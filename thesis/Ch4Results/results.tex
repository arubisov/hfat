\definecolor{cts_plot_color}{rgb}{ 0.9019608 , 0.3803922 , 0.0039216}%
\definecolor{dscr_plot_color}{rgb}{ 0.3686275 , 0.2352941 , 0.6000000}%
\definecolor{cts_nFPC_plot_color}{rgb}{ 0.9921569 , 0.7215686 , 0.3882353}%
\definecolor{dscr_nFPC_plot_color}{rgb}{ 0.6980392 , 0.6705882 , 0.8235294}%
\chapter{Results}

In this chapter we explore the dynamics of the continuous time and discrete time models, and perform in-sample and out-of-sample backtests to compare the performance of the continuous and discrete time solutions to the stochastic optimal control problem.

\section{Calibration}

All tests in this chapter were run using the following global set of parameters:
\begin{center}
\begin{tabular}{rll}
time window for computing price change & $\Delta t_S$ & 1000ms \\
time window for averaging order imbalance & $\Delta t_I$ & 1000ms \\
number of imbalance bins & $\#_{bins}$ & 5 \\
fill probability constant & $\kappa$ & 100 \\
&& \\
\multicolumn{3}{p{\linewidth}}{For each daily calibration, we then computed the remaining parameters using the following formulae:} \\
&& \\
infinitesimal generator matrix & $\mat{G}$ & \autoref{eq:MLEG} \\
transition probability matrix & $\mat{P}$ & \autoref{eq:CTMCPG} \\
market order arrival intensities & $\mu^\pm$ & \autoref{eq:MLElambda}
\end{tabular}
\end{center}

Additionally, $\xi$ was computed as half of the simple average of the bid-ask spread observed during the trading day, rounded to the nearest half-cent; and the imbalance bins $\rho$ were computed as the partitioning of the interval $[-1,1]$ into percentile bins symmetric around zero, where the percentile interval was $100 \div \#_{bins}$.

As was mentioned in Chapter 2, the exploratory data analysis done on the data made use of an unorthodox Markov chain, where its state at time $t$ was actually not determinable at time $t$ because the price change $\Delta S(t)$ was computed over the \emph{future} time interval $\Delta t_S$. (See \autoref{sec:2DCTMC}.) In the optimal stochastic control formulations, the Markov chain was defined instead such the price change was computed over the \emph{past} interval $\Delta t_S$. However, it was of interest to explore what results would be obtained if the calibration was still done using the non $\cF$-predictable method. A justification for doing so is that in a given Markov state $Z$, there is a state-dependent arrival rate of price updates, and there is a state-dependent distribution of jumps when a price update occurs. So that although the price change is measured over the future when calibrating, this really is a way of getting at the state-dependence of those price changes. In the following tests, this calibration method is denoted `w nFPC', standing for `with non-$\cF$-predictable calibration'.

\section{Dynamics of the Optimal Posting Depths}

We have solved the same stochastic control problem using both continuous and discrete time, which have yielded markedly different resulting formulae for the optimal limit order posting depths. In this section we explore the calibrated results obtained on an in-sample backtest on \texttt{INTC}, calibrating on amalgamated data for the entire 2013 trading year. The dynamics of $\delta^\pm$ were obtained using a time-to-maturity of 600sec to best depict the behavior as we approached the end of day trading; as the time-to-maturity horizon increases, the postings depths tend to stabilize. 

%\begin{figure}
%\centering
%\begin{subfigure}{.45\linewidth}
%  \centering
%  \setlength\figureheight{\linewidth} 
%  \setlength\figurewidth{\linewidth}
%  \tikzsetnextfilename{dp_cts_z1}
%  % This file was created by matlab2tikz.
%
%The latest updates can be retrieved from
%  http://www.mathworks.com/matlabcentral/fileexchange/22022-matlab2tikz-matlab2tikz
%where you can also make suggestions and rate matlab2tikz.
%
\definecolor{mycolor1}{rgb}{0.00000,1.00000,0.14286}%
\definecolor{mycolor2}{rgb}{0.00000,1.00000,0.28571}%
\definecolor{mycolor3}{rgb}{0.00000,1.00000,0.42857}%
\definecolor{mycolor4}{rgb}{0.00000,1.00000,0.57143}%
\definecolor{mycolor5}{rgb}{0.00000,1.00000,0.71429}%
\definecolor{mycolor6}{rgb}{0.00000,1.00000,0.85714}%
\definecolor{mycolor7}{rgb}{0.00000,1.00000,1.00000}%
\definecolor{mycolor8}{rgb}{0.00000,0.87500,1.00000}%
\definecolor{mycolor9}{rgb}{0.00000,0.62500,1.00000}%
\definecolor{mycolor10}{rgb}{0.12500,0.00000,1.00000}%
\definecolor{mycolor11}{rgb}{0.25000,0.00000,1.00000}%
\definecolor{mycolor12}{rgb}{0.37500,0.00000,1.00000}%
\definecolor{mycolor13}{rgb}{0.50000,0.00000,1.00000}%
\definecolor{mycolor14}{rgb}{0.62500,0.00000,1.00000}%
\definecolor{mycolor15}{rgb}{0.75000,0.00000,1.00000}%
\definecolor{mycolor16}{rgb}{0.87500,0.00000,1.00000}%
\definecolor{mycolor17}{rgb}{1.00000,0.00000,1.00000}%
\definecolor{mycolor18}{rgb}{1.00000,0.00000,0.87500}%
\definecolor{mycolor19}{rgb}{1.00000,0.00000,0.62500}%
\definecolor{mycolor20}{rgb}{0.85714,0.00000,0.00000}%
\definecolor{mycolor21}{rgb}{0.71429,0.00000,0.00000}%
%
\begin{tikzpicture}[trim axis left, trim axis right]

\begin{axis}[%
width=\figurewidth,
height=\figureheight,
at={(0\figurewidth,0\figureheight)},
scale only axis,
point meta min=0,
point meta max=1,
every outer x axis line/.append style={black},
every x tick label/.append style={font=\color{black}},
xmin=0,
xmax=600,
every outer y axis line/.append style={black},
every y tick label/.append style={font=\color{black}},
ymin=0,
ymax=0.014,
axis background/.style={fill=white},
axis x line*=bottom,
axis y line*=left,
]
\addplot [color=green,solid,forget plot]
  table[row sep=crcr]{%
0.01	0\\
1.01	0\\
2.01	0\\
3.01	0\\
4.01	0\\
5.01	0\\
6.01	0\\
7.01	0\\
8.01	0\\
9.01	0\\
10.01	0\\
11.01	0\\
12.01	0\\
13.01	0\\
14.01	0\\
15.01	0\\
16.01	0\\
17.01	0\\
18.01	0\\
19.01	0\\
20.01	0\\
21.01	0\\
22.01	0\\
23.01	0\\
24.01	0\\
25.01	0\\
26.01	0\\
27.01	0\\
28.01	0\\
29.01	0\\
30.01	0\\
31.01	0\\
32.01	0\\
33.01	0\\
34.01	0\\
35.01	0\\
36.01	0\\
37.01	0\\
38.01	0\\
39.01	0\\
40.01	0\\
41.01	0\\
42.01	0\\
43.01	0\\
44.01	0\\
45.01	0\\
46.01	0\\
47.01	0\\
48.01	0\\
49.01	0\\
50.01	0\\
51.01	0\\
52.01	0\\
53.01	0\\
54.01	0\\
55.01	0\\
56.01	0\\
57.01	0\\
58.01	0\\
59.01	0\\
60.01	0\\
61.01	0\\
62.01	0\\
63.01	0\\
64.01	0\\
65.01	0\\
66.01	0\\
67.01	0\\
68.01	0\\
69.01	0\\
70.01	0\\
71.01	0\\
72.01	0\\
73.01	0\\
74.01	0\\
75.01	0\\
76.01	0\\
77.01	0\\
78.01	0\\
79.01	0\\
80.01	0\\
81.01	0\\
82.01	0\\
83.01	0\\
84.01	0\\
85.01	0\\
86.01	0\\
87.01	0\\
88.01	0\\
89.01	0\\
90.01	0\\
91.01	0\\
92.01	0\\
93.01	0\\
94.01	0\\
95.01	0\\
96.01	0\\
97.01	0\\
98.01	0\\
99.01	0\\
100.01	0\\
101.01	0\\
102.01	0\\
103.01	0\\
104.01	0\\
105.01	0\\
106.01	0\\
107.01	0\\
108.01	0\\
109.01	0\\
110.01	0\\
111.01	0\\
112.01	0\\
113.01	0\\
114.01	0\\
115.01	0\\
116.01	0\\
117.01	0\\
118.01	0\\
119.01	0\\
120.01	0\\
121.01	0\\
122.01	0\\
123.01	0\\
124.01	0\\
125.01	0\\
126.01	0\\
127.01	0\\
128.01	0\\
129.01	0\\
130.01	0\\
131.01	0\\
132.01	0\\
133.01	0\\
134.01	0\\
135.01	0\\
136.01	0\\
137.01	0\\
138.01	0\\
139.01	0\\
140.01	0\\
141.01	0\\
142.01	0\\
143.01	0\\
144.01	0\\
145.01	0\\
146.01	0\\
147.01	0\\
148.01	0\\
149.01	0\\
150.01	0\\
151.01	0\\
152.01	0\\
153.01	0\\
154.01	0\\
155.01	0\\
156.01	0\\
157.01	0\\
158.01	0\\
159.01	0\\
160.01	0\\
161.01	0\\
162.01	0\\
163.01	0\\
164.01	0\\
165.01	0\\
166.01	0\\
167.01	0\\
168.01	0\\
169.01	0\\
170.01	0\\
171.01	0\\
172.01	0\\
173.01	0\\
174.01	0\\
175.01	0\\
176.01	0\\
177.01	0\\
178.01	0\\
179.01	0\\
180.01	0\\
181.01	0\\
182.01	0\\
183.01	0\\
184.01	0\\
185.01	0\\
186.01	0\\
187.01	0\\
188.01	0\\
189.01	0\\
190.01	0\\
191.01	0\\
192.01	0\\
193.01	0\\
194.01	0\\
195.01	0\\
196.01	0\\
197.01	0\\
198.01	0\\
199.01	0\\
200.01	0\\
201.01	0\\
202.01	0\\
203.01	0\\
204.01	0\\
205.01	0\\
206.01	0\\
207.01	0\\
208.01	0\\
209.01	0\\
210.01	0\\
211.01	0\\
212.01	0\\
213.01	0\\
214.01	0\\
215.01	0\\
216.01	0\\
217.01	0\\
218.01	0\\
219.01	0\\
220.01	0\\
221.01	0\\
222.01	0\\
223.01	0\\
224.01	0\\
225.01	0\\
226.01	0\\
227.01	0\\
228.01	0\\
229.01	0\\
230.01	0\\
231.01	0\\
232.01	0\\
233.01	0\\
234.01	0\\
235.01	0\\
236.01	0\\
237.01	0\\
238.01	0\\
239.01	0\\
240.01	0\\
241.01	0\\
242.01	0\\
243.01	0\\
244.01	0\\
245.01	0\\
246.01	0\\
247.01	0\\
248.01	0\\
249.01	0\\
250.01	0\\
251.01	0\\
252.01	0\\
253.01	0\\
254.01	0\\
255.01	0\\
256.01	0\\
257.01	0\\
258.01	0\\
259.01	0\\
260.01	0\\
261.01	0\\
262.01	0\\
263.01	0\\
264.01	0\\
265.01	0\\
266.01	0\\
267.01	0\\
268.01	0\\
269.01	0\\
270.01	0\\
271.01	0\\
272.01	0\\
273.01	0\\
274.01	0\\
275.01	0\\
276.01	0\\
277.01	0\\
278.01	0\\
279.01	0\\
280.01	0\\
281.01	0\\
282.01	0\\
283.01	0\\
284.01	0\\
285.01	0\\
286.01	0\\
287.01	0\\
288.01	0\\
289.01	0\\
290.01	0\\
291.01	0\\
292.01	0\\
293.01	0\\
294.01	0\\
295.01	0\\
296.01	0\\
297.01	0\\
298.01	0\\
299.01	0\\
300.01	0\\
301.01	0\\
302.01	0\\
303.01	0\\
304.01	0\\
305.01	0\\
306.01	0\\
307.01	0\\
308.01	0\\
309.01	0\\
310.01	0\\
311.01	0\\
312.01	0\\
313.01	0\\
314.01	0\\
315.01	0\\
316.01	0\\
317.01	0\\
318.01	0\\
319.01	0\\
320.01	0\\
321.01	0\\
322.01	0\\
323.01	0\\
324.01	0\\
325.01	0\\
326.01	0\\
327.01	0\\
328.01	0\\
329.01	0\\
330.01	0\\
331.01	0\\
332.01	0\\
333.01	0\\
334.01	0\\
335.01	0\\
336.01	0\\
337.01	0\\
338.01	0\\
339.01	0\\
340.01	0\\
341.01	0\\
342.01	0\\
343.01	0\\
344.01	0\\
345.01	0\\
346.01	0\\
347.01	0\\
348.01	0\\
349.01	0\\
350.01	0\\
351.01	0\\
352.01	0\\
353.01	0\\
354.01	0\\
355.01	0\\
356.01	0\\
357.01	0\\
358.01	0\\
359.01	0\\
360.01	0\\
361.01	0\\
362.01	0\\
363.01	0\\
364.01	0\\
365.01	0\\
366.01	0\\
367.01	0\\
368.01	0\\
369.01	0\\
370.01	0\\
371.01	0\\
372.01	0\\
373.01	0\\
374.01	0\\
375.01	0\\
376.01	0\\
377.01	0\\
378.01	0\\
379.01	0\\
380.01	0\\
381.01	0\\
382.01	0\\
383.01	0\\
384.01	0\\
385.01	0\\
386.01	0\\
387.01	0\\
388.01	0\\
389.01	0\\
390.01	0\\
391.01	0\\
392.01	0\\
393.01	0\\
394.01	0\\
395.01	0\\
396.01	0\\
397.01	0\\
398.01	0\\
399.01	0\\
400.01	0\\
401.01	0\\
402.01	0\\
403.01	0\\
404.01	0\\
405.01	0\\
406.01	0\\
407.01	0\\
408.01	0\\
409.01	0\\
410.01	0\\
411.01	0\\
412.01	0\\
413.01	0\\
414.01	0\\
415.01	0\\
416.01	0\\
417.01	0\\
418.01	0\\
419.01	0\\
420.01	0\\
421.01	0\\
422.01	0\\
423.01	0\\
424.01	0\\
425.01	0\\
426.01	0\\
427.01	0\\
428.01	0\\
429.01	0\\
430.01	0\\
431.01	0\\
432.01	0\\
433.01	0\\
434.01	0\\
435.01	0\\
436.01	0\\
437.01	0\\
438.01	0\\
439.01	0\\
440.01	0\\
441.01	0\\
442.01	0\\
443.01	0\\
444.01	0\\
445.01	0\\
446.01	0\\
447.01	0\\
448.01	0\\
449.01	0\\
450.01	0\\
451.01	1.73472347597681e-18\\
452.01	0\\
453.01	0\\
454.01	0\\
455.01	0\\
456.01	0\\
457.01	0\\
458.01	0\\
459.01	0\\
460.01	0\\
461.01	0\\
462.01	0\\
463.01	0\\
464.01	0\\
465.01	0\\
466.01	0\\
467.01	0\\
468.01	0\\
469.01	1.73472347597681e-18\\
470.01	0\\
471.01	0\\
472.01	0\\
473.01	0\\
474.01	0\\
475.01	0\\
476.01	1.73472347597681e-18\\
477.01	0\\
478.01	0\\
479.01	0\\
480.01	0\\
481.01	0\\
482.01	0\\
483.01	0\\
484.01	0\\
485.01	0\\
486.01	0\\
487.01	0\\
488.01	0\\
489.01	0\\
490.01	0\\
491.01	0\\
492.01	0\\
493.01	0\\
494.01	0\\
495.01	0\\
496.01	0\\
497.01	0\\
498.01	0\\
499.01	0\\
500.01	0\\
501.01	0\\
502.01	0\\
503.01	0\\
504.01	0\\
505.01	0\\
506.01	0\\
507.01	0\\
508.01	0\\
509.01	0\\
510.01	0\\
511.01	0\\
512.01	1.73472347597681e-18\\
513.01	0\\
514.01	0\\
515.01	0\\
516.01	0\\
517.01	0\\
518.01	0\\
519.01	0\\
520.01	0\\
521.01	0\\
522.01	0\\
523.01	1.73472347597681e-18\\
524.01	0\\
525.01	0\\
526.01	0\\
527.01	0\\
528.01	0\\
529.01	0\\
530.01	0\\
531.01	0\\
532.01	0\\
533.01	0\\
534.01	0\\
535.01	0\\
536.01	1.73472347597681e-18\\
537.01	1.73472347597681e-18\\
538.01	1.73472347597681e-18\\
539.01	0\\
540.01	0\\
541.01	0\\
542.01	0\\
543.01	0\\
544.01	0\\
545.01	1.73472347597681e-18\\
546.01	1.73472347597681e-18\\
547.01	0\\
548.01	0\\
549.01	0\\
550.01	0\\
551.01	0\\
552.01	0\\
553.01	0\\
554.01	0\\
555.01	0\\
556.01	1.73472347597681e-18\\
557.01	0\\
558.01	0\\
559.01	0\\
560.01	1.73472347597681e-18\\
561.01	1.73472347597681e-18\\
562.01	0\\
563.01	0\\
564.01	0\\
565.01	0\\
566.01	0\\
567.01	0\\
568.01	0\\
569.01	0\\
570.01	0\\
571.01	0\\
572.01	0\\
573.01	0\\
574.01	0\\
575.01	0\\
576.01	0\\
577.01	0\\
578.01	1.73472347597681e-18\\
579.01	0\\
580.01	0\\
581.01	0\\
582.01	0\\
583.01	1.73472347597681e-18\\
584.01	0\\
585.01	0\\
586.01	0\\
587.01	0\\
588.01	0\\
589.01	0\\
590.01	0\\
591.01	0\\
592.01	0\\
593.01	0\\
594.01	0\\
595.01	0\\
596.01	0\\
597.01	0\\
598.01	0\\
599.01	0\\
599.02	0\\
599.03	0\\
599.04	0\\
599.05	0\\
599.06	0\\
599.07	0\\
599.08	0\\
599.09	0\\
599.1	0\\
599.11	0\\
599.12	0\\
599.13	0\\
599.14	0\\
599.15	0\\
599.16	0\\
599.17	0\\
599.18	0\\
599.19	0\\
599.2	0\\
599.21	0\\
599.22	0\\
599.23	0\\
599.24	0\\
599.25	0\\
599.26	0\\
599.27	0\\
599.28	0\\
599.29	0\\
599.3	0\\
599.31	0\\
599.32	0\\
599.33	0\\
599.34	0\\
599.35	0\\
599.36	0\\
599.37	0\\
599.38	0\\
599.39	0\\
599.4	0\\
599.41	0\\
599.42	0\\
599.43	0\\
599.44	0\\
599.45	0\\
599.46	0\\
599.47	0\\
599.48	0\\
599.49	0\\
599.5	0\\
599.51	0\\
599.52	0\\
599.53	0\\
599.54	0\\
599.55	0\\
599.56	0\\
599.57	0\\
599.58	0\\
599.59	0\\
599.6	0\\
599.61	0\\
599.62	0\\
599.63	0\\
599.64	0\\
599.65	0\\
599.66	0\\
599.67	0\\
599.68	0\\
599.69	0\\
599.7	0\\
599.71	0\\
599.72	0\\
599.73	0\\
599.74	0\\
599.75	0\\
599.76	0\\
599.77	0\\
599.78	0\\
599.79	0\\
599.8	0\\
599.81	0\\
599.82	0\\
599.83	0\\
599.84	0\\
599.85	0\\
599.86	0\\
599.87	0\\
599.88	0\\
599.89	0\\
599.9	0\\
599.91	0\\
599.92	0\\
599.93	0\\
599.94	0\\
599.95	0\\
599.96	0\\
599.97	0\\
599.98	0\\
599.99	0\\
600	0\\
};
\addplot [color=mycolor1,solid,forget plot]
  table[row sep=crcr]{%
0.01	0\\
1.01	0\\
2.01	0\\
3.01	0\\
4.01	0\\
5.01	0\\
6.01	0\\
7.01	0\\
8.01	0\\
9.01	0\\
10.01	0\\
11.01	0\\
12.01	0\\
13.01	0\\
14.01	0\\
15.01	0\\
16.01	0\\
17.01	0\\
18.01	0\\
19.01	0\\
20.01	0\\
21.01	0\\
22.01	0\\
23.01	0\\
24.01	0\\
25.01	0\\
26.01	0\\
27.01	0\\
28.01	0\\
29.01	0\\
30.01	0\\
31.01	0\\
32.01	0\\
33.01	0\\
34.01	0\\
35.01	0\\
36.01	0\\
37.01	0\\
38.01	0\\
39.01	0\\
40.01	0\\
41.01	0\\
42.01	0\\
43.01	0\\
44.01	0\\
45.01	0\\
46.01	0\\
47.01	0\\
48.01	0\\
49.01	0\\
50.01	0\\
51.01	0\\
52.01	0\\
53.01	0\\
54.01	0\\
55.01	0\\
56.01	0\\
57.01	0\\
58.01	0\\
59.01	0\\
60.01	0\\
61.01	0\\
62.01	0\\
63.01	0\\
64.01	0\\
65.01	0\\
66.01	0\\
67.01	0\\
68.01	0\\
69.01	0\\
70.01	0\\
71.01	0\\
72.01	0\\
73.01	0\\
74.01	0\\
75.01	0\\
76.01	0\\
77.01	0\\
78.01	0\\
79.01	0\\
80.01	0\\
81.01	0\\
82.01	0\\
83.01	0\\
84.01	0\\
85.01	0\\
86.01	0\\
87.01	0\\
88.01	0\\
89.01	0\\
90.01	0\\
91.01	0\\
92.01	0\\
93.01	0\\
94.01	0\\
95.01	0\\
96.01	0\\
97.01	0\\
98.01	0\\
99.01	0\\
100.01	0\\
101.01	0\\
102.01	0\\
103.01	0\\
104.01	0\\
105.01	0\\
106.01	0\\
107.01	0\\
108.01	0\\
109.01	0\\
110.01	0\\
111.01	0\\
112.01	0\\
113.01	0\\
114.01	0\\
115.01	0\\
116.01	0\\
117.01	0\\
118.01	0\\
119.01	0\\
120.01	0\\
121.01	0\\
122.01	0\\
123.01	0\\
124.01	0\\
125.01	0\\
126.01	0\\
127.01	0\\
128.01	0\\
129.01	0\\
130.01	0\\
131.01	0\\
132.01	0\\
133.01	0\\
134.01	0\\
135.01	0\\
136.01	0\\
137.01	0\\
138.01	0\\
139.01	0\\
140.01	0\\
141.01	0\\
142.01	0\\
143.01	0\\
144.01	0\\
145.01	0\\
146.01	0\\
147.01	0\\
148.01	0\\
149.01	0\\
150.01	0\\
151.01	0\\
152.01	0\\
153.01	0\\
154.01	0\\
155.01	0\\
156.01	0\\
157.01	0\\
158.01	0\\
159.01	0\\
160.01	0\\
161.01	0\\
162.01	0\\
163.01	0\\
164.01	0\\
165.01	0\\
166.01	0\\
167.01	0\\
168.01	0\\
169.01	0\\
170.01	0\\
171.01	0\\
172.01	0\\
173.01	0\\
174.01	0\\
175.01	0\\
176.01	0\\
177.01	0\\
178.01	0\\
179.01	0\\
180.01	0\\
181.01	0\\
182.01	0\\
183.01	0\\
184.01	0\\
185.01	0\\
186.01	0\\
187.01	0\\
188.01	0\\
189.01	0\\
190.01	0\\
191.01	0\\
192.01	0\\
193.01	0\\
194.01	0\\
195.01	0\\
196.01	0\\
197.01	0\\
198.01	0\\
199.01	0\\
200.01	0\\
201.01	0\\
202.01	0\\
203.01	0\\
204.01	0\\
205.01	0\\
206.01	0\\
207.01	0\\
208.01	0\\
209.01	0\\
210.01	0\\
211.01	0\\
212.01	0\\
213.01	0\\
214.01	0\\
215.01	0\\
216.01	0\\
217.01	0\\
218.01	0\\
219.01	0\\
220.01	0\\
221.01	0\\
222.01	0\\
223.01	0\\
224.01	0\\
225.01	0\\
226.01	0\\
227.01	0\\
228.01	0\\
229.01	0\\
230.01	0\\
231.01	0\\
232.01	0\\
233.01	0\\
234.01	0\\
235.01	0\\
236.01	0\\
237.01	0\\
238.01	0\\
239.01	0\\
240.01	0\\
241.01	0\\
242.01	0\\
243.01	0\\
244.01	0\\
245.01	0\\
246.01	0\\
247.01	0\\
248.01	0\\
249.01	0\\
250.01	0\\
251.01	0\\
252.01	0\\
253.01	0\\
254.01	0\\
255.01	0\\
256.01	0\\
257.01	0\\
258.01	0\\
259.01	0\\
260.01	0\\
261.01	0\\
262.01	0\\
263.01	0\\
264.01	0\\
265.01	0\\
266.01	0\\
267.01	0\\
268.01	0\\
269.01	0\\
270.01	0\\
271.01	0\\
272.01	0\\
273.01	0\\
274.01	0\\
275.01	0\\
276.01	0\\
277.01	0\\
278.01	0\\
279.01	0\\
280.01	0\\
281.01	0\\
282.01	0\\
283.01	0\\
284.01	0\\
285.01	0\\
286.01	0\\
287.01	0\\
288.01	0\\
289.01	0\\
290.01	0\\
291.01	0\\
292.01	0\\
293.01	0\\
294.01	0\\
295.01	0\\
296.01	0\\
297.01	0\\
298.01	0\\
299.01	0\\
300.01	0\\
301.01	0\\
302.01	0\\
303.01	0\\
304.01	0\\
305.01	0\\
306.01	0\\
307.01	0\\
308.01	0\\
309.01	0\\
310.01	0\\
311.01	0\\
312.01	0\\
313.01	0\\
314.01	0\\
315.01	0\\
316.01	0\\
317.01	0\\
318.01	0\\
319.01	0\\
320.01	0\\
321.01	0\\
322.01	0\\
323.01	0\\
324.01	0\\
325.01	0\\
326.01	0\\
327.01	0\\
328.01	0\\
329.01	0\\
330.01	0\\
331.01	0\\
332.01	0\\
333.01	0\\
334.01	0\\
335.01	0\\
336.01	0\\
337.01	0\\
338.01	0\\
339.01	0\\
340.01	0\\
341.01	0\\
342.01	0\\
343.01	0\\
344.01	0\\
345.01	0\\
346.01	0\\
347.01	0\\
348.01	0\\
349.01	0\\
350.01	0\\
351.01	0\\
352.01	0\\
353.01	0\\
354.01	0\\
355.01	0\\
356.01	0\\
357.01	0\\
358.01	0\\
359.01	0\\
360.01	0\\
361.01	0\\
362.01	0\\
363.01	0\\
364.01	0\\
365.01	0\\
366.01	0\\
367.01	0\\
368.01	0\\
369.01	0\\
370.01	0\\
371.01	0\\
372.01	0\\
373.01	0\\
374.01	0\\
375.01	0\\
376.01	0\\
377.01	0\\
378.01	0\\
379.01	0\\
380.01	0\\
381.01	0\\
382.01	0\\
383.01	0\\
384.01	0\\
385.01	0\\
386.01	0\\
387.01	0\\
388.01	0\\
389.01	0\\
390.01	0\\
391.01	0\\
392.01	0\\
393.01	0\\
394.01	0\\
395.01	0\\
396.01	0\\
397.01	0\\
398.01	0\\
399.01	0\\
400.01	0\\
401.01	0\\
402.01	0\\
403.01	0\\
404.01	0\\
405.01	0\\
406.01	0\\
407.01	0\\
408.01	0\\
409.01	0\\
410.01	0\\
411.01	0\\
412.01	0\\
413.01	0\\
414.01	0\\
415.01	0\\
416.01	0\\
417.01	0\\
418.01	0\\
419.01	0\\
420.01	0\\
421.01	0\\
422.01	0\\
423.01	0\\
424.01	0\\
425.01	0\\
426.01	0\\
427.01	0\\
428.01	0\\
429.01	0\\
430.01	0\\
431.01	0\\
432.01	0\\
433.01	0\\
434.01	0\\
435.01	0\\
436.01	0\\
437.01	0\\
438.01	0\\
439.01	0\\
440.01	0\\
441.01	0\\
442.01	0\\
443.01	0\\
444.01	0\\
445.01	0\\
446.01	0\\
447.01	0\\
448.01	0\\
449.01	0\\
450.01	0\\
451.01	1.73472347597681e-18\\
452.01	0\\
453.01	0\\
454.01	0\\
455.01	0\\
456.01	0\\
457.01	0\\
458.01	0\\
459.01	0\\
460.01	0\\
461.01	0\\
462.01	0\\
463.01	0\\
464.01	0\\
465.01	0\\
466.01	0\\
467.01	0\\
468.01	0\\
469.01	1.73472347597681e-18\\
470.01	0\\
471.01	0\\
472.01	0\\
473.01	0\\
474.01	0\\
475.01	0\\
476.01	1.73472347597681e-18\\
477.01	0\\
478.01	0\\
479.01	0\\
480.01	0\\
481.01	0\\
482.01	0\\
483.01	0\\
484.01	0\\
485.01	0\\
486.01	0\\
487.01	0\\
488.01	0\\
489.01	0\\
490.01	0\\
491.01	0\\
492.01	0\\
493.01	0\\
494.01	0\\
495.01	0\\
496.01	0\\
497.01	0\\
498.01	0\\
499.01	0\\
500.01	0\\
501.01	0\\
502.01	0\\
503.01	0\\
504.01	0\\
505.01	0\\
506.01	0\\
507.01	0\\
508.01	0\\
509.01	0\\
510.01	0\\
511.01	0\\
512.01	1.73472347597681e-18\\
513.01	0\\
514.01	0\\
515.01	0\\
516.01	0\\
517.01	0\\
518.01	0\\
519.01	0\\
520.01	0\\
521.01	0\\
522.01	0\\
523.01	1.73472347597681e-18\\
524.01	0\\
525.01	0\\
526.01	0\\
527.01	0\\
528.01	0\\
529.01	0\\
530.01	0\\
531.01	0\\
532.01	0\\
533.01	0\\
534.01	0\\
535.01	0\\
536.01	1.73472347597681e-18\\
537.01	1.73472347597681e-18\\
538.01	1.73472347597681e-18\\
539.01	0\\
540.01	0\\
541.01	0\\
542.01	0\\
543.01	0\\
544.01	0\\
545.01	1.73472347597681e-18\\
546.01	1.73472347597681e-18\\
547.01	0\\
548.01	0\\
549.01	0\\
550.01	0\\
551.01	0\\
552.01	0\\
553.01	0\\
554.01	0\\
555.01	0\\
556.01	1.73472347597681e-18\\
557.01	0\\
558.01	0\\
559.01	0\\
560.01	1.73472347597681e-18\\
561.01	1.73472347597681e-18\\
562.01	0\\
563.01	0\\
564.01	0\\
565.01	0\\
566.01	0\\
567.01	0\\
568.01	0\\
569.01	0\\
570.01	0\\
571.01	0\\
572.01	0\\
573.01	0\\
574.01	0\\
575.01	0\\
576.01	0\\
577.01	0\\
578.01	1.73472347597681e-18\\
579.01	0\\
580.01	0\\
581.01	0\\
582.01	0\\
583.01	1.73472347597681e-18\\
584.01	0\\
585.01	0\\
586.01	0\\
587.01	0\\
588.01	0\\
589.01	0\\
590.01	0\\
591.01	0\\
592.01	0\\
593.01	0\\
594.01	0\\
595.01	0\\
596.01	0\\
597.01	0\\
598.01	0\\
599.01	0\\
599.02	0\\
599.03	0\\
599.04	0\\
599.05	0\\
599.06	0\\
599.07	0\\
599.08	0\\
599.09	0\\
599.1	0\\
599.11	0\\
599.12	0\\
599.13	0\\
599.14	0\\
599.15	0\\
599.16	0\\
599.17	0\\
599.18	0\\
599.19	0\\
599.2	0\\
599.21	0\\
599.22	0\\
599.23	0\\
599.24	0\\
599.25	0\\
599.26	0\\
599.27	0\\
599.28	0\\
599.29	0\\
599.3	0\\
599.31	0\\
599.32	0\\
599.33	0\\
599.34	0\\
599.35	0\\
599.36	0\\
599.37	0\\
599.38	0\\
599.39	0\\
599.4	0\\
599.41	0\\
599.42	0\\
599.43	0\\
599.44	0\\
599.45	0\\
599.46	0\\
599.47	0\\
599.48	0\\
599.49	0\\
599.5	0\\
599.51	0\\
599.52	0\\
599.53	0\\
599.54	0\\
599.55	0\\
599.56	0\\
599.57	0\\
599.58	0\\
599.59	0\\
599.6	0\\
599.61	0\\
599.62	0\\
599.63	0\\
599.64	0\\
599.65	0\\
599.66	0\\
599.67	0\\
599.68	0\\
599.69	0\\
599.7	0\\
599.71	0\\
599.72	0\\
599.73	0\\
599.74	0\\
599.75	0\\
599.76	0\\
599.77	0\\
599.78	0\\
599.79	0\\
599.8	0\\
599.81	0\\
599.82	0\\
599.83	0\\
599.84	0\\
599.85	0\\
599.86	0\\
599.87	0\\
599.88	0\\
599.89	0\\
599.9	0\\
599.91	0\\
599.92	0\\
599.93	0\\
599.94	0\\
599.95	0\\
599.96	0\\
599.97	0\\
599.98	0\\
599.99	0\\
600	0\\
};
\addplot [color=mycolor2,solid,forget plot]
  table[row sep=crcr]{%
0.01	0\\
1.01	0\\
2.01	0\\
3.01	0\\
4.01	0\\
5.01	0\\
6.01	0\\
7.01	0\\
8.01	0\\
9.01	0\\
10.01	0\\
11.01	0\\
12.01	0\\
13.01	0\\
14.01	0\\
15.01	0\\
16.01	0\\
17.01	0\\
18.01	0\\
19.01	0\\
20.01	0\\
21.01	0\\
22.01	0\\
23.01	0\\
24.01	0\\
25.01	0\\
26.01	0\\
27.01	0\\
28.01	0\\
29.01	0\\
30.01	0\\
31.01	0\\
32.01	0\\
33.01	0\\
34.01	0\\
35.01	0\\
36.01	0\\
37.01	0\\
38.01	0\\
39.01	0\\
40.01	0\\
41.01	0\\
42.01	0\\
43.01	0\\
44.01	0\\
45.01	0\\
46.01	0\\
47.01	0\\
48.01	0\\
49.01	0\\
50.01	0\\
51.01	0\\
52.01	0\\
53.01	0\\
54.01	0\\
55.01	0\\
56.01	0\\
57.01	0\\
58.01	0\\
59.01	0\\
60.01	0\\
61.01	0\\
62.01	0\\
63.01	0\\
64.01	0\\
65.01	0\\
66.01	0\\
67.01	0\\
68.01	0\\
69.01	0\\
70.01	0\\
71.01	0\\
72.01	0\\
73.01	0\\
74.01	0\\
75.01	0\\
76.01	0\\
77.01	0\\
78.01	0\\
79.01	0\\
80.01	0\\
81.01	0\\
82.01	0\\
83.01	0\\
84.01	0\\
85.01	0\\
86.01	0\\
87.01	0\\
88.01	0\\
89.01	0\\
90.01	0\\
91.01	0\\
92.01	0\\
93.01	0\\
94.01	0\\
95.01	0\\
96.01	0\\
97.01	0\\
98.01	0\\
99.01	0\\
100.01	0\\
101.01	0\\
102.01	0\\
103.01	0\\
104.01	0\\
105.01	0\\
106.01	0\\
107.01	0\\
108.01	0\\
109.01	0\\
110.01	0\\
111.01	0\\
112.01	0\\
113.01	0\\
114.01	0\\
115.01	0\\
116.01	0\\
117.01	0\\
118.01	0\\
119.01	0\\
120.01	0\\
121.01	0\\
122.01	0\\
123.01	0\\
124.01	0\\
125.01	0\\
126.01	0\\
127.01	0\\
128.01	0\\
129.01	0\\
130.01	0\\
131.01	0\\
132.01	0\\
133.01	0\\
134.01	0\\
135.01	0\\
136.01	0\\
137.01	0\\
138.01	0\\
139.01	0\\
140.01	0\\
141.01	0\\
142.01	0\\
143.01	0\\
144.01	0\\
145.01	0\\
146.01	0\\
147.01	0\\
148.01	0\\
149.01	0\\
150.01	0\\
151.01	0\\
152.01	0\\
153.01	0\\
154.01	0\\
155.01	0\\
156.01	0\\
157.01	0\\
158.01	0\\
159.01	0\\
160.01	0\\
161.01	0\\
162.01	0\\
163.01	0\\
164.01	0\\
165.01	0\\
166.01	0\\
167.01	0\\
168.01	0\\
169.01	0\\
170.01	0\\
171.01	0\\
172.01	0\\
173.01	0\\
174.01	0\\
175.01	0\\
176.01	0\\
177.01	0\\
178.01	0\\
179.01	0\\
180.01	0\\
181.01	0\\
182.01	0\\
183.01	0\\
184.01	0\\
185.01	0\\
186.01	0\\
187.01	0\\
188.01	0\\
189.01	0\\
190.01	0\\
191.01	0\\
192.01	0\\
193.01	0\\
194.01	0\\
195.01	0\\
196.01	0\\
197.01	0\\
198.01	0\\
199.01	0\\
200.01	0\\
201.01	0\\
202.01	0\\
203.01	0\\
204.01	0\\
205.01	0\\
206.01	0\\
207.01	0\\
208.01	0\\
209.01	0\\
210.01	0\\
211.01	0\\
212.01	0\\
213.01	0\\
214.01	0\\
215.01	0\\
216.01	0\\
217.01	0\\
218.01	0\\
219.01	0\\
220.01	0\\
221.01	0\\
222.01	0\\
223.01	0\\
224.01	0\\
225.01	0\\
226.01	0\\
227.01	0\\
228.01	0\\
229.01	0\\
230.01	0\\
231.01	0\\
232.01	0\\
233.01	0\\
234.01	0\\
235.01	0\\
236.01	0\\
237.01	0\\
238.01	0\\
239.01	0\\
240.01	0\\
241.01	0\\
242.01	0\\
243.01	0\\
244.01	0\\
245.01	0\\
246.01	0\\
247.01	0\\
248.01	0\\
249.01	0\\
250.01	0\\
251.01	0\\
252.01	0\\
253.01	0\\
254.01	0\\
255.01	0\\
256.01	0\\
257.01	0\\
258.01	0\\
259.01	0\\
260.01	0\\
261.01	0\\
262.01	0\\
263.01	0\\
264.01	0\\
265.01	0\\
266.01	0\\
267.01	0\\
268.01	0\\
269.01	0\\
270.01	0\\
271.01	0\\
272.01	0\\
273.01	0\\
274.01	0\\
275.01	0\\
276.01	0\\
277.01	0\\
278.01	0\\
279.01	0\\
280.01	0\\
281.01	0\\
282.01	0\\
283.01	0\\
284.01	0\\
285.01	0\\
286.01	0\\
287.01	0\\
288.01	0\\
289.01	0\\
290.01	0\\
291.01	0\\
292.01	0\\
293.01	0\\
294.01	0\\
295.01	0\\
296.01	0\\
297.01	0\\
298.01	0\\
299.01	0\\
300.01	0\\
301.01	0\\
302.01	0\\
303.01	0\\
304.01	0\\
305.01	0\\
306.01	0\\
307.01	0\\
308.01	0\\
309.01	0\\
310.01	0\\
311.01	0\\
312.01	0\\
313.01	0\\
314.01	0\\
315.01	0\\
316.01	0\\
317.01	0\\
318.01	0\\
319.01	0\\
320.01	0\\
321.01	0\\
322.01	0\\
323.01	0\\
324.01	0\\
325.01	0\\
326.01	0\\
327.01	0\\
328.01	0\\
329.01	0\\
330.01	0\\
331.01	0\\
332.01	0\\
333.01	0\\
334.01	0\\
335.01	0\\
336.01	0\\
337.01	0\\
338.01	0\\
339.01	0\\
340.01	0\\
341.01	0\\
342.01	0\\
343.01	0\\
344.01	0\\
345.01	0\\
346.01	0\\
347.01	0\\
348.01	0\\
349.01	0\\
350.01	0\\
351.01	0\\
352.01	0\\
353.01	0\\
354.01	0\\
355.01	0\\
356.01	0\\
357.01	0\\
358.01	0\\
359.01	0\\
360.01	0\\
361.01	0\\
362.01	0\\
363.01	0\\
364.01	0\\
365.01	0\\
366.01	0\\
367.01	0\\
368.01	0\\
369.01	0\\
370.01	0\\
371.01	0\\
372.01	0\\
373.01	0\\
374.01	0\\
375.01	0\\
376.01	0\\
377.01	0\\
378.01	0\\
379.01	0\\
380.01	0\\
381.01	0\\
382.01	0\\
383.01	0\\
384.01	0\\
385.01	0\\
386.01	0\\
387.01	0\\
388.01	0\\
389.01	0\\
390.01	0\\
391.01	0\\
392.01	0\\
393.01	0\\
394.01	0\\
395.01	0\\
396.01	0\\
397.01	0\\
398.01	0\\
399.01	0\\
400.01	0\\
401.01	0\\
402.01	0\\
403.01	0\\
404.01	0\\
405.01	0\\
406.01	0\\
407.01	0\\
408.01	0\\
409.01	0\\
410.01	0\\
411.01	0\\
412.01	0\\
413.01	0\\
414.01	0\\
415.01	0\\
416.01	0\\
417.01	0\\
418.01	0\\
419.01	0\\
420.01	0\\
421.01	0\\
422.01	0\\
423.01	0\\
424.01	0\\
425.01	0\\
426.01	0\\
427.01	0\\
428.01	0\\
429.01	0\\
430.01	0\\
431.01	0\\
432.01	0\\
433.01	0\\
434.01	0\\
435.01	0\\
436.01	0\\
437.01	0\\
438.01	0\\
439.01	0\\
440.01	0\\
441.01	0\\
442.01	0\\
443.01	0\\
444.01	0\\
445.01	0\\
446.01	0\\
447.01	0\\
448.01	0\\
449.01	0\\
450.01	0\\
451.01	1.73472347597681e-18\\
452.01	0\\
453.01	0\\
454.01	0\\
455.01	0\\
456.01	0\\
457.01	0\\
458.01	0\\
459.01	0\\
460.01	0\\
461.01	0\\
462.01	0\\
463.01	0\\
464.01	0\\
465.01	0\\
466.01	0\\
467.01	0\\
468.01	0\\
469.01	1.73472347597681e-18\\
470.01	0\\
471.01	0\\
472.01	0\\
473.01	0\\
474.01	0\\
475.01	0\\
476.01	1.73472347597681e-18\\
477.01	0\\
478.01	0\\
479.01	0\\
480.01	0\\
481.01	0\\
482.01	0\\
483.01	0\\
484.01	0\\
485.01	0\\
486.01	0\\
487.01	0\\
488.01	0\\
489.01	0\\
490.01	0\\
491.01	0\\
492.01	0\\
493.01	0\\
494.01	0\\
495.01	0\\
496.01	0\\
497.01	0\\
498.01	0\\
499.01	0\\
500.01	0\\
501.01	0\\
502.01	0\\
503.01	0\\
504.01	0\\
505.01	0\\
506.01	0\\
507.01	0\\
508.01	0\\
509.01	0\\
510.01	0\\
511.01	0\\
512.01	1.73472347597681e-18\\
513.01	0\\
514.01	0\\
515.01	0\\
516.01	0\\
517.01	0\\
518.01	0\\
519.01	0\\
520.01	0\\
521.01	0\\
522.01	0\\
523.01	1.73472347597681e-18\\
524.01	0\\
525.01	0\\
526.01	0\\
527.01	0\\
528.01	0\\
529.01	0\\
530.01	0\\
531.01	0\\
532.01	0\\
533.01	0\\
534.01	0\\
535.01	0\\
536.01	1.73472347597681e-18\\
537.01	1.73472347597681e-18\\
538.01	1.73472347597681e-18\\
539.01	0\\
540.01	0\\
541.01	0\\
542.01	0\\
543.01	0\\
544.01	0\\
545.01	1.73472347597681e-18\\
546.01	1.73472347597681e-18\\
547.01	0\\
548.01	0\\
549.01	0\\
550.01	0\\
551.01	0\\
552.01	0\\
553.01	0\\
554.01	0\\
555.01	0\\
556.01	1.73472347597681e-18\\
557.01	0\\
558.01	0\\
559.01	0\\
560.01	1.73472347597681e-18\\
561.01	1.73472347597681e-18\\
562.01	0\\
563.01	0\\
564.01	0\\
565.01	0\\
566.01	0\\
567.01	0\\
568.01	0\\
569.01	0\\
570.01	0\\
571.01	0\\
572.01	0\\
573.01	0\\
574.01	0\\
575.01	0\\
576.01	0\\
577.01	0\\
578.01	1.73472347597681e-18\\
579.01	0\\
580.01	0\\
581.01	0\\
582.01	0\\
583.01	1.73472347597681e-18\\
584.01	0\\
585.01	0\\
586.01	0\\
587.01	0\\
588.01	0\\
589.01	0\\
590.01	0\\
591.01	0\\
592.01	0\\
593.01	0\\
594.01	0\\
595.01	0\\
596.01	0\\
597.01	0\\
598.01	0\\
599.01	0\\
599.02	0\\
599.03	0\\
599.04	0\\
599.05	0\\
599.06	0\\
599.07	0\\
599.08	0\\
599.09	0\\
599.1	0\\
599.11	0\\
599.12	0\\
599.13	0\\
599.14	0\\
599.15	0\\
599.16	0\\
599.17	0\\
599.18	0\\
599.19	0\\
599.2	0\\
599.21	0\\
599.22	0\\
599.23	0\\
599.24	0\\
599.25	0\\
599.26	0\\
599.27	0\\
599.28	0\\
599.29	0\\
599.3	0\\
599.31	0\\
599.32	0\\
599.33	0\\
599.34	0\\
599.35	0\\
599.36	0\\
599.37	0\\
599.38	0\\
599.39	0\\
599.4	0\\
599.41	0\\
599.42	0\\
599.43	0\\
599.44	0\\
599.45	0\\
599.46	0\\
599.47	0\\
599.48	0\\
599.49	0\\
599.5	0\\
599.51	0\\
599.52	0\\
599.53	0\\
599.54	0\\
599.55	0\\
599.56	0\\
599.57	0\\
599.58	0\\
599.59	0\\
599.6	0\\
599.61	0\\
599.62	0\\
599.63	0\\
599.64	0\\
599.65	0\\
599.66	0\\
599.67	0\\
599.68	0\\
599.69	0\\
599.7	0\\
599.71	0\\
599.72	0\\
599.73	0\\
599.74	0\\
599.75	0\\
599.76	0\\
599.77	0\\
599.78	0\\
599.79	0\\
599.8	0\\
599.81	0\\
599.82	0\\
599.83	0\\
599.84	0\\
599.85	0\\
599.86	0\\
599.87	0\\
599.88	0\\
599.89	0\\
599.9	0\\
599.91	0\\
599.92	0\\
599.93	0\\
599.94	0\\
599.95	0\\
599.96	0\\
599.97	0\\
599.98	0\\
599.99	0\\
600	0\\
};
\addplot [color=mycolor3,solid,forget plot]
  table[row sep=crcr]{%
0.01	0\\
1.01	0\\
2.01	0\\
3.01	0\\
4.01	0\\
5.01	0\\
6.01	0\\
7.01	0\\
8.01	0\\
9.01	0\\
10.01	0\\
11.01	0\\
12.01	0\\
13.01	0\\
14.01	0\\
15.01	0\\
16.01	0\\
17.01	0\\
18.01	0\\
19.01	0\\
20.01	0\\
21.01	0\\
22.01	0\\
23.01	0\\
24.01	0\\
25.01	0\\
26.01	0\\
27.01	0\\
28.01	0\\
29.01	0\\
30.01	0\\
31.01	0\\
32.01	0\\
33.01	0\\
34.01	0\\
35.01	0\\
36.01	0\\
37.01	0\\
38.01	0\\
39.01	0\\
40.01	0\\
41.01	0\\
42.01	0\\
43.01	0\\
44.01	0\\
45.01	0\\
46.01	0\\
47.01	0\\
48.01	0\\
49.01	0\\
50.01	0\\
51.01	0\\
52.01	0\\
53.01	0\\
54.01	0\\
55.01	0\\
56.01	0\\
57.01	0\\
58.01	0\\
59.01	0\\
60.01	0\\
61.01	0\\
62.01	0\\
63.01	0\\
64.01	0\\
65.01	0\\
66.01	0\\
67.01	0\\
68.01	0\\
69.01	0\\
70.01	0\\
71.01	0\\
72.01	0\\
73.01	0\\
74.01	0\\
75.01	0\\
76.01	0\\
77.01	0\\
78.01	0\\
79.01	0\\
80.01	0\\
81.01	0\\
82.01	0\\
83.01	0\\
84.01	0\\
85.01	0\\
86.01	0\\
87.01	0\\
88.01	0\\
89.01	0\\
90.01	0\\
91.01	0\\
92.01	0\\
93.01	0\\
94.01	0\\
95.01	0\\
96.01	0\\
97.01	0\\
98.01	0\\
99.01	0\\
100.01	0\\
101.01	0\\
102.01	0\\
103.01	0\\
104.01	0\\
105.01	0\\
106.01	0\\
107.01	0\\
108.01	0\\
109.01	0\\
110.01	0\\
111.01	0\\
112.01	0\\
113.01	0\\
114.01	0\\
115.01	0\\
116.01	0\\
117.01	0\\
118.01	0\\
119.01	0\\
120.01	0\\
121.01	0\\
122.01	0\\
123.01	0\\
124.01	0\\
125.01	0\\
126.01	0\\
127.01	0\\
128.01	0\\
129.01	0\\
130.01	0\\
131.01	0\\
132.01	0\\
133.01	0\\
134.01	0\\
135.01	0\\
136.01	0\\
137.01	0\\
138.01	0\\
139.01	0\\
140.01	0\\
141.01	0\\
142.01	0\\
143.01	0\\
144.01	0\\
145.01	0\\
146.01	0\\
147.01	0\\
148.01	0\\
149.01	0\\
150.01	0\\
151.01	0\\
152.01	0\\
153.01	0\\
154.01	0\\
155.01	0\\
156.01	0\\
157.01	0\\
158.01	0\\
159.01	0\\
160.01	0\\
161.01	0\\
162.01	0\\
163.01	0\\
164.01	0\\
165.01	0\\
166.01	0\\
167.01	0\\
168.01	0\\
169.01	0\\
170.01	0\\
171.01	0\\
172.01	0\\
173.01	0\\
174.01	0\\
175.01	0\\
176.01	0\\
177.01	0\\
178.01	0\\
179.01	0\\
180.01	0\\
181.01	0\\
182.01	0\\
183.01	0\\
184.01	0\\
185.01	0\\
186.01	0\\
187.01	0\\
188.01	0\\
189.01	0\\
190.01	0\\
191.01	0\\
192.01	0\\
193.01	0\\
194.01	0\\
195.01	0\\
196.01	0\\
197.01	0\\
198.01	0\\
199.01	0\\
200.01	0\\
201.01	0\\
202.01	0\\
203.01	0\\
204.01	0\\
205.01	0\\
206.01	0\\
207.01	0\\
208.01	0\\
209.01	0\\
210.01	0\\
211.01	0\\
212.01	0\\
213.01	0\\
214.01	0\\
215.01	0\\
216.01	0\\
217.01	0\\
218.01	0\\
219.01	0\\
220.01	0\\
221.01	0\\
222.01	0\\
223.01	0\\
224.01	0\\
225.01	0\\
226.01	0\\
227.01	0\\
228.01	0\\
229.01	0\\
230.01	0\\
231.01	0\\
232.01	0\\
233.01	0\\
234.01	0\\
235.01	0\\
236.01	0\\
237.01	0\\
238.01	0\\
239.01	0\\
240.01	0\\
241.01	0\\
242.01	0\\
243.01	0\\
244.01	0\\
245.01	0\\
246.01	0\\
247.01	0\\
248.01	0\\
249.01	0\\
250.01	0\\
251.01	0\\
252.01	0\\
253.01	0\\
254.01	0\\
255.01	0\\
256.01	0\\
257.01	0\\
258.01	0\\
259.01	0\\
260.01	0\\
261.01	0\\
262.01	0\\
263.01	0\\
264.01	0\\
265.01	0\\
266.01	0\\
267.01	0\\
268.01	0\\
269.01	0\\
270.01	0\\
271.01	0\\
272.01	0\\
273.01	0\\
274.01	0\\
275.01	0\\
276.01	0\\
277.01	0\\
278.01	0\\
279.01	0\\
280.01	0\\
281.01	0\\
282.01	0\\
283.01	0\\
284.01	0\\
285.01	0\\
286.01	0\\
287.01	0\\
288.01	0\\
289.01	0\\
290.01	0\\
291.01	0\\
292.01	0\\
293.01	0\\
294.01	0\\
295.01	0\\
296.01	0\\
297.01	0\\
298.01	0\\
299.01	0\\
300.01	0\\
301.01	0\\
302.01	0\\
303.01	0\\
304.01	0\\
305.01	0\\
306.01	0\\
307.01	0\\
308.01	0\\
309.01	0\\
310.01	0\\
311.01	0\\
312.01	0\\
313.01	0\\
314.01	0\\
315.01	0\\
316.01	0\\
317.01	0\\
318.01	0\\
319.01	0\\
320.01	0\\
321.01	0\\
322.01	0\\
323.01	0\\
324.01	0\\
325.01	0\\
326.01	0\\
327.01	0\\
328.01	0\\
329.01	0\\
330.01	0\\
331.01	0\\
332.01	0\\
333.01	0\\
334.01	0\\
335.01	0\\
336.01	0\\
337.01	0\\
338.01	0\\
339.01	0\\
340.01	0\\
341.01	0\\
342.01	0\\
343.01	0\\
344.01	0\\
345.01	0\\
346.01	0\\
347.01	0\\
348.01	0\\
349.01	0\\
350.01	0\\
351.01	0\\
352.01	0\\
353.01	0\\
354.01	0\\
355.01	0\\
356.01	0\\
357.01	0\\
358.01	0\\
359.01	0\\
360.01	0\\
361.01	0\\
362.01	0\\
363.01	0\\
364.01	0\\
365.01	0\\
366.01	0\\
367.01	0\\
368.01	0\\
369.01	0\\
370.01	0\\
371.01	0\\
372.01	0\\
373.01	0\\
374.01	0\\
375.01	0\\
376.01	0\\
377.01	0\\
378.01	0\\
379.01	0\\
380.01	0\\
381.01	0\\
382.01	0\\
383.01	0\\
384.01	0\\
385.01	0\\
386.01	0\\
387.01	0\\
388.01	0\\
389.01	0\\
390.01	0\\
391.01	0\\
392.01	0\\
393.01	0\\
394.01	0\\
395.01	0\\
396.01	0\\
397.01	0\\
398.01	0\\
399.01	0\\
400.01	0\\
401.01	0\\
402.01	0\\
403.01	0\\
404.01	0\\
405.01	0\\
406.01	0\\
407.01	0\\
408.01	0\\
409.01	0\\
410.01	0\\
411.01	0\\
412.01	0\\
413.01	0\\
414.01	0\\
415.01	0\\
416.01	0\\
417.01	0\\
418.01	0\\
419.01	0\\
420.01	0\\
421.01	0\\
422.01	0\\
423.01	0\\
424.01	0\\
425.01	0\\
426.01	0\\
427.01	0\\
428.01	0\\
429.01	0\\
430.01	0\\
431.01	0\\
432.01	0\\
433.01	0\\
434.01	0\\
435.01	0\\
436.01	0\\
437.01	0\\
438.01	0\\
439.01	0\\
440.01	0\\
441.01	0\\
442.01	0\\
443.01	0\\
444.01	0\\
445.01	0\\
446.01	0\\
447.01	0\\
448.01	0\\
449.01	0\\
450.01	0\\
451.01	1.73472347597681e-18\\
452.01	0\\
453.01	0\\
454.01	0\\
455.01	0\\
456.01	0\\
457.01	0\\
458.01	0\\
459.01	0\\
460.01	0\\
461.01	0\\
462.01	0\\
463.01	0\\
464.01	0\\
465.01	0\\
466.01	0\\
467.01	0\\
468.01	0\\
469.01	1.73472347597681e-18\\
470.01	0\\
471.01	0\\
472.01	0\\
473.01	0\\
474.01	0\\
475.01	0\\
476.01	1.73472347597681e-18\\
477.01	0\\
478.01	0\\
479.01	0\\
480.01	0\\
481.01	0\\
482.01	0\\
483.01	0\\
484.01	0\\
485.01	0\\
486.01	0\\
487.01	0\\
488.01	0\\
489.01	0\\
490.01	0\\
491.01	0\\
492.01	0\\
493.01	0\\
494.01	0\\
495.01	0\\
496.01	0\\
497.01	0\\
498.01	0\\
499.01	0\\
500.01	0\\
501.01	0\\
502.01	0\\
503.01	0\\
504.01	0\\
505.01	0\\
506.01	0\\
507.01	0\\
508.01	0\\
509.01	0\\
510.01	0\\
511.01	0\\
512.01	1.73472347597681e-18\\
513.01	0\\
514.01	0\\
515.01	0\\
516.01	0\\
517.01	0\\
518.01	0\\
519.01	0\\
520.01	0\\
521.01	0\\
522.01	0\\
523.01	1.73472347597681e-18\\
524.01	0\\
525.01	0\\
526.01	0\\
527.01	0\\
528.01	0\\
529.01	0\\
530.01	0\\
531.01	0\\
532.01	0\\
533.01	0\\
534.01	0\\
535.01	0\\
536.01	1.73472347597681e-18\\
537.01	1.73472347597681e-18\\
538.01	1.73472347597681e-18\\
539.01	0\\
540.01	0\\
541.01	0\\
542.01	0\\
543.01	0\\
544.01	0\\
545.01	1.73472347597681e-18\\
546.01	1.73472347597681e-18\\
547.01	0\\
548.01	0\\
549.01	0\\
550.01	0\\
551.01	0\\
552.01	0\\
553.01	0\\
554.01	0\\
555.01	0\\
556.01	1.73472347597681e-18\\
557.01	0\\
558.01	0\\
559.01	0\\
560.01	1.73472347597681e-18\\
561.01	1.73472347597681e-18\\
562.01	0\\
563.01	0\\
564.01	0\\
565.01	0\\
566.01	0\\
567.01	0\\
568.01	0\\
569.01	0\\
570.01	0\\
571.01	0\\
572.01	0\\
573.01	0\\
574.01	0\\
575.01	0\\
576.01	0\\
577.01	0\\
578.01	1.73472347597681e-18\\
579.01	0\\
580.01	0\\
581.01	0\\
582.01	0\\
583.01	1.73472347597681e-18\\
584.01	0\\
585.01	0\\
586.01	0\\
587.01	0\\
588.01	0\\
589.01	0\\
590.01	0\\
591.01	0\\
592.01	0\\
593.01	0\\
594.01	0\\
595.01	0\\
596.01	0\\
597.01	0\\
598.01	0\\
599.01	0\\
599.02	0\\
599.03	0\\
599.04	0\\
599.05	0\\
599.06	0\\
599.07	0\\
599.08	0\\
599.09	0\\
599.1	0\\
599.11	0\\
599.12	0\\
599.13	0\\
599.14	0\\
599.15	0\\
599.16	0\\
599.17	0\\
599.18	0\\
599.19	0\\
599.2	0\\
599.21	0\\
599.22	0\\
599.23	0\\
599.24	0\\
599.25	0\\
599.26	0\\
599.27	0\\
599.28	0\\
599.29	0\\
599.3	0\\
599.31	0\\
599.32	0\\
599.33	0\\
599.34	0\\
599.35	0\\
599.36	0\\
599.37	0\\
599.38	0\\
599.39	0\\
599.4	0\\
599.41	0\\
599.42	0\\
599.43	0\\
599.44	0\\
599.45	0\\
599.46	0\\
599.47	0\\
599.48	0\\
599.49	0\\
599.5	0\\
599.51	0\\
599.52	0\\
599.53	0\\
599.54	0\\
599.55	0\\
599.56	0\\
599.57	0\\
599.58	0\\
599.59	0\\
599.6	0\\
599.61	0\\
599.62	0\\
599.63	0\\
599.64	0\\
599.65	0\\
599.66	0\\
599.67	0\\
599.68	0\\
599.69	0\\
599.7	0\\
599.71	0\\
599.72	0\\
599.73	0\\
599.74	0\\
599.75	0\\
599.76	0\\
599.77	0\\
599.78	0\\
599.79	0\\
599.8	0\\
599.81	0\\
599.82	0\\
599.83	0\\
599.84	0\\
599.85	0\\
599.86	0\\
599.87	0\\
599.88	0\\
599.89	0\\
599.9	0\\
599.91	0\\
599.92	0\\
599.93	0\\
599.94	0\\
599.95	0\\
599.96	0\\
599.97	0\\
599.98	0\\
599.99	0\\
600	0\\
};
\addplot [color=mycolor4,solid,forget plot]
  table[row sep=crcr]{%
0.01	0\\
1.01	0\\
2.01	0\\
3.01	0\\
4.01	0\\
5.01	0\\
6.01	0\\
7.01	0\\
8.01	0\\
9.01	0\\
10.01	0\\
11.01	0\\
12.01	0\\
13.01	0\\
14.01	0\\
15.01	0\\
16.01	0\\
17.01	0\\
18.01	0\\
19.01	0\\
20.01	0\\
21.01	0\\
22.01	0\\
23.01	0\\
24.01	0\\
25.01	0\\
26.01	0\\
27.01	0\\
28.01	0\\
29.01	0\\
30.01	0\\
31.01	0\\
32.01	0\\
33.01	0\\
34.01	0\\
35.01	0\\
36.01	0\\
37.01	0\\
38.01	0\\
39.01	0\\
40.01	0\\
41.01	0\\
42.01	0\\
43.01	0\\
44.01	0\\
45.01	0\\
46.01	0\\
47.01	0\\
48.01	0\\
49.01	0\\
50.01	0\\
51.01	0\\
52.01	0\\
53.01	0\\
54.01	0\\
55.01	0\\
56.01	0\\
57.01	0\\
58.01	0\\
59.01	0\\
60.01	0\\
61.01	0\\
62.01	0\\
63.01	0\\
64.01	0\\
65.01	0\\
66.01	0\\
67.01	0\\
68.01	0\\
69.01	0\\
70.01	0\\
71.01	0\\
72.01	0\\
73.01	0\\
74.01	0\\
75.01	0\\
76.01	0\\
77.01	0\\
78.01	0\\
79.01	0\\
80.01	0\\
81.01	0\\
82.01	0\\
83.01	0\\
84.01	0\\
85.01	0\\
86.01	0\\
87.01	0\\
88.01	0\\
89.01	0\\
90.01	0\\
91.01	0\\
92.01	0\\
93.01	0\\
94.01	0\\
95.01	0\\
96.01	0\\
97.01	0\\
98.01	0\\
99.01	0\\
100.01	0\\
101.01	0\\
102.01	0\\
103.01	0\\
104.01	0\\
105.01	0\\
106.01	0\\
107.01	0\\
108.01	0\\
109.01	0\\
110.01	0\\
111.01	0\\
112.01	0\\
113.01	0\\
114.01	0\\
115.01	0\\
116.01	0\\
117.01	0\\
118.01	0\\
119.01	0\\
120.01	0\\
121.01	0\\
122.01	0\\
123.01	0\\
124.01	0\\
125.01	0\\
126.01	0\\
127.01	0\\
128.01	0\\
129.01	0\\
130.01	0\\
131.01	0\\
132.01	0\\
133.01	0\\
134.01	0\\
135.01	0\\
136.01	0\\
137.01	0\\
138.01	0\\
139.01	0\\
140.01	0\\
141.01	0\\
142.01	0\\
143.01	0\\
144.01	0\\
145.01	0\\
146.01	0\\
147.01	0\\
148.01	0\\
149.01	0\\
150.01	0\\
151.01	0\\
152.01	0\\
153.01	0\\
154.01	0\\
155.01	0\\
156.01	0\\
157.01	0\\
158.01	0\\
159.01	0\\
160.01	0\\
161.01	0\\
162.01	0\\
163.01	0\\
164.01	0\\
165.01	0\\
166.01	0\\
167.01	0\\
168.01	0\\
169.01	0\\
170.01	0\\
171.01	0\\
172.01	0\\
173.01	0\\
174.01	0\\
175.01	0\\
176.01	0\\
177.01	0\\
178.01	0\\
179.01	0\\
180.01	0\\
181.01	0\\
182.01	0\\
183.01	0\\
184.01	0\\
185.01	0\\
186.01	0\\
187.01	0\\
188.01	0\\
189.01	0\\
190.01	0\\
191.01	0\\
192.01	0\\
193.01	0\\
194.01	0\\
195.01	0\\
196.01	0\\
197.01	0\\
198.01	0\\
199.01	0\\
200.01	0\\
201.01	0\\
202.01	0\\
203.01	0\\
204.01	0\\
205.01	0\\
206.01	0\\
207.01	0\\
208.01	0\\
209.01	0\\
210.01	0\\
211.01	0\\
212.01	0\\
213.01	0\\
214.01	0\\
215.01	0\\
216.01	0\\
217.01	0\\
218.01	0\\
219.01	0\\
220.01	0\\
221.01	0\\
222.01	0\\
223.01	0\\
224.01	0\\
225.01	0\\
226.01	0\\
227.01	0\\
228.01	0\\
229.01	0\\
230.01	0\\
231.01	0\\
232.01	0\\
233.01	0\\
234.01	0\\
235.01	0\\
236.01	0\\
237.01	0\\
238.01	0\\
239.01	0\\
240.01	0\\
241.01	0\\
242.01	0\\
243.01	0\\
244.01	0\\
245.01	0\\
246.01	0\\
247.01	0\\
248.01	0\\
249.01	0\\
250.01	0\\
251.01	0\\
252.01	0\\
253.01	0\\
254.01	0\\
255.01	0\\
256.01	0\\
257.01	0\\
258.01	0\\
259.01	0\\
260.01	0\\
261.01	0\\
262.01	0\\
263.01	0\\
264.01	0\\
265.01	0\\
266.01	0\\
267.01	0\\
268.01	0\\
269.01	0\\
270.01	0\\
271.01	0\\
272.01	0\\
273.01	0\\
274.01	0\\
275.01	0\\
276.01	0\\
277.01	0\\
278.01	0\\
279.01	0\\
280.01	0\\
281.01	0\\
282.01	0\\
283.01	0\\
284.01	0\\
285.01	0\\
286.01	0\\
287.01	0\\
288.01	0\\
289.01	0\\
290.01	0\\
291.01	0\\
292.01	0\\
293.01	0\\
294.01	0\\
295.01	0\\
296.01	0\\
297.01	0\\
298.01	0\\
299.01	0\\
300.01	0\\
301.01	0\\
302.01	0\\
303.01	0\\
304.01	0\\
305.01	0\\
306.01	0\\
307.01	0\\
308.01	0\\
309.01	0\\
310.01	0\\
311.01	0\\
312.01	0\\
313.01	0\\
314.01	0\\
315.01	0\\
316.01	0\\
317.01	0\\
318.01	0\\
319.01	0\\
320.01	0\\
321.01	0\\
322.01	0\\
323.01	0\\
324.01	0\\
325.01	0\\
326.01	0\\
327.01	0\\
328.01	0\\
329.01	0\\
330.01	0\\
331.01	0\\
332.01	0\\
333.01	0\\
334.01	0\\
335.01	0\\
336.01	0\\
337.01	0\\
338.01	0\\
339.01	0\\
340.01	0\\
341.01	0\\
342.01	0\\
343.01	0\\
344.01	0\\
345.01	0\\
346.01	0\\
347.01	0\\
348.01	0\\
349.01	0\\
350.01	0\\
351.01	0\\
352.01	0\\
353.01	0\\
354.01	0\\
355.01	0\\
356.01	0\\
357.01	0\\
358.01	0\\
359.01	0\\
360.01	0\\
361.01	0\\
362.01	0\\
363.01	0\\
364.01	0\\
365.01	0\\
366.01	0\\
367.01	0\\
368.01	0\\
369.01	0\\
370.01	0\\
371.01	0\\
372.01	0\\
373.01	0\\
374.01	0\\
375.01	0\\
376.01	0\\
377.01	0\\
378.01	0\\
379.01	0\\
380.01	0\\
381.01	0\\
382.01	0\\
383.01	0\\
384.01	0\\
385.01	0\\
386.01	0\\
387.01	0\\
388.01	0\\
389.01	0\\
390.01	0\\
391.01	0\\
392.01	0\\
393.01	0\\
394.01	0\\
395.01	0\\
396.01	0\\
397.01	0\\
398.01	0\\
399.01	0\\
400.01	0\\
401.01	0\\
402.01	0\\
403.01	0\\
404.01	0\\
405.01	0\\
406.01	0\\
407.01	0\\
408.01	0\\
409.01	0\\
410.01	0\\
411.01	0\\
412.01	0\\
413.01	0\\
414.01	0\\
415.01	0\\
416.01	0\\
417.01	0\\
418.01	0\\
419.01	0\\
420.01	0\\
421.01	0\\
422.01	0\\
423.01	0\\
424.01	0\\
425.01	0\\
426.01	0\\
427.01	0\\
428.01	0\\
429.01	0\\
430.01	0\\
431.01	0\\
432.01	0\\
433.01	0\\
434.01	0\\
435.01	0\\
436.01	0\\
437.01	0\\
438.01	0\\
439.01	0\\
440.01	0\\
441.01	0\\
442.01	0\\
443.01	0\\
444.01	0\\
445.01	0\\
446.01	0\\
447.01	0\\
448.01	0\\
449.01	0\\
450.01	0\\
451.01	1.73472347597681e-18\\
452.01	0\\
453.01	0\\
454.01	0\\
455.01	0\\
456.01	0\\
457.01	0\\
458.01	0\\
459.01	0\\
460.01	0\\
461.01	0\\
462.01	0\\
463.01	0\\
464.01	0\\
465.01	0\\
466.01	0\\
467.01	0\\
468.01	0\\
469.01	1.73472347597681e-18\\
470.01	0\\
471.01	0\\
472.01	0\\
473.01	0\\
474.01	0\\
475.01	0\\
476.01	1.73472347597681e-18\\
477.01	0\\
478.01	0\\
479.01	0\\
480.01	0\\
481.01	0\\
482.01	0\\
483.01	0\\
484.01	0\\
485.01	0\\
486.01	0\\
487.01	0\\
488.01	0\\
489.01	0\\
490.01	0\\
491.01	0\\
492.01	0\\
493.01	0\\
494.01	0\\
495.01	0\\
496.01	0\\
497.01	0\\
498.01	0\\
499.01	0\\
500.01	0\\
501.01	0\\
502.01	0\\
503.01	0\\
504.01	0\\
505.01	0\\
506.01	0\\
507.01	0\\
508.01	0\\
509.01	0\\
510.01	0\\
511.01	0\\
512.01	1.73472347597681e-18\\
513.01	0\\
514.01	0\\
515.01	0\\
516.01	0\\
517.01	0\\
518.01	0\\
519.01	0\\
520.01	0\\
521.01	0\\
522.01	0\\
523.01	1.73472347597681e-18\\
524.01	0\\
525.01	0\\
526.01	0\\
527.01	0\\
528.01	0\\
529.01	0\\
530.01	0\\
531.01	0\\
532.01	0\\
533.01	0\\
534.01	0\\
535.01	0\\
536.01	1.73472347597681e-18\\
537.01	1.73472347597681e-18\\
538.01	1.73472347597681e-18\\
539.01	0\\
540.01	0\\
541.01	0\\
542.01	0\\
543.01	0\\
544.01	0\\
545.01	1.73472347597681e-18\\
546.01	1.73472347597681e-18\\
547.01	0\\
548.01	0\\
549.01	0\\
550.01	0\\
551.01	0\\
552.01	0\\
553.01	0\\
554.01	0\\
555.01	0\\
556.01	1.73472347597681e-18\\
557.01	0\\
558.01	0\\
559.01	0\\
560.01	1.73472347597681e-18\\
561.01	1.73472347597681e-18\\
562.01	0\\
563.01	0\\
564.01	0\\
565.01	0\\
566.01	0\\
567.01	0\\
568.01	0\\
569.01	0\\
570.01	0\\
571.01	0\\
572.01	0\\
573.01	0\\
574.01	0\\
575.01	0\\
576.01	0\\
577.01	0\\
578.01	1.73472347597681e-18\\
579.01	0\\
580.01	0\\
581.01	0\\
582.01	0\\
583.01	1.73472347597681e-18\\
584.01	0\\
585.01	0\\
586.01	0\\
587.01	0\\
588.01	0\\
589.01	0\\
590.01	0\\
591.01	0\\
592.01	0\\
593.01	0\\
594.01	0\\
595.01	0\\
596.01	0\\
597.01	0\\
598.01	0\\
599.01	0\\
599.02	0\\
599.03	0\\
599.04	0\\
599.05	0\\
599.06	0\\
599.07	0\\
599.08	0\\
599.09	0\\
599.1	0\\
599.11	0\\
599.12	0\\
599.13	0\\
599.14	0\\
599.15	0\\
599.16	0\\
599.17	0\\
599.18	0\\
599.19	0\\
599.2	0\\
599.21	0\\
599.22	0\\
599.23	0\\
599.24	0\\
599.25	0\\
599.26	0\\
599.27	0\\
599.28	0\\
599.29	0\\
599.3	0\\
599.31	0\\
599.32	0\\
599.33	0\\
599.34	0\\
599.35	0\\
599.36	0\\
599.37	0\\
599.38	0\\
599.39	0\\
599.4	0\\
599.41	0\\
599.42	0\\
599.43	0\\
599.44	0\\
599.45	0\\
599.46	0\\
599.47	0\\
599.48	0\\
599.49	0\\
599.5	0\\
599.51	0\\
599.52	0\\
599.53	0\\
599.54	0\\
599.55	0\\
599.56	0\\
599.57	0\\
599.58	0\\
599.59	0\\
599.6	0\\
599.61	0\\
599.62	0\\
599.63	0\\
599.64	0\\
599.65	0\\
599.66	0\\
599.67	0\\
599.68	0\\
599.69	0\\
599.7	0\\
599.71	0\\
599.72	0\\
599.73	0\\
599.74	0\\
599.75	0\\
599.76	0\\
599.77	0\\
599.78	0\\
599.79	0\\
599.8	0\\
599.81	0\\
599.82	0\\
599.83	0\\
599.84	0\\
599.85	0\\
599.86	0\\
599.87	0\\
599.88	0\\
599.89	0\\
599.9	0\\
599.91	0\\
599.92	0\\
599.93	0\\
599.94	0\\
599.95	0\\
599.96	0\\
599.97	0\\
599.98	0\\
599.99	0\\
600	0\\
};
\addplot [color=mycolor5,solid,forget plot]
  table[row sep=crcr]{%
0.01	0\\
1.01	0\\
2.01	0\\
3.01	0\\
4.01	0\\
5.01	0\\
6.01	0\\
7.01	0\\
8.01	0\\
9.01	0\\
10.01	0\\
11.01	0\\
12.01	0\\
13.01	0\\
14.01	0\\
15.01	0\\
16.01	0\\
17.01	0\\
18.01	0\\
19.01	0\\
20.01	0\\
21.01	0\\
22.01	0\\
23.01	0\\
24.01	0\\
25.01	0\\
26.01	0\\
27.01	0\\
28.01	0\\
29.01	0\\
30.01	0\\
31.01	0\\
32.01	0\\
33.01	0\\
34.01	0\\
35.01	0\\
36.01	0\\
37.01	0\\
38.01	0\\
39.01	0\\
40.01	0\\
41.01	0\\
42.01	0\\
43.01	0\\
44.01	0\\
45.01	0\\
46.01	0\\
47.01	0\\
48.01	0\\
49.01	0\\
50.01	0\\
51.01	0\\
52.01	0\\
53.01	0\\
54.01	0\\
55.01	0\\
56.01	0\\
57.01	0\\
58.01	0\\
59.01	0\\
60.01	0\\
61.01	0\\
62.01	0\\
63.01	0\\
64.01	0\\
65.01	0\\
66.01	0\\
67.01	0\\
68.01	0\\
69.01	0\\
70.01	0\\
71.01	0\\
72.01	0\\
73.01	0\\
74.01	0\\
75.01	0\\
76.01	0\\
77.01	0\\
78.01	0\\
79.01	0\\
80.01	0\\
81.01	0\\
82.01	0\\
83.01	0\\
84.01	0\\
85.01	0\\
86.01	0\\
87.01	0\\
88.01	0\\
89.01	0\\
90.01	0\\
91.01	0\\
92.01	0\\
93.01	0\\
94.01	0\\
95.01	0\\
96.01	0\\
97.01	0\\
98.01	0\\
99.01	0\\
100.01	0\\
101.01	0\\
102.01	0\\
103.01	0\\
104.01	0\\
105.01	0\\
106.01	0\\
107.01	0\\
108.01	0\\
109.01	0\\
110.01	0\\
111.01	0\\
112.01	0\\
113.01	0\\
114.01	0\\
115.01	0\\
116.01	0\\
117.01	0\\
118.01	0\\
119.01	0\\
120.01	0\\
121.01	0\\
122.01	0\\
123.01	0\\
124.01	0\\
125.01	0\\
126.01	0\\
127.01	0\\
128.01	0\\
129.01	0\\
130.01	0\\
131.01	0\\
132.01	0\\
133.01	0\\
134.01	0\\
135.01	0\\
136.01	0\\
137.01	0\\
138.01	0\\
139.01	0\\
140.01	0\\
141.01	0\\
142.01	0\\
143.01	0\\
144.01	0\\
145.01	0\\
146.01	0\\
147.01	0\\
148.01	0\\
149.01	0\\
150.01	0\\
151.01	0\\
152.01	0\\
153.01	0\\
154.01	0\\
155.01	0\\
156.01	0\\
157.01	0\\
158.01	0\\
159.01	0\\
160.01	0\\
161.01	0\\
162.01	0\\
163.01	0\\
164.01	0\\
165.01	0\\
166.01	0\\
167.01	0\\
168.01	0\\
169.01	0\\
170.01	0\\
171.01	0\\
172.01	0\\
173.01	0\\
174.01	0\\
175.01	0\\
176.01	0\\
177.01	0\\
178.01	0\\
179.01	0\\
180.01	0\\
181.01	0\\
182.01	0\\
183.01	0\\
184.01	0\\
185.01	0\\
186.01	0\\
187.01	0\\
188.01	0\\
189.01	0\\
190.01	0\\
191.01	0\\
192.01	0\\
193.01	0\\
194.01	0\\
195.01	0\\
196.01	0\\
197.01	0\\
198.01	0\\
199.01	0\\
200.01	0\\
201.01	0\\
202.01	0\\
203.01	0\\
204.01	0\\
205.01	0\\
206.01	0\\
207.01	0\\
208.01	0\\
209.01	0\\
210.01	0\\
211.01	0\\
212.01	0\\
213.01	0\\
214.01	0\\
215.01	0\\
216.01	0\\
217.01	0\\
218.01	0\\
219.01	0\\
220.01	0\\
221.01	0\\
222.01	0\\
223.01	0\\
224.01	0\\
225.01	0\\
226.01	0\\
227.01	0\\
228.01	0\\
229.01	0\\
230.01	0\\
231.01	0\\
232.01	0\\
233.01	0\\
234.01	0\\
235.01	0\\
236.01	0\\
237.01	0\\
238.01	0\\
239.01	0\\
240.01	0\\
241.01	0\\
242.01	0\\
243.01	0\\
244.01	0\\
245.01	0\\
246.01	0\\
247.01	0\\
248.01	0\\
249.01	0\\
250.01	0\\
251.01	0\\
252.01	0\\
253.01	0\\
254.01	0\\
255.01	0\\
256.01	0\\
257.01	0\\
258.01	0\\
259.01	0\\
260.01	0\\
261.01	0\\
262.01	0\\
263.01	0\\
264.01	0\\
265.01	0\\
266.01	0\\
267.01	0\\
268.01	0\\
269.01	0\\
270.01	0\\
271.01	0\\
272.01	0\\
273.01	0\\
274.01	0\\
275.01	0\\
276.01	0\\
277.01	0\\
278.01	0\\
279.01	0\\
280.01	0\\
281.01	0\\
282.01	0\\
283.01	0\\
284.01	0\\
285.01	0\\
286.01	0\\
287.01	0\\
288.01	0\\
289.01	0\\
290.01	0\\
291.01	0\\
292.01	0\\
293.01	0\\
294.01	0\\
295.01	0\\
296.01	0\\
297.01	0\\
298.01	0\\
299.01	0\\
300.01	0\\
301.01	0\\
302.01	0\\
303.01	0\\
304.01	0\\
305.01	0\\
306.01	0\\
307.01	0\\
308.01	0\\
309.01	0\\
310.01	0\\
311.01	0\\
312.01	0\\
313.01	0\\
314.01	0\\
315.01	0\\
316.01	0\\
317.01	0\\
318.01	0\\
319.01	0\\
320.01	0\\
321.01	0\\
322.01	0\\
323.01	0\\
324.01	0\\
325.01	0\\
326.01	0\\
327.01	0\\
328.01	0\\
329.01	0\\
330.01	0\\
331.01	0\\
332.01	0\\
333.01	0\\
334.01	0\\
335.01	0\\
336.01	0\\
337.01	0\\
338.01	0\\
339.01	0\\
340.01	0\\
341.01	0\\
342.01	0\\
343.01	0\\
344.01	0\\
345.01	0\\
346.01	0\\
347.01	0\\
348.01	0\\
349.01	0\\
350.01	0\\
351.01	0\\
352.01	0\\
353.01	0\\
354.01	0\\
355.01	0\\
356.01	0\\
357.01	0\\
358.01	0\\
359.01	0\\
360.01	0\\
361.01	0\\
362.01	0\\
363.01	0\\
364.01	0\\
365.01	0\\
366.01	0\\
367.01	0\\
368.01	0\\
369.01	0\\
370.01	0\\
371.01	0\\
372.01	0\\
373.01	0\\
374.01	0\\
375.01	0\\
376.01	0\\
377.01	0\\
378.01	0\\
379.01	0\\
380.01	0\\
381.01	0\\
382.01	0\\
383.01	0\\
384.01	0\\
385.01	0\\
386.01	0\\
387.01	0\\
388.01	0\\
389.01	0\\
390.01	0\\
391.01	0\\
392.01	0\\
393.01	0\\
394.01	0\\
395.01	0\\
396.01	0\\
397.01	0\\
398.01	0\\
399.01	0\\
400.01	0\\
401.01	0\\
402.01	0\\
403.01	0\\
404.01	0\\
405.01	0\\
406.01	0\\
407.01	0\\
408.01	0\\
409.01	0\\
410.01	0\\
411.01	0\\
412.01	0\\
413.01	0\\
414.01	0\\
415.01	0\\
416.01	0\\
417.01	0\\
418.01	0\\
419.01	0\\
420.01	0\\
421.01	0\\
422.01	0\\
423.01	0\\
424.01	0\\
425.01	0\\
426.01	0\\
427.01	0\\
428.01	0\\
429.01	0\\
430.01	0\\
431.01	0\\
432.01	0\\
433.01	0\\
434.01	0\\
435.01	0\\
436.01	0\\
437.01	0\\
438.01	0\\
439.01	0\\
440.01	0\\
441.01	0\\
442.01	0\\
443.01	0\\
444.01	0\\
445.01	0\\
446.01	0\\
447.01	0\\
448.01	0\\
449.01	0\\
450.01	0\\
451.01	1.73472347597681e-18\\
452.01	0\\
453.01	0\\
454.01	0\\
455.01	0\\
456.01	0\\
457.01	0\\
458.01	0\\
459.01	0\\
460.01	0\\
461.01	0\\
462.01	0\\
463.01	0\\
464.01	0\\
465.01	0\\
466.01	0\\
467.01	0\\
468.01	0\\
469.01	1.73472347597681e-18\\
470.01	0\\
471.01	0\\
472.01	0\\
473.01	0\\
474.01	0\\
475.01	0\\
476.01	1.73472347597681e-18\\
477.01	0\\
478.01	0\\
479.01	0\\
480.01	0\\
481.01	0\\
482.01	0\\
483.01	0\\
484.01	0\\
485.01	0\\
486.01	0\\
487.01	0\\
488.01	0\\
489.01	0\\
490.01	0\\
491.01	0\\
492.01	0\\
493.01	0\\
494.01	0\\
495.01	0\\
496.01	0\\
497.01	0\\
498.01	0\\
499.01	0\\
500.01	0\\
501.01	0\\
502.01	0\\
503.01	0\\
504.01	0\\
505.01	0\\
506.01	0\\
507.01	0\\
508.01	0\\
509.01	0\\
510.01	0\\
511.01	0\\
512.01	1.73472347597681e-18\\
513.01	0\\
514.01	0\\
515.01	0\\
516.01	0\\
517.01	0\\
518.01	0\\
519.01	0\\
520.01	0\\
521.01	0\\
522.01	0\\
523.01	1.73472347597681e-18\\
524.01	0\\
525.01	0\\
526.01	0\\
527.01	0\\
528.01	0\\
529.01	0\\
530.01	0\\
531.01	0\\
532.01	0\\
533.01	0\\
534.01	0\\
535.01	0\\
536.01	1.73472347597681e-18\\
537.01	1.73472347597681e-18\\
538.01	1.73472347597681e-18\\
539.01	0\\
540.01	0\\
541.01	0\\
542.01	0\\
543.01	0\\
544.01	0\\
545.01	1.73472347597681e-18\\
546.01	1.73472347597681e-18\\
547.01	0\\
548.01	0\\
549.01	0\\
550.01	0\\
551.01	0\\
552.01	0\\
553.01	0\\
554.01	0\\
555.01	0\\
556.01	1.73472347597681e-18\\
557.01	0\\
558.01	0\\
559.01	0\\
560.01	1.73472347597681e-18\\
561.01	1.73472347597681e-18\\
562.01	0\\
563.01	0\\
564.01	0\\
565.01	0\\
566.01	0\\
567.01	0\\
568.01	0\\
569.01	0\\
570.01	0\\
571.01	0\\
572.01	0\\
573.01	0\\
574.01	0\\
575.01	0\\
576.01	0\\
577.01	0\\
578.01	1.73472347597681e-18\\
579.01	0\\
580.01	0\\
581.01	0\\
582.01	0\\
583.01	1.73472347597681e-18\\
584.01	0\\
585.01	0\\
586.01	0\\
587.01	0\\
588.01	0\\
589.01	0\\
590.01	0\\
591.01	0\\
592.01	0\\
593.01	0\\
594.01	0\\
595.01	0\\
596.01	0\\
597.01	0\\
598.01	0\\
599.01	0\\
599.02	0\\
599.03	0\\
599.04	0\\
599.05	0\\
599.06	0\\
599.07	0\\
599.08	0\\
599.09	0\\
599.1	0\\
599.11	0\\
599.12	0\\
599.13	0\\
599.14	0\\
599.15	0\\
599.16	0\\
599.17	0\\
599.18	0\\
599.19	0\\
599.2	0\\
599.21	0\\
599.22	0\\
599.23	0\\
599.24	0\\
599.25	0\\
599.26	0\\
599.27	0\\
599.28	0\\
599.29	0\\
599.3	0\\
599.31	0\\
599.32	0\\
599.33	0\\
599.34	0\\
599.35	0\\
599.36	0\\
599.37	0\\
599.38	0\\
599.39	0\\
599.4	0\\
599.41	0\\
599.42	0\\
599.43	0\\
599.44	0\\
599.45	0\\
599.46	0\\
599.47	0\\
599.48	0\\
599.49	0\\
599.5	0\\
599.51	0\\
599.52	0\\
599.53	0\\
599.54	0\\
599.55	0\\
599.56	0\\
599.57	0\\
599.58	0\\
599.59	0\\
599.6	0\\
599.61	0\\
599.62	0\\
599.63	0\\
599.64	0\\
599.65	0\\
599.66	0\\
599.67	0\\
599.68	0\\
599.69	0\\
599.7	0\\
599.71	0\\
599.72	0\\
599.73	0\\
599.74	0\\
599.75	0\\
599.76	0\\
599.77	0\\
599.78	0\\
599.79	0\\
599.8	0\\
599.81	0\\
599.82	0\\
599.83	0\\
599.84	0\\
599.85	0\\
599.86	0\\
599.87	0\\
599.88	0\\
599.89	0\\
599.9	0\\
599.91	0\\
599.92	0\\
599.93	0\\
599.94	0\\
599.95	0\\
599.96	0\\
599.97	0\\
599.98	0\\
599.99	0\\
600	0\\
};
\addplot [color=mycolor6,solid,forget plot]
  table[row sep=crcr]{%
0.01	0\\
1.01	0\\
2.01	0\\
3.01	0\\
4.01	0\\
5.01	0\\
6.01	0\\
7.01	0\\
8.01	0\\
9.01	0\\
10.01	0\\
11.01	0\\
12.01	0\\
13.01	0\\
14.01	0\\
15.01	0\\
16.01	0\\
17.01	0\\
18.01	0\\
19.01	0\\
20.01	0\\
21.01	0\\
22.01	0\\
23.01	0\\
24.01	0\\
25.01	0\\
26.01	0\\
27.01	0\\
28.01	0\\
29.01	0\\
30.01	0\\
31.01	0\\
32.01	0\\
33.01	0\\
34.01	0\\
35.01	0\\
36.01	0\\
37.01	0\\
38.01	0\\
39.01	0\\
40.01	0\\
41.01	0\\
42.01	0\\
43.01	0\\
44.01	0\\
45.01	0\\
46.01	0\\
47.01	0\\
48.01	0\\
49.01	0\\
50.01	0\\
51.01	0\\
52.01	0\\
53.01	0\\
54.01	0\\
55.01	0\\
56.01	0\\
57.01	0\\
58.01	0\\
59.01	0\\
60.01	0\\
61.01	0\\
62.01	0\\
63.01	0\\
64.01	0\\
65.01	0\\
66.01	0\\
67.01	0\\
68.01	0\\
69.01	0\\
70.01	0\\
71.01	0\\
72.01	0\\
73.01	0\\
74.01	0\\
75.01	0\\
76.01	0\\
77.01	0\\
78.01	0\\
79.01	0\\
80.01	0\\
81.01	0\\
82.01	0\\
83.01	0\\
84.01	0\\
85.01	0\\
86.01	0\\
87.01	0\\
88.01	0\\
89.01	0\\
90.01	0\\
91.01	0\\
92.01	0\\
93.01	0\\
94.01	0\\
95.01	0\\
96.01	0\\
97.01	0\\
98.01	0\\
99.01	0\\
100.01	0\\
101.01	0\\
102.01	0\\
103.01	0\\
104.01	0\\
105.01	0\\
106.01	0\\
107.01	0\\
108.01	0\\
109.01	0\\
110.01	0\\
111.01	0\\
112.01	0\\
113.01	0\\
114.01	0\\
115.01	0\\
116.01	0\\
117.01	0\\
118.01	0\\
119.01	0\\
120.01	0\\
121.01	0\\
122.01	0\\
123.01	0\\
124.01	0\\
125.01	0\\
126.01	0\\
127.01	0\\
128.01	0\\
129.01	0\\
130.01	0\\
131.01	0\\
132.01	0\\
133.01	0\\
134.01	0\\
135.01	0\\
136.01	0\\
137.01	0\\
138.01	0\\
139.01	0\\
140.01	0\\
141.01	0\\
142.01	0\\
143.01	0\\
144.01	0\\
145.01	0\\
146.01	0\\
147.01	0\\
148.01	0\\
149.01	0\\
150.01	0\\
151.01	0\\
152.01	0\\
153.01	0\\
154.01	0\\
155.01	0\\
156.01	0\\
157.01	0\\
158.01	0\\
159.01	0\\
160.01	0\\
161.01	0\\
162.01	0\\
163.01	0\\
164.01	0\\
165.01	0\\
166.01	0\\
167.01	0\\
168.01	0\\
169.01	0\\
170.01	0\\
171.01	0\\
172.01	0\\
173.01	0\\
174.01	0\\
175.01	0\\
176.01	0\\
177.01	0\\
178.01	0\\
179.01	0\\
180.01	0\\
181.01	0\\
182.01	0\\
183.01	0\\
184.01	0\\
185.01	0\\
186.01	0\\
187.01	0\\
188.01	0\\
189.01	0\\
190.01	0\\
191.01	0\\
192.01	0\\
193.01	0\\
194.01	0\\
195.01	0\\
196.01	0\\
197.01	0\\
198.01	0\\
199.01	0\\
200.01	0\\
201.01	0\\
202.01	0\\
203.01	0\\
204.01	0\\
205.01	0\\
206.01	0\\
207.01	0\\
208.01	0\\
209.01	0\\
210.01	0\\
211.01	0\\
212.01	0\\
213.01	0\\
214.01	0\\
215.01	0\\
216.01	0\\
217.01	0\\
218.01	0\\
219.01	0\\
220.01	0\\
221.01	0\\
222.01	0\\
223.01	0\\
224.01	0\\
225.01	0\\
226.01	0\\
227.01	0\\
228.01	0\\
229.01	0\\
230.01	0\\
231.01	0\\
232.01	0\\
233.01	0\\
234.01	0\\
235.01	0\\
236.01	0\\
237.01	0\\
238.01	0\\
239.01	0\\
240.01	0\\
241.01	0\\
242.01	0\\
243.01	0\\
244.01	0\\
245.01	0\\
246.01	0\\
247.01	0\\
248.01	0\\
249.01	0\\
250.01	0\\
251.01	0\\
252.01	0\\
253.01	0\\
254.01	0\\
255.01	0\\
256.01	0\\
257.01	0\\
258.01	0\\
259.01	0\\
260.01	0\\
261.01	0\\
262.01	0\\
263.01	0\\
264.01	0\\
265.01	0\\
266.01	0\\
267.01	0\\
268.01	0\\
269.01	0\\
270.01	0\\
271.01	0\\
272.01	0\\
273.01	0\\
274.01	0\\
275.01	0\\
276.01	0\\
277.01	0\\
278.01	0\\
279.01	0\\
280.01	0\\
281.01	0\\
282.01	0\\
283.01	0\\
284.01	0\\
285.01	0\\
286.01	0\\
287.01	0\\
288.01	0\\
289.01	0\\
290.01	0\\
291.01	0\\
292.01	0\\
293.01	0\\
294.01	0\\
295.01	0\\
296.01	0\\
297.01	0\\
298.01	0\\
299.01	0\\
300.01	0\\
301.01	0\\
302.01	0\\
303.01	0\\
304.01	0\\
305.01	0\\
306.01	0\\
307.01	0\\
308.01	0\\
309.01	0\\
310.01	0\\
311.01	0\\
312.01	0\\
313.01	0\\
314.01	0\\
315.01	0\\
316.01	0\\
317.01	0\\
318.01	0\\
319.01	0\\
320.01	0\\
321.01	0\\
322.01	0\\
323.01	0\\
324.01	0\\
325.01	0\\
326.01	0\\
327.01	0\\
328.01	0\\
329.01	0\\
330.01	0\\
331.01	0\\
332.01	0\\
333.01	0\\
334.01	0\\
335.01	0\\
336.01	0\\
337.01	0\\
338.01	0\\
339.01	0\\
340.01	0\\
341.01	0\\
342.01	0\\
343.01	0\\
344.01	0\\
345.01	0\\
346.01	0\\
347.01	0\\
348.01	0\\
349.01	0\\
350.01	0\\
351.01	0\\
352.01	0\\
353.01	0\\
354.01	0\\
355.01	0\\
356.01	0\\
357.01	0\\
358.01	0\\
359.01	0\\
360.01	0\\
361.01	0\\
362.01	0\\
363.01	0\\
364.01	0\\
365.01	0\\
366.01	0\\
367.01	0\\
368.01	0\\
369.01	0\\
370.01	0\\
371.01	0\\
372.01	0\\
373.01	0\\
374.01	0\\
375.01	0\\
376.01	0\\
377.01	0\\
378.01	0\\
379.01	0\\
380.01	0\\
381.01	0\\
382.01	0\\
383.01	0\\
384.01	0\\
385.01	0\\
386.01	0\\
387.01	0\\
388.01	0\\
389.01	0\\
390.01	0\\
391.01	0\\
392.01	0\\
393.01	0\\
394.01	0\\
395.01	0\\
396.01	0\\
397.01	0\\
398.01	0\\
399.01	0\\
400.01	0\\
401.01	0\\
402.01	0\\
403.01	0\\
404.01	0\\
405.01	0\\
406.01	0\\
407.01	0\\
408.01	0\\
409.01	0\\
410.01	0\\
411.01	0\\
412.01	0\\
413.01	0\\
414.01	0\\
415.01	0\\
416.01	0\\
417.01	0\\
418.01	0\\
419.01	0\\
420.01	0\\
421.01	0\\
422.01	0\\
423.01	0\\
424.01	0\\
425.01	0\\
426.01	0\\
427.01	0\\
428.01	0\\
429.01	0\\
430.01	0\\
431.01	0\\
432.01	0\\
433.01	0\\
434.01	0\\
435.01	0\\
436.01	0\\
437.01	0\\
438.01	0\\
439.01	0\\
440.01	0\\
441.01	0\\
442.01	0\\
443.01	0\\
444.01	0\\
445.01	0\\
446.01	0\\
447.01	0\\
448.01	0\\
449.01	0\\
450.01	0\\
451.01	1.73472347597681e-18\\
452.01	0\\
453.01	0\\
454.01	0\\
455.01	0\\
456.01	0\\
457.01	0\\
458.01	0\\
459.01	0\\
460.01	0\\
461.01	0\\
462.01	0\\
463.01	0\\
464.01	0\\
465.01	0\\
466.01	0\\
467.01	0\\
468.01	0\\
469.01	1.73472347597681e-18\\
470.01	0\\
471.01	0\\
472.01	0\\
473.01	0\\
474.01	0\\
475.01	0\\
476.01	1.73472347597681e-18\\
477.01	0\\
478.01	0\\
479.01	0\\
480.01	0\\
481.01	0\\
482.01	0\\
483.01	0\\
484.01	0\\
485.01	0\\
486.01	0\\
487.01	0\\
488.01	0\\
489.01	0\\
490.01	0\\
491.01	0\\
492.01	0\\
493.01	0\\
494.01	0\\
495.01	0\\
496.01	0\\
497.01	0\\
498.01	0\\
499.01	0\\
500.01	0\\
501.01	0\\
502.01	0\\
503.01	0\\
504.01	0\\
505.01	0\\
506.01	0\\
507.01	0\\
508.01	0\\
509.01	0\\
510.01	0\\
511.01	0\\
512.01	1.73472347597681e-18\\
513.01	0\\
514.01	0\\
515.01	0\\
516.01	0\\
517.01	0\\
518.01	0\\
519.01	0\\
520.01	0\\
521.01	0\\
522.01	0\\
523.01	1.73472347597681e-18\\
524.01	0\\
525.01	0\\
526.01	0\\
527.01	0\\
528.01	0\\
529.01	0\\
530.01	0\\
531.01	0\\
532.01	0\\
533.01	0\\
534.01	0\\
535.01	0\\
536.01	1.73472347597681e-18\\
537.01	1.73472347597681e-18\\
538.01	1.73472347597681e-18\\
539.01	0\\
540.01	0\\
541.01	0\\
542.01	0\\
543.01	0\\
544.01	0\\
545.01	1.73472347597681e-18\\
546.01	1.73472347597681e-18\\
547.01	0\\
548.01	0\\
549.01	0\\
550.01	0\\
551.01	0\\
552.01	0\\
553.01	0\\
554.01	0\\
555.01	0\\
556.01	1.73472347597681e-18\\
557.01	0\\
558.01	0\\
559.01	0\\
560.01	1.73472347597681e-18\\
561.01	1.73472347597681e-18\\
562.01	0\\
563.01	0\\
564.01	0\\
565.01	0\\
566.01	0\\
567.01	0\\
568.01	0\\
569.01	0\\
570.01	0\\
571.01	0\\
572.01	0\\
573.01	0\\
574.01	0\\
575.01	0\\
576.01	0\\
577.01	0\\
578.01	1.73472347597681e-18\\
579.01	0\\
580.01	0\\
581.01	0\\
582.01	0\\
583.01	1.73472347597681e-18\\
584.01	0\\
585.01	0\\
586.01	0\\
587.01	0\\
588.01	0\\
589.01	0\\
590.01	0\\
591.01	0\\
592.01	0\\
593.01	0\\
594.01	0\\
595.01	0\\
596.01	0\\
597.01	0\\
598.01	0\\
599.01	0\\
599.02	0\\
599.03	0\\
599.04	0\\
599.05	0\\
599.06	0\\
599.07	0\\
599.08	0\\
599.09	0\\
599.1	0\\
599.11	0\\
599.12	0\\
599.13	0\\
599.14	0\\
599.15	0\\
599.16	0\\
599.17	0\\
599.18	0\\
599.19	0\\
599.2	0\\
599.21	0\\
599.22	0\\
599.23	0\\
599.24	0\\
599.25	0\\
599.26	0\\
599.27	0\\
599.28	0\\
599.29	0\\
599.3	0\\
599.31	0\\
599.32	0\\
599.33	0\\
599.34	0\\
599.35	0\\
599.36	0\\
599.37	0\\
599.38	0\\
599.39	0\\
599.4	0\\
599.41	0\\
599.42	0\\
599.43	0\\
599.44	0\\
599.45	0\\
599.46	0\\
599.47	0\\
599.48	0\\
599.49	0\\
599.5	0\\
599.51	0\\
599.52	0\\
599.53	0\\
599.54	0\\
599.55	0\\
599.56	0\\
599.57	0\\
599.58	0\\
599.59	0\\
599.6	0\\
599.61	0\\
599.62	0\\
599.63	0\\
599.64	0\\
599.65	0\\
599.66	0\\
599.67	0\\
599.68	0\\
599.69	0\\
599.7	0\\
599.71	0\\
599.72	0\\
599.73	0\\
599.74	0\\
599.75	0\\
599.76	0\\
599.77	0\\
599.78	0\\
599.79	0\\
599.8	0\\
599.81	0\\
599.82	0\\
599.83	0\\
599.84	0\\
599.85	0\\
599.86	0\\
599.87	0\\
599.88	0\\
599.89	0\\
599.9	0\\
599.91	0\\
599.92	0\\
599.93	0\\
599.94	0\\
599.95	0\\
599.96	0\\
599.97	0\\
599.98	0\\
599.99	0\\
600	0\\
};
\addplot [color=mycolor7,solid,forget plot]
  table[row sep=crcr]{%
0.01	0\\
1.01	0\\
2.01	0\\
3.01	0\\
4.01	0\\
5.01	0\\
6.01	0\\
7.01	0\\
8.01	0\\
9.01	0\\
10.01	0\\
11.01	0\\
12.01	0\\
13.01	0\\
14.01	0\\
15.01	0\\
16.01	0\\
17.01	0\\
18.01	0\\
19.01	0\\
20.01	0\\
21.01	0\\
22.01	0\\
23.01	0\\
24.01	0\\
25.01	0\\
26.01	0\\
27.01	0\\
28.01	0\\
29.01	0\\
30.01	0\\
31.01	0\\
32.01	0\\
33.01	0\\
34.01	0\\
35.01	0\\
36.01	0\\
37.01	0\\
38.01	0\\
39.01	0\\
40.01	0\\
41.01	0\\
42.01	0\\
43.01	0\\
44.01	0\\
45.01	0\\
46.01	0\\
47.01	0\\
48.01	0\\
49.01	0\\
50.01	0\\
51.01	0\\
52.01	0\\
53.01	0\\
54.01	0\\
55.01	0\\
56.01	0\\
57.01	0\\
58.01	0\\
59.01	0\\
60.01	0\\
61.01	0\\
62.01	0\\
63.01	0\\
64.01	0\\
65.01	0\\
66.01	0\\
67.01	0\\
68.01	0\\
69.01	0\\
70.01	0\\
71.01	0\\
72.01	0\\
73.01	0\\
74.01	0\\
75.01	0\\
76.01	0\\
77.01	0\\
78.01	0\\
79.01	0\\
80.01	0\\
81.01	0\\
82.01	0\\
83.01	0\\
84.01	0\\
85.01	0\\
86.01	0\\
87.01	0\\
88.01	0\\
89.01	0\\
90.01	0\\
91.01	0\\
92.01	0\\
93.01	0\\
94.01	0\\
95.01	0\\
96.01	0\\
97.01	0\\
98.01	0\\
99.01	0\\
100.01	0\\
101.01	0\\
102.01	0\\
103.01	0\\
104.01	0\\
105.01	0\\
106.01	0\\
107.01	0\\
108.01	0\\
109.01	0\\
110.01	0\\
111.01	0\\
112.01	0\\
113.01	0\\
114.01	0\\
115.01	0\\
116.01	0\\
117.01	0\\
118.01	0\\
119.01	0\\
120.01	0\\
121.01	0\\
122.01	0\\
123.01	0\\
124.01	0\\
125.01	0\\
126.01	0\\
127.01	0\\
128.01	0\\
129.01	0\\
130.01	0\\
131.01	0\\
132.01	0\\
133.01	0\\
134.01	0\\
135.01	0\\
136.01	0\\
137.01	0\\
138.01	0\\
139.01	0\\
140.01	0\\
141.01	0\\
142.01	0\\
143.01	0\\
144.01	0\\
145.01	0\\
146.01	0\\
147.01	0\\
148.01	0\\
149.01	0\\
150.01	0\\
151.01	0\\
152.01	0\\
153.01	0\\
154.01	0\\
155.01	0\\
156.01	0\\
157.01	0\\
158.01	0\\
159.01	0\\
160.01	0\\
161.01	0\\
162.01	0\\
163.01	0\\
164.01	0\\
165.01	0\\
166.01	0\\
167.01	0\\
168.01	0\\
169.01	0\\
170.01	0\\
171.01	0\\
172.01	0\\
173.01	0\\
174.01	0\\
175.01	0\\
176.01	0\\
177.01	0\\
178.01	0\\
179.01	0\\
180.01	0\\
181.01	0\\
182.01	0\\
183.01	0\\
184.01	0\\
185.01	0\\
186.01	0\\
187.01	0\\
188.01	0\\
189.01	0\\
190.01	0\\
191.01	0\\
192.01	0\\
193.01	0\\
194.01	0\\
195.01	0\\
196.01	0\\
197.01	0\\
198.01	0\\
199.01	0\\
200.01	0\\
201.01	0\\
202.01	0\\
203.01	0\\
204.01	0\\
205.01	0\\
206.01	0\\
207.01	0\\
208.01	0\\
209.01	0\\
210.01	0\\
211.01	0\\
212.01	0\\
213.01	0\\
214.01	0\\
215.01	0\\
216.01	0\\
217.01	0\\
218.01	0\\
219.01	0\\
220.01	0\\
221.01	0\\
222.01	0\\
223.01	0\\
224.01	0\\
225.01	0\\
226.01	0\\
227.01	0\\
228.01	0\\
229.01	0\\
230.01	0\\
231.01	0\\
232.01	0\\
233.01	0\\
234.01	0\\
235.01	0\\
236.01	0\\
237.01	0\\
238.01	0\\
239.01	0\\
240.01	0\\
241.01	0\\
242.01	0\\
243.01	0\\
244.01	0\\
245.01	0\\
246.01	0\\
247.01	0\\
248.01	0\\
249.01	0\\
250.01	0\\
251.01	0\\
252.01	0\\
253.01	0\\
254.01	0\\
255.01	0\\
256.01	0\\
257.01	0\\
258.01	0\\
259.01	0\\
260.01	0\\
261.01	0\\
262.01	0\\
263.01	0\\
264.01	0\\
265.01	0\\
266.01	0\\
267.01	0\\
268.01	0\\
269.01	0\\
270.01	0\\
271.01	0\\
272.01	0\\
273.01	0\\
274.01	0\\
275.01	0\\
276.01	0\\
277.01	0\\
278.01	0\\
279.01	0\\
280.01	0\\
281.01	0\\
282.01	0\\
283.01	0\\
284.01	0\\
285.01	0\\
286.01	0\\
287.01	0\\
288.01	0\\
289.01	0\\
290.01	0\\
291.01	0\\
292.01	0\\
293.01	0\\
294.01	0\\
295.01	0\\
296.01	0\\
297.01	0\\
298.01	0\\
299.01	0\\
300.01	0\\
301.01	0\\
302.01	0\\
303.01	0\\
304.01	0\\
305.01	0\\
306.01	0\\
307.01	0\\
308.01	0\\
309.01	0\\
310.01	0\\
311.01	0\\
312.01	0\\
313.01	0\\
314.01	0\\
315.01	0\\
316.01	0\\
317.01	0\\
318.01	0\\
319.01	0\\
320.01	0\\
321.01	0\\
322.01	0\\
323.01	0\\
324.01	0\\
325.01	0\\
326.01	0\\
327.01	0\\
328.01	0\\
329.01	0\\
330.01	0\\
331.01	0\\
332.01	0\\
333.01	0\\
334.01	0\\
335.01	0\\
336.01	0\\
337.01	0\\
338.01	0\\
339.01	0\\
340.01	0\\
341.01	0\\
342.01	0\\
343.01	0\\
344.01	0\\
345.01	0\\
346.01	0\\
347.01	0\\
348.01	0\\
349.01	0\\
350.01	0\\
351.01	0\\
352.01	0\\
353.01	0\\
354.01	0\\
355.01	0\\
356.01	0\\
357.01	0\\
358.01	0\\
359.01	0\\
360.01	0\\
361.01	0\\
362.01	0\\
363.01	0\\
364.01	0\\
365.01	0\\
366.01	0\\
367.01	0\\
368.01	0\\
369.01	0\\
370.01	0\\
371.01	0\\
372.01	0\\
373.01	0\\
374.01	0\\
375.01	0\\
376.01	0\\
377.01	0\\
378.01	0\\
379.01	0\\
380.01	0\\
381.01	0\\
382.01	0\\
383.01	0\\
384.01	0\\
385.01	0\\
386.01	0\\
387.01	0\\
388.01	0\\
389.01	0\\
390.01	0\\
391.01	0\\
392.01	0\\
393.01	0\\
394.01	0\\
395.01	0\\
396.01	0\\
397.01	0\\
398.01	0\\
399.01	0\\
400.01	0\\
401.01	0\\
402.01	0\\
403.01	0\\
404.01	0\\
405.01	0\\
406.01	0\\
407.01	0\\
408.01	0\\
409.01	0\\
410.01	0\\
411.01	0\\
412.01	0\\
413.01	0\\
414.01	0\\
415.01	0\\
416.01	0\\
417.01	0\\
418.01	0\\
419.01	0\\
420.01	0\\
421.01	0\\
422.01	0\\
423.01	0\\
424.01	0\\
425.01	0\\
426.01	0\\
427.01	0\\
428.01	0\\
429.01	0\\
430.01	0\\
431.01	0\\
432.01	0\\
433.01	0\\
434.01	0\\
435.01	0\\
436.01	0\\
437.01	0\\
438.01	0\\
439.01	0\\
440.01	0\\
441.01	0\\
442.01	0\\
443.01	0\\
444.01	0\\
445.01	0\\
446.01	0\\
447.01	0\\
448.01	0\\
449.01	0\\
450.01	0\\
451.01	1.73472347597681e-18\\
452.01	0\\
453.01	0\\
454.01	0\\
455.01	0\\
456.01	0\\
457.01	0\\
458.01	0\\
459.01	0\\
460.01	0\\
461.01	0\\
462.01	0\\
463.01	0\\
464.01	0\\
465.01	0\\
466.01	0\\
467.01	0\\
468.01	0\\
469.01	1.73472347597681e-18\\
470.01	0\\
471.01	0\\
472.01	0\\
473.01	0\\
474.01	0\\
475.01	0\\
476.01	1.73472347597681e-18\\
477.01	0\\
478.01	0\\
479.01	0\\
480.01	0\\
481.01	0\\
482.01	0\\
483.01	0\\
484.01	0\\
485.01	0\\
486.01	0\\
487.01	0\\
488.01	0\\
489.01	0\\
490.01	0\\
491.01	0\\
492.01	0\\
493.01	0\\
494.01	0\\
495.01	0\\
496.01	0\\
497.01	0\\
498.01	0\\
499.01	0\\
500.01	0\\
501.01	0\\
502.01	0\\
503.01	0\\
504.01	0\\
505.01	0\\
506.01	0\\
507.01	0\\
508.01	0\\
509.01	0\\
510.01	0\\
511.01	0\\
512.01	1.73472347597681e-18\\
513.01	0\\
514.01	0\\
515.01	0\\
516.01	0\\
517.01	0\\
518.01	0\\
519.01	0\\
520.01	0\\
521.01	0\\
522.01	0\\
523.01	1.73472347597681e-18\\
524.01	0\\
525.01	0\\
526.01	0\\
527.01	0\\
528.01	0\\
529.01	0\\
530.01	0\\
531.01	0\\
532.01	0\\
533.01	0\\
534.01	0\\
535.01	0\\
536.01	1.73472347597681e-18\\
537.01	1.73472347597681e-18\\
538.01	1.73472347597681e-18\\
539.01	0\\
540.01	0\\
541.01	0\\
542.01	0\\
543.01	0\\
544.01	0\\
545.01	1.73472347597681e-18\\
546.01	1.73472347597681e-18\\
547.01	0\\
548.01	0\\
549.01	0\\
550.01	0\\
551.01	0\\
552.01	0\\
553.01	0\\
554.01	0\\
555.01	0\\
556.01	1.73472347597681e-18\\
557.01	0\\
558.01	0\\
559.01	0\\
560.01	1.73472347597681e-18\\
561.01	1.73472347597681e-18\\
562.01	0\\
563.01	0\\
564.01	0\\
565.01	0\\
566.01	0\\
567.01	0\\
568.01	0\\
569.01	0\\
570.01	0\\
571.01	0\\
572.01	0\\
573.01	0\\
574.01	0\\
575.01	0\\
576.01	0\\
577.01	0\\
578.01	1.73472347597681e-18\\
579.01	0\\
580.01	0\\
581.01	0\\
582.01	0\\
583.01	1.73472347597681e-18\\
584.01	0\\
585.01	0\\
586.01	0\\
587.01	0\\
588.01	0\\
589.01	0\\
590.01	0\\
591.01	0\\
592.01	0\\
593.01	0\\
594.01	0\\
595.01	0\\
596.01	0\\
597.01	0\\
598.01	0\\
599.01	0\\
599.02	0\\
599.03	0\\
599.04	0\\
599.05	0\\
599.06	0\\
599.07	0\\
599.08	0\\
599.09	0\\
599.1	0\\
599.11	0\\
599.12	0\\
599.13	0\\
599.14	0\\
599.15	0\\
599.16	0\\
599.17	0\\
599.18	0\\
599.19	0\\
599.2	0\\
599.21	0\\
599.22	0\\
599.23	0\\
599.24	0\\
599.25	0\\
599.26	0\\
599.27	0\\
599.28	0\\
599.29	0\\
599.3	0\\
599.31	0\\
599.32	0\\
599.33	0\\
599.34	0\\
599.35	0\\
599.36	0\\
599.37	0\\
599.38	0\\
599.39	0\\
599.4	0\\
599.41	0\\
599.42	0\\
599.43	0\\
599.44	0\\
599.45	0\\
599.46	0\\
599.47	0\\
599.48	0\\
599.49	0\\
599.5	0\\
599.51	0\\
599.52	0\\
599.53	0\\
599.54	0\\
599.55	0\\
599.56	0\\
599.57	0\\
599.58	0\\
599.59	0\\
599.6	0\\
599.61	0\\
599.62	0\\
599.63	0\\
599.64	0\\
599.65	0\\
599.66	0\\
599.67	0\\
599.68	0\\
599.69	0\\
599.7	0\\
599.71	0\\
599.72	0\\
599.73	0\\
599.74	0\\
599.75	0\\
599.76	0\\
599.77	0\\
599.78	0\\
599.79	0\\
599.8	0\\
599.81	0\\
599.82	0\\
599.83	0\\
599.84	0\\
599.85	0\\
599.86	0\\
599.87	0\\
599.88	0\\
599.89	0\\
599.9	0\\
599.91	0\\
599.92	0\\
599.93	0\\
599.94	0\\
599.95	0\\
599.96	0\\
599.97	0\\
599.98	0\\
599.99	0\\
600	0\\
};
\addplot [color=mycolor8,solid,forget plot]
  table[row sep=crcr]{%
0.01	0\\
1.01	0\\
2.01	0\\
3.01	0\\
4.01	0\\
5.01	0\\
6.01	0\\
7.01	0\\
8.01	0\\
9.01	0\\
10.01	0\\
11.01	0\\
12.01	0\\
13.01	0\\
14.01	0\\
15.01	0\\
16.01	0\\
17.01	0\\
18.01	0\\
19.01	0\\
20.01	0\\
21.01	0\\
22.01	0\\
23.01	0\\
24.01	0\\
25.01	0\\
26.01	0\\
27.01	0\\
28.01	0\\
29.01	0\\
30.01	0\\
31.01	0\\
32.01	0\\
33.01	0\\
34.01	0\\
35.01	0\\
36.01	0\\
37.01	0\\
38.01	0\\
39.01	0\\
40.01	0\\
41.01	0\\
42.01	0\\
43.01	0\\
44.01	0\\
45.01	0\\
46.01	0\\
47.01	0\\
48.01	0\\
49.01	0\\
50.01	0\\
51.01	0\\
52.01	0\\
53.01	0\\
54.01	0\\
55.01	0\\
56.01	0\\
57.01	0\\
58.01	0\\
59.01	0\\
60.01	0\\
61.01	0\\
62.01	0\\
63.01	0\\
64.01	0\\
65.01	0\\
66.01	0\\
67.01	0\\
68.01	0\\
69.01	0\\
70.01	0\\
71.01	0\\
72.01	0\\
73.01	0\\
74.01	0\\
75.01	0\\
76.01	0\\
77.01	0\\
78.01	0\\
79.01	0\\
80.01	0\\
81.01	0\\
82.01	0\\
83.01	0\\
84.01	0\\
85.01	0\\
86.01	0\\
87.01	0\\
88.01	0\\
89.01	0\\
90.01	0\\
91.01	0\\
92.01	0\\
93.01	0\\
94.01	0\\
95.01	0\\
96.01	0\\
97.01	0\\
98.01	0\\
99.01	0\\
100.01	0\\
101.01	0\\
102.01	0\\
103.01	0\\
104.01	0\\
105.01	0\\
106.01	0\\
107.01	0\\
108.01	0\\
109.01	0\\
110.01	0\\
111.01	0\\
112.01	0\\
113.01	0\\
114.01	0\\
115.01	0\\
116.01	0\\
117.01	0\\
118.01	0\\
119.01	0\\
120.01	0\\
121.01	0\\
122.01	0\\
123.01	0\\
124.01	0\\
125.01	0\\
126.01	0\\
127.01	0\\
128.01	0\\
129.01	0\\
130.01	0\\
131.01	0\\
132.01	0\\
133.01	0\\
134.01	0\\
135.01	0\\
136.01	0\\
137.01	0\\
138.01	0\\
139.01	0\\
140.01	0\\
141.01	0\\
142.01	0\\
143.01	0\\
144.01	0\\
145.01	0\\
146.01	0\\
147.01	0\\
148.01	0\\
149.01	0\\
150.01	0\\
151.01	0\\
152.01	0\\
153.01	0\\
154.01	0\\
155.01	0\\
156.01	0\\
157.01	0\\
158.01	0\\
159.01	0\\
160.01	0\\
161.01	0\\
162.01	0\\
163.01	0\\
164.01	0\\
165.01	0\\
166.01	0\\
167.01	0\\
168.01	0\\
169.01	0\\
170.01	0\\
171.01	0\\
172.01	0\\
173.01	0\\
174.01	0\\
175.01	0\\
176.01	0\\
177.01	0\\
178.01	0\\
179.01	0\\
180.01	0\\
181.01	0\\
182.01	0\\
183.01	0\\
184.01	0\\
185.01	0\\
186.01	0\\
187.01	0\\
188.01	0\\
189.01	0\\
190.01	0\\
191.01	0\\
192.01	0\\
193.01	0\\
194.01	0\\
195.01	0\\
196.01	0\\
197.01	0\\
198.01	0\\
199.01	0\\
200.01	0\\
201.01	0\\
202.01	0\\
203.01	0\\
204.01	0\\
205.01	0\\
206.01	0\\
207.01	0\\
208.01	0\\
209.01	0\\
210.01	0\\
211.01	0\\
212.01	0\\
213.01	0\\
214.01	0\\
215.01	0\\
216.01	0\\
217.01	0\\
218.01	0\\
219.01	0\\
220.01	0\\
221.01	0\\
222.01	0\\
223.01	0\\
224.01	0\\
225.01	0\\
226.01	0\\
227.01	0\\
228.01	0\\
229.01	0\\
230.01	0\\
231.01	0\\
232.01	0\\
233.01	0\\
234.01	0\\
235.01	0\\
236.01	0\\
237.01	0\\
238.01	0\\
239.01	0\\
240.01	0\\
241.01	0\\
242.01	0\\
243.01	0\\
244.01	0\\
245.01	0\\
246.01	0\\
247.01	0\\
248.01	0\\
249.01	0\\
250.01	0\\
251.01	0\\
252.01	0\\
253.01	0\\
254.01	0\\
255.01	0\\
256.01	0\\
257.01	0\\
258.01	0\\
259.01	0\\
260.01	0\\
261.01	0\\
262.01	0\\
263.01	0\\
264.01	0\\
265.01	0\\
266.01	0\\
267.01	0\\
268.01	0\\
269.01	0\\
270.01	0\\
271.01	0\\
272.01	0\\
273.01	0\\
274.01	0\\
275.01	0\\
276.01	0\\
277.01	0\\
278.01	0\\
279.01	0\\
280.01	0\\
281.01	0\\
282.01	0\\
283.01	0\\
284.01	0\\
285.01	0\\
286.01	0\\
287.01	0\\
288.01	0\\
289.01	0\\
290.01	0\\
291.01	0\\
292.01	0\\
293.01	0\\
294.01	0\\
295.01	0\\
296.01	0\\
297.01	0\\
298.01	0\\
299.01	0\\
300.01	0\\
301.01	0\\
302.01	0\\
303.01	0\\
304.01	0\\
305.01	0\\
306.01	0\\
307.01	0\\
308.01	0\\
309.01	0\\
310.01	0\\
311.01	0\\
312.01	0\\
313.01	0\\
314.01	0\\
315.01	0\\
316.01	0\\
317.01	0\\
318.01	0\\
319.01	0\\
320.01	0\\
321.01	0\\
322.01	0\\
323.01	0\\
324.01	0\\
325.01	0\\
326.01	0\\
327.01	0\\
328.01	0\\
329.01	0\\
330.01	0\\
331.01	0\\
332.01	0\\
333.01	0\\
334.01	0\\
335.01	0\\
336.01	0\\
337.01	0\\
338.01	0\\
339.01	0\\
340.01	0\\
341.01	0\\
342.01	0\\
343.01	0\\
344.01	0\\
345.01	0\\
346.01	0\\
347.01	0\\
348.01	0\\
349.01	0\\
350.01	0\\
351.01	0\\
352.01	0\\
353.01	0\\
354.01	0\\
355.01	0\\
356.01	0\\
357.01	0\\
358.01	0\\
359.01	0\\
360.01	0\\
361.01	0\\
362.01	0\\
363.01	0\\
364.01	0\\
365.01	0\\
366.01	0\\
367.01	0\\
368.01	0\\
369.01	0\\
370.01	0\\
371.01	0\\
372.01	0\\
373.01	0\\
374.01	0\\
375.01	0\\
376.01	0\\
377.01	0\\
378.01	0\\
379.01	0\\
380.01	0\\
381.01	0\\
382.01	0\\
383.01	0\\
384.01	0\\
385.01	0\\
386.01	0\\
387.01	0\\
388.01	0\\
389.01	0\\
390.01	0\\
391.01	0\\
392.01	0\\
393.01	0\\
394.01	0\\
395.01	0\\
396.01	0\\
397.01	0\\
398.01	0\\
399.01	0\\
400.01	0\\
401.01	0\\
402.01	0\\
403.01	0\\
404.01	0\\
405.01	0\\
406.01	0\\
407.01	0\\
408.01	0\\
409.01	0\\
410.01	0\\
411.01	0\\
412.01	0\\
413.01	0\\
414.01	0\\
415.01	0\\
416.01	0\\
417.01	0\\
418.01	0\\
419.01	0\\
420.01	0\\
421.01	0\\
422.01	0\\
423.01	0\\
424.01	0\\
425.01	0\\
426.01	0\\
427.01	0\\
428.01	0\\
429.01	0\\
430.01	0\\
431.01	0\\
432.01	0\\
433.01	0\\
434.01	0\\
435.01	0\\
436.01	0\\
437.01	0\\
438.01	0\\
439.01	0\\
440.01	0\\
441.01	0\\
442.01	0\\
443.01	0\\
444.01	0\\
445.01	0\\
446.01	0\\
447.01	0\\
448.01	0\\
449.01	0\\
450.01	0\\
451.01	1.73472347597681e-18\\
452.01	0\\
453.01	0\\
454.01	0\\
455.01	0\\
456.01	0\\
457.01	0\\
458.01	0\\
459.01	0\\
460.01	0\\
461.01	0\\
462.01	0\\
463.01	0\\
464.01	0\\
465.01	0\\
466.01	0\\
467.01	0\\
468.01	0\\
469.01	1.73472347597681e-18\\
470.01	0\\
471.01	0\\
472.01	0\\
473.01	0\\
474.01	0\\
475.01	0\\
476.01	1.73472347597681e-18\\
477.01	0\\
478.01	0\\
479.01	0\\
480.01	0\\
481.01	0\\
482.01	0\\
483.01	0\\
484.01	0\\
485.01	0\\
486.01	0\\
487.01	0\\
488.01	0\\
489.01	0\\
490.01	0\\
491.01	0\\
492.01	0\\
493.01	0\\
494.01	0\\
495.01	0\\
496.01	0\\
497.01	0\\
498.01	0\\
499.01	0\\
500.01	0\\
501.01	0\\
502.01	0\\
503.01	0\\
504.01	0\\
505.01	0\\
506.01	0\\
507.01	0\\
508.01	0\\
509.01	0\\
510.01	0\\
511.01	0\\
512.01	1.73472347597681e-18\\
513.01	0\\
514.01	0\\
515.01	0\\
516.01	0\\
517.01	0\\
518.01	0\\
519.01	0\\
520.01	0\\
521.01	0\\
522.01	0\\
523.01	1.73472347597681e-18\\
524.01	0\\
525.01	0\\
526.01	0\\
527.01	0\\
528.01	0\\
529.01	0\\
530.01	0\\
531.01	0\\
532.01	0\\
533.01	0\\
534.01	0\\
535.01	0\\
536.01	1.73472347597681e-18\\
537.01	1.73472347597681e-18\\
538.01	1.73472347597681e-18\\
539.01	0\\
540.01	0\\
541.01	0\\
542.01	0\\
543.01	0\\
544.01	0\\
545.01	1.73472347597681e-18\\
546.01	1.73472347597681e-18\\
547.01	0\\
548.01	0\\
549.01	0\\
550.01	0\\
551.01	0\\
552.01	0\\
553.01	0\\
554.01	0\\
555.01	0\\
556.01	1.73472347597681e-18\\
557.01	0\\
558.01	0\\
559.01	0\\
560.01	1.73472347597681e-18\\
561.01	1.73472347597681e-18\\
562.01	0\\
563.01	0\\
564.01	0\\
565.01	0\\
566.01	0\\
567.01	0\\
568.01	0\\
569.01	0\\
570.01	0\\
571.01	0\\
572.01	0\\
573.01	0\\
574.01	0\\
575.01	0\\
576.01	0\\
577.01	0\\
578.01	1.73472347597681e-18\\
579.01	0\\
580.01	0\\
581.01	0\\
582.01	0\\
583.01	1.73472347597681e-18\\
584.01	0\\
585.01	0\\
586.01	0\\
587.01	0\\
588.01	0\\
589.01	0\\
590.01	0\\
591.01	0\\
592.01	0\\
593.01	0\\
594.01	0\\
595.01	0\\
596.01	0\\
597.01	0\\
598.01	0\\
599.01	0\\
599.02	0\\
599.03	0\\
599.04	0\\
599.05	0\\
599.06	0\\
599.07	0\\
599.08	0\\
599.09	0\\
599.1	0\\
599.11	0\\
599.12	0\\
599.13	0\\
599.14	0\\
599.15	0\\
599.16	0\\
599.17	0\\
599.18	0\\
599.19	0\\
599.2	0\\
599.21	0\\
599.22	0\\
599.23	0\\
599.24	0\\
599.25	0\\
599.26	0\\
599.27	0\\
599.28	0\\
599.29	0\\
599.3	0\\
599.31	0\\
599.32	0\\
599.33	0\\
599.34	0\\
599.35	0\\
599.36	0\\
599.37	0\\
599.38	0\\
599.39	0\\
599.4	0\\
599.41	0\\
599.42	0\\
599.43	0\\
599.44	0\\
599.45	0\\
599.46	0\\
599.47	0\\
599.48	0\\
599.49	0\\
599.5	0\\
599.51	0\\
599.52	0\\
599.53	0\\
599.54	0\\
599.55	0\\
599.56	0\\
599.57	0\\
599.58	0\\
599.59	0\\
599.6	0\\
599.61	0\\
599.62	0\\
599.63	0\\
599.64	0\\
599.65	0\\
599.66	0\\
599.67	0\\
599.68	0\\
599.69	0\\
599.7	0\\
599.71	0\\
599.72	0\\
599.73	0\\
599.74	0\\
599.75	0\\
599.76	0\\
599.77	0\\
599.78	0\\
599.79	0\\
599.8	0\\
599.81	0\\
599.82	0\\
599.83	0\\
599.84	0\\
599.85	0\\
599.86	0\\
599.87	0\\
599.88	0\\
599.89	0\\
599.9	0\\
599.91	0\\
599.92	0\\
599.93	0\\
599.94	0\\
599.95	0\\
599.96	0\\
599.97	0\\
599.98	0\\
599.99	0\\
600	0\\
};
\addplot [color=blue!25!mycolor7,solid,forget plot]
  table[row sep=crcr]{%
0.01	0\\
1.01	0\\
2.01	0\\
3.01	0\\
4.01	0\\
5.01	0\\
6.01	0\\
7.01	0\\
8.01	0\\
9.01	0\\
10.01	0\\
11.01	0\\
12.01	0\\
13.01	0\\
14.01	0\\
15.01	0\\
16.01	0\\
17.01	0\\
18.01	0\\
19.01	0\\
20.01	0\\
21.01	0\\
22.01	0\\
23.01	0\\
24.01	0\\
25.01	0\\
26.01	0\\
27.01	0\\
28.01	0\\
29.01	0\\
30.01	0\\
31.01	0\\
32.01	0\\
33.01	0\\
34.01	0\\
35.01	0\\
36.01	0\\
37.01	0\\
38.01	0\\
39.01	0\\
40.01	0\\
41.01	0\\
42.01	0\\
43.01	0\\
44.01	0\\
45.01	0\\
46.01	0\\
47.01	0\\
48.01	0\\
49.01	0\\
50.01	0\\
51.01	0\\
52.01	0\\
53.01	0\\
54.01	0\\
55.01	0\\
56.01	0\\
57.01	0\\
58.01	0\\
59.01	0\\
60.01	0\\
61.01	0\\
62.01	0\\
63.01	0\\
64.01	0\\
65.01	0\\
66.01	0\\
67.01	0\\
68.01	0\\
69.01	0\\
70.01	0\\
71.01	0\\
72.01	0\\
73.01	0\\
74.01	0\\
75.01	0\\
76.01	0\\
77.01	0\\
78.01	0\\
79.01	0\\
80.01	0\\
81.01	0\\
82.01	0\\
83.01	0\\
84.01	0\\
85.01	0\\
86.01	0\\
87.01	0\\
88.01	0\\
89.01	0\\
90.01	0\\
91.01	0\\
92.01	0\\
93.01	0\\
94.01	0\\
95.01	0\\
96.01	0\\
97.01	0\\
98.01	0\\
99.01	0\\
100.01	0\\
101.01	0\\
102.01	0\\
103.01	0\\
104.01	0\\
105.01	0\\
106.01	0\\
107.01	0\\
108.01	0\\
109.01	0\\
110.01	0\\
111.01	0\\
112.01	0\\
113.01	0\\
114.01	0\\
115.01	0\\
116.01	0\\
117.01	0\\
118.01	0\\
119.01	0\\
120.01	0\\
121.01	0\\
122.01	0\\
123.01	0\\
124.01	0\\
125.01	0\\
126.01	0\\
127.01	0\\
128.01	0\\
129.01	0\\
130.01	0\\
131.01	0\\
132.01	0\\
133.01	0\\
134.01	0\\
135.01	0\\
136.01	0\\
137.01	0\\
138.01	0\\
139.01	0\\
140.01	0\\
141.01	0\\
142.01	0\\
143.01	0\\
144.01	0\\
145.01	0\\
146.01	0\\
147.01	0\\
148.01	0\\
149.01	0\\
150.01	0\\
151.01	0\\
152.01	0\\
153.01	0\\
154.01	0\\
155.01	0\\
156.01	0\\
157.01	0\\
158.01	0\\
159.01	0\\
160.01	0\\
161.01	0\\
162.01	0\\
163.01	0\\
164.01	0\\
165.01	0\\
166.01	0\\
167.01	0\\
168.01	0\\
169.01	0\\
170.01	0\\
171.01	0\\
172.01	0\\
173.01	0\\
174.01	0\\
175.01	0\\
176.01	0\\
177.01	0\\
178.01	0\\
179.01	0\\
180.01	0\\
181.01	0\\
182.01	0\\
183.01	0\\
184.01	0\\
185.01	0\\
186.01	0\\
187.01	0\\
188.01	0\\
189.01	0\\
190.01	0\\
191.01	0\\
192.01	0\\
193.01	0\\
194.01	0\\
195.01	0\\
196.01	0\\
197.01	0\\
198.01	0\\
199.01	0\\
200.01	0\\
201.01	0\\
202.01	0\\
203.01	0\\
204.01	0\\
205.01	0\\
206.01	0\\
207.01	0\\
208.01	0\\
209.01	0\\
210.01	0\\
211.01	0\\
212.01	0\\
213.01	0\\
214.01	0\\
215.01	0\\
216.01	0\\
217.01	0\\
218.01	0\\
219.01	0\\
220.01	0\\
221.01	0\\
222.01	0\\
223.01	0\\
224.01	0\\
225.01	0\\
226.01	0\\
227.01	0\\
228.01	0\\
229.01	0\\
230.01	0\\
231.01	0\\
232.01	0\\
233.01	0\\
234.01	0\\
235.01	0\\
236.01	0\\
237.01	0\\
238.01	0\\
239.01	0\\
240.01	0\\
241.01	0\\
242.01	0\\
243.01	0\\
244.01	0\\
245.01	0\\
246.01	0\\
247.01	0\\
248.01	0\\
249.01	0\\
250.01	0\\
251.01	0\\
252.01	0\\
253.01	0\\
254.01	0\\
255.01	0\\
256.01	0\\
257.01	0\\
258.01	0\\
259.01	0\\
260.01	0\\
261.01	0\\
262.01	0\\
263.01	0\\
264.01	0\\
265.01	0\\
266.01	0\\
267.01	0\\
268.01	0\\
269.01	0\\
270.01	0\\
271.01	0\\
272.01	0\\
273.01	0\\
274.01	0\\
275.01	0\\
276.01	0\\
277.01	0\\
278.01	0\\
279.01	0\\
280.01	0\\
281.01	0\\
282.01	0\\
283.01	0\\
284.01	0\\
285.01	0\\
286.01	0\\
287.01	0\\
288.01	0\\
289.01	0\\
290.01	0\\
291.01	0\\
292.01	0\\
293.01	0\\
294.01	0\\
295.01	0\\
296.01	0\\
297.01	0\\
298.01	0\\
299.01	0\\
300.01	0\\
301.01	0\\
302.01	0\\
303.01	0\\
304.01	0\\
305.01	0\\
306.01	0\\
307.01	0\\
308.01	0\\
309.01	0\\
310.01	0\\
311.01	0\\
312.01	0\\
313.01	0\\
314.01	0\\
315.01	0\\
316.01	0\\
317.01	0\\
318.01	0\\
319.01	0\\
320.01	0\\
321.01	0\\
322.01	0\\
323.01	0\\
324.01	0\\
325.01	0\\
326.01	0\\
327.01	0\\
328.01	0\\
329.01	0\\
330.01	0\\
331.01	0\\
332.01	0\\
333.01	0\\
334.01	0\\
335.01	0\\
336.01	0\\
337.01	0\\
338.01	0\\
339.01	0\\
340.01	0\\
341.01	0\\
342.01	0\\
343.01	0\\
344.01	0\\
345.01	0\\
346.01	0\\
347.01	0\\
348.01	0\\
349.01	0\\
350.01	0\\
351.01	0\\
352.01	0\\
353.01	0\\
354.01	0\\
355.01	0\\
356.01	0\\
357.01	0\\
358.01	0\\
359.01	0\\
360.01	0\\
361.01	0\\
362.01	0\\
363.01	0\\
364.01	0\\
365.01	0\\
366.01	0\\
367.01	0\\
368.01	0\\
369.01	0\\
370.01	0\\
371.01	0\\
372.01	0\\
373.01	0\\
374.01	0\\
375.01	0\\
376.01	0\\
377.01	0\\
378.01	0\\
379.01	0\\
380.01	0\\
381.01	0\\
382.01	0\\
383.01	0\\
384.01	0\\
385.01	0\\
386.01	0\\
387.01	0\\
388.01	0\\
389.01	0\\
390.01	0\\
391.01	0\\
392.01	0\\
393.01	0\\
394.01	0\\
395.01	0\\
396.01	0\\
397.01	0\\
398.01	0\\
399.01	0\\
400.01	0\\
401.01	0\\
402.01	0\\
403.01	0\\
404.01	0\\
405.01	0\\
406.01	0\\
407.01	0\\
408.01	0\\
409.01	0\\
410.01	0\\
411.01	0\\
412.01	0\\
413.01	0\\
414.01	0\\
415.01	0\\
416.01	0\\
417.01	0\\
418.01	0\\
419.01	0\\
420.01	0\\
421.01	0\\
422.01	0\\
423.01	0\\
424.01	0\\
425.01	0\\
426.01	0\\
427.01	0\\
428.01	0\\
429.01	0\\
430.01	0\\
431.01	0\\
432.01	0\\
433.01	0\\
434.01	0\\
435.01	0\\
436.01	0\\
437.01	0\\
438.01	0\\
439.01	0\\
440.01	0\\
441.01	0\\
442.01	0\\
443.01	0\\
444.01	0\\
445.01	0\\
446.01	0\\
447.01	0\\
448.01	0\\
449.01	0\\
450.01	0\\
451.01	1.73472347597681e-18\\
452.01	0\\
453.01	0\\
454.01	0\\
455.01	0\\
456.01	0\\
457.01	0\\
458.01	0\\
459.01	0\\
460.01	0\\
461.01	0\\
462.01	0\\
463.01	0\\
464.01	0\\
465.01	0\\
466.01	0\\
467.01	0\\
468.01	0\\
469.01	1.73472347597681e-18\\
470.01	0\\
471.01	0\\
472.01	0\\
473.01	0\\
474.01	0\\
475.01	0\\
476.01	1.73472347597681e-18\\
477.01	0\\
478.01	0\\
479.01	0\\
480.01	0\\
481.01	0\\
482.01	0\\
483.01	0\\
484.01	0\\
485.01	0\\
486.01	0\\
487.01	0\\
488.01	0\\
489.01	0\\
490.01	0\\
491.01	0\\
492.01	0\\
493.01	0\\
494.01	0\\
495.01	0\\
496.01	0\\
497.01	0\\
498.01	0\\
499.01	0\\
500.01	0\\
501.01	0\\
502.01	0\\
503.01	0\\
504.01	0\\
505.01	0\\
506.01	0\\
507.01	0\\
508.01	0\\
509.01	0\\
510.01	0\\
511.01	0\\
512.01	1.73472347597681e-18\\
513.01	0\\
514.01	0\\
515.01	0\\
516.01	0\\
517.01	0\\
518.01	0\\
519.01	0\\
520.01	0\\
521.01	0\\
522.01	0\\
523.01	1.73472347597681e-18\\
524.01	0\\
525.01	0\\
526.01	0\\
527.01	0\\
528.01	0\\
529.01	0\\
530.01	0\\
531.01	0\\
532.01	0\\
533.01	0\\
534.01	0\\
535.01	0\\
536.01	1.73472347597681e-18\\
537.01	1.73472347597681e-18\\
538.01	1.73472347597681e-18\\
539.01	0\\
540.01	0\\
541.01	0\\
542.01	0\\
543.01	0\\
544.01	0\\
545.01	1.73472347597681e-18\\
546.01	1.73472347597681e-18\\
547.01	0\\
548.01	0\\
549.01	0\\
550.01	0\\
551.01	0\\
552.01	0\\
553.01	0\\
554.01	0\\
555.01	0\\
556.01	1.73472347597681e-18\\
557.01	0\\
558.01	0\\
559.01	0\\
560.01	1.73472347597681e-18\\
561.01	1.73472347597681e-18\\
562.01	0\\
563.01	0\\
564.01	0\\
565.01	0\\
566.01	0\\
567.01	0\\
568.01	0\\
569.01	0\\
570.01	0\\
571.01	0\\
572.01	0\\
573.01	0\\
574.01	0\\
575.01	0\\
576.01	0\\
577.01	0\\
578.01	1.73472347597681e-18\\
579.01	0\\
580.01	0\\
581.01	0\\
582.01	0\\
583.01	1.73472347597681e-18\\
584.01	0\\
585.01	0\\
586.01	0\\
587.01	0\\
588.01	0\\
589.01	0\\
590.01	0\\
591.01	0\\
592.01	0\\
593.01	0\\
594.01	0\\
595.01	0\\
596.01	0\\
597.01	0\\
598.01	0\\
599.01	0\\
599.02	0\\
599.03	0\\
599.04	0\\
599.05	0\\
599.06	0\\
599.07	0\\
599.08	0\\
599.09	0\\
599.1	0\\
599.11	0\\
599.12	0\\
599.13	0\\
599.14	0\\
599.15	0\\
599.16	0\\
599.17	0\\
599.18	0\\
599.19	0\\
599.2	0\\
599.21	0\\
599.22	0\\
599.23	0\\
599.24	0\\
599.25	0\\
599.26	0\\
599.27	0\\
599.28	0\\
599.29	0\\
599.3	0\\
599.31	0\\
599.32	0\\
599.33	0\\
599.34	0\\
599.35	0\\
599.36	0\\
599.37	0\\
599.38	0\\
599.39	0\\
599.4	0\\
599.41	0\\
599.42	0\\
599.43	0\\
599.44	0\\
599.45	0\\
599.46	0\\
599.47	0\\
599.48	0\\
599.49	0\\
599.5	0\\
599.51	0\\
599.52	0\\
599.53	0\\
599.54	0\\
599.55	0\\
599.56	0\\
599.57	0\\
599.58	0\\
599.59	0\\
599.6	0\\
599.61	0\\
599.62	0\\
599.63	0\\
599.64	0\\
599.65	0\\
599.66	0\\
599.67	0\\
599.68	0\\
599.69	0\\
599.7	0\\
599.71	0\\
599.72	0\\
599.73	0\\
599.74	0\\
599.75	0\\
599.76	0\\
599.77	0\\
599.78	0\\
599.79	0\\
599.8	0\\
599.81	0\\
599.82	0\\
599.83	0\\
599.84	0\\
599.85	0\\
599.86	0\\
599.87	0\\
599.88	0\\
599.89	0\\
599.9	0\\
599.91	0\\
599.92	0\\
599.93	0\\
599.94	0\\
599.95	0\\
599.96	0\\
599.97	0\\
599.98	0\\
599.99	0\\
600	0\\
};
\addplot [color=mycolor9,solid,forget plot]
  table[row sep=crcr]{%
0.01	0\\
1.01	0\\
2.01	0\\
3.01	0\\
4.01	0\\
5.01	0\\
6.01	0\\
7.01	0\\
8.01	0\\
9.01	0\\
10.01	0\\
11.01	0\\
12.01	0\\
13.01	0\\
14.01	0\\
15.01	0\\
16.01	0\\
17.01	0\\
18.01	0\\
19.01	0\\
20.01	0\\
21.01	0\\
22.01	0\\
23.01	0\\
24.01	0\\
25.01	0\\
26.01	0\\
27.01	0\\
28.01	0\\
29.01	0\\
30.01	0\\
31.01	0\\
32.01	0\\
33.01	0\\
34.01	0\\
35.01	0\\
36.01	0\\
37.01	0\\
38.01	0\\
39.01	0\\
40.01	0\\
41.01	0\\
42.01	0\\
43.01	0\\
44.01	0\\
45.01	0\\
46.01	0\\
47.01	0\\
48.01	0\\
49.01	0\\
50.01	0\\
51.01	0\\
52.01	0\\
53.01	0\\
54.01	0\\
55.01	0\\
56.01	0\\
57.01	0\\
58.01	0\\
59.01	0\\
60.01	0\\
61.01	0\\
62.01	0\\
63.01	0\\
64.01	0\\
65.01	0\\
66.01	0\\
67.01	0\\
68.01	0\\
69.01	0\\
70.01	0\\
71.01	0\\
72.01	0\\
73.01	0\\
74.01	0\\
75.01	0\\
76.01	0\\
77.01	0\\
78.01	0\\
79.01	0\\
80.01	0\\
81.01	0\\
82.01	0\\
83.01	0\\
84.01	0\\
85.01	0\\
86.01	0\\
87.01	0\\
88.01	0\\
89.01	0\\
90.01	0\\
91.01	0\\
92.01	0\\
93.01	0\\
94.01	0\\
95.01	0\\
96.01	0\\
97.01	0\\
98.01	0\\
99.01	0\\
100.01	0\\
101.01	0\\
102.01	0\\
103.01	0\\
104.01	0\\
105.01	0\\
106.01	0\\
107.01	0\\
108.01	0\\
109.01	0\\
110.01	0\\
111.01	0\\
112.01	0\\
113.01	0\\
114.01	0\\
115.01	0\\
116.01	0\\
117.01	0\\
118.01	0\\
119.01	0\\
120.01	0\\
121.01	0\\
122.01	0\\
123.01	0\\
124.01	0\\
125.01	0\\
126.01	0\\
127.01	0\\
128.01	0\\
129.01	0\\
130.01	0\\
131.01	0\\
132.01	0\\
133.01	0\\
134.01	0\\
135.01	0\\
136.01	0\\
137.01	0\\
138.01	0\\
139.01	0\\
140.01	0\\
141.01	0\\
142.01	0\\
143.01	0\\
144.01	0\\
145.01	0\\
146.01	0\\
147.01	0\\
148.01	0\\
149.01	0\\
150.01	0\\
151.01	0\\
152.01	0\\
153.01	0\\
154.01	0\\
155.01	0\\
156.01	0\\
157.01	0\\
158.01	0\\
159.01	0\\
160.01	0\\
161.01	0\\
162.01	0\\
163.01	0\\
164.01	0\\
165.01	0\\
166.01	0\\
167.01	0\\
168.01	0\\
169.01	0\\
170.01	0\\
171.01	0\\
172.01	0\\
173.01	0\\
174.01	0\\
175.01	0\\
176.01	0\\
177.01	0\\
178.01	0\\
179.01	0\\
180.01	0\\
181.01	0\\
182.01	0\\
183.01	0\\
184.01	0\\
185.01	0\\
186.01	0\\
187.01	0\\
188.01	0\\
189.01	0\\
190.01	0\\
191.01	0\\
192.01	0\\
193.01	0\\
194.01	0\\
195.01	0\\
196.01	0\\
197.01	0\\
198.01	0\\
199.01	0\\
200.01	0\\
201.01	0\\
202.01	0\\
203.01	0\\
204.01	0\\
205.01	0\\
206.01	0\\
207.01	0\\
208.01	0\\
209.01	0\\
210.01	0\\
211.01	0\\
212.01	0\\
213.01	0\\
214.01	0\\
215.01	0\\
216.01	0\\
217.01	0\\
218.01	0\\
219.01	0\\
220.01	0\\
221.01	0\\
222.01	0\\
223.01	0\\
224.01	0\\
225.01	0\\
226.01	0\\
227.01	0\\
228.01	0\\
229.01	0\\
230.01	0\\
231.01	0\\
232.01	0\\
233.01	0\\
234.01	0\\
235.01	0\\
236.01	0\\
237.01	0\\
238.01	0\\
239.01	0\\
240.01	0\\
241.01	0\\
242.01	0\\
243.01	0\\
244.01	0\\
245.01	0\\
246.01	0\\
247.01	0\\
248.01	0\\
249.01	0\\
250.01	0\\
251.01	0\\
252.01	0\\
253.01	0\\
254.01	0\\
255.01	0\\
256.01	0\\
257.01	0\\
258.01	0\\
259.01	0\\
260.01	0\\
261.01	0\\
262.01	0\\
263.01	0\\
264.01	0\\
265.01	0\\
266.01	0\\
267.01	0\\
268.01	0\\
269.01	0\\
270.01	0\\
271.01	0\\
272.01	0\\
273.01	0\\
274.01	0\\
275.01	0\\
276.01	0\\
277.01	0\\
278.01	0\\
279.01	0\\
280.01	0\\
281.01	0\\
282.01	0\\
283.01	0\\
284.01	0\\
285.01	0\\
286.01	0\\
287.01	0\\
288.01	0\\
289.01	0\\
290.01	0\\
291.01	0\\
292.01	0\\
293.01	0\\
294.01	0\\
295.01	0\\
296.01	0\\
297.01	0\\
298.01	0\\
299.01	0\\
300.01	0\\
301.01	0\\
302.01	0\\
303.01	0\\
304.01	0\\
305.01	0\\
306.01	0\\
307.01	0\\
308.01	0\\
309.01	0\\
310.01	0\\
311.01	0\\
312.01	0\\
313.01	0\\
314.01	0\\
315.01	0\\
316.01	0\\
317.01	0\\
318.01	0\\
319.01	0\\
320.01	0\\
321.01	0\\
322.01	0\\
323.01	0\\
324.01	0\\
325.01	0\\
326.01	0\\
327.01	0\\
328.01	0\\
329.01	0\\
330.01	0\\
331.01	0\\
332.01	0\\
333.01	0\\
334.01	0\\
335.01	0\\
336.01	0\\
337.01	0\\
338.01	0\\
339.01	0\\
340.01	0\\
341.01	0\\
342.01	0\\
343.01	0\\
344.01	0\\
345.01	0\\
346.01	0\\
347.01	0\\
348.01	0\\
349.01	0\\
350.01	0\\
351.01	0\\
352.01	0\\
353.01	0\\
354.01	0\\
355.01	0\\
356.01	0\\
357.01	0\\
358.01	0\\
359.01	0\\
360.01	0\\
361.01	0\\
362.01	0\\
363.01	0\\
364.01	0\\
365.01	0\\
366.01	0\\
367.01	0\\
368.01	0\\
369.01	0\\
370.01	0\\
371.01	0\\
372.01	0\\
373.01	0\\
374.01	0\\
375.01	0\\
376.01	0\\
377.01	0\\
378.01	0\\
379.01	0\\
380.01	0\\
381.01	0\\
382.01	0\\
383.01	0\\
384.01	0\\
385.01	0\\
386.01	0\\
387.01	0\\
388.01	0\\
389.01	0\\
390.01	0\\
391.01	0\\
392.01	0\\
393.01	0\\
394.01	0\\
395.01	0\\
396.01	0\\
397.01	0\\
398.01	0\\
399.01	0\\
400.01	0\\
401.01	0\\
402.01	0\\
403.01	0\\
404.01	0\\
405.01	0\\
406.01	0\\
407.01	0\\
408.01	0\\
409.01	0\\
410.01	0\\
411.01	0\\
412.01	0\\
413.01	0\\
414.01	0\\
415.01	0\\
416.01	0\\
417.01	0\\
418.01	0\\
419.01	0\\
420.01	0\\
421.01	0\\
422.01	0\\
423.01	0\\
424.01	0\\
425.01	0\\
426.01	0\\
427.01	0\\
428.01	0\\
429.01	0\\
430.01	0\\
431.01	0\\
432.01	0\\
433.01	0\\
434.01	0\\
435.01	0\\
436.01	0\\
437.01	0\\
438.01	0\\
439.01	0\\
440.01	0\\
441.01	0\\
442.01	0\\
443.01	0\\
444.01	0\\
445.01	0\\
446.01	0\\
447.01	0\\
448.01	0\\
449.01	0\\
450.01	0\\
451.01	1.73472347597681e-18\\
452.01	0\\
453.01	0\\
454.01	0\\
455.01	0\\
456.01	0\\
457.01	0\\
458.01	0\\
459.01	0\\
460.01	0\\
461.01	0\\
462.01	0\\
463.01	0\\
464.01	0\\
465.01	0\\
466.01	0\\
467.01	0\\
468.01	0\\
469.01	1.73472347597681e-18\\
470.01	0\\
471.01	0\\
472.01	0\\
473.01	0\\
474.01	0\\
475.01	0\\
476.01	1.73472347597681e-18\\
477.01	0\\
478.01	0\\
479.01	0\\
480.01	0\\
481.01	0\\
482.01	0\\
483.01	0\\
484.01	0\\
485.01	0\\
486.01	0\\
487.01	0\\
488.01	0\\
489.01	0\\
490.01	0\\
491.01	0\\
492.01	0\\
493.01	0\\
494.01	0\\
495.01	0\\
496.01	0\\
497.01	0\\
498.01	0\\
499.01	0\\
500.01	0\\
501.01	0\\
502.01	0\\
503.01	0\\
504.01	0\\
505.01	0\\
506.01	0\\
507.01	0\\
508.01	0\\
509.01	0\\
510.01	0\\
511.01	0\\
512.01	1.73472347597681e-18\\
513.01	0\\
514.01	0\\
515.01	0\\
516.01	0\\
517.01	0\\
518.01	0\\
519.01	0\\
520.01	0\\
521.01	0\\
522.01	0\\
523.01	1.73472347597681e-18\\
524.01	0\\
525.01	0\\
526.01	0\\
527.01	0\\
528.01	0\\
529.01	0\\
530.01	0\\
531.01	0\\
532.01	0\\
533.01	0\\
534.01	0\\
535.01	0\\
536.01	1.73472347597681e-18\\
537.01	1.73472347597681e-18\\
538.01	1.73472347597681e-18\\
539.01	0\\
540.01	0\\
541.01	0\\
542.01	0\\
543.01	0\\
544.01	0\\
545.01	1.73472347597681e-18\\
546.01	1.73472347597681e-18\\
547.01	0\\
548.01	0\\
549.01	0\\
550.01	0\\
551.01	0\\
552.01	0\\
553.01	0\\
554.01	0\\
555.01	0\\
556.01	1.73472347597681e-18\\
557.01	0\\
558.01	0\\
559.01	0\\
560.01	1.73472347597681e-18\\
561.01	1.73472347597681e-18\\
562.01	0\\
563.01	0\\
564.01	0\\
565.01	0\\
566.01	0\\
567.01	0\\
568.01	0\\
569.01	0\\
570.01	0\\
571.01	0\\
572.01	0\\
573.01	0\\
574.01	0\\
575.01	0\\
576.01	0\\
577.01	0\\
578.01	1.73472347597681e-18\\
579.01	0\\
580.01	0\\
581.01	0\\
582.01	0\\
583.01	1.73472347597681e-18\\
584.01	0\\
585.01	0\\
586.01	0\\
587.01	0\\
588.01	0\\
589.01	0\\
590.01	0\\
591.01	0\\
592.01	0\\
593.01	0\\
594.01	0\\
595.01	0\\
596.01	0\\
597.01	0\\
598.01	0\\
599.01	0\\
599.02	0\\
599.03	0\\
599.04	0\\
599.05	0\\
599.06	0\\
599.07	0\\
599.08	0\\
599.09	0\\
599.1	0\\
599.11	0\\
599.12	0\\
599.13	0\\
599.14	0\\
599.15	0\\
599.16	0\\
599.17	0\\
599.18	0\\
599.19	0\\
599.2	0\\
599.21	0\\
599.22	0\\
599.23	0\\
599.24	0\\
599.25	0\\
599.26	0\\
599.27	0\\
599.28	0\\
599.29	0\\
599.3	0\\
599.31	0\\
599.32	0\\
599.33	0\\
599.34	0\\
599.35	0\\
599.36	0\\
599.37	0\\
599.38	0\\
599.39	0\\
599.4	0\\
599.41	0\\
599.42	0\\
599.43	0\\
599.44	0\\
599.45	0\\
599.46	0\\
599.47	0\\
599.48	0\\
599.49	0\\
599.5	0\\
599.51	0\\
599.52	0\\
599.53	0\\
599.54	0\\
599.55	0\\
599.56	0\\
599.57	0\\
599.58	0\\
599.59	0\\
599.6	0\\
599.61	0\\
599.62	0\\
599.63	0\\
599.64	0\\
599.65	0\\
599.66	0\\
599.67	0\\
599.68	0\\
599.69	0\\
599.7	0\\
599.71	0\\
599.72	0\\
599.73	0\\
599.74	0\\
599.75	0\\
599.76	0\\
599.77	0\\
599.78	0\\
599.79	0\\
599.8	0\\
599.81	0\\
599.82	0\\
599.83	0\\
599.84	0\\
599.85	0\\
599.86	0\\
599.87	0\\
599.88	0\\
599.89	0\\
599.9	0\\
599.91	0\\
599.92	0\\
599.93	0\\
599.94	0\\
599.95	0\\
599.96	0\\
599.97	0\\
599.98	0\\
599.99	0\\
600	0\\
};
\addplot [color=blue!50!mycolor7,solid,forget plot]
  table[row sep=crcr]{%
0.01	0\\
1.01	0\\
2.01	0\\
3.01	0\\
4.01	0\\
5.01	0\\
6.01	0\\
7.01	0\\
8.01	0\\
9.01	0\\
10.01	0\\
11.01	0\\
12.01	0\\
13.01	0\\
14.01	0\\
15.01	0\\
16.01	0\\
17.01	0\\
18.01	0\\
19.01	0\\
20.01	0\\
21.01	0\\
22.01	0\\
23.01	0\\
24.01	0\\
25.01	0\\
26.01	0\\
27.01	0\\
28.01	0\\
29.01	0\\
30.01	0\\
31.01	0\\
32.01	0\\
33.01	0\\
34.01	0\\
35.01	0\\
36.01	0\\
37.01	0\\
38.01	0\\
39.01	0\\
40.01	0\\
41.01	0\\
42.01	0\\
43.01	0\\
44.01	0\\
45.01	0\\
46.01	0\\
47.01	0\\
48.01	0\\
49.01	0\\
50.01	0\\
51.01	0\\
52.01	0\\
53.01	0\\
54.01	0\\
55.01	0\\
56.01	0\\
57.01	0\\
58.01	0\\
59.01	0\\
60.01	0\\
61.01	0\\
62.01	0\\
63.01	0\\
64.01	0\\
65.01	0\\
66.01	0\\
67.01	0\\
68.01	0\\
69.01	0\\
70.01	0\\
71.01	0\\
72.01	0\\
73.01	0\\
74.01	0\\
75.01	0\\
76.01	0\\
77.01	0\\
78.01	0\\
79.01	0\\
80.01	0\\
81.01	0\\
82.01	0\\
83.01	0\\
84.01	0\\
85.01	0\\
86.01	0\\
87.01	0\\
88.01	0\\
89.01	0\\
90.01	0\\
91.01	0\\
92.01	0\\
93.01	0\\
94.01	0\\
95.01	0\\
96.01	0\\
97.01	0\\
98.01	0\\
99.01	0\\
100.01	0\\
101.01	0\\
102.01	0\\
103.01	0\\
104.01	0\\
105.01	0\\
106.01	0\\
107.01	0\\
108.01	0\\
109.01	0\\
110.01	0\\
111.01	0\\
112.01	0\\
113.01	0\\
114.01	0\\
115.01	0\\
116.01	0\\
117.01	0\\
118.01	0\\
119.01	0\\
120.01	0\\
121.01	0\\
122.01	0\\
123.01	0\\
124.01	0\\
125.01	0\\
126.01	0\\
127.01	0\\
128.01	0\\
129.01	0\\
130.01	0\\
131.01	0\\
132.01	0\\
133.01	0\\
134.01	0\\
135.01	0\\
136.01	0\\
137.01	0\\
138.01	0\\
139.01	0\\
140.01	0\\
141.01	0\\
142.01	0\\
143.01	0\\
144.01	0\\
145.01	0\\
146.01	0\\
147.01	0\\
148.01	0\\
149.01	0\\
150.01	0\\
151.01	0\\
152.01	0\\
153.01	0\\
154.01	0\\
155.01	0\\
156.01	0\\
157.01	0\\
158.01	0\\
159.01	0\\
160.01	0\\
161.01	0\\
162.01	0\\
163.01	0\\
164.01	0\\
165.01	0\\
166.01	0\\
167.01	0\\
168.01	0\\
169.01	0\\
170.01	0\\
171.01	0\\
172.01	0\\
173.01	0\\
174.01	0\\
175.01	0\\
176.01	0\\
177.01	0\\
178.01	0\\
179.01	0\\
180.01	0\\
181.01	0\\
182.01	0\\
183.01	0\\
184.01	0\\
185.01	0\\
186.01	0\\
187.01	0\\
188.01	0\\
189.01	0\\
190.01	0\\
191.01	0\\
192.01	0\\
193.01	0\\
194.01	0\\
195.01	0\\
196.01	0\\
197.01	0\\
198.01	0\\
199.01	0\\
200.01	0\\
201.01	0\\
202.01	0\\
203.01	0\\
204.01	0\\
205.01	0\\
206.01	0\\
207.01	0\\
208.01	0\\
209.01	0\\
210.01	0\\
211.01	0\\
212.01	0\\
213.01	0\\
214.01	0\\
215.01	0\\
216.01	0\\
217.01	0\\
218.01	0\\
219.01	0\\
220.01	0\\
221.01	0\\
222.01	0\\
223.01	0\\
224.01	0\\
225.01	0\\
226.01	0\\
227.01	0\\
228.01	0\\
229.01	0\\
230.01	0\\
231.01	0\\
232.01	0\\
233.01	0\\
234.01	0\\
235.01	0\\
236.01	0\\
237.01	0\\
238.01	0\\
239.01	0\\
240.01	0\\
241.01	0\\
242.01	0\\
243.01	0\\
244.01	0\\
245.01	0\\
246.01	0\\
247.01	0\\
248.01	0\\
249.01	0\\
250.01	0\\
251.01	0\\
252.01	0\\
253.01	0\\
254.01	0\\
255.01	0\\
256.01	0\\
257.01	0\\
258.01	0\\
259.01	0\\
260.01	0\\
261.01	0\\
262.01	0\\
263.01	0\\
264.01	0\\
265.01	0\\
266.01	0\\
267.01	0\\
268.01	0\\
269.01	0\\
270.01	0\\
271.01	0\\
272.01	0\\
273.01	0\\
274.01	0\\
275.01	0\\
276.01	0\\
277.01	0\\
278.01	0\\
279.01	0\\
280.01	0\\
281.01	0\\
282.01	0\\
283.01	0\\
284.01	0\\
285.01	0\\
286.01	0\\
287.01	0\\
288.01	0\\
289.01	0\\
290.01	0\\
291.01	0\\
292.01	0\\
293.01	0\\
294.01	0\\
295.01	0\\
296.01	0\\
297.01	0\\
298.01	0\\
299.01	0\\
300.01	0\\
301.01	0\\
302.01	0\\
303.01	0\\
304.01	0\\
305.01	0\\
306.01	0\\
307.01	0\\
308.01	0\\
309.01	0\\
310.01	0\\
311.01	0\\
312.01	0\\
313.01	0\\
314.01	0\\
315.01	0\\
316.01	0\\
317.01	0\\
318.01	0\\
319.01	0\\
320.01	0\\
321.01	0\\
322.01	0\\
323.01	0\\
324.01	0\\
325.01	0\\
326.01	0\\
327.01	0\\
328.01	0\\
329.01	0\\
330.01	0\\
331.01	0\\
332.01	0\\
333.01	0\\
334.01	0\\
335.01	0\\
336.01	0\\
337.01	0\\
338.01	0\\
339.01	0\\
340.01	0\\
341.01	0\\
342.01	0\\
343.01	0\\
344.01	0\\
345.01	0\\
346.01	0\\
347.01	0\\
348.01	0\\
349.01	0\\
350.01	0\\
351.01	0\\
352.01	0\\
353.01	0\\
354.01	0\\
355.01	0\\
356.01	0\\
357.01	0\\
358.01	0\\
359.01	0\\
360.01	0\\
361.01	0\\
362.01	0\\
363.01	0\\
364.01	0\\
365.01	0\\
366.01	0\\
367.01	0\\
368.01	0\\
369.01	0\\
370.01	0\\
371.01	0\\
372.01	0\\
373.01	0\\
374.01	0\\
375.01	0\\
376.01	0\\
377.01	0\\
378.01	0\\
379.01	0\\
380.01	0\\
381.01	0\\
382.01	0\\
383.01	0\\
384.01	0\\
385.01	0\\
386.01	0\\
387.01	0\\
388.01	0\\
389.01	0\\
390.01	0\\
391.01	0\\
392.01	0\\
393.01	0\\
394.01	0\\
395.01	0\\
396.01	0\\
397.01	0\\
398.01	0\\
399.01	0\\
400.01	0\\
401.01	0\\
402.01	0\\
403.01	0\\
404.01	0\\
405.01	0\\
406.01	0\\
407.01	0\\
408.01	0\\
409.01	0\\
410.01	0\\
411.01	0\\
412.01	0\\
413.01	0\\
414.01	0\\
415.01	0\\
416.01	0\\
417.01	0\\
418.01	0\\
419.01	0\\
420.01	0\\
421.01	0\\
422.01	0\\
423.01	0\\
424.01	0\\
425.01	0\\
426.01	0\\
427.01	0\\
428.01	0\\
429.01	0\\
430.01	0\\
431.01	0\\
432.01	0\\
433.01	0\\
434.01	0\\
435.01	0\\
436.01	0\\
437.01	0\\
438.01	0\\
439.01	0\\
440.01	0\\
441.01	0\\
442.01	0\\
443.01	0\\
444.01	0\\
445.01	0\\
446.01	0\\
447.01	0\\
448.01	0\\
449.01	0\\
450.01	0\\
451.01	1.73472347597681e-18\\
452.01	0\\
453.01	0\\
454.01	0\\
455.01	0\\
456.01	0\\
457.01	0\\
458.01	0\\
459.01	0\\
460.01	0\\
461.01	0\\
462.01	0\\
463.01	0\\
464.01	0\\
465.01	0\\
466.01	0\\
467.01	0\\
468.01	0\\
469.01	1.73472347597681e-18\\
470.01	0\\
471.01	0\\
472.01	0\\
473.01	0\\
474.01	0\\
475.01	0\\
476.01	1.73472347597681e-18\\
477.01	0\\
478.01	0\\
479.01	0\\
480.01	0\\
481.01	0\\
482.01	0\\
483.01	0\\
484.01	0\\
485.01	0\\
486.01	0\\
487.01	0\\
488.01	0\\
489.01	0\\
490.01	0\\
491.01	0\\
492.01	0\\
493.01	0\\
494.01	0\\
495.01	0\\
496.01	0\\
497.01	0\\
498.01	0\\
499.01	0\\
500.01	0\\
501.01	0\\
502.01	0\\
503.01	0\\
504.01	0\\
505.01	0\\
506.01	0\\
507.01	0\\
508.01	0\\
509.01	0\\
510.01	0\\
511.01	0\\
512.01	1.73472347597681e-18\\
513.01	0\\
514.01	0\\
515.01	0\\
516.01	0\\
517.01	0\\
518.01	0\\
519.01	0\\
520.01	0\\
521.01	0\\
522.01	0\\
523.01	1.73472347597681e-18\\
524.01	0\\
525.01	0\\
526.01	0\\
527.01	0\\
528.01	0\\
529.01	0\\
530.01	0\\
531.01	0\\
532.01	0\\
533.01	0\\
534.01	0\\
535.01	0\\
536.01	1.73472347597681e-18\\
537.01	1.73472347597681e-18\\
538.01	1.73472347597681e-18\\
539.01	0\\
540.01	0\\
541.01	0\\
542.01	0\\
543.01	0\\
544.01	0\\
545.01	1.73472347597681e-18\\
546.01	1.73472347597681e-18\\
547.01	0\\
548.01	0\\
549.01	0\\
550.01	0\\
551.01	0\\
552.01	0\\
553.01	0\\
554.01	0\\
555.01	0\\
556.01	1.73472347597681e-18\\
557.01	0\\
558.01	0\\
559.01	0\\
560.01	1.73472347597681e-18\\
561.01	1.73472347597681e-18\\
562.01	0\\
563.01	0\\
564.01	0\\
565.01	0\\
566.01	0\\
567.01	0\\
568.01	0\\
569.01	0\\
570.01	0\\
571.01	0\\
572.01	0\\
573.01	0\\
574.01	0\\
575.01	0\\
576.01	0\\
577.01	0\\
578.01	1.73472347597681e-18\\
579.01	0\\
580.01	0\\
581.01	0\\
582.01	0\\
583.01	1.73472347597681e-18\\
584.01	0\\
585.01	0\\
586.01	0\\
587.01	0\\
588.01	0\\
589.01	0\\
590.01	0\\
591.01	0\\
592.01	0\\
593.01	0\\
594.01	0\\
595.01	0\\
596.01	0\\
597.01	0\\
598.01	0\\
599.01	0\\
599.02	0\\
599.03	0\\
599.04	0\\
599.05	0\\
599.06	0\\
599.07	0\\
599.08	0\\
599.09	0\\
599.1	0\\
599.11	0\\
599.12	0\\
599.13	0\\
599.14	0\\
599.15	0\\
599.16	0\\
599.17	0\\
599.18	0\\
599.19	0\\
599.2	0\\
599.21	0\\
599.22	0\\
599.23	0\\
599.24	0\\
599.25	0\\
599.26	0\\
599.27	0\\
599.28	0\\
599.29	0\\
599.3	0\\
599.31	0\\
599.32	0\\
599.33	0\\
599.34	0\\
599.35	0\\
599.36	0\\
599.37	0\\
599.38	0\\
599.39	0\\
599.4	0\\
599.41	0\\
599.42	0\\
599.43	0\\
599.44	0\\
599.45	0\\
599.46	0\\
599.47	0\\
599.48	0\\
599.49	0\\
599.5	0\\
599.51	0\\
599.52	0\\
599.53	0\\
599.54	0\\
599.55	0\\
599.56	0\\
599.57	0\\
599.58	0\\
599.59	0\\
599.6	0\\
599.61	0\\
599.62	0\\
599.63	0\\
599.64	0\\
599.65	0\\
599.66	0\\
599.67	0\\
599.68	0\\
599.69	0\\
599.7	0\\
599.71	0\\
599.72	0\\
599.73	0\\
599.74	0\\
599.75	0\\
599.76	0\\
599.77	0\\
599.78	0\\
599.79	0\\
599.8	0\\
599.81	0\\
599.82	0\\
599.83	0\\
599.84	0\\
599.85	0\\
599.86	0\\
599.87	0\\
599.88	0\\
599.89	0\\
599.9	0\\
599.91	0\\
599.92	0\\
599.93	0\\
599.94	0\\
599.95	0\\
599.96	0\\
599.97	0\\
599.98	0\\
599.99	0\\
600	0\\
};
\addplot [color=blue!40!mycolor9,solid,forget plot]
  table[row sep=crcr]{%
0.01	0\\
1.01	0\\
2.01	0\\
3.01	0\\
4.01	0\\
5.01	0\\
6.01	0\\
7.01	0\\
8.01	0\\
9.01	0\\
10.01	0\\
11.01	0\\
12.01	0\\
13.01	0\\
14.01	0\\
15.01	0\\
16.01	0\\
17.01	0\\
18.01	0\\
19.01	0\\
20.01	0\\
21.01	0\\
22.01	0\\
23.01	0\\
24.01	0\\
25.01	0\\
26.01	0\\
27.01	0\\
28.01	0\\
29.01	0\\
30.01	0\\
31.01	0\\
32.01	0\\
33.01	0\\
34.01	0\\
35.01	0\\
36.01	0\\
37.01	0\\
38.01	0\\
39.01	0\\
40.01	0\\
41.01	0\\
42.01	0\\
43.01	0\\
44.01	0\\
45.01	0\\
46.01	0\\
47.01	0\\
48.01	0\\
49.01	0\\
50.01	0\\
51.01	0\\
52.01	0\\
53.01	0\\
54.01	0\\
55.01	0\\
56.01	0\\
57.01	0\\
58.01	0\\
59.01	0\\
60.01	0\\
61.01	0\\
62.01	0\\
63.01	0\\
64.01	0\\
65.01	0\\
66.01	0\\
67.01	0\\
68.01	0\\
69.01	0\\
70.01	0\\
71.01	0\\
72.01	0\\
73.01	0\\
74.01	0\\
75.01	0\\
76.01	0\\
77.01	0\\
78.01	0\\
79.01	0\\
80.01	0\\
81.01	0\\
82.01	0\\
83.01	0\\
84.01	0\\
85.01	0\\
86.01	0\\
87.01	0\\
88.01	0\\
89.01	0\\
90.01	0\\
91.01	0\\
92.01	0\\
93.01	0\\
94.01	0\\
95.01	0\\
96.01	0\\
97.01	0\\
98.01	0\\
99.01	0\\
100.01	0\\
101.01	0\\
102.01	0\\
103.01	0\\
104.01	0\\
105.01	0\\
106.01	0\\
107.01	0\\
108.01	0\\
109.01	0\\
110.01	0\\
111.01	0\\
112.01	0\\
113.01	0\\
114.01	0\\
115.01	0\\
116.01	0\\
117.01	0\\
118.01	0\\
119.01	0\\
120.01	0\\
121.01	0\\
122.01	0\\
123.01	0\\
124.01	0\\
125.01	0\\
126.01	0\\
127.01	0\\
128.01	0\\
129.01	0\\
130.01	0\\
131.01	0\\
132.01	0\\
133.01	0\\
134.01	0\\
135.01	0\\
136.01	0\\
137.01	0\\
138.01	0\\
139.01	0\\
140.01	0\\
141.01	0\\
142.01	0\\
143.01	0\\
144.01	0\\
145.01	0\\
146.01	0\\
147.01	0\\
148.01	0\\
149.01	0\\
150.01	0\\
151.01	0\\
152.01	0\\
153.01	0\\
154.01	0\\
155.01	0\\
156.01	0\\
157.01	0\\
158.01	0\\
159.01	0\\
160.01	0\\
161.01	0\\
162.01	0\\
163.01	0\\
164.01	0\\
165.01	0\\
166.01	0\\
167.01	0\\
168.01	0\\
169.01	0\\
170.01	0\\
171.01	0\\
172.01	0\\
173.01	0\\
174.01	0\\
175.01	0\\
176.01	0\\
177.01	0\\
178.01	0\\
179.01	0\\
180.01	0\\
181.01	0\\
182.01	0\\
183.01	0\\
184.01	0\\
185.01	0\\
186.01	0\\
187.01	0\\
188.01	0\\
189.01	0\\
190.01	0\\
191.01	0\\
192.01	0\\
193.01	0\\
194.01	0\\
195.01	0\\
196.01	0\\
197.01	0\\
198.01	0\\
199.01	0\\
200.01	0\\
201.01	0\\
202.01	0\\
203.01	0\\
204.01	0\\
205.01	0\\
206.01	0\\
207.01	0\\
208.01	0\\
209.01	0\\
210.01	0\\
211.01	0\\
212.01	0\\
213.01	0\\
214.01	0\\
215.01	0\\
216.01	0\\
217.01	0\\
218.01	0\\
219.01	0\\
220.01	0\\
221.01	0\\
222.01	0\\
223.01	0\\
224.01	0\\
225.01	0\\
226.01	0\\
227.01	0\\
228.01	0\\
229.01	0\\
230.01	0\\
231.01	0\\
232.01	0\\
233.01	0\\
234.01	0\\
235.01	0\\
236.01	0\\
237.01	0\\
238.01	0\\
239.01	0\\
240.01	0\\
241.01	0\\
242.01	0\\
243.01	0\\
244.01	0\\
245.01	0\\
246.01	0\\
247.01	0\\
248.01	0\\
249.01	0\\
250.01	0\\
251.01	0\\
252.01	0\\
253.01	0\\
254.01	0\\
255.01	0\\
256.01	0\\
257.01	0\\
258.01	0\\
259.01	0\\
260.01	0\\
261.01	0\\
262.01	0\\
263.01	0\\
264.01	0\\
265.01	0\\
266.01	0\\
267.01	0\\
268.01	0\\
269.01	0\\
270.01	0\\
271.01	0\\
272.01	0\\
273.01	0\\
274.01	0\\
275.01	0\\
276.01	0\\
277.01	0\\
278.01	0\\
279.01	0\\
280.01	0\\
281.01	0\\
282.01	0\\
283.01	0\\
284.01	0\\
285.01	0\\
286.01	0\\
287.01	0\\
288.01	0\\
289.01	0\\
290.01	0\\
291.01	0\\
292.01	0\\
293.01	0\\
294.01	0\\
295.01	0\\
296.01	0\\
297.01	0\\
298.01	0\\
299.01	0\\
300.01	0\\
301.01	0\\
302.01	0\\
303.01	0\\
304.01	0\\
305.01	0\\
306.01	0\\
307.01	0\\
308.01	0\\
309.01	0\\
310.01	0\\
311.01	0\\
312.01	0\\
313.01	0\\
314.01	0\\
315.01	0\\
316.01	0\\
317.01	0\\
318.01	0\\
319.01	0\\
320.01	0\\
321.01	0\\
322.01	0\\
323.01	0\\
324.01	0\\
325.01	0\\
326.01	0\\
327.01	0\\
328.01	0\\
329.01	0\\
330.01	0\\
331.01	0\\
332.01	0\\
333.01	0\\
334.01	0\\
335.01	0\\
336.01	0\\
337.01	0\\
338.01	0\\
339.01	0\\
340.01	0\\
341.01	0\\
342.01	0\\
343.01	0\\
344.01	0\\
345.01	0\\
346.01	0\\
347.01	0\\
348.01	0\\
349.01	0\\
350.01	0\\
351.01	0\\
352.01	0\\
353.01	0\\
354.01	0\\
355.01	0\\
356.01	0\\
357.01	0\\
358.01	0\\
359.01	0\\
360.01	0\\
361.01	0\\
362.01	0\\
363.01	0\\
364.01	0\\
365.01	0\\
366.01	0\\
367.01	0\\
368.01	0\\
369.01	0\\
370.01	0\\
371.01	0\\
372.01	0\\
373.01	0\\
374.01	0\\
375.01	0\\
376.01	0\\
377.01	0\\
378.01	0\\
379.01	0\\
380.01	0\\
381.01	0\\
382.01	0\\
383.01	0\\
384.01	0\\
385.01	0\\
386.01	0\\
387.01	0\\
388.01	0\\
389.01	0\\
390.01	0\\
391.01	0\\
392.01	0\\
393.01	0\\
394.01	0\\
395.01	0\\
396.01	0\\
397.01	0\\
398.01	0\\
399.01	0\\
400.01	0\\
401.01	0\\
402.01	0\\
403.01	0\\
404.01	0\\
405.01	0\\
406.01	0\\
407.01	0\\
408.01	0\\
409.01	0\\
410.01	0\\
411.01	0\\
412.01	0\\
413.01	0\\
414.01	0\\
415.01	0\\
416.01	0\\
417.01	0\\
418.01	0\\
419.01	0\\
420.01	0\\
421.01	0\\
422.01	0\\
423.01	0\\
424.01	0\\
425.01	0\\
426.01	0\\
427.01	0\\
428.01	0\\
429.01	0\\
430.01	0\\
431.01	0\\
432.01	0\\
433.01	0\\
434.01	0\\
435.01	0\\
436.01	0\\
437.01	0\\
438.01	0\\
439.01	0\\
440.01	0\\
441.01	0\\
442.01	0\\
443.01	0\\
444.01	0\\
445.01	0\\
446.01	0\\
447.01	0\\
448.01	0\\
449.01	0\\
450.01	0\\
451.01	1.73472347597681e-18\\
452.01	0\\
453.01	0\\
454.01	0\\
455.01	0\\
456.01	0\\
457.01	0\\
458.01	0\\
459.01	0\\
460.01	0\\
461.01	0\\
462.01	0\\
463.01	0\\
464.01	0\\
465.01	0\\
466.01	0\\
467.01	0\\
468.01	0\\
469.01	1.73472347597681e-18\\
470.01	0\\
471.01	0\\
472.01	0\\
473.01	0\\
474.01	0\\
475.01	0\\
476.01	1.73472347597681e-18\\
477.01	0\\
478.01	0\\
479.01	0\\
480.01	0\\
481.01	0\\
482.01	0\\
483.01	0\\
484.01	0\\
485.01	0\\
486.01	0\\
487.01	0\\
488.01	0\\
489.01	0\\
490.01	0\\
491.01	0\\
492.01	0\\
493.01	0\\
494.01	0\\
495.01	0\\
496.01	0\\
497.01	0\\
498.01	0\\
499.01	0\\
500.01	0\\
501.01	0\\
502.01	0\\
503.01	0\\
504.01	0\\
505.01	0\\
506.01	0\\
507.01	0\\
508.01	0\\
509.01	0\\
510.01	0\\
511.01	0\\
512.01	1.73472347597681e-18\\
513.01	0\\
514.01	0\\
515.01	0\\
516.01	0\\
517.01	0\\
518.01	0\\
519.01	0\\
520.01	0\\
521.01	0\\
522.01	0\\
523.01	1.73472347597681e-18\\
524.01	0\\
525.01	0\\
526.01	0\\
527.01	0\\
528.01	0\\
529.01	0\\
530.01	0\\
531.01	0\\
532.01	0\\
533.01	0\\
534.01	0\\
535.01	0\\
536.01	1.73472347597681e-18\\
537.01	1.73472347597681e-18\\
538.01	1.73472347597681e-18\\
539.01	0\\
540.01	0\\
541.01	0\\
542.01	0\\
543.01	0\\
544.01	0\\
545.01	1.73472347597681e-18\\
546.01	1.73472347597681e-18\\
547.01	0\\
548.01	0\\
549.01	0\\
550.01	0\\
551.01	0\\
552.01	0\\
553.01	0\\
554.01	0\\
555.01	0\\
556.01	1.73472347597681e-18\\
557.01	0\\
558.01	0\\
559.01	0\\
560.01	1.73472347597681e-18\\
561.01	1.73472347597681e-18\\
562.01	0\\
563.01	0\\
564.01	0\\
565.01	0\\
566.01	0\\
567.01	0\\
568.01	0\\
569.01	0\\
570.01	0\\
571.01	0\\
572.01	0\\
573.01	0\\
574.01	0\\
575.01	0\\
576.01	0\\
577.01	0\\
578.01	1.73472347597681e-18\\
579.01	0\\
580.01	0\\
581.01	0\\
582.01	0\\
583.01	1.73472347597681e-18\\
584.01	0\\
585.01	0\\
586.01	0\\
587.01	0\\
588.01	0\\
589.01	0\\
590.01	0\\
591.01	0\\
592.01	0\\
593.01	0\\
594.01	0\\
595.01	0\\
596.01	0\\
597.01	0\\
598.01	0\\
599.01	0\\
599.02	0\\
599.03	0\\
599.04	0\\
599.05	0\\
599.06	0\\
599.07	0\\
599.08	0\\
599.09	0\\
599.1	0\\
599.11	0\\
599.12	0\\
599.13	0\\
599.14	0\\
599.15	0\\
599.16	0\\
599.17	0\\
599.18	0\\
599.19	0\\
599.2	0\\
599.21	0\\
599.22	0\\
599.23	0\\
599.24	0\\
599.25	0\\
599.26	0\\
599.27	0\\
599.28	0\\
599.29	0\\
599.3	0\\
599.31	0\\
599.32	0\\
599.33	0\\
599.34	0\\
599.35	0\\
599.36	0\\
599.37	0\\
599.38	0\\
599.39	0\\
599.4	0\\
599.41	0\\
599.42	0\\
599.43	0\\
599.44	0\\
599.45	0\\
599.46	0\\
599.47	0\\
599.48	0\\
599.49	0\\
599.5	0\\
599.51	0\\
599.52	0\\
599.53	0\\
599.54	0\\
599.55	0\\
599.56	0\\
599.57	0\\
599.58	0\\
599.59	0\\
599.6	0\\
599.61	0\\
599.62	0\\
599.63	0\\
599.64	0\\
599.65	0\\
599.66	0\\
599.67	0\\
599.68	0\\
599.69	0\\
599.7	0\\
599.71	0\\
599.72	0\\
599.73	0\\
599.74	0\\
599.75	0\\
599.76	0\\
599.77	0\\
599.78	0\\
599.79	0\\
599.8	0\\
599.81	0\\
599.82	0\\
599.83	0\\
599.84	0\\
599.85	0\\
599.86	0\\
599.87	0\\
599.88	0\\
599.89	0\\
599.9	0\\
599.91	0\\
599.92	0\\
599.93	0\\
599.94	0\\
599.95	0\\
599.96	0\\
599.97	0\\
599.98	0\\
599.99	0\\
600	0\\
};
\addplot [color=blue!75!mycolor7,solid,forget plot]
  table[row sep=crcr]{%
0.01	0\\
1.01	0\\
2.01	0\\
3.01	0\\
4.01	0\\
5.01	0\\
6.01	0\\
7.01	0\\
8.01	0\\
9.01	0\\
10.01	0\\
11.01	0\\
12.01	0\\
13.01	0\\
14.01	0\\
15.01	0\\
16.01	0\\
17.01	0\\
18.01	0\\
19.01	0\\
20.01	0\\
21.01	0\\
22.01	0\\
23.01	0\\
24.01	0\\
25.01	0\\
26.01	0\\
27.01	0\\
28.01	0\\
29.01	0\\
30.01	0\\
31.01	0\\
32.01	0\\
33.01	0\\
34.01	0\\
35.01	0\\
36.01	0\\
37.01	0\\
38.01	0\\
39.01	0\\
40.01	0\\
41.01	0\\
42.01	0\\
43.01	0\\
44.01	0\\
45.01	0\\
46.01	0\\
47.01	0\\
48.01	0\\
49.01	0\\
50.01	0\\
51.01	0\\
52.01	0\\
53.01	0\\
54.01	0\\
55.01	0\\
56.01	0\\
57.01	0\\
58.01	0\\
59.01	0\\
60.01	0\\
61.01	0\\
62.01	0\\
63.01	0\\
64.01	0\\
65.01	0\\
66.01	0\\
67.01	0\\
68.01	0\\
69.01	0\\
70.01	0\\
71.01	0\\
72.01	0\\
73.01	0\\
74.01	0\\
75.01	0\\
76.01	0\\
77.01	0\\
78.01	0\\
79.01	0\\
80.01	0\\
81.01	0\\
82.01	0\\
83.01	0\\
84.01	0\\
85.01	0\\
86.01	0\\
87.01	0\\
88.01	0\\
89.01	0\\
90.01	0\\
91.01	0\\
92.01	0\\
93.01	0\\
94.01	0\\
95.01	0\\
96.01	0\\
97.01	0\\
98.01	0\\
99.01	0\\
100.01	0\\
101.01	0\\
102.01	0\\
103.01	0\\
104.01	0\\
105.01	0\\
106.01	0\\
107.01	0\\
108.01	0\\
109.01	0\\
110.01	0\\
111.01	0\\
112.01	0\\
113.01	0\\
114.01	0\\
115.01	0\\
116.01	0\\
117.01	0\\
118.01	0\\
119.01	0\\
120.01	0\\
121.01	0\\
122.01	0\\
123.01	0\\
124.01	0\\
125.01	0\\
126.01	0\\
127.01	0\\
128.01	0\\
129.01	0\\
130.01	0\\
131.01	0\\
132.01	0\\
133.01	0\\
134.01	0\\
135.01	0\\
136.01	0\\
137.01	0\\
138.01	0\\
139.01	0\\
140.01	0\\
141.01	0\\
142.01	0\\
143.01	0\\
144.01	0\\
145.01	0\\
146.01	0\\
147.01	0\\
148.01	0\\
149.01	0\\
150.01	0\\
151.01	0\\
152.01	0\\
153.01	0\\
154.01	0\\
155.01	0\\
156.01	0\\
157.01	0\\
158.01	0\\
159.01	0\\
160.01	0\\
161.01	0\\
162.01	0\\
163.01	0\\
164.01	0\\
165.01	0\\
166.01	0\\
167.01	0\\
168.01	0\\
169.01	0\\
170.01	0\\
171.01	0\\
172.01	0\\
173.01	0\\
174.01	0\\
175.01	0\\
176.01	0\\
177.01	0\\
178.01	0\\
179.01	0\\
180.01	0\\
181.01	0\\
182.01	0\\
183.01	0\\
184.01	0\\
185.01	0\\
186.01	0\\
187.01	0\\
188.01	0\\
189.01	0\\
190.01	0\\
191.01	0\\
192.01	0\\
193.01	0\\
194.01	0\\
195.01	0\\
196.01	0\\
197.01	0\\
198.01	0\\
199.01	0\\
200.01	0\\
201.01	0\\
202.01	0\\
203.01	0\\
204.01	0\\
205.01	0\\
206.01	0\\
207.01	0\\
208.01	0\\
209.01	0\\
210.01	0\\
211.01	0\\
212.01	0\\
213.01	0\\
214.01	0\\
215.01	0\\
216.01	0\\
217.01	0\\
218.01	0\\
219.01	0\\
220.01	0\\
221.01	0\\
222.01	0\\
223.01	0\\
224.01	0\\
225.01	0\\
226.01	0\\
227.01	0\\
228.01	0\\
229.01	0\\
230.01	0\\
231.01	0\\
232.01	0\\
233.01	0\\
234.01	0\\
235.01	0\\
236.01	0\\
237.01	0\\
238.01	0\\
239.01	0\\
240.01	0\\
241.01	0\\
242.01	0\\
243.01	0\\
244.01	0\\
245.01	0\\
246.01	0\\
247.01	0\\
248.01	0\\
249.01	0\\
250.01	0\\
251.01	0\\
252.01	0\\
253.01	0\\
254.01	0\\
255.01	0\\
256.01	0\\
257.01	0\\
258.01	0\\
259.01	0\\
260.01	0\\
261.01	0\\
262.01	0\\
263.01	0\\
264.01	0\\
265.01	0\\
266.01	0\\
267.01	0\\
268.01	0\\
269.01	0\\
270.01	0\\
271.01	0\\
272.01	0\\
273.01	0\\
274.01	0\\
275.01	0\\
276.01	0\\
277.01	0\\
278.01	0\\
279.01	0\\
280.01	0\\
281.01	0\\
282.01	0\\
283.01	0\\
284.01	0\\
285.01	0\\
286.01	0\\
287.01	0\\
288.01	0\\
289.01	0\\
290.01	0\\
291.01	0\\
292.01	0\\
293.01	0\\
294.01	0\\
295.01	0\\
296.01	0\\
297.01	0\\
298.01	0\\
299.01	0\\
300.01	0\\
301.01	0\\
302.01	0\\
303.01	0\\
304.01	0\\
305.01	0\\
306.01	0\\
307.01	0\\
308.01	0\\
309.01	0\\
310.01	0\\
311.01	0\\
312.01	0\\
313.01	0\\
314.01	0\\
315.01	0\\
316.01	0\\
317.01	0\\
318.01	0\\
319.01	0\\
320.01	0\\
321.01	0\\
322.01	0\\
323.01	0\\
324.01	0\\
325.01	0\\
326.01	0\\
327.01	0\\
328.01	0\\
329.01	0\\
330.01	0\\
331.01	0\\
332.01	0\\
333.01	0\\
334.01	0\\
335.01	0\\
336.01	0\\
337.01	0\\
338.01	0\\
339.01	0\\
340.01	0\\
341.01	0\\
342.01	0\\
343.01	0\\
344.01	0\\
345.01	0\\
346.01	0\\
347.01	0\\
348.01	0\\
349.01	0\\
350.01	0\\
351.01	0\\
352.01	0\\
353.01	0\\
354.01	0\\
355.01	0\\
356.01	0\\
357.01	0\\
358.01	0\\
359.01	0\\
360.01	0\\
361.01	0\\
362.01	0\\
363.01	0\\
364.01	0\\
365.01	0\\
366.01	0\\
367.01	0\\
368.01	0\\
369.01	0\\
370.01	0\\
371.01	0\\
372.01	0\\
373.01	0\\
374.01	0\\
375.01	0\\
376.01	0\\
377.01	0\\
378.01	0\\
379.01	0\\
380.01	0\\
381.01	0\\
382.01	0\\
383.01	0\\
384.01	0\\
385.01	0\\
386.01	0\\
387.01	0\\
388.01	0\\
389.01	0\\
390.01	0\\
391.01	0\\
392.01	0\\
393.01	0\\
394.01	0\\
395.01	0\\
396.01	0\\
397.01	0\\
398.01	0\\
399.01	0\\
400.01	0\\
401.01	0\\
402.01	0\\
403.01	0\\
404.01	0\\
405.01	0\\
406.01	0\\
407.01	0\\
408.01	0\\
409.01	0\\
410.01	0\\
411.01	0\\
412.01	0\\
413.01	0\\
414.01	0\\
415.01	0\\
416.01	0\\
417.01	0\\
418.01	0\\
419.01	0\\
420.01	0\\
421.01	0\\
422.01	0\\
423.01	0\\
424.01	0\\
425.01	0\\
426.01	0\\
427.01	0\\
428.01	0\\
429.01	0\\
430.01	0\\
431.01	0\\
432.01	0\\
433.01	0\\
434.01	0\\
435.01	0\\
436.01	0\\
437.01	0\\
438.01	0\\
439.01	0\\
440.01	0\\
441.01	0\\
442.01	0\\
443.01	0\\
444.01	0\\
445.01	0\\
446.01	0\\
447.01	0\\
448.01	0\\
449.01	0\\
450.01	0\\
451.01	1.73472347597681e-18\\
452.01	0\\
453.01	0\\
454.01	0\\
455.01	0\\
456.01	0\\
457.01	0\\
458.01	0\\
459.01	0\\
460.01	0\\
461.01	0\\
462.01	0\\
463.01	0\\
464.01	0\\
465.01	0\\
466.01	0\\
467.01	0\\
468.01	0\\
469.01	1.73472347597681e-18\\
470.01	0\\
471.01	0\\
472.01	0\\
473.01	0\\
474.01	0\\
475.01	0\\
476.01	1.73472347597681e-18\\
477.01	0\\
478.01	0\\
479.01	0\\
480.01	0\\
481.01	0\\
482.01	0\\
483.01	0\\
484.01	0\\
485.01	0\\
486.01	0\\
487.01	0\\
488.01	0\\
489.01	0\\
490.01	0\\
491.01	0\\
492.01	0\\
493.01	0\\
494.01	0\\
495.01	0\\
496.01	0\\
497.01	0\\
498.01	0\\
499.01	0\\
500.01	0\\
501.01	0\\
502.01	0\\
503.01	0\\
504.01	0\\
505.01	0\\
506.01	0\\
507.01	0\\
508.01	0\\
509.01	0\\
510.01	0\\
511.01	0\\
512.01	1.73472347597681e-18\\
513.01	0\\
514.01	0\\
515.01	0\\
516.01	0\\
517.01	0\\
518.01	0\\
519.01	0\\
520.01	0\\
521.01	0\\
522.01	0\\
523.01	1.73472347597681e-18\\
524.01	0\\
525.01	0\\
526.01	0\\
527.01	0\\
528.01	0\\
529.01	0\\
530.01	0\\
531.01	0\\
532.01	0\\
533.01	0\\
534.01	0\\
535.01	0\\
536.01	1.73472347597681e-18\\
537.01	1.73472347597681e-18\\
538.01	1.73472347597681e-18\\
539.01	0\\
540.01	0\\
541.01	0\\
542.01	0\\
543.01	0\\
544.01	0\\
545.01	1.73472347597681e-18\\
546.01	1.73472347597681e-18\\
547.01	0\\
548.01	0\\
549.01	0\\
550.01	0\\
551.01	0\\
552.01	0\\
553.01	0\\
554.01	0\\
555.01	0\\
556.01	1.73472347597681e-18\\
557.01	0\\
558.01	0\\
559.01	0\\
560.01	1.73472347597681e-18\\
561.01	1.73472347597681e-18\\
562.01	0\\
563.01	0\\
564.01	0\\
565.01	0\\
566.01	0\\
567.01	0\\
568.01	0\\
569.01	0\\
570.01	0\\
571.01	0\\
572.01	0\\
573.01	0\\
574.01	0\\
575.01	0\\
576.01	0\\
577.01	0\\
578.01	1.73472347597681e-18\\
579.01	0\\
580.01	0\\
581.01	0\\
582.01	0\\
583.01	1.73472347597681e-18\\
584.01	0\\
585.01	0\\
586.01	0\\
587.01	0\\
588.01	0\\
589.01	0\\
590.01	0\\
591.01	0\\
592.01	0\\
593.01	0\\
594.01	0\\
595.01	0\\
596.01	0\\
597.01	0\\
598.01	0\\
599.01	0\\
599.02	0\\
599.03	0\\
599.04	0\\
599.05	0\\
599.06	0\\
599.07	0\\
599.08	0\\
599.09	0\\
599.1	0\\
599.11	0\\
599.12	0\\
599.13	0\\
599.14	0\\
599.15	0\\
599.16	0\\
599.17	0\\
599.18	0\\
599.19	0\\
599.2	0\\
599.21	0\\
599.22	0\\
599.23	0\\
599.24	0\\
599.25	0\\
599.26	0\\
599.27	0\\
599.28	0\\
599.29	0\\
599.3	0\\
599.31	0\\
599.32	0\\
599.33	0\\
599.34	0\\
599.35	0\\
599.36	0\\
599.37	0\\
599.38	0\\
599.39	0\\
599.4	0\\
599.41	0\\
599.42	0\\
599.43	0\\
599.44	0\\
599.45	0\\
599.46	0\\
599.47	0\\
599.48	0\\
599.49	0\\
599.5	0\\
599.51	0\\
599.52	0\\
599.53	0\\
599.54	0\\
599.55	0\\
599.56	0\\
599.57	0\\
599.58	0\\
599.59	0\\
599.6	0\\
599.61	0\\
599.62	0\\
599.63	0\\
599.64	0\\
599.65	0\\
599.66	0\\
599.67	0\\
599.68	0\\
599.69	0\\
599.7	0\\
599.71	0\\
599.72	0\\
599.73	0\\
599.74	0\\
599.75	0\\
599.76	0\\
599.77	0\\
599.78	0\\
599.79	0\\
599.8	0\\
599.81	0\\
599.82	0\\
599.83	0\\
599.84	0\\
599.85	0\\
599.86	0\\
599.87	0\\
599.88	0\\
599.89	0\\
599.9	0\\
599.91	0\\
599.92	0\\
599.93	0\\
599.94	0\\
599.95	0\\
599.96	0\\
599.97	0\\
599.98	0\\
599.99	0\\
600	0\\
};
\addplot [color=blue!80!mycolor9,solid,forget plot]
  table[row sep=crcr]{%
0.01	0\\
1.01	0\\
2.01	0\\
3.01	0\\
4.01	0\\
5.01	0\\
6.01	0\\
7.01	0\\
8.01	0\\
9.01	0\\
10.01	0\\
11.01	0\\
12.01	0\\
13.01	0\\
14.01	0\\
15.01	0\\
16.01	0\\
17.01	0\\
18.01	0\\
19.01	0\\
20.01	0\\
21.01	0\\
22.01	0\\
23.01	0\\
24.01	0\\
25.01	0\\
26.01	0\\
27.01	0\\
28.01	0\\
29.01	0\\
30.01	0\\
31.01	0\\
32.01	0\\
33.01	0\\
34.01	0\\
35.01	0\\
36.01	0\\
37.01	0\\
38.01	0\\
39.01	0\\
40.01	0\\
41.01	0\\
42.01	0\\
43.01	0\\
44.01	0\\
45.01	0\\
46.01	0\\
47.01	0\\
48.01	0\\
49.01	0\\
50.01	0\\
51.01	0\\
52.01	0\\
53.01	0\\
54.01	0\\
55.01	0\\
56.01	0\\
57.01	0\\
58.01	0\\
59.01	0\\
60.01	0\\
61.01	0\\
62.01	0\\
63.01	0\\
64.01	0\\
65.01	0\\
66.01	0\\
67.01	0\\
68.01	0\\
69.01	0\\
70.01	0\\
71.01	0\\
72.01	0\\
73.01	0\\
74.01	0\\
75.01	0\\
76.01	0\\
77.01	0\\
78.01	0\\
79.01	0\\
80.01	0\\
81.01	0\\
82.01	0\\
83.01	0\\
84.01	0\\
85.01	0\\
86.01	0\\
87.01	0\\
88.01	0\\
89.01	0\\
90.01	0\\
91.01	0\\
92.01	0\\
93.01	0\\
94.01	0\\
95.01	0\\
96.01	0\\
97.01	0\\
98.01	0\\
99.01	0\\
100.01	0\\
101.01	0\\
102.01	0\\
103.01	0\\
104.01	0\\
105.01	0\\
106.01	0\\
107.01	0\\
108.01	0\\
109.01	0\\
110.01	0\\
111.01	0\\
112.01	0\\
113.01	0\\
114.01	0\\
115.01	0\\
116.01	0\\
117.01	0\\
118.01	0\\
119.01	0\\
120.01	0\\
121.01	0\\
122.01	0\\
123.01	0\\
124.01	0\\
125.01	0\\
126.01	0\\
127.01	0\\
128.01	0\\
129.01	0\\
130.01	0\\
131.01	0\\
132.01	0\\
133.01	0\\
134.01	0\\
135.01	0\\
136.01	0\\
137.01	0\\
138.01	0\\
139.01	0\\
140.01	0\\
141.01	0\\
142.01	0\\
143.01	0\\
144.01	0\\
145.01	0\\
146.01	0\\
147.01	0\\
148.01	0\\
149.01	0\\
150.01	0\\
151.01	0\\
152.01	0\\
153.01	0\\
154.01	0\\
155.01	0\\
156.01	0\\
157.01	0\\
158.01	0\\
159.01	0\\
160.01	0\\
161.01	0\\
162.01	0\\
163.01	0\\
164.01	0\\
165.01	0\\
166.01	0\\
167.01	0\\
168.01	0\\
169.01	0\\
170.01	0\\
171.01	0\\
172.01	0\\
173.01	0\\
174.01	0\\
175.01	0\\
176.01	0\\
177.01	0\\
178.01	0\\
179.01	0\\
180.01	0\\
181.01	0\\
182.01	0\\
183.01	0\\
184.01	0\\
185.01	0\\
186.01	0\\
187.01	0\\
188.01	0\\
189.01	0\\
190.01	0\\
191.01	0\\
192.01	0\\
193.01	0\\
194.01	0\\
195.01	0\\
196.01	0\\
197.01	0\\
198.01	0\\
199.01	0\\
200.01	0\\
201.01	0\\
202.01	0\\
203.01	0\\
204.01	0\\
205.01	0\\
206.01	0\\
207.01	0\\
208.01	0\\
209.01	0\\
210.01	0\\
211.01	0\\
212.01	0\\
213.01	0\\
214.01	0\\
215.01	0\\
216.01	0\\
217.01	0\\
218.01	0\\
219.01	0\\
220.01	0\\
221.01	0\\
222.01	0\\
223.01	0\\
224.01	0\\
225.01	0\\
226.01	0\\
227.01	0\\
228.01	0\\
229.01	0\\
230.01	0\\
231.01	0\\
232.01	0\\
233.01	0\\
234.01	0\\
235.01	0\\
236.01	0\\
237.01	0\\
238.01	0\\
239.01	0\\
240.01	0\\
241.01	0\\
242.01	0\\
243.01	0\\
244.01	0\\
245.01	0\\
246.01	0\\
247.01	0\\
248.01	0\\
249.01	0\\
250.01	0\\
251.01	0\\
252.01	0\\
253.01	0\\
254.01	0\\
255.01	0\\
256.01	0\\
257.01	0\\
258.01	0\\
259.01	0\\
260.01	0\\
261.01	0\\
262.01	0\\
263.01	0\\
264.01	0\\
265.01	0\\
266.01	0\\
267.01	0\\
268.01	0\\
269.01	0\\
270.01	0\\
271.01	0\\
272.01	0\\
273.01	0\\
274.01	0\\
275.01	0\\
276.01	0\\
277.01	0\\
278.01	0\\
279.01	0\\
280.01	0\\
281.01	0\\
282.01	0\\
283.01	0\\
284.01	0\\
285.01	0\\
286.01	0\\
287.01	0\\
288.01	0\\
289.01	0\\
290.01	0\\
291.01	0\\
292.01	0\\
293.01	0\\
294.01	0\\
295.01	0\\
296.01	0\\
297.01	0\\
298.01	0\\
299.01	0\\
300.01	0\\
301.01	0\\
302.01	0\\
303.01	0\\
304.01	0\\
305.01	0\\
306.01	0\\
307.01	0\\
308.01	0\\
309.01	0\\
310.01	0\\
311.01	0\\
312.01	0\\
313.01	0\\
314.01	0\\
315.01	0\\
316.01	0\\
317.01	0\\
318.01	0\\
319.01	0\\
320.01	0\\
321.01	0\\
322.01	0\\
323.01	0\\
324.01	0\\
325.01	0\\
326.01	0\\
327.01	0\\
328.01	0\\
329.01	0\\
330.01	0\\
331.01	0\\
332.01	0\\
333.01	0\\
334.01	0\\
335.01	0\\
336.01	0\\
337.01	0\\
338.01	0\\
339.01	0\\
340.01	0\\
341.01	0\\
342.01	0\\
343.01	0\\
344.01	0\\
345.01	0\\
346.01	0\\
347.01	0\\
348.01	0\\
349.01	0\\
350.01	0\\
351.01	0\\
352.01	0\\
353.01	0\\
354.01	0\\
355.01	0\\
356.01	0\\
357.01	0\\
358.01	0\\
359.01	0\\
360.01	0\\
361.01	0\\
362.01	0\\
363.01	0\\
364.01	0\\
365.01	0\\
366.01	0\\
367.01	0\\
368.01	0\\
369.01	0\\
370.01	0\\
371.01	0\\
372.01	0\\
373.01	0\\
374.01	0\\
375.01	0\\
376.01	0\\
377.01	0\\
378.01	0\\
379.01	0\\
380.01	0\\
381.01	0\\
382.01	0\\
383.01	0\\
384.01	0\\
385.01	0\\
386.01	0\\
387.01	0\\
388.01	0\\
389.01	0\\
390.01	0\\
391.01	0\\
392.01	0\\
393.01	0\\
394.01	0\\
395.01	0\\
396.01	0\\
397.01	0\\
398.01	0\\
399.01	0\\
400.01	0\\
401.01	0\\
402.01	0\\
403.01	0\\
404.01	0\\
405.01	0\\
406.01	0\\
407.01	0\\
408.01	0\\
409.01	0\\
410.01	0\\
411.01	0\\
412.01	0\\
413.01	0\\
414.01	0\\
415.01	0\\
416.01	0\\
417.01	0\\
418.01	0\\
419.01	0\\
420.01	0\\
421.01	0\\
422.01	0\\
423.01	0\\
424.01	0\\
425.01	0\\
426.01	0\\
427.01	0\\
428.01	0\\
429.01	0\\
430.01	0\\
431.01	0\\
432.01	0\\
433.01	0\\
434.01	0\\
435.01	0\\
436.01	0\\
437.01	0\\
438.01	0\\
439.01	0\\
440.01	0\\
441.01	0\\
442.01	0\\
443.01	0\\
444.01	0\\
445.01	0\\
446.01	0\\
447.01	0\\
448.01	0\\
449.01	0\\
450.01	0\\
451.01	1.73472347597681e-18\\
452.01	0\\
453.01	0\\
454.01	0\\
455.01	0\\
456.01	0\\
457.01	0\\
458.01	0\\
459.01	0\\
460.01	0\\
461.01	0\\
462.01	0\\
463.01	0\\
464.01	0\\
465.01	0\\
466.01	0\\
467.01	0\\
468.01	0\\
469.01	1.73472347597681e-18\\
470.01	0\\
471.01	0\\
472.01	0\\
473.01	0\\
474.01	0\\
475.01	0\\
476.01	1.73472347597681e-18\\
477.01	0\\
478.01	0\\
479.01	0\\
480.01	0\\
481.01	0\\
482.01	0\\
483.01	0\\
484.01	0\\
485.01	0\\
486.01	0\\
487.01	0\\
488.01	0\\
489.01	0\\
490.01	0\\
491.01	0\\
492.01	0\\
493.01	0\\
494.01	0\\
495.01	0\\
496.01	0\\
497.01	0\\
498.01	0\\
499.01	0\\
500.01	0\\
501.01	0\\
502.01	0\\
503.01	0\\
504.01	0\\
505.01	0\\
506.01	0\\
507.01	0\\
508.01	0\\
509.01	0\\
510.01	0\\
511.01	0\\
512.01	1.73472347597681e-18\\
513.01	0\\
514.01	0\\
515.01	0\\
516.01	0\\
517.01	0\\
518.01	0\\
519.01	0\\
520.01	0\\
521.01	0\\
522.01	0\\
523.01	1.73472347597681e-18\\
524.01	0\\
525.01	0\\
526.01	0\\
527.01	0\\
528.01	0\\
529.01	0\\
530.01	0\\
531.01	0\\
532.01	0\\
533.01	0\\
534.01	0\\
535.01	0\\
536.01	1.73472347597681e-18\\
537.01	1.73472347597681e-18\\
538.01	1.73472347597681e-18\\
539.01	0\\
540.01	0\\
541.01	0\\
542.01	0\\
543.01	0\\
544.01	0\\
545.01	1.73472347597681e-18\\
546.01	1.73472347597681e-18\\
547.01	0\\
548.01	0\\
549.01	0\\
550.01	0\\
551.01	0\\
552.01	0\\
553.01	0\\
554.01	0\\
555.01	0\\
556.01	1.73472347597681e-18\\
557.01	0\\
558.01	0\\
559.01	0\\
560.01	1.73472347597681e-18\\
561.01	1.73472347597681e-18\\
562.01	0\\
563.01	0\\
564.01	0\\
565.01	0\\
566.01	0\\
567.01	0\\
568.01	0\\
569.01	0\\
570.01	0\\
571.01	0\\
572.01	0\\
573.01	0\\
574.01	0\\
575.01	0\\
576.01	0\\
577.01	0\\
578.01	1.73472347597681e-18\\
579.01	0\\
580.01	0\\
581.01	0\\
582.01	0\\
583.01	1.73472347597681e-18\\
584.01	0\\
585.01	0\\
586.01	0\\
587.01	0\\
588.01	0\\
589.01	0\\
590.01	0\\
591.01	0\\
592.01	0\\
593.01	0\\
594.01	0\\
595.01	0\\
596.01	0\\
597.01	0\\
598.01	0\\
599.01	0\\
599.02	0\\
599.03	0\\
599.04	0\\
599.05	0\\
599.06	0\\
599.07	0\\
599.08	0\\
599.09	0\\
599.1	0\\
599.11	0\\
599.12	0\\
599.13	0\\
599.14	0\\
599.15	0\\
599.16	0\\
599.17	0\\
599.18	0\\
599.19	0\\
599.2	0\\
599.21	0\\
599.22	0\\
599.23	0\\
599.24	0\\
599.25	0\\
599.26	0\\
599.27	0\\
599.28	0\\
599.29	0\\
599.3	0\\
599.31	0\\
599.32	0\\
599.33	0\\
599.34	0\\
599.35	0\\
599.36	0\\
599.37	0\\
599.38	0\\
599.39	0\\
599.4	0\\
599.41	0\\
599.42	0\\
599.43	0\\
599.44	0\\
599.45	0\\
599.46	0\\
599.47	0\\
599.48	0\\
599.49	0\\
599.5	0\\
599.51	0\\
599.52	0\\
599.53	0\\
599.54	0\\
599.55	0\\
599.56	0\\
599.57	0\\
599.58	0\\
599.59	0\\
599.6	0\\
599.61	0\\
599.62	0\\
599.63	0\\
599.64	0\\
599.65	0\\
599.66	0\\
599.67	0\\
599.68	0\\
599.69	0\\
599.7	0\\
599.71	0\\
599.72	0\\
599.73	0\\
599.74	0\\
599.75	0\\
599.76	0\\
599.77	0\\
599.78	0\\
599.79	0\\
599.8	0\\
599.81	0\\
599.82	0\\
599.83	0\\
599.84	0\\
599.85	0\\
599.86	0\\
599.87	0\\
599.88	0\\
599.89	0\\
599.9	0\\
599.91	0\\
599.92	0\\
599.93	0\\
599.94	0\\
599.95	0\\
599.96	0\\
599.97	0\\
599.98	0\\
599.99	0\\
600	0\\
};
\addplot [color=blue,solid,forget plot]
  table[row sep=crcr]{%
0.01	0\\
1.01	0\\
2.01	0\\
3.01	0\\
4.01	0\\
5.01	0\\
6.01	0\\
7.01	0\\
8.01	0\\
9.01	0\\
10.01	0\\
11.01	0\\
12.01	0\\
13.01	0\\
14.01	0\\
15.01	0\\
16.01	0\\
17.01	0\\
18.01	0\\
19.01	0\\
20.01	0\\
21.01	0\\
22.01	0\\
23.01	0\\
24.01	0\\
25.01	0\\
26.01	0\\
27.01	0\\
28.01	0\\
29.01	0\\
30.01	0\\
31.01	0\\
32.01	0\\
33.01	0\\
34.01	0\\
35.01	0\\
36.01	0\\
37.01	0\\
38.01	0\\
39.01	0\\
40.01	0\\
41.01	0\\
42.01	0\\
43.01	0\\
44.01	0\\
45.01	0\\
46.01	0\\
47.01	0\\
48.01	0\\
49.01	0\\
50.01	0\\
51.01	0\\
52.01	0\\
53.01	0\\
54.01	0\\
55.01	0\\
56.01	0\\
57.01	0\\
58.01	0\\
59.01	0\\
60.01	0\\
61.01	0\\
62.01	0\\
63.01	0\\
64.01	0\\
65.01	0\\
66.01	0\\
67.01	0\\
68.01	0\\
69.01	0\\
70.01	0\\
71.01	0\\
72.01	0\\
73.01	0\\
74.01	0\\
75.01	0\\
76.01	0\\
77.01	0\\
78.01	0\\
79.01	0\\
80.01	0\\
81.01	0\\
82.01	0\\
83.01	0\\
84.01	0\\
85.01	0\\
86.01	0\\
87.01	0\\
88.01	0\\
89.01	0\\
90.01	0\\
91.01	0\\
92.01	0\\
93.01	0\\
94.01	0\\
95.01	0\\
96.01	0\\
97.01	0\\
98.01	0\\
99.01	0\\
100.01	0\\
101.01	0\\
102.01	0\\
103.01	0\\
104.01	0\\
105.01	0\\
106.01	0\\
107.01	0\\
108.01	0\\
109.01	0\\
110.01	0\\
111.01	0\\
112.01	0\\
113.01	0\\
114.01	0\\
115.01	0\\
116.01	0\\
117.01	0\\
118.01	0\\
119.01	0\\
120.01	0\\
121.01	0\\
122.01	0\\
123.01	0\\
124.01	0\\
125.01	0\\
126.01	0\\
127.01	0\\
128.01	0\\
129.01	0\\
130.01	0\\
131.01	0\\
132.01	0\\
133.01	0\\
134.01	0\\
135.01	0\\
136.01	0\\
137.01	0\\
138.01	0\\
139.01	0\\
140.01	0\\
141.01	0\\
142.01	0\\
143.01	0\\
144.01	0\\
145.01	0\\
146.01	0\\
147.01	0\\
148.01	0\\
149.01	0\\
150.01	0\\
151.01	0\\
152.01	0\\
153.01	0\\
154.01	0\\
155.01	0\\
156.01	0\\
157.01	0\\
158.01	0\\
159.01	0\\
160.01	0\\
161.01	0\\
162.01	0\\
163.01	0\\
164.01	0\\
165.01	0\\
166.01	0\\
167.01	0\\
168.01	0\\
169.01	0\\
170.01	0\\
171.01	0\\
172.01	0\\
173.01	0\\
174.01	0\\
175.01	0\\
176.01	0\\
177.01	0\\
178.01	0\\
179.01	0\\
180.01	0\\
181.01	0\\
182.01	0\\
183.01	0\\
184.01	0\\
185.01	0\\
186.01	0\\
187.01	0\\
188.01	0\\
189.01	0\\
190.01	0\\
191.01	0\\
192.01	0\\
193.01	0\\
194.01	0\\
195.01	0\\
196.01	0\\
197.01	0\\
198.01	0\\
199.01	0\\
200.01	0\\
201.01	0\\
202.01	0\\
203.01	0\\
204.01	0\\
205.01	0\\
206.01	0\\
207.01	0\\
208.01	0\\
209.01	0\\
210.01	0\\
211.01	0\\
212.01	0\\
213.01	0\\
214.01	0\\
215.01	0\\
216.01	0\\
217.01	0\\
218.01	0\\
219.01	0\\
220.01	0\\
221.01	0\\
222.01	0\\
223.01	0\\
224.01	0\\
225.01	0\\
226.01	0\\
227.01	0\\
228.01	0\\
229.01	0\\
230.01	0\\
231.01	0\\
232.01	0\\
233.01	0\\
234.01	0\\
235.01	0\\
236.01	0\\
237.01	0\\
238.01	0\\
239.01	0\\
240.01	0\\
241.01	0\\
242.01	0\\
243.01	0\\
244.01	0\\
245.01	0\\
246.01	0\\
247.01	0\\
248.01	0\\
249.01	0\\
250.01	0\\
251.01	0\\
252.01	0\\
253.01	0\\
254.01	0\\
255.01	0\\
256.01	0\\
257.01	0\\
258.01	0\\
259.01	0\\
260.01	0\\
261.01	0\\
262.01	0\\
263.01	0\\
264.01	0\\
265.01	0\\
266.01	0\\
267.01	0\\
268.01	0\\
269.01	0\\
270.01	0\\
271.01	0\\
272.01	0\\
273.01	0\\
274.01	0\\
275.01	0\\
276.01	0\\
277.01	0\\
278.01	0\\
279.01	0\\
280.01	0\\
281.01	0\\
282.01	0\\
283.01	0\\
284.01	0\\
285.01	0\\
286.01	0\\
287.01	0\\
288.01	0\\
289.01	0\\
290.01	0\\
291.01	0\\
292.01	0\\
293.01	0\\
294.01	0\\
295.01	0\\
296.01	0\\
297.01	0\\
298.01	0\\
299.01	0\\
300.01	0\\
301.01	0\\
302.01	0\\
303.01	0\\
304.01	0\\
305.01	0\\
306.01	0\\
307.01	0\\
308.01	0\\
309.01	0\\
310.01	0\\
311.01	0\\
312.01	0\\
313.01	0\\
314.01	0\\
315.01	0\\
316.01	0\\
317.01	0\\
318.01	0\\
319.01	0\\
320.01	0\\
321.01	0\\
322.01	0\\
323.01	0\\
324.01	0\\
325.01	0\\
326.01	0\\
327.01	0\\
328.01	0\\
329.01	0\\
330.01	0\\
331.01	0\\
332.01	0\\
333.01	0\\
334.01	0\\
335.01	0\\
336.01	0\\
337.01	0\\
338.01	0\\
339.01	0\\
340.01	0\\
341.01	0\\
342.01	0\\
343.01	0\\
344.01	0\\
345.01	0\\
346.01	0\\
347.01	0\\
348.01	0\\
349.01	0\\
350.01	0\\
351.01	0\\
352.01	0\\
353.01	0\\
354.01	0\\
355.01	0\\
356.01	0\\
357.01	0\\
358.01	0\\
359.01	0\\
360.01	0\\
361.01	0\\
362.01	0\\
363.01	0\\
364.01	0\\
365.01	0\\
366.01	0\\
367.01	0\\
368.01	0\\
369.01	0\\
370.01	0\\
371.01	0\\
372.01	0\\
373.01	0\\
374.01	0\\
375.01	0\\
376.01	0\\
377.01	0\\
378.01	0\\
379.01	0\\
380.01	0\\
381.01	0\\
382.01	0\\
383.01	0\\
384.01	0\\
385.01	0\\
386.01	0\\
387.01	0\\
388.01	0\\
389.01	0\\
390.01	0\\
391.01	0\\
392.01	0\\
393.01	0\\
394.01	0\\
395.01	0\\
396.01	0\\
397.01	0\\
398.01	0\\
399.01	0\\
400.01	0\\
401.01	0\\
402.01	0\\
403.01	0\\
404.01	0\\
405.01	0\\
406.01	0\\
407.01	0\\
408.01	0\\
409.01	0\\
410.01	0\\
411.01	0\\
412.01	0\\
413.01	0\\
414.01	0\\
415.01	0\\
416.01	0\\
417.01	0\\
418.01	0\\
419.01	0\\
420.01	0\\
421.01	0\\
422.01	0\\
423.01	0\\
424.01	0\\
425.01	0\\
426.01	0\\
427.01	0\\
428.01	0\\
429.01	0\\
430.01	0\\
431.01	0\\
432.01	0\\
433.01	0\\
434.01	0\\
435.01	0\\
436.01	0\\
437.01	0\\
438.01	0\\
439.01	0\\
440.01	0\\
441.01	0\\
442.01	0\\
443.01	0\\
444.01	0\\
445.01	0\\
446.01	0\\
447.01	0\\
448.01	0\\
449.01	0\\
450.01	0\\
451.01	1.73472347597681e-18\\
452.01	0\\
453.01	0\\
454.01	0\\
455.01	0\\
456.01	0\\
457.01	0\\
458.01	0\\
459.01	0\\
460.01	0\\
461.01	0\\
462.01	0\\
463.01	0\\
464.01	0\\
465.01	0\\
466.01	0\\
467.01	0\\
468.01	0\\
469.01	1.73472347597681e-18\\
470.01	0\\
471.01	0\\
472.01	0\\
473.01	0\\
474.01	0\\
475.01	0\\
476.01	1.73472347597681e-18\\
477.01	0\\
478.01	0\\
479.01	0\\
480.01	0\\
481.01	0\\
482.01	0\\
483.01	0\\
484.01	0\\
485.01	0\\
486.01	0\\
487.01	0\\
488.01	0\\
489.01	0\\
490.01	0\\
491.01	0\\
492.01	0\\
493.01	0\\
494.01	0\\
495.01	0\\
496.01	0\\
497.01	0\\
498.01	0\\
499.01	0\\
500.01	0\\
501.01	0\\
502.01	0\\
503.01	0\\
504.01	0\\
505.01	0\\
506.01	0\\
507.01	0\\
508.01	0\\
509.01	0\\
510.01	0\\
511.01	0\\
512.01	1.73472347597681e-18\\
513.01	0\\
514.01	0\\
515.01	0\\
516.01	0\\
517.01	0\\
518.01	0\\
519.01	0\\
520.01	0\\
521.01	0\\
522.01	0\\
523.01	1.73472347597681e-18\\
524.01	0\\
525.01	0\\
526.01	0\\
527.01	0\\
528.01	0\\
529.01	0\\
530.01	0\\
531.01	0\\
532.01	0\\
533.01	0\\
534.01	0\\
535.01	0\\
536.01	1.73472347597681e-18\\
537.01	1.73472347597681e-18\\
538.01	1.73472347597681e-18\\
539.01	0\\
540.01	0\\
541.01	0\\
542.01	0\\
543.01	0\\
544.01	0\\
545.01	1.73472347597681e-18\\
546.01	1.73472347597681e-18\\
547.01	0\\
548.01	0\\
549.01	0\\
550.01	0\\
551.01	0\\
552.01	0\\
553.01	0\\
554.01	0\\
555.01	0\\
556.01	1.73472347597681e-18\\
557.01	0\\
558.01	0\\
559.01	0\\
560.01	1.73472347597681e-18\\
561.01	1.73472347597681e-18\\
562.01	0\\
563.01	0\\
564.01	0\\
565.01	0\\
566.01	0\\
567.01	0\\
568.01	0\\
569.01	0\\
570.01	0\\
571.01	0\\
572.01	0\\
573.01	0\\
574.01	0\\
575.01	0\\
576.01	0\\
577.01	0\\
578.01	1.73472347597681e-18\\
579.01	0\\
580.01	0\\
581.01	0\\
582.01	0\\
583.01	1.73472347597681e-18\\
584.01	0\\
585.01	0\\
586.01	0\\
587.01	0\\
588.01	0\\
589.01	0\\
590.01	0\\
591.01	0\\
592.01	0\\
593.01	0\\
594.01	0\\
595.01	0\\
596.01	0\\
597.01	0\\
598.01	0\\
599.01	0\\
599.02	0\\
599.03	0\\
599.04	0\\
599.05	0\\
599.06	0\\
599.07	0\\
599.08	0\\
599.09	0\\
599.1	0\\
599.11	0\\
599.12	0\\
599.13	0\\
599.14	0\\
599.15	0\\
599.16	0\\
599.17	0\\
599.18	0\\
599.19	0\\
599.2	0\\
599.21	0\\
599.22	0\\
599.23	0\\
599.24	0\\
599.25	0\\
599.26	0\\
599.27	0\\
599.28	0\\
599.29	0\\
599.3	0\\
599.31	0\\
599.32	0\\
599.33	0\\
599.34	0\\
599.35	0\\
599.36	0\\
599.37	0\\
599.38	0\\
599.39	0\\
599.4	0\\
599.41	0\\
599.42	0\\
599.43	0\\
599.44	0\\
599.45	0\\
599.46	0\\
599.47	0\\
599.48	0\\
599.49	0\\
599.5	0\\
599.51	0\\
599.52	0\\
599.53	0\\
599.54	0\\
599.55	0\\
599.56	0\\
599.57	0\\
599.58	0\\
599.59	0\\
599.6	0\\
599.61	0\\
599.62	0\\
599.63	0\\
599.64	0\\
599.65	0\\
599.66	0\\
599.67	0\\
599.68	0\\
599.69	0\\
599.7	0\\
599.71	0\\
599.72	0\\
599.73	0\\
599.74	0\\
599.75	0\\
599.76	0\\
599.77	0\\
599.78	0\\
599.79	0\\
599.8	0\\
599.81	0\\
599.82	0\\
599.83	0\\
599.84	0\\
599.85	0\\
599.86	0\\
599.87	0\\
599.88	0\\
599.89	0\\
599.9	0\\
599.91	0\\
599.92	0\\
599.93	0\\
599.94	0\\
599.95	0\\
599.96	0\\
599.97	0\\
599.98	0\\
599.99	0\\
600	0\\
};
\addplot [color=mycolor10,solid,forget plot]
  table[row sep=crcr]{%
0.01	3.6332990175629e-05\\
1.01	3.6332990175629e-05\\
2.01	3.6332990175629e-05\\
3.01	3.6332990175629e-05\\
4.01	3.6332990175629e-05\\
5.01	3.6332990175629e-05\\
6.01	3.6332990175629e-05\\
7.01	3.6332990175629e-05\\
8.01	3.6332990175629e-05\\
9.01	3.6332990175629e-05\\
10.01	3.6332990175629e-05\\
11.01	3.6332990175629e-05\\
12.01	3.6332990175629e-05\\
13.01	3.6332990175629e-05\\
14.01	3.6332990175629e-05\\
15.01	3.6332990175629e-05\\
16.01	3.6332990175629e-05\\
17.01	3.6332990175629e-05\\
18.01	3.6332990175629e-05\\
19.01	3.6332990175629e-05\\
20.01	3.6332990175629e-05\\
21.01	3.6332990175629e-05\\
22.01	3.6332990175629e-05\\
23.01	3.6332990175629e-05\\
24.01	3.6332990175629e-05\\
25.01	3.6332990175629e-05\\
26.01	3.6332990175629e-05\\
27.01	3.6332990175629e-05\\
28.01	3.6332990175629e-05\\
29.01	3.6332990175629e-05\\
30.01	3.6332990175629e-05\\
31.01	3.6332990175629e-05\\
32.01	3.6332990175629e-05\\
33.01	3.6332990175629e-05\\
34.01	3.6332990175629e-05\\
35.01	3.6332990175629e-05\\
36.01	3.6332990175629e-05\\
37.01	3.6332990175629e-05\\
38.01	3.6332990175629e-05\\
39.01	3.6332990175629e-05\\
40.01	3.6332990175629e-05\\
41.01	3.6332990175629e-05\\
42.01	3.6332990175629e-05\\
43.01	3.6332990175629e-05\\
44.01	3.6332990175629e-05\\
45.01	3.6332990175629e-05\\
46.01	3.6332990175629e-05\\
47.01	3.6332990175629e-05\\
48.01	3.6332990175629e-05\\
49.01	3.6332990175629e-05\\
50.01	3.6332990175629e-05\\
51.01	3.6332990175629e-05\\
52.01	3.6332990175629e-05\\
53.01	3.6332990175629e-05\\
54.01	3.6332990175629e-05\\
55.01	3.6332990175629e-05\\
56.01	3.6332990175629e-05\\
57.01	3.6332990175629e-05\\
58.01	3.6332990175629e-05\\
59.01	3.6332990175629e-05\\
60.01	3.6332990175629e-05\\
61.01	3.6332990175629e-05\\
62.01	3.6332990175629e-05\\
63.01	3.6332990175629e-05\\
64.01	3.6332990175629e-05\\
65.01	3.6332990175629e-05\\
66.01	3.6332990175629e-05\\
67.01	3.6332990175629e-05\\
68.01	3.6332990175629e-05\\
69.01	3.6332990175629e-05\\
70.01	3.6332990175629e-05\\
71.01	3.6332990175629e-05\\
72.01	3.6332990175629e-05\\
73.01	3.6332990175629e-05\\
74.01	3.6332990175629e-05\\
75.01	3.6332990175629e-05\\
76.01	3.6332990175629e-05\\
77.01	3.6332990175629e-05\\
78.01	3.6332990175629e-05\\
79.01	3.6332990175629e-05\\
80.01	3.6332990175629e-05\\
81.01	3.6332990175629e-05\\
82.01	3.6332990175629e-05\\
83.01	3.6332990175629e-05\\
84.01	3.6332990175629e-05\\
85.01	3.6332990175629e-05\\
86.01	3.6332990175629e-05\\
87.01	3.6332990175629e-05\\
88.01	3.6332990175629e-05\\
89.01	3.6332990175629e-05\\
90.01	3.6332990175629e-05\\
91.01	3.6332990175629e-05\\
92.01	3.6332990175629e-05\\
93.01	3.6332990175629e-05\\
94.01	3.6332990175629e-05\\
95.01	3.6332990175629e-05\\
96.01	3.6332990175629e-05\\
97.01	3.6332990175629e-05\\
98.01	3.6332990175629e-05\\
99.01	3.6332990175629e-05\\
100.01	3.6332990175629e-05\\
101.01	3.6332990175629e-05\\
102.01	3.6332990175629e-05\\
103.01	3.6332990175629e-05\\
104.01	3.6332990175629e-05\\
105.01	3.6332990175629e-05\\
106.01	3.6332990175629e-05\\
107.01	3.6332990175629e-05\\
108.01	3.6332990175629e-05\\
109.01	3.6332990175629e-05\\
110.01	3.6332990175629e-05\\
111.01	3.6332990175629e-05\\
112.01	3.6332990175629e-05\\
113.01	3.6332990175629e-05\\
114.01	3.6332990175629e-05\\
115.01	3.6332990175629e-05\\
116.01	3.6332990175629e-05\\
117.01	3.6332990175629e-05\\
118.01	3.6332990175629e-05\\
119.01	3.6332990175629e-05\\
120.01	3.6332990175629e-05\\
121.01	3.6332990175629e-05\\
122.01	3.6332990175629e-05\\
123.01	3.6332990175629e-05\\
124.01	3.6332990175629e-05\\
125.01	3.6332990175629e-05\\
126.01	3.6332990175629e-05\\
127.01	3.6332990175629e-05\\
128.01	3.6332990175629e-05\\
129.01	3.6332990175629e-05\\
130.01	3.6332990175629e-05\\
131.01	3.6332990175629e-05\\
132.01	3.6332990175629e-05\\
133.01	3.6332990175629e-05\\
134.01	3.6332990175629e-05\\
135.01	3.6332990175629e-05\\
136.01	3.6332990175629e-05\\
137.01	3.6332990175629e-05\\
138.01	3.6332990175629e-05\\
139.01	3.6332990175629e-05\\
140.01	3.6332990175629e-05\\
141.01	3.6332990175629e-05\\
142.01	3.6332990175629e-05\\
143.01	3.6332990175629e-05\\
144.01	3.6332990175629e-05\\
145.01	3.6332990175629e-05\\
146.01	3.6332990175629e-05\\
147.01	3.6332990175629e-05\\
148.01	3.6332990175629e-05\\
149.01	3.6332990175629e-05\\
150.01	3.6332990175629e-05\\
151.01	3.6332990175629e-05\\
152.01	3.6332990175629e-05\\
153.01	3.6332990175629e-05\\
154.01	3.6332990175629e-05\\
155.01	3.6332990175629e-05\\
156.01	3.6332990175629e-05\\
157.01	3.6332990175629e-05\\
158.01	3.6332990175629e-05\\
159.01	3.6332990175629e-05\\
160.01	3.6332990175629e-05\\
161.01	3.6332990175629e-05\\
162.01	3.6332990175629e-05\\
163.01	3.6332990175629e-05\\
164.01	3.6332990175629e-05\\
165.01	3.6332990175629e-05\\
166.01	3.6332990175629e-05\\
167.01	3.6332990175629e-05\\
168.01	3.6332990175629e-05\\
169.01	3.6332990175629e-05\\
170.01	3.6332990175629e-05\\
171.01	3.6332990175629e-05\\
172.01	3.6332990175629e-05\\
173.01	3.6332990175629e-05\\
174.01	3.6332990175629e-05\\
175.01	3.6332990175629e-05\\
176.01	3.6332990175629e-05\\
177.01	3.6332990175629e-05\\
178.01	3.6332990175629e-05\\
179.01	3.6332990175629e-05\\
180.01	3.6332990175629e-05\\
181.01	3.6332990175629e-05\\
182.01	3.6332990175629e-05\\
183.01	3.6332990175629e-05\\
184.01	3.6332990175629e-05\\
185.01	3.6332990175629e-05\\
186.01	3.6332990175629e-05\\
187.01	3.6332990175629e-05\\
188.01	3.6332990175629e-05\\
189.01	3.6332990175629e-05\\
190.01	3.6332990175629e-05\\
191.01	3.6332990175629e-05\\
192.01	3.6332990175629e-05\\
193.01	3.6332990175629e-05\\
194.01	3.6332990175629e-05\\
195.01	3.6332990175629e-05\\
196.01	3.6332990175629e-05\\
197.01	3.6332990175629e-05\\
198.01	3.6332990175629e-05\\
199.01	3.6332990175629e-05\\
200.01	3.6332990175629e-05\\
201.01	3.6332990175629e-05\\
202.01	3.6332990175629e-05\\
203.01	3.6332990175629e-05\\
204.01	3.6332990175629e-05\\
205.01	3.6332990175629e-05\\
206.01	3.6332990175629e-05\\
207.01	3.6332990175629e-05\\
208.01	3.6332990175629e-05\\
209.01	3.6332990175629e-05\\
210.01	3.6332990175629e-05\\
211.01	3.6332990175629e-05\\
212.01	3.6332990175629e-05\\
213.01	3.6332990175629e-05\\
214.01	3.6332990175629e-05\\
215.01	3.6332990175629e-05\\
216.01	3.6332990175629e-05\\
217.01	3.6332990175629e-05\\
218.01	3.6332990175629e-05\\
219.01	3.6332990175629e-05\\
220.01	3.6332990175629e-05\\
221.01	3.6332990175629e-05\\
222.01	3.6332990175629e-05\\
223.01	3.6332990175629e-05\\
224.01	3.6332990175629e-05\\
225.01	3.6332990175629e-05\\
226.01	3.6332990175629e-05\\
227.01	3.6332990175629e-05\\
228.01	3.6332990175629e-05\\
229.01	3.6332990175629e-05\\
230.01	3.6332990175629e-05\\
231.01	3.6332990175629e-05\\
232.01	3.6332990175629e-05\\
233.01	3.6332990175629e-05\\
234.01	3.6332990175629e-05\\
235.01	3.6332990175629e-05\\
236.01	3.6332990175629e-05\\
237.01	3.6332990175629e-05\\
238.01	3.6332990175629e-05\\
239.01	3.6332990175629e-05\\
240.01	3.6332990175629e-05\\
241.01	3.6332990175629e-05\\
242.01	3.6332990175629e-05\\
243.01	3.6332990175629e-05\\
244.01	3.6332990175629e-05\\
245.01	3.6332990175629e-05\\
246.01	3.6332990175629e-05\\
247.01	3.6332990175629e-05\\
248.01	3.6332990175629e-05\\
249.01	3.6332990175629e-05\\
250.01	3.6332990175629e-05\\
251.01	3.6332990175629e-05\\
252.01	3.6332990175629e-05\\
253.01	3.6332990175629e-05\\
254.01	3.6332990175629e-05\\
255.01	3.6332990175629e-05\\
256.01	3.6332990175629e-05\\
257.01	3.6332990175629e-05\\
258.01	3.6332990175629e-05\\
259.01	3.6332990175629e-05\\
260.01	3.6332990175629e-05\\
261.01	3.6332990175629e-05\\
262.01	3.6332990175629e-05\\
263.01	3.6332990175629e-05\\
264.01	3.6332990175629e-05\\
265.01	3.6332990175629e-05\\
266.01	3.6332990175629e-05\\
267.01	3.6332990175629e-05\\
268.01	3.6332990175629e-05\\
269.01	3.6332990175629e-05\\
270.01	3.6332990175629e-05\\
271.01	3.6332990175629e-05\\
272.01	3.6332990175629e-05\\
273.01	3.6332990175629e-05\\
274.01	3.6332990175629e-05\\
275.01	3.6332990175629e-05\\
276.01	3.6332990175629e-05\\
277.01	3.6332990175629e-05\\
278.01	3.6332990175629e-05\\
279.01	3.6332990175629e-05\\
280.01	3.6332990175629e-05\\
281.01	3.6332990175629e-05\\
282.01	3.6332990175629e-05\\
283.01	3.6332990175629e-05\\
284.01	3.6332990175629e-05\\
285.01	3.6332990175629e-05\\
286.01	3.6332990175629e-05\\
287.01	3.6332990175629e-05\\
288.01	3.6332990175629e-05\\
289.01	3.6332990175629e-05\\
290.01	3.6332990175629e-05\\
291.01	3.6332990175629e-05\\
292.01	3.6332990175629e-05\\
293.01	3.6332990175629e-05\\
294.01	3.6332990175629e-05\\
295.01	3.6332990175629e-05\\
296.01	3.6332990175629e-05\\
297.01	3.6332990175629e-05\\
298.01	3.6332990175629e-05\\
299.01	3.6332990175629e-05\\
300.01	3.6332990175629e-05\\
301.01	3.6332990175629e-05\\
302.01	3.6332990175629e-05\\
303.01	3.6332990175629e-05\\
304.01	3.6332990175629e-05\\
305.01	3.6332990175629e-05\\
306.01	3.6332990175629e-05\\
307.01	3.6332990175629e-05\\
308.01	3.6332990175629e-05\\
309.01	3.6332990175629e-05\\
310.01	3.6332990175629e-05\\
311.01	3.6332990175629e-05\\
312.01	3.6332990175629e-05\\
313.01	3.6332990175629e-05\\
314.01	3.6332990175629e-05\\
315.01	3.6332990175629e-05\\
316.01	3.6332990175629e-05\\
317.01	3.6332990175629e-05\\
318.01	3.6332990175629e-05\\
319.01	3.6332990175629e-05\\
320.01	3.6332990175629e-05\\
321.01	3.6332990175629e-05\\
322.01	3.6332990175629e-05\\
323.01	3.6332990175629e-05\\
324.01	3.6332990175629e-05\\
325.01	3.6332990175629e-05\\
326.01	3.6332990175629e-05\\
327.01	3.6332990175629e-05\\
328.01	3.6332990175629e-05\\
329.01	3.6332990175629e-05\\
330.01	3.6332990175629e-05\\
331.01	3.6332990175629e-05\\
332.01	3.6332990175629e-05\\
333.01	3.6332990175629e-05\\
334.01	3.6332990175629e-05\\
335.01	3.6332990175629e-05\\
336.01	3.6332990175629e-05\\
337.01	3.6332990175629e-05\\
338.01	3.6332990175629e-05\\
339.01	3.6332990175629e-05\\
340.01	3.6332990175629e-05\\
341.01	3.6332990175629e-05\\
342.01	3.6332990175629e-05\\
343.01	3.6332990175629e-05\\
344.01	3.6332990175629e-05\\
345.01	3.6332990175629e-05\\
346.01	3.6332990175629e-05\\
347.01	3.6332990175629e-05\\
348.01	3.6332990175629e-05\\
349.01	3.6332990175629e-05\\
350.01	3.6332990175629e-05\\
351.01	3.6332990175629e-05\\
352.01	3.6332990175629e-05\\
353.01	3.6332990175629e-05\\
354.01	3.6332990175629e-05\\
355.01	3.6332990175629e-05\\
356.01	3.6332990175629e-05\\
357.01	3.6332990175629e-05\\
358.01	3.6332990175629e-05\\
359.01	3.6332990175629e-05\\
360.01	3.6332990175629e-05\\
361.01	3.6332990175629e-05\\
362.01	3.6332990175629e-05\\
363.01	3.6332990175629e-05\\
364.01	3.6332990175629e-05\\
365.01	3.6332990175629e-05\\
366.01	3.6332990175629e-05\\
367.01	3.6332990175629e-05\\
368.01	3.6332990175629e-05\\
369.01	3.6332990175629e-05\\
370.01	3.6332990175629e-05\\
371.01	3.6332990175629e-05\\
372.01	3.6332990175629e-05\\
373.01	3.6332990175629e-05\\
374.01	3.6332990175629e-05\\
375.01	3.6332990175629e-05\\
376.01	3.6332990175629e-05\\
377.01	3.6332990175629e-05\\
378.01	3.6332990175629e-05\\
379.01	3.6332990175629e-05\\
380.01	3.6332990175629e-05\\
381.01	3.6332990175629e-05\\
382.01	3.6332990175629e-05\\
383.01	3.6332990175629e-05\\
384.01	3.6332990175629e-05\\
385.01	3.6332990175629e-05\\
386.01	3.6332990175629e-05\\
387.01	3.6332990175629e-05\\
388.01	3.6332990175629e-05\\
389.01	3.6332990175629e-05\\
390.01	3.6332990175629e-05\\
391.01	3.6332990175629e-05\\
392.01	3.6332990175629e-05\\
393.01	3.6332990175629e-05\\
394.01	3.6332990175629e-05\\
395.01	3.6332990175629e-05\\
396.01	3.6332990175629e-05\\
397.01	3.6332990175629e-05\\
398.01	3.6332990175629e-05\\
399.01	3.6332990175629e-05\\
400.01	3.6332990175629e-05\\
401.01	3.6332990175629e-05\\
402.01	3.6332990175629e-05\\
403.01	3.6332990175629e-05\\
404.01	3.6332990175629e-05\\
405.01	3.6332990175629e-05\\
406.01	3.6332990175629e-05\\
407.01	3.6332990175629e-05\\
408.01	3.6332990175629e-05\\
409.01	3.6332990175629e-05\\
410.01	3.6332990175629e-05\\
411.01	3.6332990175629e-05\\
412.01	3.6332990175629e-05\\
413.01	3.6332990175629e-05\\
414.01	3.6332990175629e-05\\
415.01	3.6332990175629e-05\\
416.01	3.6332990175629e-05\\
417.01	3.6332990175629e-05\\
418.01	3.6332990175629e-05\\
419.01	3.6332990175629e-05\\
420.01	3.6332990175629e-05\\
421.01	3.6332990175629e-05\\
422.01	3.6332990175629e-05\\
423.01	3.6332990175629e-05\\
424.01	3.6332990175629e-05\\
425.01	3.6332990175629e-05\\
426.01	3.6332990175629e-05\\
427.01	3.6332990175629e-05\\
428.01	3.6332990175629e-05\\
429.01	3.6332990175629e-05\\
430.01	3.6332990175629e-05\\
431.01	3.6332990175629e-05\\
432.01	3.6332990175629e-05\\
433.01	3.6332990175629e-05\\
434.01	3.6332990175629e-05\\
435.01	3.6332990175629e-05\\
436.01	3.6332990175629e-05\\
437.01	3.6332990175629e-05\\
438.01	3.6332990175629e-05\\
439.01	3.6332990175629e-05\\
440.01	3.6332990175629e-05\\
441.01	3.6332990175629e-05\\
442.01	3.6332990175629e-05\\
443.01	3.6332990175629e-05\\
444.01	3.6332990175629e-05\\
445.01	3.63329901756273e-05\\
446.01	3.63329901756099e-05\\
447.01	3.633299017557e-05\\
448.01	3.63329901754642e-05\\
449.01	3.63329901751815e-05\\
450.01	3.63329901744112e-05\\
451.01	3.63329901723244e-05\\
452.01	3.63329901666726e-05\\
453.01	3.63329901513759e-05\\
454.01	3.63329901100322e-05\\
455.01	3.63329899984756e-05\\
456.01	3.63329896982331e-05\\
457.01	3.63329888930751e-05\\
458.01	3.63329867448508e-05\\
459.01	3.63329810541529e-05\\
460.01	3.63329661298185e-05\\
461.01	3.63329275367384e-05\\
462.01	3.63328297068363e-05\\
463.01	3.63325887119854e-05\\
464.01	3.63320195061381e-05\\
465.01	3.63307585765377e-05\\
466.01	3.63282378046548e-05\\
467.01	3.63240050212116e-05\\
468.01	3.63186526850575e-05\\
469.01	3.63131005853765e-05\\
470.01	3.63074235777052e-05\\
471.01	3.6301634700522e-05\\
472.01	3.62957697946677e-05\\
473.01	3.62899174158309e-05\\
474.01	3.62842786465652e-05\\
475.01	3.62792618784288e-05\\
476.01	3.6275545979637e-05\\
477.01	3.62738038318083e-05\\
478.01	3.62736023077313e-05\\
479.01	3.62736023077313e-05\\
480.01	3.62736023077313e-05\\
481.01	3.62736023077313e-05\\
482.01	3.62736023077313e-05\\
483.01	3.62736023077313e-05\\
484.01	3.62736023077313e-05\\
485.01	3.62736023077313e-05\\
486.01	3.62736023077313e-05\\
487.01	3.62736023077313e-05\\
488.01	3.62736023077313e-05\\
489.01	3.62736023077313e-05\\
490.01	3.62736023077313e-05\\
491.01	3.62736023077313e-05\\
492.01	3.62736023077313e-05\\
493.01	3.62736023077313e-05\\
494.01	3.62736023077313e-05\\
495.01	3.62736023077313e-05\\
496.01	3.62736023077313e-05\\
497.01	3.62736023077313e-05\\
498.01	3.62736023077313e-05\\
499.01	3.62736023077313e-05\\
500.01	3.62736023077313e-05\\
501.01	3.62736023077313e-05\\
502.01	3.62736023077313e-05\\
503.01	3.62736023077313e-05\\
504.01	3.62736023077313e-05\\
505.01	3.62736023077313e-05\\
506.01	3.62736023077313e-05\\
507.01	3.62736023077331e-05\\
508.01	3.62736023077331e-05\\
509.01	3.6273602307714e-05\\
510.01	3.6273602307688e-05\\
511.01	3.62736023076082e-05\\
512.01	3.62736023073896e-05\\
513.01	3.62736023067703e-05\\
514.01	3.62736023050304e-05\\
515.01	3.62736023001211e-05\\
516.01	3.62736022862329e-05\\
517.01	3.62736022467593e-05\\
518.01	3.62736021340768e-05\\
519.01	3.62736018107348e-05\\
520.01	3.62736008775299e-05\\
521.01	3.62735981668961e-05\\
522.01	3.62735902382626e-05\\
523.01	3.62735668706579e-05\\
524.01	3.62734974387602e-05\\
525.01	3.6273289345785e-05\\
526.01	3.62726599625867e-05\\
527.01	3.62707381364666e-05\\
528.01	3.62648115461908e-05\\
529.01	3.62463488691057e-05\\
530.01	3.61882420091067e-05\\
531.01	3.6003504478764e-05\\
532.01	3.54104452939367e-05\\
533.01	3.34895990750511e-05\\
534.01	2.86780701776249e-05\\
535.01	2.33349897528021e-05\\
536.01	1.7891461666756e-05\\
537.01	1.23300729334211e-05\\
538.01	6.77559521603671e-06\\
539.01	1.90599427558891e-06\\
540.01	0\\
541.01	0\\
542.01	0\\
543.01	0\\
544.01	0\\
545.01	1.73472347597681e-18\\
546.01	1.73472347597681e-18\\
547.01	0\\
548.01	0\\
549.01	0\\
550.01	0\\
551.01	0\\
552.01	0\\
553.01	0\\
554.01	0\\
555.01	0\\
556.01	1.73472347597681e-18\\
557.01	0\\
558.01	0\\
559.01	0\\
560.01	1.73472347597681e-18\\
561.01	1.73472347597681e-18\\
562.01	0\\
563.01	0\\
564.01	0\\
565.01	0\\
566.01	0\\
567.01	0\\
568.01	0\\
569.01	0\\
570.01	0\\
571.01	0\\
572.01	0\\
573.01	0\\
574.01	0\\
575.01	0\\
576.01	0\\
577.01	0\\
578.01	1.73472347597681e-18\\
579.01	0\\
580.01	0\\
581.01	0\\
582.01	0\\
583.01	1.73472347597681e-18\\
584.01	0\\
585.01	0\\
586.01	0\\
587.01	0\\
588.01	0\\
589.01	0\\
590.01	0\\
591.01	0\\
592.01	0\\
593.01	0\\
594.01	0\\
595.01	0\\
596.01	0\\
597.01	0\\
598.01	0\\
599.01	0\\
599.02	0\\
599.03	0\\
599.04	0\\
599.05	0\\
599.06	0\\
599.07	0\\
599.08	0\\
599.09	0\\
599.1	0\\
599.11	0\\
599.12	0\\
599.13	0\\
599.14	0\\
599.15	0\\
599.16	0\\
599.17	0\\
599.18	0\\
599.19	0\\
599.2	0\\
599.21	0\\
599.22	0\\
599.23	0\\
599.24	0\\
599.25	0\\
599.26	0\\
599.27	0\\
599.28	0\\
599.29	0\\
599.3	0\\
599.31	0\\
599.32	0\\
599.33	0\\
599.34	0\\
599.35	0\\
599.36	0\\
599.37	0\\
599.38	0\\
599.39	0\\
599.4	0\\
599.41	0\\
599.42	0\\
599.43	0\\
599.44	0\\
599.45	0\\
599.46	0\\
599.47	0\\
599.48	0\\
599.49	0\\
599.5	0\\
599.51	0\\
599.52	0\\
599.53	0\\
599.54	0\\
599.55	0\\
599.56	0\\
599.57	0\\
599.58	0\\
599.59	0\\
599.6	0\\
599.61	0\\
599.62	0\\
599.63	0\\
599.64	0\\
599.65	0\\
599.66	0\\
599.67	0\\
599.68	0\\
599.69	0\\
599.7	0\\
599.71	0\\
599.72	0\\
599.73	0\\
599.74	0\\
599.75	0\\
599.76	0\\
599.77	0\\
599.78	0\\
599.79	0\\
599.8	0\\
599.81	0\\
599.82	0\\
599.83	0\\
599.84	0\\
599.85	0\\
599.86	0\\
599.87	0\\
599.88	0\\
599.89	0\\
599.9	0\\
599.91	0\\
599.92	0\\
599.93	0\\
599.94	0\\
599.95	0\\
599.96	0\\
599.97	0\\
599.98	0\\
599.99	0\\
600	0\\
};
\addplot [color=mycolor11,solid,forget plot]
  table[row sep=crcr]{%
0.01	0.00032348326751387\\
1.01	0.00032348326751387\\
2.01	0.00032348326751387\\
3.01	0.00032348326751387\\
4.01	0.00032348326751387\\
5.01	0.00032348326751387\\
6.01	0.00032348326751387\\
7.01	0.00032348326751387\\
8.01	0.00032348326751387\\
9.01	0.00032348326751387\\
10.01	0.00032348326751387\\
11.01	0.00032348326751387\\
12.01	0.00032348326751387\\
13.01	0.00032348326751387\\
14.01	0.00032348326751387\\
15.01	0.00032348326751387\\
16.01	0.00032348326751387\\
17.01	0.00032348326751387\\
18.01	0.00032348326751387\\
19.01	0.00032348326751387\\
20.01	0.00032348326751387\\
21.01	0.00032348326751387\\
22.01	0.00032348326751387\\
23.01	0.00032348326751387\\
24.01	0.00032348326751387\\
25.01	0.00032348326751387\\
26.01	0.00032348326751387\\
27.01	0.00032348326751387\\
28.01	0.00032348326751387\\
29.01	0.00032348326751387\\
30.01	0.00032348326751387\\
31.01	0.00032348326751387\\
32.01	0.00032348326751387\\
33.01	0.00032348326751387\\
34.01	0.00032348326751387\\
35.01	0.00032348326751387\\
36.01	0.00032348326751387\\
37.01	0.00032348326751387\\
38.01	0.00032348326751387\\
39.01	0.00032348326751387\\
40.01	0.00032348326751387\\
41.01	0.00032348326751387\\
42.01	0.00032348326751387\\
43.01	0.00032348326751387\\
44.01	0.00032348326751387\\
45.01	0.00032348326751387\\
46.01	0.00032348326751387\\
47.01	0.00032348326751387\\
48.01	0.00032348326751387\\
49.01	0.00032348326751387\\
50.01	0.00032348326751387\\
51.01	0.00032348326751387\\
52.01	0.00032348326751387\\
53.01	0.00032348326751387\\
54.01	0.00032348326751387\\
55.01	0.00032348326751387\\
56.01	0.00032348326751387\\
57.01	0.00032348326751387\\
58.01	0.00032348326751387\\
59.01	0.00032348326751387\\
60.01	0.00032348326751387\\
61.01	0.00032348326751387\\
62.01	0.00032348326751387\\
63.01	0.00032348326751387\\
64.01	0.00032348326751387\\
65.01	0.00032348326751387\\
66.01	0.00032348326751387\\
67.01	0.00032348326751387\\
68.01	0.00032348326751387\\
69.01	0.00032348326751387\\
70.01	0.00032348326751387\\
71.01	0.00032348326751387\\
72.01	0.00032348326751387\\
73.01	0.00032348326751387\\
74.01	0.00032348326751387\\
75.01	0.00032348326751387\\
76.01	0.00032348326751387\\
77.01	0.00032348326751387\\
78.01	0.00032348326751387\\
79.01	0.00032348326751387\\
80.01	0.00032348326751387\\
81.01	0.00032348326751387\\
82.01	0.00032348326751387\\
83.01	0.00032348326751387\\
84.01	0.00032348326751387\\
85.01	0.00032348326751387\\
86.01	0.00032348326751387\\
87.01	0.00032348326751387\\
88.01	0.00032348326751387\\
89.01	0.00032348326751387\\
90.01	0.00032348326751387\\
91.01	0.00032348326751387\\
92.01	0.00032348326751387\\
93.01	0.00032348326751387\\
94.01	0.00032348326751387\\
95.01	0.00032348326751387\\
96.01	0.00032348326751387\\
97.01	0.00032348326751387\\
98.01	0.00032348326751387\\
99.01	0.00032348326751387\\
100.01	0.00032348326751387\\
101.01	0.00032348326751387\\
102.01	0.00032348326751387\\
103.01	0.00032348326751387\\
104.01	0.00032348326751387\\
105.01	0.00032348326751387\\
106.01	0.00032348326751387\\
107.01	0.00032348326751387\\
108.01	0.00032348326751387\\
109.01	0.00032348326751387\\
110.01	0.00032348326751387\\
111.01	0.00032348326751387\\
112.01	0.00032348326751387\\
113.01	0.00032348326751387\\
114.01	0.00032348326751387\\
115.01	0.00032348326751387\\
116.01	0.00032348326751387\\
117.01	0.00032348326751387\\
118.01	0.00032348326751387\\
119.01	0.00032348326751387\\
120.01	0.00032348326751387\\
121.01	0.00032348326751387\\
122.01	0.00032348326751387\\
123.01	0.00032348326751387\\
124.01	0.00032348326751387\\
125.01	0.00032348326751387\\
126.01	0.00032348326751387\\
127.01	0.00032348326751387\\
128.01	0.00032348326751387\\
129.01	0.00032348326751387\\
130.01	0.00032348326751387\\
131.01	0.00032348326751387\\
132.01	0.00032348326751387\\
133.01	0.00032348326751387\\
134.01	0.00032348326751387\\
135.01	0.00032348326751387\\
136.01	0.00032348326751387\\
137.01	0.00032348326751387\\
138.01	0.00032348326751387\\
139.01	0.00032348326751387\\
140.01	0.00032348326751387\\
141.01	0.00032348326751387\\
142.01	0.00032348326751387\\
143.01	0.00032348326751387\\
144.01	0.00032348326751387\\
145.01	0.00032348326751387\\
146.01	0.00032348326751387\\
147.01	0.00032348326751387\\
148.01	0.00032348326751387\\
149.01	0.00032348326751387\\
150.01	0.00032348326751387\\
151.01	0.00032348326751387\\
152.01	0.00032348326751387\\
153.01	0.00032348326751387\\
154.01	0.00032348326751387\\
155.01	0.00032348326751387\\
156.01	0.00032348326751387\\
157.01	0.00032348326751387\\
158.01	0.00032348326751387\\
159.01	0.00032348326751387\\
160.01	0.00032348326751387\\
161.01	0.00032348326751387\\
162.01	0.00032348326751387\\
163.01	0.00032348326751387\\
164.01	0.00032348326751387\\
165.01	0.00032348326751387\\
166.01	0.00032348326751387\\
167.01	0.00032348326751387\\
168.01	0.00032348326751387\\
169.01	0.00032348326751387\\
170.01	0.00032348326751387\\
171.01	0.00032348326751387\\
172.01	0.00032348326751387\\
173.01	0.00032348326751387\\
174.01	0.00032348326751387\\
175.01	0.00032348326751387\\
176.01	0.00032348326751387\\
177.01	0.00032348326751387\\
178.01	0.00032348326751387\\
179.01	0.00032348326751387\\
180.01	0.00032348326751387\\
181.01	0.00032348326751387\\
182.01	0.00032348326751387\\
183.01	0.00032348326751387\\
184.01	0.00032348326751387\\
185.01	0.00032348326751387\\
186.01	0.00032348326751387\\
187.01	0.00032348326751387\\
188.01	0.00032348326751387\\
189.01	0.00032348326751387\\
190.01	0.00032348326751387\\
191.01	0.00032348326751387\\
192.01	0.00032348326751387\\
193.01	0.00032348326751387\\
194.01	0.00032348326751387\\
195.01	0.00032348326751387\\
196.01	0.00032348326751387\\
197.01	0.00032348326751387\\
198.01	0.00032348326751387\\
199.01	0.00032348326751387\\
200.01	0.00032348326751387\\
201.01	0.00032348326751387\\
202.01	0.00032348326751387\\
203.01	0.00032348326751387\\
204.01	0.00032348326751387\\
205.01	0.00032348326751387\\
206.01	0.00032348326751387\\
207.01	0.00032348326751387\\
208.01	0.00032348326751387\\
209.01	0.00032348326751387\\
210.01	0.00032348326751387\\
211.01	0.00032348326751387\\
212.01	0.00032348326751387\\
213.01	0.00032348326751387\\
214.01	0.00032348326751387\\
215.01	0.00032348326751387\\
216.01	0.00032348326751387\\
217.01	0.00032348326751387\\
218.01	0.00032348326751387\\
219.01	0.00032348326751387\\
220.01	0.00032348326751387\\
221.01	0.00032348326751387\\
222.01	0.00032348326751387\\
223.01	0.00032348326751387\\
224.01	0.00032348326751387\\
225.01	0.00032348326751387\\
226.01	0.00032348326751387\\
227.01	0.00032348326751387\\
228.01	0.00032348326751387\\
229.01	0.00032348326751387\\
230.01	0.00032348326751387\\
231.01	0.00032348326751387\\
232.01	0.00032348326751387\\
233.01	0.00032348326751387\\
234.01	0.00032348326751387\\
235.01	0.00032348326751387\\
236.01	0.00032348326751387\\
237.01	0.00032348326751387\\
238.01	0.00032348326751387\\
239.01	0.00032348326751387\\
240.01	0.00032348326751387\\
241.01	0.00032348326751387\\
242.01	0.00032348326751387\\
243.01	0.00032348326751387\\
244.01	0.00032348326751387\\
245.01	0.00032348326751387\\
246.01	0.00032348326751387\\
247.01	0.00032348326751387\\
248.01	0.00032348326751387\\
249.01	0.00032348326751387\\
250.01	0.00032348326751387\\
251.01	0.00032348326751387\\
252.01	0.00032348326751387\\
253.01	0.00032348326751387\\
254.01	0.00032348326751387\\
255.01	0.00032348326751387\\
256.01	0.00032348326751387\\
257.01	0.00032348326751387\\
258.01	0.00032348326751387\\
259.01	0.00032348326751387\\
260.01	0.00032348326751387\\
261.01	0.00032348326751387\\
262.01	0.00032348326751387\\
263.01	0.00032348326751387\\
264.01	0.00032348326751387\\
265.01	0.00032348326751387\\
266.01	0.00032348326751387\\
267.01	0.00032348326751387\\
268.01	0.00032348326751387\\
269.01	0.00032348326751387\\
270.01	0.00032348326751387\\
271.01	0.00032348326751387\\
272.01	0.00032348326751387\\
273.01	0.00032348326751387\\
274.01	0.00032348326751387\\
275.01	0.00032348326751387\\
276.01	0.00032348326751387\\
277.01	0.00032348326751387\\
278.01	0.00032348326751387\\
279.01	0.00032348326751387\\
280.01	0.00032348326751387\\
281.01	0.00032348326751387\\
282.01	0.00032348326751387\\
283.01	0.00032348326751387\\
284.01	0.00032348326751387\\
285.01	0.00032348326751387\\
286.01	0.00032348326751387\\
287.01	0.00032348326751387\\
288.01	0.00032348326751387\\
289.01	0.00032348326751387\\
290.01	0.00032348326751387\\
291.01	0.00032348326751387\\
292.01	0.00032348326751387\\
293.01	0.00032348326751387\\
294.01	0.00032348326751387\\
295.01	0.00032348326751387\\
296.01	0.00032348326751387\\
297.01	0.00032348326751387\\
298.01	0.00032348326751387\\
299.01	0.00032348326751387\\
300.01	0.00032348326751387\\
301.01	0.00032348326751387\\
302.01	0.00032348326751387\\
303.01	0.00032348326751387\\
304.01	0.00032348326751387\\
305.01	0.00032348326751387\\
306.01	0.00032348326751387\\
307.01	0.00032348326751387\\
308.01	0.00032348326751387\\
309.01	0.00032348326751387\\
310.01	0.00032348326751387\\
311.01	0.00032348326751387\\
312.01	0.00032348326751387\\
313.01	0.00032348326751387\\
314.01	0.00032348326751387\\
315.01	0.00032348326751387\\
316.01	0.00032348326751387\\
317.01	0.00032348326751387\\
318.01	0.00032348326751387\\
319.01	0.00032348326751387\\
320.01	0.00032348326751387\\
321.01	0.00032348326751387\\
322.01	0.00032348326751387\\
323.01	0.00032348326751387\\
324.01	0.00032348326751387\\
325.01	0.00032348326751387\\
326.01	0.00032348326751387\\
327.01	0.00032348326751387\\
328.01	0.00032348326751387\\
329.01	0.00032348326751387\\
330.01	0.00032348326751387\\
331.01	0.00032348326751387\\
332.01	0.00032348326751387\\
333.01	0.00032348326751387\\
334.01	0.00032348326751387\\
335.01	0.00032348326751387\\
336.01	0.00032348326751387\\
337.01	0.00032348326751387\\
338.01	0.00032348326751387\\
339.01	0.00032348326751387\\
340.01	0.00032348326751387\\
341.01	0.00032348326751387\\
342.01	0.00032348326751387\\
343.01	0.00032348326751387\\
344.01	0.00032348326751387\\
345.01	0.00032348326751387\\
346.01	0.00032348326751387\\
347.01	0.00032348326751387\\
348.01	0.00032348326751387\\
349.01	0.00032348326751387\\
350.01	0.00032348326751387\\
351.01	0.00032348326751387\\
352.01	0.00032348326751387\\
353.01	0.00032348326751387\\
354.01	0.00032348326751387\\
355.01	0.00032348326751387\\
356.01	0.00032348326751387\\
357.01	0.00032348326751387\\
358.01	0.00032348326751387\\
359.01	0.00032348326751387\\
360.01	0.00032348326751387\\
361.01	0.00032348326751387\\
362.01	0.00032348326751387\\
363.01	0.00032348326751387\\
364.01	0.00032348326751387\\
365.01	0.00032348326751387\\
366.01	0.00032348326751387\\
367.01	0.00032348326751387\\
368.01	0.00032348326751387\\
369.01	0.00032348326751387\\
370.01	0.00032348326751387\\
371.01	0.00032348326751387\\
372.01	0.00032348326751387\\
373.01	0.00032348326751387\\
374.01	0.00032348326751387\\
375.01	0.00032348326751387\\
376.01	0.00032348326751387\\
377.01	0.00032348326751387\\
378.01	0.00032348326751387\\
379.01	0.00032348326751387\\
380.01	0.00032348326751387\\
381.01	0.00032348326751387\\
382.01	0.00032348326751387\\
383.01	0.00032348326751387\\
384.01	0.00032348326751387\\
385.01	0.00032348326751387\\
386.01	0.00032348326751387\\
387.01	0.00032348326751387\\
388.01	0.00032348326751387\\
389.01	0.00032348326751387\\
390.01	0.00032348326751387\\
391.01	0.00032348326751387\\
392.01	0.00032348326751387\\
393.01	0.00032348326751387\\
394.01	0.00032348326751387\\
395.01	0.00032348326751387\\
396.01	0.00032348326751387\\
397.01	0.00032348326751387\\
398.01	0.00032348326751387\\
399.01	0.00032348326751387\\
400.01	0.00032348326751387\\
401.01	0.00032348326751387\\
402.01	0.00032348326751387\\
403.01	0.00032348326751387\\
404.01	0.00032348326751387\\
405.01	0.00032348326751387\\
406.01	0.00032348326751387\\
407.01	0.00032348326751387\\
408.01	0.00032348326751387\\
409.01	0.00032348326751387\\
410.01	0.00032348326751387\\
411.01	0.00032348326751387\\
412.01	0.00032348326751387\\
413.01	0.00032348326751387\\
414.01	0.00032348326751387\\
415.01	0.00032348326751387\\
416.01	0.00032348326751387\\
417.01	0.00032348326751387\\
418.01	0.00032348326751387\\
419.01	0.00032348326751387\\
420.01	0.00032348326751387\\
421.01	0.00032348326751387\\
422.01	0.00032348326751387\\
423.01	0.00032348326751387\\
424.01	0.00032348326751387\\
425.01	0.00032348326751387\\
426.01	0.00032348326751387\\
427.01	0.00032348326751387\\
428.01	0.00032348326751387\\
429.01	0.00032348326751387\\
430.01	0.00032348326751387\\
431.01	0.00032348326751387\\
432.01	0.00032348326751387\\
433.01	0.00032348326751387\\
434.01	0.00032348326751387\\
435.01	0.00032348326751387\\
436.01	0.00032348326751387\\
437.01	0.00032348326751387\\
438.01	0.00032348326751387\\
439.01	0.00032348326751387\\
440.01	0.00032348326751387\\
441.01	0.00032348326751387\\
442.01	0.00032348326751387\\
443.01	0.00032348326751387\\
444.01	0.000323483267513872\\
445.01	0.000323483267513872\\
446.01	0.000323483267513854\\
447.01	0.000323483267513818\\
448.01	0.000323483267513719\\
449.01	0.00032348326751346\\
450.01	0.000323483267512777\\
451.01	0.00032348326751098\\
452.01	0.000323483267506265\\
453.01	0.000323483267493957\\
454.01	0.00032348326746201\\
455.01	0.000323483267379661\\
456.01	0.000323483267169111\\
457.01	0.000323483266636138\\
458.01	0.000323483265303676\\
459.01	0.000323483262024351\\
460.01	0.000323483254114245\\
461.01	0.000323483235527324\\
462.01	0.000323483193342078\\
463.01	0.000323483101985822\\
464.01	0.000323482916574127\\
465.01	0.000323482573337544\\
466.01	0.000323482016935115\\
467.01	0.000323481268148873\\
468.01	0.000323480441666813\\
469.01	0.000323479595467246\\
470.01	0.000323478733894969\\
471.01	0.000323477863201176\\
472.01	0.000323476997294479\\
473.01	0.000323476164510996\\
474.01	0.000323475416246714\\
475.01	0.000323474829585597\\
476.01	0.000323474481653553\\
477.01	0.000323474371361708\\
478.01	0.000323474365043552\\
479.01	0.000323474365043552\\
480.01	0.000323474365043552\\
481.01	0.000323474365043552\\
482.01	0.000323474365043552\\
483.01	0.000323474365043552\\
484.01	0.000323474365043552\\
485.01	0.000323474365043552\\
486.01	0.000323474365043552\\
487.01	0.000323474365043552\\
488.01	0.000323474365043552\\
489.01	0.000323474365043552\\
490.01	0.000323474365043552\\
491.01	0.000323474365043552\\
492.01	0.000323474365043552\\
493.01	0.000323474365043552\\
494.01	0.000323474365043552\\
495.01	0.000323474365043552\\
496.01	0.000323474365043552\\
497.01	0.000323474365043552\\
498.01	0.000323474365043552\\
499.01	0.000323474365043552\\
500.01	0.000323474365043552\\
501.01	0.000323474365043552\\
502.01	0.000323474365043552\\
503.01	0.000323474365043552\\
504.01	0.000323474365043552\\
505.01	0.000323474365043552\\
506.01	0.000323474365043552\\
507.01	0.000323474365043552\\
508.01	0.000323474365043552\\
509.01	0.000323474365043541\\
510.01	0.000323474365043508\\
511.01	0.000323474365043427\\
512.01	0.000323474365043201\\
513.01	0.000323474365042582\\
514.01	0.000323474365040866\\
515.01	0.000323474365036124\\
516.01	0.000323474365023004\\
517.01	0.000323474364986712\\
518.01	0.000323474364886271\\
519.01	0.000323474364608155\\
520.01	0.000323474363837714\\
521.01	0.000323474361702531\\
522.01	0.000323474355783794\\
523.01	0.00032347433937965\\
524.01	0.000323474293952497\\
525.01	0.000323474168395149\\
526.01	0.000323473822617017\\
527.01	0.000323472876248249\\
528.01	0.000323470312193687\\
529.01	0.000323463476687595\\
530.01	0.0003234457191502\\
531.01	0.000323401507088766\\
532.01	0.000323299349623227\\
533.01	0.000323096583750704\\
534.01	0.000322820494104223\\
535.01	0.000322544563133137\\
536.01	0.000322274405627094\\
537.01	0.000322012012008299\\
538.01	0.000321781721082232\\
539.01	0.000321635878047019\\
540.01	0.000321609567441481\\
541.01	0.000321609567440196\\
542.01	0.00032160956743655\\
543.01	0.000321609567426226\\
544.01	0.000321609567396994\\
545.01	0.000321609567314168\\
546.01	0.000321609567079316\\
547.01	0.00032160956641262\\
548.01	0.000321609564516894\\
549.01	0.000321609559114573\\
550.01	0.000321609543674653\\
551.01	0.00032160949938256\\
552.01	0.00032160937172819\\
553.01	0.000321609001700078\\
554.01	0.000321607921710718\\
555.01	0.000321604744132882\\
556.01	0.00032159530906575\\
557.01	0.000321567009926508\\
558.01	0.00032148121484275\\
559.01	0.000321218239453271\\
560.01	0.000320403516967051\\
561.01	0.000317854518528006\\
562.01	0.000309812765753915\\
563.01	0.000290521225180146\\
564.01	0.000258815636837589\\
565.01	0.00022319689057226\\
566.01	0.000180005113740115\\
567.01	0.000117402829674417\\
568.01	5.15310646992881e-05\\
569.01	4.39279151884069e-06\\
570.01	0\\
571.01	0\\
572.01	0\\
573.01	0\\
574.01	0\\
575.01	0\\
576.01	0\\
577.01	0\\
578.01	1.73472347597681e-18\\
579.01	0\\
580.01	0\\
581.01	0\\
582.01	0\\
583.01	1.73472347597681e-18\\
584.01	0\\
585.01	0\\
586.01	0\\
587.01	0\\
588.01	0\\
589.01	0\\
590.01	0\\
591.01	0\\
592.01	0\\
593.01	0\\
594.01	0\\
595.01	0\\
596.01	0\\
597.01	0\\
598.01	0\\
599.01	0\\
599.02	0\\
599.03	0\\
599.04	0\\
599.05	0\\
599.06	0\\
599.07	0\\
599.08	0\\
599.09	0\\
599.1	0\\
599.11	0\\
599.12	0\\
599.13	0\\
599.14	0\\
599.15	0\\
599.16	0\\
599.17	0\\
599.18	0\\
599.19	0\\
599.2	0\\
599.21	0\\
599.22	0\\
599.23	0\\
599.24	0\\
599.25	0\\
599.26	0\\
599.27	0\\
599.28	0\\
599.29	0\\
599.3	0\\
599.31	0\\
599.32	0\\
599.33	0\\
599.34	0\\
599.35	0\\
599.36	0\\
599.37	0\\
599.38	0\\
599.39	0\\
599.4	0\\
599.41	0\\
599.42	0\\
599.43	0\\
599.44	0\\
599.45	0\\
599.46	0\\
599.47	0\\
599.48	0\\
599.49	0\\
599.5	0\\
599.51	0\\
599.52	0\\
599.53	0\\
599.54	0\\
599.55	0\\
599.56	0\\
599.57	0\\
599.58	0\\
599.59	0\\
599.6	0\\
599.61	0\\
599.62	0\\
599.63	0\\
599.64	0\\
599.65	0\\
599.66	0\\
599.67	0\\
599.68	0\\
599.69	0\\
599.7	0\\
599.71	0\\
599.72	0\\
599.73	0\\
599.74	0\\
599.75	0\\
599.76	0\\
599.77	0\\
599.78	0\\
599.79	0\\
599.8	0\\
599.81	0\\
599.82	0\\
599.83	0\\
599.84	0\\
599.85	0\\
599.86	0\\
599.87	0\\
599.88	0\\
599.89	0\\
599.9	0\\
599.91	0\\
599.92	0\\
599.93	0\\
599.94	0\\
599.95	0\\
599.96	0\\
599.97	0\\
599.98	0\\
599.99	0\\
600	0\\
};
\addplot [color=mycolor12,solid,forget plot]
  table[row sep=crcr]{%
0.01	0.00230832677588388\\
1.01	0.00230832677588388\\
2.01	0.00230832677588388\\
3.01	0.00230832677588388\\
4.01	0.00230832677588388\\
5.01	0.00230832677588388\\
6.01	0.00230832677588388\\
7.01	0.00230832677588388\\
8.01	0.00230832677588388\\
9.01	0.00230832677588388\\
10.01	0.00230832677588388\\
11.01	0.00230832677588388\\
12.01	0.00230832677588388\\
13.01	0.00230832677588388\\
14.01	0.00230832677588388\\
15.01	0.00230832677588388\\
16.01	0.00230832677588388\\
17.01	0.00230832677588388\\
18.01	0.00230832677588388\\
19.01	0.00230832677588388\\
20.01	0.00230832677588388\\
21.01	0.00230832677588388\\
22.01	0.00230832677588388\\
23.01	0.00230832677588388\\
24.01	0.00230832677588388\\
25.01	0.00230832677588388\\
26.01	0.00230832677588388\\
27.01	0.00230832677588388\\
28.01	0.00230832677588388\\
29.01	0.00230832677588388\\
30.01	0.00230832677588388\\
31.01	0.00230832677588388\\
32.01	0.00230832677588388\\
33.01	0.00230832677588388\\
34.01	0.00230832677588388\\
35.01	0.00230832677588388\\
36.01	0.00230832677588388\\
37.01	0.00230832677588388\\
38.01	0.00230832677588388\\
39.01	0.00230832677588388\\
40.01	0.00230832677588388\\
41.01	0.00230832677588388\\
42.01	0.00230832677588388\\
43.01	0.00230832677588388\\
44.01	0.00230832677588388\\
45.01	0.00230832677588388\\
46.01	0.00230832677588388\\
47.01	0.00230832677588388\\
48.01	0.00230832677588388\\
49.01	0.00230832677588388\\
50.01	0.00230832677588388\\
51.01	0.00230832677588388\\
52.01	0.00230832677588388\\
53.01	0.00230832677588388\\
54.01	0.00230832677588388\\
55.01	0.00230832677588388\\
56.01	0.00230832677588388\\
57.01	0.00230832677588388\\
58.01	0.00230832677588388\\
59.01	0.00230832677588388\\
60.01	0.00230832677588388\\
61.01	0.00230832677588388\\
62.01	0.00230832677588388\\
63.01	0.00230832677588388\\
64.01	0.00230832677588388\\
65.01	0.00230832677588388\\
66.01	0.00230832677588388\\
67.01	0.00230832677588388\\
68.01	0.00230832677588388\\
69.01	0.00230832677588388\\
70.01	0.00230832677588388\\
71.01	0.00230832677588388\\
72.01	0.00230832677588388\\
73.01	0.00230832677588388\\
74.01	0.00230832677588388\\
75.01	0.00230832677588388\\
76.01	0.00230832677588388\\
77.01	0.00230832677588388\\
78.01	0.00230832677588388\\
79.01	0.00230832677588388\\
80.01	0.00230832677588388\\
81.01	0.00230832677588388\\
82.01	0.00230832677588388\\
83.01	0.00230832677588388\\
84.01	0.00230832677588388\\
85.01	0.00230832677588388\\
86.01	0.00230832677588388\\
87.01	0.00230832677588388\\
88.01	0.00230832677588388\\
89.01	0.00230832677588388\\
90.01	0.00230832677588388\\
91.01	0.00230832677588388\\
92.01	0.00230832677588388\\
93.01	0.00230832677588388\\
94.01	0.00230832677588388\\
95.01	0.00230832677588388\\
96.01	0.00230832677588388\\
97.01	0.00230832677588388\\
98.01	0.00230832677588388\\
99.01	0.00230832677588388\\
100.01	0.00230832677588388\\
101.01	0.00230832677588388\\
102.01	0.00230832677588388\\
103.01	0.00230832677588388\\
104.01	0.00230832677588388\\
105.01	0.00230832677588388\\
106.01	0.00230832677588388\\
107.01	0.00230832677588388\\
108.01	0.00230832677588388\\
109.01	0.00230832677588388\\
110.01	0.00230832677588388\\
111.01	0.00230832677588388\\
112.01	0.00230832677588388\\
113.01	0.00230832677588388\\
114.01	0.00230832677588388\\
115.01	0.00230832677588388\\
116.01	0.00230832677588388\\
117.01	0.00230832677588388\\
118.01	0.00230832677588388\\
119.01	0.00230832677588388\\
120.01	0.00230832677588388\\
121.01	0.00230832677588388\\
122.01	0.00230832677588388\\
123.01	0.00230832677588388\\
124.01	0.00230832677588388\\
125.01	0.00230832677588388\\
126.01	0.00230832677588388\\
127.01	0.00230832677588388\\
128.01	0.00230832677588388\\
129.01	0.00230832677588388\\
130.01	0.00230832677588388\\
131.01	0.00230832677588388\\
132.01	0.00230832677588388\\
133.01	0.00230832677588388\\
134.01	0.00230832677588388\\
135.01	0.00230832677588388\\
136.01	0.00230832677588388\\
137.01	0.00230832677588388\\
138.01	0.00230832677588388\\
139.01	0.00230832677588388\\
140.01	0.00230832677588388\\
141.01	0.00230832677588388\\
142.01	0.00230832677588388\\
143.01	0.00230832677588388\\
144.01	0.00230832677588388\\
145.01	0.00230832677588388\\
146.01	0.00230832677588388\\
147.01	0.00230832677588388\\
148.01	0.00230832677588388\\
149.01	0.00230832677588388\\
150.01	0.00230832677588388\\
151.01	0.00230832677588388\\
152.01	0.00230832677588388\\
153.01	0.00230832677588388\\
154.01	0.00230832677588388\\
155.01	0.00230832677588388\\
156.01	0.00230832677588388\\
157.01	0.00230832677588388\\
158.01	0.00230832677588388\\
159.01	0.00230832677588388\\
160.01	0.00230832677588388\\
161.01	0.00230832677588388\\
162.01	0.00230832677588388\\
163.01	0.00230832677588388\\
164.01	0.00230832677588388\\
165.01	0.00230832677588388\\
166.01	0.00230832677588388\\
167.01	0.00230832677588388\\
168.01	0.00230832677588388\\
169.01	0.00230832677588388\\
170.01	0.00230832677588388\\
171.01	0.00230832677588388\\
172.01	0.00230832677588388\\
173.01	0.00230832677588388\\
174.01	0.00230832677588388\\
175.01	0.00230832677588388\\
176.01	0.00230832677588388\\
177.01	0.00230832677588388\\
178.01	0.00230832677588388\\
179.01	0.00230832677588388\\
180.01	0.00230832677588388\\
181.01	0.00230832677588388\\
182.01	0.00230832677588388\\
183.01	0.00230832677588388\\
184.01	0.00230832677588388\\
185.01	0.00230832677588388\\
186.01	0.00230832677588388\\
187.01	0.00230832677588388\\
188.01	0.00230832677588388\\
189.01	0.00230832677588388\\
190.01	0.00230832677588388\\
191.01	0.00230832677588388\\
192.01	0.00230832677588388\\
193.01	0.00230832677588388\\
194.01	0.00230832677588388\\
195.01	0.00230832677588388\\
196.01	0.00230832677588388\\
197.01	0.00230832677588388\\
198.01	0.00230832677588388\\
199.01	0.00230832677588388\\
200.01	0.00230832677588388\\
201.01	0.00230832677588388\\
202.01	0.00230832677588388\\
203.01	0.00230832677588388\\
204.01	0.00230832677588388\\
205.01	0.00230832677588388\\
206.01	0.00230832677588388\\
207.01	0.00230832677588388\\
208.01	0.00230832677588388\\
209.01	0.00230832677588388\\
210.01	0.00230832677588388\\
211.01	0.00230832677588388\\
212.01	0.00230832677588388\\
213.01	0.00230832677588388\\
214.01	0.00230832677588388\\
215.01	0.00230832677588388\\
216.01	0.00230832677588388\\
217.01	0.00230832677588388\\
218.01	0.00230832677588388\\
219.01	0.00230832677588388\\
220.01	0.00230832677588388\\
221.01	0.00230832677588388\\
222.01	0.00230832677588388\\
223.01	0.00230832677588388\\
224.01	0.00230832677588388\\
225.01	0.00230832677588388\\
226.01	0.00230832677588388\\
227.01	0.00230832677588388\\
228.01	0.00230832677588388\\
229.01	0.00230832677588388\\
230.01	0.00230832677588388\\
231.01	0.00230832677588388\\
232.01	0.00230832677588388\\
233.01	0.00230832677588388\\
234.01	0.00230832677588388\\
235.01	0.00230832677588388\\
236.01	0.00230832677588388\\
237.01	0.00230832677588388\\
238.01	0.00230832677588388\\
239.01	0.00230832677588388\\
240.01	0.00230832677588388\\
241.01	0.00230832677588388\\
242.01	0.00230832677588388\\
243.01	0.00230832677588388\\
244.01	0.00230832677588388\\
245.01	0.00230832677588388\\
246.01	0.00230832677588388\\
247.01	0.00230832677588388\\
248.01	0.00230832677588388\\
249.01	0.00230832677588388\\
250.01	0.00230832677588388\\
251.01	0.00230832677588388\\
252.01	0.00230832677588388\\
253.01	0.00230832677588388\\
254.01	0.00230832677588388\\
255.01	0.00230832677588388\\
256.01	0.00230832677588388\\
257.01	0.00230832677588388\\
258.01	0.00230832677588388\\
259.01	0.00230832677588388\\
260.01	0.00230832677588388\\
261.01	0.00230832677588388\\
262.01	0.00230832677588388\\
263.01	0.00230832677588388\\
264.01	0.00230832677588388\\
265.01	0.00230832677588388\\
266.01	0.00230832677588388\\
267.01	0.00230832677588388\\
268.01	0.00230832677588388\\
269.01	0.00230832677588388\\
270.01	0.00230832677588388\\
271.01	0.00230832677588388\\
272.01	0.00230832677588388\\
273.01	0.00230832677588388\\
274.01	0.00230832677588388\\
275.01	0.00230832677588388\\
276.01	0.00230832677588388\\
277.01	0.00230832677588388\\
278.01	0.00230832677588388\\
279.01	0.00230832677588388\\
280.01	0.00230832677588388\\
281.01	0.00230832677588388\\
282.01	0.00230832677588388\\
283.01	0.00230832677588388\\
284.01	0.00230832677588388\\
285.01	0.00230832677588388\\
286.01	0.00230832677588388\\
287.01	0.00230832677588388\\
288.01	0.00230832677588388\\
289.01	0.00230832677588388\\
290.01	0.00230832677588388\\
291.01	0.00230832677588388\\
292.01	0.00230832677588388\\
293.01	0.00230832677588388\\
294.01	0.00230832677588388\\
295.01	0.00230832677588388\\
296.01	0.00230832677588388\\
297.01	0.00230832677588388\\
298.01	0.00230832677588388\\
299.01	0.00230832677588388\\
300.01	0.00230832677588388\\
301.01	0.00230832677588388\\
302.01	0.00230832677588388\\
303.01	0.00230832677588388\\
304.01	0.00230832677588388\\
305.01	0.00230832677588388\\
306.01	0.00230832677588388\\
307.01	0.00230832677588388\\
308.01	0.00230832677588388\\
309.01	0.00230832677588388\\
310.01	0.00230832677588388\\
311.01	0.00230832677588388\\
312.01	0.00230832677588388\\
313.01	0.00230832677588388\\
314.01	0.00230832677588388\\
315.01	0.00230832677588388\\
316.01	0.00230832677588388\\
317.01	0.00230832677588388\\
318.01	0.00230832677588388\\
319.01	0.00230832677588388\\
320.01	0.00230832677588388\\
321.01	0.00230832677588388\\
322.01	0.00230832677588388\\
323.01	0.00230832677588388\\
324.01	0.00230832677588388\\
325.01	0.00230832677588388\\
326.01	0.00230832677588388\\
327.01	0.00230832677588388\\
328.01	0.00230832677588388\\
329.01	0.00230832677588388\\
330.01	0.00230832677588388\\
331.01	0.00230832677588388\\
332.01	0.00230832677588388\\
333.01	0.00230832677588388\\
334.01	0.00230832677588388\\
335.01	0.00230832677588388\\
336.01	0.00230832677588388\\
337.01	0.00230832677588388\\
338.01	0.00230832677588388\\
339.01	0.00230832677588388\\
340.01	0.00230832677588388\\
341.01	0.00230832677588388\\
342.01	0.00230832677588388\\
343.01	0.00230832677588388\\
344.01	0.00230832677588388\\
345.01	0.00230832677588388\\
346.01	0.00230832677588388\\
347.01	0.00230832677588388\\
348.01	0.00230832677588388\\
349.01	0.00230832677588388\\
350.01	0.00230832677588388\\
351.01	0.00230832677588388\\
352.01	0.00230832677588388\\
353.01	0.00230832677588388\\
354.01	0.00230832677588388\\
355.01	0.00230832677588388\\
356.01	0.00230832677588388\\
357.01	0.00230832677588388\\
358.01	0.00230832677588388\\
359.01	0.00230832677588388\\
360.01	0.00230832677588388\\
361.01	0.00230832677588388\\
362.01	0.00230832677588388\\
363.01	0.00230832677588388\\
364.01	0.00230832677588388\\
365.01	0.00230832677588388\\
366.01	0.00230832677588388\\
367.01	0.00230832677588388\\
368.01	0.00230832677588388\\
369.01	0.00230832677588388\\
370.01	0.00230832677588388\\
371.01	0.00230832677588388\\
372.01	0.00230832677588388\\
373.01	0.00230832677588388\\
374.01	0.00230832677588388\\
375.01	0.00230832677588388\\
376.01	0.00230832677588388\\
377.01	0.00230832677588388\\
378.01	0.00230832677588388\\
379.01	0.00230832677588388\\
380.01	0.00230832677588388\\
381.01	0.00230832677588388\\
382.01	0.00230832677588388\\
383.01	0.00230832677588388\\
384.01	0.00230832677588388\\
385.01	0.00230832677588388\\
386.01	0.00230832677588388\\
387.01	0.00230832677588388\\
388.01	0.00230832677588388\\
389.01	0.00230832677588388\\
390.01	0.00230832677588388\\
391.01	0.00230832677588388\\
392.01	0.00230832677588388\\
393.01	0.00230832677588388\\
394.01	0.00230832677588388\\
395.01	0.00230832677588388\\
396.01	0.00230832677588388\\
397.01	0.00230832677588388\\
398.01	0.00230832677588388\\
399.01	0.00230832677588388\\
400.01	0.00230832677588388\\
401.01	0.00230832677588388\\
402.01	0.00230832677588388\\
403.01	0.00230832677588388\\
404.01	0.00230832677588388\\
405.01	0.00230832677588388\\
406.01	0.00230832677588388\\
407.01	0.00230832677588388\\
408.01	0.00230832677588388\\
409.01	0.00230832677588388\\
410.01	0.00230832677588388\\
411.01	0.00230832677588388\\
412.01	0.00230832677588388\\
413.01	0.00230832677588388\\
414.01	0.00230832677588388\\
415.01	0.00230832677588388\\
416.01	0.00230832677588388\\
417.01	0.00230832677588388\\
418.01	0.00230832677588388\\
419.01	0.00230832677588388\\
420.01	0.00230832677588388\\
421.01	0.00230832677588388\\
422.01	0.00230832677588388\\
423.01	0.00230832677588388\\
424.01	0.00230832677588388\\
425.01	0.00230832677588388\\
426.01	0.00230832677588388\\
427.01	0.00230832677588388\\
428.01	0.00230832677588388\\
429.01	0.00230832677588388\\
430.01	0.00230832677588388\\
431.01	0.00230832677588388\\
432.01	0.00230832677588388\\
433.01	0.00230832677588388\\
434.01	0.00230832677588388\\
435.01	0.00230832677588388\\
436.01	0.00230832677588388\\
437.01	0.00230832677588388\\
438.01	0.00230832677588388\\
439.01	0.00230832677588388\\
440.01	0.00230832677588388\\
441.01	0.00230832677588388\\
442.01	0.00230832677588388\\
443.01	0.00230832677588388\\
444.01	0.00230832677588388\\
445.01	0.00230832677588388\\
446.01	0.00230832677588388\\
447.01	0.00230832677588385\\
448.01	0.00230832677588378\\
449.01	0.00230832677588358\\
450.01	0.00230832677588308\\
451.01	0.00230832677588178\\
452.01	0.00230832677587848\\
453.01	0.00230832677587017\\
454.01	0.00230832677584946\\
455.01	0.00230832677579852\\
456.01	0.00230832677567521\\
457.01	0.00230832677538211\\
458.01	0.00230832677470067\\
459.01	0.00230832677315791\\
460.01	0.00230832676977686\\
461.01	0.00230832676265952\\
462.01	0.00230832674841642\\
463.01	0.00230832672169671\\
464.01	0.00230832667559282\\
465.01	0.00230832660424001\\
466.01	0.00230832650799597\\
467.01	0.00230832639644697\\
468.01	0.00230832628012502\\
469.01	0.00230832616191159\\
470.01	0.00230832604287346\\
471.01	0.00230832592509532\\
472.01	0.00230832581240644\\
473.01	0.00230832571105018\\
474.01	0.00230832562949901\\
475.01	0.00230832557582186\\
476.01	0.00230832555157799\\
477.01	0.00230832554657476\\
478.01	0.00230832554643708\\
479.01	0.00230832554643708\\
480.01	0.00230832554643708\\
481.01	0.00230832554643708\\
482.01	0.00230832554643708\\
483.01	0.00230832554643708\\
484.01	0.00230832554643708\\
485.01	0.00230832554643708\\
486.01	0.00230832554643708\\
487.01	0.00230832554643708\\
488.01	0.00230832554643708\\
489.01	0.00230832554643708\\
490.01	0.00230832554643708\\
491.01	0.00230832554643708\\
492.01	0.00230832554643708\\
493.01	0.00230832554643708\\
494.01	0.00230832554643708\\
495.01	0.00230832554643708\\
496.01	0.00230832554643708\\
497.01	0.00230832554643708\\
498.01	0.00230832554643708\\
499.01	0.00230832554643708\\
500.01	0.00230832554643708\\
501.01	0.00230832554643708\\
502.01	0.00230832554643708\\
503.01	0.00230832554643708\\
504.01	0.00230832554643708\\
505.01	0.00230832554643708\\
506.01	0.00230832554643708\\
507.01	0.00230832554643708\\
508.01	0.00230832554643708\\
509.01	0.00230832554643708\\
510.01	0.00230832554643706\\
511.01	0.00230832554643699\\
512.01	0.00230832554643681\\
513.01	0.00230832554643632\\
514.01	0.00230832554643497\\
515.01	0.0023083255464313\\
516.01	0.00230832554642139\\
517.01	0.00230832554639465\\
518.01	0.00230832554632288\\
519.01	0.00230832554613126\\
520.01	0.00230832554562292\\
521.01	0.00230832554428438\\
522.01	0.00230832554079122\\
523.01	0.0023083255317732\\
524.01	0.00230832550879643\\
525.01	0.00230832545119599\\
526.01	0.002308325309691\\
527.01	0.00230832497087839\\
528.01	0.00230832418619398\\
529.01	0.00230832244749176\\
530.01	0.00230831882192436\\
531.01	0.00230831189250172\\
532.01	0.00230830028466374\\
533.01	0.00230828450470487\\
534.01	0.00230826815361379\\
535.01	0.00230825300739821\\
536.01	0.00230823904377992\\
537.01	0.00230822681893579\\
538.01	0.00230821812485519\\
539.01	0.00230821447379591\\
540.01	0.00230821417331235\\
541.01	0.002308214173311\\
542.01	0.0023082141733072\\
543.01	0.00230821417329649\\
544.01	0.00230821417326634\\
545.01	0.00230821417318166\\
546.01	0.00230821417294417\\
547.01	0.00230821417227934\\
548.01	0.00230821417042241\\
549.01	0.00230821416524946\\
550.01	0.00230821415088388\\
551.01	0.00230821411113991\\
552.01	0.0023082140016871\\
553.01	0.00230821370195917\\
554.01	0.00230821288695113\\
555.01	0.00230821069054941\\
556.01	0.00230820483924602\\
557.01	0.00230818948614138\\
558.01	0.00230815002122632\\
559.01	0.00230805146354952\\
560.01	0.00230781562226196\\
561.01	0.00230728866371109\\
562.01	0.00230625128843328\\
563.01	0.00230463539557428\\
564.01	0.00230261463947325\\
565.01	0.00230027902002805\\
566.01	0.00229739045615409\\
567.01	0.00229414404146691\\
568.01	0.00229160501562282\\
569.01	0.00229055117153649\\
570.01	0.00229052498630496\\
571.01	0.00229052423870807\\
572.01	0.00229052200145302\\
573.01	0.00229051531235272\\
574.01	0.00229049533245125\\
575.01	0.00229043571706054\\
576.01	0.00229025804248841\\
577.01	0.00228972916792507\\
578.01	0.00228815701010785\\
579.01	0.00228349029843025\\
580.01	0.00226965902075653\\
581.01	0.00222878167195284\\
582.01	0.00213964864116229\\
583.01	0.00201726281566172\\
584.01	0.00185939417254293\\
585.01	0.00164117735174267\\
586.01	0.00140842561566351\\
587.01	0.00116447021169677\\
588.01	0.000899880963021769\\
589.01	0.00058797839348404\\
590.01	0.000224847652437047\\
591.01	3.75983539933726e-06\\
592.01	0\\
593.01	0\\
594.01	0\\
595.01	0\\
596.01	0\\
597.01	0\\
598.01	0\\
599.01	0\\
599.02	0\\
599.03	0\\
599.04	0\\
599.05	0\\
599.06	0\\
599.07	0\\
599.08	0\\
599.09	0\\
599.1	0\\
599.11	0\\
599.12	0\\
599.13	0\\
599.14	0\\
599.15	0\\
599.16	0\\
599.17	0\\
599.18	0\\
599.19	0\\
599.2	0\\
599.21	0\\
599.22	0\\
599.23	0\\
599.24	0\\
599.25	0\\
599.26	0\\
599.27	0\\
599.28	0\\
599.29	0\\
599.3	0\\
599.31	0\\
599.32	0\\
599.33	0\\
599.34	0\\
599.35	0\\
599.36	0\\
599.37	0\\
599.38	0\\
599.39	0\\
599.4	0\\
599.41	0\\
599.42	0\\
599.43	0\\
599.44	0\\
599.45	0\\
599.46	0\\
599.47	0\\
599.48	0\\
599.49	0\\
599.5	0\\
599.51	0\\
599.52	0\\
599.53	0\\
599.54	0\\
599.55	0\\
599.56	0\\
599.57	0\\
599.58	0\\
599.59	0\\
599.6	0\\
599.61	0\\
599.62	0\\
599.63	0\\
599.64	0\\
599.65	0\\
599.66	0\\
599.67	0\\
599.68	0\\
599.69	0\\
599.7	0\\
599.71	0\\
599.72	0\\
599.73	0\\
599.74	0\\
599.75	0\\
599.76	0\\
599.77	0\\
599.78	0\\
599.79	0\\
599.8	0\\
599.81	0\\
599.82	0\\
599.83	0\\
599.84	0\\
599.85	0\\
599.86	0\\
599.87	0\\
599.88	0\\
599.89	0\\
599.9	0\\
599.91	0\\
599.92	0\\
599.93	0\\
599.94	0\\
599.95	0\\
599.96	0\\
599.97	0\\
599.98	0\\
599.99	0\\
600	0\\
};
\addplot [color=mycolor13,solid,forget plot]
  table[row sep=crcr]{%
0.01	0\\
1.01	0\\
2.01	0\\
3.01	0\\
4.01	0\\
5.01	0\\
6.01	0\\
7.01	0\\
8.01	0\\
9.01	0\\
10.01	0\\
11.01	0\\
12.01	0\\
13.01	0\\
14.01	0\\
15.01	0\\
16.01	0\\
17.01	0\\
18.01	0\\
19.01	0\\
20.01	0\\
21.01	0\\
22.01	0\\
23.01	0\\
24.01	0\\
25.01	0\\
26.01	0\\
27.01	0\\
28.01	0\\
29.01	0\\
30.01	0\\
31.01	0\\
32.01	0\\
33.01	0\\
34.01	0\\
35.01	0\\
36.01	0\\
37.01	0\\
38.01	0\\
39.01	0\\
40.01	0\\
41.01	0\\
42.01	0\\
43.01	0\\
44.01	0\\
45.01	0\\
46.01	0\\
47.01	0\\
48.01	0\\
49.01	0\\
50.01	0\\
51.01	0\\
52.01	0\\
53.01	0\\
54.01	0\\
55.01	0\\
56.01	0\\
57.01	0\\
58.01	0\\
59.01	0\\
60.01	0\\
61.01	0\\
62.01	0\\
63.01	0\\
64.01	0\\
65.01	0\\
66.01	0\\
67.01	0\\
68.01	0\\
69.01	0\\
70.01	0\\
71.01	0\\
72.01	0\\
73.01	0\\
74.01	0\\
75.01	0\\
76.01	0\\
77.01	0\\
78.01	0\\
79.01	0\\
80.01	0\\
81.01	0\\
82.01	0\\
83.01	0\\
84.01	0\\
85.01	0\\
86.01	0\\
87.01	0\\
88.01	0\\
89.01	0\\
90.01	0\\
91.01	0\\
92.01	0\\
93.01	0\\
94.01	0\\
95.01	0\\
96.01	0\\
97.01	0\\
98.01	0\\
99.01	0\\
100.01	0\\
101.01	0\\
102.01	0\\
103.01	0\\
104.01	0\\
105.01	0\\
106.01	0\\
107.01	0\\
108.01	0\\
109.01	0\\
110.01	0\\
111.01	0\\
112.01	0\\
113.01	0\\
114.01	0\\
115.01	0\\
116.01	0\\
117.01	0\\
118.01	0\\
119.01	0\\
120.01	0\\
121.01	0\\
122.01	0\\
123.01	0\\
124.01	0\\
125.01	0\\
126.01	0\\
127.01	0\\
128.01	0\\
129.01	0\\
130.01	0\\
131.01	0\\
132.01	0\\
133.01	0\\
134.01	0\\
135.01	0\\
136.01	0\\
137.01	0\\
138.01	0\\
139.01	0\\
140.01	0\\
141.01	0\\
142.01	0\\
143.01	0\\
144.01	0\\
145.01	0\\
146.01	0\\
147.01	0\\
148.01	0\\
149.01	0\\
150.01	0\\
151.01	0\\
152.01	0\\
153.01	0\\
154.01	0\\
155.01	0\\
156.01	0\\
157.01	0\\
158.01	0\\
159.01	0\\
160.01	0\\
161.01	0\\
162.01	0\\
163.01	0\\
164.01	0\\
165.01	0\\
166.01	0\\
167.01	0\\
168.01	0\\
169.01	0\\
170.01	0\\
171.01	0\\
172.01	0\\
173.01	0\\
174.01	0\\
175.01	0\\
176.01	0\\
177.01	0\\
178.01	0\\
179.01	0\\
180.01	0\\
181.01	0\\
182.01	0\\
183.01	0\\
184.01	0\\
185.01	0\\
186.01	0\\
187.01	0\\
188.01	0\\
189.01	0\\
190.01	0\\
191.01	0\\
192.01	0\\
193.01	0\\
194.01	0\\
195.01	0\\
196.01	0\\
197.01	0\\
198.01	0\\
199.01	0\\
200.01	0\\
201.01	0\\
202.01	0\\
203.01	0\\
204.01	0\\
205.01	0\\
206.01	0\\
207.01	0\\
208.01	0\\
209.01	0\\
210.01	0\\
211.01	0\\
212.01	0\\
213.01	0\\
214.01	0\\
215.01	0\\
216.01	0\\
217.01	0\\
218.01	0\\
219.01	0\\
220.01	0\\
221.01	0\\
222.01	0\\
223.01	0\\
224.01	0\\
225.01	0\\
226.01	0\\
227.01	0\\
228.01	0\\
229.01	0\\
230.01	0\\
231.01	0\\
232.01	0\\
233.01	0\\
234.01	0\\
235.01	0\\
236.01	0\\
237.01	0\\
238.01	0\\
239.01	0\\
240.01	0\\
241.01	0\\
242.01	0\\
243.01	0\\
244.01	0\\
245.01	0\\
246.01	0\\
247.01	0\\
248.01	0\\
249.01	0\\
250.01	0\\
251.01	0\\
252.01	0\\
253.01	0\\
254.01	0\\
255.01	0\\
256.01	0\\
257.01	0\\
258.01	0\\
259.01	0\\
260.01	0\\
261.01	0\\
262.01	0\\
263.01	0\\
264.01	0\\
265.01	0\\
266.01	0\\
267.01	0\\
268.01	0\\
269.01	0\\
270.01	0\\
271.01	0\\
272.01	0\\
273.01	0\\
274.01	0\\
275.01	0\\
276.01	0\\
277.01	0\\
278.01	0\\
279.01	0\\
280.01	0\\
281.01	0\\
282.01	0\\
283.01	0\\
284.01	0\\
285.01	0\\
286.01	0\\
287.01	0\\
288.01	0\\
289.01	0\\
290.01	0\\
291.01	0\\
292.01	0\\
293.01	0\\
294.01	0\\
295.01	0\\
296.01	0\\
297.01	0\\
298.01	0\\
299.01	0\\
300.01	0\\
301.01	0\\
302.01	0\\
303.01	0\\
304.01	0\\
305.01	0\\
306.01	0\\
307.01	0\\
308.01	0\\
309.01	0\\
310.01	0\\
311.01	0\\
312.01	0\\
313.01	0\\
314.01	0\\
315.01	0\\
316.01	0\\
317.01	0\\
318.01	0\\
319.01	0\\
320.01	0\\
321.01	0\\
322.01	0\\
323.01	0\\
324.01	0\\
325.01	0\\
326.01	0\\
327.01	0\\
328.01	0\\
329.01	0\\
330.01	0\\
331.01	0\\
332.01	0\\
333.01	0\\
334.01	0\\
335.01	0\\
336.01	0\\
337.01	0\\
338.01	0\\
339.01	0\\
340.01	0\\
341.01	0\\
342.01	0\\
343.01	0\\
344.01	0\\
345.01	0\\
346.01	0\\
347.01	0\\
348.01	0\\
349.01	0\\
350.01	0\\
351.01	0\\
352.01	0\\
353.01	0\\
354.01	0\\
355.01	0\\
356.01	0\\
357.01	0\\
358.01	0\\
359.01	0\\
360.01	0\\
361.01	0\\
362.01	0\\
363.01	0\\
364.01	0\\
365.01	0\\
366.01	0\\
367.01	0\\
368.01	0\\
369.01	0\\
370.01	0\\
371.01	0\\
372.01	0\\
373.01	0\\
374.01	0\\
375.01	0\\
376.01	0\\
377.01	0\\
378.01	0\\
379.01	0\\
380.01	0\\
381.01	0\\
382.01	0\\
383.01	0\\
384.01	0\\
385.01	0\\
386.01	0\\
387.01	0\\
388.01	0\\
389.01	0\\
390.01	0\\
391.01	0\\
392.01	0\\
393.01	0\\
394.01	0\\
395.01	0\\
396.01	0\\
397.01	0\\
398.01	0\\
399.01	0\\
400.01	0\\
401.01	0\\
402.01	0\\
403.01	0\\
404.01	0\\
405.01	0\\
406.01	0\\
407.01	0\\
408.01	0\\
409.01	0\\
410.01	0\\
411.01	0\\
412.01	0\\
413.01	0\\
414.01	0\\
415.01	0\\
416.01	0\\
417.01	0\\
418.01	0\\
419.01	0\\
420.01	0\\
421.01	0\\
422.01	0\\
423.01	0\\
424.01	0\\
425.01	0\\
426.01	0\\
427.01	0\\
428.01	0\\
429.01	0\\
430.01	0\\
431.01	0\\
432.01	0\\
433.01	0\\
434.01	0\\
435.01	0\\
436.01	0\\
437.01	0\\
438.01	0\\
439.01	0\\
440.01	0\\
441.01	0\\
442.01	0\\
443.01	0\\
444.01	0\\
445.01	0\\
446.01	0\\
447.01	0\\
448.01	0\\
449.01	0\\
450.01	0\\
451.01	0\\
452.01	0\\
453.01	0\\
454.01	0\\
455.01	0\\
456.01	0\\
457.01	0\\
458.01	0\\
459.01	0\\
460.01	0\\
461.01	0\\
462.01	0\\
463.01	0\\
464.01	0\\
465.01	0\\
466.01	0\\
467.01	0\\
468.01	0\\
469.01	0\\
470.01	0\\
471.01	0\\
472.01	0\\
473.01	0\\
474.01	0\\
475.01	0\\
476.01	0\\
477.01	0\\
478.01	0\\
479.01	0\\
480.01	0\\
481.01	0\\
482.01	0\\
483.01	0\\
484.01	0\\
485.01	0\\
486.01	0\\
487.01	0\\
488.01	0\\
489.01	0\\
490.01	0\\
491.01	0\\
492.01	0\\
493.01	0\\
494.01	0\\
495.01	0\\
496.01	0\\
497.01	0\\
498.01	0\\
499.01	0\\
500.01	0\\
501.01	0\\
502.01	0\\
503.01	0\\
504.01	0\\
505.01	0\\
506.01	0\\
507.01	0\\
508.01	0\\
509.01	0\\
510.01	0\\
511.01	0\\
512.01	0\\
513.01	0\\
514.01	0\\
515.01	0\\
516.01	0\\
517.01	0\\
518.01	0\\
519.01	0\\
520.01	0\\
521.01	0\\
522.01	0\\
523.01	0\\
524.01	0\\
525.01	0\\
526.01	0\\
527.01	0\\
528.01	0\\
529.01	0\\
530.01	0\\
531.01	0\\
532.01	0\\
533.01	0\\
534.01	0\\
535.01	0\\
536.01	0\\
537.01	0\\
538.01	0\\
539.01	0\\
540.01	0\\
541.01	0\\
542.01	0\\
543.01	0\\
544.01	0\\
545.01	0\\
546.01	0\\
547.01	0\\
548.01	0\\
549.01	0\\
550.01	0\\
551.01	0\\
552.01	0\\
553.01	0\\
554.01	0\\
555.01	0\\
556.01	0\\
557.01	0\\
558.01	0\\
559.01	0\\
560.01	0\\
561.01	0\\
562.01	0\\
563.01	0\\
564.01	0\\
565.01	0\\
566.01	0\\
567.01	0\\
568.01	0\\
569.01	0\\
570.01	0\\
571.01	0\\
572.01	0\\
573.01	0\\
574.01	0\\
575.01	0\\
576.01	0\\
577.01	0\\
578.01	0\\
579.01	0\\
580.01	0\\
581.01	0\\
582.01	0\\
583.01	0\\
584.01	0\\
585.01	0\\
586.01	0\\
587.01	0\\
588.01	0\\
589.01	0\\
590.01	0\\
591.01	0\\
592.01	0\\
593.01	0\\
594.01	0\\
595.01	0\\
596.01	0\\
597.01	0\\
598.01	0\\
599.01	0.0022382818960328\\
599.02	0.00228829715400941\\
599.03	0.00233870844988766\\
599.04	0.00238951982934219\\
599.05	0.00244073538661882\\
599.06	0.00249235926524579\\
599.07	0.00254439565875786\\
599.08	0.00259684881143348\\
599.09	0.00264972301904528\\
599.1	0.0027030226296243\\
599.11	0.00275675204423814\\
599.12	0.00281091571778341\\
599.13	0.00286551815979286\\
599.14	0.00292056393525741\\
599.15	0.00297605766546351\\
599.16	0.00303200402884618\\
599.17	0.00308840776185803\\
599.18	0.00314527365985481\\
599.19	0.00320260657799763\\
599.2	0.00326041143217246\\
599.21	0.00331869319992726\\
599.22	0.00337745692142712\\
599.23	0.00343670770042793\\
599.24	0.00349645070526896\\
599.25	0.00355669116988489\\
599.26	0.0036174343948377\\
599.27	0.00367868574836898\\
599.28	0.00374045066747309\\
599.29	0.00380273465899182\\
599.3	0.00386554330073107\\
599.31	0.00392888224260002\\
599.32	0.00399275720777354\\
599.33	0.00405717399387833\\
599.34	0.00412213847420342\\
599.35	0.00418765659893578\\
599.36	0.00425373439642159\\
599.37	0.00432037797445393\\
599.38	0.00438759352158759\\
599.39	0.00445538730848178\\
599.4	0.00452376568927147\\
599.41	0.0045927351029681\\
599.42	0.00466230207489068\\
599.43	0.0047324732181279\\
599.44	0.00480325523503231\\
599.45	0.0048746549187474\\
599.46	0.00494667915476945\\
599.47	0.005019334922583\\
599.48	0.00509262929729296\\
599.49	0.00516656945129368\\
599.5	0.00524116265597606\\
599.51	0.00531641628347389\\
599.52	0.00539233780845056\\
599.53	0.00546893480992739\\
599.54	0.00554621497315491\\
599.55	0.00562418609152839\\
599.56	0.00570285606854896\\
599.57	0.00578223291983187\\
599.58	0.00586232477516331\\
599.59	0.00594313988060741\\
599.6	0.006024686600665\\
599.61	0.00610697342048581\\
599.62	0.00619000894813599\\
599.63	0.00627380191692264\\
599.64	0.00635836118777732\\
599.65	0.00644369575170061\\
599.66	0.00652981473226971\\
599.67	0.00661672738821123\\
599.68	0.00670444311604154\\
599.69	0.00679297145277696\\
599.7	0.00688232207871629\\
599.71	0.00697250482029821\\
599.72	0.0070635296530363\\
599.73	0.00715540670453446\\
599.74	0.00724814625758565\\
599.75	0.00734175875335703\\
599.76	0.00743625479466474\\
599.77	0.0075316451493416\\
599.78	0.00762794075370135\\
599.79	0.00772515271610301\\
599.8	0.00782329232061922\\
599.81	0.00792237103081269\\
599.82	0.00802240049362481\\
599.83	0.00812339254338108\\
599.84	0.00822535920591771\\
599.85	0.00832831270283452\\
599.86	0.00843226545587912\\
599.87	0.00853723009146771\\
599.88	0.00864321944534821\\
599.89	0.00875024656741154\\
599.9	0.0088583247266573\\
599.91	0.00896746741632041\\
599.92	0.00907768835916546\\
599.93	0.0091890015129561\\
599.94	0.00930142107610699\\
599.95	0.00941496149352625\\
599.96	0.009529637462657\\
599.97	0.00964546393972657\\
599.98	0.00976245614621301\\
599.99	0.00988062957553847\\
600	0.01\\
};
\addplot [color=mycolor14,solid,forget plot]
  table[row sep=crcr]{%
0.01	0\\
1.01	0\\
2.01	0\\
3.01	0\\
4.01	0\\
5.01	0\\
6.01	0\\
7.01	0\\
8.01	0\\
9.01	0\\
10.01	0\\
11.01	0\\
12.01	0\\
13.01	0\\
14.01	0\\
15.01	0\\
16.01	0\\
17.01	0\\
18.01	0\\
19.01	0\\
20.01	0\\
21.01	0\\
22.01	0\\
23.01	0\\
24.01	0\\
25.01	0\\
26.01	0\\
27.01	0\\
28.01	0\\
29.01	0\\
30.01	0\\
31.01	0\\
32.01	0\\
33.01	0\\
34.01	0\\
35.01	0\\
36.01	0\\
37.01	0\\
38.01	0\\
39.01	0\\
40.01	0\\
41.01	0\\
42.01	0\\
43.01	0\\
44.01	0\\
45.01	0\\
46.01	0\\
47.01	0\\
48.01	0\\
49.01	0\\
50.01	0\\
51.01	0\\
52.01	0\\
53.01	0\\
54.01	0\\
55.01	0\\
56.01	0\\
57.01	0\\
58.01	0\\
59.01	0\\
60.01	0\\
61.01	0\\
62.01	0\\
63.01	0\\
64.01	0\\
65.01	0\\
66.01	0\\
67.01	0\\
68.01	0\\
69.01	0\\
70.01	0\\
71.01	0\\
72.01	0\\
73.01	0\\
74.01	0\\
75.01	0\\
76.01	0\\
77.01	0\\
78.01	0\\
79.01	0\\
80.01	0\\
81.01	0\\
82.01	0\\
83.01	0\\
84.01	0\\
85.01	0\\
86.01	0\\
87.01	0\\
88.01	0\\
89.01	0\\
90.01	0\\
91.01	0\\
92.01	0\\
93.01	0\\
94.01	0\\
95.01	0\\
96.01	0\\
97.01	0\\
98.01	0\\
99.01	0\\
100.01	0\\
101.01	0\\
102.01	0\\
103.01	0\\
104.01	0\\
105.01	0\\
106.01	0\\
107.01	0\\
108.01	0\\
109.01	0\\
110.01	0\\
111.01	0\\
112.01	0\\
113.01	0\\
114.01	0\\
115.01	0\\
116.01	0\\
117.01	0\\
118.01	0\\
119.01	0\\
120.01	0\\
121.01	0\\
122.01	0\\
123.01	0\\
124.01	0\\
125.01	0\\
126.01	0\\
127.01	0\\
128.01	0\\
129.01	0\\
130.01	0\\
131.01	0\\
132.01	0\\
133.01	0\\
134.01	0\\
135.01	0\\
136.01	0\\
137.01	0\\
138.01	0\\
139.01	0\\
140.01	0\\
141.01	0\\
142.01	0\\
143.01	0\\
144.01	0\\
145.01	0\\
146.01	0\\
147.01	0\\
148.01	0\\
149.01	0\\
150.01	0\\
151.01	0\\
152.01	0\\
153.01	0\\
154.01	0\\
155.01	0\\
156.01	0\\
157.01	0\\
158.01	0\\
159.01	0\\
160.01	0\\
161.01	0\\
162.01	0\\
163.01	0\\
164.01	0\\
165.01	0\\
166.01	0\\
167.01	0\\
168.01	0\\
169.01	0\\
170.01	0\\
171.01	0\\
172.01	0\\
173.01	0\\
174.01	0\\
175.01	0\\
176.01	0\\
177.01	0\\
178.01	0\\
179.01	0\\
180.01	0\\
181.01	0\\
182.01	0\\
183.01	0\\
184.01	0\\
185.01	0\\
186.01	0\\
187.01	0\\
188.01	0\\
189.01	0\\
190.01	0\\
191.01	0\\
192.01	0\\
193.01	0\\
194.01	0\\
195.01	0\\
196.01	0\\
197.01	0\\
198.01	0\\
199.01	0\\
200.01	0\\
201.01	0\\
202.01	0\\
203.01	0\\
204.01	0\\
205.01	0\\
206.01	0\\
207.01	0\\
208.01	0\\
209.01	0\\
210.01	0\\
211.01	0\\
212.01	0\\
213.01	0\\
214.01	0\\
215.01	0\\
216.01	0\\
217.01	0\\
218.01	0\\
219.01	0\\
220.01	0\\
221.01	0\\
222.01	0\\
223.01	0\\
224.01	0\\
225.01	0\\
226.01	0\\
227.01	0\\
228.01	0\\
229.01	0\\
230.01	0\\
231.01	0\\
232.01	0\\
233.01	0\\
234.01	0\\
235.01	0\\
236.01	0\\
237.01	0\\
238.01	0\\
239.01	0\\
240.01	0\\
241.01	0\\
242.01	0\\
243.01	0\\
244.01	0\\
245.01	0\\
246.01	0\\
247.01	0\\
248.01	0\\
249.01	0\\
250.01	0\\
251.01	0\\
252.01	0\\
253.01	0\\
254.01	0\\
255.01	0\\
256.01	0\\
257.01	0\\
258.01	0\\
259.01	0\\
260.01	0\\
261.01	0\\
262.01	0\\
263.01	0\\
264.01	0\\
265.01	0\\
266.01	0\\
267.01	0\\
268.01	0\\
269.01	0\\
270.01	0\\
271.01	0\\
272.01	0\\
273.01	0\\
274.01	0\\
275.01	0\\
276.01	0\\
277.01	0\\
278.01	0\\
279.01	0\\
280.01	0\\
281.01	0\\
282.01	0\\
283.01	0\\
284.01	0\\
285.01	0\\
286.01	0\\
287.01	0\\
288.01	0\\
289.01	0\\
290.01	0\\
291.01	0\\
292.01	0\\
293.01	0\\
294.01	0\\
295.01	0\\
296.01	0\\
297.01	0\\
298.01	0\\
299.01	0\\
300.01	0\\
301.01	0\\
302.01	0\\
303.01	0\\
304.01	0\\
305.01	0\\
306.01	0\\
307.01	0\\
308.01	0\\
309.01	0\\
310.01	0\\
311.01	0\\
312.01	0\\
313.01	0\\
314.01	0\\
315.01	0\\
316.01	0\\
317.01	0\\
318.01	0\\
319.01	0\\
320.01	0\\
321.01	0\\
322.01	0\\
323.01	0\\
324.01	0\\
325.01	0\\
326.01	0\\
327.01	0\\
328.01	0\\
329.01	0\\
330.01	0\\
331.01	0\\
332.01	0\\
333.01	0\\
334.01	0\\
335.01	0\\
336.01	0\\
337.01	0\\
338.01	0\\
339.01	0\\
340.01	0\\
341.01	0\\
342.01	0\\
343.01	0\\
344.01	0\\
345.01	0\\
346.01	0\\
347.01	0\\
348.01	0\\
349.01	0\\
350.01	0\\
351.01	0\\
352.01	0\\
353.01	0\\
354.01	0\\
355.01	0\\
356.01	0\\
357.01	0\\
358.01	0\\
359.01	0\\
360.01	0\\
361.01	0\\
362.01	0\\
363.01	0\\
364.01	0\\
365.01	0\\
366.01	0\\
367.01	0\\
368.01	0\\
369.01	0\\
370.01	0\\
371.01	0\\
372.01	0\\
373.01	0\\
374.01	0\\
375.01	0\\
376.01	0\\
377.01	0\\
378.01	0\\
379.01	0\\
380.01	0\\
381.01	0\\
382.01	0\\
383.01	0\\
384.01	0\\
385.01	0\\
386.01	0\\
387.01	0\\
388.01	0\\
389.01	0\\
390.01	0\\
391.01	0\\
392.01	0\\
393.01	0\\
394.01	0\\
395.01	0\\
396.01	0\\
397.01	0\\
398.01	0\\
399.01	0\\
400.01	0\\
401.01	0\\
402.01	0\\
403.01	0\\
404.01	0\\
405.01	0\\
406.01	0\\
407.01	0\\
408.01	0\\
409.01	0\\
410.01	0\\
411.01	0\\
412.01	0\\
413.01	0\\
414.01	0\\
415.01	0\\
416.01	0\\
417.01	0\\
418.01	0\\
419.01	0\\
420.01	0\\
421.01	0\\
422.01	0\\
423.01	0\\
424.01	0\\
425.01	0\\
426.01	0\\
427.01	0\\
428.01	0\\
429.01	0\\
430.01	0\\
431.01	0\\
432.01	0\\
433.01	0\\
434.01	0\\
435.01	0\\
436.01	0\\
437.01	0\\
438.01	0\\
439.01	0\\
440.01	0\\
441.01	0\\
442.01	0\\
443.01	0\\
444.01	0\\
445.01	0\\
446.01	0\\
447.01	0\\
448.01	0\\
449.01	0\\
450.01	0\\
451.01	0\\
452.01	0\\
453.01	0\\
454.01	0\\
455.01	0\\
456.01	0\\
457.01	0\\
458.01	0\\
459.01	0\\
460.01	0\\
461.01	0\\
462.01	0\\
463.01	0\\
464.01	0\\
465.01	0\\
466.01	0\\
467.01	0\\
468.01	0\\
469.01	0\\
470.01	0\\
471.01	0\\
472.01	0\\
473.01	0\\
474.01	0\\
475.01	0\\
476.01	0\\
477.01	0\\
478.01	0\\
479.01	0\\
480.01	0\\
481.01	0\\
482.01	0\\
483.01	0\\
484.01	0\\
485.01	0\\
486.01	0\\
487.01	0\\
488.01	0\\
489.01	0\\
490.01	0\\
491.01	0\\
492.01	0\\
493.01	0\\
494.01	0\\
495.01	0\\
496.01	0\\
497.01	0\\
498.01	0\\
499.01	0\\
500.01	0\\
501.01	0\\
502.01	0\\
503.01	0\\
504.01	0\\
505.01	0\\
506.01	0\\
507.01	0\\
508.01	0\\
509.01	0\\
510.01	0\\
511.01	0\\
512.01	0\\
513.01	0\\
514.01	0\\
515.01	0\\
516.01	0\\
517.01	0\\
518.01	0\\
519.01	0\\
520.01	0\\
521.01	0\\
522.01	0\\
523.01	0\\
524.01	0\\
525.01	0\\
526.01	0\\
527.01	0\\
528.01	0\\
529.01	0\\
530.01	0\\
531.01	0\\
532.01	0\\
533.01	0\\
534.01	0\\
535.01	0\\
536.01	0\\
537.01	0\\
538.01	0\\
539.01	0\\
540.01	0\\
541.01	0\\
542.01	0\\
543.01	0\\
544.01	0\\
545.01	0\\
546.01	0\\
547.01	0\\
548.01	0\\
549.01	0\\
550.01	0\\
551.01	0\\
552.01	0\\
553.01	0\\
554.01	0\\
555.01	0\\
556.01	0\\
557.01	0\\
558.01	0\\
559.01	0\\
560.01	0\\
561.01	0\\
562.01	0\\
563.01	0\\
564.01	0\\
565.01	0\\
566.01	0\\
567.01	0\\
568.01	0\\
569.01	0\\
570.01	0\\
571.01	0\\
572.01	0\\
573.01	0\\
574.01	0\\
575.01	0\\
576.01	0\\
577.01	0\\
578.01	0\\
579.01	0\\
580.01	0\\
581.01	0\\
582.01	0\\
583.01	0\\
584.01	0\\
585.01	0\\
586.01	0\\
587.01	0\\
588.01	0\\
589.01	0\\
590.01	0\\
591.01	0\\
592.01	0\\
593.01	0\\
594.01	0\\
595.01	0\\
596.01	0\\
597.01	0\\
598.01	3.69463863061463e-05\\
599.01	0.00379444900894968\\
599.02	0.00383270374370267\\
599.03	0.00387131777133739\\
599.04	0.00391029453419965\\
599.05	0.0039496375063856\\
599.06	0.003989350193997\\
599.07	0.00402943613539713\\
599.08	0.00406989890146739\\
599.09	0.00411074209586436\\
599.1	0.00415196935527735\\
599.11	0.00419358434968641\\
599.12	0.00423559078262063\\
599.13	0.00427799239141662\\
599.14	0.00432079294747728\\
599.15	0.00436399625653049\\
599.16	0.00440760615888791\\
599.17	0.00445162652970357\\
599.18	0.00449606127923222\\
599.19	0.00454091435308739\\
599.2	0.00458618973249894\\
599.21	0.00463189143457006\\
599.22	0.00467802351253352\\
599.23	0.00472459005600711\\
599.24	0.00477159519124805\\
599.25	0.00481904308140627\\
599.26	0.00486693792678842\\
599.27	0.00491528396512382\\
599.28	0.00496408547181374\\
599.29	0.00501334676017881\\
599.3	0.00506307218170441\\
599.31	0.00511326612628387\\
599.32	0.00516393302245909\\
599.33	0.00521507733765866\\
599.34	0.00526670357843294\\
599.35	0.00531881629068601\\
599.36	0.00537142005990418\\
599.37	0.00542451951138086\\
599.38	0.00547811931043734\\
599.39	0.00553222416263933\\
599.4	0.00558683881400888\\
599.41	0.00564196805123132\\
599.42	0.00569761670185686\\
599.43	0.00575378963449661\\
599.44	0.00581049175901236\\
599.45	0.00586772802670002\\
599.46	0.00592550342982103\\
599.47	0.00598382297386091\\
599.48	0.00604269170566519\\
599.49	0.00610211471358306\\
599.5	0.00616209712760199\\
599.51	0.00622264411947269\\
599.52	0.00628376090282387\\
599.53	0.00634545273326615\\
599.54	0.00640772490848447\\
599.55	0.00647058276831837\\
599.56	0.00653403169482929\\
599.57	0.00659807711235419\\
599.58	0.00666272448754477\\
599.59	0.00672797932939122\\
599.6	0.00679384718922988\\
599.61	0.0068603336607336\\
599.62	0.00692744437988406\\
599.63	0.00699518502492475\\
599.64	0.00706356131629375\\
599.65	0.007132579016535\\
599.66	0.00720224393018678\\
599.67	0.00727256190364638\\
599.68	0.00734353882500926\\
599.69	0.00741518062388145\\
599.7	0.00748749327116373\\
599.71	0.00756048277880581\\
599.72	0.00763415519952897\\
599.73	0.00770851662651528\\
599.74	0.00778357319306161\\
599.75	0.00785933107219639\\
599.76	0.00793579647625698\\
599.77	0.00801297565642555\\
599.78	0.00809087490222101\\
599.79	0.00816950054094457\\
599.8	0.00824885893707632\\
599.81	0.00832895649161999\\
599.82	0.00840979964139308\\
599.83	0.00849139485825914\\
599.84	0.00857374864829899\\
599.85	0.0086568675509174\\
599.86	0.00874075813788151\\
599.87	0.00882542701228713\\
599.88	0.00891088080744883\\
599.89	0.00899712618570936\\
599.9	0.00908416983716382\\
599.91	0.00917201847829369\\
599.92	0.00926067885050539\\
599.93	0.00935015771856803\\
599.94	0.00944046186894425\\
599.95	0.00953159810800815\\
599.96	0.00962357326014348\\
599.97	0.00971639416571523\\
599.98	0.00981006767890704\\
599.99	0.00990460066541651\\
600	0.01\\
};
\addplot [color=mycolor15,solid,forget plot]
  table[row sep=crcr]{%
0.01	0\\
1.01	0\\
2.01	0\\
3.01	0\\
4.01	0\\
5.01	0\\
6.01	0\\
7.01	0\\
8.01	0\\
9.01	0\\
10.01	0\\
11.01	0\\
12.01	0\\
13.01	0\\
14.01	0\\
15.01	0\\
16.01	0\\
17.01	0\\
18.01	0\\
19.01	0\\
20.01	0\\
21.01	0\\
22.01	0\\
23.01	0\\
24.01	0\\
25.01	0\\
26.01	0\\
27.01	0\\
28.01	0\\
29.01	0\\
30.01	0\\
31.01	0\\
32.01	0\\
33.01	0\\
34.01	0\\
35.01	0\\
36.01	0\\
37.01	0\\
38.01	0\\
39.01	0\\
40.01	0\\
41.01	0\\
42.01	0\\
43.01	0\\
44.01	0\\
45.01	0\\
46.01	0\\
47.01	0\\
48.01	0\\
49.01	0\\
50.01	0\\
51.01	0\\
52.01	0\\
53.01	0\\
54.01	0\\
55.01	0\\
56.01	0\\
57.01	0\\
58.01	0\\
59.01	0\\
60.01	0\\
61.01	0\\
62.01	0\\
63.01	0\\
64.01	0\\
65.01	0\\
66.01	0\\
67.01	0\\
68.01	0\\
69.01	0\\
70.01	0\\
71.01	0\\
72.01	0\\
73.01	0\\
74.01	0\\
75.01	0\\
76.01	0\\
77.01	0\\
78.01	0\\
79.01	0\\
80.01	0\\
81.01	0\\
82.01	0\\
83.01	0\\
84.01	0\\
85.01	0\\
86.01	0\\
87.01	0\\
88.01	0\\
89.01	0\\
90.01	0\\
91.01	0\\
92.01	0\\
93.01	0\\
94.01	0\\
95.01	0\\
96.01	0\\
97.01	0\\
98.01	0\\
99.01	0\\
100.01	0\\
101.01	0\\
102.01	0\\
103.01	0\\
104.01	0\\
105.01	0\\
106.01	0\\
107.01	0\\
108.01	0\\
109.01	0\\
110.01	0\\
111.01	0\\
112.01	0\\
113.01	0\\
114.01	0\\
115.01	0\\
116.01	0\\
117.01	0\\
118.01	0\\
119.01	0\\
120.01	0\\
121.01	0\\
122.01	0\\
123.01	0\\
124.01	0\\
125.01	0\\
126.01	0\\
127.01	0\\
128.01	0\\
129.01	0\\
130.01	0\\
131.01	0\\
132.01	0\\
133.01	0\\
134.01	0\\
135.01	0\\
136.01	0\\
137.01	0\\
138.01	0\\
139.01	0\\
140.01	0\\
141.01	0\\
142.01	0\\
143.01	0\\
144.01	0\\
145.01	0\\
146.01	0\\
147.01	0\\
148.01	0\\
149.01	0\\
150.01	0\\
151.01	0\\
152.01	0\\
153.01	0\\
154.01	0\\
155.01	0\\
156.01	0\\
157.01	0\\
158.01	0\\
159.01	0\\
160.01	0\\
161.01	0\\
162.01	0\\
163.01	0\\
164.01	0\\
165.01	0\\
166.01	0\\
167.01	0\\
168.01	0\\
169.01	0\\
170.01	0\\
171.01	0\\
172.01	0\\
173.01	0\\
174.01	0\\
175.01	0\\
176.01	0\\
177.01	0\\
178.01	0\\
179.01	0\\
180.01	0\\
181.01	0\\
182.01	0\\
183.01	0\\
184.01	0\\
185.01	0\\
186.01	0\\
187.01	0\\
188.01	0\\
189.01	0\\
190.01	0\\
191.01	0\\
192.01	0\\
193.01	0\\
194.01	0\\
195.01	0\\
196.01	0\\
197.01	0\\
198.01	0\\
199.01	0\\
200.01	0\\
201.01	0\\
202.01	0\\
203.01	0\\
204.01	0\\
205.01	0\\
206.01	0\\
207.01	0\\
208.01	0\\
209.01	0\\
210.01	0\\
211.01	0\\
212.01	0\\
213.01	0\\
214.01	0\\
215.01	0\\
216.01	0\\
217.01	0\\
218.01	0\\
219.01	0\\
220.01	0\\
221.01	0\\
222.01	0\\
223.01	0\\
224.01	0\\
225.01	0\\
226.01	0\\
227.01	0\\
228.01	0\\
229.01	0\\
230.01	0\\
231.01	0\\
232.01	0\\
233.01	0\\
234.01	0\\
235.01	0\\
236.01	0\\
237.01	0\\
238.01	0\\
239.01	0\\
240.01	0\\
241.01	0\\
242.01	0\\
243.01	0\\
244.01	0\\
245.01	0\\
246.01	0\\
247.01	0\\
248.01	0\\
249.01	0\\
250.01	0\\
251.01	0\\
252.01	0\\
253.01	0\\
254.01	0\\
255.01	0\\
256.01	0\\
257.01	0\\
258.01	0\\
259.01	0\\
260.01	0\\
261.01	0\\
262.01	0\\
263.01	0\\
264.01	0\\
265.01	0\\
266.01	0\\
267.01	0\\
268.01	0\\
269.01	0\\
270.01	0\\
271.01	0\\
272.01	0\\
273.01	0\\
274.01	0\\
275.01	0\\
276.01	0\\
277.01	0\\
278.01	0\\
279.01	0\\
280.01	0\\
281.01	0\\
282.01	0\\
283.01	0\\
284.01	0\\
285.01	0\\
286.01	0\\
287.01	0\\
288.01	0\\
289.01	0\\
290.01	0\\
291.01	0\\
292.01	0\\
293.01	0\\
294.01	0\\
295.01	0\\
296.01	0\\
297.01	0\\
298.01	0\\
299.01	0\\
300.01	0\\
301.01	0\\
302.01	0\\
303.01	0\\
304.01	0\\
305.01	0\\
306.01	0\\
307.01	0\\
308.01	0\\
309.01	0\\
310.01	0\\
311.01	0\\
312.01	0\\
313.01	0\\
314.01	0\\
315.01	0\\
316.01	0\\
317.01	0\\
318.01	0\\
319.01	0\\
320.01	0\\
321.01	0\\
322.01	0\\
323.01	0\\
324.01	0\\
325.01	0\\
326.01	0\\
327.01	0\\
328.01	0\\
329.01	0\\
330.01	0\\
331.01	0\\
332.01	0\\
333.01	0\\
334.01	0\\
335.01	0\\
336.01	0\\
337.01	0\\
338.01	0\\
339.01	0\\
340.01	0\\
341.01	0\\
342.01	0\\
343.01	0\\
344.01	0\\
345.01	0\\
346.01	0\\
347.01	0\\
348.01	0\\
349.01	0\\
350.01	0\\
351.01	0\\
352.01	0\\
353.01	0\\
354.01	0\\
355.01	0\\
356.01	0\\
357.01	0\\
358.01	0\\
359.01	0\\
360.01	0\\
361.01	0\\
362.01	0\\
363.01	0\\
364.01	0\\
365.01	0\\
366.01	0\\
367.01	0\\
368.01	0\\
369.01	0\\
370.01	0\\
371.01	0\\
372.01	0\\
373.01	0\\
374.01	0\\
375.01	0\\
376.01	0\\
377.01	0\\
378.01	0\\
379.01	0\\
380.01	0\\
381.01	0\\
382.01	0\\
383.01	0\\
384.01	0\\
385.01	0\\
386.01	0\\
387.01	0\\
388.01	0\\
389.01	0\\
390.01	0\\
391.01	0\\
392.01	0\\
393.01	0\\
394.01	0\\
395.01	0\\
396.01	0\\
397.01	0\\
398.01	0\\
399.01	0\\
400.01	0\\
401.01	0\\
402.01	0\\
403.01	0\\
404.01	0\\
405.01	0\\
406.01	0\\
407.01	0\\
408.01	0\\
409.01	0\\
410.01	0\\
411.01	0\\
412.01	0\\
413.01	0\\
414.01	0\\
415.01	0\\
416.01	0\\
417.01	0\\
418.01	0\\
419.01	0\\
420.01	0\\
421.01	0\\
422.01	0\\
423.01	0\\
424.01	0\\
425.01	0\\
426.01	0\\
427.01	0\\
428.01	0\\
429.01	0\\
430.01	0\\
431.01	0\\
432.01	0\\
433.01	0\\
434.01	0\\
435.01	0\\
436.01	0\\
437.01	0\\
438.01	0\\
439.01	0\\
440.01	0\\
441.01	0\\
442.01	0\\
443.01	0\\
444.01	0\\
445.01	0\\
446.01	0\\
447.01	0\\
448.01	0\\
449.01	0\\
450.01	0\\
451.01	0\\
452.01	0\\
453.01	0\\
454.01	0\\
455.01	0\\
456.01	0\\
457.01	0\\
458.01	0\\
459.01	0\\
460.01	0\\
461.01	0\\
462.01	0\\
463.01	0\\
464.01	0\\
465.01	0\\
466.01	0\\
467.01	0\\
468.01	0\\
469.01	0\\
470.01	0\\
471.01	0\\
472.01	0\\
473.01	0\\
474.01	0\\
475.01	0\\
476.01	0\\
477.01	0\\
478.01	0\\
479.01	0\\
480.01	0\\
481.01	0\\
482.01	0\\
483.01	0\\
484.01	0\\
485.01	0\\
486.01	0\\
487.01	0\\
488.01	0\\
489.01	0\\
490.01	0\\
491.01	0\\
492.01	0\\
493.01	0\\
494.01	0\\
495.01	0\\
496.01	0\\
497.01	0\\
498.01	0\\
499.01	0\\
500.01	0\\
501.01	0\\
502.01	0\\
503.01	0\\
504.01	0\\
505.01	0\\
506.01	0\\
507.01	0\\
508.01	0\\
509.01	0\\
510.01	0\\
511.01	0\\
512.01	0\\
513.01	0\\
514.01	0\\
515.01	0\\
516.01	0\\
517.01	0\\
518.01	0\\
519.01	0\\
520.01	0\\
521.01	0\\
522.01	0\\
523.01	0\\
524.01	0\\
525.01	0\\
526.01	0\\
527.01	0\\
528.01	0\\
529.01	0\\
530.01	0\\
531.01	0\\
532.01	0\\
533.01	0\\
534.01	0\\
535.01	0\\
536.01	0\\
537.01	0\\
538.01	0\\
539.01	0\\
540.01	0\\
541.01	0\\
542.01	0\\
543.01	0\\
544.01	0\\
545.01	0\\
546.01	0\\
547.01	0\\
548.01	0\\
549.01	0\\
550.01	0\\
551.01	0\\
552.01	0\\
553.01	0\\
554.01	0\\
555.01	0\\
556.01	0\\
557.01	0\\
558.01	0\\
559.01	0\\
560.01	0\\
561.01	0\\
562.01	0\\
563.01	0\\
564.01	0\\
565.01	0\\
566.01	0\\
567.01	0\\
568.01	0\\
569.01	0\\
570.01	0\\
571.01	0\\
572.01	0\\
573.01	0\\
574.01	0\\
575.01	0\\
576.01	0\\
577.01	0\\
578.01	0\\
579.01	0\\
580.01	0\\
581.01	0\\
582.01	0\\
583.01	0\\
584.01	0\\
585.01	0\\
586.01	0\\
587.01	0\\
588.01	0\\
589.01	0\\
590.01	0\\
591.01	0\\
592.01	0\\
593.01	0\\
594.01	0\\
595.01	0\\
596.01	0\\
597.01	0\\
598.01	0.00141368560759025\\
599.01	0.00385460611783226\\
599.02	0.00389217668082419\\
599.03	0.00393010478508591\\
599.04	0.00396839388118488\\
599.05	0.00400704745291263\\
599.06	0.00404606901760485\\
599.07	0.00408546212646469\\
599.08	0.00412523036488917\\
599.09	0.00416537735279883\\
599.1	0.00420590674497062\\
599.11	0.00424682223137403\\
599.12	0.0042881275375106\\
599.13	0.00432982642475674\\
599.14	0.0043719226907099\\
599.15	0.0044144201695383\\
599.16	0.00445732273233393\\
599.17	0.00450063428746929\\
599.18	0.00454435878095746\\
599.19	0.00458850019681592\\
599.2	0.00463306255743389\\
599.21	0.00467804992394345\\
599.22	0.0047234663965943\\
599.23	0.00476931611513231\\
599.24	0.00481560325918193\\
599.25	0.00486233204863238\\
599.26	0.00490950673609608\\
599.27	0.00495713160496067\\
599.28	0.0050052109796917\\
599.29	0.00505374922622871\\
599.3	0.00510275075238528\\
599.31	0.00515222000825318\\
599.32	0.00520216148661058\\
599.33	0.00525257972333446\\
599.34	0.00530347929781725\\
599.35	0.00535486483338777\\
599.36	0.00540674099773647\\
599.37	0.00545911250334512\\
599.38	0.00551198410792095\\
599.39	0.00556536061483532\\
599.4	0.00561924687356703\\
599.41	0.00567364778015027\\
599.42	0.00572856827762734\\
599.43	0.00578401335650615\\
599.44	0.00583998805522266\\
599.45	0.00589649746060833\\
599.46	0.00595354670836325\\
599.47	0.00601114098356581\\
599.48	0.00606928552115735\\
599.49	0.00612798560643227\\
599.5	0.00618724657553358\\
599.51	0.00624707381595411\\
599.52	0.00630747276704345\\
599.53	0.00636844892052068\\
599.54	0.00643000782099308\\
599.55	0.00649215506648088\\
599.56	0.00655489630894826\\
599.57	0.00661823725484062\\
599.58	0.00668218366562836\\
599.59	0.00674674135835722\\
599.6	0.00681191620620549\\
599.61	0.006877714139048\\
599.62	0.00694414114402729\\
599.63	0.00701120326613206\\
599.64	0.00707890660878295\\
599.65	0.00714725733442608\\
599.66	0.00721626166513437\\
599.67	0.0072859258832169\\
599.68	0.00735625633183653\\
599.69	0.00742725941563611\\
599.7	0.00749894160137328\\
599.71	0.00757130941856446\\
599.72	0.00764436946013798\\
599.73	0.00771812838309687\\
599.74	0.00779259290919147\\
599.75	0.00786776982560227\\
599.76	0.0079436659856333\\
599.77	0.00802028830941633\\
599.78	0.00809764378462653\\
599.79	0.00817573946720956\\
599.8	0.00825458248212104\\
599.81	0.00833418002407833\\
599.82	0.00841453935832557\\
599.83	0.00849566782141209\\
599.84	0.00857757282198503\\
599.85	0.00866026184159656\\
599.86	0.00874374243552642\\
599.87	0.00882802223362036\\
599.88	0.00891310894114515\\
599.89	0.00899901033966106\\
599.9	0.0090857342879123\\
599.91	0.00917328872273657\\
599.92	0.00926168165999438\\
599.93	0.00935092119551921\\
599.94	0.0094410155060894\\
599.95	0.009531972850423\\
599.96	0.00962380157019665\\
599.97	0.00971651009108966\\
599.98	0.0098101069238547\\
599.99	0.00990460066541651\\
600	0.01\\
};
\addplot [color=mycolor16,solid,forget plot]
  table[row sep=crcr]{%
0.01	0\\
1.01	0\\
2.01	0\\
3.01	0\\
4.01	0\\
5.01	0\\
6.01	0\\
7.01	0\\
8.01	0\\
9.01	0\\
10.01	0\\
11.01	0\\
12.01	0\\
13.01	0\\
14.01	0\\
15.01	0\\
16.01	0\\
17.01	0\\
18.01	0\\
19.01	0\\
20.01	0\\
21.01	0\\
22.01	0\\
23.01	0\\
24.01	0\\
25.01	0\\
26.01	0\\
27.01	0\\
28.01	0\\
29.01	0\\
30.01	0\\
31.01	0\\
32.01	0\\
33.01	0\\
34.01	0\\
35.01	0\\
36.01	0\\
37.01	0\\
38.01	0\\
39.01	0\\
40.01	0\\
41.01	0\\
42.01	0\\
43.01	0\\
44.01	0\\
45.01	0\\
46.01	0\\
47.01	0\\
48.01	0\\
49.01	0\\
50.01	0\\
51.01	0\\
52.01	0\\
53.01	0\\
54.01	0\\
55.01	0\\
56.01	0\\
57.01	0\\
58.01	0\\
59.01	0\\
60.01	0\\
61.01	0\\
62.01	0\\
63.01	0\\
64.01	0\\
65.01	0\\
66.01	0\\
67.01	0\\
68.01	0\\
69.01	0\\
70.01	0\\
71.01	0\\
72.01	0\\
73.01	0\\
74.01	0\\
75.01	0\\
76.01	0\\
77.01	0\\
78.01	0\\
79.01	0\\
80.01	0\\
81.01	0\\
82.01	0\\
83.01	0\\
84.01	0\\
85.01	0\\
86.01	0\\
87.01	0\\
88.01	0\\
89.01	0\\
90.01	0\\
91.01	0\\
92.01	0\\
93.01	0\\
94.01	0\\
95.01	0\\
96.01	0\\
97.01	0\\
98.01	0\\
99.01	0\\
100.01	0\\
101.01	0\\
102.01	0\\
103.01	0\\
104.01	0\\
105.01	0\\
106.01	0\\
107.01	0\\
108.01	0\\
109.01	0\\
110.01	0\\
111.01	0\\
112.01	0\\
113.01	0\\
114.01	0\\
115.01	0\\
116.01	0\\
117.01	0\\
118.01	0\\
119.01	0\\
120.01	0\\
121.01	0\\
122.01	0\\
123.01	0\\
124.01	0\\
125.01	0\\
126.01	0\\
127.01	0\\
128.01	0\\
129.01	0\\
130.01	0\\
131.01	0\\
132.01	0\\
133.01	0\\
134.01	0\\
135.01	0\\
136.01	0\\
137.01	0\\
138.01	0\\
139.01	0\\
140.01	0\\
141.01	0\\
142.01	0\\
143.01	0\\
144.01	0\\
145.01	0\\
146.01	0\\
147.01	0\\
148.01	0\\
149.01	0\\
150.01	0\\
151.01	0\\
152.01	0\\
153.01	0\\
154.01	0\\
155.01	0\\
156.01	0\\
157.01	0\\
158.01	0\\
159.01	0\\
160.01	0\\
161.01	0\\
162.01	0\\
163.01	0\\
164.01	0\\
165.01	0\\
166.01	0\\
167.01	0\\
168.01	0\\
169.01	0\\
170.01	0\\
171.01	0\\
172.01	0\\
173.01	0\\
174.01	0\\
175.01	0\\
176.01	0\\
177.01	0\\
178.01	0\\
179.01	0\\
180.01	0\\
181.01	0\\
182.01	0\\
183.01	0\\
184.01	0\\
185.01	0\\
186.01	0\\
187.01	0\\
188.01	0\\
189.01	0\\
190.01	0\\
191.01	0\\
192.01	0\\
193.01	0\\
194.01	0\\
195.01	0\\
196.01	0\\
197.01	0\\
198.01	0\\
199.01	0\\
200.01	0\\
201.01	0\\
202.01	0\\
203.01	0\\
204.01	0\\
205.01	0\\
206.01	0\\
207.01	0\\
208.01	0\\
209.01	0\\
210.01	0\\
211.01	0\\
212.01	0\\
213.01	0\\
214.01	0\\
215.01	0\\
216.01	0\\
217.01	0\\
218.01	0\\
219.01	0\\
220.01	0\\
221.01	0\\
222.01	0\\
223.01	0\\
224.01	0\\
225.01	0\\
226.01	0\\
227.01	0\\
228.01	0\\
229.01	0\\
230.01	0\\
231.01	0\\
232.01	0\\
233.01	0\\
234.01	0\\
235.01	0\\
236.01	0\\
237.01	0\\
238.01	0\\
239.01	0\\
240.01	0\\
241.01	0\\
242.01	0\\
243.01	0\\
244.01	0\\
245.01	0\\
246.01	0\\
247.01	0\\
248.01	0\\
249.01	0\\
250.01	0\\
251.01	0\\
252.01	0\\
253.01	0\\
254.01	0\\
255.01	0\\
256.01	0\\
257.01	0\\
258.01	0\\
259.01	0\\
260.01	0\\
261.01	0\\
262.01	0\\
263.01	0\\
264.01	0\\
265.01	0\\
266.01	0\\
267.01	0\\
268.01	0\\
269.01	0\\
270.01	0\\
271.01	0\\
272.01	0\\
273.01	0\\
274.01	0\\
275.01	0\\
276.01	0\\
277.01	0\\
278.01	0\\
279.01	0\\
280.01	0\\
281.01	0\\
282.01	0\\
283.01	0\\
284.01	0\\
285.01	0\\
286.01	0\\
287.01	0\\
288.01	0\\
289.01	0\\
290.01	0\\
291.01	0\\
292.01	0\\
293.01	0\\
294.01	0\\
295.01	0\\
296.01	0\\
297.01	0\\
298.01	0\\
299.01	0\\
300.01	0\\
301.01	0\\
302.01	0\\
303.01	0\\
304.01	0\\
305.01	0\\
306.01	0\\
307.01	0\\
308.01	0\\
309.01	0\\
310.01	0\\
311.01	0\\
312.01	0\\
313.01	0\\
314.01	0\\
315.01	0\\
316.01	0\\
317.01	0\\
318.01	0\\
319.01	0\\
320.01	0\\
321.01	0\\
322.01	0\\
323.01	0\\
324.01	0\\
325.01	0\\
326.01	0\\
327.01	0\\
328.01	0\\
329.01	0\\
330.01	0\\
331.01	0\\
332.01	0\\
333.01	0\\
334.01	0\\
335.01	0\\
336.01	0\\
337.01	0\\
338.01	0\\
339.01	0\\
340.01	0\\
341.01	0\\
342.01	0\\
343.01	0\\
344.01	0\\
345.01	0\\
346.01	0\\
347.01	0\\
348.01	0\\
349.01	0\\
350.01	0\\
351.01	0\\
352.01	0\\
353.01	0\\
354.01	0\\
355.01	0\\
356.01	0\\
357.01	0\\
358.01	0\\
359.01	0\\
360.01	0\\
361.01	0\\
362.01	0\\
363.01	0\\
364.01	0\\
365.01	0\\
366.01	0\\
367.01	0\\
368.01	0\\
369.01	0\\
370.01	0\\
371.01	0\\
372.01	0\\
373.01	0\\
374.01	0\\
375.01	0\\
376.01	0\\
377.01	0\\
378.01	0\\
379.01	0\\
380.01	0\\
381.01	0\\
382.01	0\\
383.01	0\\
384.01	0\\
385.01	0\\
386.01	0\\
387.01	0\\
388.01	0\\
389.01	0\\
390.01	0\\
391.01	0\\
392.01	0\\
393.01	0\\
394.01	0\\
395.01	0\\
396.01	0\\
397.01	0\\
398.01	0\\
399.01	0\\
400.01	0\\
401.01	0\\
402.01	0\\
403.01	0\\
404.01	0\\
405.01	0\\
406.01	0\\
407.01	0\\
408.01	0\\
409.01	0\\
410.01	0\\
411.01	0\\
412.01	0\\
413.01	0\\
414.01	0\\
415.01	0\\
416.01	0\\
417.01	0\\
418.01	0\\
419.01	0\\
420.01	0\\
421.01	0\\
422.01	0\\
423.01	0\\
424.01	0\\
425.01	0\\
426.01	0\\
427.01	0\\
428.01	0\\
429.01	0\\
430.01	0\\
431.01	0\\
432.01	0\\
433.01	0\\
434.01	0\\
435.01	0\\
436.01	0\\
437.01	0\\
438.01	0\\
439.01	0\\
440.01	0\\
441.01	0\\
442.01	0\\
443.01	0\\
444.01	0\\
445.01	0\\
446.01	0\\
447.01	0\\
448.01	0\\
449.01	0\\
450.01	0\\
451.01	0\\
452.01	0\\
453.01	0\\
454.01	0\\
455.01	0\\
456.01	0\\
457.01	0\\
458.01	0\\
459.01	0\\
460.01	0\\
461.01	0\\
462.01	0\\
463.01	0\\
464.01	0\\
465.01	0\\
466.01	0\\
467.01	0\\
468.01	0\\
469.01	0\\
470.01	0\\
471.01	0\\
472.01	0\\
473.01	0\\
474.01	0\\
475.01	0\\
476.01	0\\
477.01	0\\
478.01	0\\
479.01	0\\
480.01	0\\
481.01	0\\
482.01	0\\
483.01	0\\
484.01	0\\
485.01	0\\
486.01	0\\
487.01	0\\
488.01	0\\
489.01	0\\
490.01	0\\
491.01	0\\
492.01	0\\
493.01	0\\
494.01	0\\
495.01	0\\
496.01	0\\
497.01	0\\
498.01	0\\
499.01	0\\
500.01	0\\
501.01	0\\
502.01	0\\
503.01	0\\
504.01	0\\
505.01	0\\
506.01	0\\
507.01	0\\
508.01	0\\
509.01	0\\
510.01	0\\
511.01	0\\
512.01	0\\
513.01	0\\
514.01	0\\
515.01	0\\
516.01	0\\
517.01	0\\
518.01	0\\
519.01	0\\
520.01	0\\
521.01	0\\
522.01	0\\
523.01	0\\
524.01	0\\
525.01	0\\
526.01	0\\
527.01	0\\
528.01	0\\
529.01	0\\
530.01	0\\
531.01	0\\
532.01	0\\
533.01	0\\
534.01	0\\
535.01	0\\
536.01	0\\
537.01	0\\
538.01	0\\
539.01	0\\
540.01	0\\
541.01	0\\
542.01	0\\
543.01	0\\
544.01	0\\
545.01	0\\
546.01	0\\
547.01	0\\
548.01	0\\
549.01	0\\
550.01	0\\
551.01	0\\
552.01	0\\
553.01	0\\
554.01	0\\
555.01	0\\
556.01	0\\
557.01	0\\
558.01	0\\
559.01	0\\
560.01	0\\
561.01	0\\
562.01	0\\
563.01	0\\
564.01	0\\
565.01	0\\
566.01	0\\
567.01	0\\
568.01	0\\
569.01	0\\
570.01	0\\
571.01	0\\
572.01	0\\
573.01	0\\
574.01	0\\
575.01	0\\
576.01	0\\
577.01	0\\
578.01	0\\
579.01	0\\
580.01	0\\
581.01	0\\
582.01	0\\
583.01	0\\
584.01	0\\
585.01	0\\
586.01	0\\
587.01	0\\
588.01	0\\
589.01	0\\
590.01	0\\
591.01	0\\
592.01	0\\
593.01	0\\
594.01	0\\
595.01	0\\
596.01	0\\
597.01	0\\
598.01	0.0014374037248278\\
599.01	0.00385656036521221\\
599.02	0.00389409724555795\\
599.03	0.00393199176826388\\
599.04	0.00397024738727775\\
599.05	0.00400886758988391\\
599.06	0.00404785589702493\\
599.07	0.0040872158636263\\
599.08	0.00412695107892428\\
599.09	0.00416706516679696\\
599.1	0.00420756178609841\\
599.11	0.00424844463099618\\
599.12	0.00428971743131196\\
599.13	0.00433138395286554\\
599.14	0.00437344799782218\\
599.15	0.00441591340504319\\
599.16	0.00445878405044\\
599.17	0.00450206384733162\\
599.18	0.0045457567468055\\
599.19	0.00458986673808198\\
599.2	0.00463439784888212\\
599.21	0.00467935414579919\\
599.22	0.00472473973467372\\
599.23	0.00477055876097212\\
599.24	0.00481681541016898\\
599.25	0.0048635139081331\\
599.26	0.00491065851083179\\
599.27	0.00495825350811072\\
599.28	0.00500630323104035\\
599.29	0.00505481205231203\\
599.3	0.00510378438663774\\
599.31	0.00515322469115379\\
599.32	0.0052031374658283\\
599.33	0.00525352725387272\\
599.34	0.00530439864215714\\
599.35	0.00535575626162978\\
599.36	0.00540760478774036\\
599.37	0.00545994894086756\\
599.38	0.00551279348675071\\
599.39	0.00556614323692543\\
599.4	0.00562000304916365\\
599.41	0.00567437782791768\\
599.42	0.00572927252476872\\
599.43	0.00578469213887949\\
599.44	0.00584064171745141\\
599.45	0.00589712635618597\\
599.46	0.00595415119975069\\
599.47	0.00601172144224942\\
599.48	0.00606984232769716\\
599.49	0.00612851915049951\\
599.5	0.00618775725593663\\
599.51	0.00624756204065183\\
599.52	0.00630793895314495\\
599.53	0.00636889349427038\\
599.54	0.00643043121773979\\
599.55	0.00649255773062987\\
599.56	0.00655527869389471\\
599.57	0.00661859982288322\\
599.58	0.00668252688786147\\
599.59	0.00674706571453995\\
599.6	0.00681222218460597\\
599.61	0.00687800223626108\\
599.62	0.00694441186476363\\
599.63	0.00701145712297651\\
599.64	0.00707914412192007\\
599.65	0.00714747903133038\\
599.66	0.00721646808022274\\
599.67	0.00728611755746052\\
599.68	0.00735643381232947\\
599.69	0.00742742325511746\\
599.7	0.00749909235769963\\
599.71	0.00757144765412921\\
599.72	0.00764449574123379\\
599.73	0.00771824327921728\\
599.74	0.00779269699226756\\
599.75	0.00786786366916977\\
599.76	0.00794375016392541\\
599.77	0.0080203633963772\\
599.78	0.00809771035283981\\
599.79	0.00817579808673647\\
599.8	0.00825463371924141\\
599.81	0.00833422443992839\\
599.82	0.00841457750742512\\
599.83	0.00849570025007373\\
599.84	0.00857760006659731\\
599.85	0.00866028442677256\\
599.86	0.0087437608721085\\
599.87	0.00882803701653141\\
599.88	0.00891312054707585\\
599.89	0.00899901922458194\\
599.9	0.00908574088439886\\
599.91	0.00917329343709451\\
599.92	0.00926168486917145\\
599.93	0.00935092324378912\\
599.94	0.00944101670149225\\
599.95	0.00953197346094551\\
599.96	0.00962380181967443\\
599.97	0.00971651015481255\\
599.98	0.0098101069238547\\
599.99	0.00990460066541651\\
600	0.01\\
};
\addplot [color=mycolor17,solid,forget plot]
  table[row sep=crcr]{%
0.01	0\\
1.01	0\\
2.01	0\\
3.01	0\\
4.01	0\\
5.01	0\\
6.01	0\\
7.01	0\\
8.01	0\\
9.01	0\\
10.01	0\\
11.01	0\\
12.01	0\\
13.01	0\\
14.01	0\\
15.01	0\\
16.01	0\\
17.01	0\\
18.01	0\\
19.01	0\\
20.01	0\\
21.01	0\\
22.01	0\\
23.01	0\\
24.01	0\\
25.01	0\\
26.01	0\\
27.01	0\\
28.01	0\\
29.01	0\\
30.01	0\\
31.01	0\\
32.01	0\\
33.01	0\\
34.01	0\\
35.01	0\\
36.01	0\\
37.01	0\\
38.01	0\\
39.01	0\\
40.01	0\\
41.01	0\\
42.01	0\\
43.01	0\\
44.01	0\\
45.01	0\\
46.01	0\\
47.01	0\\
48.01	0\\
49.01	0\\
50.01	0\\
51.01	0\\
52.01	0\\
53.01	0\\
54.01	0\\
55.01	0\\
56.01	0\\
57.01	0\\
58.01	0\\
59.01	0\\
60.01	0\\
61.01	0\\
62.01	0\\
63.01	0\\
64.01	0\\
65.01	0\\
66.01	0\\
67.01	0\\
68.01	0\\
69.01	0\\
70.01	0\\
71.01	0\\
72.01	0\\
73.01	0\\
74.01	0\\
75.01	0\\
76.01	0\\
77.01	0\\
78.01	0\\
79.01	0\\
80.01	0\\
81.01	0\\
82.01	0\\
83.01	0\\
84.01	0\\
85.01	0\\
86.01	0\\
87.01	0\\
88.01	0\\
89.01	0\\
90.01	0\\
91.01	0\\
92.01	0\\
93.01	0\\
94.01	0\\
95.01	0\\
96.01	0\\
97.01	0\\
98.01	0\\
99.01	0\\
100.01	0\\
101.01	0\\
102.01	0\\
103.01	0\\
104.01	0\\
105.01	0\\
106.01	0\\
107.01	0\\
108.01	0\\
109.01	0\\
110.01	0\\
111.01	0\\
112.01	0\\
113.01	0\\
114.01	0\\
115.01	0\\
116.01	0\\
117.01	0\\
118.01	0\\
119.01	0\\
120.01	0\\
121.01	0\\
122.01	0\\
123.01	0\\
124.01	0\\
125.01	0\\
126.01	0\\
127.01	0\\
128.01	0\\
129.01	0\\
130.01	0\\
131.01	0\\
132.01	0\\
133.01	0\\
134.01	0\\
135.01	0\\
136.01	0\\
137.01	0\\
138.01	0\\
139.01	0\\
140.01	0\\
141.01	0\\
142.01	0\\
143.01	0\\
144.01	0\\
145.01	0\\
146.01	0\\
147.01	0\\
148.01	0\\
149.01	0\\
150.01	0\\
151.01	0\\
152.01	0\\
153.01	0\\
154.01	0\\
155.01	0\\
156.01	0\\
157.01	0\\
158.01	0\\
159.01	0\\
160.01	0\\
161.01	0\\
162.01	0\\
163.01	0\\
164.01	0\\
165.01	0\\
166.01	0\\
167.01	0\\
168.01	0\\
169.01	0\\
170.01	0\\
171.01	0\\
172.01	0\\
173.01	0\\
174.01	0\\
175.01	0\\
176.01	0\\
177.01	0\\
178.01	0\\
179.01	0\\
180.01	0\\
181.01	0\\
182.01	0\\
183.01	0\\
184.01	0\\
185.01	0\\
186.01	0\\
187.01	0\\
188.01	0\\
189.01	0\\
190.01	0\\
191.01	0\\
192.01	0\\
193.01	0\\
194.01	0\\
195.01	0\\
196.01	0\\
197.01	0\\
198.01	0\\
199.01	0\\
200.01	0\\
201.01	0\\
202.01	0\\
203.01	0\\
204.01	0\\
205.01	0\\
206.01	0\\
207.01	0\\
208.01	0\\
209.01	0\\
210.01	0\\
211.01	0\\
212.01	0\\
213.01	0\\
214.01	0\\
215.01	0\\
216.01	0\\
217.01	0\\
218.01	0\\
219.01	0\\
220.01	0\\
221.01	0\\
222.01	0\\
223.01	0\\
224.01	0\\
225.01	0\\
226.01	0\\
227.01	0\\
228.01	0\\
229.01	0\\
230.01	0\\
231.01	0\\
232.01	0\\
233.01	0\\
234.01	0\\
235.01	0\\
236.01	0\\
237.01	0\\
238.01	0\\
239.01	0\\
240.01	0\\
241.01	0\\
242.01	0\\
243.01	0\\
244.01	0\\
245.01	0\\
246.01	0\\
247.01	0\\
248.01	0\\
249.01	0\\
250.01	0\\
251.01	0\\
252.01	0\\
253.01	0\\
254.01	0\\
255.01	0\\
256.01	0\\
257.01	0\\
258.01	0\\
259.01	0\\
260.01	0\\
261.01	0\\
262.01	0\\
263.01	0\\
264.01	0\\
265.01	0\\
266.01	0\\
267.01	0\\
268.01	0\\
269.01	0\\
270.01	0\\
271.01	0\\
272.01	0\\
273.01	0\\
274.01	0\\
275.01	0\\
276.01	0\\
277.01	0\\
278.01	0\\
279.01	0\\
280.01	0\\
281.01	0\\
282.01	0\\
283.01	0\\
284.01	0\\
285.01	0\\
286.01	0\\
287.01	0\\
288.01	0\\
289.01	0\\
290.01	0\\
291.01	0\\
292.01	0\\
293.01	0\\
294.01	0\\
295.01	0\\
296.01	0\\
297.01	0\\
298.01	0\\
299.01	0\\
300.01	0\\
301.01	0\\
302.01	0\\
303.01	0\\
304.01	0\\
305.01	0\\
306.01	0\\
307.01	0\\
308.01	0\\
309.01	0\\
310.01	0\\
311.01	0\\
312.01	0\\
313.01	0\\
314.01	0\\
315.01	0\\
316.01	0\\
317.01	0\\
318.01	0\\
319.01	0\\
320.01	0\\
321.01	0\\
322.01	0\\
323.01	0\\
324.01	0\\
325.01	0\\
326.01	0\\
327.01	0\\
328.01	0\\
329.01	0\\
330.01	0\\
331.01	0\\
332.01	0\\
333.01	0\\
334.01	0\\
335.01	0\\
336.01	0\\
337.01	0\\
338.01	0\\
339.01	0\\
340.01	0\\
341.01	0\\
342.01	0\\
343.01	0\\
344.01	0\\
345.01	0\\
346.01	0\\
347.01	0\\
348.01	0\\
349.01	0\\
350.01	0\\
351.01	0\\
352.01	0\\
353.01	0\\
354.01	0\\
355.01	0\\
356.01	0\\
357.01	0\\
358.01	0\\
359.01	0\\
360.01	0\\
361.01	0\\
362.01	0\\
363.01	0\\
364.01	0\\
365.01	0\\
366.01	0\\
367.01	0\\
368.01	0\\
369.01	0\\
370.01	0\\
371.01	0\\
372.01	0\\
373.01	0\\
374.01	0\\
375.01	0\\
376.01	0\\
377.01	0\\
378.01	0\\
379.01	0\\
380.01	0\\
381.01	0\\
382.01	0\\
383.01	0\\
384.01	0\\
385.01	0\\
386.01	0\\
387.01	0\\
388.01	0\\
389.01	0\\
390.01	0\\
391.01	0\\
392.01	0\\
393.01	0\\
394.01	0\\
395.01	0\\
396.01	0\\
397.01	0\\
398.01	0\\
399.01	0\\
400.01	0\\
401.01	0\\
402.01	0\\
403.01	0\\
404.01	0\\
405.01	0\\
406.01	0\\
407.01	0\\
408.01	0\\
409.01	0\\
410.01	0\\
411.01	0\\
412.01	0\\
413.01	0\\
414.01	0\\
415.01	0\\
416.01	0\\
417.01	0\\
418.01	0\\
419.01	0\\
420.01	0\\
421.01	0\\
422.01	0\\
423.01	0\\
424.01	0\\
425.01	0\\
426.01	0\\
427.01	0\\
428.01	0\\
429.01	0\\
430.01	0\\
431.01	0\\
432.01	0\\
433.01	0\\
434.01	0\\
435.01	0\\
436.01	0\\
437.01	0\\
438.01	0\\
439.01	0\\
440.01	0\\
441.01	0\\
442.01	0\\
443.01	0\\
444.01	0\\
445.01	0\\
446.01	0\\
447.01	0\\
448.01	0\\
449.01	0\\
450.01	0\\
451.01	0\\
452.01	0\\
453.01	0\\
454.01	0\\
455.01	0\\
456.01	0\\
457.01	0\\
458.01	0\\
459.01	0\\
460.01	0\\
461.01	0\\
462.01	0\\
463.01	0\\
464.01	0\\
465.01	0\\
466.01	0\\
467.01	0\\
468.01	0\\
469.01	0\\
470.01	0\\
471.01	0\\
472.01	0\\
473.01	0\\
474.01	0\\
475.01	0\\
476.01	0\\
477.01	0\\
478.01	0\\
479.01	0\\
480.01	0\\
481.01	0\\
482.01	0\\
483.01	0\\
484.01	0\\
485.01	0\\
486.01	0\\
487.01	0\\
488.01	0\\
489.01	0\\
490.01	0\\
491.01	0\\
492.01	0\\
493.01	0\\
494.01	0\\
495.01	0\\
496.01	0\\
497.01	0\\
498.01	0\\
499.01	0\\
500.01	0\\
501.01	0\\
502.01	0\\
503.01	0\\
504.01	0\\
505.01	0\\
506.01	0\\
507.01	0\\
508.01	0\\
509.01	0\\
510.01	0\\
511.01	0\\
512.01	0\\
513.01	0\\
514.01	0\\
515.01	0\\
516.01	0\\
517.01	0\\
518.01	0\\
519.01	0\\
520.01	0\\
521.01	0\\
522.01	0\\
523.01	0\\
524.01	0\\
525.01	0\\
526.01	0\\
527.01	0\\
528.01	0\\
529.01	0\\
530.01	0\\
531.01	0\\
532.01	0\\
533.01	0\\
534.01	0\\
535.01	0\\
536.01	0\\
537.01	0\\
538.01	0\\
539.01	0\\
540.01	0\\
541.01	0\\
542.01	0\\
543.01	0\\
544.01	0\\
545.01	0\\
546.01	0\\
547.01	0\\
548.01	0\\
549.01	0\\
550.01	0\\
551.01	0\\
552.01	0\\
553.01	0\\
554.01	0\\
555.01	0\\
556.01	0\\
557.01	0\\
558.01	0\\
559.01	0\\
560.01	0\\
561.01	0\\
562.01	0\\
563.01	0\\
564.01	0\\
565.01	0\\
566.01	0\\
567.01	0\\
568.01	0\\
569.01	0\\
570.01	0\\
571.01	0\\
572.01	0\\
573.01	0\\
574.01	0\\
575.01	0\\
576.01	0\\
577.01	0\\
578.01	0\\
579.01	0\\
580.01	0\\
581.01	0\\
582.01	0\\
583.01	0\\
584.01	0\\
585.01	0\\
586.01	0\\
587.01	0\\
588.01	0\\
589.01	0\\
590.01	0\\
591.01	0\\
592.01	0\\
593.01	0\\
594.01	0\\
595.01	0\\
596.01	0\\
597.01	0\\
598.01	0.00143822749983459\\
599.01	0.00385661144683571\\
599.02	0.00389414709863901\\
599.03	0.00393204040529412\\
599.04	0.00397029482087115\\
599.05	0.00400891383277854\\
599.06	0.00404790096208447\\
599.07	0.00408725976384149\\
599.08	0.00412699382741433\\
599.09	0.0041671067768107\\
599.1	0.00420760227101545\\
599.11	0.00424848400432787\\
599.12	0.00428975570670221\\
599.13	0.00433142114409155\\
599.14	0.00437348411879493\\
599.15	0.00441594846980785\\
599.16	0.00445881807317617\\
599.17	0.00450209684235336\\
599.18	0.00454578872856121\\
599.19	0.00458989772115408\\
599.2	0.00463442784798657\\
599.21	0.00467938317578476\\
599.22	0.00472476781052109\\
599.23	0.00477058589779272\\
599.24	0.00481684162320371\\
599.25	0.00486353921275068\\
599.26	0.0049106829225159\\
599.27	0.00495827704247297\\
599.28	0.00500632590381774\\
599.29	0.00505483387936404\\
599.3	0.00510380538394311\\
599.31	0.00515324487480703\\
599.32	0.00520315685203588\\
599.33	0.00525354585894884\\
599.34	0.00530441648251928\\
599.35	0.00535577335379373\\
599.36	0.005407621148315\\
599.37	0.00545996458654921\\
599.38	0.00551280843431708\\
599.39	0.00556615750322915\\
599.4	0.00562001665112538\\
599.41	0.00567439078251885\\
599.42	0.00572928484904367\\
599.43	0.00578470384990735\\
599.44	0.00584065283234732\\
599.45	0.00589713689209195\\
599.46	0.00595416117382597\\
599.47	0.00601173087166026\\
599.48	0.00606985122960629\\
599.49	0.006128527542055\\
599.5	0.0061877651542603\\
599.51	0.00624756946282722\\
599.52	0.00630794591620473\\
599.53	0.00636890001518324\\
599.54	0.00643043731339698\\
599.55	0.00649256341783105\\
599.56	0.00655528398933347\\
599.57	0.00661860474313204\\
599.58	0.00668253144935614\\
599.59	0.00674706993356369\\
599.6	0.00681222607727294\\
599.61	0.0068780058184995\\
599.62	0.0069444151522985\\
599.63	0.00701146013131191\\
599.64	0.0070791468663211\\
599.65	0.00714748152680477\\
599.66	0.00721647034150212\\
599.67	0.00728611959898148\\
599.68	0.00735643564821442\\
599.69	0.00742742489915527\\
599.7	0.00749909382332627\\
599.71	0.00757144895440832\\
599.72	0.00764449688883733\\
599.73	0.00771824428640635\\
599.74	0.00779269787087345\\
599.75	0.00786786443057545\\
599.76	0.00794375081904746\\
599.77	0.0080203639556484\\
599.78	0.00809771082619259\\
599.79	0.00817579848358721\\
599.8	0.00825463404847609\\
599.81	0.0083342247098895\\
599.82	0.0084145777259002\\
599.83	0.00849570042428592\\
599.84	0.00857760020319794\\
599.85	0.00866028453183631\\
599.86	0.00874376095113144\\
599.87	0.00882803707443219\\
599.88	0.00891312058820066\\
599.89	0.00899901925271357\\
599.9	0.00908574090277037\\
599.91	0.00917329344840818\\
599.92	0.00926168487562357\\
599.93	0.00935092324710127\\
599.94	0.00944101670294984\\
599.95	0.00953197346144448\\
599.96	0.00962380181977693\\
599.97	0.00971651015481255\\
599.98	0.0098101069238547\\
599.99	0.00990460066541651\\
600	0.01\\
};
\addplot [color=mycolor18,solid,forget plot]
  table[row sep=crcr]{%
0.01	0\\
1.01	0\\
2.01	0\\
3.01	0\\
4.01	0\\
5.01	0\\
6.01	0\\
7.01	0\\
8.01	0\\
9.01	0\\
10.01	0\\
11.01	0\\
12.01	0\\
13.01	0\\
14.01	0\\
15.01	0\\
16.01	0\\
17.01	0\\
18.01	0\\
19.01	0\\
20.01	0\\
21.01	0\\
22.01	0\\
23.01	0\\
24.01	0\\
25.01	0\\
26.01	0\\
27.01	0\\
28.01	0\\
29.01	0\\
30.01	0\\
31.01	0\\
32.01	0\\
33.01	0\\
34.01	0\\
35.01	0\\
36.01	0\\
37.01	0\\
38.01	0\\
39.01	0\\
40.01	0\\
41.01	0\\
42.01	0\\
43.01	0\\
44.01	0\\
45.01	0\\
46.01	0\\
47.01	0\\
48.01	0\\
49.01	0\\
50.01	0\\
51.01	0\\
52.01	0\\
53.01	0\\
54.01	0\\
55.01	0\\
56.01	0\\
57.01	0\\
58.01	0\\
59.01	0\\
60.01	0\\
61.01	0\\
62.01	0\\
63.01	0\\
64.01	0\\
65.01	0\\
66.01	0\\
67.01	0\\
68.01	0\\
69.01	0\\
70.01	0\\
71.01	0\\
72.01	0\\
73.01	0\\
74.01	0\\
75.01	0\\
76.01	0\\
77.01	0\\
78.01	0\\
79.01	0\\
80.01	0\\
81.01	0\\
82.01	0\\
83.01	0\\
84.01	0\\
85.01	0\\
86.01	0\\
87.01	0\\
88.01	0\\
89.01	0\\
90.01	0\\
91.01	0\\
92.01	0\\
93.01	0\\
94.01	0\\
95.01	0\\
96.01	0\\
97.01	0\\
98.01	0\\
99.01	0\\
100.01	0\\
101.01	0\\
102.01	0\\
103.01	0\\
104.01	0\\
105.01	0\\
106.01	0\\
107.01	0\\
108.01	0\\
109.01	0\\
110.01	0\\
111.01	0\\
112.01	0\\
113.01	0\\
114.01	0\\
115.01	0\\
116.01	0\\
117.01	0\\
118.01	0\\
119.01	0\\
120.01	0\\
121.01	0\\
122.01	0\\
123.01	0\\
124.01	0\\
125.01	0\\
126.01	0\\
127.01	0\\
128.01	0\\
129.01	0\\
130.01	0\\
131.01	0\\
132.01	0\\
133.01	0\\
134.01	0\\
135.01	0\\
136.01	0\\
137.01	0\\
138.01	0\\
139.01	0\\
140.01	0\\
141.01	0\\
142.01	0\\
143.01	0\\
144.01	0\\
145.01	0\\
146.01	0\\
147.01	0\\
148.01	0\\
149.01	0\\
150.01	0\\
151.01	0\\
152.01	0\\
153.01	0\\
154.01	0\\
155.01	0\\
156.01	0\\
157.01	0\\
158.01	0\\
159.01	0\\
160.01	0\\
161.01	0\\
162.01	0\\
163.01	0\\
164.01	0\\
165.01	0\\
166.01	0\\
167.01	0\\
168.01	0\\
169.01	0\\
170.01	0\\
171.01	0\\
172.01	0\\
173.01	0\\
174.01	0\\
175.01	0\\
176.01	0\\
177.01	0\\
178.01	0\\
179.01	0\\
180.01	0\\
181.01	0\\
182.01	0\\
183.01	0\\
184.01	0\\
185.01	0\\
186.01	0\\
187.01	0\\
188.01	0\\
189.01	0\\
190.01	0\\
191.01	0\\
192.01	0\\
193.01	0\\
194.01	0\\
195.01	0\\
196.01	0\\
197.01	0\\
198.01	0\\
199.01	0\\
200.01	0\\
201.01	0\\
202.01	0\\
203.01	0\\
204.01	0\\
205.01	0\\
206.01	0\\
207.01	0\\
208.01	0\\
209.01	0\\
210.01	0\\
211.01	0\\
212.01	0\\
213.01	0\\
214.01	0\\
215.01	0\\
216.01	0\\
217.01	0\\
218.01	0\\
219.01	0\\
220.01	0\\
221.01	0\\
222.01	0\\
223.01	0\\
224.01	0\\
225.01	0\\
226.01	0\\
227.01	0\\
228.01	0\\
229.01	0\\
230.01	0\\
231.01	0\\
232.01	0\\
233.01	0\\
234.01	0\\
235.01	0\\
236.01	0\\
237.01	0\\
238.01	0\\
239.01	0\\
240.01	0\\
241.01	0\\
242.01	0\\
243.01	0\\
244.01	0\\
245.01	0\\
246.01	0\\
247.01	0\\
248.01	0\\
249.01	0\\
250.01	0\\
251.01	0\\
252.01	0\\
253.01	0\\
254.01	0\\
255.01	0\\
256.01	0\\
257.01	0\\
258.01	0\\
259.01	0\\
260.01	0\\
261.01	0\\
262.01	0\\
263.01	0\\
264.01	0\\
265.01	0\\
266.01	0\\
267.01	0\\
268.01	0\\
269.01	0\\
270.01	0\\
271.01	0\\
272.01	0\\
273.01	0\\
274.01	0\\
275.01	0\\
276.01	0\\
277.01	0\\
278.01	0\\
279.01	0\\
280.01	0\\
281.01	0\\
282.01	0\\
283.01	0\\
284.01	0\\
285.01	0\\
286.01	0\\
287.01	0\\
288.01	0\\
289.01	0\\
290.01	0\\
291.01	0\\
292.01	0\\
293.01	0\\
294.01	0\\
295.01	0\\
296.01	0\\
297.01	0\\
298.01	0\\
299.01	0\\
300.01	0\\
301.01	0\\
302.01	0\\
303.01	0\\
304.01	0\\
305.01	0\\
306.01	0\\
307.01	0\\
308.01	0\\
309.01	0\\
310.01	0\\
311.01	0\\
312.01	0\\
313.01	0\\
314.01	0\\
315.01	0\\
316.01	0\\
317.01	0\\
318.01	0\\
319.01	0\\
320.01	0\\
321.01	0\\
322.01	0\\
323.01	0\\
324.01	0\\
325.01	0\\
326.01	0\\
327.01	0\\
328.01	0\\
329.01	0\\
330.01	0\\
331.01	0\\
332.01	0\\
333.01	0\\
334.01	0\\
335.01	0\\
336.01	0\\
337.01	0\\
338.01	0\\
339.01	0\\
340.01	0\\
341.01	0\\
342.01	0\\
343.01	0\\
344.01	0\\
345.01	0\\
346.01	0\\
347.01	0\\
348.01	0\\
349.01	0\\
350.01	0\\
351.01	0\\
352.01	0\\
353.01	0\\
354.01	0\\
355.01	0\\
356.01	0\\
357.01	0\\
358.01	0\\
359.01	0\\
360.01	0\\
361.01	0\\
362.01	0\\
363.01	0\\
364.01	0\\
365.01	0\\
366.01	0\\
367.01	0\\
368.01	0\\
369.01	0\\
370.01	0\\
371.01	0\\
372.01	0\\
373.01	0\\
374.01	0\\
375.01	0\\
376.01	0\\
377.01	0\\
378.01	0\\
379.01	0\\
380.01	0\\
381.01	0\\
382.01	0\\
383.01	0\\
384.01	0\\
385.01	0\\
386.01	0\\
387.01	0\\
388.01	0\\
389.01	0\\
390.01	0\\
391.01	0\\
392.01	0\\
393.01	0\\
394.01	0\\
395.01	0\\
396.01	0\\
397.01	0\\
398.01	0\\
399.01	0\\
400.01	0\\
401.01	0\\
402.01	0\\
403.01	0\\
404.01	0\\
405.01	0\\
406.01	0\\
407.01	0\\
408.01	0\\
409.01	0\\
410.01	0\\
411.01	0\\
412.01	0\\
413.01	0\\
414.01	0\\
415.01	0\\
416.01	0\\
417.01	0\\
418.01	0\\
419.01	0\\
420.01	0\\
421.01	0\\
422.01	0\\
423.01	0\\
424.01	0\\
425.01	0\\
426.01	0\\
427.01	0\\
428.01	0\\
429.01	0\\
430.01	0\\
431.01	0\\
432.01	0\\
433.01	0\\
434.01	0\\
435.01	0\\
436.01	0\\
437.01	0\\
438.01	0\\
439.01	0\\
440.01	0\\
441.01	0\\
442.01	0\\
443.01	0\\
444.01	0\\
445.01	0\\
446.01	0\\
447.01	0\\
448.01	0\\
449.01	0\\
450.01	0\\
451.01	0\\
452.01	0\\
453.01	0\\
454.01	0\\
455.01	0\\
456.01	0\\
457.01	0\\
458.01	0\\
459.01	0\\
460.01	0\\
461.01	0\\
462.01	0\\
463.01	0\\
464.01	0\\
465.01	0\\
466.01	0\\
467.01	0\\
468.01	0\\
469.01	0\\
470.01	0\\
471.01	0\\
472.01	0\\
473.01	0\\
474.01	0\\
475.01	0\\
476.01	0\\
477.01	0\\
478.01	0\\
479.01	0\\
480.01	0\\
481.01	0\\
482.01	0\\
483.01	0\\
484.01	0\\
485.01	0\\
486.01	0\\
487.01	0\\
488.01	0\\
489.01	0\\
490.01	0\\
491.01	0\\
492.01	0\\
493.01	0\\
494.01	0\\
495.01	0\\
496.01	0\\
497.01	0\\
498.01	0\\
499.01	0\\
500.01	0\\
501.01	0\\
502.01	0\\
503.01	0\\
504.01	0\\
505.01	0\\
506.01	0\\
507.01	0\\
508.01	0\\
509.01	0\\
510.01	0\\
511.01	0\\
512.01	0\\
513.01	0\\
514.01	0\\
515.01	0\\
516.01	0\\
517.01	0\\
518.01	0\\
519.01	0\\
520.01	0\\
521.01	0\\
522.01	0\\
523.01	0\\
524.01	0\\
525.01	0\\
526.01	0\\
527.01	0\\
528.01	0\\
529.01	0\\
530.01	0\\
531.01	0\\
532.01	0\\
533.01	0\\
534.01	0\\
535.01	0\\
536.01	0\\
537.01	0\\
538.01	0\\
539.01	0\\
540.01	0\\
541.01	0\\
542.01	0\\
543.01	0\\
544.01	0\\
545.01	0\\
546.01	0\\
547.01	0\\
548.01	0\\
549.01	0\\
550.01	0\\
551.01	0\\
552.01	0\\
553.01	0\\
554.01	0\\
555.01	0\\
556.01	0\\
557.01	0\\
558.01	0\\
559.01	0\\
560.01	0\\
561.01	0\\
562.01	0\\
563.01	0\\
564.01	0\\
565.01	0\\
566.01	0\\
567.01	0\\
568.01	0\\
569.01	0\\
570.01	0\\
571.01	0\\
572.01	0\\
573.01	0\\
574.01	0\\
575.01	0\\
576.01	0\\
577.01	0\\
578.01	0\\
579.01	0\\
580.01	0\\
581.01	0\\
582.01	0\\
583.01	0\\
584.01	0\\
585.01	0\\
586.01	0\\
587.01	0\\
588.01	0\\
589.01	0\\
590.01	0\\
591.01	0\\
592.01	0\\
593.01	0\\
594.01	0\\
595.01	0\\
596.01	0\\
597.01	0\\
598.01	0.00143832017857744\\
599.01	0.00385661255145558\\
599.02	0.00389414816850622\\
599.03	0.00393204144103766\\
599.04	0.00397029582311886\\
599.05	0.00400891480215704\\
599.06	0.00404790189921909\\
599.07	0.00408726066935627\\
599.08	0.00412699470193188\\
599.09	0.00416710762095217\\
599.1	0.00420760308540043\\
599.11	0.00424848478957431\\
599.12	0.00428975646342631\\
599.13	0.00433142187290768\\
599.14	0.00437348482031551\\
599.15	0.00441594914464327\\
599.16	0.00445881872193463\\
599.17	0.00450209746564076\\
599.18	0.00454578932698104\\
599.19	0.00458989829530726\\
599.2	0.0046344283984713\\
599.21	0.0046793837031964\\
599.22	0.00472476831545197\\
599.23	0.00477058638083201\\
599.24	0.00481684208493721\\
599.25	0.00486353965376069\\
599.26	0.00491068334338074\\
599.27	0.00495827744376709\\
599.28	0.00500632628611153\\
599.29	0.0050548342432236\\
599.3	0.00510380572993011\\
599.31	0.00515324520347841\\
599.32	0.00520315716394368\\
599.33	0.00525354615463998\\
599.34	0.00530441676253529\\
599.35	0.00535577361867056\\
599.36	0.00540762139858276\\
599.37	0.00545996482273196\\
599.38	0.00551280865693252\\
599.39	0.00556615771278843\\
599.4	0.00562001684813284\\
599.41	0.00567439096747174\\
599.42	0.00572928502243194\\
599.43	0.00578470401221333\\
599.44	0.00584065298404554\\
599.45	0.00589713703364883\\
599.46	0.00595416130569955\\
599.47	0.00601173099430001\\
599.48	0.00606985134345282\\
599.49	0.00612852764753978\\
599.5	0.00618776525180547\\
599.51	0.0062475695528453\\
599.52	0.0063079459990984\\
599.53	0.00636890009134512\\
599.54	0.00643043738320938\\
599.55	0.00649256348166578\\
599.56	0.0065552840475516\\
599.57	0.0066186047960837\\
599.58	0.00668253149738038\\
599.59	0.00674706997698825\\
599.6	0.00681222611641409\\
599.61	0.00687800585366193\\
599.62	0.00694441518377516\\
599.63	0.00701146015938389\\
599.64	0.00707914689125756\\
599.65	0.00714748154886281\\
599.66	0.00721647036092677\\
599.67	0.00728611961600567\\
599.68	0.00735643566305893\\
599.69	0.00742742491202878\\
599.7	0.00749909383442539\\
599.71	0.00757144896391765\\
599.72	0.00764449689692958\\
599.73	0.00771824429324247\\
599.74	0.0077926978766028\\
599.75	0.00786786443533595\\
599.76	0.00794375082296586\\
599.77	0.00802036395884057\\
599.78	0.00809771082876376\\
599.79	0.00817579848563237\\
599.8	0.00825463405008032\\
599.81	0.0083342247111284\\
599.82	0.00841457772684034\\
599.83	0.00849570042498529\\
599.84	0.00857760020370652\\
599.85	0.00866028453219658\\
599.86	0.00874376095137895\\
599.87	0.00882803707459619\\
599.88	0.00891312058830471\\
599.89	0.00899901925277615\\
599.9	0.00908574090280557\\
599.91	0.00917329344842634\\
599.92	0.0092616848756319\\
599.93	0.00935092324710449\\
599.94	0.00944101670295078\\
599.95	0.00953197346144465\\
599.96	0.00962380181977693\\
599.97	0.00971651015481255\\
599.98	0.0098101069238547\\
599.99	0.00990460066541651\\
600	0.01\\
};
\addplot [color=red!25!mycolor17,solid,forget plot]
  table[row sep=crcr]{%
0.01	0\\
1.01	0\\
2.01	0\\
3.01	0\\
4.01	0\\
5.01	0\\
6.01	0\\
7.01	0\\
8.01	0\\
9.01	0\\
10.01	0\\
11.01	0\\
12.01	0\\
13.01	0\\
14.01	0\\
15.01	0\\
16.01	0\\
17.01	0\\
18.01	0\\
19.01	0\\
20.01	0\\
21.01	0\\
22.01	0\\
23.01	0\\
24.01	0\\
25.01	0\\
26.01	0\\
27.01	0\\
28.01	0\\
29.01	0\\
30.01	0\\
31.01	0\\
32.01	0\\
33.01	0\\
34.01	0\\
35.01	0\\
36.01	0\\
37.01	0\\
38.01	0\\
39.01	0\\
40.01	0\\
41.01	0\\
42.01	0\\
43.01	0\\
44.01	0\\
45.01	0\\
46.01	0\\
47.01	0\\
48.01	0\\
49.01	0\\
50.01	0\\
51.01	0\\
52.01	0\\
53.01	0\\
54.01	0\\
55.01	0\\
56.01	0\\
57.01	0\\
58.01	0\\
59.01	0\\
60.01	0\\
61.01	0\\
62.01	0\\
63.01	0\\
64.01	0\\
65.01	0\\
66.01	0\\
67.01	0\\
68.01	0\\
69.01	0\\
70.01	0\\
71.01	0\\
72.01	0\\
73.01	0\\
74.01	0\\
75.01	0\\
76.01	0\\
77.01	0\\
78.01	0\\
79.01	0\\
80.01	0\\
81.01	0\\
82.01	0\\
83.01	0\\
84.01	0\\
85.01	0\\
86.01	0\\
87.01	0\\
88.01	0\\
89.01	0\\
90.01	0\\
91.01	0\\
92.01	0\\
93.01	0\\
94.01	0\\
95.01	0\\
96.01	0\\
97.01	0\\
98.01	0\\
99.01	0\\
100.01	0\\
101.01	0\\
102.01	0\\
103.01	0\\
104.01	0\\
105.01	0\\
106.01	0\\
107.01	0\\
108.01	0\\
109.01	0\\
110.01	0\\
111.01	0\\
112.01	0\\
113.01	0\\
114.01	0\\
115.01	0\\
116.01	0\\
117.01	0\\
118.01	0\\
119.01	0\\
120.01	0\\
121.01	0\\
122.01	0\\
123.01	0\\
124.01	0\\
125.01	0\\
126.01	0\\
127.01	0\\
128.01	0\\
129.01	0\\
130.01	0\\
131.01	0\\
132.01	0\\
133.01	0\\
134.01	0\\
135.01	0\\
136.01	0\\
137.01	0\\
138.01	0\\
139.01	0\\
140.01	0\\
141.01	0\\
142.01	0\\
143.01	0\\
144.01	0\\
145.01	0\\
146.01	0\\
147.01	0\\
148.01	0\\
149.01	0\\
150.01	0\\
151.01	0\\
152.01	0\\
153.01	0\\
154.01	0\\
155.01	0\\
156.01	0\\
157.01	0\\
158.01	0\\
159.01	0\\
160.01	0\\
161.01	0\\
162.01	0\\
163.01	0\\
164.01	0\\
165.01	0\\
166.01	0\\
167.01	0\\
168.01	0\\
169.01	0\\
170.01	0\\
171.01	0\\
172.01	0\\
173.01	0\\
174.01	0\\
175.01	0\\
176.01	0\\
177.01	0\\
178.01	0\\
179.01	0\\
180.01	0\\
181.01	0\\
182.01	0\\
183.01	0\\
184.01	0\\
185.01	0\\
186.01	0\\
187.01	0\\
188.01	0\\
189.01	0\\
190.01	0\\
191.01	0\\
192.01	0\\
193.01	0\\
194.01	0\\
195.01	0\\
196.01	0\\
197.01	0\\
198.01	0\\
199.01	0\\
200.01	0\\
201.01	0\\
202.01	0\\
203.01	0\\
204.01	0\\
205.01	0\\
206.01	0\\
207.01	0\\
208.01	0\\
209.01	0\\
210.01	0\\
211.01	0\\
212.01	0\\
213.01	0\\
214.01	0\\
215.01	0\\
216.01	0\\
217.01	0\\
218.01	0\\
219.01	0\\
220.01	0\\
221.01	0\\
222.01	0\\
223.01	0\\
224.01	0\\
225.01	0\\
226.01	0\\
227.01	0\\
228.01	0\\
229.01	0\\
230.01	0\\
231.01	0\\
232.01	0\\
233.01	0\\
234.01	0\\
235.01	0\\
236.01	0\\
237.01	0\\
238.01	0\\
239.01	0\\
240.01	0\\
241.01	0\\
242.01	0\\
243.01	0\\
244.01	0\\
245.01	0\\
246.01	0\\
247.01	0\\
248.01	0\\
249.01	0\\
250.01	0\\
251.01	0\\
252.01	0\\
253.01	0\\
254.01	0\\
255.01	0\\
256.01	0\\
257.01	0\\
258.01	0\\
259.01	0\\
260.01	0\\
261.01	0\\
262.01	0\\
263.01	0\\
264.01	0\\
265.01	0\\
266.01	0\\
267.01	0\\
268.01	0\\
269.01	0\\
270.01	0\\
271.01	0\\
272.01	0\\
273.01	0\\
274.01	0\\
275.01	0\\
276.01	0\\
277.01	0\\
278.01	0\\
279.01	0\\
280.01	0\\
281.01	0\\
282.01	0\\
283.01	0\\
284.01	0\\
285.01	0\\
286.01	0\\
287.01	0\\
288.01	0\\
289.01	0\\
290.01	0\\
291.01	0\\
292.01	0\\
293.01	0\\
294.01	0\\
295.01	0\\
296.01	0\\
297.01	0\\
298.01	0\\
299.01	0\\
300.01	0\\
301.01	0\\
302.01	0\\
303.01	0\\
304.01	0\\
305.01	0\\
306.01	0\\
307.01	0\\
308.01	0\\
309.01	0\\
310.01	0\\
311.01	0\\
312.01	0\\
313.01	0\\
314.01	0\\
315.01	0\\
316.01	0\\
317.01	0\\
318.01	0\\
319.01	0\\
320.01	0\\
321.01	0\\
322.01	0\\
323.01	0\\
324.01	0\\
325.01	0\\
326.01	0\\
327.01	0\\
328.01	0\\
329.01	0\\
330.01	0\\
331.01	0\\
332.01	0\\
333.01	0\\
334.01	0\\
335.01	0\\
336.01	0\\
337.01	0\\
338.01	0\\
339.01	0\\
340.01	0\\
341.01	0\\
342.01	0\\
343.01	0\\
344.01	0\\
345.01	0\\
346.01	0\\
347.01	0\\
348.01	0\\
349.01	0\\
350.01	0\\
351.01	0\\
352.01	0\\
353.01	0\\
354.01	0\\
355.01	0\\
356.01	0\\
357.01	0\\
358.01	0\\
359.01	0\\
360.01	0\\
361.01	0\\
362.01	0\\
363.01	0\\
364.01	0\\
365.01	0\\
366.01	0\\
367.01	0\\
368.01	0\\
369.01	0\\
370.01	0\\
371.01	0\\
372.01	0\\
373.01	0\\
374.01	0\\
375.01	0\\
376.01	0\\
377.01	0\\
378.01	0\\
379.01	0\\
380.01	0\\
381.01	0\\
382.01	0\\
383.01	0\\
384.01	0\\
385.01	0\\
386.01	0\\
387.01	0\\
388.01	0\\
389.01	0\\
390.01	0\\
391.01	0\\
392.01	0\\
393.01	0\\
394.01	0\\
395.01	0\\
396.01	0\\
397.01	0\\
398.01	0\\
399.01	0\\
400.01	0\\
401.01	0\\
402.01	0\\
403.01	0\\
404.01	0\\
405.01	0\\
406.01	0\\
407.01	0\\
408.01	0\\
409.01	0\\
410.01	0\\
411.01	0\\
412.01	0\\
413.01	0\\
414.01	0\\
415.01	0\\
416.01	0\\
417.01	0\\
418.01	0\\
419.01	0\\
420.01	0\\
421.01	0\\
422.01	0\\
423.01	0\\
424.01	0\\
425.01	0\\
426.01	0\\
427.01	0\\
428.01	0\\
429.01	0\\
430.01	0\\
431.01	0\\
432.01	0\\
433.01	0\\
434.01	0\\
435.01	0\\
436.01	0\\
437.01	0\\
438.01	0\\
439.01	0\\
440.01	0\\
441.01	0\\
442.01	0\\
443.01	0\\
444.01	0\\
445.01	0\\
446.01	0\\
447.01	0\\
448.01	0\\
449.01	0\\
450.01	0\\
451.01	0\\
452.01	0\\
453.01	0\\
454.01	0\\
455.01	0\\
456.01	0\\
457.01	0\\
458.01	0\\
459.01	0\\
460.01	0\\
461.01	0\\
462.01	0\\
463.01	0\\
464.01	0\\
465.01	0\\
466.01	0\\
467.01	0\\
468.01	0\\
469.01	0\\
470.01	0\\
471.01	0\\
472.01	0\\
473.01	0\\
474.01	0\\
475.01	0\\
476.01	0\\
477.01	0\\
478.01	0\\
479.01	0\\
480.01	0\\
481.01	0\\
482.01	0\\
483.01	0\\
484.01	0\\
485.01	0\\
486.01	0\\
487.01	0\\
488.01	0\\
489.01	0\\
490.01	0\\
491.01	0\\
492.01	0\\
493.01	0\\
494.01	0\\
495.01	0\\
496.01	0\\
497.01	0\\
498.01	0\\
499.01	0\\
500.01	0\\
501.01	0\\
502.01	0\\
503.01	0\\
504.01	0\\
505.01	0\\
506.01	0\\
507.01	0\\
508.01	0\\
509.01	0\\
510.01	0\\
511.01	0\\
512.01	0\\
513.01	0\\
514.01	0\\
515.01	0\\
516.01	0\\
517.01	0\\
518.01	0\\
519.01	0\\
520.01	0\\
521.01	0\\
522.01	0\\
523.01	0\\
524.01	0\\
525.01	0\\
526.01	0\\
527.01	0\\
528.01	0\\
529.01	0\\
530.01	0\\
531.01	0\\
532.01	0\\
533.01	0\\
534.01	0\\
535.01	0\\
536.01	0\\
537.01	0\\
538.01	0\\
539.01	0\\
540.01	0\\
541.01	0\\
542.01	0\\
543.01	0\\
544.01	0\\
545.01	0\\
546.01	0\\
547.01	0\\
548.01	0\\
549.01	0\\
550.01	0\\
551.01	0\\
552.01	0\\
553.01	0\\
554.01	0\\
555.01	0\\
556.01	0\\
557.01	0\\
558.01	0\\
559.01	0\\
560.01	0\\
561.01	0\\
562.01	0\\
563.01	0\\
564.01	0\\
565.01	0\\
566.01	0\\
567.01	0\\
568.01	0\\
569.01	0\\
570.01	0\\
571.01	0\\
572.01	0\\
573.01	0\\
574.01	0\\
575.01	0\\
576.01	0\\
577.01	0\\
578.01	0\\
579.01	0\\
580.01	0\\
581.01	0\\
582.01	0\\
583.01	0\\
584.01	0\\
585.01	0\\
586.01	0\\
587.01	0\\
588.01	0\\
589.01	0\\
590.01	0\\
591.01	0\\
592.01	0\\
593.01	0\\
594.01	0\\
595.01	0\\
596.01	0\\
597.01	0\\
598.01	0.00143832067364899\\
599.01	0.00385661257178434\\
599.02	0.0038941481880362\\
599.03	0.00393204145978999\\
599.04	0.00397029584111443\\
599.05	0.00400891481941645\\
599.06	0.00404790191576273\\
599.07	0.00408726068520424\\
599.08	0.00412699471710402\\
599.09	0.00416710763546808\\
599.1	0.00420760309927944\\
599.11	0.00424848480283546\\
599.12	0.00428975647608841\\
599.13	0.00433142188498926\\
599.14	0.00437348483183484\\
599.15	0.00441594915561834\\
599.16	0.00445881873238316\\
599.17	0.00450209747558018\\
599.18	0.00454578933642853\\
599.19	0.00458989830427969\\
599.2	0.00463442840698527\\
599.21	0.00467938371126824\\
599.22	0.00472476832309771\\
599.23	0.00477058638806738\\
599.24	0.00481684209177767\\
599.25	0.0048635396602214\\
599.26	0.00491068334947654\\
599.27	0.00495827744951256\\
599.28	0.00500632629152094\\
599.29	0.00505483424831089\\
599.3	0.00510380573470894\\
599.31	0.00515324520796213\\
599.32	0.00520315716814533\\
599.33	0.00525354615857229\\
599.34	0.00530441676621069\\
599.35	0.00535577362210116\\
599.36	0.00540762140178036\\
599.37	0.00545996482570803\\
599.38	0.00551280865969823\\
599.39	0.00556615771535462\\
599.4	0.00562001685051006\\
599.41	0.00567439096967018\\
599.42	0.00572928502446148\\
599.43	0.00578470401408357\\
599.44	0.00584065298576571\\
599.45	0.00589713703522788\\
599.46	0.0059541613071461\\
599.47	0.00601173099562236\\
599.48	0.00606985134465893\\
599.49	0.00612852764863735\\
599.5	0.00618776525280185\\
599.51	0.00624756955374753\\
599.52	0.00630794599991322\\
599.53	0.00636890009207898\\
599.54	0.00643043738386841\\
599.55	0.00649256348225581\\
599.56	0.00655528404807817\\
599.57	0.00661860479655207\\
599.58	0.00668253149779552\\
599.59	0.00674706997735484\\
599.6	0.00681222611673654\\
599.61	0.00687800585394439\\
599.62	0.00694441518402151\\
599.63	0.00701146015959775\\
599.64	0.0070791468914423\\
599.65	0.00714748154902157\\
599.66	0.00721647036106244\\
599.67	0.00728611961612093\\
599.68	0.00735643566315622\\
599.69	0.00742742491211035\\
599.7	0.00749909383449329\\
599.71	0.00757144896397373\\
599.72	0.0076444968969755\\
599.73	0.00771824429327973\\
599.74	0.00779269787663272\\
599.75	0.00786786443535973\\
599.76	0.00794375082298454\\
599.77	0.00802036395885504\\
599.78	0.00809771082877482\\
599.79	0.00817579848564069\\
599.8	0.00825463405008646\\
599.81	0.00833422471113284\\
599.82	0.00841457772684348\\
599.83	0.00849570042498746\\
599.84	0.00857760020370797\\
599.85	0.00866028453219752\\
599.86	0.00874376095137954\\
599.87	0.00882803707459654\\
599.88	0.0089131205883049\\
599.89	0.00899901925277625\\
599.9	0.00908574090280562\\
599.91	0.00917329344842636\\
599.92	0.0092616848756319\\
599.93	0.00935092324710449\\
599.94	0.00944101670295079\\
599.95	0.00953197346144465\\
599.96	0.00962380181977694\\
599.97	0.00971651015481255\\
599.98	0.0098101069238547\\
599.99	0.00990460066541651\\
600	0.01\\
};
\addplot [color=mycolor19,solid,forget plot]
  table[row sep=crcr]{%
0.01	0\\
1.01	0\\
2.01	0\\
3.01	0\\
4.01	0\\
5.01	0\\
6.01	0\\
7.01	0\\
8.01	0\\
9.01	0\\
10.01	0\\
11.01	0\\
12.01	0\\
13.01	0\\
14.01	0\\
15.01	0\\
16.01	0\\
17.01	0\\
18.01	0\\
19.01	0\\
20.01	0\\
21.01	0\\
22.01	0\\
23.01	0\\
24.01	0\\
25.01	0\\
26.01	0\\
27.01	0\\
28.01	0\\
29.01	0\\
30.01	0\\
31.01	0\\
32.01	0\\
33.01	0\\
34.01	0\\
35.01	0\\
36.01	0\\
37.01	0\\
38.01	0\\
39.01	0\\
40.01	0\\
41.01	0\\
42.01	0\\
43.01	0\\
44.01	0\\
45.01	0\\
46.01	0\\
47.01	0\\
48.01	0\\
49.01	0\\
50.01	0\\
51.01	0\\
52.01	0\\
53.01	0\\
54.01	0\\
55.01	0\\
56.01	0\\
57.01	0\\
58.01	0\\
59.01	0\\
60.01	0\\
61.01	0\\
62.01	0\\
63.01	0\\
64.01	0\\
65.01	0\\
66.01	0\\
67.01	0\\
68.01	0\\
69.01	0\\
70.01	0\\
71.01	0\\
72.01	0\\
73.01	0\\
74.01	0\\
75.01	0\\
76.01	0\\
77.01	0\\
78.01	0\\
79.01	0\\
80.01	0\\
81.01	0\\
82.01	0\\
83.01	0\\
84.01	0\\
85.01	0\\
86.01	0\\
87.01	0\\
88.01	0\\
89.01	0\\
90.01	0\\
91.01	0\\
92.01	0\\
93.01	0\\
94.01	0\\
95.01	0\\
96.01	0\\
97.01	0\\
98.01	0\\
99.01	0\\
100.01	0\\
101.01	0\\
102.01	0\\
103.01	0\\
104.01	0\\
105.01	0\\
106.01	0\\
107.01	0\\
108.01	0\\
109.01	0\\
110.01	0\\
111.01	0\\
112.01	0\\
113.01	0\\
114.01	0\\
115.01	0\\
116.01	0\\
117.01	0\\
118.01	0\\
119.01	0\\
120.01	0\\
121.01	0\\
122.01	0\\
123.01	0\\
124.01	0\\
125.01	0\\
126.01	0\\
127.01	0\\
128.01	0\\
129.01	0\\
130.01	0\\
131.01	0\\
132.01	0\\
133.01	0\\
134.01	0\\
135.01	0\\
136.01	0\\
137.01	0\\
138.01	0\\
139.01	0\\
140.01	0\\
141.01	0\\
142.01	0\\
143.01	0\\
144.01	0\\
145.01	0\\
146.01	0\\
147.01	0\\
148.01	0\\
149.01	0\\
150.01	0\\
151.01	0\\
152.01	0\\
153.01	0\\
154.01	0\\
155.01	0\\
156.01	0\\
157.01	0\\
158.01	0\\
159.01	0\\
160.01	0\\
161.01	0\\
162.01	0\\
163.01	0\\
164.01	0\\
165.01	0\\
166.01	0\\
167.01	0\\
168.01	0\\
169.01	0\\
170.01	0\\
171.01	0\\
172.01	0\\
173.01	0\\
174.01	0\\
175.01	0\\
176.01	0\\
177.01	0\\
178.01	0\\
179.01	0\\
180.01	0\\
181.01	0\\
182.01	0\\
183.01	0\\
184.01	0\\
185.01	0\\
186.01	0\\
187.01	0\\
188.01	0\\
189.01	0\\
190.01	0\\
191.01	0\\
192.01	0\\
193.01	0\\
194.01	0\\
195.01	0\\
196.01	0\\
197.01	0\\
198.01	0\\
199.01	0\\
200.01	0\\
201.01	0\\
202.01	0\\
203.01	0\\
204.01	0\\
205.01	0\\
206.01	0\\
207.01	0\\
208.01	0\\
209.01	0\\
210.01	0\\
211.01	0\\
212.01	0\\
213.01	0\\
214.01	0\\
215.01	0\\
216.01	0\\
217.01	0\\
218.01	0\\
219.01	0\\
220.01	0\\
221.01	0\\
222.01	0\\
223.01	0\\
224.01	0\\
225.01	0\\
226.01	0\\
227.01	0\\
228.01	0\\
229.01	0\\
230.01	0\\
231.01	0\\
232.01	0\\
233.01	0\\
234.01	0\\
235.01	0\\
236.01	0\\
237.01	0\\
238.01	0\\
239.01	0\\
240.01	0\\
241.01	0\\
242.01	0\\
243.01	0\\
244.01	0\\
245.01	0\\
246.01	0\\
247.01	0\\
248.01	0\\
249.01	0\\
250.01	0\\
251.01	0\\
252.01	0\\
253.01	0\\
254.01	0\\
255.01	0\\
256.01	0\\
257.01	0\\
258.01	0\\
259.01	0\\
260.01	0\\
261.01	0\\
262.01	0\\
263.01	0\\
264.01	0\\
265.01	0\\
266.01	0\\
267.01	0\\
268.01	0\\
269.01	0\\
270.01	0\\
271.01	0\\
272.01	0\\
273.01	0\\
274.01	0\\
275.01	0\\
276.01	0\\
277.01	0\\
278.01	0\\
279.01	0\\
280.01	0\\
281.01	0\\
282.01	0\\
283.01	0\\
284.01	0\\
285.01	0\\
286.01	0\\
287.01	0\\
288.01	0\\
289.01	0\\
290.01	0\\
291.01	0\\
292.01	0\\
293.01	0\\
294.01	0\\
295.01	0\\
296.01	0\\
297.01	0\\
298.01	0\\
299.01	0\\
300.01	0\\
301.01	0\\
302.01	0\\
303.01	0\\
304.01	0\\
305.01	0\\
306.01	0\\
307.01	0\\
308.01	0\\
309.01	0\\
310.01	0\\
311.01	0\\
312.01	0\\
313.01	0\\
314.01	0\\
315.01	0\\
316.01	0\\
317.01	0\\
318.01	0\\
319.01	0\\
320.01	0\\
321.01	0\\
322.01	0\\
323.01	0\\
324.01	0\\
325.01	0\\
326.01	0\\
327.01	0\\
328.01	0\\
329.01	0\\
330.01	0\\
331.01	0\\
332.01	0\\
333.01	0\\
334.01	0\\
335.01	0\\
336.01	0\\
337.01	0\\
338.01	0\\
339.01	0\\
340.01	0\\
341.01	0\\
342.01	0\\
343.01	0\\
344.01	0\\
345.01	0\\
346.01	0\\
347.01	0\\
348.01	0\\
349.01	0\\
350.01	0\\
351.01	0\\
352.01	0\\
353.01	0\\
354.01	0\\
355.01	0\\
356.01	0\\
357.01	0\\
358.01	0\\
359.01	0\\
360.01	0\\
361.01	0\\
362.01	0\\
363.01	0\\
364.01	0\\
365.01	0\\
366.01	0\\
367.01	0\\
368.01	0\\
369.01	0\\
370.01	0\\
371.01	0\\
372.01	0\\
373.01	0\\
374.01	0\\
375.01	0\\
376.01	0\\
377.01	0\\
378.01	0\\
379.01	0\\
380.01	0\\
381.01	0\\
382.01	0\\
383.01	0\\
384.01	0\\
385.01	0\\
386.01	0\\
387.01	0\\
388.01	0\\
389.01	0\\
390.01	0\\
391.01	0\\
392.01	0\\
393.01	0\\
394.01	0\\
395.01	0\\
396.01	0\\
397.01	0\\
398.01	0\\
399.01	0\\
400.01	0\\
401.01	0\\
402.01	0\\
403.01	0\\
404.01	0\\
405.01	0\\
406.01	0\\
407.01	0\\
408.01	0\\
409.01	0\\
410.01	0\\
411.01	0\\
412.01	0\\
413.01	0\\
414.01	0\\
415.01	0\\
416.01	0\\
417.01	0\\
418.01	0\\
419.01	0\\
420.01	0\\
421.01	0\\
422.01	0\\
423.01	0\\
424.01	0\\
425.01	0\\
426.01	0\\
427.01	0\\
428.01	0\\
429.01	0\\
430.01	0\\
431.01	0\\
432.01	0\\
433.01	0\\
434.01	0\\
435.01	0\\
436.01	0\\
437.01	0\\
438.01	0\\
439.01	0\\
440.01	0\\
441.01	0\\
442.01	0\\
443.01	0\\
444.01	0\\
445.01	0\\
446.01	0\\
447.01	0\\
448.01	0\\
449.01	0\\
450.01	0\\
451.01	0\\
452.01	0\\
453.01	0\\
454.01	0\\
455.01	0\\
456.01	0\\
457.01	0\\
458.01	0\\
459.01	0\\
460.01	0\\
461.01	0\\
462.01	0\\
463.01	0\\
464.01	0\\
465.01	0\\
466.01	0\\
467.01	0\\
468.01	0\\
469.01	0\\
470.01	0\\
471.01	0\\
472.01	0\\
473.01	0\\
474.01	0\\
475.01	0\\
476.01	0\\
477.01	0\\
478.01	0\\
479.01	0\\
480.01	0\\
481.01	0\\
482.01	0\\
483.01	0\\
484.01	0\\
485.01	0\\
486.01	0\\
487.01	0\\
488.01	0\\
489.01	0\\
490.01	0\\
491.01	0\\
492.01	0\\
493.01	0\\
494.01	0\\
495.01	0\\
496.01	0\\
497.01	0\\
498.01	0\\
499.01	0\\
500.01	0\\
501.01	0\\
502.01	0\\
503.01	0\\
504.01	0\\
505.01	0\\
506.01	0\\
507.01	0\\
508.01	0\\
509.01	0\\
510.01	0\\
511.01	0\\
512.01	0\\
513.01	0\\
514.01	0\\
515.01	0\\
516.01	0\\
517.01	0\\
518.01	0\\
519.01	0\\
520.01	0\\
521.01	0\\
522.01	0\\
523.01	0\\
524.01	0\\
525.01	0\\
526.01	0\\
527.01	0\\
528.01	0\\
529.01	0\\
530.01	0\\
531.01	0\\
532.01	0\\
533.01	0\\
534.01	0\\
535.01	0\\
536.01	0\\
537.01	0\\
538.01	0\\
539.01	0\\
540.01	0\\
541.01	0\\
542.01	0\\
543.01	0\\
544.01	0\\
545.01	0\\
546.01	0\\
547.01	0\\
548.01	0\\
549.01	0\\
550.01	0\\
551.01	0\\
552.01	0\\
553.01	0\\
554.01	0\\
555.01	0\\
556.01	0\\
557.01	0\\
558.01	0\\
559.01	0\\
560.01	0\\
561.01	0\\
562.01	0\\
563.01	0\\
564.01	0\\
565.01	0\\
566.01	0\\
567.01	0\\
568.01	0\\
569.01	0\\
570.01	0\\
571.01	0\\
572.01	0\\
573.01	0\\
574.01	0\\
575.01	0\\
576.01	0\\
577.01	0\\
578.01	0\\
579.01	0\\
580.01	0\\
581.01	0\\
582.01	0\\
583.01	0\\
584.01	0\\
585.01	0\\
586.01	0\\
587.01	0\\
588.01	0\\
589.01	0\\
590.01	0\\
591.01	0\\
592.01	0\\
593.01	0\\
594.01	0\\
595.01	0\\
596.01	0\\
597.01	0\\
598.01	0.00143832068163977\\
599.01	0.00385661257211143\\
599.02	0.00389414818834777\\
599.03	0.00393204146008657\\
599.04	0.00397029584139656\\
599.05	0.00400891481968466\\
599.06	0.00404790191601751\\
599.07	0.00408726068544609\\
599.08	0.00412699471733344\\
599.09	0.00416710763568554\\
599.1	0.0042076030994854\\
599.11	0.00424848480303038\\
599.12	0.00428975647627273\\
599.13	0.0043314218851634\\
599.14	0.00437348483199923\\
599.15	0.00441594915577338\\
599.16	0.00445881873252926\\
599.17	0.00450209747571773\\
599.18	0.00454578933655789\\
599.19	0.00458989830440124\\
599.2	0.00463442840709936\\
599.21	0.00467938371137522\\
599.22	0.0047247683231979\\
599.23	0.00477058638816113\\
599.24	0.00481684209186529\\
599.25	0.00486353966030319\\
599.26	0.0049106833495528\\
599.27	0.00495827744958357\\
599.28	0.00500632629158696\\
599.29	0.00505483424837222\\
599.3	0.00510380573476581\\
599.31	0.00515324520801481\\
599.32	0.00520315716819406\\
599.33	0.00525354615861728\\
599.34	0.00530441676625217\\
599.35	0.00535577362213935\\
599.36	0.00540762140181545\\
599.37	0.00545996482574022\\
599.38	0.0055128086597277\\
599.39	0.00556615771538157\\
599.4	0.00562001685053463\\
599.41	0.00567439096969254\\
599.42	0.00572928502448181\\
599.43	0.00578470401410199\\
599.44	0.00584065298578238\\
599.45	0.00589713703524292\\
599.46	0.00595416130715964\\
599.47	0.00601173099563451\\
599.48	0.00606985134466982\\
599.49	0.00612852764864707\\
599.5	0.0061877652528105\\
599.51	0.00624756955375522\\
599.52	0.00630794599992002\\
599.53	0.00636890009208498\\
599.54	0.00643043738387368\\
599.55	0.00649256348226043\\
599.56	0.0065552840480822\\
599.57	0.00661860479655558\\
599.58	0.00668253149779855\\
599.59	0.00674706997735745\\
599.6	0.00681222611673879\\
599.61	0.00687800585394631\\
599.62	0.00694441518402313\\
599.63	0.00701146015959912\\
599.64	0.00707914689144344\\
599.65	0.00714748154902252\\
599.66	0.00721647036106323\\
599.67	0.00728611961612157\\
599.68	0.00735643566315675\\
599.69	0.00742742491211077\\
599.7	0.00749909383449363\\
599.71	0.00757144896397399\\
599.72	0.0076444968969757\\
599.73	0.00771824429327989\\
599.74	0.00779269787663284\\
599.75	0.00786786443535982\\
599.76	0.0079437508229846\\
599.77	0.00802036395885509\\
599.78	0.00809771082877485\\
599.79	0.00817579848564071\\
599.8	0.00825463405008648\\
599.81	0.00833422471113285\\
599.82	0.00841457772684349\\
599.83	0.00849570042498746\\
599.84	0.00857760020370797\\
599.85	0.00866028453219752\\
599.86	0.00874376095137953\\
599.87	0.00882803707459654\\
599.88	0.0089131205883049\\
599.89	0.00899901925277625\\
599.9	0.00908574090280562\\
599.91	0.00917329344842636\\
599.92	0.0092616848756319\\
599.93	0.00935092324710449\\
599.94	0.00944101670295078\\
599.95	0.00953197346144465\\
599.96	0.00962380181977693\\
599.97	0.00971651015481255\\
599.98	0.0098101069238547\\
599.99	0.00990460066541651\\
600	0.01\\
};
\addplot [color=red!50!mycolor17,solid,forget plot]
  table[row sep=crcr]{%
0.01	0\\
1.01	0\\
2.01	0\\
3.01	0\\
4.01	0\\
5.01	0\\
6.01	0\\
7.01	0\\
8.01	0\\
9.01	0\\
10.01	0\\
11.01	0\\
12.01	0\\
13.01	0\\
14.01	0\\
15.01	0\\
16.01	0\\
17.01	0\\
18.01	0\\
19.01	0\\
20.01	0\\
21.01	0\\
22.01	0\\
23.01	0\\
24.01	0\\
25.01	0\\
26.01	0\\
27.01	0\\
28.01	0\\
29.01	0\\
30.01	0\\
31.01	0\\
32.01	0\\
33.01	0\\
34.01	0\\
35.01	0\\
36.01	0\\
37.01	0\\
38.01	0\\
39.01	0\\
40.01	0\\
41.01	0\\
42.01	0\\
43.01	0\\
44.01	0\\
45.01	0\\
46.01	0\\
47.01	0\\
48.01	0\\
49.01	0\\
50.01	0\\
51.01	0\\
52.01	0\\
53.01	0\\
54.01	0\\
55.01	0\\
56.01	0\\
57.01	0\\
58.01	0\\
59.01	0\\
60.01	0\\
61.01	0\\
62.01	0\\
63.01	0\\
64.01	0\\
65.01	0\\
66.01	0\\
67.01	0\\
68.01	0\\
69.01	0\\
70.01	0\\
71.01	0\\
72.01	0\\
73.01	0\\
74.01	0\\
75.01	0\\
76.01	0\\
77.01	0\\
78.01	0\\
79.01	0\\
80.01	0\\
81.01	0\\
82.01	0\\
83.01	0\\
84.01	0\\
85.01	0\\
86.01	0\\
87.01	0\\
88.01	0\\
89.01	0\\
90.01	0\\
91.01	0\\
92.01	0\\
93.01	0\\
94.01	0\\
95.01	0\\
96.01	0\\
97.01	0\\
98.01	0\\
99.01	0\\
100.01	0\\
101.01	0\\
102.01	0\\
103.01	0\\
104.01	0\\
105.01	0\\
106.01	0\\
107.01	0\\
108.01	0\\
109.01	0\\
110.01	0\\
111.01	0\\
112.01	0\\
113.01	0\\
114.01	0\\
115.01	0\\
116.01	0\\
117.01	0\\
118.01	0\\
119.01	0\\
120.01	0\\
121.01	0\\
122.01	0\\
123.01	0\\
124.01	0\\
125.01	0\\
126.01	0\\
127.01	0\\
128.01	0\\
129.01	0\\
130.01	0\\
131.01	0\\
132.01	0\\
133.01	0\\
134.01	0\\
135.01	0\\
136.01	0\\
137.01	0\\
138.01	0\\
139.01	0\\
140.01	0\\
141.01	0\\
142.01	0\\
143.01	0\\
144.01	0\\
145.01	0\\
146.01	0\\
147.01	0\\
148.01	0\\
149.01	0\\
150.01	0\\
151.01	0\\
152.01	0\\
153.01	0\\
154.01	0\\
155.01	0\\
156.01	0\\
157.01	0\\
158.01	0\\
159.01	0\\
160.01	0\\
161.01	0\\
162.01	0\\
163.01	0\\
164.01	0\\
165.01	0\\
166.01	0\\
167.01	0\\
168.01	0\\
169.01	0\\
170.01	0\\
171.01	0\\
172.01	0\\
173.01	0\\
174.01	0\\
175.01	0\\
176.01	0\\
177.01	0\\
178.01	0\\
179.01	0\\
180.01	0\\
181.01	0\\
182.01	0\\
183.01	0\\
184.01	0\\
185.01	0\\
186.01	0\\
187.01	0\\
188.01	0\\
189.01	0\\
190.01	0\\
191.01	0\\
192.01	0\\
193.01	0\\
194.01	0\\
195.01	0\\
196.01	0\\
197.01	0\\
198.01	0\\
199.01	0\\
200.01	0\\
201.01	0\\
202.01	0\\
203.01	0\\
204.01	0\\
205.01	0\\
206.01	0\\
207.01	0\\
208.01	0\\
209.01	0\\
210.01	0\\
211.01	0\\
212.01	0\\
213.01	0\\
214.01	0\\
215.01	0\\
216.01	0\\
217.01	0\\
218.01	0\\
219.01	0\\
220.01	0\\
221.01	0\\
222.01	0\\
223.01	0\\
224.01	0\\
225.01	0\\
226.01	0\\
227.01	0\\
228.01	0\\
229.01	0\\
230.01	0\\
231.01	0\\
232.01	0\\
233.01	0\\
234.01	0\\
235.01	0\\
236.01	0\\
237.01	0\\
238.01	0\\
239.01	0\\
240.01	0\\
241.01	0\\
242.01	0\\
243.01	0\\
244.01	0\\
245.01	0\\
246.01	0\\
247.01	0\\
248.01	0\\
249.01	0\\
250.01	0\\
251.01	0\\
252.01	0\\
253.01	0\\
254.01	0\\
255.01	0\\
256.01	0\\
257.01	0\\
258.01	0\\
259.01	0\\
260.01	0\\
261.01	0\\
262.01	0\\
263.01	0\\
264.01	0\\
265.01	0\\
266.01	0\\
267.01	0\\
268.01	0\\
269.01	0\\
270.01	0\\
271.01	0\\
272.01	0\\
273.01	0\\
274.01	0\\
275.01	0\\
276.01	0\\
277.01	0\\
278.01	0\\
279.01	0\\
280.01	0\\
281.01	0\\
282.01	0\\
283.01	0\\
284.01	0\\
285.01	0\\
286.01	0\\
287.01	0\\
288.01	0\\
289.01	0\\
290.01	0\\
291.01	0\\
292.01	0\\
293.01	0\\
294.01	0\\
295.01	0\\
296.01	0\\
297.01	0\\
298.01	0\\
299.01	0\\
300.01	0\\
301.01	0\\
302.01	0\\
303.01	0\\
304.01	0\\
305.01	0\\
306.01	0\\
307.01	0\\
308.01	0\\
309.01	0\\
310.01	0\\
311.01	0\\
312.01	0\\
313.01	0\\
314.01	0\\
315.01	0\\
316.01	0\\
317.01	0\\
318.01	0\\
319.01	0\\
320.01	0\\
321.01	0\\
322.01	0\\
323.01	0\\
324.01	0\\
325.01	0\\
326.01	0\\
327.01	0\\
328.01	0\\
329.01	0\\
330.01	0\\
331.01	0\\
332.01	0\\
333.01	0\\
334.01	0\\
335.01	0\\
336.01	0\\
337.01	0\\
338.01	0\\
339.01	0\\
340.01	0\\
341.01	0\\
342.01	0\\
343.01	0\\
344.01	0\\
345.01	0\\
346.01	0\\
347.01	0\\
348.01	0\\
349.01	0\\
350.01	0\\
351.01	0\\
352.01	0\\
353.01	0\\
354.01	0\\
355.01	0\\
356.01	0\\
357.01	0\\
358.01	0\\
359.01	0\\
360.01	0\\
361.01	0\\
362.01	0\\
363.01	0\\
364.01	0\\
365.01	0\\
366.01	0\\
367.01	0\\
368.01	0\\
369.01	0\\
370.01	0\\
371.01	0\\
372.01	0\\
373.01	0\\
374.01	0\\
375.01	0\\
376.01	0\\
377.01	0\\
378.01	0\\
379.01	0\\
380.01	0\\
381.01	0\\
382.01	0\\
383.01	0\\
384.01	0\\
385.01	0\\
386.01	0\\
387.01	0\\
388.01	0\\
389.01	0\\
390.01	0\\
391.01	0\\
392.01	0\\
393.01	0\\
394.01	0\\
395.01	0\\
396.01	0\\
397.01	0\\
398.01	0\\
399.01	0\\
400.01	0\\
401.01	0\\
402.01	0\\
403.01	0\\
404.01	0\\
405.01	0\\
406.01	0\\
407.01	0\\
408.01	0\\
409.01	0\\
410.01	0\\
411.01	0\\
412.01	0\\
413.01	0\\
414.01	0\\
415.01	0\\
416.01	0\\
417.01	0\\
418.01	0\\
419.01	0\\
420.01	0\\
421.01	0\\
422.01	0\\
423.01	0\\
424.01	0\\
425.01	0\\
426.01	0\\
427.01	0\\
428.01	0\\
429.01	0\\
430.01	0\\
431.01	0\\
432.01	0\\
433.01	0\\
434.01	0\\
435.01	0\\
436.01	0\\
437.01	0\\
438.01	0\\
439.01	0\\
440.01	0\\
441.01	0\\
442.01	0\\
443.01	0\\
444.01	0\\
445.01	0\\
446.01	0\\
447.01	0\\
448.01	0\\
449.01	0\\
450.01	0\\
451.01	0\\
452.01	0\\
453.01	0\\
454.01	0\\
455.01	0\\
456.01	0\\
457.01	0\\
458.01	0\\
459.01	0\\
460.01	0\\
461.01	0\\
462.01	0\\
463.01	0\\
464.01	0\\
465.01	0\\
466.01	0\\
467.01	0\\
468.01	0\\
469.01	0\\
470.01	0\\
471.01	0\\
472.01	0\\
473.01	0\\
474.01	0\\
475.01	0\\
476.01	0\\
477.01	0\\
478.01	0\\
479.01	0\\
480.01	0\\
481.01	0\\
482.01	0\\
483.01	0\\
484.01	0\\
485.01	0\\
486.01	0\\
487.01	0\\
488.01	0\\
489.01	0\\
490.01	0\\
491.01	0\\
492.01	0\\
493.01	0\\
494.01	0\\
495.01	0\\
496.01	0\\
497.01	0\\
498.01	0\\
499.01	0\\
500.01	0\\
501.01	0\\
502.01	0\\
503.01	0\\
504.01	0\\
505.01	0\\
506.01	0\\
507.01	0\\
508.01	0\\
509.01	0\\
510.01	0\\
511.01	0\\
512.01	0\\
513.01	0\\
514.01	0\\
515.01	0\\
516.01	0\\
517.01	0\\
518.01	0\\
519.01	0\\
520.01	0\\
521.01	0\\
522.01	0\\
523.01	0\\
524.01	0\\
525.01	0\\
526.01	0\\
527.01	0\\
528.01	0\\
529.01	0\\
530.01	0\\
531.01	0\\
532.01	0\\
533.01	0\\
534.01	0\\
535.01	0\\
536.01	0\\
537.01	0\\
538.01	0\\
539.01	0\\
540.01	0\\
541.01	0\\
542.01	0\\
543.01	0\\
544.01	0\\
545.01	0\\
546.01	0\\
547.01	0\\
548.01	0\\
549.01	0\\
550.01	0\\
551.01	0\\
552.01	0\\
553.01	0\\
554.01	0\\
555.01	0\\
556.01	0\\
557.01	0\\
558.01	0\\
559.01	0\\
560.01	0\\
561.01	0\\
562.01	0\\
563.01	0\\
564.01	0\\
565.01	0\\
566.01	0\\
567.01	0\\
568.01	0\\
569.01	0\\
570.01	0\\
571.01	0\\
572.01	0\\
573.01	0\\
574.01	0\\
575.01	0\\
576.01	0\\
577.01	0\\
578.01	0\\
579.01	0\\
580.01	0\\
581.01	0\\
582.01	0\\
583.01	0\\
584.01	0\\
585.01	0\\
586.01	0\\
587.01	0\\
588.01	0\\
589.01	0\\
590.01	0\\
591.01	0\\
592.01	0\\
593.01	0\\
594.01	0\\
595.01	0\\
596.01	0\\
597.01	0.000469956016972223\\
598.01	0.00143832068184645\\
599.01	0.00385661257211617\\
599.02	0.00389414818835224\\
599.03	0.00393204146009079\\
599.04	0.00397029584140053\\
599.05	0.0040089148196884\\
599.06	0.00404790191602104\\
599.07	0.00408726068544941\\
599.08	0.00412699471733656\\
599.09	0.00416710763568846\\
599.1	0.00420760309948814\\
599.11	0.00424848480303295\\
599.12	0.00428975647627514\\
599.13	0.00433142188516566\\
599.14	0.00437348483200135\\
599.15	0.00441594915577536\\
599.16	0.00445881873253109\\
599.17	0.00450209747571945\\
599.18	0.00454578933655949\\
599.19	0.00458989830440273\\
599.2	0.00463442840710074\\
599.21	0.00467938371137649\\
599.22	0.00472476832319909\\
599.23	0.00477058638816223\\
599.24	0.0048168420918663\\
599.25	0.00486353966030412\\
599.26	0.00491068334955366\\
599.27	0.00495827744958435\\
599.28	0.00500632629158769\\
599.29	0.00505483424837289\\
599.3	0.00510380573476642\\
599.31	0.00515324520801536\\
599.32	0.00520315716819456\\
599.33	0.00525354615861774\\
599.34	0.00530441676625258\\
599.35	0.00535577362213972\\
599.36	0.00540762140181579\\
599.37	0.00545996482574053\\
599.38	0.00551280865972798\\
599.39	0.00556615771538182\\
599.4	0.00562001685053486\\
599.41	0.00567439096969275\\
599.42	0.00572928502448199\\
599.43	0.00578470401410216\\
599.44	0.00584065298578252\\
599.45	0.00589713703524304\\
599.46	0.00595416130715974\\
599.47	0.0060117309956346\\
599.48	0.0060698513446699\\
599.49	0.00612852764864714\\
599.5	0.00618776525281056\\
599.51	0.00624756955375527\\
599.52	0.00630794599992007\\
599.53	0.00636890009208501\\
599.54	0.00643043738387371\\
599.55	0.00649256348226045\\
599.56	0.00655528404808222\\
599.57	0.00661860479655559\\
599.58	0.00668253149779856\\
599.59	0.00674706997735745\\
599.6	0.00681222611673878\\
599.61	0.0068780058539463\\
599.62	0.00694441518402313\\
599.63	0.00701146015959912\\
599.64	0.00707914689144345\\
599.65	0.00714748154902253\\
599.66	0.00721647036106323\\
599.67	0.00728611961612158\\
599.68	0.00735643566315675\\
599.69	0.00742742491211078\\
599.7	0.00749909383449363\\
599.71	0.007571448963974\\
599.72	0.00764449689697571\\
599.73	0.00771824429327989\\
599.74	0.00779269787663284\\
599.75	0.00786786443535982\\
599.76	0.0079437508229846\\
599.77	0.0080203639588551\\
599.78	0.00809771082877486\\
599.79	0.00817579848564071\\
599.8	0.00825463405008648\\
599.81	0.00833422471113285\\
599.82	0.00841457772684349\\
599.83	0.00849570042498747\\
599.84	0.00857760020370797\\
599.85	0.00866028453219752\\
599.86	0.00874376095137953\\
599.87	0.00882803707459654\\
599.88	0.0089131205883049\\
599.89	0.00899901925277625\\
599.9	0.00908574090280562\\
599.91	0.00917329344842635\\
599.92	0.0092616848756319\\
599.93	0.00935092324710449\\
599.94	0.00944101670295078\\
599.95	0.00953197346144465\\
599.96	0.00962380181977693\\
599.97	0.00971651015481255\\
599.98	0.0098101069238547\\
599.99	0.00990460066541651\\
600	0.01\\
};
\addplot [color=red!40!mycolor19,solid,forget plot]
  table[row sep=crcr]{%
0.01	0\\
1.01	0\\
2.01	0\\
3.01	0\\
4.01	0\\
5.01	0\\
6.01	0\\
7.01	0\\
8.01	0\\
9.01	0\\
10.01	0\\
11.01	0\\
12.01	0\\
13.01	0\\
14.01	0\\
15.01	0\\
16.01	0\\
17.01	0\\
18.01	0\\
19.01	0\\
20.01	0\\
21.01	0\\
22.01	0\\
23.01	0\\
24.01	0\\
25.01	0\\
26.01	0\\
27.01	0\\
28.01	0\\
29.01	0\\
30.01	0\\
31.01	0\\
32.01	0\\
33.01	0\\
34.01	0\\
35.01	0\\
36.01	0\\
37.01	0\\
38.01	0\\
39.01	0\\
40.01	0\\
41.01	0\\
42.01	0\\
43.01	0\\
44.01	0\\
45.01	0\\
46.01	0\\
47.01	0\\
48.01	0\\
49.01	0\\
50.01	0\\
51.01	0\\
52.01	0\\
53.01	0\\
54.01	0\\
55.01	0\\
56.01	0\\
57.01	0\\
58.01	0\\
59.01	0\\
60.01	0\\
61.01	0\\
62.01	0\\
63.01	0\\
64.01	0\\
65.01	0\\
66.01	0\\
67.01	0\\
68.01	0\\
69.01	0\\
70.01	0\\
71.01	0\\
72.01	0\\
73.01	0\\
74.01	0\\
75.01	0\\
76.01	0\\
77.01	0\\
78.01	0\\
79.01	0\\
80.01	0\\
81.01	0\\
82.01	0\\
83.01	0\\
84.01	0\\
85.01	0\\
86.01	0\\
87.01	0\\
88.01	0\\
89.01	0\\
90.01	0\\
91.01	0\\
92.01	0\\
93.01	0\\
94.01	0\\
95.01	0\\
96.01	0\\
97.01	0\\
98.01	0\\
99.01	0\\
100.01	0\\
101.01	0\\
102.01	0\\
103.01	0\\
104.01	0\\
105.01	0\\
106.01	0\\
107.01	0\\
108.01	0\\
109.01	0\\
110.01	0\\
111.01	0\\
112.01	0\\
113.01	0\\
114.01	0\\
115.01	0\\
116.01	0\\
117.01	0\\
118.01	0\\
119.01	0\\
120.01	0\\
121.01	0\\
122.01	0\\
123.01	0\\
124.01	0\\
125.01	0\\
126.01	0\\
127.01	0\\
128.01	0\\
129.01	0\\
130.01	0\\
131.01	0\\
132.01	0\\
133.01	0\\
134.01	0\\
135.01	0\\
136.01	0\\
137.01	0\\
138.01	0\\
139.01	0\\
140.01	0\\
141.01	0\\
142.01	0\\
143.01	0\\
144.01	0\\
145.01	0\\
146.01	0\\
147.01	0\\
148.01	0\\
149.01	0\\
150.01	0\\
151.01	0\\
152.01	0\\
153.01	0\\
154.01	0\\
155.01	0\\
156.01	0\\
157.01	0\\
158.01	0\\
159.01	0\\
160.01	0\\
161.01	0\\
162.01	0\\
163.01	0\\
164.01	0\\
165.01	0\\
166.01	0\\
167.01	0\\
168.01	0\\
169.01	0\\
170.01	0\\
171.01	0\\
172.01	0\\
173.01	0\\
174.01	0\\
175.01	0\\
176.01	0\\
177.01	0\\
178.01	0\\
179.01	0\\
180.01	0\\
181.01	0\\
182.01	0\\
183.01	0\\
184.01	0\\
185.01	0\\
186.01	0\\
187.01	0\\
188.01	0\\
189.01	0\\
190.01	0\\
191.01	0\\
192.01	0\\
193.01	0\\
194.01	0\\
195.01	0\\
196.01	0\\
197.01	0\\
198.01	0\\
199.01	0\\
200.01	0\\
201.01	0\\
202.01	0\\
203.01	0\\
204.01	0\\
205.01	0\\
206.01	0\\
207.01	0\\
208.01	0\\
209.01	0\\
210.01	0\\
211.01	0\\
212.01	0\\
213.01	0\\
214.01	0\\
215.01	0\\
216.01	0\\
217.01	0\\
218.01	0\\
219.01	0\\
220.01	0\\
221.01	0\\
222.01	0\\
223.01	0\\
224.01	0\\
225.01	0\\
226.01	0\\
227.01	0\\
228.01	0\\
229.01	0\\
230.01	0\\
231.01	0\\
232.01	0\\
233.01	0\\
234.01	0\\
235.01	0\\
236.01	0\\
237.01	0\\
238.01	0\\
239.01	0\\
240.01	0\\
241.01	0\\
242.01	0\\
243.01	0\\
244.01	0\\
245.01	0\\
246.01	0\\
247.01	0\\
248.01	0\\
249.01	0\\
250.01	0\\
251.01	0\\
252.01	0\\
253.01	0\\
254.01	0\\
255.01	0\\
256.01	0\\
257.01	0\\
258.01	0\\
259.01	0\\
260.01	0\\
261.01	0\\
262.01	0\\
263.01	0\\
264.01	0\\
265.01	0\\
266.01	0\\
267.01	0\\
268.01	0\\
269.01	0\\
270.01	0\\
271.01	0\\
272.01	0\\
273.01	0\\
274.01	0\\
275.01	0\\
276.01	0\\
277.01	0\\
278.01	0\\
279.01	0\\
280.01	0\\
281.01	0\\
282.01	0\\
283.01	0\\
284.01	0\\
285.01	0\\
286.01	0\\
287.01	0\\
288.01	0\\
289.01	0\\
290.01	0\\
291.01	0\\
292.01	0\\
293.01	0\\
294.01	0\\
295.01	0\\
296.01	0\\
297.01	0\\
298.01	0\\
299.01	0\\
300.01	0\\
301.01	0\\
302.01	0\\
303.01	0\\
304.01	0\\
305.01	0\\
306.01	0\\
307.01	0\\
308.01	0\\
309.01	0\\
310.01	0\\
311.01	0\\
312.01	0\\
313.01	0\\
314.01	0\\
315.01	0\\
316.01	0\\
317.01	0\\
318.01	0\\
319.01	0\\
320.01	0\\
321.01	0\\
322.01	0\\
323.01	0\\
324.01	0\\
325.01	0\\
326.01	0\\
327.01	0\\
328.01	0\\
329.01	0\\
330.01	0\\
331.01	0\\
332.01	0\\
333.01	0\\
334.01	0\\
335.01	0\\
336.01	0\\
337.01	0\\
338.01	0\\
339.01	0\\
340.01	0\\
341.01	0\\
342.01	0\\
343.01	0\\
344.01	0\\
345.01	0\\
346.01	0\\
347.01	0\\
348.01	0\\
349.01	0\\
350.01	0\\
351.01	0\\
352.01	0\\
353.01	0\\
354.01	0\\
355.01	0\\
356.01	0\\
357.01	0\\
358.01	0\\
359.01	0\\
360.01	0\\
361.01	0\\
362.01	0\\
363.01	0\\
364.01	0\\
365.01	0\\
366.01	0\\
367.01	0\\
368.01	0\\
369.01	0\\
370.01	0\\
371.01	0\\
372.01	0\\
373.01	0\\
374.01	0\\
375.01	0\\
376.01	0\\
377.01	0\\
378.01	0\\
379.01	0\\
380.01	0\\
381.01	0\\
382.01	0\\
383.01	0\\
384.01	0\\
385.01	0\\
386.01	0\\
387.01	0\\
388.01	0\\
389.01	0\\
390.01	0\\
391.01	0\\
392.01	0\\
393.01	0\\
394.01	0\\
395.01	0\\
396.01	0\\
397.01	0\\
398.01	0\\
399.01	0\\
400.01	0\\
401.01	0\\
402.01	0\\
403.01	0\\
404.01	0\\
405.01	0\\
406.01	0\\
407.01	0\\
408.01	0\\
409.01	0\\
410.01	0\\
411.01	0\\
412.01	0\\
413.01	0\\
414.01	0\\
415.01	0\\
416.01	0\\
417.01	0\\
418.01	0\\
419.01	0\\
420.01	0\\
421.01	0\\
422.01	0\\
423.01	0\\
424.01	0\\
425.01	0\\
426.01	0\\
427.01	0\\
428.01	0\\
429.01	0\\
430.01	0\\
431.01	0\\
432.01	0\\
433.01	0\\
434.01	0\\
435.01	0\\
436.01	0\\
437.01	0\\
438.01	0\\
439.01	0\\
440.01	0\\
441.01	0\\
442.01	0\\
443.01	0\\
444.01	0\\
445.01	0\\
446.01	0\\
447.01	0\\
448.01	0\\
449.01	0\\
450.01	0\\
451.01	0\\
452.01	0\\
453.01	0\\
454.01	0\\
455.01	0\\
456.01	0\\
457.01	0\\
458.01	0\\
459.01	0\\
460.01	0\\
461.01	0\\
462.01	0\\
463.01	0\\
464.01	0\\
465.01	0\\
466.01	0\\
467.01	0\\
468.01	0\\
469.01	0\\
470.01	0\\
471.01	0\\
472.01	0\\
473.01	0\\
474.01	0\\
475.01	0\\
476.01	0\\
477.01	0\\
478.01	0\\
479.01	0\\
480.01	0\\
481.01	0\\
482.01	0\\
483.01	0\\
484.01	0\\
485.01	0\\
486.01	0\\
487.01	0\\
488.01	0\\
489.01	0\\
490.01	0\\
491.01	0\\
492.01	0\\
493.01	0\\
494.01	0\\
495.01	0\\
496.01	0\\
497.01	0\\
498.01	0\\
499.01	0\\
500.01	0\\
501.01	0\\
502.01	0\\
503.01	0\\
504.01	0\\
505.01	0\\
506.01	0\\
507.01	0\\
508.01	0\\
509.01	0\\
510.01	0\\
511.01	0\\
512.01	0\\
513.01	0\\
514.01	0\\
515.01	0\\
516.01	0\\
517.01	0\\
518.01	0\\
519.01	0\\
520.01	0\\
521.01	0\\
522.01	0\\
523.01	0\\
524.01	0\\
525.01	0\\
526.01	0\\
527.01	0\\
528.01	0\\
529.01	0\\
530.01	0\\
531.01	0\\
532.01	0\\
533.01	0\\
534.01	0\\
535.01	0\\
536.01	0\\
537.01	0\\
538.01	0\\
539.01	0\\
540.01	0\\
541.01	0\\
542.01	0\\
543.01	0\\
544.01	0\\
545.01	0\\
546.01	0\\
547.01	0\\
548.01	0\\
549.01	0\\
550.01	0\\
551.01	0\\
552.01	0\\
553.01	0\\
554.01	0\\
555.01	0\\
556.01	0\\
557.01	0\\
558.01	0\\
559.01	0\\
560.01	0\\
561.01	0\\
562.01	0\\
563.01	0\\
564.01	0\\
565.01	0\\
566.01	0\\
567.01	0\\
568.01	0\\
569.01	0\\
570.01	0\\
571.01	0\\
572.01	0\\
573.01	0\\
574.01	0\\
575.01	0\\
576.01	0\\
577.01	0\\
578.01	0\\
579.01	0\\
580.01	0\\
581.01	0\\
582.01	0\\
583.01	0\\
584.01	0\\
585.01	0\\
586.01	0\\
587.01	0\\
588.01	0\\
589.01	0\\
590.01	0\\
591.01	0\\
592.01	0\\
593.01	0\\
594.01	0\\
595.01	0\\
596.01	0\\
597.01	0.000480262484284122\\
598.01	0.00143832068185175\\
599.01	0.00385661257211624\\
599.02	0.00389414818835231\\
599.03	0.00393204146009084\\
599.04	0.00397029584140058\\
599.05	0.00400891481968844\\
599.06	0.00404790191602107\\
599.07	0.00408726068544944\\
599.08	0.00412699471733659\\
599.09	0.00416710763568848\\
599.1	0.00420760309948815\\
599.11	0.00424848480303296\\
599.12	0.00428975647627514\\
599.13	0.00433142188516566\\
599.14	0.00437348483200135\\
599.15	0.00441594915577536\\
599.16	0.0044588187325311\\
599.17	0.00450209747571944\\
599.18	0.00454578933655948\\
599.19	0.00458989830440271\\
599.2	0.00463442840710074\\
599.21	0.00467938371137649\\
599.22	0.00472476832319909\\
599.23	0.00477058638816223\\
599.24	0.0048168420918663\\
599.25	0.00486353966030412\\
599.26	0.00491068334955367\\
599.27	0.00495827744958437\\
599.28	0.0050063262915877\\
599.29	0.00505483424837289\\
599.3	0.00510380573476642\\
599.31	0.00515324520801536\\
599.32	0.00520315716819456\\
599.33	0.00525354615861774\\
599.34	0.00530441676625259\\
599.35	0.00535577362213972\\
599.36	0.00540762140181578\\
599.37	0.00545996482574052\\
599.38	0.00551280865972797\\
599.39	0.0055661577153818\\
599.4	0.00562001685053484\\
599.41	0.00567439096969274\\
599.42	0.00572928502448198\\
599.43	0.00578470401410214\\
599.44	0.00584065298578251\\
599.45	0.00589713703524303\\
599.46	0.00595416130715973\\
599.47	0.0060117309956346\\
599.48	0.0060698513446699\\
599.49	0.00612852764864714\\
599.5	0.00618776525281056\\
599.51	0.00624756955375527\\
599.52	0.00630794599992007\\
599.53	0.00636890009208502\\
599.54	0.00643043738387371\\
599.55	0.00649256348226045\\
599.56	0.00655528404808223\\
599.57	0.0066186047965556\\
599.58	0.00668253149779857\\
599.59	0.00674706997735746\\
599.6	0.0068122261167388\\
599.61	0.00687800585394632\\
599.62	0.00694441518402314\\
599.63	0.00701146015959913\\
599.64	0.00707914689144346\\
599.65	0.00714748154902253\\
599.66	0.00721647036106323\\
599.67	0.00728611961612158\\
599.68	0.00735643566315675\\
599.69	0.00742742491211078\\
599.7	0.00749909383449363\\
599.71	0.007571448963974\\
599.72	0.00764449689697571\\
599.73	0.00771824429327989\\
599.74	0.00779269787663285\\
599.75	0.00786786443535982\\
599.76	0.00794375082298461\\
599.77	0.0080203639588551\\
599.78	0.00809771082877486\\
599.79	0.00817579848564071\\
599.8	0.00825463405008648\\
599.81	0.00833422471113285\\
599.82	0.00841457772684349\\
599.83	0.00849570042498747\\
599.84	0.00857760020370798\\
599.85	0.00866028453219752\\
599.86	0.00874376095137954\\
599.87	0.00882803707459654\\
599.88	0.0089131205883049\\
599.89	0.00899901925277625\\
599.9	0.00908574090280562\\
599.91	0.00917329344842636\\
599.92	0.00926168487563191\\
599.93	0.00935092324710449\\
599.94	0.00944101670295079\\
599.95	0.00953197346144465\\
599.96	0.00962380181977694\\
599.97	0.00971651015481255\\
599.98	0.0098101069238547\\
599.99	0.00990460066541651\\
600	0.01\\
};
\addplot [color=red!75!mycolor17,solid,forget plot]
  table[row sep=crcr]{%
0.01	0\\
1.01	0\\
2.01	0\\
3.01	0\\
4.01	0\\
5.01	0\\
6.01	0\\
7.01	0\\
8.01	0\\
9.01	0\\
10.01	0\\
11.01	0\\
12.01	0\\
13.01	0\\
14.01	0\\
15.01	0\\
16.01	0\\
17.01	0\\
18.01	0\\
19.01	0\\
20.01	0\\
21.01	0\\
22.01	0\\
23.01	0\\
24.01	0\\
25.01	0\\
26.01	0\\
27.01	0\\
28.01	0\\
29.01	0\\
30.01	0\\
31.01	0\\
32.01	0\\
33.01	0\\
34.01	0\\
35.01	0\\
36.01	0\\
37.01	0\\
38.01	0\\
39.01	0\\
40.01	0\\
41.01	0\\
42.01	0\\
43.01	0\\
44.01	0\\
45.01	0\\
46.01	0\\
47.01	0\\
48.01	0\\
49.01	0\\
50.01	0\\
51.01	0\\
52.01	0\\
53.01	0\\
54.01	0\\
55.01	0\\
56.01	0\\
57.01	0\\
58.01	0\\
59.01	0\\
60.01	0\\
61.01	0\\
62.01	0\\
63.01	0\\
64.01	0\\
65.01	0\\
66.01	0\\
67.01	0\\
68.01	0\\
69.01	0\\
70.01	0\\
71.01	0\\
72.01	0\\
73.01	0\\
74.01	0\\
75.01	0\\
76.01	0\\
77.01	0\\
78.01	0\\
79.01	0\\
80.01	0\\
81.01	0\\
82.01	0\\
83.01	0\\
84.01	0\\
85.01	0\\
86.01	0\\
87.01	0\\
88.01	0\\
89.01	0\\
90.01	0\\
91.01	0\\
92.01	0\\
93.01	0\\
94.01	0\\
95.01	0\\
96.01	0\\
97.01	0\\
98.01	0\\
99.01	0\\
100.01	0\\
101.01	0\\
102.01	0\\
103.01	0\\
104.01	0\\
105.01	0\\
106.01	0\\
107.01	0\\
108.01	0\\
109.01	0\\
110.01	0\\
111.01	0\\
112.01	0\\
113.01	0\\
114.01	0\\
115.01	0\\
116.01	0\\
117.01	0\\
118.01	0\\
119.01	0\\
120.01	0\\
121.01	0\\
122.01	0\\
123.01	0\\
124.01	0\\
125.01	0\\
126.01	0\\
127.01	0\\
128.01	0\\
129.01	0\\
130.01	0\\
131.01	0\\
132.01	0\\
133.01	0\\
134.01	0\\
135.01	0\\
136.01	0\\
137.01	0\\
138.01	0\\
139.01	0\\
140.01	0\\
141.01	0\\
142.01	0\\
143.01	0\\
144.01	0\\
145.01	0\\
146.01	0\\
147.01	0\\
148.01	0\\
149.01	0\\
150.01	0\\
151.01	0\\
152.01	0\\
153.01	0\\
154.01	0\\
155.01	0\\
156.01	0\\
157.01	0\\
158.01	0\\
159.01	0\\
160.01	0\\
161.01	0\\
162.01	0\\
163.01	0\\
164.01	0\\
165.01	0\\
166.01	0\\
167.01	0\\
168.01	0\\
169.01	0\\
170.01	0\\
171.01	0\\
172.01	0\\
173.01	0\\
174.01	0\\
175.01	0\\
176.01	0\\
177.01	0\\
178.01	0\\
179.01	0\\
180.01	0\\
181.01	0\\
182.01	0\\
183.01	0\\
184.01	0\\
185.01	0\\
186.01	0\\
187.01	0\\
188.01	0\\
189.01	0\\
190.01	0\\
191.01	0\\
192.01	0\\
193.01	0\\
194.01	0\\
195.01	0\\
196.01	0\\
197.01	0\\
198.01	0\\
199.01	0\\
200.01	0\\
201.01	0\\
202.01	0\\
203.01	0\\
204.01	0\\
205.01	0\\
206.01	0\\
207.01	0\\
208.01	0\\
209.01	0\\
210.01	0\\
211.01	0\\
212.01	0\\
213.01	0\\
214.01	0\\
215.01	0\\
216.01	0\\
217.01	0\\
218.01	0\\
219.01	0\\
220.01	0\\
221.01	0\\
222.01	0\\
223.01	0\\
224.01	0\\
225.01	0\\
226.01	0\\
227.01	0\\
228.01	0\\
229.01	0\\
230.01	0\\
231.01	0\\
232.01	0\\
233.01	0\\
234.01	0\\
235.01	0\\
236.01	0\\
237.01	0\\
238.01	0\\
239.01	0\\
240.01	0\\
241.01	0\\
242.01	0\\
243.01	0\\
244.01	0\\
245.01	0\\
246.01	0\\
247.01	0\\
248.01	0\\
249.01	0\\
250.01	0\\
251.01	0\\
252.01	0\\
253.01	0\\
254.01	0\\
255.01	0\\
256.01	0\\
257.01	0\\
258.01	0\\
259.01	0\\
260.01	0\\
261.01	0\\
262.01	0\\
263.01	0\\
264.01	0\\
265.01	0\\
266.01	0\\
267.01	0\\
268.01	0\\
269.01	0\\
270.01	0\\
271.01	0\\
272.01	0\\
273.01	0\\
274.01	0\\
275.01	0\\
276.01	0\\
277.01	0\\
278.01	0\\
279.01	0\\
280.01	0\\
281.01	0\\
282.01	0\\
283.01	0\\
284.01	0\\
285.01	0\\
286.01	0\\
287.01	0\\
288.01	0\\
289.01	0\\
290.01	0\\
291.01	0\\
292.01	0\\
293.01	0\\
294.01	0\\
295.01	0\\
296.01	0\\
297.01	0\\
298.01	0\\
299.01	0\\
300.01	0\\
301.01	0\\
302.01	0\\
303.01	0\\
304.01	0\\
305.01	0\\
306.01	0\\
307.01	0\\
308.01	0\\
309.01	0\\
310.01	0\\
311.01	0\\
312.01	0\\
313.01	0\\
314.01	0\\
315.01	0\\
316.01	0\\
317.01	0\\
318.01	0\\
319.01	0\\
320.01	0\\
321.01	0\\
322.01	0\\
323.01	0\\
324.01	0\\
325.01	0\\
326.01	0\\
327.01	0\\
328.01	0\\
329.01	0\\
330.01	0\\
331.01	0\\
332.01	0\\
333.01	0\\
334.01	0\\
335.01	0\\
336.01	0\\
337.01	0\\
338.01	0\\
339.01	0\\
340.01	0\\
341.01	0\\
342.01	0\\
343.01	0\\
344.01	0\\
345.01	0\\
346.01	0\\
347.01	0\\
348.01	0\\
349.01	0\\
350.01	0\\
351.01	0\\
352.01	0\\
353.01	0\\
354.01	0\\
355.01	0\\
356.01	0\\
357.01	0\\
358.01	0\\
359.01	0\\
360.01	0\\
361.01	0\\
362.01	0\\
363.01	0\\
364.01	0\\
365.01	0\\
366.01	0\\
367.01	0\\
368.01	0\\
369.01	0\\
370.01	0\\
371.01	0\\
372.01	0\\
373.01	0\\
374.01	0\\
375.01	0\\
376.01	0\\
377.01	0\\
378.01	0\\
379.01	0\\
380.01	0\\
381.01	0\\
382.01	0\\
383.01	0\\
384.01	0\\
385.01	0\\
386.01	0\\
387.01	0\\
388.01	0\\
389.01	0\\
390.01	0\\
391.01	0\\
392.01	0\\
393.01	0\\
394.01	0\\
395.01	0\\
396.01	0\\
397.01	0\\
398.01	0\\
399.01	0\\
400.01	0\\
401.01	0\\
402.01	0\\
403.01	0\\
404.01	0\\
405.01	0\\
406.01	0\\
407.01	0\\
408.01	0\\
409.01	0\\
410.01	0\\
411.01	0\\
412.01	0\\
413.01	0\\
414.01	0\\
415.01	0\\
416.01	0\\
417.01	0\\
418.01	0\\
419.01	0\\
420.01	0\\
421.01	0\\
422.01	0\\
423.01	0\\
424.01	0\\
425.01	0\\
426.01	0\\
427.01	0\\
428.01	0\\
429.01	0\\
430.01	0\\
431.01	0\\
432.01	0\\
433.01	0\\
434.01	0\\
435.01	0\\
436.01	0\\
437.01	0\\
438.01	0\\
439.01	0\\
440.01	0\\
441.01	0\\
442.01	0\\
443.01	0\\
444.01	0\\
445.01	0\\
446.01	0\\
447.01	0\\
448.01	0\\
449.01	0\\
450.01	0\\
451.01	0\\
452.01	0\\
453.01	0\\
454.01	0\\
455.01	0\\
456.01	0\\
457.01	0\\
458.01	0\\
459.01	0\\
460.01	0\\
461.01	0\\
462.01	0\\
463.01	0\\
464.01	0\\
465.01	0\\
466.01	0\\
467.01	0\\
468.01	0\\
469.01	0\\
470.01	0\\
471.01	0\\
472.01	0\\
473.01	0\\
474.01	0\\
475.01	0\\
476.01	0\\
477.01	0\\
478.01	0\\
479.01	0\\
480.01	0\\
481.01	0\\
482.01	0\\
483.01	0\\
484.01	0\\
485.01	0\\
486.01	0\\
487.01	0\\
488.01	0\\
489.01	0\\
490.01	0\\
491.01	0\\
492.01	0\\
493.01	0\\
494.01	0\\
495.01	0\\
496.01	0\\
497.01	0\\
498.01	0\\
499.01	0\\
500.01	0\\
501.01	0\\
502.01	0\\
503.01	0\\
504.01	0\\
505.01	0\\
506.01	0\\
507.01	0\\
508.01	0\\
509.01	0\\
510.01	0\\
511.01	0\\
512.01	0\\
513.01	0\\
514.01	0\\
515.01	0\\
516.01	0\\
517.01	0\\
518.01	0\\
519.01	0\\
520.01	0\\
521.01	0\\
522.01	0\\
523.01	0\\
524.01	0\\
525.01	0\\
526.01	0\\
527.01	0\\
528.01	0\\
529.01	0\\
530.01	0\\
531.01	0\\
532.01	0\\
533.01	0\\
534.01	0\\
535.01	0\\
536.01	0\\
537.01	0\\
538.01	0\\
539.01	0\\
540.01	0\\
541.01	0\\
542.01	0\\
543.01	0\\
544.01	0\\
545.01	0\\
546.01	0\\
547.01	0\\
548.01	0\\
549.01	0\\
550.01	0\\
551.01	0\\
552.01	0\\
553.01	0\\
554.01	0\\
555.01	0\\
556.01	0\\
557.01	0\\
558.01	0\\
559.01	0\\
560.01	0\\
561.01	0\\
562.01	0\\
563.01	0\\
564.01	0\\
565.01	0\\
566.01	0\\
567.01	0\\
568.01	0\\
569.01	0\\
570.01	0\\
571.01	0\\
572.01	0\\
573.01	0\\
574.01	0\\
575.01	0\\
576.01	0\\
577.01	0\\
578.01	0\\
579.01	0\\
580.01	0\\
581.01	0\\
582.01	0\\
583.01	0\\
584.01	0\\
585.01	0\\
586.01	0\\
587.01	0\\
588.01	0\\
589.01	0\\
590.01	0\\
591.01	0\\
592.01	0\\
593.01	0\\
594.01	0\\
595.01	0\\
596.01	0\\
597.01	0.000480601427308949\\
598.01	0.0014383206818518\\
599.01	0.00385661257211624\\
599.02	0.00389414818835231\\
599.03	0.00393204146009085\\
599.04	0.00397029584140059\\
599.05	0.00400891481968844\\
599.06	0.00404790191602107\\
599.07	0.00408726068544946\\
599.08	0.00412699471733659\\
599.09	0.00416710763568849\\
599.1	0.00420760309948817\\
599.11	0.00424848480303298\\
599.12	0.00428975647627516\\
599.13	0.00433142188516568\\
599.14	0.00437348483200135\\
599.15	0.00441594915577537\\
599.16	0.00445881873253111\\
599.17	0.00450209747571945\\
599.18	0.00454578933655949\\
599.19	0.00458989830440273\\
599.2	0.00463442840710075\\
599.21	0.00467938371137651\\
599.22	0.00472476832319909\\
599.23	0.00477058638816223\\
599.24	0.00481684209186631\\
599.25	0.00486353966030413\\
599.26	0.00491068334955367\\
599.27	0.00495827744958437\\
599.28	0.0050063262915877\\
599.29	0.00505483424837289\\
599.3	0.00510380573476643\\
599.31	0.00515324520801537\\
599.32	0.00520315716819457\\
599.33	0.00525354615861775\\
599.34	0.0053044167662526\\
599.35	0.00535577362213973\\
599.36	0.0054076214018158\\
599.37	0.00545996482574054\\
599.38	0.00551280865972798\\
599.39	0.00556615771538182\\
599.4	0.00562001685053486\\
599.41	0.00567439096969275\\
599.42	0.00572928502448199\\
599.43	0.00578470401410217\\
599.44	0.00584065298578253\\
599.45	0.00589713703524306\\
599.46	0.00595416130715976\\
599.47	0.00601173099563462\\
599.48	0.00606985134466991\\
599.49	0.00612852764864715\\
599.5	0.00618776525281057\\
599.51	0.00624756955375528\\
599.52	0.00630794599992008\\
599.53	0.00636890009208502\\
599.54	0.00643043738387372\\
599.55	0.00649256348226046\\
599.56	0.00655528404808223\\
599.57	0.0066186047965556\\
599.58	0.00668253149779857\\
599.59	0.00674706997735747\\
599.6	0.0068122261167388\\
599.61	0.00687800585394632\\
599.62	0.00694441518402314\\
599.63	0.00701146015959912\\
599.64	0.00707914689144345\\
599.65	0.00714748154902253\\
599.66	0.00721647036106324\\
599.67	0.00728611961612158\\
599.68	0.00735643566315675\\
599.69	0.00742742491211078\\
599.7	0.00749909383449363\\
599.71	0.00757144896397399\\
599.72	0.0076444968969757\\
599.73	0.00771824429327989\\
599.74	0.00779269787663285\\
599.75	0.00786786443535982\\
599.76	0.00794375082298461\\
599.77	0.0080203639588551\\
599.78	0.00809771082877485\\
599.79	0.00817579848564071\\
599.8	0.00825463405008648\\
599.81	0.00833422471113285\\
599.82	0.00841457772684349\\
599.83	0.00849570042498746\\
599.84	0.00857760020370797\\
599.85	0.00866028453219752\\
599.86	0.00874376095137953\\
599.87	0.00882803707459653\\
599.88	0.0089131205883049\\
599.89	0.00899901925277625\\
599.9	0.00908574090280562\\
599.91	0.00917329344842636\\
599.92	0.0092616848756319\\
599.93	0.00935092324710449\\
599.94	0.00944101670295079\\
599.95	0.00953197346144465\\
599.96	0.00962380181977694\\
599.97	0.00971651015481255\\
599.98	0.0098101069238547\\
599.99	0.00990460066541651\\
600	0.01\\
};
\addplot [color=red!80!mycolor19,solid,forget plot]
  table[row sep=crcr]{%
0.01	0\\
1.01	0\\
2.01	0\\
3.01	0\\
4.01	0\\
5.01	0\\
6.01	0\\
7.01	0\\
8.01	0\\
9.01	0\\
10.01	0\\
11.01	0\\
12.01	0\\
13.01	0\\
14.01	0\\
15.01	0\\
16.01	0\\
17.01	0\\
18.01	0\\
19.01	0\\
20.01	0\\
21.01	0\\
22.01	0\\
23.01	0\\
24.01	0\\
25.01	0\\
26.01	0\\
27.01	0\\
28.01	0\\
29.01	0\\
30.01	0\\
31.01	0\\
32.01	0\\
33.01	0\\
34.01	0\\
35.01	0\\
36.01	0\\
37.01	0\\
38.01	0\\
39.01	0\\
40.01	0\\
41.01	0\\
42.01	0\\
43.01	0\\
44.01	0\\
45.01	0\\
46.01	0\\
47.01	0\\
48.01	0\\
49.01	0\\
50.01	0\\
51.01	0\\
52.01	0\\
53.01	0\\
54.01	0\\
55.01	0\\
56.01	0\\
57.01	0\\
58.01	0\\
59.01	0\\
60.01	0\\
61.01	0\\
62.01	0\\
63.01	0\\
64.01	0\\
65.01	0\\
66.01	0\\
67.01	0\\
68.01	0\\
69.01	0\\
70.01	0\\
71.01	0\\
72.01	0\\
73.01	0\\
74.01	0\\
75.01	0\\
76.01	0\\
77.01	0\\
78.01	0\\
79.01	0\\
80.01	0\\
81.01	0\\
82.01	0\\
83.01	0\\
84.01	0\\
85.01	0\\
86.01	0\\
87.01	0\\
88.01	0\\
89.01	0\\
90.01	0\\
91.01	0\\
92.01	0\\
93.01	0\\
94.01	0\\
95.01	0\\
96.01	0\\
97.01	0\\
98.01	0\\
99.01	0\\
100.01	0\\
101.01	0\\
102.01	0\\
103.01	0\\
104.01	0\\
105.01	0\\
106.01	0\\
107.01	0\\
108.01	0\\
109.01	0\\
110.01	0\\
111.01	0\\
112.01	0\\
113.01	0\\
114.01	0\\
115.01	0\\
116.01	0\\
117.01	0\\
118.01	0\\
119.01	0\\
120.01	0\\
121.01	0\\
122.01	0\\
123.01	0\\
124.01	0\\
125.01	0\\
126.01	0\\
127.01	0\\
128.01	0\\
129.01	0\\
130.01	0\\
131.01	0\\
132.01	0\\
133.01	0\\
134.01	0\\
135.01	0\\
136.01	0\\
137.01	0\\
138.01	0\\
139.01	0\\
140.01	0\\
141.01	0\\
142.01	0\\
143.01	0\\
144.01	0\\
145.01	0\\
146.01	0\\
147.01	0\\
148.01	0\\
149.01	0\\
150.01	0\\
151.01	0\\
152.01	0\\
153.01	0\\
154.01	0\\
155.01	0\\
156.01	0\\
157.01	0\\
158.01	0\\
159.01	0\\
160.01	0\\
161.01	0\\
162.01	0\\
163.01	0\\
164.01	0\\
165.01	0\\
166.01	0\\
167.01	0\\
168.01	0\\
169.01	0\\
170.01	0\\
171.01	0\\
172.01	0\\
173.01	0\\
174.01	0\\
175.01	0\\
176.01	0\\
177.01	0\\
178.01	0\\
179.01	0\\
180.01	0\\
181.01	0\\
182.01	0\\
183.01	0\\
184.01	0\\
185.01	0\\
186.01	0\\
187.01	0\\
188.01	0\\
189.01	0\\
190.01	0\\
191.01	0\\
192.01	0\\
193.01	0\\
194.01	0\\
195.01	0\\
196.01	0\\
197.01	0\\
198.01	0\\
199.01	0\\
200.01	0\\
201.01	0\\
202.01	0\\
203.01	0\\
204.01	0\\
205.01	0\\
206.01	0\\
207.01	0\\
208.01	0\\
209.01	0\\
210.01	0\\
211.01	0\\
212.01	0\\
213.01	0\\
214.01	0\\
215.01	0\\
216.01	0\\
217.01	0\\
218.01	0\\
219.01	0\\
220.01	0\\
221.01	0\\
222.01	0\\
223.01	0\\
224.01	0\\
225.01	0\\
226.01	0\\
227.01	0\\
228.01	0\\
229.01	0\\
230.01	0\\
231.01	0\\
232.01	0\\
233.01	0\\
234.01	0\\
235.01	0\\
236.01	0\\
237.01	0\\
238.01	0\\
239.01	0\\
240.01	0\\
241.01	0\\
242.01	0\\
243.01	0\\
244.01	0\\
245.01	0\\
246.01	0\\
247.01	0\\
248.01	0\\
249.01	0\\
250.01	0\\
251.01	0\\
252.01	0\\
253.01	0\\
254.01	0\\
255.01	0\\
256.01	0\\
257.01	0\\
258.01	0\\
259.01	0\\
260.01	0\\
261.01	0\\
262.01	0\\
263.01	0\\
264.01	0\\
265.01	0\\
266.01	0\\
267.01	0\\
268.01	0\\
269.01	0\\
270.01	0\\
271.01	0\\
272.01	0\\
273.01	0\\
274.01	0\\
275.01	0\\
276.01	0\\
277.01	0\\
278.01	0\\
279.01	0\\
280.01	0\\
281.01	0\\
282.01	0\\
283.01	0\\
284.01	0\\
285.01	0\\
286.01	0\\
287.01	0\\
288.01	0\\
289.01	0\\
290.01	0\\
291.01	0\\
292.01	0\\
293.01	0\\
294.01	0\\
295.01	0\\
296.01	0\\
297.01	0\\
298.01	0\\
299.01	0\\
300.01	0\\
301.01	0\\
302.01	0\\
303.01	0\\
304.01	0\\
305.01	0\\
306.01	0\\
307.01	0\\
308.01	0\\
309.01	0\\
310.01	0\\
311.01	0\\
312.01	0\\
313.01	0\\
314.01	0\\
315.01	0\\
316.01	0\\
317.01	0\\
318.01	0\\
319.01	0\\
320.01	0\\
321.01	0\\
322.01	0\\
323.01	0\\
324.01	0\\
325.01	0\\
326.01	0\\
327.01	0\\
328.01	0\\
329.01	0\\
330.01	0\\
331.01	0\\
332.01	0\\
333.01	0\\
334.01	0\\
335.01	0\\
336.01	0\\
337.01	0\\
338.01	0\\
339.01	0\\
340.01	0\\
341.01	0\\
342.01	0\\
343.01	0\\
344.01	0\\
345.01	0\\
346.01	0\\
347.01	0\\
348.01	0\\
349.01	0\\
350.01	0\\
351.01	0\\
352.01	0\\
353.01	0\\
354.01	0\\
355.01	0\\
356.01	0\\
357.01	0\\
358.01	0\\
359.01	0\\
360.01	0\\
361.01	0\\
362.01	0\\
363.01	0\\
364.01	0\\
365.01	0\\
366.01	0\\
367.01	0\\
368.01	0\\
369.01	0\\
370.01	0\\
371.01	0\\
372.01	0\\
373.01	0\\
374.01	0\\
375.01	0\\
376.01	0\\
377.01	0\\
378.01	0\\
379.01	0\\
380.01	0\\
381.01	0\\
382.01	0\\
383.01	0\\
384.01	0\\
385.01	0\\
386.01	0\\
387.01	0\\
388.01	0\\
389.01	0\\
390.01	0\\
391.01	0\\
392.01	0\\
393.01	0\\
394.01	0\\
395.01	0\\
396.01	0\\
397.01	0\\
398.01	0\\
399.01	0\\
400.01	0\\
401.01	0\\
402.01	0\\
403.01	0\\
404.01	0\\
405.01	0\\
406.01	0\\
407.01	0\\
408.01	0\\
409.01	0\\
410.01	0\\
411.01	0\\
412.01	0\\
413.01	0\\
414.01	0\\
415.01	0\\
416.01	0\\
417.01	0\\
418.01	0\\
419.01	0\\
420.01	0\\
421.01	0\\
422.01	0\\
423.01	0\\
424.01	0\\
425.01	0\\
426.01	0\\
427.01	0\\
428.01	0\\
429.01	0\\
430.01	0\\
431.01	0\\
432.01	0\\
433.01	0\\
434.01	0\\
435.01	0\\
436.01	0\\
437.01	0\\
438.01	0\\
439.01	0\\
440.01	0\\
441.01	0\\
442.01	0\\
443.01	0\\
444.01	0\\
445.01	0\\
446.01	0\\
447.01	0\\
448.01	0\\
449.01	0\\
450.01	0\\
451.01	0\\
452.01	0\\
453.01	0\\
454.01	0\\
455.01	0\\
456.01	0\\
457.01	0\\
458.01	0\\
459.01	0\\
460.01	0\\
461.01	0\\
462.01	0\\
463.01	0\\
464.01	0\\
465.01	0\\
466.01	0\\
467.01	0\\
468.01	0\\
469.01	0\\
470.01	0\\
471.01	0\\
472.01	0\\
473.01	0\\
474.01	0\\
475.01	0\\
476.01	0\\
477.01	0\\
478.01	0\\
479.01	0\\
480.01	0\\
481.01	0\\
482.01	0\\
483.01	0\\
484.01	0\\
485.01	0\\
486.01	0\\
487.01	0\\
488.01	0\\
489.01	0\\
490.01	0\\
491.01	0\\
492.01	0\\
493.01	0\\
494.01	0\\
495.01	0\\
496.01	0\\
497.01	0\\
498.01	0\\
499.01	0\\
500.01	0\\
501.01	0\\
502.01	0\\
503.01	0\\
504.01	0\\
505.01	0\\
506.01	0\\
507.01	0\\
508.01	0\\
509.01	0\\
510.01	0\\
511.01	0\\
512.01	0\\
513.01	0\\
514.01	0\\
515.01	0\\
516.01	0\\
517.01	0\\
518.01	0\\
519.01	0\\
520.01	0\\
521.01	0\\
522.01	0\\
523.01	0\\
524.01	0\\
525.01	0\\
526.01	0\\
527.01	0\\
528.01	0\\
529.01	0\\
530.01	0\\
531.01	0\\
532.01	0\\
533.01	0\\
534.01	0\\
535.01	0\\
536.01	0\\
537.01	0\\
538.01	0\\
539.01	0\\
540.01	0\\
541.01	0\\
542.01	0\\
543.01	0\\
544.01	0\\
545.01	0\\
546.01	0\\
547.01	0\\
548.01	0\\
549.01	0\\
550.01	0\\
551.01	0\\
552.01	0\\
553.01	0\\
554.01	0\\
555.01	0\\
556.01	0\\
557.01	0\\
558.01	0\\
559.01	0\\
560.01	0\\
561.01	0\\
562.01	0\\
563.01	0\\
564.01	0\\
565.01	0\\
566.01	0\\
567.01	0\\
568.01	0\\
569.01	0\\
570.01	0\\
571.01	0\\
572.01	0\\
573.01	0\\
574.01	0\\
575.01	0\\
576.01	0\\
577.01	0\\
578.01	0\\
579.01	0\\
580.01	0\\
581.01	0\\
582.01	0\\
583.01	0\\
584.01	0\\
585.01	0\\
586.01	0\\
587.01	0\\
588.01	0\\
589.01	0\\
590.01	0\\
591.01	0\\
592.01	0\\
593.01	0\\
594.01	0\\
595.01	0\\
596.01	0\\
597.01	0.000480804875715751\\
598.01	0.00143832068185185\\
599.01	0.00385661257211618\\
599.02	0.00389414818835225\\
599.03	0.0039320414600908\\
599.04	0.00397029584140053\\
599.05	0.00400891481968839\\
599.06	0.00404790191602102\\
599.07	0.00408726068544939\\
599.08	0.00412699471733653\\
599.09	0.00416710763568844\\
599.1	0.00420760309948812\\
599.11	0.00424848480303291\\
599.12	0.00428975647627511\\
599.13	0.00433142188516562\\
599.14	0.00437348483200131\\
599.15	0.00441594915577533\\
599.16	0.00445881873253107\\
599.17	0.00450209747571942\\
599.18	0.00454578933655946\\
599.19	0.00458989830440269\\
599.2	0.00463442840710071\\
599.21	0.00467938371137647\\
599.22	0.00472476832319907\\
599.23	0.0047705863881622\\
599.24	0.00481684209186628\\
599.25	0.00486353966030409\\
599.26	0.00491068334955363\\
599.27	0.00495827744958433\\
599.28	0.00500632629158766\\
599.29	0.00505483424837287\\
599.3	0.0051038057347664\\
599.31	0.00515324520801534\\
599.32	0.00520315716819453\\
599.33	0.00525354615861772\\
599.34	0.00530441676625257\\
599.35	0.0053557736221397\\
599.36	0.00540762140181576\\
599.37	0.0054599648257405\\
599.38	0.00551280865972795\\
599.39	0.0055661577153818\\
599.4	0.00562001685053483\\
599.41	0.00567439096969272\\
599.42	0.00572928502448197\\
599.43	0.00578470401410213\\
599.44	0.0058406529857825\\
599.45	0.00589713703524302\\
599.46	0.00595416130715973\\
599.47	0.00601173099563459\\
599.48	0.00606985134466988\\
599.49	0.00612852764864712\\
599.5	0.00618776525281054\\
599.51	0.00624756955375525\\
599.52	0.00630794599992006\\
599.53	0.006368900092085\\
599.54	0.0064304373838737\\
599.55	0.00649256348226045\\
599.56	0.00655528404808222\\
599.57	0.00661860479655559\\
599.58	0.00668253149779856\\
599.59	0.00674706997735745\\
599.6	0.00681222611673878\\
599.61	0.0068780058539463\\
599.62	0.00694441518402313\\
599.63	0.00701146015959912\\
599.64	0.00707914689144345\\
599.65	0.00714748154902253\\
599.66	0.00721647036106324\\
599.67	0.00728611961612158\\
599.68	0.00735643566315675\\
599.69	0.00742742491211078\\
599.7	0.00749909383449363\\
599.71	0.00757144896397399\\
599.72	0.00764449689697571\\
599.73	0.00771824429327988\\
599.74	0.00779269787663284\\
599.75	0.00786786443535982\\
599.76	0.0079437508229846\\
599.77	0.00802036395885509\\
599.78	0.00809771082877485\\
599.79	0.00817579848564071\\
599.8	0.00825463405008648\\
599.81	0.00833422471113284\\
599.82	0.00841457772684349\\
599.83	0.00849570042498746\\
599.84	0.00857760020370797\\
599.85	0.00866028453219752\\
599.86	0.00874376095137954\\
599.87	0.00882803707459654\\
599.88	0.0089131205883049\\
599.89	0.00899901925277625\\
599.9	0.00908574090280562\\
599.91	0.00917329344842635\\
599.92	0.0092616848756319\\
599.93	0.00935092324710449\\
599.94	0.00944101670295078\\
599.95	0.00953197346144465\\
599.96	0.00962380181977693\\
599.97	0.00971651015481255\\
599.98	0.0098101069238547\\
599.99	0.00990460066541651\\
600	0.01\\
};
\addplot [color=red,solid,forget plot]
  table[row sep=crcr]{%
0.01	0\\
1.01	0\\
2.01	0\\
3.01	0\\
4.01	0\\
5.01	0\\
6.01	0\\
7.01	0\\
8.01	0\\
9.01	0\\
10.01	0\\
11.01	0\\
12.01	0\\
13.01	0\\
14.01	0\\
15.01	0\\
16.01	0\\
17.01	0\\
18.01	0\\
19.01	0\\
20.01	0\\
21.01	0\\
22.01	0\\
23.01	0\\
24.01	0\\
25.01	0\\
26.01	0\\
27.01	0\\
28.01	0\\
29.01	0\\
30.01	0\\
31.01	0\\
32.01	0\\
33.01	0\\
34.01	0\\
35.01	0\\
36.01	0\\
37.01	0\\
38.01	0\\
39.01	0\\
40.01	0\\
41.01	0\\
42.01	0\\
43.01	0\\
44.01	0\\
45.01	0\\
46.01	0\\
47.01	0\\
48.01	0\\
49.01	0\\
50.01	0\\
51.01	0\\
52.01	0\\
53.01	0\\
54.01	0\\
55.01	0\\
56.01	0\\
57.01	0\\
58.01	0\\
59.01	0\\
60.01	0\\
61.01	0\\
62.01	0\\
63.01	0\\
64.01	0\\
65.01	0\\
66.01	0\\
67.01	0\\
68.01	0\\
69.01	0\\
70.01	0\\
71.01	0\\
72.01	0\\
73.01	0\\
74.01	0\\
75.01	0\\
76.01	0\\
77.01	0\\
78.01	0\\
79.01	0\\
80.01	0\\
81.01	0\\
82.01	0\\
83.01	0\\
84.01	0\\
85.01	0\\
86.01	0\\
87.01	0\\
88.01	0\\
89.01	0\\
90.01	0\\
91.01	0\\
92.01	0\\
93.01	0\\
94.01	0\\
95.01	0\\
96.01	0\\
97.01	0\\
98.01	0\\
99.01	0\\
100.01	0\\
101.01	0\\
102.01	0\\
103.01	0\\
104.01	0\\
105.01	0\\
106.01	0\\
107.01	0\\
108.01	0\\
109.01	0\\
110.01	0\\
111.01	0\\
112.01	0\\
113.01	0\\
114.01	0\\
115.01	0\\
116.01	0\\
117.01	0\\
118.01	0\\
119.01	0\\
120.01	0\\
121.01	0\\
122.01	0\\
123.01	0\\
124.01	0\\
125.01	0\\
126.01	0\\
127.01	0\\
128.01	0\\
129.01	0\\
130.01	0\\
131.01	0\\
132.01	0\\
133.01	0\\
134.01	0\\
135.01	0\\
136.01	0\\
137.01	0\\
138.01	0\\
139.01	0\\
140.01	0\\
141.01	0\\
142.01	0\\
143.01	0\\
144.01	0\\
145.01	0\\
146.01	0\\
147.01	0\\
148.01	0\\
149.01	0\\
150.01	0\\
151.01	0\\
152.01	0\\
153.01	0\\
154.01	0\\
155.01	0\\
156.01	0\\
157.01	0\\
158.01	0\\
159.01	0\\
160.01	0\\
161.01	0\\
162.01	0\\
163.01	0\\
164.01	0\\
165.01	0\\
166.01	0\\
167.01	0\\
168.01	0\\
169.01	0\\
170.01	0\\
171.01	0\\
172.01	0\\
173.01	0\\
174.01	0\\
175.01	0\\
176.01	0\\
177.01	0\\
178.01	0\\
179.01	0\\
180.01	0\\
181.01	0\\
182.01	0\\
183.01	0\\
184.01	0\\
185.01	0\\
186.01	0\\
187.01	0\\
188.01	0\\
189.01	0\\
190.01	0\\
191.01	0\\
192.01	0\\
193.01	0\\
194.01	0\\
195.01	0\\
196.01	0\\
197.01	0\\
198.01	0\\
199.01	0\\
200.01	0\\
201.01	0\\
202.01	0\\
203.01	0\\
204.01	0\\
205.01	0\\
206.01	0\\
207.01	0\\
208.01	0\\
209.01	0\\
210.01	0\\
211.01	0\\
212.01	0\\
213.01	0\\
214.01	0\\
215.01	0\\
216.01	0\\
217.01	0\\
218.01	0\\
219.01	0\\
220.01	0\\
221.01	0\\
222.01	0\\
223.01	0\\
224.01	0\\
225.01	0\\
226.01	0\\
227.01	0\\
228.01	0\\
229.01	0\\
230.01	0\\
231.01	0\\
232.01	0\\
233.01	0\\
234.01	0\\
235.01	0\\
236.01	0\\
237.01	0\\
238.01	0\\
239.01	0\\
240.01	0\\
241.01	0\\
242.01	0\\
243.01	0\\
244.01	0\\
245.01	0\\
246.01	0\\
247.01	0\\
248.01	0\\
249.01	0\\
250.01	0\\
251.01	0\\
252.01	0\\
253.01	0\\
254.01	0\\
255.01	0\\
256.01	0\\
257.01	0\\
258.01	0\\
259.01	0\\
260.01	0\\
261.01	0\\
262.01	0\\
263.01	0\\
264.01	0\\
265.01	0\\
266.01	0\\
267.01	0\\
268.01	0\\
269.01	0\\
270.01	0\\
271.01	0\\
272.01	0\\
273.01	0\\
274.01	0\\
275.01	0\\
276.01	0\\
277.01	0\\
278.01	0\\
279.01	0\\
280.01	0\\
281.01	0\\
282.01	0\\
283.01	0\\
284.01	0\\
285.01	0\\
286.01	0\\
287.01	0\\
288.01	0\\
289.01	0\\
290.01	0\\
291.01	0\\
292.01	0\\
293.01	0\\
294.01	0\\
295.01	0\\
296.01	0\\
297.01	0\\
298.01	0\\
299.01	0\\
300.01	0\\
301.01	0\\
302.01	0\\
303.01	0\\
304.01	0\\
305.01	0\\
306.01	0\\
307.01	0\\
308.01	0\\
309.01	0\\
310.01	0\\
311.01	0\\
312.01	0\\
313.01	0\\
314.01	0\\
315.01	0\\
316.01	0\\
317.01	0\\
318.01	0\\
319.01	0\\
320.01	0\\
321.01	0\\
322.01	0\\
323.01	0\\
324.01	0\\
325.01	0\\
326.01	0\\
327.01	0\\
328.01	0\\
329.01	0\\
330.01	0\\
331.01	0\\
332.01	0\\
333.01	0\\
334.01	0\\
335.01	0\\
336.01	0\\
337.01	0\\
338.01	0\\
339.01	0\\
340.01	0\\
341.01	0\\
342.01	0\\
343.01	0\\
344.01	0\\
345.01	0\\
346.01	0\\
347.01	0\\
348.01	0\\
349.01	0\\
350.01	0\\
351.01	0\\
352.01	0\\
353.01	0\\
354.01	0\\
355.01	0\\
356.01	0\\
357.01	0\\
358.01	0\\
359.01	0\\
360.01	0\\
361.01	0\\
362.01	0\\
363.01	0\\
364.01	0\\
365.01	0\\
366.01	0\\
367.01	0\\
368.01	0\\
369.01	0\\
370.01	0\\
371.01	0\\
372.01	0\\
373.01	0\\
374.01	0\\
375.01	0\\
376.01	0\\
377.01	0\\
378.01	0\\
379.01	0\\
380.01	0\\
381.01	0\\
382.01	0\\
383.01	0\\
384.01	0\\
385.01	0\\
386.01	0\\
387.01	0\\
388.01	0\\
389.01	0\\
390.01	0\\
391.01	0\\
392.01	0\\
393.01	0\\
394.01	0\\
395.01	0\\
396.01	0\\
397.01	0\\
398.01	0\\
399.01	0\\
400.01	0\\
401.01	0\\
402.01	0\\
403.01	0\\
404.01	0\\
405.01	0\\
406.01	0\\
407.01	0\\
408.01	0\\
409.01	0\\
410.01	0\\
411.01	0\\
412.01	0\\
413.01	0\\
414.01	0\\
415.01	0\\
416.01	0\\
417.01	0\\
418.01	0\\
419.01	0\\
420.01	0\\
421.01	0\\
422.01	0\\
423.01	0\\
424.01	0\\
425.01	0\\
426.01	0\\
427.01	0\\
428.01	0\\
429.01	0\\
430.01	0\\
431.01	0\\
432.01	0\\
433.01	0\\
434.01	0\\
435.01	0\\
436.01	0\\
437.01	0\\
438.01	0\\
439.01	0\\
440.01	0\\
441.01	0\\
442.01	0\\
443.01	0\\
444.01	0\\
445.01	0\\
446.01	0\\
447.01	0\\
448.01	0\\
449.01	0\\
450.01	0\\
451.01	0\\
452.01	0\\
453.01	0\\
454.01	0\\
455.01	0\\
456.01	0\\
457.01	0\\
458.01	0\\
459.01	0\\
460.01	0\\
461.01	0\\
462.01	0\\
463.01	0\\
464.01	0\\
465.01	0\\
466.01	0\\
467.01	0\\
468.01	0\\
469.01	0\\
470.01	0\\
471.01	0\\
472.01	0\\
473.01	0\\
474.01	0\\
475.01	0\\
476.01	0\\
477.01	0\\
478.01	0\\
479.01	0\\
480.01	0\\
481.01	0\\
482.01	0\\
483.01	0\\
484.01	0\\
485.01	0\\
486.01	0\\
487.01	0\\
488.01	0\\
489.01	0\\
490.01	0\\
491.01	0\\
492.01	0\\
493.01	0\\
494.01	0\\
495.01	0\\
496.01	0\\
497.01	0\\
498.01	0\\
499.01	0\\
500.01	0\\
501.01	0\\
502.01	0\\
503.01	0\\
504.01	0\\
505.01	0\\
506.01	0\\
507.01	0\\
508.01	0\\
509.01	0\\
510.01	0\\
511.01	0\\
512.01	0\\
513.01	0\\
514.01	0\\
515.01	0\\
516.01	0\\
517.01	0\\
518.01	0\\
519.01	0\\
520.01	0\\
521.01	0\\
522.01	0\\
523.01	0\\
524.01	0\\
525.01	0\\
526.01	0\\
527.01	0\\
528.01	0\\
529.01	0\\
530.01	0\\
531.01	0\\
532.01	0\\
533.01	0\\
534.01	0\\
535.01	0\\
536.01	0\\
537.01	0\\
538.01	0\\
539.01	0\\
540.01	0\\
541.01	0\\
542.01	0\\
543.01	0\\
544.01	0\\
545.01	0\\
546.01	0\\
547.01	0\\
548.01	0\\
549.01	0\\
550.01	0\\
551.01	0\\
552.01	0\\
553.01	0\\
554.01	0\\
555.01	0\\
556.01	0\\
557.01	0\\
558.01	0\\
559.01	0\\
560.01	0\\
561.01	0\\
562.01	0\\
563.01	0\\
564.01	0\\
565.01	0\\
566.01	0\\
567.01	0\\
568.01	0\\
569.01	0\\
570.01	0\\
571.01	0\\
572.01	0\\
573.01	0\\
574.01	0\\
575.01	0\\
576.01	0\\
577.01	0\\
578.01	0\\
579.01	0\\
580.01	0\\
581.01	0\\
582.01	0\\
583.01	0\\
584.01	0\\
585.01	0\\
586.01	0\\
587.01	0\\
588.01	0\\
589.01	0\\
590.01	0\\
591.01	0\\
592.01	0\\
593.01	0\\
594.01	0\\
595.01	0\\
596.01	0\\
597.01	0.000480974774395043\\
598.01	0.00143832068185183\\
599.01	0.00385661257211624\\
599.02	0.00389414818835231\\
599.03	0.00393204146009085\\
599.04	0.0039702958414006\\
599.05	0.00400891481968847\\
599.06	0.00404790191602111\\
599.07	0.00408726068544948\\
599.08	0.00412699471733663\\
599.09	0.00416710763568853\\
599.1	0.0042076030994882\\
599.11	0.00424848480303301\\
599.12	0.00428975647627519\\
599.13	0.00433142188516572\\
599.14	0.0043734848320014\\
599.15	0.00441594915577541\\
599.16	0.00445881873253114\\
599.17	0.00450209747571947\\
599.18	0.00454578933655952\\
599.19	0.00458989830440275\\
599.2	0.00463442840710077\\
599.21	0.00467938371137651\\
599.22	0.0047247683231991\\
599.23	0.00477058638816223\\
599.24	0.00481684209186631\\
599.25	0.00486353966030413\\
599.26	0.00491068334955368\\
599.27	0.00495827744958438\\
599.28	0.0050063262915877\\
599.29	0.0050548342483729\\
599.3	0.00510380573476643\\
599.31	0.00515324520801537\\
599.32	0.00520315716819457\\
599.33	0.00525354615861776\\
599.34	0.00530441676625259\\
599.35	0.00535577362213973\\
599.36	0.00540762140181579\\
599.37	0.00545996482574052\\
599.38	0.00551280865972798\\
599.39	0.00556615771538182\\
599.4	0.00562001685053486\\
599.41	0.00567439096969275\\
599.42	0.00572928502448199\\
599.43	0.00578470401410216\\
599.44	0.00584065298578253\\
599.45	0.00589713703524305\\
599.46	0.00595416130715976\\
599.47	0.00601173099563462\\
599.48	0.00606985134466992\\
599.49	0.00612852764864716\\
599.5	0.00618776525281058\\
599.51	0.00624756955375529\\
599.52	0.00630794599992009\\
599.53	0.00636890009208504\\
599.54	0.00643043738387374\\
599.55	0.00649256348226048\\
599.56	0.00655528404808225\\
599.57	0.00661860479655562\\
599.58	0.00668253149779859\\
599.59	0.00674706997735749\\
599.6	0.00681222611673882\\
599.61	0.00687800585394634\\
599.62	0.00694441518402316\\
599.63	0.00701146015959914\\
599.64	0.00707914689144347\\
599.65	0.00714748154902254\\
599.66	0.00721647036106325\\
599.67	0.00728611961612159\\
599.68	0.00735643566315677\\
599.69	0.0074274249121108\\
599.7	0.00749909383449365\\
599.71	0.00757144896397401\\
599.72	0.00764449689697572\\
599.73	0.0077182442932799\\
599.74	0.00779269787663286\\
599.75	0.00786786443535983\\
599.76	0.00794375082298461\\
599.77	0.0080203639588551\\
599.78	0.00809771082877486\\
599.79	0.00817579848564072\\
599.8	0.00825463405008648\\
599.81	0.00833422471113285\\
599.82	0.0084145777268435\\
599.83	0.00849570042498747\\
599.84	0.00857760020370798\\
599.85	0.00866028453219753\\
599.86	0.00874376095137954\\
599.87	0.00882803707459654\\
599.88	0.0089131205883049\\
599.89	0.00899901925277625\\
599.9	0.00908574090280562\\
599.91	0.00917329344842636\\
599.92	0.00926168487563191\\
599.93	0.00935092324710449\\
599.94	0.00944101670295078\\
599.95	0.00953197346144465\\
599.96	0.00962380181977693\\
599.97	0.00971651015481255\\
599.98	0.0098101069238547\\
599.99	0.00990460066541651\\
600	0.01\\
};
\addplot [color=mycolor20,solid,forget plot]
  table[row sep=crcr]{%
0.01	0\\
1.01	0\\
2.01	0\\
3.01	0\\
4.01	0\\
5.01	0\\
6.01	0\\
7.01	0\\
8.01	0\\
9.01	0\\
10.01	0\\
11.01	0\\
12.01	0\\
13.01	0\\
14.01	0\\
15.01	0\\
16.01	0\\
17.01	0\\
18.01	0\\
19.01	0\\
20.01	0\\
21.01	0\\
22.01	0\\
23.01	0\\
24.01	0\\
25.01	0\\
26.01	0\\
27.01	0\\
28.01	0\\
29.01	0\\
30.01	0\\
31.01	0\\
32.01	0\\
33.01	0\\
34.01	0\\
35.01	0\\
36.01	0\\
37.01	0\\
38.01	0\\
39.01	0\\
40.01	0\\
41.01	0\\
42.01	0\\
43.01	0\\
44.01	0\\
45.01	0\\
46.01	0\\
47.01	0\\
48.01	0\\
49.01	0\\
50.01	0\\
51.01	0\\
52.01	0\\
53.01	0\\
54.01	0\\
55.01	0\\
56.01	0\\
57.01	0\\
58.01	0\\
59.01	0\\
60.01	0\\
61.01	0\\
62.01	0\\
63.01	0\\
64.01	0\\
65.01	0\\
66.01	0\\
67.01	0\\
68.01	0\\
69.01	0\\
70.01	0\\
71.01	0\\
72.01	0\\
73.01	0\\
74.01	0\\
75.01	0\\
76.01	0\\
77.01	0\\
78.01	0\\
79.01	0\\
80.01	0\\
81.01	0\\
82.01	0\\
83.01	0\\
84.01	0\\
85.01	0\\
86.01	0\\
87.01	0\\
88.01	0\\
89.01	0\\
90.01	0\\
91.01	0\\
92.01	0\\
93.01	0\\
94.01	0\\
95.01	0\\
96.01	0\\
97.01	0\\
98.01	0\\
99.01	0\\
100.01	0\\
101.01	0\\
102.01	0\\
103.01	0\\
104.01	0\\
105.01	0\\
106.01	0\\
107.01	0\\
108.01	0\\
109.01	0\\
110.01	0\\
111.01	0\\
112.01	0\\
113.01	0\\
114.01	0\\
115.01	0\\
116.01	0\\
117.01	0\\
118.01	0\\
119.01	0\\
120.01	0\\
121.01	0\\
122.01	0\\
123.01	0\\
124.01	0\\
125.01	0\\
126.01	0\\
127.01	0\\
128.01	0\\
129.01	0\\
130.01	0\\
131.01	0\\
132.01	0\\
133.01	0\\
134.01	0\\
135.01	0\\
136.01	0\\
137.01	0\\
138.01	0\\
139.01	0\\
140.01	0\\
141.01	0\\
142.01	0\\
143.01	0\\
144.01	0\\
145.01	0\\
146.01	0\\
147.01	0\\
148.01	0\\
149.01	0\\
150.01	0\\
151.01	0\\
152.01	0\\
153.01	0\\
154.01	0\\
155.01	0\\
156.01	0\\
157.01	0\\
158.01	0\\
159.01	0\\
160.01	0\\
161.01	0\\
162.01	0\\
163.01	0\\
164.01	0\\
165.01	0\\
166.01	0\\
167.01	0\\
168.01	0\\
169.01	0\\
170.01	0\\
171.01	0\\
172.01	0\\
173.01	0\\
174.01	0\\
175.01	0\\
176.01	0\\
177.01	0\\
178.01	0\\
179.01	0\\
180.01	0\\
181.01	0\\
182.01	0\\
183.01	0\\
184.01	0\\
185.01	0\\
186.01	0\\
187.01	0\\
188.01	0\\
189.01	0\\
190.01	0\\
191.01	0\\
192.01	0\\
193.01	0\\
194.01	0\\
195.01	0\\
196.01	0\\
197.01	0\\
198.01	0\\
199.01	0\\
200.01	0\\
201.01	0\\
202.01	0\\
203.01	0\\
204.01	0\\
205.01	0\\
206.01	0\\
207.01	0\\
208.01	0\\
209.01	0\\
210.01	0\\
211.01	0\\
212.01	0\\
213.01	0\\
214.01	0\\
215.01	0\\
216.01	0\\
217.01	0\\
218.01	0\\
219.01	0\\
220.01	0\\
221.01	0\\
222.01	0\\
223.01	0\\
224.01	0\\
225.01	0\\
226.01	0\\
227.01	0\\
228.01	0\\
229.01	0\\
230.01	0\\
231.01	0\\
232.01	0\\
233.01	0\\
234.01	0\\
235.01	0\\
236.01	0\\
237.01	0\\
238.01	0\\
239.01	0\\
240.01	0\\
241.01	0\\
242.01	0\\
243.01	0\\
244.01	0\\
245.01	0\\
246.01	0\\
247.01	0\\
248.01	0\\
249.01	0\\
250.01	0\\
251.01	0\\
252.01	0\\
253.01	0\\
254.01	0\\
255.01	0\\
256.01	0\\
257.01	0\\
258.01	0\\
259.01	0\\
260.01	0\\
261.01	0\\
262.01	0\\
263.01	0\\
264.01	0\\
265.01	0\\
266.01	0\\
267.01	0\\
268.01	0\\
269.01	0\\
270.01	0\\
271.01	0\\
272.01	0\\
273.01	0\\
274.01	0\\
275.01	0\\
276.01	0\\
277.01	0\\
278.01	0\\
279.01	0\\
280.01	0\\
281.01	0\\
282.01	0\\
283.01	0\\
284.01	0\\
285.01	0\\
286.01	0\\
287.01	0\\
288.01	0\\
289.01	0\\
290.01	0\\
291.01	0\\
292.01	0\\
293.01	0\\
294.01	0\\
295.01	0\\
296.01	0\\
297.01	0\\
298.01	0\\
299.01	0\\
300.01	0\\
301.01	0\\
302.01	0\\
303.01	0\\
304.01	0\\
305.01	0\\
306.01	0\\
307.01	0\\
308.01	0\\
309.01	0\\
310.01	0\\
311.01	0\\
312.01	0\\
313.01	0\\
314.01	0\\
315.01	0\\
316.01	0\\
317.01	0\\
318.01	0\\
319.01	0\\
320.01	0\\
321.01	0\\
322.01	0\\
323.01	0\\
324.01	0\\
325.01	0\\
326.01	0\\
327.01	0\\
328.01	0\\
329.01	0\\
330.01	0\\
331.01	0\\
332.01	0\\
333.01	0\\
334.01	0\\
335.01	0\\
336.01	0\\
337.01	0\\
338.01	0\\
339.01	0\\
340.01	0\\
341.01	0\\
342.01	0\\
343.01	0\\
344.01	0\\
345.01	0\\
346.01	0\\
347.01	0\\
348.01	0\\
349.01	0\\
350.01	0\\
351.01	0\\
352.01	0\\
353.01	0\\
354.01	0\\
355.01	0\\
356.01	0\\
357.01	0\\
358.01	0\\
359.01	0\\
360.01	0\\
361.01	0\\
362.01	0\\
363.01	0\\
364.01	0\\
365.01	0\\
366.01	0\\
367.01	0\\
368.01	0\\
369.01	0\\
370.01	0\\
371.01	0\\
372.01	0\\
373.01	0\\
374.01	0\\
375.01	0\\
376.01	0\\
377.01	0\\
378.01	0\\
379.01	0\\
380.01	0\\
381.01	0\\
382.01	0\\
383.01	0\\
384.01	0\\
385.01	0\\
386.01	0\\
387.01	0\\
388.01	0\\
389.01	0\\
390.01	0\\
391.01	0\\
392.01	0\\
393.01	0\\
394.01	0\\
395.01	0\\
396.01	0\\
397.01	0\\
398.01	0\\
399.01	0\\
400.01	0\\
401.01	0\\
402.01	0\\
403.01	0\\
404.01	0\\
405.01	0\\
406.01	0\\
407.01	0\\
408.01	0\\
409.01	0\\
410.01	0\\
411.01	0\\
412.01	0\\
413.01	0\\
414.01	0\\
415.01	0\\
416.01	0\\
417.01	0\\
418.01	0\\
419.01	0\\
420.01	0\\
421.01	0\\
422.01	0\\
423.01	0\\
424.01	0\\
425.01	0\\
426.01	0\\
427.01	0\\
428.01	0\\
429.01	0\\
430.01	0\\
431.01	0\\
432.01	0\\
433.01	0\\
434.01	0\\
435.01	0\\
436.01	0\\
437.01	0\\
438.01	0\\
439.01	0\\
440.01	0\\
441.01	0\\
442.01	0\\
443.01	0\\
444.01	0\\
445.01	0\\
446.01	0\\
447.01	0\\
448.01	0\\
449.01	0\\
450.01	0\\
451.01	0\\
452.01	0\\
453.01	0\\
454.01	0\\
455.01	0\\
456.01	0\\
457.01	0\\
458.01	0\\
459.01	0\\
460.01	0\\
461.01	0\\
462.01	0\\
463.01	0\\
464.01	0\\
465.01	0\\
466.01	0\\
467.01	0\\
468.01	0\\
469.01	0\\
470.01	0\\
471.01	0\\
472.01	0\\
473.01	0\\
474.01	0\\
475.01	0\\
476.01	0\\
477.01	0\\
478.01	0\\
479.01	0\\
480.01	0\\
481.01	0\\
482.01	0\\
483.01	0\\
484.01	0\\
485.01	0\\
486.01	0\\
487.01	0\\
488.01	0\\
489.01	0\\
490.01	0\\
491.01	0\\
492.01	0\\
493.01	0\\
494.01	0\\
495.01	0\\
496.01	0\\
497.01	0\\
498.01	0\\
499.01	0\\
500.01	0\\
501.01	0\\
502.01	0\\
503.01	0\\
504.01	0\\
505.01	0\\
506.01	0\\
507.01	0\\
508.01	0\\
509.01	0\\
510.01	0\\
511.01	0\\
512.01	0\\
513.01	0\\
514.01	0\\
515.01	0\\
516.01	0\\
517.01	0\\
518.01	0\\
519.01	0\\
520.01	0\\
521.01	0\\
522.01	0\\
523.01	0\\
524.01	0\\
525.01	0\\
526.01	0\\
527.01	0\\
528.01	0\\
529.01	0\\
530.01	0\\
531.01	0\\
532.01	0\\
533.01	0\\
534.01	0\\
535.01	0\\
536.01	0\\
537.01	0\\
538.01	0\\
539.01	0\\
540.01	0\\
541.01	0\\
542.01	0\\
543.01	0\\
544.01	0\\
545.01	0\\
546.01	0\\
547.01	0\\
548.01	0\\
549.01	0\\
550.01	0\\
551.01	0\\
552.01	0\\
553.01	0\\
554.01	0\\
555.01	0\\
556.01	0\\
557.01	0\\
558.01	0\\
559.01	0\\
560.01	0\\
561.01	0\\
562.01	0\\
563.01	0\\
564.01	0\\
565.01	0\\
566.01	0\\
567.01	0\\
568.01	0\\
569.01	0\\
570.01	0\\
571.01	0\\
572.01	0\\
573.01	0\\
574.01	0\\
575.01	0\\
576.01	0\\
577.01	0\\
578.01	0\\
579.01	0\\
580.01	0\\
581.01	0\\
582.01	0\\
583.01	0\\
584.01	0\\
585.01	0\\
586.01	0\\
587.01	0\\
588.01	0\\
589.01	0\\
590.01	0\\
591.01	0\\
592.01	0\\
593.01	0\\
594.01	0\\
595.01	0\\
596.01	0\\
597.01	0.000481110898971432\\
598.01	0.00143832068185183\\
599.01	0.00385661257211628\\
599.02	0.00389414818835235\\
599.03	0.00393204146009088\\
599.04	0.00397029584140061\\
599.05	0.00400891481968847\\
599.06	0.0040479019160211\\
599.07	0.00408726068544947\\
599.08	0.00412699471733662\\
599.09	0.00416710763568851\\
599.1	0.00420760309948819\\
599.11	0.004248484803033\\
599.12	0.00428975647627518\\
599.13	0.00433142188516569\\
599.14	0.00437348483200137\\
599.15	0.00441594915577538\\
599.16	0.00445881873253112\\
599.17	0.00450209747571947\\
599.18	0.00454578933655951\\
599.19	0.00458989830440275\\
599.2	0.00463442840710077\\
599.21	0.00467938371137652\\
599.22	0.00472476832319911\\
599.23	0.00477058638816225\\
599.24	0.00481684209186631\\
599.25	0.00486353966030412\\
599.26	0.00491068334955366\\
599.27	0.00495827744958435\\
599.28	0.00500632629158769\\
599.29	0.00505483424837289\\
599.3	0.00510380573476643\\
599.31	0.00515324520801537\\
599.32	0.00520315716819457\\
599.33	0.00525354615861776\\
599.34	0.00530441676625261\\
599.35	0.00535577362213975\\
599.36	0.0054076214018158\\
599.37	0.00545996482574054\\
599.38	0.00551280865972799\\
599.39	0.00556615771538183\\
599.4	0.00562001685053486\\
599.41	0.00567439096969276\\
599.42	0.00572928502448199\\
599.43	0.00578470401410217\\
599.44	0.00584065298578253\\
599.45	0.00589713703524306\\
599.46	0.00595416130715976\\
599.47	0.00601173099563461\\
599.48	0.00606985134466991\\
599.49	0.00612852764864715\\
599.5	0.00618776525281058\\
599.51	0.00624756955375528\\
599.52	0.00630794599992008\\
599.53	0.00636890009208502\\
599.54	0.00643043738387371\\
599.55	0.00649256348226045\\
599.56	0.00655528404808223\\
599.57	0.00661860479655559\\
599.58	0.00668253149779856\\
599.59	0.00674706997735745\\
599.6	0.00681222611673879\\
599.61	0.00687800585394631\\
599.62	0.00694441518402313\\
599.63	0.00701146015959912\\
599.64	0.00707914689144344\\
599.65	0.00714748154902252\\
599.66	0.00721647036106322\\
599.67	0.00728611961612156\\
599.68	0.00735643566315675\\
599.69	0.00742742491211077\\
599.7	0.00749909383449363\\
599.71	0.00757144896397399\\
599.72	0.00764449689697571\\
599.73	0.00771824429327989\\
599.74	0.00779269787663284\\
599.75	0.00786786443535982\\
599.76	0.0079437508229846\\
599.77	0.00802036395885509\\
599.78	0.00809771082877485\\
599.79	0.0081757984856407\\
599.8	0.00825463405008647\\
599.81	0.00833422471113284\\
599.82	0.00841457772684348\\
599.83	0.00849570042498746\\
599.84	0.00857760020370798\\
599.85	0.00866028453219752\\
599.86	0.00874376095137953\\
599.87	0.00882803707459654\\
599.88	0.00891312058830489\\
599.89	0.00899901925277625\\
599.9	0.00908574090280562\\
599.91	0.00917329344842636\\
599.92	0.00926168487563191\\
599.93	0.00935092324710449\\
599.94	0.00944101670295078\\
599.95	0.00953197346144465\\
599.96	0.00962380181977694\\
599.97	0.00971651015481255\\
599.98	0.0098101069238547\\
599.99	0.00990460066541651\\
600	0.01\\
};
\addplot [color=mycolor21,solid,forget plot]
  table[row sep=crcr]{%
0.01	0\\
1.01	0\\
2.01	0\\
3.01	0\\
4.01	0\\
5.01	0\\
6.01	0\\
7.01	0\\
8.01	0\\
9.01	0\\
10.01	0\\
11.01	0\\
12.01	0\\
13.01	0\\
14.01	0\\
15.01	0\\
16.01	0\\
17.01	0\\
18.01	0\\
19.01	0\\
20.01	0\\
21.01	0\\
22.01	0\\
23.01	0\\
24.01	0\\
25.01	0\\
26.01	0\\
27.01	0\\
28.01	0\\
29.01	0\\
30.01	0\\
31.01	0\\
32.01	0\\
33.01	0\\
34.01	0\\
35.01	0\\
36.01	0\\
37.01	0\\
38.01	0\\
39.01	0\\
40.01	0\\
41.01	0\\
42.01	0\\
43.01	0\\
44.01	0\\
45.01	0\\
46.01	0\\
47.01	0\\
48.01	0\\
49.01	0\\
50.01	0\\
51.01	0\\
52.01	0\\
53.01	0\\
54.01	0\\
55.01	0\\
56.01	0\\
57.01	0\\
58.01	0\\
59.01	0\\
60.01	0\\
61.01	0\\
62.01	0\\
63.01	0\\
64.01	0\\
65.01	0\\
66.01	0\\
67.01	0\\
68.01	0\\
69.01	0\\
70.01	0\\
71.01	0\\
72.01	0\\
73.01	0\\
74.01	0\\
75.01	0\\
76.01	0\\
77.01	0\\
78.01	0\\
79.01	0\\
80.01	0\\
81.01	0\\
82.01	0\\
83.01	0\\
84.01	0\\
85.01	0\\
86.01	0\\
87.01	0\\
88.01	0\\
89.01	0\\
90.01	0\\
91.01	0\\
92.01	0\\
93.01	0\\
94.01	0\\
95.01	0\\
96.01	0\\
97.01	0\\
98.01	0\\
99.01	0\\
100.01	0\\
101.01	0\\
102.01	0\\
103.01	0\\
104.01	0\\
105.01	0\\
106.01	0\\
107.01	0\\
108.01	0\\
109.01	0\\
110.01	0\\
111.01	0\\
112.01	0\\
113.01	0\\
114.01	0\\
115.01	0\\
116.01	0\\
117.01	0\\
118.01	0\\
119.01	0\\
120.01	0\\
121.01	0\\
122.01	0\\
123.01	0\\
124.01	0\\
125.01	0\\
126.01	0\\
127.01	0\\
128.01	0\\
129.01	0\\
130.01	0\\
131.01	0\\
132.01	0\\
133.01	0\\
134.01	0\\
135.01	0\\
136.01	0\\
137.01	0\\
138.01	0\\
139.01	0\\
140.01	0\\
141.01	0\\
142.01	0\\
143.01	0\\
144.01	0\\
145.01	0\\
146.01	0\\
147.01	0\\
148.01	0\\
149.01	0\\
150.01	0\\
151.01	0\\
152.01	0\\
153.01	0\\
154.01	0\\
155.01	0\\
156.01	0\\
157.01	0\\
158.01	0\\
159.01	0\\
160.01	0\\
161.01	0\\
162.01	0\\
163.01	0\\
164.01	0\\
165.01	0\\
166.01	0\\
167.01	0\\
168.01	0\\
169.01	0\\
170.01	0\\
171.01	0\\
172.01	0\\
173.01	0\\
174.01	0\\
175.01	0\\
176.01	0\\
177.01	0\\
178.01	0\\
179.01	0\\
180.01	0\\
181.01	0\\
182.01	0\\
183.01	0\\
184.01	0\\
185.01	0\\
186.01	0\\
187.01	0\\
188.01	0\\
189.01	0\\
190.01	0\\
191.01	0\\
192.01	0\\
193.01	0\\
194.01	0\\
195.01	0\\
196.01	0\\
197.01	0\\
198.01	0\\
199.01	0\\
200.01	0\\
201.01	0\\
202.01	0\\
203.01	0\\
204.01	0\\
205.01	0\\
206.01	0\\
207.01	0\\
208.01	0\\
209.01	0\\
210.01	0\\
211.01	0\\
212.01	0\\
213.01	0\\
214.01	0\\
215.01	0\\
216.01	0\\
217.01	0\\
218.01	0\\
219.01	0\\
220.01	0\\
221.01	0\\
222.01	0\\
223.01	0\\
224.01	0\\
225.01	0\\
226.01	0\\
227.01	0\\
228.01	0\\
229.01	0\\
230.01	0\\
231.01	0\\
232.01	0\\
233.01	0\\
234.01	0\\
235.01	0\\
236.01	0\\
237.01	0\\
238.01	0\\
239.01	0\\
240.01	0\\
241.01	0\\
242.01	0\\
243.01	0\\
244.01	0\\
245.01	0\\
246.01	0\\
247.01	0\\
248.01	0\\
249.01	0\\
250.01	0\\
251.01	0\\
252.01	0\\
253.01	0\\
254.01	0\\
255.01	0\\
256.01	0\\
257.01	0\\
258.01	0\\
259.01	0\\
260.01	0\\
261.01	0\\
262.01	0\\
263.01	0\\
264.01	0\\
265.01	0\\
266.01	0\\
267.01	0\\
268.01	0\\
269.01	0\\
270.01	0\\
271.01	0\\
272.01	0\\
273.01	0\\
274.01	0\\
275.01	0\\
276.01	0\\
277.01	0\\
278.01	0\\
279.01	0\\
280.01	0\\
281.01	0\\
282.01	0\\
283.01	0\\
284.01	0\\
285.01	0\\
286.01	0\\
287.01	0\\
288.01	0\\
289.01	0\\
290.01	0\\
291.01	0\\
292.01	0\\
293.01	0\\
294.01	0\\
295.01	0\\
296.01	0\\
297.01	0\\
298.01	0\\
299.01	0\\
300.01	0\\
301.01	0\\
302.01	0\\
303.01	0\\
304.01	0\\
305.01	0\\
306.01	0\\
307.01	0\\
308.01	0\\
309.01	0\\
310.01	0\\
311.01	0\\
312.01	0\\
313.01	0\\
314.01	0\\
315.01	0\\
316.01	0\\
317.01	0\\
318.01	0\\
319.01	0\\
320.01	0\\
321.01	0\\
322.01	0\\
323.01	0\\
324.01	0\\
325.01	0\\
326.01	0\\
327.01	0\\
328.01	0\\
329.01	0\\
330.01	0\\
331.01	0\\
332.01	0\\
333.01	0\\
334.01	0\\
335.01	0\\
336.01	0\\
337.01	0\\
338.01	0\\
339.01	0\\
340.01	0\\
341.01	0\\
342.01	0\\
343.01	0\\
344.01	0\\
345.01	0\\
346.01	0\\
347.01	0\\
348.01	0\\
349.01	0\\
350.01	0\\
351.01	0\\
352.01	0\\
353.01	0\\
354.01	0\\
355.01	0\\
356.01	0\\
357.01	0\\
358.01	0\\
359.01	0\\
360.01	0\\
361.01	0\\
362.01	0\\
363.01	0\\
364.01	0\\
365.01	0\\
366.01	0\\
367.01	0\\
368.01	0\\
369.01	0\\
370.01	0\\
371.01	0\\
372.01	0\\
373.01	0\\
374.01	0\\
375.01	0\\
376.01	0\\
377.01	0\\
378.01	0\\
379.01	0\\
380.01	0\\
381.01	0\\
382.01	0\\
383.01	0\\
384.01	0\\
385.01	0\\
386.01	0\\
387.01	0\\
388.01	0\\
389.01	0\\
390.01	0\\
391.01	0\\
392.01	0\\
393.01	0\\
394.01	0\\
395.01	0\\
396.01	0\\
397.01	0\\
398.01	0\\
399.01	0\\
400.01	0\\
401.01	0\\
402.01	0\\
403.01	0\\
404.01	0\\
405.01	0\\
406.01	0\\
407.01	0\\
408.01	0\\
409.01	0\\
410.01	0\\
411.01	0\\
412.01	0\\
413.01	0\\
414.01	0\\
415.01	0\\
416.01	0\\
417.01	0\\
418.01	0\\
419.01	0\\
420.01	0\\
421.01	0\\
422.01	0\\
423.01	0\\
424.01	0\\
425.01	0\\
426.01	0\\
427.01	0\\
428.01	0\\
429.01	0\\
430.01	0\\
431.01	0\\
432.01	0\\
433.01	0\\
434.01	0\\
435.01	0\\
436.01	0\\
437.01	0\\
438.01	0\\
439.01	0\\
440.01	0\\
441.01	0\\
442.01	0\\
443.01	0\\
444.01	0\\
445.01	0\\
446.01	0\\
447.01	0\\
448.01	0\\
449.01	0\\
450.01	0\\
451.01	0\\
452.01	0\\
453.01	0\\
454.01	0\\
455.01	0\\
456.01	0\\
457.01	0\\
458.01	0\\
459.01	0\\
460.01	0\\
461.01	0\\
462.01	0\\
463.01	0\\
464.01	0\\
465.01	0\\
466.01	0\\
467.01	0\\
468.01	0\\
469.01	0\\
470.01	0\\
471.01	0\\
472.01	0\\
473.01	0\\
474.01	0\\
475.01	0\\
476.01	0\\
477.01	0\\
478.01	0\\
479.01	0\\
480.01	0\\
481.01	0\\
482.01	0\\
483.01	0\\
484.01	0\\
485.01	0\\
486.01	0\\
487.01	0\\
488.01	0\\
489.01	0\\
490.01	0\\
491.01	0\\
492.01	0\\
493.01	0\\
494.01	0\\
495.01	0\\
496.01	0\\
497.01	0\\
498.01	0\\
499.01	0\\
500.01	0\\
501.01	0\\
502.01	0\\
503.01	0\\
504.01	0\\
505.01	0\\
506.01	0\\
507.01	0\\
508.01	0\\
509.01	0\\
510.01	0\\
511.01	0\\
512.01	0\\
513.01	0\\
514.01	0\\
515.01	0\\
516.01	0\\
517.01	0\\
518.01	0\\
519.01	0\\
520.01	0\\
521.01	0\\
522.01	0\\
523.01	0\\
524.01	0\\
525.01	0\\
526.01	0\\
527.01	0\\
528.01	0\\
529.01	0\\
530.01	0\\
531.01	0\\
532.01	0\\
533.01	0\\
534.01	0\\
535.01	0\\
536.01	0\\
537.01	0\\
538.01	0\\
539.01	0\\
540.01	0\\
541.01	0\\
542.01	0\\
543.01	0\\
544.01	0\\
545.01	0\\
546.01	0\\
547.01	0\\
548.01	0\\
549.01	0\\
550.01	0\\
551.01	0\\
552.01	0\\
553.01	0\\
554.01	0\\
555.01	0\\
556.01	0\\
557.01	0\\
558.01	0\\
559.01	0\\
560.01	0\\
561.01	0\\
562.01	0\\
563.01	0\\
564.01	0\\
565.01	0\\
566.01	0\\
567.01	0\\
568.01	0\\
569.01	0\\
570.01	0\\
571.01	0\\
572.01	0\\
573.01	0\\
574.01	0\\
575.01	0\\
576.01	0\\
577.01	0\\
578.01	0\\
579.01	0\\
580.01	0\\
581.01	0\\
582.01	0\\
583.01	0\\
584.01	0\\
585.01	0\\
586.01	0\\
587.01	0\\
588.01	0\\
589.01	0\\
590.01	0\\
591.01	0\\
592.01	0\\
593.01	0\\
594.01	0\\
595.01	0\\
596.01	0\\
597.01	0.000481187488961005\\
598.01	0.00143832068185176\\
599.01	0.00385661257211614\\
599.02	0.00389414818835221\\
599.03	0.00393204146009075\\
599.04	0.00397029584140049\\
599.05	0.00400891481968836\\
599.06	0.004047901916021\\
599.07	0.00408726068544937\\
599.08	0.00412699471733652\\
599.09	0.00416710763568842\\
599.1	0.00420760309948809\\
599.11	0.0042484848030329\\
599.12	0.00428975647627509\\
599.13	0.00433142188516562\\
599.14	0.00437348483200131\\
599.15	0.00441594915577533\\
599.16	0.00445881873253107\\
599.17	0.0045020974757194\\
599.18	0.00454578933655944\\
599.19	0.00458989830440268\\
599.2	0.0046344284071007\\
599.21	0.00467938371137645\\
599.22	0.00472476832319906\\
599.23	0.0047705863881622\\
599.24	0.00481684209186628\\
599.25	0.0048635396603041\\
599.26	0.00491068334955364\\
599.27	0.00495827744958434\\
599.28	0.00500632629158766\\
599.29	0.00505483424837286\\
599.3	0.00510380573476639\\
599.31	0.00515324520801533\\
599.32	0.00520315716819451\\
599.33	0.0052535461586177\\
599.34	0.00530441676625255\\
599.35	0.00535577362213968\\
599.36	0.00540762140181575\\
599.37	0.00545996482574049\\
599.38	0.00551280865972795\\
599.39	0.00556615771538178\\
599.4	0.00562001685053481\\
599.41	0.0056743909696927\\
599.42	0.00572928502448196\\
599.43	0.00578470401410212\\
599.44	0.0058406529857825\\
599.45	0.00589713703524301\\
599.46	0.00595416130715971\\
599.47	0.00601173099563457\\
599.48	0.00606985134466987\\
599.49	0.00612852764864711\\
599.5	0.00618776525281053\\
599.51	0.00624756955375524\\
599.52	0.00630794599992004\\
599.53	0.006368900092085\\
599.54	0.00643043738387369\\
599.55	0.00649256348226044\\
599.56	0.00655528404808221\\
599.57	0.00661860479655558\\
599.58	0.00668253149779856\\
599.59	0.00674706997735745\\
599.6	0.00681222611673878\\
599.61	0.0068780058539463\\
599.62	0.00694441518402313\\
599.63	0.00701146015959912\\
599.64	0.00707914689144345\\
599.65	0.00714748154902253\\
599.66	0.00721647036106324\\
599.67	0.00728611961612158\\
599.68	0.00735643566315675\\
599.69	0.00742742491211078\\
599.7	0.00749909383449363\\
599.71	0.00757144896397399\\
599.72	0.00764449689697571\\
599.73	0.00771824429327989\\
599.74	0.00779269787663285\\
599.75	0.00786786443535983\\
599.76	0.00794375082298461\\
599.77	0.0080203639588551\\
599.78	0.00809771082877486\\
599.79	0.00817579848564071\\
599.8	0.00825463405008649\\
599.81	0.00833422471113285\\
599.82	0.00841457772684349\\
599.83	0.00849570042498747\\
599.84	0.00857760020370797\\
599.85	0.00866028453219752\\
599.86	0.00874376095137953\\
599.87	0.00882803707459654\\
599.88	0.0089131205883049\\
599.89	0.00899901925277625\\
599.9	0.00908574090280562\\
599.91	0.00917329344842636\\
599.92	0.0092616848756319\\
599.93	0.00935092324710449\\
599.94	0.00944101670295078\\
599.95	0.00953197346144465\\
599.96	0.00962380181977693\\
599.97	0.00971651015481255\\
599.98	0.0098101069238547\\
599.99	0.00990460066541651\\
600	0.01\\
};
\addplot [color=black!20!mycolor21,solid,forget plot]
  table[row sep=crcr]{%
0.01	0\\
1.01	0\\
2.01	0\\
3.01	0\\
4.01	0\\
5.01	0\\
6.01	0\\
7.01	0\\
8.01	0\\
9.01	0\\
10.01	0\\
11.01	0\\
12.01	0\\
13.01	0\\
14.01	0\\
15.01	0\\
16.01	0\\
17.01	0\\
18.01	0\\
19.01	0\\
20.01	0\\
21.01	0\\
22.01	0\\
23.01	0\\
24.01	0\\
25.01	0\\
26.01	0\\
27.01	0\\
28.01	0\\
29.01	0\\
30.01	0\\
31.01	0\\
32.01	0\\
33.01	0\\
34.01	0\\
35.01	0\\
36.01	0\\
37.01	0\\
38.01	0\\
39.01	0\\
40.01	0\\
41.01	0\\
42.01	0\\
43.01	0\\
44.01	0\\
45.01	0\\
46.01	0\\
47.01	0\\
48.01	0\\
49.01	0\\
50.01	0\\
51.01	0\\
52.01	0\\
53.01	0\\
54.01	0\\
55.01	0\\
56.01	0\\
57.01	0\\
58.01	0\\
59.01	0\\
60.01	0\\
61.01	0\\
62.01	0\\
63.01	0\\
64.01	0\\
65.01	0\\
66.01	0\\
67.01	0\\
68.01	0\\
69.01	0\\
70.01	0\\
71.01	0\\
72.01	0\\
73.01	0\\
74.01	0\\
75.01	0\\
76.01	0\\
77.01	0\\
78.01	0\\
79.01	0\\
80.01	0\\
81.01	0\\
82.01	0\\
83.01	0\\
84.01	0\\
85.01	0\\
86.01	0\\
87.01	0\\
88.01	0\\
89.01	0\\
90.01	0\\
91.01	0\\
92.01	0\\
93.01	0\\
94.01	0\\
95.01	0\\
96.01	0\\
97.01	0\\
98.01	0\\
99.01	0\\
100.01	0\\
101.01	0\\
102.01	0\\
103.01	0\\
104.01	0\\
105.01	0\\
106.01	0\\
107.01	0\\
108.01	0\\
109.01	0\\
110.01	0\\
111.01	0\\
112.01	0\\
113.01	0\\
114.01	0\\
115.01	0\\
116.01	0\\
117.01	0\\
118.01	0\\
119.01	0\\
120.01	0\\
121.01	0\\
122.01	0\\
123.01	0\\
124.01	0\\
125.01	0\\
126.01	0\\
127.01	0\\
128.01	0\\
129.01	0\\
130.01	0\\
131.01	0\\
132.01	0\\
133.01	0\\
134.01	0\\
135.01	0\\
136.01	0\\
137.01	0\\
138.01	0\\
139.01	0\\
140.01	0\\
141.01	0\\
142.01	0\\
143.01	0\\
144.01	0\\
145.01	0\\
146.01	0\\
147.01	0\\
148.01	0\\
149.01	0\\
150.01	0\\
151.01	0\\
152.01	0\\
153.01	0\\
154.01	0\\
155.01	0\\
156.01	0\\
157.01	0\\
158.01	0\\
159.01	0\\
160.01	0\\
161.01	0\\
162.01	0\\
163.01	0\\
164.01	0\\
165.01	0\\
166.01	0\\
167.01	0\\
168.01	0\\
169.01	0\\
170.01	0\\
171.01	0\\
172.01	0\\
173.01	0\\
174.01	0\\
175.01	0\\
176.01	0\\
177.01	0\\
178.01	0\\
179.01	0\\
180.01	0\\
181.01	0\\
182.01	0\\
183.01	0\\
184.01	0\\
185.01	0\\
186.01	0\\
187.01	0\\
188.01	0\\
189.01	0\\
190.01	0\\
191.01	0\\
192.01	0\\
193.01	0\\
194.01	0\\
195.01	0\\
196.01	0\\
197.01	0\\
198.01	0\\
199.01	0\\
200.01	0\\
201.01	0\\
202.01	0\\
203.01	0\\
204.01	0\\
205.01	0\\
206.01	0\\
207.01	0\\
208.01	0\\
209.01	0\\
210.01	0\\
211.01	0\\
212.01	0\\
213.01	0\\
214.01	0\\
215.01	0\\
216.01	0\\
217.01	0\\
218.01	0\\
219.01	0\\
220.01	0\\
221.01	0\\
222.01	0\\
223.01	0\\
224.01	0\\
225.01	0\\
226.01	0\\
227.01	0\\
228.01	0\\
229.01	0\\
230.01	0\\
231.01	0\\
232.01	0\\
233.01	0\\
234.01	0\\
235.01	0\\
236.01	0\\
237.01	0\\
238.01	0\\
239.01	0\\
240.01	0\\
241.01	0\\
242.01	0\\
243.01	0\\
244.01	0\\
245.01	0\\
246.01	0\\
247.01	0\\
248.01	0\\
249.01	0\\
250.01	0\\
251.01	0\\
252.01	0\\
253.01	0\\
254.01	0\\
255.01	0\\
256.01	0\\
257.01	0\\
258.01	0\\
259.01	0\\
260.01	0\\
261.01	0\\
262.01	0\\
263.01	0\\
264.01	0\\
265.01	0\\
266.01	0\\
267.01	0\\
268.01	0\\
269.01	0\\
270.01	0\\
271.01	0\\
272.01	0\\
273.01	0\\
274.01	0\\
275.01	0\\
276.01	0\\
277.01	0\\
278.01	0\\
279.01	0\\
280.01	0\\
281.01	0\\
282.01	0\\
283.01	0\\
284.01	0\\
285.01	0\\
286.01	0\\
287.01	0\\
288.01	0\\
289.01	0\\
290.01	0\\
291.01	0\\
292.01	0\\
293.01	0\\
294.01	0\\
295.01	0\\
296.01	0\\
297.01	0\\
298.01	0\\
299.01	0\\
300.01	0\\
301.01	0\\
302.01	0\\
303.01	0\\
304.01	0\\
305.01	0\\
306.01	0\\
307.01	0\\
308.01	0\\
309.01	0\\
310.01	0\\
311.01	0\\
312.01	0\\
313.01	0\\
314.01	0\\
315.01	0\\
316.01	0\\
317.01	0\\
318.01	0\\
319.01	0\\
320.01	0\\
321.01	0\\
322.01	0\\
323.01	0\\
324.01	0\\
325.01	0\\
326.01	0\\
327.01	0\\
328.01	0\\
329.01	0\\
330.01	0\\
331.01	0\\
332.01	0\\
333.01	0\\
334.01	0\\
335.01	0\\
336.01	0\\
337.01	0\\
338.01	0\\
339.01	0\\
340.01	0\\
341.01	0\\
342.01	0\\
343.01	0\\
344.01	0\\
345.01	0\\
346.01	0\\
347.01	0\\
348.01	0\\
349.01	0\\
350.01	0\\
351.01	0\\
352.01	0\\
353.01	0\\
354.01	0\\
355.01	0\\
356.01	0\\
357.01	0\\
358.01	0\\
359.01	0\\
360.01	0\\
361.01	0\\
362.01	0\\
363.01	0\\
364.01	0\\
365.01	0\\
366.01	0\\
367.01	0\\
368.01	0\\
369.01	0\\
370.01	0\\
371.01	0\\
372.01	0\\
373.01	0\\
374.01	0\\
375.01	0\\
376.01	0\\
377.01	0\\
378.01	0\\
379.01	0\\
380.01	0\\
381.01	0\\
382.01	0\\
383.01	0\\
384.01	0\\
385.01	0\\
386.01	0\\
387.01	0\\
388.01	0\\
389.01	0\\
390.01	0\\
391.01	0\\
392.01	0\\
393.01	0\\
394.01	0\\
395.01	0\\
396.01	0\\
397.01	0\\
398.01	0\\
399.01	0\\
400.01	0\\
401.01	0\\
402.01	0\\
403.01	0\\
404.01	0\\
405.01	0\\
406.01	0\\
407.01	0\\
408.01	0\\
409.01	0\\
410.01	0\\
411.01	0\\
412.01	0\\
413.01	0\\
414.01	0\\
415.01	0\\
416.01	0\\
417.01	0\\
418.01	0\\
419.01	0\\
420.01	0\\
421.01	0\\
422.01	0\\
423.01	0\\
424.01	0\\
425.01	0\\
426.01	0\\
427.01	0\\
428.01	0\\
429.01	0\\
430.01	0\\
431.01	0\\
432.01	0\\
433.01	0\\
434.01	0\\
435.01	0\\
436.01	0\\
437.01	0\\
438.01	0\\
439.01	0\\
440.01	0\\
441.01	0\\
442.01	0\\
443.01	0\\
444.01	0\\
445.01	0\\
446.01	0\\
447.01	0\\
448.01	0\\
449.01	0\\
450.01	0\\
451.01	0\\
452.01	0\\
453.01	0\\
454.01	0\\
455.01	0\\
456.01	0\\
457.01	0\\
458.01	0\\
459.01	0\\
460.01	0\\
461.01	0\\
462.01	0\\
463.01	0\\
464.01	0\\
465.01	0\\
466.01	0\\
467.01	0\\
468.01	0\\
469.01	0\\
470.01	0\\
471.01	0\\
472.01	0\\
473.01	0\\
474.01	0\\
475.01	0\\
476.01	0\\
477.01	0\\
478.01	0\\
479.01	0\\
480.01	0\\
481.01	0\\
482.01	0\\
483.01	0\\
484.01	0\\
485.01	0\\
486.01	0\\
487.01	0\\
488.01	0\\
489.01	0\\
490.01	0\\
491.01	0\\
492.01	0\\
493.01	0\\
494.01	0\\
495.01	0\\
496.01	0\\
497.01	0\\
498.01	0\\
499.01	0\\
500.01	0\\
501.01	0\\
502.01	0\\
503.01	0\\
504.01	0\\
505.01	0\\
506.01	0\\
507.01	0\\
508.01	0\\
509.01	0\\
510.01	0\\
511.01	0\\
512.01	0\\
513.01	0\\
514.01	0\\
515.01	0\\
516.01	0\\
517.01	0\\
518.01	0\\
519.01	0\\
520.01	0\\
521.01	0\\
522.01	0\\
523.01	0\\
524.01	0\\
525.01	0\\
526.01	0\\
527.01	0\\
528.01	0\\
529.01	0\\
530.01	0\\
531.01	0\\
532.01	0\\
533.01	0\\
534.01	0\\
535.01	0\\
536.01	0\\
537.01	0\\
538.01	0\\
539.01	0\\
540.01	0\\
541.01	0\\
542.01	0\\
543.01	0\\
544.01	0\\
545.01	0\\
546.01	0\\
547.01	0\\
548.01	0\\
549.01	0\\
550.01	0\\
551.01	0\\
552.01	0\\
553.01	0\\
554.01	0\\
555.01	0\\
556.01	0\\
557.01	0\\
558.01	0\\
559.01	0\\
560.01	0\\
561.01	0\\
562.01	0\\
563.01	0\\
564.01	0\\
565.01	0\\
566.01	0\\
567.01	0\\
568.01	0\\
569.01	0\\
570.01	0\\
571.01	0\\
572.01	0\\
573.01	0\\
574.01	0\\
575.01	0\\
576.01	0\\
577.01	0\\
578.01	0\\
579.01	0\\
580.01	0\\
581.01	0\\
582.01	0\\
583.01	0\\
584.01	0\\
585.01	0\\
586.01	0\\
587.01	0\\
588.01	0\\
589.01	0\\
590.01	0\\
591.01	0\\
592.01	0\\
593.01	0\\
594.01	0\\
595.01	0\\
596.01	0\\
597.01	0.000481249826739166\\
598.01	0.00143832068185196\\
599.01	0.00385661257211625\\
599.02	0.00389414818835232\\
599.03	0.00393204146009087\\
599.04	0.0039702958414006\\
599.05	0.00400891481968846\\
599.06	0.00404790191602108\\
599.07	0.00408726068544946\\
599.08	0.0041269947173366\\
599.09	0.00416710763568851\\
599.1	0.00420760309948819\\
599.11	0.00424848480303298\\
599.12	0.00428975647627516\\
599.13	0.00433142188516568\\
599.14	0.00437348483200135\\
599.15	0.00441594915577537\\
599.16	0.00445881873253111\\
599.17	0.00450209747571946\\
599.18	0.00454578933655951\\
599.19	0.00458989830440275\\
599.2	0.00463442840710078\\
599.21	0.00467938371137652\\
599.22	0.00472476832319911\\
599.23	0.00477058638816225\\
599.24	0.00481684209186631\\
599.25	0.00486353966030413\\
599.26	0.00491068334955368\\
599.27	0.00495827744958437\\
599.28	0.0050063262915877\\
599.29	0.0050548342483729\\
599.3	0.00510380573476643\\
599.31	0.00515324520801537\\
599.32	0.00520315716819457\\
599.33	0.00525354615861774\\
599.34	0.00530441676625258\\
599.35	0.00535577362213972\\
599.36	0.00540762140181579\\
599.37	0.00545996482574053\\
599.38	0.00551280865972797\\
599.39	0.00556615771538181\\
599.4	0.00562001685053486\\
599.41	0.00567439096969276\\
599.42	0.00572928502448199\\
599.43	0.00578470401410215\\
599.44	0.00584065298578251\\
599.45	0.00589713703524303\\
599.46	0.00595416130715974\\
599.47	0.00601173099563461\\
599.48	0.0060698513446699\\
599.49	0.00612852764864714\\
599.5	0.00618776525281057\\
599.51	0.00624756955375527\\
599.52	0.00630794599992007\\
599.53	0.00636890009208502\\
599.54	0.00643043738387371\\
599.55	0.00649256348226046\\
599.56	0.00655528404808223\\
599.57	0.00661860479655561\\
599.58	0.00668253149779858\\
599.59	0.00674706997735747\\
599.6	0.0068122261167388\\
599.61	0.00687800585394632\\
599.62	0.00694441518402315\\
599.63	0.00701146015959913\\
599.64	0.00707914689144346\\
599.65	0.00714748154902253\\
599.66	0.00721647036106324\\
599.67	0.00728611961612158\\
599.68	0.00735643566315675\\
599.69	0.00742742491211078\\
599.7	0.00749909383449363\\
599.71	0.00757144896397399\\
599.72	0.0076444968969757\\
599.73	0.00771824429327989\\
599.74	0.00779269787663284\\
599.75	0.00786786443535982\\
599.76	0.0079437508229846\\
599.77	0.0080203639588551\\
599.78	0.00809771082877485\\
599.79	0.00817579848564071\\
599.8	0.00825463405008648\\
599.81	0.00833422471113284\\
599.82	0.00841457772684348\\
599.83	0.00849570042498746\\
599.84	0.00857760020370798\\
599.85	0.00866028453219752\\
599.86	0.00874376095137953\\
599.87	0.00882803707459653\\
599.88	0.00891312058830489\\
599.89	0.00899901925277624\\
599.9	0.00908574090280562\\
599.91	0.00917329344842635\\
599.92	0.0092616848756319\\
599.93	0.00935092324710449\\
599.94	0.00944101670295078\\
599.95	0.00953197346144465\\
599.96	0.00962380181977694\\
599.97	0.00971651015481255\\
599.98	0.0098101069238547\\
599.99	0.00990460066541651\\
600	0.01\\
};
\addplot [color=black!50!mycolor20,solid,forget plot]
  table[row sep=crcr]{%
0.01	0\\
1.01	0\\
2.01	0\\
3.01	0\\
4.01	0\\
5.01	0\\
6.01	0\\
7.01	0\\
8.01	0\\
9.01	0\\
10.01	0\\
11.01	0\\
12.01	0\\
13.01	0\\
14.01	0\\
15.01	0\\
16.01	0\\
17.01	0\\
18.01	0\\
19.01	0\\
20.01	0\\
21.01	0\\
22.01	0\\
23.01	0\\
24.01	0\\
25.01	0\\
26.01	0\\
27.01	0\\
28.01	0\\
29.01	0\\
30.01	0\\
31.01	0\\
32.01	0\\
33.01	0\\
34.01	0\\
35.01	0\\
36.01	0\\
37.01	0\\
38.01	0\\
39.01	0\\
40.01	0\\
41.01	0\\
42.01	0\\
43.01	0\\
44.01	0\\
45.01	0\\
46.01	0\\
47.01	0\\
48.01	0\\
49.01	0\\
50.01	0\\
51.01	0\\
52.01	0\\
53.01	0\\
54.01	0\\
55.01	0\\
56.01	0\\
57.01	0\\
58.01	0\\
59.01	0\\
60.01	0\\
61.01	0\\
62.01	0\\
63.01	0\\
64.01	0\\
65.01	0\\
66.01	0\\
67.01	0\\
68.01	0\\
69.01	0\\
70.01	0\\
71.01	0\\
72.01	0\\
73.01	0\\
74.01	0\\
75.01	0\\
76.01	0\\
77.01	0\\
78.01	0\\
79.01	0\\
80.01	0\\
81.01	0\\
82.01	0\\
83.01	0\\
84.01	0\\
85.01	0\\
86.01	0\\
87.01	0\\
88.01	0\\
89.01	0\\
90.01	0\\
91.01	0\\
92.01	0\\
93.01	0\\
94.01	0\\
95.01	0\\
96.01	0\\
97.01	0\\
98.01	0\\
99.01	0\\
100.01	0\\
101.01	0\\
102.01	0\\
103.01	0\\
104.01	0\\
105.01	0\\
106.01	0\\
107.01	0\\
108.01	0\\
109.01	0\\
110.01	0\\
111.01	0\\
112.01	0\\
113.01	0\\
114.01	0\\
115.01	0\\
116.01	0\\
117.01	0\\
118.01	0\\
119.01	0\\
120.01	0\\
121.01	0\\
122.01	0\\
123.01	0\\
124.01	0\\
125.01	0\\
126.01	0\\
127.01	0\\
128.01	0\\
129.01	0\\
130.01	0\\
131.01	0\\
132.01	0\\
133.01	0\\
134.01	0\\
135.01	0\\
136.01	0\\
137.01	0\\
138.01	0\\
139.01	0\\
140.01	0\\
141.01	0\\
142.01	0\\
143.01	0\\
144.01	0\\
145.01	0\\
146.01	0\\
147.01	0\\
148.01	0\\
149.01	0\\
150.01	0\\
151.01	0\\
152.01	0\\
153.01	0\\
154.01	0\\
155.01	0\\
156.01	0\\
157.01	0\\
158.01	0\\
159.01	0\\
160.01	0\\
161.01	0\\
162.01	0\\
163.01	0\\
164.01	0\\
165.01	0\\
166.01	0\\
167.01	0\\
168.01	0\\
169.01	0\\
170.01	0\\
171.01	0\\
172.01	0\\
173.01	0\\
174.01	0\\
175.01	0\\
176.01	0\\
177.01	0\\
178.01	0\\
179.01	0\\
180.01	0\\
181.01	0\\
182.01	0\\
183.01	0\\
184.01	0\\
185.01	0\\
186.01	0\\
187.01	0\\
188.01	0\\
189.01	0\\
190.01	0\\
191.01	0\\
192.01	0\\
193.01	0\\
194.01	0\\
195.01	0\\
196.01	0\\
197.01	0\\
198.01	0\\
199.01	0\\
200.01	0\\
201.01	0\\
202.01	0\\
203.01	0\\
204.01	0\\
205.01	0\\
206.01	0\\
207.01	0\\
208.01	0\\
209.01	0\\
210.01	0\\
211.01	0\\
212.01	0\\
213.01	0\\
214.01	0\\
215.01	0\\
216.01	0\\
217.01	0\\
218.01	0\\
219.01	0\\
220.01	0\\
221.01	0\\
222.01	0\\
223.01	0\\
224.01	0\\
225.01	0\\
226.01	0\\
227.01	0\\
228.01	0\\
229.01	0\\
230.01	0\\
231.01	0\\
232.01	0\\
233.01	0\\
234.01	0\\
235.01	0\\
236.01	0\\
237.01	0\\
238.01	0\\
239.01	0\\
240.01	0\\
241.01	0\\
242.01	0\\
243.01	0\\
244.01	0\\
245.01	0\\
246.01	0\\
247.01	0\\
248.01	0\\
249.01	0\\
250.01	0\\
251.01	0\\
252.01	0\\
253.01	0\\
254.01	0\\
255.01	0\\
256.01	0\\
257.01	0\\
258.01	0\\
259.01	0\\
260.01	0\\
261.01	0\\
262.01	0\\
263.01	0\\
264.01	0\\
265.01	0\\
266.01	0\\
267.01	0\\
268.01	0\\
269.01	0\\
270.01	0\\
271.01	0\\
272.01	0\\
273.01	0\\
274.01	0\\
275.01	0\\
276.01	0\\
277.01	0\\
278.01	0\\
279.01	0\\
280.01	0\\
281.01	0\\
282.01	0\\
283.01	0\\
284.01	0\\
285.01	0\\
286.01	0\\
287.01	0\\
288.01	0\\
289.01	0\\
290.01	0\\
291.01	0\\
292.01	0\\
293.01	0\\
294.01	0\\
295.01	0\\
296.01	0\\
297.01	0\\
298.01	0\\
299.01	0\\
300.01	0\\
301.01	0\\
302.01	0\\
303.01	0\\
304.01	0\\
305.01	0\\
306.01	0\\
307.01	0\\
308.01	0\\
309.01	0\\
310.01	0\\
311.01	0\\
312.01	0\\
313.01	0\\
314.01	0\\
315.01	0\\
316.01	0\\
317.01	0\\
318.01	0\\
319.01	0\\
320.01	0\\
321.01	0\\
322.01	0\\
323.01	0\\
324.01	0\\
325.01	0\\
326.01	0\\
327.01	0\\
328.01	0\\
329.01	0\\
330.01	0\\
331.01	0\\
332.01	0\\
333.01	0\\
334.01	0\\
335.01	0\\
336.01	0\\
337.01	0\\
338.01	0\\
339.01	0\\
340.01	0\\
341.01	0\\
342.01	0\\
343.01	0\\
344.01	0\\
345.01	0\\
346.01	0\\
347.01	0\\
348.01	0\\
349.01	0\\
350.01	0\\
351.01	0\\
352.01	0\\
353.01	0\\
354.01	0\\
355.01	0\\
356.01	0\\
357.01	0\\
358.01	0\\
359.01	0\\
360.01	0\\
361.01	0\\
362.01	0\\
363.01	0\\
364.01	0\\
365.01	0\\
366.01	0\\
367.01	0\\
368.01	0\\
369.01	0\\
370.01	0\\
371.01	0\\
372.01	0\\
373.01	0\\
374.01	0\\
375.01	0\\
376.01	0\\
377.01	0\\
378.01	0\\
379.01	0\\
380.01	0\\
381.01	0\\
382.01	0\\
383.01	0\\
384.01	0\\
385.01	0\\
386.01	0\\
387.01	0\\
388.01	0\\
389.01	0\\
390.01	0\\
391.01	0\\
392.01	0\\
393.01	0\\
394.01	0\\
395.01	0\\
396.01	0\\
397.01	0\\
398.01	0\\
399.01	0\\
400.01	0\\
401.01	0\\
402.01	0\\
403.01	0\\
404.01	0\\
405.01	0\\
406.01	0\\
407.01	0\\
408.01	0\\
409.01	0\\
410.01	0\\
411.01	0\\
412.01	0\\
413.01	0\\
414.01	0\\
415.01	0\\
416.01	0\\
417.01	0\\
418.01	0\\
419.01	0\\
420.01	0\\
421.01	0\\
422.01	0\\
423.01	0\\
424.01	0\\
425.01	0\\
426.01	0\\
427.01	0\\
428.01	0\\
429.01	0\\
430.01	0\\
431.01	0\\
432.01	0\\
433.01	0\\
434.01	0\\
435.01	0\\
436.01	0\\
437.01	0\\
438.01	0\\
439.01	0\\
440.01	0\\
441.01	0\\
442.01	0\\
443.01	0\\
444.01	0\\
445.01	0\\
446.01	0\\
447.01	0\\
448.01	0\\
449.01	0\\
450.01	0\\
451.01	0\\
452.01	0\\
453.01	0\\
454.01	0\\
455.01	0\\
456.01	0\\
457.01	0\\
458.01	0\\
459.01	0\\
460.01	0\\
461.01	0\\
462.01	0\\
463.01	0\\
464.01	0\\
465.01	0\\
466.01	0\\
467.01	0\\
468.01	0\\
469.01	0\\
470.01	0\\
471.01	0\\
472.01	0\\
473.01	0\\
474.01	0\\
475.01	0\\
476.01	0\\
477.01	0\\
478.01	0\\
479.01	0\\
480.01	0\\
481.01	0\\
482.01	0\\
483.01	0\\
484.01	0\\
485.01	0\\
486.01	0\\
487.01	0\\
488.01	0\\
489.01	0\\
490.01	0\\
491.01	0\\
492.01	0\\
493.01	0\\
494.01	0\\
495.01	0\\
496.01	0\\
497.01	0\\
498.01	0\\
499.01	0\\
500.01	0\\
501.01	0\\
502.01	0\\
503.01	0\\
504.01	0\\
505.01	0\\
506.01	0\\
507.01	0\\
508.01	0\\
509.01	0\\
510.01	0\\
511.01	0\\
512.01	0\\
513.01	0\\
514.01	0\\
515.01	0\\
516.01	0\\
517.01	0\\
518.01	0\\
519.01	0\\
520.01	0\\
521.01	0\\
522.01	0\\
523.01	0\\
524.01	0\\
525.01	0\\
526.01	0\\
527.01	0\\
528.01	0\\
529.01	0\\
530.01	0\\
531.01	0\\
532.01	0\\
533.01	0\\
534.01	0\\
535.01	0\\
536.01	0\\
537.01	0\\
538.01	0\\
539.01	0\\
540.01	0\\
541.01	0\\
542.01	0\\
543.01	0\\
544.01	0\\
545.01	0\\
546.01	0\\
547.01	0\\
548.01	0\\
549.01	0\\
550.01	0\\
551.01	0\\
552.01	0\\
553.01	0\\
554.01	0\\
555.01	0\\
556.01	0\\
557.01	0\\
558.01	0\\
559.01	0\\
560.01	0\\
561.01	0\\
562.01	0\\
563.01	0\\
564.01	0\\
565.01	0\\
566.01	0\\
567.01	0\\
568.01	0\\
569.01	0\\
570.01	0\\
571.01	0\\
572.01	0\\
573.01	0\\
574.01	0\\
575.01	0\\
576.01	0\\
577.01	0\\
578.01	0\\
579.01	0\\
580.01	0\\
581.01	0\\
582.01	0\\
583.01	0\\
584.01	0\\
585.01	0\\
586.01	0\\
587.01	0\\
588.01	0\\
589.01	0\\
590.01	0\\
591.01	0\\
592.01	0\\
593.01	0\\
594.01	0\\
595.01	0\\
596.01	0\\
597.01	0.000481304278336731\\
598.01	0.0014383206818519\\
599.01	0.00385661257211628\\
599.02	0.00389414818835235\\
599.03	0.00393204146009089\\
599.04	0.00397029584140063\\
599.05	0.00400891481968849\\
599.06	0.00404790191602111\\
599.07	0.00408726068544948\\
599.08	0.00412699471733663\\
599.09	0.00416710763568853\\
599.1	0.00420760309948821\\
599.11	0.00424848480303303\\
599.12	0.00428975647627521\\
599.13	0.00433142188516572\\
599.14	0.0043734848320014\\
599.15	0.00441594915577541\\
599.16	0.00445881873253114\\
599.17	0.00450209747571947\\
599.18	0.00454578933655951\\
599.19	0.00458989830440273\\
599.2	0.00463442840710075\\
599.21	0.00467938371137651\\
599.22	0.0047247683231991\\
599.23	0.00477058638816223\\
599.24	0.00481684209186631\\
599.25	0.00486353966030412\\
599.26	0.00491068334955366\\
599.27	0.00495827744958437\\
599.28	0.00500632629158769\\
599.29	0.00505483424837289\\
599.3	0.00510380573476643\\
599.31	0.00515324520801538\\
599.32	0.00520315716819458\\
599.33	0.00525354615861777\\
599.34	0.00530441676625261\\
599.35	0.00535577362213975\\
599.36	0.00540762140181582\\
599.37	0.00545996482574056\\
599.38	0.005512808659728\\
599.39	0.00556615771538184\\
599.4	0.00562001685053487\\
599.41	0.00567439096969276\\
599.42	0.005729285024482\\
599.43	0.00578470401410217\\
599.44	0.00584065298578254\\
599.45	0.00589713703524307\\
599.46	0.00595416130715977\\
599.47	0.00601173099563464\\
599.48	0.00606985134466993\\
599.49	0.00612852764864717\\
599.5	0.00618776525281059\\
599.51	0.00624756955375529\\
599.52	0.00630794599992009\\
599.53	0.00636890009208503\\
599.54	0.00643043738387374\\
599.55	0.00649256348226047\\
599.56	0.00655528404808225\\
599.57	0.00661860479655561\\
599.58	0.00668253149779858\\
599.59	0.00674706997735747\\
599.6	0.0068122261167388\\
599.61	0.00687800585394632\\
599.62	0.00694441518402315\\
599.63	0.00701146015959914\\
599.64	0.00707914689144346\\
599.65	0.00714748154902254\\
599.66	0.00721647036106324\\
599.67	0.00728611961612158\\
599.68	0.00735643566315676\\
599.69	0.00742742491211078\\
599.7	0.00749909383449363\\
599.71	0.007571448963974\\
599.72	0.00764449689697572\\
599.73	0.00771824429327989\\
599.74	0.00779269787663285\\
599.75	0.00786786443535983\\
599.76	0.00794375082298461\\
599.77	0.0080203639588551\\
599.78	0.00809771082877486\\
599.79	0.00817579848564072\\
599.8	0.00825463405008649\\
599.81	0.00833422471113286\\
599.82	0.00841457772684349\\
599.83	0.00849570042498747\\
599.84	0.00857760020370798\\
599.85	0.00866028453219753\\
599.86	0.00874376095137954\\
599.87	0.00882803707459654\\
599.88	0.0089131205883049\\
599.89	0.00899901925277626\\
599.9	0.00908574090280562\\
599.91	0.00917329344842636\\
599.92	0.00926168487563191\\
599.93	0.00935092324710449\\
599.94	0.00944101670295079\\
599.95	0.00953197346144465\\
599.96	0.00962380181977693\\
599.97	0.00971651015481255\\
599.98	0.0098101069238547\\
599.99	0.00990460066541651\\
600	0.01\\
};
\addplot [color=black!60!mycolor21,solid,forget plot]
  table[row sep=crcr]{%
0.01	0\\
1.01	0\\
2.01	0\\
3.01	0\\
4.01	0\\
5.01	0\\
6.01	0\\
7.01	0\\
8.01	0\\
9.01	0\\
10.01	0\\
11.01	0\\
12.01	0\\
13.01	0\\
14.01	0\\
15.01	0\\
16.01	0\\
17.01	0\\
18.01	0\\
19.01	0\\
20.01	0\\
21.01	0\\
22.01	0\\
23.01	0\\
24.01	0\\
25.01	0\\
26.01	0\\
27.01	0\\
28.01	0\\
29.01	0\\
30.01	0\\
31.01	0\\
32.01	0\\
33.01	0\\
34.01	0\\
35.01	0\\
36.01	0\\
37.01	0\\
38.01	0\\
39.01	0\\
40.01	0\\
41.01	0\\
42.01	0\\
43.01	0\\
44.01	0\\
45.01	0\\
46.01	0\\
47.01	0\\
48.01	0\\
49.01	0\\
50.01	0\\
51.01	0\\
52.01	0\\
53.01	0\\
54.01	0\\
55.01	0\\
56.01	0\\
57.01	0\\
58.01	0\\
59.01	0\\
60.01	0\\
61.01	0\\
62.01	0\\
63.01	0\\
64.01	0\\
65.01	0\\
66.01	0\\
67.01	0\\
68.01	0\\
69.01	0\\
70.01	0\\
71.01	0\\
72.01	0\\
73.01	0\\
74.01	0\\
75.01	0\\
76.01	0\\
77.01	0\\
78.01	0\\
79.01	0\\
80.01	0\\
81.01	0\\
82.01	0\\
83.01	0\\
84.01	0\\
85.01	0\\
86.01	0\\
87.01	0\\
88.01	0\\
89.01	0\\
90.01	0\\
91.01	0\\
92.01	0\\
93.01	0\\
94.01	0\\
95.01	0\\
96.01	0\\
97.01	0\\
98.01	0\\
99.01	0\\
100.01	0\\
101.01	0\\
102.01	0\\
103.01	0\\
104.01	0\\
105.01	0\\
106.01	0\\
107.01	0\\
108.01	0\\
109.01	0\\
110.01	0\\
111.01	0\\
112.01	0\\
113.01	0\\
114.01	0\\
115.01	0\\
116.01	0\\
117.01	0\\
118.01	0\\
119.01	0\\
120.01	0\\
121.01	0\\
122.01	0\\
123.01	0\\
124.01	0\\
125.01	0\\
126.01	0\\
127.01	0\\
128.01	0\\
129.01	0\\
130.01	0\\
131.01	0\\
132.01	0\\
133.01	0\\
134.01	0\\
135.01	0\\
136.01	0\\
137.01	0\\
138.01	0\\
139.01	0\\
140.01	0\\
141.01	0\\
142.01	0\\
143.01	0\\
144.01	0\\
145.01	0\\
146.01	0\\
147.01	0\\
148.01	0\\
149.01	0\\
150.01	0\\
151.01	0\\
152.01	0\\
153.01	0\\
154.01	0\\
155.01	0\\
156.01	0\\
157.01	0\\
158.01	0\\
159.01	0\\
160.01	0\\
161.01	0\\
162.01	0\\
163.01	0\\
164.01	0\\
165.01	0\\
166.01	0\\
167.01	0\\
168.01	0\\
169.01	0\\
170.01	0\\
171.01	0\\
172.01	0\\
173.01	0\\
174.01	0\\
175.01	0\\
176.01	0\\
177.01	0\\
178.01	0\\
179.01	0\\
180.01	0\\
181.01	0\\
182.01	0\\
183.01	0\\
184.01	0\\
185.01	0\\
186.01	0\\
187.01	0\\
188.01	0\\
189.01	0\\
190.01	0\\
191.01	0\\
192.01	0\\
193.01	0\\
194.01	0\\
195.01	0\\
196.01	0\\
197.01	0\\
198.01	0\\
199.01	0\\
200.01	0\\
201.01	0\\
202.01	0\\
203.01	0\\
204.01	0\\
205.01	0\\
206.01	0\\
207.01	0\\
208.01	0\\
209.01	0\\
210.01	0\\
211.01	0\\
212.01	0\\
213.01	0\\
214.01	0\\
215.01	0\\
216.01	0\\
217.01	0\\
218.01	0\\
219.01	0\\
220.01	0\\
221.01	0\\
222.01	0\\
223.01	0\\
224.01	0\\
225.01	0\\
226.01	0\\
227.01	0\\
228.01	0\\
229.01	0\\
230.01	0\\
231.01	0\\
232.01	0\\
233.01	0\\
234.01	0\\
235.01	0\\
236.01	0\\
237.01	0\\
238.01	0\\
239.01	0\\
240.01	0\\
241.01	0\\
242.01	0\\
243.01	0\\
244.01	0\\
245.01	0\\
246.01	0\\
247.01	0\\
248.01	0\\
249.01	0\\
250.01	0\\
251.01	0\\
252.01	0\\
253.01	0\\
254.01	0\\
255.01	0\\
256.01	0\\
257.01	0\\
258.01	0\\
259.01	0\\
260.01	0\\
261.01	0\\
262.01	0\\
263.01	0\\
264.01	0\\
265.01	0\\
266.01	0\\
267.01	0\\
268.01	0\\
269.01	0\\
270.01	0\\
271.01	0\\
272.01	0\\
273.01	0\\
274.01	0\\
275.01	0\\
276.01	0\\
277.01	0\\
278.01	0\\
279.01	0\\
280.01	0\\
281.01	0\\
282.01	0\\
283.01	0\\
284.01	0\\
285.01	0\\
286.01	0\\
287.01	0\\
288.01	0\\
289.01	0\\
290.01	0\\
291.01	0\\
292.01	0\\
293.01	0\\
294.01	0\\
295.01	0\\
296.01	0\\
297.01	0\\
298.01	0\\
299.01	0\\
300.01	0\\
301.01	0\\
302.01	0\\
303.01	0\\
304.01	0\\
305.01	0\\
306.01	0\\
307.01	0\\
308.01	0\\
309.01	0\\
310.01	0\\
311.01	0\\
312.01	0\\
313.01	0\\
314.01	0\\
315.01	0\\
316.01	0\\
317.01	0\\
318.01	0\\
319.01	0\\
320.01	0\\
321.01	0\\
322.01	0\\
323.01	0\\
324.01	0\\
325.01	0\\
326.01	0\\
327.01	0\\
328.01	0\\
329.01	0\\
330.01	0\\
331.01	0\\
332.01	0\\
333.01	0\\
334.01	0\\
335.01	0\\
336.01	0\\
337.01	0\\
338.01	0\\
339.01	0\\
340.01	0\\
341.01	0\\
342.01	0\\
343.01	0\\
344.01	0\\
345.01	0\\
346.01	0\\
347.01	0\\
348.01	0\\
349.01	0\\
350.01	0\\
351.01	0\\
352.01	0\\
353.01	0\\
354.01	0\\
355.01	0\\
356.01	0\\
357.01	0\\
358.01	0\\
359.01	0\\
360.01	0\\
361.01	0\\
362.01	0\\
363.01	0\\
364.01	0\\
365.01	0\\
366.01	0\\
367.01	0\\
368.01	0\\
369.01	0\\
370.01	0\\
371.01	0\\
372.01	0\\
373.01	0\\
374.01	0\\
375.01	0\\
376.01	0\\
377.01	0\\
378.01	0\\
379.01	0\\
380.01	0\\
381.01	0\\
382.01	0\\
383.01	0\\
384.01	0\\
385.01	0\\
386.01	0\\
387.01	0\\
388.01	0\\
389.01	0\\
390.01	0\\
391.01	0\\
392.01	0\\
393.01	0\\
394.01	0\\
395.01	0\\
396.01	0\\
397.01	0\\
398.01	0\\
399.01	0\\
400.01	0\\
401.01	0\\
402.01	0\\
403.01	0\\
404.01	0\\
405.01	0\\
406.01	0\\
407.01	0\\
408.01	0\\
409.01	0\\
410.01	0\\
411.01	0\\
412.01	0\\
413.01	0\\
414.01	0\\
415.01	0\\
416.01	0\\
417.01	0\\
418.01	0\\
419.01	0\\
420.01	0\\
421.01	0\\
422.01	0\\
423.01	0\\
424.01	0\\
425.01	0\\
426.01	0\\
427.01	0\\
428.01	0\\
429.01	0\\
430.01	0\\
431.01	0\\
432.01	0\\
433.01	0\\
434.01	0\\
435.01	0\\
436.01	0\\
437.01	0\\
438.01	0\\
439.01	0\\
440.01	0\\
441.01	0\\
442.01	0\\
443.01	0\\
444.01	0\\
445.01	0\\
446.01	0\\
447.01	0\\
448.01	0\\
449.01	0\\
450.01	0\\
451.01	0\\
452.01	0\\
453.01	0\\
454.01	0\\
455.01	0\\
456.01	0\\
457.01	0\\
458.01	0\\
459.01	0\\
460.01	0\\
461.01	0\\
462.01	0\\
463.01	0\\
464.01	0\\
465.01	0\\
466.01	0\\
467.01	0\\
468.01	0\\
469.01	0\\
470.01	0\\
471.01	0\\
472.01	0\\
473.01	0\\
474.01	0\\
475.01	0\\
476.01	0\\
477.01	0\\
478.01	0\\
479.01	0\\
480.01	0\\
481.01	0\\
482.01	0\\
483.01	0\\
484.01	0\\
485.01	0\\
486.01	0\\
487.01	0\\
488.01	0\\
489.01	0\\
490.01	0\\
491.01	0\\
492.01	0\\
493.01	0\\
494.01	0\\
495.01	0\\
496.01	0\\
497.01	0\\
498.01	0\\
499.01	0\\
500.01	0\\
501.01	0\\
502.01	0\\
503.01	0\\
504.01	0\\
505.01	0\\
506.01	0\\
507.01	0\\
508.01	0\\
509.01	0\\
510.01	0\\
511.01	0\\
512.01	0\\
513.01	0\\
514.01	0\\
515.01	0\\
516.01	0\\
517.01	0\\
518.01	0\\
519.01	0\\
520.01	0\\
521.01	0\\
522.01	0\\
523.01	0\\
524.01	0\\
525.01	0\\
526.01	0\\
527.01	0\\
528.01	0\\
529.01	0\\
530.01	0\\
531.01	0\\
532.01	0\\
533.01	0\\
534.01	0\\
535.01	0\\
536.01	0\\
537.01	0\\
538.01	0\\
539.01	0\\
540.01	0\\
541.01	0\\
542.01	0\\
543.01	0\\
544.01	0\\
545.01	0\\
546.01	0\\
547.01	0\\
548.01	0\\
549.01	0\\
550.01	0\\
551.01	0\\
552.01	0\\
553.01	0\\
554.01	0\\
555.01	0\\
556.01	0\\
557.01	0\\
558.01	0\\
559.01	0\\
560.01	0\\
561.01	0\\
562.01	0\\
563.01	0\\
564.01	0\\
565.01	0\\
566.01	0\\
567.01	0\\
568.01	0\\
569.01	0\\
570.01	0\\
571.01	0\\
572.01	0\\
573.01	0\\
574.01	0\\
575.01	0\\
576.01	0\\
577.01	0\\
578.01	0\\
579.01	0\\
580.01	0\\
581.01	0\\
582.01	0\\
583.01	0\\
584.01	0\\
585.01	0\\
586.01	0\\
587.01	0\\
588.01	0\\
589.01	0\\
590.01	0\\
591.01	0\\
592.01	0\\
593.01	0\\
594.01	0\\
595.01	0\\
596.01	0\\
597.01	0.000481322506726112\\
598.01	0.00143832068185176\\
599.01	0.00385661257211617\\
599.02	0.00389414818835224\\
599.03	0.00393204146009077\\
599.04	0.00397029584140052\\
599.05	0.00400891481968839\\
599.06	0.00404790191602103\\
599.07	0.0040872606854494\\
599.08	0.00412699471733655\\
599.09	0.00416710763568844\\
599.1	0.0042076030994881\\
599.11	0.00424848480303291\\
599.12	0.00428975647627511\\
599.13	0.00433142188516564\\
599.14	0.00437348483200133\\
599.15	0.00441594915577534\\
599.16	0.00445881873253108\\
599.17	0.00450209747571942\\
599.18	0.00454578933655947\\
599.19	0.00458989830440271\\
599.2	0.00463442840710072\\
599.21	0.00467938371137648\\
599.22	0.00472476832319908\\
599.23	0.00477058638816223\\
599.24	0.00481684209186631\\
599.25	0.00486353966030413\\
599.26	0.00491068334955366\\
599.27	0.00495827744958435\\
599.28	0.00500632629158769\\
599.29	0.00505483424837289\\
599.3	0.00510380573476642\\
599.31	0.00515324520801536\\
599.32	0.00520315716819454\\
599.33	0.00525354615861773\\
599.34	0.00530441676625257\\
599.35	0.0053557736221397\\
599.36	0.00540762140181576\\
599.37	0.0054599648257405\\
599.38	0.00551280865972796\\
599.39	0.00556615771538181\\
599.4	0.00562001685053484\\
599.41	0.00567439096969274\\
599.42	0.00572928502448199\\
599.43	0.00578470401410215\\
599.44	0.00584065298578251\\
599.45	0.00589713703524303\\
599.46	0.00595416130715973\\
599.47	0.00601173099563458\\
599.48	0.00606985134466989\\
599.49	0.00612852764864713\\
599.5	0.00618776525281055\\
599.51	0.00624756955375526\\
599.52	0.00630794599992006\\
599.53	0.006368900092085\\
599.54	0.00643043738387369\\
599.55	0.00649256348226043\\
599.56	0.00655528404808221\\
599.57	0.00661860479655559\\
599.58	0.00668253149779856\\
599.59	0.00674706997735745\\
599.6	0.00681222611673879\\
599.61	0.00687800585394631\\
599.62	0.00694441518402313\\
599.63	0.00701146015959912\\
599.64	0.00707914689144344\\
599.65	0.00714748154902252\\
599.66	0.00721647036106323\\
599.67	0.00728611961612157\\
599.68	0.00735643566315675\\
599.69	0.00742742491211077\\
599.7	0.00749909383449363\\
599.71	0.00757144896397399\\
599.72	0.0076444968969757\\
599.73	0.00771824429327989\\
599.74	0.00779269787663284\\
599.75	0.00786786443535981\\
599.76	0.0079437508229846\\
599.77	0.00802036395885509\\
599.78	0.00809771082877485\\
599.79	0.00817579848564071\\
599.8	0.00825463405008648\\
599.81	0.00833422471113284\\
599.82	0.00841457772684349\\
599.83	0.00849570042498746\\
599.84	0.00857760020370797\\
599.85	0.00866028453219752\\
599.86	0.00874376095137954\\
599.87	0.00882803707459654\\
599.88	0.0089131205883049\\
599.89	0.00899901925277625\\
599.9	0.00908574090280562\\
599.91	0.00917329344842635\\
599.92	0.0092616848756319\\
599.93	0.00935092324710449\\
599.94	0.00944101670295078\\
599.95	0.00953197346144465\\
599.96	0.00962380181977693\\
599.97	0.00971651015481255\\
599.98	0.0098101069238547\\
599.99	0.00990460066541651\\
600	0.01\\
};
\addplot [color=black!80!mycolor21,solid,forget plot]
  table[row sep=crcr]{%
0.01	0\\
1.01	0\\
2.01	0\\
3.01	0\\
4.01	0\\
5.01	0\\
6.01	0\\
7.01	0\\
8.01	0\\
9.01	0\\
10.01	0\\
11.01	0\\
12.01	0\\
13.01	0\\
14.01	0\\
15.01	0\\
16.01	0\\
17.01	0\\
18.01	0\\
19.01	0\\
20.01	0\\
21.01	0\\
22.01	0\\
23.01	0\\
24.01	0\\
25.01	0\\
26.01	0\\
27.01	0\\
28.01	0\\
29.01	0\\
30.01	0\\
31.01	0\\
32.01	0\\
33.01	0\\
34.01	0\\
35.01	0\\
36.01	0\\
37.01	0\\
38.01	0\\
39.01	0\\
40.01	0\\
41.01	0\\
42.01	0\\
43.01	0\\
44.01	0\\
45.01	0\\
46.01	0\\
47.01	0\\
48.01	0\\
49.01	0\\
50.01	0\\
51.01	0\\
52.01	0\\
53.01	0\\
54.01	0\\
55.01	0\\
56.01	0\\
57.01	0\\
58.01	0\\
59.01	0\\
60.01	0\\
61.01	0\\
62.01	0\\
63.01	0\\
64.01	0\\
65.01	0\\
66.01	0\\
67.01	0\\
68.01	0\\
69.01	0\\
70.01	0\\
71.01	0\\
72.01	0\\
73.01	0\\
74.01	0\\
75.01	0\\
76.01	0\\
77.01	0\\
78.01	0\\
79.01	0\\
80.01	0\\
81.01	0\\
82.01	0\\
83.01	0\\
84.01	0\\
85.01	0\\
86.01	0\\
87.01	0\\
88.01	0\\
89.01	0\\
90.01	0\\
91.01	0\\
92.01	0\\
93.01	0\\
94.01	0\\
95.01	0\\
96.01	0\\
97.01	0\\
98.01	0\\
99.01	0\\
100.01	0\\
101.01	0\\
102.01	0\\
103.01	0\\
104.01	0\\
105.01	0\\
106.01	0\\
107.01	0\\
108.01	0\\
109.01	0\\
110.01	0\\
111.01	0\\
112.01	0\\
113.01	0\\
114.01	0\\
115.01	0\\
116.01	0\\
117.01	0\\
118.01	0\\
119.01	0\\
120.01	0\\
121.01	0\\
122.01	0\\
123.01	0\\
124.01	0\\
125.01	0\\
126.01	0\\
127.01	0\\
128.01	0\\
129.01	0\\
130.01	0\\
131.01	0\\
132.01	0\\
133.01	0\\
134.01	0\\
135.01	0\\
136.01	0\\
137.01	0\\
138.01	0\\
139.01	0\\
140.01	0\\
141.01	0\\
142.01	0\\
143.01	0\\
144.01	0\\
145.01	0\\
146.01	0\\
147.01	0\\
148.01	0\\
149.01	0\\
150.01	0\\
151.01	0\\
152.01	0\\
153.01	0\\
154.01	0\\
155.01	0\\
156.01	0\\
157.01	0\\
158.01	0\\
159.01	0\\
160.01	0\\
161.01	0\\
162.01	0\\
163.01	0\\
164.01	0\\
165.01	0\\
166.01	0\\
167.01	0\\
168.01	0\\
169.01	0\\
170.01	0\\
171.01	0\\
172.01	0\\
173.01	0\\
174.01	0\\
175.01	0\\
176.01	0\\
177.01	0\\
178.01	0\\
179.01	0\\
180.01	0\\
181.01	0\\
182.01	0\\
183.01	0\\
184.01	0\\
185.01	0\\
186.01	0\\
187.01	0\\
188.01	0\\
189.01	0\\
190.01	0\\
191.01	0\\
192.01	0\\
193.01	0\\
194.01	0\\
195.01	0\\
196.01	0\\
197.01	0\\
198.01	0\\
199.01	0\\
200.01	0\\
201.01	0\\
202.01	0\\
203.01	0\\
204.01	0\\
205.01	0\\
206.01	0\\
207.01	0\\
208.01	0\\
209.01	0\\
210.01	0\\
211.01	0\\
212.01	0\\
213.01	0\\
214.01	0\\
215.01	0\\
216.01	0\\
217.01	0\\
218.01	0\\
219.01	0\\
220.01	0\\
221.01	0\\
222.01	0\\
223.01	0\\
224.01	0\\
225.01	0\\
226.01	0\\
227.01	0\\
228.01	0\\
229.01	0\\
230.01	0\\
231.01	0\\
232.01	0\\
233.01	0\\
234.01	0\\
235.01	0\\
236.01	0\\
237.01	0\\
238.01	0\\
239.01	0\\
240.01	0\\
241.01	0\\
242.01	0\\
243.01	0\\
244.01	0\\
245.01	0\\
246.01	0\\
247.01	0\\
248.01	0\\
249.01	0\\
250.01	0\\
251.01	0\\
252.01	0\\
253.01	0\\
254.01	0\\
255.01	0\\
256.01	0\\
257.01	0\\
258.01	0\\
259.01	0\\
260.01	0\\
261.01	0\\
262.01	0\\
263.01	0\\
264.01	0\\
265.01	0\\
266.01	0\\
267.01	0\\
268.01	0\\
269.01	0\\
270.01	0\\
271.01	0\\
272.01	0\\
273.01	0\\
274.01	0\\
275.01	0\\
276.01	0\\
277.01	0\\
278.01	0\\
279.01	0\\
280.01	0\\
281.01	0\\
282.01	0\\
283.01	0\\
284.01	0\\
285.01	0\\
286.01	0\\
287.01	0\\
288.01	0\\
289.01	0\\
290.01	0\\
291.01	0\\
292.01	0\\
293.01	0\\
294.01	0\\
295.01	0\\
296.01	0\\
297.01	0\\
298.01	0\\
299.01	0\\
300.01	0\\
301.01	0\\
302.01	0\\
303.01	0\\
304.01	0\\
305.01	0\\
306.01	0\\
307.01	0\\
308.01	0\\
309.01	0\\
310.01	0\\
311.01	0\\
312.01	0\\
313.01	0\\
314.01	0\\
315.01	0\\
316.01	0\\
317.01	0\\
318.01	0\\
319.01	0\\
320.01	0\\
321.01	0\\
322.01	0\\
323.01	0\\
324.01	0\\
325.01	0\\
326.01	0\\
327.01	0\\
328.01	0\\
329.01	0\\
330.01	0\\
331.01	0\\
332.01	0\\
333.01	0\\
334.01	0\\
335.01	0\\
336.01	0\\
337.01	0\\
338.01	0\\
339.01	0\\
340.01	0\\
341.01	0\\
342.01	0\\
343.01	0\\
344.01	0\\
345.01	0\\
346.01	0\\
347.01	0\\
348.01	0\\
349.01	0\\
350.01	0\\
351.01	0\\
352.01	0\\
353.01	0\\
354.01	0\\
355.01	0\\
356.01	0\\
357.01	0\\
358.01	0\\
359.01	0\\
360.01	0\\
361.01	0\\
362.01	0\\
363.01	0\\
364.01	0\\
365.01	0\\
366.01	0\\
367.01	0\\
368.01	0\\
369.01	0\\
370.01	0\\
371.01	0\\
372.01	0\\
373.01	0\\
374.01	0\\
375.01	0\\
376.01	0\\
377.01	0\\
378.01	0\\
379.01	0\\
380.01	0\\
381.01	0\\
382.01	0\\
383.01	0\\
384.01	0\\
385.01	0\\
386.01	0\\
387.01	0\\
388.01	0\\
389.01	0\\
390.01	0\\
391.01	0\\
392.01	0\\
393.01	0\\
394.01	0\\
395.01	0\\
396.01	0\\
397.01	0\\
398.01	0\\
399.01	0\\
400.01	0\\
401.01	0\\
402.01	0\\
403.01	0\\
404.01	0\\
405.01	0\\
406.01	0\\
407.01	0\\
408.01	0\\
409.01	0\\
410.01	0\\
411.01	0\\
412.01	0\\
413.01	0\\
414.01	0\\
415.01	0\\
416.01	0\\
417.01	0\\
418.01	0\\
419.01	0\\
420.01	0\\
421.01	0\\
422.01	0\\
423.01	0\\
424.01	0\\
425.01	0\\
426.01	0\\
427.01	0\\
428.01	0\\
429.01	0\\
430.01	0\\
431.01	0\\
432.01	0\\
433.01	0\\
434.01	0\\
435.01	0\\
436.01	0\\
437.01	0\\
438.01	0\\
439.01	0\\
440.01	0\\
441.01	0\\
442.01	0\\
443.01	0\\
444.01	0\\
445.01	0\\
446.01	0\\
447.01	0\\
448.01	0\\
449.01	0\\
450.01	0\\
451.01	0\\
452.01	0\\
453.01	0\\
454.01	0\\
455.01	0\\
456.01	0\\
457.01	0\\
458.01	0\\
459.01	0\\
460.01	0\\
461.01	0\\
462.01	0\\
463.01	0\\
464.01	0\\
465.01	0\\
466.01	0\\
467.01	0\\
468.01	0\\
469.01	0\\
470.01	0\\
471.01	0\\
472.01	0\\
473.01	0\\
474.01	0\\
475.01	0\\
476.01	0\\
477.01	0\\
478.01	0\\
479.01	0\\
480.01	0\\
481.01	0\\
482.01	0\\
483.01	0\\
484.01	0\\
485.01	0\\
486.01	0\\
487.01	0\\
488.01	0\\
489.01	0\\
490.01	0\\
491.01	0\\
492.01	0\\
493.01	0\\
494.01	0\\
495.01	0\\
496.01	0\\
497.01	0\\
498.01	0\\
499.01	0\\
500.01	0\\
501.01	0\\
502.01	0\\
503.01	0\\
504.01	0\\
505.01	0\\
506.01	0\\
507.01	0\\
508.01	0\\
509.01	0\\
510.01	0\\
511.01	0\\
512.01	0\\
513.01	0\\
514.01	0\\
515.01	0\\
516.01	0\\
517.01	0\\
518.01	0\\
519.01	0\\
520.01	0\\
521.01	0\\
522.01	0\\
523.01	0\\
524.01	0\\
525.01	0\\
526.01	0\\
527.01	0\\
528.01	0\\
529.01	0\\
530.01	0\\
531.01	0\\
532.01	0\\
533.01	0\\
534.01	0\\
535.01	0\\
536.01	0\\
537.01	0\\
538.01	0\\
539.01	0\\
540.01	0\\
541.01	0\\
542.01	0\\
543.01	0\\
544.01	0\\
545.01	0\\
546.01	0\\
547.01	0\\
548.01	0\\
549.01	0\\
550.01	0\\
551.01	0\\
552.01	0\\
553.01	0\\
554.01	0\\
555.01	0\\
556.01	0\\
557.01	0\\
558.01	0\\
559.01	0\\
560.01	0\\
561.01	0\\
562.01	0\\
563.01	0\\
564.01	0\\
565.01	0\\
566.01	0\\
567.01	0\\
568.01	0\\
569.01	0\\
570.01	0\\
571.01	0\\
572.01	0\\
573.01	0\\
574.01	0\\
575.01	0\\
576.01	0\\
577.01	0\\
578.01	0\\
579.01	0\\
580.01	0\\
581.01	0\\
582.01	0\\
583.01	0\\
584.01	0\\
585.01	0\\
586.01	0\\
587.01	0\\
588.01	0\\
589.01	0\\
590.01	0\\
591.01	0\\
592.01	0\\
593.01	0\\
594.01	0\\
595.01	0\\
596.01	0\\
597.01	0.000481328854172736\\
598.01	0.00143832068185187\\
599.01	0.00385661257211624\\
599.02	0.00389414818835231\\
599.03	0.00393204146009085\\
599.04	0.00397029584140059\\
599.05	0.00400891481968844\\
599.06	0.00404790191602107\\
599.07	0.00408726068544944\\
599.08	0.00412699471733657\\
599.09	0.00416710763568848\\
599.1	0.00420760309948816\\
599.11	0.00424848480303296\\
599.12	0.00428975647627514\\
599.13	0.00433142188516565\\
599.14	0.00437348483200133\\
599.15	0.00441594915577534\\
599.16	0.00445881873253108\\
599.17	0.00450209747571943\\
599.18	0.00454578933655947\\
599.19	0.00458989830440271\\
599.2	0.00463442840710074\\
599.21	0.00467938371137648\\
599.22	0.00472476832319907\\
599.23	0.0047705863881622\\
599.24	0.00481684209186627\\
599.25	0.00486353966030409\\
599.26	0.00491068334955364\\
599.27	0.00495827744958434\\
599.28	0.00500632629158768\\
599.29	0.00505483424837287\\
599.3	0.0051038057347664\\
599.31	0.00515324520801534\\
599.32	0.00520315716819454\\
599.33	0.00525354615861773\\
599.34	0.00530441676625258\\
599.35	0.00535577362213972\\
599.36	0.00540762140181578\\
599.37	0.0054599648257405\\
599.38	0.00551280865972795\\
599.39	0.00556615771538178\\
599.4	0.00562001685053483\\
599.41	0.00567439096969272\\
599.42	0.00572928502448196\\
599.43	0.00578470401410212\\
599.44	0.00584065298578249\\
599.45	0.005897137035243\\
599.46	0.00595416130715971\\
599.47	0.00601173099563457\\
599.48	0.00606985134466986\\
599.49	0.0061285276486471\\
599.5	0.00618776525281052\\
599.51	0.00624756955375523\\
599.52	0.00630794599992004\\
599.53	0.006368900092085\\
599.54	0.00643043738387369\\
599.55	0.00649256348226045\\
599.56	0.00655528404808221\\
599.57	0.00661860479655557\\
599.58	0.00668253149779854\\
599.59	0.00674706997735744\\
599.6	0.00681222611673878\\
599.61	0.0068780058539463\\
599.62	0.00694441518402312\\
599.63	0.00701146015959912\\
599.64	0.00707914689144344\\
599.65	0.00714748154902253\\
599.66	0.00721647036106324\\
599.67	0.00728611961612158\\
599.68	0.00735643566315676\\
599.69	0.00742742491211079\\
599.7	0.00749909383449364\\
599.71	0.007571448963974\\
599.72	0.00764449689697572\\
599.73	0.0077182442932799\\
599.74	0.00779269787663286\\
599.75	0.00786786443535983\\
599.76	0.00794375082298462\\
599.77	0.00802036395885511\\
599.78	0.00809771082877486\\
599.79	0.00817579848564072\\
599.8	0.00825463405008649\\
599.81	0.00833422471113286\\
599.82	0.0084145777268435\\
599.83	0.00849570042498747\\
599.84	0.00857760020370798\\
599.85	0.00866028453219752\\
599.86	0.00874376095137954\\
599.87	0.00882803707459654\\
599.88	0.0089131205883049\\
599.89	0.00899901925277625\\
599.9	0.00908574090280562\\
599.91	0.00917329344842635\\
599.92	0.0092616848756319\\
599.93	0.00935092324710449\\
599.94	0.00944101670295079\\
599.95	0.00953197346144465\\
599.96	0.00962380181977693\\
599.97	0.00971651015481255\\
599.98	0.0098101069238547\\
599.99	0.00990460066541651\\
600	0.01\\
};
\addplot [color=black,solid,forget plot]
  table[row sep=crcr]{%
0.01	0\\
1.01	0\\
2.01	0\\
3.01	0\\
4.01	0\\
5.01	0\\
6.01	0\\
7.01	0\\
8.01	0\\
9.01	0\\
10.01	0\\
11.01	0\\
12.01	0\\
13.01	0\\
14.01	0\\
15.01	0\\
16.01	0\\
17.01	0\\
18.01	0\\
19.01	0\\
20.01	0\\
21.01	0\\
22.01	0\\
23.01	0\\
24.01	0\\
25.01	0\\
26.01	0\\
27.01	0\\
28.01	0\\
29.01	0\\
30.01	0\\
31.01	0\\
32.01	0\\
33.01	0\\
34.01	0\\
35.01	0\\
36.01	0\\
37.01	0\\
38.01	0\\
39.01	0\\
40.01	0\\
41.01	0\\
42.01	0\\
43.01	0\\
44.01	0\\
45.01	0\\
46.01	0\\
47.01	0\\
48.01	0\\
49.01	0\\
50.01	0\\
51.01	0\\
52.01	0\\
53.01	0\\
54.01	0\\
55.01	0\\
56.01	0\\
57.01	0\\
58.01	0\\
59.01	0\\
60.01	0\\
61.01	0\\
62.01	0\\
63.01	0\\
64.01	0\\
65.01	0\\
66.01	0\\
67.01	0\\
68.01	0\\
69.01	0\\
70.01	0\\
71.01	0\\
72.01	0\\
73.01	0\\
74.01	0\\
75.01	0\\
76.01	0\\
77.01	0\\
78.01	0\\
79.01	0\\
80.01	0\\
81.01	0\\
82.01	0\\
83.01	0\\
84.01	0\\
85.01	0\\
86.01	0\\
87.01	0\\
88.01	0\\
89.01	0\\
90.01	0\\
91.01	0\\
92.01	0\\
93.01	0\\
94.01	0\\
95.01	0\\
96.01	0\\
97.01	0\\
98.01	0\\
99.01	0\\
100.01	0\\
101.01	0\\
102.01	0\\
103.01	0\\
104.01	0\\
105.01	0\\
106.01	0\\
107.01	0\\
108.01	0\\
109.01	0\\
110.01	0\\
111.01	0\\
112.01	0\\
113.01	0\\
114.01	0\\
115.01	0\\
116.01	0\\
117.01	0\\
118.01	0\\
119.01	0\\
120.01	0\\
121.01	0\\
122.01	0\\
123.01	0\\
124.01	0\\
125.01	0\\
126.01	0\\
127.01	0\\
128.01	0\\
129.01	0\\
130.01	0\\
131.01	0\\
132.01	0\\
133.01	0\\
134.01	0\\
135.01	0\\
136.01	0\\
137.01	0\\
138.01	0\\
139.01	0\\
140.01	0\\
141.01	0\\
142.01	0\\
143.01	0\\
144.01	0\\
145.01	0\\
146.01	0\\
147.01	0\\
148.01	0\\
149.01	0\\
150.01	0\\
151.01	0\\
152.01	0\\
153.01	0\\
154.01	0\\
155.01	0\\
156.01	0\\
157.01	0\\
158.01	0\\
159.01	0\\
160.01	0\\
161.01	0\\
162.01	0\\
163.01	0\\
164.01	0\\
165.01	0\\
166.01	0\\
167.01	0\\
168.01	0\\
169.01	0\\
170.01	0\\
171.01	0\\
172.01	0\\
173.01	0\\
174.01	0\\
175.01	0\\
176.01	0\\
177.01	0\\
178.01	0\\
179.01	0\\
180.01	0\\
181.01	0\\
182.01	0\\
183.01	0\\
184.01	0\\
185.01	0\\
186.01	0\\
187.01	0\\
188.01	0\\
189.01	0\\
190.01	0\\
191.01	0\\
192.01	0\\
193.01	0\\
194.01	0\\
195.01	0\\
196.01	0\\
197.01	0\\
198.01	0\\
199.01	0\\
200.01	0\\
201.01	0\\
202.01	0\\
203.01	0\\
204.01	0\\
205.01	0\\
206.01	0\\
207.01	0\\
208.01	0\\
209.01	0\\
210.01	0\\
211.01	0\\
212.01	0\\
213.01	0\\
214.01	0\\
215.01	0\\
216.01	0\\
217.01	0\\
218.01	0\\
219.01	0\\
220.01	0\\
221.01	0\\
222.01	0\\
223.01	0\\
224.01	0\\
225.01	0\\
226.01	0\\
227.01	0\\
228.01	0\\
229.01	0\\
230.01	0\\
231.01	0\\
232.01	0\\
233.01	0\\
234.01	0\\
235.01	0\\
236.01	0\\
237.01	0\\
238.01	0\\
239.01	0\\
240.01	0\\
241.01	0\\
242.01	0\\
243.01	0\\
244.01	0\\
245.01	0\\
246.01	0\\
247.01	0\\
248.01	0\\
249.01	0\\
250.01	0\\
251.01	0\\
252.01	0\\
253.01	0\\
254.01	0\\
255.01	0\\
256.01	0\\
257.01	0\\
258.01	0\\
259.01	0\\
260.01	0\\
261.01	0\\
262.01	0\\
263.01	0\\
264.01	0\\
265.01	0\\
266.01	0\\
267.01	0\\
268.01	0\\
269.01	0\\
270.01	0\\
271.01	0\\
272.01	0\\
273.01	0\\
274.01	0\\
275.01	0\\
276.01	0\\
277.01	0\\
278.01	0\\
279.01	0\\
280.01	0\\
281.01	0\\
282.01	0\\
283.01	0\\
284.01	0\\
285.01	0\\
286.01	0\\
287.01	0\\
288.01	0\\
289.01	0\\
290.01	0\\
291.01	0\\
292.01	0\\
293.01	0\\
294.01	0\\
295.01	0\\
296.01	0\\
297.01	0\\
298.01	0\\
299.01	0\\
300.01	0\\
301.01	0\\
302.01	0\\
303.01	0\\
304.01	0\\
305.01	0\\
306.01	0\\
307.01	0\\
308.01	0\\
309.01	0\\
310.01	0\\
311.01	0\\
312.01	0\\
313.01	0\\
314.01	0\\
315.01	0\\
316.01	0\\
317.01	0\\
318.01	0\\
319.01	0\\
320.01	0\\
321.01	0\\
322.01	0\\
323.01	0\\
324.01	0\\
325.01	0\\
326.01	0\\
327.01	0\\
328.01	0\\
329.01	0\\
330.01	0\\
331.01	0\\
332.01	0\\
333.01	0\\
334.01	0\\
335.01	0\\
336.01	0\\
337.01	0\\
338.01	0\\
339.01	0\\
340.01	0\\
341.01	0\\
342.01	0\\
343.01	0\\
344.01	0\\
345.01	0\\
346.01	0\\
347.01	0\\
348.01	0\\
349.01	0\\
350.01	0\\
351.01	0\\
352.01	0\\
353.01	0\\
354.01	0\\
355.01	0\\
356.01	0\\
357.01	0\\
358.01	0\\
359.01	0\\
360.01	0\\
361.01	0\\
362.01	0\\
363.01	0\\
364.01	0\\
365.01	0\\
366.01	0\\
367.01	0\\
368.01	0\\
369.01	0\\
370.01	0\\
371.01	0\\
372.01	0\\
373.01	0\\
374.01	0\\
375.01	0\\
376.01	0\\
377.01	0\\
378.01	0\\
379.01	0\\
380.01	0\\
381.01	0\\
382.01	0\\
383.01	0\\
384.01	0\\
385.01	0\\
386.01	0\\
387.01	0\\
388.01	0\\
389.01	0\\
390.01	0\\
391.01	0\\
392.01	0\\
393.01	0\\
394.01	0\\
395.01	0\\
396.01	0\\
397.01	0\\
398.01	0\\
399.01	0\\
400.01	0\\
401.01	0\\
402.01	0\\
403.01	0\\
404.01	0\\
405.01	0\\
406.01	0\\
407.01	0\\
408.01	0\\
409.01	0\\
410.01	0\\
411.01	0\\
412.01	0\\
413.01	0\\
414.01	0\\
415.01	0\\
416.01	0\\
417.01	0\\
418.01	0\\
419.01	0\\
420.01	0\\
421.01	0\\
422.01	0\\
423.01	0\\
424.01	0\\
425.01	0\\
426.01	0\\
427.01	0\\
428.01	0\\
429.01	0\\
430.01	0\\
431.01	0\\
432.01	0\\
433.01	0\\
434.01	0\\
435.01	0\\
436.01	0\\
437.01	0\\
438.01	0\\
439.01	0\\
440.01	0\\
441.01	0\\
442.01	0\\
443.01	0\\
444.01	0\\
445.01	0\\
446.01	0\\
447.01	0\\
448.01	0\\
449.01	0\\
450.01	0\\
451.01	0\\
452.01	0\\
453.01	0\\
454.01	0\\
455.01	0\\
456.01	0\\
457.01	0\\
458.01	0\\
459.01	0\\
460.01	0\\
461.01	0\\
462.01	0\\
463.01	0\\
464.01	0\\
465.01	0\\
466.01	0\\
467.01	0\\
468.01	0\\
469.01	0\\
470.01	0\\
471.01	0\\
472.01	0\\
473.01	0\\
474.01	0\\
475.01	0\\
476.01	0\\
477.01	0\\
478.01	0\\
479.01	0\\
480.01	0\\
481.01	0\\
482.01	0\\
483.01	0\\
484.01	0\\
485.01	0\\
486.01	0\\
487.01	0\\
488.01	0\\
489.01	0\\
490.01	0\\
491.01	0\\
492.01	0\\
493.01	0\\
494.01	0\\
495.01	0\\
496.01	0\\
497.01	0\\
498.01	0\\
499.01	0\\
500.01	0\\
501.01	0\\
502.01	0\\
503.01	0\\
504.01	0\\
505.01	0\\
506.01	0\\
507.01	0\\
508.01	0\\
509.01	0\\
510.01	0\\
511.01	0\\
512.01	0\\
513.01	0\\
514.01	0\\
515.01	0\\
516.01	0\\
517.01	0\\
518.01	0\\
519.01	0\\
520.01	0\\
521.01	0\\
522.01	0\\
523.01	0\\
524.01	0\\
525.01	0\\
526.01	0\\
527.01	0\\
528.01	0\\
529.01	0\\
530.01	0\\
531.01	0\\
532.01	0\\
533.01	0\\
534.01	0\\
535.01	0\\
536.01	0\\
537.01	0\\
538.01	0\\
539.01	0\\
540.01	0\\
541.01	0\\
542.01	0\\
543.01	0\\
544.01	0\\
545.01	0\\
546.01	0\\
547.01	0\\
548.01	0\\
549.01	0\\
550.01	0\\
551.01	0\\
552.01	0\\
553.01	0\\
554.01	0\\
555.01	0\\
556.01	0\\
557.01	0\\
558.01	0\\
559.01	0\\
560.01	0\\
561.01	0\\
562.01	0\\
563.01	0\\
564.01	0\\
565.01	0\\
566.01	0\\
567.01	0\\
568.01	0\\
569.01	0\\
570.01	0\\
571.01	0\\
572.01	0\\
573.01	0\\
574.01	0\\
575.01	0\\
576.01	0\\
577.01	0\\
578.01	0\\
579.01	0\\
580.01	0\\
581.01	0\\
582.01	0\\
583.01	0\\
584.01	0\\
585.01	0\\
586.01	0\\
587.01	0\\
588.01	0\\
589.01	0\\
590.01	0\\
591.01	0\\
592.01	0\\
593.01	0\\
594.01	0\\
595.01	0\\
596.01	0\\
597.01	0.00048133368794398\\
598.01	0.00143832068185187\\
599.01	0.00385661257211624\\
599.02	0.00389414818835231\\
599.03	0.00393204146009084\\
599.04	0.00397029584140057\\
599.05	0.00400891481968844\\
599.06	0.00404790191602107\\
599.07	0.00408726068544944\\
599.08	0.00412699471733659\\
599.09	0.00416710763568849\\
599.1	0.00420760309948817\\
599.11	0.00424848480303298\\
599.12	0.00428975647627516\\
599.13	0.00433142188516569\\
599.14	0.00437348483200138\\
599.15	0.00441594915577539\\
599.16	0.00445881873253112\\
599.17	0.00450209747571946\\
599.18	0.0045457893365595\\
599.19	0.00458989830440273\\
599.2	0.00463442840710075\\
599.21	0.00467938371137651\\
599.22	0.0047247683231991\\
599.23	0.00477058638816223\\
599.24	0.00481684209186631\\
599.25	0.00486353966030412\\
599.26	0.00491068334955366\\
599.27	0.00495827744958435\\
599.28	0.00500632629158768\\
599.29	0.00505483424837287\\
599.3	0.00510380573476642\\
599.31	0.00515324520801536\\
599.32	0.00520315716819456\\
599.33	0.00525354615861774\\
599.34	0.00530441676625258\\
599.35	0.00535577362213972\\
599.36	0.00540762140181579\\
599.37	0.00545996482574053\\
599.38	0.00551280865972797\\
599.39	0.00556615771538181\\
599.4	0.00562001685053484\\
599.41	0.00567439096969274\\
599.42	0.00572928502448199\\
599.43	0.00578470401410215\\
599.44	0.00584065298578253\\
599.45	0.00589713703524306\\
599.46	0.00595416130715976\\
599.47	0.00601173099563462\\
599.48	0.00606985134466992\\
599.49	0.00612852764864716\\
599.5	0.00618776525281058\\
599.51	0.00624756955375529\\
599.52	0.00630794599992009\\
599.53	0.00636890009208503\\
599.54	0.00643043738387374\\
599.55	0.00649256348226047\\
599.56	0.00655528404808225\\
599.57	0.00661860479655563\\
599.58	0.0066825314977986\\
599.59	0.00674706997735748\\
599.6	0.0068122261167388\\
599.61	0.00687800585394632\\
599.62	0.00694441518402315\\
599.63	0.00701146015959914\\
599.64	0.00707914689144346\\
599.65	0.00714748154902254\\
599.66	0.00721647036106325\\
599.67	0.00728611961612159\\
599.68	0.00735643566315677\\
599.69	0.00742742491211079\\
599.7	0.00749909383449364\\
599.71	0.00757144896397401\\
599.72	0.00764449689697572\\
599.73	0.00771824429327989\\
599.74	0.00779269787663285\\
599.75	0.00786786443535983\\
599.76	0.00794375082298461\\
599.77	0.0080203639588551\\
599.78	0.00809771082877486\\
599.79	0.00817579848564071\\
599.8	0.00825463405008648\\
599.81	0.00833422471113285\\
599.82	0.00841457772684349\\
599.83	0.00849570042498746\\
599.84	0.00857760020370797\\
599.85	0.00866028453219752\\
599.86	0.00874376095137953\\
599.87	0.00882803707459654\\
599.88	0.0089131205883049\\
599.89	0.00899901925277625\\
599.9	0.00908574090280562\\
599.91	0.00917329344842635\\
599.92	0.0092616848756319\\
599.93	0.00935092324710449\\
599.94	0.00944101670295078\\
599.95	0.00953197346144464\\
599.96	0.00962380181977693\\
599.97	0.00971651015481255\\
599.98	0.0098101069238547\\
599.99	0.00990460066541651\\
600	0.01\\
};
\end{axis}
\end{tikzpicture}%

%  \caption{Continuous Time}
%\end{subfigure}%
%\hfill%
%\begin{subfigure}{.45\linewidth}
%  \centering
%  \setlength\figureheight{\linewidth} 
%  \setlength\figurewidth{\linewidth}
%  \tikzsetnextfilename{dp_dscr_z1}
%  % This file was created by matlab2tikz.
%
%The latest updates can be retrieved from
%  http://www.mathworks.com/matlabcentral/fileexchange/22022-matlab2tikz-matlab2tikz
%where you can also make suggestions and rate matlab2tikz.
%
\definecolor{mycolor1}{rgb}{0.00000,1.00000,0.14286}%
\definecolor{mycolor2}{rgb}{0.00000,1.00000,0.28571}%
\definecolor{mycolor3}{rgb}{0.00000,1.00000,0.42857}%
\definecolor{mycolor4}{rgb}{0.00000,1.00000,0.57143}%
\definecolor{mycolor5}{rgb}{0.00000,1.00000,0.71429}%
\definecolor{mycolor6}{rgb}{0.00000,1.00000,0.85714}%
\definecolor{mycolor7}{rgb}{0.00000,1.00000,1.00000}%
\definecolor{mycolor8}{rgb}{0.00000,0.87500,1.00000}%
\definecolor{mycolor9}{rgb}{0.00000,0.62500,1.00000}%
\definecolor{mycolor10}{rgb}{0.12500,0.00000,1.00000}%
\definecolor{mycolor11}{rgb}{0.25000,0.00000,1.00000}%
\definecolor{mycolor12}{rgb}{0.37500,0.00000,1.00000}%
\definecolor{mycolor13}{rgb}{0.50000,0.00000,1.00000}%
\definecolor{mycolor14}{rgb}{0.62500,0.00000,1.00000}%
\definecolor{mycolor15}{rgb}{0.75000,0.00000,1.00000}%
\definecolor{mycolor16}{rgb}{0.87500,0.00000,1.00000}%
\definecolor{mycolor17}{rgb}{1.00000,0.00000,1.00000}%
\definecolor{mycolor18}{rgb}{1.00000,0.00000,0.87500}%
\definecolor{mycolor19}{rgb}{1.00000,0.00000,0.62500}%
\definecolor{mycolor20}{rgb}{0.85714,0.00000,0.00000}%
\definecolor{mycolor21}{rgb}{0.71429,0.00000,0.00000}%
%
\begin{tikzpicture}[trim axis left, trim axis right]

\begin{axis}[%
width=\figurewidth,
height=\figureheight,
at={(0\figurewidth,0\figureheight)},
scale only axis,
point meta min=0,
point meta max=1,
every outer x axis line/.append style={black},
every x tick label/.append style={font=\color{black}},
xmin=0,
xmax=600,
every outer y axis line/.append style={black},
every y tick label/.append style={font=\color{black}},
ymin=0,
ymax=0.014,
axis background/.style={fill=white},
axis x line*=bottom,
axis y line*=left,
]
\addplot [color=green,solid,forget plot]
  table[row sep=crcr]{%
1	0.0120168416619381\\
2	0.012016841040951\\
3	0.0120168404084185\\
4	0.0120168397641242\\
5	0.0120168391078476\\
6	0.0120168384393642\\
7	0.0120168377584453\\
8	0.0120168370648576\\
9	0.0120168363583636\\
10	0.0120168356387214\\
11	0.0120168349056843\\
12	0.0120168341590011\\
13	0.0120168333984157\\
14	0.0120168326236673\\
15	0.0120168318344903\\
16	0.0120168310306136\\
17	0.0120168302117615\\
18	0.0120168293776528\\
19	0.012016828528001\\
20	0.0120168276625143\\
21	0.0120168267808953\\
22	0.0120168258828409\\
23	0.0120168249680424\\
24	0.0120168240361851\\
25	0.0120168230869486\\
26	0.0120168221200061\\
27	0.0120168211350248\\
28	0.0120168201316658\\
29	0.0120168191095832\\
30	0.0120168180684252\\
31	0.0120168170078329\\
32	0.0120168159274408\\
33	0.0120168148268764\\
34	0.0120168137057601\\
35	0.0120168125637052\\
36	0.0120168114003176\\
37	0.0120168102151957\\
38	0.0120168090079304\\
39	0.0120168077781048\\
40	0.012016806525294\\
41	0.0120168052490652\\
42	0.0120168039489771\\
43	0.0120168026245804\\
44	0.012016801275417\\
45	0.0120167999010202\\
46	0.0120167985009144\\
47	0.0120167970746151\\
48	0.0120167956216284\\
49	0.0120167941414512\\
50	0.0120167926335707\\
51	0.0120167910974644\\
52	0.0120167895325999\\
53	0.0120167879384347\\
54	0.012016786314416\\
55	0.0120167846599805\\
56	0.012016782974554\\
57	0.0120167812575515\\
58	0.0120167795083771\\
59	0.0120167777264233\\
60	0.0120167759110711\\
61	0.0120167740616897\\
62	0.0120167721776364\\
63	0.0120167702582561\\
64	0.0120167683028816\\
65	0.0120167663108325\\
66	0.0120167642814158\\
67	0.0120167622139252\\
68	0.0120167601076409\\
69	0.0120167579618295\\
70	0.0120167557757436\\
71	0.0120167535486216\\
72	0.0120167512796871\\
73	0.0120167489681494\\
74	0.0120167466132024\\
75	0.0120167442140246\\
76	0.0120167417697792\\
77	0.0120167392796129\\
78	0.0120167367426567\\
79	0.0120167341580247\\
80	0.0120167315248141\\
81	0.0120167288421051\\
82	0.01201672610896\\
83	0.0120167233244237\\
84	0.0120167204875224\\
85	0.012016717597264\\
86	0.0120167146526374\\
87	0.0120167116526122\\
88	0.0120167085961383\\
89	0.0120167054821456\\
90	0.0120167023095436\\
91	0.0120166990772208\\
92	0.0120166957840448\\
93	0.0120166924288611\\
94	0.0120166890104937\\
95	0.0120166855277437\\
96	0.0120166819793895\\
97	0.012016678364186\\
98	0.0120166746808645\\
99	0.0120166709281319\\
100	0.0120166671046705\\
101	0.0120166632091374\\
102	0.0120166592401638\\
103	0.012016655196355\\
104	0.0120166510762895\\
105	0.0120166468785187\\
106	0.0120166426015661\\
107	0.0120166382439272\\
108	0.0120166338040684\\
109	0.0120166292804271\\
110	0.0120166246714106\\
111	0.0120166199753957\\
112	0.0120166151907282\\
113	0.0120166103157222\\
114	0.0120166053486595\\
115	0.012016600287789\\
116	0.0120165951313261\\
117	0.012016589877452\\
118	0.0120165845243131\\
119	0.0120165790700202\\
120	0.0120165735126478\\
121	0.0120165678502339\\
122	0.0120165620807784\\
123	0.0120165562022433\\
124	0.0120165502125512\\
125	0.012016544109585\\
126	0.0120165378911871\\
127	0.0120165315551584\\
128	0.0120165250992576\\
129	0.0120165185212005\\
130	0.0120165118186591\\
131	0.0120165049892606\\
132	0.0120164980305868\\
133	0.0120164909401731\\
134	0.0120164837155075\\
135	0.0120164763540298\\
136	0.0120164688531309\\
137	0.0120164612101511\\
138	0.0120164534223802\\
139	0.0120164454870556\\
140	0.0120164374013618\\
141	0.0120164291624291\\
142	0.0120164207673328\\
143	0.0120164122130919\\
144	0.0120164034966684\\
145	0.0120163946149655\\
146	0.0120163855648274\\
147	0.0120163763430371\\
148	0.0120163669463162\\
149	0.0120163573713232\\
150	0.0120163476146523\\
151	0.0120163376728321\\
152	0.0120163275423248\\
153	0.0120163172195243\\
154	0.0120163067007552\\
155	0.0120162959822716\\
156	0.0120162850602554\\
157	0.012016273930815\\
158	0.0120162625899842\\
159	0.0120162510337203\\
160	0.012016239257903\\
161	0.0120162272583326\\
162	0.0120162150307287\\
163	0.0120162025707284\\
164	0.0120161898738851\\
165	0.0120161769356666\\
166	0.0120161637514533\\
167	0.0120161503165369\\
168	0.0120161366261185\\
169	0.012016122675307\\
170	0.0120161084591168\\
171	0.0120160939724668\\
172	0.0120160792101779\\
173	0.0120160641669715\\
174	0.0120160488374673\\
175	0.0120160332161817\\
176	0.0120160172975255\\
177	0.0120160010758018\\
178	0.0120159845452046\\
179	0.0120159676998159\\
180	0.0120159505336038\\
181	0.0120159330404208\\
182	0.0120159152140008\\
183	0.0120158970479575\\
184	0.0120158785357816\\
185	0.012015859670839\\
186	0.0120158404463678\\
187	0.0120158208554763\\
188	0.0120158008911404\\
189	0.0120157805462009\\
190	0.0120157598133615\\
191	0.0120157386851853\\
192	0.0120157171540931\\
193	0.0120156952123596\\
194	0.0120156728521112\\
195	0.0120156500653218\\
196	0.0120156268438103\\
197	0.0120156031792407\\
198	0.0120155790631165\\
199	0.0120155544867777\\
200	0.0120155294413975\\
201	0.0120155039179793\\
202	0.0120154779073529\\
203	0.0120154514001719\\
204	0.0120154243869094\\
205	0.0120153968578554\\
206	0.0120153688031126\\
207	0.0120153402125928\\
208	0.0120153110760136\\
209	0.0120152813828943\\
210	0.012015251122552\\
211	0.0120152202840978\\
212	0.0120151888564325\\
213	0.012015156828243\\
214	0.0120151241879974\\
215	0.0120150909239412\\
216	0.0120150570240928\\
217	0.0120150224762386\\
218	0.0120149872679291\\
219	0.0120149513864735\\
220	0.0120149148189352\\
221	0.0120148775521267\\
222	0.0120148395726048\\
223	0.0120148008666651\\
224	0.0120147614203367\\
225	0.0120147212193768\\
226	0.0120146802492652\\
227	0.0120146384951982\\
228	0.0120145959420828\\
229	0.0120145525745308\\
230	0.012014508376852\\
231	0.0120144633330484\\
232	0.0120144174268068\\
233	0.0120143706414924\\
234	0.0120143229601414\\
235	0.0120142743654538\\
236	0.0120142248397856\\
237	0.0120141743651412\\
238	0.0120141229231651\\
239	0.012014070495133\\
240	0.012014017061944\\
241	0.0120139626041104\\
242	0.0120139071017492\\
243	0.0120138505345714\\
244	0.0120137928818725\\
245	0.0120137341225212\\
246	0.012013674234949\\
247	0.0120136131971381\\
248	0.0120135509866095\\
249	0.0120134875804106\\
250	0.0120134229551023\\
251	0.012013357086745\\
252	0.0120132899508847\\
253	0.0120132215225385\\
254	0.012013151776179\\
255	0.0120130806857185\\
256	0.012013008224493\\
257	0.0120129343652448\\
258	0.0120128590801049\\
259	0.0120127823405759\\
260	0.0120127041175125\\
261	0.0120126243811036\\
262	0.012012543100853\\
263	0.0120124602455603\\
264	0.012012375783302\\
265	0.0120122896814129\\
266	0.0120122019064676\\
267	0.0120121124242639\\
268	0.0120120211998068\\
269	0.0120119281972951\\
270	0.0120118333801124\\
271	0.0120117367108245\\
272	0.0120116381511926\\
273	0.0120115376622204\\
274	0.0120114352042807\\
275	0.0120113307374105\\
276	0.0120112242218546\\
277	0.0120111156183025\\
278	0.0120110048827092\\
279	0.0120108919663135\\
280	0.012010776825855\\
281	0.0120106594172084\\
282	0.0120105396953649\\
283	0.0120104176144154\\
284	0.0120102931275316\\
285	0.0120101661869474\\
286	0.0120100367439404\\
287	0.012009904748812\\
288	0.0120097701508679\\
289	0.0120096328983978\\
290	0.012009492938655\\
291	0.0120093502178347\\
292	0.0120092046810536\\
293	0.012009056272327\\
294	0.0120089049345467\\
295	0.0120087506094586\\
296	0.0120085932376387\\
297	0.0120084327584698\\
298	0.0120082691101167\\
299	0.0120081022295019\\
300	0.01200793205228\\
301	0.0120077585128114\\
302	0.0120075815441369\\
303	0.0120074010779494\\
304	0.0120072170445675\\
305	0.0120070293729065\\
306	0.0120068379904501\\
307	0.0120066428232209\\
308	0.0120064437957503\\
309	0.012006240831048\\
310	0.0120060338505704\\
311	0.0120058227741891\\
312	0.0120056075201576\\
313	0.0120053880050783\\
314	0.0120051641438682\\
315	0.012004935849724\\
316	0.0120047030340867\\
317	0.0120044656066045\\
318	0.0120042234750963\\
319	0.012003976545513\\
320	0.0120037247218989\\
321	0.0120034679063515\\
322	0.0120032059989814\\
323	0.0120029388978699\\
324	0.0120026664990267\\
325	0.0120023886963465\\
326	0.0120021053815641\\
327	0.0120018164442088\\
328	0.0120015217715578\\
329	0.0120012212485885\\
330	0.0120009147579291\\
331	0.0120006021798094\\
332	0.0120002833920086\\
333	0.0119999582698036\\
334	0.0119996266859145\\
335	0.0119992885104505\\
336	0.0119989436108528\\
337	0.0119985918518372\\
338	0.0119982330953351\\
339	0.0119978672004328\\
340	0.0119974940233094\\
341	0.0119971134171738\\
342	0.011996725232199\\
343	0.0119963293154559\\
344	0.0119959255108448\\
345	0.0119955136590257\\
346	0.0119950935973465\\
347	0.0119946651597696\\
348	0.0119942281767972\\
349	0.0119937824753938\\
350	0.0119933278789079\\
351	0.0119928642069913\\
352	0.0119923912755166\\
353	0.0119919088964929\\
354	0.0119914168779795\\
355	0.0119909150239977\\
356	0.0119904031344405\\
357	0.0119898810049811\\
358	0.0119893484269779\\
359	0.0119888051873798\\
360	0.0119882510686273\\
361	0.0119876858485533\\
362	0.0119871093002814\\
363	0.0119865211921224\\
364	0.0119859212874694\\
365	0.0119853093446905\\
366	0.0119846851170202\\
367	0.0119840483524494\\
368	0.0119833987936132\\
369	0.0119827361776774\\
370	0.011982060236224\\
371	0.0119813706951342\\
372	0.0119806672744704\\
373	0.0119799496883571\\
374	0.0119792176448592\\
375	0.0119784708458597\\
376	0.0119777089869351\\
377	0.0119769317572299\\
378	0.0119761388393308\\
379	0.0119753299091428\\
380	0.0119745046357735\\
381	0.0119736626814303\\
382	0.0119728037013296\\
383	0.0119719273435596\\
384	0.0119710332486776\\
385	0.0119701210486854\\
386	0.0119691903672383\\
387	0.0119682408210636\\
388	0.011967272018416\\
389	0.0119662835588548\\
390	0.0119652750330143\\
391	0.0119642460223643\\
392	0.0119631960989623\\
393	0.011962124825196\\
394	0.0119610317535156\\
395	0.0119599164261553\\
396	0.0119587783748436\\
397	0.0119576171205018\\
398	0.0119564321729289\\
399	0.0119552230304736\\
400	0.0119539891796917\\
401	0.0119527300949877\\
402	0.0119514452382403\\
403	0.0119501340584101\\
404	0.0119487959911288\\
405	0.011947430458268\\
406	0.0119460368674869\\
407	0.0119446146117566\\
408	0.0119431630688604\\
409	0.0119416816008673\\
410	0.0119401695535767\\
411	0.0119386262559338\\
412	0.0119370510194114\\
413	0.0119354431373573\\
414	0.0119338018843034\\
415	0.0119321265152356\\
416	0.011930416264819\\
417	0.011928670346578\\
418	0.0119268879520266\\
419	0.0119250682497446\\
420	0.0119232103843938\\
421	0.0119213134756592\\
422	0.0119193766171024\\
423	0.0119173988749376\\
424	0.0119153792868497\\
425	0.0119133168607032\\
426	0.0119112105730325\\
427	0.0119090593675221\\
428	0.0119068621533832\\
429	0.0119046178036216\\
430	0.0119023251531855\\
431	0.0118999829969861\\
432	0.0118975900877788\\
433	0.0118951451338961\\
434	0.011892646796818\\
435	0.0118900936885688\\
436	0.0118874843689248\\
437	0.0118848173424181\\
438	0.0118820910551198\\
439	0.0118793038911842\\
440	0.0118764541691341\\
441	0.0118735401378653\\
442	0.0118705599723482\\
443	0.0118675117689975\\
444	0.0118643935406848\\
445	0.011861203211361\\
446	0.0118579386102535\\
447	0.0118545974656052\\
448	0.0118511773979167\\
449	0.0118476759126506\\
450	0.0118440903923217\\
451	0.0118404180879478\\
452	0.0118366561097923\\
453	0.0118328014173301\\
454	0.0118288508083663\\
455	0.0118248009072259\\
456	0.0118206481519238\\
457	0.0118163887802169\\
458	0.0118120188144275\\
459	0.0118075340449137\\
460	0.0118029300120497\\
461	0.0117982019865639\\
462	0.0117933449480621\\
463	0.0117883535615499\\
464	0.011783222151751\\
465	0.011777944675019\\
466	0.0117725146886901\\
467	0.0117669253179064\\
468	0.0117611692205262\\
469	0.0117552385524497\\
470	0.0117491249405385\\
471	0.0117428194837972\\
472	0.0117363128403338\\
473	0.0117295955560796\\
474	0.0117226590459481\\
475	0.0117154944841619\\
476	0.0117081007677849\\
477	0.0117004399492539\\
478	0.0116924911639723\\
479	0.0116842371389706\\
480	0.0116756592807096\\
481	0.011666737420585\\
482	0.011657449643096\\
483	0.0116477720973909\\
484	0.0116376787859447\\
485	0.011627141326173\\
486	0.0116161286806223\\
487	0.0116046068509186\\
488	0.0115925385297241\\
489	0.0115798827037196\\
490	0.0115665941987047\\
491	0.0115526231543386\\
492	0.011537914408131\\
493	0.0115224067484298\\
494	0.0115060319430138\\
495	0.0114887133045863\\
496	0.0114703631593111\\
497	0.0114508775402919\\
498	0.0114301238292634\\
499	0.011407911351236\\
500	0.0113799225560012\\
501	0.0113348917331087\\
502	0.0112878339916798\\
503	0.0112381899578923\\
504	0.0111211579001085\\
505	0.0109822138907321\\
506	0.0108406034240144\\
507	0.0107131600808254\\
508	0.0106794477019033\\
509	0.0106506709685816\\
510	0.0106281808630873\\
511	0.0106106296017312\\
512	0.0105936254972074\\
513	0.01057693397818\\
514	0.0105601789387908\\
515	0.0105431859749238\\
516	0.0105258935013059\\
517	0.0105082776463844\\
518	0.0104903245000416\\
519	0.010471970931807\\
520	0.010453204501249\\
521	0.0104340146738591\\
522	0.0104143910900067\\
523	0.0103943234840548\\
524	0.0103738016024305\\
525	0.0103528150266084\\
526	0.0103313531309081\\
527	0.0103094050049212\\
528	0.0102869592772049\\
529	0.0102640037208883\\
530	0.0102404079739401\\
531	0.0102159793504059\\
532	0.0101911034024066\\
533	0.0101638539240202\\
534	0.0101367367488312\\
535	0.0101113248467772\\
536	0.0100866450965714\\
537	0.0100615241813055\\
538	0.010035862866537\\
539	0.0100096435719068\\
540	0.00998285235584197\\
541	0.00995547442818611\\
542	0.00992749540687192\\
543	0.0098989013482874\\
544	0.00986967913586614\\
545	0.00983981882940722\\
546	0.00980932051751061\\
547	0.00977821424882196\\
548	0.00974648723673855\\
549	0.00971403640883855\\
550	0.00968084144507715\\
551	0.00964606093351074\\
552	0.00960798190194539\\
553	0.00957049440978274\\
554	0.00953772512363284\\
555	0.00950433882860447\\
556	0.00947028681113497\\
557	0.00943555341890868\\
558	0.00940012264022913\\
559	0.00936397796935504\\
560	0.009327102375457\\
561	0.00928947824470587\\
562	0.00925108724655004\\
563	0.00921191000844813\\
564	0.00917192533096021\\
565	0.00893409944319607\\
566	0.00851312471020558\\
567	0.00845082836229412\\
568	0.00838749699508529\\
569	0.00832310828726401\\
570	0.00825763909266039\\
571	0.00819106539479088\\
572	0.00812336225919132\\
573	0.00805450378356905\\
574	0.00798446304588267\\
575	0.00791321205051155\\
576	0.0078407216727229\\
577	0.00776696160163765\\
578	0.00769190028172314\\
579	0.00761550485219266\\
580	0.00753774108175192\\
581	0.00745857329076528\\
582	0.007377964238543\\
583	0.00729587491554804\\
584	0.00721226408110737\\
585	0.00712708712854453\\
586	0.0070402931868614\\
587	0.00695181762108359\\
588	0.00686156256306144\\
589	0.00676934637110727\\
590	0.00667477256501157\\
591	0.00657713355197375\\
592	0.00647481662499992\\
593	0.00636364764633836\\
594	0.00623271716130669\\
595	0.00605340994148281\\
596	0.00575058001197164\\
597	0.00512683753504546\\
598	0.00366374385960312\\
599	0\\
600	0\\
};
\addplot [color=mycolor1,solid,forget plot]
  table[row sep=crcr]{%
1	0.012017858178407\\
2	0.0120178574763449\\
3	0.0120178567613835\\
4	0.012017856033284\\
5	0.0120178552918031\\
6	0.012017854536693\\
7	0.0120178537677013\\
8	0.0120178529845709\\
9	0.0120178521870398\\
10	0.0120178513748412\\
11	0.0120178505477033\\
12	0.0120178497053491\\
13	0.0120178488474967\\
14	0.0120178479738587\\
15	0.0120178470841425\\
16	0.0120178461780497\\
17	0.0120178452552768\\
18	0.0120178443155144\\
19	0.0120178433584471\\
20	0.012017842383754\\
21	0.0120178413911079\\
22	0.0120178403801756\\
23	0.0120178393506177\\
24	0.0120178383020884\\
25	0.0120178372342354\\
26	0.0120178361467\\
27	0.0120178350391165\\
28	0.0120178339111125\\
29	0.0120178327623086\\
30	0.0120178315923185\\
31	0.0120178304007482\\
32	0.0120178291871969\\
33	0.0120178279512558\\
34	0.0120178266925088\\
35	0.0120178254105318\\
36	0.0120178241048927\\
37	0.0120178227751515\\
38	0.0120178214208599\\
39	0.0120178200415611\\
40	0.0120178186367898\\
41	0.0120178172060721\\
42	0.0120178157489249\\
43	0.0120178142648563\\
44	0.0120178127533651\\
45	0.0120178112139408\\
46	0.0120178096460631\\
47	0.0120178080492022\\
48	0.0120178064228181\\
49	0.0120178047663608\\
50	0.01201780307927\\
51	0.0120178013609748\\
52	0.0120177996108937\\
53	0.0120177978284342\\
54	0.0120177960129926\\
55	0.012017794163954\\
56	0.0120177922806919\\
57	0.012017790362568\\
58	0.0120177884089321\\
59	0.0120177864191216\\
60	0.0120177843924616\\
61	0.0120177823282645\\
62	0.0120177802258298\\
63	0.0120177780844436\\
64	0.012017775903379\\
65	0.012017773681895\\
66	0.0120177714192369\\
67	0.0120177691146358\\
68	0.0120177667673082\\
69	0.012017764376456\\
70	0.0120177619412661\\
71	0.01201775946091\\
72	0.0120177569345436\\
73	0.0120177543613069\\
74	0.0120177517403239\\
75	0.0120177490707019\\
76	0.0120177463515315\\
77	0.0120177435818861\\
78	0.0120177407608217\\
79	0.0120177378873766\\
80	0.0120177349605707\\
81	0.0120177319794059\\
82	0.0120177289428647\\
83	0.0120177258499111\\
84	0.012017722699489\\
85	0.0120177194905227\\
86	0.0120177162219162\\
87	0.0120177128925527\\
88	0.0120177095012946\\
89	0.0120177060469826\\
90	0.0120177025284356\\
91	0.0120176989444504\\
92	0.0120176952938008\\
93	0.0120176915752379\\
94	0.0120176877874887\\
95	0.0120176839292568\\
96	0.0120176799992207\\
97	0.0120176759960346\\
98	0.0120176719183269\\
99	0.0120176677647001\\
100	0.0120176635337307\\
101	0.0120176592239678\\
102	0.0120176548339337\\
103	0.0120176503621223\\
104	0.0120176458069994\\
105	0.0120176411670017\\
106	0.0120176364405365\\
107	0.0120176316259808\\
108	0.0120176267216811\\
109	0.0120176217259528\\
110	0.0120176166370794\\
111	0.0120176114533118\\
112	0.0120176061728681\\
113	0.0120176007939327\\
114	0.0120175953146557\\
115	0.0120175897331522\\
116	0.0120175840475018\\
117	0.0120175782557476\\
118	0.0120175723558961\\
119	0.0120175663459158\\
120	0.0120175602237371\\
121	0.012017553987251\\
122	0.0120175476343089\\
123	0.0120175411627216\\
124	0.0120175345702583\\
125	0.0120175278546464\\
126	0.01201752101357\\
127	0.0120175140446697\\
128	0.0120175069455413\\
129	0.0120174997137356\\
130	0.0120174923467565\\
131	0.0120174848420612\\
132	0.0120174771970588\\
133	0.0120174694091092\\
134	0.0120174614755226\\
135	0.0120174533935583\\
136	0.012017445160424\\
137	0.0120174367732743\\
138	0.0120174282292103\\
139	0.0120174195252782\\
140	0.0120174106584685\\
141	0.0120174016257147\\
142	0.0120173924238924\\
143	0.0120173830498183\\
144	0.0120173735002487\\
145	0.0120173637718789\\
146	0.0120173538613416\\
147	0.012017343765206\\
148	0.0120173334799762\\
149	0.0120173230020907\\
150	0.0120173123279205\\
151	0.012017301453768\\
152	0.0120172903758658\\
153	0.0120172790903755\\
154	0.0120172675933862\\
155	0.012017255880913\\
156	0.0120172439488959\\
157	0.0120172317931981\\
158	0.0120172194096051\\
159	0.0120172067938225\\
160	0.0120171939414749\\
161	0.0120171808481045\\
162	0.0120171675091693\\
163	0.0120171539200418\\
164	0.0120171400760071\\
165	0.0120171259722613\\
166	0.0120171116039102\\
167	0.0120170969659673\\
168	0.012017082053352\\
169	0.0120170668608882\\
170	0.0120170513833022\\
171	0.0120170356152211\\
172	0.0120170195511709\\
173	0.0120170031855745\\
174	0.0120169865127502\\
175	0.0120169695269092\\
176	0.0120169522221543\\
177	0.0120169345924772\\
178	0.0120169166317571\\
179	0.0120168983337581\\
180	0.0120168796921278\\
181	0.0120168607003946\\
182	0.0120168413519656\\
183	0.0120168216401248\\
184	0.0120168015580306\\
185	0.0120167810987138\\
186	0.012016760255075\\
187	0.0120167390198826\\
188	0.0120167173857706\\
189	0.0120166953452363\\
190	0.0120166728906383\\
191	0.0120166500141946\\
192	0.0120166267079812\\
193	0.0120166029639306\\
194	0.012016578773829\\
195	0.012016554129308\\
196	0.0120165290218211\\
197	0.0120165034426105\\
198	0.0120164773828393\\
199	0.0120164508335118\\
200	0.0120164237854672\\
201	0.0120163962293771\\
202	0.0120163681557422\\
203	0.0120163395548901\\
204	0.0120163104169717\\
205	0.0120162807319585\\
206	0.01201625048964\\
207	0.01201621967962\\
208	0.0120161882913144\\
209	0.0120161563139471\\
210	0.0120161237365481\\
211	0.0120160905479491\\
212	0.0120160567367816\\
213	0.0120160222914728\\
214	0.0120159872002426\\
215	0.0120159514511009\\
216	0.0120159150318435\\
217	0.0120158779300494\\
218	0.0120158401330775\\
219	0.0120158016280629\\
220	0.0120157624019139\\
221	0.0120157224413085\\
222	0.012015681732691\\
223	0.0120156402622687\\
224	0.0120155980160086\\
225	0.0120155549796336\\
226	0.0120155111386194\\
227	0.0120154664781908\\
228	0.0120154209833188\\
229	0.0120153746387161\\
230	0.0120153274288347\\
231	0.0120152793378615\\
232	0.0120152303497153\\
233	0.0120151804480431\\
234	0.0120151296162164\\
235	0.0120150778373275\\
236	0.0120150250941859\\
237	0.0120149713693149\\
238	0.012014916644947\\
239	0.0120148609030205\\
240	0.0120148041251754\\
241	0.0120147462927493\\
242	0.0120146873867728\\
243	0.0120146273879651\\
244	0.0120145662767293\\
245	0.0120145040331471\\
246	0.0120144406369735\\
247	0.0120143760676312\\
248	0.0120143103042038\\
249	0.0120142433254288\\
250	0.0120141751096906\\
251	0.0120141056350112\\
252	0.0120140348790412\\
253	0.0120139628190491\\
254	0.0120138894319089\\
255	0.0120138146940869\\
256	0.0120137385816259\\
257	0.0120136610701276\\
258	0.0120135821347321\\
259	0.0120135017500945\\
260	0.0120134198903585\\
261	0.0120133365291248\\
262	0.0120132516394155\\
263	0.012013165193633\\
264	0.0120130771635115\\
265	0.012012987520062\\
266	0.0120128962335073\\
267	0.0120128032732078\\
268	0.0120127086075736\\
269	0.0120126122039643\\
270	0.0120125140285728\\
271	0.0120124140462938\\
272	0.0120123122205904\\
273	0.0120122085133997\\
274	0.0120121028852558\\
275	0.0120119952962719\\
276	0.0120118857103786\\
277	0.0120117741118422\\
278	0.0120116605686941\\
279	0.0120115450443472\\
280	0.0120114272814002\\
281	0.0120113072363395\\
282	0.0120111848648041\\
283	0.0120110601215693\\
284	0.0120109329605297\\
285	0.0120108033346824\\
286	0.0120106711961095\\
287	0.0120105364959599\\
288	0.0120103991844317\\
289	0.0120102592107534\\
290	0.0120101165231651\\
291	0.0120099710688996\\
292	0.0120098227941623\\
293	0.0120096716441117\\
294	0.012009517562839\\
295	0.012009360493347\\
296	0.0120092003775294\\
297	0.0120090371561492\\
298	0.0120088707688166\\
299	0.0120087011539667\\
300	0.0120085282488365\\
301	0.012008351989442\\
302	0.0120081723105543\\
303	0.0120079891456753\\
304	0.0120078024270133\\
305	0.0120076120854576\\
306	0.0120074180505531\\
307	0.0120072202504741\\
308	0.0120070186119977\\
309	0.0120068130604765\\
310	0.0120066035198109\\
311	0.012006389912421\\
312	0.0120061721592176\\
313	0.0120059501795729\\
314	0.0120057238912902\\
315	0.0120054932105738\\
316	0.0120052580519972\\
317	0.0120050183284715\\
318	0.0120047739512128\\
319	0.0120045248297087\\
320	0.0120042708716848\\
321	0.0120040119830695\\
322	0.0120037480679588\\
323	0.0120034790285802\\
324	0.0120032047652555\\
325	0.0120029251763628\\
326	0.0120026401582981\\
327	0.0120023496054355\\
328	0.0120020534100868\\
329	0.0120017514624597\\
330	0.0120014436506155\\
331	0.0120011298604256\\
332	0.0120008099755267\\
333	0.012000483877275\\
334	0.0120001514446992\\
335	0.0119998125544523\\
336	0.0119994670807622\\
337	0.0119991148953802\\
338	0.0119987558675294\\
339	0.0119983898638503\\
340	0.0119980167483451\\
341	0.0119976363823211\\
342	0.0119972486243314\\
343	0.0119968533301138\\
344	0.0119964503525286\\
345	0.0119960395414932\\
346	0.0119956207439149\\
347	0.011995193803622\\
348	0.0119947585612914\\
349	0.0119943148543741\\
350	0.0119938625170185\\
351	0.0119934013799895\\
352	0.0119929312705857\\
353	0.0119924520125531\\
354	0.0119919634259951\\
355	0.0119914653272795\\
356	0.0119909575289408\\
357	0.0119904398395802\\
358	0.0119899120637601\\
359	0.011989374001895\\
360	0.0119888254501385\\
361	0.0119882662002649\\
362	0.0119876960395472\\
363	0.0119871147506296\\
364	0.0119865221113965\\
365	0.0119859178948358\\
366	0.0119853018688982\\
367	0.0119846737963526\\
368	0.0119840334346364\\
369	0.011983380535703\\
370	0.0119827148458664\\
371	0.0119820361056425\\
372	0.0119813440495901\\
373	0.0119806384061511\\
374	0.0119799188974922\\
375	0.0119791852393511\\
376	0.0119784371408883\\
377	0.0119776743045518\\
378	0.0119768964259612\\
379	0.0119761031938335\\
380	0.0119752942899961\\
381	0.0119744693896065\\
382	0.0119736281618532\\
383	0.0119727702716925\\
384	0.0119718953831306\\
385	0.0119710031607734\\
386	0.0119700932390592\\
387	0.0119691652183333\\
388	0.0119682187313951\\
389	0.0119672534030775\\
390	0.0119662688500361\\
391	0.0119652646805306\\
392	0.0119642404941992\\
393	0.0119631958818241\\
394	0.0119621304250888\\
395	0.0119610436963249\\
396	0.0119599352582509\\
397	0.0119588046637009\\
398	0.0119576514553435\\
399	0.0119564751653896\\
400	0.0119552753152894\\
401	0.011954051415417\\
402	0.0119528029647425\\
403	0.0119515294504905\\
404	0.0119502303477852\\
405	0.0119489051192814\\
406	0.0119475532147793\\
407	0.0119461740708232\\
408	0.011944767110283\\
409	0.0119433317419166\\
410	0.0119418673599135\\
411	0.0119403733434165\\
412	0.0119388490560209\\
413	0.0119372938452492\\
414	0.0119357070420005\\
415	0.0119340879599738\\
416	0.0119324358950677\\
417	0.0119307501247405\\
418	0.0119290299073486\\
419	0.0119272744814688\\
420	0.0119254830652124\\
421	0.0119236548555195\\
422	0.0119217890273048\\
423	0.0119198847319841\\
424	0.0119179410947082\\
425	0.0119159572148796\\
426	0.0119139321671055\\
427	0.0119118649977743\\
428	0.0119097547237217\\
429	0.0119076003308079\\
430	0.0119054007724006\\
431	0.011903154967755\\
432	0.0119008618002837\\
433	0.0118985201157074\\
434	0.0118961287200778\\
435	0.0118936863776612\\
436	0.0118911918086727\\
437	0.0118886436868493\\
438	0.0118860406368477\\
439	0.0118833812314542\\
440	0.0118806639885902\\
441	0.0118778873680984\\
442	0.0118750497682902\\
443	0.0118721495222375\\
444	0.0118691848937853\\
445	0.0118661540732619\\
446	0.0118630551728496\\
447	0.0118598862215725\\
448	0.0118566451598622\\
449	0.0118533298337747\\
450	0.0118499379890396\\
451	0.011846467264141\\
452	0.0118429151830773\\
453	0.0118392791475423\\
454	0.0118355564284757\\
455	0.0118317441569271\\
456	0.0118278393141703\\
457	0.0118238387209992\\
458	0.0118197390261304\\
459	0.0118155366936276\\
460	0.0118112279892551\\
461	0.0118068089656577\\
462	0.0118022754462527\\
463	0.0117976230077073\\
464	0.011792846960863\\
465	0.0117879423299594\\
466	0.0117829038300097\\
467	0.0117777258422164\\
468	0.0117724023874159\\
469	0.0117669270978056\\
470	0.0117612931880113\\
471	0.0117554934284298\\
472	0.0117495201282214\\
473	0.0117433651452857\\
474	0.0117370199610621\\
475	0.011730476079478\\
476	0.0117237265715313\\
477	0.0117167590747527\\
478	0.0117095701305726\\
479	0.0117021427379843\\
480	0.0116944459668972\\
481	0.0116864641480356\\
482	0.0116781805388822\\
483	0.0116695770703071\\
484	0.0116606341770892\\
485	0.0116513306364963\\
486	0.0116416433940128\\
487	0.0116315473667719\\
488	0.0116210152216209\\
489	0.0116100171235214\\
490	0.0115985204492672\\
491	0.0115864894600743\\
492	0.0115738849241734\\
493	0.0115606636758725\\
494	0.011546778088046\\
495	0.0115321754158476\\
496	0.0115167969349481\\
497	0.0115005767553568\\
498	0.0114834402288102\\
499	0.0114653023812198\\
500	0.01144606909077\\
501	0.0114256296401743\\
502	0.0114038072068435\\
503	0.0113803357780602\\
504	0.0113408015744358\\
505	0.0112948584770326\\
506	0.0112461743411741\\
507	0.0111816724789651\\
508	0.0110411234212246\\
509	0.0108981549051104\\
510	0.0107529165082137\\
511	0.0106566838649621\\
512	0.0106223392082801\\
513	0.0105932999974131\\
514	0.0105709305585536\\
515	0.0105516640313174\\
516	0.0105334593702064\\
517	0.010515535853531\\
518	0.0104974977882872\\
519	0.0104792075890283\\
520	0.0104605664268592\\
521	0.0104415263594337\\
522	0.0104220693344566\\
523	0.0104021790567106\\
524	0.0103818410645504\\
525	0.0103610436927962\\
526	0.0103397754415959\\
527	0.0103180248780044\\
528	0.0102957805435841\\
529	0.0102730307843135\\
530	0.0102497636816225\\
531	0.010225965968753\\
532	0.0102016152128647\\
533	0.0101762803903668\\
534	0.0101504779428705\\
535	0.010122974133396\\
536	0.010094618998852\\
537	0.0100668160198629\\
538	0.0100411100980949\\
539	0.0100149580061252\\
540	0.00998828200125511\\
541	0.00996102821975749\\
542	0.00993317705082673\\
543	0.00990471400124363\\
544	0.00987562528901442\\
545	0.00984589955554137\\
546	0.00981553271219468\\
547	0.00978454227666683\\
548	0.00975296444485807\\
549	0.00972066649117154\\
550	0.00968762890003407\\
551	0.00965383086217386\\
552	0.0096186803283981\\
553	0.00958132318008507\\
554	0.00954096285964986\\
555	0.00950678442230384\\
556	0.00947277922344007\\
557	0.00943809556783066\\
558	0.00940271546732934\\
559	0.00936662234550874\\
560	0.00932979912718505\\
561	0.00929222817532882\\
562	0.00925389122203143\\
563	0.00921476922165203\\
564	0.00917484201313446\\
565	0.00911444624945732\\
566	0.00879955097520018\\
567	0.00845082836277874\\
568	0.00838749699510775\\
569	0.00832310828727373\\
570	0.00825763909266504\\
571	0.00819106539479303\\
572	0.00812336225919221\\
573	0.00805450378356939\\
574	0.00798446304588278\\
575	0.00791321205051157\\
576	0.00784072167272291\\
577	0.00776696160163765\\
578	0.00769190028172313\\
579	0.00761550485219263\\
580	0.0075377410817519\\
581	0.00745857329076528\\
582	0.007377964238543\\
583	0.00729587491554803\\
584	0.00721226408110737\\
585	0.00712708712854452\\
586	0.0070402931868614\\
587	0.00695181762108361\\
588	0.00686156256306144\\
589	0.00676934637110728\\
590	0.00667477256501156\\
591	0.00657713355197374\\
592	0.00647481662499992\\
593	0.00636364764633836\\
594	0.0062327171613067\\
595	0.00605340994148282\\
596	0.00575058001197165\\
597	0.00512683753504546\\
598	0.00366374385960312\\
599	0\\
600	0\\
};
\addplot [color=mycolor2,solid,forget plot]
  table[row sep=crcr]{%
1	0.0120212943025742\\
2	0.0120212933501643\\
3	0.0120212923806444\\
4	0.0120212913937051\\
5	0.0120212903890313\\
6	0.0120212893663023\\
7	0.0120212883251915\\
8	0.0120212872653662\\
9	0.0120212861864879\\
10	0.0120212850882118\\
11	0.0120212839701868\\
12	0.0120212828320555\\
13	0.0120212816734538\\
14	0.0120212804940111\\
15	0.0120212792933501\\
16	0.0120212780710864\\
17	0.0120212768268288\\
18	0.012021275560179\\
19	0.0120212742707312\\
20	0.0120212729580723\\
21	0.0120212716217818\\
22	0.0120212702614313\\
23	0.0120212688765848\\
24	0.0120212674667982\\
25	0.0120212660316192\\
26	0.0120212645705876\\
27	0.0120212630832343\\
28	0.0120212615690821\\
29	0.0120212600276449\\
30	0.0120212584584275\\
31	0.0120212568609261\\
32	0.0120212552346272\\
33	0.0120212535790083\\
34	0.0120212518935373\\
35	0.0120212501776721\\
36	0.012021248430861\\
37	0.012021246652542\\
38	0.012021244842143\\
39	0.0120212429990813\\
40	0.0120212411227636\\
41	0.0120212392125858\\
42	0.0120212372679326\\
43	0.0120212352881775\\
44	0.0120212332726826\\
45	0.0120212312207983\\
46	0.0120212291318631\\
47	0.0120212270052033\\
48	0.0120212248401331\\
49	0.0120212226359538\\
50	0.0120212203919543\\
51	0.01202121810741\\
52	0.0120212157815835\\
53	0.0120212134137237\\
54	0.0120212110030655\\
55	0.0120212085488302\\
56	0.0120212060502246\\
57	0.0120212035064409\\
58	0.0120212009166567\\
59	0.0120211982800342\\
60	0.0120211955957206\\
61	0.0120211928628473\\
62	0.0120211900805297\\
63	0.012021187247867\\
64	0.012021184363942\\
65	0.0120211814278205\\
66	0.0120211784385513\\
67	0.0120211753951657\\
68	0.0120211722966772\\
69	0.0120211691420811\\
70	0.0120211659303544\\
71	0.0120211626604553\\
72	0.0120211593313229\\
73	0.0120211559418766\\
74	0.0120211524910163\\
75	0.0120211489776213\\
76	0.0120211454005507\\
77	0.0120211417586423\\
78	0.0120211380507128\\
79	0.012021134275557\\
80	0.0120211304319477\\
81	0.0120211265186349\\
82	0.012021122534346\\
83	0.0120211184777846\\
84	0.0120211143476307\\
85	0.01202111014254\\
86	0.0120211058611434\\
87	0.0120211015020468\\
88	0.0120210970638303\\
89	0.0120210925450479\\
90	0.0120210879442271\\
91	0.0120210832598681\\
92	0.0120210784904439\\
93	0.0120210736343989\\
94	0.0120210686901493\\
95	0.012021063656082\\
96	0.0120210585305542\\
97	0.0120210533118928\\
98	0.0120210479983941\\
99	0.0120210425883228\\
100	0.012021037079912\\
101	0.0120210314713618\\
102	0.0120210257608396\\
103	0.0120210199464789\\
104	0.0120210140263788\\
105	0.0120210079986035\\
106	0.0120210018611814\\
107	0.0120209956121047\\
108	0.0120209892493287\\
109	0.0120209827707709\\
110	0.0120209761743106\\
111	0.0120209694577879\\
112	0.0120209626190033\\
113	0.0120209556557167\\
114	0.0120209485656467\\
115	0.01202094134647\\
116	0.0120209339958204\\
117	0.0120209265112883\\
118	0.0120209188904195\\
119	0.0120209111307147\\
120	0.0120209032296287\\
121	0.0120208951845693\\
122	0.0120208869928967\\
123	0.0120208786519222\\
124	0.012020870158908\\
125	0.0120208615110656\\
126	0.0120208527055553\\
127	0.012020843739485\\
128	0.0120208346099095\\
129	0.0120208253138293\\
130	0.0120208158481898\\
131	0.0120208062098801\\
132	0.012020796395732\\
133	0.012020786402519\\
134	0.0120207762269554\\
135	0.0120207658656948\\
136	0.0120207553153295\\
137	0.012020744572389\\
138	0.0120207336333388\\
139	0.0120207224945797\\
140	0.0120207111524461\\
141	0.0120206996032051\\
142	0.0120206878430552\\
143	0.0120206758681248\\
144	0.0120206636744714\\
145	0.01202065125808\\
146	0.0120206386148617\\
147	0.0120206257406526\\
148	0.0120206126312123\\
149	0.0120205992822223\\
150	0.0120205856892852\\
151	0.0120205718479226\\
152	0.0120205577535739\\
153	0.0120205434015948\\
154	0.0120205287872558\\
155	0.0120205139057405\\
156	0.0120204987521444\\
157	0.0120204833214727\\
158	0.0120204676086393\\
159	0.0120204516084647\\
160	0.0120204353156743\\
161	0.0120204187248969\\
162	0.0120204018306631\\
163	0.0120203846274028\\
164	0.0120203671094442\\
165	0.0120203492710114\\
166	0.0120203311062229\\
167	0.0120203126090892\\
168	0.0120202937735112\\
169	0.0120202745932783\\
170	0.0120202550620659\\
171	0.0120202351734337\\
172	0.0120202149208237\\
173	0.0120201942975576\\
174	0.0120201732968349\\
175	0.0120201519117307\\
176	0.0120201301351935\\
177	0.0120201079600423\\
178	0.0120200853789651\\
179	0.0120200623845158\\
180	0.0120200389691121\\
181	0.0120200151250329\\
182	0.0120199908444157\\
183	0.0120199661192541\\
184	0.0120199409413948\\
185	0.0120199153025354\\
186	0.0120198891942214\\
187	0.0120198626078434\\
188	0.012019835534635\\
189	0.0120198079656703\\
190	0.0120197798918636\\
191	0.0120197513039714\\
192	0.0120197221926004\\
193	0.0120196925482286\\
194	0.0120196623612438\\
195	0.012019631621997\\
196	0.0120196003207862\\
197	0.0120195684473535\\
198	0.0120195359881855\\
199	0.0120195029324462\\
200	0.0120194692692369\\
201	0.0120194349874642\\
202	0.0120194000758374\\
203	0.0120193645228647\\
204	0.0120193283168502\\
205	0.0120192914458908\\
206	0.0120192538978724\\
207	0.0120192156604671\\
208	0.0120191767211296\\
209	0.0120191370670938\\
210	0.0120190966853692\\
211	0.0120190555627381\\
212	0.0120190136857514\\
213	0.012018971040726\\
214	0.0120189276137403\\
215	0.0120188833906318\\
216	0.0120188383569931\\
217	0.0120187924981685\\
218	0.0120187457992508\\
219	0.0120186982450778\\
220	0.0120186498202288\\
221	0.0120186005090218\\
222	0.0120185502955096\\
223	0.0120184991634773\\
224	0.0120184470964386\\
225	0.0120183940776333\\
226	0.0120183400900242\\
227	0.0120182851162943\\
228	0.0120182291388446\\
229	0.0120181721397913\\
230	0.0120181141009638\\
231	0.0120180550039028\\
232	0.0120179948298586\\
233	0.0120179335597897\\
234	0.0120178711743619\\
235	0.0120178076539481\\
236	0.0120177429786273\\
237	0.0120176771281859\\
238	0.0120176100821185\\
239	0.0120175418196294\\
240	0.0120174723196353\\
241	0.0120174015607688\\
242	0.0120173295213823\\
243	0.012017256179554\\
244	0.0120171815130946\\
245	0.012017105499555\\
246	0.0120170281162367\\
247	0.0120169493402033\\
248	0.0120168691482939\\
249	0.0120167875171396\\
250	0.0120167044231821\\
251	0.0120166198426952\\
252	0.0120165337518103\\
253	0.0120164461265451\\
254	0.0120163569428371\\
255	0.0120162661765821\\
256	0.0120161738036776\\
257	0.0120160798000737\\
258	0.0120159841418299\\
259	0.0120158868051817\\
260	0.0120157877666151\\
261	0.0120156870029525\\
262	0.0120155844914507\\
263	0.0120154802099127\\
264	0.0120153741368156\\
265	0.0120152662514562\\
266	0.0120151565341183\\
267	0.0120150449662634\\
268	0.0120149315307494\\
269	0.0120148162120812\\
270	0.0120146989966962\\
271	0.0120145798732875\\
272	0.0120144588331593\\
273	0.0120143358705786\\
274	0.0120142109829975\\
275	0.0120140841706996\\
276	0.0120139554343474\\
277	0.0120138247651544\\
278	0.0120136921090047\\
279	0.0120135618480969\\
280	0.012013436982326\\
281	0.0120133097613101\\
282	0.012013180140791\\
283	0.0120130480756863\\
284	0.0120129135200739\\
285	0.0120127764271771\\
286	0.0120126367493488\\
287	0.0120124944380558\\
288	0.0120123494438624\\
289	0.0120122017164143\\
290	0.0120120512044217\\
291	0.0120118978556424\\
292	0.0120117416168646\\
293	0.0120115824338894\\
294	0.0120114202515127\\
295	0.0120112550135073\\
296	0.0120110866626045\\
297	0.0120109151404753\\
298	0.0120107403877112\\
299	0.012010562343805\\
300	0.0120103809471313\\
301	0.0120101961349261\\
302	0.0120100078432669\\
303	0.0120098160070518\\
304	0.0120096205599789\\
305	0.0120094214345247\\
306	0.0120092185619225\\
307	0.0120090118721409\\
308	0.0120088012938612\\
309	0.0120085867544547\\
310	0.0120083681799603\\
311	0.0120081454950609\\
312	0.0120079186230598\\
313	0.012007687485857\\
314	0.0120074520039248\\
315	0.0120072120962835\\
316	0.0120069676804761\\
317	0.0120067186725437\\
318	0.0120064649869991\\
319	0.0120062065368019\\
320	0.0120059432333316\\
321	0.0120056749863614\\
322	0.0120054017040314\\
323	0.0120051232928213\\
324	0.0120048396575232\\
325	0.0120045507012139\\
326	0.0120042563252265\\
327	0.0120039564291225\\
328	0.0120036509106629\\
329	0.0120033396657796\\
330	0.0120030225885459\\
331	0.0120026995711472\\
332	0.0120023705038515\\
333	0.0120020352749788\\
334	0.0120016937708717\\
335	0.012001345875864\\
336	0.0120009914722503\\
337	0.0120006304402548\\
338	0.012000262658\\
339	0.0119998880014746\\
340	0.0119995063445021\\
341	0.0119991175587074\\
342	0.0119987215134849\\
343	0.0119983180759645\\
344	0.0119979071109782\\
345	0.0119974884810256\\
346	0.0119970620462388\\
347	0.0119966276643467\\
348	0.0119961851906381\\
349	0.0119957344779242\\
350	0.0119952753764991\\
351	0.0119948077341001\\
352	0.0119943313958647\\
353	0.0119938462042868\\
354	0.0119933519991702\\
355	0.0119928486175787\\
356	0.0119923358937839\\
357	0.011991813659208\\
358	0.0119912817423628\\
359	0.011990739968783\\
360	0.0119901881609529\\
361	0.011989626138226\\
362	0.0119890537167359\\
363	0.0119884707092964\\
364	0.0119878769252905\\
365	0.011987272170545\\
366	0.0119866562471886\\
367	0.0119860289534915\\
368	0.0119853900836829\\
369	0.0119847394277427\\
370	0.0119840767711626\\
371	0.0119834018946737\\
372	0.0119827145739322\\
373	0.0119820145791579\\
374	0.0119813016747184\\
375	0.0119805756186474\\
376	0.0119798361620893\\
377	0.0119790830486569\\
378	0.0119783160136945\\
379	0.011977534783451\\
380	0.0119767390742242\\
381	0.0119759285917464\\
382	0.0119751030318458\\
383	0.0119742620862168\\
384	0.011973405467553\\
385	0.0119725330076814\\
386	0.0119716450343202\\
387	0.0119707413139761\\
388	0.0119698201724342\\
389	0.0119688812692837\\
390	0.0119679242569677\\
391	0.0119669487805691\\
392	0.0119659544775831\\
393	0.0119649409776798\\
394	0.0119639079024865\\
395	0.0119628548653285\\
396	0.0119617814709377\\
397	0.0119606873151641\\
398	0.0119595719846748\\
399	0.011958435056641\\
400	0.0119572760984121\\
401	0.0119560946671755\\
402	0.0119548903095999\\
403	0.0119536625614684\\
404	0.0119524109472724\\
405	0.0119511349797859\\
406	0.01194983415966\\
407	0.0119485079749752\\
408	0.0119471559007755\\
409	0.011945777398583\\
410	0.0119443719158949\\
411	0.0119429388856611\\
412	0.0119414777257444\\
413	0.0119399878383559\\
414	0.0119384686094617\\
415	0.0119369194081628\\
416	0.0119353395861165\\
417	0.0119337284771905\\
418	0.0119320853968156\\
419	0.0119304096416062\\
420	0.0119287004893926\\
421	0.0119269572002021\\
422	0.0119251790192836\\
423	0.0119233651830204\\
424	0.0119215149199847\\
425	0.0119196273795223\\
426	0.0119177016602142\\
427	0.0119157369004203\\
428	0.0119137322123761\\
429	0.0119116866809773\\
430	0.011909599362491\\
431	0.0119074692831836\\
432	0.0119052954378612\\
433	0.0119030767883182\\
434	0.0119008122616852\\
435	0.0118985007486698\\
436	0.0118961411016814\\
437	0.0118937321328313\\
438	0.0118912726117987\\
439	0.0118887612635533\\
440	0.0118861967659203\\
441	0.0118835777469791\\
442	0.0118809027822839\\
443	0.0118781703918965\\
444	0.0118753790372304\\
445	0.011872527117709\\
446	0.0118696129672333\\
447	0.011866634850359\\
448	0.0118635909577247\\
449	0.0118604793995762\\
450	0.0118572981979823\\
451	0.0118540452906383\\
452	0.0118507185195902\\
453	0.0118473156249637\\
454	0.0118438342382083\\
455	0.0118402718748141\\
456	0.0118366259264537\\
457	0.0118328936525002\\
458	0.0118290721708644\\
459	0.0118251584480907\\
460	0.0118211492886442\\
461	0.0118170413233156\\
462	0.0118128309966606\\
463	0.0118085145533854\\
464	0.0118040880235743\\
465	0.0117995472066425\\
466	0.0117948876538708\\
467	0.0117901046493747\\
468	0.0117851931894531\\
469	0.0117801479605565\\
470	0.0117749633151626\\
471	0.0117696332460602\\
472	0.0117641513599051\\
473	0.0117585108519717\\
474	0.0117527044874203\\
475	0.0117467245944295\\
476	0.0117405630496169\\
477	0.0117342116691335\\
478	0.0117276628023863\\
479	0.0117209063970582\\
480	0.0117139320718934\\
481	0.0117067434107674\\
482	0.0116993046617714\\
483	0.0116915962327533\\
484	0.0116836024774085\\
485	0.0116753067087091\\
486	0.0116666908925494\\
487	0.011657735512381\\
488	0.0116484194114408\\
489	0.0116387196211643\\
490	0.0116286111676767\\
491	0.0116180668531775\\
492	0.0116070570078661\\
493	0.0115955492072172\\
494	0.0115835079480167\\
495	0.011570894274926\\
496	0.0115576653481582\\
497	0.0115437739440877\\
498	0.0115291678845862\\
499	0.0115137893699679\\
500	0.0114975739586083\\
501	0.0114804489455737\\
502	0.0114623322038971\\
503	0.0114431313060926\\
504	0.0114227500657037\\
505	0.0114010442641983\\
506	0.0113778066135288\\
507	0.0113500196181607\\
508	0.0113041321689353\\
509	0.0112558742749655\\
510	0.0112045326765312\\
511	0.0111097332188489\\
512	0.0109651003693197\\
513	0.0108179424333279\\
514	0.0106685963528404\\
515	0.0105982671538382\\
516	0.0105628667937857\\
517	0.0105330914377936\\
518	0.0105104258309053\\
519	0.0104892880612886\\
520	0.0104696410556042\\
521	0.0104503322380931\\
522	0.010430778893258\\
523	0.010410924496438\\
524	0.0103907109350068\\
525	0.010370065689303\\
526	0.0103489689272352\\
527	0.0103274027573535\\
528	0.0103053510070502\\
529	0.0102828000380246\\
530	0.0102597370891189\\
531	0.0102361494788378\\
532	0.0102120245456834\\
533	0.0101873495147147\\
534	0.0101621075080765\\
535	0.0101360149873251\\
536	0.0101092330629758\\
537	0.010081606928069\\
538	0.0100519479721389\\
539	0.0100223711098094\\
540	0.00999451555122586\\
541	0.00996726818527192\\
542	0.00993953621963588\\
543	0.00991120468542803\\
544	0.00988225526518685\\
545	0.00985267250871387\\
546	0.00982244948141088\\
547	0.00979159559488473\\
548	0.00976016293484706\\
549	0.00972803578235161\\
550	0.00969517692126071\\
551	0.00966156706231893\\
552	0.00962718609996824\\
553	0.00959174752024606\\
554	0.00955495298772372\\
555	0.00951380796051269\\
556	0.00947563820042506\\
557	0.00944096945327738\\
558	0.00940564223266134\\
559	0.00936960350752038\\
560	0.00933283573091086\\
561	0.00929532122306513\\
562	0.00925704172702615\\
563	0.009217978314658\\
564	0.00917811122032248\\
565	0.00913741944841397\\
566	0.00906187965960978\\
567	0.00869695649895891\\
568	0.00838749700177051\\
569	0.00832310828745102\\
570	0.00825763909273647\\
571	0.00819106539482789\\
572	0.00812336225920869\\
573	0.00805450378357674\\
574	0.00798446304588583\\
575	0.00791321205051272\\
576	0.00784072167272329\\
577	0.00776696160163777\\
578	0.00769190028172315\\
579	0.00761550485219264\\
580	0.00753774108175188\\
581	0.00745857329076526\\
582	0.00737796423854298\\
583	0.00729587491554803\\
584	0.00721226408110736\\
585	0.00712708712854453\\
586	0.00704029318686139\\
587	0.00695181762108358\\
588	0.00686156256306142\\
589	0.00676934637110724\\
590	0.00667477256501155\\
591	0.00657713355197373\\
592	0.00647481662499991\\
593	0.00636364764633834\\
594	0.00623271716130668\\
595	0.0060534099414828\\
596	0.00575058001197164\\
597	0.00512683753504545\\
598	0.00366374385960312\\
599	0\\
600	0\\
};
\addplot [color=mycolor3,solid,forget plot]
  table[row sep=crcr]{%
1	0.0120294184481938\\
2	0.012029417555819\\
3	0.0120294166474337\\
4	0.0120294157227489\\
5	0.0120294147814703\\
6	0.0120294138232981\\
7	0.0120294128479273\\
8	0.0120294118550471\\
9	0.0120294108443413\\
10	0.0120294098154878\\
11	0.0120294087681586\\
12	0.0120294077020199\\
13	0.0120294066167317\\
14	0.0120294055119477\\
15	0.0120294043873156\\
16	0.0120294032424764\\
17	0.0120294020770648\\
18	0.0120294008907087\\
19	0.0120293996830292\\
20	0.0120293984536406\\
21	0.0120293972021501\\
22	0.0120293959281578\\
23	0.0120293946312565\\
24	0.0120293933110315\\
25	0.0120293919670607\\
26	0.0120293905989142\\
27	0.0120293892061542\\
28	0.0120293877883351\\
29	0.012029386345003\\
30	0.0120293848756958\\
31	0.0120293833799429\\
32	0.0120293818572654\\
33	0.0120293803071752\\
34	0.0120293787291756\\
35	0.0120293771227608\\
36	0.0120293754874158\\
37	0.012029373822616\\
38	0.0120293721278274\\
39	0.0120293704025062\\
40	0.0120293686460988\\
41	0.0120293668580412\\
42	0.0120293650377593\\
43	0.0120293631846684\\
44	0.0120293612981734\\
45	0.0120293593776679\\
46	0.0120293574225348\\
47	0.0120293554321453\\
48	0.0120293534058596\\
49	0.0120293513430258\\
50	0.0120293492429802\\
51	0.0120293471050471\\
52	0.0120293449285382\\
53	0.0120293427127528\\
54	0.0120293404569773\\
55	0.0120293381604851\\
56	0.0120293358225361\\
57	0.0120293334423769\\
58	0.0120293310192402\\
59	0.0120293285523446\\
60	0.0120293260408945\\
61	0.0120293234840795\\
62	0.0120293208810748\\
63	0.0120293182310399\\
64	0.0120293155331194\\
65	0.012029312786442\\
66	0.0120293099901204\\
67	0.0120293071432511\\
68	0.0120293042449139\\
69	0.0120293012941721\\
70	0.0120292982900713\\
71	0.01202929523164\\
72	0.0120292921178886\\
73	0.0120292889478096\\
74	0.0120292857203768\\
75	0.0120292824345452\\
76	0.0120292790892507\\
77	0.0120292756834096\\
78	0.0120292722159184\\
79	0.0120292686856531\\
80	0.0120292650914693\\
81	0.0120292614322015\\
82	0.0120292577066627\\
83	0.0120292539136441\\
84	0.0120292500519149\\
85	0.0120292461202213\\
86	0.0120292421172868\\
87	0.0120292380418112\\
88	0.0120292338924706\\
89	0.0120292296679165\\
90	0.0120292253667758\\
91	0.0120292209876499\\
92	0.0120292165291149\\
93	0.0120292119897202\\
94	0.0120292073679888\\
95	0.0120292026624165\\
96	0.0120291978714712\\
97	0.0120291929935929\\
98	0.0120291880271927\\
99	0.0120291829706523\\
100	0.0120291778223239\\
101	0.0120291725805291\\
102	0.0120291672435587\\
103	0.012029161809672\\
104	0.0120291562770962\\
105	0.0120291506440258\\
106	0.0120291449086221\\
107	0.0120291390690125\\
108	0.0120291331232897\\
109	0.0120291270695115\\
110	0.0120291209056998\\
111	0.0120291146298399\\
112	0.01202910823988\\
113	0.0120291017337306\\
114	0.0120290951092636\\
115	0.0120290883643115\\
116	0.0120290814966669\\
117	0.0120290745040818\\
118	0.0120290673842665\\
119	0.0120290601348893\\
120	0.0120290527535754\\
121	0.0120290452379059\\
122	0.0120290375854178\\
123	0.0120290297936021\\
124	0.0120290218599039\\
125	0.012029013781721\\
126	0.0120290055564032\\
127	0.0120289971812513\\
128	0.0120289886535164\\
129	0.0120289799703989\\
130	0.0120289711290475\\
131	0.0120289621265582\\
132	0.0120289529599735\\
133	0.0120289436262814\\
134	0.0120289341224142\\
135	0.0120289244452476\\
136	0.0120289145915999\\
137	0.0120289045582304\\
138	0.0120288943418389\\
139	0.0120288839390641\\
140	0.0120288733464828\\
141	0.0120288625606089\\
142	0.0120288515778917\\
143	0.0120288403947153\\
144	0.012028829007397\\
145	0.0120288174121864\\
146	0.0120288056052639\\
147	0.0120287935827395\\
148	0.0120287813406517\\
149	0.012028768874966\\
150	0.0120287561815737\\
151	0.0120287432562902\\
152	0.0120287300948542\\
153	0.012028716692926\\
154	0.0120287030460859\\
155	0.0120286891498331\\
156	0.0120286749995839\\
157	0.0120286605906705\\
158	0.0120286459183392\\
159	0.0120286309777491\\
160	0.0120286157639703\\
161	0.0120286002719823\\
162	0.0120285844966728\\
163	0.0120285684328351\\
164	0.0120285520751674\\
165	0.0120285354182704\\
166	0.0120285184566459\\
167	0.0120285011846948\\
168	0.0120284835967152\\
169	0.0120284656869009\\
170	0.0120284474493389\\
171	0.0120284288780082\\
172	0.0120284099667771\\
173	0.0120283907094017\\
174	0.0120283710995235\\
175	0.0120283511306677\\
176	0.0120283307962404\\
177	0.0120283100895271\\
178	0.0120282890036903\\
179	0.0120282675317669\\
180	0.0120282456666662\\
181	0.0120282234011675\\
182	0.0120282007279176\\
183	0.0120281776394285\\
184	0.0120281541280747\\
185	0.0120281301860908\\
186	0.0120281058055692\\
187	0.0120280809784568\\
188	0.0120280556965533\\
189	0.0120280299515078\\
190	0.0120280037348172\\
191	0.0120279770378239\\
192	0.0120279498517154\\
193	0.0120279221675255\\
194	0.0120278939761393\\
195	0.0120278652683013\\
196	0.0120278360346111\\
197	0.0120278062654331\\
198	0.0120277759504069\\
199	0.0120277450795075\\
200	0.0120277136425499\\
201	0.0120276816291618\\
202	0.0120276490287804\\
203	0.0120276158306487\\
204	0.0120275820238121\\
205	0.012027547597114\\
206	0.0120275125391925\\
207	0.0120274768384759\\
208	0.0120274404831791\\
209	0.0120274034612994\\
210	0.0120273657606117\\
211	0.0120273273686646\\
212	0.0120272882727759\\
213	0.0120272484600275\\
214	0.0120272079172613\\
215	0.0120271666310738\\
216	0.012027124587811\\
217	0.0120270817735637\\
218	0.0120270381741617\\
219	0.0120269937751683\\
220	0.0120269485618749\\
221	0.0120269025192947\\
222	0.0120268556321566\\
223	0.0120268078848991\\
224	0.0120267592616635\\
225	0.0120267097462871\\
226	0.0120266593222959\\
227	0.0120266079728973\\
228	0.0120265556809724\\
229	0.0120265024290674\\
230	0.0120264481993857\\
231	0.0120263929737783\\
232	0.0120263367337346\\
233	0.0120262794603728\\
234	0.0120262211344286\\
235	0.0120261617362449\\
236	0.0120261012457592\\
237	0.0120260396424918\\
238	0.0120259769055316\\
239	0.0120259130135226\\
240	0.012025847944648\\
241	0.0120257816766139\\
242	0.0120257141866318\\
243	0.0120256454513992\\
244	0.0120255754470795\\
245	0.0120255041492789\\
246	0.0120254315330229\\
247	0.0120253575727294\\
248	0.0120252822421802\\
249	0.0120252055144893\\
250	0.0120251273620687\\
251	0.0120250477565904\\
252	0.0120249666689445\\
253	0.0120248840691939\\
254	0.0120247999265234\\
255	0.0120247142091835\\
256	0.0120246268844287\\
257	0.0120245379184485\\
258	0.0120244472762907\\
259	0.0120243549217768\\
260	0.0120242608174063\\
261	0.0120241649242511\\
262	0.0120240672018367\\
263	0.0120239676080095\\
264	0.0120238660987877\\
265	0.0120237626281936\\
266	0.0120236571480655\\
267	0.0120235496078454\\
268	0.0120234399543379\\
269	0.0120233281314393\\
270	0.0120232140798276\\
271	0.0120230977366076\\
272	0.0120229790348989\\
273	0.012022857903336\\
274	0.0120227342654058\\
275	0.0120226080383852\\
276	0.012022479131077\\
277	0.0120223474375137\\
278	0.0120222128162072\\
279	0.0120220715723733\\
280	0.0120219213295351\\
281	0.0120217683596326\\
282	0.012021612613815\\
283	0.0120214540423758\\
284	0.0120212925947387\\
285	0.0120211282194428\\
286	0.012020960864128\\
287	0.0120207904755191\\
288	0.012020616999411\\
289	0.0120204403806528\\
290	0.0120202605631322\\
291	0.0120200774897588\\
292	0.012019891102448\\
293	0.0120197013421045\\
294	0.012019508148605\\
295	0.0120193114607812\\
296	0.0120191112164021\\
297	0.0120189073521568\\
298	0.0120186998036358\\
299	0.0120184885053131\\
300	0.0120182733905276\\
301	0.0120180543914639\\
302	0.0120178314391339\\
303	0.0120176044633563\\
304	0.0120173733927379\\
305	0.0120171381546528\\
306	0.0120168986752226\\
307	0.0120166548792956\\
308	0.0120164066904259\\
309	0.0120161540308522\\
310	0.0120158968214763\\
311	0.0120156349818416\\
312	0.0120153684301107\\
313	0.0120150970830431\\
314	0.0120148208559728\\
315	0.0120145396627854\\
316	0.0120142534158947\\
317	0.0120139620262195\\
318	0.01201366540316\\
319	0.0120133634545736\\
320	0.0120130560867512\\
321	0.0120127432043928\\
322	0.0120124247105826\\
323	0.0120121005067651\\
324	0.0120117704927196\\
325	0.012011434566536\\
326	0.0120110926245892\\
327	0.0120107445615147\\
328	0.0120103902701834\\
329	0.0120100296416768\\
330	0.0120096625652621\\
331	0.0120092889283681\\
332	0.0120089086165606\\
333	0.0120085215135186\\
334	0.0120081275010108\\
335	0.0120077264588732\\
336	0.0120073182649862\\
337	0.0120069027952543\\
338	0.0120064799235848\\
339	0.0120060495218696\\
340	0.0120056114599669\\
341	0.0120051656056856\\
342	0.0120047118247706\\
343	0.0120042499808916\\
344	0.012003779935633\\
345	0.0120033015484878\\
346	0.0120028146768541\\
347	0.0120023191760361\\
348	0.012001814899249\\
349	0.0120013016976288\\
350	0.0120007794202484\\
351	0.0120002479141394\\
352	0.0119997070243222\\
353	0.0119991565938432\\
354	0.011998596463823\\
355	0.0119980264735143\\
356	0.0119974464603725\\
357	0.0119968562601403\\
358	0.0119962557069481\\
359	0.011995644633432\\
360	0.0119950228708724\\
361	0.0119943902493562\\
362	0.0119937465979637\\
363	0.0119930917449871\\
364	0.0119924255181811\\
365	0.0119917477450529\\
366	0.0119910582531953\\
367	0.011990356870669\\
368	0.011989643426442\\
369	0.0119889177508927\\
370	0.011988179676386\\
371	0.0119874290379337\\
372	0.0119866656739499\\
373	0.0119858894271167\\
374	0.0119851001453745\\
375	0.0119842976830583\\
376	0.011983481902199\\
377	0.0119826526740142\\
378	0.0119818098806119\\
379	0.0119809534169174\\
380	0.0119800831927854\\
381	0.0119791991350906\\
382	0.0119783011890194\\
383	0.01197738931588\\
384	0.0119764634782775\\
385	0.0119755235810857\\
386	0.0119745692569808\\
387	0.011973625409027\\
388	0.0119727124513343\\
389	0.0119717826992378\\
390	0.0119708358443789\\
391	0.0119698715731102\\
392	0.0119688895663477\\
393	0.0119678894992354\\
394	0.0119668710405976\\
395	0.011965833853782\\
396	0.0119647775967581\\
397	0.0119637019212927\\
398	0.0119626064726859\\
399	0.0119614908894618\\
400	0.0119603548030381\\
401	0.0119591978374037\\
402	0.0119580196087587\\
403	0.0119568197251915\\
404	0.0119555977866664\\
405	0.0119543533842356\\
406	0.0119530860987276\\
407	0.0119517955009753\\
408	0.0119504811512473\\
409	0.0119491425986175\\
410	0.0119477793802661\\
411	0.0119463910207103\\
412	0.0119449770309564\\
413	0.0119435369075439\\
414	0.0119420701313624\\
415	0.0119405761658887\\
416	0.0119390544541572\\
417	0.011937504414738\\
418	0.0119359254489342\\
419	0.0119343169344911\\
420	0.0119326782233582\\
421	0.0119310086428773\\
422	0.011929307507584\\
423	0.0119275741712286\\
424	0.0119258082305455\\
425	0.0119240103084827\\
426	0.0119221792712286\\
427	0.0119203122165478\\
428	0.0119184083403242\\
429	0.0119164668132416\\
430	0.0119144867795492\\
431	0.0119124673558056\\
432	0.0119104076294985\\
433	0.0119083066575484\\
434	0.0119061634647583\\
435	0.0119039770421726\\
436	0.0119017463453406\\
437	0.0118994702924773\\
438	0.0118971477625141\\
439	0.0118947775930574\\
440	0.0118923585782448\\
441	0.0118898894664903\\
442	0.0118873689581361\\
443	0.0118847957030411\\
444	0.0118821682981956\\
445	0.0118794852855955\\
446	0.0118767451509667\\
447	0.0118739463246458\\
448	0.011871087186619\\
449	0.0118681660736066\\
450	0.0118651812470562\\
451	0.0118621307121703\\
452	0.0118590125533093\\
453	0.0118558247686228\\
454	0.0118525652646353\\
455	0.0118492318504519\\
456	0.0118458222315362\\
457	0.0118423340030213\\
458	0.0118387646425096\\
459	0.0118351115023127\\
460	0.0118313718010989\\
461	0.0118275426148839\\
462	0.0118236208673154\\
463	0.0118196033191998\\
464	0.0118154865572239\\
465	0.0118112669818219\\
466	0.0118069407940596\\
467	0.0118025039810784\\
468	0.0117979522987619\\
469	0.0117932812501646\\
470	0.0117884860748163\\
471	0.011783561727839\\
472	0.0117785028554338\\
473	0.0117733037714346\\
474	0.0117679584324752\\
475	0.0117624604127193\\
476	0.0117568028825618\\
477	0.0117509785853709\\
478	0.0117449798086852\\
479	0.011738798513994\\
480	0.0117324267035955\\
481	0.0117258577258535\\
482	0.0117190787737757\\
483	0.0117120827645168\\
484	0.0117048720754383\\
485	0.0116974051805671\\
486	0.0116896676075278\\
487	0.0116816438179188\\
488	0.0116733172216274\\
489	0.0116646699106196\\
490	0.0116556825278185\\
491	0.01164633411655\\
492	0.0116366019565618\\
493	0.0116264613796394\\
494	0.0116158855619995\\
495	0.0116048452897468\\
496	0.0115933086932196\\
497	0.011581240944953\\
498	0.0115686039134893\\
499	0.0115553557601971\\
500	0.0115414504689702\\
501	0.0115268373147794\\
502	0.0115114602484086\\
503	0.0114952571479728\\
504	0.0114781582844479\\
505	0.0114600852174173\\
506	0.0114409493222894\\
507	0.0114206527983975\\
508	0.0113990867611616\\
509	0.0113760762083767\\
510	0.0113513752036678\\
511	0.0113159381479182\\
512	0.0112679636595168\\
513	0.0112173754871497\\
514	0.0111633610544983\\
515	0.0110428004028452\\
516	0.0108941234215818\\
517	0.0107427987078863\\
518	0.0105892321186162\\
519	0.010537604940941\\
520	0.010500770558179\\
521	0.0104697504356829\\
522	0.0104461025364528\\
523	0.0104232374023355\\
524	0.0104012318717967\\
525	0.0103802574261422\\
526	0.0103590187928536\\
527	0.0103374538335167\\
528	0.0103155080760614\\
529	0.0102931128473157\\
530	0.0102702293080313\\
531	0.0102468372365915\\
532	0.0102229185854888\\
533	0.0101984567735062\\
534	0.0101734373743994\\
535	0.010147846227661\\
536	0.0101216669001464\\
537	0.0100948073762948\\
538	0.0100669889691556\\
539	0.0100386403761204\\
540	0.0100084781527846\\
541	0.00997746649239294\\
542	0.00994664059615901\\
543	0.00991824089606246\\
544	0.00988933523322398\\
545	0.00985987564026084\\
546	0.00982978318555431\\
547	0.00979905747736052\\
548	0.00976773860790732\\
549	0.00973575981902065\\
550	0.0097030515171097\\
551	0.00966959432886941\\
552	0.00963536803785071\\
553	0.00960035150018078\\
554	0.00956451998046889\\
555	0.00952705413323322\\
556	0.00948710602387265\\
557	0.00944503338855385\\
558	0.00940887643030145\\
559	0.00937288992527579\\
560	0.00933618315537469\\
561	0.00929873143952121\\
562	0.00926051622101601\\
563	0.00922151860131325\\
564	0.00918171898438378\\
565	0.00914109688559529\\
566	0.00909963057080181\\
567	0.00901421719622151\\
568	0.00862731211635968\\
569	0.00832310839907635\\
570	0.00825763909419514\\
571	0.00819106539533925\\
572	0.0081233622594618\\
573	0.00805450378369977\\
574	0.00798446304594252\\
575	0.00791321205053708\\
576	0.0078407216727329\\
577	0.00776696160164116\\
578	0.00769190028172421\\
579	0.00761550485219292\\
580	0.00753774108175196\\
581	0.00745857329076527\\
582	0.00737796423854298\\
583	0.00729587491554802\\
584	0.00721226408110735\\
585	0.00712708712854451\\
586	0.00704029318686139\\
587	0.00695181762108358\\
588	0.00686156256306142\\
589	0.00676934637110726\\
590	0.00667477256501154\\
591	0.00657713355197374\\
592	0.0064748166249999\\
593	0.00636364764633834\\
594	0.00623271716130668\\
595	0.0060534099414828\\
596	0.00575058001197162\\
597	0.00512683753504545\\
598	0.00366374385960312\\
599	0\\
600	0\\
};
\addplot [color=mycolor4,solid,forget plot]
  table[row sep=crcr]{%
1	0.0120359606790683\\
2	0.0120359597145902\\
3	0.0120359587329864\\
4	0.0120359577339511\\
5	0.0120359567171732\\
6	0.0120359556823358\\
7	0.0120359546291166\\
8	0.0120359535571873\\
9	0.0120359524662137\\
10	0.0120359513558557\\
11	0.0120359502257671\\
12	0.0120359490755953\\
13	0.0120359479049816\\
14	0.0120359467135608\\
15	0.012035945500961\\
16	0.0120359442668036\\
17	0.0120359430107035\\
18	0.0120359417322684\\
19	0.0120359404310989\\
20	0.0120359391067887\\
21	0.0120359377589239\\
22	0.0120359363870833\\
23	0.0120359349908382\\
24	0.0120359335697521\\
25	0.0120359321233805\\
26	0.0120359306512711\\
27	0.0120359291529634\\
28	0.0120359276279886\\
29	0.0120359260758695\\
30	0.0120359244961203\\
31	0.0120359228882463\\
32	0.0120359212517442\\
33	0.0120359195861013\\
34	0.0120359178907959\\
35	0.0120359161652967\\
36	0.0120359144090631\\
37	0.0120359126215446\\
38	0.0120359108021806\\
39	0.0120359089504007\\
40	0.0120359070656242\\
41	0.0120359051472595\\
42	0.012035903194705\\
43	0.0120359012073476\\
44	0.0120358991845636\\
45	0.0120358971257179\\
46	0.0120358950301637\\
47	0.012035892897243\\
48	0.0120358907262856\\
49	0.0120358885166092\\
50	0.0120358862675193\\
51	0.0120358839783088\\
52	0.0120358816482579\\
53	0.0120358792766335\\
54	0.0120358768626898\\
55	0.012035874405667\\
56	0.0120358719047918\\
57	0.012035869359277\\
58	0.0120358667683209\\
59	0.0120358641311076\\
60	0.0120358614468063\\
61	0.0120358587145711\\
62	0.0120358559335409\\
63	0.012035853102839\\
64	0.0120358502215729\\
65	0.0120358472888338\\
66	0.0120358443036965\\
67	0.012035841265219\\
68	0.0120358381724425\\
69	0.0120358350243904\\
70	0.0120358318200688\\
71	0.0120358285584655\\
72	0.0120358252385501\\
73	0.0120358218592735\\
74	0.0120358184195676\\
75	0.0120358149183448\\
76	0.0120358113544979\\
77	0.0120358077268996\\
78	0.012035804034402\\
79	0.0120358002758366\\
80	0.0120357964500134\\
81	0.012035792555721\\
82	0.0120357885917259\\
83	0.0120357845567722\\
84	0.0120357804495813\\
85	0.0120357762688511\\
86	0.0120357720132559\\
87	0.0120357676814461\\
88	0.0120357632720474\\
89	0.0120357587836603\\
90	0.0120357542148601\\
91	0.0120357495641962\\
92	0.0120357448301912\\
93	0.0120357400113412\\
94	0.0120357351061147\\
95	0.0120357301129522\\
96	0.0120357250302659\\
97	0.0120357198564392\\
98	0.0120357145898256\\
99	0.0120357092287489\\
100	0.0120357037715022\\
101	0.0120356982163473\\
102	0.0120356925615144\\
103	0.0120356868052013\\
104	0.012035680945573\\
105	0.0120356749807607\\
106	0.0120356689088616\\
107	0.0120356627279381\\
108	0.0120356564360171\\
109	0.0120356500310893\\
110	0.0120356435111087\\
111	0.0120356368739918\\
112	0.0120356301176169\\
113	0.0120356232398234\\
114	0.0120356162384112\\
115	0.0120356091111397\\
116	0.0120356018557272\\
117	0.0120355944698502\\
118	0.0120355869511425\\
119	0.0120355792971943\\
120	0.0120355715055517\\
121	0.0120355635737156\\
122	0.0120355554991411\\
123	0.0120355472792362\\
124	0.0120355389113615\\
125	0.0120355303928288\\
126	0.0120355217209005\\
127	0.0120355128927886\\
128	0.0120355039056537\\
129	0.012035494756604\\
130	0.0120354854426944\\
131	0.0120354759609256\\
132	0.0120354663082429\\
133	0.0120354564815352\\
134	0.012035446477634\\
135	0.0120354362933123\\
136	0.0120354259252834\\
137	0.0120354153702001\\
138	0.0120354046246532\\
139	0.0120353936851705\\
140	0.0120353825482156\\
141	0.012035371210187\\
142	0.0120353596674161\\
143	0.0120353479161669\\
144	0.0120353359526342\\
145	0.0120353237729423\\
146	0.0120353113731439\\
147	0.0120352987492188\\
148	0.0120352858970722\\
149	0.0120352728125338\\
150	0.0120352594913562\\
151	0.0120352459292132\\
152	0.012035232121699\\
153	0.0120352180643262\\
154	0.0120352037525244\\
155	0.012035189181639\\
156	0.0120351743469293\\
157	0.0120351592435672\\
158	0.0120351438666356\\
159	0.0120351282111265\\
160	0.01203511227194\\
161	0.0120350960438821\\
162	0.0120350795216631\\
163	0.0120350626998962\\
164	0.0120350455730957\\
165	0.0120350281356749\\
166	0.0120350103819449\\
167	0.0120349923061124\\
168	0.012034973902278\\
169	0.0120349551644345\\
170	0.0120349360864647\\
171	0.0120349166621398\\
172	0.0120348968851174\\
173	0.0120348767489394\\
174	0.0120348562470302\\
175	0.0120348353726944\\
176	0.012034814119115\\
177	0.0120347924793509\\
178	0.0120347704463351\\
179	0.0120347480128722\\
180	0.0120347251716363\\
181	0.0120347019151684\\
182	0.0120346782358744\\
183	0.0120346541260219\\
184	0.0120346295777385\\
185	0.0120346045830083\\
186	0.0120345791336699\\
187	0.0120345532214131\\
188	0.0120345268377759\\
189	0.0120344999741422\\
190	0.0120344726217377\\
191	0.0120344447716273\\
192	0.0120344164147119\\
193	0.0120343875417244\\
194	0.0120343581432269\\
195	0.0120343282096068\\
196	0.0120342977310737\\
197	0.0120342666976557\\
198	0.012034235099196\\
199	0.012034202925348\\
200	0.0120341701655715\\
201	0.0120341368091284\\
202	0.0120341028450792\\
203	0.0120340682622785\\
204	0.0120340330493707\\
205	0.0120339971947859\\
206	0.0120339606867355\\
207	0.0120339235132071\\
208	0.0120338856619602\\
209	0.0120338471205213\\
210	0.0120338078761788\\
211	0.0120337679159779\\
212	0.0120337272267153\\
213	0.0120336857949339\\
214	0.0120336436069171\\
215	0.012033600648683\\
216	0.0120335569059784\\
217	0.0120335123642729\\
218	0.0120334670087527\\
219	0.0120334208243135\\
220	0.0120333737955544\\
221	0.0120333259067709\\
222	0.0120332771419474\\
223	0.0120332274847502\\
224	0.0120331769185197\\
225	0.0120331254262626\\
226	0.0120330729906438\\
227	0.0120330195939781\\
228	0.0120329652182214\\
229	0.0120329098449618\\
230	0.0120328534554103\\
231	0.0120327960303916\\
232	0.0120327375503336\\
233	0.012032677995258\\
234	0.0120326173447693\\
235	0.0120325555780439\\
236	0.0120324926738197\\
237	0.0120324286103835\\
238	0.0120323633655602\\
239	0.0120322969167001\\
240	0.0120322292406665\\
241	0.0120321603138232\\
242	0.0120320901120211\\
243	0.0120320186105852\\
244	0.0120319457843009\\
245	0.0120318716074002\\
246	0.0120317960535482\\
247	0.0120317190958286\\
248	0.0120316407067303\\
249	0.0120315608581331\\
250	0.0120314795212942\\
251	0.0120313966668353\\
252	0.0120313122647291\\
253	0.0120312262842875\\
254	0.0120311386941505\\
255	0.0120310494622759\\
256	0.0120309585559307\\
257	0.012030865941684\\
258	0.0120307715854032\\
259	0.0120306754522507\\
260	0.0120305775066864\\
261	0.0120304777124717\\
262	0.0120303760326797\\
263	0.0120302724297095\\
264	0.012030166865308\\
265	0.0120300593005975\\
266	0.0120299496961136\\
267	0.0120298380118518\\
268	0.0120297242073262\\
269	0.0120296082416417\\
270	0.0120294900735813\\
271	0.0120293696617117\\
272	0.0120292469645062\\
273	0.0120291219404863\\
274	0.0120289945483684\\
275	0.0120288647471818\\
276	0.0120287324962426\\
277	0.0120285977546729\\
278	0.0120284604796759\\
279	0.0120283198503998\\
280	0.0120281753226869\\
281	0.0120280282403208\\
282	0.0120278785593939\\
283	0.0120277262352805\\
284	0.0120275712226256\\
285	0.0120274134753339\\
286	0.0120272529465588\\
287	0.0120270895886905\\
288	0.0120269233533449\\
289	0.0120267541913517\\
290	0.0120265820527423\\
291	0.0120264068867378\\
292	0.0120262286417368\\
293	0.0120260472653024\\
294	0.0120258627041503\\
295	0.0120256749041348\\
296	0.0120254838102365\\
297	0.0120252893665483\\
298	0.0120250915162623\\
299	0.0120248902016553\\
300	0.012024685364075\\
301	0.0120244769439255\\
302	0.0120242648806526\\
303	0.0120240491127286\\
304	0.0120238295776373\\
305	0.012023606211858\\
306	0.0120233789508497\\
307	0.0120231477290347\\
308	0.012022912479782\\
309	0.0120226731353898\\
310	0.0120224296270684\\
311	0.0120221818849219\\
312	0.0120219298379298\\
313	0.0120216734139283\\
314	0.0120214125395901\\
315	0.0120211471404055\\
316	0.0120208771406605\\
317	0.0120206024634166\\
318	0.0120203230304882\\
319	0.0120200387624201\\
320	0.0120197495784643\\
321	0.0120194553965554\\
322	0.0120191561332859\\
323	0.0120188517038797\\
324	0.0120185420221655\\
325	0.0120182270005482\\
326	0.0120179065499801\\
327	0.0120175805799298\\
328	0.012017248998351\\
329	0.0120169117116484\\
330	0.0120165686246435\\
331	0.0120162196405377\\
332	0.0120158646608736\\
333	0.0120155035854954\\
334	0.0120151363125056\\
335	0.0120147627382206\\
336	0.0120143827571233\\
337	0.012013996261813\\
338	0.0120136031429519\\
339	0.0120132032892096\\
340	0.0120127965872026\\
341	0.0120123829214307\\
342	0.0120119621742093\\
343	0.0120115342255967\\
344	0.0120110989533162\\
345	0.0120106562326738\\
346	0.0120102059364678\\
347	0.0120097479348941\\
348	0.012009282095442\\
349	0.0120088082827839\\
350	0.012008326358654\\
351	0.0120078361817192\\
352	0.0120073376074382\\
353	0.0120068304879084\\
354	0.0120063146717\\
355	0.0120057900036757\\
356	0.0120052563247937\\
357	0.012004713471893\\
358	0.0120041612774585\\
359	0.0120035995693643\\
360	0.012003028170592\\
361	0.0120024468989218\\
362	0.0120018555665928\\
363	0.012001253979929\\
364	0.0120006419389284\\
365	0.0120000192368075\\
366	0.0119993856595003\\
367	0.0119987409851025\\
368	0.0119980849832564\\
369	0.0119974174144692\\
370	0.0119967380293545\\
371	0.0119960465677902\\
372	0.011995342757979\\
373	0.0119946263154014\\
374	0.0119938969416451\\
375	0.0119931543230939\\
376	0.0119923981294571\\
377	0.0119916280121149\\
378	0.0119908436022476\\
379	0.0119900445087076\\
380	0.0119892303155564\\
381	0.0119884005791028\\
382	0.0119875548239905\\
383	0.0119866925369336\\
384	0.0119858131533778\\
385	0.0119849160203859\\
386	0.0119840002743995\\
387	0.0119830447810233\\
388	0.0119820332865597\\
389	0.0119810043420775\\
390	0.0119799576332635\\
391	0.0119788928404135\\
392	0.0119778096411205\\
393	0.0119767077110179\\
394	0.0119755867194906\\
395	0.0119744463010784\\
396	0.0119732860919018\\
397	0.0119721057495234\\
398	0.0119709049305588\\
399	0.0119696832912273\\
400	0.0119684404875616\\
401	0.0119671761753424\\
402	0.0119658900101433\\
403	0.0119645816462572\\
404	0.0119632507360533\\
405	0.0119618969494199\\
406	0.0119605199699264\\
407	0.0119591194537006\\
408	0.0119576950576613\\
409	0.0119562464404785\\
410	0.0119547732637695\\
411	0.0119532751936683\\
412	0.0119517519030298\\
413	0.0119502030747196\\
414	0.0119486284065388\\
415	0.0119470276175077\\
416	0.0119454004498282\\
417	0.011943746637176\\
418	0.0119420657191672\\
419	0.0119403574311319\\
420	0.0119386215620075\\
421	0.0119368579280512\\
422	0.0119350663669119\\
423	0.0119332467052732\\
424	0.011931398634631\\
425	0.0119295212651841\\
426	0.0119276712979806\\
427	0.0119258618224502\\
428	0.0119240185054655\\
429	0.011922140633935\\
430	0.0119202274726783\\
431	0.0119182782625387\\
432	0.0119162922207086\\
433	0.0119142685397912\\
434	0.0119122063852172\\
435	0.0119101048935508\\
436	0.0119079631706583\\
437	0.0119057802897224\\
438	0.0119035552889629\\
439	0.0119012871686918\\
440	0.011898974888498\\
441	0.011896617364291\\
442	0.0118942134647693\\
443	0.0118917620075053\\
444	0.0118892617547244\\
445	0.0118867114092115\\
446	0.0118841096121044\\
447	0.011881454949172\\
448	0.0118787459900734\\
449	0.0118759814522358\\
450	0.0118731608374308\\
451	0.0118702852378632\\
452	0.0118673477930897\\
453	0.0118643467299626\\
454	0.0118612801968073\\
455	0.011858146258142\\
456	0.0118549428894319\\
457	0.0118516679715537\\
458	0.0118483192849539\\
459	0.0118448945034086\\
460	0.0118413911870938\\
461	0.0118378067754983\\
462	0.0118341385798008\\
463	0.0118303837748577\\
464	0.0118265393909793\\
465	0.0118226023060342\\
466	0.0118185692391883\\
467	0.0118144367487465\\
468	0.0118102012351429\\
469	0.01180585892488\\
470	0.0118014056451756\\
471	0.0117968370573102\\
472	0.011792148653276\\
473	0.0117873356601239\\
474	0.0117823930201877\\
475	0.0117773153697698\\
476	0.0117720970162177\\
477	0.0117667319140896\\
478	0.011761213642902\\
479	0.0117555353867836\\
480	0.0117496899160052\\
481	0.0117436695365881\\
482	0.0117374663740204\\
483	0.0117310727138272\\
484	0.0117244818640581\\
485	0.0117176807134393\\
486	0.0117106650736784\\
487	0.0117034291790748\\
488	0.0116959369600259\\
489	0.0116881740726698\\
490	0.0116801251570261\\
491	0.0116717738069703\\
492	0.0116631023328707\\
493	0.0116540916358305\\
494	0.0116447210641159\\
495	0.0116349682571253\\
496	0.0116248089702437\\
497	0.011614216878034\\
498	0.0116031633521892\\
499	0.0115916172107801\\
500	0.0115795444347776\\
501	0.0115669078443119\\
502	0.0115536667262967\\
503	0.0115397763968374\\
504	0.0115251877077862\\
505	0.0115098464716553\\
506	0.011493692798663\\
507	0.0114766601459512\\
508	0.011458673693352\\
509	0.0114396495179366\\
510	0.0114194931741318\\
511	0.0113981021064543\\
512	0.0113753435529281\\
513	0.0113510298642051\\
514	0.0113248481248743\\
515	0.0112826994026206\\
516	0.0112326650254009\\
517	0.0111798074843413\\
518	0.011123210112933\\
519	0.0109825812882413\\
520	0.0108299743020159\\
521	0.0106745698670279\\
522	0.0105172459195764\\
523	0.0104744672804384\\
524	0.0104361482762094\\
525	0.0104031045229496\\
526	0.0103774141853245\\
527	0.0103525128845029\\
528	0.0103282500875483\\
529	0.0103049500232545\\
530	0.010281860072798\\
531	0.0102584265610034\\
532	0.0102345853612172\\
533	0.0102102885372687\\
534	0.0101854675139607\\
535	0.0101600974623455\\
536	0.0101341575278408\\
537	0.010107629058335\\
538	0.0100804944109962\\
539	0.0100527338341062\\
540	0.0100240505803467\\
541	0.0099946343130577\\
542	0.00996455191135411\\
543	0.00993204582059295\\
544	0.00989954848574274\\
545	0.00986799151672372\\
546	0.00983787325018791\\
547	0.00980723263760193\\
548	0.00977603928561901\\
549	0.00974423691066886\\
550	0.00971171393718171\\
551	0.00967844964677866\\
552	0.00964442416272868\\
553	0.00960961707724233\\
554	0.00957400721930803\\
555	0.00953757253722866\\
556	0.00949991012500304\\
557	0.00946102814078937\\
558	0.00941785857696371\\
559	0.00937663850101491\\
560	0.00933985709546176\\
561	0.00930246673014417\\
562	0.00926431963294322\\
563	0.00922539226211695\\
564	0.00918566486579048\\
565	0.00914511719909901\\
566	0.00910372811490353\\
567	0.00906147522229277\\
568	0.00897133540686007\\
569	0.00859185485320159\\
570	0.0082576416188635\\
571	0.00819106540855579\\
572	0.00812336226305488\\
573	0.00805450378548458\\
574	0.0079844630468339\\
575	0.00791321205096173\\
576	0.0078407216729221\\
577	0.00776696160171876\\
578	0.00769190028175292\\
579	0.00761550485220227\\
580	0.00753774108175452\\
581	0.00745857329076583\\
582	0.00737796423854308\\
583	0.00729587491554804\\
584	0.00721226408110735\\
585	0.00712708712854453\\
586	0.00704029318686139\\
587	0.00695181762108359\\
588	0.00686156256306143\\
589	0.00676934637110726\\
590	0.00667477256501155\\
591	0.00657713355197374\\
592	0.00647481662499991\\
593	0.00636364764633835\\
594	0.00623271716130668\\
595	0.00605340994148281\\
596	0.00575058001197164\\
597	0.00512683753504545\\
598	0.00366374385960312\\
599	0\\
600	0\\
};
\addplot [color=mycolor5,solid,forget plot]
  table[row sep=crcr]{%
1	0.0120472198053511\\
2	0.012047218586242\\
3	0.0120472173458397\\
4	0.0120472160837709\\
5	0.012047214799656\\
6	0.0120472134931088\\
7	0.0120472121637361\\
8	0.0120472108111379\\
9	0.0120472094349072\\
10	0.0120472080346297\\
11	0.012047206609884\\
12	0.0120472051602411\\
13	0.0120472036852646\\
14	0.0120472021845103\\
15	0.0120472006575265\\
16	0.012047199103853\\
17	0.0120471975230221\\
18	0.0120471959145574\\
19	0.0120471942779744\\
20	0.01204719261278\\
21	0.0120471909184724\\
22	0.0120471891945409\\
23	0.012047187440466\\
24	0.0120471856557189\\
25	0.0120471838397615\\
26	0.0120471819920463\\
27	0.0120471801120162\\
28	0.0120471781991042\\
29	0.0120471762527332\\
30	0.0120471742723163\\
31	0.0120471722572559\\
32	0.012047170206944\\
33	0.012047168120762\\
34	0.0120471659980803\\
35	0.0120471638382582\\
36	0.0120471616406438\\
37	0.0120471594045736\\
38	0.0120471571293724\\
39	0.0120471548143533\\
40	0.0120471524588171\\
41	0.0120471500620524\\
42	0.0120471476233351\\
43	0.0120471451419285\\
44	0.0120471426170829\\
45	0.0120471400480352\\
46	0.0120471374340091\\
47	0.0120471347742145\\
48	0.0120471320678472\\
49	0.0120471293140892\\
50	0.0120471265121077\\
51	0.0120471236610554\\
52	0.0120471207600699\\
53	0.0120471178082739\\
54	0.0120471148047742\\
55	0.0120471117486622\\
56	0.0120471086390131\\
57	0.0120471054748856\\
58	0.0120471022553221\\
59	0.012047098979348\\
60	0.0120470956459714\\
61	0.0120470922541828\\
62	0.0120470888029552\\
63	0.0120470852912431\\
64	0.0120470817179827\\
65	0.0120470780820914\\
66	0.0120470743824674\\
67	0.0120470706179896\\
68	0.0120470667875167\\
69	0.0120470628898877\\
70	0.0120470589239207\\
71	0.012047054888413\\
72	0.0120470507821407\\
73	0.0120470466038582\\
74	0.0120470423522978\\
75	0.0120470380261695\\
76	0.0120470336241601\\
77	0.0120470291449337\\
78	0.0120470245871303\\
79	0.0120470199493659\\
80	0.0120470152302321\\
81	0.0120470104282955\\
82	0.0120470055420971\\
83	0.0120470005701523\\
84	0.0120469955109501\\
85	0.0120469903629524\\
86	0.0120469851245942\\
87	0.0120469797942825\\
88	0.0120469743703962\\
89	0.0120469688512851\\
90	0.01204696323527\\
91	0.0120469575206416\\
92	0.0120469517056604\\
93	0.0120469457885559\\
94	0.012046939767526\\
95	0.0120469336407368\\
96	0.0120469274063215\\
97	0.0120469210623801\\
98	0.0120469146069788\\
99	0.0120469080381494\\
100	0.0120469013538884\\
101	0.0120468945521568\\
102	0.012046887630879\\
103	0.0120468805879426\\
104	0.0120468734211972\\
105	0.0120468661284541\\
106	0.0120468587074856\\
107	0.0120468511560238\\
108	0.0120468434717607\\
109	0.0120468356523464\\
110	0.0120468276953893\\
111	0.012046819598455\\
112	0.012046811359065\\
113	0.0120468029746967\\
114	0.012046794442782\\
115	0.0120467857607068\\
116	0.0120467769258098\\
117	0.0120467679353819\\
118	0.0120467587866653\\
119	0.0120467494768524\\
120	0.0120467400030849\\
121	0.012046730362453\\
122	0.0120467205519943\\
123	0.012046710568693\\
124	0.0120467004094784\\
125	0.0120466900712244\\
126	0.0120466795507483\\
127	0.0120466688448096\\
128	0.0120466579501087\\
129	0.0120466468632865\\
130	0.0120466355809223\\
131	0.0120466240995334\\
132	0.0120466124155735\\
133	0.0120466005254315\\
134	0.0120465884254305\\
135	0.0120465761118261\\
136	0.0120465635808057\\
137	0.0120465508284863\\
138	0.012046537850914\\
139	0.0120465246440622\\
140	0.0120465112038301\\
141	0.0120464975260414\\
142	0.0120464836064429\\
143	0.0120464694407027\\
144	0.0120464550244089\\
145	0.0120464403530679\\
146	0.012046425422103\\
147	0.0120464102268524\\
148	0.0120463947625677\\
149	0.0120463790244124\\
150	0.0120463630074598\\
151	0.0120463467066914\\
152	0.012046330116995\\
153	0.0120463132331632\\
154	0.0120462960498911\\
155	0.0120462785617744\\
156	0.0120462607633077\\
157	0.0120462426488828\\
158	0.0120462242127857\\
159	0.0120462054491958\\
160	0.012046186352183\\
161	0.012046166915706\\
162	0.01204614713361\\
163	0.0120461269996247\\
164	0.0120461065073624\\
165	0.0120460856503152\\
166	0.0120460644218537\\
167	0.0120460428152241\\
168	0.0120460208235462\\
169	0.0120459984398118\\
170	0.0120459756568817\\
171	0.012045952467484\\
172	0.012045928864212\\
173	0.0120459048395219\\
174	0.0120458803857309\\
175	0.0120458554950147\\
176	0.0120458301594061\\
177	0.0120458043707924\\
178	0.0120457781209137\\
179	0.012045751401361\\
180	0.0120457242035742\\
181	0.01204569651884\\
182	0.0120456683382905\\
183	0.0120456396529009\\
184	0.0120456104534878\\
185	0.0120455807307074\\
186	0.0120455504750532\\
187	0.0120455196768545\\
188	0.0120454883262736\\
189	0.0120454564133039\\
190	0.0120454239277672\\
191	0.0120453908593108\\
192	0.012045357197404\\
193	0.0120453229313345\\
194	0.0120452880502034\\
195	0.01204525254292\\
196	0.0120452163981968\\
197	0.0120451796045461\\
198	0.0120451421502713\\
199	0.0120451040234629\\
200	0.0120450652119942\\
201	0.0120450257035177\\
202	0.0120449854854608\\
203	0.0120449445450216\\
204	0.0120449028691647\\
205	0.0120448604446167\\
206	0.0120448172578622\\
207	0.0120447732951389\\
208	0.0120447285424332\\
209	0.0120446829854758\\
210	0.0120446366097365\\
211	0.0120445894004198\\
212	0.01204454134246\\
213	0.0120444924205162\\
214	0.0120444426189671\\
215	0.0120443919219064\\
216	0.0120443403131373\\
217	0.0120442877761674\\
218	0.0120442342942033\\
219	0.0120441798501456\\
220	0.0120441244265833\\
221	0.0120440680057882\\
222	0.0120440105697099\\
223	0.0120439520999699\\
224	0.0120438925778561\\
225	0.0120438319843176\\
226	0.0120437702999586\\
227	0.0120437075050332\\
228	0.0120436435794399\\
229	0.0120435785027158\\
230	0.0120435122540312\\
231	0.0120434448121842\\
232	0.0120433761555956\\
233	0.0120433062623031\\
234	0.0120432351099565\\
235	0.0120431626758128\\
236	0.012043088936731\\
237	0.0120430138691678\\
238	0.012042937449173\\
239	0.0120428596523851\\
240	0.0120427804540282\\
241	0.0120426998289075\\
242	0.0120426177514068\\
243	0.0120425341954854\\
244	0.0120424491346762\\
245	0.0120423625420832\\
246	0.012042274390381\\
247	0.0120421846518138\\
248	0.0120420932981953\\
249	0.0120420003009097\\
250	0.012041905630913\\
251	0.0120418092587352\\
252	0.0120417111544835\\
253	0.0120416112878463\\
254	0.0120415096280978\\
255	0.0120414061441044\\
256	0.0120413008043309\\
257	0.0120411935768484\\
258	0.0120410844293431\\
259	0.0120409733291258\\
260	0.012040860243142\\
261	0.0120407451379833\\
262	0.0120406279798987\\
263	0.012040508734807\\
264	0.0120403873683088\\
265	0.0120402638456985\\
266	0.0120401381319763\\
267	0.0120400101918581\\
268	0.0120398799897854\\
269	0.0120397474899317\\
270	0.0120396126562066\\
271	0.0120394754522559\\
272	0.0120393358414557\\
273	0.0120391937868992\\
274	0.0120390492513723\\
275	0.0120389021973127\\
276	0.0120387525867447\\
277	0.0120386003811919\\
278	0.0120384455416318\\
279	0.01203828802883\\
280	0.0120381278016652\\
281	0.0120379648152236\\
282	0.0120377990239095\\
283	0.0120376303814367\\
284	0.0120374588408205\\
285	0.0120372843543689\\
286	0.0120371068736742\\
287	0.0120369263496043\\
288	0.0120367427322945\\
289	0.0120365559711384\\
290	0.0120363660147796\\
291	0.0120361728111025\\
292	0.0120359763072244\\
293	0.0120357764494858\\
294	0.0120355731834423\\
295	0.0120353664538554\\
296	0.0120351562046838\\
297	0.0120349423790742\\
298	0.0120347249193527\\
299	0.0120345037670158\\
300	0.0120342788627211\\
301	0.0120340501462785\\
302	0.0120338175566411\\
303	0.012033581031896\\
304	0.012033340509255\\
305	0.0120330959250457\\
306	0.012032847214702\\
307	0.0120325943127546\\
308	0.0120323371528221\\
309	0.012032075667601\\
310	0.0120318097888566\\
311	0.0120315394474131\\
312	0.0120312645731437\\
313	0.0120309850949613\\
314	0.0120307009408081\\
315	0.0120304120376458\\
316	0.0120301183114449\\
317	0.0120298196871748\\
318	0.0120295160887931\\
319	0.0120292074392344\\
320	0.0120288936603999\\
321	0.0120285746731454\\
322	0.0120282503972702\\
323	0.012027920751505\\
324	0.0120275856534996\\
325	0.0120272450198105\\
326	0.0120268987658871\\
327	0.0120265468060588\\
328	0.0120261890535205\\
329	0.0120258254203179\\
330	0.012025455817332\\
331	0.0120250801542631\\
332	0.0120246983396139\\
333	0.0120243102806713\\
334	0.0120239158834881\\
335	0.012023515052863\\
336	0.0120231076923196\\
337	0.0120226937040842\\
338	0.012022272989062\\
339	0.0120218454468127\\
340	0.0120214109755229\\
341	0.0120209694719786\\
342	0.0120205208315345\\
343	0.0120200649480815\\
344	0.0120196017140126\\
345	0.0120191310201861\\
346	0.0120186527558858\\
347	0.0120181668087795\\
348	0.012017673064874\\
349	0.0120171714084669\\
350	0.0120166617220958\\
351	0.0120161438864836\\
352	0.0120156177804806\\
353	0.0120150832810022\\
354	0.0120145402629637\\
355	0.0120139885992098\\
356	0.0120134281604413\\
357	0.0120128588151362\\
358	0.0120122804294677\\
359	0.0120116928672164\\
360	0.0120110959896799\\
361	0.0120104896555765\\
362	0.0120098737209457\\
363	0.0120092480390451\\
364	0.0120086124602429\\
365	0.0120079668319086\\
366	0.0120073109983002\\
367	0.0120066448004511\\
368	0.0120059680760553\\
369	0.0120052806593548\\
370	0.0120045823810283\\
371	0.0120038730680846\\
372	0.0120031525437631\\
373	0.0120024206274433\\
374	0.0120016771345673\\
375	0.0120009218765798\\
376	0.0120001546608895\\
377	0.0119993752908586\\
378	0.0119985835658262\\
379	0.011997779281172\\
380	0.0119969622284209\\
381	0.0119961321953737\\
382	0.0119952889661968\\
383	0.0119944323212464\\
384	0.0119935620359645\\
385	0.0119926778770069\\
386	0.0119917795909458\\
387	0.0119908624777247\\
388	0.0119899228214573\\
389	0.0119889688611929\\
390	0.0119880002896239\\
391	0.0119870167795216\\
392	0.0119860179826227\\
393	0.0119850035282624\\
394	0.0119839730213251\\
395	0.0119829260361939\\
396	0.0119818621230247\\
397	0.0119807808121115\\
398	0.0119796816122004\\
399	0.0119785640148432\\
400	0.0119774275006134\\
401	0.0119762715470793\\
402	0.0119750956427804\\
403	0.0119738993289339\\
404	0.0119726822097602\\
405	0.0119714438794811\\
406	0.011970183920554\\
407	0.0119689018950762\\
408	0.0119675973484922\\
409	0.0119662698082343\\
410	0.0119649187822249\\
411	0.0119635437572433\\
412	0.0119621441971765\\
413	0.0119607195412046\\
414	0.0119592692019899\\
415	0.0119577925637893\\
416	0.0119562889794543\\
417	0.011954757760998\\
418	0.0119531981416374\\
419	0.011951609340498\\
420	0.0119499905309098\\
421	0.0119483408280381\\
422	0.0119466592767789\\
423	0.0119449448252672\\
424	0.0119431962497694\\
425	0.0119414119050794\\
426	0.011939544330476\\
427	0.0119375833170973\\
428	0.0119355886718335\\
429	0.0119335597978319\\
430	0.0119314960902991\\
431	0.0119293969288426\\
432	0.0119272616306781\\
433	0.0119250895084772\\
434	0.0119228799029329\\
435	0.0119206321448455\\
436	0.011918345556092\\
437	0.0119160194517908\\
438	0.0119136531452393\\
439	0.0119112459549549\\
440	0.0119087971915859\\
441	0.0119063061586946\\
442	0.0119037721636091\\
443	0.0119011945212235\\
444	0.0118985725583592\\
445	0.0118959056183421\\
446	0.011893193064395\\
447	0.011890434276953\\
448	0.0118876286282989\\
449	0.011884775377955\\
450	0.01188187329274\\
451	0.0118789359570394\\
452	0.0118761187290379\\
453	0.0118732441583786\\
454	0.0118703106824494\\
455	0.0118673166788937\\
456	0.0118642604514183\\
457	0.0118611402226762\\
458	0.0118579541274538\\
459	0.0118547002077002\\
460	0.0118513764098386\\
461	0.0118479805695112\\
462	0.0118445104058764\\
463	0.0118409635122189\\
464	0.0118373373462516\\
465	0.0118336292212986\\
466	0.0118298363038166\\
467	0.0118259556373484\\
468	0.0118219842676887\\
469	0.0118179197503552\\
470	0.0118137621140052\\
471	0.01180950425274\\
472	0.011805139893548\\
473	0.0118006649908582\\
474	0.0117960752680153\\
475	0.0117913662006552\\
476	0.0117865329993363\\
477	0.0117815705898811\\
478	0.0117764735924932\\
479	0.0117712362997029\\
480	0.0117658526539322\\
481	0.0117603162295635\\
482	0.0117546202164943\\
483	0.0117487574127413\\
484	0.0117427202157511\\
485	0.0117365009825856\\
486	0.0117300922678666\\
487	0.0117234860181015\\
488	0.0117166705592115\\
489	0.0117096430946846\\
490	0.0117023924143927\\
491	0.0116948861812014\\
492	0.011687110203106\\
493	0.0116790493082437\\
494	0.0116706873039627\\
495	0.0116620067521509\\
496	0.0116529888513808\\
497	0.0116436132998294\\
498	0.0116338581469455\\
499	0.0116236996237757\\
500	0.0116131119545055\\
501	0.0116020671638132\\
502	0.0115905348320248\\
503	0.0115784818297717\\
504	0.0115658720170299\\
505	0.0115526658996702\\
506	0.0115388202319084\\
507	0.0115242875535829\\
508	0.0115090156703155\\
509	0.0114929470364955\\
510	0.0114760180503333\\
511	0.011458157917053\\
512	0.0114392873377584\\
513	0.0114193168259383\\
514	0.0113981416489737\\
515	0.0113756586727428\\
516	0.0113517080155585\\
517	0.011326059144954\\
518	0.0112984078901283\\
519	0.0112508252791484\\
520	0.01119879958361\\
521	0.0111437690299501\\
522	0.0110844158859491\\
523	0.0109311406455557\\
524	0.0107748825979114\\
525	0.0106155354733288\\
526	0.0104549496916452\\
527	0.0104095204059356\\
528	0.0103686165457181\\
529	0.0103328767538324\\
530	0.0103041717760264\\
531	0.010276904865161\\
532	0.0102503142032637\\
533	0.0102241546718795\\
534	0.010198960119134\\
535	0.0101734677595411\\
536	0.0101475515969584\\
537	0.0101211494948978\\
538	0.0100942094563074\\
539	0.0100666774092843\\
540	0.0100385289808489\\
541	0.0100097403033765\\
542	0.00998027217526259\\
543	0.00994971931279848\\
544	0.0099185772867922\\
545	0.00988611173148685\\
546	0.00985198060028316\\
547	0.0098178438786619\\
548	0.00978520062244087\\
549	0.00975339491219885\\
550	0.00972099753197099\\
551	0.0096879072496547\\
552	0.00965406861345961\\
553	0.00961945651714474\\
554	0.00958404815898549\\
555	0.00954782144334343\\
556	0.00951075368679787\\
557	0.00947281788527594\\
558	0.00943323273142403\\
559	0.0093918204066566\\
560	0.00934731903904993\\
561	0.00930667756038113\\
562	0.00926850736250741\\
563	0.00922964619803856\\
564	0.00918999094494944\\
565	0.00914951749255589\\
566	0.0091082046432926\\
567	0.00906603063041869\\
568	0.00902297257848852\\
569	0.00893326721015658\\
570	0.00859168342714514\\
571	0.00819114043826589\\
572	0.0081233624115439\\
573	0.00805450381049785\\
574	0.0079844630590578\\
575	0.00791321205721573\\
576	0.00784072167600322\\
577	0.00776696160314408\\
578	0.00769190028236148\\
579	0.00761550485243745\\
580	0.00753774108183466\\
581	0.007458573290789\\
582	0.00737796423854843\\
583	0.00729587491554892\\
584	0.00721226408110744\\
585	0.00712708712854452\\
586	0.00704029318686139\\
587	0.0069518176210836\\
588	0.00686156256306143\\
589	0.00676934637110726\\
590	0.00667477256501155\\
591	0.00657713355197374\\
592	0.00647481662499991\\
593	0.00636364764633835\\
594	0.00623271716130668\\
595	0.00605340994148281\\
596	0.00575058001197164\\
597	0.00512683753504545\\
598	0.00366374385960312\\
599	0\\
600	0\\
};
\addplot [color=mycolor6,solid,forget plot]
  table[row sep=crcr]{%
1	0.012076866751432\\
2	0.012076864946228\\
3	0.0120768631100084\\
4	0.0120768612422401\\
5	0.0120768593423807\\
6	0.0120768574098786\\
7	0.0120768554441726\\
8	0.012076853444692\\
9	0.0120768514108561\\
10	0.0120768493420744\\
11	0.012076847237746\\
12	0.0120768450972601\\
13	0.012076842919995\\
14	0.0120768407053184\\
15	0.0120768384525873\\
16	0.0120768361611475\\
17	0.0120768338303335\\
18	0.0120768314594685\\
19	0.0120768290478642\\
20	0.01207682659482\\
21	0.0120768240996237\\
22	0.0120768215615508\\
23	0.0120768189798641\\
24	0.0120768163538139\\
25	0.0120768136826377\\
26	0.0120768109655598\\
27	0.0120768082017911\\
28	0.0120768053905291\\
29	0.0120768025309573\\
30	0.0120767996222455\\
31	0.0120767966635488\\
32	0.0120767936540082\\
33	0.0120767905927497\\
34	0.0120767874788843\\
35	0.0120767843115078\\
36	0.0120767810897005\\
37	0.0120767778125267\\
38	0.0120767744790349\\
39	0.012076771088257\\
40	0.0120767676392084\\
41	0.0120767641308876\\
42	0.0120767605622759\\
43	0.012076756932337\\
44	0.0120767532400169\\
45	0.0120767494842434\\
46	0.0120767456639261\\
47	0.0120767417779556\\
48	0.0120767378252036\\
49	0.0120767338045224\\
50	0.0120767297147445\\
51	0.0120767255546825\\
52	0.0120767213231285\\
53	0.012076717018854\\
54	0.0120767126406092\\
55	0.0120767081871229\\
56	0.0120767036571022\\
57	0.0120766990492318\\
58	0.012076694362174\\
59	0.012076689594568\\
60	0.0120766847450298\\
61	0.0120766798121513\\
62	0.0120766747945007\\
63	0.0120766696906212\\
64	0.0120766644990311\\
65	0.0120766592182235\\
66	0.0120766538466653\\
67	0.0120766483827973\\
68	0.0120766428250334\\
69	0.0120766371717603\\
70	0.012076631421337\\
71	0.0120766255720944\\
72	0.0120766196223347\\
73	0.0120766135703309\\
74	0.0120766074143265\\
75	0.0120766011525345\\
76	0.0120765947831377\\
77	0.0120765883042874\\
78	0.0120765817141031\\
79	0.0120765750106723\\
80	0.0120765681920494\\
81	0.0120765612562555\\
82	0.0120765542012777\\
83	0.0120765470250685\\
84	0.0120765397255455\\
85	0.0120765323005901\\
86	0.0120765247480476\\
87	0.0120765170657263\\
88	0.0120765092513968\\
89	0.0120765013027913\\
90	0.0120764932176033\\
91	0.0120764849934865\\
92	0.0120764766280542\\
93	0.0120764681188789\\
94	0.0120764594634912\\
95	0.0120764506593796\\
96	0.012076441703989\\
97	0.0120764325947205\\
98	0.0120764233289308\\
99	0.0120764139039309\\
100	0.0120764043169855\\
101	0.0120763945653125\\
102	0.0120763846460816\\
103	0.0120763745564142\\
104	0.0120763642933818\\
105	0.0120763538540057\\
106	0.0120763432352558\\
107	0.01207633243405\\
108	0.0120763214472528\\
109	0.0120763102716749\\
110	0.012076298904072\\
111	0.0120762873411437\\
112	0.0120762755795329\\
113	0.0120762636158243\\
114	0.0120762514465438\\
115	0.0120762390681574\\
116	0.0120762264770697\\
117	0.0120762136696235\\
118	0.012076200642098\\
119	0.0120761873907084\\
120	0.012076173911604\\
121	0.0120761602008674\\
122	0.0120761462545135\\
123	0.0120761320684879\\
124	0.0120761176386657\\
125	0.0120761029608505\\
126	0.0120760880307727\\
127	0.0120760728440884\\
128	0.0120760573963782\\
129	0.0120760416831454\\
130	0.0120760256998146\\
131	0.0120760094417307\\
132	0.012075992904157\\
133	0.0120759760822737\\
134	0.0120759589711764\\
135	0.0120759415658745\\
136	0.0120759238612895\\
137	0.0120759058522534\\
138	0.0120758875335067\\
139	0.0120758688996972\\
140	0.0120758499453774\\
141	0.0120758306650031\\
142	0.0120758110529315\\
143	0.012075791103419\\
144	0.0120757708106194\\
145	0.0120757501685813\\
146	0.0120757291712467\\
147	0.0120757078124482\\
148	0.0120756860859068\\
149	0.0120756639852299\\
150	0.0120756415039082\\
151	0.0120756186353139\\
152	0.0120755953726978\\
153	0.0120755717091865\\
154	0.0120755476377799\\
155	0.0120755231513482\\
156	0.012075498242629\\
157	0.0120754729042245\\
158	0.012075447128598\\
159	0.0120754209080708\\
160	0.0120753942348191\\
161	0.0120753671008704\\
162	0.0120753394980998\\
163	0.0120753114182267\\
164	0.0120752828528108\\
165	0.0120752537932484\\
166	0.0120752242307682\\
167	0.0120751941564275\\
168	0.0120751635611079\\
169	0.012075132435511\\
170	0.0120751007701542\\
171	0.0120750685553661\\
172	0.0120750357812821\\
173	0.0120750024378402\\
174	0.0120749685147758\\
175	0.0120749340016178\\
176	0.0120748988876841\\
177	0.0120748631620766\\
178	0.0120748268136778\\
179	0.0120747898311461\\
180	0.0120747522029123\\
181	0.012074713917176\\
182	0.0120746749619026\\
183	0.0120746353248205\\
184	0.0120745949934195\\
185	0.0120745539549491\\
186	0.0120745121964184\\
187	0.0120744697045965\\
188	0.0120744264660147\\
189	0.0120743824669689\\
190	0.012074337693525\\
191	0.0120742921315251\\
192	0.0120742457665956\\
193	0.0120741985841573\\
194	0.0120741505694352\\
195	0.0120741017074662\\
196	0.0120740519831008\\
197	0.0120740013810086\\
198	0.0120739498857166\\
199	0.012073897481487\\
200	0.0120738441523073\\
201	0.0120737898818858\\
202	0.0120737346536473\\
203	0.0120736784507279\\
204	0.0120736212559707\\
205	0.0120735630519208\\
206	0.0120735038208204\\
207	0.0120734435446043\\
208	0.0120733822048946\\
209	0.0120733197829958\\
210	0.0120732562598897\\
211	0.0120731916162307\\
212	0.0120731258323401\\
213	0.0120730588882014\\
214	0.012072990763455\\
215	0.0120729214373928\\
216	0.0120728508889531\\
217	0.0120727790967152\\
218	0.0120727060388944\\
219	0.012072631693336\\
220	0.0120725560375106\\
221	0.0120724790485085\\
222	0.0120724007030342\\
223	0.012072320977401\\
224	0.012072239847526\\
225	0.012072157288924\\
226	0.0120720732767029\\
227	0.0120719877855578\\
228	0.0120719007897658\\
229	0.0120718122631808\\
230	0.0120717221792279\\
231	0.0120716305108983\\
232	0.0120715372307442\\
233	0.0120714423108734\\
234	0.012071345722944\\
235	0.0120712474381596\\
236	0.012071147427264\\
237	0.0120710456605363\\
238	0.0120709421077857\\
239	0.0120708367383467\\
240	0.0120707295210742\\
241	0.0120706204243385\\
242	0.0120705094160205\\
243	0.0120703964635068\\
244	0.0120702815336848\\
245	0.0120701645929382\\
246	0.0120700456071415\\
247	0.0120699245416553\\
248	0.0120698013613215\\
249	0.0120696760304578\\
250	0.0120695485128524\\
251	0.0120694187717586\\
252	0.0120692867698889\\
253	0.0120691524694088\\
254	0.0120690158319308\\
255	0.012068876818507\\
256	0.012068735389622\\
257	0.012068591505185\\
258	0.0120684451245209\\
259	0.0120682962063611\\
260	0.0120681447088335\\
261	0.012067990589451\\
262	0.0120678338050993\\
263	0.0120676743120241\\
264	0.0120675120658159\\
265	0.012067347021395\\
266	0.0120671791329933\\
267	0.0120670083541369\\
268	0.012066834637625\\
269	0.0120666579355091\\
270	0.0120664781990698\\
271	0.0120662953787928\\
272	0.0120661094243431\\
273	0.0120659202845386\\
274	0.0120657279073227\\
275	0.0120655322397366\\
276	0.0120653332278923\\
277	0.0120651308169491\\
278	0.0120649249510965\\
279	0.0120647155735271\\
280	0.0120645026264644\\
281	0.01206428605126\\
282	0.012064065788384\\
283	0.0120638417774153\\
284	0.0120636139570311\\
285	0.0120633822649979\\
286	0.0120631466381613\\
287	0.0120629070124367\\
288	0.0120626633227997\\
289	0.012062415503277\\
290	0.0120621634869367\\
291	0.0120619072058799\\
292	0.012061646591231\\
293	0.0120613815731299\\
294	0.0120611120807228\\
295	0.0120608380421543\\
296	0.0120605593845593\\
297	0.0120602760340553\\
298	0.0120599879157352\\
299	0.0120596949536598\\
300	0.0120593970708514\\
301	0.0120590941892876\\
302	0.0120587862298951\\
303	0.0120584731125445\\
304	0.0120581547560456\\
305	0.0120578310781425\\
306	0.0120575019955104\\
307	0.0120571674237524\\
308	0.0120568272773966\\
309	0.0120564814698954\\
310	0.0120561299136236\\
311	0.0120557725198792\\
312	0.0120554091988837\\
313	0.0120550398597848\\
314	0.0120546644106587\\
315	0.0120542827585149\\
316	0.0120538948093013\\
317	0.0120535004679109\\
318	0.0120530996381906\\
319	0.0120526922229503\\
320	0.0120522781239747\\
321	0.0120518572420367\\
322	0.0120514294769123\\
323	0.0120509947273977\\
324	0.0120505528913292\\
325	0.0120501038656046\\
326	0.0120496475462076\\
327	0.0120491838282352\\
328	0.0120487126059271\\
329	0.0120482337726995\\
330	0.0120477472211812\\
331	0.0120472528432538\\
332	0.0120467505300961\\
333	0.0120462401722314\\
334	0.0120457216595811\\
335	0.0120451948815214\\
336	0.0120446597269462\\
337	0.0120441160843349\\
338	0.0120435638418263\\
339	0.0120430028872988\\
340	0.0120424331084575\\
341	0.0120418543929279\\
342	0.0120412666283579\\
343	0.0120406697025282\\
344	0.0120400635034707\\
345	0.0120394479195972\\
346	0.0120388228398373\\
347	0.0120381881537882\\
348	0.0120375437518744\\
349	0.0120368895255208\\
350	0.0120362253673385\\
351	0.0120355511713236\\
352	0.0120348668330712\\
353	0.0120341722500051\\
354	0.0120334673216228\\
355	0.0120327519497585\\
356	0.012032026038864\\
357	0.0120312894963081\\
358	0.0120305422326963\\
359	0.0120297841622109\\
360	0.0120290152029721\\
361	0.0120282352774221\\
362	0.0120274443127307\\
363	0.0120266422412243\\
364	0.0120258290008383\\
365	0.0120250045355921\\
366	0.0120241687960872\\
367	0.0120233217400265\\
368	0.0120224633327547\\
369	0.0120215935478165\\
370	0.0120207123675309\\
371	0.0120198197835769\\
372	0.0120189157975863\\
373	0.0120180004217369\\
374	0.012017073679337\\
375	0.0120161356053925\\
376	0.0120151862471411\\
377	0.0120142256645391\\
378	0.0120132539306772\\
379	0.0120122711321011\\
380	0.0120112773690004\\
381	0.0120102727552214\\
382	0.0120092574180418\\
383	0.0120082314976216\\
384	0.0120071951460218\\
385	0.0120061485257101\\
386	0.0120050918078144\\
387	0.0120040251719482\\
388	0.0120029487940964\\
389	0.0120018628303055\\
390	0.0120007674303481\\
391	0.0119996627306729\\
392	0.0119985488451377\\
393	0.011997425852885\\
394	0.0119962937825089\\
395	0.0119951525912521\\
396	0.0119940021382087\\
397	0.0119928421501233\\
398	0.0119916721777353\\
399	0.0119904915401138\\
400	0.0119892992517129\\
401	0.0119880939164827\\
402	0.0119868735446339\\
403	0.0119856352281617\\
404	0.0119843764414293\\
405	0.0119830968517885\\
406	0.0119817961207085\\
407	0.0119804739037199\\
408	0.0119791298503657\\
409	0.0119777636041755\\
410	0.0119763748026687\\
411	0.0119749630773915\\
412	0.0119735280539953\\
413	0.0119720693523654\\
414	0.0119705865868082\\
415	0.0119690793663086\\
416	0.011967547294871\\
417	0.0119659899719639\\
418	0.0119644069930839\\
419	0.0119627979503317\\
420	0.0119611624331332\\
421	0.011959500028912\\
422	0.011957810323254\\
423	0.0119560928982393\\
424	0.0119543473252919\\
425	0.0119525731433739\\
426	0.0119507598264658\\
427	0.0119489048710412\\
428	0.0119470210362922\\
429	0.0119451078047242\\
430	0.0119431646339061\\
431	0.0119411909523468\\
432	0.0119391861479556\\
433	0.011937149574883\\
434	0.0119350805554316\\
435	0.0119329783688772\\
436	0.0119308422467828\\
437	0.0119286713679691\\
438	0.0119264648533749\\
439	0.011924221760646\\
440	0.0119219410743917\\
441	0.0119196216981448\\
442	0.0119172624472902\\
443	0.0119148620398431\\
444	0.011912419086394\\
445	0.0119099320790559\\
446	0.0119073993788302\\
447	0.0119048191992663\\
448	0.0119021895787571\\
449	0.011899508313721\\
450	0.0118967727505276\\
451	0.0118939666686319\\
452	0.0118909639850025\\
453	0.0118879074422213\\
454	0.0118847955020424\\
455	0.0118816267045609\\
456	0.0118783998005142\\
457	0.0118751135389869\\
458	0.0118717666282401\\
459	0.0118683576837656\\
460	0.0118648852376284\\
461	0.0118613479889332\\
462	0.0118577444982513\\
463	0.0118540732931081\\
464	0.0118503328656751\\
465	0.01184652167443\\
466	0.0118426381373628\\
467	0.011838680604638\\
468	0.0118346472641248\\
469	0.0118305358263164\\
470	0.0118263424675444\\
471	0.0118222282569492\\
472	0.0118181027549173\\
473	0.0118138810706521\\
474	0.0118095596161384\\
475	0.0118051345909209\\
476	0.0118006019615102\\
477	0.0117959574684708\\
478	0.0117911966009365\\
479	0.0117863145756489\\
480	0.011781306318024\\
481	0.0117761664405187\\
482	0.0117708892267801\\
483	0.0117654686386757\\
484	0.0117598984188618\\
485	0.0117541725345987\\
486	0.0117482869273429\\
487	0.0117422339262941\\
488	0.0117360003946928\\
489	0.0117295791986597\\
490	0.0117229616374423\\
491	0.0117161368740211\\
492	0.0117091026909894\\
493	0.0117018461542213\\
494	0.0116943361423598\\
495	0.0116865586014639\\
496	0.0116784985030909\\
497	0.011670139874852\\
498	0.0116614655694812\\
499	0.0116524571751639\\
500	0.0116430947865314\\
501	0.0116333566231288\\
502	0.011623219448945\\
503	0.0116126581133254\\
504	0.0116016453560485\\
505	0.0115901515885614\\
506	0.0115781446466673\\
507	0.0115655895123386\\
508	0.0115524479980813\\
509	0.0115386783866561\\
510	0.0115242350110054\\
511	0.0115090677645258\\
512	0.011493121532545\\
513	0.0114763355745899\\
514	0.0114586430323312\\
515	0.0114399690933532\\
516	0.0114202280920987\\
517	0.0113993226981617\\
518	0.0113771493036721\\
519	0.0113535717316111\\
520	0.0113284178717591\\
521	0.0113014744125244\\
522	0.0112723452388385\\
523	0.0112210253920977\\
524	0.011167116116309\\
525	0.0111100574647757\\
526	0.011047856114441\\
527	0.0108911870199347\\
528	0.0107314148744869\\
529	0.0105683625241118\\
530	0.010402582700154\\
531	0.0103433714657556\\
532	0.0102987208813198\\
533	0.0102594596397759\\
534	0.0102264173729237\\
535	0.0101964187124358\\
536	0.0101671495870776\\
537	0.0101384129584845\\
538	0.0101101347405005\\
539	0.0100823555888032\\
540	0.0100541461508269\\
541	0.0100254329653264\\
542	0.00999616110359175\\
543	0.00996627402406363\\
544	0.00993571988088713\\
545	0.00990431666479844\\
546	0.00987196571049124\\
547	0.00983898715188358\\
548	0.00980423349694136\\
549	0.00976841201801745\\
550	0.00973260294245857\\
551	0.00969848078071835\\
552	0.00966466365009313\\
553	0.00963019578557018\\
554	0.0095949827847948\\
555	0.0095589666279459\\
556	0.00952211973215971\\
557	0.00948441614080153\\
558	0.00944583115156459\\
559	0.00940611033237629\\
560	0.00936494180771247\\
561	0.00932091705477405\\
562	0.00927537531431877\\
563	0.00923449360961306\\
564	0.00919483593703645\\
565	0.00915443320240743\\
566	0.00911319714743514\\
567	0.00907110057975673\\
568	0.00902811986267383\\
569	0.00898423067887847\\
570	0.00890021985209273\\
571	0.00862745807166071\\
572	0.00812595060641661\\
573	0.00805450627922674\\
574	0.00798446323608824\\
575	0.00791321213868608\\
576	0.00784072171843002\\
577	0.00776696162475285\\
578	0.00769190029275072\\
579	0.00761550485706315\\
580	0.00753774108370465\\
581	0.00745857329145751\\
582	0.00737796423875186\\
583	0.00729587491559849\\
584	0.00721226408111606\\
585	0.00712708712854533\\
586	0.00704029318686139\\
587	0.00695181762108359\\
588	0.00686156256306143\\
589	0.00676934637110726\\
590	0.00667477256501156\\
591	0.00657713355197374\\
592	0.00647481662499991\\
593	0.00636364764633835\\
594	0.00623271716130669\\
595	0.00605340994148281\\
596	0.00575058001197164\\
597	0.00512683753504546\\
598	0.00366374385960312\\
599	0\\
600	0\\
};
\addplot [color=mycolor7,solid,forget plot]
  table[row sep=crcr]{%
1	0.0121672499298442\\
2	0.0121672467288244\\
3	0.0121672434734494\\
4	0.0121672401627974\\
5	0.0121672367959313\\
6	0.012167233371898\\
7	0.0121672298897283\\
8	0.0121672263484366\\
9	0.0121672227470206\\
10	0.0121672190844611\\
11	0.0121672153597217\\
12	0.0121672115717485\\
13	0.0121672077194695\\
14	0.0121672038017951\\
15	0.0121671998176168\\
16	0.0121671957658075\\
17	0.0121671916452213\\
18	0.0121671874546926\\
19	0.0121671831930363\\
20	0.0121671788590472\\
21	0.0121671744514998\\
22	0.0121671699691477\\
23	0.0121671654107239\\
24	0.0121671607749395\\
25	0.0121671560604841\\
26	0.012167151266025\\
27	0.0121671463902074\\
28	0.0121671414316531\\
29	0.0121671363889609\\
30	0.0121671312607061\\
31	0.0121671260454395\\
32	0.012167120741688\\
33	0.0121671153479532\\
34	0.0121671098627117\\
35	0.0121671042844141\\
36	0.0121670986114852\\
37	0.0121670928423231\\
38	0.0121670869752988\\
39	0.0121670810087559\\
40	0.01216707494101\\
41	0.0121670687703485\\
42	0.0121670624950297\\
43	0.0121670561132827\\
44	0.0121670496233066\\
45	0.0121670430232704\\
46	0.0121670363113119\\
47	0.0121670294855379\\
48	0.0121670225440231\\
49	0.0121670154848098\\
50	0.0121670083059073\\
51	0.0121670010052914\\
52	0.012166993580904\\
53	0.012166986030652\\
54	0.0121669783524075\\
55	0.0121669705440064\\
56	0.0121669626032483\\
57	0.012166954527896\\
58	0.0121669463156745\\
59	0.0121669379642704\\
60	0.0121669294713317\\
61	0.0121669208344665\\
62	0.0121669120512431\\
63	0.0121669031191886\\
64	0.0121668940357887\\
65	0.0121668847984868\\
66	0.0121668754046835\\
67	0.0121668658517354\\
68	0.012166856136955\\
69	0.0121668462576097\\
70	0.0121668362109208\\
71	0.0121668259940632\\
72	0.0121668156041641\\
73	0.0121668050383029\\
74	0.0121667942935097\\
75	0.0121667833667649\\
76	0.0121667722549983\\
77	0.012166760955088\\
78	0.0121667494638602\\
79	0.0121667377780875\\
80	0.0121667258944887\\
81	0.0121667138097274\\
82	0.0121667015204117\\
83	0.0121666890230924\\
84	0.0121666763142631\\
85	0.0121666633903583\\
86	0.012166650247753\\
87	0.0121666368827616\\
88	0.0121666232916368\\
89	0.0121666094705688\\
90	0.0121665954156838\\
91	0.0121665811230435\\
92	0.0121665665886437\\
93	0.0121665518084135\\
94	0.0121665367782137\\
95	0.0121665214938364\\
96	0.012166505951003\\
97	0.0121664901453638\\
98	0.0121664740724966\\
99	0.0121664577279053\\
100	0.0121664411070188\\
101	0.0121664242051899\\
102	0.0121664070176939\\
103	0.0121663895397274\\
104	0.0121663717664072\\
105	0.0121663536927684\\
106	0.0121663353137638\\
107	0.012166316624262\\
108	0.0121662976190465\\
109	0.0121662782928136\\
110	0.0121662586401719\\
111	0.0121662386556401\\
112	0.0121662183336458\\
113	0.0121661976685243\\
114	0.0121661766545165\\
115	0.012166155285768\\
116	0.0121661335563269\\
117	0.0121661114601427\\
118	0.0121660889910647\\
119	0.0121660661428399\\
120	0.0121660429091118\\
121	0.0121660192834186\\
122	0.0121659952591912\\
123	0.0121659708297518\\
124	0.0121659459883121\\
125	0.0121659207279713\\
126	0.0121658950417143\\
127	0.0121658689224099\\
128	0.0121658423628089\\
129	0.0121658153555424\\
130	0.0121657878931191\\
131	0.0121657599679242\\
132	0.0121657315722168\\
133	0.012165702698128\\
134	0.0121656733376586\\
135	0.0121656434826773\\
136	0.0121656131249184\\
137	0.0121655822559791\\
138	0.012165550867318\\
139	0.0121655189502521\\
140	0.0121654864959549\\
141	0.0121654534954536\\
142	0.0121654199396268\\
143	0.012165385819202\\
144	0.012165351124753\\
145	0.012165315846697\\
146	0.0121652799752923\\
147	0.0121652435006351\\
148	0.0121652064126569\\
149	0.0121651687011214\\
150	0.0121651303556214\\
151	0.0121650913655759\\
152	0.0121650517202267\\
153	0.0121650114086352\\
154	0.0121649704196785\\
155	0.0121649287420465\\
156	0.0121648863642376\\
157	0.0121648432745551\\
158	0.0121647994611032\\
159	0.0121647549117825\\
160	0.0121647096142858\\
161	0.0121646635560934\\
162	0.0121646167244682\\
163	0.0121645691064507\\
164	0.0121645206888533\\
165	0.0121644714582547\\
166	0.0121644214009938\\
167	0.0121643705031631\\
168	0.0121643187506019\\
169	0.0121642661288887\\
170	0.0121642126233333\\
171	0.0121641582189684\\
172	0.0121641029005403\\
173	0.0121640466524988\\
174	0.0121639894589868\\
175	0.0121639313038287\\
176	0.0121638721705174\\
177	0.0121638120422012\\
178	0.0121637509016689\\
179	0.0121636887313334\\
180	0.0121636255132146\\
181	0.0121635612289203\\
182	0.0121634958596253\\
183	0.0121634293860493\\
184	0.0121633617884322\\
185	0.0121632930465078\\
186	0.0121632231394751\\
187	0.0121631520459674\\
188	0.012163079744019\\
189	0.0121630062110298\\
190	0.0121629314237279\\
191	0.0121628553581313\\
192	0.0121627779895098\\
193	0.0121626992923517\\
194	0.0121626192403392\\
195	0.0121625378063422\\
196	0.0121624549624381\\
197	0.0121623706800006\\
198	0.0121622849301965\\
199	0.0121621976874229\\
200	0.0121621089257971\\
201	0.0121620186189936\\
202	0.0121619267402364\\
203	0.0121618332622918\\
204	0.012161738157461\\
205	0.0121616413975721\\
206	0.012161542953973\\
207	0.0121614427975231\\
208	0.0121613408985856\\
209	0.0121612372270193\\
210	0.0121611317521708\\
211	0.0121610244428658\\
212	0.012160915267401\\
213	0.0121608041935358\\
214	0.012160691188483\\
215	0.012160576218901\\
216	0.0121604592508843\\
217	0.0121603402499546\\
218	0.0121602191810522\\
219	0.0121600960085262\\
220	0.0121599706961255\\
221	0.0121598432069892\\
222	0.0121597135036372\\
223	0.0121595815479602\\
224	0.0121594473012101\\
225	0.0121593107239896\\
226	0.0121591717762425\\
227	0.0121590304172432\\
228	0.0121588866055861\\
229	0.0121587402991751\\
230	0.012158591455213\\
231	0.0121584400301901\\
232	0.0121582859798734\\
233	0.0121581292592954\\
234	0.0121579698227421\\
235	0.0121578076237419\\
236	0.0121576426150529\\
237	0.0121574747486517\\
238	0.0121573039757201\\
239	0.0121571302466333\\
240	0.0121569535109464\\
241	0.0121567737173816\\
242	0.0121565908138146\\
243	0.0121564047472612\\
244	0.0121562154638628\\
245	0.0121560229088727\\
246	0.0121558270266408\\
247	0.012155627760599\\
248	0.0121554250532457\\
249	0.01215521884613\\
250	0.0121550090798356\\
251	0.0121547956939642\\
252	0.0121545786271188\\
253	0.0121543578168857\\
254	0.0121541331998175\\
255	0.0121539047114142\\
256	0.0121536722861046\\
257	0.0121534358572276\\
258	0.0121531953570121\\
259	0.0121529507165576\\
260	0.0121527018658132\\
261	0.0121524487335575\\
262	0.0121521912473771\\
263	0.0121519293336456\\
264	0.0121516629175014\\
265	0.0121513919228266\\
266	0.0121511162722246\\
267	0.0121508358869984\\
268	0.0121505506871282\\
269	0.0121502605912499\\
270	0.0121499655166334\\
271	0.012149665379161\\
272	0.0121493600933067\\
273	0.0121490495721154\\
274	0.0121487337271832\\
275	0.0121484124686382\\
276	0.0121480857051216\\
277	0.0121477533437706\\
278	0.0121474152902003\\
279	0.0121470714484874\\
280	0.0121467217211516\\
281	0.0121463660091334\\
282	0.0121460042117708\\
283	0.0121456362267765\\
284	0.0121452619502152\\
285	0.0121448812764796\\
286	0.0121444940982674\\
287	0.0121441003065579\\
288	0.0121436997905878\\
289	0.0121432924378277\\
290	0.0121428781339582\\
291	0.0121424567628458\\
292	0.0121420282065189\\
293	0.012141592345144\\
294	0.0121411490570012\\
295	0.0121406982184603\\
296	0.0121402397039568\\
297	0.0121397733859678\\
298	0.0121392991349881\\
299	0.0121388168195068\\
300	0.012138326305983\\
301	0.0121378274588228\\
302	0.0121373201403563\\
303	0.0121368042108143\\
304	0.0121362795283057\\
305	0.012135745948796\\
306	0.0121352033260846\\
307	0.0121346515117843\\
308	0.0121340903553003\\
309	0.0121335197038105\\
310	0.0121329394022456\\
311	0.0121323492932712\\
312	0.0121317492172701\\
313	0.0121311390123254\\
314	0.0121305185142055\\
315	0.0121298875563495\\
316	0.0121292459698542\\
317	0.0121285935834632\\
318	0.0121279302235567\\
319	0.0121272557141434\\
320	0.0121265698768552\\
321	0.0121258725309424\\
322	0.0121251634932731\\
323	0.0121244425783338\\
324	0.0121237095982333\\
325	0.0121229643627103\\
326	0.0121222066791431\\
327	0.0121214363525644\\
328	0.0121206531856789\\
329	0.0121198569788863\\
330	0.0121190475303081\\
331	0.0121182246358208\\
332	0.0121173880890939\\
333	0.0121165376816341\\
334	0.0121156732028371\\
335	0.0121147944400461\\
336	0.0121139011786185\\
337	0.0121129932020016\\
338	0.0121120702918176\\
339	0.0121111322279596\\
340	0.0121101787886986\\
341	0.0121092097508031\\
342	0.0121082248896726\\
343	0.0121072239794858\\
344	0.0121062067933654\\
345	0.0121051731035605\\
346	0.012104122681649\\
347	0.0121030552987609\\
348	0.0121019707258257\\
349	0.0121008687338454\\
350	0.0120997490941957\\
351	0.0120986115789588\\
352	0.0120974559612898\\
353	0.0120962820158215\\
354	0.0120950895191102\\
355	0.0120938782501282\\
356	0.0120926479908055\\
357	0.0120913985266291\\
358	0.0120901296473035\\
359	0.0120888411474793\\
360	0.0120875328275592\\
361	0.012086204494588\\
362	0.0120848559632369\\
363	0.0120834870568932\\
364	0.0120820976088669\\
365	0.012080687463729\\
366	0.0120792564787962\\
367	0.0120778045257813\\
368	0.0120763314926286\\
369	0.0120748372855583\\
370	0.0120733218313474\\
371	0.0120717850798769\\
372	0.0120702270069822\\
373	0.0120686476176476\\
374	0.0120670469495931\\
375	0.0120654250773081\\
376	0.0120637821165994\\
377	0.012062118229727\\
378	0.012060433631219\\
379	0.0120587285944694\\
380	0.0120570034592443\\
381	0.0120552586402422\\
382	0.0120534946368853\\
383	0.0120517120445521\\
384	0.0120499115675062\\
385	0.0120480940338354\\
386	0.0120462604127716\\
387	0.012044411834686\\
388	0.0120425496144177\\
389	0.0120406752780346\\
390	0.0120387905969147\\
391	0.0120368976264479\\
392	0.0120349987516109\\
393	0.0120330967416673\\
394	0.0120311948169823\\
395	0.0120292967332153\\
396	0.0120274068732542\\
397	0.0120255303582682\\
398	0.0120236731943461\\
399	0.012021842462964\\
400	0.0120200465818283\\
401	0.0120182956815154\\
402	0.0120166021249311\\
403	0.0120149808186783\\
404	0.0120134051966291\\
405	0.0120118050173729\\
406	0.0120101800162975\\
407	0.012008529934249\\
408	0.0120068545182357\\
409	0.0120051535221704\\
410	0.0120034267076519\\
411	0.012001673844785\\
412	0.0119998947130367\\
413	0.0119980891021299\\
414	0.0119962568129681\\
415	0.0119943976585893\\
416	0.0119925114651317\\
417	0.0119905980727707\\
418	0.0119886573365104\\
419	0.0119866891271247\\
420	0.0119846933320899\\
421	0.0119826698564447\\
422	0.0119806186235152\\
423	0.0119785395754274\\
424	0.0119764326734344\\
425	0.0119742978988363\\
426	0.0119721352585366\\
427	0.0119699447634414\\
428	0.0119677264179037\\
429	0.0119654802499883\\
430	0.0119632063124931\\
431	0.0119609046837995\\
432	0.0119585754681763\\
433	0.0119562187958807\\
434	0.0119538348232256\\
435	0.0119514237322228\\
436	0.0119489857296504\\
437	0.0119465210453638\\
438	0.0119440299296241\\
439	0.0119415126491543\\
440	0.0119389694815311\\
441	0.0119364007074169\\
442	0.0119338066000023\\
443	0.0119311874108364\\
444	0.0119285433509625\\
445	0.0119258745659243\\
446	0.0119231811027\\
447	0.0119204628658186\\
448	0.0119177195584786\\
449	0.0119149506015577\\
450	0.0119121550169998\\
451	0.0119093284710531\\
452	0.0119064416637074\\
453	0.0119035232120867\\
454	0.0119005688437335\\
455	0.0118975724442471\\
456	0.0118945250215855\\
457	0.0118914243002373\\
458	0.0118882689223223\\
459	0.0118850574337212\\
460	0.0118817882765181\\
461	0.0118784598240128\\
462	0.0118750703146075\\
463	0.0118716178580883\\
464	0.0118681004207486\\
465	0.0118645158088869\\
466	0.0118608616469971\\
467	0.0118571353440785\\
468	0.011853334024718\\
469	0.0118494543430632\\
470	0.0118454918806472\\
471	0.0118413124685387\\
472	0.0118369871889903\\
473	0.0118325803753584\\
474	0.0118280893871499\\
475	0.0118235114289388\\
476	0.0118188434515252\\
477	0.0118140816567455\\
478	0.0118092224264765\\
479	0.0118042620391943\\
480	0.0117991965645841\\
481	0.01179402190825\\
482	0.0117887337676045\\
483	0.0117833276471263\\
484	0.0117777988265974\\
485	0.0117721421668744\\
486	0.0117663511629183\\
487	0.0117605354910614\\
488	0.0117547488071647\\
489	0.0117488018490309\\
490	0.011742687265597\\
491	0.011736397640294\\
492	0.011729926003789\\
493	0.0117232633521966\\
494	0.0117163987620694\\
495	0.0117093287142977\\
496	0.0117020467809634\\
497	0.0116945174299802\\
498	0.0116867268287663\\
499	0.0116786606184768\\
500	0.0116703055773252\\
501	0.0116616484001961\\
502	0.0116526620427354\\
503	0.0116433269590323\\
504	0.0116336221032504\\
505	0.0116235247802049\\
506	0.0116130104872893\\
507	0.0116020527128669\\
508	0.0115906227332482\\
509	0.0115786893916337\\
510	0.011566218853616\\
511	0.0115531743406828\\
512	0.0115395158224479\\
513	0.011525199490575\\
514	0.0115101771254553\\
515	0.0114943961763612\\
516	0.0114777989848389\\
517	0.0114603221763643\\
518	0.0114418955432169\\
519	0.0114224366743477\\
520	0.0114018578664005\\
521	0.0113800623340896\\
522	0.0113568983493816\\
523	0.0113322570736846\\
524	0.0113059704473728\\
525	0.0112778184763133\\
526	0.0112472523946274\\
527	0.0111943146120311\\
528	0.0111386976309552\\
529	0.0110798443997108\\
530	0.0110167833311675\\
531	0.010865529087143\\
532	0.0107024788368962\\
533	0.0105360981423921\\
534	0.0103667082895005\\
535	0.0102765910278294\\
536	0.010227222569959\\
537	0.0101827890109943\\
538	0.0101445658151101\\
539	0.010111371615044\\
540	0.0100789632735897\\
541	0.0100472067397383\\
542	0.010015832865925\\
543	0.00998498731553718\\
544	0.0099542235926231\\
545	0.00992295635263974\\
546	0.00989110830015612\\
547	0.00985862235006408\\
548	0.00982519568092307\\
549	0.00979095127016024\\
550	0.00975614293230419\\
551	0.00971929000415757\\
552	0.00968151833576426\\
553	0.00964363277383124\\
554	0.009607393326479\\
555	0.00957140391933581\\
556	0.00953471211834849\\
557	0.00949723437876546\\
558	0.00945889624815701\\
559	0.00941966748288923\\
560	0.00937951610888176\\
561	0.009337970523975\\
562	0.00929525080985312\\
563	0.00924896092881901\\
564	0.00920212158492495\\
565	0.00916014253809604\\
566	0.00911890457344965\\
567	0.00907690288812684\\
568	0.00903403143963363\\
569	0.00899025801317528\\
570	0.00894555664214493\\
571	0.00887271235043344\\
572	0.00869850024695905\\
573	0.00814933114564015\\
574	0.00798452580157154\\
575	0.00791321352136201\\
576	0.00784072225130142\\
577	0.00776696190277687\\
578	0.00769190043889123\\
579	0.00761550493014874\\
580	0.00753774111769362\\
581	0.00745857330586257\\
582	0.00737796424416958\\
583	0.00729587491733925\\
584	0.00721226408156575\\
585	0.00712708712862869\\
586	0.00704029318686975\\
587	0.00695181762108359\\
588	0.00686156256306142\\
589	0.00676934637110726\\
590	0.00667477256501155\\
591	0.00657713355197373\\
592	0.0064748166249999\\
593	0.00636364764633835\\
594	0.00623271716130668\\
595	0.0060534099414828\\
596	0.00575058001197163\\
597	0.00512683753504545\\
598	0.00366374385960312\\
599	0\\
600	0\\
};
\addplot [color=mycolor8,solid,forget plot]
  table[row sep=crcr]{%
1	0.012490636149034\\
2	0.0124906286463844\\
3	0.0124906210170971\\
4	0.0124906132590396\\
5	0.0124906053700436\\
6	0.0124905973479044\\
7	0.0124905891903803\\
8	0.0124905808951921\\
9	0.0124905724600222\\
10	0.0124905638825141\\
11	0.012490555160272\\
12	0.0124905462908596\\
13	0.0124905372718\\
14	0.0124905281005745\\
15	0.0124905187746225\\
16	0.0124905092913402\\
17	0.0124904996480803\\
18	0.0124904898421509\\
19	0.0124904798708151\\
20	0.0124904697312903\\
21	0.0124904594207471\\
22	0.0124904489363085\\
23	0.0124904382750496\\
24	0.0124904274339964\\
25	0.0124904164101251\\
26	0.012490405200361\\
27	0.0124903938015782\\
28	0.0124903822105984\\
29	0.0124903704241899\\
30	0.0124903584390671\\
31	0.0124903462518893\\
32	0.0124903338592598\\
33	0.0124903212577252\\
34	0.0124903084437742\\
35	0.0124902954138367\\
36	0.0124902821642829\\
37	0.0124902686914224\\
38	0.0124902549915029\\
39	0.0124902410607093\\
40	0.012490226895163\\
41	0.0124902124909201\\
42	0.0124901978439712\\
43	0.0124901829502394\\
44	0.0124901678055802\\
45	0.0124901524057793\\
46	0.0124901367465523\\
47	0.0124901208235431\\
48	0.0124901046323229\\
49	0.0124900881683888\\
50	0.0124900714271629\\
51	0.0124900544039905\\
52	0.0124900370941396\\
53	0.0124900194927989\\
54	0.0124900015950769\\
55	0.0124899833960004\\
56	0.0124899648905133\\
57	0.0124899460734752\\
58	0.0124899269396598\\
59	0.0124899074837537\\
60	0.0124898877003549\\
61	0.0124898675839714\\
62	0.0124898471290194\\
63	0.0124898263298224\\
64	0.0124898051806091\\
65	0.0124897836755119\\
66	0.0124897618085659\\
67	0.0124897395737063\\
68	0.0124897169647678\\
69	0.0124896939754822\\
70	0.0124896705994771\\
71	0.0124896468302738\\
72	0.0124896226612861\\
73	0.0124895980858181\\
74	0.0124895730970625\\
75	0.0124895476880991\\
76	0.0124895218518921\\
77	0.0124894955812893\\
78	0.0124894688690195\\
79	0.0124894417076904\\
80	0.0124894140897874\\
81	0.0124893860076709\\
82	0.0124893574535743\\
83	0.0124893284196023\\
84	0.0124892988977286\\
85	0.0124892688797938\\
86	0.0124892383575029\\
87	0.0124892073224237\\
88	0.012489175765984\\
89	0.0124891436794697\\
90	0.0124891110540223\\
91	0.0124890778806365\\
92	0.012489044150158\\
93	0.0124890098532808\\
94	0.0124889749805453\\
95	0.0124889395223349\\
96	0.0124889034688744\\
97	0.0124888668102267\\
98	0.0124888295362907\\
99	0.0124887916367982\\
100	0.0124887531013116\\
101	0.0124887139192208\\
102	0.0124886740797407\\
103	0.0124886335719083\\
104	0.0124885923845796\\
105	0.012488550506427\\
106	0.0124885079259363\\
107	0.0124884646314033\\
108	0.0124884206109314\\
109	0.012488375852428\\
110	0.0124883303436014\\
111	0.012488284071958\\
112	0.0124882370247985\\
113	0.0124881891892151\\
114	0.0124881405520878\\
115	0.0124880911000814\\
116	0.0124880408196416\\
117	0.0124879896969918\\
118	0.0124879377181297\\
119	0.0124878848688232\\
120	0.0124878311346073\\
121	0.0124877765007802\\
122	0.0124877209523993\\
123	0.0124876644742779\\
124	0.0124876070509808\\
125	0.0124875486668209\\
126	0.0124874893058548\\
127	0.012487428951879\\
128	0.0124873675884257\\
129	0.012487305198759\\
130	0.0124872417658701\\
131	0.0124871772724736\\
132	0.0124871117010029\\
133	0.012487045033606\\
134	0.0124869772521407\\
135	0.0124869083381709\\
136	0.012486838272961\\
137	0.0124867670374724\\
138	0.0124866946123579\\
139	0.0124866209779576\\
140	0.0124865461142939\\
141	0.0124864700010668\\
142	0.0124863926176488\\
143	0.0124863139430802\\
144	0.012486233956064\\
145	0.0124861526349609\\
146	0.0124860699577844\\
147	0.0124859859021952\\
148	0.0124859004454965\\
149	0.0124858135646288\\
150	0.0124857252361641\\
151	0.0124856354363014\\
152	0.0124855441408607\\
153	0.012485451325278\\
154	0.0124853569646003\\
155	0.0124852610334793\\
156	0.0124851635061672\\
157	0.0124850643565102\\
158	0.0124849635579443\\
159	0.012484861083489\\
160	0.0124847569057426\\
161	0.012484650996877\\
162	0.0124845433286323\\
163	0.012484433872312\\
164	0.012484322598778\\
165	0.0124842094784457\\
166	0.0124840944812796\\
167	0.0124839775767887\\
168	0.0124838587340227\\
169	0.0124837379215675\\
170	0.012483615107542\\
171	0.0124834902595951\\
172	0.0124833633449024\\
173	0.0124832343301642\\
174	0.0124831031816039\\
175	0.0124829698649667\\
176	0.0124828343455194\\
177	0.0124826965880507\\
178	0.012482556556873\\
179	0.0124824142158243\\
180	0.0124822695282726\\
181	0.0124821224571204\\
182	0.0124819729648115\\
183	0.0124818210133389\\
184	0.0124816665642556\\
185	0.0124815095786859\\
186	0.0124813500173406\\
187	0.0124811878405344\\
188	0.0124810230082078\\
189	0.0124808554799551\\
190	0.0124806852150662\\
191	0.0124805121726009\\
192	0.0124803363115452\\
193	0.012480157591189\\
194	0.0124799759721103\\
195	0.0124797914188318\\
196	0.0124796039071297\\
197	0.0124794134443417\\
198	0.0124792201261817\\
199	0.0124790234906164\\
200	0.0124788234627171\\
201	0.0124786199842485\\
202	0.0124784129959842\\
203	0.0124782024376896\\
204	0.0124779882481053\\
205	0.0124777703649293\\
206	0.0124775487248002\\
207	0.012477323263279\\
208	0.012477093914831\\
209	0.0124768606128075\\
210	0.0124766232894273\\
211	0.0124763818757571\\
212	0.0124761363016928\\
213	0.0124758864959394\\
214	0.0124756323859912\\
215	0.0124753738981114\\
216	0.0124751109573116\\
217	0.0124748434873302\\
218	0.0124745714106118\\
219	0.0124742946482848\\
220	0.0124740131201396\\
221	0.0124737267446061\\
222	0.0124734354387304\\
223	0.0124731391181521\\
224	0.0124728376970802\\
225	0.012472531088269\\
226	0.0124722192029938\\
227	0.0124719019510258\\
228	0.0124715792406066\\
229	0.0124712509784227\\
230	0.0124709170695786\\
231	0.0124705774175709\\
232	0.0124702319242603\\
233	0.0124698804898441\\
234	0.0124695230128285\\
235	0.0124691593899991\\
236	0.0124687895163922\\
237	0.0124684132852652\\
238	0.0124680305880657\\
239	0.0124676413144014\\
240	0.0124672453520085\\
241	0.0124668425867195\\
242	0.0124664329024314\\
243	0.0124660161810721\\
244	0.012465592302567\\
245	0.0124651611448049\\
246	0.0124647225836032\\
247	0.0124642764926722\\
248	0.0124638227435796\\
249	0.0124633612057136\\
250	0.0124628917462456\\
251	0.0124624142300926\\
252	0.0124619285198785\\
253	0.0124614344758952\\
254	0.0124609319560628\\
255	0.0124604208158893\\
256	0.0124599009084292\\
257	0.0124593720842425\\
258	0.0124588341913522\\
259	0.0124582870752011\\
260	0.0124577305786088\\
261	0.0124571645417272\\
262	0.0124565888019956\\
263	0.0124560031940958\\
264	0.0124554075499056\\
265	0.0124548016984523\\
266	0.0124541854658652\\
267	0.0124535586753277\\
268	0.0124529211470288\\
269	0.0124522726981133\\
270	0.0124516131426324\\
271	0.0124509422914922\\
272	0.0124502599524026\\
273	0.0124495659298245\\
274	0.0124488600249171\\
275	0.0124481420354833\\
276	0.0124474117559145\\
277	0.0124466689771347\\
278	0.0124459134865431\\
279	0.0124451450679556\\
280	0.012444363501545\\
281	0.0124435685637808\\
282	0.0124427600273675\\
283	0.0124419376611817\\
284	0.012441101230209\\
285	0.012440250495479\\
286	0.0124393852139998\\
287	0.0124385051386907\\
288	0.012437610018315\\
289	0.0124366995974106\\
290	0.0124357736162197\\
291	0.0124348318106178\\
292	0.0124338739120407\\
293	0.0124328996474114\\
294	0.0124319087390648\\
295	0.0124309009046711\\
296	0.0124298758571593\\
297	0.0124288333046373\\
298	0.0124277729503128\\
299	0.0124266944924109\\
300	0.0124255976240919\\
301	0.0124244820333668\\
302	0.0124233474030115\\
303	0.0124221934104797\\
304	0.0124210197278143\\
305	0.0124198260215568\\
306	0.0124186119526559\\
307	0.0124173771763737\\
308	0.0124161213421908\\
309	0.0124148440937093\\
310	0.0124135450685544\\
311	0.0124122238982738\\
312	0.0124108802082359\\
313	0.0124095136175251\\
314	0.0124081237388363\\
315	0.0124067101783664\\
316	0.0124052725357044\\
317	0.0124038104037189\\
318	0.0124023233684436\\
319	0.0124008110089607\\
320	0.0123992728972814\\
321	0.0123977085982243\\
322	0.0123961176692914\\
323	0.0123944996605409\\
324	0.0123928541144578\\
325	0.0123911805658216\\
326	0.0123894785415702\\
327	0.0123877475606619\\
328	0.0123859871339331\\
329	0.0123841967639533\\
330	0.0123823759448762\\
331	0.012380524162287\\
332	0.012378640893046\\
333	0.0123767256051278\\
334	0.0123747777574568\\
335	0.0123727967997372\\
336	0.0123707821722788\\
337	0.012368733305818\\
338	0.012366649621333\\
339	0.0123645305298533\\
340	0.0123623754322637\\
341	0.0123601837191021\\
342	0.01235795477035\\
343	0.0123556879552169\\
344	0.0123533826319166\\
345	0.0123510381474363\\
346	0.0123486538372969\\
347	0.0123462290253051\\
348	0.0123437630232958\\
349	0.0123412551308656\\
350	0.0123387046350953\\
351	0.0123361108102618\\
352	0.0123334729175395\\
353	0.0123307902046884\\
354	0.0123280619057314\\
355	0.0123252872406174\\
356	0.0123224654148719\\
357	0.0123195956192325\\
358	0.0123166770292713\\
359	0.0123137088050016\\
360	0.0123106900904701\\
361	0.0123076200133332\\
362	0.0123044976844193\\
363	0.0123013221972749\\
364	0.0122980926276977\\
365	0.0122948080332549\\
366	0.0122914674527901\\
367	0.01228806990592\\
368	0.0122846143925232\\
369	0.0122810998922244\\
370	0.0122775253638797\\
371	0.0122738897450669\\
372	0.0122701919515896\\
373	0.0122664308770044\\
374	0.0122626053921818\\
375	0.0122587143449171\\
376	0.012254756559608\\
377	0.0122507308370242\\
378	0.0122466359541946\\
379	0.0122424706644498\\
380	0.012238233697663\\
381	0.0122339237607445\\
382	0.0122295395384605\\
383	0.0122250796946701\\
384	0.0122205428741229\\
385	0.012215927705065\\
386	0.0122112328032047\\
387	0.0122064567784605\\
388	0.0122015982484379\\
389	0.0121966558695895\\
390	0.012191628246412\\
391	0.012186513987498\\
392	0.0121813117276193\\
393	0.0121760201323573\\
394	0.0121706378852504\\
395	0.01216516362561\\
396	0.0121595962902764\\
397	0.0121539352215202\\
398	0.012148180032555\\
399	0.0121423307017629\\
400	0.0121363877419276\\
401	0.0121303526073261\\
402	0.0121242289501595\\
403	0.0121180270678624\\
404	0.0121130536747471\\
405	0.0121104523023287\\
406	0.0121078083185364\\
407	0.01210512115808\\
408	0.0121023902599384\\
409	0.0120996150686293\\
410	0.0120967950356053\\
411	0.0120939296207917\\
412	0.0120910182942822\\
413	0.0120880605382211\\
414	0.0120850558489222\\
415	0.0120820037393318\\
416	0.0120789037420903\\
417	0.0120757554137948\\
418	0.0120725583418423\\
419	0.0120693121427133\\
420	0.0120660164652204\\
421	0.012062670995064\\
422	0.0120592754599662\\
423	0.0120558296354722\\
424	0.0120523333515216\\
425	0.0120487864998532\\
426	0.0120451890418797\\
427	0.0120415410171614\\
428	0.012037842556226\\
429	0.0120340938949395\\
430	0.0120302953893922\\
431	0.0120264475346821\\
432	0.0120225509918269\\
433	0.0120186066032817\\
434	0.0120146154068519\\
435	0.0120105786619206\\
436	0.0120064978795476\\
437	0.0120023748571068\\
438	0.0119982117182366\\
439	0.0119940109591714\\
440	0.0119897755030979\\
441	0.011985508763572\\
442	0.0119812147186527\\
443	0.0119768979982757\\
444	0.0119725639877198\\
445	0.0119682189507369\\
446	0.0119638701768505\\
447	0.0119595261585687\\
448	0.0119551968060185\\
449	0.0119508937093172\\
450	0.0119466304644099\\
451	0.0119424230905271\\
452	0.0119382905852469\\
453	0.0119342556219057\\
454	0.0119303457517199\\
455	0.0119265948528963\\
456	0.0119230429179789\\
457	0.0119194710046449\\
458	0.0119158476151377\\
459	0.011912172431185\\
460	0.0119084451603485\\
461	0.0119046655341048\\
462	0.011900833308525\\
463	0.0118969482645121\\
464	0.0118930102074505\\
465	0.0118890189659581\\
466	0.0118849743891262\\
467	0.0118808763408364\\
468	0.0118767246876381\\
469	0.011872519271243\\
470	0.0118682598441434\\
471	0.0118639175456341\\
472	0.0118595082453442\\
473	0.0118550461576761\\
474	0.0118505306681751\\
475	0.0118459609439259\\
476	0.0118413358455098\\
477	0.0118366537456646\\
478	0.0118319125774443\\
479	0.011827109630832\\
480	0.0118222413387052\\
481	0.0118173030358881\\
482	0.0118122886282725\\
483	0.0118071901462051\\
484	0.0118019970039349\\
485	0.011796694591656\\
486	0.0117912646905644\\
487	0.0117856076044617\\
488	0.0117796737059394\\
489	0.0117736075515142\\
490	0.0117674039004969\\
491	0.0117610571584606\\
492	0.0117545613395641\\
493	0.0117479101315451\\
494	0.0117410971372608\\
495	0.0117341162723479\\
496	0.01172696046367\\
497	0.0117196180438407\\
498	0.0117120827193301\\
499	0.0117043623189843\\
500	0.0116964204462794\\
501	0.0116882973672571\\
502	0.0116801970713572\\
503	0.0116718146623912\\
504	0.0116631326896446\\
505	0.0116541326997863\\
506	0.0116447950870197\\
507	0.0116350993701108\\
508	0.0116250235166712\\
509	0.0116145437033562\\
510	0.0116036342101411\\
511	0.0115922673594586\\
512	0.0115804138023002\\
513	0.0115680442561159\\
514	0.011555125983707\\
515	0.0115416112288776\\
516	0.0115274580251271\\
517	0.0115126205783209\\
518	0.0114970488302267\\
519	0.011480688169082\\
520	0.0114634785901327\\
521	0.0114453538456015\\
522	0.0114262353226874\\
523	0.0114060436987103\\
524	0.0113846838721412\\
525	0.0113620164291072\\
526	0.0113379306283637\\
527	0.0113123151655743\\
528	0.0112849995616158\\
529	0.0112557631806695\\
530	0.0112242777088344\\
531	0.0111717754723663\\
532	0.011114671874946\\
533	0.0110543070849606\\
534	0.0109898061926997\\
535	0.010857563607989\\
536	0.0106916036541881\\
537	0.0105221471154174\\
538	0.0103495172527279\\
539	0.0102113165682049\\
540	0.0101563443194975\\
541	0.0101056483225631\\
542	0.0100605936551069\\
543	0.0100224638458455\\
544	0.0099863417802264\\
545	0.0099509678935687\\
546	0.00991617660108073\\
547	0.0098816420316696\\
548	0.00984777035242589\\
549	0.00981361797051381\\
550	0.00977893593183029\\
551	0.00974344075680954\\
552	0.00970717472075751\\
553	0.00967020301537289\\
554	0.00963110590679491\\
555	0.00959105434587335\\
556	0.00955084132009492\\
557	0.00951190359616305\\
558	0.00947355548253387\\
559	0.00943445149762873\\
560	0.00939453485050878\\
561	0.00935369861231558\\
562	0.00931190536002771\\
563	0.009268550527102\\
564	0.00922417461261898\\
565	0.00917597825417376\\
566	0.00912753705130518\\
567	0.00908351322316861\\
568	0.00904056753481854\\
569	0.00899687644885152\\
570	0.00895228057669682\\
571	0.0089067404232288\\
572	0.00885045980593461\\
573	0.00873067527223201\\
574	0.00827491996561498\\
575	0.00791523178946964\\
576	0.00784073719470976\\
577	0.0077669654453666\\
578	0.00769190220158509\\
579	0.0076155058810726\\
580	0.00753774161250178\\
581	0.00745857354656985\\
582	0.007377964351367\\
583	0.00729587495987698\\
584	0.00721226409604604\\
585	0.00712708713261023\\
586	0.00704029318765944\\
587	0.00695181762116886\\
588	0.00686156256306143\\
589	0.00676934637110726\\
590	0.00667477256501155\\
591	0.00657713355197373\\
592	0.00647481662499992\\
593	0.00636364764633835\\
594	0.00623271716130669\\
595	0.00605340994148281\\
596	0.00575058001197164\\
597	0.00512683753504545\\
598	0.00366374385960312\\
599	0\\
600	0\\
};
\addplot [color=blue!25!mycolor7,solid,forget plot]
  table[row sep=crcr]{%
1	0.0126627991686105\\
2	0.0126627964209655\\
3	0.0126627936269292\\
4	0.0126627907857206\\
5	0.0126627878965456\\
6	0.0126627849585968\\
7	0.0126627819710535\\
8	0.0126627789330809\\
9	0.0126627758438304\\
10	0.0126627727024394\\
11	0.0126627695080304\\
12	0.0126627662597116\\
13	0.012662762956576\\
14	0.0126627595977018\\
15	0.0126627561821514\\
16	0.0126627527089716\\
17	0.0126627491771935\\
18	0.0126627455858316\\
19	0.0126627419338844\\
20	0.0126627382203332\\
21	0.0126627344441425\\
22	0.0126627306042596\\
23	0.0126627266996141\\
24	0.0126627227291175\\
25	0.0126627186916636\\
26	0.0126627145861273\\
27	0.012662710411365\\
28	0.0126627061662138\\
29	0.0126627018494916\\
30	0.0126626974599964\\
31	0.0126626929965063\\
32	0.0126626884577789\\
33	0.0126626838425512\\
34	0.0126626791495391\\
35	0.0126626743774371\\
36	0.0126626695249178\\
37	0.0126626645906321\\
38	0.012662659573208\\
39	0.0126626544712508\\
40	0.0126626492833428\\
41	0.0126626440080423\\
42	0.0126626386438841\\
43	0.0126626331893782\\
44	0.0126626276430101\\
45	0.01266262200324\\
46	0.0126626162685025\\
47	0.0126626104372064\\
48	0.0126626045077338\\
49	0.0126625984784401\\
50	0.0126625923476534\\
51	0.012662586113674\\
52	0.0126625797747739\\
53	0.0126625733291968\\
54	0.012662566775157\\
55	0.0126625601108392\\
56	0.0126625533343981\\
57	0.012662546443958\\
58	0.0126625394376119\\
59	0.0126625323134214\\
60	0.0126625250694159\\
61	0.0126625177035924\\
62	0.0126625102139147\\
63	0.0126625025983128\\
64	0.0126624948546829\\
65	0.012662486980886\\
66	0.0126624789747483\\
67	0.0126624708340597\\
68	0.0126624625565739\\
69	0.0126624541400075\\
70	0.0126624455820396\\
71	0.012662436880311\\
72	0.0126624280324236\\
73	0.01266241903594\\
74	0.0126624098883826\\
75	0.0126624005872332\\
76	0.0126623911299322\\
77	0.012662381513878\\
78	0.0126623717364262\\
79	0.0126623617948892\\
80	0.0126623516865353\\
81	0.012662341408588\\
82	0.0126623309582254\\
83	0.0126623203325796\\
84	0.0126623095287355\\
85	0.0126622985437306\\
86	0.0126622873745539\\
87	0.0126622760181454\\
88	0.0126622644713951\\
89	0.0126622527311424\\
90	0.0126622407941752\\
91	0.0126622286572292\\
92	0.0126622163169869\\
93	0.012662203770077\\
94	0.0126621910130735\\
95	0.0126621780424948\\
96	0.0126621648548028\\
97	0.0126621514464022\\
98	0.0126621378136396\\
99	0.0126621239528024\\
100	0.0126621098601182\\
101	0.0126620955317535\\
102	0.0126620809638133\\
103	0.0126620661523397\\
104	0.0126620510933112\\
105	0.0126620357826416\\
106	0.0126620202161793\\
107	0.012662004389706\\
108	0.0126619882989357\\
109	0.0126619719395141\\
110	0.0126619553070173\\
111	0.0126619383969505\\
112	0.0126619212047477\\
113	0.0126619037257699\\
114	0.0126618859553044\\
115	0.0126618678885638\\
116	0.0126618495206847\\
117	0.0126618308467269\\
118	0.0126618118616717\\
119	0.0126617925604217\\
120	0.0126617729377989\\
121	0.0126617529885438\\
122	0.0126617327073145\\
123	0.012661712088685\\
124	0.0126616911271448\\
125	0.0126616698170971\\
126	0.0126616481528576\\
127	0.012661626128654\\
128	0.0126616037386239\\
129	0.0126615809768143\\
130	0.0126615578371799\\
131	0.0126615343135822\\
132	0.0126615103997882\\
133	0.0126614860894689\\
134	0.0126614613761983\\
135	0.0126614362534524\\
136	0.0126614107146072\\
137	0.0126613847529382\\
138	0.0126613583616186\\
139	0.0126613315337183\\
140	0.0126613042622025\\
141	0.0126612765399306\\
142	0.0126612483596545\\
143	0.0126612197140177\\
144	0.0126611905955538\\
145	0.0126611609966852\\
146	0.0126611309097219\\
147	0.0126611003268598\\
148	0.0126610692401799\\
149	0.0126610376416467\\
150	0.0126610055231065\\
151	0.0126609728762867\\
152	0.0126609396927937\\
153	0.0126609059641122\\
154	0.012660871681603\\
155	0.0126608368365022\\
156	0.0126608014199192\\
157	0.0126607654228354\\
158	0.0126607288361026\\
159	0.0126606916504412\\
160	0.0126606538564384\\
161	0.0126606154445467\\
162	0.0126605764050815\\
163	0.0126605367282192\\
164	0.0126604964039951\\
165	0.0126604554223009\\
166	0.0126604137728819\\
167	0.0126603714453341\\
168	0.0126603284291013\\
169	0.0126602847134714\\
170	0.012660240287572\\
171	0.0126601951403666\\
172	0.0126601492606492\\
173	0.0126601026370384\\
174	0.0126600552579714\\
175	0.0126600071116959\\
176	0.0126599581862623\\
177	0.0126599084695136\\
178	0.0126598579490742\\
179	0.0126598066123372\\
180	0.0126597544464499\\
181	0.0126597014382966\\
182	0.0126596475744791\\
183	0.0126595928412948\\
184	0.0126595372247106\\
185	0.0126594807103342\\
186	0.0126594232833799\\
187	0.0126593649286309\\
188	0.0126593056303967\\
189	0.0126592453724657\\
190	0.0126591841380556\\
191	0.0126591219097689\\
192	0.0126590586695676\\
193	0.0126589943988123\\
194	0.0126589290784635\\
195	0.0126588626896697\\
196	0.0126587952151837\\
197	0.0126587266422602\\
198	0.012658656966977\\
199	0.0126585860957209\\
200	0.0126585140048563\\
201	0.0126584406735843\\
202	0.0126583660807569\\
203	0.0126582902048715\\
204	0.0126582130240652\\
205	0.0126581345161083\\
206	0.0126580546583992\\
207	0.0126579734279578\\
208	0.0126578908014192\\
209	0.0126578067550279\\
210	0.0126577212646309\\
211	0.0126576343056717\\
212	0.0126575458531834\\
213	0.0126574558817819\\
214	0.0126573643656596\\
215	0.0126572712785779\\
216	0.0126571765938606\\
217	0.0126570802843864\\
218	0.0126569823225821\\
219	0.0126568826804148\\
220	0.0126567813293844\\
221	0.0126566782405163\\
222	0.0126565733843536\\
223	0.0126564667309488\\
224	0.0126563582498562\\
225	0.0126562479101238\\
226	0.0126561356802845\\
227	0.0126560215283485\\
228	0.012655905421794\\
229	0.0126557873275588\\
230	0.0126556672120317\\
231	0.0126555450410431\\
232	0.0126554207798562\\
233	0.0126552943931575\\
234	0.0126551658450475\\
235	0.0126550350990311\\
236	0.0126549021180079\\
237	0.0126547668642622\\
238	0.012654629299453\\
239	0.0126544893846038\\
240	0.0126543470800925\\
241	0.0126542023456403\\
242	0.0126540551403015\\
243	0.0126539054224525\\
244	0.0126537531497808\\
245	0.0126535982792737\\
246	0.0126534407672071\\
247	0.0126532805691341\\
248	0.012653117639873\\
249	0.0126529519334958\\
250	0.0126527834033159\\
251	0.0126526120018764\\
252	0.0126524376809374\\
253	0.0126522603914633\\
254	0.012652080083611\\
255	0.0126518967067161\\
256	0.0126517102092807\\
257	0.0126515205389597\\
258	0.0126513276425479\\
259	0.0126511314659663\\
260	0.0126509319542486\\
261	0.0126507290515273\\
262	0.0126505227010196\\
263	0.0126503128450135\\
264	0.0126500994248532\\
265	0.0126498823809246\\
266	0.0126496616526407\\
267	0.0126494371784264\\
268	0.0126492088957035\\
269	0.0126489767408751\\
270	0.0126487406493103\\
271	0.0126485005553278\\
272	0.0126482563921802\\
273	0.0126480080920375\\
274	0.01264775558597\\
275	0.0126474988039321\\
276	0.0126472376747443\\
277	0.012646972126076\\
278	0.0126467020844278\\
279	0.0126464274751133\\
280	0.0126461482222406\\
281	0.0126458642486937\\
282	0.0126455754761138\\
283	0.0126452818248799\\
284	0.0126449832140889\\
285	0.0126446795615367\\
286	0.012644370783697\\
287	0.0126440567957016\\
288	0.012643737511319\\
289	0.0126434128429339\\
290	0.0126430827015251\\
291	0.012642746996644\\
292	0.0126424056363922\\
293	0.0126420585273991\\
294	0.0126417055747988\\
295	0.0126413466822065\\
296	0.012640981751695\\
297	0.0126406106837703\\
298	0.012640233377347\\
299	0.0126398497297228\\
300	0.0126394596365532\\
301	0.0126390629918254\\
302	0.0126386596878313\\
303	0.0126382496151403\\
304	0.0126378326625718\\
305	0.0126374087171665\\
306	0.0126369776641577\\
307	0.0126365393869412\\
308	0.0126360937670452\\
309	0.0126356406840993\\
310	0.0126351800158024\\
311	0.0126347116378901\\
312	0.0126342354241012\\
313	0.0126337512461437\\
314	0.0126332589736588\\
315	0.0126327584741849\\
316	0.0126322496131201\\
317	0.0126317322536837\\
318	0.0126312062568764\\
319	0.0126306714814394\\
320	0.0126301277838115\\
321	0.0126295750180859\\
322	0.0126290130359642\\
323	0.0126284416867095\\
324	0.0126278608170973\\
325	0.012627270271365\\
326	0.0126266698911585\\
327	0.0126260595154773\\
328	0.012625438980617\\
329	0.0126248081201088\\
330	0.012624166764657\\
331	0.0126235147420725\\
332	0.0126228518772038\\
333	0.0126221779918646\\
334	0.0126214929047562\\
335	0.0126207964313878\\
336	0.0126200883839903\\
337	0.0126193685714266\\
338	0.0126186367990958\\
339	0.0126178928688317\\
340	0.0126171365787953\\
341	0.0126163677233595\\
342	0.0126155860929871\\
343	0.0126147914741002\\
344	0.0126139836489396\\
345	0.0126131623954158\\
346	0.0126123274869473\\
347	0.0126114786922876\\
348	0.0126106157753389\\
349	0.0126097384949502\\
350	0.0126088466047003\\
351	0.0126079398526609\\
352	0.0126070179811412\\
353	0.0126060807264093\\
354	0.0126051278183887\\
355	0.0126041589803275\\
356	0.0126031739284356\\
357	0.0126021723714884\\
358	0.0126011540103906\\
359	0.0126001185376966\\
360	0.0125990656370813\\
361	0.0125979949827559\\
362	0.0125969062388204\\
363	0.0125957990585441\\
364	0.0125946730835669\\
365	0.0125935279430064\\
366	0.0125923632524616\\
367	0.0125911786128946\\
368	0.0125899736093762\\
369	0.0125887478096708\\
370	0.0125875007626403\\
371	0.0125862319964359\\
372	0.0125849410164452\\
373	0.0125836273029556\\
374	0.0125822903084857\\
375	0.0125809294547307\\
376	0.0125795441290541\\
377	0.0125781336804476\\
378	0.0125766974148622\\
379	0.0125752345897976\\
380	0.0125737444080068\\
381	0.0125722260101416\\
382	0.0125706784661053\\
383	0.0125691007647705\\
384	0.0125674918014544\\
385	0.0125658503618055\\
386	0.0125641750984852\\
387	0.0125624644897927\\
388	0.012560716745713\\
389	0.0125589295484406\\
390	0.012557101861502\\
391	0.0125552316388122\\
392	0.012553316370943\\
393	0.0125513533295656\\
394	0.0125493397699444\\
395	0.0125472736080522\\
396	0.0125451474116995\\
397	0.0125429522436535\\
398	0.0125406813320105\\
399	0.0125383266460232\\
400	0.0125358785399175\\
401	0.0125333251353853\\
402	0.0125306509543421\\
403	0.0125278331046089\\
404	0.0125238189354777\\
405	0.0125177247103424\\
406	0.0125115283167343\\
407	0.0125052279616382\\
408	0.0124988218172144\\
409	0.012492308020313\\
410	0.0124856846719898\\
411	0.0124789498369781\\
412	0.0124721015430077\\
413	0.0124651377797076\\
414	0.0124580564964506\\
415	0.0124508555975605\\
416	0.0124435329309816\\
417	0.0124360862607481\\
418	0.0124285131994143\\
419	0.0124208115287634\\
420	0.0124129790141578\\
421	0.0124050133710544\\
422	0.0123969122630009\\
423	0.0123886732997546\\
424	0.0123802940360336\\
425	0.0123717719724966\\
426	0.0123631045637948\\
427	0.0123542892476439\\
428	0.0123453233451612\\
429	0.0123362040273265\\
430	0.0123269283548499\\
431	0.0123174932190651\\
432	0.0123078951951936\\
433	0.0122981310331372\\
434	0.0122881977596803\\
435	0.0122780923491395\\
436	0.0122678117263272\\
437	0.0122573527712034\\
438	0.012246712325635\\
439	0.012235887199955\\
440	0.0122248741689738\\
441	0.0122136699878926\\
442	0.012202271417721\\
443	0.0121906752468003\\
444	0.0121788783183176\\
445	0.0121668775653097\\
446	0.012154670055046\\
447	0.0121422530451988\\
448	0.0121296240549782\\
449	0.0121167809557951\\
450	0.0121037220893232\\
451	0.0120904464307419\\
452	0.0120769538495803\\
453	0.0120632456526816\\
454	0.0120493260736496\\
455	0.0120352072503406\\
456	0.0120209269542652\\
457	0.0120141474630193\\
458	0.0120083362709143\\
459	0.0120024389967806\\
460	0.0119964557937994\\
461	0.0119903870880293\\
462	0.0119842335388187\\
463	0.0119779960722039\\
464	0.011971675919305\\
465	0.0119652746603985\\
466	0.0119587942756917\\
467	0.011952237203908\\
468	0.0119456064099843\\
469	0.0119389054636415\\
470	0.0119321386323924\\
471	0.0119253110049611\\
472	0.0119184285521223\\
473	0.0119114983090526\\
474	0.0119045285924051\\
475	0.0118975292233623\\
476	0.0118905117983631\\
477	0.0118834900224544\\
478	0.0118764800886545\\
479	0.0118695011579454\\
480	0.0118625759625213\\
481	0.0118557315769333\\
482	0.0118490004670204\\
483	0.011842421972757\\
484	0.0118360444205979\\
485	0.0118299272661661\\
486	0.0118240711065785\\
487	0.011818093823366\\
488	0.0118119836632568\\
489	0.0118057712209277\\
490	0.0117994541064894\\
491	0.0117930297142576\\
492	0.011786495183732\\
493	0.0117798473523641\\
494	0.0117730826932324\\
495	0.0117661972219266\\
496	0.011759186416015\\
497	0.0117520453057056\\
498	0.0117447684750535\\
499	0.0117373515372546\\
500	0.0117297811306234\\
501	0.0117219943323928\\
502	0.0117138105032538\\
503	0.0117054677285961\\
504	0.011696953652332\\
505	0.0116882207881493\\
506	0.0116792440778941\\
507	0.0116699905659572\\
508	0.0116604422902908\\
509	0.0116505823945872\\
510	0.0116403918843002\\
511	0.0116298506306759\\
512	0.0116189369472603\\
513	0.011607625654809\\
514	0.0115959861465712\\
515	0.0115841790707076\\
516	0.0115718780158105\\
517	0.0115590485282023\\
518	0.0115456531052509\\
519	0.0115316508974598\\
520	0.0115169974011455\\
521	0.0115016441650013\\
522	0.0114855387896299\\
523	0.0114686243726267\\
524	0.0114508390277682\\
525	0.0114321136290695\\
526	0.0114123788587613\\
527	0.0113915364004079\\
528	0.0113694545366788\\
529	0.0113460319457778\\
530	0.011321156324708\\
531	0.0112947145386821\\
532	0.0112665397799165\\
533	0.0112364169914739\\
534	0.0112040357446609\\
535	0.0111551667727741\\
536	0.011096919169057\\
537	0.0110354652627191\\
538	0.0109700878101411\\
539	0.0108708039195637\\
540	0.0107023961700645\\
541	0.0105303385885915\\
542	0.0103548327473627\\
543	0.010175555371547\\
544	0.0100891311152285\\
545	0.0100313394797322\\
546	0.00997814790714204\\
547	0.0099310988020503\\
548	0.00989068920853819\\
549	0.00985091345286371\\
550	0.0098118748301417\\
551	0.00977342506205893\\
552	0.0097353412845486\\
553	0.00969803635282579\\
554	0.00965987302914852\\
555	0.00962100115308273\\
556	0.00958148348237549\\
557	0.00954011250497676\\
558	0.00949743746264783\\
559	0.00945456395837911\\
560	0.00941223298613685\\
561	0.00937127836630832\\
562	0.00932951898304717\\
563	0.00928692394318492\\
564	0.0092433602842606\\
565	0.00919817203828855\\
566	0.00915199392923645\\
567	0.00910223641591428\\
568	0.00905183590070086\\
569	0.00900473865587824\\
570	0.00895991368126304\\
571	0.00891439948712266\\
572	0.00886794299834819\\
573	0.00882049631800914\\
574	0.00871402909880057\\
575	0.00843939249643411\\
576	0.00791197803111671\\
577	0.00776725103636003\\
578	0.00769192980440901\\
579	0.00761551684140999\\
580	0.00753774755897855\\
581	0.00745857676192871\\
582	0.00737796598891016\\
583	0.00729587572800404\\
584	0.00721226441867625\\
585	0.00712708724941016\\
586	0.00704029322199904\\
587	0.00695181762849465\\
588	0.00686156256391835\\
589	0.00676934637110726\\
590	0.00667477256501155\\
591	0.00657713355197375\\
592	0.00647481662499991\\
593	0.00636364764633835\\
594	0.00623271716130668\\
595	0.00605340994148281\\
596	0.00575058001197164\\
597	0.00512683753504545\\
598	0.00366374385960312\\
599	0\\
600	0\\
};
\addplot [color=mycolor9,solid,forget plot]
  table[row sep=crcr]{%
1	0.0127187035361118\\
2	0.012718702153434\\
3	0.0127187007474257\\
4	0.0127186993176947\\
5	0.0127186978638424\\
6	0.0127186963854635\\
7	0.012718694882146\\
8	0.0127186933534709\\
9	0.0127186917990122\\
10	0.0127186902183371\\
11	0.0127186886110052\\
12	0.012718686976569\\
13	0.0127186853145735\\
14	0.012718683624556\\
15	0.0127186819060464\\
16	0.0127186801585664\\
17	0.0127186783816301\\
18	0.0127186765747432\\
19	0.0127186747374033\\
20	0.0127186728690999\\
21	0.0127186709693135\\
22	0.0127186690375165\\
23	0.0127186670731722\\
24	0.0127186650757351\\
25	0.0127186630446506\\
26	0.0127186609793551\\
27	0.0127186588792753\\
28	0.0127186567438287\\
29	0.0127186545724229\\
30	0.012718652364456\\
31	0.0127186501193159\\
32	0.0127186478363803\\
33	0.0127186455150169\\
34	0.0127186431545826\\
35	0.0127186407544239\\
36	0.0127186383138764\\
37	0.0127186358322648\\
38	0.0127186333089026\\
39	0.0127186307430919\\
40	0.0127186281341233\\
41	0.0127186254812759\\
42	0.0127186227838166\\
43	0.0127186200410005\\
44	0.0127186172520702\\
45	0.0127186144162561\\
46	0.0127186115327755\\
47	0.0127186086008334\\
48	0.0127186056196213\\
49	0.0127186025883175\\
50	0.0127185995060869\\
51	0.0127185963720807\\
52	0.0127185931854361\\
53	0.0127185899452761\\
54	0.0127185866507094\\
55	0.0127185833008303\\
56	0.0127185798947179\\
57	0.0127185764314365\\
58	0.012718572910035\\
59	0.0127185693295469\\
60	0.0127185656889896\\
61	0.0127185619873649\\
62	0.012718558223658\\
63	0.0127185543968376\\
64	0.0127185505058558\\
65	0.0127185465496475\\
66	0.0127185425271304\\
67	0.0127185384372043\\
68	0.0127185342787515\\
69	0.0127185300506361\\
70	0.0127185257517036\\
71	0.0127185213807809\\
72	0.0127185169366761\\
73	0.0127185124181777\\
74	0.0127185078240548\\
75	0.0127185031530566\\
76	0.0127184984039122\\
77	0.0127184935753302\\
78	0.0127184886659982\\
79	0.012718483674583\\
80	0.0127184785997296\\
81	0.0127184734400617\\
82	0.0127184681941806\\
83	0.0127184628606653\\
84	0.0127184574380719\\
85	0.0127184519249336\\
86	0.0127184463197601\\
87	0.0127184406210372\\
88	0.0127184348272269\\
89	0.0127184289367663\\
90	0.0127184229480678\\
91	0.0127184168595186\\
92	0.0127184106694803\\
93	0.0127184043762886\\
94	0.0127183979782526\\
95	0.0127183914736548\\
96	0.0127183848607506\\
97	0.012718378137768\\
98	0.0127183713029067\\
99	0.0127183643543384\\
100	0.0127183572902059\\
101	0.0127183501086229\\
102	0.0127183428076735\\
103	0.0127183353854117\\
104	0.0127183278398614\\
105	0.0127183201690151\\
106	0.0127183123708346\\
107	0.0127183044432494\\
108	0.0127182963841571\\
109	0.0127182881914226\\
110	0.0127182798628776\\
111	0.0127182713963202\\
112	0.0127182627895144\\
113	0.0127182540401898\\
114	0.0127182451460409\\
115	0.0127182361047264\\
116	0.0127182269138694\\
117	0.012718217571056\\
118	0.0127182080738355\\
119	0.0127181984197195\\
120	0.0127181886061816\\
121	0.0127181786306566\\
122	0.0127181684905399\\
123	0.0127181581831876\\
124	0.0127181477059148\\
125	0.0127181370559963\\
126	0.0127181262306648\\
127	0.0127181152271113\\
128	0.0127181040424836\\
129	0.0127180926738865\\
130	0.0127180811183804\\
131	0.0127180693729812\\
132	0.0127180574346594\\
133	0.0127180453003393\\
134	0.0127180329668983\\
135	0.0127180204311662\\
136	0.0127180076899248\\
137	0.0127179947399063\\
138	0.0127179815777932\\
139	0.012717968200217\\
140	0.0127179546037576\\
141	0.0127179407849422\\
142	0.0127179267402445\\
143	0.0127179124660833\\
144	0.0127178979588221\\
145	0.0127178832147672\\
146	0.0127178682301672\\
147	0.0127178530012115\\
148	0.0127178375240289\\
149	0.0127178217946866\\
150	0.0127178058091882\\
151	0.0127177895634728\\
152	0.0127177730534127\\
153	0.0127177562748124\\
154	0.0127177392234061\\
155	0.0127177218948562\\
156	0.012717704284751\\
157	0.0127176863886023\\
158	0.0127176682018436\\
159	0.0127176497198269\\
160	0.0127176309378202\\
161	0.0127176118510048\\
162	0.0127175924544719\\
163	0.0127175727432191\\
164	0.0127175527121475\\
165	0.0127175323560569\\
166	0.0127175116696424\\
167	0.0127174906474899\\
168	0.0127174692840711\\
169	0.0127174475737387\\
170	0.0127174255107212\\
171	0.0127174030891168\\
172	0.0127173803028879\\
173	0.0127173571458545\\
174	0.0127173336116874\\
175	0.0127173096939014\\
176	0.0127172853858477\\
177	0.0127172606807064\\
178	0.0127172355714785\\
179	0.0127172100509778\\
180	0.0127171841118226\\
181	0.0127171577464273\\
182	0.0127171309469943\\
183	0.0127171037055057\\
184	0.0127170760137159\\
185	0.0127170478631442\\
186	0.0127170192450689\\
187	0.0127169901505219\\
188	0.0127169605702862\\
189	0.012716930494894\\
190	0.0127168999146298\\
191	0.012716868819537\\
192	0.0127168371994322\\
193	0.0127168050439309\\
194	0.0127167723424889\\
195	0.0127167390844593\\
196	0.0127167052591334\\
197	0.0127166708556113\\
198	0.012716635861977\\
199	0.0127166002681403\\
200	0.012716564063932\\
201	0.0127165272390142\\
202	0.0127164897828772\\
203	0.0127164516848373\\
204	0.0127164129340335\\
205	0.0127163735194248\\
206	0.0127163334297875\\
207	0.012716292653712\\
208	0.0127162511795998\\
209	0.0127162089956605\\
210	0.012716166089909\\
211	0.0127161224501618\\
212	0.0127160780640342\\
213	0.0127160329189369\\
214	0.0127159870020729\\
215	0.0127159403004337\\
216	0.0127158928007965\\
217	0.0127158444897201\\
218	0.012715795353542\\
219	0.0127157453783743\\
220	0.0127156945501004\\
221	0.0127156428543712\\
222	0.0127155902766013\\
223	0.0127155368019653\\
224	0.0127154824153938\\
225	0.0127154271015695\\
226	0.0127153708449233\\
227	0.0127153136296303\\
228	0.0127152554396052\\
229	0.0127151962584989\\
230	0.0127151360696935\\
231	0.0127150748562983\\
232	0.0127150126011455\\
233	0.0127149492867857\\
234	0.0127148848954829\\
235	0.0127148194092109\\
236	0.0127147528096475\\
237	0.0127146850781705\\
238	0.0127146161958528\\
239	0.012714546143457\\
240	0.0127144749014313\\
241	0.0127144024499037\\
242	0.012714328768677\\
243	0.0127142538372241\\
244	0.0127141776346823\\
245	0.0127141001398477\\
246	0.0127140213311706\\
247	0.0127139411867493\\
248	0.0127138596843244\\
249	0.0127137768012738\\
250	0.0127136925146064\\
251	0.0127136068009562\\
252	0.0127135196365767\\
253	0.0127134309973346\\
254	0.0127133408587036\\
255	0.0127132491957584\\
256	0.0127131559831682\\
257	0.0127130611951902\\
258	0.0127129648056632\\
259	0.0127128667880008\\
260	0.0127127671151846\\
261	0.0127126657597574\\
262	0.012712562693816\\
263	0.0127124578890042\\
264	0.0127123513165053\\
265	0.012712242947035\\
266	0.0127121327508333\\
267	0.0127120206976571\\
268	0.0127119067567724\\
269	0.0127117908969458\\
270	0.0127116730864366\\
271	0.0127115532929884\\
272	0.0127114314838201\\
273	0.0127113076256177\\
274	0.0127111816845249\\
275	0.0127110536261341\\
276	0.0127109234154768\\
277	0.0127107910170144\\
278	0.0127106563946279\\
279	0.0127105195116083\\
280	0.0127103803306458\\
281	0.0127102388138198\\
282	0.0127100949225878\\
283	0.0127099486177742\\
284	0.0127097998595593\\
285	0.0127096486074672\\
286	0.0127094948203544\\
287	0.0127093384563968\\
288	0.0127091794730776\\
289	0.0127090178271741\\
290	0.0127088534747444\\
291	0.0127086863711136\\
292	0.0127085164708595\\
293	0.0127083437277981\\
294	0.0127081680949687\\
295	0.012707989524618\\
296	0.0127078079681843\\
297	0.0127076233762807\\
298	0.0127074356986784\\
299	0.0127072448842884\\
300	0.0127070508811435\\
301	0.0127068536363791\\
302	0.0127066530962137\\
303	0.0127064492059284\\
304	0.0127062419098454\\
305	0.0127060311513064\\
306	0.0127058168726496\\
307	0.0127055990151857\\
308	0.0127053775191734\\
309	0.0127051523237934\\
310	0.0127049233671216\\
311	0.0127046905861013\\
312	0.0127044539165136\\
313	0.0127042132929472\\
314	0.0127039686487663\\
315	0.0127037199160774\\
316	0.0127034670256944\\
317	0.0127032099071021\\
318	0.0127029484884179\\
319	0.0127026826963519\\
320	0.0127024124561644\\
321	0.0127021376916225\\
322	0.0127018583249531\\
323	0.0127015742767945\\
324	0.0127012854661449\\
325	0.0127009918103091\\
326	0.0127006932248411\\
327	0.0127003896234845\\
328	0.0127000809181096\\
329	0.0126997670186471\\
330	0.0126994478330172\\
331	0.0126991232670562\\
332	0.0126987932244378\\
333	0.0126984576065904\\
334	0.0126981163126094\\
335	0.0126977692391644\\
336	0.0126974162804011\\
337	0.0126970573278366\\
338	0.0126966922702498\\
339	0.0126963209935632\\
340	0.0126959433807189\\
341	0.0126955593115466\\
342	0.0126951686626221\\
343	0.0126947713071185\\
344	0.0126943671146468\\
345	0.012693955951086\\
346	0.0126935376784026\\
347	0.0126931121544579\\
348	0.0126926792328023\\
349	0.0126922387624557\\
350	0.0126917905876727\\
351	0.0126913345476924\\
352	0.0126908704764694\\
353	0.0126903982023869\\
354	0.0126899175479491\\
355	0.0126894283294512\\
356	0.0126889303566258\\
357	0.0126884234322631\\
358	0.0126879073518029\\
359	0.0126873819028967\\
360	0.0126868468649355\\
361	0.0126863020085431\\
362	0.0126857470950287\\
363	0.0126851818757982\\
364	0.012684606091719\\
365	0.0126840194724339\\
366	0.0126834217356205\\
367	0.0126828125861917\\
368	0.0126821917154295\\
369	0.0126815588000498\\
370	0.0126809135011889\\
371	0.0126802554633061\\
372	0.0126795843129959\\
373	0.012678899657699\\
374	0.0126782010843071\\
375	0.0126774881576493\\
376	0.0126767604188521\\
377	0.0126760173835618\\
378	0.0126752585400196\\
379	0.0126744833469777\\
380	0.0126736912314439\\
381	0.0126728815862414\\
382	0.0126720537673614\\
383	0.0126712070910646\\
384	0.0126703408306282\\
385	0.0126694542124642\\
386	0.0126685464109121\\
387	0.0126676165399881\\
388	0.0126666636383049\\
389	0.0126656866406527\\
390	0.0126646846641529\\
391	0.0126636566193529\\
392	0.0126626013223191\\
393	0.0126615175508678\\
394	0.0126604040894902\\
395	0.0126592598112124\\
396	0.0126580827354889\\
397	0.0126568707871128\\
398	0.012655622418407\\
399	0.0126543360438814\\
400	0.0126530100315048\\
401	0.0126516426367741\\
402	0.0126502317655991\\
403	0.0126487742863174\\
404	0.0126470412231565\\
405	0.0126448394498039\\
406	0.0126426028906695\\
407	0.0126403310062329\\
408	0.012638023246909\\
409	0.0126356790527071\\
410	0.0126332978528633\\
411	0.0126308790654331\\
412	0.0126284220968131\\
413	0.0126259263411253\\
414	0.0126233911793037\\
415	0.0126208159775242\\
416	0.0126182000842078\\
417	0.0126155428240428\\
418	0.0126128434853768\\
419	0.0126101013742487\\
420	0.012607315778792\\
421	0.0126044859608663\\
422	0.0126016111524406\\
423	0.0125986905497337\\
424	0.0125957233009573\\
425	0.0125927084742787\\
426	0.0125896449622146\\
427	0.0125865311776374\\
428	0.0125833669648658\\
429	0.0125801524841931\\
430	0.012576887062375\\
431	0.0125735706441166\\
432	0.0125702050071502\\
433	0.0125667842615524\\
434	0.0125633011882153\\
435	0.0125597536815803\\
436	0.0125561394328797\\
437	0.0125524558990647\\
438	0.0125487002668484\\
439	0.0125448694163338\\
440	0.012540959859719\\
441	0.0125369676699011\\
442	0.0125328884048699\\
443	0.0125287170147654\\
444	0.01252444772953\\
445	0.0125200739225858\\
446	0.0125155879447591\\
447	0.0125109809210655\\
448	0.0125062425007236\\
449	0.0125013605473103\\
450	0.0124963207497012\\
451	0.0124911061204436\\
452	0.0124856963104355\\
453	0.0124800665523575\\
454	0.0124741856582942\\
455	0.0124680111442012\\
456	0.012461474571577\\
457	0.0124484630794266\\
458	0.0124343528801997\\
459	0.0124200022239117\\
460	0.0124054040698584\\
461	0.0123905512943999\\
462	0.0123754387233565\\
463	0.0123600610383275\\
464	0.0123444127726757\\
465	0.0123284883137113\\
466	0.0123122819039344\\
467	0.0122957876450154\\
468	0.0122789995066305\\
469	0.0122619113455549\\
470	0.012244516950215\\
471	0.0122268099785803\\
472	0.0122087839361696\\
473	0.0121904322792884\\
474	0.0121717484434963\\
475	0.0121527258665641\\
476	0.0121333579836693\\
477	0.0121136381365269\\
478	0.0120935604840553\\
479	0.0120731197552663\\
480	0.0120523113363066\\
481	0.0120311316413393\\
482	0.0120095781935255\\
483	0.0119876508158609\\
484	0.0119653550104346\\
485	0.011942713668386\\
486	0.0119217535111116\\
487	0.0119126553535508\\
488	0.0119034242335742\\
489	0.0118940621988752\\
490	0.0118845720085456\\
491	0.0118749572607399\\
492	0.0118652225447763\\
493	0.0118553736229872\\
494	0.0118454176490934\\
495	0.0118353634324121\\
496	0.0118252217593097\\
497	0.0118150057807003\\
498	0.0118047314801465\\
499	0.011794418152376\\
500	0.0117840892563953\\
501	0.0117737621951146\\
502	0.0117634343642958\\
503	0.0117532021922648\\
504	0.0117431177111872\\
505	0.0117332405380037\\
506	0.0117236518228485\\
507	0.0117144490297488\\
508	0.0117050984101696\\
509	0.011695540811201\\
510	0.0116857764281301\\
511	0.0116757512054881\\
512	0.0116654511142765\\
513	0.0116548598861948\\
514	0.0116438810727602\\
515	0.0116323495871931\\
516	0.0116204881962589\\
517	0.0116082752126049\\
518	0.0115956857098154\\
519	0.0115826912890469\\
520	0.0115692592239029\\
521	0.0115553509133539\\
522	0.0115409201179023\\
523	0.0115259101095428\\
524	0.0115102555767572\\
525	0.0114939090193393\\
526	0.0114769171695554\\
527	0.0114594426525852\\
528	0.011441086993902\\
529	0.0114217833923259\\
530	0.0114014574654767\\
531	0.0113799831495963\\
532	0.011357261802147\\
533	0.011333190732684\\
534	0.0113076561706216\\
535	0.0112805418077311\\
536	0.0112516938359571\\
537	0.011220917937706\\
538	0.0111879266671607\\
539	0.0111459660044578\\
540	0.0110869382246407\\
541	0.0110248217006492\\
542	0.0109590646109739\\
543	0.010888563122888\\
544	0.010738916237394\\
545	0.0105648875611786\\
546	0.0103871213951295\\
547	0.0102056152951943\\
548	0.0100299719568395\\
549	0.00996449037444903\\
550	0.00990245020666905\\
551	0.00984512024416929\\
552	0.00979412211129612\\
553	0.00974912543464925\\
554	0.00970501505029354\\
555	0.00966179010838528\\
556	0.00961907147515163\\
557	0.00957650089008366\\
558	0.00953431631936958\\
559	0.00949161053152337\\
560	0.00944769833697887\\
561	0.00940185771275797\\
562	0.00935583469263743\\
563	0.00930959204929004\\
564	0.00926525542069476\\
565	0.00922050872016028\\
566	0.00917487623137639\\
567	0.00912775069549462\\
568	0.00907955321167675\\
569	0.00902855805055227\\
570	0.00897578861335284\\
571	0.00892450720709081\\
572	0.00887761633395281\\
573	0.00883006113839492\\
574	0.00878163864994064\\
575	0.00870460980912668\\
576	0.00858817430045391\\
577	0.00813530455527196\\
578	0.00770022413322697\\
579	0.007615873705758\\
580	0.00753781942160739\\
581	0.00745861268072716\\
582	0.00737798602976268\\
583	0.00729588639332857\\
584	0.00721226969867924\\
585	0.0071270896051591\\
586	0.0070402941328682\\
587	0.00695181791635961\\
588	0.0068615626303918\\
589	0.00676934637959575\\
590	0.00667477256501156\\
591	0.00657713355197374\\
592	0.00647481662499991\\
593	0.00636364764633835\\
594	0.00623271716130669\\
595	0.00605340994148282\\
596	0.00575058001197164\\
597	0.00512683753504545\\
598	0.00366374385960312\\
599	0\\
600	0\\
};
\addplot [color=blue!50!mycolor7,solid,forget plot]
  table[row sep=crcr]{%
1	0.012792992079254\\
2	0.0127929907447039\\
3	0.0127929893876757\\
4	0.012792988007792\\
5	0.0127929866046692\\
6	0.0127929851779174\\
7	0.0127929837271399\\
8	0.0127929822519337\\
9	0.0127929807518888\\
10	0.0127929792265886\\
11	0.0127929776756093\\
12	0.0127929760985203\\
13	0.0127929744948836\\
14	0.012792972864254\\
15	0.0127929712061789\\
16	0.0127929695201981\\
17	0.0127929678058438\\
18	0.0127929660626403\\
19	0.0127929642901041\\
20	0.0127929624877435\\
21	0.0127929606550588\\
22	0.0127929587915418\\
23	0.012792956896676\\
24	0.0127929549699361\\
25	0.0127929530107883\\
26	0.0127929510186898\\
27	0.0127929489930888\\
28	0.0127929469334242\\
29	0.0127929448391258\\
30	0.0127929427096137\\
31	0.0127929405442986\\
32	0.0127929383425812\\
33	0.0127929361038523\\
34	0.0127929338274927\\
35	0.0127929315128728\\
36	0.0127929291593526\\
37	0.0127929267662814\\
38	0.0127929243329978\\
39	0.0127929218588295\\
40	0.0127929193430928\\
41	0.0127929167850929\\
42	0.0127929141841235\\
43	0.0127929115394664\\
44	0.0127929088503916\\
45	0.0127929061161572\\
46	0.0127929033360086\\
47	0.0127929005091791\\
48	0.0127928976348892\\
49	0.0127928947123465\\
50	0.0127928917407453\\
51	0.012792888719267\\
52	0.0127928856470791\\
53	0.0127928825233355\\
54	0.0127928793471761\\
55	0.0127928761177267\\
56	0.0127928728340985\\
57	0.0127928694953882\\
58	0.0127928661006774\\
59	0.0127928626490328\\
60	0.0127928591395054\\
61	0.012792855571131\\
62	0.0127928519429291\\
63	0.0127928482539032\\
64	0.0127928445030406\\
65	0.0127928406893116\\
66	0.0127928368116697\\
67	0.0127928328690513\\
68	0.0127928288603752\\
69	0.0127928247845423\\
70	0.0127928206404358\\
71	0.0127928164269201\\
72	0.0127928121428414\\
73	0.0127928077870268\\
74	0.0127928033582839\\
75	0.0127927988554011\\
76	0.0127927942771468\\
77	0.0127927896222691\\
78	0.0127927848894959\\
79	0.0127927800775338\\
80	0.0127927751850687\\
81	0.0127927702107647\\
82	0.012792765153264\\
83	0.0127927600111868\\
84	0.0127927547831306\\
85	0.01279274946767\\
86	0.0127927440633562\\
87	0.012792738568717\\
88	0.0127927329822558\\
89	0.0127927273024518\\
90	0.0127927215277594\\
91	0.0127927156566074\\
92	0.0127927096873995\\
93	0.0127927036185128\\
94	0.0127926974482984\\
95	0.0127926911750802\\
96	0.0127926847971548\\
97	0.0127926783127911\\
98	0.0127926717202297\\
99	0.0127926650176826\\
100	0.0127926582033325\\
101	0.0127926512753326\\
102	0.0127926442318059\\
103	0.0127926370708448\\
104	0.0127926297905106\\
105	0.012792622388833\\
106	0.0127926148638093\\
107	0.0127926072134045\\
108	0.0127925994355501\\
109	0.0127925915281439\\
110	0.0127925834890493\\
111	0.0127925753160948\\
112	0.0127925670070733\\
113	0.0127925585597417\\
114	0.0127925499718201\\
115	0.0127925412409911\\
116	0.0127925323648993\\
117	0.0127925233411507\\
118	0.012792514167312\\
119	0.0127925048409095\\
120	0.012792495359429\\
121	0.0127924857203148\\
122	0.0127924759209686\\
123	0.0127924659587495\\
124	0.0127924558309725\\
125	0.012792445534908\\
126	0.0127924350677809\\
127	0.0127924244267698\\
128	0.0127924136090061\\
129	0.0127924026115729\\
130	0.0127923914315044\\
131	0.0127923800657848\\
132	0.0127923685113468\\
133	0.0127923567650716\\
134	0.0127923448237867\\
135	0.0127923326842658\\
136	0.0127923203432267\\
137	0.0127923077973311\\
138	0.0127922950431825\\
139	0.0127922820773255\\
140	0.0127922688962442\\
141	0.0127922554963612\\
142	0.0127922418740356\\
143	0.0127922280255621\\
144	0.0127922139471691\\
145	0.0127921996350176\\
146	0.0127921850851988\\
147	0.0127921702937333\\
148	0.0127921552565688\\
149	0.0127921399695783\\
150	0.0127921244285586\\
151	0.0127921086292277\\
152	0.0127920925672236\\
153	0.0127920762381015\\
154	0.012792059637332\\
155	0.0127920427602989\\
156	0.0127920256022968\\
157	0.0127920081585285\\
158	0.012791990424103\\
159	0.0127919723940329\\
160	0.0127919540632313\\
161	0.0127919354265099\\
162	0.012791916478576\\
163	0.0127918972140295\\
164	0.0127918776273604\\
165	0.0127918577129463\\
166	0.0127918374650486\\
167	0.0127918168778108\\
168	0.0127917959452548\\
169	0.0127917746612785\\
170	0.0127917530196529\\
171	0.0127917310140195\\
172	0.0127917086378874\\
173	0.0127916858846313\\
174	0.0127916627474886\\
175	0.0127916392195576\\
176	0.0127916152937956\\
177	0.0127915909630169\\
178	0.0127915662198916\\
179	0.0127915410569447\\
180	0.0127915154665551\\
181	0.0127914894409557\\
182	0.0127914629722335\\
183	0.01279143605233\\
184	0.0127914086730433\\
185	0.0127913808260287\\
186	0.0127913525028021\\
187	0.0127913236947424\\
188	0.012791294393095\\
189	0.0127912645889755\\
190	0.0127912342733744\\
191	0.0127912034371615\\
192	0.0127911720710905\\
193	0.0127911401658038\\
194	0.0127911077118359\\
195	0.0127910746996145\\
196	0.0127910411194561\\
197	0.0127910069615579\\
198	0.0127909722160215\\
199	0.0127909368727825\\
200	0.0127909009216049\\
201	0.0127908643520776\\
202	0.0127908271536117\\
203	0.0127907893154376\\
204	0.0127907508266016\\
205	0.0127907116759629\\
206	0.0127906718521901\\
207	0.0127906313437582\\
208	0.0127905901389453\\
209	0.0127905482258289\\
210	0.0127905055922825\\
211	0.0127904622259724\\
212	0.0127904181143535\\
213	0.0127903732446662\\
214	0.0127903276039325\\
215	0.0127902811789519\\
216	0.012790233956298\\
217	0.0127901859223143\\
218	0.01279013706311\\
219	0.0127900873645565\\
220	0.0127900368122826\\
221	0.0127899853916706\\
222	0.0127899330878521\\
223	0.0127898798857032\\
224	0.0127898257698403\\
225	0.0127897707246156\\
226	0.0127897147341123\\
227	0.0127896577821399\\
228	0.0127895998522294\\
229	0.0127895409276283\\
230	0.0127894809912959\\
231	0.0127894200258978\\
232	0.0127893580138008\\
233	0.0127892949370681\\
234	0.012789230777453\\
235	0.0127891655163943\\
236	0.0127890991350101\\
237	0.0127890316140924\\
238	0.0127889629341011\\
239	0.0127888930751582\\
240	0.0127888220170417\\
241	0.0127887497391793\\
242	0.0127886762206425\\
243	0.0127886014401395\\
244	0.0127885253760096\\
245	0.0127884480062155\\
246	0.0127883693083373\\
247	0.0127882892595651\\
248	0.0127882078366921\\
249	0.0127881250161072\\
250	0.0127880407737878\\
251	0.0127879550852921\\
252	0.0127878679257514\\
253	0.0127877792698621\\
254	0.0127876890918782\\
255	0.0127875973656023\\
256	0.0127875040643776\\
257	0.0127874091610794\\
258	0.012787312628106\\
259	0.01278721443737\\
260	0.0127871145602889\\
261	0.0127870129677757\\
262	0.0127869096302293\\
263	0.0127868045175246\\
264	0.0127866975990026\\
265	0.0127865888434598\\
266	0.0127864782191376\\
267	0.0127863656937116\\
268	0.0127862512342805\\
269	0.0127861348073544\\
270	0.0127860163788436\\
271	0.0127858959140459\\
272	0.0127857733776352\\
273	0.0127856487336482\\
274	0.0127855219454717\\
275	0.0127853929758292\\
276	0.0127852617867676\\
277	0.0127851283396429\\
278	0.0127849925951058\\
279	0.0127848545130872\\
280	0.0127847140527828\\
281	0.0127845711726375\\
282	0.0127844258303296\\
283	0.0127842779827541\\
284	0.0127841275860058\\
285	0.0127839745953618\\
286	0.0127838189652637\\
287	0.0127836606492988\\
288	0.0127834996001816\\
289	0.0127833357697337\\
290	0.0127831691088637\\
291	0.0127829995675466\\
292	0.0127828270948023\\
293	0.0127826516386731\\
294	0.0127824731462014\\
295	0.012782291563406\\
296	0.0127821068352577\\
297	0.0127819189056544\\
298	0.0127817277173949\\
299	0.0127815332121529\\
300	0.0127813353304485\\
301	0.0127811340116206\\
302	0.0127809291937968\\
303	0.0127807208138632\\
304	0.0127805088074332\\
305	0.0127802931088144\\
306	0.0127800736509757\\
307	0.0127798503655119\\
308	0.0127796231826082\\
309	0.0127793920310026\\
310	0.0127791568379478\\
311	0.0127789175291708\\
312	0.0127786740288324\\
313	0.0127784262594837\\
314	0.0127781741420226\\
315	0.0127779175956476\\
316	0.0127776565378107\\
317	0.0127773908841681\\
318	0.0127771205485295\\
319	0.0127768454428058\\
320	0.0127765654769539\\
321	0.0127762805589213\\
322	0.0127759905945869\\
323	0.0127756954877008\\
324	0.0127753951398218\\
325	0.0127750894502525\\
326	0.0127747783159723\\
327	0.0127744616315679\\
328	0.0127741392891616\\
329	0.0127738111783369\\
330	0.0127734771860616\\
331	0.0127731371966085\\
332	0.0127727910914735\\
333	0.01277243874929\\
334	0.0127720800457421\\
335	0.0127717148534735\\
336	0.0127713430419948\\
337	0.0127709644775862\\
338	0.0127705790231993\\
339	0.012770186538354\\
340	0.0127697868790338\\
341	0.0127693798975773\\
342	0.0127689654425666\\
343	0.0127685433587135\\
344	0.0127681134867421\\
345	0.0127676756632684\\
346	0.0127672297206776\\
347	0.0127667754869984\\
348	0.0127663127857746\\
349	0.0127658414359342\\
350	0.0127653612516568\\
351	0.0127648720422383\\
352	0.0127643736119541\\
353	0.0127638657599209\\
354	0.0127633482799573\\
355	0.012762820960444\\
356	0.0127622835841832\\
357	0.012761735928259\\
358	0.0127611777638984\\
359	0.0127606088563344\\
360	0.0127600289646714\\
361	0.0127594378417537\\
362	0.0127588352340393\\
363	0.012758220881479\\
364	0.0127575945174025\\
365	0.0127569558684133\\
366	0.0127563046542935\\
367	0.0127556405879219\\
368	0.0127549633752047\\
369	0.0127542727150254\\
370	0.0127535682992122\\
371	0.0127528498125304\\
372	0.0127521169326994\\
373	0.0127513693304417\\
374	0.0127506066695671\\
375	0.0127498286070967\\
376	0.0127490347934348\\
377	0.0127482248725942\\
378	0.0127473984824831\\
379	0.0127465552552642\\
380	0.0127456948177941\\
381	0.0127448167921566\\
382	0.0127439207963021\\
383	0.0127430064448041\\
384	0.012742073349745\\
385	0.0127411211217393\\
386	0.0127401493711167\\
387	0.012739157709372\\
388	0.0127381457513711\\
389	0.0127371131201528\\
390	0.0127360594423528\\
391	0.0127349843552758\\
392	0.0127338875118904\\
393	0.0127327685852119\\
394	0.0127316272714285\\
395	0.0127304632856456\\
396	0.0127292763875721\\
397	0.0127280663904961\\
398	0.012726833150965\\
399	0.0127255765731411\\
400	0.0127242966105811\\
401	0.0127229932632295\\
402	0.0127216665707705\\
403	0.0127203166263298\\
404	0.0127189437435923\\
405	0.0127175481960241\\
406	0.0127161295954536\\
407	0.012714687532925\\
408	0.0127132215866604\\
409	0.0127117313210684\\
410	0.0127102162856905\\
411	0.0127086760141024\\
412	0.0127071100228092\\
413	0.0127055178101971\\
414	0.0127038988556273\\
415	0.0127022526187338\\
416	0.0127005785388669\\
417	0.0126988760349283\\
418	0.0126971445116799\\
419	0.0126953834233624\\
420	0.0126935922143988\\
421	0.0126917703081474\\
422	0.0126899171042727\\
423	0.0126880319751953\\
424	0.0126861142606266\\
425	0.0126841632578285\\
426	0.0126821782024126\\
427	0.0126801582305813\\
428	0.0126781026854419\\
429	0.0126760109028805\\
430	0.0126738820307764\\
431	0.012671715277872\\
432	0.0126695100338271\\
433	0.0126672643388395\\
434	0.0126649760781305\\
435	0.0126626440029699\\
436	0.0126602667674305\\
437	0.012657842907867\\
438	0.0126553707916276\\
439	0.0126528485190949\\
440	0.0126502744425797\\
441	0.0126476471696468\\
442	0.0126449652765735\\
443	0.0126422273153552\\
444	0.012639431823736\\
445	0.0126365773392313\\
446	0.0126336624183972\\
447	0.0126306856629217\\
448	0.0126276457543936\\
449	0.0126245414995428\\
450	0.0126213718863682\\
451	0.0126181361459287\\
452	0.012614833795841\\
453	0.0126114645838651\\
454	0.0126080280793682\\
455	0.012604522172994\\
456	0.0126009465470349\\
457	0.0125959707843802\\
458	0.0125907847269098\\
459	0.0125855907676074\\
460	0.0125803992631018\\
461	0.0125752072479814\\
462	0.0125699213757532\\
463	0.0125645386175389\\
464	0.0125590556880374\\
465	0.0125534689939647\\
466	0.0125477745882916\\
467	0.0125419681164783\\
468	0.0125360447533458\\
469	0.0125299991293525\\
470	0.012523825246106\\
471	0.0125175163548903\\
472	0.0125110648195536\\
473	0.0125044619735459\\
474	0.0124976979292119\\
475	0.0124907613433337\\
476	0.0124836391248707\\
477	0.012476316062334\\
478	0.0124687745419361\\
479	0.0124609939533591\\
480	0.0124529500114997\\
481	0.0124446139091603\\
482	0.0124359509911232\\
483	0.012426918794812\\
484	0.0124174629365302\\
485	0.012407506191671\\
486	0.0123953754340449\\
487	0.0123730025552894\\
488	0.0123502157826629\\
489	0.0123270054750613\\
490	0.0123033616289481\\
491	0.0122792738628729\\
492	0.0122547314024656\\
493	0.0122297230661976\\
494	0.0122042372522723\\
495	0.0121782619271292\\
496	0.0121517846162918\\
497	0.0121247923990123\\
498	0.0120972719094384\\
499	0.0120692093531988\\
500	0.0120405905413115\\
501	0.012011400956291\\
502	0.0119816258239681\\
503	0.0119512494044773\\
504	0.0119202583698927\\
505	0.0118886477107224\\
506	0.0118564281356895\\
507	0.0118236678927037\\
508	0.0118083044304957\\
509	0.0117949512162318\\
510	0.0117813795387892\\
511	0.0117675871516953\\
512	0.0117535813066232\\
513	0.0117393716166701\\
514	0.0117249531908821\\
515	0.0117103083315483\\
516	0.0116955111774054\\
517	0.0116806120037169\\
518	0.0116656720414887\\
519	0.0116507048545402\\
520	0.0116357214973517\\
521	0.0116207871970443\\
522	0.0116059855372909\\
523	0.0115914234563041\\
524	0.0115770818101273\\
525	0.011562231204689\\
526	0.0115467564795575\\
527	0.0115304506965384\\
528	0.0115135466893872\\
529	0.0114960040829992\\
530	0.0114777778829336\\
531	0.0114588120208159\\
532	0.0114390506818788\\
533	0.0114184308008632\\
534	0.011396877210855\\
535	0.0113742460102425\\
536	0.0113504353777355\\
537	0.0113253189725327\\
538	0.0112991495883925\\
539	0.0112714730354775\\
540	0.0112420912224632\\
541	0.0112108193672537\\
542	0.0111774211860306\\
543	0.0111415588186748\\
544	0.0110865857833767\\
545	0.0110241580600701\\
546	0.0109583457434496\\
547	0.010888416417644\\
548	0.0108052632907422\\
549	0.0106299429228413\\
550	0.0104510105857241\\
551	0.0102682287949327\\
552	0.0100804500499112\\
553	0.00990956786620671\\
554	0.00983838302665969\\
555	0.00977038980910216\\
556	0.00970680215494745\\
557	0.00964931687322562\\
558	0.009599162085229\\
559	0.00954967206260671\\
560	0.00950071602098221\\
561	0.00945204581332923\\
562	0.00940398971229489\\
563	0.00935663222758508\\
564	0.00930699225591728\\
565	0.00925693302201336\\
566	0.00920676347091189\\
567	0.00915704191852477\\
568	0.00910875713763971\\
569	0.00905921407971458\\
570	0.00900853798820848\\
571	0.00895649782086392\\
572	0.00890079172342068\\
573	0.00884574283971147\\
574	0.00879414734479222\\
575	0.00874417034575634\\
576	0.00869347303764065\\
577	0.00858507079136236\\
578	0.00839695950345147\\
579	0.00789852302704922\\
580	0.00754640048314225\\
581	0.00745926952565165\\
582	0.00737820372283263\\
583	0.00729600763418925\\
584	0.00721233608898569\\
585	0.00712712428178828\\
586	0.00704031063046766\\
587	0.00695182476096253\\
588	0.00686156497012343\\
589	0.00676934696886435\\
590	0.00667477264796359\\
591	0.00657713355197374\\
592	0.00647481662499991\\
593	0.00636364764633835\\
594	0.00623271716130668\\
595	0.0060534099414828\\
596	0.00575058001197163\\
597	0.00512683753504545\\
598	0.00366374385960312\\
599	0\\
600	0\\
};
\addplot [color=blue!40!mycolor9,solid,forget plot]
  table[row sep=crcr]{%
1	0.0131062123008924\\
2	0.0131062097189333\\
3	0.0131062070935401\\
4	0.0131062044239834\\
5	0.013106201709521\\
6	0.0131061989493986\\
7	0.0131061961428491\\
8	0.0131061932890925\\
9	0.0131061903873358\\
10	0.0131061874367725\\
11	0.0131061844365826\\
12	0.0131061813859325\\
13	0.0131061782839744\\
14	0.0131061751298463\\
15	0.0131061719226716\\
16	0.0131061686615592\\
17	0.0131061653456028\\
18	0.0131061619738811\\
19	0.0131061585454569\\
20	0.0131061550593777\\
21	0.0131061515146748\\
22	0.013106147910363\\
23	0.0131061442454409\\
24	0.01310614051889\\
25	0.0131061367296746\\
26	0.0131061328767419\\
27	0.0131061289590211\\
28	0.0131061249754234\\
29	0.0131061209248418\\
30	0.0131061168061506\\
31	0.0131061126182051\\
32	0.0131061083598414\\
33	0.0131061040298759\\
34	0.0131060996271052\\
35	0.0131060951503055\\
36	0.0131060905982325\\
37	0.0131060859696209\\
38	0.0131060812631838\\
39	0.0131060764776132\\
40	0.0131060716115784\\
41	0.0131060666637267\\
42	0.0131060616326823\\
43	0.0131060565170466\\
44	0.0131060513153969\\
45	0.0131060460262868\\
46	0.0131060406482455\\
47	0.0131060351797772\\
48	0.013106029619361\\
49	0.0131060239654501\\
50	0.0131060182164719\\
51	0.013106012370827\\
52	0.0131060064268889\\
53	0.0131060003830039\\
54	0.0131059942374901\\
55	0.0131059879886372\\
56	0.013105981634706\\
57	0.0131059751739278\\
58	0.0131059686045042\\
59	0.0131059619246061\\
60	0.0131059551323734\\
61	0.0131059482259147\\
62	0.0131059412033064\\
63	0.0131059340625925\\
64	0.0131059268017835\\
65	0.0131059194188564\\
66	0.0131059119117538\\
67	0.0131059042783835\\
68	0.0131058965166175\\
69	0.013105888624292\\
70	0.0131058805992063\\
71	0.0131058724391221\\
72	0.0131058641417634\\
73	0.0131058557048152\\
74	0.0131058471259232\\
75	0.0131058384026932\\
76	0.0131058295326898\\
77	0.0131058205134367\\
78	0.0131058113424148\\
79	0.0131058020170624\\
80	0.0131057925347741\\
81	0.0131057828928998\\
82	0.0131057730887443\\
83	0.0131057631195664\\
84	0.0131057529825779\\
85	0.013105742674943\\
86	0.0131057321937775\\
87	0.0131057215361474\\
88	0.013105710699069\\
89	0.013105699679507\\
90	0.0131056884743742\\
91	0.0131056770805305\\
92	0.0131056654947818\\
93	0.0131056537138792\\
94	0.0131056417345177\\
95	0.0131056295533356\\
96	0.0131056171669134\\
97	0.0131056045717727\\
98	0.0131055917643748\\
99	0.0131055787411203\\
100	0.0131055654983475\\
101	0.0131055520323315\\
102	0.013105538339283\\
103	0.0131055244153468\\
104	0.0131055102566015\\
105	0.0131054958590572\\
106	0.013105481218655\\
107	0.0131054663312656\\
108	0.0131054511926879\\
109	0.0131054357986475\\
110	0.0131054201447959\\
111	0.0131054042267086\\
112	0.013105388039884\\
113	0.013105371579742\\
114	0.0131053548416222\\
115	0.0131053378207831\\
116	0.0131053205123996\\
117	0.0131053029115626\\
118	0.0131052850132764\\
119	0.0131052668124578\\
120	0.0131052483039343\\
121	0.013105229482442\\
122	0.0131052103426246\\
123	0.0131051908790311\\
124	0.0131051710861142\\
125	0.0131051509582288\\
126	0.0131051304896297\\
127	0.0131051096744697\\
128	0.0131050885067981\\
129	0.0131050669805585\\
130	0.0131050450895869\\
131	0.0131050228276092\\
132	0.01310500018824\\
133	0.0131049771649795\\
134	0.0131049537512121\\
135	0.0131049299402039\\
136	0.0131049057251003\\
137	0.0131048810989239\\
138	0.0131048560545722\\
139	0.0131048305848152\\
140	0.0131048046822928\\
141	0.0131047783395126\\
142	0.0131047515488471\\
143	0.0131047243025316\\
144	0.013104696592661\\
145	0.0131046684111877\\
146	0.0131046397499186\\
147	0.0131046106005124\\
148	0.0131045809544773\\
149	0.0131045508031673\\
150	0.0131045201377802\\
151	0.0131044889493543\\
152	0.0131044572287656\\
153	0.0131044249667248\\
154	0.0131043921537743\\
155	0.0131043587802853\\
156	0.0131043248364546\\
157	0.0131042903123017\\
158	0.0131042551976655\\
159	0.0131042194822013\\
160	0.0131041831553779\\
161	0.0131041462064738\\
162	0.0131041086245749\\
163	0.0131040703985704\\
164	0.0131040315171507\\
165	0.0131039919688031\\
166	0.0131039517418096\\
167	0.013103910824243\\
168	0.0131038692039644\\
169	0.0131038268686193\\
170	0.0131037838056353\\
171	0.0131037400022184\\
172	0.01310369544535\\
173	0.013103650121784\\
174	0.0131036040180436\\
175	0.0131035571204182\\
176	0.0131035094149604\\
177	0.013103460887483\\
178	0.0131034115235556\\
179	0.0131033613085019\\
180	0.0131033102273963\\
181	0.0131032582650607\\
182	0.0131032054060611\\
183	0.0131031516347044\\
184	0.0131030969350345\\
185	0.013103041290829\\
186	0.0131029846855945\\
187	0.0131029271025632\\
188	0.0131028685246876\\
189	0.013102808934636\\
190	0.013102748314787\\
191	0.013102686647224\\
192	0.0131026239137285\\
193	0.0131025600957738\\
194	0.0131024951745173\\
195	0.013102429130793\\
196	0.013102361945103\\
197	0.0131022935976096\\
198	0.0131022240681262\\
199	0.0131021533361108\\
200	0.0131020813806595\\
201	0.0131020081805004\\
202	0.0131019337139861\\
203	0.0131018579590877\\
204	0.0131017808933873\\
205	0.0131017024940715\\
206	0.0131016227379234\\
207	0.013101541601316\\
208	0.0131014590602043\\
209	0.0131013750901179\\
210	0.0131012896661529\\
211	0.0131012027629645\\
212	0.0131011143547584\\
213	0.0131010244152828\\
214	0.0131009329178199\\
215	0.0131008398351778\\
216	0.0131007451396809\\
217	0.013100648803162\\
218	0.0131005507969523\\
219	0.013100451091873\\
220	0.0131003496582252\\
221	0.0131002464657809\\
222	0.0131001414837726\\
223	0.0131000346808838\\
224	0.0130999260252386\\
225	0.0130998154843912\\
226	0.0130997030253156\\
227	0.0130995886143948\\
228	0.0130994722174093\\
229	0.0130993537995267\\
230	0.0130992333252894\\
231	0.0130991107586034\\
232	0.0130989860627266\\
233	0.0130988592002558\\
234	0.0130987301331155\\
235	0.0130985988225441\\
236	0.0130984652290819\\
237	0.0130983293125573\\
238	0.013098191032074\\
239	0.0130980503459966\\
240	0.0130979072119374\\
241	0.0130977615867416\\
242	0.0130976134264728\\
243	0.0130974626863987\\
244	0.0130973093209751\\
245	0.0130971532838313\\
246	0.0130969945277537\\
247	0.0130968330046702\\
248	0.0130966686656334\\
249	0.0130965014608042\\
250	0.0130963313394342\\
251	0.0130961582498489\\
252	0.0130959821394293\\
253	0.0130958029545939\\
254	0.0130956206407805\\
255	0.0130954351424265\\
256	0.0130952464029504\\
257	0.0130950543647312\\
258	0.0130948589690888\\
259	0.0130946601562635\\
260	0.0130944578653943\\
261	0.013094252034498\\
262	0.0130940426004473\\
263	0.013093829498948\\
264	0.0130936126645164\\
265	0.0130933920304559\\
266	0.0130931675288334\\
267	0.0130929390904542\\
268	0.0130927066448383\\
269	0.0130924701201937\\
270	0.0130922294433916\\
271	0.0130919845399391\\
272	0.0130917353339526\\
273	0.0130914817481298\\
274	0.0130912237037218\\
275	0.0130909611205041\\
276	0.0130906939167472\\
277	0.0130904220091863\\
278	0.0130901453129908\\
279	0.0130898637417328\\
280	0.013089577207355\\
281	0.0130892856201379\\
282	0.0130889888886666\\
283	0.0130886869197962\\
284	0.0130883796186169\\
285	0.0130880668884187\\
286	0.0130877486306546\\
287	0.0130874247449034\\
288	0.0130870951288316\\
289	0.0130867596781547\\
290	0.0130864182865976\\
291	0.0130860708458535\\
292	0.0130857172455431\\
293	0.0130853573731718\\
294	0.0130849911140868\\
295	0.0130846183514328\\
296	0.0130842389661068\\
297	0.0130838528367124\\
298	0.0130834598395123\\
299	0.0130830598483807\\
300	0.0130826527347539\\
301	0.0130822383675805\\
302	0.01308181661327\\
303	0.0130813873356411\\
304	0.0130809503958676\\
305	0.0130805056524247\\
306	0.0130800529610334\\
307	0.0130795921746034\\
308	0.0130791231431758\\
309	0.013078645713864\\
310	0.0130781597307933\\
311	0.0130776650350397\\
312	0.0130771614645675\\
313	0.0130766488541653\\
314	0.0130761270353807\\
315	0.0130755958364548\\
316	0.0130750550822536\\
317	0.0130745045942002\\
318	0.0130739441902039\\
319	0.0130733736845894\\
320	0.0130727928880237\\
321	0.0130722016074424\\
322	0.0130715996459745\\
323	0.0130709868028659\\
324	0.0130703628734013\\
325	0.0130697276488255\\
326	0.0130690809162632\\
327	0.0130684224586368\\
328	0.0130677520545845\\
329	0.0130670694783756\\
330	0.0130663744998258\\
331	0.0130656668842106\\
332	0.0130649463921783\\
333	0.0130642127796611\\
334	0.0130634657977858\\
335	0.0130627051927833\\
336	0.0130619307058973\\
337	0.0130611420732919\\
338	0.0130603390259586\\
339	0.0130595212896225\\
340	0.0130586885846479\\
341	0.0130578406259437\\
342	0.0130569771228672\\
343	0.0130560977791293\\
344	0.0130552022926978\\
345	0.0130542903557015\\
346	0.0130533616543344\\
347	0.0130524158687594\\
348	0.0130514526730125\\
349	0.0130504717349081\\
350	0.0130494727159439\\
351	0.0130484552712069\\
352	0.0130474190492809\\
353	0.013046363692154\\
354	0.0130452888351284\\
355	0.0130441941067309\\
356	0.0130430791286257\\
357	0.0130419435155281\\
358	0.0130407868751212\\
359	0.013039608807974\\
360	0.0130384089074623\\
361	0.0130371867596918\\
362	0.013035941943424\\
363	0.0130346740300047\\
364	0.0130333825832956\\
365	0.013032067159609\\
366	0.0130307273076444\\
367	0.0130293625684301\\
368	0.0130279724752654\\
369	0.0130265565536678\\
370	0.0130251143213216\\
371	0.0130236452880295\\
372	0.0130221489556663\\
373	0.0130206248181345\\
374	0.0130190723613211\\
375	0.0130174910630551\\
376	0.0130158803930647\\
377	0.013014239812934\\
378	0.0130125687760568\\
379	0.0130108667275874\\
380	0.0130091331043857\\
381	0.0130073673349556\\
382	0.0130055688393732\\
383	0.0130037370292029\\
384	0.013001871307398\\
385	0.0129999710681821\\
386	0.0129980356969119\\
387	0.0129960645699217\\
388	0.0129940570543403\\
389	0.012992012507765\\
390	0.0129899302781134\\
391	0.0129878097032638\\
392	0.0129856501105441\\
393	0.0129834508160503\\
394	0.0129812111237702\\
395	0.012978930324836\\
396	0.0129766076959477\\
397	0.0129742424974519\\
398	0.0129718339716314\\
399	0.0129693813405619\\
400	0.0129668838033543\\
401	0.0129643405324378\\
402	0.012961750667796\\
403	0.0129591133040516\\
404	0.0129564274444927\\
405	0.0129536918679756\\
406	0.0129509058552706\\
407	0.0129480694158987\\
408	0.0129451816044291\\
409	0.0129422414651227\\
410	0.0129392480333532\\
411	0.0129362003376072\\
412	0.0129330974028044\\
413	0.0129299382569691\\
414	0.0129267219468194\\
415	0.0129234475775764\\
416	0.0129201144191165\\
417	0.0129167221947262\\
418	0.0129132718739799\\
419	0.0129097555658597\\
420	0.012906171085146\\
421	0.0129025172046721\\
422	0.0128987927050607\\
423	0.0128949963802688\\
424	0.0128911270441335\\
425	0.0128871835382241\\
426	0.0128831647418413\\
427	0.0128790695868142\\
428	0.012874897065317\\
429	0.0128706462456258\\
430	0.0128663162974818\\
431	0.0128619065136225\\
432	0.0128574163239453\\
433	0.0128528453643652\\
434	0.012848193529859\\
435	0.0128434610143885\\
436	0.0128386484115166\\
437	0.01283375687313\\
438	0.0128287883138137\\
439	0.0128237453121304\\
440	0.0128186124555014\\
441	0.012813376780429\\
442	0.012808036415159\\
443	0.012802589509648\\
444	0.0127970342466934\\
445	0.0127913688551719\\
446	0.0127855916257636\\
447	0.0127797009295912\\
448	0.012773695240214\\
449	0.012767573159328\\
450	0.0127613334461588\\
451	0.0127549750495214\\
452	0.0127484971391772\\
453	0.0127418991287986\\
454	0.0127351806789257\\
455	0.0127283416846906\\
456	0.0127213833968186\\
457	0.012714308182166\\
458	0.0127071175974745\\
459	0.0126998154614038\\
460	0.0126924110083193\\
461	0.0126849649695689\\
462	0.0126815145191367\\
463	0.0126779900021133\\
464	0.0126743884876139\\
465	0.012670707415893\\
466	0.0126669440878218\\
467	0.0126630956550081\\
468	0.0126591591094476\\
469	0.0126551312729997\\
470	0.0126510087853592\\
471	0.0126467880715858\\
472	0.0126424653298302\\
473	0.0126380365452358\\
474	0.012633497484306\\
475	0.0126288436920041\\
476	0.0126240704921423\\
477	0.0126191730039485\\
478	0.0126141462939614\\
479	0.0126089851664669\\
480	0.0126036838360823\\
481	0.0125982349089815\\
482	0.0125926372838476\\
483	0.0125868874724391\\
484	0.0125809801308669\\
485	0.0125749087287736\\
486	0.0125683235025073\\
487	0.0125594520239075\\
488	0.0125504280157903\\
489	0.0125412480341287\\
490	0.0125319083517077\\
491	0.0125224049054904\\
492	0.0125127332333894\\
493	0.012502888398347\\
494	0.0124928648972797\\
495	0.0124826565522084\\
496	0.0124722563810587\\
497	0.0124616564470727\\
498	0.0124508476907447\\
499	0.0124398197614155\\
500	0.0124285608946239\\
501	0.0124170579151011\\
502	0.012405296325065\\
503	0.0123932589960691\\
504	0.0123808206868512\\
505	0.0123677880674256\\
506	0.0123540612035565\\
507	0.0123394798057046\\
508	0.0123097390461796\\
509	0.0122774957760457\\
510	0.0122445595306122\\
511	0.0122109080176509\\
512	0.0121765185710484\\
513	0.0121413675045351\\
514	0.0121054300689422\\
515	0.0120686803431523\\
516	0.0120310919183313\\
517	0.011992641441605\\
518	0.0119533047159617\\
519	0.0119130471285004\\
520	0.0118718323983246\\
521	0.0118296329947836\\
522	0.0117864287093839\\
523	0.0117422289446107\\
524	0.0117012329633552\\
525	0.0116824680885595\\
526	0.0116633095787144\\
527	0.0116437397366907\\
528	0.0116238488887752\\
529	0.0116035371698284\\
530	0.0115828089704705\\
531	0.0115616813864128\\
532	0.0115401795632423\\
533	0.0115183388984585\\
534	0.0114962078795074\\
535	0.01147384555441\\
536	0.0114513452201563\\
537	0.0114288300780705\\
538	0.011405969759028\\
539	0.0113820128478913\\
540	0.0113568919014069\\
541	0.0113305207212212\\
542	0.011302802021007\\
543	0.0112736266560262\\
544	0.0112428814717731\\
545	0.011210408426141\\
546	0.0111760045349491\\
547	0.0111393997893615\\
548	0.0110985141478726\\
549	0.0110359958009047\\
550	0.0109703350304709\\
551	0.0109010455269748\\
552	0.0108272236335095\\
553	0.0107311067430109\\
554	0.0105521204355113\\
555	0.0103686484462171\\
556	0.0101803219317228\\
557	0.00998651054621628\\
558	0.00979370685284906\\
559	0.00971537036235166\\
560	0.00963954231455041\\
561	0.00956767902320676\\
562	0.00950150185643653\\
563	0.00944190396085486\\
564	0.00938514629032857\\
565	0.00932901308938975\\
566	0.0092735379510463\\
567	0.00921824308278224\\
568	0.00916273140610261\\
569	0.00910719383632812\\
570	0.00905161174129055\\
571	0.00899684129174705\\
572	0.00894254650472999\\
573	0.00888762933099036\\
574	0.00882982792747635\\
575	0.00877077156997362\\
576	0.00871241435043279\\
577	0.00865852168006084\\
578	0.00859208551626859\\
579	0.00847571802280514\\
580	0.00822405621502198\\
581	0.00773536955592913\\
582	0.00739020877460958\\
583	0.00729759652126533\\
584	0.00721310701082822\\
585	0.0071275233835895\\
586	0.00704052757817003\\
587	0.00695193522767474\\
588	0.00686161434916973\\
589	0.00676936536373383\\
590	0.00667477774794946\\
591	0.00657713435287329\\
592	0.00647481662499991\\
593	0.00636364764633836\\
594	0.00623271716130668\\
595	0.00605340994148281\\
596	0.00575058001197164\\
597	0.00512683753504545\\
598	0.00366374385960312\\
599	0\\
600	0\\
};
\addplot [color=blue!75!mycolor7,solid,forget plot]
  table[row sep=crcr]{%
1	0.013299675720752\\
2	0.0132996740620042\\
3	0.0132996723753582\\
4	0.0132996706603451\\
5	0.0132996689164881\\
6	0.0132996671433026\\
7	0.0132996653402954\\
8	0.0132996635069653\\
9	0.0132996616428026\\
10	0.013299659747289\\
11	0.0132996578198975\\
12	0.0132996558600922\\
13	0.0132996538673282\\
14	0.0132996518410513\\
15	0.0132996497806981\\
16	0.0132996476856956\\
17	0.0132996455554613\\
18	0.0132996433894027\\
19	0.0132996411869173\\
20	0.0132996389473927\\
21	0.0132996366702058\\
22	0.0132996343547233\\
23	0.0132996320003011\\
24	0.013299629606284\\
25	0.0132996271720062\\
26	0.0132996246967903\\
27	0.0132996221799476\\
28	0.0132996196207776\\
29	0.0132996170185683\\
30	0.0132996143725954\\
31	0.0132996116821224\\
32	0.0132996089464004\\
33	0.0132996061646678\\
34	0.0132996033361501\\
35	0.0132996004600598\\
36	0.0132995975355959\\
37	0.013299594561944\\
38	0.0132995915382758\\
39	0.0132995884637491\\
40	0.0132995853375073\\
41	0.0132995821586794\\
42	0.0132995789263796\\
43	0.0132995756397069\\
44	0.0132995722977453\\
45	0.0132995688995631\\
46	0.0132995654442129\\
47	0.013299561930731\\
48	0.0132995583581376\\
49	0.0132995547254362\\
50	0.0132995510316131\\
51	0.0132995472756377\\
52	0.0132995434564617\\
53	0.0132995395730191\\
54	0.0132995356242256\\
55	0.0132995316089785\\
56	0.0132995275261564\\
57	0.0132995233746187\\
58	0.0132995191532054\\
59	0.0132995148607368\\
60	0.0132995104960128\\
61	0.0132995060578131\\
62	0.0132995015448966\\
63	0.0132994969560007\\
64	0.0132994922898415\\
65	0.0132994875451131\\
66	0.0132994827204871\\
67	0.0132994778146127\\
68	0.0132994728261155\\
69	0.0132994677535981\\
70	0.0132994625956387\\
71	0.0132994573507914\\
72	0.0132994520175853\\
73	0.0132994465945244\\
74	0.0132994410800871\\
75	0.0132994354727253\\
76	0.0132994297708646\\
77	0.0132994239729035\\
78	0.0132994180772128\\
79	0.0132994120821353\\
80	0.0132994059859852\\
81	0.0132993997870477\\
82	0.0132993934835785\\
83	0.013299387073803\\
84	0.013299380555916\\
85	0.0132993739280812\\
86	0.0132993671884303\\
87	0.013299360335063\\
88	0.0132993533660459\\
89	0.013299346279412\\
90	0.0132993390731603\\
91	0.0132993317452552\\
92	0.0132993242936255\\
93	0.0132993167161641\\
94	0.0132993090107273\\
95	0.013299301175134\\
96	0.0132992932071653\\
97	0.0132992851045634\\
98	0.0132992768650312\\
99	0.0132992684862315\\
100	0.0132992599657864\\
101	0.0132992513012761\\
102	0.013299242490239\\
103	0.0132992335301697\\
104	0.0132992244185196\\
105	0.0132992151526949\\
106	0.0132992057300566\\
107	0.0132991961479191\\
108	0.0132991864035498\\
109	0.0132991764941679\\
110	0.0132991664169437\\
111	0.0132991561689977\\
112	0.0132991457473996\\
113	0.0132991351491673\\
114	0.0132991243712663\\
115	0.0132991134106081\\
116	0.0132991022640499\\
117	0.0132990909283932\\
118	0.0132990794003828\\
119	0.0132990676767059\\
120	0.0132990557539908\\
121	0.0132990436288063\\
122	0.0132990312976598\\
123	0.0132990187569971\\
124	0.0132990060032005\\
125	0.013298993032588\\
126	0.0132989798414122\\
127	0.0132989664258587\\
128	0.0132989527820455\\
129	0.013298938906021\\
130	0.0132989247937634\\
131	0.0132989104411789\\
132	0.013298895844101\\
133	0.0132988809982883\\
134	0.0132988658994241\\
135	0.0132988505431143\\
136	0.0132988349248863\\
137	0.0132988190401877\\
138	0.0132988028843846\\
139	0.0132987864527604\\
140	0.013298769740514\\
141	0.0132987527427587\\
142	0.0132987354545204\\
143	0.0132987178707359\\
144	0.013298699986252\\
145	0.0132986817958233\\
146	0.0132986632941105\\
147	0.0132986444756796\\
148	0.0132986253349993\\
149	0.0132986058664399\\
150	0.0132985860642716\\
151	0.0132985659226627\\
152	0.0132985454356775\\
153	0.0132985245972756\\
154	0.0132985034013089\\
155	0.0132984818415208\\
156	0.0132984599115439\\
157	0.0132984376048985\\
158	0.0132984149149905\\
159	0.0132983918351097\\
160	0.0132983683584281\\
161	0.0132983444779979\\
162	0.0132983201867494\\
163	0.0132982954774894\\
164	0.0132982703428994\\
165	0.0132982447755332\\
166	0.0132982187678151\\
167	0.0132981923120382\\
168	0.0132981654003621\\
169	0.0132981380248108\\
170	0.0132981101772708\\
171	0.013298081849489\\
172	0.0132980530330702\\
173	0.0132980237194754\\
174	0.0132979939000191\\
175	0.0132979635658671\\
176	0.0132979327080345\\
177	0.0132979013173826\\
178	0.0132978693846167\\
179	0.0132978369002837\\
180	0.0132978038547691\\
181	0.0132977702382942\\
182	0.0132977360409131\\
183	0.0132977012525102\\
184	0.0132976658627964\\
185	0.013297629861306\\
186	0.0132975932373936\\
187	0.0132975559802302\\
188	0.0132975180787999\\
189	0.0132974795218955\\
190	0.0132974402981155\\
191	0.0132974003958591\\
192	0.0132973598033231\\
193	0.0132973185084971\\
194	0.0132972764991594\\
195	0.0132972337628732\\
196	0.0132971902869821\\
197	0.013297146058606\\
198	0.0132971010646371\\
199	0.0132970552917359\\
200	0.0132970087263266\\
201	0.0132969613545932\\
202	0.0132969131624749\\
203	0.0132968641356615\\
204	0.0132968142595893\\
205	0.013296763519436\\
206	0.0132967119001162\\
207	0.0132966593862763\\
208	0.0132966059622901\\
209	0.0132965516122533\\
210	0.0132964963199788\\
211	0.0132964400689911\\
212	0.0132963828425212\\
213	0.0132963246235013\\
214	0.013296265394559\\
215	0.0132962051380119\\
216	0.0132961438358621\\
217	0.0132960814697897\\
218	0.0132960180211474\\
219	0.0132959534709544\\
220	0.0132958877998899\\
221	0.0132958209882872\\
222	0.0132957530161269\\
223	0.0132956838630307\\
224	0.0132956135082543\\
225	0.013295541930681\\
226	0.0132954691088147\\
227	0.0132953950207725\\
228	0.0132953196442779\\
229	0.013295242956653\\
230	0.0132951649348116\\
231	0.0132950855552508\\
232	0.013295004794044\\
233	0.0132949226268323\\
234	0.0132948390288167\\
235	0.0132947539747498\\
236	0.0132946674389274\\
237	0.0132945793951798\\
238	0.013294489816863\\
239	0.0132943986768499\\
240	0.0132943059475214\\
241	0.0132942116007563\\
242	0.0132941156079228\\
243	0.0132940179398683\\
244	0.0132939185669096\\
245	0.0132938174588227\\
246	0.013293714584833\\
247	0.0132936099136045\\
248	0.0132935034132292\\
249	0.0132933950512161\\
250	0.0132932847944803\\
251	0.013293172609332\\
252	0.013293058461464\\
253	0.0132929423159411\\
254	0.0132928241371873\\
255	0.0132927038889737\\
256	0.0132925815344065\\
257	0.0132924570359137\\
258	0.0132923303552325\\
259	0.0132922014533959\\
260	0.0132920702907192\\
261	0.0132919368267865\\
262	0.0132918010204365\\
263	0.013291662829748\\
264	0.0132915222120259\\
265	0.0132913791237858\\
266	0.0132912335207389\\
267	0.0132910853577772\\
268	0.0132909345889567\\
269	0.0132907811674824\\
270	0.0132906250456914\\
271	0.013290466175036\\
272	0.0132903045060676\\
273	0.0132901399884183\\
274	0.0132899725707841\\
275	0.0132898022009064\\
276	0.0132896288255542\\
277	0.0132894523905047\\
278	0.0132892728405251\\
279	0.0132890901193526\\
280	0.0132889041696748\\
281	0.0132887149331099\\
282	0.013288522350186\\
283	0.0132883263603201\\
284	0.0132881269017972\\
285	0.0132879239117483\\
286	0.0132877173261288\\
287	0.0132875070796957\\
288	0.013287293105985\\
289	0.0132870753372887\\
290	0.0132868537046304\\
291	0.0132866281377423\\
292	0.01328639856504\\
293	0.0132861649135978\\
294	0.0132859271091234\\
295	0.0132856850759324\\
296	0.0132854387369218\\
297	0.0132851880135434\\
298	0.0132849328257772\\
299	0.0132846730921032\\
300	0.0132844087294743\\
301	0.0132841396532871\\
302	0.0132838657773537\\
303	0.0132835870138722\\
304	0.0132833032733969\\
305	0.0132830144648082\\
306	0.0132827204952822\\
307	0.0132824212702593\\
308	0.0132821166934128\\
309	0.0132818066666168\\
310	0.0132814910899141\\
311	0.0132811698614832\\
312	0.0132808428776046\\
313	0.0132805100326276\\
314	0.0132801712189357\\
315	0.0132798263269123\\
316	0.013279475244905\\
317	0.0132791178591908\\
318	0.0132787540539397\\
319	0.0132783837111783\\
320	0.0132780067107532\\
321	0.0132776229302937\\
322	0.0132772322451739\\
323	0.0132768345284749\\
324	0.0132764296509462\\
325	0.0132760174809667\\
326	0.0132755978845057\\
327	0.0132751707250825\\
328	0.013274735863727\\
329	0.0132742931589386\\
330	0.0132738424666455\\
331	0.0132733836401631\\
332	0.0132729165301522\\
333	0.0132724409845768\\
334	0.0132719568486613\\
335	0.0132714639648473\\
336	0.0132709621727497\\
337	0.013270451309113\\
338	0.0132699312077659\\
339	0.0132694016995764\\
340	0.0132688626124059\\
341	0.0132683137710623\\
342	0.0132677549972528\\
343	0.013267186109536\\
344	0.0132666069232725\\
345	0.0132660172505756\\
346	0.0132654169002599\\
347	0.0132648056777896\\
348	0.013264183385225\\
349	0.0132635498211678\\
350	0.0132629047807052\\
351	0.0132622480553515\\
352	0.0132615794329885\\
353	0.0132608986978037\\
354	0.0132602056302259\\
355	0.013259500006858\\
356	0.0132587816004079\\
357	0.0132580501796152\\
358	0.0132573055091747\\
359	0.0132565473496562\\
360	0.01325577545742\\
361	0.0132549895845276\\
362	0.0132541894786472\\
363	0.0132533748829535\\
364	0.013252545536022\\
365	0.0132517011717154\\
366	0.0132508415190639\\
367	0.0132499663021366\\
368	0.0132490752399053\\
369	0.013248168046098\\
370	0.0132472444290431\\
371	0.0132463040915032\\
372	0.0132453467304968\\
373	0.0132443720371084\\
374	0.0132433796962854\\
375	0.0132423693866218\\
376	0.0132413407801275\\
377	0.013240293541982\\
378	0.0132392273302733\\
379	0.0132381417957202\\
380	0.0132370365813775\\
381	0.0132359113223241\\
382	0.0132347656453337\\
383	0.0132335991685267\\
384	0.0132324115010047\\
385	0.0132312022424661\\
386	0.0132299709828036\\
387	0.0132287173016828\\
388	0.0132274407681013\\
389	0.0132261409399316\\
390	0.0132248173634389\\
391	0.0132234695727793\\
392	0.0132220970894823\\
393	0.01322069942192\\
394	0.0132192760647692\\
395	0.013217826498462\\
396	0.0132163501886507\\
397	0.0132148465856894\\
398	0.0132133151241093\\
399	0.0132117552220694\\
400	0.0132101662807385\\
401	0.013208547683455\\
402	0.013206898794218\\
403	0.0132052189543544\\
404	0.0132035074752677\\
405	0.013201763623997\\
406	0.0131999867324026\\
407	0.0131981762026784\\
408	0.0131963312391025\\
409	0.0131944510159958\\
410	0.0131925346757647\\
411	0.0131905813269307\\
412	0.0131885900424134\\
413	0.0131865598587482\\
414	0.0131844897778328\\
415	0.0131823787746775\\
416	0.0131802258177056\\
417	0.0131780299097869\\
418	0.0131757901401662\\
419	0.0131735042328075\\
420	0.0131711709547028\\
421	0.013168789225222\\
422	0.013166357945463\\
423	0.0131638759998031\\
424	0.0131613422578505\\
425	0.0131587555768702\\
426	0.0131561148047549\\
427	0.0131534187834936\\
428	0.0131506663535254\\
429	0.0131478563588932\\
430	0.013144987653059\\
431	0.0131420591057864\\
432	0.0131390696121651\\
433	0.0131360181036858\\
434	0.0131329035668836\\
435	0.0131297250864497\\
436	0.0131264819667674\\
437	0.0131231741246124\\
438	0.013119803437367\\
439	0.0131163784993131\\
440	0.0131137359531825\\
441	0.0131116413008394\\
442	0.013109502564207\\
443	0.0131073184069381\\
444	0.0131050873694988\\
445	0.0131028078459677\\
446	0.0131004780558094\\
447	0.0130980960094271\\
448	0.0130956594659778\\
449	0.0130931658815169\\
450	0.0130906123449841\\
451	0.013087995498833\\
452	0.0130853114402517\\
453	0.0130825555980898\\
454	0.0130797225802734\\
455	0.0130768059868708\\
456	0.0130737981273337\\
457	0.0130706896151962\\
458	0.0130674687677329\\
459	0.0130641201752447\\
460	0.0130606208769816\\
461	0.0130569023893881\\
462	0.013049724837475\\
463	0.013042422609164\\
464	0.013034994567441\\
465	0.0130274155614741\\
466	0.013019682130087\\
467	0.0130117907274541\\
468	0.0130037376847426\\
469	0.012995519078651\\
470	0.0129871302867297\\
471	0.0129785689671273\\
472	0.0129698337951504\\
473	0.0129609211607287\\
474	0.0129518275786133\\
475	0.0129425499668079\\
476	0.0129330864017597\\
477	0.0129234381878213\\
478	0.012913584446632\\
479	0.0129035167419422\\
480	0.0128932304355082\\
481	0.0128827207061132\\
482	0.0128719837966622\\
483	0.0128610155427485\\
484	0.012849811625554\\
485	0.012838368064323\\
486	0.0128266819624904\\
487	0.0128147488374125\\
488	0.0128025584739185\\
489	0.0127901045718158\\
490	0.0127773808194113\\
491	0.0127643809417649\\
492	0.012751098762356\\
493	0.012737528281468\\
494	0.0127236637754498\\
495	0.0127094999222009\\
496	0.0126950319601255\\
497	0.0126802558916504\\
498	0.0126651687524712\\
499	0.0126497689991266\\
500	0.0126340571747312\\
501	0.0126180373935847\\
502	0.0126017215656472\\
503	0.0125851563441973\\
504	0.0125721011719379\\
505	0.0125639330051409\\
506	0.0125555005130236\\
507	0.0125467853759751\\
508	0.0125346611310508\\
509	0.0125218528537396\\
510	0.0125087480954591\\
511	0.012495330994115\\
512	0.012481583868136\\
513	0.0124674865405492\\
514	0.0124530158565289\\
515	0.0124381451063683\\
516	0.012422843315943\\
517	0.0124070742452977\\
518	0.0123907951739118\\
519	0.0123739555914239\\
520	0.0123564951170039\\
521	0.0123383397028977\\
522	0.0123193938071307\\
523	0.0122995191922814\\
524	0.0122752467093493\\
525	0.0122315354138963\\
526	0.0121869359452254\\
527	0.0121414467128114\\
528	0.0120950706271324\\
529	0.0120478054049151\\
530	0.0119995131456163\\
531	0.0119499286206226\\
532	0.0118989940718964\\
533	0.0118466471540871\\
534	0.0117928199997251\\
535	0.0117374369605637\\
536	0.0116804650467312\\
537	0.0116219596772739\\
538	0.0115666270542502\\
539	0.0115400648252121\\
540	0.0115127153882832\\
541	0.0114845647858405\\
542	0.0114556020442396\\
543	0.011425820358175\\
544	0.0113952192429664\\
545	0.0113637493192464\\
546	0.0113314022173966\\
547	0.0112982095221342\\
548	0.0112641817595797\\
549	0.0112291150552606\\
550	0.0111935039058502\\
551	0.0111561000943894\\
552	0.0111163453495756\\
553	0.0110701841428973\\
554	0.0110037507814572\\
555	0.0109341683335819\\
556	0.0108608936701748\\
557	0.0107831014065912\\
558	0.0106962357997787\\
559	0.0105129233178253\\
560	0.0103249861752007\\
561	0.0101317886664457\\
562	0.00993267883641647\\
563	0.00972720135460083\\
564	0.00960036565758178\\
565	0.0095149101347733\\
566	0.00943259613537835\\
567	0.00935440460098425\\
568	0.00928095805545381\\
569	0.00921567801913973\\
570	0.0091508731428269\\
571	0.00908578738650886\\
572	0.00902035159158171\\
573	0.00895696432559356\\
574	0.00889381827605941\\
575	0.00883133125598114\\
576	0.00877034863198628\\
577	0.00870594004185222\\
578	0.00864172161605577\\
579	0.00857843594035942\\
580	0.00849639660057001\\
581	0.00837926229131001\\
582	0.00812898741429046\\
583	0.00765189796216335\\
584	0.00723415767715551\\
585	0.00713458485247705\\
586	0.00704303138858332\\
587	0.00695323519973369\\
588	0.00686232363693157\\
589	0.00676970639866382\\
590	0.00667491749014196\\
591	0.00657717743770035\\
592	0.00647482429634717\\
593	0.00636364764633835\\
594	0.00623271716130668\\
595	0.00605340994148281\\
596	0.00575058001197163\\
597	0.00512683753504545\\
598	0.00366374385960312\\
599	0\\
600	0\\
};
\addplot [color=blue!80!mycolor9,solid,forget plot]
  table[row sep=crcr]{%
1	0.0134978178999\\
2	0.0134978172579578\\
3	0.0134978166052197\\
4	0.0134978159415042\\
5	0.0134978152666268\\
6	0.0134978145803998\\
7	0.0134978138826324\\
8	0.0134978131731304\\
9	0.0134978124516967\\
10	0.0134978117181304\\
11	0.0134978109722275\\
12	0.0134978102137807\\
13	0.0134978094425788\\
14	0.0134978086584073\\
15	0.0134978078610481\\
16	0.0134978070502793\\
17	0.0134978062258752\\
18	0.0134978053876066\\
19	0.0134978045352399\\
20	0.0134978036685382\\
21	0.01349780278726\\
22	0.0134978018911601\\
23	0.013497800979989\\
24	0.013497800053493\\
25	0.0134977991114141\\
26	0.01349779815349\\
27	0.0134977971794538\\
28	0.0134977961890343\\
29	0.0134977951819556\\
30	0.013497794157937\\
31	0.0134977931166933\\
32	0.0134977920579342\\
33	0.0134977909813647\\
34	0.0134977898866847\\
35	0.013497788773589\\
36	0.0134977876417672\\
37	0.0134977864909037\\
38	0.0134977853206775\\
39	0.0134977841307622\\
40	0.0134977829208257\\
41	0.0134977816905304\\
42	0.013497780439533\\
43	0.0134977791674841\\
44	0.0134977778740287\\
45	0.0134977765588055\\
46	0.0134977752214472\\
47	0.0134977738615801\\
48	0.0134977724788243\\
49	0.0134977710727933\\
50	0.013497769643094\\
51	0.0134977681893266\\
52	0.0134977667110846\\
53	0.0134977652079544\\
54	0.0134977636795155\\
55	0.01349776212534\\
56	0.0134977605449928\\
57	0.0134977589380314\\
58	0.0134977573040056\\
59	0.0134977556424576\\
60	0.0134977539529217\\
61	0.0134977522349243\\
62	0.0134977504879836\\
63	0.0134977487116094\\
64	0.0134977469053034\\
65	0.0134977450685585\\
66	0.0134977432008589\\
67	0.01349774130168\\
68	0.0134977393704882\\
69	0.0134977374067407\\
70	0.0134977354098852\\
71	0.0134977333793601\\
72	0.0134977313145939\\
73	0.0134977292150055\\
74	0.0134977270800034\\
75	0.0134977249089863\\
76	0.0134977227013422\\
77	0.0134977204564487\\
78	0.0134977181736724\\
79	0.0134977158523692\\
80	0.0134977134918836\\
81	0.0134977110915489\\
82	0.0134977086506867\\
83	0.013497706168607\\
84	0.0134977036446076\\
85	0.0134977010779743\\
86	0.0134976984679802\\
87	0.013497695813886\\
88	0.0134976931149394\\
89	0.013497690370375\\
90	0.013497687579414\\
91	0.0134976847412642\\
92	0.0134976818551193\\
93	0.013497678920159\\
94	0.0134976759355487\\
95	0.0134976729004391\\
96	0.0134976698139662\\
97	0.0134976666752506\\
98	0.0134976634833976\\
99	0.0134976602374969\\
100	0.0134976569366219\\
101	0.0134976535798301\\
102	0.0134976501661622\\
103	0.013497646694642\\
104	0.013497643164276\\
105	0.0134976395740536\\
106	0.013497635922946\\
107	0.0134976322099063\\
108	0.0134976284338693\\
109	0.0134976245937507\\
110	0.0134976206884474\\
111	0.0134976167168363\\
112	0.0134976126777749\\
113	0.0134976085701001\\
114	0.0134976043926285\\
115	0.0134976001441555\\
116	0.0134975958234553\\
117	0.0134975914292802\\
118	0.0134975869603604\\
119	0.0134975824154037\\
120	0.0134975777930947\\
121	0.0134975730920949\\
122	0.0134975683110417\\
123	0.0134975634485486\\
124	0.0134975585032043\\
125	0.0134975534735723\\
126	0.0134975483581907\\
127	0.0134975431555715\\
128	0.0134975378642004\\
129	0.0134975324825359\\
130	0.0134975270090093\\
131	0.013497521442024\\
132	0.0134975157799547\\
133	0.0134975100211477\\
134	0.0134975041639194\\
135	0.0134974982065567\\
136	0.0134974921473157\\
137	0.0134974859844219\\
138	0.0134974797160689\\
139	0.0134974733404187\\
140	0.0134974668556004\\
141	0.0134974602597101\\
142	0.01349745355081\\
143	0.0134974467269282\\
144	0.013497439786058\\
145	0.0134974327261571\\
146	0.0134974255451471\\
147	0.0134974182409133\\
148	0.0134974108113034\\
149	0.0134974032541275\\
150	0.0134973955671571\\
151	0.0134973877481245\\
152	0.0134973797947226\\
153	0.0134973717046035\\
154	0.0134973634753786\\
155	0.0134973551046174\\
156	0.0134973465898471\\
157	0.0134973379285518\\
158	0.013497329118172\\
159	0.0134973201561037\\
160	0.0134973110396975\\
161	0.0134973017662586\\
162	0.0134972923330452\\
163	0.0134972827372684\\
164	0.0134972729760911\\
165	0.0134972630466271\\
166	0.013497252945941\\
167	0.0134972426710464\\
168	0.0134972322189059\\
169	0.0134972215864299\\
170	0.0134972107704756\\
171	0.0134971997678466\\
172	0.0134971885752912\\
173	0.0134971771895024\\
174	0.0134971656071162\\
175	0.0134971538247108\\
176	0.0134971418388059\\
177	0.013497129645861\\
178	0.0134971172422751\\
179	0.013497104624385\\
180	0.0134970917884642\\
181	0.0134970787307219\\
182	0.0134970654473019\\
183	0.0134970519342809\\
184	0.0134970381876675\\
185	0.013497024203401\\
186	0.0134970099773498\\
187	0.0134969955053099\\
188	0.0134969807830041\\
189	0.01349696580608\\
190	0.0134969505701086\\
191	0.0134969350705833\\
192	0.0134969193029179\\
193	0.0134969032624452\\
194	0.013496886944416\\
195	0.013496870343997\\
196	0.0134968534562693\\
197	0.0134968362762275\\
198	0.0134968187987772\\
199	0.0134968010187341\\
200	0.0134967829308221\\
201	0.0134967645296715\\
202	0.0134967458098175\\
203	0.0134967267656986\\
204	0.0134967073916543\\
205	0.0134966876819239\\
206	0.0134966676306442\\
207	0.0134966472318479\\
208	0.0134966264794617\\
209	0.0134966053673042\\
210	0.013496583889084\\
211	0.0134965620383977\\
212	0.0134965398087279\\
213	0.0134965171934409\\
214	0.0134964941857848\\
215	0.0134964707788873\\
216	0.0134964469657533\\
217	0.0134964227392627\\
218	0.0134963980921684\\
219	0.0134963730170935\\
220	0.0134963475065292\\
221	0.0134963215528323\\
222	0.0134962951482228\\
223	0.013496268284781\\
224	0.0134962409544455\\
225	0.0134962131490101\\
226	0.0134961848601215\\
227	0.0134961560792762\\
228	0.013496126797818\\
229	0.0134960970069349\\
230	0.0134960666976567\\
231	0.0134960358608514\\
232	0.0134960044872227\\
233	0.0134959725673068\\
234	0.0134959400914694\\
235	0.013495907049902\\
236	0.0134958734326196\\
237	0.0134958392294564\\
238	0.0134958044300633\\
239	0.0134957690239038\\
240	0.013495733000251\\
241	0.013495696348184\\
242	0.0134956590565837\\
243	0.0134956211141301\\
244	0.0134955825092976\\
245	0.0134955432303518\\
246	0.0134955032653452\\
247	0.0134954626021137\\
248	0.013495421228272\\
249	0.0134953791312099\\
250	0.0134953362980878\\
251	0.0134952927158327\\
252	0.0134952483711337\\
253	0.0134952032504375\\
254	0.0134951573399441\\
255	0.0134951106256019\\
256	0.0134950630931033\\
257	0.0134950147278797\\
258	0.0134949655150968\\
259	0.0134949154396496\\
260	0.0134948644861573\\
261	0.0134948126389584\\
262	0.0134947598821051\\
263	0.0134947061993585\\
264	0.0134946515741826\\
265	0.0134945959897392\\
266	0.0134945394288824\\
267	0.0134944818741525\\
268	0.0134944233077706\\
269	0.0134943637116323\\
270	0.0134943030673021\\
271	0.0134942413560073\\
272	0.0134941785586313\\
273	0.0134941146557079\\
274	0.0134940496274146\\
275	0.0134939834535662\\
276	0.0134939161136079\\
277	0.0134938475866091\\
278	0.013493777851256\\
279	0.0134937068858451\\
280	0.0134936346682759\\
281	0.0134935611760438\\
282	0.0134934863862329\\
283	0.0134934102755085\\
284	0.0134933328201099\\
285	0.0134932539958423\\
286	0.0134931737780695\\
287	0.0134930921417059\\
288	0.0134930090612088\\
289	0.0134929245105699\\
290	0.0134928384633077\\
291	0.0134927508924586\\
292	0.0134926617705692\\
293	0.0134925710696874\\
294	0.0134924787613537\\
295	0.013492384816593\\
296	0.0134922892059054\\
297	0.0134921918992571\\
298	0.013492092866072\\
299	0.013491992075222\\
300	0.013491889495018\\
301	0.0134917850932007\\
302	0.0134916788369308\\
303	0.0134915706927797\\
304	0.01349146062672\\
305	0.0134913486041155\\
306	0.0134912345897113\\
307	0.0134911185476244\\
308	0.0134910004413329\\
309	0.0134908802336668\\
310	0.0134907578867968\\
311	0.0134906333622247\\
312	0.0134905066207731\\
313	0.0134903776225741\\
314	0.0134902463270597\\
315	0.0134901126929506\\
316	0.0134899766782455\\
317	0.0134898382402102\\
318	0.0134896973353671\\
319	0.0134895539194837\\
320	0.0134894079475616\\
321	0.0134892593738252\\
322	0.0134891081517108\\
323	0.0134889542338546\\
324	0.0134887975720812\\
325	0.0134886381173923\\
326	0.0134884758199544\\
327	0.0134883106290873\\
328	0.0134881424932514\\
329	0.0134879713600358\\
330	0.0134877971761456\\
331	0.0134876198873892\\
332	0.0134874394386652\\
333	0.0134872557739495\\
334	0.0134870688362815\\
335	0.01348687856775\\
336	0.0134866849094798\\
337	0.0134864878016161\\
338	0.0134862871833102\\
339	0.0134860829927034\\
340	0.0134858751669114\\
341	0.013485663642007\\
342	0.0134854483530034\\
343	0.0134852292338353\\
344	0.0134850062173405\\
345	0.0134847792352399\\
346	0.0134845482181164\\
347	0.0134843130953929\\
348	0.0134840737953097\\
349	0.0134838302448991\\
350	0.0134835823699599\\
351	0.0134833300950298\\
352	0.013483073343356\\
353	0.013482812036864\\
354	0.0134825460961245\\
355	0.0134822754403174\\
356	0.0134819999871948\\
357	0.0134817196530399\\
358	0.0134814343526238\\
359	0.0134811439991597\\
360	0.013480848504253\\
361	0.0134805477778486\\
362	0.0134802417281746\\
363	0.0134799302616814\\
364	0.0134796132829776\\
365	0.0134792906947609\\
366	0.0134789623977443\\
367	0.0134786282905782\\
368	0.013478288269766\\
369	0.013477942229576\\
370	0.0134775900619459\\
371	0.013477231656383\\
372	0.0134768668998567\\
373	0.0134764956766858\\
374	0.0134761178684185\\
375	0.0134757333537053\\
376	0.013475342008165\\
377	0.013474943704243\\
378	0.0134745383110612\\
379	0.0134741256942608\\
380	0.0134737057158351\\
381	0.0134732782339542\\
382	0.0134728431027795\\
383	0.0134724001722683\\
384	0.0134719492879671\\
385	0.0134714902907932\\
386	0.0134710230168033\\
387	0.0134705472969484\\
388	0.0134700629568116\\
389	0.0134695698163304\\
390	0.0134690676894971\\
391	0.0134685563840394\\
392	0.0134680357010739\\
393	0.0134675054347311\\
394	0.0134669653717453\\
395	0.0134664152910036\\
396	0.0134658549630444\\
397	0.0134652841494936\\
398	0.0134647026024256\\
399	0.0134641100636296\\
400	0.0134635062637539\\
401	0.0134628909212936\\
402	0.013462263741382\\
403	0.0134616244144063\\
404	0.0134609726147129\\
405	0.0134603080004289\\
406	0.0134596302091251\\
407	0.0134589388521121\\
408	0.0134582335196413\\
409	0.013457513779399\\
410	0.0134567791749599\\
411	0.0134560292242418\\
412	0.0134552634180291\\
413	0.0134544812186721\\
414	0.0134536820590924\\
415	0.013452865342125\\
416	0.0134520304396445\\
417	0.0134511766889951\\
418	0.0134503033799075\\
419	0.0134494098053107\\
420	0.0134484952161293\\
421	0.0134475588091908\\
422	0.0134465997209443\\
423	0.0134456170202447\\
424	0.0134446097000415\\
425	0.0134435766677741\\
426	0.0134425167342392\\
427	0.0134414286006498\\
428	0.0134403108435325\\
429	0.0134391618970388\\
430	0.0134379800321376\\
431	0.0134367633319616\\
432	0.0134355096621471\\
433	0.0134342166340143\\
434	0.0134328815553418\\
435	0.0134315013537595\\
436	0.0134300724257486\\
437	0.0134285902553484\\
438	0.0134270482693102\\
439	0.0134254340662547\\
440	0.0134230701205249\\
441	0.0134201546775523\\
442	0.0134171699905031\\
443	0.0134141137796265\\
444	0.0134109836592755\\
445	0.0134077771331828\\
446	0.0134044915899481\\
447	0.0134011242989055\\
448	0.0133976724065945\\
449	0.0133941329341324\\
450	0.013390502775878\\
451	0.0133867786998934\\
452	0.0133829573508258\\
453	0.0133790352558841\\
454	0.013375008834235\\
455	0.0133708744083719\\
456	0.0133666282131116\\
457	0.0133622663856774\\
458	0.0133577848874649\\
459	0.013353179229957\\
460	0.0133484436952482\\
461	0.0133435636803895\\
462	0.013337830751605\\
463	0.0133319767046786\\
464	0.0133260221190616\\
465	0.0133211591017198\\
466	0.0133162002847412\\
467	0.0133111432370257\\
468	0.0133059854032918\\
469	0.0133007240718036\\
470	0.0132953563194378\\
471	0.0132898794486607\\
472	0.0132842906941347\\
473	0.01327858670593\\
474	0.0132727639740842\\
475	0.013266818869552\\
476	0.0132607477374956\\
477	0.0132545470174358\\
478	0.0132482095608695\\
479	0.0132417303888866\\
480	0.0132351049762897\\
481	0.0132283285391178\\
482	0.0132213959758157\\
483	0.0132143018473306\\
484	0.0132070403490988\\
485	0.0131996052655958\\
486	0.0131919898787866\\
487	0.0131841869234585\\
488	0.0131761887111239\\
489	0.0131679869559688\\
490	0.0131595726981769\\
491	0.0131509362141319\\
492	0.0131420669108388\\
493	0.0131329532012644\\
494	0.0131235823564324\\
495	0.0131139403288144\\
496	0.0131040115393305\\
497	0.0130937786156154\\
498	0.0130832220576497\\
499	0.0130723197743846\\
500	0.0130610463365686\\
501	0.0130493714782708\\
502	0.0130372563518538\\
503	0.0130246440918271\\
504	0.0130085421346071\\
505	0.0129902929400844\\
506	0.0129743442802123\\
507	0.0129580375375457\\
508	0.012941368696631\\
509	0.0129243216642253\\
510	0.012906892506889\\
511	0.0128890507053126\\
512	0.0128707638287753\\
513	0.0128520174734674\\
514	0.0128327968478921\\
515	0.0128130867950133\\
516	0.0127928718190546\\
517	0.0127721361185569\\
518	0.0127508636208616\\
519	0.0127290379884259\\
520	0.0127066425217938\\
521	0.0126836597216777\\
522	0.0126600697331428\\
523	0.0126358452120687\\
524	0.0126102012522706\\
525	0.0125797637674508\\
526	0.0125486008261555\\
527	0.012516701834906\\
528	0.0124840859968582\\
529	0.0124506403646017\\
530	0.012421882458765\\
531	0.0124011496190387\\
532	0.012379760895246\\
533	0.0123576581795014\\
534	0.0123347715309565\\
535	0.0123110147450163\\
536	0.0122862828872557\\
537	0.0122603233418825\\
538	0.0122289258663907\\
539	0.0121717976209731\\
540	0.0121131111946628\\
541	0.0120528091740456\\
542	0.011990832808659\\
543	0.0119271218968977\\
544	0.0118616123748213\\
545	0.0117942284389449\\
546	0.0117248949720159\\
547	0.0116536580977404\\
548	0.0115806189640546\\
549	0.0115053699585196\\
550	0.0114274468585565\\
551	0.011381698839189\\
552	0.0113431571959004\\
553	0.0113030362611606\\
554	0.0112612795220593\\
555	0.0112177965128137\\
556	0.0111724851618522\\
557	0.0111252104114285\\
558	0.0110746474923157\\
559	0.0110014333785487\\
560	0.0109261242630267\\
561	0.0108485432095232\\
562	0.0107673722301904\\
563	0.0106831091580284\\
564	0.0105280498514088\\
565	0.0103348601344522\\
566	0.0101363722907275\\
567	0.0099316542325717\\
568	0.00972020831209711\\
569	0.00950501606326739\\
570	0.00940828015028893\\
571	0.00931333402576754\\
572	0.00922080861130992\\
573	0.0091312653549379\\
574	0.00904753506838823\\
575	0.00897092077645637\\
576	0.00889344865767624\\
577	0.00881637725871559\\
578	0.0087410276765376\\
579	0.00866753613908096\\
580	0.0085942859796107\\
581	0.00852160538424648\\
582	0.00842757392739441\\
583	0.00830398138734459\\
584	0.00812183334290264\\
585	0.0076557836760961\\
586	0.00717616754392714\\
587	0.00697576093390482\\
588	0.00687016646845473\\
589	0.00677423682775788\\
590	0.00667717773539374\\
591	0.00657820653135076\\
592	0.00647518044191995\\
593	0.00636372123806462\\
594	0.00623271716130668\\
595	0.00605340994148281\\
596	0.00575058001197164\\
597	0.00512683753504545\\
598	0.00366374385960312\\
599	0\\
600	0\\
};
\addplot [color=blue,solid,forget plot]
  table[row sep=crcr]{%
1	0.0135820465983962\\
2	0.0135820465214251\\
3	0.0135820464431596\\
4	0.0135820463635778\\
5	0.0135820462826577\\
6	0.0135820462003768\\
7	0.0135820461167122\\
8	0.0135820460316406\\
9	0.0135820459451383\\
10	0.0135820458571813\\
11	0.0135820457677451\\
12	0.0135820456768049\\
13	0.0135820455843352\\
14	0.0135820454903105\\
15	0.0135820453947045\\
16	0.0135820452974906\\
17	0.0135820451986418\\
18	0.0135820450981306\\
19	0.0135820449959289\\
20	0.0135820448920084\\
21	0.01358204478634\\
22	0.0135820446788944\\
23	0.0135820445696416\\
24	0.0135820444585513\\
25	0.0135820443455925\\
26	0.0135820442307336\\
27	0.0135820441139428\\
28	0.0135820439951874\\
29	0.0135820438744345\\
30	0.0135820437516502\\
31	0.0135820436268004\\
32	0.0135820434998503\\
33	0.0135820433707646\\
34	0.0135820432395071\\
35	0.0135820431060413\\
36	0.0135820429703299\\
37	0.0135820428323352\\
38	0.0135820426920185\\
39	0.0135820425493407\\
40	0.0135820424042621\\
41	0.013582042256742\\
42	0.0135820421067394\\
43	0.0135820419542122\\
44	0.0135820417991179\\
45	0.0135820416414132\\
46	0.0135820414810539\\
47	0.0135820413179953\\
48	0.0135820411521918\\
49	0.0135820409835969\\
50	0.0135820408121636\\
51	0.0135820406378438\\
52	0.0135820404605887\\
53	0.0135820402803489\\
54	0.0135820400970737\\
55	0.0135820399107118\\
56	0.013582039721211\\
57	0.0135820395285183\\
58	0.0135820393325795\\
59	0.0135820391333399\\
60	0.0135820389307433\\
61	0.0135820387247331\\
62	0.0135820385152513\\
63	0.0135820383022393\\
64	0.013582038085637\\
65	0.0135820378653837\\
66	0.0135820376414175\\
67	0.0135820374136754\\
68	0.0135820371820934\\
69	0.0135820369466062\\
70	0.0135820367071475\\
71	0.0135820364636501\\
72	0.0135820362160452\\
73	0.013582035964263\\
74	0.0135820357082327\\
75	0.0135820354478819\\
76	0.0135820351831372\\
77	0.0135820349139238\\
78	0.0135820346401658\\
79	0.0135820343617857\\
80	0.0135820340787049\\
81	0.0135820337908432\\
82	0.0135820334981193\\
83	0.0135820332004502\\
84	0.0135820328977516\\
85	0.0135820325899378\\
86	0.0135820322769214\\
87	0.0135820319586137\\
88	0.0135820316349243\\
89	0.0135820313057612\\
90	0.0135820309710309\\
91	0.0135820306306382\\
92	0.0135820302844863\\
93	0.0135820299324767\\
94	0.013582029574509\\
95	0.0135820292104813\\
96	0.0135820288402897\\
97	0.0135820284638286\\
98	0.0135820280809905\\
99	0.013582027691666\\
100	0.0135820272957438\\
101	0.0135820268931106\\
102	0.0135820264836511\\
103	0.0135820260672482\\
104	0.0135820256437823\\
105	0.013582025213132\\
106	0.0135820247751737\\
107	0.0135820243297816\\
108	0.0135820238768277\\
109	0.0135820234161817\\
110	0.0135820229477109\\
111	0.0135820224712804\\
112	0.0135820219867529\\
113	0.0135820214939887\\
114	0.0135820209928454\\
115	0.0135820204831783\\
116	0.01358201996484\\
117	0.0135820194376807\\
118	0.0135820189015476\\
119	0.0135820183562856\\
120	0.0135820178017363\\
121	0.0135820172377391\\
122	0.0135820166641301\\
123	0.0135820160807426\\
124	0.0135820154874071\\
125	0.0135820148839508\\
126	0.0135820142701982\\
127	0.0135820136459703\\
128	0.0135820130110851\\
129	0.0135820123653575\\
130	0.0135820117085988\\
131	0.0135820110406172\\
132	0.0135820103612174\\
133	0.0135820096702007\\
134	0.0135820089673648\\
135	0.0135820082525038\\
136	0.0135820075254081\\
137	0.0135820067858647\\
138	0.0135820060336564\\
139	0.0135820052685625\\
140	0.0135820044903582\\
141	0.0135820036988148\\
142	0.0135820028936996\\
143	0.0135820020747757\\
144	0.0135820012418021\\
145	0.0135820003945335\\
146	0.0135819995327205\\
147	0.013581998656109\\
148	0.0135819977644407\\
149	0.0135819968574525\\
150	0.0135819959348771\\
151	0.0135819949964423\\
152	0.0135819940418709\\
153	0.0135819930708814\\
154	0.0135819920831871\\
155	0.0135819910784962\\
156	0.0135819900565121\\
157	0.013581989016933\\
158	0.0135819879594516\\
159	0.0135819868837557\\
160	0.0135819857895275\\
161	0.0135819846764436\\
162	0.0135819835441754\\
163	0.0135819823923881\\
164	0.0135819812207417\\
165	0.01358198002889\\
166	0.0135819788164809\\
167	0.0135819775831565\\
168	0.0135819763285523\\
169	0.013581975052298\\
170	0.0135819737540167\\
171	0.0135819724333251\\
172	0.0135819710898334\\
173	0.0135819697231449\\
174	0.0135819683328563\\
175	0.0135819669185572\\
176	0.0135819654798302\\
177	0.0135819640162508\\
178	0.0135819625273871\\
179	0.0135819610127997\\
180	0.0135819594720415\\
181	0.0135819579046579\\
182	0.0135819563101863\\
183	0.0135819546881559\\
184	0.0135819530380879\\
185	0.0135819513594951\\
186	0.0135819496518816\\
187	0.0135819479147429\\
188	0.0135819461475659\\
189	0.013581944349828\\
190	0.0135819425209979\\
191	0.0135819406605345\\
192	0.0135819387678875\\
193	0.0135819368424968\\
194	0.0135819348837922\\
195	0.0135819328911938\\
196	0.013581930864111\\
197	0.0135819288019433\\
198	0.0135819267040791\\
199	0.0135819245698963\\
200	0.0135819223987615\\
201	0.0135819201900305\\
202	0.0135819179430472\\
203	0.0135819156571442\\
204	0.0135819133316421\\
205	0.0135819109658496\\
206	0.013581908559063\\
207	0.0135819061105662\\
208	0.0135819036196302\\
209	0.0135819010855133\\
210	0.0135818985074604\\
211	0.013581895884703\\
212	0.013581893216459\\
213	0.0135818905019323\\
214	0.0135818877403126\\
215	0.0135818849307752\\
216	0.0135818820724806\\
217	0.0135818791645746\\
218	0.0135818762061873\\
219	0.0135818731964336\\
220	0.0135818701344125\\
221	0.0135818670192067\\
222	0.0135818638498828\\
223	0.0135818606254904\\
224	0.0135818573450623\\
225	0.0135818540076139\\
226	0.0135818506121428\\
227	0.0135818471576288\\
228	0.0135818436430334\\
229	0.0135818400672992\\
230	0.0135818364293501\\
231	0.0135818327280905\\
232	0.0135818289624052\\
233	0.0135818251311588\\
234	0.0135818212331955\\
235	0.0135818172673387\\
236	0.0135818132323908\\
237	0.0135818091271322\\
238	0.0135818049503216\\
239	0.0135818007006953\\
240	0.0135817963769666\\
241	0.0135817919778257\\
242	0.0135817875019391\\
243	0.0135817829479493\\
244	0.013581778314474\\
245	0.0135817736001062\\
246	0.0135817688034131\\
247	0.0135817639229363\\
248	0.0135817589571907\\
249	0.0135817539046644\\
250	0.0135817487638182\\
251	0.0135817435330849\\
252	0.0135817382108687\\
253	0.013581732795545\\
254	0.0135817272854599\\
255	0.0135817216789292\\
256	0.0135817159742381\\
257	0.0135817101696408\\
258	0.0135817042633598\\
259	0.0135816982535851\\
260	0.013581692138474\\
261	0.0135816859161502\\
262	0.0135816795847033\\
263	0.0135816731421881\\
264	0.0135816665866242\\
265	0.013581659915995\\
266	0.0135816531282472\\
267	0.0135816462212903\\
268	0.0135816391929958\\
269	0.0135816320411962\\
270	0.013581624763685\\
271	0.0135816173582152\\
272	0.0135816098224992\\
273	0.0135816021542076\\
274	0.0135815943509689\\
275	0.0135815864103683\\
276	0.0135815783299473\\
277	0.0135815701072028\\
278	0.013581561739586\\
279	0.0135815532245021\\
280	0.0135815445593093\\
281	0.0135815357413177\\
282	0.0135815267677888\\
283	0.0135815176359345\\
284	0.0135815083429164\\
285	0.0135814988858446\\
286	0.0135814892617772\\
287	0.0135814794677191\\
288	0.0135814695006212\\
289	0.0135814593573797\\
290	0.0135814490348345\\
291	0.0135814385297693\\
292	0.0135814278389096\\
293	0.0135814169589225\\
294	0.0135814058864152\\
295	0.0135813946179344\\
296	0.0135813831499652\\
297	0.0135813714789299\\
298	0.0135813596011872\\
299	0.013581347513031\\
300	0.0135813352106898\\
301	0.013581322690325\\
302	0.0135813099480303\\
303	0.0135812969798305\\
304	0.0135812837816806\\
305	0.0135812703494642\\
306	0.0135812566789932\\
307	0.013581242766006\\
308	0.0135812286061669\\
309	0.0135812141950646\\
310	0.0135811995282114\\
311	0.0135811846010417\\
312	0.0135811694089113\\
313	0.013581153947096\\
314	0.0135811382107905\\
315	0.0135811221951072\\
316	0.0135811058950751\\
317	0.0135810893056386\\
318	0.0135810724216564\\
319	0.0135810552379\\
320	0.013581037749053\\
321	0.0135810199497093\\
322	0.0135810018343726\\
323	0.0135809833974543\\
324	0.0135809646332729\\
325	0.0135809455360527\\
326	0.0135809260999219\\
327	0.0135809063189123\\
328	0.0135808861869568\\
329	0.0135808656978893\\
330	0.0135808448454421\\
331	0.0135808236232456\\
332	0.013580802024826\\
333	0.0135807800436043\\
334	0.0135807576728946\\
335	0.0135807349059027\\
336	0.0135807117357242\\
337	0.0135806881553432\\
338	0.0135806641576301\\
339	0.0135806397353402\\
340	0.0135806148811115\\
341	0.0135805895874628\\
342	0.0135805638467918\\
343	0.0135805376513725\\
344	0.0135805109933533\\
345	0.0135804838647541\\
346	0.0135804562574641\\
347	0.0135804281632386\\
348	0.0135803995736966\\
349	0.013580370480317\\
350	0.0135803408744358\\
351	0.013580310747242\\
352	0.0135802800897743\\
353	0.0135802488929164\\
354	0.0135802171473932\\
355	0.0135801848437656\\
356	0.0135801519724254\\
357	0.0135801185235906\\
358	0.0135800844872985\\
359	0.0135800498534004\\
360	0.0135800146115544\\
361	0.0135799787512186\\
362	0.0135799422616432\\
363	0.0135799051318624\\
364	0.0135798673506859\\
365	0.0135798289066893\\
366	0.0135797897882046\\
367	0.0135797499833094\\
368	0.0135797094798155\\
369	0.0135796682652576\\
370	0.0135796263268804\\
371	0.013579583651625\\
372	0.013579540226115\\
373	0.0135794960366418\\
374	0.0135794510691482\\
375	0.0135794053092124\\
376	0.0135793587420297\\
377	0.0135793113523947\\
378	0.013579263124681\\
379	0.0135792140428211\\
380	0.0135791640902844\\
381	0.0135791132500541\\
382	0.0135790615046026\\
383	0.0135790088358664\\
384	0.0135789552252177\\
385	0.0135789006534355\\
386	0.0135788451006745\\
387	0.013578788546431\\
388	0.0135787309695064\\
389	0.0135786723479684\\
390	0.0135786126591077\\
391	0.0135785518793907\\
392	0.0135784899844079\\
393	0.0135784269488162\\
394	0.0135783627462741\\
395	0.0135782973493692\\
396	0.0135782307295354\\
397	0.0135781628569592\\
398	0.0135780937004717\\
399	0.013578023227424\\
400	0.0135779514035437\\
401	0.0135778781927702\\
402	0.0135778035570696\\
403	0.0135777274562328\\
404	0.0135776498476507\\
405	0.0135775706859901\\
406	0.0135774899230165\\
407	0.0135774075077132\\
408	0.013577323386081\\
409	0.0135772375009321\\
410	0.0135771497916844\\
411	0.0135770601941591\\
412	0.0135769686403918\\
413	0.013576875058459\\
414	0.0135767793723166\\
415	0.0135766815016254\\
416	0.0135765813615334\\
417	0.0135764788625137\\
418	0.0135763739108464\\
419	0.0135762664057732\\
420	0.0135761562383114\\
421	0.0135760432902424\\
422	0.0135759274329503\\
423	0.013575808526082\\
424	0.0135756864160007\\
425	0.0135755609339938\\
426	0.0135754318941911\\
427	0.0135752990911401\\
428	0.0135751622969668\\
429	0.0135750212580314\\
430	0.0135748756909365\\
431	0.0135747252776382\\
432	0.013574569659137\\
433	0.0135744084265066\\
434	0.0135742411061021\\
435	0.0135740671307341\\
436	0.013573885775618\\
437	0.0135736960065876\\
438	0.0135734961215769\\
439	0.0135732829690905\\
440	0.0135729081704589\\
441	0.0135724156562382\\
442	0.013571911372972\\
443	0.0135713949382745\\
444	0.0135708659545813\\
445	0.0135703240090924\\
446	0.0135697686739248\\
447	0.0135691995065429\\
448	0.0135686160505555\\
449	0.0135680178369965\\
450	0.0135674043862314\\
451	0.0135667752106715\\
452	0.0135661298185056\\
453	0.0135654677186674\\
454	0.0135647884271903\\
455	0.0135640914748855\\
456	0.0135633764156356\\
457	0.0135626428333041\\
458	0.0135618903432951\\
459	0.0135611185839216\\
460	0.0135603272023142\\
461	0.0135595158870753\\
462	0.0135586845841885\\
463	0.013557829555261\\
464	0.013556930497075\\
465	0.0135550240133501\\
466	0.0135530808659335\\
467	0.0135510999818061\\
468	0.0135490802236954\\
469	0.0135470203857685\\
470	0.0135449191936659\\
471	0.0135427752867835\\
472	0.0135405872104085\\
473	0.0135383534301489\\
474	0.0135360723311704\\
475	0.0135337422170446\\
476	0.013531361301743\\
477	0.0135289276778965\\
478	0.0135264394440204\\
479	0.0135238946049664\\
480	0.0135212910345299\\
481	0.0135186264629605\\
482	0.0135158984639776\\
483	0.0135131044396253\\
484	0.0135102416025157\\
485	0.0135073069549623\\
486	0.0135042972658295\\
487	0.013501209043831\\
488	0.0134980385020666\\
489	0.0134947815206354\\
490	0.0134914336030213\\
491	0.0134879898249848\\
492	0.0134844447743847\\
493	0.013480792479944\\
494	0.0134770263263802\\
495	0.0134731389523509\\
496	0.0134691221258066\\
497	0.0134649665871592\\
498	0.0134606618401376\\
499	0.0134561958417997\\
500	0.0134515544642281\\
501	0.0134467203786268\\
502	0.0134416703910603\\
503	0.0134363684584538\\
504	0.0134301109338322\\
505	0.0134206996946467\\
506	0.0134088559696673\\
507	0.013396743056697\\
508	0.0133843486283437\\
509	0.0133716599451185\\
510	0.013358663810878\\
511	0.0133453418125183\\
512	0.0133316747671679\\
513	0.013317645481152\\
514	0.0133032352912366\\
515	0.0132884239069768\\
516	0.0132731892362347\\
517	0.0132575071945262\\
518	0.013241351505\\
519	0.0132246935185095\\
520	0.0132075021638263\\
521	0.01318974442653\\
522	0.0131713877897399\\
523	0.0131524097829329\\
524	0.0131338902835204\\
525	0.0131174619770685\\
526	0.0131004259760457\\
527	0.0130827263013375\\
528	0.0130642915501048\\
529	0.013044982201245\\
530	0.0130202912901368\\
531	0.0129876063032289\\
532	0.0129539722259956\\
533	0.0129193559278304\\
534	0.0128837225475325\\
535	0.0128470504617317\\
536	0.0128092533794661\\
537	0.0127742465044304\\
538	0.0127424674144049\\
539	0.0127042975904532\\
540	0.0126649903423508\\
541	0.0126244739772393\\
542	0.0125826653456361\\
543	0.012539477527505\\
544	0.0124948524307789\\
545	0.0124487142253323\\
546	0.0124009538939307\\
547	0.0123513106965161\\
548	0.0122996579653759\\
549	0.0122470277426339\\
550	0.0122112561399698\\
551	0.0121456663671696\\
552	0.0120708318601181\\
553	0.0119937286458208\\
554	0.011914267974539\\
555	0.0118323524226605\\
556	0.0117477215100425\\
557	0.0116602661639905\\
558	0.0115700082060289\\
559	0.0114766789108114\\
560	0.0113796625377174\\
561	0.011278648796618\\
562	0.0111980143005405\\
563	0.0111421292706777\\
564	0.0110695374751004\\
565	0.0109874399878422\\
566	0.0109017749549262\\
567	0.0108119877133317\\
568	0.0107195897617478\\
569	0.0106209820844399\\
570	0.0104193965800069\\
571	0.0102138125244415\\
572	0.0100038158503817\\
573	0.00978711498739125\\
574	0.00956243326742183\\
575	0.00933812331163025\\
576	0.00922701819998303\\
577	0.00911696641490741\\
578	0.00900868084044329\\
579	0.00890286078583315\\
580	0.00879963372380029\\
581	0.00870465551720041\\
582	0.00861264400964136\\
583	0.00851949463509417\\
584	0.00841846506097205\\
585	0.0082781774636422\\
586	0.00814370437258449\\
587	0.00776443978473074\\
588	0.00729624552515174\\
589	0.00683356619974124\\
590	0.00671171289148691\\
591	0.00659293263946882\\
592	0.00648270502684632\\
593	0.00636662498220103\\
594	0.00623343894857411\\
595	0.00605340994148281\\
596	0.00575058001197164\\
597	0.00512683753504545\\
598	0.00366374385960312\\
599	0\\
600	0\\
};
\addplot [color=mycolor10,solid,forget plot]
  table[row sep=crcr]{%
1	0.01359190851983\\
2	0.013591908519541\\
3	0.0135919085192472\\
4	0.0135919085189484\\
5	0.0135919085186446\\
6	0.0135919085183357\\
7	0.0135919085180216\\
8	0.0135919085177023\\
9	0.0135919085173775\\
10	0.0135919085170473\\
11	0.0135919085167115\\
12	0.0135919085163701\\
13	0.013591908516023\\
14	0.01359190851567\\
15	0.0135919085153111\\
16	0.0135919085149461\\
17	0.013591908514575\\
18	0.0135919085141977\\
19	0.013591908513814\\
20	0.0135919085134238\\
21	0.0135919085130271\\
22	0.0135919085126238\\
23	0.0135919085122136\\
24	0.0135919085117965\\
25	0.0135919085113725\\
26	0.0135919085109413\\
27	0.0135919085105028\\
28	0.013591908510057\\
29	0.0135919085096036\\
30	0.0135919085091427\\
31	0.013591908508674\\
32	0.0135919085081974\\
33	0.0135919085077128\\
34	0.01359190850722\\
35	0.0135919085067189\\
36	0.0135919085062094\\
37	0.0135919085056914\\
38	0.0135919085051646\\
39	0.013591908504629\\
40	0.0135919085040843\\
41	0.0135919085035305\\
42	0.0135919085029673\\
43	0.0135919085023947\\
44	0.0135919085018124\\
45	0.0135919085012204\\
46	0.0135919085006183\\
47	0.0135919085000062\\
48	0.0135919084993837\\
49	0.0135919084987508\\
50	0.0135919084981072\\
51	0.0135919084974527\\
52	0.0135919084967873\\
53	0.0135919084961106\\
54	0.0135919084954225\\
55	0.0135919084947229\\
56	0.0135919084940114\\
57	0.013591908493288\\
58	0.0135919084925524\\
59	0.0135919084918044\\
60	0.0135919084910438\\
61	0.0135919084902704\\
62	0.0135919084894839\\
63	0.0135919084886842\\
64	0.013591908487871\\
65	0.0135919084870441\\
66	0.0135919084862032\\
67	0.0135919084853482\\
68	0.0135919084844788\\
69	0.0135919084835947\\
70	0.0135919084826957\\
71	0.0135919084817815\\
72	0.0135919084808519\\
73	0.0135919084799066\\
74	0.0135919084789454\\
75	0.0135919084779679\\
76	0.013591908476974\\
77	0.0135919084759632\\
78	0.0135919084749354\\
79	0.0135919084738903\\
80	0.0135919084728274\\
81	0.0135919084717467\\
82	0.0135919084706477\\
83	0.0135919084695301\\
84	0.0135919084683936\\
85	0.0135919084672379\\
86	0.0135919084660627\\
87	0.0135919084648676\\
88	0.0135919084636523\\
89	0.0135919084624165\\
90	0.0135919084611597\\
91	0.0135919084598817\\
92	0.013591908458582\\
93	0.0135919084572604\\
94	0.0135919084559164\\
95	0.0135919084545496\\
96	0.0135919084531597\\
97	0.0135919084517462\\
98	0.0135919084503088\\
99	0.013591908448847\\
100	0.0135919084473604\\
101	0.0135919084458487\\
102	0.0135919084443113\\
103	0.0135919084427478\\
104	0.0135919084411578\\
105	0.0135919084395408\\
106	0.0135919084378964\\
107	0.013591908436224\\
108	0.0135919084345233\\
109	0.0135919084327936\\
110	0.0135919084310346\\
111	0.0135919084292457\\
112	0.0135919084274263\\
113	0.0135919084255761\\
114	0.0135919084236943\\
115	0.0135919084217806\\
116	0.0135919084198342\\
117	0.0135919084178548\\
118	0.0135919084158416\\
119	0.0135919084137942\\
120	0.0135919084117118\\
121	0.013591908409594\\
122	0.0135919084074401\\
123	0.0135919084052494\\
124	0.0135919084030214\\
125	0.0135919084007553\\
126	0.0135919083984506\\
127	0.0135919083961066\\
128	0.0135919083937225\\
129	0.0135919083912976\\
130	0.0135919083888314\\
131	0.013591908386323\\
132	0.0135919083837717\\
133	0.0135919083811767\\
134	0.0135919083785374\\
135	0.0135919083758529\\
136	0.0135919083731224\\
137	0.0135919083703451\\
138	0.0135919083675203\\
139	0.0135919083646471\\
140	0.0135919083617247\\
141	0.0135919083587521\\
142	0.0135919083557286\\
143	0.0135919083526531\\
144	0.0135919083495249\\
145	0.013591908346343\\
146	0.0135919083431065\\
147	0.0135919083398144\\
148	0.0135919083364657\\
149	0.0135919083330595\\
150	0.0135919083295947\\
151	0.0135919083260704\\
152	0.0135919083224854\\
153	0.0135919083188388\\
154	0.0135919083151294\\
155	0.0135919083113562\\
156	0.013591908307518\\
157	0.0135919083036138\\
158	0.0135919082996423\\
159	0.0135919082956024\\
160	0.0135919082914929\\
161	0.0135919082873125\\
162	0.0135919082830601\\
163	0.0135919082787344\\
164	0.0135919082743341\\
165	0.0135919082698579\\
166	0.0135919082653046\\
167	0.0135919082606726\\
168	0.0135919082559608\\
169	0.0135919082511676\\
170	0.0135919082462918\\
171	0.0135919082413317\\
172	0.0135919082362861\\
173	0.0135919082311533\\
174	0.0135919082259319\\
175	0.0135919082206204\\
176	0.0135919082152171\\
177	0.0135919082097205\\
178	0.013591908204129\\
179	0.0135919081984408\\
180	0.0135919081926544\\
181	0.0135919081867681\\
182	0.01359190818078\\
183	0.0135919081746884\\
184	0.0135919081684916\\
185	0.0135919081621876\\
186	0.0135919081557747\\
187	0.013591908149251\\
188	0.0135919081426144\\
189	0.0135919081358632\\
190	0.0135919081289951\\
191	0.0135919081220084\\
192	0.0135919081149007\\
193	0.0135919081076702\\
194	0.0135919081003145\\
195	0.0135919080928317\\
196	0.0135919080852193\\
197	0.0135919080774752\\
198	0.0135919080695971\\
199	0.0135919080615827\\
200	0.0135919080534295\\
201	0.0135919080451352\\
202	0.0135919080366973\\
203	0.0135919080281132\\
204	0.0135919080193805\\
205	0.0135919080104966\\
206	0.0135919080014587\\
207	0.0135919079922643\\
208	0.0135919079829106\\
209	0.0135919079733947\\
210	0.0135919079637139\\
211	0.0135919079538653\\
212	0.013591907943846\\
213	0.0135919079336529\\
214	0.013591907923283\\
215	0.0135919079127333\\
216	0.0135919079020006\\
217	0.0135919078910816\\
218	0.0135919078799731\\
219	0.0135919078686719\\
220	0.0135919078571744\\
221	0.0135919078454774\\
222	0.0135919078335772\\
223	0.0135919078214703\\
224	0.013591907809153\\
225	0.0135919077966218\\
226	0.0135919077838728\\
227	0.0135919077709022\\
228	0.0135919077577061\\
229	0.0135919077442806\\
230	0.0135919077306216\\
231	0.0135919077167249\\
232	0.0135919077025865\\
233	0.0135919076882021\\
234	0.0135919076735673\\
235	0.0135919076586777\\
236	0.0135919076435288\\
237	0.0135919076281161\\
238	0.0135919076124349\\
239	0.0135919075964805\\
240	0.013591907580248\\
241	0.0135919075637325\\
242	0.0135919075469291\\
243	0.0135919075298326\\
244	0.0135919075124379\\
245	0.0135919074947397\\
246	0.0135919074767327\\
247	0.0135919074584113\\
248	0.01359190743977\\
249	0.0135919074208031\\
250	0.013591907401505\\
251	0.0135919073818696\\
252	0.013591907361891\\
253	0.0135919073415632\\
254	0.0135919073208799\\
255	0.0135919072998349\\
256	0.0135919072784217\\
257	0.0135919072566337\\
258	0.0135919072344644\\
259	0.013591907211907\\
260	0.0135919071889544\\
261	0.0135919071655998\\
262	0.013591907141836\\
263	0.0135919071176557\\
264	0.0135919070930515\\
265	0.0135919070680158\\
266	0.0135919070425409\\
267	0.0135919070166191\\
268	0.0135919069902424\\
269	0.0135919069634026\\
270	0.0135919069360915\\
271	0.0135919069083008\\
272	0.0135919068800217\\
273	0.0135919068512458\\
274	0.013591906821964\\
275	0.0135919067921673\\
276	0.0135919067618466\\
277	0.0135919067309925\\
278	0.0135919066995955\\
279	0.0135919066676459\\
280	0.0135919066351338\\
281	0.0135919066020491\\
282	0.0135919065683817\\
283	0.0135919065341211\\
284	0.0135919064992567\\
285	0.0135919064637778\\
286	0.0135919064276733\\
287	0.0135919063909321\\
288	0.0135919063535427\\
289	0.0135919063154937\\
290	0.0135919062767732\\
291	0.0135919062373693\\
292	0.0135919061972696\\
293	0.0135919061564619\\
294	0.0135919061149334\\
295	0.0135919060726712\\
296	0.0135919060296624\\
297	0.0135919059858934\\
298	0.0135919059413509\\
299	0.0135919058960209\\
300	0.0135919058498895\\
301	0.0135919058029423\\
302	0.0135919057551647\\
303	0.0135919057065421\\
304	0.0135919056570594\\
305	0.0135919056067012\\
306	0.013591905555452\\
307	0.0135919055032959\\
308	0.0135919054502168\\
309	0.0135919053961984\\
310	0.0135919053412239\\
311	0.0135919052852764\\
312	0.0135919052283387\\
313	0.0135919051703932\\
314	0.0135919051114221\\
315	0.0135919050514072\\
316	0.0135919049903302\\
317	0.0135919049281722\\
318	0.0135919048649143\\
319	0.013591904800537\\
320	0.0135919047350207\\
321	0.0135919046683454\\
322	0.0135919046004906\\
323	0.0135919045314357\\
324	0.0135919044611597\\
325	0.0135919043896413\\
326	0.0135919043168587\\
327	0.0135919042427899\\
328	0.0135919041674124\\
329	0.0135919040907035\\
330	0.0135919040126401\\
331	0.0135919039331986\\
332	0.0135919038523551\\
333	0.0135919037700855\\
334	0.013591903686365\\
335	0.0135919036011686\\
336	0.0135919035144709\\
337	0.013591903426246\\
338	0.0135919033364676\\
339	0.0135919032451092\\
340	0.0135919031521436\\
341	0.0135919030575433\\
342	0.0135919029612804\\
343	0.0135919028633264\\
344	0.0135919027636526\\
345	0.0135919026622295\\
346	0.0135919025590274\\
347	0.013591902454016\\
348	0.0135919023471646\\
349	0.0135919022384418\\
350	0.0135919021278159\\
351	0.0135919020152546\\
352	0.013591901900725\\
353	0.0135919017841936\\
354	0.0135919016656264\\
355	0.0135919015449888\\
356	0.0135919014222457\\
357	0.013591901297361\\
358	0.0135919011702984\\
359	0.0135919010410205\\
360	0.0135919009094895\\
361	0.0135919007756667\\
362	0.0135919006395127\\
363	0.0135919005009873\\
364	0.0135919003600493\\
365	0.013591900216657\\
366	0.0135919000707675\\
367	0.013591899922337\\
368	0.0135918997713208\\
369	0.0135918996176731\\
370	0.0135918994613471\\
371	0.0135918993022949\\
372	0.0135918991404671\\
373	0.0135918989758135\\
374	0.0135918988082823\\
375	0.0135918986378206\\
376	0.0135918984643738\\
377	0.0135918982878859\\
378	0.0135918981082995\\
379	0.0135918979255553\\
380	0.0135918977395924\\
381	0.0135918975503481\\
382	0.0135918973577577\\
383	0.0135918971617546\\
384	0.0135918969622701\\
385	0.0135918967592331\\
386	0.0135918965525703\\
387	0.013591896342206\\
388	0.0135918961280615\\
389	0.0135918959100558\\
390	0.0135918956881046\\
391	0.0135918954621207\\
392	0.0135918952320132\\
393	0.013591894997688\\
394	0.0135918947590467\\
395	0.013591894515987\\
396	0.0135918942684016\\
397	0.0135918940161786\\
398	0.0135918937591999\\
399	0.0135918934973415\\
400	0.0135918932304722\\
401	0.013591892958454\\
402	0.0135918926811415\\
403	0.0135918923983827\\
404	0.0135918921100159\\
405	0.0135918918158676\\
406	0.0135918915157543\\
407	0.0135918912094824\\
408	0.0135918908968471\\
409	0.0135918905776323\\
410	0.0135918902516104\\
411	0.0135918899185423\\
412	0.0135918895781768\\
413	0.0135918892302491\\
414	0.0135918888744765\\
415	0.0135918885105519\\
416	0.0135918881381426\\
417	0.0135918877569098\\
418	0.0135918873664904\\
419	0.0135918869664911\\
420	0.0135918865564846\\
421	0.0135918861360053\\
422	0.0135918857045439\\
423	0.0135918852615422\\
424	0.0135918848063852\\
425	0.0135918843383937\\
426	0.0135918838568133\\
427	0.0135918833608008\\
428	0.0135918828494039\\
429	0.0135918823215263\\
430	0.0135918817758588\\
431	0.0135918812107291\\
432	0.0135918806237562\\
433	0.0135918800110476\\
434	0.0135918793653702\\
435	0.0135918786721928\\
436	0.0135918779019084\\
437	0.0135918769973993\\
438	0.0135918758646862\\
439	0.0135918744110811\\
440	0.013591872788802\\
441	0.013591871129571\\
442	0.0135918694322569\\
443	0.013591867695688\\
444	0.013591865918653\\
445	0.0135918640999026\\
446	0.0135918622381519\\
447	0.0135918603320848\\
448	0.0135918583803587\\
449	0.0135918563816111\\
450	0.0135918543344665\\
451	0.013591852237538\\
452	0.0135918500894173\\
453	0.0135918478886263\\
454	0.0135918456334797\\
455	0.0135918433217444\\
456	0.0135918409498552\\
457	0.0135918385111969\\
458	0.0135918359923967\\
459	0.0135918333649876\\
460	0.0135918305642624\\
461	0.0135918274258205\\
462	0.0135918234734875\\
463	0.0135918174912898\\
464	0.0135918045174755\\
465	0.0135915740084851\\
466	0.0135913391041115\\
467	0.013591099669409\\
468	0.0135908555611643\\
469	0.0135906066273529\\
470	0.0135903527063131\\
471	0.0135900936268359\\
472	0.0135898292088706\\
473	0.013589559263522\\
474	0.0135892835930784\\
475	0.0135890019910042\\
476	0.0135887142422068\\
477	0.013588420125188\\
478	0.0135881194057676\\
479	0.013587811834604\\
480	0.0135874971458467\\
481	0.0135871750556311\\
482	0.0135868452603463\\
483	0.0135865074346414\\
484	0.0135861612291345\\
485	0.0135858062678069\\
486	0.0135854421450077\\
487	0.0135850684219721\\
488	0.0135846846229339\\
489	0.013584290230632\\
490	0.0135838846810895\\
491	0.0135834673575133\\
492	0.013583037583117\\
493	0.0135825946125993\\
494	0.013582137621883\\
495	0.0135816656954558\\
496	0.0135811778100659\\
497	0.013580672812118\\
498	0.0135801493827262\\
499	0.0135796059762644\\
500	0.0135790406994847\\
501	0.0135784510576631\\
502	0.0135778334172429\\
503	0.0135771819374208\\
504	0.0135764867838359\\
505	0.0135753188135231\\
506	0.0135736371624971\\
507	0.013571917310551\\
508	0.0135701573334925\\
509	0.0135683551256225\\
510	0.0135665083422997\\
511	0.0135646145612687\\
512	0.0135626712709945\\
513	0.0135606757178118\\
514	0.013558624876504\\
515	0.0135565154165732\\
516	0.0135543436626437\\
517	0.0135521055452458\\
518	0.0135497965315038\\
519	0.0135474115039713\\
520	0.0135449444864745\\
521	0.0135423878839583\\
522	0.0135397301108489\\
523	0.0135369477205464\\
524	0.0135331290459678\\
525	0.0135269367067296\\
526	0.0135205313336136\\
527	0.0135138937138777\\
528	0.0135069985011208\\
529	0.0134998096724892\\
530	0.0134913259654143\\
531	0.0134809596289797\\
532	0.013470240349157\\
533	0.0134591408298378\\
534	0.0134476274755564\\
535	0.0134356546878034\\
536	0.013423130212144\\
537	0.0134068337452876\\
538	0.0133857363659662\\
539	0.0133640067904496\\
540	0.0133415975166393\\
541	0.0133184549341791\\
542	0.0132945181338923\\
543	0.0132697195946383\\
544	0.0132439887419286\\
545	0.0132172412743684\\
546	0.0131893720312445\\
547	0.0131602259087115\\
548	0.0131296221170271\\
549	0.0130963173310625\\
550	0.0130457326939099\\
551	0.0129871618734068\\
552	0.012924988837248\\
553	0.0128606163998723\\
554	0.01279391552148\\
555	0.0127247750666354\\
556	0.0126590448696326\\
557	0.0125915651237542\\
558	0.0125212588405926\\
559	0.0124479598293873\\
560	0.0123818455907285\\
561	0.0123161653451569\\
562	0.0122272630202049\\
563	0.0121118495694013\\
564	0.0119919511824275\\
565	0.0118673938931047\\
566	0.011764611763121\\
567	0.0116578802124658\\
568	0.0115464435767891\\
569	0.0114294267343144\\
570	0.0112859427377856\\
571	0.0111362830903561\\
572	0.0109797626709145\\
573	0.0108586765742158\\
574	0.0107547619357607\\
575	0.0106388887178325\\
576	0.0104222260330238\\
577	0.0101998222515477\\
578	0.0099713298658148\\
579	0.00973555084991577\\
580	0.00949143485806943\\
581	0.00923867627785478\\
582	0.00907524279500573\\
583	0.0089436314279156\\
584	0.00881001385781117\\
585	0.00867625419638909\\
586	0.00854558854619178\\
587	0.00837875618173115\\
588	0.00821243744553536\\
589	0.00803422941175562\\
590	0.00755482615660152\\
591	0.00706957814576492\\
592	0.00658807525000149\\
593	0.00642698023275806\\
594	0.00625725512634445\\
595	0.0060609403391104\\
596	0.00575058001197164\\
597	0.00512683753504545\\
598	0.00366374385960312\\
599	0\\
600	0\\
};
\addplot [color=mycolor11,solid,forget plot]
  table[row sep=crcr]{%
1	0.0135940453118077\\
2	0.0135940453117958\\
3	0.0135940453117837\\
4	0.0135940453117713\\
5	0.0135940453117588\\
6	0.0135940453117461\\
7	0.0135940453117331\\
8	0.0135940453117199\\
9	0.0135940453117065\\
10	0.0135940453116929\\
11	0.0135940453116791\\
12	0.013594045311665\\
13	0.0135940453116507\\
14	0.0135940453116361\\
15	0.0135940453116213\\
16	0.0135940453116063\\
17	0.013594045311591\\
18	0.0135940453115754\\
19	0.0135940453115596\\
20	0.0135940453115435\\
21	0.0135940453115271\\
22	0.0135940453115105\\
23	0.0135940453114936\\
24	0.0135940453114764\\
25	0.0135940453114589\\
26	0.0135940453114411\\
27	0.013594045311423\\
28	0.0135940453114046\\
29	0.0135940453113859\\
30	0.0135940453113669\\
31	0.0135940453113476\\
32	0.0135940453113279\\
33	0.0135940453113079\\
34	0.0135940453112876\\
35	0.0135940453112669\\
36	0.0135940453112459\\
37	0.0135940453112246\\
38	0.0135940453112028\\
39	0.0135940453111807\\
40	0.0135940453111583\\
41	0.0135940453111354\\
42	0.0135940453111122\\
43	0.0135940453110886\\
44	0.0135940453110646\\
45	0.0135940453110402\\
46	0.0135940453110153\\
47	0.0135940453109901\\
48	0.0135940453109644\\
49	0.0135940453109383\\
50	0.0135940453109118\\
51	0.0135940453108848\\
52	0.0135940453108573\\
53	0.0135940453108294\\
54	0.013594045310801\\
55	0.0135940453107722\\
56	0.0135940453107428\\
57	0.013594045310713\\
58	0.0135940453106827\\
59	0.0135940453106518\\
60	0.0135940453106204\\
61	0.0135940453105885\\
62	0.0135940453105561\\
63	0.0135940453105231\\
64	0.0135940453104896\\
65	0.0135940453104555\\
66	0.0135940453104208\\
67	0.0135940453103855\\
68	0.0135940453103497\\
69	0.0135940453103132\\
70	0.0135940453102761\\
71	0.0135940453102384\\
72	0.0135940453102001\\
73	0.0135940453101611\\
74	0.0135940453101214\\
75	0.0135940453100811\\
76	0.0135940453100401\\
77	0.0135940453099984\\
78	0.013594045309956\\
79	0.0135940453099129\\
80	0.0135940453098691\\
81	0.0135940453098245\\
82	0.0135940453097792\\
83	0.0135940453097331\\
84	0.0135940453096862\\
85	0.0135940453096385\\
86	0.01359404530959\\
87	0.0135940453095408\\
88	0.0135940453094906\\
89	0.0135940453094396\\
90	0.0135940453093878\\
91	0.0135940453093351\\
92	0.0135940453092815\\
93	0.013594045309227\\
94	0.0135940453091715\\
95	0.0135940453091151\\
96	0.0135940453090578\\
97	0.0135940453089995\\
98	0.0135940453089402\\
99	0.0135940453088799\\
100	0.0135940453088186\\
101	0.0135940453087562\\
102	0.0135940453086928\\
103	0.0135940453086283\\
104	0.0135940453085627\\
105	0.013594045308496\\
106	0.0135940453084282\\
107	0.0135940453083592\\
108	0.013594045308289\\
109	0.0135940453082177\\
110	0.0135940453081451\\
111	0.0135940453080713\\
112	0.0135940453079962\\
113	0.0135940453079199\\
114	0.0135940453078423\\
115	0.0135940453077633\\
116	0.013594045307683\\
117	0.0135940453076014\\
118	0.0135940453075183\\
119	0.0135940453074338\\
120	0.0135940453073479\\
121	0.0135940453072606\\
122	0.0135940453071717\\
123	0.0135940453070813\\
124	0.0135940453069894\\
125	0.0135940453068959\\
126	0.0135940453068008\\
127	0.0135940453067041\\
128	0.0135940453066057\\
129	0.0135940453065057\\
130	0.0135940453064039\\
131	0.0135940453063004\\
132	0.0135940453061952\\
133	0.0135940453060881\\
134	0.0135940453059792\\
135	0.0135940453058684\\
136	0.0135940453057558\\
137	0.0135940453056412\\
138	0.0135940453055246\\
139	0.0135940453054061\\
140	0.0135940453052855\\
141	0.0135940453051628\\
142	0.0135940453050381\\
143	0.0135940453049112\\
144	0.0135940453047821\\
145	0.0135940453046508\\
146	0.0135940453045172\\
147	0.0135940453043814\\
148	0.0135940453042432\\
149	0.0135940453041026\\
150	0.0135940453039597\\
151	0.0135940453038142\\
152	0.0135940453036663\\
153	0.0135940453035158\\
154	0.0135940453033627\\
155	0.013594045303207\\
156	0.0135940453030486\\
157	0.0135940453028875\\
158	0.0135940453027236\\
159	0.0135940453025569\\
160	0.0135940453023873\\
161	0.0135940453022148\\
162	0.0135940453020393\\
163	0.0135940453018608\\
164	0.0135940453016792\\
165	0.0135940453014945\\
166	0.0135940453013066\\
167	0.0135940453011154\\
168	0.013594045300921\\
169	0.0135940453007232\\
170	0.013594045300522\\
171	0.0135940453003173\\
172	0.013594045300109\\
173	0.0135940452998972\\
174	0.0135940452996817\\
175	0.0135940452994625\\
176	0.0135940452992395\\
177	0.0135940452990127\\
178	0.0135940452987819\\
179	0.0135940452985472\\
180	0.0135940452983084\\
181	0.0135940452980655\\
182	0.0135940452978183\\
183	0.0135940452975669\\
184	0.0135940452973112\\
185	0.013594045297051\\
186	0.0135940452967864\\
187	0.0135940452965171\\
188	0.0135940452962432\\
189	0.0135940452959646\\
190	0.0135940452956812\\
191	0.0135940452953928\\
192	0.0135940452950995\\
193	0.0135940452948011\\
194	0.0135940452944975\\
195	0.0135940452941887\\
196	0.0135940452938745\\
197	0.0135940452935549\\
198	0.0135940452932298\\
199	0.013594045292899\\
200	0.0135940452925625\\
201	0.0135940452922202\\
202	0.013594045291872\\
203	0.0135940452915177\\
204	0.0135940452911573\\
205	0.0135940452907906\\
206	0.0135940452904176\\
207	0.0135940452900382\\
208	0.0135940452896521\\
209	0.0135940452892594\\
210	0.0135940452888598\\
211	0.0135940452884534\\
212	0.0135940452880398\\
213	0.0135940452876191\\
214	0.0135940452871912\\
215	0.0135940452867557\\
216	0.0135940452863128\\
217	0.0135940452858621\\
218	0.0135940452854036\\
219	0.0135940452849372\\
220	0.0135940452844627\\
221	0.0135940452839799\\
222	0.0135940452834887\\
223	0.013594045282989\\
224	0.0135940452824806\\
225	0.0135940452819634\\
226	0.0135940452814372\\
227	0.0135940452809019\\
228	0.0135940452803572\\
229	0.0135940452798031\\
230	0.0135940452792393\\
231	0.0135940452786657\\
232	0.0135940452780822\\
233	0.0135940452774884\\
234	0.0135940452768844\\
235	0.0135940452762698\\
236	0.0135940452756445\\
237	0.0135940452750084\\
238	0.0135940452743611\\
239	0.0135940452737026\\
240	0.0135940452730326\\
241	0.0135940452723509\\
242	0.0135940452716573\\
243	0.0135940452709516\\
244	0.0135940452702336\\
245	0.0135940452695031\\
246	0.0135940452687598\\
247	0.0135940452680035\\
248	0.0135940452672341\\
249	0.0135940452664511\\
250	0.0135940452656546\\
251	0.013594045264844\\
252	0.0135940452640194\\
253	0.0135940452631803\\
254	0.0135940452623265\\
255	0.0135940452614578\\
256	0.0135940452605739\\
257	0.0135940452596745\\
258	0.0135940452587593\\
259	0.0135940452578282\\
260	0.0135940452568807\\
261	0.0135940452559166\\
262	0.0135940452549357\\
263	0.0135940452539375\\
264	0.0135940452529218\\
265	0.0135940452518883\\
266	0.0135940452508367\\
267	0.0135940452497666\\
268	0.0135940452486778\\
269	0.0135940452475698\\
270	0.0135940452464424\\
271	0.0135940452452951\\
272	0.0135940452441277\\
273	0.0135940452429398\\
274	0.013594045241731\\
275	0.0135940452405009\\
276	0.0135940452392492\\
277	0.0135940452379755\\
278	0.0135940452366793\\
279	0.0135940452353604\\
280	0.0135940452340182\\
281	0.0135940452326524\\
282	0.0135940452312625\\
283	0.0135940452298481\\
284	0.0135940452284088\\
285	0.0135940452269441\\
286	0.0135940452254536\\
287	0.0135940452239368\\
288	0.0135940452223932\\
289	0.0135940452208224\\
290	0.0135940452192239\\
291	0.0135940452175972\\
292	0.0135940452159417\\
293	0.013594045214257\\
294	0.0135940452125426\\
295	0.0135940452107978\\
296	0.0135940452090223\\
297	0.0135940452072153\\
298	0.0135940452053764\\
299	0.0135940452035051\\
300	0.0135940452016006\\
301	0.0135940451996625\\
302	0.01359404519769\\
303	0.0135940451956827\\
304	0.0135940451936399\\
305	0.013594045191561\\
306	0.0135940451894453\\
307	0.0135940451872921\\
308	0.0135940451851009\\
309	0.0135940451828709\\
310	0.0135940451806014\\
311	0.0135940451782918\\
312	0.0135940451759414\\
313	0.0135940451735493\\
314	0.0135940451711149\\
315	0.0135940451686375\\
316	0.0135940451661162\\
317	0.0135940451635503\\
318	0.013594045160939\\
319	0.0135940451582816\\
320	0.0135940451555772\\
321	0.0135940451528249\\
322	0.013594045150024\\
323	0.0135940451471736\\
324	0.0135940451442729\\
325	0.0135940451413209\\
326	0.0135940451383167\\
327	0.0135940451352595\\
328	0.0135940451321484\\
329	0.0135940451289823\\
330	0.0135940451257604\\
331	0.0135940451224817\\
332	0.0135940451191452\\
333	0.0135940451157498\\
334	0.0135940451122947\\
335	0.0135940451087787\\
336	0.0135940451052008\\
337	0.01359404510156\\
338	0.0135940450978552\\
339	0.0135940450940852\\
340	0.013594045090249\\
341	0.0135940450863455\\
342	0.0135940450823734\\
343	0.0135940450783316\\
344	0.013594045074219\\
345	0.0135940450700343\\
346	0.0135940450657763\\
347	0.0135940450614438\\
348	0.0135940450570355\\
349	0.01359404505255\\
350	0.0135940450479862\\
351	0.0135940450433426\\
352	0.013594045038618\\
353	0.0135940450338109\\
354	0.01359404502892\\
355	0.0135940450239437\\
356	0.0135940450188808\\
357	0.0135940450137296\\
358	0.0135940450084887\\
359	0.0135940450031565\\
360	0.0135940449977316\\
361	0.0135940449922122\\
362	0.0135940449865968\\
363	0.0135940449808836\\
364	0.0135940449750711\\
365	0.0135940449691574\\
366	0.0135940449631408\\
367	0.0135940449570195\\
368	0.0135940449507916\\
369	0.0135940449444553\\
370	0.0135940449380084\\
371	0.0135940449314492\\
372	0.0135940449247755\\
373	0.0135940449179853\\
374	0.0135940449110763\\
375	0.0135940449040464\\
376	0.0135940448968933\\
377	0.0135940448896146\\
378	0.013594044882208\\
379	0.013594044874671\\
380	0.013594044867001\\
381	0.0135940448591953\\
382	0.0135940448512514\\
383	0.0135940448431664\\
384	0.0135940448349373\\
385	0.0135940448265613\\
386	0.0135940448180352\\
387	0.0135940448093558\\
388	0.0135940448005198\\
389	0.0135940447915238\\
390	0.0135940447823642\\
391	0.0135940447730373\\
392	0.0135940447635393\\
393	0.0135940447538661\\
394	0.0135940447440135\\
395	0.0135940447339772\\
396	0.0135940447237525\\
397	0.0135940447133346\\
398	0.0135940447027184\\
399	0.0135940446918985\\
400	0.0135940446808691\\
401	0.0135940446696245\\
402	0.0135940446581582\\
403	0.0135940446464636\\
404	0.0135940446345335\\
405	0.0135940446223604\\
406	0.0135940446099363\\
407	0.0135940445972529\\
408	0.0135940445843011\\
409	0.0135940445710718\\
410	0.013594044557555\\
411	0.0135940445437405\\
412	0.0135940445296173\\
413	0.0135940445151736\\
414	0.0135940445003966\\
415	0.0135940444852728\\
416	0.0135940444697881\\
417	0.0135940444539272\\
418	0.0135940444376739\\
419	0.01359404442101\\
420	0.0135940444039162\\
421	0.013594044386371\\
422	0.0135940443683509\\
423	0.0135940443498298\\
424	0.013594044330779\\
425	0.0135940443111661\\
426	0.0135940442909547\\
427	0.0135940442701025\\
428	0.0135940442485587\\
429	0.0135940442262581\\
430	0.0135940442031082\\
431	0.0135940441789624\\
432	0.0135940441535643\\
433	0.0135940441264396\\
434	0.0135940440967092\\
435	0.0135940440628293\\
436	0.013594044022417\\
437	0.0135940439727302\\
438	0.0135940439129532\\
439	0.0135940438482276\\
440	0.0135940437820306\\
441	0.0135940437143174\\
442	0.0135940436450416\\
443	0.0135940435741556\\
444	0.0135940435016101\\
445	0.0135940434273548\\
446	0.0135940433513381\\
447	0.0135940432735077\\
448	0.0135940431938102\\
449	0.0135940431121919\\
450	0.0135940430285971\\
451	0.0135940429429669\\
452	0.0135940428552326\\
453	0.0135940427653014\\
454	0.013594042673022\\
455	0.0135940425781079\\
456	0.0135940424799621\\
457	0.0135940423772787\\
458	0.013594042267121\\
459	0.0135940421427694\\
460	0.0135940419888291\\
461	0.0135940417713454\\
462	0.0135940414246864\\
463	0.0135940408468957\\
464	0.0135940399671601\\
465	0.0135940390712872\\
466	0.0135940381587915\\
467	0.0135940372291632\\
468	0.0135940362818697\\
469	0.0135940353163476\\
470	0.0135940343320011\\
471	0.01359403332821\\
472	0.0135940323043265\\
473	0.0135940312596649\\
474	0.0135940301934835\\
475	0.0135940291049847\\
476	0.0135940279933754\\
477	0.0135940268578239\\
478	0.0135940256974439\\
479	0.0135940245112887\\
480	0.0135940232983459\\
481	0.0135940220575301\\
482	0.013594020787675\\
483	0.0135940194875245\\
484	0.013594018155723\\
485	0.0135940167908042\\
486	0.0135940153911767\\
487	0.0135940139551084\\
488	0.0135940124807073\\
489	0.0135940109659\\
490	0.0135940094084036\\
491	0.0135940078056898\\
492	0.0135940061549331\\
493	0.0135940044529279\\
494	0.0135940026959357\\
495	0.0135940008793747\\
496	0.0135939989971485\\
497	0.0135939970401605\\
498	0.0135939949930508\\
499	0.0135939928272663\\
500	0.0135939904872408\\
501	0.0135939878657466\\
502	0.0135939847690407\\
503	0.01359398089622\\
504	0.013593975930866\\
505	0.0135939702662746\\
506	0.013593964498087\\
507	0.0135939586222699\\
508	0.0135939526333629\\
509	0.0135939465230392\\
510	0.013593940283437\\
511	0.0135939339079869\\
512	0.013593927389427\\
513	0.0135939207196255\\
514	0.0135939138892086\\
515	0.0135939068866672\\
516	0.0135938996960378\\
517	0.013593892290677\\
518	0.0135938846163582\\
519	0.0135938765454359\\
520	0.0135938677539973\\
521	0.0135938574003501\\
522	0.0135938433194432\\
523	0.0135938201658801\\
524	0.0135935886646046\\
525	0.0135928584855939\\
526	0.0135921057763526\\
527	0.0135913287160958\\
528	0.0135905250633436\\
529	0.01358969224065\\
530	0.0135888284581564\\
531	0.0135879321713535\\
532	0.0135869997092166\\
533	0.0135860263605028\\
534	0.0135850054939404\\
535	0.0135839264419661\\
536	0.013582770721826\\
537	0.0135808328121808\\
538	0.0135778902427088\\
539	0.013574859417277\\
540	0.0135717334431941\\
541	0.0135685045635268\\
542	0.0135651640488818\\
543	0.0135617020033984\\
544	0.0135581068285905\\
545	0.013554364611253\\
546	0.0135504579475602\\
547	0.0135463642743097\\
548	0.013542049651676\\
549	0.0135372492329162\\
550	0.0135287751582943\\
551	0.0135200154047505\\
552	0.0135109460369536\\
553	0.0135015375256678\\
554	0.0134917495954427\\
555	0.0134814949641083\\
556	0.0134657542586436\\
557	0.0134485344487215\\
558	0.0134304383644168\\
559	0.0134112877355424\\
560	0.0133826077339614\\
561	0.0133495839111871\\
562	0.0133107027875717\\
563	0.0132651074942661\\
564	0.0132174550510279\\
565	0.0131673829773579\\
566	0.0130935034775975\\
567	0.0130163853987856\\
568	0.0129361503549666\\
569	0.0128524923811328\\
570	0.0127649775367202\\
571	0.0126730522243384\\
572	0.0125759874486968\\
573	0.0124399283302884\\
574	0.0122820145229851\\
575	0.0121151381544788\\
576	0.0119201782367025\\
577	0.0117158816622068\\
578	0.0115013017629722\\
579	0.0113047644849782\\
580	0.0111157007364945\\
581	0.0109411421379581\\
582	0.0106781870066112\\
583	0.0103761124222753\\
584	0.0101250412749154\\
585	0.00986943912818245\\
586	0.00960283510563681\\
587	0.00932247274674759\\
588	0.00902904009979084\\
589	0.00884170618614843\\
590	0.00861050494941614\\
591	0.00837652661868457\\
592	0.00812230596562506\\
593	0.00756248410970657\\
594	0.00695994490712135\\
595	0.00626582126936309\\
596	0.00583905595420081\\
597	0.00512683753504545\\
598	0.00366374385960312\\
599	0\\
600	0\\
};
\addplot [color=mycolor12,solid,forget plot]
  table[row sep=crcr]{%
1	0.0136268878279312\\
2	0.0136268878279307\\
3	0.0136268878279302\\
4	0.0136268878279297\\
5	0.0136268878279292\\
6	0.0136268878279286\\
7	0.0136268878279281\\
8	0.0136268878279275\\
9	0.0136268878279269\\
10	0.0136268878279264\\
11	0.0136268878279258\\
12	0.0136268878279252\\
13	0.0136268878279246\\
14	0.013626887827924\\
15	0.0136268878279234\\
16	0.0136268878279227\\
17	0.0136268878279221\\
18	0.0136268878279214\\
19	0.0136268878279208\\
20	0.0136268878279201\\
21	0.0136268878279194\\
22	0.0136268878279187\\
23	0.013626887827918\\
24	0.0136268878279172\\
25	0.0136268878279165\\
26	0.0136268878279158\\
27	0.013626887827915\\
28	0.0136268878279142\\
29	0.0136268878279134\\
30	0.0136268878279126\\
31	0.0136268878279118\\
32	0.013626887827911\\
33	0.0136268878279101\\
34	0.0136268878279093\\
35	0.0136268878279084\\
36	0.0136268878279075\\
37	0.0136268878279066\\
38	0.0136268878279057\\
39	0.0136268878279048\\
40	0.0136268878279038\\
41	0.0136268878279029\\
42	0.0136268878279019\\
43	0.0136268878279009\\
44	0.0136268878278999\\
45	0.0136268878278989\\
46	0.0136268878278978\\
47	0.0136268878278967\\
48	0.0136268878278957\\
49	0.0136268878278946\\
50	0.0136268878278934\\
51	0.0136268878278923\\
52	0.0136268878278911\\
53	0.01362688782789\\
54	0.0136268878278888\\
55	0.0136268878278876\\
56	0.0136268878278863\\
57	0.0136268878278851\\
58	0.0136268878278838\\
59	0.0136268878278825\\
60	0.0136268878278812\\
61	0.0136268878278798\\
62	0.0136268878278784\\
63	0.013626887827877\\
64	0.0136268878278756\\
65	0.0136268878278742\\
66	0.0136268878278727\\
67	0.0136268878278712\\
68	0.0136268878278697\\
69	0.0136268878278682\\
70	0.0136268878278666\\
71	0.013626887827865\\
72	0.0136268878278634\\
73	0.0136268878278618\\
74	0.0136268878278601\\
75	0.0136268878278584\\
76	0.0136268878278567\\
77	0.0136268878278549\\
78	0.0136268878278531\\
79	0.0136268878278513\\
80	0.0136268878278495\\
81	0.0136268878278476\\
82	0.0136268878278457\\
83	0.0136268878278437\\
84	0.0136268878278418\\
85	0.0136268878278397\\
86	0.0136268878278377\\
87	0.0136268878278356\\
88	0.0136268878278335\\
89	0.0136268878278314\\
90	0.0136268878278292\\
91	0.013626887827827\\
92	0.0136268878278247\\
93	0.0136268878278224\\
94	0.0136268878278201\\
95	0.0136268878278177\\
96	0.0136268878278153\\
97	0.0136268878278128\\
98	0.0136268878278103\\
99	0.0136268878278078\\
100	0.0136268878278052\\
101	0.0136268878278025\\
102	0.0136268878277999\\
103	0.0136268878277971\\
104	0.0136268878277944\\
105	0.0136268878277916\\
106	0.0136268878277887\\
107	0.0136268878277858\\
108	0.0136268878277828\\
109	0.0136268878277798\\
110	0.0136268878277768\\
111	0.0136268878277737\\
112	0.0136268878277705\\
113	0.0136268878277673\\
114	0.013626887827764\\
115	0.0136268878277607\\
116	0.0136268878277573\\
117	0.0136268878277538\\
118	0.0136268878277503\\
119	0.0136268878277468\\
120	0.0136268878277431\\
121	0.0136268878277395\\
122	0.0136268878277357\\
123	0.0136268878277319\\
124	0.013626887827728\\
125	0.0136268878277241\\
126	0.0136268878277201\\
127	0.013626887827716\\
128	0.0136268878277118\\
129	0.0136268878277076\\
130	0.0136268878277033\\
131	0.013626887827699\\
132	0.0136268878276945\\
133	0.01362688782769\\
134	0.0136268878276854\\
135	0.0136268878276807\\
136	0.013626887827676\\
137	0.0136268878276712\\
138	0.0136268878276662\\
139	0.0136268878276612\\
140	0.0136268878276561\\
141	0.013626887827651\\
142	0.0136268878276457\\
143	0.0136268878276404\\
144	0.0136268878276349\\
145	0.0136268878276294\\
146	0.0136268878276237\\
147	0.013626887827618\\
148	0.0136268878276122\\
149	0.0136268878276062\\
150	0.0136268878276002\\
151	0.0136268878275941\\
152	0.0136268878275878\\
153	0.0136268878275815\\
154	0.013626887827575\\
155	0.0136268878275685\\
156	0.0136268878275618\\
157	0.013626887827555\\
158	0.0136268878275481\\
159	0.013626887827541\\
160	0.0136268878275339\\
161	0.0136268878275266\\
162	0.0136268878275192\\
163	0.0136268878275117\\
164	0.013626887827504\\
165	0.0136268878274962\\
166	0.0136268878274883\\
167	0.0136268878274802\\
168	0.013626887827472\\
169	0.0136268878274637\\
170	0.0136268878274552\\
171	0.0136268878274465\\
172	0.0136268878274378\\
173	0.0136268878274288\\
174	0.0136268878274197\\
175	0.0136268878274105\\
176	0.0136268878274011\\
177	0.0136268878273915\\
178	0.0136268878273818\\
179	0.0136268878273719\\
180	0.0136268878273618\\
181	0.0136268878273515\\
182	0.0136268878273411\\
183	0.0136268878273305\\
184	0.0136268878273197\\
185	0.0136268878273087\\
186	0.0136268878272976\\
187	0.0136268878272862\\
188	0.0136268878272747\\
189	0.0136268878272629\\
190	0.0136268878272509\\
191	0.0136268878272388\\
192	0.0136268878272264\\
193	0.0136268878272138\\
194	0.013626887827201\\
195	0.013626887827188\\
196	0.0136268878271747\\
197	0.0136268878271612\\
198	0.0136268878271475\\
199	0.0136268878271335\\
200	0.0136268878271194\\
201	0.0136268878271049\\
202	0.0136268878270902\\
203	0.0136268878270753\\
204	0.0136268878270601\\
205	0.0136268878270446\\
206	0.0136268878270288\\
207	0.0136268878270128\\
208	0.0136268878269965\\
209	0.01362688782698\\
210	0.0136268878269631\\
211	0.013626887826946\\
212	0.0136268878269285\\
213	0.0136268878269108\\
214	0.0136268878268927\\
215	0.0136268878268743\\
216	0.0136268878268556\\
217	0.0136268878268366\\
218	0.0136268878268173\\
219	0.0136268878267976\\
220	0.0136268878267776\\
221	0.0136268878267572\\
222	0.0136268878267365\\
223	0.0136268878267154\\
224	0.0136268878266939\\
225	0.0136268878266721\\
226	0.0136268878266499\\
227	0.0136268878266273\\
228	0.0136268878266043\\
229	0.013626887826581\\
230	0.0136268878265572\\
231	0.013626887826533\\
232	0.0136268878265083\\
233	0.0136268878264833\\
234	0.0136268878264578\\
235	0.0136268878264319\\
236	0.0136268878264055\\
237	0.0136268878263786\\
238	0.0136268878263513\\
239	0.0136268878263235\\
240	0.0136268878262952\\
241	0.0136268878262665\\
242	0.0136268878262372\\
243	0.0136268878262074\\
244	0.0136268878261771\\
245	0.0136268878261463\\
246	0.0136268878261149\\
247	0.013626887826083\\
248	0.0136268878260505\\
249	0.0136268878260175\\
250	0.0136268878259839\\
251	0.0136268878259497\\
252	0.0136268878259149\\
253	0.0136268878258795\\
254	0.0136268878258434\\
255	0.0136268878258068\\
256	0.0136268878257695\\
257	0.0136268878257315\\
258	0.0136268878256929\\
259	0.0136268878256536\\
260	0.0136268878256136\\
261	0.0136268878255729\\
262	0.0136268878255315\\
263	0.0136268878254894\\
264	0.0136268878254465\\
265	0.0136268878254029\\
266	0.0136268878253585\\
267	0.0136268878253133\\
268	0.0136268878252674\\
269	0.0136268878252206\\
270	0.013626887825173\\
271	0.0136268878251246\\
272	0.0136268878250754\\
273	0.0136268878250252\\
274	0.0136268878249742\\
275	0.0136268878249223\\
276	0.0136268878248694\\
277	0.0136268878248157\\
278	0.013626887824761\\
279	0.0136268878247053\\
280	0.0136268878246487\\
281	0.013626887824591\\
282	0.0136268878245323\\
283	0.0136268878244726\\
284	0.0136268878244119\\
285	0.0136268878243501\\
286	0.0136268878242872\\
287	0.0136268878242231\\
288	0.013626887824158\\
289	0.0136268878240917\\
290	0.0136268878240242\\
291	0.0136268878239555\\
292	0.0136268878238857\\
293	0.0136268878238146\\
294	0.0136268878237422\\
295	0.0136268878236685\\
296	0.0136268878235936\\
297	0.0136268878235173\\
298	0.0136268878234397\\
299	0.0136268878233607\\
300	0.0136268878232803\\
301	0.0136268878231985\\
302	0.0136268878231153\\
303	0.0136268878230305\\
304	0.0136268878229443\\
305	0.0136268878228566\\
306	0.0136268878227673\\
307	0.0136268878226764\\
308	0.0136268878225839\\
309	0.0136268878224898\\
310	0.013626887822394\\
311	0.0136268878222965\\
312	0.0136268878221973\\
313	0.0136268878220963\\
314	0.0136268878219936\\
315	0.013626887821889\\
316	0.0136268878217826\\
317	0.0136268878216743\\
318	0.0136268878215641\\
319	0.0136268878214519\\
320	0.0136268878213378\\
321	0.0136268878212216\\
322	0.0136268878211034\\
323	0.0136268878209831\\
324	0.0136268878208607\\
325	0.0136268878207361\\
326	0.0136268878206093\\
327	0.0136268878204803\\
328	0.013626887820349\\
329	0.0136268878202154\\
330	0.0136268878200794\\
331	0.0136268878199411\\
332	0.0136268878198003\\
333	0.013626887819657\\
334	0.0136268878195112\\
335	0.0136268878193628\\
336	0.0136268878192119\\
337	0.0136268878190583\\
338	0.0136268878189019\\
339	0.0136268878187429\\
340	0.013626887818581\\
341	0.0136268878184163\\
342	0.0136268878182487\\
343	0.0136268878180782\\
344	0.0136268878179047\\
345	0.0136268878177282\\
346	0.0136268878175485\\
347	0.0136268878173658\\
348	0.0136268878171798\\
349	0.0136268878169906\\
350	0.0136268878167981\\
351	0.0136268878166022\\
352	0.0136268878164029\\
353	0.0136268878162001\\
354	0.0136268878159938\\
355	0.013626887815784\\
356	0.0136268878155704\\
357	0.0136268878153532\\
358	0.0136268878151321\\
359	0.0136268878149072\\
360	0.0136268878146784\\
361	0.0136268878144457\\
362	0.0136268878142088\\
363	0.0136268878139679\\
364	0.0136268878137228\\
365	0.0136268878134734\\
366	0.0136268878132196\\
367	0.0136268878129615\\
368	0.0136268878126988\\
369	0.0136268878124316\\
370	0.0136268878121597\\
371	0.0136268878118831\\
372	0.0136268878116017\\
373	0.0136268878113153\\
374	0.0136268878110239\\
375	0.0136268878107274\\
376	0.0136268878104257\\
377	0.0136268878101187\\
378	0.0136268878098063\\
379	0.0136268878094884\\
380	0.0136268878091649\\
381	0.0136268878088356\\
382	0.0136268878085005\\
383	0.0136268878081594\\
384	0.0136268878078122\\
385	0.0136268878074587\\
386	0.0136268878070989\\
387	0.0136268878067327\\
388	0.0136268878063597\\
389	0.01362688780598\\
390	0.0136268878055934\\
391	0.0136268878051996\\
392	0.0136268878047985\\
393	0.01362688780439\\
394	0.0136268878039739\\
395	0.0136268878035499\\
396	0.0136268878031179\\
397	0.0136268878026777\\
398	0.0136268878022289\\
399	0.0136268878017715\\
400	0.013626887801305\\
401	0.0136268878008294\\
402	0.0136268878003442\\
403	0.0136268877998492\\
404	0.0136268877993441\\
405	0.0136268877988285\\
406	0.0136268877983021\\
407	0.0136268877977644\\
408	0.0136268877972152\\
409	0.013626887796654\\
410	0.0136268877960803\\
411	0.0136268877954937\\
412	0.0136268877948937\\
413	0.0136268877942797\\
414	0.0136268877936513\\
415	0.0136268877930076\\
416	0.0136268877923483\\
417	0.0136268877916724\\
418	0.0136268877909792\\
419	0.013626887790268\\
420	0.0136268877895377\\
421	0.0136268877887874\\
422	0.0136268877880159\\
423	0.013626887787222\\
424	0.0136268877864041\\
425	0.0136268877855607\\
426	0.0136268877846897\\
427	0.0136268877837885\\
428	0.0136268877828534\\
429	0.0136268877818787\\
430	0.0136268877808552\\
431	0.0136268877797667\\
432	0.0136268877785851\\
433	0.0136268877772648\\
434	0.0136268877757375\\
435	0.0136268877739199\\
436	0.0136268877717504\\
437	0.0136268877692613\\
438	0.0136268877666197\\
439	0.0136268877639182\\
440	0.0136268877611548\\
441	0.0136268877583276\\
442	0.0136268877554348\\
443	0.0136268877524744\\
444	0.0136268877494442\\
445	0.0136268877463424\\
446	0.0136268877431666\\
447	0.0136268877399149\\
448	0.0136268877365848\\
449	0.0136268877331741\\
450	0.0136268877296796\\
451	0.0136268877260972\\
452	0.0136268877224199\\
453	0.0136268877186344\\
454	0.0136268877147134\\
455	0.0136268877105991\\
456	0.0136268877061689\\
457	0.0136268877011652\\
458	0.0136268876950624\\
459	0.0136268876868414\\
460	0.0136268876747063\\
461	0.0136268876560219\\
462	0.0136268876281971\\
463	0.0136268875918093\\
464	0.0136268875547535\\
465	0.0136268875170094\\
466	0.0136268874785559\\
467	0.0136268874393708\\
468	0.0136268873994307\\
469	0.0136268873587108\\
470	0.0136268873171855\\
471	0.0136268872748276\\
472	0.0136268872316083\\
473	0.0136268871874966\\
474	0.0136268871424598\\
475	0.0136268870964648\\
476	0.0136268870494767\\
477	0.0136268870014582\\
478	0.0136268869523698\\
479	0.0136268869021687\\
480	0.0136268868508092\\
481	0.0136268867982423\\
482	0.0136268867444148\\
483	0.0136268866892695\\
484	0.0136268866327443\\
485	0.0136268865747716\\
486	0.0136268865152779\\
487	0.0136268864541824\\
488	0.0136268863913966\\
489	0.013626886326822\\
490	0.0136268862603486\\
491	0.0136268861918511\\
492	0.0136268861211822\\
493	0.0136268860481596\\
494	0.0136268859725398\\
495	0.0136268858939634\\
496	0.0136268858118456\\
497	0.0136268857251639\\
498	0.0136268856320766\\
499	0.0136268855293147\\
500	0.0136268854114232\\
501	0.0136268852703876\\
502	0.0136268850972404\\
503	0.0136268848886566\\
504	0.0136268846601735\\
505	0.0136268844275599\\
506	0.0136268841906322\\
507	0.0136268839491408\\
508	0.0136268837027589\\
509	0.0136268834511831\\
510	0.0136268831941374\\
511	0.0136268829313053\\
512	0.013626882662304\\
513	0.0136268823866249\\
514	0.013626882103487\\
515	0.0136268818114839\\
516	0.0136268815077355\\
517	0.0136268811858971\\
518	0.013626880831645\\
519	0.013626880412956\\
520	0.0136268798608944\\
521	0.0136268790373815\\
522	0.0136268777013764\\
523	0.0136268755541293\\
524	0.0136268726781752\\
525	0.0136268697003445\\
526	0.013626866600694\\
527	0.0136268633550434\\
528	0.0136268599452971\\
529	0.0136268563819244\\
530	0.0136268526770398\\
531	0.0136268487878813\\
532	0.013626844625871\\
533	0.0136268400092747\\
534	0.0136268345890394\\
535	0.013626827800565\\
536	0.0136268190273637\\
537	0.0136268088632286\\
538	0.0136267984599363\\
539	0.0136267877978721\\
540	0.0136267768548274\\
541	0.01362676560487\\
542	0.0136267540121859\\
543	0.0136267420127902\\
544	0.0136267294854962\\
545	0.0136267161994809\\
546	0.0136267017334953\\
547	0.0136266853329548\\
548	0.0136266658993872\\
549	0.0136266428322623\\
550	0.0136266190480606\\
551	0.0136265941294955\\
552	0.0136265670640316\\
553	0.0136265352048811\\
554	0.0136264915555887\\
555	0.0136264192183494\\
556	0.0136252301555777\\
557	0.0136238102218841\\
558	0.0136222972782903\\
559	0.0136206564703397\\
560	0.013617039838498\\
561	0.0136126015173747\\
562	0.0136079768228116\\
563	0.0136031411721728\\
564	0.0135980533199211\\
565	0.0135926446344186\\
566	0.0135822278964534\\
567	0.0135713356376805\\
568	0.0135600086889985\\
569	0.0135481945730692\\
570	0.0135358286051002\\
571	0.0135228252962376\\
572	0.0135090622348228\\
573	0.0134871046514208\\
574	0.0134607346999303\\
575	0.0134331506381204\\
576	0.0134042215466356\\
577	0.0133735941471642\\
578	0.013340749615476\\
579	0.0132830563963814\\
580	0.0132124138014288\\
581	0.0131172554217482\\
582	0.0130009559658341\\
583	0.0128731876267521\\
584	0.0126887187729786\\
585	0.0124916626897068\\
586	0.0122889547607575\\
587	0.0120852960853785\\
588	0.0118698675681612\\
589	0.0115422726391937\\
590	0.0111928033866319\\
591	0.010825014474107\\
592	0.010434224581246\\
593	0.00998273020622688\\
594	0.00950216794138242\\
595	0.00896778970282203\\
596	0.00790389929454537\\
597	0.00638278975403702\\
598	0.00366374385960312\\
599	0\\
600	0\\
};
\addplot [color=mycolor13,solid,forget plot]
  table[row sep=crcr]{%
1	0.00612333152159056\\
2	0.00612333152160971\\
3	0.0061233315216292\\
4	0.00612333152164904\\
5	0.00612333152166924\\
6	0.0061233315216898\\
7	0.00612333152171073\\
8	0.00612333152173202\\
9	0.00612333152175371\\
10	0.00612333152177577\\
11	0.00612333152179824\\
12	0.0061233315218211\\
13	0.00612333152184437\\
14	0.00612333152186806\\
15	0.00612333152189217\\
16	0.00612333152191671\\
17	0.0061233315219417\\
18	0.00612333152196712\\
19	0.00612333152199301\\
20	0.00612333152201935\\
21	0.00612333152204617\\
22	0.00612333152207346\\
23	0.00612333152210125\\
24	0.00612333152212952\\
25	0.00612333152215831\\
26	0.00612333152218761\\
27	0.00612333152221743\\
28	0.00612333152224779\\
29	0.00612333152227868\\
30	0.00612333152231013\\
31	0.00612333152234215\\
32	0.00612333152237473\\
33	0.00612333152240789\\
34	0.00612333152244166\\
35	0.00612333152247602\\
36	0.00612333152251099\\
37	0.00612333152254659\\
38	0.00612333152258283\\
39	0.00612333152261971\\
40	0.00612333152265725\\
41	0.00612333152269546\\
42	0.00612333152273436\\
43	0.00612333152277395\\
44	0.00612333152281424\\
45	0.00612333152285526\\
46	0.00612333152289701\\
47	0.00612333152293951\\
48	0.00612333152298276\\
49	0.00612333152302679\\
50	0.0061233315230716\\
51	0.00612333152311721\\
52	0.00612333152316364\\
53	0.00612333152321089\\
54	0.00612333152325899\\
55	0.00612333152330795\\
56	0.00612333152335778\\
57	0.00612333152340851\\
58	0.00612333152346013\\
59	0.00612333152351268\\
60	0.00612333152356617\\
61	0.00612333152362062\\
62	0.00612333152367603\\
63	0.00612333152373244\\
64	0.00612333152378985\\
65	0.00612333152384828\\
66	0.00612333152390776\\
67	0.0061233315239683\\
68	0.00612333152402993\\
69	0.00612333152409265\\
70	0.00612333152415649\\
71	0.00612333152422147\\
72	0.00612333152428761\\
73	0.00612333152435493\\
74	0.00612333152442345\\
75	0.0061233315244932\\
76	0.00612333152456419\\
77	0.00612333152463645\\
78	0.00612333152471\\
79	0.00612333152478486\\
80	0.00612333152486105\\
81	0.0061233315249386\\
82	0.00612333152501754\\
83	0.00612333152509789\\
84	0.00612333152517968\\
85	0.00612333152526292\\
86	0.00612333152534764\\
87	0.00612333152543388\\
88	0.00612333152552166\\
89	0.006123331525611\\
90	0.00612333152570194\\
91	0.0061233315257945\\
92	0.00612333152588871\\
93	0.00612333152598461\\
94	0.00612333152608221\\
95	0.00612333152618156\\
96	0.00612333152628268\\
97	0.0061233315263856\\
98	0.00612333152649035\\
99	0.00612333152659698\\
100	0.00612333152670551\\
101	0.00612333152681598\\
102	0.00612333152692842\\
103	0.00612333152704286\\
104	0.00612333152715934\\
105	0.00612333152727791\\
106	0.00612333152739859\\
107	0.00612333152752141\\
108	0.00612333152764644\\
109	0.00612333152777369\\
110	0.00612333152790321\\
111	0.00612333152803505\\
112	0.00612333152816923\\
113	0.00612333152830581\\
114	0.00612333152844483\\
115	0.00612333152858632\\
116	0.00612333152873034\\
117	0.00612333152887693\\
118	0.00612333152902614\\
119	0.006123331529178\\
120	0.00612333152933258\\
121	0.00612333152948991\\
122	0.00612333152965005\\
123	0.00612333152981305\\
124	0.00612333152997896\\
125	0.00612333153014782\\
126	0.0061233315303197\\
127	0.00612333153049464\\
128	0.00612333153067271\\
129	0.00612333153085395\\
130	0.00612333153103843\\
131	0.0061233315312262\\
132	0.00612333153141732\\
133	0.00612333153161184\\
134	0.00612333153180984\\
135	0.00612333153201138\\
136	0.0061233315322165\\
137	0.00612333153242529\\
138	0.00612333153263781\\
139	0.00612333153285411\\
140	0.00612333153307428\\
141	0.00612333153329838\\
142	0.00612333153352647\\
143	0.00612333153375864\\
144	0.00612333153399495\\
145	0.00612333153423547\\
146	0.00612333153448029\\
147	0.00612333153472948\\
148	0.00612333153498312\\
149	0.00612333153524129\\
150	0.00612333153550406\\
151	0.00612333153577153\\
152	0.00612333153604377\\
153	0.00612333153632087\\
154	0.00612333153660292\\
155	0.00612333153689\\
156	0.00612333153718221\\
157	0.00612333153747964\\
158	0.00612333153778238\\
159	0.00612333153809053\\
160	0.00612333153840418\\
161	0.00612333153872344\\
162	0.00612333153904839\\
163	0.00612333153937916\\
164	0.00612333153971583\\
165	0.00612333154005851\\
166	0.00612333154040732\\
167	0.00612333154076236\\
168	0.00612333154112374\\
169	0.00612333154149159\\
170	0.00612333154186601\\
171	0.00612333154224712\\
172	0.00612333154263504\\
173	0.0061233315430299\\
174	0.00612333154343182\\
175	0.00612333154384092\\
176	0.00612333154425734\\
177	0.00612333154468121\\
178	0.00612333154511266\\
179	0.00612333154555183\\
180	0.00612333154599885\\
181	0.00612333154645387\\
182	0.00612333154691703\\
183	0.00612333154738848\\
184	0.00612333154786837\\
185	0.00612333154835685\\
186	0.00612333154885407\\
187	0.00612333154936019\\
188	0.00612333154987538\\
189	0.00612333155039979\\
190	0.0061233315509336\\
191	0.00612333155147697\\
192	0.00612333155203007\\
193	0.00612333155259308\\
194	0.00612333155316619\\
195	0.00612333155374956\\
196	0.00612333155434339\\
197	0.00612333155494787\\
198	0.00612333155556319\\
199	0.00612333155618955\\
200	0.00612333155682714\\
201	0.00612333155747617\\
202	0.00612333155813684\\
203	0.00612333155880937\\
204	0.00612333155949397\\
205	0.00612333156019087\\
206	0.00612333156090028\\
207	0.00612333156162243\\
208	0.00612333156235755\\
209	0.00612333156310589\\
210	0.00612333156386767\\
211	0.00612333156464314\\
212	0.00612333156543255\\
213	0.00612333156623616\\
214	0.00612333156705423\\
215	0.00612333156788701\\
216	0.00612333156873478\\
217	0.0061233315695978\\
218	0.00612333157047637\\
219	0.00612333157137075\\
220	0.00612333157228125\\
221	0.00612333157320814\\
222	0.00612333157415174\\
223	0.00612333157511235\\
224	0.00612333157609028\\
225	0.00612333157708585\\
226	0.00612333157809938\\
227	0.00612333157913119\\
228	0.00612333158018164\\
229	0.00612333158125104\\
230	0.00612333158233976\\
231	0.00612333158344815\\
232	0.00612333158457657\\
233	0.00612333158572539\\
234	0.00612333158689498\\
235	0.00612333158808572\\
236	0.00612333158929801\\
237	0.00612333159053225\\
238	0.00612333159178882\\
239	0.00612333159306816\\
240	0.00612333159437068\\
241	0.00612333159569679\\
242	0.00612333159704696\\
243	0.00612333159842161\\
244	0.0061233315998212\\
245	0.00612333160124619\\
246	0.00612333160269706\\
247	0.00612333160417427\\
248	0.00612333160567832\\
249	0.00612333160720971\\
250	0.00612333160876895\\
251	0.00612333161035655\\
252	0.00612333161197304\\
253	0.00612333161361895\\
254	0.00612333161529484\\
255	0.00612333161700126\\
256	0.00612333161873879\\
257	0.00612333162050799\\
258	0.00612333162230947\\
259	0.00612333162414382\\
260	0.00612333162601166\\
261	0.00612333162791361\\
262	0.00612333162985031\\
263	0.00612333163182242\\
264	0.00612333163383059\\
265	0.0061233316358755\\
266	0.00612333163795784\\
267	0.0061233316400783\\
268	0.00612333164223762\\
269	0.00612333164443651\\
270	0.00612333164667572\\
271	0.006123331648956\\
272	0.00612333165127813\\
273	0.00612333165364289\\
274	0.00612333165605109\\
275	0.00612333165850355\\
276	0.00612333166100109\\
277	0.00612333166354458\\
278	0.00612333166613486\\
279	0.00612333166877283\\
280	0.00612333167145939\\
281	0.00612333167419545\\
282	0.00612333167698195\\
283	0.00612333167981984\\
284	0.00612333168271008\\
285	0.00612333168565368\\
286	0.00612333168865163\\
287	0.00612333169170497\\
288	0.00612333169481475\\
289	0.00612333169798202\\
290	0.00612333170120788\\
291	0.00612333170449343\\
292	0.00612333170783981\\
293	0.00612333171124817\\
294	0.00612333171471967\\
295	0.00612333171825551\\
296	0.00612333172185692\\
297	0.00612333172552512\\
298	0.00612333172926138\\
299	0.00612333173306699\\
300	0.00612333173694327\\
301	0.00612333174089153\\
302	0.00612333174491316\\
303	0.00612333174900952\\
304	0.00612333175318203\\
305	0.00612333175743213\\
306	0.00612333176176128\\
307	0.00612333176617098\\
308	0.00612333177066273\\
309	0.00612333177523808\\
310	0.00612333177989862\\
311	0.00612333178464593\\
312	0.00612333178948165\\
313	0.00612333179440743\\
314	0.00612333179942498\\
315	0.006123331804536\\
316	0.00612333180974224\\
317	0.0061233318150455\\
318	0.00612333182044757\\
319	0.00612333182595031\\
320	0.00612333183155559\\
321	0.00612333183726532\\
322	0.00612333184308144\\
323	0.00612333184900594\\
324	0.00612333185504081\\
325	0.00612333186118812\\
326	0.00612333186744993\\
327	0.00612333187382837\\
328	0.00612333188032559\\
329	0.00612333188694378\\
330	0.00612333189368517\\
331	0.00612333190055203\\
332	0.00612333190754666\\
333	0.0061233319146714\\
334	0.00612333192192864\\
335	0.00612333192932081\\
336	0.00612333193685038\\
337	0.00612333194451985\\
338	0.00612333195233178\\
339	0.00612333196028876\\
340	0.00612333196839344\\
341	0.0061233319766485\\
342	0.00612333198505669\\
343	0.00612333199362078\\
344	0.0061233320023436\\
345	0.00612333201122803\\
346	0.00612333202027701\\
347	0.00612333202949351\\
348	0.00612333203888058\\
349	0.0061233320484413\\
350	0.00612333205817882\\
351	0.00612333206809635\\
352	0.00612333207819715\\
353	0.00612333208848454\\
354	0.00612333209896192\\
355	0.00612333210963274\\
356	0.00612333212050051\\
357	0.00612333213156884\\
358	0.00612333214284137\\
359	0.00612333215432186\\
360	0.00612333216601411\\
361	0.00612333217792204\\
362	0.00612333219004961\\
363	0.0061233322024009\\
364	0.00612333221498009\\
365	0.00612333222779143\\
366	0.00612333224083928\\
367	0.00612333225412815\\
368	0.00612333226766259\\
369	0.00612333228144734\\
370	0.00612333229548723\\
371	0.00612333230978722\\
372	0.00612333232435245\\
373	0.00612333233918817\\
374	0.00612333235429981\\
375	0.00612333236969299\\
376	0.00612333238537349\\
377	0.00612333240134729\\
378	0.00612333241762059\\
379	0.00612333243419982\\
380	0.00612333245109164\\
381	0.006123332468303\\
382	0.00612333248584112\\
383	0.00612333250371355\\
384	0.00612333252192822\\
385	0.00612333254049343\\
386	0.00612333255941793\\
387	0.00612333257871095\\
388	0.00612333259838223\\
389	0.00612333261844197\\
390	0.00612333263890084\\
391	0.00612333265976994\\
392	0.00612333268106099\\
393	0.00612333270278657\\
394	0.0061233327249601\\
395	0.00612333274759572\\
396	0.0061233327707083\\
397	0.00612333279431344\\
398	0.00612333281842746\\
399	0.00612333284306739\\
400	0.00612333286825098\\
401	0.00612333289399696\\
402	0.00612333292032533\\
403	0.00612333294725796\\
404	0.00612333297481895\\
405	0.00612333300303422\\
406	0.00612333303193053\\
407	0.00612333306153556\\
408	0.00612333309187924\\
409	0.00612333312299409\\
410	0.0061233331549155\\
411	0.0061233331876822\\
412	0.00612333322133662\\
413	0.00612333325592553\\
414	0.00612333329150067\\
415	0.00612333332811961\\
416	0.00612333336584694\\
417	0.00612333340475602\\
418	0.00612333344493191\\
419	0.00612333348647652\\
420	0.00612333352951816\\
421	0.0061233335742297\\
422	0.00612333362086176\\
423	0.00612333366980066\\
424	0.00612333372165936\\
425	0.00612333377739507\\
426	0.00612333383839862\\
427	0.00612333390639754\\
428	0.00612333398288767\\
429	0.00612333406789588\\
430	0.00612333415878554\\
431	0.00612333425173984\\
432	0.00612333434682346\\
433	0.00612333444410418\\
434	0.00612333454365384\\
435	0.00612333464555074\\
436	0.00612333474988377\\
437	0.00612333485675714\\
438	0.0061233349662959\\
439	0.00612333507867288\\
440	0.00612333519417509\\
441	0.00612333531335248\\
442	0.00612333543735627\\
443	0.006123335568684\\
444	0.00612333571274892\\
445	0.00612333588098476\\
446	0.00612333609643078\\
447	0.0061233364022547\\
448	0.00612333687062499\\
449	0.00612333760035324\\
450	0.00612333867582629\\
451	0.00612334006033904\\
452	0.00612334151805552\\
453	0.00612334300283723\\
454	0.00612334451497056\\
455	0.00612334605477471\\
456	0.00612334762261034\\
457	0.00612334921888093\\
458	0.00612335084402158\\
459	0.00612335249847965\\
460	0.00612335418271947\\
461	0.00612335589731353\\
462	0.0061233576431221\\
463	0.00612335942126231\\
464	0.00612336123264819\\
465	0.00612336307821368\\
466	0.00612336495893177\\
467	0.00612336687583198\\
468	0.00612336882998634\\
469	0.00612337082254784\\
470	0.00612337285480842\\
471	0.00612337492822521\\
472	0.0061233770443228\\
473	0.00612337920451999\\
474	0.00612338141032804\\
475	0.00612338366335916\\
476	0.00612338596533573\\
477	0.00612338831810056\\
478	0.00612339072362854\\
479	0.00612339318403979\\
480	0.00612339570161451\\
481	0.00612339827881008\\
482	0.00612340091828052\\
483	0.00612340362289906\\
484	0.00612340639578422\\
485	0.00612340924033031\\
486	0.00612341216024336\\
487	0.00612341515958419\\
488	0.0061234182428213\\
489	0.00612342141489905\\
490	0.00612342468133214\\
491	0.00612342804834999\\
492	0.00612343152314132\\
493	0.00612343511430621\\
494	0.00612343883273651\\
495	0.0061234426933556\\
496	0.0061234467184931\\
497	0.00612345094410432\\
498	0.00612345543019226\\
499	0.00612346027535257\\
500	0.00612346562911269\\
501	0.00612347167931795\\
502	0.00612347856496256\\
503	0.00612348617018915\\
504	0.00612349399886649\\
505	0.00612350194271402\\
506	0.00612351001672883\\
507	0.00612351823541286\\
508	0.00612352660578226\\
509	0.00612353513630956\\
510	0.00612354383829377\\
511	0.00612355272923643\\
512	0.00612356184112266\\
513	0.0061235712405534\\
514	0.00612358107670167\\
515	0.00612359169192493\\
516	0.00612360386495179\\
517	0.00612361930707401\\
518	0.00612364155179061\\
519	0.00612367715832927\\
520	0.00612373606591696\\
521	0.00612382697681332\\
522	0.00612394161842495\\
523	0.00612405925830685\\
524	0.00612418014996012\\
525	0.0061243045842138\\
526	0.00612443291031743\\
527	0.0061245654992447\\
528	0.00612470263138651\\
529	0.00612484436620328\\
530	0.00612499084176697\\
531	0.00612514344412451\\
532	0.00612530515912619\\
533	0.00612548106565689\\
534	0.00612567850675068\\
535	0.00612590690897643\\
536	0.00612617082184989\\
537	0.0061264523687656\\
538	0.00612673858468065\\
539	0.00612702980938359\\
540	0.00612732643867049\\
541	0.00612762895759816\\
542	0.00612793801356077\\
543	0.00612825457131476\\
544	0.00612858020262912\\
545	0.00612891755369671\\
546	0.00612927095494212\\
547	0.00612964676683896\\
548	0.00613005186831006\\
549	0.00613048819487575\\
550	0.006130956352701\\
551	0.00613149812525851\\
552	0.00613223680990467\\
553	0.00613349561535179\\
554	0.00613590184192054\\
555	0.00614018159167847\\
556	0.00614488258111959\\
557	0.00615000623395128\\
558	0.00615584271042338\\
559	0.00616286636953639\\
560	0.00617134956848266\\
561	0.00618001076769368\\
562	0.00618890086222981\\
563	0.00619812775862577\\
564	0.00620787765005245\\
565	0.00621836770034789\\
566	0.00622925516894777\\
567	0.00624017357319983\\
568	0.00625114413080589\\
569	0.00626218623723885\\
570	0.0062732588449742\\
571	0.00628396345137723\\
572	0.00629359538457764\\
573	0.0063024565884034\\
574	0.00631193880828286\\
575	0.00632311293512667\\
576	0.00633912660832106\\
577	0.00636892477166563\\
578	0.00643825590142264\\
579	0.00662260426925778\\
580	0.00687560526233001\\
581	0.00716140658477241\\
582	0.00745664658486484\\
583	0.0077664865553155\\
584	0.00808893954834654\\
585	0.00841648692008067\\
586	0.00875120222867915\\
587	0.00909545523576583\\
588	0.0094551562817773\\
589	0.00982732353045407\\
590	0.0102219626909677\\
591	0.0106195309101401\\
592	0.0110105218793053\\
593	0.0113688158692212\\
594	0.0116257831346641\\
595	0.0118959542233756\\
596	0.0121471190838585\\
597	0.0124382624750838\\
598	0.0128122611038263\\
599	0\\
600	0\\
};
\addplot [color=mycolor14,solid,forget plot]
  table[row sep=crcr]{%
1	0.00582958538134794\\
2	0.0058295853817239\\
3	0.00582958538210659\\
4	0.00582958538249612\\
5	0.00582958538289261\\
6	0.00582958538329619\\
7	0.00582958538370699\\
8	0.00582958538412513\\
9	0.00582958538455074\\
10	0.00582958538498397\\
11	0.00582958538542493\\
12	0.00582958538587378\\
13	0.00582958538633065\\
14	0.00582958538679569\\
15	0.00582958538726904\\
16	0.00582958538775085\\
17	0.00582958538824128\\
18	0.00582958538874046\\
19	0.00582958538924857\\
20	0.00582958538976576\\
21	0.00582958539029219\\
22	0.00582958539082803\\
23	0.00582958539137344\\
24	0.0058295853919286\\
25	0.00582958539249368\\
26	0.00582958539306886\\
27	0.00582958539365431\\
28	0.00582958539425023\\
29	0.00582958539485679\\
30	0.00582958539547419\\
31	0.00582958539610262\\
32	0.00582958539674228\\
33	0.00582958539739337\\
34	0.00582958539805608\\
35	0.00582958539873064\\
36	0.00582958539941725\\
37	0.00582958540011612\\
38	0.00582958540082747\\
39	0.00582958540155153\\
40	0.00582958540228853\\
41	0.00582958540303869\\
42	0.00582958540380224\\
43	0.00582958540457944\\
44	0.00582958540537052\\
45	0.00582958540617572\\
46	0.0058295854069953\\
47	0.00582958540782953\\
48	0.00582958540867865\\
49	0.00582958540954293\\
50	0.00582958541042265\\
51	0.00582958541131808\\
52	0.00582958541222949\\
53	0.00582958541315718\\
54	0.00582958541410144\\
55	0.00582958541506256\\
56	0.00582958541604083\\
57	0.00582958541703658\\
58	0.0058295854180501\\
59	0.00582958541908171\\
60	0.00582958542013175\\
61	0.00582958542120053\\
62	0.00582958542228839\\
63	0.00582958542339567\\
64	0.00582958542452272\\
65	0.00582958542566989\\
66	0.00582958542683754\\
67	0.00582958542802603\\
68	0.00582958542923573\\
69	0.00582958543046703\\
70	0.00582958543172031\\
71	0.00582958543299595\\
72	0.00582958543429437\\
73	0.00582958543561595\\
74	0.00582958543696113\\
75	0.00582958543833032\\
76	0.00582958543972394\\
77	0.00582958544114243\\
78	0.00582958544258623\\
79	0.00582958544405581\\
80	0.00582958544555161\\
81	0.0058295854470741\\
82	0.00582958544862377\\
83	0.00582958545020108\\
84	0.00582958545180655\\
85	0.00582958545344066\\
86	0.00582958545510393\\
87	0.00582958545679688\\
88	0.00582958545852004\\
89	0.00582958546027394\\
90	0.00582958546205914\\
91	0.00582958546387619\\
92	0.00582958546572567\\
93	0.00582958546760814\\
94	0.0058295854695242\\
95	0.00582958547147444\\
96	0.00582958547345948\\
97	0.00582958547547994\\
98	0.00582958547753644\\
99	0.00582958547962963\\
100	0.00582958548176017\\
101	0.00582958548392871\\
102	0.00582958548613595\\
103	0.00582958548838256\\
104	0.00582958549066925\\
105	0.00582958549299674\\
106	0.00582958549536574\\
107	0.00582958549777701\\
108	0.0058295855002313\\
109	0.00582958550272937\\
110	0.005829585505272\\
111	0.00582958550785999\\
112	0.00582958551049416\\
113	0.00582958551317531\\
114	0.00582958551590429\\
115	0.00582958551868195\\
116	0.00582958552150917\\
117	0.00582958552438682\\
118	0.00582958552731581\\
119	0.00582958553029705\\
120	0.00582958553333147\\
121	0.00582958553642002\\
122	0.00582958553956367\\
123	0.00582958554276341\\
124	0.00582958554602022\\
125	0.00582958554933514\\
126	0.0058295855527092\\
127	0.00582958555614345\\
128	0.00582958555963896\\
129	0.00582958556319684\\
130	0.0058295855668182\\
131	0.00582958557050416\\
132	0.00582958557425589\\
133	0.00582958557807455\\
134	0.00582958558196135\\
135	0.00582958558591749\\
136	0.00582958558994423\\
137	0.00582958559404281\\
138	0.00582958559821452\\
139	0.00582958560246067\\
140	0.0058295856067826\\
141	0.00582958561118164\\
142	0.00582958561565919\\
143	0.00582958562021664\\
144	0.00582958562485542\\
145	0.00582958562957699\\
146	0.00582958563438283\\
147	0.00582958563927444\\
148	0.00582958564425336\\
149	0.00582958564932115\\
150	0.0058295856544794\\
151	0.00582958565972972\\
152	0.00582958566507377\\
153	0.00582958567051322\\
154	0.00582958567604978\\
155	0.00582958568168519\\
156	0.00582958568742122\\
157	0.00582958569325967\\
158	0.00582958569920236\\
159	0.00582958570525118\\
160	0.00582958571140802\\
161	0.00582958571767482\\
162	0.00582958572405355\\
163	0.00582958573054621\\
164	0.00582958573715484\\
165	0.00582958574388153\\
166	0.00582958575072839\\
167	0.00582958575769757\\
168	0.00582958576479128\\
169	0.00582958577201175\\
170	0.00582958577936124\\
171	0.00582958578684208\\
172	0.00582958579445663\\
173	0.00582958580220727\\
174	0.00582958581009648\\
175	0.00582958581812671\\
176	0.00582958582630053\\
177	0.00582958583462049\\
178	0.00582958584308924\\
179	0.00582958585170944\\
180	0.00582958586048383\\
181	0.00582958586941516\\
182	0.00582958587850628\\
183	0.00582958588776005\\
184	0.00582958589717941\\
185	0.00582958590676732\\
186	0.00582958591652684\\
187	0.00582958592646106\\
188	0.00582958593657311\\
189	0.00582958594686622\\
190	0.00582958595734363\\
191	0.00582958596800869\\
192	0.00582958597886477\\
193	0.00582958598991533\\
194	0.00582958600116386\\
195	0.00582958601261396\\
196	0.00582958602426926\\
197	0.00582958603613347\\
198	0.00582958604821037\\
199	0.00582958606050381\\
200	0.0058295860730177\\
201	0.00582958608575603\\
202	0.00582958609872287\\
203	0.00582958611192235\\
204	0.0058295861253587\\
205	0.0058295861390362\\
206	0.00582958615295923\\
207	0.00582958616713224\\
208	0.00582958618155977\\
209	0.00582958619624644\\
210	0.00582958621119695\\
211	0.0058295862264161\\
212	0.00582958624190878\\
213	0.00582958625767995\\
214	0.00582958627373469\\
215	0.00582958629007815\\
216	0.0058295863067156\\
217	0.00582958632365238\\
218	0.00582958634089395\\
219	0.00582958635844588\\
220	0.00582958637631382\\
221	0.00582958639450355\\
222	0.00582958641302093\\
223	0.00582958643187195\\
224	0.00582958645106271\\
225	0.00582958647059943\\
226	0.00582958649048843\\
227	0.00582958651073616\\
228	0.00582958653134919\\
229	0.00582958655233423\\
230	0.00582958657369808\\
231	0.00582958659544769\\
232	0.00582958661759016\\
233	0.00582958664013268\\
234	0.00582958666308262\\
235	0.00582958668644746\\
236	0.00582958671023483\\
237	0.00582958673445251\\
238	0.00582958675910843\\
239	0.00582958678421065\\
240	0.00582958680976741\\
241	0.0058295868357871\\
242	0.00582958686227825\\
243	0.00582958688924957\\
244	0.00582958691670994\\
245	0.0058295869446684\\
246	0.00582958697313417\\
247	0.00582958700211662\\
248	0.00582958703162535\\
249	0.00582958706167009\\
250	0.0058295870922608\\
251	0.00582958712340759\\
252	0.00582958715512081\\
253	0.00582958718741096\\
254	0.00582958722028878\\
255	0.00582958725376519\\
256	0.00582958728785134\\
257	0.00582958732255859\\
258	0.0058295873578985\\
259	0.00582958739388288\\
260	0.00582958743052374\\
261	0.00582958746783335\\
262	0.0058295875058242\\
263	0.00582958754450901\\
264	0.00582958758390076\\
265	0.00582958762401269\\
266	0.00582958766485827\\
267	0.00582958770645125\\
268	0.00582958774880565\\
269	0.00582958779193572\\
270	0.00582958783585604\\
271	0.00582958788058144\\
272	0.00582958792612704\\
273	0.00582958797250824\\
274	0.00582958801974077\\
275	0.00582958806784064\\
276	0.00582958811682418\\
277	0.00582958816670801\\
278	0.0058295882175091\\
279	0.00582958826924473\\
280	0.00582958832193252\\
281	0.00582958837559043\\
282	0.00582958843023677\\
283	0.00582958848589019\\
284	0.00582958854256969\\
285	0.00582958860029467\\
286	0.00582958865908486\\
287	0.0058295887189604\\
288	0.0058295887799418\\
289	0.00582958884204996\\
290	0.00582958890530619\\
291	0.00582958896973219\\
292	0.0058295890353501\\
293	0.00582958910218246\\
294	0.00582958917025223\\
295	0.00582958923958282\\
296	0.00582958931019809\\
297	0.00582958938212232\\
298	0.00582958945538028\\
299	0.00582958952999718\\
300	0.00582958960599873\\
301	0.00582958968341109\\
302	0.00582958976226092\\
303	0.00582958984257539\\
304	0.00582958992438216\\
305	0.0058295900077094\\
306	0.00582959009258581\\
307	0.00582959017904061\\
308	0.00582959026710356\\
309	0.00582959035680496\\
310	0.00582959044817567\\
311	0.0058295905412471\\
312	0.00582959063605123\\
313	0.00582959073262064\\
314	0.00582959083098847\\
315	0.00582959093118847\\
316	0.00582959103325497\\
317	0.00582959113722295\\
318	0.00582959124312798\\
319	0.00582959135100626\\
320	0.00582959146089465\\
321	0.00582959157283064\\
322	0.00582959168685238\\
323	0.0058295918029987\\
324	0.00582959192130909\\
325	0.00582959204182373\\
326	0.0058295921645835\\
327	0.00582959228962998\\
328	0.00582959241700549\\
329	0.00582959254675305\\
330	0.00582959267891644\\
331	0.00582959281354018\\
332	0.00582959295066956\\
333	0.00582959309035065\\
334	0.00582959323263029\\
335	0.00582959337755615\\
336	0.00582959352517671\\
337	0.00582959367554126\\
338	0.00582959382869997\\
339	0.00582959398470383\\
340	0.00582959414360476\\
341	0.00582959430545553\\
342	0.00582959447030984\\
343	0.00582959463822232\\
344	0.00582959480924854\\
345	0.00582959498344505\\
346	0.0058295951608694\\
347	0.00582959534158011\\
348	0.00582959552563678\\
349	0.00582959571310003\\
350	0.00582959590403161\\
351	0.00582959609849436\\
352	0.00582959629655224\\
353	0.00582959649827044\\
354	0.00582959670371531\\
355	0.0058295969129545\\
356	0.00582959712605689\\
357	0.00582959734309276\\
358	0.00582959756413371\\
359	0.0058295977892528\\
360	0.00582959801852459\\
361	0.00582959825202515\\
362	0.00582959848983218\\
363	0.00582959873202506\\
364	0.00582959897868491\\
365	0.0058295992298947\\
366	0.00582959948573932\\
367	0.00582959974630568\\
368	0.0058296000116828\\
369	0.00582960028196196\\
370	0.00582960055723677\\
371	0.00582960083760336\\
372	0.00582960112316047\\
373	0.00582960141400965\\
374	0.00582960171025541\\
375	0.00582960201200542\\
376	0.00582960231937068\\
377	0.0058296026324658\\
378	0.00582960295140919\\
379	0.00582960327632338\\
380	0.00582960360733529\\
381	0.00582960394457657\\
382	0.00582960428818403\\
383	0.00582960463830002\\
384	0.00582960499507305\\
385	0.00582960535865839\\
386	0.00582960572921891\\
387	0.005829606106926\\
388	0.00582960649196056\\
389	0.00582960688451369\\
390	0.00582960728478655\\
391	0.00582960769298912\\
392	0.00582960810933893\\
393	0.00582960853406322\\
394	0.00582960896740667\\
395	0.00582960940963215\\
396	0.0058296098610169\\
397	0.00582961032185298\\
398	0.00582961079244737\\
399	0.0058296112731215\\
400	0.00582961176421056\\
401	0.0058296122660629\\
402	0.00582961277904154\\
403	0.00582961330353036\\
404	0.00582961383994705\\
405	0.00582961438875788\\
406	0.00582961495047681\\
407	0.0058296155256341\\
408	0.00582961611476347\\
409	0.00582961671844148\\
410	0.00582961733729288\\
411	0.00582961797199679\\
412	0.00582961862329385\\
413	0.00582961929199461\\
414	0.0058296199789892\\
415	0.00582962068525896\\
416	0.00582962141189038\\
417	0.00582962216009264\\
418	0.00582962293122101\\
419	0.00582962372681105\\
420	0.0058296245486348\\
421	0.00582962539880278\\
422	0.00582962627996381\\
423	0.00582962719570759\\
424	0.00582962815136812\\
425	0.00582962915555119\\
426	0.00582963022276766\\
427	0.0058296313771673\\
428	0.00582963265555936\\
429	0.00582963410313205\\
430	0.00582963574838076\\
431	0.00582963755119696\\
432	0.00582963939479503\\
433	0.00582964128046762\\
434	0.00582964320957619\\
435	0.00582964518355714\\
436	0.00582964720393022\\
437	0.0058296492723123\\
438	0.0058296513904448\\
439	0.00582965356025649\\
440	0.00582965578401772\\
441	0.00582965806473182\\
442	0.00582966040713372\\
443	0.00582966282022291\\
444	0.0058296653235937\\
445	0.005829667962891\\
446	0.00582967084626833\\
447	0.00582967422602506\\
448	0.00582967866672493\\
449	0.00582968534224265\\
450	0.00582969639559791\\
451	0.00582971480129136\\
452	0.00582974191409959\\
453	0.005829770986764\\
454	0.00582980060131154\\
455	0.00582983076316588\\
456	0.00582986147834398\\
457	0.00582989275374301\\
458	0.00582992459738322\\
459	0.00582995701844631\\
460	0.00582999002693113\\
461	0.0058300236331148\\
462	0.00583005784812319\\
463	0.00583009268716015\\
464	0.00583012816791207\\
465	0.00583016430857424\\
466	0.0058302011275361\\
467	0.00583023864378673\\
468	0.00583027687760269\\
469	0.00583031584971009\\
470	0.00583035558186035\\
471	0.00583039609833999\\
472	0.00583043742780714\\
473	0.0058304796021032\\
474	0.00583052264882924\\
475	0.00583056659735822\\
476	0.00583061147899847\\
477	0.00583065732716743\\
478	0.00583070417758578\\
479	0.00583075206849526\\
480	0.00583080104090415\\
481	0.00583085113886539\\
482	0.00583090240979286\\
483	0.00583095490482302\\
484	0.00583100867923024\\
485	0.0058310637929063\\
486	0.00583112031091637\\
487	0.00583117830414725\\
488	0.00583123785006905\\
489	0.00583129903363768\\
490	0.0058313619483873\\
491	0.00583142669779906\\
492	0.00583149339713185\\
493	0.00583156217611772\\
494	0.00583163318344877\\
495	0.00583170659522542\\
496	0.00583178263227508\\
497	0.00583186159713117\\
498	0.00583194395304776\\
499	0.00583203048701399\\
500	0.00583212262081674\\
501	0.00583222291731705\\
502	0.0058323356133636\\
503	0.00583246613881396\\
504	0.00583261617545312\\
505	0.00583277179295458\\
506	0.00583292954933207\\
507	0.00583308975210002\\
508	0.00583325277222019\\
509	0.00583341873451183\\
510	0.00583358777851466\\
511	0.0058337600649959\\
512	0.00583393579219153\\
513	0.00583411523759399\\
514	0.00583429887008517\\
515	0.00583448765542392\\
516	0.00583468388718613\\
517	0.00583489341990013\\
518	0.00583513152082242\\
519	0.00583543748972403\\
520	0.00583590785938274\\
521	0.00583675591049733\\
522	0.00583833584480593\\
523	0.00584061277155873\\
524	0.00584294839343069\\
525	0.00584534686938407\\
526	0.00584781342617747\\
527	0.00585035481848645\\
528	0.00585297937938078\\
529	0.00585569497024925\\
530	0.00585850195453839\\
531	0.00586139016917696\\
532	0.00586437042898095\\
533	0.00586748028582171\\
534	0.00587079707657696\\
535	0.00587444466791646\\
536	0.00587866180477956\\
537	0.00588377530467723\\
538	0.00588937856467497\\
539	0.00589507424030023\\
540	0.00590086861589458\\
541	0.00590676874521496\\
542	0.00591278268876506\\
543	0.0059189200740091\\
544	0.00592519362455466\\
545	0.00593162277075402\\
546	0.00593824114150121\\
547	0.00594511143620707\\
548	0.0059523522096728\\
549	0.00596015137784947\\
550	0.00596856464983976\\
551	0.00597718259522469\\
552	0.0059859840032945\\
553	0.0059953471520073\\
554	0.00600863216314313\\
555	0.00603766671196547\\
556	0.00612112468559537\\
557	0.00621400588477447\\
558	0.00631333641990335\\
559	0.00642328807872805\\
560	0.00655649508322337\\
561	0.00673744096005576\\
562	0.00692495206043795\\
563	0.00711975569152903\\
564	0.00732361062886509\\
565	0.00754074109644888\\
566	0.00778401979610116\\
567	0.00804733790929998\\
568	0.00831811668216551\\
569	0.00859730677617569\\
570	0.00888668648769891\\
571	0.00919074662973911\\
572	0.00950307203274264\\
573	0.0097867009811926\\
574	0.0100321540985466\\
575	0.0102859780417936\\
576	0.0105482685783307\\
577	0.0108211031448169\\
578	0.0111064746220399\\
579	0.0113025369869154\\
580	0.0114465247212759\\
581	0.0115638816196064\\
582	0.0116972840244324\\
583	0.0118316131749101\\
584	0.0119494792138259\\
585	0.0120691663285315\\
586	0.0121908257470425\\
587	0.0123147668054481\\
588	0.012443283960156\\
589	0.0125623922825788\\
590	0.0126929540601676\\
591	0.0128425876192921\\
592	0.0129891468420991\\
593	0.0131272830991681\\
594	0.0132426187737169\\
595	0.0133471971667729\\
596	0.0134467378825905\\
597	0.0135473485564097\\
598	0.0136637438596031\\
599	0\\
600	0\\
};
\addplot [color=mycolor15,solid,forget plot]
  table[row sep=crcr]{%
1	0.00599032346902476\\
2	0.00599032347639302\\
3	0.00599032348389301\\
4	0.00599032349152711\\
5	0.00599032349929768\\
6	0.00599032350720719\\
7	0.0059903235152581\\
8	0.00599032352345294\\
9	0.00599032353179428\\
10	0.00599032354028474\\
11	0.00599032354892697\\
12	0.00599032355772368\\
13	0.00599032356667764\\
14	0.00599032357579165\\
15	0.00599032358506856\\
16	0.00599032359451129\\
17	0.00599032360412279\\
18	0.00599032361390607\\
19	0.00599032362386421\\
20	0.00599032363400032\\
21	0.00599032364431757\\
22	0.0059903236548192\\
23	0.00599032366550851\\
24	0.00599032367638883\\
25	0.00599032368746358\\
26	0.00599032369873622\\
27	0.00599032371021029\\
28	0.00599032372188939\\
29	0.00599032373377716\\
30	0.00599032374587734\\
31	0.0059903237581937\\
32	0.00599032377073012\\
33	0.00599032378349051\\
34	0.00599032379647887\\
35	0.00599032380969927\\
36	0.00599032382315584\\
37	0.00599032383685281\\
38	0.00599032385079445\\
39	0.00599032386498513\\
40	0.0059903238794293\\
41	0.00599032389413147\\
42	0.00599032390909625\\
43	0.00599032392432833\\
44	0.00599032393983246\\
45	0.0059903239556135\\
46	0.0059903239716764\\
47	0.00599032398802616\\
48	0.00599032400466793\\
49	0.00599032402160689\\
50	0.00599032403884835\\
51	0.00599032405639771\\
52	0.00599032407426046\\
53	0.00599032409244217\\
54	0.00599032411094856\\
55	0.00599032412978539\\
56	0.00599032414895856\\
57	0.00599032416847408\\
58	0.00599032418833805\\
59	0.00599032420855667\\
60	0.00599032422913628\\
61	0.00599032425008331\\
62	0.00599032427140431\\
63	0.00599032429310594\\
64	0.005990324315195\\
65	0.0059903243376784\\
66	0.00599032436056316\\
67	0.00599032438385643\\
68	0.00599032440756551\\
69	0.00599032443169781\\
70	0.00599032445626087\\
71	0.00599032448126237\\
72	0.00599032450671012\\
73	0.00599032453261209\\
74	0.00599032455897638\\
75	0.00599032458581121\\
76	0.00599032461312499\\
77	0.00599032464092625\\
78	0.00599032466922368\\
79	0.00599032469802612\\
80	0.00599032472734259\\
81	0.00599032475718224\\
82	0.00599032478755439\\
83	0.00599032481846856\\
84	0.00599032484993439\\
85	0.00599032488196172\\
86	0.00599032491456056\\
87	0.0059903249477411\\
88	0.00599032498151371\\
89	0.00599032501588895\\
90	0.00599032505087755\\
91	0.00599032508649046\\
92	0.0059903251227388\\
93	0.00599032515963391\\
94	0.00599032519718731\\
95	0.00599032523541073\\
96	0.00599032527431614\\
97	0.00599032531391568\\
98	0.00599032535422172\\
99	0.00599032539524688\\
100	0.00599032543700397\\
101	0.00599032547950603\\
102	0.00599032552276636\\
103	0.00599032556679847\\
104	0.00599032561161612\\
105	0.00599032565723333\\
106	0.00599032570366435\\
107	0.00599032575092369\\
108	0.00599032579902613\\
109	0.00599032584798669\\
110	0.00599032589782069\\
111	0.0059903259485437\\
112	0.00599032600017158\\
113	0.00599032605272047\\
114	0.00599032610620679\\
115	0.00599032616064727\\
116	0.00599032621605892\\
117	0.00599032627245907\\
118	0.00599032632986536\\
119	0.00599032638829573\\
120	0.00599032644776846\\
121	0.00599032650830214\\
122	0.0059903265699157\\
123	0.00599032663262842\\
124	0.00599032669645991\\
125	0.00599032676143013\\
126	0.00599032682755941\\
127	0.00599032689486842\\
128	0.00599032696337824\\
129	0.0059903270331103\\
130	0.00599032710408641\\
131	0.00599032717632878\\
132	0.00599032724986002\\
133	0.00599032732470315\\
134	0.00599032740088159\\
135	0.00599032747841918\\
136	0.0059903275573402\\
137	0.00599032763766937\\
138	0.00599032771943183\\
139	0.00599032780265318\\
140	0.0059903278873595\\
141	0.00599032797357731\\
142	0.00599032806133362\\
143	0.00599032815065594\\
144	0.00599032824157224\\
145	0.00599032833411101\\
146	0.00599032842830128\\
147	0.00599032852417254\\
148	0.00599032862175488\\
149	0.00599032872107888\\
150	0.00599032882217569\\
151	0.00599032892507703\\
152	0.00599032902981517\\
153	0.00599032913642299\\
154	0.00599032924493392\\
155	0.00599032935538204\\
156	0.00599032946780201\\
157	0.00599032958222914\\
158	0.00599032969869937\\
159	0.00599032981724927\\
160	0.00599032993791608\\
161	0.00599033006073774\\
162	0.00599033018575285\\
163	0.0059903303130007\\
164	0.00599033044252131\\
165	0.00599033057435542\\
166	0.00599033070854449\\
167	0.00599033084513077\\
168	0.00599033098415723\\
169	0.00599033112566765\\
170	0.00599033126970659\\
171	0.00599033141631942\\
172	0.00599033156555234\\
173	0.00599033171745239\\
174	0.00599033187206744\\
175	0.00599033202944626\\
176	0.0059903321896385\\
177	0.0059903323526947\\
178	0.00599033251866633\\
179	0.00599033268760579\\
180	0.00599033285956645\\
181	0.00599033303460263\\
182	0.00599033321276967\\
183	0.00599033339412388\\
184	0.00599033357872264\\
185	0.00599033376662434\\
186	0.00599033395788847\\
187	0.00599033415257559\\
188	0.00599033435074736\\
189	0.00599033455246659\\
190	0.00599033475779723\\
191	0.00599033496680439\\
192	0.00599033517955439\\
193	0.00599033539611476\\
194	0.00599033561655427\\
195	0.00599033584094296\\
196	0.00599033606935214\\
197	0.00599033630185446\\
198	0.00599033653852387\\
199	0.00599033677943572\\
200	0.00599033702466671\\
201	0.00599033727429499\\
202	0.00599033752840012\\
203	0.00599033778706316\\
204	0.00599033805036664\\
205	0.00599033831839463\\
206	0.00599033859123276\\
207	0.00599033886896823\\
208	0.00599033915168986\\
209	0.00599033943948813\\
210	0.00599033973245518\\
211	0.00599034003068488\\
212	0.00599034033427282\\
213	0.00599034064331638\\
214	0.00599034095791476\\
215	0.005990341278169\\
216	0.00599034160418201\\
217	0.00599034193605864\\
218	0.00599034227390568\\
219	0.00599034261783194\\
220	0.00599034296794822\\
221	0.00599034332436744\\
222	0.0059903436872046\\
223	0.00599034405657687\\
224	0.00599034443260361\\
225	0.00599034481540643\\
226	0.0059903452051092\\
227	0.00599034560183816\\
228	0.00599034600572187\\
229	0.00599034641689136\\
230	0.0059903468354801\\
231	0.00599034726162408\\
232	0.00599034769546186\\
233	0.00599034813713463\\
234	0.00599034858678621\\
235	0.00599034904456317\\
236	0.00599034951061486\\
237	0.00599034998509344\\
238	0.00599035046815397\\
239	0.00599035095995443\\
240	0.00599035146065582\\
241	0.00599035197042219\\
242	0.00599035248942072\\
243	0.00599035301782176\\
244	0.0059903535557989\\
245	0.00599035410352907\\
246	0.00599035466119253\\
247	0.00599035522897301\\
248	0.00599035580705774\\
249	0.00599035639563753\\
250	0.00599035699490685\\
251	0.00599035760506387\\
252	0.00599035822631057\\
253	0.0059903588588528\\
254	0.00599035950290036\\
255	0.00599036015866707\\
256	0.00599036082637085\\
257	0.00599036150623383\\
258	0.00599036219848237\\
259	0.00599036290334722\\
260	0.00599036362106355\\
261	0.00599036435187106\\
262	0.00599036509601407\\
263	0.00599036585374161\\
264	0.00599036662530749\\
265	0.00599036741097044\\
266	0.00599036821099415\\
267	0.00599036902564742\\
268	0.00599036985520423\\
269	0.00599037069994383\\
270	0.00599037156015089\\
271	0.00599037243611555\\
272	0.00599037332813355\\
273	0.00599037423650636\\
274	0.00599037516154125\\
275	0.00599037610355142\\
276	0.00599037706285613\\
277	0.00599037803978077\\
278	0.00599037903465704\\
279	0.005990380047823\\
280	0.00599038107962327\\
281	0.00599038213040906\\
282	0.00599038320053837\\
283	0.00599038429037609\\
284	0.00599038540029411\\
285	0.00599038653067149\\
286	0.00599038768189454\\
287	0.00599038885435699\\
288	0.00599039004846013\\
289	0.00599039126461291\\
290	0.00599039250323212\\
291	0.00599039376474249\\
292	0.00599039504957685\\
293	0.0059903963581763\\
294	0.00599039769099029\\
295	0.00599039904847684\\
296	0.00599040043110261\\
297	0.00599040183934313\\
298	0.00599040327368287\\
299	0.00599040473461546\\
300	0.0059904062226438\\
301	0.00599040773828022\\
302	0.00599040928204666\\
303	0.00599041085447477\\
304	0.00599041245610616\\
305	0.00599041408749244\\
306	0.00599041574919549\\
307	0.00599041744178756\\
308	0.00599041916585143\\
309	0.00599042092198059\\
310	0.00599042271077943\\
311	0.00599042453286332\\
312	0.00599042638885889\\
313	0.0059904282794041\\
314	0.00599043020514846\\
315	0.00599043216675317\\
316	0.00599043416489132\\
317	0.00599043620024807\\
318	0.00599043827352076\\
319	0.00599044038541915\\
320	0.0059904425366656\\
321	0.00599044472799519\\
322	0.00599044696015596\\
323	0.0059904492339091\\
324	0.00599045155002908\\
325	0.00599045390930391\\
326	0.00599045631253529\\
327	0.00599045876053885\\
328	0.00599046125414432\\
329	0.00599046379419578\\
330	0.00599046638155185\\
331	0.00599046901708592\\
332	0.0059904717016864\\
333	0.00599047443625695\\
334	0.0059904772217167\\
335	0.00599048005900056\\
336	0.00599048294905946\\
337	0.00599048589286061\\
338	0.00599048889138779\\
339	0.00599049194564168\\
340	0.00599049505664012\\
341	0.00599049822541846\\
342	0.00599050145302987\\
343	0.00599050474054569\\
344	0.00599050808905576\\
345	0.00599051149966887\\
346	0.00599051497351304\\
347	0.00599051851173602\\
348	0.00599052211550567\\
349	0.00599052578601042\\
350	0.00599052952445978\\
351	0.00599053333208481\\
352	0.00599053721013874\\
353	0.00599054115989747\\
354	0.00599054518266026\\
355	0.0059905492797504\\
356	0.00599055345251592\\
357	0.00599055770233039\\
358	0.00599056203059375\\
359	0.00599056643873321\\
360	0.00599057092820429\\
361	0.00599057550049181\\
362	0.00599058015711107\\
363	0.00599058489960911\\
364	0.00599058972956601\\
365	0.00599059464859639\\
366	0.00599059965835097\\
367	0.00599060476051826\\
368	0.00599060995682645\\
369	0.00599061524904537\\
370	0.00599062063898871\\
371	0.00599062612851638\\
372	0.00599063171953705\\
373	0.00599063741401096\\
374	0.00599064321395296\\
375	0.00599064912143581\\
376	0.00599065513859376\\
377	0.00599066126762651\\
378	0.0059906675108035\\
379	0.00599067387046862\\
380	0.00599068034904542\\
381	0.00599068694904281\\
382	0.00599069367306152\\
383	0.00599070052380127\\
384	0.00599070750406913\\
385	0.00599071461678928\\
386	0.00599072186501503\\
387	0.00599072925194376\\
388	0.00599073678093576\\
389	0.00599074445553644\\
390	0.00599075227949734\\
391	0.0059907602567838\\
392	0.00599076839154974\\
393	0.00599077668807921\\
394	0.00599078515078226\\
395	0.00599079378441388\\
396	0.00599080259412864\\
397	0.00599081158536669\\
398	0.00599082076386549\\
399	0.00599083013566539\\
400	0.00599083970710465\\
401	0.00599084948480091\\
402	0.00599085947562256\\
403	0.00599086968667511\\
404	0.00599088012536707\\
405	0.00599089079965561\\
406	0.00599090171848987\\
407	0.0059909128920625\\
408	0.00599092433100616\\
409	0.00599093604566406\\
410	0.00599094804719197\\
411	0.00599096034765849\\
412	0.00599097296016049\\
413	0.00599098589895647\\
414	0.00599099917962111\\
415	0.00599101281922494\\
416	0.00599102683654438\\
417	0.00599104125230952\\
418	0.00599105608950113\\
419	0.00599107137371775\\
420	0.00599108713365582\\
421	0.00599110340179911\\
422	0.00599112021554305\\
423	0.00599113761928811\\
424	0.00599115566874995\\
425	0.00599117444029749\\
426	0.0059911940512634\\
427	0.00599121470242646\\
428	0.00599123675903439\\
429	0.00599126087725931\\
430	0.00599128810623883\\
431	0.00599131962843508\\
432	0.00599135533124139\\
433	0.00599139183795031\\
434	0.00599142917377634\\
435	0.00599146736525415\\
436	0.00599150644033957\\
437	0.00599154642852507\\
438	0.00599158736098331\\
439	0.00599162927077509\\
440	0.00599167219322376\\
441	0.00599171616674131\\
442	0.0059917612349068\\
443	0.005991807452041\\
444	0.00599185489854579\\
445	0.00599190372341149\\
446	0.00599195426168655\\
447	0.00599200735582206\\
448	0.00599206521814722\\
449	0.00599213367019972\\
450	0.00599222761755948\\
451	0.00599238287805609\\
452	0.00599267419358242\\
453	0.00599320142795372\\
454	0.00599378085836966\\
455	0.00599437115849134\\
456	0.00599497243327278\\
457	0.00599558479693493\\
458	0.00599620837908925\\
459	0.00599684333147334\\
460	0.00599748983234564\\
461	0.00599814808180438\\
462	0.00599881828174953\\
463	0.00599950062359239\\
464	0.00600019539386675\\
465	0.00600090294588935\\
466	0.0060016236460515\\
467	0.00600235786094702\\
468	0.00600310596310707\\
469	0.00600386835861614\\
470	0.00600464545737945\\
471	0.0060054376764517\\
472	0.00600624546795066\\
473	0.00600706938373473\\
474	0.00600791010689268\\
475	0.00600876818034188\\
476	0.00600964418139982\\
477	0.00601053872500006\\
478	0.00601145246702651\\
479	0.00601238610802862\\
480	0.00601334039738017\\
481	0.00601431613795586\\
482	0.00601531419141411\\
483	0.00601633548419161\\
484	0.00601738101433689\\
485	0.00601845185933789\\
486	0.00601954918513108\\
487	0.00602067425651219\\
488	0.00602182844921712\\
489	0.00602301326403406\\
490	0.00602423034330445\\
491	0.00602548149043597\\
492	0.00602676869321937\\
493	0.00602809415256294\\
494	0.00602946031968202\\
495	0.00603086994855116\\
496	0.00603232618116426\\
497	0.00603383270871072\\
498	0.00603539411666303\\
499	0.00603701668047269\\
500	0.00603871024996858\\
501	0.00604049266486954\\
502	0.00604239962764516\\
503	0.00604450459405451\\
504	0.00604694951891126\\
505	0.0060499082676736\\
506	0.00605300983125695\\
507	0.00605615210312746\\
508	0.0060593407813001\\
509	0.00606258565355876\\
510	0.00606588912726207\\
511	0.00606925384372765\\
512	0.00607268271492154\\
513	0.00607617900036558\\
514	0.00607974646740256\\
515	0.00608338980084082\\
516	0.00608711579538251\\
517	0.00609093702913546\\
518	0.00609488359173495\\
519	0.00609904156391954\\
520	0.00610368229955621\\
521	0.00610970625651236\\
522	0.00612019689083974\\
523	0.00614595910776007\\
524	0.0061918955482278\\
525	0.00623920729408083\\
526	0.00628798809020923\\
527	0.0063383513221512\\
528	0.0063904436183794\\
529	0.00644446606450121\\
530	0.00650067913415457\\
531	0.00655921987394357\\
532	0.00661971651637051\\
533	0.0066821693836296\\
534	0.00674702570064966\\
535	0.00681553185147688\\
536	0.00688929070934847\\
537	0.00697272484397712\\
538	0.00708050838487127\\
539	0.00720430346932848\\
540	0.00733164834814307\\
541	0.00746279371554648\\
542	0.00759801929082208\\
543	0.00773763840445583\\
544	0.00788200385676951\\
545	0.00803152280858739\\
546	0.00818669749733148\\
547	0.00834822243528515\\
548	0.0085172314789705\\
549	0.00869615410810298\\
550	0.00889173048717438\\
551	0.00911191901563923\\
552	0.0093426467605882\\
553	0.00957820888500259\\
554	0.00978978294048548\\
555	0.00996250278044234\\
556	0.0100939384828728\\
557	0.0102245759089651\\
558	0.0103588032242156\\
559	0.0104969545349647\\
560	0.0106343065755494\\
561	0.010736880126034\\
562	0.0108496866690711\\
563	0.0109730606430093\\
564	0.0110985099408717\\
565	0.0112260664279247\\
566	0.0113424818351038\\
567	0.0114496888862977\\
568	0.0115590069721942\\
569	0.0116701319268866\\
570	0.0117835969687971\\
571	0.011899948581223\\
572	0.0120224663297457\\
573	0.012132495226828\\
574	0.0122315824166699\\
575	0.0123339138430404\\
576	0.01244517147675\\
577	0.0125563893121482\\
578	0.0126670103729859\\
579	0.0127598557290707\\
580	0.0128432753767443\\
581	0.0129212754827187\\
582	0.0129882112395968\\
583	0.0130515036111677\\
584	0.0131112486684305\\
585	0.0131701425753821\\
586	0.0132269000042697\\
587	0.013281671558656\\
588	0.0133309066443553\\
589	0.0133736658661722\\
590	0.0134129455168408\\
591	0.0134481178433173\\
592	0.0134797612143452\\
593	0.0135085207755725\\
594	0.0135350984511593\\
595	0.0135592314909786\\
596	0.0135833251697673\\
597	0.0136131889078992\\
598	0.0136637438596031\\
599	0\\
600	0\\
};
\addplot [color=mycolor16,solid,forget plot]
  table[row sep=crcr]{%
1	0.00604107245767237\\
2	0.00604107260381874\\
3	0.00604107275257828\\
4	0.00604107290399764\\
5	0.0060410730581243\\
6	0.00604107321500661\\
7	0.00604107337469376\\
8	0.00604107353723582\\
9	0.00604107370268375\\
10	0.00604107387108944\\
11	0.00604107404250568\\
12	0.00604107421698621\\
13	0.00604107439458573\\
14	0.0060410745753599\\
15	0.00604107475936541\\
16	0.00604107494665991\\
17	0.00604107513730212\\
18	0.00604107533135178\\
19	0.0060410755288697\\
20	0.00604107572991779\\
21	0.00604107593455903\\
22	0.00604107614285757\\
23	0.00604107635487865\\
24	0.00604107657068871\\
25	0.00604107679035536\\
26	0.00604107701394743\\
27	0.00604107724153494\\
28	0.0060410774731892\\
29	0.00604107770898275\\
30	0.00604107794898947\\
31	0.00604107819328452\\
32	0.0060410784419444\\
33	0.00604107869504699\\
34	0.00604107895267155\\
35	0.00604107921489876\\
36	0.00604107948181072\\
37	0.00604107975349099\\
38	0.00604108003002466\\
39	0.00604108031149828\\
40	0.00604108059799999\\
41	0.00604108088961946\\
42	0.006041081186448\\
43	0.00604108148857849\\
44	0.00604108179610553\\
45	0.00604108210912535\\
46	0.00604108242773594\\
47	0.00604108275203699\\
48	0.00604108308213\\
49	0.00604108341811827\\
50	0.00604108376010693\\
51	0.00604108410820299\\
52	0.00604108446251537\\
53	0.00604108482315494\\
54	0.00604108519023451\\
55	0.00604108556386894\\
56	0.0060410859441751\\
57	0.00604108633127197\\
58	0.00604108672528064\\
59	0.00604108712632436\\
60	0.00604108753452857\\
61	0.00604108795002095\\
62	0.00604108837293144\\
63	0.00604108880339232\\
64	0.00604108924153821\\
65	0.00604108968750613\\
66	0.00604109014143555\\
67	0.00604109060346842\\
68	0.00604109107374921\\
69	0.00604109155242498\\
70	0.00604109203964539\\
71	0.0060410925355628\\
72	0.00604109304033224\\
73	0.00604109355411155\\
74	0.00604109407706134\\
75	0.00604109460934511\\
76	0.00604109515112927\\
77	0.00604109570258319\\
78	0.00604109626387925\\
79	0.00604109683519292\\
80	0.0060410974167028\\
81	0.00604109800859065\\
82	0.00604109861104148\\
83	0.00604109922424361\\
84	0.00604109984838871\\
85	0.00604110048367186\\
86	0.00604110113029162\\
87	0.0060411017884501\\
88	0.00604110245835301\\
89	0.00604110314020972\\
90	0.00604110383423335\\
91	0.00604110454064079\\
92	0.00604110525965283\\
93	0.00604110599149419\\
94	0.00604110673639358\\
95	0.00604110749458382\\
96	0.00604110826630186\\
97	0.00604110905178889\\
98	0.00604110985129039\\
99	0.00604111066505623\\
100	0.00604111149334074\\
101	0.00604111233640278\\
102	0.00604111319450585\\
103	0.00604111406791813\\
104	0.0060411149569126\\
105	0.0060411158617671\\
106	0.00604111678276446\\
107	0.00604111772019253\\
108	0.00604111867434431\\
109	0.00604111964551804\\
110	0.00604112063401728\\
111	0.00604112164015101\\
112	0.00604112266423375\\
113	0.00604112370658561\\
114	0.00604112476753244\\
115	0.00604112584740592\\
116	0.00604112694654363\\
117	0.00604112806528921\\
118	0.00604112920399245\\
119	0.00604113036300936\\
120	0.00604113154270236\\
121	0.00604113274344032\\
122	0.00604113396559873\\
123	0.00604113520955978\\
124	0.00604113647571251\\
125	0.00604113776445293\\
126	0.00604113907618413\\
127	0.00604114041131642\\
128	0.00604114177026745\\
129	0.00604114315346236\\
130	0.0060411445613339\\
131	0.00604114599432259\\
132	0.00604114745287682\\
133	0.00604114893745302\\
134	0.00604115044851583\\
135	0.00604115198653819\\
136	0.00604115355200153\\
137	0.00604115514539594\\
138	0.00604115676722027\\
139	0.00604115841798234\\
140	0.00604116009819909\\
141	0.00604116180839674\\
142	0.00604116354911094\\
143	0.00604116532088699\\
144	0.00604116712427997\\
145	0.00604116895985495\\
146	0.00604117082818714\\
147	0.00604117272986213\\
148	0.006041174665476\\
149	0.00604117663563558\\
150	0.00604117864095861\\
151	0.00604118068207396\\
152	0.00604118275962181\\
153	0.0060411848742539\\
154	0.00604118702663369\\
155	0.00604118921743659\\
156	0.00604119144735022\\
157	0.00604119371707457\\
158	0.00604119602732228\\
159	0.00604119837881883\\
160	0.00604120077230283\\
161	0.0060412032085262\\
162	0.00604120568825446\\
163	0.00604120821226695\\
164	0.00604121078135712\\
165	0.00604121339633276\\
166	0.00604121605801628\\
167	0.00604121876724496\\
168	0.00604122152487127\\
169	0.00604122433176311\\
170	0.00604122718880408\\
171	0.00604123009689385\\
172	0.00604123305694836\\
173	0.00604123606990021\\
174	0.00604123913669891\\
175	0.0060412422583112\\
176	0.00604124543572141\\
177	0.00604124866993176\\
178	0.00604125196196269\\
179	0.0060412553128532\\
180	0.0060412587236612\\
181	0.0060412621954639\\
182	0.00604126572935811\\
183	0.00604126932646064\\
184	0.00604127298790867\\
185	0.00604127671486014\\
186	0.00604128050849412\\
187	0.00604128437001124\\
188	0.00604128830063406\\
189	0.00604129230160751\\
190	0.00604129637419929\\
191	0.00604130051970032\\
192	0.00604130473942516\\
193	0.00604130903471248\\
194	0.0060413134069255\\
195	0.00604131785745244\\
196	0.00604132238770704\\
197	0.006041326999129\\
198	0.0060413316931845\\
199	0.00604133647136667\\
200	0.00604134133519617\\
201	0.00604134628622164\\
202	0.00604135132602027\\
203	0.00604135645619836\\
204	0.00604136167839183\\
205	0.00604136699426683\\
206	0.00604137240552032\\
207	0.00604137791388061\\
208	0.00604138352110804\\
209	0.00604138922899551\\
210	0.00604139503936919\\
211	0.00604140095408909\\
212	0.00604140697504975\\
213	0.00604141310418091\\
214	0.00604141934344817\\
215	0.00604142569485369\\
216	0.00604143216043692\\
217	0.00604143874227528\\
218	0.00604144544248493\\
219	0.0060414522632215\\
220	0.0060414592066809\\
221	0.00604146627510004\\
222	0.00604147347075768\\
223	0.00604148079597522\\
224	0.00604148825311754\\
225	0.00604149584459385\\
226	0.00604150357285854\\
227	0.00604151144041212\\
228	0.00604151944980203\\
229	0.00604152760362364\\
230	0.00604153590452117\\
231	0.00604154435518861\\
232	0.00604155295837076\\
233	0.00604156171686419\\
234	0.00604157063351824\\
235	0.00604157971123613\\
236	0.00604158895297595\\
237	0.00604159836175177\\
238	0.00604160794063474\\
239	0.00604161769275425\\
240	0.00604162762129901\\
241	0.0060416377295183\\
242	0.00604164802072311\\
243	0.0060416584982874\\
244	0.00604166916564931\\
245	0.00604168002631246\\
246	0.00604169108384724\\
247	0.00604170234189212\\
248	0.00604171380415501\\
249	0.00604172547441465\\
250	0.00604173735652197\\
251	0.00604174945440157\\
252	0.00604176177205316\\
253	0.00604177431355303\\
254	0.0060417870830556\\
255	0.00604180008479495\\
256	0.00604181332308638\\
257	0.00604182680232807\\
258	0.00604184052700265\\
259	0.00604185450167894\\
260	0.0060418687310136\\
261	0.00604188321975289\\
262	0.00604189797273442\\
263	0.00604191299488896\\
264	0.0060419282912423\\
265	0.00604194386691706\\
266	0.00604195972713465\\
267	0.00604197587721716\\
268	0.00604199232258936\\
269	0.00604200906878071\\
270	0.00604202612142738\\
271	0.00604204348627433\\
272	0.00604206116917744\\
273	0.00604207917610567\\
274	0.00604209751314321\\
275	0.00604211618649171\\
276	0.00604213520247258\\
277	0.00604215456752925\\
278	0.0060421742882295\\
279	0.00604219437126785\\
280	0.00604221482346796\\
281	0.00604223565178506\\
282	0.00604225686330848\\
283	0.00604227846526408\\
284	0.0060423004650169\\
285	0.00604232287007367\\
286	0.00604234568808549\\
287	0.00604236892685046\\
288	0.00604239259431639\\
289	0.00604241669858353\\
290	0.00604244124790734\\
291	0.00604246625070126\\
292	0.00604249171553958\\
293	0.0060425176511603\\
294	0.00604254406646799\\
295	0.00604257097053676\\
296	0.00604259837261321\\
297	0.00604262628211939\\
298	0.00604265470865585\\
299	0.00604268366200469\\
300	0.00604271315213258\\
301	0.00604274318919392\\
302	0.00604277378353395\\
303	0.00604280494569188\\
304	0.00604283668640407\\
305	0.00604286901660724\\
306	0.0060429019474417\\
307	0.00604293549025455\\
308	0.00604296965660301\\
309	0.00604300445825766\\
310	0.00604303990720574\\
311	0.00604307601565452\\
312	0.00604311279603465\\
313	0.00604315026100349\\
314	0.00604318842344856\\
315	0.00604322729649096\\
316	0.00604326689348877\\
317	0.00604330722804057\\
318	0.00604334831398897\\
319	0.00604339016542409\\
320	0.00604343279668721\\
321	0.00604347622237433\\
322	0.0060435204573399\\
323	0.00604356551670048\\
324	0.00604361141583855\\
325	0.00604365817040637\\
326	0.00604370579632987\\
327	0.00604375430981263\\
328	0.00604380372734001\\
329	0.0060438540656833\\
330	0.00604390534190403\\
331	0.00604395757335837\\
332	0.00604401077770169\\
333	0.0060440649728932\\
334	0.00604412017720086\\
335	0.00604417640920629\\
336	0.00604423368780999\\
337	0.00604429203223669\\
338	0.00604435146204086\\
339	0.00604441199711243\\
340	0.00604447365768277\\
341	0.00604453646433083\\
342	0.00604460043798947\\
343	0.00604466559995213\\
344	0.00604473197187965\\
345	0.00604479957580744\\
346	0.00604486843415292\\
347	0.00604493856972325\\
348	0.00604501000572354\\
349	0.0060450827657654\\
350	0.00604515687387593\\
351	0.00604523235450739\\
352	0.00604530923254732\\
353	0.00604538753332946\\
354	0.0060454672826454\\
355	0.00604554850675698\\
356	0.00604563123240967\\
357	0.00604571548684694\\
358	0.0060458012978257\\
359	0.00604588869363288\\
360	0.00604597770310341\\
361	0.00604606835563952\\
362	0.00604616068123166\\
363	0.00604625471048106\\
364	0.00604635047462414\\
365	0.00604644800555896\\
366	0.00604654733587378\\
367	0.00604664849887804\\
368	0.00604675152863593\\
369	0.0060468564600027\\
370	0.00604696332866412\\
371	0.00604707217117922\\
372	0.00604718302502664\\
373	0.00604729592865502\\
374	0.00604741092153761\\
375	0.00604752804423182\\
376	0.00604764733844376\\
377	0.00604776884709875\\
378	0.0060478926144181\\
379	0.00604801868600297\\
380	0.00604814710892623\\
381	0.00604827793183324\\
382	0.00604841120505288\\
383	0.0060485469807205\\
384	0.00604868531291534\\
385	0.00604882625781609\\
386	0.00604896987388125\\
387	0.00604911622206545\\
388	0.00604926536609071\\
389	0.00604941737280087\\
390	0.00604957231262645\\
391	0.00604973026014073\\
392	0.00604989129451384\\
393	0.00605005549929356\\
394	0.00605022296069095\\
395	0.00605039376545726\\
396	0.00605056800689307\\
397	0.00605074578751558\\
398	0.00605092721555956\\
399	0.00605111240527569\\
400	0.00605130147713777\\
401	0.00605149455786511\\
402	0.00605169178012603\\
403	0.00605189328180619\\
404	0.00605209920498479\\
405	0.00605230969563584\\
406	0.00605252490688595\\
407	0.00605274500993518\\
408	0.0060529702088865\\
409	0.00605320072346447\\
410	0.00605343675722967\\
411	0.00605367852939864\\
412	0.00605392627674797\\
413	0.00605418025580464\\
414	0.00605444074537174\\
415	0.00605470804944785\\
416	0.00605498250061092\\
417	0.00605526446395273\\
418	0.00605555434167216\\
419	0.00605585257847013\\
420	0.00605615966795491\\
421	0.00605647616040805\\
422	0.00605680267260717\\
423	0.00605713990129653\\
424	0.00605748864427628\\
425	0.00605784983945151\\
426	0.00605822464906579\\
427	0.00605861466002468\\
428	0.00605902237900171\\
429	0.00605945244313382\\
430	0.00605991439894868\\
431	0.00606042809967847\\
432	0.00606102918372906\\
433	0.00606174518355961\\
434	0.00606247727923921\\
435	0.00606322597220445\\
436	0.00606399178991594\\
437	0.00606477528778837\\
438	0.00606557705128793\\
439	0.00606639769823559\\
440	0.00606723788140878\\
441	0.00606809829169523\\
442	0.00606897966253552\\
443	0.00606988277785511\\
444	0.00607080849019261\\
445	0.00607175776981034\\
446	0.0060727318503275\\
447	0.00607373268129346\\
448	0.00607476437618227\\
449	0.00607583795301545\\
450	0.00607698718544002\\
451	0.00607832269276952\\
452	0.00608022018707292\\
453	0.0060839880453354\\
454	0.00609427677510399\\
455	0.00610602598612627\\
456	0.00611801045428564\\
457	0.00613023305681882\\
458	0.00614269683042821\\
459	0.00615540509118221\\
460	0.00616836159537399\\
461	0.0061815707187161\\
462	0.00619503751751249\\
463	0.00620876727626111\\
464	0.00622276419136081\\
465	0.00623703469742191\\
466	0.00625158713579219\\
467	0.00626643030583339\\
468	0.00628157302414865\\
469	0.00629702404398672\\
470	0.00631279317771494\\
471	0.00632889036639022\\
472	0.00634532552123743\\
473	0.00636210877493495\\
474	0.00637925218463179\\
475	0.0063967732779601\\
476	0.00641468508647931\\
477	0.00643300146736872\\
478	0.00645173718233791\\
479	0.00647090797796981\\
480	0.0064905306751564\\
481	0.00651062326907722\\
482	0.00653120504144705\\
483	0.00655229668708556\\
484	0.00657392045725157\\
485	0.00659610032266658\\
486	0.00661886215977514\\
487	0.00664223396456578\\
488	0.00666624609900567\\
489	0.0066909315760426\\
490	0.00671632639188002\\
491	0.00674246991249381\\
492	0.00676940532775348\\
493	0.0067971801847125\\
494	0.00682584702209267\\
495	0.00685546413475298\\
496	0.0068860965102351\\
497	0.00691781705813337\\
498	0.00695070840770705\\
499	0.00698486600954099\\
500	0.00702040460012362\\
501	0.00705747392433718\\
502	0.0070963007897498\\
503	0.0071373072041076\\
504	0.0071814500473006\\
505	0.00723162610810303\\
506	0.00729745314543209\\
507	0.00736801731413665\\
508	0.00743995105733167\\
509	0.00751337158735593\\
510	0.00758861983498507\\
511	0.00766578718907356\\
512	0.00774497343798484\\
513	0.00782628766354034\\
514	0.00790984955612919\\
515	0.00799579078492592\\
516	0.00808425609444664\\
517	0.00817540328038095\\
518	0.00826939821203734\\
519	0.00836639020145198\\
520	0.00846641204396294\\
521	0.00856899519194826\\
522	0.00867171404819373\\
523	0.0087647045328796\\
524	0.00884301530583041\\
525	0.00892393530751383\\
526	0.00900765861368781\\
527	0.00909442832702381\\
528	0.00918460888670943\\
529	0.00927892176734096\\
530	0.00937920083063166\\
531	0.00949088647832879\\
532	0.00962035111008971\\
533	0.0097474577377326\\
534	0.0098590903546554\\
535	0.00994932493360521\\
536	0.0100416125047448\\
537	0.0101361613614959\\
538	0.010217173924014\\
539	0.010288560840392\\
540	0.0103617101615038\\
541	0.0104366645799178\\
542	0.0105134646396043\\
543	0.0105921497950417\\
544	0.0106727647579807\\
545	0.0107553072623498\\
546	0.0108396471253698\\
547	0.0109256324321949\\
548	0.0110136279709086\\
549	0.0111040797378401\\
550	0.0111938998042921\\
551	0.0112799328341239\\
552	0.0113737245489534\\
553	0.0114714566879937\\
554	0.0115689818513626\\
555	0.0116584763545022\\
556	0.0117397844527682\\
557	0.0118200973228264\\
558	0.0118996361358905\\
559	0.0119781018230238\\
560	0.0120544708172926\\
561	0.0121231089225033\\
562	0.0121844999004352\\
563	0.0122397314293899\\
564	0.0122953412749118\\
565	0.0123511564035504\\
566	0.0124050036232864\\
567	0.0124576550265305\\
568	0.0125113556956924\\
569	0.012570391603119\\
570	0.0126387511495328\\
571	0.0127067897504844\\
572	0.0127746116796269\\
573	0.012838925819619\\
574	0.0129017946708741\\
575	0.0129608236903396\\
576	0.0130103047378455\\
577	0.0130582618703364\\
578	0.0131044278302121\\
579	0.0131492299830451\\
580	0.0131922728506593\\
581	0.0132325781836974\\
582	0.0132679389451329\\
583	0.0133018403599666\\
584	0.0133343810661122\\
585	0.0133644452420943\\
586	0.0133926358436795\\
587	0.0134186816413615\\
588	0.0134418521091651\\
589	0.0134628106883845\\
590	0.0134816861851665\\
591	0.0134994363358936\\
592	0.0135162644577772\\
593	0.0135324897661906\\
594	0.0135486051791571\\
595	0.0135654641563757\\
596	0.0135851815686143\\
597	0.0136131889078992\\
598	0.0136637438596031\\
599	0\\
600	0\\
};
\addplot [color=mycolor17,solid,forget plot]
  table[row sep=crcr]{%
1	0.00693592557646952\\
2	0.0069359287400365\\
3	0.00693593196017262\\
4	0.00693593523788794\\
5	0.00693593857421057\\
6	0.00693594197018693\\
7	0.00693594542688211\\
8	0.00693594894538016\\
9	0.00693595252678446\\
10	0.0069359561722181\\
11	0.00693595988282414\\
12	0.00693596365976603\\
13	0.00693596750422798\\
14	0.00693597141741529\\
15	0.00693597540055474\\
16	0.00693597945489498\\
17	0.00693598358170693\\
18	0.00693598778228414\\
19	0.00693599205794324\\
20	0.00693599641002429\\
21	0.00693600083989127\\
22	0.00693600534893245\\
23	0.00693600993856085\\
24	0.00693601461021466\\
25	0.0069360193653577\\
26	0.0069360242054799\\
27	0.00693602913209772\\
28	0.00693603414675466\\
29	0.0069360392510217\\
30	0.00693604444649783\\
31	0.00693604973481054\\
32	0.00693605511761632\\
33	0.00693606059660116\\
34	0.00693606617348111\\
35	0.00693607185000279\\
36	0.00693607762794394\\
37	0.00693608350911399\\
38	0.00693608949535461\\
39	0.00693609558854026\\
40	0.00693610179057883\\
41	0.00693610810341218\\
42	0.00693611452901679\\
43	0.00693612106940435\\
44	0.00693612772662238\\
45	0.00693613450275491\\
46	0.00693614139992307\\
47	0.00693614842028581\\
48	0.00693615556604053\\
49	0.0069361628394238\\
50	0.00693617024271202\\
51	0.00693617777822215\\
52	0.00693618544831245\\
53	0.0069361932553832\\
54	0.00693620120187743\\
55	0.0069362092902817\\
56	0.0069362175231269\\
57	0.00693622590298898\\
58	0.00693623443248983\\
59	0.00693624311429801\\
60	0.00693625195112968\\
61	0.00693626094574937\\
62	0.00693627010097087\\
63	0.00693627941965812\\
64	0.0069362889047261\\
65	0.00693629855914172\\
66	0.00693630838592476\\
67	0.00693631838814885\\
68	0.00693632856894234\\
69	0.00693633893148938\\
70	0.00693634947903084\\
71	0.00693636021486536\\
72	0.00693637114235037\\
73	0.00693638226490314\\
74	0.00693639358600186\\
75	0.0069364051091867\\
76	0.00693641683806093\\
77	0.00693642877629208\\
78	0.00693644092761304\\
79	0.00693645329582324\\
80	0.00693646588478985\\
81	0.00693647869844899\\
82	0.00693649174080697\\
83	0.0069365050159415\\
84	0.00693651852800304\\
85	0.00693653228121604\\
86	0.00693654627988026\\
87	0.00693656052837221\\
88	0.00693657503114637\\
89	0.00693658979273672\\
90	0.0069366048177581\\
91	0.00693662011090768\\
92	0.00693663567696641\\
93	0.00693665152080054\\
94	0.00693666764736313\\
95	0.00693668406169562\\
96	0.00693670076892944\\
97	0.00693671777428754\\
98	0.00693673508308614\\
99	0.00693675270073629\\
100	0.00693677063274568\\
101	0.00693678888472027\\
102	0.00693680746236614\\
103	0.00693682637149123\\
104	0.0069368456180072\\
105	0.00693686520793127\\
106	0.00693688514738815\\
107	0.0069369054426119\\
108	0.006936926099948\\
109	0.00693694712585525\\
110	0.00693696852690788\\
111	0.00693699030979759\\
112	0.00693701248133567\\
113	0.00693703504845516\\
114	0.00693705801821303\\
115	0.00693708139779242\\
116	0.00693710519450491\\
117	0.00693712941579284\\
118	0.00693715406923166\\
119	0.00693717916253233\\
120	0.00693720470354375\\
121	0.00693723070025531\\
122	0.00693725716079934\\
123	0.00693728409345375\\
124	0.00693731150664463\\
125	0.00693733940894895\\
126	0.0069373678090973\\
127	0.0069373967159766\\
128	0.006937426138633\\
129	0.00693745608627473\\
130	0.00693748656827506\\
131	0.00693751759417525\\
132	0.00693754917368762\\
133	0.00693758131669867\\
134	0.00693761403327218\\
135	0.0069376473336525\\
136	0.00693768122826777\\
137	0.00693771572773331\\
138	0.00693775084285496\\
139	0.0069377865846326\\
140	0.00693782296426367\\
141	0.00693785999314673\\
142	0.00693789768288517\\
143	0.00693793604529091\\
144	0.00693797509238822\\
145	0.0069380148364176\\
146	0.00693805528983968\\
147	0.0069380964653393\\
148	0.00693813837582959\\
149	0.00693818103445611\\
150	0.00693822445460114\\
151	0.00693826864988801\\
152	0.00693831363418551\\
153	0.00693835942161236\\
154	0.00693840602654186\\
155	0.00693845346360649\\
156	0.00693850174770273\\
157	0.00693855089399587\\
158	0.00693860091792498\\
159	0.00693865183520793\\
160	0.00693870366184653\\
161	0.00693875641413177\\
162	0.00693881010864915\\
163	0.00693886476228411\\
164	0.00693892039222757\\
165	0.0069389770159816\\
166	0.00693903465136513\\
167	0.00693909331651986\\
168	0.00693915302991621\\
169	0.00693921381035942\\
170	0.0069392756769958\\
171	0.00693933864931899\\
172	0.0069394027471765\\
173	0.00693946799077621\\
174	0.00693953440069316\\
175	0.00693960199787633\\
176	0.00693967080365563\\
177	0.00693974083974905\\
178	0.00693981212826983\\
179	0.00693988469173393\\
180	0.00693995855306753\\
181	0.00694003373561471\\
182	0.00694011026314531\\
183	0.00694018815986288\\
184	0.00694026745041289\\
185	0.00694034815989098\\
186	0.00694043031385146\\
187	0.00694051393831594\\
188	0.00694059905978218\\
189	0.00694068570523302\\
190	0.00694077390214557\\
191	0.0069408636785006\\
192	0.006940955062792\\
193	0.00694104808403656\\
194	0.00694114277178389\\
195	0.00694123915612652\\
196	0.0069413372677102\\
197	0.00694143713774449\\
198	0.00694153879801346\\
199	0.00694164228088665\\
200	0.00694174761933026\\
201	0.00694185484691855\\
202	0.00694196399784546\\
203	0.00694207510693652\\
204	0.00694218820966091\\
205	0.00694230334214386\\
206	0.00694242054117922\\
207	0.00694253984424234\\
208	0.00694266128950323\\
209	0.00694278491583989\\
210	0.00694291076285202\\
211	0.00694303887087494\\
212	0.00694316928099387\\
213	0.00694330203505838\\
214	0.00694343717569727\\
215	0.00694357474633366\\
216	0.00694371479120044\\
217	0.00694385735535603\\
218	0.00694400248470044\\
219	0.00694415022599167\\
220	0.00694430062686247\\
221	0.00694445373583744\\
222	0.00694460960235046\\
223	0.00694476827676248\\
224	0.00694492981037971\\
225	0.00694509425547217\\
226	0.00694526166529259\\
227	0.00694543209409576\\
228	0.00694560559715825\\
229	0.00694578223079852\\
230	0.00694596205239747\\
231	0.00694614512041946\\
232	0.00694633149443361\\
233	0.00694652123513576\\
234	0.00694671440437071\\
235	0.00694691106515499\\
236	0.00694711128170008\\
237	0.00694731511943619\\
238	0.00694752264503637\\
239	0.00694773392644127\\
240	0.00694794903288435\\
241	0.00694816803491759\\
242	0.00694839100443779\\
243	0.00694861801471333\\
244	0.00694884914041155\\
245	0.00694908445762666\\
246	0.00694932404390824\\
247	0.00694956797829029\\
248	0.00694981634132092\\
249	0.00695006921509256\\
250	0.00695032668327294\\
251	0.00695058883113651\\
252	0.00695085574559664\\
253	0.00695112751523838\\
254	0.00695140423035197\\
255	0.00695168598296691\\
256	0.0069519728668868\\
257	0.00695226497772482\\
258	0.00695256241293997\\
259	0.00695286527187397\\
260	0.00695317365578888\\
261	0.00695348766790555\\
262	0.00695380741344268\\
263	0.00695413299965677\\
264	0.00695446453588274\\
265	0.00695480213357536\\
266	0.00695514590635153\\
267	0.00695549597003322\\
268	0.00695585244269139\\
269	0.00695621544469057\\
270	0.00695658509873436\\
271	0.00695696152991174\\
272	0.00695734486574422\\
273	0.00695773523623382\\
274	0.00695813277391196\\
275	0.00695853761388915\\
276	0.00695894989390558\\
277	0.00695936975438259\\
278	0.00695979733847502\\
279	0.00696023279212441\\
280	0.00696067626411313\\
281	0.00696112790611937\\
282	0.00696158787277305\\
283	0.00696205632171255\\
284	0.00696253341364248\\
285	0.00696301931239213\\
286	0.00696351418497501\\
287	0.00696401820164918\\
288	0.00696453153597842\\
289	0.00696505436489442\\
290	0.00696558686875968\\
291	0.00696612923143141\\
292	0.00696668164032621\\
293	0.00696724428648563\\
294	0.00696781736464258\\
295	0.00696840107328853\\
296	0.00696899561474163\\
297	0.00696960119521547\\
298	0.00697021802488886\\
299	0.00697084631797616\\
300	0.0069714862927986\\
301	0.00697213817185618\\
302	0.00697280218190045\\
303	0.00697347855400793\\
304	0.0069741675236543\\
305	0.00697486933078933\\
306	0.00697558421991238\\
307	0.00697631244014876\\
308	0.00697705424532668\\
309	0.00697780989405484\\
310	0.00697857964980084\\
311	0.00697936378097011\\
312	0.00698016256098566\\
313	0.00698097626836843\\
314	0.00698180518681845\\
315	0.00698264960529662\\
316	0.00698350981810742\\
317	0.0069843861249823\\
318	0.00698527883116396\\
319	0.00698618824749163\\
320	0.00698711469048721\\
321	0.00698805848244262\\
322	0.00698901995150818\\
323	0.00698999943178241\\
324	0.00699099726340311\\
325	0.00699201379264004\\
326	0.00699304937198935\\
327	0.00699410436026977\\
328	0.00699517912272096\\
329	0.0069962740311041\\
330	0.00699738946380493\\
331	0.0069985258059396\\
332	0.00699968344946339\\
333	0.0070008627932827\\
334	0.00700206424337056\\
335	0.00700328821288577\\
336	0.00700453512229612\\
337	0.00700580539950579\\
338	0.00700709947998727\\
339	0.00700841780691788\\
340	0.00700976083132122\\
341	0.00701112901221367\\
342	0.00701252281675612\\
343	0.00701394272041125\\
344	0.00701538920710653\\
345	0.00701686276940347\\
346	0.00701836390867339\\
347	0.00701989313528041\\
348	0.00702145096877222\\
349	0.00702303793807945\\
350	0.00702465458172487\\
351	0.00702630144804328\\
352	0.00702797909541394\\
353	0.00702968809250679\\
354	0.00703142901854438\\
355	0.00703320246358087\\
356	0.00703500902879987\\
357	0.00703684932683287\\
358	0.00703872398210015\\
359	0.00704063363117637\\
360	0.00704257892318313\\
361	0.00704456052021087\\
362	0.0070465790977729\\
363	0.0070486353452944\\
364	0.00705072996663958\\
365	0.00705286368068034\\
366	0.00705503722191009\\
367	0.00705725134110685\\
368	0.00705950680604973\\
369	0.00706180440229369\\
370	0.00706414493400754\\
371	0.00706652922488084\\
372	0.00706895811910579\\
373	0.00707143248244078\\
374	0.00707395320336307\\
375	0.0070765211943188\\
376	0.00707913739307948\\
377	0.00708180276421545\\
378	0.00708451830069771\\
379	0.00708728502564173\\
380	0.00709010399420837\\
381	0.00709297629567991\\
382	0.00709590305573201\\
383	0.00709888543892724\\
384	0.00710192465146155\\
385	0.00710502194420596\\
386	0.00710817861610449\\
387	0.00711139601802771\\
388	0.00711467555726204\\
389	0.00711801870297982\\
390	0.00712142699334772\\
391	0.00712490204539648\\
392	0.00712844556900577\\
393	0.00713205938442223\\
394	0.00713574543265157\\
395	0.00713950574021029\\
396	0.0071433422876819\\
397	0.00714725718194879\\
398	0.00715125278070888\\
399	0.00715533157393911\\
400	0.00715949619206064\\
401	0.0071637494130185\\
402	0.00716809416653157\\
403	0.00717253353219982\\
404	0.00717707072606006\\
405	0.00718170906979993\\
406	0.00718645194749512\\
407	0.00719130280262168\\
408	0.00719626535179438\\
409	0.00720134425810756\\
410	0.00720654488150459\\
411	0.00721187189326634\\
412	0.00721733031683269\\
413	0.00722292556988984\\
414	0.00722866351273491\\
415	0.0072345505039745\\
416	0.00724059346482704\\
417	0.00724679995354541\\
418	0.0072531782517778\\
419	0.00725973746505873\\
420	0.00726648764010154\\
421	0.00727343990221666\\
422	0.00728060661716942\\
423	0.00728800158358886\\
424	0.00729564026601822\\
425	0.00730354008922147\\
426	0.00731172084526395\\
427	0.00732020536153701\\
428	0.00732902089226944\\
429	0.00733820274777266\\
430	0.00734780527488531\\
431	0.00735793786200966\\
432	0.00736888822247437\\
433	0.0073815558343386\\
434	0.00739788986943627\\
435	0.00741461679079328\\
436	0.007431749828804\\
437	0.00744930292834301\\
438	0.00746729080307535\\
439	0.00748572899444429\\
440	0.00750463393570512\\
441	0.00752402302133271\\
442	0.00754391468193645\\
443	0.00756432846410616\\
444	0.00758528511215494\\
445	0.00760680664026718\\
446	0.00762891635426835\\
447	0.00765163867967215\\
448	0.00767499828909532\\
449	0.00769901671661117\\
450	0.0077236998978746\\
451	0.00774899258502784\\
452	0.00777461041323596\\
453	0.00779941497507422\\
454	0.00781906490806405\\
455	0.0078380894044167\\
456	0.00785760467298398\\
457	0.00787762175136461\\
458	0.00789815064129321\\
459	0.00791920016914555\\
460	0.00794077818616062\\
461	0.00796289261791559\\
462	0.00798555415032497\\
463	0.00800878031620292\\
464	0.00803259127499823\\
465	0.00805692439937071\\
466	0.00808175183128079\\
467	0.00810709293604959\\
468	0.00813297245462707\\
469	0.0081594042066264\\
470	0.00818637430444169\\
471	0.00821389267775709\\
472	0.00824196176545622\\
473	0.00827056760765769\\
474	0.00829966997130377\\
475	0.00832922668369863\\
476	0.00835944291153246\\
477	0.00839033815308526\\
478	0.00842193264304755\\
479	0.00845424765324777\\
480	0.00848730556827345\\
481	0.00852112996514715\\
482	0.00855574569694087\\
483	0.00859117898018266\\
484	0.00862745748583628\\
485	0.00866461043347\\
486	0.00870266868807612\\
487	0.00874166485959422\\
488	0.00878163340811694\\
489	0.00882261075596682\\
490	0.00886463539199106\\
491	0.00890774802683636\\
492	0.0089519917230595\\
493	0.00899741207554902\\
494	0.0090440573609577\\
495	0.00909197875295658\\
496	0.00914123076781363\\
497	0.0091918716796119\\
498	0.00924396430635652\\
499	0.00929757789943002\\
500	0.00935279335520063\\
501	0.00940971848270948\\
502	0.00946853384295804\\
503	0.00952963197949118\\
504	0.00959404354820472\\
505	0.00965937471217319\\
506	0.00971741718170556\\
507	0.00978100927947546\\
508	0.00984267120681476\\
509	0.00989645651890556\\
510	0.00994022900582074\\
511	0.0099850493671466\\
512	0.0100309357080922\\
513	0.0100779056943849\\
514	0.0101259711639922\\
515	0.0101751378927862\\
516	0.0102254066827219\\
517	0.0102767721325311\\
518	0.0103292205175522\\
519	0.0103827246591832\\
520	0.0104372277698731\\
521	0.0104925859218834\\
522	0.0105483541263133\\
523	0.0106031743169032\\
524	0.0106565553941962\\
525	0.0107108900777372\\
526	0.0107661194362737\\
527	0.0108222296441962\\
528	0.0108791887008099\\
529	0.0109369063091805\\
530	0.0109954246208243\\
531	0.0110554210458185\\
532	0.0111072387045219\\
533	0.0111592241307219\\
534	0.0112089994768058\\
535	0.0112555145534793\\
536	0.011302783334635\\
537	0.0113507882654209\\
538	0.0113970561054866\\
539	0.0114486499285055\\
540	0.0115050149328621\\
541	0.0115617821273209\\
542	0.0116188618623749\\
543	0.0116761532117478\\
544	0.0117335457566282\\
545	0.0117908775216927\\
546	0.0118479462871987\\
547	0.0119043952746839\\
548	0.0119598423665208\\
549	0.0120144037009114\\
550	0.0120672100335883\\
551	0.0121100067772152\\
552	0.0121481372446998\\
553	0.0121867398580826\\
554	0.0122257233689272\\
555	0.01226498370261\\
556	0.0123046282813976\\
557	0.0123445883022754\\
558	0.01238472481417\\
559	0.0124249698088742\\
560	0.0124653477302295\\
561	0.012505980730811\\
562	0.012545856282539\\
563	0.0125877993334322\\
564	0.0126437455304109\\
565	0.0126995253808788\\
566	0.0127551073637504\\
567	0.0128104401919974\\
568	0.0128653577953402\\
569	0.0129167710847116\\
570	0.0129594821584043\\
571	0.0130006793970259\\
572	0.0130396481804786\\
573	0.0130771861975367\\
574	0.0131132404057498\\
575	0.01314729680226\\
576	0.0131788280122701\\
577	0.0132093634648367\\
578	0.0132388265386339\\
579	0.0132668813072085\\
580	0.0132929590156252\\
581	0.0133185923536109\\
582	0.0133428334087323\\
583	0.0133656796464596\\
584	0.013386746951242\\
585	0.0134069339834939\\
586	0.0134258963353013\\
587	0.0134430288053913\\
588	0.0134592059786922\\
589	0.0134745052296271\\
590	0.013489420017042\\
591	0.013504239290221\\
592	0.0135190260558172\\
593	0.0135339097116538\\
594	0.0135491860906086\\
595	0.0135656125153459\\
596	0.0135851815686143\\
597	0.0136131889078992\\
598	0.0136637438596031\\
599	0\\
600	0\\
};
\addplot [color=mycolor18,solid,forget plot]
  table[row sep=crcr]{%
1	0.00840063842684203\\
2	0.00840064410390609\\
3	0.00840064988251434\\
4	0.00840065576448053\\
5	0.00840066175165082\\
6	0.00840066784590424\\
7	0.00840067404915337\\
8	0.00840068036334488\\
9	0.00840068679046016\\
10	0.00840069333251595\\
11	0.00840069999156492\\
12	0.00840070676969638\\
13	0.00840071366903687\\
14	0.00840072069175086\\
15	0.00840072784004142\\
16	0.00840073511615089\\
17	0.0084007425223616\\
18	0.00840075006099658\\
19	0.00840075773442027\\
20	0.00840076554503926\\
21	0.00840077349530306\\
22	0.00840078158770484\\
23	0.00840078982478224\\
24	0.00840079820911811\\
25	0.00840080674334138\\
26	0.00840081543012781\\
27	0.0084008242722009\\
28	0.00840083327233265\\
29	0.00840084243334454\\
30	0.00840085175810829\\
31	0.00840086124954684\\
32	0.00840087091063522\\
33	0.0084008807444015\\
34	0.00840089075392773\\
35	0.00840090094235089\\
36	0.00840091131286386\\
37	0.00840092186871645\\
38	0.0084009326132164\\
39	0.00840094354973039\\
40	0.00840095468168513\\
41	0.00840096601256838\\
42	0.00840097754593008\\
43	0.00840098928538343\\
44	0.00840100123460604\\
45	0.00840101339734108\\
46	0.00840102577739839\\
47	0.00840103837865576\\
48	0.00840105120506006\\
49	0.00840106426062852\\
50	0.00840107754944997\\
51	0.0084010910756861\\
52	0.00840110484357278\\
53	0.00840111885742137\\
54	0.00840113312162008\\
55	0.00840114764063532\\
56	0.00840116241901309\\
57	0.00840117746138044\\
58	0.00840119277244687\\
59	0.00840120835700582\\
60	0.00840122421993616\\
61	0.0084012403662037\\
62	0.0084012568008628\\
63	0.00840127352905786\\
64	0.008401290556025\\
65	0.00840130788709364\\
66	0.00840132552768823\\
67	0.00840134348332986\\
68	0.00840136175963806\\
69	0.00840138036233252\\
70	0.00840139929723488\\
71	0.00840141857027055\\
72	0.00840143818747056\\
73	0.00840145815497346\\
74	0.00840147847902722\\
75	0.0084014991659912\\
76	0.00840152022233811\\
77	0.00840154165465608\\
78	0.00840156346965066\\
79	0.00840158567414699\\
80	0.00840160827509186\\
81	0.00840163127955596\\
82	0.008401654694736\\
83	0.00840167852795706\\
84	0.00840170278667482\\
85	0.00840172747847792\\
86	0.00840175261109031\\
87	0.00840177819237371\\
88	0.00840180423033004\\
89	0.00840183073310394\\
90	0.00840185770898531\\
91	0.00840188516641195\\
92	0.00840191311397213\\
93	0.00840194156040735\\
94	0.00840197051461507\\
95	0.00840199998565147\\
96	0.00840202998273433\\
97	0.00840206051524588\\
98	0.0084020915927358\\
99	0.00840212322492417\\
100	0.00840215542170455\\
101	0.00840218819314707\\
102	0.00840222154950163\\
103	0.00840225550120109\\
104	0.00840229005886453\\
105	0.00840232523330067\\
106	0.00840236103551117\\
107	0.0084023974766942\\
108	0.00840243456824785\\
109	0.00840247232177382\\
110	0.00840251074908103\\
111	0.00840254986218933\\
112	0.00840258967333332\\
113	0.0084026301949662\\
114	0.00840267143976367\\
115	0.00840271342062798\\
116	0.00840275615069198\\
117	0.00840279964332326\\
118	0.00840284391212838\\
119	0.00840288897095719\\
120	0.00840293483390717\\
121	0.00840298151532795\\
122	0.0084030290298258\\
123	0.00840307739226826\\
124	0.00840312661778891\\
125	0.00840317672179209\\
126	0.00840322771995783\\
127	0.00840327962824682\\
128	0.00840333246290547\\
129	0.00840338624047108\\
130	0.0084034409777771\\
131	0.00840349669195845\\
132	0.00840355340045701\\
133	0.00840361112102718\\
134	0.00840366987174148\\
135	0.00840372967099638\\
136	0.00840379053751812\\
137	0.00840385249036871\\
138	0.00840391554895199\\
139	0.00840397973301986\\
140	0.00840404506267861\\
141	0.00840411155839529\\
142	0.00840417924100433\\
143	0.00840424813171417\\
144	0.0084043182521141\\
145	0.00840438962418116\\
146	0.0084044622702872\\
147	0.00840453621320608\\
148	0.00840461147612099\\
149	0.00840468808263192\\
150	0.00840476605676324\\
151	0.00840484542297146\\
152	0.00840492620615316\\
153	0.00840500843165293\\
154	0.00840509212527167\\
155	0.00840517731327487\\
156	0.00840526402240117\\
157	0.00840535227987093\\
158	0.00840544211339517\\
159	0.00840553355118452\\
160	0.00840562662195836\\
161	0.00840572135495422\\
162	0.00840581777993726\\
163	0.00840591592720998\\
164	0.00840601582762214\\
165	0.00840611751258079\\
166	0.00840622101406057\\
167	0.00840632636461419\\
168	0.00840643359738305\\
169	0.00840654274610817\\
170	0.00840665384514124\\
171	0.00840676692945591\\
172	0.00840688203465935\\
173	0.00840699919700393\\
174	0.00840711845339922\\
175	0.00840723984142419\\
176	0.00840736339933962\\
177	0.00840748916610079\\
178	0.00840761718137041\\
179	0.00840774748553177\\
180	0.00840788011970221\\
181	0.00840801512574676\\
182	0.00840815254629215\\
183	0.00840829242474103\\
184	0.00840843480528647\\
185	0.0084085797329268\\
186	0.00840872725348064\\
187	0.00840887741360238\\
188	0.00840903026079778\\
189	0.00840918584344005\\
190	0.00840934421078613\\
191	0.00840950541299334\\
192	0.0084096695011364\\
193	0.0084098365272247\\
194	0.00841000654421996\\
195	0.00841017960605429\\
196	0.00841035576764852\\
197	0.00841053508493098\\
198	0.0084107176148566\\
199	0.00841090341542639\\
200	0.00841109254570743\\
201	0.00841128506585303\\
202	0.00841148103712359\\
203	0.0084116805219076\\
204	0.00841188358374328\\
205	0.00841209028734056\\
206	0.00841230069860348\\
207	0.00841251488465316\\
208	0.00841273291385112\\
209	0.00841295485582314\\
210	0.00841318078148359\\
211	0.00841341076306026\\
212	0.00841364487411973\\
213	0.00841388318959322\\
214	0.00841412578580297\\
215	0.0084143727404892\\
216	0.00841462413283762\\
217	0.0084148800435075\\
218	0.00841514055466031\\
219	0.00841540574998895\\
220	0.00841567571474766\\
221	0.00841595053578246\\
222	0.00841623030156229\\
223	0.00841651510221073\\
224	0.00841680502953849\\
225	0.00841710017707649\\
226	0.00841740064010966\\
227	0.00841770651571147\\
228	0.00841801790277918\\
229	0.00841833490206982\\
230	0.00841865761623692\\
231	0.00841898614986805\\
232	0.00841932060952315\\
233	0.00841966110377364\\
234	0.00842000774324237\\
235	0.00842036064064445\\
236	0.00842071991082893\\
237	0.00842108567082134\\
238	0.00842145803986719\\
239	0.00842183713947632\\
240	0.00842222309346832\\
241	0.00842261602801875\\
242	0.00842301607170655\\
243	0.00842342335556227\\
244	0.00842383801311744\\
245	0.00842426018045505\\
246	0.00842468999626093\\
247	0.00842512760187642\\
248	0.00842557314135208\\
249	0.00842602676150256\\
250	0.00842648861196264\\
251	0.00842695884524453\\
252	0.0084274376167963\\
253	0.00842792508506169\\
254	0.00842842141154108\\
255	0.00842892676085386\\
256	0.00842944130080209\\
257	0.00842996520243556\\
258	0.0084304986401182\\
259	0.00843104179159596\\
260	0.00843159483806607\\
261	0.00843215796424791\\
262	0.00843273135845526\\
263	0.00843331521267014\\
264	0.00843390972261825\\
265	0.00843451508784597\\
266	0.00843513151179903\\
267	0.00843575920190283\\
268	0.00843639836964441\\
269	0.00843704923065627\\
270	0.00843771200480182\\
271	0.00843838691626268\\
272	0.00843907419362781\\
273	0.00843977406998446\\
274	0.00844048678301111\\
275	0.00844121257507213\\
276	0.00844195169331459\\
277	0.00844270438976698\\
278	0.00844347092143987\\
279	0.00844425155042873\\
280	0.00844504654401874\\
281	0.00844585617479173\\
282	0.00844668072073525\\
283	0.00844752046535381\\
284	0.00844837569778225\\
285	0.00844924671290141\\
286	0.00845013381145593\\
287	0.00845103730017439\\
288	0.00845195749189171\\
289	0.00845289470567383\\
290	0.00845384926694471\\
291	0.0084548215076157\\
292	0.00845581176621722\\
293	0.00845682038803285\\
294	0.00845784772523576\\
295	0.00845889413702753\\
296	0.00845995998977938\\
297	0.0084610456571758\\
298	0.00846215152036049\\
299	0.00846327796808484\\
300	0.00846442539685869\\
301	0.00846559421110346\\
302	0.00846678482330772\\
303	0.00846799765418506\\
304	0.0084692331328343\\
305	0.00847049169690202\\
306	0.00847177379274737\\
307	0.00847307987560912\\
308	0.00847441040977496\\
309	0.00847576586875299\\
310	0.0084771467354453\\
311	0.00847855350232376\\
312	0.0084799866716078\\
313	0.00848144675544426\\
314	0.00848293427608916\\
315	0.00848444976609151\\
316	0.00848599376847895\\
317	0.00848756683694522\\
318	0.00848916953603956\\
319	0.00849080244135781\\
320	0.00849246613973533\\
321	0.0084941612294417\\
322	0.00849588832037716\\
323	0.00849764803427094\\
324	0.00849944100488132\\
325	0.00850126787819771\\
326	0.00850312931264466\\
327	0.00850502597928813\\
328	0.00850695856204396\\
329	0.00850892775788904\\
330	0.00851093427707517\\
331	0.00851297884334615\\
332	0.00851506219415839\\
333	0.00851718508090553\\
334	0.00851934826914762\\
335	0.00852155253884539\\
336	0.00852379868460028\\
337	0.00852608751590099\\
338	0.00852841985737715\\
339	0.00853079654906097\\
340	0.00853321844665741\\
341	0.00853568642182322\\
342	0.00853820136245521\\
343	0.0085407641729874\\
344	0.00854337577469618\\
345	0.00854603710601338\\
346	0.00854874912284661\\
347	0.00855151279890608\\
348	0.00855432912603707\\
349	0.00855719911455682\\
350	0.00856012379359469\\
351	0.00856310421143466\\
352	0.00856614143585965\\
353	0.00856923655449866\\
354	0.00857239067517873\\
355	0.00857560492628687\\
356	0.00857888045714315\\
357	0.00858221843838626\\
358	0.00858562006237254\\
359	0.00858908654359013\\
360	0.00859261911908952\\
361	0.00859621904893227\\
362	0.0085998876166598\\
363	0.00860362612978406\\
364	0.00860743592030234\\
365	0.0086113183452385\\
366	0.00861527478721314\\
367	0.00861930665504517\\
368	0.00862341538438778\\
369	0.00862760243840159\\
370	0.0086318693084679\\
371	0.00863621751494532\\
372	0.0086406486079726\\
373	0.0086451641683208\\
374	0.00864976580829766\\
375	0.00865445517270669\\
376	0.00865923393986321\\
377	0.00866410382266882\\
378	0.00866906656974522\\
379	0.00867412396662695\\
380	0.00867927783701121\\
381	0.00868453004406164\\
382	0.00868988249176042\\
383	0.0086953371263025\\
384	0.00870089593752655\\
385	0.00870656096038763\\
386	0.0087123342765136\\
387	0.00871821801600059\\
388	0.00872421435993283\\
389	0.00873032554503007\\
390	0.00873655387430428\\
391	0.00874290174400483\\
392	0.00874937171252229\\
393	0.00875596666900957\\
394	0.00876269020304465\\
395	0.00876954720799981\\
396	0.00877654378363983\\
397	0.00878367967348816\\
398	0.00879095317877909\\
399	0.00879836745699547\\
400	0.00880592579819438\\
401	0.00881363162802031\\
402	0.00882148849594291\\
403	0.00882950002344964\\
404	0.00883766975101349\\
405	0.00884600074469163\\
406	0.00885449467954629\\
407	0.0088631499654199\\
408	0.00887195891188415\\
409	0.00888090810981988\\
410	0.00889000840579733\\
411	0.00889930190360543\\
412	0.00890879292868571\\
413	0.00891848583575452\\
414	0.00892838499058034\\
415	0.00893849474755592\\
416	0.00894881942191725\\
417	0.00895936325546501\\
418	0.00897013037439337\\
419	0.00898112473752426\\
420	0.00899235007286794\\
421	0.00900380979994188\\
422	0.00901550693478468\\
423	0.00902744397385639\\
424	0.00903962275202505\\
425	0.00905204426611349\\
426	0.0090647084425957\\
427	0.00907761377719326\\
428	0.00909075657317015\\
429	0.00910412872210131\\
430	0.00911770996208753\\
431	0.00913143910753164\\
432	0.00914510555251504\\
433	0.00915793959845932\\
434	0.00916812680014346\\
435	0.0091785293100627\\
436	0.00918915129984152\\
437	0.00919999693895049\\
438	0.00921107038370186\\
439	0.0092223757649605\\
440	0.0092339171744347\\
441	0.00924569864929151\\
442	0.00925772415444182\\
443	0.00926999756056873\\
444	0.00928252261203347\\
445	0.00929530286683722\\
446	0.00930834155523104\\
447	0.00932164120004579\\
448	0.00933520255013695\\
449	0.00934902159637897\\
450	0.0093630815196596\\
451	0.00937733253230136\\
452	0.00939164891864054\\
453	0.00940577529433277\\
454	0.00941948256359929\\
455	0.00943346910041966\\
456	0.00944791401931388\\
457	0.00946285406923779\\
458	0.00947833169538007\\
459	0.00949439676854473\\
460	0.00951111017649282\\
461	0.00952855289659365\\
462	0.00954685191139944\\
463	0.00956625932615199\\
464	0.00958740168856049\\
465	0.00961223868386329\\
466	0.00963907843714742\\
467	0.00966653251821401\\
468	0.00969476576736104\\
469	0.00972430675880319\\
470	0.00975555769263777\\
471	0.00978736612713335\\
472	0.00981965252879608\\
473	0.00985218440201735\\
474	0.00988430053592306\\
475	0.00991412733865065\\
476	0.00993633151347284\\
477	0.00995901323593493\\
478	0.00998219129664034\\
479	0.0100058744151527\\
480	0.0100300711817142\\
481	0.0100547900366138\\
482	0.010080039249981\\
483	0.010105826902284\\
484	0.0101321608658823\\
485	0.0101590487876677\\
486	0.0101864980703451\\
487	0.0102145158400948\\
488	0.0102431088657005\\
489	0.0102722834516399\\
490	0.0103020454023941\\
491	0.0103323989575677\\
492	0.010363347067156\\
493	0.0103948905245735\\
494	0.0104270282270168\\
495	0.0104597557159804\\
496	0.0104930617072032\\
497	0.0105269299339452\\
498	0.0105613382361768\\
499	0.0105962576073175\\
500	0.0106316514529724\\
501	0.0106674759130486\\
502	0.0107036839704322\\
503	0.0107402419634592\\
504	0.0107771858478174\\
505	0.010814432979152\\
506	0.0108497822256945\\
507	0.0108783361859115\\
508	0.0109068695375369\\
509	0.0109343329101461\\
510	0.0109602808304604\\
511	0.010986629916455\\
512	0.0110133632811433\\
513	0.0110404314592121\\
514	0.0110679381502852\\
515	0.0110959617571711\\
516	0.0111245164356795\\
517	0.01115361841351\\
518	0.0111832864721112\\
519	0.0112135424991692\\
520	0.0112444123127838\\
521	0.0112759268609692\\
522	0.0113081245732649\\
523	0.0113410870879542\\
524	0.0113749579651764\\
525	0.0114106852705433\\
526	0.0114541634769512\\
527	0.011497744098317\\
528	0.011541350761772\\
529	0.0115848799434719\\
530	0.0116282137951213\\
531	0.0116712427776204\\
532	0.0117120635829607\\
533	0.0117527503477585\\
534	0.0117934612746472\\
535	0.0118341183143062\\
536	0.0118746252819239\\
537	0.0119147515812094\\
538	0.0119543795365334\\
539	0.0119885467124485\\
540	0.0120191761477029\\
541	0.0120497131132624\\
542	0.0120800785188749\\
543	0.0121101868597433\\
544	0.0121399273081876\\
545	0.0121698618497962\\
546	0.0122000503549242\\
547	0.0122304275505205\\
548	0.0122609116642179\\
549	0.0122915131671246\\
550	0.0123221966405659\\
551	0.0123519173985611\\
552	0.0123814449602816\\
553	0.0124118819066886\\
554	0.012444036750918\\
555	0.0124777472735445\\
556	0.0125125640224238\\
557	0.0125486611876602\\
558	0.012593459146519\\
559	0.0126449216464747\\
560	0.0126961852118528\\
561	0.0127471637137742\\
562	0.0127978135291046\\
563	0.012847296247837\\
564	0.0128847976414583\\
565	0.012921448743291\\
566	0.0129570753468019\\
567	0.0129913280273469\\
568	0.013024002460064\\
569	0.0130534437148665\\
570	0.0130804277521042\\
571	0.0131081048327079\\
572	0.0131355149595239\\
573	0.0131621003091514\\
574	0.0131878670970299\\
575	0.0132118866214813\\
576	0.0132347030023971\\
577	0.0132576075025751\\
578	0.0132807780900296\\
579	0.0133029936974567\\
580	0.013323692266975\\
581	0.0133431253864752\\
582	0.0133625260871674\\
583	0.0133812244736883\\
584	0.0133986729773908\\
585	0.0134151985784402\\
586	0.0134311750708794\\
587	0.0134465365381261\\
588	0.0134614184599777\\
589	0.0134757971101556\\
590	0.0134901845159537\\
591	0.0135046456749847\\
592	0.0135192117584053\\
593	0.0135339752758402\\
594	0.0135491998980231\\
595	0.0135656125153459\\
596	0.0135851815686143\\
597	0.0136131889078992\\
598	0.0136637438596031\\
599	0\\
600	0\\
};
\addplot [color=red!25!mycolor17,solid,forget plot]
  table[row sep=crcr]{%
1	0.00940022538022224\\
2	0.00940023166219765\\
3	0.00940023805660001\\
4	0.00940024456543877\\
5	0.00940025119075924\\
6	0.00940025793464325\\
7	0.00940026479920974\\
8	0.00940027178661551\\
9	0.00940027889905582\\
10	0.00940028613876511\\
11	0.0094002935080177\\
12	0.00940030100912851\\
13	0.0094003086444537\\
14	0.00940031641639155\\
15	0.00940032432738308\\
16	0.00940033237991287\\
17	0.00940034057650985\\
18	0.00940034891974806\\
19	0.00940035741224744\\
20	0.00940036605667471\\
21	0.00940037485574413\\
22	0.0094003838122184\\
23	0.00940039292890949\\
24	0.00940040220867954\\
25	0.00940041165444174\\
26	0.00940042126916125\\
27	0.00940043105585614\\
28	0.00940044101759826\\
29	0.0094004511575143\\
30	0.00940046147878669\\
31	0.00940047198465464\\
32	0.00940048267841513\\
33	0.00940049356342391\\
34	0.00940050464309662\\
35	0.0094005159209098\\
36	0.00940052740040198\\
37	0.00940053908517481\\
38	0.00940055097889415\\
39	0.00940056308529126\\
40	0.00940057540816391\\
41	0.0094005879513776\\
42	0.00940060071886677\\
43	0.009400613714636\\
44	0.00940062694276129\\
45	0.00940064040739131\\
46	0.0094006541127487\\
47	0.00940066806313141\\
48	0.009400682262914\\
49	0.00940069671654903\\
50	0.00940071142856845\\
51	0.00940072640358503\\
52	0.00940074164629374\\
53	0.0094007571614733\\
54	0.00940077295398759\\
55	0.00940078902878723\\
56	0.00940080539091106\\
57	0.00940082204548781\\
58	0.0094008389977376\\
59	0.00940085625297363\\
60	0.0094008738166038\\
61	0.00940089169413245\\
62	0.00940090989116201\\
63	0.0094009284133948\\
64	0.0094009472666348\\
65	0.00940096645678943\\
66	0.00940098598987146\\
67	0.0094010058720008\\
68	0.0094010261094065\\
69	0.0094010467084286\\
70	0.0094010676755202\\
71	0.00940108901724943\\
72	0.00940111074030148\\
73	0.00940113285148074\\
74	0.00940115535771287\\
75	0.009401178266047\\
76	0.00940120158365793\\
77	0.00940122531784834\\
78	0.00940124947605111\\
79	0.0094012740658316\\
80	0.00940129909489005\\
81	0.00940132457106398\\
82	0.00940135050233062\\
83	0.00940137689680938\\
84	0.00940140376276447\\
85	0.0094014311086074\\
86	0.00940145894289964\\
87	0.00940148727435528\\
88	0.00940151611184381\\
89	0.00940154546439281\\
90	0.00940157534119082\\
91	0.00940160575159021\\
92	0.00940163670511011\\
93	0.00940166821143935\\
94	0.00940170028043953\\
95	0.00940173292214809\\
96	0.00940176614678146\\
97	0.00940179996473822\\
98	0.00940183438660241\\
99	0.00940186942314679\\
100	0.00940190508533625\\
101	0.00940194138433124\\
102	0.00940197833149121\\
103	0.00940201593837826\\
104	0.00940205421676068\\
105	0.00940209317861668\\
106	0.00940213283613811\\
107	0.00940217320173434\\
108	0.00940221428803609\\
109	0.0094022561078994\\
110	0.00940229867440969\\
111	0.00940234200088585\\
112	0.00940238610088441\\
113	0.00940243098820379\\
114	0.00940247667688865\\
115	0.00940252318123428\\
116	0.00940257051579113\\
117	0.00940261869536929\\
118	0.00940266773504327\\
119	0.0094027176501566\\
120	0.00940276845632678\\
121	0.00940282016945009\\
122	0.00940287280570666\\
123	0.0094029263815655\\
124	0.00940298091378976\\
125	0.00940303641944192\\
126	0.00940309291588922\\
127	0.00940315042080912\\
128	0.00940320895219488\\
129	0.00940326852836117\\
130	0.00940332916794996\\
131	0.00940339088993629\\
132	0.00940345371363431\\
133	0.00940351765870336\\
134	0.00940358274515424\\
135	0.0094036489933554\\
136	0.00940371642403953\\
137	0.00940378505831001\\
138	0.00940385491764761\\
139	0.00940392602391735\\
140	0.00940399839937532\\
141	0.00940407206667579\\
142	0.00940414704887842\\
143	0.00940422336945549\\
144	0.00940430105229942\\
145	0.0094043801217303\\
146	0.00940446060250366\\
147	0.00940454251981829\\
148	0.00940462589932429\\
149	0.00940471076713122\\
150	0.00940479714981639\\
151	0.00940488507443333\\
152	0.00940497456852043\\
153	0.00940506566010971\\
154	0.00940515837773575\\
155	0.00940525275044483\\
156	0.00940534880780415\\
157	0.00940544657991136\\
158	0.00940554609740415\\
159	0.00940564739147002\\
160	0.00940575049385635\\
161	0.0094058554368805\\
162	0.00940596225344023\\
163	0.00940607097702423\\
164	0.00940618164172293\\
165	0.00940629428223936\\
166	0.00940640893390045\\
167	0.00940652563266829\\
168	0.0094066444151518\\
169	0.00940676531861851\\
170	0.00940688838100659\\
171	0.00940701364093713\\
172	0.0094071411377266\\
173	0.00940727091139963\\
174	0.00940740300270194\\
175	0.00940753745311358\\
176	0.00940767430486237\\
177	0.00940781360093767\\
178	0.00940795538510432\\
179	0.00940809970191697\\
180	0.00940824659673453\\
181	0.00940839611573497\\
182	0.00940854830593051\\
183	0.0094087032151829\\
184	0.00940886089221913\\
185	0.00940902138664743\\
186	0.00940918474897353\\
187	0.00940935103061727\\
188	0.00940952028392951\\
189	0.00940969256220941\\
190	0.009409867919722\\
191	0.0094100464117161\\
192	0.0094102280944426\\
193	0.00941041302517312\\
194	0.00941060126221898\\
195	0.00941079286495059\\
196	0.00941098789381717\\
197	0.00941118641036694\\
198	0.0094113884772676\\
199	0.00941159415832733\\
200	0.0094118035185161\\
201	0.00941201662398747\\
202	0.00941223354210079\\
203	0.00941245434144385\\
204	0.009412679091856\\
205	0.00941290786445171\\
206	0.00941314073164458\\
207	0.0094133777671719\\
208	0.00941361904611963\\
209	0.0094138646449479\\
210	0.0094141146415171\\
211	0.00941436911511435\\
212	0.00941462814648064\\
213	0.00941489181783848\\
214	0.00941516021292007\\
215	0.0094154334169961\\
216	0.00941571151690512\\
217	0.00941599460108349\\
218	0.00941628275959597\\
219	0.00941657608416693\\
220	0.00941687466821222\\
221	0.00941717860687167\\
222	0.00941748799704229\\
223	0.00941780293741213\\
224	0.00941812352849486\\
225	0.00941844987266513\\
226	0.00941878207419454\\
227	0.00941912023928849\\
228	0.00941946447612374\\
229	0.00941981489488676\\
230	0.00942017160781296\\
231	0.00942053472922665\\
232	0.0094209043755819\\
233	0.00942128066550433\\
234	0.00942166371983367\\
235	0.00942205366166733\\
236	0.00942245061640486\\
237	0.00942285471179342\\
238	0.00942326607797412\\
239	0.00942368484752953\\
240	0.00942411115553208\\
241	0.00942454513959357\\
242	0.00942498693991578\\
243	0.00942543669934217\\
244	0.00942589456341071\\
245	0.00942636068040789\\
246	0.00942683520142393\\
247	0.00942731828040921\\
248	0.0094278100742319\\
249	0.00942831074273702\\
250	0.00942882044880666\\
251	0.00942933935842168\\
252	0.00942986764072468\\
253	0.00943040546808454\\
254	0.0094309530161623\\
255	0.00943151046397858\\
256	0.00943207799398257\\
257	0.00943265579212256\\
258	0.00943324404791808\\
259	0.00943384295453378\\
260	0.0094344527088549\\
261	0.00943507351156458\\
262	0.00943570556722294\\
263	0.009436349084348\\
264	0.00943700427549848\\
265	0.00943767135735856\\
266	0.00943835055082463\\
267	0.00943904208109411\\
268	0.00943974617775633\\
269	0.00944046307488556\\
270	0.00944119301113633\\
271	0.00944193622984091\\
272	0.00944269297910922\\
273	0.00944346351193113\\
274	0.00944424808628109\\
275	0.00944504696522546\\
276	0.00944586041703234\\
277	0.009446688715284\\
278	0.00944753213899218\\
279	0.00944839097271602\\
280	0.00944926550668304\\
281	0.00945015603691288\\
282	0.00945106286534422\\
283	0.00945198629996467\\
284	0.00945292665494392\\
285	0.00945388425077006\\
286	0.00945485941438938\\
287	0.00945585247934935\\
288	0.00945686378594535\\
289	0.0094578936813708\\
290	0.00945894251987102\\
291	0.00946001066290084\\
292	0.00946109847928603\\
293	0.00946220634538859\\
294	0.00946333464527616\\
295	0.00946448377089538\\
296	0.00946565412224947\\
297	0.00946684610758005\\
298	0.00946806014355325\\
299	0.00946929665545019\\
300	0.009470556077362\\
301	0.00947183885238926\\
302	0.0094731454328461\\
303	0.00947447628046895\\
304	0.00947583186662984\\
305	0.00947721267255471\\
306	0.0094786191895462\\
307	0.00948005191921142\\
308	0.00948151137369446\\
309	0.00948299807591373\\
310	0.00948451255980408\\
311	0.00948605537056373\\
312	0.00948762706490587\\
313	0.00948922821131497\\
314	0.00949085939030769\\
315	0.00949252119469822\\
316	0.00949421422986801\\
317	0.00949593911403967\\
318	0.00949769647855494\\
319	0.00949948696815639\\
320	0.00950131124127278\\
321	0.00950316997030759\\
322	0.00950506384193064\\
323	0.00950699355737229\\
324	0.00950895983271991\\
325	0.0095109633992162\\
326	0.00951300500355895\\
327	0.00951508540820172\\
328	0.00951720539165498\\
329	0.00951936574878728\\
330	0.00952156729112584\\
331	0.00952381084715627\\
332	0.00952609726262079\\
333	0.00952842740081486\\
334	0.00953080214288188\\
335	0.00953322238810603\\
336	0.00953568905420335\\
337	0.00953820307761199\\
338	0.00954076541378234\\
339	0.00954337703746915\\
340	0.00954603894302809\\
341	0.00954875214472098\\
342	0.00955151767703414\\
343	0.00955433659501601\\
344	0.00955720997463941\\
345	0.00956013891318104\\
346	0.00956312452962018\\
347	0.00956616796505788\\
348	0.00956927038315733\\
349	0.00957243297060424\\
350	0.0095756569375835\\
351	0.00957894351826476\\
352	0.00958229397128373\\
353	0.00958570958020108\\
354	0.00958919165392165\\
355	0.00959274152703758\\
356	0.00959636056015464\\
357	0.00960005014020471\\
358	0.00960381168074269\\
359	0.00960764662222614\\
360	0.00961155643227569\\
361	0.00961554260591438\\
362	0.0096196066657839\\
363	0.00962375016233579\\
364	0.00962797467399552\\
365	0.00963228180729747\\
366	0.00963667319698879\\
367	0.00964115050610019\\
368	0.00964571542598191\\
369	0.00965036967630297\\
370	0.00965511500501229\\
371	0.00965995318826026\\
372	0.00966488603027974\\
373	0.00966991536322567\\
374	0.00967504304697298\\
375	0.00968027096887291\\
376	0.00968560104346826\\
377	0.00969103521216887\\
378	0.00969657544288914\\
379	0.00970222372965012\\
380	0.00970798209214983\\
381	0.00971385257530607\\
382	0.00971983724877821\\
383	0.0097259382064777\\
384	0.00973215756608635\\
385	0.00973849746862825\\
386	0.00974496007822181\\
387	0.00975154758238821\\
388	0.00975826219407005\\
389	0.00976510615896127\\
390	0.00977208177951623\\
391	0.00977919149188917\\
392	0.00978643811257192\\
393	0.00979382563481065\\
394	0.00980136182558579\\
395	0.0098090667864548\\
396	0.00981700149961414\\
397	0.00982536414909979\\
398	0.00983408793016988\\
399	0.00984296899344703\\
400	0.0098520090471634\\
401	0.0098612096389363\\
402	0.00987057203340493\\
403	0.00988009690443604\\
404	0.00988978351906158\\
405	0.00989962751543673\\
406	0.00990961476531127\\
407	0.00991970426433803\\
408	0.00992978006952169\\
409	0.00993951524169936\\
410	0.00994798333454597\\
411	0.00995491838324685\\
412	0.00996198841372489\\
413	0.00996919548958295\\
414	0.00997654159946153\\
415	0.00998402863479882\\
416	0.00999165836872181\\
417	0.00999943243206585\\
418	0.0100073522861934\\
419	0.0100154191922237\\
420	0.0100236341762566\\
421	0.010031997990582\\
422	0.0100405110696584\\
423	0.0100491734805339\\
424	0.010057984862722\\
425	0.0100669443580834\\
426	0.0100760505258386\\
427	0.0100853012277024\\
428	0.010094693430481\\
429	0.0101042227332456\\
430	0.0101138819050338\\
431	0.0101236557866387\\
432	0.0101335027565515\\
433	0.01014329267964\\
434	0.010152780865034\\
435	0.0101624553039849\\
436	0.0101723188326044\\
437	0.0101823742111527\\
438	0.0101926241088854\\
439	0.0102030710875025\\
440	0.0102137175830946\\
441	0.0102245658864512\\
442	0.0102356181215214\\
443	0.0102468762216088\\
444	0.0102583419022861\\
445	0.0102700166282737\\
446	0.0102819015665088\\
447	0.010293997503506\\
448	0.0103063046670698\\
449	0.0103188222979053\\
450	0.010331547619344\\
451	0.0103444736083217\\
452	0.0103575857472083\\
453	0.0103708665543382\\
454	0.0103843659394169\\
455	0.0103980931536572\\
456	0.0104120452850369\\
457	0.010426218220351\\
458	0.0104406064484268\\
459	0.0104552028753707\\
460	0.010469998759181\\
461	0.010484984142475\\
462	0.0105001503655594\\
463	0.0105154974359771\\
464	0.0105310594028087\\
465	0.0105447224870884\\
466	0.0105574766869334\\
467	0.01057051836962\\
468	0.0105838793854706\\
469	0.0105976515894039\\
470	0.0106120051502292\\
471	0.0106269252441725\\
472	0.0106421295948054\\
473	0.0106575835354005\\
474	0.010673178062128\\
475	0.010688589946916\\
476	0.0107028779050528\\
477	0.0107174714743243\\
478	0.0107323758521718\\
479	0.0107475937145868\\
480	0.0107631268139347\\
481	0.0107789757985907\\
482	0.0107951400092252\\
483	0.0108116172528151\\
484	0.0108284035613643\\
485	0.0108454929568367\\
486	0.0108628772746906\\
487	0.0108805461402974\\
488	0.0108984871091588\\
489	0.0109166842737079\\
490	0.0109351163074438\\
491	0.0109537789935243\\
492	0.0109726465558139\\
493	0.0109917036341972\\
494	0.0110109171897054\\
495	0.0110302962362289\\
496	0.0110499300266453\\
497	0.0110698192119407\\
498	0.0110899651617066\\
499	0.0111103702428086\\
500	0.0111310381534936\\
501	0.0111519743023774\\
502	0.0111731861854438\\
503	0.0111946836310753\\
504	0.011216478645776\\
505	0.011238584689142\\
506	0.0112609591832853\\
507	0.0112826555779167\\
508	0.0113052072521908\\
509	0.0113291260037237\\
510	0.011359324669026\\
511	0.0113898550929772\\
512	0.0114207002532744\\
513	0.01145183489198\\
514	0.0114832570242686\\
515	0.0115149558859323\\
516	0.0115469050018112\\
517	0.0115790734695417\\
518	0.0116114261203464\\
519	0.011643923846543\\
520	0.0116765196739858\\
521	0.0117091586031359\\
522	0.0117417770214818\\
523	0.0117742984461677\\
524	0.0118066204743017\\
525	0.0118378964999197\\
526	0.0118629478512975\\
527	0.0118879336579292\\
528	0.0119127687895923\\
529	0.0119375126214545\\
530	0.0119621857565344\\
531	0.0119867611603447\\
532	0.0120112681719466\\
533	0.0120356795951325\\
534	0.0120599588300948\\
535	0.0120840468411578\\
536	0.0121078781420597\\
537	0.0121313596779242\\
538	0.0121543944822357\\
539	0.0121764699227433\\
540	0.0121983577120722\\
541	0.0122208428250783\\
542	0.0122439383126498\\
543	0.0122676584525512\\
544	0.0122920155260352\\
545	0.0123180942825718\\
546	0.0123452446050778\\
547	0.0123732279553314\\
548	0.0124020458624497\\
549	0.0124317513081825\\
550	0.0124624321916442\\
551	0.0124942611880166\\
552	0.0125275179193061\\
553	0.0125725789185033\\
554	0.0126202684013368\\
555	0.0126673374643105\\
556	0.0127141264605036\\
557	0.0127616720734353\\
558	0.0128020584216912\\
559	0.0128360640086293\\
560	0.0128691043257316\\
561	0.0129010157386838\\
562	0.0129315743184\\
563	0.0129589777156225\\
564	0.0129832273087086\\
565	0.0130071965598834\\
566	0.0130308997551807\\
567	0.0130552632004074\\
568	0.0130799921894672\\
569	0.0131039390985665\\
570	0.0131274182975934\\
571	0.0131494789434883\\
572	0.0131707902443108\\
573	0.0131919480224045\\
574	0.0132130137288525\\
575	0.013234922879128\\
576	0.0132560749400763\\
577	0.0132762429061295\\
578	0.0132951907148933\\
579	0.0133138262343339\\
580	0.0133325880297693\\
581	0.0133507372247235\\
582	0.013367860031664\\
583	0.0133844301868122\\
584	0.0134005909064923\\
585	0.0134163720167592\\
586	0.0134318302903064\\
587	0.01344693698094\\
588	0.0134616492495159\\
589	0.0134759212692747\\
590	0.0134902449069583\\
591	0.0135046704775126\\
592	0.0135192194811896\\
593	0.0135339766714342\\
594	0.0135491998980231\\
595	0.0135656125153459\\
596	0.0135851815686143\\
597	0.0136131889078992\\
598	0.0136637438596031\\
599	0\\
600	0\\
};
\addplot [color=mycolor19,solid,forget plot]
  table[row sep=crcr]{%
1	0.0102160108638796\\
2	0.010216014005395\\
3	0.0102160172032066\\
4	0.0102160204583219\\
5	0.0102160237717668\\
6	0.0102160271445851\\
7	0.0102160305778397\\
8	0.0102160340726122\\
9	0.0102160376300037\\
10	0.0102160412511347\\
11	0.0102160449371461\\
12	0.0102160486891989\\
13	0.010216052508475\\
14	0.0102160563961773\\
15	0.0102160603535304\\
16	0.0102160643817806\\
17	0.0102160684821964\\
18	0.0102160726560692\\
19	0.0102160769047134\\
20	0.0102160812294667\\
21	0.0102160856316908\\
22	0.0102160901127719\\
23	0.0102160946741208\\
24	0.0102160993171734\\
25	0.0102161040433915\\
26	0.0102161088542628\\
27	0.0102161137513016\\
28	0.0102161187360494\\
29	0.0102161238100751\\
30	0.0102161289749756\\
31	0.0102161342323764\\
32	0.010216139583932\\
33	0.0102161450313264\\
34	0.0102161505762738\\
35	0.0102161562205187\\
36	0.0102161619658371\\
37	0.0102161678140366\\
38	0.0102161737669569\\
39	0.0102161798264709\\
40	0.0102161859944845\\
41	0.0102161922729379\\
42	0.010216198663806\\
43	0.0102162051690987\\
44	0.0102162117908619\\
45	0.0102162185311781\\
46	0.0102162253921669\\
47	0.0102162323759856\\
48	0.0102162394848301\\
49	0.0102162467209357\\
50	0.0102162540865772\\
51	0.0102162615840702\\
52	0.0102162692157717\\
53	0.0102162769840806\\
54	0.0102162848914387\\
55	0.0102162929403313\\
56	0.0102163011332879\\
57	0.0102163094728836\\
58	0.0102163179617388\\
59	0.0102163266025212\\
60	0.0102163353979458\\
61	0.0102163443507759\\
62	0.0102163534638245\\
63	0.0102163627399544\\
64	0.0102163721820796\\
65	0.010216381793166\\
66	0.0102163915762325\\
67	0.0102164015343518\\
68	0.0102164116706513\\
69	0.0102164219883143\\
70	0.0102164324905807\\
71	0.0102164431807482\\
72	0.0102164540621734\\
73	0.0102164651382726\\
74	0.0102164764125231\\
75	0.0102164878884642\\
76	0.0102164995696981\\
77	0.0102165114598914\\
78	0.010216523562776\\
79	0.0102165358821503\\
80	0.0102165484218803\\
81	0.0102165611859009\\
82	0.0102165741782174\\
83	0.010216587402906\\
84	0.0102166008641159\\
85	0.0102166145660701\\
86	0.0102166285130668\\
87	0.0102166427094807\\
88	0.0102166571597647\\
89	0.0102166718684507\\
90	0.0102166868401515\\
91	0.0102167020795619\\
92	0.0102167175914606\\
93	0.0102167333807112\\
94	0.0102167494522638\\
95	0.0102167658111568\\
96	0.0102167824625184\\
97	0.010216799411568\\
98	0.0102168166636178\\
99	0.0102168342240747\\
100	0.010216852098442\\
101	0.0102168702923206\\
102	0.0102168888114115\\
103	0.0102169076615168\\
104	0.0102169268485421\\
105	0.0102169463784981\\
106	0.0102169662575024\\
107	0.0102169864917814\\
108	0.0102170070876723\\
109	0.0102170280516253\\
110	0.010217049390205\\
111	0.0102170711100931\\
112	0.01021709321809\\
113	0.010217115721117\\
114	0.0102171386262187\\
115	0.0102171619405649\\
116	0.010217185671453\\
117	0.01021720982631\\
118	0.0102172344126952\\
119	0.0102172594383022\\
120	0.0102172849109613\\
121	0.0102173108386423\\
122	0.0102173372294565\\
123	0.0102173640916595\\
124	0.0102173914336535\\
125	0.0102174192639903\\
126	0.0102174475913736\\
127	0.0102174764246616\\
128	0.0102175057728704\\
129	0.0102175356451758\\
130	0.0102175660509171\\
131	0.0102175969995993\\
132	0.0102176285008965\\
133	0.0102176605646544\\
134	0.0102176932008939\\
135	0.0102177264198139\\
136	0.0102177602317942\\
137	0.0102177946473994\\
138	0.0102178296773815\\
139	0.0102178653326834\\
140	0.0102179016244427\\
141	0.0102179385639943\\
142	0.010217976162875\\
143	0.0102180144328259\\
144	0.0102180533857969\\
145	0.01021809303395\\
146	0.0102181333896631\\
147	0.010218174465534\\
148	0.0102182162743839\\
149	0.0102182588292618\\
150	0.0102183021434483\\
151	0.0102183462304597\\
152	0.0102183911040521\\
153	0.010218436778226\\
154	0.01021848326723\\
155	0.010218530585566\\
156	0.0102185787479928\\
157	0.0102186277695315\\
158	0.0102186776654694\\
159	0.0102187284513652\\
160	0.0102187801430537\\
161	0.0102188327566504\\
162	0.010218886308557\\
163	0.0102189408154659\\
164	0.0102189962943657\\
165	0.0102190527625465\\
166	0.0102191102376047\\
167	0.0102191687374493\\
168	0.0102192282803066\\
169	0.0102192888847263\\
170	0.0102193505695869\\
171	0.0102194133541019\\
172	0.0102194772578255\\
173	0.0102195423006587\\
174	0.0102196085028552\\
175	0.010219675885028\\
176	0.0102197444681557\\
177	0.0102198142735888\\
178	0.0102198853230564\\
179	0.0102199576386727\\
180	0.0102200312429444\\
181	0.010220106158777\\
182	0.0102201824094822\\
183	0.010220260018785\\
184	0.0102203390108311\\
185	0.0102204194101943\\
186	0.0102205012418842\\
187	0.0102205845313539\\
188	0.0102206693045075\\
189	0.0102207555877089\\
190	0.010220843407789\\
191	0.0102209327920549\\
192	0.0102210237682977\\
193	0.0102211163648014\\
194	0.0102212106103516\\
195	0.0102213065342444\\
196	0.0102214041662955\\
197	0.0102215035368496\\
198	0.0102216046767893\\
199	0.0102217076175453\\
200	0.0102218123911059\\
201	0.0102219190300267\\
202	0.0102220275674412\\
203	0.0102221380370706\\
204	0.0102222504732348\\
205	0.0102223649108624\\
206	0.0102224813855024\\
207	0.0102225999333347\\
208	0.0102227205911814\\
209	0.0102228433965187\\
210	0.0102229683874881\\
211	0.0102230956029087\\
212	0.0102232250822891\\
213	0.01022335686584\\
214	0.0102234909944864\\
215	0.010223627509881\\
216	0.0102237664544167\\
217	0.0102239078712404\\
218	0.0102240518042662\\
219	0.0102241982981896\\
220	0.010224347398501\\
221	0.0102244991515008\\
222	0.0102246536043134\\
223	0.0102248108049025\\
224	0.0102249708020862\\
225	0.0102251336455524\\
226	0.0102252993858745\\
227	0.0102254680745278\\
228	0.0102256397639058\\
229	0.0102258145073367\\
230	0.0102259923591006\\
231	0.0102261733744472\\
232	0.0102263576096128\\
233	0.0102265451218391\\
234	0.0102267359693915\\
235	0.0102269302115774\\
236	0.0102271279087661\\
237	0.0102273291224079\\
238	0.0102275339150542\\
239	0.0102277423503781\\
240	0.0102279544931948\\
241	0.0102281704094831\\
242	0.0102283901664073\\
243	0.0102286138323386\\
244	0.0102288414768784\\
245	0.0102290731708812\\
246	0.0102293089864777\\
247	0.0102295489970996\\
248	0.0102297932775036\\
249	0.0102300419037967\\
250	0.0102302949534619\\
251	0.0102305525053844\\
252	0.0102308146398783\\
253	0.0102310814387143\\
254	0.0102313529851472\\
255	0.010231629363945\\
256	0.0102319106614183\\
257	0.0102321969654495\\
258	0.0102324883655245\\
259	0.0102327849527634\\
260	0.0102330868199528\\
261	0.0102333940615788\\
262	0.0102337067738608\\
263	0.0102340250547855\\
264	0.0102343490041429\\
265	0.0102346787235622\\
266	0.0102350143165489\\
267	0.0102353558885229\\
268	0.0102357035468579\\
269	0.0102360574009207\\
270	0.0102364175621132\\
271	0.0102367841439139\\
272	0.0102371572619215\\
273	0.0102375370338992\\
274	0.0102379235798206\\
275	0.0102383170219163\\
276	0.0102387174847226\\
277	0.0102391250951305\\
278	0.0102395399824371\\
279	0.010239962278398\\
280	0.0102403921172812\\
281	0.0102408296359224\\
282	0.0102412749737829\\
283	0.0102417282730078\\
284	0.0102421896784871\\
285	0.0102426593379181\\
286	0.0102431374018701\\
287	0.0102436240238508\\
288	0.010244119360375\\
289	0.0102446235710354\\
290	0.0102451368185759\\
291	0.0102456592689667\\
292	0.010246191091483\\
293	0.0102467324587851\\
294	0.0102472835470019\\
295	0.0102478445358171\\
296	0.0102484156085584\\
297	0.0102489969522896\\
298	0.0102495887579062\\
299	0.010250191220234\\
300	0.0102508045381314\\
301	0.0102514289145956\\
302	0.0102520645568722\\
303	0.0102527116765687\\
304	0.0102533704897726\\
305	0.0102540412171734\\
306	0.0102547240841893\\
307	0.0102554193210982\\
308	0.010256127163174\\
309	0.0102568478508274\\
310	0.0102575816297524\\
311	0.0102583287510776\\
312	0.0102590894715235\\
313	0.0102598640535654\\
314	0.010260652765602\\
315	0.0102614558821305\\
316	0.0102622736839272\\
317	0.0102631064582351\\
318	0.0102639544989577\\
319	0.0102648181068594\\
320	0.0102656975897724\\
321	0.0102665932628107\\
322	0.0102675054485899\\
323	0.0102684344774541\\
324	0.0102693806877097\\
325	0.0102703444258641\\
326	0.0102713260468714\\
327	0.0102723259143831\\
328	0.0102733444010033\\
329	0.0102743818885489\\
330	0.0102754387683117\\
331	0.0102765154413234\\
332	0.0102776123186206\\
333	0.0102787298215087\\
334	0.0102798683818217\\
335	0.0102810284421759\\
336	0.0102822104562145\\
337	0.0102834148888374\\
338	0.0102846422164129\\
339	0.0102858929269641\\
340	0.0102871675203215\\
341	0.0102884665082337\\
342	0.0102897904144325\\
343	0.0102911397746454\\
344	0.0102925151365476\\
345	0.0102939170598395\\
346	0.0102953461163202\\
347	0.0102968028899654\\
348	0.010298287977025\\
349	0.0102998019861563\\
350	0.010301345538624\\
351	0.0103029192686024\\
352	0.010304523823628\\
353	0.0103061598652284\\
354	0.0103078280696994\\
355	0.0103095291293405\\
356	0.0103112637528506\\
357	0.0103130326656513\\
358	0.0103148366102018\\
359	0.0103166763463019\\
360	0.0103185526513811\\
361	0.01032046632077\\
362	0.0103224181679513\\
363	0.0103244090247858\\
364	0.0103264397417111\\
365	0.0103285111879057\\
366	0.010330624251416\\
367	0.0103327798392395\\
368	0.0103349788773581\\
369	0.0103372223107155\\
370	0.0103395111031315\\
371	0.0103418462371451\\
372	0.0103442287137781\\
373	0.0103466595522112\\
374	0.0103491397893617\\
375	0.0103516704793535\\
376	0.0103542526928684\\
377	0.0103568875163673\\
378	0.0103595760511692\\
379	0.0103623194123768\\
380	0.0103651187276345\\
381	0.0103679751357077\\
382	0.01037088978487\\
383	0.0103738638310872\\
384	0.0103768984359875\\
385	0.010379994764614\\
386	0.0103831539829689\\
387	0.0103863772553981\\
388	0.0103896657419934\\
389	0.0103930205965776\\
390	0.0103964429670876\\
391	0.0103999340041957\\
392	0.0104034948971676\\
393	0.0104071269994475\\
394	0.0104108322519377\\
395	0.0104146146043903\\
396	0.0104184848226213\\
397	0.0104224769219781\\
398	0.0104266170680057\\
399	0.0104308695664695\\
400	0.0104352004506298\\
401	0.0104396104033915\\
402	0.0104441000292187\\
403	0.0104486698070544\\
404	0.0104533199583441\\
405	0.0104580500748369\\
406	0.0104628580682281\\
407	0.0104677371996859\\
408	0.0104726676393481\\
409	0.0104775923092647\\
410	0.0104823470834675\\
411	0.010486884196447\\
412	0.0104915002814511\\
413	0.0104961963054818\\
414	0.0105009734672681\\
415	0.0105058330705055\\
416	0.0105107764308712\\
417	0.0105158048741282\\
418	0.010520919733102\\
419	0.010526122344945\\
420	0.0105314140490443\\
421	0.0105367961736621\\
422	0.0105422700317515\\
423	0.0105478368998609\\
424	0.0105534980816071\\
425	0.010559254914826\\
426	0.0105651087800921\\
427	0.0105710611104227\\
428	0.0105771134015832\\
429	0.0105832672221906\\
430	0.0105895242247094\\
431	0.0105958861695747\\
432	0.0106023550128572\\
433	0.0106089341512195\\
434	0.0106156366260048\\
435	0.0106224646653094\\
436	0.0106294204940609\\
437	0.0106365063246536\\
438	0.0106437243458301\\
439	0.010651076709529\\
440	0.0106585655153817\\
441	0.0106661927924964\\
442	0.0106739604781271\\
443	0.010681870392781\\
444	0.010689924211298\\
445	0.0106981234294457\\
446	0.0107064693256434\\
447	0.0107149629173934\\
448	0.0107236049107167\\
449	0.0107323956323151\\
450	0.010741334890645\\
451	0.0107504219114158\\
452	0.0107596555063136\\
453	0.0107690339497155\\
454	0.010778554637225\\
455	0.0107882122525449\\
456	0.0107980000907883\\
457	0.0108079098889469\\
458	0.0108179316767824\\
459	0.0108280536863329\\
460	0.0108382623192952\\
461	0.0108485416956345\\
462	0.0108588669433471\\
463	0.0108692257906381\\
464	0.0108796040041826\\
465	0.0108896323702773\\
466	0.0108995751987845\\
467	0.0109096904092387\\
468	0.0109199828591609\\
469	0.0109304580797673\\
470	0.010941122976827\\
471	0.0109519880076902\\
472	0.0109630447368926\\
473	0.0109742926884899\\
474	0.0109857221422626\\
475	0.0109973602708092\\
476	0.0110092630320491\\
477	0.0110214448404314\\
478	0.0110339217960156\\
479	0.0110467119174964\\
480	0.0110598354063832\\
481	0.0110733149468219\\
482	0.0110871760472935\\
483	0.0111014474364632\\
484	0.011116161545282\\
485	0.0111313551747629\\
486	0.0111470706830823\\
487	0.0111633588603293\\
488	0.0111802877053137\\
489	0.0111985015705012\\
490	0.0112208323070155\\
491	0.0112434270011014\\
492	0.01126627719628\\
493	0.0112893756449881\\
494	0.0113127111015965\\
495	0.0113362783055923\\
496	0.0113600840848991\\
497	0.0113841177708688\\
498	0.0114083670366784\\
499	0.0114328176700319\\
500	0.0114574533068956\\
501	0.0114822551174988\\
502	0.0115072014309169\\
503	0.0115322672922947\\
504	0.0115574239746017\\
505	0.0115826385773421\\
506	0.0116078753247936\\
507	0.0116331185117995\\
508	0.0116582972416487\\
509	0.011682976487573\\
510	0.0117029182183745\\
511	0.0117229753407755\\
512	0.0117431304848434\\
513	0.0117633648421504\\
514	0.0117836580049199\\
515	0.0118039885361277\\
516	0.0118243355970798\\
517	0.0118446696352771\\
518	0.0118649301769579\\
519	0.0118850454386762\\
520	0.0119050625039699\\
521	0.0119249878098595\\
522	0.0119447913272117\\
523	0.0119644403906871\\
524	0.0119838967563132\\
525	0.0120029939191967\\
526	0.0120208686342032\\
527	0.0120388810354602\\
528	0.0120570239371964\\
529	0.012075317979301\\
530	0.012093771103102\\
531	0.0121123671487895\\
532	0.0121310863890163\\
533	0.0121499070509248\\
534	0.0121688761207841\\
535	0.0121889273014959\\
536	0.0122101674742018\\
537	0.0122319959865568\\
538	0.012254436864366\\
539	0.0122776213607122\\
540	0.0123016298278577\\
541	0.0123264885210862\\
542	0.0123522271906262\\
543	0.0123788508145031\\
544	0.0124064111370428\\
545	0.0124341899271275\\
546	0.012462879016296\\
547	0.0124929842955957\\
548	0.0125346753326465\\
549	0.0125795783189477\\
550	0.0126242494224955\\
551	0.0126695650210792\\
552	0.0127147548130075\\
553	0.0127495048228508\\
554	0.012780481350985\\
555	0.0128103328863578\\
556	0.0128389907278093\\
557	0.0128651902329027\\
558	0.0128887593915927\\
559	0.0129108394513219\\
560	0.0129325533708364\\
561	0.0129540156508846\\
562	0.0129752787946581\\
563	0.0129964965610991\\
564	0.0130192184412886\\
565	0.0130416663316402\\
566	0.0130638233245856\\
567	0.0130848801536914\\
568	0.0131050486548192\\
569	0.0131251862555491\\
570	0.0131452751480645\\
571	0.0131651374840012\\
572	0.013185614029845\\
573	0.0132064222300329\\
574	0.0132266477375825\\
575	0.0132453129505523\\
576	0.0132637709093837\\
577	0.0132819565315644\\
578	0.0133003146967692\\
579	0.0133184309476972\\
580	0.0133355606149827\\
581	0.0133522925858001\\
582	0.0133686912531438\\
583	0.0133848547771268\\
584	0.0134007994899512\\
585	0.0134164953173271\\
586	0.0134319025775093\\
587	0.013446976659015\\
588	0.0134616692113963\\
589	0.0134759300888777\\
590	0.0134902481999006\\
591	0.0135046713967894\\
592	0.0135192196273521\\
593	0.0135339766714342\\
594	0.0135491998980231\\
595	0.0135656125153459\\
596	0.0135851815686143\\
597	0.0136131889078992\\
598	0.0136637438596031\\
599	0\\
600	0\\
};
\addplot [color=red!50!mycolor17,solid,forget plot]
  table[row sep=crcr]{%
1	0.0104922631019662\\
2	0.0104922658281164\\
3	0.0104922686031995\\
4	0.010492271428093\\
5	0.0104922743036902\\
6	0.0104922772309002\\
7	0.0104922802106486\\
8	0.0104922832438776\\
9	0.0104922863315461\\
10	0.0104922894746302\\
11	0.0104922926741236\\
12	0.0104922959310376\\
13	0.0104922992464019\\
14	0.0104923026212644\\
15	0.0104923060566917\\
16	0.0104923095537698\\
17	0.0104923131136039\\
18	0.0104923167373189\\
19	0.0104923204260602\\
20	0.0104923241809933\\
21	0.0104923280033049\\
22	0.0104923318942026\\
23	0.010492335854916\\
24	0.0104923398866963\\
25	0.0104923439908175\\
26	0.010492348168576\\
27	0.0104923524212916\\
28	0.0104923567503079\\
29	0.0104923611569921\\
30	0.0104923656427363\\
31	0.0104923702089571\\
32	0.0104923748570969\\
33	0.0104923795886234\\
34	0.010492384405031\\
35	0.0104923893078406\\
36	0.0104923942986002\\
37	0.0104923993788858\\
38	0.0104924045503013\\
39	0.0104924098144795\\
40	0.0104924151730822\\
41	0.010492420627801\\
42	0.0104924261803578\\
43	0.0104924318325052\\
44	0.0104924375860271\\
45	0.0104924434427395\\
46	0.0104924494044907\\
47	0.0104924554731619\\
48	0.0104924616506682\\
49	0.0104924679389588\\
50	0.0104924743400177\\
51	0.0104924808558644\\
52	0.0104924874885545\\
53	0.0104924942401803\\
54	0.0104925011128714\\
55	0.0104925081087957\\
56	0.0104925152301594\\
57	0.0104925224792086\\
58	0.0104925298582292\\
59	0.0104925373695479\\
60	0.0104925450155332\\
61	0.0104925527985957\\
62	0.010492560721189\\
63	0.0104925687858107\\
64	0.0104925769950028\\
65	0.0104925853513528\\
66	0.0104925938574944\\
67	0.0104926025161081\\
68	0.0104926113299224\\
69	0.0104926203017146\\
70	0.0104926294343114\\
71	0.0104926387305898\\
72	0.0104926481934785\\
73	0.010492657825958\\
74	0.0104926676310624\\
75	0.0104926776118796\\
76	0.0104926877715526\\
77	0.0104926981132806\\
78	0.0104927086403197\\
79	0.0104927193559842\\
80	0.0104927302636472\\
81	0.0104927413667422\\
82	0.0104927526687637\\
83	0.0104927641732687\\
84	0.0104927758838773\\
85	0.0104927878042743\\
86	0.0104927999382101\\
87	0.0104928122895018\\
88	0.0104928248620345\\
89	0.0104928376597627\\
90	0.0104928506867109\\
91	0.0104928639469758\\
92	0.0104928774447264\\
93	0.0104928911842065\\
94	0.0104929051697349\\
95	0.0104929194057077\\
96	0.0104929338965988\\
97	0.0104929486469621\\
98	0.0104929636614321\\
99	0.0104929789447259\\
100	0.0104929945016447\\
101	0.0104930103370746\\
102	0.0104930264559889\\
103	0.0104930428634492\\
104	0.010493059564607\\
105	0.0104930765647054\\
106	0.0104930938690805\\
107	0.0104931114831635\\
108	0.0104931294124817\\
109	0.0104931476626607\\
110	0.010493166239426\\
111	0.0104931851486048\\
112	0.0104932043961274\\
113	0.0104932239880297\\
114	0.0104932439304545\\
115	0.0104932642296535\\
116	0.0104932848919895\\
117	0.0104933059239378\\
118	0.0104933273320889\\
119	0.0104933491231497\\
120	0.0104933713039462\\
121	0.0104933938814253\\
122	0.0104934168626569\\
123	0.0104934402548361\\
124	0.0104934640652854\\
125	0.0104934883014571\\
126	0.0104935129709352\\
127	0.0104935380814379\\
128	0.0104935636408201\\
129	0.0104935896570754\\
130	0.0104936161383389\\
131	0.0104936430928895\\
132	0.0104936705291525\\
133	0.0104936984557019\\
134	0.0104937268812632\\
135	0.0104937558147162\\
136	0.0104937852650972\\
137	0.0104938152416022\\
138	0.0104938457535896\\
139	0.0104938768105825\\
140	0.0104939084222723\\
141	0.0104939405985213\\
142	0.0104939733493656\\
143	0.0104940066850179\\
144	0.0104940406158713\\
145	0.0104940751525017\\
146	0.0104941103056712\\
147	0.0104941460863313\\
148	0.0104941825056264\\
149	0.0104942195748969\\
150	0.0104942573056825\\
151	0.0104942957097259\\
152	0.0104943347989761\\
153	0.0104943745855921\\
154	0.0104944150819465\\
155	0.0104944563006289\\
156	0.0104944982544502\\
157	0.0104945409564457\\
158	0.0104945844198797\\
159	0.0104946286582487\\
160	0.010494673685286\\
161	0.0104947195149653\\
162	0.0104947661615053\\
163	0.0104948136393733\\
164	0.01049486196329\\
165	0.0104949111482334\\
166	0.0104949612094436\\
167	0.0104950121624269\\
168	0.0104950640229607\\
169	0.0104951168070978\\
170	0.0104951705311712\\
171	0.010495225211799\\
172	0.0104952808658891\\
173	0.010495337510644\\
174	0.0104953951635663\\
175	0.0104954538424632\\
176	0.0104955135654519\\
177	0.0104955743509652\\
178	0.010495636217756\\
179	0.0104956991849036\\
180	0.0104957632718188\\
181	0.0104958284982492\\
182	0.0104958948842857\\
183	0.0104959624503672\\
184	0.0104960312172877\\
185	0.0104961012062009\\
186	0.0104961724386275\\
187	0.0104962449364606\\
188	0.010496318721972\\
189	0.0104963938178189\\
190	0.0104964702470499\\
191	0.0104965480331121\\
192	0.0104966271998571\\
193	0.0104967077715482\\
194	0.0104967897728671\\
195	0.0104968732289208\\
196	0.0104969581652489\\
197	0.0104970446078305\\
198	0.0104971325830917\\
199	0.0104972221179128\\
200	0.010497313239636\\
201	0.0104974059760731\\
202	0.0104975003555128\\
203	0.0104975964067293\\
204	0.0104976941589894\\
205	0.0104977936420615\\
206	0.0104978948862232\\
207	0.01049799792227\\
208	0.0104981027815234\\
209	0.0104982094958401\\
210	0.0104983180976203\\
211	0.0104984286198166\\
212	0.0104985410959434\\
213	0.0104986555600853\\
214	0.0104987720469073\\
215	0.0104988905916636\\
216	0.0104990112302072\\
217	0.010499133999\\
218	0.0104992589351221\\
219	0.0104993860762823\\
220	0.0104995154608281\\
221	0.0104996471277557\\
222	0.0104997811167208\\
223	0.0104999174680493\\
224	0.0105000562227476\\
225	0.0105001974225139\\
226	0.0105003411097492\\
227	0.0105004873275684\\
228	0.0105006361198119\\
229	0.0105007875310573\\
230	0.0105009416066306\\
231	0.0105010983926189\\
232	0.0105012579358818\\
233	0.0105014202840642\\
234	0.0105015854856083\\
235	0.0105017535897667\\
236	0.010501924646615\\
237	0.0105020987070646\\
238	0.0105022758228765\\
239	0.0105024560466742\\
240	0.0105026394319573\\
241	0.0105028260331158\\
242	0.0105030159054434\\
243	0.0105032091051526\\
244	0.0105034056893881\\
245	0.0105036057162423\\
246	0.01050380924477\\
247	0.0105040163350029\\
248	0.0105042270479657\\
249	0.0105044414456914\\
250	0.0105046595912368\\
251	0.0105048815486988\\
252	0.0105051073832308\\
253	0.0105053371610589\\
254	0.010505570949499\\
255	0.0105058088169737\\
256	0.0105060508330297\\
257	0.0105062970683554\\
258	0.0105065475947986\\
259	0.0105068024853851\\
260	0.010507061814337\\
261	0.0105073256570913\\
262	0.0105075940903192\\
263	0.0105078671919457\\
264	0.010508145041169\\
265	0.0105084277184811\\
266	0.010508715305688\\
267	0.0105090078859304\\
268	0.0105093055437056\\
269	0.0105096083648883\\
270	0.0105099164367536\\
271	0.0105102298479987\\
272	0.0105105486887663\\
273	0.010510873050668\\
274	0.0105112030268079\\
275	0.0105115387118076\\
276	0.0105118802018305\\
277	0.0105122275946078\\
278	0.0105125809894644\\
279	0.0105129404873456\\
280	0.0105133061908446\\
281	0.0105136782042302\\
282	0.0105140566334759\\
283	0.0105144415862893\\
284	0.0105148331721422\\
285	0.0105152315023017\\
286	0.0105156366898627\\
287	0.0105160488497803\\
288	0.0105164680989036\\
289	0.0105168945560115\\
290	0.0105173283418478\\
291	0.0105177695791592\\
292	0.0105182183927334\\
293	0.0105186749094392\\
294	0.010519139258268\\
295	0.010519611570376\\
296	0.0105200919791296\\
297	0.0105205806201507\\
298	0.0105210776313655\\
299	0.0105215831530543\\
300	0.0105220973279038\\
301	0.0105226203010616\\
302	0.0105231522201933\\
303	0.0105236932355421\\
304	0.0105242434999914\\
305	0.0105248031691303\\
306	0.0105253724013227\\
307	0.0105259513577794\\
308	0.0105265402026347\\
309	0.0105271391030267\\
310	0.0105277482291823\\
311	0.0105283677545068\\
312	0.0105289978556789\\
313	0.0105296387127513\\
314	0.0105302905092574\\
315	0.0105309534323245\\
316	0.0105316276727947\\
317	0.0105323134253526\\
318	0.0105330108886626\\
319	0.0105337202655145\\
320	0.0105344417629798\\
321	0.0105351755925783\\
322	0.0105359219704566\\
323	0.0105366811175805\\
324	0.01053745325994\\
325	0.0105382386287713\\
326	0.0105390374607944\\
327	0.0105398499984705\\
328	0.0105406764902786\\
329	0.0105415171910157\\
330	0.0105423723621204\\
331	0.0105432422720251\\
332	0.0105441271965372\\
333	0.0105450274192547\\
334	0.0105459432320183\\
335	0.0105468749354065\\
336	0.0105478228392777\\
337	0.0105487872633681\\
338	0.010549768537953\\
339	0.0105507670045796\\
340	0.0105517830168645\\
341	0.0105528169413062\\
342	0.0105538691580087\\
343	0.0105549400616397\\
344	0.0105560300627568\\
345	0.0105571395849841\\
346	0.0105582690650129\\
347	0.0105594189524104\\
348	0.0105605897091741\\
349	0.0105617818089554\\
350	0.0105629957358704\\
351	0.0105642319828643\\
352	0.0105654910497801\\
353	0.0105667734416641\\
354	0.0105680796674353\\
355	0.0105694102299327\\
356	0.010570765644879\\
357	0.0105721464428852\\
358	0.0105735531700711\\
359	0.0105749863887159\\
360	0.0105764466779409\\
361	0.0105779346344247\\
362	0.0105794508731527\\
363	0.0105809960282012\\
364	0.010582570753558\\
365	0.0105841757239788\\
366	0.0105858116358823\\
367	0.0105874792082816\\
368	0.0105891791837536\\
369	0.0105909123294459\\
370	0.0105926794381194\\
371	0.0105944813292259\\
372	0.0105963188500177\\
373	0.0105981928766863\\
374	0.010600104315526\\
375	0.0106020541041159\\
376	0.0106040432125143\\
377	0.0106060726444541\\
378	0.0106081434385303\\
379	0.0106102566693627\\
380	0.0106124134487185\\
381	0.0106146149265712\\
382	0.0106168622920723\\
383	0.0106191567744025\\
384	0.0106214996434668\\
385	0.0106238922103874\\
386	0.0106263358277431\\
387	0.0106288318894902\\
388	0.0106313818304932\\
389	0.0106339871255754\\
390	0.0106366492879889\\
391	0.0106393698671846\\
392	0.010642150445741\\
393	0.0106449926352949\\
394	0.0106478980712935\\
395	0.0106508684063931\\
396	0.0106539053024393\\
397	0.0106570104214851\\
398	0.0106601854181536\\
399	0.010663431923342\\
400	0.0106667515211934\\
401	0.0106701457270135\\
402	0.0106736159610154\\
403	0.0106771635172982\\
404	0.0106807895274217\\
405	0.0106844949179383\\
406	0.0106882803613391\\
407	0.0106921462202038\\
408	0.0106960924852302\\
409	0.0107001187097593\\
410	0.0107042239465306\\
411	0.010708406692336\\
412	0.0107126648710745\\
413	0.0107169956907362\\
414	0.0107213938194012\\
415	0.0107258577179076\\
416	0.0107303884949005\\
417	0.0107349873499031\\
418	0.0107396555825343\\
419	0.0107443945491256\\
420	0.0107492056447915\\
421	0.010754090654461\\
422	0.0107590516533602\\
423	0.0107640914643021\\
424	0.0107692114401413\\
425	0.0107744129987577\\
426	0.0107796976298033\\
427	0.010785066901977\\
428	0.01079052247092\\
429	0.0107960660881549\\
430	0.0108016996122699\\
431	0.0108074250244273\\
432	0.0108132444475376\\
433	0.01081916012765\\
434	0.0108251741797216\\
435	0.0108312888626474\\
436	0.0108375065981016\\
437	0.0108438299920576\\
438	0.0108502618593683\\
439	0.0108568052518437\\
440	0.0108634634903231\\
441	0.0108702402013144\\
442	0.0108771393588687\\
443	0.0108841653325115\\
444	0.0108913229423706\\
445	0.0108986175234693\\
446	0.0109060550036153\\
447	0.0109136420071115\\
448	0.0109213860218328\\
449	0.0109292957507719\\
450	0.0109373820478817\\
451	0.0109456536434308\\
452	0.0109541105728118\\
453	0.0109627553679178\\
454	0.0109715952766986\\
455	0.0109806977758795\\
456	0.0109900845475758\\
457	0.0109997801986754\\
458	0.0110098129116675\\
459	0.0110202158358084\\
460	0.0110310310121471\\
461	0.0110423224146991\\
462	0.0110564064852099\\
463	0.0110708948770853\\
464	0.0110855803674594\\
465	0.0111004767213673\\
466	0.0111155915247348\\
467	0.0111309255843516\\
468	0.0111464794308322\\
469	0.0111622533008335\\
470	0.0111782472053681\\
471	0.0111944612900714\\
472	0.0112108918541861\\
473	0.0112275359633692\\
474	0.0112443882060358\\
475	0.0112614484415217\\
476	0.0112787200694372\\
477	0.0112961973842977\\
478	0.0113138733222272\\
479	0.011331739249686\\
480	0.0113497847213654\\
481	0.011367997203887\\
482	0.0113863617612842\\
483	0.0114048606963048\\
484	0.0114234731324198\\
485	0.0114421744806877\\
486	0.0114609355615265\\
487	0.0114797204229935\\
488	0.0114984788764643\\
489	0.0115166660066114\\
490	0.0115318354217178\\
491	0.0115471363708895\\
492	0.0115625613392904\\
493	0.0115781020607263\\
494	0.0115937494686824\\
495	0.0116094935508971\\
496	0.0116253232256831\\
497	0.0116412265127655\\
498	0.0116571905147545\\
499	0.011673201409273\\
500	0.0116892444520922\\
501	0.0117053039621049\\
502	0.0117213633644552\\
503	0.0117374051852127\\
504	0.0117534110359217\\
505	0.0117693627276014\\
506	0.0117852395549783\\
507	0.0118010171733033\\
508	0.0118166667318458\\
509	0.0118320992866367\\
510	0.0118465763301248\\
511	0.0118611299157394\\
512	0.0118758315580416\\
513	0.0118906792584423\\
514	0.0119056711094668\\
515	0.0119208055951756\\
516	0.0119360845568445\\
517	0.0119515095264179\\
518	0.0119670778812097\\
519	0.011982786226415\\
520	0.0119986534641948\\
521	0.0120146925648888\\
522	0.0120309116528699\\
523	0.0120473209130739\\
524	0.0120639339183285\\
525	0.0120813714269544\\
526	0.0120994853871426\\
527	0.0121177974319214\\
528	0.012136285660672\\
529	0.0121549258671513\\
530	0.0121738599535043\\
531	0.0121934101250772\\
532	0.0122135910540104\\
533	0.0122344195004237\\
534	0.012255936213804\\
535	0.01227781202766\\
536	0.012299986826479\\
537	0.0123229992346631\\
538	0.0123468690334845\\
539	0.0123715943710749\\
540	0.0123972508259376\\
541	0.0124239552551283\\
542	0.012451953159002\\
543	0.0124888780177537\\
544	0.0125317851836834\\
545	0.0125749626463171\\
546	0.0126183997896953\\
547	0.012661041093234\\
548	0.0126932500319898\\
549	0.0127217852363899\\
550	0.0127492609051859\\
551	0.0127746658052106\\
552	0.012798208348949\\
553	0.0128191314755905\\
554	0.0128392130479626\\
555	0.0128590674270762\\
556	0.0128787040689377\\
557	0.0128979632954015\\
558	0.0129171275060748\\
559	0.0129364774215315\\
560	0.0129571735862411\\
561	0.0129784817787838\\
562	0.012999586534313\\
563	0.0130202237919064\\
564	0.013039402335484\\
565	0.0130586139383252\\
566	0.0130778453899885\\
567	0.0130969414096854\\
568	0.0131159704008785\\
569	0.0131351261912954\\
570	0.0131554993448754\\
571	0.0131756759297228\\
572	0.013194772201013\\
573	0.0132130537859863\\
574	0.0132311740231579\\
575	0.0132489771434035\\
576	0.0132670622357414\\
577	0.0132852418138994\\
578	0.0133025407158314\\
579	0.0133194135767091\\
580	0.0133360118213147\\
581	0.0133524572942046\\
582	0.0133687604594872\\
583	0.0133848934143346\\
584	0.0134008219604018\\
585	0.0134165078757943\\
586	0.0134319091005627\\
587	0.0134469797340211\\
588	0.0134616704768085\\
589	0.0134759305176615\\
590	0.0134902483085425\\
591	0.0135046714122469\\
592	0.0135192196273521\\
593	0.0135339766714342\\
594	0.0135491998980231\\
595	0.0135656125153459\\
596	0.0135851815686143\\
597	0.0136131889078992\\
598	0.0136637438596031\\
599	0\\
600	0\\
};
\addplot [color=red!40!mycolor19,solid,forget plot]
  table[row sep=crcr]{%
1	0.0106114853119478\\
2	0.0106114886185054\\
3	0.0106114919844647\\
4	0.0106114954108921\\
5	0.0106114988988731\\
6	0.0106115024495128\\
7	0.0106115060639359\\
8	0.0106115097432875\\
9	0.0106115134887331\\
10	0.0106115173014591\\
11	0.0106115211826732\\
12	0.0106115251336047\\
13	0.0106115291555051\\
14	0.0106115332496482\\
15	0.0106115374173307\\
16	0.0106115416598725\\
17	0.0106115459786171\\
18	0.0106115503749323\\
19	0.0106115548502102\\
20	0.010611559405868\\
21	0.0106115640433483\\
22	0.0106115687641193\\
23	0.0106115735696759\\
24	0.0106115784615396\\
25	0.0106115834412592\\
26	0.0106115885104112\\
27	0.0106115936706003\\
28	0.0106115989234603\\
29	0.0106116042706538\\
30	0.0106116097138735\\
31	0.0106116152548425\\
32	0.0106116208953144\\
33	0.0106116266370745\\
34	0.0106116324819402\\
35	0.0106116384317611\\
36	0.0106116444884203\\
37	0.0106116506538343\\
38	0.0106116569299542\\
39	0.0106116633187659\\
40	0.0106116698222909\\
41	0.0106116764425869\\
42	0.0106116831817486\\
43	0.0106116900419079\\
44	0.0106116970252352\\
45	0.0106117041339396\\
46	0.0106117113702697\\
47	0.0106117187365145\\
48	0.010611726235004\\
49	0.0106117338681097\\
50	0.0106117416382458\\
51	0.0106117495478696\\
52	0.0106117575994823\\
53	0.01061176579563\\
54	0.0106117741389045\\
55	0.0106117826319436\\
56	0.0106117912774326\\
57	0.0106118000781048\\
58	0.0106118090367423\\
59	0.0106118181561772\\
60	0.0106118274392921\\
61	0.0106118368890211\\
62	0.010611846508351\\
63	0.0106118563003219\\
64	0.0106118662680282\\
65	0.0106118764146198\\
66	0.0106118867433028\\
67	0.0106118972573408\\
68	0.0106119079600556\\
69	0.0106119188548285\\
70	0.0106119299451011\\
71	0.0106119412343767\\
72	0.010611952726221\\
73	0.0106119644242638\\
74	0.0106119763321994\\
75	0.0106119884537883\\
76	0.0106120007928581\\
77	0.010612013353305\\
78	0.0106120261390946\\
79	0.0106120391542633\\
80	0.0106120524029199\\
81	0.0106120658892463\\
82	0.0106120796174991\\
83	0.0106120935920112\\
84	0.0106121078171925\\
85	0.0106121222975319\\
86	0.0106121370375983\\
87	0.0106121520420422\\
88	0.010612167315597\\
89	0.0106121828630808\\
90	0.0106121986893976\\
91	0.0106122147995387\\
92	0.0106122311985847\\
93	0.0106122478917066\\
94	0.010612264884168\\
95	0.0106122821813259\\
96	0.0106122997886332\\
97	0.0106123177116399\\
98	0.0106123359559948\\
99	0.0106123545274477\\
100	0.0106123734318505\\
101	0.0106123926751598\\
102	0.0106124122634379\\
103	0.0106124322028555\\
104	0.010612452499693\\
105	0.0106124731603427\\
106	0.0106124941913108\\
107	0.0106125155992192\\
108	0.0106125373908078\\
109	0.0106125595729364\\
110	0.0106125821525871\\
111	0.0106126051368659\\
112	0.0106126285330055\\
113	0.0106126523483672\\
114	0.0106126765904431\\
115	0.0106127012668587\\
116	0.010612726385375\\
117	0.0106127519538908\\
118	0.0106127779804457\\
119	0.0106128044732218\\
120	0.0106128314405467\\
121	0.010612858890896\\
122	0.0106128868328956\\
123	0.0106129152753247\\
124	0.0106129442271185\\
125	0.0106129736973706\\
126	0.0106130036953361\\
127	0.0106130342304341\\
128	0.010613065312251\\
129	0.0106130969505431\\
130	0.0106131291552397\\
131	0.0106131619364462\\
132	0.0106131953044469\\
133	0.0106132292697085\\
134	0.0106132638428831\\
135	0.0106132990348112\\
136	0.0106133348565256\\
137	0.0106133713192541\\
138	0.0106134084344234\\
139	0.0106134462136622\\
140	0.0106134846688049\\
141	0.0106135238118953\\
142	0.01061356365519\\
143	0.0106136042111621\\
144	0.0106136454925053\\
145	0.0106136875121373\\
146	0.0106137302832041\\
147	0.0106137738190836\\
148	0.0106138181333898\\
149	0.0106138632399769\\
150	0.0106139091529434\\
151	0.0106139558866363\\
152	0.0106140034556557\\
153	0.0106140518748587\\
154	0.0106141011593641\\
155	0.010614151324557\\
156	0.0106142023860931\\
157	0.0106142543599038\\
158	0.0106143072622006\\
159	0.0106143611094798\\
160	0.0106144159185279\\
161	0.0106144717064263\\
162	0.0106145284905562\\
163	0.010614586288604\\
164	0.0106146451185665\\
165	0.0106147049987562\\
166	0.0106147659478065\\
167	0.0106148279846774\\
168	0.0106148911286613\\
169	0.0106149553993881\\
170	0.0106150208168314\\
171	0.0106150874013143\\
172	0.0106151551735151\\
173	0.0106152241544738\\
174	0.0106152943655977\\
175	0.010615365828668\\
176	0.0106154385658462\\
177	0.0106155125996802\\
178	0.0106155879531111\\
179	0.0106156646494798\\
180	0.0106157427125338\\
181	0.0106158221664337\\
182	0.0106159030357609\\
183	0.0106159853455239\\
184	0.0106160691211659\\
185	0.0106161543885722\\
186	0.0106162411740772\\
187	0.0106163295044722\\
188	0.0106164194070132\\
189	0.0106165109094282\\
190	0.0106166040399256\\
191	0.0106166988272018\\
192	0.0106167953004496\\
193	0.0106168934893662\\
194	0.0106169934241618\\
195	0.010617095135568\\
196	0.0106171986548464\\
197	0.0106173040137976\\
198	0.0106174112447697\\
199	0.0106175203806675\\
200	0.0106176314549618\\
201	0.0106177445016987\\
202	0.0106178595555088\\
203	0.0106179766516168\\
204	0.0106180958258514\\
205	0.0106182171146553\\
206	0.0106183405550946\\
207	0.0106184661848695\\
208	0.0106185940423246\\
209	0.010618724166459\\
210	0.0106188565969369\\
211	0.0106189913740988\\
212	0.0106191285389719\\
213	0.0106192681332814\\
214	0.0106194101994619\\
215	0.010619554780668\\
216	0.0106197019207869\\
217	0.0106198516644492\\
218	0.0106200040570411\\
219	0.0106201591447165\\
220	0.010620316974409\\
221	0.0106204775938443\\
222	0.0106206410515526\\
223	0.0106208073968814\\
224	0.0106209766800083\\
225	0.0106211489519541\\
226	0.0106213242645958\\
227	0.0106215026706799\\
228	0.0106216842238363\\
229	0.0106218689785917\\
230	0.0106220569903833\\
231	0.0106222483155734\\
232	0.0106224430114632\\
233	0.0106226411363072\\
234	0.010622842749328\\
235	0.010623047910731\\
236	0.0106232566817192\\
237	0.0106234691245085\\
238	0.0106236853023429\\
239	0.01062390527951\\
240	0.0106241291213568\\
241	0.0106243568943053\\
242	0.0106245886658686\\
243	0.0106248245046674\\
244	0.0106250644804459\\
245	0.0106253086640887\\
246	0.0106255571276377\\
247	0.0106258099443086\\
248	0.0106260671885082\\
249	0.0106263289358519\\
250	0.0106265952631809\\
251	0.01062686624858\\
252	0.0106271419713954\\
253	0.0106274225122526\\
254	0.0106277079530748\\
255	0.0106279983771009\\
256	0.0106282938689045\\
257	0.0106285945144121\\
258	0.0106289004009223\\
259	0.0106292116171247\\
260	0.0106295282531188\\
261	0.010629850400434\\
262	0.0106301781520484\\
263	0.0106305116024089\\
264	0.0106308508474508\\
265	0.0106311959846179\\
266	0.0106315471128821\\
267	0.0106319043327643\\
268	0.0106322677463545\\
269	0.0106326374573318\\
270	0.0106330135709858\\
271	0.0106333961942367\\
272	0.0106337854356564\\
273	0.0106341814054895\\
274	0.010634584215674\\
275	0.0106349939798628\\
276	0.0106354108134446\\
277	0.0106358348335654\\
278	0.0106362661591498\\
279	0.0106367049109225\\
280	0.0106371512114293\\
281	0.0106376051850594\\
282	0.0106380669580663\\
283	0.0106385366585895\\
284	0.0106390144166765\\
285	0.0106395003643038\\
286	0.0106399946353989\\
287	0.0106404973658617\\
288	0.0106410086935861\\
289	0.0106415287584815\\
290	0.0106420577024943\\
291	0.0106425956696292\\
292	0.0106431428059706\\
293	0.010643699259704\\
294	0.0106442651811368\\
295	0.0106448407227195\\
296	0.0106454260390668\\
297	0.0106460212869784\\
298	0.0106466266254592\\
299	0.0106472422157404\\
300	0.0106478682212995\\
301	0.0106485048078808\\
302	0.0106491521435151\\
303	0.0106498103985398\\
304	0.0106504797456186\\
305	0.0106511603597607\\
306	0.0106518524183406\\
307	0.0106525561011172\\
308	0.0106532715902523\\
309	0.0106539990703303\\
310	0.0106547387283762\\
311	0.0106554907538748\\
312	0.010656255338789\\
313	0.0106570326775785\\
314	0.0106578229672179\\
315	0.0106586264072161\\
316	0.0106594431996341\\
317	0.0106602735491048\\
318	0.0106611176628514\\
319	0.0106619757507073\\
320	0.0106628480251361\\
321	0.0106637347012522\\
322	0.010664635996842\\
323	0.0106655521323865\\
324	0.0106664833310841\\
325	0.0106674298188756\\
326	0.0106683918244704\\
327	0.0106693695793736\\
328	0.0106703633179169\\
329	0.0106713732772895\\
330	0.0106723996975737\\
331	0.010673442821782\\
332	0.0106745028958983\\
333	0.010675580168923\\
334	0.010676674892924\\
335	0.010677787323098\\
336	0.0106789177178506\\
337	0.0106800663389225\\
338	0.010681233451625\\
339	0.0106824193253267\\
340	0.0106836242344602\\
341	0.0106848484602662\\
342	0.0106860922918499\\
343	0.010687356014712\\
344	0.010688639902096\\
345	0.0106899443237689\\
346	0.0106912696713181\\
347	0.0106926163614273\\
348	0.0106939848395193\\
349	0.0106953755835486\\
350	0.0106967891073743\\
351	0.0106982259625612\\
352	0.0106996867372792\\
353	0.0107011720552247\\
354	0.0107026826054926\\
355	0.0107042193671596\\
356	0.0107057828480992\\
357	0.0107073735302338\\
358	0.0107089919070945\\
359	0.0107106384843793\\
360	0.0107123137805568\\
361	0.0107140183275212\\
362	0.0107157526713025\\
363	0.0107175173728395\\
364	0.0107193130088206\\
365	0.0107211401726009\\
366	0.010722999475204\\
367	0.0107248915464169\\
368	0.0107268170359897\\
369	0.0107287766149518\\
370	0.0107307709770569\\
371	0.0107328008403744\\
372	0.0107348669490421\\
373	0.0107369700752005\\
374	0.010739111021132\\
375	0.0107412906216275\\
376	0.0107435097466122\\
377	0.0107457693040592\\
378	0.0107480702432313\\
379	0.0107504135582904\\
380	0.010752800292323\\
381	0.0107552315418365\\
382	0.0107577084617878\\
383	0.0107602322712147\\
384	0.010762804259551\\
385	0.0107654257937172\\
386	0.0107680983260908\\
387	0.0107708234034773\\
388	0.0107736026772162\\
389	0.010776437914579\\
390	0.0107793310116346\\
391	0.0107822840077848\\
392	0.0107852991021959\\
393	0.0107883786723875\\
394	0.0107915252952693\\
395	0.010794741770959\\
396	0.0107980311497552\\
397	0.0108013967626701\\
398	0.0108048422558438\\
399	0.0108083716295327\\
400	0.0108119892823958\\
401	0.0108157000615815\\
402	0.010819509319323\\
403	0.0108234229768225\\
404	0.0108274475962958\\
405	0.010831590462225\\
406	0.0108358596733072\\
407	0.010840264247912\\
408	0.0108448142501159\\
409	0.0108495209574173\\
410	0.0108543971385928\\
411	0.0108594576730917\\
412	0.0108647213124221\\
413	0.0108702164268151\\
414	0.0108766774365729\\
415	0.0108836668553089\\
416	0.0108907591835057\\
417	0.0108979550636151\\
418	0.0109052553715621\\
419	0.0109126619799257\\
420	0.0109201748625144\\
421	0.0109277817741522\\
422	0.0109354738079255\\
423	0.0109432372541542\\
424	0.0109511253813273\\
425	0.0109591397210496\\
426	0.0109672817936892\\
427	0.0109755531043966\\
428	0.0109839551385272\\
429	0.0109924893564195\\
430	0.0110011571874733\\
431	0.0110099600232569\\
432	0.0110188992087364\\
433	0.0110279760321781\\
434	0.0110371917269777\\
435	0.0110465474564168\\
436	0.01105604430109\\
437	0.0110656832443159\\
438	0.011075465155203\\
439	0.0110853907689842\\
440	0.0110954606641771\\
441	0.0111056752360574\\
442	0.0111160346658579\\
443	0.0111265388850299\\
444	0.0111371875338444\\
445	0.0111479799136248\\
446	0.0111589149321948\\
447	0.0111699910433492\\
448	0.0111812061855367\\
449	0.0111925577401043\\
450	0.011204042581828\\
451	0.0112156562171155\\
452	0.0112273915059033\\
453	0.0112392403998852\\
454	0.0112511939583147\\
455	0.0112632515995506\\
456	0.0112754022603004\\
457	0.011287632793238\\
458	0.0112999275196998\\
459	0.0113122672387725\\
460	0.0113246260726944\\
461	0.0113369594759117\\
462	0.0113472939212519\\
463	0.0113575827214334\\
464	0.011367996979037\\
465	0.0113785348476215\\
466	0.0113891939746021\\
467	0.0113999716439867\\
468	0.0114108647499814\\
469	0.0114218697692738\\
470	0.011432982735025\\
471	0.0114441992153791\\
472	0.011455514308206\\
473	0.0114669225876084\\
474	0.0114784180701259\\
475	0.0114899941171424\\
476	0.0115016434001388\\
477	0.0115133579299799\\
478	0.0115251290859775\\
479	0.011536947611601\\
480	0.0115488036211499\\
481	0.0115606866186767\\
482	0.0115725855281146\\
483	0.0115844887303875\\
484	0.0115963840900195\\
485	0.0116082589167907\\
486	0.0116200997160952\\
487	0.0116318914087207\\
488	0.011643615696632\\
489	0.0116551704574625\\
490	0.0116661567153559\\
491	0.0116772663746034\\
492	0.0116884987836458\\
493	0.0116998533286945\\
494	0.0117113294407693\\
495	0.0117229266232878\\
496	0.0117346444885439\\
497	0.0117464827476359\\
498	0.0117584412535207\\
499	0.0117705200049101\\
500	0.0117827192596956\\
501	0.0117950402275496\\
502	0.0118074829898157\\
503	0.0118200470595363\\
504	0.011832729066091\\
505	0.0118454981484023\\
506	0.0118584090597587\\
507	0.0118714790623285\\
508	0.0118847170630902\\
509	0.0118981368719263\\
510	0.0119117776830312\\
511	0.0119256399778211\\
512	0.0119397437702992\\
513	0.0119540965950328\\
514	0.0119687880522303\\
515	0.0119846257642538\\
516	0.0120006792035105\\
517	0.012016948460691\\
518	0.012033433378248\\
519	0.0120501337038229\\
520	0.0120670487552183\\
521	0.0120841782478731\\
522	0.0121015230854084\\
523	0.0121190769550382\\
524	0.0121368289945457\\
525	0.0121542558384702\\
526	0.0121715689980968\\
527	0.0121894895629998\\
528	0.0122080315784222\\
529	0.0122272072725048\\
530	0.0122470601524105\\
531	0.0122676663885992\\
532	0.0122890490541949\\
533	0.0123112314739784\\
534	0.01233419358811\\
535	0.0123578916613893\\
536	0.0123824196727018\\
537	0.0124080735661055\\
538	0.0124376739012499\\
539	0.0124791957473463\\
540	0.012521320936842\\
541	0.0125628551667318\\
542	0.0126035863118111\\
543	0.0126363933286878\\
544	0.0126629956736317\\
545	0.0126879212814179\\
546	0.0127110005266763\\
547	0.0127327507761874\\
548	0.0127519089492287\\
549	0.0127702379348839\\
550	0.0127883494525872\\
551	0.0128061143307652\\
552	0.0128237030271585\\
553	0.0128414080727687\\
554	0.0128592914679222\\
555	0.0128773788132809\\
556	0.0128956830257798\\
557	0.0129159388381864\\
558	0.0129361789885641\\
559	0.0129562582617236\\
560	0.0129751695978427\\
561	0.0129934934782369\\
562	0.0130118829813184\\
563	0.0130302735780219\\
564	0.013048498827268\\
565	0.013066882287389\\
566	0.0130854146291867\\
567	0.0131041261948512\\
568	0.0131242851927021\\
569	0.0131439992612982\\
570	0.0131622809802794\\
571	0.0131802393603465\\
572	0.0131979881782582\\
573	0.013215611798017\\
574	0.0132333125336978\\
575	0.0132515920089052\\
576	0.0132691646651249\\
577	0.013286128642129\\
578	0.0133028709966482\\
579	0.0133194950173628\\
580	0.0133360358211135\\
581	0.0133524695865987\\
582	0.0133687674364789\\
583	0.0133848973350109\\
584	0.0134008240469672\\
585	0.0134165089013565\\
586	0.0134319095548253\\
587	0.01344697990865\\
588	0.0134616705317511\\
589	0.0134759305303028\\
590	0.0134902483101722\\
591	0.0135046714122469\\
592	0.0135192196273521\\
593	0.0135339766714342\\
594	0.0135491998980231\\
595	0.0135656125153459\\
596	0.0135851815686143\\
597	0.0136131889078992\\
598	0.0136637438596031\\
599	0\\
600	0\\
};
\addplot [color=red!75!mycolor17,solid,forget plot]
  table[row sep=crcr]{%
1	0.0107560004113626\\
2	0.0107560057046717\\
3	0.0107560110930745\\
4	0.0107560165782783\\
5	0.010756022162021\\
6	0.0107560278460718\\
7	0.0107560336322315\\
8	0.0107560395223334\\
9	0.0107560455182434\\
10	0.010756051621861\\
11	0.01075605783512\\
12	0.0107560641599884\\
13	0.01075607059847\\
14	0.0107560771526043\\
15	0.0107560838244673\\
16	0.0107560906161724\\
17	0.0107560975298708\\
18	0.0107561045677524\\
19	0.0107561117320462\\
20	0.0107561190250214\\
21	0.0107561264489876\\
22	0.0107561340062961\\
23	0.0107561416993402\\
24	0.0107561495305562\\
25	0.010756157502424\\
26	0.0107561656174681\\
27	0.0107561738782581\\
28	0.0107561822874099\\
29	0.0107561908475859\\
30	0.0107561995614968\\
31	0.0107562084319014\\
32	0.0107562174616083\\
33	0.0107562266534761\\
34	0.0107562360104151\\
35	0.0107562455353874\\
36	0.0107562552314083\\
37	0.0107562651015473\\
38	0.0107562751489288\\
39	0.0107562853767331\\
40	0.0107562957881976\\
41	0.0107563063866179\\
42	0.0107563171753484\\
43	0.0107563281578037\\
44	0.0107563393374597\\
45	0.0107563507178545\\
46	0.0107563623025897\\
47	0.0107563740953312\\
48	0.0107563860998108\\
49	0.0107563983198272\\
50	0.0107564107592471\\
51	0.0107564234220064\\
52	0.0107564363121116\\
53	0.0107564494336411\\
54	0.0107564627907461\\
55	0.0107564763876524\\
56	0.0107564902286614\\
57	0.0107565043181516\\
58	0.0107565186605798\\
59	0.0107565332604829\\
60	0.0107565481224789\\
61	0.0107565632512683\\
62	0.0107565786516363\\
63	0.0107565943284533\\
64	0.0107566102866771\\
65	0.0107566265313545\\
66	0.0107566430676224\\
67	0.0107566599007097\\
68	0.0107566770359391\\
69	0.0107566944787284\\
70	0.0107567122345926\\
71	0.0107567303091453\\
72	0.0107567487081006\\
73	0.0107567674372749\\
74	0.0107567865025889\\
75	0.0107568059100689\\
76	0.0107568256658493\\
77	0.0107568457761745\\
78	0.0107568662474001\\
79	0.0107568870859961\\
80	0.0107569082985477\\
81	0.0107569298917582\\
82	0.010756951872451\\
83	0.0107569742475712\\
84	0.0107569970241885\\
85	0.0107570202094988\\
86	0.010757043810827\\
87	0.0107570678356288\\
88	0.0107570922914934\\
89	0.0107571171861456\\
90	0.0107571425274485\\
91	0.0107571683234057\\
92	0.0107571945821641\\
93	0.010757221312016\\
94	0.0107572485214021\\
95	0.010757276218914\\
96	0.0107573044132971\\
97	0.0107573331134527\\
98	0.0107573623284415\\
99	0.0107573920674862\\
100	0.010757422339974\\
101	0.0107574531554603\\
102	0.0107574845236709\\
103	0.0107575164545054\\
104	0.0107575489580405\\
105	0.0107575820445327\\
106	0.0107576157244218\\
107	0.010757650008334\\
108	0.0107576849070855\\
109	0.0107577204316854\\
110	0.0107577565933395\\
111	0.0107577934034538\\
112	0.0107578308736376\\
113	0.010757869015708\\
114	0.0107579078416924\\
115	0.0107579473638334\\
116	0.0107579875945918\\
117	0.010758028546651\\
118	0.0107580702329203\\
119	0.0107581126665399\\
120	0.0107581558608839\\
121	0.0107581998295652\\
122	0.0107582445864395\\
123	0.0107582901456095\\
124	0.0107583365214296\\
125	0.0107583837285099\\
126	0.0107584317817211\\
127	0.0107584806961988\\
128	0.0107585304873485\\
129	0.0107585811708504\\
130	0.0107586327626638\\
131	0.0107586852790326\\
132	0.0107587387364898\\
133	0.0107587931518631\\
134	0.0107588485422799\\
135	0.0107589049251726\\
136	0.0107589623182837\\
137	0.0107590207396719\\
138	0.0107590802077172\\
139	0.0107591407411267\\
140	0.0107592023589404\\
141	0.010759265080537\\
142	0.01075932892564\\
143	0.0107593939143237\\
144	0.0107594600670194\\
145	0.0107595274045216\\
146	0.0107595959479945\\
147	0.0107596657189784\\
148	0.0107597367393967\\
149	0.010759809031562\\
150	0.0107598826181834\\
151	0.0107599575223734\\
152	0.010760033767655\\
153	0.0107601113779687\\
154	0.01076019037768\\
155	0.0107602707915871\\
156	0.0107603526449278\\
157	0.0107604359633882\\
158	0.0107605207731094\\
159	0.0107606071006967\\
160	0.0107606949732266\\
161	0.0107607844182559\\
162	0.0107608754638296\\
163	0.0107609681384897\\
164	0.0107610624712838\\
165	0.0107611584917739\\
166	0.0107612562300455\\
167	0.0107613557167166\\
168	0.010761456982947\\
169	0.0107615600604481\\
170	0.0107616649814919\\
171	0.0107617717789212\\
172	0.0107618804861596\\
173	0.0107619911372213\\
174	0.0107621037667216\\
175	0.0107622184098872\\
176	0.0107623351025669\\
177	0.0107624538812425\\
178	0.0107625747830398\\
179	0.0107626978457393\\
180	0.0107628231077884\\
181	0.010762950608312\\
182	0.0107630803871251\\
183	0.0107632124847441\\
184	0.0107633469423993\\
185	0.010763483802047\\
186	0.0107636231063823\\
187	0.0107637648988515\\
188	0.0107639092236656\\
189	0.0107640561258127\\
190	0.0107642056510721\\
191	0.0107643578460276\\
192	0.0107645127580812\\
193	0.0107646704354675\\
194	0.0107648309272677\\
195	0.0107649942834243\\
196	0.0107651605547557\\
197	0.0107653297929714\\
198	0.010765502050687\\
199	0.0107656773814401\\
200	0.0107658558397055\\
201	0.0107660374809116\\
202	0.0107662223614564\\
203	0.0107664105387244\\
204	0.0107666020711029\\
205	0.0107667970179993\\
206	0.0107669954398586\\
207	0.0107671973981806\\
208	0.0107674029555382\\
209	0.0107676121755951\\
210	0.0107678251231247\\
211	0.0107680418640286\\
212	0.0107682624653558\\
213	0.0107684869953218\\
214	0.0107687155233284\\
215	0.0107689481199838\\
216	0.0107691848571225\\
217	0.0107694258078264\\
218	0.010769671046445\\
219	0.0107699206486172\\
220	0.0107701746912928\\
221	0.0107704332527541\\
222	0.0107706964126381\\
223	0.0107709642519596\\
224	0.0107712368531335\\
225	0.0107715142999982\\
226	0.0107717966778393\\
227	0.0107720840734139\\
228	0.010772376574974\\
229	0.010772674272292\\
230	0.0107729772566856\\
231	0.0107732856210429\\
232	0.0107735994598484\\
233	0.0107739188692093\\
234	0.0107742439468819\\
235	0.0107745747922986\\
236	0.0107749115065951\\
237	0.0107752541926383\\
238	0.0107756029550546\\
239	0.010775957900258\\
240	0.0107763191364794\\
241	0.0107766867737958\\
242	0.0107770609241601\\
243	0.0107774417014312\\
244	0.0107778292214049\\
245	0.0107782236018444\\
246	0.0107786249625125\\
247	0.0107790334252026\\
248	0.0107794491137717\\
249	0.010779872154173\\
250	0.0107803026744887\\
251	0.0107807408049642\\
252	0.0107811866780418\\
253	0.0107816404283953\\
254	0.0107821021929649\\
255	0.0107825721109926\\
256	0.0107830503240583\\
257	0.0107835369761157\\
258	0.0107840322135292\\
259	0.0107845361851114\\
260	0.0107850490421601\\
261	0.0107855709384969\\
262	0.0107861020305056\\
263	0.0107866424771711\\
264	0.0107871924401189\\
265	0.0107877520836549\\
266	0.0107883215748057\\
267	0.0107889010833595\\
268	0.0107894907819068\\
269	0.0107900908458827\\
270	0.0107907014536082\\
271	0.0107913227863329\\
272	0.0107919550282781\\
273	0.0107925983666794\\
274	0.0107932529918309\\
275	0.0107939190971286\\
276	0.0107945968791154\\
277	0.0107952865375252\\
278	0.010795988275328\\
279	0.0107967022987757\\
280	0.0107974288174471\\
281	0.0107981680442942\\
282	0.0107989201956885\\
283	0.010799685491467\\
284	0.0108004641549791\\
285	0.0108012564131333\\
286	0.0108020624964442\\
287	0.0108028826390798\\
288	0.0108037170789083\\
289	0.0108045660575453\\
290	0.0108054298204016\\
291	0.0108063086167297\\
292	0.0108072026996711\\
293	0.0108081123263034\\
294	0.0108090377576868\\
295	0.0108099792589106\\
296	0.0108109370991395\\
297	0.010811911551659\\
298	0.0108129028939207\\
299	0.010813911407587\\
300	0.010814937378575\\
301	0.0108159810970992\\
302	0.0108170428577147\\
303	0.0108181229593573\\
304	0.0108192217053844\\
305	0.0108203394036132\\
306	0.0108214763663583\\
307	0.0108226329104675\\
308	0.0108238093573551\\
309	0.0108250060330344\\
310	0.010826223268147\\
311	0.0108274613979899\\
312	0.0108287207625402\\
313	0.0108300017064766\\
314	0.0108313045791973\\
315	0.0108326297348349\\
316	0.0108339775322666\\
317	0.010835348335121\\
318	0.010836742511779\\
319	0.0108381604353704\\
320	0.0108396024837644\\
321	0.0108410690395536\\
322	0.0108425604900315\\
323	0.0108440772271626\\
324	0.0108456196475437\\
325	0.010847188152357\\
326	0.0108487831473131\\
327	0.0108504050425831\\
328	0.0108520542527196\\
329	0.0108537311965649\\
330	0.0108554362971454\\
331	0.0108571699815516\\
332	0.0108589326808018\\
333	0.0108607248296908\\
334	0.0108625468666244\\
335	0.0108643992334509\\
336	0.0108662823753215\\
337	0.0108681967406926\\
338	0.0108701427818187\\
339	0.0108721209568684\\
340	0.0108741317373076\\
341	0.0108761756324051\\
342	0.0108782532697422\\
343	0.0108803656603173\\
344	0.0108825136733782\\
345	0.0108846950074879\\
346	0.0108869099389341\\
347	0.0108891587176373\\
348	0.0108914415609099\\
349	0.0108937586432021\\
350	0.0108961100758233\\
351	0.0108984958600843\\
352	0.0109009157678355\\
353	0.010903369020424\\
354	0.0109058534021411\\
355	0.0109083627740358\\
356	0.0109109142482262\\
357	0.010913509776825\\
358	0.010916150074059\\
359	0.010918835863868\\
360	0.0109215678799238\\
361	0.0109243468656362\\
362	0.010927173574143\\
363	0.0109300487682823\\
364	0.0109329732205445\\
365	0.0109359477129989\\
366	0.0109389730371938\\
367	0.0109420499940228\\
368	0.0109451793935537\\
369	0.0109483620548137\\
370	0.0109515988055234\\
371	0.0109548904817712\\
372	0.0109582379276189\\
373	0.0109616419946274\\
374	0.0109651035412889\\
375	0.0109686234323498\\
376	0.010972202538009\\
377	0.0109758417329675\\
378	0.0109795418953075\\
379	0.0109833039051726\\
380	0.0109871286432153\\
381	0.0109910169887747\\
382	0.0109949698177385\\
383	0.01099898800004\\
384	0.0110030723967259\\
385	0.0110072238565285\\
386	0.0110114432118573\\
387	0.0110157312741172\\
388	0.0110200888282406\\
389	0.0110245166263069\\
390	0.0110290153800974\\
391	0.0110335857524139\\
392	0.0110382283469545\\
393	0.0110429436965023\\
394	0.011047732249122\\
395	0.0110525943519436\\
396	0.0110575302318528\\
397	0.0110625399717327\\
398	0.0110676234923378\\
399	0.0110727805255804\\
400	0.0110780105755054\\
401	0.0110833128798142\\
402	0.0110886863655734\\
403	0.0110941295983592\\
404	0.0110996407240481\\
405	0.0111052174024484\\
406	0.011110856731972\\
407	0.0111165551644781\\
408	0.011122308408644\\
409	0.0111281113158106\\
410	0.0111339577211439\\
411	0.0111398400761058\\
412	0.0111457483486511\\
413	0.0111516660817744\\
414	0.011156964484749\\
415	0.0111619878578353\\
416	0.0111670862602773\\
417	0.0111722603121029\\
418	0.0111775106805761\\
419	0.0111828382298948\\
420	0.0111882435328969\\
421	0.0111937250392224\\
422	0.0111992817844259\\
423	0.0112049117878577\\
424	0.0112106247419682\\
425	0.0112164209651369\\
426	0.0112223007049886\\
427	0.0112282641317941\\
428	0.0112343113313877\\
429	0.0112404422975732\\
430	0.011246656923983\\
431	0.0112529549953617\\
432	0.011259336178302\\
433	0.0112658000116168\\
434	0.0112723458962629\\
435	0.0112789730832777\\
436	0.0112856806618511\\
437	0.0112924675467517\\
438	0.0112993324651341\\
439	0.0113062739427732\\
440	0.0113132902897978\\
441	0.0113203795860234\\
442	0.0113275396660291\\
443	0.0113347681041641\\
444	0.0113420621997308\\
445	0.0113494189626589\\
446	0.0113568351000644\\
447	0.0113643070041782\\
448	0.0113718307422455\\
449	0.0113794020490534\\
450	0.0113870163231759\\
451	0.0113946686344911\\
452	0.011402353738525\\
453	0.0114100660683377\\
454	0.0114177997226611\\
455	0.011425548407258\\
456	0.0114333055592898\\
457	0.0114410643381417\\
458	0.0114488175162195\\
459	0.0114565570982927\\
460	0.0114642733882142\\
461	0.0114719538126416\\
462	0.0114792554329784\\
463	0.0114866145247477\\
464	0.0114940688632244\\
465	0.0115016183534721\\
466	0.011509262862467\\
467	0.011517002226292\\
468	0.0115248362592818\\
469	0.0115327647654544\\
470	0.0115407875525308\\
471	0.0115489044487946\\
472	0.0115571153226117\\
473	0.0115654201064713\\
474	0.0115738188254873\\
475	0.0115823116332402\\
476	0.0115908988541599\\
477	0.0115995810310029\\
478	0.011608358977777\\
479	0.0116172338434812\\
480	0.0116262071176267\\
481	0.0116352806456348\\
482	0.0116444566723041\\
483	0.0116537378811409\\
484	0.0116631274226973\\
485	0.011672628922337\\
486	0.0116822464671054\\
487	0.0116919846370314\\
488	0.0117018488687023\\
489	0.0117118493223799\\
490	0.011722011854751\\
491	0.0117323382542566\\
492	0.0117428300020873\\
493	0.0117534904296702\\
494	0.0117643234498819\\
495	0.0117753332026987\\
496	0.0117865240435149\\
497	0.011797901522342\\
498	0.0118094698782025\\
499	0.0118212320871047\\
500	0.0118331862380586\\
501	0.0118453179781838\\
502	0.0118576821609804\\
503	0.0118703887331382\\
504	0.0118841323940433\\
505	0.0118980764872054\\
506	0.0119122308395106\\
507	0.0119265983264816\\
508	0.011941179670215\\
509	0.0119559759730999\\
510	0.011970987860511\\
511	0.0119862140672117\\
512	0.0120016532879103\\
513	0.0120173076330059\\
514	0.012033118226287\\
515	0.0120484035798358\\
516	0.012063927380855\\
517	0.0120796898961159\\
518	0.0120956921160588\\
519	0.0121119360931016\\
520	0.0121284139056513\\
521	0.0121451067230527\\
522	0.0121619933687936\\
523	0.0121792491352064\\
524	0.0121970856955373\\
525	0.0122154113898604\\
526	0.0122342977760097\\
527	0.0122539065579063\\
528	0.0122742599460032\\
529	0.0122953871171892\\
530	0.0123172649128693\\
531	0.0123399762252244\\
532	0.0123636472920366\\
533	0.0123887825764232\\
534	0.0124248948785538\\
535	0.0124653916607193\\
536	0.0125053032607519\\
537	0.0125444621519648\\
538	0.0125803186824415\\
539	0.0126049937783567\\
540	0.0126277827489329\\
541	0.0126494802269713\\
542	0.0126699082864006\\
543	0.0126880748582784\\
544	0.0127048644068053\\
545	0.0127213251630922\\
546	0.0127375132712684\\
547	0.0127536481786705\\
548	0.0127699093946887\\
549	0.0127863626598456\\
550	0.0128030284818492\\
551	0.0128199335315367\\
552	0.0128371151409249\\
553	0.01285476873447\\
554	0.0128743449254445\\
555	0.0128937619879908\\
556	0.0129129779678697\\
557	0.0129305492966507\\
558	0.0129481015008704\\
559	0.0129657315095173\\
560	0.012983253107484\\
561	0.0130008155217328\\
562	0.0130185598472839\\
563	0.0130364816534675\\
564	0.0130545809182304\\
565	0.0130728998578937\\
566	0.0130926915917238\\
567	0.013112029740397\\
568	0.0131297784569176\\
569	0.0131474445936999\\
570	0.0131648698721828\\
571	0.0131823037325431\\
572	0.0131997893033244\\
573	0.0132177150871859\\
574	0.0132356555798346\\
575	0.0132526603539149\\
576	0.0132694963051091\\
577	0.0132862068721776\\
578	0.0133028805932743\\
579	0.0133194990134759\\
580	0.0133360379891867\\
581	0.0133524707975475\\
582	0.013368768090595\\
583	0.0133848976667294\\
584	0.0134008242015108\\
585	0.0134165089658764\\
586	0.0134319095780783\\
587	0.0134469799154684\\
588	0.013461670533205\\
589	0.0134759305304731\\
590	0.0134902483101722\\
591	0.0135046714122469\\
592	0.0135192196273521\\
593	0.0135339766714342\\
594	0.0135491998980231\\
595	0.0135656125153459\\
596	0.0135851815686143\\
597	0.0136131889078992\\
598	0.0136637438596031\\
599	0\\
600	0\\
};
\addplot [color=red!80!mycolor19,solid,forget plot]
  table[row sep=crcr]{%
1	0.0110113268894945\\
2	0.0110113306646321\\
3	0.0110113345077589\\
4	0.0110113384200993\\
5	0.0110113424028998\\
6	0.0110113464574294\\
7	0.0110113505849797\\
8	0.011011354786866\\
9	0.0110113590644268\\
10	0.0110113634190252\\
11	0.0110113678520483\\
12	0.0110113723649087\\
13	0.0110113769590441\\
14	0.0110113816359182\\
15	0.0110113863970211\\
16	0.0110113912438696\\
17	0.0110113961780081\\
18	0.0110114012010084\\
19	0.011011406314471\\
20	0.0110114115200249\\
21	0.0110114168193288\\
22	0.0110114222140708\\
23	0.0110114277059698\\
24	0.0110114332967754\\
25	0.011011438988269\\
26	0.0110114447822638\\
27	0.0110114506806057\\
28	0.0110114566851739\\
29	0.0110114627978814\\
30	0.0110114690206758\\
31	0.0110114753555396\\
32	0.011011481804491\\
33	0.0110114883695847\\
34	0.0110114950529121\\
35	0.0110115018566026\\
36	0.0110115087828238\\
37	0.0110115158337822\\
38	0.0110115230117244\\
39	0.0110115303189369\\
40	0.011011537757748\\
41	0.0110115453305274\\
42	0.0110115530396877\\
43	0.0110115608876851\\
44	0.0110115688770197\\
45	0.0110115770102368\\
46	0.0110115852899275\\
47	0.0110115937187296\\
48	0.0110116022993283\\
49	0.0110116110344572\\
50	0.0110116199268992\\
51	0.0110116289794871\\
52	0.011011638195105\\
53	0.0110116475766886\\
54	0.0110116571272268\\
55	0.011011666849762\\
56	0.0110116767473915\\
57	0.0110116868232685\\
58	0.0110116970806027\\
59	0.0110117075226618\\
60	0.0110117181527724\\
61	0.0110117289743206\\
62	0.0110117399907539\\
63	0.0110117512055816\\
64	0.0110117626223763\\
65	0.0110117742447749\\
66	0.0110117860764796\\
67	0.0110117981212596\\
68	0.0110118103829517\\
69	0.0110118228654617\\
70	0.0110118355727659\\
71	0.0110118485089122\\
72	0.0110118616780211\\
73	0.0110118750842875\\
74	0.0110118887319818\\
75	0.0110119026254512\\
76	0.0110119167691211\\
77	0.0110119311674966\\
78	0.011011945825164\\
79	0.011011960746792\\
80	0.0110119759371333\\
81	0.0110119914010263\\
82	0.0110120071433964\\
83	0.0110120231692575\\
84	0.011012039483714\\
85	0.0110120560919619\\
86	0.0110120729992908\\
87	0.0110120902110857\\
88	0.0110121077328281\\
89	0.0110121255700984\\
90	0.0110121437285773\\
91	0.0110121622140479\\
92	0.0110121810323971\\
93	0.0110122001896179\\
94	0.0110122196918109\\
95	0.0110122395451868\\
96	0.0110122597560677\\
97	0.0110122803308895\\
98	0.0110123012762042\\
99	0.0110123225986811\\
100	0.0110123443051099\\
101	0.0110123664024023\\
102	0.0110123888975943\\
103	0.0110124117978484\\
104	0.0110124351104559\\
105	0.0110124588428394\\
106	0.0110124830025548\\
107	0.011012507597294\\
108	0.011012532634887\\
109	0.0110125581233049\\
110	0.0110125840706619\\
111	0.0110126104852181\\
112	0.0110126373753821\\
113	0.0110126647497138\\
114	0.0110126926169268\\
115	0.0110127209858914\\
116	0.0110127498656373\\
117	0.0110127792653567\\
118	0.0110128091944065\\
119	0.0110128396623124\\
120	0.0110128706787707\\
121	0.0110129022536522\\
122	0.0110129343970052\\
123	0.0110129671190581\\
124	0.0110130004302235\\
125	0.0110130343411007\\
126	0.0110130688624796\\
127	0.0110131040053439\\
128	0.0110131397808743\\
129	0.0110131762004526\\
130	0.0110132132756648\\
131	0.0110132510183051\\
132	0.0110132894403791\\
133	0.0110133285541083\\
134	0.0110133683719334\\
135	0.0110134089065183\\
136	0.0110134501707545\\
137	0.0110134921777645\\
138	0.0110135349409064\\
139	0.0110135784737781\\
140	0.0110136227902211\\
141	0.0110136679043256\\
142	0.0110137138304342\\
143	0.0110137605831468\\
144	0.0110138081773253\\
145	0.0110138566280977\\
146	0.0110139059508637\\
147	0.0110139561612985\\
148	0.0110140072753586\\
149	0.0110140593092863\\
150	0.0110141122796152\\
151	0.0110141662031747\\
152	0.011014221097096\\
153	0.011014276978817\\
154	0.0110143338660879\\
155	0.0110143917769767\\
156	0.0110144507298749\\
157	0.0110145107435033\\
158	0.0110145718369176\\
159	0.0110146340295145\\
160	0.0110146973410379\\
161	0.0110147617915846\\
162	0.0110148274016111\\
163	0.0110148941919395\\
164	0.0110149621837643\\
165	0.0110150313986588\\
166	0.0110151018585818\\
167	0.0110151735858845\\
168	0.0110152466033175\\
169	0.0110153209340375\\
170	0.0110153966016149\\
171	0.0110154736300409\\
172	0.0110155520437347\\
173	0.0110156318675515\\
174	0.0110157131267897\\
175	0.0110157958471993\\
176	0.011015880054989\\
177	0.011015965776835\\
178	0.0110160530398891\\
179	0.0110161418717864\\
180	0.0110162323006549\\
181	0.0110163243551229\\
182	0.0110164180643288\\
183	0.0110165134579295\\
184	0.0110166105661096\\
185	0.0110167094195904\\
186	0.0110168100496398\\
187	0.0110169124880813\\
188	0.0110170167673041\\
189	0.0110171229202726\\
190	0.0110172309805367\\
191	0.011017340982242\\
192	0.0110174529601399\\
193	0.0110175669495983\\
194	0.0110176829866123\\
195	0.0110178011078152\\
196	0.0110179213504891\\
197	0.0110180437525767\\
198	0.0110181683526926\\
199	0.0110182951901345\\
200	0.0110184243048954\\
201	0.0110185557376755\\
202	0.0110186895298944\\
203	0.0110188257237033\\
204	0.0110189643619979\\
205	0.011019105488431\\
206	0.0110192491474252\\
207	0.0110193953841868\\
208	0.0110195442447184\\
209	0.0110196957758331\\
210	0.0110198500251682\\
211	0.011020007041199\\
212	0.0110201668732534\\
213	0.0110203295715262\\
214	0.0110204951870939\\
215	0.0110206637719295\\
216	0.0110208353789179\\
217	0.0110210100618712\\
218	0.0110211878755444\\
219	0.0110213688756511\\
220	0.0110215531188802\\
221	0.0110217406629113\\
222	0.0110219315664325\\
223	0.0110221258891561\\
224	0.0110223236918367\\
225	0.0110225250362882\\
226	0.0110227299854012\\
227	0.0110229386031614\\
228	0.0110231509546673\\
229	0.0110233671061491\\
230	0.0110235871249868\\
231	0.0110238110797298\\
232	0.0110240390401159\\
233	0.0110242710770905\\
234	0.011024507262827\\
235	0.0110247476707463\\
236	0.0110249923755376\\
237	0.0110252414531787\\
238	0.0110254949809572\\
239	0.0110257530374914\\
240	0.0110260157027521\\
241	0.0110262830580844\\
242	0.0110265551862293\\
243	0.0110268321713468\\
244	0.0110271140990378\\
245	0.0110274010563679\\
246	0.0110276931318897\\
247	0.0110279904156673\\
248	0.0110282929992993\\
249	0.0110286009759437\\
250	0.0110289144403419\\
251	0.0110292334888433\\
252	0.0110295582194311\\
253	0.0110298887317465\\
254	0.0110302251271156\\
255	0.0110305675085742\\
256	0.0110309159808949\\
257	0.0110312706506131\\
258	0.0110316316260539\\
259	0.0110319990173595\\
260	0.0110323729365161\\
261	0.0110327534973822\\
262	0.0110331408157159\\
263	0.0110335350092037\\
264	0.0110339361974885\\
265	0.011034344502199\\
266	0.0110347600469782\\
267	0.0110351829575129\\
268	0.0110356133615633\\
269	0.0110360513889927\\
270	0.0110364971717979\\
271	0.011036950844139\\
272	0.01103741254237\\
273	0.0110378824050698\\
274	0.0110383605730728\\
275	0.0110388471895002\\
276	0.0110393423997912\\
277	0.0110398463517345\\
278	0.0110403591954997\\
279	0.0110408810836694\\
280	0.0110414121712706\\
281	0.011041952615807\\
282	0.0110425025772909\\
283	0.0110430622182754\\
284	0.011043631703886\\
285	0.0110442112018536\\
286	0.0110448008825458\\
287	0.0110454009189994\\
288	0.0110460114869523\\
289	0.0110466327648753\\
290	0.0110472649340037\\
291	0.0110479081783692\\
292	0.0110485626848308\\
293	0.011049228643106\\
294	0.0110499062458016\\
295	0.0110505956884438\\
296	0.0110512971695082\\
297	0.0110520108904489\\
298	0.0110527370557281\\
299	0.0110534758728432\\
300	0.0110542275523552\\
301	0.0110549923079149\\
302	0.0110557703562888\\
303	0.0110565619173842\\
304	0.0110573672142724\\
305	0.011058186473212\\
306	0.0110590199236697\\
307	0.0110598677983403\\
308	0.011060730333165\\
309	0.0110616077673483\\
310	0.0110625003433725\\
311	0.0110634083070104\\
312	0.0110643319073363\\
313	0.0110652713967339\\
314	0.0110662270309021\\
315	0.0110671990688578\\
316	0.0110681877729359\\
317	0.0110691934087859\\
318	0.011070216245365\\
319	0.0110712565549278\\
320	0.011072314613011\\
321	0.0110733906984149\\
322	0.0110744850931795\\
323	0.0110755980825559\\
324	0.0110767299549716\\
325	0.011077881001991\\
326	0.0110790515182692\\
327	0.0110802418014988\\
328	0.0110814521523498\\
329	0.0110826828744022\\
330	0.0110839342740697\\
331	0.0110852066605152\\
332	0.0110865003455558\\
333	0.0110878156435595\\
334	0.01108915287133\\
335	0.0110905123479843\\
336	0.0110918943948259\\
337	0.0110932993352358\\
338	0.0110947274946449\\
339	0.0110961792007985\\
340	0.0110976547849935\\
341	0.011099154586508\\
342	0.011100678967581\\
343	0.0111022283636207\\
344	0.0111038031975171\\
345	0.011105403345186\\
346	0.0111070291553165\\
347	0.0111086809821176\\
348	0.0111103591858499\\
349	0.0111120641329243\\
350	0.0111137961944744\\
351	0.0111155557403061\\
352	0.0111173431194767\\
353	0.0111191586026878\\
354	0.0111210022153724\\
355	0.0111228732544067\\
356	0.0111247750935166\\
357	0.0111267083682706\\
358	0.0111286734887712\\
359	0.0111306708636312\\
360	0.0111327008993839\\
361	0.0111347639998411\\
362	0.0111368605653959\\
363	0.0111389909922651\\
364	0.0111411556716668\\
365	0.0111433549889267\\
366	0.0111455893225098\\
367	0.0111478590429682\\
368	0.0111501645118016\\
369	0.0111525060802205\\
370	0.0111548840878072\\
371	0.0111572988610641\\
372	0.0111597507118427\\
373	0.0111622399356427\\
374	0.0111647668097724\\
375	0.0111673315913599\\
376	0.011169934515205\\
377	0.0111725757914596\\
378	0.0111752556031271\\
379	0.0111779741033681\\
380	0.0111807314126009\\
381	0.0111835276153863\\
382	0.0111863627570851\\
383	0.0111892368402792\\
384	0.011192149820946\\
385	0.0111951016043811\\
386	0.0111980920408634\\
387	0.0112011209210616\\
388	0.011204187971186\\
389	0.0112072928478934\\
390	0.0112104351329604\\
391	0.0112136143277499\\
392	0.0112168298475046\\
393	0.0112200810155127\\
394	0.0112233670572052\\
395	0.0112266870942536\\
396	0.0112300401387341\\
397	0.0112334250873728\\
398	0.0112368407176134\\
399	0.0112402856833753\\
400	0.0112437585104393\\
401	0.0112472575939932\\
402	0.0112507811976449\\
403	0.0112543274542802\\
404	0.0112578943691687\\
405	0.0112614798256961\\
406	0.0112650815939077\\
407	0.0112686973413479\\
408	0.011272324643515\\
409	0.0112759609850413\\
410	0.0112796037263457\\
411	0.0112832499676989\\
412	0.011286896203401\\
413	0.0112905378364156\\
414	0.0112940651081415\\
415	0.0112975687563616\\
416	0.0113011259956259\\
417	0.0113047373442405\\
418	0.0113084033125058\\
419	0.0113121244028058\\
420	0.0113159011126722\\
421	0.0113197339371375\\
422	0.0113236233494445\\
423	0.0113275697786033\\
424	0.0113315735650039\\
425	0.0113356350264169\\
426	0.0113397544571508\\
427	0.0113439321274055\\
428	0.011348168282849\\
429	0.0113524631444036\\
430	0.0113568169082831\\
431	0.0113612297463309\\
432	0.01136570180671\\
433	0.0113702332149925\\
434	0.0113748240757068\\
435	0.0113794744744569\\
436	0.0113841844806643\\
437	0.0113889541510194\\
438	0.0113937835337255\\
439	0.0113986726736303\\
440	0.0114036216183354\\
441	0.011408630425375\\
442	0.0114136991705488\\
443	0.0114188279574836\\
444	0.0114240169284743\\
445	0.0114292662766282\\
446	0.0114345762592995\\
447	0.0114399472127696\\
448	0.0114453795679461\\
449	0.0114508738693692\\
450	0.0114564307972314\\
451	0.011462051191343\\
452	0.0114677360767602\\
453	0.0114734866915618\\
454	0.0114793045164547\\
455	0.0114851913066866\\
456	0.0114911491170186\\
457	0.0114971803174441\\
458	0.0115032875961655\\
459	0.0115094739718772\\
460	0.0115157429445744\\
461	0.0115220991578076\\
462	0.0115285602767802\\
463	0.0115351293988298\\
464	0.0115418084414028\\
465	0.0115485994530532\\
466	0.0115555045519506\\
467	0.0115625259248113\\
468	0.011569665826814\\
469	0.011576926580244\\
470	0.0115843105715111\\
471	0.0115918202461318\\
472	0.0115994581011852\\
473	0.0116072266746456\\
474	0.0116151285310103\\
475	0.0116231662425808\\
476	0.0116313423658861\\
477	0.0116396594119471\\
478	0.0116481198015513\\
479	0.0116567257452577\\
480	0.0116654806395043\\
481	0.0116743888158184\\
482	0.0116834548673434\\
483	0.0116926836607282\\
484	0.0117020803479166\\
485	0.0117116503806736\\
486	0.0117213995366113\\
487	0.0117313339738432\\
488	0.0117414603097175\\
489	0.0117517855108233\\
490	0.0117623162204974\\
491	0.0117730600510751\\
492	0.0117846848635083\\
493	0.0117966360849784\\
494	0.0118087710848869\\
495	0.0118210886492517\\
496	0.0118335832452196\\
497	0.0118462250557034\\
498	0.0118590629481398\\
499	0.0118720983719465\\
500	0.0118853297638309\\
501	0.0118987538190686\\
502	0.0119123819322364\\
503	0.0119261386581738\\
504	0.0119394393424958\\
505	0.0119529683357346\\
506	0.0119667271448993\\
507	0.0119807171131835\\
508	0.0119949393507152\\
509	0.0120093948705986\\
510	0.0120240846035681\\
511	0.0120390087729354\\
512	0.012054166513775\\
513	0.0120695536440546\\
514	0.0120851450454524\\
515	0.0121008147449223\\
516	0.0121167319980734\\
517	0.0121328897458616\\
518	0.0121492694316621\\
519	0.0121658504724211\\
520	0.0121828274656008\\
521	0.012200391844306\\
522	0.0122185534629314\\
523	0.0122373639516898\\
524	0.0122568861919771\\
525	0.0122771432756959\\
526	0.0122981280436679\\
527	0.0123200037486382\\
528	0.0123441135063749\\
529	0.0123693950649721\\
530	0.0124083258512722\\
531	0.0124468460085786\\
532	0.0124846674899516\\
533	0.0125213967902214\\
534	0.0125483389305038\\
535	0.0125700394526752\\
536	0.0125907092933434\\
537	0.0126101954873346\\
538	0.0126280281792202\\
539	0.0126433950276803\\
540	0.0126584046595101\\
541	0.012673295349401\\
542	0.0126881234894621\\
543	0.0127030360854984\\
544	0.0127181359679043\\
545	0.0127334484122643\\
546	0.0127490011044552\\
547	0.0127648170780659\\
548	0.0127809056773097\\
549	0.0127972861353565\\
550	0.0128141964225559\\
551	0.0128329970503251\\
552	0.0128516714703716\\
553	0.0128700447187205\\
554	0.0128868360530066\\
555	0.0129036699683316\\
556	0.0129205718181761\\
557	0.0129372904245884\\
558	0.0129541842441257\\
559	0.012971274017061\\
560	0.0129885620297398\\
561	0.0130060470272279\\
562	0.0130237210647826\\
563	0.0130415890533139\\
564	0.0130609026410381\\
565	0.013079927001214\\
566	0.0130973634653552\\
567	0.0131147325999733\\
568	0.0131318544672218\\
569	0.0131490561835143\\
570	0.0131663382726025\\
571	0.0131836936800372\\
572	0.013201786544463\\
573	0.0132191763080944\\
574	0.0132360655357439\\
575	0.0132527843975138\\
576	0.0132695026506452\\
577	0.0132862082766414\\
578	0.0133028812714239\\
579	0.0133194993844997\\
580	0.0133360381907542\\
581	0.0133524709020724\\
582	0.0133687681411751\\
583	0.0133848976891148\\
584	0.0134008242103469\\
585	0.013416508968873\\
586	0.0134319095789007\\
587	0.0134469799156316\\
588	0.0134616705332227\\
589	0.0134759305304731\\
590	0.0134902483101722\\
591	0.0135046714122469\\
592	0.0135192196273521\\
593	0.0135339766714342\\
594	0.0135491998980231\\
595	0.0135656125153459\\
596	0.0135851815686143\\
597	0.0136131889078992\\
598	0.0136637438596031\\
599	0\\
600	0\\
};
\addplot [color=red,solid,forget plot]
  table[row sep=crcr]{%
1	0.0111473102704322\\
2	0.0111473127269773\\
3	0.0111473152279834\\
4	0.0111473177742563\\
5	0.011147320366616\\
6	0.0111473230058974\\
7	0.0111473256929507\\
8	0.0111473284286414\\
9	0.0111473312138509\\
10	0.0111473340494763\\
11	0.0111473369364313\\
12	0.0111473398756458\\
13	0.0111473428680668\\
14	0.0111473459146586\\
15	0.0111473490164028\\
16	0.0111473521742988\\
17	0.0111473553893642\\
18	0.011147358662635\\
19	0.0111473619951663\\
20	0.0111473653880319\\
21	0.0111473688423254\\
22	0.0111473723591601\\
23	0.0111473759396695\\
24	0.0111473795850078\\
25	0.0111473832963501\\
26	0.0111473870748927\\
27	0.0111473909218538\\
28	0.0111473948384736\\
29	0.0111473988260148\\
30	0.011147402885763\\
31	0.0111474070190274\\
32	0.0111474112271406\\
33	0.0111474155114596\\
34	0.0111474198733659\\
35	0.0111474243142664\\
36	0.0111474288355931\\
37	0.0111474334388043\\
38	0.0111474381253846\\
39	0.0111474428968458\\
40	0.011147447754727\\
41	0.0111474527005951\\
42	0.0111474577360457\\
43	0.0111474628627032\\
44	0.0111474680822217\\
45	0.0111474733962852\\
46	0.0111474788066081\\
47	0.0111474843149364\\
48	0.0111474899230474\\
49	0.0111474956327508\\
50	0.0111475014458893\\
51	0.0111475073643389\\
52	0.0111475133900098\\
53	0.0111475195248467\\
54	0.0111475257708299\\
55	0.0111475321299756\\
56	0.0111475386043364\\
57	0.0111475451960024\\
58	0.0111475519071017\\
59	0.0111475587398009\\
60	0.0111475656963062\\
61	0.0111475727788636\\
62	0.0111475799897601\\
63	0.0111475873313242\\
64	0.0111475948059267\\
65	0.0111476024159815\\
66	0.0111476101639464\\
67	0.0111476180523238\\
68	0.0111476260836616\\
69	0.0111476342605539\\
70	0.0111476425856422\\
71	0.0111476510616157\\
72	0.0111476596912128\\
73	0.0111476684772213\\
74	0.01114767742248\\
75	0.0111476865298791\\
76	0.0111476958023613\\
77	0.0111477052429231\\
78	0.0111477148546151\\
79	0.0111477246405436\\
80	0.0111477346038711\\
81	0.0111477447478179\\
82	0.0111477550756627\\
83	0.0111477655907437\\
84	0.01114777629646\\
85	0.0111477871962723\\
86	0.0111477982937044\\
87	0.0111478095923439\\
88	0.011147821095844\\
89	0.011147832807924\\
90	0.011147844732371\\
91	0.0111478568730408\\
92	0.0111478692338595\\
93	0.0111478818188245\\
94	0.0111478946320058\\
95	0.0111479076775476\\
96	0.0111479209596692\\
97	0.0111479344826669\\
98	0.0111479482509149\\
99	0.0111479622688672\\
100	0.0111479765410584\\
101	0.0111479910721059\\
102	0.0111480058667111\\
103	0.0111480209296605\\
104	0.0111480362658281\\
105	0.0111480518801763\\
106	0.0111480677777578\\
107	0.0111480839637171\\
108	0.0111481004432924\\
109	0.0111481172218171\\
110	0.0111481343047216\\
111	0.0111481516975351\\
112	0.0111481694058872\\
113	0.0111481874355101\\
114	0.0111482057922401\\
115	0.0111482244820196\\
116	0.0111482435108994\\
117	0.0111482628850398\\
118	0.0111482826107137\\
119	0.0111483026943076\\
120	0.0111483231423245\\
121	0.0111483439613854\\
122	0.011148365158232\\
123	0.0111483867397282\\
124	0.0111484087128631\\
125	0.0111484310847527\\
126	0.0111484538626424\\
127	0.0111484770539095\\
128	0.0111485006660653\\
129	0.0111485247067576\\
130	0.0111485491837736\\
131	0.0111485741050417\\
132	0.0111485994786347\\
133	0.0111486253127721\\
134	0.0111486516158229\\
135	0.011148678396308\\
136	0.0111487056629036\\
137	0.0111487334244433\\
138	0.0111487616899214\\
139	0.0111487904684957\\
140	0.0111488197694905\\
141	0.0111488496023995\\
142	0.0111488799768889\\
143	0.0111489109028007\\
144	0.0111489423901557\\
145	0.0111489744491566\\
146	0.0111490070901919\\
147	0.0111490403238385\\
148	0.0111490741608656\\
149	0.011149108612238\\
150	0.0111491436891196\\
151	0.0111491794028774\\
152	0.0111492157650844\\
153	0.0111492527875241\\
154	0.0111492904821938\\
155	0.0111493288613088\\
156	0.011149367937306\\
157	0.0111494077228482\\
158	0.0111494482308281\\
159	0.0111494894743721\\
160	0.0111495314668453\\
161	0.0111495742218549\\
162	0.0111496177532551\\
163	0.0111496620751515\\
164	0.0111497072019056\\
165	0.0111497531481392\\
166	0.0111497999287394\\
167	0.011149847558863\\
168	0.0111498960539419\\
169	0.0111499454296874\\
170	0.0111499957020957\\
171	0.0111500468874528\\
172	0.0111500990023396\\
173	0.0111501520636376\\
174	0.0111502060885339\\
175	0.0111502610945266\\
176	0.0111503170994307\\
177	0.0111503741213839\\
178	0.0111504321788515\\
179	0.0111504912906334\\
180	0.0111505514758692\\
181	0.0111506127540448\\
182	0.0111506751449981\\
183	0.0111507386689257\\
184	0.0111508033463892\\
185	0.0111508691983215\\
186	0.0111509362460335\\
187	0.0111510045112209\\
188	0.0111510740159709\\
189	0.0111511447827693\\
190	0.0111512168345074\\
191	0.0111512901944894\\
192	0.0111513648864393\\
193	0.0111514409345088\\
194	0.0111515183632845\\
195	0.0111515971977959\\
196	0.0111516774635228\\
197	0.0111517591864036\\
198	0.0111518423928429\\
199	0.0111519271097205\\
200	0.0111520133643988\\
201	0.0111521011847319\\
202	0.0111521905990739\\
203	0.0111522816362877\\
204	0.0111523743257542\\
205	0.0111524686973807\\
206	0.0111525647816107\\
207	0.0111526626094328\\
208	0.0111527622123903\\
209	0.0111528636225909\\
210	0.0111529668727165\\
211	0.0111530719960329\\
212	0.0111531790264\\
213	0.0111532879982823\\
214	0.0111533989467587\\
215	0.0111535119075338\\
216	0.0111536269169478\\
217	0.0111537440119883\\
218	0.0111538632303006\\
219	0.0111539846101992\\
220	0.0111541081906792\\
221	0.011154234011428\\
222	0.0111543621128366\\
223	0.0111544925360121\\
224	0.0111546253227891\\
225	0.0111547605157427\\
226	0.0111548981582002\\
227	0.0111550382942542\\
228	0.0111551809687753\\
229	0.0111553262274249\\
230	0.0111554741166686\\
231	0.0111556246837893\\
232	0.0111557779769008\\
233	0.0111559340449616\\
234	0.0111560929377888\\
235	0.011156254706072\\
236	0.0111564194013876\\
237	0.0111565870762135\\
238	0.0111567577839435\\
239	0.0111569315789022\\
240	0.01115710851636\\
241	0.0111572886525482\\
242	0.0111574720446747\\
243	0.0111576587509395\\
244	0.0111578488305501\\
245	0.0111580423437379\\
246	0.0111582393517742\\
247	0.0111584399169865\\
248	0.0111586441027751\\
249	0.0111588519736297\\
250	0.0111590635951465\\
251	0.0111592790340448\\
252	0.0111594983581851\\
253	0.0111597216365856\\
254	0.0111599489394404\\
255	0.0111601803381372\\
256	0.0111604159052752\\
257	0.0111606557146831\\
258	0.0111608998414377\\
259	0.0111611483618822\\
260	0.0111614013536447\\
261	0.0111616588956573\\
262	0.011161921068175\\
263	0.0111621879527944\\
264	0.0111624596324735\\
265	0.0111627361915508\\
266	0.0111630177157649\\
267	0.0111633042922742\\
268	0.0111635960096764\\
269	0.0111638929580291\\
270	0.0111641952288691\\
271	0.0111645029152331\\
272	0.0111648161116779\\
273	0.0111651349143002\\
274	0.0111654594207581\\
275	0.0111657897302906\\
276	0.0111661259437391\\
277	0.0111664681635675\\
278	0.0111668164938833\\
279	0.0111671710404585\\
280	0.0111675319107503\\
281	0.0111678992139222\\
282	0.0111682730608647\\
283	0.0111686535642168\\
284	0.0111690408383868\\
285	0.0111694349995731\\
286	0.0111698361657856\\
287	0.0111702444568667\\
288	0.011170659994512\\
289	0.0111710829022917\\
290	0.0111715133056715\\
291	0.0111719513320332\\
292	0.011172397110696\\
293	0.0111728507729371\\
294	0.0111733124520122\\
295	0.0111737822831767\\
296	0.0111742604037057\\
297	0.0111747469529147\\
298	0.0111752420721799\\
299	0.0111757459049585\\
300	0.0111762585968084\\
301	0.0111767802954084\\
302	0.0111773111505778\\
303	0.011177851314296\\
304	0.0111784009407219\\
305	0.0111789601862125\\
306	0.0111795292093429\\
307	0.0111801081709236\\
308	0.0111806972340199\\
309	0.0111812965639697\\
310	0.0111819063284009\\
311	0.0111825266972494\\
312	0.0111831578427759\\
313	0.0111837999395826\\
314	0.0111844531646293\\
315	0.0111851176972496\\
316	0.0111857937191659\\
317	0.0111864814145041\\
318	0.011187180969808\\
319	0.0111878925740528\\
320	0.0111886164186577\\
321	0.0111893526974983\\
322	0.011190101606918\\
323	0.0111908633457379\\
324	0.0111916381152674\\
325	0.011192426119312\\
326	0.0111932275641813\\
327	0.0111940426586961\\
328	0.0111948716141934\\
329	0.0111957146445319\\
330	0.0111965719660951\\
331	0.0111974437977952\\
332	0.0111983303610746\\
333	0.0111992318799085\\
334	0.0112001485808067\\
335	0.011201080692816\\
336	0.0112020284475241\\
337	0.0112029920790656\\
338	0.0112039718241326\\
339	0.0112049679219903\\
340	0.0112059806145044\\
341	0.0112070101461895\\
342	0.011208056764326\\
343	0.0112091207193404\\
344	0.0112102022661628\\
345	0.0112113016661054\\
346	0.0112124191835345\\
347	0.0112135550858418\\
348	0.0112147096433907\\
349	0.0112158831294242\\
350	0.0112170758199153\\
351	0.0112182879933285\\
352	0.011219519930215\\
353	0.0112207719124187\\
354	0.0112220442211199\\
355	0.0112233371311262\\
356	0.0112246508941022\\
357	0.0112259857587908\\
358	0.0112273419723362\\
359	0.0112287197799926\\
360	0.0112301194248212\\
361	0.0112315411473759\\
362	0.0112329851853761\\
363	0.01123445177337\\
364	0.0112359411423863\\
365	0.0112374535195782\\
366	0.0112389891278582\\
367	0.0112405481855274\\
368	0.0112421309058998\\
369	0.0112437374969252\\
370	0.0112453681608113\\
371	0.0112470230936513\\
372	0.0112487024850573\\
373	0.0112504065178066\\
374	0.0112521353675051\\
375	0.011253889202273\\
376	0.0112556681824605\\
377	0.0112574724604017\\
378	0.0112593021802138\\
379	0.0112611574776544\\
380	0.0112630384800454\\
381	0.0112649453062801\\
382	0.0112668780669247\\
383	0.0112688368644336\\
384	0.0112708217934956\\
385	0.0112728329415323\\
386	0.011274870389372\\
387	0.0112769342121245\\
388	0.0112790244802867\\
389	0.0112811412611087\\
390	0.0112832846202571\\
391	0.0112854546238134\\
392	0.0112876513406494\\
393	0.0112898748452273\\
394	0.0112921252208737\\
395	0.0112944025635879\\
396	0.0112967069864456\\
397	0.0112990386246729\\
398	0.0113013976414172\\
399	0.0113037842343011\\
400	0.0113061986428035\\
401	0.0113086411564646\\
402	0.0113111121239322\\
403	0.0113136119628371\\
404	0.0113161411704452\\
405	0.0113187003349482\\
406	0.0113212901470929\\
407	0.0113239114115065\\
408	0.0113265650564488\\
409	0.0113292521398623\\
410	0.0113319738499135\\
411	0.0113347315064679\\
412	0.0113375266058579\\
413	0.0113403610395717\\
414	0.0113432408659266\\
415	0.0113461693643335\\
416	0.0113491472737491\\
417	0.0113521753452866\\
418	0.0113552543429474\\
419	0.0113583850444081\\
420	0.0113615682417523\\
421	0.0113648047420469\\
422	0.0113680953682887\\
423	0.0113714409607659\\
424	0.0113748423798136\\
425	0.0113783005065397\\
426	0.0113818162431911\\
427	0.011385390512941\\
428	0.0113890242600531\\
429	0.0113927184512245\\
430	0.0113964740770638\\
431	0.0114002921537191\\
432	0.011404173724678\\
433	0.0114081198627595\\
434	0.0114121316723295\\
435	0.0114162102917705\\
436	0.0114203568962482\\
437	0.0114245727008251\\
438	0.0114288589639846\\
439	0.0114332169916449\\
440	0.0114376481417603\\
441	0.0114421538296371\\
442	0.0114467355341217\\
443	0.0114513948048625\\
444	0.0114561332708902\\
445	0.0114609526507881\\
446	0.0114658547646797\\
447	0.0114708415483316\\
448	0.0114759150724162\\
449	0.0114810775146525\\
450	0.0114863311410385\\
451	0.0114916783040837\\
452	0.0114971214393143\\
453	0.0115026630596356\\
454	0.0115083057470238\\
455	0.0115140521407895\\
456	0.0115199049220752\\
457	0.0115258667950607\\
458	0.0115319404682885\\
459	0.0115381286445818\\
460	0.011544434024117\\
461	0.0115508592630822\\
462	0.0115574063440456\\
463	0.0115640770860273\\
464	0.0115708731461204\\
465	0.0115777958834864\\
466	0.0115848482998898\\
467	0.0115920335970919\\
468	0.0115993551458568\\
469	0.0116068165037789\\
470	0.0116144214354527\\
471	0.0116221739351382\\
472	0.0116300782532637\\
473	0.0116381389273681\\
474	0.0116463608188084\\
475	0.0116547491591382\\
476	0.0116633096165548\\
477	0.011672048420183\\
478	0.0116809726580689\\
479	0.0116905204863379\\
480	0.0117005044269762\\
481	0.0117106532130661\\
482	0.0117209681075114\\
483	0.0117314501600944\\
484	0.0117421001737969\\
485	0.0117529186690691\\
486	0.0117639058492852\\
487	0.0117750615454548\\
488	0.0117863852481278\\
489	0.0117978764165742\\
490	0.0118095363454837\\
491	0.0118213717483199\\
492	0.0118328210307557\\
493	0.0118443243259626\\
494	0.0118560051910933\\
495	0.0118678939000001\\
496	0.0118799920426209\\
497	0.0118922968115098\\
498	0.0119048192289019\\
499	0.0119175617424726\\
500	0.0119305261740929\\
501	0.0119437134999593\\
502	0.011957121494349\\
503	0.0119707250869657\\
504	0.0119844175624323\\
505	0.0119983501189346\\
506	0.012012523925746\\
507	0.0120269399018215\\
508	0.0120415986912665\\
509	0.0120565005861726\\
510	0.0120716454611987\\
511	0.0120870327709694\\
512	0.012102662138615\\
513	0.0121185333071197\\
514	0.0121346391863825\\
515	0.0121509646394995\\
516	0.0121674846253816\\
517	0.0121843730191093\\
518	0.0122018418940742\\
519	0.0122199093285341\\
520	0.0122386373285286\\
521	0.0122580808793175\\
522	0.0122795220208374\\
523	0.0123016310832037\\
524	0.012324565846525\\
525	0.0123504689525452\\
526	0.0123881498461421\\
527	0.012424800444479\\
528	0.0124595768080078\\
529	0.0124933976748602\\
530	0.012514740800854\\
531	0.012534652282812\\
532	0.0125534699405362\\
533	0.0125709809393902\\
534	0.0125859580147234\\
535	0.012599834001914\\
536	0.0126135773576437\\
537	0.0126272341096508\\
538	0.0126408953663348\\
539	0.012654733078678\\
540	0.0126687744623121\\
541	0.0126830426454391\\
542	0.0126975607232427\\
543	0.012712343339761\\
544	0.0127273981388023\\
545	0.0127427298886337\\
546	0.0127583537406449\\
547	0.0127743389760305\\
548	0.0127924266957907\\
549	0.0128103931501383\\
550	0.0128280275808347\\
551	0.0128441396444115\\
552	0.0128603480007628\\
553	0.0128766037615932\\
554	0.0128926685840816\\
555	0.0129088943648097\\
556	0.0129253240933278\\
557	0.0129419662203101\\
558	0.0129588173154488\\
559	0.012975873011209\\
560	0.0129931272774625\\
561	0.0130105799293654\\
562	0.0130291935362831\\
563	0.013048015192612\\
564	0.0130652952278806\\
565	0.0130823786395838\\
566	0.0130992303318434\\
567	0.0131161823503507\\
568	0.0131332384982756\\
569	0.013150388897885\\
570	0.0131678002979293\\
571	0.0131857189375081\\
572	0.0132026379726609\\
573	0.0132193845885735\\
574	0.0132360735367549\\
575	0.0132527850905522\\
576	0.0132695028713411\\
577	0.0132862083899486\\
578	0.0133028813326694\\
579	0.0133194994167125\\
580	0.013336038206793\\
581	0.0133524709094905\\
582	0.013368768144301\\
583	0.0133848976902849\\
584	0.0134008242107216\\
585	0.0134165089689696\\
586	0.0134319095789186\\
587	0.0134469799156334\\
588	0.0134616705332227\\
589	0.0134759305304731\\
590	0.0134902483101722\\
591	0.0135046714122469\\
592	0.0135192196273521\\
593	0.0135339766714342\\
594	0.0135491998980231\\
595	0.0135656125153459\\
596	0.0135851815686143\\
597	0.0136131889078992\\
598	0.0136637438596031\\
599	0\\
600	0\\
};
\addplot [color=mycolor20,solid,forget plot]
  table[row sep=crcr]{%
1	0.0112011782251379\\
2	0.0112011799967281\\
3	0.0112011818005927\\
4	0.0112011836373212\\
5	0.0112011855075136\\
6	0.0112011874117807\\
7	0.0112011893507449\\
8	0.0112011913250396\\
9	0.0112011933353101\\
10	0.0112011953822134\\
11	0.0112011974664184\\
12	0.0112011995886065\\
13	0.0112012017494714\\
14	0.0112012039497195\\
15	0.0112012061900702\\
16	0.0112012084712562\\
17	0.0112012107940234\\
18	0.0112012131591314\\
19	0.0112012155673539\\
20	0.0112012180194785\\
21	0.0112012205163075\\
22	0.0112012230586577\\
23	0.011201225647361\\
24	0.0112012282832644\\
25	0.0112012309672307\\
26	0.0112012337001381\\
27	0.0112012364828813\\
28	0.0112012393163712\\
29	0.0112012422015354\\
30	0.0112012451393184\\
31	0.0112012481306824\\
32	0.0112012511766068\\
33	0.0112012542780892\\
34	0.0112012574361455\\
35	0.01120126065181\\
36	0.0112012639261364\\
37	0.0112012672601973\\
38	0.0112012706550852\\
39	0.0112012741119125\\
40	0.0112012776318122\\
41	0.0112012812159379\\
42	0.0112012848654645\\
43	0.0112012885815884\\
44	0.0112012923655279\\
45	0.0112012962185238\\
46	0.0112013001418395\\
47	0.0112013041367618\\
48	0.0112013082046008\\
49	0.011201312346691\\
50	0.0112013165643911\\
51	0.0112013208590851\\
52	0.011201325232182\\
53	0.011201329685117\\
54	0.0112013342193514\\
55	0.0112013388363736\\
56	0.0112013435376991\\
57	0.0112013483248713\\
58	0.0112013531994621\\
59	0.011201358163072\\
60	0.0112013632173312\\
61	0.0112013683638996\\
62	0.0112013736044677\\
63	0.0112013789407572\\
64	0.0112013843745212\\
65	0.0112013899075452\\
66	0.0112013955416473\\
67	0.0112014012786793\\
68	0.0112014071205268\\
69	0.0112014130691102\\
70	0.0112014191263851\\
71	0.0112014252943432\\
72	0.0112014315750128\\
73	0.0112014379704592\\
74	0.0112014444827862\\
75	0.0112014511141358\\
76	0.0112014578666897\\
77	0.0112014647426696\\
78	0.0112014717443381\\
79	0.0112014788739994\\
80	0.0112014861340002\\
81	0.0112014935267303\\
82	0.0112015010546234\\
83	0.0112015087201581\\
84	0.0112015165258585\\
85	0.0112015244742953\\
86	0.0112015325680864\\
87	0.011201540809898\\
88	0.0112015492024453\\
89	0.0112015577484934\\
90	0.0112015664508586\\
91	0.0112015753124087\\
92	0.0112015843360645\\
93	0.0112015935248007\\
94	0.0112016028816466\\
95	0.0112016124096873\\
96	0.0112016221120649\\
97	0.0112016319919791\\
98	0.0112016420526888\\
99	0.0112016522975129\\
100	0.0112016627298313\\
101	0.0112016733530864\\
102	0.0112016841707839\\
103	0.0112016951864941\\
104	0.0112017064038532\\
105	0.0112017178265644\\
106	0.011201729458399\\
107	0.0112017413031981\\
108	0.0112017533648734\\
109	0.0112017656474089\\
110	0.0112017781548618\\
111	0.0112017908913645\\
112	0.0112018038611252\\
113	0.0112018170684301\\
114	0.0112018305176441\\
115	0.0112018442132128\\
116	0.0112018581596639\\
117	0.0112018723616084\\
118	0.0112018868237427\\
119	0.0112019015508494\\
120	0.0112019165477998\\
121	0.0112019318195548\\
122	0.0112019473711671\\
123	0.0112019632077823\\
124	0.0112019793346412\\
125	0.0112019957570816\\
126	0.0112020124805392\\
127	0.0112020295105507\\
128	0.0112020468527546\\
129	0.0112020645128936\\
130	0.0112020824968166\\
131	0.0112021008104803\\
132	0.0112021194599515\\
133	0.0112021384514091\\
134	0.0112021577911459\\
135	0.0112021774855712\\
136	0.0112021975412126\\
137	0.0112022179647182\\
138	0.0112022387628589\\
139	0.0112022599425309\\
140	0.0112022815107575\\
141	0.011202303474692\\
142	0.0112023258416198\\
143	0.0112023486189609\\
144	0.0112023718142725\\
145	0.0112023954352513\\
146	0.0112024194897364\\
147	0.0112024439857117\\
148	0.0112024689313086\\
149	0.0112024943348089\\
150	0.0112025202046474\\
151	0.0112025465494149\\
152	0.0112025733778609\\
153	0.0112026006988967\\
154	0.0112026285215983\\
155	0.0112026568552095\\
156	0.0112026857091449\\
157	0.0112027150929934\\
158	0.0112027450165208\\
159	0.0112027754896737\\
160	0.0112028065225827\\
161	0.0112028381255654\\
162	0.0112028703091302\\
163	0.0112029030839801\\
164	0.0112029364610156\\
165	0.011202970451339\\
166	0.0112030050662577\\
167	0.0112030403172881\\
168	0.0112030762161599\\
169	0.0112031127748191\\
170	0.011203150005433\\
171	0.0112031879203936\\
172	0.0112032265323222\\
173	0.0112032658540731\\
174	0.0112033058987385\\
175	0.0112033466796524\\
176	0.0112033882103955\\
177	0.0112034305047993\\
178	0.0112034735769511\\
179	0.0112035174411983\\
180	0.0112035621121538\\
181	0.0112036076047002\\
182	0.0112036539339954\\
183	0.0112037011154769\\
184	0.0112037491648679\\
185	0.0112037980981818\\
186	0.0112038479317277\\
187	0.0112038986821162\\
188	0.0112039503662644\\
189	0.0112040030014018\\
190	0.0112040566050762\\
191	0.0112041111951591\\
192	0.011204166789852\\
193	0.0112042234076924\\
194	0.0112042810675595\\
195	0.0112043397886813\\
196	0.0112043995906402\\
197	0.0112044604933797\\
198	0.0112045225172113\\
199	0.0112045856828209\\
200	0.0112046500112754\\
201	0.0112047155240304\\
202	0.0112047822429364\\
203	0.0112048501902468\\
204	0.0112049193886243\\
205	0.0112049898611493\\
206	0.0112050616313269\\
207	0.0112051347230945\\
208	0.0112052091608301\\
209	0.01120528496936\\
210	0.0112053621739668\\
211	0.0112054408003978\\
212	0.0112055208748735\\
213	0.0112056024240956\\
214	0.0112056854752563\\
215	0.0112057700560469\\
216	0.0112058561946664\\
217	0.0112059439198311\\
218	0.0112060332607837\\
219	0.0112061242473024\\
220	0.0112062169097111\\
221	0.0112063112788883\\
222	0.0112064073862779\\
223	0.0112065052638984\\
224	0.0112066049443535\\
225	0.0112067064608427\\
226	0.0112068098471713\\
227	0.0112069151377614\\
228	0.0112070223676628\\
229	0.0112071315725639\\
230	0.011207242788803\\
231	0.0112073560533795\\
232	0.0112074714039659\\
233	0.0112075888789189\\
234	0.0112077085172918\\
235	0.0112078303588465\\
236	0.0112079544440655\\
237	0.0112080808141649\\
238	0.0112082095111062\\
239	0.01120834057761\\
240	0.0112084740571685\\
241	0.0112086099940586\\
242	0.0112087484333557\\
243	0.011208889420947\\
244	0.011209033003545\\
245	0.0112091792287022\\
246	0.0112093281448242\\
247	0.0112094798011848\\
248	0.0112096342479404\\
249	0.0112097915361442\\
250	0.0112099517177615\\
251	0.0112101148456848\\
252	0.0112102809737489\\
253	0.0112104501567461\\
254	0.0112106224504424\\
255	0.0112107979115929\\
256	0.0112109765979577\\
257	0.0112111585683184\\
258	0.0112113438824943\\
259	0.011211532601359\\
260	0.0112117247868568\\
261	0.0112119205020201\\
262	0.0112121198109859\\
263	0.0112123227790136\\
264	0.0112125294725019\\
265	0.0112127399590064\\
266	0.0112129543072578\\
267	0.0112131725871792\\
268	0.0112133948699043\\
269	0.0112136212277956\\
270	0.0112138517344629\\
271	0.0112140864647814\\
272	0.0112143254949105\\
273	0.0112145689023124\\
274	0.0112148167657712\\
275	0.0112150691654116\\
276	0.0112153261827181\\
277	0.0112155879005541\\
278	0.0112158544031813\\
279	0.0112161257762791\\
280	0.0112164021069643\\
281	0.0112166834838102\\
282	0.0112169699968666\\
283	0.0112172617376797\\
284	0.0112175587993117\\
285	0.0112178612763608\\
286	0.0112181692649812\\
287	0.0112184828629032\\
288	0.0112188021694534\\
289	0.0112191272855747\\
290	0.0112194583138468\\
291	0.0112197953585066\\
292	0.0112201385254683\\
293	0.0112204879223442\\
294	0.0112208436584652\\
295	0.0112212058449017\\
296	0.011221574594484\\
297	0.0112219500218237\\
298	0.0112223322433345\\
299	0.0112227213772532\\
300	0.0112231175436618\\
301	0.0112235208645085\\
302	0.0112239314636297\\
303	0.0112243494667723\\
304	0.0112247750016162\\
305	0.0112252081977967\\
306	0.0112256491869282\\
307	0.0112260981026274\\
308	0.011226555080538\\
309	0.0112270202583552\\
310	0.0112274937758511\\
311	0.0112279757749011\\
312	0.0112284663995106\\
313	0.0112289657958432\\
314	0.0112294741122495\\
315	0.011229991499297\\
316	0.0112305181098013\\
317	0.0112310540988585\\
318	0.011231599623879\\
319	0.0112321548446228\\
320	0.0112327199232361\\
321	0.0112332950242897\\
322	0.0112338803148189\\
323	0.0112344759643651\\
324	0.0112350821450188\\
325	0.0112356990314649\\
326	0.011236326801028\\
327	0.0112369656337208\\
328	0.0112376157122921\\
329	0.011238277222276\\
330	0.0112389503520416\\
331	0.0112396352928422\\
332	0.0112403322388634\\
333	0.0112410413872696\\
334	0.0112417629382481\\
335	0.0112424970950486\\
336	0.0112432440640184\\
337	0.0112440040546304\\
338	0.0112447772795042\\
339	0.0112455639544163\\
340	0.0112463642982997\\
341	0.0112471785332304\\
342	0.0112480068844006\\
343	0.0112488495800786\\
344	0.0112497068515277\\
345	0.0112505789328555\\
346	0.0112514660609589\\
347	0.0112523684754778\\
348	0.011253286418743\\
349	0.0112542201357159\\
350	0.0112551698739232\\
351	0.0112561358833897\\
352	0.0112571184165753\\
353	0.0112581177283186\\
354	0.0112591340757955\\
355	0.0112601677185754\\
356	0.0112612189192329\\
357	0.0112622879434293\\
358	0.0112633750599579\\
359	0.0112644805407952\\
360	0.0112656046611613\\
361	0.0112667476995882\\
362	0.0112679099379983\\
363	0.0112690916617956\\
364	0.0112702931599693\\
365	0.0112715147252129\\
366	0.0112727566540603\\
367	0.0112740192470409\\
368	0.0112753028088574\\
369	0.0112766076485859\\
370	0.0112779340799054\\
371	0.0112792824213557\\
372	0.0112806529966303\\
373	0.0112820461349056\\
374	0.0112834621712126\\
375	0.0112849014468532\\
376	0.0112863643098675\\
377	0.0112878511155551\\
378	0.0112893622270574\\
379	0.0112908980160046\\
380	0.0112924588632329\\
381	0.0112940451595784\\
382	0.0112956573067507\\
383	0.0112972957182934\\
384	0.0112989608206337\\
385	0.0113006530542256\\
386	0.0113023728747881\\
387	0.0113041207546382\\
388	0.0113058971841162\\
389	0.011307702673097\\
390	0.0113095377525747\\
391	0.0113114029763052\\
392	0.0113132989224773\\
393	0.0113152261953781\\
394	0.0113171854269967\\
395	0.0113191772784982\\
396	0.0113212024414711\\
397	0.0113232616388324\\
398	0.0113253556252256\\
399	0.0113274851871978\\
400	0.0113296511440915\\
401	0.0113318543487182\\
402	0.0113340956877552\\
403	0.0113363760818018\\
404	0.0113386964850165\\
405	0.0113410578842378\\
406	0.0113434612974713\\
407	0.0113459077716017\\
408	0.0113483983792447\\
409	0.0113509342149807\\
410	0.0113535163922435\\
411	0.011356146044051\\
412	0.0113588243302582\\
413	0.0113615524348826\\
414	0.0113643313799227\\
415	0.0113671621095503\\
416	0.0113700455847795\\
417	0.0113729827843173\\
418	0.0113759747057078\\
419	0.0113790223668585\\
420	0.0113821268080629\\
421	0.0113852890946732\\
422	0.0113885103206019\\
423	0.0113917916128862\\
424	0.0113951341375146\\
425	0.0113985391068362\\
426	0.0114020077888496\\
427	0.0114055415200318\\
428	0.0114091416976793\\
429	0.0114128097568815\\
430	0.0114165471707832\\
431	0.0114203554505699\\
432	0.0114242361451091\\
433	0.0114281908401637\\
434	0.0114322211570809\\
435	0.0114363287508342\\
436	0.0114405153072744\\
437	0.011444782539418\\
438	0.0114491321825672\\
439	0.0114535659880168\\
440	0.0114580857150578\\
441	0.0114626931209378\\
442	0.0114673899483952\\
443	0.011472177910368\\
444	0.0114770586715192\\
445	0.0114820338263178\\
446	0.0114871048729088\\
447	0.0114922731763596\\
448	0.011497539878388\\
449	0.011502906924413\\
450	0.0115083767907457\\
451	0.0115139520612988\\
452	0.0115196354368832\\
453	0.0115254297460682\\
454	0.0115313379580668\\
455	0.0115373631984863\\
456	0.0115435087673774\\
457	0.011549778161005\\
458	0.0115561750983072\\
459	0.0115627035521213\\
460	0.0115693677827651\\
461	0.0115761723752888\\
462	0.0115831223321389\\
463	0.0115902232038116\\
464	0.0115974814175953\\
465	0.0116054693123334\\
466	0.0116136130143116\\
467	0.0116218957646072\\
468	0.011630318948227\\
469	0.01163888373421\\
470	0.0116475909885039\\
471	0.0116564415174651\\
472	0.0116654359411272\\
473	0.0116745746217214\\
474	0.0116838578127286\\
475	0.0116932858082149\\
476	0.0117028602097702\\
477	0.0117125822800973\\
478	0.0117224579922088\\
479	0.0117321264017504\\
480	0.0117417390738246\\
481	0.0117515218576962\\
482	0.0117614769395935\\
483	0.0117716064679172\\
484	0.0117819125050376\\
485	0.0117923969951987\\
486	0.0118030617210243\\
487	0.0118139087702765\\
488	0.0118249385493135\\
489	0.0118361476498823\\
490	0.0118475167294026\\
491	0.0118590600272656\\
492	0.0118706883881438\\
493	0.0118825050948001\\
494	0.0118945395506284\\
495	0.011906799486869\\
496	0.0119192874847334\\
497	0.0119320059738172\\
498	0.0119449571486634\\
499	0.0119581429964553\\
500	0.0119715652296089\\
501	0.0119852251124589\\
502	0.0119991230552591\\
503	0.0120132587699048\\
504	0.0120276364674216\\
505	0.0120422563024704\\
506	0.0120571178099907\\
507	0.0120722197737664\\
508	0.012087560591793\\
509	0.0121031388393444\\
510	0.0121189534581114\\
511	0.0121350029211934\\
512	0.0121512724497392\\
513	0.0121677458213481\\
514	0.012184548696405\\
515	0.0122019601923641\\
516	0.012221190571426\\
517	0.0122409443092564\\
518	0.0122612294233469\\
519	0.0122821227877054\\
520	0.0123037993365533\\
521	0.0123292709005782\\
522	0.0123642235092747\\
523	0.012398281359588\\
524	0.0124314689725784\\
525	0.0124619690331083\\
526	0.012481401372793\\
527	0.0124997518719848\\
528	0.0125167592473171\\
529	0.0125325076729113\\
530	0.0125454955282533\\
531	0.0125582429988684\\
532	0.0125708758906019\\
533	0.012583449496072\\
534	0.0125961214399987\\
535	0.0126089786809454\\
536	0.0126220434902936\\
537	0.012635338574974\\
538	0.0126488834125194\\
539	0.0126626860399153\\
540	0.0126767537815583\\
541	0.012691092848305\\
542	0.0127057068913743\\
543	0.0127206055553765\\
544	0.0127358087999329\\
545	0.0127528558503585\\
546	0.0127701680106764\\
547	0.0127872922293651\\
548	0.0128027255822829\\
549	0.012818260263768\\
550	0.0128338478465422\\
551	0.0128492851458301\\
552	0.01286492608112\\
553	0.0128807601707231\\
554	0.0128967675656984\\
555	0.012912965406474\\
556	0.0129293765963599\\
557	0.012945997188559\\
558	0.0129628230671786\\
559	0.0129798480402323\\
560	0.0129975888140626\\
561	0.0130162293940542\\
562	0.0130335904484122\\
563	0.0130503947638634\\
564	0.0130669819616154\\
565	0.0130836539210182\\
566	0.0131004466716772\\
567	0.013117352028506\\
568	0.0131343604620953\\
569	0.0131517943041389\\
570	0.0131693600552189\\
571	0.0131861241189776\\
572	0.0132027353319319\\
573	0.0132193852557846\\
574	0.0132360736255491\\
575	0.0132527851257368\\
576	0.0132695028897663\\
577	0.013286208399693\\
578	0.0133028813376224\\
579	0.0133194994190823\\
580	0.0133360382078424\\
581	0.0133524709099125\\
582	0.0133687681444512\\
583	0.0133848976903304\\
584	0.0134008242107326\\
585	0.0134165089689715\\
586	0.0134319095789188\\
587	0.0134469799156334\\
588	0.0134616705332227\\
589	0.0134759305304731\\
590	0.0134902483101722\\
591	0.0135046714122469\\
592	0.0135192196273521\\
593	0.0135339766714342\\
594	0.0135491998980231\\
595	0.0135656125153459\\
596	0.0135851815686143\\
597	0.0136131889078992\\
598	0.0136637438596031\\
599	0\\
600	0\\
};
\addplot [color=mycolor21,solid,forget plot]
  table[row sep=crcr]{%
1	0.0112221845287511\\
2	0.0112221859756682\\
3	0.011222187449112\\
4	0.0112221889495701\\
5	0.0112221904775395\\
6	0.0112221920335261\\
7	0.0112221936180453\\
8	0.0112221952316218\\
9	0.0112221968747901\\
10	0.0112221985480948\\
11	0.0112222002520901\\
12	0.0112222019873409\\
13	0.0112222037544221\\
14	0.0112222055539196\\
15	0.0112222073864299\\
16	0.0112222092525606\\
17	0.0112222111529305\\
18	0.0112222130881697\\
19	0.0112222150589202\\
20	0.0112222170658357\\
21	0.0112222191095821\\
22	0.0112222211908374\\
23	0.0112222233102924\\
24	0.0112222254686505\\
25	0.011222227666628\\
26	0.0112222299049547\\
27	0.0112222321843738\\
28	0.0112222345056421\\
29	0.0112222368695307\\
30	0.0112222392768246\\
31	0.0112222417283235\\
32	0.0112222442248421\\
33	0.0112222467672098\\
34	0.0112222493562715\\
35	0.0112222519928878\\
36	0.0112222546779351\\
37	0.0112222574123062\\
38	0.0112222601969103\\
39	0.0112222630326735\\
40	0.0112222659205388\\
41	0.011222268861467\\
42	0.0112222718564366\\
43	0.011222274906444\\
44	0.0112222780125044\\
45	0.0112222811756515\\
46	0.0112222843969383\\
47	0.0112222876774373\\
48	0.0112222910182408\\
49	0.0112222944204615\\
50	0.0112222978852325\\
51	0.0112223014137081\\
52	0.0112223050070638\\
53	0.0112223086664971\\
54	0.0112223123932275\\
55	0.0112223161884973\\
56	0.0112223200535717\\
57	0.0112223239897394\\
58	0.011222327998313\\
59	0.0112223320806295\\
60	0.0112223362380505\\
61	0.0112223404719632\\
62	0.0112223447837803\\
63	0.0112223491749408\\
64	0.0112223536469102\\
65	0.0112223582011816\\
66	0.0112223628392754\\
67	0.0112223675627405\\
68	0.0112223723731545\\
69	0.0112223772721242\\
70	0.0112223822612863\\
71	0.011222387342308\\
72	0.0112223925168872\\
73	0.0112223977867536\\
74	0.011222403153669\\
75	0.0112224086194277\\
76	0.0112224141858577\\
77	0.0112224198548208\\
78	0.0112224256282132\\
79	0.0112224315079667\\
80	0.0112224374960489\\
81	0.0112224435944638\\
82	0.0112224498052529\\
83	0.0112224561304956\\
84	0.01122246257231\\
85	0.0112224691328534\\
86	0.0112224758143235\\
87	0.0112224826189588\\
88	0.0112224895490394\\
89	0.0112224966068878\\
90	0.01122250379487\\
91	0.0112225111153957\\
92	0.0112225185709197\\
93	0.0112225261639425\\
94	0.0112225338970109\\
95	0.0112225417727194\\
96	0.0112225497937108\\
97	0.0112225579626771\\
98	0.0112225662823604\\
99	0.011222574755554\\
100	0.0112225833851032\\
101	0.0112225921739063\\
102	0.0112226011249158\\
103	0.011222610241139\\
104	0.0112226195256394\\
105	0.0112226289815377\\
106	0.0112226386120126\\
107	0.0112226484203025\\
108	0.0112226584097057\\
109	0.0112226685835825\\
110	0.0112226789453555\\
111	0.0112226894985116\\
112	0.0112227002466025\\
113	0.0112227111932462\\
114	0.0112227223421285\\
115	0.0112227336970038\\
116	0.0112227452616968\\
117	0.0112227570401034\\
118	0.0112227690361926\\
119	0.0112227812540073\\
120	0.0112227936976662\\
121	0.0112228063713646\\
122	0.0112228192793767\\
123	0.0112228324260564\\
124	0.0112228458158388\\
125	0.0112228594532423\\
126	0.0112228733428697\\
127	0.0112228874894098\\
128	0.0112229018976392\\
129	0.0112229165724239\\
130	0.011222931518721\\
131	0.0112229467415804\\
132	0.0112229622461465\\
133	0.01122297803766\\
134	0.0112229941214598\\
135	0.0112230105029848\\
136	0.0112230271877759\\
137	0.0112230441814775\\
138	0.0112230614898401\\
139	0.0112230791187218\\
140	0.0112230970740908\\
141	0.0112231153620267\\
142	0.0112231339887236\\
143	0.0112231529604914\\
144	0.0112231722837587\\
145	0.0112231919650744\\
146	0.0112232120111105\\
147	0.0112232324286641\\
148	0.01122325322466\\
149	0.0112232744061528\\
150	0.0112232959803299\\
151	0.0112233179545134\\
152	0.0112233403361631\\
153	0.0112233631328788\\
154	0.0112233863524031\\
155	0.0112234100026244\\
156	0.0112234340915791\\
157	0.0112234586274547\\
158	0.0112234836185928\\
159	0.0112235090734917\\
160	0.0112235350008097\\
161	0.011223561409368\\
162	0.0112235883081536\\
163	0.0112236157063229\\
164	0.0112236436132045\\
165	0.0112236720383026\\
166	0.0112237009913005\\
167	0.0112237304820639\\
168	0.0112237605206441\\
169	0.011223791117282\\
170	0.0112238222824115\\
171	0.0112238540266629\\
172	0.0112238863608673\\
173	0.0112239192960596\\
174	0.0112239528434832\\
175	0.0112239870145933\\
176	0.0112240218210613\\
177	0.0112240572747788\\
178	0.0112240933878619\\
179	0.0112241301726553\\
180	0.0112241676417369\\
181	0.011224205807922\\
182	0.0112242446842681\\
183	0.0112242842840792\\
184	0.0112243246209107\\
185	0.0112243657085742\\
186	0.0112244075611425\\
187	0.0112244501929542\\
188	0.0112244936186192\\
189	0.0112245378530236\\
190	0.0112245829113352\\
191	0.0112246288090087\\
192	0.0112246755617913\\
193	0.0112247231857282\\
194	0.0112247716971683\\
195	0.0112248211127701\\
196	0.0112248714495076\\
197	0.0112249227246763\\
198	0.0112249749558993\\
199	0.0112250281611337\\
200	0.0112250823586769\\
201	0.0112251375671731\\
202	0.0112251938056199\\
203	0.0112252510933753\\
204	0.0112253094501642\\
205	0.011225368896086\\
206	0.011225429451621\\
207	0.0112254911376385\\
208	0.0112255539754037\\
209	0.0112256179865854\\
210	0.0112256831932639\\
211	0.0112257496179387\\
212	0.0112258172835366\\
213	0.01122588621342\\
214	0.0112259564313949\\
215	0.0112260279617198\\
216	0.0112261008291141\\
217	0.0112261750587671\\
218	0.0112262506763467\\
219	0.0112263277080092\\
220	0.0112264061804078\\
221	0.0112264861207027\\
222	0.0112265675565707\\
223	0.0112266505162151\\
224	0.0112267350283756\\
225	0.0112268211223386\\
226	0.011226908827948\\
227	0.0112269981756153\\
228	0.0112270891963311\\
229	0.0112271819216755\\
230	0.0112272763838299\\
231	0.0112273726155881\\
232	0.0112274706503684\\
233	0.0112275705222251\\
234	0.011227672265861\\
235	0.0112277759166393\\
236	0.0112278815105967\\
237	0.0112279890844561\\
238	0.0112280986756394\\
239	0.0112282103222809\\
240	0.0112283240632411\\
241	0.0112284399381203\\
242	0.0112285579872724\\
243	0.0112286782518199\\
244	0.0112288007736676\\
245	0.0112289255955179\\
246	0.011229052760886\\
247	0.011229182314115\\
248	0.0112293143003913\\
249	0.0112294487657612\\
250	0.0112295857571464\\
251	0.0112297253223607\\
252	0.011229867510127\\
253	0.0112300123700938\\
254	0.0112301599528529\\
255	0.0112303103099569\\
256	0.0112304634939374\\
257	0.0112306195583225\\
258	0.011230778557656\\
259	0.011230940547516\\
260	0.0112311055845337\\
261	0.0112312737264134\\
262	0.0112314450319516\\
263	0.0112316195610575\\
264	0.0112317973747732\\
265	0.0112319785352941\\
266	0.0112321631059903\\
267	0.0112323511514274\\
268	0.0112325427373884\\
269	0.0112327379308954\\
270	0.0112329368002318\\
271	0.0112331394149648\\
272	0.011233345845968\\
273	0.0112335561654447\\
274	0.0112337704469511\\
275	0.0112339887654198\\
276	0.011234211197184\\
277	0.0112344378200014\\
278	0.0112346687130787\\
279	0.0112349039570962\\
280	0.0112351436342329\\
281	0.0112353878281913\\
282	0.0112356366242233\\
283	0.0112358901091551\\
284	0.0112361483714135\\
285	0.0112364115010516\\
286	0.0112366795897748\\
287	0.0112369527309674\\
288	0.0112372310197184\\
289	0.0112375145528484\\
290	0.011237803428936\\
291	0.0112380977483446\\
292	0.0112383976132487\\
293	0.0112387031276607\\
294	0.0112390143974577\\
295	0.011239331530408\\
296	0.0112396546361978\\
297	0.0112399838264574\\
298	0.0112403192147879\\
299	0.0112406609167875\\
300	0.0112410090500772\\
301	0.0112413637343272\\
302	0.0112417250912825\\
303	0.0112420932447884\\
304	0.0112424683208162\\
305	0.0112428504474881\\
306	0.0112432397551022\\
307	0.0112436363761576\\
308	0.0112440404453791\\
309	0.0112444520997416\\
310	0.011244871478495\\
311	0.0112452987231886\\
312	0.0112457339776962\\
313	0.0112461773882414\\
314	0.0112466291034225\\
315	0.0112470892742399\\
316	0.0112475580541222\\
317	0.0112480355989557\\
318	0.0112485220671135\\
319	0.0112490176194881\\
320	0.0112495224195257\\
321	0.0112500366332634\\
322	0.0112505604293706\\
323	0.0112510939791941\\
324	0.0112516374568092\\
325	0.0112521910390758\\
326	0.0112527549057027\\
327	0.0112533292393193\\
328	0.0112539142255573\\
329	0.0112545100531425\\
330	0.0112551169139992\\
331	0.0112557350033674\\
332	0.011256364519935\\
333	0.0112570056659853\\
334	0.0112576586475614\\
335	0.0112583236746474\\
336	0.0112590009613661\\
337	0.0112596907261927\\
338	0.0112603931921805\\
339	0.0112611085871956\\
340	0.0112618371441518\\
341	0.0112625791012361\\
342	0.0112633347021083\\
343	0.0112641041960519\\
344	0.0112648878380441\\
345	0.011265685888711\\
346	0.0112664986141052\\
347	0.0112673262854511\\
348	0.0112681691792728\\
349	0.0112690275775281\\
350	0.0112699017677485\\
351	0.0112707920431862\\
352	0.0112716987029678\\
353	0.0112726220522521\\
354	0.0112735624023946\\
355	0.0112745200711203\\
356	0.0112754953826846\\
357	0.0112764886680353\\
358	0.0112775002649788\\
359	0.0112785305183488\\
360	0.0112795797801781\\
361	0.0112806484098733\\
362	0.0112817367743908\\
363	0.011282845248415\\
364	0.011283974214537\\
365	0.0112851240634348\\
366	0.0112862951940518\\
367	0.0112874880137752\\
368	0.0112887029386123\\
369	0.0112899403933638\\
370	0.0112912008117934\\
371	0.0112924846367924\\
372	0.0112937923205388\\
373	0.0112951243246491\\
374	0.0112964811203223\\
375	0.0112978631884755\\
376	0.0112992710198701\\
377	0.0113007051152286\\
378	0.0113021659853424\\
379	0.0113036541511709\\
380	0.0113051701439351\\
381	0.0113067145052077\\
382	0.0113082877870055\\
383	0.0113098905518903\\
384	0.0113115233730906\\
385	0.0113131868346545\\
386	0.0113148815316564\\
387	0.0113166080704795\\
388	0.0113183670692088\\
389	0.0113201591581762\\
390	0.0113219849807153\\
391	0.0113238451941965\\
392	0.0113257404714362\\
393	0.0113276715025997\\
394	0.0113296389977477\\
395	0.0113316436902041\\
396	0.0113336863409353\\
397	0.0113357677441374\\
398	0.0113378887348318\\
399	0.0113400501891257\\
400	0.0113422530048768\\
401	0.0113444981016797\\
402	0.0113467864207639\\
403	0.0113491189247945\\
404	0.0113514965975696\\
405	0.0113539204436045\\
406	0.0113563914876015\\
407	0.0113589107738157\\
408	0.0113614793653651\\
409	0.0113640983435972\\
410	0.0113667688076252\\
411	0.0113694918737642\\
412	0.0113722686734777\\
413	0.0113751003485068\\
414	0.0113779880549139\\
415	0.0113809329646181\\
416	0.0113839362623907\\
417	0.011386999142006\\
418	0.0113901228013718\\
419	0.0113933084364245\\
420	0.0113965572335405\\
421	0.0113998703601832\\
422	0.0114032489534711\\
423	0.011406694106363\\
424	0.0114102068511853\\
425	0.0114137881400628\\
426	0.0114174388188876\\
427	0.0114211595720168\\
428	0.0114249513806757\\
429	0.0114288158262218\\
430	0.011432754551876\\
431	0.0114367692681713\\
432	0.0114408617591386\\
433	0.0114450338893178\\
434	0.0114492876117109\\
435	0.0114536249769666\\
436	0.0114580481439158\\
437	0.0114625593916818\\
438	0.0114671611336305\\
439	0.011471855933476\\
440	0.0114766465239209\\
441	0.011481535828314\\
442	0.0114865269859937\\
443	0.0114916233824512\\
444	0.011496828686832\\
445	0.0115021469039296\\
446	0.0115075824644448\\
447	0.0115131404395331\\
448	0.0115191322062942\\
449	0.0115253991608814\\
450	0.0115317762313072\\
451	0.0115382647247336\\
452	0.0115448659241668\\
453	0.0115515811223587\\
454	0.0115584117023178\\
455	0.0115653587311498\\
456	0.0115724231632056\\
457	0.0115796058215558\\
458	0.0115869073784527\\
459	0.0115943283382833\\
460	0.0116018690360585\\
461	0.0116095296963436\\
462	0.0116173107300015\\
463	0.0116252138172487\\
464	0.011633245881548\\
465	0.0116409291186393\\
466	0.011648737192218\\
467	0.0116566872445538\\
468	0.0116647814716776\\
469	0.011673022064531\\
470	0.0116814112027402\\
471	0.0116899511018213\\
472	0.0116986439963342\\
473	0.0117074921346454\\
474	0.011716497799903\\
475	0.0117256632804195\\
476	0.0117349907755795\\
477	0.0117444810454503\\
478	0.0117541301058064\\
479	0.0117638676743984\\
480	0.011773735668206\\
481	0.011783785700611\\
482	0.0117940201873895\\
483	0.0118044413792205\\
484	0.0118150518355674\\
485	0.011825852454593\\
486	0.0118368405715724\\
487	0.0118479980445204\\
488	0.0118593537395021\\
489	0.0118709228414562\\
490	0.0118827043690247\\
491	0.0118947033303379\\
492	0.0119069278351825\\
493	0.0119193808282175\\
494	0.0119320644612938\\
495	0.011944980676853\\
496	0.0119581312197565\\
497	0.0119715176530992\\
498	0.0119851412676121\\
499	0.0119990028801546\\
500	0.0120131027660469\\
501	0.0120274406003132\\
502	0.0120420154112648\\
503	0.0120568252786763\\
504	0.0120718684325769\\
505	0.0120871484815884\\
506	0.0121026704547876\\
507	0.0121184413034275\\
508	0.0121344611355934\\
509	0.0121508368205792\\
510	0.01216842867978\\
511	0.0121862142017653\\
512	0.0122044455249945\\
513	0.0122231229061308\\
514	0.0122422283685575\\
515	0.0122618673473782\\
516	0.0122812016952955\\
517	0.0123039654967153\\
518	0.0123375914224916\\
519	0.0123703867767683\\
520	0.0124021639739272\\
521	0.0124304090221293\\
522	0.0124483093410831\\
523	0.012465186770289\\
524	0.0124809448608415\\
525	0.0124952318226891\\
526	0.0125071168720933\\
527	0.0125188577351294\\
528	0.0125305005114574\\
529	0.0125421071992136\\
530	0.0125538705882579\\
531	0.0125658181268851\\
532	0.0125779716808804\\
533	0.0125903529367803\\
534	0.0126029746354097\\
535	0.012615844187871\\
536	0.012628968201082\\
537	0.0126423524358235\\
538	0.0126560016126776\\
539	0.0126699196902501\\
540	0.0126841108125561\\
541	0.0126985931866322\\
542	0.0127143652210051\\
543	0.0127310739668326\\
544	0.0127476395016023\\
545	0.0127627334645779\\
546	0.0127776116276101\\
547	0.012792569092646\\
548	0.0128073408324845\\
549	0.012822322360404\\
550	0.0128375136959082\\
551	0.0128529220491045\\
552	0.0128685412994251\\
553	0.012884357167586\\
554	0.0129003310487705\\
555	0.012916510225626\\
556	0.012932898194355\\
557	0.0129494900945344\\
558	0.0129662896703516\\
559	0.0129846714157593\\
560	0.0130023054258907\\
561	0.0130188302252141\\
562	0.0130351869307859\\
563	0.0130515579693191\\
564	0.0130680628266491\\
565	0.0130846955837323\\
566	0.0131014483412305\\
567	0.0131183126116483\\
568	0.0131357162880282\\
569	0.0131529761041365\\
570	0.0131696048889977\\
571	0.0131861379800626\\
572	0.0132027353963806\\
573	0.0132193852681561\\
574	0.0132360736311067\\
575	0.0132527851286413\\
576	0.0132695028912621\\
577	0.0132862084004275\\
578	0.0133028813379604\\
579	0.0133194994192258\\
580	0.0133360382078977\\
581	0.0133524709099312\\
582	0.0133687681444566\\
583	0.0133848976903317\\
584	0.0134008242107329\\
585	0.0134165089689716\\
586	0.0134319095789188\\
587	0.0134469799156334\\
588	0.0134616705332227\\
589	0.0134759305304731\\
590	0.0134902483101722\\
591	0.0135046714122469\\
592	0.0135192196273521\\
593	0.0135339766714342\\
594	0.0135491998980231\\
595	0.0135656125153459\\
596	0.0135851815686143\\
597	0.0136131889078992\\
598	0.0136637438596031\\
599	0\\
600	0\\
};
\addplot [color=black!20!mycolor21,solid,forget plot]
  table[row sep=crcr]{%
1	0.0112307667062576\\
2	0.0112307680392395\\
3	0.0112307693967611\\
4	0.0112307707792757\\
5	0.011230772187245\\
6	0.0112307736211395\\
7	0.0112307750814381\\
8	0.0112307765686288\\
9	0.0112307780832086\\
10	0.0112307796256837\\
11	0.0112307811965698\\
12	0.0112307827963919\\
13	0.0112307844256851\\
14	0.0112307860849943\\
15	0.0112307877748743\\
16	0.0112307894958906\\
17	0.0112307912486188\\
18	0.0112307930336455\\
19	0.0112307948515681\\
20	0.0112307967029949\\
21	0.0112307985885459\\
22	0.0112308005088522\\
23	0.011230802464557\\
24	0.0112308044563151\\
25	0.0112308064847937\\
26	0.0112308085506724\\
27	0.0112308106546434\\
28	0.0112308127974115\\
29	0.0112308149796952\\
30	0.0112308172022257\\
31	0.0112308194657483\\
32	0.0112308217710218\\
33	0.0112308241188194\\
34	0.0112308265099286\\
35	0.0112308289451514\\
36	0.0112308314253048\\
37	0.0112308339512212\\
38	0.0112308365237481\\
39	0.0112308391437492\\
40	0.0112308418121039\\
41	0.0112308445297083\\
42	0.0112308472974749\\
43	0.0112308501163333\\
44	0.0112308529872305\\
45	0.0112308559111311\\
46	0.0112308588890175\\
47	0.0112308619218905\\
48	0.0112308650107697\\
49	0.0112308681566934\\
50	0.0112308713607196\\
51	0.0112308746239256\\
52	0.0112308779474092\\
53	0.0112308813322883\\
54	0.0112308847797018\\
55	0.01123088829081\\
56	0.0112308918667945\\
57	0.0112308955088591\\
58	0.0112308992182301\\
59	0.0112309029961566\\
60	0.0112309068439111\\
61	0.0112309107627896\\
62	0.0112309147541125\\
63	0.0112309188192248\\
64	0.0112309229594966\\
65	0.0112309271763235\\
66	0.0112309314711272\\
67	0.0112309358453559\\
68	0.011230940300485\\
69	0.0112309448380172\\
70	0.0112309494594835\\
71	0.0112309541664433\\
72	0.0112309589604853\\
73	0.0112309638432277\\
74	0.0112309688163191\\
75	0.0112309738814389\\
76	0.0112309790402978\\
77	0.0112309842946385\\
78	0.0112309896462362\\
79	0.0112309950968995\\
80	0.0112310006484706\\
81	0.0112310063028264\\
82	0.0112310120618786\\
83	0.0112310179275749\\
84	0.0112310239018994\\
85	0.0112310299868732\\
86	0.0112310361845553\\
87	0.0112310424970433\\
88	0.011231048926474\\
89	0.0112310554750243\\
90	0.0112310621449116\\
91	0.0112310689383952\\
92	0.0112310758577764\\
93	0.0112310829053999\\
94	0.0112310900836539\\
95	0.0112310973949718\\
96	0.0112311048418324\\
97	0.011231112426761\\
98	0.0112311201523302\\
99	0.0112311280211608\\
100	0.0112311360359229\\
101	0.0112311441993365\\
102	0.0112311525141728\\
103	0.0112311609832548\\
104	0.0112311696094586\\
105	0.0112311783957143\\
106	0.0112311873450069\\
107	0.0112311964603775\\
108	0.0112312057449242\\
109	0.0112312152018035\\
110	0.0112312248342309\\
111	0.0112312346454825\\
112	0.0112312446388959\\
113	0.0112312548178714\\
114	0.0112312651858733\\
115	0.0112312757464309\\
116	0.0112312865031398\\
117	0.0112312974596635\\
118	0.0112313086197342\\
119	0.0112313199871543\\
120	0.0112313315657976\\
121	0.0112313433596113\\
122	0.0112313553726163\\
123	0.0112313676089095\\
124	0.011231380072665\\
125	0.0112313927681353\\
126	0.0112314056996532\\
127	0.011231418871633\\
128	0.0112314322885721\\
129	0.0112314459550529\\
130	0.0112314598757441\\
131	0.0112314740554022\\
132	0.0112314884988737\\
133	0.0112315032110965\\
134	0.0112315181971013\\
135	0.0112315334620142\\
136	0.0112315490110578\\
137	0.0112315648495532\\
138	0.0112315809829222\\
139	0.011231597416689\\
140	0.0112316141564819\\
141	0.0112316312080358\\
142	0.0112316485771938\\
143	0.0112316662699096\\
144	0.0112316842922494\\
145	0.0112317026503942\\
146	0.0112317213506418\\
147	0.0112317403994092\\
148	0.0112317598032349\\
149	0.0112317795687812\\
150	0.0112317997028365\\
151	0.0112318202123177\\
152	0.0112318411042729\\
153	0.0112318623858836\\
154	0.0112318840644675\\
155	0.011231906147481\\
156	0.0112319286425221\\
157	0.0112319515573326\\
158	0.0112319748998015\\
159	0.0112319986779674\\
160	0.0112320229000214\\
161	0.0112320475743104\\
162	0.0112320727093397\\
163	0.0112320983137763\\
164	0.0112321243964519\\
165	0.011232150966366\\
166	0.0112321780326895\\
167	0.0112322056047677\\
168	0.0112322336921235\\
169	0.0112322623044614\\
170	0.0112322914516706\\
171	0.0112323211438286\\
172	0.011232351391205\\
173	0.0112323822042651\\
174	0.0112324135936739\\
175	0.0112324455702996\\
176	0.0112324781452179\\
177	0.0112325113297159\\
178	0.0112325451352962\\
179	0.0112325795736812\\
180	0.011232614656817\\
181	0.0112326503968785\\
182	0.0112326868062731\\
183	0.0112327238976455\\
184	0.0112327616838827\\
185	0.011232800178118\\
186	0.0112328393937365\\
187	0.0112328793443798\\
188	0.0112329200439507\\
189	0.0112329615066188\\
190	0.0112330037468255\\
191	0.0112330467792893\\
192	0.0112330906190113\\
193	0.0112331352812807\\
194	0.0112331807816806\\
195	0.0112332271360936\\
196	0.0112332743607078\\
197	0.0112333224720229\\
198	0.011233371486856\\
199	0.0112334214223485\\
200	0.0112334722959716\\
201	0.0112335241255339\\
202	0.0112335769291871\\
203	0.0112336307254334\\
204	0.0112336855331321\\
205	0.0112337413715071\\
206	0.0112337982601535\\
207	0.0112338562190457\\
208	0.0112339152685445\\
209	0.0112339754294047\\
210	0.0112340367227834\\
211	0.0112340991702475\\
212	0.0112341627937824\\
213	0.0112342276158001\\
214	0.0112342936591474\\
215	0.0112343609471154\\
216	0.0112344295034477\\
217	0.0112344993523498\\
218	0.0112345705184982\\
219	0.0112346430270503\\
220	0.0112347169036533\\
221	0.0112347921744551\\
222	0.0112348688661133\\
223	0.0112349470058063\\
224	0.0112350266212436\\
225	0.0112351077406765\\
226	0.0112351903929089\\
227	0.0112352746073088\\
228	0.0112353604138199\\
229	0.0112354478429727\\
230	0.011235536925897\\
231	0.011235627694334\\
232	0.0112357201806485\\
233	0.0112358144178421\\
234	0.0112359104395658\\
235	0.0112360082801335\\
236	0.0112361079745356\\
237	0.0112362095584531\\
238	0.0112363130682711\\
239	0.0112364185410943\\
240	0.0112365260147609\\
241	0.0112366355278583\\
242	0.0112367471197386\\
243	0.0112368608305339\\
244	0.0112369767011731\\
245	0.0112370947733981\\
246	0.0112372150897807\\
247	0.01123733769374\\
248	0.0112374626295598\\
249	0.0112375899424069\\
250	0.0112377196783496\\
251	0.0112378518843761\\
252	0.0112379866084142\\
253	0.0112381238993509\\
254	0.0112382638070524\\
255	0.0112384063823847\\
256	0.011238551677235\\
257	0.0112386997445326\\
258	0.0112388506382713\\
259	0.0112390044135318\\
260	0.0112391611265046\\
261	0.0112393208345137\\
262	0.0112394835960401\\
263	0.0112396494707471\\
264	0.0112398185195046\\
265	0.0112399908044153\\
266	0.0112401663888406\\
267	0.0112403453374274\\
268	0.0112405277161357\\
269	0.0112407135922663\\
270	0.0112409030344894\\
271	0.0112410961128739\\
272	0.011241292898917\\
273	0.011241493465575\\
274	0.0112416978872941\\
275	0.011241906240042\\
276	0.0112421186013408\\
277	0.0112423350502996\\
278	0.0112425556676486\\
279	0.0112427805357733\\
280	0.0112430097387497\\
281	0.0112432433623799\\
282	0.0112434814942289\\
283	0.0112437242236615\\
284	0.0112439716418803\\
285	0.0112442238419642\\
286	0.0112444809189072\\
287	0.0112447429696587\\
288	0.0112450100931639\\
289	0.0112452823904045\\
290	0.0112455599644406\\
291	0.0112458429204529\\
292	0.0112461313657854\\
293	0.0112464254099887\\
294	0.0112467251648633\\
295	0.011247030744504\\
296	0.0112473422653442\\
297	0.0112476598462\\
298	0.0112479836083154\\
299	0.0112483136754067\\
300	0.0112486501737073\\
301	0.0112489932320123\\
302	0.0112493429817223\\
303	0.0112496995568871\\
304	0.011250063094249\\
305	0.0112504337332846\\
306	0.0112508116162455\\
307	0.0112511968881987\\
308	0.011251589697064\\
309	0.0112519901936504\\
310	0.0112523985316899\\
311	0.0112528148678689\\
312	0.0112532393618569\\
313	0.0112536721763309\\
314	0.0112541134769972\\
315	0.0112545634326081\\
316	0.0112550222149743\\
317	0.0112554899989719\\
318	0.0112559669625436\\
319	0.0112564532866937\\
320	0.0112569491554766\\
321	0.0112574547559778\\
322	0.0112579702782874\\
323	0.011258495915466\\
324	0.0112590318635021\\
325	0.0112595783212621\\
326	0.0112601354904324\\
327	0.0112607035754546\\
328	0.011261282783455\\
329	0.0112618733241702\\
330	0.0112624754098717\\
331	0.0112630892552927\\
332	0.0112637150775627\\
333	0.0112643530961564\\
334	0.011265003532867\\
335	0.0112656666118139\\
336	0.0112663425595027\\
337	0.0112670316049559\\
338	0.0112677339799408\\
339	0.011268449919328\\
340	0.0112691796616234\\
341	0.0112699234497262\\
342	0.011270681531983\\
343	0.0112714541636173\\
344	0.0112722416086195\\
345	0.0112730441421911\\
346	0.011273862054093\\
347	0.0112746956487724\\
348	0.0112755452373909\\
349	0.0112764111380161\\
350	0.01127729367582\\
351	0.0112781931832859\\
352	0.0112791100004236\\
353	0.0112800444749947\\
354	0.011280996962749\\
355	0.0112819678276726\\
356	0.0112829574422343\\
357	0.0112839661876405\\
358	0.011284994454098\\
359	0.011286042641085\\
360	0.0112871111576312\\
361	0.0112882004226044\\
362	0.0112893108650061\\
363	0.0112904429242726\\
364	0.0112915970505834\\
365	0.0112927737051744\\
366	0.0112939733606545\\
367	0.0112951965013248\\
368	0.0112964436234978\\
369	0.0112977152358139\\
370	0.0112990118595516\\
371	0.0113003340289287\\
372	0.0113016822913876\\
373	0.0113030572078605\\
374	0.011304459353006\\
375	0.0113058893154091\\
376	0.0113073476977328\\
377	0.0113088351168096\\
378	0.0113103522036562\\
379	0.0113118996033932\\
380	0.0113134779750471\\
381	0.0113150879912066\\
382	0.0113167303375016\\
383	0.011318405711864\\
384	0.0113201148235244\\
385	0.011321858391686\\
386	0.0113236371438089\\
387	0.0113254518134204\\
388	0.0113273031373548\\
389	0.0113291918523007\\
390	0.0113311186905176\\
391	0.0113330843745502\\
392	0.0113350896107483\\
393	0.0113371350813761\\
394	0.0113392214350959\\
395	0.0113413492756428\\
396	0.011343519148421\\
397	0.011345731523411\\
398	0.0113479867633781\\
399	0.0113502852931765\\
400	0.0113526280027205\\
401	0.0113550158061234\\
402	0.0113574496431801\\
403	0.0113599304810429\\
404	0.0113624593161152\\
405	0.0113650371761958\\
406	0.0113676651229043\\
407	0.0113703442544188\\
408	0.0113730757085532\\
409	0.0113758606661902\\
410	0.0113787003551336\\
411	0.0113815960549853\\
412	0.0113845491032701\\
413	0.0113875609018832\\
414	0.0113906329241603\\
415	0.011393766722834\\
416	0.0113969639392339\\
417	0.0114002263138842\\
418	0.0114035556987356\\
419	0.0114069540713054\\
420	0.0114104235510158\\
421	0.0114139664181129\\
422	0.0114175851361284\\
423	0.0114212823795039\\
424	0.0114250610712117\\
425	0.0114289244426929\\
426	0.0114328761654531\\
427	0.0114370640982274\\
428	0.0114415379374501\\
429	0.0114460924029805\\
430	0.0114507286537309\\
431	0.0114554478303431\\
432	0.0114602510426257\\
433	0.0114651393613016\\
434	0.0114701138533264\\
435	0.011475175547086\\
436	0.0114803254241217\\
437	0.0114855644096148\\
438	0.0114908933614573\\
439	0.0114963130577435\\
440	0.0115018241826379\\
441	0.0115074273110264\\
442	0.0115131228940036\\
443	0.0115189112529816\\
444	0.0115247926103114\\
445	0.0115307672557026\\
446	0.0115368362056364\\
447	0.0115430036618903\\
448	0.0115490132673341\\
449	0.0115549947015359\\
450	0.0115610862595482\\
451	0.0115672897545382\\
452	0.0115736070318654\\
453	0.0115800399785694\\
454	0.0115865905355455\\
455	0.0115932606339144\\
456	0.0116000522373306\\
457	0.0116069673451786\\
458	0.0116140079956677\\
459	0.0116211762663818\\
460	0.0116284742624648\\
461	0.0116359040542482\\
462	0.0116434674208401\\
463	0.011651164826483\\
464	0.0116589913997526\\
465	0.0116668580464109\\
466	0.0116748712939301\\
467	0.0116830370779039\\
468	0.0116913580790573\\
469	0.011699837004147\\
470	0.0117084765842934\\
471	0.0117172795714657\\
472	0.01172624873421\\
473	0.0117353868512342\\
474	0.0117446966973311\\
475	0.0117541810099138\\
476	0.0117638423974898\\
477	0.0117736831347504\\
478	0.0117837046803641\\
479	0.0117939107459867\\
480	0.0118043046563185\\
481	0.0118148888470297\\
482	0.0118256641281432\\
483	0.0118366276489269\\
484	0.0118477610158207\\
485	0.011859093544736\\
486	0.011870639935368\\
487	0.011882399334264\\
488	0.0118943790589139\\
489	0.0119065838765882\\
490	0.0119190160358446\\
491	0.0119316777135256\\
492	0.0119445707170306\\
493	0.011957696419546\\
494	0.0119710557136896\\
495	0.0119846489551514\\
496	0.0119984758715109\\
497	0.0120125350473707\\
498	0.0120268270333737\\
499	0.0120413562861641\\
500	0.0120561278030908\\
501	0.0120711473781491\\
502	0.0120864222598817\\
503	0.0121022226856904\\
504	0.0121190727498333\\
505	0.0121360117600989\\
506	0.0121530092632708\\
507	0.0121700361156674\\
508	0.012187250361415\\
509	0.0122047689873851\\
510	0.0122217619396384\\
511	0.012239260157526\\
512	0.0122574551592304\\
513	0.0122775888917459\\
514	0.0123100310886368\\
515	0.0123416998353115\\
516	0.0123720190910494\\
517	0.0123991861831648\\
518	0.0124160639725055\\
519	0.0124319379072306\\
520	0.0124466770974654\\
521	0.0124598547014337\\
522	0.0124708480141983\\
523	0.012481708239206\\
524	0.0124924783118715\\
525	0.0125032377562866\\
526	0.012514155670743\\
527	0.0125252528002105\\
528	0.0125365503121378\\
529	0.0125480686772887\\
530	0.0125598158865116\\
531	0.0125717988484442\\
532	0.0125840236760444\\
533	0.0125964954365076\\
534	0.012609218675859\\
535	0.0126221979545071\\
536	0.0126354374974166\\
537	0.0126489398102175\\
538	0.0126627172712948\\
539	0.0126770675913136\\
540	0.0126932238069243\\
541	0.0127092401128757\\
542	0.0127242536202772\\
543	0.0127385024089967\\
544	0.0127528396853093\\
545	0.0127670273929129\\
546	0.012781355090042\\
547	0.0127958929273647\\
548	0.0128106515387539\\
549	0.0128256300351418\\
550	0.0128408265723288\\
551	0.0128562370638659\\
552	0.0128718528282949\\
553	0.0128876526856466\\
554	0.012903615942414\\
555	0.012919788613919\\
556	0.0129361672580484\\
557	0.0129533934080588\\
558	0.0129714059924364\\
559	0.0129877694482098\\
560	0.0130039237299284\\
561	0.0130199787653225\\
562	0.0130361763128579\\
563	0.0130525147202513\\
564	0.0130689870656023\\
565	0.0130855862276066\\
566	0.0131023047069769\\
567	0.0131196261688764\\
568	0.0131366518311452\\
569	0.0131531500754486\\
570	0.0131696057265081\\
571	0.0131861379869476\\
572	0.0132027353981748\\
573	0.0132193852690165\\
574	0.0132360736315502\\
575	0.0132527851288632\\
576	0.0132695028913673\\
577	0.0132862084004741\\
578	0.0133028813379795\\
579	0.0133194994192329\\
580	0.0133360382078999\\
581	0.0133524709099318\\
582	0.0133687681444567\\
583	0.0133848976903317\\
584	0.0134008242107329\\
585	0.0134165089689716\\
586	0.0134319095789188\\
587	0.0134469799156334\\
588	0.0134616705332227\\
589	0.0134759305304731\\
590	0.0134902483101722\\
591	0.0135046714122469\\
592	0.0135192196273521\\
593	0.0135339766714342\\
594	0.0135491998980231\\
595	0.0135656125153459\\
596	0.0135851815686143\\
597	0.0136131889078992\\
598	0.0136637438596031\\
599	0\\
600	0\\
};
\addplot [color=black!50!mycolor20,solid,forget plot]
  table[row sep=crcr]{%
1	0.0112351944782826\\
2	0.0112351958259356\\
3	0.0112351971984351\\
4	0.0112351985962409\\
5	0.0112352000198214\\
6	0.0112352014696533\\
7	0.0112352029462226\\
8	0.0112352044500242\\
9	0.011235205981562\\
10	0.0112352075413496\\
11	0.0112352091299098\\
12	0.0112352107477753\\
13	0.0112352123954888\\
14	0.0112352140736029\\
15	0.0112352157826806\\
16	0.0112352175232953\\
17	0.011235219296031\\
18	0.0112352211014828\\
19	0.0112352229402567\\
20	0.0112352248129698\\
21	0.0112352267202511\\
22	0.011235228662741\\
23	0.0112352306410919\\
24	0.0112352326559683\\
25	0.011235234708047\\
26	0.0112352367980177\\
27	0.0112352389265827\\
28	0.0112352410944572\\
29	0.0112352433023702\\
30	0.0112352455510638\\
31	0.0112352478412942\\
32	0.0112352501738315\\
33	0.0112352525494604\\
34	0.01123525496898\\
35	0.0112352574332043\\
36	0.0112352599429625\\
37	0.0112352624990993\\
38	0.0112352651024751\\
39	0.0112352677539663\\
40	0.0112352704544658\\
41	0.0112352732048828\\
42	0.0112352760061437\\
43	0.0112352788591922\\
44	0.0112352817649896\\
45	0.0112352847245149\\
46	0.0112352877387655\\
47	0.0112352908087575\\
48	0.0112352939355258\\
49	0.0112352971201247\\
50	0.0112353003636281\\
51	0.01123530366713\\
52	0.0112353070317447\\
53	0.0112353104586075\\
54	0.0112353139488747\\
55	0.0112353175037242\\
56	0.0112353211243561\\
57	0.0112353248119926\\
58	0.0112353285678789\\
59	0.0112353323932834\\
60	0.0112353362894982\\
61	0.0112353402578396\\
62	0.0112353442996484\\
63	0.0112353484162904\\
64	0.0112353526091571\\
65	0.0112353568796658\\
66	0.0112353612292604\\
67	0.0112353656594118\\
68	0.0112353701716183\\
69	0.0112353747674064\\
70	0.0112353794483308\\
71	0.0112353842159756\\
72	0.0112353890719545\\
73	0.011235394017911\\
74	0.0112353990555199\\
75	0.0112354041864869\\
76	0.0112354094125497\\
77	0.0112354147354788\\
78	0.0112354201570774\\
79	0.0112354256791828\\
80	0.0112354313036664\\
81	0.0112354370324349\\
82	0.0112354428674306\\
83	0.011235448810632\\
84	0.011235454864055\\
85	0.0112354610297528\\
86	0.0112354673098176\\
87	0.0112354737063802\\
88	0.0112354802216119\\
89	0.0112354868577243\\
90	0.0112354936169706\\
91	0.0112355005016462\\
92	0.0112355075140895\\
93	0.0112355146566828\\
94	0.011235521931853\\
95	0.0112355293420726\\
96	0.0112355368898604\\
97	0.0112355445777826\\
98	0.0112355524084532\\
99	0.0112355603845357\\
100	0.0112355685087431\\
101	0.0112355767838398\\
102	0.0112355852126418\\
103	0.011235593798018\\
104	0.0112356025428913\\
105	0.0112356114502392\\
106	0.0112356205230955\\
107	0.0112356297645507\\
108	0.0112356391777536\\
109	0.0112356487659119\\
110	0.0112356585322938\\
111	0.0112356684802288\\
112	0.011235678613109\\
113	0.0112356889343905\\
114	0.0112356994475941\\
115	0.0112357101563069\\
116	0.0112357210641835\\
117	0.0112357321749474\\
118	0.0112357434923919\\
119	0.0112357550203819\\
120	0.011235766762855\\
121	0.0112357787238227\\
122	0.0112357909073725\\
123	0.0112358033176684\\
124	0.0112358159589532\\
125	0.0112358288355493\\
126	0.0112358419518609\\
127	0.011235855312375\\
128	0.0112358689216631\\
129	0.0112358827843831\\
130	0.0112358969052805\\
131	0.0112359112891905\\
132	0.0112359259410393\\
133	0.0112359408658463\\
134	0.0112359560687254\\
135	0.0112359715548871\\
136	0.0112359873296402\\
137	0.011236003398394\\
138	0.0112360197666595\\
139	0.0112360364400522\\
140	0.0112360534242935\\
141	0.0112360707252129\\
142	0.0112360883487502\\
143	0.0112361063009574\\
144	0.011236124588001\\
145	0.0112361432161641\\
146	0.0112361621918486\\
147	0.0112361815215777\\
148	0.0112362012119977\\
149	0.0112362212698812\\
150	0.0112362417021287\\
151	0.0112362625157715\\
152	0.011236283717974\\
153	0.0112363053160365\\
154	0.0112363273173976\\
155	0.0112363497296369\\
156	0.0112363725604777\\
157	0.0112363958177899\\
158	0.0112364195095927\\
159	0.0112364436440572\\
160	0.01123646822951\\
161	0.0112364932744355\\
162	0.0112365187874795\\
163	0.0112365447774519\\
164	0.0112365712533299\\
165	0.0112365982242617\\
166	0.0112366256995691\\
167	0.0112366536887513\\
168	0.0112366822014883\\
169	0.0112367112476444\\
170	0.0112367408372715\\
171	0.011236770980613\\
172	0.0112368016881074\\
173	0.0112368329703923\\
174	0.0112368648383078\\
175	0.011236897302901\\
176	0.0112369303754294\\
177	0.0112369640673656\\
178	0.011236998390401\\
179	0.0112370333564505\\
180	0.0112370689776563\\
181	0.0112371052663928\\
182	0.0112371422352709\\
183	0.0112371798971428\\
184	0.0112372182651065\\
185	0.0112372573525106\\
186	0.0112372971729593\\
187	0.0112373377403177\\
188	0.0112373790687162\\
189	0.0112374211725562\\
190	0.0112374640665155\\
191	0.0112375077655533\\
192	0.0112375522849161\\
193	0.0112375976401432\\
194	0.0112376438470724\\
195	0.0112376909218461\\
196	0.0112377388809171\\
197	0.0112377877410546\\
198	0.011237837519351\\
199	0.0112378882332277\\
200	0.0112379399004419\\
201	0.0112379925390931\\
202	0.0112380461676301\\
203	0.0112381008048578\\
204	0.0112381564699443\\
205	0.011238213182428\\
206	0.0112382709622254\\
207	0.0112383298296379\\
208	0.0112383898053603\\
209	0.0112384509104882\\
210	0.0112385131665259\\
211	0.0112385765953951\\
212	0.0112386412194427\\
213	0.0112387070614498\\
214	0.0112387741446402\\
215	0.0112388424926893\\
216	0.0112389121297333\\
217	0.0112389830803786\\
218	0.011239055369711\\
219	0.0112391290233056\\
220	0.011239204067237\\
221	0.0112392805280889\\
222	0.0112393584329647\\
223	0.0112394378094984\\
224	0.011239518685865\\
225	0.0112396010907916\\
226	0.0112396850535691\\
227	0.0112397706040632\\
228	0.0112398577727267\\
229	0.0112399465906113\\
230	0.0112400370893801\\
231	0.0112401293013203\\
232	0.0112402232593558\\
233	0.0112403189970606\\
234	0.0112404165486726\\
235	0.0112405159491068\\
236	0.0112406172339701\\
237	0.0112407204395753\\
238	0.0112408256029562\\
239	0.0112409327618823\\
240	0.011241041954875\\
241	0.0112411532212228\\
242	0.0112412666009978\\
243	0.0112413821350724\\
244	0.0112414998651363\\
245	0.0112416198337136\\
246	0.0112417420841811\\
247	0.0112418666607862\\
248	0.0112419936086659\\
249	0.0112421229738654\\
250	0.0112422548033587\\
251	0.0112423891450676\\
252	0.0112425260478831\\
253	0.0112426655616863\\
254	0.0112428077373697\\
255	0.0112429526268598\\
256	0.0112431002831399\\
257	0.011243250760273\\
258	0.0112434041134258\\
259	0.0112435603988937\\
260	0.0112437196741254\\
261	0.0112438819977489\\
262	0.0112440474295982\\
263	0.0112442160307399\\
264	0.011244387863502\\
265	0.0112445629915017\\
266	0.0112447414796753\\
267	0.0112449233943081\\
268	0.0112451088030658\\
269	0.011245297775026\\
270	0.0112454903807114\\
271	0.011245686692123\\
272	0.011245886782775\\
273	0.0112460907277304\\
274	0.0112462986036377\\
275	0.011246510488768\\
276	0.0112467264630548\\
277	0.0112469466081329\\
278	0.01124717100738\\
279	0.0112473997459587\\
280	0.0112476329108604\\
281	0.01124787059095\\
282	0.0112481128770119\\
283	0.0112483598617982\\
284	0.0112486116400772\\
285	0.0112488683086845\\
286	0.0112491299665753\\
287	0.011249396714878\\
288	0.01124966865695\\
289	0.0112499458984355\\
290	0.0112502285473242\\
291	0.0112505167140129\\
292	0.0112508105113688\\
293	0.0112511100547948\\
294	0.0112514154622969\\
295	0.0112517268545542\\
296	0.0112520443549908\\
297	0.0112523680898506\\
298	0.0112526981882742\\
299	0.0112530347823781\\
300	0.0112533780073377\\
301	0.0112537280014717\\
302	0.01125408490633\\
303	0.0112544488667842\\
304	0.0112548200311212\\
305	0.0112551985511393\\
306	0.0112555845822474\\
307	0.0112559782835671\\
308	0.0112563798180377\\
309	0.0112567893525232\\
310	0.0112572070579229\\
311	0.011257633109283\\
312	0.0112580676859115\\
313	0.0112585109714935\\
314	0.011258963154208\\
315	0.0112594244268451\\
316	0.0112598949869225\\
317	0.0112603750368003\\
318	0.0112608647837929\\
319	0.0112613644402765\\
320	0.0112618742237889\\
321	0.0112623943571217\\
322	0.0112629250683987\\
323	0.0112634665911391\\
324	0.0112640191643\\
325	0.0112645830322926\\
326	0.0112651584449661\\
327	0.0112657456575503\\
328	0.0112663449305484\\
329	0.0112669565295663\\
330	0.0112675807250669\\
331	0.0112682177920301\\
332	0.011268868009498\\
333	0.0112695316599826\\
334	0.0112702090287028\\
335	0.0112709004026179\\
336	0.0112716060692117\\
337	0.0112723263149765\\
338	0.0112730614235338\\
339	0.0112738116733184\\
340	0.0112745773347391\\
341	0.0112753586667209\\
342	0.0112761559125331\\
343	0.0112769692948167\\
344	0.0112777990096898\\
345	0.0112786452192335\\
346	0.0112795080376671\\
347	0.0112803876049575\\
348	0.0112812842512423\\
349	0.0112821983134854\\
350	0.0112831301356409\\
351	0.0112840800688197\\
352	0.0112850484714556\\
353	0.0112860357094653\\
354	0.0112870421563975\\
355	0.0112880681936037\\
356	0.0112891142106824\\
357	0.0112901806057492\\
358	0.0112912677857093\\
359	0.011292376166548\\
360	0.0112935061736426\\
361	0.0112946582420975\\
362	0.0112958328171055\\
363	0.0112970303543392\\
364	0.0112982513203751\\
365	0.0112994961931557\\
366	0.0113007654624944\\
367	0.0113020596306281\\
368	0.0113033792128247\\
369	0.0113047247380533\\
370	0.0113060967497241\\
371	0.0113074958065102\\
372	0.0113089224832606\\
373	0.0113103773720188\\
374	0.0113118610831618\\
375	0.0113133742466777\\
376	0.0113149175136011\\
377	0.0113164915576308\\
378	0.0113180970769578\\
379	0.011319734796334\\
380	0.0113214054694192\\
381	0.0113231098814511\\
382	0.0113248488522854\\
383	0.0113266232398675\\
384	0.0113284339442032\\
385	0.0113302819119076\\
386	0.0113321681414276\\
387	0.0113340936890466\\
388	0.0113360596758014\\
389	0.0113380672954645\\
390	0.0113401178237735\\
391	0.0113422126291345\\
392	0.0113443531850954\\
393	0.0113465410850429\\
394	0.011348778060013\\
395	0.0113510660018995\\
396	0.0113534069992335\\
397	0.0113558034107334\\
398	0.0113583168250304\\
399	0.0113610460379201\\
400	0.0113638261856476\\
401	0.0113666581148067\\
402	0.0113695426763981\\
403	0.0113724807243164\\
404	0.01137547311312\\
405	0.0113785206945055\\
406	0.0113816243116338\\
407	0.0113847847909651\\
408	0.0113880029361386\\
409	0.0113912795517672\\
410	0.0113946156180478\\
411	0.0113980122560836\\
412	0.011401470277601\\
413	0.0114049904425198\\
414	0.0114085735015855\\
415	0.0114122201905185\\
416	0.0114159312069635\\
417	0.0114197072026157\\
418	0.0114235487710558\\
419	0.011427456419056\\
420	0.0114314305383849\\
421	0.0114354714701621\\
422	0.0114395795153607\\
423	0.0114437550997129\\
424	0.0114479984198532\\
425	0.011452310039942\\
426	0.0114566922528386\\
427	0.0114610260806607\\
428	0.0114652689912649\\
429	0.0114695916077998\\
430	0.0114739952737607\\
431	0.0114784813487846\\
432	0.0114830512078456\\
433	0.0114877062419988\\
434	0.0114924478660921\\
435	0.0114972775148057\\
436	0.0115021966439266\\
437	0.0115072067319429\\
438	0.0115123092820202\\
439	0.0115175058244099\\
440	0.0115227979192723\\
441	0.011528187159605\\
442	0.0115336751728246\\
443	0.0115392636152083\\
444	0.0115449541367792\\
445	0.0115507482301072\\
446	0.0115566466262736\\
447	0.0115626469125934\\
448	0.0115686995666977\\
449	0.0115748386193224\\
450	0.0115810961834864\\
451	0.0115874744552185\\
452	0.0115939756710901\\
453	0.0116006021089091\\
454	0.0116073560874389\\
455	0.0116142399679179\\
456	0.0116212561545987\\
457	0.0116284070949366\\
458	0.0116356952788893\\
459	0.0116431232357747\\
460	0.0116506935242219\\
461	0.0116584087027417\\
462	0.011666271247919\\
463	0.0116742833448325\\
464	0.0116824464369669\\
465	0.011690765514842\\
466	0.0116992432784483\\
467	0.0117078823670914\\
468	0.0117166854303381\\
469	0.011725655123599\\
470	0.0117347941028349\\
471	0.0117441050180588\\
472	0.0117535905049195\\
473	0.0117632531731367\\
474	0.0117730955903263\\
475	0.011783120243332\\
476	0.011793329460267\\
477	0.0118037253839663\\
478	0.0118143103311685\\
479	0.0118250852272747\\
480	0.0118360472282428\\
481	0.0118471787530638\\
482	0.0118585066197922\\
483	0.0118700470539891\\
484	0.0118817992347845\\
485	0.011893770724185\\
486	0.0119059661922436\\
487	0.011918387666641\\
488	0.0119310367172285\\
489	0.0119439144308247\\
490	0.0119570213167615\\
491	0.0119703565722088\\
492	0.0119839236319086\\
493	0.011997728078787\\
494	0.0120117761109523\\
495	0.0120260748158329\\
496	0.012040632949565\\
497	0.0120559168278183\\
498	0.0120719814775316\\
499	0.0120881446193133\\
500	0.0121043905968925\\
501	0.0121207034716817\\
502	0.0121370704421017\\
503	0.0121532550532704\\
504	0.0121687851616581\\
505	0.0121844771200109\\
506	0.0122005760501311\\
507	0.0122171054119467\\
508	0.012234205817976\\
509	0.0122521990039042\\
510	0.0122816849601013\\
511	0.0123121442619949\\
512	0.0123416091951965\\
513	0.0123691027454942\\
514	0.0123851353787613\\
515	0.012400191923301\\
516	0.0124141165882071\\
517	0.0124265510429925\\
518	0.0124367580222875\\
519	0.0124468328337305\\
520	0.0124568189466466\\
521	0.0124668029325324\\
522	0.0124769382084773\\
523	0.0124872447203615\\
524	0.0124977427497673\\
525	0.0125084502254296\\
526	0.0125193742630316\\
527	0.0125305213519228\\
528	0.0125418971964977\\
529	0.0125535064874172\\
530	0.012565353730664\\
531	0.0125774432769447\\
532	0.0125897794886255\\
533	0.0126023667317016\\
534	0.0126152088468381\\
535	0.0126283102321478\\
536	0.0126416900885925\\
537	0.0126566327469089\\
538	0.0126721436449114\\
539	0.0126872593255511\\
540	0.0127008999274142\\
541	0.0127146303254365\\
542	0.0127282920377257\\
543	0.0127419703158875\\
544	0.012755856263815\\
545	0.0127699596130699\\
546	0.0127842824736393\\
547	0.0127988246644857\\
548	0.0128135851567183\\
549	0.0128285622593643\\
550	0.0128437531543193\\
551	0.0128591526408781\\
552	0.0128747495286285\\
553	0.0128905167409036\\
554	0.0129064587151011\\
555	0.0129226098866358\\
556	0.012940262197184\\
557	0.0129572514711595\\
558	0.0129732190329192\\
559	0.0129889692473418\\
560	0.0130048428449475\\
561	0.0130208678413011\\
562	0.0130370385205704\\
563	0.0130533485311235\\
564	0.0130697914384829\\
565	0.0130863606440552\\
566	0.0131035466640366\\
567	0.0131204071165013\\
568	0.0131367816466169\\
569	0.0131531501361782\\
570	0.0131696057273078\\
571	0.0131861379872111\\
572	0.0132027353983047\\
573	0.0132193852690821\\
574	0.0132360736315821\\
575	0.0132527851288778\\
576	0.0132695028913736\\
577	0.0132862084004766\\
578	0.0133028813379803\\
579	0.0133194994192331\\
580	0.0133360382079\\
581	0.0133524709099319\\
582	0.0133687681444567\\
583	0.0133848976903317\\
584	0.0134008242107329\\
585	0.0134165089689716\\
586	0.0134319095789188\\
587	0.0134469799156334\\
588	0.0134616705332227\\
589	0.0134759305304731\\
590	0.0134902483101722\\
591	0.0135046714122469\\
592	0.0135192196273521\\
593	0.0135339766714342\\
594	0.0135491998980231\\
595	0.0135656125153459\\
596	0.0135851815686143\\
597	0.0136131889078992\\
598	0.0136637438596031\\
599	0\\
600	0\\
};
\addplot [color=black!60!mycolor21,solid,forget plot]
  table[row sep=crcr]{%
1	0.011239041352767\\
2	0.0112390428002761\\
3	0.0112390442744595\\
4	0.0112390457758103\\
5	0.0112390473048309\\
6	0.0112390488620331\\
7	0.011239050447938\\
8	0.0112390520630764\\
9	0.0112390537079891\\
10	0.0112390553832268\\
11	0.0112390570893506\\
12	0.0112390588269317\\
13	0.0112390605965523\\
14	0.0112390623988051\\
15	0.0112390642342939\\
16	0.0112390661036339\\
17	0.0112390680074515\\
18	0.0112390699463848\\
19	0.0112390719210837\\
20	0.0112390739322104\\
21	0.0112390759804391\\
22	0.0112390780664569\\
23	0.0112390801909632\\
24	0.0112390823546708\\
25	0.0112390845583056\\
26	0.0112390868026069\\
27	0.011239089088328\\
28	0.011239091416236\\
29	0.0112390937871122\\
30	0.0112390962017527\\
31	0.0112390986609682\\
32	0.0112391011655845\\
33	0.0112391037164428\\
34	0.0112391063144001\\
35	0.0112391089603289\\
36	0.0112391116551185\\
37	0.0112391143996743\\
38	0.0112391171949187\\
39	0.0112391200417915\\
40	0.0112391229412496\\
41	0.0112391258942679\\
42	0.0112391289018394\\
43	0.0112391319649758\\
44	0.0112391350847073\\
45	0.0112391382620836\\
46	0.0112391414981736\\
47	0.0112391447940665\\
48	0.0112391481508715\\
49	0.0112391515697185\\
50	0.0112391550517585\\
51	0.0112391585981641\\
52	0.0112391622101294\\
53	0.0112391658888711\\
54	0.0112391696356283\\
55	0.0112391734516633\\
56	0.011239177338262\\
57	0.0112391812967343\\
58	0.0112391853284142\\
59	0.0112391894346609\\
60	0.0112391936168589\\
61	0.0112391978764184\\
62	0.0112392022147759\\
63	0.0112392066333948\\
64	0.0112392111337658\\
65	0.0112392157174072\\
66	0.0112392203858659\\
67	0.0112392251407175\\
68	0.0112392299835671\\
69	0.0112392349160495\\
70	0.0112392399398303\\
71	0.0112392450566061\\
72	0.0112392502681051\\
73	0.0112392555760878\\
74	0.0112392609823477\\
75	0.0112392664887115\\
76	0.0112392720970403\\
77	0.0112392778092297\\
78	0.011239283627211\\
79	0.0112392895529513\\
80	0.0112392955884546\\
81	0.0112393017357622\\
82	0.0112393079969537\\
83	0.0112393143741475\\
84	0.0112393208695016\\
85	0.0112393274852143\\
86	0.0112393342235249\\
87	0.0112393410867147\\
88	0.0112393480771077\\
89	0.0112393551970711\\
90	0.0112393624490167\\
91	0.0112393698354011\\
92	0.011239377358727\\
93	0.0112393850215438\\
94	0.0112393928264488\\
95	0.0112394007760878\\
96	0.0112394088731559\\
97	0.0112394171203988\\
98	0.0112394255206137\\
99	0.01123943407665\\
100	0.0112394427914104\\
101	0.011239451667852\\
102	0.0112394607089874\\
103	0.0112394699178854\\
104	0.0112394792976723\\
105	0.011239488851533\\
106	0.0112394985827122\\
107	0.011239508494515\\
108	0.0112395185903088\\
109	0.0112395288735239\\
110	0.011239539347655\\
111	0.0112395500162621\\
112	0.0112395608829721\\
113	0.0112395719514798\\
114	0.0112395832255493\\
115	0.0112395947090153\\
116	0.0112396064057844\\
117	0.0112396183198363\\
118	0.0112396304552257\\
119	0.011239642816083\\
120	0.0112396554066162\\
121	0.0112396682311124\\
122	0.011239681293939\\
123	0.0112396945995453\\
124	0.0112397081524644\\
125	0.0112397219573142\\
126	0.0112397360187993\\
127	0.011239750341713\\
128	0.0112397649309381\\
129	0.0112397797914495\\
130	0.0112397949283154\\
131	0.0112398103466992\\
132	0.0112398260518614\\
133	0.0112398420491613\\
134	0.0112398583440588\\
135	0.0112398749421167\\
136	0.0112398918490021\\
137	0.0112399090704889\\
138	0.0112399266124593\\
139	0.0112399444809064\\
140	0.0112399626819358\\
141	0.0112399812217681\\
142	0.011240000106741\\
143	0.0112400193433113\\
144	0.0112400389380573\\
145	0.0112400588976814\\
146	0.0112400792290118\\
147	0.0112400999390056\\
148	0.0112401210347507\\
149	0.0112401425234686\\
150	0.0112401644125168\\
151	0.0112401867093915\\
152	0.0112402094217301\\
153	0.0112402325573139\\
154	0.0112402561240709\\
155	0.0112402801300785\\
156	0.0112403045835667\\
157	0.0112403294929203\\
158	0.0112403548666826\\
159	0.0112403807135579\\
160	0.0112404070424147\\
161	0.0112404338622891\\
162	0.0112404611823875\\
163	0.0112404890120904\\
164	0.0112405173609551\\
165	0.0112405462387196\\
166	0.0112405756553059\\
167	0.0112406056208233\\
168	0.0112406361455723\\
169	0.011240667240048\\
170	0.011240698914944\\
171	0.0112407311811559\\
172	0.0112407640497856\\
173	0.0112407975321448\\
174	0.0112408316397595\\
175	0.0112408663843736\\
176	0.0112409017779533\\
177	0.0112409378326915\\
178	0.0112409745610118\\
179	0.0112410119755735\\
180	0.0112410500892752\\
181	0.0112410889152605\\
182	0.0112411284669217\\
183	0.0112411687579053\\
184	0.0112412098021164\\
185	0.011241251613724\\
186	0.0112412942071659\\
187	0.0112413375971539\\
188	0.0112413817986791\\
189	0.0112414268270174\\
190	0.0112414726977347\\
191	0.0112415194266926\\
192	0.0112415670300544\\
193	0.0112416155242904\\
194	0.0112416649261842\\
195	0.0112417152528389\\
196	0.0112417665216826\\
197	0.0112418187504754\\
198	0.0112418719573156\\
199	0.011241926160646\\
200	0.0112419813792609\\
201	0.0112420376323127\\
202	0.0112420949393192\\
203	0.01124215332017\\
204	0.0112422127951346\\
205	0.0112422733848691\\
206	0.011242335110424\\
207	0.011242397993252\\
208	0.0112424620552155\\
209	0.0112425273185951\\
210	0.011242593806097\\
211	0.0112426615408622\\
212	0.0112427305464744\\
213	0.0112428008469688\\
214	0.011242872466841\\
215	0.0112429454310562\\
216	0.0112430197650579\\
217	0.011243095494778\\
218	0.0112431726466456\\
219	0.0112432512475977\\
220	0.0112433313250885\\
221	0.0112434129070996\\
222	0.0112434960221511\\
223	0.0112435806993115\\
224	0.0112436669682089\\
225	0.0112437548590421\\
226	0.0112438444025918\\
227	0.0112439356302323\\
228	0.011244028573943\\
229	0.011244123266321\\
230	0.0112442197405929\\
231	0.0112443180306275\\
232	0.0112444181709489\\
233	0.0112445201967493\\
234	0.0112446241439026\\
235	0.0112447300489779\\
236	0.0112448379492539\\
237	0.0112449478827324\\
238	0.0112450598881539\\
239	0.0112451740050118\\
240	0.0112452902735678\\
241	0.0112454087348678\\
242	0.0112455294307574\\
243	0.0112456524038985\\
244	0.0112457776977859\\
245	0.011245905356764\\
246	0.0112460354260447\\
247	0.011246167951725\\
248	0.0112463029808051\\
249	0.0112464405612071\\
250	0.0112465807417942\\
251	0.0112467235723898\\
252	0.011246869103798\\
253	0.0112470173878234\\
254	0.0112471684772925\\
255	0.0112473224260747\\
256	0.0112474792891046\\
257	0.0112476391224038\\
258	0.0112478019831046\\
259	0.0112479679294726\\
260	0.0112481370209317\\
261	0.0112483093180882\\
262	0.011248484882756\\
263	0.0112486637779829\\
264	0.0112488460680768\\
265	0.0112490318186331\\
266	0.0112492210965625\\
267	0.0112494139701195\\
268	0.0112496105089323\\
269	0.0112498107840322\\
270	0.0112500148678849\\
271	0.0112502228344225\\
272	0.0112504347590754\\
273	0.0112506507188065\\
274	0.0112508707921453\\
275	0.0112510950592233\\
276	0.0112513236018102\\
277	0.0112515565033518\\
278	0.0112517938490079\\
279	0.0112520357256926\\
280	0.0112522822221144\\
281	0.0112525334288191\\
282	0.0112527894382324\\
283	0.0112530503447052\\
284	0.0112533162445595\\
285	0.0112535872361362\\
286	0.0112538634198447\\
287	0.0112541448982132\\
288	0.0112544317759423\\
289	0.0112547241599591\\
290	0.0112550221594742\\
291	0.0112553258860407\\
292	0.0112556354536151\\
293	0.0112559509786208\\
294	0.0112562725800151\\
295	0.0112566003793575\\
296	0.0112569345008823\\
297	0.011257275071574\\
298	0.0112576222212465\\
299	0.0112579760826259\\
300	0.0112583367914379\\
301	0.0112587044864993\\
302	0.0112590793098156\\
303	0.0112594614066828\\
304	0.0112598509257967\\
305	0.011260248019368\\
306	0.0112606528432454\\
307	0.0112610655570471\\
308	0.011261486324301\\
309	0.0112619153125962\\
310	0.0112623526937454\\
311	0.0112627986439604\\
312	0.0112632533440423\\
313	0.0112637169795881\\
314	0.0112641897412159\\
315	0.0112646718248108\\
316	0.0112651634317953\\
317	0.0112656647694251\\
318	0.0112661760511175\\
319	0.011266697496813\\
320	0.0112672293333778\\
321	0.011267771795051\\
322	0.0112683251239445\\
323	0.0112688895706017\\
324	0.011269465394626\\
325	0.011270052865386\\
326	0.0112706522628133\\
327	0.011271263878303\\
328	0.0112718880157353\\
329	0.0112725249926361\\
330	0.0112731751414983\\
331	0.011273838811289\\
332	0.0112745163691734\\
333	0.0112752082024888\\
334	0.0112759147210093\\
335	0.011276636359551\\
336	0.0112773735809702\\
337	0.0112781268796244\\
338	0.011278896785372\\
339	0.0112796838682095\\
340	0.0112804887436729\\
341	0.0112813120791945\\
342	0.0112821546017914\\
343	0.0112830171080404\\
344	0.0112839004793356\\
345	0.0112848057129946\\
346	0.0112857590681972\\
347	0.0112868000115571\\
348	0.0112878610524782\\
349	0.0112889425743934\\
350	0.0112900449666733\\
351	0.0112911686238055\\
352	0.0112923139443557\\
353	0.0112934813317574\\
354	0.0112946712069793\\
355	0.0112958840735392\\
356	0.0112971203706034\\
357	0.0112983805387522\\
358	0.0112996650256684\\
359	0.011300974286136\\
360	0.0113023087820242\\
361	0.0113036689822559\\
362	0.0113050553627564\\
363	0.0113064684063821\\
364	0.0113079086028251\\
365	0.01130937644849\\
366	0.0113108724463407\\
367	0.0113123971057115\\
368	0.0113139509420787\\
369	0.0113155344767859\\
370	0.0113171482367189\\
371	0.0113187927539204\\
372	0.0113204685651377\\
373	0.0113221762112941\\
374	0.0113239162368713\\
375	0.0113256891891924\\
376	0.0113274956175895\\
377	0.0113293360724393\\
378	0.0113312111040474\\
379	0.0113331212613591\\
380	0.0113350670904709\\
381	0.011337049132911\\
382	0.0113390679236565\\
383	0.0113411239888445\\
384	0.0113432178431313\\
385	0.0113453499866442\\
386	0.0113475209014642\\
387	0.0113497310475676\\
388	0.0113519808581482\\
389	0.0113542707342486\\
390	0.0113566010386649\\
391	0.0113589720892446\\
392	0.0113613841522097\\
393	0.0113638374378285\\
394	0.011366332106384\\
395	0.01136886831157\\
396	0.0113714463758554\\
397	0.0113740674361707\\
398	0.0113766836405187\\
399	0.0113792123482623\\
400	0.0113817903305588\\
401	0.0113844184879349\\
402	0.0113870977339388\\
403	0.0113898289950864\\
404	0.0113926132106861\\
405	0.011395451332449\\
406	0.0113983443238048\\
407	0.0114012931591304\\
408	0.0114042988243542\\
409	0.0114073623244064\\
410	0.0114104847127292\\
411	0.0114136670790637\\
412	0.0114169104751911\\
413	0.0114202159635445\\
414	0.0114235846263744\\
415	0.0114270175662835\\
416	0.0114305159043431\\
417	0.0114340807812689\\
418	0.0114377133581531\\
419	0.0114414148151829\\
420	0.0114451863522635\\
421	0.0114490292045197\\
422	0.011452944643883\\
423	0.0114569339804359\\
424	0.0114609984057528\\
425	0.0114651387043717\\
426	0.0114693539460858\\
427	0.0114736198897073\\
428	0.011477933198766\\
429	0.0114823326681331\\
430	0.0114868199698509\\
431	0.0114913968077287\\
432	0.0114960649180925\\
433	0.0115008260706244\\
434	0.0115056820690209\\
435	0.0115106347517691\\
436	0.0115156859929351\\
437	0.0115208377029543\\
438	0.0115260918294043\\
439	0.0115314503577167\\
440	0.0115369153117144\\
441	0.0115424887536511\\
442	0.0115481727828192\\
443	0.011553969530018\\
444	0.0115598811402087\\
445	0.0115659097227945\\
446	0.0115720572205754\\
447	0.0115783251137235\\
448	0.0115847165823101\\
449	0.0115912345328617\\
450	0.0115978812282788\\
451	0.011604658955138\\
452	0.0116115700225534\\
453	0.0116186167608869\\
454	0.0116258015203381\\
455	0.0116331266693448\\
456	0.0116405945927863\\
457	0.0116482076899403\\
458	0.0116559683720931\\
459	0.0116638790595942\\
460	0.0116719421779993\\
461	0.0116801601529917\\
462	0.0116885354053477\\
463	0.0116970703555997\\
464	0.0117057674803261\\
465	0.0117146291956841\\
466	0.0117236579083083\\
467	0.0117328560135997\\
468	0.0117422258916711\\
469	0.0117517699022312\\
470	0.0117614903773437\\
471	0.0117713896097924\\
472	0.0117814698319882\\
473	0.011791733172256\\
474	0.0118021815560944\\
475	0.0118128167528688\\
476	0.0118236400101319\\
477	0.0118346485947756\\
478	0.0118458276192941\\
479	0.0118571954671726\\
480	0.0118687733282496\\
481	0.0118805600567881\\
482	0.0118925619641299\\
483	0.0119047830617299\\
484	0.0119172242011226\\
485	0.0119298854033905\\
486	0.0119427725223105\\
487	0.0119558920220597\\
488	0.0119692511356302\\
489	0.0119828581950146\\
490	0.0119967236138377\\
491	0.0120115765680266\\
492	0.0120268598345071\\
493	0.0120422451105933\\
494	0.0120577189585024\\
495	0.0120732676044461\\
496	0.0120888795717372\\
497	0.0121041561963896\\
498	0.0121190235880096\\
499	0.0121340269579018\\
500	0.0121491673421512\\
501	0.0121644366243783\\
502	0.0121798048427859\\
503	0.0121953751414982\\
504	0.0122112964868549\\
505	0.0122279984447559\\
506	0.0122530267148885\\
507	0.0122826169348248\\
508	0.0123113768672566\\
509	0.0123388641799021\\
510	0.0123555690371973\\
511	0.012369964685551\\
512	0.0123832779073909\\
513	0.0123952743523283\\
514	0.0124047811299181\\
515	0.0124141527954976\\
516	0.0124234312987462\\
517	0.0124326974282635\\
518	0.0124421053633064\\
519	0.0124516742957978\\
520	0.0124614236240149\\
521	0.0124713696781674\\
522	0.0124815191433029\\
523	0.0124918780997577\\
524	0.0125024518623407\\
525	0.0125132449070785\\
526	0.0125242615999327\\
527	0.0125355062107447\\
528	0.0125469829271314\\
529	0.0125586959141213\\
530	0.0125706495456539\\
531	0.0125828481043012\\
532	0.0125952945913644\\
533	0.0126079997885039\\
534	0.0126212665972596\\
535	0.0126363143644848\\
536	0.012651210100543\\
537	0.0126648475890322\\
538	0.0126779993348058\\
539	0.012691187853797\\
540	0.0127042330856999\\
541	0.0127174825159092\\
542	0.0127309431386996\\
543	0.0127446210527922\\
544	0.0127585169094392\\
545	0.012772630736796\\
546	0.0127869621423264\\
547	0.0128015103267207\\
548	0.0128162740111645\\
549	0.0128312512294928\\
550	0.0128464387796428\\
551	0.0128618307298279\\
552	0.0128774144513983\\
553	0.0128931568941647\\
554	0.0129093833575267\\
555	0.0129269612101602\\
556	0.0129429560467004\\
557	0.0129585621734848\\
558	0.0129740989886379\\
559	0.0129897947280717\\
560	0.0130056457798629\\
561	0.0130216467641238\\
562	0.0130377919156873\\
563	0.0130540753963924\\
564	0.013070491452824\\
565	0.0130874896882337\\
566	0.0131042520123742\\
567	0.0131205097152076\\
568	0.0131367816514045\\
569	0.0131531501362776\\
570	0.0131696057273463\\
571	0.0131861379872302\\
572	0.0132027353983141\\
573	0.0132193852690865\\
574	0.013236073631584\\
575	0.0132527851288787\\
576	0.0132695028913739\\
577	0.0132862084004767\\
578	0.0133028813379804\\
579	0.0133194994192331\\
580	0.0133360382079\\
581	0.0133524709099319\\
582	0.0133687681444567\\
583	0.0133848976903317\\
584	0.0134008242107329\\
585	0.0134165089689716\\
586	0.0134319095789188\\
587	0.0134469799156334\\
588	0.0134616705332227\\
589	0.0134759305304731\\
590	0.0134902483101722\\
591	0.0135046714122469\\
592	0.0135192196273521\\
593	0.0135339766714342\\
594	0.0135491998980231\\
595	0.0135656125153459\\
596	0.0135851815686143\\
597	0.0136131889078992\\
598	0.0136637438596031\\
599	0\\
600	0\\
};
\addplot [color=black!80!mycolor21,solid,forget plot]
  table[row sep=crcr]{%
1	0.0112465140737006\\
2	0.0112465158814963\\
3	0.0112465177224853\\
4	0.0112465195972787\\
5	0.0112465215064991\\
6	0.0112465234507803\\
7	0.0112465254307683\\
8	0.0112465274471205\\
9	0.0112465295005068\\
10	0.0112465315916093\\
11	0.0112465337211228\\
12	0.011246535889755\\
13	0.0112465380982265\\
14	0.0112465403472714\\
15	0.0112465426376371\\
16	0.0112465449700852\\
17	0.0112465473453911\\
18	0.0112465497643446\\
19	0.0112465522277501\\
20	0.0112465547364269\\
21	0.0112465572912095\\
22	0.0112465598929477\\
23	0.0112465625425072\\
24	0.0112465652407694\\
25	0.0112465679886323\\
26	0.0112465707870105\\
27	0.0112465736368354\\
28	0.0112465765390555\\
29	0.0112465794946372\\
30	0.0112465825045645\\
31	0.0112465855698397\\
32	0.0112465886914837\\
33	0.0112465918705363\\
34	0.0112465951080563\\
35	0.0112465984051224\\
36	0.0112466017628331\\
37	0.0112466051823073\\
38	0.0112466086646847\\
39	0.0112466122111258\\
40	0.0112466158228129\\
41	0.0112466195009501\\
42	0.0112466232467637\\
43	0.0112466270615028\\
44	0.0112466309464397\\
45	0.01124663490287\\
46	0.0112466389321137\\
47	0.0112466430355148\\
48	0.0112466472144426\\
49	0.0112466514702916\\
50	0.011246655804482\\
51	0.0112466602184605\\
52	0.0112466647137005\\
53	0.0112466692917029\\
54	0.0112466739539961\\
55	0.0112466787021372\\
56	0.0112466835377118\\
57	0.0112466884623351\\
58	0.0112466934776523\\
59	0.0112466985853388\\
60	0.0112467037871013\\
61	0.0112467090846782\\
62	0.0112467144798401\\
63	0.0112467199743901\\
64	0.0112467255701653\\
65	0.0112467312690364\\
66	0.0112467370729091\\
67	0.0112467429837243\\
68	0.0112467490034589\\
69	0.0112467551341265\\
70	0.0112467613777781\\
71	0.0112467677365027\\
72	0.0112467742124282\\
73	0.0112467808077217\\
74	0.0112467875245909\\
75	0.0112467943652841\\
76	0.0112468013320916\\
77	0.011246808427346\\
78	0.0112468156534235\\
79	0.0112468230127442\\
80	0.011246830507773\\
81	0.0112468381410208\\
82	0.0112468459150452\\
83	0.011246853832451\\
84	0.0112468618958916\\
85	0.0112468701080696\\
86	0.0112468784717378\\
87	0.0112468869897002\\
88	0.0112468956648127\\
89	0.0112469044999845\\
90	0.0112469134981788\\
91	0.0112469226624137\\
92	0.0112469319957635\\
93	0.0112469415013597\\
94	0.0112469511823919\\
95	0.0112469610421089\\
96	0.0112469710838201\\
97	0.0112469813108963\\
98	0.0112469917267709\\
99	0.0112470023349413\\
100	0.0112470131389698\\
101	0.011247024142485\\
102	0.0112470353491829\\
103	0.0112470467628283\\
104	0.0112470583872561\\
105	0.0112470702263723\\
106	0.0112470822841558\\
107	0.0112470945646594\\
108	0.0112471070720112\\
109	0.0112471198104163\\
110	0.011247132784158\\
111	0.0112471459975992\\
112	0.011247159455184\\
113	0.0112471731614394\\
114	0.0112471871209764\\
115	0.0112472013384919\\
116	0.0112472158187704\\
117	0.0112472305666853\\
118	0.0112472455872007\\
119	0.0112472608853734\\
120	0.0112472764663539\\
121	0.011247292335389\\
122	0.0112473084978231\\
123	0.0112473249591001\\
124	0.0112473417247653\\
125	0.0112473588004673\\
126	0.0112473761919599\\
127	0.0112473939051042\\
128	0.0112474119458703\\
129	0.0112474303203398\\
130	0.0112474490347075\\
131	0.0112474680952834\\
132	0.0112474875084956\\
133	0.0112475072808915\\
134	0.0112475274191409\\
135	0.0112475479300378\\
136	0.0112475688205028\\
137	0.0112475900975856\\
138	0.0112476117684673\\
139	0.011247633840463\\
140	0.0112476563210241\\
141	0.0112476792177409\\
142	0.0112477025383454\\
143	0.0112477262907138\\
144	0.0112477504828692\\
145	0.0112477751229843\\
146	0.0112478002193843\\
147	0.0112478257805498\\
148	0.0112478518151196\\
149	0.0112478783318937\\
150	0.0112479053398361\\
151	0.0112479328480786\\
152	0.0112479608659229\\
153	0.0112479894028446\\
154	0.0112480184684962\\
155	0.0112480480727103\\
156	0.011248078225503\\
157	0.0112481089370777\\
158	0.011248140217828\\
159	0.0112481720783417\\
160	0.0112482045294044\\
161	0.0112482375820031\\
162	0.01124827124733\\
163	0.0112483055367864\\
164	0.0112483404619864\\
165	0.0112483760347613\\
166	0.0112484122671634\\
167	0.0112484491714701\\
168	0.0112484867601884\\
169	0.0112485250460588\\
170	0.01124856404206\\
171	0.0112486037614133\\
172	0.0112486442175872\\
173	0.0112486854243018\\
174	0.0112487273955338\\
175	0.0112487701455211\\
176	0.0112488136887681\\
177	0.01124885804005\\
178	0.0112489032144185\\
179	0.0112489492272069\\
180	0.011248996094035\\
181	0.0112490438308151\\
182	0.0112490924537568\\
183	0.0112491419793732\\
184	0.0112491924244862\\
185	0.0112492438062323\\
186	0.011249296142069\\
187	0.0112493494497801\\
188	0.0112494037474825\\
189	0.011249459053632\\
190	0.0112495153870298\\
191	0.0112495727668291\\
192	0.0112496312125417\\
193	0.0112496907440445\\
194	0.0112497513815866\\
195	0.0112498131457964\\
196	0.0112498760576884\\
197	0.0112499401386705\\
198	0.0112500054105519\\
199	0.01125007189555\\
200	0.0112501396162984\\
201	0.0112502085958546\\
202	0.0112502788577082\\
203	0.0112503504257888\\
204	0.0112504233244743\\
205	0.0112504975785995\\
206	0.0112505732134646\\
207	0.0112506502548437\\
208	0.0112507287289943\\
209	0.0112508086626659\\
210	0.0112508900831093\\
211	0.0112509730180863\\
212	0.0112510574958793\\
213	0.0112511435453007\\
214	0.0112512311957033\\
215	0.0112513204769905\\
216	0.0112514114196262\\
217	0.0112515040546458\\
218	0.011251598413667\\
219	0.0112516945289003\\
220	0.0112517924331608\\
221	0.0112518921598789\\
222	0.0112519937431127\\
223	0.0112520972175589\\
224	0.0112522026185657\\
225	0.0112523099821446\\
226	0.0112524193449831\\
227	0.0112525307444572\\
228	0.0112526442186446\\
229	0.0112527598063381\\
230	0.0112528775470586\\
231	0.0112529974810693\\
232	0.0112531196493895\\
233	0.0112532440938087\\
234	0.0112533708569017\\
235	0.0112534999820427\\
236	0.011253631513421\\
237	0.0112537654960559\\
238	0.0112539019758129\\
239	0.0112540409994192\\
240	0.0112541826144801\\
241	0.011254326869496\\
242	0.0112544738138786\\
243	0.0112546234979686\\
244	0.0112547759730534\\
245	0.0112549312913845\\
246	0.011255089506196\\
247	0.0112552506717232\\
248	0.0112554148432215\\
249	0.0112555820769857\\
250	0.0112557524303696\\
251	0.0112559259618061\\
252	0.0112561027308276\\
253	0.0112562827980867\\
254	0.0112564662253776\\
255	0.0112566530756575\\
256	0.0112568434130687\\
257	0.0112570373029611\\
258	0.011257234811915\\
259	0.011257436007764\\
260	0.0112576409596196\\
261	0.0112578497378943\\
262	0.0112580624143271\\
263	0.0112582790620081\\
264	0.0112584997554041\\
265	0.0112587245703849\\
266	0.0112589535842494\\
267	0.0112591868757529\\
268	0.0112594245251342\\
269	0.0112596666141443\\
270	0.0112599132260741\\
271	0.0112601644457839\\
272	0.0112604203597327\\
273	0.0112606810560084\\
274	0.0112609466243581\\
275	0.0112612171562196\\
276	0.0112614927447525\\
277	0.0112617734848708\\
278	0.0112620594732754\\
279	0.0112623508084872\\
280	0.0112626475908807\\
281	0.0112629499227187\\
282	0.0112632579081863\\
283	0.0112635716534267\\
284	0.0112638912665764\\
285	0.0112642168578017\\
286	0.0112645485393349\\
287	0.0112648864255112\\
288	0.0112652306328062\\
289	0.0112655812798733\\
290	0.0112659384875815\\
291	0.0112663023790532\\
292	0.0112666730797028\\
293	0.0112670507172746\\
294	0.0112674354218808\\
295	0.0112678273260394\\
296	0.0112682265647121\\
297	0.0112686332753413\\
298	0.0112690475978865\\
299	0.0112694696748603\\
300	0.0112698996513627\\
301	0.0112703376751146\\
302	0.0112707838964896\\
303	0.0112712384685432\\
304	0.0112717015470409\\
305	0.0112721732904816\\
306	0.0112726538601196\\
307	0.0112731434199812\\
308	0.0112736421368771\\
309	0.0112741501804091\\
310	0.0112746677229712\\
311	0.0112751949397417\\
312	0.0112757320086684\\
313	0.0112762791104423\\
314	0.0112768364284604\\
315	0.011277404148775\\
316	0.0112779824600271\\
317	0.0112785715533625\\
318	0.0112791716223264\\
319	0.0112797828627351\\
320	0.0112804054725191\\
321	0.0112810396515359\\
322	0.0112816856013452\\
323	0.0112823435249429\\
324	0.0112830136264467\\
325	0.011283696110726\\
326	0.011284391182968\\
327	0.0112850990481702\\
328	0.0112858199105477\\
329	0.0112865539728434\\
330	0.0112873014355252\\
331	0.0112880624958532\\
332	0.0112888373467973\\
333	0.0112896261757813\\
334	0.0112904291632275\\
335	0.0112912464808683\\
336	0.011292078289789\\
337	0.0112929247381596\\
338	0.0112937859586129\\
339	0.0112946620652585\\
340	0.0112955531504619\\
341	0.0112964592820748\\
342	0.011297380503763\\
343	0.0112983168477426\\
344	0.0112992684081234\\
345	0.0113002356615827\\
346	0.0113011981866483\\
347	0.0113021229912858\\
348	0.011303066083019\\
349	0.011304027822147\\
350	0.0113050085755559\\
351	0.0113060087166638\\
352	0.0113070286254258\\
353	0.0113080686889163\\
354	0.0113091293041964\\
355	0.0113102108877347\\
356	0.0113113138503988\\
357	0.0113124386094914\\
358	0.0113135855898635\\
359	0.0113147552240034\\
360	0.011315947952124\\
361	0.011317164222248\\
362	0.0113184044902896\\
363	0.0113196692201357\\
364	0.0113209588837223\\
365	0.0113222739611102\\
366	0.0113236149405574\\
367	0.0113249823185897\\
368	0.0113263766000688\\
369	0.01132779829826\\
370	0.0113292479348981\\
371	0.0113307260402538\\
372	0.0113322331532014\\
373	0.0113337698212891\\
374	0.0113353366008134\\
375	0.011336934056901\\
376	0.0113385627636001\\
377	0.0113402233039846\\
378	0.0113419162702771\\
379	0.0113436422639938\\
380	0.0113454018961199\\
381	0.0113471957873225\\
382	0.0113490245682099\\
383	0.0113508888796513\\
384	0.0113527893731675\\
385	0.0113547267114124\\
386	0.0113567015687627\\
387	0.0113587146320412\\
388	0.0113607666013978\\
389	0.0113628581913758\\
390	0.0113649901321715\\
391	0.0113671631710247\\
392	0.0113693780733711\\
393	0.0113716356222197\\
394	0.0113739366096414\\
395	0.0113762817964131\\
396	0.0113786717458\\
397	0.0113811061612395\\
398	0.0113835744974096\\
399	0.0113860631236216\\
400	0.0113886028964451\\
401	0.0113911948733196\\
402	0.011393840134293\\
403	0.0113965397826018\\
404	0.0113992949452713\\
405	0.0114021067737427\\
406	0.0114049764445453\\
407	0.0114079051600715\\
408	0.0114108941495445\\
409	0.0114139446701564\\
410	0.0114170580072881\\
411	0.0114202354742386\\
412	0.0114234784141378\\
413	0.0114267882010151\\
414	0.0114301662407148\\
415	0.0114336139718114\\
416	0.0114371328666062\\
417	0.0114407244321219\\
418	0.0114443902110759\\
419	0.0114481317828855\\
420	0.0114519507646593\\
421	0.0114558488111916\\
422	0.0114598276126126\\
423	0.0114638888840721\\
424	0.0114680343388791\\
425	0.0114722656124757\\
426	0.0114765840818516\\
427	0.011480991769219\\
428	0.0114854913360267\\
429	0.0114900846649395\\
430	0.0114947736731113\\
431	0.011499560312358\\
432	0.0115044465692753\\
433	0.0115094344652891\\
434	0.0115145260566377\\
435	0.0115197234342728\\
436	0.0115250287236716\\
437	0.0115304440845463\\
438	0.0115359717104374\\
439	0.0115416138281704\\
440	0.0115473726971375\\
441	0.0115532506083381\\
442	0.011559249883036\\
443	0.0115653728707932\\
444	0.0115716219466123\\
445	0.0115779995077365\\
446	0.0115845079753972\\
447	0.0115911498259805\\
448	0.0115979275453705\\
449	0.0116048436257181\\
450	0.0116119005789811\\
451	0.0116191009345577\\
452	0.0116264472366378\\
453	0.0116339420412495\\
454	0.0116415879129732\\
455	0.0116493874212933\\
456	0.0116573431365553\\
457	0.0116654576254935\\
458	0.0116737334462865\\
459	0.0116821731431068\\
460	0.0116907792401442\\
461	0.0116995542351541\\
462	0.0117085005926779\\
463	0.0117176207368288\\
464	0.0117269170404246\\
465	0.0117363918167596\\
466	0.0117460473087593\\
467	0.0117558856761953\\
468	0.0117659089793798\\
469	0.0117761191552671\\
470	0.0117865179759158\\
471	0.0117971069657494\\
472	0.0118078872277539\\
473	0.0118188590973616\\
474	0.0118300214282654\\
475	0.0118413650894999\\
476	0.0118528811076133\\
477	0.0118646067501865\\
478	0.011876540298591\\
479	0.0118886848485102\\
480	0.0119010507400625\\
481	0.0119136454255147\\
482	0.0119264770762642\\
483	0.0119395550052135\\
484	0.0119529685670179\\
485	0.0119675075241263\\
486	0.0119821595371004\\
487	0.0119969122178181\\
488	0.0120117518763348\\
489	0.012026664669177\\
490	0.0120416408877232\\
491	0.0120560576879281\\
492	0.0120703513483297\\
493	0.0120847749973027\\
494	0.0120993227183574\\
495	0.0121139883995913\\
496	0.0121287638935181\\
497	0.0121435734228761\\
498	0.0121584259457046\\
499	0.0121733979081672\\
500	0.0121885337910087\\
501	0.0122042508721673\\
502	0.0122243248823405\\
503	0.012253126402231\\
504	0.0122811631091917\\
505	0.0123080968354763\\
506	0.0123271506439781\\
507	0.0123410130276573\\
508	0.0123538565201679\\
509	0.0123655060424077\\
510	0.0123746905951724\\
511	0.0123834342839682\\
512	0.0123920759355795\\
513	0.0124006804561605\\
514	0.0124094164154852\\
515	0.0124183022457615\\
516	0.0124273565678601\\
517	0.0124365953200095\\
518	0.0124460248223508\\
519	0.0124556508094527\\
520	0.0124654782759953\\
521	0.0124755114782222\\
522	0.0124857545724964\\
523	0.0124962116296147\\
524	0.012506886662141\\
525	0.0125177836824298\\
526	0.0125289067221078\\
527	0.0125402598750669\\
528	0.012551847360068\\
529	0.0125636734422102\\
530	0.0125757412289918\\
531	0.0125880677380157\\
532	0.0126016825351311\\
533	0.0126161648684101\\
534	0.0126302379574037\\
535	0.0126428367023998\\
536	0.0126555156544702\\
537	0.0126680745095018\\
538	0.0126807062746515\\
539	0.0126935408327277\\
540	0.0127065889381891\\
541	0.0127198518206078\\
542	0.0127333302740618\\
543	0.0127470246364701\\
544	0.0127609349332875\\
545	0.0127750608635254\\
546	0.0127894017721699\\
547	0.0128039566102232\\
548	0.0128187238632145\\
549	0.0128337013809305\\
550	0.0128488858805986\\
551	0.0128642713518595\\
552	0.0128798436938607\\
553	0.0128963032992081\\
554	0.0129132511743546\\
555	0.0129287107054386\\
556	0.0129439677980492\\
557	0.0129593224543217\\
558	0.0129748400360881\\
559	0.012990516155719\\
560	0.0130063459378701\\
561	0.013022324171331\\
562	0.0130384455273938\\
563	0.0130547049389776\\
564	0.0130714646410262\\
565	0.0130881947021808\\
566	0.0131043428556408\\
567	0.0131205097155993\\
568	0.0131367816514175\\
569	0.0131531501362831\\
570	0.0131696057273491\\
571	0.0131861379872315\\
572	0.0132027353983147\\
573	0.0132193852690868\\
574	0.0132360736315842\\
575	0.0132527851288787\\
576	0.0132695028913739\\
577	0.0132862084004767\\
578	0.0133028813379804\\
579	0.0133194994192331\\
580	0.0133360382079\\
581	0.0133524709099319\\
582	0.0133687681444567\\
583	0.0133848976903317\\
584	0.0134008242107329\\
585	0.0134165089689716\\
586	0.0134319095789188\\
587	0.0134469799156334\\
588	0.0134616705332227\\
589	0.0134759305304731\\
590	0.0134902483101722\\
591	0.0135046714122469\\
592	0.0135192196273521\\
593	0.0135339766714342\\
594	0.0135491998980231\\
595	0.0135656125153459\\
596	0.0135851815686143\\
597	0.0136131889078992\\
598	0.0136637438596031\\
599	0\\
600	0\\
};
\addplot [color=black,solid,forget plot]
  table[row sep=crcr]{%
1	0.0112576820807805\\
2	0.0112576836631175\\
3	0.0112576852745897\\
4	0.0112576869157352\\
5	0.0112576885871022\\
6	0.0112576902892489\\
7	0.011257692022744\\
8	0.0112576937881667\\
9	0.011257695586107\\
10	0.0112576974171655\\
11	0.0112576992819545\\
12	0.0112577011810972\\
13	0.0112577031152285\\
14	0.0112577050849952\\
15	0.0112577070910559\\
16	0.0112577091340816\\
17	0.0112577112147556\\
18	0.0112577133337738\\
19	0.0112577154918454\\
20	0.0112577176896923\\
21	0.01125771992805\\
22	0.0112577222076678\\
23	0.0112577245293086\\
24	0.0112577268937497\\
25	0.0112577293017827\\
26	0.0112577317542139\\
27	0.0112577342518647\\
28	0.0112577367955716\\
29	0.0112577393861866\\
30	0.0112577420245777\\
31	0.011257744711629\\
32	0.0112577474482407\\
33	0.0112577502353301\\
34	0.0112577530738314\\
35	0.0112577559646961\\
36	0.0112577589088935\\
37	0.0112577619074107\\
38	0.0112577649612534\\
39	0.0112577680714457\\
40	0.011257771239031\\
41	0.011257774465072\\
42	0.011257777750651\\
43	0.0112577810968705\\
44	0.0112577845048537\\
45	0.0112577879757444\\
46	0.0112577915107077\\
47	0.0112577951109305\\
48	0.0112577987776216\\
49	0.0112578025120125\\
50	0.0112578063153574\\
51	0.0112578101889339\\
52	0.0112578141340432\\
53	0.011257818152011\\
54	0.0112578222441875\\
55	0.0112578264119479\\
56	0.0112578306566932\\
57	0.0112578349798503\\
58	0.0112578393828726\\
59	0.0112578438672409\\
60	0.0112578484344631\\
61	0.0112578530860754\\
62	0.0112578578236426\\
63	0.0112578626487585\\
64	0.0112578675630467\\
65	0.0112578725681608\\
66	0.0112578776657855\\
67	0.0112578828576367\\
68	0.0112578881454621\\
69	0.0112578935310423\\
70	0.0112578990161906\\
71	0.0112579046027544\\
72	0.0112579102926155\\
73	0.0112579160876905\\
74	0.011257921989932\\
75	0.0112579280013287\\
76	0.0112579341239065\\
77	0.0112579403597291\\
78	0.0112579467108984\\
79	0.0112579531795557\\
80	0.0112579597678822\\
81	0.0112579664780995\\
82	0.0112579733124708\\
83	0.0112579802733016\\
84	0.01125798736294\\
85	0.0112579945837782\\
86	0.0112580019382528\\
87	0.0112580094288459\\
88	0.011258017058086\\
89	0.0112580248285484\\
90	0.0112580327428568\\
91	0.0112580408036835\\
92	0.0112580490137509\\
93	0.011258057375832\\
94	0.0112580658927517\\
95	0.0112580745673872\\
96	0.0112580834026699\\
97	0.0112580924015855\\
98	0.0112581015671754\\
99	0.0112581109025381\\
100	0.0112581204108295\\
101	0.0112581300952646\\
102	0.0112581399591183\\
103	0.0112581500057267\\
104	0.0112581602384882\\
105	0.0112581706608644\\
106	0.0112581812763818\\
107	0.0112581920886325\\
108	0.0112582031012757\\
109	0.0112582143180391\\
110	0.0112582257427198\\
111	0.0112582373791857\\
112	0.0112582492313773\\
113	0.0112582613033083\\
114	0.0112582735990675\\
115	0.0112582861228201\\
116	0.0112582988788091\\
117	0.0112583118713566\\
118	0.0112583251048657\\
119	0.0112583385838214\\
120	0.0112583523127927\\
121	0.011258366296434\\
122	0.0112583805394865\\
123	0.0112583950467801\\
124	0.011258409823235\\
125	0.0112584248738632\\
126	0.0112584402037703\\
127	0.0112584558181575\\
128	0.0112584717223232\\
129	0.0112584879216648\\
130	0.0112585044216804\\
131	0.011258521227971\\
132	0.0112585383462425\\
133	0.0112585557823071\\
134	0.0112585735420859\\
135	0.0112585916316107\\
136	0.011258610057026\\
137	0.0112586288245911\\
138	0.0112586479406826\\
139	0.011258667411796\\
140	0.0112586872445485\\
141	0.0112587074456811\\
142	0.0112587280220607\\
143	0.0112587489806826\\
144	0.0112587703286731\\
145	0.0112587920732918\\
146	0.0112588142219338\\
147	0.0112588367821329\\
148	0.0112588597615637\\
149	0.0112588831680442\\
150	0.0112589070095388\\
151	0.0112589312941609\\
152	0.0112589560301758\\
153	0.0112589812260031\\
154	0.0112590068902203\\
155	0.0112590330315652\\
156	0.0112590596589391\\
157	0.01125908678141\\
158	0.0112591144082153\\
159	0.0112591425487655\\
160	0.0112591712126471\\
161	0.0112592004096261\\
162	0.011259230149651\\
163	0.0112592604428568\\
164	0.011259291299568\\
165	0.0112593227303022\\
166	0.0112593547457741\\
167	0.0112593873568988\\
168	0.0112594205747956\\
169	0.011259454410792\\
170	0.0112594888764274\\
171	0.0112595239834574\\
172	0.0112595597438574\\
173	0.0112595961698268\\
174	0.0112596332737938\\
175	0.0112596710684187\\
176	0.0112597095665989\\
177	0.0112597487814733\\
178	0.0112597887264267\\
179	0.0112598294150941\\
180	0.0112598708613661\\
181	0.0112599130793928\\
182	0.0112599560835895\\
183	0.011259999888641\\
184	0.011260044509507\\
185	0.0112600899614271\\
186	0.0112601362599261\\
187	0.0112601834208193\\
188	0.0112602314602177\\
189	0.0112602803945341\\
190	0.011260330240488\\
191	0.0112603810151119\\
192	0.0112604327357569\\
193	0.0112604854200987\\
194	0.0112605390861434\\
195	0.0112605937522342\\
196	0.0112606494370571\\
197	0.0112607061596477\\
198	0.0112607639393976\\
199	0.0112608227960609\\
200	0.011260882749761\\
201	0.0112609438209977\\
202	0.0112610060306537\\
203	0.0112610694000025\\
204	0.0112611339507147\\
205	0.0112611997048663\\
206	0.0112612666849456\\
207	0.0112613349138612\\
208	0.0112614044149494\\
209	0.0112614752119827\\
210	0.0112615473291776\\
211	0.0112616207912027\\
212	0.0112616956231872\\
213	0.0112617718507296\\
214	0.011261849499906\\
215	0.0112619285972795\\
216	0.0112620091699085\\
217	0.0112620912453566\\
218	0.0112621748517012\\
219	0.0112622600175435\\
220	0.0112623467720183\\
221	0.0112624351448031\\
222	0.0112625251661289\\
223	0.0112626168667898\\
224	0.0112627102781538\\
225	0.0112628054321732\\
226	0.0112629023613951\\
227	0.0112630010989727\\
228	0.0112631016786763\\
229	0.0112632041349044\\
230	0.0112633085026956\\
231	0.0112634148177399\\
232	0.011263523116391\\
233	0.0112636334356782\\
234	0.0112637458133186\\
235	0.0112638602877301\\
236	0.0112639768980438\\
237	0.011264095684117\\
238	0.0112642166865466\\
239	0.0112643399466824\\
240	0.0112644655066406\\
241	0.0112645934093183\\
242	0.0112647236984067\\
243	0.0112648564184064\\
244	0.0112649916146413\\
245	0.0112651293332739\\
246	0.0112652696213201\\
247	0.0112654125266648\\
248	0.0112655580980772\\
249	0.0112657063852272\\
250	0.0112658574387008\\
251	0.0112660113100169\\
252	0.0112661680516442\\
253	0.0112663277170174\\
254	0.011266490360555\\
255	0.0112666560376767\\
256	0.0112668248048206\\
257	0.0112669967194621\\
258	0.0112671718401313\\
259	0.0112673502264324\\
260	0.0112675319390618\\
261	0.0112677170398279\\
262	0.0112679055916703\\
263	0.0112680976586796\\
264	0.0112682933061171\\
265	0.011268492600436\\
266	0.0112686956093012\\
267	0.0112689024016106\\
268	0.0112691130475165\\
269	0.011269327618447\\
270	0.011269546187128\\
271	0.0112697688276052\\
272	0.0112699956152668\\
273	0.0112702266268665\\
274	0.0112704619405461\\
275	0.0112707016358595\\
276	0.0112709457937964\\
277	0.0112711944968063\\
278	0.0112714478288227\\
279	0.0112717058752885\\
280	0.0112719687231806\\
281	0.0112722364610355\\
282	0.0112725091789749\\
283	0.0112727869687322\\
284	0.0112730699236784\\
285	0.0112733581388494\\
286	0.0112736517109728\\
287	0.0112739507384954\\
288	0.0112742553216112\\
289	0.0112745655622895\\
290	0.0112748815643033\\
291	0.0112752034332587\\
292	0.0112755312766235\\
293	0.0112758652037578\\
294	0.0112762053259433\\
295	0.0112765517564144\\
296	0.0112769046103891\\
297	0.0112772640051001\\
298	0.0112776300598274\\
299	0.0112780028959302\\
300	0.0112783826368803\\
301	0.0112787694082955\\
302	0.011279163337974\\
303	0.0112795645559294\\
304	0.0112799731944262\\
305	0.0112803893880167\\
306	0.011280813273578\\
307	0.0112812449903512\\
308	0.0112816846799803\\
309	0.0112821324865534\\
310	0.0112825885566452\\
311	0.0112830530393604\\
312	0.01128352608638\\
313	0.0112840078520083\\
314	0.0112844984932235\\
315	0.0112849981697299\\
316	0.0112855070440136\\
317	0.0112860252814014\\
318	0.0112865530501233\\
319	0.01128709052138\\
320	0.0112876378694145\\
321	0.0112881952715901\\
322	0.0112887629084743\\
323	0.0112893409639309\\
324	0.0112899296252196\\
325	0.0112905290831059\\
326	0.0112911395319821\\
327	0.0112917611700007\\
328	0.0112923941992221\\
329	0.0112930388257796\\
330	0.0112936952600627\\
331	0.0112943637169221\\
332	0.0112950444158994\\
333	0.0112957375814848\\
334	0.0112964434434048\\
335	0.0112971622369433\\
336	0.0112978942032914\\
337	0.0112986395899087\\
338	0.0112993986508361\\
339	0.0113001716467717\\
340	0.0113009588443532\\
341	0.0113017605129985\\
342	0.011302576914436\\
343	0.0113034082704757\\
344	0.0113042546679323\\
345	0.0113051157788013\\
346	0.0113059870847612\\
347	0.0113068634602973\\
348	0.0113077574729328\\
349	0.0113086694838385\\
350	0.0113095998618589\\
351	0.0113105489837004\\
352	0.0113115172341461\\
353	0.0113125050063101\\
354	0.0113135127018809\\
355	0.01131454073094\\
356	0.0113155895124201\\
357	0.0113166594743456\\
358	0.0113177510540604\\
359	0.011318864698463\\
360	0.01132000086425\\
361	0.0113211600181681\\
362	0.0113223426372747\\
363	0.0113235492092085\\
364	0.0113247802324689\\
365	0.0113260362167065\\
366	0.0113273176830242\\
367	0.0113286251642894\\
368	0.0113299592054583\\
369	0.0113313203639124\\
370	0.0113327092098079\\
371	0.0113341263264394\\
372	0.011335572310617\\
373	0.0113370477730589\\
374	0.011338553338799\\
375	0.0113400896476113\\
376	0.0113416573544503\\
377	0.0113432571299097\\
378	0.0113448896606983\\
379	0.011346555650135\\
380	0.011348255818663\\
381	0.0113499909043826\\
382	0.0113517616636048\\
383	0.0113535688714243\\
384	0.0113554133223126\\
385	0.0113572958307299\\
386	0.0113592172317557\\
387	0.0113611783817345\\
388	0.0113631801589288\\
389	0.0113652234641618\\
390	0.0113673092214005\\
391	0.0113694383781505\\
392	0.011371611905316\\
393	0.0113738307956219\\
394	0.0113760960582649\\
395	0.0113784087040299\\
396	0.0113807697074683\\
397	0.0113831799184833\\
398	0.0113856403676593\\
399	0.0113881527962993\\
400	0.0113907183294125\\
401	0.0113933381170359\\
402	0.0113960133347759\\
403	0.0113987451843574\\
404	0.0114015348941784\\
405	0.0114043837198715\\
406	0.0114072929448722\\
407	0.0114102638809921\\
408	0.0114132978689921\\
409	0.0114163962791378\\
410	0.011419560511758\\
411	0.0114227919978371\\
412	0.011426092199583\\
413	0.0114294626109854\\
414	0.0114329047583655\\
415	0.0114364202009167\\
416	0.0114400105312306\\
417	0.0114436773758051\\
418	0.0114474223955253\\
419	0.0114512472861004\\
420	0.011455153778408\\
421	0.0114591436386726\\
422	0.011463218668338\\
423	0.0114673807035545\\
424	0.0114716316144898\\
425	0.0114759733070446\\
426	0.0114804077392078\\
427	0.0114849369131814\\
428	0.0114895628525918\\
429	0.011494287621703\\
430	0.0114991133259348\\
431	0.0115040421123582\\
432	0.0115090761701658\\
433	0.0115142177311134\\
434	0.0115194690699294\\
435	0.0115248325046872\\
436	0.0115303103971372\\
437	0.0115359051529925\\
438	0.011541619222163\\
439	0.0115474550989317\\
440	0.0115534153220635\\
441	0.0115595024748401\\
442	0.0115657191850129\\
443	0.0115720681246817\\
444	0.0115785520101438\\
445	0.0115851736018212\\
446	0.0115919357042626\\
447	0.0115988411646546\\
448	0.0116058928723007\\
449	0.0116130937581414\\
450	0.0116204467936933\\
451	0.0116279549897897\\
452	0.0116356213951004\\
453	0.011643449094412\\
454	0.0116514412066435\\
455	0.0116596008825729\\
456	0.0116679313022476\\
457	0.0116764356720505\\
458	0.0116851172213899\\
459	0.0116939791989828\\
460	0.0117030248686985\\
461	0.0117122575049196\\
462	0.0117216803873619\\
463	0.0117312967952552\\
464	0.0117411100008999\\
465	0.0117511232622166\\
466	0.0117613398136458\\
467	0.0117717628534318\\
468	0.0117823955214753\\
469	0.0117932408502661\\
470	0.0118043016357616\\
471	0.0118155800651896\\
472	0.0118270765966559\\
473	0.0118387865091484\\
474	0.0118506924852321\\
475	0.0118628414200771\\
476	0.0118752354806045\\
477	0.0118878854083289\\
478	0.0119007971910868\\
479	0.0119149717250346\\
480	0.0119292492072798\\
481	0.0119436199748858\\
482	0.0119580713142637\\
483	0.0119725845726865\\
484	0.0119870708740365\\
485	0.0120009119529355\\
486	0.0120149260477244\\
487	0.0120291096423274\\
488	0.0120434579807057\\
489	0.0120579636188418\\
490	0.0120726122065619\\
491	0.0120872825221991\\
492	0.0121020742309012\\
493	0.0121170377351672\\
494	0.0121321673614265\\
495	0.0121474616644167\\
496	0.0121629343637475\\
497	0.0121786530264737\\
498	0.0121948547977051\\
499	0.012221457378298\\
500	0.0122497871374526\\
501	0.0122767903833679\\
502	0.0122990963844984\\
503	0.0123126577049702\\
504	0.0123251859336564\\
505	0.0123365518283011\\
506	0.0123458266959497\\
507	0.012354006909702\\
508	0.0123620707926574\\
509	0.0123700731811967\\
510	0.0123781756799936\\
511	0.0123864162622617\\
512	0.0123948132252244\\
513	0.0124033832834578\\
514	0.0124121326522197\\
515	0.0124210669970876\\
516	0.0124301912993531\\
517	0.0124395098495792\\
518	0.0124490268827363\\
519	0.0124587465968323\\
520	0.0124686731839293\\
521	0.0124788108770463\\
522	0.0124891639522178\\
523	0.0124997367307636\\
524	0.0125105335854768\\
525	0.012521558958305\\
526	0.0125328174269855\\
527	0.0125443139228337\\
528	0.0125560544133927\\
529	0.0125680480407597\\
530	0.0125818596302423\\
531	0.0125958734095194\\
532	0.0126087894960663\\
533	0.0126209508645745\\
534	0.0126331344642834\\
535	0.012645168775862\\
536	0.012657399769164\\
537	0.0126698368094777\\
538	0.0126824850373347\\
539	0.0126953459640422\\
540	0.0127084205674275\\
541	0.0127217095726791\\
542	0.0127352134366965\\
543	0.0127489323316446\\
544	0.0127628661204725\\
545	0.0127770143306754\\
546	0.0127913761272473\\
547	0.0128059502864518\\
548	0.0128207351763738\\
549	0.0128357287676952\\
550	0.0128509287688814\\
551	0.0128663332651597\\
552	0.0128829974101976\\
553	0.0128992255349403\\
554	0.0129144037198701\\
555	0.0129294166252927\\
556	0.0129445962025297\\
557	0.0129599414039861\\
558	0.0129754481860902\\
559	0.0129911120830214\\
560	0.0130069282939308\\
561	0.0130228919582639\\
562	0.0130389989624544\\
563	0.0130554776437141\\
564	0.0130722424729182\\
565	0.0130882887545376\\
566	0.0131043428556736\\
567	0.013120509715601\\
568	0.0131367816514183\\
569	0.0131531501362835\\
570	0.0131696057273492\\
571	0.0131861379872316\\
572	0.0132027353983147\\
573	0.0132193852690868\\
574	0.0132360736315842\\
575	0.0132527851288787\\
576	0.0132695028913739\\
577	0.0132862084004767\\
578	0.0133028813379804\\
579	0.0133194994192331\\
580	0.0133360382079\\
581	0.0133524709099319\\
582	0.0133687681444567\\
583	0.0133848976903317\\
584	0.0134008242107329\\
585	0.0134165089689716\\
586	0.0134319095789188\\
587	0.0134469799156334\\
588	0.0134616705332227\\
589	0.0134759305304731\\
590	0.0134902483101722\\
591	0.0135046714122469\\
592	0.0135192196273521\\
593	0.0135339766714342\\
594	0.0135491998980231\\
595	0.0135656125153459\\
596	0.0135851815686143\\
597	0.0136131889078992\\
598	0.0136637438596031\\
599	0\\
600	0\\
};
\end{axis}
\end{tikzpicture}%
 
%  \caption{Discrete Time}
%\end{subfigure}\\
%\vspace{1cm}
%\begin{subfigure}{.45\linewidth}
%  \centering
%  \setlength\figureheight{\linewidth} 
%  \setlength\figurewidth{\linewidth}
%  \tikzsetnextfilename{dp_cts_nFPC_z1}
%  % This file was created by matlab2tikz.
%
%The latest updates can be retrieved from
%  http://www.mathworks.com/matlabcentral/fileexchange/22022-matlab2tikz-matlab2tikz
%where you can also make suggestions and rate matlab2tikz.
%
\definecolor{mycolor1}{rgb}{0.00000,1.00000,0.14286}%
\definecolor{mycolor2}{rgb}{0.00000,1.00000,0.28571}%
\definecolor{mycolor3}{rgb}{0.00000,1.00000,0.42857}%
\definecolor{mycolor4}{rgb}{0.00000,1.00000,0.57143}%
\definecolor{mycolor5}{rgb}{0.00000,1.00000,0.71429}%
\definecolor{mycolor6}{rgb}{0.00000,1.00000,0.85714}%
\definecolor{mycolor7}{rgb}{0.00000,1.00000,1.00000}%
\definecolor{mycolor8}{rgb}{0.00000,0.87500,1.00000}%
\definecolor{mycolor9}{rgb}{0.00000,0.62500,1.00000}%
\definecolor{mycolor10}{rgb}{0.12500,0.00000,1.00000}%
\definecolor{mycolor11}{rgb}{0.25000,0.00000,1.00000}%
\definecolor{mycolor12}{rgb}{0.37500,0.00000,1.00000}%
\definecolor{mycolor13}{rgb}{0.50000,0.00000,1.00000}%
\definecolor{mycolor14}{rgb}{0.62500,0.00000,1.00000}%
\definecolor{mycolor15}{rgb}{0.75000,0.00000,1.00000}%
\definecolor{mycolor16}{rgb}{0.87500,0.00000,1.00000}%
\definecolor{mycolor17}{rgb}{1.00000,0.00000,1.00000}%
\definecolor{mycolor18}{rgb}{1.00000,0.00000,0.87500}%
\definecolor{mycolor19}{rgb}{1.00000,0.00000,0.62500}%
\definecolor{mycolor20}{rgb}{0.85714,0.00000,0.00000}%
\definecolor{mycolor21}{rgb}{0.71429,0.00000,0.00000}%
%
\begin{tikzpicture}[trim axis left, trim axis right]

\begin{axis}[%
width=\figurewidth,
height=\figureheight,
at={(0\figurewidth,0\figureheight)},
scale only axis,
every outer x axis line/.append style={black},
every x tick label/.append style={font=\color{black}},
xmin=0,
xmax=600,
every outer y axis line/.append style={black},
every y tick label/.append style={font=\color{black}},
ymin=0,
ymax=0.014,
axis background/.style={fill=white},
axis x line*=bottom,
axis y line*=left,
yticklabel style={
        /pgf/number format/fixed,
        /pgf/number format/precision=3
},
scaled y ticks=false
]
\addplot [color=green,solid,forget plot]
  table[row sep=crcr]{%
0.01	1.73472347597681e-18\\
1.01	1.73472347597681e-18\\
2.01	0\\
3.01	1.73472347597681e-18\\
4.01	1.73472347597681e-18\\
5.01	0\\
6.01	0\\
7.01	1.73472347597681e-18\\
8.01	0\\
9.01	0\\
10.01	0\\
11.01	0\\
12.01	1.73472347597681e-18\\
13.01	1.73472347597681e-18\\
14.01	1.73472347597681e-18\\
15.01	1.73472347597681e-18\\
16.01	1.73472347597681e-18\\
17.01	1.73472347597681e-18\\
18.01	0\\
19.01	0\\
20.01	1.73472347597681e-18\\
21.01	0\\
22.01	1.73472347597681e-18\\
23.01	1.73472347597681e-18\\
24.01	1.73472347597681e-18\\
25.01	0\\
26.01	1.73472347597681e-18\\
27.01	0\\
28.01	1.73472347597681e-18\\
29.01	1.73472347597681e-18\\
30.01	0\\
31.01	0\\
32.01	1.73472347597681e-18\\
33.01	1.73472347597681e-18\\
34.01	1.73472347597681e-18\\
35.01	0\\
36.01	0\\
37.01	0\\
38.01	0\\
39.01	1.73472347597681e-18\\
40.01	1.73472347597681e-18\\
41.01	1.73472347597681e-18\\
42.01	1.73472347597681e-18\\
43.01	1.73472347597681e-18\\
44.01	0\\
45.01	1.73472347597681e-18\\
46.01	1.73472347597681e-18\\
47.01	1.73472347597681e-18\\
48.01	1.73472347597681e-18\\
49.01	0\\
50.01	1.73472347597681e-18\\
51.01	1.73472347597681e-18\\
52.01	0\\
53.01	0\\
54.01	0\\
55.01	1.73472347597681e-18\\
56.01	0\\
57.01	1.73472347597681e-18\\
58.01	0\\
59.01	0\\
60.01	1.73472347597681e-18\\
61.01	1.73472347597681e-18\\
62.01	1.73472347597681e-18\\
63.01	0\\
64.01	1.73472347597681e-18\\
65.01	0\\
66.01	1.73472347597681e-18\\
67.01	1.73472347597681e-18\\
68.01	0\\
69.01	1.73472347597681e-18\\
70.01	1.73472347597681e-18\\
71.01	0\\
72.01	0\\
73.01	1.73472347597681e-18\\
74.01	0\\
75.01	1.73472347597681e-18\\
76.01	0\\
77.01	0\\
78.01	0\\
79.01	0\\
80.01	1.73472347597681e-18\\
81.01	1.73472347597681e-18\\
82.01	0\\
83.01	0\\
84.01	0\\
85.01	1.73472347597681e-18\\
86.01	1.73472347597681e-18\\
87.01	0\\
88.01	0\\
89.01	0\\
90.01	1.73472347597681e-18\\
91.01	1.73472347597681e-18\\
92.01	0\\
93.01	1.73472347597681e-18\\
94.01	1.73472347597681e-18\\
95.01	0\\
96.01	1.73472347597681e-18\\
97.01	0\\
98.01	0\\
99.01	0\\
100.01	1.73472347597681e-18\\
101.01	0\\
102.01	1.73472347597681e-18\\
103.01	1.73472347597681e-18\\
104.01	0\\
105.01	0\\
106.01	0\\
107.01	0\\
108.01	1.73472347597681e-18\\
109.01	0\\
110.01	1.73472347597681e-18\\
111.01	0\\
112.01	0\\
113.01	0\\
114.01	1.73472347597681e-18\\
115.01	1.73472347597681e-18\\
116.01	0\\
117.01	0\\
118.01	0\\
119.01	0\\
120.01	0\\
121.01	0\\
122.01	0\\
123.01	1.73472347597681e-18\\
124.01	1.73472347597681e-18\\
125.01	1.73472347597681e-18\\
126.01	1.73472347597681e-18\\
127.01	0\\
128.01	1.73472347597681e-18\\
129.01	0\\
130.01	1.73472347597681e-18\\
131.01	1.73472347597681e-18\\
132.01	0\\
133.01	1.73472347597681e-18\\
134.01	1.73472347597681e-18\\
135.01	0\\
136.01	0\\
137.01	1.73472347597681e-18\\
138.01	0\\
139.01	1.73472347597681e-18\\
140.01	1.73472347597681e-18\\
141.01	1.73472347597681e-18\\
142.01	1.73472347597681e-18\\
143.01	0\\
144.01	1.73472347597681e-18\\
145.01	1.73472347597681e-18\\
146.01	1.73472347597681e-18\\
147.01	1.73472347597681e-18\\
148.01	1.73472347597681e-18\\
149.01	0\\
150.01	0\\
151.01	1.73472347597681e-18\\
152.01	1.73472347597681e-18\\
153.01	1.73472347597681e-18\\
154.01	1.73472347597681e-18\\
155.01	1.73472347597681e-18\\
156.01	1.73472347597681e-18\\
157.01	0\\
158.01	1.73472347597681e-18\\
159.01	1.73472347597681e-18\\
160.01	1.73472347597681e-18\\
161.01	1.73472347597681e-18\\
162.01	1.73472347597681e-18\\
163.01	1.73472347597681e-18\\
164.01	0\\
165.01	0\\
166.01	1.73472347597681e-18\\
167.01	0\\
168.01	1.73472347597681e-18\\
169.01	0\\
170.01	1.73472347597681e-18\\
171.01	0\\
172.01	0\\
173.01	0\\
174.01	1.73472347597681e-18\\
175.01	1.73472347597681e-18\\
176.01	1.73472347597681e-18\\
177.01	0\\
178.01	1.73472347597681e-18\\
179.01	0\\
180.01	0\\
181.01	1.73472347597681e-18\\
182.01	0\\
183.01	0\\
184.01	0\\
185.01	1.73472347597681e-18\\
186.01	0\\
187.01	0\\
188.01	0\\
189.01	0\\
190.01	1.73472347597681e-18\\
191.01	0\\
192.01	1.73472347597681e-18\\
193.01	0\\
194.01	1.73472347597681e-18\\
195.01	0\\
196.01	1.73472347597681e-18\\
197.01	0\\
198.01	0\\
199.01	0\\
200.01	0\\
201.01	0\\
202.01	1.73472347597681e-18\\
203.01	0\\
204.01	1.73472347597681e-18\\
205.01	1.73472347597681e-18\\
206.01	1.73472347597681e-18\\
207.01	1.73472347597681e-18\\
208.01	0\\
209.01	1.73472347597681e-18\\
210.01	0\\
211.01	1.73472347597681e-18\\
212.01	0\\
213.01	1.73472347597681e-18\\
214.01	0\\
215.01	0\\
216.01	0\\
217.01	0\\
218.01	1.73472347597681e-18\\
219.01	1.73472347597681e-18\\
220.01	0\\
221.01	0\\
222.01	0\\
223.01	1.73472347597681e-18\\
224.01	1.73472347597681e-18\\
225.01	0\\
226.01	1.73472347597681e-18\\
227.01	0\\
228.01	0\\
229.01	1.73472347597681e-18\\
230.01	0\\
231.01	0\\
232.01	0\\
233.01	0\\
234.01	1.73472347597681e-18\\
235.01	0\\
236.01	1.73472347597681e-18\\
237.01	1.73472347597681e-18\\
238.01	0\\
239.01	1.73472347597681e-18\\
240.01	0\\
241.01	1.73472347597681e-18\\
242.01	0\\
243.01	1.73472347597681e-18\\
244.01	1.73472347597681e-18\\
245.01	1.73472347597681e-18\\
246.01	1.73472347597681e-18\\
247.01	0\\
248.01	0\\
249.01	0\\
250.01	1.73472347597681e-18\\
251.01	0\\
252.01	1.73472347597681e-18\\
253.01	1.73472347597681e-18\\
254.01	1.73472347597681e-18\\
255.01	1.73472347597681e-18\\
256.01	1.73472347597681e-18\\
257.01	1.73472347597681e-18\\
258.01	0\\
259.01	0\\
260.01	0\\
261.01	0\\
262.01	1.73472347597681e-18\\
263.01	1.73472347597681e-18\\
264.01	0\\
265.01	1.73472347597681e-18\\
266.01	1.73472347597681e-18\\
267.01	0\\
268.01	0\\
269.01	1.73472347597681e-18\\
270.01	0\\
271.01	1.73472347597681e-18\\
272.01	0\\
273.01	1.73472347597681e-18\\
274.01	0\\
275.01	0\\
276.01	1.73472347597681e-18\\
277.01	1.73472347597681e-18\\
278.01	0\\
279.01	1.73472347597681e-18\\
280.01	0\\
281.01	1.73472347597681e-18\\
282.01	1.73472347597681e-18\\
283.01	1.73472347597681e-18\\
284.01	0\\
285.01	1.73472347597681e-18\\
286.01	1.73472347597681e-18\\
287.01	1.73472347597681e-18\\
288.01	1.73472347597681e-18\\
289.01	0\\
290.01	0\\
291.01	0\\
292.01	0\\
293.01	1.73472347597681e-18\\
294.01	1.73472347597681e-18\\
295.01	0\\
296.01	1.73472347597681e-18\\
297.01	0\\
298.01	1.73472347597681e-18\\
299.01	0\\
300.01	1.73472347597681e-18\\
301.01	0\\
302.01	0\\
303.01	0\\
304.01	1.73472347597681e-18\\
305.01	1.73472347597681e-18\\
306.01	0\\
307.01	1.73472347597681e-18\\
308.01	0\\
309.01	1.73472347597681e-18\\
310.01	1.73472347597681e-18\\
311.01	0\\
312.01	1.73472347597681e-18\\
313.01	1.73472347597681e-18\\
314.01	0\\
315.01	1.73472347597681e-18\\
316.01	0\\
317.01	1.73472347597681e-18\\
318.01	0\\
319.01	0\\
320.01	0\\
321.01	0\\
322.01	1.73472347597681e-18\\
323.01	1.73472347597681e-18\\
324.01	0\\
325.01	1.73472347597681e-18\\
326.01	1.73472347597681e-18\\
327.01	0\\
328.01	1.73472347597681e-18\\
329.01	0\\
330.01	0\\
331.01	1.73472347597681e-18\\
332.01	1.73472347597681e-18\\
333.01	1.73472347597681e-18\\
334.01	1.73472347597681e-18\\
335.01	1.73472347597681e-18\\
336.01	0\\
337.01	1.73472347597681e-18\\
338.01	0\\
339.01	0\\
340.01	1.73472347597681e-18\\
341.01	1.73472347597681e-18\\
342.01	1.73472347597681e-18\\
343.01	1.73472347597681e-18\\
344.01	0\\
345.01	0\\
346.01	0\\
347.01	1.73472347597681e-18\\
348.01	0\\
349.01	1.73472347597681e-18\\
350.01	0\\
351.01	0\\
352.01	0\\
353.01	1.73472347597681e-18\\
354.01	1.73472347597681e-18\\
355.01	0\\
356.01	0\\
357.01	1.73472347597681e-18\\
358.01	1.73472347597681e-18\\
359.01	1.73472347597681e-18\\
360.01	0\\
361.01	0\\
362.01	0\\
363.01	0\\
364.01	0\\
365.01	1.73472347597681e-18\\
366.01	1.73472347597681e-18\\
367.01	1.73472347597681e-18\\
368.01	1.73472347597681e-18\\
369.01	0\\
370.01	0\\
371.01	1.73472347597681e-18\\
372.01	1.73472347597681e-18\\
373.01	1.73472347597681e-18\\
374.01	0\\
375.01	1.73472347597681e-18\\
376.01	0\\
377.01	0\\
378.01	1.73472347597681e-18\\
379.01	0\\
380.01	1.73472347597681e-18\\
381.01	1.73472347597681e-18\\
382.01	1.73472347597681e-18\\
383.01	0\\
384.01	1.73472347597681e-18\\
385.01	0\\
386.01	1.73472347597681e-18\\
387.01	1.73472347597681e-18\\
388.01	1.73472347597681e-18\\
389.01	0\\
390.01	1.73472347597681e-18\\
391.01	1.73472347597681e-18\\
392.01	0\\
393.01	0\\
394.01	1.73472347597681e-18\\
395.01	1.73472347597681e-18\\
396.01	1.73472347597681e-18\\
397.01	1.73472347597681e-18\\
398.01	0\\
399.01	1.73472347597681e-18\\
400.01	0\\
401.01	0\\
402.01	0\\
403.01	1.73472347597681e-18\\
404.01	1.73472347597681e-18\\
405.01	1.73472347597681e-18\\
406.01	1.73472347597681e-18\\
407.01	1.73472347597681e-18\\
408.01	1.73472347597681e-18\\
409.01	0\\
410.01	1.73472347597681e-18\\
411.01	0\\
412.01	1.73472347597681e-18\\
413.01	1.73472347597681e-18\\
414.01	0\\
415.01	0\\
416.01	1.73472347597681e-18\\
417.01	1.73472347597681e-18\\
418.01	1.73472347597681e-18\\
419.01	1.73472347597681e-18\\
420.01	0\\
421.01	1.73472347597681e-18\\
422.01	1.73472347597681e-18\\
423.01	1.73472347597681e-18\\
424.01	1.73472347597681e-18\\
425.01	0\\
426.01	1.73472347597681e-18\\
427.01	1.73472347597681e-18\\
428.01	1.73472347597681e-18\\
429.01	0\\
430.01	1.73472347597681e-18\\
431.01	1.73472347597681e-18\\
432.01	1.73472347597681e-18\\
433.01	0\\
434.01	1.73472347597681e-18\\
435.01	1.73472347597681e-18\\
436.01	1.73472347597681e-18\\
437.01	1.73472347597681e-18\\
438.01	1.73472347597681e-18\\
439.01	0\\
440.01	1.73472347597681e-18\\
441.01	1.73472347597681e-18\\
442.01	0\\
443.01	0\\
444.01	0\\
445.01	1.73472347597681e-18\\
446.01	1.73472347597681e-18\\
447.01	0\\
448.01	1.73472347597681e-18\\
449.01	1.73472347597681e-18\\
450.01	0\\
451.01	0\\
452.01	0\\
453.01	1.73472347597681e-18\\
454.01	1.73472347597681e-18\\
455.01	0\\
456.01	1.73472347597681e-18\\
457.01	1.73472347597681e-18\\
458.01	0\\
459.01	1.73472347597681e-18\\
460.01	0\\
461.01	1.73472347597681e-18\\
462.01	0\\
463.01	0\\
464.01	0\\
465.01	1.73472347597681e-18\\
466.01	0\\
467.01	1.73472347597681e-18\\
468.01	0\\
469.01	1.73472347597681e-18\\
470.01	1.73472347597681e-18\\
471.01	1.73472347597681e-18\\
472.01	0\\
473.01	0\\
474.01	0\\
475.01	1.73472347597681e-18\\
476.01	0\\
477.01	0\\
478.01	1.73472347597681e-18\\
479.01	1.73472347597681e-18\\
480.01	0\\
481.01	0\\
482.01	1.73472347597681e-18\\
483.01	1.73472347597681e-18\\
484.01	0\\
485.01	0\\
486.01	0\\
487.01	1.73472347597681e-18\\
488.01	1.73472347597681e-18\\
489.01	1.73472347597681e-18\\
490.01	1.73472347597681e-18\\
491.01	0\\
492.01	1.73472347597681e-18\\
493.01	1.73472347597681e-18\\
494.01	0\\
495.01	1.73472347597681e-18\\
496.01	0\\
497.01	1.73472347597681e-18\\
498.01	1.73472347597681e-18\\
499.01	0\\
500.01	1.73472347597681e-18\\
501.01	1.73472347597681e-18\\
502.01	0\\
503.01	1.73472347597681e-18\\
504.01	1.73472347597681e-18\\
505.01	1.73472347597681e-18\\
506.01	0\\
507.01	0\\
508.01	0\\
509.01	0\\
510.01	1.73472347597681e-18\\
511.01	1.73472347597681e-18\\
512.01	0\\
513.01	1.73472347597681e-18\\
514.01	1.73472347597681e-18\\
515.01	0\\
516.01	0\\
517.01	1.73472347597681e-18\\
518.01	1.73472347597681e-18\\
519.01	1.73472347597681e-18\\
520.01	0\\
521.01	0\\
522.01	0\\
523.01	1.73472347597681e-18\\
524.01	0\\
525.01	1.73472347597681e-18\\
526.01	1.73472347597681e-18\\
527.01	0\\
528.01	0\\
529.01	0\\
530.01	1.73472347597681e-18\\
531.01	0\\
532.01	1.73472347597681e-18\\
533.01	0\\
534.01	1.73472347597681e-18\\
535.01	0\\
536.01	1.73472347597681e-18\\
537.01	1.73472347597681e-18\\
538.01	1.73472347597681e-18\\
539.01	0\\
540.01	0\\
541.01	1.73472347597681e-18\\
542.01	1.73472347597681e-18\\
543.01	0\\
544.01	1.73472347597681e-18\\
545.01	0\\
546.01	1.73472347597681e-18\\
547.01	0\\
548.01	0\\
549.01	0\\
550.01	0\\
551.01	0\\
552.01	0\\
553.01	0\\
554.01	0\\
555.01	1.73472347597681e-18\\
556.01	0\\
557.01	0\\
558.01	0\\
559.01	1.73472347597681e-18\\
560.01	1.73472347597681e-18\\
561.01	1.73472347597681e-18\\
562.01	1.73472347597681e-18\\
563.01	0\\
564.01	0\\
565.01	1.73472347597681e-18\\
566.01	0\\
567.01	0\\
568.01	0\\
569.01	0\\
570.01	0\\
571.01	0\\
572.01	1.73472347597681e-18\\
573.01	0\\
574.01	1.73472347597681e-18\\
575.01	0\\
576.01	1.73472347597681e-18\\
577.01	0\\
578.01	1.73472347597681e-18\\
579.01	0\\
580.01	0\\
581.01	1.73472347597681e-18\\
582.01	0\\
583.01	0\\
584.01	0\\
585.01	0\\
586.01	0\\
587.01	0\\
588.01	0\\
589.01	0\\
590.01	0\\
591.01	0\\
592.01	1.73472347597681e-18\\
593.01	0\\
594.01	0\\
595.01	0\\
596.01	0\\
597.01	0\\
598.01	0\\
599.01	0\\
599.02	0\\
599.03	1.73472347597681e-18\\
599.04	1.73472347597681e-18\\
599.05	0\\
599.06	0\\
599.07	1.73472347597681e-18\\
599.08	0\\
599.09	0\\
599.1	0\\
599.11	1.73472347597681e-18\\
599.12	1.73472347597681e-18\\
599.13	0\\
599.14	0\\
599.15	1.73472347597681e-18\\
599.16	1.73472347597681e-18\\
599.17	1.73472347597681e-18\\
599.18	0\\
599.19	0\\
599.2	0\\
599.21	0\\
599.22	0\\
599.23	1.73472347597681e-18\\
599.24	1.73472347597681e-18\\
599.25	1.73472347597681e-18\\
599.26	1.73472347597681e-18\\
599.27	0\\
599.28	0\\
599.29	0\\
599.3	0\\
599.31	0\\
599.32	1.73472347597681e-18\\
599.33	1.73472347597681e-18\\
599.34	0\\
599.35	0\\
599.36	0\\
599.37	1.73472347597681e-18\\
599.38	0\\
599.39	0\\
599.4	0\\
599.41	0\\
599.42	0\\
599.43	0\\
599.44	0\\
599.45	0\\
599.46	1.73472347597681e-18\\
599.47	1.73472347597681e-18\\
599.48	1.73472347597681e-18\\
599.49	1.73472347597681e-18\\
599.5	1.73472347597681e-18\\
599.51	0\\
599.52	0\\
599.53	1.73472347597681e-18\\
599.54	0\\
599.55	0\\
599.56	0\\
599.57	1.73472347597681e-18\\
599.58	0\\
599.59	0\\
599.6	0\\
599.61	0\\
599.62	1.73472347597681e-18\\
599.63	0\\
599.64	1.73472347597681e-18\\
599.65	0\\
599.66	0\\
599.67	0\\
599.68	0\\
599.69	1.73472347597681e-18\\
599.7	0\\
599.71	1.73472347597681e-18\\
599.72	1.73472347597681e-18\\
599.73	1.73472347597681e-18\\
599.74	0\\
599.75	0\\
599.76	0\\
599.77	0\\
599.78	0\\
599.79	0\\
599.8	0\\
599.81	0\\
599.82	0\\
599.83	1.73472347597681e-18\\
599.84	0\\
599.85	0\\
599.86	0\\
599.87	0\\
599.88	0\\
599.89	0\\
599.9	0\\
599.91	0\\
599.92	0\\
599.93	0\\
599.94	0\\
599.95	0\\
599.96	0\\
599.97	0\\
599.98	0\\
599.99	0\\
600	0\\
};
\addplot [color=mycolor1,solid,forget plot]
  table[row sep=crcr]{%
0.01	1.73472347597681e-18\\
1.01	1.73472347597681e-18\\
2.01	0\\
3.01	1.73472347597681e-18\\
4.01	1.73472347597681e-18\\
5.01	0\\
6.01	0\\
7.01	1.73472347597681e-18\\
8.01	0\\
9.01	0\\
10.01	0\\
11.01	0\\
12.01	1.73472347597681e-18\\
13.01	1.73472347597681e-18\\
14.01	1.73472347597681e-18\\
15.01	1.73472347597681e-18\\
16.01	1.73472347597681e-18\\
17.01	1.73472347597681e-18\\
18.01	0\\
19.01	0\\
20.01	1.73472347597681e-18\\
21.01	0\\
22.01	1.73472347597681e-18\\
23.01	1.73472347597681e-18\\
24.01	1.73472347597681e-18\\
25.01	0\\
26.01	1.73472347597681e-18\\
27.01	0\\
28.01	1.73472347597681e-18\\
29.01	1.73472347597681e-18\\
30.01	0\\
31.01	0\\
32.01	1.73472347597681e-18\\
33.01	1.73472347597681e-18\\
34.01	1.73472347597681e-18\\
35.01	0\\
36.01	0\\
37.01	0\\
38.01	0\\
39.01	1.73472347597681e-18\\
40.01	1.73472347597681e-18\\
41.01	1.73472347597681e-18\\
42.01	1.73472347597681e-18\\
43.01	1.73472347597681e-18\\
44.01	0\\
45.01	1.73472347597681e-18\\
46.01	1.73472347597681e-18\\
47.01	1.73472347597681e-18\\
48.01	1.73472347597681e-18\\
49.01	0\\
50.01	1.73472347597681e-18\\
51.01	1.73472347597681e-18\\
52.01	0\\
53.01	0\\
54.01	0\\
55.01	1.73472347597681e-18\\
56.01	0\\
57.01	1.73472347597681e-18\\
58.01	0\\
59.01	0\\
60.01	1.73472347597681e-18\\
61.01	1.73472347597681e-18\\
62.01	1.73472347597681e-18\\
63.01	0\\
64.01	1.73472347597681e-18\\
65.01	0\\
66.01	1.73472347597681e-18\\
67.01	1.73472347597681e-18\\
68.01	0\\
69.01	1.73472347597681e-18\\
70.01	1.73472347597681e-18\\
71.01	0\\
72.01	0\\
73.01	1.73472347597681e-18\\
74.01	0\\
75.01	1.73472347597681e-18\\
76.01	0\\
77.01	0\\
78.01	0\\
79.01	0\\
80.01	1.73472347597681e-18\\
81.01	1.73472347597681e-18\\
82.01	0\\
83.01	0\\
84.01	0\\
85.01	1.73472347597681e-18\\
86.01	1.73472347597681e-18\\
87.01	0\\
88.01	0\\
89.01	0\\
90.01	1.73472347597681e-18\\
91.01	1.73472347597681e-18\\
92.01	0\\
93.01	1.73472347597681e-18\\
94.01	1.73472347597681e-18\\
95.01	0\\
96.01	1.73472347597681e-18\\
97.01	0\\
98.01	0\\
99.01	0\\
100.01	1.73472347597681e-18\\
101.01	0\\
102.01	1.73472347597681e-18\\
103.01	1.73472347597681e-18\\
104.01	0\\
105.01	0\\
106.01	0\\
107.01	0\\
108.01	1.73472347597681e-18\\
109.01	0\\
110.01	1.73472347597681e-18\\
111.01	0\\
112.01	0\\
113.01	0\\
114.01	1.73472347597681e-18\\
115.01	1.73472347597681e-18\\
116.01	0\\
117.01	0\\
118.01	0\\
119.01	0\\
120.01	0\\
121.01	0\\
122.01	0\\
123.01	1.73472347597681e-18\\
124.01	1.73472347597681e-18\\
125.01	1.73472347597681e-18\\
126.01	1.73472347597681e-18\\
127.01	0\\
128.01	1.73472347597681e-18\\
129.01	0\\
130.01	1.73472347597681e-18\\
131.01	1.73472347597681e-18\\
132.01	0\\
133.01	1.73472347597681e-18\\
134.01	1.73472347597681e-18\\
135.01	0\\
136.01	0\\
137.01	1.73472347597681e-18\\
138.01	0\\
139.01	1.73472347597681e-18\\
140.01	1.73472347597681e-18\\
141.01	1.73472347597681e-18\\
142.01	1.73472347597681e-18\\
143.01	0\\
144.01	1.73472347597681e-18\\
145.01	1.73472347597681e-18\\
146.01	1.73472347597681e-18\\
147.01	1.73472347597681e-18\\
148.01	1.73472347597681e-18\\
149.01	0\\
150.01	0\\
151.01	1.73472347597681e-18\\
152.01	1.73472347597681e-18\\
153.01	1.73472347597681e-18\\
154.01	1.73472347597681e-18\\
155.01	1.73472347597681e-18\\
156.01	1.73472347597681e-18\\
157.01	0\\
158.01	1.73472347597681e-18\\
159.01	1.73472347597681e-18\\
160.01	1.73472347597681e-18\\
161.01	1.73472347597681e-18\\
162.01	1.73472347597681e-18\\
163.01	1.73472347597681e-18\\
164.01	0\\
165.01	0\\
166.01	1.73472347597681e-18\\
167.01	0\\
168.01	1.73472347597681e-18\\
169.01	0\\
170.01	1.73472347597681e-18\\
171.01	0\\
172.01	0\\
173.01	0\\
174.01	1.73472347597681e-18\\
175.01	1.73472347597681e-18\\
176.01	1.73472347597681e-18\\
177.01	0\\
178.01	1.73472347597681e-18\\
179.01	0\\
180.01	0\\
181.01	1.73472347597681e-18\\
182.01	0\\
183.01	0\\
184.01	0\\
185.01	1.73472347597681e-18\\
186.01	0\\
187.01	0\\
188.01	0\\
189.01	0\\
190.01	1.73472347597681e-18\\
191.01	0\\
192.01	1.73472347597681e-18\\
193.01	0\\
194.01	1.73472347597681e-18\\
195.01	0\\
196.01	1.73472347597681e-18\\
197.01	0\\
198.01	0\\
199.01	0\\
200.01	0\\
201.01	0\\
202.01	1.73472347597681e-18\\
203.01	0\\
204.01	1.73472347597681e-18\\
205.01	1.73472347597681e-18\\
206.01	1.73472347597681e-18\\
207.01	1.73472347597681e-18\\
208.01	0\\
209.01	1.73472347597681e-18\\
210.01	0\\
211.01	1.73472347597681e-18\\
212.01	0\\
213.01	1.73472347597681e-18\\
214.01	0\\
215.01	0\\
216.01	0\\
217.01	0\\
218.01	1.73472347597681e-18\\
219.01	1.73472347597681e-18\\
220.01	0\\
221.01	0\\
222.01	0\\
223.01	1.73472347597681e-18\\
224.01	1.73472347597681e-18\\
225.01	0\\
226.01	1.73472347597681e-18\\
227.01	0\\
228.01	0\\
229.01	1.73472347597681e-18\\
230.01	0\\
231.01	0\\
232.01	0\\
233.01	0\\
234.01	1.73472347597681e-18\\
235.01	0\\
236.01	1.73472347597681e-18\\
237.01	1.73472347597681e-18\\
238.01	0\\
239.01	1.73472347597681e-18\\
240.01	0\\
241.01	1.73472347597681e-18\\
242.01	0\\
243.01	1.73472347597681e-18\\
244.01	1.73472347597681e-18\\
245.01	1.73472347597681e-18\\
246.01	1.73472347597681e-18\\
247.01	0\\
248.01	0\\
249.01	0\\
250.01	1.73472347597681e-18\\
251.01	0\\
252.01	1.73472347597681e-18\\
253.01	1.73472347597681e-18\\
254.01	1.73472347597681e-18\\
255.01	1.73472347597681e-18\\
256.01	1.73472347597681e-18\\
257.01	1.73472347597681e-18\\
258.01	0\\
259.01	0\\
260.01	0\\
261.01	0\\
262.01	1.73472347597681e-18\\
263.01	1.73472347597681e-18\\
264.01	0\\
265.01	1.73472347597681e-18\\
266.01	1.73472347597681e-18\\
267.01	0\\
268.01	0\\
269.01	1.73472347597681e-18\\
270.01	0\\
271.01	1.73472347597681e-18\\
272.01	0\\
273.01	1.73472347597681e-18\\
274.01	0\\
275.01	0\\
276.01	1.73472347597681e-18\\
277.01	1.73472347597681e-18\\
278.01	0\\
279.01	1.73472347597681e-18\\
280.01	0\\
281.01	1.73472347597681e-18\\
282.01	1.73472347597681e-18\\
283.01	1.73472347597681e-18\\
284.01	0\\
285.01	1.73472347597681e-18\\
286.01	1.73472347597681e-18\\
287.01	1.73472347597681e-18\\
288.01	1.73472347597681e-18\\
289.01	0\\
290.01	0\\
291.01	0\\
292.01	0\\
293.01	1.73472347597681e-18\\
294.01	1.73472347597681e-18\\
295.01	0\\
296.01	1.73472347597681e-18\\
297.01	0\\
298.01	1.73472347597681e-18\\
299.01	0\\
300.01	1.73472347597681e-18\\
301.01	0\\
302.01	0\\
303.01	0\\
304.01	1.73472347597681e-18\\
305.01	1.73472347597681e-18\\
306.01	0\\
307.01	1.73472347597681e-18\\
308.01	0\\
309.01	1.73472347597681e-18\\
310.01	1.73472347597681e-18\\
311.01	0\\
312.01	1.73472347597681e-18\\
313.01	1.73472347597681e-18\\
314.01	0\\
315.01	1.73472347597681e-18\\
316.01	0\\
317.01	1.73472347597681e-18\\
318.01	0\\
319.01	0\\
320.01	0\\
321.01	0\\
322.01	1.73472347597681e-18\\
323.01	1.73472347597681e-18\\
324.01	0\\
325.01	1.73472347597681e-18\\
326.01	1.73472347597681e-18\\
327.01	0\\
328.01	1.73472347597681e-18\\
329.01	0\\
330.01	0\\
331.01	1.73472347597681e-18\\
332.01	1.73472347597681e-18\\
333.01	1.73472347597681e-18\\
334.01	1.73472347597681e-18\\
335.01	1.73472347597681e-18\\
336.01	0\\
337.01	1.73472347597681e-18\\
338.01	0\\
339.01	0\\
340.01	1.73472347597681e-18\\
341.01	1.73472347597681e-18\\
342.01	1.73472347597681e-18\\
343.01	1.73472347597681e-18\\
344.01	0\\
345.01	0\\
346.01	0\\
347.01	1.73472347597681e-18\\
348.01	0\\
349.01	1.73472347597681e-18\\
350.01	0\\
351.01	0\\
352.01	0\\
353.01	1.73472347597681e-18\\
354.01	1.73472347597681e-18\\
355.01	0\\
356.01	0\\
357.01	1.73472347597681e-18\\
358.01	1.73472347597681e-18\\
359.01	1.73472347597681e-18\\
360.01	0\\
361.01	0\\
362.01	0\\
363.01	0\\
364.01	0\\
365.01	1.73472347597681e-18\\
366.01	1.73472347597681e-18\\
367.01	1.73472347597681e-18\\
368.01	1.73472347597681e-18\\
369.01	0\\
370.01	0\\
371.01	1.73472347597681e-18\\
372.01	1.73472347597681e-18\\
373.01	1.73472347597681e-18\\
374.01	0\\
375.01	1.73472347597681e-18\\
376.01	0\\
377.01	0\\
378.01	1.73472347597681e-18\\
379.01	0\\
380.01	1.73472347597681e-18\\
381.01	1.73472347597681e-18\\
382.01	1.73472347597681e-18\\
383.01	0\\
384.01	1.73472347597681e-18\\
385.01	0\\
386.01	1.73472347597681e-18\\
387.01	1.73472347597681e-18\\
388.01	1.73472347597681e-18\\
389.01	0\\
390.01	1.73472347597681e-18\\
391.01	1.73472347597681e-18\\
392.01	0\\
393.01	0\\
394.01	1.73472347597681e-18\\
395.01	1.73472347597681e-18\\
396.01	1.73472347597681e-18\\
397.01	1.73472347597681e-18\\
398.01	0\\
399.01	1.73472347597681e-18\\
400.01	0\\
401.01	0\\
402.01	0\\
403.01	1.73472347597681e-18\\
404.01	1.73472347597681e-18\\
405.01	1.73472347597681e-18\\
406.01	1.73472347597681e-18\\
407.01	1.73472347597681e-18\\
408.01	1.73472347597681e-18\\
409.01	0\\
410.01	1.73472347597681e-18\\
411.01	0\\
412.01	1.73472347597681e-18\\
413.01	1.73472347597681e-18\\
414.01	0\\
415.01	0\\
416.01	1.73472347597681e-18\\
417.01	1.73472347597681e-18\\
418.01	1.73472347597681e-18\\
419.01	1.73472347597681e-18\\
420.01	0\\
421.01	1.73472347597681e-18\\
422.01	1.73472347597681e-18\\
423.01	1.73472347597681e-18\\
424.01	1.73472347597681e-18\\
425.01	0\\
426.01	1.73472347597681e-18\\
427.01	1.73472347597681e-18\\
428.01	1.73472347597681e-18\\
429.01	0\\
430.01	1.73472347597681e-18\\
431.01	1.73472347597681e-18\\
432.01	1.73472347597681e-18\\
433.01	0\\
434.01	1.73472347597681e-18\\
435.01	1.73472347597681e-18\\
436.01	1.73472347597681e-18\\
437.01	1.73472347597681e-18\\
438.01	1.73472347597681e-18\\
439.01	0\\
440.01	1.73472347597681e-18\\
441.01	1.73472347597681e-18\\
442.01	0\\
443.01	0\\
444.01	0\\
445.01	1.73472347597681e-18\\
446.01	1.73472347597681e-18\\
447.01	0\\
448.01	1.73472347597681e-18\\
449.01	1.73472347597681e-18\\
450.01	0\\
451.01	0\\
452.01	0\\
453.01	1.73472347597681e-18\\
454.01	1.73472347597681e-18\\
455.01	0\\
456.01	1.73472347597681e-18\\
457.01	1.73472347597681e-18\\
458.01	0\\
459.01	1.73472347597681e-18\\
460.01	0\\
461.01	1.73472347597681e-18\\
462.01	0\\
463.01	0\\
464.01	0\\
465.01	1.73472347597681e-18\\
466.01	0\\
467.01	1.73472347597681e-18\\
468.01	0\\
469.01	1.73472347597681e-18\\
470.01	1.73472347597681e-18\\
471.01	1.73472347597681e-18\\
472.01	0\\
473.01	0\\
474.01	0\\
475.01	1.73472347597681e-18\\
476.01	0\\
477.01	0\\
478.01	1.73472347597681e-18\\
479.01	1.73472347597681e-18\\
480.01	0\\
481.01	0\\
482.01	1.73472347597681e-18\\
483.01	1.73472347597681e-18\\
484.01	0\\
485.01	0\\
486.01	0\\
487.01	1.73472347597681e-18\\
488.01	1.73472347597681e-18\\
489.01	1.73472347597681e-18\\
490.01	1.73472347597681e-18\\
491.01	0\\
492.01	1.73472347597681e-18\\
493.01	1.73472347597681e-18\\
494.01	0\\
495.01	1.73472347597681e-18\\
496.01	0\\
497.01	1.73472347597681e-18\\
498.01	1.73472347597681e-18\\
499.01	0\\
500.01	1.73472347597681e-18\\
501.01	1.73472347597681e-18\\
502.01	0\\
503.01	1.73472347597681e-18\\
504.01	1.73472347597681e-18\\
505.01	1.73472347597681e-18\\
506.01	0\\
507.01	0\\
508.01	0\\
509.01	0\\
510.01	1.73472347597681e-18\\
511.01	1.73472347597681e-18\\
512.01	0\\
513.01	1.73472347597681e-18\\
514.01	1.73472347597681e-18\\
515.01	0\\
516.01	0\\
517.01	1.73472347597681e-18\\
518.01	1.73472347597681e-18\\
519.01	1.73472347597681e-18\\
520.01	0\\
521.01	0\\
522.01	0\\
523.01	1.73472347597681e-18\\
524.01	0\\
525.01	1.73472347597681e-18\\
526.01	1.73472347597681e-18\\
527.01	0\\
528.01	0\\
529.01	0\\
530.01	1.73472347597681e-18\\
531.01	0\\
532.01	1.73472347597681e-18\\
533.01	0\\
534.01	1.73472347597681e-18\\
535.01	0\\
536.01	1.73472347597681e-18\\
537.01	1.73472347597681e-18\\
538.01	1.73472347597681e-18\\
539.01	0\\
540.01	0\\
541.01	1.73472347597681e-18\\
542.01	1.73472347597681e-18\\
543.01	0\\
544.01	1.73472347597681e-18\\
545.01	0\\
546.01	1.73472347597681e-18\\
547.01	0\\
548.01	0\\
549.01	0\\
550.01	0\\
551.01	0\\
552.01	0\\
553.01	0\\
554.01	0\\
555.01	1.73472347597681e-18\\
556.01	0\\
557.01	0\\
558.01	0\\
559.01	1.73472347597681e-18\\
560.01	1.73472347597681e-18\\
561.01	1.73472347597681e-18\\
562.01	1.73472347597681e-18\\
563.01	0\\
564.01	0\\
565.01	1.73472347597681e-18\\
566.01	0\\
567.01	0\\
568.01	0\\
569.01	0\\
570.01	0\\
571.01	0\\
572.01	1.73472347597681e-18\\
573.01	0\\
574.01	1.73472347597681e-18\\
575.01	0\\
576.01	1.73472347597681e-18\\
577.01	0\\
578.01	1.73472347597681e-18\\
579.01	0\\
580.01	0\\
581.01	1.73472347597681e-18\\
582.01	0\\
583.01	0\\
584.01	0\\
585.01	0\\
586.01	0\\
587.01	0\\
588.01	0\\
589.01	0\\
590.01	0\\
591.01	0\\
592.01	1.73472347597681e-18\\
593.01	0\\
594.01	0\\
595.01	0\\
596.01	0\\
597.01	0\\
598.01	0\\
599.01	0\\
599.02	0\\
599.03	1.73472347597681e-18\\
599.04	1.73472347597681e-18\\
599.05	0\\
599.06	0\\
599.07	1.73472347597681e-18\\
599.08	0\\
599.09	0\\
599.1	0\\
599.11	1.73472347597681e-18\\
599.12	1.73472347597681e-18\\
599.13	0\\
599.14	0\\
599.15	1.73472347597681e-18\\
599.16	1.73472347597681e-18\\
599.17	1.73472347597681e-18\\
599.18	0\\
599.19	0\\
599.2	0\\
599.21	0\\
599.22	0\\
599.23	1.73472347597681e-18\\
599.24	1.73472347597681e-18\\
599.25	1.73472347597681e-18\\
599.26	1.73472347597681e-18\\
599.27	0\\
599.28	0\\
599.29	0\\
599.3	0\\
599.31	0\\
599.32	1.73472347597681e-18\\
599.33	1.73472347597681e-18\\
599.34	0\\
599.35	0\\
599.36	0\\
599.37	1.73472347597681e-18\\
599.38	0\\
599.39	0\\
599.4	0\\
599.41	0\\
599.42	0\\
599.43	0\\
599.44	0\\
599.45	0\\
599.46	1.73472347597681e-18\\
599.47	1.73472347597681e-18\\
599.48	1.73472347597681e-18\\
599.49	1.73472347597681e-18\\
599.5	1.73472347597681e-18\\
599.51	0\\
599.52	0\\
599.53	1.73472347597681e-18\\
599.54	0\\
599.55	0\\
599.56	0\\
599.57	1.73472347597681e-18\\
599.58	0\\
599.59	0\\
599.6	0\\
599.61	0\\
599.62	1.73472347597681e-18\\
599.63	0\\
599.64	1.73472347597681e-18\\
599.65	0\\
599.66	0\\
599.67	0\\
599.68	0\\
599.69	1.73472347597681e-18\\
599.7	0\\
599.71	1.73472347597681e-18\\
599.72	1.73472347597681e-18\\
599.73	1.73472347597681e-18\\
599.74	0\\
599.75	0\\
599.76	0\\
599.77	0\\
599.78	0\\
599.79	0\\
599.8	0\\
599.81	0\\
599.82	0\\
599.83	1.73472347597681e-18\\
599.84	0\\
599.85	0\\
599.86	0\\
599.87	0\\
599.88	0\\
599.89	0\\
599.9	0\\
599.91	0\\
599.92	0\\
599.93	0\\
599.94	0\\
599.95	0\\
599.96	0\\
599.97	0\\
599.98	0\\
599.99	0\\
600	0\\
};
\addplot [color=mycolor2,solid,forget plot]
  table[row sep=crcr]{%
0.01	1.73472347597681e-18\\
1.01	1.73472347597681e-18\\
2.01	0\\
3.01	1.73472347597681e-18\\
4.01	1.73472347597681e-18\\
5.01	0\\
6.01	0\\
7.01	1.73472347597681e-18\\
8.01	0\\
9.01	0\\
10.01	0\\
11.01	0\\
12.01	1.73472347597681e-18\\
13.01	1.73472347597681e-18\\
14.01	1.73472347597681e-18\\
15.01	1.73472347597681e-18\\
16.01	1.73472347597681e-18\\
17.01	1.73472347597681e-18\\
18.01	0\\
19.01	0\\
20.01	1.73472347597681e-18\\
21.01	0\\
22.01	1.73472347597681e-18\\
23.01	1.73472347597681e-18\\
24.01	1.73472347597681e-18\\
25.01	0\\
26.01	1.73472347597681e-18\\
27.01	0\\
28.01	1.73472347597681e-18\\
29.01	1.73472347597681e-18\\
30.01	0\\
31.01	0\\
32.01	1.73472347597681e-18\\
33.01	1.73472347597681e-18\\
34.01	1.73472347597681e-18\\
35.01	0\\
36.01	0\\
37.01	0\\
38.01	0\\
39.01	1.73472347597681e-18\\
40.01	1.73472347597681e-18\\
41.01	1.73472347597681e-18\\
42.01	1.73472347597681e-18\\
43.01	1.73472347597681e-18\\
44.01	0\\
45.01	1.73472347597681e-18\\
46.01	1.73472347597681e-18\\
47.01	1.73472347597681e-18\\
48.01	1.73472347597681e-18\\
49.01	0\\
50.01	1.73472347597681e-18\\
51.01	1.73472347597681e-18\\
52.01	0\\
53.01	0\\
54.01	0\\
55.01	1.73472347597681e-18\\
56.01	0\\
57.01	1.73472347597681e-18\\
58.01	0\\
59.01	0\\
60.01	1.73472347597681e-18\\
61.01	1.73472347597681e-18\\
62.01	1.73472347597681e-18\\
63.01	0\\
64.01	1.73472347597681e-18\\
65.01	0\\
66.01	1.73472347597681e-18\\
67.01	1.73472347597681e-18\\
68.01	0\\
69.01	1.73472347597681e-18\\
70.01	1.73472347597681e-18\\
71.01	0\\
72.01	0\\
73.01	1.73472347597681e-18\\
74.01	0\\
75.01	1.73472347597681e-18\\
76.01	0\\
77.01	0\\
78.01	0\\
79.01	0\\
80.01	1.73472347597681e-18\\
81.01	1.73472347597681e-18\\
82.01	0\\
83.01	0\\
84.01	0\\
85.01	1.73472347597681e-18\\
86.01	1.73472347597681e-18\\
87.01	0\\
88.01	0\\
89.01	0\\
90.01	1.73472347597681e-18\\
91.01	1.73472347597681e-18\\
92.01	0\\
93.01	1.73472347597681e-18\\
94.01	1.73472347597681e-18\\
95.01	0\\
96.01	1.73472347597681e-18\\
97.01	0\\
98.01	0\\
99.01	0\\
100.01	1.73472347597681e-18\\
101.01	0\\
102.01	1.73472347597681e-18\\
103.01	1.73472347597681e-18\\
104.01	0\\
105.01	0\\
106.01	0\\
107.01	0\\
108.01	1.73472347597681e-18\\
109.01	0\\
110.01	1.73472347597681e-18\\
111.01	0\\
112.01	0\\
113.01	0\\
114.01	1.73472347597681e-18\\
115.01	1.73472347597681e-18\\
116.01	0\\
117.01	0\\
118.01	0\\
119.01	0\\
120.01	0\\
121.01	0\\
122.01	0\\
123.01	1.73472347597681e-18\\
124.01	1.73472347597681e-18\\
125.01	1.73472347597681e-18\\
126.01	1.73472347597681e-18\\
127.01	0\\
128.01	1.73472347597681e-18\\
129.01	0\\
130.01	1.73472347597681e-18\\
131.01	1.73472347597681e-18\\
132.01	0\\
133.01	1.73472347597681e-18\\
134.01	1.73472347597681e-18\\
135.01	0\\
136.01	0\\
137.01	1.73472347597681e-18\\
138.01	0\\
139.01	1.73472347597681e-18\\
140.01	1.73472347597681e-18\\
141.01	1.73472347597681e-18\\
142.01	1.73472347597681e-18\\
143.01	0\\
144.01	1.73472347597681e-18\\
145.01	1.73472347597681e-18\\
146.01	1.73472347597681e-18\\
147.01	1.73472347597681e-18\\
148.01	1.73472347597681e-18\\
149.01	0\\
150.01	0\\
151.01	1.73472347597681e-18\\
152.01	1.73472347597681e-18\\
153.01	1.73472347597681e-18\\
154.01	1.73472347597681e-18\\
155.01	1.73472347597681e-18\\
156.01	1.73472347597681e-18\\
157.01	0\\
158.01	1.73472347597681e-18\\
159.01	1.73472347597681e-18\\
160.01	1.73472347597681e-18\\
161.01	1.73472347597681e-18\\
162.01	1.73472347597681e-18\\
163.01	1.73472347597681e-18\\
164.01	0\\
165.01	0\\
166.01	1.73472347597681e-18\\
167.01	0\\
168.01	1.73472347597681e-18\\
169.01	0\\
170.01	1.73472347597681e-18\\
171.01	0\\
172.01	0\\
173.01	0\\
174.01	1.73472347597681e-18\\
175.01	1.73472347597681e-18\\
176.01	1.73472347597681e-18\\
177.01	0\\
178.01	1.73472347597681e-18\\
179.01	0\\
180.01	0\\
181.01	1.73472347597681e-18\\
182.01	0\\
183.01	0\\
184.01	0\\
185.01	1.73472347597681e-18\\
186.01	0\\
187.01	0\\
188.01	0\\
189.01	0\\
190.01	1.73472347597681e-18\\
191.01	0\\
192.01	1.73472347597681e-18\\
193.01	0\\
194.01	1.73472347597681e-18\\
195.01	0\\
196.01	1.73472347597681e-18\\
197.01	0\\
198.01	0\\
199.01	0\\
200.01	0\\
201.01	0\\
202.01	1.73472347597681e-18\\
203.01	0\\
204.01	1.73472347597681e-18\\
205.01	1.73472347597681e-18\\
206.01	1.73472347597681e-18\\
207.01	1.73472347597681e-18\\
208.01	0\\
209.01	1.73472347597681e-18\\
210.01	0\\
211.01	1.73472347597681e-18\\
212.01	0\\
213.01	1.73472347597681e-18\\
214.01	0\\
215.01	0\\
216.01	0\\
217.01	0\\
218.01	1.73472347597681e-18\\
219.01	1.73472347597681e-18\\
220.01	0\\
221.01	0\\
222.01	0\\
223.01	1.73472347597681e-18\\
224.01	1.73472347597681e-18\\
225.01	0\\
226.01	1.73472347597681e-18\\
227.01	0\\
228.01	0\\
229.01	1.73472347597681e-18\\
230.01	0\\
231.01	0\\
232.01	0\\
233.01	0\\
234.01	1.73472347597681e-18\\
235.01	0\\
236.01	1.73472347597681e-18\\
237.01	1.73472347597681e-18\\
238.01	0\\
239.01	1.73472347597681e-18\\
240.01	0\\
241.01	1.73472347597681e-18\\
242.01	0\\
243.01	1.73472347597681e-18\\
244.01	1.73472347597681e-18\\
245.01	1.73472347597681e-18\\
246.01	1.73472347597681e-18\\
247.01	0\\
248.01	0\\
249.01	0\\
250.01	1.73472347597681e-18\\
251.01	0\\
252.01	1.73472347597681e-18\\
253.01	1.73472347597681e-18\\
254.01	1.73472347597681e-18\\
255.01	1.73472347597681e-18\\
256.01	1.73472347597681e-18\\
257.01	1.73472347597681e-18\\
258.01	0\\
259.01	0\\
260.01	0\\
261.01	0\\
262.01	1.73472347597681e-18\\
263.01	1.73472347597681e-18\\
264.01	0\\
265.01	1.73472347597681e-18\\
266.01	1.73472347597681e-18\\
267.01	0\\
268.01	0\\
269.01	1.73472347597681e-18\\
270.01	0\\
271.01	1.73472347597681e-18\\
272.01	0\\
273.01	1.73472347597681e-18\\
274.01	0\\
275.01	0\\
276.01	1.73472347597681e-18\\
277.01	1.73472347597681e-18\\
278.01	0\\
279.01	1.73472347597681e-18\\
280.01	0\\
281.01	1.73472347597681e-18\\
282.01	1.73472347597681e-18\\
283.01	1.73472347597681e-18\\
284.01	0\\
285.01	1.73472347597681e-18\\
286.01	1.73472347597681e-18\\
287.01	1.73472347597681e-18\\
288.01	1.73472347597681e-18\\
289.01	0\\
290.01	0\\
291.01	0\\
292.01	0\\
293.01	1.73472347597681e-18\\
294.01	1.73472347597681e-18\\
295.01	0\\
296.01	1.73472347597681e-18\\
297.01	0\\
298.01	1.73472347597681e-18\\
299.01	0\\
300.01	1.73472347597681e-18\\
301.01	0\\
302.01	0\\
303.01	0\\
304.01	1.73472347597681e-18\\
305.01	1.73472347597681e-18\\
306.01	0\\
307.01	1.73472347597681e-18\\
308.01	0\\
309.01	1.73472347597681e-18\\
310.01	1.73472347597681e-18\\
311.01	0\\
312.01	1.73472347597681e-18\\
313.01	1.73472347597681e-18\\
314.01	0\\
315.01	1.73472347597681e-18\\
316.01	0\\
317.01	1.73472347597681e-18\\
318.01	0\\
319.01	0\\
320.01	0\\
321.01	0\\
322.01	1.73472347597681e-18\\
323.01	1.73472347597681e-18\\
324.01	0\\
325.01	1.73472347597681e-18\\
326.01	1.73472347597681e-18\\
327.01	0\\
328.01	1.73472347597681e-18\\
329.01	0\\
330.01	0\\
331.01	1.73472347597681e-18\\
332.01	1.73472347597681e-18\\
333.01	1.73472347597681e-18\\
334.01	1.73472347597681e-18\\
335.01	1.73472347597681e-18\\
336.01	0\\
337.01	1.73472347597681e-18\\
338.01	0\\
339.01	0\\
340.01	1.73472347597681e-18\\
341.01	1.73472347597681e-18\\
342.01	1.73472347597681e-18\\
343.01	1.73472347597681e-18\\
344.01	0\\
345.01	0\\
346.01	0\\
347.01	1.73472347597681e-18\\
348.01	0\\
349.01	1.73472347597681e-18\\
350.01	0\\
351.01	0\\
352.01	0\\
353.01	1.73472347597681e-18\\
354.01	1.73472347597681e-18\\
355.01	0\\
356.01	0\\
357.01	1.73472347597681e-18\\
358.01	1.73472347597681e-18\\
359.01	1.73472347597681e-18\\
360.01	0\\
361.01	0\\
362.01	0\\
363.01	0\\
364.01	0\\
365.01	1.73472347597681e-18\\
366.01	1.73472347597681e-18\\
367.01	1.73472347597681e-18\\
368.01	1.73472347597681e-18\\
369.01	0\\
370.01	0\\
371.01	1.73472347597681e-18\\
372.01	1.73472347597681e-18\\
373.01	1.73472347597681e-18\\
374.01	0\\
375.01	1.73472347597681e-18\\
376.01	0\\
377.01	0\\
378.01	1.73472347597681e-18\\
379.01	0\\
380.01	1.73472347597681e-18\\
381.01	1.73472347597681e-18\\
382.01	1.73472347597681e-18\\
383.01	0\\
384.01	1.73472347597681e-18\\
385.01	0\\
386.01	1.73472347597681e-18\\
387.01	1.73472347597681e-18\\
388.01	1.73472347597681e-18\\
389.01	0\\
390.01	1.73472347597681e-18\\
391.01	1.73472347597681e-18\\
392.01	0\\
393.01	0\\
394.01	1.73472347597681e-18\\
395.01	1.73472347597681e-18\\
396.01	1.73472347597681e-18\\
397.01	1.73472347597681e-18\\
398.01	0\\
399.01	1.73472347597681e-18\\
400.01	0\\
401.01	0\\
402.01	0\\
403.01	1.73472347597681e-18\\
404.01	1.73472347597681e-18\\
405.01	1.73472347597681e-18\\
406.01	1.73472347597681e-18\\
407.01	1.73472347597681e-18\\
408.01	1.73472347597681e-18\\
409.01	0\\
410.01	1.73472347597681e-18\\
411.01	0\\
412.01	1.73472347597681e-18\\
413.01	1.73472347597681e-18\\
414.01	0\\
415.01	0\\
416.01	1.73472347597681e-18\\
417.01	1.73472347597681e-18\\
418.01	1.73472347597681e-18\\
419.01	1.73472347597681e-18\\
420.01	0\\
421.01	1.73472347597681e-18\\
422.01	1.73472347597681e-18\\
423.01	1.73472347597681e-18\\
424.01	1.73472347597681e-18\\
425.01	0\\
426.01	1.73472347597681e-18\\
427.01	1.73472347597681e-18\\
428.01	1.73472347597681e-18\\
429.01	0\\
430.01	1.73472347597681e-18\\
431.01	1.73472347597681e-18\\
432.01	1.73472347597681e-18\\
433.01	0\\
434.01	1.73472347597681e-18\\
435.01	1.73472347597681e-18\\
436.01	1.73472347597681e-18\\
437.01	1.73472347597681e-18\\
438.01	1.73472347597681e-18\\
439.01	0\\
440.01	1.73472347597681e-18\\
441.01	1.73472347597681e-18\\
442.01	0\\
443.01	0\\
444.01	0\\
445.01	1.73472347597681e-18\\
446.01	1.73472347597681e-18\\
447.01	0\\
448.01	1.73472347597681e-18\\
449.01	1.73472347597681e-18\\
450.01	0\\
451.01	0\\
452.01	0\\
453.01	1.73472347597681e-18\\
454.01	1.73472347597681e-18\\
455.01	0\\
456.01	1.73472347597681e-18\\
457.01	1.73472347597681e-18\\
458.01	0\\
459.01	1.73472347597681e-18\\
460.01	0\\
461.01	1.73472347597681e-18\\
462.01	0\\
463.01	0\\
464.01	0\\
465.01	1.73472347597681e-18\\
466.01	0\\
467.01	1.73472347597681e-18\\
468.01	0\\
469.01	1.73472347597681e-18\\
470.01	1.73472347597681e-18\\
471.01	1.73472347597681e-18\\
472.01	0\\
473.01	0\\
474.01	0\\
475.01	1.73472347597681e-18\\
476.01	0\\
477.01	0\\
478.01	1.73472347597681e-18\\
479.01	1.73472347597681e-18\\
480.01	0\\
481.01	0\\
482.01	1.73472347597681e-18\\
483.01	1.73472347597681e-18\\
484.01	0\\
485.01	0\\
486.01	0\\
487.01	1.73472347597681e-18\\
488.01	1.73472347597681e-18\\
489.01	1.73472347597681e-18\\
490.01	1.73472347597681e-18\\
491.01	0\\
492.01	1.73472347597681e-18\\
493.01	1.73472347597681e-18\\
494.01	0\\
495.01	1.73472347597681e-18\\
496.01	0\\
497.01	1.73472347597681e-18\\
498.01	1.73472347597681e-18\\
499.01	0\\
500.01	1.73472347597681e-18\\
501.01	1.73472347597681e-18\\
502.01	0\\
503.01	1.73472347597681e-18\\
504.01	1.73472347597681e-18\\
505.01	1.73472347597681e-18\\
506.01	0\\
507.01	0\\
508.01	0\\
509.01	0\\
510.01	1.73472347597681e-18\\
511.01	1.73472347597681e-18\\
512.01	0\\
513.01	1.73472347597681e-18\\
514.01	1.73472347597681e-18\\
515.01	0\\
516.01	0\\
517.01	1.73472347597681e-18\\
518.01	1.73472347597681e-18\\
519.01	1.73472347597681e-18\\
520.01	0\\
521.01	0\\
522.01	0\\
523.01	1.73472347597681e-18\\
524.01	0\\
525.01	1.73472347597681e-18\\
526.01	1.73472347597681e-18\\
527.01	0\\
528.01	0\\
529.01	0\\
530.01	1.73472347597681e-18\\
531.01	0\\
532.01	1.73472347597681e-18\\
533.01	0\\
534.01	1.73472347597681e-18\\
535.01	0\\
536.01	1.73472347597681e-18\\
537.01	1.73472347597681e-18\\
538.01	1.73472347597681e-18\\
539.01	0\\
540.01	0\\
541.01	1.73472347597681e-18\\
542.01	1.73472347597681e-18\\
543.01	0\\
544.01	1.73472347597681e-18\\
545.01	0\\
546.01	1.73472347597681e-18\\
547.01	0\\
548.01	0\\
549.01	0\\
550.01	0\\
551.01	0\\
552.01	0\\
553.01	0\\
554.01	0\\
555.01	1.73472347597681e-18\\
556.01	0\\
557.01	0\\
558.01	0\\
559.01	1.73472347597681e-18\\
560.01	1.73472347597681e-18\\
561.01	1.73472347597681e-18\\
562.01	1.73472347597681e-18\\
563.01	0\\
564.01	0\\
565.01	1.73472347597681e-18\\
566.01	0\\
567.01	0\\
568.01	0\\
569.01	0\\
570.01	0\\
571.01	0\\
572.01	1.73472347597681e-18\\
573.01	0\\
574.01	1.73472347597681e-18\\
575.01	0\\
576.01	1.73472347597681e-18\\
577.01	0\\
578.01	1.73472347597681e-18\\
579.01	0\\
580.01	0\\
581.01	1.73472347597681e-18\\
582.01	0\\
583.01	0\\
584.01	0\\
585.01	0\\
586.01	0\\
587.01	0\\
588.01	0\\
589.01	0\\
590.01	0\\
591.01	0\\
592.01	1.73472347597681e-18\\
593.01	0\\
594.01	0\\
595.01	0\\
596.01	0\\
597.01	0\\
598.01	0\\
599.01	0\\
599.02	0\\
599.03	1.73472347597681e-18\\
599.04	1.73472347597681e-18\\
599.05	0\\
599.06	0\\
599.07	1.73472347597681e-18\\
599.08	0\\
599.09	0\\
599.1	0\\
599.11	1.73472347597681e-18\\
599.12	1.73472347597681e-18\\
599.13	0\\
599.14	0\\
599.15	1.73472347597681e-18\\
599.16	1.73472347597681e-18\\
599.17	1.73472347597681e-18\\
599.18	0\\
599.19	0\\
599.2	0\\
599.21	0\\
599.22	0\\
599.23	1.73472347597681e-18\\
599.24	1.73472347597681e-18\\
599.25	1.73472347597681e-18\\
599.26	1.73472347597681e-18\\
599.27	0\\
599.28	0\\
599.29	0\\
599.3	0\\
599.31	0\\
599.32	1.73472347597681e-18\\
599.33	1.73472347597681e-18\\
599.34	0\\
599.35	0\\
599.36	0\\
599.37	1.73472347597681e-18\\
599.38	0\\
599.39	0\\
599.4	0\\
599.41	0\\
599.42	0\\
599.43	0\\
599.44	0\\
599.45	0\\
599.46	1.73472347597681e-18\\
599.47	1.73472347597681e-18\\
599.48	1.73472347597681e-18\\
599.49	1.73472347597681e-18\\
599.5	1.73472347597681e-18\\
599.51	0\\
599.52	0\\
599.53	1.73472347597681e-18\\
599.54	0\\
599.55	0\\
599.56	0\\
599.57	1.73472347597681e-18\\
599.58	0\\
599.59	0\\
599.6	0\\
599.61	0\\
599.62	1.73472347597681e-18\\
599.63	0\\
599.64	1.73472347597681e-18\\
599.65	0\\
599.66	0\\
599.67	0\\
599.68	0\\
599.69	1.73472347597681e-18\\
599.7	0\\
599.71	1.73472347597681e-18\\
599.72	1.73472347597681e-18\\
599.73	1.73472347597681e-18\\
599.74	0\\
599.75	0\\
599.76	0\\
599.77	0\\
599.78	0\\
599.79	0\\
599.8	0\\
599.81	0\\
599.82	0\\
599.83	1.73472347597681e-18\\
599.84	0\\
599.85	0\\
599.86	0\\
599.87	0\\
599.88	0\\
599.89	0\\
599.9	0\\
599.91	0\\
599.92	0\\
599.93	0\\
599.94	0\\
599.95	0\\
599.96	0\\
599.97	0\\
599.98	0\\
599.99	0\\
600	0\\
};
\addplot [color=mycolor3,solid,forget plot]
  table[row sep=crcr]{%
0.01	1.73472347597681e-18\\
1.01	1.73472347597681e-18\\
2.01	0\\
3.01	1.73472347597681e-18\\
4.01	1.73472347597681e-18\\
5.01	0\\
6.01	0\\
7.01	1.73472347597681e-18\\
8.01	0\\
9.01	0\\
10.01	0\\
11.01	0\\
12.01	1.73472347597681e-18\\
13.01	1.73472347597681e-18\\
14.01	1.73472347597681e-18\\
15.01	1.73472347597681e-18\\
16.01	1.73472347597681e-18\\
17.01	1.73472347597681e-18\\
18.01	0\\
19.01	0\\
20.01	1.73472347597681e-18\\
21.01	0\\
22.01	1.73472347597681e-18\\
23.01	1.73472347597681e-18\\
24.01	1.73472347597681e-18\\
25.01	0\\
26.01	1.73472347597681e-18\\
27.01	0\\
28.01	1.73472347597681e-18\\
29.01	1.73472347597681e-18\\
30.01	0\\
31.01	0\\
32.01	1.73472347597681e-18\\
33.01	1.73472347597681e-18\\
34.01	1.73472347597681e-18\\
35.01	0\\
36.01	0\\
37.01	0\\
38.01	0\\
39.01	1.73472347597681e-18\\
40.01	1.73472347597681e-18\\
41.01	1.73472347597681e-18\\
42.01	1.73472347597681e-18\\
43.01	1.73472347597681e-18\\
44.01	0\\
45.01	1.73472347597681e-18\\
46.01	1.73472347597681e-18\\
47.01	1.73472347597681e-18\\
48.01	1.73472347597681e-18\\
49.01	0\\
50.01	1.73472347597681e-18\\
51.01	1.73472347597681e-18\\
52.01	0\\
53.01	0\\
54.01	0\\
55.01	1.73472347597681e-18\\
56.01	0\\
57.01	1.73472347597681e-18\\
58.01	0\\
59.01	0\\
60.01	1.73472347597681e-18\\
61.01	1.73472347597681e-18\\
62.01	1.73472347597681e-18\\
63.01	0\\
64.01	1.73472347597681e-18\\
65.01	0\\
66.01	1.73472347597681e-18\\
67.01	1.73472347597681e-18\\
68.01	0\\
69.01	1.73472347597681e-18\\
70.01	1.73472347597681e-18\\
71.01	0\\
72.01	0\\
73.01	1.73472347597681e-18\\
74.01	0\\
75.01	1.73472347597681e-18\\
76.01	0\\
77.01	0\\
78.01	0\\
79.01	0\\
80.01	1.73472347597681e-18\\
81.01	1.73472347597681e-18\\
82.01	0\\
83.01	0\\
84.01	0\\
85.01	1.73472347597681e-18\\
86.01	1.73472347597681e-18\\
87.01	0\\
88.01	0\\
89.01	0\\
90.01	1.73472347597681e-18\\
91.01	1.73472347597681e-18\\
92.01	0\\
93.01	1.73472347597681e-18\\
94.01	1.73472347597681e-18\\
95.01	0\\
96.01	1.73472347597681e-18\\
97.01	0\\
98.01	0\\
99.01	0\\
100.01	1.73472347597681e-18\\
101.01	0\\
102.01	1.73472347597681e-18\\
103.01	1.73472347597681e-18\\
104.01	0\\
105.01	0\\
106.01	0\\
107.01	0\\
108.01	1.73472347597681e-18\\
109.01	0\\
110.01	1.73472347597681e-18\\
111.01	0\\
112.01	0\\
113.01	0\\
114.01	1.73472347597681e-18\\
115.01	1.73472347597681e-18\\
116.01	0\\
117.01	0\\
118.01	0\\
119.01	0\\
120.01	0\\
121.01	0\\
122.01	0\\
123.01	1.73472347597681e-18\\
124.01	1.73472347597681e-18\\
125.01	1.73472347597681e-18\\
126.01	1.73472347597681e-18\\
127.01	0\\
128.01	1.73472347597681e-18\\
129.01	0\\
130.01	1.73472347597681e-18\\
131.01	1.73472347597681e-18\\
132.01	0\\
133.01	1.73472347597681e-18\\
134.01	1.73472347597681e-18\\
135.01	0\\
136.01	0\\
137.01	1.73472347597681e-18\\
138.01	0\\
139.01	1.73472347597681e-18\\
140.01	1.73472347597681e-18\\
141.01	1.73472347597681e-18\\
142.01	1.73472347597681e-18\\
143.01	0\\
144.01	1.73472347597681e-18\\
145.01	1.73472347597681e-18\\
146.01	1.73472347597681e-18\\
147.01	1.73472347597681e-18\\
148.01	1.73472347597681e-18\\
149.01	0\\
150.01	0\\
151.01	1.73472347597681e-18\\
152.01	1.73472347597681e-18\\
153.01	1.73472347597681e-18\\
154.01	1.73472347597681e-18\\
155.01	1.73472347597681e-18\\
156.01	1.73472347597681e-18\\
157.01	0\\
158.01	1.73472347597681e-18\\
159.01	1.73472347597681e-18\\
160.01	1.73472347597681e-18\\
161.01	1.73472347597681e-18\\
162.01	1.73472347597681e-18\\
163.01	1.73472347597681e-18\\
164.01	0\\
165.01	0\\
166.01	1.73472347597681e-18\\
167.01	0\\
168.01	1.73472347597681e-18\\
169.01	0\\
170.01	1.73472347597681e-18\\
171.01	0\\
172.01	0\\
173.01	0\\
174.01	1.73472347597681e-18\\
175.01	1.73472347597681e-18\\
176.01	1.73472347597681e-18\\
177.01	0\\
178.01	1.73472347597681e-18\\
179.01	0\\
180.01	0\\
181.01	1.73472347597681e-18\\
182.01	0\\
183.01	0\\
184.01	0\\
185.01	1.73472347597681e-18\\
186.01	0\\
187.01	0\\
188.01	0\\
189.01	0\\
190.01	1.73472347597681e-18\\
191.01	0\\
192.01	1.73472347597681e-18\\
193.01	0\\
194.01	1.73472347597681e-18\\
195.01	0\\
196.01	1.73472347597681e-18\\
197.01	0\\
198.01	0\\
199.01	0\\
200.01	0\\
201.01	0\\
202.01	1.73472347597681e-18\\
203.01	0\\
204.01	1.73472347597681e-18\\
205.01	1.73472347597681e-18\\
206.01	1.73472347597681e-18\\
207.01	1.73472347597681e-18\\
208.01	0\\
209.01	1.73472347597681e-18\\
210.01	0\\
211.01	1.73472347597681e-18\\
212.01	0\\
213.01	1.73472347597681e-18\\
214.01	0\\
215.01	0\\
216.01	0\\
217.01	0\\
218.01	1.73472347597681e-18\\
219.01	1.73472347597681e-18\\
220.01	0\\
221.01	0\\
222.01	0\\
223.01	1.73472347597681e-18\\
224.01	1.73472347597681e-18\\
225.01	0\\
226.01	1.73472347597681e-18\\
227.01	0\\
228.01	0\\
229.01	1.73472347597681e-18\\
230.01	0\\
231.01	0\\
232.01	0\\
233.01	0\\
234.01	1.73472347597681e-18\\
235.01	0\\
236.01	1.73472347597681e-18\\
237.01	1.73472347597681e-18\\
238.01	0\\
239.01	1.73472347597681e-18\\
240.01	0\\
241.01	1.73472347597681e-18\\
242.01	0\\
243.01	1.73472347597681e-18\\
244.01	1.73472347597681e-18\\
245.01	1.73472347597681e-18\\
246.01	1.73472347597681e-18\\
247.01	0\\
248.01	0\\
249.01	0\\
250.01	1.73472347597681e-18\\
251.01	0\\
252.01	1.73472347597681e-18\\
253.01	1.73472347597681e-18\\
254.01	1.73472347597681e-18\\
255.01	1.73472347597681e-18\\
256.01	1.73472347597681e-18\\
257.01	1.73472347597681e-18\\
258.01	0\\
259.01	0\\
260.01	0\\
261.01	0\\
262.01	1.73472347597681e-18\\
263.01	1.73472347597681e-18\\
264.01	0\\
265.01	1.73472347597681e-18\\
266.01	1.73472347597681e-18\\
267.01	0\\
268.01	0\\
269.01	1.73472347597681e-18\\
270.01	0\\
271.01	1.73472347597681e-18\\
272.01	0\\
273.01	1.73472347597681e-18\\
274.01	0\\
275.01	0\\
276.01	1.73472347597681e-18\\
277.01	1.73472347597681e-18\\
278.01	0\\
279.01	1.73472347597681e-18\\
280.01	0\\
281.01	1.73472347597681e-18\\
282.01	1.73472347597681e-18\\
283.01	1.73472347597681e-18\\
284.01	0\\
285.01	1.73472347597681e-18\\
286.01	1.73472347597681e-18\\
287.01	1.73472347597681e-18\\
288.01	1.73472347597681e-18\\
289.01	0\\
290.01	0\\
291.01	0\\
292.01	0\\
293.01	1.73472347597681e-18\\
294.01	1.73472347597681e-18\\
295.01	0\\
296.01	1.73472347597681e-18\\
297.01	0\\
298.01	1.73472347597681e-18\\
299.01	0\\
300.01	1.73472347597681e-18\\
301.01	0\\
302.01	0\\
303.01	0\\
304.01	1.73472347597681e-18\\
305.01	1.73472347597681e-18\\
306.01	0\\
307.01	1.73472347597681e-18\\
308.01	0\\
309.01	1.73472347597681e-18\\
310.01	1.73472347597681e-18\\
311.01	0\\
312.01	1.73472347597681e-18\\
313.01	1.73472347597681e-18\\
314.01	0\\
315.01	1.73472347597681e-18\\
316.01	0\\
317.01	1.73472347597681e-18\\
318.01	0\\
319.01	0\\
320.01	0\\
321.01	0\\
322.01	1.73472347597681e-18\\
323.01	1.73472347597681e-18\\
324.01	0\\
325.01	1.73472347597681e-18\\
326.01	1.73472347597681e-18\\
327.01	0\\
328.01	1.73472347597681e-18\\
329.01	0\\
330.01	0\\
331.01	1.73472347597681e-18\\
332.01	1.73472347597681e-18\\
333.01	1.73472347597681e-18\\
334.01	1.73472347597681e-18\\
335.01	1.73472347597681e-18\\
336.01	0\\
337.01	1.73472347597681e-18\\
338.01	0\\
339.01	0\\
340.01	1.73472347597681e-18\\
341.01	1.73472347597681e-18\\
342.01	1.73472347597681e-18\\
343.01	1.73472347597681e-18\\
344.01	0\\
345.01	0\\
346.01	0\\
347.01	1.73472347597681e-18\\
348.01	0\\
349.01	1.73472347597681e-18\\
350.01	0\\
351.01	0\\
352.01	0\\
353.01	1.73472347597681e-18\\
354.01	1.73472347597681e-18\\
355.01	0\\
356.01	0\\
357.01	1.73472347597681e-18\\
358.01	1.73472347597681e-18\\
359.01	1.73472347597681e-18\\
360.01	0\\
361.01	0\\
362.01	0\\
363.01	0\\
364.01	0\\
365.01	1.73472347597681e-18\\
366.01	1.73472347597681e-18\\
367.01	1.73472347597681e-18\\
368.01	1.73472347597681e-18\\
369.01	0\\
370.01	0\\
371.01	1.73472347597681e-18\\
372.01	1.73472347597681e-18\\
373.01	1.73472347597681e-18\\
374.01	0\\
375.01	1.73472347597681e-18\\
376.01	0\\
377.01	0\\
378.01	1.73472347597681e-18\\
379.01	0\\
380.01	1.73472347597681e-18\\
381.01	1.73472347597681e-18\\
382.01	1.73472347597681e-18\\
383.01	0\\
384.01	1.73472347597681e-18\\
385.01	0\\
386.01	1.73472347597681e-18\\
387.01	1.73472347597681e-18\\
388.01	1.73472347597681e-18\\
389.01	0\\
390.01	1.73472347597681e-18\\
391.01	1.73472347597681e-18\\
392.01	0\\
393.01	0\\
394.01	1.73472347597681e-18\\
395.01	1.73472347597681e-18\\
396.01	1.73472347597681e-18\\
397.01	1.73472347597681e-18\\
398.01	0\\
399.01	1.73472347597681e-18\\
400.01	0\\
401.01	0\\
402.01	0\\
403.01	1.73472347597681e-18\\
404.01	1.73472347597681e-18\\
405.01	1.73472347597681e-18\\
406.01	1.73472347597681e-18\\
407.01	1.73472347597681e-18\\
408.01	1.73472347597681e-18\\
409.01	0\\
410.01	1.73472347597681e-18\\
411.01	0\\
412.01	1.73472347597681e-18\\
413.01	1.73472347597681e-18\\
414.01	0\\
415.01	0\\
416.01	1.73472347597681e-18\\
417.01	1.73472347597681e-18\\
418.01	1.73472347597681e-18\\
419.01	1.73472347597681e-18\\
420.01	0\\
421.01	1.73472347597681e-18\\
422.01	1.73472347597681e-18\\
423.01	1.73472347597681e-18\\
424.01	1.73472347597681e-18\\
425.01	0\\
426.01	1.73472347597681e-18\\
427.01	1.73472347597681e-18\\
428.01	1.73472347597681e-18\\
429.01	0\\
430.01	1.73472347597681e-18\\
431.01	1.73472347597681e-18\\
432.01	1.73472347597681e-18\\
433.01	0\\
434.01	1.73472347597681e-18\\
435.01	1.73472347597681e-18\\
436.01	1.73472347597681e-18\\
437.01	1.73472347597681e-18\\
438.01	1.73472347597681e-18\\
439.01	0\\
440.01	1.73472347597681e-18\\
441.01	1.73472347597681e-18\\
442.01	0\\
443.01	0\\
444.01	0\\
445.01	1.73472347597681e-18\\
446.01	1.73472347597681e-18\\
447.01	0\\
448.01	1.73472347597681e-18\\
449.01	1.73472347597681e-18\\
450.01	0\\
451.01	0\\
452.01	0\\
453.01	1.73472347597681e-18\\
454.01	1.73472347597681e-18\\
455.01	0\\
456.01	1.73472347597681e-18\\
457.01	1.73472347597681e-18\\
458.01	0\\
459.01	1.73472347597681e-18\\
460.01	0\\
461.01	1.73472347597681e-18\\
462.01	0\\
463.01	0\\
464.01	0\\
465.01	1.73472347597681e-18\\
466.01	0\\
467.01	1.73472347597681e-18\\
468.01	0\\
469.01	1.73472347597681e-18\\
470.01	1.73472347597681e-18\\
471.01	1.73472347597681e-18\\
472.01	0\\
473.01	0\\
474.01	0\\
475.01	1.73472347597681e-18\\
476.01	0\\
477.01	0\\
478.01	1.73472347597681e-18\\
479.01	1.73472347597681e-18\\
480.01	0\\
481.01	0\\
482.01	1.73472347597681e-18\\
483.01	1.73472347597681e-18\\
484.01	0\\
485.01	0\\
486.01	0\\
487.01	1.73472347597681e-18\\
488.01	1.73472347597681e-18\\
489.01	1.73472347597681e-18\\
490.01	1.73472347597681e-18\\
491.01	0\\
492.01	1.73472347597681e-18\\
493.01	1.73472347597681e-18\\
494.01	0\\
495.01	1.73472347597681e-18\\
496.01	0\\
497.01	1.73472347597681e-18\\
498.01	1.73472347597681e-18\\
499.01	0\\
500.01	1.73472347597681e-18\\
501.01	1.73472347597681e-18\\
502.01	0\\
503.01	1.73472347597681e-18\\
504.01	1.73472347597681e-18\\
505.01	1.73472347597681e-18\\
506.01	0\\
507.01	0\\
508.01	0\\
509.01	0\\
510.01	1.73472347597681e-18\\
511.01	1.73472347597681e-18\\
512.01	0\\
513.01	1.73472347597681e-18\\
514.01	1.73472347597681e-18\\
515.01	0\\
516.01	0\\
517.01	1.73472347597681e-18\\
518.01	1.73472347597681e-18\\
519.01	1.73472347597681e-18\\
520.01	0\\
521.01	0\\
522.01	0\\
523.01	1.73472347597681e-18\\
524.01	0\\
525.01	1.73472347597681e-18\\
526.01	1.73472347597681e-18\\
527.01	0\\
528.01	0\\
529.01	0\\
530.01	1.73472347597681e-18\\
531.01	0\\
532.01	1.73472347597681e-18\\
533.01	0\\
534.01	1.73472347597681e-18\\
535.01	0\\
536.01	1.73472347597681e-18\\
537.01	1.73472347597681e-18\\
538.01	1.73472347597681e-18\\
539.01	0\\
540.01	0\\
541.01	1.73472347597681e-18\\
542.01	1.73472347597681e-18\\
543.01	0\\
544.01	1.73472347597681e-18\\
545.01	0\\
546.01	1.73472347597681e-18\\
547.01	0\\
548.01	0\\
549.01	0\\
550.01	0\\
551.01	0\\
552.01	0\\
553.01	0\\
554.01	0\\
555.01	1.73472347597681e-18\\
556.01	0\\
557.01	0\\
558.01	0\\
559.01	1.73472347597681e-18\\
560.01	1.73472347597681e-18\\
561.01	1.73472347597681e-18\\
562.01	1.73472347597681e-18\\
563.01	0\\
564.01	0\\
565.01	1.73472347597681e-18\\
566.01	0\\
567.01	0\\
568.01	0\\
569.01	0\\
570.01	0\\
571.01	0\\
572.01	1.73472347597681e-18\\
573.01	0\\
574.01	1.73472347597681e-18\\
575.01	0\\
576.01	1.73472347597681e-18\\
577.01	0\\
578.01	1.73472347597681e-18\\
579.01	0\\
580.01	0\\
581.01	1.73472347597681e-18\\
582.01	0\\
583.01	0\\
584.01	0\\
585.01	0\\
586.01	0\\
587.01	0\\
588.01	0\\
589.01	0\\
590.01	0\\
591.01	0\\
592.01	1.73472347597681e-18\\
593.01	0\\
594.01	0\\
595.01	0\\
596.01	0\\
597.01	0\\
598.01	0\\
599.01	0\\
599.02	0\\
599.03	1.73472347597681e-18\\
599.04	1.73472347597681e-18\\
599.05	0\\
599.06	0\\
599.07	1.73472347597681e-18\\
599.08	0\\
599.09	0\\
599.1	0\\
599.11	1.73472347597681e-18\\
599.12	1.73472347597681e-18\\
599.13	0\\
599.14	0\\
599.15	1.73472347597681e-18\\
599.16	1.73472347597681e-18\\
599.17	1.73472347597681e-18\\
599.18	0\\
599.19	0\\
599.2	0\\
599.21	0\\
599.22	0\\
599.23	1.73472347597681e-18\\
599.24	1.73472347597681e-18\\
599.25	1.73472347597681e-18\\
599.26	1.73472347597681e-18\\
599.27	0\\
599.28	0\\
599.29	0\\
599.3	0\\
599.31	0\\
599.32	1.73472347597681e-18\\
599.33	1.73472347597681e-18\\
599.34	0\\
599.35	0\\
599.36	0\\
599.37	1.73472347597681e-18\\
599.38	0\\
599.39	0\\
599.4	0\\
599.41	0\\
599.42	0\\
599.43	0\\
599.44	0\\
599.45	0\\
599.46	1.73472347597681e-18\\
599.47	1.73472347597681e-18\\
599.48	1.73472347597681e-18\\
599.49	1.73472347597681e-18\\
599.5	1.73472347597681e-18\\
599.51	0\\
599.52	0\\
599.53	1.73472347597681e-18\\
599.54	0\\
599.55	0\\
599.56	0\\
599.57	1.73472347597681e-18\\
599.58	0\\
599.59	0\\
599.6	0\\
599.61	0\\
599.62	1.73472347597681e-18\\
599.63	0\\
599.64	1.73472347597681e-18\\
599.65	0\\
599.66	0\\
599.67	0\\
599.68	0\\
599.69	1.73472347597681e-18\\
599.7	0\\
599.71	1.73472347597681e-18\\
599.72	1.73472347597681e-18\\
599.73	1.73472347597681e-18\\
599.74	0\\
599.75	0\\
599.76	0\\
599.77	0\\
599.78	0\\
599.79	0\\
599.8	0\\
599.81	0\\
599.82	0\\
599.83	1.73472347597681e-18\\
599.84	0\\
599.85	0\\
599.86	0\\
599.87	0\\
599.88	0\\
599.89	0\\
599.9	0\\
599.91	0\\
599.92	0\\
599.93	0\\
599.94	0\\
599.95	0\\
599.96	0\\
599.97	0\\
599.98	0\\
599.99	0\\
600	0\\
};
\addplot [color=mycolor4,solid,forget plot]
  table[row sep=crcr]{%
0.01	1.73472347597681e-18\\
1.01	1.73472347597681e-18\\
2.01	0\\
3.01	1.73472347597681e-18\\
4.01	1.73472347597681e-18\\
5.01	0\\
6.01	0\\
7.01	1.73472347597681e-18\\
8.01	0\\
9.01	0\\
10.01	0\\
11.01	0\\
12.01	1.73472347597681e-18\\
13.01	1.73472347597681e-18\\
14.01	1.73472347597681e-18\\
15.01	1.73472347597681e-18\\
16.01	1.73472347597681e-18\\
17.01	1.73472347597681e-18\\
18.01	0\\
19.01	0\\
20.01	1.73472347597681e-18\\
21.01	0\\
22.01	1.73472347597681e-18\\
23.01	1.73472347597681e-18\\
24.01	1.73472347597681e-18\\
25.01	0\\
26.01	1.73472347597681e-18\\
27.01	0\\
28.01	1.73472347597681e-18\\
29.01	1.73472347597681e-18\\
30.01	0\\
31.01	0\\
32.01	1.73472347597681e-18\\
33.01	1.73472347597681e-18\\
34.01	1.73472347597681e-18\\
35.01	0\\
36.01	0\\
37.01	0\\
38.01	0\\
39.01	1.73472347597681e-18\\
40.01	1.73472347597681e-18\\
41.01	1.73472347597681e-18\\
42.01	1.73472347597681e-18\\
43.01	1.73472347597681e-18\\
44.01	0\\
45.01	1.73472347597681e-18\\
46.01	1.73472347597681e-18\\
47.01	1.73472347597681e-18\\
48.01	1.73472347597681e-18\\
49.01	0\\
50.01	1.73472347597681e-18\\
51.01	1.73472347597681e-18\\
52.01	0\\
53.01	0\\
54.01	0\\
55.01	1.73472347597681e-18\\
56.01	0\\
57.01	1.73472347597681e-18\\
58.01	0\\
59.01	0\\
60.01	1.73472347597681e-18\\
61.01	1.73472347597681e-18\\
62.01	1.73472347597681e-18\\
63.01	0\\
64.01	1.73472347597681e-18\\
65.01	0\\
66.01	1.73472347597681e-18\\
67.01	1.73472347597681e-18\\
68.01	0\\
69.01	1.73472347597681e-18\\
70.01	1.73472347597681e-18\\
71.01	0\\
72.01	0\\
73.01	1.73472347597681e-18\\
74.01	0\\
75.01	1.73472347597681e-18\\
76.01	0\\
77.01	0\\
78.01	0\\
79.01	0\\
80.01	1.73472347597681e-18\\
81.01	1.73472347597681e-18\\
82.01	0\\
83.01	0\\
84.01	0\\
85.01	1.73472347597681e-18\\
86.01	1.73472347597681e-18\\
87.01	0\\
88.01	0\\
89.01	0\\
90.01	1.73472347597681e-18\\
91.01	1.73472347597681e-18\\
92.01	0\\
93.01	1.73472347597681e-18\\
94.01	1.73472347597681e-18\\
95.01	0\\
96.01	1.73472347597681e-18\\
97.01	0\\
98.01	0\\
99.01	0\\
100.01	1.73472347597681e-18\\
101.01	0\\
102.01	1.73472347597681e-18\\
103.01	1.73472347597681e-18\\
104.01	0\\
105.01	0\\
106.01	0\\
107.01	0\\
108.01	1.73472347597681e-18\\
109.01	0\\
110.01	1.73472347597681e-18\\
111.01	0\\
112.01	0\\
113.01	0\\
114.01	1.73472347597681e-18\\
115.01	1.73472347597681e-18\\
116.01	0\\
117.01	0\\
118.01	0\\
119.01	0\\
120.01	0\\
121.01	0\\
122.01	0\\
123.01	1.73472347597681e-18\\
124.01	1.73472347597681e-18\\
125.01	1.73472347597681e-18\\
126.01	1.73472347597681e-18\\
127.01	0\\
128.01	1.73472347597681e-18\\
129.01	0\\
130.01	1.73472347597681e-18\\
131.01	1.73472347597681e-18\\
132.01	0\\
133.01	1.73472347597681e-18\\
134.01	1.73472347597681e-18\\
135.01	0\\
136.01	0\\
137.01	1.73472347597681e-18\\
138.01	0\\
139.01	1.73472347597681e-18\\
140.01	1.73472347597681e-18\\
141.01	1.73472347597681e-18\\
142.01	1.73472347597681e-18\\
143.01	0\\
144.01	1.73472347597681e-18\\
145.01	1.73472347597681e-18\\
146.01	1.73472347597681e-18\\
147.01	1.73472347597681e-18\\
148.01	1.73472347597681e-18\\
149.01	0\\
150.01	0\\
151.01	1.73472347597681e-18\\
152.01	1.73472347597681e-18\\
153.01	1.73472347597681e-18\\
154.01	1.73472347597681e-18\\
155.01	1.73472347597681e-18\\
156.01	1.73472347597681e-18\\
157.01	0\\
158.01	1.73472347597681e-18\\
159.01	1.73472347597681e-18\\
160.01	1.73472347597681e-18\\
161.01	1.73472347597681e-18\\
162.01	1.73472347597681e-18\\
163.01	1.73472347597681e-18\\
164.01	0\\
165.01	0\\
166.01	1.73472347597681e-18\\
167.01	0\\
168.01	1.73472347597681e-18\\
169.01	0\\
170.01	1.73472347597681e-18\\
171.01	0\\
172.01	0\\
173.01	0\\
174.01	1.73472347597681e-18\\
175.01	1.73472347597681e-18\\
176.01	1.73472347597681e-18\\
177.01	0\\
178.01	1.73472347597681e-18\\
179.01	0\\
180.01	0\\
181.01	1.73472347597681e-18\\
182.01	0\\
183.01	0\\
184.01	0\\
185.01	1.73472347597681e-18\\
186.01	0\\
187.01	0\\
188.01	0\\
189.01	0\\
190.01	1.73472347597681e-18\\
191.01	0\\
192.01	1.73472347597681e-18\\
193.01	0\\
194.01	1.73472347597681e-18\\
195.01	0\\
196.01	1.73472347597681e-18\\
197.01	0\\
198.01	0\\
199.01	0\\
200.01	0\\
201.01	0\\
202.01	1.73472347597681e-18\\
203.01	0\\
204.01	1.73472347597681e-18\\
205.01	1.73472347597681e-18\\
206.01	1.73472347597681e-18\\
207.01	1.73472347597681e-18\\
208.01	0\\
209.01	1.73472347597681e-18\\
210.01	0\\
211.01	1.73472347597681e-18\\
212.01	0\\
213.01	1.73472347597681e-18\\
214.01	0\\
215.01	0\\
216.01	0\\
217.01	0\\
218.01	1.73472347597681e-18\\
219.01	1.73472347597681e-18\\
220.01	0\\
221.01	0\\
222.01	0\\
223.01	1.73472347597681e-18\\
224.01	1.73472347597681e-18\\
225.01	0\\
226.01	1.73472347597681e-18\\
227.01	0\\
228.01	0\\
229.01	1.73472347597681e-18\\
230.01	0\\
231.01	0\\
232.01	0\\
233.01	0\\
234.01	1.73472347597681e-18\\
235.01	0\\
236.01	1.73472347597681e-18\\
237.01	1.73472347597681e-18\\
238.01	0\\
239.01	1.73472347597681e-18\\
240.01	0\\
241.01	1.73472347597681e-18\\
242.01	0\\
243.01	1.73472347597681e-18\\
244.01	1.73472347597681e-18\\
245.01	1.73472347597681e-18\\
246.01	1.73472347597681e-18\\
247.01	0\\
248.01	0\\
249.01	0\\
250.01	1.73472347597681e-18\\
251.01	0\\
252.01	1.73472347597681e-18\\
253.01	1.73472347597681e-18\\
254.01	1.73472347597681e-18\\
255.01	1.73472347597681e-18\\
256.01	1.73472347597681e-18\\
257.01	1.73472347597681e-18\\
258.01	0\\
259.01	0\\
260.01	0\\
261.01	0\\
262.01	1.73472347597681e-18\\
263.01	1.73472347597681e-18\\
264.01	0\\
265.01	1.73472347597681e-18\\
266.01	1.73472347597681e-18\\
267.01	0\\
268.01	0\\
269.01	1.73472347597681e-18\\
270.01	0\\
271.01	1.73472347597681e-18\\
272.01	0\\
273.01	1.73472347597681e-18\\
274.01	0\\
275.01	0\\
276.01	1.73472347597681e-18\\
277.01	1.73472347597681e-18\\
278.01	0\\
279.01	1.73472347597681e-18\\
280.01	0\\
281.01	1.73472347597681e-18\\
282.01	1.73472347597681e-18\\
283.01	1.73472347597681e-18\\
284.01	0\\
285.01	1.73472347597681e-18\\
286.01	1.73472347597681e-18\\
287.01	1.73472347597681e-18\\
288.01	1.73472347597681e-18\\
289.01	0\\
290.01	0\\
291.01	0\\
292.01	0\\
293.01	1.73472347597681e-18\\
294.01	1.73472347597681e-18\\
295.01	0\\
296.01	1.73472347597681e-18\\
297.01	0\\
298.01	1.73472347597681e-18\\
299.01	0\\
300.01	1.73472347597681e-18\\
301.01	0\\
302.01	0\\
303.01	0\\
304.01	1.73472347597681e-18\\
305.01	1.73472347597681e-18\\
306.01	0\\
307.01	1.73472347597681e-18\\
308.01	0\\
309.01	1.73472347597681e-18\\
310.01	1.73472347597681e-18\\
311.01	0\\
312.01	1.73472347597681e-18\\
313.01	1.73472347597681e-18\\
314.01	0\\
315.01	1.73472347597681e-18\\
316.01	0\\
317.01	1.73472347597681e-18\\
318.01	0\\
319.01	0\\
320.01	0\\
321.01	0\\
322.01	1.73472347597681e-18\\
323.01	1.73472347597681e-18\\
324.01	0\\
325.01	1.73472347597681e-18\\
326.01	1.73472347597681e-18\\
327.01	0\\
328.01	1.73472347597681e-18\\
329.01	0\\
330.01	0\\
331.01	1.73472347597681e-18\\
332.01	1.73472347597681e-18\\
333.01	1.73472347597681e-18\\
334.01	1.73472347597681e-18\\
335.01	1.73472347597681e-18\\
336.01	0\\
337.01	1.73472347597681e-18\\
338.01	0\\
339.01	0\\
340.01	1.73472347597681e-18\\
341.01	1.73472347597681e-18\\
342.01	1.73472347597681e-18\\
343.01	1.73472347597681e-18\\
344.01	0\\
345.01	0\\
346.01	0\\
347.01	1.73472347597681e-18\\
348.01	0\\
349.01	1.73472347597681e-18\\
350.01	0\\
351.01	0\\
352.01	0\\
353.01	1.73472347597681e-18\\
354.01	1.73472347597681e-18\\
355.01	0\\
356.01	0\\
357.01	1.73472347597681e-18\\
358.01	1.73472347597681e-18\\
359.01	1.73472347597681e-18\\
360.01	0\\
361.01	0\\
362.01	0\\
363.01	0\\
364.01	0\\
365.01	1.73472347597681e-18\\
366.01	1.73472347597681e-18\\
367.01	1.73472347597681e-18\\
368.01	1.73472347597681e-18\\
369.01	0\\
370.01	0\\
371.01	1.73472347597681e-18\\
372.01	1.73472347597681e-18\\
373.01	1.73472347597681e-18\\
374.01	0\\
375.01	1.73472347597681e-18\\
376.01	0\\
377.01	0\\
378.01	1.73472347597681e-18\\
379.01	0\\
380.01	1.73472347597681e-18\\
381.01	1.73472347597681e-18\\
382.01	1.73472347597681e-18\\
383.01	0\\
384.01	1.73472347597681e-18\\
385.01	0\\
386.01	1.73472347597681e-18\\
387.01	1.73472347597681e-18\\
388.01	1.73472347597681e-18\\
389.01	0\\
390.01	1.73472347597681e-18\\
391.01	1.73472347597681e-18\\
392.01	0\\
393.01	0\\
394.01	1.73472347597681e-18\\
395.01	1.73472347597681e-18\\
396.01	1.73472347597681e-18\\
397.01	1.73472347597681e-18\\
398.01	0\\
399.01	1.73472347597681e-18\\
400.01	0\\
401.01	0\\
402.01	0\\
403.01	1.73472347597681e-18\\
404.01	1.73472347597681e-18\\
405.01	1.73472347597681e-18\\
406.01	1.73472347597681e-18\\
407.01	1.73472347597681e-18\\
408.01	1.73472347597681e-18\\
409.01	0\\
410.01	1.73472347597681e-18\\
411.01	0\\
412.01	1.73472347597681e-18\\
413.01	1.73472347597681e-18\\
414.01	0\\
415.01	0\\
416.01	1.73472347597681e-18\\
417.01	1.73472347597681e-18\\
418.01	1.73472347597681e-18\\
419.01	1.73472347597681e-18\\
420.01	0\\
421.01	1.73472347597681e-18\\
422.01	1.73472347597681e-18\\
423.01	1.73472347597681e-18\\
424.01	1.73472347597681e-18\\
425.01	0\\
426.01	1.73472347597681e-18\\
427.01	1.73472347597681e-18\\
428.01	1.73472347597681e-18\\
429.01	0\\
430.01	1.73472347597681e-18\\
431.01	1.73472347597681e-18\\
432.01	1.73472347597681e-18\\
433.01	0\\
434.01	1.73472347597681e-18\\
435.01	1.73472347597681e-18\\
436.01	1.73472347597681e-18\\
437.01	1.73472347597681e-18\\
438.01	1.73472347597681e-18\\
439.01	0\\
440.01	1.73472347597681e-18\\
441.01	1.73472347597681e-18\\
442.01	0\\
443.01	0\\
444.01	0\\
445.01	1.73472347597681e-18\\
446.01	1.73472347597681e-18\\
447.01	0\\
448.01	1.73472347597681e-18\\
449.01	1.73472347597681e-18\\
450.01	0\\
451.01	0\\
452.01	0\\
453.01	1.73472347597681e-18\\
454.01	1.73472347597681e-18\\
455.01	0\\
456.01	1.73472347597681e-18\\
457.01	1.73472347597681e-18\\
458.01	0\\
459.01	1.73472347597681e-18\\
460.01	0\\
461.01	1.73472347597681e-18\\
462.01	0\\
463.01	0\\
464.01	0\\
465.01	1.73472347597681e-18\\
466.01	0\\
467.01	1.73472347597681e-18\\
468.01	0\\
469.01	1.73472347597681e-18\\
470.01	1.73472347597681e-18\\
471.01	1.73472347597681e-18\\
472.01	0\\
473.01	0\\
474.01	0\\
475.01	1.73472347597681e-18\\
476.01	0\\
477.01	0\\
478.01	1.73472347597681e-18\\
479.01	1.73472347597681e-18\\
480.01	0\\
481.01	0\\
482.01	1.73472347597681e-18\\
483.01	1.73472347597681e-18\\
484.01	0\\
485.01	0\\
486.01	0\\
487.01	1.73472347597681e-18\\
488.01	1.73472347597681e-18\\
489.01	1.73472347597681e-18\\
490.01	1.73472347597681e-18\\
491.01	0\\
492.01	1.73472347597681e-18\\
493.01	1.73472347597681e-18\\
494.01	0\\
495.01	1.73472347597681e-18\\
496.01	0\\
497.01	1.73472347597681e-18\\
498.01	1.73472347597681e-18\\
499.01	0\\
500.01	1.73472347597681e-18\\
501.01	1.73472347597681e-18\\
502.01	0\\
503.01	1.73472347597681e-18\\
504.01	1.73472347597681e-18\\
505.01	1.73472347597681e-18\\
506.01	0\\
507.01	0\\
508.01	0\\
509.01	0\\
510.01	1.73472347597681e-18\\
511.01	1.73472347597681e-18\\
512.01	0\\
513.01	1.73472347597681e-18\\
514.01	1.73472347597681e-18\\
515.01	0\\
516.01	0\\
517.01	1.73472347597681e-18\\
518.01	1.73472347597681e-18\\
519.01	1.73472347597681e-18\\
520.01	0\\
521.01	0\\
522.01	0\\
523.01	1.73472347597681e-18\\
524.01	0\\
525.01	1.73472347597681e-18\\
526.01	1.73472347597681e-18\\
527.01	0\\
528.01	0\\
529.01	0\\
530.01	1.73472347597681e-18\\
531.01	0\\
532.01	1.73472347597681e-18\\
533.01	0\\
534.01	1.73472347597681e-18\\
535.01	0\\
536.01	1.73472347597681e-18\\
537.01	1.73472347597681e-18\\
538.01	1.73472347597681e-18\\
539.01	0\\
540.01	0\\
541.01	1.73472347597681e-18\\
542.01	1.73472347597681e-18\\
543.01	0\\
544.01	1.73472347597681e-18\\
545.01	0\\
546.01	1.73472347597681e-18\\
547.01	0\\
548.01	0\\
549.01	0\\
550.01	0\\
551.01	0\\
552.01	0\\
553.01	0\\
554.01	0\\
555.01	1.73472347597681e-18\\
556.01	0\\
557.01	0\\
558.01	0\\
559.01	1.73472347597681e-18\\
560.01	1.73472347597681e-18\\
561.01	1.73472347597681e-18\\
562.01	1.73472347597681e-18\\
563.01	0\\
564.01	0\\
565.01	1.73472347597681e-18\\
566.01	0\\
567.01	0\\
568.01	0\\
569.01	0\\
570.01	0\\
571.01	0\\
572.01	1.73472347597681e-18\\
573.01	0\\
574.01	1.73472347597681e-18\\
575.01	0\\
576.01	1.73472347597681e-18\\
577.01	0\\
578.01	1.73472347597681e-18\\
579.01	0\\
580.01	0\\
581.01	1.73472347597681e-18\\
582.01	0\\
583.01	0\\
584.01	0\\
585.01	0\\
586.01	0\\
587.01	0\\
588.01	0\\
589.01	0\\
590.01	0\\
591.01	0\\
592.01	1.73472347597681e-18\\
593.01	0\\
594.01	0\\
595.01	0\\
596.01	0\\
597.01	0\\
598.01	0\\
599.01	0\\
599.02	0\\
599.03	1.73472347597681e-18\\
599.04	1.73472347597681e-18\\
599.05	0\\
599.06	0\\
599.07	1.73472347597681e-18\\
599.08	0\\
599.09	0\\
599.1	0\\
599.11	1.73472347597681e-18\\
599.12	1.73472347597681e-18\\
599.13	0\\
599.14	0\\
599.15	1.73472347597681e-18\\
599.16	1.73472347597681e-18\\
599.17	1.73472347597681e-18\\
599.18	0\\
599.19	0\\
599.2	0\\
599.21	0\\
599.22	0\\
599.23	1.73472347597681e-18\\
599.24	1.73472347597681e-18\\
599.25	1.73472347597681e-18\\
599.26	1.73472347597681e-18\\
599.27	0\\
599.28	0\\
599.29	0\\
599.3	0\\
599.31	0\\
599.32	1.73472347597681e-18\\
599.33	1.73472347597681e-18\\
599.34	0\\
599.35	0\\
599.36	0\\
599.37	1.73472347597681e-18\\
599.38	0\\
599.39	0\\
599.4	0\\
599.41	0\\
599.42	0\\
599.43	0\\
599.44	0\\
599.45	0\\
599.46	1.73472347597681e-18\\
599.47	1.73472347597681e-18\\
599.48	1.73472347597681e-18\\
599.49	1.73472347597681e-18\\
599.5	1.73472347597681e-18\\
599.51	0\\
599.52	0\\
599.53	1.73472347597681e-18\\
599.54	0\\
599.55	0\\
599.56	0\\
599.57	1.73472347597681e-18\\
599.58	0\\
599.59	0\\
599.6	0\\
599.61	0\\
599.62	1.73472347597681e-18\\
599.63	0\\
599.64	1.73472347597681e-18\\
599.65	0\\
599.66	0\\
599.67	0\\
599.68	0\\
599.69	1.73472347597681e-18\\
599.7	0\\
599.71	1.73472347597681e-18\\
599.72	1.73472347597681e-18\\
599.73	1.73472347597681e-18\\
599.74	0\\
599.75	0\\
599.76	0\\
599.77	0\\
599.78	0\\
599.79	0\\
599.8	0\\
599.81	0\\
599.82	0\\
599.83	1.73472347597681e-18\\
599.84	0\\
599.85	0\\
599.86	0\\
599.87	0\\
599.88	0\\
599.89	0\\
599.9	0\\
599.91	0\\
599.92	0\\
599.93	0\\
599.94	0\\
599.95	0\\
599.96	0\\
599.97	0\\
599.98	0\\
599.99	0\\
600	0\\
};
\addplot [color=mycolor5,solid,forget plot]
  table[row sep=crcr]{%
0.01	1.73472347597681e-18\\
1.01	1.73472347597681e-18\\
2.01	0\\
3.01	1.73472347597681e-18\\
4.01	1.73472347597681e-18\\
5.01	0\\
6.01	0\\
7.01	1.73472347597681e-18\\
8.01	0\\
9.01	0\\
10.01	0\\
11.01	0\\
12.01	1.73472347597681e-18\\
13.01	1.73472347597681e-18\\
14.01	1.73472347597681e-18\\
15.01	1.73472347597681e-18\\
16.01	1.73472347597681e-18\\
17.01	1.73472347597681e-18\\
18.01	0\\
19.01	0\\
20.01	1.73472347597681e-18\\
21.01	0\\
22.01	1.73472347597681e-18\\
23.01	1.73472347597681e-18\\
24.01	1.73472347597681e-18\\
25.01	0\\
26.01	1.73472347597681e-18\\
27.01	0\\
28.01	1.73472347597681e-18\\
29.01	1.73472347597681e-18\\
30.01	0\\
31.01	0\\
32.01	1.73472347597681e-18\\
33.01	1.73472347597681e-18\\
34.01	1.73472347597681e-18\\
35.01	0\\
36.01	0\\
37.01	0\\
38.01	0\\
39.01	1.73472347597681e-18\\
40.01	1.73472347597681e-18\\
41.01	1.73472347597681e-18\\
42.01	1.73472347597681e-18\\
43.01	1.73472347597681e-18\\
44.01	0\\
45.01	1.73472347597681e-18\\
46.01	1.73472347597681e-18\\
47.01	1.73472347597681e-18\\
48.01	1.73472347597681e-18\\
49.01	0\\
50.01	1.73472347597681e-18\\
51.01	1.73472347597681e-18\\
52.01	0\\
53.01	0\\
54.01	0\\
55.01	1.73472347597681e-18\\
56.01	0\\
57.01	1.73472347597681e-18\\
58.01	0\\
59.01	0\\
60.01	1.73472347597681e-18\\
61.01	1.73472347597681e-18\\
62.01	1.73472347597681e-18\\
63.01	0\\
64.01	1.73472347597681e-18\\
65.01	0\\
66.01	1.73472347597681e-18\\
67.01	1.73472347597681e-18\\
68.01	0\\
69.01	1.73472347597681e-18\\
70.01	1.73472347597681e-18\\
71.01	0\\
72.01	0\\
73.01	1.73472347597681e-18\\
74.01	0\\
75.01	1.73472347597681e-18\\
76.01	0\\
77.01	0\\
78.01	0\\
79.01	0\\
80.01	1.73472347597681e-18\\
81.01	1.73472347597681e-18\\
82.01	0\\
83.01	0\\
84.01	0\\
85.01	1.73472347597681e-18\\
86.01	1.73472347597681e-18\\
87.01	0\\
88.01	0\\
89.01	0\\
90.01	1.73472347597681e-18\\
91.01	1.73472347597681e-18\\
92.01	0\\
93.01	1.73472347597681e-18\\
94.01	1.73472347597681e-18\\
95.01	0\\
96.01	1.73472347597681e-18\\
97.01	0\\
98.01	0\\
99.01	0\\
100.01	1.73472347597681e-18\\
101.01	0\\
102.01	1.73472347597681e-18\\
103.01	1.73472347597681e-18\\
104.01	0\\
105.01	0\\
106.01	0\\
107.01	0\\
108.01	1.73472347597681e-18\\
109.01	0\\
110.01	1.73472347597681e-18\\
111.01	0\\
112.01	0\\
113.01	0\\
114.01	1.73472347597681e-18\\
115.01	1.73472347597681e-18\\
116.01	0\\
117.01	0\\
118.01	0\\
119.01	0\\
120.01	0\\
121.01	0\\
122.01	0\\
123.01	1.73472347597681e-18\\
124.01	1.73472347597681e-18\\
125.01	1.73472347597681e-18\\
126.01	1.73472347597681e-18\\
127.01	0\\
128.01	1.73472347597681e-18\\
129.01	0\\
130.01	1.73472347597681e-18\\
131.01	1.73472347597681e-18\\
132.01	0\\
133.01	1.73472347597681e-18\\
134.01	1.73472347597681e-18\\
135.01	0\\
136.01	0\\
137.01	1.73472347597681e-18\\
138.01	0\\
139.01	1.73472347597681e-18\\
140.01	1.73472347597681e-18\\
141.01	1.73472347597681e-18\\
142.01	1.73472347597681e-18\\
143.01	0\\
144.01	1.73472347597681e-18\\
145.01	1.73472347597681e-18\\
146.01	1.73472347597681e-18\\
147.01	1.73472347597681e-18\\
148.01	1.73472347597681e-18\\
149.01	0\\
150.01	0\\
151.01	1.73472347597681e-18\\
152.01	1.73472347597681e-18\\
153.01	1.73472347597681e-18\\
154.01	1.73472347597681e-18\\
155.01	1.73472347597681e-18\\
156.01	1.73472347597681e-18\\
157.01	0\\
158.01	1.73472347597681e-18\\
159.01	1.73472347597681e-18\\
160.01	1.73472347597681e-18\\
161.01	1.73472347597681e-18\\
162.01	1.73472347597681e-18\\
163.01	1.73472347597681e-18\\
164.01	0\\
165.01	0\\
166.01	1.73472347597681e-18\\
167.01	0\\
168.01	1.73472347597681e-18\\
169.01	0\\
170.01	1.73472347597681e-18\\
171.01	0\\
172.01	0\\
173.01	0\\
174.01	1.73472347597681e-18\\
175.01	1.73472347597681e-18\\
176.01	1.73472347597681e-18\\
177.01	0\\
178.01	1.73472347597681e-18\\
179.01	0\\
180.01	0\\
181.01	1.73472347597681e-18\\
182.01	0\\
183.01	0\\
184.01	0\\
185.01	1.73472347597681e-18\\
186.01	0\\
187.01	0\\
188.01	0\\
189.01	0\\
190.01	1.73472347597681e-18\\
191.01	0\\
192.01	1.73472347597681e-18\\
193.01	0\\
194.01	1.73472347597681e-18\\
195.01	0\\
196.01	1.73472347597681e-18\\
197.01	0\\
198.01	0\\
199.01	0\\
200.01	0\\
201.01	0\\
202.01	1.73472347597681e-18\\
203.01	0\\
204.01	1.73472347597681e-18\\
205.01	1.73472347597681e-18\\
206.01	1.73472347597681e-18\\
207.01	1.73472347597681e-18\\
208.01	0\\
209.01	1.73472347597681e-18\\
210.01	0\\
211.01	1.73472347597681e-18\\
212.01	0\\
213.01	1.73472347597681e-18\\
214.01	0\\
215.01	0\\
216.01	0\\
217.01	0\\
218.01	1.73472347597681e-18\\
219.01	1.73472347597681e-18\\
220.01	0\\
221.01	0\\
222.01	0\\
223.01	1.73472347597681e-18\\
224.01	1.73472347597681e-18\\
225.01	0\\
226.01	1.73472347597681e-18\\
227.01	0\\
228.01	0\\
229.01	1.73472347597681e-18\\
230.01	0\\
231.01	0\\
232.01	0\\
233.01	0\\
234.01	1.73472347597681e-18\\
235.01	0\\
236.01	1.73472347597681e-18\\
237.01	1.73472347597681e-18\\
238.01	0\\
239.01	1.73472347597681e-18\\
240.01	0\\
241.01	1.73472347597681e-18\\
242.01	0\\
243.01	1.73472347597681e-18\\
244.01	1.73472347597681e-18\\
245.01	1.73472347597681e-18\\
246.01	1.73472347597681e-18\\
247.01	0\\
248.01	0\\
249.01	0\\
250.01	1.73472347597681e-18\\
251.01	0\\
252.01	1.73472347597681e-18\\
253.01	1.73472347597681e-18\\
254.01	1.73472347597681e-18\\
255.01	1.73472347597681e-18\\
256.01	1.73472347597681e-18\\
257.01	1.73472347597681e-18\\
258.01	0\\
259.01	0\\
260.01	0\\
261.01	0\\
262.01	1.73472347597681e-18\\
263.01	1.73472347597681e-18\\
264.01	0\\
265.01	1.73472347597681e-18\\
266.01	1.73472347597681e-18\\
267.01	0\\
268.01	0\\
269.01	1.73472347597681e-18\\
270.01	0\\
271.01	1.73472347597681e-18\\
272.01	0\\
273.01	1.73472347597681e-18\\
274.01	0\\
275.01	0\\
276.01	1.73472347597681e-18\\
277.01	1.73472347597681e-18\\
278.01	0\\
279.01	1.73472347597681e-18\\
280.01	0\\
281.01	1.73472347597681e-18\\
282.01	1.73472347597681e-18\\
283.01	1.73472347597681e-18\\
284.01	0\\
285.01	1.73472347597681e-18\\
286.01	1.73472347597681e-18\\
287.01	1.73472347597681e-18\\
288.01	1.73472347597681e-18\\
289.01	0\\
290.01	0\\
291.01	0\\
292.01	0\\
293.01	1.73472347597681e-18\\
294.01	1.73472347597681e-18\\
295.01	0\\
296.01	1.73472347597681e-18\\
297.01	0\\
298.01	1.73472347597681e-18\\
299.01	0\\
300.01	1.73472347597681e-18\\
301.01	0\\
302.01	0\\
303.01	0\\
304.01	1.73472347597681e-18\\
305.01	1.73472347597681e-18\\
306.01	0\\
307.01	1.73472347597681e-18\\
308.01	0\\
309.01	1.73472347597681e-18\\
310.01	1.73472347597681e-18\\
311.01	0\\
312.01	1.73472347597681e-18\\
313.01	1.73472347597681e-18\\
314.01	0\\
315.01	1.73472347597681e-18\\
316.01	0\\
317.01	1.73472347597681e-18\\
318.01	0\\
319.01	0\\
320.01	0\\
321.01	0\\
322.01	1.73472347597681e-18\\
323.01	1.73472347597681e-18\\
324.01	0\\
325.01	1.73472347597681e-18\\
326.01	1.73472347597681e-18\\
327.01	0\\
328.01	1.73472347597681e-18\\
329.01	0\\
330.01	0\\
331.01	1.73472347597681e-18\\
332.01	1.73472347597681e-18\\
333.01	1.73472347597681e-18\\
334.01	1.73472347597681e-18\\
335.01	1.73472347597681e-18\\
336.01	0\\
337.01	1.73472347597681e-18\\
338.01	0\\
339.01	0\\
340.01	1.73472347597681e-18\\
341.01	1.73472347597681e-18\\
342.01	1.73472347597681e-18\\
343.01	1.73472347597681e-18\\
344.01	0\\
345.01	0\\
346.01	0\\
347.01	1.73472347597681e-18\\
348.01	0\\
349.01	1.73472347597681e-18\\
350.01	0\\
351.01	0\\
352.01	0\\
353.01	1.73472347597681e-18\\
354.01	1.73472347597681e-18\\
355.01	0\\
356.01	0\\
357.01	1.73472347597681e-18\\
358.01	1.73472347597681e-18\\
359.01	1.73472347597681e-18\\
360.01	0\\
361.01	0\\
362.01	0\\
363.01	0\\
364.01	0\\
365.01	1.73472347597681e-18\\
366.01	1.73472347597681e-18\\
367.01	1.73472347597681e-18\\
368.01	1.73472347597681e-18\\
369.01	0\\
370.01	0\\
371.01	1.73472347597681e-18\\
372.01	1.73472347597681e-18\\
373.01	1.73472347597681e-18\\
374.01	0\\
375.01	1.73472347597681e-18\\
376.01	0\\
377.01	0\\
378.01	1.73472347597681e-18\\
379.01	0\\
380.01	1.73472347597681e-18\\
381.01	1.73472347597681e-18\\
382.01	1.73472347597681e-18\\
383.01	0\\
384.01	1.73472347597681e-18\\
385.01	0\\
386.01	1.73472347597681e-18\\
387.01	1.73472347597681e-18\\
388.01	1.73472347597681e-18\\
389.01	0\\
390.01	1.73472347597681e-18\\
391.01	1.73472347597681e-18\\
392.01	0\\
393.01	0\\
394.01	1.73472347597681e-18\\
395.01	1.73472347597681e-18\\
396.01	1.73472347597681e-18\\
397.01	1.73472347597681e-18\\
398.01	0\\
399.01	1.73472347597681e-18\\
400.01	0\\
401.01	0\\
402.01	0\\
403.01	1.73472347597681e-18\\
404.01	1.73472347597681e-18\\
405.01	1.73472347597681e-18\\
406.01	1.73472347597681e-18\\
407.01	1.73472347597681e-18\\
408.01	1.73472347597681e-18\\
409.01	0\\
410.01	1.73472347597681e-18\\
411.01	0\\
412.01	1.73472347597681e-18\\
413.01	1.73472347597681e-18\\
414.01	0\\
415.01	0\\
416.01	1.73472347597681e-18\\
417.01	1.73472347597681e-18\\
418.01	1.73472347597681e-18\\
419.01	1.73472347597681e-18\\
420.01	0\\
421.01	1.73472347597681e-18\\
422.01	1.73472347597681e-18\\
423.01	1.73472347597681e-18\\
424.01	1.73472347597681e-18\\
425.01	0\\
426.01	1.73472347597681e-18\\
427.01	1.73472347597681e-18\\
428.01	1.73472347597681e-18\\
429.01	0\\
430.01	1.73472347597681e-18\\
431.01	1.73472347597681e-18\\
432.01	1.73472347597681e-18\\
433.01	0\\
434.01	1.73472347597681e-18\\
435.01	1.73472347597681e-18\\
436.01	1.73472347597681e-18\\
437.01	1.73472347597681e-18\\
438.01	1.73472347597681e-18\\
439.01	0\\
440.01	1.73472347597681e-18\\
441.01	1.73472347597681e-18\\
442.01	0\\
443.01	0\\
444.01	0\\
445.01	1.73472347597681e-18\\
446.01	1.73472347597681e-18\\
447.01	0\\
448.01	1.73472347597681e-18\\
449.01	1.73472347597681e-18\\
450.01	0\\
451.01	0\\
452.01	0\\
453.01	1.73472347597681e-18\\
454.01	1.73472347597681e-18\\
455.01	0\\
456.01	1.73472347597681e-18\\
457.01	1.73472347597681e-18\\
458.01	0\\
459.01	1.73472347597681e-18\\
460.01	0\\
461.01	1.73472347597681e-18\\
462.01	0\\
463.01	0\\
464.01	0\\
465.01	1.73472347597681e-18\\
466.01	0\\
467.01	1.73472347597681e-18\\
468.01	0\\
469.01	1.73472347597681e-18\\
470.01	1.73472347597681e-18\\
471.01	1.73472347597681e-18\\
472.01	0\\
473.01	0\\
474.01	0\\
475.01	1.73472347597681e-18\\
476.01	0\\
477.01	0\\
478.01	1.73472347597681e-18\\
479.01	1.73472347597681e-18\\
480.01	0\\
481.01	0\\
482.01	1.73472347597681e-18\\
483.01	1.73472347597681e-18\\
484.01	0\\
485.01	0\\
486.01	0\\
487.01	1.73472347597681e-18\\
488.01	1.73472347597681e-18\\
489.01	1.73472347597681e-18\\
490.01	1.73472347597681e-18\\
491.01	0\\
492.01	1.73472347597681e-18\\
493.01	1.73472347597681e-18\\
494.01	0\\
495.01	1.73472347597681e-18\\
496.01	0\\
497.01	1.73472347597681e-18\\
498.01	1.73472347597681e-18\\
499.01	0\\
500.01	1.73472347597681e-18\\
501.01	1.73472347597681e-18\\
502.01	0\\
503.01	1.73472347597681e-18\\
504.01	1.73472347597681e-18\\
505.01	1.73472347597681e-18\\
506.01	0\\
507.01	0\\
508.01	0\\
509.01	0\\
510.01	1.73472347597681e-18\\
511.01	1.73472347597681e-18\\
512.01	0\\
513.01	1.73472347597681e-18\\
514.01	1.73472347597681e-18\\
515.01	0\\
516.01	0\\
517.01	1.73472347597681e-18\\
518.01	1.73472347597681e-18\\
519.01	1.73472347597681e-18\\
520.01	0\\
521.01	0\\
522.01	0\\
523.01	1.73472347597681e-18\\
524.01	0\\
525.01	1.73472347597681e-18\\
526.01	1.73472347597681e-18\\
527.01	0\\
528.01	0\\
529.01	0\\
530.01	1.73472347597681e-18\\
531.01	0\\
532.01	1.73472347597681e-18\\
533.01	0\\
534.01	1.73472347597681e-18\\
535.01	0\\
536.01	1.73472347597681e-18\\
537.01	1.73472347597681e-18\\
538.01	1.73472347597681e-18\\
539.01	0\\
540.01	0\\
541.01	1.73472347597681e-18\\
542.01	1.73472347597681e-18\\
543.01	0\\
544.01	1.73472347597681e-18\\
545.01	0\\
546.01	1.73472347597681e-18\\
547.01	0\\
548.01	0\\
549.01	0\\
550.01	0\\
551.01	0\\
552.01	0\\
553.01	0\\
554.01	0\\
555.01	1.73472347597681e-18\\
556.01	0\\
557.01	0\\
558.01	0\\
559.01	1.73472347597681e-18\\
560.01	1.73472347597681e-18\\
561.01	1.73472347597681e-18\\
562.01	1.73472347597681e-18\\
563.01	0\\
564.01	0\\
565.01	1.73472347597681e-18\\
566.01	0\\
567.01	0\\
568.01	0\\
569.01	0\\
570.01	0\\
571.01	0\\
572.01	1.73472347597681e-18\\
573.01	0\\
574.01	1.73472347597681e-18\\
575.01	0\\
576.01	1.73472347597681e-18\\
577.01	0\\
578.01	1.73472347597681e-18\\
579.01	0\\
580.01	0\\
581.01	1.73472347597681e-18\\
582.01	0\\
583.01	0\\
584.01	0\\
585.01	0\\
586.01	0\\
587.01	0\\
588.01	0\\
589.01	0\\
590.01	0\\
591.01	0\\
592.01	1.73472347597681e-18\\
593.01	0\\
594.01	0\\
595.01	0\\
596.01	0\\
597.01	0\\
598.01	0\\
599.01	0\\
599.02	0\\
599.03	1.73472347597681e-18\\
599.04	1.73472347597681e-18\\
599.05	0\\
599.06	0\\
599.07	1.73472347597681e-18\\
599.08	0\\
599.09	0\\
599.1	0\\
599.11	1.73472347597681e-18\\
599.12	1.73472347597681e-18\\
599.13	0\\
599.14	0\\
599.15	1.73472347597681e-18\\
599.16	1.73472347597681e-18\\
599.17	1.73472347597681e-18\\
599.18	0\\
599.19	0\\
599.2	0\\
599.21	0\\
599.22	0\\
599.23	1.73472347597681e-18\\
599.24	1.73472347597681e-18\\
599.25	1.73472347597681e-18\\
599.26	1.73472347597681e-18\\
599.27	0\\
599.28	0\\
599.29	0\\
599.3	0\\
599.31	0\\
599.32	1.73472347597681e-18\\
599.33	1.73472347597681e-18\\
599.34	0\\
599.35	0\\
599.36	0\\
599.37	1.73472347597681e-18\\
599.38	0\\
599.39	0\\
599.4	0\\
599.41	0\\
599.42	0\\
599.43	0\\
599.44	0\\
599.45	0\\
599.46	1.73472347597681e-18\\
599.47	1.73472347597681e-18\\
599.48	1.73472347597681e-18\\
599.49	1.73472347597681e-18\\
599.5	1.73472347597681e-18\\
599.51	0\\
599.52	0\\
599.53	1.73472347597681e-18\\
599.54	0\\
599.55	0\\
599.56	0\\
599.57	1.73472347597681e-18\\
599.58	0\\
599.59	0\\
599.6	0\\
599.61	0\\
599.62	1.73472347597681e-18\\
599.63	0\\
599.64	1.73472347597681e-18\\
599.65	0\\
599.66	0\\
599.67	0\\
599.68	0\\
599.69	1.73472347597681e-18\\
599.7	0\\
599.71	1.73472347597681e-18\\
599.72	1.73472347597681e-18\\
599.73	1.73472347597681e-18\\
599.74	0\\
599.75	0\\
599.76	0\\
599.77	0\\
599.78	0\\
599.79	0\\
599.8	0\\
599.81	0\\
599.82	0\\
599.83	1.73472347597681e-18\\
599.84	0\\
599.85	0\\
599.86	0\\
599.87	0\\
599.88	0\\
599.89	0\\
599.9	0\\
599.91	0\\
599.92	0\\
599.93	0\\
599.94	0\\
599.95	0\\
599.96	0\\
599.97	0\\
599.98	0\\
599.99	0\\
600	0\\
};
\addplot [color=mycolor6,solid,forget plot]
  table[row sep=crcr]{%
0.01	1.73472347597681e-18\\
1.01	1.73472347597681e-18\\
2.01	0\\
3.01	1.73472347597681e-18\\
4.01	1.73472347597681e-18\\
5.01	0\\
6.01	0\\
7.01	1.73472347597681e-18\\
8.01	0\\
9.01	0\\
10.01	0\\
11.01	0\\
12.01	1.73472347597681e-18\\
13.01	1.73472347597681e-18\\
14.01	1.73472347597681e-18\\
15.01	1.73472347597681e-18\\
16.01	1.73472347597681e-18\\
17.01	1.73472347597681e-18\\
18.01	0\\
19.01	0\\
20.01	1.73472347597681e-18\\
21.01	0\\
22.01	1.73472347597681e-18\\
23.01	1.73472347597681e-18\\
24.01	1.73472347597681e-18\\
25.01	0\\
26.01	1.73472347597681e-18\\
27.01	0\\
28.01	1.73472347597681e-18\\
29.01	1.73472347597681e-18\\
30.01	0\\
31.01	0\\
32.01	1.73472347597681e-18\\
33.01	1.73472347597681e-18\\
34.01	1.73472347597681e-18\\
35.01	0\\
36.01	0\\
37.01	0\\
38.01	0\\
39.01	1.73472347597681e-18\\
40.01	1.73472347597681e-18\\
41.01	1.73472347597681e-18\\
42.01	1.73472347597681e-18\\
43.01	1.73472347597681e-18\\
44.01	0\\
45.01	1.73472347597681e-18\\
46.01	1.73472347597681e-18\\
47.01	1.73472347597681e-18\\
48.01	1.73472347597681e-18\\
49.01	0\\
50.01	1.73472347597681e-18\\
51.01	1.73472347597681e-18\\
52.01	0\\
53.01	0\\
54.01	0\\
55.01	1.73472347597681e-18\\
56.01	0\\
57.01	1.73472347597681e-18\\
58.01	0\\
59.01	0\\
60.01	1.73472347597681e-18\\
61.01	1.73472347597681e-18\\
62.01	1.73472347597681e-18\\
63.01	0\\
64.01	1.73472347597681e-18\\
65.01	0\\
66.01	1.73472347597681e-18\\
67.01	1.73472347597681e-18\\
68.01	0\\
69.01	1.73472347597681e-18\\
70.01	1.73472347597681e-18\\
71.01	0\\
72.01	0\\
73.01	1.73472347597681e-18\\
74.01	0\\
75.01	1.73472347597681e-18\\
76.01	0\\
77.01	0\\
78.01	0\\
79.01	0\\
80.01	1.73472347597681e-18\\
81.01	1.73472347597681e-18\\
82.01	0\\
83.01	0\\
84.01	0\\
85.01	1.73472347597681e-18\\
86.01	1.73472347597681e-18\\
87.01	0\\
88.01	0\\
89.01	0\\
90.01	1.73472347597681e-18\\
91.01	1.73472347597681e-18\\
92.01	0\\
93.01	1.73472347597681e-18\\
94.01	1.73472347597681e-18\\
95.01	0\\
96.01	1.73472347597681e-18\\
97.01	0\\
98.01	0\\
99.01	0\\
100.01	1.73472347597681e-18\\
101.01	0\\
102.01	1.73472347597681e-18\\
103.01	1.73472347597681e-18\\
104.01	0\\
105.01	0\\
106.01	0\\
107.01	0\\
108.01	1.73472347597681e-18\\
109.01	0\\
110.01	1.73472347597681e-18\\
111.01	0\\
112.01	0\\
113.01	0\\
114.01	1.73472347597681e-18\\
115.01	1.73472347597681e-18\\
116.01	0\\
117.01	0\\
118.01	0\\
119.01	0\\
120.01	0\\
121.01	0\\
122.01	0\\
123.01	1.73472347597681e-18\\
124.01	1.73472347597681e-18\\
125.01	1.73472347597681e-18\\
126.01	1.73472347597681e-18\\
127.01	0\\
128.01	1.73472347597681e-18\\
129.01	0\\
130.01	1.73472347597681e-18\\
131.01	1.73472347597681e-18\\
132.01	0\\
133.01	1.73472347597681e-18\\
134.01	1.73472347597681e-18\\
135.01	0\\
136.01	0\\
137.01	1.73472347597681e-18\\
138.01	0\\
139.01	1.73472347597681e-18\\
140.01	1.73472347597681e-18\\
141.01	1.73472347597681e-18\\
142.01	1.73472347597681e-18\\
143.01	0\\
144.01	1.73472347597681e-18\\
145.01	1.73472347597681e-18\\
146.01	1.73472347597681e-18\\
147.01	1.73472347597681e-18\\
148.01	1.73472347597681e-18\\
149.01	0\\
150.01	0\\
151.01	1.73472347597681e-18\\
152.01	1.73472347597681e-18\\
153.01	1.73472347597681e-18\\
154.01	1.73472347597681e-18\\
155.01	1.73472347597681e-18\\
156.01	1.73472347597681e-18\\
157.01	0\\
158.01	1.73472347597681e-18\\
159.01	1.73472347597681e-18\\
160.01	1.73472347597681e-18\\
161.01	1.73472347597681e-18\\
162.01	1.73472347597681e-18\\
163.01	1.73472347597681e-18\\
164.01	0\\
165.01	0\\
166.01	1.73472347597681e-18\\
167.01	0\\
168.01	1.73472347597681e-18\\
169.01	0\\
170.01	1.73472347597681e-18\\
171.01	0\\
172.01	0\\
173.01	0\\
174.01	1.73472347597681e-18\\
175.01	1.73472347597681e-18\\
176.01	1.73472347597681e-18\\
177.01	0\\
178.01	1.73472347597681e-18\\
179.01	0\\
180.01	0\\
181.01	1.73472347597681e-18\\
182.01	0\\
183.01	0\\
184.01	0\\
185.01	1.73472347597681e-18\\
186.01	0\\
187.01	0\\
188.01	0\\
189.01	0\\
190.01	1.73472347597681e-18\\
191.01	0\\
192.01	1.73472347597681e-18\\
193.01	0\\
194.01	1.73472347597681e-18\\
195.01	0\\
196.01	1.73472347597681e-18\\
197.01	0\\
198.01	0\\
199.01	0\\
200.01	0\\
201.01	0\\
202.01	1.73472347597681e-18\\
203.01	0\\
204.01	1.73472347597681e-18\\
205.01	1.73472347597681e-18\\
206.01	1.73472347597681e-18\\
207.01	1.73472347597681e-18\\
208.01	0\\
209.01	1.73472347597681e-18\\
210.01	0\\
211.01	1.73472347597681e-18\\
212.01	0\\
213.01	1.73472347597681e-18\\
214.01	0\\
215.01	0\\
216.01	0\\
217.01	0\\
218.01	1.73472347597681e-18\\
219.01	1.73472347597681e-18\\
220.01	0\\
221.01	0\\
222.01	0\\
223.01	1.73472347597681e-18\\
224.01	1.73472347597681e-18\\
225.01	0\\
226.01	1.73472347597681e-18\\
227.01	0\\
228.01	0\\
229.01	1.73472347597681e-18\\
230.01	0\\
231.01	0\\
232.01	0\\
233.01	0\\
234.01	1.73472347597681e-18\\
235.01	0\\
236.01	1.73472347597681e-18\\
237.01	1.73472347597681e-18\\
238.01	0\\
239.01	1.73472347597681e-18\\
240.01	0\\
241.01	1.73472347597681e-18\\
242.01	0\\
243.01	1.73472347597681e-18\\
244.01	1.73472347597681e-18\\
245.01	1.73472347597681e-18\\
246.01	1.73472347597681e-18\\
247.01	0\\
248.01	0\\
249.01	0\\
250.01	1.73472347597681e-18\\
251.01	0\\
252.01	1.73472347597681e-18\\
253.01	1.73472347597681e-18\\
254.01	1.73472347597681e-18\\
255.01	1.73472347597681e-18\\
256.01	1.73472347597681e-18\\
257.01	1.73472347597681e-18\\
258.01	0\\
259.01	0\\
260.01	0\\
261.01	0\\
262.01	1.73472347597681e-18\\
263.01	1.73472347597681e-18\\
264.01	0\\
265.01	1.73472347597681e-18\\
266.01	1.73472347597681e-18\\
267.01	0\\
268.01	0\\
269.01	1.73472347597681e-18\\
270.01	0\\
271.01	1.73472347597681e-18\\
272.01	0\\
273.01	1.73472347597681e-18\\
274.01	0\\
275.01	0\\
276.01	1.73472347597681e-18\\
277.01	1.73472347597681e-18\\
278.01	0\\
279.01	1.73472347597681e-18\\
280.01	0\\
281.01	1.73472347597681e-18\\
282.01	1.73472347597681e-18\\
283.01	1.73472347597681e-18\\
284.01	0\\
285.01	1.73472347597681e-18\\
286.01	1.73472347597681e-18\\
287.01	1.73472347597681e-18\\
288.01	1.73472347597681e-18\\
289.01	0\\
290.01	0\\
291.01	0\\
292.01	0\\
293.01	1.73472347597681e-18\\
294.01	1.73472347597681e-18\\
295.01	0\\
296.01	1.73472347597681e-18\\
297.01	0\\
298.01	1.73472347597681e-18\\
299.01	0\\
300.01	1.73472347597681e-18\\
301.01	0\\
302.01	0\\
303.01	0\\
304.01	1.73472347597681e-18\\
305.01	1.73472347597681e-18\\
306.01	0\\
307.01	1.73472347597681e-18\\
308.01	0\\
309.01	1.73472347597681e-18\\
310.01	1.73472347597681e-18\\
311.01	0\\
312.01	1.73472347597681e-18\\
313.01	1.73472347597681e-18\\
314.01	0\\
315.01	1.73472347597681e-18\\
316.01	0\\
317.01	1.73472347597681e-18\\
318.01	0\\
319.01	0\\
320.01	0\\
321.01	0\\
322.01	1.73472347597681e-18\\
323.01	1.73472347597681e-18\\
324.01	0\\
325.01	1.73472347597681e-18\\
326.01	1.73472347597681e-18\\
327.01	0\\
328.01	1.73472347597681e-18\\
329.01	0\\
330.01	0\\
331.01	1.73472347597681e-18\\
332.01	1.73472347597681e-18\\
333.01	1.73472347597681e-18\\
334.01	1.73472347597681e-18\\
335.01	1.73472347597681e-18\\
336.01	0\\
337.01	1.73472347597681e-18\\
338.01	0\\
339.01	0\\
340.01	1.73472347597681e-18\\
341.01	1.73472347597681e-18\\
342.01	1.73472347597681e-18\\
343.01	1.73472347597681e-18\\
344.01	0\\
345.01	0\\
346.01	0\\
347.01	1.73472347597681e-18\\
348.01	0\\
349.01	1.73472347597681e-18\\
350.01	0\\
351.01	0\\
352.01	0\\
353.01	1.73472347597681e-18\\
354.01	1.73472347597681e-18\\
355.01	0\\
356.01	0\\
357.01	1.73472347597681e-18\\
358.01	1.73472347597681e-18\\
359.01	1.73472347597681e-18\\
360.01	0\\
361.01	0\\
362.01	0\\
363.01	0\\
364.01	0\\
365.01	1.73472347597681e-18\\
366.01	1.73472347597681e-18\\
367.01	1.73472347597681e-18\\
368.01	1.73472347597681e-18\\
369.01	0\\
370.01	0\\
371.01	1.73472347597681e-18\\
372.01	1.73472347597681e-18\\
373.01	1.73472347597681e-18\\
374.01	0\\
375.01	1.73472347597681e-18\\
376.01	0\\
377.01	0\\
378.01	1.73472347597681e-18\\
379.01	0\\
380.01	1.73472347597681e-18\\
381.01	1.73472347597681e-18\\
382.01	1.73472347597681e-18\\
383.01	0\\
384.01	1.73472347597681e-18\\
385.01	0\\
386.01	1.73472347597681e-18\\
387.01	1.73472347597681e-18\\
388.01	1.73472347597681e-18\\
389.01	0\\
390.01	1.73472347597681e-18\\
391.01	1.73472347597681e-18\\
392.01	0\\
393.01	0\\
394.01	1.73472347597681e-18\\
395.01	1.73472347597681e-18\\
396.01	1.73472347597681e-18\\
397.01	1.73472347597681e-18\\
398.01	0\\
399.01	1.73472347597681e-18\\
400.01	0\\
401.01	0\\
402.01	0\\
403.01	1.73472347597681e-18\\
404.01	1.73472347597681e-18\\
405.01	1.73472347597681e-18\\
406.01	1.73472347597681e-18\\
407.01	1.73472347597681e-18\\
408.01	1.73472347597681e-18\\
409.01	0\\
410.01	1.73472347597681e-18\\
411.01	0\\
412.01	1.73472347597681e-18\\
413.01	1.73472347597681e-18\\
414.01	0\\
415.01	0\\
416.01	1.73472347597681e-18\\
417.01	1.73472347597681e-18\\
418.01	1.73472347597681e-18\\
419.01	1.73472347597681e-18\\
420.01	0\\
421.01	1.73472347597681e-18\\
422.01	1.73472347597681e-18\\
423.01	1.73472347597681e-18\\
424.01	1.73472347597681e-18\\
425.01	0\\
426.01	1.73472347597681e-18\\
427.01	1.73472347597681e-18\\
428.01	1.73472347597681e-18\\
429.01	0\\
430.01	1.73472347597681e-18\\
431.01	1.73472347597681e-18\\
432.01	1.73472347597681e-18\\
433.01	0\\
434.01	1.73472347597681e-18\\
435.01	1.73472347597681e-18\\
436.01	1.73472347597681e-18\\
437.01	1.73472347597681e-18\\
438.01	1.73472347597681e-18\\
439.01	0\\
440.01	1.73472347597681e-18\\
441.01	1.73472347597681e-18\\
442.01	0\\
443.01	0\\
444.01	0\\
445.01	1.73472347597681e-18\\
446.01	1.73472347597681e-18\\
447.01	0\\
448.01	1.73472347597681e-18\\
449.01	1.73472347597681e-18\\
450.01	0\\
451.01	0\\
452.01	0\\
453.01	1.73472347597681e-18\\
454.01	1.73472347597681e-18\\
455.01	0\\
456.01	1.73472347597681e-18\\
457.01	1.73472347597681e-18\\
458.01	0\\
459.01	1.73472347597681e-18\\
460.01	0\\
461.01	1.73472347597681e-18\\
462.01	0\\
463.01	0\\
464.01	0\\
465.01	1.73472347597681e-18\\
466.01	0\\
467.01	1.73472347597681e-18\\
468.01	0\\
469.01	1.73472347597681e-18\\
470.01	1.73472347597681e-18\\
471.01	1.73472347597681e-18\\
472.01	0\\
473.01	0\\
474.01	0\\
475.01	1.73472347597681e-18\\
476.01	0\\
477.01	0\\
478.01	1.73472347597681e-18\\
479.01	1.73472347597681e-18\\
480.01	0\\
481.01	0\\
482.01	1.73472347597681e-18\\
483.01	1.73472347597681e-18\\
484.01	0\\
485.01	0\\
486.01	0\\
487.01	1.73472347597681e-18\\
488.01	1.73472347597681e-18\\
489.01	1.73472347597681e-18\\
490.01	1.73472347597681e-18\\
491.01	0\\
492.01	1.73472347597681e-18\\
493.01	1.73472347597681e-18\\
494.01	0\\
495.01	1.73472347597681e-18\\
496.01	0\\
497.01	1.73472347597681e-18\\
498.01	1.73472347597681e-18\\
499.01	0\\
500.01	1.73472347597681e-18\\
501.01	1.73472347597681e-18\\
502.01	0\\
503.01	1.73472347597681e-18\\
504.01	1.73472347597681e-18\\
505.01	1.73472347597681e-18\\
506.01	0\\
507.01	0\\
508.01	0\\
509.01	0\\
510.01	1.73472347597681e-18\\
511.01	1.73472347597681e-18\\
512.01	0\\
513.01	1.73472347597681e-18\\
514.01	1.73472347597681e-18\\
515.01	0\\
516.01	0\\
517.01	1.73472347597681e-18\\
518.01	1.73472347597681e-18\\
519.01	1.73472347597681e-18\\
520.01	0\\
521.01	0\\
522.01	0\\
523.01	1.73472347597681e-18\\
524.01	0\\
525.01	1.73472347597681e-18\\
526.01	1.73472347597681e-18\\
527.01	0\\
528.01	0\\
529.01	0\\
530.01	1.73472347597681e-18\\
531.01	0\\
532.01	1.73472347597681e-18\\
533.01	0\\
534.01	1.73472347597681e-18\\
535.01	0\\
536.01	1.73472347597681e-18\\
537.01	1.73472347597681e-18\\
538.01	1.73472347597681e-18\\
539.01	0\\
540.01	0\\
541.01	1.73472347597681e-18\\
542.01	1.73472347597681e-18\\
543.01	0\\
544.01	1.73472347597681e-18\\
545.01	0\\
546.01	1.73472347597681e-18\\
547.01	0\\
548.01	0\\
549.01	0\\
550.01	0\\
551.01	0\\
552.01	0\\
553.01	0\\
554.01	0\\
555.01	1.73472347597681e-18\\
556.01	0\\
557.01	0\\
558.01	0\\
559.01	1.73472347597681e-18\\
560.01	1.73472347597681e-18\\
561.01	1.73472347597681e-18\\
562.01	1.73472347597681e-18\\
563.01	0\\
564.01	0\\
565.01	1.73472347597681e-18\\
566.01	0\\
567.01	0\\
568.01	0\\
569.01	0\\
570.01	0\\
571.01	0\\
572.01	1.73472347597681e-18\\
573.01	0\\
574.01	1.73472347597681e-18\\
575.01	0\\
576.01	1.73472347597681e-18\\
577.01	0\\
578.01	1.73472347597681e-18\\
579.01	0\\
580.01	0\\
581.01	1.73472347597681e-18\\
582.01	0\\
583.01	0\\
584.01	0\\
585.01	0\\
586.01	0\\
587.01	0\\
588.01	0\\
589.01	0\\
590.01	0\\
591.01	0\\
592.01	1.73472347597681e-18\\
593.01	0\\
594.01	0\\
595.01	0\\
596.01	0\\
597.01	0\\
598.01	0\\
599.01	0\\
599.02	0\\
599.03	1.73472347597681e-18\\
599.04	1.73472347597681e-18\\
599.05	0\\
599.06	0\\
599.07	1.73472347597681e-18\\
599.08	0\\
599.09	0\\
599.1	0\\
599.11	1.73472347597681e-18\\
599.12	1.73472347597681e-18\\
599.13	0\\
599.14	0\\
599.15	1.73472347597681e-18\\
599.16	1.73472347597681e-18\\
599.17	1.73472347597681e-18\\
599.18	0\\
599.19	0\\
599.2	0\\
599.21	0\\
599.22	0\\
599.23	1.73472347597681e-18\\
599.24	1.73472347597681e-18\\
599.25	1.73472347597681e-18\\
599.26	1.73472347597681e-18\\
599.27	0\\
599.28	0\\
599.29	0\\
599.3	0\\
599.31	0\\
599.32	1.73472347597681e-18\\
599.33	1.73472347597681e-18\\
599.34	0\\
599.35	0\\
599.36	0\\
599.37	1.73472347597681e-18\\
599.38	0\\
599.39	0\\
599.4	0\\
599.41	0\\
599.42	0\\
599.43	0\\
599.44	0\\
599.45	0\\
599.46	1.73472347597681e-18\\
599.47	1.73472347597681e-18\\
599.48	1.73472347597681e-18\\
599.49	1.73472347597681e-18\\
599.5	1.73472347597681e-18\\
599.51	0\\
599.52	0\\
599.53	1.73472347597681e-18\\
599.54	0\\
599.55	0\\
599.56	0\\
599.57	1.73472347597681e-18\\
599.58	0\\
599.59	0\\
599.6	0\\
599.61	0\\
599.62	1.73472347597681e-18\\
599.63	0\\
599.64	1.73472347597681e-18\\
599.65	0\\
599.66	0\\
599.67	0\\
599.68	0\\
599.69	1.73472347597681e-18\\
599.7	0\\
599.71	1.73472347597681e-18\\
599.72	1.73472347597681e-18\\
599.73	1.73472347597681e-18\\
599.74	0\\
599.75	0\\
599.76	0\\
599.77	0\\
599.78	0\\
599.79	0\\
599.8	0\\
599.81	0\\
599.82	0\\
599.83	1.73472347597681e-18\\
599.84	0\\
599.85	0\\
599.86	0\\
599.87	0\\
599.88	0\\
599.89	0\\
599.9	0\\
599.91	0\\
599.92	0\\
599.93	0\\
599.94	0\\
599.95	0\\
599.96	0\\
599.97	0\\
599.98	0\\
599.99	0\\
600	0\\
};
\addplot [color=mycolor7,solid,forget plot]
  table[row sep=crcr]{%
0.01	1.73472347597681e-18\\
1.01	1.73472347597681e-18\\
2.01	0\\
3.01	1.73472347597681e-18\\
4.01	1.73472347597681e-18\\
5.01	0\\
6.01	0\\
7.01	1.73472347597681e-18\\
8.01	0\\
9.01	0\\
10.01	0\\
11.01	0\\
12.01	1.73472347597681e-18\\
13.01	1.73472347597681e-18\\
14.01	1.73472347597681e-18\\
15.01	1.73472347597681e-18\\
16.01	1.73472347597681e-18\\
17.01	1.73472347597681e-18\\
18.01	0\\
19.01	0\\
20.01	1.73472347597681e-18\\
21.01	0\\
22.01	1.73472347597681e-18\\
23.01	1.73472347597681e-18\\
24.01	1.73472347597681e-18\\
25.01	0\\
26.01	1.73472347597681e-18\\
27.01	0\\
28.01	1.73472347597681e-18\\
29.01	1.73472347597681e-18\\
30.01	0\\
31.01	0\\
32.01	1.73472347597681e-18\\
33.01	1.73472347597681e-18\\
34.01	1.73472347597681e-18\\
35.01	0\\
36.01	0\\
37.01	0\\
38.01	0\\
39.01	1.73472347597681e-18\\
40.01	1.73472347597681e-18\\
41.01	1.73472347597681e-18\\
42.01	1.73472347597681e-18\\
43.01	1.73472347597681e-18\\
44.01	0\\
45.01	1.73472347597681e-18\\
46.01	1.73472347597681e-18\\
47.01	1.73472347597681e-18\\
48.01	1.73472347597681e-18\\
49.01	0\\
50.01	1.73472347597681e-18\\
51.01	1.73472347597681e-18\\
52.01	0\\
53.01	0\\
54.01	0\\
55.01	1.73472347597681e-18\\
56.01	0\\
57.01	1.73472347597681e-18\\
58.01	0\\
59.01	0\\
60.01	1.73472347597681e-18\\
61.01	1.73472347597681e-18\\
62.01	1.73472347597681e-18\\
63.01	0\\
64.01	1.73472347597681e-18\\
65.01	0\\
66.01	1.73472347597681e-18\\
67.01	1.73472347597681e-18\\
68.01	0\\
69.01	1.73472347597681e-18\\
70.01	1.73472347597681e-18\\
71.01	0\\
72.01	0\\
73.01	1.73472347597681e-18\\
74.01	0\\
75.01	1.73472347597681e-18\\
76.01	0\\
77.01	0\\
78.01	0\\
79.01	0\\
80.01	1.73472347597681e-18\\
81.01	1.73472347597681e-18\\
82.01	0\\
83.01	0\\
84.01	0\\
85.01	1.73472347597681e-18\\
86.01	1.73472347597681e-18\\
87.01	0\\
88.01	0\\
89.01	0\\
90.01	1.73472347597681e-18\\
91.01	1.73472347597681e-18\\
92.01	0\\
93.01	1.73472347597681e-18\\
94.01	1.73472347597681e-18\\
95.01	0\\
96.01	1.73472347597681e-18\\
97.01	0\\
98.01	0\\
99.01	0\\
100.01	1.73472347597681e-18\\
101.01	0\\
102.01	1.73472347597681e-18\\
103.01	1.73472347597681e-18\\
104.01	0\\
105.01	0\\
106.01	0\\
107.01	0\\
108.01	1.73472347597681e-18\\
109.01	0\\
110.01	1.73472347597681e-18\\
111.01	0\\
112.01	0\\
113.01	0\\
114.01	1.73472347597681e-18\\
115.01	1.73472347597681e-18\\
116.01	0\\
117.01	0\\
118.01	0\\
119.01	0\\
120.01	0\\
121.01	0\\
122.01	0\\
123.01	1.73472347597681e-18\\
124.01	1.73472347597681e-18\\
125.01	1.73472347597681e-18\\
126.01	1.73472347597681e-18\\
127.01	0\\
128.01	1.73472347597681e-18\\
129.01	0\\
130.01	1.73472347597681e-18\\
131.01	1.73472347597681e-18\\
132.01	0\\
133.01	1.73472347597681e-18\\
134.01	1.73472347597681e-18\\
135.01	0\\
136.01	0\\
137.01	1.73472347597681e-18\\
138.01	0\\
139.01	1.73472347597681e-18\\
140.01	1.73472347597681e-18\\
141.01	1.73472347597681e-18\\
142.01	1.73472347597681e-18\\
143.01	0\\
144.01	1.73472347597681e-18\\
145.01	1.73472347597681e-18\\
146.01	1.73472347597681e-18\\
147.01	1.73472347597681e-18\\
148.01	1.73472347597681e-18\\
149.01	0\\
150.01	0\\
151.01	1.73472347597681e-18\\
152.01	1.73472347597681e-18\\
153.01	1.73472347597681e-18\\
154.01	1.73472347597681e-18\\
155.01	1.73472347597681e-18\\
156.01	1.73472347597681e-18\\
157.01	0\\
158.01	1.73472347597681e-18\\
159.01	1.73472347597681e-18\\
160.01	1.73472347597681e-18\\
161.01	1.73472347597681e-18\\
162.01	1.73472347597681e-18\\
163.01	1.73472347597681e-18\\
164.01	0\\
165.01	0\\
166.01	1.73472347597681e-18\\
167.01	0\\
168.01	1.73472347597681e-18\\
169.01	0\\
170.01	1.73472347597681e-18\\
171.01	0\\
172.01	0\\
173.01	0\\
174.01	1.73472347597681e-18\\
175.01	1.73472347597681e-18\\
176.01	1.73472347597681e-18\\
177.01	0\\
178.01	1.73472347597681e-18\\
179.01	0\\
180.01	0\\
181.01	1.73472347597681e-18\\
182.01	0\\
183.01	0\\
184.01	0\\
185.01	1.73472347597681e-18\\
186.01	0\\
187.01	0\\
188.01	0\\
189.01	0\\
190.01	1.73472347597681e-18\\
191.01	0\\
192.01	1.73472347597681e-18\\
193.01	0\\
194.01	1.73472347597681e-18\\
195.01	0\\
196.01	1.73472347597681e-18\\
197.01	0\\
198.01	0\\
199.01	0\\
200.01	0\\
201.01	0\\
202.01	1.73472347597681e-18\\
203.01	0\\
204.01	1.73472347597681e-18\\
205.01	1.73472347597681e-18\\
206.01	1.73472347597681e-18\\
207.01	1.73472347597681e-18\\
208.01	0\\
209.01	1.73472347597681e-18\\
210.01	0\\
211.01	1.73472347597681e-18\\
212.01	0\\
213.01	1.73472347597681e-18\\
214.01	0\\
215.01	0\\
216.01	0\\
217.01	0\\
218.01	1.73472347597681e-18\\
219.01	1.73472347597681e-18\\
220.01	0\\
221.01	0\\
222.01	0\\
223.01	1.73472347597681e-18\\
224.01	1.73472347597681e-18\\
225.01	0\\
226.01	1.73472347597681e-18\\
227.01	0\\
228.01	0\\
229.01	1.73472347597681e-18\\
230.01	0\\
231.01	0\\
232.01	0\\
233.01	0\\
234.01	1.73472347597681e-18\\
235.01	0\\
236.01	1.73472347597681e-18\\
237.01	1.73472347597681e-18\\
238.01	0\\
239.01	1.73472347597681e-18\\
240.01	0\\
241.01	1.73472347597681e-18\\
242.01	0\\
243.01	1.73472347597681e-18\\
244.01	1.73472347597681e-18\\
245.01	1.73472347597681e-18\\
246.01	1.73472347597681e-18\\
247.01	0\\
248.01	0\\
249.01	0\\
250.01	1.73472347597681e-18\\
251.01	0\\
252.01	1.73472347597681e-18\\
253.01	1.73472347597681e-18\\
254.01	1.73472347597681e-18\\
255.01	1.73472347597681e-18\\
256.01	1.73472347597681e-18\\
257.01	1.73472347597681e-18\\
258.01	0\\
259.01	0\\
260.01	0\\
261.01	0\\
262.01	1.73472347597681e-18\\
263.01	1.73472347597681e-18\\
264.01	0\\
265.01	1.73472347597681e-18\\
266.01	1.73472347597681e-18\\
267.01	0\\
268.01	0\\
269.01	1.73472347597681e-18\\
270.01	0\\
271.01	1.73472347597681e-18\\
272.01	0\\
273.01	1.73472347597681e-18\\
274.01	0\\
275.01	0\\
276.01	1.73472347597681e-18\\
277.01	1.73472347597681e-18\\
278.01	0\\
279.01	1.73472347597681e-18\\
280.01	0\\
281.01	1.73472347597681e-18\\
282.01	1.73472347597681e-18\\
283.01	1.73472347597681e-18\\
284.01	0\\
285.01	1.73472347597681e-18\\
286.01	1.73472347597681e-18\\
287.01	1.73472347597681e-18\\
288.01	1.73472347597681e-18\\
289.01	0\\
290.01	0\\
291.01	0\\
292.01	0\\
293.01	1.73472347597681e-18\\
294.01	1.73472347597681e-18\\
295.01	0\\
296.01	1.73472347597681e-18\\
297.01	0\\
298.01	1.73472347597681e-18\\
299.01	0\\
300.01	1.73472347597681e-18\\
301.01	0\\
302.01	0\\
303.01	0\\
304.01	1.73472347597681e-18\\
305.01	1.73472347597681e-18\\
306.01	0\\
307.01	1.73472347597681e-18\\
308.01	0\\
309.01	1.73472347597681e-18\\
310.01	1.73472347597681e-18\\
311.01	0\\
312.01	1.73472347597681e-18\\
313.01	1.73472347597681e-18\\
314.01	0\\
315.01	1.73472347597681e-18\\
316.01	0\\
317.01	1.73472347597681e-18\\
318.01	0\\
319.01	0\\
320.01	0\\
321.01	0\\
322.01	1.73472347597681e-18\\
323.01	1.73472347597681e-18\\
324.01	0\\
325.01	1.73472347597681e-18\\
326.01	1.73472347597681e-18\\
327.01	0\\
328.01	1.73472347597681e-18\\
329.01	0\\
330.01	0\\
331.01	1.73472347597681e-18\\
332.01	1.73472347597681e-18\\
333.01	1.73472347597681e-18\\
334.01	1.73472347597681e-18\\
335.01	1.73472347597681e-18\\
336.01	0\\
337.01	1.73472347597681e-18\\
338.01	0\\
339.01	0\\
340.01	1.73472347597681e-18\\
341.01	1.73472347597681e-18\\
342.01	1.73472347597681e-18\\
343.01	1.73472347597681e-18\\
344.01	0\\
345.01	0\\
346.01	0\\
347.01	1.73472347597681e-18\\
348.01	0\\
349.01	1.73472347597681e-18\\
350.01	0\\
351.01	0\\
352.01	0\\
353.01	1.73472347597681e-18\\
354.01	1.73472347597681e-18\\
355.01	0\\
356.01	0\\
357.01	1.73472347597681e-18\\
358.01	1.73472347597681e-18\\
359.01	1.73472347597681e-18\\
360.01	0\\
361.01	0\\
362.01	0\\
363.01	0\\
364.01	0\\
365.01	1.73472347597681e-18\\
366.01	1.73472347597681e-18\\
367.01	1.73472347597681e-18\\
368.01	1.73472347597681e-18\\
369.01	0\\
370.01	0\\
371.01	1.73472347597681e-18\\
372.01	1.73472347597681e-18\\
373.01	1.73472347597681e-18\\
374.01	0\\
375.01	1.73472347597681e-18\\
376.01	0\\
377.01	0\\
378.01	1.73472347597681e-18\\
379.01	0\\
380.01	1.73472347597681e-18\\
381.01	1.73472347597681e-18\\
382.01	1.73472347597681e-18\\
383.01	0\\
384.01	1.73472347597681e-18\\
385.01	0\\
386.01	1.73472347597681e-18\\
387.01	1.73472347597681e-18\\
388.01	1.73472347597681e-18\\
389.01	0\\
390.01	1.73472347597681e-18\\
391.01	1.73472347597681e-18\\
392.01	0\\
393.01	0\\
394.01	1.73472347597681e-18\\
395.01	1.73472347597681e-18\\
396.01	1.73472347597681e-18\\
397.01	1.73472347597681e-18\\
398.01	0\\
399.01	1.73472347597681e-18\\
400.01	0\\
401.01	0\\
402.01	0\\
403.01	1.73472347597681e-18\\
404.01	1.73472347597681e-18\\
405.01	1.73472347597681e-18\\
406.01	1.73472347597681e-18\\
407.01	1.73472347597681e-18\\
408.01	1.73472347597681e-18\\
409.01	0\\
410.01	1.73472347597681e-18\\
411.01	0\\
412.01	1.73472347597681e-18\\
413.01	1.73472347597681e-18\\
414.01	0\\
415.01	0\\
416.01	1.73472347597681e-18\\
417.01	1.73472347597681e-18\\
418.01	1.73472347597681e-18\\
419.01	1.73472347597681e-18\\
420.01	0\\
421.01	1.73472347597681e-18\\
422.01	1.73472347597681e-18\\
423.01	1.73472347597681e-18\\
424.01	1.73472347597681e-18\\
425.01	0\\
426.01	1.73472347597681e-18\\
427.01	1.73472347597681e-18\\
428.01	1.73472347597681e-18\\
429.01	0\\
430.01	1.73472347597681e-18\\
431.01	1.73472347597681e-18\\
432.01	1.73472347597681e-18\\
433.01	0\\
434.01	1.73472347597681e-18\\
435.01	1.73472347597681e-18\\
436.01	1.73472347597681e-18\\
437.01	1.73472347597681e-18\\
438.01	1.73472347597681e-18\\
439.01	0\\
440.01	1.73472347597681e-18\\
441.01	1.73472347597681e-18\\
442.01	0\\
443.01	0\\
444.01	0\\
445.01	1.73472347597681e-18\\
446.01	1.73472347597681e-18\\
447.01	0\\
448.01	1.73472347597681e-18\\
449.01	1.73472347597681e-18\\
450.01	0\\
451.01	0\\
452.01	0\\
453.01	1.73472347597681e-18\\
454.01	1.73472347597681e-18\\
455.01	0\\
456.01	1.73472347597681e-18\\
457.01	1.73472347597681e-18\\
458.01	0\\
459.01	1.73472347597681e-18\\
460.01	0\\
461.01	1.73472347597681e-18\\
462.01	0\\
463.01	0\\
464.01	0\\
465.01	1.73472347597681e-18\\
466.01	0\\
467.01	1.73472347597681e-18\\
468.01	0\\
469.01	1.73472347597681e-18\\
470.01	1.73472347597681e-18\\
471.01	1.73472347597681e-18\\
472.01	0\\
473.01	0\\
474.01	0\\
475.01	1.73472347597681e-18\\
476.01	0\\
477.01	0\\
478.01	1.73472347597681e-18\\
479.01	1.73472347597681e-18\\
480.01	0\\
481.01	0\\
482.01	1.73472347597681e-18\\
483.01	1.73472347597681e-18\\
484.01	0\\
485.01	0\\
486.01	0\\
487.01	1.73472347597681e-18\\
488.01	1.73472347597681e-18\\
489.01	1.73472347597681e-18\\
490.01	1.73472347597681e-18\\
491.01	0\\
492.01	1.73472347597681e-18\\
493.01	1.73472347597681e-18\\
494.01	0\\
495.01	1.73472347597681e-18\\
496.01	0\\
497.01	1.73472347597681e-18\\
498.01	1.73472347597681e-18\\
499.01	0\\
500.01	1.73472347597681e-18\\
501.01	1.73472347597681e-18\\
502.01	0\\
503.01	1.73472347597681e-18\\
504.01	1.73472347597681e-18\\
505.01	1.73472347597681e-18\\
506.01	0\\
507.01	0\\
508.01	0\\
509.01	0\\
510.01	1.73472347597681e-18\\
511.01	1.73472347597681e-18\\
512.01	0\\
513.01	1.73472347597681e-18\\
514.01	1.73472347597681e-18\\
515.01	0\\
516.01	0\\
517.01	1.73472347597681e-18\\
518.01	1.73472347597681e-18\\
519.01	1.73472347597681e-18\\
520.01	0\\
521.01	0\\
522.01	0\\
523.01	1.73472347597681e-18\\
524.01	0\\
525.01	1.73472347597681e-18\\
526.01	1.73472347597681e-18\\
527.01	0\\
528.01	0\\
529.01	0\\
530.01	1.73472347597681e-18\\
531.01	0\\
532.01	1.73472347597681e-18\\
533.01	0\\
534.01	1.73472347597681e-18\\
535.01	0\\
536.01	1.73472347597681e-18\\
537.01	1.73472347597681e-18\\
538.01	1.73472347597681e-18\\
539.01	0\\
540.01	0\\
541.01	1.73472347597681e-18\\
542.01	1.73472347597681e-18\\
543.01	0\\
544.01	1.73472347597681e-18\\
545.01	0\\
546.01	1.73472347597681e-18\\
547.01	0\\
548.01	0\\
549.01	0\\
550.01	0\\
551.01	0\\
552.01	0\\
553.01	0\\
554.01	0\\
555.01	1.73472347597681e-18\\
556.01	0\\
557.01	0\\
558.01	0\\
559.01	1.73472347597681e-18\\
560.01	1.73472347597681e-18\\
561.01	1.73472347597681e-18\\
562.01	1.73472347597681e-18\\
563.01	0\\
564.01	0\\
565.01	1.73472347597681e-18\\
566.01	0\\
567.01	0\\
568.01	0\\
569.01	0\\
570.01	0\\
571.01	0\\
572.01	1.73472347597681e-18\\
573.01	0\\
574.01	1.73472347597681e-18\\
575.01	0\\
576.01	1.73472347597681e-18\\
577.01	0\\
578.01	1.73472347597681e-18\\
579.01	0\\
580.01	0\\
581.01	1.73472347597681e-18\\
582.01	0\\
583.01	0\\
584.01	0\\
585.01	0\\
586.01	0\\
587.01	0\\
588.01	0\\
589.01	0\\
590.01	0\\
591.01	0\\
592.01	1.73472347597681e-18\\
593.01	0\\
594.01	0\\
595.01	0\\
596.01	0\\
597.01	0\\
598.01	0\\
599.01	0\\
599.02	0\\
599.03	1.73472347597681e-18\\
599.04	1.73472347597681e-18\\
599.05	0\\
599.06	0\\
599.07	1.73472347597681e-18\\
599.08	0\\
599.09	0\\
599.1	0\\
599.11	1.73472347597681e-18\\
599.12	1.73472347597681e-18\\
599.13	0\\
599.14	0\\
599.15	1.73472347597681e-18\\
599.16	1.73472347597681e-18\\
599.17	1.73472347597681e-18\\
599.18	0\\
599.19	0\\
599.2	0\\
599.21	0\\
599.22	0\\
599.23	1.73472347597681e-18\\
599.24	1.73472347597681e-18\\
599.25	1.73472347597681e-18\\
599.26	1.73472347597681e-18\\
599.27	0\\
599.28	0\\
599.29	0\\
599.3	0\\
599.31	0\\
599.32	1.73472347597681e-18\\
599.33	1.73472347597681e-18\\
599.34	0\\
599.35	0\\
599.36	0\\
599.37	1.73472347597681e-18\\
599.38	0\\
599.39	0\\
599.4	0\\
599.41	0\\
599.42	0\\
599.43	0\\
599.44	0\\
599.45	0\\
599.46	1.73472347597681e-18\\
599.47	1.73472347597681e-18\\
599.48	1.73472347597681e-18\\
599.49	1.73472347597681e-18\\
599.5	1.73472347597681e-18\\
599.51	0\\
599.52	0\\
599.53	1.73472347597681e-18\\
599.54	0\\
599.55	0\\
599.56	0\\
599.57	1.73472347597681e-18\\
599.58	0\\
599.59	0\\
599.6	0\\
599.61	0\\
599.62	1.73472347597681e-18\\
599.63	0\\
599.64	1.73472347597681e-18\\
599.65	0\\
599.66	0\\
599.67	0\\
599.68	0\\
599.69	1.73472347597681e-18\\
599.7	0\\
599.71	1.73472347597681e-18\\
599.72	1.73472347597681e-18\\
599.73	1.73472347597681e-18\\
599.74	0\\
599.75	0\\
599.76	0\\
599.77	0\\
599.78	0\\
599.79	0\\
599.8	0\\
599.81	0\\
599.82	0\\
599.83	1.73472347597681e-18\\
599.84	0\\
599.85	0\\
599.86	0\\
599.87	0\\
599.88	0\\
599.89	0\\
599.9	0\\
599.91	0\\
599.92	0\\
599.93	0\\
599.94	0\\
599.95	0\\
599.96	0\\
599.97	0\\
599.98	0\\
599.99	0\\
600	0\\
};
\addplot [color=mycolor8,solid,forget plot]
  table[row sep=crcr]{%
0.01	1.73472347597681e-18\\
1.01	1.73472347597681e-18\\
2.01	0\\
3.01	1.73472347597681e-18\\
4.01	1.73472347597681e-18\\
5.01	0\\
6.01	0\\
7.01	1.73472347597681e-18\\
8.01	0\\
9.01	0\\
10.01	0\\
11.01	0\\
12.01	1.73472347597681e-18\\
13.01	1.73472347597681e-18\\
14.01	1.73472347597681e-18\\
15.01	1.73472347597681e-18\\
16.01	1.73472347597681e-18\\
17.01	1.73472347597681e-18\\
18.01	0\\
19.01	0\\
20.01	1.73472347597681e-18\\
21.01	0\\
22.01	1.73472347597681e-18\\
23.01	1.73472347597681e-18\\
24.01	1.73472347597681e-18\\
25.01	0\\
26.01	1.73472347597681e-18\\
27.01	0\\
28.01	1.73472347597681e-18\\
29.01	1.73472347597681e-18\\
30.01	0\\
31.01	0\\
32.01	1.73472347597681e-18\\
33.01	1.73472347597681e-18\\
34.01	1.73472347597681e-18\\
35.01	0\\
36.01	0\\
37.01	0\\
38.01	0\\
39.01	1.73472347597681e-18\\
40.01	1.73472347597681e-18\\
41.01	1.73472347597681e-18\\
42.01	1.73472347597681e-18\\
43.01	1.73472347597681e-18\\
44.01	0\\
45.01	1.73472347597681e-18\\
46.01	1.73472347597681e-18\\
47.01	1.73472347597681e-18\\
48.01	1.73472347597681e-18\\
49.01	0\\
50.01	1.73472347597681e-18\\
51.01	1.73472347597681e-18\\
52.01	0\\
53.01	0\\
54.01	0\\
55.01	1.73472347597681e-18\\
56.01	0\\
57.01	1.73472347597681e-18\\
58.01	0\\
59.01	0\\
60.01	1.73472347597681e-18\\
61.01	1.73472347597681e-18\\
62.01	1.73472347597681e-18\\
63.01	0\\
64.01	1.73472347597681e-18\\
65.01	0\\
66.01	1.73472347597681e-18\\
67.01	1.73472347597681e-18\\
68.01	0\\
69.01	1.73472347597681e-18\\
70.01	1.73472347597681e-18\\
71.01	0\\
72.01	0\\
73.01	1.73472347597681e-18\\
74.01	0\\
75.01	1.73472347597681e-18\\
76.01	0\\
77.01	0\\
78.01	0\\
79.01	0\\
80.01	1.73472347597681e-18\\
81.01	1.73472347597681e-18\\
82.01	0\\
83.01	0\\
84.01	0\\
85.01	1.73472347597681e-18\\
86.01	1.73472347597681e-18\\
87.01	0\\
88.01	0\\
89.01	0\\
90.01	1.73472347597681e-18\\
91.01	1.73472347597681e-18\\
92.01	0\\
93.01	1.73472347597681e-18\\
94.01	1.73472347597681e-18\\
95.01	0\\
96.01	1.73472347597681e-18\\
97.01	0\\
98.01	0\\
99.01	0\\
100.01	1.73472347597681e-18\\
101.01	0\\
102.01	1.73472347597681e-18\\
103.01	1.73472347597681e-18\\
104.01	0\\
105.01	0\\
106.01	0\\
107.01	0\\
108.01	1.73472347597681e-18\\
109.01	0\\
110.01	1.73472347597681e-18\\
111.01	0\\
112.01	0\\
113.01	0\\
114.01	1.73472347597681e-18\\
115.01	1.73472347597681e-18\\
116.01	0\\
117.01	0\\
118.01	0\\
119.01	0\\
120.01	0\\
121.01	0\\
122.01	0\\
123.01	1.73472347597681e-18\\
124.01	1.73472347597681e-18\\
125.01	1.73472347597681e-18\\
126.01	1.73472347597681e-18\\
127.01	0\\
128.01	1.73472347597681e-18\\
129.01	0\\
130.01	1.73472347597681e-18\\
131.01	1.73472347597681e-18\\
132.01	0\\
133.01	1.73472347597681e-18\\
134.01	1.73472347597681e-18\\
135.01	0\\
136.01	0\\
137.01	1.73472347597681e-18\\
138.01	0\\
139.01	1.73472347597681e-18\\
140.01	1.73472347597681e-18\\
141.01	1.73472347597681e-18\\
142.01	1.73472347597681e-18\\
143.01	0\\
144.01	1.73472347597681e-18\\
145.01	1.73472347597681e-18\\
146.01	1.73472347597681e-18\\
147.01	1.73472347597681e-18\\
148.01	1.73472347597681e-18\\
149.01	0\\
150.01	0\\
151.01	1.73472347597681e-18\\
152.01	1.73472347597681e-18\\
153.01	1.73472347597681e-18\\
154.01	1.73472347597681e-18\\
155.01	1.73472347597681e-18\\
156.01	1.73472347597681e-18\\
157.01	0\\
158.01	1.73472347597681e-18\\
159.01	1.73472347597681e-18\\
160.01	1.73472347597681e-18\\
161.01	1.73472347597681e-18\\
162.01	1.73472347597681e-18\\
163.01	1.73472347597681e-18\\
164.01	0\\
165.01	0\\
166.01	1.73472347597681e-18\\
167.01	0\\
168.01	1.73472347597681e-18\\
169.01	0\\
170.01	1.73472347597681e-18\\
171.01	0\\
172.01	0\\
173.01	0\\
174.01	1.73472347597681e-18\\
175.01	1.73472347597681e-18\\
176.01	1.73472347597681e-18\\
177.01	0\\
178.01	1.73472347597681e-18\\
179.01	0\\
180.01	0\\
181.01	1.73472347597681e-18\\
182.01	0\\
183.01	0\\
184.01	0\\
185.01	1.73472347597681e-18\\
186.01	0\\
187.01	0\\
188.01	0\\
189.01	0\\
190.01	1.73472347597681e-18\\
191.01	0\\
192.01	1.73472347597681e-18\\
193.01	0\\
194.01	1.73472347597681e-18\\
195.01	0\\
196.01	1.73472347597681e-18\\
197.01	0\\
198.01	0\\
199.01	0\\
200.01	0\\
201.01	0\\
202.01	1.73472347597681e-18\\
203.01	0\\
204.01	1.73472347597681e-18\\
205.01	1.73472347597681e-18\\
206.01	1.73472347597681e-18\\
207.01	1.73472347597681e-18\\
208.01	0\\
209.01	1.73472347597681e-18\\
210.01	0\\
211.01	1.73472347597681e-18\\
212.01	0\\
213.01	1.73472347597681e-18\\
214.01	0\\
215.01	0\\
216.01	0\\
217.01	0\\
218.01	1.73472347597681e-18\\
219.01	1.73472347597681e-18\\
220.01	0\\
221.01	0\\
222.01	0\\
223.01	1.73472347597681e-18\\
224.01	1.73472347597681e-18\\
225.01	0\\
226.01	1.73472347597681e-18\\
227.01	0\\
228.01	0\\
229.01	1.73472347597681e-18\\
230.01	0\\
231.01	0\\
232.01	0\\
233.01	0\\
234.01	1.73472347597681e-18\\
235.01	0\\
236.01	1.73472347597681e-18\\
237.01	1.73472347597681e-18\\
238.01	0\\
239.01	1.73472347597681e-18\\
240.01	0\\
241.01	1.73472347597681e-18\\
242.01	0\\
243.01	1.73472347597681e-18\\
244.01	1.73472347597681e-18\\
245.01	1.73472347597681e-18\\
246.01	1.73472347597681e-18\\
247.01	0\\
248.01	0\\
249.01	0\\
250.01	1.73472347597681e-18\\
251.01	0\\
252.01	1.73472347597681e-18\\
253.01	1.73472347597681e-18\\
254.01	1.73472347597681e-18\\
255.01	1.73472347597681e-18\\
256.01	1.73472347597681e-18\\
257.01	1.73472347597681e-18\\
258.01	0\\
259.01	0\\
260.01	0\\
261.01	0\\
262.01	1.73472347597681e-18\\
263.01	1.73472347597681e-18\\
264.01	0\\
265.01	1.73472347597681e-18\\
266.01	1.73472347597681e-18\\
267.01	0\\
268.01	0\\
269.01	1.73472347597681e-18\\
270.01	0\\
271.01	1.73472347597681e-18\\
272.01	0\\
273.01	1.73472347597681e-18\\
274.01	0\\
275.01	0\\
276.01	1.73472347597681e-18\\
277.01	1.73472347597681e-18\\
278.01	0\\
279.01	1.73472347597681e-18\\
280.01	0\\
281.01	1.73472347597681e-18\\
282.01	1.73472347597681e-18\\
283.01	1.73472347597681e-18\\
284.01	0\\
285.01	1.73472347597681e-18\\
286.01	1.73472347597681e-18\\
287.01	1.73472347597681e-18\\
288.01	1.73472347597681e-18\\
289.01	0\\
290.01	0\\
291.01	0\\
292.01	0\\
293.01	1.73472347597681e-18\\
294.01	1.73472347597681e-18\\
295.01	0\\
296.01	1.73472347597681e-18\\
297.01	0\\
298.01	1.73472347597681e-18\\
299.01	0\\
300.01	1.73472347597681e-18\\
301.01	0\\
302.01	0\\
303.01	0\\
304.01	1.73472347597681e-18\\
305.01	1.73472347597681e-18\\
306.01	0\\
307.01	1.73472347597681e-18\\
308.01	0\\
309.01	1.73472347597681e-18\\
310.01	1.73472347597681e-18\\
311.01	0\\
312.01	1.73472347597681e-18\\
313.01	1.73472347597681e-18\\
314.01	0\\
315.01	1.73472347597681e-18\\
316.01	0\\
317.01	1.73472347597681e-18\\
318.01	0\\
319.01	0\\
320.01	0\\
321.01	0\\
322.01	1.73472347597681e-18\\
323.01	1.73472347597681e-18\\
324.01	0\\
325.01	1.73472347597681e-18\\
326.01	1.73472347597681e-18\\
327.01	0\\
328.01	1.73472347597681e-18\\
329.01	0\\
330.01	0\\
331.01	1.73472347597681e-18\\
332.01	1.73472347597681e-18\\
333.01	1.73472347597681e-18\\
334.01	1.73472347597681e-18\\
335.01	1.73472347597681e-18\\
336.01	0\\
337.01	1.73472347597681e-18\\
338.01	0\\
339.01	0\\
340.01	1.73472347597681e-18\\
341.01	1.73472347597681e-18\\
342.01	1.73472347597681e-18\\
343.01	1.73472347597681e-18\\
344.01	0\\
345.01	0\\
346.01	0\\
347.01	1.73472347597681e-18\\
348.01	0\\
349.01	1.73472347597681e-18\\
350.01	0\\
351.01	0\\
352.01	0\\
353.01	1.73472347597681e-18\\
354.01	1.73472347597681e-18\\
355.01	0\\
356.01	0\\
357.01	1.73472347597681e-18\\
358.01	1.73472347597681e-18\\
359.01	1.73472347597681e-18\\
360.01	0\\
361.01	0\\
362.01	0\\
363.01	0\\
364.01	0\\
365.01	1.73472347597681e-18\\
366.01	1.73472347597681e-18\\
367.01	1.73472347597681e-18\\
368.01	1.73472347597681e-18\\
369.01	0\\
370.01	0\\
371.01	1.73472347597681e-18\\
372.01	1.73472347597681e-18\\
373.01	1.73472347597681e-18\\
374.01	0\\
375.01	1.73472347597681e-18\\
376.01	0\\
377.01	0\\
378.01	1.73472347597681e-18\\
379.01	0\\
380.01	1.73472347597681e-18\\
381.01	1.73472347597681e-18\\
382.01	1.73472347597681e-18\\
383.01	0\\
384.01	1.73472347597681e-18\\
385.01	0\\
386.01	1.73472347597681e-18\\
387.01	1.73472347597681e-18\\
388.01	1.73472347597681e-18\\
389.01	0\\
390.01	1.73472347597681e-18\\
391.01	1.73472347597681e-18\\
392.01	0\\
393.01	0\\
394.01	1.73472347597681e-18\\
395.01	1.73472347597681e-18\\
396.01	1.73472347597681e-18\\
397.01	1.73472347597681e-18\\
398.01	0\\
399.01	1.73472347597681e-18\\
400.01	0\\
401.01	0\\
402.01	0\\
403.01	1.73472347597681e-18\\
404.01	1.73472347597681e-18\\
405.01	1.73472347597681e-18\\
406.01	1.73472347597681e-18\\
407.01	1.73472347597681e-18\\
408.01	1.73472347597681e-18\\
409.01	0\\
410.01	1.73472347597681e-18\\
411.01	0\\
412.01	1.73472347597681e-18\\
413.01	1.73472347597681e-18\\
414.01	0\\
415.01	0\\
416.01	1.73472347597681e-18\\
417.01	1.73472347597681e-18\\
418.01	1.73472347597681e-18\\
419.01	1.73472347597681e-18\\
420.01	0\\
421.01	1.73472347597681e-18\\
422.01	1.73472347597681e-18\\
423.01	1.73472347597681e-18\\
424.01	1.73472347597681e-18\\
425.01	0\\
426.01	1.73472347597681e-18\\
427.01	1.73472347597681e-18\\
428.01	1.73472347597681e-18\\
429.01	0\\
430.01	1.73472347597681e-18\\
431.01	1.73472347597681e-18\\
432.01	1.73472347597681e-18\\
433.01	0\\
434.01	1.73472347597681e-18\\
435.01	1.73472347597681e-18\\
436.01	1.73472347597681e-18\\
437.01	1.73472347597681e-18\\
438.01	1.73472347597681e-18\\
439.01	0\\
440.01	1.73472347597681e-18\\
441.01	1.73472347597681e-18\\
442.01	0\\
443.01	0\\
444.01	0\\
445.01	1.73472347597681e-18\\
446.01	1.73472347597681e-18\\
447.01	0\\
448.01	1.73472347597681e-18\\
449.01	1.73472347597681e-18\\
450.01	0\\
451.01	0\\
452.01	0\\
453.01	1.73472347597681e-18\\
454.01	1.73472347597681e-18\\
455.01	0\\
456.01	1.73472347597681e-18\\
457.01	1.73472347597681e-18\\
458.01	0\\
459.01	1.73472347597681e-18\\
460.01	0\\
461.01	1.73472347597681e-18\\
462.01	0\\
463.01	0\\
464.01	0\\
465.01	1.73472347597681e-18\\
466.01	0\\
467.01	1.73472347597681e-18\\
468.01	0\\
469.01	1.73472347597681e-18\\
470.01	1.73472347597681e-18\\
471.01	1.73472347597681e-18\\
472.01	0\\
473.01	0\\
474.01	0\\
475.01	1.73472347597681e-18\\
476.01	0\\
477.01	0\\
478.01	1.73472347597681e-18\\
479.01	1.73472347597681e-18\\
480.01	0\\
481.01	0\\
482.01	1.73472347597681e-18\\
483.01	1.73472347597681e-18\\
484.01	0\\
485.01	0\\
486.01	0\\
487.01	1.73472347597681e-18\\
488.01	1.73472347597681e-18\\
489.01	1.73472347597681e-18\\
490.01	1.73472347597681e-18\\
491.01	0\\
492.01	1.73472347597681e-18\\
493.01	1.73472347597681e-18\\
494.01	0\\
495.01	1.73472347597681e-18\\
496.01	0\\
497.01	1.73472347597681e-18\\
498.01	1.73472347597681e-18\\
499.01	0\\
500.01	1.73472347597681e-18\\
501.01	1.73472347597681e-18\\
502.01	0\\
503.01	1.73472347597681e-18\\
504.01	1.73472347597681e-18\\
505.01	1.73472347597681e-18\\
506.01	0\\
507.01	0\\
508.01	0\\
509.01	0\\
510.01	1.73472347597681e-18\\
511.01	1.73472347597681e-18\\
512.01	0\\
513.01	1.73472347597681e-18\\
514.01	1.73472347597681e-18\\
515.01	0\\
516.01	0\\
517.01	1.73472347597681e-18\\
518.01	1.73472347597681e-18\\
519.01	1.73472347597681e-18\\
520.01	0\\
521.01	0\\
522.01	0\\
523.01	1.73472347597681e-18\\
524.01	0\\
525.01	1.73472347597681e-18\\
526.01	1.73472347597681e-18\\
527.01	0\\
528.01	0\\
529.01	0\\
530.01	1.73472347597681e-18\\
531.01	0\\
532.01	1.73472347597681e-18\\
533.01	0\\
534.01	1.73472347597681e-18\\
535.01	0\\
536.01	1.73472347597681e-18\\
537.01	1.73472347597681e-18\\
538.01	1.73472347597681e-18\\
539.01	0\\
540.01	0\\
541.01	1.73472347597681e-18\\
542.01	1.73472347597681e-18\\
543.01	0\\
544.01	1.73472347597681e-18\\
545.01	0\\
546.01	1.73472347597681e-18\\
547.01	0\\
548.01	0\\
549.01	0\\
550.01	0\\
551.01	0\\
552.01	0\\
553.01	0\\
554.01	0\\
555.01	1.73472347597681e-18\\
556.01	0\\
557.01	0\\
558.01	0\\
559.01	1.73472347597681e-18\\
560.01	1.73472347597681e-18\\
561.01	1.73472347597681e-18\\
562.01	1.73472347597681e-18\\
563.01	0\\
564.01	0\\
565.01	1.73472347597681e-18\\
566.01	0\\
567.01	0\\
568.01	0\\
569.01	0\\
570.01	0\\
571.01	0\\
572.01	1.73472347597681e-18\\
573.01	0\\
574.01	1.73472347597681e-18\\
575.01	0\\
576.01	1.73472347597681e-18\\
577.01	0\\
578.01	1.73472347597681e-18\\
579.01	0\\
580.01	0\\
581.01	1.73472347597681e-18\\
582.01	0\\
583.01	0\\
584.01	0\\
585.01	0\\
586.01	0\\
587.01	0\\
588.01	0\\
589.01	0\\
590.01	0\\
591.01	0\\
592.01	1.73472347597681e-18\\
593.01	0\\
594.01	0\\
595.01	0\\
596.01	0\\
597.01	0\\
598.01	0\\
599.01	0\\
599.02	0\\
599.03	1.73472347597681e-18\\
599.04	1.73472347597681e-18\\
599.05	0\\
599.06	0\\
599.07	1.73472347597681e-18\\
599.08	0\\
599.09	0\\
599.1	0\\
599.11	1.73472347597681e-18\\
599.12	1.73472347597681e-18\\
599.13	0\\
599.14	0\\
599.15	1.73472347597681e-18\\
599.16	1.73472347597681e-18\\
599.17	1.73472347597681e-18\\
599.18	0\\
599.19	0\\
599.2	0\\
599.21	0\\
599.22	0\\
599.23	1.73472347597681e-18\\
599.24	1.73472347597681e-18\\
599.25	1.73472347597681e-18\\
599.26	1.73472347597681e-18\\
599.27	0\\
599.28	0\\
599.29	0\\
599.3	0\\
599.31	0\\
599.32	1.73472347597681e-18\\
599.33	1.73472347597681e-18\\
599.34	0\\
599.35	0\\
599.36	0\\
599.37	1.73472347597681e-18\\
599.38	0\\
599.39	0\\
599.4	0\\
599.41	0\\
599.42	0\\
599.43	0\\
599.44	0\\
599.45	0\\
599.46	1.73472347597681e-18\\
599.47	1.73472347597681e-18\\
599.48	1.73472347597681e-18\\
599.49	1.73472347597681e-18\\
599.5	1.73472347597681e-18\\
599.51	0\\
599.52	0\\
599.53	1.73472347597681e-18\\
599.54	0\\
599.55	0\\
599.56	0\\
599.57	1.73472347597681e-18\\
599.58	0\\
599.59	0\\
599.6	0\\
599.61	0\\
599.62	1.73472347597681e-18\\
599.63	0\\
599.64	1.73472347597681e-18\\
599.65	0\\
599.66	0\\
599.67	0\\
599.68	0\\
599.69	1.73472347597681e-18\\
599.7	0\\
599.71	1.73472347597681e-18\\
599.72	1.73472347597681e-18\\
599.73	1.73472347597681e-18\\
599.74	0\\
599.75	0\\
599.76	0\\
599.77	0\\
599.78	0\\
599.79	0\\
599.8	0\\
599.81	0\\
599.82	0\\
599.83	1.73472347597681e-18\\
599.84	0\\
599.85	0\\
599.86	0\\
599.87	0\\
599.88	0\\
599.89	0\\
599.9	0\\
599.91	0\\
599.92	0\\
599.93	0\\
599.94	0\\
599.95	0\\
599.96	0\\
599.97	0\\
599.98	0\\
599.99	0\\
600	0\\
};
\addplot [color=blue!25!mycolor7,solid,forget plot]
  table[row sep=crcr]{%
0.01	1.73472347597681e-18\\
1.01	1.73472347597681e-18\\
2.01	0\\
3.01	1.73472347597681e-18\\
4.01	1.73472347597681e-18\\
5.01	0\\
6.01	0\\
7.01	1.73472347597681e-18\\
8.01	0\\
9.01	0\\
10.01	0\\
11.01	0\\
12.01	1.73472347597681e-18\\
13.01	1.73472347597681e-18\\
14.01	1.73472347597681e-18\\
15.01	1.73472347597681e-18\\
16.01	1.73472347597681e-18\\
17.01	1.73472347597681e-18\\
18.01	0\\
19.01	0\\
20.01	1.73472347597681e-18\\
21.01	0\\
22.01	1.73472347597681e-18\\
23.01	1.73472347597681e-18\\
24.01	1.73472347597681e-18\\
25.01	0\\
26.01	1.73472347597681e-18\\
27.01	0\\
28.01	1.73472347597681e-18\\
29.01	1.73472347597681e-18\\
30.01	0\\
31.01	0\\
32.01	1.73472347597681e-18\\
33.01	1.73472347597681e-18\\
34.01	1.73472347597681e-18\\
35.01	0\\
36.01	0\\
37.01	0\\
38.01	0\\
39.01	1.73472347597681e-18\\
40.01	1.73472347597681e-18\\
41.01	1.73472347597681e-18\\
42.01	1.73472347597681e-18\\
43.01	1.73472347597681e-18\\
44.01	0\\
45.01	1.73472347597681e-18\\
46.01	1.73472347597681e-18\\
47.01	1.73472347597681e-18\\
48.01	1.73472347597681e-18\\
49.01	0\\
50.01	1.73472347597681e-18\\
51.01	1.73472347597681e-18\\
52.01	0\\
53.01	0\\
54.01	0\\
55.01	1.73472347597681e-18\\
56.01	0\\
57.01	1.73472347597681e-18\\
58.01	0\\
59.01	0\\
60.01	1.73472347597681e-18\\
61.01	1.73472347597681e-18\\
62.01	1.73472347597681e-18\\
63.01	0\\
64.01	1.73472347597681e-18\\
65.01	0\\
66.01	1.73472347597681e-18\\
67.01	1.73472347597681e-18\\
68.01	0\\
69.01	1.73472347597681e-18\\
70.01	1.73472347597681e-18\\
71.01	0\\
72.01	0\\
73.01	1.73472347597681e-18\\
74.01	0\\
75.01	1.73472347597681e-18\\
76.01	0\\
77.01	0\\
78.01	0\\
79.01	0\\
80.01	1.73472347597681e-18\\
81.01	1.73472347597681e-18\\
82.01	0\\
83.01	0\\
84.01	0\\
85.01	1.73472347597681e-18\\
86.01	1.73472347597681e-18\\
87.01	0\\
88.01	0\\
89.01	0\\
90.01	1.73472347597681e-18\\
91.01	1.73472347597681e-18\\
92.01	0\\
93.01	1.73472347597681e-18\\
94.01	1.73472347597681e-18\\
95.01	0\\
96.01	1.73472347597681e-18\\
97.01	0\\
98.01	0\\
99.01	0\\
100.01	1.73472347597681e-18\\
101.01	0\\
102.01	1.73472347597681e-18\\
103.01	1.73472347597681e-18\\
104.01	0\\
105.01	0\\
106.01	0\\
107.01	0\\
108.01	1.73472347597681e-18\\
109.01	0\\
110.01	1.73472347597681e-18\\
111.01	0\\
112.01	0\\
113.01	0\\
114.01	1.73472347597681e-18\\
115.01	1.73472347597681e-18\\
116.01	0\\
117.01	0\\
118.01	0\\
119.01	0\\
120.01	0\\
121.01	0\\
122.01	0\\
123.01	1.73472347597681e-18\\
124.01	1.73472347597681e-18\\
125.01	1.73472347597681e-18\\
126.01	1.73472347597681e-18\\
127.01	0\\
128.01	1.73472347597681e-18\\
129.01	0\\
130.01	1.73472347597681e-18\\
131.01	1.73472347597681e-18\\
132.01	0\\
133.01	1.73472347597681e-18\\
134.01	1.73472347597681e-18\\
135.01	0\\
136.01	0\\
137.01	1.73472347597681e-18\\
138.01	0\\
139.01	1.73472347597681e-18\\
140.01	1.73472347597681e-18\\
141.01	1.73472347597681e-18\\
142.01	1.73472347597681e-18\\
143.01	0\\
144.01	1.73472347597681e-18\\
145.01	1.73472347597681e-18\\
146.01	1.73472347597681e-18\\
147.01	1.73472347597681e-18\\
148.01	1.73472347597681e-18\\
149.01	0\\
150.01	0\\
151.01	1.73472347597681e-18\\
152.01	1.73472347597681e-18\\
153.01	1.73472347597681e-18\\
154.01	1.73472347597681e-18\\
155.01	1.73472347597681e-18\\
156.01	1.73472347597681e-18\\
157.01	0\\
158.01	1.73472347597681e-18\\
159.01	1.73472347597681e-18\\
160.01	1.73472347597681e-18\\
161.01	1.73472347597681e-18\\
162.01	1.73472347597681e-18\\
163.01	1.73472347597681e-18\\
164.01	0\\
165.01	0\\
166.01	1.73472347597681e-18\\
167.01	0\\
168.01	1.73472347597681e-18\\
169.01	0\\
170.01	1.73472347597681e-18\\
171.01	0\\
172.01	0\\
173.01	0\\
174.01	1.73472347597681e-18\\
175.01	1.73472347597681e-18\\
176.01	1.73472347597681e-18\\
177.01	0\\
178.01	1.73472347597681e-18\\
179.01	0\\
180.01	0\\
181.01	1.73472347597681e-18\\
182.01	0\\
183.01	0\\
184.01	0\\
185.01	1.73472347597681e-18\\
186.01	0\\
187.01	0\\
188.01	0\\
189.01	0\\
190.01	1.73472347597681e-18\\
191.01	0\\
192.01	1.73472347597681e-18\\
193.01	0\\
194.01	1.73472347597681e-18\\
195.01	0\\
196.01	1.73472347597681e-18\\
197.01	0\\
198.01	0\\
199.01	0\\
200.01	0\\
201.01	0\\
202.01	1.73472347597681e-18\\
203.01	0\\
204.01	1.73472347597681e-18\\
205.01	1.73472347597681e-18\\
206.01	1.73472347597681e-18\\
207.01	1.73472347597681e-18\\
208.01	0\\
209.01	1.73472347597681e-18\\
210.01	0\\
211.01	1.73472347597681e-18\\
212.01	0\\
213.01	1.73472347597681e-18\\
214.01	0\\
215.01	0\\
216.01	0\\
217.01	0\\
218.01	1.73472347597681e-18\\
219.01	1.73472347597681e-18\\
220.01	0\\
221.01	0\\
222.01	0\\
223.01	1.73472347597681e-18\\
224.01	1.73472347597681e-18\\
225.01	0\\
226.01	1.73472347597681e-18\\
227.01	0\\
228.01	0\\
229.01	1.73472347597681e-18\\
230.01	0\\
231.01	0\\
232.01	0\\
233.01	0\\
234.01	1.73472347597681e-18\\
235.01	0\\
236.01	1.73472347597681e-18\\
237.01	1.73472347597681e-18\\
238.01	0\\
239.01	1.73472347597681e-18\\
240.01	0\\
241.01	1.73472347597681e-18\\
242.01	0\\
243.01	1.73472347597681e-18\\
244.01	1.73472347597681e-18\\
245.01	1.73472347597681e-18\\
246.01	1.73472347597681e-18\\
247.01	0\\
248.01	0\\
249.01	0\\
250.01	1.73472347597681e-18\\
251.01	0\\
252.01	1.73472347597681e-18\\
253.01	1.73472347597681e-18\\
254.01	1.73472347597681e-18\\
255.01	1.73472347597681e-18\\
256.01	1.73472347597681e-18\\
257.01	1.73472347597681e-18\\
258.01	0\\
259.01	0\\
260.01	0\\
261.01	0\\
262.01	1.73472347597681e-18\\
263.01	1.73472347597681e-18\\
264.01	0\\
265.01	1.73472347597681e-18\\
266.01	1.73472347597681e-18\\
267.01	0\\
268.01	0\\
269.01	1.73472347597681e-18\\
270.01	0\\
271.01	1.73472347597681e-18\\
272.01	0\\
273.01	1.73472347597681e-18\\
274.01	0\\
275.01	0\\
276.01	1.73472347597681e-18\\
277.01	1.73472347597681e-18\\
278.01	0\\
279.01	1.73472347597681e-18\\
280.01	0\\
281.01	1.73472347597681e-18\\
282.01	1.73472347597681e-18\\
283.01	1.73472347597681e-18\\
284.01	0\\
285.01	1.73472347597681e-18\\
286.01	1.73472347597681e-18\\
287.01	1.73472347597681e-18\\
288.01	1.73472347597681e-18\\
289.01	0\\
290.01	0\\
291.01	0\\
292.01	0\\
293.01	1.73472347597681e-18\\
294.01	1.73472347597681e-18\\
295.01	0\\
296.01	1.73472347597681e-18\\
297.01	0\\
298.01	1.73472347597681e-18\\
299.01	0\\
300.01	1.73472347597681e-18\\
301.01	0\\
302.01	0\\
303.01	0\\
304.01	1.73472347597681e-18\\
305.01	1.73472347597681e-18\\
306.01	0\\
307.01	1.73472347597681e-18\\
308.01	0\\
309.01	1.73472347597681e-18\\
310.01	1.73472347597681e-18\\
311.01	0\\
312.01	1.73472347597681e-18\\
313.01	1.73472347597681e-18\\
314.01	0\\
315.01	1.73472347597681e-18\\
316.01	0\\
317.01	1.73472347597681e-18\\
318.01	0\\
319.01	0\\
320.01	0\\
321.01	0\\
322.01	1.73472347597681e-18\\
323.01	1.73472347597681e-18\\
324.01	0\\
325.01	1.73472347597681e-18\\
326.01	1.73472347597681e-18\\
327.01	0\\
328.01	1.73472347597681e-18\\
329.01	0\\
330.01	0\\
331.01	1.73472347597681e-18\\
332.01	1.73472347597681e-18\\
333.01	1.73472347597681e-18\\
334.01	1.73472347597681e-18\\
335.01	1.73472347597681e-18\\
336.01	0\\
337.01	1.73472347597681e-18\\
338.01	0\\
339.01	0\\
340.01	1.73472347597681e-18\\
341.01	1.73472347597681e-18\\
342.01	1.73472347597681e-18\\
343.01	1.73472347597681e-18\\
344.01	0\\
345.01	0\\
346.01	0\\
347.01	1.73472347597681e-18\\
348.01	0\\
349.01	1.73472347597681e-18\\
350.01	0\\
351.01	0\\
352.01	0\\
353.01	1.73472347597681e-18\\
354.01	1.73472347597681e-18\\
355.01	0\\
356.01	0\\
357.01	1.73472347597681e-18\\
358.01	1.73472347597681e-18\\
359.01	1.73472347597681e-18\\
360.01	0\\
361.01	0\\
362.01	0\\
363.01	0\\
364.01	0\\
365.01	1.73472347597681e-18\\
366.01	1.73472347597681e-18\\
367.01	1.73472347597681e-18\\
368.01	1.73472347597681e-18\\
369.01	0\\
370.01	0\\
371.01	1.73472347597681e-18\\
372.01	1.73472347597681e-18\\
373.01	1.73472347597681e-18\\
374.01	0\\
375.01	1.73472347597681e-18\\
376.01	0\\
377.01	0\\
378.01	1.73472347597681e-18\\
379.01	0\\
380.01	1.73472347597681e-18\\
381.01	1.73472347597681e-18\\
382.01	1.73472347597681e-18\\
383.01	0\\
384.01	1.73472347597681e-18\\
385.01	0\\
386.01	1.73472347597681e-18\\
387.01	1.73472347597681e-18\\
388.01	1.73472347597681e-18\\
389.01	0\\
390.01	1.73472347597681e-18\\
391.01	1.73472347597681e-18\\
392.01	0\\
393.01	0\\
394.01	1.73472347597681e-18\\
395.01	1.73472347597681e-18\\
396.01	1.73472347597681e-18\\
397.01	1.73472347597681e-18\\
398.01	0\\
399.01	1.73472347597681e-18\\
400.01	0\\
401.01	0\\
402.01	0\\
403.01	1.73472347597681e-18\\
404.01	1.73472347597681e-18\\
405.01	1.73472347597681e-18\\
406.01	1.73472347597681e-18\\
407.01	1.73472347597681e-18\\
408.01	1.73472347597681e-18\\
409.01	0\\
410.01	1.73472347597681e-18\\
411.01	0\\
412.01	1.73472347597681e-18\\
413.01	1.73472347597681e-18\\
414.01	0\\
415.01	0\\
416.01	1.73472347597681e-18\\
417.01	1.73472347597681e-18\\
418.01	1.73472347597681e-18\\
419.01	1.73472347597681e-18\\
420.01	0\\
421.01	1.73472347597681e-18\\
422.01	1.73472347597681e-18\\
423.01	1.73472347597681e-18\\
424.01	1.73472347597681e-18\\
425.01	0\\
426.01	1.73472347597681e-18\\
427.01	1.73472347597681e-18\\
428.01	1.73472347597681e-18\\
429.01	0\\
430.01	1.73472347597681e-18\\
431.01	1.73472347597681e-18\\
432.01	1.73472347597681e-18\\
433.01	0\\
434.01	1.73472347597681e-18\\
435.01	1.73472347597681e-18\\
436.01	1.73472347597681e-18\\
437.01	1.73472347597681e-18\\
438.01	1.73472347597681e-18\\
439.01	0\\
440.01	1.73472347597681e-18\\
441.01	1.73472347597681e-18\\
442.01	0\\
443.01	0\\
444.01	0\\
445.01	1.73472347597681e-18\\
446.01	1.73472347597681e-18\\
447.01	0\\
448.01	1.73472347597681e-18\\
449.01	1.73472347597681e-18\\
450.01	0\\
451.01	0\\
452.01	0\\
453.01	1.73472347597681e-18\\
454.01	1.73472347597681e-18\\
455.01	0\\
456.01	1.73472347597681e-18\\
457.01	1.73472347597681e-18\\
458.01	0\\
459.01	1.73472347597681e-18\\
460.01	0\\
461.01	1.73472347597681e-18\\
462.01	0\\
463.01	0\\
464.01	0\\
465.01	1.73472347597681e-18\\
466.01	0\\
467.01	1.73472347597681e-18\\
468.01	0\\
469.01	1.73472347597681e-18\\
470.01	1.73472347597681e-18\\
471.01	1.73472347597681e-18\\
472.01	0\\
473.01	0\\
474.01	0\\
475.01	1.73472347597681e-18\\
476.01	0\\
477.01	0\\
478.01	1.73472347597681e-18\\
479.01	1.73472347597681e-18\\
480.01	0\\
481.01	0\\
482.01	1.73472347597681e-18\\
483.01	1.73472347597681e-18\\
484.01	0\\
485.01	0\\
486.01	0\\
487.01	1.73472347597681e-18\\
488.01	1.73472347597681e-18\\
489.01	1.73472347597681e-18\\
490.01	1.73472347597681e-18\\
491.01	0\\
492.01	1.73472347597681e-18\\
493.01	1.73472347597681e-18\\
494.01	0\\
495.01	1.73472347597681e-18\\
496.01	0\\
497.01	1.73472347597681e-18\\
498.01	1.73472347597681e-18\\
499.01	0\\
500.01	1.73472347597681e-18\\
501.01	1.73472347597681e-18\\
502.01	0\\
503.01	1.73472347597681e-18\\
504.01	1.73472347597681e-18\\
505.01	1.73472347597681e-18\\
506.01	0\\
507.01	0\\
508.01	0\\
509.01	0\\
510.01	1.73472347597681e-18\\
511.01	1.73472347597681e-18\\
512.01	0\\
513.01	1.73472347597681e-18\\
514.01	1.73472347597681e-18\\
515.01	0\\
516.01	0\\
517.01	1.73472347597681e-18\\
518.01	1.73472347597681e-18\\
519.01	1.73472347597681e-18\\
520.01	0\\
521.01	0\\
522.01	0\\
523.01	1.73472347597681e-18\\
524.01	0\\
525.01	1.73472347597681e-18\\
526.01	1.73472347597681e-18\\
527.01	0\\
528.01	0\\
529.01	0\\
530.01	1.73472347597681e-18\\
531.01	0\\
532.01	1.73472347597681e-18\\
533.01	0\\
534.01	1.73472347597681e-18\\
535.01	0\\
536.01	1.73472347597681e-18\\
537.01	1.73472347597681e-18\\
538.01	1.73472347597681e-18\\
539.01	0\\
540.01	0\\
541.01	1.73472347597681e-18\\
542.01	1.73472347597681e-18\\
543.01	0\\
544.01	1.73472347597681e-18\\
545.01	0\\
546.01	1.73472347597681e-18\\
547.01	0\\
548.01	0\\
549.01	0\\
550.01	0\\
551.01	0\\
552.01	0\\
553.01	0\\
554.01	0\\
555.01	1.73472347597681e-18\\
556.01	0\\
557.01	0\\
558.01	0\\
559.01	1.73472347597681e-18\\
560.01	1.73472347597681e-18\\
561.01	1.73472347597681e-18\\
562.01	1.73472347597681e-18\\
563.01	0\\
564.01	0\\
565.01	1.73472347597681e-18\\
566.01	0\\
567.01	0\\
568.01	0\\
569.01	0\\
570.01	0\\
571.01	0\\
572.01	1.73472347597681e-18\\
573.01	0\\
574.01	1.73472347597681e-18\\
575.01	0\\
576.01	1.73472347597681e-18\\
577.01	0\\
578.01	1.73472347597681e-18\\
579.01	0\\
580.01	0\\
581.01	1.73472347597681e-18\\
582.01	0\\
583.01	0\\
584.01	0\\
585.01	0\\
586.01	0\\
587.01	0\\
588.01	0\\
589.01	0\\
590.01	0\\
591.01	0\\
592.01	1.73472347597681e-18\\
593.01	0\\
594.01	0\\
595.01	0\\
596.01	0\\
597.01	0\\
598.01	0\\
599.01	0\\
599.02	0\\
599.03	1.73472347597681e-18\\
599.04	1.73472347597681e-18\\
599.05	0\\
599.06	0\\
599.07	1.73472347597681e-18\\
599.08	0\\
599.09	0\\
599.1	0\\
599.11	1.73472347597681e-18\\
599.12	1.73472347597681e-18\\
599.13	0\\
599.14	0\\
599.15	1.73472347597681e-18\\
599.16	1.73472347597681e-18\\
599.17	1.73472347597681e-18\\
599.18	0\\
599.19	0\\
599.2	0\\
599.21	0\\
599.22	0\\
599.23	1.73472347597681e-18\\
599.24	1.73472347597681e-18\\
599.25	1.73472347597681e-18\\
599.26	1.73472347597681e-18\\
599.27	0\\
599.28	0\\
599.29	0\\
599.3	0\\
599.31	0\\
599.32	1.73472347597681e-18\\
599.33	1.73472347597681e-18\\
599.34	0\\
599.35	0\\
599.36	0\\
599.37	1.73472347597681e-18\\
599.38	0\\
599.39	0\\
599.4	0\\
599.41	0\\
599.42	0\\
599.43	0\\
599.44	0\\
599.45	0\\
599.46	1.73472347597681e-18\\
599.47	1.73472347597681e-18\\
599.48	1.73472347597681e-18\\
599.49	1.73472347597681e-18\\
599.5	1.73472347597681e-18\\
599.51	0\\
599.52	0\\
599.53	1.73472347597681e-18\\
599.54	0\\
599.55	0\\
599.56	0\\
599.57	1.73472347597681e-18\\
599.58	0\\
599.59	0\\
599.6	0\\
599.61	0\\
599.62	1.73472347597681e-18\\
599.63	0\\
599.64	1.73472347597681e-18\\
599.65	0\\
599.66	0\\
599.67	0\\
599.68	0\\
599.69	1.73472347597681e-18\\
599.7	0\\
599.71	1.73472347597681e-18\\
599.72	1.73472347597681e-18\\
599.73	1.73472347597681e-18\\
599.74	0\\
599.75	0\\
599.76	0\\
599.77	0\\
599.78	0\\
599.79	0\\
599.8	0\\
599.81	0\\
599.82	0\\
599.83	1.73472347597681e-18\\
599.84	0\\
599.85	0\\
599.86	0\\
599.87	0\\
599.88	0\\
599.89	0\\
599.9	0\\
599.91	0\\
599.92	0\\
599.93	0\\
599.94	0\\
599.95	0\\
599.96	0\\
599.97	0\\
599.98	0\\
599.99	0\\
600	0\\
};
\addplot [color=mycolor9,solid,forget plot]
  table[row sep=crcr]{%
0.01	1.73472347597681e-18\\
1.01	1.73472347597681e-18\\
2.01	0\\
3.01	1.73472347597681e-18\\
4.01	1.73472347597681e-18\\
5.01	0\\
6.01	0\\
7.01	1.73472347597681e-18\\
8.01	0\\
9.01	0\\
10.01	0\\
11.01	0\\
12.01	1.73472347597681e-18\\
13.01	1.73472347597681e-18\\
14.01	1.73472347597681e-18\\
15.01	1.73472347597681e-18\\
16.01	1.73472347597681e-18\\
17.01	1.73472347597681e-18\\
18.01	0\\
19.01	0\\
20.01	1.73472347597681e-18\\
21.01	0\\
22.01	1.73472347597681e-18\\
23.01	1.73472347597681e-18\\
24.01	1.73472347597681e-18\\
25.01	0\\
26.01	1.73472347597681e-18\\
27.01	0\\
28.01	1.73472347597681e-18\\
29.01	1.73472347597681e-18\\
30.01	0\\
31.01	0\\
32.01	1.73472347597681e-18\\
33.01	1.73472347597681e-18\\
34.01	1.73472347597681e-18\\
35.01	0\\
36.01	0\\
37.01	0\\
38.01	0\\
39.01	1.73472347597681e-18\\
40.01	1.73472347597681e-18\\
41.01	1.73472347597681e-18\\
42.01	1.73472347597681e-18\\
43.01	1.73472347597681e-18\\
44.01	0\\
45.01	1.73472347597681e-18\\
46.01	1.73472347597681e-18\\
47.01	1.73472347597681e-18\\
48.01	1.73472347597681e-18\\
49.01	0\\
50.01	1.73472347597681e-18\\
51.01	1.73472347597681e-18\\
52.01	0\\
53.01	0\\
54.01	0\\
55.01	1.73472347597681e-18\\
56.01	0\\
57.01	1.73472347597681e-18\\
58.01	0\\
59.01	0\\
60.01	1.73472347597681e-18\\
61.01	1.73472347597681e-18\\
62.01	1.73472347597681e-18\\
63.01	0\\
64.01	1.73472347597681e-18\\
65.01	0\\
66.01	1.73472347597681e-18\\
67.01	1.73472347597681e-18\\
68.01	0\\
69.01	1.73472347597681e-18\\
70.01	1.73472347597681e-18\\
71.01	0\\
72.01	0\\
73.01	1.73472347597681e-18\\
74.01	0\\
75.01	1.73472347597681e-18\\
76.01	0\\
77.01	0\\
78.01	0\\
79.01	0\\
80.01	1.73472347597681e-18\\
81.01	1.73472347597681e-18\\
82.01	0\\
83.01	0\\
84.01	0\\
85.01	1.73472347597681e-18\\
86.01	1.73472347597681e-18\\
87.01	0\\
88.01	0\\
89.01	0\\
90.01	1.73472347597681e-18\\
91.01	1.73472347597681e-18\\
92.01	0\\
93.01	1.73472347597681e-18\\
94.01	1.73472347597681e-18\\
95.01	0\\
96.01	1.73472347597681e-18\\
97.01	0\\
98.01	0\\
99.01	0\\
100.01	1.73472347597681e-18\\
101.01	0\\
102.01	1.73472347597681e-18\\
103.01	1.73472347597681e-18\\
104.01	0\\
105.01	0\\
106.01	0\\
107.01	0\\
108.01	1.73472347597681e-18\\
109.01	0\\
110.01	1.73472347597681e-18\\
111.01	0\\
112.01	0\\
113.01	0\\
114.01	1.73472347597681e-18\\
115.01	1.73472347597681e-18\\
116.01	0\\
117.01	0\\
118.01	0\\
119.01	0\\
120.01	0\\
121.01	0\\
122.01	0\\
123.01	1.73472347597681e-18\\
124.01	1.73472347597681e-18\\
125.01	1.73472347597681e-18\\
126.01	1.73472347597681e-18\\
127.01	0\\
128.01	1.73472347597681e-18\\
129.01	0\\
130.01	1.73472347597681e-18\\
131.01	1.73472347597681e-18\\
132.01	0\\
133.01	1.73472347597681e-18\\
134.01	1.73472347597681e-18\\
135.01	0\\
136.01	0\\
137.01	1.73472347597681e-18\\
138.01	0\\
139.01	1.73472347597681e-18\\
140.01	1.73472347597681e-18\\
141.01	1.73472347597681e-18\\
142.01	1.73472347597681e-18\\
143.01	0\\
144.01	1.73472347597681e-18\\
145.01	1.73472347597681e-18\\
146.01	1.73472347597681e-18\\
147.01	1.73472347597681e-18\\
148.01	1.73472347597681e-18\\
149.01	0\\
150.01	0\\
151.01	1.73472347597681e-18\\
152.01	1.73472347597681e-18\\
153.01	1.73472347597681e-18\\
154.01	1.73472347597681e-18\\
155.01	1.73472347597681e-18\\
156.01	1.73472347597681e-18\\
157.01	0\\
158.01	1.73472347597681e-18\\
159.01	1.73472347597681e-18\\
160.01	1.73472347597681e-18\\
161.01	1.73472347597681e-18\\
162.01	1.73472347597681e-18\\
163.01	1.73472347597681e-18\\
164.01	0\\
165.01	0\\
166.01	1.73472347597681e-18\\
167.01	0\\
168.01	1.73472347597681e-18\\
169.01	0\\
170.01	1.73472347597681e-18\\
171.01	0\\
172.01	0\\
173.01	0\\
174.01	1.73472347597681e-18\\
175.01	1.73472347597681e-18\\
176.01	1.73472347597681e-18\\
177.01	0\\
178.01	1.73472347597681e-18\\
179.01	0\\
180.01	0\\
181.01	1.73472347597681e-18\\
182.01	0\\
183.01	0\\
184.01	0\\
185.01	1.73472347597681e-18\\
186.01	0\\
187.01	0\\
188.01	0\\
189.01	0\\
190.01	1.73472347597681e-18\\
191.01	0\\
192.01	1.73472347597681e-18\\
193.01	0\\
194.01	1.73472347597681e-18\\
195.01	0\\
196.01	1.73472347597681e-18\\
197.01	0\\
198.01	0\\
199.01	0\\
200.01	0\\
201.01	0\\
202.01	1.73472347597681e-18\\
203.01	0\\
204.01	1.73472347597681e-18\\
205.01	1.73472347597681e-18\\
206.01	1.73472347597681e-18\\
207.01	1.73472347597681e-18\\
208.01	0\\
209.01	1.73472347597681e-18\\
210.01	0\\
211.01	1.73472347597681e-18\\
212.01	0\\
213.01	1.73472347597681e-18\\
214.01	0\\
215.01	0\\
216.01	0\\
217.01	0\\
218.01	1.73472347597681e-18\\
219.01	1.73472347597681e-18\\
220.01	0\\
221.01	0\\
222.01	0\\
223.01	1.73472347597681e-18\\
224.01	1.73472347597681e-18\\
225.01	0\\
226.01	1.73472347597681e-18\\
227.01	0\\
228.01	0\\
229.01	1.73472347597681e-18\\
230.01	0\\
231.01	0\\
232.01	0\\
233.01	0\\
234.01	1.73472347597681e-18\\
235.01	0\\
236.01	1.73472347597681e-18\\
237.01	1.73472347597681e-18\\
238.01	0\\
239.01	1.73472347597681e-18\\
240.01	0\\
241.01	1.73472347597681e-18\\
242.01	0\\
243.01	1.73472347597681e-18\\
244.01	1.73472347597681e-18\\
245.01	1.73472347597681e-18\\
246.01	1.73472347597681e-18\\
247.01	0\\
248.01	0\\
249.01	0\\
250.01	1.73472347597681e-18\\
251.01	0\\
252.01	1.73472347597681e-18\\
253.01	1.73472347597681e-18\\
254.01	1.73472347597681e-18\\
255.01	1.73472347597681e-18\\
256.01	1.73472347597681e-18\\
257.01	1.73472347597681e-18\\
258.01	0\\
259.01	0\\
260.01	0\\
261.01	0\\
262.01	1.73472347597681e-18\\
263.01	1.73472347597681e-18\\
264.01	0\\
265.01	1.73472347597681e-18\\
266.01	1.73472347597681e-18\\
267.01	0\\
268.01	0\\
269.01	1.73472347597681e-18\\
270.01	0\\
271.01	1.73472347597681e-18\\
272.01	0\\
273.01	1.73472347597681e-18\\
274.01	0\\
275.01	0\\
276.01	1.73472347597681e-18\\
277.01	1.73472347597681e-18\\
278.01	0\\
279.01	1.73472347597681e-18\\
280.01	0\\
281.01	1.73472347597681e-18\\
282.01	1.73472347597681e-18\\
283.01	1.73472347597681e-18\\
284.01	0\\
285.01	1.73472347597681e-18\\
286.01	1.73472347597681e-18\\
287.01	1.73472347597681e-18\\
288.01	1.73472347597681e-18\\
289.01	0\\
290.01	0\\
291.01	0\\
292.01	0\\
293.01	1.73472347597681e-18\\
294.01	1.73472347597681e-18\\
295.01	0\\
296.01	1.73472347597681e-18\\
297.01	0\\
298.01	1.73472347597681e-18\\
299.01	0\\
300.01	1.73472347597681e-18\\
301.01	0\\
302.01	0\\
303.01	0\\
304.01	1.73472347597681e-18\\
305.01	1.73472347597681e-18\\
306.01	0\\
307.01	1.73472347597681e-18\\
308.01	0\\
309.01	1.73472347597681e-18\\
310.01	1.73472347597681e-18\\
311.01	0\\
312.01	1.73472347597681e-18\\
313.01	1.73472347597681e-18\\
314.01	0\\
315.01	1.73472347597681e-18\\
316.01	0\\
317.01	1.73472347597681e-18\\
318.01	0\\
319.01	0\\
320.01	0\\
321.01	0\\
322.01	1.73472347597681e-18\\
323.01	1.73472347597681e-18\\
324.01	0\\
325.01	1.73472347597681e-18\\
326.01	1.73472347597681e-18\\
327.01	0\\
328.01	1.73472347597681e-18\\
329.01	0\\
330.01	0\\
331.01	1.73472347597681e-18\\
332.01	1.73472347597681e-18\\
333.01	1.73472347597681e-18\\
334.01	1.73472347597681e-18\\
335.01	1.73472347597681e-18\\
336.01	0\\
337.01	1.73472347597681e-18\\
338.01	0\\
339.01	0\\
340.01	1.73472347597681e-18\\
341.01	1.73472347597681e-18\\
342.01	1.73472347597681e-18\\
343.01	1.73472347597681e-18\\
344.01	0\\
345.01	0\\
346.01	0\\
347.01	1.73472347597681e-18\\
348.01	0\\
349.01	1.73472347597681e-18\\
350.01	0\\
351.01	0\\
352.01	0\\
353.01	1.73472347597681e-18\\
354.01	1.73472347597681e-18\\
355.01	0\\
356.01	0\\
357.01	1.73472347597681e-18\\
358.01	1.73472347597681e-18\\
359.01	1.73472347597681e-18\\
360.01	0\\
361.01	0\\
362.01	0\\
363.01	0\\
364.01	0\\
365.01	1.73472347597681e-18\\
366.01	1.73472347597681e-18\\
367.01	1.73472347597681e-18\\
368.01	1.73472347597681e-18\\
369.01	0\\
370.01	0\\
371.01	1.73472347597681e-18\\
372.01	1.73472347597681e-18\\
373.01	1.73472347597681e-18\\
374.01	0\\
375.01	1.73472347597681e-18\\
376.01	0\\
377.01	0\\
378.01	1.73472347597681e-18\\
379.01	0\\
380.01	1.73472347597681e-18\\
381.01	1.73472347597681e-18\\
382.01	1.73472347597681e-18\\
383.01	0\\
384.01	1.73472347597681e-18\\
385.01	0\\
386.01	1.73472347597681e-18\\
387.01	1.73472347597681e-18\\
388.01	1.73472347597681e-18\\
389.01	0\\
390.01	1.73472347597681e-18\\
391.01	1.73472347597681e-18\\
392.01	0\\
393.01	0\\
394.01	1.73472347597681e-18\\
395.01	1.73472347597681e-18\\
396.01	1.73472347597681e-18\\
397.01	1.73472347597681e-18\\
398.01	0\\
399.01	1.73472347597681e-18\\
400.01	0\\
401.01	0\\
402.01	0\\
403.01	1.73472347597681e-18\\
404.01	1.73472347597681e-18\\
405.01	1.73472347597681e-18\\
406.01	1.73472347597681e-18\\
407.01	1.73472347597681e-18\\
408.01	1.73472347597681e-18\\
409.01	0\\
410.01	1.73472347597681e-18\\
411.01	0\\
412.01	1.73472347597681e-18\\
413.01	1.73472347597681e-18\\
414.01	0\\
415.01	0\\
416.01	1.73472347597681e-18\\
417.01	1.73472347597681e-18\\
418.01	1.73472347597681e-18\\
419.01	1.73472347597681e-18\\
420.01	0\\
421.01	1.73472347597681e-18\\
422.01	1.73472347597681e-18\\
423.01	1.73472347597681e-18\\
424.01	1.73472347597681e-18\\
425.01	0\\
426.01	1.73472347597681e-18\\
427.01	1.73472347597681e-18\\
428.01	1.73472347597681e-18\\
429.01	0\\
430.01	1.73472347597681e-18\\
431.01	1.73472347597681e-18\\
432.01	1.73472347597681e-18\\
433.01	0\\
434.01	1.73472347597681e-18\\
435.01	1.73472347597681e-18\\
436.01	1.73472347597681e-18\\
437.01	1.73472347597681e-18\\
438.01	1.73472347597681e-18\\
439.01	0\\
440.01	1.73472347597681e-18\\
441.01	1.73472347597681e-18\\
442.01	0\\
443.01	0\\
444.01	0\\
445.01	1.73472347597681e-18\\
446.01	1.73472347597681e-18\\
447.01	0\\
448.01	1.73472347597681e-18\\
449.01	1.73472347597681e-18\\
450.01	0\\
451.01	0\\
452.01	0\\
453.01	1.73472347597681e-18\\
454.01	1.73472347597681e-18\\
455.01	0\\
456.01	1.73472347597681e-18\\
457.01	1.73472347597681e-18\\
458.01	0\\
459.01	1.73472347597681e-18\\
460.01	0\\
461.01	1.73472347597681e-18\\
462.01	0\\
463.01	0\\
464.01	0\\
465.01	1.73472347597681e-18\\
466.01	0\\
467.01	1.73472347597681e-18\\
468.01	0\\
469.01	1.73472347597681e-18\\
470.01	1.73472347597681e-18\\
471.01	1.73472347597681e-18\\
472.01	0\\
473.01	0\\
474.01	0\\
475.01	1.73472347597681e-18\\
476.01	0\\
477.01	0\\
478.01	1.73472347597681e-18\\
479.01	1.73472347597681e-18\\
480.01	0\\
481.01	0\\
482.01	1.73472347597681e-18\\
483.01	1.73472347597681e-18\\
484.01	0\\
485.01	0\\
486.01	0\\
487.01	1.73472347597681e-18\\
488.01	1.73472347597681e-18\\
489.01	1.73472347597681e-18\\
490.01	1.73472347597681e-18\\
491.01	0\\
492.01	1.73472347597681e-18\\
493.01	1.73472347597681e-18\\
494.01	0\\
495.01	1.73472347597681e-18\\
496.01	0\\
497.01	1.73472347597681e-18\\
498.01	1.73472347597681e-18\\
499.01	0\\
500.01	1.73472347597681e-18\\
501.01	1.73472347597681e-18\\
502.01	0\\
503.01	1.73472347597681e-18\\
504.01	1.73472347597681e-18\\
505.01	1.73472347597681e-18\\
506.01	0\\
507.01	0\\
508.01	0\\
509.01	0\\
510.01	1.73472347597681e-18\\
511.01	1.73472347597681e-18\\
512.01	0\\
513.01	1.73472347597681e-18\\
514.01	1.73472347597681e-18\\
515.01	0\\
516.01	0\\
517.01	1.73472347597681e-18\\
518.01	1.73472347597681e-18\\
519.01	1.73472347597681e-18\\
520.01	0\\
521.01	0\\
522.01	0\\
523.01	1.73472347597681e-18\\
524.01	0\\
525.01	1.73472347597681e-18\\
526.01	1.73472347597681e-18\\
527.01	0\\
528.01	0\\
529.01	0\\
530.01	1.73472347597681e-18\\
531.01	0\\
532.01	1.73472347597681e-18\\
533.01	0\\
534.01	1.73472347597681e-18\\
535.01	0\\
536.01	1.73472347597681e-18\\
537.01	1.73472347597681e-18\\
538.01	1.73472347597681e-18\\
539.01	0\\
540.01	0\\
541.01	1.73472347597681e-18\\
542.01	1.73472347597681e-18\\
543.01	0\\
544.01	1.73472347597681e-18\\
545.01	0\\
546.01	1.73472347597681e-18\\
547.01	0\\
548.01	0\\
549.01	0\\
550.01	0\\
551.01	0\\
552.01	0\\
553.01	0\\
554.01	0\\
555.01	1.73472347597681e-18\\
556.01	0\\
557.01	0\\
558.01	0\\
559.01	1.73472347597681e-18\\
560.01	1.73472347597681e-18\\
561.01	1.73472347597681e-18\\
562.01	1.73472347597681e-18\\
563.01	0\\
564.01	0\\
565.01	1.73472347597681e-18\\
566.01	0\\
567.01	0\\
568.01	0\\
569.01	0\\
570.01	0\\
571.01	0\\
572.01	1.73472347597681e-18\\
573.01	0\\
574.01	1.73472347597681e-18\\
575.01	0\\
576.01	1.73472347597681e-18\\
577.01	0\\
578.01	1.73472347597681e-18\\
579.01	0\\
580.01	0\\
581.01	1.73472347597681e-18\\
582.01	0\\
583.01	0\\
584.01	0\\
585.01	0\\
586.01	0\\
587.01	0\\
588.01	0\\
589.01	0\\
590.01	0\\
591.01	0\\
592.01	1.73472347597681e-18\\
593.01	0\\
594.01	0\\
595.01	0\\
596.01	0\\
597.01	0\\
598.01	0\\
599.01	0\\
599.02	0\\
599.03	1.73472347597681e-18\\
599.04	1.73472347597681e-18\\
599.05	0\\
599.06	0\\
599.07	1.73472347597681e-18\\
599.08	0\\
599.09	0\\
599.1	0\\
599.11	1.73472347597681e-18\\
599.12	1.73472347597681e-18\\
599.13	0\\
599.14	0\\
599.15	1.73472347597681e-18\\
599.16	1.73472347597681e-18\\
599.17	1.73472347597681e-18\\
599.18	0\\
599.19	0\\
599.2	0\\
599.21	0\\
599.22	0\\
599.23	1.73472347597681e-18\\
599.24	1.73472347597681e-18\\
599.25	1.73472347597681e-18\\
599.26	1.73472347597681e-18\\
599.27	0\\
599.28	0\\
599.29	0\\
599.3	0\\
599.31	0\\
599.32	1.73472347597681e-18\\
599.33	1.73472347597681e-18\\
599.34	0\\
599.35	0\\
599.36	0\\
599.37	1.73472347597681e-18\\
599.38	0\\
599.39	0\\
599.4	0\\
599.41	0\\
599.42	0\\
599.43	0\\
599.44	0\\
599.45	0\\
599.46	1.73472347597681e-18\\
599.47	1.73472347597681e-18\\
599.48	1.73472347597681e-18\\
599.49	1.73472347597681e-18\\
599.5	1.73472347597681e-18\\
599.51	0\\
599.52	0\\
599.53	1.73472347597681e-18\\
599.54	0\\
599.55	0\\
599.56	0\\
599.57	1.73472347597681e-18\\
599.58	0\\
599.59	0\\
599.6	0\\
599.61	0\\
599.62	1.73472347597681e-18\\
599.63	0\\
599.64	1.73472347597681e-18\\
599.65	0\\
599.66	0\\
599.67	0\\
599.68	0\\
599.69	1.73472347597681e-18\\
599.7	0\\
599.71	1.73472347597681e-18\\
599.72	1.73472347597681e-18\\
599.73	1.73472347597681e-18\\
599.74	0\\
599.75	0\\
599.76	0\\
599.77	0\\
599.78	0\\
599.79	0\\
599.8	0\\
599.81	0\\
599.82	0\\
599.83	1.73472347597681e-18\\
599.84	0\\
599.85	0\\
599.86	0\\
599.87	0\\
599.88	0\\
599.89	0\\
599.9	0\\
599.91	0\\
599.92	0\\
599.93	0\\
599.94	0\\
599.95	0\\
599.96	0\\
599.97	0\\
599.98	0\\
599.99	0\\
600	0\\
};
\addplot [color=blue!50!mycolor7,solid,forget plot]
  table[row sep=crcr]{%
0.01	1.73472347597681e-18\\
1.01	1.73472347597681e-18\\
2.01	0\\
3.01	1.73472347597681e-18\\
4.01	1.73472347597681e-18\\
5.01	0\\
6.01	0\\
7.01	1.73472347597681e-18\\
8.01	0\\
9.01	0\\
10.01	0\\
11.01	0\\
12.01	1.73472347597681e-18\\
13.01	1.73472347597681e-18\\
14.01	1.73472347597681e-18\\
15.01	1.73472347597681e-18\\
16.01	1.73472347597681e-18\\
17.01	1.73472347597681e-18\\
18.01	0\\
19.01	0\\
20.01	1.73472347597681e-18\\
21.01	0\\
22.01	1.73472347597681e-18\\
23.01	1.73472347597681e-18\\
24.01	1.73472347597681e-18\\
25.01	0\\
26.01	1.73472347597681e-18\\
27.01	0\\
28.01	1.73472347597681e-18\\
29.01	1.73472347597681e-18\\
30.01	0\\
31.01	0\\
32.01	1.73472347597681e-18\\
33.01	1.73472347597681e-18\\
34.01	1.73472347597681e-18\\
35.01	0\\
36.01	0\\
37.01	0\\
38.01	0\\
39.01	1.73472347597681e-18\\
40.01	1.73472347597681e-18\\
41.01	1.73472347597681e-18\\
42.01	1.73472347597681e-18\\
43.01	1.73472347597681e-18\\
44.01	0\\
45.01	1.73472347597681e-18\\
46.01	1.73472347597681e-18\\
47.01	1.73472347597681e-18\\
48.01	1.73472347597681e-18\\
49.01	0\\
50.01	1.73472347597681e-18\\
51.01	1.73472347597681e-18\\
52.01	0\\
53.01	0\\
54.01	0\\
55.01	1.73472347597681e-18\\
56.01	0\\
57.01	1.73472347597681e-18\\
58.01	0\\
59.01	0\\
60.01	1.73472347597681e-18\\
61.01	1.73472347597681e-18\\
62.01	1.73472347597681e-18\\
63.01	0\\
64.01	1.73472347597681e-18\\
65.01	0\\
66.01	1.73472347597681e-18\\
67.01	1.73472347597681e-18\\
68.01	0\\
69.01	1.73472347597681e-18\\
70.01	1.73472347597681e-18\\
71.01	0\\
72.01	0\\
73.01	1.73472347597681e-18\\
74.01	0\\
75.01	1.73472347597681e-18\\
76.01	0\\
77.01	0\\
78.01	0\\
79.01	0\\
80.01	1.73472347597681e-18\\
81.01	1.73472347597681e-18\\
82.01	0\\
83.01	0\\
84.01	0\\
85.01	1.73472347597681e-18\\
86.01	1.73472347597681e-18\\
87.01	0\\
88.01	0\\
89.01	0\\
90.01	1.73472347597681e-18\\
91.01	1.73472347597681e-18\\
92.01	0\\
93.01	1.73472347597681e-18\\
94.01	1.73472347597681e-18\\
95.01	0\\
96.01	1.73472347597681e-18\\
97.01	0\\
98.01	0\\
99.01	0\\
100.01	1.73472347597681e-18\\
101.01	0\\
102.01	1.73472347597681e-18\\
103.01	1.73472347597681e-18\\
104.01	0\\
105.01	0\\
106.01	0\\
107.01	0\\
108.01	1.73472347597681e-18\\
109.01	0\\
110.01	1.73472347597681e-18\\
111.01	0\\
112.01	0\\
113.01	0\\
114.01	1.73472347597681e-18\\
115.01	1.73472347597681e-18\\
116.01	0\\
117.01	0\\
118.01	0\\
119.01	0\\
120.01	0\\
121.01	0\\
122.01	0\\
123.01	1.73472347597681e-18\\
124.01	1.73472347597681e-18\\
125.01	1.73472347597681e-18\\
126.01	1.73472347597681e-18\\
127.01	0\\
128.01	1.73472347597681e-18\\
129.01	0\\
130.01	1.73472347597681e-18\\
131.01	1.73472347597681e-18\\
132.01	0\\
133.01	1.73472347597681e-18\\
134.01	1.73472347597681e-18\\
135.01	0\\
136.01	0\\
137.01	1.73472347597681e-18\\
138.01	0\\
139.01	1.73472347597681e-18\\
140.01	1.73472347597681e-18\\
141.01	1.73472347597681e-18\\
142.01	1.73472347597681e-18\\
143.01	0\\
144.01	1.73472347597681e-18\\
145.01	1.73472347597681e-18\\
146.01	1.73472347597681e-18\\
147.01	1.73472347597681e-18\\
148.01	1.73472347597681e-18\\
149.01	0\\
150.01	0\\
151.01	1.73472347597681e-18\\
152.01	1.73472347597681e-18\\
153.01	1.73472347597681e-18\\
154.01	1.73472347597681e-18\\
155.01	1.73472347597681e-18\\
156.01	1.73472347597681e-18\\
157.01	0\\
158.01	1.73472347597681e-18\\
159.01	1.73472347597681e-18\\
160.01	1.73472347597681e-18\\
161.01	1.73472347597681e-18\\
162.01	1.73472347597681e-18\\
163.01	1.73472347597681e-18\\
164.01	0\\
165.01	0\\
166.01	1.73472347597681e-18\\
167.01	0\\
168.01	1.73472347597681e-18\\
169.01	0\\
170.01	1.73472347597681e-18\\
171.01	0\\
172.01	0\\
173.01	0\\
174.01	1.73472347597681e-18\\
175.01	1.73472347597681e-18\\
176.01	1.73472347597681e-18\\
177.01	0\\
178.01	1.73472347597681e-18\\
179.01	0\\
180.01	0\\
181.01	1.73472347597681e-18\\
182.01	0\\
183.01	0\\
184.01	0\\
185.01	1.73472347597681e-18\\
186.01	0\\
187.01	0\\
188.01	0\\
189.01	0\\
190.01	1.73472347597681e-18\\
191.01	0\\
192.01	1.73472347597681e-18\\
193.01	0\\
194.01	1.73472347597681e-18\\
195.01	0\\
196.01	1.73472347597681e-18\\
197.01	0\\
198.01	0\\
199.01	0\\
200.01	0\\
201.01	0\\
202.01	1.73472347597681e-18\\
203.01	0\\
204.01	1.73472347597681e-18\\
205.01	1.73472347597681e-18\\
206.01	1.73472347597681e-18\\
207.01	1.73472347597681e-18\\
208.01	0\\
209.01	1.73472347597681e-18\\
210.01	0\\
211.01	1.73472347597681e-18\\
212.01	0\\
213.01	1.73472347597681e-18\\
214.01	0\\
215.01	0\\
216.01	0\\
217.01	0\\
218.01	1.73472347597681e-18\\
219.01	1.73472347597681e-18\\
220.01	0\\
221.01	0\\
222.01	0\\
223.01	1.73472347597681e-18\\
224.01	1.73472347597681e-18\\
225.01	0\\
226.01	1.73472347597681e-18\\
227.01	0\\
228.01	0\\
229.01	1.73472347597681e-18\\
230.01	0\\
231.01	0\\
232.01	0\\
233.01	0\\
234.01	1.73472347597681e-18\\
235.01	0\\
236.01	1.73472347597681e-18\\
237.01	1.73472347597681e-18\\
238.01	0\\
239.01	1.73472347597681e-18\\
240.01	0\\
241.01	1.73472347597681e-18\\
242.01	0\\
243.01	1.73472347597681e-18\\
244.01	1.73472347597681e-18\\
245.01	1.73472347597681e-18\\
246.01	1.73472347597681e-18\\
247.01	0\\
248.01	0\\
249.01	0\\
250.01	1.73472347597681e-18\\
251.01	0\\
252.01	1.73472347597681e-18\\
253.01	1.73472347597681e-18\\
254.01	1.73472347597681e-18\\
255.01	1.73472347597681e-18\\
256.01	1.73472347597681e-18\\
257.01	1.73472347597681e-18\\
258.01	0\\
259.01	0\\
260.01	0\\
261.01	0\\
262.01	1.73472347597681e-18\\
263.01	1.73472347597681e-18\\
264.01	0\\
265.01	1.73472347597681e-18\\
266.01	1.73472347597681e-18\\
267.01	0\\
268.01	0\\
269.01	1.73472347597681e-18\\
270.01	0\\
271.01	1.73472347597681e-18\\
272.01	0\\
273.01	1.73472347597681e-18\\
274.01	0\\
275.01	0\\
276.01	1.73472347597681e-18\\
277.01	1.73472347597681e-18\\
278.01	0\\
279.01	1.73472347597681e-18\\
280.01	0\\
281.01	1.73472347597681e-18\\
282.01	1.73472347597681e-18\\
283.01	1.73472347597681e-18\\
284.01	0\\
285.01	1.73472347597681e-18\\
286.01	1.73472347597681e-18\\
287.01	1.73472347597681e-18\\
288.01	1.73472347597681e-18\\
289.01	0\\
290.01	0\\
291.01	0\\
292.01	0\\
293.01	1.73472347597681e-18\\
294.01	1.73472347597681e-18\\
295.01	0\\
296.01	1.73472347597681e-18\\
297.01	0\\
298.01	1.73472347597681e-18\\
299.01	0\\
300.01	1.73472347597681e-18\\
301.01	0\\
302.01	0\\
303.01	0\\
304.01	1.73472347597681e-18\\
305.01	1.73472347597681e-18\\
306.01	0\\
307.01	1.73472347597681e-18\\
308.01	0\\
309.01	1.73472347597681e-18\\
310.01	1.73472347597681e-18\\
311.01	0\\
312.01	1.73472347597681e-18\\
313.01	1.73472347597681e-18\\
314.01	0\\
315.01	1.73472347597681e-18\\
316.01	0\\
317.01	1.73472347597681e-18\\
318.01	0\\
319.01	0\\
320.01	0\\
321.01	0\\
322.01	1.73472347597681e-18\\
323.01	1.73472347597681e-18\\
324.01	0\\
325.01	1.73472347597681e-18\\
326.01	1.73472347597681e-18\\
327.01	0\\
328.01	1.73472347597681e-18\\
329.01	0\\
330.01	0\\
331.01	1.73472347597681e-18\\
332.01	1.73472347597681e-18\\
333.01	1.73472347597681e-18\\
334.01	1.73472347597681e-18\\
335.01	1.73472347597681e-18\\
336.01	0\\
337.01	1.73472347597681e-18\\
338.01	0\\
339.01	0\\
340.01	1.73472347597681e-18\\
341.01	1.73472347597681e-18\\
342.01	1.73472347597681e-18\\
343.01	1.73472347597681e-18\\
344.01	0\\
345.01	0\\
346.01	0\\
347.01	1.73472347597681e-18\\
348.01	0\\
349.01	1.73472347597681e-18\\
350.01	0\\
351.01	0\\
352.01	0\\
353.01	1.73472347597681e-18\\
354.01	1.73472347597681e-18\\
355.01	0\\
356.01	0\\
357.01	1.73472347597681e-18\\
358.01	1.73472347597681e-18\\
359.01	1.73472347597681e-18\\
360.01	0\\
361.01	0\\
362.01	0\\
363.01	0\\
364.01	0\\
365.01	1.73472347597681e-18\\
366.01	1.73472347597681e-18\\
367.01	1.73472347597681e-18\\
368.01	1.73472347597681e-18\\
369.01	0\\
370.01	0\\
371.01	1.73472347597681e-18\\
372.01	1.73472347597681e-18\\
373.01	1.73472347597681e-18\\
374.01	0\\
375.01	1.73472347597681e-18\\
376.01	0\\
377.01	0\\
378.01	1.73472347597681e-18\\
379.01	0\\
380.01	1.73472347597681e-18\\
381.01	1.73472347597681e-18\\
382.01	1.73472347597681e-18\\
383.01	0\\
384.01	1.73472347597681e-18\\
385.01	0\\
386.01	1.73472347597681e-18\\
387.01	1.73472347597681e-18\\
388.01	1.73472347597681e-18\\
389.01	0\\
390.01	1.73472347597681e-18\\
391.01	1.73472347597681e-18\\
392.01	0\\
393.01	0\\
394.01	1.73472347597681e-18\\
395.01	1.73472347597681e-18\\
396.01	1.73472347597681e-18\\
397.01	1.73472347597681e-18\\
398.01	0\\
399.01	1.73472347597681e-18\\
400.01	0\\
401.01	0\\
402.01	0\\
403.01	1.73472347597681e-18\\
404.01	1.73472347597681e-18\\
405.01	1.73472347597681e-18\\
406.01	1.73472347597681e-18\\
407.01	1.73472347597681e-18\\
408.01	1.73472347597681e-18\\
409.01	0\\
410.01	1.73472347597681e-18\\
411.01	0\\
412.01	1.73472347597681e-18\\
413.01	1.73472347597681e-18\\
414.01	0\\
415.01	0\\
416.01	1.73472347597681e-18\\
417.01	1.73472347597681e-18\\
418.01	1.73472347597681e-18\\
419.01	1.73472347597681e-18\\
420.01	0\\
421.01	1.73472347597681e-18\\
422.01	1.73472347597681e-18\\
423.01	1.73472347597681e-18\\
424.01	1.73472347597681e-18\\
425.01	0\\
426.01	1.73472347597681e-18\\
427.01	1.73472347597681e-18\\
428.01	1.73472347597681e-18\\
429.01	0\\
430.01	1.73472347597681e-18\\
431.01	1.73472347597681e-18\\
432.01	1.73472347597681e-18\\
433.01	0\\
434.01	1.73472347597681e-18\\
435.01	1.73472347597681e-18\\
436.01	1.73472347597681e-18\\
437.01	1.73472347597681e-18\\
438.01	1.73472347597681e-18\\
439.01	0\\
440.01	1.73472347597681e-18\\
441.01	1.73472347597681e-18\\
442.01	0\\
443.01	0\\
444.01	0\\
445.01	1.73472347597681e-18\\
446.01	1.73472347597681e-18\\
447.01	0\\
448.01	1.73472347597681e-18\\
449.01	1.73472347597681e-18\\
450.01	0\\
451.01	0\\
452.01	0\\
453.01	1.73472347597681e-18\\
454.01	1.73472347597681e-18\\
455.01	0\\
456.01	1.73472347597681e-18\\
457.01	1.73472347597681e-18\\
458.01	0\\
459.01	1.73472347597681e-18\\
460.01	0\\
461.01	1.73472347597681e-18\\
462.01	0\\
463.01	0\\
464.01	0\\
465.01	1.73472347597681e-18\\
466.01	0\\
467.01	1.73472347597681e-18\\
468.01	0\\
469.01	1.73472347597681e-18\\
470.01	1.73472347597681e-18\\
471.01	1.73472347597681e-18\\
472.01	0\\
473.01	0\\
474.01	0\\
475.01	1.73472347597681e-18\\
476.01	0\\
477.01	0\\
478.01	1.73472347597681e-18\\
479.01	1.73472347597681e-18\\
480.01	0\\
481.01	0\\
482.01	1.73472347597681e-18\\
483.01	1.73472347597681e-18\\
484.01	0\\
485.01	0\\
486.01	0\\
487.01	1.73472347597681e-18\\
488.01	1.73472347597681e-18\\
489.01	1.73472347597681e-18\\
490.01	1.73472347597681e-18\\
491.01	0\\
492.01	1.73472347597681e-18\\
493.01	1.73472347597681e-18\\
494.01	0\\
495.01	1.73472347597681e-18\\
496.01	0\\
497.01	1.73472347597681e-18\\
498.01	1.73472347597681e-18\\
499.01	0\\
500.01	1.73472347597681e-18\\
501.01	1.73472347597681e-18\\
502.01	0\\
503.01	1.73472347597681e-18\\
504.01	1.73472347597681e-18\\
505.01	1.73472347597681e-18\\
506.01	0\\
507.01	0\\
508.01	0\\
509.01	0\\
510.01	1.73472347597681e-18\\
511.01	1.73472347597681e-18\\
512.01	0\\
513.01	1.73472347597681e-18\\
514.01	1.73472347597681e-18\\
515.01	0\\
516.01	0\\
517.01	1.73472347597681e-18\\
518.01	1.73472347597681e-18\\
519.01	1.73472347597681e-18\\
520.01	0\\
521.01	0\\
522.01	0\\
523.01	1.73472347597681e-18\\
524.01	0\\
525.01	1.73472347597681e-18\\
526.01	1.73472347597681e-18\\
527.01	0\\
528.01	0\\
529.01	0\\
530.01	1.73472347597681e-18\\
531.01	0\\
532.01	1.73472347597681e-18\\
533.01	0\\
534.01	1.73472347597681e-18\\
535.01	0\\
536.01	1.73472347597681e-18\\
537.01	1.73472347597681e-18\\
538.01	1.73472347597681e-18\\
539.01	0\\
540.01	0\\
541.01	1.73472347597681e-18\\
542.01	1.73472347597681e-18\\
543.01	0\\
544.01	1.73472347597681e-18\\
545.01	0\\
546.01	1.73472347597681e-18\\
547.01	0\\
548.01	0\\
549.01	0\\
550.01	0\\
551.01	0\\
552.01	0\\
553.01	0\\
554.01	0\\
555.01	1.73472347597681e-18\\
556.01	0\\
557.01	0\\
558.01	0\\
559.01	1.73472347597681e-18\\
560.01	1.73472347597681e-18\\
561.01	1.73472347597681e-18\\
562.01	1.73472347597681e-18\\
563.01	0\\
564.01	0\\
565.01	1.73472347597681e-18\\
566.01	0\\
567.01	0\\
568.01	0\\
569.01	0\\
570.01	0\\
571.01	0\\
572.01	1.73472347597681e-18\\
573.01	0\\
574.01	1.73472347597681e-18\\
575.01	0\\
576.01	1.73472347597681e-18\\
577.01	0\\
578.01	1.73472347597681e-18\\
579.01	0\\
580.01	0\\
581.01	1.73472347597681e-18\\
582.01	0\\
583.01	0\\
584.01	0\\
585.01	0\\
586.01	0\\
587.01	0\\
588.01	0\\
589.01	0\\
590.01	0\\
591.01	0\\
592.01	1.73472347597681e-18\\
593.01	0\\
594.01	0\\
595.01	0\\
596.01	0\\
597.01	0\\
598.01	0\\
599.01	0\\
599.02	0\\
599.03	1.73472347597681e-18\\
599.04	1.73472347597681e-18\\
599.05	0\\
599.06	0\\
599.07	1.73472347597681e-18\\
599.08	0\\
599.09	0\\
599.1	0\\
599.11	1.73472347597681e-18\\
599.12	1.73472347597681e-18\\
599.13	0\\
599.14	0\\
599.15	1.73472347597681e-18\\
599.16	1.73472347597681e-18\\
599.17	1.73472347597681e-18\\
599.18	0\\
599.19	0\\
599.2	0\\
599.21	0\\
599.22	0\\
599.23	1.73472347597681e-18\\
599.24	1.73472347597681e-18\\
599.25	1.73472347597681e-18\\
599.26	1.73472347597681e-18\\
599.27	0\\
599.28	0\\
599.29	0\\
599.3	0\\
599.31	0\\
599.32	1.73472347597681e-18\\
599.33	1.73472347597681e-18\\
599.34	0\\
599.35	0\\
599.36	0\\
599.37	1.73472347597681e-18\\
599.38	0\\
599.39	0\\
599.4	0\\
599.41	0\\
599.42	0\\
599.43	0\\
599.44	0\\
599.45	0\\
599.46	1.73472347597681e-18\\
599.47	1.73472347597681e-18\\
599.48	1.73472347597681e-18\\
599.49	1.73472347597681e-18\\
599.5	1.73472347597681e-18\\
599.51	0\\
599.52	0\\
599.53	1.73472347597681e-18\\
599.54	0\\
599.55	0\\
599.56	0\\
599.57	1.73472347597681e-18\\
599.58	0\\
599.59	0\\
599.6	0\\
599.61	0\\
599.62	1.73472347597681e-18\\
599.63	0\\
599.64	1.73472347597681e-18\\
599.65	0\\
599.66	0\\
599.67	0\\
599.68	0\\
599.69	1.73472347597681e-18\\
599.7	0\\
599.71	1.73472347597681e-18\\
599.72	1.73472347597681e-18\\
599.73	1.73472347597681e-18\\
599.74	0\\
599.75	0\\
599.76	0\\
599.77	0\\
599.78	0\\
599.79	0\\
599.8	0\\
599.81	0\\
599.82	0\\
599.83	1.73472347597681e-18\\
599.84	0\\
599.85	0\\
599.86	0\\
599.87	0\\
599.88	0\\
599.89	0\\
599.9	0\\
599.91	0\\
599.92	0\\
599.93	0\\
599.94	0\\
599.95	0\\
599.96	0\\
599.97	0\\
599.98	0\\
599.99	0\\
600	0\\
};
\addplot [color=blue!40!mycolor9,solid,forget plot]
  table[row sep=crcr]{%
0.01	0.000191675496803481\\
1.01	0.000191675330892803\\
2.01	0.000191675161516676\\
3.01	0.000191674988602361\\
4.01	0.00019167481207555\\
5.01	0.000191674631860398\\
6.01	0.000191674447879429\\
7.01	0.000191674260053553\\
8.01	0.00019167406830198\\
9.01	0.000191673872542245\\
10.01	0.000191673672690089\\
11.01	0.000191673468659526\\
12.01	0.000191673260362736\\
13.01	0.000191673047709994\\
14.01	0.000191672830609725\\
15.01	0.000191672608968402\\
16.01	0.000191672382690531\\
17.01	0.00019167215167854\\
18.01	0.000191671915832857\\
19.01	0.000191671675051788\\
20.01	0.000191671429231424\\
21.01	0.000191671178265693\\
22.01	0.000191670922046345\\
23.01	0.000191670660462698\\
24.01	0.000191670393401821\\
25.01	0.000191670120748349\\
26.01	0.000191669842384471\\
27.01	0.00019166955818988\\
28.01	0.000191669268041728\\
29.01	0.000191668971814529\\
30.01	0.000191668669380132\\
31.01	0.000191668360607715\\
32.01	0.000191668045363609\\
33.01	0.000191667723511353\\
34.01	0.000191667394911549\\
35.01	0.000191667059421871\\
36.01	0.000191666716896939\\
37.01	0.000191666367188282\\
38.01	0.000191666010144287\\
39.01	0.000191665645610099\\
40.01	0.000191665273427564\\
41.01	0.000191664893435169\\
42.01	0.000191664505467959\\
43.01	0.000191664109357467\\
44.01	0.000191663704931634\\
45.01	0.000191663292014728\\
46.01	0.000191662870427297\\
47.01	0.00019166243998602\\
48.01	0.000191662000503702\\
49.01	0.000191661551789108\\
50.01	0.000191661093646997\\
51.01	0.00019166062587792\\
52.01	0.000191660148278124\\
53.01	0.000191659660639594\\
54.01	0.000191659162749795\\
55.01	0.000191658654391696\\
56.01	0.000191658135343617\\
57.01	0.000191657605379132\\
58.01	0.000191657064266977\\
59.01	0.000191656511770949\\
60.01	0.000191655947649768\\
61.01	0.000191655371657018\\
62.01	0.000191654783541004\\
63.01	0.000191654183044605\\
64.01	0.00019165356990526\\
65.01	0.000191652943854736\\
66.01	0.000191652304619051\\
67.01	0.000191651651918359\\
68.01	0.000191650985466823\\
69.01	0.000191650304972474\\
70.01	0.000191649610137045\\
71.01	0.000191648900655933\\
72.01	0.000191648176217947\\
73.01	0.000191647436505226\\
74.01	0.000191646681193073\\
75.01	0.000191645909949826\\
76.01	0.000191645122436699\\
77.01	0.000191644318307609\\
78.01	0.000191643497209059\\
79.01	0.000191642658779875\\
80.01	0.000191641802651221\\
81.01	0.000191640928446217\\
82.01	0.000191640035779957\\
83.01	0.000191639124259171\\
84.01	0.000191638193482143\\
85.01	0.000191637243038544\\
86.01	0.00019163627250912\\
87.01	0.000191635281465622\\
88.01	0.000191634269470541\\
89.01	0.000191633236076969\\
90.01	0.000191632180828296\\
91.01	0.000191631103258075\\
92.01	0.000191630002889799\\
93.01	0.000191628879236625\\
94.01	0.000191627731801232\\
95.01	0.000191626560075505\\
96.01	0.000191625363540377\\
97.01	0.000191624141665498\\
98.01	0.000191622893909069\\
99.01	0.000191621619717577\\
100.01	0.000191620318525479\\
101.01	0.000191618989754996\\
102.01	0.000191617632815813\\
103.01	0.000191616247104844\\
104.01	0.000191614832005879\\
105.01	0.000191613386889363\\
106.01	0.000191611911112053\\
107.01	0.000191610404016747\\
108.01	0.000191608864931999\\
109.01	0.000191607293171666\\
110.01	0.000191605688034826\\
111.01	0.000191604048805189\\
112.01	0.00019160237475095\\
113.01	0.000191600665124332\\
114.01	0.000191598919161316\\
115.01	0.000191597136081202\\
116.01	0.000191595315086281\\
117.01	0.000191593455361469\\
118.01	0.000191591556073898\\
119.01	0.000191589616372502\\
120.01	0.00019158763538764\\
121.01	0.000191585612230731\\
122.01	0.000191583545993712\\
123.01	0.000191581435748706\\
124.01	0.000191579280547553\\
125.01	0.00019157707942131\\
126.01	0.000191574831379859\\
127.01	0.000191572535411412\\
128.01	0.000191570190482023\\
129.01	0.000191567795535015\\
130.01	0.000191565349490651\\
131.01	0.000191562851245426\\
132.01	0.000191560299671658\\
133.01	0.000191557693616928\\
134.01	0.000191555031903436\\
135.01	0.000191552313327568\\
136.01	0.000191549536659218\\
137.01	0.000191546700641252\\
138.01	0.000191543803988821\\
139.01	0.000191540845388851\\
140.01	0.000191537823499318\\
141.01	0.000191534736948608\\
142.01	0.000191531584334914\\
143.01	0.000191528364225474\\
144.01	0.000191525075155902\\
145.01	0.000191521715629498\\
146.01	0.000191518284116505\\
147.01	0.000191514779053362\\
148.01	0.000191511198841912\\
149.01	0.000191507541848648\\
150.01	0.000191503806403944\\
151.01	0.000191499990801149\\
152.01	0.00019149609329582\\
153.01	0.000191492112104849\\
154.01	0.000191488045405591\\
155.01	0.000191483891334927\\
156.01	0.000191479647988385\\
157.01	0.000191475313419223\\
158.01	0.000191470885637363\\
159.01	0.00019146636260853\\
160.01	0.00019146174225315\\
161.01	0.000191457022445369\\
162.01	0.000191452201012006\\
163.01	0.000191447275731413\\
164.01	0.000191442244332407\\
165.01	0.000191437104493158\\
166.01	0.000191431853839992\\
167.01	0.000191426489946219\\
168.01	0.000191421010330909\\
169.01	0.000191415412457695\\
170.01	0.000191409693733423\\
171.01	0.00019140385150687\\
172.01	0.000191397883067509\\
173.01	0.000191391785643999\\
174.01	0.000191385556402869\\
175.01	0.000191379192447093\\
176.01	0.000191372690814566\\
177.01	0.000191366048476678\\
178.01	0.000191359262336715\\
179.01	0.000191352329228336\\
180.01	0.000191345245913957\\
181.01	0.000191338009083048\\
182.01	0.000191330615350512\\
183.01	0.000191323061254954\\
184.01	0.000191315343256863\\
185.01	0.000191307457736866\\
186.01	0.000191299400993794\\
187.01	0.000191291169242888\\
188.01	0.000191282758613754\\
189.01	0.000191274165148468\\
190.01	0.000191265384799423\\
191.01	0.000191256413427357\\
192.01	0.000191247246799158\\
193.01	0.000191237880585672\\
194.01	0.000191228310359505\\
195.01	0.00019121853159265\\
196.01	0.000191208539654252\\
197.01	0.000191198329808099\\
198.01	0.000191187897210164\\
199.01	0.000191177236906195\\
200.01	0.000191166343828963\\
201.01	0.000191155212795753\\
202.01	0.000191143838505591\\
203.01	0.000191132215536462\\
204.01	0.000191120338342484\\
205.01	0.000191108201251003\\
206.01	0.000191095798459611\\
207.01	0.000191083124033034\\
208.01	0.000191070171900077\\
209.01	0.00019105693585039\\
210.01	0.00019104340953117\\
211.01	0.000191029586443845\\
212.01	0.000191015459940531\\
213.01	0.000191001023220659\\
214.01	0.000190986269327157\\
215.01	0.000190971191142959\\
216.01	0.000190955781387095\\
217.01	0.000190940032610802\\
218.01	0.000190923937193574\\
219.01	0.000190907487339088\\
220.01	0.000190890675071028\\
221.01	0.00019087349222882\\
222.01	0.000190855930463219\\
223.01	0.000190837981231809\\
224.01	0.000190819635794494\\
225.01	0.000190800885208684\\
226.01	0.000190781720324528\\
227.01	0.000190762131779978\\
228.01	0.000190742109995661\\
229.01	0.000190721645169816\\
230.01	0.000190700727272822\\
231.01	0.000190679346041904\\
232.01	0.000190657490975478\\
233.01	0.000190635151327464\\
234.01	0.000190612316101414\\
235.01	0.000190588974044529\\
236.01	0.00019056511364157\\
237.01	0.000190540723108494\\
238.01	0.000190515790386035\\
239.01	0.000190490303133175\\
240.01	0.000190464248720256\\
241.01	0.000190437614222197\\
242.01	0.000190410386411289\\
243.01	0.000190382551750001\\
244.01	0.000190354096383507\\
245.01	0.000190325006132088\\
246.01	0.000190295266483299\\
247.01	0.000190264862584002\\
248.01	0.000190233779232185\\
249.01	0.000190202000868537\\
250.01	0.000190169511567884\\
251.01	0.000190136295030381\\
252.01	0.000190102334572563\\
253.01	0.000190067613118012\\
254.01	0.000190032113188016\\
255.01	0.000189995816891825\\
256.01	0.000189958705916796\\
257.01	0.000189920761518214\\
258.01	0.000189881964508922\\
259.01	0.000189842295248722\\
260.01	0.000189801733633413\\
261.01	0.000189760259083733\\
262.01	0.000189717850533866\\
263.01	0.000189674486419855\\
264.01	0.000189630144667585\\
265.01	0.000189584802680576\\
266.01	0.000189538437327408\\
267.01	0.000189491024929015\\
268.01	0.000189442541245458\\
269.01	0.000189392961462578\\
270.01	0.000189342260178209\\
271.01	0.00018929041138817\\
272.01	0.000189237388471852\\
273.01	0.000189183164177547\\
274.01	0.000189127710607388\\
275.01	0.000189070999201965\\
276.01	0.000189013000724573\\
277.01	0.000188953685245135\\
278.01	0.000188893022123765\\
279.01	0.000188830979993853\\
280.01	0.000188767526744975\\
281.01	0.000188702629505247\\
282.01	0.000188636254623302\\
283.01	0.000188568367650034\\
284.01	0.000188498933319672\\
285.01	0.000188427915530752\\
286.01	0.000188355277326345\\
287.01	0.000188280980874143\\
288.01	0.000188204987445961\\
289.01	0.000188127257396848\\
290.01	0.000188047750143716\\
291.01	0.000187966424143584\\
292.01	0.000187883236871319\\
293.01	0.000187798144796898\\
294.01	0.00018771110336223\\
295.01	0.000187622066957565\\
296.01	0.000187530988897249\\
297.01	0.000187437821395105\\
298.01	0.000187342515539398\\
299.01	0.00018724502126709\\
300.01	0.000187145287337759\\
301.01	0.000187043261306917\\
302.01	0.000186938889498877\\
303.01	0.000186832116978937\\
304.01	0.000186722887525284\\
305.01	0.000186611143600105\\
306.01	0.000186496826320265\\
307.01	0.000186379875427514\\
308.01	0.000186260229257921\\
309.01	0.000186137824710943\\
310.01	0.000186012597217818\\
311.01	0.000185884480709398\\
312.01	0.0001857534075834\\
313.01	0.000185619308671029\\
314.01	0.00018548211320304\\
315.01	0.000185341748775244\\
316.01	0.000185198141313155\\
317.01	0.000185051215036326\\
318.01	0.000184900892421789\\
319.01	0.000184747094167022\\
320.01	0.000184589739152115\\
321.01	0.000184428744401327\\
322.01	0.000184264025043996\\
323.01	0.000184095494274667\\
324.01	0.000183923063312526\\
325.01	0.000183746641360142\\
326.01	0.000183566135561407\\
327.01	0.000183381450958727\\
328.01	0.000183192490449445\\
329.01	0.000182999154741398\\
330.01	0.000182801342307663\\
331.01	0.000182598949340461\\
332.01	0.000182391869704155\\
333.01	0.000182179994887342\\
334.01	0.000181963213953909\\
335.01	0.000181741413493261\\
336.01	0.000181514477569386\\
337.01	0.000181282287668954\\
338.01	0.000181044722648302\\
339.01	0.00018080165867925\\
340.01	0.000180552969193797\\
341.01	0.000180298524827574\\
342.01	0.000180038193362023\\
343.01	0.000179771839665348\\
344.01	0.000179499325631981\\
345.01	0.000179220510120906\\
346.01	0.000178935248892264\\
347.01	0.000178643394542722\\
348.01	0.000178344796439215\\
349.01	0.000178039300651247\\
350.01	0.000177726749881571\\
351.01	0.000177406983395285\\
352.01	0.000177079836947357\\
353.01	0.000176745142708478\\
354.01	0.000176402729189255\\
355.01	0.0001760524211628\\
356.01	0.000175694039585566\\
357.01	0.000175327401516415\\
358.01	0.000174952320033924\\
359.01	0.00017456860415173\\
360.01	0.000174176058731738\\
361.01	0.000173774484395069\\
362.01	0.000173363677429981\\
363.01	0.00017294342969648\\
364.01	0.000172513528526505\\
365.01	0.000172073756618337\\
366.01	0.000171623891923186\\
367.01	0.000171163707521389\\
368.01	0.000170692971484062\\
369.01	0.000170211446714636\\
370.01	0.000169718890762957\\
371.01	0.000169215055600849\\
372.01	0.000168699687345003\\
373.01	0.000168172525906693\\
374.01	0.0001676333045418\\
375.01	0.000167081749269502\\
376.01	0.000166517578175698\\
377.01	0.000165940501199465\\
378.01	0.000165350222575527\\
379.01	0.000164746422497702\\
380.01	0.000164128763725698\\
381.01	0.000163496908888791\\
382.01	0.000162850511652208\\
383.01	0.000162189216550953\\
384.01	0.000161512658727167\\
385.01	0.000160820463659077\\
386.01	0.000160112246881736\\
387.01	0.000159387613698577\\
388.01	0.00015864615888426\\
389.01	0.000157887466377809\\
390.01	0.00015711110896618\\
391.01	0.0001563166479577\\
392.01	0.000155503632845158\\
393.01	0.000154671600958131\\
394.01	0.000153820077103977\\
395.01	0.00015294857319764\\
396.01	0.00015205658787922\\
397.01	0.000151143606119361\\
398.01	0.000150209098811858\\
399.01	0.000149252522352905\\
400.01	0.000148273318206845\\
401.01	0.000147270912457522\\
402.01	0.00014624471534524\\
403.01	0.000145194120788212\\
404.01	0.000144118505888446\\
405.01	0.000143017230421419\\
406.01	0.000141889636308597\\
407.01	0.00014073504707275\\
408.01	0.000139552767275\\
409.01	0.000138342081933239\\
410.01	0.000137102255921043\\
411.01	0.000135832533346527\\
412.01	0.000134532136910491\\
413.01	0.000133200267242934\\
414.01	0.000131836102217392\\
415.01	0.000130438796242293\\
416.01	0.000129007479528492\\
417.01	0.000127541257332353\\
418.01	0.000126039209173491\\
419.01	0.000124500388026453\\
420.01	0.000122923819485697\\
421.01	0.000121308500902943\\
422.01	0.000119653400496437\\
423.01	0.000117957456431445\\
424.01	0.000116219575871322\\
425.01	0.000114438633999054\\
426.01	0.000112613473008732\\
427.01	0.000110742901067063\\
428.01	0.000108825691245052\\
429.01	0.000106860580420495\\
430.01	0.00010484626815214\\
431.01	0.000102781415526892\\
432.01	0.000100664643982487\\
433.01	9.84945341082743e-05\\
434.01	9.62696244282846e-05\\
435.01	9.39884101717244e-05\\
436.01	9.16493420380481e-05\\
437.01	8.92508249650172e-05\\
438.01	8.67912169117014e-05\\
439.01	8.42688276705365e-05\\
440.01	8.16819177270559e-05\\
441.01	7.90286971904582e-05\\
442.01	7.63073248244862e-05\\
443.01	7.35159072154431e-05\\
444.01	7.06524981237393e-05\\
445.01	6.77150980773252e-05\\
446.01	6.4701654280153e-05\\
447.01	6.16100609273206e-05\\
448.01	5.84381600423397e-05\\
449.01	5.51837429808386e-05\\
450.01	5.18445527828617e-05\\
451.01	4.84182876016714e-05\\
452.01	4.49026054967338e-05\\
453.01	4.12951309523273e-05\\
454.01	3.75934635778053e-05\\
455.01	3.37951895642719e-05\\
456.01	2.98978966246349e-05\\
457.01	2.58991933472629e-05\\
458.01	2.17967342929452e-05\\
459.01	1.75882546981764e-05\\
460.01	1.32716547695307e-05\\
461.01	8.84584227558473e-06\\
462.01	4.32849815507237e-06\\
463.01	2.87754591011061e-07\\
464.01	0\\
465.01	1.73472347597681e-18\\
466.01	0\\
467.01	1.73472347597681e-18\\
468.01	0\\
469.01	1.73472347597681e-18\\
470.01	1.73472347597681e-18\\
471.01	1.73472347597681e-18\\
472.01	0\\
473.01	0\\
474.01	0\\
475.01	1.73472347597681e-18\\
476.01	0\\
477.01	0\\
478.01	1.73472347597681e-18\\
479.01	1.73472347597681e-18\\
480.01	0\\
481.01	0\\
482.01	1.73472347597681e-18\\
483.01	1.73472347597681e-18\\
484.01	0\\
485.01	0\\
486.01	0\\
487.01	1.73472347597681e-18\\
488.01	1.73472347597681e-18\\
489.01	1.73472347597681e-18\\
490.01	1.73472347597681e-18\\
491.01	0\\
492.01	1.73472347597681e-18\\
493.01	1.73472347597681e-18\\
494.01	0\\
495.01	1.73472347597681e-18\\
496.01	0\\
497.01	1.73472347597681e-18\\
498.01	1.73472347597681e-18\\
499.01	0\\
500.01	1.73472347597681e-18\\
501.01	1.73472347597681e-18\\
502.01	0\\
503.01	1.73472347597681e-18\\
504.01	1.73472347597681e-18\\
505.01	1.73472347597681e-18\\
506.01	0\\
507.01	0\\
508.01	0\\
509.01	0\\
510.01	1.73472347597681e-18\\
511.01	1.73472347597681e-18\\
512.01	0\\
513.01	1.73472347597681e-18\\
514.01	1.73472347597681e-18\\
515.01	0\\
516.01	0\\
517.01	1.73472347597681e-18\\
518.01	1.73472347597681e-18\\
519.01	1.73472347597681e-18\\
520.01	0\\
521.01	0\\
522.01	0\\
523.01	1.73472347597681e-18\\
524.01	0\\
525.01	1.73472347597681e-18\\
526.01	1.73472347597681e-18\\
527.01	0\\
528.01	0\\
529.01	0\\
530.01	1.73472347597681e-18\\
531.01	0\\
532.01	1.73472347597681e-18\\
533.01	0\\
534.01	1.73472347597681e-18\\
535.01	0\\
536.01	1.73472347597681e-18\\
537.01	1.73472347597681e-18\\
538.01	1.73472347597681e-18\\
539.01	0\\
540.01	0\\
541.01	1.73472347597681e-18\\
542.01	1.73472347597681e-18\\
543.01	0\\
544.01	1.73472347597681e-18\\
545.01	0\\
546.01	1.73472347597681e-18\\
547.01	0\\
548.01	0\\
549.01	0\\
550.01	0\\
551.01	0\\
552.01	0\\
553.01	0\\
554.01	0\\
555.01	1.73472347597681e-18\\
556.01	0\\
557.01	0\\
558.01	0\\
559.01	1.73472347597681e-18\\
560.01	1.73472347597681e-18\\
561.01	1.73472347597681e-18\\
562.01	1.73472347597681e-18\\
563.01	0\\
564.01	0\\
565.01	1.73472347597681e-18\\
566.01	0\\
567.01	0\\
568.01	0\\
569.01	0\\
570.01	0\\
571.01	0\\
572.01	1.73472347597681e-18\\
573.01	0\\
574.01	1.73472347597681e-18\\
575.01	0\\
576.01	1.73472347597681e-18\\
577.01	0\\
578.01	1.73472347597681e-18\\
579.01	0\\
580.01	0\\
581.01	1.73472347597681e-18\\
582.01	0\\
583.01	0\\
584.01	0\\
585.01	0\\
586.01	0\\
587.01	0\\
588.01	0\\
589.01	0\\
590.01	0\\
591.01	0\\
592.01	1.73472347597681e-18\\
593.01	0\\
594.01	0\\
595.01	0\\
596.01	0\\
597.01	0\\
598.01	0\\
599.01	0\\
599.02	0\\
599.03	1.73472347597681e-18\\
599.04	1.73472347597681e-18\\
599.05	0\\
599.06	0\\
599.07	1.73472347597681e-18\\
599.08	0\\
599.09	0\\
599.1	0\\
599.11	1.73472347597681e-18\\
599.12	1.73472347597681e-18\\
599.13	0\\
599.14	0\\
599.15	1.73472347597681e-18\\
599.16	1.73472347597681e-18\\
599.17	1.73472347597681e-18\\
599.18	0\\
599.19	0\\
599.2	0\\
599.21	0\\
599.22	0\\
599.23	1.73472347597681e-18\\
599.24	1.73472347597681e-18\\
599.25	1.73472347597681e-18\\
599.26	1.73472347597681e-18\\
599.27	0\\
599.28	0\\
599.29	0\\
599.3	0\\
599.31	0\\
599.32	1.73472347597681e-18\\
599.33	1.73472347597681e-18\\
599.34	0\\
599.35	0\\
599.36	0\\
599.37	1.73472347597681e-18\\
599.38	0\\
599.39	0\\
599.4	0\\
599.41	0\\
599.42	0\\
599.43	0\\
599.44	0\\
599.45	0\\
599.46	1.73472347597681e-18\\
599.47	1.73472347597681e-18\\
599.48	1.73472347597681e-18\\
599.49	1.73472347597681e-18\\
599.5	1.73472347597681e-18\\
599.51	0\\
599.52	0\\
599.53	1.73472347597681e-18\\
599.54	0\\
599.55	0\\
599.56	0\\
599.57	1.73472347597681e-18\\
599.58	0\\
599.59	0\\
599.6	0\\
599.61	0\\
599.62	1.73472347597681e-18\\
599.63	0\\
599.64	1.73472347597681e-18\\
599.65	0\\
599.66	0\\
599.67	0\\
599.68	0\\
599.69	1.73472347597681e-18\\
599.7	0\\
599.71	1.73472347597681e-18\\
599.72	1.73472347597681e-18\\
599.73	1.73472347597681e-18\\
599.74	0\\
599.75	0\\
599.76	0\\
599.77	0\\
599.78	0\\
599.79	0\\
599.8	0\\
599.81	0\\
599.82	0\\
599.83	1.73472347597681e-18\\
599.84	0\\
599.85	0\\
599.86	0\\
599.87	0\\
599.88	0\\
599.89	0\\
599.9	0\\
599.91	0\\
599.92	0\\
599.93	0\\
599.94	0\\
599.95	0\\
599.96	0\\
599.97	0\\
599.98	0\\
599.99	0\\
600	0\\
};
\addplot [color=blue!75!mycolor7,solid,forget plot]
  table[row sep=crcr]{%
0.01	0.00078891105654883\\
1.01	0.000788910902421567\\
2.01	0.000788910745062972\\
3.01	0.00078891058440487\\
4.01	0.000788910420377644\\
5.01	0.000788910252910195\\
6.01	0.000788910081929878\\
7.01	0.000788909907362559\\
8.01	0.000788909729132462\\
9.01	0.000788909547162256\\
10.01	0.000788909361372913\\
11.01	0.000788909171683764\\
12.01	0.000788908978012372\\
13.01	0.000788908780274612\\
14.01	0.000788908578384479\\
15.01	0.000788908372254217\\
16.01	0.000788908161794137\\
17.01	0.000788907946912666\\
18.01	0.000788907727516256\\
19.01	0.000788907503509398\\
20.01	0.00078890727479451\\
21.01	0.000788907041271941\\
22.01	0.000788906802839886\\
23.01	0.0007889065593944\\
24.01	0.000788906310829304\\
25.01	0.00078890605703613\\
26.01	0.000788905797904129\\
27.01	0.000788905533320125\\
28.01	0.000788905263168578\\
29.01	0.000788904987331447\\
30.01	0.000788904705688169\\
31.01	0.000788904418115579\\
32.01	0.000788904124487925\\
33.01	0.00078890382467669\\
34.01	0.00078890351855066\\
35.01	0.000788903205975777\\
36.01	0.000788902886815131\\
37.01	0.000788902560928855\\
38.01	0.000788902228174059\\
39.01	0.000788901888404853\\
40.01	0.000788901541472166\\
41.01	0.000788901187223726\\
42.01	0.00078890082550402\\
43.01	0.000788900456154132\\
44.01	0.000788900079011801\\
45.01	0.000788899693911235\\
46.01	0.000788899300683077\\
47.01	0.000788898899154339\\
48.01	0.000788898489148303\\
49.01	0.000788898070484484\\
50.01	0.000788897642978422\\
51.01	0.000788897206441761\\
52.01	0.000788896760682065\\
53.01	0.000788896305502723\\
54.01	0.000788895840702955\\
55.01	0.000788895366077536\\
56.01	0.000788894881416926\\
57.01	0.000788894386507\\
58.01	0.000788893881129028\\
59.01	0.000788893365059553\\
60.01	0.000788892838070274\\
61.01	0.000788892299928\\
62.01	0.000788891750394432\\
63.01	0.000788891189226168\\
64.01	0.000788890616174551\\
65.01	0.000788890030985487\\
66.01	0.00078888943339942\\
67.01	0.000788888823151162\\
68.01	0.00078888819996979\\
69.01	0.000788887563578497\\
70.01	0.000788886913694475\\
71.01	0.000788886250028766\\
72.01	0.000788885572286206\\
73.01	0.000788884880165141\\
74.01	0.000788884173357431\\
75.01	0.000788883451548204\\
76.01	0.000788882714415772\\
77.01	0.000788881961631431\\
78.01	0.000788881192859378\\
79.01	0.000788880407756412\\
80.01	0.000788879605971982\\
81.01	0.000788878787147817\\
82.01	0.000788877950917871\\
83.01	0.000788877096908122\\
84.01	0.00078887622473643\\
85.01	0.000788875334012255\\
86.01	0.00078887442433658\\
87.01	0.000788873495301717\\
88.01	0.000788872546491006\\
89.01	0.000788871577478716\\
90.01	0.000788870587829853\\
91.01	0.000788869577099864\\
92.01	0.0007888685448345\\
93.01	0.000788867490569582\\
94.01	0.000788866413830766\\
95.01	0.000788865314133308\\
96.01	0.000788864190981885\\
97.01	0.000788863043870299\\
98.01	0.000788861872281251\\
99.01	0.000788860675686101\\
100.01	0.000788859453544634\\
101.01	0.000788858205304757\\
102.01	0.000788856930402268\\
103.01	0.000788855628260558\\
104.01	0.000788854298290361\\
105.01	0.000788852939889459\\
106.01	0.000788851552442375\\
107.01	0.000788850135320105\\
108.01	0.00078884868787976\\
109.01	0.000788847209464356\\
110.01	0.000788845699402378\\
111.01	0.000788844157007544\\
112.01	0.000788842581578427\\
113.01	0.000788840972398121\\
114.01	0.000788839328733875\\
115.01	0.000788837649836794\\
116.01	0.000788835934941419\\
117.01	0.000788834183265393\\
118.01	0.000788832394009001\\
119.01	0.000788830566354913\\
120.01	0.000788828699467647\\
121.01	0.000788826792493228\\
122.01	0.000788824844558807\\
123.01	0.000788822854772131\\
124.01	0.00078882082222117\\
125.01	0.000788818745973708\\
126.01	0.000788816625076768\\
127.01	0.000788814458556259\\
128.01	0.000788812245416408\\
129.01	0.000788809984639332\\
130.01	0.00078880767518451\\
131.01	0.000788805315988251\\
132.01	0.000788802905963197\\
133.01	0.000788800443997753\\
134.01	0.000788797928955589\\
135.01	0.000788795359674981\\
136.01	0.000788792734968348\\
137.01	0.000788790053621561\\
138.01	0.000788787314393425\\
139.01	0.000788784516014942\\
140.01	0.000788781657188853\\
141.01	0.000788778736588799\\
142.01	0.000788775752858769\\
143.01	0.000788772704612408\\
144.01	0.000788769590432328\\
145.01	0.000788766408869329\\
146.01	0.000788763158441766\\
147.01	0.00078875983763472\\
148.01	0.000788756444899312\\
149.01	0.000788752978651832\\
150.01	0.000788749437273041\\
151.01	0.000788745819107253\\
152.01	0.000788742122461597\\
153.01	0.000788738345605078\\
154.01	0.000788734486767721\\
155.01	0.000788730544139702\\
156.01	0.000788726515870407\\
157.01	0.000788722400067483\\
158.01	0.000788718194795915\\
159.01	0.000788713898076951\\
160.01	0.000788709507887205\\
161.01	0.000788705022157556\\
162.01	0.00078870043877209\\
163.01	0.000788695755567068\\
164.01	0.000788690970329726\\
165.01	0.000788686080797253\\
166.01	0.000788681084655537\\
167.01	0.000788675979538028\\
168.01	0.000788670763024512\\
169.01	0.000788665432639836\\
170.01	0.000788659985852697\\
171.01	0.000788654420074244\\
172.01	0.000788648732656822\\
173.01	0.000788642920892575\\
174.01	0.000788636982012052\\
175.01	0.000788630913182762\\
176.01	0.000788624711507665\\
177.01	0.000788618374023794\\
178.01	0.000788611897700556\\
179.01	0.000788605279438333\\
180.01	0.000788598516066665\\
181.01	0.000788591604342812\\
182.01	0.000788584540949889\\
183.01	0.000788577322495232\\
184.01	0.000788569945508607\\
185.01	0.000788562406440407\\
186.01	0.00078855470165975\\
187.01	0.00078854682745262\\
188.01	0.000788538780019927\\
189.01	0.000788530555475493\\
190.01	0.00078852214984401\\
191.01	0.000788513559058962\\
192.01	0.000788504778960515\\
193.01	0.000788495805293268\\
194.01	0.000788486633704068\\
195.01	0.000788477259739682\\
196.01	0.000788467678844487\\
197.01	0.000788457886358023\\
198.01	0.000788447877512555\\
199.01	0.000788437647430565\\
200.01	0.000788427191122159\\
201.01	0.000788416503482438\\
202.01	0.00078840557928876\\
203.01	0.000788394413198038\\
204.01	0.00078838299974387\\
205.01	0.000788371333333598\\
206.01	0.000788359408245455\\
207.01	0.000788347218625406\\
208.01	0.00078833475848415\\
209.01	0.000788322021693823\\
210.01	0.000788309001984828\\
211.01	0.000788295692942457\\
212.01	0.000788282088003524\\
213.01	0.000788268180452809\\
214.01	0.000788253963419512\\
215.01	0.000788239429873624\\
216.01	0.000788224572622129\\
217.01	0.000788209384305181\\
218.01	0.000788193857392228\\
219.01	0.000788177984177919\\
220.01	0.000788161756778085\\
221.01	0.000788145167125427\\
222.01	0.00078812820696531\\
223.01	0.00078811086785128\\
224.01	0.000788093141140625\\
225.01	0.000788075017989711\\
226.01	0.000788056489349239\\
227.01	0.000788037545959538\\
228.01	0.000788018178345467\\
229.01	0.00078799837681145\\
230.01	0.000787978131436272\\
231.01	0.000787957432067809\\
232.01	0.000787936268317536\\
233.01	0.000787914629555147\\
234.01	0.000787892504902665\\
235.01	0.000787869883228829\\
236.01	0.000787846753143055\\
237.01	0.000787823102989403\\
238.01	0.000787798920840367\\
239.01	0.000787774194490517\\
240.01	0.000787748911449991\\
241.01	0.00078772305893786\\
242.01	0.00078769662387536\\
243.01	0.000787669592878909\\
244.01	0.000787641952252989\\
245.01	0.000787613687982894\\
246.01	0.000787584785727274\\
247.01	0.000787555230810635\\
248.01	0.000787525008215393\\
249.01	0.000787494102574034\\
250.01	0.000787462498161017\\
251.01	0.000787430178884366\\
252.01	0.000787397128277257\\
253.01	0.000787363329489305\\
254.01	0.000787328765277663\\
255.01	0.000787293417998018\\
256.01	0.000787257269595219\\
257.01	0.000787220301593854\\
258.01	0.000787182495088598\\
259.01	0.000787143830734178\\
260.01	0.00078710428873542\\
261.01	0.000787063848836733\\
262.01	0.000787022490311684\\
263.01	0.000786980191952091\\
264.01	0.00078693693205706\\
265.01	0.000786892688421579\\
266.01	0.000786847438325204\\
267.01	0.000786801158520057\\
268.01	0.000786753825218952\\
269.01	0.000786705414082977\\
270.01	0.00078665590020908\\
271.01	0.000786605258117102\\
272.01	0.00078655346173674\\
273.01	0.000786500484394157\\
274.01	0.000786446298798269\\
275.01	0.000786390877026812\\
276.01	0.000786334190512007\\
277.01	0.000786276210026056\\
278.01	0.000786216905666182\\
279.01	0.00078615624683947\\
280.01	0.000786094202247263\\
281.01	0.000786030739869352\\
282.01	0.000785965826947682\\
283.01	0.000785899429969874\\
284.01	0.000785831514652226\\
285.01	0.000785762045922447\\
286.01	0.000785690987902011\\
287.01	0.000785618303888092\\
288.01	0.000785543956335085\\
289.01	0.00078546790683586\\
290.01	0.000785390116102453\\
291.01	0.000785310543946314\\
292.01	0.000785229149258338\\
293.01	0.000785145889988228\\
294.01	0.00078506072312348\\
295.01	0.000784973604667913\\
296.01	0.00078488448961969\\
297.01	0.000784793331948931\\
298.01	0.000784700084574588\\
299.01	0.000784604699341039\\
300.01	0.000784507126994091\\
301.01	0.000784407317156196\\
302.01	0.000784305218301451\\
303.01	0.000784200777729714\\
304.01	0.000784093941540187\\
305.01	0.000783984654604412\\
306.01	0.000783872860538581\\
307.01	0.000783758501675053\\
308.01	0.000783641519033389\\
309.01	0.000783521852290371\\
310.01	0.000783399439749493\\
311.01	0.00078327421830956\\
312.01	0.000783146123432442\\
313.01	0.00078301508910999\\
314.01	0.000782881047830098\\
315.01	0.000782743930541766\\
316.01	0.000782603666619338\\
317.01	0.000782460183825507\\
318.01	0.000782313408273569\\
319.01	0.0007821632643883\\
320.01	0.000782009674865846\\
321.01	0.00078185256063249\\
322.01	0.000781691840802004\\
323.01	0.000781527432631936\\
324.01	0.000781359251478332\\
325.01	0.00078118721074919\\
326.01	0.000781011221856488\\
327.01	0.00078083119416658\\
328.01	0.000780647034949086\\
329.01	0.000780458649324122\\
330.01	0.000780265940207759\\
331.01	0.000780068808255792\\
332.01	0.000779867151805514\\
333.01	0.000779660866815643\\
334.01	0.000779449846804213\\
335.01	0.000779233982784356\\
336.01	0.000779013163197848\\
337.01	0.000778787273846523\\
338.01	0.000778556197821188\\
339.01	0.00077831981542823\\
340.01	0.00077807800411356\\
341.01	0.000777830638384129\\
342.01	0.000777577589726542\\
343.01	0.000777318726522981\\
344.01	0.000777053913964357\\
345.01	0.000776783013960237\\
346.01	0.000776505885045875\\
347.01	0.000776222382286187\\
348.01	0.000775932357176179\\
349.01	0.000775635657538421\\
350.01	0.000775332127416773\\
351.01	0.00077502160696691\\
352.01	0.000774703932343029\\
353.01	0.000774378935581061\\
354.01	0.000774046444478005\\
355.01	0.000773706282467592\\
356.01	0.000773358268491885\\
357.01	0.000773002216868957\\
358.01	0.000772637937156579\\
359.01	0.000772265234011556\\
360.01	0.00077188390704499\\
361.01	0.000771493750673131\\
362.01	0.000771094553963665\\
363.01	0.00077068610047761\\
364.01	0.000770268168106261\\
365.01	0.000769840528903462\\
366.01	0.00076940294891264\\
367.01	0.000768955187988649\\
368.01	0.000768496999613977\\
369.01	0.000768028130709331\\
370.01	0.000767548321437874\\
371.01	0.000767057305003109\\
372.01	0.000766554807439862\\
373.01	0.000766040547398124\\
374.01	0.000765514235919773\\
375.01	0.000764975576211037\\
376.01	0.000764424263447626\\
377.01	0.000763859985052047\\
378.01	0.000763282426039914\\
379.01	0.000762691257241056\\
380.01	0.000762086116124052\\
381.01	0.00076146665513232\\
382.01	0.000760832518163562\\
383.01	0.000760183339841269\\
384.01	0.00075951874526531\\
385.01	0.000758838349755616\\
386.01	0.000758141758588808\\
387.01	0.000757428566727588\\
388.01	0.000756698358542631\\
389.01	0.000755950707526735\\
390.01	0.000755185176001079\\
391.01	0.000754401314813273\\
392.01	0.000753598663026918\\
393.01	0.000752776747602541\\
394.01	0.000751935083069472\\
395.01	0.00075107317118832\\
396.01	0.000750190500604151\\
397.01	0.000749286546489444\\
398.01	0.000748360770176862\\
399.01	0.000747412618781562\\
400.01	0.000746441524812328\\
401.01	0.000745446905771377\\
402.01	0.000744428163742374\\
403.01	0.000743384684966221\\
404.01	0.000742315839404125\\
405.01	0.000741220980287475\\
406.01	0.000740099443653964\\
407.01	0.000738950547869563\\
408.01	0.000737773593135567\\
409.01	0.000736567860980265\\
410.01	0.000735332613734411\\
411.01	0.0007340670939901\\
412.01	0.000732770524041962\\
413.01	0.000731442105310129\\
414.01	0.000730081017744163\\
415.01	0.000728686419206944\\
416.01	0.00072725744483775\\
417.01	0.000725793206393379\\
418.01	0.000724292791566437\\
419.01	0.00072275526327968\\
420.01	0.000721179658955065\\
421.01	0.000719564989756555\\
422.01	0.000717910239805197\\
423.01	0.000716214365365056\\
424.01	0.000714476293998756\\
425.01	0.00071269492369091\\
426.01	0.000710869121937891\\
427.01	0.000708997724802247\\
428.01	0.000707079535929895\\
429.01	0.000705113325528412\\
430.01	0.000703097829304186\\
431.01	0.00070103174735661\\
432.01	0.00069891374302713\\
433.01	0.000696742441700702\\
434.01	0.000694516429557755\\
435.01	0.000692234252274012\\
436.01	0.000689894413665718\\
437.01	0.000687495374278013\\
438.01	0.000685035549913665\\
439.01	0.000682513310099797\\
440.01	0.000679926976489867\\
441.01	0.000677274821198443\\
442.01	0.00067455506506625\\
443.01	0.000671765875852692\\
444.01	0.00066890536635388\\
445.01	0.000665971592443181\\
446.01	0.000662962551032277\\
447.01	0.000659876177950198\\
448.01	0.000656710345737976\\
449.01	0.000653462861356197\\
450.01	0.000650131463802535\\
451.01	0.000646713821635564\\
452.01	0.000643207530400097\\
453.01	0.000639610109947413\\
454.01	0.000635919001640934\\
455.01	0.000632131565433734\\
456.01	0.000628245076798572\\
457.01	0.000624256723488157\\
458.01	0.00062016360216313\\
459.01	0.000615962715856013\\
460.01	0.000611650986581258\\
461.01	0.000607225500882805\\
462.01	0.000602687385822366\\
463.01	0.000598088340347467\\
464.01	0.000593484151361715\\
465.01	0.000588754218462854\\
466.01	0.000583874643218333\\
467.01	0.000578921336074694\\
468.01	0.000573954233032756\\
469.01	0.000568843481288115\\
470.01	0.000563592175141936\\
471.01	0.000558201097387594\\
472.01	0.000552647359145746\\
473.01	0.000546896934269207\\
474.01	0.000540740339582005\\
475.01	0.000535169185104023\\
476.01	0.000529759474592844\\
477.01	0.000524208045358894\\
478.01	0.000518510722889253\\
479.01	0.00051266320239155\\
480.01	0.000506661047067395\\
481.01	0.00050049968717973\\
482.01	0.000494174420170959\\
483.01	0.000487680412158522\\
484.01	0.000481012701221703\\
485.01	0.000474166203010365\\
486.01	0.000467135719422815\\
487.01	0.00045991595217851\\
488.01	0.000452501535153515\\
489.01	0.00044488726797943\\
490.01	0.000437070164910716\\
491.01	0.000429044529699786\\
492.01	0.000420809292186642\\
493.01	0.000412343419361179\\
494.01	0.000403634904553368\\
495.01	0.000394677209356745\\
496.01	0.000385464091523431\\
497.01	0.000375989603330807\\
498.01	0.000366245969979332\\
499.01	0.000356221230692162\\
500.01	0.000345899599529418\\
501.01	0.000335260647672983\\
502.01	0.000324344773929204\\
503.01	0.000313173748962269\\
504.01	0.000301649210040845\\
505.01	0.000289751300194941\\
506.01	0.000277458978449925\\
507.01	0.000264749128852363\\
508.01	0.000251596268624513\\
509.01	0.00023797220579236\\
510.01	0.000223845635329447\\
511.01	0.000209181653300443\\
512.01	0.00019394110106879\\
513.01	0.000178079053299222\\
514.01	0.000161536394333217\\
515.01	0.000144182427327151\\
516.01	0.000125783370083116\\
517.01	0.000107718309217698\\
518.01	8.36312000147641e-05\\
519.01	3.99456113915325e-05\\
520.01	1.85314525037901e-06\\
521.01	0\\
522.01	0\\
523.01	1.73472347597681e-18\\
524.01	0\\
525.01	1.73472347597681e-18\\
526.01	1.73472347597681e-18\\
527.01	0\\
528.01	0\\
529.01	0\\
530.01	1.73472347597681e-18\\
531.01	0\\
532.01	1.73472347597681e-18\\
533.01	0\\
534.01	1.73472347597681e-18\\
535.01	0\\
536.01	1.73472347597681e-18\\
537.01	1.73472347597681e-18\\
538.01	1.73472347597681e-18\\
539.01	0\\
540.01	0\\
541.01	1.73472347597681e-18\\
542.01	1.73472347597681e-18\\
543.01	0\\
544.01	1.73472347597681e-18\\
545.01	0\\
546.01	1.73472347597681e-18\\
547.01	0\\
548.01	0\\
549.01	0\\
550.01	0\\
551.01	0\\
552.01	0\\
553.01	0\\
554.01	0\\
555.01	1.73472347597681e-18\\
556.01	0\\
557.01	0\\
558.01	0\\
559.01	1.73472347597681e-18\\
560.01	1.73472347597681e-18\\
561.01	1.73472347597681e-18\\
562.01	1.73472347597681e-18\\
563.01	0\\
564.01	0\\
565.01	1.73472347597681e-18\\
566.01	0\\
567.01	0\\
568.01	0\\
569.01	0\\
570.01	0\\
571.01	0\\
572.01	1.73472347597681e-18\\
573.01	0\\
574.01	1.73472347597681e-18\\
575.01	0\\
576.01	1.73472347597681e-18\\
577.01	0\\
578.01	1.73472347597681e-18\\
579.01	0\\
580.01	0\\
581.01	1.73472347597681e-18\\
582.01	0\\
583.01	0\\
584.01	0\\
585.01	0\\
586.01	0\\
587.01	0\\
588.01	0\\
589.01	0\\
590.01	0\\
591.01	0\\
592.01	1.73472347597681e-18\\
593.01	0\\
594.01	0\\
595.01	0\\
596.01	0\\
597.01	0\\
598.01	0\\
599.01	0\\
599.02	0\\
599.03	1.73472347597681e-18\\
599.04	1.73472347597681e-18\\
599.05	0\\
599.06	0\\
599.07	1.73472347597681e-18\\
599.08	0\\
599.09	0\\
599.1	0\\
599.11	1.73472347597681e-18\\
599.12	1.73472347597681e-18\\
599.13	0\\
599.14	0\\
599.15	1.73472347597681e-18\\
599.16	1.73472347597681e-18\\
599.17	1.73472347597681e-18\\
599.18	0\\
599.19	0\\
599.2	0\\
599.21	0\\
599.22	0\\
599.23	1.73472347597681e-18\\
599.24	1.73472347597681e-18\\
599.25	1.73472347597681e-18\\
599.26	1.73472347597681e-18\\
599.27	0\\
599.28	0\\
599.29	0\\
599.3	0\\
599.31	0\\
599.32	1.73472347597681e-18\\
599.33	1.73472347597681e-18\\
599.34	0\\
599.35	0\\
599.36	0\\
599.37	1.73472347597681e-18\\
599.38	0\\
599.39	0\\
599.4	0\\
599.41	0\\
599.42	0\\
599.43	0\\
599.44	0\\
599.45	0\\
599.46	1.73472347597681e-18\\
599.47	1.73472347597681e-18\\
599.48	1.73472347597681e-18\\
599.49	1.73472347597681e-18\\
599.5	1.73472347597681e-18\\
599.51	0\\
599.52	0\\
599.53	1.73472347597681e-18\\
599.54	0\\
599.55	0\\
599.56	0\\
599.57	1.73472347597681e-18\\
599.58	0\\
599.59	0\\
599.6	0\\
599.61	0\\
599.62	1.73472347597681e-18\\
599.63	0\\
599.64	1.73472347597681e-18\\
599.65	0\\
599.66	0\\
599.67	0\\
599.68	0\\
599.69	1.73472347597681e-18\\
599.7	0\\
599.71	1.73472347597681e-18\\
599.72	1.73472347597681e-18\\
599.73	1.73472347597681e-18\\
599.74	0\\
599.75	0\\
599.76	0\\
599.77	0\\
599.78	0\\
599.79	0\\
599.8	0\\
599.81	0\\
599.82	0\\
599.83	1.73472347597681e-18\\
599.84	0\\
599.85	0\\
599.86	0\\
599.87	0\\
599.88	0\\
599.89	0\\
599.9	0\\
599.91	0\\
599.92	0\\
599.93	0\\
599.94	0\\
599.95	0\\
599.96	0\\
599.97	0\\
599.98	0\\
599.99	0\\
600	0\\
};
\addplot [color=blue!80!mycolor9,solid,forget plot]
  table[row sep=crcr]{%
0.01	0.00148667125888009\\
1.01	0.00148667120328983\\
2.01	0.00148667114653197\\
3.01	0.00148667108858188\\
4.01	0.00148667102941426\\
5.01	0.00148667096900342\\
6.01	0.00148667090732303\\
7.01	0.0014866708443462\\
8.01	0.00148667078004551\\
9.01	0.00148667071439291\\
10.01	0.0014866706473598\\
11.01	0.00148667057891692\\
12.01	0.00148667050903441\\
13.01	0.00148667043768172\\
14.01	0.00148667036482778\\
15.01	0.00148667029044068\\
16.01	0.00148667021448795\\
17.01	0.00148667013693641\\
18.01	0.0014866700577521\\
19.01	0.00148666997690039\\
20.01	0.00148666989434592\\
21.01	0.00148666981005255\\
22.01	0.00148666972398332\\
23.01	0.00148666963610057\\
24.01	0.00148666954636574\\
25.01	0.00148666945473954\\
26.01	0.00148666936118173\\
27.01	0.00148666926565128\\
28.01	0.00148666916810626\\
29.01	0.00148666906850383\\
30.01	0.00148666896680022\\
31.01	0.00148666886295076\\
32.01	0.00148666875690977\\
33.01	0.00148666864863063\\
34.01	0.00148666853806568\\
35.01	0.00148666842516626\\
36.01	0.00148666830988267\\
37.01	0.00148666819216408\\
38.01	0.00148666807195865\\
39.01	0.00148666794921335\\
40.01	0.00148666782387405\\
41.01	0.00148666769588546\\
42.01	0.00148666756519102\\
43.01	0.00148666743173307\\
44.01	0.0014866672954526\\
45.01	0.00148666715628939\\
46.01	0.00148666701418188\\
47.01	0.00148666686906721\\
48.01	0.00148666672088115\\
49.01	0.00148666656955804\\
50.01	0.00148666641503089\\
51.01	0.00148666625723117\\
52.01	0.00148666609608892\\
53.01	0.00148666593153265\\
54.01	0.00148666576348933\\
55.01	0.00148666559188432\\
56.01	0.00148666541664142\\
57.01	0.00148666523768268\\
58.01	0.00148666505492855\\
59.01	0.00148666486829774\\
60.01	0.00148666467770715\\
61.01	0.00148666448307192\\
62.01	0.00148666428430532\\
63.01	0.00148666408131874\\
64.01	0.00148666387402166\\
65.01	0.00148666366232158\\
66.01	0.00148666344612398\\
67.01	0.0014866632253323\\
68.01	0.00148666299984784\\
69.01	0.00148666276956983\\
70.01	0.00148666253439522\\
71.01	0.00148666229421875\\
72.01	0.00148666204893285\\
73.01	0.00148666179842765\\
74.01	0.00148666154259082\\
75.01	0.00148666128130761\\
76.01	0.00148666101446077\\
77.01	0.00148666074193048\\
78.01	0.00148666046359427\\
79.01	0.00148666017932705\\
80.01	0.00148665988900098\\
81.01	0.00148665959248535\\
82.01	0.00148665928964671\\
83.01	0.00148665898034858\\
84.01	0.00148665866445154\\
85.01	0.00148665834181312\\
86.01	0.00148665801228771\\
87.01	0.0014866576757265\\
88.01	0.00148665733197743\\
89.01	0.0014866569808851\\
90.01	0.00148665662229066\\
91.01	0.00148665625603183\\
92.01	0.00148665588194273\\
93.01	0.00148665549985383\\
94.01	0.00148665510959186\\
95.01	0.00148665471097977\\
96.01	0.00148665430383658\\
97.01	0.0014866538879773\\
98.01	0.00148665346321296\\
99.01	0.00148665302935031\\
100.01	0.00148665258619189\\
101.01	0.00148665213353589\\
102.01	0.00148665167117602\\
103.01	0.00148665119890144\\
104.01	0.00148665071649666\\
105.01	0.00148665022374141\\
106.01	0.00148664972041055\\
107.01	0.00148664920627398\\
108.01	0.00148664868109645\\
109.01	0.00148664814463758\\
110.01	0.00148664759665158\\
111.01	0.00148664703688726\\
112.01	0.00148664646508784\\
113.01	0.00148664588099085\\
114.01	0.001486645284328\\
115.01	0.00148664467482499\\
116.01	0.0014866440522015\\
117.01	0.0014866434161709\\
118.01	0.00148664276644025\\
119.01	0.00148664210271004\\
120.01	0.00148664142467412\\
121.01	0.00148664073201952\\
122.01	0.00148664002442625\\
123.01	0.00148663930156723\\
124.01	0.00148663856310809\\
125.01	0.00148663780870695\\
126.01	0.00148663703801433\\
127.01	0.00148663625067292\\
128.01	0.00148663544631744\\
129.01	0.00148663462457442\\
130.01	0.00148663378506205\\
131.01	0.00148663292738998\\
132.01	0.0014866320511591\\
133.01	0.00148663115596133\\
134.01	0.00148663024137949\\
135.01	0.00148662930698705\\
136.01	0.0014866283523478\\
137.01	0.00148662737701587\\
138.01	0.00148662638053528\\
139.01	0.00148662536243987\\
140.01	0.00148662432225292\\
141.01	0.00148662325948703\\
142.01	0.00148662217364386\\
143.01	0.00148662106421376\\
144.01	0.00148661993067566\\
145.01	0.00148661877249675\\
146.01	0.00148661758913215\\
147.01	0.0014866163800247\\
148.01	0.00148661514460468\\
149.01	0.00148661388228951\\
150.01	0.00148661259248339\\
151.01	0.0014866112745771\\
152.01	0.00148660992794762\\
153.01	0.00148660855195781\\
154.01	0.00148660714595611\\
155.01	0.00148660570927622\\
156.01	0.00148660424123671\\
157.01	0.00148660274114071\\
158.01	0.00148660120827551\\
159.01	0.00148659964191226\\
160.01	0.00148659804130553\\
161.01	0.00148659640569296\\
162.01	0.00148659473429484\\
163.01	0.00148659302631375\\
164.01	0.00148659128093412\\
165.01	0.00148658949732178\\
166.01	0.0014865876746236\\
167.01	0.00148658581196699\\
168.01	0.00148658390845943\\
169.01	0.0014865819631881\\
170.01	0.00148657997521934\\
171.01	0.00148657794359812\\
172.01	0.00148657586734763\\
173.01	0.00148657374546878\\
174.01	0.00148657157693952\\
175.01	0.00148656936071453\\
176.01	0.00148656709572455\\
177.01	0.00148656478087578\\
178.01	0.00148656241504947\\
179.01	0.0014865599971011\\
180.01	0.00148655752586004\\
181.01	0.00148655500012872\\
182.01	0.00148655241868216\\
183.01	0.00148654978026723\\
184.01	0.00148654708360201\\
185.01	0.00148654432737518\\
186.01	0.00148654151024523\\
187.01	0.00148653863083984\\
188.01	0.00148653568775512\\
189.01	0.00148653267955487\\
190.01	0.00148652960476984\\
191.01	0.00148652646189695\\
192.01	0.00148652324939849\\
193.01	0.00148651996570133\\
194.01	0.00148651660919605\\
195.01	0.00148651317823616\\
196.01	0.00148650967113715\\
197.01	0.00148650608617565\\
198.01	0.00148650242158854\\
199.01	0.00148649867557194\\
200.01	0.00148649484628034\\
201.01	0.00148649093182559\\
202.01	0.00148648693027587\\
203.01	0.00148648283965474\\
204.01	0.00148647865794004\\
205.01	0.00148647438306282\\
206.01	0.00148647001290629\\
207.01	0.00148646554530464\\
208.01	0.00148646097804195\\
209.01	0.001486456308851\\
210.01	0.00148645153541199\\
211.01	0.00148644665535152\\
212.01	0.00148644166624103\\
213.01	0.00148643656559584\\
214.01	0.00148643135087357\\
215.01	0.00148642601947294\\
216.01	0.00148642056873238\\
217.01	0.00148641499592855\\
218.01	0.00148640929827498\\
219.01	0.00148640347292058\\
220.01	0.00148639751694806\\
221.01	0.00148639142737249\\
222.01	0.00148638520113966\\
223.01	0.00148637883512451\\
224.01	0.00148637232612942\\
225.01	0.00148636567088258\\
226.01	0.00148635886603625\\
227.01	0.00148635190816494\\
228.01	0.00148634479376369\\
229.01	0.00148633751924618\\
230.01	0.00148633008094285\\
231.01	0.00148632247509892\\
232.01	0.00148631469787253\\
233.01	0.0014863067453326\\
234.01	0.00148629861345685\\
235.01	0.00148629029812964\\
236.01	0.00148628179513982\\
237.01	0.00148627310017856\\
238.01	0.00148626420883703\\
239.01	0.00148625511660417\\
240.01	0.00148624581886427\\
241.01	0.00148623631089461\\
242.01	0.00148622658786293\\
243.01	0.00148621664482497\\
244.01	0.0014862064767219\\
245.01	0.00148619607837764\\
246.01	0.0014861854444962\\
247.01	0.00148617456965896\\
248.01	0.00148616344832181\\
249.01	0.00148615207481229\\
250.01	0.00148614044332668\\
251.01	0.00148612854792698\\
252.01	0.00148611638253788\\
253.01	0.0014861039409436\\
254.01	0.00148609121678466\\
255.01	0.00148607820355467\\
256.01	0.00148606489459698\\
257.01	0.00148605128310123\\
258.01	0.00148603736209989\\
259.01	0.00148602312446468\\
260.01	0.00148600856290296\\
261.01	0.00148599366995398\\
262.01	0.00148597843798509\\
263.01	0.0014859628591878\\
264.01	0.00148594692557395\\
265.01	0.00148593062897149\\
266.01	0.00148591396102048\\
267.01	0.00148589691316872\\
268.01	0.00148587947666754\\
269.01	0.00148586164256732\\
270.01	0.00148584340171296\\
271.01	0.00148582474473932\\
272.01	0.00148580566206644\\
273.01	0.00148578614389478\\
274.01	0.00148576618020022\\
275.01	0.00148574576072908\\
276.01	0.00148572487499295\\
277.01	0.00148570351226347\\
278.01	0.00148568166156683\\
279.01	0.00148565931167841\\
280.01	0.00148563645111707\\
281.01	0.00148561306813944\\
282.01	0.00148558915073401\\
283.01	0.00148556468661512\\
284.01	0.00148553966321682\\
285.01	0.00148551406768654\\
286.01	0.00148548788687875\\
287.01	0.00148546110734816\\
288.01	0.0014854337153432\\
289.01	0.00148540569679897\\
290.01	0.00148537703733022\\
291.01	0.00148534772222409\\
292.01	0.0014853177364327\\
293.01	0.0014852870645655\\
294.01	0.00148525569088158\\
295.01	0.0014852235992816\\
296.01	0.00148519077329961\\
297.01	0.0014851571960947\\
298.01	0.00148512285044237\\
299.01	0.00148508771872568\\
300.01	0.00148505178292618\\
301.01	0.00148501502461468\\
302.01	0.00148497742494158\\
303.01	0.00148493896462712\\
304.01	0.00148489962395129\\
305.01	0.00148485938274348\\
306.01	0.00148481822037178\\
307.01	0.00148477611573207\\
308.01	0.00148473304723673\\
309.01	0.00148468899280306\\
310.01	0.00148464392984129\\
311.01	0.00148459783524234\\
312.01	0.00148455068536518\\
313.01	0.00148450245602375\\
314.01	0.00148445312247352\\
315.01	0.00148440265939768\\
316.01	0.00148435104089279\\
317.01	0.00148429824045417\\
318.01	0.00148424423096053\\
319.01	0.00148418898465847\\
320.01	0.00148413247314618\\
321.01	0.00148407466735677\\
322.01	0.00148401553754105\\
323.01	0.00148395505324966\\
324.01	0.00148389318331468\\
325.01	0.00148382989583073\\
326.01	0.00148376515813515\\
327.01	0.00148369893678785\\
328.01	0.00148363119755033\\
329.01	0.00148356190536394\\
330.01	0.00148349102432757\\
331.01	0.00148341851767448\\
332.01	0.00148334434774837\\
333.01	0.00148326847597874\\
334.01	0.00148319086285529\\
335.01	0.00148311146790156\\
336.01	0.00148303024964776\\
337.01	0.00148294716560248\\
338.01	0.00148286217222378\\
339.01	0.00148277522488909\\
340.01	0.00148268627786429\\
341.01	0.00148259528427165\\
342.01	0.00148250219605695\\
343.01	0.00148240696395542\\
344.01	0.00148230953745664\\
345.01	0.0014822098647684\\
346.01	0.00148210789277943\\
347.01	0.00148200356702099\\
348.01	0.00148189683162741\\
349.01	0.00148178762929527\\
350.01	0.00148167590124167\\
351.01	0.00148156158716107\\
352.01	0.00148144462518104\\
353.01	0.0014813249518167\\
354.01	0.00148120250192407\\
355.01	0.00148107720865193\\
356.01	0.00148094900339259\\
357.01	0.0014808178157313\\
358.01	0.00148068357339438\\
359.01	0.00148054620219602\\
360.01	0.00148040562598376\\
361.01	0.00148026176658271\\
362.01	0.00148011454373837\\
363.01	0.00147996387505812\\
364.01	0.00147980967595145\\
365.01	0.0014796518595688\\
366.01	0.00147949033673916\\
367.01	0.00147932501590623\\
368.01	0.00147915580306354\\
369.01	0.00147898260168817\\
370.01	0.0014788053126732\\
371.01	0.00147862383425929\\
372.01	0.00147843806196481\\
373.01	0.00147824788851521\\
374.01	0.00147805320377142\\
375.01	0.00147785389465882\\
376.01	0.00147764984510826\\
377.01	0.0014774409361123\\
378.01	0.00147722704622314\\
379.01	0.0014770080491819\\
380.01	0.00147678381519769\\
381.01	0.00147655421277934\\
382.01	0.00147631910696703\\
383.01	0.00147607835922564\\
384.01	0.00147583182734266\\
385.01	0.00147557936532291\\
386.01	0.00147532082327951\\
387.01	0.00147505604732124\\
388.01	0.00147478487943591\\
389.01	0.00147450715736971\\
390.01	0.00147422271450212\\
391.01	0.00147393137971652\\
392.01	0.00147363297726579\\
393.01	0.00147332732663323\\
394.01	0.00147301424238788\\
395.01	0.00147269353403461\\
396.01	0.00147236500585822\\
397.01	0.00147202845676139\\
398.01	0.0014716836800962\\
399.01	0.00147133046348868\\
400.01	0.00147096858865618\\
401.01	0.00147059783121699\\
402.01	0.00147021796049185\\
403.01	0.00146982873929673\\
404.01	0.00146942992372654\\
405.01	0.00146902126292897\\
406.01	0.0014686024988681\\
407.01	0.00146817336607682\\
408.01	0.00146773359139768\\
409.01	0.00146728289371101\\
410.01	0.00146682098364997\\
411.01	0.00146634756330103\\
412.01	0.00146586232588942\\
413.01	0.00146536495544833\\
414.01	0.00146485512647047\\
415.01	0.00146433250354117\\
416.01	0.00146379674095133\\
417.01	0.00146324748228891\\
418.01	0.00146268436000735\\
419.01	0.00146210699496918\\
420.01	0.00146151499596284\\
421.01	0.00146090795919073\\
422.01	0.00146028546772621\\
423.01	0.00145964709093706\\
424.01	0.00145899238387248\\
425.01	0.001458320886611\\
426.01	0.00145763212356564\\
427.01	0.00145692560274273\\
428.01	0.00145620081495063\\
429.01	0.00145545723295329\\
430.01	0.00145469431056425\\
431.01	0.0014539114816752\\
432.01	0.00145310815921283\\
433.01	0.00145228373401726\\
434.01	0.00145143757363426\\
435.01	0.00145056902101202\\
436.01	0.00144967739309335\\
437.01	0.00144876197929147\\
438.01	0.00144782203983707\\
439.01	0.00144685680398216\\
440.01	0.00144586546804417\\
441.01	0.00144484719327167\\
442.01	0.00144380110351006\\
443.01	0.00144272628264251\\
444.01	0.00144162177177731\\
445.01	0.00144048656614887\\
446.01	0.00143931961169353\\
447.01	0.00143811980125565\\
448.01	0.00143688597037131\\
449.01	0.00143561689256835\\
450.01	0.00143431127411032\\
451.01	0.00143296774809906\\
452.01	0.00143158486783438\\
453.01	0.00143016109931075\\
454.01	0.00142869481270724\\
455.01	0.00142718427269863\\
456.01	0.00142562762738081\\
457.01	0.00142402289556433\\
458.01	0.00142236795216188\\
459.01	0.0014206605116214\\
460.01	0.00141889811261707\\
461.01	0.00141707814489417\\
462.01	0.00141519830956914\\
463.01	0.00141325871671744\\
464.01	0.00141125511419306\\
465.01	0.00140917775241511\\
466.01	0.00140700083786376\\
467.01	0.00140473385456332\\
468.01	0.00140266671023066\\
469.01	0.00140051700651366\\
470.01	0.00139828407406295\\
471.01	0.00139598936318125\\
472.01	0.00139359556751587\\
473.01	0.00139107893837405\\
474.01	0.00138819797988816\\
475.01	0.00138318856130055\\
476.01	0.00137765691238938\\
477.01	0.00137198396486754\\
478.01	0.00136616572388151\\
479.01	0.00136019805855384\\
480.01	0.00135407669562829\\
481.01	0.00134779721267852\\
482.01	0.00134135503083478\\
483.01	0.00133474540697633\\
484.01	0.00132796342532756\\
485.01	0.00132100398838981\\
486.01	0.00131386180716428\\
487.01	0.00130653139098239\\
488.01	0.00129900704106502\\
489.01	0.00129128288397996\\
490.01	0.00128335298276893\\
491.01	0.00127521089758948\\
492.01	0.00126685010857471\\
493.01	0.00125826233474716\\
494.01	0.0012494401758256\\
495.01	0.00124037612333986\\
496.01	0.00123106232795712\\
497.01	0.0012214905381718\\
498.01	0.00121165191211895\\
499.01	0.00120153667441694\\
500.01	0.00119113210578245\\
501.01	0.00118041491583364\\
502.01	0.0011694069334897\\
503.01	0.00115825522553226\\
504.01	0.00114678010314293\\
505.01	0.00113496280434723\\
506.01	0.00112278951008377\\
507.01	0.00111024557025976\\
508.01	0.00109731542923216\\
509.01	0.00108398254178966\\
510.01	0.00107022927732357\\
511.01	0.00105603680460775\\
512.01	0.00104138490774409\\
513.01	0.00102625130966424\\
514.01	0.00101060650128075\\
515.01	0.00099436616793677\\
516.01	0.000977084734376887\\
517.01	0.000960691326306129\\
518.01	0.000944700367329772\\
519.01	0.000927791988661471\\
520.01	0.000911023289324185\\
521.01	0.000894775923625113\\
522.01	0.000878423206032023\\
523.01	0.000861850718137483\\
524.01	0.000844707102916892\\
525.01	0.000826955246655789\\
526.01	0.000808553839524584\\
527.01	0.000789456733757346\\
528.01	0.000769612375607442\\
529.01	0.000748963556818137\\
530.01	0.000727441266626895\\
531.01	0.000704969255035168\\
532.01	0.00068145902477487\\
533.01	0.000656768768742986\\
534.01	0.000629864955380999\\
535.01	0.000582592877329985\\
536.01	0.000519929998022089\\
537.01	0.000456021358906175\\
538.01	0.000390832352851327\\
539.01	0.000324198365391418\\
540.01	0.000255826929879356\\
541.01	0.000186537130846801\\
542.01	0.000118055601462464\\
543.01	4.8892884100234e-05\\
544.01	1.73472347597681e-18\\
545.01	0\\
546.01	1.73472347597681e-18\\
547.01	0\\
548.01	0\\
549.01	0\\
550.01	0\\
551.01	0\\
552.01	0\\
553.01	0\\
554.01	0\\
555.01	1.73472347597681e-18\\
556.01	0\\
557.01	0\\
558.01	0\\
559.01	1.73472347597681e-18\\
560.01	1.73472347597681e-18\\
561.01	1.73472347597681e-18\\
562.01	1.73472347597681e-18\\
563.01	0\\
564.01	0\\
565.01	1.73472347597681e-18\\
566.01	0\\
567.01	0\\
568.01	0\\
569.01	0\\
570.01	0\\
571.01	0\\
572.01	1.73472347597681e-18\\
573.01	0\\
574.01	1.73472347597681e-18\\
575.01	0\\
576.01	1.73472347597681e-18\\
577.01	0\\
578.01	1.73472347597681e-18\\
579.01	0\\
580.01	0\\
581.01	1.73472347597681e-18\\
582.01	0\\
583.01	0\\
584.01	0\\
585.01	0\\
586.01	0\\
587.01	0\\
588.01	0\\
589.01	0\\
590.01	0\\
591.01	0\\
592.01	1.73472347597681e-18\\
593.01	0\\
594.01	0\\
595.01	0\\
596.01	0\\
597.01	0\\
598.01	0\\
599.01	0\\
599.02	0\\
599.03	1.73472347597681e-18\\
599.04	1.73472347597681e-18\\
599.05	0\\
599.06	0\\
599.07	1.73472347597681e-18\\
599.08	0\\
599.09	0\\
599.1	0\\
599.11	1.73472347597681e-18\\
599.12	1.73472347597681e-18\\
599.13	0\\
599.14	0\\
599.15	1.73472347597681e-18\\
599.16	1.73472347597681e-18\\
599.17	1.73472347597681e-18\\
599.18	0\\
599.19	0\\
599.2	0\\
599.21	0\\
599.22	0\\
599.23	1.73472347597681e-18\\
599.24	1.73472347597681e-18\\
599.25	1.73472347597681e-18\\
599.26	1.73472347597681e-18\\
599.27	0\\
599.28	0\\
599.29	0\\
599.3	0\\
599.31	0\\
599.32	1.73472347597681e-18\\
599.33	1.73472347597681e-18\\
599.34	0\\
599.35	0\\
599.36	0\\
599.37	1.73472347597681e-18\\
599.38	0\\
599.39	0\\
599.4	0\\
599.41	0\\
599.42	0\\
599.43	0\\
599.44	0\\
599.45	0\\
599.46	1.73472347597681e-18\\
599.47	1.73472347597681e-18\\
599.48	1.73472347597681e-18\\
599.49	1.73472347597681e-18\\
599.5	1.73472347597681e-18\\
599.51	0\\
599.52	0\\
599.53	1.73472347597681e-18\\
599.54	0\\
599.55	0\\
599.56	0\\
599.57	1.73472347597681e-18\\
599.58	0\\
599.59	0\\
599.6	0\\
599.61	0\\
599.62	1.73472347597681e-18\\
599.63	0\\
599.64	1.73472347597681e-18\\
599.65	0\\
599.66	0\\
599.67	0\\
599.68	0\\
599.69	1.73472347597681e-18\\
599.7	0\\
599.71	1.73472347597681e-18\\
599.72	1.73472347597681e-18\\
599.73	1.73472347597681e-18\\
599.74	0\\
599.75	0\\
599.76	0\\
599.77	0\\
599.78	0\\
599.79	0\\
599.8	0\\
599.81	0\\
599.82	0\\
599.83	1.73472347597681e-18\\
599.84	0\\
599.85	0\\
599.86	0\\
599.87	0\\
599.88	0\\
599.89	0\\
599.9	0\\
599.91	0\\
599.92	0\\
599.93	0\\
599.94	0\\
599.95	0\\
599.96	0\\
599.97	0\\
599.98	0\\
599.99	0\\
600	0\\
};
\addplot [color=blue,solid,forget plot]
  table[row sep=crcr]{%
0.01	0.00240288571436453\\
1.01	0.00240288569658394\\
2.01	0.00240288567842936\\
3.01	0.00240288565989285\\
4.01	0.00240288564096636\\
5.01	0.00240288562164156\\
6.01	0.00240288560191004\\
7.01	0.00240288558176315\\
8.01	0.00240288556119211\\
9.01	0.00240288554018791\\
10.01	0.00240288551874136\\
11.01	0.00240288549684306\\
12.01	0.00240288547448342\\
13.01	0.00240288545165266\\
14.01	0.00240288542834077\\
15.01	0.0024028854045375\\
16.01	0.00240288538023244\\
17.01	0.00240288535541497\\
18.01	0.00240288533007409\\
19.01	0.00240288530419873\\
20.01	0.00240288527777753\\
21.01	0.00240288525079886\\
22.01	0.00240288522325087\\
23.01	0.00240288519512146\\
24.01	0.00240288516639823\\
25.01	0.00240288513706853\\
26.01	0.00240288510711945\\
27.01	0.00240288507653782\\
28.01	0.00240288504531012\\
29.01	0.00240288501342262\\
30.01	0.00240288498086125\\
31.01	0.00240288494761162\\
32.01	0.00240288491365906\\
33.01	0.00240288487898859\\
34.01	0.00240288484358486\\
35.01	0.00240288480743225\\
36.01	0.00240288477051476\\
37.01	0.00240288473281605\\
38.01	0.00240288469431945\\
39.01	0.0024028846550079\\
40.01	0.00240288461486397\\
41.01	0.00240288457386991\\
42.01	0.00240288453200749\\
43.01	0.00240288448925817\\
44.01	0.00240288444560298\\
45.01	0.00240288440102251\\
46.01	0.00240288435549698\\
47.01	0.00240288430900614\\
48.01	0.00240288426152931\\
49.01	0.00240288421304538\\
50.01	0.00240288416353279\\
51.01	0.00240288411296945\\
52.01	0.00240288406133287\\
53.01	0.00240288400859999\\
54.01	0.00240288395474732\\
55.01	0.00240288389975084\\
56.01	0.00240288384358595\\
57.01	0.00240288378622761\\
58.01	0.00240288372765017\\
59.01	0.00240288366782744\\
60.01	0.00240288360673266\\
61.01	0.00240288354433849\\
62.01	0.00240288348061699\\
63.01	0.00240288341553959\\
64.01	0.00240288334907712\\
65.01	0.00240288328119979\\
66.01	0.00240288321187712\\
67.01	0.00240288314107799\\
68.01	0.0024028830687706\\
69.01	0.00240288299492244\\
70.01	0.00240288291950029\\
71.01	0.00240288284247022\\
72.01	0.00240288276379753\\
73.01	0.00240288268344679\\
74.01	0.00240288260138179\\
75.01	0.0024028825175655\\
76.01	0.00240288243196009\\
77.01	0.00240288234452692\\
78.01	0.00240288225522649\\
79.01	0.00240288216401844\\
80.01	0.00240288207086146\\
81.01	0.00240288197571344\\
82.01	0.0024028818785313\\
83.01	0.00240288177927093\\
84.01	0.00240288167788737\\
85.01	0.00240288157433463\\
86.01	0.00240288146856569\\
87.01	0.00240288136053251\\
88.01	0.00240288125018597\\
89.01	0.0024028811374759\\
90.01	0.002402881022351\\
91.01	0.00240288090475882\\
92.01	0.0024028807846458\\
93.01	0.00240288066195717\\
94.01	0.00240288053663695\\
95.01	0.00240288040862791\\
96.01	0.0024028802778716\\
97.01	0.00240288014430822\\
98.01	0.00240288000787666\\
99.01	0.00240287986851449\\
100.01	0.00240287972615789\\
101.01	0.00240287958074156\\
102.01	0.00240287943219884\\
103.01	0.00240287928046153\\
104.01	0.00240287912545996\\
105.01	0.00240287896712286\\
106.01	0.00240287880537744\\
107.01	0.00240287864014922\\
108.01	0.00240287847136214\\
109.01	0.00240287829893839\\
110.01	0.00240287812279843\\
111.01	0.00240287794286099\\
112.01	0.00240287775904297\\
113.01	0.00240287757125939\\
114.01	0.00240287737942339\\
115.01	0.0024028771834462\\
116.01	0.00240287698323703\\
117.01	0.0024028767787031\\
118.01	0.00240287656974948\\
119.01	0.00240287635627922\\
120.01	0.00240287613819315\\
121.01	0.00240287591538986\\
122.01	0.00240287568776571\\
123.01	0.00240287545521472\\
124.01	0.00240287521762849\\
125.01	0.0024028749748963\\
126.01	0.00240287472690484\\
127.01	0.00240287447353829\\
128.01	0.00240287421467826\\
129.01	0.00240287395020368\\
130.01	0.00240287367999072\\
131.01	0.00240287340391281\\
132.01	0.00240287312184053\\
133.01	0.00240287283364155\\
134.01	0.00240287253918049\\
135.01	0.00240287223831905\\
136.01	0.00240287193091568\\
137.01	0.00240287161682571\\
138.01	0.0024028712959012\\
139.01	0.00240287096799085\\
140.01	0.00240287063293997\\
141.01	0.00240287029059032\\
142.01	0.00240286994078015\\
143.01	0.00240286958334401\\
144.01	0.00240286921811269\\
145.01	0.00240286884491319\\
146.01	0.0024028684635686\\
147.01	0.00240286807389793\\
148.01	0.00240286767571614\\
149.01	0.002402867268834\\
150.01	0.00240286685305795\\
151.01	0.00240286642819005\\
152.01	0.00240286599402787\\
153.01	0.00240286555036435\\
154.01	0.00240286509698779\\
155.01	0.00240286463368159\\
156.01	0.00240286416022426\\
157.01	0.00240286367638928\\
158.01	0.00240286318194495\\
159.01	0.00240286267665429\\
160.01	0.00240286216027489\\
161.01	0.0024028616325589\\
162.01	0.0024028610932527\\
163.01	0.00240286054209695\\
164.01	0.00240285997882638\\
165.01	0.00240285940316962\\
166.01	0.00240285881484913\\
167.01	0.00240285821358105\\
168.01	0.00240285759907495\\
169.01	0.00240285697103376\\
170.01	0.0024028563291537\\
171.01	0.00240285567312395\\
172.01	0.00240285500262658\\
173.01	0.00240285431733637\\
174.01	0.00240285361692066\\
175.01	0.00240285290103911\\
176.01	0.00240285216934366\\
177.01	0.00240285142147813\\
178.01	0.00240285065707829\\
179.01	0.00240284987577142\\
180.01	0.00240284907717633\\
181.01	0.00240284826090299\\
182.01	0.00240284742655246\\
183.01	0.00240284657371658\\
184.01	0.00240284570197779\\
185.01	0.00240284481090893\\
186.01	0.00240284390007302\\
187.01	0.00240284296902298\\
188.01	0.00240284201730143\\
189.01	0.00240284104444045\\
190.01	0.00240284004996135\\
191.01	0.00240283903337435\\
192.01	0.0024028379941784\\
193.01	0.00240283693186089\\
194.01	0.00240283584589736\\
195.01	0.00240283473575125\\
196.01	0.0024028336008736\\
197.01	0.00240283244070274\\
198.01	0.00240283125466409\\
199.01	0.00240283004216972\\
200.01	0.00240282880261815\\
201.01	0.00240282753539396\\
202.01	0.00240282623986752\\
203.01	0.00240282491539463\\
204.01	0.00240282356131619\\
205.01	0.00240282217695783\\
206.01	0.00240282076162963\\
207.01	0.00240281931462561\\
208.01	0.00240281783522351\\
209.01	0.00240281632268436\\
210.01	0.00240281477625204\\
211.01	0.00240281319515294\\
212.01	0.00240281157859556\\
213.01	0.00240280992577001\\
214.01	0.0024028082358477\\
215.01	0.00240280650798084\\
216.01	0.00240280474130197\\
217.01	0.00240280293492357\\
218.01	0.00240280108793759\\
219.01	0.00240279919941486\\
220.01	0.00240279726840477\\
221.01	0.00240279529393467\\
222.01	0.00240279327500939\\
223.01	0.00240279121061069\\
224.01	0.00240278909969677\\
225.01	0.0024027869412017\\
226.01	0.00240278473403488\\
227.01	0.00240278247708047\\
228.01	0.00240278016919677\\
229.01	0.00240277780921569\\
230.01	0.0024027753959421\\
231.01	0.00240277292815322\\
232.01	0.00240277040459795\\
233.01	0.00240276782399633\\
234.01	0.00240276518503872\\
235.01	0.00240276248638527\\
236.01	0.00240275972666509\\
237.01	0.00240275690447568\\
238.01	0.0024027540183821\\
239.01	0.00240275106691628\\
240.01	0.00240274804857626\\
241.01	0.00240274496182537\\
242.01	0.00240274180509153\\
243.01	0.00240273857676635\\
244.01	0.00240273527520439\\
245.01	0.00240273189872223\\
246.01	0.00240272844559766\\
247.01	0.00240272491406882\\
248.01	0.00240272130233324\\
249.01	0.002402717608547\\
250.01	0.00240271383082367\\
251.01	0.00240270996723346\\
252.01	0.00240270601580224\\
253.01	0.00240270197451042\\
254.01	0.00240269784129208\\
255.01	0.00240269361403378\\
256.01	0.00240268929057362\\
257.01	0.00240268486870005\\
258.01	0.00240268034615077\\
259.01	0.00240267572061163\\
260.01	0.00240267098971545\\
261.01	0.00240266615104076\\
262.01	0.00240266120211068\\
263.01	0.00240265614039162\\
264.01	0.00240265096329199\\
265.01	0.00240264566816094\\
266.01	0.00240264025228699\\
267.01	0.0024026347128967\\
268.01	0.00240262904715328\\
269.01	0.00240262325215511\\
270.01	0.00240261732493434\\
271.01	0.00240261126245542\\
272.01	0.00240260506161346\\
273.01	0.00240259871923284\\
274.01	0.00240259223206548\\
275.01	0.00240258559678927\\
276.01	0.00240257881000639\\
277.01	0.0024025718682416\\
278.01	0.00240256476794048\\
279.01	0.00240255750546768\\
280.01	0.00240255007710502\\
281.01	0.00240254247904967\\
282.01	0.00240253470741222\\
283.01	0.00240252675821469\\
284.01	0.00240251862738849\\
285.01	0.00240251031077243\\
286.01	0.00240250180411052\\
287.01	0.00240249310304982\\
288.01	0.00240248420313826\\
289.01	0.00240247509982224\\
290.01	0.00240246578844441\\
291.01	0.0024024562642412\\
292.01	0.00240244652234034\\
293.01	0.00240243655775832\\
294.01	0.00240242636539788\\
295.01	0.00240241594004516\\
296.01	0.00240240527636709\\
297.01	0.00240239436890853\\
298.01	0.00240238321208935\\
299.01	0.00240237180020147\\
300.01	0.0024023601274057\\
301.01	0.00240234818772876\\
302.01	0.00240233597505979\\
303.01	0.00240232348314723\\
304.01	0.00240231070559523\\
305.01	0.00240229763586016\\
306.01	0.00240228426724694\\
307.01	0.00240227059290527\\
308.01	0.00240225660582575\\
309.01	0.00240224229883593\\
310.01	0.00240222766459611\\
311.01	0.00240221269559514\\
312.01	0.00240219738414605\\
313.01	0.00240218172238142\\
314.01	0.00240216570224893\\
315.01	0.00240214931550629\\
316.01	0.00240213255371653\\
317.01	0.00240211540824264\\
318.01	0.00240209787024245\\
319.01	0.00240207993066313\\
320.01	0.0024020615802355\\
321.01	0.00240204280946832\\
322.01	0.00240202360864218\\
323.01	0.00240200396780334\\
324.01	0.0024019838767574\\
325.01	0.00240196332506262\\
326.01	0.00240194230202314\\
327.01	0.00240192079668196\\
328.01	0.00240189879781362\\
329.01	0.00240187629391681\\
330.01	0.00240185327320662\\
331.01	0.0024018297236065\\
332.01	0.00240180563274017\\
333.01	0.00240178098792303\\
334.01	0.00240175577615351\\
335.01	0.00240172998410406\\
336.01	0.00240170359811179\\
337.01	0.00240167660416911\\
338.01	0.00240164898791365\\
339.01	0.00240162073461831\\
340.01	0.00240159182918077\\
341.01	0.00240156225611275\\
342.01	0.00240153199952899\\
343.01	0.0024015010431359\\
344.01	0.0024014693702198\\
345.01	0.00240143696363506\\
346.01	0.00240140380579167\\
347.01	0.00240136987864255\\
348.01	0.00240133516367057\\
349.01	0.00240129964187519\\
350.01	0.00240126329375869\\
351.01	0.00240122609931209\\
352.01	0.0024011880380007\\
353.01	0.00240114908874926\\
354.01	0.00240110922992672\\
355.01	0.00240106843933059\\
356.01	0.00240102669417101\\
357.01	0.00240098397105425\\
358.01	0.00240094024596589\\
359.01	0.00240089549425364\\
360.01	0.00240084969060961\\
361.01	0.00240080280905227\\
362.01	0.00240075482290794\\
363.01	0.00240070570479182\\
364.01	0.00240065542658876\\
365.01	0.00240060395943334\\
366.01	0.00240055127368982\\
367.01	0.00240049733893152\\
368.01	0.00240044212391992\\
369.01	0.00240038559658326\\
370.01	0.00240032772399498\\
371.01	0.00240026847235161\\
372.01	0.00240020780695057\\
373.01	0.00240014569216741\\
374.01	0.00240008209143306\\
375.01	0.00240001696721089\\
376.01	0.0023999502809769\\
377.01	0.00239988199321554\\
378.01	0.00239981206340619\\
379.01	0.00239974044978371\\
380.01	0.00239966710956389\\
381.01	0.0023995919988901\\
382.01	0.00239951507270891\\
383.01	0.00239943628473266\\
384.01	0.00239935558740052\\
385.01	0.00239927293183783\\
386.01	0.00239918826781396\\
387.01	0.00239910154369817\\
388.01	0.00239901270641361\\
389.01	0.00239892170138943\\
390.01	0.00239882847251061\\
391.01	0.00239873296206552\\
392.01	0.00239863511069123\\
393.01	0.00239853485731606\\
394.01	0.00239843213909956\\
395.01	0.00239832689136959\\
396.01	0.00239821904755624\\
397.01	0.0023981085391226\\
398.01	0.00239799529549201\\
399.01	0.00239787924397152\\
400.01	0.00239776030967159\\
401.01	0.00239763841542143\\
402.01	0.00239751348167983\\
403.01	0.00239738542644139\\
404.01	0.00239725416513736\\
405.01	0.0023971196105312\\
406.01	0.00239698167260827\\
407.01	0.00239684025845907\\
408.01	0.00239669527215606\\
409.01	0.00239654661462303\\
410.01	0.0023963941834969\\
411.01	0.00239623787298124\\
412.01	0.0023960775736909\\
413.01	0.00239591317248699\\
414.01	0.00239574455230188\\
415.01	0.00239557159195282\\
416.01	0.00239539416594393\\
417.01	0.00239521214425515\\
418.01	0.00239502539211761\\
419.01	0.00239483376977369\\
420.01	0.00239463713222116\\
421.01	0.00239443532893965\\
422.01	0.00239422820359799\\
423.01	0.00239401559374111\\
424.01	0.00239379733045407\\
425.01	0.0023935732380018\\
426.01	0.00239334313344181\\
427.01	0.0023931068262076\\
428.01	0.00239286411765984\\
429.01	0.00239261480060215\\
430.01	0.00239235865875773\\
431.01	0.00239209546620304\\
432.01	0.00239182498675366\\
433.01	0.00239154697329714\\
434.01	0.00239126116706697\\
435.01	0.00239096729685058\\
436.01	0.00239066507812369\\
437.01	0.00239035421210179\\
438.01	0.00239003438469846\\
439.01	0.00238970526537812\\
440.01	0.00238936650588943\\
441.01	0.00238901773886284\\
442.01	0.00238865857625296\\
443.01	0.00238828860760366\\
444.01	0.00238790739810927\\
445.01	0.00238751448644124\\
446.01	0.00238710938230342\\
447.01	0.00238669156367283\\
448.01	0.00238626047367503\\
449.01	0.00238581551703253\\
450.01	0.00238535605601448\\
451.01	0.00238488140580067\\
452.01	0.00238439082915637\\
453.01	0.00238388353029419\\
454.01	0.00238335864777369\\
455.01	0.00238281524625954\\
456.01	0.00238225230692208\\
457.01	0.00238166871622011\\
458.01	0.00238106325275859\\
459.01	0.00238043457192086\\
460.01	0.00237978118854724\\
461.01	0.00237910146288947\\
462.01	0.0023783936176163\\
463.01	0.00237765573008159\\
464.01	0.00237688521573682\\
465.01	0.00237607804645507\\
466.01	0.00237521680126685\\
467.01	0.00237406363369648\\
468.01	0.0023723996421659\\
469.01	0.00237068404364005\\
470.01	0.00236890837206177\\
471.01	0.002367135533743\\
472.01	0.0023653201884989\\
473.01	0.00236344932990166\\
474.01	0.00236148983298677\\
475.01	0.00235938935000563\\
476.01	0.00235722803957903\\
477.01	0.0023550099223541\\
478.01	0.00235273318309994\\
479.01	0.00235039592526171\\
480.01	0.00234799616578743\\
481.01	0.00234553182953742\\
482.01	0.00234300074323746\\
483.01	0.0023404006289316\\
484.01	0.00233772909688802\\
485.01	0.00233498363790802\\
486.01	0.00233216161499807\\
487.01	0.00232926025447013\\
488.01	0.0023262766372841\\
489.01	0.00232320769423005\\
490.01	0.00232005019424852\\
491.01	0.00231680071685399\\
492.01	0.00231345562778019\\
493.01	0.00231001101689176\\
494.01	0.00230646282222201\\
495.01	0.00230280674238715\\
496.01	0.00229903821072988\\
497.01	0.00229515236771408\\
498.01	0.00229114400296414\\
499.01	0.00228700730893898\\
500.01	0.00228273423043378\\
501.01	0.00227830360804612\\
502.01	0.00227366869047343\\
503.01	0.00226919996914999\\
504.01	0.00226462357559349\\
505.01	0.00225988950061276\\
506.01	0.0022549887410712\\
507.01	0.00224991140125685\\
508.01	0.00224464656536145\\
509.01	0.00223918214737417\\
510.01	0.00223350471338938\\
511.01	0.00222759926749472\\
512.01	0.00222144896884423\\
513.01	0.00221503455724645\\
514.01	0.00220833150147027\\
515.01	0.00220128357537459\\
516.01	0.0021934419223718\\
517.01	0.00218051080749184\\
518.01	0.00216530521470991\\
519.01	0.00214963885173311\\
520.01	0.00213352591913959\\
521.01	0.00211682686193053\\
522.01	0.00209990162455342\\
523.01	0.00208334881810736\\
524.01	0.00206629842201661\\
525.01	0.0020487293273293\\
526.01	0.00203061912117578\\
527.01	0.0020119439851682\\
528.01	0.00199267861544261\\
529.01	0.00197279603566874\\
530.01	0.00195226717990642\\
531.01	0.0019310611918816\\
532.01	0.00190914400220951\\
533.01	0.00188646584897037\\
534.01	0.00186281675633515\\
535.01	0.00183733817879279\\
536.01	0.00181068563975129\\
537.01	0.00178316148253556\\
538.01	0.00175483070660634\\
539.01	0.00172543183520467\\
540.01	0.00169458528263656\\
541.01	0.0016619117138034\\
542.01	0.00163319129547519\\
543.01	0.00160397032611999\\
544.01	0.00157532742552503\\
545.01	0.00154672149291808\\
546.01	0.00151676918156295\\
547.01	0.00148528325632334\\
548.01	0.00145122918066596\\
549.01	0.00139645842151118\\
550.01	0.00131573204815777\\
551.01	0.00123401894266624\\
552.01	0.00115056170306921\\
553.01	0.00106527540330119\\
554.01	0.000978066999797552\\
555.01	0.000888833971152863\\
556.01	0.000797461203723393\\
557.01	0.00070380649333056\\
558.01	0.00060762032763984\\
559.01	0.000508366950882333\\
560.01	0.00040767854432403\\
561.01	0.000303109299583869\\
562.01	0.000206994944679002\\
563.01	0.000119489589999968\\
564.01	3.32849885533466e-05\\
565.01	1.73472347597681e-18\\
566.01	0\\
567.01	0\\
568.01	0\\
569.01	0\\
570.01	0\\
571.01	0\\
572.01	1.73472347597681e-18\\
573.01	0\\
574.01	1.73472347597681e-18\\
575.01	0\\
576.01	1.73472347597681e-18\\
577.01	0\\
578.01	1.73472347597681e-18\\
579.01	0\\
580.01	0\\
581.01	1.73472347597681e-18\\
582.01	0\\
583.01	0\\
584.01	0\\
585.01	0\\
586.01	0\\
587.01	0\\
588.01	0\\
589.01	0\\
590.01	0\\
591.01	0\\
592.01	1.73472347597681e-18\\
593.01	0\\
594.01	0\\
595.01	0\\
596.01	0\\
597.01	0\\
598.01	0\\
599.01	0\\
599.02	0\\
599.03	1.73472347597681e-18\\
599.04	1.73472347597681e-18\\
599.05	0\\
599.06	0\\
599.07	1.73472347597681e-18\\
599.08	0\\
599.09	0\\
599.1	0\\
599.11	1.73472347597681e-18\\
599.12	1.73472347597681e-18\\
599.13	0\\
599.14	0\\
599.15	1.73472347597681e-18\\
599.16	1.73472347597681e-18\\
599.17	1.73472347597681e-18\\
599.18	0\\
599.19	0\\
599.2	0\\
599.21	0\\
599.22	0\\
599.23	1.73472347597681e-18\\
599.24	1.73472347597681e-18\\
599.25	1.73472347597681e-18\\
599.26	1.73472347597681e-18\\
599.27	0\\
599.28	0\\
599.29	0\\
599.3	0\\
599.31	0\\
599.32	1.73472347597681e-18\\
599.33	1.73472347597681e-18\\
599.34	0\\
599.35	0\\
599.36	0\\
599.37	1.73472347597681e-18\\
599.38	0\\
599.39	0\\
599.4	0\\
599.41	0\\
599.42	0\\
599.43	0\\
599.44	0\\
599.45	0\\
599.46	1.73472347597681e-18\\
599.47	1.73472347597681e-18\\
599.48	1.73472347597681e-18\\
599.49	1.73472347597681e-18\\
599.5	1.73472347597681e-18\\
599.51	0\\
599.52	0\\
599.53	1.73472347597681e-18\\
599.54	0\\
599.55	0\\
599.56	0\\
599.57	1.73472347597681e-18\\
599.58	0\\
599.59	0\\
599.6	0\\
599.61	0\\
599.62	1.73472347597681e-18\\
599.63	0\\
599.64	1.73472347597681e-18\\
599.65	0\\
599.66	0\\
599.67	0\\
599.68	0\\
599.69	1.73472347597681e-18\\
599.7	0\\
599.71	1.73472347597681e-18\\
599.72	1.73472347597681e-18\\
599.73	1.73472347597681e-18\\
599.74	0\\
599.75	0\\
599.76	0\\
599.77	0\\
599.78	0\\
599.79	0\\
599.8	0\\
599.81	0\\
599.82	0\\
599.83	1.73472347597681e-18\\
599.84	0\\
599.85	0\\
599.86	0\\
599.87	0\\
599.88	0\\
599.89	0\\
599.9	0\\
599.91	0\\
599.92	0\\
599.93	0\\
599.94	0\\
599.95	0\\
599.96	0\\
599.97	0\\
599.98	0\\
599.99	0\\
600	0\\
};
\addplot [color=mycolor10,solid,forget plot]
  table[row sep=crcr]{%
0.01	0.00377125663032789\\
1.01	0.00377125662923481\\
2.01	0.00377125662811874\\
3.01	0.00377125662697919\\
4.01	0.00377125662581567\\
5.01	0.00377125662462766\\
6.01	0.00377125662341464\\
7.01	0.0037712566221761\\
8.01	0.00377125662091146\\
9.01	0.00377125661962021\\
10.01	0.00377125661830175\\
11.01	0.00377125661695552\\
12.01	0.00377125661558092\\
13.01	0.00377125661417738\\
14.01	0.00377125661274424\\
15.01	0.00377125661128089\\
16.01	0.0037712566097867\\
17.01	0.00377125660826095\\
18.01	0.00377125660670309\\
19.01	0.00377125660511236\\
20.01	0.00377125660348806\\
21.01	0.00377125660182949\\
22.01	0.00377125660013591\\
23.01	0.00377125659840658\\
24.01	0.00377125659664075\\
25.01	0.00377125659483764\\
26.01	0.00377125659299645\\
27.01	0.00377125659111635\\
28.01	0.00377125658919654\\
29.01	0.00377125658723616\\
30.01	0.00377125658523435\\
31.01	0.00377125658319023\\
32.01	0.00377125658110289\\
33.01	0.00377125657897141\\
34.01	0.00377125657679484\\
35.01	0.00377125657457225\\
36.01	0.0037712565723026\\
37.01	0.00377125656998494\\
38.01	0.00377125656761822\\
39.01	0.0037712565652014\\
40.01	0.00377125656273339\\
41.01	0.00377125656021311\\
42.01	0.00377125655763946\\
43.01	0.00377125655501125\\
44.01	0.00377125655232737\\
45.01	0.00377125654958658\\
46.01	0.0037712565467877\\
47.01	0.00377125654392947\\
48.01	0.00377125654101061\\
49.01	0.00377125653802985\\
50.01	0.00377125653498578\\
51.01	0.00377125653187715\\
52.01	0.00377125652870254\\
53.01	0.00377125652546051\\
54.01	0.00377125652214963\\
55.01	0.00377125651876842\\
56.01	0.00377125651531538\\
57.01	0.00377125651178895\\
58.01	0.00377125650818757\\
59.01	0.00377125650450962\\
60.01	0.00377125650075346\\
61.01	0.0037712564969174\\
62.01	0.00377125649299973\\
63.01	0.00377125648899868\\
64.01	0.00377125648491248\\
65.01	0.00377125648073927\\
66.01	0.0037712564764772\\
67.01	0.00377125647212436\\
68.01	0.00377125646767877\\
69.01	0.00377125646313843\\
70.01	0.00377125645850132\\
71.01	0.00377125645376534\\
72.01	0.00377125644892838\\
73.01	0.00377125644398821\\
74.01	0.00377125643894264\\
75.01	0.0037712564337894\\
76.01	0.00377125642852614\\
77.01	0.0037712564231505\\
78.01	0.00377125641766003\\
79.01	0.00377125641205227\\
80.01	0.00377125640632467\\
81.01	0.00377125640047466\\
82.01	0.00377125639449953\\
83.01	0.00377125638839665\\
84.01	0.00377125638216321\\
85.01	0.00377125637579638\\
86.01	0.00377125636929328\\
87.01	0.00377125636265095\\
88.01	0.00377125635586637\\
89.01	0.00377125634893645\\
90.01	0.00377125634185805\\
91.01	0.00377125633462793\\
92.01	0.0037712563272428\\
93.01	0.00377125631969928\\
94.01	0.00377125631199396\\
95.01	0.00377125630412329\\
96.01	0.00377125629608368\\
97.01	0.00377125628787146\\
98.01	0.00377125627948288\\
99.01	0.00377125627091409\\
100.01	0.00377125626216116\\
101.01	0.00377125625322007\\
102.01	0.00377125624408673\\
103.01	0.00377125623475696\\
104.01	0.00377125622522645\\
105.01	0.00377125621549082\\
106.01	0.0037712562055456\\
107.01	0.00377125619538622\\
108.01	0.00377125618500799\\
109.01	0.00377125617440612\\
110.01	0.00377125616357572\\
111.01	0.0037712561525118\\
112.01	0.00377125614120924\\
113.01	0.00377125612966283\\
114.01	0.0037712561178672\\
115.01	0.00377125610581691\\
116.01	0.00377125609350638\\
117.01	0.00377125608092989\\
118.01	0.00377125606808161\\
119.01	0.00377125605495556\\
120.01	0.00377125604154567\\
121.01	0.00377125602784567\\
122.01	0.00377125601384922\\
123.01	0.00377125599954977\\
124.01	0.00377125598494068\\
125.01	0.00377125597001511\\
126.01	0.00377125595476611\\
127.01	0.00377125593918656\\
128.01	0.00377125592326918\\
129.01	0.00377125590700649\\
130.01	0.00377125589039092\\
131.01	0.00377125587341465\\
132.01	0.00377125585606974\\
133.01	0.00377125583834803\\
134.01	0.00377125582024122\\
135.01	0.0037712558017408\\
136.01	0.00377125578283805\\
137.01	0.00377125576352404\\
138.01	0.00377125574378972\\
139.01	0.00377125572362578\\
140.01	0.00377125570302267\\
141.01	0.00377125568197071\\
142.01	0.00377125566045989\\
143.01	0.00377125563848008\\
144.01	0.00377125561602085\\
145.01	0.00377125559307157\\
146.01	0.00377125556962133\\
147.01	0.00377125554565905\\
148.01	0.00377125552117329\\
149.01	0.00377125549615244\\
150.01	0.00377125547058459\\
151.01	0.00377125544445756\\
152.01	0.00377125541775891\\
153.01	0.0037712553904759\\
154.01	0.00377125536259548\\
155.01	0.00377125533410436\\
156.01	0.0037712553049889\\
157.01	0.00377125527523517\\
158.01	0.00377125524482892\\
159.01	0.00377125521375555\\
160.01	0.00377125518200019\\
161.01	0.00377125514954756\\
162.01	0.00377125511638205\\
163.01	0.00377125508248775\\
164.01	0.0037712550478483\\
165.01	0.00377125501244702\\
166.01	0.00377125497626686\\
167.01	0.00377125493929032\\
168.01	0.00377125490149956\\
169.01	0.00377125486287629\\
170.01	0.00377125482340183\\
171.01	0.00377125478305704\\
172.01	0.00377125474182241\\
173.01	0.00377125469967789\\
174.01	0.00377125465660303\\
175.01	0.00377125461257688\\
176.01	0.00377125456757805\\
177.01	0.00377125452158462\\
178.01	0.00377125447457417\\
179.01	0.00377125442652378\\
180.01	0.00377125437741001\\
181.01	0.00377125432720886\\
182.01	0.00377125427589576\\
183.01	0.00377125422344564\\
184.01	0.0037712541698328\\
185.01	0.00377125411503094\\
186.01	0.00377125405901324\\
187.01	0.00377125400175211\\
188.01	0.00377125394321946\\
189.01	0.0037712538833865\\
190.01	0.00377125382222378\\
191.01	0.00377125375970115\\
192.01	0.00377125369578781\\
193.01	0.00377125363045223\\
194.01	0.0037712535636621\\
195.01	0.00377125349538445\\
196.01	0.0037712534255855\\
197.01	0.00377125335423071\\
198.01	0.00377125328128466\\
199.01	0.00377125320671129\\
200.01	0.00377125313047352\\
201.01	0.00377125305253351\\
202.01	0.00377125297285252\\
203.01	0.00377125289139092\\
204.01	0.00377125280810816\\
205.01	0.00377125272296276\\
206.01	0.00377125263591226\\
207.01	0.00377125254691324\\
208.01	0.00377125245592126\\
209.01	0.00377125236289087\\
210.01	0.00377125226777554\\
211.01	0.00377125217052766\\
212.01	0.00377125207109857\\
213.01	0.0037712519694384\\
214.01	0.00377125186549624\\
215.01	0.00377125175921985\\
216.01	0.00377125165055593\\
217.01	0.00377125153944987\\
218.01	0.00377125142584576\\
219.01	0.00377125130968648\\
220.01	0.00377125119091354\\
221.01	0.00377125106946709\\
222.01	0.0037712509452859\\
223.01	0.00377125081830735\\
224.01	0.00377125068846734\\
225.01	0.00377125055570028\\
226.01	0.00377125041993908\\
227.01	0.0037712502811151\\
228.01	0.0037712501391581\\
229.01	0.00377124999399622\\
230.01	0.00377124984555593\\
231.01	0.003771249693762\\
232.01	0.00377124953853749\\
233.01	0.00377124937980362\\
234.01	0.00377124921747985\\
235.01	0.00377124905148371\\
236.01	0.00377124888173092\\
237.01	0.00377124870813518\\
238.01	0.0037712485306082\\
239.01	0.00377124834905967\\
240.01	0.00377124816339722\\
241.01	0.00377124797352628\\
242.01	0.00377124777935015\\
243.01	0.00377124758076991\\
244.01	0.00377124737768429\\
245.01	0.00377124716998976\\
246.01	0.00377124695758038\\
247.01	0.00377124674034776\\
248.01	0.00377124651818101\\
249.01	0.00377124629096671\\
250.01	0.0037712460585888\\
251.01	0.00377124582092861\\
252.01	0.00377124557786464\\
253.01	0.0037712453292727\\
254.01	0.00377124507502568\\
255.01	0.00377124481499361\\
256.01	0.00377124454904345\\
257.01	0.00377124427703921\\
258.01	0.0037712439988417\\
259.01	0.00377124371430856\\
260.01	0.00377124342329422\\
261.01	0.00377124312564969\\
262.01	0.00377124282122264\\
263.01	0.00377124250985719\\
264.01	0.00377124219139397\\
265.01	0.00377124186566991\\
266.01	0.00377124153251821\\
267.01	0.00377124119176828\\
268.01	0.00377124084324565\\
269.01	0.00377124048677181\\
270.01	0.00377124012216424\\
271.01	0.0037712397492362\\
272.01	0.00377123936779675\\
273.01	0.00377123897765053\\
274.01	0.00377123857859778\\
275.01	0.00377123817043414\\
276.01	0.00377123775295064\\
277.01	0.0037712373259335\\
278.01	0.00377123688916414\\
279.01	0.00377123644241893\\
280.01	0.00377123598546919\\
281.01	0.00377123551808104\\
282.01	0.00377123504001524\\
283.01	0.0037712345510271\\
284.01	0.00377123405086641\\
285.01	0.00377123353927726\\
286.01	0.0037712330159978\\
287.01	0.00377123248076038\\
288.01	0.00377123193329115\\
289.01	0.00377123137331005\\
290.01	0.00377123080053061\\
291.01	0.00377123021465986\\
292.01	0.00377122961539813\\
293.01	0.00377122900243893\\
294.01	0.0037712283754687\\
295.01	0.00377122773416679\\
296.01	0.00377122707820523\\
297.01	0.00377122640724845\\
298.01	0.00377122572095324\\
299.01	0.00377122501896854\\
300.01	0.00377122430093521\\
301.01	0.00377122356648583\\
302.01	0.00377122281524456\\
303.01	0.00377122204682687\\
304.01	0.00377122126083937\\
305.01	0.00377122045687959\\
306.01	0.00377121963453571\\
307.01	0.00377121879338637\\
308.01	0.00377121793300043\\
309.01	0.0037712170529367\\
310.01	0.00377121615274374\\
311.01	0.00377121523195949\\
312.01	0.0037712142901111\\
313.01	0.00377121332671464\\
314.01	0.00377121234127477\\
315.01	0.00377121133328445\\
316.01	0.00377121030222468\\
317.01	0.00377120924756412\\
318.01	0.00377120816875886\\
319.01	0.00377120706525192\\
320.01	0.00377120593647308\\
321.01	0.00377120478183842\\
322.01	0.00377120360074998\\
323.01	0.00377120239259536\\
324.01	0.00377120115674734\\
325.01	0.00377119989256349\\
326.01	0.00377119859938572\\
327.01	0.00377119727653983\\
328.01	0.00377119592333516\\
329.01	0.00377119453906398\\
330.01	0.00377119312300113\\
331.01	0.00377119167440351\\
332.01	0.00377119019250953\\
333.01	0.00377118867653863\\
334.01	0.0037711871256907\\
335.01	0.00377118553914557\\
336.01	0.00377118391606246\\
337.01	0.00377118225557928\\
338.01	0.00377118055681215\\
339.01	0.0037711788188547\\
340.01	0.00377117704077747\\
341.01	0.0037711752216272\\
342.01	0.00377117336042621\\
343.01	0.00377117145617166\\
344.01	0.00377116950783488\\
345.01	0.00377116751436053\\
346.01	0.00377116547466594\\
347.01	0.00377116338764031\\
348.01	0.0037711612521439\\
349.01	0.00377115906700718\\
350.01	0.00377115683103001\\
351.01	0.0037711545429808\\
352.01	0.00377115220159558\\
353.01	0.0037711498055771\\
354.01	0.00377114735359391\\
355.01	0.00377114484427938\\
356.01	0.00377114227623072\\
357.01	0.00377113964800797\\
358.01	0.003771136958133\\
359.01	0.00377113420508843\\
360.01	0.00377113138731648\\
361.01	0.00377112850321798\\
362.01	0.00377112555115114\\
363.01	0.00377112252943045\\
364.01	0.00377111943632548\\
365.01	0.0037711162700596\\
366.01	0.00377111302880885\\
367.01	0.00377110971070064\\
368.01	0.00377110631381242\\
369.01	0.00377110283617048\\
370.01	0.00377109927574847\\
371.01	0.00377109563046624\\
372.01	0.00377109189818833\\
373.01	0.00377108807672269\\
374.01	0.00377108416381922\\
375.01	0.00377108015716865\\
376.01	0.0037710760544019\\
377.01	0.00377107185309136\\
378.01	0.00377106755074423\\
379.01	0.00377106314478657\\
380.01	0.0037710586325869\\
381.01	0.00377105401144644\\
382.01	0.00377104927859274\\
383.01	0.00377104443117732\\
384.01	0.00377103946627339\\
385.01	0.00377103438087322\\
386.01	0.0037710291718856\\
387.01	0.00377102383613311\\
388.01	0.00377101837034927\\
389.01	0.00377101277117559\\
390.01	0.0037710070351585\\
391.01	0.003771001158746\\
392.01	0.00377099513828446\\
393.01	0.0037709889700149\\
394.01	0.00377098265006942\\
395.01	0.00377097617446723\\
396.01	0.00377096953911052\\
397.01	0.00377096273978032\\
398.01	0.00377095577213189\\
399.01	0.00377094863169\\
400.01	0.00377094131384401\\
401.01	0.00377093381384256\\
402.01	0.00377092612678817\\
403.01	0.00377091824763121\\
404.01	0.00377091017116405\\
405.01	0.00377090189201435\\
406.01	0.00377089340463834\\
407.01	0.00377088470331359\\
408.01	0.00377087578213129\\
409.01	0.00377086663498825\\
410.01	0.00377085725557829\\
411.01	0.00377084763738311\\
412.01	0.00377083777366273\\
413.01	0.00377082765744519\\
414.01	0.0037708172815157\\
415.01	0.00377080663840508\\
416.01	0.00377079572037746\\
417.01	0.00377078451941714\\
418.01	0.00377077302721465\\
419.01	0.00377076123515187\\
420.01	0.00377074913428604\\
421.01	0.0037707367153328\\
422.01	0.003770723968648\\
423.01	0.00377071088420809\\
424.01	0.0037706974515894\\
425.01	0.00377068365994545\\
426.01	0.00377066949798302\\
427.01	0.00377065495393602\\
428.01	0.00377064001553759\\
429.01	0.00377062466998979\\
430.01	0.00377060890393099\\
431.01	0.00377059270340034\\
432.01	0.00377057605379947\\
433.01	0.00377055893985072\\
434.01	0.00377054134555156\\
435.01	0.00377052325412504\\
436.01	0.00377050464796542\\
437.01	0.00377048550857853\\
438.01	0.00377046581651641\\
439.01	0.00377044555130495\\
440.01	0.00377042469136421\\
441.01	0.00377040321391984\\
442.01	0.00377038109490479\\
443.01	0.00377035830884953\\
444.01	0.00377033482875936\\
445.01	0.00377031062597658\\
446.01	0.00377028567002527\\
447.01	0.0037702599284359\\
448.01	0.00377023336654646\\
449.01	0.00377020594727602\\
450.01	0.00377017763086607\\
451.01	0.00377014837458425\\
452.01	0.00377011813238344\\
453.01	0.00377008685450823\\
454.01	0.00377005448703902\\
455.01	0.00377002097136231\\
456.01	0.00376998624355294\\
457.01	0.0037699502336513\\
458.01	0.00376991286481851\\
459.01	0.0037698740523633\\
460.01	0.0037698337027361\\
461.01	0.00376979171303983\\
462.01	0.00376974797145467\\
463.01	0.00376970234428547\\
464.01	0.00376965461408954\\
465.01	0.00376960414450828\\
466.01	0.00376954711044049\\
467.01	0.00376946369118717\\
468.01	0.00376936110710564\\
469.01	0.00376925352522068\\
470.01	0.00376912119316176\\
471.01	0.00376888136306918\\
472.01	0.00376862514051226\\
473.01	0.00376836119045994\\
474.01	0.00376808652744747\\
475.01	0.00376780087669021\\
476.01	0.00376750749543886\\
477.01	0.00376720617289869\\
478.01	0.00376689660393025\\
479.01	0.00376657846271649\\
480.01	0.0037662514006514\\
481.01	0.0037659150439715\\
482.01	0.00376556899109603\\
483.01	0.00376521280963743\\
484.01	0.00376484603303844\\
485.01	0.00376446815678627\\
486.01	0.00376407863415389\\
487.01	0.00376367687143755\\
488.01	0.0037632622227638\\
489.01	0.00376283398453866\\
490.01	0.00376239138774652\\
491.01	0.00376193358974755\\
492.01	0.00376145966125149\\
493.01	0.00376096857969392\\
494.01	0.00376045922297468\\
495.01	0.00375993035225818\\
496.01	0.00375938059668709\\
497.01	0.00375880843395434\\
498.01	0.00375821215331006\\
499.01	0.00375758970411473\\
500.01	0.00375693769098714\\
501.01	0.00375624304836522\\
502.01	0.00375538283410615\\
503.01	0.00375388380446347\\
504.01	0.00375228680245878\\
505.01	0.00375063691587537\\
506.01	0.00374893144156919\\
507.01	0.00374716743881713\\
508.01	0.00374534169691954\\
509.01	0.00374345069696541\\
510.01	0.00374149056631271\\
511.01	0.00373945702259677\\
512.01	0.00373734529289359\\
513.01	0.00373514991110968\\
514.01	0.00373286365162165\\
515.01	0.00373046958084579\\
516.01	0.00372788180107596\\
517.01	0.00372490743254982\\
518.01	0.00372176839390527\\
519.01	0.00371849609881487\\
520.01	0.00371507009073484\\
521.01	0.00371137097610994\\
522.01	0.00370612686786882\\
523.01	0.00369939575242447\\
524.01	0.00369245046530768\\
525.01	0.00368527987059093\\
526.01	0.00367787189140608\\
527.01	0.00367021340166731\\
528.01	0.00366229010320468\\
529.01	0.00365408636436933\\
530.01	0.00364558505231754\\
531.01	0.00363676733036109\\
532.01	0.00362761208451302\\
533.01	0.00361809309921056\\
534.01	0.00360816454433315\\
535.01	0.00359776897808293\\
536.01	0.0035869023128831\\
537.01	0.00357551025994714\\
538.01	0.00356393476086114\\
539.01	0.00355182238397022\\
540.01	0.00353885783760149\\
541.01	0.00352207049789789\\
542.01	0.00349558486704233\\
543.01	0.00346850137793974\\
544.01	0.00344126642932281\\
545.01	0.00341313220386299\\
546.01	0.00338401104553813\\
547.01	0.0033538230294319\\
548.01	0.00332226370052044\\
549.01	0.00328792935660126\\
550.01	0.00325212899512041\\
551.01	0.00321710530789461\\
552.01	0.00318087389906738\\
553.01	0.00314334260828293\\
554.01	0.00310440778428242\\
555.01	0.00306395216153003\\
556.01	0.00302184109964779\\
557.01	0.00297790873028086\\
558.01	0.00293187370834995\\
559.01	0.00288288946532628\\
560.01	0.00283179093759702\\
561.01	0.00278149436602371\\
562.01	0.00269128150390921\\
563.01	0.00258475698378168\\
564.01	0.00247655079411002\\
565.01	0.0023686312360737\\
566.01	0.00225884532024653\\
567.01	0.00214636471298463\\
568.01	0.00203101614877091\\
569.01	0.00191260384401076\\
570.01	0.00179089309187064\\
571.01	0.00166546488008717\\
572.01	0.00153473123580838\\
573.01	0.00139480841276273\\
574.01	0.00127843382944641\\
575.01	0.00116237717509533\\
576.01	0.00104405713628715\\
577.01	0.000923474008441835\\
578.01	0.000800467496979472\\
579.01	0.000674981078808047\\
580.01	0.000547004326546188\\
581.01	0.000416546308886746\\
582.01	0.00028365363948627\\
583.01	0.000148672106345117\\
584.01	2.10202085324119e-05\\
585.01	0\\
586.01	0\\
587.01	0\\
588.01	0\\
589.01	0\\
590.01	0\\
591.01	0\\
592.01	1.73472347597681e-18\\
593.01	0\\
594.01	0\\
595.01	0\\
596.01	0\\
597.01	0\\
598.01	0\\
599.01	0\\
599.02	0\\
599.03	1.73472347597681e-18\\
599.04	1.73472347597681e-18\\
599.05	0\\
599.06	0\\
599.07	1.73472347597681e-18\\
599.08	0\\
599.09	0\\
599.1	0\\
599.11	1.73472347597681e-18\\
599.12	1.73472347597681e-18\\
599.13	0\\
599.14	0\\
599.15	1.73472347597681e-18\\
599.16	1.73472347597681e-18\\
599.17	1.73472347597681e-18\\
599.18	0\\
599.19	0\\
599.2	0\\
599.21	0\\
599.22	0\\
599.23	1.73472347597681e-18\\
599.24	1.73472347597681e-18\\
599.25	1.73472347597681e-18\\
599.26	1.73472347597681e-18\\
599.27	0\\
599.28	0\\
599.29	0\\
599.3	0\\
599.31	0\\
599.32	1.73472347597681e-18\\
599.33	1.73472347597681e-18\\
599.34	0\\
599.35	0\\
599.36	0\\
599.37	1.73472347597681e-18\\
599.38	0\\
599.39	0\\
599.4	0\\
599.41	0\\
599.42	0\\
599.43	0\\
599.44	0\\
599.45	0\\
599.46	1.73472347597681e-18\\
599.47	1.73472347597681e-18\\
599.48	1.73472347597681e-18\\
599.49	1.73472347597681e-18\\
599.5	1.73472347597681e-18\\
599.51	0\\
599.52	0\\
599.53	1.73472347597681e-18\\
599.54	0\\
599.55	0\\
599.56	0\\
599.57	1.73472347597681e-18\\
599.58	0\\
599.59	0\\
599.6	0\\
599.61	0\\
599.62	1.73472347597681e-18\\
599.63	0\\
599.64	1.73472347597681e-18\\
599.65	0\\
599.66	0\\
599.67	0\\
599.68	0\\
599.69	1.73472347597681e-18\\
599.7	0\\
599.71	1.73472347597681e-18\\
599.72	1.73472347597681e-18\\
599.73	1.73472347597681e-18\\
599.74	0\\
599.75	0\\
599.76	0\\
599.77	0\\
599.78	0\\
599.79	0\\
599.8	0\\
599.81	0\\
599.82	0\\
599.83	1.73472347597681e-18\\
599.84	0\\
599.85	0\\
599.86	0\\
599.87	0\\
599.88	0\\
599.89	0\\
599.9	0\\
599.91	0\\
599.92	0\\
599.93	0\\
599.94	0\\
599.95	0\\
599.96	0\\
599.97	0\\
599.98	0\\
599.99	0\\
600	0\\
};
\addplot [color=mycolor11,solid,forget plot]
  table[row sep=crcr]{%
0.01	0.00628526379179407\\
1.01	0.00628526379170105\\
2.01	0.00628526379160609\\
3.01	0.00628526379150912\\
4.01	0.00628526379141011\\
5.01	0.00628526379130903\\
6.01	0.0062852637912058\\
7.01	0.00628526379110041\\
8.01	0.0062852637909928\\
9.01	0.00628526379088292\\
10.01	0.00628526379077073\\
11.01	0.00628526379065617\\
12.01	0.0062852637905392\\
13.01	0.00628526379041977\\
14.01	0.00628526379029781\\
15.01	0.0062852637901733\\
16.01	0.00628526379004615\\
17.01	0.00628526378991631\\
18.01	0.00628526378978375\\
19.01	0.00628526378964839\\
20.01	0.00628526378951017\\
21.01	0.00628526378936903\\
22.01	0.00628526378922492\\
23.01	0.00628526378907776\\
24.01	0.00628526378892749\\
25.01	0.00628526378877405\\
26.01	0.00628526378861738\\
27.01	0.00628526378845739\\
28.01	0.00628526378829402\\
29.01	0.0062852637881272\\
30.01	0.00628526378795685\\
31.01	0.00628526378778291\\
32.01	0.00628526378760528\\
33.01	0.0062852637874239\\
34.01	0.00628526378723868\\
35.01	0.00628526378704954\\
36.01	0.00628526378685641\\
37.01	0.00628526378665918\\
38.01	0.00628526378645778\\
39.01	0.00628526378625211\\
40.01	0.00628526378604209\\
41.01	0.00628526378582762\\
42.01	0.00628526378560861\\
43.01	0.00628526378538495\\
44.01	0.00628526378515656\\
45.01	0.00628526378492332\\
46.01	0.00628526378468514\\
47.01	0.00628526378444191\\
48.01	0.00628526378419352\\
49.01	0.00628526378393985\\
50.01	0.00628526378368081\\
51.01	0.00628526378341627\\
52.01	0.0062852637831461\\
53.01	0.00628526378287021\\
54.01	0.00628526378258845\\
55.01	0.00628526378230071\\
56.01	0.00628526378200686\\
57.01	0.00628526378170676\\
58.01	0.00628526378140028\\
59.01	0.00628526378108728\\
60.01	0.00628526378076763\\
61.01	0.00628526378044117\\
62.01	0.00628526378010777\\
63.01	0.00628526377976728\\
64.01	0.00628526377941954\\
65.01	0.00628526377906439\\
66.01	0.00628526377870168\\
67.01	0.00628526377833124\\
68.01	0.00628526377795291\\
69.01	0.00628526377756652\\
70.01	0.00628526377717189\\
71.01	0.00628526377676884\\
72.01	0.00628526377635719\\
73.01	0.00628526377593677\\
74.01	0.00628526377550737\\
75.01	0.00628526377506882\\
76.01	0.00628526377462089\\
77.01	0.00628526377416341\\
78.01	0.00628526377369614\\
79.01	0.00628526377321889\\
80.01	0.00628526377273145\\
81.01	0.00628526377223358\\
82.01	0.00628526377172507\\
83.01	0.00628526377120568\\
84.01	0.00628526377067519\\
85.01	0.00628526377013333\\
86.01	0.00628526376957987\\
87.01	0.00628526376901458\\
88.01	0.00628526376843716\\
89.01	0.00628526376784738\\
90.01	0.00628526376724496\\
91.01	0.00628526376662963\\
92.01	0.0062852637660011\\
93.01	0.0062852637653591\\
94.01	0.00628526376470331\\
95.01	0.00628526376403345\\
96.01	0.00628526376334921\\
97.01	0.00628526376265028\\
98.01	0.00628526376193635\\
99.01	0.00628526376120706\\
100.01	0.00628526376046211\\
101.01	0.00628526375970114\\
102.01	0.00628526375892381\\
103.01	0.00628526375812975\\
104.01	0.0062852637573186\\
105.01	0.00628526375649001\\
106.01	0.00628526375564357\\
107.01	0.00628526375477889\\
108.01	0.00628526375389559\\
109.01	0.00628526375299326\\
110.01	0.00628526375207147\\
111.01	0.00628526375112981\\
112.01	0.00628526375016782\\
113.01	0.00628526374918508\\
114.01	0.00628526374818112\\
115.01	0.00628526374715549\\
116.01	0.0062852637461077\\
117.01	0.00628526374503727\\
118.01	0.00628526374394372\\
119.01	0.00628526374282652\\
120.01	0.00628526374168514\\
121.01	0.00628526374051908\\
122.01	0.00628526373932778\\
123.01	0.00628526373811068\\
124.01	0.00628526373686723\\
125.01	0.00628526373559685\\
126.01	0.00628526373429891\\
127.01	0.00628526373297285\\
128.01	0.00628526373161802\\
129.01	0.0062852637302338\\
130.01	0.00628526372881954\\
131.01	0.00628526372737458\\
132.01	0.00628526372589823\\
133.01	0.00628526372438981\\
134.01	0.0062852637228486\\
135.01	0.00628526372127384\\
136.01	0.00628526371966487\\
137.01	0.00628526371802091\\
138.01	0.00628526371634115\\
139.01	0.00628526371462481\\
140.01	0.00628526371287109\\
141.01	0.00628526371107915\\
142.01	0.00628526370924815\\
143.01	0.00628526370737723\\
144.01	0.00628526370546549\\
145.01	0.00628526370351202\\
146.01	0.00628526370151591\\
147.01	0.0062852636994762\\
148.01	0.00628526369739193\\
149.01	0.0062852636952621\\
150.01	0.00628526369308571\\
151.01	0.00628526369086171\\
152.01	0.00628526368858903\\
153.01	0.00628526368626661\\
154.01	0.00628526368389333\\
155.01	0.00628526368146806\\
156.01	0.00628526367898962\\
157.01	0.00628526367645685\\
158.01	0.00628526367386852\\
159.01	0.00628526367122339\\
160.01	0.0062852636685202\\
161.01	0.00628526366575764\\
162.01	0.0062852636629344\\
163.01	0.00628526366004909\\
164.01	0.00628526365710035\\
165.01	0.00628526365408674\\
166.01	0.00628526365100682\\
167.01	0.00628526364785907\\
168.01	0.00628526364464202\\
169.01	0.0062852636413541\\
170.01	0.00628526363799368\\
171.01	0.00628526363455917\\
172.01	0.00628526363104889\\
173.01	0.00628526362746113\\
174.01	0.00628526362379417\\
175.01	0.00628526362004621\\
176.01	0.00628526361621543\\
177.01	0.00628526361229996\\
178.01	0.0062852636082979\\
179.01	0.00628526360420729\\
180.01	0.00628526360002613\\
181.01	0.00628526359575239\\
182.01	0.00628526359138397\\
183.01	0.00628526358691873\\
184.01	0.00628526358235449\\
185.01	0.00628526357768902\\
186.01	0.00628526357291998\\
187.01	0.0062852635680451\\
188.01	0.00628526356306194\\
189.01	0.00628526355796806\\
190.01	0.00628526355276094\\
191.01	0.00628526354743804\\
192.01	0.00628526354199671\\
193.01	0.00628526353643427\\
194.01	0.00628526353074799\\
195.01	0.00628526352493503\\
196.01	0.00628526351899252\\
197.01	0.00628526351291754\\
198.01	0.00628526350670707\\
199.01	0.00628526350035799\\
200.01	0.00628526349386721\\
201.01	0.00628526348723147\\
202.01	0.00628526348044748\\
203.01	0.00628526347351186\\
204.01	0.00628526346642117\\
205.01	0.00628526345917185\\
206.01	0.00628526345176031\\
207.01	0.00628526344418285\\
208.01	0.00628526343643567\\
209.01	0.0062852634285149\\
210.01	0.00628526342041658\\
211.01	0.00628526341213667\\
212.01	0.00628526340367101\\
213.01	0.00628526339501537\\
214.01	0.00628526338616539\\
215.01	0.00628526337711664\\
216.01	0.00628526336786457\\
217.01	0.00628526335840451\\
218.01	0.00628526334873175\\
219.01	0.00628526333884138\\
220.01	0.00628526332872843\\
221.01	0.00628526331838782\\
222.01	0.00628526330781432\\
223.01	0.00628526329700258\\
224.01	0.00628526328594717\\
225.01	0.0062852632746425\\
226.01	0.00628526326308284\\
227.01	0.00628526325126236\\
228.01	0.00628526323917507\\
229.01	0.00628526322681484\\
230.01	0.00628526321417543\\
231.01	0.00628526320125042\\
232.01	0.00628526318803326\\
233.01	0.00628526317451723\\
234.01	0.00628526316069548\\
235.01	0.006285263146561\\
236.01	0.00628526313210659\\
237.01	0.00628526311732491\\
238.01	0.00628526310220844\\
239.01	0.00628526308674948\\
240.01	0.00628526307094018\\
241.01	0.00628526305477247\\
242.01	0.00628526303823814\\
243.01	0.00628526302132872\\
244.01	0.00628526300403563\\
245.01	0.00628526298635002\\
246.01	0.0062852629682629\\
247.01	0.00628526294976503\\
248.01	0.00628526293084694\\
249.01	0.006285262911499\\
250.01	0.00628526289171131\\
251.01	0.00628526287147376\\
252.01	0.00628526285077603\\
253.01	0.00628526282960753\\
254.01	0.00628526280795741\\
255.01	0.00628526278581463\\
256.01	0.00628526276316786\\
257.01	0.00628526274000552\\
258.01	0.00628526271631574\\
259.01	0.00628526269208641\\
260.01	0.00628526266730512\\
261.01	0.0062852626419592\\
262.01	0.00628526261603565\\
263.01	0.00628526258952125\\
264.01	0.00628526256240237\\
265.01	0.00628526253466515\\
266.01	0.00628526250629539\\
267.01	0.00628526247727856\\
268.01	0.00628526244759979\\
269.01	0.00628526241724389\\
270.01	0.00628526238619531\\
271.01	0.00628526235443817\\
272.01	0.00628526232195617\\
273.01	0.00628526228873271\\
274.01	0.00628526225475076\\
275.01	0.00628526221999294\\
276.01	0.00628526218444143\\
277.01	0.00628526214807805\\
278.01	0.00628526211088416\\
279.01	0.00628526207284073\\
280.01	0.00628526203392831\\
281.01	0.00628526199412694\\
282.01	0.00628526195341628\\
283.01	0.00628526191177548\\
284.01	0.00628526186918324\\
285.01	0.00628526182561777\\
286.01	0.00628526178105677\\
287.01	0.00628526173547744\\
288.01	0.00628526168885647\\
289.01	0.00628526164117002\\
290.01	0.00628526159239368\\
291.01	0.00628526154250251\\
292.01	0.00628526149147098\\
293.01	0.00628526143927302\\
294.01	0.00628526138588189\\
295.01	0.00628526133127032\\
296.01	0.00628526127541035\\
297.01	0.00628526121827342\\
298.01	0.00628526115983031\\
299.01	0.00628526110005111\\
300.01	0.00628526103890524\\
301.01	0.00628526097636141\\
302.01	0.00628526091238761\\
303.01	0.0062852608469511\\
304.01	0.00628526078001837\\
305.01	0.00628526071155514\\
306.01	0.00628526064152635\\
307.01	0.00628526056989611\\
308.01	0.00628526049662768\\
309.01	0.0062852604216835\\
310.01	0.00628526034502512\\
311.01	0.00628526026661318\\
312.01	0.00628526018640739\\
313.01	0.00628526010436657\\
314.01	0.00628526002044848\\
315.01	0.00628525993460997\\
316.01	0.00628525984680679\\
317.01	0.0062852597569937\\
318.01	0.00628525966512435\\
319.01	0.0062852595711513\\
320.01	0.00628525947502597\\
321.01	0.00628525937669862\\
322.01	0.00628525927611828\\
323.01	0.00628525917323279\\
324.01	0.00628525906798871\\
325.01	0.00628525896033128\\
326.01	0.00628525885020445\\
327.01	0.00628525873755075\\
328.01	0.00628525862231133\\
329.01	0.00628525850442589\\
330.01	0.00628525838383262\\
331.01	0.00628525826046818\\
332.01	0.00628525813426766\\
333.01	0.00628525800516456\\
334.01	0.00628525787309066\\
335.01	0.00628525773797607\\
336.01	0.00628525759974909\\
337.01	0.00628525745833627\\
338.01	0.00628525731366225\\
339.01	0.00628525716564976\\
340.01	0.00628525701421957\\
341.01	0.00628525685929041\\
342.01	0.00628525670077893\\
343.01	0.00628525653859963\\
344.01	0.00628525637266481\\
345.01	0.00628525620288447\\
346.01	0.00628525602916632\\
347.01	0.00628525585141562\\
348.01	0.00628525566953522\\
349.01	0.00628525548342537\\
350.01	0.00628525529298375\\
351.01	0.00628525509810531\\
352.01	0.00628525489868229\\
353.01	0.00628525469460406\\
354.01	0.00628525448575706\\
355.01	0.00628525427202476\\
356.01	0.00628525405328753\\
357.01	0.00628525382942255\\
358.01	0.00628525360030377\\
359.01	0.00628525336580178\\
360.01	0.00628525312578374\\
361.01	0.00628525288011325\\
362.01	0.00628525262865029\\
363.01	0.00628525237125112\\
364.01	0.00628525210776814\\
365.01	0.00628525183804985\\
366.01	0.00628525156194069\\
367.01	0.00628525127928093\\
368.01	0.00628525098990664\\
369.01	0.00628525069364946\\
370.01	0.00628525039033662\\
371.01	0.0062852500797907\\
372.01	0.00628524976182957\\
373.01	0.00628524943626634\\
374.01	0.00628524910290913\\
375.01	0.00628524876156111\\
376.01	0.00628524841202061\\
377.01	0.00628524805408113\\
378.01	0.00628524768752958\\
379.01	0.0062852473121461\\
380.01	0.00628524692770632\\
381.01	0.00628524653398002\\
382.01	0.00628524613073069\\
383.01	0.00628524571771536\\
384.01	0.00628524529468427\\
385.01	0.00628524486138085\\
386.01	0.00628524441754133\\
387.01	0.00628524396289458\\
388.01	0.00628524349716187\\
389.01	0.00628524302005659\\
390.01	0.00628524253128397\\
391.01	0.0062852420305409\\
392.01	0.00628524151751552\\
393.01	0.00628524099188693\\
394.01	0.00628524045332497\\
395.01	0.00628523990148974\\
396.01	0.00628523933603142\\
397.01	0.00628523875658973\\
398.01	0.00628523816279363\\
399.01	0.00628523755426096\\
400.01	0.00628523693059789\\
401.01	0.00628523629139856\\
402.01	0.00628523563624455\\
403.01	0.00628523496470447\\
404.01	0.0062852342763333\\
405.01	0.00628523357067193\\
406.01	0.00628523284724651\\
407.01	0.00628523210556788\\
408.01	0.00628523134513089\\
409.01	0.00628523056541366\\
410.01	0.0062852297658769\\
411.01	0.00628522894596309\\
412.01	0.00628522810509564\\
413.01	0.00628522724267802\\
414.01	0.00628522635809286\\
415.01	0.00628522545070084\\
416.01	0.0062852245198397\\
417.01	0.00628522356482316\\
418.01	0.00628522258493956\\
419.01	0.0062852215794507\\
420.01	0.0062852205475904\\
421.01	0.00628521948856302\\
422.01	0.0062852184015419\\
423.01	0.00628521728566767\\
424.01	0.00628521614004638\\
425.01	0.00628521496374766\\
426.01	0.00628521375580245\\
427.01	0.00628521251520094\\
428.01	0.00628521124088995\\
429.01	0.00628520993177042\\
430.01	0.00628520858669443\\
431.01	0.00628520720446228\\
432.01	0.00628520578381897\\
433.01	0.00628520432345067\\
434.01	0.00628520282198063\\
435.01	0.0062852012779649\\
436.01	0.00628519968988755\\
437.01	0.00628519805615545\\
438.01	0.00628519637509245\\
439.01	0.00628519464493309\\
440.01	0.00628519286381552\\
441.01	0.00628519102977374\\
442.01	0.00628518914072883\\
443.01	0.00628518719447931\\
444.01	0.0062851851886902\\
445.01	0.00628518312088083\\
446.01	0.00628518098841099\\
447.01	0.00628517878846539\\
448.01	0.00628517651803577\\
449.01	0.00628517417390082\\
450.01	0.00628517175260294\\
451.01	0.00628516925042169\\
452.01	0.00628516666334323\\
453.01	0.00628516398702483\\
454.01	0.00628516121675388\\
455.01	0.0062851583474001\\
456.01	0.00628515537335966\\
457.01	0.00628515228849003\\
458.01	0.00628514908603428\\
459.01	0.00628514575853754\\
460.01	0.00628514229777216\\
461.01	0.00628513869468703\\
462.01	0.0062851349389533\\
463.01	0.00628513101533705\\
464.01	0.0062851268862169\\
465.01	0.00628512239141112\\
466.01	0.00628511682057988\\
467.01	0.00628510891601267\\
468.01	0.00628510004372786\\
469.01	0.00628509011464676\\
470.01	0.00628507547351595\\
471.01	0.00628505436363106\\
472.01	0.00628503251003922\\
473.01	0.00628500992174695\\
474.01	0.00628498638191543\\
475.01	0.00628496205009947\\
476.01	0.00628493706343639\\
477.01	0.00628491139816609\\
478.01	0.00628488502794176\\
479.01	0.00628485792460767\\
480.01	0.00628483005801303\\
481.01	0.00628480139580312\\
482.01	0.00628477190318461\\
483.01	0.00628474154266182\\
484.01	0.00628471027373981\\
485.01	0.00628467805259078\\
486.01	0.00628464483168046\\
487.01	0.00628461055935842\\
488.01	0.00628457517942485\\
489.01	0.00628453863062311\\
490.01	0.00628450084588446\\
491.01	0.00628446175155596\\
492.01	0.00628442126618958\\
493.01	0.00628437930045945\\
494.01	0.00628433575613927\\
495.01	0.00628429052466838\\
496.01	0.00628424348566458\\
497.01	0.00628419450420187\\
498.01	0.00628414342030574\\
499.01	0.00628408998362221\\
500.01	0.00628403340176519\\
501.01	0.00628396934316834\\
502.01	0.00628387594674586\\
503.01	0.00628374474219451\\
504.01	0.00628360861067663\\
505.01	0.00628346796240913\\
506.01	0.00628332256427076\\
507.01	0.0062831721622902\\
508.01	0.00628301647875245\\
509.01	0.00628285520876357\\
510.01	0.00628268801605152\\
511.01	0.00628251452706827\\
512.01	0.00628233431759219\\
513.01	0.00628214685325238\\
514.01	0.00628195114215573\\
515.01	0.00628174387613744\\
516.01	0.00628151401165013\\
517.01	0.00628125786679415\\
518.01	0.00628099012826855\\
519.01	0.00628071031397971\\
520.01	0.00628041243823779\\
521.01	0.0062800628102503\\
522.01	0.00627955818186459\\
523.01	0.00627898773279302\\
524.01	0.00627839941801704\\
525.01	0.00627779231594755\\
526.01	0.00627716542706608\\
527.01	0.00627651766509509\\
528.01	0.00627584784634711\\
529.01	0.00627515467492565\\
530.01	0.00627443672923637\\
531.01	0.00627369243125106\\
532.01	0.00627291992734269\\
533.01	0.00627211658830004\\
534.01	0.00627127799423567\\
535.01	0.00627040002835317\\
536.01	0.0062694664043186\\
537.01	0.00626831754531038\\
538.01	0.00626649145606629\\
539.01	0.00626451741086056\\
540.01	0.0062623421422299\\
541.01	0.00625954994824356\\
542.01	0.00625607368878011\\
543.01	0.00625117487701888\\
544.01	0.00624511955552116\\
545.01	0.00623882134227631\\
546.01	0.00623225770111419\\
547.01	0.00622539543531563\\
548.01	0.00621814372336758\\
549.01	0.00621017509655851\\
550.01	0.00619919570469964\\
551.01	0.00618475779904904\\
552.01	0.00616977537202722\\
553.01	0.00615420605909365\\
554.01	0.00613800225184393\\
555.01	0.00612111014558708\\
556.01	0.00610346788073663\\
557.01	0.00608499731021465\\
558.01	0.00606554234287161\\
559.01	0.00604431311954412\\
560.01	0.00601401718810351\\
561.01	0.00597000049674904\\
562.01	0.00592183900587292\\
563.01	0.00587175812428921\\
564.01	0.00582006423633348\\
565.01	0.00576668023270784\\
566.01	0.0057113088810085\\
567.01	0.00565376273732039\\
568.01	0.00559384842142432\\
569.01	0.00553134735277533\\
570.01	0.00546601127051573\\
571.01	0.00539748147462386\\
572.01	0.00532425881929858\\
573.01	0.00523223672153153\\
574.01	0.00508205373471091\\
575.01	0.00492532175758901\\
576.01	0.00476471806081075\\
577.01	0.004600083778585\\
578.01	0.00443145190274027\\
579.01	0.00425857178856824\\
580.01	0.00408110912050806\\
581.01	0.00389868410416532\\
582.01	0.00371086719773854\\
583.01	0.00351722652953084\\
584.01	0.00331751222637499\\
585.01	0.00308986283227971\\
586.01	0.00292716497304702\\
587.01	0.00278898404456788\\
588.01	0.00264703419695185\\
589.01	0.00250118297809676\\
590.01	0.0023512946155854\\
591.01	0.00219724220680509\\
592.01	0.00203890843405107\\
593.01	0.0018761942862383\\
594.01	0.00170902362200746\\
595.01	0.00153731355151368\\
596.01	0.00136061732988777\\
597.01	0.00117486750438617\\
598.01	0.000944118809282362\\
599.01	0.000412707488818912\\
599.02	0.000403763707257631\\
599.03	0.000394738618952119\\
599.04	0.000385633799629855\\
599.05	0.000376451016479014\\
599.06	0.000367192239998626\\
599.07	0.000357859656498377\\
599.08	0.000348455681284034\\
599.09	0.000338982972567762\\
599.1	0.00032944444613737\\
599.11	0.000319843290829534\\
599.12	0.000310182984851588\\
599.13	0.000300467312994089\\
599.14	0.000290700384783934\\
599.15	0.000280886653628904\\
599.16	0.000271030937007258\\
599.17	0.000261138437758839\\
599.18	0.000251214766537227\\
599.19	0.000241265965485694\\
599.2	0.000231298533203026\\
599.21	0.00022132044869011\\
599.22	0.000211340509823443\\
599.23	0.000201366838978293\\
599.24	0.000191408133016347\\
599.25	0.000181474851944197\\
599.26	0.000171581039711705\\
599.27	0.000161737538486348\\
599.28	0.000151963437707387\\
599.29	0.000142273737523882\\
599.3	0.000132693943086811\\
599.31	0.000123249963771816\\
599.32	0.00011395817999805\\
599.33	0.000104835993820664\\
599.34	9.59018791959702e-05\\
599.35	8.7175440601309e-05\\
599.36	7.8677474740033e-05\\
599.37	7.04300354867297e-05\\
599.38	6.24565022345654e-05\\
599.39	5.47816518136814e-05\\
599.4	4.74317341564879e-05\\
599.41	4.04345518931083e-05\\
599.42	3.38517737233143e-05\\
599.43	2.77322506769592e-05\\
599.44	2.21111563562509e-05\\
599.45	1.70256734628362e-05\\
599.46	1.25150998088776e-05\\
599.47	8.62095959768341e-06\\
599.48	5.38712020206437e-06\\
599.49	2.8585274958904e-06\\
599.5	1.0833850832611e-06\\
599.51	1.13220873875983e-07\\
599.52	0\\
599.53	1.73472347597681e-18\\
599.54	0\\
599.55	0\\
599.56	0\\
599.57	1.73472347597681e-18\\
599.58	0\\
599.59	0\\
599.6	0\\
599.61	0\\
599.62	1.73472347597681e-18\\
599.63	0\\
599.64	1.73472347597681e-18\\
599.65	0\\
599.66	0\\
599.67	0\\
599.68	0\\
599.69	1.73472347597681e-18\\
599.7	0\\
599.71	1.73472347597681e-18\\
599.72	1.73472347597681e-18\\
599.73	1.73472347597681e-18\\
599.74	0\\
599.75	0\\
599.76	0\\
599.77	0\\
599.78	0\\
599.79	0\\
599.8	0\\
599.81	0\\
599.82	0\\
599.83	1.73472347597681e-18\\
599.84	0\\
599.85	0\\
599.86	0\\
599.87	0\\
599.88	0\\
599.89	0\\
599.9	0\\
599.91	0\\
599.92	0\\
599.93	0\\
599.94	0\\
599.95	0\\
599.96	0\\
599.97	0\\
599.98	0\\
599.99	0\\
600	0\\
};
\addplot [color=mycolor12,solid,forget plot]
  table[row sep=crcr]{%
0.01	0.01\\
1.01	0.01\\
2.01	0.01\\
3.01	0.01\\
4.01	0.01\\
5.01	0.01\\
6.01	0.01\\
7.01	0.01\\
8.01	0.01\\
9.01	0.01\\
10.01	0.01\\
11.01	0.01\\
12.01	0.01\\
13.01	0.01\\
14.01	0.01\\
15.01	0.01\\
16.01	0.01\\
17.01	0.01\\
18.01	0.01\\
19.01	0.01\\
20.01	0.01\\
21.01	0.01\\
22.01	0.01\\
23.01	0.01\\
24.01	0.01\\
25.01	0.01\\
26.01	0.01\\
27.01	0.01\\
28.01	0.01\\
29.01	0.01\\
30.01	0.01\\
31.01	0.01\\
32.01	0.01\\
33.01	0.01\\
34.01	0.01\\
35.01	0.01\\
36.01	0.01\\
37.01	0.01\\
38.01	0.01\\
39.01	0.01\\
40.01	0.01\\
41.01	0.01\\
42.01	0.01\\
43.01	0.01\\
44.01	0.01\\
45.01	0.01\\
46.01	0.01\\
47.01	0.01\\
48.01	0.01\\
49.01	0.01\\
50.01	0.01\\
51.01	0.01\\
52.01	0.01\\
53.01	0.01\\
54.01	0.01\\
55.01	0.01\\
56.01	0.01\\
57.01	0.01\\
58.01	0.01\\
59.01	0.01\\
60.01	0.01\\
61.01	0.01\\
62.01	0.01\\
63.01	0.01\\
64.01	0.01\\
65.01	0.01\\
66.01	0.01\\
67.01	0.01\\
68.01	0.01\\
69.01	0.01\\
70.01	0.01\\
71.01	0.01\\
72.01	0.01\\
73.01	0.01\\
74.01	0.01\\
75.01	0.01\\
76.01	0.01\\
77.01	0.01\\
78.01	0.01\\
79.01	0.01\\
80.01	0.01\\
81.01	0.01\\
82.01	0.01\\
83.01	0.01\\
84.01	0.01\\
85.01	0.01\\
86.01	0.01\\
87.01	0.01\\
88.01	0.01\\
89.01	0.01\\
90.01	0.01\\
91.01	0.01\\
92.01	0.01\\
93.01	0.01\\
94.01	0.01\\
95.01	0.01\\
96.01	0.01\\
97.01	0.01\\
98.01	0.01\\
99.01	0.01\\
100.01	0.01\\
101.01	0.01\\
102.01	0.01\\
103.01	0.01\\
104.01	0.01\\
105.01	0.01\\
106.01	0.01\\
107.01	0.01\\
108.01	0.01\\
109.01	0.01\\
110.01	0.01\\
111.01	0.01\\
112.01	0.01\\
113.01	0.01\\
114.01	0.01\\
115.01	0.01\\
116.01	0.01\\
117.01	0.01\\
118.01	0.01\\
119.01	0.01\\
120.01	0.01\\
121.01	0.01\\
122.01	0.01\\
123.01	0.01\\
124.01	0.01\\
125.01	0.01\\
126.01	0.01\\
127.01	0.01\\
128.01	0.01\\
129.01	0.01\\
130.01	0.01\\
131.01	0.01\\
132.01	0.01\\
133.01	0.01\\
134.01	0.01\\
135.01	0.01\\
136.01	0.01\\
137.01	0.01\\
138.01	0.01\\
139.01	0.01\\
140.01	0.01\\
141.01	0.01\\
142.01	0.01\\
143.01	0.01\\
144.01	0.01\\
145.01	0.01\\
146.01	0.01\\
147.01	0.01\\
148.01	0.01\\
149.01	0.01\\
150.01	0.01\\
151.01	0.01\\
152.01	0.01\\
153.01	0.01\\
154.01	0.01\\
155.01	0.01\\
156.01	0.01\\
157.01	0.01\\
158.01	0.01\\
159.01	0.01\\
160.01	0.01\\
161.01	0.01\\
162.01	0.01\\
163.01	0.01\\
164.01	0.01\\
165.01	0.01\\
166.01	0.01\\
167.01	0.01\\
168.01	0.01\\
169.01	0.01\\
170.01	0.01\\
171.01	0.01\\
172.01	0.01\\
173.01	0.01\\
174.01	0.01\\
175.01	0.01\\
176.01	0.01\\
177.01	0.01\\
178.01	0.01\\
179.01	0.01\\
180.01	0.01\\
181.01	0.01\\
182.01	0.01\\
183.01	0.01\\
184.01	0.01\\
185.01	0.01\\
186.01	0.01\\
187.01	0.01\\
188.01	0.01\\
189.01	0.01\\
190.01	0.01\\
191.01	0.01\\
192.01	0.01\\
193.01	0.01\\
194.01	0.01\\
195.01	0.01\\
196.01	0.01\\
197.01	0.01\\
198.01	0.01\\
199.01	0.01\\
200.01	0.01\\
201.01	0.01\\
202.01	0.01\\
203.01	0.01\\
204.01	0.01\\
205.01	0.01\\
206.01	0.01\\
207.01	0.01\\
208.01	0.01\\
209.01	0.01\\
210.01	0.01\\
211.01	0.01\\
212.01	0.01\\
213.01	0.01\\
214.01	0.01\\
215.01	0.01\\
216.01	0.01\\
217.01	0.01\\
218.01	0.01\\
219.01	0.01\\
220.01	0.01\\
221.01	0.01\\
222.01	0.01\\
223.01	0.01\\
224.01	0.01\\
225.01	0.01\\
226.01	0.01\\
227.01	0.01\\
228.01	0.01\\
229.01	0.01\\
230.01	0.01\\
231.01	0.01\\
232.01	0.01\\
233.01	0.01\\
234.01	0.01\\
235.01	0.01\\
236.01	0.01\\
237.01	0.01\\
238.01	0.01\\
239.01	0.01\\
240.01	0.01\\
241.01	0.01\\
242.01	0.01\\
243.01	0.01\\
244.01	0.01\\
245.01	0.01\\
246.01	0.01\\
247.01	0.01\\
248.01	0.01\\
249.01	0.01\\
250.01	0.01\\
251.01	0.01\\
252.01	0.01\\
253.01	0.01\\
254.01	0.01\\
255.01	0.01\\
256.01	0.01\\
257.01	0.01\\
258.01	0.01\\
259.01	0.01\\
260.01	0.01\\
261.01	0.01\\
262.01	0.01\\
263.01	0.01\\
264.01	0.01\\
265.01	0.01\\
266.01	0.01\\
267.01	0.01\\
268.01	0.01\\
269.01	0.01\\
270.01	0.01\\
271.01	0.01\\
272.01	0.01\\
273.01	0.01\\
274.01	0.01\\
275.01	0.01\\
276.01	0.01\\
277.01	0.01\\
278.01	0.01\\
279.01	0.01\\
280.01	0.01\\
281.01	0.01\\
282.01	0.01\\
283.01	0.01\\
284.01	0.01\\
285.01	0.01\\
286.01	0.01\\
287.01	0.01\\
288.01	0.01\\
289.01	0.01\\
290.01	0.01\\
291.01	0.01\\
292.01	0.01\\
293.01	0.01\\
294.01	0.01\\
295.01	0.01\\
296.01	0.01\\
297.01	0.01\\
298.01	0.01\\
299.01	0.01\\
300.01	0.01\\
301.01	0.01\\
302.01	0.01\\
303.01	0.01\\
304.01	0.01\\
305.01	0.01\\
306.01	0.01\\
307.01	0.01\\
308.01	0.01\\
309.01	0.01\\
310.01	0.01\\
311.01	0.01\\
312.01	0.01\\
313.01	0.01\\
314.01	0.01\\
315.01	0.01\\
316.01	0.01\\
317.01	0.01\\
318.01	0.01\\
319.01	0.01\\
320.01	0.01\\
321.01	0.01\\
322.01	0.01\\
323.01	0.01\\
324.01	0.01\\
325.01	0.01\\
326.01	0.01\\
327.01	0.01\\
328.01	0.01\\
329.01	0.01\\
330.01	0.01\\
331.01	0.01\\
332.01	0.01\\
333.01	0.01\\
334.01	0.01\\
335.01	0.01\\
336.01	0.01\\
337.01	0.01\\
338.01	0.01\\
339.01	0.01\\
340.01	0.01\\
341.01	0.01\\
342.01	0.01\\
343.01	0.01\\
344.01	0.01\\
345.01	0.01\\
346.01	0.01\\
347.01	0.01\\
348.01	0.01\\
349.01	0.01\\
350.01	0.01\\
351.01	0.01\\
352.01	0.01\\
353.01	0.01\\
354.01	0.01\\
355.01	0.01\\
356.01	0.01\\
357.01	0.01\\
358.01	0.01\\
359.01	0.01\\
360.01	0.01\\
361.01	0.01\\
362.01	0.01\\
363.01	0.01\\
364.01	0.01\\
365.01	0.01\\
366.01	0.01\\
367.01	0.01\\
368.01	0.01\\
369.01	0.01\\
370.01	0.01\\
371.01	0.01\\
372.01	0.01\\
373.01	0.01\\
374.01	0.01\\
375.01	0.01\\
376.01	0.01\\
377.01	0.01\\
378.01	0.01\\
379.01	0.01\\
380.01	0.01\\
381.01	0.01\\
382.01	0.01\\
383.01	0.01\\
384.01	0.01\\
385.01	0.01\\
386.01	0.01\\
387.01	0.01\\
388.01	0.01\\
389.01	0.01\\
390.01	0.01\\
391.01	0.01\\
392.01	0.01\\
393.01	0.01\\
394.01	0.01\\
395.01	0.01\\
396.01	0.01\\
397.01	0.01\\
398.01	0.01\\
399.01	0.01\\
400.01	0.01\\
401.01	0.01\\
402.01	0.01\\
403.01	0.01\\
404.01	0.01\\
405.01	0.01\\
406.01	0.01\\
407.01	0.01\\
408.01	0.01\\
409.01	0.01\\
410.01	0.01\\
411.01	0.01\\
412.01	0.01\\
413.01	0.01\\
414.01	0.01\\
415.01	0.01\\
416.01	0.01\\
417.01	0.01\\
418.01	0.01\\
419.01	0.01\\
420.01	0.01\\
421.01	0.01\\
422.01	0.01\\
423.01	0.01\\
424.01	0.01\\
425.01	0.01\\
426.01	0.01\\
427.01	0.01\\
428.01	0.01\\
429.01	0.01\\
430.01	0.01\\
431.01	0.01\\
432.01	0.01\\
433.01	0.01\\
434.01	0.01\\
435.01	0.01\\
436.01	0.01\\
437.01	0.01\\
438.01	0.01\\
439.01	0.01\\
440.01	0.01\\
441.01	0.01\\
442.01	0.01\\
443.01	0.01\\
444.01	0.01\\
445.01	0.01\\
446.01	0.01\\
447.01	0.01\\
448.01	0.01\\
449.01	0.01\\
450.01	0.01\\
451.01	0.01\\
452.01	0.01\\
453.01	0.01\\
454.01	0.01\\
455.01	0.01\\
456.01	0.01\\
457.01	0.01\\
458.01	0.01\\
459.01	0.01\\
460.01	0.01\\
461.01	0.01\\
462.01	0.01\\
463.01	0.01\\
464.01	0.01\\
465.01	0.01\\
466.01	0.01\\
467.01	0.01\\
468.01	0.01\\
469.01	0.01\\
470.01	0.01\\
471.01	0.01\\
472.01	0.01\\
473.01	0.01\\
474.01	0.01\\
475.01	0.01\\
476.01	0.01\\
477.01	0.01\\
478.01	0.01\\
479.01	0.01\\
480.01	0.01\\
481.01	0.01\\
482.01	0.01\\
483.01	0.01\\
484.01	0.01\\
485.01	0.01\\
486.01	0.01\\
487.01	0.01\\
488.01	0.01\\
489.01	0.01\\
490.01	0.01\\
491.01	0.01\\
492.01	0.01\\
493.01	0.01\\
494.01	0.01\\
495.01	0.01\\
496.01	0.01\\
497.01	0.01\\
498.01	0.01\\
499.01	0.01\\
500.01	0.01\\
501.01	0.01\\
502.01	0.01\\
503.01	0.01\\
504.01	0.01\\
505.01	0.01\\
506.01	0.01\\
507.01	0.01\\
508.01	0.01\\
509.01	0.01\\
510.01	0.01\\
511.01	0.01\\
512.01	0.01\\
513.01	0.01\\
514.01	0.01\\
515.01	0.01\\
516.01	0.01\\
517.01	0.01\\
518.01	0.01\\
519.01	0.01\\
520.01	0.01\\
521.01	0.01\\
522.01	0.01\\
523.01	0.01\\
524.01	0.01\\
525.01	0.01\\
526.01	0.01\\
527.01	0.01\\
528.01	0.01\\
529.01	0.01\\
530.01	0.01\\
531.01	0.01\\
532.01	0.01\\
533.01	0.01\\
534.01	0.01\\
535.01	0.01\\
536.01	0.01\\
537.01	0.01\\
538.01	0.01\\
539.01	0.01\\
540.01	0.01\\
541.01	0.01\\
542.01	0.01\\
543.01	0.01\\
544.01	0.01\\
545.01	0.01\\
546.01	0.01\\
547.01	0.01\\
548.01	0.01\\
549.01	0.01\\
550.01	0.01\\
551.01	0.01\\
552.01	0.01\\
553.01	0.01\\
554.01	0.01\\
555.01	0.01\\
556.01	0.01\\
557.01	0.01\\
558.01	0.01\\
559.01	0.01\\
560.01	0.01\\
561.01	0.01\\
562.01	0.01\\
563.01	0.01\\
564.01	0.01\\
565.01	0.01\\
566.01	0.01\\
567.01	0.01\\
568.01	0.01\\
569.01	0.01\\
570.01	0.01\\
571.01	0.01\\
572.01	0.01\\
573.01	0.01\\
574.01	0.01\\
575.01	0.01\\
576.01	0.01\\
577.01	0.01\\
578.01	0.01\\
579.01	0.01\\
580.01	0.01\\
581.01	0.01\\
582.01	0.01\\
583.01	0.01\\
584.01	0.01\\
585.01	0.01\\
586.01	0.0098241999093732\\
587.01	0.00961107658299926\\
588.01	0.00939241819816822\\
589.01	0.00916777261536746\\
590.01	0.00893660772687111\\
591.01	0.00869831500266637\\
592.01	0.00845219178373418\\
593.01	0.00819742465069798\\
594.01	0.00793306018611432\\
595.01	0.00765792511149237\\
596.01	0.00737019251449302\\
597.01	0.00706398766549632\\
598.01	0.00669942449313359\\
599.01	0.00590910258711776\\
599.02	0.00589375841662357\\
599.03	0.00587812370952707\\
599.04	0.0058621910335398\\
599.05	0.00584595274653486\\
599.06	0.00582940098980269\\
599.07	0.00581252768105277\\
599.08	0.00579532450714961\\
599.09	0.00577778291657132\\
599.1	0.00575989411157794\\
599.11	0.0057416490400764\\
599.12	0.00572303838716803\\
599.13	0.00570405256636397\\
599.14	0.0056846817104528\\
599.15	0.00566491566200403\\
599.16	0.00564474396349013\\
599.17	0.00562415584700877\\
599.18	0.00560314022358585\\
599.19	0.00558168567203894\\
599.2	0.00555978042737953\\
599.21	0.00553741236731075\\
599.22	0.00551456899916912\\
599.23	0.00549123744848741\\
599.24	0.00546740444529818\\
599.25	0.00544305630820636\\
599.26	0.00541817892527457\\
599.27	0.00539275774382929\\
599.28	0.00536677774354027\\
599.29	0.00534022342605056\\
599.3	0.00531307878341821\\
599.31	0.0052853272797771\\
599.32	0.00525695184621571\\
599.33	0.00522793486025739\\
599.34	0.00519825812438913\\
599.35	0.00516790284357519\\
599.36	0.00513684960169748\\
599.37	0.00510507833686053\\
599.38	0.00507256831549451\\
599.39	0.00503929810518537\\
599.4	0.00500524554615594\\
599.41	0.00497038772131652\\
599.42	0.00493470087968157\\
599.43	0.00489816042254417\\
599.44	0.00486074088969181\\
599.45	0.00482241592249718\\
599.46	0.00478315822505225\\
599.47	0.00474293952321862\\
599.48	0.00470173052145797\\
599.49	0.00465950220083545\\
599.5	0.00461622374084655\\
599.51	0.00457186255544324\\
599.52	0.00452638719662633\\
599.53	0.0044797659435815\\
599.54	0.00443196991276706\\
599.55	0.00438296832083659\\
599.56	0.00433274130624297\\
599.57	0.00428125092075455\\
599.58	0.00422845780611208\\
599.59	0.00417432112792187\\
599.6	0.00411879850560301\\
599.61	0.0040618459380975\\
599.62	0.00400344767773484\\
599.63	0.00394356701252722\\
599.64	0.0038821559017218\\
599.65	0.00381916439423846\\
599.66	0.00375454053122845\\
599.67	0.00368823024230304\\
599.68	0.00362017723488921\\
599.69	0.00355032287614973\\
599.7	0.00347860606684995\\
599.71	0.00340496310649239\\
599.72	0.00332932754897175\\
599.73	0.00325163004792685\\
599.74	0.00317179819083822\\
599.75	0.00308975632093623\\
599.76	0.00300542534578846\\
599.77	0.00291872253132427\\
599.78	0.00282956127992724\\
599.79	0.00273785089105788\\
599.8	0.00264349630273556\\
599.81	0.00254639781197214\\
599.82	0.00244645077204294\\
599.83	0.00234354526422315\\
599.84	0.00223756574132511\\
599.85	0.0021283906400423\\
599.86	0.00201589195872525\\
599.87	0.00189993479677869\\
599.88	0.00178037685136625\\
599.89	0.00165706786652964\\
599.9	0.00152984902915766\\
599.91	0.00139855230546207\\
599.92	0.00126299971071054\\
599.93	0.00112300250390927\\
599.94	0.000978360297888684\\
599.95	0.000828860073790389\\
599.96	0.000674275087237755\\
599.97	0.000514363651443209\\
599.98	0.00034886778009568\\
599.99	0.000177511669999688\\
600	0\\
};
\addplot [color=mycolor13,solid,forget plot]
  table[row sep=crcr]{%
0.01	0\\
1.01	0\\
2.01	0\\
3.01	0\\
4.01	0\\
5.01	0\\
6.01	0\\
7.01	0\\
8.01	0\\
9.01	0\\
10.01	0\\
11.01	0\\
12.01	0\\
13.01	0\\
14.01	0\\
15.01	0\\
16.01	0\\
17.01	0\\
18.01	0\\
19.01	0\\
20.01	0\\
21.01	0\\
22.01	0\\
23.01	0\\
24.01	0\\
25.01	0\\
26.01	0\\
27.01	0\\
28.01	0\\
29.01	0\\
30.01	0\\
31.01	0\\
32.01	0\\
33.01	0\\
34.01	0\\
35.01	0\\
36.01	0\\
37.01	0\\
38.01	0\\
39.01	0\\
40.01	0\\
41.01	0\\
42.01	0\\
43.01	0\\
44.01	0\\
45.01	0\\
46.01	0\\
47.01	0\\
48.01	0\\
49.01	0\\
50.01	0\\
51.01	0\\
52.01	0\\
53.01	0\\
54.01	0\\
55.01	0\\
56.01	0\\
57.01	0\\
58.01	0\\
59.01	0\\
60.01	0\\
61.01	0\\
62.01	0\\
63.01	0\\
64.01	0\\
65.01	0\\
66.01	0\\
67.01	0\\
68.01	0\\
69.01	0\\
70.01	0\\
71.01	0\\
72.01	0\\
73.01	0\\
74.01	0\\
75.01	0\\
76.01	0\\
77.01	0\\
78.01	0\\
79.01	0\\
80.01	0\\
81.01	0\\
82.01	0\\
83.01	0\\
84.01	0\\
85.01	0\\
86.01	0\\
87.01	0\\
88.01	0\\
89.01	0\\
90.01	0\\
91.01	0\\
92.01	0\\
93.01	0\\
94.01	0\\
95.01	0\\
96.01	0\\
97.01	0\\
98.01	0\\
99.01	0\\
100.01	0\\
101.01	0\\
102.01	0\\
103.01	0\\
104.01	0\\
105.01	0\\
106.01	0\\
107.01	0\\
108.01	0\\
109.01	0\\
110.01	0\\
111.01	0\\
112.01	0\\
113.01	0\\
114.01	0\\
115.01	0\\
116.01	0\\
117.01	0\\
118.01	0\\
119.01	0\\
120.01	0\\
121.01	0\\
122.01	0\\
123.01	0\\
124.01	0\\
125.01	0\\
126.01	0\\
127.01	0\\
128.01	0\\
129.01	0\\
130.01	0\\
131.01	0\\
132.01	0\\
133.01	0\\
134.01	0\\
135.01	0\\
136.01	0\\
137.01	0\\
138.01	0\\
139.01	0\\
140.01	0\\
141.01	0\\
142.01	0\\
143.01	0\\
144.01	0\\
145.01	0\\
146.01	0\\
147.01	0\\
148.01	0\\
149.01	0\\
150.01	0\\
151.01	0\\
152.01	0\\
153.01	0\\
154.01	0\\
155.01	0\\
156.01	0\\
157.01	0\\
158.01	0\\
159.01	0\\
160.01	0\\
161.01	0\\
162.01	0\\
163.01	0\\
164.01	0\\
165.01	0\\
166.01	0\\
167.01	0\\
168.01	0\\
169.01	0\\
170.01	0\\
171.01	0\\
172.01	0\\
173.01	0\\
174.01	0\\
175.01	0\\
176.01	0\\
177.01	0\\
178.01	0\\
179.01	0\\
180.01	0\\
181.01	0\\
182.01	0\\
183.01	0\\
184.01	0\\
185.01	0\\
186.01	0\\
187.01	0\\
188.01	0\\
189.01	0\\
190.01	0\\
191.01	0\\
192.01	0\\
193.01	0\\
194.01	0\\
195.01	0\\
196.01	0\\
197.01	0\\
198.01	0\\
199.01	0\\
200.01	0\\
201.01	0\\
202.01	0\\
203.01	0\\
204.01	0\\
205.01	0\\
206.01	0\\
207.01	0\\
208.01	0\\
209.01	0\\
210.01	0\\
211.01	0\\
212.01	0\\
213.01	0\\
214.01	0\\
215.01	0\\
216.01	0\\
217.01	0\\
218.01	0\\
219.01	0\\
220.01	0\\
221.01	0\\
222.01	0\\
223.01	0\\
224.01	0\\
225.01	0\\
226.01	0\\
227.01	0\\
228.01	0\\
229.01	0\\
230.01	0\\
231.01	0\\
232.01	0\\
233.01	0\\
234.01	0\\
235.01	0\\
236.01	0\\
237.01	0\\
238.01	0\\
239.01	0\\
240.01	0\\
241.01	0\\
242.01	0\\
243.01	0\\
244.01	0\\
245.01	0\\
246.01	0\\
247.01	0\\
248.01	0\\
249.01	0\\
250.01	0\\
251.01	0\\
252.01	0\\
253.01	0\\
254.01	0\\
255.01	0\\
256.01	0\\
257.01	0\\
258.01	0\\
259.01	0\\
260.01	0\\
261.01	0\\
262.01	0\\
263.01	0\\
264.01	0\\
265.01	0\\
266.01	0\\
267.01	0\\
268.01	0\\
269.01	0\\
270.01	0\\
271.01	0\\
272.01	0\\
273.01	0\\
274.01	0\\
275.01	0\\
276.01	0\\
277.01	0\\
278.01	0\\
279.01	0\\
280.01	0\\
281.01	0\\
282.01	0\\
283.01	0\\
284.01	0\\
285.01	0\\
286.01	0\\
287.01	0\\
288.01	0\\
289.01	0\\
290.01	0\\
291.01	0\\
292.01	0\\
293.01	0\\
294.01	0\\
295.01	0\\
296.01	0\\
297.01	0\\
298.01	0\\
299.01	0\\
300.01	0\\
301.01	0\\
302.01	0\\
303.01	0\\
304.01	0\\
305.01	0\\
306.01	0\\
307.01	0\\
308.01	0\\
309.01	0\\
310.01	0\\
311.01	0\\
312.01	0\\
313.01	0\\
314.01	0\\
315.01	0\\
316.01	0\\
317.01	0\\
318.01	0\\
319.01	0\\
320.01	0\\
321.01	0\\
322.01	0\\
323.01	0\\
324.01	0\\
325.01	0\\
326.01	0\\
327.01	0\\
328.01	0\\
329.01	0\\
330.01	0\\
331.01	0\\
332.01	0\\
333.01	0\\
334.01	0\\
335.01	0\\
336.01	0\\
337.01	0\\
338.01	0\\
339.01	0\\
340.01	0\\
341.01	0\\
342.01	0\\
343.01	0\\
344.01	0\\
345.01	0\\
346.01	0\\
347.01	0\\
348.01	0\\
349.01	0\\
350.01	0\\
351.01	0\\
352.01	0\\
353.01	0\\
354.01	0\\
355.01	0\\
356.01	0\\
357.01	0\\
358.01	0\\
359.01	0\\
360.01	0\\
361.01	0\\
362.01	0\\
363.01	0\\
364.01	0\\
365.01	0\\
366.01	0\\
367.01	0\\
368.01	0\\
369.01	0\\
370.01	0\\
371.01	0\\
372.01	0\\
373.01	0\\
374.01	0\\
375.01	0\\
376.01	0\\
377.01	0\\
378.01	0\\
379.01	0\\
380.01	0\\
381.01	0\\
382.01	0\\
383.01	0\\
384.01	0\\
385.01	0\\
386.01	0\\
387.01	0\\
388.01	0\\
389.01	0\\
390.01	0\\
391.01	0\\
392.01	0\\
393.01	0\\
394.01	0\\
395.01	0\\
396.01	0\\
397.01	0\\
398.01	0\\
399.01	0\\
400.01	0\\
401.01	0\\
402.01	0\\
403.01	0\\
404.01	0\\
405.01	0\\
406.01	0\\
407.01	0\\
408.01	0\\
409.01	0\\
410.01	0\\
411.01	0\\
412.01	0\\
413.01	0\\
414.01	0\\
415.01	0\\
416.01	0\\
417.01	0\\
418.01	0\\
419.01	0\\
420.01	0\\
421.01	0\\
422.01	0\\
423.01	0\\
424.01	0\\
425.01	0\\
426.01	0\\
427.01	0\\
428.01	0\\
429.01	0\\
430.01	0\\
431.01	0\\
432.01	0\\
433.01	0\\
434.01	0\\
435.01	0\\
436.01	0\\
437.01	0\\
438.01	0\\
439.01	0\\
440.01	0\\
441.01	0\\
442.01	0\\
443.01	0\\
444.01	0\\
445.01	0\\
446.01	0\\
447.01	0\\
448.01	0\\
449.01	0\\
450.01	0\\
451.01	0\\
452.01	0\\
453.01	0\\
454.01	0\\
455.01	0\\
456.01	0\\
457.01	0\\
458.01	0\\
459.01	0\\
460.01	0\\
461.01	0\\
462.01	0\\
463.01	0\\
464.01	0\\
465.01	0\\
466.01	0\\
467.01	0\\
468.01	0\\
469.01	0\\
470.01	0\\
471.01	0\\
472.01	0\\
473.01	0\\
474.01	0\\
475.01	0\\
476.01	0\\
477.01	0\\
478.01	0\\
479.01	0\\
480.01	0\\
481.01	0\\
482.01	0\\
483.01	0\\
484.01	0\\
485.01	0\\
486.01	0\\
487.01	0\\
488.01	0\\
489.01	0\\
490.01	0\\
491.01	0\\
492.01	0\\
493.01	0\\
494.01	0\\
495.01	0\\
496.01	0\\
497.01	0\\
498.01	0\\
499.01	0\\
500.01	0\\
501.01	0\\
502.01	0\\
503.01	0\\
504.01	0\\
505.01	0\\
506.01	0\\
507.01	0\\
508.01	0\\
509.01	0\\
510.01	0\\
511.01	0\\
512.01	0\\
513.01	0\\
514.01	0\\
515.01	0\\
516.01	0\\
517.01	0\\
518.01	0\\
519.01	0\\
520.01	0\\
521.01	0\\
522.01	0\\
523.01	0\\
524.01	0\\
525.01	0\\
526.01	0\\
527.01	0\\
528.01	0\\
529.01	0\\
530.01	0\\
531.01	0\\
532.01	0\\
533.01	0\\
534.01	0\\
535.01	0\\
536.01	0\\
537.01	0\\
538.01	0\\
539.01	0\\
540.01	0\\
541.01	0\\
542.01	0\\
543.01	0\\
544.01	0\\
545.01	0\\
546.01	0\\
547.01	0\\
548.01	0\\
549.01	0\\
550.01	0\\
551.01	0\\
552.01	0\\
553.01	0\\
554.01	0\\
555.01	0\\
556.01	0\\
557.01	0\\
558.01	0\\
559.01	0\\
560.01	0\\
561.01	0\\
562.01	0\\
563.01	0\\
564.01	0\\
565.01	0\\
566.01	0\\
567.01	0\\
568.01	0\\
569.01	0\\
570.01	0\\
571.01	0\\
572.01	0\\
573.01	0\\
574.01	0\\
575.01	0\\
576.01	0\\
577.01	0\\
578.01	0\\
579.01	0\\
580.01	0\\
581.01	0\\
582.01	0\\
583.01	0\\
584.01	0\\
585.01	0\\
586.01	0\\
587.01	0\\
588.01	0\\
589.01	0\\
590.01	0\\
591.01	0\\
592.01	0\\
593.01	0\\
594.01	0\\
595.01	0\\
596.01	0\\
597.01	0\\
598.01	0\\
599.01	0\\
599.02	0\\
599.03	0\\
599.04	0\\
599.05	0\\
599.06	0\\
599.07	0\\
599.08	0\\
599.09	0\\
599.1	0\\
599.11	0\\
599.12	0\\
599.13	0\\
599.14	0\\
599.15	0\\
599.16	0\\
599.17	0\\
599.18	0\\
599.19	0\\
599.2	0\\
599.21	0\\
599.22	0\\
599.23	0\\
599.24	0\\
599.25	0\\
599.26	0\\
599.27	0\\
599.28	0\\
599.29	0\\
599.3	0\\
599.31	0\\
599.32	0\\
599.33	0\\
599.34	0\\
599.35	0\\
599.36	0\\
599.37	0\\
599.38	0\\
599.39	0\\
599.4	0\\
599.41	0\\
599.42	0\\
599.43	0\\
599.44	0\\
599.45	0\\
599.46	0\\
599.47	0\\
599.48	0\\
599.49	0.000108462067765987\\
599.5	0.000274956428996023\\
599.51	0.000442304418925576\\
599.52	0.000610516577113952\\
599.53	0.000779597867790803\\
599.54	0.000949552333876835\\
599.55	0.00112039122267384\\
599.56	0.00129212590665678\\
599.57	0.00146476792988524\\
599.58	0.0016383296358674\\
599.59	0.00181282359304911\\
599.6	0.00198826293302643\\
599.61	0.00216466121189597\\
599.62	0.0023420324165617\\
599.63	0.00252039097831623\\
599.64	0.00269975178686579\\
599.65	0.00288013020481843\\
599.66	0.00306154208265583\\
599.67	0.00324400377459614\\
599.68	0.00342753215478128\\
599.69	0.00361214463388089\\
599.7	0.00379785917627751\\
599.71	0.0039846943178602\\
599.72	0.00417266918642947\\
599.73	0.00436180352525035\\
599.74	0.00455211771380802\\
599.75	0.00474363278903193\\
599.76	0.00493637046861567\\
599.77	0.00513035317607476\\
599.78	0.00532560406487874\\
599.79	0.0055221470435591\\
599.8	0.00572000680184516\\
599.81	0.00591920883788277\\
599.82	0.00611977948659422\\
599.83	0.00632174594924113\\
599.84	0.00652513632425557\\
599.85	0.00672997963940852\\
599.86	0.00693630588538855\\
599.87	0.00714414605086752\\
599.88	0.00735353215913409\\
599.89	0.00756449730637987\\
599.9	0.00777707570172686\\
599.91	0.00799130270908877\\
599.92	0.00820721489096223\\
599.93	0.00842485005424691\\
599.94	0.00864424729819635\\
599.95	0.00886544706460237\\
599.96	0.00908849119031689\\
599.97	0.00931342296221292\\
599.98	0.00954028717468348\\
599.99	0.00976913018976985\\
600	0.01\\
};
\addplot [color=mycolor14,solid,forget plot]
  table[row sep=crcr]{%
0.01	0\\
1.01	0\\
2.01	0\\
3.01	0\\
4.01	0\\
5.01	0\\
6.01	0\\
7.01	0\\
8.01	0\\
9.01	0\\
10.01	0\\
11.01	0\\
12.01	0\\
13.01	0\\
14.01	0\\
15.01	0\\
16.01	0\\
17.01	0\\
18.01	0\\
19.01	0\\
20.01	0\\
21.01	0\\
22.01	0\\
23.01	0\\
24.01	0\\
25.01	0\\
26.01	0\\
27.01	0\\
28.01	0\\
29.01	0\\
30.01	0\\
31.01	0\\
32.01	0\\
33.01	0\\
34.01	0\\
35.01	0\\
36.01	0\\
37.01	0\\
38.01	0\\
39.01	0\\
40.01	0\\
41.01	0\\
42.01	0\\
43.01	0\\
44.01	0\\
45.01	0\\
46.01	0\\
47.01	0\\
48.01	0\\
49.01	0\\
50.01	0\\
51.01	0\\
52.01	0\\
53.01	0\\
54.01	0\\
55.01	0\\
56.01	0\\
57.01	0\\
58.01	0\\
59.01	0\\
60.01	0\\
61.01	0\\
62.01	0\\
63.01	0\\
64.01	0\\
65.01	0\\
66.01	0\\
67.01	0\\
68.01	0\\
69.01	0\\
70.01	0\\
71.01	0\\
72.01	0\\
73.01	0\\
74.01	0\\
75.01	0\\
76.01	0\\
77.01	0\\
78.01	0\\
79.01	0\\
80.01	0\\
81.01	0\\
82.01	0\\
83.01	0\\
84.01	0\\
85.01	0\\
86.01	0\\
87.01	0\\
88.01	0\\
89.01	0\\
90.01	0\\
91.01	0\\
92.01	0\\
93.01	0\\
94.01	0\\
95.01	0\\
96.01	0\\
97.01	0\\
98.01	0\\
99.01	0\\
100.01	0\\
101.01	0\\
102.01	0\\
103.01	0\\
104.01	0\\
105.01	0\\
106.01	0\\
107.01	0\\
108.01	0\\
109.01	0\\
110.01	0\\
111.01	0\\
112.01	0\\
113.01	0\\
114.01	0\\
115.01	0\\
116.01	0\\
117.01	0\\
118.01	0\\
119.01	0\\
120.01	0\\
121.01	0\\
122.01	0\\
123.01	0\\
124.01	0\\
125.01	0\\
126.01	0\\
127.01	0\\
128.01	0\\
129.01	0\\
130.01	0\\
131.01	0\\
132.01	0\\
133.01	0\\
134.01	0\\
135.01	0\\
136.01	0\\
137.01	0\\
138.01	0\\
139.01	0\\
140.01	0\\
141.01	0\\
142.01	0\\
143.01	0\\
144.01	0\\
145.01	0\\
146.01	0\\
147.01	0\\
148.01	0\\
149.01	0\\
150.01	0\\
151.01	0\\
152.01	0\\
153.01	0\\
154.01	0\\
155.01	0\\
156.01	0\\
157.01	0\\
158.01	0\\
159.01	0\\
160.01	0\\
161.01	0\\
162.01	0\\
163.01	0\\
164.01	0\\
165.01	0\\
166.01	0\\
167.01	0\\
168.01	0\\
169.01	0\\
170.01	0\\
171.01	0\\
172.01	0\\
173.01	0\\
174.01	0\\
175.01	0\\
176.01	0\\
177.01	0\\
178.01	0\\
179.01	0\\
180.01	0\\
181.01	0\\
182.01	0\\
183.01	0\\
184.01	0\\
185.01	0\\
186.01	0\\
187.01	0\\
188.01	0\\
189.01	0\\
190.01	0\\
191.01	0\\
192.01	0\\
193.01	0\\
194.01	0\\
195.01	0\\
196.01	0\\
197.01	0\\
198.01	0\\
199.01	0\\
200.01	0\\
201.01	0\\
202.01	0\\
203.01	0\\
204.01	0\\
205.01	0\\
206.01	0\\
207.01	0\\
208.01	0\\
209.01	0\\
210.01	0\\
211.01	0\\
212.01	0\\
213.01	0\\
214.01	0\\
215.01	0\\
216.01	0\\
217.01	0\\
218.01	0\\
219.01	0\\
220.01	0\\
221.01	0\\
222.01	0\\
223.01	0\\
224.01	0\\
225.01	0\\
226.01	0\\
227.01	0\\
228.01	0\\
229.01	0\\
230.01	0\\
231.01	0\\
232.01	0\\
233.01	0\\
234.01	0\\
235.01	0\\
236.01	0\\
237.01	0\\
238.01	0\\
239.01	0\\
240.01	0\\
241.01	0\\
242.01	0\\
243.01	0\\
244.01	0\\
245.01	0\\
246.01	0\\
247.01	0\\
248.01	0\\
249.01	0\\
250.01	0\\
251.01	0\\
252.01	0\\
253.01	0\\
254.01	0\\
255.01	0\\
256.01	0\\
257.01	0\\
258.01	0\\
259.01	0\\
260.01	0\\
261.01	0\\
262.01	0\\
263.01	0\\
264.01	0\\
265.01	0\\
266.01	0\\
267.01	0\\
268.01	0\\
269.01	0\\
270.01	0\\
271.01	0\\
272.01	0\\
273.01	0\\
274.01	0\\
275.01	0\\
276.01	0\\
277.01	0\\
278.01	0\\
279.01	0\\
280.01	0\\
281.01	0\\
282.01	0\\
283.01	0\\
284.01	0\\
285.01	0\\
286.01	0\\
287.01	0\\
288.01	0\\
289.01	0\\
290.01	0\\
291.01	0\\
292.01	0\\
293.01	0\\
294.01	0\\
295.01	0\\
296.01	0\\
297.01	0\\
298.01	0\\
299.01	0\\
300.01	0\\
301.01	0\\
302.01	0\\
303.01	0\\
304.01	0\\
305.01	0\\
306.01	0\\
307.01	0\\
308.01	0\\
309.01	0\\
310.01	0\\
311.01	0\\
312.01	0\\
313.01	0\\
314.01	0\\
315.01	0\\
316.01	0\\
317.01	0\\
318.01	0\\
319.01	0\\
320.01	0\\
321.01	0\\
322.01	0\\
323.01	0\\
324.01	0\\
325.01	0\\
326.01	0\\
327.01	0\\
328.01	0\\
329.01	0\\
330.01	0\\
331.01	0\\
332.01	0\\
333.01	0\\
334.01	0\\
335.01	0\\
336.01	0\\
337.01	0\\
338.01	0\\
339.01	0\\
340.01	0\\
341.01	0\\
342.01	0\\
343.01	0\\
344.01	0\\
345.01	0\\
346.01	0\\
347.01	0\\
348.01	0\\
349.01	0\\
350.01	0\\
351.01	0\\
352.01	0\\
353.01	0\\
354.01	0\\
355.01	0\\
356.01	0\\
357.01	0\\
358.01	0\\
359.01	0\\
360.01	0\\
361.01	0\\
362.01	0\\
363.01	0\\
364.01	0\\
365.01	0\\
366.01	0\\
367.01	0\\
368.01	0\\
369.01	0\\
370.01	0\\
371.01	0\\
372.01	0\\
373.01	0\\
374.01	0\\
375.01	0\\
376.01	0\\
377.01	0\\
378.01	0\\
379.01	0\\
380.01	0\\
381.01	0\\
382.01	0\\
383.01	0\\
384.01	0\\
385.01	0\\
386.01	0\\
387.01	0\\
388.01	0\\
389.01	0\\
390.01	0\\
391.01	0\\
392.01	0\\
393.01	0\\
394.01	0\\
395.01	0\\
396.01	0\\
397.01	0\\
398.01	0\\
399.01	0\\
400.01	0\\
401.01	0\\
402.01	0\\
403.01	0\\
404.01	0\\
405.01	0\\
406.01	0\\
407.01	0\\
408.01	0\\
409.01	0\\
410.01	0\\
411.01	0\\
412.01	0\\
413.01	0\\
414.01	0\\
415.01	0\\
416.01	0\\
417.01	0\\
418.01	0\\
419.01	0\\
420.01	0\\
421.01	0\\
422.01	0\\
423.01	0\\
424.01	0\\
425.01	0\\
426.01	0\\
427.01	0\\
428.01	0\\
429.01	0\\
430.01	0\\
431.01	0\\
432.01	0\\
433.01	0\\
434.01	0\\
435.01	0\\
436.01	0\\
437.01	0\\
438.01	0\\
439.01	0\\
440.01	0\\
441.01	0\\
442.01	0\\
443.01	0\\
444.01	0\\
445.01	0\\
446.01	0\\
447.01	0\\
448.01	0\\
449.01	0\\
450.01	0\\
451.01	0\\
452.01	0\\
453.01	0\\
454.01	0\\
455.01	0\\
456.01	0\\
457.01	0\\
458.01	0\\
459.01	0\\
460.01	0\\
461.01	0\\
462.01	0\\
463.01	0\\
464.01	0\\
465.01	0\\
466.01	0\\
467.01	0\\
468.01	0\\
469.01	0\\
470.01	0\\
471.01	0\\
472.01	0\\
473.01	0\\
474.01	0\\
475.01	0\\
476.01	0\\
477.01	0\\
478.01	0\\
479.01	0\\
480.01	0\\
481.01	0\\
482.01	0\\
483.01	0\\
484.01	0\\
485.01	0\\
486.01	0\\
487.01	0\\
488.01	0\\
489.01	0\\
490.01	0\\
491.01	0\\
492.01	0\\
493.01	0\\
494.01	0\\
495.01	0\\
496.01	0\\
497.01	0\\
498.01	0\\
499.01	0\\
500.01	0\\
501.01	0\\
502.01	0\\
503.01	0\\
504.01	0\\
505.01	0\\
506.01	0\\
507.01	0\\
508.01	0\\
509.01	0\\
510.01	0\\
511.01	0\\
512.01	0\\
513.01	0\\
514.01	0\\
515.01	0\\
516.01	0\\
517.01	0\\
518.01	0\\
519.01	0\\
520.01	0\\
521.01	0\\
522.01	0\\
523.01	0\\
524.01	0\\
525.01	0\\
526.01	0\\
527.01	0\\
528.01	0\\
529.01	0\\
530.01	0\\
531.01	0\\
532.01	0\\
533.01	0\\
534.01	0\\
535.01	0\\
536.01	0\\
537.01	0\\
538.01	0\\
539.01	0\\
540.01	0\\
541.01	0\\
542.01	0\\
543.01	0\\
544.01	0\\
545.01	0\\
546.01	0\\
547.01	0\\
548.01	0\\
549.01	0\\
550.01	0\\
551.01	0\\
552.01	0\\
553.01	0\\
554.01	0\\
555.01	0\\
556.01	0\\
557.01	0\\
558.01	0\\
559.01	0\\
560.01	0\\
561.01	0\\
562.01	0\\
563.01	0\\
564.01	0\\
565.01	0\\
566.01	0\\
567.01	0\\
568.01	0\\
569.01	0\\
570.01	0\\
571.01	0\\
572.01	0\\
573.01	0\\
574.01	0\\
575.01	0\\
576.01	0\\
577.01	0\\
578.01	0\\
579.01	0\\
580.01	0\\
581.01	0\\
582.01	0\\
583.01	0\\
584.01	0\\
585.01	0\\
586.01	0\\
587.01	0\\
588.01	0\\
589.01	0\\
590.01	0\\
591.01	0\\
592.01	0\\
593.01	0\\
594.01	0\\
595.01	0\\
596.01	0\\
597.01	0\\
598.01	0\\
599.01	0\\
599.02	0\\
599.03	0\\
599.04	0\\
599.05	0\\
599.06	0\\
599.07	0\\
599.08	0\\
599.09	0\\
599.1	0\\
599.11	0\\
599.12	0\\
599.13	0\\
599.14	0\\
599.15	0\\
599.16	0\\
599.17	0\\
599.18	0\\
599.19	0\\
599.2	0\\
599.21	7.43240303883695e-05\\
599.22	0.000264470482307808\\
599.23	0.000455841366224598\\
599.24	0.000648453467216172\\
599.25	0.000842324105170485\\
599.26	0.00103747115493353\\
599.27	0.00123390811714826\\
599.28	0.001431643963674\\
599.29	0.00163069794137809\\
599.3	0.00183108993013174\\
599.31	0.00203284008212145\\
599.32	0.00223596924269511\\
599.33	0.00244049910317951\\
599.34	0.00264645207197108\\
599.35	0.00285385117329218\\
599.36	0.00306272030937642\\
599.37	0.00327308425972395\\
599.38	0.00348496865944219\\
599.39	0.00369840003378491\\
599.4	0.003913405834474\\
599.41	0.00413001447782843\\
599.42	0.00434825538458751\\
599.43	0.00456815902172324\\
599.44	0.00478975694637176\\
599.45	0.00501308185202392\\
599.46	0.00523816761712656\\
599.47	0.00546504935625849\\
599.48	0.00569376347405871\\
599.49	0.0058156133798876\\
599.5	0.00588090468334206\\
599.51	0.00594682986883297\\
599.52	0.00601339241646541\\
599.53	0.00608060161398367\\
599.54	0.00614846796314744\\
599.55	0.00621699500263725\\
599.56	0.0062861863936304\\
599.57	0.00635604587967757\\
599.58	0.00642657666087133\\
599.59	0.00649778197508687\\
599.6	0.00656966476333187\\
599.61	0.00664222781355835\\
599.62	0.00671547375969177\\
599.63	0.00678940507349391\\
599.64	0.00686402405609307\\
599.65	0.00693933282916512\\
599.66	0.00701533332574808\\
599.67	0.0070920272542144\\
599.68	0.00716941610179131\\
599.69	0.00724750113333021\\
599.7	0.0073262833801967\\
599.71	0.00740576362867076\\
599.72	0.00748594227390502\\
599.73	0.00756681913272001\\
599.74	0.00764839366574154\\
599.75	0.00773066498139155\\
599.76	0.0078136317257257\\
599.77	0.00789729199735544\\
599.78	0.00798164345615663\\
599.79	0.00806668329920511\\
599.8	0.00815240823537674\\
599.81	0.00823881445852518\\
599.82	0.00832589761914421\\
599.83	0.00841365279441413\\
599.84	0.00850207445652472\\
599.85	0.00859115643915884\\
599.86	0.00868089190201197\\
599.87	0.00877127329321369\\
599.88	0.00886229230950642\\
599.89	0.00895393985402594\\
599.9	0.00904620599151582\\
599.91	0.00913907990079482\\
599.92	0.00923254982428205\\
599.93	0.00932660301436908\\
599.94	0.00942122567641143\\
599.95	0.00951640290809338\\
599.96	0.00961211863490027\\
599.97	0.00970835554141063\\
599.98	0.00980509499809699\\
599.99	0.00990231698329844\\
600	0.01\\
};
\addplot [color=mycolor15,solid,forget plot]
  table[row sep=crcr]{%
0.01	0\\
1.01	0\\
2.01	0\\
3.01	0\\
4.01	0\\
5.01	0\\
6.01	0\\
7.01	0\\
8.01	0\\
9.01	0\\
10.01	0\\
11.01	0\\
12.01	0\\
13.01	0\\
14.01	0\\
15.01	0\\
16.01	0\\
17.01	0\\
18.01	0\\
19.01	0\\
20.01	0\\
21.01	0\\
22.01	0\\
23.01	0\\
24.01	0\\
25.01	0\\
26.01	0\\
27.01	0\\
28.01	0\\
29.01	0\\
30.01	0\\
31.01	0\\
32.01	0\\
33.01	0\\
34.01	0\\
35.01	0\\
36.01	0\\
37.01	0\\
38.01	0\\
39.01	0\\
40.01	0\\
41.01	0\\
42.01	0\\
43.01	0\\
44.01	0\\
45.01	0\\
46.01	0\\
47.01	0\\
48.01	0\\
49.01	0\\
50.01	0\\
51.01	0\\
52.01	0\\
53.01	0\\
54.01	0\\
55.01	0\\
56.01	0\\
57.01	0\\
58.01	0\\
59.01	0\\
60.01	0\\
61.01	0\\
62.01	0\\
63.01	0\\
64.01	0\\
65.01	0\\
66.01	0\\
67.01	0\\
68.01	0\\
69.01	0\\
70.01	0\\
71.01	0\\
72.01	0\\
73.01	0\\
74.01	0\\
75.01	0\\
76.01	0\\
77.01	0\\
78.01	0\\
79.01	0\\
80.01	0\\
81.01	0\\
82.01	0\\
83.01	0\\
84.01	0\\
85.01	0\\
86.01	0\\
87.01	0\\
88.01	0\\
89.01	0\\
90.01	0\\
91.01	0\\
92.01	0\\
93.01	0\\
94.01	0\\
95.01	0\\
96.01	0\\
97.01	0\\
98.01	0\\
99.01	0\\
100.01	0\\
101.01	0\\
102.01	0\\
103.01	0\\
104.01	0\\
105.01	0\\
106.01	0\\
107.01	0\\
108.01	0\\
109.01	0\\
110.01	0\\
111.01	0\\
112.01	0\\
113.01	0\\
114.01	0\\
115.01	0\\
116.01	0\\
117.01	0\\
118.01	0\\
119.01	0\\
120.01	0\\
121.01	0\\
122.01	0\\
123.01	0\\
124.01	0\\
125.01	0\\
126.01	0\\
127.01	0\\
128.01	0\\
129.01	0\\
130.01	0\\
131.01	0\\
132.01	0\\
133.01	0\\
134.01	0\\
135.01	0\\
136.01	0\\
137.01	0\\
138.01	0\\
139.01	0\\
140.01	0\\
141.01	0\\
142.01	0\\
143.01	0\\
144.01	0\\
145.01	0\\
146.01	0\\
147.01	0\\
148.01	0\\
149.01	0\\
150.01	0\\
151.01	0\\
152.01	0\\
153.01	0\\
154.01	0\\
155.01	0\\
156.01	0\\
157.01	0\\
158.01	0\\
159.01	0\\
160.01	0\\
161.01	0\\
162.01	0\\
163.01	0\\
164.01	0\\
165.01	0\\
166.01	0\\
167.01	0\\
168.01	0\\
169.01	0\\
170.01	0\\
171.01	0\\
172.01	0\\
173.01	0\\
174.01	0\\
175.01	0\\
176.01	0\\
177.01	0\\
178.01	0\\
179.01	0\\
180.01	0\\
181.01	0\\
182.01	0\\
183.01	0\\
184.01	0\\
185.01	0\\
186.01	0\\
187.01	0\\
188.01	0\\
189.01	0\\
190.01	0\\
191.01	0\\
192.01	0\\
193.01	0\\
194.01	0\\
195.01	0\\
196.01	0\\
197.01	0\\
198.01	0\\
199.01	0\\
200.01	0\\
201.01	0\\
202.01	0\\
203.01	0\\
204.01	0\\
205.01	0\\
206.01	0\\
207.01	0\\
208.01	0\\
209.01	0\\
210.01	0\\
211.01	0\\
212.01	0\\
213.01	0\\
214.01	0\\
215.01	0\\
216.01	0\\
217.01	0\\
218.01	0\\
219.01	0\\
220.01	0\\
221.01	0\\
222.01	0\\
223.01	0\\
224.01	0\\
225.01	0\\
226.01	0\\
227.01	0\\
228.01	0\\
229.01	0\\
230.01	0\\
231.01	0\\
232.01	0\\
233.01	0\\
234.01	0\\
235.01	0\\
236.01	0\\
237.01	0\\
238.01	0\\
239.01	0\\
240.01	0\\
241.01	0\\
242.01	0\\
243.01	0\\
244.01	0\\
245.01	0\\
246.01	0\\
247.01	0\\
248.01	0\\
249.01	0\\
250.01	0\\
251.01	0\\
252.01	0\\
253.01	0\\
254.01	0\\
255.01	0\\
256.01	0\\
257.01	0\\
258.01	0\\
259.01	0\\
260.01	0\\
261.01	0\\
262.01	0\\
263.01	0\\
264.01	0\\
265.01	0\\
266.01	0\\
267.01	0\\
268.01	0\\
269.01	0\\
270.01	0\\
271.01	0\\
272.01	0\\
273.01	0\\
274.01	0\\
275.01	0\\
276.01	0\\
277.01	0\\
278.01	0\\
279.01	0\\
280.01	0\\
281.01	0\\
282.01	0\\
283.01	0\\
284.01	0\\
285.01	0\\
286.01	0\\
287.01	0\\
288.01	0\\
289.01	0\\
290.01	0\\
291.01	0\\
292.01	0\\
293.01	0\\
294.01	0\\
295.01	0\\
296.01	0\\
297.01	0\\
298.01	0\\
299.01	0\\
300.01	0\\
301.01	0\\
302.01	0\\
303.01	0\\
304.01	0\\
305.01	0\\
306.01	0\\
307.01	0\\
308.01	0\\
309.01	0\\
310.01	0\\
311.01	0\\
312.01	0\\
313.01	0\\
314.01	0\\
315.01	0\\
316.01	0\\
317.01	0\\
318.01	0\\
319.01	0\\
320.01	0\\
321.01	0\\
322.01	0\\
323.01	0\\
324.01	0\\
325.01	0\\
326.01	0\\
327.01	0\\
328.01	0\\
329.01	0\\
330.01	0\\
331.01	0\\
332.01	0\\
333.01	0\\
334.01	0\\
335.01	0\\
336.01	0\\
337.01	0\\
338.01	0\\
339.01	0\\
340.01	0\\
341.01	0\\
342.01	0\\
343.01	0\\
344.01	0\\
345.01	0\\
346.01	0\\
347.01	0\\
348.01	0\\
349.01	0\\
350.01	0\\
351.01	0\\
352.01	0\\
353.01	0\\
354.01	0\\
355.01	0\\
356.01	0\\
357.01	0\\
358.01	0\\
359.01	0\\
360.01	0\\
361.01	0\\
362.01	0\\
363.01	0\\
364.01	0\\
365.01	0\\
366.01	0\\
367.01	0\\
368.01	0\\
369.01	0\\
370.01	0\\
371.01	0\\
372.01	0\\
373.01	0\\
374.01	0\\
375.01	0\\
376.01	0\\
377.01	0\\
378.01	0\\
379.01	0\\
380.01	0\\
381.01	0\\
382.01	0\\
383.01	0\\
384.01	0\\
385.01	0\\
386.01	0\\
387.01	0\\
388.01	0\\
389.01	0\\
390.01	0\\
391.01	0\\
392.01	0\\
393.01	0\\
394.01	0\\
395.01	0\\
396.01	0\\
397.01	0\\
398.01	0\\
399.01	0\\
400.01	0\\
401.01	0\\
402.01	0\\
403.01	0\\
404.01	0\\
405.01	0\\
406.01	0\\
407.01	0\\
408.01	0\\
409.01	0\\
410.01	0\\
411.01	0\\
412.01	0\\
413.01	0\\
414.01	0\\
415.01	0\\
416.01	0\\
417.01	0\\
418.01	0\\
419.01	0\\
420.01	0\\
421.01	0\\
422.01	0\\
423.01	0\\
424.01	0\\
425.01	0\\
426.01	0\\
427.01	0\\
428.01	0\\
429.01	0\\
430.01	0\\
431.01	0\\
432.01	0\\
433.01	0\\
434.01	0\\
435.01	0\\
436.01	0\\
437.01	0\\
438.01	0\\
439.01	0\\
440.01	0\\
441.01	0\\
442.01	0\\
443.01	0\\
444.01	0\\
445.01	0\\
446.01	0\\
447.01	0\\
448.01	0\\
449.01	0\\
450.01	0\\
451.01	0\\
452.01	0\\
453.01	0\\
454.01	0\\
455.01	0\\
456.01	0\\
457.01	0\\
458.01	0\\
459.01	0\\
460.01	0\\
461.01	0\\
462.01	0\\
463.01	0\\
464.01	0\\
465.01	0\\
466.01	0\\
467.01	0\\
468.01	0\\
469.01	0\\
470.01	0\\
471.01	0\\
472.01	0\\
473.01	0\\
474.01	0\\
475.01	0\\
476.01	0\\
477.01	0\\
478.01	0\\
479.01	0\\
480.01	0\\
481.01	0\\
482.01	0\\
483.01	0\\
484.01	0\\
485.01	0\\
486.01	0\\
487.01	0\\
488.01	0\\
489.01	0\\
490.01	0\\
491.01	0\\
492.01	0\\
493.01	0\\
494.01	0\\
495.01	0\\
496.01	0\\
497.01	0\\
498.01	0\\
499.01	0\\
500.01	0\\
501.01	0\\
502.01	0\\
503.01	0\\
504.01	0\\
505.01	0\\
506.01	0\\
507.01	0\\
508.01	0\\
509.01	0\\
510.01	0\\
511.01	0\\
512.01	0\\
513.01	0\\
514.01	0\\
515.01	0\\
516.01	0\\
517.01	0\\
518.01	0\\
519.01	0\\
520.01	0\\
521.01	0\\
522.01	0\\
523.01	0\\
524.01	0\\
525.01	0\\
526.01	0\\
527.01	0\\
528.01	0\\
529.01	0\\
530.01	0\\
531.01	0\\
532.01	0\\
533.01	0\\
534.01	0\\
535.01	0\\
536.01	0\\
537.01	0\\
538.01	0\\
539.01	0\\
540.01	0\\
541.01	0\\
542.01	0\\
543.01	0\\
544.01	0\\
545.01	0\\
546.01	0\\
547.01	0\\
548.01	0\\
549.01	0\\
550.01	0\\
551.01	0\\
552.01	0\\
553.01	0\\
554.01	0\\
555.01	0\\
556.01	0\\
557.01	0\\
558.01	0\\
559.01	0\\
560.01	0\\
561.01	0\\
562.01	0\\
563.01	0\\
564.01	0\\
565.01	0\\
566.01	0\\
567.01	0\\
568.01	0\\
569.01	0\\
570.01	0\\
571.01	0\\
572.01	0\\
573.01	0\\
574.01	0\\
575.01	0\\
576.01	0\\
577.01	0\\
578.01	0\\
579.01	0\\
580.01	0\\
581.01	0\\
582.01	0\\
583.01	0\\
584.01	0\\
585.01	0\\
586.01	0\\
587.01	0\\
588.01	0\\
589.01	0\\
590.01	0\\
591.01	0\\
592.01	0\\
593.01	0\\
594.01	0\\
595.01	0\\
596.01	0\\
597.01	0\\
598.01	0\\
599.01	0.000112606315978678\\
599.02	0.000319256592928665\\
599.03	0.000527406673816788\\
599.04	0.000737078609145313\\
599.05	0.000948295168053847\\
599.06	0.00116107986778447\\
599.07	0.00137545700451787\\
599.08	0.00159144811590323\\
599.09	0.00180906712192634\\
599.1	0.00202833953770087\\
599.11	0.00224929233225141\\
599.12	0.00247195326892454\\
599.13	0.00269635053012227\\
599.14	0.00292251398260875\\
599.15	0.00315047455348861\\
599.16	0.00338026427669825\\
599.17	0.00361191627693269\\
599.18	0.00384546479123293\\
599.19	0.00408094547119403\\
599.2	0.00431839528698449\\
599.21	0.00448336522131882\\
599.22	0.00453412889007315\\
599.23	0.00458529775335187\\
599.24	0.00463686864801558\\
599.25	0.00468883802852874\\
599.26	0.00474120194778565\\
599.27	0.00479396101985254\\
599.28	0.00484712061004249\\
599.29	0.00490067597179399\\
599.3	0.00495462188970531\\
599.31	0.00500895304553924\\
599.32	0.00506366359833286\\
599.33	0.0051187468066271\\
599.34	0.00517419536070024\\
599.35	0.00523000150170405\\
599.36	0.00528615675861414\\
599.37	0.00534265194840582\\
599.38	0.00539947719780977\\
599.39	0.00545662190877039\\
599.4	0.00551407471367819\\
599.41	0.00557182343363529\\
599.42	0.00562985504934023\\
599.43	0.00568815565847897\\
599.44	0.00574671043077997\\
599.45	0.00580550356057858\\
599.46	0.00586451821672388\\
599.47	0.00592373648964772\\
599.48	0.0059831393354014\\
599.49	0.00604297960426825\\
599.5	0.00610340045448543\\
599.51	0.00616440733117743\\
599.52	0.00622600573181369\\
599.53	0.00628820117461682\\
599.54	0.00635099918988433\\
599.55	0.00641440535722492\\
599.56	0.00647842530500787\\
599.57	0.00654306470995203\\
599.58	0.00660832930056135\\
599.59	0.00667422485699413\\
599.6	0.00674075721317562\\
599.61	0.00680793225800099\\
599.62	0.00687575593656359\\
599.63	0.00694423425145826\\
599.64	0.00701337326416413\\
599.65	0.00708317909651245\\
599.66	0.00715365793224462\\
599.67	0.00722481601874126\\
599.68	0.00729665966881985\\
599.69	0.00736919526261727\\
599.7	0.00744242924958997\\
599.71	0.00751636815063912\\
599.72	0.0075910185607637\\
599.73	0.00766638715239962\\
599.74	0.00774248067826639\\
599.75	0.00781930597433837\\
599.76	0.00789686996329332\\
599.77	0.0079751796583929\\
599.78	0.00805424216722275\\
599.79	0.00813406469569006\\
599.8	0.00821465455229681\\
599.81	0.00829601915270807\\
599.82	0.0083781660246363\\
599.83	0.008461102813064\\
599.84	0.00854483728582868\\
599.85	0.00862937733959608\\
599.86	0.00871473100624913\\
599.87	0.00880090645972257\\
599.88	0.00888791202331493\\
599.89	0.00897575617751244\\
599.9	0.00906444756836147\\
599.91	0.00915399501642923\\
599.92	0.00924440752639532\\
599.93	0.00933569429731976\\
599.94	0.00942786473363691\\
599.95	0.00952092845692796\\
599.96	0.00961489531852919\\
599.97	0.00970977541303696\\
599.98	0.00980557909277551\\
599.99	0.00990231698329844\\
600	0.01\\
};
\addplot [color=mycolor16,solid,forget plot]
  table[row sep=crcr]{%
0.01	0\\
1.01	0\\
2.01	0\\
3.01	0\\
4.01	0\\
5.01	0\\
6.01	0\\
7.01	0\\
8.01	0\\
9.01	0\\
10.01	0\\
11.01	0\\
12.01	0\\
13.01	0\\
14.01	0\\
15.01	0\\
16.01	0\\
17.01	0\\
18.01	0\\
19.01	0\\
20.01	0\\
21.01	0\\
22.01	0\\
23.01	0\\
24.01	0\\
25.01	0\\
26.01	0\\
27.01	0\\
28.01	0\\
29.01	0\\
30.01	0\\
31.01	0\\
32.01	0\\
33.01	0\\
34.01	0\\
35.01	0\\
36.01	0\\
37.01	0\\
38.01	0\\
39.01	0\\
40.01	0\\
41.01	0\\
42.01	0\\
43.01	0\\
44.01	0\\
45.01	0\\
46.01	0\\
47.01	0\\
48.01	0\\
49.01	0\\
50.01	0\\
51.01	0\\
52.01	0\\
53.01	0\\
54.01	0\\
55.01	0\\
56.01	0\\
57.01	0\\
58.01	0\\
59.01	0\\
60.01	0\\
61.01	0\\
62.01	0\\
63.01	0\\
64.01	0\\
65.01	0\\
66.01	0\\
67.01	0\\
68.01	0\\
69.01	0\\
70.01	0\\
71.01	0\\
72.01	0\\
73.01	0\\
74.01	0\\
75.01	0\\
76.01	0\\
77.01	0\\
78.01	0\\
79.01	0\\
80.01	0\\
81.01	0\\
82.01	0\\
83.01	0\\
84.01	0\\
85.01	0\\
86.01	0\\
87.01	0\\
88.01	0\\
89.01	0\\
90.01	0\\
91.01	0\\
92.01	0\\
93.01	0\\
94.01	0\\
95.01	0\\
96.01	0\\
97.01	0\\
98.01	0\\
99.01	0\\
100.01	0\\
101.01	0\\
102.01	0\\
103.01	0\\
104.01	0\\
105.01	0\\
106.01	0\\
107.01	0\\
108.01	0\\
109.01	0\\
110.01	0\\
111.01	0\\
112.01	0\\
113.01	0\\
114.01	0\\
115.01	0\\
116.01	0\\
117.01	0\\
118.01	0\\
119.01	0\\
120.01	0\\
121.01	0\\
122.01	0\\
123.01	0\\
124.01	0\\
125.01	0\\
126.01	0\\
127.01	0\\
128.01	0\\
129.01	0\\
130.01	0\\
131.01	0\\
132.01	0\\
133.01	0\\
134.01	0\\
135.01	0\\
136.01	0\\
137.01	0\\
138.01	0\\
139.01	0\\
140.01	0\\
141.01	0\\
142.01	0\\
143.01	0\\
144.01	0\\
145.01	0\\
146.01	0\\
147.01	0\\
148.01	0\\
149.01	0\\
150.01	0\\
151.01	0\\
152.01	0\\
153.01	0\\
154.01	0\\
155.01	0\\
156.01	0\\
157.01	0\\
158.01	0\\
159.01	0\\
160.01	0\\
161.01	0\\
162.01	0\\
163.01	0\\
164.01	0\\
165.01	0\\
166.01	0\\
167.01	0\\
168.01	0\\
169.01	0\\
170.01	0\\
171.01	0\\
172.01	0\\
173.01	0\\
174.01	0\\
175.01	0\\
176.01	0\\
177.01	0\\
178.01	0\\
179.01	0\\
180.01	0\\
181.01	0\\
182.01	0\\
183.01	0\\
184.01	0\\
185.01	0\\
186.01	0\\
187.01	0\\
188.01	0\\
189.01	0\\
190.01	0\\
191.01	0\\
192.01	0\\
193.01	0\\
194.01	0\\
195.01	0\\
196.01	0\\
197.01	0\\
198.01	0\\
199.01	0\\
200.01	0\\
201.01	0\\
202.01	0\\
203.01	0\\
204.01	0\\
205.01	0\\
206.01	0\\
207.01	0\\
208.01	0\\
209.01	0\\
210.01	0\\
211.01	0\\
212.01	0\\
213.01	0\\
214.01	0\\
215.01	0\\
216.01	0\\
217.01	0\\
218.01	0\\
219.01	0\\
220.01	0\\
221.01	0\\
222.01	0\\
223.01	0\\
224.01	0\\
225.01	0\\
226.01	0\\
227.01	0\\
228.01	0\\
229.01	0\\
230.01	0\\
231.01	0\\
232.01	0\\
233.01	0\\
234.01	0\\
235.01	0\\
236.01	0\\
237.01	0\\
238.01	0\\
239.01	0\\
240.01	0\\
241.01	0\\
242.01	0\\
243.01	0\\
244.01	0\\
245.01	0\\
246.01	0\\
247.01	0\\
248.01	0\\
249.01	0\\
250.01	0\\
251.01	0\\
252.01	0\\
253.01	0\\
254.01	0\\
255.01	0\\
256.01	0\\
257.01	0\\
258.01	0\\
259.01	0\\
260.01	0\\
261.01	0\\
262.01	0\\
263.01	0\\
264.01	0\\
265.01	0\\
266.01	0\\
267.01	0\\
268.01	0\\
269.01	0\\
270.01	0\\
271.01	0\\
272.01	0\\
273.01	0\\
274.01	0\\
275.01	0\\
276.01	0\\
277.01	0\\
278.01	0\\
279.01	0\\
280.01	0\\
281.01	0\\
282.01	0\\
283.01	0\\
284.01	0\\
285.01	0\\
286.01	0\\
287.01	0\\
288.01	0\\
289.01	0\\
290.01	0\\
291.01	0\\
292.01	0\\
293.01	0\\
294.01	0\\
295.01	0\\
296.01	0\\
297.01	0\\
298.01	0\\
299.01	0\\
300.01	0\\
301.01	0\\
302.01	0\\
303.01	0\\
304.01	0\\
305.01	0\\
306.01	0\\
307.01	0\\
308.01	0\\
309.01	0\\
310.01	0\\
311.01	0\\
312.01	0\\
313.01	0\\
314.01	0\\
315.01	0\\
316.01	0\\
317.01	0\\
318.01	0\\
319.01	0\\
320.01	0\\
321.01	0\\
322.01	0\\
323.01	0\\
324.01	0\\
325.01	0\\
326.01	0\\
327.01	0\\
328.01	0\\
329.01	0\\
330.01	0\\
331.01	0\\
332.01	0\\
333.01	0\\
334.01	0\\
335.01	0\\
336.01	0\\
337.01	0\\
338.01	0\\
339.01	0\\
340.01	0\\
341.01	0\\
342.01	0\\
343.01	0\\
344.01	0\\
345.01	0\\
346.01	0\\
347.01	0\\
348.01	0\\
349.01	0\\
350.01	0\\
351.01	0\\
352.01	0\\
353.01	0\\
354.01	0\\
355.01	0\\
356.01	0\\
357.01	0\\
358.01	0\\
359.01	0\\
360.01	0\\
361.01	0\\
362.01	0\\
363.01	0\\
364.01	0\\
365.01	0\\
366.01	0\\
367.01	0\\
368.01	0\\
369.01	0\\
370.01	0\\
371.01	0\\
372.01	0\\
373.01	0\\
374.01	0\\
375.01	0\\
376.01	0\\
377.01	0\\
378.01	0\\
379.01	0\\
380.01	0\\
381.01	0\\
382.01	0\\
383.01	0\\
384.01	0\\
385.01	0\\
386.01	0\\
387.01	0\\
388.01	0\\
389.01	0\\
390.01	0\\
391.01	0\\
392.01	0\\
393.01	0\\
394.01	0\\
395.01	0\\
396.01	0\\
397.01	0\\
398.01	0\\
399.01	0\\
400.01	0\\
401.01	0\\
402.01	0\\
403.01	0\\
404.01	0\\
405.01	0\\
406.01	0\\
407.01	0\\
408.01	0\\
409.01	0\\
410.01	0\\
411.01	0\\
412.01	0\\
413.01	0\\
414.01	0\\
415.01	0\\
416.01	0\\
417.01	0\\
418.01	0\\
419.01	0\\
420.01	0\\
421.01	0\\
422.01	0\\
423.01	0\\
424.01	0\\
425.01	0\\
426.01	0\\
427.01	0\\
428.01	0\\
429.01	0\\
430.01	0\\
431.01	0\\
432.01	0\\
433.01	0\\
434.01	0\\
435.01	0\\
436.01	0\\
437.01	0\\
438.01	0\\
439.01	0\\
440.01	0\\
441.01	0\\
442.01	0\\
443.01	0\\
444.01	0\\
445.01	0\\
446.01	0\\
447.01	0\\
448.01	0\\
449.01	0\\
450.01	0\\
451.01	0\\
452.01	0\\
453.01	0\\
454.01	0\\
455.01	0\\
456.01	0\\
457.01	0\\
458.01	0\\
459.01	0\\
460.01	0\\
461.01	0\\
462.01	0\\
463.01	0\\
464.01	0\\
465.01	0\\
466.01	0\\
467.01	0\\
468.01	0\\
469.01	0\\
470.01	0\\
471.01	0\\
472.01	0\\
473.01	0\\
474.01	0\\
475.01	0\\
476.01	0\\
477.01	0\\
478.01	0\\
479.01	0\\
480.01	0\\
481.01	0\\
482.01	0\\
483.01	0\\
484.01	0\\
485.01	0\\
486.01	0\\
487.01	0\\
488.01	0\\
489.01	0\\
490.01	0\\
491.01	0\\
492.01	0\\
493.01	0\\
494.01	0\\
495.01	0\\
496.01	0\\
497.01	0\\
498.01	0\\
499.01	0\\
500.01	0\\
501.01	0\\
502.01	0\\
503.01	0\\
504.01	0\\
505.01	0\\
506.01	0\\
507.01	0\\
508.01	0\\
509.01	0\\
510.01	0\\
511.01	0\\
512.01	0\\
513.01	0\\
514.01	0\\
515.01	0\\
516.01	0\\
517.01	0\\
518.01	0\\
519.01	0\\
520.01	0\\
521.01	0\\
522.01	0\\
523.01	0\\
524.01	0\\
525.01	0\\
526.01	0\\
527.01	0\\
528.01	0\\
529.01	0\\
530.01	0\\
531.01	0\\
532.01	0\\
533.01	0\\
534.01	0\\
535.01	0\\
536.01	0\\
537.01	0\\
538.01	0\\
539.01	0\\
540.01	0\\
541.01	0\\
542.01	0\\
543.01	0\\
544.01	0\\
545.01	0\\
546.01	0\\
547.01	0\\
548.01	0\\
549.01	0\\
550.01	0\\
551.01	0\\
552.01	0\\
553.01	0\\
554.01	0\\
555.01	0\\
556.01	0\\
557.01	0\\
558.01	0\\
559.01	0\\
560.01	0\\
561.01	0\\
562.01	0\\
563.01	0\\
564.01	0\\
565.01	0\\
566.01	0\\
567.01	0\\
568.01	0\\
569.01	0\\
570.01	0\\
571.01	0\\
572.01	0\\
573.01	0\\
574.01	0\\
575.01	0\\
576.01	0\\
577.01	0\\
578.01	0\\
579.01	0\\
580.01	0\\
581.01	0\\
582.01	0\\
583.01	0\\
584.01	0\\
585.01	0\\
586.01	0\\
587.01	0\\
588.01	0\\
589.01	0\\
590.01	0\\
591.01	0\\
592.01	0\\
593.01	0\\
594.01	0\\
595.01	0\\
596.01	0\\
597.01	0\\
598.01	0\\
599.01	0.00370208209411713\\
599.02	0.00374423082044583\\
599.03	0.00378664015556109\\
599.04	0.00382930214308572\\
599.05	0.00387220842322091\\
599.06	0.00391535001457705\\
599.07	0.00395871728279321\\
599.08	0.0040023034946127\\
599.09	0.00404610972041236\\
599.1	0.00409012556930278\\
599.11	0.00413433908054605\\
599.12	0.00417873761917283\\
599.13	0.00422330825718109\\
599.14	0.00426803650362734\\
599.15	0.00431290692702228\\
599.16	0.00435790310699198\\
599.17	0.00440300764863001\\
599.18	0.00444820215913953\\
599.19	0.00449346694222096\\
599.2	0.00453878109118118\\
599.21	0.00458428620748516\\
599.22	0.00463022774404547\\
599.23	0.00467660989602732\\
599.24	0.00472343691632505\\
599.25	0.00477071311787237\\
599.26	0.00481844287609376\\
599.27	0.00486663060728154\\
599.28	0.00491528074441994\\
599.29	0.00496439778463096\\
599.3	0.00501398629186756\\
599.31	0.00506405089766463\\
599.32	0.00511459630406084\\
599.33	0.00516562728865405\\
599.34	0.00521714870799032\\
599.35	0.00526916550026689\\
599.36	0.00532168268960642\\
599.37	0.00537470539040875\\
599.38	0.00542823881163046\\
599.39	0.00548228825765701\\
599.4	0.00553685912202147\\
599.41	0.00559195690690256\\
599.42	0.00564758722852464\\
599.43	0.00570375582288129\\
599.44	0.00576046855180099\\
599.45	0.00581773140937498\\
599.46	0.00587555052876837\\
599.47	0.00593393218943659\\
599.48	0.00599288282477091\\
599.49	0.00605240833238633\\
599.5	0.00611251432580253\\
599.51	0.00617320647360776\\
599.52	0.00623449050000827\\
599.53	0.00629637218551971\\
599.54	0.00635885736771363\\
599.55	0.00642195194182304\\
599.56	0.00648566186136196\\
599.57	0.00654999313875884\\
599.58	0.00661495184598612\\
599.59	0.00668054411520325\\
599.6	0.00674677613940176\\
599.61	0.00681365417305723\\
599.62	0.00688118453278823\\
599.63	0.00694937359802226\\
599.64	0.00701822781166827\\
599.65	0.00708775368079585\\
599.66	0.00715795777732064\\
599.67	0.00722884673869562\\
599.68	0.007300427268608\\
599.69	0.0073727061376814\\
599.7	0.00744569018418299\\
599.71	0.00751938631473504\\
599.72	0.00759380150502925\\
599.73	0.00766894280054146\\
599.74	0.00774481731724789\\
599.75	0.00782143224234223\\
599.76	0.00789879483495157\\
599.77	0.00797691242684947\\
599.78	0.00805579242316594\\
599.79	0.00813544230309276\\
599.8	0.00821586962058269\\
599.81	0.00829708200504056\\
599.82	0.0083790871620044\\
599.83	0.00846189287381431\\
599.84	0.00854550700026665\\
599.85	0.00862993747925072\\
599.86	0.00871519232736509\\
599.87	0.00880127964051009\\
599.88	0.00888820759445287\\
599.89	0.00897598444536097\\
599.9	0.00906461853029984\\
599.91	0.00915411826768946\\
599.92	0.00924449215771456\\
599.93	0.00933574878268246\\
599.94	0.00942789680732179\\
599.95	0.00952094497901499\\
599.96	0.00961490212795633\\
599.97	0.00970977716722672\\
599.98	0.00980557909277551\\
599.99	0.00990231698329844\\
600	0.01\\
};
\addplot [color=mycolor17,solid,forget plot]
  table[row sep=crcr]{%
0.01	0\\
1.01	0\\
2.01	0\\
3.01	0\\
4.01	0\\
5.01	0\\
6.01	0\\
7.01	0\\
8.01	0\\
9.01	0\\
10.01	0\\
11.01	0\\
12.01	0\\
13.01	0\\
14.01	0\\
15.01	0\\
16.01	0\\
17.01	0\\
18.01	0\\
19.01	0\\
20.01	0\\
21.01	0\\
22.01	0\\
23.01	0\\
24.01	0\\
25.01	0\\
26.01	0\\
27.01	0\\
28.01	0\\
29.01	0\\
30.01	0\\
31.01	0\\
32.01	0\\
33.01	0\\
34.01	0\\
35.01	0\\
36.01	0\\
37.01	0\\
38.01	0\\
39.01	0\\
40.01	0\\
41.01	0\\
42.01	0\\
43.01	0\\
44.01	0\\
45.01	0\\
46.01	0\\
47.01	0\\
48.01	0\\
49.01	0\\
50.01	0\\
51.01	0\\
52.01	0\\
53.01	0\\
54.01	0\\
55.01	0\\
56.01	0\\
57.01	0\\
58.01	0\\
59.01	0\\
60.01	0\\
61.01	0\\
62.01	0\\
63.01	0\\
64.01	0\\
65.01	0\\
66.01	0\\
67.01	0\\
68.01	0\\
69.01	0\\
70.01	0\\
71.01	0\\
72.01	0\\
73.01	0\\
74.01	0\\
75.01	0\\
76.01	0\\
77.01	0\\
78.01	0\\
79.01	0\\
80.01	0\\
81.01	0\\
82.01	0\\
83.01	0\\
84.01	0\\
85.01	0\\
86.01	0\\
87.01	0\\
88.01	0\\
89.01	0\\
90.01	0\\
91.01	0\\
92.01	0\\
93.01	0\\
94.01	0\\
95.01	0\\
96.01	0\\
97.01	0\\
98.01	0\\
99.01	0\\
100.01	0\\
101.01	0\\
102.01	0\\
103.01	0\\
104.01	0\\
105.01	0\\
106.01	0\\
107.01	0\\
108.01	0\\
109.01	0\\
110.01	0\\
111.01	0\\
112.01	0\\
113.01	0\\
114.01	0\\
115.01	0\\
116.01	0\\
117.01	0\\
118.01	0\\
119.01	0\\
120.01	0\\
121.01	0\\
122.01	0\\
123.01	0\\
124.01	0\\
125.01	0\\
126.01	0\\
127.01	0\\
128.01	0\\
129.01	0\\
130.01	0\\
131.01	0\\
132.01	0\\
133.01	0\\
134.01	0\\
135.01	0\\
136.01	0\\
137.01	0\\
138.01	0\\
139.01	0\\
140.01	0\\
141.01	0\\
142.01	0\\
143.01	0\\
144.01	0\\
145.01	0\\
146.01	0\\
147.01	0\\
148.01	0\\
149.01	0\\
150.01	0\\
151.01	0\\
152.01	0\\
153.01	0\\
154.01	0\\
155.01	0\\
156.01	0\\
157.01	0\\
158.01	0\\
159.01	0\\
160.01	0\\
161.01	0\\
162.01	0\\
163.01	0\\
164.01	0\\
165.01	0\\
166.01	0\\
167.01	0\\
168.01	0\\
169.01	0\\
170.01	0\\
171.01	0\\
172.01	0\\
173.01	0\\
174.01	0\\
175.01	0\\
176.01	0\\
177.01	0\\
178.01	0\\
179.01	0\\
180.01	0\\
181.01	0\\
182.01	0\\
183.01	0\\
184.01	0\\
185.01	0\\
186.01	0\\
187.01	0\\
188.01	0\\
189.01	0\\
190.01	0\\
191.01	0\\
192.01	0\\
193.01	0\\
194.01	0\\
195.01	0\\
196.01	0\\
197.01	0\\
198.01	0\\
199.01	0\\
200.01	0\\
201.01	0\\
202.01	0\\
203.01	0\\
204.01	0\\
205.01	0\\
206.01	0\\
207.01	0\\
208.01	0\\
209.01	0\\
210.01	0\\
211.01	0\\
212.01	0\\
213.01	0\\
214.01	0\\
215.01	0\\
216.01	0\\
217.01	0\\
218.01	0\\
219.01	0\\
220.01	0\\
221.01	0\\
222.01	0\\
223.01	0\\
224.01	0\\
225.01	0\\
226.01	0\\
227.01	0\\
228.01	0\\
229.01	0\\
230.01	0\\
231.01	0\\
232.01	0\\
233.01	0\\
234.01	0\\
235.01	0\\
236.01	0\\
237.01	0\\
238.01	0\\
239.01	0\\
240.01	0\\
241.01	0\\
242.01	0\\
243.01	0\\
244.01	0\\
245.01	0\\
246.01	0\\
247.01	0\\
248.01	0\\
249.01	0\\
250.01	0\\
251.01	0\\
252.01	0\\
253.01	0\\
254.01	0\\
255.01	0\\
256.01	0\\
257.01	0\\
258.01	0\\
259.01	0\\
260.01	0\\
261.01	0\\
262.01	0\\
263.01	0\\
264.01	0\\
265.01	0\\
266.01	0\\
267.01	0\\
268.01	0\\
269.01	0\\
270.01	0\\
271.01	0\\
272.01	0\\
273.01	0\\
274.01	0\\
275.01	0\\
276.01	0\\
277.01	0\\
278.01	0\\
279.01	0\\
280.01	0\\
281.01	0\\
282.01	0\\
283.01	0\\
284.01	0\\
285.01	0\\
286.01	0\\
287.01	0\\
288.01	0\\
289.01	0\\
290.01	0\\
291.01	0\\
292.01	0\\
293.01	0\\
294.01	0\\
295.01	0\\
296.01	0\\
297.01	0\\
298.01	0\\
299.01	0\\
300.01	0\\
301.01	0\\
302.01	0\\
303.01	0\\
304.01	0\\
305.01	0\\
306.01	0\\
307.01	0\\
308.01	0\\
309.01	0\\
310.01	0\\
311.01	0\\
312.01	0\\
313.01	0\\
314.01	0\\
315.01	0\\
316.01	0\\
317.01	0\\
318.01	0\\
319.01	0\\
320.01	0\\
321.01	0\\
322.01	0\\
323.01	0\\
324.01	0\\
325.01	0\\
326.01	0\\
327.01	0\\
328.01	0\\
329.01	0\\
330.01	0\\
331.01	0\\
332.01	0\\
333.01	0\\
334.01	0\\
335.01	0\\
336.01	0\\
337.01	0\\
338.01	0\\
339.01	0\\
340.01	0\\
341.01	0\\
342.01	0\\
343.01	0\\
344.01	0\\
345.01	0\\
346.01	0\\
347.01	0\\
348.01	0\\
349.01	0\\
350.01	0\\
351.01	0\\
352.01	0\\
353.01	0\\
354.01	0\\
355.01	0\\
356.01	0\\
357.01	0\\
358.01	0\\
359.01	0\\
360.01	0\\
361.01	0\\
362.01	0\\
363.01	0\\
364.01	0\\
365.01	0\\
366.01	0\\
367.01	0\\
368.01	0\\
369.01	0\\
370.01	0\\
371.01	0\\
372.01	0\\
373.01	0\\
374.01	0\\
375.01	0\\
376.01	0\\
377.01	0\\
378.01	0\\
379.01	0\\
380.01	0\\
381.01	0\\
382.01	0\\
383.01	0\\
384.01	0\\
385.01	0\\
386.01	0\\
387.01	0\\
388.01	0\\
389.01	0\\
390.01	0\\
391.01	0\\
392.01	0\\
393.01	0\\
394.01	0\\
395.01	0\\
396.01	0\\
397.01	0\\
398.01	0\\
399.01	0\\
400.01	0\\
401.01	0\\
402.01	0\\
403.01	0\\
404.01	0\\
405.01	0\\
406.01	0\\
407.01	0\\
408.01	0\\
409.01	0\\
410.01	0\\
411.01	0\\
412.01	0\\
413.01	0\\
414.01	0\\
415.01	0\\
416.01	0\\
417.01	0\\
418.01	0\\
419.01	0\\
420.01	0\\
421.01	0\\
422.01	0\\
423.01	0\\
424.01	0\\
425.01	0\\
426.01	0\\
427.01	0\\
428.01	0\\
429.01	0\\
430.01	0\\
431.01	0\\
432.01	0\\
433.01	0\\
434.01	0\\
435.01	0\\
436.01	0\\
437.01	0\\
438.01	0\\
439.01	0\\
440.01	0\\
441.01	0\\
442.01	0\\
443.01	0\\
444.01	0\\
445.01	0\\
446.01	0\\
447.01	0\\
448.01	0\\
449.01	0\\
450.01	0\\
451.01	0\\
452.01	0\\
453.01	0\\
454.01	0\\
455.01	0\\
456.01	0\\
457.01	0\\
458.01	0\\
459.01	0\\
460.01	0\\
461.01	0\\
462.01	0\\
463.01	0\\
464.01	0\\
465.01	0\\
466.01	0\\
467.01	0\\
468.01	0\\
469.01	0\\
470.01	0\\
471.01	0\\
472.01	0\\
473.01	0\\
474.01	0\\
475.01	0\\
476.01	0\\
477.01	0\\
478.01	0\\
479.01	0\\
480.01	0\\
481.01	0\\
482.01	0\\
483.01	0\\
484.01	0\\
485.01	0\\
486.01	0\\
487.01	0\\
488.01	0\\
489.01	0\\
490.01	0\\
491.01	0\\
492.01	0\\
493.01	0\\
494.01	0\\
495.01	0\\
496.01	0\\
497.01	0\\
498.01	0\\
499.01	0\\
500.01	0\\
501.01	0\\
502.01	0\\
503.01	0\\
504.01	0\\
505.01	0\\
506.01	0\\
507.01	0\\
508.01	0\\
509.01	0\\
510.01	0\\
511.01	0\\
512.01	0\\
513.01	0\\
514.01	0\\
515.01	0\\
516.01	0\\
517.01	0\\
518.01	0\\
519.01	0\\
520.01	0\\
521.01	0\\
522.01	0\\
523.01	0\\
524.01	0\\
525.01	0\\
526.01	0\\
527.01	0\\
528.01	0\\
529.01	0\\
530.01	0\\
531.01	0\\
532.01	0\\
533.01	0\\
534.01	0\\
535.01	0\\
536.01	0\\
537.01	0\\
538.01	0\\
539.01	0\\
540.01	0\\
541.01	0\\
542.01	0\\
543.01	0\\
544.01	0\\
545.01	0\\
546.01	0\\
547.01	0\\
548.01	0\\
549.01	0\\
550.01	0\\
551.01	0\\
552.01	0\\
553.01	0\\
554.01	0\\
555.01	0\\
556.01	0\\
557.01	0\\
558.01	0\\
559.01	0\\
560.01	0\\
561.01	0\\
562.01	0\\
563.01	0\\
564.01	0\\
565.01	0\\
566.01	0\\
567.01	0\\
568.01	0\\
569.01	0\\
570.01	0\\
571.01	0\\
572.01	0\\
573.01	0\\
574.01	0\\
575.01	0\\
576.01	0\\
577.01	0\\
578.01	0\\
579.01	0\\
580.01	0\\
581.01	0\\
582.01	0\\
583.01	0\\
584.01	0\\
585.01	0\\
586.01	0\\
587.01	0\\
588.01	0\\
589.01	0\\
590.01	0\\
591.01	0\\
592.01	0\\
593.01	0\\
594.01	0\\
595.01	0\\
596.01	0\\
597.01	0\\
598.01	0\\
599.01	0.0037573870629064\\
599.02	0.00379519556360755\\
599.03	0.00383336480333374\\
599.04	0.00387189858986088\\
599.05	0.00391080060823229\\
599.06	0.00395007461228412\\
599.07	0.00398972442806679\\
599.08	0.00402975394099463\\
599.09	0.00407016706079863\\
599.1	0.00411096777072455\\
599.11	0.00415216013486262\\
599.12	0.00419374830159151\\
599.13	0.00423573650508695\\
599.14	0.00427812907311622\\
599.15	0.00432093043192307\\
599.16	0.00436414511140291\\
599.17	0.00440777775022103\\
599.18	0.00445183310105154\\
599.19	0.00449631603752963\\
599.2	0.00454123156074996\\
599.21	0.00458658444372716\\
599.22	0.00463237898770875\\
599.23	0.00467861953666444\\
599.24	0.00472531047768157\\
599.25	0.00477245624135829\\
599.26	0.00482006130219408\\
599.27	0.00486813017906603\\
599.28	0.00491666743580523\\
599.29	0.00496567768161336\\
599.3	0.00501516557147688\\
599.31	0.00506513580658655\\
599.32	0.00511559313475406\\
599.33	0.00516654235081657\\
599.34	0.00521798829703587\\
599.35	0.00526993586349498\\
599.36	0.00532238998848544\\
599.37	0.00537535565888448\\
599.38	0.00542883791052227\\
599.39	0.00548284182457115\\
599.4	0.00553737251745407\\
599.41	0.00559243515521324\\
599.42	0.00564803495381788\\
599.43	0.00570417717945396\\
599.44	0.0057608671487941\\
599.45	0.00581811022924536\\
599.46	0.00587591183917258\\
599.47	0.00593427744809491\\
599.48	0.0059932125768525\\
599.49	0.00605272279952553\\
599.5	0.00611281374479376\\
599.51	0.00617349109647296\\
599.52	0.00623476059405683\\
599.53	0.00629662803326369\\
599.54	0.006359099266588\\
599.55	0.00642218020385709\\
599.56	0.00648587681279326\\
599.57	0.00655019511958122\\
599.58	0.00661514120944084\\
599.59	0.00668072122720545\\
599.6	0.00674694137790552\\
599.61	0.00681380792735788\\
599.62	0.00688132720276059\\
599.63	0.00694950559329325\\
599.64	0.00701834955072308\\
599.65	0.00708786559001664\\
599.66	0.00715806028995723\\
599.67	0.00722894029376811\\
599.68	0.00730051230974149\\
599.69	0.00737278311187335\\
599.7	0.00744575954050419\\
599.71	0.00751944850296568\\
599.72	0.00759385697423329\\
599.73	0.007668991997585\\
599.74	0.00774486068526607\\
599.75	0.00782147021915993\\
599.76	0.0078988278514654\\
599.77	0.0079769409053801\\
599.78	0.00805581677579022\\
599.79	0.00813546292996683\\
599.8	0.00821588690826864\\
599.81	0.00829709632485142\\
599.82	0.00837909886838419\\
599.83	0.00846190230277219\\
599.84	0.008545514467887\\
599.85	0.00862994328030364\\
599.86	0.00871519673404507\\
599.87	0.00880128290133424\\
599.88	0.00888820993335373\\
599.89	0.00897598606101348\\
599.9	0.00906461959572664\\
599.91	0.009154118930194\\
599.92	0.00924449253919727\\
599.93	0.00933574898040158\\
599.94	0.00942789689516777\\
599.95	0.00952094500937471\\
599.96	0.00961490213425244\\
599.97	0.00970977716722672\\
599.98	0.00980557909277551\\
599.99	0.00990231698329844\\
600	0.01\\
};
\addplot [color=mycolor18,solid,forget plot]
  table[row sep=crcr]{%
0.01	0\\
1.01	0\\
2.01	0\\
3.01	0\\
4.01	0\\
5.01	0\\
6.01	0\\
7.01	0\\
8.01	0\\
9.01	0\\
10.01	0\\
11.01	0\\
12.01	0\\
13.01	0\\
14.01	0\\
15.01	0\\
16.01	0\\
17.01	0\\
18.01	0\\
19.01	0\\
20.01	0\\
21.01	0\\
22.01	0\\
23.01	0\\
24.01	0\\
25.01	0\\
26.01	0\\
27.01	0\\
28.01	0\\
29.01	0\\
30.01	0\\
31.01	0\\
32.01	0\\
33.01	0\\
34.01	0\\
35.01	0\\
36.01	0\\
37.01	0\\
38.01	0\\
39.01	0\\
40.01	0\\
41.01	0\\
42.01	0\\
43.01	0\\
44.01	0\\
45.01	0\\
46.01	0\\
47.01	0\\
48.01	0\\
49.01	0\\
50.01	0\\
51.01	0\\
52.01	0\\
53.01	0\\
54.01	0\\
55.01	0\\
56.01	0\\
57.01	0\\
58.01	0\\
59.01	0\\
60.01	0\\
61.01	0\\
62.01	0\\
63.01	0\\
64.01	0\\
65.01	0\\
66.01	0\\
67.01	0\\
68.01	0\\
69.01	0\\
70.01	0\\
71.01	0\\
72.01	0\\
73.01	0\\
74.01	0\\
75.01	0\\
76.01	0\\
77.01	0\\
78.01	0\\
79.01	0\\
80.01	0\\
81.01	0\\
82.01	0\\
83.01	0\\
84.01	0\\
85.01	0\\
86.01	0\\
87.01	0\\
88.01	0\\
89.01	0\\
90.01	0\\
91.01	0\\
92.01	0\\
93.01	0\\
94.01	0\\
95.01	0\\
96.01	0\\
97.01	0\\
98.01	0\\
99.01	0\\
100.01	0\\
101.01	0\\
102.01	0\\
103.01	0\\
104.01	0\\
105.01	0\\
106.01	0\\
107.01	0\\
108.01	0\\
109.01	0\\
110.01	0\\
111.01	0\\
112.01	0\\
113.01	0\\
114.01	0\\
115.01	0\\
116.01	0\\
117.01	0\\
118.01	0\\
119.01	0\\
120.01	0\\
121.01	0\\
122.01	0\\
123.01	0\\
124.01	0\\
125.01	0\\
126.01	0\\
127.01	0\\
128.01	0\\
129.01	0\\
130.01	0\\
131.01	0\\
132.01	0\\
133.01	0\\
134.01	0\\
135.01	0\\
136.01	0\\
137.01	0\\
138.01	0\\
139.01	0\\
140.01	0\\
141.01	0\\
142.01	0\\
143.01	0\\
144.01	0\\
145.01	0\\
146.01	0\\
147.01	0\\
148.01	0\\
149.01	0\\
150.01	0\\
151.01	0\\
152.01	0\\
153.01	0\\
154.01	0\\
155.01	0\\
156.01	0\\
157.01	0\\
158.01	0\\
159.01	0\\
160.01	0\\
161.01	0\\
162.01	0\\
163.01	0\\
164.01	0\\
165.01	0\\
166.01	0\\
167.01	0\\
168.01	0\\
169.01	0\\
170.01	0\\
171.01	0\\
172.01	0\\
173.01	0\\
174.01	0\\
175.01	0\\
176.01	0\\
177.01	0\\
178.01	0\\
179.01	0\\
180.01	0\\
181.01	0\\
182.01	0\\
183.01	0\\
184.01	0\\
185.01	0\\
186.01	0\\
187.01	0\\
188.01	0\\
189.01	0\\
190.01	0\\
191.01	0\\
192.01	0\\
193.01	0\\
194.01	0\\
195.01	0\\
196.01	0\\
197.01	0\\
198.01	0\\
199.01	0\\
200.01	0\\
201.01	0\\
202.01	0\\
203.01	0\\
204.01	0\\
205.01	0\\
206.01	0\\
207.01	0\\
208.01	0\\
209.01	0\\
210.01	0\\
211.01	0\\
212.01	0\\
213.01	0\\
214.01	0\\
215.01	0\\
216.01	0\\
217.01	0\\
218.01	0\\
219.01	0\\
220.01	0\\
221.01	0\\
222.01	0\\
223.01	0\\
224.01	0\\
225.01	0\\
226.01	0\\
227.01	0\\
228.01	0\\
229.01	0\\
230.01	0\\
231.01	0\\
232.01	0\\
233.01	0\\
234.01	0\\
235.01	0\\
236.01	0\\
237.01	0\\
238.01	0\\
239.01	0\\
240.01	0\\
241.01	0\\
242.01	0\\
243.01	0\\
244.01	0\\
245.01	0\\
246.01	0\\
247.01	0\\
248.01	0\\
249.01	0\\
250.01	0\\
251.01	0\\
252.01	0\\
253.01	0\\
254.01	0\\
255.01	0\\
256.01	0\\
257.01	0\\
258.01	0\\
259.01	0\\
260.01	0\\
261.01	0\\
262.01	0\\
263.01	0\\
264.01	0\\
265.01	0\\
266.01	0\\
267.01	0\\
268.01	0\\
269.01	0\\
270.01	0\\
271.01	0\\
272.01	0\\
273.01	0\\
274.01	0\\
275.01	0\\
276.01	0\\
277.01	0\\
278.01	0\\
279.01	0\\
280.01	0\\
281.01	0\\
282.01	0\\
283.01	0\\
284.01	0\\
285.01	0\\
286.01	0\\
287.01	0\\
288.01	0\\
289.01	0\\
290.01	0\\
291.01	0\\
292.01	0\\
293.01	0\\
294.01	0\\
295.01	0\\
296.01	0\\
297.01	0\\
298.01	0\\
299.01	0\\
300.01	0\\
301.01	0\\
302.01	0\\
303.01	0\\
304.01	0\\
305.01	0\\
306.01	0\\
307.01	0\\
308.01	0\\
309.01	0\\
310.01	0\\
311.01	0\\
312.01	0\\
313.01	0\\
314.01	0\\
315.01	0\\
316.01	0\\
317.01	0\\
318.01	0\\
319.01	0\\
320.01	0\\
321.01	0\\
322.01	0\\
323.01	0\\
324.01	0\\
325.01	0\\
326.01	0\\
327.01	0\\
328.01	0\\
329.01	0\\
330.01	0\\
331.01	0\\
332.01	0\\
333.01	0\\
334.01	0\\
335.01	0\\
336.01	0\\
337.01	0\\
338.01	0\\
339.01	0\\
340.01	0\\
341.01	0\\
342.01	0\\
343.01	0\\
344.01	0\\
345.01	0\\
346.01	0\\
347.01	0\\
348.01	0\\
349.01	0\\
350.01	0\\
351.01	0\\
352.01	0\\
353.01	0\\
354.01	0\\
355.01	0\\
356.01	0\\
357.01	0\\
358.01	0\\
359.01	0\\
360.01	0\\
361.01	0\\
362.01	0\\
363.01	0\\
364.01	0\\
365.01	0\\
366.01	0\\
367.01	0\\
368.01	0\\
369.01	0\\
370.01	0\\
371.01	0\\
372.01	0\\
373.01	0\\
374.01	0\\
375.01	0\\
376.01	0\\
377.01	0\\
378.01	0\\
379.01	0\\
380.01	0\\
381.01	0\\
382.01	0\\
383.01	0\\
384.01	0\\
385.01	0\\
386.01	0\\
387.01	0\\
388.01	0\\
389.01	0\\
390.01	0\\
391.01	0\\
392.01	0\\
393.01	0\\
394.01	0\\
395.01	0\\
396.01	0\\
397.01	0\\
398.01	0\\
399.01	0\\
400.01	0\\
401.01	0\\
402.01	0\\
403.01	0\\
404.01	0\\
405.01	0\\
406.01	0\\
407.01	0\\
408.01	0\\
409.01	0\\
410.01	0\\
411.01	0\\
412.01	0\\
413.01	0\\
414.01	0\\
415.01	0\\
416.01	0\\
417.01	0\\
418.01	0\\
419.01	0\\
420.01	0\\
421.01	0\\
422.01	0\\
423.01	0\\
424.01	0\\
425.01	0\\
426.01	0\\
427.01	0\\
428.01	0\\
429.01	0\\
430.01	0\\
431.01	0\\
432.01	0\\
433.01	0\\
434.01	0\\
435.01	0\\
436.01	0\\
437.01	0\\
438.01	0\\
439.01	0\\
440.01	0\\
441.01	0\\
442.01	0\\
443.01	0\\
444.01	0\\
445.01	0\\
446.01	0\\
447.01	0\\
448.01	0\\
449.01	0\\
450.01	0\\
451.01	0\\
452.01	0\\
453.01	0\\
454.01	0\\
455.01	0\\
456.01	0\\
457.01	0\\
458.01	0\\
459.01	0\\
460.01	0\\
461.01	0\\
462.01	0\\
463.01	0\\
464.01	0\\
465.01	0\\
466.01	0\\
467.01	0\\
468.01	0\\
469.01	0\\
470.01	0\\
471.01	0\\
472.01	0\\
473.01	0\\
474.01	0\\
475.01	0\\
476.01	0\\
477.01	0\\
478.01	0\\
479.01	0\\
480.01	0\\
481.01	0\\
482.01	0\\
483.01	0\\
484.01	0\\
485.01	0\\
486.01	0\\
487.01	0\\
488.01	0\\
489.01	0\\
490.01	0\\
491.01	0\\
492.01	0\\
493.01	0\\
494.01	0\\
495.01	0\\
496.01	0\\
497.01	0\\
498.01	0\\
499.01	0\\
500.01	0\\
501.01	0\\
502.01	0\\
503.01	0\\
504.01	0\\
505.01	0\\
506.01	0\\
507.01	0\\
508.01	0\\
509.01	0\\
510.01	0\\
511.01	0\\
512.01	0\\
513.01	0\\
514.01	0\\
515.01	0\\
516.01	0\\
517.01	0\\
518.01	0\\
519.01	0\\
520.01	0\\
521.01	0\\
522.01	0\\
523.01	0\\
524.01	0\\
525.01	0\\
526.01	0\\
527.01	0\\
528.01	0\\
529.01	0\\
530.01	0\\
531.01	0\\
532.01	0\\
533.01	0\\
534.01	0\\
535.01	0\\
536.01	0\\
537.01	0\\
538.01	0\\
539.01	0\\
540.01	0\\
541.01	0\\
542.01	0\\
543.01	0\\
544.01	0\\
545.01	0\\
546.01	0\\
547.01	0\\
548.01	0\\
549.01	0\\
550.01	0\\
551.01	0\\
552.01	0\\
553.01	0\\
554.01	0\\
555.01	0\\
556.01	0\\
557.01	0\\
558.01	0\\
559.01	0\\
560.01	0\\
561.01	0\\
562.01	0\\
563.01	0\\
564.01	0\\
565.01	0\\
566.01	0\\
567.01	0\\
568.01	0\\
569.01	0\\
570.01	0\\
571.01	0\\
572.01	0\\
573.01	0\\
574.01	0\\
575.01	0\\
576.01	0\\
577.01	0\\
578.01	0\\
579.01	0\\
580.01	0\\
581.01	0\\
582.01	0\\
583.01	0\\
584.01	0\\
585.01	0\\
586.01	0\\
587.01	0\\
588.01	0\\
589.01	0\\
590.01	0\\
591.01	0\\
592.01	0\\
593.01	0\\
594.01	0\\
595.01	0\\
596.01	0\\
597.01	0\\
598.01	0\\
599.01	0.00375812205680727\\
599.02	0.00379584782681452\\
599.03	0.00383394044928508\\
599.04	0.00387240353649723\\
599.05	0.00391124073635877\\
599.06	0.00395045573268432\\
599.07	0.00399005224546545\\
599.08	0.00403003403118953\\
599.09	0.0040704048832826\\
599.1	0.00411116863239879\\
599.11	0.0041523291466905\\
599.12	0.00419389033207165\\
599.13	0.00423585613248164\\
599.14	0.00427823053012622\\
599.15	0.00432101754570572\\
599.16	0.00436422123862898\\
599.17	0.00440784570721336\\
599.18	0.0044518950888699\\
599.19	0.00449637356026637\\
599.2	0.00454128533746953\\
599.21	0.00458663467686692\\
599.22	0.00463242587666335\\
599.23	0.00467866327729075\\
599.24	0.00472535126182209\\
599.25	0.00477249425638942\\
599.26	0.00482009673060606\\
599.27	0.00486816319799273\\
599.28	0.00491669821640737\\
599.29	0.00496570638847912\\
599.3	0.00501519236204661\\
599.31	0.00506516083060044\\
599.32	0.00511561653373015\\
599.33	0.00516656425757546\\
599.34	0.00521800883528218\\
599.35	0.00526995514746255\\
599.36	0.0053224081226603\\
599.37	0.00537537273782044\\
599.38	0.00542885401876388\\
599.39	0.00548285703669257\\
599.4	0.00553738689825535\\
599.41	0.00559244876006856\\
599.42	0.00564804782920575\\
599.43	0.00570418936369266\\
599.44	0.00576087867300769\\
599.45	0.00581812111858793\\
599.46	0.00587592211434108\\
599.47	0.00593428712716332\\
599.48	0.00599322167746327\\
599.49	0.00605273133968783\\
599.5	0.00611282174285112\\
599.51	0.00617349857106859\\
599.52	0.00623476756409629\\
599.53	0.00629663451787551\\
599.54	0.00635910528508266\\
599.55	0.00642218577568459\\
599.56	0.00648588195749926\\
599.57	0.00655019985676195\\
599.58	0.00661514555869698\\
599.59	0.00668072520809509\\
599.6	0.00674694500989631\\
599.61	0.00681381122977868\\
599.62	0.00688133019475265\\
599.63	0.00694950829376133\\
599.64	0.00701835197828654\\
599.65	0.00708786776296088\\
599.66	0.00715806222618573\\
599.67	0.0072289420107553\\
599.68	0.00730051382448678\\
599.69	0.00737278444085667\\
599.7	0.00744576069964336\\
599.71	0.00751944950757594\\
599.72	0.00759385783898938\\
599.73	0.0076689927364862\\
599.74	0.00774486131160447\\
599.75	0.00782147074549251\\
599.76	0.00789882828959007\\
599.77	0.0079769412663163\\
599.78	0.00805581706976437\\
599.79	0.00813546316640295\\
599.8	0.00821588709578454\\
599.81	0.00829709647126081\\
599.82	0.00837909898070485\\
599.83	0.00846190238724062\\
599.84	0.00854551452997949\\
599.85	0.00862994332476403\\
599.86	0.00871519676491913\\
599.87	0.00880128292201045\\
599.88	0.00888820994661034\\
599.89	0.00897598606907129\\
599.9	0.00906461960030698\\
599.91	0.00915411893258095\\
599.92	0.00924449254030311\\
599.93	0.00933574898083391\\
599.94	0.00942789689529656\\
599.95	0.00952094500939709\\
599.96	0.00961490213425245\\
599.97	0.00970977716722672\\
599.98	0.00980557909277551\\
599.99	0.00990231698329844\\
600	0.01\\
};
\addplot [color=red!25!mycolor17,solid,forget plot]
  table[row sep=crcr]{%
0.01	0\\
1.01	0\\
2.01	0\\
3.01	0\\
4.01	0\\
5.01	0\\
6.01	0\\
7.01	0\\
8.01	0\\
9.01	0\\
10.01	0\\
11.01	0\\
12.01	0\\
13.01	0\\
14.01	0\\
15.01	0\\
16.01	0\\
17.01	0\\
18.01	0\\
19.01	0\\
20.01	0\\
21.01	0\\
22.01	0\\
23.01	0\\
24.01	0\\
25.01	0\\
26.01	0\\
27.01	0\\
28.01	0\\
29.01	0\\
30.01	0\\
31.01	0\\
32.01	0\\
33.01	0\\
34.01	0\\
35.01	0\\
36.01	0\\
37.01	0\\
38.01	0\\
39.01	0\\
40.01	0\\
41.01	0\\
42.01	0\\
43.01	0\\
44.01	0\\
45.01	0\\
46.01	0\\
47.01	0\\
48.01	0\\
49.01	0\\
50.01	0\\
51.01	0\\
52.01	0\\
53.01	0\\
54.01	0\\
55.01	0\\
56.01	0\\
57.01	0\\
58.01	0\\
59.01	0\\
60.01	0\\
61.01	0\\
62.01	0\\
63.01	0\\
64.01	0\\
65.01	0\\
66.01	0\\
67.01	0\\
68.01	0\\
69.01	0\\
70.01	0\\
71.01	0\\
72.01	0\\
73.01	0\\
74.01	0\\
75.01	0\\
76.01	0\\
77.01	0\\
78.01	0\\
79.01	0\\
80.01	0\\
81.01	0\\
82.01	0\\
83.01	0\\
84.01	0\\
85.01	0\\
86.01	0\\
87.01	0\\
88.01	0\\
89.01	0\\
90.01	0\\
91.01	0\\
92.01	0\\
93.01	0\\
94.01	0\\
95.01	0\\
96.01	0\\
97.01	0\\
98.01	0\\
99.01	0\\
100.01	0\\
101.01	0\\
102.01	0\\
103.01	0\\
104.01	0\\
105.01	0\\
106.01	0\\
107.01	0\\
108.01	0\\
109.01	0\\
110.01	0\\
111.01	0\\
112.01	0\\
113.01	0\\
114.01	0\\
115.01	0\\
116.01	0\\
117.01	0\\
118.01	0\\
119.01	0\\
120.01	0\\
121.01	0\\
122.01	0\\
123.01	0\\
124.01	0\\
125.01	0\\
126.01	0\\
127.01	0\\
128.01	0\\
129.01	0\\
130.01	0\\
131.01	0\\
132.01	0\\
133.01	0\\
134.01	0\\
135.01	0\\
136.01	0\\
137.01	0\\
138.01	0\\
139.01	0\\
140.01	0\\
141.01	0\\
142.01	0\\
143.01	0\\
144.01	0\\
145.01	0\\
146.01	0\\
147.01	0\\
148.01	0\\
149.01	0\\
150.01	0\\
151.01	0\\
152.01	0\\
153.01	0\\
154.01	0\\
155.01	0\\
156.01	0\\
157.01	0\\
158.01	0\\
159.01	0\\
160.01	0\\
161.01	0\\
162.01	0\\
163.01	0\\
164.01	0\\
165.01	0\\
166.01	0\\
167.01	0\\
168.01	0\\
169.01	0\\
170.01	0\\
171.01	0\\
172.01	0\\
173.01	0\\
174.01	0\\
175.01	0\\
176.01	0\\
177.01	0\\
178.01	0\\
179.01	0\\
180.01	0\\
181.01	0\\
182.01	0\\
183.01	0\\
184.01	0\\
185.01	0\\
186.01	0\\
187.01	0\\
188.01	0\\
189.01	0\\
190.01	0\\
191.01	0\\
192.01	0\\
193.01	0\\
194.01	0\\
195.01	0\\
196.01	0\\
197.01	0\\
198.01	0\\
199.01	0\\
200.01	0\\
201.01	0\\
202.01	0\\
203.01	0\\
204.01	0\\
205.01	0\\
206.01	0\\
207.01	0\\
208.01	0\\
209.01	0\\
210.01	0\\
211.01	0\\
212.01	0\\
213.01	0\\
214.01	0\\
215.01	0\\
216.01	0\\
217.01	0\\
218.01	0\\
219.01	0\\
220.01	0\\
221.01	0\\
222.01	0\\
223.01	0\\
224.01	0\\
225.01	0\\
226.01	0\\
227.01	0\\
228.01	0\\
229.01	0\\
230.01	0\\
231.01	0\\
232.01	0\\
233.01	0\\
234.01	0\\
235.01	0\\
236.01	0\\
237.01	0\\
238.01	0\\
239.01	0\\
240.01	0\\
241.01	0\\
242.01	0\\
243.01	0\\
244.01	0\\
245.01	0\\
246.01	0\\
247.01	0\\
248.01	0\\
249.01	0\\
250.01	0\\
251.01	0\\
252.01	0\\
253.01	0\\
254.01	0\\
255.01	0\\
256.01	0\\
257.01	0\\
258.01	0\\
259.01	0\\
260.01	0\\
261.01	0\\
262.01	0\\
263.01	0\\
264.01	0\\
265.01	0\\
266.01	0\\
267.01	0\\
268.01	0\\
269.01	0\\
270.01	0\\
271.01	0\\
272.01	0\\
273.01	0\\
274.01	0\\
275.01	0\\
276.01	0\\
277.01	0\\
278.01	0\\
279.01	0\\
280.01	0\\
281.01	0\\
282.01	0\\
283.01	0\\
284.01	0\\
285.01	0\\
286.01	0\\
287.01	0\\
288.01	0\\
289.01	0\\
290.01	0\\
291.01	0\\
292.01	0\\
293.01	0\\
294.01	0\\
295.01	0\\
296.01	0\\
297.01	0\\
298.01	0\\
299.01	0\\
300.01	0\\
301.01	0\\
302.01	0\\
303.01	0\\
304.01	0\\
305.01	0\\
306.01	0\\
307.01	0\\
308.01	0\\
309.01	0\\
310.01	0\\
311.01	0\\
312.01	0\\
313.01	0\\
314.01	0\\
315.01	0\\
316.01	0\\
317.01	0\\
318.01	0\\
319.01	0\\
320.01	0\\
321.01	0\\
322.01	0\\
323.01	0\\
324.01	0\\
325.01	0\\
326.01	0\\
327.01	0\\
328.01	0\\
329.01	0\\
330.01	0\\
331.01	0\\
332.01	0\\
333.01	0\\
334.01	0\\
335.01	0\\
336.01	0\\
337.01	0\\
338.01	0\\
339.01	0\\
340.01	0\\
341.01	0\\
342.01	0\\
343.01	0\\
344.01	0\\
345.01	0\\
346.01	0\\
347.01	0\\
348.01	0\\
349.01	0\\
350.01	0\\
351.01	0\\
352.01	0\\
353.01	0\\
354.01	0\\
355.01	0\\
356.01	0\\
357.01	0\\
358.01	0\\
359.01	0\\
360.01	0\\
361.01	0\\
362.01	0\\
363.01	0\\
364.01	0\\
365.01	0\\
366.01	0\\
367.01	0\\
368.01	0\\
369.01	0\\
370.01	0\\
371.01	0\\
372.01	0\\
373.01	0\\
374.01	0\\
375.01	0\\
376.01	0\\
377.01	0\\
378.01	0\\
379.01	0\\
380.01	0\\
381.01	0\\
382.01	0\\
383.01	0\\
384.01	0\\
385.01	0\\
386.01	0\\
387.01	0\\
388.01	0\\
389.01	0\\
390.01	0\\
391.01	0\\
392.01	0\\
393.01	0\\
394.01	0\\
395.01	0\\
396.01	0\\
397.01	0\\
398.01	0\\
399.01	0\\
400.01	0\\
401.01	0\\
402.01	0\\
403.01	0\\
404.01	0\\
405.01	0\\
406.01	0\\
407.01	0\\
408.01	0\\
409.01	0\\
410.01	0\\
411.01	0\\
412.01	0\\
413.01	0\\
414.01	0\\
415.01	0\\
416.01	0\\
417.01	0\\
418.01	0\\
419.01	0\\
420.01	0\\
421.01	0\\
422.01	0\\
423.01	0\\
424.01	0\\
425.01	0\\
426.01	0\\
427.01	0\\
428.01	0\\
429.01	0\\
430.01	0\\
431.01	0\\
432.01	0\\
433.01	0\\
434.01	0\\
435.01	0\\
436.01	0\\
437.01	0\\
438.01	0\\
439.01	0\\
440.01	0\\
441.01	0\\
442.01	0\\
443.01	0\\
444.01	0\\
445.01	0\\
446.01	0\\
447.01	0\\
448.01	0\\
449.01	0\\
450.01	0\\
451.01	0\\
452.01	0\\
453.01	0\\
454.01	0\\
455.01	0\\
456.01	0\\
457.01	0\\
458.01	0\\
459.01	0\\
460.01	0\\
461.01	0\\
462.01	0\\
463.01	0\\
464.01	0\\
465.01	0\\
466.01	0\\
467.01	0\\
468.01	0\\
469.01	0\\
470.01	0\\
471.01	0\\
472.01	0\\
473.01	0\\
474.01	0\\
475.01	0\\
476.01	0\\
477.01	0\\
478.01	0\\
479.01	0\\
480.01	0\\
481.01	0\\
482.01	0\\
483.01	0\\
484.01	0\\
485.01	0\\
486.01	0\\
487.01	0\\
488.01	0\\
489.01	0\\
490.01	0\\
491.01	0\\
492.01	0\\
493.01	0\\
494.01	0\\
495.01	0\\
496.01	0\\
497.01	0\\
498.01	0\\
499.01	0\\
500.01	0\\
501.01	0\\
502.01	0\\
503.01	0\\
504.01	0\\
505.01	0\\
506.01	0\\
507.01	0\\
508.01	0\\
509.01	0\\
510.01	0\\
511.01	0\\
512.01	0\\
513.01	0\\
514.01	0\\
515.01	0\\
516.01	0\\
517.01	0\\
518.01	0\\
519.01	0\\
520.01	0\\
521.01	0\\
522.01	0\\
523.01	0\\
524.01	0\\
525.01	0\\
526.01	0\\
527.01	0\\
528.01	0\\
529.01	0\\
530.01	0\\
531.01	0\\
532.01	0\\
533.01	0\\
534.01	0\\
535.01	0\\
536.01	0\\
537.01	0\\
538.01	0\\
539.01	0\\
540.01	0\\
541.01	0\\
542.01	0\\
543.01	0\\
544.01	0\\
545.01	0\\
546.01	0\\
547.01	0\\
548.01	0\\
549.01	0\\
550.01	0\\
551.01	0\\
552.01	0\\
553.01	0\\
554.01	0\\
555.01	0\\
556.01	0\\
557.01	0\\
558.01	0\\
559.01	0\\
560.01	0\\
561.01	0\\
562.01	0\\
563.01	0\\
564.01	0\\
565.01	0\\
566.01	0\\
567.01	0\\
568.01	0\\
569.01	0\\
570.01	0\\
571.01	0\\
572.01	0\\
573.01	0\\
574.01	0\\
575.01	0\\
576.01	0\\
577.01	0\\
578.01	0\\
579.01	0\\
580.01	0\\
581.01	0\\
582.01	0\\
583.01	0\\
584.01	0\\
585.01	0\\
586.01	0\\
587.01	0\\
588.01	0\\
589.01	0\\
590.01	0\\
591.01	0\\
592.01	0\\
593.01	0\\
594.01	0\\
595.01	0\\
596.01	0\\
597.01	0\\
598.01	0\\
599.01	0.00375813078223648\\
599.02	0.00379585547607227\\
599.03	0.00383394714291217\\
599.04	0.00387240938738669\\
599.05	0.00391124584952775\\
599.06	0.00395046020511587\\
599.07	0.00399005616603103\\
599.08	0.00403003748060689\\
599.09	0.00407040793398825\\
599.1	0.00411117134849209\\
599.11	0.00415233158397238\\
599.12	0.00419389253818866\\
599.13	0.00423585814717847\\
599.14	0.00427823238563368\\
599.15	0.00432101926728097\\
599.16	0.00436422284526639\\
599.17	0.00440784721254407\\
599.18	0.00445189650226941\\
599.19	0.00449637488819665\\
599.2	0.00454128658508102\\
599.21	0.00458663584908382\\
599.22	0.00463242697817922\\
599.23	0.00467866431256501\\
599.24	0.00472535223507736\\
599.25	0.00477249517160974\\
599.26	0.00482009759153586\\
599.27	0.0048681640081368\\
599.28	0.00491669897903243\\
599.29	0.00496570710661689\\
599.3	0.00501519303849851\\
599.31	0.005065161467944\\
599.32	0.00511561713432699\\
599.33	0.00516656482358099\\
599.34	0.00521800936865682\\
599.35	0.00526995564998455\\
599.36	0.0053224085959399\\
599.37	0.00537537318331537\\
599.38	0.00542885443779583\\
599.39	0.00548285743046435\\
599.4	0.00553738726786889\\
599.41	0.0055924491065429\\
599.42	0.0056480481534946\\
599.43	0.00570418966670113\\
599.44	0.00576087895560742\\
599.45	0.0058181213816301\\
599.46	0.00587592235866615\\
599.47	0.00593428735360669\\
599.48	0.00599322188685565\\
599.49	0.00605273153285365\\
599.5	0.00611282192060684\\
599.51	0.00617349873422108\\
599.52	0.00623476771344123\\
599.53	0.00629663465419579\\
599.54	0.00635910540914682\\
599.55	0.00642218588824529\\
599.56	0.00648588205929183\\
599.57	0.006550199948503\\
599.58	0.00661514564108305\\
599.59	0.00668072528180133\\
599.6	0.00674694507557537\\
599.61	0.00681381128805956\\
599.62	0.00688133024623971\\
599.63	0.00694950833903336\\
599.64	0.00701835201789602\\
599.65	0.00708786779743327\\
599.66	0.00715806225601893\\
599.67	0.00722894203641922\\
599.68	0.00730051384642307\\
599.69	0.00737278445947858\\
599.7	0.00744576071533571\\
599.71	0.00751944952069528\\
599.72	0.00759385784986428\\
599.73	0.00766899274541769\\
599.74	0.0077448613188666\\
599.75	0.0078214707513331\\
599.76	0.00789882829423158\\
599.77	0.00797694126995682\\
599.78	0.00805581707257874\\
599.79	0.00813546316854402\\
599.8	0.00821588709738451\\
599.81	0.00829709647243267\\
599.82	0.00837909898154391\\
599.83	0.00846190238782606\\
599.84	0.00854551453037601\\
599.85	0.00862994332502348\\
599.86	0.00871519676508213\\
599.87	0.00880128292210799\\
599.88	0.00888820994666535\\
599.89	0.0089759860691001\\
599.9	0.0090646196003207\\
599.91	0.00915411893258671\\
599.92	0.00924449254030512\\
599.93	0.00933574898083444\\
599.94	0.00942789689529664\\
599.95	0.00952094500939709\\
599.96	0.00961490213425244\\
599.97	0.00970977716722672\\
599.98	0.00980557909277551\\
599.99	0.00990231698329844\\
600	0.01\\
};
\addplot [color=mycolor19,solid,forget plot]
  table[row sep=crcr]{%
0.01	0\\
1.01	0\\
2.01	0\\
3.01	0\\
4.01	0\\
5.01	0\\
6.01	0\\
7.01	0\\
8.01	0\\
9.01	0\\
10.01	0\\
11.01	0\\
12.01	0\\
13.01	0\\
14.01	0\\
15.01	0\\
16.01	0\\
17.01	0\\
18.01	0\\
19.01	0\\
20.01	0\\
21.01	0\\
22.01	0\\
23.01	0\\
24.01	0\\
25.01	0\\
26.01	0\\
27.01	0\\
28.01	0\\
29.01	0\\
30.01	0\\
31.01	0\\
32.01	0\\
33.01	0\\
34.01	0\\
35.01	0\\
36.01	0\\
37.01	0\\
38.01	0\\
39.01	0\\
40.01	0\\
41.01	0\\
42.01	0\\
43.01	0\\
44.01	0\\
45.01	0\\
46.01	0\\
47.01	0\\
48.01	0\\
49.01	0\\
50.01	0\\
51.01	0\\
52.01	0\\
53.01	0\\
54.01	0\\
55.01	0\\
56.01	0\\
57.01	0\\
58.01	0\\
59.01	0\\
60.01	0\\
61.01	0\\
62.01	0\\
63.01	0\\
64.01	0\\
65.01	0\\
66.01	0\\
67.01	0\\
68.01	0\\
69.01	0\\
70.01	0\\
71.01	0\\
72.01	0\\
73.01	0\\
74.01	0\\
75.01	0\\
76.01	0\\
77.01	0\\
78.01	0\\
79.01	0\\
80.01	0\\
81.01	0\\
82.01	0\\
83.01	0\\
84.01	0\\
85.01	0\\
86.01	0\\
87.01	0\\
88.01	0\\
89.01	0\\
90.01	0\\
91.01	0\\
92.01	0\\
93.01	0\\
94.01	0\\
95.01	0\\
96.01	0\\
97.01	0\\
98.01	0\\
99.01	0\\
100.01	0\\
101.01	0\\
102.01	0\\
103.01	0\\
104.01	0\\
105.01	0\\
106.01	0\\
107.01	0\\
108.01	0\\
109.01	0\\
110.01	0\\
111.01	0\\
112.01	0\\
113.01	0\\
114.01	0\\
115.01	0\\
116.01	0\\
117.01	0\\
118.01	0\\
119.01	0\\
120.01	0\\
121.01	0\\
122.01	0\\
123.01	0\\
124.01	0\\
125.01	0\\
126.01	0\\
127.01	0\\
128.01	0\\
129.01	0\\
130.01	0\\
131.01	0\\
132.01	0\\
133.01	0\\
134.01	0\\
135.01	0\\
136.01	0\\
137.01	0\\
138.01	0\\
139.01	0\\
140.01	0\\
141.01	0\\
142.01	0\\
143.01	0\\
144.01	0\\
145.01	0\\
146.01	0\\
147.01	0\\
148.01	0\\
149.01	0\\
150.01	0\\
151.01	0\\
152.01	0\\
153.01	0\\
154.01	0\\
155.01	0\\
156.01	0\\
157.01	0\\
158.01	0\\
159.01	0\\
160.01	0\\
161.01	0\\
162.01	0\\
163.01	0\\
164.01	0\\
165.01	0\\
166.01	0\\
167.01	0\\
168.01	0\\
169.01	0\\
170.01	0\\
171.01	0\\
172.01	0\\
173.01	0\\
174.01	0\\
175.01	0\\
176.01	0\\
177.01	0\\
178.01	0\\
179.01	0\\
180.01	0\\
181.01	0\\
182.01	0\\
183.01	0\\
184.01	0\\
185.01	0\\
186.01	0\\
187.01	0\\
188.01	0\\
189.01	0\\
190.01	0\\
191.01	0\\
192.01	0\\
193.01	0\\
194.01	0\\
195.01	0\\
196.01	0\\
197.01	0\\
198.01	0\\
199.01	0\\
200.01	0\\
201.01	0\\
202.01	0\\
203.01	0\\
204.01	0\\
205.01	0\\
206.01	0\\
207.01	0\\
208.01	0\\
209.01	0\\
210.01	0\\
211.01	0\\
212.01	0\\
213.01	0\\
214.01	0\\
215.01	0\\
216.01	0\\
217.01	0\\
218.01	0\\
219.01	0\\
220.01	0\\
221.01	0\\
222.01	0\\
223.01	0\\
224.01	0\\
225.01	0\\
226.01	0\\
227.01	0\\
228.01	0\\
229.01	0\\
230.01	0\\
231.01	0\\
232.01	0\\
233.01	0\\
234.01	0\\
235.01	0\\
236.01	0\\
237.01	0\\
238.01	0\\
239.01	0\\
240.01	0\\
241.01	0\\
242.01	0\\
243.01	0\\
244.01	0\\
245.01	0\\
246.01	0\\
247.01	0\\
248.01	0\\
249.01	0\\
250.01	0\\
251.01	0\\
252.01	0\\
253.01	0\\
254.01	0\\
255.01	0\\
256.01	0\\
257.01	0\\
258.01	0\\
259.01	0\\
260.01	0\\
261.01	0\\
262.01	0\\
263.01	0\\
264.01	0\\
265.01	0\\
266.01	0\\
267.01	0\\
268.01	0\\
269.01	0\\
270.01	0\\
271.01	0\\
272.01	0\\
273.01	0\\
274.01	0\\
275.01	0\\
276.01	0\\
277.01	0\\
278.01	0\\
279.01	0\\
280.01	0\\
281.01	0\\
282.01	0\\
283.01	0\\
284.01	0\\
285.01	0\\
286.01	0\\
287.01	0\\
288.01	0\\
289.01	0\\
290.01	0\\
291.01	0\\
292.01	0\\
293.01	0\\
294.01	0\\
295.01	0\\
296.01	0\\
297.01	0\\
298.01	0\\
299.01	0\\
300.01	0\\
301.01	0\\
302.01	0\\
303.01	0\\
304.01	0\\
305.01	0\\
306.01	0\\
307.01	0\\
308.01	0\\
309.01	0\\
310.01	0\\
311.01	0\\
312.01	0\\
313.01	0\\
314.01	0\\
315.01	0\\
316.01	0\\
317.01	0\\
318.01	0\\
319.01	0\\
320.01	0\\
321.01	0\\
322.01	0\\
323.01	0\\
324.01	0\\
325.01	0\\
326.01	0\\
327.01	0\\
328.01	0\\
329.01	0\\
330.01	0\\
331.01	0\\
332.01	0\\
333.01	0\\
334.01	0\\
335.01	0\\
336.01	0\\
337.01	0\\
338.01	0\\
339.01	0\\
340.01	0\\
341.01	0\\
342.01	0\\
343.01	0\\
344.01	0\\
345.01	0\\
346.01	0\\
347.01	0\\
348.01	0\\
349.01	0\\
350.01	0\\
351.01	0\\
352.01	0\\
353.01	0\\
354.01	0\\
355.01	0\\
356.01	0\\
357.01	0\\
358.01	0\\
359.01	0\\
360.01	0\\
361.01	0\\
362.01	0\\
363.01	0\\
364.01	0\\
365.01	0\\
366.01	0\\
367.01	0\\
368.01	0\\
369.01	0\\
370.01	0\\
371.01	0\\
372.01	0\\
373.01	0\\
374.01	0\\
375.01	0\\
376.01	0\\
377.01	0\\
378.01	0\\
379.01	0\\
380.01	0\\
381.01	0\\
382.01	0\\
383.01	0\\
384.01	0\\
385.01	0\\
386.01	0\\
387.01	0\\
388.01	0\\
389.01	0\\
390.01	0\\
391.01	0\\
392.01	0\\
393.01	0\\
394.01	0\\
395.01	0\\
396.01	0\\
397.01	0\\
398.01	0\\
399.01	0\\
400.01	0\\
401.01	0\\
402.01	0\\
403.01	0\\
404.01	0\\
405.01	0\\
406.01	0\\
407.01	0\\
408.01	0\\
409.01	0\\
410.01	0\\
411.01	0\\
412.01	0\\
413.01	0\\
414.01	0\\
415.01	0\\
416.01	0\\
417.01	0\\
418.01	0\\
419.01	0\\
420.01	0\\
421.01	0\\
422.01	0\\
423.01	0\\
424.01	0\\
425.01	0\\
426.01	0\\
427.01	0\\
428.01	0\\
429.01	0\\
430.01	0\\
431.01	0\\
432.01	0\\
433.01	0\\
434.01	0\\
435.01	0\\
436.01	0\\
437.01	0\\
438.01	0\\
439.01	0\\
440.01	0\\
441.01	0\\
442.01	0\\
443.01	0\\
444.01	0\\
445.01	0\\
446.01	0\\
447.01	0\\
448.01	0\\
449.01	0\\
450.01	0\\
451.01	0\\
452.01	0\\
453.01	0\\
454.01	0\\
455.01	0\\
456.01	0\\
457.01	0\\
458.01	0\\
459.01	0\\
460.01	0\\
461.01	0\\
462.01	0\\
463.01	0\\
464.01	0\\
465.01	0\\
466.01	0\\
467.01	0\\
468.01	0\\
469.01	0\\
470.01	0\\
471.01	0\\
472.01	0\\
473.01	0\\
474.01	0\\
475.01	0\\
476.01	0\\
477.01	0\\
478.01	0\\
479.01	0\\
480.01	0\\
481.01	0\\
482.01	0\\
483.01	0\\
484.01	0\\
485.01	0\\
486.01	0\\
487.01	0\\
488.01	0\\
489.01	0\\
490.01	0\\
491.01	0\\
492.01	0\\
493.01	0\\
494.01	0\\
495.01	0\\
496.01	0\\
497.01	0\\
498.01	0\\
499.01	0\\
500.01	0\\
501.01	0\\
502.01	0\\
503.01	0\\
504.01	0\\
505.01	0\\
506.01	0\\
507.01	0\\
508.01	0\\
509.01	0\\
510.01	0\\
511.01	0\\
512.01	0\\
513.01	0\\
514.01	0\\
515.01	0\\
516.01	0\\
517.01	0\\
518.01	0\\
519.01	0\\
520.01	0\\
521.01	0\\
522.01	0\\
523.01	0\\
524.01	0\\
525.01	0\\
526.01	0\\
527.01	0\\
528.01	0\\
529.01	0\\
530.01	0\\
531.01	0\\
532.01	0\\
533.01	0\\
534.01	0\\
535.01	0\\
536.01	0\\
537.01	0\\
538.01	0\\
539.01	0\\
540.01	0\\
541.01	0\\
542.01	0\\
543.01	0\\
544.01	0\\
545.01	0\\
546.01	0\\
547.01	0\\
548.01	0\\
549.01	0\\
550.01	0\\
551.01	0\\
552.01	0\\
553.01	0\\
554.01	0\\
555.01	0\\
556.01	0\\
557.01	0\\
558.01	0\\
559.01	0\\
560.01	0\\
561.01	0\\
562.01	0\\
563.01	0\\
564.01	0\\
565.01	0\\
566.01	0\\
567.01	0\\
568.01	0\\
569.01	0\\
570.01	0\\
571.01	0\\
572.01	0\\
573.01	0\\
574.01	0\\
575.01	0\\
576.01	0\\
577.01	0\\
578.01	0\\
579.01	0\\
580.01	0\\
581.01	0\\
582.01	0\\
583.01	0\\
584.01	0\\
585.01	0\\
586.01	0\\
587.01	0\\
588.01	0\\
589.01	0\\
590.01	0\\
591.01	0\\
592.01	0\\
593.01	0\\
594.01	0\\
595.01	0\\
596.01	0\\
597.01	0\\
598.01	0\\
599.01	0.00375813090363212\\
599.02	0.0037958555854239\\
599.03	0.00383394724176133\\
599.04	0.00387240947710218\\
599.05	0.00391124593131475\\
599.06	0.00395046028002561\\
599.07	0.00399005623497103\\
599.08	0.00403003754435165\\
599.09	0.00407040799319081\\
599.1	0.00411117140369623\\
599.11	0.00415233163562539\\
599.12	0.00419389258665448\\
599.13	0.00423585819275083\\
599.14	0.00427823242854921\\
599.15	0.00432101930773168\\
599.16	0.0043642228834112\\
599.17	0.00440784724851909\\
599.18	0.00445189653619617\\
599.19	0.00449637492018781\\
599.2	0.00454128661524282\\
599.21	0.0045866358775163\\
599.22	0.00463242700497642\\
599.23	0.00467866433781515\\
599.24	0.00472535225886313\\
599.25	0.00477249519400854\\
599.26	0.00482009761262005\\
599.27	0.00486816402797405\\
599.28	0.00491669899768597\\
599.29	0.00496570712414586\\
599.3	0.00501519305495828\\
599.31	0.00506516148338648\\
599.32	0.00511561714880095\\
599.33	0.00516656483713241\\
599.34	0.00521800938132919\\
599.35	0.00526995566181915\\
599.36	0.00532240860697612\\
599.37	0.00537537319359091\\
599.38	0.005428854447347\\
599.39	0.00548285743932622\\
599.4	0.0055373872760755\\
599.41	0.00559244911412737\\
599.42	0.00564804816048927\\
599.43	0.00570418967313756\\
599.44	0.0057608789615165\\
599.45	0.00581812138704197\\
599.46	0.00587592236361025\\
599.47	0.00593428735811169\\
599.48	0.00599322189094943\\
599.49	0.00605273153656322\\
599.5	0.00611282192395836\\
599.51	0.00617349873723978\\
599.52	0.0062347677161514\\
599.53	0.00629663465662077\\
599.54	0.00635910541130898\\
599.55	0.00642218589016599\\
599.56	0.00648588206099141\\
599.57	0.00655019995000079\\
599.58	0.00661514564239734\\
599.59	0.00668072528294941\\
599.6	0.00674694507657347\\
599.61	0.00681381128892291\\
599.62	0.00688133024698253\\
599.63	0.00694950833966888\\
599.64	0.00701835201843649\\
599.65	0.00708786779788998\\
599.66	0.00715806225640225\\
599.67	0.00722894203673861\\
599.68	0.00730051384668715\\
599.69	0.00737278445969511\\
599.7	0.00744576071551167\\
599.71	0.00751944952083691\\
599.72	0.00759385784997709\\
599.73	0.00766899274550653\\
599.74	0.00774486131893572\\
599.75	0.00782147075138616\\
599.76	0.00789882829427172\\
599.77	0.00797694126998669\\
599.78	0.00805581707260058\\
599.79	0.00813546316855967\\
599.8	0.00821588709739548\\
599.81	0.00829709647244017\\
599.82	0.00837909898154888\\
599.83	0.00846190238782926\\
599.84	0.00854551453037799\\
599.85	0.00862994332502465\\
599.86	0.00871519676508279\\
599.87	0.00880128292210834\\
599.88	0.00888820994666552\\
599.89	0.00897598606910018\\
599.9	0.00906461960032072\\
599.91	0.00915411893258672\\
599.92	0.00924449254030512\\
599.93	0.00933574898083444\\
599.94	0.00942789689529664\\
599.95	0.00952094500939709\\
599.96	0.00961490213425244\\
599.97	0.00970977716722672\\
599.98	0.00980557909277551\\
599.99	0.00990231698329844\\
600	0.01\\
};
\addplot [color=red!50!mycolor17,solid,forget plot]
  table[row sep=crcr]{%
0.01	0\\
1.01	0\\
2.01	0\\
3.01	0\\
4.01	0\\
5.01	0\\
6.01	0\\
7.01	0\\
8.01	0\\
9.01	0\\
10.01	0\\
11.01	0\\
12.01	0\\
13.01	0\\
14.01	0\\
15.01	0\\
16.01	0\\
17.01	0\\
18.01	0\\
19.01	0\\
20.01	0\\
21.01	0\\
22.01	0\\
23.01	0\\
24.01	0\\
25.01	0\\
26.01	0\\
27.01	0\\
28.01	0\\
29.01	0\\
30.01	0\\
31.01	0\\
32.01	0\\
33.01	0\\
34.01	0\\
35.01	0\\
36.01	0\\
37.01	0\\
38.01	0\\
39.01	0\\
40.01	0\\
41.01	0\\
42.01	0\\
43.01	0\\
44.01	0\\
45.01	0\\
46.01	0\\
47.01	0\\
48.01	0\\
49.01	0\\
50.01	0\\
51.01	0\\
52.01	0\\
53.01	0\\
54.01	0\\
55.01	0\\
56.01	0\\
57.01	0\\
58.01	0\\
59.01	0\\
60.01	0\\
61.01	0\\
62.01	0\\
63.01	0\\
64.01	0\\
65.01	0\\
66.01	0\\
67.01	0\\
68.01	0\\
69.01	0\\
70.01	0\\
71.01	0\\
72.01	0\\
73.01	0\\
74.01	0\\
75.01	0\\
76.01	0\\
77.01	0\\
78.01	0\\
79.01	0\\
80.01	0\\
81.01	0\\
82.01	0\\
83.01	0\\
84.01	0\\
85.01	0\\
86.01	0\\
87.01	0\\
88.01	0\\
89.01	0\\
90.01	0\\
91.01	0\\
92.01	0\\
93.01	0\\
94.01	0\\
95.01	0\\
96.01	0\\
97.01	0\\
98.01	0\\
99.01	0\\
100.01	0\\
101.01	0\\
102.01	0\\
103.01	0\\
104.01	0\\
105.01	0\\
106.01	0\\
107.01	0\\
108.01	0\\
109.01	0\\
110.01	0\\
111.01	0\\
112.01	0\\
113.01	0\\
114.01	0\\
115.01	0\\
116.01	0\\
117.01	0\\
118.01	0\\
119.01	0\\
120.01	0\\
121.01	0\\
122.01	0\\
123.01	0\\
124.01	0\\
125.01	0\\
126.01	0\\
127.01	0\\
128.01	0\\
129.01	0\\
130.01	0\\
131.01	0\\
132.01	0\\
133.01	0\\
134.01	0\\
135.01	0\\
136.01	0\\
137.01	0\\
138.01	0\\
139.01	0\\
140.01	0\\
141.01	0\\
142.01	0\\
143.01	0\\
144.01	0\\
145.01	0\\
146.01	0\\
147.01	0\\
148.01	0\\
149.01	0\\
150.01	0\\
151.01	0\\
152.01	0\\
153.01	0\\
154.01	0\\
155.01	0\\
156.01	0\\
157.01	0\\
158.01	0\\
159.01	0\\
160.01	0\\
161.01	0\\
162.01	0\\
163.01	0\\
164.01	0\\
165.01	0\\
166.01	0\\
167.01	0\\
168.01	0\\
169.01	0\\
170.01	0\\
171.01	0\\
172.01	0\\
173.01	0\\
174.01	0\\
175.01	0\\
176.01	0\\
177.01	0\\
178.01	0\\
179.01	0\\
180.01	0\\
181.01	0\\
182.01	0\\
183.01	0\\
184.01	0\\
185.01	0\\
186.01	0\\
187.01	0\\
188.01	0\\
189.01	0\\
190.01	0\\
191.01	0\\
192.01	0\\
193.01	0\\
194.01	0\\
195.01	0\\
196.01	0\\
197.01	0\\
198.01	0\\
199.01	0\\
200.01	0\\
201.01	0\\
202.01	0\\
203.01	0\\
204.01	0\\
205.01	0\\
206.01	0\\
207.01	0\\
208.01	0\\
209.01	0\\
210.01	0\\
211.01	0\\
212.01	0\\
213.01	0\\
214.01	0\\
215.01	0\\
216.01	0\\
217.01	0\\
218.01	0\\
219.01	0\\
220.01	0\\
221.01	0\\
222.01	0\\
223.01	0\\
224.01	0\\
225.01	0\\
226.01	0\\
227.01	0\\
228.01	0\\
229.01	0\\
230.01	0\\
231.01	0\\
232.01	0\\
233.01	0\\
234.01	0\\
235.01	0\\
236.01	0\\
237.01	0\\
238.01	0\\
239.01	0\\
240.01	0\\
241.01	0\\
242.01	0\\
243.01	0\\
244.01	0\\
245.01	0\\
246.01	0\\
247.01	0\\
248.01	0\\
249.01	0\\
250.01	0\\
251.01	0\\
252.01	0\\
253.01	0\\
254.01	0\\
255.01	0\\
256.01	0\\
257.01	0\\
258.01	0\\
259.01	0\\
260.01	0\\
261.01	0\\
262.01	0\\
263.01	0\\
264.01	0\\
265.01	0\\
266.01	0\\
267.01	0\\
268.01	0\\
269.01	0\\
270.01	0\\
271.01	0\\
272.01	0\\
273.01	0\\
274.01	0\\
275.01	0\\
276.01	0\\
277.01	0\\
278.01	0\\
279.01	0\\
280.01	0\\
281.01	0\\
282.01	0\\
283.01	0\\
284.01	0\\
285.01	0\\
286.01	0\\
287.01	0\\
288.01	0\\
289.01	0\\
290.01	0\\
291.01	0\\
292.01	0\\
293.01	0\\
294.01	0\\
295.01	0\\
296.01	0\\
297.01	0\\
298.01	0\\
299.01	0\\
300.01	0\\
301.01	0\\
302.01	0\\
303.01	0\\
304.01	0\\
305.01	0\\
306.01	0\\
307.01	0\\
308.01	0\\
309.01	0\\
310.01	0\\
311.01	0\\
312.01	0\\
313.01	0\\
314.01	0\\
315.01	0\\
316.01	0\\
317.01	0\\
318.01	0\\
319.01	0\\
320.01	0\\
321.01	0\\
322.01	0\\
323.01	0\\
324.01	0\\
325.01	0\\
326.01	0\\
327.01	0\\
328.01	0\\
329.01	0\\
330.01	0\\
331.01	0\\
332.01	0\\
333.01	0\\
334.01	0\\
335.01	0\\
336.01	0\\
337.01	0\\
338.01	0\\
339.01	0\\
340.01	0\\
341.01	0\\
342.01	0\\
343.01	0\\
344.01	0\\
345.01	0\\
346.01	0\\
347.01	0\\
348.01	0\\
349.01	0\\
350.01	0\\
351.01	0\\
352.01	0\\
353.01	0\\
354.01	0\\
355.01	0\\
356.01	0\\
357.01	0\\
358.01	0\\
359.01	0\\
360.01	0\\
361.01	0\\
362.01	0\\
363.01	0\\
364.01	0\\
365.01	0\\
366.01	0\\
367.01	0\\
368.01	0\\
369.01	0\\
370.01	0\\
371.01	0\\
372.01	0\\
373.01	0\\
374.01	0\\
375.01	0\\
376.01	0\\
377.01	0\\
378.01	0\\
379.01	0\\
380.01	0\\
381.01	0\\
382.01	0\\
383.01	0\\
384.01	0\\
385.01	0\\
386.01	0\\
387.01	0\\
388.01	0\\
389.01	0\\
390.01	0\\
391.01	0\\
392.01	0\\
393.01	0\\
394.01	0\\
395.01	0\\
396.01	0\\
397.01	0\\
398.01	0\\
399.01	0\\
400.01	0\\
401.01	0\\
402.01	0\\
403.01	0\\
404.01	0\\
405.01	0\\
406.01	0\\
407.01	0\\
408.01	0\\
409.01	0\\
410.01	0\\
411.01	0\\
412.01	0\\
413.01	0\\
414.01	0\\
415.01	0\\
416.01	0\\
417.01	0\\
418.01	0\\
419.01	0\\
420.01	0\\
421.01	0\\
422.01	0\\
423.01	0\\
424.01	0\\
425.01	0\\
426.01	0\\
427.01	0\\
428.01	0\\
429.01	0\\
430.01	0\\
431.01	0\\
432.01	0\\
433.01	0\\
434.01	0\\
435.01	0\\
436.01	0\\
437.01	0\\
438.01	0\\
439.01	0\\
440.01	0\\
441.01	0\\
442.01	0\\
443.01	0\\
444.01	0\\
445.01	0\\
446.01	0\\
447.01	0\\
448.01	0\\
449.01	0\\
450.01	0\\
451.01	0\\
452.01	0\\
453.01	0\\
454.01	0\\
455.01	0\\
456.01	0\\
457.01	0\\
458.01	0\\
459.01	0\\
460.01	0\\
461.01	0\\
462.01	0\\
463.01	0\\
464.01	0\\
465.01	0\\
466.01	0\\
467.01	0\\
468.01	0\\
469.01	0\\
470.01	0\\
471.01	0\\
472.01	0\\
473.01	0\\
474.01	0\\
475.01	0\\
476.01	0\\
477.01	0\\
478.01	0\\
479.01	0\\
480.01	0\\
481.01	0\\
482.01	0\\
483.01	0\\
484.01	0\\
485.01	0\\
486.01	0\\
487.01	0\\
488.01	0\\
489.01	0\\
490.01	0\\
491.01	0\\
492.01	0\\
493.01	0\\
494.01	0\\
495.01	0\\
496.01	0\\
497.01	0\\
498.01	0\\
499.01	0\\
500.01	0\\
501.01	0\\
502.01	0\\
503.01	0\\
504.01	0\\
505.01	0\\
506.01	0\\
507.01	0\\
508.01	0\\
509.01	0\\
510.01	0\\
511.01	0\\
512.01	0\\
513.01	0\\
514.01	0\\
515.01	0\\
516.01	0\\
517.01	0\\
518.01	0\\
519.01	0\\
520.01	0\\
521.01	0\\
522.01	0\\
523.01	0\\
524.01	0\\
525.01	0\\
526.01	0\\
527.01	0\\
528.01	0\\
529.01	0\\
530.01	0\\
531.01	0\\
532.01	0\\
533.01	0\\
534.01	0\\
535.01	0\\
536.01	0\\
537.01	0\\
538.01	0\\
539.01	0\\
540.01	0\\
541.01	0\\
542.01	0\\
543.01	0\\
544.01	0\\
545.01	0\\
546.01	0\\
547.01	0\\
548.01	0\\
549.01	0\\
550.01	0\\
551.01	0\\
552.01	0\\
553.01	0\\
554.01	0\\
555.01	0\\
556.01	0\\
557.01	0\\
558.01	0\\
559.01	0\\
560.01	0\\
561.01	0\\
562.01	0\\
563.01	0\\
564.01	0\\
565.01	0\\
566.01	0\\
567.01	0\\
568.01	0\\
569.01	0\\
570.01	0\\
571.01	0\\
572.01	0\\
573.01	0\\
574.01	0\\
575.01	0\\
576.01	0\\
577.01	0\\
578.01	0\\
579.01	0\\
580.01	0\\
581.01	0\\
582.01	0\\
583.01	0\\
584.01	0\\
585.01	0\\
586.01	0\\
587.01	0\\
588.01	0\\
589.01	0\\
590.01	0\\
591.01	0\\
592.01	0\\
593.01	0\\
594.01	0\\
595.01	0\\
596.01	0\\
597.01	0\\
598.01	0\\
599.01	0.00375813090589555\\
599.02	0.00379585558753093\\
599.03	0.00383394724372792\\
599.04	0.00387240947894213\\
599.05	0.00391124593303989\\
599.06	0.00395046028164608\\
599.07	0.0039900562364955\\
599.08	0.0040300375457875\\
599.09	0.00407040799454436\\
599.1	0.00411117140497293\\
599.11	0.00415233163682999\\
599.12	0.00419389258779109\\
599.13	0.00423585819382318\\
599.14	0.00427823242956064\\
599.15	0.00432101930868529\\
599.16	0.00436422288430989\\
599.17	0.00440784724936557\\
599.18	0.004451896536993\\
599.19	0.00449637492093741\\
599.2	0.00454128661594752\\
599.21	0.00458663587817829\\
599.22	0.00463242700559778\\
599.23	0.00467866433839786\\
599.24	0.00472535225940909\\
599.25	0.00477249519451952\\
599.26	0.00482009761309778\\
599.27	0.00486816402842016\\
599.28	0.00491669899810202\\
599.29	0.00496570712453336\\
599.3	0.00501519305531867\\
599.31	0.00506516148372115\\
599.32	0.00511561714911123\\
599.33	0.0051665648374196\\
599.34	0.00521800938159454\\
599.35	0.00526995566206385\\
599.36	0.00532240860720133\\
599.37	0.00537537319379777\\
599.38	0.0054288544475366\\
599.39	0.00548285743949962\\
599.4	0.00553738727623371\\
599.41	0.00559244911427138\\
599.42	0.00564804816062001\\
599.43	0.00570418967325595\\
599.44	0.00576087896162341\\
599.45	0.00581812138713825\\
599.46	0.00587592236369672\\
599.47	0.0059342873581891\\
599.48	0.00599322189101851\\
599.49	0.00605273153662467\\
599.5	0.00611282192401282\\
599.51	0.00617349873728788\\
599.52	0.00623476771619373\\
599.53	0.00629663465665787\\
599.54	0.00635910541134135\\
599.55	0.00642218589019411\\
599.56	0.00648588206101574\\
599.57	0.00655019995002173\\
599.58	0.00661514564241528\\
599.59	0.00668072528296469\\
599.6	0.00674694507658642\\
599.61	0.00681381128893382\\
599.62	0.00688133024699166\\
599.63	0.00694950833967647\\
599.64	0.00701835201844275\\
599.65	0.00708786779789511\\
599.66	0.00715806225640642\\
599.67	0.00722894203674198\\
599.68	0.00730051384668983\\
599.69	0.00737278445969722\\
599.7	0.00744576071551332\\
599.71	0.00751944952083818\\
599.72	0.00759385784997806\\
599.73	0.00766899274550726\\
599.74	0.00774486131893626\\
599.75	0.00782147075138655\\
599.76	0.00789882829427199\\
599.77	0.00797694126998688\\
599.78	0.00805581707260071\\
599.79	0.00813546316855976\\
599.8	0.00821588709739554\\
599.81	0.0082970964724402\\
599.82	0.0083790989815489\\
599.83	0.00846190238782927\\
599.84	0.008545514530378\\
599.85	0.00862994332502466\\
599.86	0.00871519676508279\\
599.87	0.00880128292210833\\
599.88	0.00888820994666551\\
599.89	0.00897598606910017\\
599.9	0.00906461960032072\\
599.91	0.00915411893258671\\
599.92	0.00924449254030512\\
599.93	0.00933574898083444\\
599.94	0.00942789689529664\\
599.95	0.00952094500939709\\
599.96	0.00961490213425244\\
599.97	0.00970977716722672\\
599.98	0.00980557909277551\\
599.99	0.00990231698329844\\
600	0.01\\
};
\addplot [color=red!40!mycolor19,solid,forget plot]
  table[row sep=crcr]{%
0.01	0\\
1.01	0\\
2.01	0\\
3.01	0\\
4.01	0\\
5.01	0\\
6.01	0\\
7.01	0\\
8.01	0\\
9.01	0\\
10.01	0\\
11.01	0\\
12.01	0\\
13.01	0\\
14.01	0\\
15.01	0\\
16.01	0\\
17.01	0\\
18.01	0\\
19.01	0\\
20.01	0\\
21.01	0\\
22.01	0\\
23.01	0\\
24.01	0\\
25.01	0\\
26.01	0\\
27.01	0\\
28.01	0\\
29.01	0\\
30.01	0\\
31.01	0\\
32.01	0\\
33.01	0\\
34.01	0\\
35.01	0\\
36.01	0\\
37.01	0\\
38.01	0\\
39.01	0\\
40.01	0\\
41.01	0\\
42.01	0\\
43.01	0\\
44.01	0\\
45.01	0\\
46.01	0\\
47.01	0\\
48.01	0\\
49.01	0\\
50.01	0\\
51.01	0\\
52.01	0\\
53.01	0\\
54.01	0\\
55.01	0\\
56.01	0\\
57.01	0\\
58.01	0\\
59.01	0\\
60.01	0\\
61.01	0\\
62.01	0\\
63.01	0\\
64.01	0\\
65.01	0\\
66.01	0\\
67.01	0\\
68.01	0\\
69.01	0\\
70.01	0\\
71.01	0\\
72.01	0\\
73.01	0\\
74.01	0\\
75.01	0\\
76.01	0\\
77.01	0\\
78.01	0\\
79.01	0\\
80.01	0\\
81.01	0\\
82.01	0\\
83.01	0\\
84.01	0\\
85.01	0\\
86.01	0\\
87.01	0\\
88.01	0\\
89.01	0\\
90.01	0\\
91.01	0\\
92.01	0\\
93.01	0\\
94.01	0\\
95.01	0\\
96.01	0\\
97.01	0\\
98.01	0\\
99.01	0\\
100.01	0\\
101.01	0\\
102.01	0\\
103.01	0\\
104.01	0\\
105.01	0\\
106.01	0\\
107.01	0\\
108.01	0\\
109.01	0\\
110.01	0\\
111.01	0\\
112.01	0\\
113.01	0\\
114.01	0\\
115.01	0\\
116.01	0\\
117.01	0\\
118.01	0\\
119.01	0\\
120.01	0\\
121.01	0\\
122.01	0\\
123.01	0\\
124.01	0\\
125.01	0\\
126.01	0\\
127.01	0\\
128.01	0\\
129.01	0\\
130.01	0\\
131.01	0\\
132.01	0\\
133.01	0\\
134.01	0\\
135.01	0\\
136.01	0\\
137.01	0\\
138.01	0\\
139.01	0\\
140.01	0\\
141.01	0\\
142.01	0\\
143.01	0\\
144.01	0\\
145.01	0\\
146.01	0\\
147.01	0\\
148.01	0\\
149.01	0\\
150.01	0\\
151.01	0\\
152.01	0\\
153.01	0\\
154.01	0\\
155.01	0\\
156.01	0\\
157.01	0\\
158.01	0\\
159.01	0\\
160.01	0\\
161.01	0\\
162.01	0\\
163.01	0\\
164.01	0\\
165.01	0\\
166.01	0\\
167.01	0\\
168.01	0\\
169.01	0\\
170.01	0\\
171.01	0\\
172.01	0\\
173.01	0\\
174.01	0\\
175.01	0\\
176.01	0\\
177.01	0\\
178.01	0\\
179.01	0\\
180.01	0\\
181.01	0\\
182.01	0\\
183.01	0\\
184.01	0\\
185.01	0\\
186.01	0\\
187.01	0\\
188.01	0\\
189.01	0\\
190.01	0\\
191.01	0\\
192.01	0\\
193.01	0\\
194.01	0\\
195.01	0\\
196.01	0\\
197.01	0\\
198.01	0\\
199.01	0\\
200.01	0\\
201.01	0\\
202.01	0\\
203.01	0\\
204.01	0\\
205.01	0\\
206.01	0\\
207.01	0\\
208.01	0\\
209.01	0\\
210.01	0\\
211.01	0\\
212.01	0\\
213.01	0\\
214.01	0\\
215.01	0\\
216.01	0\\
217.01	0\\
218.01	0\\
219.01	0\\
220.01	0\\
221.01	0\\
222.01	0\\
223.01	0\\
224.01	0\\
225.01	0\\
226.01	0\\
227.01	0\\
228.01	0\\
229.01	0\\
230.01	0\\
231.01	0\\
232.01	0\\
233.01	0\\
234.01	0\\
235.01	0\\
236.01	0\\
237.01	0\\
238.01	0\\
239.01	0\\
240.01	0\\
241.01	0\\
242.01	0\\
243.01	0\\
244.01	0\\
245.01	0\\
246.01	0\\
247.01	0\\
248.01	0\\
249.01	0\\
250.01	0\\
251.01	0\\
252.01	0\\
253.01	0\\
254.01	0\\
255.01	0\\
256.01	0\\
257.01	0\\
258.01	0\\
259.01	0\\
260.01	0\\
261.01	0\\
262.01	0\\
263.01	0\\
264.01	0\\
265.01	0\\
266.01	0\\
267.01	0\\
268.01	0\\
269.01	0\\
270.01	0\\
271.01	0\\
272.01	0\\
273.01	0\\
274.01	0\\
275.01	0\\
276.01	0\\
277.01	0\\
278.01	0\\
279.01	0\\
280.01	0\\
281.01	0\\
282.01	0\\
283.01	0\\
284.01	0\\
285.01	0\\
286.01	0\\
287.01	0\\
288.01	0\\
289.01	0\\
290.01	0\\
291.01	0\\
292.01	0\\
293.01	0\\
294.01	0\\
295.01	0\\
296.01	0\\
297.01	0\\
298.01	0\\
299.01	0\\
300.01	0\\
301.01	0\\
302.01	0\\
303.01	0\\
304.01	0\\
305.01	0\\
306.01	0\\
307.01	0\\
308.01	0\\
309.01	0\\
310.01	0\\
311.01	0\\
312.01	0\\
313.01	0\\
314.01	0\\
315.01	0\\
316.01	0\\
317.01	0\\
318.01	0\\
319.01	0\\
320.01	0\\
321.01	0\\
322.01	0\\
323.01	0\\
324.01	0\\
325.01	0\\
326.01	0\\
327.01	0\\
328.01	0\\
329.01	0\\
330.01	0\\
331.01	0\\
332.01	0\\
333.01	0\\
334.01	0\\
335.01	0\\
336.01	0\\
337.01	0\\
338.01	0\\
339.01	0\\
340.01	0\\
341.01	0\\
342.01	0\\
343.01	0\\
344.01	0\\
345.01	0\\
346.01	0\\
347.01	0\\
348.01	0\\
349.01	0\\
350.01	0\\
351.01	0\\
352.01	0\\
353.01	0\\
354.01	0\\
355.01	0\\
356.01	0\\
357.01	0\\
358.01	0\\
359.01	0\\
360.01	0\\
361.01	0\\
362.01	0\\
363.01	0\\
364.01	0\\
365.01	0\\
366.01	0\\
367.01	0\\
368.01	0\\
369.01	0\\
370.01	0\\
371.01	0\\
372.01	0\\
373.01	0\\
374.01	0\\
375.01	0\\
376.01	0\\
377.01	0\\
378.01	0\\
379.01	0\\
380.01	0\\
381.01	0\\
382.01	0\\
383.01	0\\
384.01	0\\
385.01	0\\
386.01	0\\
387.01	0\\
388.01	0\\
389.01	0\\
390.01	0\\
391.01	0\\
392.01	0\\
393.01	0\\
394.01	0\\
395.01	0\\
396.01	0\\
397.01	0\\
398.01	0\\
399.01	0\\
400.01	0\\
401.01	0\\
402.01	0\\
403.01	0\\
404.01	0\\
405.01	0\\
406.01	0\\
407.01	0\\
408.01	0\\
409.01	0\\
410.01	0\\
411.01	0\\
412.01	0\\
413.01	0\\
414.01	0\\
415.01	0\\
416.01	0\\
417.01	0\\
418.01	0\\
419.01	0\\
420.01	0\\
421.01	0\\
422.01	0\\
423.01	0\\
424.01	0\\
425.01	0\\
426.01	0\\
427.01	0\\
428.01	0\\
429.01	0\\
430.01	0\\
431.01	0\\
432.01	0\\
433.01	0\\
434.01	0\\
435.01	0\\
436.01	0\\
437.01	0\\
438.01	0\\
439.01	0\\
440.01	0\\
441.01	0\\
442.01	0\\
443.01	0\\
444.01	0\\
445.01	0\\
446.01	0\\
447.01	0\\
448.01	0\\
449.01	0\\
450.01	0\\
451.01	0\\
452.01	0\\
453.01	0\\
454.01	0\\
455.01	0\\
456.01	0\\
457.01	0\\
458.01	0\\
459.01	0\\
460.01	0\\
461.01	0\\
462.01	0\\
463.01	0\\
464.01	0\\
465.01	0\\
466.01	0\\
467.01	0\\
468.01	0\\
469.01	0\\
470.01	0\\
471.01	0\\
472.01	0\\
473.01	0\\
474.01	0\\
475.01	0\\
476.01	0\\
477.01	0\\
478.01	0\\
479.01	0\\
480.01	0\\
481.01	0\\
482.01	0\\
483.01	0\\
484.01	0\\
485.01	0\\
486.01	0\\
487.01	0\\
488.01	0\\
489.01	0\\
490.01	0\\
491.01	0\\
492.01	0\\
493.01	0\\
494.01	0\\
495.01	0\\
496.01	0\\
497.01	0\\
498.01	0\\
499.01	0\\
500.01	0\\
501.01	0\\
502.01	0\\
503.01	0\\
504.01	0\\
505.01	0\\
506.01	0\\
507.01	0\\
508.01	0\\
509.01	0\\
510.01	0\\
511.01	0\\
512.01	0\\
513.01	0\\
514.01	0\\
515.01	0\\
516.01	0\\
517.01	0\\
518.01	0\\
519.01	0\\
520.01	0\\
521.01	0\\
522.01	0\\
523.01	0\\
524.01	0\\
525.01	0\\
526.01	0\\
527.01	0\\
528.01	0\\
529.01	0\\
530.01	0\\
531.01	0\\
532.01	0\\
533.01	0\\
534.01	0\\
535.01	0\\
536.01	0\\
537.01	0\\
538.01	0\\
539.01	0\\
540.01	0\\
541.01	0\\
542.01	0\\
543.01	0\\
544.01	0\\
545.01	0\\
546.01	0\\
547.01	0\\
548.01	0\\
549.01	0\\
550.01	0\\
551.01	0\\
552.01	0\\
553.01	0\\
554.01	0\\
555.01	0\\
556.01	0\\
557.01	0\\
558.01	0\\
559.01	0\\
560.01	0\\
561.01	0\\
562.01	0\\
563.01	0\\
564.01	0\\
565.01	0\\
566.01	0\\
567.01	0\\
568.01	0\\
569.01	0\\
570.01	0\\
571.01	0\\
572.01	0\\
573.01	0\\
574.01	0\\
575.01	0\\
576.01	0\\
577.01	0\\
578.01	0\\
579.01	0\\
580.01	0\\
581.01	0\\
582.01	0\\
583.01	0\\
584.01	0\\
585.01	0\\
586.01	0\\
587.01	0\\
588.01	0\\
589.01	0\\
590.01	0\\
591.01	0\\
592.01	0\\
593.01	0\\
594.01	0\\
595.01	0\\
596.01	0\\
597.01	0\\
598.01	0\\
599.01	0.00375813090594476\\
599.02	0.00379585558757729\\
599.03	0.00383394724377162\\
599.04	0.00387240947898333\\
599.05	0.00391124593307875\\
599.06	0.00395046028168274\\
599.07	0.00399005623653004\\
599.08	0.00403003754582007\\
599.09	0.00407040799457505\\
599.1	0.00411117140500182\\
599.11	0.00415233163685717\\
599.12	0.00419389258781667\\
599.13	0.00423585819384723\\
599.14	0.00427823242958322\\
599.15	0.00432101930870647\\
599.16	0.00436422288432972\\
599.17	0.00440784724938413\\
599.18	0.00445189653701036\\
599.19	0.00449637492095364\\
599.2	0.00454128661596266\\
599.21	0.00458663587819241\\
599.22	0.00463242700561092\\
599.23	0.00467866433841008\\
599.24	0.00472535225942043\\
599.25	0.00477249519453003\\
599.26	0.0048200976131075\\
599.27	0.00486816402842916\\
599.28	0.00491669899811033\\
599.29	0.004965707124541\\
599.3	0.0050151930553257\\
599.31	0.0050651614837276\\
599.32	0.00511561714911714\\
599.33	0.005166564837425\\
599.34	0.00521800938159945\\
599.35	0.00526995566206832\\
599.36	0.00532240860720539\\
599.37	0.00537537319380144\\
599.38	0.00542885444753992\\
599.39	0.0054828574395026\\
599.4	0.00553738727623639\\
599.41	0.00559244911427378\\
599.42	0.00564804816062216\\
599.43	0.00570418967325787\\
599.44	0.00576087896162512\\
599.45	0.00581812138713977\\
599.46	0.00587592236369806\\
599.47	0.00593428735819028\\
599.48	0.00599322189101955\\
599.49	0.00605273153662558\\
599.5	0.00611282192401361\\
599.51	0.00617349873728856\\
599.52	0.00623476771619432\\
599.53	0.00629663465665838\\
599.54	0.00635910541134179\\
599.55	0.00642218589019449\\
599.56	0.00648588206101606\\
599.57	0.006550199950022\\
599.58	0.00661514564241551\\
599.59	0.00668072528296489\\
599.6	0.00674694507658658\\
599.61	0.00681381128893395\\
599.62	0.00688133024699177\\
599.63	0.00694950833967656\\
599.64	0.00701835201844282\\
599.65	0.00708786779789516\\
599.66	0.00715806225640646\\
599.67	0.00722894203674201\\
599.68	0.00730051384668985\\
599.69	0.00737278445969725\\
599.7	0.00744576071551335\\
599.71	0.0075194495208382\\
599.72	0.00759385784997808\\
599.73	0.00766899274550727\\
599.74	0.00774486131893627\\
599.75	0.00782147075138656\\
599.76	0.00789882829427201\\
599.77	0.0079769412699869\\
599.78	0.00805581707260072\\
599.79	0.00813546316855976\\
599.8	0.00821588709739554\\
599.81	0.00829709647244021\\
599.82	0.00837909898154891\\
599.83	0.00846190238782927\\
599.84	0.008545514530378\\
599.85	0.00862994332502466\\
599.86	0.00871519676508279\\
599.87	0.00880128292210833\\
599.88	0.00888820994666551\\
599.89	0.00897598606910018\\
599.9	0.00906461960032072\\
599.91	0.00915411893258671\\
599.92	0.00924449254030512\\
599.93	0.00933574898083444\\
599.94	0.00942789689529664\\
599.95	0.00952094500939709\\
599.96	0.00961490213425244\\
599.97	0.00970977716722672\\
599.98	0.00980557909277551\\
599.99	0.00990231698329844\\
600	0.01\\
};
\addplot [color=red!75!mycolor17,solid,forget plot]
  table[row sep=crcr]{%
0.01	0\\
1.01	0\\
2.01	0\\
3.01	0\\
4.01	0\\
5.01	0\\
6.01	0\\
7.01	0\\
8.01	0\\
9.01	0\\
10.01	0\\
11.01	0\\
12.01	0\\
13.01	0\\
14.01	0\\
15.01	0\\
16.01	0\\
17.01	0\\
18.01	0\\
19.01	0\\
20.01	0\\
21.01	0\\
22.01	0\\
23.01	0\\
24.01	0\\
25.01	0\\
26.01	0\\
27.01	0\\
28.01	0\\
29.01	0\\
30.01	0\\
31.01	0\\
32.01	0\\
33.01	0\\
34.01	0\\
35.01	0\\
36.01	0\\
37.01	0\\
38.01	0\\
39.01	0\\
40.01	0\\
41.01	0\\
42.01	0\\
43.01	0\\
44.01	0\\
45.01	0\\
46.01	0\\
47.01	0\\
48.01	0\\
49.01	0\\
50.01	0\\
51.01	0\\
52.01	0\\
53.01	0\\
54.01	0\\
55.01	0\\
56.01	0\\
57.01	0\\
58.01	0\\
59.01	0\\
60.01	0\\
61.01	0\\
62.01	0\\
63.01	0\\
64.01	0\\
65.01	0\\
66.01	0\\
67.01	0\\
68.01	0\\
69.01	0\\
70.01	0\\
71.01	0\\
72.01	0\\
73.01	0\\
74.01	0\\
75.01	0\\
76.01	0\\
77.01	0\\
78.01	0\\
79.01	0\\
80.01	0\\
81.01	0\\
82.01	0\\
83.01	0\\
84.01	0\\
85.01	0\\
86.01	0\\
87.01	0\\
88.01	0\\
89.01	0\\
90.01	0\\
91.01	0\\
92.01	0\\
93.01	0\\
94.01	0\\
95.01	0\\
96.01	0\\
97.01	0\\
98.01	0\\
99.01	0\\
100.01	0\\
101.01	0\\
102.01	0\\
103.01	0\\
104.01	0\\
105.01	0\\
106.01	0\\
107.01	0\\
108.01	0\\
109.01	0\\
110.01	0\\
111.01	0\\
112.01	0\\
113.01	0\\
114.01	0\\
115.01	0\\
116.01	0\\
117.01	0\\
118.01	0\\
119.01	0\\
120.01	0\\
121.01	0\\
122.01	0\\
123.01	0\\
124.01	0\\
125.01	0\\
126.01	0\\
127.01	0\\
128.01	0\\
129.01	0\\
130.01	0\\
131.01	0\\
132.01	0\\
133.01	0\\
134.01	0\\
135.01	0\\
136.01	0\\
137.01	0\\
138.01	0\\
139.01	0\\
140.01	0\\
141.01	0\\
142.01	0\\
143.01	0\\
144.01	0\\
145.01	0\\
146.01	0\\
147.01	0\\
148.01	0\\
149.01	0\\
150.01	0\\
151.01	0\\
152.01	0\\
153.01	0\\
154.01	0\\
155.01	0\\
156.01	0\\
157.01	0\\
158.01	0\\
159.01	0\\
160.01	0\\
161.01	0\\
162.01	0\\
163.01	0\\
164.01	0\\
165.01	0\\
166.01	0\\
167.01	0\\
168.01	0\\
169.01	0\\
170.01	0\\
171.01	0\\
172.01	0\\
173.01	0\\
174.01	0\\
175.01	0\\
176.01	0\\
177.01	0\\
178.01	0\\
179.01	0\\
180.01	0\\
181.01	0\\
182.01	0\\
183.01	0\\
184.01	0\\
185.01	0\\
186.01	0\\
187.01	0\\
188.01	0\\
189.01	0\\
190.01	0\\
191.01	0\\
192.01	0\\
193.01	0\\
194.01	0\\
195.01	0\\
196.01	0\\
197.01	0\\
198.01	0\\
199.01	0\\
200.01	0\\
201.01	0\\
202.01	0\\
203.01	0\\
204.01	0\\
205.01	0\\
206.01	0\\
207.01	0\\
208.01	0\\
209.01	0\\
210.01	0\\
211.01	0\\
212.01	0\\
213.01	0\\
214.01	0\\
215.01	0\\
216.01	0\\
217.01	0\\
218.01	0\\
219.01	0\\
220.01	0\\
221.01	0\\
222.01	0\\
223.01	0\\
224.01	0\\
225.01	0\\
226.01	0\\
227.01	0\\
228.01	0\\
229.01	0\\
230.01	0\\
231.01	0\\
232.01	0\\
233.01	0\\
234.01	0\\
235.01	0\\
236.01	0\\
237.01	0\\
238.01	0\\
239.01	0\\
240.01	0\\
241.01	0\\
242.01	0\\
243.01	0\\
244.01	0\\
245.01	0\\
246.01	0\\
247.01	0\\
248.01	0\\
249.01	0\\
250.01	0\\
251.01	0\\
252.01	0\\
253.01	0\\
254.01	0\\
255.01	0\\
256.01	0\\
257.01	0\\
258.01	0\\
259.01	0\\
260.01	0\\
261.01	0\\
262.01	0\\
263.01	0\\
264.01	0\\
265.01	0\\
266.01	0\\
267.01	0\\
268.01	0\\
269.01	0\\
270.01	0\\
271.01	0\\
272.01	0\\
273.01	0\\
274.01	0\\
275.01	0\\
276.01	0\\
277.01	0\\
278.01	0\\
279.01	0\\
280.01	0\\
281.01	0\\
282.01	0\\
283.01	0\\
284.01	0\\
285.01	0\\
286.01	0\\
287.01	0\\
288.01	0\\
289.01	0\\
290.01	0\\
291.01	0\\
292.01	0\\
293.01	0\\
294.01	0\\
295.01	0\\
296.01	0\\
297.01	0\\
298.01	0\\
299.01	0\\
300.01	0\\
301.01	0\\
302.01	0\\
303.01	0\\
304.01	0\\
305.01	0\\
306.01	0\\
307.01	0\\
308.01	0\\
309.01	0\\
310.01	0\\
311.01	0\\
312.01	0\\
313.01	0\\
314.01	0\\
315.01	0\\
316.01	0\\
317.01	0\\
318.01	0\\
319.01	0\\
320.01	0\\
321.01	0\\
322.01	0\\
323.01	0\\
324.01	0\\
325.01	0\\
326.01	0\\
327.01	0\\
328.01	0\\
329.01	0\\
330.01	0\\
331.01	0\\
332.01	0\\
333.01	0\\
334.01	0\\
335.01	0\\
336.01	0\\
337.01	0\\
338.01	0\\
339.01	0\\
340.01	0\\
341.01	0\\
342.01	0\\
343.01	0\\
344.01	0\\
345.01	0\\
346.01	0\\
347.01	0\\
348.01	0\\
349.01	0\\
350.01	0\\
351.01	0\\
352.01	0\\
353.01	0\\
354.01	0\\
355.01	0\\
356.01	0\\
357.01	0\\
358.01	0\\
359.01	0\\
360.01	0\\
361.01	0\\
362.01	0\\
363.01	0\\
364.01	0\\
365.01	0\\
366.01	0\\
367.01	0\\
368.01	0\\
369.01	0\\
370.01	0\\
371.01	0\\
372.01	0\\
373.01	0\\
374.01	0\\
375.01	0\\
376.01	0\\
377.01	0\\
378.01	0\\
379.01	0\\
380.01	0\\
381.01	0\\
382.01	0\\
383.01	0\\
384.01	0\\
385.01	0\\
386.01	0\\
387.01	0\\
388.01	0\\
389.01	0\\
390.01	0\\
391.01	0\\
392.01	0\\
393.01	0\\
394.01	0\\
395.01	0\\
396.01	0\\
397.01	0\\
398.01	0\\
399.01	0\\
400.01	0\\
401.01	0\\
402.01	0\\
403.01	0\\
404.01	0\\
405.01	0\\
406.01	0\\
407.01	0\\
408.01	0\\
409.01	0\\
410.01	0\\
411.01	0\\
412.01	0\\
413.01	0\\
414.01	0\\
415.01	0\\
416.01	0\\
417.01	0\\
418.01	0\\
419.01	0\\
420.01	0\\
421.01	0\\
422.01	0\\
423.01	0\\
424.01	0\\
425.01	0\\
426.01	0\\
427.01	0\\
428.01	0\\
429.01	0\\
430.01	0\\
431.01	0\\
432.01	0\\
433.01	0\\
434.01	0\\
435.01	0\\
436.01	0\\
437.01	0\\
438.01	0\\
439.01	0\\
440.01	0\\
441.01	0\\
442.01	0\\
443.01	0\\
444.01	0\\
445.01	0\\
446.01	0\\
447.01	0\\
448.01	0\\
449.01	0\\
450.01	0\\
451.01	0\\
452.01	0\\
453.01	0\\
454.01	0\\
455.01	0\\
456.01	0\\
457.01	0\\
458.01	0\\
459.01	0\\
460.01	0\\
461.01	0\\
462.01	0\\
463.01	0\\
464.01	0\\
465.01	0\\
466.01	0\\
467.01	0\\
468.01	0\\
469.01	0\\
470.01	0\\
471.01	0\\
472.01	0\\
473.01	0\\
474.01	0\\
475.01	0\\
476.01	0\\
477.01	0\\
478.01	0\\
479.01	0\\
480.01	0\\
481.01	0\\
482.01	0\\
483.01	0\\
484.01	0\\
485.01	0\\
486.01	0\\
487.01	0\\
488.01	0\\
489.01	0\\
490.01	0\\
491.01	0\\
492.01	0\\
493.01	0\\
494.01	0\\
495.01	0\\
496.01	0\\
497.01	0\\
498.01	0\\
499.01	0\\
500.01	0\\
501.01	0\\
502.01	0\\
503.01	0\\
504.01	0\\
505.01	0\\
506.01	0\\
507.01	0\\
508.01	0\\
509.01	0\\
510.01	0\\
511.01	0\\
512.01	0\\
513.01	0\\
514.01	0\\
515.01	0\\
516.01	0\\
517.01	0\\
518.01	0\\
519.01	0\\
520.01	0\\
521.01	0\\
522.01	0\\
523.01	0\\
524.01	0\\
525.01	0\\
526.01	0\\
527.01	0\\
528.01	0\\
529.01	0\\
530.01	0\\
531.01	0\\
532.01	0\\
533.01	0\\
534.01	0\\
535.01	0\\
536.01	0\\
537.01	0\\
538.01	0\\
539.01	0\\
540.01	0\\
541.01	0\\
542.01	0\\
543.01	0\\
544.01	0\\
545.01	0\\
546.01	0\\
547.01	0\\
548.01	0\\
549.01	0\\
550.01	0\\
551.01	0\\
552.01	0\\
553.01	0\\
554.01	0\\
555.01	0\\
556.01	0\\
557.01	0\\
558.01	0\\
559.01	0\\
560.01	0\\
561.01	0\\
562.01	0\\
563.01	0\\
564.01	0\\
565.01	0\\
566.01	0\\
567.01	0\\
568.01	0\\
569.01	0\\
570.01	0\\
571.01	0\\
572.01	0\\
573.01	0\\
574.01	0\\
575.01	0\\
576.01	0\\
577.01	0\\
578.01	0\\
579.01	0\\
580.01	0\\
581.01	0\\
582.01	0\\
583.01	0\\
584.01	0\\
585.01	0\\
586.01	0\\
587.01	0\\
588.01	0\\
589.01	0\\
590.01	0\\
591.01	0\\
592.01	0\\
593.01	0\\
594.01	0\\
595.01	0\\
596.01	0\\
597.01	0\\
598.01	0\\
599.01	0.0037581309059459\\
599.02	0.00379585558757836\\
599.03	0.00383394724377263\\
599.04	0.00387240947898429\\
599.05	0.00391124593307965\\
599.06	0.00395046028168357\\
599.07	0.00399005623653083\\
599.08	0.00403003754582079\\
599.09	0.00407040799457573\\
599.1	0.00411117140500247\\
599.11	0.00415233163685778\\
599.12	0.00419389258781724\\
599.13	0.00423585819384774\\
599.14	0.00427823242958369\\
599.15	0.0043210193087069\\
599.16	0.00436422288433014\\
599.17	0.00440784724938452\\
599.18	0.00445189653701071\\
599.19	0.00449637492095395\\
599.2	0.00454128661596294\\
599.21	0.00458663587819265\\
599.22	0.00463242700561113\\
599.23	0.00467866433841027\\
599.24	0.0047253522594206\\
599.25	0.00477249519453019\\
599.26	0.00482009761310764\\
599.27	0.00486816402842927\\
599.28	0.00491669899811042\\
599.29	0.0049657071245411\\
599.3	0.00501519305532579\\
599.31	0.00506516148372766\\
599.32	0.00511561714911721\\
599.33	0.00516656483742505\\
599.34	0.00521800938159949\\
599.35	0.00526995566206836\\
599.36	0.00532240860720542\\
599.37	0.00537537319380147\\
599.38	0.00542885444753993\\
599.39	0.00548285743950262\\
599.4	0.00553738727623639\\
599.41	0.00559244911427378\\
599.42	0.00564804816062215\\
599.43	0.00570418967325786\\
599.44	0.00576087896162511\\
599.45	0.00581812138713976\\
599.46	0.00587592236369804\\
599.47	0.00593428735819027\\
599.48	0.00599322189101953\\
599.49	0.00605273153662556\\
599.5	0.00611282192401359\\
599.51	0.00617349873728854\\
599.52	0.0062347677161943\\
599.53	0.00629663465665836\\
599.54	0.00635910541134178\\
599.55	0.00642218589019448\\
599.56	0.00648588206101605\\
599.57	0.00655019995002199\\
599.58	0.0066151456424155\\
599.59	0.00668072528296487\\
599.6	0.00674694507658657\\
599.61	0.00681381128893395\\
599.62	0.00688133024699177\\
599.63	0.00694950833967656\\
599.64	0.00701835201844282\\
599.65	0.00708786779789517\\
599.66	0.00715806225640647\\
599.67	0.00722894203674201\\
599.68	0.00730051384668985\\
599.69	0.00737278445969725\\
599.7	0.00744576071551335\\
599.71	0.0075194495208382\\
599.72	0.00759385784997808\\
599.73	0.00766899274550728\\
599.74	0.00774486131893627\\
599.75	0.00782147075138656\\
599.76	0.00789882829427201\\
599.77	0.0079769412699869\\
599.78	0.00805581707260072\\
599.79	0.00813546316855976\\
599.8	0.00821588709739554\\
599.81	0.00829709647244021\\
599.82	0.00837909898154891\\
599.83	0.00846190238782927\\
599.84	0.008545514530378\\
599.85	0.00862994332502466\\
599.86	0.00871519676508279\\
599.87	0.00880128292210834\\
599.88	0.00888820994666552\\
599.89	0.00897598606910018\\
599.9	0.00906461960032073\\
599.91	0.00915411893258672\\
599.92	0.00924449254030512\\
599.93	0.00933574898083444\\
599.94	0.00942789689529664\\
599.95	0.00952094500939709\\
599.96	0.00961490213425244\\
599.97	0.00970977716722672\\
599.98	0.00980557909277551\\
599.99	0.00990231698329844\\
600	0.01\\
};
\addplot [color=red!80!mycolor19,solid,forget plot]
  table[row sep=crcr]{%
0.01	0\\
1.01	0\\
2.01	0\\
3.01	0\\
4.01	0\\
5.01	0\\
6.01	0\\
7.01	0\\
8.01	0\\
9.01	0\\
10.01	0\\
11.01	0\\
12.01	0\\
13.01	0\\
14.01	0\\
15.01	0\\
16.01	0\\
17.01	0\\
18.01	0\\
19.01	0\\
20.01	0\\
21.01	0\\
22.01	0\\
23.01	0\\
24.01	0\\
25.01	0\\
26.01	0\\
27.01	0\\
28.01	0\\
29.01	0\\
30.01	0\\
31.01	0\\
32.01	0\\
33.01	0\\
34.01	0\\
35.01	0\\
36.01	0\\
37.01	0\\
38.01	0\\
39.01	0\\
40.01	0\\
41.01	0\\
42.01	0\\
43.01	0\\
44.01	0\\
45.01	0\\
46.01	0\\
47.01	0\\
48.01	0\\
49.01	0\\
50.01	0\\
51.01	0\\
52.01	0\\
53.01	0\\
54.01	0\\
55.01	0\\
56.01	0\\
57.01	0\\
58.01	0\\
59.01	0\\
60.01	0\\
61.01	0\\
62.01	0\\
63.01	0\\
64.01	0\\
65.01	0\\
66.01	0\\
67.01	0\\
68.01	0\\
69.01	0\\
70.01	0\\
71.01	0\\
72.01	0\\
73.01	0\\
74.01	0\\
75.01	0\\
76.01	0\\
77.01	0\\
78.01	0\\
79.01	0\\
80.01	0\\
81.01	0\\
82.01	0\\
83.01	0\\
84.01	0\\
85.01	0\\
86.01	0\\
87.01	0\\
88.01	0\\
89.01	0\\
90.01	0\\
91.01	0\\
92.01	0\\
93.01	0\\
94.01	0\\
95.01	0\\
96.01	0\\
97.01	0\\
98.01	0\\
99.01	0\\
100.01	0\\
101.01	0\\
102.01	0\\
103.01	0\\
104.01	0\\
105.01	0\\
106.01	0\\
107.01	0\\
108.01	0\\
109.01	0\\
110.01	0\\
111.01	0\\
112.01	0\\
113.01	0\\
114.01	0\\
115.01	0\\
116.01	0\\
117.01	0\\
118.01	0\\
119.01	0\\
120.01	0\\
121.01	0\\
122.01	0\\
123.01	0\\
124.01	0\\
125.01	0\\
126.01	0\\
127.01	0\\
128.01	0\\
129.01	0\\
130.01	0\\
131.01	0\\
132.01	0\\
133.01	0\\
134.01	0\\
135.01	0\\
136.01	0\\
137.01	0\\
138.01	0\\
139.01	0\\
140.01	0\\
141.01	0\\
142.01	0\\
143.01	0\\
144.01	0\\
145.01	0\\
146.01	0\\
147.01	0\\
148.01	0\\
149.01	0\\
150.01	0\\
151.01	0\\
152.01	0\\
153.01	0\\
154.01	0\\
155.01	0\\
156.01	0\\
157.01	0\\
158.01	0\\
159.01	0\\
160.01	0\\
161.01	0\\
162.01	0\\
163.01	0\\
164.01	0\\
165.01	0\\
166.01	0\\
167.01	0\\
168.01	0\\
169.01	0\\
170.01	0\\
171.01	0\\
172.01	0\\
173.01	0\\
174.01	0\\
175.01	0\\
176.01	0\\
177.01	0\\
178.01	0\\
179.01	0\\
180.01	0\\
181.01	0\\
182.01	0\\
183.01	0\\
184.01	0\\
185.01	0\\
186.01	0\\
187.01	0\\
188.01	0\\
189.01	0\\
190.01	0\\
191.01	0\\
192.01	0\\
193.01	0\\
194.01	0\\
195.01	0\\
196.01	0\\
197.01	0\\
198.01	0\\
199.01	0\\
200.01	0\\
201.01	0\\
202.01	0\\
203.01	0\\
204.01	0\\
205.01	0\\
206.01	0\\
207.01	0\\
208.01	0\\
209.01	0\\
210.01	0\\
211.01	0\\
212.01	0\\
213.01	0\\
214.01	0\\
215.01	0\\
216.01	0\\
217.01	0\\
218.01	0\\
219.01	0\\
220.01	0\\
221.01	0\\
222.01	0\\
223.01	0\\
224.01	0\\
225.01	0\\
226.01	0\\
227.01	0\\
228.01	0\\
229.01	0\\
230.01	0\\
231.01	0\\
232.01	0\\
233.01	0\\
234.01	0\\
235.01	0\\
236.01	0\\
237.01	0\\
238.01	0\\
239.01	0\\
240.01	0\\
241.01	0\\
242.01	0\\
243.01	0\\
244.01	0\\
245.01	0\\
246.01	0\\
247.01	0\\
248.01	0\\
249.01	0\\
250.01	0\\
251.01	0\\
252.01	0\\
253.01	0\\
254.01	0\\
255.01	0\\
256.01	0\\
257.01	0\\
258.01	0\\
259.01	0\\
260.01	0\\
261.01	0\\
262.01	0\\
263.01	0\\
264.01	0\\
265.01	0\\
266.01	0\\
267.01	0\\
268.01	0\\
269.01	0\\
270.01	0\\
271.01	0\\
272.01	0\\
273.01	0\\
274.01	0\\
275.01	0\\
276.01	0\\
277.01	0\\
278.01	0\\
279.01	0\\
280.01	0\\
281.01	0\\
282.01	0\\
283.01	0\\
284.01	0\\
285.01	0\\
286.01	0\\
287.01	0\\
288.01	0\\
289.01	0\\
290.01	0\\
291.01	0\\
292.01	0\\
293.01	0\\
294.01	0\\
295.01	0\\
296.01	0\\
297.01	0\\
298.01	0\\
299.01	0\\
300.01	0\\
301.01	0\\
302.01	0\\
303.01	0\\
304.01	0\\
305.01	0\\
306.01	0\\
307.01	0\\
308.01	0\\
309.01	0\\
310.01	0\\
311.01	0\\
312.01	0\\
313.01	0\\
314.01	0\\
315.01	0\\
316.01	0\\
317.01	0\\
318.01	0\\
319.01	0\\
320.01	0\\
321.01	0\\
322.01	0\\
323.01	0\\
324.01	0\\
325.01	0\\
326.01	0\\
327.01	0\\
328.01	0\\
329.01	0\\
330.01	0\\
331.01	0\\
332.01	0\\
333.01	0\\
334.01	0\\
335.01	0\\
336.01	0\\
337.01	0\\
338.01	0\\
339.01	0\\
340.01	0\\
341.01	0\\
342.01	0\\
343.01	0\\
344.01	0\\
345.01	0\\
346.01	0\\
347.01	0\\
348.01	0\\
349.01	0\\
350.01	0\\
351.01	0\\
352.01	0\\
353.01	0\\
354.01	0\\
355.01	0\\
356.01	0\\
357.01	0\\
358.01	0\\
359.01	0\\
360.01	0\\
361.01	0\\
362.01	0\\
363.01	0\\
364.01	0\\
365.01	0\\
366.01	0\\
367.01	0\\
368.01	0\\
369.01	0\\
370.01	0\\
371.01	0\\
372.01	0\\
373.01	0\\
374.01	0\\
375.01	0\\
376.01	0\\
377.01	0\\
378.01	0\\
379.01	0\\
380.01	0\\
381.01	0\\
382.01	0\\
383.01	0\\
384.01	0\\
385.01	0\\
386.01	0\\
387.01	0\\
388.01	0\\
389.01	0\\
390.01	0\\
391.01	0\\
392.01	0\\
393.01	0\\
394.01	0\\
395.01	0\\
396.01	0\\
397.01	0\\
398.01	0\\
399.01	0\\
400.01	0\\
401.01	0\\
402.01	0\\
403.01	0\\
404.01	0\\
405.01	0\\
406.01	0\\
407.01	0\\
408.01	0\\
409.01	0\\
410.01	0\\
411.01	0\\
412.01	0\\
413.01	0\\
414.01	0\\
415.01	0\\
416.01	0\\
417.01	0\\
418.01	0\\
419.01	0\\
420.01	0\\
421.01	0\\
422.01	0\\
423.01	0\\
424.01	0\\
425.01	0\\
426.01	0\\
427.01	0\\
428.01	0\\
429.01	0\\
430.01	0\\
431.01	0\\
432.01	0\\
433.01	0\\
434.01	0\\
435.01	0\\
436.01	0\\
437.01	0\\
438.01	0\\
439.01	0\\
440.01	0\\
441.01	0\\
442.01	0\\
443.01	0\\
444.01	0\\
445.01	0\\
446.01	0\\
447.01	0\\
448.01	0\\
449.01	0\\
450.01	0\\
451.01	0\\
452.01	0\\
453.01	0\\
454.01	0\\
455.01	0\\
456.01	0\\
457.01	0\\
458.01	0\\
459.01	0\\
460.01	0\\
461.01	0\\
462.01	0\\
463.01	0\\
464.01	0\\
465.01	0\\
466.01	0\\
467.01	0\\
468.01	0\\
469.01	0\\
470.01	0\\
471.01	0\\
472.01	0\\
473.01	0\\
474.01	0\\
475.01	0\\
476.01	0\\
477.01	0\\
478.01	0\\
479.01	0\\
480.01	0\\
481.01	0\\
482.01	0\\
483.01	0\\
484.01	0\\
485.01	0\\
486.01	0\\
487.01	0\\
488.01	0\\
489.01	0\\
490.01	0\\
491.01	0\\
492.01	0\\
493.01	0\\
494.01	0\\
495.01	0\\
496.01	0\\
497.01	0\\
498.01	0\\
499.01	0\\
500.01	0\\
501.01	0\\
502.01	0\\
503.01	0\\
504.01	0\\
505.01	0\\
506.01	0\\
507.01	0\\
508.01	0\\
509.01	0\\
510.01	0\\
511.01	0\\
512.01	0\\
513.01	0\\
514.01	0\\
515.01	0\\
516.01	0\\
517.01	0\\
518.01	0\\
519.01	0\\
520.01	0\\
521.01	0\\
522.01	0\\
523.01	0\\
524.01	0\\
525.01	0\\
526.01	0\\
527.01	0\\
528.01	0\\
529.01	0\\
530.01	0\\
531.01	0\\
532.01	0\\
533.01	0\\
534.01	0\\
535.01	0\\
536.01	0\\
537.01	0\\
538.01	0\\
539.01	0\\
540.01	0\\
541.01	0\\
542.01	0\\
543.01	0\\
544.01	0\\
545.01	0\\
546.01	0\\
547.01	0\\
548.01	0\\
549.01	0\\
550.01	0\\
551.01	0\\
552.01	0\\
553.01	0\\
554.01	0\\
555.01	0\\
556.01	0\\
557.01	0\\
558.01	0\\
559.01	0\\
560.01	0\\
561.01	0\\
562.01	0\\
563.01	0\\
564.01	0\\
565.01	0\\
566.01	0\\
567.01	0\\
568.01	0\\
569.01	0\\
570.01	0\\
571.01	0\\
572.01	0\\
573.01	0\\
574.01	0\\
575.01	0\\
576.01	0\\
577.01	0\\
578.01	0\\
579.01	0\\
580.01	0\\
581.01	0\\
582.01	0\\
583.01	0\\
584.01	0\\
585.01	0\\
586.01	0\\
587.01	0\\
588.01	0\\
589.01	0\\
590.01	0\\
591.01	0\\
592.01	0\\
593.01	0\\
594.01	0\\
595.01	0\\
596.01	0\\
597.01	0\\
598.01	0\\
599.01	0.0037581309059459\\
599.02	0.00379585558757838\\
599.03	0.00383394724377265\\
599.04	0.0038724094789843\\
599.05	0.00391124593307965\\
599.06	0.00395046028168358\\
599.07	0.00399005623653084\\
599.08	0.00403003754582082\\
599.09	0.00407040799457575\\
599.1	0.00411117140500249\\
599.11	0.0041523316368578\\
599.12	0.00419389258781726\\
599.13	0.00423585819384777\\
599.14	0.00427823242958374\\
599.15	0.00432101930870694\\
599.16	0.00436422288433017\\
599.17	0.00440784724938455\\
599.18	0.00445189653701075\\
599.19	0.004496374920954\\
599.2	0.00454128661596299\\
599.21	0.00458663587819272\\
599.22	0.00463242700561121\\
599.23	0.00467866433841035\\
599.24	0.00472535225942068\\
599.25	0.00477249519453027\\
599.26	0.00482009761310773\\
599.27	0.00486816402842935\\
599.28	0.00491669899811049\\
599.29	0.00496570712454115\\
599.3	0.00501519305532583\\
599.31	0.00506516148372772\\
599.32	0.00511561714911725\\
599.33	0.0051665648374251\\
599.34	0.00521800938159954\\
599.35	0.00526995566206839\\
599.36	0.00532240860720545\\
599.37	0.00537537319380151\\
599.38	0.00542885444753997\\
599.39	0.00548285743950264\\
599.4	0.00553738727623643\\
599.41	0.00559244911427381\\
599.42	0.00564804816062218\\
599.43	0.00570418967325789\\
599.44	0.00576087896162513\\
599.45	0.00581812138713978\\
599.46	0.00587592236369807\\
599.47	0.00593428735819029\\
599.48	0.00599322189101956\\
599.49	0.00605273153662558\\
599.5	0.00611282192401362\\
599.51	0.00617349873728857\\
599.52	0.00623476771619433\\
599.53	0.00629663465665839\\
599.54	0.0063591054113418\\
599.55	0.0064221858901945\\
599.56	0.00648588206101607\\
599.57	0.00655019995002201\\
599.58	0.00661514564241552\\
599.59	0.0066807252829649\\
599.6	0.0067469450765866\\
599.61	0.00681381128893396\\
599.62	0.00688133024699178\\
599.63	0.00694950833967657\\
599.64	0.00701835201844284\\
599.65	0.00708786779789518\\
599.66	0.00715806225640647\\
599.67	0.00722894203674202\\
599.68	0.00730051384668987\\
599.69	0.00737278445969725\\
599.7	0.00744576071551335\\
599.71	0.0075194495208382\\
599.72	0.00759385784997808\\
599.73	0.00766899274550727\\
599.74	0.00774486131893627\\
599.75	0.00782147075138656\\
599.76	0.007898828294272\\
599.77	0.00797694126998689\\
599.78	0.00805581707260072\\
599.79	0.00813546316855976\\
599.8	0.00821588709739554\\
599.81	0.00829709647244021\\
599.82	0.00837909898154891\\
599.83	0.00846190238782927\\
599.84	0.008545514530378\\
599.85	0.00862994332502466\\
599.86	0.00871519676508279\\
599.87	0.00880128292210834\\
599.88	0.00888820994666552\\
599.89	0.00897598606910018\\
599.9	0.00906461960032073\\
599.91	0.00915411893258671\\
599.92	0.00924449254030512\\
599.93	0.00933574898083444\\
599.94	0.00942789689529664\\
599.95	0.00952094500939709\\
599.96	0.00961490213425245\\
599.97	0.00970977716722672\\
599.98	0.00980557909277551\\
599.99	0.00990231698329844\\
600	0.01\\
};
\addplot [color=red,solid,forget plot]
  table[row sep=crcr]{%
0.01	0\\
1.01	0\\
2.01	0\\
3.01	0\\
4.01	0\\
5.01	0\\
6.01	0\\
7.01	0\\
8.01	0\\
9.01	0\\
10.01	0\\
11.01	0\\
12.01	0\\
13.01	0\\
14.01	0\\
15.01	0\\
16.01	0\\
17.01	0\\
18.01	0\\
19.01	0\\
20.01	0\\
21.01	0\\
22.01	0\\
23.01	0\\
24.01	0\\
25.01	0\\
26.01	0\\
27.01	0\\
28.01	0\\
29.01	0\\
30.01	0\\
31.01	0\\
32.01	0\\
33.01	0\\
34.01	0\\
35.01	0\\
36.01	0\\
37.01	0\\
38.01	0\\
39.01	0\\
40.01	0\\
41.01	0\\
42.01	0\\
43.01	0\\
44.01	0\\
45.01	0\\
46.01	0\\
47.01	0\\
48.01	0\\
49.01	0\\
50.01	0\\
51.01	0\\
52.01	0\\
53.01	0\\
54.01	0\\
55.01	0\\
56.01	0\\
57.01	0\\
58.01	0\\
59.01	0\\
60.01	0\\
61.01	0\\
62.01	0\\
63.01	0\\
64.01	0\\
65.01	0\\
66.01	0\\
67.01	0\\
68.01	0\\
69.01	0\\
70.01	0\\
71.01	0\\
72.01	0\\
73.01	0\\
74.01	0\\
75.01	0\\
76.01	0\\
77.01	0\\
78.01	0\\
79.01	0\\
80.01	0\\
81.01	0\\
82.01	0\\
83.01	0\\
84.01	0\\
85.01	0\\
86.01	0\\
87.01	0\\
88.01	0\\
89.01	0\\
90.01	0\\
91.01	0\\
92.01	0\\
93.01	0\\
94.01	0\\
95.01	0\\
96.01	0\\
97.01	0\\
98.01	0\\
99.01	0\\
100.01	0\\
101.01	0\\
102.01	0\\
103.01	0\\
104.01	0\\
105.01	0\\
106.01	0\\
107.01	0\\
108.01	0\\
109.01	0\\
110.01	0\\
111.01	0\\
112.01	0\\
113.01	0\\
114.01	0\\
115.01	0\\
116.01	0\\
117.01	0\\
118.01	0\\
119.01	0\\
120.01	0\\
121.01	0\\
122.01	0\\
123.01	0\\
124.01	0\\
125.01	0\\
126.01	0\\
127.01	0\\
128.01	0\\
129.01	0\\
130.01	0\\
131.01	0\\
132.01	0\\
133.01	0\\
134.01	0\\
135.01	0\\
136.01	0\\
137.01	0\\
138.01	0\\
139.01	0\\
140.01	0\\
141.01	0\\
142.01	0\\
143.01	0\\
144.01	0\\
145.01	0\\
146.01	0\\
147.01	0\\
148.01	0\\
149.01	0\\
150.01	0\\
151.01	0\\
152.01	0\\
153.01	0\\
154.01	0\\
155.01	0\\
156.01	0\\
157.01	0\\
158.01	0\\
159.01	0\\
160.01	0\\
161.01	0\\
162.01	0\\
163.01	0\\
164.01	0\\
165.01	0\\
166.01	0\\
167.01	0\\
168.01	0\\
169.01	0\\
170.01	0\\
171.01	0\\
172.01	0\\
173.01	0\\
174.01	0\\
175.01	0\\
176.01	0\\
177.01	0\\
178.01	0\\
179.01	0\\
180.01	0\\
181.01	0\\
182.01	0\\
183.01	0\\
184.01	0\\
185.01	0\\
186.01	0\\
187.01	0\\
188.01	0\\
189.01	0\\
190.01	0\\
191.01	0\\
192.01	0\\
193.01	0\\
194.01	0\\
195.01	0\\
196.01	0\\
197.01	0\\
198.01	0\\
199.01	0\\
200.01	0\\
201.01	0\\
202.01	0\\
203.01	0\\
204.01	0\\
205.01	0\\
206.01	0\\
207.01	0\\
208.01	0\\
209.01	0\\
210.01	0\\
211.01	0\\
212.01	0\\
213.01	0\\
214.01	0\\
215.01	0\\
216.01	0\\
217.01	0\\
218.01	0\\
219.01	0\\
220.01	0\\
221.01	0\\
222.01	0\\
223.01	0\\
224.01	0\\
225.01	0\\
226.01	0\\
227.01	0\\
228.01	0\\
229.01	0\\
230.01	0\\
231.01	0\\
232.01	0\\
233.01	0\\
234.01	0\\
235.01	0\\
236.01	0\\
237.01	0\\
238.01	0\\
239.01	0\\
240.01	0\\
241.01	0\\
242.01	0\\
243.01	0\\
244.01	0\\
245.01	0\\
246.01	0\\
247.01	0\\
248.01	0\\
249.01	0\\
250.01	0\\
251.01	0\\
252.01	0\\
253.01	0\\
254.01	0\\
255.01	0\\
256.01	0\\
257.01	0\\
258.01	0\\
259.01	0\\
260.01	0\\
261.01	0\\
262.01	0\\
263.01	0\\
264.01	0\\
265.01	0\\
266.01	0\\
267.01	0\\
268.01	0\\
269.01	0\\
270.01	0\\
271.01	0\\
272.01	0\\
273.01	0\\
274.01	0\\
275.01	0\\
276.01	0\\
277.01	0\\
278.01	0\\
279.01	0\\
280.01	0\\
281.01	0\\
282.01	0\\
283.01	0\\
284.01	0\\
285.01	0\\
286.01	0\\
287.01	0\\
288.01	0\\
289.01	0\\
290.01	0\\
291.01	0\\
292.01	0\\
293.01	0\\
294.01	0\\
295.01	0\\
296.01	0\\
297.01	0\\
298.01	0\\
299.01	0\\
300.01	0\\
301.01	0\\
302.01	0\\
303.01	0\\
304.01	0\\
305.01	0\\
306.01	0\\
307.01	0\\
308.01	0\\
309.01	0\\
310.01	0\\
311.01	0\\
312.01	0\\
313.01	0\\
314.01	0\\
315.01	0\\
316.01	0\\
317.01	0\\
318.01	0\\
319.01	0\\
320.01	0\\
321.01	0\\
322.01	0\\
323.01	0\\
324.01	0\\
325.01	0\\
326.01	0\\
327.01	0\\
328.01	0\\
329.01	0\\
330.01	0\\
331.01	0\\
332.01	0\\
333.01	0\\
334.01	0\\
335.01	0\\
336.01	0\\
337.01	0\\
338.01	0\\
339.01	0\\
340.01	0\\
341.01	0\\
342.01	0\\
343.01	0\\
344.01	0\\
345.01	0\\
346.01	0\\
347.01	0\\
348.01	0\\
349.01	0\\
350.01	0\\
351.01	0\\
352.01	0\\
353.01	0\\
354.01	0\\
355.01	0\\
356.01	0\\
357.01	0\\
358.01	0\\
359.01	0\\
360.01	0\\
361.01	0\\
362.01	0\\
363.01	0\\
364.01	0\\
365.01	0\\
366.01	0\\
367.01	0\\
368.01	0\\
369.01	0\\
370.01	0\\
371.01	0\\
372.01	0\\
373.01	0\\
374.01	0\\
375.01	0\\
376.01	0\\
377.01	0\\
378.01	0\\
379.01	0\\
380.01	0\\
381.01	0\\
382.01	0\\
383.01	0\\
384.01	0\\
385.01	0\\
386.01	0\\
387.01	0\\
388.01	0\\
389.01	0\\
390.01	0\\
391.01	0\\
392.01	0\\
393.01	0\\
394.01	0\\
395.01	0\\
396.01	0\\
397.01	0\\
398.01	0\\
399.01	0\\
400.01	0\\
401.01	0\\
402.01	0\\
403.01	0\\
404.01	0\\
405.01	0\\
406.01	0\\
407.01	0\\
408.01	0\\
409.01	0\\
410.01	0\\
411.01	0\\
412.01	0\\
413.01	0\\
414.01	0\\
415.01	0\\
416.01	0\\
417.01	0\\
418.01	0\\
419.01	0\\
420.01	0\\
421.01	0\\
422.01	0\\
423.01	0\\
424.01	0\\
425.01	0\\
426.01	0\\
427.01	0\\
428.01	0\\
429.01	0\\
430.01	0\\
431.01	0\\
432.01	0\\
433.01	0\\
434.01	0\\
435.01	0\\
436.01	0\\
437.01	0\\
438.01	0\\
439.01	0\\
440.01	0\\
441.01	0\\
442.01	0\\
443.01	0\\
444.01	0\\
445.01	0\\
446.01	0\\
447.01	0\\
448.01	0\\
449.01	0\\
450.01	0\\
451.01	0\\
452.01	0\\
453.01	0\\
454.01	0\\
455.01	0\\
456.01	0\\
457.01	0\\
458.01	0\\
459.01	0\\
460.01	0\\
461.01	0\\
462.01	0\\
463.01	0\\
464.01	0\\
465.01	0\\
466.01	0\\
467.01	0\\
468.01	0\\
469.01	0\\
470.01	0\\
471.01	0\\
472.01	0\\
473.01	0\\
474.01	0\\
475.01	0\\
476.01	0\\
477.01	0\\
478.01	0\\
479.01	0\\
480.01	0\\
481.01	0\\
482.01	0\\
483.01	0\\
484.01	0\\
485.01	0\\
486.01	0\\
487.01	0\\
488.01	0\\
489.01	0\\
490.01	0\\
491.01	0\\
492.01	0\\
493.01	0\\
494.01	0\\
495.01	0\\
496.01	0\\
497.01	0\\
498.01	0\\
499.01	0\\
500.01	0\\
501.01	0\\
502.01	0\\
503.01	0\\
504.01	0\\
505.01	0\\
506.01	0\\
507.01	0\\
508.01	0\\
509.01	0\\
510.01	0\\
511.01	0\\
512.01	0\\
513.01	0\\
514.01	0\\
515.01	0\\
516.01	0\\
517.01	0\\
518.01	0\\
519.01	0\\
520.01	0\\
521.01	0\\
522.01	0\\
523.01	0\\
524.01	0\\
525.01	0\\
526.01	0\\
527.01	0\\
528.01	0\\
529.01	0\\
530.01	0\\
531.01	0\\
532.01	0\\
533.01	0\\
534.01	0\\
535.01	0\\
536.01	0\\
537.01	0\\
538.01	0\\
539.01	0\\
540.01	0\\
541.01	0\\
542.01	0\\
543.01	0\\
544.01	0\\
545.01	0\\
546.01	0\\
547.01	0\\
548.01	0\\
549.01	0\\
550.01	0\\
551.01	0\\
552.01	0\\
553.01	0\\
554.01	0\\
555.01	0\\
556.01	0\\
557.01	0\\
558.01	0\\
559.01	0\\
560.01	0\\
561.01	0\\
562.01	0\\
563.01	0\\
564.01	0\\
565.01	0\\
566.01	0\\
567.01	0\\
568.01	0\\
569.01	0\\
570.01	0\\
571.01	0\\
572.01	0\\
573.01	0\\
574.01	0\\
575.01	0\\
576.01	0\\
577.01	0\\
578.01	0\\
579.01	0\\
580.01	0\\
581.01	0\\
582.01	0\\
583.01	0\\
584.01	0\\
585.01	0\\
586.01	0\\
587.01	0\\
588.01	0\\
589.01	0\\
590.01	0\\
591.01	0\\
592.01	0\\
593.01	0\\
594.01	0\\
595.01	0\\
596.01	0\\
597.01	0\\
598.01	0\\
599.01	0.00375813090594593\\
599.02	0.0037958555875784\\
599.03	0.00383394724377267\\
599.04	0.00387240947898433\\
599.05	0.00391124593307969\\
599.06	0.00395046028168361\\
599.07	0.00399005623653087\\
599.08	0.00403003754582085\\
599.09	0.00407040799457577\\
599.1	0.0041111714050025\\
599.11	0.00415233163685781\\
599.12	0.00419389258781727\\
599.13	0.00423585819384778\\
599.14	0.00427823242958374\\
599.15	0.00432101930870694\\
599.16	0.00436422288433018\\
599.17	0.00440784724938456\\
599.18	0.00445189653701075\\
599.19	0.004496374920954\\
599.2	0.00454128661596299\\
599.21	0.0045866358781927\\
599.22	0.00463242700561119\\
599.23	0.00467866433841031\\
599.24	0.00472535225942064\\
599.25	0.00477249519453023\\
599.26	0.00482009761310769\\
599.27	0.00486816402842932\\
599.28	0.00491669899811048\\
599.29	0.00496570712454115\\
599.3	0.00501519305532583\\
599.31	0.00506516148372771\\
599.32	0.00511561714911725\\
599.33	0.0051665648374251\\
599.34	0.00521800938159955\\
599.35	0.00526995566206842\\
599.36	0.00532240860720548\\
599.37	0.00537537319380152\\
599.38	0.00542885444753999\\
599.39	0.00548285743950266\\
599.4	0.00553738727623644\\
599.41	0.00559244911427383\\
599.42	0.0056480481606222\\
599.43	0.00570418967325791\\
599.44	0.00576087896162515\\
599.45	0.00581812138713979\\
599.46	0.00587592236369808\\
599.47	0.00593428735819029\\
599.48	0.00599322189101956\\
599.49	0.00605273153662558\\
599.5	0.00611282192401361\\
599.51	0.00617349873728858\\
599.52	0.00623476771619433\\
599.53	0.00629663465665838\\
599.54	0.0063591054113418\\
599.55	0.00642218589019449\\
599.56	0.00648588206101606\\
599.57	0.006550199950022\\
599.58	0.00661514564241551\\
599.59	0.00668072528296488\\
599.6	0.00674694507658657\\
599.61	0.00681381128893395\\
599.62	0.00688133024699177\\
599.63	0.00694950833967656\\
599.64	0.00701835201844282\\
599.65	0.00708786779789516\\
599.66	0.00715806225640646\\
599.67	0.00722894203674201\\
599.68	0.00730051384668985\\
599.69	0.00737278445969725\\
599.7	0.00744576071551335\\
599.71	0.0075194495208382\\
599.72	0.00759385784997808\\
599.73	0.00766899274550727\\
599.74	0.00774486131893627\\
599.75	0.00782147075138656\\
599.76	0.007898828294272\\
599.77	0.00797694126998689\\
599.78	0.00805581707260072\\
599.79	0.00813546316855976\\
599.8	0.00821588709739554\\
599.81	0.00829709647244021\\
599.82	0.0083790989815489\\
599.83	0.00846190238782927\\
599.84	0.008545514530378\\
599.85	0.00862994332502466\\
599.86	0.00871519676508279\\
599.87	0.00880128292210834\\
599.88	0.00888820994666552\\
599.89	0.00897598606910018\\
599.9	0.00906461960032072\\
599.91	0.00915411893258671\\
599.92	0.00924449254030512\\
599.93	0.00933574898083444\\
599.94	0.00942789689529664\\
599.95	0.00952094500939709\\
599.96	0.00961490213425244\\
599.97	0.00970977716722672\\
599.98	0.00980557909277551\\
599.99	0.00990231698329844\\
600	0.01\\
};
\addplot [color=mycolor20,solid,forget plot]
  table[row sep=crcr]{%
0.01	0\\
1.01	0\\
2.01	0\\
3.01	0\\
4.01	0\\
5.01	0\\
6.01	0\\
7.01	0\\
8.01	0\\
9.01	0\\
10.01	0\\
11.01	0\\
12.01	0\\
13.01	0\\
14.01	0\\
15.01	0\\
16.01	0\\
17.01	0\\
18.01	0\\
19.01	0\\
20.01	0\\
21.01	0\\
22.01	0\\
23.01	0\\
24.01	0\\
25.01	0\\
26.01	0\\
27.01	0\\
28.01	0\\
29.01	0\\
30.01	0\\
31.01	0\\
32.01	0\\
33.01	0\\
34.01	0\\
35.01	0\\
36.01	0\\
37.01	0\\
38.01	0\\
39.01	0\\
40.01	0\\
41.01	0\\
42.01	0\\
43.01	0\\
44.01	0\\
45.01	0\\
46.01	0\\
47.01	0\\
48.01	0\\
49.01	0\\
50.01	0\\
51.01	0\\
52.01	0\\
53.01	0\\
54.01	0\\
55.01	0\\
56.01	0\\
57.01	0\\
58.01	0\\
59.01	0\\
60.01	0\\
61.01	0\\
62.01	0\\
63.01	0\\
64.01	0\\
65.01	0\\
66.01	0\\
67.01	0\\
68.01	0\\
69.01	0\\
70.01	0\\
71.01	0\\
72.01	0\\
73.01	0\\
74.01	0\\
75.01	0\\
76.01	0\\
77.01	0\\
78.01	0\\
79.01	0\\
80.01	0\\
81.01	0\\
82.01	0\\
83.01	0\\
84.01	0\\
85.01	0\\
86.01	0\\
87.01	0\\
88.01	0\\
89.01	0\\
90.01	0\\
91.01	0\\
92.01	0\\
93.01	0\\
94.01	0\\
95.01	0\\
96.01	0\\
97.01	0\\
98.01	0\\
99.01	0\\
100.01	0\\
101.01	0\\
102.01	0\\
103.01	0\\
104.01	0\\
105.01	0\\
106.01	0\\
107.01	0\\
108.01	0\\
109.01	0\\
110.01	0\\
111.01	0\\
112.01	0\\
113.01	0\\
114.01	0\\
115.01	0\\
116.01	0\\
117.01	0\\
118.01	0\\
119.01	0\\
120.01	0\\
121.01	0\\
122.01	0\\
123.01	0\\
124.01	0\\
125.01	0\\
126.01	0\\
127.01	0\\
128.01	0\\
129.01	0\\
130.01	0\\
131.01	0\\
132.01	0\\
133.01	0\\
134.01	0\\
135.01	0\\
136.01	0\\
137.01	0\\
138.01	0\\
139.01	0\\
140.01	0\\
141.01	0\\
142.01	0\\
143.01	0\\
144.01	0\\
145.01	0\\
146.01	0\\
147.01	0\\
148.01	0\\
149.01	0\\
150.01	0\\
151.01	0\\
152.01	0\\
153.01	0\\
154.01	0\\
155.01	0\\
156.01	0\\
157.01	0\\
158.01	0\\
159.01	0\\
160.01	0\\
161.01	0\\
162.01	0\\
163.01	0\\
164.01	0\\
165.01	0\\
166.01	0\\
167.01	0\\
168.01	0\\
169.01	0\\
170.01	0\\
171.01	0\\
172.01	0\\
173.01	0\\
174.01	0\\
175.01	0\\
176.01	0\\
177.01	0\\
178.01	0\\
179.01	0\\
180.01	0\\
181.01	0\\
182.01	0\\
183.01	0\\
184.01	0\\
185.01	0\\
186.01	0\\
187.01	0\\
188.01	0\\
189.01	0\\
190.01	0\\
191.01	0\\
192.01	0\\
193.01	0\\
194.01	0\\
195.01	0\\
196.01	0\\
197.01	0\\
198.01	0\\
199.01	0\\
200.01	0\\
201.01	0\\
202.01	0\\
203.01	0\\
204.01	0\\
205.01	0\\
206.01	0\\
207.01	0\\
208.01	0\\
209.01	0\\
210.01	0\\
211.01	0\\
212.01	0\\
213.01	0\\
214.01	0\\
215.01	0\\
216.01	0\\
217.01	0\\
218.01	0\\
219.01	0\\
220.01	0\\
221.01	0\\
222.01	0\\
223.01	0\\
224.01	0\\
225.01	0\\
226.01	0\\
227.01	0\\
228.01	0\\
229.01	0\\
230.01	0\\
231.01	0\\
232.01	0\\
233.01	0\\
234.01	0\\
235.01	0\\
236.01	0\\
237.01	0\\
238.01	0\\
239.01	0\\
240.01	0\\
241.01	0\\
242.01	0\\
243.01	0\\
244.01	0\\
245.01	0\\
246.01	0\\
247.01	0\\
248.01	0\\
249.01	0\\
250.01	0\\
251.01	0\\
252.01	0\\
253.01	0\\
254.01	0\\
255.01	0\\
256.01	0\\
257.01	0\\
258.01	0\\
259.01	0\\
260.01	0\\
261.01	0\\
262.01	0\\
263.01	0\\
264.01	0\\
265.01	0\\
266.01	0\\
267.01	0\\
268.01	0\\
269.01	0\\
270.01	0\\
271.01	0\\
272.01	0\\
273.01	0\\
274.01	0\\
275.01	0\\
276.01	0\\
277.01	0\\
278.01	0\\
279.01	0\\
280.01	0\\
281.01	0\\
282.01	0\\
283.01	0\\
284.01	0\\
285.01	0\\
286.01	0\\
287.01	0\\
288.01	0\\
289.01	0\\
290.01	0\\
291.01	0\\
292.01	0\\
293.01	0\\
294.01	0\\
295.01	0\\
296.01	0\\
297.01	0\\
298.01	0\\
299.01	0\\
300.01	0\\
301.01	0\\
302.01	0\\
303.01	0\\
304.01	0\\
305.01	0\\
306.01	0\\
307.01	0\\
308.01	0\\
309.01	0\\
310.01	0\\
311.01	0\\
312.01	0\\
313.01	0\\
314.01	0\\
315.01	0\\
316.01	0\\
317.01	0\\
318.01	0\\
319.01	0\\
320.01	0\\
321.01	0\\
322.01	0\\
323.01	0\\
324.01	0\\
325.01	0\\
326.01	0\\
327.01	0\\
328.01	0\\
329.01	0\\
330.01	0\\
331.01	0\\
332.01	0\\
333.01	0\\
334.01	0\\
335.01	0\\
336.01	0\\
337.01	0\\
338.01	0\\
339.01	0\\
340.01	0\\
341.01	0\\
342.01	0\\
343.01	0\\
344.01	0\\
345.01	0\\
346.01	0\\
347.01	0\\
348.01	0\\
349.01	0\\
350.01	0\\
351.01	0\\
352.01	0\\
353.01	0\\
354.01	0\\
355.01	0\\
356.01	0\\
357.01	0\\
358.01	0\\
359.01	0\\
360.01	0\\
361.01	0\\
362.01	0\\
363.01	0\\
364.01	0\\
365.01	0\\
366.01	0\\
367.01	0\\
368.01	0\\
369.01	0\\
370.01	0\\
371.01	0\\
372.01	0\\
373.01	0\\
374.01	0\\
375.01	0\\
376.01	0\\
377.01	0\\
378.01	0\\
379.01	0\\
380.01	0\\
381.01	0\\
382.01	0\\
383.01	0\\
384.01	0\\
385.01	0\\
386.01	0\\
387.01	0\\
388.01	0\\
389.01	0\\
390.01	0\\
391.01	0\\
392.01	0\\
393.01	0\\
394.01	0\\
395.01	0\\
396.01	0\\
397.01	0\\
398.01	0\\
399.01	0\\
400.01	0\\
401.01	0\\
402.01	0\\
403.01	0\\
404.01	0\\
405.01	0\\
406.01	0\\
407.01	0\\
408.01	0\\
409.01	0\\
410.01	0\\
411.01	0\\
412.01	0\\
413.01	0\\
414.01	0\\
415.01	0\\
416.01	0\\
417.01	0\\
418.01	0\\
419.01	0\\
420.01	0\\
421.01	0\\
422.01	0\\
423.01	0\\
424.01	0\\
425.01	0\\
426.01	0\\
427.01	0\\
428.01	0\\
429.01	0\\
430.01	0\\
431.01	0\\
432.01	0\\
433.01	0\\
434.01	0\\
435.01	0\\
436.01	0\\
437.01	0\\
438.01	0\\
439.01	0\\
440.01	0\\
441.01	0\\
442.01	0\\
443.01	0\\
444.01	0\\
445.01	0\\
446.01	0\\
447.01	0\\
448.01	0\\
449.01	0\\
450.01	0\\
451.01	0\\
452.01	0\\
453.01	0\\
454.01	0\\
455.01	0\\
456.01	0\\
457.01	0\\
458.01	0\\
459.01	0\\
460.01	0\\
461.01	0\\
462.01	0\\
463.01	0\\
464.01	0\\
465.01	0\\
466.01	0\\
467.01	0\\
468.01	0\\
469.01	0\\
470.01	0\\
471.01	0\\
472.01	0\\
473.01	0\\
474.01	0\\
475.01	0\\
476.01	0\\
477.01	0\\
478.01	0\\
479.01	0\\
480.01	0\\
481.01	0\\
482.01	0\\
483.01	0\\
484.01	0\\
485.01	0\\
486.01	0\\
487.01	0\\
488.01	0\\
489.01	0\\
490.01	0\\
491.01	0\\
492.01	0\\
493.01	0\\
494.01	0\\
495.01	0\\
496.01	0\\
497.01	0\\
498.01	0\\
499.01	0\\
500.01	0\\
501.01	0\\
502.01	0\\
503.01	0\\
504.01	0\\
505.01	0\\
506.01	0\\
507.01	0\\
508.01	0\\
509.01	0\\
510.01	0\\
511.01	0\\
512.01	0\\
513.01	0\\
514.01	0\\
515.01	0\\
516.01	0\\
517.01	0\\
518.01	0\\
519.01	0\\
520.01	0\\
521.01	0\\
522.01	0\\
523.01	0\\
524.01	0\\
525.01	0\\
526.01	0\\
527.01	0\\
528.01	0\\
529.01	0\\
530.01	0\\
531.01	0\\
532.01	0\\
533.01	0\\
534.01	0\\
535.01	0\\
536.01	0\\
537.01	0\\
538.01	0\\
539.01	0\\
540.01	0\\
541.01	0\\
542.01	0\\
543.01	0\\
544.01	0\\
545.01	0\\
546.01	0\\
547.01	0\\
548.01	0\\
549.01	0\\
550.01	0\\
551.01	0\\
552.01	0\\
553.01	0\\
554.01	0\\
555.01	0\\
556.01	0\\
557.01	0\\
558.01	0\\
559.01	0\\
560.01	0\\
561.01	0\\
562.01	0\\
563.01	0\\
564.01	0\\
565.01	0\\
566.01	0\\
567.01	0\\
568.01	0\\
569.01	0\\
570.01	0\\
571.01	0\\
572.01	0\\
573.01	0\\
574.01	0\\
575.01	0\\
576.01	0\\
577.01	0\\
578.01	0\\
579.01	0\\
580.01	0\\
581.01	0\\
582.01	0\\
583.01	0\\
584.01	0\\
585.01	0\\
586.01	0\\
587.01	0\\
588.01	0\\
589.01	0\\
590.01	0\\
591.01	0\\
592.01	0\\
593.01	0\\
594.01	0\\
595.01	0\\
596.01	0\\
597.01	0\\
598.01	0\\
599.01	0.0037581309059459\\
599.02	0.00379585558757836\\
599.03	0.00383394724377262\\
599.04	0.00387240947898428\\
599.05	0.00391124593307962\\
599.06	0.00395046028168355\\
599.07	0.00399005623653081\\
599.08	0.00403003754582079\\
599.09	0.00407040799457573\\
599.1	0.00411117140500246\\
599.11	0.00415233163685777\\
599.12	0.00419389258781722\\
599.13	0.00423585819384772\\
599.14	0.00427823242958369\\
599.15	0.0043210193087069\\
599.16	0.00436422288433012\\
599.17	0.00440784724938451\\
599.18	0.00445189653701071\\
599.19	0.00449637492095396\\
599.2	0.00454128661596297\\
599.21	0.00458663587819269\\
599.22	0.00463242700561117\\
599.23	0.00467866433841031\\
599.24	0.00472535225942064\\
599.25	0.00477249519453023\\
599.26	0.00482009761310769\\
599.27	0.00486816402842931\\
599.28	0.00491669899811047\\
599.29	0.00496570712454113\\
599.3	0.00501519305532581\\
599.31	0.0050651614837277\\
599.32	0.00511561714911722\\
599.33	0.00516656483742507\\
599.34	0.00521800938159951\\
599.35	0.00526995566206837\\
599.36	0.00532240860720543\\
599.37	0.00537537319380148\\
599.38	0.00542885444753995\\
599.39	0.00548285743950262\\
599.4	0.0055373872762364\\
599.41	0.00559244911427378\\
599.42	0.00564804816062216\\
599.43	0.00570418967325786\\
599.44	0.00576087896162511\\
599.45	0.00581812138713977\\
599.46	0.00587592236369805\\
599.47	0.00593428735819027\\
599.48	0.00599322189101954\\
599.49	0.00605273153662557\\
599.5	0.00611282192401361\\
599.51	0.00617349873728855\\
599.52	0.00623476771619431\\
599.53	0.00629663465665838\\
599.54	0.00635910541134178\\
599.55	0.00642218589019448\\
599.56	0.00648588206101606\\
599.57	0.006550199950022\\
599.58	0.00661514564241551\\
599.59	0.00668072528296489\\
599.6	0.00674694507658658\\
599.61	0.00681381128893395\\
599.62	0.00688133024699177\\
599.63	0.00694950833967656\\
599.64	0.00701835201844282\\
599.65	0.00708786779789517\\
599.66	0.00715806225640647\\
599.67	0.00722894203674201\\
599.68	0.00730051384668985\\
599.69	0.00737278445969725\\
599.7	0.00744576071551334\\
599.71	0.00751944952083819\\
599.72	0.00759385784997808\\
599.73	0.00766899274550727\\
599.74	0.00774486131893627\\
599.75	0.00782147075138655\\
599.76	0.007898828294272\\
599.77	0.00797694126998689\\
599.78	0.00805581707260072\\
599.79	0.00813546316855976\\
599.8	0.00821588709739554\\
599.81	0.0082970964724402\\
599.82	0.0083790989815489\\
599.83	0.00846190238782927\\
599.84	0.008545514530378\\
599.85	0.00862994332502466\\
599.86	0.00871519676508279\\
599.87	0.00880128292210833\\
599.88	0.00888820994666552\\
599.89	0.00897598606910018\\
599.9	0.00906461960032073\\
599.91	0.00915411893258672\\
599.92	0.00924449254030512\\
599.93	0.00933574898083444\\
599.94	0.00942789689529664\\
599.95	0.00952094500939709\\
599.96	0.00961490213425244\\
599.97	0.00970977716722672\\
599.98	0.00980557909277551\\
599.99	0.00990231698329844\\
600	0.01\\
};
\addplot [color=mycolor21,solid,forget plot]
  table[row sep=crcr]{%
0.01	0\\
1.01	0\\
2.01	0\\
3.01	0\\
4.01	0\\
5.01	0\\
6.01	0\\
7.01	0\\
8.01	0\\
9.01	0\\
10.01	0\\
11.01	0\\
12.01	0\\
13.01	0\\
14.01	0\\
15.01	0\\
16.01	0\\
17.01	0\\
18.01	0\\
19.01	0\\
20.01	0\\
21.01	0\\
22.01	0\\
23.01	0\\
24.01	0\\
25.01	0\\
26.01	0\\
27.01	0\\
28.01	0\\
29.01	0\\
30.01	0\\
31.01	0\\
32.01	0\\
33.01	0\\
34.01	0\\
35.01	0\\
36.01	0\\
37.01	0\\
38.01	0\\
39.01	0\\
40.01	0\\
41.01	0\\
42.01	0\\
43.01	0\\
44.01	0\\
45.01	0\\
46.01	0\\
47.01	0\\
48.01	0\\
49.01	0\\
50.01	0\\
51.01	0\\
52.01	0\\
53.01	0\\
54.01	0\\
55.01	0\\
56.01	0\\
57.01	0\\
58.01	0\\
59.01	0\\
60.01	0\\
61.01	0\\
62.01	0\\
63.01	0\\
64.01	0\\
65.01	0\\
66.01	0\\
67.01	0\\
68.01	0\\
69.01	0\\
70.01	0\\
71.01	0\\
72.01	0\\
73.01	0\\
74.01	0\\
75.01	0\\
76.01	0\\
77.01	0\\
78.01	0\\
79.01	0\\
80.01	0\\
81.01	0\\
82.01	0\\
83.01	0\\
84.01	0\\
85.01	0\\
86.01	0\\
87.01	0\\
88.01	0\\
89.01	0\\
90.01	0\\
91.01	0\\
92.01	0\\
93.01	0\\
94.01	0\\
95.01	0\\
96.01	0\\
97.01	0\\
98.01	0\\
99.01	0\\
100.01	0\\
101.01	0\\
102.01	0\\
103.01	0\\
104.01	0\\
105.01	0\\
106.01	0\\
107.01	0\\
108.01	0\\
109.01	0\\
110.01	0\\
111.01	0\\
112.01	0\\
113.01	0\\
114.01	0\\
115.01	0\\
116.01	0\\
117.01	0\\
118.01	0\\
119.01	0\\
120.01	0\\
121.01	0\\
122.01	0\\
123.01	0\\
124.01	0\\
125.01	0\\
126.01	0\\
127.01	0\\
128.01	0\\
129.01	0\\
130.01	0\\
131.01	0\\
132.01	0\\
133.01	0\\
134.01	0\\
135.01	0\\
136.01	0\\
137.01	0\\
138.01	0\\
139.01	0\\
140.01	0\\
141.01	0\\
142.01	0\\
143.01	0\\
144.01	0\\
145.01	0\\
146.01	0\\
147.01	0\\
148.01	0\\
149.01	0\\
150.01	0\\
151.01	0\\
152.01	0\\
153.01	0\\
154.01	0\\
155.01	0\\
156.01	0\\
157.01	0\\
158.01	0\\
159.01	0\\
160.01	0\\
161.01	0\\
162.01	0\\
163.01	0\\
164.01	0\\
165.01	0\\
166.01	0\\
167.01	0\\
168.01	0\\
169.01	0\\
170.01	0\\
171.01	0\\
172.01	0\\
173.01	0\\
174.01	0\\
175.01	0\\
176.01	0\\
177.01	0\\
178.01	0\\
179.01	0\\
180.01	0\\
181.01	0\\
182.01	0\\
183.01	0\\
184.01	0\\
185.01	0\\
186.01	0\\
187.01	0\\
188.01	0\\
189.01	0\\
190.01	0\\
191.01	0\\
192.01	0\\
193.01	0\\
194.01	0\\
195.01	0\\
196.01	0\\
197.01	0\\
198.01	0\\
199.01	0\\
200.01	0\\
201.01	0\\
202.01	0\\
203.01	0\\
204.01	0\\
205.01	0\\
206.01	0\\
207.01	0\\
208.01	0\\
209.01	0\\
210.01	0\\
211.01	0\\
212.01	0\\
213.01	0\\
214.01	0\\
215.01	0\\
216.01	0\\
217.01	0\\
218.01	0\\
219.01	0\\
220.01	0\\
221.01	0\\
222.01	0\\
223.01	0\\
224.01	0\\
225.01	0\\
226.01	0\\
227.01	0\\
228.01	0\\
229.01	0\\
230.01	0\\
231.01	0\\
232.01	0\\
233.01	0\\
234.01	0\\
235.01	0\\
236.01	0\\
237.01	0\\
238.01	0\\
239.01	0\\
240.01	0\\
241.01	0\\
242.01	0\\
243.01	0\\
244.01	0\\
245.01	0\\
246.01	0\\
247.01	0\\
248.01	0\\
249.01	0\\
250.01	0\\
251.01	0\\
252.01	0\\
253.01	0\\
254.01	0\\
255.01	0\\
256.01	0\\
257.01	0\\
258.01	0\\
259.01	0\\
260.01	0\\
261.01	0\\
262.01	0\\
263.01	0\\
264.01	0\\
265.01	0\\
266.01	0\\
267.01	0\\
268.01	0\\
269.01	0\\
270.01	0\\
271.01	0\\
272.01	0\\
273.01	0\\
274.01	0\\
275.01	0\\
276.01	0\\
277.01	0\\
278.01	0\\
279.01	0\\
280.01	0\\
281.01	0\\
282.01	0\\
283.01	0\\
284.01	0\\
285.01	0\\
286.01	0\\
287.01	0\\
288.01	0\\
289.01	0\\
290.01	0\\
291.01	0\\
292.01	0\\
293.01	0\\
294.01	0\\
295.01	0\\
296.01	0\\
297.01	0\\
298.01	0\\
299.01	0\\
300.01	0\\
301.01	0\\
302.01	0\\
303.01	0\\
304.01	0\\
305.01	0\\
306.01	0\\
307.01	0\\
308.01	0\\
309.01	0\\
310.01	0\\
311.01	0\\
312.01	0\\
313.01	0\\
314.01	0\\
315.01	0\\
316.01	0\\
317.01	0\\
318.01	0\\
319.01	0\\
320.01	0\\
321.01	0\\
322.01	0\\
323.01	0\\
324.01	0\\
325.01	0\\
326.01	0\\
327.01	0\\
328.01	0\\
329.01	0\\
330.01	0\\
331.01	0\\
332.01	0\\
333.01	0\\
334.01	0\\
335.01	0\\
336.01	0\\
337.01	0\\
338.01	0\\
339.01	0\\
340.01	0\\
341.01	0\\
342.01	0\\
343.01	0\\
344.01	0\\
345.01	0\\
346.01	0\\
347.01	0\\
348.01	0\\
349.01	0\\
350.01	0\\
351.01	0\\
352.01	0\\
353.01	0\\
354.01	0\\
355.01	0\\
356.01	0\\
357.01	0\\
358.01	0\\
359.01	0\\
360.01	0\\
361.01	0\\
362.01	0\\
363.01	0\\
364.01	0\\
365.01	0\\
366.01	0\\
367.01	0\\
368.01	0\\
369.01	0\\
370.01	0\\
371.01	0\\
372.01	0\\
373.01	0\\
374.01	0\\
375.01	0\\
376.01	0\\
377.01	0\\
378.01	0\\
379.01	0\\
380.01	0\\
381.01	0\\
382.01	0\\
383.01	0\\
384.01	0\\
385.01	0\\
386.01	0\\
387.01	0\\
388.01	0\\
389.01	0\\
390.01	0\\
391.01	0\\
392.01	0\\
393.01	0\\
394.01	0\\
395.01	0\\
396.01	0\\
397.01	0\\
398.01	0\\
399.01	0\\
400.01	0\\
401.01	0\\
402.01	0\\
403.01	0\\
404.01	0\\
405.01	0\\
406.01	0\\
407.01	0\\
408.01	0\\
409.01	0\\
410.01	0\\
411.01	0\\
412.01	0\\
413.01	0\\
414.01	0\\
415.01	0\\
416.01	0\\
417.01	0\\
418.01	0\\
419.01	0\\
420.01	0\\
421.01	0\\
422.01	0\\
423.01	0\\
424.01	0\\
425.01	0\\
426.01	0\\
427.01	0\\
428.01	0\\
429.01	0\\
430.01	0\\
431.01	0\\
432.01	0\\
433.01	0\\
434.01	0\\
435.01	0\\
436.01	0\\
437.01	0\\
438.01	0\\
439.01	0\\
440.01	0\\
441.01	0\\
442.01	0\\
443.01	0\\
444.01	0\\
445.01	0\\
446.01	0\\
447.01	0\\
448.01	0\\
449.01	0\\
450.01	0\\
451.01	0\\
452.01	0\\
453.01	0\\
454.01	0\\
455.01	0\\
456.01	0\\
457.01	0\\
458.01	0\\
459.01	0\\
460.01	0\\
461.01	0\\
462.01	0\\
463.01	0\\
464.01	0\\
465.01	0\\
466.01	0\\
467.01	0\\
468.01	0\\
469.01	0\\
470.01	0\\
471.01	0\\
472.01	0\\
473.01	0\\
474.01	0\\
475.01	0\\
476.01	0\\
477.01	0\\
478.01	0\\
479.01	0\\
480.01	0\\
481.01	0\\
482.01	0\\
483.01	0\\
484.01	0\\
485.01	0\\
486.01	0\\
487.01	0\\
488.01	0\\
489.01	0\\
490.01	0\\
491.01	0\\
492.01	0\\
493.01	0\\
494.01	0\\
495.01	0\\
496.01	0\\
497.01	0\\
498.01	0\\
499.01	0\\
500.01	0\\
501.01	0\\
502.01	0\\
503.01	0\\
504.01	0\\
505.01	0\\
506.01	0\\
507.01	0\\
508.01	0\\
509.01	0\\
510.01	0\\
511.01	0\\
512.01	0\\
513.01	0\\
514.01	0\\
515.01	0\\
516.01	0\\
517.01	0\\
518.01	0\\
519.01	0\\
520.01	0\\
521.01	0\\
522.01	0\\
523.01	0\\
524.01	0\\
525.01	0\\
526.01	0\\
527.01	0\\
528.01	0\\
529.01	0\\
530.01	0\\
531.01	0\\
532.01	0\\
533.01	0\\
534.01	0\\
535.01	0\\
536.01	0\\
537.01	0\\
538.01	0\\
539.01	0\\
540.01	0\\
541.01	0\\
542.01	0\\
543.01	0\\
544.01	0\\
545.01	0\\
546.01	0\\
547.01	0\\
548.01	0\\
549.01	0\\
550.01	0\\
551.01	0\\
552.01	0\\
553.01	0\\
554.01	0\\
555.01	0\\
556.01	0\\
557.01	0\\
558.01	0\\
559.01	0\\
560.01	0\\
561.01	0\\
562.01	0\\
563.01	0\\
564.01	0\\
565.01	0\\
566.01	0\\
567.01	0\\
568.01	0\\
569.01	0\\
570.01	0\\
571.01	0\\
572.01	0\\
573.01	0\\
574.01	0\\
575.01	0\\
576.01	0\\
577.01	0\\
578.01	0\\
579.01	0\\
580.01	0\\
581.01	0\\
582.01	0\\
583.01	0\\
584.01	0\\
585.01	0\\
586.01	0\\
587.01	0\\
588.01	0\\
589.01	0\\
590.01	0\\
591.01	0\\
592.01	0\\
593.01	0\\
594.01	0\\
595.01	0\\
596.01	0\\
597.01	0\\
598.01	0.000700248929397768\\
599.01	0.00375813090594587\\
599.02	0.00379585558757835\\
599.03	0.00383394724377262\\
599.04	0.00387240947898428\\
599.05	0.00391124593307964\\
599.06	0.00395046028168355\\
599.07	0.00399005623653081\\
599.08	0.00403003754582079\\
599.09	0.00407040799457573\\
599.1	0.00411117140500247\\
599.11	0.00415233163685777\\
599.12	0.00419389258781723\\
599.13	0.00423585819384774\\
599.14	0.00427823242958369\\
599.15	0.00432101930870692\\
599.16	0.00436422288433015\\
599.17	0.00440784724938452\\
599.18	0.00445189653701071\\
599.19	0.00449637492095395\\
599.2	0.00454128661596294\\
599.21	0.00458663587819266\\
599.22	0.00463242700561114\\
599.23	0.00467866433841028\\
599.24	0.00472535225942061\\
599.25	0.0047724951945302\\
599.26	0.00482009761310766\\
599.27	0.00486816402842928\\
599.28	0.00491669899811044\\
599.29	0.00496570712454111\\
599.3	0.00501519305532579\\
599.31	0.00506516148372767\\
599.32	0.00511561714911721\\
599.33	0.00516656483742504\\
599.34	0.0052180093815995\\
599.35	0.00526995566206837\\
599.36	0.00532240860720543\\
599.37	0.00537537319380148\\
599.38	0.00542885444753995\\
599.39	0.00548285743950262\\
599.4	0.0055373872762364\\
599.41	0.00559244911427378\\
599.42	0.00564804816062216\\
599.43	0.00570418967325786\\
599.44	0.00576087896162512\\
599.45	0.00581812138713976\\
599.46	0.00587592236369805\\
599.47	0.00593428735819028\\
599.48	0.00599322189101954\\
599.49	0.00605273153662557\\
599.5	0.00611282192401361\\
599.51	0.00617349873728855\\
599.52	0.0062347677161943\\
599.53	0.00629663465665836\\
599.54	0.00635910541134178\\
599.55	0.00642218589019448\\
599.56	0.00648588206101605\\
599.57	0.006550199950022\\
599.58	0.0066151456424155\\
599.59	0.00668072528296488\\
599.6	0.00674694507658658\\
599.61	0.00681381128893396\\
599.62	0.00688133024699177\\
599.63	0.00694950833967656\\
599.64	0.00701835201844283\\
599.65	0.00708786779789517\\
599.66	0.00715806225640647\\
599.67	0.00722894203674201\\
599.68	0.00730051384668986\\
599.69	0.00737278445969725\\
599.7	0.00744576071551335\\
599.71	0.0075194495208382\\
599.72	0.00759385784997809\\
599.73	0.00766899274550728\\
599.74	0.00774486131893628\\
599.75	0.00782147075138657\\
599.76	0.00789882829427201\\
599.77	0.0079769412699869\\
599.78	0.00805581707260072\\
599.79	0.00813546316855976\\
599.8	0.00821588709739554\\
599.81	0.00829709647244021\\
599.82	0.00837909898154891\\
599.83	0.00846190238782927\\
599.84	0.008545514530378\\
599.85	0.00862994332502466\\
599.86	0.00871519676508279\\
599.87	0.00880128292210834\\
599.88	0.00888820994666552\\
599.89	0.00897598606910018\\
599.9	0.00906461960032072\\
599.91	0.00915411893258671\\
599.92	0.00924449254030512\\
599.93	0.00933574898083444\\
599.94	0.00942789689529664\\
599.95	0.00952094500939709\\
599.96	0.00961490213425245\\
599.97	0.00970977716722672\\
599.98	0.00980557909277551\\
599.99	0.00990231698329844\\
600	0.01\\
};
\addplot [color=black!20!mycolor21,solid,forget plot]
  table[row sep=crcr]{%
0.01	0\\
1.01	0\\
2.01	0\\
3.01	0\\
4.01	0\\
5.01	0\\
6.01	0\\
7.01	0\\
8.01	0\\
9.01	0\\
10.01	0\\
11.01	0\\
12.01	0\\
13.01	0\\
14.01	0\\
15.01	0\\
16.01	0\\
17.01	0\\
18.01	0\\
19.01	0\\
20.01	0\\
21.01	0\\
22.01	0\\
23.01	0\\
24.01	0\\
25.01	0\\
26.01	0\\
27.01	0\\
28.01	0\\
29.01	0\\
30.01	0\\
31.01	0\\
32.01	0\\
33.01	0\\
34.01	0\\
35.01	0\\
36.01	0\\
37.01	0\\
38.01	0\\
39.01	0\\
40.01	0\\
41.01	0\\
42.01	0\\
43.01	0\\
44.01	0\\
45.01	0\\
46.01	0\\
47.01	0\\
48.01	0\\
49.01	0\\
50.01	0\\
51.01	0\\
52.01	0\\
53.01	0\\
54.01	0\\
55.01	0\\
56.01	0\\
57.01	0\\
58.01	0\\
59.01	0\\
60.01	0\\
61.01	0\\
62.01	0\\
63.01	0\\
64.01	0\\
65.01	0\\
66.01	0\\
67.01	0\\
68.01	0\\
69.01	0\\
70.01	0\\
71.01	0\\
72.01	0\\
73.01	0\\
74.01	0\\
75.01	0\\
76.01	0\\
77.01	0\\
78.01	0\\
79.01	0\\
80.01	0\\
81.01	0\\
82.01	0\\
83.01	0\\
84.01	0\\
85.01	0\\
86.01	0\\
87.01	0\\
88.01	0\\
89.01	0\\
90.01	0\\
91.01	0\\
92.01	0\\
93.01	0\\
94.01	0\\
95.01	0\\
96.01	0\\
97.01	0\\
98.01	0\\
99.01	0\\
100.01	0\\
101.01	0\\
102.01	0\\
103.01	0\\
104.01	0\\
105.01	0\\
106.01	0\\
107.01	0\\
108.01	0\\
109.01	0\\
110.01	0\\
111.01	0\\
112.01	0\\
113.01	0\\
114.01	0\\
115.01	0\\
116.01	0\\
117.01	0\\
118.01	0\\
119.01	0\\
120.01	0\\
121.01	0\\
122.01	0\\
123.01	0\\
124.01	0\\
125.01	0\\
126.01	0\\
127.01	0\\
128.01	0\\
129.01	0\\
130.01	0\\
131.01	0\\
132.01	0\\
133.01	0\\
134.01	0\\
135.01	0\\
136.01	0\\
137.01	0\\
138.01	0\\
139.01	0\\
140.01	0\\
141.01	0\\
142.01	0\\
143.01	0\\
144.01	0\\
145.01	0\\
146.01	0\\
147.01	0\\
148.01	0\\
149.01	0\\
150.01	0\\
151.01	0\\
152.01	0\\
153.01	0\\
154.01	0\\
155.01	0\\
156.01	0\\
157.01	0\\
158.01	0\\
159.01	0\\
160.01	0\\
161.01	0\\
162.01	0\\
163.01	0\\
164.01	0\\
165.01	0\\
166.01	0\\
167.01	0\\
168.01	0\\
169.01	0\\
170.01	0\\
171.01	0\\
172.01	0\\
173.01	0\\
174.01	0\\
175.01	0\\
176.01	0\\
177.01	0\\
178.01	0\\
179.01	0\\
180.01	0\\
181.01	0\\
182.01	0\\
183.01	0\\
184.01	0\\
185.01	0\\
186.01	0\\
187.01	0\\
188.01	0\\
189.01	0\\
190.01	0\\
191.01	0\\
192.01	0\\
193.01	0\\
194.01	0\\
195.01	0\\
196.01	0\\
197.01	0\\
198.01	0\\
199.01	0\\
200.01	0\\
201.01	0\\
202.01	0\\
203.01	0\\
204.01	0\\
205.01	0\\
206.01	0\\
207.01	0\\
208.01	0\\
209.01	0\\
210.01	0\\
211.01	0\\
212.01	0\\
213.01	0\\
214.01	0\\
215.01	0\\
216.01	0\\
217.01	0\\
218.01	0\\
219.01	0\\
220.01	0\\
221.01	0\\
222.01	0\\
223.01	0\\
224.01	0\\
225.01	0\\
226.01	0\\
227.01	0\\
228.01	0\\
229.01	0\\
230.01	0\\
231.01	0\\
232.01	0\\
233.01	0\\
234.01	0\\
235.01	0\\
236.01	0\\
237.01	0\\
238.01	0\\
239.01	0\\
240.01	0\\
241.01	0\\
242.01	0\\
243.01	0\\
244.01	0\\
245.01	0\\
246.01	0\\
247.01	0\\
248.01	0\\
249.01	0\\
250.01	0\\
251.01	0\\
252.01	0\\
253.01	0\\
254.01	0\\
255.01	0\\
256.01	0\\
257.01	0\\
258.01	0\\
259.01	0\\
260.01	0\\
261.01	0\\
262.01	0\\
263.01	0\\
264.01	0\\
265.01	0\\
266.01	0\\
267.01	0\\
268.01	0\\
269.01	0\\
270.01	0\\
271.01	0\\
272.01	0\\
273.01	0\\
274.01	0\\
275.01	0\\
276.01	0\\
277.01	0\\
278.01	0\\
279.01	0\\
280.01	0\\
281.01	0\\
282.01	0\\
283.01	0\\
284.01	0\\
285.01	0\\
286.01	0\\
287.01	0\\
288.01	0\\
289.01	0\\
290.01	0\\
291.01	0\\
292.01	0\\
293.01	0\\
294.01	0\\
295.01	0\\
296.01	0\\
297.01	0\\
298.01	0\\
299.01	0\\
300.01	0\\
301.01	0\\
302.01	0\\
303.01	0\\
304.01	0\\
305.01	0\\
306.01	0\\
307.01	0\\
308.01	0\\
309.01	0\\
310.01	0\\
311.01	0\\
312.01	0\\
313.01	0\\
314.01	0\\
315.01	0\\
316.01	0\\
317.01	0\\
318.01	0\\
319.01	0\\
320.01	0\\
321.01	0\\
322.01	0\\
323.01	0\\
324.01	0\\
325.01	0\\
326.01	0\\
327.01	0\\
328.01	0\\
329.01	0\\
330.01	0\\
331.01	0\\
332.01	0\\
333.01	0\\
334.01	0\\
335.01	0\\
336.01	0\\
337.01	0\\
338.01	0\\
339.01	0\\
340.01	0\\
341.01	0\\
342.01	0\\
343.01	0\\
344.01	0\\
345.01	0\\
346.01	0\\
347.01	0\\
348.01	0\\
349.01	0\\
350.01	0\\
351.01	0\\
352.01	0\\
353.01	0\\
354.01	0\\
355.01	0\\
356.01	0\\
357.01	0\\
358.01	0\\
359.01	0\\
360.01	0\\
361.01	0\\
362.01	0\\
363.01	0\\
364.01	0\\
365.01	0\\
366.01	0\\
367.01	0\\
368.01	0\\
369.01	0\\
370.01	0\\
371.01	0\\
372.01	0\\
373.01	0\\
374.01	0\\
375.01	0\\
376.01	0\\
377.01	0\\
378.01	0\\
379.01	0\\
380.01	0\\
381.01	0\\
382.01	0\\
383.01	0\\
384.01	0\\
385.01	0\\
386.01	0\\
387.01	0\\
388.01	0\\
389.01	0\\
390.01	0\\
391.01	0\\
392.01	0\\
393.01	0\\
394.01	0\\
395.01	0\\
396.01	0\\
397.01	0\\
398.01	0\\
399.01	0\\
400.01	0\\
401.01	0\\
402.01	0\\
403.01	0\\
404.01	0\\
405.01	0\\
406.01	0\\
407.01	0\\
408.01	0\\
409.01	0\\
410.01	0\\
411.01	0\\
412.01	0\\
413.01	0\\
414.01	0\\
415.01	0\\
416.01	0\\
417.01	0\\
418.01	0\\
419.01	0\\
420.01	0\\
421.01	0\\
422.01	0\\
423.01	0\\
424.01	0\\
425.01	0\\
426.01	0\\
427.01	0\\
428.01	0\\
429.01	0\\
430.01	0\\
431.01	0\\
432.01	0\\
433.01	0\\
434.01	0\\
435.01	0\\
436.01	0\\
437.01	0\\
438.01	0\\
439.01	0\\
440.01	0\\
441.01	0\\
442.01	0\\
443.01	0\\
444.01	0\\
445.01	0\\
446.01	0\\
447.01	0\\
448.01	0\\
449.01	0\\
450.01	0\\
451.01	0\\
452.01	0\\
453.01	0\\
454.01	0\\
455.01	0\\
456.01	0\\
457.01	0\\
458.01	0\\
459.01	0\\
460.01	0\\
461.01	0\\
462.01	0\\
463.01	0\\
464.01	0\\
465.01	0\\
466.01	0\\
467.01	0\\
468.01	0\\
469.01	0\\
470.01	0\\
471.01	0\\
472.01	0\\
473.01	0\\
474.01	0\\
475.01	0\\
476.01	0\\
477.01	0\\
478.01	0\\
479.01	0\\
480.01	0\\
481.01	0\\
482.01	0\\
483.01	0\\
484.01	0\\
485.01	0\\
486.01	0\\
487.01	0\\
488.01	0\\
489.01	0\\
490.01	0\\
491.01	0\\
492.01	0\\
493.01	0\\
494.01	0\\
495.01	0\\
496.01	0\\
497.01	0\\
498.01	0\\
499.01	0\\
500.01	0\\
501.01	0\\
502.01	0\\
503.01	0\\
504.01	0\\
505.01	0\\
506.01	0\\
507.01	0\\
508.01	0\\
509.01	0\\
510.01	0\\
511.01	0\\
512.01	0\\
513.01	0\\
514.01	0\\
515.01	0\\
516.01	0\\
517.01	0\\
518.01	0\\
519.01	0\\
520.01	0\\
521.01	0\\
522.01	0\\
523.01	0\\
524.01	0\\
525.01	0\\
526.01	0\\
527.01	0\\
528.01	0\\
529.01	0\\
530.01	0\\
531.01	0\\
532.01	0\\
533.01	0\\
534.01	0\\
535.01	0\\
536.01	0\\
537.01	0\\
538.01	0\\
539.01	0\\
540.01	0\\
541.01	0\\
542.01	0\\
543.01	0\\
544.01	0\\
545.01	0\\
546.01	0\\
547.01	0\\
548.01	0\\
549.01	0\\
550.01	0\\
551.01	0\\
552.01	0\\
553.01	0\\
554.01	0\\
555.01	0\\
556.01	0\\
557.01	0\\
558.01	0\\
559.01	0\\
560.01	0\\
561.01	0\\
562.01	0\\
563.01	0\\
564.01	0\\
565.01	0\\
566.01	0\\
567.01	0\\
568.01	0\\
569.01	0\\
570.01	0\\
571.01	0\\
572.01	0\\
573.01	0\\
574.01	0\\
575.01	0\\
576.01	0\\
577.01	0\\
578.01	0\\
579.01	0\\
580.01	0\\
581.01	0\\
582.01	0\\
583.01	0\\
584.01	0\\
585.01	0\\
586.01	0\\
587.01	0\\
588.01	0\\
589.01	0\\
590.01	0\\
591.01	0\\
592.01	0\\
593.01	0\\
594.01	0\\
595.01	0\\
596.01	0\\
597.01	0\\
598.01	0.00134745983720699\\
599.01	0.0037581309059459\\
599.02	0.00379585558757836\\
599.03	0.00383394724377263\\
599.04	0.0038724094789843\\
599.05	0.00391124593307965\\
599.06	0.00395046028168358\\
599.07	0.00399005623653084\\
599.08	0.00403003754582081\\
599.09	0.00407040799457574\\
599.1	0.00411117140500247\\
599.11	0.00415233163685778\\
599.12	0.00419389258781724\\
599.13	0.00423585819384774\\
599.14	0.00427823242958371\\
599.15	0.00432101930870692\\
599.16	0.00436422288433015\\
599.17	0.00440784724938453\\
599.18	0.00445189653701074\\
599.19	0.00449637492095399\\
599.2	0.00454128661596298\\
599.21	0.00458663587819269\\
599.22	0.00463242700561119\\
599.23	0.00467866433841033\\
599.24	0.00472535225942064\\
599.25	0.00477249519453023\\
599.26	0.00482009761310769\\
599.27	0.00486816402842932\\
599.28	0.00491669899811048\\
599.29	0.00496570712454114\\
599.3	0.00501519305532581\\
599.31	0.00506516148372771\\
599.32	0.00511561714911724\\
599.33	0.00516656483742509\\
599.34	0.00521800938159953\\
599.35	0.00526995566206838\\
599.36	0.00532240860720544\\
599.37	0.00537537319380149\\
599.38	0.00542885444753996\\
599.39	0.00548285743950264\\
599.4	0.00553738727623643\\
599.41	0.0055924491142738\\
599.42	0.00564804816062217\\
599.43	0.00570418967325788\\
599.44	0.00576087896162512\\
599.45	0.00581812138713977\\
599.46	0.00587592236369805\\
599.47	0.00593428735819027\\
599.48	0.00599322189101953\\
599.49	0.00605273153662556\\
599.5	0.00611282192401359\\
599.51	0.00617349873728855\\
599.52	0.00623476771619431\\
599.53	0.00629663465665838\\
599.54	0.00635910541134178\\
599.55	0.00642218589019448\\
599.56	0.00648588206101606\\
599.57	0.006550199950022\\
599.58	0.00661514564241551\\
599.59	0.00668072528296489\\
599.6	0.00674694507658658\\
599.61	0.00681381128893396\\
599.62	0.00688133024699177\\
599.63	0.00694950833967656\\
599.64	0.00701835201844282\\
599.65	0.00708786779789517\\
599.66	0.00715806225640647\\
599.67	0.00722894203674201\\
599.68	0.00730051384668986\\
599.69	0.00737278445969725\\
599.7	0.00744576071551335\\
599.71	0.00751944952083819\\
599.72	0.00759385784997808\\
599.73	0.00766899274550727\\
599.74	0.00774486131893627\\
599.75	0.00782147075138655\\
599.76	0.007898828294272\\
599.77	0.00797694126998689\\
599.78	0.00805581707260072\\
599.79	0.00813546316855976\\
599.8	0.00821588709739554\\
599.81	0.00829709647244021\\
599.82	0.00837909898154891\\
599.83	0.00846190238782927\\
599.84	0.008545514530378\\
599.85	0.00862994332502466\\
599.86	0.00871519676508279\\
599.87	0.00880128292210834\\
599.88	0.00888820994666551\\
599.89	0.00897598606910018\\
599.9	0.00906461960032073\\
599.91	0.00915411893258671\\
599.92	0.00924449254030512\\
599.93	0.00933574898083444\\
599.94	0.00942789689529665\\
599.95	0.00952094500939709\\
599.96	0.00961490213425244\\
599.97	0.00970977716722672\\
599.98	0.00980557909277551\\
599.99	0.00990231698329844\\
600	0.01\\
};
\addplot [color=black!50!mycolor20,solid,forget plot]
  table[row sep=crcr]{%
0.01	0\\
1.01	0\\
2.01	0\\
3.01	0\\
4.01	0\\
5.01	0\\
6.01	0\\
7.01	0\\
8.01	0\\
9.01	0\\
10.01	0\\
11.01	0\\
12.01	0\\
13.01	0\\
14.01	0\\
15.01	0\\
16.01	0\\
17.01	0\\
18.01	0\\
19.01	0\\
20.01	0\\
21.01	0\\
22.01	0\\
23.01	0\\
24.01	0\\
25.01	0\\
26.01	0\\
27.01	0\\
28.01	0\\
29.01	0\\
30.01	0\\
31.01	0\\
32.01	0\\
33.01	0\\
34.01	0\\
35.01	0\\
36.01	0\\
37.01	0\\
38.01	0\\
39.01	0\\
40.01	0\\
41.01	0\\
42.01	0\\
43.01	0\\
44.01	0\\
45.01	0\\
46.01	0\\
47.01	0\\
48.01	0\\
49.01	0\\
50.01	0\\
51.01	0\\
52.01	0\\
53.01	0\\
54.01	0\\
55.01	0\\
56.01	0\\
57.01	0\\
58.01	0\\
59.01	0\\
60.01	0\\
61.01	0\\
62.01	0\\
63.01	0\\
64.01	0\\
65.01	0\\
66.01	0\\
67.01	0\\
68.01	0\\
69.01	0\\
70.01	0\\
71.01	0\\
72.01	0\\
73.01	0\\
74.01	0\\
75.01	0\\
76.01	0\\
77.01	0\\
78.01	0\\
79.01	0\\
80.01	0\\
81.01	0\\
82.01	0\\
83.01	0\\
84.01	0\\
85.01	0\\
86.01	0\\
87.01	0\\
88.01	0\\
89.01	0\\
90.01	0\\
91.01	0\\
92.01	0\\
93.01	0\\
94.01	0\\
95.01	0\\
96.01	0\\
97.01	0\\
98.01	0\\
99.01	0\\
100.01	0\\
101.01	0\\
102.01	0\\
103.01	0\\
104.01	0\\
105.01	0\\
106.01	0\\
107.01	0\\
108.01	0\\
109.01	0\\
110.01	0\\
111.01	0\\
112.01	0\\
113.01	0\\
114.01	0\\
115.01	0\\
116.01	0\\
117.01	0\\
118.01	0\\
119.01	0\\
120.01	0\\
121.01	0\\
122.01	0\\
123.01	0\\
124.01	0\\
125.01	0\\
126.01	0\\
127.01	0\\
128.01	0\\
129.01	0\\
130.01	0\\
131.01	0\\
132.01	0\\
133.01	0\\
134.01	0\\
135.01	0\\
136.01	0\\
137.01	0\\
138.01	0\\
139.01	0\\
140.01	0\\
141.01	0\\
142.01	0\\
143.01	0\\
144.01	0\\
145.01	0\\
146.01	0\\
147.01	0\\
148.01	0\\
149.01	0\\
150.01	0\\
151.01	0\\
152.01	0\\
153.01	0\\
154.01	0\\
155.01	0\\
156.01	0\\
157.01	0\\
158.01	0\\
159.01	0\\
160.01	0\\
161.01	0\\
162.01	0\\
163.01	0\\
164.01	0\\
165.01	0\\
166.01	0\\
167.01	0\\
168.01	0\\
169.01	0\\
170.01	0\\
171.01	0\\
172.01	0\\
173.01	0\\
174.01	0\\
175.01	0\\
176.01	0\\
177.01	0\\
178.01	0\\
179.01	0\\
180.01	0\\
181.01	0\\
182.01	0\\
183.01	0\\
184.01	0\\
185.01	0\\
186.01	0\\
187.01	0\\
188.01	0\\
189.01	0\\
190.01	0\\
191.01	0\\
192.01	0\\
193.01	0\\
194.01	0\\
195.01	0\\
196.01	0\\
197.01	0\\
198.01	0\\
199.01	0\\
200.01	0\\
201.01	0\\
202.01	0\\
203.01	0\\
204.01	0\\
205.01	0\\
206.01	0\\
207.01	0\\
208.01	0\\
209.01	0\\
210.01	0\\
211.01	0\\
212.01	0\\
213.01	0\\
214.01	0\\
215.01	0\\
216.01	0\\
217.01	0\\
218.01	0\\
219.01	0\\
220.01	0\\
221.01	0\\
222.01	0\\
223.01	0\\
224.01	0\\
225.01	0\\
226.01	0\\
227.01	0\\
228.01	0\\
229.01	0\\
230.01	0\\
231.01	0\\
232.01	0\\
233.01	0\\
234.01	0\\
235.01	0\\
236.01	0\\
237.01	0\\
238.01	0\\
239.01	0\\
240.01	0\\
241.01	0\\
242.01	0\\
243.01	0\\
244.01	0\\
245.01	0\\
246.01	0\\
247.01	0\\
248.01	0\\
249.01	0\\
250.01	0\\
251.01	0\\
252.01	0\\
253.01	0\\
254.01	0\\
255.01	0\\
256.01	0\\
257.01	0\\
258.01	0\\
259.01	0\\
260.01	0\\
261.01	0\\
262.01	0\\
263.01	0\\
264.01	0\\
265.01	0\\
266.01	0\\
267.01	0\\
268.01	0\\
269.01	0\\
270.01	0\\
271.01	0\\
272.01	0\\
273.01	0\\
274.01	0\\
275.01	0\\
276.01	0\\
277.01	0\\
278.01	0\\
279.01	0\\
280.01	0\\
281.01	0\\
282.01	0\\
283.01	0\\
284.01	0\\
285.01	0\\
286.01	0\\
287.01	0\\
288.01	0\\
289.01	0\\
290.01	0\\
291.01	0\\
292.01	0\\
293.01	0\\
294.01	0\\
295.01	0\\
296.01	0\\
297.01	0\\
298.01	0\\
299.01	0\\
300.01	0\\
301.01	0\\
302.01	0\\
303.01	0\\
304.01	0\\
305.01	0\\
306.01	0\\
307.01	0\\
308.01	0\\
309.01	0\\
310.01	0\\
311.01	0\\
312.01	0\\
313.01	0\\
314.01	0\\
315.01	0\\
316.01	0\\
317.01	0\\
318.01	0\\
319.01	0\\
320.01	0\\
321.01	0\\
322.01	0\\
323.01	0\\
324.01	0\\
325.01	0\\
326.01	0\\
327.01	0\\
328.01	0\\
329.01	0\\
330.01	0\\
331.01	0\\
332.01	0\\
333.01	0\\
334.01	0\\
335.01	0\\
336.01	0\\
337.01	0\\
338.01	0\\
339.01	0\\
340.01	0\\
341.01	0\\
342.01	0\\
343.01	0\\
344.01	0\\
345.01	0\\
346.01	0\\
347.01	0\\
348.01	0\\
349.01	0\\
350.01	0\\
351.01	0\\
352.01	0\\
353.01	0\\
354.01	0\\
355.01	0\\
356.01	0\\
357.01	0\\
358.01	0\\
359.01	0\\
360.01	0\\
361.01	0\\
362.01	0\\
363.01	0\\
364.01	0\\
365.01	0\\
366.01	0\\
367.01	0\\
368.01	0\\
369.01	0\\
370.01	0\\
371.01	0\\
372.01	0\\
373.01	0\\
374.01	0\\
375.01	0\\
376.01	0\\
377.01	0\\
378.01	0\\
379.01	0\\
380.01	0\\
381.01	0\\
382.01	0\\
383.01	0\\
384.01	0\\
385.01	0\\
386.01	0\\
387.01	0\\
388.01	0\\
389.01	0\\
390.01	0\\
391.01	0\\
392.01	0\\
393.01	0\\
394.01	0\\
395.01	0\\
396.01	0\\
397.01	0\\
398.01	0\\
399.01	0\\
400.01	0\\
401.01	0\\
402.01	0\\
403.01	0\\
404.01	0\\
405.01	0\\
406.01	0\\
407.01	0\\
408.01	0\\
409.01	0\\
410.01	0\\
411.01	0\\
412.01	0\\
413.01	0\\
414.01	0\\
415.01	0\\
416.01	0\\
417.01	0\\
418.01	0\\
419.01	0\\
420.01	0\\
421.01	0\\
422.01	0\\
423.01	0\\
424.01	0\\
425.01	0\\
426.01	0\\
427.01	0\\
428.01	0\\
429.01	0\\
430.01	0\\
431.01	0\\
432.01	0\\
433.01	0\\
434.01	0\\
435.01	0\\
436.01	0\\
437.01	0\\
438.01	0\\
439.01	0\\
440.01	0\\
441.01	0\\
442.01	0\\
443.01	0\\
444.01	0\\
445.01	0\\
446.01	0\\
447.01	0\\
448.01	0\\
449.01	0\\
450.01	0\\
451.01	0\\
452.01	0\\
453.01	0\\
454.01	0\\
455.01	0\\
456.01	0\\
457.01	0\\
458.01	0\\
459.01	0\\
460.01	0\\
461.01	0\\
462.01	0\\
463.01	0\\
464.01	0\\
465.01	0\\
466.01	0\\
467.01	0\\
468.01	0\\
469.01	0\\
470.01	0\\
471.01	0\\
472.01	0\\
473.01	0\\
474.01	0\\
475.01	0\\
476.01	0\\
477.01	0\\
478.01	0\\
479.01	0\\
480.01	0\\
481.01	0\\
482.01	0\\
483.01	0\\
484.01	0\\
485.01	0\\
486.01	0\\
487.01	0\\
488.01	0\\
489.01	0\\
490.01	0\\
491.01	0\\
492.01	0\\
493.01	0\\
494.01	0\\
495.01	0\\
496.01	0\\
497.01	0\\
498.01	0\\
499.01	0\\
500.01	0\\
501.01	0\\
502.01	0\\
503.01	0\\
504.01	0\\
505.01	0\\
506.01	0\\
507.01	0\\
508.01	0\\
509.01	0\\
510.01	0\\
511.01	0\\
512.01	0\\
513.01	0\\
514.01	0\\
515.01	0\\
516.01	0\\
517.01	0\\
518.01	0\\
519.01	0\\
520.01	0\\
521.01	0\\
522.01	0\\
523.01	0\\
524.01	0\\
525.01	0\\
526.01	0\\
527.01	0\\
528.01	0\\
529.01	0\\
530.01	0\\
531.01	0\\
532.01	0\\
533.01	0\\
534.01	0\\
535.01	0\\
536.01	0\\
537.01	0\\
538.01	0\\
539.01	0\\
540.01	0\\
541.01	0\\
542.01	0\\
543.01	0\\
544.01	0\\
545.01	0\\
546.01	0\\
547.01	0\\
548.01	0\\
549.01	0\\
550.01	0\\
551.01	0\\
552.01	0\\
553.01	0\\
554.01	0\\
555.01	0\\
556.01	0\\
557.01	0\\
558.01	0\\
559.01	0\\
560.01	0\\
561.01	0\\
562.01	0\\
563.01	0\\
564.01	0\\
565.01	0\\
566.01	0\\
567.01	0\\
568.01	0\\
569.01	0\\
570.01	0\\
571.01	0\\
572.01	0\\
573.01	0\\
574.01	0\\
575.01	0\\
576.01	0\\
577.01	0\\
578.01	0\\
579.01	0\\
580.01	0\\
581.01	0\\
582.01	0\\
583.01	0\\
584.01	0\\
585.01	0\\
586.01	0\\
587.01	0\\
588.01	0\\
589.01	0\\
590.01	0\\
591.01	0\\
592.01	0\\
593.01	0\\
594.01	0\\
595.01	0\\
596.01	0\\
597.01	0\\
598.01	0.00134849457212285\\
599.01	0.0037581309059459\\
599.02	0.00379585558757838\\
599.03	0.00383394724377265\\
599.04	0.0038724094789843\\
599.05	0.00391124593307966\\
599.06	0.0039504602816836\\
599.07	0.00399005623653084\\
599.08	0.00403003754582082\\
599.09	0.00407040799457574\\
599.1	0.00411117140500247\\
599.11	0.00415233163685778\\
599.12	0.00419389258781723\\
599.13	0.00423585819384774\\
599.14	0.00427823242958369\\
599.15	0.0043210193087069\\
599.16	0.00436422288433012\\
599.17	0.00440784724938451\\
599.18	0.00445189653701071\\
599.19	0.00449637492095396\\
599.2	0.00454128661596297\\
599.21	0.00458663587819269\\
599.22	0.00463242700561117\\
599.23	0.00467866433841031\\
599.24	0.00472535225942064\\
599.25	0.00477249519453023\\
599.26	0.00482009761310769\\
599.27	0.00486816402842931\\
599.28	0.00491669899811045\\
599.29	0.00496570712454113\\
599.3	0.00501519305532581\\
599.31	0.0050651614837277\\
599.32	0.00511561714911724\\
599.33	0.00516656483742509\\
599.34	0.00521800938159954\\
599.35	0.00526995566206841\\
599.36	0.00532240860720548\\
599.37	0.00537537319380152\\
599.38	0.00542885444753999\\
599.39	0.00548285743950268\\
599.4	0.00553738727623646\\
599.41	0.00559244911427385\\
599.42	0.00564804816062223\\
599.43	0.00570418967325793\\
599.44	0.00576087896162518\\
599.45	0.00581812138713983\\
599.46	0.00587592236369812\\
599.47	0.00593428735819033\\
599.48	0.0059932218910196\\
599.49	0.00605273153662562\\
599.5	0.00611282192401366\\
599.51	0.00617349873728862\\
599.52	0.00623476771619437\\
599.53	0.00629663465665842\\
599.54	0.00635910541134183\\
599.55	0.00642218589019451\\
599.56	0.00648588206101608\\
599.57	0.00655019995002201\\
599.58	0.00661514564241553\\
599.59	0.00668072528296489\\
599.6	0.0067469450765866\\
599.61	0.00681381128893397\\
599.62	0.00688133024699178\\
599.63	0.00694950833967657\\
599.64	0.00701835201844284\\
599.65	0.00708786779789519\\
599.66	0.00715806225640649\\
599.67	0.00722894203674203\\
599.68	0.00730051384668987\\
599.69	0.00737278445969726\\
599.7	0.00744576071551335\\
599.71	0.0075194495208382\\
599.72	0.00759385784997809\\
599.73	0.00766899274550729\\
599.74	0.00774486131893628\\
599.75	0.00782147075138657\\
599.76	0.00789882829427201\\
599.77	0.0079769412699869\\
599.78	0.00805581707260073\\
599.79	0.00813546316855977\\
599.8	0.00821588709739555\\
599.81	0.00829709647244021\\
599.82	0.00837909898154891\\
599.83	0.00846190238782928\\
599.84	0.008545514530378\\
599.85	0.00862994332502466\\
599.86	0.0087151967650828\\
599.87	0.00880128292210834\\
599.88	0.00888820994666552\\
599.89	0.00897598606910018\\
599.9	0.00906461960032073\\
599.91	0.00915411893258672\\
599.92	0.00924449254030512\\
599.93	0.00933574898083444\\
599.94	0.00942789689529664\\
599.95	0.00952094500939709\\
599.96	0.00961490213425244\\
599.97	0.00970977716722672\\
599.98	0.00980557909277551\\
599.99	0.00990231698329844\\
600	0.01\\
};
\addplot [color=black!60!mycolor21,solid,forget plot]
  table[row sep=crcr]{%
0.01	0\\
1.01	0\\
2.01	0\\
3.01	0\\
4.01	0\\
5.01	0\\
6.01	0\\
7.01	0\\
8.01	0\\
9.01	0\\
10.01	0\\
11.01	0\\
12.01	0\\
13.01	0\\
14.01	0\\
15.01	0\\
16.01	0\\
17.01	0\\
18.01	0\\
19.01	0\\
20.01	0\\
21.01	0\\
22.01	0\\
23.01	0\\
24.01	0\\
25.01	0\\
26.01	0\\
27.01	0\\
28.01	0\\
29.01	0\\
30.01	0\\
31.01	0\\
32.01	0\\
33.01	0\\
34.01	0\\
35.01	0\\
36.01	0\\
37.01	0\\
38.01	0\\
39.01	0\\
40.01	0\\
41.01	0\\
42.01	0\\
43.01	0\\
44.01	0\\
45.01	0\\
46.01	0\\
47.01	0\\
48.01	0\\
49.01	0\\
50.01	0\\
51.01	0\\
52.01	0\\
53.01	0\\
54.01	0\\
55.01	0\\
56.01	0\\
57.01	0\\
58.01	0\\
59.01	0\\
60.01	0\\
61.01	0\\
62.01	0\\
63.01	0\\
64.01	0\\
65.01	0\\
66.01	0\\
67.01	0\\
68.01	0\\
69.01	0\\
70.01	0\\
71.01	0\\
72.01	0\\
73.01	0\\
74.01	0\\
75.01	0\\
76.01	0\\
77.01	0\\
78.01	0\\
79.01	0\\
80.01	0\\
81.01	0\\
82.01	0\\
83.01	0\\
84.01	0\\
85.01	0\\
86.01	0\\
87.01	0\\
88.01	0\\
89.01	0\\
90.01	0\\
91.01	0\\
92.01	0\\
93.01	0\\
94.01	0\\
95.01	0\\
96.01	0\\
97.01	0\\
98.01	0\\
99.01	0\\
100.01	0\\
101.01	0\\
102.01	0\\
103.01	0\\
104.01	0\\
105.01	0\\
106.01	0\\
107.01	0\\
108.01	0\\
109.01	0\\
110.01	0\\
111.01	0\\
112.01	0\\
113.01	0\\
114.01	0\\
115.01	0\\
116.01	0\\
117.01	0\\
118.01	0\\
119.01	0\\
120.01	0\\
121.01	0\\
122.01	0\\
123.01	0\\
124.01	0\\
125.01	0\\
126.01	0\\
127.01	0\\
128.01	0\\
129.01	0\\
130.01	0\\
131.01	0\\
132.01	0\\
133.01	0\\
134.01	0\\
135.01	0\\
136.01	0\\
137.01	0\\
138.01	0\\
139.01	0\\
140.01	0\\
141.01	0\\
142.01	0\\
143.01	0\\
144.01	0\\
145.01	0\\
146.01	0\\
147.01	0\\
148.01	0\\
149.01	0\\
150.01	0\\
151.01	0\\
152.01	0\\
153.01	0\\
154.01	0\\
155.01	0\\
156.01	0\\
157.01	0\\
158.01	0\\
159.01	0\\
160.01	0\\
161.01	0\\
162.01	0\\
163.01	0\\
164.01	0\\
165.01	0\\
166.01	0\\
167.01	0\\
168.01	0\\
169.01	0\\
170.01	0\\
171.01	0\\
172.01	0\\
173.01	0\\
174.01	0\\
175.01	0\\
176.01	0\\
177.01	0\\
178.01	0\\
179.01	0\\
180.01	0\\
181.01	0\\
182.01	0\\
183.01	0\\
184.01	0\\
185.01	0\\
186.01	0\\
187.01	0\\
188.01	0\\
189.01	0\\
190.01	0\\
191.01	0\\
192.01	0\\
193.01	0\\
194.01	0\\
195.01	0\\
196.01	0\\
197.01	0\\
198.01	0\\
199.01	0\\
200.01	0\\
201.01	0\\
202.01	0\\
203.01	0\\
204.01	0\\
205.01	0\\
206.01	0\\
207.01	0\\
208.01	0\\
209.01	0\\
210.01	0\\
211.01	0\\
212.01	0\\
213.01	0\\
214.01	0\\
215.01	0\\
216.01	0\\
217.01	0\\
218.01	0\\
219.01	0\\
220.01	0\\
221.01	0\\
222.01	0\\
223.01	0\\
224.01	0\\
225.01	0\\
226.01	0\\
227.01	0\\
228.01	0\\
229.01	0\\
230.01	0\\
231.01	0\\
232.01	0\\
233.01	0\\
234.01	0\\
235.01	0\\
236.01	0\\
237.01	0\\
238.01	0\\
239.01	0\\
240.01	0\\
241.01	0\\
242.01	0\\
243.01	0\\
244.01	0\\
245.01	0\\
246.01	0\\
247.01	0\\
248.01	0\\
249.01	0\\
250.01	0\\
251.01	0\\
252.01	0\\
253.01	0\\
254.01	0\\
255.01	0\\
256.01	0\\
257.01	0\\
258.01	0\\
259.01	0\\
260.01	0\\
261.01	0\\
262.01	0\\
263.01	0\\
264.01	0\\
265.01	0\\
266.01	0\\
267.01	0\\
268.01	0\\
269.01	0\\
270.01	0\\
271.01	0\\
272.01	0\\
273.01	0\\
274.01	0\\
275.01	0\\
276.01	0\\
277.01	0\\
278.01	0\\
279.01	0\\
280.01	0\\
281.01	0\\
282.01	0\\
283.01	0\\
284.01	0\\
285.01	0\\
286.01	0\\
287.01	0\\
288.01	0\\
289.01	0\\
290.01	0\\
291.01	0\\
292.01	0\\
293.01	0\\
294.01	0\\
295.01	0\\
296.01	0\\
297.01	0\\
298.01	0\\
299.01	0\\
300.01	0\\
301.01	0\\
302.01	0\\
303.01	0\\
304.01	0\\
305.01	0\\
306.01	0\\
307.01	0\\
308.01	0\\
309.01	0\\
310.01	0\\
311.01	0\\
312.01	0\\
313.01	0\\
314.01	0\\
315.01	0\\
316.01	0\\
317.01	0\\
318.01	0\\
319.01	0\\
320.01	0\\
321.01	0\\
322.01	0\\
323.01	0\\
324.01	0\\
325.01	0\\
326.01	0\\
327.01	0\\
328.01	0\\
329.01	0\\
330.01	0\\
331.01	0\\
332.01	0\\
333.01	0\\
334.01	0\\
335.01	0\\
336.01	0\\
337.01	0\\
338.01	0\\
339.01	0\\
340.01	0\\
341.01	0\\
342.01	0\\
343.01	0\\
344.01	0\\
345.01	0\\
346.01	0\\
347.01	0\\
348.01	0\\
349.01	0\\
350.01	0\\
351.01	0\\
352.01	0\\
353.01	0\\
354.01	0\\
355.01	0\\
356.01	0\\
357.01	0\\
358.01	0\\
359.01	0\\
360.01	0\\
361.01	0\\
362.01	0\\
363.01	0\\
364.01	0\\
365.01	0\\
366.01	0\\
367.01	0\\
368.01	0\\
369.01	0\\
370.01	0\\
371.01	0\\
372.01	0\\
373.01	0\\
374.01	0\\
375.01	0\\
376.01	0\\
377.01	0\\
378.01	0\\
379.01	0\\
380.01	0\\
381.01	0\\
382.01	0\\
383.01	0\\
384.01	0\\
385.01	0\\
386.01	0\\
387.01	0\\
388.01	0\\
389.01	0\\
390.01	0\\
391.01	0\\
392.01	0\\
393.01	0\\
394.01	0\\
395.01	0\\
396.01	0\\
397.01	0\\
398.01	0\\
399.01	0\\
400.01	0\\
401.01	0\\
402.01	0\\
403.01	0\\
404.01	0\\
405.01	0\\
406.01	0\\
407.01	0\\
408.01	0\\
409.01	0\\
410.01	0\\
411.01	0\\
412.01	0\\
413.01	0\\
414.01	0\\
415.01	0\\
416.01	0\\
417.01	0\\
418.01	0\\
419.01	0\\
420.01	0\\
421.01	0\\
422.01	0\\
423.01	0\\
424.01	0\\
425.01	0\\
426.01	0\\
427.01	0\\
428.01	0\\
429.01	0\\
430.01	0\\
431.01	0\\
432.01	0\\
433.01	0\\
434.01	0\\
435.01	0\\
436.01	0\\
437.01	0\\
438.01	0\\
439.01	0\\
440.01	0\\
441.01	0\\
442.01	0\\
443.01	0\\
444.01	0\\
445.01	0\\
446.01	0\\
447.01	0\\
448.01	0\\
449.01	0\\
450.01	0\\
451.01	0\\
452.01	0\\
453.01	0\\
454.01	0\\
455.01	0\\
456.01	0\\
457.01	0\\
458.01	0\\
459.01	0\\
460.01	0\\
461.01	0\\
462.01	0\\
463.01	0\\
464.01	0\\
465.01	0\\
466.01	0\\
467.01	0\\
468.01	0\\
469.01	0\\
470.01	0\\
471.01	0\\
472.01	0\\
473.01	0\\
474.01	0\\
475.01	0\\
476.01	0\\
477.01	0\\
478.01	0\\
479.01	0\\
480.01	0\\
481.01	0\\
482.01	0\\
483.01	0\\
484.01	0\\
485.01	0\\
486.01	0\\
487.01	0\\
488.01	0\\
489.01	0\\
490.01	0\\
491.01	0\\
492.01	0\\
493.01	0\\
494.01	0\\
495.01	0\\
496.01	0\\
497.01	0\\
498.01	0\\
499.01	0\\
500.01	0\\
501.01	0\\
502.01	0\\
503.01	0\\
504.01	0\\
505.01	0\\
506.01	0\\
507.01	0\\
508.01	0\\
509.01	0\\
510.01	0\\
511.01	0\\
512.01	0\\
513.01	0\\
514.01	0\\
515.01	0\\
516.01	0\\
517.01	0\\
518.01	0\\
519.01	0\\
520.01	0\\
521.01	0\\
522.01	0\\
523.01	0\\
524.01	0\\
525.01	0\\
526.01	0\\
527.01	0\\
528.01	0\\
529.01	0\\
530.01	0\\
531.01	0\\
532.01	0\\
533.01	0\\
534.01	0\\
535.01	0\\
536.01	0\\
537.01	0\\
538.01	0\\
539.01	0\\
540.01	0\\
541.01	0\\
542.01	0\\
543.01	0\\
544.01	0\\
545.01	0\\
546.01	0\\
547.01	0\\
548.01	0\\
549.01	0\\
550.01	0\\
551.01	0\\
552.01	0\\
553.01	0\\
554.01	0\\
555.01	0\\
556.01	0\\
557.01	0\\
558.01	0\\
559.01	0\\
560.01	0\\
561.01	0\\
562.01	0\\
563.01	0\\
564.01	0\\
565.01	0\\
566.01	0\\
567.01	0\\
568.01	0\\
569.01	0\\
570.01	0\\
571.01	0\\
572.01	0\\
573.01	0\\
574.01	0\\
575.01	0\\
576.01	0\\
577.01	0\\
578.01	0\\
579.01	0\\
580.01	0\\
581.01	0\\
582.01	0\\
583.01	0\\
584.01	0\\
585.01	0\\
586.01	0\\
587.01	0\\
588.01	0\\
589.01	0\\
590.01	0\\
591.01	0\\
592.01	0\\
593.01	0\\
594.01	0\\
595.01	0\\
596.01	0\\
597.01	0\\
598.01	0.00134872349121326\\
599.01	0.0037581309059459\\
599.02	0.00379585558757838\\
599.03	0.00383394724377262\\
599.04	0.00387240947898428\\
599.05	0.00391124593307964\\
599.06	0.00395046028168355\\
599.07	0.00399005623653081\\
599.08	0.00403003754582079\\
599.09	0.00407040799457573\\
599.1	0.00411117140500246\\
599.11	0.00415233163685777\\
599.12	0.00419389258781723\\
599.13	0.00423585819384774\\
599.14	0.00427823242958371\\
599.15	0.00432101930870692\\
599.16	0.00436422288433015\\
599.17	0.00440784724938453\\
599.18	0.00445189653701072\\
599.19	0.00449637492095398\\
599.2	0.00454128661596297\\
599.21	0.00458663587819268\\
599.22	0.00463242700561116\\
599.23	0.0046786643384103\\
599.24	0.00472535225942063\\
599.25	0.0047724951945302\\
599.26	0.00482009761310764\\
599.27	0.00486816402842928\\
599.28	0.00491669899811044\\
599.29	0.00496570712454111\\
599.3	0.00501519305532579\\
599.31	0.00506516148372767\\
599.32	0.00511561714911721\\
599.33	0.00516656483742506\\
599.34	0.0052180093815995\\
599.35	0.00526995566206835\\
599.36	0.00532240860720541\\
599.37	0.00537537319380146\\
599.38	0.00542885444753993\\
599.39	0.00548285743950261\\
599.4	0.00553738727623639\\
599.41	0.00559244911427376\\
599.42	0.00564804816062213\\
599.43	0.00570418967325784\\
599.44	0.00576087896162508\\
599.45	0.00581812138713973\\
599.46	0.00587592236369801\\
599.47	0.00593428735819024\\
599.48	0.0059932218910195\\
599.49	0.00605273153662553\\
599.5	0.00611282192401356\\
599.51	0.00617349873728851\\
599.52	0.00623476771619427\\
599.53	0.00629663465665833\\
599.54	0.00635910541134176\\
599.55	0.00642218589019446\\
599.56	0.00648588206101604\\
599.57	0.00655019995002198\\
599.58	0.00661514564241548\\
599.59	0.00668072528296487\\
599.6	0.00674694507658657\\
599.61	0.00681381128893394\\
599.62	0.00688133024699176\\
599.63	0.00694950833967655\\
599.64	0.00701835201844282\\
599.65	0.00708786779789516\\
599.66	0.00715806225640646\\
599.67	0.00722894203674201\\
599.68	0.00730051384668985\\
599.69	0.00737278445969725\\
599.7	0.00744576071551335\\
599.71	0.0075194495208382\\
599.72	0.00759385784997808\\
599.73	0.00766899274550727\\
599.74	0.00774486131893627\\
599.75	0.00782147075138656\\
599.76	0.00789882829427201\\
599.77	0.0079769412699869\\
599.78	0.00805581707260072\\
599.79	0.00813546316855976\\
599.8	0.00821588709739554\\
599.81	0.0082970964724402\\
599.82	0.00837909898154891\\
599.83	0.00846190238782927\\
599.84	0.008545514530378\\
599.85	0.00862994332502466\\
599.86	0.00871519676508279\\
599.87	0.00880128292210833\\
599.88	0.00888820994666551\\
599.89	0.00897598606910018\\
599.9	0.00906461960032073\\
599.91	0.00915411893258671\\
599.92	0.00924449254030512\\
599.93	0.00933574898083444\\
599.94	0.00942789689529664\\
599.95	0.00952094500939709\\
599.96	0.00961490213425245\\
599.97	0.00970977716722672\\
599.98	0.00980557909277551\\
599.99	0.00990231698329843\\
600	0.01\\
};
\addplot [color=black!80!mycolor21,solid,forget plot]
  table[row sep=crcr]{%
0.01	0\\
1.01	0\\
2.01	0\\
3.01	0\\
4.01	0\\
5.01	0\\
6.01	0\\
7.01	0\\
8.01	0\\
9.01	0\\
10.01	0\\
11.01	0\\
12.01	0\\
13.01	0\\
14.01	0\\
15.01	0\\
16.01	0\\
17.01	0\\
18.01	0\\
19.01	0\\
20.01	0\\
21.01	0\\
22.01	0\\
23.01	0\\
24.01	0\\
25.01	0\\
26.01	0\\
27.01	0\\
28.01	0\\
29.01	0\\
30.01	0\\
31.01	0\\
32.01	0\\
33.01	0\\
34.01	0\\
35.01	0\\
36.01	0\\
37.01	0\\
38.01	0\\
39.01	0\\
40.01	0\\
41.01	0\\
42.01	0\\
43.01	0\\
44.01	0\\
45.01	0\\
46.01	0\\
47.01	0\\
48.01	0\\
49.01	0\\
50.01	0\\
51.01	0\\
52.01	0\\
53.01	0\\
54.01	0\\
55.01	0\\
56.01	0\\
57.01	0\\
58.01	0\\
59.01	0\\
60.01	0\\
61.01	0\\
62.01	0\\
63.01	0\\
64.01	0\\
65.01	0\\
66.01	0\\
67.01	0\\
68.01	0\\
69.01	0\\
70.01	0\\
71.01	0\\
72.01	0\\
73.01	0\\
74.01	0\\
75.01	0\\
76.01	0\\
77.01	0\\
78.01	0\\
79.01	0\\
80.01	0\\
81.01	0\\
82.01	0\\
83.01	0\\
84.01	0\\
85.01	0\\
86.01	0\\
87.01	0\\
88.01	0\\
89.01	0\\
90.01	0\\
91.01	0\\
92.01	0\\
93.01	0\\
94.01	0\\
95.01	0\\
96.01	0\\
97.01	0\\
98.01	0\\
99.01	0\\
100.01	0\\
101.01	0\\
102.01	0\\
103.01	0\\
104.01	0\\
105.01	0\\
106.01	0\\
107.01	0\\
108.01	0\\
109.01	0\\
110.01	0\\
111.01	0\\
112.01	0\\
113.01	0\\
114.01	0\\
115.01	0\\
116.01	0\\
117.01	0\\
118.01	0\\
119.01	0\\
120.01	0\\
121.01	0\\
122.01	0\\
123.01	0\\
124.01	0\\
125.01	0\\
126.01	0\\
127.01	0\\
128.01	0\\
129.01	0\\
130.01	0\\
131.01	0\\
132.01	0\\
133.01	0\\
134.01	0\\
135.01	0\\
136.01	0\\
137.01	0\\
138.01	0\\
139.01	0\\
140.01	0\\
141.01	0\\
142.01	0\\
143.01	0\\
144.01	0\\
145.01	0\\
146.01	0\\
147.01	0\\
148.01	0\\
149.01	0\\
150.01	0\\
151.01	0\\
152.01	0\\
153.01	0\\
154.01	0\\
155.01	0\\
156.01	0\\
157.01	0\\
158.01	0\\
159.01	0\\
160.01	0\\
161.01	0\\
162.01	0\\
163.01	0\\
164.01	0\\
165.01	0\\
166.01	0\\
167.01	0\\
168.01	0\\
169.01	0\\
170.01	0\\
171.01	0\\
172.01	0\\
173.01	0\\
174.01	0\\
175.01	0\\
176.01	0\\
177.01	0\\
178.01	0\\
179.01	0\\
180.01	0\\
181.01	0\\
182.01	0\\
183.01	0\\
184.01	0\\
185.01	0\\
186.01	0\\
187.01	0\\
188.01	0\\
189.01	0\\
190.01	0\\
191.01	0\\
192.01	0\\
193.01	0\\
194.01	0\\
195.01	0\\
196.01	0\\
197.01	0\\
198.01	0\\
199.01	0\\
200.01	0\\
201.01	0\\
202.01	0\\
203.01	0\\
204.01	0\\
205.01	0\\
206.01	0\\
207.01	0\\
208.01	0\\
209.01	0\\
210.01	0\\
211.01	0\\
212.01	0\\
213.01	0\\
214.01	0\\
215.01	0\\
216.01	0\\
217.01	0\\
218.01	0\\
219.01	0\\
220.01	0\\
221.01	0\\
222.01	0\\
223.01	0\\
224.01	0\\
225.01	0\\
226.01	0\\
227.01	0\\
228.01	0\\
229.01	0\\
230.01	0\\
231.01	0\\
232.01	0\\
233.01	0\\
234.01	0\\
235.01	0\\
236.01	0\\
237.01	0\\
238.01	0\\
239.01	0\\
240.01	0\\
241.01	0\\
242.01	0\\
243.01	0\\
244.01	0\\
245.01	0\\
246.01	0\\
247.01	0\\
248.01	0\\
249.01	0\\
250.01	0\\
251.01	0\\
252.01	0\\
253.01	0\\
254.01	0\\
255.01	0\\
256.01	0\\
257.01	0\\
258.01	0\\
259.01	0\\
260.01	0\\
261.01	0\\
262.01	0\\
263.01	0\\
264.01	0\\
265.01	0\\
266.01	0\\
267.01	0\\
268.01	0\\
269.01	0\\
270.01	0\\
271.01	0\\
272.01	0\\
273.01	0\\
274.01	0\\
275.01	0\\
276.01	0\\
277.01	0\\
278.01	0\\
279.01	0\\
280.01	0\\
281.01	0\\
282.01	0\\
283.01	0\\
284.01	0\\
285.01	0\\
286.01	0\\
287.01	0\\
288.01	0\\
289.01	0\\
290.01	0\\
291.01	0\\
292.01	0\\
293.01	0\\
294.01	0\\
295.01	0\\
296.01	0\\
297.01	0\\
298.01	0\\
299.01	0\\
300.01	0\\
301.01	0\\
302.01	0\\
303.01	0\\
304.01	0\\
305.01	0\\
306.01	0\\
307.01	0\\
308.01	0\\
309.01	0\\
310.01	0\\
311.01	0\\
312.01	0\\
313.01	0\\
314.01	0\\
315.01	0\\
316.01	0\\
317.01	0\\
318.01	0\\
319.01	0\\
320.01	0\\
321.01	0\\
322.01	0\\
323.01	0\\
324.01	0\\
325.01	0\\
326.01	0\\
327.01	0\\
328.01	0\\
329.01	0\\
330.01	0\\
331.01	0\\
332.01	0\\
333.01	0\\
334.01	0\\
335.01	0\\
336.01	0\\
337.01	0\\
338.01	0\\
339.01	0\\
340.01	0\\
341.01	0\\
342.01	0\\
343.01	0\\
344.01	0\\
345.01	0\\
346.01	0\\
347.01	0\\
348.01	0\\
349.01	0\\
350.01	0\\
351.01	0\\
352.01	0\\
353.01	0\\
354.01	0\\
355.01	0\\
356.01	0\\
357.01	0\\
358.01	0\\
359.01	0\\
360.01	0\\
361.01	0\\
362.01	0\\
363.01	0\\
364.01	0\\
365.01	0\\
366.01	0\\
367.01	0\\
368.01	0\\
369.01	0\\
370.01	0\\
371.01	0\\
372.01	0\\
373.01	0\\
374.01	0\\
375.01	0\\
376.01	0\\
377.01	0\\
378.01	0\\
379.01	0\\
380.01	0\\
381.01	0\\
382.01	0\\
383.01	0\\
384.01	0\\
385.01	0\\
386.01	0\\
387.01	0\\
388.01	0\\
389.01	0\\
390.01	0\\
391.01	0\\
392.01	0\\
393.01	0\\
394.01	0\\
395.01	0\\
396.01	0\\
397.01	0\\
398.01	0\\
399.01	0\\
400.01	0\\
401.01	0\\
402.01	0\\
403.01	0\\
404.01	0\\
405.01	0\\
406.01	0\\
407.01	0\\
408.01	0\\
409.01	0\\
410.01	0\\
411.01	0\\
412.01	0\\
413.01	0\\
414.01	0\\
415.01	0\\
416.01	0\\
417.01	0\\
418.01	0\\
419.01	0\\
420.01	0\\
421.01	0\\
422.01	0\\
423.01	0\\
424.01	0\\
425.01	0\\
426.01	0\\
427.01	0\\
428.01	0\\
429.01	0\\
430.01	0\\
431.01	0\\
432.01	0\\
433.01	0\\
434.01	0\\
435.01	0\\
436.01	0\\
437.01	0\\
438.01	0\\
439.01	0\\
440.01	0\\
441.01	0\\
442.01	0\\
443.01	0\\
444.01	0\\
445.01	0\\
446.01	0\\
447.01	0\\
448.01	0\\
449.01	0\\
450.01	0\\
451.01	0\\
452.01	0\\
453.01	0\\
454.01	0\\
455.01	0\\
456.01	0\\
457.01	0\\
458.01	0\\
459.01	0\\
460.01	0\\
461.01	0\\
462.01	0\\
463.01	0\\
464.01	0\\
465.01	0\\
466.01	0\\
467.01	0\\
468.01	0\\
469.01	0\\
470.01	0\\
471.01	0\\
472.01	0\\
473.01	0\\
474.01	0\\
475.01	0\\
476.01	0\\
477.01	0\\
478.01	0\\
479.01	0\\
480.01	0\\
481.01	0\\
482.01	0\\
483.01	0\\
484.01	0\\
485.01	0\\
486.01	0\\
487.01	0\\
488.01	0\\
489.01	0\\
490.01	0\\
491.01	0\\
492.01	0\\
493.01	0\\
494.01	0\\
495.01	0\\
496.01	0\\
497.01	0\\
498.01	0\\
499.01	0\\
500.01	0\\
501.01	0\\
502.01	0\\
503.01	0\\
504.01	0\\
505.01	0\\
506.01	0\\
507.01	0\\
508.01	0\\
509.01	0\\
510.01	0\\
511.01	0\\
512.01	0\\
513.01	0\\
514.01	0\\
515.01	0\\
516.01	0\\
517.01	0\\
518.01	0\\
519.01	0\\
520.01	0\\
521.01	0\\
522.01	0\\
523.01	0\\
524.01	0\\
525.01	0\\
526.01	0\\
527.01	0\\
528.01	0\\
529.01	0\\
530.01	0\\
531.01	0\\
532.01	0\\
533.01	0\\
534.01	0\\
535.01	0\\
536.01	0\\
537.01	0\\
538.01	0\\
539.01	0\\
540.01	0\\
541.01	0\\
542.01	0\\
543.01	0\\
544.01	0\\
545.01	0\\
546.01	0\\
547.01	0\\
548.01	0\\
549.01	0\\
550.01	0\\
551.01	0\\
552.01	0\\
553.01	0\\
554.01	0\\
555.01	0\\
556.01	0\\
557.01	0\\
558.01	0\\
559.01	0\\
560.01	0\\
561.01	0\\
562.01	0\\
563.01	0\\
564.01	0\\
565.01	0\\
566.01	0\\
567.01	0\\
568.01	0\\
569.01	0\\
570.01	0\\
571.01	0\\
572.01	0\\
573.01	0\\
574.01	0\\
575.01	0\\
576.01	0\\
577.01	0\\
578.01	0\\
579.01	0\\
580.01	0\\
581.01	0\\
582.01	0\\
583.01	0\\
584.01	0\\
585.01	0\\
586.01	0\\
587.01	0\\
588.01	0\\
589.01	0\\
590.01	0\\
591.01	0\\
592.01	0\\
593.01	0\\
594.01	0\\
595.01	0\\
596.01	0\\
597.01	0\\
598.01	0.00134893218521348\\
599.01	0.0037581309059459\\
599.02	0.00379585558757836\\
599.03	0.00383394724377265\\
599.04	0.00387240947898429\\
599.05	0.00391124593307965\\
599.06	0.00395046028168355\\
599.07	0.00399005623653083\\
599.08	0.00403003754582081\\
599.09	0.00407040799457575\\
599.1	0.00411117140500249\\
599.11	0.0041523316368578\\
599.12	0.00419389258781726\\
599.13	0.00423585819384777\\
599.14	0.00427823242958372\\
599.15	0.00432101930870693\\
599.16	0.00436422288433015\\
599.17	0.00440784724938453\\
599.18	0.00445189653701074\\
599.19	0.00449637492095398\\
599.2	0.00454128661596297\\
599.21	0.00458663587819269\\
599.22	0.00463242700561119\\
599.23	0.00467866433841033\\
599.24	0.00472535225942065\\
599.25	0.00477249519453024\\
599.26	0.0048200976131077\\
599.27	0.00486816402842932\\
599.28	0.00491669899811048\\
599.29	0.00496570712454114\\
599.3	0.00501519305532583\\
599.31	0.00506516148372771\\
599.32	0.00511561714911724\\
599.33	0.00516656483742507\\
599.34	0.00521800938159953\\
599.35	0.00526995566206839\\
599.36	0.00532240860720545\\
599.37	0.00537537319380151\\
599.38	0.00542885444753997\\
599.39	0.00548285743950265\\
599.4	0.00553738727623644\\
599.41	0.00559244911427383\\
599.42	0.0056480481606222\\
599.43	0.00570418967325791\\
599.44	0.00576087896162515\\
599.45	0.0058181213871398\\
599.46	0.00587592236369808\\
599.47	0.00593428735819029\\
599.48	0.00599322189101957\\
599.49	0.00605273153662558\\
599.5	0.00611282192401361\\
599.51	0.00617349873728856\\
599.52	0.00623476771619431\\
599.53	0.00629663465665838\\
599.54	0.00635910541134178\\
599.55	0.00642218589019448\\
599.56	0.00648588206101605\\
599.57	0.00655019995002198\\
599.58	0.0066151456424155\\
599.59	0.00668072528296487\\
599.6	0.00674694507658657\\
599.61	0.00681381128893395\\
599.62	0.00688133024699177\\
599.63	0.00694950833967656\\
599.64	0.00701835201844282\\
599.65	0.00708786779789516\\
599.66	0.00715806225640645\\
599.67	0.00722894203674201\\
599.68	0.00730051384668984\\
599.69	0.00737278445969725\\
599.7	0.00744576071551334\\
599.71	0.00751944952083819\\
599.72	0.00759385784997808\\
599.73	0.00766899274550727\\
599.74	0.00774486131893627\\
599.75	0.00782147075138656\\
599.76	0.00789882829427201\\
599.77	0.0079769412699869\\
599.78	0.00805581707260072\\
599.79	0.00813546316855977\\
599.8	0.00821588709739555\\
599.81	0.00829709647244022\\
599.82	0.00837909898154891\\
599.83	0.00846190238782928\\
599.84	0.008545514530378\\
599.85	0.00862994332502466\\
599.86	0.00871519676508279\\
599.87	0.00880128292210834\\
599.88	0.00888820994666552\\
599.89	0.00897598606910018\\
599.9	0.00906461960032073\\
599.91	0.00915411893258672\\
599.92	0.00924449254030512\\
599.93	0.00933574898083444\\
599.94	0.00942789689529664\\
599.95	0.00952094500939709\\
599.96	0.00961490213425244\\
599.97	0.00970977716722672\\
599.98	0.00980557909277551\\
599.99	0.00990231698329844\\
600	0.01\\
};
\addplot [color=black,solid,forget plot]
  table[row sep=crcr]{%
0.01	0\\
1.01	0\\
2.01	0\\
3.01	0\\
4.01	0\\
5.01	0\\
6.01	0\\
7.01	0\\
8.01	0\\
9.01	0\\
10.01	0\\
11.01	0\\
12.01	0\\
13.01	0\\
14.01	0\\
15.01	0\\
16.01	0\\
17.01	0\\
18.01	0\\
19.01	0\\
20.01	0\\
21.01	0\\
22.01	0\\
23.01	0\\
24.01	0\\
25.01	0\\
26.01	0\\
27.01	0\\
28.01	0\\
29.01	0\\
30.01	0\\
31.01	0\\
32.01	0\\
33.01	0\\
34.01	0\\
35.01	0\\
36.01	0\\
37.01	0\\
38.01	0\\
39.01	0\\
40.01	0\\
41.01	0\\
42.01	0\\
43.01	0\\
44.01	0\\
45.01	0\\
46.01	0\\
47.01	0\\
48.01	0\\
49.01	0\\
50.01	0\\
51.01	0\\
52.01	0\\
53.01	0\\
54.01	0\\
55.01	0\\
56.01	0\\
57.01	0\\
58.01	0\\
59.01	0\\
60.01	0\\
61.01	0\\
62.01	0\\
63.01	0\\
64.01	0\\
65.01	0\\
66.01	0\\
67.01	0\\
68.01	0\\
69.01	0\\
70.01	0\\
71.01	0\\
72.01	0\\
73.01	0\\
74.01	0\\
75.01	0\\
76.01	0\\
77.01	0\\
78.01	0\\
79.01	0\\
80.01	0\\
81.01	0\\
82.01	0\\
83.01	0\\
84.01	0\\
85.01	0\\
86.01	0\\
87.01	0\\
88.01	0\\
89.01	0\\
90.01	0\\
91.01	0\\
92.01	0\\
93.01	0\\
94.01	0\\
95.01	0\\
96.01	0\\
97.01	0\\
98.01	0\\
99.01	0\\
100.01	0\\
101.01	0\\
102.01	0\\
103.01	0\\
104.01	0\\
105.01	0\\
106.01	0\\
107.01	0\\
108.01	0\\
109.01	0\\
110.01	0\\
111.01	0\\
112.01	0\\
113.01	0\\
114.01	0\\
115.01	0\\
116.01	0\\
117.01	0\\
118.01	0\\
119.01	0\\
120.01	0\\
121.01	0\\
122.01	0\\
123.01	0\\
124.01	0\\
125.01	0\\
126.01	0\\
127.01	0\\
128.01	0\\
129.01	0\\
130.01	0\\
131.01	0\\
132.01	0\\
133.01	0\\
134.01	0\\
135.01	0\\
136.01	0\\
137.01	0\\
138.01	0\\
139.01	0\\
140.01	0\\
141.01	0\\
142.01	0\\
143.01	0\\
144.01	0\\
145.01	0\\
146.01	0\\
147.01	0\\
148.01	0\\
149.01	0\\
150.01	0\\
151.01	0\\
152.01	0\\
153.01	0\\
154.01	0\\
155.01	0\\
156.01	0\\
157.01	0\\
158.01	0\\
159.01	0\\
160.01	0\\
161.01	0\\
162.01	0\\
163.01	0\\
164.01	0\\
165.01	0\\
166.01	0\\
167.01	0\\
168.01	0\\
169.01	0\\
170.01	0\\
171.01	0\\
172.01	0\\
173.01	0\\
174.01	0\\
175.01	0\\
176.01	0\\
177.01	0\\
178.01	0\\
179.01	0\\
180.01	0\\
181.01	0\\
182.01	0\\
183.01	0\\
184.01	0\\
185.01	0\\
186.01	0\\
187.01	0\\
188.01	0\\
189.01	0\\
190.01	0\\
191.01	0\\
192.01	0\\
193.01	0\\
194.01	0\\
195.01	0\\
196.01	0\\
197.01	0\\
198.01	0\\
199.01	0\\
200.01	0\\
201.01	0\\
202.01	0\\
203.01	0\\
204.01	0\\
205.01	0\\
206.01	0\\
207.01	0\\
208.01	0\\
209.01	0\\
210.01	0\\
211.01	0\\
212.01	0\\
213.01	0\\
214.01	0\\
215.01	0\\
216.01	0\\
217.01	0\\
218.01	0\\
219.01	0\\
220.01	0\\
221.01	0\\
222.01	0\\
223.01	0\\
224.01	0\\
225.01	0\\
226.01	0\\
227.01	0\\
228.01	0\\
229.01	0\\
230.01	0\\
231.01	0\\
232.01	0\\
233.01	0\\
234.01	0\\
235.01	0\\
236.01	0\\
237.01	0\\
238.01	0\\
239.01	0\\
240.01	0\\
241.01	0\\
242.01	0\\
243.01	0\\
244.01	0\\
245.01	0\\
246.01	0\\
247.01	0\\
248.01	0\\
249.01	0\\
250.01	0\\
251.01	0\\
252.01	0\\
253.01	0\\
254.01	0\\
255.01	0\\
256.01	0\\
257.01	0\\
258.01	0\\
259.01	0\\
260.01	0\\
261.01	0\\
262.01	0\\
263.01	0\\
264.01	0\\
265.01	0\\
266.01	0\\
267.01	0\\
268.01	0\\
269.01	0\\
270.01	0\\
271.01	0\\
272.01	0\\
273.01	0\\
274.01	0\\
275.01	0\\
276.01	0\\
277.01	0\\
278.01	0\\
279.01	0\\
280.01	0\\
281.01	0\\
282.01	0\\
283.01	0\\
284.01	0\\
285.01	0\\
286.01	0\\
287.01	0\\
288.01	0\\
289.01	0\\
290.01	0\\
291.01	0\\
292.01	0\\
293.01	0\\
294.01	0\\
295.01	0\\
296.01	0\\
297.01	0\\
298.01	0\\
299.01	0\\
300.01	0\\
301.01	0\\
302.01	0\\
303.01	0\\
304.01	0\\
305.01	0\\
306.01	0\\
307.01	0\\
308.01	0\\
309.01	0\\
310.01	0\\
311.01	0\\
312.01	0\\
313.01	0\\
314.01	0\\
315.01	0\\
316.01	0\\
317.01	0\\
318.01	0\\
319.01	0\\
320.01	0\\
321.01	0\\
322.01	0\\
323.01	0\\
324.01	0\\
325.01	0\\
326.01	0\\
327.01	0\\
328.01	0\\
329.01	0\\
330.01	0\\
331.01	0\\
332.01	0\\
333.01	0\\
334.01	0\\
335.01	0\\
336.01	0\\
337.01	0\\
338.01	0\\
339.01	0\\
340.01	0\\
341.01	0\\
342.01	0\\
343.01	0\\
344.01	0\\
345.01	0\\
346.01	0\\
347.01	0\\
348.01	0\\
349.01	0\\
350.01	0\\
351.01	0\\
352.01	0\\
353.01	0\\
354.01	0\\
355.01	0\\
356.01	0\\
357.01	0\\
358.01	0\\
359.01	0\\
360.01	0\\
361.01	0\\
362.01	0\\
363.01	0\\
364.01	0\\
365.01	0\\
366.01	0\\
367.01	0\\
368.01	0\\
369.01	0\\
370.01	0\\
371.01	0\\
372.01	0\\
373.01	0\\
374.01	0\\
375.01	0\\
376.01	0\\
377.01	0\\
378.01	0\\
379.01	0\\
380.01	0\\
381.01	0\\
382.01	0\\
383.01	0\\
384.01	0\\
385.01	0\\
386.01	0\\
387.01	0\\
388.01	0\\
389.01	0\\
390.01	0\\
391.01	0\\
392.01	0\\
393.01	0\\
394.01	0\\
395.01	0\\
396.01	0\\
397.01	0\\
398.01	0\\
399.01	0\\
400.01	0\\
401.01	0\\
402.01	0\\
403.01	0\\
404.01	0\\
405.01	0\\
406.01	0\\
407.01	0\\
408.01	0\\
409.01	0\\
410.01	0\\
411.01	0\\
412.01	0\\
413.01	0\\
414.01	0\\
415.01	0\\
416.01	0\\
417.01	0\\
418.01	0\\
419.01	0\\
420.01	0\\
421.01	0\\
422.01	0\\
423.01	0\\
424.01	0\\
425.01	0\\
426.01	0\\
427.01	0\\
428.01	0\\
429.01	0\\
430.01	0\\
431.01	0\\
432.01	0\\
433.01	0\\
434.01	0\\
435.01	0\\
436.01	0\\
437.01	0\\
438.01	0\\
439.01	0\\
440.01	0\\
441.01	0\\
442.01	0\\
443.01	0\\
444.01	0\\
445.01	0\\
446.01	0\\
447.01	0\\
448.01	0\\
449.01	0\\
450.01	0\\
451.01	0\\
452.01	0\\
453.01	0\\
454.01	0\\
455.01	0\\
456.01	0\\
457.01	0\\
458.01	0\\
459.01	0\\
460.01	0\\
461.01	0\\
462.01	0\\
463.01	0\\
464.01	0\\
465.01	0\\
466.01	0\\
467.01	0\\
468.01	0\\
469.01	0\\
470.01	0\\
471.01	0\\
472.01	0\\
473.01	0\\
474.01	0\\
475.01	0\\
476.01	0\\
477.01	0\\
478.01	0\\
479.01	0\\
480.01	0\\
481.01	0\\
482.01	0\\
483.01	0\\
484.01	0\\
485.01	0\\
486.01	0\\
487.01	0\\
488.01	0\\
489.01	0\\
490.01	0\\
491.01	0\\
492.01	0\\
493.01	0\\
494.01	0\\
495.01	0\\
496.01	0\\
497.01	0\\
498.01	0\\
499.01	0\\
500.01	0\\
501.01	0\\
502.01	0\\
503.01	0\\
504.01	0\\
505.01	0\\
506.01	0\\
507.01	0\\
508.01	0\\
509.01	0\\
510.01	0\\
511.01	0\\
512.01	0\\
513.01	0\\
514.01	0\\
515.01	0\\
516.01	0\\
517.01	0\\
518.01	0\\
519.01	0\\
520.01	0\\
521.01	0\\
522.01	0\\
523.01	0\\
524.01	0\\
525.01	0\\
526.01	0\\
527.01	0\\
528.01	0\\
529.01	0\\
530.01	0\\
531.01	0\\
532.01	0\\
533.01	0\\
534.01	0\\
535.01	0\\
536.01	0\\
537.01	0\\
538.01	0\\
539.01	0\\
540.01	0\\
541.01	0\\
542.01	0\\
543.01	0\\
544.01	0\\
545.01	0\\
546.01	0\\
547.01	0\\
548.01	0\\
549.01	0\\
550.01	0\\
551.01	0\\
552.01	0\\
553.01	0\\
554.01	0\\
555.01	0\\
556.01	0\\
557.01	0\\
558.01	0\\
559.01	0\\
560.01	0\\
561.01	0\\
562.01	0\\
563.01	0\\
564.01	0\\
565.01	0\\
566.01	0\\
567.01	0\\
568.01	0\\
569.01	0\\
570.01	0\\
571.01	0\\
572.01	0\\
573.01	0\\
574.01	0\\
575.01	0\\
576.01	0\\
577.01	0\\
578.01	0\\
579.01	0\\
580.01	0\\
581.01	0\\
582.01	0\\
583.01	0\\
584.01	0\\
585.01	0\\
586.01	0\\
587.01	0\\
588.01	0\\
589.01	0\\
590.01	0\\
591.01	0\\
592.01	0\\
593.01	0\\
594.01	0\\
595.01	0\\
596.01	0\\
597.01	0\\
598.01	0.00134910574383743\\
599.01	0.00375813090594593\\
599.02	0.00379585558757839\\
599.03	0.00383394724377265\\
599.04	0.00387240947898429\\
599.05	0.00391124593307965\\
599.06	0.00395046028168358\\
599.07	0.00399005623653081\\
599.08	0.00403003754582079\\
599.09	0.00407040799457573\\
599.1	0.00411117140500245\\
599.11	0.00415233163685777\\
599.12	0.00419389258781722\\
599.13	0.00423585819384772\\
599.14	0.00427823242958368\\
599.15	0.00432101930870689\\
599.16	0.00436422288433012\\
599.17	0.00440784724938452\\
599.18	0.0044518965370107\\
599.19	0.00449637492095395\\
599.2	0.00454128661596295\\
599.21	0.00458663587819266\\
599.22	0.00463242700561114\\
599.23	0.00467866433841028\\
599.24	0.00472535225942061\\
599.25	0.0047724951945302\\
599.26	0.00482009761310766\\
599.27	0.00486816402842929\\
599.28	0.00491669899811045\\
599.29	0.00496570712454113\\
599.3	0.0050151930553258\\
599.31	0.00506516148372768\\
599.32	0.00511561714911722\\
599.33	0.00516656483742507\\
599.34	0.00521800938159951\\
599.35	0.00526995566206837\\
599.36	0.00532240860720543\\
599.37	0.00537537319380148\\
599.38	0.00542885444753995\\
599.39	0.00548285743950262\\
599.4	0.0055373872762364\\
599.41	0.00559244911427378\\
599.42	0.00564804816062216\\
599.43	0.00570418967325786\\
599.44	0.00576087896162512\\
599.45	0.00581812138713977\\
599.46	0.00587592236369806\\
599.47	0.00593428735819028\\
599.48	0.00599322189101954\\
599.49	0.00605273153662557\\
599.5	0.00611282192401361\\
599.51	0.00617349873728855\\
599.52	0.0062347677161943\\
599.53	0.00629663465665836\\
599.54	0.00635910541134177\\
599.55	0.00642218589019447\\
599.56	0.00648588206101604\\
599.57	0.00655019995002198\\
599.58	0.0066151456424155\\
599.59	0.00668072528296487\\
599.6	0.00674694507658656\\
599.61	0.00681381128893394\\
599.62	0.00688133024699175\\
599.63	0.00694950833967654\\
599.64	0.00701835201844282\\
599.65	0.00708786779789516\\
599.66	0.00715806225640647\\
599.67	0.00722894203674201\\
599.68	0.00730051384668986\\
599.69	0.00737278445969724\\
599.7	0.00744576071551335\\
599.71	0.0075194495208382\\
599.72	0.00759385784997808\\
599.73	0.00766899274550727\\
599.74	0.00774486131893626\\
599.75	0.00782147075138655\\
599.76	0.00789882829427199\\
599.77	0.00797694126998689\\
599.78	0.00805581707260071\\
599.79	0.00813546316855975\\
599.8	0.00821588709739553\\
599.81	0.00829709647244019\\
599.82	0.0083790989815489\\
599.83	0.00846190238782927\\
599.84	0.008545514530378\\
599.85	0.00862994332502465\\
599.86	0.00871519676508278\\
599.87	0.00880128292210833\\
599.88	0.00888820994666551\\
599.89	0.00897598606910017\\
599.9	0.00906461960032073\\
599.91	0.00915411893258671\\
599.92	0.00924449254030512\\
599.93	0.00933574898083444\\
599.94	0.00942789689529664\\
599.95	0.00952094500939709\\
599.96	0.00961490213425244\\
599.97	0.00970977716722672\\
599.98	0.00980557909277551\\
599.99	0.00990231698329844\\
600	0.01\\
};
\end{axis}
\end{tikzpicture}%
 
%  \caption{Continuous Time w/ nFPC}
%\end{subfigure}%
%\hfill%
%\begin{subfigure}{.45\linewidth}
%  \centering
%  \setlength\figureheight{\linewidth} 
%  \setlength\figurewidth{\linewidth}
%  \tikzsetnextfilename{dp_dscr_nFPC_z1}
%  % This file was created by matlab2tikz.
%
%The latest updates can be retrieved from
%  http://www.mathworks.com/matlabcentral/fileexchange/22022-matlab2tikz-matlab2tikz
%where you can also make suggestions and rate matlab2tikz.
%
\definecolor{mycolor1}{rgb}{0.00000,1.00000,0.14286}%
\definecolor{mycolor2}{rgb}{0.00000,1.00000,0.28571}%
\definecolor{mycolor3}{rgb}{0.00000,1.00000,0.42857}%
\definecolor{mycolor4}{rgb}{0.00000,1.00000,0.57143}%
\definecolor{mycolor5}{rgb}{0.00000,1.00000,0.71429}%
\definecolor{mycolor6}{rgb}{0.00000,1.00000,0.85714}%
\definecolor{mycolor7}{rgb}{0.00000,1.00000,1.00000}%
\definecolor{mycolor8}{rgb}{0.00000,0.87500,1.00000}%
\definecolor{mycolor9}{rgb}{0.00000,0.62500,1.00000}%
\definecolor{mycolor10}{rgb}{0.12500,0.00000,1.00000}%
\definecolor{mycolor11}{rgb}{0.25000,0.00000,1.00000}%
\definecolor{mycolor12}{rgb}{0.37500,0.00000,1.00000}%
\definecolor{mycolor13}{rgb}{0.50000,0.00000,1.00000}%
\definecolor{mycolor14}{rgb}{0.62500,0.00000,1.00000}%
\definecolor{mycolor15}{rgb}{0.75000,0.00000,1.00000}%
\definecolor{mycolor16}{rgb}{0.87500,0.00000,1.00000}%
\definecolor{mycolor17}{rgb}{1.00000,0.00000,1.00000}%
\definecolor{mycolor18}{rgb}{1.00000,0.00000,0.87500}%
\definecolor{mycolor19}{rgb}{1.00000,0.00000,0.62500}%
\definecolor{mycolor20}{rgb}{0.85714,0.00000,0.00000}%
\definecolor{mycolor21}{rgb}{0.71429,0.00000,0.00000}%
%
\begin{tikzpicture}

\begin{axis}[%
width=4.1in,
height=3.803in,
at={(0.809in,0.513in)},
scale only axis,
point meta min=0,
point meta max=1,
every outer x axis line/.append style={black},
every x tick label/.append style={font=\color{black}},
xmin=0,
xmax=600,
every outer y axis line/.append style={black},
every y tick label/.append style={font=\color{black}},
ymin=0,
ymax=0.014,
axis background/.style={fill=white},
axis x line*=bottom,
axis y line*=left,
colormap={mymap}{[1pt] rgb(0pt)=(0,1,0); rgb(7pt)=(0,1,1); rgb(15pt)=(0,0,1); rgb(23pt)=(1,0,1); rgb(31pt)=(1,0,0); rgb(38pt)=(0,0,0)},
colorbar,
colorbar style={separate axis lines,every outer x axis line/.append style={black},every x tick label/.append style={font=\color{black}},every outer y axis line/.append style={black},every y tick label/.append style={font=\color{black}},yticklabels={{-19},{-17},{-15},{-13},{-11},{-9},{-7},{-5},{-3},{-1},{1},{3},{5},{7},{9},{11},{13},{15},{17},{19}}}
]
\addplot [color=green,solid,forget plot]
  table[row sep=crcr]{%
1	0.0124897681842806\\
2	0.0124897499947743\\
3	0.0124897313920624\\
4	0.0124897123667764\\
5	0.0124896929093365\\
6	0.0124896730099469\\
7	0.012489652658591\\
8	0.0124896318450267\\
9	0.0124896105587808\\
10	0.0124895887891448\\
11	0.0124895665251686\\
12	0.0124895437556562\\
13	0.0124895204691594\\
14	0.0124894966539725\\
15	0.012489472298127\\
16	0.0124894473893851\\
17	0.0124894219152343\\
18	0.012489395862881\\
19	0.0124893692192444\\
20	0.0124893419709503\\
21	0.0124893141043243\\
22	0.0124892856053857\\
23	0.0124892564598401\\
24	0.0124892266530732\\
25	0.0124891961701431\\
26	0.0124891649957736\\
27	0.0124891331143468\\
28	0.0124891005098951\\
29	0.0124890671660944\\
30	0.0124890330662554\\
31	0.0124889981933163\\
32	0.012488962529834\\
33	0.0124889260579766\\
34	0.012488888759514\\
35	0.0124888506158097\\
36	0.012488811607812\\
37	0.0124887717160446\\
38	0.0124887309205976\\
39	0.0124886892011177\\
40	0.012488646536799\\
41	0.0124886029063727\\
42	0.0124885582880974\\
43	0.0124885126597483\\
44	0.012488465998607\\
45	0.0124884182814507\\
46	0.012488369484541\\
47	0.0124883195836128\\
48	0.0124882685538628\\
49	0.0124882163699377\\
50	0.0124881630059218\\
51	0.0124881084353252\\
52	0.012488052631071\\
53	0.0124879955654823\\
54	0.012487937210269\\
55	0.0124878775365142\\
56	0.0124878165146607\\
57	0.0124877541144963\\
58	0.0124876903051394\\
59	0.0124876250550243\\
60	0.0124875583318856\\
61	0.0124874901027424\\
62	0.0124874203338824\\
63	0.0124873489908453\\
64	0.0124872760384054\\
65	0.0124872014405546\\
66	0.0124871251604837\\
67	0.0124870471605643\\
68	0.0124869674023295\\
69	0.0124868858464539\\
70	0.0124868024527337\\
71	0.0124867171800652\\
72	0.012486629986423\\
73	0.0124865408288378\\
74	0.012486449663373\\
75	0.0124863564451002\\
76	0.0124862611280743\\
77	0.0124861636653076\\
78	0.0124860640087424\\
79	0.0124859621092229\\
80	0.0124858579164657\\
81	0.0124857513790295\\
82	0.0124856424442823\\
83	0.0124855310583685\\
84	0.0124854171661736\\
85	0.0124853007112872\\
86	0.0124851816359643\\
87	0.0124850598810849\\
88	0.0124849353861111\\
89	0.0124848080890416\\
90	0.0124846779263645\\
91	0.0124845448330064\\
92	0.0124844087422797\\
93	0.0124842695858253\\
94	0.0124841272935532\\
95	0.012483981793578\\
96	0.0124838330121518\\
97	0.0124836808735911\\
98	0.0124835253001996\\
99	0.0124833662121859\\
100	0.0124832035275752\\
101	0.0124830371621138\\
102	0.0124828670291687\\
103	0.0124826930396173\\
104	0.0124825151017309\\
105	0.0124823331210479\\
106	0.0124821470002375\\
107	0.0124819566389527\\
108	0.0124817619336714\\
109	0.012481562777524\\
110	0.0124813590601073\\
111	0.0124811506672822\\
112	0.012480937480954\\
113	0.0124807193788339\\
114	0.0124804962341784\\
115	0.0124802679155066\\
116	0.0124800342862908\\
117	0.0124797952046179\\
118	0.0124795505228206\\
119	0.0124793000870714\\
120	0.0124790437369384\\
121	0.0124787813048972\\
122	0.0124785126157933\\
123	0.0124782374862508\\
124	0.0124779557240191\\
125	0.0124776671272519\\
126	0.0124773714837087\\
127	0.0124770685698707\\
128	0.0124767581499585\\
129	0.0124764399748413\\
130	0.0124761137808209\\
131	0.0124757792882774\\
132	0.0124754362001546\\
133	0.0124750842002665\\
134	0.0124747229513997\\
135	0.012474352093183\\
136	0.012473971239693\\
137	0.0124735799767584\\
138	0.0124731778589205\\
139	0.0124727644060008\\
140	0.0124723390992182\\
141	0.0124719013767853\\
142	0.0124714506288953\\
143	0.0124709861919651\\
144	0.0124705073418964\\
145	0.0124700132858296\\
146	0.0124695031510399\\
147	0.0124689759672094\\
148	0.0124684306310268\\
149	0.0124678658199048\\
150	0.0124672797549955\\
151	0.0124666695506285\\
152	0.0124655220176865\\
153	0.0124643209304721\\
154	0.0124631068994203\\
155	0.012461879779158\\
156	0.012460639422368\\
157	0.0124593856797469\\
158	0.0124581183999614\\
159	0.0124568374296027\\
160	0.0124555426131394\\
161	0.0124542337928678\\
162	0.012452910808861\\
163	0.0124515734989151\\
164	0.0124502216984937\\
165	0.0124488552406697\\
166	0.0124474739560649\\
167	0.0124460776727863\\
168	0.0124446662163604\\
169	0.0124432394096641\\
170	0.0124417970728524\\
171	0.0124403390232828\\
172	0.0124388650754367\\
173	0.0124373750408364\\
174	0.0124358687279584\\
175	0.0124343459421428\\
176	0.0124328064854981\\
177	0.0124312501568013\\
178	0.0124296767513929\\
179	0.0124280860610673\\
180	0.0124264778739565\\
181	0.0124248519744094\\
182	0.0124232081428632\\
183	0.0124215461557094\\
184	0.012419865785152\\
185	0.0124181667990585\\
186	0.0124164489608024\\
187	0.012414712029098\\
188	0.012412955757825\\
189	0.0124111798958441\\
190	0.0124093841868022\\
191	0.0124075683689261\\
192	0.0124057321748046\\
193	0.0124038753311582\\
194	0.0124019975585945\\
195	0.0124000985713498\\
196	0.0123981780770149\\
197	0.012396235776244\\
198	0.0123942713624457\\
199	0.0123922845214549\\
200	0.0123902749311838\\
201	0.0123882422612506\\
202	0.0123861861725835\\
203	0.0123841063169999\\
204	0.0123820023367559\\
205	0.0123798738640662\\
206	0.0123777205205909\\
207	0.0123755419168863\\
208	0.012373337651816\\
209	0.0123711073119209\\
210	0.0123688504707413\\
211	0.01236656668809\\
212	0.0123642555092687\\
213	0.0123619164642245\\
214	0.0123595490666397\\
215	0.0123571528129485\\
216	0.0123547271812729\\
217	0.0123522716302694\\
218	0.0123497855978778\\
219	0.0123472684999595\\
220	0.0123447197288151\\
221	0.0123421386515643\\
222	0.0123395246083695\\
223	0.0123368769104764\\
224	0.0123341948380201\\
225	0.012331477637488\\
226	0.0123287245185589\\
227	0.0123259346495284\\
228	0.0123231071489932\\
229	0.0123202410667435\\
230	0.0123173353319925\\
231	0.0123143885998255\\
232	0.012311398776063\\
233	0.0123083615872519\\
234	0.0123052684376352\\
235	0.0123021313433887\\
236	0.0122989492523277\\
237	0.0122957208417754\\
238	0.0122924447005439\\
239	0.0122891193196325\\
240	0.0122857430818107\\
241	0.0122823142501185\\
242	0.0122788309555883\\
243	0.0122752911851106\\
244	0.0122716927717539\\
245	0.0122680333929044\\
246	0.0122643105881803\\
247	0.0122605218229948\\
248	0.0122566646517401\\
249	0.0122527370840316\\
250	0.0122487382931443\\
251	0.0122446693060966\\
252	0.0122405279664819\\
253	0.0122362918059262\\
254	0.0122319544356621\\
255	0.012227508575915\\
256	0.0122229457722666\\
257	0.0122182556703026\\
258	0.0122131454626804\\
259	0.0122062744447709\\
260	0.0121993190353875\\
261	0.0121922780549319\\
262	0.012185150285039\\
263	0.0121779344268315\\
264	0.0121706289758907\\
265	0.0121632318755543\\
266	0.0121557401381313\\
267	0.0121481559197659\\
268	0.0121404785863628\\
269	0.0121327068436496\\
270	0.0121248393777778\\
271	0.012116874855248\\
272	0.0121088119228444\\
273	0.0121006492074875\\
274	0.0120923853156354\\
275	0.0120840188307729\\
276	0.0120755483030845\\
277	0.0120669722061883\\
278	0.0120582887441587\\
279	0.0120494948788807\\
280	0.0120405801522957\\
281	0.0120315452171778\\
282	0.0120223999677997\\
283	0.0120131434828155\\
284	0.0120037753301137\\
285	0.0119942959563901\\
286	0.0119847063235465\\
287	0.0119749990802778\\
288	0.0119651713396973\\
289	0.0119552250222492\\
290	0.0119451697753187\\
291	0.0119350259652305\\
292	0.0119247492573933\\
293	0.0119143380895385\\
294	0.0119037909564472\\
295	0.0118931064255736\\
296	0.0118822831554732\\
297	0.0118713199172857\\
298	0.0118602156188022\\
299	0.0118489693274923\\
300	0.0118375802751465\\
301	0.0118260477619261\\
302	0.0118143705196092\\
303	0.0118025425611362\\
304	0.0117905725623206\\
305	0.0117784636773141\\
306	0.0117662178431268\\
307	0.0117538377931612\\
308	0.011741327249895\\
309	0.0117286912184618\\
310	0.0117159365391645\\
311	0.0117030732499434\\
312	0.0116901190741196\\
313	0.0116770867132529\\
314	0.0116639745525467\\
315	0.0116569764559215\\
316	0.011651037953093\\
317	0.0116450685689349\\
318	0.0116390703016166\\
319	0.0116330453323779\\
320	0.0116269960005527\\
321	0.0116209248376083\\
322	0.0116148346067817\\
323	0.0116087283242837\\
324	0.0116026092856854\\
325	0.0115964811017978\\
326	0.011590347746216\\
327	0.0115842136682991\\
328	0.0115780837950509\\
329	0.0115719635163849\\
330	0.0115658587290047\\
331	0.0115597758843811\\
332	0.0115537220401241\\
333	0.011547704906536\\
334	0.0115417328920142\\
335	0.0115358152154615\\
336	0.0115299619918528\\
337	0.0115241843294441\\
338	0.0115184944422059\\
339	0.0115129057842033\\
340	0.01150743323468\\
341	0.0115020932665117\\
342	0.011496904043422\\
343	0.0114918856144818\\
344	0.0114870601341707\\
345	0.0114824521110562\\
346	0.0114780886866855\\
347	0.0114739999391547\\
348	0.0114702191719366\\
349	0.0114667568316486\\
350	0.0114633132849764\\
351	0.0114598906781047\\
352	0.0114564912071508\\
353	0.0114531171098194\\
354	0.0114497706511324\\
355	0.0114464541090349\\
356	0.0114431697585837\\
357	0.0114399198534554\\
358	0.0114367066029506\\
359	0.011433532143429\\
360	0.0114303985036616\\
361	0.0114273075627111\\
362	0.0114242609980642\\
363	0.0114212602230584\\
364	0.0114183063149119\\
365	0.0114153999288342\\
366	0.01141254119559\\
367	0.0114097295994013\\
368	0.0114069638325477\\
369	0.0114042416226381\\
370	0.0114015595285162\\
371	0.0113989126942821\\
372	0.0113962945554893\\
373	0.0113936964877304\\
374	0.0113911073853027\\
375	0.01138851473969\\
376	0.0113859181971972\\
377	0.0113833173119853\\
378	0.011380711538328\\
379	0.0113781002228792\\
380	0.0113754825971527\\
381	0.0113728577705123\\
382	0.0113702247239388\\
383	0.0113675823049288\\
384	0.0113649292240269\\
385	0.0113622640536292\\
386	0.0113595852298662\\
387	0.0113568910585767\\
388	0.011354179726601\\
389	0.0113514493200365\\
390	0.0113486978515352\\
391	0.0113459232992453\\
392	0.0113431236607245\\
393	0.0113402970259769\\
394	0.0113374416746488\\
395	0.0113345562038304\\
396	0.0113316396945492\\
397	0.0113286912786436\\
398	0.0113257100640905\\
399	0.0113226951375787\\
400	0.0113196455675965\\
401	0.0113165604080537\\
402	0.011313438702445\\
403	0.0113102794885385\\
404	0.0113070818035454\\
405	0.0113038446896826\\
406	0.0113005671999783\\
407	0.011297248404088\\
408	0.0112938873937712\\
409	0.0112904832875337\\
410	0.0112870352337436\\
411	0.0112835424112656\\
412	0.0112800040263188\\
413	0.0112764193038087\\
414	0.0112727874708002\\
415	0.0112691077489393\\
416	0.0112653793550945\\
417	0.0112616015019377\\
418	0.0112577733984404\\
419	0.0112538942502539\\
420	0.0112499632599395\\
421	0.011245979627018\\
422	0.0112419425478029\\
423	0.0112378512149911\\
424	0.0112337048169925\\
425	0.0112295025369944\\
426	0.0112252435517858\\
427	0.0112209270304042\\
428	0.0112165521327284\\
429	0.0112121180082258\\
430	0.0112076237951866\\
431	0.0112030686204205\\
432	0.0111984515989128\\
433	0.0111937718334402\\
434	0.0111890284141468\\
435	0.0111842204180812\\
436	0.0111793469087012\\
437	0.0111744069353487\\
438	0.0111693995327043\\
439	0.0111643237202288\\
440	0.0111591785016042\\
441	0.0111539628641816\\
442	0.0111486757784466\\
443	0.0111433161975055\\
444	0.0111378830565872\\
445	0.0111323752725393\\
446	0.0111267917433053\\
447	0.0111211313473823\\
448	0.0111153929432602\\
449	0.0111095753688422\\
450	0.0111036774408473\\
451	0.0110976979541931\\
452	0.0110916356813608\\
453	0.01108548937174\\
454	0.0110792577509524\\
455	0.0110729395201534\\
456	0.011066533355308\\
457	0.0110600379064397\\
458	0.011053451796849\\
459	0.0110467736222994\\
460	0.011040001950171\\
461	0.0110331353185774\\
462	0.011026172235447\\
463	0.0110191111775638\\
464	0.0110119505895676\\
465	0.0110046888829107\\
466	0.0109973244347679\\
467	0.0109898555868982\\
468	0.0109822806444541\\
469	0.0109745978747372\\
470	0.0109668055058956\\
471	0.0109589017255597\\
472	0.0109508846794146\\
473	0.0109427524697038\\
474	0.0109345031536602\\
475	0.0109261347418621\\
476	0.010917645196507\\
477	0.0109090324296014\\
478	0.0109002943010583\\
479	0.0108914286167004\\
480	0.0108824331261609\\
481	0.0108733055206763\\
482	0.0108640434307657\\
483	0.010854644423789\\
484	0.0108451060013766\\
485	0.0108354255967231\\
486	0.0108256005717376\\
487	0.0108156282140407\\
488	0.0108055057338\\
489	0.0107952302603946\\
490	0.0107847988388982\\
491	0.0107742084263705\\
492	0.0107634558879466\\
493	0.0107525379927126\\
494	0.0107414514093556\\
495	0.0107301927015777\\
496	0.0107187583232606\\
497	0.0107071446133703\\
498	0.0106953477905887\\
499	0.010683363947663\\
500	0.0106711890454607\\
501	0.0106588189067216\\
502	0.0106462492094991\\
503	0.010633475480284\\
504	0.01062049308681\\
505	0.010607297230538\\
506	0.010593882938827\\
507	0.0105802450567994\\
508	0.0105663782389195\\
509	0.0105522730167099\\
510	0.0105379282758502\\
511	0.0105233398131128\\
512	0.0105084982627157\\
513	0.0104933982717844\\
514	0.0104780387368352\\
515	0.0104624117159062\\
516	0.0104465168741575\\
517	0.010430338110904\\
518	0.0104138601073215\\
519	0.0103970752891333\\
520	0.0103799740894401\\
521	0.0103625472625547\\
522	0.0103447860033705\\
523	0.0103266810616174\\
524	0.0103082228900916\\
525	0.010289391268734\\
526	0.0102701819114095\\
527	0.0102506040189008\\
528	0.0102306249539969\\
529	0.0102102812552429\\
530	0.0101895704585159\\
531	0.0101684943826255\\
532	0.0101469696050343\\
533	0.0101249805933305\\
534	0.0101025062038\\
535	0.0100795317365609\\
536	0.0100560416729332\\
537	0.0100320209551519\\
538	0.0100074541320563\\
539	0.00998232879437465\\
540	0.00995661836385641\\
541	0.00993030072187439\\
542	0.00990331483917391\\
543	0.00987561760665483\\
544	0.00984723732256093\\
545	0.00981830109818854\\
546	0.009789145095759\\
547	0.00975939644714741\\
548	0.00972891729681173\\
549	0.00969767543412559\\
550	0.00966564626747568\\
551	0.00963280526397163\\
552	0.00959912703850655\\
553	0.00956458512770733\\
554	0.00952915185033005\\
555	0.00949279808571401\\
556	0.00945506717019782\\
557	0.00941596384115013\\
558	0.0093765693029433\\
559	0.00933810399915141\\
560	0.00929866190376292\\
561	0.00925816686011643\\
562	0.00921657377355508\\
563	0.00915714596620506\\
564	0.00888374711558651\\
565	0.00850308588273132\\
566	0.00844099822861057\\
567	0.0083778796614207\\
568	0.00831365319536097\\
569	0.00824828895484398\\
570	0.00818175575864172\\
571	0.00811402081575833\\
572	0.00804504960172116\\
573	0.00797480573046632\\
574	0.0079032508165112\\
575	0.00783034432612527\\
576	0.00775604341652007\\
577	0.00768030276197866\\
578	0.00760307436565794\\
579	0.00752430735533364\\
580	0.00744394776027091\\
581	0.00736193826362867\\
582	0.00727821791754122\\
583	0.00719272178875226\\
584	0.00710538045132748\\
585	0.00701611910584812\\
586	0.00692485573802265\\
587	0.00683149674994525\\
588	0.00673592587771182\\
589	0.00663797520465071\\
590	0.00653734834917362\\
591	0.00643341581933105\\
592	0.00632466858477865\\
593	0.00620725772627072\\
594	0.0060710901180166\\
595	0.00588956027306328\\
596	0.00559184182825265\\
597	0.00498881296807123\\
598	0.00357511483354343\\
599	0\\
600	0\\
};
\addplot [color=mycolor1,solid,forget plot]
  table[row sep=crcr]{%
1	0.01248982130969\\
2	0.0124898047810677\\
3	0.012489787892337\\
4	0.0124897706358697\\
5	0.0124897530038844\\
6	0.0124897349884446\\
7	0.0124897165814558\\
8	0.0124896977746626\\
9	0.012489678559646\\
10	0.0124896589278208\\
11	0.0124896388704325\\
12	0.0124896183785548\\
13	0.0124895974430862\\
14	0.0124895760547478\\
15	0.0124895542040798\\
16	0.0124895318814389\\
17	0.0124895090769953\\
18	0.0124894857807294\\
19	0.0124894619824294\\
20	0.0124894376716878\\
21	0.0124894128378984\\
22	0.0124893874702536\\
23	0.0124893615577411\\
24	0.0124893350891407\\
25	0.0124893080530215\\
26	0.0124892804377389\\
27	0.012489252231431\\
28	0.0124892234220162\\
29	0.0124891939971895\\
30	0.0124891639444198\\
31	0.0124891332509468\\
32	0.0124891019037778\\
33	0.0124890698896848\\
34	0.0124890371952012\\
35	0.012489003806619\\
36	0.012488969709986\\
37	0.0124889348911024\\
38	0.012488899335518\\
39	0.0124888630285294\\
40	0.0124888259551773\\
41	0.0124887881002431\\
42	0.0124887494482467\\
43	0.0124887099834434\\
44	0.0124886696898215\\
45	0.0124886285510993\\
46	0.012488586550723\\
47	0.0124885436718638\\
48	0.0124884998974159\\
49	0.0124884552099938\\
50	0.01248840959193\\
51	0.0124883630252734\\
52	0.0124883154917866\\
53	0.0124882669729444\\
54	0.0124882174499317\\
55	0.0124881669036418\\
56	0.0124881153146748\\
57	0.0124880626633362\\
58	0.0124880089296354\\
59	0.0124879540932847\\
60	0.012487898133698\\
61	0.0124878410299899\\
62	0.0124877827609752\\
63	0.0124877233051679\\
64	0.0124876626407812\\
65	0.0124876007457268\\
66	0.0124875375976154\\
67	0.0124874731737561\\
68	0.0124874074511573\\
69	0.012487340406527\\
70	0.0124872720162735\\
71	0.0124872022565061\\
72	0.0124871311030369\\
73	0.0124870585313812\\
74	0.0124869845167597\\
75	0.0124869090341002\\
76	0.0124868320580392\\
77	0.0124867535629247\\
78	0.0124866735228179\\
79	0.0124865919114967\\
80	0.0124865087024577\\
81	0.01248642386892\\
82	0.0124863373838279\\
83	0.0124862492198548\\
84	0.0124861593494067\\
85	0.012486067744626\\
86	0.0124859743773959\\
87	0.0124858792193441\\
88	0.0124857822418478\\
89	0.0124856834160377\\
90	0.012485582712803\\
91	0.012485480102796\\
92	0.0124853755564371\\
93	0.0124852690439197\\
94	0.0124851605352154\\
95	0.0124850500000788\\
96	0.0124849374080525\\
97	0.0124848227284729\\
98	0.0124847059304741\\
99	0.012484586982994\\
100	0.0124844658547785\\
101	0.0124843425143864\\
102	0.0124842169301945\\
103	0.0124840890704017\\
104	0.0124839589030336\\
105	0.0124838263959465\\
106	0.0124836915168317\\
107	0.0124835542332191\\
108	0.0124834145124811\\
109	0.0124832723218357\\
110	0.0124831276283501\\
111	0.0124829803989441\\
112	0.0124828306003928\\
113	0.0124826781993304\\
114	0.012482523162253\\
115	0.012482365455523\\
116	0.0124822050453726\\
117	0.0124820418979092\\
118	0.0124818759791209\\
119	0.0124817072548838\\
120	0.0124815356909711\\
121	0.0124813612530642\\
122	0.0124811839067668\\
123	0.012481003617623\\
124	0.0124808203511399\\
125	0.0124806340728157\\
126	0.0124804447481754\\
127	0.0124802523428154\\
128	0.0124800568224586\\
129	0.0124798581530236\\
130	0.0124796563007091\\
131	0.0124794512321\\
132	0.0124792429142964\\
133	0.0124790313150731\\
134	0.0124788164030755\\
135	0.0124785981480597\\
136	0.0124783765211862\\
137	0.0124781514953799\\
138	0.0124779230457703\\
139	0.0124776911502283\\
140	0.012477455790022\\
141	0.0124772169506114\\
142	0.0124769746226051\\
143	0.0124767288028848\\
144	0.0124764794958574\\
145	0.0124762267146486\\
146	0.0124759704816643\\
147	0.0124757108269577\\
148	0.0124754477806711\\
149	0.0124751813528551\\
150	0.0124749115024053\\
151	0.0124746382066063\\
152	0.012474362172165\\
153	0.012474083429475\\
154	0.0124738019515424\\
155	0.0124735177109414\\
156	0.0124732306797966\\
157	0.0124729408297643\\
158	0.0124726481320126\\
159	0.0124723525572006\\
160	0.0124720540754561\\
161	0.0124717526563525\\
162	0.0124714482688839\\
163	0.0124711408814397\\
164	0.0124708304617764\\
165	0.0124705169769896\\
166	0.0124702003934824\\
167	0.0124698806769344\\
168	0.0124695577922667\\
169	0.012469231703607\\
170	0.012468902374251\\
171	0.0124685697666232\\
172	0.0124682338422346\\
173	0.0124678945616387\\
174	0.0124675518843843\\
175	0.012467205768967\\
176	0.0124668561727765\\
177	0.0124665030520427\\
178	0.0124661463617774\\
179	0.012465786055714\\
180	0.012465422086243\\
181	0.0124650544043449\\
182	0.0124646829595185\\
183	0.0124643076997062\\
184	0.0124639285712149\\
185	0.0124635455186322\\
186	0.0124631584847388\\
187	0.0124627674104151\\
188	0.0124623722345443\\
189	0.0124619728939081\\
190	0.0124615693230784\\
191	0.0124611614543021\\
192	0.0124607492173796\\
193	0.0124603325395366\\
194	0.012459911345289\\
195	0.0124594855562993\\
196	0.0124590550912255\\
197	0.0124586198655612\\
198	0.012458179791466\\
199	0.0124577347775864\\
200	0.0124572847288657\\
201	0.0124568295463427\\
202	0.0124563691269379\\
203	0.0124559033632271\\
204	0.0124554321431998\\
205	0.0124549553500034\\
206	0.0124544728616703\\
207	0.0124539845508267\\
208	0.0124534902843832\\
209	0.012452989923203\\
210	0.012452483321748\\
211	0.0124519703276998\\
212	0.0124514507815527\\
213	0.0124509245161775\\
214	0.0124503913563522\\
215	0.0124498511182564\\
216	0.0124493036089256\\
217	0.0124487486256623\\
218	0.0124481859553975\\
219	0.012447615373998\\
220	0.0124470366455137\\
221	0.0124464495213551\\
222	0.0124458537393924\\
223	0.0124452490229572\\
224	0.0124446350797139\\
225	0.0124440116003234\\
226	0.0124433782567031\\
227	0.0124427346993644\\
228	0.0124420805524397\\
229	0.0124414154027706\\
230	0.01244073877421\\
231	0.0124400500696235\\
232	0.0124393484737726\\
233	0.012438633038978\\
234	0.012437904250913\\
235	0.012437161548183\\
236	0.0124364043130293\\
237	0.0124356318819047\\
238	0.0124348435408024\\
239	0.0124340385200036\\
240	0.0124332159881911\\
241	0.0124323750459106\\
242	0.0124315147184497\\
243	0.0124306339484011\\
244	0.0124297315886051\\
245	0.0124288063970172\\
246	0.0124278570365894\\
247	0.0124268820854545\\
248	0.0124258800631654\\
249	0.0124248494635782\\
250	0.0124237886938405\\
251	0.0124226955095855\\
252	0.0124215660834259\\
253	0.0124203975479223\\
254	0.0124191866615405\\
255	0.0124179296989049\\
256	0.0124166221971726\\
257	0.0124152581972048\\
258	0.0124135998771043\\
259	0.0124105464902129\\
260	0.0124074533112881\\
261	0.0124043196111587\\
262	0.0124011446274988\\
263	0.0123979275498318\\
264	0.01239466749984\\
265	0.012391363575555\\
266	0.0123880152332381\\
267	0.0123846216440767\\
268	0.0123811818915475\\
269	0.0123776950252452\\
270	0.0123741600587974\\
271	0.0123705759675712\\
272	0.0123669416860765\\
273	0.0123632561048083\\
274	0.0123595180657178\\
275	0.0123557263536714\\
276	0.0123518796749791\\
277	0.0123479765922832\\
278	0.0123440153100879\\
279	0.0123399929826119\\
280	0.012335904356084\\
281	0.012331749671693\\
282	0.0123275323610815\\
283	0.0123232508899895\\
284	0.0123189036977558\\
285	0.0123144890652403\\
286	0.0123100046228365\\
287	0.0123054485013464\\
288	0.0123008195614031\\
289	0.0122961181720901\\
290	0.0122913480976104\\
291	0.012286513195336\\
292	0.0122815914432025\\
293	0.0122765797181411\\
294	0.0122714746760928\\
295	0.0122662727285626\\
296	0.0122609700157525\\
297	0.0122555623751948\\
298	0.0122500453034464\\
299	0.0122444139040322\\
300	0.012238662800327\\
301	0.0122327859450087\\
302	0.0122267761397827\\
303	0.0122206244342492\\
304	0.0122143269975641\\
305	0.0122078763474986\\
306	0.0122012632738793\\
307	0.0121944775512155\\
308	0.0121875078110607\\
309	0.0121803414402418\\
310	0.0121729645878634\\
311	0.0121653624380007\\
312	0.0121575193403631\\
313	0.0121494114067925\\
314	0.0121410012302823\\
315	0.0121289991707545\\
316	0.0121162237446487\\
317	0.0121032760675475\\
318	0.0120901510109772\\
319	0.0120768300389825\\
320	0.0120633227376904\\
321	0.0120496380803449\\
322	0.0120357743204656\\
323	0.0120217303333258\\
324	0.012007506849554\\
325	0.0119931106168703\\
326	0.0119785707010117\\
327	0.0119638612863687\\
328	0.0119489529131119\\
329	0.0119338434367098\\
330	0.011918530725327\\
331	0.0119030122789395\\
332	0.0118872829361935\\
333	0.011871337033803\\
334	0.0118551852840994\\
335	0.0118388271871127\\
336	0.0118222628867497\\
337	0.0118054937113584\\
338	0.0117885238166105\\
339	0.011771366243537\\
340	0.01175402276002\\
341	0.0117364733917679\\
342	0.0117187228969905\\
343	0.0117007776331837\\
344	0.0116826458992591\\
345	0.0116643383754695\\
346	0.011645868740675\\
347	0.0116272547205502\\
348	0.0116085206182026\\
349	0.0115905098056133\\
350	0.0115827699634876\\
351	0.0115750160182305\\
352	0.0115672532899309\\
353	0.0115594876452058\\
354	0.0115517256661517\\
355	0.0115439746198808\\
356	0.0115362424670674\\
357	0.0115285378968242\\
358	0.0115208704188005\\
359	0.0115132504750047\\
360	0.0115056895429897\\
361	0.0114982002618669\\
362	0.0114907966116018\\
363	0.0114834940803675\\
364	0.0114763097184594\\
365	0.0114692623095194\\
366	0.0114623725629731\\
367	0.0114556633299413\\
368	0.0114491598433331\\
369	0.0114428899759606\\
370	0.0114368844969976\\
371	0.0114311774964501\\
372	0.011425806796031\\
373	0.0114208144276655\\
374	0.0114162472061001\\
375	0.0114121109389486\\
376	0.0114080082149506\\
377	0.0114039419716387\\
378	0.011399915135869\\
379	0.0113959305946254\\
380	0.0113919911587714\\
381	0.0113880995168144\\
382	0.0113842581813307\\
383	0.0113804694262423\\
384	0.0113767352115892\\
385	0.0113730570934725\\
386	0.0113694361164496\\
387	0.0113658726853779\\
388	0.0113623664141238\\
389	0.0113589159438883\\
390	0.0113555187249788\\
391	0.0113521707553167\\
392	0.0113488662655676\\
393	0.0113455973413804\\
394	0.0113423534754387\\
395	0.0113391210316181\\
396	0.0113358826040086\\
397	0.0113326349535495\\
398	0.0113293771781688\\
399	0.0113261082255134\\
400	0.0113228268844138\\
401	0.0113195317775774\\
402	0.0113162213560974\\
403	0.0113128938965208\\
404	0.0113095475013784\\
405	0.0113061801043885\\
406	0.0113027894819316\\
407	0.0112993732727906\\
408	0.0112959290088091\\
409	0.0112924541594801\\
410	0.0112889461943988\\
411	0.0112854026685634\\
412	0.0112818213366601\\
413	0.0112782003041048\\
414	0.0112745382245752\\
415	0.0112708339878973\\
416	0.0112670864592815\\
417	0.0112632944834878\\
418	0.0112594568896728\\
419	0.0112555724969145\\
420	0.0112516401203762\\
421	0.0112476585780188\\
422	0.0112436266976918\\
423	0.0112395433243453\\
424	0.0112354073269749\\
425	0.0112312176047389\\
426	0.0112269730914619\\
427	0.0112226727574288\\
428	0.0112183156069769\\
429	0.0112139006698625\\
430	0.0112094269836878\\
431	0.0112048935786911\\
432	0.0112002994783756\\
433	0.0111956437000278\\
434	0.0111909252550896\\
435	0.0111861431493417\\
436	0.0111812963828571\\
437	0.0111763839496849\\
438	0.011171404837229\\
439	0.0111663580253013\\
440	0.0111612424848455\\
441	0.0111560571763681\\
442	0.0111508010481581\\
443	0.0111454730344616\\
444	0.0111400720538851\\
445	0.0111345970084639\\
446	0.0111290467831078\\
447	0.0111234202449903\\
448	0.0111177162428817\\
449	0.0111119336064288\\
450	0.0111060711453838\\
451	0.0111001276487916\\
452	0.0110941018841419\\
453	0.011087992596497\\
454	0.0110817985076091\\
455	0.0110755183150369\\
456	0.0110691506912718\\
457	0.011062694282878\\
458	0.0110561477096353\\
459	0.0110495095636552\\
460	0.011042778408461\\
461	0.0110359527780285\\
462	0.0110290311757904\\
463	0.0110220120736005\\
464	0.0110148939106599\\
465	0.0110076750924007\\
466	0.0110003539893278\\
467	0.010992928935815\\
468	0.0109853982288538\\
469	0.0109777601267485\\
470	0.0109700128477561\\
471	0.0109621545686647\\
472	0.0109541834233072\\
473	0.0109460975010078\\
474	0.0109378948449566\\
475	0.0109295734505097\\
476	0.0109211312634098\\
477	0.0109125661779232\\
478	0.0109038760348892\\
479	0.0108950586196768\\
480	0.0108861116600428\\
481	0.0108770328238872\\
482	0.0108678197168994\\
483	0.0108584698800903\\
484	0.010848980787203\\
485	0.0108393498419968\\
486	0.0108295743753977\\
487	0.0108196516425083\\
488	0.010809578819469\\
489	0.010799353000165\\
490	0.0107889711927688\\
491	0.0107784303161116\\
492	0.010767727195875\\
493	0.0107568585605932\\
494	0.0107458210374583\\
495	0.0107346111479178\\
496	0.0107232253030569\\
497	0.0107116597987542\\
498	0.0106999108106023\\
499	0.0106879743885845\\
500	0.0106758464514961\\
501	0.0106635227811036\\
502	0.0106509990160317\\
503	0.0106382706453697\\
504	0.0106253330019917\\
505	0.0106121812555819\\
506	0.0105988104053619\\
507	0.0105852152725145\\
508	0.0105713904923051\\
509	0.0105573305058982\\
510	0.0105430294943534\\
511	0.0105284814171098\\
512	0.0105136777412571\\
513	0.0104986126775228\\
514	0.010483284269303\\
515	0.0104676848722985\\
516	0.0104518031448841\\
517	0.0104356359690962\\
518	0.0104191789036504\\
519	0.0104024228651959\\
520	0.010385367415071\\
521	0.010367993787575\\
522	0.010350286570482\\
523	0.0103322369326052\\
524	0.0103138343349133\\
525	0.0102950680180483\\
526	0.010275927693054\\
527	0.0102564017972843\\
528	0.0102364586923393\\
529	0.0102161261957729\\
530	0.0101953841960509\\
531	0.0101742141241981\\
532	0.0101526393269767\\
533	0.0101306616217065\\
534	0.0101082692106229\\
535	0.0100853868658679\\
536	0.0100619982790481\\
537	0.0100380807490736\\
538	0.0100136185087147\\
539	0.00998859739201236\\
540	0.00996299658999363\\
541	0.00993678699714986\\
542	0.00990995441930058\\
543	0.0098824840583619\\
544	0.00985426557139988\\
545	0.00982538631410531\\
546	0.00979576034912938\\
547	0.00976557228047118\\
548	0.00973513952339307\\
549	0.00970404442958007\\
550	0.0096721805993989\\
551	0.00963951287670261\\
552	0.00960601403348662\\
553	0.00957165740280535\\
554	0.00953641548189309\\
555	0.00950025959413018\\
556	0.00946315979362414\\
557	0.00942476702323156\\
558	0.00938505629975076\\
559	0.00934423285200159\\
560	0.00930470285384079\\
561	0.00926443506641366\\
562	0.00922308762457783\\
563	0.00918061162809973\\
564	0.00910086899317433\\
565	0.00882974345939778\\
566	0.00844143787727468\\
567	0.0083778817235865\\
568	0.00831365328902626\\
569	0.0082482889682748\\
570	0.00818175576141668\\
571	0.00811402081654658\\
572	0.0080450496020305\\
573	0.00797480573061697\\
574	0.00790325081658399\\
575	0.00783034432616349\\
576	0.00775604341653988\\
577	0.00768030276198882\\
578	0.00760307436566271\\
579	0.00752430735533561\\
580	0.00744394776027147\\
581	0.00736193826362876\\
582	0.00727821791754121\\
583	0.00719272178875226\\
584	0.00710538045132749\\
585	0.00701611910584812\\
586	0.00692485573802267\\
587	0.00683149674994525\\
588	0.00673592587771183\\
589	0.00663797520465071\\
590	0.00653734834917362\\
591	0.00643341581933106\\
592	0.00632466858477866\\
593	0.00620725772627072\\
594	0.00607109011801661\\
595	0.0058895602730633\\
596	0.00559184182825266\\
597	0.00498881296807123\\
598	0.00357511483354343\\
599	0\\
600	0\\
};
\addplot [color=mycolor2,solid,forget plot]
  table[row sep=crcr]{%
1	0.0124899058695097\\
2	0.0124898918345811\\
3	0.012489877513426\\
4	0.0124898629006011\\
5	0.0124898479905745\\
6	0.0124898327777244\\
7	0.0124898172563383\\
8	0.0124898014206125\\
9	0.0124897852646508\\
10	0.0124897687824639\\
11	0.0124897519679687\\
12	0.0124897348149873\\
13	0.0124897173172466\\
14	0.0124896994683773\\
15	0.0124896812619135\\
16	0.0124896626912916\\
17	0.0124896437498503\\
18	0.0124896244308294\\
19	0.0124896047273698\\
20	0.0124895846325127\\
21	0.0124895641391991\\
22	0.0124895432402696\\
23	0.0124895219284638\\
24	0.0124895001964202\\
25	0.0124894780366758\\
26	0.0124894554416659\\
27	0.0124894324037239\\
28	0.0124894089150813\\
29	0.0124893849678677\\
30	0.0124893605541104\\
31	0.0124893356657352\\
32	0.012489310294566\\
33	0.0124892844323252\\
34	0.012489258070634\\
35	0.012489231201013\\
36	0.0124892038148822\\
37	0.0124891759035619\\
38	0.0124891474582734\\
39	0.0124891184701396\\
40	0.0124890889301855\\
41	0.0124890588293399\\
42	0.0124890281584357\\
43	0.0124889969082112\\
44	0.0124889650693115\\
45	0.01248893263229\\
46	0.0124888995876092\\
47	0.0124888659256429\\
48	0.0124888316366775\\
49	0.0124887967109141\\
50	0.0124887611384702\\
51	0.0124887249093817\\
52	0.0124886880136055\\
53	0.0124886504410211\\
54	0.0124886121814337\\
55	0.0124885732245765\\
56	0.0124885335601134\\
57	0.0124884931776418\\
58	0.0124884520666956\\
59	0.0124884102167486\\
60	0.0124883676172173\\
61	0.0124883242574646\\
62	0.0124882801268035\\
63	0.0124882352145002\\
64	0.0124881895097784\\
65	0.0124881430018234\\
66	0.0124880956797856\\
67	0.0124880475327853\\
68	0.0124879985499165\\
69	0.0124879487202522\\
70	0.0124878980328481\\
71	0.0124878464767479\\
72	0.0124877940409881\\
73	0.0124877407146025\\
74	0.012487686486628\\
75	0.012487631346109\\
76	0.0124875752821032\\
77	0.0124875182836864\\
78	0.0124874603399584\\
79	0.012487401440048\\
80	0.0124873415731186\\
81	0.0124872807283737\\
82	0.0124872188950626\\
83	0.0124871560624855\\
84	0.0124870922199994\\
85	0.0124870273570234\\
86	0.0124869614630442\\
87	0.0124868945276211\\
88	0.012486826540392\\
89	0.0124867574910779\\
90	0.0124866873694881\\
91	0.0124866161655255\\
92	0.0124865438691906\\
93	0.0124864704705868\\
94	0.0124863959599242\\
95	0.0124863203275237\\
96	0.0124862435638211\\
97	0.0124861656593704\\
98	0.0124860866048471\\
99	0.0124860063910508\\
100	0.0124859250089081\\
101	0.0124858424494745\\
102	0.0124857587039359\\
103	0.0124856737636104\\
104	0.0124855876199487\\
105	0.0124855002645343\\
106	0.0124854116890838\\
107	0.0124853218854462\\
108	0.0124852308456014\\
109	0.0124851385616586\\
110	0.0124850450258544\\
111	0.0124849502305498\\
112	0.012484854168227\\
113	0.0124847568314853\\
114	0.0124846582130375\\
115	0.0124845583057042\\
116	0.0124844571024092\\
117	0.0124843545961735\\
118	0.012484250780109\\
119	0.0124841456474125\\
120	0.0124840391913581\\
121	0.0124839314052907\\
122	0.0124838222826185\\
123	0.0124837118168051\\
124	0.0124836000013626\\
125	0.0124834868298434\\
126	0.0124833722958328\\
127	0.0124832563929416\\
128	0.0124831391147989\\
129	0.0124830204550447\\
130	0.0124829004073231\\
131	0.0124827789652761\\
132	0.0124826561225369\\
133	0.0124825318727236\\
134	0.0124824062094329\\
135	0.0124822791262337\\
136	0.0124821506166592\\
137	0.0124820206741983\\
138	0.0124818892922846\\
139	0.0124817564642805\\
140	0.0124816221834558\\
141	0.0124814864429543\\
142	0.0124813492357403\\
143	0.0124812105545025\\
144	0.0124810703914728\\
145	0.0124809287380585\\
146	0.0124807855840914\\
147	0.0124806409163868\\
148	0.0124804947165609\\
149	0.0124803469604029\\
150	0.0124801976301356\\
151	0.0124800467548402\\
152	0.0124798943186481\\
153	0.0124797403015055\\
154	0.0124795846829034\\
155	0.0124794274418618\\
156	0.0124792685569136\\
157	0.0124791080060879\\
158	0.0124789457668929\\
159	0.0124787818162977\\
160	0.0124786161307143\\
161	0.0124784486859786\\
162	0.0124782794573303\\
163	0.0124781084193936\\
164	0.0124779355461554\\
165	0.0124777608109448\\
166	0.0124775841864102\\
167	0.012477405644497\\
168	0.0124772251564241\\
169	0.0124770426926598\\
170	0.0124768582228971\\
171	0.0124766717160279\\
172	0.0124764831401175\\
173	0.012476292462377\\
174	0.0124760996491364\\
175	0.0124759046658158\\
176	0.0124757074768966\\
177	0.012475508045892\\
178	0.0124753063353161\\
179	0.0124751023066531\\
180	0.012474895920325\\
181	0.0124746871356594\\
182	0.0124744759108559\\
183	0.0124742622029519\\
184	0.0124740459677879\\
185	0.012473827159972\\
186	0.0124736057328432\\
187	0.0124733816384351\\
188	0.0124731548274376\\
189	0.0124729252491587\\
190	0.0124726928514852\\
191	0.0124724575808432\\
192	0.012472219382157\\
193	0.0124719781988085\\
194	0.012471733972595\\
195	0.0124714866436863\\
196	0.0124712361505819\\
197	0.012470982430067\\
198	0.0124707254171673\\
199	0.0124704650451046\\
200	0.0124702012452499\\
201	0.0124699339470775\\
202	0.0124696630781169\\
203	0.0124693885639056\\
204	0.0124691103279396\\
205	0.0124688282916242\\
206	0.0124685423742239\\
207	0.0124682524928114\\
208	0.0124679585622155\\
209	0.012467660494969\\
210	0.0124673582012553\\
211	0.0124670515888537\\
212	0.0124667405630851\\
213	0.012466425026755\\
214	0.0124661048800976\\
215	0.0124657800207173\\
216	0.0124654503435301\\
217	0.0124651157407039\\
218	0.0124647761015979\\
219	0.0124644313127009\\
220	0.0124640812575685\\
221	0.012463725816759\\
222	0.0124633648677655\\
223	0.0124629982849404\\
224	0.0124626259394003\\
225	0.0124622476988772\\
226	0.0124618634274375\\
227	0.0124614729848716\\
228	0.0124610762253366\\
229	0.0124606729945205\\
230	0.0124602631248934\\
231	0.0124598464331194\\
232	0.0124594227435349\\
233	0.012458991978805\\
234	0.0124585539768254\\
235	0.0124581085708109\\
236	0.0124576555905474\\
237	0.0124571948625038\\
238	0.0124567262099996\\
239	0.0124562494534469\\
240	0.0124557644106965\\
241	0.0124552708975352\\
242	0.0124547687284186\\
243	0.0124542577175781\\
244	0.0124537376807467\\
245	0.0124532084378732\\
246	0.0124526698172451\\
247	0.0124521216608792\\
248	0.0124515638282067\\
249	0.0124509961864292\\
250	0.0124504185655711\\
251	0.0124498307268511\\
252	0.0124492325529148\\
253	0.0124486239497388\\
254	0.0124480048444645\\
255	0.0124473751665518\\
256	0.0124467347878819\\
257	0.0124460834245486\\
258	0.0124454210160211\\
259	0.0124447491737744\\
260	0.0124440676550846\\
261	0.0124433762057747\\
262	0.0124426745585101\\
263	0.0124419624323464\\
264	0.0124412395407332\\
265	0.0124405056153793\\
266	0.0124397603516958\\
267	0.0124390034265005\\
268	0.0124382345021435\\
269	0.0124374532256369\\
270	0.0124366592277048\\
271	0.0124358521217202\\
272	0.0124350315024562\\
273	0.012434196944436\\
274	0.0124333479992884\\
275	0.0124324841904358\\
276	0.0124316050006829\\
277	0.0124307098426495\\
278	0.0124297980009628\\
279	0.0124288686203389\\
280	0.0124279212851715\\
281	0.0124269557482776\\
282	0.012425971349151\\
283	0.0124249673909836\\
284	0.0124239431226206\\
285	0.0124228977130159\\
286	0.012421830355329\\
287	0.0124207402963908\\
288	0.0124196269044363\\
289	0.0124184897233157\\
290	0.0124173279624923\\
291	0.0124161389944833\\
292	0.0124149215629971\\
293	0.0124136743190888\\
294	0.0124123958115562\\
295	0.0124110844760014\\
296	0.0124097386222167\\
297	0.0124083564192622\\
298	0.0124069358768688\\
299	0.0124054748201287\\
300	0.0124039708519022\\
301	0.012402421302669\\
302	0.0124008232446478\\
303	0.0123991739232649\\
304	0.0123974700071286\\
305	0.0123957077359013\\
306	0.0123938829443561\\
307	0.0123919910088375\\
308	0.0123900267900115\\
309	0.0123879845731706\\
310	0.0123858579884471\\
311	0.0123836397747981\\
312	0.0123813208717979\\
313	0.0123788894398068\\
314	0.0123763285252103\\
315	0.012370921442076\\
316	0.0123649355844777\\
317	0.0123588561857783\\
318	0.0123526784241549\\
319	0.0123463935582653\\
320	0.0123400051360701\\
321	0.0123335154806087\\
322	0.0123269220701389\\
323	0.0123202227683259\\
324	0.0123134167450671\\
325	0.0123065066784082\\
326	0.0122995006730084\\
327	0.0122923819741245\\
328	0.0122851339179322\\
329	0.0122777516449987\\
330	0.0122702297905421\\
331	0.012262562134915\\
332	0.0122547408923101\\
333	0.0122467586426726\\
334	0.0122386136013522\\
335	0.0122302982138006\\
336	0.0122218044076812\\
337	0.0122131238511732\\
338	0.0122042486884234\\
339	0.012195172287013\\
340	0.0121858797646555\\
341	0.0121763465999835\\
342	0.012166557109027\\
343	0.01215649371462\\
344	0.0121461365985616\\
345	0.0121354631663459\\
346	0.012124446885626\\
347	0.012113053267976\\
348	0.0121012179216949\\
349	0.0120884965445485\\
350	0.0120698539880368\\
351	0.0120509823334755\\
352	0.0120318921016879\\
353	0.0120126267926436\\
354	0.0119931343073231\\
355	0.0119733861784105\\
356	0.0119533717973654\\
357	0.0119330957513884\\
358	0.011912564331459\\
359	0.0118917774008529\\
360	0.011870737833019\\
361	0.0118494585751671\\
362	0.0118279417678206\\
363	0.0118061564915626\\
364	0.0117841039848333\\
365	0.0117617864993389\\
366	0.0117392075201951\\
367	0.0117163720547322\\
368	0.0116932870613034\\
369	0.0116699622532959\\
370	0.0116464116899545\\
371	0.0116226501263682\\
372	0.0115986959567423\\
373	0.0115745717392626\\
374	0.0115503044350653\\
375	0.0115273377065095\\
376	0.0115179002283825\\
377	0.0115084824047774\\
378	0.0114990947983604\\
379	0.0114897490429226\\
380	0.0114804579830704\\
381	0.01147123589981\\
382	0.0114620985812708\\
383	0.0114530634187262\\
384	0.0114441495803473\\
385	0.011435378205776\\
386	0.0114267726226372\\
387	0.0114183585814081\\
388	0.0114101644821448\\
389	0.0114022217175612\\
390	0.0113945650690661\\
391	0.0113872331429737\\
392	0.0113802689147056\\
393	0.0113737203175515\\
394	0.0113676407667043\\
395	0.0113620898826448\\
396	0.0113571343228428\\
397	0.0113523059218674\\
398	0.0113475279039228\\
399	0.0113428035994453\\
400	0.0113381361493827\\
401	0.0113335284315835\\
402	0.0113289829722901\\
403	0.0113245018401987\\
404	0.0113200865209816\\
405	0.0113157377665044\\
406	0.0113114554121766\\
407	0.0113072381568359\\
408	0.0113030832941315\\
409	0.0112989863955218\\
410	0.0112949409309041\\
411	0.0112909378133669\\
412	0.011286964856682\\
413	0.0112830061260558\\
414	0.0112790411617502\\
415	0.0112750602871158\\
416	0.0112710620903373\\
417	0.0112670449604429\\
418	0.0112630070793571\\
419	0.0112589464166073\\
420	0.0112548607278733\\
421	0.0112507475588533\\
422	0.0112466042564819\\
423	0.0112424279896433\\
424	0.0112382157822969\\
425	0.011233964562621\\
426	0.0112296712327711\\
427	0.0112253327650429\\
428	0.0112209463317157\\
429	0.0112165094777162\\
430	0.0112120203475584\\
431	0.0112074775595681\\
432	0.0112028797061604\\
433	0.0111982253596221\\
434	0.011193513078722\\
435	0.0111887414160902\\
436	0.0111839089262452\\
437	0.0111790141740625\\
438	0.0111740557433569\\
439	0.0111690322450853\\
440	0.0111639423244585\\
441	0.0111587846659548\\
442	0.0111535579948385\\
443	0.0111482610732717\\
444	0.0111428926884298\\
445	0.0111374516291415\\
446	0.01113193667288\\
447	0.0111263465862123\\
448	0.0111206801250633\\
449	0.0111149360347422\\
450	0.0111091130496806\\
451	0.0111032098928331\\
452	0.0110972252746971\\
453	0.0110911578919272\\
454	0.0110850064255433\\
455	0.0110787695387785\\
456	0.0110724458746793\\
457	0.0110660340536702\\
458	0.0110595326714435\\
459	0.0110529402977432\\
460	0.0110462554753646\\
461	0.0110394767190718\\
462	0.0110326025144355\\
463	0.011025631316593\\
464	0.0110185615489396\\
465	0.011011391601758\\
466	0.0110041198307999\\
467	0.0109967445558312\\
468	0.0109892640591567\\
469	0.010981676584134\\
470	0.0109739803336804\\
471	0.010966173468764\\
472	0.0109582541068399\\
473	0.0109502203202189\\
474	0.0109420701343649\\
475	0.0109338015261205\\
476	0.0109254124218581\\
477	0.0109169006955545\\
478	0.0109082641667855\\
479	0.0108995005986375\\
480	0.0108906076955305\\
481	0.0108815831009482\\
482	0.0108724243950681\\
483	0.0108631290922849\\
484	0.0108536946386212\\
485	0.0108441184090199\\
486	0.0108343977045128\\
487	0.0108245297492601\\
488	0.0108145116874535\\
489	0.0108043405800779\\
490	0.0107940134015238\\
491	0.0107835270360439\\
492	0.0107728782740463\\
493	0.0107620638082176\\
494	0.0107510802294666\\
495	0.010739924022682\\
496	0.0107285915622965\\
497	0.0107170791076466\\
498	0.0107053827981228\\
499	0.0106934986480998\\
500	0.0106814225416374\\
501	0.0106691502269463\\
502	0.0106566773106061\\
503	0.0106439992515312\\
504	0.0106311113546724\\
505	0.010618008764449\\
506	0.0106046864579025\\
507	0.0105911392375658\\
508	0.0105773617240414\\
509	0.0105633483482841\\
510	0.0105490933451519\\
511	0.0105345907453926\\
512	0.0105198343666647\\
513	0.0105048177690006\\
514	0.0104895342397086\\
515	0.0104739768385048\\
516	0.010458135053507\\
517	0.010442004924358\\
518	0.0104255816028362\\
519	0.0104088564084067\\
520	0.0103918166774931\\
521	0.0103744590490279\\
522	0.0103567773162321\\
523	0.01033876119676\\
524	0.0103204077833572\\
525	0.0103017001063452\\
526	0.0102826195717509\\
527	0.0102631558089421\\
528	0.0102432975992738\\
529	0.0102230319184114\\
530	0.0102023414680112\\
531	0.0101812073683544\\
532	0.0101596464122478\\
533	0.0101376354415696\\
534	0.0101151618026545\\
535	0.0100922424281249\\
536	0.010068876465799\\
537	0.0100450542649095\\
538	0.0100206969305763\\
539	0.00999579045777402\\
540	0.00997031114870067\\
541	0.00994422540220995\\
542	0.00991751576362549\\
543	0.00989016716918039\\
544	0.0098621643912705\\
545	0.00983349114382052\\
546	0.00980401782346135\\
547	0.00977388035971209\\
548	0.00974295049822557\\
549	0.00971143590720675\\
550	0.00967961836267487\\
551	0.00964711305777354\\
552	0.00961380442103104\\
553	0.00957964880248824\\
554	0.00954461545166724\\
555	0.00950867526035693\\
556	0.00947179845028549\\
557	0.00943395392448495\\
558	0.00939495787475029\\
559	0.00935457285869148\\
560	0.00931275940407491\\
561	0.00927137351937616\\
562	0.00923024825685196\\
563	0.00918803432212118\\
564	0.00914466123074735\\
565	0.00906001065198841\\
566	0.00880282936135181\\
567	0.00838176684114641\\
568	0.00831367114185934\\
569	0.00824828975106419\\
570	0.00818175587048123\\
571	0.00811402083827017\\
572	0.00804504960790682\\
573	0.00797480573282636\\
574	0.00790325081764849\\
575	0.00783034432666621\\
576	0.00775604341680255\\
577	0.00768030276212431\\
578	0.00760307436573265\\
579	0.00752430735536871\\
580	0.00744394776028546\\
581	0.00736193826363283\\
582	0.00727821791754183\\
583	0.00719272178875225\\
584	0.0071053804513275\\
585	0.00701611910584812\\
586	0.00692485573802268\\
587	0.00683149674994525\\
588	0.00673592587771182\\
589	0.00663797520465072\\
590	0.00653734834917363\\
591	0.00643341581933105\\
592	0.00632466858477865\\
593	0.00620725772627071\\
594	0.0060710901180166\\
595	0.00588956027306328\\
596	0.00559184182825266\\
597	0.00498881296807124\\
598	0.00357511483354343\\
599	0\\
600	0\\
};
\addplot [color=mycolor3,solid,forget plot]
  table[row sep=crcr]{%
1	0.0124900117213761\\
2	0.0124900005757721\\
3	0.0124899892213819\\
4	0.0124899776547542\\
5	0.0124899658723958\\
6	0.0124899538707719\\
7	0.0124899416463056\\
8	0.0124899291953785\\
9	0.0124899165143306\\
10	0.0124899035994603\\
11	0.0124898904470247\\
12	0.0124898770532395\\
13	0.0124898634142798\\
14	0.0124898495262796\\
15	0.0124898353853323\\
16	0.0124898209874914\\
17	0.0124898063287702\\
18	0.0124897914051424\\
19	0.0124897762125426\\
20	0.0124897607468664\\
21	0.0124897450039712\\
22	0.0124897289796763\\
23	0.0124897126697639\\
24	0.0124896960699789\\
25	0.0124896791760304\\
26	0.0124896619835914\\
27	0.0124896444883002\\
28	0.0124896266857606\\
29	0.0124896085715428\\
30	0.0124895901411842\\
31	0.0124895713901902\\
32	0.0124895523140351\\
33	0.0124895329081626\\
34	0.0124895131679874\\
35	0.0124894930888953\\
36	0.0124894726662452\\
37	0.0124894518953693\\
38	0.0124894307715744\\
39	0.0124894092901435\\
40	0.0124893874463362\\
41	0.0124893652353904\\
42	0.0124893426525235\\
43	0.0124893196929333\\
44	0.0124892963517999\\
45	0.0124892726242864\\
46	0.0124892485055408\\
47	0.0124892239906969\\
48	0.0124891990748763\\
49	0.0124891737531894\\
50	0.0124891480207372\\
51	0.0124891218726124\\
52	0.0124890953039015\\
53	0.012489068309686\\
54	0.0124890408850438\\
55	0.0124890130250512\\
56	0.0124889847247843\\
57	0.0124889559793207\\
58	0.0124889267837408\\
59	0.0124888971331298\\
60	0.0124888670225789\\
61	0.0124888364471874\\
62	0.0124888054020637\\
63	0.0124887738823273\\
64	0.0124887418831101\\
65	0.012488709399558\\
66	0.0124886764268322\\
67	0.0124886429601109\\
68	0.0124886089945903\\
69	0.0124885745254862\\
70	0.012488539548035\\
71	0.0124885040574953\\
72	0.0124884680491484\\
73	0.0124884315182997\\
74	0.0124883944602793\\
75	0.0124883568704432\\
76	0.0124883187441736\\
77	0.0124882800768796\\
78	0.0124882408639976\\
79	0.0124882011009915\\
80	0.012488160783353\\
81	0.0124881199066013\\
82	0.012488078466283\\
83	0.0124880364579716\\
84	0.012487993877267\\
85	0.0124879507197948\\
86	0.0124879069812049\\
87	0.0124878626571703\\
88	0.0124878177433859\\
89	0.0124877722355667\\
90	0.0124877261294453\\
91	0.0124876794207702\\
92	0.0124876321053031\\
93	0.0124875841788157\\
94	0.0124875356370873\\
95	0.0124874864759004\\
96	0.012487436691038\\
97	0.0124873862782788\\
98	0.0124873352333933\\
99	0.0124872835521386\\
100	0.0124872312302543\\
101	0.0124871782634561\\
102	0.012487124647431\\
103	0.012487070377831\\
104	0.0124870154502668\\
105	0.0124869598603013\\
106	0.0124869036034428\\
107	0.0124868466751375\\
108	0.0124867890707623\\
109	0.012486730785617\\
110	0.0124866718149159\\
111	0.0124866121537802\\
112	0.0124865517972289\\
113	0.0124864907401702\\
114	0.0124864289773925\\
115	0.0124863665035556\\
116	0.0124863033131811\\
117	0.0124862394006428\\
118	0.0124861747601577\\
119	0.012486109385776\\
120	0.0124860432713715\\
121	0.0124859764106317\\
122	0.0124859087970485\\
123	0.0124858404239081\\
124	0.0124857712842813\\
125	0.0124857013710141\\
126	0.012485630676718\\
127	0.0124855591937606\\
128	0.012485486914256\\
129	0.0124854138300561\\
130	0.0124853399327408\\
131	0.0124852652136095\\
132	0.0124851896636722\\
133	0.0124851132736404\\
134	0.012485036033919\\
135	0.0124849579345971\\
136	0.01248487896544\\
137	0.0124847991158804\\
138	0.0124847183750098\\
139	0.0124846367315694\\
140	0.0124845541739403\\
141	0.0124844706901321\\
142	0.0124843862677663\\
143	0.0124843008940506\\
144	0.0124842145557326\\
145	0.0124841272390174\\
146	0.0124840389294419\\
147	0.0124839496117726\\
148	0.0124838592702673\\
149	0.0124837678900993\\
150	0.0124836754594427\\
151	0.0124835819624927\\
152	0.012483487382789\\
153	0.0124833917035093\\
154	0.012483294907461\\
155	0.0124831969770725\\
156	0.0124830978943846\\
157	0.012482997641042\\
158	0.0124828961982841\\
159	0.012482793546936\\
160	0.0124826896673999\\
161	0.0124825845396455\\
162	0.0124824781432011\\
163	0.0124823704571443\\
164	0.0124822614600923\\
165	0.0124821511301933\\
166	0.0124820394451161\\
167	0.0124819263820413\\
168	0.0124818119176518\\
169	0.0124816960281228\\
170	0.0124815786891126\\
171	0.0124814598757532\\
172	0.0124813395626401\\
173	0.0124812177238238\\
174	0.0124810943327992\\
175	0.012480969362497\\
176	0.0124808427852736\\
177	0.0124807145729023\\
178	0.0124805846965635\\
179	0.0124804531268356\\
180	0.0124803198336859\\
181	0.0124801847864616\\
182	0.0124800479538802\\
183	0.0124799093040214\\
184	0.0124797688043175\\
185	0.0124796264215453\\
186	0.0124794821218172\\
187	0.0124793358705726\\
188	0.0124791876325697\\
189	0.0124790373718772\\
190	0.0124788850518661\\
191	0.0124787306352014\\
192	0.0124785740838347\\
193	0.0124784153589955\\
194	0.0124782544211841\\
195	0.0124780912301634\\
196	0.0124779257449516\\
197	0.012477757923814\\
198	0.0124775877242555\\
199	0.0124774151030131\\
200	0.0124772400160479\\
201	0.0124770624185374\\
202	0.0124768822648671\\
203	0.0124766995086228\\
204	0.0124765141025823\\
205	0.0124763259987061\\
206	0.0124761351481293\\
207	0.0124759415011515\\
208	0.0124757450072277\\
209	0.0124755456149576\\
210	0.0124753432720753\\
211	0.0124751379254372\\
212	0.0124749295210105\\
213	0.0124747180038601\\
214	0.0124745033181348\\
215	0.0124742854070527\\
216	0.0124740642128855\\
217	0.0124738396769417\\
218	0.0124736117395489\\
219	0.0124733803400345\\
220	0.0124731454167046\\
221	0.0124729069068219\\
222	0.0124726647465798\\
223	0.0124724188710718\\
224	0.0124721692142512\\
225	0.0124719157088681\\
226	0.0124716582863629\\
227	0.0124713968766753\\
228	0.0124711314079379\\
229	0.0124708618061522\\
230	0.0124705879954977\\
231	0.0124703099010313\\
232	0.0124700274534102\\
233	0.0124697405752607\\
234	0.0124694491876557\\
235	0.0124691532101689\\
236	0.0124688525608331\\
237	0.0124685471560972\\
238	0.0124682369107845\\
239	0.0124679217380536\\
240	0.0124676015493646\\
241	0.0124672762544591\\
242	0.0124669457613653\\
243	0.0124666099764479\\
244	0.0124662688045272\\
245	0.0124659221490664\\
246	0.012465569912315\\
247	0.0124652119949602\\
248	0.0124648482942536\\
249	0.0124644786999544\\
250	0.0124641030937212\\
251	0.0124637213620397\\
252	0.0124633333870466\\
253	0.0124629390444868\\
254	0.0124625381996382\\
255	0.0124621307008008\\
256	0.0124617163777749\\
257	0.0124612950843974\\
258	0.0124608667782466\\
259	0.0124604312930264\\
260	0.0124599884577046\\
261	0.0124595380963293\\
262	0.012459080028112\\
263	0.0124586140683\\
264	0.0124581400293418\\
265	0.0124576577162257\\
266	0.012457166928084\\
267	0.0124566674584745\\
268	0.0124561590952074\\
269	0.012455641620162\\
270	0.0124551148090884\\
271	0.0124545784313791\\
272	0.0124540322497752\\
273	0.0124534760199154\\
274	0.0124529094895125\\
275	0.0124523323967338\\
276	0.0124517444672789\\
277	0.0124511454112616\\
278	0.0124505349298039\\
279	0.0124499127552903\\
280	0.0124492786218922\\
281	0.0124486322248941\\
282	0.0124479732504704\\
283	0.0124473013740687\\
284	0.0124466162600645\\
285	0.0124459175714466\\
286	0.0124452049726032\\
287	0.0124444781329385\\
288	0.0124437367206503\\
289	0.0124429803494812\\
290	0.0124422085043243\\
291	0.0124414207738386\\
292	0.0124406167344618\\
293	0.0124397959501881\\
294	0.01243895797239\\
295	0.0124381023396796\\
296	0.0124372285777689\\
297	0.0124363361992258\\
298	0.0124354247029345\\
299	0.0124344935732505\\
300	0.012433542280599\\
301	0.0124325702919805\\
302	0.012431577101701\\
303	0.0124305621744038\\
304	0.0124295249666692\\
305	0.0124284649379104\\
306	0.0124273815564308\\
307	0.0124262743074417\\
308	0.0124251427026119\\
309	0.0124239862867158\\
310	0.0124228046244221\\
311	0.0124215972348075\\
312	0.0124203635229112\\
313	0.0124191026464338\\
314	0.0124178134236839\\
315	0.0124164985515049\\
316	0.0124151576251906\\
317	0.0124137893669169\\
318	0.0124123922899686\\
319	0.01241096579974\\
320	0.0124095091010523\\
321	0.0124080209768983\\
322	0.0124065002097119\\
323	0.0124049456755524\\
324	0.012403356474694\\
325	0.0124017317356611\\
326	0.0124000685809139\\
327	0.0123983641545376\\
328	0.012396616370332\\
329	0.0123948229548816\\
330	0.0123929814020112\\
331	0.0123910889404741\\
332	0.012389142736172\\
333	0.0123871401553977\\
334	0.0123850778474462\\
335	0.0123829521799028\\
336	0.012380759239627\\
337	0.0123784948415397\\
338	0.0123761544033203\\
339	0.0123737321287599\\
340	0.0123712211093611\\
341	0.0123686146560381\\
342	0.0123659053050938\\
343	0.0123630846840279\\
344	0.0123601433096615\\
345	0.0123570702010703\\
346	0.0123538519145613\\
347	0.0123504697763404\\
348	0.012346894154993\\
349	0.0123427558617095\\
350	0.0123338818010271\\
351	0.0123248706417058\\
352	0.0123157265279167\\
353	0.0123064608165831\\
354	0.0122970436401246\\
355	0.0122874567199428\\
356	0.0122776905100024\\
357	0.0122677416625208\\
358	0.0122576063654486\\
359	0.0122472774433936\\
360	0.0122367486737082\\
361	0.0122260158450916\\
362	0.0122150663140686\\
363	0.0122038739015477\\
364	0.0121924251772472\\
365	0.012180705271155\\
366	0.0121686976221371\\
367	0.0121563835550873\\
368	0.0121437411931459\\
369	0.0121307408423981\\
370	0.0121173316477296\\
371	0.0121035191999451\\
372	0.0120892716440285\\
373	0.012074555217779\\
374	0.0120593399796327\\
375	0.0120428858819619\\
376	0.0120186426323913\\
377	0.0119940963925576\\
378	0.0119692490533195\\
379	0.0119441049902936\\
380	0.0119186860346378\\
381	0.0118929643586608\\
382	0.0118669198083094\\
383	0.0118405519421216\\
384	0.0118138611523436\\
385	0.0117868488637125\\
386	0.0117595178276716\\
387	0.0117318726798215\\
388	0.0117039214955012\\
389	0.0116756733381913\\
390	0.0116471384998108\\
391	0.0116183295054687\\
392	0.0115892603065783\\
393	0.0115599481408537\\
394	0.0115304191576603\\
395	0.01150070591498\\
396	0.0114708489372037\\
397	0.0114571246110284\\
398	0.0114461287698232\\
399	0.0114352074925657\\
400	0.0114243793106374\\
401	0.0114136646750546\\
402	0.0114030861647579\\
403	0.0113926687125702\\
404	0.0113824398248891\\
405	0.0113724298927981\\
406	0.0113626725928505\\
407	0.0113532053208341\\
408	0.0113440697912537\\
409	0.0113353124306812\\
410	0.0113269849679598\\
411	0.0113191451465055\\
412	0.0113118574369489\\
413	0.0113051939321546\\
414	0.0112992353672767\\
415	0.0112936057619535\\
416	0.0112880377296427\\
417	0.0112825347915983\\
418	0.0112771000917152\\
419	0.0112717362787034\\
420	0.0112664453606393\\
421	0.0112612285277664\\
422	0.0112560859330031\\
423	0.0112510164344459\\
424	0.0112460172862472\\
425	0.0112410837696491\\
426	0.0112362087507133\\
427	0.0112313821499731\\
428	0.0112265903065674\\
429	0.0112218152161048\\
430	0.0112170336179759\\
431	0.0112122278156015\\
432	0.0112073958616761\\
433	0.0112025355606028\\
434	0.0111976444619761\\
435	0.0111927198591086\\
436	0.0111877587942\\
437	0.0111827580723855\\
438	0.0111777142875351\\
439	0.0111726238634414\\
440	0.0111674831149901\\
441	0.0111622883351075\\
442	0.0111570359147843\\
443	0.0111517225053463\\
444	0.0111463452344593\\
445	0.0111409019903933\\
446	0.0111353910541228\\
447	0.0111298106795012\\
448	0.0111241591009583\\
449	0.0111184345421164\\
450	0.0111126352251774\\
451	0.0111067593808233\\
452	0.0111008052582152\\
453	0.0110947711344761\\
454	0.0110886553227582\\
455	0.011082456177628\\
456	0.0110761720959998\\
457	0.0110698015111858\\
458	0.0110633428767625\\
459	0.0110567946358115\\
460	0.0110501552119634\\
461	0.0110434230094554\\
462	0.0110365964129051\\
463	0.0110296737867357\\
464	0.0110226534741914\\
465	0.0110155337958875\\
466	0.0110083130478594\\
467	0.0110009894991038\\
468	0.0109935613886583\\
469	0.0109860269223436\\
470	0.0109783842694136\\
471	0.0109706315595301\\
472	0.0109627668807314\\
473	0.0109547882777024\\
474	0.0109466937499229\\
475	0.0109384812496959\\
476	0.0109301486800601\\
477	0.0109216938925958\\
478	0.0109131146851347\\
479	0.0109044087993883\\
480	0.0108955739185107\\
481	0.0108866076646103\\
482	0.0108775075962197\\
483	0.010868271205721\\
484	0.0108588959166984\\
485	0.0108493790811731\\
486	0.010839717976718\\
487	0.0108299098034481\\
488	0.0108199516808828\\
489	0.0108098406446772\\
490	0.0107995736432165\\
491	0.0107891475340695\\
492	0.0107785590802925\\
493	0.010767804946579\\
494	0.0107568816952437\\
495	0.0107457857820342\\
496	0.0107345135517603\\
497	0.0107230612337341\\
498	0.0107114249370147\\
499	0.0106996006454486\\
500	0.0106875842125\\
501	0.0106753713558623\\
502	0.0106629576518436\\
503	0.0106503385295184\\
504	0.0106375092646387\\
505	0.0106244649732957\\
506	0.0106112006053275\\
507	0.0105977109374638\\
508	0.0105839905662037\\
509	0.0105700339004211\\
510	0.0105558351536465\\
511	0.0105413883360576\\
512	0.0105266872461938\\
513	0.0105117254633539\\
514	0.0104964963398891\\
515	0.0104809929917471\\
516	0.0104652082888149\\
517	0.0104491347938747\\
518	0.0104327647825303\\
519	0.0104160902676691\\
520	0.0103990993857909\\
521	0.0103817877930226\\
522	0.0103641489966369\\
523	0.0103461731401486\\
524	0.0103278468408664\\
525	0.0103091639362961\\
526	0.0102901179732998\\
527	0.0102706974394484\\
528	0.0102508949463511\\
529	0.0102306996381357\\
530	0.0102100867601676\\
531	0.0101890440677434\\
532	0.0101675595467004\\
533	0.0101456104950239\\
534	0.0101231826669715\\
535	0.010100287680872\\
536	0.0100769016474549\\
537	0.0100530087982048\\
538	0.0100286259503887\\
539	0.0100037425761063\\
540	0.00997837563815114\\
541	0.00995241604058662\\
542	0.00992584375934148\\
543	0.00989863439054779\\
544	0.00987077093169437\\
545	0.00984223690468954\\
546	0.00981301526993071\\
547	0.00978308845229256\\
548	0.0097523199963511\\
549	0.00972086204257419\\
550	0.00968857864483313\\
551	0.00965565846421692\\
552	0.00962233801165364\\
553	0.00958835066292521\\
554	0.00955352863102991\\
555	0.00951781367241968\\
556	0.00948117010442207\\
557	0.00944356668974041\\
558	0.00940497068589146\\
559	0.00936534712600867\\
560	0.00932413031827952\\
561	0.00928171998247948\\
562	0.00923840442122668\\
563	0.00919625530291374\\
564	0.00915316897197163\\
565	0.00910890374113917\\
566	0.00902976035098331\\
567	0.00880259997933173\\
568	0.0083478056989328\\
569	0.00824844307218399\\
570	0.00818176236454212\\
571	0.00811402171848672\\
572	0.00804504977729814\\
573	0.00797480577650455\\
574	0.00790325083332592\\
575	0.0078303443341488\\
576	0.00775604342024438\\
577	0.00768030276391219\\
578	0.00760307436664889\\
579	0.00752430735584519\\
580	0.007443947760513\\
581	0.0073619382637318\\
582	0.00727821791757129\\
583	0.00719272178875675\\
584	0.00710538045132748\\
585	0.00701611910584813\\
586	0.00692485573802266\\
587	0.00683149674994524\\
588	0.00673592587771182\\
589	0.00663797520465071\\
590	0.00653734834917364\\
591	0.00643341581933105\\
592	0.00632466858477866\\
593	0.00620725772627071\\
594	0.00607109011801661\\
595	0.00588956027306329\\
596	0.00559184182825266\\
597	0.00498881296807123\\
598	0.00357511483354343\\
599	0\\
600	0\\
};
\addplot [color=mycolor4,solid,forget plot]
  table[row sep=crcr]{%
1	0.0124901248826776\\
2	0.0124901165401922\\
3	0.0124901080555749\\
4	0.0124900994268116\\
5	0.0124900906518713\\
6	0.012490081728706\\
7	0.0124900726552512\\
8	0.0124900634294259\\
9	0.0124900540491329\\
10	0.0124900445122589\\
11	0.012490034816675\\
12	0.0124900249602366\\
13	0.0124900149407839\\
14	0.0124900047561419\\
15	0.0124899944041212\\
16	0.0124899838825176\\
17	0.0124899731891129\\
18	0.0124899623216748\\
19	0.0124899512779577\\
20	0.0124899400557024\\
21	0.012489928652637\\
22	0.0124899170664768\\
23	0.012489905294925\\
24	0.0124898933356726\\
25	0.0124898811863992\\
26	0.0124898688447731\\
27	0.0124898563084516\\
28	0.0124898435750815\\
29	0.0124898306422995\\
30	0.0124898175077324\\
31	0.0124898041689977\\
32	0.0124897906237037\\
33	0.01248977686945\\
34	0.0124897629038278\\
35	0.0124897487244206\\
36	0.0124897343288039\\
37	0.0124897197145463\\
38	0.0124897048792093\\
39	0.0124896898203477\\
40	0.0124896745355104\\
41	0.0124896590222401\\
42	0.0124896432780741\\
43	0.0124896273005443\\
44	0.0124896110871775\\
45	0.0124895946354959\\
46	0.0124895779430171\\
47	0.0124895610072544\\
48	0.0124895438257171\\
49	0.0124895263959105\\
50	0.0124895087153361\\
51	0.0124894907814918\\
52	0.0124894725918722\\
53	0.0124894541439679\\
54	0.0124894354352663\\
55	0.0124894164632514\\
56	0.0124893972254034\\
57	0.0124893777191988\\
58	0.0124893579421104\\
59	0.0124893378916069\\
60	0.0124893175651526\\
61	0.0124892969602073\\
62	0.0124892760742256\\
63	0.0124892549046567\\
64	0.0124892334489439\\
65	0.012489211704524\\
66	0.0124891896688266\\
67	0.0124891673392731\\
68	0.0124891447132767\\
69	0.0124891217882407\\
70	0.012489098561558\\
71	0.0124890750306098\\
72	0.0124890511927648\\
73	0.0124890270453776\\
74	0.0124890025857876\\
75	0.0124889778113176\\
76	0.0124889527192721\\
77	0.012488927306936\\
78	0.0124889015715725\\
79	0.0124888755104216\\
80	0.0124888491206979\\
81	0.012488822399589\\
82	0.0124887953442526\\
83	0.0124887679518152\\
84	0.0124887402193688\\
85	0.0124887121439692\\
86	0.0124886837226327\\
87	0.0124886549523341\\
88	0.0124886258300033\\
89	0.0124885963525227\\
90	0.0124885665167242\\
91	0.0124885363193858\\
92	0.0124885057572285\\
93	0.012488474826913\\
94	0.0124884435250364\\
95	0.0124884118481283\\
96	0.0124883797926474\\
97	0.0124883473549778\\
98	0.0124883145314252\\
99	0.012488281318213\\
100	0.0124882477114785\\
101	0.0124882137072685\\
102	0.0124881793015358\\
103	0.0124881444901349\\
104	0.0124881092688176\\
105	0.0124880736332292\\
106	0.0124880375789042\\
107	0.0124880011012621\\
108	0.0124879641956029\\
109	0.0124879268571035\\
110	0.0124878890808128\\
111	0.0124878508616483\\
112	0.0124878121943913\\
113	0.012487773073683\\
114	0.0124877334940207\\
115	0.0124876934497535\\
116	0.0124876529350784\\
117	0.0124876119440367\\
118	0.0124875704705098\\
119	0.012487528508216\\
120	0.0124874860507064\\
121	0.0124874430913617\\
122	0.0124873996233887\\
123	0.012487355639817\\
124	0.0124873111334956\\
125	0.0124872660970903\\
126	0.0124872205230798\\
127	0.0124871744037537\\
128	0.0124871277312092\\
129	0.0124870804973482\\
130	0.012487032693875\\
131	0.0124869843122935\\
132	0.0124869353439047\\
133	0.0124868857798043\\
134	0.0124868356108801\\
135	0.0124867848278096\\
136	0.0124867334210577\\
137	0.0124866813808744\\
138	0.0124866286972919\\
139	0.0124865753601224\\
140	0.0124865213589551\\
141	0.0124864666831522\\
142	0.0124864113218445\\
143	0.0124863552639239\\
144	0.0124862984980327\\
145	0.0124862410125527\\
146	0.0124861827956046\\
147	0.0124861238350929\\
148	0.012486064118833\\
149	0.0124860036346186\\
150	0.0124859423696853\\
151	0.012485880310991\\
152	0.0124858174452364\\
153	0.0124857537588613\\
154	0.0124856892380398\\
155	0.0124856238686756\\
156	0.0124855576363979\\
157	0.0124854905265568\\
158	0.0124854225242184\\
159	0.0124853536141608\\
160	0.0124852837808688\\
161	0.01248521300853\\
162	0.0124851412810294\\
163	0.0124850685819453\\
164	0.0124849948945443\\
165	0.0124849202017767\\
166	0.0124848444862714\\
167	0.0124847677303316\\
168	0.0124846899159295\\
169	0.0124846110247015\\
170	0.0124845310379437\\
171	0.0124844499366062\\
172	0.0124843677012886\\
173	0.012484284312235\\
174	0.0124841997493283\\
175	0.0124841139920857\\
176	0.0124840270196528\\
177	0.0124839388107986\\
178	0.0124838493439101\\
179	0.0124837585969864\\
180	0.0124836665476336\\
181	0.0124835731730583\\
182	0.0124834784500624\\
183	0.0124833823550366\\
184	0.0124832848639545\\
185	0.0124831859523657\\
186	0.01248308559539\\
187	0.0124829837677099\\
188	0.0124828804435639\\
189	0.0124827755967394\\
190	0.0124826692005647\\
191	0.0124825612279017\\
192	0.0124824516511376\\
193	0.0124823404421762\\
194	0.0124822275724294\\
195	0.0124821130128081\\
196	0.012481996733712\\
197	0.0124818787050203\\
198	0.0124817588960806\\
199	0.0124816372756983\\
200	0.0124815138121247\\
201	0.0124813884730453\\
202	0.012481261225567\\
203	0.0124811320362044\\
204	0.0124810008708668\\
205	0.0124808676948427\\
206	0.0124807324727849\\
207	0.0124805951686948\\
208	0.0124804557459049\\
209	0.0124803141670621\\
210	0.0124801703941086\\
211	0.0124800243882632\\
212	0.0124798761100013\\
213	0.0124797255190338\\
214	0.0124795725742855\\
215	0.0124794172338726\\
216	0.0124792594550789\\
217	0.0124790991943316\\
218	0.0124789364071762\\
219	0.0124787710482499\\
220	0.0124786030712543\\
221	0.0124784324289277\\
222	0.0124782590730148\\
223	0.0124780829542349\\
224	0.012477904022247\\
225	0.0124777222256086\\
226	0.0124775375117266\\
227	0.0124773498268022\\
228	0.0124771591157911\\
229	0.0124769653224477\\
230	0.0124767683895453\\
231	0.012476568259038\\
232	0.0124763648710206\\
233	0.0124761581642267\\
234	0.0124759480760004\\
235	0.0124757345422605\\
236	0.0124755174974643\\
237	0.0124752968745714\\
238	0.0124750726050081\\
239	0.0124748446186323\\
240	0.0124746128437004\\
241	0.0124743772068374\\
242	0.0124741376330116\\
243	0.0124738940455159\\
244	0.0124736463659501\\
245	0.0124733945141879\\
246	0.0124731384082771\\
247	0.0124728779642054\\
248	0.0124726130955754\\
249	0.0124723437137052\\
250	0.0124720697285101\\
251	0.0124717910477025\\
252	0.012471507576558\\
253	0.0124712192175515\\
254	0.0124709258700459\\
255	0.012470627431071\\
256	0.012470323799531\\
257	0.0124700148796904\\
258	0.0124697005642099\\
259	0.0124693807436389\\
260	0.0124690553063781\\
261	0.0124687241386772\\
262	0.012468387124689\\
263	0.0124680441464679\\
264	0.0124676950835441\\
265	0.0124673398130423\\
266	0.0124669782096682\\
267	0.0124666101456524\\
268	0.0124662354906901\\
269	0.0124658541118766\\
270	0.0124654658736353\\
271	0.0124650706376342\\
272	0.0124646682626768\\
273	0.0124642586045469\\
274	0.0124638415157779\\
275	0.0124634168453613\\
276	0.0124629844386467\\
277	0.0124625441383024\\
278	0.0124620957869119\\
279	0.0124616392239826\\
280	0.0124611742830414\\
281	0.0124607007939315\\
282	0.0124602185826081\\
283	0.0124597274711809\\
284	0.0124592272786975\\
285	0.0124587178212171\\
286	0.0124581989116626\\
287	0.0124576703582858\\
288	0.0124571319598345\\
289	0.0124565835026562\\
290	0.0124560247774795\\
291	0.0124554555699073\\
292	0.0124548756602314\\
293	0.0124542848232372\\
294	0.0124536828279935\\
295	0.0124530694376166\\
296	0.0124524444089975\\
297	0.0124518074924801\\
298	0.0124511584315414\\
299	0.0124504969627406\\
300	0.0124498228165644\\
301	0.0124491357187345\\
302	0.0124484353848962\\
303	0.0124477215220361\\
304	0.0124469938286734\\
305	0.0124462519944237\\
306	0.0124454956994415\\
307	0.0124447246134749\\
308	0.0124439383937158\\
309	0.0124431366796135\\
310	0.012442319082969\\
311	0.0124414851786714\\
312	0.0124406344982282\\
313	0.0124397665622936\\
314	0.0124388811282895\\
315	0.0124379777391157\\
316	0.0124370558673665\\
317	0.0124361149707234\\
318	0.0124351545618017\\
319	0.0124341741192239\\
320	0.0124331730767785\\
321	0.0124321508580897\\
322	0.0124311068838852\\
323	0.0124300405738951\\
324	0.012428951307811\\
325	0.0124278382979973\\
326	0.0124267007582024\\
327	0.0124255379479097\\
328	0.0124243490966155\\
329	0.0124231334013247\\
330	0.012421890029075\\
331	0.0124206181379127\\
332	0.0124193168877339\\
333	0.0124179853703915\\
334	0.0124166226517812\\
335	0.0124152277755932\\
336	0.0124137997648297\\
337	0.0124123376054178\\
338	0.0124108401907023\\
339	0.0124093063615074\\
340	0.0124077350176528\\
341	0.0124061250444717\\
342	0.0124044753165431\\
343	0.0124027846952309\\
344	0.0124010520051072\\
345	0.0123992759537307\\
346	0.012397454948314\\
347	0.0123955870131983\\
348	0.0123936714362393\\
349	0.0123917066494807\\
350	0.0123896971275026\\
351	0.0123876415631515\\
352	0.0123855385826638\\
353	0.0123833836191224\\
354	0.0123811729224236\\
355	0.0123789033123675\\
356	0.0123765719355767\\
357	0.0123741756328625\\
358	0.0123717107336897\\
359	0.0123691733800568\\
360	0.0123665594116891\\
361	0.0123638634374432\\
362	0.0123610787492131\\
363	0.0123581992738843\\
364	0.0123552183015072\\
365	0.0123521283812704\\
366	0.0123489211592767\\
367	0.0123455870488059\\
368	0.0123421143719694\\
369	0.0123384870582913\\
370	0.0123346833061634\\
371	0.0123307044088364\\
372	0.0123265351574494\\
373	0.0123221600587547\\
374	0.0123175651069471\\
375	0.0123121181459393\\
376	0.0123004270121501\\
377	0.0122885332526515\\
378	0.0122764313610203\\
379	0.0122641167566825\\
380	0.0122515875703521\\
381	0.012238819515977\\
382	0.0122257920709943\\
383	0.0122124920527585\\
384	0.0121989049264763\\
385	0.0121850145044644\\
386	0.012170802221424\\
387	0.0121562443704533\\
388	0.0121412961328583\\
389	0.0121259490876007\\
390	0.0121101932839557\\
391	0.0120940059810074\\
392	0.0120773782769752\\
393	0.0120602799353118\\
394	0.0120426225520414\\
395	0.0120243592889549\\
396	0.0120054477854116\\
397	0.0119772530535398\\
398	0.0119472087565024\\
399	0.0119167897449176\\
400	0.0118859945880998\\
401	0.01185482266128\\
402	0.0118232743914859\\
403	0.0117913517432048\\
404	0.0117590596238437\\
405	0.0117264040495483\\
406	0.0116933909071998\\
407	0.0116600270972709\\
408	0.0116263176559654\\
409	0.0115922756940728\\
410	0.0115579186717889\\
411	0.0115232682246787\\
412	0.0114883546096828\\
413	0.0114532158136441\\
414	0.0114178997337225\\
415	0.0113962785871438\\
416	0.0113837072405572\\
417	0.0113712561017825\\
418	0.011358950878782\\
419	0.0113468199248681\\
420	0.0113348946539114\\
421	0.0113232099670389\\
422	0.0113118048583576\\
423	0.0113007227250728\\
424	0.0112900119126424\\
425	0.011279726281361\\
426	0.0112699259081022\\
427	0.011260677901755\\
428	0.0112520573444275\\
429	0.011244148378936\\
430	0.0112370454582933\\
431	0.0112305161061403\\
432	0.0112240566039267\\
433	0.0112176705630141\\
434	0.0112113610049963\\
435	0.0112051301763297\\
436	0.0111989793318932\\
437	0.0111929084752741\\
438	0.0111869160461271\\
439	0.0111809985440657\\
440	0.0111751500764268\\
441	0.0111693618147172\\
442	0.0111636213415874\\
443	0.011157911866852\\
444	0.0111522112879962\\
445	0.0111464910620393\\
446	0.0111407354884382\\
447	0.0111349419349087\\
448	0.0111291074670474\\
449	0.0111232288464062\\
450	0.0111173025366058\\
451	0.011111324720289\\
452	0.011105291330471\\
453	0.0110991981007967\\
454	0.0110930406403892\\
455	0.0110868145404838\\
456	0.0110805155221153\\
457	0.0110741396364773\\
458	0.0110676835325153\\
459	0.0110611448099239\\
460	0.0110545213142401\\
461	0.0110478108591665\\
462	0.0110410112361793\\
463	0.0110341202250999\\
464	0.0110271356053456\\
465	0.0110200551674\\
466	0.0110128767237927\\
467	0.0110055981185488\\
468	0.0109982172336279\\
469	0.0109907319902822\\
470	0.0109831403424805\\
471	0.0109754402585081\\
472	0.0109676296854838\\
473	0.0109597065361618\\
474	0.0109516686883546\\
475	0.0109435139839672\\
476	0.0109352402275613\\
477	0.0109268451843773\\
478	0.0109183265777507\\
479	0.0109096820858918\\
480	0.0109009093380414\\
481	0.010892005910093\\
482	0.0108829693198906\\
483	0.010873797022585\\
484	0.0108644864066869\\
485	0.0108550347912124\\
486	0.0108454394226476\\
487	0.0108356974717292\\
488	0.010825806030049\\
489	0.0108157621064867\\
490	0.0108055626234857\\
491	0.0107952044131859\\
492	0.0107846842134338\\
493	0.0107739986636859\\
494	0.0107631443008188\\
495	0.0107521175548441\\
496	0.0107409147444969\\
497	0.0107295320726408\\
498	0.010717965621484\\
499	0.0107062113476029\\
500	0.0106942650767707\\
501	0.0106821224985848\\
502	0.0106697791608902\\
503	0.0106572304639928\\
504	0.0106444716546572\\
505	0.0106314978198817\\
506	0.0106183038804425\\
507	0.0106048845842016\\
508	0.0105912344991705\\
509	0.0105773480063279\\
510	0.0105632192921886\\
511	0.010548842341124\\
512	0.0105342109274303\\
513	0.0105193186071187\\
514	0.0105041587094202\\
515	0.0104887243280487\\
516	0.0104730083122233\\
517	0.0104570032588406\\
518	0.0104407015036354\\
519	0.0104240951111839\\
520	0.0104071758647688\\
521	0.0103899352010984\\
522	0.0103723642390185\\
523	0.0103544537984531\\
524	0.0103361912858935\\
525	0.0103175697913127\\
526	0.0102985823644446\\
527	0.0102792179033952\\
528	0.0102594628736049\\
529	0.0102393058891938\\
530	0.0102187416612845\\
531	0.0101977572891147\\
532	0.0101763391650815\\
533	0.0101544858517146\\
534	0.0101321650698094\\
535	0.0101093618892514\\
536	0.0100860552896666\\
537	0.0100622279883189\\
538	0.0100378898760208\\
539	0.0100130190160264\\
540	0.00998759235516145\\
541	0.00996162910243391\\
542	0.00993509312227768\\
543	0.00990801253630866\\
544	0.00988029775326948\\
545	0.00985192229656894\\
546	0.00982286658872227\\
547	0.00979310716199265\\
548	0.00976262660047411\\
549	0.00973140497514659\\
550	0.00969931219528145\\
551	0.00966648344620747\\
552	0.00963281165389633\\
553	0.00959842102443266\\
554	0.00956349370009603\\
555	0.00952797230121755\\
556	0.00949159145210537\\
557	0.00945426789194161\\
558	0.00941596751559718\\
559	0.00937665349780352\\
560	0.00933629049997299\\
561	0.00929457458365241\\
562	0.00925160407166974\\
563	0.00920708885269026\\
564	0.00916271247427281\\
565	0.00911868685878312\\
566	0.00907356620335069\\
567	0.00900971250905663\\
568	0.00881541843488129\\
569	0.00839061335662357\\
570	0.00818306859050345\\
571	0.00811407518943064\\
572	0.0080450568354251\\
573	0.00797480709176036\\
574	0.00790325115715567\\
575	0.00783034444470094\\
576	0.00775604347260888\\
577	0.00768030278729031\\
578	0.00760307437871544\\
579	0.00752430736197342\\
580	0.00744394776372258\\
581	0.00736193826527503\\
582	0.00727821791826307\\
583	0.00719272178896829\\
584	0.00710538045136045\\
585	0.00701611910584813\\
586	0.00692485573802267\\
587	0.00683149674994525\\
588	0.00673592587771181\\
589	0.0066379752046507\\
590	0.00653734834917363\\
591	0.00643341581933105\\
592	0.00632466858477865\\
593	0.00620725772627071\\
594	0.0060710901180166\\
595	0.00588956027306329\\
596	0.00559184182825266\\
597	0.00498881296807123\\
598	0.00357511483354343\\
599	0\\
600	0\\
};
\addplot [color=mycolor5,solid,forget plot]
  table[row sep=crcr]{%
1	0.0124902353284585\\
2	0.0124902293664923\\
3	0.0124902233118918\\
4	0.0124902171635064\\
5	0.0124902109201776\\
6	0.012490204580739\\
7	0.0124901981440165\\
8	0.0124901916088281\\
9	0.0124901849739839\\
10	0.0124901782382866\\
11	0.0124901714005312\\
12	0.012490164459505\\
13	0.0124901574139876\\
14	0.0124901502627513\\
15	0.0124901430045607\\
16	0.0124901356381731\\
17	0.012490128162338\\
18	0.0124901205757979\\
19	0.0124901128772873\\
20	0.0124901050655338\\
21	0.012490097139257\\
22	0.0124900890971696\\
23	0.0124900809379764\\
24	0.0124900726603747\\
25	0.0124900642630546\\
26	0.0124900557446983\\
27	0.0124900471039804\\
28	0.012490038339568\\
29	0.0124900294501201\\
30	0.012490020434288\\
31	0.012490011290715\\
32	0.0124900020180364\\
33	0.0124899926148792\\
34	0.0124899830798619\\
35	0.0124899734115947\\
36	0.012489963608679\\
37	0.0124899536697074\\
38	0.0124899435932632\\
39	0.0124899333779204\\
40	0.0124899230222437\\
41	0.0124899125247875\\
42	0.0124899018840964\\
43	0.0124898910987043\\
44	0.0124898801671344\\
45	0.0124898690878988\\
46	0.0124898578594979\\
47	0.0124898464804201\\
48	0.0124898349491415\\
49	0.0124898232641253\\
50	0.0124898114238211\\
51	0.0124897994266649\\
52	0.0124897872710779\\
53	0.0124897749554665\\
54	0.0124897624782213\\
55	0.0124897498377165\\
56	0.0124897370323093\\
57	0.0124897240603392\\
58	0.012489710920127\\
59	0.0124896976099744\\
60	0.0124896841281627\\
61	0.0124896704729525\\
62	0.0124896566425823\\
63	0.0124896426352675\\
64	0.0124896284492\\
65	0.0124896140825465\\
66	0.0124895995334482\\
67	0.0124895848000186\\
68	0.0124895698803436\\
69	0.0124895547724793\\
70	0.0124895394744515\\
71	0.0124895239842538\\
72	0.0124895082998469\\
73	0.0124894924191568\\
74	0.0124894763400736\\
75	0.0124894600604502\\
76	0.0124894435781005\\
77	0.0124894268907983\\
78	0.0124894099962753\\
79	0.01248939289222\\
80	0.0124893755762759\\
81	0.0124893580460395\\
82	0.0124893402990593\\
83	0.0124893223328336\\
84	0.0124893041448087\\
85	0.0124892857323777\\
86	0.012489267092878\\
87	0.0124892482235898\\
88	0.0124892291217345\\
89	0.0124892097844723\\
90	0.0124891902089006\\
91	0.0124891703920523\\
92	0.0124891503308933\\
93	0.0124891300223211\\
94	0.0124891094631626\\
95	0.0124890886501721\\
96	0.0124890675800295\\
97	0.0124890462493382\\
98	0.0124890246546232\\
99	0.0124890027923289\\
100	0.0124889806588174\\
101	0.0124889582503665\\
102	0.0124889355631674\\
103	0.0124889125933234\\
104	0.0124888893368471\\
105	0.0124888657896591\\
106	0.012488841947586\\
107	0.0124888178063581\\
108	0.012488793361608\\
109	0.0124887686088683\\
110	0.01248874354357\\
111	0.0124887181610407\\
112	0.0124886924565025\\
113	0.0124886664250702\\
114	0.01248864006175\\
115	0.0124886133614371\\
116	0.0124885863189142\\
117	0.0124885589288498\\
118	0.0124885311857965\\
119	0.0124885030841889\\
120	0.0124884746183425\\
121	0.0124884457824513\\
122	0.0124884165705866\\
123	0.0124883869766949\\
124	0.0124883569945965\\
125	0.0124883266179836\\
126	0.0124882958404183\\
127	0.0124882646553313\\
128	0.0124882330560198\\
129	0.0124882010356458\\
130	0.0124881685872339\\
131	0.0124881357036699\\
132	0.0124881023776987\\
133	0.0124880686019221\\
134	0.0124880343687968\\
135	0.0124879996706327\\
136	0.0124879644995903\\
137	0.0124879288476787\\
138	0.0124878927067533\\
139	0.0124878560685132\\
140	0.0124878189244991\\
141	0.0124877812660901\\
142	0.0124877430845006\\
143	0.0124877043707775\\
144	0.0124876651157965\\
145	0.0124876253102614\\
146	0.0124875849447066\\
147	0.0124875440095055\\
148	0.0124875024948651\\
149	0.0124874603907828\\
150	0.0124874176870681\\
151	0.0124873743733414\\
152	0.0124873304390307\\
153	0.0124872858733691\\
154	0.0124872406653904\\
155	0.0124871948039269\\
156	0.012487148277605\\
157	0.0124871010748425\\
158	0.0124870531838447\\
159	0.0124870045926009\\
160	0.0124869552888807\\
161	0.0124869052602303\\
162	0.0124868544939689\\
163	0.0124868029771844\\
164	0.0124867506967297\\
165	0.0124866976392188\\
166	0.0124866437910222\\
167	0.0124865891382632\\
168	0.012486533666813\\
169	0.0124864773622869\\
170	0.0124864202100392\\
171	0.0124863621951586\\
172	0.0124863033024637\\
173	0.0124862435164981\\
174	0.0124861828215248\\
175	0.0124861212015215\\
176	0.0124860586401755\\
177	0.0124859951208774\\
178	0.0124859306267162\\
179	0.0124858651404732\\
180	0.012485798644616\\
181	0.0124857311212924\\
182	0.0124856625523242\\
183	0.0124855929192003\\
184	0.0124855222030705\\
185	0.012485450384738\\
186	0.0124853774446527\\
187	0.0124853033629034\\
188	0.0124852281192108\\
189	0.0124851516929191\\
190	0.012485074062988\\
191	0.0124849952079849\\
192	0.0124849151060755\\
193	0.0124848337350158\\
194	0.0124847510721424\\
195	0.0124846670943634\\
196	0.0124845817781487\\
197	0.01248449509952\\
198	0.012484407034041\\
199	0.0124843175568061\\
200	0.0124842266424307\\
201	0.0124841342650394\\
202	0.0124840403982546\\
203	0.0124839450151857\\
204	0.0124838480884163\\
205	0.0124837495899925\\
206	0.0124836494914103\\
207	0.012483547763603\\
208	0.0124834443769283\\
209	0.0124833393011548\\
210	0.0124832325054486\\
211	0.0124831239583597\\
212	0.012483013627808\\
213	0.0124829014810694\\
214	0.0124827874847608\\
215	0.0124826716048263\\
216	0.0124825538065221\\
217	0.0124824340544015\\
218	0.0124823123123004\\
219	0.0124821885433217\\
220	0.01248206270982\\
221	0.0124819347733864\\
222	0.0124818046948328\\
223	0.0124816724341753\\
224	0.0124815379506183\\
225	0.0124814012025367\\
226	0.0124812621474594\\
227	0.012481120742055\\
228	0.0124809769421269\\
229	0.0124808307026182\\
230	0.0124806819775986\\
231	0.0124805307201748\\
232	0.012480376882519\\
233	0.0124802204158544\\
234	0.0124800612704389\\
235	0.0124798993955506\\
236	0.0124797347394715\\
237	0.0124795672494726\\
238	0.0124793968717981\\
239	0.0124792235516502\\
240	0.0124790472331739\\
241	0.0124788678594421\\
242	0.0124786853724408\\
243	0.0124784997130536\\
244	0.012478310821043\\
245	0.0124781186350241\\
246	0.0124779230924268\\
247	0.0124777241294588\\
248	0.0124775216811095\\
249	0.0124773156811916\\
250	0.0124771060622543\\
251	0.012476892755541\\
252	0.0124766756909423\\
253	0.0124764547969787\\
254	0.0124762300009113\\
255	0.0124760012290689\\
256	0.0124757684068646\\
257	0.0124755314575005\\
258	0.0124752903027308\\
259	0.0124750448628285\\
260	0.0124747950565549\\
261	0.0124745408011309\\
262	0.0124742820121952\\
263	0.0124740186037331\\
264	0.0124737504880505\\
265	0.0124734775757329\\
266	0.0124731997756005\\
267	0.0124729169946593\\
268	0.0124726291380521\\
269	0.0124723361090053\\
270	0.0124720378087735\\
271	0.0124717341365791\\
272	0.0124714249895452\\
273	0.0124711102626228\\
274	0.0124707898485164\\
275	0.0124704636376436\\
276	0.0124701315181834\\
277	0.0124697933761739\\
278	0.0124694490951707\\
279	0.0124690985559843\\
280	0.0124687416368001\\
281	0.0124683782130985\\
282	0.0124680081576039\\
283	0.0124676313402836\\
284	0.0124672476282676\\
285	0.0124668568857217\\
286	0.0124664589735756\\
287	0.0124660537490618\\
288	0.0124656410656329\\
289	0.0124652207740919\\
290	0.0124647927217348\\
291	0.0124643567522475\\
292	0.012463912705599\\
293	0.012463460417931\\
294	0.012462999721442\\
295	0.0124625304442654\\
296	0.0124620524103428\\
297	0.0124615654393033\\
298	0.0124610693463769\\
299	0.0124605639423743\\
300	0.0124600490336132\\
301	0.0124595244213592\\
302	0.0124589899018524\\
303	0.0124584452662088\\
304	0.0124578903002627\\
305	0.0124573247843877\\
306	0.0124567484932576\\
307	0.0124561611954554\\
308	0.0124555626527871\\
309	0.0124549526193216\\
310	0.0124543308408411\\
311	0.0124536970554987\\
312	0.0124530509990368\\
313	0.012452392418624\\
314	0.012451721038384\\
315	0.0124510365714501\\
316	0.0124503387249437\\
317	0.0124496272039903\\
318	0.0124489017043215\\
319	0.0124481619117428\\
320	0.0124474075047502\\
321	0.0124466381546771\\
322	0.0124458535246763\\
323	0.0124450532649725\\
324	0.0124442370051025\\
325	0.0124434043675366\\
326	0.0124425549696913\\
327	0.0124416884174781\\
328	0.0124408043048335\\
329	0.0124399022138936\\
330	0.0124389817163567\\
331	0.0124380423728597\\
332	0.012437083726933\\
333	0.0124361053081795\\
334	0.0124351066317933\\
335	0.0124340871973913\\
336	0.0124330464858498\\
337	0.0124319839543395\\
338	0.0124308990402332\\
339	0.0124297911680455\\
340	0.0124286597409069\\
341	0.0124275041382953\\
342	0.0124263237122196\\
343	0.0124251177801756\\
344	0.0124238856125472\\
345	0.0124226264163516\\
346	0.0124213393436645\\
347	0.012420023597371\\
348	0.0124186783311737\\
349	0.0124173030564975\\
350	0.0124158968898539\\
351	0.0124144588741656\\
352	0.0124129877934208\\
353	0.0124114824825798\\
354	0.0124099418033697\\
355	0.0124083646192341\\
356	0.0124067497381982\\
357	0.0124050959010704\\
358	0.0124034018065269\\
359	0.0124016660890458\\
360	0.0123998872475998\\
361	0.0123980636728532\\
362	0.0123961937996587\\
363	0.0123942760012171\\
364	0.0123923085856317\\
365	0.0123902897861636\\
366	0.0123882177334646\\
367	0.0123860903869722\\
368	0.0123839054314689\\
369	0.012381660513089\\
370	0.0123793549436391\\
371	0.0123769869410401\\
372	0.0123745549049348\\
373	0.0123720573981417\\
374	0.0123694921657331\\
375	0.0123668525718493\\
376	0.0123641434723573\\
377	0.0123613612308389\\
378	0.0123585019905699\\
379	0.0123555614765225\\
380	0.012352533254238\\
381	0.0123494109560399\\
382	0.0123461883904216\\
383	0.0123428587357301\\
384	0.012339414416079\\
385	0.0123358468573877\\
386	0.0123321458613464\\
387	0.0123282977885207\\
388	0.012324282093484\\
389	0.0123200946013244\\
390	0.0123157300149129\\
391	0.01231117831966\\
392	0.0123064341129794\\
393	0.0123014792917886\\
394	0.0122962721209553\\
395	0.0122907884681141\\
396	0.0122849957863516\\
397	0.0122717496953838\\
398	0.0122570401769288\\
399	0.0122420484885861\\
400	0.0122267608862323\\
401	0.0122111621527093\\
402	0.0121952349369981\\
403	0.0121789572001542\\
404	0.012162287564809\\
405	0.012145210527299\\
406	0.0121277276505989\\
407	0.0121098184262074\\
408	0.0120915212598722\\
409	0.012072736941335\\
410	0.0120534270861025\\
411	0.0120335748564956\\
412	0.0120130715916027\\
413	0.0119918571744927\\
414	0.0119698626984003\\
415	0.0119397099674708\\
416	0.0119041788690955\\
417	0.0118682198352206\\
418	0.0118318339627868\\
419	0.0117950261756174\\
420	0.0117577990437274\\
421	0.0117201546414928\\
422	0.0116820909786053\\
423	0.0116436150224078\\
424	0.0116047369665029\\
425	0.0115654727440228\\
426	0.0115258431628656\\
427	0.0114858746588462\\
428	0.0114456005294182\\
429	0.0114050625288854\\
430	0.0113643131762869\\
431	0.011333352114047\\
432	0.0113191539921617\\
433	0.0113051183353929\\
434	0.011291278209906\\
435	0.011277670594299\\
436	0.0112643366313084\\
437	0.0112513221058282\\
438	0.0112386780243766\\
439	0.0112264612790195\\
440	0.0112147354101465\\
441	0.011203571485085\\
442	0.0111930491118217\\
443	0.0111832576062024\\
444	0.0111742973131731\\
445	0.0111662812296339\\
446	0.011158757195801\\
447	0.0111513097040762\\
448	0.0111439421227419\\
449	0.0111366569094769\\
450	0.0111294553533079\\
451	0.0111223372626344\\
452	0.0111153005885831\\
453	0.0111083409708036\\
454	0.0111014511906871\\
455	0.0110946205129858\\
456	0.0110878338881117\\
457	0.0110810709892686\\
458	0.0110743050536873\\
459	0.0110675014940417\\
460	0.0110606483815249\\
461	0.0110537422847705\\
462	0.0110467794114533\\
463	0.0110397556126491\\
464	0.0110326663993794\\
465	0.0110255069755062\\
466	0.011018272292372\\
467	0.0110109571319796\\
468	0.0110035562271484\\
469	0.0109960644293726\\
470	0.0109884769379745\\
471	0.0109807896077161\\
472	0.0109729993564934\\
473	0.0109651034414379\\
474	0.0109570990735483\\
475	0.0109489834288188\\
476	0.0109407536604105\\
477	0.0109324069114435\\
478	0.010923940327742\\
479	0.0109153510695292\\
480	0.010906636320608\\
481	0.0108977932929509\\
482	0.0108888192237977\\
483	0.0108797113612692\\
484	0.0108704669330506\\
485	0.0108610831069969\\
486	0.0108515569896736\\
487	0.0108418856244206\\
488	0.010832065988842\\
489	0.0108220949916262\\
490	0.0108119694686164\\
491	0.0108016861780791\\
492	0.0107912417951752\\
493	0.0107806329057225\\
494	0.0107698559994798\\
495	0.0107589074633869\\
496	0.0107477835754997\\
497	0.0107364805003073\\
498	0.0107249942837879\\
499	0.010713320848206\\
500	0.0107014559866558\\
501	0.010689395357361\\
502	0.0106771344777492\\
503	0.0106646687183215\\
504	0.0106519932963427\\
505	0.0106391032693764\\
506	0.0106259935286803\\
507	0.0106126587924579\\
508	0.0105990935989242\\
509	0.0105852922991342\\
510	0.0105712490495774\\
511	0.0105569578045387\\
512	0.0105424123082307\\
513	0.0105276060867004\\
514	0.0105125324395163\\
515	0.0104971844312409\\
516	0.010481554882691\\
517	0.0104656363619528\\
518	0.0104494211751765\\
519	0.0104329013571813\\
520	0.0104160686618839\\
521	0.010398914554042\\
522	0.0103814301998287\\
523	0.010363606456479\\
524	0.0103454338618867\\
525	0.0103269025770548\\
526	0.0103080023982409\\
527	0.0102887227844879\\
528	0.0102690509745637\\
529	0.0102489754108345\\
530	0.0102284897740996\\
531	0.0102075815648557\\
532	0.0101862375709781\\
533	0.0101644392206717\\
534	0.0101421829086899\\
535	0.010119454573835\\
536	0.0100962391115006\\
537	0.0100725256838767\\
538	0.0100482986053227\\
539	0.0100235261175771\\
540	0.00999818247812129\\
541	0.00997228941853672\\
542	0.00994581507060125\\
543	0.00991870999703562\\
544	0.00989099442621836\\
545	0.00986265677447937\\
546	0.00983370215581083\\
547	0.0098041036567173\\
548	0.00977379119656501\\
549	0.00974274914025589\\
550	0.00971094972091856\\
551	0.00967837072564764\\
552	0.00964490231605043\\
553	0.0096106266264305\\
554	0.00957550432320249\\
555	0.0095395458041511\\
556	0.00950287238659638\\
557	0.00946573856617259\\
558	0.0094276828600046\\
559	0.00938865888075084\\
560	0.00934860316479576\\
561	0.00930747516300949\\
562	0.00926523327873259\\
563	0.00922131617172663\\
564	0.00917625117725328\\
565	0.00913000690891885\\
566	0.0090841631426429\\
567	0.00903806347343111\\
568	0.00899074899630517\\
569	0.008809378738224\\
570	0.00846359421886279\\
571	0.00812512099772187\\
572	0.00804549373171191\\
573	0.00797486330201451\\
574	0.00790326132056656\\
575	0.0078303468390966\\
576	0.0077560442470378\\
577	0.0076803031522189\\
578	0.00760307453625226\\
579	0.00752430744274301\\
580	0.00744394780425085\\
581	0.00736193828664806\\
582	0.00727821792858083\\
583	0.00719272179374158\\
584	0.00710538045286112\\
585	0.00701611910608783\\
586	0.00692485573802267\\
587	0.00683149674994525\\
588	0.00673592587771182\\
589	0.00663797520465071\\
590	0.00653734834917363\\
591	0.00643341581933105\\
592	0.00632466858477866\\
593	0.00620725772627071\\
594	0.0060710901180166\\
595	0.00588956027306329\\
596	0.00559184182825266\\
597	0.00498881296807123\\
598	0.00357511483354343\\
599	0\\
600	0\\
};
\addplot [color=mycolor6,solid,forget plot]
  table[row sep=crcr]{%
1	0.0124903499082909\\
2	0.0124903457713311\\
3	0.0124903415746704\\
4	0.0124903373176105\\
5	0.0124903329994472\\
6	0.01249032861947\\
7	0.0124903241769623\\
8	0.0124903196712012\\
9	0.0124903151014572\\
10	0.0124903104669946\\
11	0.0124903057670707\\
12	0.0124903010009364\\
13	0.0124902961678356\\
14	0.0124902912670051\\
15	0.0124902862976747\\
16	0.0124902812590668\\
17	0.0124902761503966\\
18	0.0124902709708714\\
19	0.012490265719691\\
20	0.012490260396047\\
21	0.0124902549991232\\
22	0.0124902495280949\\
23	0.0124902439821287\\
24	0.0124902383603829\\
25	0.0124902326620065\\
26	0.0124902268861394\\
27	0.0124902210319122\\
28	0.0124902150984455\\
29	0.0124902090848503\\
30	0.0124902029902272\\
31	0.0124901968136663\\
32	0.0124901905542469\\
33	0.0124901842110371\\
34	0.0124901777830937\\
35	0.0124901712694616\\
36	0.0124901646691735\\
37	0.0124901579812498\\
38	0.0124901512046978\\
39	0.0124901443385116\\
40	0.0124901373816717\\
41	0.0124901303331444\\
42	0.0124901231918815\\
43	0.0124901159568197\\
44	0.0124901086268805\\
45	0.0124901012009692\\
46	0.0124900936779748\\
47	0.0124900860567693\\
48	0.0124900783362073\\
49	0.0124900705151254\\
50	0.0124900625923415\\
51	0.0124900545666544\\
52	0.0124900464368432\\
53	0.0124900382016665\\
54	0.0124900298598623\\
55	0.0124900214101464\\
56	0.0124900128512128\\
57	0.0124900041817323\\
58	0.0124899954003521\\
59	0.012489986505695\\
60	0.0124899774963588\\
61	0.0124899683709154\\
62	0.0124899591279101\\
63	0.012489949765861\\
64	0.0124899402832576\\
65	0.0124899306785611\\
66	0.0124899209502022\\
67	0.0124899110965816\\
68	0.0124899011160681\\
69	0.0124898910069984\\
70	0.0124898807676758\\
71	0.0124898703963696\\
72	0.0124898598913141\\
73	0.0124898492507074\\
74	0.012489838472711\\
75	0.0124898275554484\\
76	0.0124898164970044\\
77	0.0124898052954239\\
78	0.012489793948711\\
79	0.0124897824548284\\
80	0.0124897708116956\\
81	0.0124897590171886\\
82	0.0124897470691385\\
83	0.0124897349653306\\
84	0.0124897227035032\\
85	0.012489710281347\\
86	0.0124896976965033\\
87	0.0124896849465636\\
88	0.0124896720290683\\
89	0.0124896589415056\\
90	0.0124896456813102\\
91	0.0124896322458628\\
92	0.0124896186324884\\
93	0.0124896048384553\\
94	0.0124895908609746\\
95	0.0124895766971982\\
96	0.0124895623442183\\
97	0.0124895477990661\\
98	0.0124895330587106\\
99	0.0124895181200576\\
100	0.0124895029799485\\
101	0.0124894876351591\\
102	0.0124894720823987\\
103	0.0124894563183087\\
104	0.0124894403394615\\
105	0.0124894241423594\\
106	0.0124894077234333\\
107	0.0124893910790418\\
108	0.0124893742054697\\
109	0.0124893570989268\\
110	0.0124893397555472\\
111	0.0124893221713872\\
112	0.0124893043424251\\
113	0.0124892862645589\\
114	0.0124892679336058\\
115	0.0124892493453008\\
116	0.0124892304952948\\
117	0.0124892113791542\\
118	0.0124891919923588\\
119	0.0124891723303009\\
120	0.0124891523882836\\
121	0.0124891321615195\\
122	0.0124891116451293\\
123	0.0124890908341404\\
124	0.0124890697234852\\
125	0.0124890483079995\\
126	0.0124890265824214\\
127	0.0124890045413891\\
128	0.0124889821794397\\
129	0.0124889594910074\\
130	0.0124889364704216\\
131	0.0124889131119055\\
132	0.012488889409574\\
133	0.012488865357432\\
134	0.0124888409493727\\
135	0.0124888161791753\\
136	0.0124887910405034\\
137	0.0124887655269029\\
138	0.0124887396317999\\
139	0.0124887133484985\\
140	0.0124886866701788\\
141	0.0124886595898949\\
142	0.0124886321005722\\
143	0.0124886041950052\\
144	0.012488575865856\\
145	0.0124885471056511\\
146	0.0124885179067796\\
147	0.0124884882614893\\
148	0.0124884581618857\\
149	0.0124884275999282\\
150	0.0124883965674282\\
151	0.0124883650560457\\
152	0.0124883330572873\\
153	0.0124883005625026\\
154	0.0124882675628817\\
155	0.0124882340494525\\
156	0.0124882000130769\\
157	0.0124881654444488\\
158	0.0124881303340898\\
159	0.0124880946723469\\
160	0.0124880584493885\\
161	0.0124880216552018\\
162	0.0124879842795884\\
163	0.0124879463121616\\
164	0.0124879077423424\\
165	0.0124878685593557\\
166	0.0124878287522271\\
167	0.0124877883097781\\
168	0.0124877472206231\\
169	0.0124877054731645\\
170	0.0124876630555894\\
171	0.0124876199558645\\
172	0.0124875761617325\\
173	0.0124875316607072\\
174	0.012487486440069\\
175	0.0124874404868604\\
176	0.0124873937878812\\
177	0.0124873463296835\\
178	0.0124872980985669\\
179	0.0124872490805731\\
180	0.0124871992614812\\
181	0.0124871486268017\\
182	0.0124870971617717\\
183	0.0124870448513489\\
184	0.0124869916802059\\
185	0.0124869376327249\\
186	0.0124868826929909\\
187	0.0124868268447865\\
188	0.0124867700715851\\
189	0.0124867123565449\\
190	0.0124866536825024\\
191	0.0124865940319656\\
192	0.0124865333871077\\
193	0.01248647172976\\
194	0.012486409041405\\
195	0.0124863453031693\\
196	0.0124862804958165\\
197	0.0124862145997398\\
198	0.0124861475949546\\
199	0.0124860794610907\\
200	0.0124860101773851\\
201	0.0124859397226737\\
202	0.0124858680753834\\
203	0.0124857952135245\\
204	0.0124857211146819\\
205	0.0124856457560075\\
206	0.0124855691142111\\
207	0.0124854911655526\\
208	0.0124854118858327\\
209	0.0124853312503847\\
210	0.0124852492340655\\
211	0.0124851658112465\\
212	0.0124850809558048\\
213	0.0124849946411139\\
214	0.0124849068400345\\
215	0.012484817524905\\
216	0.0124847266675323\\
217	0.0124846342391818\\
218	0.0124845402105681\\
219	0.0124844445518448\\
220	0.0124843472325947\\
221	0.0124842482218196\\
222	0.0124841474879302\\
223	0.0124840449987352\\
224	0.0124839407214309\\
225	0.0124838346225904\\
226	0.0124837266681531\\
227	0.0124836168234147\\
228	0.0124835050530174\\
229	0.0124833913209377\\
230	0.012483275590469\\
231	0.0124831578242137\\
232	0.0124830379840704\\
233	0.0124829160312219\\
234	0.0124827919261219\\
235	0.012482665628482\\
236	0.0124825370972582\\
237	0.0124824062906366\\
238	0.0124822731660197\\
239	0.0124821376800112\\
240	0.012481999788401\\
241	0.0124818594461497\\
242	0.0124817166073725\\
243	0.0124815712253218\\
244	0.0124814232523689\\
245	0.0124812726399846\\
246	0.0124811193387193\\
247	0.0124809632981872\\
248	0.0124808044670495\\
249	0.0124806427929892\\
250	0.0124804782226875\\
251	0.0124803107018014\\
252	0.0124801401749449\\
253	0.0124799665856824\\
254	0.0124797898765304\\
255	0.0124796099889194\\
256	0.0124794268630742\\
257	0.0124792404380561\\
258	0.0124790506517356\\
259	0.0124788574407657\\
260	0.0124786607405545\\
261	0.0124784604852348\\
262	0.0124782566076328\\
263	0.0124780490392382\\
264	0.0124778377101742\\
265	0.0124776225491646\\
266	0.0124774034835018\\
267	0.0124771804390127\\
268	0.0124769533400247\\
269	0.0124767221093299\\
270	0.0124764866681489\\
271	0.0124762469360934\\
272	0.0124760028311273\\
273	0.0124757542695289\\
274	0.012475501165856\\
275	0.0124752434329172\\
276	0.0124749809817402\\
277	0.0124747137215045\\
278	0.0124744415594877\\
279	0.0124741644010382\\
280	0.0124738821495321\\
281	0.0124735947063317\\
282	0.0124733019707468\\
283	0.0124730038399861\\
284	0.0124727002091024\\
285	0.0124723909709238\\
286	0.012472076015978\\
287	0.0124717552324584\\
288	0.0124714285062603\\
289	0.0124710957208649\\
290	0.0124707567572845\\
291	0.0124704114940054\\
292	0.0124700598069298\\
293	0.0124697015693148\\
294	0.0124693366517105\\
295	0.0124689649218955\\
296	0.0124685862448123\\
297	0.0124682004825048\\
298	0.0124678074940573\\
299	0.0124674071355199\\
300	0.0124669992597981\\
301	0.0124665837165906\\
302	0.0124661603523118\\
303	0.0124657290100063\\
304	0.0124652895292559\\
305	0.0124648417460756\\
306	0.0124643854927882\\
307	0.0124639205978698\\
308	0.0124634468857859\\
309	0.012462964176897\\
310	0.0124624722875289\\
311	0.0124619710304049\\
312	0.0124614602151652\\
313	0.0124609396456024\\
314	0.0124604091206361\\
315	0.0124598684346352\\
316	0.012459317377499\\
317	0.0124587557339302\\
318	0.012458183283296\\
319	0.0124575997996936\\
320	0.0124570050517805\\
321	0.0124563988024518\\
322	0.0124557808082306\\
323	0.0124551508186964\\
324	0.0124545085774243\\
325	0.0124538538218701\\
326	0.0124531862826627\\
327	0.0124525056833877\\
328	0.0124518117404315\\
329	0.0124511041628745\\
330	0.0124503826521337\\
331	0.012449646901259\\
332	0.012448896594983\\
333	0.01244813140941\\
334	0.0124473510116013\\
335	0.0124465550589769\\
336	0.0124457431987077\\
337	0.0124449150678924\\
338	0.012444070293746\\
339	0.0124432084925006\\
340	0.0124423292688365\\
341	0.0124414322151302\\
342	0.0124405169103919\\
343	0.0124395829188694\\
344	0.0124386297890041\\
345	0.0124376570554372\\
346	0.0124366642467256\\
347	0.0124356508814378\\
348	0.0124346164902014\\
349	0.0124335605588321\\
350	0.0124324825511272\\
351	0.0124313818982378\\
352	0.0124302580222779\\
353	0.0124291103333074\\
354	0.0124279382259728\\
355	0.0124267410735406\\
356	0.0124255182263614\\
357	0.0124242690126032\\
358	0.0124229927343146\\
359	0.0124216886610799\\
360	0.0124203560335352\\
361	0.0124189940735666\\
362	0.0124176019732497\\
363	0.0124161788924943\\
364	0.0124147239555812\\
365	0.0124132362456239\\
366	0.0124117147965536\\
367	0.012410158589031\\
368	0.0124085665820927\\
369	0.0124069378036266\\
370	0.0124052711457885\\
371	0.0124035654530817\\
372	0.0124018194942684\\
373	0.0124000318633614\\
374	0.012398200882086\\
375	0.0123963256058482\\
376	0.0123944044999658\\
377	0.0123924359608527\\
378	0.0123904182783274\\
379	0.0123883495256649\\
380	0.0123862277531902\\
381	0.0123840509848023\\
382	0.0123818171483689\\
383	0.0123795240637716\\
384	0.012377169416278\\
385	0.0123747506945979\\
386	0.0123722650765578\\
387	0.0123697094640008\\
388	0.0123670820071721\\
389	0.0123643807447834\\
390	0.0123616034346911\\
391	0.0123587480375856\\
392	0.0123558114438562\\
393	0.0123527892847153\\
394	0.012349678499905\\
395	0.0123464751769515\\
396	0.0123431723992934\\
397	0.0123397719775921\\
398	0.0123362693978358\\
399	0.0123326580223497\\
400	0.0123289305131901\\
401	0.0123250786061563\\
402	0.0123210925229418\\
403	0.0123169592462861\\
404	0.0123126590341093\\
405	0.0123081849892889\\
406	0.0123035371390292\\
407	0.0122987089766026\\
408	0.0122937117987223\\
409	0.012288499239166\\
410	0.0122830527393027\\
411	0.0122773569247035\\
412	0.0122713590208821\\
413	0.01226502611952\\
414	0.0122583126776527\\
415	0.0122451202917145\\
416	0.0122275962977738\\
417	0.0122097245096778\\
418	0.0121914798875531\\
419	0.0121728035719554\\
420	0.0121537201252082\\
421	0.0121342283835598\\
422	0.0121143944015873\\
423	0.0120941035622057\\
424	0.0120733228868494\\
425	0.0120519704093852\\
426	0.0120300009381597\\
427	0.0120073630960932\\
428	0.0119839982400571\\
429	0.0119598391929415\\
430	0.0119348088642074\\
431	0.0119035718607269\\
432	0.0118626644383289\\
433	0.0118212840867717\\
434	0.0117794275159407\\
435	0.0117370890367737\\
436	0.0116942726696179\\
437	0.0116509860369291\\
438	0.0116072397585825\\
439	0.0115630480879793\\
440	0.0115184296888694\\
441	0.0114734085958313\\
442	0.0114280154387347\\
443	0.0113822891369773\\
444	0.0113362797848445\\
445	0.0112900485108053\\
446	0.0112604417470562\\
447	0.0112445871699175\\
448	0.0112289467408647\\
449	0.0112135628538908\\
450	0.0111984827755775\\
451	0.0111837592613211\\
452	0.0111694512583896\\
453	0.0111556247100162\\
454	0.0111423534654628\\
455	0.0111297203450589\\
456	0.0111178185251644\\
457	0.0111067530814234\\
458	0.0110966427290473\\
459	0.0110876217424687\\
460	0.0110789511509886\\
461	0.0110703603876142\\
462	0.0110618523880424\\
463	0.01105342880017\\
464	0.0110450896302439\\
465	0.0110368328130501\\
466	0.0110286536879707\\
467	0.0110205443636153\\
468	0.0110124929529589\\
469	0.0110044826511298\\
470	0.0109964906225358\\
471	0.0109884866589396\\
472	0.0109804315636604\\
473	0.010972309402298\\
474	0.0109641158735388\\
475	0.0109558462453186\\
476	0.0109474953634065\\
477	0.0109390576763981\\
478	0.0109305272825391\\
479	0.0109218980054371\\
480	0.0109131635076103\\
481	0.0109043174532553\\
482	0.0108953537346557\\
483	0.0108862667804902\\
484	0.0108770519691698\\
485	0.0108677057439626\\
486	0.0108582244698403\\
487	0.0108486044452207\\
488	0.0108388419150288\\
489	0.0108289330846297\\
490	0.0108188741338943\\
491	0.0108086612302689\\
492	0.0107982905391795\\
493	0.0107877582293795\\
494	0.0107770604698671\\
495	0.0107661934137049\\
496	0.010755153162328\\
497	0.0107439357145074\\
498	0.0107325369632783\\
499	0.0107209526922572\\
500	0.010709178571227\\
501	0.0106972101508757\\
502	0.0106850428565912\\
503	0.010672671981256\\
504	0.0106600926770576\\
505	0.01064729994645\\
506	0.0106342886325849\\
507	0.0106210534098167\\
508	0.0106075887752942\\
509	0.010593889041686\\
510	0.0105799483295842\\
511	0.0105657605595954\\
512	0.0105513194441342\\
513	0.010536618478944\\
514	0.010521650934377\\
515	0.0105064098464739\\
516	0.0104908880078865\\
517	0.0104750779586877\\
518	0.0104589719770975\\
519	0.0104425620701236\\
520	0.0104258399640507\\
521	0.0104087970947348\\
522	0.0103914245977489\\
523	0.0103737132984263\\
524	0.0103556537018252\\
525	0.010337235983902\\
526	0.0103184499820798\\
527	0.0102992851848024\\
528	0.0102797307211344\\
529	0.0102597753222085\\
530	0.0102394072935273\\
531	0.0102186145643185\\
532	0.0101973846774478\\
533	0.0101756997798491\\
534	0.0101535545461412\\
535	0.0101309351733925\\
536	0.0101078269266932\\
537	0.0100842124706025\\
538	0.0100600770696729\\
539	0.0100354112287115\\
540	0.0100101981158954\\
541	0.00998442768607106\\
542	0.00995808447059193\\
543	0.00993111835930172\\
544	0.00990352491215651\\
545	0.00987528721777292\\
546	0.0098463777269695\\
547	0.00981678641520482\\
548	0.00978652235414809\\
549	0.00975556467757381\\
550	0.00972393918831744\\
551	0.0096915577078595\\
552	0.00965838440114489\\
553	0.00962440005291455\\
554	0.00958951789549794\\
555	0.00955373339702645\\
556	0.00951711378512381\\
557	0.00947950003892096\\
558	0.0094411129256838\\
559	0.00940205772885252\\
560	0.00936225442642091\\
561	0.00932144327520121\\
562	0.0092795526745926\\
563	0.00923652889629026\\
564	0.00919214483043075\\
565	0.00914631274642346\\
566	0.0090990966384936\\
567	0.0090509252644886\\
568	0.00900320175753261\\
569	0.00895486878153357\\
570	0.0088151415759191\\
571	0.00856330285569335\\
572	0.00813856590265964\\
573	0.00797840592915509\\
574	0.00790370578700228\\
575	0.00783042495555452\\
576	0.00775606190032862\\
577	0.00768030853877576\\
578	0.00760307706941287\\
579	0.00752430849590189\\
580	0.0074439483406399\\
581	0.00736193855163164\\
582	0.00727821806930863\\
583	0.00719272186170981\\
584	0.00710538048536775\\
585	0.00701611911660432\\
586	0.00692485573974563\\
587	0.00683149674994525\\
588	0.00673592587771182\\
589	0.00663797520465072\\
590	0.00653734834917363\\
591	0.00643341581933105\\
592	0.00632466858477866\\
593	0.00620725772627072\\
594	0.00607109011801661\\
595	0.0058895602730633\\
596	0.00559184182825266\\
597	0.00498881296807123\\
598	0.00357511483354343\\
599	0\\
600	0\\
};
\addplot [color=mycolor7,solid,forget plot]
  table[row sep=crcr]{%
1	0.0124905348581479\\
2	0.0124905320101193\\
3	0.0124905291226254\\
4	0.0124905261951872\\
5	0.0124905232273194\\
6	0.0124905202185305\\
7	0.0124905171683227\\
8	0.0124905140761919\\
9	0.012490510941627\\
10	0.0124905077641104\\
11	0.0124905045431175\\
12	0.0124905012781165\\
13	0.0124904979685683\\
14	0.0124904946139264\\
15	0.0124904912136367\\
16	0.012490487767137\\
17	0.0124904842738573\\
18	0.0124904807332194\\
19	0.0124904771446363\\
20	0.0124904735075128\\
21	0.0124904698212445\\
22	0.012490466085218\\
23	0.0124904622988105\\
24	0.0124904584613897\\
25	0.0124904545723135\\
26	0.0124904506309296\\
27	0.0124904466365754\\
28	0.0124904425885778\\
29	0.0124904384862527\\
30	0.0124904343289049\\
31	0.0124904301158279\\
32	0.0124904258463031\\
33	0.0124904215196001\\
34	0.0124904171349763\\
35	0.012490412691676\\
36	0.012490408188931\\
37	0.0124904036259594\\
38	0.0124903990019657\\
39	0.0124903943161406\\
40	0.0124903895676602\\
41	0.012490384755686\\
42	0.0124903798793642\\
43	0.0124903749378258\\
44	0.0124903699301856\\
45	0.0124903648555425\\
46	0.0124903597129784\\
47	0.0124903545015582\\
48	0.0124903492203295\\
49	0.0124903438683217\\
50	0.0124903384445459\\
51	0.0124903329479946\\
52	0.0124903273776407\\
53	0.0124903217324375\\
54	0.0124903160113183\\
55	0.0124903102131955\\
56	0.0124903043369605\\
57	0.0124902983814829\\
58	0.0124902923456103\\
59	0.0124902862281677\\
60	0.0124902800279567\\
61	0.0124902737437554\\
62	0.0124902673743178\\
63	0.0124902609183729\\
64	0.0124902543746245\\
65	0.0124902477417506\\
66	0.0124902410184028\\
67	0.0124902342032057\\
68	0.0124902272947564\\
69	0.0124902202916237\\
70	0.0124902131923478\\
71	0.0124902059954398\\
72	0.0124901986993805\\
73	0.0124901913026205\\
74	0.012490183803579\\
75	0.0124901762006435\\
76	0.0124901684921693\\
77	0.0124901606764784\\
78	0.0124901527518593\\
79	0.012490144716566\\
80	0.0124901365688176\\
81	0.0124901283067975\\
82	0.0124901199286527\\
83	0.0124901114324933\\
84	0.0124901028163915\\
85	0.0124900940783813\\
86	0.0124900852164573\\
87	0.0124900762285744\\
88	0.0124900671126469\\
89	0.0124900578665479\\
90	0.0124900484881081\\
91	0.0124900389751158\\
92	0.0124900293253154\\
93	0.0124900195364071\\
94	0.0124900096060461\\
95	0.0124899995318414\\
96	0.0124899893113553\\
97	0.0124899789421028\\
98	0.0124899684215502\\
99	0.0124899577471148\\
100	0.0124899469161636\\
101	0.0124899359260128\\
102	0.0124899247739268\\
103	0.0124899134571169\\
104	0.012489901972741\\
105	0.0124898903179025\\
106	0.0124898784896489\\
107	0.0124898664849714\\
108	0.0124898543008035\\
109	0.0124898419340203\\
110	0.0124898293814372\\
111	0.0124898166398092\\
112	0.0124898037058292\\
113	0.0124897905761276\\
114	0.0124897772472707\\
115	0.0124897637157599\\
116	0.0124897499780301\\
117	0.0124897360304489\\
118	0.0124897218693152\\
119	0.0124897074908578\\
120	0.0124896928912346\\
121	0.0124896780665306\\
122	0.0124896630127571\\
123	0.0124896477258502\\
124	0.0124896322016691\\
125	0.0124896164359952\\
126	0.01248960042453\\
127	0.0124895841628944\\
128	0.0124895676466263\\
129	0.0124895508711796\\
130	0.0124895338319227\\
131	0.0124895165241365\\
132	0.012489498943013\\
133	0.0124894810836536\\
134	0.0124894629410675\\
135	0.01248944451017\\
136	0.0124894257857809\\
137	0.0124894067626225\\
138	0.0124893874353184\\
139	0.0124893677983916\\
140	0.0124893478462627\\
141	0.0124893275732491\\
142	0.0124893069735627\\
143	0.0124892860413098\\
144	0.0124892647704892\\
145	0.0124892431549916\\
146	0.0124892211885953\\
147	0.0124891988649551\\
148	0.0124891761775888\\
149	0.0124891531199295\\
150	0.0124891296852879\\
151	0.0124891058668498\\
152	0.0124890816576739\\
153	0.0124890570506893\\
154	0.012489032038693\\
155	0.0124890066143472\\
156	0.0124889807701772\\
157	0.0124889544985682\\
158	0.0124889277917629\\
159	0.0124889006418589\\
160	0.0124888730408054\\
161	0.0124888449804008\\
162	0.0124888164522896\\
163	0.0124887874479595\\
164	0.0124887579587381\\
165	0.0124887279757901\\
166	0.0124886974901141\\
167	0.0124886664925391\\
168	0.0124886349737216\\
169	0.0124886029241418\\
170	0.0124885703341007\\
171	0.0124885371937159\\
172	0.0124885034929189\\
173	0.0124884692214506\\
174	0.0124884343688584\\
175	0.0124883989244917\\
176	0.0124883628774985\\
177	0.0124883262168214\\
178	0.0124882889311937\\
179	0.0124882510091348\\
180	0.0124882124389469\\
181	0.0124881732087102\\
182	0.0124881333062785\\
183	0.0124880927192753\\
184	0.0124880514350891\\
185	0.0124880094408688\\
186	0.0124879667235192\\
187	0.0124879232696962\\
188	0.012487879065802\\
189	0.0124878340979806\\
190	0.0124877883521122\\
191	0.0124877418138088\\
192	0.0124876944684088\\
193	0.0124876463009715\\
194	0.0124875972962724\\
195	0.0124875474387975\\
196	0.0124874967127378\\
197	0.0124874451019837\\
198	0.0124873925901195\\
199	0.0124873391604176\\
200	0.0124872847958326\\
201	0.0124872294789953\\
202	0.0124871731922067\\
203	0.0124871159174319\\
204	0.0124870576362937\\
205	0.0124869983300665\\
206	0.0124869379796693\\
207	0.0124868765656599\\
208	0.0124868140682273\\
209	0.0124867504671856\\
210	0.0124866857419665\\
211	0.0124866198716126\\
212	0.0124865528347702\\
213	0.0124864846096814\\
214	0.0124864151741774\\
215	0.0124863445056701\\
216	0.0124862725811449\\
217	0.0124861993771526\\
218	0.0124861248698009\\
219	0.0124860490347467\\
220	0.0124859718471871\\
221	0.0124858932818512\\
222	0.012485813312991\\
223	0.0124857319143724\\
224	0.0124856490592661\\
225	0.0124855647204379\\
226	0.0124854788701395\\
227	0.0124853914800986\\
228	0.0124853025215084\\
229	0.0124852119650172\\
230	0.0124851197807177\\
231	0.0124850259381366\\
232	0.0124849304062229\\
233	0.0124848331533369\\
234	0.0124847341472384\\
235	0.0124846333550744\\
236	0.0124845307433674\\
237	0.0124844262780019\\
238	0.0124843199242127\\
239	0.0124842116465705\\
240	0.012484101408969\\
241	0.0124839891746109\\
242	0.0124838749059936\\
243	0.0124837585648943\\
244	0.0124836401123552\\
245	0.0124835195086681\\
246	0.0124833967133588\\
247	0.0124832716851716\\
248	0.0124831443820525\\
249	0.0124830147611318\\
250	0.0124828827787077\\
251	0.0124827483902288\\
252	0.0124826115502786\\
253	0.0124824722125582\\
254	0.0124823303298655\\
255	0.0124821858540693\\
256	0.0124820387360955\\
257	0.0124818889259074\\
258	0.012481736372485\\
259	0.0124815810238052\\
260	0.0124814228268198\\
261	0.0124812617274338\\
262	0.012481097670484\\
263	0.0124809305997161\\
264	0.012480760457762\\
265	0.0124805871861166\\
266	0.0124804107251132\\
267	0.0124802310139002\\
268	0.0124800479904154\\
269	0.0124798615913607\\
270	0.0124796717521767\\
271	0.0124794784070152\\
272	0.0124792814887133\\
273	0.0124790809287654\\
274	0.0124788766572966\\
275	0.0124786686030352\\
276	0.0124784566932851\\
277	0.0124782408538995\\
278	0.0124780210092316\\
279	0.0124777970821148\\
280	0.012477568993832\\
281	0.0124773366640827\\
282	0.0124771000109498\\
283	0.0124768589508636\\
284	0.0124766133985649\\
285	0.0124763632670672\\
286	0.012476108467622\\
287	0.0124758489096868\\
288	0.0124755845008812\\
289	0.012475315146946\\
290	0.0124750407517022\\
291	0.0124747612170079\\
292	0.012474476442715\\
293	0.0124741863266231\\
294	0.0124738907644337\\
295	0.0124735896497025\\
296	0.0124732828737903\\
297	0.0124729703258132\\
298	0.0124726518925897\\
299	0.0124723274585832\\
300	0.0124719969058491\\
301	0.0124716601139765\\
302	0.0124713169600303\\
303	0.0124709673184884\\
304	0.0124706110611779\\
305	0.0124702480572067\\
306	0.0124698781728918\\
307	0.0124695012716871\\
308	0.0124691172141201\\
309	0.0124687258577437\\
310	0.0124683270571089\\
311	0.0124679206637006\\
312	0.012467506525651\\
313	0.0124670844877404\\
314	0.0124666543913352\\
315	0.0124662160743862\\
316	0.0124657693712707\\
317	0.012465314112685\\
318	0.0124648501255666\\
319	0.0124643772329907\\
320	0.0124638952540501\\
321	0.012463404003713\\
322	0.0124629032927018\\
323	0.0124623929274717\\
324	0.0124618727100949\\
325	0.0124613424380998\\
326	0.0124608019043495\\
327	0.0124602508969206\\
328	0.0124596891989781\\
329	0.0124591165886197\\
330	0.0124585328386973\\
331	0.0124579377166939\\
332	0.012457330984562\\
333	0.0124567123985447\\
334	0.0124560817089774\\
335	0.0124554386600965\\
336	0.0124547829899094\\
337	0.0124541144300366\\
338	0.0124534327054414\\
339	0.0124527375341964\\
340	0.0124520286272237\\
341	0.0124513056880012\\
342	0.0124505684122523\\
343	0.0124498164877157\\
344	0.0124490495942073\\
345	0.0124482674039933\\
346	0.0124474695814075\\
347	0.0124466557835481\\
348	0.0124458256556707\\
349	0.0124449788327121\\
350	0.0124441149385019\\
351	0.0124432335872951\\
352	0.0124423343831529\\
353	0.0124414169192525\\
354	0.0124404807769839\\
355	0.0124395255254527\\
356	0.0124385507210872\\
357	0.0124375559068055\\
358	0.0124365406111347\\
359	0.0124355043482407\\
360	0.0124344466181821\\
361	0.0124333669053959\\
362	0.0124322646779233\\
363	0.0124311393864985\\
364	0.01242999046346\\
365	0.0124288173216336\\
366	0.0124276193540648\\
367	0.0124263959366591\\
368	0.012425146431859\\
369	0.0124238701729671\\
370	0.0124225664698571\\
371	0.0124212346050074\\
372	0.0124198738264399\\
373	0.012418483351476\\
374	0.0124170624278787\\
375	0.0124156102335757\\
376	0.0124141259156042\\
377	0.0124126085844373\\
378	0.0124110573064146\\
379	0.0124094711190097\\
380	0.012407849027586\\
381	0.0124061899971926\\
382	0.0124044929490002\\
383	0.0124027567549661\\
384	0.0124009802299257\\
385	0.0123991621244678\\
386	0.0123973011425606\\
387	0.0123953960353144\\
388	0.0123934454878276\\
389	0.0123914481037175\\
390	0.0123894024190804\\
391	0.0123873068038163\\
392	0.0123851594726926\\
393	0.012382958627836\\
394	0.0123807022797639\\
395	0.0123783881948406\\
396	0.0123760146205097\\
397	0.0123735793781997\\
398	0.0123710800760692\\
399	0.0123685141895256\\
400	0.0123658790321492\\
401	0.0123631716916794\\
402	0.0123603889115399\\
403	0.0123575270926111\\
404	0.0123545836608289\\
405	0.0123515564514014\\
406	0.0123484428830265\\
407	0.0123452410479942\\
408	0.012341945132935\\
409	0.0123385514087179\\
410	0.0123350559571012\\
411	0.0123314522659224\\
412	0.0123277355502006\\
413	0.0123238999417505\\
414	0.0123199372047438\\
415	0.0123158466433408\\
416	0.01231162464631\\
417	0.0123072605199656\\
418	0.0123027388411858\\
419	0.0122980347031105\\
420	0.0122931599545735\\
421	0.0122881184015738\\
422	0.0122829323480974\\
423	0.0122775474812349\\
424	0.0122719426269842\\
425	0.0122660775374262\\
426	0.0122599279472396\\
427	0.0122534663204804\\
428	0.0122466610753029\\
429	0.0122394748244065\\
430	0.0122318573162748\\
431	0.0122193934971236\\
432	0.01219907003095\\
433	0.0121783223572718\\
434	0.012157211188889\\
435	0.0121357821290657\\
436	0.0121138883848964\\
437	0.0120914713783048\\
438	0.0120684947526461\\
439	0.0120449175919429\\
440	0.0120206936737558\\
441	0.0119957705816139\\
442	0.0119700886624083\\
443	0.0119435798577895\\
444	0.0119161666529473\\
445	0.0118877591033575\\
446	0.0118493950643996\\
447	0.0118029847814933\\
448	0.0117560370758049\\
449	0.0117085550393122\\
450	0.0116605438013708\\
451	0.0116120111665148\\
452	0.0115629683312149\\
453	0.0115134307519362\\
454	0.0114634195782015\\
455	0.0114129623471881\\
456	0.0113620897836564\\
457	0.0113108397072014\\
458	0.0112592610890864\\
459	0.0112074202653949\\
460	0.0111807966783627\\
461	0.0111631539198066\\
462	0.0111457754608839\\
463	0.0111287137085472\\
464	0.0111120272758899\\
465	0.0110957818296251\\
466	0.0110800511382366\\
467	0.0110649182204461\\
468	0.0110504765446275\\
469	0.0110368314774263\\
470	0.0110241020251619\\
471	0.0110124228876155\\
472	0.0110019468752839\\
473	0.0109919135989446\\
474	0.0109819581271541\\
475	0.0109720831320626\\
476	0.0109622896634912\\
477	0.0109525766955367\\
478	0.0109429405795266\\
479	0.0109333743798174\\
480	0.0109238670689868\\
481	0.0109144025526902\\
482	0.0109049584893634\\
483	0.0108955048627365\\
484	0.010886002256269\\
485	0.0108764113287698\\
486	0.0108667269013731\\
487	0.0108569432753038\\
488	0.0108470542364019\\
489	0.0108370530777549\\
490	0.0108269326468195\\
491	0.010816685425205\\
492	0.0108063036515764\\
493	0.010795779500978\\
494	0.0107851053375398\\
495	0.0107742740619926\\
496	0.0107632795814143\\
497	0.0107521170989154\\
498	0.0107407816914889\\
499	0.0107292683223882\\
500	0.0107175718550653\\
501	0.0107056870681011\\
502	0.0106936086701972\\
503	0.0106813313137836\\
504	0.0106688496051062\\
505	0.0106561581077001\\
506	0.0106432513348758\\
507	0.0106301237251207\\
508	0.0106167695920173\\
509	0.0106031830772331\\
510	0.0105893581450665\\
511	0.0105752885762286\\
512	0.0105609679607158\\
513	0.0105463896896371\\
514	0.0105315469458894\\
515	0.0105164326936348\\
516	0.0105010396666399\\
517	0.0104853603557148\\
518	0.0104693869957639\\
519	0.0104531115533758\\
520	0.0104365257164948\\
521	0.0104196208844651\\
522	0.010402388157826\\
523	0.0103848183278934\\
524	0.0103669018661714\\
525	0.0103486289136151\\
526	0.0103299892698269\\
527	0.0103109723822915\\
528	0.0102915673357285\\
529	0.0102717628423874\\
530	0.0102515472330569\\
531	0.0102309084467258\\
532	0.0102098340203716\\
533	0.0101883110790446\\
534	0.0101663262511507\\
535	0.010143865743747\\
536	0.0101209153372074\\
537	0.0100974583940287\\
538	0.0100734798563586\\
539	0.0100489691822816\\
540	0.0100239098760763\\
541	0.00999828775940519\\
542	0.00997208773349542\\
543	0.00994528797321891\\
544	0.00991786997241802\\
545	0.00988981093756304\\
546	0.00986106789000255\\
547	0.00983163917964649\\
548	0.00980152321313676\\
549	0.00977069207681917\\
550	0.00973910583825725\\
551	0.00970678396253116\\
552	0.00967371302342107\\
553	0.00963987135698235\\
554	0.0096052821126277\\
555	0.00956983538884356\\
556	0.00953350247176351\\
557	0.00949615392586526\\
558	0.00945792993494292\\
559	0.00941872899557266\\
560	0.0093785649396831\\
561	0.00933753561474518\\
562	0.00929582166875998\\
563	0.00925314354820956\\
564	0.00920935743036538\\
565	0.00916438007450519\\
566	0.00911786024556143\\
567	0.00907002470464621\\
568	0.00902064875706754\\
569	0.00897037472747782\\
570	0.00892037973865424\\
571	0.00883136476694636\\
572	0.00865022597339645\\
573	0.00828928204857107\\
574	0.00793224760636447\\
575	0.00783391425016302\\
576	0.00775665882326094\\
577	0.00768043829538653\\
578	0.00760311425029811\\
579	0.00752432601413681\\
580	0.00744395532478318\\
581	0.00736194208712794\\
582	0.00727821978182829\\
583	0.00719272277813591\\
584	0.00710538092623947\\
585	0.00701611933500737\\
586	0.00692485581252974\\
587	0.00683149676218332\\
588	0.00673592587771182\\
589	0.00663797520465071\\
590	0.00653734834917363\\
591	0.00643341581933105\\
592	0.00632466858477865\\
593	0.00620725772627071\\
594	0.0060710901180166\\
595	0.00588956027306328\\
596	0.00559184182825266\\
597	0.00498881296807123\\
598	0.00357511483354343\\
599	0\\
600	0\\
};
\addplot [color=mycolor8,solid,forget plot]
  table[row sep=crcr]{%
1	0.0124911041531259\\
2	0.012491102138032\\
3	0.012491100094959\\
4	0.0124910980235295\\
5	0.0124910959233599\\
6	0.0124910937940607\\
7	0.0124910916352357\\
8	0.0124910894464824\\
9	0.0124910872273918\\
10	0.0124910849775477\\
11	0.0124910826965273\\
12	0.0124910803839006\\
13	0.0124910780392302\\
14	0.0124910756620713\\
15	0.0124910732519718\\
16	0.0124910708084714\\
17	0.0124910683311021\\
18	0.0124910658193876\\
19	0.0124910632728435\\
20	0.0124910606909768\\
21	0.0124910580732857\\
22	0.0124910554192597\\
23	0.012491052728379\\
24	0.0124910500001146\\
25	0.0124910472339282\\
26	0.0124910444292714\\
27	0.0124910415855863\\
28	0.0124910387023045\\
29	0.0124910357788474\\
30	0.0124910328146258\\
31	0.0124910298090396\\
32	0.0124910267614777\\
33	0.0124910236713177\\
34	0.0124910205379255\\
35	0.0124910173606555\\
36	0.0124910141388498\\
37	0.0124910108718382\\
38	0.012491007558938\\
39	0.0124910041994536\\
40	0.0124910007926764\\
41	0.0124909973378842\\
42	0.0124909938343414\\
43	0.0124909902812982\\
44	0.0124909866779906\\
45	0.0124909830236402\\
46	0.0124909793174537\\
47	0.0124909755586225\\
48	0.0124909717463228\\
49	0.0124909678797148\\
50	0.0124909639579429\\
51	0.0124909599801347\\
52	0.0124909559454014\\
53	0.0124909518528371\\
54	0.0124909477015184\\
55	0.0124909434905042\\
56	0.0124909392188353\\
57	0.0124909348855341\\
58	0.0124909304896041\\
59	0.0124909260300299\\
60	0.0124909215057763\\
61	0.0124909169157883\\
62	0.0124909122589908\\
63	0.0124909075342878\\
64	0.0124909027405624\\
65	0.0124908978766762\\
66	0.0124908929414691\\
67	0.0124908879337587\\
68	0.01249088285234\\
69	0.0124908776959849\\
70	0.0124908724634418\\
71	0.0124908671534353\\
72	0.0124908617646657\\
73	0.0124908562958084\\
74	0.0124908507455136\\
75	0.0124908451124061\\
76	0.0124908393950843\\
77	0.0124908335921201\\
78	0.0124908277020584\\
79	0.0124908217234166\\
80	0.012490815654684\\
81	0.0124908094943214\\
82	0.0124908032407609\\
83	0.0124907968924046\\
84	0.012490790447625\\
85	0.0124907839047639\\
86	0.0124907772621319\\
87	0.0124907705180083\\
88	0.01249076367064\\
89	0.0124907567182412\\
90	0.012490749658993\\
91	0.0124907424910425\\
92	0.0124907352125023\\
93	0.0124907278214501\\
94	0.0124907203159279\\
95	0.0124907126939414\\
96	0.0124907049534594\\
97	0.0124906970924132\\
98	0.0124906891086959\\
99	0.0124906810001617\\
100	0.0124906727646252\\
101	0.0124906643998609\\
102	0.0124906559036023\\
103	0.0124906472735411\\
104	0.0124906385073266\\
105	0.0124906296025651\\
106	0.0124906205568187\\
107	0.012490611367605\\
108	0.0124906020323956\\
109	0.0124905925486162\\
110	0.0124905829136448\\
111	0.0124905731248115\\
112	0.0124905631793972\\
113	0.0124905530746329\\
114	0.0124905428076984\\
115	0.0124905323757219\\
116	0.0124905217757781\\
117	0.0124905110048882\\
118	0.0124905000600176\\
119	0.0124904889380758\\
120	0.0124904776359146\\
121	0.012490466150327\\
122	0.0124904544780457\\
123	0.0124904426157421\\
124	0.0124904305600244\\
125	0.0124904183074365\\
126	0.012490405854456\\
127	0.0124903931974924\\
128	0.0124903803328859\\
129	0.0124903672569045\\
130	0.0124903539657427\\
131	0.0124903404555188\\
132	0.0124903267222724\\
133	0.0124903127619621\\
134	0.0124902985704625\\
135	0.0124902841435613\\
136	0.0124902694769555\\
137	0.0124902545662488\\
138	0.0124902394069469\\
139	0.0124902239944543\\
140	0.0124902083240696\\
141	0.0124901923909816\\
142	0.0124901761902656\\
143	0.0124901597168796\\
144	0.0124901429656638\\
145	0.0124901259313421\\
146	0.0124901086085269\\
147	0.0124900909917171\\
148	0.0124900730752285\\
149	0.0124900548530119\\
150	0.0124900363194713\\
151	0.0124900174689059\\
152	0.012489998295508\\
153	0.0124899787933609\\
154	0.0124899589564368\\
155	0.012489938778595\\
156	0.0124899182535792\\
157	0.0124898973750159\\
158	0.0124898761364117\\
159	0.0124898545311512\\
160	0.0124898325524947\\
161	0.0124898101935757\\
162	0.0124897874473986\\
163	0.0124897643068363\\
164	0.0124897407646275\\
165	0.0124897168133744\\
166	0.0124896924455396\\
167	0.0124896676534442\\
168	0.0124896424292644\\
169	0.0124896167650292\\
170	0.0124895906526174\\
171	0.0124895640837547\\
172	0.0124895370500111\\
173	0.0124895095427974\\
174	0.0124894815533627\\
175	0.0124894530727911\\
176	0.0124894240919986\\
177	0.0124893946017298\\
178	0.0124893645925551\\
179	0.0124893340548666\\
180	0.0124893029788755\\
181	0.0124892713546085\\
182	0.0124892391719037\\
183	0.0124892064204081\\
184	0.012489173089573\\
185	0.012489139168651\\
186	0.0124891046466917\\
187	0.0124890695125386\\
188	0.0124890337548242\\
189	0.0124889973619672\\
190	0.0124889603221673\\
191	0.0124889226234021\\
192	0.0124888842534222\\
193	0.0124888451997473\\
194	0.0124888054496618\\
195	0.0124887649902103\\
196	0.0124887238081929\\
197	0.0124886818901612\\
198	0.0124886392224129\\
199	0.0124885957909875\\
200	0.0124885515816612\\
201	0.0124885065799421\\
202	0.0124884607710651\\
203	0.0124884141399864\\
204	0.0124883666713791\\
205	0.0124883183496266\\
206	0.0124882691588184\\
207	0.0124882190827435\\
208	0.0124881681048854\\
209	0.0124881162084159\\
210	0.0124880633761894\\
211	0.0124880095907364\\
212	0.0124879548342582\\
213	0.0124878990886197\\
214	0.0124878423353433\\
215	0.0124877845556027\\
216	0.0124877257302155\\
217	0.012487665839637\\
218	0.0124876048639526\\
219	0.0124875427828714\\
220	0.0124874795757182\\
221	0.0124874152214263\\
222	0.0124873496985303\\
223	0.0124872829851575\\
224	0.0124872150590207\\
225	0.01248714589741\\
226	0.0124870754771841\\
227	0.0124870037747623\\
228	0.0124869307661155\\
229	0.0124868564267578\\
230	0.012486780731737\\
231	0.0124867036556259\\
232	0.0124866251725127\\
233	0.0124865452559911\\
234	0.0124864638791511\\
235	0.0124863810145688\\
236	0.0124862966342958\\
237	0.0124862107098496\\
238	0.0124861232122022\\
239	0.0124860341117697\\
240	0.0124859433784011\\
241	0.0124858509813671\\
242	0.0124857568893484\\
243	0.0124856610704243\\
244	0.0124855634920603\\
245	0.0124854641210963\\
246	0.0124853629237338\\
247	0.0124852598655234\\
248	0.0124851549113518\\
249	0.0124850480254286\\
250	0.0124849391712727\\
251	0.0124848283116992\\
252	0.0124847154088048\\
253	0.012484600423954\\
254	0.0124844833177631\\
255	0.0124843640500866\\
256	0.0124842425800017\\
257	0.0124841188657925\\
258	0.0124839928649349\\
259	0.0124838645340795\\
260	0.0124837338290358\\
261	0.0124836007047547\\
262	0.0124834651153118\\
263	0.0124833270138889\\
264	0.0124831863527567\\
265	0.0124830430832552\\
266	0.0124828971557754\\
267	0.0124827485197389\\
268	0.0124825971235783\\
269	0.0124824429147154\\
270	0.0124822858395403\\
271	0.0124821258433883\\
272	0.0124819628705165\\
273	0.0124817968640789\\
274	0.0124816277660993\\
275	0.0124814555174431\\
276	0.0124812800577948\\
277	0.0124811013256614\\
278	0.0124809192584328\\
279	0.0124807337921161\\
280	0.012480544861438\\
281	0.0124803523998258\\
282	0.0124801563393797\\
283	0.0124799566108442\\
284	0.0124797531435784\\
285	0.0124795458655267\\
286	0.0124793347031885\\
287	0.0124791195815857\\
288	0.0124789004242308\\
289	0.0124786771530937\\
290	0.0124784496885671\\
291	0.0124782179494323\\
292	0.0124779818528227\\
293	0.0124777413141876\\
294	0.0124774962472545\\
295	0.0124772465639906\\
296	0.0124769921745635\\
297	0.0124767329873011\\
298	0.0124764689086499\\
299	0.0124761998431337\\
300	0.0124759256933106\\
301	0.0124756463597299\\
302	0.0124753617408879\\
303	0.0124750717331831\\
304	0.012474776230871\\
305	0.0124744751260184\\
306	0.0124741683084588\\
307	0.0124738556657481\\
308	0.0124735370831249\\
309	0.0124732124434715\\
310	0.0124728816272755\\
311	0.0124725445125801\\
312	0.0124722009749567\\
313	0.0124718508874366\\
314	0.0124714941203138\\
315	0.0124711305407571\\
316	0.0124707600136177\\
317	0.0124703824011359\\
318	0.0124699975626429\\
319	0.0124696053544896\\
320	0.0124692056299724\\
321	0.0124687982392594\\
322	0.0124683830293228\\
323	0.0124679598438589\\
324	0.0124675285232051\\
325	0.0124670889042563\\
326	0.0124666408203799\\
327	0.0124661841013269\\
328	0.0124657185731383\\
329	0.0124652440580472\\
330	0.0124647603743824\\
331	0.0124642673364666\\
332	0.0124637647545081\\
333	0.0124632524344899\\
334	0.0124627301780566\\
335	0.0124621977824021\\
336	0.0124616550401511\\
337	0.012461101739226\\
338	0.0124605376627145\\
339	0.0124599625887298\\
340	0.0124593762902648\\
341	0.0124587785350435\\
342	0.012458169085379\\
343	0.0124575476980541\\
344	0.0124569141242101\\
345	0.0124562681091739\\
346	0.0124556093923005\\
347	0.0124549377064665\\
348	0.0124542527780373\\
349	0.0124535543266637\\
350	0.0124528420652271\\
351	0.0124521156996018\\
352	0.0124513749284031\\
353	0.0124506194427104\\
354	0.0124498489258216\\
355	0.0124490630530038\\
356	0.0124482614912072\\
357	0.0124474438987931\\
358	0.0124466099252914\\
359	0.0124457592111046\\
360	0.0124448913871545\\
361	0.0124440060745378\\
362	0.0124431028841549\\
363	0.012442181416315\\
364	0.0124412412603495\\
365	0.0124402819943223\\
366	0.012439303184906\\
367	0.0124383043870957\\
368	0.0124372851426542\\
369	0.0124362449801065\\
370	0.0124351834139574\\
371	0.0124340999439149\\
372	0.0124329940554009\\
373	0.0124318652225133\\
374	0.0124307128998605\\
375	0.0124295365251154\\
376	0.0124283355178597\\
377	0.0124271092785167\\
378	0.0124258571889347\\
379	0.012424578611565\\
380	0.0124232728879237\\
381	0.0124219393373652\\
382	0.012420577255762\\
383	0.0124191859140688\\
384	0.01241776455739\\
385	0.0124163124064769\\
386	0.0124148286623609\\
387	0.0124133124961171\\
388	0.0124117630465355\\
389	0.0124101794189236\\
390	0.0124085606771264\\
391	0.0124069058450466\\
392	0.0124052139132507\\
393	0.0124034838251913\\
394	0.0124017144798726\\
395	0.0123999047758579\\
396	0.0123980535423976\\
397	0.0123961595527441\\
398	0.0123942215274171\\
399	0.0123922381284047\\
400	0.0123902079507506\\
401	0.0123881295144388\\
402	0.0123860012789051\\
403	0.0123838217331691\\
404	0.0123815893319603\\
405	0.0123793024339158\\
406	0.0123769593353914\\
407	0.0123745579848499\\
408	0.012372096395045\\
409	0.0123695724308074\\
410	0.0123669836613538\\
411	0.0123643276713864\\
412	0.0123616018110315\\
413	0.0123588032073114\\
414	0.0123559293759664\\
415	0.012352977516097\\
416	0.0123499442104931\\
417	0.0123468255804046\\
418	0.0123436174626817\\
419	0.0123403182842309\\
420	0.0123369258799185\\
421	0.0123334384239431\\
422	0.0123298486515786\\
423	0.0123261514279178\\
424	0.0123223403454993\\
425	0.0123184103549643\\
426	0.0123143561538332\\
427	0.0123101721200259\\
428	0.0123058520719514\\
429	0.0123013884515147\\
430	0.0122967702770947\\
431	0.0122919872839269\\
432	0.0122870262947262\\
433	0.012281893121785\\
434	0.0122766172968557\\
435	0.0122712086989549\\
436	0.012265596747146\\
437	0.0122597519638662\\
438	0.0122536544500453\\
439	0.0122472818652323\\
440	0.012240609006385\\
441	0.0122336071818781\\
442	0.0122262429145973\\
443	0.0122184736049118\\
444	0.0122102262152079\\
445	0.0122014694010157\\
446	0.0121848916675161\\
447	0.0121619150759289\\
448	0.0121384537630224\\
449	0.0121144778828846\\
450	0.012089954056728\\
451	0.0120648449151303\\
452	0.0120391085326023\\
453	0.0120126977550609\\
454	0.0119855595646001\\
455	0.0119576338161843\\
456	0.0119288499825325\\
457	0.0118991267200434\\
458	0.0118683709474048\\
459	0.0118364772457061\\
460	0.0117899293887976\\
461	0.0117379330702364\\
462	0.0116853363928404\\
463	0.0116321444789565\\
464	0.0115783659745756\\
465	0.0115240119462968\\
466	0.0114690944420999\\
467	0.0114136296117527\\
468	0.0113576430991244\\
469	0.011301169870704\\
470	0.0112442545989068\\
471	0.011186953692074\\
472	0.0111293383178927\\
473	0.011097764439338\\
474	0.011078069400534\\
475	0.0110586702906042\\
476	0.0110396284626611\\
477	0.0110210129708733\\
478	0.0110029014976747\\
479	0.0109853815511372\\
480	0.0109685518350611\\
481	0.0109525238704125\\
482	0.0109374238939215\\
483	0.0109233951058549\\
484	0.0109106003704035\\
485	0.0108989145299648\\
486	0.0108872964848506\\
487	0.010875749312009\\
488	0.0108642743277868\\
489	0.0108528705813431\\
490	0.0108415342366413\\
491	0.0108302578195344\\
492	0.0108190293011009\\
493	0.0108078309834536\\
494	0.0107966381456029\\
495	0.0107854174027605\\
496	0.0107741247138657\\
497	0.0107627118024001\\
498	0.0107511720771215\\
499	0.0107394983103662\\
500	0.0107276826416192\\
501	0.0107157166032271\\
502	0.010703591176167\\
503	0.0106912968861572\\
504	0.0106788239532819\\
505	0.0106661625121368\\
506	0.0106533029241167\\
507	0.0106402362093446\\
508	0.0106269546333935\\
509	0.0106134514714393\\
510	0.0105997197988757\\
511	0.010585752504419\\
512	0.0105715423050824\\
513	0.0105570817622631\\
514	0.0105423632976747\\
515	0.010527379207154\\
516	0.0105121216693857\\
517	0.0104965827452528\\
518	0.0104807543617196\\
519	0.0104646282717054\\
520	0.0104481959781213\\
521	0.0104314487084238\\
522	0.0104143774066446\\
523	0.0103969727245863\\
524	0.0103792250120199\\
525	0.0103611243057383\\
526	0.0103426603173705\\
527	0.0103238224199568\\
528	0.0103045996334536\\
529	0.0102849806095972\\
530	0.0102649536169924\\
531	0.010244506528012\\
532	0.0102236268082363\\
533	0.01020230150594\\
534	0.0101805172436589\\
535	0.0101582602076974\\
536	0.0101355161377006\\
537	0.0101122703165035\\
538	0.0100885075307361\\
539	0.0100642120485176\\
540	0.01003936766622\\
541	0.0100139576998519\\
542	0.00998797089944421\\
543	0.00996138846326938\\
544	0.00993419115715787\\
545	0.00990635887018058\\
546	0.00987786587109372\\
547	0.00984868848857995\\
548	0.00981879802970254\\
549	0.00978818044588218\\
550	0.00975679906793436\\
551	0.00972466302807263\\
552	0.00969176439712891\\
553	0.00965808014673742\\
554	0.00962355671545677\\
555	0.00958821951216768\\
556	0.00955204355678784\\
557	0.00951500838590981\\
558	0.00947712588481273\\
559	0.00943823158357318\\
560	0.00939832134866453\\
561	0.00935745843282328\\
562	0.00931549302033459\\
563	0.00927253839052881\\
564	0.00922861881642663\\
565	0.00918392533577964\\
566	0.00913812936805613\\
567	0.00909110508593811\\
568	0.00904243698427758\\
569	0.00899247235927465\\
570	0.00894090723682039\\
571	0.00888831520764092\\
572	0.00883560063672551\\
573	0.00868299667447052\\
574	0.0084611546732484\\
575	0.00806442919934086\\
576	0.00778388453850264\\
577	0.00768497233769221\\
578	0.00760406498573608\\
579	0.00752458055450901\\
580	0.00744407604424876\\
581	0.00736198802737892\\
582	0.00727824292195357\\
583	0.00719273371534652\\
584	0.00710538683074545\\
585	0.00701612214882769\\
586	0.00692485725962403\\
587	0.00683149725950674\\
588	0.00673592596358478\\
589	0.00663797520465071\\
590	0.00653734834917363\\
591	0.00643341581933105\\
592	0.00632466858477866\\
593	0.00620725772627071\\
594	0.00607109011801661\\
595	0.00588956027306329\\
596	0.00559184182825266\\
597	0.00498881296807123\\
598	0.00357511483354343\\
599	0\\
600	0\\
};
\addplot [color=blue!25!mycolor7,solid,forget plot]
  table[row sep=crcr]{%
1	0.0124934896979223\\
2	0.0124934880957596\\
3	0.0124934864703468\\
4	0.012493484821332\\
5	0.012493483148357\\
6	0.0124934814510573\\
7	0.0124934797290618\\
8	0.0124934779819929\\
9	0.0124934762094662\\
10	0.0124934744110903\\
11	0.0124934725864671\\
12	0.012493470735191\\
13	0.0124934688568492\\
14	0.0124934669510214\\
15	0.0124934650172798\\
16	0.0124934630551889\\
17	0.0124934610643049\\
18	0.0124934590441764\\
19	0.0124934569943435\\
20	0.012493454914338\\
21	0.0124934528036832\\
22	0.0124934506618934\\
23	0.0124934484884744\\
24	0.0124934462829227\\
25	0.0124934440447255\\
26	0.0124934417733608\\
27	0.0124934394682967\\
28	0.0124934371289919\\
29	0.0124934347548947\\
30	0.0124934323454437\\
31	0.0124934299000666\\
32	0.012493427418181\\
33	0.0124934248991935\\
34	0.0124934223424998\\
35	0.0124934197474845\\
36	0.0124934171135208\\
37	0.0124934144399702\\
38	0.0124934117261825\\
39	0.0124934089714955\\
40	0.0124934061752348\\
41	0.0124934033367134\\
42	0.0124934004552317\\
43	0.0124933975300772\\
44	0.012493394560524\\
45	0.0124933915458333\\
46	0.0124933884852521\\
47	0.0124933853780139\\
48	0.012493382223338\\
49	0.0124933790204293\\
50	0.0124933757684779\\
51	0.0124933724666592\\
52	0.0124933691141334\\
53	0.0124933657100453\\
54	0.0124933622535239\\
55	0.0124933587436824\\
56	0.0124933551796176\\
57	0.0124933515604099\\
58	0.0124933478851228\\
59	0.0124933441528028\\
60	0.0124933403624789\\
61	0.0124933365131626\\
62	0.0124933326038472\\
63	0.0124933286335079\\
64	0.0124933246011012\\
65	0.0124933205055648\\
66	0.012493316345817\\
67	0.0124933121207569\\
68	0.0124933078292635\\
69	0.0124933034701956\\
70	0.0124932990423916\\
71	0.0124932945446691\\
72	0.0124932899758244\\
73	0.0124932853346323\\
74	0.012493280619846\\
75	0.0124932758301961\\
76	0.0124932709643908\\
77	0.0124932660211155\\
78	0.0124932609990322\\
79	0.0124932558967793\\
80	0.0124932507129712\\
81	0.012493245446198\\
82	0.012493240095025\\
83	0.0124932346579924\\
84	0.0124932291336149\\
85	0.0124932235203815\\
86	0.0124932178167547\\
87	0.0124932120211706\\
88	0.0124932061320383\\
89	0.0124932001477395\\
90	0.012493194066628\\
91	0.0124931878870296\\
92	0.0124931816072417\\
93	0.0124931752255326\\
94	0.0124931687401414\\
95	0.0124931621492777\\
96	0.0124931554511208\\
97	0.01249314864382\\
98	0.0124931417254936\\
99	0.012493134694229\\
100	0.012493127548082\\
101	0.0124931202850768\\
102	0.0124931129032055\\
103	0.0124931054004278\\
104	0.0124930977746709\\
105	0.0124930900238288\\
106	0.0124930821457626\\
107	0.0124930741382997\\
108	0.0124930659992342\\
109	0.0124930577263262\\
110	0.0124930493173018\\
111	0.0124930407698534\\
112	0.0124930320816387\\
113	0.0124930232502819\\
114	0.0124930142733726\\
115	0.0124930051484666\\
116	0.0124929958730855\\
117	0.0124929864447175\\
118	0.012492976860817\\
119	0.0124929671188054\\
120	0.0124929572160715\\
121	0.0124929471499716\\
122	0.0124929369178305\\
123	0.0124929265169421\\
124	0.01249291594457\\
125	0.0124929051979488\\
126	0.0124928942742848\\
127	0.012492883170757\\
128	0.0124928718845191\\
129	0.0124928604126997\\
130	0.012492848752405\\
131	0.0124928369007194\\
132	0.0124928248547077\\
133	0.0124928126114164\\
134	0.0124928001678754\\
135	0.0124927875210997\\
136	0.0124927746680904\\
137	0.0124927616058361\\
138	0.0124927483313133\\
139	0.0124927348414866\\
140	0.0124927211333086\\
141	0.0124927072037204\\
142	0.0124926930496555\\
143	0.0124926786680577\\
144	0.0124926640559419\\
145	0.0124926492105886\\
146	0.0124926341301557\\
147	0.0124926188156553\\
148	0.0124926032778825\\
149	0.0124925875454267\\
150	0.0124925715367068\\
151	0.0124925552466545\\
152	0.0124925386701069\\
153	0.0124925218018051\\
154	0.0124925046363921\\
155	0.0124924871684115\\
156	0.0124924693923051\\
157	0.012492451302411\\
158	0.0124924328929624\\
159	0.0124924141580844\\
160	0.0124923950917933\\
161	0.0124923756879937\\
162	0.0124923559404764\\
163	0.012492335842917\\
164	0.012492315388873\\
165	0.0124922945717819\\
166	0.0124922733849591\\
167	0.0124922518215951\\
168	0.012492229874754\\
169	0.0124922075373703\\
170	0.0124921848022471\\
171	0.0124921616620532\\
172	0.0124921381093209\\
173	0.0124921141364434\\
174	0.0124920897356721\\
175	0.0124920648991141\\
176	0.0124920396187294\\
177	0.012492013886328\\
178	0.0124919876935677\\
179	0.0124919610319504\\
180	0.0124919338928201\\
181	0.0124919062673592\\
182	0.0124918781465859\\
183	0.0124918495213511\\
184	0.0124918203823353\\
185	0.012491790720045\\
186	0.0124917605248103\\
187	0.0124917297867809\\
188	0.0124916984959227\\
189	0.0124916666420151\\
190	0.0124916342146468\\
191	0.0124916012032126\\
192	0.0124915675969094\\
193	0.0124915333847333\\
194	0.0124914985554749\\
195	0.012491463097716\\
196	0.0124914269998258\\
197	0.0124913902499564\\
198	0.0124913528360393\\
199	0.0124913147457809\\
200	0.0124912759666586\\
201	0.0124912364859161\\
202	0.0124911962905594\\
203	0.0124911553673521\\
204	0.012491113702811\\
205	0.0124910712832015\\
206	0.0124910280945325\\
207	0.0124909841225523\\
208	0.0124909393527428\\
209	0.0124908937703154\\
210	0.0124908473602051\\
211	0.012490800107066\\
212	0.0124907519952654\\
213	0.0124907030088788\\
214	0.0124906531316842\\
215	0.0124906023471566\\
216	0.0124905506384623\\
217	0.0124904979884528\\
218	0.0124904443796593\\
219	0.0124903897942864\\
220	0.0124903342142056\\
221	0.0124902776209498\\
222	0.0124902199957062\\
223	0.0124901613193101\\
224	0.0124901015722381\\
225	0.0124900407346015\\
226	0.0124899787861392\\
227	0.0124899157062108\\
228	0.0124898514737893\\
229	0.0124897860674542\\
230	0.0124897194653835\\
231	0.0124896516453464\\
232	0.0124895825846959\\
233	0.0124895122603604\\
234	0.0124894406488363\\
235	0.0124893677261793\\
236	0.0124892934679966\\
237	0.0124892178494386\\
238	0.0124891408451896\\
239	0.0124890624294603\\
240	0.0124889825759777\\
241	0.0124889012579772\\
242	0.012488818448193\\
243	0.0124887341188488\\
244	0.0124886482416487\\
245	0.0124885607877676\\
246	0.0124884717278416\\
247	0.012488381031958\\
248	0.0124882886696458\\
249	0.0124881946098656\\
250	0.0124880988209994\\
251	0.0124880012708406\\
252	0.0124879019265838\\
253	0.0124878007548143\\
254	0.0124876977214977\\
255	0.01248759279197\\
256	0.0124874859309268\\
257	0.0124873771024131\\
258	0.0124872662698131\\
259	0.01248715339584\\
260	0.0124870384425261\\
261	0.0124869213712125\\
262	0.01248680214254\\
263	0.0124866807164395\\
264	0.0124865570521227\\
265	0.012486431108074\\
266	0.0124863028420416\\
267	0.0124861722110299\\
268	0.0124860391712924\\
269	0.0124859036783241\\
270	0.0124857656868552\\
271	0.012485625150844\\
272	0.0124854820234694\\
273	0.0124853362571207\\
274	0.0124851878033816\\
275	0.0124850366130013\\
276	0.0124848826358461\\
277	0.0124847258208445\\
278	0.0124845661161088\\
279	0.0124844034702233\\
280	0.0124842378286065\\
281	0.0124840691355065\\
282	0.012483897334114\\
283	0.0124837223665395\\
284	0.0124835441737902\\
285	0.0124833626957451\\
286	0.0124831778711305\\
287	0.0124829896374928\\
288	0.0124827979311713\\
289	0.0124826026872696\\
290	0.0124824038396243\\
291	0.0124822013207743\\
292	0.0124819950619258\\
293	0.0124817849929178\\
294	0.0124815710421833\\
295	0.0124813531367093\\
296	0.0124811312019941\\
297	0.0124809051620007\\
298	0.0124806749391079\\
299	0.012480440454057\\
300	0.0124802016258953\\
301	0.012479958371914\\
302	0.0124797106075826\\
303	0.0124794582464776\\
304	0.0124792012002055\\
305	0.0124789393783213\\
306	0.0124786726882417\\
307	0.0124784010351532\\
308	0.0124781243219197\\
309	0.0124778424489904\\
310	0.0124775553143208\\
311	0.0124772628133248\\
312	0.0124769648388866\\
313	0.0124766612814485\\
314	0.0124763520290608\\
315	0.0124760369665911\\
316	0.012475715971063\\
317	0.0124753889246491\\
318	0.0124750557108248\\
319	0.0124747162107271\\
320	0.0124743703031029\\
321	0.0124740178642565\\
322	0.0124736587679943\\
323	0.0124732928855684\\
324	0.012472920085619\\
325	0.0124725402341145\\
326	0.0124721531942905\\
327	0.0124717588265863\\
328	0.0124713569885796\\
329	0.0124709475349203\\
330	0.0124705303172609\\
331	0.0124701051841856\\
332	0.0124696719811369\\
333	0.0124692305503398\\
334	0.012468780730725\\
335	0.0124683223578482\\
336	0.0124678552638066\\
337	0.0124673792771525\\
338	0.0124668942228049\\
339	0.0124663999219567\\
340	0.0124658961919803\\
341	0.0124653828463309\\
342	0.0124648596944482\\
343	0.0124643265416551\\
344	0.0124637831890467\\
345	0.0124632294333734\\
346	0.012462665066898\\
347	0.0124620898772859\\
348	0.0124615036474782\\
349	0.0124609061555697\\
350	0.0124602971746649\\
351	0.0124596764727285\\
352	0.0124590438124252\\
353	0.0124583989509545\\
354	0.0124577416398742\\
355	0.0124570716249094\\
356	0.0124563886457649\\
357	0.0124556924360045\\
358	0.0124549827231162\\
359	0.0124542592282132\\
360	0.0124535216651536\\
361	0.0124527697408658\\
362	0.0124520031551274\\
363	0.0124512216003319\\
364	0.0124504247612395\\
365	0.0124496123147366\\
366	0.0124487839296874\\
367	0.0124479392665319\\
368	0.0124470779770451\\
369	0.0124461997040359\\
370	0.0124453040810644\\
371	0.0124443907322648\\
372	0.0124434592722166\\
373	0.0124425093050651\\
374	0.0124415404244795\\
375	0.0124405522133178\\
376	0.0124395442433062\\
377	0.0124385160746847\\
378	0.0124374672552204\\
379	0.0124363973184357\\
380	0.012435305786179\\
381	0.0124341921676227\\
382	0.0124330559576152\\
383	0.012431896636168\\
384	0.0124307136681108\\
385	0.0124295065027671\\
386	0.0124282745725388\\
387	0.0124270172920972\\
388	0.0124257340575738\\
389	0.0124244242453342\\
390	0.0124230872114575\\
391	0.0124217222912526\\
392	0.0124203287975411\\
393	0.0124189060204611\\
394	0.0124174532291352\\
395	0.0124159696654504\\
396	0.0124144545440628\\
397	0.0124129070514929\\
398	0.0124113263444193\\
399	0.0124097115478212\\
400	0.0124080617535538\\
401	0.0124063760212524\\
402	0.0124046533828143\\
403	0.012402892834977\\
404	0.012401093333085\\
405	0.0123992537896467\\
406	0.0123973730553565\\
407	0.0123954499460333\\
408	0.0123934832239011\\
409	0.0123914715905959\\
410	0.0123894137133371\\
411	0.0123873082072819\\
412	0.012385153627286\\
413	0.0123829484324725\\
414	0.0123806910363104\\
415	0.0123783797876818\\
416	0.0123760129475657\\
417	0.0123735887310139\\
418	0.0123711054781399\\
419	0.0123685614077457\\
420	0.0123659546179662\\
421	0.0123632827294819\\
422	0.0123605434145629\\
423	0.0123577341676305\\
424	0.0123548524919875\\
425	0.0123518957476068\\
426	0.0123488611260978\\
427	0.0123457456021872\\
428	0.0123425458399115\\
429	0.0123392580886468\\
430	0.0123358785207434\\
431	0.0123324033774937\\
432	0.0123288306324937\\
433	0.0123251594699614\\
434	0.0123213863320206\\
435	0.0123175019945348\\
436	0.0123135004459836\\
437	0.0123093763866789\\
438	0.0123051242227054\\
439	0.0123007380391701\\
440	0.0122962115337123\\
441	0.0122915378104163\\
442	0.0122867086988441\\
443	0.0122817126384959\\
444	0.0122765313924144\\
445	0.0122711730094806\\
446	0.0122656783141936\\
447	0.0122600460436909\\
448	0.0122542044383031\\
449	0.0122481370496886\\
450	0.0122418255387772\\
451	0.0122352493046356\\
452	0.0122283847413856\\
453	0.0122212028493017\\
454	0.012213657087294\\
455	0.0122057179546538\\
456	0.0121974177852449\\
457	0.0121887350120657\\
458	0.0121796026867837\\
459	0.012169910760579\\
460	0.0121484525807216\\
461	0.0121225439671586\\
462	0.0120960687551813\\
463	0.0120689901445763\\
464	0.0120412671721926\\
465	0.0120128532454812\\
466	0.0119836943681672\\
467	0.0119537292121411\\
468	0.0119228897545696\\
469	0.0118910991619444\\
470	0.0118582694796218\\
471	0.0118242995178252\\
472	0.011789072511394\\
473	0.0117386084559066\\
474	0.0116810405201921\\
475	0.0116228033167768\\
476	0.0115638986607994\\
477	0.0115043304356759\\
478	0.0114441107324253\\
479	0.0113832567636374\\
480	0.0113217927406389\\
481	0.0112597506258005\\
482	0.0111971719801715\\
483	0.0111341098150239\\
484	0.0110706301898503\\
485	0.0110154864063124\\
486	0.0109933315999276\\
487	0.0109714636791583\\
488	0.0109499493266452\\
489	0.0109288637691311\\
490	0.0109082919550504\\
491	0.0108883299229187\\
492	0.0108690864107806\\
493	0.0108506847274538\\
494	0.0108332649923816\\
495	0.0108169867104067\\
496	0.0108020320341635\\
497	0.0107883763847002\\
498	0.0107747702255624\\
499	0.0107612163754469\\
500	0.0107477155937402\\
501	0.0107342659735556\\
502	0.0107208622008048\\
503	0.0107074946475425\\
504	0.0106941482646422\\
505	0.0106808012267657\\
506	0.0106674232797821\\
507	0.010653973729537\\
508	0.0106403989941711\\
509	0.0106266558014479\\
510	0.0106127353669084\\
511	0.0105986281112579\\
512	0.0105843236646759\\
513	0.0105698108994153\\
514	0.0105550780014204\\
515	0.0105401125946322\\
516	0.0105249019360662\\
517	0.0105094332044547\\
518	0.0104936939115285\\
519	0.0104776724734513\\
520	0.0104613589903642\\
521	0.0104447437527462\\
522	0.0104278167420056\\
523	0.0104105676451535\\
524	0.0103929858717878\\
525	0.0103750605723023\\
526	0.0103567806554969\\
527	0.0103381348027425\\
528	0.0103191114744017\\
529	0.0102996989022278\\
530	0.0102798850587414\\
531	0.0102596575908799\\
532	0.0102390037430913\\
533	0.0102179103492875\\
534	0.0101963638242602\\
535	0.0101743501544096\\
536	0.010151854887576\\
537	0.0101288631217804\\
538	0.0101053594935276\\
539	0.0100813281652646\\
540	0.0100567528104512\\
541	0.0100316165983594\\
542	0.0100059025314179\\
543	0.00997960737655468\\
544	0.00995269842410414\\
545	0.00992515671607556\\
546	0.00989696291884511\\
547	0.00986809291814794\\
548	0.00983852351201753\\
549	0.00980822536473573\\
550	0.00977717950896099\\
551	0.00974536590116786\\
552	0.00971276030523118\\
553	0.00967934025184406\\
554	0.00964507332415307\\
555	0.00960997266384452\\
556	0.0095740133838415\\
557	0.00953716611704428\\
558	0.0094994041575164\\
559	0.00946072850645546\\
560	0.00942110960231462\\
561	0.00938051491477584\\
562	0.00933888295206795\\
563	0.00929620306916978\\
564	0.00925246879248052\\
565	0.00920752969012616\\
566	0.00916151708845009\\
567	0.00911442729326383\\
568	0.00906639133305028\\
569	0.00901719711457537\\
570	0.00896631970186846\\
571	0.00891408668734256\\
572	0.00886028503627413\\
573	0.00880513086802375\\
574	0.00872464282348522\\
575	0.00855320411017449\\
576	0.00828883645919685\\
577	0.0078978638763854\\
578	0.00763833815339484\\
579	0.00753152624116889\\
580	0.00744580350857311\\
581	0.00736281717889\\
582	0.00727854262256624\\
583	0.00719288419631409\\
584	0.00710545584045065\\
585	0.00701615980828359\\
586	0.0069248749165042\\
587	0.00683150671091789\\
588	0.00673592931731008\\
589	0.00663797579971203\\
590	0.00653734834917363\\
591	0.00643341581933106\\
592	0.00632466858477866\\
593	0.00620725772627072\\
594	0.00607109011801661\\
595	0.00588956027306329\\
596	0.00559184182825266\\
597	0.00498881296807123\\
598	0.00357511483354343\\
599	0\\
600	0\\
};
\addplot [color=mycolor9,solid,forget plot]
  table[row sep=crcr]{%
1	0.0125042813964841\\
2	0.012504279863156\\
3	0.0125042783059943\\
4	0.0125042767246081\\
5	0.0125042751185993\\
6	0.0125042734875627\\
7	0.0125042718310859\\
8	0.0125042701487491\\
9	0.012504268440125\\
10	0.0125042667047787\\
11	0.0125042649422673\\
12	0.0125042631521402\\
13	0.0125042613339387\\
14	0.0125042594871957\\
15	0.0125042576114357\\
16	0.012504255706175\\
17	0.0125042537709208\\
18	0.0125042518051718\\
19	0.0125042498084175\\
20	0.0125042477801382\\
21	0.0125042457198049\\
22	0.0125042436268793\\
23	0.0125042415008131\\
24	0.0125042393410485\\
25	0.0125042371470174\\
26	0.0125042349181415\\
27	0.0125042326538324\\
28	0.0125042303534909\\
29	0.012504228016507\\
30	0.01250422564226\\
31	0.0125042232301178\\
32	0.0125042207794371\\
33	0.0125042182895631\\
34	0.0125042157598292\\
35	0.0125042131895569\\
36	0.0125042105780553\\
37	0.0125042079246216\\
38	0.01250420522854\\
39	0.0125042024890821\\
40	0.0125041997055065\\
41	0.0125041968770583\\
42	0.0125041940029694\\
43	0.0125041910824579\\
44	0.0125041881147279\\
45	0.0125041850989693\\
46	0.0125041820343575\\
47	0.0125041789200535\\
48	0.012504175755203\\
49	0.0125041725389366\\
50	0.0125041692703697\\
51	0.0125041659486015\\
52	0.0125041625727157\\
53	0.0125041591417793\\
54	0.0125041556548432\\
55	0.012504152110941\\
56	0.0125041485090895\\
57	0.012504144848288\\
58	0.0125041411275181\\
59	0.0125041373457435\\
60	0.0125041335019094\\
61	0.0125041295949425\\
62	0.0125041256237507\\
63	0.0125041215872223\\
64	0.0125041174842265\\
65	0.0125041133136123\\
66	0.0125041090742084\\
67	0.0125041047648231\\
68	0.0125041003842438\\
69	0.0125040959312366\\
70	0.0125040914045459\\
71	0.0125040868028942\\
72	0.0125040821249817\\
73	0.0125040773694858\\
74	0.0125040725350609\\
75	0.0125040676203378\\
76	0.0125040626239237\\
77	0.0125040575444012\\
78	0.0125040523803285\\
79	0.0125040471302388\\
80	0.0125040417926397\\
81	0.0125040363660129\\
82	0.0125040308488141\\
83	0.012504025239472\\
84	0.0125040195363882\\
85	0.0125040137379369\\
86	0.012504007842464\\
87	0.0125040018482873\\
88	0.0125039957536955\\
89	0.0125039895569478\\
90	0.0125039832562738\\
91	0.0125039768498727\\
92	0.012503970335913\\
93	0.012503963712532\\
94	0.0125039569778351\\
95	0.0125039501298957\\
96	0.0125039431667544\\
97	0.0125039360864189\\
98	0.0125039288868629\\
99	0.0125039215660261\\
100	0.0125039141218136\\
101	0.0125039065520955\\
102	0.0125038988547059\\
103	0.0125038910274432\\
104	0.0125038830680689\\
105	0.0125038749743075\\
106	0.0125038667438459\\
107	0.0125038583743329\\
108	0.0125038498633787\\
109	0.0125038412085545\\
110	0.0125038324073921\\
111	0.012503823457383\\
112	0.0125038143559786\\
113	0.0125038051005892\\
114	0.0125037956885839\\
115	0.0125037861172899\\
116	0.0125037763839924\\
117	0.0125037664859339\\
118	0.0125037564203142\\
119	0.0125037461842894\\
120	0.0125037357749724\\
121	0.0125037251894319\\
122	0.012503714424692\\
123	0.0125037034777327\\
124	0.0125036923454888\\
125	0.01250368102485\\
126	0.0125036695126603\\
127	0.0125036578057182\\
128	0.0125036459007761\\
129	0.0125036337945399\\
130	0.0125036214836688\\
131	0.012503608964775\\
132	0.0125035962344226\\
133	0.0125035832891278\\
134	0.0125035701253572\\
135	0.0125035567395276\\
136	0.0125035431280038\\
137	0.0125035292870977\\
138	0.0125035152130657\\
139	0.0125035009021064\\
140	0.0125034863503579\\
141	0.012503471553897\\
142	0.0125034565087424\\
143	0.0125034412108735\\
144	0.0125034256562931\\
145	0.0125034098412112\\
146	0.0125033937625495\\
147	0.0125033774192035\\
148	0.01250336081429\\
149	0.0125033439507468\\
150	0.0125033267890371\\
151	0.0125033093237698\\
152	0.0125032915494555\\
153	0.0125032734605051\\
154	0.0125032550512279\\
155	0.0125032363158293\\
156	0.0125032172484099\\
157	0.0125031978429624\\
158	0.0125031780933706\\
159	0.012503157993407\\
160	0.0125031375367307\\
161	0.0125031167168857\\
162	0.0125030955272985\\
163	0.012503073961276\\
164	0.0125030520120036\\
165	0.0125030296725425\\
166	0.012503006935828\\
167	0.012502983794667\\
168	0.0125029602417355\\
169	0.0125029362695764\\
170	0.0125029118705971\\
171	0.0125028870370672\\
172	0.0125028617611154\\
173	0.0125028360347278\\
174	0.0125028098497447\\
175	0.0125027831978582\\
176	0.0125027560706094\\
177	0.0125027284593859\\
178	0.0125027003554187\\
179	0.0125026717497797\\
180	0.0125026426333787\\
181	0.0125026129969601\\
182	0.0125025828311008\\
183	0.0125025521262061\\
184	0.0125025208725074\\
185	0.0125024890600587\\
186	0.0125024566787337\\
187	0.0125024237182219\\
188	0.012502390168026\\
189	0.0125023560174582\\
190	0.0125023212556367\\
191	0.0125022858714824\\
192	0.0125022498537149\\
193	0.0125022131908496\\
194	0.0125021758711933\\
195	0.012502137882841\\
196	0.0125020992136714\\
197	0.0125020598513437\\
198	0.0125020197832933\\
199	0.0125019789967277\\
200	0.0125019374786227\\
201	0.0125018952157176\\
202	0.0125018521945117\\
203	0.0125018084012593\\
204	0.0125017638219658\\
205	0.0125017184423829\\
206	0.0125016722480039\\
207	0.0125016252240594\\
208	0.0125015773555123\\
209	0.0125015286270531\\
210	0.0125014790230948\\
211	0.0125014285277679\\
212	0.0125013771249155\\
213	0.012501324798088\\
214	0.0125012715305376\\
215	0.0125012173052132\\
216	0.0125011621047545\\
217	0.012501105911487\\
218	0.0125010487074156\\
219	0.0125009904742192\\
220	0.0125009311932449\\
221	0.0125008708455015\\
222	0.0125008094116536\\
223	0.0125007468720152\\
224	0.0125006832065436\\
225	0.0125006183948325\\
226	0.0125005524161053\\
227	0.0125004852492088\\
228	0.0125004168726057\\
229	0.0125003472643677\\
230	0.0125002764021684\\
231	0.0125002042632757\\
232	0.0125001308245446\\
233	0.0125000560624088\\
234	0.0124999799528736\\
235	0.0124999024715071\\
236	0.0124998235934327\\
237	0.0124997432933199\\
238	0.012499661545376\\
239	0.0124995783233373\\
240	0.0124994936004597\\
241	0.0124994073495096\\
242	0.0124993195427543\\
243	0.0124992301519518\\
244	0.0124991391483412\\
245	0.0124990465026319\\
246	0.012498952184993\\
247	0.0124988561650423\\
248	0.0124987584118348\\
249	0.012498658893851\\
250	0.0124985575789849\\
251	0.0124984544345317\\
252	0.0124983494271742\\
253	0.0124982425229702\\
254	0.0124981336873387\\
255	0.0124980228850451\\
256	0.0124979100801874\\
257	0.0124977952361805\\
258	0.012497678315741\\
259	0.0124975592808711\\
260	0.0124974380928425\\
261	0.0124973147121796\\
262	0.0124971890986425\\
263	0.0124970612112107\\
264	0.0124969310080653\\
265	0.0124967984465734\\
266	0.0124966634832716\\
267	0.0124965260738513\\
268	0.012496386173146\\
269	0.0124962437351195\\
270	0.0124960987128586\\
271	0.0124959510585669\\
272	0.0124958007235603\\
273	0.0124956476582492\\
274	0.0124954918120677\\
275	0.0124953331331845\\
276	0.0124951715673754\\
277	0.0124950070533906\\
278	0.0124948395015193\\
279	0.0124946686712862\\
280	0.0124944948102393\\
281	0.0124943178863465\\
282	0.0124941378483797\\
283	0.0124939546444442\\
284	0.0124937682219852\\
285	0.0124935785277959\\
286	0.0124933855080263\\
287	0.0124931891081952\\
288	0.0124929892732033\\
289	0.012492785947348\\
290	0.0124925790743417\\
291	0.0124923685973309\\
292	0.0124921544589191\\
293	0.0124919366011912\\
294	0.0124917149657412\\
295	0.0124914894937022\\
296	0.0124912601257788\\
297	0.0124910268022822\\
298	0.0124907894631671\\
299	0.0124905480480709\\
300	0.0124903024963525\\
301	0.0124900527471327\\
302	0.0124897987393324\\
303	0.0124895404117075\\
304	0.012489277702879\\
305	0.0124890105513551\\
306	0.0124887388955438\\
307	0.0124884626737555\\
308	0.0124881818242108\\
309	0.0124878962850951\\
310	0.0124876059948107\\
311	0.0124873108928717\\
312	0.0124870109228012\\
313	0.0124867060412385\\
314	0.0124863962469504\\
315	0.0124860816786911\\
316	0.0124857629914215\\
317	0.0124854390499873\\
318	0.0124851092585999\\
319	0.0124847735129621\\
320	0.0124844317069656\\
321	0.0124840837326609\\
322	0.0124837294802262\\
323	0.0124833688379362\\
324	0.0124830016921301\\
325	0.0124826279271796\\
326	0.012482247425456\\
327	0.0124818600672971\\
328	0.0124814657309735\\
329	0.0124810642926547\\
330	0.0124806556263746\\
331	0.0124802396039965\\
332	0.0124798160951785\\
333	0.0124793849673377\\
334	0.0124789460856146\\
335	0.0124784993128374\\
336	0.012478044509486\\
337	0.0124775815336559\\
338	0.0124771102410219\\
339	0.012476630484803\\
340	0.0124761421157266\\
341	0.0124756449819938\\
342	0.0124751389292448\\
343	0.0124746238005248\\
344	0.0124740994362498\\
345	0.0124735656741724\\
346	0.0124730223493508\\
347	0.0124724692941191\\
348	0.0124719063380572\\
349	0.0124713333079614\\
350	0.0124707500278139\\
351	0.0124701563187508\\
352	0.0124695519990265\\
353	0.0124689368839671\\
354	0.0124683107859021\\
355	0.0124676735140504\\
356	0.0124670248743245\\
357	0.0124663646690444\\
358	0.0124656926969143\\
359	0.0124650087556106\\
360	0.0124643126403343\\
361	0.012463604133633\\
362	0.0124628830135796\\
363	0.0124621490536067\\
364	0.0124614020222679\\
365	0.0124606416827807\\
366	0.0124598677923641\\
367	0.0124590801032656\\
368	0.0124582783618723\\
369	0.0124574623083692\\
370	0.0124566316764213\\
371	0.0124557861928318\\
372	0.0124549255771381\\
373	0.0124540495412789\\
374	0.01245315778928\\
375	0.012452250017062\\
376	0.0124513259124923\\
377	0.0124503851557323\\
378	0.0124494274193314\\
379	0.012448452365398\\
380	0.0124474596283017\\
381	0.0124464488599826\\
382	0.012445419721733\\
383	0.0124443718672787\\
384	0.0124433049425044\\
385	0.0124422185850834\\
386	0.0124411124241317\\
387	0.0124399860798356\\
388	0.0124388391630253\\
389	0.0124376712747637\\
390	0.0124364820058789\\
391	0.0124352709363835\\
392	0.0124340376349621\\
393	0.0124327816584449\\
394	0.0124315025506512\\
395	0.0124301998416107\\
396	0.0124288730465825\\
397	0.0124275216648607\\
398	0.0124261451783975\\
399	0.0124247430502708\\
400	0.012423314723067\\
401	0.0124218596170114\\
402	0.0124203771267264\\
403	0.0124188666175301\\
404	0.0124173274211892\\
405	0.0124157588300651\\
406	0.0124141600946032\\
407	0.0124125304153304\\
408	0.0124108689318808\\
409	0.0124091747201955\\
410	0.0124074467894693\\
411	0.0124056840961264\\
412	0.0124038855800256\\
413	0.0124020501093927\\
414	0.0124001754678643\\
415	0.0123982599804733\\
416	0.012396302559117\\
417	0.0123943020900616\\
418	0.0123922574125513\\
419	0.0123901673130505\\
420	0.0123880305021444\\
421	0.0123858456507439\\
422	0.0123836113723348\\
423	0.0123813262319508\\
424	0.0123789887331281\\
425	0.0123765973159215\\
426	0.0123741503452912\\
427	0.0123716461012829\\
428	0.0123690827752048\\
429	0.0123664585001967\\
430	0.0123637713795259\\
431	0.0123610195900388\\
432	0.0123582012731023\\
433	0.0123553141942935\\
434	0.0123523556350233\\
435	0.0123493230463589\\
436	0.0123462138357786\\
437	0.0123430252714849\\
438	0.0123397544677696\\
439	0.0123363983619402\\
440	0.0123329536711212\\
441	0.0123294168069508\\
442	0.0123257837522329\\
443	0.0123220502643853\\
444	0.0123182142760293\\
445	0.0123142750316562\\
446	0.012310227560643\\
447	0.0123060624933546\\
448	0.0123017745491636\\
449	0.0122973581259335\\
450	0.012292807248134\\
451	0.0122881154333784\\
452	0.0122832752883144\\
453	0.0122782772912278\\
454	0.0122731077643023\\
455	0.0122677658691481\\
456	0.0122622753072779\\
457	0.0122566342913368\\
458	0.0122508214742564\\
459	0.0122448030757952\\
460	0.012238569051616\\
461	0.0122321038581154\\
462	0.0122253871779358\\
463	0.0122183877425042\\
464	0.0122110801628658\\
465	0.0122034644462425\\
466	0.012195571063476\\
467	0.012187361779633\\
468	0.0121787567379619\\
469	0.0121696938498687\\
470	0.012160125571854\\
471	0.0121499975585025\\
472	0.0121392474514685\\
473	0.012116281940348\\
474	0.0120875117292555\\
475	0.0120580980725064\\
476	0.0120279958636458\\
477	0.0119971541043067\\
478	0.0119655174753707\\
479	0.0119330235342684\\
480	0.0118996018511712\\
481	0.0118651722843274\\
482	0.0118296433086857\\
483	0.0117929098040878\\
484	0.0117548500491487\\
485	0.0117107522708203\\
486	0.0116477830587934\\
487	0.0115840543321413\\
488	0.0115195655239011\\
489	0.0114543215079448\\
490	0.0113883319428579\\
491	0.011321612301923\\
492	0.0112541847215448\\
493	0.0111860799831065\\
494	0.0111173382814737\\
495	0.0110480137380876\\
496	0.0109781715812912\\
497	0.0109142993365318\\
498	0.0108894084779344\\
499	0.0108648011256737\\
500	0.0108405517327763\\
501	0.0108167447507493\\
502	0.0107934759728688\\
503	0.0107708541884305\\
504	0.0107490031066106\\
505	0.0107280637373214\\
506	0.0107081971347483\\
507	0.0106895876388807\\
508	0.0106724468797263\\
509	0.0106563430009931\\
510	0.010640256212759\\
511	0.010624188281311\\
512	0.0106081384693105\\
513	0.0105921027885619\\
514	0.0105760730741135\\
515	0.0105600358460041\\
516	0.0105439709004266\\
517	0.0105278495835077\\
518	0.0105116326831166\\
519	0.010495267849565\\
520	0.0104786864472087\\
521	0.0104618629634467\\
522	0.010444785281913\\
523	0.0104274402750027\\
524	0.010409813813195\\
525	0.0103918908125962\\
526	0.0103736553352649\\
527	0.010355090761378\\
528	0.0103361800582437\\
529	0.0103169061783647\\
530	0.0102972526281948\\
531	0.0102772042613134\\
532	0.0102567473374211\\
533	0.010235867642011\\
534	0.0102145505011398\\
535	0.0101927808003158\\
536	0.0101705430067024\\
537	0.0101478211930789\\
538	0.0101245990608679\\
539	0.0101008599579727\\
540	0.0100765868850004\\
541	0.010051762480306\\
542	0.0100263689700029\\
543	0.0100003928707983\\
544	0.00997382390675952\\
545	0.00994663028234151\\
546	0.00991879207359581\\
547	0.00989028880719799\\
548	0.00986109951605072\\
549	0.00983119736487534\\
550	0.0098005491034348\\
551	0.00976913474017009\\
552	0.00973693074790581\\
553	0.00970391543383491\\
554	0.00967007172010916\\
555	0.00963537658421949\\
556	0.00959980612120528\\
557	0.00956333290141669\\
558	0.00952590825509406\\
559	0.00948755089445722\\
560	0.00944823206997917\\
561	0.00940792046637175\\
562	0.00936656211640174\\
563	0.00932417644149528\\
564	0.0092807369301494\\
565	0.00923610933042622\\
566	0.0091903927955922\\
567	0.00914356761387222\\
568	0.00909544730654022\\
569	0.00904609248987307\\
570	0.00899554686237181\\
571	0.00894379302847974\\
572	0.00889070804722582\\
573	0.0088360322748802\\
574	0.00877996467453759\\
575	0.00872201521456251\\
576	0.00861400657524299\\
577	0.00844456166084187\\
578	0.00817323631074009\\
579	0.00779191677837049\\
580	0.00749650363288286\\
581	0.00737443998447818\\
582	0.00728422197907741\\
583	0.00719482330235922\\
584	0.0071064288968317\\
585	0.00701658985358563\\
586	0.006925112872923\\
587	0.00683161554949806\\
588	0.00673599013947704\\
589	0.00663799811207612\\
590	0.00653735241989114\\
591	0.00643341581933105\\
592	0.00632466858477866\\
593	0.00620725772627072\\
594	0.00607109011801661\\
595	0.00588956027306329\\
596	0.00559184182825266\\
597	0.00498881296807123\\
598	0.00357511483354343\\
599	0\\
600	0\\
};
\addplot [color=blue!50!mycolor7,solid,forget plot]
  table[row sep=crcr]{%
1	0.0125539685419108\\
2	0.012553966399082\\
3	0.0125539642208014\\
4	0.0125539620064649\\
5	0.012553959755458\\
6	0.0125539574671551\\
7	0.0125539551409199\\
8	0.0125539527761048\\
9	0.012553950372051\\
10	0.012553947928088\\
11	0.0125539454435336\\
12	0.0125539429176937\\
13	0.012553940349862\\
14	0.0125539377393198\\
15	0.0125539350853357\\
16	0.0125539323871658\\
17	0.0125539296440529\\
18	0.0125539268552267\\
19	0.0125539240199031\\
20	0.0125539211372846\\
21	0.0125539182065597\\
22	0.0125539152269025\\
23	0.0125539121974728\\
24	0.0125539091174155\\
25	0.0125539059858609\\
26	0.0125539028019235\\
27	0.0125538995647029\\
28	0.0125538962732826\\
29	0.01255389292673\\
30	0.0125538895240965\\
31	0.0125538860644166\\
32	0.0125538825467081\\
33	0.0125538789699714\\
34	0.0125538753331897\\
35	0.0125538716353281\\
36	0.012553867875334\\
37	0.012553864052136\\
38	0.0125538601646442\\
39	0.0125538562117495\\
40	0.0125538521923237\\
41	0.0125538481052185\\
42	0.0125538439492658\\
43	0.0125538397232769\\
44	0.0125538354260426\\
45	0.0125538310563323\\
46	0.0125538266128941\\
47	0.0125538220944541\\
48	0.0125538174997161\\
49	0.0125538128273615\\
50	0.0125538080760486\\
51	0.012553803244412\\
52	0.0125537983310629\\
53	0.0125537933345879\\
54	0.0125537882535491\\
55	0.0125537830864834\\
56	0.0125537778319023\\
57	0.0125537724882912\\
58	0.0125537670541091\\
59	0.0125537615277882\\
60	0.012553755907733\\
61	0.0125537501923207\\
62	0.0125537443798997\\
63	0.0125537384687897\\
64	0.0125537324572813\\
65	0.012553726343635\\
66	0.012553720126081\\
67	0.0125537138028187\\
68	0.012553707372016\\
69	0.0125537008318089\\
70	0.0125536941803007\\
71	0.0125536874155618\\
72	0.0125536805356286\\
73	0.0125536735385036\\
74	0.0125536664221542\\
75	0.0125536591845124\\
76	0.0125536518234739\\
77	0.0125536443368978\\
78	0.0125536367226059\\
79	0.0125536289783816\\
80	0.0125536211019701\\
81	0.0125536130910766\\
82	0.0125536049433666\\
83	0.0125535966564646\\
84	0.0125535882279538\\
85	0.0125535796553749\\
86	0.0125535709362256\\
87	0.01255356206796\\
88	0.0125535530479875\\
89	0.0125535438736722\\
90	0.0125535345423322\\
91	0.0125535250512385\\
92	0.0125535153976142\\
93	0.0125535055786342\\
94	0.0125534955914235\\
95	0.0125534854330569\\
96	0.0125534751005578\\
97	0.0125534645908978\\
98	0.0125534539009949\\
99	0.0125534430277135\\
100	0.0125534319678626\\
101	0.0125534207181957\\
102	0.0125534092754088\\
103	0.0125533976361403\\
104	0.0125533857969695\\
105	0.0125533737544155\\
106	0.0125533615049363\\
107	0.0125533490449279\\
108	0.0125533363707226\\
109	0.0125533234785884\\
110	0.0125533103647278\\
111	0.0125532970252763\\
112	0.0125532834563015\\
113	0.0125532696538018\\
114	0.012553255613705\\
115	0.0125532413318674\\
116	0.0125532268040721\\
117	0.0125532120260279\\
118	0.0125531969933678\\
119	0.0125531817016478\\
120	0.0125531661463453\\
121	0.0125531503228579\\
122	0.0125531342265015\\
123	0.0125531178525091\\
124	0.0125531011960292\\
125	0.012553084252124\\
126	0.0125530670157678\\
127	0.0125530494818452\\
128	0.0125530316451497\\
129	0.0125530135003811\\
130	0.0125529950421442\\
131	0.0125529762649465\\
132	0.0125529571631961\\
133	0.0125529377311993\\
134	0.0125529179631586\\
135	0.01255289785317\\
136	0.0125528773952204\\
137	0.0125528565831849\\
138	0.0125528354108241\\
139	0.0125528138717809\\
140	0.0125527919595785\\
141	0.0125527696676186\\
142	0.012552746989183\\
143	0.0125527239174435\\
144	0.0125527004454877\\
145	0.0125526765663813\\
146	0.0125526522732679\\
147	0.0125526275593746\\
148	0.012552602417158\\
149	0.0125525768359298\\
150	0.0125525508079685\\
151	0.0125525243254168\\
152	0.0125524973802789\\
153	0.0125524699644186\\
154	0.0125524420695562\\
155	0.0125524136872667\\
156	0.0125523848089767\\
157	0.0125523554259619\\
158	0.0125523255293448\\
159	0.0125522951100915\\
160	0.0125522641590094\\
161	0.0125522326667443\\
162	0.0125522006237774\\
163	0.0125521680204227\\
164	0.0125521348468238\\
165	0.0125521010929512\\
166	0.012552066748599\\
167	0.012552031803382\\
168	0.0125519962467326\\
169	0.0125519600678976\\
170	0.0125519232559347\\
171	0.0125518857997096\\
172	0.0125518476878923\\
173	0.0125518089089542\\
174	0.0125517694511639\\
175	0.0125517293025843\\
176	0.0125516884510687\\
177	0.0125516468842574\\
178	0.0125516045895737\\
179	0.0125515615542202\\
180	0.0125515177651753\\
181	0.0125514732091887\\
182	0.0125514278727782\\
183	0.0125513817422249\\
184	0.0125513348035696\\
185	0.0125512870426085\\
186	0.0125512384448888\\
187	0.0125511889957047\\
188	0.0125511386800928\\
189	0.0125510874828277\\
190	0.0125510353884175\\
191	0.0125509823810991\\
192	0.0125509284448337\\
193	0.0125508735633018\\
194	0.0125508177198986\\
195	0.0125507608977289\\
196	0.0125507030796022\\
197	0.0125506442480276\\
198	0.0125505843852087\\
199	0.012550523473038\\
200	0.0125504614930922\\
201	0.0125503984266262\\
202	0.0125503342545677\\
203	0.012550268957512\\
204	0.0125502025157157\\
205	0.0125501349090913\\
206	0.0125500661172013\\
207	0.0125499961192519\\
208	0.0125499248940875\\
209	0.0125498524201839\\
210	0.0125497786756425\\
211	0.0125497036381835\\
212	0.0125496272851399\\
213	0.0125495495934505\\
214	0.0125494705396535\\
215	0.0125493900998794\\
216	0.0125493082498444\\
217	0.0125492249648432\\
218	0.0125491402197419\\
219	0.0125490539889709\\
220	0.0125489662465171\\
221	0.012548876965917\\
222	0.0125487861202487\\
223	0.0125486936821244\\
224	0.0125485996236822\\
225	0.0125485039165785\\
226	0.0125484065319798\\
227	0.0125483074405545\\
228	0.0125482066124642\\
229	0.0125481040173556\\
230	0.0125479996243518\\
231	0.0125478934020433\\
232	0.0125477853184791\\
233	0.012547675341158\\
234	0.0125475634370188\\
235	0.0125474495724314\\
236	0.012547333713187\\
237	0.0125472158244885\\
238	0.0125470958709407\\
239	0.0125469738165402\\
240	0.012546849624665\\
241	0.0125467232580648\\
242	0.0125465946788495\\
243	0.0125464638484795\\
244	0.0125463307277538\\
245	0.0125461952767997\\
246	0.0125460574550612\\
247	0.0125459172212873\\
248	0.0125457745335209\\
249	0.0125456293490864\\
250	0.0125454816245778\\
251	0.0125453313158465\\
252	0.0125451783779891\\
253	0.0125450227653341\\
254	0.01254486443143\\
255	0.0125447033290315\\
256	0.0125445394100871\\
257	0.0125443726257255\\
258	0.0125442029262424\\
259	0.0125440302610872\\
260	0.0125438545788501\\
261	0.0125436758272484\\
262	0.0125434939531138\\
263	0.0125433089023801\\
264	0.0125431206200705\\
265	0.0125429290502868\\
266	0.0125427341361986\\
267	0.0125425358200341\\
268	0.0125423340430724\\
269	0.0125421287456377\\
270	0.0125419198670949\\
271	0.012541707345846\\
272	0.0125414911193207\\
273	0.0125412711239449\\
274	0.0125410472950238\\
275	0.0125408195663476\\
276	0.0125405878688892\\
277	0.0125403521266324\\
278	0.0125401122446321\\
279	0.0125398680975298\\
280	0.0125396197518342\\
281	0.0125393671467273\\
282	0.012539110212118\\
283	0.0125388488769173\\
284	0.0125385830690327\\
285	0.012538312715364\\
286	0.012538037741799\\
287	0.0125377580732105\\
288	0.0125374736334527\\
289	0.0125371843453599\\
290	0.0125368901307449\\
291	0.0125365909103974\\
292	0.0125362866040847\\
293	0.012535977130551\\
294	0.0125356624075185\\
295	0.0125353423516877\\
296	0.0125350168787384\\
297	0.0125346859033298\\
298	0.0125343493391\\
299	0.0125340070986645\\
300	0.0125336590936123\\
301	0.0125333052344994\\
302	0.0125329454308381\\
303	0.0125325795910816\\
304	0.0125322076226013\\
305	0.0125318294316566\\
306	0.0125314449233571\\
307	0.0125310540016207\\
308	0.012530656569147\\
309	0.0125302525274589\\
310	0.0125298417771743\\
311	0.0125294242189521\\
312	0.0125289997563365\\
313	0.0125285683037279\\
314	0.012528129807243\\
315	0.0125276842903027\\
316	0.0125272318628926\\
317	0.012526771893439\\
318	0.0125263040346894\\
319	0.0125258281580859\\
320	0.0125253441332608\\
321	0.0125248518280226\\
322	0.0125243511083424\\
323	0.0125238418383405\\
324	0.0125233238802737\\
325	0.0125227970945226\\
326	0.0125222613395799\\
327	0.0125217164720388\\
328	0.012521162346582\\
329	0.0125205988159718\\
330	0.0125200257310399\\
331	0.0125194429406791\\
332	0.0125188502918345\\
333	0.0125182476294967\\
334	0.0125176347966954\\
335	0.0125170116344945\\
336	0.0125163779819879\\
337	0.0125157336762976\\
338	0.0125150785525734\\
339	0.0125144124439944\\
340	0.012513735181774\\
341	0.0125130465951674\\
342	0.0125123465114832\\
343	0.0125116347561004\\
344	0.0125109111524907\\
345	0.0125101755222495\\
346	0.0125094276851356\\
347	0.0125086674591237\\
348	0.0125078946604713\\
349	0.0125071091038047\\
350	0.0125063106022268\\
351	0.012505498967451\\
352	0.0125046740099563\\
353	0.0125038355391378\\
354	0.0125029833633412\\
355	0.0125021172893665\\
356	0.012501237119844\\
357	0.0125003426418489\\
358	0.0124994335750054\\
359	0.0124985092887477\\
360	0.0124975694338948\\
361	0.0124966150013712\\
362	0.0124956458335584\\
363	0.0124946617760865\\
364	0.012493662678593\\
365	0.0124926483951677\\
366	0.0124916187822288\\
367	0.0124905736896656\\
368	0.0124895129817267\\
369	0.0124884365284323\\
370	0.0124873442056209\\
371	0.0124862358953432\\
372	0.012485111486295\\
373	0.0124839708744501\\
374	0.0124828139644341\\
375	0.0124816406732979\\
376	0.0124804509417551\\
377	0.0124792447686313\\
378	0.0124780223194872\\
379	0.0124767842878654\\
380	0.0124755332908602\\
381	0.0124742656829281\\
382	0.0124729790186189\\
383	0.0124716731448712\\
384	0.0124703479191356\\
385	0.0124690032108148\\
386	0.0124676389028829\\
387	0.0124662548937047\\
388	0.0124648510990928\\
389	0.0124634274546289\\
390	0.0124619839182908\\
391	0.0124605204734457\\
392	0.012459037132254\\
393	0.012457533939517\\
394	0.0124560109771211\\
395	0.0124544683691344\\
396	0.0124529062876691\\
397	0.0124513249596597\\
398	0.0124497246747286\\
399	0.0124481057943539\\
400	0.0124464687625691\\
401	0.0124448141184359\\
402	0.0124431425107317\\
403	0.012441454715226\\
404	0.012439751654775\\
405	0.0124380344225396\\
406	0.0124363043092761\\
407	0.0124345628547054\\
408	0.0124328118954527\\
409	0.0124310535745122\\
410	0.0124292904567631\\
411	0.0124275256758275\\
412	0.0124257632536774\\
413	0.0124240089223275\\
414	0.0124222719390955\\
415	0.0124205205792832\\
416	0.0124187349564255\\
417	0.0124169142951179\\
418	0.0124150577916102\\
419	0.0124131646114102\\
420	0.012411233891286\\
421	0.0124092647380639\\
422	0.0124072562293581\\
423	0.0124052074077761\\
424	0.0124031172530648\\
425	0.0124009847230644\\
426	0.0123988088003222\\
427	0.0123965884314126\\
428	0.0123943225285245\\
429	0.0123920099710475\\
430	0.0123896496092874\\
431	0.0123872402488307\\
432	0.012384780621131\\
433	0.0123822693822572\\
434	0.0123797051597763\\
435	0.0123770865322508\\
436	0.0123744120181398\\
437	0.0123716800713922\\
438	0.0123688890758683\\
439	0.0123660373378166\\
440	0.0123631230766227\\
441	0.0123601444222086\\
442	0.0123570994589875\\
443	0.0123539863805868\\
444	0.0123508033216867\\
445	0.0123475479501921\\
446	0.0123442175795041\\
447	0.0123408097466111\\
448	0.0123373218590584\\
449	0.0123337511725555\\
450	0.0123300947644995\\
451	0.012326349482727\\
452	0.0123225118709252\\
453	0.0123185782771384\\
454	0.0123145462179362\\
455	0.0123104143066162\\
456	0.0123061785491335\\
457	0.0123018330595863\\
458	0.012297371306301\\
459	0.012292788197805\\
460	0.0122880781215863\\
461	0.0122832347049287\\
462	0.0122782499857507\\
463	0.0122731128283286\\
464	0.0122678176838001\\
465	0.0122623723303896\\
466	0.0122567932091308\\
467	0.0122510652800892\\
468	0.0122451583799879\\
469	0.0122390532822709\\
470	0.0122327351178461\\
471	0.0122261911498775\\
472	0.0122193999742811\\
473	0.0122123492021528\\
474	0.0122050282442201\\
475	0.0121974853517383\\
476	0.0121896739902676\\
477	0.012181584337354\\
478	0.0121731272355642\\
479	0.0121642686454854\\
480	0.012154970113043\\
481	0.0121451880492609\\
482	0.0121348728608001\\
483	0.0121239678834108\\
484	0.0121124081923805\\
485	0.0120963155019349\\
486	0.0120649176038134\\
487	0.0120328128768236\\
488	0.0119999514025083\\
489	0.0119662782897598\\
490	0.0119317320930044\\
491	0.0118962437302807\\
492	0.0118597350033185\\
493	0.0118221172405603\\
494	0.0117832889149886\\
495	0.0117431343588629\\
496	0.0117015184283364\\
497	0.0116549116336663\\
498	0.0115860121225372\\
499	0.0115162499669502\\
500	0.0114456167315407\\
501	0.0113741133849686\\
502	0.0113017488788742\\
503	0.0112285384910014\\
504	0.0111545057219939\\
505	0.0110796805654126\\
506	0.0110041039437151\\
507	0.010927831037003\\
508	0.0108509311393083\\
509	0.0107916350492174\\
510	0.0107635466787545\\
511	0.0107357347046062\\
512	0.0107082829141176\\
513	0.010681286929017\\
514	0.010654856098325\\
515	0.0106291155566186\\
516	0.0106042089535487\\
517	0.0105803013466249\\
518	0.0105575825649882\\
519	0.010536271481845\\
520	0.0105166211944132\\
521	0.0104973368161173\\
522	0.0104780134087807\\
523	0.0104586506232094\\
524	0.0104392449365551\\
525	0.0104197887016708\\
526	0.0104002689680724\\
527	0.0103806660200109\\
528	0.0103609515612236\\
529	0.0103410864712358\\
530	0.0103210180366013\\
531	0.0103006765426013\\
532	0.0102799958356522\\
533	0.0102589602236068\\
534	0.0102375527368014\\
535	0.0102157551054318\\
536	0.0101935477733669\\
537	0.010170909964095\\
538	0.0101478198195821\\
539	0.0101242546394033\\
540	0.0101001912560033\\
541	0.0100756065930546\\
542	0.0100504784681263\\
543	0.0100247867189688\\
544	0.00999851978886971\\
545	0.00997166200880703\\
546	0.00994417963842086\\
547	0.00991605082722985\\
548	0.00988725312771732\\
549	0.00985776361556839\\
550	0.00982755393573083\\
551	0.00979658948103769\\
552	0.00976484812125456\\
553	0.00973230776167854\\
554	0.00969894570077598\\
555	0.00966473846632073\\
556	0.00962966177862186\\
557	0.00959369033328697\\
558	0.00955679163728458\\
559	0.00951894667692255\\
560	0.00948012721252465\\
561	0.00944030388348263\\
562	0.00939942773091895\\
563	0.00935749447232193\\
564	0.00931447279380433\\
565	0.00927033353760196\\
566	0.0092250271380538\\
567	0.0091785227373342\\
568	0.00913074503562235\\
569	0.00908172329564748\\
570	0.00903146923488505\\
571	0.00897992032434836\\
572	0.00892692134903677\\
573	0.00887258532325634\\
574	0.00881679907641235\\
575	0.00875932341700503\\
576	0.00870064929330184\\
577	0.00864017654027085\\
578	0.00852481707220463\\
579	0.00835776302369814\\
580	0.00811579247804831\\
581	0.00774959098955616\\
582	0.00736199746753604\\
583	0.00723369909236408\\
584	0.00711888015058533\\
585	0.00702285427221594\\
586	0.00692775913869259\\
587	0.00683310670274828\\
588	0.00673664854901047\\
589	0.00663838358992553\\
590	0.00653749881589685\\
591	0.00643344329816635\\
592	0.00632466858477865\\
593	0.00620725772627071\\
594	0.0060710901180166\\
595	0.00588956027306329\\
596	0.00559184182825266\\
597	0.00498881296807123\\
598	0.00357511483354343\\
599	0\\
600	0\\
};
\addplot [color=blue!40!mycolor9,solid,forget plot]
  table[row sep=crcr]{%
1	0.0128221241513979\\
2	0.0128221186474875\\
3	0.0128221130499332\\
4	0.0128221073571268\\
5	0.0128221015674324\\
6	0.0128220956791858\\
7	0.0128220896906941\\
8	0.0128220836002352\\
9	0.0128220774060573\\
10	0.0128220711063784\\
11	0.0128220646993857\\
12	0.0128220581832353\\
13	0.0128220515560513\\
14	0.0128220448159256\\
15	0.012822037960917\\
16	0.0128220309890511\\
17	0.0128220238983193\\
18	0.0128220166866783\\
19	0.0128220093520498\\
20	0.0128220018923194\\
21	0.0128219943053364\\
22	0.0128219865889129\\
23	0.0128219787408236\\
24	0.0128219707588045\\
25	0.0128219626405527\\
26	0.0128219543837257\\
27	0.0128219459859404\\
28	0.0128219374447729\\
29	0.0128219287577573\\
30	0.0128219199223853\\
31	0.0128219109361054\\
32	0.0128219017963222\\
33	0.0128218925003954\\
34	0.0128218830456394\\
35	0.0128218734293221\\
36	0.0128218636486645\\
37	0.0128218537008397\\
38	0.0128218435829722\\
39	0.0128218332921366\\
40	0.0128218228253576\\
41	0.0128218121796083\\
42	0.0128218013518097\\
43	0.01282179033883\\
44	0.0128217791374831\\
45	0.0128217677445283\\
46	0.012821756156669\\
47	0.0128217443705517\\
48	0.0128217323827655\\
49	0.0128217201898403\\
50	0.0128217077882465\\
51	0.0128216951743938\\
52	0.0128216823446297\\
53	0.0128216692952393\\
54	0.0128216560224432\\
55	0.0128216425223974\\
56	0.0128216287911915\\
57	0.0128216148248475\\
58	0.0128216006193194\\
59	0.0128215861704912\\
60	0.0128215714741761\\
61	0.0128215565261152\\
62	0.0128215413219763\\
63	0.0128215258573526\\
64	0.0128215101277613\\
65	0.0128214941286427\\
66	0.0128214778553582\\
67	0.0128214613031896\\
68	0.0128214444673375\\
69	0.0128214273429196\\
70	0.0128214099249698\\
71	0.0128213922084363\\
72	0.0128213741881804\\
73	0.012821355858975\\
74	0.0128213372155029\\
75	0.0128213182523553\\
76	0.0128212989640304\\
77	0.0128212793449317\\
78	0.012821259389366\\
79	0.0128212390915426\\
80	0.0128212184455707\\
81	0.0128211974454583\\
82	0.0128211760851101\\
83	0.0128211543583258\\
84	0.0128211322587987\\
85	0.0128211097801131\\
86	0.0128210869157432\\
87	0.0128210636590505\\
88	0.0128210400032826\\
89	0.0128210159415705\\
90	0.0128209914669273\\
91	0.0128209665722455\\
92	0.0128209412502954\\
93	0.0128209154937231\\
94	0.0128208892950477\\
95	0.0128208626466598\\
96	0.0128208355408189\\
97	0.0128208079696515\\
98	0.0128207799251484\\
99	0.0128207513991627\\
100	0.0128207223834072\\
101	0.0128206928694522\\
102	0.012820662848723\\
103	0.0128206323124972\\
104	0.0128206012519027\\
105	0.0128205696579143\\
106	0.0128205375213521\\
107	0.0128205048328779\\
108	0.0128204715829931\\
109	0.0128204377620357\\
110	0.0128204033601777\\
111	0.0128203683674219\\
112	0.0128203327735994\\
113	0.0128202965683666\\
114	0.0128202597412018\\
115	0.0128202222814028\\
116	0.0128201841780833\\
117	0.0128201454201699\\
118	0.0128201059963993\\
119	0.0128200658953142\\
120	0.0128200251052609\\
121	0.0128199836143853\\
122	0.0128199414106298\\
123	0.0128198984817297\\
124	0.0128198548152098\\
125	0.0128198103983806\\
126	0.0128197652183347\\
127	0.0128197192619432\\
128	0.0128196725158518\\
129	0.012819624966477\\
130	0.012819576600002\\
131	0.0128195274023729\\
132	0.0128194773592948\\
133	0.0128194264562272\\
134	0.0128193746783801\\
135	0.0128193220107098\\
136	0.0128192684379143\\
137	0.0128192139444292\\
138	0.012819158514423\\
139	0.0128191021317929\\
140	0.0128190447801604\\
141	0.0128189864428677\\
142	0.0128189271029748\\
143	0.012818866743259\\
144	0.012818805346216\\
145	0.0128187428940601\\
146	0.0128186793686993\\
147	0.0128186147516327\\
148	0.0128185490238399\\
149	0.0128184821662585\\
150	0.0128184141595004\\
151	0.0128183449838462\\
152	0.0128182746192392\\
153	0.0128182030452802\\
154	0.0128181302412215\\
155	0.0128180561859607\\
156	0.0128179808580352\\
157	0.0128179042356155\\
158	0.0128178262964995\\
159	0.0128177470181059\\
160	0.0128176663774676\\
161	0.0128175843512257\\
162	0.0128175009156222\\
163	0.0128174160464938\\
164	0.0128173297192645\\
165	0.0128172419089391\\
166	0.0128171525900958\\
167	0.012817061736879\\
168	0.012816969322992\\
169	0.0128168753216895\\
170	0.0128167797057699\\
171	0.012816682447568\\
172	0.0128165835189464\\
173	0.0128164828912883\\
174	0.0128163805354888\\
175	0.012816276421947\\
176	0.0128161705205573\\
177	0.0128160628007011\\
178	0.0128159532312381\\
179	0.0128158417804973\\
180	0.0128157284162684\\
181	0.0128156131057922\\
182	0.0128154958157517\\
183	0.0128153765122623\\
184	0.0128152551608629\\
185	0.0128151317265053\\
186	0.012815006173545\\
187	0.0128148784657306\\
188	0.012814748566194\\
189	0.0128146164374397\\
190	0.0128144820413344\\
191	0.0128143453390961\\
192	0.0128142062912833\\
193	0.0128140648577838\\
194	0.0128139209978035\\
195	0.0128137746698547\\
196	0.0128136258317447\\
197	0.0128134744405639\\
198	0.0128133204526734\\
199	0.0128131638236932\\
200	0.0128130045084895\\
201	0.012812842461162\\
202	0.0128126776350311\\
203	0.012812509982625\\
204	0.0128123394556659\\
205	0.0128121660050569\\
206	0.0128119895808684\\
207	0.0128118101323236\\
208	0.0128116276077846\\
209	0.012811441954738\\
210	0.0128112531197804\\
211	0.0128110610486031\\
212	0.0128108656859772\\
213	0.0128106669757382\\
214	0.0128104648607704\\
215	0.0128102592829907\\
216	0.0128100501833332\\
217	0.0128098375017318\\
218	0.0128096211771042\\
219	0.0128094011473348\\
220	0.0128091773492575\\
221	0.012808949718638\\
222	0.0128087181901563\\
223	0.0128084826973885\\
224	0.0128082431727885\\
225	0.0128079995476693\\
226	0.0128077517521841\\
227	0.0128074997153071\\
228	0.0128072433648138\\
229	0.0128069826272612\\
230	0.0128067174279681\\
231	0.0128064476909936\\
232	0.0128061733391172\\
233	0.0128058942938175\\
234	0.0128056104752502\\
235	0.012805321802227\\
236	0.012805028192193\\
237	0.0128047295612044\\
238	0.0128044258239057\\
239	0.0128041168935065\\
240	0.0128038026817578\\
241	0.0128034830989285\\
242	0.0128031580537809\\
243	0.0128028274535461\\
244	0.0128024912038992\\
245	0.0128021492089338\\
246	0.0128018013711366\\
247	0.0128014475913609\\
248	0.0128010877688008\\
249	0.012800721800964\\
250	0.0128003495836446\\
251	0.012799971010896\\
252	0.0127995859750026\\
253	0.0127991943664522\\
254	0.0127987960739067\\
255	0.0127983909841739\\
256	0.0127979789821779\\
257	0.0127975599509297\\
258	0.0127971337714976\\
259	0.0127967003229768\\
260	0.0127962594824592\\
261	0.012795811125003\\
262	0.0127953551236015\\
263	0.0127948913491526\\
264	0.0127944196704277\\
265	0.0127939399540401\\
266	0.0127934520644142\\
267	0.012792955863754\\
268	0.0127924512120117\\
269	0.0127919379668566\\
270	0.0127914159836421\\
271	0.0127908851153716\\
272	0.0127903452126572\\
273	0.012789796123658\\
274	0.012789237693966\\
275	0.0127886697663585\\
276	0.012788092180285\\
277	0.012787504771209\\
278	0.0127869073725014\\
279	0.0127862998293162\\
280	0.0127856819722864\\
281	0.012785053628519\\
282	0.0127844146223136\\
283	0.0127837647751204\\
284	0.0127831039054983\\
285	0.0127824318290726\\
286	0.0127817483584917\\
287	0.0127810533033836\\
288	0.0127803464703117\\
289	0.0127796276627299\\
290	0.0127788966809378\\
291	0.0127781533220338\\
292	0.012777397379869\\
293	0.0127766286449993\\
294	0.0127758469046371\\
295	0.0127750519426019\\
296	0.0127742435392698\\
297	0.012773421471522\\
298	0.0127725855126917\\
299	0.01277173543251\\
300	0.0127708709970495\\
301	0.0127699919686672\\
302	0.0127690981059437\\
303	0.0127681891636222\\
304	0.012767264892544\\
305	0.0127663250395824\\
306	0.0127653693475776\\
307	0.0127643975552741\\
308	0.0127634093972768\\
309	0.0127624046040533\\
310	0.012761382902063\\
311	0.0127603440141851\\
312	0.0127592876607808\\
313	0.0127582135617308\\
314	0.0127571214380861\\
315	0.0127560110017227\\
316	0.0127548819023352\\
317	0.0127537338167616\\
318	0.0127525664375331\\
319	0.0127513794527453\\
320	0.0127501725460195\\
321	0.0127489453964651\\
322	0.0127476976786425\\
323	0.0127464290625272\\
324	0.0127451392134742\\
325	0.0127438277921845\\
326	0.0127424944546712\\
327	0.0127411388522275\\
328	0.0127397606313953\\
329	0.0127383594339348\\
330	0.0127369348967955\\
331	0.0127354866520871\\
332	0.0127340143270527\\
333	0.0127325175440416\\
334	0.0127309959204837\\
335	0.012729449068864\\
336	0.0127278765966982\\
337	0.0127262781065082\\
338	0.0127246531957985\\
339	0.0127230014570321\\
340	0.0127213224776067\\
341	0.0127196158398309\\
342	0.0127178811208987\\
343	0.012716117892865\\
344	0.0127143257226191\\
345	0.012712504171857\\
346	0.0127106527970532\\
347	0.0127087711494308\\
348	0.0127068587749304\\
349	0.012704915214177\\
350	0.0127029400024452\\
351	0.0127009326696169\\
352	0.0126988927401189\\
353	0.0126968197327897\\
354	0.0126947131605148\\
355	0.0126925725290967\\
356	0.0126903973335947\\
357	0.0126881870464451\\
358	0.0126859410815036\\
359	0.0126836587326256\\
360	0.0126813395091237\\
361	0.0126789834048322\\
362	0.0126765898920321\\
363	0.0126741584365697\\
364	0.0126716884997981\\
365	0.0126691795434975\\
366	0.0126666310385116\\
367	0.0126640424590216\\
368	0.0126614131256183\\
369	0.0126587424619835\\
370	0.0126560298704718\\
371	0.0126532747474923\\
372	0.0126504764835539\\
373	0.0126476344635425\\
374	0.0126447480678077\\
375	0.0126418166757314\\
376	0.0126388396765391\\
377	0.0126358165005232\\
378	0.0126327467048574\\
379	0.0126296301807253\\
380	0.0126264671049288\\
381	0.0126232549673068\\
382	0.0126199919893131\\
383	0.0126166774047729\\
384	0.0126133104398394\\
385	0.012609890313639\\
386	0.0126064162390709\\
387	0.0126028874237943\\
388	0.0125993030714365\\
389	0.0125956623830687\\
390	0.0125919645589999\\
391	0.0125882088009529\\
392	0.0125843943146948\\
393	0.0125805203132207\\
394	0.0125765860205964\\
395	0.0125725906765956\\
396	0.0125685335422928\\
397	0.012564413906809\\
398	0.0125602310954449\\
399	0.0125559844794782\\
400	0.0125516734879364\\
401	0.0125472976216594\\
402	0.0125428564697679\\
403	0.0125383497277677\\
404	0.0125337772129918\\
405	0.0125291388585198\\
406	0.0125244346009591\\
407	0.0125196637162193\\
408	0.0125148251632131\\
409	0.0125099245732156\\
410	0.0125049637052244\\
411	0.0124999449000599\\
412	0.0124948714160985\\
413	0.0124897487860816\\
414	0.0124845907621011\\
415	0.0124813395821847\\
416	0.0124791097948332\\
417	0.0124768435857581\\
418	0.0124745405489178\\
419	0.0124722002892683\\
420	0.012469822439971\\
421	0.0124674067170455\\
422	0.0124649530934125\\
423	0.0124624623825015\\
424	0.0124599384475586\\
425	0.0124573787205636\\
426	0.0124547743978464\\
427	0.0124521247917136\\
428	0.0124494292085696\\
429	0.0124466869492808\\
430	0.0124438973079154\\
431	0.0124410595699248\\
432	0.0124381730132808\\
433	0.0124352369125776\\
434	0.0124322505378799\\
435	0.0124292131544308\\
436	0.0124261240230152\\
437	0.0124229824003044\\
438	0.0124197875391983\\
439	0.0124165386894555\\
440	0.0124132350998524\\
441	0.0124098760247526\\
442	0.0124064607339108\\
443	0.0124029884936748\\
444	0.0123994585404522\\
445	0.0123958701013909\\
446	0.0123922224374629\\
447	0.0123885148041532\\
448	0.012384746465052\\
449	0.0123809167314593\\
450	0.0123770248782612\\
451	0.0123730701783668\\
452	0.0123690519602327\\
453	0.0123649697201982\\
454	0.0123608230758601\\
455	0.0123566115534947\\
456	0.0123523347708978\\
457	0.012347992523575\\
458	0.0123435836487435\\
459	0.0123391080984532\\
460	0.0123345663514086\\
461	0.0123299590957613\\
462	0.0123252873618982\\
463	0.0123205533906122\\
464	0.0123157610411223\\
465	0.012310914948296\\
466	0.0123060179926898\\
467	0.0123010737395066\\
468	0.0122960890425682\\
469	0.0122910761707491\\
470	0.0122860532579802\\
471	0.0122808911831758\\
472	0.0122755799847244\\
473	0.012270114742827\\
474	0.0122644967037555\\
475	0.0122587475957867\\
476	0.0122528499138829\\
477	0.0122467985890392\\
478	0.0122405583835642\\
479	0.0122341196501383\\
480	0.0122274713422001\\
481	0.012220597866337\\
482	0.0122134732558553\\
483	0.0122060936535532\\
484	0.012198460194408\\
485	0.0121906110028726\\
486	0.0121825203336693\\
487	0.0121741841374296\\
488	0.0121655485355651\\
489	0.0121565490614545\\
490	0.012147152934935\\
491	0.0121373231973741\\
492	0.0121270180414425\\
493	0.0121161900061922\\
494	0.0121047849719389\\
495	0.0120927410471809\\
496	0.0120799872030659\\
497	0.0120636331394824\\
498	0.012029218695471\\
499	0.0119940190150183\\
500	0.0119579738482337\\
501	0.0119210183688601\\
502	0.0118830809599045\\
503	0.0118440805517895\\
504	0.0118039251460552\\
505	0.011762511791863\\
506	0.0117197229902744\\
507	0.0116754242022384\\
508	0.0116294590585493\\
509	0.0115721076825285\\
510	0.0114966338628686\\
511	0.0114201756042756\\
512	0.0113427275167932\\
513	0.0112642867859971\\
514	0.0111848523399804\\
515	0.0111044314852391\\
516	0.0110230325129274\\
517	0.0109406771844838\\
518	0.0108574055503264\\
519	0.0107732722674924\\
520	0.0106883499578858\\
521	0.0106441427838155\\
522	0.010612107277377\\
523	0.0105803246125866\\
524	0.0105488907673634\\
525	0.0105179159298167\\
526	0.0104875267373642\\
527	0.010457868989134\\
528	0.0104291110278388\\
529	0.0104014476627862\\
530	0.0103751049387132\\
531	0.0103503458851833\\
532	0.0103268742489999\\
533	0.0103032680621459\\
534	0.0102795267589314\\
535	0.0102556468073683\\
536	0.0102316207352366\\
537	0.0102074359058474\\
538	0.0101830729830778\\
539	0.0101585040114805\\
540	0.0101336900207503\\
541	0.010108578040636\\
542	0.0100830973907246\\
543	0.0100571550882699\\
544	0.0100307088934362\\
545	0.0100037458230299\\
546	0.00997625334123156\\
547	0.00994818649442175\\
548	0.00991951861369859\\
549	0.00989022144399231\\
550	0.00986026544414329\\
551	0.00982961512878818\\
552	0.00979823029458212\\
553	0.00976607959939507\\
554	0.00973313584000431\\
555	0.00969937447888449\\
556	0.0096647699938608\\
557	0.00962929585758961\\
558	0.00959292452181416\\
559	0.00955562739215151\\
560	0.00951737483940085\\
561	0.00947813619386819\\
562	0.0094378797362552\\
563	0.00939656988361341\\
564	0.00935417160700795\\
565	0.00931065457927453\\
566	0.00926597358132882\\
567	0.00922009446115416\\
568	0.00917299651438998\\
569	0.00912463165109635\\
570	0.0090749525458088\\
571	0.00902388281798735\\
572	0.00897134716389313\\
573	0.00891744112791506\\
574	0.00886209983796009\\
575	0.00880519743323962\\
576	0.00874672803872491\\
577	0.00868656413408832\\
578	0.00862465950032357\\
579	0.00856132452076371\\
580	0.00845850078546796\\
581	0.00829534704719994\\
582	0.00811899472346424\\
583	0.00776287265425107\\
584	0.00738782812585071\\
585	0.00710250401533227\\
586	0.00696803363265066\\
587	0.00684919446956261\\
588	0.00674593216677205\\
589	0.0066422892676759\\
590	0.00653990407331532\\
591	0.00643439021582487\\
592	0.00632485151876375\\
593	0.00620725772627071\\
594	0.0060710901180166\\
595	0.00588956027306329\\
596	0.00559184182825266\\
597	0.00498881296807123\\
598	0.00357511483354343\\
599	0\\
600	0\\
};
\addplot [color=blue!75!mycolor7,solid,forget plot]
  table[row sep=crcr]{%
1	0.0130225150736602\\
2	0.0130225120193724\\
3	0.0130225089129799\\
4	0.0130225057535874\\
5	0.0130225025402839\\
6	0.013022499272143\\
7	0.0130224959482223\\
8	0.0130224925675629\\
9	0.0130224891291897\\
10	0.0130224856321107\\
11	0.0130224820753167\\
12	0.0130224784577813\\
13	0.0130224747784603\\
14	0.0130224710362915\\
15	0.0130224672301946\\
16	0.0130224633590705\\
17	0.0130224594218011\\
18	0.0130224554172494\\
19	0.0130224513442585\\
20	0.0130224472016516\\
21	0.0130224429882319\\
22	0.0130224387027817\\
23	0.0130224343440625\\
24	0.0130224299108144\\
25	0.0130224254017559\\
26	0.0130224208155833\\
27	0.0130224161509707\\
28	0.0130224114065691\\
29	0.0130224065810064\\
30	0.013022401672887\\
31	0.013022396680791\\
32	0.0130223916032744\\
33	0.0130223864388681\\
34	0.0130223811860778\\
35	0.0130223758433836\\
36	0.0130223704092391\\
37	0.0130223648820717\\
38	0.0130223592602815\\
39	0.0130223535422411\\
40	0.0130223477262952\\
41	0.0130223418107598\\
42	0.0130223357939222\\
43	0.0130223296740401\\
44	0.0130223234493411\\
45	0.0130223171180226\\
46	0.0130223106782507\\
47	0.01302230412816\\
48	0.0130222974658533\\
49	0.0130222906894003\\
50	0.0130222837968378\\
51	0.0130222767861688\\
52	0.0130222696553618\\
53	0.0130222624023506\\
54	0.0130222550250332\\
55	0.0130222475212715\\
56	0.0130222398888909\\
57	0.0130222321256789\\
58	0.0130222242293854\\
59	0.0130222161977214\\
60	0.0130222080283584\\
61	0.0130221997189281\\
62	0.0130221912670214\\
63	0.0130221826701876\\
64	0.0130221739259341\\
65	0.0130221650317253\\
66	0.013022155984982\\
67	0.0130221467830806\\
68	0.0130221374233525\\
69	0.0130221279030832\\
70	0.0130221182195113\\
71	0.0130221083698283\\
72	0.0130220983511771\\
73	0.0130220881606517\\
74	0.0130220777952959\\
75	0.0130220672521029\\
76	0.013022056528014\\
77	0.0130220456199182\\
78	0.0130220345246508\\
79	0.0130220232389928\\
80	0.0130220117596698\\
81	0.0130220000833512\\
82	0.0130219882066492\\
83	0.0130219761261178\\
84	0.0130219638382516\\
85	0.0130219513394851\\
86	0.0130219386261916\\
87	0.0130219256946821\\
88	0.0130219125412042\\
89	0.0130218991619409\\
90	0.01302188555301\\
91	0.0130218717104624\\
92	0.0130218576302811\\
93	0.0130218433083805\\
94	0.0130218287406045\\
95	0.0130218139227259\\
96	0.0130217988504449\\
97	0.0130217835193878\\
98	0.0130217679251062\\
99	0.0130217520630749\\
100	0.0130217359286916\\
101	0.0130217195172747\\
102	0.0130217028240625\\
103	0.0130216858442116\\
104	0.0130216685727958\\
105	0.0130216510048041\\
106	0.01302163313514\\
107	0.0130216149586195\\
108	0.0130215964699698\\
109	0.0130215776638279\\
110	0.0130215585347389\\
111	0.0130215390771545\\
112	0.0130215192854316\\
113	0.0130214991538303\\
114	0.0130214786765127\\
115	0.0130214578475409\\
116	0.0130214366608754\\
117	0.0130214151103737\\
118	0.013021393189788\\
119	0.0130213708927639\\
120	0.0130213482128382\\
121	0.0130213251434375\\
122	0.0130213016778759\\
123	0.0130212778093534\\
124	0.0130212535309538\\
125	0.013021228835643\\
126	0.0130212037162665\\
127	0.0130211781655479\\
128	0.0130211521760866\\
129	0.0130211257403557\\
130	0.0130210988506997\\
131	0.0130210714993327\\
132	0.0130210436783359\\
133	0.0130210153796555\\
134	0.0130209865951002\\
135	0.0130209573163393\\
136	0.0130209275348998\\
137	0.0130208972421644\\
138	0.0130208664293689\\
139	0.0130208350875998\\
140	0.0130208032077922\\
141	0.013020770780727\\
142	0.0130207377970292\\
143	0.0130207042471656\\
144	0.0130206701214413\\
145	0.0130206354099936\\
146	0.0130206001027788\\
147	0.0130205641895691\\
148	0.0130205276599894\\
149	0.0130204905034875\\
150	0.0130204527093318\\
151	0.0130204142666074\\
152	0.0130203751642137\\
153	0.0130203353908611\\
154	0.0130202949350675\\
155	0.0130202537851553\\
156	0.0130202119292483\\
157	0.0130201693552677\\
158	0.0130201260509294\\
159	0.0130200820037399\\
160	0.0130200372009933\\
161	0.0130199916297673\\
162	0.0130199452769198\\
163	0.0130198981290851\\
164	0.0130198501726701\\
165	0.0130198013938504\\
166	0.0130197517785665\\
167	0.0130197013125199\\
168	0.0130196499811689\\
169	0.0130195977697245\\
170	0.0130195446631464\\
171	0.0130194906461387\\
172	0.0130194357031453\\
173	0.0130193798183459\\
174	0.0130193229756516\\
175	0.0130192651586999\\
176	0.0130192063508504\\
177	0.0130191465351802\\
178	0.0130190856944789\\
179	0.0130190238112441\\
180	0.0130189608676761\\
181	0.013018896845673\\
182	0.0130188317268259\\
183	0.0130187654924134\\
184	0.0130186981233965\\
185	0.0130186296004131\\
186	0.013018559903773\\
187	0.0130184890134518\\
188	0.0130184169090858\\
189	0.0130183435699658\\
190	0.0130182689750319\\
191	0.0130181931028671\\
192	0.0130181159316915\\
193	0.0130180374393562\\
194	0.0130179576033374\\
195	0.0130178764007295\\
196	0.0130177938082393\\
197	0.0130177098021791\\
198	0.0130176243584603\\
199	0.0130175374525867\\
200	0.0130174490596477\\
201	0.0130173591543111\\
202	0.0130172677108163\\
203	0.0130171747029673\\
204	0.013017080104125\\
205	0.0130169838872\\
206	0.0130168860246453\\
207	0.0130167864884482\\
208	0.0130166852501232\\
209	0.0130165822807033\\
210	0.0130164775507325\\
211	0.0130163710302578\\
212	0.0130162626888204\\
213	0.0130161524954476\\
214	0.0130160404186443\\
215	0.0130159264263842\\
216	0.013015810486101\\
217	0.0130156925646794\\
218	0.0130155726284459\\
219	0.0130154506431599\\
220	0.013015326574004\\
221	0.0130152003855742\\
222	0.0130150720418708\\
223	0.0130149415062881\\
224	0.0130148087416045\\
225	0.0130146737099722\\
226	0.0130145363729071\\
227	0.013014396691278\\
228	0.0130142546252963\\
229	0.0130141101345046\\
230	0.0130139631777664\\
231	0.0130138137132542\\
232	0.0130136616984386\\
233	0.0130135070900766\\
234	0.0130133498442\\
235	0.0130131899161033\\
236	0.0130130272603318\\
237	0.0130128618306691\\
238	0.0130126935801252\\
239	0.0130125224609229\\
240	0.013012348424486\\
241	0.0130121714214256\\
242	0.0130119914015271\\
243	0.0130118083137367\\
244	0.0130116221061478\\
245	0.0130114327259875\\
246	0.013011240119602\\
247	0.0130110442324428\\
248	0.0130108450090522\\
249	0.0130106423930486\\
250	0.0130104363271119\\
251	0.013010226752968\\
252	0.0130100136113741\\
253	0.0130097968421031\\
254	0.0130095763839278\\
255	0.0130093521746051\\
256	0.0130091241508601\\
257	0.0130088922483701\\
258	0.0130086564017474\\
259	0.0130084165445237\\
260	0.0130081726091322\\
261	0.0130079245268914\\
262	0.0130076722279879\\
263	0.0130074156414583\\
264	0.0130071546951722\\
265	0.0130068893158146\\
266	0.0130066194288675\\
267	0.0130063449585917\\
268	0.0130060658280088\\
269	0.0130057819588823\\
270	0.0130054932716976\\
271	0.0130051996856409\\
272	0.013004901118574\\
273	0.0130045974870004\\
274	0.0130042887060179\\
275	0.0130039746892596\\
276	0.0130036553488967\\
277	0.013003330595985\\
278	0.0130030003412403\\
279	0.0130026644927914\\
280	0.013002322957221\\
281	0.0130019756396162\\
282	0.0130016224435491\\
283	0.0130012632710565\\
284	0.0130008980226195\\
285	0.0130005265971432\\
286	0.0130001488919369\\
287	0.0129997648026934\\
288	0.0129993742234691\\
289	0.0129989770466638\\
290	0.0129985731630008\\
291	0.0129981624615067\\
292	0.0129977448294915\\
293	0.0129973201525295\\
294	0.0129968883144391\\
295	0.0129964491972639\\
296	0.0129960026812535\\
297	0.0129955486448445\\
298	0.0129950869646423\\
299	0.0129946175154027\\
300	0.0129941401700143\\
301	0.0129936547994806\\
302	0.0129931612729041\\
303	0.0129926594574693\\
304	0.0129921492184279\\
305	0.012991630419085\\
306	0.0129911029207873\\
307	0.0129905665829175\\
308	0.0129900212628987\\
309	0.0129894668162209\\
310	0.01298890309651\\
311	0.0129883299556555\\
312	0.0129877472439461\\
313	0.0129871548098513\\
314	0.0129865524983448\\
315	0.0129859401478064\\
316	0.0129853175979958\\
317	0.012984684688532\\
318	0.0129840412570151\\
319	0.0129833871390295\\
320	0.0129827221681496\\
321	0.0129820461759471\\
322	0.0129813589920006\\
323	0.0129806604439079\\
324	0.0129799503573005\\
325	0.0129792285558602\\
326	0.0129784948613397\\
327	0.0129777490935848\\
328	0.0129769910705603\\
329	0.0129762206083789\\
330	0.0129754375213335\\
331	0.0129746416219327\\
332	0.0129738327209403\\
333	0.0129730106274183\\
334	0.0129721751487738\\
335	0.0129713260908104\\
336	0.012970463257783\\
337	0.0129695864524581\\
338	0.0129686954761775\\
339	0.0129677901289272\\
340	0.0129668702094111\\
341	0.0129659355151289\\
342	0.0129649858424594\\
343	0.012964020986748\\
344	0.0129630407423988\\
345	0.0129620449029715\\
346	0.0129610332612806\\
347	0.0129600056094995\\
348	0.0129589617392641\\
349	0.0129579014417778\\
350	0.0129568245079108\\
351	0.0129557307282876\\
352	0.0129546198933432\\
353	0.0129534917933025\\
354	0.0129523462179658\\
355	0.0129511829560209\\
356	0.012950001793339\\
357	0.0129488025100273\\
358	0.0129475848818788\\
359	0.0129463487117659\\
360	0.0129450938341597\\
361	0.0129438200333205\\
362	0.0129425270913271\\
363	0.0129412147885994\\
364	0.0129398829076805\\
365	0.0129385312479636\\
366	0.0129371596757962\\
367	0.0129357682822976\\
368	0.0129343577230402\\
369	0.012932924364349\\
370	0.0129314675434969\\
371	0.0129299868273339\\
372	0.0129284817700042\\
373	0.0129269519124138\\
374	0.0129253967818844\\
375	0.012923815892023\\
376	0.0129222087416671\\
377	0.0129205748041308\\
378	0.0129189134528469\\
379	0.0129172234793178\\
380	0.0129155048955577\\
381	0.0129137576337257\\
382	0.0129119810979793\\
383	0.0129101746681613\\
384	0.0129083376978115\\
385	0.0129064695119542\\
386	0.0129045694046296\\
387	0.0129026366361339\\
388	0.0129006704299246\\
389	0.0128986699691414\\
390	0.0128966343926832\\
391	0.0128945627907712\\
392	0.0128924541999154\\
393	0.0128903075971844\\
394	0.0128881218936615\\
395	0.0128858959269434\\
396	0.0128836284525145\\
397	0.0128813181337972\\
398	0.0128789635306518\\
399	0.0128765630860954\\
400	0.0128741151110797\\
401	0.0128716177674844\\
402	0.0128690690505257\\
403	0.0128664667750224\\
404	0.0128638085800816\\
405	0.0128610920005398\\
406	0.0128583147862191\\
407	0.0128554763748126\\
408	0.0128525667828647\\
409	0.0128495819374439\\
410	0.0128465168804551\\
411	0.012843365947154\\
412	0.0128401225690261\\
413	0.0128367789591462\\
414	0.0128333239203598\\
415	0.0128286990332797\\
416	0.0128234021262148\\
417	0.0128180166341239\\
418	0.0128125412308796\\
419	0.0128069745855871\\
420	0.0128013153800388\\
421	0.0127955623585586\\
422	0.0127897144625424\\
423	0.0127837711430261\\
424	0.0127777325604093\\
425	0.012771595603294\\
426	0.012765355119604\\
427	0.0127590095802008\\
428	0.0127525574495674\\
429	0.0127459971871596\\
430	0.0127393272489458\\
431	0.0127325460894349\\
432	0.0127256521640962\\
433	0.0127186439316089\\
434	0.0127115198564744\\
435	0.01270427841201\\
436	0.0126969180837037\\
437	0.0126894373729905\\
438	0.0126818348015502\\
439	0.0126741089163148\\
440	0.0126662582953799\\
441	0.0126582815543499\\
442	0.0126501773510782\\
443	0.0126419443916143\\
444	0.0126335814411897\\
445	0.0126250873377048\\
446	0.0126164610081213\\
447	0.0126077015314601\\
448	0.0125988069413607\\
449	0.0125897710377402\\
450	0.0125805998022605\\
451	0.0125712953924108\\
452	0.0125618575066632\\
453	0.012552286091783\\
454	0.0125425814447039\\
455	0.0125327445211897\\
456	0.0125227777856401\\
457	0.01251268777782\\
458	0.0125024928731723\\
459	0.0124921574130544\\
460	0.0124816607799896\\
461	0.0124710019331076\\
462	0.0124601798385086\\
463	0.0124491946459253\\
464	0.0124380475091773\\
465	0.0124267402040968\\
466	0.012415275367245\\
467	0.0124036567011664\\
468	0.0123918903601347\\
469	0.0123799886388254\\
470	0.0123681460639805\\
471	0.0123627380329631\\
472	0.0123572185042465\\
473	0.0123515856473842\\
474	0.0123458384320011\\
475	0.0123399726850037\\
476	0.0123339849132016\\
477	0.0123278696733113\\
478	0.0123216232513006\\
479	0.0123152415491759\\
480	0.0123087195710508\\
481	0.0123020499839554\\
482	0.0122952229451359\\
483	0.0122882391449178\\
484	0.0122811035814963\\
485	0.0122738329847773\\
486	0.0122664179415157\\
487	0.0122588585366582\\
488	0.0122511326954372\\
489	0.0122432175139735\\
490	0.0122351059750949\\
491	0.0122267903369046\\
492	0.0122182598862883\\
493	0.0122094885089349\\
494	0.0122004822387657\\
495	0.0121912061038296\\
496	0.0121817025186965\\
497	0.0121719836708441\\
498	0.0121621011380786\\
499	0.0121519932315128\\
500	0.0121417056217666\\
501	0.012131141935661\\
502	0.0121202500288472\\
503	0.0121090082357356\\
504	0.0120973930403815\\
505	0.0120851934508869\\
506	0.0120723389791654\\
507	0.0120587588538445\\
508	0.0120443720487467\\
509	0.0120211361412554\\
510	0.0119832050072543\\
511	0.0119443784207802\\
512	0.0119045898585676\\
513	0.0118637640360684\\
514	0.0118218148014813\\
515	0.0117786462247138\\
516	0.0117341464586208\\
517	0.0116881902035418\\
518	0.0116406370936603\\
519	0.0115913253925226\\
520	0.0115400677765833\\
521	0.0114649020041444\\
522	0.011381940911945\\
523	0.0112978096631671\\
524	0.0112125015917795\\
525	0.0111260164534872\\
526	0.0110383597403553\\
527	0.0109495423004872\\
528	0.0108595777399665\\
529	0.0107684884570977\\
530	0.0106763072113456\\
531	0.0105830834829332\\
532	0.0105043775785049\\
533	0.0104671142982166\\
534	0.0104299383117761\\
535	0.0103929426892914\\
536	0.0103562355725284\\
537	0.0103199426752268\\
538	0.0102842102569721\\
539	0.0102492086725187\\
540	0.0102151366268864\\
541	0.0101822263814329\\
542	0.0101507501044137\\
543	0.010121027558569\\
544	0.0100915954141609\\
545	0.0100618746939995\\
546	0.0100318607180066\\
547	0.0100015838587952\\
548	0.00997100666511039\\
549	0.00994009336497352\\
550	0.00990881845987775\\
551	0.00987714631783721\\
552	0.0098450239144289\\
553	0.00981237947121874\\
554	0.00977911752878245\\
555	0.00974512451415165\\
556	0.00971037583367489\\
557	0.00967484493140955\\
558	0.00963850313244329\\
559	0.0096013195092261\\
560	0.00956326079011629\\
561	0.00952429134141864\\
562	0.00948437325964169\\
563	0.00944346663225815\\
564	0.00940153003079942\\
565	0.009358521353756\\
566	0.0093143991753416\\
567	0.00926912467092151\\
568	0.00922265706987281\\
569	0.00917495068242558\\
570	0.00912596001895447\\
571	0.00907563019759986\\
572	0.00902392558975723\\
573	0.0089708046766837\\
574	0.00891620968789234\\
575	0.00886003700472624\\
576	0.00880221738574126\\
577	0.00874278903473106\\
578	0.00868170440886518\\
579	0.00861883520774838\\
580	0.00855387888569596\\
581	0.0084870822310664\\
582	0.00841538272261392\\
583	0.00825264072841415\\
584	0.0080871163608675\\
585	0.00784265262501878\\
586	0.00748707723152733\\
587	0.00711097509202256\\
588	0.006842981126967\\
589	0.00669995455437386\\
590	0.00656262541880796\\
591	0.00644916937420928\\
592	0.00633088788857397\\
593	0.00620845766590564\\
594	0.0060710901180166\\
595	0.00588956027306329\\
596	0.00559184182825266\\
597	0.00498881296807123\\
598	0.00357511483354343\\
599	0\\
600	0\\
};
\addplot [color=blue!80!mycolor9,solid,forget plot]
  table[row sep=crcr]{%
1	0.0132347781023203\\
2	0.0132347758295092\\
3	0.0132347735178182\\
4	0.0132347711665798\\
5	0.0132347687751148\\
6	0.0132347663427328\\
7	0.0132347638687312\\
8	0.0132347613523954\\
9	0.0132347587929987\\
10	0.0132347561898018\\
11	0.0132347535420527\\
12	0.0132347508489866\\
13	0.0132347481098255\\
14	0.0132347453237781\\
15	0.0132347424900394\\
16	0.0132347396077909\\
17	0.0132347366761998\\
18	0.013234733694419\\
19	0.013234730661587\\
20	0.0132347275768276\\
21	0.0132347244392494\\
22	0.0132347212479457\\
23	0.0132347180019945\\
24	0.0132347147004578\\
25	0.0132347113423815\\
26	0.0132347079267953\\
27	0.0132347044527122\\
28	0.0132347009191282\\
29	0.0132346973250222\\
30	0.0132346936693556\\
31	0.013234689951072\\
32	0.0132346861690969\\
33	0.0132346823223373\\
34	0.0132346784096817\\
35	0.0132346744299992\\
36	0.01323467038214\\
37	0.0132346662649343\\
38	0.0132346620771923\\
39	0.0132346578177039\\
40	0.0132346534852381\\
41	0.0132346490785432\\
42	0.0132346445963457\\
43	0.0132346400373505\\
44	0.0132346354002403\\
45	0.0132346306836753\\
46	0.0132346258862925\\
47	0.013234621006706\\
48	0.0132346160435058\\
49	0.0132346109952581\\
50	0.0132346058605044\\
51	0.0132346006377612\\
52	0.0132345953255198\\
53	0.0132345899222455\\
54	0.0132345844263776\\
55	0.0132345788363284\\
56	0.0132345731504833\\
57	0.0132345673672001\\
58	0.0132345614848083\\
59	0.0132345555016091\\
60	0.0132345494158745\\
61	0.0132345432258471\\
62	0.0132345369297393\\
63	0.0132345305257332\\
64	0.0132345240119795\\
65	0.0132345173865976\\
66	0.0132345106476747\\
67	0.0132345037932653\\
68	0.0132344968213907\\
69	0.0132344897300383\\
70	0.0132344825171614\\
71	0.0132344751806782\\
72	0.0132344677184712\\
73	0.013234460128387\\
74	0.0132344524082354\\
75	0.0132344445557888\\
76	0.0132344365687815\\
77	0.0132344284449094\\
78	0.0132344201818288\\
79	0.0132344117771562\\
80	0.0132344032284675\\
81	0.0132343945332971\\
82	0.0132343856891375\\
83	0.0132343766934383\\
84	0.0132343675436058\\
85	0.0132343582370021\\
86	0.0132343487709442\\
87	0.0132343391427035\\
88	0.0132343293495048\\
89	0.0132343193885258\\
90	0.0132343092568961\\
91	0.0132342989516962\\
92	0.0132342884699572\\
93	0.0132342778086595\\
94	0.0132342669647321\\
95	0.0132342559350517\\
96	0.0132342447164419\\
97	0.0132342333056724\\
98	0.0132342216994577\\
99	0.0132342098944566\\
100	0.0132341978872709\\
101	0.0132341856744447\\
102	0.0132341732524635\\
103	0.0132341606177527\\
104	0.0132341477666773\\
105	0.0132341346955404\\
106	0.013234121400582\\
107	0.0132341078779787\\
108	0.0132340941238417\\
109	0.0132340801342165\\
110	0.0132340659050812\\
111	0.0132340514323457\\
112	0.0132340367118504\\
113	0.013234021739365\\
114	0.0132340065105876\\
115	0.0132339910211431\\
116	0.0132339752665822\\
117	0.0132339592423801\\
118	0.0132339429439351\\
119	0.0132339263665676\\
120	0.0132339095055186\\
121	0.0132338923559482\\
122	0.0132338749129344\\
123	0.013233857171472\\
124	0.0132338391264705\\
125	0.0132338207727534\\
126	0.0132338021050563\\
127	0.0132337831180255\\
128	0.0132337638062165\\
129	0.0132337441640926\\
130	0.0132337241860232\\
131	0.0132337038662822\\
132	0.0132336831990464\\
133	0.0132336621783939\\
134	0.0132336407983026\\
135	0.013233619052648\\
136	0.0132335969352022\\
137	0.0132335744396313\\
138	0.0132335515594945\\
139	0.0132335282882416\\
140	0.0132335046192117\\
141	0.013233480545631\\
142	0.0132334560606112\\
143	0.0132334311571472\\
144	0.0132334058281146\\
145	0.0132333800662666\\
146	0.0132333538642328\\
147	0.0132333272145206\\
148	0.0132333001095103\\
149	0.0132332725414534\\
150	0.0132332445024699\\
151	0.0132332159845462\\
152	0.0132331869795333\\
153	0.0132331574791437\\
154	0.0132331274749498\\
155	0.0132330969583812\\
156	0.0132330659207223\\
157	0.0132330343531097\\
158	0.0132330022465301\\
159	0.0132329695918174\\
160	0.0132329363796503\\
161	0.0132329026005495\\
162	0.0132328682448754\\
163	0.0132328333028247\\
164	0.0132327977644286\\
165	0.0132327616195489\\
166	0.0132327248578761\\
167	0.0132326874689259\\
168	0.0132326494420364\\
169	0.0132326107663652\\
170	0.0132325714308862\\
171	0.0132325314243865\\
172	0.0132324907354632\\
173	0.0132324493525206\\
174	0.013232407263766\\
175	0.0132323644572074\\
176	0.0132323209206495\\
177	0.0132322766416902\\
178	0.0132322316077175\\
179	0.0132321858059057\\
180	0.0132321392232115\\
181	0.0132320918463709\\
182	0.0132320436618949\\
183	0.0132319946560659\\
184	0.013231944814934\\
185	0.0132318941243125\\
186	0.0132318425697743\\
187	0.0132317901366479\\
188	0.0132317368100128\\
189	0.0132316825746955\\
190	0.0132316274152651\\
191	0.0132315713160292\\
192	0.0132315142610289\\
193	0.0132314562340347\\
194	0.0132313972185415\\
195	0.0132313371977643\\
196	0.0132312761546329\\
197	0.0132312140717877\\
198	0.0132311509315738\\
199	0.0132310867160367\\
200	0.0132310214069171\\
201	0.0132309549856451\\
202	0.0132308874333354\\
203	0.0132308187307815\\
204	0.0132307488584507\\
205	0.0132306777964777\\
206	0.0132306055246597\\
207	0.0132305320224497\\
208	0.0132304572689512\\
209	0.0132303812429122\\
210	0.0132303039227183\\
211	0.0132302252863873\\
212	0.0132301453115623\\
213	0.0132300639755051\\
214	0.0132299812550902\\
215	0.0132298971267972\\
216	0.0132298115667046\\
217	0.0132297245504827\\
218	0.013229636053386\\
219	0.0132295460502464\\
220	0.0132294545154658\\
221	0.0132293614230082\\
222	0.0132292667463924\\
223	0.0132291704586841\\
224	0.0132290725324875\\
225	0.0132289729399379\\
226	0.0132288716526929\\
227	0.0132287686419241\\
228	0.0132286638783086\\
229	0.01322855733202\\
230	0.0132284489727197\\
231	0.013228338769548\\
232	0.0132282266911141\\
233	0.0132281127054875\\
234	0.0132279967801877\\
235	0.0132278788821747\\
236	0.013227758977839\\
237	0.0132276370329913\\
238	0.0132275130128521\\
239	0.013227386882041\\
240	0.013227258604566\\
241	0.0132271281438124\\
242	0.0132269954625315\\
243	0.0132268605228293\\
244	0.0132267232861545\\
245	0.0132265837132867\\
246	0.0132264417643243\\
247	0.0132262973986719\\
248	0.0132261505750277\\
249	0.0132260012513705\\
250	0.0132258493849466\\
251	0.0132256949322561\\
252	0.0132255378490395\\
253	0.0132253780902632\\
254	0.0132252156101057\\
255	0.0132250503619426\\
256	0.0132248822983317\\
257	0.0132247113709979\\
258	0.0132245375308178\\
259	0.0132243607278034\\
260	0.0132241809110861\\
261	0.0132239980289003\\
262	0.0132238120285662\\
263	0.0132236228564729\\
264	0.0132234304580604\\
265	0.0132232347778017\\
266	0.0132230357591846\\
267	0.0132228333446926\\
268	0.0132226274757858\\
269	0.013222418092881\\
270	0.013222205135331\\
271	0.0132219885414033\\
272	0.0132217682482567\\
273	0.0132215441919161\\
274	0.0132213163072492\\
275	0.0132210845279557\\
276	0.0132208487865939\\
277	0.0132206090146007\\
278	0.0132203651420567\\
279	0.0132201170977708\\
280	0.0132198648092627\\
281	0.0132196082027371\\
282	0.0132193472030574\\
283	0.0132190817337192\\
284	0.0132188117168224\\
285	0.0132185370730435\\
286	0.0132182577216067\\
287	0.0132179735802545\\
288	0.0132176845652177\\
289	0.0132173905911842\\
290	0.013217091571268\\
291	0.0132167874169769\\
292	0.0132164780381788\\
293	0.0132161633430689\\
294	0.0132158432381344\\
295	0.0132155176281194\\
296	0.0132151864159887\\
297	0.013214849502891\\
298	0.0132145067881207\\
299	0.0132141581690795\\
300	0.0132138035412366\\
301	0.0132134427980889\\
302	0.0132130758311191\\
303	0.0132127025297543\\
304	0.0132123227813227\\
305	0.0132119364710111\\
306	0.0132115434818211\\
307	0.0132111436945272\\
308	0.0132107369876367\\
309	0.0132103232373546\\
310	0.0132099023175512\\
311	0.0132094740997167\\
312	0.013209038452846\\
313	0.0132085952431701\\
314	0.0132081443339324\\
315	0.0132076855861243\\
316	0.0132072188582528\\
317	0.0132067440061001\\
318	0.0132062608826734\\
319	0.0132057693381547\\
320	0.0132052692198509\\
321	0.0132047603721447\\
322	0.0132042426364462\\
323	0.0132037158511462\\
324	0.0132031798515709\\
325	0.013202634469938\\
326	0.0132020795353164\\
327	0.0132015148735874\\
328	0.0132009403074109\\
329	0.0132003556561944\\
330	0.013199760736068\\
331	0.0131991553598638\\
332	0.0131985393371037\\
333	0.013197912473993\\
334	0.0131972745734243\\
335	0.013196625434991\\
336	0.0131959648550124\\
337	0.0131952926265729\\
338	0.0131946085395756\\
339	0.0131939123808146\\
340	0.0131932039340663\\
341	0.0131924829802043\\
342	0.0131917492973398\\
343	0.0131910026609915\\
344	0.01319024284429\\
345	0.0131894696182193\\
346	0.0131886827519027\\
347	0.0131878820129377\\
348	0.0131870671677877\\
349	0.0131862379822367\\
350	0.0131853942219163\\
351	0.0131845356529115\\
352	0.0131836620424491\\
353	0.0131827731596655\\
354	0.0131818687764362\\
355	0.0131809486682588\\
356	0.0131800126153635\\
357	0.0131790604049756\\
358	0.0131780918361405\\
359	0.013177106720808\\
360	0.0131761048789324\\
361	0.0131750861461626\\
362	0.0131740503778041\\
363	0.0131729974552272\\
364	0.01317192729905\\
365	0.0131708399031935\\
366	0.0131697354388771\\
367	0.0131686146244958\\
368	0.013167480455131\\
369	0.0131665168884495\\
370	0.0131655504787687\\
371	0.0131645672718775\\
372	0.0131635669259919\\
373	0.013162549089619\\
374	0.0131615134009261\\
375	0.0131604594860665\\
376	0.0131593869534426\\
377	0.0131582953737402\\
378	0.0131571842210047\\
379	0.0131560528341121\\
380	0.0131549011737056\\
381	0.0131537290664285\\
382	0.013152536054818\\
383	0.0131513216642898\\
384	0.0131500854019684\\
385	0.0131488267553962\\
386	0.0131475451911057\\
387	0.0131462401530352\\
388	0.0131449110607665\\
389	0.0131435573075603\\
390	0.013142178258158\\
391	0.0131407732463172\\
392	0.0131393415720401\\
393	0.0131378824984483\\
394	0.0131363952482487\\
395	0.0131348789997277\\
396	0.0131333328822022\\
397	0.0131317559708472\\
398	0.0131301472808314\\
399	0.0131285057607295\\
400	0.0131268302853374\\
401	0.0131251196484726\\
402	0.0131233725576429\\
403	0.0131215876360022\\
404	0.0131197634462196\\
405	0.0131178985721574\\
406	0.0131159918130475\\
407	0.0131140420985683\\
408	0.013112044706827\\
409	0.013109996486079\\
410	0.0131078948063959\\
411	0.013105736880694\\
412	0.0131035200203141\\
413	0.0131012374186524\\
414	0.0130988728798661\\
415	0.0130955680561331\\
416	0.0130917248225261\\
417	0.0130878146556782\\
418	0.0130838363593167\\
419	0.0130797887193233\\
420	0.013075670511126\\
421	0.0130714805151593\\
422	0.0130672175342066\\
423	0.0130628803382718\\
424	0.013058467354028\\
425	0.0130539768918502\\
426	0.0130494075817094\\
427	0.0130447580314616\\
428	0.0130400268268571\\
429	0.0130352125316226\\
430	0.0130303136876508\\
431	0.0130253288152923\\
432	0.0130202564137235\\
433	0.0130150949614558\\
434	0.0130098429170114\\
435	0.0130044987197909\\
436	0.0129990607911719\\
437	0.0129935275358878\\
438	0.0129878973437463\\
439	0.0129821685917298\\
440	0.0129763396464148\\
441	0.0129704088664562\\
442	0.012964374604684\\
443	0.0129582352067792\\
444	0.0129519889928034\\
445	0.0129456341636788\\
446	0.0129391683426349\\
447	0.0129325858670673\\
448	0.0129258909366011\\
449	0.0129190806169245\\
450	0.0129121567789233\\
451	0.0129051193176962\\
452	0.0128979670257757\\
453	0.0128906988273655\\
454	0.0128833138909198\\
455	0.012875811934488\\
456	0.0128681941311641\\
457	0.0128604659923326\\
458	0.0128526479478858\\
459	0.0128463432843658\\
460	0.0128402288433069\\
461	0.0128339852225642\\
462	0.0128276133713435\\
463	0.0128210948494779\\
464	0.01281441344157\\
465	0.0128075590747409\\
466	0.0128005205059016\\
467	0.0127932853917727\\
468	0.0127858339892735\\
469	0.0127781449890987\\
470	0.0127701016006216\\
471	0.0127582393463909\\
472	0.0127461733117576\\
473	0.0127338998980457\\
474	0.0127214152470802\\
475	0.0127087155480823\\
476	0.0126957968660497\\
477	0.0126826553162722\\
478	0.0126692866612546\\
479	0.0126556855465089\\
480	0.0126418427421058\\
481	0.0126277562256423\\
482	0.0126134338845041\\
483	0.0125988738737036\\
484	0.0125840749030257\\
485	0.01256903385509\\
486	0.0125537489084793\\
487	0.0125382194244606\\
488	0.0125224523292342\\
489	0.0125064490906939\\
490	0.0124901683099635\\
491	0.0124735867336545\\
492	0.0124566958023023\\
493	0.0124394826754598\\
494	0.0124219466772415\\
495	0.0124040745937468\\
496	0.0123858858916424\\
497	0.0123673886652956\\
498	0.0123486014693713\\
499	0.0123295029307065\\
500	0.0123101108101237\\
501	0.0122903853911431\\
502	0.0122703136627916\\
503	0.0122499026135327\\
504	0.0122291898553464\\
505	0.0122182341084881\\
506	0.0122075537815293\\
507	0.0121965778959416\\
508	0.0121853129971356\\
509	0.0121736477797267\\
510	0.012161602506892\\
511	0.0121491619363014\\
512	0.0121363035640099\\
513	0.0121230173672913\\
514	0.0121093475310175\\
515	0.0120951592706932\\
516	0.0120804790687188\\
517	0.0120652203880827\\
518	0.0120492328688599\\
519	0.0120324415501122\\
520	0.0120147603405112\\
521	0.011977952271915\\
522	0.0119348584053488\\
523	0.0118907851469197\\
524	0.0118456684488627\\
525	0.0117994378583295\\
526	0.0117520144632674\\
527	0.01170330891857\\
528	0.0116532180701453\\
529	0.0116016248096776\\
530	0.0115483953376152\\
531	0.011493093288488\\
532	0.0114273515266461\\
533	0.0113373290609614\\
534	0.0112459131750253\\
535	0.0111530808453629\\
536	0.0110588120087396\\
537	0.0109630906304736\\
538	0.0108659061000613\\
539	0.0107672551111521\\
540	0.0106671440848492\\
541	0.0105655906119601\\
542	0.0104626264988062\\
543	0.0103583043008088\\
544	0.0102980335130526\\
545	0.0102537894756843\\
546	0.01020948323368\\
547	0.0101652120206003\\
548	0.0101210898973623\\
549	0.0100772506471942\\
550	0.0100339538313825\\
551	0.00999133077126591\\
552	0.00994952019604942\\
553	0.00990879691879962\\
554	0.0098694885703536\\
555	0.00983193644346207\\
556	0.00979382750513552\\
557	0.00975515995707443\\
558	0.00971592989186324\\
559	0.0096761302407511\\
560	0.00963574938160194\\
561	0.00959476919138996\\
562	0.00955316263743115\\
563	0.00951089063585328\\
564	0.00946789796833104\\
565	0.00942410799241927\\
566	0.00937941580458399\\
567	0.0093336795217075\\
568	0.00928685833691943\\
569	0.00923891336036805\\
570	0.00918980277362079\\
571	0.00913948161419208\\
572	0.00908790133297858\\
573	0.00903500913785611\\
574	0.00898074783306652\\
575	0.00892505467335488\\
576	0.0088678514388995\\
577	0.00880907800309159\\
578	0.00874867907362212\\
579	0.00868655222200205\\
580	0.00862260334329572\\
581	0.00855684893553334\\
582	0.00848916944141833\\
583	0.00841938629888952\\
584	0.008347256035793\\
585	0.00823589926147293\\
586	0.0080735141024222\\
587	0.00790696828781916\\
588	0.00764918419075849\\
589	0.0072961566882175\\
590	0.0069260267358603\\
591	0.00657947678076211\\
592	0.00642061328358282\\
593	0.00624640749833555\\
594	0.00607883230598045\\
595	0.00588956027306329\\
596	0.00559184182825266\\
597	0.00498881296807123\\
598	0.00357511483354343\\
599	0\\
600	0\\
};
\addplot [color=blue,solid,forget plot]
  table[row sep=crcr]{%
1	0.013414928077931\\
2	0.0134149270175419\\
3	0.0134149259390028\\
4	0.0134149248420022\\
5	0.0134149237262232\\
6	0.0134149225913437\\
7	0.0134149214370357\\
8	0.013414920262966\\
9	0.0134149190687953\\
10	0.0134149178541788\\
11	0.0134149166187655\\
12	0.0134149153621987\\
13	0.0134149140841154\\
14	0.0134149127841463\\
15	0.0134149114619161\\
16	0.0134149101170427\\
17	0.0134149087491376\\
18	0.0134149073578057\\
19	0.0134149059426452\\
20	0.0134149045032472\\
21	0.013414903039196\\
22	0.0134149015500686\\
23	0.0134149000354349\\
24	0.0134148984948575\\
25	0.0134148969278912\\
26	0.0134148953340836\\
27	0.0134148937129743\\
28	0.013414892064095\\
29	0.0134148903869696\\
30	0.0134148886811136\\
31	0.0134148869460343\\
32	0.0134148851812307\\
33	0.0134148833861931\\
34	0.0134148815604031\\
35	0.0134148797033334\\
36	0.0134148778144479\\
37	0.013414875893201\\
38	0.013414873939038\\
39	0.0134148719513947\\
40	0.0134148699296971\\
41	0.0134148678733617\\
42	0.0134148657817947\\
43	0.0134148636543923\\
44	0.0134148614905404\\
45	0.0134148592896143\\
46	0.0134148570509788\\
47	0.0134148547739878\\
48	0.013414852457984\\
49	0.0134148501022991\\
50	0.0134148477062533\\
51	0.013414845269155\\
52	0.0134148427903012\\
53	0.0134148402689766\\
54	0.0134148377044538\\
55	0.0134148350959929\\
56	0.0134148324428414\\
57	0.0134148297442341\\
58	0.0134148269993926\\
59	0.0134148242075253\\
60	0.0134148213678269\\
61	0.0134148184794787\\
62	0.0134148155416478\\
63	0.0134148125534871\\
64	0.0134148095141353\\
65	0.0134148064227161\\
66	0.0134148032783385\\
67	0.0134148000800962\\
68	0.0134147968270675\\
69	0.0134147935183149\\
70	0.0134147901528851\\
71	0.0134147867298085\\
72	0.0134147832480989\\
73	0.0134147797067535\\
74	0.0134147761047521\\
75	0.0134147724410576\\
76	0.0134147687146148\\
77	0.0134147649243507\\
78	0.0134147610691743\\
79	0.0134147571479756\\
80	0.013414753159626\\
81	0.0134147491029777\\
82	0.0134147449768633\\
83	0.0134147407800955\\
84	0.0134147365114669\\
85	0.0134147321697497\\
86	0.0134147277536949\\
87	0.0134147232620327\\
88	0.0134147186934712\\
89	0.0134147140466971\\
90	0.0134147093203745\\
91	0.0134147045131448\\
92	0.0134146996236264\\
93	0.0134146946504142\\
94	0.0134146895920793\\
95	0.0134146844471685\\
96	0.013414679214204\\
97	0.0134146738916828\\
98	0.0134146684780766\\
99	0.013414662971831\\
100	0.0134146573713652\\
101	0.0134146516750717\\
102	0.0134146458813158\\
103	0.013414639988435\\
104	0.0134146339947384\\
105	0.0134146278985069\\
106	0.0134146216979918\\
107	0.0134146153914152\\
108	0.0134146089769685\\
109	0.0134146024528131\\
110	0.0134145958170788\\
111	0.0134145890678639\\
112	0.0134145822032344\\
113	0.0134145752212236\\
114	0.0134145681198316\\
115	0.0134145608970243\\
116	0.0134145535507335\\
117	0.0134145460788558\\
118	0.0134145384792523\\
119	0.0134145307497478\\
120	0.0134145228881303\\
121	0.0134145148921504\\
122	0.0134145067595206\\
123	0.0134144984879146\\
124	0.013414490074967\\
125	0.013414481518272\\
126	0.0134144728153834\\
127	0.0134144639638135\\
128	0.0134144549610325\\
129	0.0134144458044677\\
130	0.0134144364915032\\
131	0.0134144270194785\\
132	0.0134144173856882\\
133	0.0134144075873813\\
134	0.0134143976217601\\
135	0.0134143874859796\\
136	0.0134143771771468\\
137	0.0134143666923196\\
138	0.0134143560285064\\
139	0.0134143451826647\\
140	0.013414334151701\\
141	0.0134143229324692\\
142	0.0134143115217701\\
143	0.0134142999163502\\
144	0.0134142881129009\\
145	0.0134142761080575\\
146	0.013414263898399\\
147	0.0134142514804463\\
148	0.0134142388506616\\
149	0.0134142260054473\\
150	0.0134142129411449\\
151	0.0134141996540344\\
152	0.0134141861403325\\
153	0.0134141723961921\\
154	0.0134141584177013\\
155	0.0134141442008818\\
156	0.0134141297416882\\
157	0.0134141150360067\\
158	0.013414100079654\\
159	0.0134140848683761\\
160	0.0134140693978471\\
161	0.0134140536636681\\
162	0.0134140376613657\\
163	0.013414021386391\\
164	0.0134140048341181\\
165	0.013413987999843\\
166	0.0134139708787824\\
167	0.0134139534660718\\
168	0.0134139357567647\\
169	0.013413917745831\\
170	0.0134138994281554\\
171	0.0134138807985364\\
172	0.0134138618516842\\
173	0.01341384258222\\
174	0.0134138229846737\\
175	0.013413803053483\\
176	0.0134137827829913\\
177	0.0134137621674464\\
178	0.013413741200999\\
179	0.0134137198777005\\
180	0.0134136981915021\\
181	0.0134136761362524\\
182	0.0134136537056959\\
183	0.0134136308934714\\
184	0.01341360769311\\
185	0.0134135840980331\\
186	0.0134135601015509\\
187	0.0134135356968605\\
188	0.0134135108770435\\
189	0.0134134856350643\\
190	0.0134134599637685\\
191	0.0134134338558799\\
192	0.0134134073039996\\
193	0.0134133803006027\\
194	0.0134133528380371\\
195	0.0134133249085209\\
196	0.0134132965041399\\
197	0.013413267616846\\
198	0.0134132382384542\\
199	0.0134132083606407\\
200	0.0134131779749403\\
201	0.0134131470727441\\
202	0.0134131156452968\\
203	0.0134130836836943\\
204	0.0134130511788811\\
205	0.0134130181216477\\
206	0.0134129845026279\\
207	0.013412950312296\\
208	0.0134129155409642\\
209	0.0134128801787796\\
210	0.0134128442157214\\
211	0.013412807641598\\
212	0.0134127704460437\\
213	0.0134127326185163\\
214	0.0134126941482932\\
215	0.013412655024469\\
216	0.0134126152359517\\
217	0.0134125747714596\\
218	0.0134125336195182\\
219	0.0134124917684562\\
220	0.0134124492064028\\
221	0.0134124059212835\\
222	0.0134123619008166\\
223	0.01341231713251\\
224	0.0134122716036567\\
225	0.0134122253013316\\
226	0.0134121782123873\\
227	0.01341213032345\\
228	0.0134120816209157\\
229	0.0134120320909459\\
230	0.0134119817194636\\
231	0.0134119304921485\\
232	0.0134118783944332\\
233	0.0134118254114981\\
234	0.0134117715282674\\
235	0.0134117167294041\\
236	0.0134116609993051\\
237	0.0134116043220968\\
238	0.0134115466816296\\
239	0.0134114880614734\\
240	0.0134114284449117\\
241	0.013411367814937\\
242	0.0134113061542451\\
243	0.0134112434452295\\
244	0.013411179669976\\
245	0.0134111148102567\\
246	0.0134110488475242\\
247	0.013410981762906\\
248	0.0134109135371976\\
249	0.0134108441508569\\
250	0.0134107735839975\\
251	0.0134107018163823\\
252	0.0134106288274165\\
253	0.0134105545961411\\
254	0.0134104791012259\\
255	0.0134104023209621\\
256	0.0134103242332551\\
257	0.0134102448156171\\
258	0.0134101640451593\\
259	0.0134100818985841\\
260	0.0134099983521772\\
261	0.0134099133817993\\
262	0.0134098269628775\\
263	0.0134097390703971\\
264	0.0134096496788924\\
265	0.0134095587624378\\
266	0.0134094662946385\\
267	0.0134093722486211\\
268	0.0134092765970235\\
269	0.0134091793119852\\
270	0.0134090803651367\\
271	0.0134089797275882\\
272	0.0134088773699192\\
273	0.0134087732621672\\
274	0.0134086673738195\\
275	0.0134085596738074\\
276	0.013408450130494\\
277	0.0134083387116404\\
278	0.0134082253844058\\
279	0.0134081101153357\\
280	0.0134079928703478\\
281	0.0134078736147188\\
282	0.0134077523130701\\
283	0.0134076289293528\\
284	0.0134075034268335\\
285	0.013407375768078\\
286	0.0134072459149358\\
287	0.0134071138285235\\
288	0.0134069794692081\\
289	0.0134068427965894\\
290	0.0134067037694818\\
291	0.0134065623458963\\
292	0.0134064184830208\\
293	0.0134062721372005\\
294	0.0134061232639177\\
295	0.0134059718177704\\
296	0.0134058177524507\\
297	0.0134056610207223\\
298	0.0134055015743972\\
299	0.0134053393643115\\
300	0.0134051743403007\\
301	0.013405006451174\\
302	0.0134048356446874\\
303	0.0134046618675164\\
304	0.0134044850652274\\
305	0.0134043051822483\\
306	0.0134041221618388\\
307	0.0134039359460592\\
308	0.0134037464757396\\
309	0.0134035536904462\\
310	0.0134033575284439\\
311	0.0134031579266446\\
312	0.0134029548205422\\
313	0.0134027481441729\\
314	0.0134025378301587\\
315	0.0134023238096377\\
316	0.0134021060121973\\
317	0.0134018843658272\\
318	0.0134016587968699\\
319	0.0134014292299697\\
320	0.013401195588019\\
321	0.013400957792102\\
322	0.0134007157614362\\
323	0.0134004694133106\\
324	0.0134002186630216\\
325	0.0133999634238046\\
326	0.0133997036067635\\
327	0.0133994391207952\\
328	0.0133991698725107\\
329	0.0133988957661521\\
330	0.0133986167035043\\
331	0.0133983325838023\\
332	0.0133980433036318\\
333	0.0133977487568256\\
334	0.0133974488343512\\
335	0.0133971434241934\\
336	0.0133968324112274\\
337	0.0133965156770837\\
338	0.0133961931000041\\
339	0.0133958645546863\\
340	0.0133955299121172\\
341	0.0133951890393943\\
342	0.0133948417995313\\
343	0.0133944880512496\\
344	0.0133941276487516\\
345	0.0133937604414753\\
346	0.0133933862738272\\
347	0.0133930049848914\\
348	0.0133926164081135\\
349	0.0133922203709535\\
350	0.0133918166945069\\
351	0.0133914051930872\\
352	0.0133909856737642\\
353	0.0133905579358524\\
354	0.0133901217703497\\
355	0.0133896769593673\\
356	0.0133892232756554\\
357	0.0133887604822079\\
358	0.0133882883312323\\
359	0.0133878065626549\\
360	0.0133873149038293\\
361	0.0133868130687519\\
362	0.0133863007575523\\
363	0.0133857776569514\\
364	0.0133852434437845\\
365	0.013384697797695\\
366	0.0133841404379945\\
367	0.013383571187586\\
368	0.0133829895015238\\
369	0.0133822925247362\\
370	0.0133815762058525\\
371	0.013380847870976\\
372	0.0133801073094401\\
373	0.0133793543065388\\
374	0.0133785886430582\\
375	0.013377810093949\\
376	0.0133770184250384\\
377	0.0133762133884424\\
378	0.013375394734917\\
379	0.0133745622727953\\
380	0.0133737157824122\\
381	0.0133728550086828\\
382	0.0133719796906904\\
383	0.0133710895614253\\
384	0.0133701843475\\
385	0.0133692637688379\\
386	0.0133683275383337\\
387	0.0133673753614823\\
388	0.0133664069359745\\
389	0.0133654219512558\\
390	0.0133644200880473\\
391	0.0133634010178258\\
392	0.0133623644022626\\
393	0.0133613098926188\\
394	0.0133602371291007\\
395	0.0133591457401762\\
396	0.013358035341861\\
397	0.0133569055369912\\
398	0.0133557559145165\\
399	0.0133545860488946\\
400	0.0133533954997756\\
401	0.013352183812435\\
402	0.0133509505200123\\
403	0.0133496951496614\\
404	0.0133484172347461\\
405	0.0133471163227695\\
406	0.0133457918972152\\
407	0.0133444431307396\\
408	0.0133430694071201\\
409	0.0133416701955969\\
410	0.0133402449973673\\
411	0.0133387933723088\\
412	0.0133373146689459\\
413	0.0133358062911943\\
414	0.0133342644378274\\
415	0.01333269178558\\
416	0.0133310898335028\\
417	0.0133294579161511\\
418	0.0133277953474446\\
419	0.0133261014209448\\
420	0.0133243754105315\\
421	0.0133226165686372\\
422	0.013320824112104\\
423	0.0133189971931415\\
424	0.0133171349303551\\
425	0.0133152364461519\\
426	0.0133133008294374\\
427	0.0133113271336002\\
428	0.0133093143743227\\
429	0.013307261527198\\
430	0.0133051675251265\\
431	0.0133030312554618\\
432	0.0133008515568784\\
433	0.0132986272159265\\
434	0.013296356963232\\
435	0.0132940394692921\\
436	0.0132916733398147\\
437	0.0132892571105308\\
438	0.013286789241395\\
439	0.0132842681100409\\
440	0.0132816920042748\\
441	0.0132790591131363\\
442	0.0132763675150305\\
443	0.0132736151577758\\
444	0.0132707998127569\\
445	0.0132679189455976\\
446	0.0132649693828738\\
447	0.0132619473679989\\
448	0.0132588529433789\\
449	0.0132556835118719\\
450	0.0132524358996166\\
451	0.0132491065188316\\
452	0.0132456914464316\\
453	0.013242186386854\\
454	0.013238586649609\\
455	0.0132348871720032\\
456	0.0132310826168241\\
457	0.0132271672621339\\
458	0.013223130999815\\
459	0.0132180806965968\\
460	0.0132127388466801\\
461	0.0132072770582642\\
462	0.013201692785451\\
463	0.0131959752525703\\
464	0.0131901152721448\\
465	0.0131841051296938\\
466	0.0131779407292673\\
467	0.0131716244691192\\
468	0.0131654521055893\\
469	0.0131592267559905\\
470	0.0131527181562933\\
471	0.0131431202552702\\
472	0.0131333539943084\\
473	0.0131234151909817\\
474	0.0131132995360054\\
475	0.0131030026145868\\
476	0.0130925200630023\\
477	0.0130818481749884\\
478	0.013070983554991\\
479	0.0130599231643694\\
480	0.0130486448834303\\
481	0.0130371458135307\\
482	0.0130254259153766\\
483	0.0130134805420878\\
484	0.0130013048442184\\
485	0.0129888941927184\\
486	0.0129762447744911\\
487	0.0129633563459099\\
488	0.0129502438180365\\
489	0.0129379219495849\\
490	0.0129258140433674\\
491	0.0129134477691533\\
492	0.0129008132372538\\
493	0.012887900786866\\
494	0.0128746993445672\\
495	0.0128611992433007\\
496	0.0128473889037068\\
497	0.0128332557361546\\
498	0.0128187832680602\\
499	0.012803957731058\\
500	0.0127887667988419\\
501	0.0127731878736328\\
502	0.0127571691255732\\
503	0.0127406813699044\\
504	0.0127236798947353\\
505	0.012700707185963\\
506	0.0126769958202807\\
507	0.0126528336782187\\
508	0.0126293103073263\\
509	0.0126084183583063\\
510	0.0125870971357296\\
511	0.0125653406545835\\
512	0.0125431424245305\\
513	0.0125205059407741\\
514	0.0124974491568105\\
515	0.0124739309632564\\
516	0.0124500018553647\\
517	0.0124255678586512\\
518	0.0124005546324037\\
519	0.0123749475495255\\
520	0.0123487218301903\\
521	0.012321853105402\\
522	0.0122943109761287\\
523	0.0122660582332452\\
524	0.0122370489072429\\
525	0.0122071786946597\\
526	0.0121764208705988\\
527	0.0121447461470646\\
528	0.0121121607906459\\
529	0.0120786276241799\\
530	0.0120441829525235\\
531	0.012023255979046\\
532	0.0119949957240226\\
533	0.0119474905219952\\
534	0.0118988839602698\\
535	0.011849078594133\\
536	0.0117979877137699\\
537	0.011745513356831\\
538	0.0116915443412781\\
539	0.0116359539156808\\
540	0.0115785968561852\\
541	0.0115192449372247\\
542	0.0114576605551375\\
543	0.0113936126473879\\
544	0.0113031030936526\\
545	0.0112020883418527\\
546	0.0110994968145173\\
547	0.0109953160618321\\
548	0.0108895405722017\\
549	0.0107821735232877\\
550	0.0106732284943602\\
551	0.0105627303086983\\
552	0.0104506429843039\\
553	0.0103366487398546\\
554	0.0102207496829703\\
555	0.0101043963541868\\
556	0.0100509336587227\\
557	0.00999693072323994\\
558	0.00994246687712072\\
559	0.0098876392311834\\
560	0.00983254675135841\\
561	0.00977733583477016\\
562	0.00972218376679349\\
563	0.00966730481245616\\
564	0.00961295804394078\\
565	0.00955945714285719\\
566	0.00950718321543156\\
567	0.00945659810990165\\
568	0.0094052073164819\\
569	0.00935288772157468\\
570	0.00929962135438986\\
571	0.00924538513148849\\
572	0.00919015291893353\\
573	0.00913390631613418\\
574	0.0090766196163146\\
575	0.00901825650753089\\
576	0.00895876561842256\\
577	0.00889807436218399\\
578	0.00883608062306942\\
579	0.00877264139895938\\
580	0.00870759893646839\\
581	0.00864089830205915\\
582	0.00857246413507161\\
583	0.00850225076988874\\
584	0.00843013586447\\
585	0.00835611259214755\\
586	0.0082801197384134\\
587	0.00820181731099155\\
588	0.0080795783656128\\
589	0.00791554451502019\\
590	0.00774951452924046\\
591	0.0075392339828192\\
592	0.00718230859247705\\
593	0.00679628126808101\\
594	0.00631581315310309\\
595	0.00593854324538158\\
596	0.00559184182825266\\
597	0.00498881296807123\\
598	0.00357511483354343\\
599	0\\
600	0\\
};
\addplot [color=mycolor10,solid,forget plot]
  table[row sep=crcr]{%
1	0.0134827010444315\\
2	0.0134827009245038\\
3	0.0134827008025232\\
4	0.0134827006784545\\
5	0.013482700552262\\
6	0.0134827004239092\\
7	0.0134827002933591\\
8	0.0134827001605738\\
9	0.0134827000255151\\
10	0.0134826998881441\\
11	0.0134826997484209\\
12	0.0134826996063053\\
13	0.0134826994617562\\
14	0.013482699314732\\
15	0.01348269916519\\
16	0.0134826990130872\\
17	0.0134826988583797\\
18	0.0134826987010227\\
19	0.0134826985409708\\
20	0.0134826983781778\\
21	0.0134826982125967\\
22	0.0134826980441797\\
23	0.0134826978728782\\
24	0.0134826976986426\\
25	0.0134826975214228\\
26	0.0134826973411675\\
27	0.0134826971578247\\
28	0.0134826969713414\\
29	0.0134826967816639\\
30	0.0134826965887373\\
31	0.013482696392506\\
32	0.0134826961929134\\
33	0.0134826959899017\\
34	0.0134826957834125\\
35	0.0134826955733862\\
36	0.013482695359762\\
37	0.0134826951424784\\
38	0.0134826949214727\\
39	0.0134826946966811\\
40	0.0134826944680387\\
41	0.0134826942354795\\
42	0.0134826939989366\\
43	0.0134826937583415\\
44	0.013482693513625\\
45	0.0134826932647163\\
46	0.0134826930115438\\
47	0.0134826927540344\\
48	0.0134826924921138\\
49	0.0134826922257066\\
50	0.0134826919547358\\
51	0.0134826916791233\\
52	0.0134826913987897\\
53	0.0134826911136542\\
54	0.0134826908236346\\
55	0.0134826905286472\\
56	0.0134826902286071\\
57	0.0134826899234277\\
58	0.0134826896130211\\
59	0.0134826892972979\\
60	0.0134826889761672\\
61	0.0134826886495363\\
62	0.0134826883173112\\
63	0.0134826879793962\\
64	0.013482687635694\\
65	0.0134826872861056\\
66	0.0134826869305304\\
67	0.0134826865688659\\
68	0.013482686201008\\
69	0.0134826858268508\\
70	0.0134826854462867\\
71	0.0134826850592061\\
72	0.0134826846654977\\
73	0.0134826842650481\\
74	0.0134826838577421\\
75	0.0134826834434626\\
76	0.0134826830220905\\
77	0.0134826825935044\\
78	0.0134826821575813\\
79	0.0134826817141958\\
80	0.0134826812632203\\
81	0.0134826808045253\\
82	0.013482680337979\\
83	0.0134826798634472\\
84	0.0134826793807937\\
85	0.0134826788898798\\
86	0.0134826783905644\\
87	0.0134826778827042\\
88	0.0134826773661534\\
89	0.0134826768407636\\
90	0.013482676306384\\
91	0.0134826757628613\\
92	0.0134826752100394\\
93	0.0134826746477598\\
94	0.0134826740758611\\
95	0.0134826734941793\\
96	0.0134826729025476\\
97	0.0134826723007963\\
98	0.0134826716887529\\
99	0.0134826710662419\\
100	0.013482670433085\\
101	0.0134826697891007\\
102	0.0134826691341046\\
103	0.0134826684679089\\
104	0.0134826677903229\\
105	0.0134826671011526\\
106	0.0134826664002006\\
107	0.0134826656872663\\
108	0.0134826649621457\\
109	0.0134826642246313\\
110	0.013482663474512\\
111	0.0134826627115733\\
112	0.013482661935597\\
113	0.0134826611463613\\
114	0.0134826603436404\\
115	0.013482659527205\\
116	0.0134826586968217\\
117	0.0134826578522533\\
118	0.0134826569932585\\
119	0.013482656119592\\
120	0.0134826552310042\\
121	0.0134826543272414\\
122	0.0134826534080458\\
123	0.0134826524731549\\
124	0.0134826515223019\\
125	0.0134826505552157\\
126	0.0134826495716203\\
127	0.0134826485712352\\
128	0.0134826475537752\\
129	0.0134826465189503\\
130	0.0134826454664654\\
131	0.0134826443960207\\
132	0.0134826433073112\\
133	0.0134826422000268\\
134	0.0134826410738521\\
135	0.0134826399284663\\
136	0.0134826387635436\\
137	0.0134826375787521\\
138	0.0134826363737549\\
139	0.0134826351482091\\
140	0.0134826339017659\\
141	0.013482632634071\\
142	0.0134826313447638\\
143	0.0134826300334778\\
144	0.01348262869984\\
145	0.0134826273434716\\
146	0.013482625963987\\
147	0.0134826245609945\\
148	0.0134826231340954\\
149	0.0134826216828846\\
150	0.01348262020695\\
151	0.0134826187058726\\
152	0.0134826171792264\\
153	0.0134826156265782\\
154	0.0134826140474875\\
155	0.0134826124415063\\
156	0.0134826108081791\\
157	0.0134826091470429\\
158	0.0134826074576267\\
159	0.0134826057394516\\
160	0.0134826039920307\\
161	0.0134826022148689\\
162	0.0134826004074627\\
163	0.0134825985693002\\
164	0.0134825966998609\\
165	0.0134825947986154\\
166	0.0134825928650256\\
167	0.0134825908985442\\
168	0.0134825888986146\\
169	0.0134825868646711\\
170	0.0134825847961383\\
171	0.0134825826924311\\
172	0.0134825805529546\\
173	0.013482578377104\\
174	0.0134825761642641\\
175	0.0134825739138095\\
176	0.0134825716251043\\
177	0.0134825692975017\\
178	0.0134825669303443\\
179	0.0134825645229634\\
180	0.0134825620746792\\
181	0.0134825595848003\\
182	0.0134825570526237\\
183	0.0134825544774348\\
184	0.0134825518585066\\
185	0.0134825491951002\\
186	0.013482546486464\\
187	0.0134825437318339\\
188	0.0134825409304328\\
189	0.0134825380814707\\
190	0.013482535184144\\
191	0.013482532237636\\
192	0.0134825292411157\\
193	0.0134825261937386\\
194	0.0134825230946456\\
195	0.0134825199429633\\
196	0.0134825167378035\\
197	0.0134825134782631\\
198	0.0134825101634237\\
199	0.0134825067923515\\
200	0.0134825033640968\\
201	0.0134824998776939\\
202	0.0134824963321608\\
203	0.0134824927264992\\
204	0.0134824890596934\\
205	0.013482485330711\\
206	0.0134824815385019\\
207	0.0134824776819985\\
208	0.0134824737601147\\
209	0.0134824697717465\\
210	0.013482465715771\\
211	0.0134824615910463\\
212	0.0134824573964112\\
213	0.0134824531306847\\
214	0.013482448792666\\
215	0.0134824443811338\\
216	0.0134824398948461\\
217	0.0134824353325398\\
218	0.0134824306929305\\
219	0.0134824259747118\\
220	0.0134824211765551\\
221	0.0134824162971094\\
222	0.0134824113350005\\
223	0.013482406288831\\
224	0.0134824011571795\\
225	0.0134823959386006\\
226	0.0134823906316239\\
227	0.0134823852347544\\
228	0.0134823797464711\\
229	0.0134823741652273\\
230	0.0134823684894497\\
231	0.0134823627175382\\
232	0.0134823568478652\\
233	0.0134823508787753\\
234	0.0134823448085845\\
235	0.0134823386355801\\
236	0.0134823323580199\\
237	0.0134823259741318\\
238	0.013482319482113\\
239	0.0134823128801298\\
240	0.0134823061663169\\
241	0.0134822993387766\\
242	0.0134822923955787\\
243	0.0134822853347593\\
244	0.0134822781543207\\
245	0.0134822708522304\\
246	0.0134822634264207\\
247	0.0134822558747879\\
248	0.0134822481951918\\
249	0.0134822403854547\\
250	0.0134822324433612\\
251	0.0134822243666569\\
252	0.013482216153048\\
253	0.0134822078002005\\
254	0.0134821993057396\\
255	0.0134821906672486\\
256	0.0134821818822682\\
257	0.0134821729482957\\
258	0.0134821638627844\\
259	0.0134821546231423\\
260	0.0134821452267315\\
261	0.0134821356708671\\
262	0.0134821259528166\\
263	0.0134821160697987\\
264	0.0134821060189821\\
265	0.0134820957974851\\
266	0.013482085402374\\
267	0.0134820748306624\\
268	0.0134820640793096\\
269	0.0134820531452202\\
270	0.0134820420252421\\
271	0.0134820307161657\\
272	0.013482019214723\\
273	0.0134820075175864\\
274	0.0134819956213686\\
275	0.0134819835226207\\
276	0.0134819712178281\\
277	0.0134819587034108\\
278	0.0134819459757223\\
279	0.0134819330310478\\
280	0.0134819198656028\\
281	0.0134819064755315\\
282	0.0134818928569048\\
283	0.0134818790057194\\
284	0.013481864917895\\
285	0.0134818505892731\\
286	0.0134818360156151\\
287	0.0134818211925997\\
288	0.0134818061158217\\
289	0.0134817907807892\\
290	0.0134817751829214\\
291	0.0134817593175467\\
292	0.0134817431798998\\
293	0.0134817267651193\\
294	0.0134817100682453\\
295	0.0134816930842162\\
296	0.013481675807866\\
297	0.0134816582339214\\
298	0.0134816403569982\\
299	0.0134816221715984\\
300	0.0134816036721062\\
301	0.0134815848527846\\
302	0.0134815657077716\\
303	0.0134815462310755\\
304	0.0134815264165713\\
305	0.0134815062579958\\
306	0.013481485748943\\
307	0.0134814648828592\\
308	0.0134814436530372\\
309	0.0134814220526096\\
310	0.0134814000745399\\
311	0.0134813777116123\\
312	0.0134813549564259\\
313	0.0134813318013979\\
314	0.0134813082387514\\
315	0.0134812842605035\\
316	0.0134812598584552\\
317	0.0134812350241814\\
318	0.0134812097490192\\
319	0.0134811840240555\\
320	0.0134811578401136\\
321	0.0134811311877385\\
322	0.0134811040571813\\
323	0.0134810764383817\\
324	0.0134810483209489\\
325	0.0134810196941409\\
326	0.0134809905468423\\
327	0.0134809608675387\\
328	0.0134809306442897\\
329	0.013480899864699\\
330	0.0134808685158808\\
331	0.0134808365844235\\
332	0.0134808040563496\\
333	0.0134807709170706\\
334	0.0134807371513378\\
335	0.0134807027431879\\
336	0.0134806676758817\\
337	0.0134806319318374\\
338	0.0134805954925551\\
339	0.013480558338534\\
340	0.0134805204491793\\
341	0.0134804818026987\\
342	0.0134804423759873\\
343	0.0134804021444987\\
344	0.0134803610821009\\
345	0.0134803191609154\\
346	0.0134802763511367\\
347	0.0134802326208307\\
348	0.0134801879357078\\
349	0.0134801422588681\\
350	0.0134800955505154\\
351	0.0134800477676343\\
352	0.0134799988636274\\
353	0.0134799487879092\\
354	0.0134798974854586\\
355	0.0134798448963313\\
356	0.0134797909551102\\
357	0.0134797355901977\\
358	0.0134796787229736\\
359	0.013479620266971\\
360	0.013479560126764\\
361	0.0134794981964129\\
362	0.0134794343568317\\
363	0.0134793684701662\\
364	0.0134793003652451\\
365	0.0134792297950084\\
366	0.0134791563025685\\
367	0.0134790787814495\\
368	0.0134789940540261\\
369	0.0134788225905738\\
370	0.0134786421622733\\
371	0.0134784588125423\\
372	0.0134782724952062\\
373	0.0134780831631753\\
374	0.0134778907681663\\
375	0.0134776952603716\\
376	0.0134774965886828\\
377	0.0134772947032948\\
378	0.0134770895610133\\
379	0.01347688111418\\
380	0.0134766693104359\\
381	0.0134764540964931\\
382	0.0134762354180778\\
383	0.0134760132198659\\
384	0.0134757874454072\\
385	0.0134755580370403\\
386	0.0134753249357949\\
387	0.0134750880812801\\
388	0.0134748474115577\\
389	0.013474602862999\\
390	0.0134743543701214\\
391	0.0134741018654059\\
392	0.0134738452790901\\
393	0.0134735845389376\\
394	0.0134733195699808\\
395	0.0134730502942381\\
396	0.0134727766304081\\
397	0.0134724984935524\\
398	0.013472215794791\\
399	0.0134719284410651\\
400	0.0134716363350594\\
401	0.0134713393754015\\
402	0.0134710374571164\\
403	0.0134707304715531\\
404	0.0134704183026849\\
405	0.0134701008133592\\
406	0.0134697778267466\\
407	0.0134694491868978\\
408	0.0134691147405925\\
409	0.0134687743277694\\
410	0.0134684277697531\\
411	0.0134680747943538\\
412	0.0134677148855409\\
413	0.0134673475368608\\
414	0.0134669729739231\\
415	0.0134665912598975\\
416	0.0134662022158768\\
417	0.0134658056562991\\
418	0.0134654013887533\\
419	0.0134649892136385\\
420	0.0134645689231513\\
421	0.0134641402987486\\
422	0.0134637031079143\\
423	0.0134632571081398\\
424	0.0134628020505101\\
425	0.0134623376746722\\
426	0.0134618637080433\\
427	0.0134613798649436\\
428	0.0134608858456454\\
429	0.0134603813353257\\
430	0.013459866002912\\
431	0.013459339499804\\
432	0.0134588014584575\\
433	0.0134582514908082\\
434	0.0134576891865135\\
435	0.0134571141109814\\
436	0.0134565258031496\\
437	0.0134559237729594\\
438	0.0134553074984349\\
439	0.0134546764222034\\
440	0.0134540299471266\\
441	0.0134533674303381\\
442	0.0134526881742701\\
443	0.0134519914123622\\
444	0.0134512762883091\\
445	0.0134505418424898\\
446	0.0134497870852128\\
447	0.0134490113382093\\
448	0.0134482135369743\\
449	0.0134473924689366\\
450	0.013446546782067\\
451	0.0134456749784278\\
452	0.0134447753873113\\
453	0.0134438461213334\\
454	0.0134428849824512\\
455	0.0134418892079458\\
456	0.0134408546882444\\
457	0.0134397734222369\\
458	0.0134386253703978\\
459	0.0134367047619298\\
460	0.0134346007627936\\
461	0.0134324534544652\\
462	0.013430261091304\\
463	0.0134280221939364\\
464	0.0134257353518306\\
465	0.0134233989451662\\
466	0.0134210127654728\\
467	0.0134185746922587\\
468	0.0134159137836394\\
469	0.0134131253562001\\
470	0.0134102722380929\\
471	0.013407366022477\\
472	0.0134044046970774\\
473	0.0134013861008012\\
474	0.0133983079142918\\
475	0.0133951676661396\\
476	0.0133919627790866\\
477	0.0133886906994249\\
478	0.0133853584756237\\
479	0.0133819540096636\\
480	0.0133784670549945\\
481	0.0133748946406184\\
482	0.0133712330759159\\
483	0.0133674783849652\\
484	0.013363626298833\\
485	0.0133596721851578\\
486	0.0133556108381455\\
487	0.0133514354691886\\
488	0.0133471319214509\\
489	0.0133421365324358\\
490	0.0133367640115309\\
491	0.0133312607324878\\
492	0.0133256206557325\\
493	0.0133198371639317\\
494	0.0133139031896577\\
495	0.0133078108486197\\
496	0.0133015514080919\\
497	0.0132951150274935\\
498	0.0132884914106425\\
499	0.0132816700229335\\
500	0.0132746405374766\\
501	0.0132673828862216\\
502	0.0132598624666167\\
503	0.0132520638943832\\
504	0.0132439309280994\\
505	0.0132310060311702\\
506	0.013217506362061\\
507	0.0132036498742862\\
508	0.0131888657849896\\
509	0.0131720523245974\\
510	0.0131548950503627\\
511	0.013137382373354\\
512	0.01311950271671\\
513	0.0131012437962938\\
514	0.0130825865701937\\
515	0.0130635250231859\\
516	0.0130440558425903\\
517	0.0130241282541478\\
518	0.0130037110193216\\
519	0.0129827769137441\\
520	0.0129613858402691\\
521	0.0129394617355794\\
522	0.012916941040038\\
523	0.0128937960044298\\
524	0.0128699291158758\\
525	0.0128452834311488\\
526	0.0128198413153205\\
527	0.0127935851421469\\
528	0.0127674028256812\\
529	0.0127407523233205\\
530	0.0127132324135985\\
531	0.0126769985315014\\
532	0.0126397275406422\\
533	0.0126030266030857\\
534	0.0125670606395804\\
535	0.0125301138866154\\
536	0.0124921310664894\\
537	0.0124530517957935\\
538	0.0124128105316109\\
539	0.0123713403308501\\
540	0.0123286044295985\\
541	0.0122885437677416\\
542	0.012249640215702\\
543	0.0122092177372384\\
544	0.0121471371196391\\
545	0.0120759788892357\\
546	0.0120030094141084\\
547	0.0119281663124486\\
548	0.0118513316197983\\
549	0.0117723787037577\\
550	0.0116911819851764\\
551	0.0116076640950326\\
552	0.0115252412543895\\
553	0.0114568516315375\\
554	0.0113858548896472\\
555	0.0113112146929633\\
556	0.0111996008565847\\
557	0.0110859072514765\\
558	0.0109700816803649\\
559	0.0108520717474002\\
560	0.0107318195872003\\
561	0.0106092810625123\\
562	0.0104844183042965\\
563	0.0103571981014711\\
564	0.010229196474915\\
565	0.0100989357460154\\
566	0.00996556898396692\\
567	0.00982906327339909\\
568	0.00975677063061848\\
569	0.00968632648367908\\
570	0.0096150755783179\\
571	0.00954315619903026\\
572	0.00947061485625028\\
573	0.00939712619316549\\
574	0.00932283394429589\\
575	0.00924792300197075\\
576	0.00917262823381608\\
577	0.00909724756957369\\
578	0.00902216065203892\\
579	0.00894785644060203\\
580	0.00887419226306399\\
581	0.00879884330774817\\
582	0.00872180054699871\\
583	0.00864305944617903\\
584	0.0085626205974466\\
585	0.008480489806672\\
586	0.00839667853748955\\
587	0.00831120341405846\\
588	0.00822407719928457\\
589	0.00813527710953944\\
590	0.00804486435975251\\
591	0.00793561760306448\\
592	0.00776256831893187\\
593	0.00758639978571355\\
594	0.00738834244898588\\
595	0.00691750222005476\\
596	0.00589506097552246\\
597	0.00498881296807123\\
598	0.00357511483354343\\
599	0\\
600	0\\
};
\addplot [color=mycolor11,solid,forget plot]
  table[row sep=crcr]{%
1	0.0135446655882236\\
2	0.0135446655790726\\
3	0.013544665569765\\
4	0.0135446655602981\\
5	0.0135446655506691\\
6	0.0135446655408753\\
7	0.0135446655309139\\
8	0.0135446655207819\\
9	0.0135446655104765\\
10	0.0135446654999946\\
11	0.0135446654893333\\
12	0.0135446654784895\\
13	0.01354466546746\\
14	0.0135446654562416\\
15	0.0135446654448312\\
16	0.0135446654332254\\
17	0.0135446654214208\\
18	0.0135446654094141\\
19	0.0135446653972018\\
20	0.0135446653847804\\
21	0.0135446653721462\\
22	0.0135446653592956\\
23	0.013544665346225\\
24	0.0135446653329305\\
25	0.0135446653194084\\
26	0.0135446653056546\\
27	0.0135446652916653\\
28	0.0135446652774363\\
29	0.0135446652629637\\
30	0.0135446652482431\\
31	0.0135446652332704\\
32	0.0135446652180413\\
33	0.0135446652025513\\
34	0.0135446651867959\\
35	0.0135446651707707\\
36	0.013544665154471\\
37	0.0135446651378921\\
38	0.0135446651210292\\
39	0.0135446651038774\\
40	0.0135446650864319\\
41	0.0135446650686875\\
42	0.0135446650506391\\
43	0.0135446650322816\\
44	0.0135446650136096\\
45	0.0135446649946178\\
46	0.0135446649753006\\
47	0.0135446649556526\\
48	0.013544664935668\\
49	0.013544664915341\\
50	0.0135446648946659\\
51	0.0135446648736366\\
52	0.0135446648522471\\
53	0.0135446648304913\\
54	0.0135446648083628\\
55	0.0135446647858552\\
56	0.0135446647629622\\
57	0.013544664739677\\
58	0.013544664715993\\
59	0.0135446646919033\\
60	0.0135446646674011\\
61	0.0135446646424792\\
62	0.0135446646171305\\
63	0.0135446645913476\\
64	0.0135446645651232\\
65	0.0135446645384497\\
66	0.0135446645113194\\
67	0.0135446644837245\\
68	0.0135446644556571\\
69	0.013544664427109\\
70	0.0135446643980721\\
71	0.0135446643685379\\
72	0.0135446643384981\\
73	0.013544664307944\\
74	0.0135446642768667\\
75	0.0135446642452573\\
76	0.0135446642131068\\
77	0.0135446641804058\\
78	0.013544664147145\\
79	0.0135446641133149\\
80	0.0135446640789057\\
81	0.0135446640439074\\
82	0.0135446640083102\\
83	0.0135446639721036\\
84	0.0135446639352773\\
85	0.0135446638978208\\
86	0.0135446638597233\\
87	0.0135446638209738\\
88	0.0135446637815612\\
89	0.0135446637414742\\
90	0.0135446637007013\\
91	0.0135446636592308\\
92	0.0135446636170507\\
93	0.013544663574149\\
94	0.0135446635305135\\
95	0.0135446634861314\\
96	0.0135446634409902\\
97	0.0135446633950769\\
98	0.0135446633483783\\
99	0.013544663300881\\
100	0.0135446632525715\\
101	0.0135446632034358\\
102	0.0135446631534599\\
103	0.0135446631026296\\
104	0.0135446630509301\\
105	0.0135446629983468\\
106	0.0135446629448646\\
107	0.0135446628904681\\
108	0.0135446628351419\\
109	0.01354466277887\\
110	0.0135446627216363\\
111	0.0135446626634246\\
112	0.0135446626042182\\
113	0.013544662544\\
114	0.013544662482753\\
115	0.0135446624204596\\
116	0.013544662357102\\
117	0.0135446622926621\\
118	0.0135446622271215\\
119	0.0135446621604615\\
120	0.013544662092663\\
121	0.0135446620237068\\
122	0.0135446619535731\\
123	0.0135446618822419\\
124	0.0135446618096928\\
125	0.0135446617359052\\
126	0.0135446616608581\\
127	0.0135446615845299\\
128	0.013544661506899\\
129	0.0135446614279433\\
130	0.0135446613476402\\
131	0.0135446612659669\\
132	0.0135446611829001\\
133	0.0135446610984161\\
134	0.013544661012491\\
135	0.0135446609251001\\
136	0.0135446608362188\\
137	0.0135446607458216\\
138	0.0135446606538829\\
139	0.0135446605603765\\
140	0.0135446604652758\\
141	0.0135446603685539\\
142	0.0135446602701831\\
143	0.0135446601701355\\
144	0.0135446600683828\\
145	0.0135446599648959\\
146	0.0135446598596456\\
147	0.0135446597526019\\
148	0.0135446596437344\\
149	0.0135446595330122\\
150	0.013544659420404\\
151	0.0135446593058777\\
152	0.0135446591894009\\
153	0.0135446590709406\\
154	0.0135446589504631\\
155	0.0135446588279344\\
156	0.0135446587033196\\
157	0.0135446585765834\\
158	0.0135446584476901\\
159	0.013544658316603\\
160	0.0135446581832851\\
161	0.0135446580476985\\
162	0.013544657909805\\
163	0.0135446577695654\\
164	0.0135446576269402\\
165	0.0135446574818889\\
166	0.0135446573343706\\
167	0.0135446571843435\\
168	0.0135446570317652\\
169	0.0135446568765926\\
170	0.0135446567187818\\
171	0.0135446565582883\\
172	0.0135446563950666\\
173	0.0135446562290708\\
174	0.013544656060254\\
175	0.0135446558885684\\
176	0.0135446557139656\\
177	0.0135446555363964\\
178	0.0135446553558107\\
179	0.0135446551721574\\
180	0.0135446549853849\\
181	0.0135446547954404\\
182	0.0135446546022704\\
183	0.0135446544058205\\
184	0.0135446542060352\\
185	0.0135446540028583\\
186	0.0135446537962326\\
187	0.0135446535860998\\
188	0.0135446533724008\\
189	0.0135446531550754\\
190	0.0135446529340624\\
191	0.0135446527092996\\
192	0.0135446524807237\\
193	0.0135446522482705\\
194	0.0135446520118745\\
195	0.0135446517714693\\
196	0.0135446515269871\\
197	0.0135446512783594\\
198	0.0135446510255161\\
199	0.0135446507683862\\
200	0.0135446505068975\\
201	0.0135446502409763\\
202	0.0135446499705481\\
203	0.0135446496955367\\
204	0.013544649415865\\
205	0.0135446491314544\\
206	0.013544648842225\\
207	0.0135446485480956\\
208	0.0135446482489836\\
209	0.0135446479448049\\
210	0.0135446476354743\\
211	0.0135446473209048\\
212	0.0135446470010081\\
213	0.0135446466756945\\
214	0.0135446463448727\\
215	0.0135446460084497\\
216	0.0135446456663313\\
217	0.0135446453184213\\
218	0.0135446449646222\\
219	0.0135446446048347\\
220	0.0135446442389578\\
221	0.0135446438668889\\
222	0.0135446434885235\\
223	0.0135446431037556\\
224	0.0135446427124771\\
225	0.0135446423145782\\
226	0.0135446419099473\\
227	0.013544641498471\\
228	0.0135446410800336\\
229	0.0135446406545177\\
230	0.0135446402218041\\
231	0.0135446397817711\\
232	0.0135446393342953\\
233	0.0135446388792511\\
234	0.0135446384165106\\
235	0.0135446379459441\\
236	0.0135446374674193\\
237	0.0135446369808019\\
238	0.0135446364859551\\
239	0.0135446359827399\\
240	0.0135446354710149\\
241	0.0135446349506363\\
242	0.0135446344214578\\
243	0.0135446338833306\\
244	0.0135446333361034\\
245	0.0135446327796221\\
246	0.0135446322137303\\
247	0.0135446316382686\\
248	0.013544631053075\\
249	0.0135446304579846\\
250	0.0135446298528298\\
251	0.01354462923744\\
252	0.0135446286116418\\
253	0.0135446279752584\\
254	0.0135446273281105\\
255	0.0135446266700152\\
256	0.0135446260007867\\
257	0.0135446253202359\\
258	0.0135446246281703\\
259	0.0135446239243941\\
260	0.0135446232087082\\
261	0.0135446224809098\\
262	0.0135446217407927\\
263	0.0135446209881469\\
264	0.013544620222759\\
265	0.0135446194444116\\
266	0.0135446186528833\\
267	0.0135446178479492\\
268	0.0135446170293801\\
269	0.0135446161969426\\
270	0.0135446153503995\\
271	0.0135446144895089\\
272	0.013544613614025\\
273	0.0135446127236976\\
274	0.0135446118182721\\
275	0.0135446108974892\\
276	0.0135446099610847\\
277	0.0135446090087897\\
278	0.0135446080403303\\
279	0.0135446070554277\\
280	0.013544606053798\\
281	0.0135446050351519\\
282	0.013544603999195\\
283	0.0135446029456273\\
284	0.0135446018741432\\
285	0.0135446007844316\\
286	0.0135445996761752\\
287	0.0135445985490511\\
288	0.01354459740273\\
289	0.0135445962368765\\
290	0.0135445950511487\\
291	0.0135445938451983\\
292	0.0135445926186701\\
293	0.013544591371202\\
294	0.0135445901024249\\
295	0.0135445888119625\\
296	0.0135445874994309\\
297	0.0135445861644386\\
298	0.0135445848065865\\
299	0.0135445834254671\\
300	0.0135445820206647\\
301	0.0135445805917552\\
302	0.0135445791383057\\
303	0.0135445776598743\\
304	0.0135445761560098\\
305	0.0135445746262516\\
306	0.0135445730701293\\
307	0.0135445714871624\\
308	0.01354456987686\\
309	0.0135445682387198\\
310	0.0135445665722272\\
311	0.0135445648768556\\
312	0.0135445631520671\\
313	0.0135445613973118\\
314	0.0135445596120268\\
315	0.0135445577956357\\
316	0.0135445559475479\\
317	0.0135445540671584\\
318	0.0135445521538467\\
319	0.0135445502069765\\
320	0.0135445482258948\\
321	0.0135445462099312\\
322	0.013544544158397\\
323	0.0135445420705845\\
324	0.0135445399457655\\
325	0.0135445377831909\\
326	0.013544535582089\\
327	0.0135445333416647\\
328	0.0135445310610977\\
329	0.0135445287395411\\
330	0.0135445263761202\\
331	0.0135445239699301\\
332	0.0135445215200338\\
333	0.0135445190254605\\
334	0.0135445164852028\\
335	0.0135445138982141\\
336	0.0135445112634057\\
337	0.0135445085796439\\
338	0.0135445058457457\\
339	0.0135445030604759\\
340	0.0135445002225416\\
341	0.0135444973305881\\
342	0.0135444943831934\\
343	0.0135444913788617\\
344	0.0135444883160172\\
345	0.0135444851929964\\
346	0.0135444820080395\\
347	0.0135444787592812\\
348	0.0135444754447402\\
349	0.0135444720623068\\
350	0.0135444686097299\\
351	0.0135444650846012\\
352	0.0135444614843391\\
353	0.0135444578061706\\
354	0.0135444540471148\\
355	0.013544450203962\\
356	0.013544446273237\\
357	0.0135444422511474\\
358	0.0135444381335358\\
359	0.0135444339157749\\
360	0.013544429592543\\
361	0.0135444251572839\\
362	0.0135444206008107\\
363	0.0135444159075856\\
364	0.0135444110456751\\
365	0.0135444059397022\\
366	0.0135444003999544\\
367	0.0135443939560351\\
368	0.0135443856817013\\
369	0.0135443771544858\\
370	0.0135443684920978\\
371	0.0135443596924323\\
372	0.0135443507533073\\
373	0.0135443416724163\\
374	0.0135443324472715\\
375	0.0135443230752331\\
376	0.0135443135539124\\
377	0.0135443038821088\\
378	0.0135442940580924\\
379	0.0135442840795147\\
380	0.0135442739439858\\
381	0.0135442636490718\\
382	0.0135442531922924\\
383	0.013544242571117\\
384	0.0135442317829619\\
385	0.0135442208251853\\
386	0.0135442096950834\\
387	0.0135441983898843\\
388	0.0135441869067426\\
389	0.0135441752427324\\
390	0.0135441633948397\\
391	0.013544151359954\\
392	0.0135441391348589\\
393	0.0135441267162224\\
394	0.0135441141005864\\
395	0.0135441012843576\\
396	0.0135440882638017\\
397	0.0135440750350464\\
398	0.0135440615941042\\
399	0.0135440479369319\\
400	0.0135440340595492\\
401	0.0135440199582147\\
402	0.0135440056295404\\
403	0.0135439910700731\\
404	0.013543976274333\\
405	0.013543961231781\\
406	0.0135439459366184\\
407	0.0135439303856296\\
408	0.0135439145765875\\
409	0.0135438985047629\\
410	0.0135438821506532\\
411	0.0135438654612502\\
412	0.0135438483892285\\
413	0.0135438309839511\\
414	0.013543813270509\\
415	0.0135437952421574\\
416	0.0135437768919685\\
417	0.0135437582128495\\
418	0.0135437391975382\\
419	0.0135437198385026\\
420	0.0135437001276234\\
421	0.0135436800557971\\
422	0.0135436596136035\\
423	0.0135436387919014\\
424	0.0135436175811696\\
425	0.0135435959714813\\
426	0.0135435739524761\\
427	0.0135435515133297\\
428	0.0135435286427206\\
429	0.0135435053287926\\
430	0.0135434815591143\\
431	0.0135434573206333\\
432	0.013543432599626\\
433	0.013543407381641\\
434	0.0135433816514343\\
435	0.0135433553928954\\
436	0.0135433285889577\\
437	0.0135433012214827\\
438	0.0135432732710955\\
439	0.013543244716918\\
440	0.013543215536086\\
441	0.0135431857028279\\
442	0.0135431551867576\\
443	0.0135431239502636\\
444	0.0135430919471355\\
445	0.0135430591343879\\
446	0.0135430255265884\\
447	0.0135429910942649\\
448	0.0135429557988381\\
449	0.0135429195955067\\
450	0.0135428824339565\\
451	0.0135428442557001\\
452	0.0135428049872605\\
453	0.0135427645212475\\
454	0.0135427226626812\\
455	0.013542678979174\\
456	0.0135426324061411\\
457	0.0135425803572847\\
458	0.0135425177949178\\
459	0.0135424524910871\\
460	0.0135423861055671\\
461	0.0135423184888302\\
462	0.0135422494887769\\
463	0.0135421789156911\\
464	0.0135421064578075\\
465	0.0135420316906338\\
466	0.0135419531202762\\
467	0.0135418654838934\\
468	0.0135416341376981\\
469	0.0135413405976787\\
470	0.0135410411476683\\
471	0.0135407354421003\\
472	0.0135404231026969\\
473	0.0135401037194476\\
474	0.0135397768556062\\
475	0.0135394420526863\\
476	0.0135390988031708\\
477	0.0135387463233793\\
478	0.0135383789935058\\
479	0.0135379997824866\\
480	0.013537608211046\\
481	0.0135372033927514\\
482	0.0135367843330014\\
483	0.0135363498948024\\
484	0.0135358987101663\\
485	0.0135354289435399\\
486	0.0135349376054903\\
487	0.0135344184725542\\
488	0.013533855475381\\
489	0.0135328021465961\\
490	0.0135315156077701\\
491	0.0135301983322477\\
492	0.0135288489897223\\
493	0.01352746616148\\
494	0.013526048305324\\
495	0.0135245937549364\\
496	0.0135231006998862\\
497	0.0135215672633874\\
498	0.0135199914415026\\
499	0.0135183708444663\\
500	0.0135167015480944\\
501	0.013514977877994\\
502	0.0135131933366922\\
503	0.0135113290126056\\
504	0.0135093670876991\\
505	0.013507331459421\\
506	0.0135052119835207\\
507	0.0135029873558751\\
508	0.0135001802069156\\
509	0.0134959150840131\\
510	0.0134915597262709\\
511	0.0134871103269274\\
512	0.0134825626886634\\
513	0.0134779115567128\\
514	0.0134731517082004\\
515	0.0134682794976019\\
516	0.0134632887333625\\
517	0.0134581716044498\\
518	0.013452920103399\\
519	0.0134475235795158\\
520	0.0134419283221539\\
521	0.0134361529326394\\
522	0.0134302058308984\\
523	0.0134240708131129\\
524	0.0134177107485215\\
525	0.0134111164356789\\
526	0.0134042736876343\\
527	0.0133971492022143\\
528	0.0133892336723207\\
529	0.0133807705780515\\
530	0.0133718730784398\\
531	0.0133561714480059\\
532	0.0133398458429537\\
533	0.0133223131893943\\
534	0.0133033809070622\\
535	0.0132839320854405\\
536	0.0132639373003346\\
537	0.0132433638151846\\
538	0.0132221737264103\\
539	0.0132003183119116\\
540	0.013177717456573\\
541	0.0131521697522508\\
542	0.0131244708335213\\
543	0.0130959542172282\\
544	0.013066498295731\\
545	0.0130360408378794\\
546	0.0130045762794272\\
547	0.0129720461920879\\
548	0.0129383631174359\\
549	0.012903431437999\\
550	0.012867149193955\\
551	0.0128294058948418\\
552	0.0127883021395426\\
553	0.012736775160571\\
554	0.0126833866717845\\
555	0.012627347866077\\
556	0.0125409032726753\\
557	0.0124521488577097\\
558	0.0123609066199172\\
559	0.0122670518760954\\
560	0.0121703879470922\\
561	0.0120706871507593\\
562	0.011967664038773\\
563	0.0118611178333308\\
564	0.0117507628222118\\
565	0.0116369864425952\\
566	0.0115243134277903\\
567	0.011407630437138\\
568	0.0112598262504462\\
569	0.0111071282536938\\
570	0.0109509091848256\\
571	0.0107908752347999\\
572	0.010631835805226\\
573	0.0104904803742406\\
574	0.0103483159232565\\
575	0.0102019742201461\\
576	0.0100513986648676\\
577	0.00989680073265635\\
578	0.00973906102328593\\
579	0.00957582759199066\\
580	0.00942261438354543\\
581	0.00932261027378465\\
582	0.00921979343461604\\
583	0.00911412364342678\\
584	0.00900557245397146\\
585	0.00889412941968206\\
586	0.0087797599448596\\
587	0.00866252558838551\\
588	0.00854253765219076\\
589	0.00841997343584366\\
590	0.00829509753613871\\
591	0.00816828987892658\\
592	0.00804008534578783\\
593	0.00791122806731336\\
594	0.00778275427780483\\
595	0.00756131189408085\\
596	0.00719876254233496\\
597	0.00551136382163314\\
598	0.00357511483354343\\
599	0\\
600	0\\
};
\addplot [color=mycolor12,solid,forget plot]
  table[row sep=crcr]{%
1	0.0135729281156313\\
2	0.0135729281149894\\
3	0.0135729281143365\\
4	0.0135729281136724\\
5	0.013572928112997\\
6	0.01357292811231\\
7	0.0135729281116113\\
8	0.0135729281109005\\
9	0.0135729281101777\\
10	0.0135729281094424\\
11	0.0135729281086946\\
12	0.0135729281079339\\
13	0.0135729281071602\\
14	0.0135729281063733\\
15	0.0135729281055729\\
16	0.0135729281047588\\
17	0.0135729281039308\\
18	0.0135729281030885\\
19	0.0135729281022319\\
20	0.0135729281013606\\
21	0.0135729281004743\\
22	0.0135729280995729\\
23	0.0135729280986561\\
24	0.0135729280977235\\
25	0.013572928096775\\
26	0.0135729280958102\\
27	0.0135729280948289\\
28	0.0135729280938308\\
29	0.0135729280928156\\
30	0.013572928091783\\
31	0.0135729280907327\\
32	0.0135729280896644\\
33	0.0135729280885778\\
34	0.0135729280874727\\
35	0.0135729280863485\\
36	0.0135729280852052\\
37	0.0135729280840422\\
38	0.0135729280828593\\
39	0.0135729280816562\\
40	0.0135729280804325\\
41	0.0135729280791877\\
42	0.0135729280779217\\
43	0.013572928076634\\
44	0.0135729280753242\\
45	0.013572928073992\\
46	0.013572928072637\\
47	0.0135729280712587\\
48	0.0135729280698569\\
49	0.013572928068431\\
50	0.0135729280669808\\
51	0.0135729280655056\\
52	0.0135729280640052\\
53	0.0135729280624792\\
54	0.0135729280609269\\
55	0.0135729280593481\\
56	0.0135729280577423\\
57	0.0135729280561089\\
58	0.0135729280544476\\
59	0.0135729280527578\\
60	0.013572928051039\\
61	0.0135729280492909\\
62	0.0135729280475128\\
63	0.0135729280457042\\
64	0.0135729280438647\\
65	0.0135729280419937\\
66	0.0135729280400906\\
67	0.013572928038155\\
68	0.0135729280361862\\
69	0.0135729280341837\\
70	0.0135729280321469\\
71	0.0135729280300752\\
72	0.0135729280279681\\
73	0.0135729280258249\\
74	0.013572928023645\\
75	0.0135729280214278\\
76	0.0135729280191726\\
77	0.0135729280168788\\
78	0.0135729280145458\\
79	0.0135729280121728\\
80	0.0135729280097592\\
81	0.0135729280073044\\
82	0.0135729280048074\\
83	0.0135729280022678\\
84	0.0135729279996847\\
85	0.0135729279970574\\
86	0.0135729279943851\\
87	0.0135729279916671\\
88	0.0135729279889026\\
89	0.0135729279860908\\
90	0.0135729279832309\\
91	0.0135729279803221\\
92	0.0135729279773635\\
93	0.0135729279743544\\
94	0.0135729279712937\\
95	0.0135729279681807\\
96	0.0135729279650144\\
97	0.013572927961794\\
98	0.0135729279585186\\
99	0.0135729279551871\\
100	0.0135729279517986\\
101	0.0135729279483523\\
102	0.013572927944847\\
103	0.0135729279412817\\
104	0.0135729279376556\\
105	0.0135729279339674\\
106	0.0135729279302162\\
107	0.0135729279264009\\
108	0.0135729279225204\\
109	0.0135729279185736\\
110	0.0135729279145594\\
111	0.0135729279104765\\
112	0.0135729279063239\\
113	0.0135729279021004\\
114	0.0135729278978047\\
115	0.0135729278934356\\
116	0.0135729278889919\\
117	0.0135729278844723\\
118	0.0135729278798756\\
119	0.0135729278752003\\
120	0.0135729278704452\\
121	0.0135729278656089\\
122	0.0135729278606901\\
123	0.0135729278556873\\
124	0.0135729278505991\\
125	0.0135729278454241\\
126	0.0135729278401607\\
127	0.0135729278348075\\
128	0.013572927829363\\
129	0.0135729278238256\\
130	0.0135729278181937\\
131	0.0135729278124657\\
132	0.0135729278066401\\
133	0.013572927800715\\
134	0.0135729277946889\\
135	0.0135729277885601\\
136	0.0135729277823267\\
137	0.0135729277759871\\
138	0.0135729277695394\\
139	0.0135729277629817\\
140	0.0135729277563123\\
141	0.0135729277495293\\
142	0.0135729277426306\\
143	0.0135729277356143\\
144	0.0135729277284785\\
145	0.0135729277212212\\
146	0.0135729277138402\\
147	0.0135729277063334\\
148	0.0135729276986988\\
149	0.0135729276909342\\
150	0.0135729276830373\\
151	0.013572927675006\\
152	0.0135729276668379\\
153	0.0135729276585308\\
154	0.0135729276500823\\
155	0.0135729276414899\\
156	0.0135729276327513\\
157	0.013572927623864\\
158	0.0135729276148255\\
159	0.0135729276056332\\
160	0.0135729275962845\\
161	0.0135729275867767\\
162	0.0135729275771073\\
163	0.0135729275672733\\
164	0.0135729275572722\\
165	0.0135729275471009\\
166	0.0135729275367568\\
167	0.0135729275262367\\
168	0.0135729275155379\\
169	0.0135729275046571\\
170	0.0135729274935915\\
171	0.0135729274823377\\
172	0.0135729274708928\\
173	0.0135729274592534\\
174	0.0135729274474162\\
175	0.013572927435378\\
176	0.0135729274231353\\
177	0.0135729274106846\\
178	0.0135729273980225\\
179	0.0135729273851454\\
180	0.0135729273720496\\
181	0.0135729273587315\\
182	0.0135729273451874\\
183	0.0135729273314133\\
184	0.0135729273174055\\
185	0.0135729273031599\\
186	0.0135729272886726\\
187	0.0135729272739395\\
188	0.0135729272589564\\
189	0.0135729272437192\\
190	0.0135729272282235\\
191	0.0135729272124649\\
192	0.0135729271964392\\
193	0.0135729271801417\\
194	0.0135729271635678\\
195	0.0135729271467129\\
196	0.0135729271295723\\
197	0.0135729271121412\\
198	0.0135729270944146\\
199	0.0135729270763876\\
200	0.0135729270580551\\
201	0.0135729270394119\\
202	0.0135729270204529\\
203	0.0135729270011727\\
204	0.0135729269815659\\
205	0.0135729269616269\\
206	0.0135729269413503\\
207	0.0135729269207302\\
208	0.0135729268997609\\
209	0.0135729268784366\\
210	0.0135729268567512\\
211	0.0135729268346987\\
212	0.0135729268122728\\
213	0.0135729267894674\\
214	0.0135729267662759\\
215	0.0135729267426919\\
216	0.0135729267187087\\
217	0.0135729266943197\\
218	0.013572926669518\\
219	0.0135729266442966\\
220	0.0135729266186486\\
221	0.0135729265925665\\
222	0.0135729265660433\\
223	0.0135729265390714\\
224	0.0135729265116432\\
225	0.0135729264837511\\
226	0.0135729264553872\\
227	0.0135729264265437\\
228	0.0135729263972123\\
229	0.0135729263673849\\
230	0.0135729263370531\\
231	0.0135729263062084\\
232	0.0135729262748421\\
233	0.0135729262429455\\
234	0.0135729262105096\\
235	0.0135729261775252\\
236	0.0135729261439831\\
237	0.013572926109874\\
238	0.0135729260751882\\
239	0.0135729260399159\\
240	0.0135729260040474\\
241	0.0135729259675724\\
242	0.0135729259304807\\
243	0.013572925892762\\
244	0.0135729258544056\\
245	0.0135729258154006\\
246	0.0135729257757362\\
247	0.0135729257354012\\
248	0.0135729256943842\\
249	0.0135729256526736\\
250	0.0135729256102578\\
251	0.0135729255671247\\
252	0.0135729255232622\\
253	0.0135729254786578\\
254	0.0135729254332991\\
255	0.0135729253871732\\
256	0.0135729253402671\\
257	0.0135729252925674\\
258	0.0135729252440608\\
259	0.0135729251947335\\
260	0.0135729251445715\\
261	0.0135729250935605\\
262	0.0135729250416862\\
263	0.0135729249889338\\
264	0.0135729249352883\\
265	0.0135729248807345\\
266	0.0135729248252569\\
267	0.0135729247688396\\
268	0.0135729247114666\\
269	0.0135729246531215\\
270	0.0135729245937876\\
271	0.013572924533448\\
272	0.0135729244720855\\
273	0.0135729244096825\\
274	0.013572924346221\\
275	0.0135729242816829\\
276	0.0135729242160496\\
277	0.0135729241493022\\
278	0.0135729240814215\\
279	0.0135729240123878\\
280	0.0135729239421813\\
281	0.0135729238707816\\
282	0.013572923798168\\
283	0.0135729237243194\\
284	0.0135729236492143\\
285	0.0135729235728309\\
286	0.0135729234951468\\
287	0.0135729234161392\\
288	0.013572923335785\\
289	0.0135729232540607\\
290	0.013572923170942\\
291	0.0135729230864045\\
292	0.0135729230004232\\
293	0.0135729229129724\\
294	0.0135729228240262\\
295	0.0135729227335581\\
296	0.0135729226415408\\
297	0.0135729225479469\\
298	0.013572922452748\\
299	0.0135729223559154\\
300	0.0135729222574197\\
301	0.0135729221572309\\
302	0.0135729220553183\\
303	0.0135729219516506\\
304	0.0135729218461959\\
305	0.0135729217389213\\
306	0.0135729216297936\\
307	0.0135729215187785\\
308	0.0135729214058411\\
309	0.0135729212909455\\
310	0.0135729211740552\\
311	0.0135729210551326\\
312	0.0135729209341393\\
313	0.013572920811036\\
314	0.0135729206857825\\
315	0.0135729205583373\\
316	0.0135729204286579\\
317	0.013572920296701\\
318	0.0135729201624217\\
319	0.013572920025774\\
320	0.0135729198867107\\
321	0.0135729197451831\\
322	0.0135729196011413\\
323	0.0135729194545334\\
324	0.0135729193053064\\
325	0.0135729191534053\\
326	0.0135729189987732\\
327	0.0135729188413517\\
328	0.0135729186810798\\
329	0.0135729185178947\\
330	0.013572918351731\\
331	0.013572918182521\\
332	0.013572918010194\\
333	0.0135729178346768\\
334	0.0135729176558925\\
335	0.0135729174737614\\
336	0.0135729172881998\\
337	0.0135729170991199\\
338	0.0135729169064299\\
339	0.0135729167100331\\
340	0.0135729165098276\\
341	0.0135729163057062\\
342	0.0135729160975551\\
343	0.0135729158852543\\
344	0.013572915668676\\
345	0.0135729154476844\\
346	0.0135729152221349\\
347	0.0135729149918729\\
348	0.0135729147567329\\
349	0.0135729145165372\\
350	0.0135729142710951\\
351	0.0135729140202008\\
352	0.013572913763632\\
353	0.0135729135011484\\
354	0.0135729132324889\\
355	0.0135729129573694\\
356	0.0135729126754785\\
357	0.0135729123864727\\
358	0.013572912089966\\
359	0.0135729117855089\\
360	0.0135729114725381\\
361	0.0135729111502544\\
362	0.0135729108173162\\
363	0.0135729104710922\\
364	0.0135729101059222\\
365	0.0135729097095322\\
366	0.0135729092580673\\
367	0.0135729087233709\\
368	0.0135729081749621\\
369	0.0135729076179268\\
370	0.0135729070521349\\
371	0.0135729064774525\\
372	0.0135729058937422\\
373	0.0135729053008627\\
374	0.0135729046986732\\
375	0.0135729040870432\\
376	0.0135729034658593\\
377	0.0135729028349909\\
378	0.0135729021942936\\
379	0.0135729015436208\\
380	0.0135729008828234\\
381	0.0135729002117499\\
382	0.0135728995302461\\
383	0.0135728988381551\\
384	0.0135728981353166\\
385	0.013572897421567\\
386	0.0135728966967393\\
387	0.0135728959606623\\
388	0.0135728952131603\\
389	0.0135728944540529\\
390	0.0135728936831546\\
391	0.0135728929002735\\
392	0.0135728921052117\\
393	0.013572891297764\\
394	0.0135728904777176\\
395	0.0135728896448513\\
396	0.0135728887989359\\
397	0.0135728879397342\\
398	0.0135728870670026\\
399	0.0135728861804934\\
400	0.0135728852799571\\
401	0.013572884365139\\
402	0.0135728834357633\\
403	0.0135728824915085\\
404	0.0135728815320447\\
405	0.0135728805572851\\
406	0.0135728795674373\\
407	0.013572878563061\\
408	0.0135728775447914\\
409	0.013572876511889\\
410	0.0135728754594386\\
411	0.0135728743825948\\
412	0.013572873284697\\
413	0.013572872167358\\
414	0.0135728710301495\\
415	0.0135728698726291\\
416	0.0135728686943397\\
417	0.0135728674948077\\
418	0.0135728662735396\\
419	0.0135728650300171\\
420	0.0135728637636954\\
421	0.0135728624740132\\
422	0.0135728611603951\\
423	0.0135728598222425\\
424	0.0135728584589321\\
425	0.0135728570698137\\
426	0.013572855654209\\
427	0.0135728542114096\\
428	0.0135728527406747\\
429	0.0135728512412291\\
430	0.0135728497122602\\
431	0.0135728481529158\\
432	0.0135728465623006\\
433	0.0135728449394728\\
434	0.0135728432834402\\
435	0.0135728415931554\\
436	0.0135728398675097\\
437	0.0135728381053252\\
438	0.0135728363053431\\
439	0.0135728344662056\\
440	0.0135728325864286\\
441	0.0135728306643691\\
442	0.01357282869821\\
443	0.0135728266860471\\
444	0.0135728246262101\\
445	0.0135728225176014\\
446	0.0135728203581798\\
447	0.0135728181456281\\
448	0.0135728158773625\\
449	0.0135728135504532\\
450	0.0135728111613897\\
451	0.0135728087054866\\
452	0.0135728061753415\\
453	0.0135728035568695\\
454	0.013572800819553\\
455	0.0135727978949762\\
456	0.0135727946434485\\
457	0.0135727908854017\\
458	0.0135727869831818\\
459	0.0135727830092267\\
460	0.0135727789551242\\
461	0.0135727748071108\\
462	0.0135727705400786\\
463	0.0135727661129216\\
464	0.0135727614816987\\
465	0.0135727564885345\\
466	0.0135727505928484\\
467	0.0135727425811198\\
468	0.0135727336577712\\
469	0.0135727245843131\\
470	0.0135727153541195\\
471	0.0135727059607591\\
472	0.0135726963991296\\
473	0.0135726866678476\\
474	0.0135726767733334\\
475	0.0135726667338239\\
476	0.0135726565702078\\
477	0.0135726462369761\\
478	0.0135726316484591\\
479	0.013572616457069\\
480	0.0135726008475058\\
481	0.0135725847901606\\
482	0.0135725682465537\\
483	0.0135725511582732\\
484	0.0135725334176537\\
485	0.0135725147868961\\
486	0.0135724946782544\\
487	0.0135724715898321\\
488	0.0135724421276752\\
489	0.0135724098707835\\
490	0.0135723769809819\\
491	0.0135723434359845\\
492	0.0135723092149093\\
493	0.0135722742954621\\
494	0.0135722386584797\\
495	0.0135722022929575\\
496	0.0135721652038956\\
497	0.013572127416783\\
498	0.0135720889542622\\
499	0.0135720496660699\\
500	0.013572009200884\\
501	0.0135719672662555\\
502	0.0135719232805466\\
503	0.0135718691154561\\
504	0.0135718027415396\\
505	0.0135717334240164\\
506	0.0135716594194468\\
507	0.0135715758487889\\
508	0.0135714800971319\\
509	0.0135713830638696\\
510	0.0135712847109463\\
511	0.0135711849868987\\
512	0.0135710837974973\\
513	0.0135709809806871\\
514	0.0135708767134498\\
515	0.0135707709678568\\
516	0.0135706635658289\\
517	0.0135705544398054\\
518	0.0135704435231976\\
519	0.013570330716213\\
520	0.0135701783501662\\
521	0.0135700082626471\\
522	0.0135698353412184\\
523	0.0135696560324198\\
524	0.0135694707937673\\
525	0.0135692794948472\\
526	0.0135690791426417\\
527	0.0135688583052168\\
528	0.0135682154156232\\
529	0.0135673676922128\\
530	0.0135664530795002\\
531	0.0135654963789684\\
532	0.0135644778383099\\
533	0.0135627961861763\\
534	0.0135603011073321\\
535	0.0135577240095517\\
536	0.0135550577791303\\
537	0.0135522940269132\\
538	0.0135494221043857\\
539	0.0135464256209235\\
540	0.0135432666778108\\
541	0.0135381572659128\\
542	0.0135317392269093\\
543	0.0135251269767831\\
544	0.0135183081130177\\
545	0.0135112748453141\\
546	0.0135040125906924\\
547	0.0134965027045445\\
548	0.0134887245497606\\
549	0.0134806554663675\\
550	0.013472269619963\\
551	0.0134635274601709\\
552	0.0134528847211895\\
553	0.0134345591688223\\
554	0.0134156886229345\\
555	0.013396223717803\\
556	0.013376162949696\\
557	0.0133554547659071\\
558	0.0133340370095081\\
559	0.0133118730060486\\
560	0.0132888952198527\\
561	0.0132650238382838\\
562	0.0132401548743918\\
563	0.0132142247716468\\
564	0.0131871345066033\\
565	0.0131583158948313\\
566	0.0131248268231682\\
567	0.013088700249165\\
568	0.0130166689939226\\
569	0.0129412540129407\\
570	0.0128635173667425\\
571	0.0127832117855989\\
572	0.0126974946905088\\
573	0.012598252501507\\
574	0.0124957378383103\\
575	0.0123896817220364\\
576	0.0122798443130587\\
577	0.0121659373337167\\
578	0.0120482103024803\\
579	0.0119252550014735\\
580	0.0117882857577192\\
581	0.0116159413948298\\
582	0.0114382083633602\\
583	0.0112541254279651\\
584	0.0110633882730042\\
585	0.0108767719528068\\
586	0.0106827430451281\\
587	0.0104803507500256\\
588	0.0102688047348806\\
589	0.0100471280721329\\
590	0.00981391836833831\\
591	0.00956853936323224\\
592	0.00930998247240804\\
593	0.00903672163075643\\
594	0.00874682866488133\\
595	0.00843773941577757\\
596	0.00810565734955028\\
597	0.00735965377900303\\
598	0.00357511483354343\\
599	0\\
600	0\\
};
\addplot [color=mycolor13,solid,forget plot]
  table[row sep=crcr]{%
1	0.00626656850273689\\
2	0.00626656850273689\\
3	0.00626656850273689\\
4	0.00626656850273689\\
5	0.00626656850273689\\
6	0.00626656850273689\\
7	0.00626656850273689\\
8	0.00626656850273689\\
9	0.00626656850273689\\
10	0.00626656850273689\\
11	0.00626656850273689\\
12	0.00626656850273689\\
13	0.00626656850273689\\
14	0.00626656850273689\\
15	0.00626656850273689\\
16	0.00626656850273689\\
17	0.00626656850273689\\
18	0.00626656850273689\\
19	0.00626656850273689\\
20	0.00626656850273689\\
21	0.00626656850273689\\
22	0.00626656850273689\\
23	0.00626656850273689\\
24	0.00626656850273689\\
25	0.00626656850273689\\
26	0.00626656850273689\\
27	0.00626656850273689\\
28	0.00626656850273689\\
29	0.00626656850273689\\
30	0.00626656850273689\\
31	0.00626656850273689\\
32	0.00626656850273689\\
33	0.00626656850273689\\
34	0.00626656850273689\\
35	0.00626656850273689\\
36	0.00626656850273689\\
37	0.00626656850273689\\
38	0.00626656850273689\\
39	0.00626656850273689\\
40	0.00626656850273689\\
41	0.00626656850273689\\
42	0.00626656850273689\\
43	0.00626656850273689\\
44	0.00626656850273689\\
45	0.00626656850273689\\
46	0.00626656850273689\\
47	0.00626656850273689\\
48	0.00626656850273689\\
49	0.00626656850273689\\
50	0.00626656850273689\\
51	0.00626656850273689\\
52	0.00626656850273689\\
53	0.00626656850273689\\
54	0.00626656850273689\\
55	0.00626656850273689\\
56	0.00626656850273689\\
57	0.00626656850273689\\
58	0.00626656850273689\\
59	0.00626656850273689\\
60	0.00626656850273689\\
61	0.00626656850273689\\
62	0.00626656850273689\\
63	0.00626656850273689\\
64	0.00626656850273689\\
65	0.00626656850273689\\
66	0.00626656850273689\\
67	0.00626656850273689\\
68	0.00626656850273689\\
69	0.00626656850273689\\
70	0.00626656850273689\\
71	0.00626656850273689\\
72	0.00626656850273689\\
73	0.00626656850273689\\
74	0.00626656850273689\\
75	0.00626656850273689\\
76	0.00626656850273689\\
77	0.00626656850273689\\
78	0.00626656850273689\\
79	0.00626656850273689\\
80	0.00626656850273689\\
81	0.00626656850273689\\
82	0.00626656850273689\\
83	0.00626656850273689\\
84	0.00626656850273689\\
85	0.00626656850273689\\
86	0.00626656850273689\\
87	0.00626656850273689\\
88	0.00626656850273689\\
89	0.00626656850273689\\
90	0.00626656850273689\\
91	0.00626656850273689\\
92	0.00626656850273689\\
93	0.00626656850273689\\
94	0.00626656850273689\\
95	0.00626656850273689\\
96	0.00626656850273689\\
97	0.00626656850273689\\
98	0.00626656850273689\\
99	0.00626656850273689\\
100	0.00626656850273689\\
101	0.00626656850273689\\
102	0.00626656850273689\\
103	0.00626656850273689\\
104	0.00626656850273689\\
105	0.00626656850273689\\
106	0.00626656850273689\\
107	0.00626656850273689\\
108	0.00626656850273689\\
109	0.00626656850273689\\
110	0.00626656850273689\\
111	0.00626656850273689\\
112	0.00626656850273689\\
113	0.00626656850273689\\
114	0.00626656850273689\\
115	0.00626656850273689\\
116	0.00626656850273689\\
117	0.00626656850273689\\
118	0.00626656850273689\\
119	0.00626656850273689\\
120	0.00626656850273689\\
121	0.00626656850273689\\
122	0.00626656850273689\\
123	0.00626656850273689\\
124	0.00626656850273689\\
125	0.00626656850273689\\
126	0.00626656850273689\\
127	0.00626656850273689\\
128	0.00626656850273689\\
129	0.00626656850273689\\
130	0.00626656850273689\\
131	0.00626656850273689\\
132	0.00626656850273689\\
133	0.00626656850273689\\
134	0.00626656850273689\\
135	0.00626656850273689\\
136	0.00626656850273689\\
137	0.00626656850273689\\
138	0.00626656850273689\\
139	0.00626656850273689\\
140	0.00626656850273689\\
141	0.00626656850273689\\
142	0.00626656850273689\\
143	0.00626656850273689\\
144	0.00626656850273689\\
145	0.00626656850273689\\
146	0.00626656850273689\\
147	0.00626656850273689\\
148	0.00626656850273689\\
149	0.00626656850273689\\
150	0.00626656850273689\\
151	0.00626656850273689\\
152	0.00626656850273689\\
153	0.00626656850273689\\
154	0.00626656850273689\\
155	0.00626656850273689\\
156	0.00626656850273689\\
157	0.00626656850273689\\
158	0.00626656850273689\\
159	0.00626656850273689\\
160	0.00626656850273689\\
161	0.00626656850273689\\
162	0.00626656850273689\\
163	0.00626656850273689\\
164	0.00626656850273689\\
165	0.00626656850273689\\
166	0.00626656850273689\\
167	0.00626656850273689\\
168	0.00626656850273689\\
169	0.00626656850273689\\
170	0.00626656850273689\\
171	0.00626656850273689\\
172	0.00626656850273689\\
173	0.00626656850273689\\
174	0.00626656850273689\\
175	0.00626656850273689\\
176	0.00626656850273689\\
177	0.00626656850273689\\
178	0.00626656850273689\\
179	0.00626656850273689\\
180	0.00626656850273689\\
181	0.00626656850273689\\
182	0.00626656850273689\\
183	0.00626656850273689\\
184	0.00626656850273689\\
185	0.00626656850273689\\
186	0.00626656850273689\\
187	0.00626656850273689\\
188	0.00626656850273689\\
189	0.00626656850273689\\
190	0.00626656850273689\\
191	0.00626656850273689\\
192	0.00626656850273689\\
193	0.00626656850273689\\
194	0.00626656850273689\\
195	0.00626656850273689\\
196	0.00626656850273689\\
197	0.00626656850273689\\
198	0.00626656850273689\\
199	0.00626656850273689\\
200	0.00626656850273689\\
201	0.00626656850273689\\
202	0.00626656850273689\\
203	0.00626656850273689\\
204	0.00626656850273689\\
205	0.00626656850273689\\
206	0.00626656850273689\\
207	0.00626656850273689\\
208	0.00626656850273689\\
209	0.00626656850273689\\
210	0.00626656850273689\\
211	0.00626656850273689\\
212	0.00626656850273689\\
213	0.00626656850273689\\
214	0.00626656850273689\\
215	0.00626656850273689\\
216	0.00626656850273689\\
217	0.00626656850273689\\
218	0.00626656850273689\\
219	0.00626656850273689\\
220	0.00626656850273689\\
221	0.00626656850273689\\
222	0.00626656850273689\\
223	0.00626656850273689\\
224	0.00626656850273689\\
225	0.00626656850273689\\
226	0.00626656850273689\\
227	0.00626656850273689\\
228	0.00626656850273689\\
229	0.00626656850273689\\
230	0.00626656850273689\\
231	0.00626656850273689\\
232	0.00626656850273689\\
233	0.00626656850273689\\
234	0.00626656850273689\\
235	0.00626656850273689\\
236	0.00626656850273689\\
237	0.00626656850273689\\
238	0.00626656850273689\\
239	0.00626656850273689\\
240	0.00626656850273689\\
241	0.00626656850273689\\
242	0.00626656850273689\\
243	0.00626656850273689\\
244	0.00626656850273689\\
245	0.00626656850273689\\
246	0.00626656850273689\\
247	0.00626656850273689\\
248	0.00626656850273689\\
249	0.00626656850273689\\
250	0.00626656850273689\\
251	0.00626656850273689\\
252	0.00626656850273689\\
253	0.00626656850273689\\
254	0.00626656850273689\\
255	0.00626656850273689\\
256	0.00626656850273689\\
257	0.00626656850273689\\
258	0.00626656850273689\\
259	0.00626656850273689\\
260	0.00626656850273689\\
261	0.00626656850273689\\
262	0.00626656850273689\\
263	0.00626656850273689\\
264	0.00626656850273689\\
265	0.00626656850273689\\
266	0.00626656850273689\\
267	0.00626656850273689\\
268	0.00626656850273689\\
269	0.00626656850273689\\
270	0.00626656850273689\\
271	0.00626656850273689\\
272	0.00626656850273689\\
273	0.00626656850273689\\
274	0.00626656850273689\\
275	0.00626656850273689\\
276	0.00626656850273689\\
277	0.00626656850273689\\
278	0.00626656850273689\\
279	0.00626656850273689\\
280	0.00626656850273689\\
281	0.00626656850273689\\
282	0.00626656850273689\\
283	0.00626656850273689\\
284	0.00626656850273689\\
285	0.00626656850273689\\
286	0.00626656850273689\\
287	0.00626656850273689\\
288	0.00626656850273689\\
289	0.00626656850273689\\
290	0.00626656850273689\\
291	0.00626656850273689\\
292	0.00626656850273689\\
293	0.00626656850273689\\
294	0.00626656850273689\\
295	0.00626656850273689\\
296	0.00626656850273689\\
297	0.00626656850273689\\
298	0.00626656850273689\\
299	0.00626656850273689\\
300	0.00626656850273689\\
301	0.00626656850273689\\
302	0.00626656850273689\\
303	0.00626656850273689\\
304	0.00626656850273689\\
305	0.00626656850273689\\
306	0.00626656850273689\\
307	0.00626656850273689\\
308	0.00626656850273689\\
309	0.00626656850273689\\
310	0.00626656850273689\\
311	0.00626656850273689\\
312	0.00626656850273689\\
313	0.00626656850273689\\
314	0.00626656850273689\\
315	0.00626656850273689\\
316	0.00626656850273689\\
317	0.00626656850273689\\
318	0.00626656850273689\\
319	0.00626656850273689\\
320	0.00626656850273689\\
321	0.00626656850273689\\
322	0.00626656850273689\\
323	0.00626656850273689\\
324	0.00626656850273689\\
325	0.00626656850273689\\
326	0.00626656850273689\\
327	0.00626656850273689\\
328	0.00626656850273689\\
329	0.00626656850273689\\
330	0.00626656850273689\\
331	0.00626656850273689\\
332	0.00626656850273689\\
333	0.00626656850273689\\
334	0.00626656850273689\\
335	0.00626656850273689\\
336	0.00626656850273689\\
337	0.00626656850273689\\
338	0.00626656850273689\\
339	0.00626656850273689\\
340	0.00626656850273689\\
341	0.00626656850273689\\
342	0.00626656850273689\\
343	0.00626656850273689\\
344	0.00626656850273689\\
345	0.00626656850273689\\
346	0.00626656850273689\\
347	0.00626656850273689\\
348	0.00626656850273689\\
349	0.00626656850273689\\
350	0.00626656850273689\\
351	0.00626656850273689\\
352	0.00626656850273689\\
353	0.00626656850273689\\
354	0.00626656850273689\\
355	0.00626656850273689\\
356	0.00626656850273689\\
357	0.00626656850273689\\
358	0.00626656850273689\\
359	0.00626656850273689\\
360	0.00626656850273689\\
361	0.00626656850273689\\
362	0.00626656850273689\\
363	0.00626656850273689\\
364	0.00626656850273689\\
365	0.00626656850273689\\
366	0.00626656850273689\\
367	0.00626656850273689\\
368	0.00626656850273689\\
369	0.00626656850273689\\
370	0.00626656850273689\\
371	0.00626656850273689\\
372	0.00626656850273689\\
373	0.00626656850273689\\
374	0.00626656850273689\\
375	0.00626656850273689\\
376	0.00626656850273689\\
377	0.00626656850273689\\
378	0.00626656850273689\\
379	0.00626656850273689\\
380	0.00626656850273689\\
381	0.00626656850273689\\
382	0.00626656850273689\\
383	0.00626656850273689\\
384	0.00626656850273689\\
385	0.00626656850273689\\
386	0.00626656850273689\\
387	0.00626656850273689\\
388	0.00626656850273689\\
389	0.00626656850273689\\
390	0.00626656850273689\\
391	0.00626656850273689\\
392	0.00626656850273689\\
393	0.00626656850273689\\
394	0.00626656850273689\\
395	0.00626656850273689\\
396	0.00626656850273689\\
397	0.00626656850273689\\
398	0.00626656850273689\\
399	0.00626656850273689\\
400	0.00626656850273689\\
401	0.00626656850273689\\
402	0.00626656850273689\\
403	0.00626656850273689\\
404	0.00626656850273689\\
405	0.00626656850273689\\
406	0.00626656850273689\\
407	0.00626656850273689\\
408	0.00626656850273689\\
409	0.00626656850273689\\
410	0.00626656850273689\\
411	0.00626656850273689\\
412	0.00626656850273689\\
413	0.00626656850273689\\
414	0.00626656850273689\\
415	0.00626656850273689\\
416	0.00626656850273689\\
417	0.00626656850273689\\
418	0.00626656850273689\\
419	0.00626656850273689\\
420	0.00626656850273689\\
421	0.00626656850273689\\
422	0.00626656850273689\\
423	0.00626656850273689\\
424	0.00626656850273689\\
425	0.00626656850273689\\
426	0.00626656850273689\\
427	0.00626656850273689\\
428	0.00626656850273689\\
429	0.00626656850273689\\
430	0.00626656850273689\\
431	0.00626656850273689\\
432	0.00626656850273689\\
433	0.00626656850273689\\
434	0.00626656850273689\\
435	0.00626656850273689\\
436	0.00626656850273689\\
437	0.00626656850273689\\
438	0.00626656850273689\\
439	0.00626656850273689\\
440	0.00626656850273689\\
441	0.00626656850273689\\
442	0.00626656850273689\\
443	0.00626656850273689\\
444	0.00626656850273689\\
445	0.00626656850273689\\
446	0.00626656850273689\\
447	0.00626656850273689\\
448	0.00626656850273689\\
449	0.00626656850273689\\
450	0.00626656850273689\\
451	0.00626656850273689\\
452	0.00626656850273689\\
453	0.00626656850273689\\
454	0.00626656850273689\\
455	0.00626656850273689\\
456	0.00626656850273689\\
457	0.00626656850273689\\
458	0.00626656850273689\\
459	0.00626656850273689\\
460	0.00626656850273689\\
461	0.00626656850273689\\
462	0.00626656850273689\\
463	0.00626656850273689\\
464	0.00626656850273689\\
465	0.00626656850273689\\
466	0.00626656850273689\\
467	0.00626656850273689\\
468	0.00626656850273689\\
469	0.00626656850273689\\
470	0.00626656850273689\\
471	0.00626656850273689\\
472	0.00626656850273689\\
473	0.00626656850273689\\
474	0.00626656850273689\\
475	0.00626656850273689\\
476	0.00626656850273689\\
477	0.00626656850273689\\
478	0.00626656850273689\\
479	0.00626656850273689\\
480	0.00626656850273689\\
481	0.00626656850273689\\
482	0.00626656850273689\\
483	0.00626656850273689\\
484	0.00626656850273689\\
485	0.00626656850273689\\
486	0.00626656850273689\\
487	0.00626656850273689\\
488	0.00626656850273689\\
489	0.00626656850273689\\
490	0.00626656850273689\\
491	0.00626656850273689\\
492	0.00626656850273689\\
493	0.00626656850273689\\
494	0.00626656850273689\\
495	0.00626656850273689\\
496	0.00626656850273689\\
497	0.00626656850273689\\
498	0.00626656850273689\\
499	0.00626656850273689\\
500	0.00626656850273689\\
501	0.00626656850273689\\
502	0.00626656850273689\\
503	0.00626656850273689\\
504	0.00626656850273689\\
505	0.00626656850273689\\
506	0.00626656850273689\\
507	0.00626656850273689\\
508	0.00626656850273689\\
509	0.00626656850273689\\
510	0.00626656850273689\\
511	0.00626656850273689\\
512	0.00626656850273689\\
513	0.00626656850273689\\
514	0.00626656850273689\\
515	0.00626656850273689\\
516	0.00626656850273689\\
517	0.00626656850273689\\
518	0.00626656850273689\\
519	0.00626656850273689\\
520	0.00626656850273689\\
521	0.00626656850273689\\
522	0.00626656850273689\\
523	0.00626656850273689\\
524	0.00626656850273689\\
525	0.00626656850273689\\
526	0.00626656850273689\\
527	0.00626656850273689\\
528	0.00626656850273689\\
529	0.00626656850273689\\
530	0.00626656850273689\\
531	0.00626656850273689\\
532	0.00626656850273689\\
533	0.00626656850273689\\
534	0.00626656850273689\\
535	0.00626656850273689\\
536	0.00626656850273689\\
537	0.00626656850273689\\
538	0.00626656850273689\\
539	0.00626656850273689\\
540	0.00626656850273689\\
541	0.00626656850273689\\
542	0.00626656850273689\\
543	0.00626656850273689\\
544	0.00626656850273689\\
545	0.00626656850273689\\
546	0.00626656850273689\\
547	0.00626656850273689\\
548	0.00626656850273689\\
549	0.00626656850273689\\
550	0.00626656850273689\\
551	0.00626656850273689\\
552	0.00626656850273689\\
553	0.00626656850273689\\
554	0.00626656850273689\\
555	0.00626656850273689\\
556	0.00626656850273689\\
557	0.00626656850273689\\
558	0.00626656850273689\\
559	0.00626656850273689\\
560	0.00626656850273689\\
561	0.00626656850273689\\
562	0.00626656850273689\\
563	0.00626656850273689\\
564	0.00626656850273689\\
565	0.0062822687199181\\
566	0.00639112209030057\\
567	0.00650258018340242\\
568	0.00662081992140465\\
569	0.00674295056371734\\
570	0.00686670043208088\\
571	0.0069920636388678\\
572	0.00711629095688794\\
573	0.00724056915523556\\
574	0.00743208496156276\\
575	0.00762095467871718\\
576	0.00780725692531322\\
577	0.00799086507431176\\
578	0.00814787332658741\\
579	0.00830507905952759\\
580	0.0084565674425073\\
581	0.00858561074290109\\
582	0.00871475916838674\\
583	0.00884326036092976\\
584	0.00897128106394158\\
585	0.0090995281368202\\
586	0.00923067908272993\\
587	0.00937316043200138\\
588	0.00952438500706335\\
589	0.00967455307485254\\
590	0.0098231620292525\\
591	0.00997019543938102\\
592	0.0101168435362631\\
593	0.0102672147782407\\
594	0.0104258886577547\\
595	0.0106039016956345\\
596	0.0108298745780857\\
597	0.0111718112421279\\
598	0.0113889057314154\\
599	0\\
600	0\\
};
\addplot [color=mycolor14,solid,forget plot]
  table[row sep=crcr]{%
1	0.00367653017611331\\
2	0.00367653056165089\\
3	0.00367653095416569\\
4	0.00367653135378372\\
5	0.00367653176063332\\
6	0.00367653217484509\\
7	0.00367653259655201\\
8	0.00367653302588946\\
9	0.00367653346299522\\
10	0.0036765339080096\\
11	0.00367653436107539\\
12	0.00367653482233799\\
13	0.00367653529194539\\
14	0.00367653577004825\\
15	0.00367653625679995\\
16	0.00367653675235662\\
17	0.0036765372568772\\
18	0.0036765377705235\\
19	0.00367653829346022\\
20	0.00367653882585503\\
21	0.00367653936787864\\
22	0.0036765399197048\\
23	0.00367654048151039\\
24	0.00367654105347547\\
25	0.00367654163578336\\
26	0.00367654222862064\\
27	0.00367654283217727\\
28	0.00367654344664661\\
29	0.0036765440722255\\
30	0.00367654470911431\\
31	0.00367654535751704\\
32	0.00367654601764131\\
33	0.00367654668969849\\
34	0.00367654737390376\\
35	0.00367654807047615\\
36	0.00367654877963863\\
37	0.00367654950161815\\
38	0.00367655023664578\\
39	0.00367655098495669\\
40	0.00367655174679032\\
41	0.00367655252239035\\
42	0.0036765533120049\\
43	0.00367655411588647\\
44	0.00367655493429216\\
45	0.00367655576748364\\
46	0.00367655661572728\\
47	0.00367655747929424\\
48	0.00367655835846054\\
49	0.00367655925350714\\
50	0.00367656016472004\\
51	0.00367656109239038\\
52	0.00367656203681451\\
53	0.00367656299829409\\
54	0.0036765639771362\\
55	0.0036765649736534\\
56	0.00367656598816389\\
57	0.00367656702099152\\
58	0.00367656807246598\\
59	0.00367656914292285\\
60	0.00367657023270374\\
61	0.00367657134215635\\
62	0.00367657247163462\\
63	0.00367657362149883\\
64	0.00367657479211571\\
65	0.00367657598385855\\
66	0.00367657719710732\\
67	0.0036765784322488\\
68	0.0036765796896767\\
69	0.00367658096979173\\
70	0.00367658227300182\\
71	0.00367658359972218\\
72	0.00367658495037544\\
73	0.00367658632539179\\
74	0.00367658772520912\\
75	0.00367658915027315\\
76	0.00367659060103755\\
77	0.00367659207796413\\
78	0.00367659358152294\\
79	0.00367659511219242\\
80	0.00367659667045958\\
81	0.00367659825682013\\
82	0.00367659987177861\\
83	0.0036766015158486\\
84	0.00367660318955285\\
85	0.00367660489342344\\
86	0.00367660662800196\\
87	0.00367660839383966\\
88	0.00367661019149763\\
89	0.00367661202154699\\
90	0.00367661388456905\\
91	0.00367661578115548\\
92	0.00367661771190854\\
93	0.00367661967744119\\
94	0.00367662167837737\\
95	0.00367662371535214\\
96	0.00367662578901186\\
97	0.00367662790001445\\
98	0.00367663004902955\\
99	0.00367663223673873\\
100	0.00367663446383574\\
101	0.00367663673102667\\
102	0.0036766390390302\\
103	0.00367664138857783\\
104	0.00367664378041408\\
105	0.00367664621529675\\
106	0.00367664869399714\\
107	0.00367665121730028\\
108	0.00367665378600518\\
109	0.00367665640092511\\
110	0.00367665906288778\\
111	0.00367666177273567\\
112	0.00367666453132624\\
113	0.00367666733953224\\
114	0.00367667019824192\\
115	0.00367667310835936\\
116	0.00367667607080472\\
117	0.00367667908651455\\
118	0.00367668215644204\\
119	0.00367668528155735\\
120	0.00367668846284789\\
121	0.00367669170131865\\
122	0.00367669499799247\\
123	0.00367669835391038\\
124	0.00367670177013193\\
125	0.00367670524773551\\
126	0.00367670878781866\\
127	0.00367671239149845\\
128	0.00367671605991178\\
129	0.00367671979421577\\
130	0.00367672359558811\\
131	0.00367672746522738\\
132	0.00367673140435348\\
133	0.00367673541420797\\
134	0.00367673949605447\\
135	0.00367674365117902\\
136	0.00367674788089052\\
137	0.00367675218652109\\
138	0.00367675656942653\\
139	0.00367676103098667\\
140	0.00367676557260587\\
141	0.00367677019571339\\
142	0.00367677490176387\\
143	0.00367677969223775\\
144	0.00367678456864176\\
145	0.00367678953250935\\
146	0.00367679458540116\\
147	0.00367679972890555\\
148	0.00367680496463902\\
149	0.00367681029424676\\
150	0.00367681571940314\\
151	0.00367682124181221\\
152	0.00367682686320823\\
153	0.00367683258535624\\
154	0.00367683841005255\\
155	0.00367684433912532\\
156	0.00367685037443511\\
157	0.00367685651787547\\
158	0.0036768627713735\\
159	0.00367686913689048\\
160	0.00367687561642242\\
161	0.00367688221200071\\
162	0.00367688892569274\\
163	0.00367689575960252\\
164	0.00367690271587136\\
165	0.00367690979667849\\
166	0.00367691700424174\\
167	0.00367692434081824\\
168	0.00367693180870512\\
169	0.00367693941024017\\
170	0.00367694714780259\\
171	0.00367695502381376\\
172	0.00367696304073789\\
173	0.00367697120108288\\
174	0.00367697950740102\\
175	0.00367698796228979\\
176	0.0036769965683927\\
177	0.00367700532840006\\
178	0.00367701424504982\\
179	0.00367702332112841\\
180	0.00367703255947162\\
181	0.00367704196296542\\
182	0.00367705153454692\\
183	0.00367706127720524\\
184	0.0036770711939824\\
185	0.00367708128797431\\
186	0.00367709156233171\\
187	0.00367710202026111\\
188	0.00367711266502583\\
189	0.00367712349994695\\
190	0.0036771345284044\\
191	0.00367714575383793\\
192	0.00367715717974825\\
193	0.00367716880969803\\
194	0.00367718064731309\\
195	0.00367719269628342\\
196	0.00367720496036441\\
197	0.00367721744337796\\
198	0.00367723014921369\\
199	0.00367724308183012\\
200	0.0036772562452559\\
201	0.00367726964359109\\
202	0.00367728328100837\\
203	0.0036772971617544\\
204	0.00367731129015109\\
205	0.00367732567059696\\
206	0.00367734030756851\\
207	0.00367735520562158\\
208	0.0036773703693928\\
209	0.00367738580360103\\
210	0.0036774015130488\\
211	0.00367741750262384\\
212	0.00367743377730055\\
213	0.00367745034214161\\
214	0.00367746720229953\\
215	0.00367748436301824\\
216	0.00367750182963476\\
217	0.00367751960758083\\
218	0.00367753770238464\\
219	0.00367755611967255\\
220	0.00367757486517082\\
221	0.00367759394470745\\
222	0.00367761336421396\\
223	0.00367763312972729\\
224	0.00367765324739164\\
225	0.00367767372346047\\
226	0.00367769456429838\\
227	0.00367771577638319\\
228	0.0036777373663079\\
229	0.00367775934078283\\
230	0.00367778170663769\\
231	0.00367780447082373\\
232	0.00367782764041598\\
233	0.00367785122261541\\
234	0.00367787522475125\\
235	0.00367789965428329\\
236	0.00367792451880426\\
237	0.00367794982604216\\
238	0.00367797558386281\\
239	0.00367800180027224\\
240	0.0036780284834193\\
241	0.00367805564159819\\
242	0.00367808328325112\\
243	0.003678111416971\\
244	0.00367814005150414\\
245	0.00367816919575303\\
246	0.00367819885877921\\
247	0.00367822904980608\\
248	0.00367825977822194\\
249	0.00367829105358287\\
250	0.00367832288561585\\
251	0.00367835528422183\\
252	0.00367838825947891\\
253	0.00367842182164553\\
254	0.00367845598116377\\
255	0.00367849074866265\\
256	0.0036785261349616\\
257	0.00367856215107382\\
258	0.00367859880820991\\
259	0.00367863611778135\\
260	0.00367867409140425\\
261	0.00367871274090302\\
262	0.00367875207831416\\
263	0.00367879211589012\\
264	0.00367883286610326\\
265	0.00367887434164981\\
266	0.00367891655545394\\
267	0.00367895952067197\\
268	0.0036790032506965\\
269	0.00367904775916076\\
270	0.0036790930599429\\
271	0.00367913916717048\\
272	0.00367918609522497\\
273	0.00367923385874648\\
274	0.00367928247263846\\
275	0.00367933195207238\\
276	0.0036793823124926\\
277	0.00367943356962131\\
278	0.00367948573946352\\
279	0.00367953883831222\\
280	0.00367959288275351\\
281	0.00367964788967194\\
282	0.00367970387625579\\
283	0.00367976086000261\\
284	0.00367981885872467\\
285	0.00367987789055463\\
286	0.0036799379739512\\
287	0.00367999912770496\\
288	0.0036800613709442\\
289	0.0036801247231409\\
290	0.00368018920411674\\
291	0.0036802548340492\\
292	0.00368032163347778\\
293	0.00368038962331024\\
294	0.00368045882482892\\
295	0.00368052925969719\\
296	0.00368060094996587\\
297	0.00368067391807978\\
298	0.00368074818688432\\
299	0.00368082377963212\\
300	0.00368090071998967\\
301	0.0036809790320441\\
302	0.00368105874030985\\
303	0.00368113986973549\\
304	0.00368122244571041\\
305	0.00368130649407168\\
306	0.00368139204111075\\
307	0.00368147911358031\\
308	0.0036815677387011\\
309	0.00368165794416871\\
310	0.00368174975816006\\
311	0.00368184320933941\\
312	0.00368193832686427\\
313	0.00368203514039226\\
314	0.00368213368008714\\
315	0.00368223397662459\\
316	0.00368233606119779\\
317	0.00368243996552272\\
318	0.00368254572184307\\
319	0.00368265336293489\\
320	0.00368276292211071\\
321	0.00368287443322316\\
322	0.00368298793066811\\
323	0.00368310344938708\\
324	0.00368322102486893\\
325	0.00368334069315084\\
326	0.00368346249081827\\
327	0.00368358645500425\\
328	0.00368371262338791\\
329	0.00368384103419306\\
330	0.00368397172618796\\
331	0.00368410473868894\\
332	0.00368424011157142\\
333	0.00368437788529088\\
334	0.00368451810090468\\
335	0.00368466080004777\\
336	0.00368480602473909\\
337	0.00368495381687434\\
338	0.00368510421777288\\
339	0.00368525726973523\\
340	0.00368541301885574\\
341	0.00368557151543321\\
342	0.00368573281946763\\
343	0.00368589701275041\\
344	0.00368606422228157\\
345	0.00368623465440683\\
346	0.00368640860434799\\
347	0.00368658630102444\\
348	0.00368676741678235\\
349	0.00368695157816927\\
350	0.00368713883560509\\
351	0.00368732924026719\\
352	0.00368752284409484\\
353	0.00368771969979251\\
354	0.0036879198608321\\
355	0.00368812338145575\\
356	0.00368833031668297\\
357	0.00368854072231974\\
358	0.00368875465495341\\
359	0.00368897217195451\\
360	0.00368919333147824\\
361	0.00368941819246552\\
362	0.00368964681464368\\
363	0.00368987925852689\\
364	0.00369011558541633\\
365	0.00369035585740017\\
366	0.00369060013735352\\
367	0.0036908484889384\\
368	0.00369110097660387\\
369	0.00369135766558651\\
370	0.00369161862191167\\
371	0.0036918839123961\\
372	0.00369215360465332\\
373	0.00369242776710292\\
374	0.00369270646898323\\
375	0.00369298978035906\\
376	0.00369327777210676\\
377	0.00369357051587518\\
378	0.00369386808408436\\
379	0.00369417054985217\\
380	0.0036944779868713\\
381	0.00369479046928054\\
382	0.00369510807174475\\
383	0.00369543087032278\\
384	0.00369575894506103\\
385	0.00369609238414613\\
386	0.00369643128324265\\
387	0.00369677572714751\\
388	0.00369712579944964\\
389	0.00369748158607326\\
390	0.00369784317553442\\
391	0.00369821065913121\\
392	0.00369858413089656\\
393	0.00369896368688419\\
394	0.0036993494227463\\
395	0.00369974142716262\\
396	0.00370013976565961\\
397	0.00370054444340995\\
398	0.00370095532587845\\
399	0.00370137198739635\\
400	0.00370179347857104\\
401	0.00370221815351734\\
402	0.00370264418994138\\
403	0.00370307214811652\\
404	0.00370350814092187\\
405	0.00370395231912532\\
406	0.003704404840081\\
407	0.00370486586827905\\
408	0.00370533557584055\\
409	0.00370581414210759\\
410	0.00370630175176115\\
411	0.0037067985954751\\
412	0.00370730488679909\\
413	0.00370782084664511\\
414	0.00370834670366121\\
415	0.00370888269462724\\
416	0.00370942906487551\\
417	0.00370998606874464\\
418	0.0037105539700848\\
419	0.00371113304284162\\
420	0.00371172357172448\\
421	0.00371232585285225\\
422	0.00371294019413211\\
423	0.00371356691562249\\
424	0.00371420635060837\\
425	0.00371485884636767\\
426	0.00371552476499325\\
427	0.00371620448427976\\
428	0.00371689839869196\\
429	0.00371760692044713\\
430	0.00371833048077769\\
431	0.00371906953150134\\
432	0.00371982454712133\\
433	0.00372059602774614\\
434	0.00372138450285962\\
435	0.00372219053442274\\
436	0.00372301471297645\\
437	0.0037238576327883\\
438	0.00372471984803585\\
439	0.00372560194682476\\
440	0.00372650463129142\\
441	0.0037274286407001\\
442	0.00372837475378901\\
443	0.00372934379128035\\
444	0.00373033661855815\\
445	0.00373135414860738\\
446	0.00373239734602282\\
447	0.00373346723597689\\
448	0.00373456492957018\\
449	0.00373569167865208\\
450	0.00373684902585686\\
451	0.00373803921625152\\
452	0.00373926631493424\\
453	0.00374053920392484\\
454	0.00374187948124994\\
455	0.00374334189873179\\
456	0.00374506636686195\\
457	0.00374739713303209\\
458	0.00375123448627122\\
459	0.00375547858829391\\
460	0.00375981539069011\\
461	0.00376426624355029\\
462	0.00376884994748991\\
463	0.00377351286961032\\
464	0.00377825610449833\\
465	0.00378308290804518\\
466	0.00378799698411069\\
467	0.00379300258192583\\
468	0.0037981044831321\\
469	0.00380330734994562\\
470	0.00380861335918627\\
471	0.00381402561349215\\
472	0.00381954488625546\\
473	0.00382516478904953\\
474	0.00383086468800139\\
475	0.00383660418703059\\
476	0.00384234651524792\\
477	0.00384818144141559\\
478	0.00385421363574154\\
479	0.00386045807863611\\
480	0.0038669329136766\\
481	0.00387366255712396\\
482	0.00388068579283263\\
483	0.00388807688206091\\
484	0.00389599855758656\\
485	0.00390481807810079\\
486	0.00391475397534097\\
487	0.00392486785415363\\
488	0.00393516434445483\\
489	0.00394564821273966\\
490	0.00395632447012295\\
491	0.00396719847114864\\
492	0.003978276296484\\
493	0.0039895669600041\\
494	0.004001083731031\\
495	0.00401282028356549\\
496	0.00402478305185202\\
497	0.00403698064252631\\
498	0.00404942382408288\\
499	0.00406212004868167\\
500	0.00407507742575319\\
501	0.00408830519022187\\
502	0.00410181481165771\\
503	0.00411562224974021\\
504	0.00412974941077046\\
505	0.00414419721295197\\
506	0.00415900509793489\\
507	0.0041742699151639\\
508	0.00419019456615506\\
509	0.00420661104572812\\
510	0.00422326330758944\\
511	0.00424026539807649\\
512	0.004257888508676\\
513	0.00427660539481829\\
514	0.00429584792052824\\
515	0.00431525143541864\\
516	0.00433480567499767\\
517	0.00435447933183099\\
518	0.00437418886870854\\
519	0.00439372433973917\\
520	0.00441259873518383\\
521	0.00442986415347352\\
522	0.00444459825959609\\
523	0.00445986664630336\\
524	0.00447619869660579\\
525	0.00449523245035704\\
526	0.00452202013800674\\
527	0.00458256534737671\\
528	0.00464972517505194\\
529	0.00471971230707819\\
530	0.00479299280265683\\
531	0.00487037494914992\\
532	0.00495353421253993\\
533	0.0050506672768503\\
534	0.00515796174870662\\
535	0.00526773094624111\\
536	0.00538009872538758\\
537	0.00549520586632014\\
538	0.00561322008825465\\
539	0.005734377560753\\
540	0.00585911618037913\\
541	0.00598854183724758\\
542	0.00612722598213259\\
543	0.00626810047367561\\
544	0.0064112384690035\\
545	0.00655671821578339\\
546	0.00670462342312696\\
547	0.00685504635426182\\
548	0.00700810064875001\\
549	0.00716396460103585\\
550	0.00732302798753163\\
551	0.00748639253875404\\
552	0.00765758657047315\\
553	0.0078491920627388\\
554	0.0080466620690786\\
555	0.00824698686594431\\
556	0.00845028978904494\\
557	0.00865645048389266\\
558	0.00886543892503706\\
559	0.00907723732830055\\
560	0.00929166693914143\\
561	0.00950801999602768\\
562	0.00972414417303286\\
563	0.00993396175185058\\
564	0.010120992459127\\
565	0.0102532054663847\\
566	0.0103229888747475\\
567	0.0103922951339935\\
568	0.010455303889071\\
569	0.0105180462780884\\
570	0.0105805760211889\\
571	0.0106428416486077\\
572	0.0107042883450541\\
573	0.010763259951551\\
574	0.0108311877217053\\
575	0.0108998445256091\\
576	0.0109692295188531\\
577	0.0110393269431159\\
578	0.0111100636614941\\
579	0.0111814896114134\\
580	0.0112536358673312\\
581	0.011324914505972\\
582	0.0113967666038473\\
583	0.0114695405407929\\
584	0.0115432987726396\\
585	0.0116182821028575\\
586	0.0116962863978285\\
587	0.0117808462511651\\
588	0.0118879148142366\\
589	0.0120123879998692\\
590	0.0121374196899412\\
591	0.0122629658449694\\
592	0.0123891236352706\\
593	0.0125163895508632\\
594	0.0126463415783053\\
595	0.0127834870314329\\
596	0.0129402012848292\\
597	0.0131792323843585\\
598	0.0135751148335434\\
599	0\\
600	0\\
};
\addplot [color=mycolor15,solid,forget plot]
  table[row sep=crcr]{%
1	0.00554917154551039\\
2	0.00554917302704827\\
3	0.00554917453539954\\
4	0.00554917607104861\\
5	0.00554917763448863\\
6	0.00554917922622163\\
7	0.00554918084675867\\
8	0.00554918249662003\\
9	0.00554918417633534\\
10	0.0055491858864438\\
11	0.00554918762749428\\
12	0.00554918940004555\\
13	0.00554919120466645\\
14	0.00554919304193603\\
15	0.00554919491244378\\
16	0.0055491968167898\\
17	0.00554919875558499\\
18	0.00554920072945123\\
19	0.0055492027390216\\
20	0.00554920478494056\\
21	0.00554920686786417\\
22	0.00554920898846029\\
23	0.00554921114740879\\
24	0.00554921334540176\\
25	0.00554921558314375\\
26	0.00554921786135196\\
27	0.0055492201807565\\
28	0.00554922254210059\\
29	0.00554922494614082\\
30	0.00554922739364737\\
31	0.00554922988540426\\
32	0.00554923242220962\\
33	0.00554923500487589\\
34	0.00554923763423012\\
35	0.00554924031111422\\
36	0.00554924303638522\\
37	0.00554924581091554\\
38	0.00554924863559327\\
39	0.00554925151132244\\
40	0.00554925443902332\\
41	0.0055492574196327\\
42	0.00554926045410419\\
43	0.0055492635434085\\
44	0.00554926668853378\\
45	0.0055492698904859\\
46	0.0055492731502888\\
47	0.00554927646898475\\
48	0.00554927984763476\\
49	0.00554928328731886\\
50	0.00554928678913644\\
51	0.00554929035420661\\
52	0.00554929398366856\\
53	0.00554929767868188\\
54	0.00554930144042699\\
55	0.00554930527010542\\
56	0.00554930916894028\\
57	0.00554931313817658\\
58	0.00554931717908164\\
59	0.0055493212929455\\
60	0.0055493254810813\\
61	0.00554932974482571\\
62	0.00554933408553934\\
63	0.00554933850460718\\
64	0.005549343003439\\
65	0.00554934758346984\\
66	0.00554935224616042\\
67	0.00554935699299761\\
68	0.00554936182549489\\
69	0.00554936674519285\\
70	0.00554937175365962\\
71	0.00554937685249142\\
72	0.00554938204331301\\
73	0.00554938732777823\\
74	0.0055493927075705\\
75	0.00554939818440334\\
76	0.00554940376002093\\
77	0.00554940943619863\\
78	0.00554941521474355\\
79	0.0055494210974951\\
80	0.00554942708632558\\
81	0.00554943318314074\\
82	0.00554943938988042\\
83	0.00554944570851908\\
84	0.0055494521410665\\
85	0.00554945868956835\\
86	0.00554946535610684\\
87	0.00554947214280138\\
88	0.00554947905180921\\
89	0.00554948608532613\\
90	0.00554949324558711\\
91	0.00554950053486706\\
92	0.00554950795548146\\
93	0.00554951550978714\\
94	0.005549523200183\\
95	0.00554953102911074\\
96	0.00554953899905563\\
97	0.00554954711254728\\
98	0.00554955537216042\\
99	0.00554956378051571\\
100	0.00554957234028057\\
101	0.00554958105416994\\
102	0.00554958992494722\\
103	0.00554959895542506\\
104	0.00554960814846624\\
105	0.00554961750698458\\
106	0.00554962703394583\\
107	0.00554963673236861\\
108	0.0055496466053253\\
109	0.00554965665594304\\
110	0.00554966688740468\\
111	0.00554967730294975\\
112	0.00554968790587549\\
113	0.00554969869953787\\
114	0.00554970968735261\\
115	0.00554972087279625\\
116	0.00554973225940723\\
117	0.00554974385078697\\
118	0.00554975565060099\\
119	0.00554976766258006\\
120	0.00554977989052133\\
121	0.00554979233828953\\
122	0.00554980500981814\\
123	0.00554981790911062\\
124	0.00554983104024164\\
125	0.00554984440735834\\
126	0.00554985801468164\\
127	0.00554987186650748\\
128	0.00554988596720822\\
129	0.00554990032123394\\
130	0.00554991493311383\\
131	0.00554992980745759\\
132	0.00554994494895686\\
133	0.00554996036238665\\
134	0.00554997605260683\\
135	0.00554999202456361\\
136	0.00555000828329111\\
137	0.00555002483391284\\
138	0.00555004168164336\\
139	0.00555005883178983\\
140	0.00555007628975369\\
141	0.00555009406103226\\
142	0.00555011215122052\\
143	0.00555013056601277\\
144	0.00555014931120444\\
145	0.00555016839269379\\
146	0.00555018781648384\\
147	0.00555020758868412\\
148	0.00555022771551262\\
149	0.00555024820329768\\
150	0.00555026905847991\\
151	0.00555029028761426\\
152	0.00555031189737193\\
153	0.00555033389454252\\
154	0.00555035628603607\\
155	0.00555037907888518\\
156	0.00555040228024722\\
157	0.00555042589740652\\
158	0.00555044993777658\\
159	0.00555047440890239\\
160	0.00555049931846274\\
161	0.0055505246742726\\
162	0.00555055048428549\\
163	0.00555057675659599\\
164	0.00555060349944217\\
165	0.00555063072120818\\
166	0.00555065843042678\\
167	0.00555068663578204\\
168	0.00555071534611194\\
169	0.00555074457041114\\
170	0.00555077431783375\\
171	0.00555080459769613\\
172	0.00555083541947978\\
173	0.00555086679283425\\
174	0.00555089872758013\\
175	0.00555093123371205\\
176	0.00555096432140182\\
177	0.00555099800100148\\
178	0.00555103228304659\\
179	0.0055510671782594\\
180	0.00555110269755221\\
181	0.00555113885203072\\
182	0.00555117565299745\\
183	0.00555121311195524\\
184	0.00555125124061082\\
185	0.00555129005087839\\
186	0.00555132955488332\\
187	0.00555136976496592\\
188	0.00555141069368523\\
189	0.00555145235382288\\
190	0.00555149475838711\\
191	0.00555153792061673\\
192	0.00555158185398528\\
193	0.00555162657220513\\
194	0.0055516720892318\\
195	0.00555171841926823\\
196	0.00555176557676923\\
197	0.00555181357644591\\
198	0.00555186243327029\\
199	0.00555191216247993\\
200	0.00555196277958265\\
201	0.00555201430036137\\
202	0.00555206674087901\\
203	0.00555212011748346\\
204	0.00555217444681268\\
205	0.00555222974579992\\
206	0.00555228603167892\\
207	0.00555234332198933\\
208	0.00555240163458216\\
209	0.00555246098762535\\
210	0.00555252139960944\\
211	0.00555258288935336\\
212	0.00555264547601026\\
213	0.00555270917907356\\
214	0.00555277401838301\\
215	0.0055528400141309\\
216	0.0055529071868684\\
217	0.00555297555751199\\
218	0.00555304514735003\\
219	0.00555311597804942\\
220	0.00555318807166242\\
221	0.00555326145063358\\
222	0.00555333613780678\\
223	0.00555341215643244\\
224	0.00555348953017482\\
225	0.0055535682831195\\
226	0.00555364843978096\\
227	0.00555373002511031\\
228	0.00555381306450319\\
229	0.00555389758380775\\
230	0.0055539836093329\\
231	0.00555407116785656\\
232	0.00555416028663419\\
233	0.00555425099340741\\
234	0.0055543433164128\\
235	0.00555443728439086\\
236	0.00555453292659516\\
237	0.00555463027280163\\
238	0.00555472935331802\\
239	0.00555483019899356\\
240	0.0055549328412288\\
241	0.00555503731198561\\
242	0.0055551436437974\\
243	0.00555525186977948\\
244	0.00555536202363966\\
245	0.00555547413968905\\
246	0.00555558825285302\\
247	0.00555570439868238\\
248	0.00555582261336481\\
249	0.00555594293373644\\
250	0.00555606539729368\\
251	0.00555619004220528\\
252	0.00555631690732457\\
253	0.00555644603220201\\
254	0.00555657745709787\\
255	0.00555671122299522\\
256	0.00555684737161315\\
257	0.00555698594542021\\
258	0.0055571269876481\\
259	0.00555727054230569\\
260	0.00555741665419315\\
261	0.00555756536891654\\
262	0.00555771673290246\\
263	0.00555787079341314\\
264	0.00555802759856171\\
265	0.00555818719732783\\
266	0.00555834963957351\\
267	0.00555851497605933\\
268	0.00555868325846091\\
269	0.00555885453938563\\
270	0.00555902887238973\\
271	0.00555920631199553\\
272	0.00555938691370899\\
273	0.00555957073403767\\
274	0.00555975783050963\\
275	0.00555994826169223\\
276	0.00556014208721073\\
277	0.00556033936776763\\
278	0.00556054016516232\\
279	0.00556074454231111\\
280	0.00556095256326765\\
281	0.00556116429324352\\
282	0.00556137979862943\\
283	0.0055615991470165\\
284	0.00556182240721812\\
285	0.00556204964929201\\
286	0.00556228094456272\\
287	0.00556251636564436\\
288	0.0055627559864639\\
289	0.0055629998822846\\
290	0.00556324812972992\\
291	0.00556350080680767\\
292	0.00556375799293457\\
293	0.00556401976896115\\
294	0.0055642862171969\\
295	0.00556455742143578\\
296	0.00556483346698204\\
297	0.00556511444067629\\
298	0.00556540043092187\\
299	0.00556569152771148\\
300	0.00556598782265403\\
301	0.00556628940900166\\
302	0.00556659638167706\\
303	0.00556690883730083\\
304	0.00556722687421898\\
305	0.00556755059253054\\
306	0.00556788009411519\\
307	0.00556821548266091\\
308	0.00556855686369184\\
309	0.00556890434459633\\
310	0.00556925803465529\\
311	0.0055696180450699\\
312	0.0055699844889865\\
313	0.00557035748151957\\
314	0.00557073713978149\\
315	0.00557112358290785\\
316	0.00557151693208222\\
317	0.00557191731055991\\
318	0.00557232484369089\\
319	0.00557273965894149\\
320	0.0055731618859148\\
321	0.0055735916563695\\
322	0.00557402910423693\\
323	0.00557447436563612\\
324	0.00557492757888645\\
325	0.00557538888451763\\
326	0.00557585842527666\\
327	0.00557633634613128\\
328	0.00557682279426963\\
329	0.00557731791909589\\
330	0.00557782187222188\\
331	0.00557833480745608\\
332	0.00557885688079342\\
333	0.00557938825041474\\
334	0.00557992907671277\\
335	0.00558047952236603\\
336	0.00558103975243689\\
337	0.00558160993421012\\
338	0.00558219023571588\\
339	0.00558278082173439\\
340	0.00558338185697873\\
341	0.00558399351843381\\
342	0.00558461598996123\\
343	0.00558524947227382\\
344	0.00558589421064353\\
345	0.00558655056895177\\
346	0.00558721921073454\\
347	0.00558790143648038\\
348	0.00558859917276615\\
349	0.00558931119697032\\
350	0.00559003525001013\\
351	0.00559077153301133\\
352	0.00559152025019985\\
353	0.00559228160892518\\
354	0.00559305581967932\\
355	0.00559384309610832\\
356	0.00559464365501986\\
357	0.0055954577164114\\
358	0.0055962855035362\\
359	0.00559712724287495\\
360	0.00559798316414822\\
361	0.00559885350032749\\
362	0.00559973848764452\\
363	0.00560063836559943\\
364	0.00560155337696753\\
365	0.00560248376780521\\
366	0.005603429787455\\
367	0.00560439168855038\\
368	0.00560536972702051\\
369	0.00560636416209527\\
370	0.00560737525631138\\
371	0.00560840327551999\\
372	0.00560944848889809\\
373	0.00561051116896803\\
374	0.00561159159163486\\
375	0.00561269003625283\\
376	0.00561380678570863\\
377	0.00561494212642573\\
378	0.00561609634817915\\
379	0.00561726974429356\\
380	0.00561846261159394\\
381	0.0056196752502201\\
382	0.00562090796323909\\
383	0.00562216105619101\\
384	0.0056234348376587\\
385	0.00562472962480594\\
386	0.00562604576180873\\
387	0.00562738364234743\\
388	0.00562874361106573\\
389	0.00563012600508707\\
390	0.00563153117065767\\
391	0.00563295946429848\\
392	0.00563441125400398\\
393	0.00563588692027197\\
394	0.00563738685635825\\
395	0.00563891146612651\\
396	0.00564046115520807\\
397	0.00564203630438714\\
398	0.00564363719711188\\
399	0.00564526383219889\\
400	0.00564691546332293\\
401	0.00564858955428353\\
402	0.00565027981729866\\
403	0.00565197457579428\\
404	0.00565366593167166\\
405	0.00565538934452754\\
406	0.00565714542080634\\
407	0.00565893479283458\\
408	0.00566075812098736\\
409	0.00566261609718918\\
410	0.00566450944641961\\
411	0.00566643891717215\\
412	0.00566840525874305\\
413	0.00567040933327525\\
414	0.00567245203278873\\
415	0.00567453428069853\\
416	0.00567665703341976\\
417	0.00567882128206196\\
418	0.00568102805422755\\
419	0.00568327841596929\\
420	0.00568557347405532\\
421	0.00568791437877067\\
422	0.00569030232708791\\
423	0.00569273856433907\\
424	0.00569522438449874\\
425	0.00569776113514944\\
426	0.00570035022057722\\
427	0.00570299310508007\\
428	0.00570569131650847\\
429	0.00570844645006433\\
430	0.00571126017239986\\
431	0.00571413422609405\\
432	0.00571707043467132\\
433	0.00572007070853657\\
434	0.0057231370526813\\
435	0.00572627157793447\\
436	0.00572947651814869\\
437	0.00573275424871647\\
438	0.00573610725214837\\
439	0.00573953783983038\\
440	0.00574304836304245\\
441	0.0057466416638279\\
442	0.00575032073486893\\
443	0.00575408872900602\\
444	0.00575794896945599\\
445	0.005761904960634\\
446	0.00576596039876006\\
447	0.0057701191800909\\
448	0.00577438540903259\\
449	0.00577876345010988\\
450	0.00578325798230108\\
451	0.00578787413772847\\
452	0.00579261788333079\\
453	0.00579749707736224\\
454	0.00580252446374141\\
455	0.00580772623961451\\
456	0.00581316667848552\\
457	0.0058190368898063\\
458	0.00582571327832045\\
459	0.00584059042965996\\
460	0.00585773982093341\\
461	0.00587526023831121\\
462	0.00589324647962475\\
463	0.00591191430334263\\
464	0.00593094298113521\\
465	0.00595033503583898\\
466	0.00597010524595364\\
467	0.00599027022057246\\
468	0.00601084903345737\\
469	0.00603186440538486\\
470	0.00605334328932895\\
471	0.00607529614807655\\
472	0.00609774357483375\\
473	0.00612070573904595\\
474	0.00614418625135804\\
475	0.00616813886537289\\
476	0.00619238784280072\\
477	0.00621653461603059\\
478	0.00624093325313133\\
479	0.00626620758177765\\
480	0.00629242239479999\\
481	0.00631965084802119\\
482	0.00634797811094961\\
483	0.00637751059040254\\
484	0.00640840380342836\\
485	0.00644096571105693\\
486	0.0064770832840262\\
487	0.00652002518815927\\
488	0.00656393309169494\\
489	0.0066088396029996\\
490	0.00665477853556085\\
491	0.00670178546472533\\
492	0.00674989769604734\\
493	0.00679915242175098\\
494	0.00684958398774613\\
495	0.00690124491949539\\
496	0.00695418094922131\\
497	0.00700844006767366\\
498	0.00706407789785076\\
499	0.00712116504867729\\
500	0.00717975879219979\\
501	0.00723992021112285\\
502	0.00730171515664672\\
503	0.00736521714647497\\
504	0.00743051996773778\\
505	0.00749780697214923\\
506	0.00756703696234098\\
507	0.00763831259437869\\
508	0.00771190792031724\\
509	0.00778968158317259\\
510	0.00787110445015054\\
511	0.0079541675000024\\
512	0.00803913862598498\\
513	0.00812701441602769\\
514	0.00822278708979622\\
515	0.0083231838402092\\
516	0.00842553243644917\\
517	0.00852987839857534\\
518	0.00863622999382598\\
519	0.00874448590634122\\
520	0.00885419865111339\\
521	0.00896377401270416\\
522	0.00906759136888438\\
523	0.00914987739233347\\
524	0.00923459905345162\\
525	0.00932209233857995\\
526	0.0094132450287492\\
527	0.00948313344972559\\
528	0.0095511033115845\\
529	0.00962013191647017\\
530	0.00969020651136166\\
531	0.00976192860983588\\
532	0.00983654852695251\\
533	0.009905044886232\\
534	0.00996871227497252\\
535	0.010033213055165\\
536	0.0100985335652152\\
537	0.0101646565910092\\
538	0.010231566322025\\
539	0.0102992570504745\\
540	0.0103677613917222\\
541	0.0104371367358114\\
542	0.0105028361840887\\
543	0.0105698440644865\\
544	0.0106382189867722\\
545	0.0107079864248684\\
546	0.0107791824425344\\
547	0.0108518507671211\\
548	0.0109260426747462\\
549	0.0110018298799197\\
550	0.0110793484931092\\
551	0.0111589520417366\\
552	0.0112417508293042\\
553	0.0113293973419312\\
554	0.0114360310476757\\
555	0.0115467165314578\\
556	0.0116590405219288\\
557	0.0117737184096062\\
558	0.0118887522695475\\
559	0.0120040017738611\\
560	0.0121192339440008\\
561	0.0122339869402639\\
562	0.0123472020556088\\
563	0.012456211370228\\
564	0.0125541195156636\\
565	0.0126272129719429\\
566	0.0126698298096857\\
567	0.0127106595773791\\
568	0.0127463815351436\\
569	0.0127799532292079\\
570	0.0128131432611447\\
571	0.0128458920068395\\
572	0.0128779099188228\\
573	0.0129079583442177\\
574	0.0129377268471323\\
575	0.0129671630339223\\
576	0.0129961970959654\\
577	0.0130247542793291\\
578	0.0130527538829558\\
579	0.0130801305617351\\
580	0.0131067980943907\\
581	0.0131319344905414\\
582	0.013156257928101\\
583	0.0131798583419052\\
584	0.0132026276096981\\
585	0.0132244499636923\\
586	0.013245665840773\\
587	0.0132665132865002\\
588	0.0132869135184188\\
589	0.0133067844580865\\
590	0.0133260406985482\\
591	0.0133445930207368\\
592	0.0133623477620165\\
593	0.0133792066748557\\
594	0.0133950830302497\\
595	0.0134144577918238\\
596	0.0134418744311713\\
597	0.0135044530401156\\
598	0.0135751148335434\\
599	0\\
600	0\\
};
\addplot [color=mycolor16,solid,forget plot]
  table[row sep=crcr]{%
1	0.00696143322487153\\
2	0.00696144146032242\\
3	0.00696144984483376\\
4	0.00696145838109889\\
5	0.00696146707185975\\
6	0.00696147591990765\\
7	0.00696148492808418\\
8	0.00696149409928215\\
9	0.00696150343644645\\
10	0.00696151294257503\\
11	0.00696152262071983\\
12	0.00696153247398774\\
13	0.00696154250554162\\
14	0.00696155271860125\\
15	0.00696156311644441\\
16	0.00696157370240786\\
17	0.00696158447988844\\
18	0.00696159545234414\\
19	0.00696160662329517\\
20	0.00696161799632509\\
21	0.00696162957508196\\
22	0.00696164136327947\\
23	0.00696165336469813\\
24	0.00696166558318646\\
25	0.00696167802266219\\
26	0.00696169068711355\\
27	0.00696170358060049\\
28	0.00696171670725595\\
29	0.00696173007128721\\
30	0.00696174367697718\\
31	0.00696175752868578\\
32	0.00696177163085131\\
33	0.00696178598799185\\
34	0.00696180060470668\\
35	0.00696181548567772\\
36	0.00696183063567105\\
37	0.00696184605953839\\
38	0.00696186176221861\\
39	0.00696187774873931\\
40	0.00696189402421841\\
41	0.00696191059386575\\
42	0.00696192746298473\\
43	0.00696194463697401\\
44	0.00696196212132916\\
45	0.00696197992164443\\
46	0.00696199804361451\\
47	0.00696201649303632\\
48	0.0069620352758108\\
49	0.00696205439794482\\
50	0.00696207386555304\\
51	0.00696209368485984\\
52	0.00696211386220129\\
53	0.0069621344040271\\
54	0.00696215531690271\\
55	0.00696217660751129\\
56	0.00696219828265589\\
57	0.00696222034926157\\
58	0.00696224281437753\\
59	0.00696226568517939\\
60	0.00696228896897139\\
61	0.00696231267318872\\
62	0.00696233680539982\\
63	0.00696236137330879\\
64	0.00696238638475776\\
65	0.00696241184772942\\
66	0.00696243777034944\\
67	0.00696246416088909\\
68	0.00696249102776779\\
69	0.00696251837955576\\
70	0.00696254622497671\\
71	0.00696257457291057\\
72	0.00696260343239627\\
73	0.00696263281263455\\
74	0.00696266272299091\\
75	0.00696269317299844\\
76	0.00696272417236088\\
77	0.00696275573095565\\
78	0.00696278785883688\\
79	0.00696282056623865\\
80	0.00696285386357808\\
81	0.00696288776145869\\
82	0.00696292227067364\\
83	0.00696295740220915\\
84	0.00696299316724791\\
85	0.00696302957717258\\
86	0.00696306664356933\\
87	0.00696310437823148\\
88	0.00696314279316319\\
89	0.00696318190058316\\
90	0.0069632217129285\\
91	0.00696326224285858\\
92	0.00696330350325897\\
93	0.00696334550724549\\
94	0.0069633882681683\\
95	0.00696343179961601\\
96	0.00696347611542\\
97	0.00696352122965865\\
98	0.00696356715666178\\
99	0.00696361391101511\\
100	0.00696366150756478\\
101	0.006963709961422\\
102	0.00696375928796774\\
103	0.00696380950285753\\
104	0.00696386062202632\\
105	0.00696391266169348\\
106	0.00696396563836779\\
107	0.00696401956885262\\
108	0.00696407447025114\\
109	0.00696413035997168\\
110	0.00696418725573307\\
111	0.00696424517557023\\
112	0.00696430413783972\\
113	0.00696436416122549\\
114	0.00696442526474467\\
115	0.00696448746775348\\
116	0.00696455078995324\\
117	0.00696461525139654\\
118	0.0069646808724934\\
119	0.00696474767401765\\
120	0.00696481567711341\\
121	0.00696488490330162\\
122	0.00696495537448672\\
123	0.00696502711296349\\
124	0.00696510014142396\\
125	0.00696517448296441\\
126	0.00696525016109264\\
127	0.00696532719973518\\
128	0.00696540562324474\\
129	0.00696548545640784\\
130	0.00696556672445236\\
131	0.00696564945305554\\
132	0.00696573366835182\\
133	0.006965819396941\\
134	0.00696590666589651\\
135	0.00696599550277379\\
136	0.00696608593561885\\
137	0.00696617799297695\\
138	0.00696627170390153\\
139	0.00696636709796312\\
140	0.00696646420525861\\
141	0.00696656305642053\\
142	0.00696666368262657\\
143	0.00696676611560927\\
144	0.00696687038766588\\
145	0.00696697653166832\\
146	0.00696708458107346\\
147	0.00696719456993342\\
148	0.00696730653290622\\
149	0.00696742050526649\\
150	0.0069675365229164\\
151	0.00696765462239688\\
152	0.00696777484089888\\
153	0.00696789721627495\\
154	0.00696802178705102\\
155	0.0069681485924383\\
156	0.00696827767234553\\
157	0.00696840906739131\\
158	0.00696854281891675\\
159	0.00696867896899833\\
160	0.00696881756046091\\
161	0.00696895863689111\\
162	0.00696910224265082\\
163	0.00696924842289099\\
164	0.00696939722356568\\
165	0.00696954869144635\\
166	0.00696970287413637\\
167	0.00696985982008592\\
168	0.00697001957860697\\
169	0.00697018219988867\\
170	0.00697034773501297\\
171	0.00697051623597057\\
172	0.00697068775567703\\
173	0.00697086234798932\\
174	0.00697104006772262\\
175	0.00697122097066734\\
176	0.0069714051136066\\
177	0.00697159255433388\\
178	0.0069717833516711\\
179	0.00697197756548698\\
180	0.00697217525671572\\
181	0.00697237648737606\\
182	0.00697258132059065\\
183	0.00697278982060581\\
184	0.0069730020528116\\
185	0.00697321808376235\\
186	0.00697343798119741\\
187	0.00697366181406246\\
188	0.00697388965253106\\
189	0.00697412156802669\\
190	0.00697435763324516\\
191	0.00697459792217744\\
192	0.00697484251013287\\
193	0.00697509147376288\\
194	0.00697534489108506\\
195	0.00697560284150774\\
196	0.00697586540585497\\
197	0.00697613266639202\\
198	0.00697640470685131\\
199	0.0069766816124588\\
200	0.00697696346996095\\
201	0.00697725036765211\\
202	0.00697754239540244\\
203	0.00697783964468634\\
204	0.00697814220861145\\
205	0.00697845018194814\\
206	0.00697876366115959\\
207	0.00697908274443239\\
208	0.00697940753170776\\
209	0.00697973812471333\\
210	0.00698007462699552\\
211	0.00698041714395254\\
212	0.00698076578286798\\
213	0.00698112065294507\\
214	0.00698148186534155\\
215	0.00698184953320526\\
216	0.00698222377171029\\
217	0.00698260469809399\\
218	0.00698299243169447\\
219	0.00698338709398898\\
220	0.00698378880863296\\
221	0.0069841977014998\\
222	0.00698461390072141\\
223	0.00698503753672954\\
224	0.00698546874229788\\
225	0.00698590765258501\\
226	0.00698635440517809\\
227	0.00698680914013747\\
228	0.0069872720000421\\
229	0.00698774313003584\\
230	0.00698822267787467\\
231	0.00698871079397477\\
232	0.00698920763146153\\
233	0.00698971334621958\\
234	0.00699022809694367\\
235	0.00699075204519062\\
236	0.00699128535543224\\
237	0.00699182819510928\\
238	0.00699238073468644\\
239	0.00699294314770841\\
240	0.00699351561085704\\
241	0.00699409830400958\\
242	0.00699469141029811\\
243	0.00699529511617007\\
244	0.00699590961144998\\
245	0.00699653508940242\\
246	0.00699717174679613\\
247	0.00699781978396948\\
248	0.00699847940489714\\
249	0.00699915081725808\\
250	0.00699983423250492\\
251	0.00700052986593458\\
252	0.00700123793676039\\
253	0.00700195866818557\\
254	0.00700269228747813\\
255	0.0070034390260473\\
256	0.00700419911952142\\
257	0.00700497280782739\\
258	0.00700576033527162\\
259	0.00700656195062264\\
260	0.00700737790719535\\
261	0.00700820846293681\\
262	0.00700905388051388\\
263	0.00700991442740246\\
264	0.00701079037597854\\
265	0.00701168200361107\\
266	0.00701258959275662\\
267	0.0070135134310559\\
268	0.00701445381143224\\
269	0.00701541103219203\\
270	0.00701638539712703\\
271	0.00701737721561869\\
272	0.00701838680274421\\
273	0.00701941447938431\\
274	0.00702046057233346\\
275	0.0070215254144148\\
276	0.00702260934459632\\
277	0.00702371270810586\\
278	0.00702483585655055\\
279	0.00702597914803856\\
280	0.00702714294730321\\
281	0.00702832762582932\\
282	0.00702953356198208\\
283	0.0070307611411382\\
284	0.0070320107558196\\
285	0.00703328280582942\\
286	0.00703457769839071\\
287	0.0070358958482874\\
288	0.00703723767800791\\
289	0.00703860361789132\\
290	0.00703999410627592\\
291	0.00704140958965045\\
292	0.0070428505228078\\
293	0.00704431736900126\\
294	0.00704581060010326\\
295	0.00704733069676665\\
296	0.00704887814858842\\
297	0.00705045345427586\\
298	0.00705205712181509\\
299	0.00705368966864196\\
300	0.00705535162181517\\
301	0.00705704351819158\\
302	0.00705876590460362\\
303	0.00706051933803868\\
304	0.00706230438582024\\
305	0.00706412162579081\\
306	0.00706597164649623\\
307	0.00706785504737133\\
308	0.00706977243892669\\
309	0.00707172444293687\\
310	0.0070737116926303\\
311	0.00707573483288177\\
312	0.00707779452040469\\
313	0.00707989142393294\\
314	0.00708202622438877\\
315	0.00708419961508777\\
316	0.00708641230192291\\
317	0.00708866500354615\\
318	0.00709095845154684\\
319	0.00709329339062627\\
320	0.00709567057876735\\
321	0.00709809078739845\\
322	0.00710055480155039\\
323	0.00710306342000517\\
324	0.00710561745543499\\
325	0.00710821773453001\\
326	0.00711086509811281\\
327	0.0071135604012376\\
328	0.00711630451327133\\
329	0.0071190983179542\\
330	0.00712194271343594\\
331	0.00712483861228473\\
332	0.00712778694146582\\
333	0.00713078864229163\\
334	0.00713384467035772\\
335	0.00713695599551142\\
336	0.00714012360195022\\
337	0.00714334848850996\\
338	0.00714663166846703\\
339	0.00714997416463147\\
340	0.00715337698804149\\
341	0.00715684114724788\\
342	0.00716036771180513\\
343	0.00716395777212828\\
344	0.0071676124663467\\
345	0.00717133307328994\\
346	0.00717512132099987\\
347	0.00717898035561633\\
348	0.00718291742133764\\
349	0.00718694812492824\\
350	0.0071910664840161\\
351	0.00719525605491353\\
352	0.00719951808601074\\
353	0.00720385384708993\\
354	0.00720826462959008\\
355	0.007212751746859\\
356	0.00721731653437655\\
357	0.00722196034994399\\
358	0.00722668457394563\\
359	0.00723149060987166\\
360	0.00723637988433051\\
361	0.00724135384727935\\
362	0.00724641397224998\\
363	0.00725156175657058\\
364	0.00725679872158399\\
365	0.00726212641286367\\
366	0.00726754640042848\\
367	0.00727306027895821\\
368	0.00727866966801178\\
369	0.00728437621225086\\
370	0.00729018158167167\\
371	0.00729608747184848\\
372	0.00730209560419309\\
373	0.0073082077262375\\
374	0.00731442561195582\\
375	0.00732075106216334\\
376	0.00732718590506092\\
377	0.00733373199695315\\
378	0.00734039122278259\\
379	0.00734716549549618\\
380	0.00735405675769685\\
381	0.00736106698230057\\
382	0.00736819817295515\\
383	0.00737545236391039\\
384	0.00738283161938419\\
385	0.00739033803538435\\
386	0.00739797375866508\\
387	0.00740574106682857\\
388	0.00741364253887084\\
389	0.00742168049563502\\
390	0.00742985722816107\\
391	0.00743817509169642\\
392	0.00744663651316554\\
393	0.00745524399957295\\
394	0.00746400014702808\\
395	0.00747290764916912\\
396	0.00748196930100744\\
397	0.00749118798598605\\
398	0.00750056660970462\\
399	0.00751010787273188\\
400	0.00751981357245011\\
401	0.00752968257217448\\
402	0.00753970521337361\\
403	0.0075498494324153\\
404	0.00756003533420194\\
405	0.00757015939809591\\
406	0.0075804852622011\\
407	0.00759101712502076\\
408	0.00760175936052149\\
409	0.00761271653085893\\
410	0.00762389340746015\\
411	0.00763529499385015\\
412	0.00764692651424247\\
413	0.00765879325027115\\
414	0.00767090117412906\\
415	0.00768325647949881\\
416	0.00769586559325436\\
417	0.00770873518791299\\
418	0.00772187219487307\\
419	0.00773528381848334\\
420	0.00774897755107084\\
421	0.00776296118937236\\
422	0.00777724285353389\\
423	0.00779183100969855\\
424	0.00780673448794035\\
425	0.007821962490855\\
426	0.00783752462608151\\
427	0.00785343093061688\\
428	0.00786969189686395\\
429	0.00788631850054416\\
430	0.00790332223061324\\
431	0.00792071512132135\\
432	0.00793850978657212\\
433	0.00795671945680637\\
434	0.00797535801896725\\
435	0.00799444006130549\\
436	0.0080139809283583\\
437	0.00803399679912899\\
438	0.00805450480306563\\
439	0.00807552306308724\\
440	0.00809706933016405\\
441	0.00811916185548849\\
442	0.00814182179655631\\
443	0.00816507150804996\\
444	0.00818893462286462\\
445	0.00821343613903194\\
446	0.00823860251300918\\
447	0.00826446175758123\\
448	0.00829104353127232\\
449	0.00831837918308929\\
450	0.00834650198159369\\
451	0.00837544721419212\\
452	0.00840525228742067\\
453	0.00843595687049254\\
454	0.00846760314389626\\
455	0.00850023662132078\\
456	0.00853390941509973\\
457	0.00856868516294145\\
458	0.00860508663371908\\
459	0.00863660635565966\\
460	0.00866741315777336\\
461	0.00869886515522766\\
462	0.00873108124645628\\
463	0.00876454255342039\\
464	0.00880149541322494\\
465	0.00883919605348062\\
466	0.00887755606675658\\
467	0.00891657571202679\\
468	0.0089562547891163\\
469	0.00899659522467287\\
470	0.00903761013241903\\
471	0.0090793556040983\\
472	0.00912171397531188\\
473	0.00916463320757009\\
474	0.00920806236299831\\
475	0.00925185723714607\\
476	0.00929557787366925\\
477	0.00933781980842638\\
478	0.00937392481136324\\
479	0.00940555137214414\\
480	0.0094377079498843\\
481	0.00947038608381698\\
482	0.00950357471372327\\
483	0.00953726045095219\\
484	0.00957143027223439\\
485	0.00960608874650984\\
486	0.00964018797110519\\
487	0.00967109526646436\\
488	0.00970267192128993\\
489	0.00973493564128096\\
490	0.00976790584651184\\
491	0.00980160034254924\\
492	0.0098360369782581\\
493	0.00987123282150973\\
494	0.00990720440162515\\
495	0.00994397293568609\\
496	0.00998155569775987\\
497	0.0100199672566739\\
498	0.0100592141609306\\
499	0.0100993145340216\\
500	0.0101403632205247\\
501	0.0101823792078284\\
502	0.0102253804611579\\
503	0.010269384236512\\
504	0.010314410402072\\
505	0.0103605044547501\\
506	0.0104089455982218\\
507	0.0104584430138407\\
508	0.0105086808662988\\
509	0.0105585199560321\\
510	0.0106081368440963\\
511	0.0106587980734855\\
512	0.0107105831147677\\
513	0.0107638093492153\\
514	0.0108202617208306\\
515	0.0108832440905393\\
516	0.0109493411670594\\
517	0.0110164229674561\\
518	0.0110844758487037\\
519	0.011153436878803\\
520	0.0112231160368238\\
521	0.0112928785938636\\
522	0.0113604152267201\\
523	0.0114189106180415\\
524	0.0114781302422692\\
525	0.0115380184175737\\
526	0.0115984924560614\\
527	0.0116475134592567\\
528	0.0116950283817215\\
529	0.0117424929243432\\
530	0.0117898115791825\\
531	0.0118365633076319\\
532	0.0118819804741919\\
533	0.0119237491743803\\
534	0.0119627022017451\\
535	0.012001785933573\\
536	0.0120409602379146\\
537	0.0120801795329016\\
538	0.0121193937642504\\
539	0.0121585482894317\\
540	0.0121975824734986\\
541	0.0122363778195194\\
542	0.0122725438386972\\
543	0.0123088361792697\\
544	0.0123452218900791\\
545	0.0123830652189747\\
546	0.0124209737098417\\
547	0.0124583196058804\\
548	0.0124950331107372\\
549	0.0125310410219464\\
550	0.0125662667460497\\
551	0.0126006301311529\\
552	0.0126340462254209\\
553	0.0126653782393577\\
554	0.0126955728633947\\
555	0.012724847087084\\
556	0.012752807126695\\
557	0.0127788004957964\\
558	0.0128037596527478\\
559	0.0128275761396023\\
560	0.0128507411189602\\
561	0.0128732667339028\\
562	0.01289507328928\\
563	0.0129160749825545\\
564	0.0129361810864091\\
565	0.0129553557382554\\
566	0.012973893166586\\
567	0.0129917465047786\\
568	0.0130087234136089\\
569	0.0130249776205532\\
570	0.0130412538691682\\
571	0.0130575426055185\\
572	0.0130738402601048\\
573	0.0130901740700694\\
574	0.0131065312929425\\
575	0.0131228989683055\\
576	0.0131392640595054\\
577	0.0131556134451633\\
578	0.0131719338770583\\
579	0.0131882118427065\\
580	0.0132044337148521\\
581	0.0132206236488861\\
582	0.0132367682457335\\
583	0.0132528443750539\\
584	0.0132688278512317\\
585	0.0132846953190783\\
586	0.0133004226334248\\
587	0.0133159862811858\\
588	0.0133323317216059\\
589	0.0133490678702832\\
590	0.0133656835516412\\
591	0.0133817357881134\\
592	0.0133970600889429\\
593	0.0134115514637899\\
594	0.0134262135232344\\
595	0.0134435354835885\\
596	0.013477883826111\\
597	0.0135160557778789\\
598	0.0135751148335434\\
599	0\\
600	0\\
};
\addplot [color=mycolor17,solid,forget plot]
  table[row sep=crcr]{%
1	0.00892804838704115\\
2	0.00892805640983568\\
3	0.00892806457795805\\
4	0.0089280728940377\\
5	0.00892808136075152\\
6	0.00892808998082476\\
7	0.00892809875703188\\
8	0.00892810769219741\\
9	0.0089281167891969\\
10	0.00892812605095781\\
11	0.00892813548046043\\
12	0.00892814508073886\\
13	0.00892815485488195\\
14	0.0089281648060343\\
15	0.00892817493739729\\
16	0.00892818525223003\\
17	0.00892819575385047\\
18	0.0089282064456364\\
19	0.00892821733102659\\
20	0.0089282284135218\\
21	0.00892823969668598\\
22	0.00892825118414736\\
23	0.0089282628795996\\
24	0.00892827478680298\\
25	0.00892828690958562\\
26	0.00892829925184464\\
27	0.00892831181754744\\
28	0.00892832461073297\\
29	0.00892833763551297\\
30	0.00892835089607334\\
31	0.00892836439667542\\
32	0.00892837814165735\\
33	0.00892839213543547\\
34	0.00892840638250572\\
35	0.00892842088744504\\
36	0.00892843565491286\\
37	0.00892845068965257\\
38	0.008928465996493\\
39	0.00892848158034997\\
40	0.00892849744622789\\
41	0.00892851359922127\\
42	0.00892853004451639\\
43	0.00892854678739294\\
44	0.00892856383322567\\
45	0.00892858118748613\\
46	0.00892859885574438\\
47	0.00892861684367074\\
48	0.00892863515703764\\
49	0.00892865380172138\\
50	0.00892867278370407\\
51	0.00892869210907545\\
52	0.00892871178403485\\
53	0.00892873181489319\\
54	0.0089287522080749\\
55	0.00892877297012003\\
56	0.00892879410768626\\
57	0.00892881562755104\\
58	0.00892883753661372\\
59	0.00892885984189774\\
60	0.00892888255055286\\
61	0.0089289056698574\\
62	0.00892892920722054\\
63	0.00892895317018468\\
64	0.00892897756642782\\
65	0.00892900240376597\\
66	0.00892902769015562\\
67	0.00892905343369629\\
68	0.00892907964263303\\
69	0.00892910632535908\\
70	0.00892913349041845\\
71	0.00892916114650871\\
72	0.00892918930248363\\
73	0.00892921796735607\\
74	0.00892924715030075\\
75	0.00892927686065718\\
76	0.0089293071079326\\
77	0.00892933790180497\\
78	0.00892936925212603\\
79	0.00892940116892439\\
80	0.00892943366240869\\
81	0.00892946674297086\\
82	0.00892950042118933\\
83	0.00892953470783239\\
84	0.00892956961386161\\
85	0.00892960515043523\\
86	0.00892964132891171\\
87	0.00892967816085331\\
88	0.00892971565802973\\
89	0.00892975383242178\\
90	0.00892979269622517\\
91	0.00892983226185436\\
92	0.00892987254194646\\
93	0.00892991354936516\\
94	0.00892995529720484\\
95	0.00892999779879464\\
96	0.00893004106770267\\
97	0.00893008511774027\\
98	0.00893012996296635\\
99	0.00893017561769179\\
100	0.00893022209648395\\
101	0.00893026941417128\\
102	0.0089303175858479\\
103	0.00893036662687841\\
104	0.00893041655290266\\
105	0.00893046737984072\\
106	0.0089305191238978\\
107	0.00893057180156941\\
108	0.00893062542964648\\
109	0.00893068002522067\\
110	0.0089307356056897\\
111	0.00893079218876283\\
112	0.00893084979246641\\
113	0.00893090843514949\\
114	0.00893096813548967\\
115	0.00893102891249884\\
116	0.00893109078552924\\
117	0.00893115377427948\\
118	0.00893121789880069\\
119	0.00893128317950286\\
120	0.00893134963716121\\
121	0.00893141729292267\\
122	0.00893148616831257\\
123	0.00893155628524133\\
124	0.00893162766601135\\
125	0.00893170033332399\\
126	0.00893177431028667\\
127	0.00893184962042012\\
128	0.00893192628766576\\
129	0.00893200433639316\\
130	0.00893208379140772\\
131	0.00893216467795836\\
132	0.00893224702174554\\
133	0.0089323308489292\\
134	0.00893241618613701\\
135	0.00893250306047269\\
136	0.00893259149952448\\
137	0.00893268153137383\\
138	0.0089327731846041\\
139	0.00893286648830962\\
140	0.00893296147210469\\
141	0.00893305816613293\\
142	0.00893315660107668\\
143	0.00893325680816662\\
144	0.00893335881919151\\
145	0.00893346266650821\\
146	0.00893356838305174\\
147	0.00893367600234563\\
148	0.00893378555851242\\
149	0.00893389708628431\\
150	0.00893401062101406\\
151	0.00893412619868607\\
152	0.00893424385592761\\
153	0.00893436363002035\\
154	0.00893448555891196\\
155	0.00893460968122806\\
156	0.00893473603628429\\
157	0.00893486466409862\\
158	0.00893499560540393\\
159	0.00893512890166072\\
160	0.00893526459507013\\
161	0.00893540272858717\\
162	0.00893554334593417\\
163	0.00893568649161449\\
164	0.00893583221092651\\
165	0.00893598054997778\\
166	0.00893613155569951\\
167	0.00893628527586131\\
168	0.00893644175908615\\
169	0.00893660105486564\\
170	0.00893676321357556\\
171	0.00893692828649168\\
172	0.00893709632580585\\
173	0.00893726738464244\\
174	0.00893744151707497\\
175	0.00893761877814317\\
176	0.00893779922387023\\
177	0.00893798291128047\\
178	0.00893816989841723\\
179	0.00893836024436119\\
180	0.00893855400924893\\
181	0.00893875125429188\\
182	0.00893895204179562\\
183	0.0089391564351795\\
184	0.00893936449899665\\
185	0.00893957629895431\\
186	0.00893979190193462\\
187	0.00894001137601568\\
188	0.0089402347904931\\
189	0.00894046221590187\\
190	0.00894069372403866\\
191	0.00894092938798455\\
192	0.00894116928212814\\
193	0.00894141348218914\\
194	0.00894166206524232\\
195	0.008941915109742\\
196	0.00894217269554687\\
197	0.0089424349039454\\
198	0.00894270181768164\\
199	0.00894297352098154\\
200	0.0089432500995797\\
201	0.0089435316407467\\
202	0.00894381823331691\\
203	0.00894410996771675\\
204	0.00894440693599363\\
205	0.00894470923184526\\
206	0.00894501695064965\\
207	0.0089453301894956\\
208	0.00894564904721376\\
209	0.00894597362440836\\
210	0.0089463040234894\\
211	0.00894664034870562\\
212	0.00894698270617792\\
213	0.00894733120393355\\
214	0.00894768595194088\\
215	0.00894804706214487\\
216	0.00894841464850319\\
217	0.00894878882702303\\
218	0.00894916971579866\\
219	0.00894955743504967\\
220	0.00894995210715993\\
221	0.00895035385671736\\
222	0.00895076281055444\\
223	0.00895117909778948\\
224	0.00895160284986871\\
225	0.00895203420060922\\
226	0.0089524732862427\\
227	0.00895292024546002\\
228	0.00895337521945672\\
229	0.00895383835197938\\
230	0.00895430978937292\\
231	0.00895478968062875\\
232	0.008955278177434\\
233	0.00895577543422161\\
234	0.00895628160822148\\
235	0.0089567968595126\\
236	0.00895732135107625\\
237	0.00895785524885025\\
238	0.00895839872178423\\
239	0.00895895194189618\\
240	0.0089595150843299\\
241	0.00896008832741384\\
242	0.00896067185272095\\
243	0.00896126584512982\\
244	0.00896187049288706\\
245	0.00896248598767096\\
246	0.00896311252465636\\
247	0.00896375030258097\\
248	0.00896439952381293\\
249	0.00896506039441982\\
250	0.00896573312423908\\
251	0.0089664179269499\\
252	0.00896711502014647\\
253	0.00896782462541293\\
254	0.00896854696839975\\
255	0.00896928227890172\\
256	0.00897003079093755\\
257	0.00897079274283118\\
258	0.00897156837729477\\
259	0.00897235794151336\\
260	0.00897316168723145\\
261	0.00897397987084129\\
262	0.00897481275347308\\
263	0.008975660601087\\
264	0.0089765236845675\\
265	0.00897740227981922\\
266	0.00897829666786521\\
267	0.00897920713494724\\
268	0.00898013397262827\\
269	0.00898107747789723\\
270	0.00898203795327612\\
271	0.0089830157069295\\
272	0.00898401105277622\\
273	0.00898502431060269\\
274	0.00898605580617691\\
275	0.00898710587136502\\
276	0.00898817484426421\\
277	0.00898926306933571\\
278	0.00899037089752115\\
279	0.00899149868637397\\
280	0.00899264680019391\\
281	0.00899381561016477\\
282	0.00899500549449545\\
283	0.00899621683856444\\
284	0.00899745003506788\\
285	0.00899870548417121\\
286	0.00899998359366466\\
287	0.00900128477912276\\
288	0.00900260946406756\\
289	0.00900395808013647\\
290	0.00900533106725425\\
291	0.00900672887380945\\
292	0.00900815195683561\\
293	0.00900960078219738\\
294	0.00901107582478151\\
295	0.0090125775686932\\
296	0.00901410650745788\\
297	0.00901566314422855\\
298	0.00901724799199918\\
299	0.00901886157382402\\
300	0.00902050442304349\\
301	0.00902217708351661\\
302	0.00902388010986039\\
303	0.00902561406769645\\
304	0.00902737953390522\\
305	0.00902917709688795\\
306	0.00903100735683696\\
307	0.0090328709260144\\
308	0.00903476842903988\\
309	0.0090367005031876\\
310	0.00903866779869467\\
311	0.00904067097908386\\
312	0.00904271072150657\\
313	0.00904478771710377\\
314	0.00904690267133902\\
315	0.00904905630424975\\
316	0.00905124935093033\\
317	0.00905348256192723\\
318	0.00905575670365084\\
319	0.00905807255880501\\
320	0.00906043092683612\\
321	0.00906283262440331\\
322	0.00906527848587189\\
323	0.00906776936383229\\
324	0.00907030612964714\\
325	0.00907288967402974\\
326	0.00907552090765718\\
327	0.00907820076182264\\
328	0.0090809301891315\\
329	0.00908371016424696\\
330	0.00908654168469219\\
331	0.00908942577171742\\
332	0.00909236347124423\\
333	0.00909535585490737\\
334	0.00909840402123719\\
335	0.00910150909708704\\
336	0.00910467223956859\\
337	0.0091078946391192\\
338	0.00911117752489224\\
339	0.00911452217297208\\
340	0.009117929904183\\
341	0.00912140195114907\\
342	0.00912493944544214\\
343	0.00912854384149498\\
344	0.00913221667553634\\
345	0.0091359596499675\\
346	0.00913977493487116\\
347	0.00914366629969959\\
348	0.00914764358037652\\
349	0.00915174158444089\\
350	0.00915610935722876\\
351	0.00916071683785196\\
352	0.00916540651822822\\
353	0.00917017996260316\\
354	0.00917503876912129\\
355	0.00917998457066177\\
356	0.0091850190356612\\
357	0.00919014386884601\\
358	0.00919536081177479\\
359	0.00920067164361595\\
360	0.00920607818379611\\
361	0.00921158229156297\\
362	0.00921718586685545\\
363	0.00922289085116554\\
364	0.00922869922838611\\
365	0.00923461302563759\\
366	0.00924063431406558\\
367	0.00924676520959955\\
368	0.00925300787366149\\
369	0.00925936451381083\\
370	0.00926583738430968\\
371	0.00927242878658873\\
372	0.00927914106958991\\
373	0.009285976629957\\
374	0.0092929379120448\\
375	0.00930002740773486\\
376	0.00930724765611737\\
377	0.00931460124326209\\
378	0.00932209080235283\\
379	0.00932971901285813\\
380	0.0093374885902979\\
381	0.00934540228883939\\
382	0.00935346289644827\\
383	0.00936167322680546\\
384	0.00937003610528839\\
385	0.00937855434557265\\
386	0.00938723072045549\\
387	0.00939606797809044\\
388	0.00940506916934261\\
389	0.00941423935269848\\
390	0.00942358173118622\\
391	0.0094330988947954\\
392	0.00944279334612153\\
393	0.00945266747968104\\
394	0.00946272355793335\\
395	0.00947296368211432\\
396	0.00948338975256834\\
397	0.0094940034028413\\
398	0.00950480585988164\\
399	0.00951579758471138\\
400	0.00952697724586378\\
401	0.00953833864042698\\
402	0.00954986123602938\\
403	0.00956148061555462\\
404	0.0095729943293389\\
405	0.00958375414271156\\
406	0.00959222741999822\\
407	0.00960087756335211\\
408	0.00960970682560881\\
409	0.0096187173843982\\
410	0.00962791133042151\\
411	0.00963729069277058\\
412	0.00964685751165343\\
413	0.00965661394791322\\
414	0.00966656103481888\\
415	0.00967670279046636\\
416	0.0096870433218048\\
417	0.00969758682580962\\
418	0.00970833759051284\\
419	0.00971929999582829\\
420	0.00973047851406489\\
421	0.00974187771002526\\
422	0.00975350224103201\\
423	0.00976535685971458\\
424	0.00977744643006985\\
425	0.00978977592808905\\
426	0.00980235037981236\\
427	0.00981517490393675\\
428	0.00982825471004588\\
429	0.00984159509623094\\
430	0.00985520144601245\\
431	0.00986907922445911\\
432	0.00988323397338755\\
433	0.00989767130553936\\
434	0.00991239689776872\\
435	0.00992741648391694\\
436	0.00994273585059723\\
437	0.00995836084872592\\
438	0.00997429747010315\\
439	0.00999055218448742\\
440	0.0100071334877347\\
441	0.0100240438283082\\
442	0.0100412803436629\\
443	0.0100588483236861\\
444	0.0100767529587988\\
445	0.0100949993130888\\
446	0.0101135922958009\\
447	0.0101325366334713\\
448	0.0101518368468658\\
449	0.0101714972189087\\
450	0.0101915213798393\\
451	0.0102119127915055\\
452	0.0102326745812786\\
453	0.010253809258963\\
454	0.0102753183246448\\
455	0.0102972011577886\\
456	0.0103194511812423\\
457	0.0103420359173635\\
458	0.0103637599402654\\
459	0.010383191872\\
460	0.010402425628083\\
461	0.0104221546761598\\
462	0.01044242798672\\
463	0.0104634208839125\\
464	0.0104860432899365\\
465	0.010510615181872\\
466	0.0105357824310886\\
467	0.0105615127165632\\
468	0.0105878161380539\\
469	0.0106147031780618\\
470	0.0106421883015759\\
471	0.0106703058621089\\
472	0.0106997937195528\\
473	0.0107298931211558\\
474	0.0107603355654729\\
475	0.0107910604253917\\
476	0.0108218943414592\\
477	0.0108522775798314\\
478	0.0108801729146689\\
479	0.01090620775102\\
480	0.0109325095111674\\
481	0.0109590612965513\\
482	0.0109858369789297\\
483	0.0110128092622672\\
484	0.0110399491483798\\
485	0.0110672221140436\\
486	0.0110939716621178\\
487	0.0111187027085108\\
488	0.011143730164065\\
489	0.0111690445719119\\
490	0.0111946377319795\\
491	0.0112205190083465\\
492	0.011246678405485\\
493	0.0112731044986012\\
494	0.0112997846888265\\
495	0.0113267046482608\\
496	0.0113538481267316\\
497	0.0113811961224248\\
498	0.011408723906216\\
499	0.0114364086728828\\
500	0.0114642608042559\\
501	0.0114922556416281\\
502	0.0115203663822708\\
503	0.0115485640225937\\
504	0.0115768172754413\\
505	0.0116050923032829\\
506	0.0116326034873794\\
507	0.0116600285432745\\
508	0.0116875251793058\\
509	0.0117143157622327\\
510	0.0117405416631368\\
511	0.0117670484620547\\
512	0.0117938236606359\\
513	0.0118208529626045\\
514	0.0118481195309724\\
515	0.0118756083873784\\
516	0.0119033001665179\\
517	0.0119311845161186\\
518	0.0119592431864288\\
519	0.0119874515861353\\
520	0.0120157829496113\\
521	0.0120442086216939\\
522	0.0120726975450495\\
523	0.0121012173182994\\
524	0.0121297368341328\\
525	0.0121582224191919\\
526	0.0121866360774382\\
527	0.0122150593357926\\
528	0.0122434837731262\\
529	0.012271884608684\\
530	0.0123002348686295\\
531	0.0123282700170947\\
532	0.0123554997781726\\
533	0.0123829798826811\\
534	0.0124122116241502\\
535	0.0124410994665472\\
536	0.0124696062633754\\
537	0.0124976937165482\\
538	0.0125253225092751\\
539	0.0125524525053561\\
540	0.0125790427938751\\
541	0.0126050524613397\\
542	0.0126304764673356\\
543	0.0126552669333829\\
544	0.0126793745126733\\
545	0.0127019060328393\\
546	0.0127235404985523\\
547	0.0127445530443311\\
548	0.012764908444034\\
549	0.0127845717118121\\
550	0.0128035082420549\\
551	0.0128216841297711\\
552	0.0128390658052865\\
553	0.012855631894578\\
554	0.0128713391491722\\
555	0.0128864409532959\\
556	0.0129008913195492\\
557	0.012914325326947\\
558	0.0129275565336294\\
559	0.012940563094706\\
560	0.0129535966559269\\
561	0.0129666978623545\\
562	0.0129798688870274\\
563	0.0129931133670592\\
564	0.0130064365716933\\
565	0.0130198454289273\\
566	0.0130333460586304\\
567	0.0130469457085717\\
568	0.0130606659746072\\
569	0.0130745389185016\\
570	0.0130885627309631\\
571	0.0131027353386474\\
572	0.0131170541424563\\
573	0.0131315150127258\\
574	0.0131461132773175\\
575	0.0131608436180769\\
576	0.0131756999131583\\
577	0.013190677909612\\
578	0.0132057732483407\\
579	0.013220981360728\\
580	0.013236298980741\\
581	0.0132519674981059\\
582	0.0132687631281697\\
583	0.013285468939127\\
584	0.013302672513284\\
585	0.0133195446593288\\
586	0.0133364249559234\\
587	0.013352829393449\\
588	0.0133682594581791\\
589	0.0133825611506923\\
590	0.0133957579545669\\
591	0.0134085298188763\\
592	0.0134221484363206\\
593	0.0134355399849467\\
594	0.0134504085075599\\
595	0.0134677297271409\\
596	0.0134889777516421\\
597	0.0135160557778789\\
598	0.0135751148335434\\
599	0\\
600	0\\
};
\addplot [color=mycolor18,solid,forget plot]
  table[row sep=crcr]{%
1	0.0101435220610527\\
2	0.0101435269761815\\
3	0.0101435319804809\\
4	0.0101435370755683\\
5	0.0101435422630904\\
6	0.0101435475447237\\
7	0.0101435529221753\\
8	0.0101435583971832\\
9	0.0101435639715168\\
10	0.0101435696469776\\
11	0.0101435754253999\\
12	0.0101435813086511\\
13	0.0101435872986326\\
14	0.0101435933972802\\
15	0.0101435996065647\\
16	0.0101436059284929\\
17	0.0101436123651077\\
18	0.0101436189184893\\
19	0.0101436255907552\\
20	0.0101436323840618\\
21	0.0101436393006042\\
22	0.0101436463426174\\
23	0.010143653512377\\
24	0.0101436608121998\\
25	0.0101436682444445\\
26	0.0101436758115126\\
27	0.0101436835158492\\
28	0.0101436913599436\\
29	0.0101436993463302\\
30	0.0101437074775895\\
31	0.0101437157563485\\
32	0.010143724185282\\
33	0.0101437327671131\\
34	0.0101437415046144\\
35	0.0101437504006087\\
36	0.0101437594579697\\
37	0.0101437686796235\\
38	0.0101437780685489\\
39	0.0101437876277789\\
40	0.0101437973604013\\
41	0.0101438072695598\\
42	0.0101438173584551\\
43	0.0101438276303458\\
44	0.0101438380885495\\
45	0.0101438487364439\\
46	0.0101438595774679\\
47	0.0101438706151226\\
48	0.0101438818529725\\
49	0.0101438932946466\\
50	0.0101439049438395\\
51	0.010143916804313\\
52	0.0101439288798965\\
53	0.0101439411744891\\
54	0.0101439536920604\\
55	0.0101439664366515\\
56	0.010143979412377\\
57	0.0101439926234259\\
58	0.0101440060740627\\
59	0.0101440197686294\\
60	0.0101440337115463\\
61	0.0101440479073138\\
62	0.0101440623605135\\
63	0.0101440770758101\\
64	0.0101440920579525\\
65	0.0101441073117756\\
66	0.0101441228422015\\
67	0.0101441386542415\\
68	0.0101441547529975\\
69	0.0101441711436634\\
70	0.0101441878315273\\
71	0.0101442048219727\\
72	0.0101442221204803\\
73	0.0101442397326301\\
74	0.0101442576641027\\
75	0.0101442759206816\\
76	0.0101442945082545\\
77	0.0101443134328156\\
78	0.0101443327004675\\
79	0.0101443523174228\\
80	0.0101443722900065\\
81	0.0101443926246579\\
82	0.0101444133279322\\
83	0.0101444344065035\\
84	0.0101444558671659\\
85	0.0101444777168366\\
86	0.0101444999625574\\
87	0.0101445226114972\\
88	0.0101445456709546\\
89	0.0101445691483596\\
90	0.0101445930512764\\
91	0.0101446173874057\\
92	0.0101446421645873\\
93	0.0101446673908023\\
94	0.0101446930741756\\
95	0.0101447192229791\\
96	0.0101447458456336\\
97	0.0101447729507119\\
98	0.0101448005469413\\
99	0.0101448286432065\\
100	0.0101448572485525\\
101	0.0101448863721874\\
102	0.0101449160234851\\
103	0.0101449462119888\\
104	0.0101449769474133\\
105	0.0101450082396489\\
106	0.0101450400987639\\
107	0.0101450725350081\\
108	0.0101451055588159\\
109	0.01014513918081\\
110	0.010145173411804\\
111	0.0101452082628068\\
112	0.0101452437450251\\
113	0.0101452798698677\\
114	0.0101453166489489\\
115	0.0101453540940921\\
116	0.0101453922173334\\
117	0.0101454310309259\\
118	0.010145470547343\\
119	0.0101455107792828\\
120	0.0101455517396719\\
121	0.0101455934416695\\
122	0.0101456358986716\\
123	0.0101456791243153\\
124	0.010145723132483\\
125	0.010145767937307\\
126	0.0101458135531735\\
127	0.0101458599947277\\
128	0.0101459072768782\\
129	0.0101459554148017\\
130	0.0101460044239475\\
131	0.0101460543200432\\
132	0.0101461051190986\\
133	0.0101461568374118\\
134	0.0101462094915734\\
135	0.0101462630984726\\
136	0.0101463176753017\\
137	0.0101463732395623\\
138	0.0101464298090702\\
139	0.0101464874019615\\
140	0.0101465460366978\\
141	0.0101466057320725\\
142	0.0101466665072166\\
143	0.0101467283816045\\
144	0.0101467913750604\\
145	0.0101468555077645\\
146	0.0101469208002591\\
147	0.0101469872734555\\
148	0.0101470549486401\\
149	0.0101471238474816\\
150	0.0101471939920373\\
151	0.0101472654047606\\
152	0.0101473381085074\\
153	0.0101474121265441\\
154	0.0101474874825541\\
155	0.0101475642006457\\
156	0.0101476423053596\\
157	0.0101477218216767\\
158	0.0101478027750256\\
159	0.010147885191291\\
160	0.0101479690968217\\
161	0.0101480545184386\\
162	0.0101481414834436\\
163	0.0101482300196276\\
164	0.0101483201552797\\
165	0.0101484119191959\\
166	0.0101485053406881\\
167	0.0101486004495932\\
168	0.0101486972762827\\
169	0.0101487958516719\\
170	0.01014889620723\\
171	0.0101489983749895\\
172	0.0101491023875564\\
173	0.0101492082781206\\
174	0.010149316080466\\
175	0.0101494258289814\\
176	0.0101495375586709\\
177	0.0101496513051649\\
178	0.0101497671047315\\
179	0.0101498849942877\\
180	0.010150005011411\\
181	0.0101501271943507\\
182	0.0101502515820406\\
183	0.0101503782141105\\
184	0.0101505071308992\\
185	0.0101506383734664\\
186	0.0101507719836059\\
187	0.0101509080038589\\
188	0.0101510464775266\\
189	0.0101511874486843\\
190	0.010151330962195\\
191	0.0101514770637232\\
192	0.0101516257997493\\
193	0.0101517772175841\\
194	0.0101519313653835\\
195	0.0101520882921635\\
196	0.0101522480478154\\
197	0.0101524106831217\\
198	0.0101525762497715\\
199	0.0101527448003767\\
200	0.0101529163884886\\
201	0.0101530910686143\\
202	0.010153268896234\\
203	0.0101534499278179\\
204	0.0101536342208441\\
205	0.0101538218338163\\
206	0.010154012826282\\
207	0.0101542072588514\\
208	0.0101544051932159\\
209	0.0101546066921672\\
210	0.0101548118196174\\
211	0.0101550206406184\\
212	0.0101552332213824\\
213	0.0101554496293024\\
214	0.0101556699329734\\
215	0.0101558942022134\\
216	0.0101561225080857\\
217	0.0101563549229206\\
218	0.0101565915203381\\
219	0.010156832375271\\
220	0.0101570775639879\\
221	0.0101573271641175\\
222	0.0101575812546725\\
223	0.0101578399160745\\
224	0.0101581032301788\\
225	0.0101583712803005\\
226	0.0101586441512399\\
227	0.0101589219293099\\
228	0.0101592047023622\\
229	0.0101594925598153\\
230	0.0101597855926826\\
231	0.0101600838936007\\
232	0.0101603875568586\\
233	0.0101606966784276\\
234	0.0101610113559911\\
235	0.0101613316889758\\
236	0.0101616577785826\\
237	0.0101619897278191\\
238	0.0101623276415318\\
239	0.0101626716264394\\
240	0.0101630217911665\\
241	0.0101633782462786\\
242	0.0101637411043163\\
243	0.0101641104798322\\
244	0.0101644864894266\\
245	0.0101648692517853\\
246	0.0101652588877171\\
247	0.010165655520193\\
248	0.0101660592743853\\
249	0.0101664702777079\\
250	0.0101668886598575\\
251	0.0101673145528555\\
252	0.0101677480910905\\
253	0.0101681894113619\\
254	0.0101686386529243\\
255	0.0101690959575331\\
256	0.0101695614694902\\
257	0.0101700353356913\\
258	0.0101705177056742\\
259	0.0101710087316675\\
260	0.0101715085686408\\
261	0.0101720173743553\\
262	0.0101725353094163\\
263	0.0101730625373258\\
264	0.0101735992245366\\
265	0.0101741455405075\\
266	0.0101747016577595\\
267	0.0101752677519328\\
268	0.0101758440018454\\
269	0.0101764305895529\\
270	0.0101770277004085\\
271	0.0101776355231263\\
272	0.0101782542498442\\
273	0.0101788840761891\\
274	0.0101795252013414\\
275	0.0101801778280897\\
276	0.0101808421628714\\
277	0.0101815184158827\\
278	0.0101822068012022\\
279	0.0101829075368072\\
280	0.0101836208446486\\
281	0.0101843469507277\\
282	0.0101850860851745\\
283	0.0101858384823286\\
284	0.0101866043808211\\
285	0.0101873840236591\\
286	0.0101881776583126\\
287	0.0101889855368028\\
288	0.0101898079157935\\
289	0.0101906450566845\\
290	0.0101914972257075\\
291	0.0101923646940257\\
292	0.0101932477378342\\
293	0.0101941466384657\\
294	0.0101950616824979\\
295	0.0101959931618645\\
296	0.0101969413739702\\
297	0.010197906621809\\
298	0.0101988892140869\\
299	0.0101998894653484\\
300	0.0102009076961081\\
301	0.0102019442329865\\
302	0.0102029994088518\\
303	0.0102040735629662\\
304	0.0102051670411389\\
305	0.0102062801958849\\
306	0.0102074133865907\\
307	0.010208566979686\\
308	0.0102097413488239\\
309	0.0102109368750664\\
310	0.0102121539470784\\
311	0.0102133929613301\\
312	0.0102146543223186\\
313	0.0102159384428383\\
314	0.0102172457443439\\
315	0.0102185766572453\\
316	0.0102199316204754\\
317	0.0102213110822513\\
318	0.010222715500353\\
319	0.0102241453424153\\
320	0.0102256010862335\\
321	0.010227083220083\\
322	0.0102285922430552\\
323	0.0102301286654081\\
324	0.0102316930089353\\
325	0.0102332858073515\\
326	0.0102349076066967\\
327	0.0102365589657609\\
328	0.0102382404565271\\
329	0.0102399526646372\\
330	0.0102416961898784\\
331	0.0102434716466929\\
332	0.0102452796647128\\
333	0.0102471208893248\\
334	0.0102489959822803\\
335	0.0102509056224011\\
336	0.0102528505065336\\
337	0.0102548313512401\\
338	0.0102568488967948\\
339	0.010258903918802\\
340	0.0102609972675927\\
341	0.0102631300339618\\
342	0.0102653028529559\\
343	0.0102675151877864\\
344	0.0102697678584674\\
345	0.0102720617428466\\
346	0.0102743978672238\\
347	0.0102767777604592\\
348	0.0102792049484136\\
349	0.0102816920028306\\
350	0.0102843005072455\\
351	0.0102870664702293\\
352	0.0102899277453679\\
353	0.0102928389923943\\
354	0.0102958011327493\\
355	0.0102988151088651\\
356	0.0103018818849584\\
357	0.0103050024478722\\
358	0.0103081778077773\\
359	0.0103114089978145\\
360	0.0103146970723185\\
361	0.0103180431168744\\
362	0.010321448243885\\
363	0.0103249135936208\\
364	0.0103284403353321\\
365	0.0103320296684272\\
366	0.0103356828237209\\
367	0.0103394010647611\\
368	0.0103431856892365\\
369	0.0103470380304741\\
370	0.0103509594590322\\
371	0.0103549513843953\\
372	0.0103590152567756\\
373	0.0103631525690239\\
374	0.0103673648586404\\
375	0.0103716537098718\\
376	0.0103760207558824\\
377	0.0103804676811158\\
378	0.0103849962245748\\
379	0.0103896081868925\\
380	0.0103943054470678\\
381	0.0103990899238339\\
382	0.0104039636024742\\
383	0.0104089285361219\\
384	0.010413986842631\\
385	0.0104191406903532\\
386	0.0104243922558833\\
387	0.0104297436113626\\
388	0.0104351964228281\\
389	0.0104407509761606\\
390	0.0104464196538632\\
391	0.0104522080299345\\
392	0.0104581194279328\\
393	0.0104641572857469\\
394	0.0104703251558597\\
395	0.0104766267037832\\
396	0.01048306570269\\
397	0.0104896460186078\\
398	0.0104963715693859\\
399	0.0105032462061919\\
400	0.0105102733588599\\
401	0.0105174549464263\\
402	0.0105247879530466\\
403	0.0105322533669631\\
404	0.0105397788970637\\
405	0.0105471029250194\\
406	0.0105535268595702\\
407	0.0105601291534413\\
408	0.0105669162125581\\
409	0.0105738943422493\\
410	0.0105810700748047\\
411	0.0105884499409389\\
412	0.0105960404610984\\
413	0.0106038658691581\\
414	0.0106122895659858\\
415	0.0106208573604243\\
416	0.0106295710251152\\
417	0.0106384322996213\\
418	0.0106474428845753\\
419	0.0106566044350936\\
420	0.0106659185530677\\
421	0.010675386777421\\
422	0.0106850105703862\\
423	0.0106947912976457\\
424	0.010704730215568\\
425	0.0107148286257336\\
426	0.0107250879841503\\
427	0.0107355092024648\\
428	0.0107460930581462\\
429	0.0107568401812159\\
430	0.0107677510399862\\
431	0.0107788259257431\\
432	0.0107900649363112\\
433	0.0108014679584443\\
434	0.0108130346490223\\
435	0.0108247644151944\\
436	0.0108366563942658\\
437	0.0108487094369896\\
438	0.010860922111729\\
439	0.0108732928239393\\
440	0.0108858207015065\\
441	0.0108985019412572\\
442	0.0109113297678965\\
443	0.0109243005956562\\
444	0.0109374103444363\\
445	0.0109506544058948\\
446	0.0109640276099815\\
447	0.0109775241975619\\
448	0.0109911378224695\\
449	0.0110048616870874\\
450	0.0110186893065778\\
451	0.0110326093642179\\
452	0.0110466113642805\\
453	0.0110606851567003\\
454	0.0110748195903616\\
455	0.0110890020489438\\
456	0.0111032166937586\\
457	0.0111174374262609\\
458	0.0111310565936648\\
459	0.011144832545784\\
460	0.0111587758046904\\
461	0.0111728884152916\\
462	0.0111871676077102\\
463	0.0112016102047753\\
464	0.0112162126175109\\
465	0.0112309710075095\\
466	0.0112458811319446\\
467	0.0112609383465852\\
468	0.0112761376221032\\
469	0.0112914735669991\\
470	0.0113069404449298\\
471	0.0113225321037189\\
472	0.011337796997148\\
473	0.0113531357235195\\
474	0.0113686819050201\\
475	0.011384434526993\\
476	0.0114003923651173\\
477	0.0114165539977096\\
478	0.0114329177748791\\
479	0.0114494815644976\\
480	0.0114662429064611\\
481	0.0114832000198309\\
482	0.0115003480617858\\
483	0.0115176828887694\\
484	0.0115352013489387\\
485	0.0115529002564451\\
486	0.0115707835106369\\
487	0.0115888716723064\\
488	0.0116071607815357\\
489	0.0116256455427747\\
490	0.0116443210030474\\
491	0.0116631904884664\\
492	0.0116822489302158\\
493	0.0117014910401946\\
494	0.0117209115558965\\
495	0.0117405027695083\\
496	0.0117602584080895\\
497	0.011780171903838\\
498	0.011800236200024\\
499	0.011820444148541\\
500	0.0118407935260595\\
501	0.0118612772505659\\
502	0.0118818882735183\\
503	0.0119026197141651\\
504	0.0119234649558649\\
505	0.0119444175491668\\
506	0.0119648998097392\\
507	0.0119854864178446\\
508	0.0120063293698489\\
509	0.0120274371614115\\
510	0.012048816863781\\
511	0.012070463165724\\
512	0.0120923735388314\\
513	0.012114547599813\\
514	0.0121369835697775\\
515	0.0121596803118904\\
516	0.0121826356160977\\
517	0.0122058442662609\\
518	0.0122292996415255\\
519	0.0122529930214906\\
520	0.0122769126604267\\
521	0.0123010450348892\\
522	0.0123258353070334\\
523	0.012351865357181\\
524	0.0123776251241155\\
525	0.0124030882997499\\
526	0.0124282280941118\\
527	0.0124530159285305\\
528	0.012477422428172\\
529	0.0125014173827645\\
530	0.0125249700318251\\
531	0.0125480542233763\\
532	0.0125706527362421\\
533	0.0125926624148569\\
534	0.0126131517479606\\
535	0.0126332368725144\\
536	0.0126528951154892\\
537	0.0126721041700234\\
538	0.0126908422164513\\
539	0.0127090880194813\\
540	0.0127268210890813\\
541	0.0127440219513844\\
542	0.012760671536731\\
543	0.0127767519874682\\
544	0.012792246830458\\
545	0.0128065450720889\\
546	0.0128202368570078\\
547	0.0128335546706431\\
548	0.0128464796173316\\
549	0.0128589922721929\\
550	0.0128710728488695\\
551	0.0128826940736109\\
552	0.0128938462067012\\
553	0.0129049029111274\\
554	0.0129158596195024\\
555	0.0129268489577111\\
556	0.0129379717486856\\
557	0.0129492673407897\\
558	0.012960739688428\\
559	0.0129723928196637\\
560	0.0129842307690847\\
561	0.0129962576435906\\
562	0.0130084775898677\\
563	0.0130208947316017\\
564	0.0130335130929776\\
565	0.0130463365095152\\
566	0.013059368561547\\
567	0.0130726124286204\\
568	0.0130860700317245\\
569	0.0130997412816743\\
570	0.0131136255795875\\
571	0.0131277224368694\\
572	0.0131420332510873\\
573	0.0131565596205997\\
574	0.01317130349398\\
575	0.0131862674076862\\
576	0.0132027377827662\\
577	0.0132193260454859\\
578	0.0132359221615345\\
579	0.013253010425182\\
580	0.0132701523258826\\
581	0.013287008557868\\
582	0.0133033612106936\\
583	0.0133193574536422\\
584	0.0133344529366027\\
585	0.0133492502469239\\
586	0.0133643892379398\\
587	0.0133798012348973\\
588	0.0133938336010688\\
589	0.0134067477204202\\
590	0.0134188993505772\\
591	0.0134303077081716\\
592	0.0134403402753647\\
593	0.0134503526516272\\
594	0.0134614306417842\\
595	0.0134740407759367\\
596	0.0134889777516421\\
597	0.0135160557778789\\
598	0.0135751148335434\\
599	0\\
600	0\\
};
\addplot [color=red!25!mycolor17,solid,forget plot]
  table[row sep=crcr]{%
1	0.0106118810148568\\
2	0.0106118860083902\\
3	0.0106118910926437\\
4	0.0106118962692665\\
5	0.0106119015399378\\
6	0.0106119069063671\\
7	0.0106119123702954\\
8	0.0106119179334951\\
9	0.0106119235977711\\
10	0.0106119293649611\\
11	0.0106119352369363\\
12	0.0106119412156019\\
13	0.0106119473028979\\
14	0.0106119535007995\\
15	0.0106119598113181\\
16	0.0106119662365016\\
17	0.0106119727784353\\
18	0.0106119794392422\\
19	0.0106119862210844\\
20	0.0106119931261632\\
21	0.0106120001567199\\
22	0.0106120073150367\\
23	0.0106120146034375\\
24	0.0106120220242884\\
25	0.0106120295799987\\
26	0.0106120372730215\\
27	0.0106120451058547\\
28	0.0106120530810416\\
29	0.010612061201172\\
30	0.0106120694688827\\
31	0.0106120778868587\\
32	0.0106120864578339\\
33	0.0106120951845918\\
34	0.0106121040699669\\
35	0.0106121131168453\\
36	0.0106121223281655\\
37	0.0106121317069197\\
38	0.0106121412561546\\
39	0.0106121509789724\\
40	0.0106121608785317\\
41	0.0106121709580489\\
42	0.0106121812207989\\
43	0.010612191670116\\
44	0.0106122023093958\\
45	0.0106122131420951\\
46	0.0106122241717343\\
47	0.0106122354018975\\
48	0.0106122468362343\\
49	0.0106122584784608\\
50	0.0106122703323607\\
51	0.0106122824017865\\
52	0.0106122946906613\\
53	0.0106123072029793\\
54	0.0106123199428075\\
55	0.0106123329142872\\
56	0.0106123461216347\\
57	0.0106123595691437\\
58	0.0106123732611855\\
59	0.0106123872022113\\
60	0.0106124013967535\\
61	0.0106124158494268\\
62	0.01061243056493\\
63	0.0106124455480475\\
64	0.0106124608036508\\
65	0.0106124763367001\\
66	0.010612492152246\\
67	0.0106125082554308\\
68	0.0106125246514906\\
69	0.0106125413457569\\
70	0.0106125583436582\\
71	0.0106125756507216\\
72	0.0106125932725752\\
73	0.0106126112149491\\
74	0.010612629483678\\
75	0.0106126480847027\\
76	0.010612667024072\\
77	0.010612686307945\\
78	0.0106127059425927\\
79	0.0106127259344001\\
80	0.0106127462898686\\
81	0.0106127670156178\\
82	0.0106127881183876\\
83	0.0106128096050406\\
84	0.0106128314825642\\
85	0.0106128537580728\\
86	0.0106128764388105\\
87	0.0106128995321526\\
88	0.0106129230456089\\
89	0.0106129469868257\\
90	0.0106129713635882\\
91	0.0106129961838232\\
92	0.0106130214556015\\
93	0.0106130471871406\\
94	0.0106130733868076\\
95	0.0106131000631213\\
96	0.0106131272247555\\
97	0.0106131548805416\\
98	0.0106131830394713\\
99	0.0106132117106999\\
100	0.0106132409035488\\
101	0.0106132706275088\\
102	0.010613300892243\\
103	0.01061333170759\\
104	0.0106133630835672\\
105	0.0106133950303737\\
106	0.0106134275583938\\
107	0.0106134606782003\\
108	0.010613494400558\\
109	0.0106135287364271\\
110	0.0106135636969664\\
111	0.0106135992935376\\
112	0.0106136355377083\\
113	0.0106136724412559\\
114	0.0106137100161717\\
115	0.0106137482746642\\
116	0.0106137872291636\\
117	0.0106138268923253\\
118	0.0106138672770343\\
119	0.010613908396409\\
120	0.0106139502638059\\
121	0.0106139928928234\\
122	0.0106140362973064\\
123	0.0106140804913507\\
124	0.0106141254893075\\
125	0.010614171305788\\
126	0.0106142179556682\\
127	0.0106142654540933\\
128	0.010614313816483\\
129	0.0106143630585362\\
130	0.0106144131962359\\
131	0.0106144642458546\\
132	0.0106145162239593\\
133	0.0106145691474168\\
134	0.0106146230333993\\
135	0.0106146778993894\\
136	0.0106147337631863\\
137	0.0106147906429111\\
138	0.0106148485570125\\
139	0.010614907524273\\
140	0.0106149675638146\\
141	0.010615028695105\\
142	0.0106150909379639\\
143	0.0106151543125691\\
144	0.0106152188394632\\
145	0.0106152845395597\\
146	0.0106153514341502\\
147	0.0106154195449106\\
148	0.0106154888939084\\
149	0.0106155595036097\\
150	0.0106156313968858\\
151	0.0106157045970213\\
152	0.0106157791277208\\
153	0.0106158550131165\\
154	0.0106159322777764\\
155	0.0106160109467112\\
156	0.010616091045383\\
157	0.0106161725997129\\
158	0.0106162556360893\\
159	0.0106163401813764\\
160	0.0106164262629225\\
161	0.0106165139085685\\
162	0.010616603146657\\
163	0.0106166940060412\\
164	0.0106167865160937\\
165	0.0106168807067161\\
166	0.0106169766083481\\
167	0.0106170742519773\\
168	0.0106171736691488\\
169	0.0106172748919752\\
170	0.0106173779531469\\
171	0.0106174828859416\\
172	0.0106175897242357\\
173	0.0106176985025142\\
174	0.0106178092558819\\
175	0.0106179220200742\\
176	0.0106180368314683\\
177	0.0106181537270945\\
178	0.010618272744648\\
179	0.0106183939225004\\
180	0.0106185172997118\\
181	0.0106186429160428\\
182	0.0106187708119673\\
183	0.0106189010286845\\
184	0.010619033608132\\
185	0.0106191685929991\\
186	0.0106193060267394\\
187	0.0106194459535847\\
188	0.0106195884185588\\
189	0.0106197334674911\\
190	0.0106198811470308\\
191	0.0106200315046615\\
192	0.0106201845887159\\
193	0.0106203404483901\\
194	0.0106204991337597\\
195	0.0106206606957942\\
196	0.0106208251863734\\
197	0.010620992658303\\
198	0.0106211631653308\\
199	0.0106213367621631\\
200	0.0106215135044816\\
201	0.0106216934489602\\
202	0.0106218766532824\\
203	0.010622063176159\\
204	0.0106222530773456\\
205	0.010622446417661\\
206	0.0106226432590057\\
207	0.0106228436643807\\
208	0.0106230476979061\\
209	0.0106232554248411\\
210	0.0106234669116033\\
211	0.0106236822257888\\
212	0.0106239014361927\\
213	0.0106241246128294\\
214	0.0106243518269541\\
215	0.0106245831510835\\
216	0.0106248186590183\\
217	0.0106250584258644\\
218	0.0106253025280559\\
219	0.0106255510433777\\
220	0.0106258040509881\\
221	0.0106260616314433\\
222	0.0106263238667203\\
223	0.0106265908402416\\
224	0.0106268626368996\\
225	0.010627139343082\\
226	0.0106274210466967\\
227	0.0106277078371979\\
228	0.0106279998056124\\
229	0.0106282970445658\\
230	0.0106285996483102\\
231	0.0106289077127511\\
232	0.0106292213354756\\
233	0.0106295406157806\\
234	0.0106298656547018\\
235	0.0106301965550427\\
236	0.0106305334214048\\
237	0.010630876360217\\
238	0.0106312254797671\\
239	0.0106315808902325\\
240	0.0106319427037119\\
241	0.0106323110342575\\
242	0.010632685997908\\
243	0.0106330677127212\\
244	0.0106334562988083\\
245	0.010633851878368\\
246	0.0106342545757216\\
247	0.010634664517348\\
248	0.0106350818319202\\
249	0.010635506650342\\
250	0.0106359391057849\\
251	0.0106363793337266\\
252	0.010636827471989\\
253	0.0106372836607783\\
254	0.010637748042724\\
255	0.0106382207629208\\
256	0.0106387019689689\\
257	0.0106391918110173\\
258	0.010639690441806\\
259	0.0106401980167104\\
260	0.0106407146937859\\
261	0.0106412406338131\\
262	0.0106417760003446\\
263	0.0106423209597525\\
264	0.0106428756812758\\
265	0.0106434403370706\\
266	0.0106440151022595\\
267	0.0106446001549829\\
268	0.010645195676451\\
269	0.0106458018509966\\
270	0.0106464188661295\\
271	0.0106470469125917\\
272	0.0106476861844155\\
273	0.0106483368789849\\
274	0.0106489991971045\\
275	0.0106496733430687\\
276	0.0106503595246674\\
277	0.0106510579528655\\
278	0.0106517688421623\\
279	0.0106524924113025\\
280	0.0106532288827512\\
281	0.0106539784827518\\
282	0.0106547414413842\\
283	0.010655517992624\\
284	0.0106563083744015\\
285	0.0106571128286613\\
286	0.0106579316014228\\
287	0.0106587649428398\\
288	0.0106596131072611\\
289	0.0106604763532912\\
290	0.0106613549438508\\
291	0.0106622491462371\\
292	0.0106631592321841\\
293	0.0106640854779232\\
294	0.0106650281642423\\
295	0.0106659875765456\\
296	0.0106669640049118\\
297	0.0106679577441526\\
298	0.0106689690938695\\
299	0.0106699983585101\\
300	0.0106710458474235\\
301	0.0106721118749139\\
302	0.0106731967602937\\
303	0.0106743008279346\\
304	0.010675424407318\\
305	0.010676567833083\\
306	0.0106777314450737\\
307	0.0106789155883842\\
308	0.0106801206134012\\
309	0.0106813468758425\\
310	0.0106825947367876\\
311	0.0106838645626918\\
312	0.0106851567253783\\
313	0.0106864716020311\\
314	0.0106878095754062\\
315	0.0106891710350984\\
316	0.0106905563795825\\
317	0.0106919660081205\\
318	0.0106934003259084\\
319	0.0106948597441366\\
320	0.0106963446800506\\
321	0.0106978555570098\\
322	0.0106993928045455\\
323	0.0107009568584179\\
324	0.0107025481606708\\
325	0.0107041671596848\\
326	0.0107058143102274\\
327	0.0107074900735009\\
328	0.0107091949171853\\
329	0.010710929315478\\
330	0.0107126937491269\\
331	0.0107144887054575\\
332	0.0107163146783914\\
333	0.0107181721684569\\
334	0.0107200616827896\\
335	0.0107219837351306\\
336	0.010723938845848\\
337	0.0107259275420947\\
338	0.0107279503585661\\
339	0.0107300078409435\\
340	0.0107321005627026\\
341	0.0107342292239689\\
342	0.0107363943480571\\
343	0.0107385958861782\\
344	0.0107408343932305\\
345	0.0107431104299265\\
346	0.0107454245626258\\
347	0.010747777363131\\
348	0.0107501694084429\\
349	0.0107526012805055\\
350	0.0107550735662609\\
351	0.0107575868679606\\
352	0.0107601418041516\\
353	0.0107627389999947\\
354	0.0107653790872289\\
355	0.0107680627041626\\
356	0.0107707904957619\\
357	0.0107735631140218\\
358	0.0107763812190675\\
359	0.0107792454811975\\
360	0.0107821565739881\\
361	0.0107851151247211\\
362	0.0107881218220222\\
363	0.0107911773602134\\
364	0.0107942824390169\\
365	0.0107974377632159\\
366	0.0108006440422689\\
367	0.0108039019898689\\
368	0.0108072123234445\\
369	0.0108105757635928\\
370	0.0108139930334371\\
371	0.010817464857899\\
372	0.0108209919628695\\
373	0.0108245750742546\\
374	0.0108282149168441\\
375	0.0108319122128838\\
376	0.0108356676800647\\
377	0.0108394820282724\\
378	0.0108433559539194\\
379	0.0108472901320757\\
380	0.010851285228192\\
381	0.010855342141211\\
382	0.0108594615026474\\
383	0.0108636439247445\\
384	0.0108678899989933\\
385	0.0108722002891287\\
386	0.0108765753131043\\
387	0.010881015487897\\
388	0.0108855209364959\\
389	0.0108900906808207\\
390	0.0108947293294273\\
391	0.0108994384454068\\
392	0.0109042184936531\\
393	0.0109090698933269\\
394	0.0109139930137216\\
395	0.0109189881700072\\
396	0.010924055618897\\
397	0.0109291955542944\\
398	0.0109344081029983\\
399	0.010939693320565\\
400	0.0109450511874538\\
401	0.0109504816056073\\
402	0.0109559843956196\\
403	0.0109615592945567\\
404	0.0109672059540069\\
405	0.0109729239359125\\
406	0.0109787126903555\\
407	0.0109845714473305\\
408	0.0109904994688752\\
409	0.0109964963741616\\
410	0.0110025592449037\\
411	0.011008686953288\\
412	0.0110148782860777\\
413	0.0110211218543512\\
414	0.0110272124634939\\
415	0.011033403370867\\
416	0.0110396956533818\\
417	0.0110460903728176\\
418	0.0110525885748834\\
419	0.0110591912881104\\
420	0.0110658995219711\\
421	0.011072714262334\\
422	0.0110796364584252\\
423	0.0110866669831162\\
424	0.0110938065086708\\
425	0.0111010551110109\\
426	0.011108413941341\\
427	0.011115888277376\\
428	0.0111234789162698\\
429	0.0111311866042172\\
430	0.0111390120335635\\
431	0.0111469558399703\\
432	0.0111550185996715\\
433	0.0111632008268637\\
434	0.0111715029712725\\
435	0.011179925415924\\
436	0.011188468475186\\
437	0.0111971323931798\\
438	0.0112059173425158\\
439	0.0112148234231933\\
440	0.0112238506605836\\
441	0.0112329990288847\\
442	0.0112422684650907\\
443	0.011251658841322\\
444	0.011261169968479\\
445	0.0112708016013562\\
446	0.0112805534457133\\
447	0.0112904251693042\\
448	0.0113004164263859\\
449	0.0113105269413681\\
450	0.0113207569657761\\
451	0.011331104920786\\
452	0.0113415699663022\\
453	0.0113521518584216\\
454	0.011362850437223\\
455	0.0113736656422712\\
456	0.0113845975019663\\
457	0.0113956462149702\\
458	0.0114068201012005\\
459	0.0114181166633648\\
460	0.0114295375127436\\
461	0.0114410832029517\\
462	0.0114527522639768\\
463	0.011464543201394\\
464	0.0114764545193957\\
465	0.0114884847953951\\
466	0.0115006327062504\\
467	0.0115128970455312\\
468	0.0115252767847158\\
469	0.0115377711302869\\
470	0.0115503795495704\\
471	0.011563101670566\\
472	0.0115755976807566\\
473	0.0115882001439084\\
474	0.0116010202544778\\
475	0.0116140605745417\\
476	0.0116273238798091\\
477	0.0116408127119817\\
478	0.0116545295377051\\
479	0.0116684768044543\\
480	0.0116826569483087\\
481	0.0116970724148852\\
482	0.0117117257691259\\
483	0.0117266193775953\\
484	0.0117417544509667\\
485	0.0117571330750902\\
486	0.0117727567101795\\
487	0.0117886298437595\\
488	0.0118047550760018\\
489	0.0118211339007117\\
490	0.011837767558486\\
491	0.0118546570139821\\
492	0.0118718029897175\\
493	0.0118892060005306\\
494	0.0119068664993603\\
495	0.0119247837122073\\
496	0.011942957382418\\
497	0.0119613869949593\\
498	0.0119800716843142\\
499	0.0119990103335995\\
500	0.0120182035427005\\
501	0.0120376548269237\\
502	0.0120573684259205\\
503	0.0120773476901518\\
504	0.0120975951168858\\
505	0.012118112180513\\
506	0.0121389099196749\\
507	0.0121599872239522\\
508	0.0121813383573198\\
509	0.0122029558282275\\
510	0.012224828193531\\
511	0.0122477291121339\\
512	0.0122713864550201\\
513	0.0122948475956738\\
514	0.0123180885618494\\
515	0.012341084307241\\
516	0.012363808750135\\
517	0.0123862348531717\\
518	0.0124083347783927\\
519	0.0124300798698871\\
520	0.0124514407075311\\
521	0.0124723877471211\\
522	0.0124926180680551\\
523	0.0125117262421685\\
524	0.0125305169289231\\
525	0.0125489744278018\\
526	0.0125670837191241\\
527	0.012584830616434\\
528	0.012602201847976\\
529	0.0126191852959122\\
530	0.0126357702239889\\
531	0.0126519472870393\\
532	0.0126677084749431\\
533	0.0126829988156499\\
534	0.0126971869694241\\
535	0.0127111858830165\\
536	0.0127249879971069\\
537	0.0127385859261309\\
538	0.0127519724329252\\
539	0.0127651403910796\\
540	0.0127780827307337\\
541	0.0127907923608671\\
542	0.0128032620862291\\
543	0.0128154844882925\\
544	0.0128274517754999\\
545	0.0128391813037485\\
546	0.0128506648505122\\
547	0.0128618822363213\\
548	0.0128728117435316\\
549	0.012883430252791\\
550	0.0128937030963321\\
551	0.0129038504641312\\
552	0.0129140184754871\\
553	0.0129243824503119\\
554	0.0129349487963308\\
555	0.0129457237688582\\
556	0.0129567129855521\\
557	0.0129679204772345\\
558	0.0129793502337498\\
559	0.0129910061859714\\
560	0.0130028921879905\\
561	0.013015011997797\\
562	0.0130273692250912\\
563	0.0130399672839607\\
564	0.01305280935119\\
565	0.0130658983185893\\
566	0.0130792371516619\\
567	0.0130928313304846\\
568	0.0131066869229398\\
569	0.0131208108625806\\
570	0.013135528507557\\
571	0.0131515524531138\\
572	0.0131676617139305\\
573	0.0131838436215691\\
574	0.0132000978213791\\
575	0.0132172308088553\\
576	0.0132334596928721\\
577	0.0132494718756232\\
578	0.0132655223898101\\
579	0.0132812630730379\\
580	0.0132964667288486\\
581	0.0133119372502694\\
582	0.0133278177619972\\
583	0.0133433021010561\\
584	0.0133579650643655\\
585	0.0133720933187266\\
586	0.0133849338761277\\
587	0.0133966572273018\\
588	0.013407796560705\\
589	0.013418470452656\\
590	0.0134287051149722\\
591	0.0134382039281232\\
592	0.0134466094533236\\
593	0.0134548583791426\\
594	0.0134637112677116\\
595	0.0134740407759367\\
596	0.0134889777516421\\
597	0.0135160557778789\\
598	0.0135751148335434\\
599	0\\
600	0\\
};
\addplot [color=mycolor19,solid,forget plot]
  table[row sep=crcr]{%
1	0.0108784198979435\\
2	0.0108784238033297\\
3	0.0108784277798183\\
4	0.010878431828705\\
5	0.0108784359513092\\
6	0.0108784401489745\\
7	0.0108784444230692\\
8	0.0108784487749868\\
9	0.0108784532061459\\
10	0.0108784577179917\\
11	0.0108784623119954\\
12	0.0108784669896555\\
13	0.0108784717524978\\
14	0.0108784766020762\\
15	0.0108784815399732\\
16	0.0108784865678001\\
17	0.0108784916871979\\
18	0.0108784968998379\\
19	0.0108785022074219\\
20	0.0108785076116828\\
21	0.0108785131143857\\
22	0.0108785187173279\\
23	0.0108785244223397\\
24	0.0108785302312852\\
25	0.0108785361460625\\
26	0.0108785421686049\\
27	0.0108785483008811\\
28	0.0108785545448958\\
29	0.0108785609026908\\
30	0.0108785673763455\\
31	0.0108785739679774\\
32	0.010878580679743\\
33	0.0108785875138385\\
34	0.0108785944725004\\
35	0.0108786015580065\\
36	0.0108786087726765\\
37	0.0108786161188728\\
38	0.0108786235990012\\
39	0.0108786312155119\\
40	0.0108786389709002\\
41	0.0108786468677073\\
42	0.0108786549085214\\
43	0.010878663095978\\
44	0.0108786714327616\\
45	0.0108786799216059\\
46	0.010878688565295\\
47	0.0108786973666643\\
48	0.0108787063286016\\
49	0.0108787154540478\\
50	0.010878724745998\\
51	0.0108787342075026\\
52	0.0108787438416683\\
53	0.0108787536516589\\
54	0.0108787636406967\\
55	0.0108787738120634\\
56	0.0108787841691012\\
57	0.0108787947152141\\
58	0.0108788054538685\\
59	0.0108788163885953\\
60	0.0108788275229901\\
61	0.0108788388607152\\
62	0.0108788504055001\\
63	0.0108788621611436\\
64	0.0108788741315141\\
65	0.0108788863205518\\
66	0.0108788987322694\\
67	0.0108789113707538\\
68	0.0108789242401673\\
69	0.0108789373447491\\
70	0.0108789506888165\\
71	0.0108789642767667\\
72	0.010878978113078\\
73	0.0108789922023113\\
74	0.010879006549112\\
75	0.010879021158211\\
76	0.0108790360344267\\
77	0.0108790511826664\\
78	0.010879066607928\\
79	0.010879082315302\\
80	0.0108790983099725\\
81	0.0108791145972195\\
82	0.0108791311824207\\
83	0.0108791480710528\\
84	0.010879165268694\\
85	0.0108791827810252\\
86	0.0108792006138323\\
87	0.0108792187730083\\
88	0.0108792372645547\\
89	0.0108792560945841\\
90	0.0108792752693219\\
91	0.0108792947951083\\
92	0.010879314678401\\
93	0.0108793349257766\\
94	0.0108793555439332\\
95	0.0108793765396928\\
96	0.0108793979200032\\
97	0.0108794196919405\\
98	0.0108794418627117\\
99	0.0108794644396564\\
100	0.0108794874302503\\
101	0.0108795108421066\\
102	0.0108795346829795\\
103	0.010879558960766\\
104	0.0108795836835091\\
105	0.0108796088594003\\
106	0.0108796344967822\\
107	0.0108796606041513\\
108	0.0108796871901612\\
109	0.0108797142636249\\
110	0.0108797418335184\\
111	0.0108797699089829\\
112	0.0108797984993287\\
113	0.0108798276140376\\
114	0.0108798572627665\\
115	0.0108798874553504\\
116	0.0108799182018056\\
117	0.0108799495123335\\
118	0.0108799813973232\\
119	0.0108800138673558\\
120	0.0108800469332073\\
121	0.0108800806058524\\
122	0.0108801148964684\\
123	0.0108801498164383\\
124	0.0108801853773552\\
125	0.0108802215910258\\
126	0.0108802584694744\\
127	0.0108802960249469\\
128	0.0108803342699148\\
129	0.0108803732170796\\
130	0.0108804128793766\\
131	0.0108804532699793\\
132	0.0108804944023041\\
133	0.0108805362900143\\
134	0.0108805789470248\\
135	0.0108806223875069\\
136	0.0108806666258925\\
137	0.0108807116768795\\
138	0.010880757555436\\
139	0.0108808042768057\\
140	0.010880851856513\\
141	0.0108809003103675\\
142	0.01088094965447\\
143	0.0108809999052172\\
144	0.0108810510793076\\
145	0.0108811031937466\\
146	0.0108811562658522\\
147	0.0108812103132609\\
148	0.0108812653539333\\
149	0.01088132140616\\
150	0.0108813784885678\\
151	0.0108814366201258\\
152	0.0108814958201512\\
153	0.0108815561083163\\
154	0.0108816175046545\\
155	0.0108816800295671\\
156	0.0108817437038298\\
157	0.0108818085485997\\
158	0.010881874585422\\
159	0.0108819418362374\\
160	0.010882010323389\\
161	0.0108820800696295\\
162	0.0108821510981292\\
163	0.0108822234324828\\
164	0.0108822970967176\\
165	0.0108823721153015\\
166	0.0108824485131504\\
167	0.0108825263156366\\
168	0.0108826055485975\\
169	0.0108826862383431\\
170	0.0108827684116654\\
171	0.0108828520958466\\
172	0.0108829373186683\\
173	0.0108830241084202\\
174	0.0108831124939094\\
175	0.01088320250447\\
176	0.010883294169972\\
177	0.0108833875208317\\
178	0.0108834825880209\\
179	0.0108835794030773\\
180	0.0108836779981144\\
181	0.0108837784058321\\
182	0.0108838806595273\\
183	0.0108839847931041\\
184	0.0108840908410853\\
185	0.0108841988386233\\
186	0.0108843088215112\\
187	0.0108844208261943\\
188	0.0108845348897822\\
189	0.0108846510500601\\
190	0.0108847693455011\\
191	0.0108848898152787\\
192	0.0108850124992789\\
193	0.0108851374381133\\
194	0.0108852646731315\\
195	0.0108853942464348\\
196	0.0108855262008892\\
197	0.0108856605801389\\
198	0.0108857974286206\\
199	0.0108859367915769\\
200	0.0108860787150708\\
201	0.0108862232460005\\
202	0.0108863704321138\\
203	0.010886520322023\\
204	0.0108866729652205\\
205	0.0108868284120938\\
206	0.0108869867139417\\
207	0.0108871479229899\\
208	0.0108873120924075\\
209	0.0108874792763233\\
210	0.0108876495298424\\
211	0.0108878229090638\\
212	0.010887999471097\\
213	0.0108881792740799\\
214	0.0108883623771967\\
215	0.0108885488406958\\
216	0.0108887387259082\\
217	0.0108889320952662\\
218	0.0108891290123224\\
219	0.0108893295417687\\
220	0.0108895337494562\\
221	0.0108897417024143\\
222	0.0108899534688717\\
223	0.0108901691182759\\
224	0.0108903887213145\\
225	0.0108906123499357\\
226	0.0108908400773699\\
227	0.0108910719781508\\
228	0.0108913081281377\\
229	0.0108915486045374\\
230	0.0108917934859265\\
231	0.0108920428522746\\
232	0.0108922967849665\\
233	0.0108925553668262\\
234	0.0108928186821403\\
235	0.0108930868166816\\
236	0.0108933598577338\\
237	0.0108936378941154\\
238	0.0108939210162049\\
239	0.0108942093159653\\
240	0.0108945028869702\\
241	0.0108948018244284\\
242	0.0108951062252106\\
243	0.0108954161878751\\
244	0.0108957318126941\\
245	0.0108960532016807\\
246	0.0108963804586157\\
247	0.0108967136890745\\
248	0.010897053000455\\
249	0.0108973985020049\\
250	0.0108977503048499\\
251	0.0108981085220217\\
252	0.0108984732684866\\
253	0.0108988446611744\\
254	0.0108992228190069\\
255	0.0108996078629279\\
256	0.010899999915932\\
257	0.010900399103095\\
258	0.0109008055516042\\
259	0.0109012193907886\\
260	0.0109016407521501\\
261	0.0109020697693951\\
262	0.0109025065784662\\
263	0.0109029513175749\\
264	0.0109034041272348\\
265	0.0109038651502957\\
266	0.0109043345319781\\
267	0.0109048124199093\\
268	0.0109052989641607\\
269	0.010905794317286\\
270	0.0109062986343622\\
271	0.0109068120730327\\
272	0.0109073347935559\\
273	0.0109078669588656\\
274	0.0109084087346604\\
275	0.0109089602895707\\
276	0.0109095217954896\\
277	0.01091009342786\\
278	0.0109106753608225\\
279	0.0109112677692281\\
280	0.010911870837355\\
281	0.0109124847526095\\
282	0.0109131097055786\\
283	0.0109137458900847\\
284	0.01091439350324\\
285	0.0109150527455036\\
286	0.0109157238207387\\
287	0.0109164069362715\\
288	0.0109171023029514\\
289	0.0109178101352125\\
290	0.0109185306511361\\
291	0.0109192640725147\\
292	0.0109200106249174\\
293	0.0109207705377558\\
294	0.010921544044352\\
295	0.0109223313820069\\
296	0.0109231327920697\\
297	0.0109239485200079\\
298	0.0109247788154782\\
299	0.0109256239323974\\
300	0.010926484129013\\
301	0.0109273596679739\\
302	0.0109282508163996\\
303	0.0109291578459477\\
304	0.0109300810328794\\
305	0.0109310206581215\\
306	0.010931977007324\\
307	0.0109329503709109\\
308	0.0109339410441221\\
309	0.0109349493270397\\
310	0.0109359755245847\\
311	0.0109370199464508\\
312	0.0109380829068948\\
313	0.0109391647242095\\
314	0.0109402657196373\\
315	0.0109413862163965\\
316	0.0109425265493869\\
317	0.0109436871051978\\
318	0.0109448682190518\\
319	0.0109460702306741\\
320	0.0109472934843042\\
321	0.010948538328704\\
322	0.0109498051171631\\
323	0.0109510942075015\\
324	0.0109524059620685\\
325	0.0109537407477376\\
326	0.0109550989358991\\
327	0.0109564809024469\\
328	0.0109578870277628\\
329	0.0109593176966948\\
330	0.0109607732985324\\
331	0.0109622542269752\\
332	0.0109637608800978\\
333	0.0109652936603085\\
334	0.0109668529743026\\
335	0.0109684392330102\\
336	0.0109700528515367\\
337	0.0109716942490973\\
338	0.0109733638489393\\
339	0.0109750620782429\\
340	0.0109767893679639\\
341	0.0109785461525363\\
342	0.0109803328720018\\
343	0.0109821499752176\\
344	0.0109839979147761\\
345	0.0109858771468368\\
346	0.0109877881309379\\
347	0.0109897313297781\\
348	0.0109917072089557\\
349	0.0109937162366201\\
350	0.0109957588828856\\
351	0.0109978356204898\\
352	0.0109999469246025\\
353	0.0110020932713079\\
354	0.0110042751371526\\
355	0.0110064929986853\\
356	0.0110087473320744\\
357	0.0110110386131371\\
358	0.0110133673190802\\
359	0.0110157339373973\\
360	0.0110181390073736\\
361	0.0110205831461489\\
362	0.0110230662340049\\
363	0.0110255887305893\\
364	0.0110281510924594\\
365	0.0110307537725681\\
366	0.0110333972197378\\
367	0.0110360818781258\\
368	0.0110388081866868\\
369	0.0110415765786356\\
370	0.0110443874809169\\
371	0.0110472413136864\\
372	0.0110501384898063\\
373	0.0110530794143414\\
374	0.0110560644840054\\
375	0.0110590940863745\\
376	0.0110621685983069\\
377	0.0110652883818427\\
378	0.0110684537722806\\
379	0.0110716650418028\\
380	0.0110749222850876\\
381	0.0110782250516535\\
382	0.0110815763520183\\
383	0.0110849764849384\\
384	0.0110884257011822\\
385	0.0110919242237003\\
386	0.0110954722459129\\
387	0.0110990699300318\\
388	0.0111027174054295\\
389	0.0111064147665355\\
390	0.0111101620326892\\
391	0.0111139591744608\\
392	0.0111178061214048\\
393	0.0111217027612495\\
394	0.0111256489394152\\
395	0.0111296444589321\\
396	0.0111336890808424\\
397	0.0111377825251812\\
398	0.0111419244726446\\
399	0.0111461145670698\\
400	0.011150352418867\\
401	0.0111546376095694\\
402	0.0111589696977059\\
403	0.0111633482262796\\
404	0.0111677727323193\\
405	0.0111722427594448\\
406	0.0111767578739891\\
407	0.0111813176719716\\
408	0.0111859218596327\\
409	0.0111905704596121\\
410	0.0111952624676152\\
411	0.0111999978281606\\
412	0.0112047765992291\\
413	0.0112095912177072\\
414	0.0112142863462451\\
415	0.0112190672101078\\
416	0.0112239352313215\\
417	0.0112288918548266\\
418	0.0112339385493594\\
419	0.0112390768083189\\
420	0.0112443081503926\\
421	0.0112496341192204\\
422	0.0112550562798195\\
423	0.0112605762041164\\
424	0.0112661954178444\\
425	0.0112719151897949\\
426	0.0112777372397674\\
427	0.0112836652336045\\
428	0.0112897008790337\\
429	0.0112958459076668\\
430	0.0113021020757675\\
431	0.0113084711650986\\
432	0.0113149549838525\\
433	0.0113215553676806\\
434	0.0113282741809034\\
435	0.0113351133178814\\
436	0.0113420747043324\\
437	0.0113491602985643\\
438	0.0113563720933495\\
439	0.0113637121177307\\
440	0.0113711824390299\\
441	0.0113787851647043\\
442	0.0113865224441316\\
443	0.0113943964708904\\
444	0.0114024094850894\\
445	0.0114105637765687\\
446	0.0114188616887625\\
447	0.0114273056217474\\
448	0.0114358980320699\\
449	0.0114446414381959\\
450	0.0114535384248832\\
451	0.011462591653238\\
452	0.0114718038578371\\
453	0.0114811778457777\\
454	0.0114907164967417\\
455	0.0115004227586098\\
456	0.011510299669983\\
457	0.0115203503648837\\
458	0.0115305779215258\\
459	0.0115409855019593\\
460	0.0115515763502002\\
461	0.0115623538105289\\
462	0.011573321345073\\
463	0.0115844825408848\\
464	0.0115958410946219\\
465	0.011607400741314\\
466	0.0116191654543405\\
467	0.0116311395543891\\
468	0.0116433273324563\\
469	0.0116557331313184\\
470	0.0116683614040626\\
471	0.0116812167210318\\
472	0.0116943100675087\\
473	0.0117076466145337\\
474	0.0117212288587474\\
475	0.0117350586523905\\
476	0.0117491379682479\\
477	0.0117634720641831\\
478	0.0117780637472259\\
479	0.0117929149586821\\
480	0.0118080273423896\\
481	0.0118234022401821\\
482	0.0118390407265265\\
483	0.0118549434573832\\
484	0.0118711102209801\\
485	0.0118875408495325\\
486	0.0119042345399909\\
487	0.0119211916958068\\
488	0.0119384114516301\\
489	0.0119558939986645\\
490	0.0119736426188383\\
491	0.0119916598958243\\
492	0.0120099476874576\\
493	0.0120285067717896\\
494	0.0120473366367891\\
495	0.0120664353977148\\
496	0.0120857996136615\\
497	0.0121054241229139\\
498	0.0121253019467027\\
499	0.0121454245213979\\
500	0.0121669864780814\\
501	0.0121886598665547\\
502	0.0122101545398785\\
503	0.0122314520477386\\
504	0.012252530409118\\
505	0.0122733666930203\\
506	0.0122939370252466\\
507	0.0123142168386173\\
508	0.0123341810788034\\
509	0.0123538044294089\\
510	0.0123730608856005\\
511	0.0123914575771536\\
512	0.0124090854357912\\
513	0.0124264484037729\\
514	0.0124435345677051\\
515	0.0124603328222785\\
516	0.0124768330174607\\
517	0.0124930261129636\\
518	0.0125089043328037\\
519	0.0125244613817106\\
520	0.0125396927277238\\
521	0.0125545956971607\\
522	0.012568974804646\\
523	0.0125825910149726\\
524	0.0125961058527152\\
525	0.0126095177936548\\
526	0.0126228257858893\\
527	0.0126360292417792\\
528	0.012649128020776\\
529	0.0126621223960243\\
530	0.0126750130027323\\
531	0.012687800772377\\
532	0.0127004868511807\\
533	0.0127130744938489\\
534	0.0127255934823642\\
535	0.0127380358133138\\
536	0.0127503928385556\\
537	0.0127626551809634\\
538	0.0127748126411372\\
539	0.0127868540945184\\
540	0.0127987673784488\\
541	0.0128105391689269\\
542	0.0128221548461821\\
543	0.0128335983486842\\
544	0.0128448520149445\\
545	0.0128558952342837\\
546	0.0128667053002203\\
547	0.0128772577225203\\
548	0.0128875263142159\\
549	0.0128974718243012\\
550	0.01290741787158\\
551	0.0129175138458269\\
552	0.0129278290503619\\
553	0.0129383682213716\\
554	0.0129491360671807\\
555	0.0129601372657898\\
556	0.0129713764884326\\
557	0.0129828584761428\\
558	0.0129945880030828\\
559	0.0130065698691541\\
560	0.013018808890978\\
561	0.0130313098860317\\
562	0.0130440799239086\\
563	0.013057127769762\\
564	0.0130704633598196\\
565	0.0130842809431066\\
566	0.0130996126366805\\
567	0.013115059447439\\
568	0.0131306115337819\\
569	0.013146304511449\\
570	0.0131620042189724\\
571	0.0131777619914875\\
572	0.0131935571303295\\
573	0.0132092235907178\\
574	0.0132247963192703\\
575	0.0132399000096878\\
576	0.0132543756406824\\
577	0.0132704253583581\\
578	0.0132865863105213\\
579	0.0133020081118863\\
580	0.0133173908781042\\
581	0.0133320104924939\\
582	0.0133451046723532\\
583	0.0133576903926825\\
584	0.0133699320550395\\
585	0.0133816601464576\\
586	0.0133924615300803\\
587	0.0134024915452123\\
588	0.0134122145702396\\
589	0.0134215984264863\\
590	0.0134305710816451\\
591	0.0134390764668377\\
592	0.0134472000954262\\
593	0.0134551345125145\\
594	0.0134637112677116\\
595	0.0134740407759367\\
596	0.0134889777516421\\
597	0.0135160557778789\\
598	0.0135751148335434\\
599	0\\
600	0\\
};
\addplot [color=red!50!mycolor17,solid,forget plot]
  table[row sep=crcr]{%
1	0.0110134653792917\\
2	0.0110134679347945\\
3	0.0110134705370157\\
4	0.0110134731868078\\
5	0.0110134758850386\\
6	0.0110134786325922\\
7	0.0110134814303687\\
8	0.0110134842792851\\
9	0.0110134871802752\\
10	0.0110134901342898\\
11	0.0110134931422977\\
12	0.0110134962052852\\
13	0.0110134993242571\\
14	0.0110135025002365\\
15	0.0110135057342657\\
16	0.0110135090274062\\
17	0.0110135123807391\\
18	0.0110135157953654\\
19	0.0110135192724068\\
20	0.0110135228130056\\
21	0.0110135264183253\\
22	0.0110135300895511\\
23	0.0110135338278901\\
24	0.0110135376345717\\
25	0.0110135415108485\\
26	0.011013545457996\\
27	0.0110135494773136\\
28	0.011013553570125\\
29	0.0110135577377783\\
30	0.0110135619816468\\
31	0.0110135663031296\\
32	0.0110135707036514\\
33	0.011013575184664\\
34	0.011013579747646\\
35	0.0110135843941036\\
36	0.0110135891255713\\
37	0.011013593943612\\
38	0.011013598849818\\
39	0.0110136038458115\\
40	0.0110136089332448\\
41	0.0110136141138013\\
42	0.0110136193891958\\
43	0.0110136247611753\\
44	0.0110136302315195\\
45	0.0110136358020417\\
46	0.0110136414745888\\
47	0.0110136472510427\\
48	0.0110136531333206\\
49	0.0110136591233756\\
50	0.0110136652231975\\
51	0.0110136714348137\\
52	0.0110136777602894\\
53	0.0110136842017291\\
54	0.0110136907612765\\
55	0.011013697441116\\
56	0.0110137042434728\\
57	0.0110137111706145\\
58	0.011013718224851\\
59	0.0110137254085362\\
60	0.011013732724068\\
61	0.0110137401738898\\
62	0.0110137477604912\\
63	0.0110137554864086\\
64	0.0110137633542266\\
65	0.0110137713665785\\
66	0.0110137795261475\\
67	0.0110137878356673\\
68	0.0110137962979238\\
69	0.0110138049157552\\
70	0.0110138136920537\\
71	0.0110138226297663\\
72	0.0110138317318956\\
73	0.0110138410015015\\
74	0.0110138504417015\\
75	0.0110138600556726\\
76	0.0110138698466519\\
77	0.011013879817938\\
78	0.011013889972892\\
79	0.011013900314939\\
80	0.011013910847569\\
81	0.0110139215743385\\
82	0.0110139324988714\\
83	0.0110139436248606\\
84	0.0110139549560693\\
85	0.011013966496332\\
86	0.0110139782495564\\
87	0.0110139902197245\\
88	0.011014002410894\\
89	0.0110140148272\\
90	0.0110140274728561\\
91	0.0110140403521563\\
92	0.0110140534694764\\
93	0.0110140668292756\\
94	0.0110140804360978\\
95	0.0110140942945738\\
96	0.0110141084094226\\
97	0.011014122785453\\
98	0.0110141374275656\\
99	0.0110141523407544\\
100	0.0110141675301088\\
101	0.011014183000815\\
102	0.0110141987581583\\
103	0.0110142148075249\\
104	0.0110142311544036\\
105	0.011014247804388\\
106	0.0110142647631784\\
107	0.011014282036584\\
108	0.0110142996305247\\
109	0.0110143175510335\\
110	0.0110143358042587\\
111	0.0110143543964657\\
112	0.0110143733340396\\
113	0.0110143926234875\\
114	0.0110144122714405\\
115	0.0110144322846566\\
116	0.0110144526700226\\
117	0.0110144734345569\\
118	0.0110144945854119\\
119	0.0110145161298763\\
120	0.0110145380753784\\
121	0.0110145604294881\\
122	0.0110145831999196\\
123	0.0110146063945349\\
124	0.0110146300213455\\
125	0.0110146540885164\\
126	0.011014678604368\\
127	0.0110147035773798\\
128	0.0110147290161931\\
129	0.0110147549296141\\
130	0.011014781326617\\
131	0.0110148082163475\\
132	0.0110148356081256\\
133	0.0110148635114492\\
134	0.0110148919359975\\
135	0.0110149208916342\\
136	0.0110149503884112\\
137	0.0110149804365724\\
138	0.0110150110465567\\
139	0.0110150422290022\\
140	0.0110150739947501\\
141	0.011015106354848\\
142	0.0110151393205542\\
143	0.0110151729033416\\
144	0.0110152071149019\\
145	0.0110152419671495\\
146	0.0110152774722258\\
147	0.0110153136425035\\
148	0.0110153504905912\\
149	0.0110153880293374\\
150	0.0110154262718355\\
151	0.0110154652314281\\
152	0.0110155049217118\\
153	0.0110155453565422\\
154	0.0110155865500383\\
155	0.0110156285165882\\
156	0.0110156712708533\\
157	0.0110157148277744\\
158	0.011015759202576\\
159	0.0110158044107726\\
160	0.0110158504681733\\
161	0.011015897390888\\
162	0.0110159451953327\\
163	0.0110159938982354\\
164	0.0110160435166419\\
165	0.011016094067922\\
166	0.0110161455697752\\
167	0.0110161980402373\\
168	0.0110162514976864\\
169	0.0110163059608496\\
170	0.0110163614488093\\
171	0.0110164179810101\\
172	0.0110164755772656\\
173	0.0110165342577649\\
174	0.0110165940430804\\
175	0.0110166549541744\\
176	0.0110167170124063\\
177	0.0110167802395407\\
178	0.0110168446577545\\
179	0.0110169102896447\\
180	0.0110169771582362\\
181	0.0110170452869901\\
182	0.0110171146998116\\
183	0.0110171854210582\\
184	0.0110172574755486\\
185	0.0110173308885709\\
186	0.0110174056858914\\
187	0.0110174818937639\\
188	0.0110175595389384\\
189	0.0110176386486703\\
190	0.0110177192507304\\
191	0.0110178013734138\\
192	0.01101788504555\\
193	0.011017970296513\\
194	0.0110180571562311\\
195	0.0110181456551975\\
196	0.0110182358244805\\
197	0.0110183276957343\\
198	0.0110184213012102\\
199	0.011018516673767\\
200	0.011018613846883\\
201	0.0110187128546667\\
202	0.0110188137318694\\
203	0.0110189165138961\\
204	0.0110190212368184\\
205	0.0110191279373861\\
206	0.0110192366530405\\
207	0.0110193474219263\\
208	0.0110194602829054\\
209	0.0110195752755692\\
210	0.011019692440253\\
211	0.0110198118180488\\
212	0.0110199334508196\\
213	0.0110200573812137\\
214	0.0110201836526786\\
215	0.0110203123094759\\
216	0.0110204433966964\\
217	0.0110205769602746\\
218	0.0110207130470049\\
219	0.0110208517045564\\
220	0.0110209929814895\\
221	0.0110211369272717\\
222	0.011021283592294\\
223	0.0110214330278879\\
224	0.0110215852863422\\
225	0.0110217404209203\\
226	0.0110218984858778\\
227	0.0110220595364801\\
228	0.0110222236290209\\
229	0.0110223908208401\\
230	0.0110225611703432\\
231	0.0110227347370196\\
232	0.011022911581462\\
233	0.0110230917653863\\
234	0.0110232753516512\\
235	0.0110234624042784\\
236	0.0110236529884728\\
237	0.0110238471706438\\
238	0.0110240450184256\\
239	0.011024246600699\\
240	0.0110244519876127\\
241	0.0110246612506055\\
242	0.0110248744624279\\
243	0.0110250916971648\\
244	0.0110253130302581\\
245	0.0110255385385295\\
246	0.0110257683002036\\
247	0.0110260023949311\\
248	0.0110262409038126\\
249	0.0110264839094217\\
250	0.0110267314958295\\
251	0.0110269837486279\\
252	0.0110272407549538\\
253	0.0110275026035131\\
254	0.011027769384605\\
255	0.0110280411901456\\
256	0.0110283181136916\\
257	0.0110286002504644\\
258	0.0110288876973734\\
259	0.0110291805530386\\
260	0.0110294789178134\\
261	0.0110297828938067\\
262	0.0110300925849039\\
263	0.011030408096787\\
264	0.0110307295369543\\
265	0.0110310570147378\\
266	0.0110313906413199\\
267	0.0110317305297478\\
268	0.0110320767949457\\
269	0.0110324295537251\\
270	0.0110327889247916\\
271	0.0110331550287499\\
272	0.0110335279881075\\
273	0.0110339079272848\\
274	0.0110342949726584\\
275	0.0110346892527373\\
276	0.0110350908988745\\
277	0.0110355000482535\\
278	0.0110359168578037\\
279	0.011036341441527\\
280	0.0110367738693236\\
281	0.0110372142814829\\
282	0.0110376628206231\\
283	0.011038119631726\\
284	0.0110385848621722\\
285	0.011039058661778\\
286	0.0110395411828326\\
287	0.0110400325801372\\
288	0.0110405330110451\\
289	0.0110410426355035\\
290	0.011041561616097\\
291	0.0110420901180937\\
292	0.0110426283094929\\
293	0.0110431763610761\\
294	0.0110437344464607\\
295	0.0110443027421577\\
296	0.0110448814276331\\
297	0.0110454706853738\\
298	0.0110460707009598\\
299	0.0110466816631413\\
300	0.0110473037639238\\
301	0.0110479371986613\\
302	0.0110485821661579\\
303	0.0110492388687817\\
304	0.0110499075125894\\
305	0.0110505883074666\\
306	0.0110512814672827\\
307	0.0110519872100641\\
308	0.0110527057581858\\
309	0.0110534373385759\\
310	0.0110541821829188\\
311	0.0110549405277995\\
312	0.0110557126146155\\
313	0.0110564986887312\\
314	0.0110572989962471\\
315	0.0110581137732551\\
316	0.0110589432109055\\
317	0.0110597874847164\\
318	0.0110606474748093\\
319	0.0110615234635714\\
320	0.0110624157380557\\
321	0.0110633245900424\\
322	0.0110642503161012\\
323	0.0110651932176533\\
324	0.0110661536010337\\
325	0.0110671317775536\\
326	0.0110681280635622\\
327	0.0110691427805092\\
328	0.0110701762550067\\
329	0.0110712288188905\\
330	0.0110723008092819\\
331	0.0110733925686482\\
332	0.0110745044448631\\
333	0.0110756367912661\\
334	0.0110767899667217\\
335	0.0110779643356769\\
336	0.0110791602682179\\
337	0.0110803781401258\\
338	0.0110816183329304\\
339	0.0110828812339609\\
340	0.011084167236392\\
341	0.0110854767393046\\
342	0.011086810147729\\
343	0.0110881678725967\\
344	0.0110895503307751\\
345	0.0110909579450985\\
346	0.0110923911443969\\
347	0.0110938503635219\\
348	0.0110953360433703\\
349	0.0110968486309073\\
350	0.0110983885791906\\
351	0.0110999563473751\\
352	0.011101552400711\\
353	0.0111031772105596\\
354	0.0111048312544102\\
355	0.0111065150159073\\
356	0.0111082289849186\\
357	0.0111099736577607\\
358	0.0111117495380706\\
359	0.0111135571406408\\
360	0.0111153970115238\\
361	0.0111172697514782\\
362	0.0111191756327446\\
363	0.0111211151955518\\
364	0.0111230889882079\\
365	0.0111250975673466\\
366	0.0111271414982031\\
367	0.0111292213549222\\
368	0.011131337720903\\
369	0.0111334911891851\\
370	0.0111356823628799\\
371	0.0111379118556521\\
372	0.0111401802922549\\
373	0.0111424883091155\\
374	0.0111448365549561\\
375	0.0111472256913834\\
376	0.0111496563932419\\
377	0.0111521293480687\\
378	0.0111546452525182\\
379	0.0111572047984563\\
380	0.0111598086209993\\
381	0.0111624570801157\\
382	0.0111651521421757\\
383	0.0111678945878793\\
384	0.0111706852044207\\
385	0.0111735247963825\\
386	0.0111764141869729\\
387	0.0111793542193985\\
388	0.0111823457583774\\
389	0.0111853896917847\\
390	0.0111884869330546\\
391	0.0111916384233961\\
392	0.0111948451340808\\
393	0.0111981080689366\\
394	0.0112014282670564\\
395	0.0112048068057313\\
396	0.0112082448036124\\
397	0.0112117434241048\\
398	0.0112153038789915\\
399	0.0112189274322798\\
400	0.0112226154042574\\
401	0.0112263691757384\\
402	0.0112301901924618\\
403	0.0112340799695989\\
404	0.0112380400963005\\
405	0.0112420722401841\\
406	0.0112461781516689\\
407	0.0112503596683496\\
408	0.0112546187186486\\
409	0.0112589573251298\\
410	0.0112633776095775\\
411	0.0112678817873419\\
412	0.0112724721642422\\
413	0.0112771512785521\\
414	0.011281924554925\\
415	0.0112867938806169\\
416	0.0112917611800265\\
417	0.0112968284152576\\
418	0.0113019975866518\\
419	0.011307270733284\\
420	0.0113126499334187\\
421	0.0113181373049183\\
422	0.0113237350055213\\
423	0.0113294452330873\\
424	0.011335270225525\\
425	0.0113412122597856\\
426	0.011347273646975\\
427	0.0113534567280712\\
428	0.0113597638858074\\
429	0.01136619754502\\
430	0.0113727601729619\\
431	0.0113794542795683\\
432	0.0113862824176476\\
433	0.0113932471829123\\
434	0.0114003512137686\\
435	0.011407597191815\\
436	0.0114149878431512\\
437	0.0114225259380576\\
438	0.0114302142869262\\
439	0.0114380557423199\\
440	0.01144605319828\\
441	0.0114542095894351\\
442	0.0114625278898838\\
443	0.011471011111739\\
444	0.0114796623026787\\
445	0.0114884845401612\\
446	0.011497480932532\\
447	0.0115066546281675\\
448	0.0115160088204975\\
449	0.0115255467007211\\
450	0.0115352714778931\\
451	0.0115451863788122\\
452	0.0115552946425947\\
453	0.011565599518629\\
454	0.0115761042645763\\
455	0.0115868121337473\\
456	0.0115977262533013\\
457	0.0116088498542276\\
458	0.0116201863377686\\
459	0.0116317389752061\\
460	0.011643510893824\\
461	0.0116555051725634\\
462	0.0116677248295728\\
463	0.0116801728071662\\
464	0.0116928520231295\\
465	0.0117057652304036\\
466	0.011718913527052\\
467	0.0117322984327091\\
468	0.0117459242297021\\
469	0.0117597933232408\\
470	0.0117739069796817\\
471	0.011788266162539\\
472	0.0118028713846374\\
473	0.0118177228665846\\
474	0.0118328202434625\\
475	0.011848162602471\\
476	0.01186374874409\\
477	0.0118795793659911\\
478	0.0118956582779787\\
479	0.0119119883690854\\
480	0.0119285716844974\\
481	0.011945409426486\\
482	0.0119625018437398\\
483	0.011979848070994\\
484	0.0119974457623746\\
485	0.0120152910551991\\
486	0.0120333784233839\\
487	0.0120517005971582\\
488	0.0120707353628826\\
489	0.0120906807389837\\
490	0.0121104767839451\\
491	0.0121301036789161\\
492	0.0121495413620184\\
493	0.0121687709909655\\
494	0.0121877726395686\\
495	0.0122065248240623\\
496	0.0122250056082003\\
497	0.0122431927167854\\
498	0.0122610636721047\\
499	0.0122785959587757\\
500	0.0122950518261643\\
501	0.0123111336151311\\
502	0.0123269797860266\\
503	0.0123425823202101\\
504	0.0123579329236886\\
505	0.0123730243018242\\
506	0.0123878503906345\\
507	0.0124024064961853\\
508	0.0124166894563258\\
509	0.0124306977496756\\
510	0.0124444317962187\\
511	0.0124575606061362\\
512	0.0124702173215123\\
513	0.0124828196346015\\
514	0.0124953691812993\\
515	0.0125078681970754\\
516	0.0125203195175986\\
517	0.0125327265692689\\
518	0.0125450933481989\\
519	0.0125574243837514\\
520	0.0125697246818943\\
521	0.0125819996524807\\
522	0.0125942628780827\\
523	0.0126065385041128\\
524	0.0126188240929494\\
525	0.0126311169093886\\
526	0.0126434138652448\\
527	0.0126557114580474\\
528	0.0126680057035716\\
529	0.0126802920621906\\
530	0.0126925653590862\\
531	0.0127048196981507\\
532	0.0127170483693706\\
533	0.0127292436616183\\
534	0.0127413955781831\\
535	0.0127534932868318\\
536	0.0127655250522041\\
537	0.0127774781635861\\
538	0.0127893388578269\\
539	0.0128010922371522\\
540	0.0128127221816049\\
541	0.0128242112558164\\
542	0.0128355406097937\\
543	0.0128466898733903\\
544	0.0128576370440959\\
545	0.0128683584230701\\
546	0.0128788284969462\\
547	0.0128890197168221\\
548	0.0128988938146329\\
549	0.0129088204236672\\
550	0.0129189536911403\\
551	0.0129293183806309\\
552	0.0129399199659765\\
553	0.0129507641132976\\
554	0.0129618566619033\\
555	0.0129732036393451\\
556	0.0129848112750305\\
557	0.012996686004706\\
558	0.013008837736785\\
559	0.0130212777315602\\
560	0.0130340188542051\\
561	0.0130480619090636\\
562	0.0130627409734786\\
563	0.0130775458061291\\
564	0.0130924969899236\\
565	0.0131075709060964\\
566	0.013122007812545\\
567	0.0131368560289957\\
568	0.0131521961419429\\
569	0.0131674037980389\\
570	0.0131823061747648\\
571	0.0131962067427473\\
572	0.0132101656552701\\
573	0.0132264507710498\\
574	0.0132427750691417\\
575	0.0132583286052144\\
576	0.0132737542621561\\
577	0.0132880160445443\\
578	0.013301888510655\\
579	0.0133153070109448\\
580	0.0133282044538216\\
581	0.0133403078355908\\
582	0.013351430271074\\
583	0.013362268821379\\
584	0.0133728864246924\\
585	0.0133832261555067\\
586	0.0133932918913093\\
587	0.0134031094508248\\
588	0.0134126433937004\\
589	0.0134218515519248\\
590	0.0134306952401147\\
591	0.0134391556363722\\
592	0.0134472338645631\\
593	0.0134551345125145\\
594	0.0134637112677116\\
595	0.0134740407759367\\
596	0.0134889777516421\\
597	0.0135160557778789\\
598	0.0135751148335434\\
599	0\\
600	0\\
};
\addplot [color=red!40!mycolor19,solid,forget plot]
  table[row sep=crcr]{%
1	0.0110522556087146\\
2	0.0110522577840676\\
3	0.0110522599994303\\
4	0.0110522622555279\\
5	0.0110522645530989\\
6	0.0110522668928955\\
7	0.0110522692756832\\
8	0.0110522717022416\\
9	0.0110522741733645\\
10	0.0110522766898602\\
11	0.0110522792525516\\
12	0.0110522818622766\\
13	0.0110522845198887\\
14	0.0110522872262565\\
15	0.011052289982265\\
16	0.0110522927888151\\
17	0.0110522956468241\\
18	0.0110522985572262\\
19	0.0110523015209729\\
20	0.011052304539033\\
21	0.0110523076123929\\
22	0.0110523107420575\\
23	0.0110523139290498\\
24	0.011052317174412\\
25	0.0110523204792051\\
26	0.0110523238445099\\
27	0.0110523272714271\\
28	0.0110523307610776\\
29	0.0110523343146032\\
30	0.0110523379331666\\
31	0.0110523416179521\\
32	0.0110523453701659\\
33	0.0110523491910367\\
34	0.0110523530818157\\
35	0.0110523570437776\\
36	0.0110523610782207\\
37	0.0110523651864673\\
38	0.0110523693698644\\
39	0.0110523736297843\\
40	0.0110523779676245\\
41	0.0110523823848088\\
42	0.0110523868827876\\
43	0.0110523914630384\\
44	0.0110523961270662\\
45	0.0110524008764043\\
46	0.0110524057126148\\
47	0.0110524106372891\\
48	0.0110524156520482\\
49	0.0110524207585438\\
50	0.0110524259584588\\
51	0.0110524312535075\\
52	0.0110524366454367\\
53	0.011052442136026\\
54	0.0110524477270886\\
55	0.0110524534204722\\
56	0.0110524592180591\\
57	0.0110524651217675\\
58	0.0110524711335517\\
59	0.0110524772554034\\
60	0.0110524834893517\\
61	0.0110524898374646\\
62	0.0110524963018492\\
63	0.0110525028846528\\
64	0.0110525095880635\\
65	0.0110525164143112\\
66	0.0110525233656684\\
67	0.0110525304444509\\
68	0.0110525376530188\\
69	0.0110525449937773\\
70	0.0110525524691778\\
71	0.0110525600817185\\
72	0.0110525678339456\\
73	0.0110525757284543\\
74	0.0110525837678894\\
75	0.0110525919549468\\
76	0.011052600292374\\
77	0.0110526087829718\\
78	0.0110526174295946\\
79	0.0110526262351521\\
80	0.0110526352026101\\
81	0.0110526443349917\\
82	0.0110526536353785\\
83	0.0110526631069116\\
84	0.0110526727527931\\
85	0.011052682576287\\
86	0.0110526925807206\\
87	0.0110527027694859\\
88	0.0110527131460406\\
89	0.0110527237139099\\
90	0.0110527344766871\\
91	0.0110527454380357\\
92	0.0110527566016906\\
93	0.0110527679714593\\
94	0.0110527795512236\\
95	0.011052791344941\\
96	0.0110528033566464\\
97	0.0110528155904532\\
98	0.0110528280505555\\
99	0.0110528407412293\\
100	0.0110528536668343\\
101	0.0110528668318153\\
102	0.0110528802407044\\
103	0.0110528938981226\\
104	0.0110529078087813\\
105	0.0110529219774843\\
106	0.0110529364091297\\
107	0.0110529511087119\\
108	0.0110529660813232\\
109	0.0110529813321562\\
110	0.0110529968665052\\
111	0.0110530126897691\\
112	0.0110530288074526\\
113	0.0110530452251689\\
114	0.0110530619486417\\
115	0.0110530789837075\\
116	0.0110530963363175\\
117	0.0110531140125406\\
118	0.0110531320185649\\
119	0.0110531503607007\\
120	0.0110531690453828\\
121	0.011053188079173\\
122	0.0110532074687622\\
123	0.0110532272209739\\
124	0.0110532473427658\\
125	0.0110532678412332\\
126	0.0110532887236116\\
127	0.0110533099972791\\
128	0.0110533316697599\\
129	0.0110533537487267\\
130	0.0110533762420036\\
131	0.0110533991575695\\
132	0.0110534225035611\\
133	0.0110534462882758\\
134	0.0110534705201748\\
135	0.0110534952078868\\
136	0.011053520360211\\
137	0.0110535459861205\\
138	0.0110535720947659\\
139	0.0110535986954786\\
140	0.0110536257977745\\
141	0.0110536534113575\\
142	0.0110536815461236\\
143	0.0110537102121641\\
144	0.01105373941977\\
145	0.0110537691794357\\
146	0.0110537995018629\\
147	0.0110538303979648\\
148	0.0110538618788705\\
149	0.0110538939559287\\
150	0.0110539266407125\\
151	0.0110539599450236\\
152	0.011053993880897\\
153	0.011054028460605\\
154	0.0110540636966628\\
155	0.0110540996018324\\
156	0.0110541361891278\\
157	0.01105417347182\\
158	0.011054211463442\\
159	0.0110542501777937\\
160	0.0110542896289476\\
161	0.0110543298312535\\
162	0.0110543707993446\\
163	0.0110544125481424\\
164	0.0110544550928631\\
165	0.0110544984490224\\
166	0.0110545426324422\\
167	0.0110545876592561\\
168	0.0110546335459158\\
169	0.0110546803091969\\
170	0.0110547279662056\\
171	0.0110547765343852\\
172	0.0110548260315221\\
173	0.0110548764757532\\
174	0.0110549278855723\\
175	0.0110549802798372\\
176	0.0110550336777769\\
177	0.0110550880989984\\
178	0.0110551435634948\\
179	0.0110552000916522\\
180	0.0110552577042573\\
181	0.0110553164225061\\
182	0.0110553762680106\\
183	0.011055437262808\\
184	0.0110554994293684\\
185	0.0110555627906032\\
186	0.011055627369874\\
187	0.0110556931910014\\
188	0.0110557602782735\\
189	0.0110558286564554\\
190	0.0110558983507982\\
191	0.011055969387049\\
192	0.0110560417914597\\
193	0.0110561155907977\\
194	0.0110561908123555\\
195	0.0110562674839607\\
196	0.011056345633987\\
197	0.0110564252913644\\
198	0.0110565064855901\\
199	0.0110565892467398\\
200	0.0110566736054786\\
201	0.0110567595930726\\
202	0.0110568472414008\\
203	0.0110569365829667\\
204	0.0110570276509108\\
205	0.0110571204790225\\
206	0.0110572151017535\\
207	0.0110573115542297\\
208	0.0110574098722653\\
209	0.0110575100923753\\
210	0.01105761225179\\
211	0.011057716388468\\
212	0.0110578225411112\\
213	0.0110579307491785\\
214	0.0110580410529011\\
215	0.011058153493297\\
216	0.0110582681121869\\
217	0.011058384952209\\
218	0.0110585040568354\\
219	0.0110586254703882\\
220	0.0110587492380556\\
221	0.0110588754059093\\
222	0.011059004020921\\
223	0.0110591351309801\\
224	0.0110592687849114\\
225	0.0110594050324932\\
226	0.0110595439244757\\
227	0.0110596855125994\\
228	0.0110598298496148\\
229	0.0110599769893014\\
230	0.0110601269864876\\
231	0.0110602798970707\\
232	0.0110604357780377\\
233	0.0110605946874859\\
234	0.0110607566846445\\
235	0.0110609218298956\\
236	0.0110610901847968\\
237	0.0110612618121034\\
238	0.011061436775791\\
239	0.0110616151410787\\
240	0.011061796974453\\
241	0.011061982343691\\
242	0.0110621713178857\\
243	0.0110623639674699\\
244	0.0110625603642416\\
245	0.0110627605813898\\
246	0.0110629646935201\\
247	0.0110631727766809\\
248	0.0110633849083906\\
249	0.0110636011676641\\
250	0.0110638216350409\\
251	0.0110640463926123\\
252	0.0110642755240501\\
253	0.011064509114635\\
254	0.0110647472512854\\
255	0.0110649900225869\\
256	0.0110652375188217\\
257	0.0110654898319986\\
258	0.0110657470558829\\
259	0.0110660092860272\\
260	0.0110662766198016\\
261	0.0110665491564247\\
262	0.0110668269969946\\
263	0.0110671102445201\\
264	0.0110673990039514\\
265	0.011067693382212\\
266	0.011067993488229\\
267	0.0110682994329649\\
268	0.0110686113294479\\
269	0.011068929292803\\
270	0.0110692534402823\\
271	0.0110695838912953\\
272	0.0110699207674399\\
273	0.0110702641925361\\
274	0.0110706142926717\\
275	0.0110709711962971\\
276	0.0110713350345256\\
277	0.0110717059424392\\
278	0.0110720840663966\\
279	0.0110724695360078\\
280	0.0110728624619412\\
281	0.0110732629880194\\
282	0.0110736712608364\\
283	0.0110740874298149\\
284	0.0110745116472644\\
285	0.0110749440684417\\
286	0.0110753848516126\\
287	0.0110758341581161\\
288	0.0110762921524307\\
289	0.0110767590022423\\
290	0.0110772348785156\\
291	0.0110777199555667\\
292	0.0110782144111398\\
293	0.011078718426486\\
294	0.0110792321864452\\
295	0.0110797558795322\\
296	0.0110802896980258\\
297	0.0110808338380618\\
298	0.0110813884997312\\
299	0.0110819538871818\\
300	0.0110825302087258\\
301	0.0110831176769525\\
302	0.0110837165088469\\
303	0.0110843269259148\\
304	0.0110849491543147\\
305	0.0110855834249976\\
306	0.0110862299738547\\
307	0.0110868890418727\\
308	0.011087560875299\\
309	0.0110882457258121\\
310	0.0110889438506927\\
311	0.0110896555129733\\
312	0.0110903809815009\\
313	0.0110911205307067\\
314	0.0110918744394063\\
315	0.0110926429862487\\
316	0.0110934264321556\\
317	0.011094225023847\\
318	0.0110950393345729\\
319	0.011095869674358\\
320	0.0110967163593643\\
321	0.0110975797120146\\
322	0.01109846006112\\
323	0.0110993577420091\\
324	0.01110027309666\\
325	0.0111012064738365\\
326	0.0111021582292256\\
327	0.0111031287255795\\
328	0.0111041183328601\\
329	0.0111051274283868\\
330	0.0111061563969883\\
331	0.0111072056311575\\
332	0.0111082755312097\\
333	0.0111093665054456\\
334	0.0111104789703173\\
335	0.0111116133505989\\
336	0.0111127700795612\\
337	0.0111139495991509\\
338	0.0111151523601744\\
339	0.0111163788224865\\
340	0.011117629455184\\
341	0.0111189047368049\\
342	0.0111202051555328\\
343	0.0111215312094084\\
344	0.0111228834065475\\
345	0.0111242622653657\\
346	0.0111256683148103\\
347	0.0111271020945992\\
348	0.011128564155468\\
349	0.0111300550594255\\
350	0.0111315753800175\\
351	0.0111331257026007\\
352	0.0111347066246264\\
353	0.0111363187559361\\
354	0.011137962719067\\
355	0.0111396391495713\\
356	0.011141348696349\\
357	0.011143092021995\\
358	0.011144869803166\\
359	0.0111466827309701\\
360	0.0111485315114099\\
361	0.01115041686609\\
362	0.011152339533704\\
363	0.0111543002689499\\
364	0.0111562998429425\\
365	0.0111583390436369\\
366	0.011160418676265\\
367	0.0111625395637837\\
368	0.0111647025473359\\
369	0.0111669084867218\\
370	0.0111691582608829\\
371	0.0111714527683958\\
372	0.0111737929279753\\
373	0.0111761796789856\\
374	0.0111786139819576\\
375	0.011181096819109\\
376	0.0111836291948633\\
377	0.0111862121363583\\
378	0.0111888466939273\\
379	0.0111915339415102\\
380	0.0111942749768615\\
381	0.0111970709209596\\
382	0.0111999229133673\\
383	0.0112028321198713\\
384	0.0112057997332585\\
385	0.0112088269740515\\
386	0.0112119150912587\\
387	0.0112150653631339\\
388	0.0112182790979441\\
389	0.0112215576347483\\
390	0.0112249023441701\\
391	0.0112283146291635\\
392	0.0112317959257731\\
393	0.0112353477038784\\
394	0.0112389714679189\\
395	0.0112426687575904\\
396	0.0112464411485052\\
397	0.0112502902528098\\
398	0.0112542177197486\\
399	0.0112582252361496\\
400	0.0112623145268282\\
401	0.0112664873548811\\
402	0.0112707455218946\\
403	0.0112750908680181\\
404	0.0112795252719004\\
405	0.0112840506504983\\
406	0.0112886689586672\\
407	0.0112933821885801\\
408	0.0112981923689454\\
409	0.0113031015640329\\
410	0.0113081118724015\\
411	0.0113132254258913\\
412	0.0113184443900822\\
413	0.0113237709645955\\
414	0.0113292073180522\\
415	0.0113347556565547\\
416	0.0113404182239359\\
417	0.0113461973018183\\
418	0.0113520952096186\\
419	0.0113581143044745\\
420	0.0113642569810519\\
421	0.0113705256713308\\
422	0.01137692284451\\
423	0.0113834510061307\\
424	0.0113901126983908\\
425	0.0113969105001876\\
426	0.0114038470264269\\
427	0.0114109249243833\\
428	0.011418146873857\\
429	0.0114255155861988\\
430	0.0114330338026575\\
431	0.0114407042925282\\
432	0.0114485298510949\\
433	0.0114565132972639\\
434	0.011464657470103\\
435	0.0114729652212656\\
436	0.0114814394144858\\
437	0.0114900829389648\\
438	0.0114988987129087\\
439	0.0115078896310828\\
440	0.0115170585920306\\
441	0.0115264084952188\\
442	0.0115359422378573\\
443	0.011545662711382\\
444	0.011555572797905\\
445	0.0115656753635626\\
446	0.0115759731968166\\
447	0.0115864690116464\\
448	0.0115971655623689\\
449	0.0116080658308683\\
450	0.0116191724296148\\
451	0.0116304878545012\\
452	0.0116420145013206\\
453	0.0116537545652828\\
454	0.0116657100377861\\
455	0.0116778827411926\\
456	0.0116902744773645\\
457	0.0117028850114115\\
458	0.0117157147228857\\
459	0.0117287663310586\\
460	0.0117420415045572\\
461	0.0117555404334236\\
462	0.0117692630482441\\
463	0.0117832090462185\\
464	0.0117973777845297\\
465	0.0118117701810837\\
466	0.0118263891698841\\
467	0.0118412373969455\\
468	0.0118563182286269\\
469	0.0118716335044038\\
470	0.0118871837351555\\
471	0.0119029684067957\\
472	0.0119189858595252\\
473	0.0119352331897081\\
474	0.0119517058678664\\
475	0.0119683977776944\\
476	0.0119855238389395\\
477	0.0120037755866312\\
478	0.0120218955928688\\
479	0.0120398671701517\\
480	0.0120576727877737\\
481	0.0120752933899803\\
482	0.0120927099554993\\
483	0.01210990385284\\
484	0.0121268570530123\\
485	0.0121435496195092\\
486	0.012159961432781\\
487	0.012176072221324\\
488	0.0121915736449661\\
489	0.0122063006062571\\
490	0.0122208204769087\\
491	0.0122351247451908\\
492	0.0122492059064224\\
493	0.0122630582556204\\
494	0.0122766769927613\\
495	0.0122900581143079\\
496	0.012303199010541\\
497	0.0123160986087916\\
498	0.0123287575041728\\
499	0.0123411780259058\\
500	0.0123528531308694\\
501	0.0123643753846295\\
502	0.0123758678113748\\
503	0.0123873337574185\\
504	0.0123987772054143\\
505	0.0124102027756199\\
506	0.0124216157165108\\
507	0.0124330218850032\\
508	0.0124444277140491\\
509	0.0124558401678491\\
510	0.0124672666755119\\
511	0.0124787282871157\\
512	0.0124902417969524\\
513	0.0125018076272554\\
514	0.0125134260723856\\
515	0.0125250972578757\\
516	0.0125368210950568\\
517	0.0125485972312262\\
518	0.0125604249953626\\
519	0.0125723033395197\\
520	0.0125842307761826\\
521	0.012596205311608\\
522	0.0126082240407639\\
523	0.0126202826040755\\
524	0.012632376194594\\
525	0.0126444995217257\\
526	0.012656646773156\\
527	0.0126688115749867\\
528	0.0126809869501136\\
529	0.0126931652748699\\
530	0.0127053382339588\\
531	0.0127174967737022\\
532	0.0127296310536341\\
533	0.0127417304003928\\
534	0.0127537833132383\\
535	0.0127657774047621\\
536	0.0127776993371164\\
537	0.0127895347533509\\
538	0.0128012682033883\\
539	0.0128128830641245\\
540	0.0128243614530708\\
541	0.0128356841348946\\
542	0.0128468304201348\\
543	0.0128577780552621\\
544	0.0128685031034433\\
545	0.0128789798035197\\
546	0.0128891806169425\\
547	0.0128990787083977\\
548	0.0129090375888559\\
549	0.0129192334113919\\
550	0.0129296727272301\\
551	0.0129403623815975\\
552	0.0129513095471166\\
553	0.0129625218062059\\
554	0.0129740104864093\\
555	0.0129857884046474\\
556	0.0129978702446768\\
557	0.0130116747206433\\
558	0.0130256634777257\\
559	0.0130397906340676\\
560	0.0130541416912179\\
561	0.0130680591566844\\
562	0.0130818328647282\\
563	0.0130957308409921\\
564	0.0131102446031075\\
565	0.0131248163214706\\
566	0.0131387593933604\\
567	0.0131525032059518\\
568	0.0131658917407363\\
569	0.0131810263323357\\
570	0.0131972575083075\\
571	0.013213044026579\\
572	0.0132287581796109\\
573	0.0132429665120816\\
574	0.0132567840953481\\
575	0.013270365677866\\
576	0.0132838830721923\\
577	0.0132962593908739\\
578	0.0133079395111771\\
579	0.0133193907373955\\
580	0.0133304822426818\\
581	0.0133413468885363\\
582	0.0133520801152746\\
583	0.0133626748577664\\
584	0.0133731047252647\\
585	0.0133833476643832\\
586	0.0133933802868045\\
587	0.013403169344532\\
588	0.0134126782949752\\
589	0.013421869140597\\
590	0.0134307057822237\\
591	0.0134391597728698\\
592	0.0134472338645631\\
593	0.0134551345125145\\
594	0.0134637112677116\\
595	0.0134740407759367\\
596	0.0134889777516421\\
597	0.0135160557778789\\
598	0.0135751148335434\\
599	0\\
600	0\\
};
\addplot [color=red!75!mycolor17,solid,forget plot]
  table[row sep=crcr]{%
1	0.0110623654486931\\
2	0.0110623678189323\\
3	0.0110623702333114\\
4	0.011062372692627\\
5	0.0110623751976894\\
6	0.0110623777493233\\
7	0.0110623803483681\\
8	0.0110623829956779\\
9	0.011062385692122\\
10	0.011062388438585\\
11	0.0110623912359672\\
12	0.0110623940851848\\
13	0.0110623969871701\\
14	0.0110623999428721\\
15	0.0110624029532564\\
16	0.011062406019306\\
17	0.0110624091420211\\
18	0.0110624123224196\\
19	0.0110624155615376\\
20	0.0110624188604296\\
21	0.0110624222201687\\
22	0.0110624256418472\\
23	0.0110624291265768\\
24	0.0110624326754889\\
25	0.0110624362897351\\
26	0.0110624399704876\\
27	0.0110624437189393\\
28	0.0110624475363046\\
29	0.0110624514238194\\
30	0.0110624553827418\\
31	0.0110624594143523\\
32	0.0110624635199545\\
33	0.0110624677008751\\
34	0.0110624719584648\\
35	0.0110624762940983\\
36	0.0110624807091752\\
37	0.0110624852051201\\
38	0.0110624897833833\\
39	0.0110624944454412\\
40	0.0110624991927968\\
41	0.0110625040269803\\
42	0.0110625089495493\\
43	0.0110625139620897\\
44	0.0110625190662162\\
45	0.0110625242635726\\
46	0.0110625295558324\\
47	0.0110625349446998\\
48	0.0110625404319096\\
49	0.0110625460192285\\
50	0.0110625517084552\\
51	0.0110625575014211\\
52	0.0110625633999912\\
53	0.0110625694060646\\
54	0.011062575521575\\
55	0.0110625817484916\\
56	0.0110625880888198\\
57	0.0110625945446018\\
58	0.0110626011179172\\
59	0.0110626078108841\\
60	0.0110626146256596\\
61	0.0110626215644406\\
62	0.0110626286294645\\
63	0.0110626358230103\\
64	0.0110626431473992\\
65	0.0110626506049955\\
66	0.0110626581982072\\
67	0.0110626659294875\\
68	0.0110626738013351\\
69	0.0110626818162954\\
70	0.0110626899769612\\
71	0.0110626982859741\\
72	0.011062706746025\\
73	0.0110627153598554\\
74	0.0110627241302584\\
75	0.0110627330600794\\
76	0.0110627421522177\\
77	0.0110627514096273\\
78	0.0110627608353178\\
79	0.0110627704323561\\
80	0.0110627802038669\\
81	0.0110627901530346\\
82	0.0110628002831037\\
83	0.0110628105973809\\
84	0.0110628210992357\\
85	0.0110628317921018\\
86	0.0110628426794789\\
87	0.0110628537649334\\
88	0.0110628650521003\\
89	0.0110628765446843\\
90	0.0110628882464612\\
91	0.0110629001612798\\
92	0.0110629122930628\\
93	0.0110629246458088\\
94	0.0110629372235936\\
95	0.0110629500305721\\
96	0.0110629630709795\\
97	0.0110629763491335\\
98	0.0110629898694354\\
99	0.0110630036363723\\
100	0.0110630176545188\\
101	0.0110630319285388\\
102	0.0110630464631872\\
103	0.0110630612633119\\
104	0.0110630763338558\\
105	0.0110630916798588\\
106	0.0110631073064596\\
107	0.0110631232188981\\
108	0.0110631394225172\\
109	0.0110631559227651\\
110	0.0110631727251974\\
111	0.0110631898354795\\
112	0.0110632072593887\\
113	0.0110632250028167\\
114	0.011063243071772\\
115	0.011063261472382\\
116	0.0110632802108961\\
117	0.0110632992936875\\
118	0.0110633187272563\\
119	0.0110633385182322\\
120	0.0110633586733768\\
121	0.0110633791995865\\
122	0.0110634001038955\\
123	0.0110634213934785\\
124	0.0110634430756536\\
125	0.0110634651578855\\
126	0.0110634876477882\\
127	0.0110635105531284\\
128	0.0110635338818285\\
129	0.0110635576419699\\
130	0.0110635818417964\\
131	0.0110636064897174\\
132	0.0110636315943114\\
133	0.0110636571643294\\
134	0.0110636832086988\\
135	0.0110637097365266\\
136	0.0110637367571034\\
137	0.0110637642799072\\
138	0.0110637923146072\\
139	0.0110638208710674\\
140	0.0110638499593515\\
141	0.011063879589726\\
142	0.011063909772665\\
143	0.0110639405188544\\
144	0.0110639718391958\\
145	0.0110640037448116\\
146	0.0110640362470491\\
147	0.011064069357485\\
148	0.0110641030879307\\
149	0.0110641374504361\\
150	0.0110641724572956\\
151	0.0110642081210522\\
152	0.0110642444545031\\
153	0.0110642814707043\\
154	0.0110643191829768\\
155	0.0110643576049109\\
156	0.0110643967503724\\
157	0.0110644366335081\\
158	0.0110644772687511\\
159	0.011064518670827\\
160	0.0110645608547597\\
161	0.0110646038358773\\
162	0.0110646476298183\\
163	0.011064692252538\\
164	0.0110647377203148\\
165	0.0110647840497565\\
166	0.0110648312578072\\
167	0.0110648793617541\\
168	0.011064928379234\\
169	0.0110649783282408\\
170	0.0110650292271322\\
171	0.0110650810946374\\
172	0.0110651339498643\\
173	0.011065187812307\\
174	0.0110652427018539\\
175	0.0110652986387951\\
176	0.0110653556438308\\
177	0.0110654137380793\\
178	0.0110654729430855\\
179	0.011065533280829\\
180	0.0110655947737334\\
181	0.0110656574446747\\
182	0.0110657213169902\\
183	0.0110657864144884\\
184	0.0110658527614575\\
185	0.0110659203826754\\
186	0.0110659893034196\\
187	0.0110660595494767\\
188	0.011066131147153\\
189	0.0110662041232844\\
190	0.0110662785052474\\
191	0.0110663543209695\\
192	0.0110664315989401\\
193	0.0110665103682221\\
194	0.0110665906584631\\
195	0.0110666724999068\\
196	0.0110667559234054\\
197	0.0110668409604311\\
198	0.011066927643089\\
199	0.0110670160041294\\
200	0.0110671060769607\\
201	0.0110671978956626\\
202	0.0110672914949997\\
203	0.0110673869104345\\
204	0.0110674841781421\\
205	0.011067583335024\\
206	0.0110676844187228\\
207	0.0110677874676369\\
208	0.0110678925209357\\
209	0.0110679996185749\\
210	0.0110681088013126\\
211	0.0110682201107248\\
212	0.0110683335892225\\
213	0.0110684492800677\\
214	0.0110685672273913\\
215	0.0110686874762097\\
216	0.0110688100724432\\
217	0.0110689350629341\\
218	0.0110690624954649\\
219	0.0110691924187775\\
220	0.0110693248825925\\
221	0.0110694599376288\\
222	0.0110695976356238\\
223	0.011069738029354\\
224	0.0110698811726557\\
225	0.0110700271204467\\
226	0.011070175928748\\
227	0.0110703276547058\\
228	0.0110704823566147\\
229	0.0110706400939406\\
230	0.0110708009273445\\
231	0.0110709649187065\\
232	0.0110711321311506\\
233	0.0110713026290697\\
234	0.0110714764781517\\
235	0.0110716537454048\\
236	0.0110718344991853\\
237	0.0110720188092241\\
238	0.0110722067466547\\
239	0.011072398384042\\
240	0.0110725937954105\\
241	0.0110727930562748\\
242	0.0110729962436688\\
243	0.0110732034361774\\
244	0.0110734147139672\\
245	0.011073630158819\\
246	0.0110738498541602\\
247	0.0110740738850985\\
248	0.0110743023384559\\
249	0.0110745353028031\\
250	0.0110747728684958\\
251	0.0110750151277102\\
252	0.0110752621744804\\
253	0.011075514104736\\
254	0.011075771016341\\
255	0.0110760330091328\\
256	0.011076300184963\\
257	0.011076572647738\\
258	0.0110768505034616\\
259	0.0110771338602773\\
260	0.0110774228285132\\
261	0.0110777175207262\\
262	0.0110780180517482\\
263	0.0110783245387335\\
264	0.0110786371012066\\
265	0.011078955861112\\
266	0.0110792809428647\\
267	0.0110796124734022\\
268	0.0110799505822378\\
269	0.0110802954015155\\
270	0.0110806470660664\\
271	0.0110810057134666\\
272	0.0110813714840972\\
273	0.011081744521206\\
274	0.0110821249709714\\
275	0.0110825129825688\\
276	0.0110829087082404\\
277	0.0110833123033692\\
278	0.0110837239265696\\
279	0.0110841437398821\\
280	0.0110845719088712\\
281	0.0110850086025246\\
282	0.0110854539933274\\
283	0.0110859082573382\\
284	0.0110863715742675\\
285	0.0110868441275568\\
286	0.0110873261044608\\
287	0.0110878176961308\\
288	0.0110883190977001\\
289	0.0110888305083719\\
290	0.0110893521315085\\
291	0.0110898841747232\\
292	0.0110904268499742\\
293	0.0110909803736601\\
294	0.0110915449667182\\
295	0.0110921208547247\\
296	0.0110927082679972\\
297	0.0110933074416991\\
298	0.0110939186159467\\
299	0.0110945420359176\\
300	0.0110951779519625\\
301	0.0110958266197177\\
302	0.0110964883002206\\
303	0.0110971632600266\\
304	0.0110978517713283\\
305	0.0110985541120752\\
306	0.0110992705660963\\
307	0.0111000014232226\\
308	0.0111007469794111\\
309	0.0111015075368689\\
310	0.0111022834041772\\
311	0.0111030748964142\\
312	0.0111038823352762\\
313	0.0111047060491931\\
314	0.0111055463734355\\
315	0.0111064036501983\\
316	0.0111072782286153\\
317	0.0111081704644863\\
318	0.0111090807194461\\
319	0.0111100093626643\\
320	0.011110956771006\\
321	0.0111119233291967\\
322	0.0111129094299904\\
323	0.0111139154743415\\
324	0.0111149418715809\\
325	0.011115989039595\\
326	0.0111170574050098\\
327	0.011118147403378\\
328	0.011119259479371\\
329	0.0111203940869744\\
330	0.0111215516896885\\
331	0.0111227327607323\\
332	0.0111239377832529\\
333	0.0111251672505386\\
334	0.0111264216662376\\
335	0.0111277015445801\\
336	0.0111290074106068\\
337	0.0111303398004013\\
338	0.0111316992613282\\
339	0.011133086352276\\
340	0.0111345016439057\\
341	0.0111359457189044\\
342	0.0111374191722452\\
343	0.0111389226114518\\
344	0.0111404566568694\\
345	0.0111420219419419\\
346	0.0111436191134946\\
347	0.0111452488320226\\
348	0.0111469117719861\\
349	0.0111486086221115\\
350	0.0111503400856987\\
351	0.0111521068809346\\
352	0.0111539097412131\\
353	0.0111557494154613\\
354	0.0111576266684717\\
355	0.0111595422812408\\
356	0.0111614970513135\\
357	0.0111634917931331\\
358	0.0111655273383964\\
359	0.0111676045364133\\
360	0.0111697242544719\\
361	0.0111718873782144\\
362	0.0111740948119858\\
363	0.0111763474792146\\
364	0.0111786463227994\\
365	0.0111809923055023\\
366	0.0111833864103454\\
367	0.011185829641011\\
368	0.0111883230222484\\
369	0.0111908676002845\\
370	0.0111934644432376\\
371	0.0111961146415384\\
372	0.0111988193083541\\
373	0.011201579580017\\
374	0.011204396616457\\
375	0.0112072716016378\\
376	0.011210205743997\\
377	0.0112132002768906\\
378	0.0112162564590431\\
379	0.0112193755750066\\
380	0.0112225589356217\\
381	0.0112258078784459\\
382	0.0112291237683384\\
383	0.0112325079979172\\
384	0.0112359619880048\\
385	0.0112394871880598\\
386	0.0112430850766003\\
387	0.0112467571616288\\
388	0.0112505049810212\\
389	0.0112543301029138\\
390	0.0112582341260679\\
391	0.011262218680213\\
392	0.0112662854263687\\
393	0.0112704360571436\\
394	0.0112746722970092\\
395	0.0112789959025473\\
396	0.0112834086626643\\
397	0.0112879123987752\\
398	0.0112925089650012\\
399	0.0112972002483958\\
400	0.0113019881690343\\
401	0.0113068746801122\\
402	0.0113118617677982\\
403	0.0113169514513408\\
404	0.0113221457830343\\
405	0.0113274468481125\\
406	0.0113328567649902\\
407	0.0113383776848614\\
408	0.0113440117914816\\
409	0.011349761300827\\
410	0.0113556284606027\\
411	0.0113616155496045\\
412	0.0113677248776123\\
413	0.0113739587842945\\
414	0.0113803196399383\\
415	0.0113868098435883\\
416	0.011393431820022\\
417	0.0114001880198675\\
418	0.0114070809179138\\
419	0.0114141130112685\\
420	0.0114212868172353\\
421	0.0114286048703343\\
422	0.0114360697193411\\
423	0.0114436839278263\\
424	0.0114514500641058\\
425	0.0114593707074665\\
426	0.0114674484547167\\
427	0.0114756859260157\\
428	0.0114840857261699\\
429	0.0114926504566801\\
430	0.0115013827176414\\
431	0.0115102851035017\\
432	0.0115193601981362\\
433	0.0115286105693645\\
434	0.0115380387634268\\
435	0.0115476472960385\\
436	0.0115574385774572\\
437	0.0115674148914464\\
438	0.011577578564922\\
439	0.011587932102672\\
440	0.0115984774852266\\
441	0.0116092165040529\\
442	0.011620150751284\\
443	0.0116312816134764\\
444	0.0116426102709697\\
445	0.0116541377182533\\
446	0.0116658648843438\\
447	0.011677791931043\\
448	0.011689918040407\\
449	0.0117022424953743\\
450	0.0117147686385806\\
451	0.011727496325985\\
452	0.0117404256078264\\
453	0.0117535601034405\\
454	0.0117669030239899\\
455	0.0117804571029016\\
456	0.0117942245922939\\
457	0.0118082062188995\\
458	0.0118224022267346\\
459	0.0118368129474516\\
460	0.0118514372985312\\
461	0.0118662723938917\\
462	0.011881314050468\\
463	0.0118965566771358\\
464	0.011912434347859\\
465	0.0119290361045897\\
466	0.0119455173330837\\
467	0.0119618628127314\\
468	0.011978056691041\\
469	0.0119940827081719\\
470	0.0120099236613982\\
471	0.0120255614815943\\
472	0.012040978130715\\
473	0.0120561558017057\\
474	0.0120710778634681\\
475	0.0120857259650814\\
476	0.0120999502055335\\
477	0.0121133080250348\\
478	0.0121264779777807\\
479	0.0121394528642926\\
480	0.0121522261655328\\
481	0.0121647918166338\\
482	0.0121771449867835\\
483	0.0121892823291729\\
484	0.0122012022135653\\
485	0.0122129036872904\\
486	0.0122243873241518\\
487	0.0122356552980776\\
488	0.0122465048325158\\
489	0.0122568647365316\\
490	0.012267203758131\\
491	0.0122775255543116\\
492	0.0122878343997599\\
493	0.0122981351875699\\
494	0.0123084334337906\\
495	0.0123187352719486\\
496	0.0123290474325015\\
497	0.0123393772108173\\
498	0.0123497324226331\\
499	0.0123601213477198\\
500	0.0123705727010044\\
501	0.0123810930067297\\
502	0.0123916841290159\\
503	0.0124023478858576\\
504	0.0124130860160422\\
505	0.0124239001425876\\
506	0.0124347917327885\\
507	0.0124457620549489\\
508	0.0124568121319474\\
509	0.0124679426917336\\
510	0.0124791541151341\\
511	0.0124904458323131\\
512	0.0125018164140504\\
513	0.0125132641802315\\
514	0.0125247871784754\\
515	0.0125363831621353\\
516	0.0125480495677779\\
517	0.0125597834922592\\
518	0.0125715816695281\\
519	0.0125834404472952\\
520	0.012595355763706\\
521	0.0126073231241715\\
522	0.0126193375929205\\
523	0.0126313938069644\\
524	0.0126434859479025\\
525	0.0126556077119702\\
526	0.0126677522781776\\
527	0.012679912274367\\
528	0.0126920797409886\\
529	0.0127042460923695\\
530	0.0127164020752147\\
531	0.0127285377240438\\
532	0.0127406423132262\\
533	0.0127527043050506\\
534	0.0127647112909694\\
535	0.012776649928001\\
536	0.0127885058698199\\
537	0.0128002636920156\\
538	0.012811906810946\\
539	0.0128234173955354\\
540	0.0128347762712923\\
541	0.0128459628157843\\
542	0.0128569548448548\\
543	0.0128677284899802\\
544	0.0128782580554183\\
545	0.0128885161898169\\
546	0.0128984663566234\\
547	0.0129084925588052\\
548	0.0129187690136387\\
549	0.012929303906619\\
550	0.0129401089208385\\
551	0.0129511976775003\\
552	0.0129626122013664\\
553	0.0129757679149621\\
554	0.0129890552764016\\
555	0.0130024848838998\\
556	0.0130161738624674\\
557	0.0130291429473715\\
558	0.0130422222229939\\
559	0.0130554270019023\\
560	0.0130686757755564\\
561	0.0130821204103835\\
562	0.0130956145942559\\
563	0.0131091155854781\\
564	0.013122239068426\\
565	0.0131360495658837\\
566	0.0131514679660248\\
567	0.0131669530769827\\
568	0.0131826927920756\\
569	0.0131974144130573\\
570	0.0132112428947385\\
571	0.0132249692959164\\
572	0.0132383599524506\\
573	0.0132505511850947\\
574	0.013262803206264\\
575	0.0132748459865571\\
576	0.0132863897399275\\
577	0.0132977162412856\\
578	0.0133087815388813\\
579	0.0133197523049772\\
580	0.0133306397973883\\
581	0.0133414408591185\\
582	0.0133521371638476\\
583	0.0133627058867585\\
584	0.0133731226054228\\
585	0.0133833603746144\\
586	0.0133933887036049\\
587	0.0134031741941115\\
588	0.013412680773945\\
589	0.0134218705377325\\
590	0.0134307062886727\\
591	0.0134391597728698\\
592	0.0134472338645631\\
593	0.0134551345125145\\
594	0.0134637112677116\\
595	0.0134740407759367\\
596	0.0134889777516421\\
597	0.0135160557778789\\
598	0.0135751148335434\\
599	0\\
600	0\\
};
\addplot [color=red!80!mycolor19,solid,forget plot]
  table[row sep=crcr]{%
1	0.0110648549379197\\
2	0.0110648578589327\\
3	0.0110648608355688\\
4	0.0110648638688409\\
5	0.0110648669597798\\
6	0.0110648701094339\\
7	0.0110648733188699\\
8	0.0110648765891732\\
9	0.0110648799214475\\
10	0.0110648833168161\\
11	0.0110648867764213\\
12	0.0110648903014255\\
13	0.0110648938930109\\
14	0.0110648975523804\\
15	0.0110649012807573\\
16	0.0110649050793863\\
17	0.0110649089495335\\
18	0.0110649128924868\\
19	0.0110649169095564\\
20	0.0110649210020748\\
21	0.0110649251713979\\
22	0.0110649294189045\\
23	0.0110649337459975\\
24	0.0110649381541037\\
25	0.0110649426446746\\
26	0.0110649472191868\\
27	0.0110649518791419\\
28	0.0110649566260679\\
29	0.0110649614615186\\
30	0.0110649663870748\\
31	0.0110649714043446\\
32	0.0110649765149634\\
33	0.011064981720595\\
34	0.011064987022932\\
35	0.0110649924236957\\
36	0.0110649979246375\\
37	0.0110650035275386\\
38	0.011065009234211\\
39	0.0110650150464981\\
40	0.0110650209662748\\
41	0.0110650269954484\\
42	0.0110650331359593\\
43	0.011065039389781\\
44	0.0110650457589214\\
45	0.0110650522454229\\
46	0.0110650588513633\\
47	0.0110650655788562\\
48	0.0110650724300517\\
49	0.0110650794071374\\
50	0.0110650865123386\\
51	0.011065093747919\\
52	0.0110651011161818\\
53	0.0110651086194702\\
54	0.0110651162601678\\
55	0.0110651240407001\\
56	0.0110651319635342\\
57	0.0110651400311808\\
58	0.0110651482461938\\
59	0.0110651566111721\\
60	0.0110651651287597\\
61	0.0110651738016469\\
62	0.0110651826325711\\
63	0.0110651916243177\\
64	0.0110652007797209\\
65	0.0110652101016648\\
66	0.0110652195930839\\
67	0.0110652292569647\\
68	0.0110652390963462\\
69	0.011065249114321\\
70	0.0110652593140365\\
71	0.0110652696986957\\
72	0.0110652802715583\\
73	0.011065291035942\\
74	0.0110653019952236\\
75	0.0110653131528397\\
76	0.0110653245122883\\
77	0.0110653360771301\\
78	0.0110653478509891\\
79	0.0110653598375547\\
80	0.0110653720405821\\
81	0.0110653844638942\\
82	0.0110653971113829\\
83	0.01106540998701\\
84	0.0110654230948091\\
85	0.0110654364388868\\
86	0.0110654500234244\\
87	0.0110654638526787\\
88	0.0110654779309845\\
89	0.0110654922627555\\
90	0.0110655068524859\\
91	0.0110655217047525\\
92	0.0110655368242159\\
93	0.0110655522156225\\
94	0.011065567883806\\
95	0.0110655838336894\\
96	0.0110656000702868\\
97	0.0110656165987052\\
98	0.0110656334241465\\
99	0.0110656505519092\\
100	0.011065667987391\\
101	0.0110656857360901\\
102	0.0110657038036078\\
103	0.0110657221956507\\
104	0.0110657409180324\\
105	0.0110657599766763\\
106	0.0110657793776175\\
107	0.0110657991270055\\
108	0.0110658192311062\\
109	0.0110658396963048\\
110	0.0110658605291079\\
111	0.0110658817361465\\
112	0.0110659033241781\\
113	0.0110659253000901\\
114	0.0110659476709017\\
115	0.0110659704437677\\
116	0.0110659936259803\\
117	0.0110660172249731\\
118	0.0110660412483234\\
119	0.0110660657037554\\
120	0.0110660905991436\\
121	0.0110661159425158\\
122	0.0110661417420566\\
123	0.0110661680061106\\
124	0.0110661947431856\\
125	0.0110662219619568\\
126	0.0110662496712697\\
127	0.0110662778801442\\
128	0.0110663065977782\\
129	0.0110663358335515\\
130	0.0110663655970296\\
131	0.0110663958979679\\
132	0.0110664267463155\\
133	0.0110664581522199\\
134	0.0110664901260306\\
135	0.0110665226783041\\
136	0.0110665558198078\\
137	0.0110665895615252\\
138	0.0110666239146598\\
139	0.0110666588906404\\
140	0.0110666945011259\\
141	0.01106673075801\\
142	0.0110667676734265\\
143	0.0110668052597542\\
144	0.0110668435296224\\
145	0.0110668824959164\\
146	0.0110669221717826\\
147	0.0110669625706342\\
148	0.0110670037061574\\
149	0.0110670455923165\\
150	0.0110670882433603\\
151	0.0110671316738282\\
152	0.0110671758985563\\
153	0.0110672209326834\\
154	0.0110672667916581\\
155	0.0110673134912449\\
156	0.0110673610475308\\
157	0.0110674094769328\\
158	0.011067458796204\\
159	0.0110675090224413\\
160	0.0110675601730927\\
161	0.0110676122659641\\
162	0.0110676653192274\\
163	0.0110677193514281\\
164	0.0110677743814927\\
165	0.0110678304287372\\
166	0.011067887512875\\
167	0.011067945654025\\
168	0.0110680048727205\\
169	0.0110680651899173\\
170	0.0110681266270027\\
171	0.0110681892058045\\
172	0.0110682529486\\
173	0.0110683178781252\\
174	0.0110683840175844\\
175	0.0110684513906597\\
176	0.0110685200215207\\
177	0.0110685899348347\\
178	0.0110686611557769\\
179	0.0110687337100405\\
180	0.0110688076238472\\
181	0.0110688829239585\\
182	0.0110689596376861\\
183	0.011069037792903\\
184	0.0110691174180554\\
185	0.0110691985421738\\
186	0.011069281194885\\
187	0.0110693654064239\\
188	0.0110694512076462\\
189	0.0110695386300402\\
190	0.01106962770574\\
191	0.0110697184675384\\
192	0.0110698109488997\\
193	0.0110699051839737\\
194	0.011070001207609\\
195	0.011070099055367\\
196	0.0110701987635363\\
197	0.0110703003691469\\
198	0.0110704039099854\\
199	0.0110705094246097\\
200	0.0110706169523645\\
201	0.0110707265333969\\
202	0.0110708382086727\\
203	0.011070952019992\\
204	0.0110710680100065\\
205	0.0110711862222363\\
206	0.0110713067010868\\
207	0.0110714294918667\\
208	0.011071554640806\\
209	0.0110716821950741\\
210	0.0110718122027989\\
211	0.0110719447130856\\
212	0.0110720797760365\\
213	0.0110722174427707\\
214	0.0110723577654445\\
215	0.0110725007972725\\
216	0.0110726465925483\\
217	0.0110727952066664\\
218	0.0110729466961447\\
219	0.0110731011186463\\
220	0.0110732585330034\\
221	0.0110734189992403\\
222	0.0110735825785978\\
223	0.0110737493335579\\
224	0.0110739193278685\\
225	0.0110740926265696\\
226	0.0110742692960191\\
227	0.0110744494039203\\
228	0.0110746330193483\\
229	0.011074820212779\\
230	0.0110750110561171\\
231	0.0110752056227257\\
232	0.0110754039874557\\
233	0.011075606226677\\
234	0.0110758124183093\\
235	0.0110760226418537\\
236	0.011076236978426\\
237	0.0110764555107893\\
238	0.0110766783233885\\
239	0.0110769055023845\\
240	0.0110771371356903\\
241	0.0110773733130067\\
242	0.0110776141258597\\
243	0.0110778596676382\\
244	0.0110781100336324\\
245	0.0110783653210738\\
246	0.011078625629175\\
247	0.0110788910591711\\
248	0.0110791617143617\\
249	0.011079437700154\\
250	0.0110797191241063\\
251	0.011080006095973\\
252	0.0110802987277507\\
253	0.0110805971337242\\
254	0.0110809014305147\\
255	0.0110812117371288\\
256	0.0110815281750074\\
257	0.0110818508680776\\
258	0.0110821799428039\\
259	0.0110825155282415\\
260	0.0110828577560907\\
261	0.0110832067607521\\
262	0.0110835626793833\\
263	0.0110839256519565\\
264	0.0110842958213179\\
265	0.0110846733332478\\
266	0.0110850583365225\\
267	0.0110854509829773\\
268	0.0110858514275708\\
269	0.0110862598284509\\
270	0.011086676347022\\
271	0.0110871011480137\\
272	0.0110875343995513\\
273	0.0110879762732271\\
274	0.0110884269441741\\
275	0.0110888865911408\\
276	0.0110893553965671\\
277	0.0110898335466623\\
278	0.0110903212314849\\
279	0.0110908186450223\\
280	0.0110913259852709\\
281	0.0110918434543203\\
282	0.0110923712584401\\
283	0.0110929096081681\\
284	0.0110934587184004\\
285	0.0110940188084837\\
286	0.0110945901023092\\
287	0.0110951728284089\\
288	0.0110957672200534\\
289	0.0110963735153521\\
290	0.0110969919573555\\
291	0.011097622794159\\
292	0.0110982662790099\\
293	0.0110989226704149\\
294	0.0110995922322519\\
295	0.0111002752338817\\
296	0.0111009719502638\\
297	0.0111016826620732\\
298	0.0111024076558199\\
299	0.011103147223971\\
300	0.0111039016650744\\
301	0.0111046712838857\\
302	0.0111054563914966\\
303	0.0111062573054662\\
304	0.0111070743499554\\
305	0.0111079078558626\\
306	0.0111087581609628\\
307	0.0111096256100496\\
308	0.0111105105550793\\
309	0.0111114133553183\\
310	0.0111123343774934\\
311	0.0111132739959458\\
312	0.0111142325927873\\
313	0.0111152105580616\\
314	0.0111162082899093\\
315	0.0111172261947381\\
316	0.0111182646873969\\
317	0.0111193241913505\\
318	0.0111204051388896\\
319	0.0111215079713165\\
320	0.0111226331391339\\
321	0.0111237811022387\\
322	0.0111249523301192\\
323	0.0111261473020572\\
324	0.0111273665073336\\
325	0.0111286104454388\\
326	0.0111298796262875\\
327	0.0111311745704377\\
328	0.0111324958093141\\
329	0.0111338438854359\\
330	0.0111352193526494\\
331	0.0111366227763646\\
332	0.0111380547337965\\
333	0.0111395158142105\\
334	0.0111410066191729\\
335	0.0111425277628045\\
336	0.0111440798720399\\
337	0.0111456635868894\\
338	0.011147279560707\\
339	0.0111489284604614\\
340	0.0111506109670131\\
341	0.0111523277753931\\
342	0.0111540795950871\\
343	0.0111558671503246\\
344	0.0111576911803718\\
345	0.0111595524398298\\
346	0.0111614516989368\\
347	0.0111633897438769\\
348	0.0111653673770919\\
349	0.0111673854175996\\
350	0.0111694447013167\\
351	0.0111715460813874\\
352	0.0111736904285166\\
353	0.0111758786313101\\
354	0.0111781115966198\\
355	0.0111803902498956\\
356	0.0111827155355408\\
357	0.0111850884172744\\
358	0.0111875098784973\\
359	0.0111899809226641\\
360	0.0111925025736589\\
361	0.0111950758761741\\
362	0.0111977018960913\\
363	0.0112003817208641\\
364	0.0112031164598979\\
365	0.0112059072449313\\
366	0.0112087552304372\\
367	0.011211661593999\\
368	0.01121462753668\\
369	0.0112176542833938\\
370	0.0112207430832663\\
371	0.011223895209981\\
372	0.011227111962129\\
373	0.0112303946635582\\
374	0.0112337446637075\\
375	0.0112371633379344\\
376	0.0112406520878365\\
377	0.0112442123415659\\
378	0.0112478455541396\\
379	0.0112515532077437\\
380	0.0112553368120285\\
381	0.0112591979044047\\
382	0.0112631380503231\\
383	0.0112671588435501\\
384	0.011271261906499\\
385	0.0112754488905291\\
386	0.0112797214761567\\
387	0.0112840813732525\\
388	0.011288530321405\\
389	0.0112930700899896\\
390	0.0112977024783915\\
391	0.0113024293161043\\
392	0.0113072524627424\\
393	0.0113121738079602\\
394	0.0113171952712657\\
395	0.0113223188017189\\
396	0.0113275463774905\\
397	0.0113328800052219\\
398	0.0113383217191028\\
399	0.0113438735801538\\
400	0.0113495376762587\\
401	0.0113553161207687\\
402	0.0113612110518829\\
403	0.0113672246288339\\
404	0.0113733590315001\\
405	0.011379616458819\\
406	0.0113859991259946\\
407	0.0113925092672923\\
408	0.0113991491318927\\
409	0.0114059209837009\\
410	0.0114128271010054\\
411	0.0114198697752345\\
412	0.0114270513060838\\
413	0.0114343740037439\\
414	0.0114418401969867\\
415	0.0114494522318439\\
416	0.0114572124648189\\
417	0.0114651232362765\\
418	0.0114731868852768\\
419	0.0114814057444074\\
420	0.011489782133807\\
421	0.0114983183538967\\
422	0.0115070166679698\\
423	0.0115158792775034\\
424	0.0115249083464358\\
425	0.011534105825837\\
426	0.0115434735116269\\
427	0.0115530131156431\\
428	0.0115627264338472\\
429	0.0115726148699541\\
430	0.0115826795884032\\
431	0.0115929215744467\\
432	0.0116033416340704\\
433	0.0116139403989606\\
434	0.0116247183394427\\
435	0.0116356758047032\\
436	0.0116468131891485\\
437	0.011658130166222\\
438	0.0116696251085954\\
439	0.0116812995788674\\
440	0.0116931609430206\\
441	0.0117052123295878\\
442	0.0117174564439994\\
443	0.0117298954678474\\
444	0.0117425309673862\\
445	0.0117553637960252\\
446	0.0117683940461984\\
447	0.0117816206196432\\
448	0.0117950409081307\\
449	0.0118086511499242\\
450	0.0118224482200987\\
451	0.0118366017004372\\
452	0.0118516965814472\\
453	0.0118666902816137\\
454	0.0118815692773868\\
455	0.0118963194924524\\
456	0.0119109263223348\\
457	0.0119253747287171\\
458	0.0119396492821493\\
459	0.0119537343753797\\
460	0.0119676129564614\\
461	0.0119812683038507\\
462	0.0119946834547688\\
463	0.012007842314727\\
464	0.0120204687221322\\
465	0.0120324695224117\\
466	0.012044296598016\\
467	0.0120559434294043\\
468	0.0120674041586191\\
469	0.0120786737931792\\
470	0.0120897480629261\\
471	0.0121006235307648\\
472	0.0121112980683367\\
473	0.012121771009318\\
474	0.0121320436906436\\
475	0.0121421182091894\\
476	0.0121519036362758\\
477	0.0121611128328961\\
478	0.0121703036927578\\
479	0.0121794798290459\\
480	0.0121886454420582\\
481	0.0121978053289121\\
482	0.0122069648802642\\
483	0.0122161300711763\\
484	0.012225307447051\\
485	0.0122345041085784\\
486	0.0122437276773052\\
487	0.0122529862511417\\
488	0.0122622962442405\\
489	0.0122716777140519\\
490	0.0122811331814007\\
491	0.0122906651825224\\
492	0.0123002762436723\\
493	0.0123099688530845\\
494	0.0123197454299354\\
495	0.0123296082904271\\
496	0.0123395596112906\\
497	0.0123496013908835\\
498	0.0123597354080664\\
499	0.0123699631789268\\
500	0.0123802850986567\\
501	0.0123907012528107\\
502	0.0124012115798434\\
503	0.012411815857483\\
504	0.0124225136890124\\
505	0.0124333044895893\\
506	0.0124441874727437\\
507	0.012455161637205\\
508	0.0124662257542208\\
509	0.0124773783555404\\
510	0.0124886177222416\\
511	0.0124999418975655\\
512	0.0125113486972977\\
513	0.0125228356956447\\
514	0.0125344002104368\\
515	0.0125460392876003\\
516	0.0125577496848353\\
517	0.0125695278544169\\
518	0.0125813699250281\\
519	0.0125932716825087\\
520	0.0126052285493902\\
521	0.0126172355630605\\
522	0.0126292873517538\\
523	0.0126413781071186\\
524	0.0126535015547546\\
525	0.0126656509225293\\
526	0.0126778189064672\\
527	0.012689997633983\\
528	0.0127021786242022\\
529	0.0127143527450922\\
530	0.0127265101670925\\
531	0.0127386403129051\\
532	0.0127507318030685\\
533	0.012762772396914\\
534	0.0127747489285695\\
535	0.0127866472375425\\
536	0.0127984520933345\\
537	0.0128101471135007\\
538	0.0128217146745347\\
539	0.0128331358157645\\
540	0.0128443901359816\\
541	0.0128554556808294\\
542	0.0128663088147238\\
543	0.0128769240150361\\
544	0.0128872740476812\\
545	0.0128973178339668\\
546	0.012907432488429\\
547	0.012917820347576\\
548	0.0129284971828948\\
549	0.0129407509586339\\
550	0.0129533441211697\\
551	0.0129660773871302\\
552	0.0129790502129792\\
553	0.012991283400045\\
554	0.013003656613236\\
555	0.013016156892264\\
556	0.0130286737560499\\
557	0.0130406673739529\\
558	0.0130532662719135\\
559	0.0130662919523494\\
560	0.0130792815800368\\
561	0.0130918882960037\\
562	0.0131064744862137\\
563	0.0131215420187478\\
564	0.0131364758718319\\
565	0.0131513443502391\\
566	0.0131654824277249\\
567	0.0131793793020266\\
568	0.0131928702337398\\
569	0.0132053903351803\\
570	0.0132172787509962\\
571	0.0132291174803308\\
572	0.0132410089728038\\
573	0.0132529107767735\\
574	0.0132644260658691\\
575	0.0132757422995737\\
576	0.0132868208689815\\
577	0.0132978465512147\\
578	0.013308835298017\\
579	0.0133197760065791\\
580	0.0133306534585244\\
581	0.0133414489699862\\
582	0.0133521416318217\\
583	0.0133627085270204\\
584	0.0133731244400726\\
585	0.0133833615638285\\
586	0.0133933893817503\\
587	0.0134031745421507\\
588	0.0134126809585648\\
589	0.0134218705998152\\
590	0.0134307062886727\\
591	0.0134391597728698\\
592	0.0134472338645631\\
593	0.0134551345125145\\
594	0.0134637112677116\\
595	0.0134740407759367\\
596	0.0134889777516421\\
597	0.0135160557778789\\
598	0.0135751148335434\\
599	0\\
600	0\\
};
\addplot [color=red,solid,forget plot]
  table[row sep=crcr]{%
1	0.0110654788392184\\
2	0.0110654825560351\\
3	0.0110654863458908\\
4	0.0110654902101524\\
5	0.0110654941502113\\
6	0.0110654981674829\\
7	0.0110655022634074\\
8	0.0110655064394505\\
9	0.011065510697103\\
10	0.0110655150378819\\
11	0.0110655194633307\\
12	0.0110655239750196\\
13	0.0110655285745461\\
14	0.0110655332635352\\
15	0.0110655380436403\\
16	0.0110655429165433\\
17	0.011065547883955\\
18	0.0110655529476159\\
19	0.0110655581092963\\
20	0.011065563370797\\
21	0.0110655687339498\\
22	0.0110655742006179\\
23	0.0110655797726964\\
24	0.0110655854521127\\
25	0.0110655912408274\\
26	0.0110655971408345\\
27	0.0110656031541618\\
28	0.0110656092828718\\
29	0.0110656155290622\\
30	0.0110656218948661\\
31	0.0110656283824531\\
32	0.0110656349940292\\
33	0.0110656417318381\\
34	0.0110656485981614\\
35	0.0110656555953192\\
36	0.0110656627256707\\
37	0.011065669991615\\
38	0.0110656773955917\\
39	0.0110656849400813\\
40	0.0110656926276062\\
41	0.0110657004607311\\
42	0.0110657084420637\\
43	0.0110657165742556\\
44	0.0110657248600027\\
45	0.0110657333020462\\
46	0.011065741903173\\
47	0.0110657506662169\\
48	0.0110657595940587\\
49	0.0110657686896276\\
50	0.0110657779559015\\
51	0.0110657873959082\\
52	0.0110657970127258\\
53	0.0110658068094837\\
54	0.0110658167893635\\
55	0.0110658269555997\\
56	0.0110658373114808\\
57	0.0110658478603496\\
58	0.011065858605605\\
59	0.011065869550702\\
60	0.0110658806991533\\
61	0.0110658920545298\\
62	0.0110659036204617\\
63	0.0110659154006397\\
64	0.0110659273988156\\
65	0.0110659396188037\\
66	0.0110659520644815\\
67	0.011065964739791\\
68	0.0110659776487397\\
69	0.0110659907954016\\
70	0.0110660041839186\\
71	0.0110660178185012\\
72	0.0110660317034302\\
73	0.0110660458430574\\
74	0.0110660602418071\\
75	0.0110660749041774\\
76	0.0110660898347412\\
77	0.0110661050381476\\
78	0.0110661205191235\\
79	0.0110661362824744\\
80	0.0110661523330863\\
81	0.0110661686759268\\
82	0.0110661853160468\\
83	0.0110662022585817\\
84	0.0110662195087529\\
85	0.0110662370718696\\
86	0.0110662549533303\\
87	0.0110662731586241\\
88	0.0110662916933329\\
89	0.0110663105631324\\
90	0.0110663297737945\\
91	0.0110663493311886\\
92	0.0110663692412836\\
93	0.0110663895101496\\
94	0.01106641014396\\
95	0.0110664311489931\\
96	0.0110664525316346\\
97	0.011066474298379\\
98	0.0110664964558322\\
99	0.0110665190107133\\
100	0.0110665419698569\\
101	0.0110665653402152\\
102	0.0110665891288605\\
103	0.0110666133429874\\
104	0.0110666379899151\\
105	0.0110666630770901\\
106	0.0110666886120884\\
107	0.0110667146026183\\
108	0.011066741056523\\
109	0.0110667679817833\\
110	0.0110667953865202\\
111	0.0110668232789979\\
112	0.0110668516676267\\
113	0.011066880560966\\
114	0.0110669099677274\\
115	0.0110669398967773\\
116	0.0110669703571411\\
117	0.0110670013580056\\
118	0.0110670329087227\\
119	0.011067065018813\\
120	0.011067097697969\\
121	0.0110671309560593\\
122	0.0110671648031314\\
123	0.0110671992494167\\
124	0.0110672343053331\\
125	0.0110672699814903\\
126	0.0110673062886928\\
127	0.011067343237945\\
128	0.0110673808404548\\
129	0.0110674191076386\\
130	0.0110674580511255\\
131	0.0110674976827619\\
132	0.0110675380146165\\
133	0.0110675790589851\\
134	0.0110676208283956\\
135	0.011067663335613\\
136	0.0110677065936447\\
137	0.0110677506157462\\
138	0.0110677954154261\\
139	0.011067841006452\\
140	0.0110678874028564\\
141	0.0110679346189423\\
142	0.0110679826692898\\
143	0.0110680315687615\\
144	0.0110680813325096\\
145	0.0110681319759822\\
146	0.0110681835149296\\
147	0.0110682359654118\\
148	0.0110682893438047\\
149	0.0110683436668078\\
150	0.0110683989514515\\
151	0.0110684552151042\\
152	0.0110685124754801\\
153	0.0110685707506473\\
154	0.0110686300590356\\
155	0.0110686904194443\\
156	0.0110687518510513\\
157	0.0110688143734211\\
158	0.0110688780065137\\
159	0.0110689427706935\\
160	0.0110690086867385\\
161	0.0110690757758495\\
162	0.0110691440596597\\
163	0.0110692135602444\\
164	0.0110692843001309\\
165	0.0110693563023085\\
166	0.0110694295902392\\
167	0.0110695041878678\\
168	0.0110695801196331\\
169	0.0110696574104787\\
170	0.0110697360858638\\
171	0.0110698161717755\\
172	0.0110698976947397\\
173	0.0110699806818333\\
174	0.0110700651606963\\
175	0.0110701511595439\\
176	0.0110702387071796\\
177	0.0110703278330076\\
178	0.0110704185670457\\
179	0.0110705109399393\\
180	0.0110706049829742\\
181	0.0110707007280906\\
182	0.0110707982078977\\
183	0.011070897455687\\
184	0.0110709985054478\\
185	0.0110711013918812\\
186	0.0110712061504159\\
187	0.0110713128172229\\
188	0.0110714214292315\\
189	0.0110715320241451\\
190	0.0110716446404572\\
191	0.0110717593174676\\
192	0.0110718760952998\\
193	0.0110719950149173\\
194	0.0110721161181409\\
195	0.0110722394476667\\
196	0.0110723650470833\\
197	0.0110724929608906\\
198	0.0110726232345176\\
199	0.0110727559143416\\
200	0.0110728910477069\\
201	0.0110730286829444\\
202	0.0110731688693913\\
203	0.0110733116574108\\
204	0.0110734570984128\\
205	0.0110736052448746\\
206	0.0110737561503616\\
207	0.0110739098695489\\
208	0.0110740664582433\\
209	0.0110742259734052\\
210	0.011074388473171\\
211	0.0110745540168765\\
212	0.0110747226650797\\
213	0.0110748944795851\\
214	0.0110750695234676\\
215	0.0110752478610972\\
216	0.0110754295581641\\
217	0.0110756146817044\\
218	0.0110758033001263\\
219	0.0110759954832362\\
220	0.0110761913022664\\
221	0.0110763908299027\\
222	0.0110765941403121\\
223	0.0110768013091723\\
224	0.0110770124137009\\
225	0.0110772275326851\\
226	0.0110774467465129\\
227	0.0110776701372042\\
228	0.0110778977884429\\
229	0.0110781297856097\\
230	0.0110783662158154\\
231	0.0110786071679353\\
232	0.0110788527326445\\
233	0.011079103002453\\
234	0.0110793580717433\\
235	0.0110796180368069\\
236	0.0110798829958835\\
237	0.0110801530491999\\
238	0.0110804282990102\\
239	0.0110807088496368\\
240	0.0110809948075128\\
241	0.0110812862812248\\
242	0.0110815833815568\\
243	0.0110818862215358\\
244	0.0110821949164773\\
245	0.0110825095840328\\
246	0.0110828303442376\\
247	0.0110831573195607\\
248	0.0110834906349545\\
249	0.0110838304179067\\
250	0.0110841767984926\\
251	0.011084529909429\\
252	0.0110848898861291\\
253	0.0110852568667581\\
254	0.0110856309922908\\
255	0.0110860124065696\\
256	0.0110864012563643\\
257	0.0110867976914325\\
258	0.0110872018645819\\
259	0.0110876139317329\\
260	0.0110880340519836\\
261	0.0110884623876753\\
262	0.011088899104459\\
263	0.0110893443713644\\
264	0.0110897983608689\\
265	0.0110902612489689\\
266	0.0110907332152518\\
267	0.0110912144429701\\
268	0.0110917051191163\\
269	0.0110922054345\\
270	0.0110927155838255\\
271	0.0110932357657722\\
272	0.0110937661830754\\
273	0.01109430704261\\
274	0.0110948585554745\\
275	0.0110954209370778\\
276	0.0110959944072277\\
277	0.0110965791902206\\
278	0.0110971755149343\\
279	0.0110977836149213\\
280	0.0110984037285057\\
281	0.0110990360988814\\
282	0.0110996809742122\\
283	0.0111003386077354\\
284	0.0111010092578667\\
285	0.0111016931883079\\
286	0.0111023906681572\\
287	0.0111031019720217\\
288	0.0111038273801333\\
289	0.0111045671784663\\
290	0.0111053216588589\\
291	0.0111060911191363\\
292	0.011106875863238\\
293	0.0111076762013467\\
294	0.0111084924500211\\
295	0.0111093249323317\\
296	0.0111101739779984\\
297	0.0111110399235329\\
298	0.0111119231123825\\
299	0.0111128238950779\\
300	0.0111137426293833\\
301	0.0111146796804497\\
302	0.0111156354209713\\
303	0.0111166102313439\\
304	0.0111176044998271\\
305	0.0111186186227094\\
306	0.0111196530044754\\
307	0.0111207080579762\\
308	0.0111217842046012\\
309	0.0111228818744546\\
310	0.0111240015065333\\
311	0.0111251435489083\\
312	0.0111263084589084\\
313	0.0111274967033076\\
314	0.0111287087585151\\
315	0.011129945110768\\
316	0.0111312062563267\\
317	0.0111324927016753\\
318	0.0111338049637239\\
319	0.0111351435700148\\
320	0.0111365090589328\\
321	0.0111379019799188\\
322	0.011139322893688\\
323	0.0111407723724531\\
324	0.0111422510001514\\
325	0.0111437593726783\\
326	0.0111452980981248\\
327	0.0111468677970206\\
328	0.0111484691025837\\
329	0.011150102660976\\
330	0.011151769131564\\
331	0.0111534691871862\\
332	0.0111552035144266\\
333	0.0111569728138959\\
334	0.0111587778005172\\
335	0.011160619203818\\
336	0.0111624977682282\\
337	0.0111644142533813\\
338	0.0111663694344195\\
339	0.011168364102302\\
340	0.0111703990641225\\
341	0.0111724751434315\\
342	0.0111745931805426\\
343	0.0111767540328467\\
344	0.0111789585751259\\
345	0.0111812076998633\\
346	0.0111835023175514\\
347	0.0111858433569973\\
348	0.011188231765631\\
349	0.0111906685098013\\
350	0.0111931545750712\\
351	0.0111956909665116\\
352	0.01119827870899\\
353	0.0112009188474486\\
354	0.0112036124471892\\
355	0.0112063605941655\\
356	0.0112091643952692\\
357	0.0112120249786188\\
358	0.0112149434938525\\
359	0.011217921112428\\
360	0.0112209590279296\\
361	0.0112240584563857\\
362	0.0112272206365981\\
363	0.0112304468304815\\
364	0.0112337383234066\\
365	0.0112370964245236\\
366	0.0112405224670688\\
367	0.0112440178089514\\
368	0.0112475838331259\\
369	0.0112512219479109\\
370	0.0112549335873582\\
371	0.0112587202115855\\
372	0.0112625833069188\\
373	0.0112665243860911\\
374	0.011270544988467\\
375	0.0112746466801043\\
376	0.0112788310537305\\
377	0.0112830997286273\\
378	0.0112874543504153\\
379	0.0112918965907371\\
380	0.0112964281468345\\
381	0.0113010507410177\\
382	0.0113057661200114\\
383	0.0113105760541221\\
384	0.0113154823361842\\
385	0.0113204867811843\\
386	0.0113255912257362\\
387	0.0113307975266277\\
388	0.0113361075590537\\
389	0.0113415232173349\\
390	0.0113470464124603\\
391	0.0113526790729982\\
392	0.0113584231454704\\
393	0.0113642805947291\\
394	0.0113702534042987\\
395	0.0113763435766419\\
396	0.0113825531332989\\
397	0.0113888841147703\\
398	0.011395338579582\\
399	0.0114019186003659\\
400	0.0114086262613894\\
401	0.0114154636685429\\
402	0.0114224329417266\\
403	0.0114295362233984\\
404	0.0114367756423113\\
405	0.0114441533192023\\
406	0.0114516713606121\\
407	0.0114593318262876\\
408	0.0114671367678746\\
409	0.0114750881348862\\
410	0.0114831877665219\\
411	0.011491437385288\\
412	0.0114998385900102\\
413	0.0115083927934869\\
414	0.0115171012326338\\
415	0.0115259650778664\\
416	0.0115349854530479\\
417	0.0115441634406638\\
418	0.0115534997632943\\
419	0.0115629950134331\\
420	0.0115726496655751\\
421	0.0115824640968761\\
422	0.0115924386311388\\
423	0.0116025734950854\\
424	0.0116128692638888\\
425	0.011623329789713\\
426	0.011633957822805\\
427	0.0116447555055206\\
428	0.0116557249137582\\
429	0.0116668713611722\\
430	0.0116781974182577\\
431	0.0116897043609441\\
432	0.0117013927699411\\
433	0.0117132624133717\\
434	0.0117253121392036\\
435	0.0117375397732391\\
436	0.0117499420961473\\
437	0.0117625144310969\\
438	0.0117758560399216\\
439	0.0117894674134734\\
440	0.0118029894575224\\
441	0.0118164099705272\\
442	0.0118297162722751\\
443	0.0118428952446038\\
444	0.0118559332399957\\
445	0.0118688161988171\\
446	0.0118815297001557\\
447	0.0118940590655509\\
448	0.0119063895998902\\
449	0.0119185058611117\\
450	0.0119303922637502\\
451	0.0119419300008739\\
452	0.0119527470665781\\
453	0.0119634065695056\\
454	0.0119739030383462\\
455	0.0119842307349347\\
456	0.0119943845632567\\
457	0.0120043601928905\\
458	0.0120141541646721\\
459	0.0120237640974739\\
460	0.0120331881808683\\
461	0.0120424260409478\\
462	0.0120514785273426\\
463	0.012060348256128\\
464	0.0120688519203595\\
465	0.0120769633773652\\
466	0.0120850565833262\\
467	0.0120931350250382\\
468	0.0121012027318994\\
469	0.0121092642855019\\
470	0.0121173248245163\\
471	0.0121253900406385\\
472	0.012133466163393\\
473	0.0121415599377372\\
474	0.0121496785936689\\
475	0.0121578298162321\\
476	0.0121660252473006\\
477	0.0121742881427895\\
478	0.0121826212162018\\
479	0.0121910272252803\\
480	0.0121995089512901\\
481	0.0122080691758438\\
482	0.0122167106555299\\
483	0.0122254360943799\\
484	0.0122342481141834\\
485	0.0122431492225112\\
486	0.0122521417789232\\
487	0.0122612279594847\\
488	0.0122704094060967\\
489	0.0122796870356028\\
490	0.0122890616833396\\
491	0.0122985340938528\\
492	0.0123081049116668\\
493	0.012317774672219\\
494	0.0123275437930923\\
495	0.0123374125656944\\
496	0.0123473811475309\\
497	0.0123574495552308\\
498	0.0123676176584917\\
499	0.0123778851751265\\
500	0.0123882517006474\\
501	0.012398716707723\\
502	0.0124092795388883\\
503	0.0124199393990324\\
504	0.0124306953476433\\
505	0.0124415462907825\\
506	0.0124524909727523\\
507	0.0124635279674112\\
508	0.0124746556690771\\
509	0.0124858722829499\\
510	0.0124971758149644\\
511	0.0125085640600067\\
512	0.0125200345883983\\
513	0.0125315847314718\\
514	0.0125432115661564\\
515	0.0125549118984843\\
516	0.0125666822459204\\
517	0.0125785188184073\\
518	0.0125904174980062\\
519	0.0126023738170057\\
520	0.0126143829343539\\
521	0.0126264396102611\\
522	0.0126385381788273\\
523	0.0126506725185891\\
524	0.0126628360207987\\
525	0.0126750215552332\\
526	0.0126872214333117\\
527	0.0126994273682752\\
528	0.012711630432161\\
529	0.0127238210092769\\
530	0.0127359887458533\\
531	0.0127481224955192\\
532	0.0127602102602153\\
533	0.0127722391260862\\
534	0.0127841951938639\\
535	0.0127960635032276\\
536	0.0128078279519002\\
537	0.0128194712084866\\
538	0.012830974617619\\
539	0.012842318056499\\
540	0.0128534797603971\\
541	0.0128644361906405\\
542	0.0128751621616999\\
543	0.0128856343763971\\
544	0.0128958158226031\\
545	0.012906873956167\\
546	0.0129187928141379\\
547	0.0129308400010039\\
548	0.0129431205246169\\
549	0.0129547693726181\\
550	0.0129664376602281\\
551	0.0129782299193289\\
552	0.0129900292355768\\
553	0.0130012990174047\\
554	0.0130127502747141\\
555	0.0130245386352892\\
556	0.0130370554092916\\
557	0.0130495984537013\\
558	0.0130620343918673\\
559	0.0130767491608591\\
560	0.01309148410142\\
561	0.0131061948644112\\
562	0.0131197439673824\\
563	0.0131334739835205\\
564	0.0131473091419805\\
565	0.0131603272205647\\
566	0.0131723874138756\\
567	0.0131842407045699\\
568	0.0131958207769496\\
569	0.0132073548170946\\
570	0.0132189563508797\\
571	0.0132308155192537\\
572	0.0132423773181155\\
573	0.0132536887823157\\
574	0.0132647731328702\\
575	0.0132758105462532\\
576	0.0132868405055488\\
577	0.013297854502466\\
578	0.01330883885008\\
579	0.0133197780028337\\
580	0.0133306546245664\\
581	0.0133414496207317\\
582	0.0133521420230989\\
583	0.0133627087929893\\
584	0.0133731246090603\\
585	0.0133833616591944\\
586	0.0133933894304777\\
587	0.0134031745665083\\
588	0.0134126809661912\\
589	0.0134218705998152\\
590	0.0134307062886727\\
591	0.0134391597728698\\
592	0.0134472338645631\\
593	0.0134551345125145\\
594	0.0134637112677116\\
595	0.0134740407759367\\
596	0.0134889777516421\\
597	0.0135160557778789\\
598	0.0135751148335434\\
599	0\\
600	0\\
};
\addplot [color=mycolor20,solid,forget plot]
  table[row sep=crcr]{%
1	0.0110656588731661\\
2	0.0110656635661861\\
3	0.011065668354941\\
4	0.011065673241298\\
5	0.0110656782271588\\
6	0.0110656833144596\\
7	0.0110656885051719\\
8	0.0110656938013029\\
9	0.0110656992048966\\
10	0.0110657047180336\\
11	0.0110657103428322\\
12	0.0110657160814492\\
13	0.01106572193608\\
14	0.0110657279089596\\
15	0.011065734002363\\
16	0.011065740218606\\
17	0.0110657465600459\\
18	0.0110657530290821\\
19	0.0110657596281567\\
20	0.011065766359755\\
21	0.011065773226407\\
22	0.0110657802306869\\
23	0.0110657873752148\\
24	0.011065794662657\\
25	0.0110658020957267\\
26	0.0110658096771849\\
27	0.011065817409841\\
28	0.0110658252965537\\
29	0.0110658333402315\\
30	0.011065841543834\\
31	0.0110658499103719\\
32	0.0110658584429087\\
33	0.0110658671445607\\
34	0.0110658760184984\\
35	0.0110658850679469\\
36	0.0110658942961871\\
37	0.0110659037065563\\
38	0.0110659133024493\\
39	0.0110659230873188\\
40	0.0110659330646768\\
41	0.0110659432380954\\
42	0.0110659536112074\\
43	0.0110659641877075\\
44	0.0110659749713531\\
45	0.0110659859659653\\
46	0.0110659971754299\\
47	0.0110660086036983\\
48	0.0110660202547882\\
49	0.0110660321327854\\
50	0.0110660442418437\\
51	0.0110660565861869\\
52	0.0110660691701093\\
53	0.0110660819979767\\
54	0.0110660950742279\\
55	0.0110661084033752\\
56	0.011066121990006\\
57	0.0110661358387835\\
58	0.0110661499544482\\
59	0.0110661643418185\\
60	0.0110661790057925\\
61	0.0110661939513485\\
62	0.0110662091835468\\
63	0.0110662247075303\\
64	0.0110662405285262\\
65	0.011066256651847\\
66	0.0110662730828915\\
67	0.0110662898271466\\
68	0.0110663068901884\\
69	0.0110663242776829\\
70	0.0110663419953884\\
71	0.0110663600491558\\
72	0.0110663784449307\\
73	0.0110663971887542\\
74	0.0110664162867648\\
75	0.0110664357451994\\
76	0.0110664555703952\\
77	0.0110664757687905\\
78	0.0110664963469269\\
79	0.0110665173114504\\
80	0.0110665386691129\\
81	0.011066560426774\\
82	0.0110665825914024\\
83	0.0110666051700776\\
84	0.0110666281699915\\
85	0.0110666515984499\\
86	0.0110666754628747\\
87	0.0110666997708051\\
88	0.0110667245298995\\
89	0.0110667497479375\\
90	0.0110667754328214\\
91	0.0110668015925783\\
92	0.011066828235362\\
93	0.0110668553694546\\
94	0.011066883003269\\
95	0.0110669111453504\\
96	0.0110669398043786\\
97	0.0110669689891701\\
98	0.01106699870868\\
99	0.0110670289720046\\
100	0.0110670597883831\\
101	0.0110670911672003\\
102	0.0110671231179887\\
103	0.0110671556504308\\
104	0.0110671887743619\\
105	0.0110672224997719\\
106	0.0110672568368086\\
107	0.0110672917957797\\
108	0.0110673273871558\\
109	0.0110673636215728\\
110	0.011067400509835\\
111	0.0110674380629178\\
112	0.0110674762919705\\
113	0.0110675152083196\\
114	0.0110675548234713\\
115	0.0110675951491154\\
116	0.0110676361971277\\
117	0.0110676779795741\\
118	0.0110677205087132\\
119	0.0110677637970005\\
120	0.0110678078570913\\
121	0.0110678527018448\\
122	0.0110678983443277\\
123	0.0110679447978183\\
124	0.0110679920758098\\
125	0.0110680401920152\\
126	0.011068089160371\\
127	0.0110681389950414\\
128	0.011068189710423\\
129	0.0110682413211494\\
130	0.0110682938420954\\
131	0.0110683472883824\\
132	0.0110684016753828\\
133	0.0110684570187255\\
134	0.0110685133343006\\
135	0.0110685706382656\\
136	0.0110686289470499\\
137	0.0110686882773614\\
138	0.0110687486461916\\
139	0.0110688100708223\\
140	0.0110688725688311\\
141	0.0110689361580985\\
142	0.0110690008568137\\
143	0.011069066683482\\
144	0.0110691336569312\\
145	0.0110692017963191\\
146	0.0110692711211408\\
147	0.0110693416512361\\
148	0.0110694134067975\\
149	0.011069486408378\\
150	0.0110695606768993\\
151	0.0110696362336606\\
152	0.0110697131003469\\
153	0.0110697912990384\\
154	0.0110698708522192\\
155	0.0110699517827873\\
156	0.011070034114064\\
157	0.0110701178698039\\
158	0.0110702030742057\\
159	0.011070289751922\\
160	0.0110703779280709\\
161	0.0110704676282469\\
162	0.011070558878532\\
163	0.0110706517055084\\
164	0.0110707461362698\\
165	0.0110708421984341\\
166	0.0110709399201563\\
167	0.0110710393301417\\
168	0.0110711404576588\\
169	0.0110712433325538\\
170	0.0110713479852643\\
171	0.0110714544468339\\
172	0.0110715627489273\\
173	0.0110716729238451\\
174	0.01107178500454\\
175	0.0110718990246322\\
176	0.0110720150184262\\
177	0.0110721330209276\\
178	0.0110722530678597\\
179	0.0110723751956818\\
180	0.0110724994416068\\
181	0.0110726258436193\\
182	0.011072754440495\\
183	0.0110728852718191\\
184	0.0110730183780066\\
185	0.0110731538003218\\
186	0.0110732915808987\\
187	0.0110734317627623\\
188	0.0110735743898491\\
189	0.0110737195070293\\
190	0.0110738671601286\\
191	0.0110740173959506\\
192	0.0110741702622998\\
193	0.0110743258080048\\
194	0.0110744840829419\\
195	0.0110746451380594\\
196	0.0110748090254018\\
197	0.0110749757981348\\
198	0.0110751455105705\\
199	0.0110753182181933\\
200	0.0110754939776854\\
201	0.0110756728469539\\
202	0.011075854885157\\
203	0.0110760401527315\\
204	0.0110762287114203\\
205	0.0110764206243\\
206	0.0110766159558096\\
207	0.0110768147717787\\
208	0.0110770171394568\\
209	0.0110772231275421\\
210	0.0110774328062119\\
211	0.0110776462471517\\
212	0.0110778635235862\\
213	0.0110780847103094\\
214	0.0110783098837161\\
215	0.0110785391218329\\
216	0.01107877250435\\
217	0.011079010112653\\
218	0.0110792520298556\\
219	0.0110794983408319\\
220	0.01107974913225\\
221	0.0110800044926049\\
222	0.0110802645122528\\
223	0.0110805292834451\\
224	0.0110807989003635\\
225	0.0110810734591544\\
226	0.0110813530579656\\
227	0.0110816377969816\\
228	0.0110819277784606\\
229	0.0110822231067719\\
230	0.0110825238884336\\
231	0.0110828302321515\\
232	0.0110831422488579\\
233	0.011083460051752\\
234	0.0110837837563407\\
235	0.01108411348048\\
236	0.0110844493444183\\
237	0.0110847914708392\\
238	0.0110851399849069\\
239	0.0110854950143115\\
240	0.0110858566893161\\
241	0.0110862251428051\\
242	0.0110866005103335\\
243	0.0110869829301781\\
244	0.0110873725433894\\
245	0.0110877694938457\\
246	0.0110881739283083\\
247	0.0110885859964784\\
248	0.0110890058510558\\
249	0.0110894336477992\\
250	0.0110898695455883\\
251	0.0110903137064869\\
252	0.0110907662958094\\
253	0.0110912274821874\\
254	0.0110916974376391\\
255	0.0110921763376401\\
256	0.011092664361196\\
257	0.011093161690917\\
258	0.0110936685130938\\
259	0.0110941850177758\\
260	0.0110947113988498\\
261	0.0110952478541224\\
262	0.0110957945854023\\
263	0.0110963517985851\\
264	0.0110969197037383\\
265	0.0110974985151895\\
266	0.0110980884516149\\
267	0.01109868973613\\
268	0.0110993025963811\\
269	0.011099927264639\\
270	0.0111005639778934\\
271	0.0111012129779492\\
272	0.0111018745115237\\
273	0.0111025488303456\\
274	0.0111032361912549\\
275	0.0111039368563045\\
276	0.0111046510928631\\
277	0.0111053791737189\\
278	0.0111061213771856\\
279	0.0111068779872093\\
280	0.0111076492934765\\
281	0.0111084355915246\\
282	0.0111092371828535\\
283	0.0111100543750385\\
284	0.0111108874818465\\
285	0.0111117368233526\\
286	0.0111126027260602\\
287	0.0111134855230232\\
288	0.0111143855539703\\
289	0.0111153031654337\\
290	0.0111162387108792\\
291	0.0111171925508416\\
292	0.0111181650530624\\
293	0.011119156592633\\
294	0.0111201675521415\\
295	0.0111211983218244\\
296	0.0111222492997236\\
297	0.0111233208918489\\
298	0.0111244135123462\\
299	0.0111255275836715\\
300	0.0111266635367701\\
301	0.0111278218112625\\
302	0.0111290028556362\\
303	0.0111302071274427\\
304	0.0111314350934994\\
305	0.0111326872300972\\
306	0.0111339640232183\\
307	0.0111352659687583\\
308	0.0111365935727425\\
309	0.0111379473515509\\
310	0.0111393278321466\\
311	0.0111407355523043\\
312	0.0111421710608418\\
313	0.0111436349178529\\
314	0.0111451276949402\\
315	0.0111466499754475\\
316	0.011148202354693\\
317	0.0111497854401991\\
318	0.0111513998519233\\
319	0.0111530462224844\\
320	0.0111547251973883\\
321	0.0111564374352496\\
322	0.0111581836080109\\
323	0.011159964401158\\
324	0.0111617805139326\\
325	0.011163632659543\\
326	0.0111655215653784\\
327	0.0111674479732104\\
328	0.0111694126394004\\
329	0.0111714163351074\\
330	0.0111734598464994\\
331	0.0111755439749672\\
332	0.0111776695373386\\
333	0.0111798373661039\\
334	0.0111820483096634\\
335	0.0111843032325815\\
336	0.0111866030158573\\
337	0.0111889485572148\\
338	0.0111913407714107\\
339	0.0111937805905531\\
340	0.0111962689644245\\
341	0.0111988068609057\\
342	0.0112013952664816\\
343	0.0112040351865738\\
344	0.011206727645988\\
345	0.011209473689358\\
346	0.0112122743815796\\
347	0.0112151308082252\\
348	0.0112180440759428\\
349	0.0112210153129062\\
350	0.0112240456691409\\
351	0.0112271363168317\\
352	0.0112302884506114\\
353	0.0112335032877874\\
354	0.011236782068394\\
355	0.0112401260552502\\
356	0.0112435365340412\\
357	0.0112470148132364\\
358	0.0112505622239224\\
359	0.0112541801195469\\
360	0.0112578698755749\\
361	0.0112616328890624\\
362	0.0112654705781572\\
363	0.0112693843815436\\
364	0.0112733757578449\\
365	0.0112774461849569\\
366	0.0112815971590794\\
367	0.0112858301929141\\
368	0.0112901468173314\\
369	0.0112945485809067\\
370	0.0112990370495442\\
371	0.0113036138072524\\
372	0.0113082804573839\\
373	0.0113130386221138\\
374	0.011317889943075\\
375	0.0113228360832951\\
376	0.0113278787281646\\
377	0.0113330195863668\\
378	0.0113382603907182\\
379	0.0113436028988527\\
380	0.0113490488936747\\
381	0.0113546001835024\\
382	0.0113602586018019\\
383	0.0113660260063192\\
384	0.011371904276819\\
385	0.0113778953087333\\
386	0.0113840010146819\\
387	0.0113902233257655\\
388	0.0113965641818177\\
389	0.0114030255123961\\
390	0.011409609248763\\
391	0.0114163172479929\\
392	0.0114231512907846\\
393	0.0114301130741022\\
394	0.011437204204364\\
395	0.0114444261914239\\
396	0.0114517804436914\\
397	0.0114592682649465\\
398	0.0114668908537145\\
399	0.0114746493042378\\
400	0.0114825445740172\\
401	0.0114905774474638\\
402	0.0114987486950691\\
403	0.0115070590027617\\
404	0.0115155092029484\\
405	0.0115240998923759\\
406	0.0115328315832527\\
407	0.0115417047900948\\
408	0.0115507199473595\\
409	0.011559880551949\\
410	0.0115691900038285\\
411	0.0115786515690362\\
412	0.0115882683573538\\
413	0.0115980434026293\\
414	0.011607978875719\\
415	0.0116180758717898\\
416	0.0116283356287589\\
417	0.01163875981301\\
418	0.0116493513417758\\
419	0.0116601098803842\\
420	0.0116710342296406\\
421	0.0116821222266719\\
422	0.0116933706785615\\
423	0.0117049271349924\\
424	0.0117172129725119\\
425	0.0117294351192178\\
426	0.0117415830819485\\
427	0.0117536457520934\\
428	0.0117656116160091\\
429	0.0117774703065973\\
430	0.0117892100098877\\
431	0.0118008181955253\\
432	0.0118122819947597\\
433	0.011823588257149\\
434	0.0118347235171335\\
435	0.0118456741218799\\
436	0.0118564263042342\\
437	0.0118669662942504\\
438	0.0118769230014001\\
439	0.0118865480721362\\
440	0.0118960284246959\\
441	0.0119053582604444\\
442	0.0119145321531706\\
443	0.0119235454410943\\
444	0.011932394946901\\
445	0.0119410772972738\\
446	0.0119495899623667\\
447	0.0119579313732811\\
448	0.0119661011118754\\
449	0.0119740996408993\\
450	0.0119819286218101\\
451	0.0119895164676886\\
452	0.011996619753397\\
453	0.012003701392746\\
454	0.01201076468025\\
455	0.0120178130323459\\
456	0.0120248503630588\\
457	0.0120318810905243\\
458	0.0120389101387957\\
459	0.01204594293424\\
460	0.012052985397651\\
461	0.0120600439208837\\
462	0.0120671253375711\\
463	0.0120742368832063\\
464	0.0120813931208556\\
465	0.0120886103077395\\
466	0.012095891125198\\
467	0.0121032383081965\\
468	0.0121106546277406\\
469	0.0121181428711796\\
470	0.0121257058204047\\
471	0.0121333462280934\\
472	0.0121410667922466\\
473	0.0121488701291379\\
474	0.0121567587448235\\
475	0.0121647350050365\\
476	0.0121728009653169\\
477	0.0121809578887974\\
478	0.0121892069921755\\
479	0.0121975494389048\\
480	0.0122059863325002\\
481	0.0122145187100659\\
482	0.0122231475361539\\
483	0.0122318736970723\\
484	0.0122406979957727\\
485	0.0122496211474671\\
486	0.0122586437761191\\
487	0.0122677664119648\\
488	0.0122769895027841\\
489	0.012286313433967\\
490	0.0122957385247837\\
491	0.0123052650246323\\
492	0.0123148931092603\\
493	0.0123246228769539\\
494	0.0123344543446787\\
495	0.0123443874441547\\
496	0.0123544220178348\\
497	0.0123645578147505\\
498	0.0123747944861783\\
499	0.0123851315810655\\
500	0.0123955685397814\\
501	0.0124061046872306\\
502	0.0124167392255343\\
503	0.0124274712262411\\
504	0.012438299622023\\
505	0.0124492231978107\\
506	0.0124602405813163\\
507	0.0124713502328876\\
508	0.0124825504346326\\
509	0.0124938392787495\\
510	0.0125052146549891\\
511	0.0125166742372163\\
512	0.0125282154690323\\
513	0.0125398355483824\\
514	0.0125515314110666\\
515	0.0125632997130623\\
516	0.0125751368115598\\
517	0.0125870387446058\\
518	0.012599001209234\\
519	0.0126110195379583\\
520	0.0126230886734865\\
521	0.0126352031415027\\
522	0.0126473570213505\\
523	0.0126595439144294\\
524	0.0126717569100985\\
525	0.0126839885488629\\
526	0.0126962307825912\\
527	0.0127084749314948\\
528	0.0127207116375674\\
529	0.0127329308141568\\
530	0.0127451215912774\\
531	0.0127572722562432\\
532	0.0127693701891896\\
533	0.0127814017945067\\
534	0.0127933524269847\\
535	0.0128052063114802\\
536	0.0128169463939496\\
537	0.0128285542015397\\
538	0.0128400097268664\\
539	0.0128512933422269\\
540	0.0128623867952452\\
541	0.0128733502284654\\
542	0.0128854094050171\\
543	0.0128970012989598\\
544	0.012908595658433\\
545	0.0129198614533375\\
546	0.0129308452656219\\
547	0.0129419518643324\\
548	0.0129530879392266\\
549	0.0129637514845416\\
550	0.0129745017007499\\
551	0.0129854097915782\\
552	0.0129964303233532\\
553	0.0130082575189452\\
554	0.0130203859606286\\
555	0.0130328503132304\\
556	0.0130472039465158\\
557	0.0130616189824782\\
558	0.0130760062244382\\
559	0.0130889106820684\\
560	0.0131018923632267\\
561	0.0131152098797295\\
562	0.0131277924658908\\
563	0.0131398059732458\\
564	0.01315155121627\\
565	0.0131629995776241\\
566	0.0131743408391387\\
567	0.0131856935415066\\
568	0.0131970799257051\\
569	0.0132087003483928\\
570	0.0132203534899281\\
571	0.0132316468187079\\
572	0.0132427546344582\\
573	0.0132537627637236\\
574	0.0132647834058537\\
575	0.0132758134659479\\
576	0.013286841680444\\
577	0.013297855034141\\
578	0.0133088391434586\\
579	0.0133197781722865\\
580	0.0133306547203085\\
581	0.0133414496789316\\
582	0.0133521420618359\\
583	0.0133627088171417\\
584	0.0133731246225379\\
585	0.0133833616660035\\
586	0.0133933894336898\\
587	0.0134031745674478\\
588	0.0134126809661912\\
589	0.0134218705998152\\
590	0.0134307062886727\\
591	0.0134391597728698\\
592	0.0134472338645631\\
593	0.0134551345125145\\
594	0.0134637112677116\\
595	0.0134740407759367\\
596	0.0134889777516421\\
597	0.0135160557778789\\
598	0.0135751148335434\\
599	0\\
600	0\\
};
\addplot [color=mycolor21,solid,forget plot]
  table[row sep=crcr]{%
1	0.0110657345758834\\
2	0.0110657403531717\\
3	0.0110657462530208\\
4	0.011065752277939\\
5	0.0110657584304834\\
6	0.0110657647132605\\
7	0.0110657711289273\\
8	0.011065777680192\\
9	0.0110657843698148\\
10	0.0110657912006091\\
11	0.0110657981754424\\
12	0.0110658052972369\\
13	0.0110658125689709\\
14	0.0110658199936793\\
15	0.0110658275744552\\
16	0.0110658353144502\\
17	0.0110658432168759\\
18	0.0110658512850048\\
19	0.0110658595221714\\
20	0.0110658679317729\\
21	0.0110658765172707\\
22	0.0110658852821914\\
23	0.0110658942301277\\
24	0.0110659033647395\\
25	0.0110659126897556\\
26	0.0110659222089739\\
27	0.0110659319262633\\
28	0.0110659418455647\\
29	0.011065951970892\\
30	0.0110659623063334\\
31	0.0110659728560528\\
32	0.0110659836242906\\
33	0.0110659946153656\\
34	0.0110660058336755\\
35	0.0110660172836989\\
36	0.0110660289699962\\
37	0.0110660408972108\\
38	0.0110660530700708\\
39	0.0110660654933902\\
40	0.01106607817207\\
41	0.0110660911111001\\
42	0.0110661043155601\\
43	0.0110661177906212\\
44	0.0110661315415474\\
45	0.011066145573697\\
46	0.011066159892524\\
47	0.0110661745035796\\
48	0.0110661894125136\\
49	0.0110662046250763\\
50	0.0110662201471196\\
51	0.0110662359845985\\
52	0.011066252143573\\
53	0.0110662686302097\\
54	0.0110662854507829\\
55	0.0110663026116766\\
56	0.0110663201193861\\
57	0.0110663379805196\\
58	0.0110663562017997\\
59	0.0110663747900654\\
60	0.0110663937522734\\
61	0.0110664130955003\\
62	0.0110664328269436\\
63	0.0110664529539244\\
64	0.0110664734838882\\
65	0.0110664944244074\\
66	0.0110665157831826\\
67	0.0110665375680447\\
68	0.0110665597869566\\
69	0.0110665824480152\\
70	0.011066605559453\\
71	0.0110666291296402\\
72	0.0110666531670863\\
73	0.0110666776804425\\
74	0.0110667026785031\\
75	0.0110667281702079\\
76	0.0110667541646437\\
77	0.0110667806710466\\
78	0.0110668076988039\\
79	0.0110668352574563\\
80	0.0110668633566995\\
81	0.0110668920063865\\
82	0.0110669212165298\\
83	0.0110669509973032\\
84	0.0110669813590442\\
85	0.0110670123122557\\
86	0.0110670438676085\\
87	0.0110670760359435\\
88	0.0110671088282735\\
89	0.0110671422557856\\
90	0.0110671763298434\\
91	0.0110672110619895\\
92	0.011067246463947\\
93	0.0110672825476228\\
94	0.0110673193251088\\
95	0.0110673568086852\\
96	0.0110673950108223\\
97	0.0110674339441826\\
98	0.011067473621624\\
99	0.0110675140562014\\
100	0.0110675552611697\\
101	0.0110675972499858\\
102	0.0110676400363113\\
103	0.0110676836340151\\
104	0.0110677280571758\\
105	0.0110677733200842\\
106	0.0110678194372461\\
107	0.0110678664233847\\
108	0.0110679142934433\\
109	0.0110679630625882\\
110	0.0110680127462113\\
111	0.0110680633599326\\
112	0.0110681149196034\\
113	0.0110681674413089\\
114	0.0110682209413712\\
115	0.0110682754363523\\
116	0.0110683309430567\\
117	0.0110683874785348\\
118	0.011068445060086\\
119	0.0110685037052616\\
120	0.0110685634318679\\
121	0.0110686242579697\\
122	0.0110686862018938\\
123	0.0110687492822317\\
124	0.0110688135178437\\
125	0.011068878927862\\
126	0.0110689455316946\\
127	0.0110690133490285\\
128	0.0110690823998342\\
129	0.0110691527043687\\
130	0.0110692242831803\\
131	0.0110692971571117\\
132	0.0110693713473049\\
133	0.0110694468752052\\
134	0.0110695237625654\\
135	0.0110696020314506\\
136	0.0110696817042425\\
137	0.0110697628036443\\
138	0.0110698453526855\\
139	0.0110699293747272\\
140	0.0110700148934668\\
141	0.0110701019329439\\
142	0.0110701905175452\\
143	0.0110702806720109\\
144	0.0110703724214399\\
145	0.0110704657912963\\
146	0.0110705608074156\\
147	0.0110706574960111\\
148	0.0110707558836807\\
149	0.0110708559974137\\
150	0.0110709578645983\\
151	0.0110710615130289\\
152	0.0110711669709136\\
153	0.0110712742668825\\
154	0.011071383429996\\
155	0.0110714944897532\\
156	0.011071607476101\\
157	0.0110717224194434\\
158	0.0110718393506507\\
159	0.01107195830107\\
160	0.0110720793025352\\
161	0.011072202387378\\
162	0.0110723275884385\\
163	0.0110724549390776\\
164	0.011072584473188\\
165	0.0110727162252075\\
166	0.0110728502301316\\
167	0.0110729865235267\\
168	0.0110731251415443\\
169	0.0110732661209355\\
170	0.0110734094990657\\
171	0.0110735553139302\\
172	0.0110737036041706\\
173	0.0110738544090913\\
174	0.0110740077686766\\
175	0.0110741637236094\\
176	0.0110743223152889\\
177	0.0110744835858506\\
178	0.011074647578186\\
179	0.0110748143359633\\
180	0.011074983903649\\
181	0.0110751563265298\\
182	0.0110753316507355\\
183	0.0110755099232633\\
184	0.0110756911920015\\
185	0.0110758755057553\\
186	0.011076062914273\\
187	0.0110762534682728\\
188	0.0110764472194705\\
189	0.0110766442206088\\
190	0.0110768445254862\\
191	0.0110770481889879\\
192	0.0110772552671174\\
193	0.0110774658170281\\
194	0.0110776798970573\\
195	0.0110778975667598\\
196	0.0110781188869433\\
197	0.0110783439197041\\
198	0.0110785727284644\\
199	0.0110788053780096\\
200	0.0110790419345273\\
201	0.0110792824656472\\
202	0.0110795270404811\\
203	0.0110797757296646\\
204	0.0110800286053994\\
205	0.0110802857414962\\
206	0.0110805472134187\\
207	0.0110808130983281\\
208	0.0110810834751289\\
209	0.0110813584245149\\
210	0.0110816380290157\\
211	0.0110819223730447\\
212	0.011082211542947\\
213	0.0110825056270477\\
214	0.0110828047157015\\
215	0.0110831089013422\\
216	0.0110834182785323\\
217	0.0110837329440138\\
218	0.0110840529967585\\
219	0.0110843785380187\\
220	0.0110847096713784\\
221	0.0110850465028041\\
222	0.011085389140696\\
223	0.0110857376959389\\
224	0.0110860922819529\\
225	0.0110864530147445\\
226	0.0110868200129565\\
227	0.011087193397919\\
228	0.0110875732936984\\
229	0.0110879598271473\\
230	0.0110883531279539\\
231	0.0110887533286901\\
232	0.0110891605648598\\
233	0.0110895749749473\\
234	0.0110899967004638\\
235	0.0110904258859951\\
236	0.0110908626792475\\
237	0.0110913072310951\\
238	0.0110917596956248\\
239	0.0110922202301832\\
240	0.0110926889954224\\
241	0.0110931661553461\\
242	0.0110936518773561\\
243	0.0110941463323\\
244	0.0110946496945182\\
245	0.0110951621418935\\
246	0.0110956838559008\\
247	0.0110962150216589\\
248	0.0110967558279836\\
249	0.0110973064674436\\
250	0.011097867136418\\
251	0.0110984380351571\\
252	0.0110990193678457\\
253	0.0110996113426702\\
254	0.0111002141718887\\
255	0.0111008280719052\\
256	0.0111014532633479\\
257	0.0111020899711511\\
258	0.0111027384246422\\
259	0.011103398857632\\
260	0.0111040715085093\\
261	0.0111047566203393\\
262	0.0111054544409684\\
263	0.0111061652231393\\
264	0.0111068892245963\\
265	0.011107626708201\\
266	0.0111083779420513\\
267	0.0111091431996037\\
268	0.0111099227597985\\
269	0.0111107169071889\\
270	0.0111115259320722\\
271	0.0111123501306243\\
272	0.0111131898050368\\
273	0.0111140452636559\\
274	0.0111149168211245\\
275	0.0111158047985245\\
276	0.0111167095235218\\
277	0.0111176313305117\\
278	0.0111185705607648\\
279	0.0111195275625731\\
280	0.0111205026913958\\
281	0.0111214963100045\\
282	0.0111225087886257\\
283	0.0111235405050835\\
284	0.0111245918449379\\
285	0.0111256632016214\\
286	0.0111267549765721\\
287	0.0111278675793624\\
288	0.0111290014278251\\
289	0.0111301569481742\\
290	0.0111313345751215\\
291	0.0111325347519904\\
292	0.0111337579308241\\
293	0.0111350045724916\\
294	0.0111362751467908\\
295	0.0111375701325493\\
296	0.0111388900177249\\
297	0.0111402352995058\\
298	0.0111416064844155\\
299	0.0111430040884249\\
300	0.0111444286370709\\
301	0.0111458806655778\\
302	0.0111473607189953\\
303	0.0111488693523518\\
304	0.0111504071308217\\
305	0.0111519746299042\\
306	0.0111535724356114\\
307	0.0111552011447466\\
308	0.0111568613652193\\
309	0.0111585537162774\\
310	0.0111602788288254\\
311	0.0111620373457564\\
312	0.0111638299222964\\
313	0.0111656572263608\\
314	0.0111675199389377\\
315	0.0111694187544753\\
316	0.0111713543812801\\
317	0.0111733275419233\\
318	0.0111753389736531\\
319	0.0111773894288088\\
320	0.0111794796752347\\
321	0.0111816104966875\\
322	0.0111837826932345\\
323	0.0111859970816334\\
324	0.0111882544956891\\
325	0.0111905557865738\\
326	0.0111929018231188\\
327	0.0111952934921429\\
328	0.0111977316986042\\
329	0.011200217365753\\
330	0.0112027514352256\\
331	0.0112053348670792\\
332	0.0112079686397508\\
333	0.011210653749863\\
334	0.0112133912119566\\
335	0.0112161820582775\\
336	0.0112190273384082\\
337	0.011221928118835\\
338	0.0112248854824623\\
339	0.0112279005280751\\
340	0.0112309743696732\\
341	0.0112341081354315\\
342	0.0112373029673959\\
343	0.0112405600222574\\
344	0.0112438804701057\\
345	0.0112472654948526\\
346	0.0112507162947592\\
347	0.0112542340830538\\
348	0.0112578200885759\\
349	0.011261475556421\\
350	0.0112652017497522\\
351	0.0112689999506558\\
352	0.0112728714612763\\
353	0.0112768176053963\\
354	0.011280839730262\\
355	0.0112849392070173\\
356	0.011289117431807\\
357	0.0112933758281439\\
358	0.0112977158483011\\
359	0.011302138974555\\
360	0.0113066467201962\\
361	0.0113112406302194\\
362	0.0113159222815943\\
363	0.0113206932830229\\
364	0.0113255552741194\\
365	0.0113305099239933\\
366	0.0113355589291006\\
367	0.0113407040083539\\
368	0.0113459468804434\\
369	0.0113512892819949\\
370	0.0113567329308793\\
371	0.0113622795031801\\
372	0.0113679306271417\\
373	0.0113736878855441\\
374	0.011379552794633\\
375	0.0113855267930921\\
376	0.011391611248355\\
377	0.0113978074525361\\
378	0.0114041166193487\\
379	0.0114105398822545\\
380	0.0114170782941147\\
381	0.0114237328286583\\
382	0.0114305043841853\\
383	0.01143739379016\\
384	0.0114444018176207\\
385	0.0114515291910225\\
386	0.0114587765444463\\
387	0.0114661444825072\\
388	0.0114736336782116\\
389	0.0114812448732057\\
390	0.011488978732885\\
391	0.0114968385746228\\
392	0.0115048276910688\\
393	0.0115129493353054\\
394	0.0115212066891198\\
395	0.0115296028262494\\
396	0.0115381406703072\\
397	0.0115468229476236\\
398	0.0115556521373787\\
399	0.0115646304311188\\
400	0.0115737597662731\\
401	0.011583041426709\\
402	0.011592475085277\\
403	0.0116020606864586\\
404	0.0116117969989971\\
405	0.0116216851604882\\
406	0.0116317236324177\\
407	0.0116419714326053\\
408	0.0116529415657599\\
409	0.011663871589142\\
410	0.0116747530579822\\
411	0.011685577103292\\
412	0.0116963344439236\\
413	0.0117070154585782\\
414	0.0117176098923509\\
415	0.0117281068040252\\
416	0.0117384951411801\\
417	0.0117487638955964\\
418	0.0117589025452438\\
419	0.0117688991696902\\
420	0.011778741639107\\
421	0.0117884176539012\\
422	0.0117979148089755\\
423	0.0118071312004235\\
424	0.0118157966930601\\
425	0.0118243385415335\\
426	0.0118327510316807\\
427	0.0118410288040422\\
428	0.0118491669519784\\
429	0.0118571605663134\\
430	0.0118650055916352\\
431	0.0118726985615377\\
432	0.0118802366168283\\
433	0.0118876179274911\\
434	0.0118948421415168\\
435	0.0119019091209185\\
436	0.0119088197806752\\
437	0.0119155761766031\\
438	0.0119219236782942\\
439	0.0119280894434835\\
440	0.011934231756636\\
441	0.0119403532393258\\
442	0.0119464568938057\\
443	0.0119525462418354\\
444	0.0119586256347657\\
445	0.0119646994594469\\
446	0.0119707725565623\\
447	0.0119768502119905\\
448	0.0119829381412316\\
449	0.0119890424683802\\
450	0.0119951696953853\\
451	0.0120013293922907\\
452	0.0120075406688709\\
453	0.0120138060099553\\
454	0.0120201279596771\\
455	0.0120265091121708\\
456	0.0120329520957963\\
457	0.0120394595555845\\
458	0.0120460341339326\\
459	0.0120526784496094\\
460	0.0120593950751\\
461	0.0120661865127087\\
462	0.0120730551695187\\
463	0.0120800033314504\\
464	0.0120870328683587\\
465	0.0120941451355333\\
466	0.0121013414602092\\
467	0.0121086231363118\\
468	0.0121159914193454\\
469	0.0121234475215211\\
470	0.0121309926072284\\
471	0.0121386277889605\\
472	0.0121463541238032\\
473	0.0121541726106076\\
474	0.0121620841879688\\
475	0.0121700897331545\\
476	0.0121781900675709\\
477	0.0121863859817356\\
478	0.0121946782333429\\
479	0.0122030675453886\\
480	0.0122115546043566\\
481	0.0122201400584669\\
482	0.0122288245159795\\
483	0.0122376085435454\\
484	0.0122464926645886\\
485	0.012255477357698\\
486	0.0122645630549977\\
487	0.012273750140458\\
488	0.0122830389475949\\
489	0.0122924297561361\\
490	0.0123019227884933\\
491	0.0123115182060202\\
492	0.0123212161050358\\
493	0.0123310165125896\\
494	0.0123409193819435\\
495	0.0123509245877424\\
496	0.0123610319208435\\
497	0.0123712410827732\\
498	0.012381551679777\\
499	0.012391963216426\\
500	0.0124024750887987\\
501	0.0124130865772172\\
502	0.0124237968385022\\
503	0.0124346048977139\\
504	0.0124455096393396\\
505	0.0124565097978854\\
506	0.0124676039478301\\
507	0.0124787904928908\\
508	0.0124900676545509\\
509	0.0125014334597928\\
510	0.0125128857279764\\
511	0.0125244220567965\\
512	0.0125360398072444\\
513	0.0125477360874936\\
514	0.0125595077356205\\
515	0.0125713513010625\\
516	0.0125832630247094\\
517	0.0125952388175095\\
518	0.0126072742374641\\
519	0.0126193644648704\\
520	0.0126315042756588\\
521	0.0126436880126554\\
522	0.0126559095545852\\
523	0.0126681622826116\\
524	0.0126804390441894\\
525	0.0126927321139865\\
526	0.0127050331516083\\
527	0.0127173331558066\\
528	0.012729622414824\\
529	0.012741890452518\\
530	0.0127541259712878\\
531	0.0127663167909335\\
532	0.0127784497822529\\
533	0.0127905107251183\\
534	0.0128024841908161\\
535	0.0128143534429356\\
536	0.0128261034464043\\
537	0.0128377199325822\\
538	0.0128497704092146\\
539	0.0128624028130973\\
540	0.0128746190719394\\
541	0.0128864437642305\\
542	0.0128969698322725\\
543	0.0129074152513112\\
544	0.0129179189471904\\
545	0.0129281262494209\\
546	0.0129381782122842\\
547	0.0129483921258864\\
548	0.0129587224354308\\
549	0.0129692334624215\\
550	0.0129800122113186\\
551	0.012991750213386\\
552	0.0130038236646002\\
553	0.0130179787643114\\
554	0.0130320825458009\\
555	0.0130461072474813\\
556	0.0130586243535662\\
557	0.0130712246843271\\
558	0.0130837088663739\\
559	0.0130954305800795\\
560	0.0131075388420725\\
561	0.0131192453219686\\
562	0.0131306904565711\\
563	0.0131418981867739\\
564	0.0131530211332214\\
565	0.0131641860434995\\
566	0.0131754046757112\\
567	0.0131866740841253\\
568	0.0131983203629706\\
569	0.0132097122331134\\
570	0.0132208481972837\\
571	0.0132317968651077\\
572	0.0132427656992653\\
573	0.0132537642725711\\
574	0.0132647838367736\\
575	0.0132758136396235\\
576	0.0132868417600207\\
577	0.0132978550775063\\
578	0.0133088391683302\\
579	0.0133197781864943\\
580	0.0133306547289956\\
581	0.0133414496845994\\
582	0.0133521420653068\\
583	0.0133627088190544\\
584	0.0133731246234883\\
585	0.0133833616664273\\
586	0.013393389433806\\
587	0.0134031745674478\\
588	0.0134126809661912\\
589	0.0134218705998152\\
590	0.0134307062886727\\
591	0.0134391597728698\\
592	0.0134472338645631\\
593	0.0134551345125145\\
594	0.0134637112677116\\
595	0.0134740407759367\\
596	0.0134889777516421\\
597	0.0135160557778789\\
598	0.0135751148335434\\
599	0\\
600	0\\
};
\addplot [color=black!20!mycolor21,solid,forget plot]
  table[row sep=crcr]{%
1	0.0110657835302299\\
2	0.0110657904019744\\
3	0.0110657974249927\\
4	0.0110658046025324\\
5	0.0110658119379086\\
6	0.0110658194345046\\
7	0.0110658270957733\\
8	0.011065834925239\\
9	0.0110658429264982\\
10	0.0110658511032214\\
11	0.0110658594591541\\
12	0.0110658679981189\\
13	0.0110658767240162\\
14	0.0110658856408263\\
15	0.0110658947526105\\
16	0.011065904063513\\
17	0.011065913577762\\
18	0.0110659232996719\\
19	0.0110659332336443\\
20	0.01106594338417\\
21	0.0110659537558307\\
22	0.0110659643533006\\
23	0.0110659751813478\\
24	0.0110659862448368\\
25	0.0110659975487296\\
26	0.0110660090980877\\
27	0.0110660208980741\\
28	0.011066032953955\\
29	0.0110660452711019\\
30	0.0110660578549929\\
31	0.0110660707112157\\
32	0.0110660838454685\\
33	0.0110660972635628\\
34	0.011066110971425\\
35	0.0110661249750985\\
36	0.0110661392807461\\
37	0.0110661538946518\\
38	0.0110661688232231\\
39	0.0110661840729933\\
40	0.0110661996506234\\
41	0.0110662155629046\\
42	0.0110662318167607\\
43	0.01106624841925\\
44	0.0110662653775681\\
45	0.0110662826990501\\
46	0.0110663003911726\\
47	0.0110663184615572\\
48	0.0110663369179716\\
49	0.0110663557683333\\
50	0.0110663750207115\\
51	0.0110663946833298\\
52	0.0110664147645689\\
53	0.0110664352729691\\
54	0.0110664562172331\\
55	0.0110664776062285\\
56	0.0110664994489909\\
57	0.0110665217547262\\
58	0.0110665445328139\\
59	0.0110665677928095\\
60	0.0110665915444474\\
61	0.0110666157976443\\
62	0.0110666405625014\\
63	0.0110666658493079\\
64	0.0110666916685437\\
65	0.0110667180308827\\
66	0.0110667449471953\\
67	0.0110667724285524\\
68	0.0110668004862274\\
69	0.0110668291317003\\
70	0.0110668583766603\\
71	0.0110668882330092\\
72	0.0110669187128647\\
73	0.0110669498285635\\
74	0.0110669815926646\\
75	0.0110670140179529\\
76	0.0110670471174419\\
77	0.0110670809043779\\
78	0.0110671153922428\\
79	0.0110671505947576\\
80	0.0110671865258861\\
81	0.011067223199838\\
82	0.0110672606310728\\
83	0.0110672988343031\\
84	0.011067337824498\\
85	0.0110673776168868\\
86	0.0110674182269626\\
87	0.0110674596704861\\
88	0.0110675019634885\\
89	0.0110675451222761\\
90	0.0110675891634331\\
91	0.0110676341038258\\
92	0.011067679960606\\
93	0.0110677267512149\\
94	0.0110677744933866\\
95	0.0110678232051519\\
96	0.0110678729048421\\
97	0.0110679236110925\\
98	0.0110679753428464\\
99	0.0110680281193587\\
100	0.0110680819601998\\
101	0.011068136885259\\
102	0.0110681929147488\\
103	0.011068250069208\\
104	0.0110683083695062\\
105	0.0110683678368469\\
106	0.0110684284927715\\
107	0.0110684903591632\\
108	0.0110685534582507\\
109	0.0110686178126115\\
110	0.0110686834451763\\
111	0.0110687503792321\\
112	0.0110688186384266\\
113	0.011068888246771\\
114	0.0110689592286446\\
115	0.0110690316087977\\
116	0.0110691054123557\\
117	0.0110691806648227\\
118	0.0110692573920848\\
119	0.0110693356204142\\
120	0.0110694153764723\\
121	0.0110694966873134\\
122	0.0110695795803883\\
123	0.0110696640835479\\
124	0.0110697502250465\\
125	0.0110698380335452\\
126	0.0110699275381154\\
127	0.0110700187682425\\
128	0.0110701117538289\\
129	0.0110702065251973\\
130	0.0110703031130946\\
131	0.0110704015486946\\
132	0.0110705018636015\\
133	0.0110706040898532\\
134	0.0110707082599248\\
135	0.0110708144067311\\
136	0.0110709225636307\\
137	0.0110710327644284\\
138	0.011071145043379\\
139	0.01107125943519\\
140	0.0110713759750252\\
141	0.0110714946985073\\
142	0.0110716156417217\\
143	0.0110717388412192\\
144	0.0110718643340192\\
145	0.0110719921576132\\
146	0.0110721223499679\\
147	0.0110722549495281\\
148	0.0110723899952204\\
149	0.0110725275264565\\
150	0.0110726675831363\\
151	0.0110728102056517\\
152	0.0110729554348896\\
153	0.0110731033122362\\
154	0.0110732538795801\\
155	0.0110734071793167\\
156	0.0110735632543514\\
157	0.0110737221481043\\
158	0.0110738839045144\\
159	0.0110740485680435\\
160	0.0110742161836816\\
161	0.0110743867969508\\
162	0.0110745604539115\\
163	0.0110747372011664\\
164	0.0110749170858674\\
165	0.0110751001557205\\
166	0.0110752864589926\\
167	0.0110754760445181\\
168	0.0110756689617054\\
169	0.011075865260545\\
170	0.0110760649916171\\
171	0.0110762682060999\\
172	0.0110764749557784\\
173	0.0110766852930544\\
174	0.0110768992709562\\
175	0.0110771169431494\\
176	0.0110773383639485\\
177	0.0110775635883289\\
178	0.0110777926719402\\
179	0.01107802567112\\
180	0.0110782626429084\\
181	0.0110785036450641\\
182	0.0110787487360809\\
183	0.0110789979752056\\
184	0.0110792514224572\\
185	0.0110795091386467\\
186	0.0110797711853988\\
187	0.0110800376251747\\
188	0.0110803085212961\\
189	0.0110805839379712\\
190	0.0110808639403217\\
191	0.0110811485944119\\
192	0.0110814379672791\\
193	0.0110817321269661\\
194	0.0110820311425552\\
195	0.011082335084205\\
196	0.0110826440231877\\
197	0.0110829580319303\\
198	0.0110832771840566\\
199	0.0110836015544319\\
200	0.0110839312192105\\
201	0.0110842662558848\\
202	0.0110846067433374\\
203	0.0110849527618957\\
204	0.011085304393389\\
205	0.0110856617212082\\
206	0.0110860248303686\\
207	0.011086393807575\\
208	0.0110867687412897\\
209	0.011087149721804\\
210	0.0110875368413115\\
211	0.0110879301939853\\
212	0.0110883298760574\\
213	0.0110887359859012\\
214	0.0110891486241169\\
215	0.0110895678936199\\
216	0.0110899938997312\\
217	0.0110904267502718\\
218	0.0110908665556579\\
219	0.0110913134290005\\
220	0.0110917674862056\\
221	0.0110922288460778\\
222	0.0110926976304249\\
223	0.0110931739641652\\
224	0.0110936579754354\\
225	0.0110941497957005\\
226	0.0110946495598638\\
227	0.0110951574063784\\
228	0.0110956734773583\\
229	0.0110961979186895\\
230	0.0110967308801408\\
231	0.011097272515473\\
232	0.0110978229825476\\
233	0.0110983824434326\\
234	0.0110989510645061\\
235	0.0110995290165575\\
236	0.0111001164748847\\
237	0.0111007136193872\\
238	0.0111013206346553\\
239	0.0111019377100537\\
240	0.0111025650397997\\
241	0.0111032028230365\\
242	0.0111038512638985\\
243	0.0111045105715719\\
244	0.0111051809603468\\
245	0.0111058626496639\\
246	0.0111065558641533\\
247	0.0111072608336665\\
248	0.0111079777933021\\
249	0.0111087069834253\\
250	0.0111094486496812\\
251	0.0111102030430039\\
252	0.0111109704196209\\
253	0.0111117510410548\\
254	0.0111125451741238\\
255	0.0111133530909416\\
256	0.0111141750689186\\
257	0.0111150113907673\\
258	0.0111158623445123\\
259	0.0111167282235074\\
260	0.0111176093264602\\
261	0.0111185059574632\\
262	0.0111194184260277\\
263	0.0111203470471425\\
264	0.0111212921414419\\
265	0.0111222540352322\\
266	0.0111232330606059\\
267	0.0111242295555637\\
268	0.0111252438641468\\
269	0.0111262763365798\\
270	0.0111273273294237\\
271	0.0111283972057422\\
272	0.0111294863352793\\
273	0.0111305950946505\\
274	0.0111317238675473\\
275	0.0111328730449559\\
276	0.0111340430253903\\
277	0.0111352342151391\\
278	0.0111364470285273\\
279	0.011137681888192\\
280	0.0111389392253713\\
281	0.011140219480206\\
282	0.0111415231020534\\
283	0.0111428505498099\\
284	0.0111442022922433\\
285	0.01114557880833\\
286	0.0111469805875953\\
287	0.011148408130454\\
288	0.0111498619485467\\
289	0.011151342565068\\
290	0.0111528505150822\\
291	0.0111543863458202\\
292	0.0111559506169531\\
293	0.0111575439008362\\
294	0.0111591667827166\\
295	0.0111608198608988\\
296	0.0111625037468601\\
297	0.0111642190653089\\
298	0.0111659664541756\\
299	0.0111677465645368\\
300	0.0111695600605198\\
301	0.0111714076191318\\
302	0.0111732899299379\\
303	0.0111752076947216\\
304	0.0111771616270852\\
305	0.0111791524519845\\
306	0.0111811809051314\\
307	0.0111832477320961\\
308	0.0111853536880729\\
309	0.0111874995379867\\
310	0.0111896860554923\\
311	0.0111919140229662\\
312	0.0111941842315383\\
313	0.0111964974811282\\
314	0.0111988545804945\\
315	0.0112012563475014\\
316	0.0112037036093745\\
317	0.0112061972030382\\
318	0.0112087379755431\\
319	0.011211326784593\\
320	0.0112139644991812\\
321	0.0112166520003437\\
322	0.0112193901820381\\
323	0.0112221799521535\\
324	0.0112250222336518\\
325	0.0112279179658189\\
326	0.0112308681055581\\
327	0.0112338736287127\\
328	0.0112369355326493\\
329	0.0112400548374137\\
330	0.0112432325875771\\
331	0.0112464698541679\\
332	0.0112497677367301\\
333	0.0112531273654906\\
334	0.0112565499025665\\
335	0.0112600365427887\\
336	0.0112635885159453\\
337	0.011267207087792\\
338	0.0112708935606567\\
339	0.0112746492735851\\
340	0.0112784756019556\\
341	0.0112823739558984\\
342	0.0112863457721734\\
343	0.0112903925107825\\
344	0.0112945156569243\\
345	0.0112987166843379\\
346	0.0113029970509625\\
347	0.0113073581944529\\
348	0.0113118015274859\\
349	0.0113163284321257\\
350	0.0113209402504951\\
351	0.0113256382897553\\
352	0.0113304238146228\\
353	0.0113352980412534\\
354	0.0113402621351728\\
355	0.0113453172148611\\
356	0.0113504643380515\\
357	0.0113557044934944\\
358	0.0113610386106566\\
359	0.0113664675606969\\
360	0.0113719921590708\\
361	0.011377613170022\\
362	0.011383331313227\\
363	0.011389147272878\\
364	0.0113950617095649\\
365	0.0114010752756061\\
366	0.0114071886354735\\
367	0.0114134024959216\\
368	0.0114197176532242\\
369	0.0114261347676039\\
370	0.0114326562650147\\
371	0.0114392851397778\\
372	0.0114460244396681\\
373	0.0114528772342998\\
374	0.0114598467309583\\
375	0.0114669360862382\\
376	0.011474148285399\\
377	0.0114814862406787\\
378	0.0114889527566934\\
379	0.0114965504907627\\
380	0.0115042819078056\\
381	0.0115121492295691\\
382	0.0115201543782327\\
383	0.0115282989152114\\
384	0.0115365839791378\\
385	0.0115450102407521\\
386	0.0115535779683069\\
387	0.0115622861432367\\
388	0.011571132674777\\
389	0.0115801153282833\\
390	0.0115897596702056\\
391	0.0115994060521122\\
392	0.0116090309193411\\
393	0.0116186274222215\\
394	0.0116281883371054\\
395	0.011637706061919\\
396	0.0116471726146568\\
397	0.0116565796355751\\
398	0.0116659183946249\\
399	0.0116751798098652\\
400	0.0116843545119192\\
401	0.0116934327385675\\
402	0.0117024039761608\\
403	0.0117112578591969\\
404	0.0117199835681743\\
405	0.0117285714708566\\
406	0.0117370109778638\\
407	0.0117452544057623\\
408	0.0117529836034152\\
409	0.0117606101620528\\
410	0.0117681286223245\\
411	0.0117755337136377\\
412	0.0117828203989786\\
413	0.0117899839464346\\
414	0.011797019990035\\
415	0.0118039246351518\\
416	0.0118106944551571\\
417	0.0118173265812444\\
418	0.0118238187820556\\
419	0.0118301694881771\\
420	0.0118363777579459\\
421	0.0118424436889591\\
422	0.011848368232014\\
423	0.011854088946802\\
424	0.011859430961938\\
425	0.0118647458904544\\
426	0.011870035512057\\
427	0.0118753019859474\\
428	0.0118805478854587\\
429	0.0118857759896692\\
430	0.0118909896248554\\
431	0.0118961925086493\\
432	0.0119013887135043\\
433	0.0119065828043419\\
434	0.0119117800054663\\
435	0.0119169855769894\\
436	0.011922205129509\\
437	0.0119274445946663\\
438	0.0119327195633107\\
439	0.0119380378591766\\
440	0.0119434017471379\\
441	0.0119488135523095\\
442	0.0119542756474879\\
443	0.0119597904378146\\
444	0.0119653603445436\\
445	0.0119709877939126\\
446	0.0119766751991778\\
447	0.0119824249413178\\
448	0.0119882393485565\\
449	0.0119941206748065\\
450	0.0120000710772602\\
451	0.0120060924897895\\
452	0.0120121862414342\\
453	0.0120183536461105\\
454	0.0120245959985104\\
455	0.0120309145699489\\
456	0.0120373106043922\\
457	0.0120437853147501\\
458	0.0120503398795261\\
459	0.0120569754399226\\
460	0.0120636930975101\\
461	0.0120704939125621\\
462	0.0120773789031652\\
463	0.0120843490452153\\
464	0.0120914052837965\\
465	0.012098548549483\\
466	0.0121057797573558\\
467	0.0121130998061113\\
468	0.012120509577265\\
469	0.012128009934452\\
470	0.0121356017228233\\
471	0.0121432857685314\\
472	0.0121510628782945\\
473	0.0121589338390237\\
474	0.0121668994174921\\
475	0.0121749603600154\\
476	0.0121831173918932\\
477	0.0121913712158188\\
478	0.0121997225101997\\
479	0.0122081719273813\\
480	0.012216720091759\\
481	0.0122253675977668\\
482	0.0122341150077287\\
483	0.0122429628495551\\
484	0.0122519116142696\\
485	0.0122609617533462\\
486	0.0122701136758401\\
487	0.0122793677452894\\
488	0.0122887242763912\\
489	0.0122981835314651\\
490	0.012307745716693\\
491	0.0123174109781171\\
492	0.0123271793973813\\
493	0.0123370509871967\\
494	0.0123470256865151\\
495	0.0123571033553883\\
496	0.0123672837694935\\
497	0.0123775666143034\\
498	0.0123879514788759\\
499	0.0123984378492395\\
500	0.0124090251013459\\
501	0.0124197124935579\\
502	0.0124304991586391\\
503	0.0124413840952073\\
504	0.0124523661586138\\
505	0.0124634440512024\\
506	0.0124746163119035\\
507	0.012485881305109\\
508	0.0124972372087743\\
509	0.0125086820016835\\
510	0.0125202134498133\\
511	0.0125318290917197\\
512	0.0125435262228691\\
513	0.0125553018788246\\
514	0.0125671528171925\\
515	0.012579075498223\\
516	0.0125910660639501\\
517	0.0126031203157439\\
518	0.0126152336901377\\
519	0.0126274012327765\\
520	0.012639617570322\\
521	0.0126518768801308\\
522	0.0126641728575067\\
523	0.0126764986803076\\
524	0.01268884697067\\
525	0.0127012097535404\\
526	0.0127135784117239\\
527	0.0127259436381541\\
528	0.0127382953853961\\
529	0.0127506228108894\\
530	0.0127629141524267\\
531	0.0127751566183118\\
532	0.0127873363026626\\
533	0.0127994416148953\\
534	0.0128114616770597\\
535	0.0128242706075797\\
536	0.012837380131529\\
537	0.0128501705703827\\
538	0.0128622857200498\\
539	0.0128734910425667\\
540	0.0128842944965308\\
541	0.0128945473872983\\
542	0.0129039241484465\\
543	0.0129134592877277\\
544	0.0129231240488488\\
545	0.0129329412547109\\
546	0.0129429458595549\\
547	0.0129531340968814\\
548	0.0129638050212203\\
549	0.012975202483788\\
550	0.0129892054093718\\
551	0.013002956613446\\
552	0.0130167974146248\\
553	0.0130290441704488\\
554	0.0130412448851652\\
555	0.0130532261696414\\
556	0.0130642989858724\\
557	0.0130755113207519\\
558	0.013087286143706\\
559	0.0130988443324463\\
560	0.0131100721444647\\
561	0.0131210855541065\\
562	0.0131320001699415\\
563	0.013142969346439\\
564	0.0131540054757782\\
565	0.0131651052578191\\
566	0.0131763235313981\\
567	0.0131878715652356\\
568	0.0131990279449963\\
569	0.0132099881684336\\
570	0.0132208706968361\\
571	0.013231798444302\\
572	0.0132427659177417\\
573	0.0132537643358688\\
574	0.0132647838624601\\
575	0.0132758136515379\\
576	0.0132868417664646\\
577	0.0132978550811885\\
578	0.0133088391704532\\
579	0.0133197781877949\\
580	0.0133306547298284\\
581	0.0133414496851007\\
582	0.0133521420655792\\
583	0.013362708819187\\
584	0.0133731246235443\\
585	0.0133833616664418\\
586	0.013393389433806\\
587	0.0134031745674478\\
588	0.0134126809661912\\
589	0.0134218705998152\\
590	0.0134307062886727\\
591	0.0134391597728698\\
592	0.0134472338645631\\
593	0.0134551345125145\\
594	0.0134637112677116\\
595	0.0134740407759367\\
596	0.0134889777516421\\
597	0.0135160557778789\\
598	0.0135751148335434\\
599	0\\
600	0\\
};
\addplot [color=black!50!mycolor20,solid,forget plot]
  table[row sep=crcr]{%
1	0.011065821102699\\
2	0.0110658289648861\\
3	0.0110658370057177\\
4	0.0110658452291939\\
5	0.0110658536394024\\
6	0.0110658622405204\\
7	0.0110658710368164\\
8	0.011065880032652\\
9	0.0110658892324841\\
10	0.0110658986408668\\
11	0.0110659082624533\\
12	0.0110659181019983\\
13	0.0110659281643598\\
14	0.0110659384545014\\
15	0.0110659489774944\\
16	0.0110659597385203\\
17	0.0110659707428727\\
18	0.01106598199596\\
19	0.0110659935033077\\
20	0.0110660052705604\\
21	0.0110660173034849\\
22	0.0110660296079724\\
23	0.0110660421900408\\
24	0.011066055055838\\
25	0.0110660682116438\\
26	0.0110660816638731\\
27	0.0110660954190784\\
28	0.011066109483953\\
29	0.0110661238653332\\
30	0.0110661385702018\\
31	0.0110661536056909\\
32	0.0110661689790847\\
33	0.011066184697823\\
34	0.0110662007695039\\
35	0.0110662172018872\\
36	0.0110662340028976\\
37	0.0110662511806283\\
38	0.0110662687433437\\
39	0.0110662866994835\\
40	0.0110663050576658\\
41	0.0110663238266906\\
42	0.0110663430155439\\
43	0.0110663626334008\\
44	0.0110663826896293\\
45	0.0110664031937946\\
46	0.0110664241556623\\
47	0.0110664455852029\\
48	0.0110664674925953\\
49	0.0110664898882313\\
50	0.0110665127827193\\
51	0.011066536186889\\
52	0.011066560111795\\
53	0.011066584568722\\
54	0.0110666095691883\\
55	0.011066635124951\\
56	0.01106666124801\\
57	0.0110666879506131\\
58	0.0110667152452604\\
59	0.0110667431447091\\
60	0.0110667716619784\\
61	0.0110668008103544\\
62	0.0110668306033951\\
63	0.0110668610549354\\
64	0.0110668921790924\\
65	0.0110669239902705\\
66	0.0110669565031665\\
67	0.0110669897327755\\
68	0.0110670236943958\\
69	0.0110670584036347\\
70	0.0110670938764143\\
71	0.0110671301289765\\
72	0.0110671671778895\\
73	0.0110672050400534\\
74	0.0110672437327058\\
75	0.0110672832734283\\
76	0.0110673236801521\\
77	0.0110673649711645\\
78	0.0110674071651149\\
79	0.0110674502810212\\
80	0.0110674943382764\\
81	0.0110675393566545\\
82	0.0110675853563174\\
83	0.0110676323578216\\
84	0.0110676803821247\\
85	0.0110677294505921\\
86	0.011067779585004\\
87	0.0110678308075621\\
88	0.0110678831408967\\
89	0.0110679366080738\\
90	0.0110679912326019\\
91	0.0110680470384397\\
92	0.0110681040500027\\
93	0.0110681622921709\\
94	0.011068221790296\\
95	0.0110682825702092\\
96	0.0110683446582278\\
97	0.0110684080811638\\
98	0.0110684728663307\\
99	0.0110685390415514\\
100	0.011068606635166\\
101	0.0110686756760393\\
102	0.0110687461935687\\
103	0.011068818217692\\
104	0.0110688917788952\\
105	0.0110689669082204\\
106	0.0110690436372737\\
107	0.0110691219982331\\
108	0.0110692020238565\\
109	0.0110692837474896\\
110	0.0110693672030739\\
111	0.0110694524251547\\
112	0.011069539448889\\
113	0.0110696283100536\\
114	0.011069719045053\\
115	0.0110698116909273\\
116	0.0110699062853603\\
117	0.011070002866687\\
118	0.0110701014739023\\
119	0.0110702021466677\\
120	0.01107030492532\\
121	0.0110704098508787\\
122	0.0110705169650537\\
123	0.0110706263102526\\
124	0.0110707379295889\\
125	0.0110708518668887\\
126	0.0110709681666985\\
127	0.0110710868742924\\
128	0.0110712080356789\\
129	0.0110713316976082\\
130	0.0110714579075792\\
131	0.0110715867138459\\
132	0.0110717181654239\\
133	0.0110718523120973\\
134	0.0110719892044245\\
135	0.0110721288937444\\
136	0.0110722714321824\\
137	0.0110724168726557\\
138	0.0110725652688791\\
139	0.0110727166753699\\
140	0.0110728711474532\\
141	0.0110730287412661\\
142	0.0110731895137625\\
143	0.0110733535227174\\
144	0.01107352082673\\
145	0.0110736914852282\\
146	0.011073865558471\\
147	0.0110740431075521\\
148	0.0110742241944016\\
149	0.0110744088817891\\
150	0.0110745972333247\\
151	0.0110747893134609\\
152	0.0110749851874935\\
153	0.0110751849215619\\
154	0.0110753885826497\\
155	0.0110755962385844\\
156	0.0110758079580363\\
157	0.0110760238105175\\
158	0.0110762438663807\\
159	0.0110764681968163\\
160	0.0110766968738505\\
161	0.0110769299703415\\
162	0.0110771675599767\\
163	0.0110774097172679\\
164	0.0110776565175468\\
165	0.0110779080369602\\
166	0.011078164352464\\
167	0.0110784255418171\\
168	0.0110786916835746\\
169	0.011078962857081\\
170	0.0110792391424619\\
171	0.0110795206206163\\
172	0.0110798073732074\\
173	0.0110800994826542\\
174	0.0110803970321212\\
175	0.0110807001055086\\
176	0.0110810087874421\\
177	0.0110813231632619\\
178	0.0110816433190119\\
179	0.0110819693414277\\
180	0.0110823013179257\\
181	0.0110826393365908\\
182	0.0110829834861647\\
183	0.0110833338560337\\
184	0.0110836905362168\\
185	0.0110840536173539\\
186	0.0110844231906939\\
187	0.0110847993480832\\
188	0.0110851821819549\\
189	0.0110855717853179\\
190	0.0110859682517471\\
191	0.0110863716753741\\
192	0.011086782150879\\
193	0.0110871997734831\\
194	0.0110876246389431\\
195	0.0110880568435463\\
196	0.0110884964841078\\
197	0.0110889436579694\\
198	0.0110893984630008\\
199	0.0110898609976025\\
200	0.0110903313607123\\
201	0.0110908096518137\\
202	0.011091295970948\\
203	0.0110917904187298\\
204	0.0110922930963662\\
205	0.0110928041056802\\
206	0.0110933235491388\\
207	0.0110938515298859\\
208	0.0110943881517803\\
209	0.0110949335194399\\
210	0.0110954877382915\\
211	0.0110960509146281\\
212	0.0110966231556722\\
213	0.0110972045696477\\
214	0.0110977952658588\\
215	0.0110983953547788\\
216	0.0110990049481464\\
217	0.011099624159073\\
218	0.011100253102159\\
219	0.0111008918936211\\
220	0.0111015406514306\\
221	0.0111021994954631\\
222	0.0111028685476601\\
223	0.0111035479322029\\
224	0.0111042377756991\\
225	0.0111049382073823\\
226	0.0111056493593247\\
227	0.0111063713666632\\
228	0.0111071043678387\\
229	0.0111078485048495\\
230	0.011108603923517\\
231	0.0111093707737648\\
232	0.0111101492099112\\
233	0.0111109393909719\\
234	0.0111117414809754\\
235	0.0111125556492884\\
236	0.0111133820709493\\
237	0.0111142209270105\\
238	0.0111150724048858\\
239	0.0111159366987018\\
240	0.0111168140096519\\
241	0.0111177045463478\\
242	0.0111186085251691\\
243	0.011119526170605\\
244	0.011120457715586\\
245	0.0111214034018017\\
246	0.0111223634800016\\
247	0.0111233382102719\\
248	0.0111243278622881\\
249	0.0111253327155356\\
250	0.0111263530594949\\
251	0.0111273891937881\\
252	0.0111284414282803\\
253	0.0111295100831344\\
254	0.0111305954888137\\
255	0.0111316979860313\\
256	0.0111328179256436\\
257	0.0111339556684872\\
258	0.0111351115851612\\
259	0.0111362860557569\\
260	0.0111374794695377\\
261	0.0111386922245708\\
262	0.011139924727283\\
263	0.0111411773918651\\
264	0.0111424506395847\\
265	0.0111437448994041\\
266	0.0111450606068011\\
267	0.0111463982036485\\
268	0.0111477581380904\\
269	0.0111491408644129\\
270	0.0111505468429148\\
271	0.0111519765397796\\
272	0.0111534304269509\\
273	0.0111549089820174\\
274	0.0111564126881083\\
275	0.0111579420338064\\
276	0.0111594975130823\\
277	0.0111610796252556\\
278	0.0111626888749901\\
279	0.0111643257723283\\
280	0.0111659908327733\\
281	0.0111676845774264\\
282	0.0111694075331867\\
283	0.0111711602330231\\
284	0.0111729432163271\\
285	0.011174757029354\\
286	0.0111766022257638\\
287	0.0111784793672676\\
288	0.0111803890243907\\
289	0.0111823317773566\\
290	0.0111843082171011\\
291	0.0111863189464177\\
292	0.0111883645812396\\
293	0.0111904457520563\\
294	0.0111925631054612\\
295	0.011194717305824\\
296	0.0111969090370743\\
297	0.0111991390045748\\
298	0.0112014079370436\\
299	0.0112037165884391\\
300	0.0112060657397183\\
301	0.0112084562011035\\
302	0.0112108888143723\\
303	0.0112133644539645\\
304	0.0112158840284228\\
305	0.0112184484815796\\
306	0.0112210587934267\\
307	0.0112237159800599\\
308	0.0112264210887\\
309	0.0112291751987576\\
310	0.0112319794208268\\
311	0.011234834875442\\
312	0.011237742691343\\
313	0.0112407040036286\\
314	0.0112437199513603\\
315	0.0112467916741458\\
316	0.011249920310454\\
317	0.0112531069948132\\
318	0.0112563528548289\\
319	0.0112596590080321\\
320	0.0112630265585776\\
321	0.0112664565938174\\
322	0.0112699501807839\\
323	0.0112735083626306\\
324	0.0112771321550935\\
325	0.0112808225430402\\
326	0.0112845804770656\\
327	0.0112884068694165\\
328	0.0112923025866838\\
329	0.0112962684592011\\
330	0.0113003052724157\\
331	0.0113044137657668\\
332	0.0113085946326452\\
333	0.0113128485225369\\
334	0.0113171760480994\\
335	0.0113215777843965\\
336	0.0113260542637323\\
337	0.0113306059934804\\
338	0.0113352334659643\\
339	0.0113399371709906\\
340	0.0113447176118106\\
341	0.0113495753265301\\
342	0.0113545109191944\\
343	0.011359525017863\\
344	0.0113646187507557\\
345	0.0113697944840235\\
346	0.0113750546600878\\
347	0.011380401794997\\
348	0.0113858384751972\\
349	0.011391367354125\\
350	0.0113969911446045\\
351	0.0114027125351027\\
352	0.0114085342971328\\
353	0.0114144592426763\\
354	0.0114204901818168\\
355	0.0114266299085963\\
356	0.011432881283791\\
357	0.0114392470923177\\
358	0.0114457299089051\\
359	0.0114523321789638\\
360	0.0114590561790719\\
361	0.0114659039726035\\
362	0.0114728773604104\\
363	0.0114799778266626\\
364	0.0114872064802676\\
365	0.0114945639930024\\
366	0.0115020505375361\\
367	0.0115096657364463\\
368	0.011517408675879\\
369	0.0115256221280624\\
370	0.0115339508772211\\
371	0.0115422821613925\\
372	0.0115506110873121\\
373	0.0115589326462482\\
374	0.0115672412416417\\
375	0.011575532689852\\
376	0.011583802533005\\
377	0.0115920445535461\\
378	0.011600252206247\\
379	0.0116084186177617\\
380	0.0116165365895334\\
381	0.0116245986049176\\
382	0.0116325968304002\\
383	0.0116405231382912\\
384	0.011648369128459\\
385	0.0116561261640855\\
386	0.0116637854724604\\
387	0.0116713378327485\\
388	0.0116787737305949\\
389	0.011686083760333\\
390	0.0116929466123258\\
391	0.0116997204530993\\
392	0.0117064106890828\\
393	0.0117130123777576\\
394	0.0117195206850028\\
395	0.0117259309243712\\
396	0.0117322386001816\\
397	0.0117384394544537\\
398	0.0117445295176025\\
399	0.0117505051625792\\
400	0.0117563631614718\\
401	0.0117621007505314\\
402	0.0117677157086127\\
403	0.0117732064075792\\
404	0.0117785718810946\\
405	0.0117838118399652\\
406	0.0117889267467188\\
407	0.0117938913382587\\
408	0.0117984888910142\\
409	0.0118030574223082\\
410	0.011807598025411\\
411	0.0118121120935789\\
412	0.0118166013245636\\
413	0.0118210676024401\\
414	0.0118255132532196\\
415	0.0118299408703243\\
416	0.0118343539043087\\
417	0.0118387557791577\\
418	0.0118431502491407\\
419	0.011847541390846\\
420	0.0118519335374563\\
421	0.0118563314129836\\
422	0.0118607399947649\\
423	0.0118651669656371\\
424	0.0118696268407276\\
425	0.0118741215359956\\
426	0.0118786530357077\\
427	0.0118832233840815\\
428	0.011887834675857\\
429	0.0118924890462112\\
430	0.0118971886563122\\
431	0.0119019356793648\\
432	0.0119067322850935\\
433	0.0119115806219581\\
434	0.0119164827994091\\
435	0.0119214408741156\\
436	0.011926456831081\\
437	0.0119315325641222\\
438	0.011936669504429\\
439	0.0119418688790101\\
440	0.0119471319053326\\
441	0.0119524597878467\\
442	0.0119578537146328\\
443	0.0119633148542813\\
444	0.0119688443530573\\
445	0.0119744433322059\\
446	0.0119801128856566\\
447	0.0119858540782158\\
448	0.0119916679443386\\
449	0.0119975554875762\\
450	0.0120035176807954\\
451	0.0120095554712216\\
452	0.0120156698005237\\
453	0.0120218616042862\\
454	0.0120281318115697\\
455	0.0120344813445731\\
456	0.0120409111184003\\
457	0.0120474220409336\\
458	0.012054015012811\\
459	0.0120606909275037\\
460	0.012067450671483\\
461	0.0120742951244625\\
462	0.0120812251596977\\
463	0.0120882416443153\\
464	0.0120953454392403\\
465	0.0121025373984851\\
466	0.0121098183683999\\
467	0.0121171891868797\\
468	0.0121246506825181\\
469	0.0121322036737005\\
470	0.0121398489676272\\
471	0.0121475873592562\\
472	0.0121554196301551\\
473	0.0121633465472496\\
474	0.0121713688614584\\
475	0.0121794873061998\\
476	0.0121877025957688\\
477	0.0121960154236081\\
478	0.0122044264604697\\
479	0.0122129363524563\\
480	0.0122215457189367\\
481	0.0122302551503262\\
482	0.0122390652057242\\
483	0.0122479764103993\\
484	0.0122569892531134\\
485	0.0122661041832746\\
486	0.0122753216079108\\
487	0.0122846418884514\\
488	0.0122940653373078\\
489	0.0123035922142385\\
490	0.0123132227224855\\
491	0.012322957004665\\
492	0.0123327951383984\\
493	0.0123427371316635\\
494	0.0123527829178492\\
495	0.0123629323504907\\
496	0.0123731851976653\\
497	0.012383541136023\\
498	0.0123939997444267\\
499	0.0124045604971727\\
500	0.0124152227567621\\
501	0.0124259857661877\\
502	0.0124368486407018\\
503	0.0124478103590238\\
504	0.012458869753945\\
505	0.0124700255022841\\
506	0.0124812761141412\\
507	0.012492619921394\\
508	0.012504055065377\\
509	0.0125155794836737\\
510	0.0125271908959512\\
511	0.0125388867887563\\
512	0.0125506643991852\\
513	0.0125625206973315\\
514	0.0125744523674068\\
515	0.0125864557874208\\
516	0.0125985270072938\\
517	0.0126106617252646\\
518	0.0126228552624458\\
519	0.0126351025353607\\
520	0.0126473980262867\\
521	0.0126597357512105\\
522	0.012672109225188\\
523	0.0126845114248838\\
524	0.0126969347480496\\
525	0.0127093709717494\\
526	0.0127218112073998\\
527	0.012734245805138\\
528	0.0127466642371578\\
529	0.0127590550256304\\
530	0.0127714089380378\\
531	0.0127837179472196\\
532	0.0127969694239702\\
533	0.0128104550459459\\
534	0.0128236937604145\\
535	0.0128361030880106\\
536	0.0128478459615153\\
537	0.0128592172165552\\
538	0.0128699146974688\\
539	0.0128798914517335\\
540	0.0128895434255866\\
541	0.0128988013504321\\
542	0.0129081214668756\\
543	0.0129176240718101\\
544	0.0129273120545926\\
545	0.0129371900133201\\
546	0.0129476699430569\\
547	0.0129606047228869\\
548	0.0129743833009077\\
549	0.0129880655941136\\
550	0.0130002935798855\\
551	0.0130121056813867\\
552	0.0130237468212249\\
553	0.0130344063417142\\
554	0.0130451432075004\\
555	0.0130558999472609\\
556	0.0130672064350691\\
557	0.0130786964131599\\
558	0.0130897324850921\\
559	0.0131006000416406\\
560	0.013111303177285\\
561	0.0131220669732257\\
562	0.013132907115845\\
563	0.0131438212924558\\
564	0.0131548047831771\\
565	0.0131659765609794\\
566	0.0131773567450786\\
567	0.0131883678346686\\
568	0.0131991666585108\\
569	0.013209991322539\\
570	0.0132208709170359\\
571	0.0132317984756346\\
572	0.0132427659270061\\
573	0.0132537643396685\\
574	0.0132647838642445\\
575	0.0132758136524997\\
576	0.0132868417670135\\
577	0.0132978550815073\\
578	0.0133088391706484\\
579	0.0133197781879177\\
580	0.0133306547299011\\
581	0.0133414496851396\\
582	0.0133521420655977\\
583	0.0133627088191944\\
584	0.0133731246235461\\
585	0.0133833616664418\\
586	0.013393389433806\\
587	0.0134031745674478\\
588	0.0134126809661912\\
589	0.0134218705998152\\
590	0.0134307062886727\\
591	0.0134391597728698\\
592	0.0134472338645631\\
593	0.0134551345125145\\
594	0.0134637112677116\\
595	0.0134740407759367\\
596	0.0134889777516421\\
597	0.0135160557778789\\
598	0.0135751148335434\\
599	0\\
600	0\\
};
\addplot [color=black!60!mycolor21,solid,forget plot]
  table[row sep=crcr]{%
1	0.011065848836443\\
2	0.0110658574824421\\
3	0.0110658663297064\\
4	0.0110658753828921\\
5	0.0110658846467622\\
6	0.0110658941261889\\
7	0.0110659038261558\\
8	0.0110659137517605\\
9	0.0110659239082174\\
10	0.01106593430086\\
11	0.011065944935144\\
12	0.0110659558166494\\
13	0.011065966951084\\
14	0.0110659783442858\\
15	0.0110659900022261\\
16	0.0110660019310124\\
17	0.0110660141368915\\
18	0.0110660266262526\\
19	0.0110660394056307\\
20	0.0110660524817093\\
21	0.0110660658613245\\
22	0.0110660795514675\\
23	0.011066093559289\\
24	0.0110661078921019\\
25	0.0110661225573856\\
26	0.0110661375627892\\
27	0.0110661529161355\\
28	0.011066168625425\\
29	0.0110661846988395\\
30	0.0110662011447463\\
31	0.0110662179717024\\
32	0.0110662351884584\\
33	0.0110662528039632\\
34	0.0110662708273681\\
35	0.0110662892680311\\
36	0.011066308135522\\
37	0.0110663274396265\\
38	0.0110663471903513\\
39	0.0110663673979288\\
40	0.011066388072822\\
41	0.0110664092257299\\
42	0.0110664308675922\\
43	0.0110664530095949\\
44	0.0110664756631754\\
45	0.0110664988400283\\
46	0.0110665225521108\\
47	0.0110665468116486\\
48	0.0110665716311415\\
49	0.0110665970233694\\
50	0.0110666230013985\\
51	0.0110666495785877\\
52	0.0110666767685942\\
53	0.0110667045853807\\
54	0.0110667330432219\\
55	0.0110667621567105\\
56	0.0110667919407652\\
57	0.0110668224106365\\
58	0.0110668535819148\\
59	0.0110668854705371\\
60	0.0110669180927945\\
61	0.0110669514653401\\
62	0.0110669856051962\\
63	0.0110670205297625\\
64	0.011067056256824\\
65	0.0110670928045592\\
66	0.0110671301915484\\
67	0.0110671684367822\\
68	0.0110672075596701\\
69	0.0110672475800496\\
70	0.0110672885181946\\
71	0.0110673303948251\\
72	0.0110673732311163\\
73	0.0110674170487081\\
74	0.0110674618697148\\
75	0.0110675077167351\\
76	0.0110675546128617\\
77	0.011067602581692\\
78	0.0110676516473385\\
79	0.0110677018344388\\
80	0.0110677531681673\\
81	0.0110678056742454\\
82	0.011067859378953\\
83	0.0110679143091399\\
84	0.0110679704922375\\
85	0.0110680279562702\\
86	0.0110680867298677\\
87	0.011068146842277\\
88	0.011068208323375\\
89	0.0110682712036808\\
90	0.011068335514369\\
91	0.0110684012872821\\
92	0.0110684685549444\\
93	0.0110685373505748\\
94	0.011068607708101\\
95	0.0110686796621732\\
96	0.0110687532481782\\
97	0.0110688285022536\\
98	0.0110689054613028\\
99	0.0110689841630091\\
100	0.0110690646458513\\
101	0.0110691469491186\\
102	0.0110692311129261\\
103	0.0110693171782304\\
104	0.0110694051868458\\
105	0.0110694951814599\\
106	0.0110695872056503\\
107	0.0110696813039009\\
108	0.0110697775216188\\
109	0.0110698759051512\\
110	0.0110699765018028\\
111	0.0110700793598526\\
112	0.0110701845285724\\
113	0.0110702920582436\\
114	0.0110704020001761\\
115	0.0110705144067258\\
116	0.0110706293313135\\
117	0.0110707468284432\\
118	0.011070866953721\\
119	0.0110709897638737\\
120	0.0110711153167686\\
121	0.011071243671432\\
122	0.0110713748880691\\
123	0.0110715090280834\\
124	0.0110716461540966\\
125	0.011071786329968\\
126	0.0110719296208149\\
127	0.0110720760930324\\
128	0.0110722258143138\\
129	0.0110723788536705\\
130	0.0110725352814524\\
131	0.0110726951693684\\
132	0.0110728585905068\\
133	0.0110730256193553\\
134	0.0110731963318221\\
135	0.0110733708052554\\
136	0.0110735491184646\\
137	0.0110737313517402\\
138	0.0110739175868737\\
139	0.0110741079071785\\
140	0.0110743023975091\\
141	0.0110745011442813\\
142	0.0110747042354916\\
143	0.0110749117607369\\
144	0.0110751238112333\\
145	0.0110753404798355\\
146	0.0110755618610546\\
147	0.0110757880510769\\
148	0.0110760191477815\\
149	0.0110762552507575\\
150	0.0110764964613212\\
151	0.0110767428825322\\
152	0.0110769946192093\\
153	0.0110772517779456\\
154	0.0110775144671229\\
155	0.0110777827969256\\
156	0.0110780568793536\\
157	0.0110783368282344\\
158	0.0110786227592342\\
159	0.0110789147898681\\
160	0.0110792130395096\\
161	0.0110795176293982\\
162	0.0110798286826463\\
163	0.0110801463242451\\
164	0.0110804706810685\\
165	0.0110808018818759\\
166	0.0110811400573137\\
167	0.0110814853399147\\
168	0.0110818378640958\\
169	0.0110821977661549\\
170	0.0110825651842643\\
171	0.0110829402584631\\
172	0.0110833231306477\\
173	0.0110837139445592\\
174	0.011084112845769\\
175	0.011084519981662\\
176	0.0110849355014169\\
177	0.0110853595559841\\
178	0.0110857922980608\\
179	0.0110862338820627\\
180	0.0110866844640931\\
181	0.0110871442019089\\
182	0.0110876132548825\\
183	0.0110880917839612\\
184	0.0110885799516224\\
185	0.0110890779218253\\
186	0.0110895858599585\\
187	0.0110901039327842\\
188	0.0110906323083774\\
189	0.0110911711560619\\
190	0.0110917206463413\\
191	0.0110922809508256\\
192	0.0110928522421534\\
193	0.0110934346939098\\
194	0.0110940284805389\\
195	0.0110946337772519\\
196	0.0110952507599306\\
197	0.0110958796050258\\
198	0.0110965204894504\\
199	0.0110971735904689\\
200	0.0110978390855803\\
201	0.0110985171523983\\
202	0.0110992079685247\\
203	0.0110999117114205\\
204	0.0111006285582705\\
205	0.0111013586858455\\
206	0.0111021022703602\\
207	0.0111028594873266\\
208	0.0111036305114064\\
209	0.0111044155162583\\
210	0.0111052146743857\\
211	0.011106028156981\\
212	0.0111068561337703\\
213	0.0111076987728578\\
214	0.0111085562405715\\
215	0.0111094287013109\\
216	0.0111103163173976\\
217	0.0111112192489312\\
218	0.0111121376536508\\
219	0.0111130716868039\\
220	0.0111140215010256\\
221	0.0111149872462281\\
222	0.0111159690695046\\
223	0.0111169671150486\\
224	0.0111179815240917\\
225	0.0111190124348626\\
226	0.0111200599825698\\
227	0.0111211242994125\\
228	0.0111222055146206\\
229	0.0111233037545307\\
230	0.0111244191426985\\
231	0.0111255518000549\\
232	0.011126701845107\\
233	0.0111278693941904\\
234	0.0111290545617769\\
235	0.0111302574608417\\
236	0.011131478203295\\
237	0.011132716900483\\
238	0.0111339736637626\\
239	0.0111352486051542\\
240	0.011136541838077\\
241	0.0111378534781709\\
242	0.0111391836442071\\
243	0.0111405324590924\\
244	0.0111419000509658\\
245	0.011143286554392\\
246	0.0111446921116473\\
247	0.0111461168741002\\
248	0.0111475610036785\\
249	0.0111490246744212\\
250	0.0111505080741037\\
251	0.011152011405928\\
252	0.0111535348902619\\
253	0.0111550787664114\\
254	0.0111566432944048\\
255	0.0111582287567642\\
256	0.0111598354602357\\
257	0.0111614637374467\\
258	0.0111631139484544\\
259	0.0111647864821524\\
260	0.0111664817574989\\
261	0.011168200224542\\
262	0.0111699423651951\\
263	0.0111717086935359\\
264	0.0111734997544875\\
265	0.0111753161187042\\
266	0.0111771583941733\\
267	0.0111790272044009\\
268	0.0111809231881949\\
269	0.0111828469993635\\
270	0.0111847993063248\\
271	0.0111867807916173\\
272	0.0111887921513036\\
273	0.011190834094261\\
274	0.0111929073413485\\
275	0.0111950126244442\\
276	0.0111971506853451\\
277	0.0111993222745221\\
278	0.0112015281497237\\
279	0.0112037690744243\\
280	0.0112060458161125\\
281	0.0112083591444165\\
282	0.0112107098290685\\
283	0.0112130986377089\\
284	0.0112155263335368\\
285	0.0112179936728162\\
286	0.0112205014022522\\
287	0.0112230502562556\\
288	0.0112256409541215\\
289	0.0112282741971546\\
290	0.0112309506657801\\
291	0.0112336710166893\\
292	0.0112364358800798\\
293	0.0112392458570581\\
294	0.0112421015172894\\
295	0.0112450033969886\\
296	0.0112479519973661\\
297	0.0112509477836606\\
298	0.0112539911848951\\
299	0.0112570825944323\\
300	0.0112602223709475\\
301	0.0112634108379538\\
302	0.0112666482907384\\
303	0.0112699350085331\\
304	0.0112732712557301\\
305	0.0112766572913039\\
306	0.0112800933809172\\
307	0.0112835798133203\\
308	0.0112871169228193\\
309	0.011290705051912\\
310	0.0112943449622133\\
311	0.0112980381668326\\
312	0.0113017862403157\\
313	0.0113055908211205\\
314	0.0113094536156605\\
315	0.0113133763998072\\
316	0.0113173610064946\\
317	0.0113214093369247\\
318	0.011325523360919\\
319	0.0113297051167896\\
320	0.0113339567106441\\
321	0.0113382803150201\\
322	0.0113426781667316\\
323	0.0113471525638106\\
324	0.0113517058614398\\
325	0.011356340466859\\
326	0.0113610588334391\\
327	0.0113658634542428\\
328	0.0113707568504109\\
329	0.011375741480718\\
330	0.0113808198777477\\
331	0.0113859945529474\\
332	0.0113912679746864\\
333	0.0113966425423589\\
334	0.0114021205620233\\
335	0.0114077042822515\\
336	0.0114133958302027\\
337	0.0114191970233761\\
338	0.0114251094447915\\
339	0.0114311343985366\\
340	0.0114372728635781\\
341	0.0114435254502255\\
342	0.0114498923782943\\
343	0.0114564757110774\\
344	0.0114633583268099\\
345	0.0114702665365942\\
346	0.0114771978509434\\
347	0.0114841495969587\\
348	0.0114911189108963\\
349	0.0114981027360209\\
350	0.0115050978444322\\
351	0.0115121009864434\\
352	0.0115191077344752\\
353	0.0115261138952198\\
354	0.0115331152649056\\
355	0.0115401073191033\\
356	0.0115470850426171\\
357	0.0115540444728333\\
358	0.0115609816187135\\
359	0.0115678909243073\\
360	0.0115747665668397\\
361	0.0115816024619457\\
362	0.0115883922741627\\
363	0.0115951294273967\\
364	0.0116018071207027\\
365	0.011608418347648\\
366	0.011614955918549\\
367	0.0116214124827961\\
368	0.0116277805356299\\
369	0.0116338499163943\\
370	0.0116398074613504\\
371	0.0116457079907705\\
372	0.0116515471516567\\
373	0.0116573206668163\\
374	0.0116630241190706\\
375	0.0116686538939094\\
376	0.0116742064822266\\
377	0.0116796778379773\\
378	0.011685064054032\\
379	0.0116903613997309\\
380	0.0116955663617011\\
381	0.0117006756880017\\
382	0.0117056864309802\\
383	0.0117105959999162\\
384	0.0117154022123978\\
385	0.0117201033468445\\
386	0.011724698194018\\
387	0.0117291861178515\\
388	0.0117335671001144\\
389	0.0117378417289012\\
390	0.0117417857985826\\
391	0.0117456930296464\\
392	0.0117495718499579\\
393	0.0117534228427067\\
394	0.0117572468299464\\
395	0.011761044885906\\
396	0.0117648183493374\\
397	0.0117685688345873\\
398	0.011772298241053\\
399	0.0117760087606568\\
400	0.0117797028829875\\
401	0.0117833833975279\\
402	0.0117870533920869\\
403	0.0117907162478317\\
404	0.0117943756295744\\
405	0.0117980354722945\\
406	0.0118016999614246\\
407	0.0118053744796544\\
408	0.0118090725038309\\
409	0.0118127954711748\\
410	0.0118165450847441\\
411	0.0118203232641424\\
412	0.0118241317785936\\
413	0.0118279723825916\\
414	0.0118318469163682\\
415	0.0118357572065733\\
416	0.0118397053253837\\
417	0.011843693180799\\
418	0.011847722657131\\
419	0.0118517956180726\\
420	0.0118559138917806\\
421	0.0118600792538731\\
422	0.0118642934106802\\
423	0.0118685578954809\\
424	0.011872873806322\\
425	0.0118772422391113\\
426	0.0118816642846503\\
427	0.0118861410256874\\
428	0.0118906735340313\\
429	0.0118952628677567\\
430	0.0118999100686662\\
431	0.0119046161600133\\
432	0.0119093821445707\\
433	0.011914209003154\\
434	0.011919097693642\\
435	0.0119240491504071\\
436	0.0119290642843765\\
437	0.0119341439838081\\
438	0.0119392891291221\\
439	0.0119445005993858\\
440	0.0119497792720521\\
441	0.0119551260227887\\
442	0.0119605417254037\\
443	0.0119660272518677\\
444	0.0119715834724338\\
445	0.0119772112558594\\
446	0.011982911469726\\
447	0.0119886849808515\\
448	0.0119945326557821\\
449	0.0120004553613493\\
450	0.0120064539652724\\
451	0.0120125293366282\\
452	0.0120186823456013\\
453	0.0120249138632233\\
454	0.012031224761097\\
455	0.012037615911101\\
456	0.0120440881850682\\
457	0.012050642454434\\
458	0.0120572795898441\\
459	0.012064000460719\\
460	0.0120708059347626\\
461	0.0120776968774113\\
462	0.0120846741512115\\
463	0.0120917386151199\\
464	0.0120988911237318\\
465	0.0121061325264567\\
466	0.0121134636666355\\
467	0.0121208853805968\\
468	0.012128398496648\\
469	0.0121360038339963\\
470	0.0121437022015959\\
471	0.0121514943969178\\
472	0.0121593812046366\\
473	0.0121673633952307\\
474	0.012175441723492\\
475	0.0121836169269392\\
476	0.0121918897241327\\
477	0.0122002608128826\\
478	0.0122087308683447\\
479	0.0122173005409972\\
480	0.0122259704544909\\
481	0.0122347412033652\\
482	0.0122436133506212\\
483	0.0122525874251436\\
484	0.0122616639189613\\
485	0.0122708432843368\\
486	0.012280125930673\\
487	0.0122895122212268\\
488	0.0122990024696139\\
489	0.0123085969360949\\
490	0.0123182958236242\\
491	0.0123280992736481\\
492	0.012338007361634\\
493	0.0123480200923118\\
494	0.0123581373946084\\
495	0.0123683591162527\\
496	0.0123786850180273\\
497	0.0123891147676427\\
498	0.0123996479332052\\
499	0.0124102839762488\\
500	0.0124210222442989\\
501	0.0124318619629314\\
502	0.0124428022272907\\
503	0.0124538419930214\\
504	0.0124649800665717\\
505	0.0124762150948152\\
506	0.0124875455539396\\
507	0.0124989697375404\\
508	0.0125104857438556\\
509	0.0125220914620709\\
510	0.012533784557616\\
511	0.0125455624563677\\
512	0.0125574223276655\\
513	0.0125693610660378\\
514	0.0125813752715246\\
515	0.0125934612284751\\
516	0.0126056148826818\\
517	0.0126178318167036\\
518	0.0126301072232128\\
519	0.0126424358761864\\
520	0.012654812099742\\
521	0.0126672297344029\\
522	0.0126796821005508\\
523	0.0126921619588054\\
524	0.0127046614670423\\
525	0.0127171720379401\\
526	0.0127296842664932\\
527	0.0127421902127353\\
528	0.012754684209797\\
529	0.0127680816133568\\
530	0.0127818538770959\\
531	0.0127954264333314\\
532	0.0128081395105162\\
533	0.0128202969787341\\
534	0.0128321162869548\\
535	0.0128432026117559\\
536	0.0128537758574423\\
537	0.0128640759468053\\
538	0.0128740848734177\\
539	0.0128838231896525\\
540	0.0128932602927639\\
541	0.012902471419193\\
542	0.0129118513862837\\
543	0.0129214188625862\\
544	0.0129324016955739\\
545	0.0129460376420307\\
546	0.0129596516018453\\
547	0.012972043748237\\
548	0.0129837885407481\\
549	0.0129951835481094\\
550	0.0130054949438586\\
551	0.0130158189828727\\
552	0.0130261636024336\\
553	0.0130366296303294\\
554	0.0130473222522103\\
555	0.0130586764567069\\
556	0.0130696869228016\\
557	0.0130804505623059\\
558	0.0130909406703917\\
559	0.0131014900219591\\
560	0.0131121238283323\\
561	0.0131228397507894\\
562	0.0131336341112898\\
563	0.0131445027539313\\
564	0.0131556054830169\\
565	0.01316682675996\\
566	0.0131777050422799\\
567	0.0131884046862925\\
568	0.0131991670871561\\
569	0.0132099913528043\\
570	0.0132208709214972\\
571	0.0132317984769864\\
572	0.013242765927568\\
573	0.0132537643399357\\
574	0.0132647838643886\\
575	0.013275813652582\\
576	0.0132868417670616\\
577	0.0132978550815367\\
578	0.0133088391706665\\
579	0.0133197781879283\\
580	0.0133306547299067\\
581	0.0133414496851422\\
582	0.0133521420655987\\
583	0.0133627088191946\\
584	0.0133731246235461\\
585	0.0133833616664418\\
586	0.013393389433806\\
587	0.0134031745674478\\
588	0.0134126809661912\\
589	0.0134218705998152\\
590	0.0134307062886727\\
591	0.0134391597728698\\
592	0.0134472338645631\\
593	0.0134551345125145\\
594	0.0134637112677116\\
595	0.0134740407759367\\
596	0.0134889777516421\\
597	0.0135160557778789\\
598	0.0135751148335434\\
599	0\\
600	0\\
};
\addplot [color=black!80!mycolor21,solid,forget plot]
  table[row sep=crcr]{%
1	0.0110658665044562\\
2	0.0110658756687729\\
3	0.0110658850498207\\
4	0.0110658946527232\\
5	0.0110659044827253\\
6	0.0110659145451954\\
7	0.011065924845629\\
8	0.0110659353896511\\
9	0.0110659461830198\\
10	0.0110659572316289\\
11	0.0110659685415116\\
12	0.0110659801188434\\
13	0.0110659919699456\\
14	0.0110660041012888\\
15	0.0110660165194963\\
16	0.0110660292313478\\
17	0.0110660422437829\\
18	0.0110660555639052\\
19	0.0110660691989856\\
20	0.0110660831564669\\
21	0.0110660974439675\\
22	0.0110661120692853\\
23	0.0110661270404025\\
24	0.0110661423654895\\
25	0.0110661580529093\\
26	0.0110661741112223\\
27	0.0110661905491909\\
28	0.011066207375784\\
29	0.0110662246001821\\
30	0.0110662422317821\\
31	0.0110662602802026\\
32	0.0110662787552889\\
33	0.0110662976671183\\
34	0.0110663170260058\\
35	0.0110663368425095\\
36	0.0110663571274362\\
37	0.0110663778918476\\
38	0.0110663991470659\\
39	0.01106642090468\\
40	0.0110664431765519\\
41	0.011066465974823\\
42	0.0110664893119208\\
43	0.0110665132005652\\
44	0.0110665376537758\\
45	0.0110665626848787\\
46	0.0110665883075138\\
47	0.0110666145356419\\
48	0.0110666413835525\\
49	0.0110666688658714\\
50	0.0110666969975684\\
51	0.0110667257939656\\
52	0.0110667552707454\\
53	0.0110667854439592\\
54	0.0110668163300358\\
55	0.0110668479457902\\
56	0.0110668803084328\\
57	0.0110669134355784\\
58	0.0110669473452559\\
59	0.0110669820559177\\
60	0.0110670175864499\\
61	0.0110670539561819\\
62	0.0110670911848974\\
63	0.0110671292928443\\
64	0.011067168300746\\
65	0.0110672082298122\\
66	0.0110672491017505\\
67	0.0110672909387776\\
68	0.0110673337636314\\
69	0.0110673775995829\\
70	0.011067422470449\\
71	0.0110674684006047\\
72	0.0110675154149961\\
73	0.0110675635391541\\
74	0.0110676127992073\\
75	0.0110676632218965\\
76	0.0110677148345883\\
77	0.0110677676652899\\
78	0.011067821742664\\
79	0.0110678770960436\\
80	0.0110679337554477\\
81	0.0110679917515973\\
82	0.0110680511159312\\
83	0.0110681118806229\\
84	0.0110681740785973\\
85	0.0110682377435481\\
86	0.0110683029099556\\
87	0.0110683696131045\\
88	0.0110684378891026\\
89	0.0110685077748996\\
90	0.0110685793083065\\
91	0.0110686525280152\\
92	0.0110687274736188\\
93	0.0110688041856321\\
94	0.0110688827055128\\
95	0.0110689630756828\\
96	0.0110690453395505\\
97	0.0110691295415331\\
98	0.0110692157270794\\
99	0.0110693039426938\\
100	0.0110693942359597\\
101	0.011069486655564\\
102	0.0110695812513228\\
103	0.0110696780742061\\
104	0.0110697771763644\\
105	0.011069878611155\\
106	0.0110699824331699\\
107	0.0110700886982625\\
108	0.0110701974635771\\
109	0.011070308787577\\
110	0.0110704227300746\\
111	0.0110705393522615\\
112	0.011070658716739\\
113	0.01107078088755\\
114	0.0110709059302108\\
115	0.0110710339117438\\
116	0.0110711649007114\\
117	0.0110712989672493\\
118	0.0110714361831021\\
119	0.0110715766216581\\
120	0.0110717203579859\\
121	0.0110718674688713\\
122	0.0110720180328544\\
123	0.0110721721302687\\
124	0.0110723298432794\\
125	0.0110724912559238\\
126	0.0110726564541517\\
127	0.0110728255258667\\
128	0.0110729985609684\\
129	0.0110731756513951\\
130	0.0110733568911678\\
131	0.0110735423764343\\
132	0.0110737322055147\\
133	0.0110739264789475\\
134	0.0110741252995363\\
135	0.0110743287723974\\
136	0.0110745370050084\\
137	0.0110747501072575\\
138	0.0110749681914937\\
139	0.011075191372577\\
140	0.0110754197679312\\
141	0.0110756534975952\\
142	0.0110758926842773\\
143	0.0110761374534085\\
144	0.0110763879331978\\
145	0.0110766442546874\\
146	0.0110769065518095\\
147	0.011077174961443\\
148	0.0110774496234716\\
149	0.0110777306808421\\
150	0.0110780182796234\\
151	0.0110783125690671\\
152	0.0110786137016668\\
153	0.0110789218332203\\
154	0.0110792371228904\\
155	0.0110795597332672\\
156	0.0110798898304309\\
157	0.0110802275840147\\
158	0.0110805731672678\\
159	0.0110809267571198\\
160	0.011081288534244\\
161	0.0110816586831217\\
162	0.0110820373921066\\
163	0.0110824248534886\\
164	0.0110828212635584\\
165	0.0110832268226712\\
166	0.0110836417353102\\
167	0.0110840662101506\\
168	0.0110845004601215\\
169	0.0110849447024687\\
170	0.0110853991588157\\
171	0.0110858640552241\\
172	0.0110863396222528\\
173	0.0110868260950164\\
174	0.0110873237132408\\
175	0.0110878327213187\\
176	0.0110883533683621\\
177	0.0110888859082526\\
178	0.01108943059969\\
179	0.0110899877062376\\
180	0.0110905574963645\\
181	0.0110911402434848\\
182	0.0110917362259929\\
183	0.0110923457272949\\
184	0.0110929690358352\\
185	0.0110936064451192\\
186	0.0110942582537292\\
187	0.011094924765336\\
188	0.0110956062887032\\
189	0.0110963031376851\\
190	0.0110970156312171\\
191	0.0110977440932984\\
192	0.0110984888529654\\
193	0.011099250244256\\
194	0.0111000286061648\\
195	0.0111008242825855\\
196	0.0111016376222437\\
197	0.0111024689786157\\
198	0.0111033187098347\\
199	0.0111041871785824\\
200	0.0111050747519647\\
201	0.0111059818013723\\
202	0.011106908702322\\
203	0.0111078558342806\\
204	0.0111088235804684\\
205	0.0111098123276412\\
206	0.0111108224658502\\
207	0.0111118543881778\\
208	0.0111129084904479\\
209	0.0111139851709095\\
210	0.0111150848298919\\
211	0.0111162078694299\\
212	0.0111173546928574\\
213	0.0111185257043679\\
214	0.0111197213085404\\
215	0.0111209419098283\\
216	0.0111221879120103\\
217	0.0111234597176016\\
218	0.0111247577272237\\
219	0.0111260823389312\\
220	0.0111274339474936\\
221	0.0111288129436318\\
222	0.0111302197132077\\
223	0.0111316546363646\\
224	0.0111331180866195\\
225	0.0111346104299051\\
226	0.0111361320235623\\
227	0.0111376832152826\\
228	0.0111392643420009\\
229	0.0111408757287406\\
230	0.0111425176874111\\
231	0.0111441905155621\\
232	0.0111458944950957\\
233	0.0111476298909424\\
234	0.0111493969497045\\
235	0.0111511958982747\\
236	0.011153026942436\\
237	0.0111548902654543\\
238	0.0111567860266728\\
239	0.0111587143601218\\
240	0.0111606753731602\\
241	0.0111626691451633\\
242	0.0111646957262801\\
243	0.0111667551362811\\
244	0.0111688473635224\\
245	0.0111709723640567\\
246	0.0111731300609223\\
247	0.0111753203436479\\
248	0.0111775430680124\\
249	0.0111797980561049\\
250	0.0111820850967317\\
251	0.0111844039462241\\
252	0.0111867543297012\\
253	0.0111891359428473\\
254	0.0111915484542651\\
255	0.0111939915084685\\
256	0.0111964647295773\\
257	0.0111989677257794\\
258	0.0112015000946204\\
259	0.0112040614291846\\
260	0.0112066513252404\\
261	0.0112092693894599\\
262	0.0112119152489404\\
263	0.0112145885624849\\
264	0.011217289033909\\
265	0.0112200164186713\\
266	0.0112227704903705\\
267	0.0112255518088507\\
268	0.0112283609678844\\
269	0.0112311985972577\\
270	0.0112340653649282\\
271	0.0112369619792517\\
272	0.0112398891912686\\
273	0.0112428477970429\\
274	0.01124583864004\\
275	0.0112488626135312\\
276	0.0112519206630064\\
277	0.0112550137885755\\
278	0.011258143047333\\
279	0.0112613095556583\\
280	0.0112645144914174\\
281	0.0112677590960289\\
282	0.0112710446763472\\
283	0.0112743726063129\\
284	0.0112777443283106\\
285	0.0112811613541667\\
286	0.0112846252657135\\
287	0.0112881377148324\\
288	0.0112917004228831\\
289	0.0112953151794132\\
290	0.0112989838400317\\
291	0.0113027083233196\\
292	0.0113064906066402\\
293	0.0113103327207012\\
294	0.0113142367427128\\
295	0.0113182047879818\\
296	0.0113222389997891\\
297	0.011326341537422\\
298	0.0113305145623161\\
299	0.0113347602224364\\
300	0.0113390806351764\\
301	0.0113434778664861\\
302	0.0113479538634801\\
303	0.0113525104463515\\
304	0.0113571493695748\\
305	0.0113618722217628\\
306	0.0113666803938898\\
307	0.0113715750501255\\
308	0.0113765571193788\\
309	0.011381707475626\\
310	0.011387049894648\\
311	0.0113924239757057\\
312	0.0113978290558508\\
313	0.01140326439794\\
314	0.0114087291882998\\
315	0.0114142225594584\\
316	0.0114197436140022\\
317	0.0114252912382915\\
318	0.0114308642143504\\
319	0.0114364612144706\\
320	0.0114420807938518\\
321	0.0114477213830169\\
322	0.011453381280053\\
323	0.0114590586427585\\
324	0.0114647514808346\\
325	0.0114704576484583\\
326	0.0114761748382996\\
327	0.0114819005811336\\
328	0.0114876322701486\\
329	0.0114933673073714\\
330	0.0114991018404361\\
331	0.0115048326684356\\
332	0.0115105563892819\\
333	0.0115162693862624\\
334	0.0115219677922677\\
335	0.0115276473766927\\
336	0.0115333044276723\\
337	0.0115389357892891\\
338	0.0115445368102847\\
339	0.0115501026601679\\
340	0.0115556283419749\\
341	0.0115611087059232\\
342	0.0115665384587616\\
343	0.0115718520637225\\
344	0.0115769909083038\\
345	0.0115821020108902\\
346	0.0115871822424816\\
347	0.0115922284051057\\
348	0.011597237240375\\
349	0.0116022054415922\\
350	0.0116071296784015\\
351	0.0116120066966827\\
352	0.0116168328023537\\
353	0.011621604512462\\
354	0.0116263184700951\\
355	0.0116309713279657\\
356	0.0116355596732972\\
357	0.0116400807569049\\
358	0.01164453206391\\
359	0.0116489106468586\\
360	0.0116532137378992\\
361	0.0116574387840926\\
362	0.011661583485679\\
363	0.0116656458346156\\
364	0.011669624155156\\
365	0.0116735171449084\\
366	0.0116773239145632\\
367	0.0116810440206136\\
368	0.0116846774678253\\
369	0.0116880779536732\\
370	0.0116914043544729\\
371	0.0116947044842502\\
372	0.0116979782518993\\
373	0.0117012257199864\\
374	0.0117044471152909\\
375	0.0117076428349863\\
376	0.0117108134609995\\
377	0.0117139597771556\\
378	0.0117170827794904\\
379	0.0117201836855822\\
380	0.0117232639426276\\
381	0.0117263252339571\\
382	0.0117293694838072\\
383	0.0117323988597782\\
384	0.0117354157727272\\
385	0.0117384228737062\\
386	0.0117414230475045\\
387	0.0117444194016274\\
388	0.0117474152504524\\
389	0.0117504141070941\\
390	0.0117534276614737\\
391	0.0117564573856178\\
392	0.0117595045329531\\
393	0.0117625704201511\\
394	0.0117656564234616\\
395	0.0117687639741982\\
396	0.0117718945533349\\
397	0.011775049685182\\
398	0.0117782309301198\\
399	0.011781439876384\\
400	0.0117846781309048\\
401	0.0117879473092122\\
402	0.0117912490244349\\
403	0.0117945848753464\\
404	0.0117979564333971\\
405	0.0118013652289198\\
406	0.0118048127417615\\
407	0.0118083003045304\\
408	0.0118118289848222\\
409	0.0118153997592706\\
410	0.0118190136094087\\
411	0.0118226715799512\\
412	0.0118263746192451\\
413	0.0118301236686515\\
414	0.0118339196593265\\
415	0.0118377635096281\\
416	0.011841656121783\\
417	0.0118455983824789\\
418	0.0118495911616511\\
419	0.0118536353114133\\
420	0.0118577316654239\\
421	0.0118618810388004\\
422	0.0118660842286248\\
423	0.0118703420183335\\
424	0.0118746551922817\\
425	0.0118790245355134\\
426	0.0118834508336072\\
427	0.0118879348726027\\
428	0.0118924774390156\\
429	0.0118970793199447\\
430	0.011901741303273\\
431	0.0119064641779606\\
432	0.0119112487344297\\
433	0.0119160957650317\\
434	0.0119210060645893\\
435	0.0119259804310075\\
436	0.0119310196659368\\
437	0.0119361245754705\\
438	0.0119412959703513\\
439	0.0119465346659483\\
440	0.0119518414822347\\
441	0.0119572172437638\\
442	0.0119626627796386\\
443	0.0119681789234718\\
444	0.0119737665133319\\
445	0.0119794263916702\\
446	0.0119851594052236\\
447	0.0119909664048864\\
448	0.0119968482455476\\
449	0.0120028057858843\\
450	0.0120088398881079\\
451	0.012014951417662\\
452	0.0120211412428938\\
453	0.0120274102346974\\
454	0.0120337592661276\\
455	0.0120401892119805\\
456	0.0120467009483396\\
457	0.0120532953520852\\
458	0.0120599733003649\\
459	0.0120667356700221\\
460	0.0120735833369826\\
461	0.0120805171755945\\
462	0.0120875380579224\\
463	0.0120946468529916\\
464	0.0121018444259829\\
465	0.0121091316373719\\
466	0.0121165093420129\\
467	0.0121239783881612\\
468	0.012131539616432\\
469	0.0121391938586919\\
470	0.0121469419368785\\
471	0.012154784661744\\
472	0.0121627228315191\\
473	0.0121707572304911\\
474	0.0121788886274922\\
475	0.0121871177742917\\
476	0.0121954454038866\\
477	0.012203872228684\\
478	0.0122123989385686\\
479	0.0122210261988474\\
480	0.0122297546480653\\
481	0.012238584895681\\
482	0.0122475175195965\\
483	0.0122565530635282\\
484	0.0122656920342115\\
485	0.0122749348984256\\
486	0.012284282079828\\
487	0.0122937339555843\\
488	0.0123032908527809\\
489	0.0123129530446041\\
490	0.0123227207462692\\
491	0.012332594110683\\
492	0.0123425732238204\\
493	0.0123526580997929\\
494	0.0123628486755882\\
495	0.0123731448054568\\
496	0.0123835462549175\\
497	0.0123940526943552\\
498	0.0124046636921786\\
499	0.0124153787075053\\
500	0.0124261970823371\\
501	0.0124371180331857\\
502	0.0124481406421071\\
503	0.0124592638470957\\
504	0.01247048643179\\
505	0.012481807014433\\
506	0.0124932240360285\\
507	0.0125047357476292\\
508	0.0125163401966863\\
509	0.012528035212386\\
510	0.0125398183898921\\
511	0.012551687073408\\
512	0.0125636383379657\\
513	0.0125756689698452\\
514	0.0125877754455178\\
515	0.0125999539090094\\
516	0.0126122001475681\\
517	0.0126245095655259\\
518	0.0126368771562392\\
519	0.0126492974719996\\
520	0.0126617645918139\\
521	0.0126742720869686\\
522	0.0126868129843154\\
523	0.0126993797272539\\
524	0.0127119641344364\\
525	0.0127245621636429\\
526	0.0127378286130893\\
527	0.012751826021275\\
528	0.0127656488468785\\
529	0.012778699366717\\
530	0.0127911762106919\\
531	0.0128033516574677\\
532	0.0128147987411692\\
533	0.0128258321847314\\
534	0.0128366389593896\\
535	0.0128472305654094\\
536	0.0128576146530734\\
537	0.0128677694856517\\
538	0.0128776751184309\\
539	0.0128873046562076\\
540	0.0128966255102941\\
541	0.0129058876807344\\
542	0.0129179735341843\\
543	0.0129314604712211\\
544	0.0129444722025304\\
545	0.0129560621552084\\
546	0.0129673691036461\\
547	0.0129776396512952\\
548	0.012987558268592\\
549	0.0129974903528408\\
550	0.0130075431523381\\
551	0.013017718660953\\
552	0.0130280221223931\\
553	0.0130388412504882\\
554	0.0130499213058635\\
555	0.0130605523893644\\
556	0.0130709501988515\\
557	0.0130812783988437\\
558	0.0130916971982159\\
559	0.0131022058233943\\
560	0.013112800233129\\
561	0.0131234770442047\\
562	0.0131342326716463\\
563	0.0131452413591923\\
564	0.0131563341090672\\
565	0.0131670883192149\\
566	0.0131777102595777\\
567	0.0131884047434965\\
568	0.013199167091273\\
569	0.0132099913534359\\
570	0.0132208709216939\\
571	0.0132317984770694\\
572	0.013242765927608\\
573	0.0132537643399573\\
574	0.0132647838644009\\
575	0.0132758136525892\\
576	0.013286841767066\\
577	0.0132978550815394\\
578	0.0133088391706681\\
579	0.013319778187929\\
580	0.013330654729907\\
581	0.0133414496851423\\
582	0.0133521420655987\\
583	0.0133627088191946\\
584	0.0133731246235461\\
585	0.0133833616664418\\
586	0.013393389433806\\
587	0.0134031745674478\\
588	0.0134126809661912\\
589	0.0134218705998152\\
590	0.0134307062886727\\
591	0.0134391597728698\\
592	0.0134472338645631\\
593	0.0134551345125145\\
594	0.0134637112677116\\
595	0.0134740407759367\\
596	0.0134889777516421\\
597	0.0135160557778789\\
598	0.0135751148335434\\
599	0\\
600	0\\
};
\addplot [color=black,solid,forget plot]
  table[row sep=crcr]{%
1	0.0110658750514072\\
2	0.0110658844718838\\
3	0.0110658941169069\\
4	0.0110659039918427\\
5	0.0110659141021865\\
6	0.0110659244535651\\
7	0.0110659350517404\\
8	0.0110659459026126\\
9	0.0110659570122233\\
10	0.0110659683867593\\
11	0.0110659800325558\\
12	0.0110659919561\\
13	0.0110660041640349\\
14	0.011066016663163\\
15	0.0110660294604502\\
16	0.0110660425630297\\
17	0.0110660559782058\\
18	0.0110660697134585\\
19	0.0110660837764475\\
20	0.0110660981750166\\
21	0.0110661129171978\\
22	0.0110661280112164\\
23	0.0110661434654955\\
24	0.0110661592886605\\
25	0.0110661754895442\\
26	0.0110661920771921\\
27	0.0110662090608669\\
28	0.0110662264500542\\
29	0.0110662442544681\\
30	0.0110662624840561\\
31	0.0110662811490052\\
32	0.0110663002597477\\
33	0.0110663198269669\\
34	0.0110663398616034\\
35	0.0110663603748612\\
36	0.0110663813782143\\
37	0.0110664028834127\\
38	0.01106642490249\\
39	0.0110664474477694\\
40	0.0110664705318711\\
41	0.0110664941677197\\
42	0.0110665183685512\\
43	0.0110665431479209\\
44	0.0110665685197107\\
45	0.0110665944981377\\
46	0.0110666210977617\\
47	0.0110666483334941\\
48	0.0110666762206057\\
49	0.0110667047747363\\
50	0.0110667340119032\\
51	0.0110667639485102\\
52	0.0110667946013576\\
53	0.0110668259876512\\
54	0.0110668581250128\\
55	0.0110668910314896\\
56	0.0110669247255654\\
57	0.0110669592261706\\
58	0.0110669945526933\\
59	0.0110670307249905\\
60	0.0110670677633996\\
61	0.0110671056887498\\
62	0.0110671445223745\\
63	0.0110671842861234\\
64	0.0110672250023751\\
65	0.0110672666940501\\
66	0.011067309384624\\
67	0.0110673530981411\\
68	0.0110673978592282\\
69	0.0110674436931089\\
70	0.0110674906256183\\
71	0.011067538683218\\
72	0.0110675878930111\\
73	0.0110676382827582\\
74	0.0110676898808937\\
75	0.0110677427165416\\
76	0.0110677968195333\\
77	0.0110678522204243\\
78	0.0110679089505122\\
79	0.011067967041855\\
80	0.0110680265272896\\
81	0.011068087440451\\
82	0.0110681498157918\\
83	0.0110682136886027\\
84	0.0110682790950326\\
85	0.0110683460721099\\
86	0.0110684146577642\\
87	0.0110684848908485\\
88	0.011068556811162\\
89	0.0110686304594731\\
90	0.0110687058775436\\
91	0.011068783108153\\
92	0.011068862195124\\
93	0.0110689431833475\\
94	0.0110690261188098\\
95	0.0110691110486191\\
96	0.0110691980210335\\
97	0.0110692870854891\\
98	0.0110693782926295\\
99	0.0110694716943358\\
100	0.0110695673437565\\
101	0.0110696652953396\\
102	0.0110697656048647\\
103	0.0110698683294757\\
104	0.0110699735277146\\
105	0.0110700812595569\\
106	0.0110701915864463\\
107	0.0110703045713315\\
108	0.011070420278704\\
109	0.0110705387746356\\
110	0.0110706601268187\\
111	0.0110707844046057\\
112	0.0110709116790509\\
113	0.0110710420229528\\
114	0.0110711755108975\\
115	0.011071312219303\\
116	0.0110714522264654\\
117	0.0110715956126055\\
118	0.0110717424599168\\
119	0.011071892852615\\
120	0.0110720468769882\\
121	0.011072204621449\\
122	0.0110723661765877\\
123	0.0110725316352265\\
124	0.0110727010924757\\
125	0.0110728746457907\\
126	0.0110730523950312\\
127	0.0110732344425212\\
128	0.0110734208931107\\
129	0.0110736118542396\\
130	0.0110738074360022\\
131	0.0110740077512143\\
132	0.0110742129154814\\
133	0.0110744230472687\\
134	0.0110746382679735\\
135	0.0110748587019986\\
136	0.0110750844768279\\
137	0.0110753157231045\\
138	0.0110755525747098\\
139	0.0110757951688453\\
140	0.0110760436461163\\
141	0.0110762981506179\\
142	0.0110765588300229\\
143	0.0110768258356718\\
144	0.0110770993226659\\
145	0.011077379449962\\
146	0.0110776663804698\\
147	0.0110779602811514\\
148	0.0110782613231244\\
149	0.011078569681766\\
150	0.0110788855368212\\
151	0.0110792090725129\\
152	0.011079540477655\\
153	0.0110798799457685\\
154	0.0110802276752001\\
155	0.0110805838692443\\
156	0.011080948736268\\
157	0.0110813224898387\\
158	0.0110817053488555\\
159	0.0110820975376836\\
160	0.0110824992862918\\
161	0.0110829108303939\\
162	0.0110833324115926\\
163	0.0110837642775282\\
164	0.0110842066820295\\
165	0.0110846598852695\\
166	0.0110851241539239\\
167	0.0110855997613339\\
168	0.0110860869876726\\
169	0.0110865861201152\\
170	0.0110870974530134\\
171	0.0110876212880731\\
172	0.0110881579345372\\
173	0.0110887077093713\\
174	0.0110892709374541\\
175	0.0110898479517721\\
176	0.0110904390936175\\
177	0.0110910447127914\\
178	0.0110916651678104\\
179	0.0110923008261176\\
180	0.0110929520642972\\
181	0.0110936192682946\\
182	0.0110943028336386\\
183	0.0110950031656692\\
184	0.0110957206797684\\
185	0.0110964558015952\\
186	0.0110972089673243\\
187	0.0110979806238882\\
188	0.0110987712292227\\
189	0.0110995812525157\\
190	0.0111004111744591\\
191	0.0111012614875032\\
192	0.0111021326961134\\
193	0.0111030253170296\\
194	0.0111039398795263\\
195	0.0111048769256752\\
196	0.0111058370106075\\
197	0.0111068207027769\\
198	0.0111078285842225\\
199	0.0111088612508298\\
200	0.0111099193125908\\
201	0.0111110033938607\\
202	0.0111121141336113\\
203	0.01111325218568\\
204	0.0111144182190122\\
205	0.011115612917897\\
206	0.0111168369821947\\
207	0.011118091127554\\
208	0.0111193760856168\\
209	0.0111206926042102\\
210	0.0111220414475223\\
211	0.0111234233962604\\
212	0.011124839247788\\
213	0.0111262898162391\\
214	0.0111277759326066\\
215	0.0111292984448007\\
216	0.0111308582176747\\
217	0.011132456133014\\
218	0.011134093089484\\
219	0.0111357700025325\\
220	0.0111374878042418\\
221	0.0111392474431245\\
222	0.0111410498838575\\
223	0.011142896106949\\
224	0.0111447871083288\\
225	0.0111467238988578\\
226	0.0111487075037465\\
227	0.0111507389618748\\
228	0.0111528193250031\\
229	0.0111549496568659\\
230	0.0111571310321366\\
231	0.0111593645352508\\
232	0.0111616512590792\\
233	0.0111639923034324\\
234	0.0111663887733889\\
235	0.0111688417774267\\
236	0.0111713524253472\\
237	0.0111739218259714\\
238	0.0111765510845945\\
239	0.0111792413001783\\
240	0.0111819935622647\\
241	0.0111848089475906\\
242	0.0111876885163837\\
243	0.0111906333083215\\
244	0.011193644338132\\
245	0.0111967225908176\\
246	0.0111998690164836\\
247	0.0112030845247539\\
248	0.0112063699787597\\
249	0.0112097261886871\\
250	0.0112131539048771\\
251	0.0112166538104739\\
252	0.011220226513625\\
253	0.0112238725392461\\
254	0.0112275923203743\\
255	0.0112313861891475\\
256	0.0112352543674674\\
257	0.011239196957426\\
258	0.0112432139316089\\
259	0.0112473051234342\\
260	0.0112514702177651\\
261	0.0112557087421901\\
262	0.011260020059759\\
263	0.0112644033651342\\
264	0.0112688576902486\\
265	0.011273388597383\\
266	0.011278120117581\\
267	0.0112828810304852\\
268	0.0112876707737252\\
269	0.0112924887445411\\
270	0.0112973342983057\\
271	0.0113022067470334\\
272	0.0113071053578852\\
273	0.011312029351671\\
274	0.0113169779013589\\
275	0.0113219501305961\\
276	0.0113269451122499\\
277	0.01133196186697\\
278	0.0113369993617633\\
279	0.0113420565087861\\
280	0.0113471321641829\\
281	0.0113522251267056\\
282	0.0113573341366179\\
283	0.0113624578747126\\
284	0.0113675949614561\\
285	0.0113727439562827\\
286	0.0113779033570593\\
287	0.011383071599742\\
288	0.0113882470582507\\
289	0.0113934280445865\\
290	0.0113986128092188\\
291	0.0114037995417722\\
292	0.0114089863720399\\
293	0.0114141713713571\\
294	0.0114193525543642\\
295	0.0114245278811968\\
296	0.0114296952601534\\
297	0.0114348525509359\\
298	0.011439997568717\\
299	0.0114451280898342\\
300	0.0114502418619611\\
301	0.01145533663055\\
302	0.0114604102453435\\
303	0.0114654601104107\\
304	0.0114704831772255\\
305	0.0114754771528777\\
306	0.0114804397432685\\
307	0.0114853686550007\\
308	0.0114902615868697\\
309	0.0114950699852063\\
310	0.0114997756240239\\
311	0.0115044649884105\\
312	0.011509136143499\\
313	0.0115137871044368\\
314	0.0115184158262472\\
315	0.0115230201818247\\
316	0.0115275982669801\\
317	0.0115321492275243\\
318	0.0115366708191833\\
319	0.011541160768341\\
320	0.0115456167766223\\
321	0.0115500365260829\\
322	0.011554417685061\\
323	0.0115587579147462\\
324	0.011563054876542\\
325	0.0115673062403616\\
326	0.0115715096942642\\
327	0.0115756629571365\\
328	0.0115797638036078\\
329	0.0115838101614979\\
330	0.0115877995386961\\
331	0.011591729898201\\
332	0.0115955992938587\\
333	0.0115994058818124\\
334	0.0116031479213682\\
335	0.0116068237270561\\
336	0.0116104320785765\\
337	0.0116139722842614\\
338	0.0116174432712505\\
339	0.0116208441901195\\
340	0.0116241744350945\\
341	0.0116274336603087\\
342	0.0116306217831731\\
343	0.0116336953750199\\
344	0.0116366214021809\\
345	0.0116395206895466\\
346	0.0116423928362576\\
347	0.0116452375251151\\
348	0.0116480545296806\\
349	0.0116508437215552\\
350	0.0116536050778247\\
351	0.0116563386888731\\
352	0.0116590447684414\\
353	0.0116617236583792\\
354	0.0116643758346477\\
355	0.0116670019139605\\
356	0.0116696026593246\\
357	0.011672178981745\\
358	0.0116747319476121\\
359	0.0116772627890906\\
360	0.0116797729079845\\
361	0.011682263878433\\
362	0.0116847374481983\\
363	0.0116871955383681\\
364	0.0116896402411757\\
365	0.0116920738155775\\
366	0.0116944986801466\\
367	0.0116969174039349\\
368	0.0116993327066238\\
369	0.0117017525737516\\
370	0.0117041795572421\\
371	0.0117066145866293\\
372	0.0117090586440313\\
373	0.0117115127627278\\
374	0.0117139780252346\\
375	0.0117164555610023\\
376	0.0117189465435208\\
377	0.0117214521866742\\
378	0.011723973740466\\
379	0.0117265124860925\\
380	0.011729069730349\\
381	0.0117316467993629\\
382	0.0117342450316503\\
383	0.0117368657705124\\
384	0.0117395103557893\\
385	0.0117421801149998\\
386	0.0117448763539552\\
387	0.011747600347295\\
388	0.0117503533304433\\
389	0.011753136491348\\
390	0.0117559506574283\\
391	0.0117587966517825\\
392	0.011761675301312\\
393	0.0117645874348435\\
394	0.011767533881213\\
395	0.0117705154673326\\
396	0.0117735330162686\\
397	0.0117765873453587\\
398	0.0117796792644017\\
399	0.0117828095739561\\
400	0.0117859790637783\\
401	0.0117891885114308\\
402	0.0117924386810585\\
403	0.0117957303222432\\
404	0.0117990641685028\\
405	0.011802440933399\\
406	0.0118058612926688\\
407	0.0118093259224047\\
408	0.0118128356691035\\
409	0.011816391248266\\
410	0.0118199933401305\\
411	0.0118236426247192\\
412	0.0118273397829966\\
413	0.0118310854969216\\
414	0.0118348804496045\\
415	0.011838725325558\\
416	0.0118426208110733\\
417	0.0118465675946009\\
418	0.011850566367199\\
419	0.0118546178230534\\
420	0.011858722660061\\
421	0.0118628815804639\\
422	0.0118670952915211\\
423	0.0118713645060784\\
424	0.0118756899426467\\
425	0.0118800723254883\\
426	0.0118845123847101\\
427	0.0118890108563599\\
428	0.0118935684825252\\
429	0.0118981860114297\\
430	0.0119028641975259\\
431	0.0119076038015782\\
432	0.0119124055907354\\
433	0.0119172703385852\\
434	0.011922198825189\\
435	0.0119271918370916\\
436	0.0119322501673012\\
437	0.0119373746152351\\
438	0.011942565986647\\
439	0.0119478250935408\\
440	0.011953152754072\\
441	0.0119585497924328\\
442	0.0119640170387223\\
443	0.0119695553287987\\
444	0.0119751655041133\\
445	0.0119808484115245\\
446	0.0119866049030919\\
447	0.0119924358358483\\
448	0.0119983420715493\\
449	0.0120043244764002\\
450	0.0120103839207597\\
451	0.0120165212788188\\
452	0.012022737428255\\
453	0.0120290332498596\\
454	0.0120354096271364\\
455	0.0120418674458711\\
456	0.0120484075936679\\
457	0.0120550309594545\\
458	0.01206173843295\\
459	0.0120685309040974\\
460	0.0120754092624557\\
461	0.0120823743965509\\
462	0.012089427193184\\
463	0.0120965685366908\\
464	0.012103799308155\\
465	0.0121111203845674\\
466	0.0121185326379307\\
467	0.0121260369343065\\
468	0.0121336341327995\\
469	0.0121413250844764\\
470	0.012149110631216\\
471	0.0121569916044837\\
472	0.0121649688240296\\
473	0.0121730430965016\\
474	0.0121812152139712\\
475	0.0121894859523644\\
476	0.0121978560697927\\
477	0.0122063263047781\\
478	0.012214897374364\\
479	0.0122235699721055\\
480	0.0122323447659308\\
481	0.012241222395866\\
482	0.0122502034716117\\
483	0.0122592885699647\\
484	0.0122684782320714\\
485	0.0122777729605017\\
486	0.0122871732161331\\
487	0.0122966794148269\\
488	0.0123062919238869\\
489	0.0123160110582802\\
490	0.0123258370766045\\
491	0.0123357701767836\\
492	0.0123458104914685\\
493	0.0123559580831224\\
494	0.0123662129387656\\
495	0.0123765749643512\\
496	0.0123870439787446\\
497	0.0123976197072715\\
498	0.0124083017748015\\
499	0.0124190896983255\\
500	0.0124299828789861\\
501	0.0124409805935121\\
502	0.0124520819850047\\
503	0.0124632860530193\\
504	0.0124745916428773\\
505	0.0124859974341389\\
506	0.0124975019281584\\
507	0.0125091034346369\\
508	0.0125208000570761\\
509	0.0125325896770302\\
510	0.0125444699370366\\
511	0.0125564382221003\\
512	0.0125684916395874\\
513	0.0125806269973728\\
514	0.0125928407800687\\
515	0.0126051291231442\\
516	0.0126174877847281\\
517	0.0126299121148679\\
518	0.0126423970219956\\
519	0.0126549369363353\\
520	0.0126675257699616\\
521	0.0126801568732061\\
522	0.0126928229870882\\
523	0.0127055161914419\\
524	0.0127196267317596\\
525	0.0127337777619999\\
526	0.012747377540542\\
527	0.0127602292836677\\
528	0.0127727542762508\\
529	0.0127845821122983\\
530	0.0127959855671431\\
531	0.012807203733555\\
532	0.0128182626229069\\
533	0.0128291679587394\\
534	0.0128399046674119\\
535	0.0128504550386671\\
536	0.0128607967821975\\
537	0.0128709061162104\\
538	0.0128807570360296\\
539	0.0128903225124733\\
540	0.0129024979729618\\
541	0.0129162104251819\\
542	0.0129285405990742\\
543	0.0129400119605311\\
544	0.0129506631698186\\
545	0.0129602132373445\\
546	0.0129697686235692\\
547	0.0129794096850582\\
548	0.0129891740306582\\
549	0.0129990701938975\\
550	0.0130090977632239\\
551	0.0130192639639011\\
552	0.0130301758884645\\
553	0.013040920155061\\
554	0.0130513026381296\\
555	0.0130614236280406\\
556	0.0130716242256193\\
557	0.0130819211329755\\
558	0.0130923104237779\\
559	0.0131027877537724\\
560	0.013113349546112\\
561	0.0131239958493138\\
562	0.0131348838140754\\
563	0.0131458992837798\\
564	0.0131565457027411\\
565	0.0131670890149139\\
566	0.0131777102671135\\
567	0.0131884047440515\\
568	0.0131991670913618\\
569	0.0132099913534643\\
570	0.0132208709217061\\
571	0.0132317984770754\\
572	0.0132427659276113\\
573	0.0132537643399592\\
574	0.013264783864402\\
575	0.0132758136525899\\
576	0.0132868417670664\\
577	0.0132978550815396\\
578	0.0133088391706682\\
579	0.0133197781879291\\
580	0.013330654729907\\
581	0.0133414496851423\\
582	0.0133521420655987\\
583	0.0133627088191946\\
584	0.0133731246235461\\
585	0.0133833616664418\\
586	0.013393389433806\\
587	0.0134031745674478\\
588	0.0134126809661912\\
589	0.0134218705998152\\
590	0.0134307062886727\\
591	0.0134391597728698\\
592	0.0134472338645631\\
593	0.0134551345125145\\
594	0.0134637112677116\\
595	0.0134740407759367\\
596	0.0134889777516421\\
597	0.0135160557778789\\
598	0.0135751148335434\\
599	0\\
600	0\\
};
\end{axis}
\end{tikzpicture}% 
%  \caption{Discrete Time w/ nFPC}
%\end{subfigure}\\
%
%\leavevmode\smash{\makebox[0pt]{\hspace{-7em}% HORIZONTAL POSITION           
%  \rotatebox[origin=l]{90}{\hspace{20em}% VERTICAL POSITION
%    Depth $\delta^+$}%
%}}\hspace{0pt plus 1filll}\null
%
%Time (s)
%
%\vspace{1cm}
%\begin{subfigure}{\linewidth}
%  \centering
%  \tikzsetnextfilename{deltalegend}
%  \definecolor{mycolor1}{rgb}{1.00000,0.00000,1.00000}%
\begin{tikzpicture}[framed]
    \begingroup
    % inits/clears the lists (which might be populated from previous
    % axes):
    \csname pgfplots@init@cleared@structures\endcsname
    \pgfplotsset{legend style={at={(0,1)},anchor=north west},legend columns=-1,legend style={draw=none,column sep=1ex},legend entries={$q=-4$,$q=-3$,$q=-2$,$q=-1$}}%
    
    \csname pgfplots@addlegendimage\endcsname{thick,green,dashed,sharp plot}
    \csname pgfplots@addlegendimage\endcsname{thick,mycolor1,dashed,sharp plot}
    \csname pgfplots@addlegendimage\endcsname{thick,red,dashed,sharp plot}
    \csname pgfplots@addlegendimage\endcsname{thick,blue,dashed,sharp plot}

    % draws the legend:
    \csname pgfplots@createlegend\endcsname
    \endgroup

    \begingroup
    % inits/clears the lists (which might be populated from previous
    % axes):
    \csname pgfplots@init@cleared@structures\endcsname
    \pgfplotsset{legend style={at={(3.75,0.5)},anchor=north west},legend columns=-1,legend style={draw=none,column sep=1ex},legend entries={$q=0$}}%

    \csname pgfplots@addlegendimage\endcsname{thick,black,sharp plot}

    % draws the legend:
    \csname pgfplots@createlegend\endcsname
    \endgroup

    \begingroup
    % inits/clears the lists (which might be populated from previous
    % axes):
    \csname pgfplots@init@cleared@structures\endcsname
    \pgfplotsset{legend style={at={(0,0)},anchor=north west},legend columns=-1,legend style={draw=none,column sep=1ex},legend entries={$q=+4$,$q=+3$,$q=+2$,$q=+1$}}%
    
    \csname pgfplots@addlegendimage\endcsname{thick,green,sharp plot}
    \csname pgfplots@addlegendimage\endcsname{thick,mycolor1,sharp plot}
    \csname pgfplots@addlegendimage\endcsname{thick,red,sharp plot}
    \csname pgfplots@addlegendimage\endcsname{thick,blue,sharp plot}

    % draws the legend:
    \csname pgfplots@createlegend\endcsname
    \endgroup
\end{tikzpicture} 
%\end{subfigure}%
%  \caption{Optimal buy depths $\delta^{+}$ for Markov state $Z=(\rho = -1, \Delta S = -1)$, implying heavy imbalance in favor of sell pressure, and having previously seen a downward price change. We expect the midprice to fall.}
%  \label{fig:comp_dp_z1}
%\end{figure}
%
%\begin{figure}
%\centering
%\begin{subfigure}{.45\linewidth}
%  \centering
%  \setlength\figureheight{\linewidth} 
%  \setlength\figurewidth{\linewidth}
%  \tikzsetnextfilename{dp_cts_z8}
%  % This file was created by matlab2tikz.
%
%The latest updates can be retrieved from
%  http://www.mathworks.com/matlabcentral/fileexchange/22022-matlab2tikz-matlab2tikz
%where you can also make suggestions and rate matlab2tikz.
%
\definecolor{mycolor1}{rgb}{0.00000,1.00000,0.14286}%
\definecolor{mycolor2}{rgb}{0.00000,1.00000,0.28571}%
\definecolor{mycolor3}{rgb}{0.00000,1.00000,0.42857}%
\definecolor{mycolor4}{rgb}{0.00000,1.00000,0.57143}%
\definecolor{mycolor5}{rgb}{0.00000,1.00000,0.71429}%
\definecolor{mycolor6}{rgb}{0.00000,1.00000,0.85714}%
\definecolor{mycolor7}{rgb}{0.00000,1.00000,1.00000}%
\definecolor{mycolor8}{rgb}{0.00000,0.87500,1.00000}%
\definecolor{mycolor9}{rgb}{0.00000,0.62500,1.00000}%
\definecolor{mycolor10}{rgb}{0.12500,0.00000,1.00000}%
\definecolor{mycolor11}{rgb}{0.25000,0.00000,1.00000}%
\definecolor{mycolor12}{rgb}{0.37500,0.00000,1.00000}%
\definecolor{mycolor13}{rgb}{0.50000,0.00000,1.00000}%
\definecolor{mycolor14}{rgb}{0.62500,0.00000,1.00000}%
\definecolor{mycolor15}{rgb}{0.75000,0.00000,1.00000}%
\definecolor{mycolor16}{rgb}{0.87500,0.00000,1.00000}%
\definecolor{mycolor17}{rgb}{1.00000,0.00000,1.00000}%
\definecolor{mycolor18}{rgb}{1.00000,0.00000,0.87500}%
\definecolor{mycolor19}{rgb}{1.00000,0.00000,0.62500}%
\definecolor{mycolor20}{rgb}{0.85714,0.00000,0.00000}%
\definecolor{mycolor21}{rgb}{0.71429,0.00000,0.00000}%
%
\begin{tikzpicture}

\begin{axis}[%
width=4.1in,
height=3.803in,
at={(0.809in,0.513in)},
scale only axis,
point meta min=0,
point meta max=1,
every outer x axis line/.append style={black},
every x tick label/.append style={font=\color{black}},
xmin=0,
xmax=600,
every outer y axis line/.append style={black},
every y tick label/.append style={font=\color{black}},
ymin=0,
ymax=0.012,
axis background/.style={fill=white},
axis x line*=bottom,
axis y line*=left,
colormap={mymap}{[1pt] rgb(0pt)=(0,1,0); rgb(7pt)=(0,1,1); rgb(15pt)=(0,0,1); rgb(23pt)=(1,0,1); rgb(31pt)=(1,0,0); rgb(38pt)=(0,0,0)},
colorbar,
colorbar style={separate axis lines,every outer x axis line/.append style={black},every x tick label/.append style={font=\color{black}},every outer y axis line/.append style={black},every y tick label/.append style={font=\color{black}},yticklabels={{-19},{-17},{-15},{-13},{-11},{-9},{-7},{-5},{-3},{-1},{1},{3},{5},{7},{9},{11},{13},{15},{17},{19}}}
]
\addplot [color=green,solid,forget plot]
  table[row sep=crcr]{%
0.01	0.00496798049573811\\
1.01	0.00496797956375651\\
2.01	0.00496797861304558\\
3.01	0.00496797764323127\\
4.01	0.00496797665393224\\
5.01	0.00496797564475905\\
6.01	0.00496797461531515\\
7.01	0.00496797356519586\\
8.01	0.00496797249398832\\
9.01	0.00496797140127171\\
10.01	0.00496797028661687\\
11.01	0.00496796914958604\\
12.01	0.00496796798973329\\
13.01	0.00496796680660276\\
14.01	0.00496796559973101\\
15.01	0.00496796436864398\\
16.01	0.00496796311285933\\
17.01	0.00496796183188498\\
18.01	0.0049679605252184\\
19.01	0.0049679591923482\\
20.01	0.00496795783275209\\
21.01	0.00496795644589804\\
22.01	0.00496795503124293\\
23.01	0.0049679535882333\\
24.01	0.00496795211630469\\
25.01	0.00496795061488138\\
26.01	0.00496794908337656\\
27.01	0.00496794752119143\\
28.01	0.00496794592771586\\
29.01	0.00496794430232737\\
30.01	0.00496794264439125\\
31.01	0.00496794095326062\\
32.01	0.00496793922827536\\
33.01	0.00496793746876279\\
34.01	0.00496793567403623\\
35.01	0.00496793384339637\\
36.01	0.00496793197612981\\
37.01	0.00496793007150893\\
38.01	0.00496792812879222\\
39.01	0.00496792614722328\\
40.01	0.00496792412603048\\
41.01	0.00496792206442768\\
42.01	0.00496791996161281\\
43.01	0.00496791781676843\\
44.01	0.00496791562906102\\
45.01	0.00496791339764023\\
46.01	0.0049679111216395\\
47.01	0.00496790880017509\\
48.01	0.00496790643234572\\
49.01	0.00496790401723289\\
50.01	0.00496790155389993\\
51.01	0.00496789904139095\\
52.01	0.00496789647873287\\
53.01	0.00496789386493225\\
54.01	0.00496789119897662\\
55.01	0.00496788847983418\\
56.01	0.00496788570645216\\
57.01	0.00496788287775749\\
58.01	0.00496787999265634\\
59.01	0.00496787705003291\\
60.01	0.0049678740487503\\
61.01	0.00496787098764873\\
62.01	0.00496786786554605\\
63.01	0.00496786468123687\\
64.01	0.00496786143349241\\
65.01	0.00496785812106\\
66.01	0.0049678547426623\\
67.01	0.00496785129699696\\
68.01	0.00496784778273635\\
69.01	0.00496784419852676\\
70.01	0.00496784054298853\\
71.01	0.00496783681471445\\
72.01	0.00496783301227056\\
73.01	0.00496782913419489\\
74.01	0.00496782517899622\\
75.01	0.00496782114515502\\
76.01	0.0049678170311221\\
77.01	0.00496781283531784\\
78.01	0.00496780855613233\\
79.01	0.00496780419192364\\
80.01	0.00496779974101881\\
81.01	0.00496779520171157\\
82.01	0.00496779057226297\\
83.01	0.0049677858509\\
84.01	0.00496778103581556\\
85.01	0.00496777612516723\\
86.01	0.00496777111707687\\
87.01	0.00496776600963035\\
88.01	0.00496776080087545\\
89.01	0.00496775548882287\\
90.01	0.00496775007144455\\
91.01	0.00496774454667329\\
92.01	0.00496773891240172\\
93.01	0.00496773316648137\\
94.01	0.00496772730672305\\
95.01	0.00496772133089444\\
96.01	0.00496771523672025\\
97.01	0.00496770902188127\\
98.01	0.00496770268401348\\
99.01	0.00496769622070757\\
100.01	0.00496768962950684\\
101.01	0.00496768290790804\\
102.01	0.00496767605335922\\
103.01	0.00496766906325914\\
104.01	0.00496766193495661\\
105.01	0.00496765466574904\\
106.01	0.00496764725288199\\
107.01	0.00496763969354833\\
108.01	0.00496763198488663\\
109.01	0.00496762412398001\\
110.01	0.00496761610785605\\
111.01	0.00496760793348489\\
112.01	0.00496759959777849\\
113.01	0.00496759109758947\\
114.01	0.00496758242970983\\
115.01	0.00496757359087045\\
116.01	0.00496756457773918\\
117.01	0.00496755538691985\\
118.01	0.00496754601495179\\
119.01	0.00496753645830684\\
120.01	0.00496752671339102\\
121.01	0.0049675167765396\\
122.01	0.00496750664401897\\
123.01	0.00496749631202413\\
124.01	0.00496748577667636\\
125.01	0.00496747503402365\\
126.01	0.00496746408003841\\
127.01	0.00496745291061582\\
128.01	0.00496744152157287\\
129.01	0.00496742990864672\\
130.01	0.00496741806749294\\
131.01	0.00496740599368548\\
132.01	0.00496739368271205\\
133.01	0.0049673811299762\\
134.01	0.00496736833079335\\
135.01	0.0049673552803895\\
136.01	0.0049673419738999\\
137.01	0.00496732840636764\\
138.01	0.00496731457274186\\
139.01	0.00496730046787498\\
140.01	0.00496728608652287\\
141.01	0.00496727142334064\\
142.01	0.00496725647288308\\
143.01	0.00496724122960136\\
144.01	0.00496722568784165\\
145.01	0.00496720984184302\\
146.01	0.00496719368573574\\
147.01	0.00496717721353888\\
148.01	0.00496716041915841\\
149.01	0.00496714329638557\\
150.01	0.0049671258388939\\
151.01	0.00496710804023749\\
152.01	0.00496708989384982\\
153.01	0.00496707139303954\\
154.01	0.00496705253098943\\
155.01	0.00496703330075438\\
156.01	0.00496701369525813\\
157.01	0.00496699370729182\\
158.01	0.00496697332951061\\
159.01	0.00496695255443192\\
160.01	0.0049669313744327\\
161.01	0.00496690978174658\\
162.01	0.00496688776846208\\
163.01	0.00496686532651897\\
164.01	0.00496684244770588\\
165.01	0.00496681912365849\\
166.01	0.00496679534585518\\
167.01	0.00496677110561573\\
168.01	0.00496674639409729\\
169.01	0.00496672120229214\\
170.01	0.00496669552102431\\
171.01	0.00496666934094662\\
172.01	0.00496664265253744\\
173.01	0.00496661544609829\\
174.01	0.00496658771175008\\
175.01	0.00496655943942979\\
176.01	0.00496653061888745\\
177.01	0.00496650123968218\\
178.01	0.00496647129118009\\
179.01	0.00496644076254919\\
180.01	0.00496640964275722\\
181.01	0.00496637792056726\\
182.01	0.00496634558453441\\
183.01	0.00496631262300174\\
184.01	0.00496627902409728\\
185.01	0.00496624477572911\\
186.01	0.00496620986558253\\
187.01	0.00496617428111516\\
188.01	0.0049661380095539\\
189.01	0.00496610103788987\\
190.01	0.00496606335287454\\
191.01	0.00496602494101573\\
192.01	0.00496598578857367\\
193.01	0.00496594588155548\\
194.01	0.0049659052057117\\
195.01	0.0049658637465313\\
196.01	0.00496582148923722\\
197.01	0.00496577841878199\\
198.01	0.00496573451984251\\
199.01	0.00496568977681478\\
200.01	0.00496564417381083\\
201.01	0.00496559769465164\\
202.01	0.00496555032286284\\
203.01	0.00496550204167005\\
204.01	0.00496545283399281\\
205.01	0.00496540268243993\\
206.01	0.00496535156930348\\
207.01	0.00496529947655403\\
208.01	0.00496524638583416\\
209.01	0.00496519227845361\\
210.01	0.00496513713538261\\
211.01	0.00496508093724666\\
212.01	0.00496502366432056\\
213.01	0.00496496529652169\\
214.01	0.00496490581340471\\
215.01	0.00496484519415411\\
216.01	0.00496478341757959\\
217.01	0.00496472046210784\\
218.01	0.00496465630577645\\
219.01	0.00496459092622768\\
220.01	0.00496452430070119\\
221.01	0.00496445640602756\\
222.01	0.00496438721862081\\
223.01	0.00496431671447074\\
224.01	0.00496424486913639\\
225.01	0.0049641716577395\\
226.01	0.00496409705495461\\
227.01	0.00496402103500305\\
228.01	0.00496394357164527\\
229.01	0.00496386463817209\\
230.01	0.00496378420739691\\
231.01	0.00496370225164814\\
232.01	0.00496361874275967\\
233.01	0.00496353365206307\\
234.01	0.00496344695037963\\
235.01	0.0049633586080099\\
236.01	0.00496326859472613\\
237.01	0.00496317687976287\\
238.01	0.00496308343180763\\
239.01	0.00496298821899086\\
240.01	0.0049628912088775\\
241.01	0.00496279236845641\\
242.01	0.00496269166413039\\
243.01	0.00496258906170711\\
244.01	0.00496248452638727\\
245.01	0.0049623780227551\\
246.01	0.00496226951476748\\
247.01	0.00496215896574246\\
248.01	0.00496204633834905\\
249.01	0.00496193159459572\\
250.01	0.00496181469581834\\
251.01	0.00496169560266875\\
252.01	0.00496157427510256\\
253.01	0.00496145067236737\\
254.01	0.00496132475298975\\
255.01	0.00496119647476285\\
256.01	0.0049610657947333\\
257.01	0.00496093266918805\\
258.01	0.00496079705364012\\
259.01	0.00496065890281581\\
260.01	0.00496051817063967\\
261.01	0.00496037481022158\\
262.01	0.00496022877383926\\
263.01	0.00496008001292559\\
264.01	0.00495992847805255\\
265.01	0.00495977411891441\\
266.01	0.00495961688431321\\
267.01	0.00495945672214072\\
268.01	0.0049592935793629\\
269.01	0.00495912740200139\\
270.01	0.00495895813511622\\
271.01	0.00495878572278778\\
272.01	0.00495861010809757\\
273.01	0.00495843123310991\\
274.01	0.00495824903885207\\
275.01	0.00495806346529388\\
276.01	0.00495787445132729\\
277.01	0.00495768193474498\\
278.01	0.00495748585221893\\
279.01	0.00495728613927838\\
280.01	0.00495708273028576\\
281.01	0.00495687555841527\\
282.01	0.00495666455562591\\
283.01	0.00495644965263901\\
284.01	0.00495623077891117\\
285.01	0.00495600786260931\\
286.01	0.00495578083058172\\
287.01	0.0049555496083324\\
288.01	0.00495531411999132\\
289.01	0.00495507428828458\\
290.01	0.00495483003450576\\
291.01	0.00495458127848298\\
292.01	0.00495432793854798\\
293.01	0.00495406993150228\\
294.01	0.00495380717258449\\
295.01	0.0049535395754343\\
296.01	0.00495326705205668\\
297.01	0.00495298951278534\\
298.01	0.00495270686624409\\
299.01	0.00495241901930786\\
300.01	0.00495212587706293\\
301.01	0.00495182734276457\\
302.01	0.00495152331779542\\
303.01	0.00495121370162072\\
304.01	0.00495089839174374\\
305.01	0.00495057728365992\\
306.01	0.00495025027080923\\
307.01	0.00494991724452621\\
308.01	0.00494957809399203\\
309.01	0.0049492327061817\\
310.01	0.0049488809658119\\
311.01	0.00494852275528771\\
312.01	0.00494815795464624\\
313.01	0.00494778644150185\\
314.01	0.00494740809098738\\
315.01	0.00494702277569646\\
316.01	0.0049466303656223\\
317.01	0.00494623072809781\\
318.01	0.00494582372773187\\
319.01	0.00494540922634768\\
320.01	0.00494498708291742\\
321.01	0.00494455715349724\\
322.01	0.00494411929116143\\
323.01	0.00494367334593533\\
324.01	0.00494321916472771\\
325.01	0.00494275659126265\\
326.01	0.00494228546601095\\
327.01	0.00494180562612141\\
328.01	0.00494131690535014\\
329.01	0.00494081913399276\\
330.01	0.00494031213881358\\
331.01	0.00493979574297719\\
332.01	0.00493926976597879\\
333.01	0.00493873402357512\\
334.01	0.00493818832771682\\
335.01	0.00493763248648003\\
336.01	0.00493706630400057\\
337.01	0.00493648958040672\\
338.01	0.00493590211175477\\
339.01	0.0049353036899649\\
340.01	0.00493469410275783\\
341.01	0.00493407313359368\\
342.01	0.00493344056161075\\
343.01	0.00493279616156606\\
344.01	0.00493213970377699\\
345.01	0.00493147095406264\\
346.01	0.00493078967368797\\
347.01	0.00493009561930683\\
348.01	0.00492938854290562\\
349.01	0.00492866819174676\\
350.01	0.00492793430831218\\
351.01	0.00492718663024442\\
352.01	0.00492642489028781\\
353.01	0.00492564881622632\\
354.01	0.00492485813081873\\
355.01	0.00492405255172952\\
356.01	0.00492323179145733\\
357.01	0.00492239555725589\\
358.01	0.00492154355104994\\
359.01	0.00492067546934425\\
360.01	0.00491979100312418\\
361.01	0.00491888983774789\\
362.01	0.00491797165283082\\
363.01	0.00491703612211642\\
364.01	0.00491608291334048\\
365.01	0.00491511168808271\\
366.01	0.00491412210160697\\
367.01	0.00491311380269122\\
368.01	0.00491208643344649\\
369.01	0.00491103962912593\\
370.01	0.00490997301792388\\
371.01	0.0049088862207663\\
372.01	0.00490777885109379\\
373.01	0.00490665051463788\\
374.01	0.0049055008091922\\
375.01	0.00490432932437882\\
376.01	0.00490313564141327\\
377.01	0.0049019193328663\\
378.01	0.00490067996242325\\
379.01	0.00489941708464335\\
380.01	0.00489813024471656\\
381.01	0.00489681897821812\\
382.01	0.00489548281086003\\
383.01	0.00489412125823797\\
384.01	0.00489273382557299\\
385.01	0.0048913200074467\\
386.01	0.00488987928753022\\
387.01	0.00488841113830419\\
388.01	0.00488691502077374\\
389.01	0.00488539038417344\\
390.01	0.00488383666566656\\
391.01	0.00488225329003651\\
392.01	0.00488063966936906\\
393.01	0.00487899520272906\\
394.01	0.00487731927582761\\
395.01	0.00487561126068354\\
396.01	0.00487387051527483\\
397.01	0.0048720963831845\\
398.01	0.00487028819323763\\
399.01	0.00486844525913183\\
400.01	0.00486656687906007\\
401.01	0.00486465233532584\\
402.01	0.0048627008939519\\
403.01	0.00486071180428182\\
404.01	0.00485868429857506\\
405.01	0.0048566175915949\\
406.01	0.00485451088019094\\
407.01	0.00485236334287564\\
408.01	0.00485017413939375\\
409.01	0.00484794241028806\\
410.01	0.00484566727645835\\
411.01	0.00484334783871655\\
412.01	0.00484098317733674\\
413.01	0.0048385723516004\\
414.01	0.00483611439933853\\
415.01	0.00483360833646877\\
416.01	0.00483105315652974\\
417.01	0.00482844783021164\\
418.01	0.0048257913048833\\
419.01	0.00482308250411575\\
420.01	0.00482032032720391\\
421.01	0.00481750364868409\\
422.01	0.00481463131784795\\
423.01	0.00481170215825402\\
424.01	0.00480871496723511\\
425.01	0.0048056685154016\\
426.01	0.0048025615461401\\
427.01	0.00479939277510788\\
428.01	0.00479616088972059\\
429.01	0.00479286454863443\\
430.01	0.00478950238122022\\
431.01	0.0047860729870299\\
432.01	0.0047825749352528\\
433.01	0.00477900676416195\\
434.01	0.00477536698054749\\
435.01	0.00477165405913904\\
436.01	0.00476786644201079\\
437.01	0.00476400253797278\\
438.01	0.0047600607219439\\
439.01	0.0047560393343079\\
440.01	0.00475193668024742\\
441.01	0.00474775102905966\\
442.01	0.0047434806134502\\
443.01	0.00473912362880386\\
444.01	0.00473467823243368\\
445.01	0.00473014254280768\\
446.01	0.00472551463875348\\
447.01	0.00472079255864059\\
448.01	0.0047159742995414\\
449.01	0.00471105781637388\\
450.01	0.00470604102102439\\
451.01	0.00470092178145423\\
452.01	0.00469569792079023\\
453.01	0.00469036721640279\\
454.01	0.00468492739897068\\
455.01	0.00467937615153474\\
456.01	0.00467371110854378\\
457.01	0.00466792985488893\\
458.01	0.00466202992492993\\
459.01	0.00465600880151068\\
460.01	0.00464986391496419\\
461.01	0.00464359264210391\\
462.01	0.0046371923051995\\
463.01	0.00463066017093431\\
464.01	0.0046239934493412\\
465.01	0.00461718929271441\\
466.01	0.00461024479449396\\
467.01	0.00460315698812097\\
468.01	0.00459592284586204\\
469.01	0.00458853927760013\\
470.01	0.00458100312959322\\
471.01	0.00457331118319749\\
472.01	0.00456546015355634\\
473.01	0.0045574466882538\\
474.01	0.00454926736593208\\
475.01	0.00454091869487459\\
476.01	0.00453239711155208\\
477.01	0.00452369897913415\\
478.01	0.0045148205859658\\
479.01	0.00450575814400844\\
480.01	0.00449650778724681\\
481.01	0.00448706557006175\\
482.01	0.00447742746556782\\
483.01	0.00446758936391779\\
484.01	0.00445754707057272\\
485.01	0.00444729630453656\\
486.01	0.00443683269655646\\
487.01	0.00442615178728698\\
488.01	0.00441524902541696\\
489.01	0.00440411976576088\\
490.01	0.00439275926730926\\
491.01	0.0043811626912425\\
492.01	0.00436932509890255\\
493.01	0.00435724144972625\\
494.01	0.00434490659913631\\
495.01	0.00433231529639334\\
496.01	0.00431946218240616\\
497.01	0.00430634178750338\\
498.01	0.00429294852916503\\
499.01	0.00427927670971572\\
500.01	0.00426532051398058\\
501.01	0.0042510740069029\\
502.01	0.0042365311311274\\
503.01	0.00422168570454761\\
504.01	0.00420653141781825\\
505.01	0.00419106183183649\\
506.01	0.00417527037519092\\
507.01	0.00415915034157993\\
508.01	0.0041426948872035\\
509.01	0.00412589702812783\\
510.01	0.00410874963762732\\
511.01	0.00409124544350514\\
512.01	0.00407337702539731\\
513.01	0.00405513681206413\\
514.01	0.00403651707867021\\
515.01	0.00401750994406269\\
516.01	0.00399810736804918\\
517.01	0.00397830114868427\\
518.01	0.00395808291956913\\
519.01	0.003937444147174\\
520.01	0.00391637612819088\\
521.01	0.00389486998692621\\
522.01	0.00387291667274421\\
523.01	0.0038505069575732\\
524.01	0.00382763143348763\\
525.01	0.00380428051038099\\
526.01	0.00378044441374601\\
527.01	0.0037561131825801\\
528.01	0.0037312766674369\\
529.01	0.00370592452864577\\
530.01	0.0036800462347252\\
531.01	0.0036536310610164\\
532.01	0.00362666808856926\\
533.01	0.00359914620331311\\
534.01	0.00357105409555154\\
535.01	0.00354238025982046\\
536.01	0.00351311299515762\\
537.01	0.00348324040583206\\
538.01	0.0034527504025914\\
539.01	0.003421630704487\\
540.01	0.0033898688413456\\
541.01	0.00335745215696268\\
542.01	0.00332436781309979\\
543.01	0.00329060279437702\\
544.01	0.00325614391416088\\
545.01	0.00322097782155683\\
546.01	0.00318509100962871\\
547.01	0.00314846982497814\\
548.01	0.00311110047882896\\
549.01	0.0030729690597782\\
550.01	0.00303406154838843\\
551.01	0.00299436383381546\\
552.01	0.00295386173268076\\
553.01	0.0029125410104198\\
554.01	0.00287038740535795\\
555.01	0.00282738665578867\\
556.01	0.00278352453035239\\
557.01	0.00273878686204152\\
558.01	0.00269315958618417\\
559.01	0.00264662878278779\\
560.01	0.00259918072365662\\
561.01	0.00255080192472444\\
562.01	0.00250147920408223\\
563.01	0.00245119974620713\\
564.01	0.0023999511729366\\
565.01	0.00234772162175995\\
566.01	0.0022944998320303\\
567.01	0.00224027523972262\\
568.01	0.00218503808138668\\
569.01	0.00212877950795229\\
570.01	0.00207149170904279\\
571.01	0.00201316804844078\\
572.01	0.00195380321131022\\
573.01	0.00189339336371602\\
574.01	0.00183193632488497\\
575.01	0.0017694317525057\\
576.01	0.00170588134116366\\
577.01	0.00164128903373116\\
578.01	0.00157566124516161\\
579.01	0.00150900709764865\\
580.01	0.00144133866547398\\
581.01	0.001372671227043\\
582.01	0.00130302352055003\\
583.01	0.00123241799836546\\
584.01	0.00116088107352443\\
585.01	0.00108844334953209\\
586.01	0.00101513982197411\\
587.01	0.000941010036997375\\
588.01	0.000866098187439544\\
589.01	0.000790453122029416\\
590.01	0.000714128236401343\\
591.01	0.000637181206352299\\
592.01	0.000559673513435248\\
593.01	0.000481669700155193\\
594.01	0.000403236276127809\\
595.01	0.00032444017686022\\
596.01	0.000245346652435533\\
597.01	0.000166140574224238\\
598.01	9.1337981519295e-05\\
599.01	2.91271958372443e-05\\
599.02	2.86192783627518e-05\\
599.03	2.81144237420042e-05\\
599.04	2.76126617978888e-05\\
599.05	2.71140226472798e-05\\
599.06	2.66185367039599e-05\\
599.07	2.61262346815533e-05\\
599.08	2.56371475965082e-05\\
599.09	2.51513067710835e-05\\
599.1	2.46687438363886e-05\\
599.11	2.41894907354479e-05\\
599.12	2.37135797262807e-05\\
599.13	2.3241043385025e-05\\
599.14	2.27719146091033e-05\\
599.15	2.23062266203888e-05\\
599.16	2.18440129684284e-05\\
599.17	2.13853075336917e-05\\
599.18	2.09301445308497e-05\\
599.19	2.04785585120812e-05\\
599.2	2.00305843704295e-05\\
599.21	1.95862573431627e-05\\
599.22	1.91456130152028e-05\\
599.23	1.87086873225627e-05\\
599.24	1.82755165558171e-05\\
599.25	1.7846137363638e-05\\
599.26	1.74205874164668e-05\\
599.27	1.69989072017502e-05\\
599.28	1.65811376101922e-05\\
599.29	1.61673199397579e-05\\
599.3	1.57574958996824e-05\\
599.31	1.53517076145714e-05\\
599.32	1.49499976284904e-05\\
599.33	1.45524089091333e-05\\
599.34	1.41589848520092e-05\\
599.35	1.37697692846814e-05\\
599.36	1.33848064710444e-05\\
599.37	1.30041411156422e-05\\
599.38	1.26278183680325e-05\\
599.39	1.22558838271912e-05\\
599.4	1.18883835459674e-05\\
599.41	1.15253640355657e-05\\
599.42	1.11668722700981e-05\\
599.43	1.08129556911519e-05\\
599.44	1.04636622124312e-05\\
599.45	1.01190402244222e-05\\
599.46	9.77913859911625e-06\\
599.47	9.44400669477576e-06\\
599.48	9.11369436074755e-06\\
599.49	8.7882519423238e-06\\
599.5	8.46773028565471e-06\\
599.51	8.15218074270464e-06\\
599.52	7.84165517625675e-06\\
599.53	7.5362059649732e-06\\
599.54	7.23588600849874e-06\\
599.55	6.94074873261973e-06\\
599.56	6.65084809447353e-06\\
599.57	6.36623858780126e-06\\
599.58	6.08697524826993e-06\\
599.59	5.81311365882228e-06\\
599.6	5.54470995511175e-06\\
599.61	5.28182083095637e-06\\
599.62	5.02450354387431e-06\\
599.63	4.7728159206558e-06\\
599.64	4.52681636300446e-06\\
599.65	4.28656385322197e-06\\
599.66	4.05211795995869e-06\\
599.67	3.82353884401457e-06\\
599.68	3.60088726420078e-06\\
599.69	3.38422458325341e-06\\
599.7	3.17361277382168e-06\\
599.71	2.96911442449442e-06\\
599.72	2.7707927458924e-06\\
599.73	2.57871157684567e-06\\
599.74	2.39293539058306e-06\\
599.75	2.21352930102926e-06\\
599.76	2.04055906913997e-06\\
599.77	1.87409110929959e-06\\
599.78	1.71419249580043e-06\\
599.79	1.56093096936177e-06\\
599.8	1.41437494372877e-06\\
599.81	1.27459351233379e-06\\
599.82	1.14165645502366e-06\\
599.83	1.01563424484766e-06\\
599.84	8.96598054920053e-07\\
599.85	7.84619765350353e-07\\
599.86	6.79771970232487e-07\\
599.87	5.82127984719016e-07\\
599.88	4.91761852147374e-07\\
599.89	4.08748351251112e-07\\
599.9	3.33163003438802e-07\\
599.91	2.6508208013365e-07\\
599.92	2.04582610201579e-07\\
599.93	1.51742387452178e-07\\
599.94	1.06639978191686e-07\\
599.95	6.93547288800611e-08\\
599.96	3.99667738487652e-08\\
599.97	1.85570430896731e-08\\
599.98	5.20727013418598e-09\\
599.99	0\\
600	0\\
};
\addplot [color=mycolor1,solid,forget plot]
  table[row sep=crcr]{%
0.01	0.00497879494448985\\
1.01	0.00497879398501761\\
2.01	0.00497879300616703\\
3.01	0.00497879200754837\\
4.01	0.00497879098876431\\
5.01	0.00497878994940958\\
6.01	0.00497878888907078\\
7.01	0.00497878780732623\\
8.01	0.00497878670374596\\
9.01	0.00497878557789113\\
10.01	0.0049787844293143\\
11.01	0.00497878325755954\\
12.01	0.00497878206216091\\
13.01	0.0049787808426441\\
14.01	0.00497877959852446\\
15.01	0.00497877832930862\\
16.01	0.00497877703449259\\
17.01	0.00497877571356219\\
18.01	0.00497877436599405\\
19.01	0.00497877299125341\\
20.01	0.00497877158879522\\
21.01	0.00497877015806326\\
22.01	0.00497876869849081\\
23.01	0.00497876720949955\\
24.01	0.00497876569049932\\
25.01	0.00497876414088867\\
26.01	0.00497876256005391\\
27.01	0.00497876094736925\\
28.01	0.00497875930219646\\
29.01	0.00497875762388458\\
30.01	0.0049787559117697\\
31.01	0.00497875416517459\\
32.01	0.00497875238340865\\
33.01	0.00497875056576746\\
34.01	0.00497874871153282\\
35.01	0.00497874681997224\\
36.01	0.00497874489033824\\
37.01	0.00497874292186881\\
38.01	0.00497874091378695\\
39.01	0.00497873886529973\\
40.01	0.00497873677559916\\
41.01	0.00497873464386062\\
42.01	0.00497873246924343\\
43.01	0.00497873025089011\\
44.01	0.00497872798792607\\
45.01	0.00497872567945961\\
46.01	0.00497872332458104\\
47.01	0.00497872092236274\\
48.01	0.00497871847185914\\
49.01	0.00497871597210512\\
50.01	0.00497871342211686\\
51.01	0.00497871082089129\\
52.01	0.00497870816740492\\
53.01	0.0049787054606144\\
54.01	0.00497870269945564\\
55.01	0.00497869988284319\\
56.01	0.00497869700967044\\
57.01	0.00497869407880886\\
58.01	0.00497869108910736\\
59.01	0.00497868803939233\\
60.01	0.00497868492846677\\
61.01	0.00497868175510988\\
62.01	0.00497867851807716\\
63.01	0.00497867521609914\\
64.01	0.00497867184788143\\
65.01	0.00497866841210389\\
66.01	0.00497866490742053\\
67.01	0.0049786613324587\\
68.01	0.00497865768581844\\
69.01	0.00497865396607265\\
70.01	0.00497865017176542\\
71.01	0.00497864630141282\\
72.01	0.00497864235350094\\
73.01	0.00497863832648619\\
74.01	0.004978634218795\\
75.01	0.00497863002882242\\
76.01	0.00497862575493193\\
77.01	0.00497862139545481\\
78.01	0.00497861694868921\\
79.01	0.00497861241290012\\
80.01	0.00497860778631811\\
81.01	0.00497860306713931\\
82.01	0.00497859825352378\\
83.01	0.00497859334359591\\
84.01	0.00497858833544299\\
85.01	0.00497858322711446\\
86.01	0.00497857801662163\\
87.01	0.00497857270193688\\
88.01	0.00497856728099273\\
89.01	0.00497856175168075\\
90.01	0.00497855611185149\\
91.01	0.00497855035931316\\
92.01	0.00497854449183084\\
93.01	0.00497853850712605\\
94.01	0.00497853240287508\\
95.01	0.00497852617670935\\
96.01	0.00497851982621329\\
97.01	0.00497851334892435\\
98.01	0.00497850674233157\\
99.01	0.0049785000038747\\
100.01	0.00497849313094353\\
101.01	0.00497848612087682\\
102.01	0.00497847897096102\\
103.01	0.00497847167842962\\
104.01	0.00497846424046198\\
105.01	0.00497845665418268\\
106.01	0.00497844891665952\\
107.01	0.00497844102490324\\
108.01	0.0049784329758661\\
109.01	0.00497842476644101\\
110.01	0.00497841639346018\\
111.01	0.00497840785369413\\
112.01	0.00497839914385016\\
113.01	0.00497839026057127\\
114.01	0.0049783812004354\\
115.01	0.00497837195995362\\
116.01	0.0049783625355689\\
117.01	0.00497835292365508\\
118.01	0.00497834312051534\\
119.01	0.00497833312238114\\
120.01	0.00497832292540997\\
121.01	0.00497831252568563\\
122.01	0.0049783019192149\\
123.01	0.00497829110192724\\
124.01	0.00497828006967328\\
125.01	0.00497826881822276\\
126.01	0.00497825734326337\\
127.01	0.00497824564039922\\
128.01	0.00497823370514904\\
129.01	0.00497822153294485\\
130.01	0.00497820911913003\\
131.01	0.00497819645895762\\
132.01	0.00497818354758924\\
133.01	0.00497817038009204\\
134.01	0.004978156951438\\
135.01	0.00497814325650218\\
136.01	0.00497812929006048\\
137.01	0.00497811504678777\\
138.01	0.00497810052125556\\
139.01	0.00497808570793124\\
140.01	0.00497807060117495\\
141.01	0.0049780551952385\\
142.01	0.00497803948426223\\
143.01	0.0049780234622741\\
144.01	0.00497800712318649\\
145.01	0.00497799046079498\\
146.01	0.0049779734687755\\
147.01	0.00497795614068234\\
148.01	0.00497793846994579\\
149.01	0.00497792044986987\\
150.01	0.00497790207363006\\
151.01	0.00497788333427072\\
152.01	0.00497786422470232\\
153.01	0.00497784473770007\\
154.01	0.00497782486590024\\
155.01	0.00497780460179779\\
156.01	0.00497778393774449\\
157.01	0.00497776286594497\\
158.01	0.0049777413784551\\
159.01	0.00497771946717878\\
160.01	0.0049776971238654\\
161.01	0.0049776743401063\\
162.01	0.00497765110733239\\
163.01	0.0049776274168112\\
164.01	0.00497760325964394\\
165.01	0.00497757862676163\\
166.01	0.00497755350892302\\
167.01	0.00497752789671095\\
168.01	0.00497750178052877\\
169.01	0.00497747515059727\\
170.01	0.00497744799695194\\
171.01	0.00497742030943834\\
172.01	0.00497739207771055\\
173.01	0.00497736329122509\\
174.01	0.00497733393923935\\
175.01	0.00497730401080691\\
176.01	0.00497727349477458\\
177.01	0.00497724237977778\\
178.01	0.00497721065423745\\
179.01	0.00497717830635548\\
180.01	0.0049771453241114\\
181.01	0.00497711169525796\\
182.01	0.00497707740731687\\
183.01	0.00497704244757503\\
184.01	0.0049770068030798\\
185.01	0.00497697046063547\\
186.01	0.00497693340679786\\
187.01	0.00497689562787066\\
188.01	0.00497685710990063\\
189.01	0.00497681783867275\\
190.01	0.00497677779970569\\
191.01	0.00497673697824758\\
192.01	0.00497669535927036\\
193.01	0.00497665292746503\\
194.01	0.00497660966723724\\
195.01	0.00497656556270144\\
196.01	0.00497652059767655\\
197.01	0.00497647475567963\\
198.01	0.00497642801992149\\
199.01	0.00497638037330122\\
200.01	0.00497633179839974\\
201.01	0.0049762822774752\\
202.01	0.00497623179245793\\
203.01	0.00497618032494245\\
204.01	0.00497612785618366\\
205.01	0.00497607436709\\
206.01	0.00497601983821763\\
207.01	0.00497596424976436\\
208.01	0.00497590758156299\\
209.01	0.00497584981307584\\
210.01	0.00497579092338769\\
211.01	0.00497573089119943\\
212.01	0.00497566969482138\\
213.01	0.00497560731216686\\
214.01	0.00497554372074528\\
215.01	0.00497547889765502\\
216.01	0.00497541281957636\\
217.01	0.00497534546276491\\
218.01	0.00497527680304393\\
219.01	0.00497520681579736\\
220.01	0.00497513547596213\\
221.01	0.00497506275802024\\
222.01	0.00497498863599211\\
223.01	0.00497491308342803\\
224.01	0.00497483607340115\\
225.01	0.00497475757849792\\
226.01	0.00497467757081176\\
227.01	0.00497459602193391\\
228.01	0.00497451290294537\\
229.01	0.00497442818440835\\
230.01	0.0049743418363581\\
231.01	0.00497425382829347\\
232.01	0.00497416412916902\\
233.01	0.0049740727073859\\
234.01	0.00497397953078169\\
235.01	0.0049738845666235\\
236.01	0.00497378778159678\\
237.01	0.00497368914179664\\
238.01	0.00497358861271796\\
239.01	0.00497348615924673\\
240.01	0.00497338174564912\\
241.01	0.00497327533556214\\
242.01	0.00497316689198343\\
243.01	0.00497305637726126\\
244.01	0.00497294375308392\\
245.01	0.00497282898046944\\
246.01	0.00497271201975529\\
247.01	0.00497259283058733\\
248.01	0.00497247137190871\\
249.01	0.00497234760194902\\
250.01	0.00497222147821346\\
251.01	0.00497209295747108\\
252.01	0.00497196199574329\\
253.01	0.00497182854829224\\
254.01	0.00497169256960917\\
255.01	0.00497155401340257\\
256.01	0.00497141283258575\\
257.01	0.00497126897926411\\
258.01	0.00497112240472345\\
259.01	0.00497097305941658\\
260.01	0.004970820892951\\
261.01	0.00497066585407497\\
262.01	0.00497050789066531\\
263.01	0.00497034694971283\\
264.01	0.00497018297730925\\
265.01	0.00497001591863312\\
266.01	0.00496984571793539\\
267.01	0.00496967231852515\\
268.01	0.00496949566275464\\
269.01	0.00496931569200448\\
270.01	0.00496913234666865\\
271.01	0.00496894556613838\\
272.01	0.00496875528878706\\
273.01	0.00496856145195312\\
274.01	0.00496836399192393\\
275.01	0.00496816284391918\\
276.01	0.00496795794207284\\
277.01	0.00496774921941665\\
278.01	0.00496753660786089\\
279.01	0.00496732003817638\\
280.01	0.00496709943997587\\
281.01	0.00496687474169378\\
282.01	0.00496664587056702\\
283.01	0.0049664127526138\\
284.01	0.00496617531261286\\
285.01	0.00496593347408155\\
286.01	0.00496568715925409\\
287.01	0.00496543628905751\\
288.01	0.00496518078308884\\
289.01	0.00496492055959027\\
290.01	0.00496465553542352\\
291.01	0.00496438562604442\\
292.01	0.00496411074547468\\
293.01	0.0049638308062757\\
294.01	0.00496354571951754\\
295.01	0.00496325539475028\\
296.01	0.00496295973997257\\
297.01	0.00496265866159888\\
298.01	0.0049623520644263\\
299.01	0.00496203985160015\\
300.01	0.00496172192457658\\
301.01	0.00496139818308617\\
302.01	0.0049610685250937\\
303.01	0.00496073284675865\\
304.01	0.00496039104239201\\
305.01	0.00496004300441223\\
306.01	0.00495968862329972\\
307.01	0.00495932778754915\\
308.01	0.00495896038361932\\
309.01	0.00495858629588227\\
310.01	0.00495820540656886\\
311.01	0.00495781759571324\\
312.01	0.0049574227410947\\
313.01	0.00495702071817684\\
314.01	0.00495661140004566\\
315.01	0.00495619465734356\\
316.01	0.00495577035820249\\
317.01	0.00495533836817307\\
318.01	0.004954898550153\\
319.01	0.00495445076431137\\
320.01	0.00495399486801113\\
321.01	0.0049535307157293\\
322.01	0.00495305815897399\\
323.01	0.00495257704619948\\
324.01	0.00495208722271928\\
325.01	0.00495158853061604\\
326.01	0.00495108080864983\\
327.01	0.00495056389216409\\
328.01	0.00495003761299109\\
329.01	0.00494950179935332\\
330.01	0.00494895627576577\\
331.01	0.00494840086293572\\
332.01	0.00494783537766226\\
333.01	0.00494725963273551\\
334.01	0.00494667343683519\\
335.01	0.00494607659443045\\
336.01	0.00494546890567955\\
337.01	0.00494485016633169\\
338.01	0.00494422016763028\\
339.01	0.0049435786962189\\
340.01	0.00494292553405081\\
341.01	0.00494226045830066\\
342.01	0.00494158324128281\\
343.01	0.00494089365037261\\
344.01	0.00494019144793501\\
345.01	0.00493947639125892\\
346.01	0.00493874823249811\\
347.01	0.00493800671862038\\
348.01	0.0049372515913643\\
349.01	0.00493648258720542\\
350.01	0.00493569943732986\\
351.01	0.00493490186761785\\
352.01	0.00493408959863593\\
353.01	0.00493326234563778\\
354.01	0.00493241981857407\\
355.01	0.00493156172210941\\
356.01	0.00493068775564597\\
357.01	0.00492979761335257\\
358.01	0.00492889098419829\\
359.01	0.00492796755198697\\
360.01	0.00492702699539246\\
361.01	0.00492606898799062\\
362.01	0.00492509319828515\\
363.01	0.0049240992897253\\
364.01	0.00492308692071073\\
365.01	0.00492205574458123\\
366.01	0.00492100540958731\\
367.01	0.00491993555883882\\
368.01	0.00491884583022844\\
369.01	0.00491773585632676\\
370.01	0.00491660526424896\\
371.01	0.00491545367549144\\
372.01	0.00491428070573694\\
373.01	0.00491308596463289\\
374.01	0.004911869055541\\
375.01	0.00491062957526776\\
376.01	0.00490936711377468\\
377.01	0.00490808125387745\\
378.01	0.00490677157094167\\
379.01	0.00490543763257748\\
380.01	0.0049040789983421\\
381.01	0.00490269521945232\\
382.01	0.0049012858385085\\
383.01	0.0048998503892306\\
384.01	0.00489838839620154\\
385.01	0.00489689937461228\\
386.01	0.00489538283000254\\
387.01	0.00489383825799477\\
388.01	0.00489226514401598\\
389.01	0.0048906629630096\\
390.01	0.00488903117913716\\
391.01	0.00488736924546662\\
392.01	0.00488567660365066\\
393.01	0.0048839526835923\\
394.01	0.00488219690309812\\
395.01	0.00488040866751922\\
396.01	0.00487858736937886\\
397.01	0.00487673238798788\\
398.01	0.00487484308904662\\
399.01	0.00487291882423342\\
400.01	0.00487095893078005\\
401.01	0.00486896273103315\\
402.01	0.00486692953200372\\
403.01	0.00486485862490164\\
404.01	0.00486274928465794\\
405.01	0.00486060076943418\\
406.01	0.00485841232011875\\
407.01	0.00485618315981099\\
408.01	0.00485391249329367\\
409.01	0.00485159950649396\\
410.01	0.00484924336593332\\
411.01	0.00484684321816763\\
412.01	0.0048443981892178\\
413.01	0.00484190738399237\\
414.01	0.00483936988570152\\
415.01	0.00483678475526597\\
416.01	0.00483415103071904\\
417.01	0.00483146772660592\\
418.01	0.00482873383337868\\
419.01	0.0048259483167901\\
420.01	0.004823110117287\\
421.01	0.00482021814940455\\
422.01	0.00481727130116219\\
423.01	0.00481426843346384\\
424.01	0.00481120837950176\\
425.01	0.00480808994416671\\
426.01	0.00480491190346474\\
427.01	0.00480167300394139\\
428.01	0.00479837196211399\\
429.01	0.00479500746391217\\
430.01	0.00479157816412741\\
431.01	0.00478808268587013\\
432.01	0.0047845196200353\\
433.01	0.00478088752477351\\
434.01	0.00477718492496825\\
435.01	0.00477341031171485\\
436.01	0.004769562141802\\
437.01	0.00476563883718955\\
438.01	0.00476163878448147\\
439.01	0.00475756033438891\\
440.01	0.00475340180118087\\
441.01	0.00474916146211554\\
442.01	0.00474483755684941\\
443.01	0.00474042828681942\\
444.01	0.00473593181459244\\
445.01	0.0047313462631784\\
446.01	0.00472666971530269\\
447.01	0.00472190021263489\\
448.01	0.00471703575497148\\
449.01	0.00471207429937067\\
450.01	0.00470701375923973\\
451.01	0.00470185200337697\\
452.01	0.00469658685497031\\
453.01	0.0046912160905575\\
454.01	0.00468573743895536\\
455.01	0.00468014858016443\\
456.01	0.00467444714425817\\
457.01	0.00466863071026671\\
458.01	0.00466269680506287\\
459.01	0.00465664290226129\\
460.01	0.00465046642113565\\
461.01	0.00464416472556011\\
462.01	0.00463773512297628\\
463.01	0.00463117486338517\\
464.01	0.00462448113835771\\
465.01	0.00461765108005793\\
466.01	0.0046106817602661\\
467.01	0.00460357018939419\\
468.01	0.00459631331547916\\
469.01	0.00458890802314913\\
470.01	0.00458135113255359\\
471.01	0.00457363939825514\\
472.01	0.00456576950807836\\
473.01	0.00455773808191323\\
474.01	0.00454954167047132\\
475.01	0.00454117675399141\\
476.01	0.00453263974089503\\
477.01	0.0045239269663906\\
478.01	0.00451503469102591\\
479.01	0.00450595909919154\\
480.01	0.00449669629757504\\
481.01	0.00448724231356893\\
482.01	0.00447759309363579\\
483.01	0.0044677445016323\\
484.01	0.00445769231709465\\
485.01	0.00444743223348967\\
486.01	0.0044369598564333\\
487.01	0.00442627070187613\\
488.01	0.00441536019425983\\
489.01	0.00440422366464025\\
490.01	0.0043928563487792\\
491.01	0.00438125338520022\\
492.01	0.00436940981320724\\
493.01	0.0043573205708602\\
494.01	0.00434498049290786\\
495.01	0.00433238430867183\\
496.01	0.00431952663988391\\
497.01	0.00430640199847244\\
498.01	0.0042930047843019\\
499.01	0.00427932928286493\\
500.01	0.00426536966292989\\
501.01	0.00425111997414721\\
502.01	0.00423657414461386\\
503.01	0.00422172597840191\\
504.01	0.00420656915305143\\
505.01	0.00419109721702921\\
506.01	0.00417530358715688\\
507.01	0.00415918154600919\\
508.01	0.00414272423928445\\
509.01	0.00412592467314963\\
510.01	0.00410877571156065\\
511.01	0.0040912700735638\\
512.01	0.0040734003305765\\
513.01	0.00405515890365519\\
514.01	0.00403653806075268\\
515.01	0.00401752991397058\\
516.01	0.00399812641681324\\
517.01	0.00397831936144701\\
518.01	0.00395810037597739\\
519.01	0.00393746092174652\\
520.01	0.00391639229066258\\
521.01	0.00389488560257104\\
522.01	0.00387293180267765\\
523.01	0.00385052165903453\\
524.01	0.00382764576010388\\
525.01	0.00380429451241306\\
526.01	0.00378045813831803\\
527.01	0.00375612667389289\\
528.01	0.00373128996696626\\
529.01	0.00370593767532707\\
530.01	0.00368005926512411\\
531.01	0.00365364400948821\\
532.01	0.00362668098740595\\
533.01	0.00359915908288108\\
534.01	0.00357106698441876\\
535.01	0.00354239318487553\\
536.01	0.00351312598172047\\
537.01	0.00348325347775771\\
538.01	0.00345276358236633\\
539.01	0.00342164401331963\\
540.01	0.0033898822992512\\
541.01	0.00335746578284247\\
542.01	0.00332438162481556\\
543.01	0.0032906168088207\\
544.01	0.00325615814731941\\
545.01	0.00322099228857354\\
546.01	0.00318510572486013\\
547.01	0.00314848480204696\\
548.01	0.00311111573067254\\
549.01	0.00307298459869323\\
550.01	0.00303407738607173\\
551.01	0.00299437998139996\\
552.01	0.00295387820076718\\
553.01	0.00291255780910315\\
554.01	0.0028704045442502\\
555.01	0.0028274041440351\\
556.01	0.00278354237664257\\
557.01	0.00273880507461447\\
558.01	0.0026931781728267\\
559.01	0.00264664775082694\\
560.01	0.00259920007994351\\
561.01	0.00255082167561239\\
562.01	0.00250149935539449\\
563.01	0.00245122030319754\\
564.01	0.00239997214024063\\
565.01	0.00234774300333622\\
566.01	0.00229452163109138\\
567.01	0.00224029745865615\\
568.01	0.00218506072166532\\
569.01	0.00212880257003134\\
570.01	0.00207151519224728\\
571.01	0.00201319195084011\\
572.01	0.00195382752958045\\
573.01	0.00189341809299105\\
574.01	0.00183196145859557\\
575.01	0.00176945728220805\\
576.01	0.00170590725635814\\
577.01	0.00164131532167358\\
578.01	0.00157568789066874\\
579.01	0.00150903408290212\\
580.01	0.00144136596982691\\
581.01	0.00137269882683558\\
582.01	0.00130305138894211\\
583.01	0.00123244610519661\\
584.01	0.00116090938521497\\
585.01	0.00108847182904192\\
586.01	0.00101516842884016\\
587.01	0.000941038727476666\\
588.01	0.000866126914790455\\
589.01	0.000790481836971451\\
590.01	0.00071415688780257\\
591.01	0.000637209742203689\\
592.01	0.000559701882183716\\
593.01	0.000481697852480112\\
594.01	0.000403264167260548\\
595.01	0.000324467769563085\\
596.01	0.000245373920771974\\
597.01	0.000166155908349813\\
598.01	9.13379815192916e-05\\
599.01	2.91271958372426e-05\\
599.02	2.86192783627535e-05\\
599.03	2.8114423742006e-05\\
599.04	2.76126617978888e-05\\
599.05	2.71140226472798e-05\\
599.06	2.66185367039581e-05\\
599.07	2.61262346815533e-05\\
599.08	2.56371475965064e-05\\
599.09	2.51513067710818e-05\\
599.1	2.46687438363886e-05\\
599.11	2.41894907354479e-05\\
599.12	2.3713579726279e-05\\
599.13	2.32410433850267e-05\\
599.14	2.27719146091033e-05\\
599.15	2.23062266203888e-05\\
599.16	2.18440129684267e-05\\
599.17	2.13853075336917e-05\\
599.18	2.0930144530848e-05\\
599.19	2.04785585120829e-05\\
599.2	2.00305843704295e-05\\
599.21	1.95862573431627e-05\\
599.22	1.91456130152045e-05\\
599.23	1.87086873225609e-05\\
599.24	1.82755165558171e-05\\
599.25	1.78461373636363e-05\\
599.26	1.74205874164651e-05\\
599.27	1.69989072017485e-05\\
599.28	1.6581137610194e-05\\
599.29	1.61673199397579e-05\\
599.3	1.57574958996841e-05\\
599.31	1.53517076145714e-05\\
599.32	1.49499976284904e-05\\
599.33	1.45524089091333e-05\\
599.34	1.41589848520092e-05\\
599.35	1.37697692846831e-05\\
599.36	1.33848064710462e-05\\
599.37	1.3004141115644e-05\\
599.38	1.26278183680325e-05\\
599.39	1.22558838271929e-05\\
599.4	1.18883835459674e-05\\
599.41	1.15253640355639e-05\\
599.42	1.11668722700964e-05\\
599.43	1.08129556911502e-05\\
599.44	1.04636622124312e-05\\
599.45	1.01190402244222e-05\\
599.46	9.77913859911798e-06\\
599.47	9.44400669477576e-06\\
599.48	9.11369436074581e-06\\
599.49	8.7882519423238e-06\\
599.5	8.46773028565471e-06\\
599.51	8.15218074270464e-06\\
599.52	7.84165517625675e-06\\
599.53	7.5362059649732e-06\\
599.54	7.23588600849874e-06\\
599.55	6.94074873262146e-06\\
599.56	6.65084809447353e-06\\
599.57	6.36623858780126e-06\\
599.58	6.08697524826819e-06\\
599.59	5.81311365882228e-06\\
599.6	5.54470995511175e-06\\
599.61	5.28182083095637e-06\\
599.62	5.02450354387431e-06\\
599.63	4.77281592065407e-06\\
599.64	4.52681636300273e-06\\
599.65	4.28656385322197e-06\\
599.66	4.05211795996042e-06\\
599.67	3.8235388440163e-06\\
599.68	3.60088726420078e-06\\
599.69	3.38422458325514e-06\\
599.7	3.17361277382341e-06\\
599.71	2.96911442449269e-06\\
599.72	2.77079274589413e-06\\
599.73	2.5787115768474e-06\\
599.74	2.3929353905848e-06\\
599.75	2.213529301031e-06\\
599.76	2.04055906913997e-06\\
599.77	1.87409110929959e-06\\
599.78	1.71419249580043e-06\\
599.79	1.56093096936004e-06\\
599.8	1.41437494372704e-06\\
599.81	1.27459351233379e-06\\
599.82	1.14165645502366e-06\\
599.83	1.01563424484766e-06\\
599.84	8.96598054920053e-07\\
599.85	7.84619765348618e-07\\
599.86	6.79771970232487e-07\\
599.87	5.82127984717282e-07\\
599.88	4.91761852145639e-07\\
599.89	4.08748351251112e-07\\
599.9	3.33163003437068e-07\\
599.91	2.65082080131915e-07\\
599.92	2.04582610201579e-07\\
599.93	1.51742387450443e-07\\
599.94	1.06639978189951e-07\\
599.95	6.93547288817958e-08\\
599.96	3.99667738487652e-08\\
599.97	1.85570430896731e-08\\
599.98	5.20727013418598e-09\\
599.99	0\\
600	0\\
};
\addplot [color=mycolor2,solid,forget plot]
  table[row sep=crcr]{%
0.01	0.00499837194810791\\
1.01	0.0049983709500581\\
2.01	0.00499836993172411\\
3.01	0.0049983688926949\\
4.01	0.00499836783255128\\
5.01	0.00499836675086541\\
6.01	0.004998365647201\\
7.01	0.00499836452111279\\
8.01	0.00499836337214647\\
9.01	0.0049983621998389\\
10.01	0.00499836100371725\\
11.01	0.00499835978329928\\
12.01	0.0049983585380931\\
13.01	0.00499835726759685\\
14.01	0.00499835597129857\\
15.01	0.00499835464867581\\
16.01	0.00499835329919573\\
17.01	0.00499835192231491\\
18.01	0.00499835051747875\\
19.01	0.00499834908412148\\
20.01	0.00499834762166594\\
21.01	0.00499834612952357\\
22.01	0.00499834460709351\\
23.01	0.00499834305376323\\
24.01	0.00499834146890773\\
25.01	0.0049983398518892\\
26.01	0.00499833820205734\\
27.01	0.00499833651874839\\
28.01	0.00499833480128571\\
29.01	0.00499833304897846\\
30.01	0.00499833126112235\\
31.01	0.00499832943699859\\
32.01	0.0049983275758743\\
33.01	0.00499832567700145\\
34.01	0.00499832373961732\\
35.01	0.00499832176294358\\
36.01	0.00499831974618637\\
37.01	0.00499831768853598\\
38.01	0.00499831558916598\\
39.01	0.0049983134472338\\
40.01	0.00499831126187974\\
41.01	0.00499830903222683\\
42.01	0.00499830675738047\\
43.01	0.00499830443642806\\
44.01	0.00499830206843873\\
45.01	0.00499829965246298\\
46.01	0.00499829718753196\\
47.01	0.00499829467265758\\
48.01	0.00499829210683183\\
49.01	0.00499828948902669\\
50.01	0.00499828681819303\\
51.01	0.00499828409326108\\
52.01	0.00499828131313945\\
53.01	0.00499827847671471\\
54.01	0.00499827558285147\\
55.01	0.00499827263039104\\
56.01	0.00499826961815198\\
57.01	0.00499826654492874\\
58.01	0.00499826340949194\\
59.01	0.00499826021058731\\
60.01	0.00499825694693557\\
61.01	0.00499825361723174\\
62.01	0.0049982502201447\\
63.01	0.0049982467543167\\
64.01	0.00499824321836263\\
65.01	0.00499823961086985\\
66.01	0.00499823593039719\\
67.01	0.00499823217547466\\
68.01	0.0049982283446032\\
69.01	0.00499822443625329\\
70.01	0.00499822044886503\\
71.01	0.00499821638084697\\
72.01	0.00499821223057625\\
73.01	0.00499820799639745\\
74.01	0.00499820367662184\\
75.01	0.00499819926952695\\
76.01	0.00499819477335602\\
77.01	0.00499819018631703\\
78.01	0.00499818550658229\\
79.01	0.00499818073228734\\
80.01	0.0049981758615306\\
81.01	0.00499817089237242\\
82.01	0.00499816582283464\\
83.01	0.00499816065089932\\
84.01	0.00499815537450828\\
85.01	0.00499814999156225\\
86.01	0.00499814449992018\\
87.01	0.00499813889739805\\
88.01	0.00499813318176858\\
89.01	0.00499812735075984\\
90.01	0.00499812140205455\\
91.01	0.00499811533328925\\
92.01	0.00499810914205335\\
93.01	0.00499810282588823\\
94.01	0.00499809638228636\\
95.01	0.00499808980868995\\
96.01	0.00499808310249072\\
97.01	0.00499807626102795\\
98.01	0.00499806928158813\\
99.01	0.00499806216140391\\
100.01	0.00499805489765244\\
101.01	0.00499804748745494\\
102.01	0.00499803992787549\\
103.01	0.00499803221591948\\
104.01	0.00499802434853285\\
105.01	0.0049980163226006\\
106.01	0.00499800813494621\\
107.01	0.00499799978232974\\
108.01	0.00499799126144694\\
109.01	0.0049979825689279\\
110.01	0.0049979737013357\\
111.01	0.00499796465516526\\
112.01	0.00499795542684211\\
113.01	0.00499794601272043\\
114.01	0.00499793640908229\\
115.01	0.00499792661213579\\
116.01	0.00499791661801425\\
117.01	0.004997906422774\\
118.01	0.00499789602239304\\
119.01	0.00499788541277001\\
120.01	0.0049978745897221\\
121.01	0.00499786354898375\\
122.01	0.00499785228620484\\
123.01	0.00499784079694906\\
124.01	0.00499782907669253\\
125.01	0.00499781712082165\\
126.01	0.00499780492463176\\
127.01	0.00499779248332488\\
128.01	0.00499777979200853\\
129.01	0.00499776684569321\\
130.01	0.00499775363929101\\
131.01	0.00499774016761349\\
132.01	0.00499772642536989\\
133.01	0.00499771240716477\\
134.01	0.00499769810749633\\
135.01	0.00499768352075453\\
136.01	0.00499766864121828\\
137.01	0.00499765346305373\\
138.01	0.00499763798031251\\
139.01	0.00499762218692859\\
140.01	0.00499760607671657\\
141.01	0.00499758964336939\\
142.01	0.00499757288045592\\
143.01	0.00499755578141847\\
144.01	0.00499753833957031\\
145.01	0.00499752054809328\\
146.01	0.00499750240003531\\
147.01	0.00499748388830772\\
148.01	0.00499746500568277\\
149.01	0.00499744574479062\\
150.01	0.00499742609811695\\
151.01	0.00499740605800016\\
152.01	0.0049973856166284\\
153.01	0.00499736476603675\\
154.01	0.00499734349810407\\
155.01	0.00499732180455063\\
156.01	0.00499729967693428\\
157.01	0.004997277106648\\
158.01	0.00499725408491626\\
159.01	0.00499723060279239\\
160.01	0.00499720665115418\\
161.01	0.0049971822207023\\
162.01	0.00499715730195526\\
163.01	0.00499713188524693\\
164.01	0.00499710596072261\\
165.01	0.00499707951833552\\
166.01	0.00499705254784331\\
167.01	0.00499702503880418\\
168.01	0.00499699698057329\\
169.01	0.00499696836229883\\
170.01	0.00499693917291786\\
171.01	0.00499690940115286\\
172.01	0.00499687903550713\\
173.01	0.00499684806426117\\
174.01	0.00499681647546797\\
175.01	0.00499678425694931\\
176.01	0.004996751396291\\
177.01	0.00499671788083855\\
178.01	0.00499668369769278\\
179.01	0.00499664883370536\\
180.01	0.00499661327547373\\
181.01	0.00499657700933676\\
182.01	0.00499654002136978\\
183.01	0.0049965022973798\\
184.01	0.00499646382290037\\
185.01	0.00499642458318667\\
186.01	0.00499638456321026\\
187.01	0.00499634374765393\\
188.01	0.00499630212090629\\
189.01	0.00499625966705638\\
190.01	0.00499621636988827\\
191.01	0.00499617221287514\\
192.01	0.00499612717917412\\
193.01	0.00499608125162022\\
194.01	0.00499603441272022\\
195.01	0.00499598664464723\\
196.01	0.00499593792923409\\
197.01	0.0049958882479677\\
198.01	0.00499583758198227\\
199.01	0.00499578591205326\\
200.01	0.00499573321859105\\
201.01	0.0049956794816339\\
202.01	0.00499562468084163\\
203.01	0.00499556879548895\\
204.01	0.00499551180445827\\
205.01	0.00499545368623272\\
206.01	0.00499539441888925\\
207.01	0.00499533398009169\\
208.01	0.00499527234708256\\
209.01	0.00499520949667663\\
210.01	0.00499514540525276\\
211.01	0.0049950800487466\\
212.01	0.00499501340264259\\
213.01	0.00499494544196635\\
214.01	0.0049948761412763\\
215.01	0.00499480547465623\\
216.01	0.00499473341570651\\
217.01	0.00499465993753581\\
218.01	0.00499458501275311\\
219.01	0.00499450861345861\\
220.01	0.00499443071123545\\
221.01	0.00499435127714077\\
222.01	0.00499427028169681\\
223.01	0.00499418769488206\\
224.01	0.00499410348612157\\
225.01	0.00499401762427875\\
226.01	0.00499393007764475\\
227.01	0.00499384081392993\\
228.01	0.00499374980025395\\
229.01	0.0049936570031358\\
230.01	0.00499356238848447\\
231.01	0.00499346592158848\\
232.01	0.00499336756710633\\
233.01	0.00499326728905578\\
234.01	0.00499316505080432\\
235.01	0.00499306081505766\\
236.01	0.00499295454385023\\
237.01	0.00499284619853421\\
238.01	0.00499273573976864\\
239.01	0.00499262312750883\\
240.01	0.00499250832099512\\
241.01	0.00499239127874211\\
242.01	0.00499227195852745\\
243.01	0.00499215031738026\\
244.01	0.00499202631157023\\
245.01	0.00499189989659601\\
246.01	0.00499177102717331\\
247.01	0.00499163965722368\\
248.01	0.00499150573986267\\
249.01	0.00499136922738777\\
250.01	0.00499123007126674\\
251.01	0.00499108822212551\\
252.01	0.00499094362973598\\
253.01	0.00499079624300416\\
254.01	0.00499064600995752\\
255.01	0.00499049287773259\\
256.01	0.00499033679256298\\
257.01	0.00499017769976685\\
258.01	0.004990015543734\\
259.01	0.00498985026791332\\
260.01	0.0049896818148004\\
261.01	0.00498951012592487\\
262.01	0.00498933514183677\\
263.01	0.00498915680209478\\
264.01	0.00498897504525236\\
265.01	0.00498878980884541\\
266.01	0.00498860102937881\\
267.01	0.00498840864231363\\
268.01	0.00498821258205379\\
269.01	0.00498801278193307\\
270.01	0.00498780917420153\\
271.01	0.00498760169001255\\
272.01	0.00498739025940909\\
273.01	0.00498717481131064\\
274.01	0.00498695527349969\\
275.01	0.0049867315726076\\
276.01	0.00498650363410184\\
277.01	0.00498627138227145\\
278.01	0.00498603474021383\\
279.01	0.00498579362982069\\
280.01	0.00498554797176367\\
281.01	0.00498529768548078\\
282.01	0.00498504268916163\\
283.01	0.00498478289973312\\
284.01	0.00498451823284513\\
285.01	0.00498424860285535\\
286.01	0.00498397392281443\\
287.01	0.00498369410445084\\
288.01	0.00498340905815486\\
289.01	0.00498311869296325\\
290.01	0.00498282291654267\\
291.01	0.00498252163517333\\
292.01	0.00498221475373195\\
293.01	0.00498190217567384\\
294.01	0.00498158380301523\\
295.01	0.00498125953631471\\
296.01	0.00498092927465335\\
297.01	0.00498059291561497\\
298.01	0.00498025035526523\\
299.01	0.00497990148812967\\
300.01	0.00497954620717112\\
301.01	0.00497918440376567\\
302.01	0.0049788159676769\\
303.01	0.00497844078703037\\
304.01	0.00497805874828512\\
305.01	0.00497766973620457\\
306.01	0.00497727363382445\\
307.01	0.00497687032242084\\
308.01	0.0049764596814739\\
309.01	0.00497604158863078\\
310.01	0.00497561591966619\\
311.01	0.00497518254843956\\
312.01	0.00497474134684996\\
313.01	0.0049742921847878\\
314.01	0.00497383493008324\\
315.01	0.00497336944845126\\
316.01	0.00497289560343311\\
317.01	0.00497241325633382\\
318.01	0.00497192226615476\\
319.01	0.00497142248952285\\
320.01	0.00497091378061419\\
321.01	0.00497039599107319\\
322.01	0.00496986896992603\\
323.01	0.0049693325634892\\
324.01	0.0049687866152714\\
325.01	0.00496823096587038\\
326.01	0.00496766545286285\\
327.01	0.0049670899106884\\
328.01	0.00496650417052665\\
329.01	0.00496590806016794\\
330.01	0.00496530140387685\\
331.01	0.004964684022249\\
332.01	0.00496405573206107\\
333.01	0.00496341634611457\\
334.01	0.00496276567307206\\
335.01	0.00496210351728766\\
336.01	0.00496142967863188\\
337.01	0.00496074395231044\\
338.01	0.00496004612867838\\
339.01	0.00495933599305066\\
340.01	0.00495861332550844\\
341.01	0.00495787790070489\\
342.01	0.00495712948766865\\
343.01	0.00495636784960901\\
344.01	0.00495559274372304\\
345.01	0.00495480392100705\\
346.01	0.00495400112607405\\
347.01	0.00495318409698125\\
348.01	0.00495235256506702\\
349.01	0.00495150625480289\\
350.01	0.00495064488366268\\
351.01	0.00494976816201098\\
352.01	0.00494887579301496\\
353.01	0.00494796747258315\\
354.01	0.00494704288933306\\
355.01	0.0049461017245915\\
356.01	0.00494514365243117\\
357.01	0.00494416833974473\\
358.01	0.00494317544635841\\
359.01	0.00494216462518827\\
360.01	0.00494113552243682\\
361.01	0.00494008777783191\\
362.01	0.00493902102490459\\
363.01	0.0049379348913036\\
364.01	0.00493682899914033\\
365.01	0.00493570296535907\\
366.01	0.00493455640212225\\
367.01	0.00493338891720158\\
368.01	0.00493220011436012\\
369.01	0.00493098959371433\\
370.01	0.00492975695205646\\
371.01	0.00492850178312323\\
372.01	0.00492722367779316\\
373.01	0.00492592222419604\\
374.01	0.00492459700772249\\
375.01	0.00492324761092191\\
376.01	0.0049218736132852\\
377.01	0.00492047459091522\\
378.01	0.00491905011609411\\
379.01	0.00491759975676965\\
380.01	0.00491612307598761\\
381.01	0.0049146196313097\\
382.01	0.00491308897425378\\
383.01	0.00491153064979922\\
384.01	0.00490994419598142\\
385.01	0.00490832914358414\\
386.01	0.00490668501590879\\
387.01	0.00490501132857611\\
388.01	0.00490330758932518\\
389.01	0.00490157329779979\\
390.01	0.00489980794532215\\
391.01	0.00489801101465234\\
392.01	0.00489618197973294\\
393.01	0.00489432030541775\\
394.01	0.0048924254471838\\
395.01	0.0048904968508254\\
396.01	0.00488853395213052\\
397.01	0.00488653617653644\\
398.01	0.00488450293876571\\
399.01	0.00488243364244059\\
400.01	0.00488032767967433\\
401.01	0.00487818443064068\\
402.01	0.00487600326311719\\
403.01	0.00487378353200537\\
404.01	0.00487152457882404\\
405.01	0.00486922573117583\\
406.01	0.00486688630218762\\
407.01	0.00486450558992194\\
408.01	0.00486208287676144\\
409.01	0.00485961742876332\\
410.01	0.00485710849498731\\
411.01	0.00485455530679276\\
412.01	0.00485195707710966\\
413.01	0.00484931299967979\\
414.01	0.00484662224827214\\
415.01	0.00484388397587165\\
416.01	0.00484109731384299\\
417.01	0.00483826137107097\\
418.01	0.00483537523307962\\
419.01	0.00483243796113237\\
420.01	0.00482944859131516\\
421.01	0.00482640613360631\\
422.01	0.00482330957093646\\
423.01	0.00482015785824115\\
424.01	0.00481694992151268\\
425.01	0.00481368465685277\\
426.01	0.00481036092953239\\
427.01	0.00480697757306484\\
428.01	0.00480353338829413\\
429.01	0.00480002714250805\\
430.01	0.00479645756857831\\
431.01	0.00479282336413425\\
432.01	0.00478912319077583\\
433.01	0.00478535567332973\\
434.01	0.00478151939915343\\
435.01	0.00477761291749091\\
436.01	0.00477363473888184\\
437.01	0.00476958333462676\\
438.01	0.00476545713630698\\
439.01	0.00476125453535923\\
440.01	0.00475697388269935\\
441.01	0.00475261348839167\\
442.01	0.00474817162135461\\
443.01	0.00474364650909256\\
444.01	0.00473903633744224\\
445.01	0.00473433925031657\\
446.01	0.00472955334943123\\
447.01	0.00472467669399176\\
448.01	0.00471970730032385\\
449.01	0.00471464314142204\\
450.01	0.00470948214639871\\
451.01	0.00470422219981142\\
452.01	0.00469886114085184\\
453.01	0.00469339676238324\\
454.01	0.00468782680981757\\
455.01	0.00468214897983199\\
456.01	0.00467636091893143\\
457.01	0.00467046022187449\\
458.01	0.00466444442998926\\
459.01	0.00465831102941401\\
460.01	0.00465205744930838\\
461.01	0.00464568106008406\\
462.01	0.0046391791717075\\
463.01	0.00463254903212104\\
464.01	0.00462578782582315\\
465.01	0.00461889267262927\\
466.01	0.00461186062661671\\
467.01	0.00460468867523036\\
468.01	0.00459737373851248\\
469.01	0.00458991266840214\\
470.01	0.00458230224805924\\
471.01	0.00457453919118591\\
472.01	0.00456662014132859\\
473.01	0.00455854167114941\\
474.01	0.0045503002816534\\
475.01	0.00454189240135999\\
476.01	0.00453331438540521\\
477.01	0.00452456251456477\\
478.01	0.00451563299418752\\
479.01	0.00450652195303194\\
480.01	0.00449722544200074\\
481.01	0.00448773943277216\\
482.01	0.00447805981632859\\
483.01	0.00446818240138797\\
484.01	0.00445810291274723\\
485.01	0.00444781698954911\\
486.01	0.00443732018348554\\
487.01	0.00442660795695588\\
488.01	0.0044156756811926\\
489.01	0.00440451863437273\\
490.01	0.00439313199972349\\
491.01	0.00438151086363007\\
492.01	0.00436965021374725\\
493.01	0.00435754493710989\\
494.01	0.00434518981822867\\
495.01	0.00433257953715834\\
496.01	0.00431970866751696\\
497.01	0.00430657167444331\\
498.01	0.0042931629124772\\
499.01	0.00427947662335989\\
500.01	0.00426550693375518\\
501.01	0.0042512478528956\\
502.01	0.00423669327016178\\
503.01	0.00422183695260067\\
504.01	0.00420667254239184\\
505.01	0.00419119355427002\\
506.01	0.00417539337290941\\
507.01	0.00415926525027823\\
508.01	0.00414280230296774\\
509.01	0.00412599750949901\\
510.01	0.00410884370761253\\
511.01	0.00409133359153772\\
512.01	0.00407345970924717\\
513.01	0.00405521445969382\\
514.01	0.00403659009003454\\
515.01	0.00401757869284169\\
516.01	0.00399817220330911\\
517.01	0.00397836239645968\\
518.01	0.00395814088436233\\
519.01	0.00393749911337053\\
520.01	0.00391642836139186\\
521.01	0.00389491973519967\\
522.01	0.00387296416779904\\
523.01	0.00385055241585978\\
524.01	0.00382767505722927\\
525.01	0.00380432248853902\\
526.01	0.0037804849229221\\
527.01	0.00375615238785891\\
528.01	0.00373131472317088\\
529.01	0.00370596157918466\\
530.01	0.00368008241509258\\
531.01	0.00365366649753617\\
532.01	0.00362670289944566\\
533.01	0.00359918049916647\\
534.01	0.00357108797991388\\
535.01	0.00354241382959441\\
536.01	0.00351314634104124\\
537.01	0.00348327361271283\\
538.01	0.00345278354991174\\
539.01	0.00342166386658361\\
540.01	0.00338990208776503\\
541.01	0.00335748555275526\\
542.01	0.00332440141909313\\
543.01	0.00329063666743112\\
544.01	0.00325617810740615\\
545.01	0.00322101238461686\\
546.01	0.00318512598882893\\
547.01	0.00314850526353988\\
548.01	0.00311113641705144\\
549.01	0.00307300553520775\\
550.01	0.00303409859597573\\
551.01	0.00299440148606015\\
552.01	0.00295390001976419\\
553.01	0.00291257996032587\\
554.01	0.00287042704398067\\
555.01	0.00282742700702653\\
556.01	0.00278356561618944\\
557.01	0.00273882870261338\\
558.01	0.00269320219982914\\
559.01	0.00264667218608162\\
560.01	0.00259922493143004\\
561.01	0.00255084695006314\\
562.01	0.00250152505830729\\
563.01	0.00245124643883594\\
564.01	0.00239999871162262\\
565.01	0.00234777001221102\\
566.01	0.00229454907790362\\
567.01	0.00224032534249755\\
568.01	0.0021850890402133\\
569.01	0.00212883131947533\\
570.01	0.00207154436720215\\
571.01	0.0020132215442478\\
572.01	0.00195385753260022\\
573.01	0.00189344849487884\\
574.01	0.00183199224657509\\
575.01	0.00176948844133388\\
576.01	0.00170593876937416\\
577.01	0.00164134716886916\\
578.01	0.00157572004973666\\
579.01	0.00150906652880276\\
580.01	0.00144139867466453\\
581.01	0.00137273175975388\\
582.01	0.00130308451604835\\
583.01	0.00123247938952499\\
584.01	0.00116094278674301\\
585.01	0.00108850530477681\\
586.01	0.00101520193299579\\
587.01	0.00094107221176625\\
588.01	0.000866160328865604\\
589.01	0.000790515129045323\\
590.01	0.00071419000550243\\
591.01	0.000637242633708195\\
592.01	0.000559734497711805\\
593.01	0.00048173014621177\\
594.01	0.000403296099787112\\
595.01	0.000324499310983717\\
596.01	0.000245405054576748\\
597.01	0.000166173366806249\\
598.01	9.13379815192916e-05\\
599.01	2.91271958372426e-05\\
599.02	2.86192783627518e-05\\
599.03	2.81144237420042e-05\\
599.04	2.76126617978888e-05\\
599.05	2.71140226472781e-05\\
599.06	2.66185367039599e-05\\
599.07	2.61262346815533e-05\\
599.08	2.56371475965082e-05\\
599.09	2.51513067710818e-05\\
599.1	2.46687438363886e-05\\
599.11	2.41894907354479e-05\\
599.12	2.3713579726279e-05\\
599.13	2.3241043385025e-05\\
599.14	2.2771914609105e-05\\
599.15	2.23062266203871e-05\\
599.16	2.18440129684284e-05\\
599.17	2.138530753369e-05\\
599.18	2.09301445308497e-05\\
599.19	2.04785585120812e-05\\
599.2	2.00305843704278e-05\\
599.21	1.95862573431644e-05\\
599.22	1.91456130152028e-05\\
599.23	1.87086873225609e-05\\
599.24	1.82755165558171e-05\\
599.25	1.7846137363638e-05\\
599.26	1.74205874164651e-05\\
599.27	1.69989072017502e-05\\
599.28	1.6581137610194e-05\\
599.29	1.61673199397579e-05\\
599.3	1.57574958996824e-05\\
599.31	1.53517076145714e-05\\
599.32	1.49499976284887e-05\\
599.33	1.45524089091333e-05\\
599.34	1.41589848520092e-05\\
599.35	1.37697692846831e-05\\
599.36	1.33848064710462e-05\\
599.37	1.30041411156422e-05\\
599.38	1.26278183680325e-05\\
599.39	1.22558838271912e-05\\
599.4	1.18883835459657e-05\\
599.41	1.15253640355657e-05\\
599.42	1.11668722700964e-05\\
599.43	1.08129556911519e-05\\
599.44	1.04636622124312e-05\\
599.45	1.01190402244222e-05\\
599.46	9.77913859911625e-06\\
599.47	9.44400669477576e-06\\
599.48	9.11369436074755e-06\\
599.49	8.78825194232206e-06\\
599.5	8.46773028565471e-06\\
599.51	8.15218074270291e-06\\
599.52	7.84165517625675e-06\\
599.53	7.53620596497147e-06\\
599.54	7.23588600849874e-06\\
599.55	6.94074873261973e-06\\
599.56	6.65084809447353e-06\\
599.57	6.366238587803e-06\\
599.58	6.08697524826819e-06\\
599.59	5.81311365882228e-06\\
599.6	5.54470995511175e-06\\
599.61	5.28182083095637e-06\\
599.62	5.02450354387257e-06\\
599.63	4.7728159206558e-06\\
599.64	4.52681636300446e-06\\
599.65	4.28656385322197e-06\\
599.66	4.05211795995869e-06\\
599.67	3.82353884401457e-06\\
599.68	3.60088726420078e-06\\
599.69	3.38422458325341e-06\\
599.7	3.17361277382168e-06\\
599.71	2.96911442449269e-06\\
599.72	2.77079274589413e-06\\
599.73	2.57871157684567e-06\\
599.74	2.39293539058306e-06\\
599.75	2.21352930102926e-06\\
599.76	2.04055906913823e-06\\
599.77	1.87409110930133e-06\\
599.78	1.71419249580043e-06\\
599.79	1.56093096936177e-06\\
599.8	1.41437494372877e-06\\
599.81	1.27459351233379e-06\\
599.82	1.14165645502366e-06\\
599.83	1.01563424484592e-06\\
599.84	8.96598054920053e-07\\
599.85	7.84619765350353e-07\\
599.86	6.79771970232487e-07\\
599.87	5.82127984719016e-07\\
599.88	4.91761852147374e-07\\
599.89	4.08748351251112e-07\\
599.9	3.33163003438802e-07\\
599.91	2.65082080131915e-07\\
599.92	2.04582610201579e-07\\
599.93	1.51742387452178e-07\\
599.94	1.06639978191686e-07\\
599.95	6.93547288800611e-08\\
599.96	3.99667738487652e-08\\
599.97	1.85570430896731e-08\\
599.98	5.20727013245126e-09\\
599.99	0\\
600	0\\
};
\addplot [color=mycolor3,solid,forget plot]
  table[row sep=crcr]{%
0.01	0.00503244980582076\\
1.01	0.0050324487594527\\
2.01	0.00503244769167412\\
3.01	0.00503244660204765\\
4.01	0.0050324454901266\\
5.01	0.00503244435545586\\
6.01	0.00503244319757019\\
7.01	0.00503244201599557\\
8.01	0.00503244081024797\\
9.01	0.00503243957983354\\
10.01	0.00503243832424839\\
11.01	0.00503243704297841\\
12.01	0.00503243573549884\\
13.01	0.00503243440127439\\
14.01	0.0050324330397588\\
15.01	0.00503243165039465\\
16.01	0.00503243023261314\\
17.01	0.00503242878583388\\
18.01	0.0050324273094648\\
19.01	0.00503242580290154\\
20.01	0.00503242426552755\\
21.01	0.00503242269671365\\
22.01	0.00503242109581802\\
23.01	0.00503241946218543\\
24.01	0.00503241779514747\\
25.01	0.0050324160940225\\
26.01	0.00503241435811418\\
27.01	0.0050324125867128\\
28.01	0.00503241077909349\\
29.01	0.00503240893451709\\
30.01	0.0050324070522293\\
31.01	0.00503240513146033\\
32.01	0.00503240317142475\\
33.01	0.00503240117132111\\
34.01	0.00503239913033189\\
35.01	0.00503239704762252\\
36.01	0.00503239492234174\\
37.01	0.00503239275362075\\
38.01	0.00503239054057329\\
39.01	0.00503238828229486\\
40.01	0.0050323859778628\\
41.01	0.00503238362633529\\
42.01	0.00503238122675157\\
43.01	0.00503237877813141\\
44.01	0.00503237627947449\\
45.01	0.00503237372976\\
46.01	0.00503237112794673\\
47.01	0.00503236847297204\\
48.01	0.00503236576375179\\
49.01	0.00503236299917937\\
50.01	0.00503236017812644\\
51.01	0.00503235729944098\\
52.01	0.0050323543619478\\
53.01	0.00503235136444798\\
54.01	0.00503234830571803\\
55.01	0.00503234518450956\\
56.01	0.0050323419995488\\
57.01	0.00503233874953634\\
58.01	0.00503233543314596\\
59.01	0.00503233204902479\\
60.01	0.0050323285957922\\
61.01	0.00503232507203969\\
62.01	0.00503232147633013\\
63.01	0.00503231780719697\\
64.01	0.0050323140631442\\
65.01	0.00503231024264517\\
66.01	0.00503230634414228\\
67.01	0.00503230236604648\\
68.01	0.00503229830673582\\
69.01	0.00503229416455597\\
70.01	0.00503228993781898\\
71.01	0.00503228562480235\\
72.01	0.00503228122374879\\
73.01	0.00503227673286538\\
74.01	0.00503227215032263\\
75.01	0.00503226747425396\\
76.01	0.00503226270275486\\
77.01	0.00503225783388235\\
78.01	0.00503225286565367\\
79.01	0.00503224779604628\\
80.01	0.00503224262299631\\
81.01	0.00503223734439799\\
82.01	0.00503223195810295\\
83.01	0.00503222646191911\\
84.01	0.00503222085361007\\
85.01	0.00503221513089406\\
86.01	0.00503220929144289\\
87.01	0.00503220333288127\\
88.01	0.00503219725278569\\
89.01	0.0050321910486836\\
90.01	0.00503218471805234\\
91.01	0.00503217825831803\\
92.01	0.00503217166685472\\
93.01	0.00503216494098317\\
94.01	0.00503215807796969\\
95.01	0.00503215107502565\\
96.01	0.00503214392930557\\
97.01	0.00503213663790648\\
98.01	0.00503212919786658\\
99.01	0.00503212160616396\\
100.01	0.00503211385971588\\
101.01	0.00503210595537693\\
102.01	0.00503209788993803\\
103.01	0.00503208966012522\\
104.01	0.00503208126259835\\
105.01	0.00503207269394952\\
106.01	0.00503206395070202\\
107.01	0.00503205502930895\\
108.01	0.0050320459261513\\
109.01	0.00503203663753715\\
110.01	0.00503202715969984\\
111.01	0.00503201748879668\\
112.01	0.00503200762090712\\
113.01	0.00503199755203162\\
114.01	0.00503198727808951\\
115.01	0.00503197679491799\\
116.01	0.00503196609826996\\
117.01	0.00503195518381265\\
118.01	0.00503194404712566\\
119.01	0.00503193268369961\\
120.01	0.00503192108893388\\
121.01	0.00503190925813501\\
122.01	0.00503189718651492\\
123.01	0.00503188486918881\\
124.01	0.00503187230117348\\
125.01	0.00503185947738504\\
126.01	0.00503184639263723\\
127.01	0.00503183304163921\\
128.01	0.00503181941899344\\
129.01	0.00503180551919353\\
130.01	0.00503179133662255\\
131.01	0.00503177686554999\\
132.01	0.00503176210012995\\
133.01	0.00503174703439897\\
134.01	0.00503173166227358\\
135.01	0.00503171597754753\\
136.01	0.0050316999738901\\
137.01	0.00503168364484295\\
138.01	0.00503166698381785\\
139.01	0.00503164998409419\\
140.01	0.00503163263881631\\
141.01	0.00503161494099051\\
142.01	0.00503159688348273\\
143.01	0.00503157845901536\\
144.01	0.00503155966016506\\
145.01	0.00503154047935915\\
146.01	0.00503152090887261\\
147.01	0.00503150094082613\\
148.01	0.00503148056718165\\
149.01	0.00503145977974037\\
150.01	0.00503143857013865\\
151.01	0.00503141692984551\\
152.01	0.00503139485015892\\
153.01	0.00503137232220257\\
154.01	0.00503134933692227\\
155.01	0.00503132588508246\\
156.01	0.005031301957263\\
157.01	0.00503127754385505\\
158.01	0.00503125263505771\\
159.01	0.00503122722087381\\
160.01	0.00503120129110709\\
161.01	0.00503117483535663\\
162.01	0.00503114784301425\\
163.01	0.00503112030326\\
164.01	0.00503109220505783\\
165.01	0.00503106353715156\\
166.01	0.00503103428806052\\
167.01	0.005031004446075\\
168.01	0.00503097399925205\\
169.01	0.00503094293541084\\
170.01	0.00503091124212806\\
171.01	0.00503087890673309\\
172.01	0.00503084591630309\\
173.01	0.00503081225765854\\
174.01	0.00503077791735762\\
175.01	0.00503074288169191\\
176.01	0.00503070713668027\\
177.01	0.00503067066806456\\
178.01	0.00503063346130337\\
179.01	0.00503059550156716\\
180.01	0.00503055677373244\\
181.01	0.00503051726237645\\
182.01	0.00503047695177069\\
183.01	0.00503043582587579\\
184.01	0.00503039386833516\\
185.01	0.00503035106246887\\
186.01	0.00503030739126768\\
187.01	0.00503026283738648\\
188.01	0.00503021738313796\\
189.01	0.00503017101048648\\
190.01	0.0050301237010407\\
191.01	0.00503007543604742\\
192.01	0.00503002619638435\\
193.01	0.00502997596255327\\
194.01	0.00502992471467295\\
195.01	0.00502987243247175\\
196.01	0.00502981909528031\\
197.01	0.00502976468202424\\
198.01	0.00502970917121621\\
199.01	0.00502965254094836\\
200.01	0.00502959476888455\\
201.01	0.00502953583225201\\
202.01	0.00502947570783336\\
203.01	0.00502941437195841\\
204.01	0.00502935180049576\\
205.01	0.00502928796884394\\
206.01	0.00502922285192303\\
207.01	0.00502915642416559\\
208.01	0.00502908865950821\\
209.01	0.00502901953138143\\
210.01	0.00502894901270135\\
211.01	0.00502887707585973\\
212.01	0.00502880369271473\\
213.01	0.00502872883458092\\
214.01	0.00502865247221946\\
215.01	0.00502857457582826\\
216.01	0.00502849511503165\\
217.01	0.0050284140588702\\
218.01	0.00502833137579014\\
219.01	0.00502824703363277\\
220.01	0.00502816099962352\\
221.01	0.00502807324036168\\
222.01	0.00502798372180807\\
223.01	0.00502789240927511\\
224.01	0.00502779926741474\\
225.01	0.00502770426020653\\
226.01	0.00502760735094691\\
227.01	0.00502750850223669\\
228.01	0.00502740767596873\\
229.01	0.00502730483331666\\
230.01	0.00502719993472168\\
231.01	0.00502709293988079\\
232.01	0.00502698380773348\\
233.01	0.00502687249644932\\
234.01	0.00502675896341495\\
235.01	0.00502664316522123\\
236.01	0.00502652505764949\\
237.01	0.00502640459565847\\
238.01	0.00502628173337099\\
239.01	0.00502615642405963\\
240.01	0.00502602862013372\\
241.01	0.00502589827312502\\
242.01	0.0050257653336735\\
243.01	0.00502562975151353\\
244.01	0.00502549147545935\\
245.01	0.00502535045339068\\
246.01	0.00502520663223826\\
247.01	0.00502505995796914\\
248.01	0.00502491037557206\\
249.01	0.00502475782904218\\
250.01	0.00502460226136691\\
251.01	0.00502444361450998\\
252.01	0.00502428182939741\\
253.01	0.00502411684590135\\
254.01	0.00502394860282549\\
255.01	0.00502377703788978\\
256.01	0.00502360208771492\\
257.01	0.00502342368780687\\
258.01	0.00502324177254212\\
259.01	0.00502305627515195\\
260.01	0.00502286712770687\\
261.01	0.00502267426110168\\
262.01	0.00502247760503997\\
263.01	0.00502227708801895\\
264.01	0.0050220726373142\\
265.01	0.00502186417896444\\
266.01	0.00502165163775659\\
267.01	0.00502143493721086\\
268.01	0.00502121399956596\\
269.01	0.00502098874576417\\
270.01	0.00502075909543693\\
271.01	0.00502052496689031\\
272.01	0.00502028627709141\\
273.01	0.00502004294165361\\
274.01	0.00501979487482338\\
275.01	0.00501954198946656\\
276.01	0.00501928419705517\\
277.01	0.00501902140765472\\
278.01	0.00501875352991081\\
279.01	0.00501848047103786\\
280.01	0.00501820213680684\\
281.01	0.00501791843153322\\
282.01	0.00501762925806636\\
283.01	0.00501733451777887\\
284.01	0.00501703411055576\\
285.01	0.00501672793478502\\
286.01	0.00501641588734835\\
287.01	0.00501609786361165\\
288.01	0.00501577375741771\\
289.01	0.0050154434610775\\
290.01	0.00501510686536327\\
291.01	0.0050147638595018\\
292.01	0.00501441433116787\\
293.01	0.00501405816647933\\
294.01	0.00501369524999145\\
295.01	0.0050133254646924\\
296.01	0.00501294869199959\\
297.01	0.00501256481175614\\
298.01	0.00501217370222772\\
299.01	0.00501177524010011\\
300.01	0.00501136930047735\\
301.01	0.00501095575687963\\
302.01	0.00501053448124273\\
303.01	0.00501010534391622\\
304.01	0.00500966821366288\\
305.01	0.0050092229576575\\
306.01	0.00500876944148657\\
307.01	0.00500830752914664\\
308.01	0.00500783708304336\\
309.01	0.00500735796398983\\
310.01	0.00500687003120323\\
311.01	0.00500637314230198\\
312.01	0.0050058671533011\\
313.01	0.0050053519186062\\
314.01	0.00500482729100569\\
315.01	0.00500429312166118\\
316.01	0.00500374926009473\\
317.01	0.00500319555417405\\
318.01	0.00500263185009457\\
319.01	0.00500205799235635\\
320.01	0.00500147382373797\\
321.01	0.0050008791852649\\
322.01	0.00500027391617245\\
323.01	0.00499965785386183\\
324.01	0.00499903083385003\\
325.01	0.00499839268971098\\
326.01	0.00499774325300887\\
327.01	0.0049970823532206\\
328.01	0.00499640981764838\\
329.01	0.00499572547132064\\
330.01	0.00499502913687951\\
331.01	0.00499432063445481\\
332.01	0.00499359978152308\\
333.01	0.0049928663927494\\
334.01	0.00499212027981295\\
335.01	0.00499136125121264\\
336.01	0.00499058911205303\\
337.01	0.00498980366380923\\
338.01	0.00498900470406936\\
339.01	0.00498819202625336\\
340.01	0.0049873654193082\\
341.01	0.00498652466737679\\
342.01	0.00498566954944258\\
343.01	0.00498479983894775\\
344.01	0.00498391530338564\\
345.01	0.00498301570386769\\
346.01	0.00498210079466709\\
347.01	0.00498117032273863\\
348.01	0.00498022402721888\\
349.01	0.00497926163890821\\
350.01	0.00497828287974025\\
351.01	0.00497728746224179\\
352.01	0.00497627508899112\\
353.01	0.00497524545208004\\
354.01	0.00497419823259\\
355.01	0.00497313310009095\\
356.01	0.00497204971217474\\
357.01	0.00497094771403626\\
358.01	0.004969826738117\\
359.01	0.0049686864038248\\
360.01	0.00496752631734916\\
361.01	0.00496634607158831\\
362.01	0.00496514524620618\\
363.01	0.00496392340783756\\
364.01	0.00496268011045868\\
365.01	0.00496141489593708\\
366.01	0.0049601272947751\\
367.01	0.00495881682705102\\
368.01	0.00495748300356199\\
369.01	0.00495612532715941\\
370.01	0.00495474329426123\\
371.01	0.00495333639651\\
372.01	0.00495190412253458\\
373.01	0.00495044595975613\\
374.01	0.00494896139616314\\
375.01	0.00494744992196539\\
376.01	0.0049459110310247\\
377.01	0.0049443442219518\\
378.01	0.00494274899876115\\
379.01	0.0049411248709896\\
380.01	0.0049394713532182\\
381.01	0.00493778796398858\\
382.01	0.00493607422418693\\
383.01	0.00493432965506405\\
384.01	0.00493255377617798\\
385.01	0.00493074610363249\\
386.01	0.00492890614899262\\
387.01	0.00492703341894679\\
388.01	0.0049251274152659\\
389.01	0.00492318763478241\\
390.01	0.00492121356935998\\
391.01	0.00491920470585568\\
392.01	0.00491716052607117\\
393.01	0.0049150805066926\\
394.01	0.00491296411921761\\
395.01	0.00491081082986713\\
396.01	0.00490862009948096\\
397.01	0.0049063913833954\\
398.01	0.0049041241312997\\
399.01	0.00490181778707027\\
400.01	0.00489947178858096\\
401.01	0.00489708556748454\\
402.01	0.004894658548966\\
403.01	0.00489219015146211\\
404.01	0.00488967978634674\\
405.01	0.00488712685757793\\
406.01	0.00488453076130294\\
407.01	0.00488189088542068\\
408.01	0.00487920660909488\\
409.01	0.00487647730221867\\
410.01	0.0048737023248231\\
411.01	0.00487088102643092\\
412.01	0.00486801274534842\\
413.01	0.00486509680789425\\
414.01	0.00486213252756273\\
415.01	0.00485911920411732\\
416.01	0.00485605612261326\\
417.01	0.00485294255234568\\
418.01	0.00484977774572384\\
419.01	0.00484656093706768\\
420.01	0.00484329134132815\\
421.01	0.00483996815273009\\
422.01	0.0048365905433399\\
423.01	0.00483315766155961\\
424.01	0.00482966863054951\\
425.01	0.0048261225465857\\
426.01	0.00482251847735665\\
427.01	0.00481885546020605\\
428.01	0.00481513250033278\\
429.01	0.00481134856895661\\
430.01	0.00480750260146232\\
431.01	0.0048035934955387\\
432.01	0.00479962010932603\\
433.01	0.00479558125959288\\
434.01	0.00479147571996046\\
435.01	0.00478730221919752\\
436.01	0.00478305943960904\\
437.01	0.00477874601554179\\
438.01	0.0047743605320341\\
439.01	0.00476990152363236\\
440.01	0.00476536747339961\\
441.01	0.00476075681213895\\
442.01	0.00475606791784981\\
443.01	0.00475129911543435\\
444.01	0.0047464486766616\\
445.01	0.00474151482039364\\
446.01	0.00473649571306537\\
447.01	0.00473138946940354\\
448.01	0.00472619415335193\\
449.01	0.00472090777916279\\
450.01	0.00471552831259486\\
451.01	0.00471005367214465\\
452.01	0.00470448173022477\\
453.01	0.00469881031418478\\
454.01	0.00469303720706679\\
455.01	0.00468716014797469\\
456.01	0.00468117683194328\\
457.01	0.0046750849091994\\
458.01	0.0046688819837303\\
459.01	0.00466256561110685\\
460.01	0.00465613329555711\\
461.01	0.00464958248634543\\
462.01	0.00464291057358258\\
463.01	0.00463611488366905\\
464.01	0.00462919267463919\\
465.01	0.00462214113172537\\
466.01	0.00461495736346909\\
467.01	0.00460763839865414\\
468.01	0.00460018118420783\\
469.01	0.00459258258401121\\
470.01	0.0045848393783379\\
471.01	0.00457694826359691\\
472.01	0.00456890585225387\\
473.01	0.00456070867290062\\
474.01	0.00455235317045032\\
475.01	0.00454383570642624\\
476.01	0.00453515255930865\\
477.01	0.00452629992489533\\
478.01	0.00451727391663021\\
479.01	0.00450807056584773\\
480.01	0.00449868582188116\\
481.01	0.00448911555198336\\
482.01	0.00447935554101247\\
483.01	0.00446940149084283\\
484.01	0.00445924901947154\\
485.01	0.00444889365980655\\
486.01	0.00443833085813825\\
487.01	0.00442755597231973\\
488.01	0.00441656426969869\\
489.01	0.00440535092486604\\
490.01	0.00439391101730345\\
491.01	0.0043822395290205\\
492.01	0.00437033134227339\\
493.01	0.00435818123744439\\
494.01	0.00434578389113603\\
495.01	0.004333133874493\\
496.01	0.004320225651719\\
497.01	0.00430705357870746\\
498.01	0.00429361190167717\\
499.01	0.00427989475570422\\
500.01	0.00426589616308109\\
501.01	0.00425161003148202\\
502.01	0.00423703015193442\\
503.01	0.00422215019660487\\
504.01	0.00420696371641547\\
505.01	0.0041914641385129\\
506.01	0.00417564476361897\\
507.01	0.00415949876329163\\
508.01	0.00414301917712979\\
509.01	0.00412619890995134\\
510.01	0.0041090307289686\\
511.01	0.00409150726097823\\
512.01	0.00407362098957318\\
513.01	0.00405536425237386\\
514.01	0.00403672923826686\\
515.01	0.00401770798463586\\
516.01	0.00399829237457026\\
517.01	0.00397847413404289\\
518.01	0.00395824482906013\\
519.01	0.00393759586279751\\
520.01	0.00391651847274203\\
521.01	0.00389500372786058\\
522.01	0.00387304252581928\\
523.01	0.00385062559026892\\
524.01	0.00382774346821949\\
525.01	0.00380438652751846\\
526.01	0.00378054495444964\\
527.01	0.00375620875146918\\
528.01	0.00373136773509426\\
529.01	0.00370601153396574\\
530.01	0.00368012958710549\\
531.01	0.00365371114239777\\
532.01	0.00362674525532476\\
533.01	0.00359922078799371\\
534.01	0.00357112640849502\\
535.01	0.00354245059063457\\
536.01	0.00351318161408584\\
537.01	0.00348330756501385\\
538.01	0.00345281633722467\\
539.01	0.00342169563390136\\
540.01	0.00338993296999411\\
541.01	0.00335751567533823\\
542.01	0.00332443089858232\\
543.01	0.00329066561201765\\
544.01	0.00325620661740762\\
545.01	0.00322104055292884\\
546.01	0.00318515390134406\\
547.01	0.00314853299953866\\
548.01	0.00311116404956736\\
549.01	0.00307303313137023\\
550.01	0.00303412621733353\\
551.01	0.00299442918888706\\
552.01	0.00295392785534823\\
553.01	0.00291260797524378\\
554.01	0.00287045528035889\\
555.01	0.00282745550278978\\
556.01	0.00278359440529631\\
557.01	0.00273885781528004\\
558.01	0.00269323166274068\\
559.01	0.00264670202259108\\
560.01	0.00259925516174355\\
561.01	0.0025508775914118\\
562.01	0.00250155612510365\\
563.01	0.00245127794281489\\
564.01	0.00240003066196555\\
565.01	0.00234780241565152\\
566.01	0.00229458193881511\\
567.01	0.00224035866296003\\
568.01	0.00218512282006012\\
569.01	0.00212886555631837\\
570.01	0.00207157905643533\\
571.01	0.00201325667902841\\
572.01	0.00195389310380886\\
573.01	0.00189348449105829\\
574.01	0.00183202865384843\\
575.01	0.00176952524330439\\
576.01	0.00170597594700881\\
577.01	0.00164138470036774\\
578.01	0.00157575791039109\\
579.01	0.00150910469085067\\
580.01	0.00144143710714304\\
581.01	0.00137277042836048\\
582.01	0.00130312338301832\\
583.01	0.00123251841353655\\
584.01	0.00116098192286425\\
585.01	0.00108854450447237\\
586.01	0.00101524114421543\\
587.01	0.000941111379143577\\
588.01	0.000866199394062414\\
589.01	0.000790554031285839\\
590.01	0.000714228682354051\\
591.01	0.000637281022179434\\
592.01	0.000559772535756788\\
593.01	0.000481767774754076\\
594.01	0.000403333265405191\\
595.01	0.00032453596943738\\
596.01	0.000245441175401659\\
597.01	0.000166193599565361\\
598.01	9.1337981519295e-05\\
599.01	2.91271958372443e-05\\
599.02	2.86192783627518e-05\\
599.03	2.81144237420042e-05\\
599.04	2.76126617978888e-05\\
599.05	2.71140226472798e-05\\
599.06	2.66185367039599e-05\\
599.07	2.61262346815533e-05\\
599.08	2.56371475965064e-05\\
599.09	2.51513067710818e-05\\
599.1	2.46687438363886e-05\\
599.11	2.41894907354479e-05\\
599.12	2.37135797262807e-05\\
599.13	2.32410433850267e-05\\
599.14	2.27719146091033e-05\\
599.15	2.23062266203888e-05\\
599.16	2.18440129684267e-05\\
599.17	2.13853075336917e-05\\
599.18	2.0930144530848e-05\\
599.19	2.04785585120812e-05\\
599.2	2.00305843704295e-05\\
599.21	1.95862573431627e-05\\
599.22	1.91456130152045e-05\\
599.23	1.87086873225609e-05\\
599.24	1.82755165558188e-05\\
599.25	1.7846137363638e-05\\
599.26	1.74205874164651e-05\\
599.27	1.69989072017485e-05\\
599.28	1.6581137610194e-05\\
599.29	1.61673199397562e-05\\
599.3	1.57574958996841e-05\\
599.31	1.53517076145696e-05\\
599.32	1.49499976284904e-05\\
599.33	1.45524089091333e-05\\
599.34	1.41589848520109e-05\\
599.35	1.37697692846831e-05\\
599.36	1.33848064710444e-05\\
599.37	1.30041411156422e-05\\
599.38	1.26278183680325e-05\\
599.39	1.22558838271912e-05\\
599.4	1.18883835459674e-05\\
599.41	1.15253640355657e-05\\
599.42	1.11668722700964e-05\\
599.43	1.08129556911519e-05\\
599.44	1.04636622124312e-05\\
599.45	1.01190402244222e-05\\
599.46	9.77913859911625e-06\\
599.47	9.44400669477576e-06\\
599.48	9.11369436074581e-06\\
599.49	8.7882519423238e-06\\
599.5	8.46773028565471e-06\\
599.51	8.15218074270464e-06\\
599.52	7.84165517625675e-06\\
599.53	7.5362059649732e-06\\
599.54	7.23588600849874e-06\\
599.55	6.94074873261973e-06\\
599.56	6.65084809447353e-06\\
599.57	6.36623858780126e-06\\
599.58	6.08697524826993e-06\\
599.59	5.81311365882228e-06\\
599.6	5.54470995511175e-06\\
599.61	5.28182083095637e-06\\
599.62	5.02450354387431e-06\\
599.63	4.7728159206558e-06\\
599.64	4.52681636300273e-06\\
599.65	4.28656385322197e-06\\
599.66	4.05211795995869e-06\\
599.67	3.8235388440163e-06\\
599.68	3.60088726420078e-06\\
599.69	3.38422458325341e-06\\
599.7	3.17361277382168e-06\\
599.71	2.96911442449269e-06\\
599.72	2.77079274589413e-06\\
599.73	2.57871157684567e-06\\
599.74	2.3929353905848e-06\\
599.75	2.213529301031e-06\\
599.76	2.04055906913997e-06\\
599.77	1.87409110929959e-06\\
599.78	1.71419249580043e-06\\
599.79	1.56093096936177e-06\\
599.8	1.41437494372877e-06\\
599.81	1.27459351233379e-06\\
599.82	1.14165645502366e-06\\
599.83	1.01563424484766e-06\\
599.84	8.96598054920053e-07\\
599.85	7.84619765350353e-07\\
599.86	6.79771970232487e-07\\
599.87	5.82127984717282e-07\\
599.88	4.91761852145639e-07\\
599.89	4.08748351251112e-07\\
599.9	3.33163003437068e-07\\
599.91	2.65082080131915e-07\\
599.92	2.04582610201579e-07\\
599.93	1.51742387450443e-07\\
599.94	1.06639978189951e-07\\
599.95	6.93547288800611e-08\\
599.96	3.99667738487652e-08\\
599.97	1.85570430896731e-08\\
599.98	5.20727013418598e-09\\
599.99	0\\
600	0\\
};
\addplot [color=mycolor4,solid,forget plot]
  table[row sep=crcr]{%
0.01	0.00509094250028793\\
1.01	0.0050909413980216\\
2.01	0.00509094027305127\\
3.01	0.00509093912490907\\
4.01	0.00509093795311797\\
5.01	0.00509093675719045\\
6.01	0.00509093553662952\\
7.01	0.00509093429092756\\
8.01	0.00509093301956685\\
9.01	0.00509093172201875\\
10.01	0.00509093039774377\\
11.01	0.00509092904619125\\
12.01	0.00509092766679937\\
13.01	0.00509092625899451\\
14.01	0.00509092482219151\\
15.01	0.00509092335579298\\
16.01	0.00509092185918925\\
17.01	0.00509092033175811\\
18.01	0.00509091877286421\\
19.01	0.00509091718185979\\
20.01	0.0050909155580832\\
21.01	0.00509091390085927\\
22.01	0.00509091220949911\\
23.01	0.00509091048329933\\
24.01	0.00509090872154248\\
25.01	0.00509090692349573\\
26.01	0.00509090508841182\\
27.01	0.00509090321552763\\
28.01	0.00509090130406455\\
29.01	0.00509089935322777\\
30.01	0.0050908973622061\\
31.01	0.0050908953301719\\
32.01	0.00509089325628013\\
33.01	0.00509089113966873\\
34.01	0.00509088897945757\\
35.01	0.00509088677474857\\
36.01	0.00509088452462498\\
37.01	0.00509088222815119\\
38.01	0.00509087988437259\\
39.01	0.00509087749231455\\
40.01	0.00509087505098255\\
41.01	0.00509087255936174\\
42.01	0.00509087001641603\\
43.01	0.00509086742108819\\
44.01	0.00509086477229912\\
45.01	0.00509086206894771\\
46.01	0.00509085930990987\\
47.01	0.00509085649403853\\
48.01	0.00509085362016291\\
49.01	0.00509085068708832\\
50.01	0.00509084769359505\\
51.01	0.00509084463843879\\
52.01	0.00509084152034938\\
53.01	0.00509083833803038\\
54.01	0.0050908350901587\\
55.01	0.00509083177538402\\
56.01	0.00509082839232848\\
57.01	0.00509082493958538\\
58.01	0.00509082141571937\\
59.01	0.00509081781926547\\
60.01	0.00509081414872855\\
61.01	0.00509081040258263\\
62.01	0.00509080657927035\\
63.01	0.00509080267720234\\
64.01	0.00509079869475621\\
65.01	0.00509079463027664\\
66.01	0.00509079048207394\\
67.01	0.00509078624842358\\
68.01	0.00509078192756587\\
69.01	0.00509077751770448\\
70.01	0.00509077301700613\\
71.01	0.00509076842360026\\
72.01	0.00509076373557727\\
73.01	0.00509075895098844\\
74.01	0.00509075406784498\\
75.01	0.00509074908411726\\
76.01	0.00509074399773357\\
77.01	0.00509073880657985\\
78.01	0.00509073350849835\\
79.01	0.00509072810128689\\
80.01	0.00509072258269822\\
81.01	0.00509071695043854\\
82.01	0.00509071120216712\\
83.01	0.00509070533549488\\
84.01	0.00509069934798343\\
85.01	0.0050906932371445\\
86.01	0.00509068700043856\\
87.01	0.00509068063527362\\
88.01	0.00509067413900453\\
89.01	0.00509066750893169\\
90.01	0.00509066074229996\\
91.01	0.00509065383629751\\
92.01	0.00509064678805472\\
93.01	0.00509063959464273\\
94.01	0.00509063225307272\\
95.01	0.0050906247602942\\
96.01	0.0050906171131939\\
97.01	0.00509060930859461\\
98.01	0.00509060134325381\\
99.01	0.00509059321386233\\
100.01	0.00509058491704274\\
101.01	0.00509057644934828\\
102.01	0.00509056780726128\\
103.01	0.0050905589871919\\
104.01	0.00509054998547617\\
105.01	0.00509054079837506\\
106.01	0.00509053142207263\\
107.01	0.00509052185267432\\
108.01	0.00509051208620588\\
109.01	0.00509050211861097\\
110.01	0.00509049194575009\\
111.01	0.00509048156339859\\
112.01	0.00509047096724497\\
113.01	0.0050904601528892\\
114.01	0.005090449115841\\
115.01	0.00509043785151748\\
116.01	0.00509042635524174\\
117.01	0.00509041462224082\\
118.01	0.00509040264764395\\
119.01	0.00509039042647985\\
120.01	0.00509037795367543\\
121.01	0.00509036522405326\\
122.01	0.0050903522323297\\
123.01	0.0050903389731127\\
124.01	0.00509032544089925\\
125.01	0.00509031163007364\\
126.01	0.00509029753490464\\
127.01	0.00509028314954362\\
128.01	0.00509026846802163\\
129.01	0.00509025348424733\\
130.01	0.00509023819200445\\
131.01	0.00509022258494903\\
132.01	0.00509020665660719\\
133.01	0.00509019040037217\\
134.01	0.00509017380950164\\
135.01	0.00509015687711486\\
136.01	0.00509013959619014\\
137.01	0.00509012195956197\\
138.01	0.00509010395991743\\
139.01	0.00509008558979416\\
140.01	0.00509006684157665\\
141.01	0.00509004770749325\\
142.01	0.00509002817961333\\
143.01	0.00509000824984351\\
144.01	0.00508998790992465\\
145.01	0.00508996715142834\\
146.01	0.00508994596575404\\
147.01	0.00508992434412425\\
148.01	0.00508990227758253\\
149.01	0.00508987975698869\\
150.01	0.00508985677301566\\
151.01	0.00508983331614541\\
152.01	0.00508980937666497\\
153.01	0.00508978494466291\\
154.01	0.00508976001002507\\
155.01	0.00508973456243036\\
156.01	0.00508970859134658\\
157.01	0.00508968208602633\\
158.01	0.00508965503550262\\
159.01	0.00508962742858412\\
160.01	0.00508959925385094\\
161.01	0.00508957049965001\\
162.01	0.0050895411540902\\
163.01	0.0050895112050376\\
164.01	0.00508948064011047\\
165.01	0.00508944944667449\\
166.01	0.0050894176118376\\
167.01	0.00508938512244456\\
168.01	0.00508935196507184\\
169.01	0.00508931812602267\\
170.01	0.00508928359132039\\
171.01	0.00508924834670393\\
172.01	0.00508921237762145\\
173.01	0.00508917566922473\\
174.01	0.0050891382063634\\
175.01	0.00508909997357835\\
176.01	0.00508906095509599\\
177.01	0.00508902113482193\\
178.01	0.00508898049633434\\
179.01	0.00508893902287745\\
180.01	0.0050888966973549\\
181.01	0.00508885350232277\\
182.01	0.00508880941998321\\
183.01	0.00508876443217666\\
184.01	0.00508871852037497\\
185.01	0.00508867166567397\\
186.01	0.00508862384878622\\
187.01	0.00508857505003305\\
188.01	0.00508852524933692\\
189.01	0.00508847442621349\\
190.01	0.00508842255976366\\
191.01	0.00508836962866526\\
192.01	0.00508831561116439\\
193.01	0.00508826048506745\\
194.01	0.00508820422773188\\
195.01	0.00508814681605782\\
196.01	0.00508808822647853\\
197.01	0.00508802843495155\\
198.01	0.00508796741694943\\
199.01	0.00508790514744991\\
200.01	0.0050878416009261\\
201.01	0.00508777675133724\\
202.01	0.00508771057211782\\
203.01	0.00508764303616791\\
204.01	0.00508757411584223\\
205.01	0.00508750378294013\\
206.01	0.0050874320086942\\
207.01	0.00508735876375952\\
208.01	0.00508728401820229\\
209.01	0.0050872077414885\\
210.01	0.00508712990247231\\
211.01	0.00508705046938434\\
212.01	0.00508696940981934\\
213.01	0.00508688669072412\\
214.01	0.00508680227838532\\
215.01	0.00508671613841619\\
216.01	0.00508662823574421\\
217.01	0.00508653853459765\\
218.01	0.00508644699849242\\
219.01	0.00508635359021819\\
220.01	0.00508625827182512\\
221.01	0.00508616100460911\\
222.01	0.00508606174909871\\
223.01	0.00508596046503921\\
224.01	0.00508585711137884\\
225.01	0.00508575164625374\\
226.01	0.00508564402697227\\
227.01	0.00508553421000003\\
228.01	0.00508542215094387\\
229.01	0.00508530780453626\\
230.01	0.00508519112461883\\
231.01	0.00508507206412603\\
232.01	0.00508495057506824\\
233.01	0.00508482660851554\\
234.01	0.00508470011457969\\
235.01	0.00508457104239697\\
236.01	0.005084439340111\\
237.01	0.00508430495485411\\
238.01	0.00508416783272951\\
239.01	0.00508402791879278\\
240.01	0.00508388515703334\\
241.01	0.00508373949035549\\
242.01	0.00508359086055902\\
243.01	0.0050834392083201\\
244.01	0.00508328447317157\\
245.01	0.005083126593483\\
246.01	0.00508296550644071\\
247.01	0.00508280114802701\\
248.01	0.00508263345300034\\
249.01	0.00508246235487425\\
250.01	0.00508228778589597\\
251.01	0.00508210967702601\\
252.01	0.00508192795791596\\
253.01	0.00508174255688761\\
254.01	0.00508155340091094\\
255.01	0.00508136041558203\\
256.01	0.00508116352510112\\
257.01	0.00508096265225067\\
258.01	0.00508075771837254\\
259.01	0.00508054864334582\\
260.01	0.0050803353455642\\
261.01	0.00508011774191309\\
262.01	0.00507989574774722\\
263.01	0.00507966927686769\\
264.01	0.00507943824149899\\
265.01	0.00507920255226643\\
266.01	0.00507896211817319\\
267.01	0.00507871684657763\\
268.01	0.00507846664317018\\
269.01	0.0050782114119512\\
270.01	0.00507795105520841\\
271.01	0.00507768547349428\\
272.01	0.00507741456560412\\
273.01	0.00507713822855394\\
274.01	0.00507685635755887\\
275.01	0.00507656884601195\\
276.01	0.00507627558546301\\
277.01	0.0050759764655979\\
278.01	0.00507567137421908\\
279.01	0.00507536019722508\\
280.01	0.00507504281859176\\
281.01	0.00507471912035403\\
282.01	0.00507438898258803\\
283.01	0.00507405228339385\\
284.01	0.00507370889887984\\
285.01	0.00507335870314669\\
286.01	0.00507300156827353\\
287.01	0.00507263736430464\\
288.01	0.00507226595923704\\
289.01	0.00507188721900948\\
290.01	0.00507150100749337\\
291.01	0.00507110718648397\\
292.01	0.00507070561569421\\
293.01	0.00507029615274887\\
294.01	0.00506987865318198\\
295.01	0.00506945297043459\\
296.01	0.00506901895585587\\
297.01	0.00506857645870491\\
298.01	0.00506812532615578\\
299.01	0.00506766540330411\\
300.01	0.00506719653317659\\
301.01	0.00506671855674299\\
302.01	0.00506623131293043\\
303.01	0.00506573463864115\\
304.01	0.00506522836877215\\
305.01	0.00506471233623913\\
306.01	0.0050641863720026\\
307.01	0.00506365030509771\\
308.01	0.00506310396266738\\
309.01	0.00506254716999854\\
310.01	0.00506197975056268\\
311.01	0.00506140152605938\\
312.01	0.00506081231646357\\
313.01	0.00506021194007659\\
314.01	0.00505960021358084\\
315.01	0.0050589769520983\\
316.01	0.00505834196925249\\
317.01	0.00505769507723388\\
318.01	0.0050570360868684\\
319.01	0.00505636480768943\\
320.01	0.00505568104801272\\
321.01	0.00505498461501344\\
322.01	0.0050542753148051\\
323.01	0.00505355295252096\\
324.01	0.00505281733239516\\
325.01	0.00505206825784484\\
326.01	0.00505130553155122\\
327.01	0.00505052895553891\\
328.01	0.00504973833125238\\
329.01	0.00504893345962804\\
330.01	0.00504811414116031\\
331.01	0.00504728017596019\\
332.01	0.00504643136380351\\
333.01	0.00504556750416772\\
334.01	0.00504468839625309\\
335.01	0.00504379383898717\\
336.01	0.00504288363100713\\
337.01	0.00504195757061827\\
338.01	0.00504101545572327\\
339.01	0.00504005708371843\\
340.01	0.00503908225135174\\
341.01	0.00503809075453745\\
342.01	0.00503708238812183\\
343.01	0.0050360569455929\\
344.01	0.00503501421872885\\
345.01	0.0050339539971773\\
346.01	0.00503287606795829\\
347.01	0.00503178021488342\\
348.01	0.00503066621788411\\
349.01	0.00502953385224091\\
350.01	0.00502838288770562\\
351.01	0.0050272130875118\\
352.01	0.00502602420726461\\
353.01	0.00502481599370699\\
354.01	0.00502358818335889\\
355.01	0.00502234050102702\\
356.01	0.00502107265818906\\
357.01	0.0050197843512555\\
358.01	0.00501847525972096\\
359.01	0.00501714504421948\\
360.01	0.00501579334450721\\
361.01	0.00501441977740002\\
362.01	0.00501302393470888\\
363.01	0.00501160538122043\\
364.01	0.00501016365278367\\
365.01	0.00500869825458022\\
366.01	0.00500720865966126\\
367.01	0.0050056943078557\\
368.01	0.00500415460516215\\
369.01	0.00500258892374896\\
370.01	0.00500099660269476\\
371.01	0.00499937694960236\\
372.01	0.00499772924321127\\
373.01	0.00499605273711389\\
374.01	0.00499434666464357\\
375.01	0.00499261024494476\\
376.01	0.00499084269014779\\
377.01	0.00498904321345464\\
378.01	0.00498721103779112\\
379.01	0.00498534540449191\\
380.01	0.00498344558127787\\
381.01	0.00498151086857018\\
382.01	0.00497954060301945\\
383.01	0.0049775341570865\\
384.01	0.00497549093374366\\
385.01	0.00497341035610876\\
386.01	0.0049712918540018\\
387.01	0.00496913485534641\\
388.01	0.00496693878503536\\
389.01	0.0049647030651768\\
390.01	0.0049624271153629\\
391.01	0.00496011035294964\\
392.01	0.00495775219334571\\
393.01	0.00495535205031138\\
394.01	0.00495290933626489\\
395.01	0.00495042346259684\\
396.01	0.00494789383998964\\
397.01	0.00494531987874166\\
398.01	0.00494270098909341\\
399.01	0.00494003658155518\\
400.01	0.00493732606723108\\
401.01	0.00493456885813996\\
402.01	0.00493176436752817\\
403.01	0.0049289120101708\\
404.01	0.00492601120265795\\
405.01	0.00492306136366175\\
406.01	0.0049200619141797\\
407.01	0.00491701227774755\\
408.01	0.00491391188061777\\
409.01	0.00491076015189521\\
410.01	0.00490755652362512\\
411.01	0.00490430043082325\\
412.01	0.00490099131144162\\
413.01	0.00489762860626115\\
414.01	0.00489421175869932\\
415.01	0.00489074021452438\\
416.01	0.00488721342146479\\
417.01	0.00488363082870177\\
418.01	0.00487999188623301\\
419.01	0.00487629604409436\\
420.01	0.00487254275142727\\
421.01	0.00486873145537775\\
422.01	0.00486486159981315\\
423.01	0.00486093262384335\\
424.01	0.00485694396013371\\
425.01	0.00485289503299578\\
426.01	0.00484878525624508\\
427.01	0.00484461403081522\\
428.01	0.00484038074211963\\
429.01	0.00483608475715432\\
430.01	0.00483172542133897\\
431.01	0.00482730205509661\\
432.01	0.00482281395017697\\
433.01	0.00481826036573343\\
434.01	0.00481364052417077\\
435.01	0.00480895360678758\\
436.01	0.0048041987492446\\
437.01	0.00479937503690344\\
438.01	0.00479448150008371\\
439.01	0.00478951710930519\\
440.01	0.00478448077058915\\
441.01	0.00477937132090532\\
442.01	0.0047741875238666\\
443.01	0.00476892806578173\\
444.01	0.00476359155219065\\
445.01	0.00475817650501215\\
446.01	0.00475268136044365\\
447.01	0.00474710446774809\\
448.01	0.00474144408906515\\
449.01	0.00473569840036508\\
450.01	0.00472986549364275\\
451.01	0.00472394338041665\\
452.01	0.00471792999654575\\
453.01	0.00471182320831667\\
454.01	0.00470562081966786\\
455.01	0.00469932058032547\\
456.01	0.00469292019450956\\
457.01	0.0046864173297475\\
458.01	0.00467980962520532\\
459.01	0.004673094698835\\
460.01	0.00466627015254277\\
461.01	0.00465933357455466\\
462.01	0.00465228253820061\\
463.01	0.00464511459651298\\
464.01	0.00463782727236597\\
465.01	0.0046304180444046\\
466.01	0.00462288432973456\\
467.01	0.00461522346520576\\
468.01	0.00460743268997008\\
469.01	0.00459950913243954\\
470.01	0.00459144980352513\\
471.01	0.00458325159387014\\
472.01	0.00457491127220498\\
473.01	0.00456642548437068\\
474.01	0.00455779075305001\\
475.01	0.00454900347823655\\
476.01	0.00454005993845263\\
477.01	0.00453095629270222\\
478.01	0.00452168858311627\\
479.01	0.00451225273821592\\
480.01	0.00450264457668138\\
481.01	0.00449285981147899\\
482.01	0.00448289405415866\\
483.01	0.00447274281910156\\
484.01	0.00446240152746782\\
485.01	0.00445186551057961\\
486.01	0.00444113001247249\\
487.01	0.00443019019136718\\
488.01	0.00441904111986471\\
489.01	0.00440767778374204\\
490.01	0.00439609507933523\\
491.01	0.00438428780963877\\
492.01	0.00437225067940555\\
493.01	0.00435997828969572\\
494.01	0.00434746513245579\\
495.01	0.00433470558577996\\
496.01	0.00432169391045496\\
497.01	0.00430842424817838\\
498.01	0.00429489062144068\\
499.01	0.00428108693452985\\
500.01	0.00426700697480779\\
501.01	0.00425264441371939\\
502.01	0.0042379928073723\\
503.01	0.0042230455966016\\
504.01	0.0042077961064585\\
505.01	0.00419223754508919\\
506.01	0.00417636300200287\\
507.01	0.00416016544577173\\
508.01	0.0041436377212398\\
509.01	0.00412677254635922\\
510.01	0.00410956250879883\\
511.01	0.00409200006248716\\
512.01	0.00407407752424347\\
513.01	0.00405578707062173\\
514.01	0.00403712073503673\\
515.01	0.00401807040516816\\
516.01	0.00399862782055691\\
517.01	0.00397878457025173\\
518.01	0.00395853209035544\\
519.01	0.00393786166138083\\
520.01	0.00391676440541885\\
521.01	0.00389523128316614\\
522.01	0.00387325309086969\\
523.01	0.00385082045725852\\
524.01	0.00382792384052074\\
525.01	0.00380455352538916\\
526.01	0.00378069962038027\\
527.01	0.00375635205522127\\
528.01	0.00373150057848589\\
529.01	0.0037061347554428\\
530.01	0.00368024396611962\\
531.01	0.00365381740358296\\
532.01	0.00362684407245007\\
533.01	0.00359931278766723\\
534.01	0.00357121217359918\\
535.01	0.00354253066349216\\
536.01	0.00351325649936497\\
537.01	0.0034833777323889\\
538.01	0.00345288222381592\\
539.01	0.0034217576465156\\
540.01	0.00338999148718528\\
541.01	0.00335757104930314\\
542.01	0.00332448345690271\\
543.01	0.00329071565925585\\
544.01	0.00325625443656603\\
545.01	0.00322108640678062\\
546.01	0.00318519803364702\\
547.01	0.00314857563614617\\
548.01	0.00311120539944831\\
549.01	0.0030730733875504\\
550.01	0.00303416555776898\\
551.01	0.00299446777727872\\
552.01	0.00295396584190528\\
553.01	0.00291264549740198\\
554.01	0.00287049246346125\\
555.01	0.00282749246073323\\
556.01	0.00278363124114932\\
557.01	0.00273889462187706\\
558.01	0.00269326852325377\\
559.01	0.00264673901108283\\
560.01	0.00259929234370175\\
561.01	0.00255091502426541\\
562.01	0.00250159385872054\\
563.01	0.0024513160199796\\
564.01	0.00240006911883537\\
565.01	0.00234784128218973\\
566.01	0.00229462123919728\\
567.01	0.00224039841595258\\
568.01	0.00218516303936678\\
569.01	0.00212890625089287\\
570.01	0.00207162023075652\\
571.01	0.00201329833333715\\
572.01	0.00195393523430371\\
573.01	0.00189352709004945\\
574.01	0.00183207170986988\\
575.01	0.00176956874118512\\
576.01	0.00170601986790419\\
577.01	0.00164142902175565\\
578.01	0.00157580260603646\\
579.01	0.00150914973074469\\
580.01	0.00144148245742554\\
581.01	0.0013728160512362\\
582.01	0.00130316923667782\\
583.01	0.00123256445209883\\
584.01	0.00116102809635882\\
585.01	0.00108859075888292\\
586.01	0.00101528742161295\\
587.01	0.000941157617941486\\
588.01	0.000866245529433705\\
589.01	0.000790599995791738\\
590.01	0.000714274406843124\\
591.01	0.000637326437030577\\
592.01	0.000559817572555011\\
593.01	0.000481812368507953\\
594.01	0.00040337735743835\\
595.01	0.00032457951111514\\
596.01	0.000245484132888262\\
597.01	0.000166217475640213\\
598.01	9.1337981519295e-05\\
599.01	2.91271958372426e-05\\
599.02	2.86192783627535e-05\\
599.03	2.8114423742006e-05\\
599.04	2.76126617978888e-05\\
599.05	2.71140226472798e-05\\
599.06	2.66185367039599e-05\\
599.07	2.61262346815533e-05\\
599.08	2.56371475965082e-05\\
599.09	2.51513067710835e-05\\
599.1	2.46687438363903e-05\\
599.11	2.41894907354479e-05\\
599.12	2.3713579726279e-05\\
599.13	2.32410433850267e-05\\
599.14	2.27719146091033e-05\\
599.15	2.23062266203871e-05\\
599.16	2.18440129684284e-05\\
599.17	2.13853075336917e-05\\
599.18	2.09301445308497e-05\\
599.19	2.04785585120829e-05\\
599.2	2.00305843704295e-05\\
599.21	1.95862573431627e-05\\
599.22	1.91456130152045e-05\\
599.23	1.87086873225609e-05\\
599.24	1.82755165558171e-05\\
599.25	1.78461373636363e-05\\
599.26	1.74205874164651e-05\\
599.27	1.69989072017502e-05\\
599.28	1.6581137610194e-05\\
599.29	1.61673199397579e-05\\
599.3	1.57574958996824e-05\\
599.31	1.53517076145714e-05\\
599.32	1.49499976284904e-05\\
599.33	1.45524089091333e-05\\
599.34	1.41589848520092e-05\\
599.35	1.37697692846814e-05\\
599.36	1.33848064710462e-05\\
599.37	1.3004141115644e-05\\
599.38	1.26278183680325e-05\\
599.39	1.22558838271912e-05\\
599.4	1.18883835459657e-05\\
599.41	1.15253640355657e-05\\
599.42	1.11668722700964e-05\\
599.43	1.08129556911519e-05\\
599.44	1.04636622124312e-05\\
599.45	1.01190402244239e-05\\
599.46	9.77913859911798e-06\\
599.47	9.44400669477576e-06\\
599.48	9.11369436074755e-06\\
599.49	8.7882519423238e-06\\
599.5	8.46773028565471e-06\\
599.51	8.15218074270464e-06\\
599.52	7.84165517625675e-06\\
599.53	7.5362059649732e-06\\
599.54	7.23588600849874e-06\\
599.55	6.94074873262146e-06\\
599.56	6.65084809447353e-06\\
599.57	6.36623858780126e-06\\
599.58	6.08697524826819e-06\\
599.59	5.81311365882402e-06\\
599.6	5.54470995511175e-06\\
599.61	5.28182083095637e-06\\
599.62	5.02450354387257e-06\\
599.63	4.77281592065407e-06\\
599.64	4.52681636300446e-06\\
599.65	4.28656385322197e-06\\
599.66	4.05211795996042e-06\\
599.67	3.8235388440163e-06\\
599.68	3.60088726420078e-06\\
599.69	3.38422458325514e-06\\
599.7	3.17361277382168e-06\\
599.71	2.96911442449269e-06\\
599.72	2.77079274589413e-06\\
599.73	2.5787115768474e-06\\
599.74	2.39293539058306e-06\\
599.75	2.213529301031e-06\\
599.76	2.04055906913997e-06\\
599.77	1.87409110929959e-06\\
599.78	1.71419249580043e-06\\
599.79	1.56093096936177e-06\\
599.8	1.41437494372877e-06\\
599.81	1.27459351233553e-06\\
599.82	1.14165645502366e-06\\
599.83	1.01563424484766e-06\\
599.84	8.96598054920053e-07\\
599.85	7.84619765348618e-07\\
599.86	6.79771970230753e-07\\
599.87	5.82127984719016e-07\\
599.88	4.91761852147374e-07\\
599.89	4.08748351252847e-07\\
599.9	3.33163003438802e-07\\
599.91	2.65082080131915e-07\\
599.92	2.04582610203313e-07\\
599.93	1.51742387450443e-07\\
599.94	1.06639978191686e-07\\
599.95	6.93547288800611e-08\\
599.96	3.99667738505e-08\\
599.97	1.85570430896731e-08\\
599.98	5.20727013418598e-09\\
599.99	0\\
600	0\\
};
\addplot [color=mycolor5,solid,forget plot]
  table[row sep=crcr]{%
0.01	0.00518985045865594\\
1.01	0.00518984929928745\\
2.01	0.00518984811589717\\
3.01	0.00518984690798687\\
4.01	0.00518984567504769\\
5.01	0.00518984441656035\\
6.01	0.00518984313199469\\
7.01	0.00518984182080953\\
8.01	0.00518984048245244\\
9.01	0.0051898391163596\\
10.01	0.00518983772195551\\
11.01	0.00518983629865253\\
12.01	0.00518983484585082\\
13.01	0.00518983336293832\\
14.01	0.00518983184928979\\
15.01	0.0051898303042675\\
16.01	0.00518982872722019\\
17.01	0.005189827117483\\
18.01	0.00518982547437757\\
19.01	0.00518982379721086\\
20.01	0.00518982208527604\\
21.01	0.00518982033785127\\
22.01	0.00518981855419968\\
23.01	0.00518981673356932\\
24.01	0.00518981487519226\\
25.01	0.00518981297828495\\
26.01	0.0051898110420471\\
27.01	0.00518980906566204\\
28.01	0.00518980704829631\\
29.01	0.00518980498909893\\
30.01	0.00518980288720111\\
31.01	0.00518980074171607\\
32.01	0.00518979855173877\\
33.01	0.00518979631634505\\
34.01	0.00518979403459159\\
35.01	0.00518979170551546\\
36.01	0.00518978932813395\\
37.01	0.00518978690144372\\
38.01	0.00518978442442033\\
39.01	0.00518978189601838\\
40.01	0.00518977931517049\\
41.01	0.00518977668078721\\
42.01	0.00518977399175633\\
43.01	0.00518977124694251\\
44.01	0.00518976844518681\\
45.01	0.00518976558530621\\
46.01	0.00518976266609294\\
47.01	0.00518975968631431\\
48.01	0.0051897566447117\\
49.01	0.00518975354000056\\
50.01	0.00518975037086952\\
51.01	0.0051897471359797\\
52.01	0.00518974383396468\\
53.01	0.00518974046342943\\
54.01	0.00518973702294982\\
55.01	0.00518973351107229\\
56.01	0.00518972992631276\\
57.01	0.00518972626715638\\
58.01	0.00518972253205689\\
59.01	0.00518971871943561\\
60.01	0.00518971482768111\\
61.01	0.00518971085514838\\
62.01	0.00518970680015819\\
63.01	0.00518970266099631\\
64.01	0.00518969843591297\\
65.01	0.00518969412312156\\
66.01	0.00518968972079873\\
67.01	0.00518968522708283\\
68.01	0.00518968064007366\\
69.01	0.00518967595783142\\
70.01	0.00518967117837573\\
71.01	0.00518966629968513\\
72.01	0.00518966131969614\\
73.01	0.00518965623630202\\
74.01	0.00518965104735233\\
75.01	0.00518964575065174\\
76.01	0.00518964034395943\\
77.01	0.00518963482498765\\
78.01	0.0051896291914012\\
79.01	0.00518962344081616\\
80.01	0.00518961757079861\\
81.01	0.00518961157886453\\
82.01	0.00518960546247765\\
83.01	0.00518959921904911\\
84.01	0.00518959284593613\\
85.01	0.00518958634044068\\
86.01	0.00518957969980875\\
87.01	0.00518957292122883\\
88.01	0.00518956600183094\\
89.01	0.00518955893868518\\
90.01	0.00518955172880054\\
91.01	0.00518954436912368\\
92.01	0.00518953685653779\\
93.01	0.00518952918786115\\
94.01	0.00518952135984562\\
95.01	0.00518951336917528\\
96.01	0.00518950521246538\\
97.01	0.00518949688626051\\
98.01	0.00518948838703316\\
99.01	0.00518947971118223\\
100.01	0.00518947085503186\\
101.01	0.00518946181482933\\
102.01	0.00518945258674392\\
103.01	0.00518944316686495\\
104.01	0.00518943355120032\\
105.01	0.00518942373567467\\
106.01	0.00518941371612753\\
107.01	0.00518940348831199\\
108.01	0.00518939304789231\\
109.01	0.0051893823904429\\
110.01	0.00518937151144555\\
111.01	0.00518936040628784\\
112.01	0.0051893490702614\\
113.01	0.00518933749855965\\
114.01	0.00518932568627586\\
115.01	0.00518931362840079\\
116.01	0.00518930131982125\\
117.01	0.00518928875531716\\
118.01	0.00518927592955957\\
119.01	0.00518926283710864\\
120.01	0.00518924947241103\\
121.01	0.00518923582979778\\
122.01	0.00518922190348179\\
123.01	0.00518920768755497\\
124.01	0.00518919317598663\\
125.01	0.00518917836261989\\
126.01	0.00518916324116988\\
127.01	0.00518914780522049\\
128.01	0.00518913204822209\\
129.01	0.00518911596348858\\
130.01	0.00518909954419419\\
131.01	0.00518908278337122\\
132.01	0.00518906567390646\\
133.01	0.00518904820853864\\
134.01	0.00518903037985504\\
135.01	0.00518901218028874\\
136.01	0.00518899360211469\\
137.01	0.00518897463744719\\
138.01	0.00518895527823605\\
139.01	0.00518893551626337\\
140.01	0.00518891534313996\\
141.01	0.00518889475030202\\
142.01	0.0051888737290068\\
143.01	0.0051888522703299\\
144.01	0.00518883036516069\\
145.01	0.00518880800419882\\
146.01	0.00518878517795015\\
147.01	0.00518876187672276\\
148.01	0.00518873809062268\\
149.01	0.00518871380954999\\
150.01	0.00518868902319427\\
151.01	0.00518866372103032\\
152.01	0.0051886378923138\\
153.01	0.00518861152607637\\
154.01	0.00518858461112139\\
155.01	0.00518855713601916\\
156.01	0.00518852908910152\\
157.01	0.00518850045845785\\
158.01	0.00518847123192905\\
159.01	0.00518844139710325\\
160.01	0.00518841094130972\\
161.01	0.00518837985161411\\
162.01	0.00518834811481294\\
163.01	0.00518831571742765\\
164.01	0.00518828264569917\\
165.01	0.00518824888558228\\
166.01	0.00518821442273881\\
167.01	0.00518817924253289\\
168.01	0.00518814333002394\\
169.01	0.00518810666996001\\
170.01	0.0051880692467723\\
171.01	0.00518803104456791\\
172.01	0.00518799204712354\\
173.01	0.00518795223787821\\
174.01	0.00518791159992654\\
175.01	0.00518787011601153\\
176.01	0.00518782776851751\\
177.01	0.00518778453946232\\
178.01	0.00518774041049012\\
179.01	0.00518769536286341\\
180.01	0.0051876493774553\\
181.01	0.00518760243474135\\
182.01	0.00518755451479137\\
183.01	0.00518750559726105\\
184.01	0.00518745566138339\\
185.01	0.00518740468596025\\
186.01	0.00518735264935298\\
187.01	0.00518729952947366\\
188.01	0.00518724530377576\\
189.01	0.00518718994924476\\
190.01	0.00518713344238819\\
191.01	0.00518707575922619\\
192.01	0.0051870168752815\\
193.01	0.00518695676556861\\
194.01	0.00518689540458407\\
195.01	0.00518683276629552\\
196.01	0.00518676882413061\\
197.01	0.00518670355096637\\
198.01	0.00518663691911726\\
199.01	0.00518656890032433\\
200.01	0.00518649946574306\\
201.01	0.00518642858593131\\
202.01	0.00518635623083735\\
203.01	0.00518628236978714\\
204.01	0.00518620697147179\\
205.01	0.00518613000393414\\
206.01	0.00518605143455629\\
207.01	0.0051859712300451\\
208.01	0.00518588935641946\\
209.01	0.00518580577899548\\
210.01	0.00518572046237251\\
211.01	0.00518563337041827\\
212.01	0.0051855444662542\\
213.01	0.00518545371224023\\
214.01	0.00518536106995914\\
215.01	0.00518526650020106\\
216.01	0.00518516996294714\\
217.01	0.00518507141735299\\
218.01	0.00518497082173251\\
219.01	0.0051848681335405\\
220.01	0.00518476330935507\\
221.01	0.00518465630486035\\
222.01	0.0051845470748279\\
223.01	0.00518443557309908\\
224.01	0.00518432175256559\\
225.01	0.00518420556515075\\
226.01	0.00518408696179002\\
227.01	0.00518396589241079\\
228.01	0.00518384230591273\\
229.01	0.00518371615014677\\
230.01	0.00518358737189447\\
231.01	0.00518345591684629\\
232.01	0.00518332172958027\\
233.01	0.00518318475353934\\
234.01	0.00518304493100925\\
235.01	0.0051829022030954\\
236.01	0.00518275650969934\\
237.01	0.00518260778949514\\
238.01	0.00518245597990508\\
239.01	0.00518230101707496\\
240.01	0.00518214283584911\\
241.01	0.00518198136974454\\
242.01	0.00518181655092525\\
243.01	0.00518164831017592\\
244.01	0.00518147657687442\\
245.01	0.00518130127896504\\
246.01	0.0051811223429304\\
247.01	0.00518093969376337\\
248.01	0.00518075325493808\\
249.01	0.00518056294838079\\
250.01	0.00518036869444017\\
251.01	0.00518017041185758\\
252.01	0.00517996801773575\\
253.01	0.00517976142750795\\
254.01	0.00517955055490675\\
255.01	0.00517933531193155\\
256.01	0.00517911560881625\\
257.01	0.00517889135399623\\
258.01	0.00517866245407491\\
259.01	0.00517842881378965\\
260.01	0.00517819033597789\\
261.01	0.00517794692154178\\
262.01	0.00517769846941321\\
263.01	0.00517744487651797\\
264.01	0.00517718603773993\\
265.01	0.00517692184588422\\
266.01	0.00517665219164034\\
267.01	0.00517637696354497\\
268.01	0.00517609604794434\\
269.01	0.00517580932895597\\
270.01	0.00517551668843058\\
271.01	0.00517521800591347\\
272.01	0.00517491315860531\\
273.01	0.0051746020213232\\
274.01	0.00517428446646141\\
275.01	0.00517396036395159\\
276.01	0.00517362958122315\\
277.01	0.0051732919831636\\
278.01	0.00517294743207849\\
279.01	0.00517259578765159\\
280.01	0.00517223690690545\\
281.01	0.00517187064416121\\
282.01	0.00517149685099972\\
283.01	0.00517111537622183\\
284.01	0.00517072606580955\\
285.01	0.00517032876288793\\
286.01	0.00516992330768668\\
287.01	0.00516950953750294\\
288.01	0.00516908728666476\\
289.01	0.00516865638649528\\
290.01	0.00516821666527762\\
291.01	0.00516776794822137\\
292.01	0.00516731005742999\\
293.01	0.00516684281186986\\
294.01	0.00516636602734045\\
295.01	0.00516587951644691\\
296.01	0.00516538308857364\\
297.01	0.00516487654986119\\
298.01	0.00516435970318472\\
299.01	0.00516383234813548\\
300.01	0.00516329428100495\\
301.01	0.00516274529477229\\
302.01	0.00516218517909523\\
303.01	0.00516161372030499\\
304.01	0.00516103070140488\\
305.01	0.00516043590207418\\
306.01	0.00515982909867575\\
307.01	0.00515921006427036\\
308.01	0.00515857856863553\\
309.01	0.00515793437829201\\
310.01	0.00515727725653543\\
311.01	0.00515660696347648\\
312.01	0.00515592325608833\\
313.01	0.00515522588826296\\
314.01	0.00515451461087604\\
315.01	0.00515378917186164\\
316.01	0.00515304931629828\\
317.01	0.00515229478650404\\
318.01	0.00515152532214645\\
319.01	0.00515074066036248\\
320.01	0.0051499405358934\\
321.01	0.0051491246812335\\
322.01	0.00514829282679385\\
323.01	0.00514744470108256\\
324.01	0.00514658003090139\\
325.01	0.00514569854156009\\
326.01	0.00514479995710914\\
327.01	0.00514388400059231\\
328.01	0.00514295039431843\\
329.01	0.00514199886015426\\
330.01	0.00514102911983835\\
331.01	0.00514004089531677\\
332.01	0.00513903390910029\\
333.01	0.00513800788464356\\
334.01	0.00513696254674596\\
335.01	0.00513589762197307\\
336.01	0.0051348128390989\\
337.01	0.00513370792956593\\
338.01	0.00513258262796248\\
339.01	0.00513143667251303\\
340.01	0.00513026980557944\\
341.01	0.00512908177416816\\
342.01	0.00512787233043716\\
343.01	0.00512664123219644\\
344.01	0.00512538824339381\\
345.01	0.00512411313457496\\
346.01	0.00512281568330687\\
347.01	0.0051214956745489\\
348.01	0.00512015290095575\\
349.01	0.00511878716309172\\
350.01	0.00511739826953352\\
351.01	0.00511598603683452\\
352.01	0.00511455028932025\\
353.01	0.00511309085868016\\
354.01	0.00511160758331573\\
355.01	0.00511010030740191\\
356.01	0.00510856887961103\\
357.01	0.00510701315144722\\
358.01	0.00510543297513202\\
359.01	0.00510382820097896\\
360.01	0.00510219867419295\\
361.01	0.0051005442310292\\
362.01	0.00509886469424607\\
363.01	0.00509715986779404\\
364.01	0.00509542953069209\\
365.01	0.00509367343005702\\
366.01	0.00509189127327863\\
367.01	0.00509008271936833\\
368.01	0.0050882473695548\\
369.01	0.00508638475726977\\
370.01	0.00508449433774629\\
371.01	0.00508257547756524\\
372.01	0.0050806274446134\\
373.01	0.0050786493990813\\
374.01	0.00507664038631425\\
375.01	0.00507459933253885\\
376.01	0.00507252504470578\\
377.01	0.00507041621589257\\
378.01	0.00506827143785643\\
379.01	0.00506608922234501\\
380.01	0.00506386803255252\\
381.01	0.0050616063254657\\
382.01	0.00505930260451195\\
383.01	0.00505695547948836\\
384.01	0.00505456372658877\\
385.01	0.00505212633438709\\
386.01	0.00504964248482725\\
387.01	0.00504711141476803\\
388.01	0.00504453235718774\\
389.01	0.00504190454022374\\
390.01	0.0050392271876562\\
391.01	0.00503649951942945\\
392.01	0.00503372075221465\\
393.01	0.00503089010001324\\
394.01	0.00502800677480556\\
395.01	0.00502506998724391\\
396.01	0.00502207894739259\\
397.01	0.00501903286551728\\
398.01	0.0050159309529246\\
399.01	0.00501277242285115\\
400.01	0.00500955649140807\\
401.01	0.00500628237857653\\
402.01	0.00500294930925779\\
403.01	0.00499955651437852\\
404.01	0.00499610323204939\\
405.01	0.00499258870877816\\
406.01	0.00498901220073488\\
407.01	0.00498537297506879\\
408.01	0.00498167031127348\\
409.01	0.00497790350259771\\
410.01	0.00497407185749765\\
411.01	0.00497017470112534\\
412.01	0.00496621137684656\\
413.01	0.00496218124778095\\
414.01	0.00495808369835382\\
415.01	0.00495391813585008\\
416.01	0.00494968399195533\\
417.01	0.0049453807242706\\
418.01	0.00494100781778096\\
419.01	0.00493656478625929\\
420.01	0.00493205117358051\\
421.01	0.00492746655492047\\
422.01	0.00492281053780932\\
423.01	0.00491808276300458\\
424.01	0.00491328290514819\\
425.01	0.00490841067316268\\
426.01	0.00490346581034312\\
427.01	0.00489844809409246\\
428.01	0.00489335733524447\\
429.01	0.00488819337691678\\
430.01	0.00488295609282807\\
431.01	0.00487764538501349\\
432.01	0.00487226118086618\\
433.01	0.00486680342943315\\
434.01	0.0048612720968921\\
435.01	0.00485566716113472\\
436.01	0.00484998860538954\\
437.01	0.00484423641081681\\
438.01	0.00483841054802516\\
439.01	0.00483251096746627\\
440.01	0.00482653758868795\\
441.01	0.0048204902884483\\
442.01	0.00481436888772832\\
443.01	0.00480817313771789\\
444.01	0.00480190270490477\\
445.01	0.00479555715545268\\
446.01	0.0047891359391279\\
447.01	0.00478263837311622\\
448.01	0.004776063626164\\
449.01	0.00476941070358255\\
450.01	0.00476267843376099\\
451.01	0.00475586545694359\\
452.01	0.00474897021713488\\
453.01	0.00474199095807932\\
454.01	0.00473492572432351\\
455.01	0.00472777236837013\\
456.01	0.00472052856486185\\
457.01	0.00471319183255276\\
458.01	0.00470575956449523\\
459.01	0.00469822906634802\\
460.01	0.00469059760196692\\
461.01	0.00468286244441021\\
462.01	0.00467502092919312\\
463.01	0.00466707050506619\\
464.01	0.00465900877590109\\
465.01	0.00465083352570604\\
466.01	0.0046425427178675\\
467.01	0.00463413446036445\\
468.01	0.00462560693245625\\
469.01	0.0046169582789292\\
470.01	0.00460818651563546\\
471.01	0.00459928949483304\\
472.01	0.00459026489236327\\
473.01	0.00458111019614933\\
474.01	0.00457182269565632\\
475.01	0.00456239947264625\\
476.01	0.00455283739358378\\
477.01	0.00454313310406019\\
478.01	0.00453328302559832\\
479.01	0.00452328335518012\\
480.01	0.00451313006779437\\
481.01	0.0045028189222251\\
482.01	0.0044923454701958\\
483.01	0.0044817050688328\\
484.01	0.00447089289622598\\
485.01	0.00445990396962706\\
486.01	0.0044487331655649\\
487.01	0.0044373752408577\\
488.01	0.00442582485320917\\
489.01	0.00441407657980662\\
490.01	0.00440212493215473\\
491.01	0.0043899643653333\\
492.01	0.00437758928005632\\
493.01	0.00436499401641587\\
494.01	0.00435217283911384\\
495.01	0.00433911991536675\\
496.01	0.00432582928848049\\
497.01	0.00431229485208758\\
498.01	0.00429851033159886\\
499.01	0.0042844692787446\\
500.01	0.00427016507750369\\
501.01	0.00425559095354556\\
502.01	0.00424073998420658\\
503.01	0.00422560510846292\\
504.01	0.00421017913641638\\
505.01	0.00419445475777532\\
506.01	0.00417842454882062\\
507.01	0.00416208097738592\\
508.01	0.00414541640548821\\
509.01	0.00412842308940153\\
510.01	0.00411109317718921\\
511.01	0.00409341870398167\\
512.01	0.00407539158558334\\
513.01	0.00405700361126819\\
514.01	0.00403824643681267\\
515.01	0.00401911157882534\\
516.01	0.00399959041117256\\
517.01	0.0039796741636992\\
518.01	0.00395935392255446\\
519.01	0.0039386206307375\\
520.01	0.00391746508787289\\
521.01	0.00389587794901097\\
522.01	0.00387384972249661\\
523.01	0.0038513707670231\\
524.01	0.00382843128805948\\
525.01	0.00380502133389307\\
526.01	0.00378113079156275\\
527.01	0.00375674938295798\\
528.01	0.00373186666131546\\
529.01	0.00370647200826305\\
530.01	0.0036805546314431\\
531.01	0.00365410356262667\\
532.01	0.0036271076561445\\
533.01	0.00359955558746853\\
534.01	0.00357143585189767\\
535.01	0.00354273676342345\\
536.01	0.00351344645390844\\
537.01	0.00348355287270617\\
538.01	0.00345304378684679\\
539.01	0.00342190678189653\\
540.01	0.00339012926358016\\
541.01	0.00335769846023385\\
542.01	0.00332460142614646\\
543.01	0.00329082504584472\\
544.01	0.00325635603939319\\
545.01	0.0032211809688097\\
546.01	0.00318528624572494\\
547.01	0.00314865814043742\\
548.01	0.0031112827925246\\
549.01	0.00307314622317565\\
550.01	0.00303423434942168\\
551.01	0.00299453300045077\\
552.01	0.00295402793620921\\
553.01	0.00291270486851015\\
554.01	0.00287054948489739\\
555.01	0.00282754747553415\\
556.01	0.0027836845634168\\
557.01	0.00273894653823668\\
558.01	0.00269331929424167\\
559.01	0.00264678887247523\\
560.01	0.00259934150780129\\
561.01	0.00255096368115431\\
562.01	0.00250164217748654\\
563.01	0.00245136414991943\\
564.01	0.00240011719063778\\
565.01	0.00234788940909758\\
566.01	0.00229466951814832\\
567.01	0.00224044692869491\\
568.01	0.00218521185354425\\
569.01	0.00212895542109336\\
570.01	0.00207166979951682\\
571.01	0.0020133483320942\\
572.01	0.00195398568428513\\
573.01	0.00189357800309545\\
574.01	0.00183212308917921\\
575.01	0.00176962058197929\\
576.01	0.00170607215800749\\
577.01	0.00164148174208821\\
578.01	0.00157585573102297\\
579.01	0.00150920322864397\\
580.01	0.00144153629058912\\
581.01	0.00137287017630887\\
582.01	0.0013032236047591\\
583.01	0.00123261900888581\\
584.01	0.00116108278229978\\
585.01	0.00108864550937499\\
586.01	0.00101534216728416\\
587.01	0.000941212285065976\\
588.01	0.000866300040538016\\
589.01	0.000790654270520734\\
590.01	0.000714328363169361\\
591.01	0.000637379992907505\\
592.01	0.000559870648136378\\
593.01	0.000481864889083335\\
594.01	0.000403429257268736\\
595.01	0.000324630738394923\\
596.01	0.000245534656112314\\
597.01	0.000166246033647394\\
598.01	9.13379815192916e-05\\
599.01	2.91271958372443e-05\\
599.02	2.86192783627518e-05\\
599.03	2.81144237420042e-05\\
599.04	2.76126617978871e-05\\
599.05	2.71140226472798e-05\\
599.06	2.66185367039581e-05\\
599.07	2.6126234681555e-05\\
599.08	2.56371475965064e-05\\
599.09	2.51513067710818e-05\\
599.1	2.46687438363886e-05\\
599.11	2.41894907354497e-05\\
599.12	2.37135797262807e-05\\
599.13	2.32410433850267e-05\\
599.14	2.2771914609105e-05\\
599.15	2.23062266203888e-05\\
599.16	2.18440129684267e-05\\
599.17	2.13853075336917e-05\\
599.18	2.0930144530848e-05\\
599.19	2.04785585120812e-05\\
599.2	2.00305843704278e-05\\
599.21	1.95862573431627e-05\\
599.22	1.91456130152028e-05\\
599.23	1.87086873225609e-05\\
599.24	1.82755165558171e-05\\
599.25	1.7846137363638e-05\\
599.26	1.74205874164668e-05\\
599.27	1.69989072017485e-05\\
599.28	1.6581137610194e-05\\
599.29	1.61673199397579e-05\\
599.3	1.57574958996841e-05\\
599.31	1.53517076145714e-05\\
599.32	1.49499976284904e-05\\
599.33	1.45524089091333e-05\\
599.34	1.41589848520092e-05\\
599.35	1.37697692846831e-05\\
599.36	1.33848064710462e-05\\
599.37	1.30041411156422e-05\\
599.38	1.26278183680325e-05\\
599.39	1.22558838271929e-05\\
599.4	1.18883835459674e-05\\
599.41	1.15253640355657e-05\\
599.42	1.11668722700981e-05\\
599.43	1.08129556911519e-05\\
599.44	1.04636622124312e-05\\
599.45	1.01190402244222e-05\\
599.46	9.77913859911798e-06\\
599.47	9.44400669477576e-06\\
599.48	9.11369436074581e-06\\
599.49	8.78825194232206e-06\\
599.5	8.46773028565471e-06\\
599.51	8.15218074270464e-06\\
599.52	7.84165517625848e-06\\
599.53	7.5362059649732e-06\\
599.54	7.23588600849874e-06\\
599.55	6.94074873261973e-06\\
599.56	6.65084809447353e-06\\
599.57	6.366238587803e-06\\
599.58	6.08697524826993e-06\\
599.59	5.81311365882228e-06\\
599.6	5.54470995511175e-06\\
599.61	5.28182083095637e-06\\
599.62	5.02450354387431e-06\\
599.63	4.7728159206558e-06\\
599.64	4.52681636300446e-06\\
599.65	4.28656385322197e-06\\
599.66	4.05211795995869e-06\\
599.67	3.82353884401457e-06\\
599.68	3.60088726420078e-06\\
599.69	3.38422458325341e-06\\
599.7	3.17361277382341e-06\\
599.71	2.96911442449269e-06\\
599.72	2.7707927458924e-06\\
599.73	2.57871157684567e-06\\
599.74	2.3929353905848e-06\\
599.75	2.21352930102926e-06\\
599.76	2.04055906913823e-06\\
599.77	1.87409110929959e-06\\
599.78	1.71419249580043e-06\\
599.79	1.56093096936004e-06\\
599.8	1.41437494372877e-06\\
599.81	1.27459351233379e-06\\
599.82	1.14165645502366e-06\\
599.83	1.01563424484766e-06\\
599.84	8.96598054920053e-07\\
599.85	7.84619765350353e-07\\
599.86	6.79771970232487e-07\\
599.87	5.82127984717282e-07\\
599.88	4.91761852145639e-07\\
599.89	4.08748351251112e-07\\
599.9	3.33163003437068e-07\\
599.91	2.65082080131915e-07\\
599.92	2.04582610201579e-07\\
599.93	1.51742387452178e-07\\
599.94	1.06639978189951e-07\\
599.95	6.93547288817958e-08\\
599.96	3.99667738487652e-08\\
599.97	1.85570430896731e-08\\
599.98	5.20727013418598e-09\\
599.99	0\\
600	0\\
};
\addplot [color=mycolor6,solid,forget plot]
  table[row sep=crcr]{%
0.01	0.00535165554740582\\
1.01	0.00535165435177768\\
2.01	0.00535165313126011\\
3.01	0.00535165188533369\\
4.01	0.00535165061346828\\
5.01	0.00535164931512245\\
6.01	0.00535164798974384\\
7.01	0.00535164663676811\\
8.01	0.00535164525561941\\
9.01	0.00535164384570965\\
10.01	0.00535164240643837\\
11.01	0.00535164093719274\\
12.01	0.00535163943734712\\
13.01	0.00535163790626262\\
14.01	0.00535163634328703\\
15.01	0.00535163474775443\\
16.01	0.00535163311898514\\
17.01	0.00535163145628507\\
18.01	0.0053516297589457\\
19.01	0.00535162802624365\\
20.01	0.00535162625744028\\
21.01	0.00535162445178166\\
22.01	0.00535162260849789\\
23.01	0.0053516207268031\\
24.01	0.00535161880589482\\
25.01	0.00535161684495391\\
26.01	0.005351614843144\\
27.01	0.00535161279961132\\
28.01	0.00535161071348379\\
29.01	0.00535160858387155\\
30.01	0.00535160640986589\\
31.01	0.00535160419053901\\
32.01	0.00535160192494384\\
33.01	0.00535159961211312\\
34.01	0.00535159725105974\\
35.01	0.00535159484077575\\
36.01	0.00535159238023177\\
37.01	0.00535158986837724\\
38.01	0.00535158730413941\\
39.01	0.00535158468642296\\
40.01	0.00535158201410972\\
41.01	0.00535157928605808\\
42.01	0.00535157650110247\\
43.01	0.00535157365805277\\
44.01	0.00535157075569409\\
45.01	0.00535156779278586\\
46.01	0.00535156476806174\\
47.01	0.00535156168022837\\
48.01	0.0053515585279657\\
49.01	0.00535155530992578\\
50.01	0.00535155202473251\\
51.01	0.00535154867098089\\
52.01	0.00535154524723616\\
53.01	0.00535154175203372\\
54.01	0.00535153818387834\\
55.01	0.00535153454124298\\
56.01	0.005351530822569\\
57.01	0.00535152702626476\\
58.01	0.00535152315070531\\
59.01	0.00535151919423157\\
60.01	0.00535151515514961\\
61.01	0.00535151103173001\\
62.01	0.00535150682220687\\
63.01	0.00535150252477731\\
64.01	0.00535149813760059\\
65.01	0.00535149365879743\\
66.01	0.00535148908644871\\
67.01	0.00535148441859534\\
68.01	0.00535147965323694\\
69.01	0.00535147478833106\\
70.01	0.00535146982179247\\
71.01	0.0053514647514921\\
72.01	0.00535145957525603\\
73.01	0.0053514542908648\\
74.01	0.00535144889605222\\
75.01	0.00535144338850447\\
76.01	0.00535143776585907\\
77.01	0.00535143202570394\\
78.01	0.00535142616557633\\
79.01	0.00535142018296148\\
80.01	0.00535141407529222\\
81.01	0.00535140783994703\\
82.01	0.00535140147424918\\
83.01	0.005351394975466\\
84.01	0.00535138834080721\\
85.01	0.00535138156742381\\
86.01	0.00535137465240689\\
87.01	0.00535136759278639\\
88.01	0.00535136038552964\\
89.01	0.0053513530275403\\
90.01	0.00535134551565688\\
91.01	0.00535133784665152\\
92.01	0.0053513300172282\\
93.01	0.00535132202402151\\
94.01	0.00535131386359549\\
95.01	0.00535130553244182\\
96.01	0.005351297026978\\
97.01	0.0053512883435465\\
98.01	0.00535127947841268\\
99.01	0.00535127042776328\\
100.01	0.00535126118770487\\
101.01	0.00535125175426191\\
102.01	0.00535124212337513\\
103.01	0.00535123229089991\\
104.01	0.00535122225260437\\
105.01	0.00535121200416742\\
106.01	0.00535120154117719\\
107.01	0.00535119085912848\\
108.01	0.0053511799534217\\
109.01	0.0053511688193598\\
110.01	0.00535115745214741\\
111.01	0.00535114584688781\\
112.01	0.00535113399858112\\
113.01	0.00535112190212204\\
114.01	0.00535110955229786\\
115.01	0.00535109694378628\\
116.01	0.00535108407115214\\
117.01	0.00535107092884647\\
118.01	0.00535105751120296\\
119.01	0.00535104381243604\\
120.01	0.00535102982663815\\
121.01	0.00535101554777718\\
122.01	0.00535100096969379\\
123.01	0.00535098608609906\\
124.01	0.00535097089057116\\
125.01	0.00535095537655314\\
126.01	0.00535093953734955\\
127.01	0.00535092336612398\\
128.01	0.00535090685589577\\
129.01	0.00535088999953703\\
130.01	0.00535087278976958\\
131.01	0.00535085521916177\\
132.01	0.00535083728012522\\
133.01	0.00535081896491143\\
134.01	0.00535080026560837\\
135.01	0.00535078117413728\\
136.01	0.00535076168224889\\
137.01	0.00535074178151985\\
138.01	0.00535072146334912\\
139.01	0.0053507007189542\\
140.01	0.00535067953936729\\
141.01	0.00535065791543123\\
142.01	0.00535063583779594\\
143.01	0.00535061329691356\\
144.01	0.00535059028303504\\
145.01	0.00535056678620547\\
146.01	0.00535054279625969\\
147.01	0.00535051830281814\\
148.01	0.0053504932952821\\
149.01	0.0053504677628291\\
150.01	0.00535044169440818\\
151.01	0.00535041507873501\\
152.01	0.00535038790428707\\
153.01	0.00535036015929866\\
154.01	0.00535033183175527\\
155.01	0.00535030290938899\\
156.01	0.00535027337967294\\
157.01	0.00535024322981521\\
158.01	0.00535021244675423\\
159.01	0.00535018101715209\\
160.01	0.00535014892738953\\
161.01	0.00535011616355948\\
162.01	0.00535008271146103\\
163.01	0.00535004855659329\\
164.01	0.00535001368414908\\
165.01	0.00534997807900831\\
166.01	0.00534994172573162\\
167.01	0.00534990460855314\\
168.01	0.00534986671137403\\
169.01	0.00534982801775518\\
170.01	0.00534978851091022\\
171.01	0.00534974817369797\\
172.01	0.00534970698861489\\
173.01	0.00534966493778769\\
174.01	0.00534962200296519\\
175.01	0.00534957816551045\\
176.01	0.00534953340639275\\
177.01	0.00534948770617875\\
178.01	0.00534944104502456\\
179.01	0.00534939340266683\\
180.01	0.00534934475841364\\
181.01	0.00534929509113565\\
182.01	0.00534924437925669\\
183.01	0.00534919260074481\\
184.01	0.00534913973310177\\
185.01	0.0053490857533536\\
186.01	0.00534903063804076\\
187.01	0.00534897436320744\\
188.01	0.00534891690439144\\
189.01	0.00534885823661272\\
190.01	0.00534879833436352\\
191.01	0.0053487371715961\\
192.01	0.005348674721712\\
193.01	0.00534861095755016\\
194.01	0.00534854585137504\\
195.01	0.00534847937486385\\
196.01	0.00534841149909541\\
197.01	0.00534834219453612\\
198.01	0.00534827143102793\\
199.01	0.00534819917777458\\
200.01	0.00534812540332819\\
201.01	0.00534805007557553\\
202.01	0.00534797316172387\\
203.01	0.00534789462828649\\
204.01	0.00534781444106817\\
205.01	0.00534773256515012\\
206.01	0.00534764896487432\\
207.01	0.00534756360382829\\
208.01	0.00534747644482889\\
209.01	0.00534738744990587\\
210.01	0.00534729658028554\\
211.01	0.00534720379637332\\
212.01	0.00534710905773658\\
213.01	0.00534701232308677\\
214.01	0.00534691355026138\\
215.01	0.0053468126962053\\
216.01	0.00534670971695187\\
217.01	0.00534660456760377\\
218.01	0.00534649720231294\\
219.01	0.0053463875742606\\
220.01	0.00534627563563692\\
221.01	0.0053461613376194\\
222.01	0.00534604463035201\\
223.01	0.00534592546292281\\
224.01	0.00534580378334183\\
225.01	0.00534567953851823\\
226.01	0.00534555267423655\\
227.01	0.00534542313513353\\
228.01	0.00534529086467326\\
229.01	0.00534515580512225\\
230.01	0.00534501789752436\\
231.01	0.00534487708167476\\
232.01	0.00534473329609347\\
233.01	0.00534458647799813\\
234.01	0.00534443656327662\\
235.01	0.00534428348645878\\
236.01	0.00534412718068752\\
237.01	0.00534396757768969\\
238.01	0.00534380460774562\\
239.01	0.005343638199659\\
240.01	0.00534346828072501\\
241.01	0.00534329477669881\\
242.01	0.0053431176117627\\
243.01	0.00534293670849275\\
244.01	0.00534275198782492\\
245.01	0.0053425633690203\\
246.01	0.00534237076962935\\
247.01	0.00534217410545625\\
248.01	0.0053419732905213\\
249.01	0.00534176823702359\\
250.01	0.0053415588553025\\
251.01	0.00534134505379794\\
252.01	0.00534112673901059\\
253.01	0.00534090381546105\\
254.01	0.00534067618564711\\
255.01	0.00534044375000232\\
256.01	0.00534020640685138\\
257.01	0.00533996405236632\\
258.01	0.00533971658052057\\
259.01	0.00533946388304308\\
260.01	0.00533920584937068\\
261.01	0.00533894236660026\\
262.01	0.00533867331943922\\
263.01	0.00533839859015534\\
264.01	0.00533811805852578\\
265.01	0.00533783160178449\\
266.01	0.00533753909456922\\
267.01	0.00533724040886725\\
268.01	0.00533693541395972\\
269.01	0.00533662397636562\\
270.01	0.00533630595978355\\
271.01	0.00533598122503373\\
272.01	0.00533564962999774\\
273.01	0.00533531102955811\\
274.01	0.00533496527553554\\
275.01	0.00533461221662648\\
276.01	0.00533425169833823\\
277.01	0.00533388356292384\\
278.01	0.00533350764931506\\
279.01	0.00533312379305492\\
280.01	0.0053327318262282\\
281.01	0.00533233157739187\\
282.01	0.00533192287150341\\
283.01	0.0053315055298484\\
284.01	0.00533107936996712\\
285.01	0.00533064420557965\\
286.01	0.00533019984651016\\
287.01	0.00532974609861\\
288.01	0.00532928276367988\\
289.01	0.0053288096393909\\
290.01	0.00532832651920482\\
291.01	0.00532783319229312\\
292.01	0.00532732944345536\\
293.01	0.0053268150530369\\
294.01	0.00532628979684592\\
295.01	0.00532575344606914\\
296.01	0.00532520576718839\\
297.01	0.00532464652189533\\
298.01	0.00532407546700686\\
299.01	0.00532349235438009\\
300.01	0.00532289693082711\\
301.01	0.00532228893803049\\
302.01	0.00532166811245883\\
303.01	0.00532103418528305\\
304.01	0.0053203868822939\\
305.01	0.00531972592381969\\
306.01	0.00531905102464723\\
307.01	0.00531836189394287\\
308.01	0.00531765823517729\\
309.01	0.0053169397460516\\
310.01	0.00531620611842825\\
311.01	0.00531545703826419\\
312.01	0.00531469218554937\\
313.01	0.00531391123424979\\
314.01	0.00531311385225667\\
315.01	0.00531229970134192\\
316.01	0.00531146843712099\\
317.01	0.00531061970902503\\
318.01	0.0053097531602816\\
319.01	0.00530886842790669\\
320.01	0.00530796514270932\\
321.01	0.00530704292930905\\
322.01	0.00530610140616988\\
323.01	0.00530514018564999\\
324.01	0.00530415887407184\\
325.01	0.00530315707181312\\
326.01	0.0053021343734214\\
327.01	0.00530109036775548\\
328.01	0.00530002463815613\\
329.01	0.00529893676264953\\
330.01	0.00529782631418644\\
331.01	0.00529669286092161\\
332.01	0.00529553596653746\\
333.01	0.00529435519061616\\
334.01	0.00529315008906588\\
335.01	0.00529192021460581\\
336.01	0.00529066511731612\\
337.01	0.00528938434525932\\
338.01	0.00528807744517931\\
339.01	0.00528674396328614\\
340.01	0.00528538344613292\\
341.01	0.00528399544159428\\
342.01	0.00528257949995408\\
343.01	0.00528113517511113\\
344.01	0.00527966202591209\\
345.01	0.00527815961762109\\
346.01	0.00527662752353379\\
347.01	0.00527506532674537\\
348.01	0.00527347262207958\\
349.01	0.00527184901818537\\
350.01	0.00527019413980575\\
351.01	0.0052685076302207\\
352.01	0.00526678915386272\\
353.01	0.00526503839909783\\
354.01	0.00526325508116023\\
355.01	0.00526143894521889\\
356.01	0.00525958976954443\\
357.01	0.00525770736873251\\
358.01	0.00525579159692007\\
359.01	0.0052538423509145\\
360.01	0.00525185957312387\\
361.01	0.00524984325415177\\
362.01	0.00524779343487412\\
363.01	0.0052457102077765\\
364.01	0.00524359371726887\\
365.01	0.005241444158633\\
366.01	0.00523926177518317\\
367.01	0.00523704685313197\\
368.01	0.00523479971356204\\
369.01	0.00523252070079974\\
370.01	0.00523021016638725\\
371.01	0.00522786844775066\\
372.01	0.00522549584059348\\
373.01	0.00522309256401391\\
374.01	0.00522065871740547\\
375.01	0.00521819422838948\\
376.01	0.00521569879144335\\
377.01	0.00521317179763731\\
378.01	0.00521061225715357\\
379.01	0.00520801871828652\\
380.01	0.00520538918976855\\
381.01	0.00520272107804836\\
382.01	0.00520001115829957\\
383.01	0.00519725560848613\\
384.01	0.00519445015123187\\
385.01	0.00519159039419118\\
386.01	0.00518867336361616\\
387.01	0.00518569786490309\\
388.01	0.00518266281706364\\
389.01	0.00517956712539902\\
390.01	0.00517640968182066\\
391.01	0.00517318936521948\\
392.01	0.00516990504188985\\
393.01	0.00516655556601164\\
394.01	0.00516313978019591\\
395.01	0.00515965651609796\\
396.01	0.00515610459510477\\
397.01	0.00515248282910091\\
398.01	0.00514879002132005\\
399.01	0.0051450249672888\\
400.01	0.00514118645586748\\
401.01	0.00513727327039889\\
402.01	0.00513328418996847\\
403.01	0.00512921799078697\\
404.01	0.00512507344770296\\
405.01	0.00512084933585352\\
406.01	0.00511654443246307\\
407.01	0.0051121575187992\\
408.01	0.00510768738229611\\
409.01	0.00510313281885506\\
410.01	0.00509849263533201\\
411.01	0.00509376565222385\\
412.01	0.00508895070656203\\
413.01	0.00508404665502501\\
414.01	0.00507905237727919\\
415.01	0.00507396677955732\\
416.01	0.00506878879848313\\
417.01	0.0050635174051499\\
418.01	0.00505815160945878\\
419.01	0.00505269046472162\\
420.01	0.0050471330725289\\
421.01	0.005041478587883\\
422.01	0.00503572622459047\\
423.01	0.00502987526090591\\
424.01	0.00502392504540896\\
425.01	0.00501787500309718\\
426.01	0.00501172464166102\\
427.01	0.00500547355790464\\
428.01	0.0049991214442613\\
429.01	0.00499266809533988\\
430.01	0.00498611341442513\\
431.01	0.00497945741983404\\
432.01	0.00497270025101373\\
433.01	0.00496584217424116\\
434.01	0.00495888358775831\\
435.01	0.00495182502614963\\
436.01	0.0049446671637332\\
437.01	0.00493741081670373\\
438.01	0.00493005694372453\\
439.01	0.00492260664462532\\
440.01	0.00491506115682083\\
441.01	0.00490742184902026\\
442.01	0.00489969021175484\\
443.01	0.00489186784421787\\
444.01	0.00488395643687475\\
445.01	0.00487595774929017\\
446.01	0.00486787358261556\\
447.01	0.00485970574621587\\
448.01	0.0048514560179773\\
449.01	0.00484312609796062\\
450.01	0.00483471755525123\\
451.01	0.00482623176813569\\
452.01	0.0048176698581155\\
453.01	0.00480903261880001\\
454.01	0.00480032044140021\\
455.01	0.00479153323943075\\
456.01	0.00478267037631216\\
457.01	0.0047737306008848\\
458.01	0.00476471199736208\\
459.01	0.00475561195792698\\
460.01	0.00474642718786406\\
461.01	0.00473715375459517\\
462.01	0.00472778719280914\\
463.01	0.0047183226773528\\
464.01	0.00470875527255164\\
465.01	0.0046990802593976\\
466.01	0.00468929352788079\\
467.01	0.00467939199655541\\
468.01	0.00466937397912064\\
469.01	0.00465923927801335\\
470.01	0.00464898837771446\\
471.01	0.00463862172466746\\
472.01	0.00462813964410839\\
473.01	0.00461754230514323\\
474.01	0.00460682968311623\\
475.01	0.00459600151964979\\
476.01	0.00458505728090664\\
477.01	0.00457399611483108\\
478.01	0.0045628168083555\\
479.01	0.00455151774581506\\
480.01	0.00454009687008133\\
481.01	0.00452855164820645\\
482.01	0.00451687904364322\\
483.01	0.00450507549734652\\
484.01	0.0044931369202116\\
485.01	0.00448105869933741\\
486.01	0.0044688357204274\\
487.01	0.0044564624081949\\
488.01	0.00444393278581005\\
489.01	0.004431240553115\\
490.01	0.00441837918142996\\
491.01	0.00440534202021139\\
492.01	0.00439212240759072\\
493.01	0.00437871377307201\\
494.01	0.00436510971680188\\
495.01	0.0043513040467084\\
496.01	0.00433729075408134\\
497.01	0.00432306391271775\\
498.01	0.00430861750177217\\
499.01	0.00429394520712976\\
500.01	0.00427904033761374\\
501.01	0.00426389583075258\\
502.01	0.00424850427227856\\
503.01	0.00423285792212514\\
504.01	0.00421694874645758\\
505.01	0.0042007684548149\\
506.01	0.00418430854090724\\
507.01	0.0041675603250459\\
508.01	0.00415051499562202\\
509.01	0.00413316364658496\\
510.01	0.00411549730760114\\
511.01	0.00409750696365053\\
512.01	0.00407918356141116\\
513.01	0.00406051800107318\\
514.01	0.00404150111435926\\
515.01	0.00402212363250523\\
516.01	0.00400237615150024\\
517.01	0.00398224910507929\\
518.01	0.00396173275631887\\
519.01	0.00394081720816598\\
520.01	0.00391949241931278\\
521.01	0.00389774821828174\\
522.01	0.00387557431448599\\
523.01	0.00385296030558535\\
524.01	0.00382989568070651\\
525.01	0.00380636981945661\\
526.01	0.0037823719871091\\
527.01	0.00375789132683927\\
528.01	0.0037329168503583\\
529.01	0.0037074374286157\\
530.01	0.00368144178426106\\
531.01	0.00365491848710693\\
532.01	0.00362785595281683\\
533.01	0.00360024244360208\\
534.01	0.00357206606891766\\
535.01	0.00354331478516688\\
536.01	0.00351397639449492\\
537.01	0.00348403854301973\\
538.01	0.00345348871893436\\
539.01	0.00342231425095655\\
540.01	0.00339050230758179\\
541.01	0.00335803989751079\\
542.01	0.00332491387146892\\
543.01	0.00329111092545075\\
544.01	0.00325661760525977\\
545.01	0.0032214203121659\\
546.01	0.00318550530962556\\
547.01	0.00314885873122135\\
548.01	0.00311146659008362\\
549.01	0.00307331479006465\\
550.01	0.00303438913891908\\
551.01	0.00299467536372011\\
552.01	0.00295415912871867\\
553.01	0.00291282605583985\\
554.01	0.00287066174801436\\
555.01	0.00282765181558129\\
556.01	0.00278378190604264\\
557.01	0.0027390377375069\\
558.01	0.00269340513618998\\
559.01	0.0026468700783629\\
560.01	0.00259941873715503\\
561.01	0.00255103753464357\\
562.01	0.00250171319969066\\
563.01	0.00245143283201888\\
564.01	0.00240018397305644\\
565.01	0.00234795468411658\\
566.01	0.00229473363250886\\
567.01	0.00224051018620399\\
568.01	0.00218527451769199\\
569.01	0.00212901771768357\\
570.01	0.00207173191930344\\
571.01	0.00201341043341097\\
572.01	0.00195404789564968\\
573.01	0.00189364042576395\\
574.01	0.00183218579962798\\
575.01	0.0017696836342872\\
576.01	0.00170613558611251\\
577.01	0.00164154556189476\\
578.01	0.00157591994233822\\
579.01	0.00150926781692674\\
580.01	0.00144160122850227\\
581.01	0.00137293542507396\\
582.01	0.00130328911532177\\
583.01	0.00123268472291186\\
584.01	0.00116114863303132\\
585.01	0.00108871142238989\\
586.01	0.00101540806121248\\
587.01	0.000941278072329516\\
588.01	0.000866365628191963\\
589.01	0.000790719561289255\\
590.01	0.000714393256781154\\
591.01	0.000637444387854076\\
592.01	0.000559934443995484\\
593.01	0.00048192798957512\\
594.01	0.000403491574243162\\
595.01	0.000324692196989477\\
596.01	0.000245595201369226\\
597.01	0.000166284632487157\\
598.01	9.1337981519295e-05\\
599.01	2.91271958372426e-05\\
599.02	2.86192783627518e-05\\
599.03	2.8114423742006e-05\\
599.04	2.76126617978888e-05\\
599.05	2.71140226472781e-05\\
599.06	2.66185367039581e-05\\
599.07	2.61262346815533e-05\\
599.08	2.56371475965082e-05\\
599.09	2.51513067710818e-05\\
599.1	2.46687438363886e-05\\
599.11	2.41894907354479e-05\\
599.12	2.3713579726279e-05\\
599.13	2.3241043385025e-05\\
599.14	2.27719146091033e-05\\
599.15	2.23062266203888e-05\\
599.16	2.18440129684284e-05\\
599.17	2.138530753369e-05\\
599.18	2.09301445308497e-05\\
599.19	2.04785585120829e-05\\
599.2	2.00305843704295e-05\\
599.21	1.95862573431644e-05\\
599.22	1.91456130152045e-05\\
599.23	1.87086873225609e-05\\
599.24	1.82755165558171e-05\\
599.25	1.7846137363638e-05\\
599.26	1.74205874164651e-05\\
599.27	1.69989072017502e-05\\
599.28	1.6581137610194e-05\\
599.29	1.61673199397579e-05\\
599.3	1.57574958996824e-05\\
599.31	1.53517076145714e-05\\
599.32	1.49499976284904e-05\\
599.33	1.45524089091333e-05\\
599.34	1.41589848520109e-05\\
599.35	1.37697692846831e-05\\
599.36	1.33848064710444e-05\\
599.37	1.30041411156422e-05\\
599.38	1.26278183680325e-05\\
599.39	1.22558838271912e-05\\
599.4	1.18883835459657e-05\\
599.41	1.15253640355657e-05\\
599.42	1.11668722700964e-05\\
599.43	1.08129556911519e-05\\
599.44	1.04636622124312e-05\\
599.45	1.01190402244222e-05\\
599.46	9.77913859911625e-06\\
599.47	9.44400669477576e-06\\
599.48	9.11369436074755e-06\\
599.49	8.7882519423238e-06\\
599.5	8.46773028565471e-06\\
599.51	8.15218074270464e-06\\
599.52	7.84165517625501e-06\\
599.53	7.5362059649732e-06\\
599.54	7.23588600849874e-06\\
599.55	6.94074873262146e-06\\
599.56	6.65084809447353e-06\\
599.57	6.36623858780126e-06\\
599.58	6.08697524826819e-06\\
599.59	5.81311365882228e-06\\
599.6	5.54470995511175e-06\\
599.61	5.28182083095637e-06\\
599.62	5.02450354387257e-06\\
599.63	4.77281592065407e-06\\
599.64	4.52681636300273e-06\\
599.65	4.28656385322197e-06\\
599.66	4.05211795996042e-06\\
599.67	3.8235388440163e-06\\
599.68	3.60088726420078e-06\\
599.69	3.38422458325514e-06\\
599.7	3.17361277382168e-06\\
599.71	2.96911442449442e-06\\
599.72	2.77079274589413e-06\\
599.73	2.5787115768474e-06\\
599.74	2.3929353905848e-06\\
599.75	2.213529301031e-06\\
599.76	2.04055906913823e-06\\
599.77	1.87409110929959e-06\\
599.78	1.71419249580043e-06\\
599.79	1.56093096936177e-06\\
599.8	1.41437494372704e-06\\
599.81	1.27459351233379e-06\\
599.82	1.14165645502366e-06\\
599.83	1.01563424484766e-06\\
599.84	8.96598054920053e-07\\
599.85	7.84619765348618e-07\\
599.86	6.79771970232487e-07\\
599.87	5.82127984719016e-07\\
599.88	4.91761852147374e-07\\
599.89	4.08748351251112e-07\\
599.9	3.33163003437068e-07\\
599.91	2.65082080131915e-07\\
599.92	2.04582610201579e-07\\
599.93	1.51742387450443e-07\\
599.94	1.06639978189951e-07\\
599.95	6.93547288800611e-08\\
599.96	3.99667738487652e-08\\
599.97	1.85570430914078e-08\\
599.98	5.20727013418598e-09\\
599.99	0\\
600	0\\
};
\addplot [color=mycolor7,solid,forget plot]
  table[row sep=crcr]{%
0.01	0.00559373691468814\\
1.01	0.00559373577872551\\
2.01	0.00559373461904626\\
3.01	0.00559373343515379\\
4.01	0.00559373222654108\\
5.01	0.0055937309926905\\
6.01	0.00559372973307341\\
7.01	0.0055937284471503\\
8.01	0.00559372713437013\\
9.01	0.00559372579417037\\
10.01	0.00559372442597669\\
11.01	0.00559372302920275\\
12.01	0.00559372160324958\\
13.01	0.00559372014750582\\
14.01	0.00559371866134741\\
15.01	0.00559371714413699\\
16.01	0.00559371559522367\\
17.01	0.00559371401394308\\
18.01	0.00559371239961689\\
19.01	0.00559371075155257\\
20.01	0.00559370906904269\\
21.01	0.00559370735136514\\
22.01	0.00559370559778273\\
23.01	0.00559370380754255\\
24.01	0.00559370197987606\\
25.01	0.00559370011399838\\
26.01	0.0055936982091083\\
27.01	0.00559369626438749\\
28.01	0.00559369427900063\\
29.01	0.00559369225209459\\
30.01	0.00559369018279852\\
31.01	0.00559368807022315\\
32.01	0.00559368591346039\\
33.01	0.0055936837115833\\
34.01	0.00559368146364506\\
35.01	0.00559367916867903\\
36.01	0.00559367682569837\\
37.01	0.00559367443369526\\
38.01	0.00559367199164091\\
39.01	0.00559366949848463\\
40.01	0.00559366695315376\\
41.01	0.00559366435455282\\
42.01	0.00559366170156327\\
43.01	0.00559365899304335\\
44.01	0.00559365622782688\\
45.01	0.00559365340472309\\
46.01	0.00559365052251629\\
47.01	0.00559364757996523\\
48.01	0.00559364457580231\\
49.01	0.00559364150873323\\
50.01	0.00559363837743648\\
51.01	0.00559363518056259\\
52.01	0.00559363191673375\\
53.01	0.00559362858454308\\
54.01	0.00559362518255385\\
55.01	0.00559362170929927\\
56.01	0.00559361816328135\\
57.01	0.00559361454297081\\
58.01	0.00559361084680584\\
59.01	0.00559360707319178\\
60.01	0.00559360322050039\\
61.01	0.00559359928706888\\
62.01	0.00559359527119959\\
63.01	0.0055935911711589\\
64.01	0.00559358698517652\\
65.01	0.00559358271144486\\
66.01	0.00559357834811817\\
67.01	0.00559357389331192\\
68.01	0.00559356934510149\\
69.01	0.00559356470152179\\
70.01	0.00559355996056631\\
71.01	0.00559355512018576\\
72.01	0.00559355017828785\\
73.01	0.00559354513273621\\
74.01	0.00559353998134907\\
75.01	0.00559353472189878\\
76.01	0.00559352935211054\\
77.01	0.00559352386966131\\
78.01	0.00559351827217927\\
79.01	0.00559351255724233\\
80.01	0.00559350672237696\\
81.01	0.00559350076505762\\
82.01	0.00559349468270554\\
83.01	0.00559348847268699\\
84.01	0.00559348213231282\\
85.01	0.00559347565883708\\
86.01	0.00559346904945529\\
87.01	0.00559346230130425\\
88.01	0.00559345541145987\\
89.01	0.00559344837693624\\
90.01	0.00559344119468431\\
91.01	0.00559343386159048\\
92.01	0.00559342637447532\\
93.01	0.00559341873009215\\
94.01	0.00559341092512545\\
95.01	0.00559340295618965\\
96.01	0.00559339481982745\\
97.01	0.00559338651250837\\
98.01	0.00559337803062708\\
99.01	0.00559336937050224\\
100.01	0.00559336052837412\\
101.01	0.00559335150040379\\
102.01	0.00559334228267094\\
103.01	0.00559333287117187\\
104.01	0.00559332326181835\\
105.01	0.00559331345043552\\
106.01	0.00559330343276005\\
107.01	0.00559329320443834\\
108.01	0.00559328276102447\\
109.01	0.00559327209797834\\
110.01	0.00559326121066336\\
111.01	0.00559325009434523\\
112.01	0.00559323874418876\\
113.01	0.00559322715525666\\
114.01	0.00559321532250665\\
115.01	0.00559320324078976\\
116.01	0.00559319090484779\\
117.01	0.0055931783093107\\
118.01	0.00559316544869513\\
119.01	0.00559315231740108\\
120.01	0.0055931389097097\\
121.01	0.00559312521978086\\
122.01	0.00559311124165062\\
123.01	0.00559309696922845\\
124.01	0.00559308239629466\\
125.01	0.0055930675164974\\
126.01	0.00559305232335015\\
127.01	0.00559303681022853\\
128.01	0.00559302097036774\\
129.01	0.00559300479685925\\
130.01	0.00559298828264788\\
131.01	0.00559297142052857\\
132.01	0.00559295420314327\\
133.01	0.00559293662297771\\
134.01	0.00559291867235801\\
135.01	0.0055929003434471\\
136.01	0.00559288162824178\\
137.01	0.00559286251856859\\
138.01	0.00559284300608037\\
139.01	0.0055928230822526\\
140.01	0.00559280273837951\\
141.01	0.0055927819655706\\
142.01	0.00559276075474589\\
143.01	0.005592739096633\\
144.01	0.0055927169817618\\
145.01	0.00559269440046124\\
146.01	0.00559267134285437\\
147.01	0.00559264779885437\\
148.01	0.00559262375815965\\
149.01	0.00559259921024959\\
150.01	0.00559257414437988\\
151.01	0.00559254854957762\\
152.01	0.0055925224146361\\
153.01	0.00559249572811038\\
154.01	0.00559246847831204\\
155.01	0.00559244065330381\\
156.01	0.00559241224089418\\
157.01	0.00559238322863223\\
158.01	0.00559235360380172\\
159.01	0.00559232335341602\\
160.01	0.00559229246421162\\
161.01	0.00559226092264278\\
162.01	0.00559222871487498\\
163.01	0.00559219582677904\\
164.01	0.00559216224392503\\
165.01	0.00559212795157504\\
166.01	0.0055920929346775\\
167.01	0.00559205717786007\\
168.01	0.00559202066542245\\
169.01	0.00559198338132997\\
170.01	0.00559194530920586\\
171.01	0.00559190643232435\\
172.01	0.00559186673360279\\
173.01	0.00559182619559434\\
174.01	0.00559178480048005\\
175.01	0.00559174253006067\\
176.01	0.00559169936574867\\
177.01	0.00559165528856021\\
178.01	0.00559161027910601\\
179.01	0.00559156431758309\\
180.01	0.00559151738376559\\
181.01	0.00559146945699598\\
182.01	0.00559142051617567\\
183.01	0.00559137053975526\\
184.01	0.00559131950572555\\
185.01	0.00559126739160681\\
186.01	0.00559121417443915\\
187.01	0.00559115983077203\\
188.01	0.00559110433665382\\
189.01	0.00559104766762112\\
190.01	0.00559098979868724\\
191.01	0.00559093070433161\\
192.01	0.0055908703584877\\
193.01	0.00559080873453166\\
194.01	0.00559074580526974\\
195.01	0.00559068154292718\\
196.01	0.00559061591913427\\
197.01	0.00559054890491436\\
198.01	0.00559048047067044\\
199.01	0.00559041058617194\\
200.01	0.00559033922054112\\
201.01	0.00559026634223887\\
202.01	0.00559019191905027\\
203.01	0.00559011591807045\\
204.01	0.00559003830568935\\
205.01	0.00558995904757671\\
206.01	0.00558987810866597\\
207.01	0.00558979545313911\\
208.01	0.00558971104440954\\
209.01	0.00558962484510629\\
210.01	0.00558953681705603\\
211.01	0.00558944692126669\\
212.01	0.00558935511790911\\
213.01	0.00558926136629889\\
214.01	0.00558916562487796\\
215.01	0.0055890678511956\\
216.01	0.00558896800188891\\
217.01	0.00558886603266324\\
218.01	0.00558876189827153\\
219.01	0.00558865555249388\\
220.01	0.00558854694811616\\
221.01	0.00558843603690878\\
222.01	0.00558832276960387\\
223.01	0.00558820709587309\\
224.01	0.00558808896430441\\
225.01	0.00558796832237821\\
226.01	0.00558784511644343\\
227.01	0.00558771929169254\\
228.01	0.00558759079213623\\
229.01	0.00558745956057783\\
230.01	0.00558732553858644\\
231.01	0.00558718866647005\\
232.01	0.00558704888324787\\
233.01	0.005586906126622\\
234.01	0.00558676033294844\\
235.01	0.00558661143720742\\
236.01	0.0055864593729732\\
237.01	0.00558630407238297\\
238.01	0.00558614546610514\\
239.01	0.005585983483307\\
240.01	0.00558581805162162\\
241.01	0.00558564909711353\\
242.01	0.00558547654424432\\
243.01	0.00558530031583694\\
244.01	0.00558512033303952\\
245.01	0.00558493651528768\\
246.01	0.00558474878026711\\
247.01	0.00558455704387397\\
248.01	0.00558436122017529\\
249.01	0.00558416122136823\\
250.01	0.00558395695773794\\
251.01	0.00558374833761504\\
252.01	0.00558353526733187\\
253.01	0.00558331765117725\\
254.01	0.00558309539135116\\
255.01	0.005582868387917\\
256.01	0.00558263653875407\\
257.01	0.00558239973950781\\
258.01	0.00558215788353952\\
259.01	0.00558191086187471\\
260.01	0.00558165856314998\\
261.01	0.00558140087355866\\
262.01	0.0055811376767955\\
263.01	0.00558086885399948\\
264.01	0.00558059428369524\\
265.01	0.00558031384173375\\
266.01	0.00558002740123074\\
267.01	0.00557973483250377\\
268.01	0.00557943600300798\\
269.01	0.00557913077726986\\
270.01	0.00557881901682009\\
271.01	0.00557850058012333\\
272.01	0.00557817532250776\\
273.01	0.0055778430960915\\
274.01	0.00557750374970861\\
275.01	0.00557715712883115\\
276.01	0.00557680307549146\\
277.01	0.00557644142820068\\
278.01	0.00557607202186592\\
279.01	0.00557569468770562\\
280.01	0.00557530925316167\\
281.01	0.00557491554180995\\
282.01	0.00557451337326819\\
283.01	0.00557410256310188\\
284.01	0.00557368292272669\\
285.01	0.00557325425930905\\
286.01	0.00557281637566413\\
287.01	0.00557236907015028\\
288.01	0.00557191213656099\\
289.01	0.00557144536401445\\
290.01	0.00557096853683888\\
291.01	0.00557048143445591\\
292.01	0.00556998383125993\\
293.01	0.0055694754964938\\
294.01	0.00556895619412241\\
295.01	0.00556842568270146\\
296.01	0.00556788371524252\\
297.01	0.00556733003907518\\
298.01	0.00556676439570424\\
299.01	0.00556618652066356\\
300.01	0.00556559614336517\\
301.01	0.00556499298694486\\
302.01	0.00556437676810192\\
303.01	0.00556374719693583\\
304.01	0.00556310397677668\\
305.01	0.00556244680401232\\
306.01	0.00556177536790905\\
307.01	0.00556108935042748\\
308.01	0.00556038842603379\\
309.01	0.00555967226150454\\
310.01	0.00555894051572644\\
311.01	0.00555819283949\\
312.01	0.00555742887527775\\
313.01	0.00555664825704568\\
314.01	0.00555585060999905\\
315.01	0.00555503555036238\\
316.01	0.00555420268514182\\
317.01	0.00555335161188214\\
318.01	0.00555248191841649\\
319.01	0.00555159318261087\\
320.01	0.00555068497210045\\
321.01	0.00554975684402067\\
322.01	0.00554880834473149\\
323.01	0.00554783900953537\\
324.01	0.00554684836238993\\
325.01	0.00554583591561429\\
326.01	0.00554480116959046\\
327.01	0.00554374361246058\\
328.01	0.00554266271981949\\
329.01	0.00554155795440458\\
330.01	0.00554042876578421\\
331.01	0.00553927459004481\\
332.01	0.00553809484947935\\
333.01	0.00553688895227882\\
334.01	0.00553565629222924\\
335.01	0.00553439624841643\\
336.01	0.00553310818494279\\
337.01	0.00553179145066014\\
338.01	0.00553044537892288\\
339.01	0.00552906928736697\\
340.01	0.00552766247772401\\
341.01	0.00552622423567475\\
342.01	0.00552475383075512\\
343.01	0.00552325051632485\\
344.01	0.00552171352961172\\
345.01	0.00552014209184854\\
346.01	0.0055185354085209\\
347.01	0.00551689266974859\\
348.01	0.00551521305082619\\
349.01	0.00551349571295421\\
350.01	0.00551173980419672\\
351.01	0.00550994446070776\\
352.01	0.00550810880827644\\
353.01	0.00550623196424911\\
354.01	0.00550431303989498\\
355.01	0.00550235114329564\\
356.01	0.00550034538284874\\
357.01	0.00549829487149175\\
358.01	0.00549619873176579\\
359.01	0.00549405610185998\\
360.01	0.00549186614279095\\
361.01	0.00548962804689502\\
362.01	0.00548734104782902\\
363.01	0.00548500443229107\\
364.01	0.00548261755369217\\
365.01	0.00548017984801175\\
366.01	0.00547769085207107\\
367.01	0.00547515022443356\\
368.01	0.00547255776909233\\
369.01	0.00546991346201305\\
370.01	0.00546721748044338\\
371.01	0.00546447023465396\\
372.01	0.00546167240139801\\
373.01	0.00545882495780955\\
374.01	0.00545592921362698\\
375.01	0.00545298683841993\\
376.01	0.00544999987875847\\
377.01	0.0054469707577876\\
378.01	0.00544390224615777\\
379.01	0.00544079738832047\\
380.01	0.00543765936125108\\
381.01	0.00543449123296023\\
382.01	0.00543129557461255\\
383.01	0.00542807386124254\\
384.01	0.0054248255699084\\
385.01	0.00542154473694886\\
386.01	0.00541820252034409\\
387.01	0.00541479053797377\\
388.01	0.00541130735626098\\
389.01	0.00540775151427604\\
390.01	0.00540412152342795\\
391.01	0.00540041586717635\\
392.01	0.00539663300076763\\
393.01	0.00539277135099786\\
394.01	0.00538882931600653\\
395.01	0.00538480526510645\\
396.01	0.00538069753865354\\
397.01	0.00537650444796261\\
398.01	0.00537222427527503\\
399.01	0.00536785527378419\\
400.01	0.00536339566772774\\
401.01	0.00535884365255251\\
402.01	0.00535419739516333\\
403.01	0.00534945503426495\\
404.01	0.00534461468080661\\
405.01	0.00533967441854422\\
406.01	0.00533463230473173\\
407.01	0.00532948637095737\\
408.01	0.00532423462414129\\
409.01	0.00531887504771252\\
410.01	0.00531340560298583\\
411.01	0.00530782423076023\\
412.01	0.00530212885316444\\
413.01	0.00529631737577533\\
414.01	0.00529038769003946\\
415.01	0.00528433767603014\\
416.01	0.00527816520557665\\
417.01	0.00527186814580291\\
418.01	0.00526544436311995\\
419.01	0.00525889172771916\\
420.01	0.00525220811861548\\
421.01	0.00524539142929793\\
422.01	0.00523843957404536\\
423.01	0.00523135049497339\\
424.01	0.0052241221698815\\
425.01	0.00521675262097326\\
426.01	0.00520923992453015\\
427.01	0.00520158222161964\\
428.01	0.00519377772992578\\
429.01	0.00518582475678883\\
430.01	0.00517772171354748\\
431.01	0.00516946713127028\\
432.01	0.00516105967796527\\
433.01	0.00515249817734653\\
434.01	0.00514378162922886\\
435.01	0.00513490923160295\\
436.01	0.00512588040441994\\
437.01	0.00511669481508393\\
438.01	0.00510735240560287\\
439.01	0.00509785342129101\\
440.01	0.00508819844083878\\
441.01	0.00507838840746606\\
442.01	0.00506842466074768\\
443.01	0.0050583089685407\\
444.01	0.00504804355824356\\
445.01	0.00503763114636934\\
446.01	0.00502707496511496\\
447.01	0.00501637878424186\\
448.01	0.00500554692614615\\
449.01	0.00499458427147987\\
450.01	0.00498349625208561\\
451.01	0.00497228882732307\\
452.01	0.00496096843911425\\
453.01	0.00494954194022756\\
454.01	0.00493801648952166\\
455.01	0.0049263994071451\\
456.01	0.00491469798218742\\
457.01	0.00490291922520205\\
458.01	0.00489106955871362\\
459.01	0.00487915444076691\\
460.01	0.00486717792052282\\
461.01	0.00485514213194154\\
462.01	0.00484304674330157\\
463.01	0.0048308883989877\\
464.01	0.00481866021894127\\
465.01	0.00480635146512563\\
466.01	0.00479394755005908\\
467.01	0.00478143065944836\\
468.01	0.00476878140324357\\
469.01	0.00475598556443265\\
470.01	0.00474304404666418\\
471.01	0.0047299610429181\\
472.01	0.00471674105847669\\
473.01	0.00470338888298013\\
474.01	0.00468990955240711\\
475.01	0.00467630829953881\\
476.01	0.00466259049144648\\
477.01	0.00464876155260209\\
478.01	0.0046348268723669\\
479.01	0.00462079169591118\\
480.01	0.00460666099807384\\
481.01	0.00459243934021112\\
482.01	0.00457813071094355\\
483.01	0.00456373835303028\\
484.01	0.0045492645803004\\
485.01	0.00453471059073541\\
486.01	0.00452007628449796\\
487.01	0.00450536009895463\\
488.01	0.00449055887648122\\
489.01	0.00447566778489476\\
490.01	0.0044606803143166\\
491.01	0.0044455883773876\\
492.01	0.00443038254075207\\
493.01	0.00441505241263782\\
494.01	0.00439958720112588\\
495.01	0.00438397643509666\\
496.01	0.00436821079696735\\
497.01	0.0043522829414424\\
498.01	0.00433618801951533\\
499.01	0.00431992275040816\\
500.01	0.00430348341949066\\
501.01	0.00428686543395026\\
502.01	0.00427006327273361\\
503.01	0.00425307045498408\\
504.01	0.00423587953069189\\
505.01	0.00421848209825503\\
506.01	0.00420086885319224\\
507.01	0.00418302967118661\\
508.01	0.00416495372677486\\
509.01	0.00414662964612302\\
510.01	0.00412804568826002\\
511.01	0.00410918994379239\\
512.01	0.00409005053361105\\
513.01	0.00407061578295647\\
514.01	0.00405087433967378\\
515.01	0.00403081520207299\\
516.01	0.00401042762599367\\
517.01	0.00398970090041908\\
518.01	0.00396862405249262\\
519.01	0.00394718570682074\\
520.01	0.00392537412349316\\
521.01	0.00390317726577278\\
522.01	0.00388058287172507\\
523.01	0.00385757852559785\\
524.01	0.00383415172398894\\
525.01	0.00381028993129461\\
526.01	0.00378598061897748\\
527.01	0.00376121128411935\\
528.01	0.00373596944483621\\
529.01	0.00371024261366338\\
530.01	0.00368401825497795\\
531.01	0.00365728373835029\\
532.01	0.00363002630467935\\
533.01	0.00360223306048554\\
534.01	0.00357389099454814\\
535.01	0.00354498699607012\\
536.01	0.00351550786737225\\
537.01	0.00348544033040776\\
538.01	0.00345477102722176\\
539.01	0.00342348651524043\\
540.01	0.00339157325907328\\
541.01	0.00335901762118412\\
542.01	0.00332580585411584\\
543.01	0.00329192409665191\\
544.01	0.00325735837513591\\
545.01	0.00322209460913947\\
546.01	0.00318611861885679\\
547.01	0.00314941613236037\\
548.01	0.00311197279279638\\
549.01	0.00307377416623498\\
550.01	0.00303480575098859\\
551.01	0.00299505298919984\\
552.01	0.00295450128137567\\
553.01	0.00291313600431742\\
554.01	0.00287094253261798\\
555.01	0.00282790626366628\\
556.01	0.00278401264606778\\
557.01	0.00273924721162163\\
558.01	0.00269359561129114\\
559.01	0.00264704365570551\\
560.01	0.0025995773607181\\
561.01	0.00255118299851699\\
562.01	0.00250184715474731\\
563.01	0.00245155679209126\\
564.01	0.00240029932076293\\
565.01	0.00234806267642719\\
566.01	0.00229483540612065\\
567.01	0.00224060676280948\\
568.01	0.0021853668092348\\
569.01	0.00212910653169041\\
570.01	0.00207181796436378\\
571.01	0.00201349432484503\\
572.01	0.00195413016137354\\
573.01	0.00189372151233586\\
574.01	0.00183226607843893\\
575.01	0.00176976340784625\\
576.01	0.00170621509436756\\
577.01	0.00164162498851609\\
578.01	0.00157599942088352\\
579.01	0.00150934743679982\\
580.01	0.00144168104061919\\
581.01	0.00137301544715818\\
582.01	0.00130336933676586\\
583.01	0.00123276510916492\\
584.01	0.00116122912949844\\
585.01	0.00108879195785878\\
586.01	0.00101548855085613\\
587.01	0.000941358420368682\\
588.01	0.000866445730334548\\
589.01	0.000790799307099266\\
590.01	0.000714472532160199\\
591.01	0.000637523077847722\\
592.01	0.000560012436164931\\
593.01	0.000482005178202047\\
594.01	0.000403567865669439\\
595.01	0.000324767516434078\\
596.01	0.00024566950161945\\
597.01	0.000166349346609286\\
598.01	9.1337981519295e-05\\
599.01	2.91271958372443e-05\\
599.02	2.86192783627535e-05\\
599.03	2.81144237420042e-05\\
599.04	2.76126617978888e-05\\
599.05	2.71140226472798e-05\\
599.06	2.66185367039599e-05\\
599.07	2.6126234681555e-05\\
599.08	2.56371475965064e-05\\
599.09	2.51513067710818e-05\\
599.1	2.46687438363903e-05\\
599.11	2.41894907354479e-05\\
599.12	2.3713579726279e-05\\
599.13	2.32410433850267e-05\\
599.14	2.27719146091033e-05\\
599.15	2.23062266203871e-05\\
599.16	2.18440129684267e-05\\
599.17	2.13853075336917e-05\\
599.18	2.0930144530848e-05\\
599.19	2.04785585120812e-05\\
599.2	2.00305843704295e-05\\
599.21	1.95862573431627e-05\\
599.22	1.91456130152028e-05\\
599.23	1.87086873225627e-05\\
599.24	1.82755165558171e-05\\
599.25	1.78461373636363e-05\\
599.26	1.74205874164668e-05\\
599.27	1.69989072017485e-05\\
599.28	1.6581137610194e-05\\
599.29	1.61673199397562e-05\\
599.3	1.57574958996824e-05\\
599.31	1.53517076145714e-05\\
599.32	1.49499976284904e-05\\
599.33	1.45524089091333e-05\\
599.34	1.41589848520109e-05\\
599.35	1.37697692846814e-05\\
599.36	1.33848064710462e-05\\
599.37	1.3004141115644e-05\\
599.38	1.26278183680325e-05\\
599.39	1.22558838271912e-05\\
599.4	1.18883835459674e-05\\
599.41	1.15253640355657e-05\\
599.42	1.11668722700964e-05\\
599.43	1.08129556911502e-05\\
599.44	1.04636622124312e-05\\
599.45	1.01190402244239e-05\\
599.46	9.77913859911798e-06\\
599.47	9.44400669477576e-06\\
599.48	9.11369436074581e-06\\
599.49	8.7882519423238e-06\\
599.5	8.46773028565471e-06\\
599.51	8.15218074270464e-06\\
599.52	7.84165517625675e-06\\
599.53	7.5362059649732e-06\\
599.54	7.23588600849874e-06\\
599.55	6.94074873261973e-06\\
599.56	6.65084809447353e-06\\
599.57	6.36623858780126e-06\\
599.58	6.08697524826819e-06\\
599.59	5.81311365882228e-06\\
599.6	5.54470995511175e-06\\
599.61	5.28182083095637e-06\\
599.62	5.02450354387431e-06\\
599.63	4.7728159206558e-06\\
599.64	4.52681636300446e-06\\
599.65	4.28656385322197e-06\\
599.66	4.05211795995869e-06\\
599.67	3.82353884401457e-06\\
599.68	3.60088726420078e-06\\
599.69	3.38422458325341e-06\\
599.7	3.17361277382168e-06\\
599.71	2.96911442449269e-06\\
599.72	2.77079274589413e-06\\
599.73	2.57871157684567e-06\\
599.74	2.39293539058306e-06\\
599.75	2.21352930102926e-06\\
599.76	2.04055906913997e-06\\
599.77	1.87409110929959e-06\\
599.78	1.71419249580043e-06\\
599.79	1.56093096936177e-06\\
599.8	1.41437494372877e-06\\
599.81	1.27459351233379e-06\\
599.82	1.14165645502366e-06\\
599.83	1.01563424484592e-06\\
599.84	8.96598054920053e-07\\
599.85	7.84619765350353e-07\\
599.86	6.79771970232487e-07\\
599.87	5.82127984717282e-07\\
599.88	4.91761852145639e-07\\
599.89	4.08748351251112e-07\\
599.9	3.33163003438802e-07\\
599.91	2.65082080131915e-07\\
599.92	2.04582610201579e-07\\
599.93	1.51742387450443e-07\\
599.94	1.06639978191686e-07\\
599.95	6.93547288800611e-08\\
599.96	3.99667738487652e-08\\
599.97	1.85570430896731e-08\\
599.98	5.20727013418598e-09\\
599.99	0\\
600	0\\
};
\addplot [color=mycolor8,solid,forget plot]
  table[row sep=crcr]{%
0.01	0.00586354776097545\\
1.01	0.00586354699753364\\
2.01	0.0058635462181614\\
3.01	0.00586354542252556\\
4.01	0.00586354461028569\\
5.01	0.00586354378109457\\
6.01	0.00586354293459732\\
7.01	0.00586354207043188\\
8.01	0.00586354118822833\\
9.01	0.00586354028760917\\
10.01	0.00586353936818904\\
11.01	0.00586353842957432\\
12.01	0.00586353747136338\\
13.01	0.00586353649314581\\
14.01	0.0058635354945027\\
15.01	0.00586353447500645\\
16.01	0.00586353343422046\\
17.01	0.00586353237169907\\
18.01	0.00586353128698678\\
19.01	0.00586353017961888\\
20.01	0.00586352904912117\\
21.01	0.00586352789500907\\
22.01	0.00586352671678787\\
23.01	0.00586352551395256\\
24.01	0.00586352428598749\\
25.01	0.00586352303236609\\
26.01	0.00586352175255092\\
27.01	0.00586352044599312\\
28.01	0.00586351911213232\\
29.01	0.00586351775039639\\
30.01	0.00586351636020111\\
31.01	0.00586351494094995\\
32.01	0.00586351349203392\\
33.01	0.00586351201283122\\
34.01	0.00586351050270696\\
35.01	0.00586350896101282\\
36.01	0.00586350738708712\\
37.01	0.0058635057802541\\
38.01	0.00586350413982343\\
39.01	0.0058635024650908\\
40.01	0.00586350075533689\\
41.01	0.00586349900982698\\
42.01	0.00586349722781158\\
43.01	0.00586349540852465\\
44.01	0.00586349355118437\\
45.01	0.00586349165499278\\
46.01	0.0058634897191345\\
47.01	0.00586348774277738\\
48.01	0.00586348572507183\\
49.01	0.00586348366515013\\
50.01	0.00586348156212629\\
51.01	0.00586347941509586\\
52.01	0.00586347722313541\\
53.01	0.00586347498530169\\
54.01	0.005863472700632\\
55.01	0.00586347036814312\\
56.01	0.00586346798683123\\
57.01	0.00586346555567125\\
58.01	0.00586346307361665\\
59.01	0.00586346053959878\\
60.01	0.00586345795252653\\
61.01	0.0058634553112857\\
62.01	0.00586345261473877\\
63.01	0.00586344986172414\\
64.01	0.00586344705105583\\
65.01	0.0058634441815228\\
66.01	0.00586344125188856\\
67.01	0.0058634382608904\\
68.01	0.00586343520723901\\
69.01	0.00586343208961823\\
70.01	0.00586342890668371\\
71.01	0.0058634256570632\\
72.01	0.00586342233935517\\
73.01	0.00586341895212866\\
74.01	0.00586341549392263\\
75.01	0.00586341196324509\\
76.01	0.00586340835857256\\
77.01	0.00586340467834979\\
78.01	0.00586340092098829\\
79.01	0.00586339708486616\\
80.01	0.00586339316832763\\
81.01	0.00586338916968147\\
82.01	0.00586338508720121\\
83.01	0.00586338091912381\\
84.01	0.00586337666364898\\
85.01	0.00586337231893833\\
86.01	0.00586336788311515\\
87.01	0.00586336335426255\\
88.01	0.00586335873042362\\
89.01	0.00586335400960003\\
90.01	0.00586334918975102\\
91.01	0.00586334426879306\\
92.01	0.00586333924459848\\
93.01	0.00586333411499474\\
94.01	0.00586332887776354\\
95.01	0.00586332353063956\\
96.01	0.00586331807130985\\
97.01	0.00586331249741234\\
98.01	0.00586330680653548\\
99.01	0.00586330099621634\\
100.01	0.00586329506394038\\
101.01	0.0058632890071395\\
102.01	0.0058632828231916\\
103.01	0.00586327650941947\\
104.01	0.00586327006308891\\
105.01	0.00586326348140817\\
106.01	0.00586325676152639\\
107.01	0.00586324990053264\\
108.01	0.00586324289545428\\
109.01	0.00586323574325608\\
110.01	0.00586322844083854\\
111.01	0.00586322098503655\\
112.01	0.00586321337261819\\
113.01	0.00586320560028331\\
114.01	0.00586319766466197\\
115.01	0.00586318956231267\\
116.01	0.00586318128972157\\
117.01	0.00586317284330049\\
118.01	0.00586316421938505\\
119.01	0.0058631554142335\\
120.01	0.00586314642402532\\
121.01	0.00586313724485888\\
122.01	0.00586312787274997\\
123.01	0.00586311830363034\\
124.01	0.0058631085333453\\
125.01	0.00586309855765268\\
126.01	0.00586308837222042\\
127.01	0.00586307797262473\\
128.01	0.00586306735434818\\
129.01	0.00586305651277781\\
130.01	0.00586304544320323\\
131.01	0.00586303414081391\\
132.01	0.00586302260069802\\
133.01	0.00586301081783939\\
134.01	0.00586299878711581\\
135.01	0.00586298650329675\\
136.01	0.00586297396104055\\
137.01	0.00586296115489279\\
138.01	0.00586294807928361\\
139.01	0.00586293472852487\\
140.01	0.00586292109680837\\
141.01	0.00586290717820243\\
142.01	0.00586289296665033\\
143.01	0.00586287845596634\\
144.01	0.00586286363983437\\
145.01	0.00586284851180404\\
146.01	0.00586283306528856\\
147.01	0.00586281729356154\\
148.01	0.00586280118975397\\
149.01	0.00586278474685129\\
150.01	0.00586276795769033\\
151.01	0.00586275081495593\\
152.01	0.00586273331117821\\
153.01	0.00586271543872859\\
154.01	0.00586269718981692\\
155.01	0.00586267855648781\\
156.01	0.0058626595306172\\
157.01	0.00586264010390873\\
158.01	0.00586262026789011\\
159.01	0.00586260001390948\\
160.01	0.00586257933313123\\
161.01	0.00586255821653237\\
162.01	0.00586253665489849\\
163.01	0.00586251463881976\\
164.01	0.00586249215868649\\
165.01	0.00586246920468516\\
166.01	0.00586244576679403\\
167.01	0.00586242183477841\\
168.01	0.0058623973981865\\
169.01	0.00586237244634459\\
170.01	0.00586234696835229\\
171.01	0.00586232095307766\\
172.01	0.00586229438915253\\
173.01	0.00586226726496695\\
174.01	0.00586223956866466\\
175.01	0.00586221128813726\\
176.01	0.00586218241101915\\
177.01	0.0058621529246819\\
178.01	0.00586212281622859\\
179.01	0.0058620920724884\\
180.01	0.00586206068001038\\
181.01	0.00586202862505758\\
182.01	0.00586199589360082\\
183.01	0.00586196247131266\\
184.01	0.00586192834356069\\
185.01	0.00586189349540151\\
186.01	0.00586185791157337\\
187.01	0.00586182157649009\\
188.01	0.00586178447423335\\
189.01	0.00586174658854625\\
190.01	0.00586170790282575\\
191.01	0.00586166840011501\\
192.01	0.00586162806309602\\
193.01	0.00586158687408193\\
194.01	0.00586154481500916\\
195.01	0.00586150186742886\\
196.01	0.00586145801249904\\
197.01	0.00586141323097598\\
198.01	0.00586136750320569\\
199.01	0.00586132080911484\\
200.01	0.00586127312820189\\
201.01	0.00586122443952771\\
202.01	0.00586117472170627\\
203.01	0.00586112395289524\\
204.01	0.00586107211078559\\
205.01	0.00586101917259156\\
206.01	0.005860965115041\\
207.01	0.00586090991436422\\
208.01	0.00586085354628339\\
209.01	0.0058607959860015\\
210.01	0.00586073720819117\\
211.01	0.00586067718698318\\
212.01	0.00586061589595468\\
213.01	0.00586055330811698\\
214.01	0.00586048939590358\\
215.01	0.00586042413115736\\
216.01	0.00586035748511773\\
217.01	0.00586028942840742\\
218.01	0.00586021993101945\\
219.01	0.00586014896230283\\
220.01	0.00586007649094895\\
221.01	0.00586000248497656\\
222.01	0.005859926911718\\
223.01	0.00585984973780342\\
224.01	0.00585977092914556\\
225.01	0.00585969045092385\\
226.01	0.00585960826756889\\
227.01	0.00585952434274532\\
228.01	0.00585943863933527\\
229.01	0.00585935111942087\\
230.01	0.00585926174426709\\
231.01	0.00585917047430313\\
232.01	0.00585907726910411\\
233.01	0.00585898208737228\\
234.01	0.00585888488691746\\
235.01	0.00585878562463736\\
236.01	0.00585868425649727\\
237.01	0.00585858073750922\\
238.01	0.00585847502171063\\
239.01	0.00585836706214264\\
240.01	0.00585825681082793\\
241.01	0.0058581442187477\\
242.01	0.0058580292358183\\
243.01	0.00585791181086724\\
244.01	0.00585779189160882\\
245.01	0.00585766942461882\\
246.01	0.00585754435530883\\
247.01	0.00585741662789988\\
248.01	0.00585728618539547\\
249.01	0.00585715296955359\\
250.01	0.00585701692085854\\
251.01	0.00585687797849164\\
252.01	0.00585673608030186\\
253.01	0.00585659116277448\\
254.01	0.00585644316100028\\
255.01	0.00585629200864336\\
256.01	0.00585613763790769\\
257.01	0.00585597997950364\\
258.01	0.00585581896261326\\
259.01	0.00585565451485434\\
260.01	0.00585548656224446\\
261.01	0.00585531502916256\\
262.01	0.0058551398383112\\
263.01	0.00585496091067647\\
264.01	0.00585477816548789\\
265.01	0.00585459152017635\\
266.01	0.00585440089033135\\
267.01	0.00585420618965691\\
268.01	0.0058540073299266\\
269.01	0.0058538042209368\\
270.01	0.00585359677045896\\
271.01	0.00585338488419084\\
272.01	0.0058531684657055\\
273.01	0.00585294741639975\\
274.01	0.00585272163544034\\
275.01	0.00585249101970952\\
276.01	0.00585225546374804\\
277.01	0.00585201485969681\\
278.01	0.00585176909723766\\
279.01	0.00585151806353051\\
280.01	0.00585126164315069\\
281.01	0.00585099971802295\\
282.01	0.00585073216735405\\
283.01	0.00585045886756262\\
284.01	0.00585017969220821\\
285.01	0.00584989451191661\\
286.01	0.00584960319430333\\
287.01	0.00584930560389508\\
288.01	0.00584900160204813\\
289.01	0.00584869104686419\\
290.01	0.00584837379310341\\
291.01	0.00584804969209476\\
292.01	0.0058477185916428\\
293.01	0.00584738033593237\\
294.01	0.0058470347654283\\
295.01	0.00584668171677343\\
296.01	0.00584632102268166\\
297.01	0.00584595251182788\\
298.01	0.00584557600873379\\
299.01	0.00584519133364926\\
300.01	0.00584479830242994\\
301.01	0.00584439672640918\\
302.01	0.0058439864122665\\
303.01	0.00584356716188987\\
304.01	0.00584313877223296\\
305.01	0.0058427010351669\\
306.01	0.00584225373732605\\
307.01	0.00584179665994759\\
308.01	0.00584132957870374\\
309.01	0.00584085226352845\\
310.01	0.00584036447843504\\
311.01	0.00583986598132764\\
312.01	0.00583935652380261\\
313.01	0.00583883585094326\\
314.01	0.00583830370110309\\
315.01	0.00583775980568039\\
316.01	0.00583720388888242\\
317.01	0.00583663566747789\\
318.01	0.00583605485053728\\
319.01	0.00583546113916102\\
320.01	0.00583485422619392\\
321.01	0.00583423379592433\\
322.01	0.00583359952376895\\
323.01	0.0058329510759397\\
324.01	0.00583228810909297\\
325.01	0.00583161026995991\\
326.01	0.00583091719495551\\
327.01	0.00583020850976483\\
328.01	0.00582948382890559\\
329.01	0.00582874275526401\\
330.01	0.00582798487960182\\
331.01	0.0058272097800327\\
332.01	0.00582641702146448\\
333.01	0.00582560615500482\\
334.01	0.00582477671732568\\
335.01	0.00582392822998462\\
336.01	0.00582306019869728\\
337.01	0.00582217211255575\\
338.01	0.00582126344318881\\
339.01	0.00582033364385754\\
340.01	0.00581938214847792\\
341.01	0.0058184083705646\\
342.01	0.00581741170208534\\
343.01	0.00581639151221692\\
344.01	0.0058153471459908\\
345.01	0.00581427792281459\\
346.01	0.00581318313485613\\
347.01	0.00581206204527136\\
348.01	0.00581091388625774\\
349.01	0.00580973785691031\\
350.01	0.00580853312085544\\
351.01	0.00580729880363309\\
352.01	0.00580603398979506\\
353.01	0.00580473771968051\\
354.01	0.0058034089858281\\
355.01	0.00580204672897402\\
356.01	0.00580064983358342\\
357.01	0.00579921712285168\\
358.01	0.00579774735311033\\
359.01	0.0057962392075587\\
360.01	0.005794691289242\\
361.01	0.00579310211318587\\
362.01	0.00579147009759589\\
363.01	0.00578979355402939\\
364.01	0.00578807067645009\\
365.01	0.00578629952909211\\
366.01	0.00578447803308131\\
367.01	0.00578260395180925\\
368.01	0.00578067487512613\\
369.01	0.00577868820252627\\
370.01	0.00577664112567159\\
371.01	0.00577453061084078\\
372.01	0.00577235338225243\\
373.01	0.00577010590772886\\
374.01	0.0057677843889034\\
375.01	0.00576538475922506\\
376.01	0.00576290269449254\\
377.01	0.00576033364272751\\
378.01	0.00575767288310775\\
379.01	0.0057549156277403\\
380.01	0.00575205718570993\\
381.01	0.00574909321669317\\
382.01	0.00574602011232909\\
383.01	0.00574283555863858\\
384.01	0.00573953935367677\\
385.01	0.00573613667950795\\
386.01	0.00573265465617449\\
387.01	0.00572909878097666\\
388.01	0.00572546746121069\\
389.01	0.00572175906798681\\
390.01	0.00571797193529673\\
391.01	0.00571410435905338\\
392.01	0.00571015459609959\\
393.01	0.00570612086318736\\
394.01	0.00570200133592409\\
395.01	0.00569779414768695\\
396.01	0.00569349738850444\\
397.01	0.00568910910390217\\
398.01	0.00568462729371486\\
399.01	0.00568004991086283\\
400.01	0.00567537486009077\\
401.01	0.00567059999667202\\
402.01	0.00566572312507429\\
403.01	0.0056607419975887\\
404.01	0.00565565431292296\\
405.01	0.0056504577147575\\
406.01	0.00564514979026664\\
407.01	0.00563972806860564\\
408.01	0.00563419001936582\\
409.01	0.00562853305100012\\
410.01	0.00562275450922289\\
411.01	0.00561685167538734\\
412.01	0.00561082176484714\\
413.01	0.00560466192530814\\
414.01	0.00559836923517888\\
415.01	0.0055919407019305\\
416.01	0.0055853732604774\\
417.01	0.00557866377159529\\
418.01	0.00557180902039386\\
419.01	0.00556480571486572\\
420.01	0.00555765048453973\\
421.01	0.0055503398792679\\
422.01	0.00554287036818504\\
423.01	0.00553523833888587\\
424.01	0.0055274400968719\\
425.01	0.00551947186533327\\
426.01	0.00551132978533816\\
427.01	0.00550300991652206\\
428.01	0.00549450823837915\\
429.01	0.00548582065228439\\
430.01	0.00547694298438933\\
431.01	0.00546787098956975\\
432.01	0.00545860035662863\\
433.01	0.00544912671499583\\
434.01	0.00543944564321008\\
435.01	0.00542955267951603\\
436.01	0.00541944333496989\\
437.01	0.0054091131095094\\
438.01	0.00539855751152491\\
439.01	0.00538777208155469\\
440.01	0.00537675242082517\\
441.01	0.00536549422547395\\
442.01	0.00535399332741376\\
443.01	0.00534224574293606\\
444.01	0.0053302477302982\\
445.01	0.00531799585769161\\
446.01	0.00530548708314142\\
447.01	0.00529271884802473\\
448.01	0.00527968918600638\\
449.01	0.0052663968492423\\
450.01	0.00525284145365738\\
451.01	0.00523902364490384\\
452.01	0.00522494528617217\\
453.01	0.00521060966822928\\
454.01	0.00519602174073492\\
455.01	0.00518118836179257\\
456.01	0.00516611855948602\\
457.01	0.00515082379435854\\
458.01	0.00513531820474412\\
459.01	0.00511961880662152\\
460.01	0.00510374560494541\\
461.01	0.0050877215523977\\
462.01	0.00507157226170662\\
463.01	0.00505532533562082\\
464.01	0.00503900911949455\\
465.01	0.00502265059858424\\
466.01	0.00500627204632678\\
467.01	0.00498988586825628\\
468.01	0.00497348660526829\\
469.01	0.00495693495608155\\
470.01	0.00494013045788559\\
471.01	0.00492307702817898\\
472.01	0.00490577951810552\\
473.01	0.00488824379014157\\
474.01	0.00487047679516144\\
475.01	0.00485248664654519\\
476.01	0.00483428268822604\\
477.01	0.00481587555260846\\
478.01	0.00479727720309214\\
479.01	0.00477850095447716\\
480.01	0.00475956146398742\\
481.01	0.0047404746854527\\
482.01	0.00472125777491141\\
483.01	0.00470192893380416\\
484.01	0.00468250717426736\\
485.01	0.00466301198928942\\
486.01	0.00464346290933059\\
487.01	0.00462387892700592\\
488.01	0.00460427777359631\\
489.01	0.00458467503696179\\
490.01	0.0045650831221016\\
491.01	0.00454551007643988\\
492.01	0.0045259583355222\\
493.01	0.00450642349236445\\
494.01	0.00448689326883421\\
495.01	0.00446734698849532\\
496.01	0.00444775602909078\\
497.01	0.00442808599578846\\
498.01	0.00440830387864604\\
499.01	0.0043884034691836\\
500.01	0.00436839295302929\\
501.01	0.00434827984393562\\
502.01	0.00432807049989737\\
503.01	0.00430776980078569\\
504.01	0.00428738081034435\\
505.01	0.00426690443532456\\
506.01	0.00424633909993131\\
507.01	0.00422568046015354\\
508.01	0.00420492118975192\\
509.01	0.00418405087717167\\
510.01	0.00416305607942218\\
511.01	0.00414192058318003\\
512.01	0.00412062592194331\\
513.01	0.00409915218592861\\
514.01	0.0040774791305508\\
515.01	0.00405558752729416\\
516.01	0.00403346058849445\\
517.01	0.00401108509045564\\
518.01	0.00398845063170133\\
519.01	0.00396554635680371\\
520.01	0.00394235996915327\\
521.01	0.00391887778858746\\
522.01	0.00389508487529542\\
523.01	0.00387096521214111\\
524.01	0.00384650194411\\
525.01	0.00382167766677865\\
526.01	0.00379647474643057\\
527.01	0.00377087564355615\\
528.01	0.00374486319990149\\
529.01	0.00371842083863281\\
530.01	0.0036915326220562\\
531.01	0.00366418311985197\\
532.01	0.00363635707841285\\
533.01	0.00360803903626365\\
534.01	0.00357921322779472\\
535.01	0.00354986368087305\\
536.01	0.00351997432567058\\
537.01	0.0034895290929901\\
538.01	0.00345851199351856\\
539.01	0.00342690716999389\\
540.01	0.00339469891645227\\
541.01	0.00336187166302356\\
542.01	0.0033284099314067\\
543.01	0.00329429827480038\\
544.01	0.00325952122485604\\
545.01	0.00322406327163148\\
546.01	0.00318790888131798\\
547.01	0.00315104252280078\\
548.01	0.00311344868702179\\
549.01	0.00307511189818222\\
550.01	0.00303601671811203\\
551.01	0.00299614774639155\\
552.01	0.00295548961988336\\
553.01	0.00291402701583492\\
554.01	0.00287174466218705\\
555.01	0.00282862735678643\\
556.01	0.00278465999400613\\
557.01	0.00273982759516147\\
558.01	0.00269411534113635\\
559.01	0.00264750860806679\\
560.01	0.00259999300743791\\
561.01	0.00255155443193755\\
562.01	0.00250217910821201\\
563.01	0.0024518536573069\\
564.01	0.00240056516314333\\
565.01	0.00234830124906577\\
566.01	0.00229505016254032\\
567.01	0.0022408008684914\\
568.01	0.0021855431520882\\
569.01	0.00212926773180824\\
570.01	0.00207196638350766\\
571.01	0.00201363207610263\\
572.01	0.00195425911933926\\
573.01	0.00189384332402592\\
574.01	0.00183238217502583\\
575.01	0.00176987501722761\\
576.01	0.00170632325455657\\
577.01	0.00164173056181327\\
578.01	0.00157610310873702\\
579.01	0.00150944979519541\\
580.01	0.00144178249576928\\
581.01	0.00137311631120603\\
582.01	0.00130346982319341\\
583.01	0.00123286534759725\\
584.01	0.00116132917962666\\
585.01	0.00108889182225053\\
586.01	0.00101558818648855\\
587.01	0.000941457748798396\\
588.01	0.000866544646519892\\
589.01	0.000790897686996644\\
590.01	0.000714570239330877\\
591.01	0.000637619969415817\\
592.01	0.000560108368556696\\
593.01	0.000482100013165267\\
594.01	0.000403661477119359\\
595.01	0.000324859798708405\\
596.01	0.000245760379760691\\
597.01	0.000166429767496855\\
598.01	9.13379815192933e-05\\
599.01	2.91271958372443e-05\\
599.02	2.86192783627518e-05\\
599.03	2.8114423742006e-05\\
599.04	2.76126617978871e-05\\
599.05	2.71140226472798e-05\\
599.06	2.66185367039581e-05\\
599.07	2.61262346815533e-05\\
599.08	2.56371475965064e-05\\
599.09	2.51513067710818e-05\\
599.1	2.46687438363886e-05\\
599.11	2.41894907354479e-05\\
599.12	2.37135797262807e-05\\
599.13	2.32410433850267e-05\\
599.14	2.2771914609105e-05\\
599.15	2.23062266203888e-05\\
599.16	2.18440129684284e-05\\
599.17	2.13853075336917e-05\\
599.18	2.09301445308497e-05\\
599.19	2.04785585120812e-05\\
599.2	2.00305843704278e-05\\
599.21	1.95862573431627e-05\\
599.22	1.91456130152045e-05\\
599.23	1.87086873225609e-05\\
599.24	1.82755165558188e-05\\
599.25	1.7846137363638e-05\\
599.26	1.74205874164651e-05\\
599.27	1.69989072017502e-05\\
599.28	1.6581137610194e-05\\
599.29	1.61673199397579e-05\\
599.3	1.57574958996841e-05\\
599.31	1.53517076145696e-05\\
599.32	1.49499976284887e-05\\
599.33	1.45524089091333e-05\\
599.34	1.41589848520092e-05\\
599.35	1.37697692846831e-05\\
599.36	1.33848064710444e-05\\
599.37	1.30041411156422e-05\\
599.38	1.26278183680325e-05\\
599.39	1.22558838271929e-05\\
599.4	1.18883835459657e-05\\
599.41	1.15253640355657e-05\\
599.42	1.11668722700964e-05\\
599.43	1.08129556911519e-05\\
599.44	1.04636622124312e-05\\
599.45	1.01190402244222e-05\\
599.46	9.77913859911625e-06\\
599.47	9.44400669477576e-06\\
599.48	9.11369436074755e-06\\
599.49	8.78825194232206e-06\\
599.5	8.46773028565471e-06\\
599.51	8.15218074270464e-06\\
599.52	7.84165517625675e-06\\
599.53	7.5362059649732e-06\\
599.54	7.23588600849874e-06\\
599.55	6.94074873262146e-06\\
599.56	6.65084809447353e-06\\
599.57	6.36623858780126e-06\\
599.58	6.08697524826993e-06\\
599.59	5.81311365882228e-06\\
599.6	5.54470995511175e-06\\
599.61	5.28182083095637e-06\\
599.62	5.02450354387257e-06\\
599.63	4.7728159206558e-06\\
599.64	4.52681636300273e-06\\
599.65	4.28656385322197e-06\\
599.66	4.05211795995869e-06\\
599.67	3.8235388440163e-06\\
599.68	3.60088726419905e-06\\
599.69	3.38422458325514e-06\\
599.7	3.17361277382168e-06\\
599.71	2.96911442449269e-06\\
599.72	2.77079274589413e-06\\
599.73	2.57871157684567e-06\\
599.74	2.39293539058306e-06\\
599.75	2.213529301031e-06\\
599.76	2.04055906913997e-06\\
599.77	1.87409110930133e-06\\
599.78	1.71419249580043e-06\\
599.79	1.56093096936177e-06\\
599.8	1.41437494372877e-06\\
599.81	1.27459351233379e-06\\
599.82	1.14165645502366e-06\\
599.83	1.01563424484766e-06\\
599.84	8.96598054920053e-07\\
599.85	7.84619765348618e-07\\
599.86	6.79771970232487e-07\\
599.87	5.82127984719016e-07\\
599.88	4.91761852147374e-07\\
599.89	4.08748351251112e-07\\
599.9	3.33163003437068e-07\\
599.91	2.65082080131915e-07\\
599.92	2.04582610201579e-07\\
599.93	1.51742387452178e-07\\
599.94	1.06639978189951e-07\\
599.95	6.93547288800611e-08\\
599.96	3.99667738487652e-08\\
599.97	1.85570430896731e-08\\
599.98	5.20727013418598e-09\\
599.99	0\\
600	0\\
};
\addplot [color=blue!25!mycolor7,solid,forget plot]
  table[row sep=crcr]{%
0.01	0.00599920774511898\\
1.01	0.00599920717618128\\
2.01	0.00599920659536472\\
3.01	0.00599920600242077\\
4.01	0.00599920539709537\\
5.01	0.00599920477912935\\
6.01	0.00599920414825806\\
7.01	0.00599920350421131\\
8.01	0.00599920284671312\\
9.01	0.00599920217548179\\
10.01	0.00599920149022968\\
11.01	0.0059992007906631\\
12.01	0.00599920007648221\\
13.01	0.00599919934738092\\
14.01	0.00599919860304673\\
15.01	0.00599919784316028\\
16.01	0.0059991970673959\\
17.01	0.00599919627542072\\
18.01	0.00599919546689532\\
19.01	0.0059991946414727\\
20.01	0.00599919379879872\\
21.01	0.00599919293851189\\
22.01	0.00599919206024303\\
23.01	0.00599919116361524\\
24.01	0.00599919024824362\\
25.01	0.00599918931373534\\
26.01	0.00599918835968907\\
27.01	0.00599918738569524\\
28.01	0.00599918639133552\\
29.01	0.00599918537618293\\
30.01	0.00599918433980144\\
31.01	0.00599918328174592\\
32.01	0.00599918220156164\\
33.01	0.00599918109878443\\
34.01	0.00599917997294056\\
35.01	0.00599917882354615\\
36.01	0.0059991776501071\\
37.01	0.00599917645211886\\
38.01	0.0059991752290666\\
39.01	0.00599917398042425\\
40.01	0.00599917270565479\\
41.01	0.00599917140421023\\
42.01	0.00599917007553037\\
43.01	0.00599916871904389\\
44.01	0.00599916733416706\\
45.01	0.00599916592030388\\
46.01	0.00599916447684598\\
47.01	0.00599916300317207\\
48.01	0.00599916149864776\\
49.01	0.00599915996262538\\
50.01	0.00599915839444355\\
51.01	0.00599915679342695\\
52.01	0.00599915515888597\\
53.01	0.00599915349011664\\
54.01	0.00599915178640011\\
55.01	0.00599915004700243\\
56.01	0.00599914827117402\\
57.01	0.00599914645814955\\
58.01	0.00599914460714769\\
59.01	0.0059991427173706\\
60.01	0.00599914078800345\\
61.01	0.00599913881821453\\
62.01	0.00599913680715434\\
63.01	0.00599913475395556\\
64.01	0.00599913265773274\\
65.01	0.00599913051758159\\
66.01	0.00599912833257874\\
67.01	0.00599912610178138\\
68.01	0.00599912382422704\\
69.01	0.00599912149893269\\
70.01	0.00599911912489449\\
71.01	0.00599911670108767\\
72.01	0.00599911422646573\\
73.01	0.00599911169996021\\
74.01	0.00599910912048001\\
75.01	0.00599910648691098\\
76.01	0.00599910379811559\\
77.01	0.00599910105293222\\
78.01	0.00599909825017465\\
79.01	0.00599909538863188\\
80.01	0.00599909246706711\\
81.01	0.00599908948421765\\
82.01	0.00599908643879377\\
83.01	0.00599908332947907\\
84.01	0.00599908015492903\\
85.01	0.00599907691377092\\
86.01	0.00599907360460263\\
87.01	0.00599907022599305\\
88.01	0.00599906677648032\\
89.01	0.00599906325457193\\
90.01	0.00599905965874401\\
91.01	0.0059990559874405\\
92.01	0.00599905223907211\\
93.01	0.00599904841201628\\
94.01	0.0059990445046161\\
95.01	0.00599904051517964\\
96.01	0.00599903644197924\\
97.01	0.00599903228325092\\
98.01	0.00599902803719312\\
99.01	0.00599902370196647\\
100.01	0.00599901927569264\\
101.01	0.00599901475645361\\
102.01	0.00599901014229074\\
103.01	0.00599900543120417\\
104.01	0.00599900062115153\\
105.01	0.00599899571004749\\
106.01	0.00599899069576272\\
107.01	0.00599898557612253\\
108.01	0.00599898034890668\\
109.01	0.00599897501184758\\
110.01	0.00599896956262994\\
111.01	0.0059989639988893\\
112.01	0.00599895831821158\\
113.01	0.00599895251813124\\
114.01	0.00599894659613085\\
115.01	0.0059989405496396\\
116.01	0.00599893437603241\\
117.01	0.00599892807262836\\
118.01	0.00599892163669021\\
119.01	0.0059989150654227\\
120.01	0.00599890835597124\\
121.01	0.00599890150542094\\
122.01	0.00599889451079527\\
123.01	0.00599888736905457\\
124.01	0.00599888007709506\\
125.01	0.005998872631747\\
126.01	0.0059988650297737\\
127.01	0.00599885726786993\\
128.01	0.00599884934266052\\
129.01	0.00599884125069864\\
130.01	0.00599883298846468\\
131.01	0.0059988245523644\\
132.01	0.00599881593872726\\
133.01	0.00599880714380532\\
134.01	0.00599879816377095\\
135.01	0.00599878899471553\\
136.01	0.00599877963264779\\
137.01	0.00599877007349139\\
138.01	0.00599876031308411\\
139.01	0.00599875034717532\\
140.01	0.00599874017142437\\
141.01	0.00599872978139846\\
142.01	0.00599871917257084\\
143.01	0.00599870834031886\\
144.01	0.00599869727992172\\
145.01	0.00599868598655861\\
146.01	0.00599867445530645\\
147.01	0.0059986626811376\\
148.01	0.00599865065891798\\
149.01	0.00599863838340458\\
150.01	0.00599862584924283\\
151.01	0.00599861305096492\\
152.01	0.00599859998298662\\
153.01	0.00599858663960552\\
154.01	0.00599857301499787\\
155.01	0.00599855910321631\\
156.01	0.00599854489818736\\
157.01	0.00599853039370847\\
158.01	0.00599851558344523\\
159.01	0.00599850046092872\\
160.01	0.00599848501955267\\
161.01	0.00599846925257051\\
162.01	0.00599845315309188\\
163.01	0.00599843671408033\\
164.01	0.00599841992834965\\
165.01	0.00599840278856103\\
166.01	0.00599838528721938\\
167.01	0.00599836741667027\\
168.01	0.00599834916909647\\
169.01	0.00599833053651428\\
170.01	0.00599831151077045\\
171.01	0.00599829208353769\\
172.01	0.00599827224631181\\
173.01	0.00599825199040755\\
174.01	0.00599823130695431\\
175.01	0.00599821018689306\\
176.01	0.00599818862097142\\
177.01	0.00599816659974013\\
178.01	0.00599814411354825\\
179.01	0.00599812115253914\\
180.01	0.00599809770664599\\
181.01	0.00599807376558748\\
182.01	0.0059980493188628\\
183.01	0.00599802435574706\\
184.01	0.00599799886528652\\
185.01	0.00599797283629355\\
186.01	0.00599794625734166\\
187.01	0.00599791911676021\\
188.01	0.00599789140262936\\
189.01	0.00599786310277441\\
190.01	0.00599783420476035\\
191.01	0.00599780469588631\\
192.01	0.00599777456317977\\
193.01	0.00599774379339065\\
194.01	0.00599771237298514\\
195.01	0.0059976802881398\\
196.01	0.00599764752473506\\
197.01	0.00599761406834871\\
198.01	0.00599757990424974\\
199.01	0.00599754501739094\\
200.01	0.00599750939240248\\
201.01	0.0059974730135848\\
202.01	0.00599743586490139\\
203.01	0.00599739792997112\\
204.01	0.00599735919206111\\
205.01	0.00599731963407892\\
206.01	0.00599727923856453\\
207.01	0.00599723798768228\\
208.01	0.00599719586321269\\
209.01	0.00599715284654393\\
210.01	0.00599710891866346\\
211.01	0.00599706406014889\\
212.01	0.00599701825115882\\
213.01	0.00599697147142395\\
214.01	0.00599692370023746\\
215.01	0.00599687491644507\\
216.01	0.0059968250984357\\
217.01	0.00599677422413087\\
218.01	0.00599672227097416\\
219.01	0.00599666921592113\\
220.01	0.00599661503542779\\
221.01	0.00599655970544041\\
222.01	0.00599650320138266\\
223.01	0.00599644549814529\\
224.01	0.00599638657007356\\
225.01	0.00599632639095526\\
226.01	0.00599626493400755\\
227.01	0.00599620217186472\\
228.01	0.00599613807656508\\
229.01	0.00599607261953705\\
230.01	0.00599600577158559\\
231.01	0.00599593750287803\\
232.01	0.00599586778292976\\
233.01	0.0059957965805889\\
234.01	0.00599572386402173\\
235.01	0.00599564960069651\\
236.01	0.0059955737573679\\
237.01	0.00599549630006043\\
238.01	0.00599541719405183\\
239.01	0.00599533640385553\\
240.01	0.00599525389320326\\
241.01	0.00599516962502701\\
242.01	0.00599508356144032\\
243.01	0.00599499566371927\\
244.01	0.00599490589228303\\
245.01	0.00599481420667374\\
246.01	0.00599472056553601\\
247.01	0.00599462492659598\\
248.01	0.00599452724663931\\
249.01	0.00599442748148934\\
250.01	0.0059943255859844\\
251.01	0.00599422151395413\\
252.01	0.00599411521819519\\
253.01	0.00599400665044746\\
254.01	0.00599389576136785\\
255.01	0.00599378250050464\\
256.01	0.0059936668162711\\
257.01	0.00599354865591747\\
258.01	0.00599342796550285\\
259.01	0.00599330468986659\\
260.01	0.00599317877259795\\
261.01	0.00599305015600591\\
262.01	0.00599291878108734\\
263.01	0.00599278458749503\\
264.01	0.00599264751350389\\
265.01	0.00599250749597694\\
266.01	0.00599236447032994\\
267.01	0.00599221837049516\\
268.01	0.00599206912888394\\
269.01	0.00599191667634849\\
270.01	0.00599176094214177\\
271.01	0.00599160185387725\\
272.01	0.00599143933748678\\
273.01	0.00599127331717706\\
274.01	0.00599110371538538\\
275.01	0.00599093045273321\\
276.01	0.0059907534479793\\
277.01	0.0059905726179706\\
278.01	0.0059903878775917\\
279.01	0.00599019913971328\\
280.01	0.00599000631513786\\
281.01	0.00598980931254505\\
282.01	0.00598960803843404\\
283.01	0.00598940239706462\\
284.01	0.00598919229039605\\
285.01	0.00598897761802438\\
286.01	0.00598875827711743\\
287.01	0.005988534162347\\
288.01	0.00598830516581982\\
289.01	0.0059880711770051\\
290.01	0.00598783208266051\\
291.01	0.00598758776675481\\
292.01	0.00598733811038847\\
293.01	0.00598708299171064\\
294.01	0.00598682228583366\\
295.01	0.00598655586474466\\
296.01	0.00598628359721345\\
297.01	0.0059860053486971\\
298.01	0.00598572098124139\\
299.01	0.00598543035337806\\
300.01	0.0059851333200183\\
301.01	0.00598482973234218\\
302.01	0.0059845194376842\\
303.01	0.00598420227941313\\
304.01	0.00598387809680867\\
305.01	0.00598354672493207\\
306.01	0.00598320799449195\\
307.01	0.00598286173170432\\
308.01	0.00598250775814763\\
309.01	0.00598214589061027\\
310.01	0.00598177594093358\\
311.01	0.00598139771584608\\
312.01	0.0059810110167925\\
313.01	0.00598061563975356\\
314.01	0.00598021137505921\\
315.01	0.00597979800719296\\
316.01	0.00597937531458689\\
317.01	0.00597894306940824\\
318.01	0.00597850103733571\\
319.01	0.00597804897732494\\
320.01	0.00597758664136327\\
321.01	0.00597711377421333\\
322.01	0.00597663011314294\\
323.01	0.00597613538764335\\
324.01	0.00597562931913234\\
325.01	0.005975111620643\\
326.01	0.00597458199649726\\
327.01	0.00597404014196246\\
328.01	0.00597348574288994\\
329.01	0.00597291847533586\\
330.01	0.00597233800516154\\
331.01	0.00597174398761329\\
332.01	0.0059711360668798\\
333.01	0.0059705138756261\\
334.01	0.00596987703450306\\
335.01	0.00596922515162998\\
336.01	0.00596855782205012\\
337.01	0.00596787462715672\\
338.01	0.00596717513408934\\
339.01	0.00596645889509628\\
340.01	0.00596572544686534\\
341.01	0.00596497430981876\\
342.01	0.00596420498737254\\
343.01	0.00596341696515803\\
344.01	0.00596260971020565\\
345.01	0.00596178267009125\\
346.01	0.00596093527204223\\
347.01	0.00596006692200784\\
348.01	0.00595917700369159\\
349.01	0.00595826487755111\\
350.01	0.00595732987976751\\
351.01	0.00595637132119052\\
352.01	0.00595538848626687\\
353.01	0.0059543806319626\\
354.01	0.00595334698669194\\
355.01	0.00595228674927193\\
356.01	0.00595119908792382\\
357.01	0.00595008313935151\\
358.01	0.00594893800793244\\
359.01	0.00594776276506671\\
360.01	0.00594655644874136\\
361.01	0.005945318063378\\
362.01	0.00594404658004975\\
363.01	0.00594274093717175\\
364.01	0.005941400041788\\
365.01	0.00594002277160655\\
366.01	0.00593860797795742\\
367.01	0.00593715448988461\\
368.01	0.00593566111960748\\
369.01	0.00593412666962958\\
370.01	0.00593254994179276\\
371.01	0.00593092974860451\\
372.01	0.00592926492716847\\
373.01	0.00592755435603001\\
374.01	0.00592579697518265\\
375.01	0.00592399180934374\\
376.01	0.00592213799436153\\
377.01	0.00592023480619791\\
378.01	0.00591828169126427\\
379.01	0.00591627829585061\\
380.01	0.00591422449080454\\
381.01	0.00591212038524867\\
382.01	0.00590996631961475\\
383.01	0.00590776282312728\\
384.01	0.00590551051338558\\
385.01	0.0059032098963159\\
386.01	0.00590086073621105\\
387.01	0.00589846203139617\\
388.01	0.00589601270542194\\
389.01	0.00589351165590037\\
390.01	0.00589095775369843\\
391.01	0.00588834984209453\\
392.01	0.00588568673589602\\
393.01	0.0058829672205153\\
394.01	0.0058801900510013\\
395.01	0.0058773539510239\\
396.01	0.00587445761180711\\
397.01	0.00587149969100976\\
398.01	0.00586847881154749\\
399.01	0.00586539356035274\\
400.01	0.00586224248707006\\
401.01	0.00585902410267831\\
402.01	0.00585573687803897\\
403.01	0.00585237924236085\\
404.01	0.00584894958157736\\
405.01	0.0058454462366283\\
406.01	0.00584186750163974\\
407.01	0.0058382116219927\\
408.01	0.00583447679227158\\
409.01	0.00583066115408409\\
410.01	0.00582676279373878\\
411.01	0.00582277973977041\\
412.01	0.00581870996029901\\
413.01	0.00581455136020655\\
414.01	0.00581030177811717\\
415.01	0.00580595898315922\\
416.01	0.00580152067149174\\
417.01	0.00579698446257012\\
418.01	0.0057923478951265\\
419.01	0.00578760842283554\\
420.01	0.00578276340963383\\
421.01	0.00577781012465565\\
422.01	0.00577274573674575\\
423.01	0.00576756730850078\\
424.01	0.00576227178978976\\
425.01	0.00575685601069225\\
426.01	0.00575131667378939\\
427.01	0.00574565034572987\\
428.01	0.00573985344798673\\
429.01	0.00573392224670478\\
430.01	0.00572785284152894\\
431.01	0.00572164115328557\\
432.01	0.00571528291037148\\
433.01	0.00570877363368667\\
434.01	0.00570210861992245\\
435.01	0.00569528292299155\\
436.01	0.00568829133335686\\
437.01	0.00568112835498331\\
438.01	0.00567378817960297\\
439.01	0.00566626465793952\\
440.01	0.00565855126750225\\
441.01	0.00565064107650711\\
442.01	0.00564252670344594\\
443.01	0.00563420027177326\\
444.01	0.00562565335914622\\
445.01	0.0056168769406252\\
446.01	0.00560786132523378\\
447.01	0.00559859608530133\\
448.01	0.00558906997808641\\
449.01	0.00557927085932592\\
450.01	0.00556918558861603\\
451.01	0.00555879992695325\\
452.01	0.00554809842741975\\
453.01	0.00553706432098528\\
454.01	0.00552567940086576\\
455.01	0.00551392391100613\\
456.01	0.00550177644733057\\
457.01	0.00548921388478204\\
458.01	0.00547621134939325\\
459.01	0.00546274226339325\\
460.01	0.005448778503656\\
461.01	0.00543429073100853\\
462.01	0.00541924897191974\\
463.01	0.00540362356749257\\
464.01	0.00538738665107438\\
465.01	0.00537051438013312\\
466.01	0.00535299023716055\\
467.01	0.00533480983766913\\
468.01	0.00531598810909099\\
469.01	0.00529667141390681\\
470.01	0.00527695873756643\\
471.01	0.00525684389729683\\
472.01	0.00523632107135242\\
473.01	0.00521538490662479\\
474.01	0.00519403064763003\\
475.01	0.00517225429072692\\
476.01	0.00515005276802421\\
477.01	0.00512742416620251\\
478.01	0.00510436798679032\\
479.01	0.00508088545394805\\
480.01	0.00505697984979161\\
481.01	0.00503265688938281\\
482.01	0.00500792516164674\\
483.01	0.00498279663299035\\
484.01	0.00495728721291673\\
485.01	0.00493141737059719\\
486.01	0.00490521278547553\\
487.01	0.00487870500103308\\
488.01	0.00485193202893283\\
489.01	0.00482493881792928\\
490.01	0.00479777745330554\\
491.01	0.00477050688384594\\
492.01	0.00474319194907003\\
493.01	0.00471590149384669\\
494.01	0.00468870505444633\\
495.01	0.00466166736378074\\
496.01	0.00463483967115599\\
497.01	0.00460824646084977\\
498.01	0.00458177047475227\\
499.01	0.00455510963545843\\
500.01	0.00452827058744963\\
501.01	0.00450128226744785\\
502.01	0.00447417550499826\\
503.01	0.00444698268078264\\
504.01	0.00441973723465563\\
505.01	0.0043924729924855\\
506.01	0.00436522327968771\\
507.01	0.00433801979095227\\
508.01	0.00431089119216303\\
509.01	0.00428386144489852\\
510.01	0.00425694787065462\\
511.01	0.00423015901772533\\
512.01	0.00420349246824837\\
513.01	0.00417693284040673\\
514.01	0.00415045042149732\\
515.01	0.00412400114070438\\
516.01	0.00409752899750191\\
517.01	0.0040709740866263\\
518.01	0.00404431015725351\\
519.01	0.00401754020522142\\
520.01	0.00399066522984844\\
521.01	0.00396368251504052\\
522.01	0.00393658516421696\\
523.01	0.00390936173362636\\
524.01	0.00388199602961315\\
525.01	0.00385446715333914\\
526.01	0.00382674988269075\\
527.01	0.00379881546461456\\
528.01	0.00377063287158874\\
529.01	0.00374217052785795\\
530.01	0.00371339840652742\\
531.01	0.00368429020704527\\
532.01	0.00365482488644382\\
533.01	0.00362498452441107\\
534.01	0.00359475008237013\\
535.01	0.0035641008951345\\
536.01	0.00353301497943292\\
537.01	0.00350146942467424\\
538.01	0.003469440847353\\
539.01	0.00343690587620745\\
540.01	0.00340384161757189\\
541.01	0.00337022603239339\\
542.01	0.0033360381432204\\
543.01	0.00330125799112801\\
544.01	0.00326586629706542\\
545.01	0.00322984392536571\\
546.01	0.00319317162090588\\
547.01	0.00315583011753067\\
548.01	0.00311780028889496\\
549.01	0.00307906327340484\\
550.01	0.00303960056088456\\
551.01	0.00299939403210085\\
552.01	0.00295842594890036\\
553.01	0.00291667890320695\\
554.01	0.00287413574665602\\
555.01	0.00283077953587981\\
556.01	0.00278659352953516\\
557.01	0.00274156123091943\\
558.01	0.00269566643484715\\
559.01	0.0026488932657813\\
560.01	0.00260122620891889\\
561.01	0.00255265013827661\\
562.01	0.00250315034739825\\
563.01	0.00245271258897792\\
564.01	0.00240132312857586\\
565.01	0.00234896881412779\\
566.01	0.00229563715800819\\
567.01	0.00224131642709743\\
568.01	0.002185995740384\\
569.01	0.00212966517598244\\
570.01	0.0020723158896105\\
571.01	0.0020139402462388\\
572.01	0.00195453196598035\\
573.01	0.00189408628446933\\
574.01	0.00183260012729144\\
575.01	0.00177007229786928\\
576.01	0.00170650367858114\\
577.01	0.00164189744502896\\
578.01	0.00157625929296629\\
579.01	0.00150959767673313\\
580.01	0.0014419240572397\\
581.01	0.00137325315661659\\
582.01	0.00130360321561083\\
583.01	0.00123299624859551\\
584.01	0.00116145828952293\\
585.01	0.0010890196201021\\
586.01	0.00101571496882195\\
587.01	0.000941583666090056\\
588.01	0.000866669736561883\\
589.01	0.000791021904488039\\
590.01	0.000714693481339309\\
591.01	0.00063774209673771\\
592.01	0.000560229223416786\\
593.01	0.000482219434073887\\
594.01	0.000403779311999168\\
595.01	0.000324975917580054\\
596.01	0.000245874688332452\\
597.01	0.000166536617556516\\
598.01	9.13379815193054e-05\\
599.01	2.91271958372426e-05\\
599.02	2.86192783627518e-05\\
599.03	2.81144237420042e-05\\
599.04	2.76126617978888e-05\\
599.05	2.71140226472798e-05\\
599.06	2.66185367039599e-05\\
599.07	2.61262346815533e-05\\
599.08	2.56371475965082e-05\\
599.09	2.51513067710835e-05\\
599.1	2.46687438363886e-05\\
599.11	2.41894907354479e-05\\
599.12	2.3713579726279e-05\\
599.13	2.32410433850267e-05\\
599.14	2.27719146091033e-05\\
599.15	2.23062266203888e-05\\
599.16	2.18440129684267e-05\\
599.17	2.13853075336917e-05\\
599.18	2.09301445308497e-05\\
599.19	2.04785585120829e-05\\
599.2	2.00305843704295e-05\\
599.21	1.95862573431627e-05\\
599.22	1.91456130152028e-05\\
599.23	1.87086873225609e-05\\
599.24	1.82755165558171e-05\\
599.25	1.7846137363638e-05\\
599.26	1.74205874164668e-05\\
599.27	1.69989072017485e-05\\
599.28	1.6581137610194e-05\\
599.29	1.61673199397579e-05\\
599.3	1.57574958996824e-05\\
599.31	1.53517076145714e-05\\
599.32	1.49499976284904e-05\\
599.33	1.45524089091333e-05\\
599.34	1.41589848520092e-05\\
599.35	1.37697692846831e-05\\
599.36	1.33848064710462e-05\\
599.37	1.30041411156422e-05\\
599.38	1.26278183680325e-05\\
599.39	1.22558838271929e-05\\
599.4	1.18883835459674e-05\\
599.41	1.15253640355657e-05\\
599.42	1.11668722700964e-05\\
599.43	1.08129556911519e-05\\
599.44	1.04636622124312e-05\\
599.45	1.01190402244222e-05\\
599.46	9.77913859911625e-06\\
599.47	9.44400669477576e-06\\
599.48	9.11369436074581e-06\\
599.49	8.7882519423238e-06\\
599.5	8.46773028565471e-06\\
599.51	8.15218074270464e-06\\
599.52	7.84165517625675e-06\\
599.53	7.5362059649732e-06\\
599.54	7.23588600849874e-06\\
599.55	6.94074873261973e-06\\
599.56	6.65084809447353e-06\\
599.57	6.366238587803e-06\\
599.58	6.08697524826819e-06\\
599.59	5.81311365882402e-06\\
599.6	5.54470995511175e-06\\
599.61	5.28182083095637e-06\\
599.62	5.02450354387431e-06\\
599.63	4.77281592065407e-06\\
599.64	4.52681636300446e-06\\
599.65	4.28656385322197e-06\\
599.66	4.05211795996042e-06\\
599.67	3.82353884401457e-06\\
599.68	3.60088726420078e-06\\
599.69	3.38422458325341e-06\\
599.7	3.17361277382168e-06\\
599.71	2.96911442449269e-06\\
599.72	2.77079274589413e-06\\
599.73	2.5787115768474e-06\\
599.74	2.39293539058306e-06\\
599.75	2.21352930102926e-06\\
599.76	2.04055906913823e-06\\
599.77	1.87409110929959e-06\\
599.78	1.71419249580043e-06\\
599.79	1.56093096936004e-06\\
599.8	1.41437494372877e-06\\
599.81	1.27459351233553e-06\\
599.82	1.14165645502366e-06\\
599.83	1.01563424484766e-06\\
599.84	8.96598054920053e-07\\
599.85	7.84619765350353e-07\\
599.86	6.79771970232487e-07\\
599.87	5.82127984717282e-07\\
599.88	4.91761852145639e-07\\
599.89	4.08748351251112e-07\\
599.9	3.33163003438802e-07\\
599.91	2.65082080131915e-07\\
599.92	2.04582610201579e-07\\
599.93	1.51742387450443e-07\\
599.94	1.06639978191686e-07\\
599.95	6.93547288817958e-08\\
599.96	3.99667738487652e-08\\
599.97	1.85570430896731e-08\\
599.98	5.20727013245126e-09\\
599.99	0\\
600	0\\
};
\addplot [color=mycolor9,solid,forget plot]
  table[row sep=crcr]{%
0.01	0.00613141925685511\\
1.01	0.00613141877852663\\
2.01	0.00613141829018322\\
3.01	0.00613141779161435\\
4.01	0.00613141728260539\\
5.01	0.00613141676293696\\
6.01	0.00613141623238522\\
7.01	0.0061314156907213\\
8.01	0.00613141513771185\\
9.01	0.00613141457311849\\
10.01	0.00613141399669758\\
11.01	0.00613141340820058\\
12.01	0.00613141280737362\\
13.01	0.00613141219395754\\
14.01	0.00613141156768764\\
15.01	0.00613141092829383\\
16.01	0.00613141027550002\\
17.01	0.00613140960902458\\
18.01	0.00613140892857955\\
19.01	0.00613140823387131\\
20.01	0.00613140752459977\\
21.01	0.00613140680045859\\
22.01	0.00613140606113511\\
23.01	0.0061314053063097\\
24.01	0.00613140453565651\\
25.01	0.00613140374884231\\
26.01	0.00613140294552697\\
27.01	0.00613140212536338\\
28.01	0.00613140128799692\\
29.01	0.00613140043306532\\
30.01	0.00613139956019907\\
31.01	0.00613139866902044\\
32.01	0.00613139775914414\\
33.01	0.00613139683017632\\
34.01	0.00613139588171504\\
35.01	0.00613139491334979\\
36.01	0.00613139392466131\\
37.01	0.00613139291522161\\
38.01	0.0061313918845936\\
39.01	0.00613139083233102\\
40.01	0.00613138975797804\\
41.01	0.00613138866106922\\
42.01	0.0061313875411294\\
43.01	0.00613138639767318\\
44.01	0.00613138523020505\\
45.01	0.00613138403821884\\
46.01	0.00613138282119798\\
47.01	0.00613138157861463\\
48.01	0.00613138030992992\\
49.01	0.00613137901459358\\
50.01	0.00613137769204372\\
51.01	0.00613137634170644\\
52.01	0.00613137496299578\\
53.01	0.00613137355531355\\
54.01	0.00613137211804844\\
55.01	0.00613137065057646\\
56.01	0.00613136915226047\\
57.01	0.00613136762244989\\
58.01	0.00613136606048014\\
59.01	0.00613136446567259\\
60.01	0.00613136283733439\\
61.01	0.00613136117475776\\
62.01	0.00613135947722029\\
63.01	0.00613135774398396\\
64.01	0.0061313559742952\\
65.01	0.00613135416738438\\
66.01	0.00613135232246574\\
67.01	0.00613135043873665\\
68.01	0.00613134851537762\\
69.01	0.00613134655155164\\
70.01	0.00613134454640412\\
71.01	0.00613134249906224\\
72.01	0.00613134040863477\\
73.01	0.00613133827421133\\
74.01	0.00613133609486241\\
75.01	0.00613133386963871\\
76.01	0.00613133159757098\\
77.01	0.00613132927766903\\
78.01	0.0061313269089219\\
79.01	0.00613132449029719\\
80.01	0.0061313220207405\\
81.01	0.00613131949917508\\
82.01	0.00613131692450155\\
83.01	0.00613131429559655\\
84.01	0.00613131161131343\\
85.01	0.00613130887048089\\
86.01	0.00613130607190306\\
87.01	0.00613130321435836\\
88.01	0.00613130029659934\\
89.01	0.00613129731735213\\
90.01	0.00613129427531566\\
91.01	0.00613129116916107\\
92.01	0.00613128799753172\\
93.01	0.00613128475904187\\
94.01	0.00613128145227612\\
95.01	0.00613127807578923\\
96.01	0.00613127462810521\\
97.01	0.00613127110771646\\
98.01	0.00613126751308354\\
99.01	0.00613126384263425\\
100.01	0.00613126009476282\\
101.01	0.00613125626782938\\
102.01	0.00613125236015923\\
103.01	0.0061312483700418\\
104.01	0.0061312442957304\\
105.01	0.00613124013544095\\
106.01	0.00613123588735144\\
107.01	0.00613123154960105\\
108.01	0.00613122712028939\\
109.01	0.00613122259747547\\
110.01	0.00613121797917708\\
111.01	0.00613121326336975\\
112.01	0.0061312084479858\\
113.01	0.00613120353091368\\
114.01	0.00613119850999654\\
115.01	0.00613119338303186\\
116.01	0.00613118814776975\\
117.01	0.00613118280191292\\
118.01	0.00613117734311466\\
119.01	0.00613117176897832\\
120.01	0.00613116607705617\\
121.01	0.00613116026484815\\
122.01	0.00613115432980107\\
123.01	0.00613114826930706\\
124.01	0.00613114208070265\\
125.01	0.00613113576126771\\
126.01	0.00613112930822355\\
127.01	0.0061311227187327\\
128.01	0.00613111598989675\\
129.01	0.00613110911875544\\
130.01	0.0061311021022852\\
131.01	0.00613109493739823\\
132.01	0.00613108762094029\\
133.01	0.00613108014968999\\
134.01	0.00613107252035707\\
135.01	0.00613106472958074\\
136.01	0.00613105677392838\\
137.01	0.00613104864989433\\
138.01	0.0061310403538974\\
139.01	0.00613103188228016\\
140.01	0.00613102323130672\\
141.01	0.00613101439716129\\
142.01	0.00613100537594639\\
143.01	0.00613099616368126\\
144.01	0.00613098675629949\\
145.01	0.00613097714964797\\
146.01	0.00613096733948448\\
147.01	0.00613095732147582\\
148.01	0.00613094709119606\\
149.01	0.00613093664412417\\
150.01	0.00613092597564249\\
151.01	0.00613091508103397\\
152.01	0.00613090395548086\\
153.01	0.0061308925940616\\
154.01	0.00613088099174931\\
155.01	0.00613086914340898\\
156.01	0.00613085704379548\\
157.01	0.00613084468755092\\
158.01	0.00613083206920266\\
159.01	0.00613081918315992\\
160.01	0.00613080602371232\\
161.01	0.00613079258502628\\
162.01	0.00613077886114318\\
163.01	0.0061307648459762\\
164.01	0.00613075053330715\\
165.01	0.00613073591678446\\
166.01	0.00613072098991972\\
167.01	0.0061307057460848\\
168.01	0.0061306901785089\\
169.01	0.00613067428027509\\
170.01	0.00613065804431758\\
171.01	0.00613064146341832\\
172.01	0.00613062453020359\\
173.01	0.00613060723714046\\
174.01	0.00613058957653357\\
175.01	0.00613057154052153\\
176.01	0.00613055312107313\\
177.01	0.00613053430998347\\
178.01	0.00613051509887085\\
179.01	0.00613049547917213\\
180.01	0.00613047544213893\\
181.01	0.00613045497883382\\
182.01	0.00613043408012581\\
183.01	0.00613041273668644\\
184.01	0.00613039093898515\\
185.01	0.00613036867728486\\
186.01	0.00613034594163763\\
187.01	0.00613032272187963\\
188.01	0.00613029900762659\\
189.01	0.00613027478826909\\
190.01	0.00613025005296706\\
191.01	0.00613022479064527\\
192.01	0.00613019898998776\\
193.01	0.00613017263943257\\
194.01	0.00613014572716603\\
195.01	0.00613011824111767\\
196.01	0.00613009016895396\\
197.01	0.00613006149807331\\
198.01	0.00613003221559867\\
199.01	0.0061300023083726\\
200.01	0.00612997176295064\\
201.01	0.00612994056559471\\
202.01	0.00612990870226678\\
203.01	0.00612987615862199\\
204.01	0.00612984292000166\\
205.01	0.00612980897142641\\
206.01	0.00612977429758902\\
207.01	0.00612973888284686\\
208.01	0.00612970271121422\\
209.01	0.00612966576635505\\
210.01	0.00612962803157442\\
211.01	0.00612958948981084\\
212.01	0.00612955012362771\\
213.01	0.00612950991520501\\
214.01	0.00612946884633029\\
215.01	0.0061294268983902\\
216.01	0.0061293840523607\\
217.01	0.0061293402887983\\
218.01	0.00612929558783022\\
219.01	0.00612924992914447\\
220.01	0.00612920329197995\\
221.01	0.00612915565511587\\
222.01	0.00612910699686189\\
223.01	0.0061290572950461\\
224.01	0.00612900652700514\\
225.01	0.00612895466957207\\
226.01	0.00612890169906489\\
227.01	0.00612884759127511\\
228.01	0.00612879232145452\\
229.01	0.00612873586430366\\
230.01	0.00612867819395805\\
231.01	0.00612861928397599\\
232.01	0.00612855910732409\\
233.01	0.00612849763636423\\
234.01	0.00612843484283882\\
235.01	0.00612837069785657\\
236.01	0.00612830517187709\\
237.01	0.00612823823469604\\
238.01	0.00612816985542942\\
239.01	0.00612810000249731\\
240.01	0.00612802864360717\\
241.01	0.00612795574573717\\
242.01	0.00612788127511858\\
243.01	0.00612780519721822\\
244.01	0.00612772747671982\\
245.01	0.00612764807750554\\
246.01	0.00612756696263651\\
247.01	0.00612748409433289\\
248.01	0.00612739943395385\\
249.01	0.00612731294197671\\
250.01	0.00612722457797518\\
251.01	0.00612713430059778\\
252.01	0.00612704206754512\\
253.01	0.00612694783554658\\
254.01	0.00612685156033707\\
255.01	0.00612675319663184\\
256.01	0.00612665269810169\\
257.01	0.00612655001734743\\
258.01	0.00612644510587337\\
259.01	0.00612633791405971\\
260.01	0.0061262283911347\\
261.01	0.0061261164851464\\
262.01	0.00612600214293276\\
263.01	0.00612588531009113\\
264.01	0.00612576593094754\\
265.01	0.00612564394852431\\
266.01	0.00612551930450735\\
267.01	0.00612539193921218\\
268.01	0.00612526179154911\\
269.01	0.00612512879898738\\
270.01	0.00612499289751862\\
271.01	0.00612485402161885\\
272.01	0.00612471210420932\\
273.01	0.00612456707661684\\
274.01	0.00612441886853247\\
275.01	0.00612426740796932\\
276.01	0.00612411262121866\\
277.01	0.00612395443280529\\
278.01	0.00612379276544136\\
279.01	0.00612362753997891\\
280.01	0.00612345867536119\\
281.01	0.00612328608857209\\
282.01	0.00612310969458471\\
283.01	0.006122929406308\\
284.01	0.00612274513453188\\
285.01	0.00612255678787113\\
286.01	0.00612236427270669\\
287.01	0.00612216749312673\\
288.01	0.00612196635086441\\
289.01	0.00612176074523501\\
290.01	0.00612155057307034\\
291.01	0.00612133572865188\\
292.01	0.00612111610364109\\
293.01	0.00612089158700883\\
294.01	0.00612066206496184\\
295.01	0.00612042742086683\\
296.01	0.00612018753517314\\
297.01	0.00611994228533281\\
298.01	0.00611969154571745\\
299.01	0.00611943518753377\\
300.01	0.00611917307873641\\
301.01	0.00611890508393728\\
302.01	0.00611863106431315\\
303.01	0.00611835087751008\\
304.01	0.00611806437754568\\
305.01	0.00611777141470708\\
306.01	0.00611747183544712\\
307.01	0.00611716548227741\\
308.01	0.00611685219365762\\
309.01	0.00611653180388203\\
310.01	0.00611620414296253\\
311.01	0.00611586903650876\\
312.01	0.00611552630560402\\
313.01	0.00611517576667837\\
314.01	0.00611481723137799\\
315.01	0.00611445050643047\\
316.01	0.00611407539350679\\
317.01	0.00611369168907982\\
318.01	0.00611329918427828\\
319.01	0.00611289766473754\\
320.01	0.0061124869104461\\
321.01	0.00611206669558833\\
322.01	0.00611163678838362\\
323.01	0.00611119695092103\\
324.01	0.00611074693899049\\
325.01	0.00611028650191061\\
326.01	0.00610981538235209\\
327.01	0.00610933331615735\\
328.01	0.00610884003215808\\
329.01	0.00610833525198795\\
330.01	0.00610781868989326\\
331.01	0.00610729005254041\\
332.01	0.00610674903882134\\
333.01	0.00610619533965678\\
334.01	0.00610562863779777\\
335.01	0.00610504860762697\\
336.01	0.00610445491495908\\
337.01	0.00610384721684288\\
338.01	0.0061032251613627\\
339.01	0.00610258838744573\\
340.01	0.00610193652467055\\
341.01	0.00610126919308258\\
342.01	0.00610058600301584\\
343.01	0.00609988655492459\\
344.01	0.00609917043922596\\
345.01	0.00609843723615625\\
346.01	0.00609768651564444\\
347.01	0.0060969178372051\\
348.01	0.00609613074985648\\
349.01	0.00609532479206568\\
350.01	0.0060944994917266\\
351.01	0.00609365436617652\\
352.01	0.00609278892225648\\
353.01	0.00609190265642249\\
354.01	0.00609099505491372\\
355.01	0.00609006559398646\\
356.01	0.00608911374022216\\
357.01	0.0060881389509176\\
358.01	0.00608714067456685\\
359.01	0.00608611835144441\\
360.01	0.00608507141429824\\
361.01	0.00608399928916177\\
362.01	0.00608290139629182\\
363.01	0.00608177715123693\\
364.01	0.00608062596603969\\
365.01	0.00607944725056773\\
366.01	0.0060782404139671\\
367.01	0.00607700486621656\\
368.01	0.00607574001975567\\
369.01	0.00607444529113813\\
370.01	0.00607312010264809\\
371.01	0.00607176388378742\\
372.01	0.00607037607251521\\
373.01	0.00606895611608468\\
374.01	0.00606750347128157\\
375.01	0.00606601760382692\\
376.01	0.00606449798666036\\
377.01	0.00606294409678792\\
378.01	0.00606135541036019\\
379.01	0.00605973139566939\\
380.01	0.00605807150384671\\
381.01	0.00605637515725608\\
382.01	0.00605464173598649\\
383.01	0.00605287056357264\\
384.01	0.00605106089427924\\
385.01	0.0060492119062795\\
386.01	0.00604732271166291\\
387.01	0.00604539238725572\\
388.01	0.00604341998557221\\
389.01	0.00604140453428834\\
390.01	0.00603934503542485\\
391.01	0.00603724046449459\\
392.01	0.00603508976961249\\
393.01	0.0060328918705657\\
394.01	0.00603064565784248\\
395.01	0.00602834999161672\\
396.01	0.00602600370068583\\
397.01	0.00602360558135929\\
398.01	0.00602115439629455\\
399.01	0.00601864887327858\\
400.01	0.00601608770394962\\
401.01	0.00601346954245762\\
402.01	0.0060107930040585\\
403.01	0.00600805666363816\\
404.01	0.0060052590541629\\
405.01	0.00600239866505064\\
406.01	0.0059994739404585\\
407.01	0.00599648327748143\\
408.01	0.00599342502425666\\
409.01	0.00599029747796719\\
410.01	0.00598709888273879\\
411.01	0.00598382742742423\\
412.01	0.00598048124326533\\
413.01	0.00597705840142861\\
414.01	0.00597355691040341\\
415.01	0.0059699747132562\\
416.01	0.00596630968473113\\
417.01	0.0059625596281877\\
418.01	0.005958722272366\\
419.01	0.00595479526796988\\
420.01	0.00595077618405627\\
421.01	0.00594666250422184\\
422.01	0.00594245162257568\\
423.01	0.00593814083948637\\
424.01	0.00593372735709521\\
425.01	0.00592920827458402\\
426.01	0.00592458058318966\\
427.01	0.00591984116095729\\
428.01	0.00591498676722701\\
429.01	0.00591001403685044\\
430.01	0.00590491947413866\\
431.01	0.00589969944654457\\
432.01	0.00589435017809278\\
433.01	0.00588886774257454\\
434.01	0.00588324805653506\\
435.01	0.00587748687209578\\
436.01	0.00587157976966431\\
437.01	0.00586552215061011\\
438.01	0.00585930923000071\\
439.01	0.00585293602952843\\
440.01	0.00584639737078868\\
441.01	0.00583968786911788\\
442.01	0.00583280192824764\\
443.01	0.00582573373609972\\
444.01	0.00581847726211837\\
445.01	0.00581102625662853\\
446.01	0.00580337425281525\\
447.01	0.00579551457204062\\
448.01	0.00578744033335727\\
449.01	0.00577914446823792\\
450.01	0.0057706197417122\\
451.01	0.00576185878129259\\
452.01	0.00575285411524921\\
453.01	0.00574359822197355\\
454.01	0.0057340835922873\\
455.01	0.00572430280659619\\
456.01	0.00571424862866944\\
457.01	0.00570391411745597\\
458.01	0.00569329275756899\\
459.01	0.00568237860767353\\
460.01	0.00567116646366767\\
461.01	0.00565965202979944\\
462.01	0.00564783208504811\\
463.01	0.00563570462326951\\
464.01	0.00562326893240791\\
465.01	0.00561052555859875\\
466.01	0.00559747607251173\\
467.01	0.00558412251397143\\
468.01	0.00557046633115286\\
469.01	0.00555650547784293\\
470.01	0.00554223133538896\\
471.01	0.00552763343169557\\
472.01	0.0055127004134383\\
473.01	0.00549741991536809\\
474.01	0.00548177841033745\\
475.01	0.00546576103898249\\
476.01	0.00544935141910451\\
477.01	0.00543253143675166\\
478.01	0.00541528102426584\\
479.01	0.00539757793587015\\
480.01	0.00537939754017152\\
481.01	0.00536071265478665\\
482.01	0.00534149324328653\\
483.01	0.00532170581780566\\
484.01	0.0053013130343658\\
485.01	0.00528027341191707\\
486.01	0.00525854117617961\\
487.01	0.00523606632551913\\
488.01	0.00521279506473645\\
489.01	0.00518867082220332\\
490.01	0.00516363616514728\\
491.01	0.00513763577635102\\
492.01	0.00511061761989139\\
493.01	0.005082533328188\\
494.01	0.00505334193289948\\
495.01	0.00502301575464103\\
496.01	0.00499154895893617\\
497.01	0.00495896985165954\\
498.01	0.00492545203691938\\
499.01	0.00489134146735274\\
500.01	0.00485667159027827\\
501.01	0.00482146026587745\\
502.01	0.00478573084122105\\
503.01	0.00474951299744011\\
504.01	0.00471284366497251\\
505.01	0.00467576798570704\\
506.01	0.0046383402842913\\
507.01	0.00460062498440062\\
508.01	0.0045626973659973\\
509.01	0.00452464400539763\\
510.01	0.004486562661826\\
511.01	0.00444856126176999\\
512.01	0.00441075547270626\\
513.01	0.00437326413161136\\
514.01	0.00433620147457205\\
515.01	0.00429966466474605\\
516.01	0.00426371448546018\\
517.01	0.00422826400791097\\
518.01	0.00419282246814783\\
519.01	0.00415735756061899\\
520.01	0.00412191637869819\\
521.01	0.00408654552542253\\
522.01	0.00405128958565977\\
523.01	0.00401618912812359\\
524.01	0.0039812781188851\\
525.01	0.00394658067462882\\
526.01	0.00391210752426048\\
527.01	0.00387785257411629\\
528.01	0.00384378984601807\\
529.01	0.00380987145583695\\
530.01	0.00377602777227343\\
531.01	0.00374217157602695\\
532.01	0.00370821547002062\\
533.01	0.00367412944135205\\
534.01	0.00363990671496194\\
535.01	0.00360553450406351\\
536.01	0.0035709934348749\\
537.01	0.00353625745259954\\
538.01	0.00350129414109099\\
539.01	0.00346606557610664\\
540.01	0.00343052981613095\\
541.01	0.00339464308259496\\
542.01	0.00335836256461435\\
543.01	0.00332164955867218\\
544.01	0.00328447223652297\\
545.01	0.00324680501038172\\
546.01	0.00320862242941572\\
547.01	0.00316989738464901\\
548.01	0.00313060163178219\\
549.01	0.00309070644051983\\
550.01	0.00305018329158036\\
551.01	0.0030090045480664\\
552.01	0.00296714399849765\\
553.01	0.0029245771474564\\
554.01	0.00288128113628113\\
555.01	0.0028372342396416\\
556.01	0.00279241518212518\\
557.01	0.00274680296146442\\
558.01	0.0027003770362638\\
559.01	0.00265311750804224\\
560.01	0.0026050052522004\\
561.01	0.00255602198689257\\
562.01	0.00250615027724643\\
563.01	0.0024553734886416\\
564.01	0.00240367572505146\\
565.01	0.00235104180713719\\
566.01	0.00229745733122292\\
567.01	0.0022429087741126\\
568.01	0.00218738359614695\\
569.01	0.00213087033732275\\
570.01	0.00207335871199557\\
571.01	0.00201483971048969\\
572.01	0.00195530571712336\\
573.01	0.0018947506524752\\
574.01	0.0018331701421967\\
575.01	0.00177056170682958\\
576.01	0.00170692496523384\\
577.01	0.00164226185009494\\
578.01	0.00157657683664837\\
579.01	0.00150987718522774\\
580.01	0.00144217319681236\\
581.01	0.00137347847869735\\
582.01	0.0013038102150756\\
583.01	0.0012331894353757\\
584.01	0.00116164127211568\\
585.01	0.00108919519887159\\
586.01	0.00101588523658629\\
587.01	0.000941750113060405\\
588.01	0.00086683335623894\\
589.01	0.000791183296731767\\
590.01	0.000714852948722578\\
591.01	0.000637899730722589\\
592.01	0.00056038497794602\\
593.01	0.000482373185672669\\
594.01	0.000403930907138838\\
595.01	0.000325125209595895\\
596.01	0.000246021567360271\\
597.01	0.000166681039954817\\
598.01	9.13379815197009e-05\\
599.01	2.91271958372443e-05\\
599.02	2.86192783627535e-05\\
599.03	2.8114423742006e-05\\
599.04	2.76126617978888e-05\\
599.05	2.71140226472798e-05\\
599.06	2.66185367039581e-05\\
599.07	2.61262346815533e-05\\
599.08	2.56371475965082e-05\\
599.09	2.51513067710818e-05\\
599.1	2.46687438363903e-05\\
599.11	2.41894907354479e-05\\
599.12	2.37135797262807e-05\\
599.13	2.3241043385025e-05\\
599.14	2.27719146091033e-05\\
599.15	2.23062266203871e-05\\
599.16	2.18440129684284e-05\\
599.17	2.138530753369e-05\\
599.18	2.09301445308497e-05\\
599.19	2.04785585120812e-05\\
599.2	2.00305843704295e-05\\
599.21	1.95862573431644e-05\\
599.22	1.91456130152045e-05\\
599.23	1.87086873225627e-05\\
599.24	1.82755165558171e-05\\
599.25	1.78461373636363e-05\\
599.26	1.74205874164651e-05\\
599.27	1.69989072017485e-05\\
599.28	1.65811376101922e-05\\
599.29	1.61673199397579e-05\\
599.3	1.57574958996841e-05\\
599.31	1.53517076145714e-05\\
599.32	1.49499976284904e-05\\
599.33	1.45524089091315e-05\\
599.34	1.41589848520109e-05\\
599.35	1.37697692846831e-05\\
599.36	1.33848064710462e-05\\
599.37	1.3004141115644e-05\\
599.38	1.26278183680325e-05\\
599.39	1.22558838271912e-05\\
599.4	1.18883835459657e-05\\
599.41	1.15253640355657e-05\\
599.42	1.11668722700964e-05\\
599.43	1.08129556911502e-05\\
599.44	1.04636622124312e-05\\
599.45	1.01190402244239e-05\\
599.46	9.77913859911798e-06\\
599.47	9.44400669477576e-06\\
599.48	9.11369436074755e-06\\
599.49	8.7882519423238e-06\\
599.5	8.46773028565471e-06\\
599.51	8.15218074270464e-06\\
599.52	7.84165517625675e-06\\
599.53	7.53620596497147e-06\\
599.54	7.23588600849874e-06\\
599.55	6.94074873261973e-06\\
599.56	6.65084809447353e-06\\
599.57	6.36623858780126e-06\\
599.58	6.08697524826993e-06\\
599.59	5.81311365882228e-06\\
599.6	5.54470995511175e-06\\
599.61	5.28182083095637e-06\\
599.62	5.02450354387257e-06\\
599.63	4.7728159206558e-06\\
599.64	4.52681636300273e-06\\
599.65	4.28656385322197e-06\\
599.66	4.05211795995869e-06\\
599.67	3.8235388440163e-06\\
599.68	3.60088726420078e-06\\
599.69	3.38422458325341e-06\\
599.7	3.17361277382341e-06\\
599.71	2.96911442449269e-06\\
599.72	2.7707927458924e-06\\
599.73	2.57871157684567e-06\\
599.74	2.3929353905848e-06\\
599.75	2.213529301031e-06\\
599.76	2.04055906913997e-06\\
599.77	1.87409110929959e-06\\
599.78	1.71419249580217e-06\\
599.79	1.56093096936177e-06\\
599.8	1.41437494372704e-06\\
599.81	1.27459351233379e-06\\
599.82	1.14165645502366e-06\\
599.83	1.01563424484766e-06\\
599.84	8.96598054920053e-07\\
599.85	7.84619765348618e-07\\
599.86	6.79771970230753e-07\\
599.87	5.82127984719016e-07\\
599.88	4.91761852147374e-07\\
599.89	4.08748351251112e-07\\
599.9	3.33163003437068e-07\\
599.91	2.65082080131915e-07\\
599.92	2.04582610201579e-07\\
599.93	1.51742387452178e-07\\
599.94	1.06639978189951e-07\\
599.95	6.93547288800611e-08\\
599.96	3.99667738505e-08\\
599.97	1.85570430896731e-08\\
599.98	5.20727013418598e-09\\
599.99	0\\
600	0\\
};
\addplot [color=blue!50!mycolor7,solid,forget plot]
  table[row sep=crcr]{%
0.01	0.00643064630349385\\
1.01	0.00643064582104349\\
2.01	0.00643064532843654\\
3.01	0.00643064482545811\\
4.01	0.00643064431188899\\
5.01	0.00643064378750511\\
6.01	0.00643064325207764\\
7.01	0.00643064270537298\\
8.01	0.00643064214715251\\
9.01	0.00643064157717261\\
10.01	0.0064306409951844\\
11.01	0.00643064040093377\\
12.01	0.00643063979416122\\
13.01	0.00643063917460158\\
14.01	0.00643063854198422\\
15.01	0.00643063789603248\\
16.01	0.00643063723646407\\
17.01	0.00643063656299057\\
18.01	0.00643063587531759\\
19.01	0.0064306351731443\\
20.01	0.0064306344561634\\
21.01	0.00643063372406137\\
22.01	0.00643063297651745\\
23.01	0.00643063221320462\\
24.01	0.00643063143378841\\
25.01	0.00643063063792766\\
26.01	0.00643062982527373\\
27.01	0.0064306289954705\\
28.01	0.00643062814815421\\
29.01	0.00643062728295375\\
30.01	0.00643062639948946\\
31.01	0.00643062549737406\\
32.01	0.00643062457621179\\
33.01	0.00643062363559861\\
34.01	0.00643062267512169\\
35.01	0.00643062169435945\\
36.01	0.00643062069288122\\
37.01	0.00643061967024741\\
38.01	0.00643061862600884\\
39.01	0.00643061755970665\\
40.01	0.00643061647087234\\
41.01	0.00643061535902733\\
42.01	0.00643061422368286\\
43.01	0.00643061306433972\\
44.01	0.00643061188048794\\
45.01	0.00643061067160677\\
46.01	0.00643060943716425\\
47.01	0.00643060817661714\\
48.01	0.00643060688941042\\
49.01	0.00643060557497739\\
50.01	0.00643060423273907\\
51.01	0.00643060286210411\\
52.01	0.00643060146246859\\
53.01	0.00643060003321559\\
54.01	0.00643059857371514\\
55.01	0.00643059708332355\\
56.01	0.0064305955613835\\
57.01	0.00643059400722356\\
58.01	0.0064305924201578\\
59.01	0.00643059079948582\\
60.01	0.00643058914449208\\
61.01	0.00643058745444574\\
62.01	0.00643058572860015\\
63.01	0.00643058396619301\\
64.01	0.00643058216644525\\
65.01	0.00643058032856141\\
66.01	0.00643057845172888\\
67.01	0.00643057653511765\\
68.01	0.00643057457787966\\
69.01	0.00643057257914896\\
70.01	0.00643057053804087\\
71.01	0.00643056845365179\\
72.01	0.00643056632505849\\
73.01	0.00643056415131834\\
74.01	0.0064305619314682\\
75.01	0.00643055966452425\\
76.01	0.00643055734948144\\
77.01	0.00643055498531337\\
78.01	0.00643055257097158\\
79.01	0.00643055010538465\\
80.01	0.00643054758745861\\
81.01	0.00643054501607576\\
82.01	0.0064305423900943\\
83.01	0.00643053970834811\\
84.01	0.00643053696964572\\
85.01	0.00643053417277017\\
86.01	0.00643053131647824\\
87.01	0.00643052839949989\\
88.01	0.00643052542053795\\
89.01	0.006430522378267\\
90.01	0.00643051927133346\\
91.01	0.00643051609835429\\
92.01	0.00643051285791676\\
93.01	0.00643050954857762\\
94.01	0.00643050616886266\\
95.01	0.00643050271726579\\
96.01	0.0064304991922484\\
97.01	0.0064304955922388\\
98.01	0.00643049191563138\\
99.01	0.00643048816078568\\
100.01	0.00643048432602608\\
101.01	0.00643048040964076\\
102.01	0.00643047640988088\\
103.01	0.00643047232495973\\
104.01	0.00643046815305227\\
105.01	0.00643046389229391\\
106.01	0.00643045954077953\\
107.01	0.00643045509656325\\
108.01	0.00643045055765674\\
109.01	0.00643044592202909\\
110.01	0.00643044118760507\\
111.01	0.00643043635226488\\
112.01	0.00643043141384262\\
113.01	0.00643042637012548\\
114.01	0.00643042121885298\\
115.01	0.00643041595771548\\
116.01	0.0064304105843536\\
117.01	0.00643040509635664\\
118.01	0.00643039949126147\\
119.01	0.00643039376655211\\
120.01	0.00643038791965752\\
121.01	0.00643038194795135\\
122.01	0.00643037584874987\\
123.01	0.00643036961931142\\
124.01	0.00643036325683483\\
125.01	0.00643035675845791\\
126.01	0.00643035012125671\\
127.01	0.00643034334224343\\
128.01	0.00643033641836551\\
129.01	0.00643032934650421\\
130.01	0.00643032212347274\\
131.01	0.00643031474601531\\
132.01	0.00643030721080514\\
133.01	0.00643029951444309\\
134.01	0.00643029165345632\\
135.01	0.00643028362429626\\
136.01	0.00643027542333703\\
137.01	0.00643026704687377\\
138.01	0.00643025849112109\\
139.01	0.00643024975221125\\
140.01	0.00643024082619188\\
141.01	0.00643023170902473\\
142.01	0.00643022239658341\\
143.01	0.00643021288465138\\
144.01	0.00643020316892044\\
145.01	0.00643019324498823\\
146.01	0.00643018310835617\\
147.01	0.00643017275442751\\
148.01	0.00643016217850524\\
149.01	0.00643015137578944\\
150.01	0.00643014034137553\\
151.01	0.00643012907025174\\
152.01	0.00643011755729637\\
153.01	0.00643010579727588\\
154.01	0.0064300937848422\\
155.01	0.00643008151453026\\
156.01	0.00643006898075504\\
157.01	0.00643005617780936\\
158.01	0.00643004309986107\\
159.01	0.00643002974095002\\
160.01	0.00643001609498536\\
161.01	0.00643000215574258\\
162.01	0.00642998791686077\\
163.01	0.00642997337183905\\
164.01	0.00642995851403413\\
165.01	0.00642994333665655\\
166.01	0.00642992783276766\\
167.01	0.00642991199527631\\
168.01	0.0064298958169353\\
169.01	0.00642987929033816\\
170.01	0.00642986240791533\\
171.01	0.00642984516193045\\
172.01	0.00642982754447661\\
173.01	0.00642980954747299\\
174.01	0.00642979116266056\\
175.01	0.006429772381598\\
176.01	0.0064297531956577\\
177.01	0.00642973359602176\\
178.01	0.00642971357367735\\
179.01	0.00642969311941259\\
180.01	0.00642967222381217\\
181.01	0.00642965087725234\\
182.01	0.00642962906989663\\
183.01	0.0064296067916905\\
184.01	0.00642958403235734\\
185.01	0.00642956078139252\\
186.01	0.00642953702805845\\
187.01	0.00642951276137988\\
188.01	0.00642948797013775\\
189.01	0.006429462642864\\
190.01	0.00642943676783618\\
191.01	0.00642941033307095\\
192.01	0.00642938332631894\\
193.01	0.0064293557350581\\
194.01	0.00642932754648815\\
195.01	0.00642929874752325\\
196.01	0.00642926932478665\\
197.01	0.00642923926460298\\
198.01	0.00642920855299219\\
199.01	0.00642917717566235\\
200.01	0.00642914511800234\\
201.01	0.00642911236507465\\
202.01	0.00642907890160809\\
203.01	0.00642904471198972\\
204.01	0.00642900978025752\\
205.01	0.00642897409009218\\
206.01	0.00642893762480835\\
207.01	0.006428900367347\\
208.01	0.00642886230026659\\
209.01	0.00642882340573394\\
210.01	0.00642878366551559\\
211.01	0.00642874306096793\\
212.01	0.0064287015730287\\
213.01	0.00642865918220635\\
214.01	0.00642861586857064\\
215.01	0.00642857161174234\\
216.01	0.00642852639088269\\
217.01	0.00642848018468272\\
218.01	0.00642843297135269\\
219.01	0.00642838472861051\\
220.01	0.00642833543367017\\
221.01	0.00642828506323037\\
222.01	0.00642823359346217\\
223.01	0.00642818099999707\\
224.01	0.00642812725791366\\
225.01	0.00642807234172527\\
226.01	0.00642801622536703\\
227.01	0.00642795888218092\\
228.01	0.00642790028490337\\
229.01	0.00642784040565012\\
230.01	0.00642777921590195\\
231.01	0.00642771668648957\\
232.01	0.00642765278757831\\
233.01	0.00642758748865235\\
234.01	0.00642752075849874\\
235.01	0.00642745256519069\\
236.01	0.00642738287607096\\
237.01	0.00642731165773421\\
238.01	0.00642723887600928\\
239.01	0.00642716449594127\\
240.01	0.00642708848177243\\
241.01	0.00642701079692326\\
242.01	0.00642693140397324\\
243.01	0.00642685026463986\\
244.01	0.00642676733975887\\
245.01	0.00642668258926261\\
246.01	0.00642659597215879\\
247.01	0.00642650744650802\\
248.01	0.0064264169694011\\
249.01	0.00642632449693564\\
250.01	0.00642622998419238\\
251.01	0.00642613338521051\\
252.01	0.00642603465296263\\
253.01	0.00642593373932892\\
254.01	0.00642583059507045\\
255.01	0.00642572516980294\\
256.01	0.00642561741196819\\
257.01	0.00642550726880583\\
258.01	0.00642539468632444\\
259.01	0.00642527960927127\\
260.01	0.0064251619811019\\
261.01	0.0064250417439487\\
262.01	0.00642491883858825\\
263.01	0.00642479320440894\\
264.01	0.00642466477937649\\
265.01	0.00642453349999969\\
266.01	0.00642439930129447\\
267.01	0.00642426211674758\\
268.01	0.00642412187827902\\
269.01	0.00642397851620397\\
270.01	0.00642383195919334\\
271.01	0.00642368213423309\\
272.01	0.00642352896658358\\
273.01	0.00642337237973691\\
274.01	0.00642321229537337\\
275.01	0.00642304863331709\\
276.01	0.00642288131149044\\
277.01	0.00642271024586754\\
278.01	0.00642253535042629\\
279.01	0.00642235653709898\\
280.01	0.00642217371572249\\
281.01	0.00642198679398665\\
282.01	0.00642179567738139\\
283.01	0.00642160026914291\\
284.01	0.00642140047019842\\
285.01	0.00642119617910903\\
286.01	0.00642098729201231\\
287.01	0.00642077370256248\\
288.01	0.00642055530186996\\
289.01	0.0064203319784387\\
290.01	0.00642010361810299\\
291.01	0.0064198701039615\\
292.01	0.00641963131631155\\
293.01	0.00641938713257997\\
294.01	0.00641913742725404\\
295.01	0.00641888207180969\\
296.01	0.00641862093463861\\
297.01	0.00641835388097361\\
298.01	0.00641808077281268\\
299.01	0.00641780146884092\\
300.01	0.00641751582435062\\
301.01	0.00641722369116083\\
302.01	0.0064169249175334\\
303.01	0.00641661934808929\\
304.01	0.00641630682372139\\
305.01	0.00641598718150675\\
306.01	0.00641566025461678\\
307.01	0.00641532587222526\\
308.01	0.00641498385941516\\
309.01	0.00641463403708344\\
310.01	0.00641427622184446\\
311.01	0.00641391022593068\\
312.01	0.00641353585709311\\
313.01	0.00641315291849869\\
314.01	0.00641276120862677\\
315.01	0.00641236052116347\\
316.01	0.00641195064489498\\
317.01	0.00641153136359834\\
318.01	0.00641110245593143\\
319.01	0.00641066369532103\\
320.01	0.00641021484984945\\
321.01	0.00640975568213994\\
322.01	0.00640928594923997\\
323.01	0.006408805402504\\
324.01	0.00640831378747466\\
325.01	0.0064078108437623\\
326.01	0.00640729630492403\\
327.01	0.00640676989834167\\
328.01	0.006406231345098\\
329.01	0.00640568035985288\\
330.01	0.00640511665071823\\
331.01	0.00640453991913279\\
332.01	0.00640394985973552\\
333.01	0.00640334616023881\\
334.01	0.00640272850130228\\
335.01	0.00640209655640494\\
336.01	0.00640144999171812\\
337.01	0.00640078846597794\\
338.01	0.00640011163035872\\
339.01	0.00639941912834587\\
340.01	0.00639871059560917\\
341.01	0.00639798565987746\\
342.01	0.00639724394081292\\
343.01	0.00639648504988657\\
344.01	0.00639570859025427\\
345.01	0.00639491415663374\\
346.01	0.0063941013351825\\
347.01	0.00639326970337583\\
348.01	0.0063924188298865\\
349.01	0.00639154827446339\\
350.01	0.00639065758781175\\
351.01	0.0063897463114725\\
352.01	0.00638881397770028\\
353.01	0.00638786010934019\\
354.01	0.00638688421970219\\
355.01	0.00638588581243139\\
356.01	0.00638486438137207\\
357.01	0.00638381941042528\\
358.01	0.00638275037339532\\
359.01	0.00638165673382387\\
360.01	0.00638053794480633\\
361.01	0.00637939344878729\\
362.01	0.00637822267732865\\
363.01	0.00637702505084507\\
364.01	0.00637579997829815\\
365.01	0.00637454685684178\\
366.01	0.00637326507140846\\
367.01	0.00637195399422571\\
368.01	0.00637061298425105\\
369.01	0.00636924138651295\\
370.01	0.00636783853134532\\
371.01	0.0063664037335029\\
372.01	0.00636493629114794\\
373.01	0.00636343548470029\\
374.01	0.00636190057554987\\
375.01	0.00636033080463596\\
376.01	0.00635872539091282\\
377.01	0.00635708352973136\\
378.01	0.00635540439119044\\
379.01	0.00635368711853009\\
380.01	0.00635193082666611\\
381.01	0.00635013460098661\\
382.01	0.0063482974965479\\
383.01	0.00634641853779242\\
384.01	0.00634449671886276\\
385.01	0.00634253100443558\\
386.01	0.00634052033068126\\
387.01	0.00633846360543564\\
388.01	0.0063363597075115\\
389.01	0.00633420748584856\\
390.01	0.0063320057586354\\
391.01	0.00632975331240574\\
392.01	0.00632744890110624\\
393.01	0.00632509124513656\\
394.01	0.00632267903036051\\
395.01	0.00632021090708838\\
396.01	0.00631768548902773\\
397.01	0.00631510135220493\\
398.01	0.00631245703385488\\
399.01	0.00630975103127821\\
400.01	0.00630698180066793\\
401.01	0.00630414775590178\\
402.01	0.00630124726730199\\
403.01	0.00629827866036199\\
404.01	0.00629524021443894\\
405.01	0.00629213016141334\\
406.01	0.00628894668431449\\
407.01	0.00628568791591252\\
408.01	0.0062823519372778\\
409.01	0.00627893677630708\\
410.01	0.00627544040621913\\
411.01	0.00627186074401835\\
412.01	0.00626819564893003\\
413.01	0.00626444292080767\\
414.01	0.00626060029851503\\
415.01	0.00625666545828518\\
416.01	0.00625263601205956\\
417.01	0.00624850950581186\\
418.01	0.00624428341785974\\
419.01	0.00623995515716993\\
420.01	0.00623552206166427\\
421.01	0.00623098139653246\\
422.01	0.00622633035255973\\
423.01	0.00622156604448145\\
424.01	0.00621668550937332\\
425.01	0.00621168570509246\\
426.01	0.00620656350878387\\
427.01	0.00620131571546979\\
428.01	0.00619593903674235\\
429.01	0.00619043009958347\\
430.01	0.00618478544533691\\
431.01	0.00617900152886527\\
432.01	0.0061730747179242\\
433.01	0.00616700129279502\\
434.01	0.0061607774462197\\
435.01	0.00615439928368823\\
436.01	0.00614786282413647\\
437.01	0.00614116400111684\\
438.01	0.00613429866451252\\
439.01	0.00612726258287525\\
440.01	0.00612005144647123\\
441.01	0.00611266087113022\\
442.01	0.00610508640299985\\
443.01	0.00609732352431131\\
444.01	0.00608936766027169\\
445.01	0.00608121418719717\\
446.01	0.00607285844200196\\
447.01	0.00606429573315153\\
448.01	0.00605552135317634\\
449.01	0.00604653059281799\\
450.01	0.00603731875684806\\
451.01	0.00602788118154597\\
452.01	0.00601821325375907\\
453.01	0.00600831043136694\\
454.01	0.00599816826485659\\
455.01	0.00598778241955891\\
456.01	0.00597714869790946\\
457.01	0.00596626306087276\\
458.01	0.00595512164741876\\
459.01	0.00594372079067494\\
460.01	0.0059320570291298\\
461.01	0.00592012711109118\\
462.01	0.00590792799059727\\
463.01	0.0058954568132993\\
464.01	0.00588271089173592\\
465.01	0.00586968767129047\\
466.01	0.00585638469160731\\
467.01	0.00584279955437026\\
468.01	0.00582892991892154\\
469.01	0.00581477357391869\\
470.01	0.00580032866011119\\
471.01	0.00578559382851982\\
472.01	0.00577056826338928\\
473.01	0.00575525166452245\\
474.01	0.00573964417868944\\
475.01	0.00572374625327787\\
476.01	0.00570755837379365\\
477.01	0.00569108063047653\\
478.01	0.00567431203588562\\
479.01	0.00565724948118748\\
480.01	0.00563988617705282\\
481.01	0.00562221074290286\\
482.01	0.00560421942995308\\
483.01	0.00558591482612433\\
484.01	0.0055672980890903\\
485.01	0.00554836727342983\\
486.01	0.00552911500826401\\
487.01	0.0055095251175396\\
488.01	0.00548956776392307\\
489.01	0.00546919253983309\\
490.01	0.00544831871406007\\
491.01	0.00542684577263245\\
492.01	0.00540473991092167\\
493.01	0.00538199139560689\\
494.01	0.00535859211425415\\
495.01	0.00533453520495696\\
496.01	0.00530981416923446\\
497.01	0.00528442116996276\\
498.01	0.00525834343539328\\
499.01	0.00523155481476314\\
500.01	0.00520402129849684\\
501.01	0.00517570600084901\\
502.01	0.00514656899797631\\
503.01	0.00511656711354349\\
504.01	0.00508565372376103\\
505.01	0.00505377859573597\\
506.01	0.00502088779737199\\
507.01	0.00498692380955012\\
508.01	0.0049518259575015\\
509.01	0.00491553127513892\\
510.01	0.00487797600476396\\
511.01	0.00483909802366951\\
512.01	0.00479884060839485\\
513.01	0.0047571581135219\\
514.01	0.00471402437318059\\
515.01	0.00466944495525777\\
516.01	0.0046234748460071\\
517.01	0.00457632493610646\\
518.01	0.00452859323452803\\
519.01	0.00448040581544159\\
520.01	0.00443181764632897\\
521.01	0.00438289402962435\\
522.01	0.00433371322031132\\
523.01	0.00428437016558703\\
524.01	0.00423498183390211\\
525.01	0.00418569014161186\\
526.01	0.00413665505471583\\
527.01	0.00408804816755842\\
528.01	0.00404004460393849\\
529.01	0.00399281066845933\\
530.01	0.00394648423884527\\
531.01	0.00390114445455031\\
532.01	0.00385652207826656\\
533.01	0.00381203123192864\\
534.01	0.00376770964252486\\
535.01	0.00372361446977168\\
536.01	0.00367979415281411\\
537.01	0.00363628455305909\\
538.01	0.00359310461580987\\
539.01	0.00355025179220907\\
540.01	0.00350769767107562\\
541.01	0.00346538460813856\\
542.01	0.00342322468084414\\
543.01	0.00338110309495784\\
544.01	0.00333889140574861\\
545.01	0.00329651627776666\\
546.01	0.00325395959978935\\
547.01	0.00321119741588083\\
548.01	0.00316819686683313\\
549.01	0.00312491664011164\\
550.01	0.00308130814178034\\
551.01	0.00303731755150926\\
552.01	0.00299288892686615\\
553.01	0.0029479682313795\\
554.01	0.0029025077891753\\
555.01	0.00285646988285535\\
556.01	0.00280982387424166\\
557.01	0.0027625388315893\\
558.01	0.00271458284913515\\
559.01	0.00266592390435199\\
560.01	0.00261653075024179\\
561.01	0.00256637376772603\\
562.01	0.00251542565903956\\
563.01	0.0024636617766947\\
564.01	0.00241105990989469\\
565.01	0.00235759950936004\\
566.01	0.00230326096645182\\
567.01	0.00224802567304017\\
568.01	0.00219187631544338\\
569.01	0.00213479711176823\\
570.01	0.00207677396396737\\
571.01	0.00201779451774308\\
572.01	0.00195784815156964\\
573.01	0.00189692594923144\\
574.01	0.00183502073790612\\
575.01	0.0017721272474868\\
576.01	0.00170824233381175\\
577.01	0.0016433652017279\\
578.01	0.00157749762286559\\
579.01	0.00151064415615248\\
580.01	0.00144281238208389\\
581.01	0.00137401316117766\\
582.01	0.0013042609204186\\
583.01	0.00123357395803039\\
584.01	0.00116197474577273\\
585.01	0.00108949021141372\\
586.01	0.00101615198804251\\
587.01	0.000941996614273563\\
588.01	0.000867065664706434\\
589.01	0.000791405783613175\\
590.01	0.000715068586964641\\
591.01	0.000638110389654499\\
592.01	0.000560591707059254\\
593.01	0.000482576471267222\\
594.01	0.000404130889408053\\
595.01	0.000325321853632796\\
596.01	0.000246214788620124\\
597.01	0.000166870791567188\\
598.01	9.1337981536533e-05\\
599.01	2.91271958373258e-05\\
599.02	2.86192783628299e-05\\
599.03	2.81144237420771e-05\\
599.04	2.76126617979548e-05\\
599.05	2.71140226473406e-05\\
599.06	2.66185367040154e-05\\
599.07	2.61262346816053e-05\\
599.08	2.5637147596555e-05\\
599.09	2.51513067711269e-05\\
599.1	2.46687438364285e-05\\
599.11	2.41894907354861e-05\\
599.12	2.37135797263137e-05\\
599.13	2.32410433850579e-05\\
599.14	2.27719146091345e-05\\
599.15	2.23062266204149e-05\\
599.16	2.1844012968451e-05\\
599.17	2.13853075337143e-05\\
599.18	2.09301445308688e-05\\
599.19	2.04785585121003e-05\\
599.2	2.00305843704451e-05\\
599.21	1.95862573431783e-05\\
599.22	1.91456130152166e-05\\
599.23	1.87086873225731e-05\\
599.24	1.82755165558292e-05\\
599.25	1.78461373636484e-05\\
599.26	1.74205874164755e-05\\
599.27	1.69989072017589e-05\\
599.28	1.65811376102026e-05\\
599.29	1.61673199397631e-05\\
599.3	1.57574958996893e-05\\
599.31	1.53517076145766e-05\\
599.32	1.49499976284957e-05\\
599.33	1.45524089091385e-05\\
599.34	1.41589848520127e-05\\
599.35	1.37697692846866e-05\\
599.36	1.33848064710479e-05\\
599.37	1.30041411156457e-05\\
599.38	1.2627818368036e-05\\
599.39	1.22558838271929e-05\\
599.4	1.18883835459691e-05\\
599.41	1.15253640355674e-05\\
599.42	1.11668722700981e-05\\
599.43	1.08129556911536e-05\\
599.44	1.04636622124312e-05\\
599.45	1.01190402244239e-05\\
599.46	9.77913859911798e-06\\
599.47	9.44400669477749e-06\\
599.48	9.11369436074755e-06\\
599.49	8.7882519423238e-06\\
599.5	8.46773028565471e-06\\
599.51	8.15218074270464e-06\\
599.52	7.84165517625848e-06\\
599.53	7.5362059649732e-06\\
599.54	7.23588600849874e-06\\
599.55	6.94074873262146e-06\\
599.56	6.65084809447526e-06\\
599.57	6.36623858780126e-06\\
599.58	6.08697524826819e-06\\
599.59	5.81311365882228e-06\\
599.6	5.54470995511175e-06\\
599.61	5.28182083095637e-06\\
599.62	5.02450354387431e-06\\
599.63	4.7728159206558e-06\\
599.64	4.52681636300446e-06\\
599.65	4.28656385322197e-06\\
599.66	4.05211795996042e-06\\
599.67	3.82353884401457e-06\\
599.68	3.60088726420078e-06\\
599.69	3.38422458325514e-06\\
599.7	3.17361277382168e-06\\
599.71	2.96911442449269e-06\\
599.72	2.77079274589413e-06\\
599.73	2.5787115768474e-06\\
599.74	2.39293539058306e-06\\
599.75	2.21352930102926e-06\\
599.76	2.04055906913997e-06\\
599.77	1.87409110929959e-06\\
599.78	1.71419249580043e-06\\
599.79	1.56093096936177e-06\\
599.8	1.41437494372877e-06\\
599.81	1.27459351233379e-06\\
599.82	1.14165645502366e-06\\
599.83	1.01563424484766e-06\\
599.84	8.96598054920053e-07\\
599.85	7.84619765350353e-07\\
599.86	6.79771970232487e-07\\
599.87	5.82127984717282e-07\\
599.88	4.91761852145639e-07\\
599.89	4.08748351251112e-07\\
599.9	3.33163003438802e-07\\
599.91	2.6508208013365e-07\\
599.92	2.04582610201579e-07\\
599.93	1.51742387450443e-07\\
599.94	1.06639978191686e-07\\
599.95	6.93547288800611e-08\\
599.96	3.99667738487652e-08\\
599.97	1.85570430896731e-08\\
599.98	5.20727013418598e-09\\
599.99	0\\
600	0\\
};
\addplot [color=blue!40!mycolor9,solid,forget plot]
  table[row sep=crcr]{%
0.01	0.00729752532496896\\
1.01	0.00729752466570372\\
2.01	0.00729752399247544\\
3.01	0.00729752330498688\\
4.01	0.00729752260293404\\
5.01	0.00729752188600661\\
6.01	0.00729752115388762\\
7.01	0.00729752040625335\\
8.01	0.00729751964277303\\
9.01	0.00729751886310865\\
10.01	0.00729751806691535\\
11.01	0.00729751725384054\\
12.01	0.00729751642352408\\
13.01	0.0072975155755983\\
14.01	0.00729751470968734\\
15.01	0.0072975138254076\\
16.01	0.00729751292236697\\
17.01	0.00729751200016488\\
18.01	0.00729751105839236\\
19.01	0.0072975100966315\\
20.01	0.00729750911445554\\
21.01	0.00729750811142823\\
22.01	0.00729750708710435\\
23.01	0.00729750604102884\\
24.01	0.00729750497273698\\
25.01	0.00729750388175379\\
26.01	0.0072975027675944\\
27.01	0.00729750162976317\\
28.01	0.00729750046775411\\
29.01	0.00729749928104998\\
30.01	0.00729749806912268\\
31.01	0.00729749683143267\\
32.01	0.00729749556742859\\
33.01	0.00729749427654743\\
34.01	0.00729749295821389\\
35.01	0.00729749161184042\\
36.01	0.00729749023682674\\
37.01	0.00729748883255935\\
38.01	0.00729748739841178\\
39.01	0.00729748593374409\\
40.01	0.00729748443790231\\
41.01	0.00729748291021846\\
42.01	0.00729748135001015\\
43.01	0.00729747975658009\\
44.01	0.00729747812921618\\
45.01	0.00729747646719074\\
46.01	0.00729747476976034\\
47.01	0.00729747303616558\\
48.01	0.00729747126563063\\
49.01	0.00729746945736278\\
50.01	0.00729746761055222\\
51.01	0.00729746572437177\\
52.01	0.00729746379797607\\
53.01	0.0072974618305016\\
54.01	0.00729745982106604\\
55.01	0.00729745776876817\\
56.01	0.00729745567268704\\
57.01	0.00729745353188178\\
58.01	0.00729745134539126\\
59.01	0.00729744911223331\\
60.01	0.00729744683140454\\
61.01	0.00729744450187991\\
62.01	0.00729744212261201\\
63.01	0.00729743969253067\\
64.01	0.00729743721054262\\
65.01	0.0072974346755307\\
66.01	0.00729743208635365\\
67.01	0.00729742944184526\\
68.01	0.00729742674081414\\
69.01	0.00729742398204288\\
70.01	0.00729742116428783\\
71.01	0.00729741828627794\\
72.01	0.00729741534671493\\
73.01	0.00729741234427186\\
74.01	0.0072974092775931\\
75.01	0.0072974061452937\\
76.01	0.00729740294595837\\
77.01	0.00729739967814097\\
78.01	0.00729739634036386\\
79.01	0.00729739293111747\\
80.01	0.00729738944885902\\
81.01	0.00729738589201225\\
82.01	0.0072973822589665\\
83.01	0.00729737854807601\\
84.01	0.00729737475765905\\
85.01	0.00729737088599732\\
86.01	0.00729736693133495\\
87.01	0.00729736289187786\\
88.01	0.00729735876579246\\
89.01	0.00729735455120554\\
90.01	0.00729735024620267\\
91.01	0.0072973458488277\\
92.01	0.00729734135708172\\
93.01	0.0072973367689221\\
94.01	0.00729733208226154\\
95.01	0.007297327294967\\
96.01	0.00729732240485881\\
97.01	0.00729731740970981\\
98.01	0.00729731230724368\\
99.01	0.00729730709513456\\
100.01	0.00729730177100555\\
101.01	0.00729729633242758\\
102.01	0.00729729077691843\\
103.01	0.00729728510194151\\
104.01	0.00729727930490448\\
105.01	0.00729727338315806\\
106.01	0.00729726733399516\\
107.01	0.007297261154649\\
108.01	0.00729725484229201\\
109.01	0.00729724839403466\\
110.01	0.00729724180692394\\
111.01	0.00729723507794183\\
112.01	0.00729722820400403\\
113.01	0.00729722118195852\\
114.01	0.00729721400858374\\
115.01	0.00729720668058757\\
116.01	0.00729719919460506\\
117.01	0.00729719154719732\\
118.01	0.00729718373485004\\
119.01	0.00729717575397083\\
120.01	0.00729716760088874\\
121.01	0.00729715927185167\\
122.01	0.00729715076302475\\
123.01	0.0072971420704886\\
124.01	0.00729713319023731\\
125.01	0.00729712411817658\\
126.01	0.00729711485012171\\
127.01	0.00729710538179554\\
128.01	0.00729709570882657\\
129.01	0.00729708582674661\\
130.01	0.00729707573098867\\
131.01	0.00729706541688462\\
132.01	0.00729705487966345\\
133.01	0.0072970441144483\\
134.01	0.00729703311625429\\
135.01	0.00729702187998626\\
136.01	0.00729701040043639\\
137.01	0.00729699867228122\\
138.01	0.00729698669007947\\
139.01	0.00729697444826889\\
140.01	0.00729696194116412\\
141.01	0.00729694916295372\\
142.01	0.00729693610769704\\
143.01	0.00729692276932173\\
144.01	0.00729690914162044\\
145.01	0.00729689521824786\\
146.01	0.00729688099271778\\
147.01	0.00729686645839981\\
148.01	0.0072968516085158\\
149.01	0.0072968364361372\\
150.01	0.00729682093418094\\
151.01	0.00729680509540626\\
152.01	0.0072967889124112\\
153.01	0.00729677237762894\\
154.01	0.00729675548332381\\
155.01	0.00729673822158769\\
156.01	0.0072967205843362\\
157.01	0.00729670256330462\\
158.01	0.00729668415004327\\
159.01	0.00729666533591438\\
160.01	0.00729664611208676\\
161.01	0.00729662646953205\\
162.01	0.00729660639902012\\
163.01	0.00729658589111428\\
164.01	0.00729656493616682\\
165.01	0.00729654352431402\\
166.01	0.00729652164547135\\
167.01	0.00729649928932837\\
168.01	0.00729647644534364\\
169.01	0.00729645310273948\\
170.01	0.00729642925049629\\
171.01	0.00729640487734771\\
172.01	0.00729637997177418\\
173.01	0.00729635452199763\\
174.01	0.00729632851597542\\
175.01	0.00729630194139441\\
176.01	0.00729627478566468\\
177.01	0.00729624703591328\\
178.01	0.00729621867897761\\
179.01	0.00729618970139889\\
180.01	0.00729616008941532\\
181.01	0.00729612982895523\\
182.01	0.00729609890562987\\
183.01	0.00729606730472636\\
184.01	0.00729603501119971\\
185.01	0.00729600200966582\\
186.01	0.00729596828439349\\
187.01	0.00729593381929612\\
188.01	0.00729589859792411\\
189.01	0.00729586260345605\\
190.01	0.00729582581869023\\
191.01	0.00729578822603618\\
192.01	0.00729574980750549\\
193.01	0.00729571054470255\\
194.01	0.00729567041881541\\
195.01	0.00729562941060609\\
196.01	0.00729558750040048\\
197.01	0.00729554466807877\\
198.01	0.00729550089306479\\
199.01	0.00729545615431571\\
200.01	0.00729541043031107\\
201.01	0.00729536369904184\\
202.01	0.00729531593799909\\
203.01	0.00729526712416246\\
204.01	0.00729521723398821\\
205.01	0.00729516624339727\\
206.01	0.00729511412776297\\
207.01	0.00729506086189784\\
208.01	0.00729500642004107\\
209.01	0.00729495077584507\\
210.01	0.00729489390236179\\
211.01	0.00729483577202908\\
212.01	0.00729477635665604\\
213.01	0.0072947156274087\\
214.01	0.007294653554795\\
215.01	0.00729459010864934\\
216.01	0.00729452525811728\\
217.01	0.00729445897163931\\
218.01	0.00729439121693432\\
219.01	0.007294321960983\\
220.01	0.00729425117001076\\
221.01	0.0072941788094698\\
222.01	0.00729410484402151\\
223.01	0.00729402923751758\\
224.01	0.00729395195298176\\
225.01	0.00729387295258993\\
226.01	0.00729379219765052\\
227.01	0.00729370964858466\\
228.01	0.00729362526490501\\
229.01	0.0072935390051947\\
230.01	0.00729345082708572\\
231.01	0.00729336068723664\\
232.01	0.00729326854130991\\
233.01	0.00729317434394859\\
234.01	0.00729307804875235\\
235.01	0.00729297960825348\\
236.01	0.00729287897389162\\
237.01	0.00729277609598836\\
238.01	0.00729267092372104\\
239.01	0.00729256340509588\\
240.01	0.00729245348692089\\
241.01	0.00729234111477758\\
242.01	0.00729222623299217\\
243.01	0.00729210878460673\\
244.01	0.00729198871134848\\
245.01	0.00729186595359962\\
246.01	0.00729174045036516\\
247.01	0.00729161213924123\\
248.01	0.00729148095638202\\
249.01	0.00729134683646588\\
250.01	0.00729120971266077\\
251.01	0.00729106951658887\\
252.01	0.00729092617829042\\
253.01	0.00729077962618685\\
254.01	0.00729062978704287\\
255.01	0.00729047658592718\\
256.01	0.00729031994617333\\
257.01	0.00729015978933894\\
258.01	0.00728999603516395\\
259.01	0.00728982860152826\\
260.01	0.00728965740440818\\
261.01	0.00728948235783168\\
262.01	0.00728930337383331\\
263.01	0.00728912036240686\\
264.01	0.00728893323145826\\
265.01	0.00728874188675637\\
266.01	0.00728854623188319\\
267.01	0.00728834616818269\\
268.01	0.00728814159470866\\
269.01	0.00728793240817102\\
270.01	0.00728771850288104\\
271.01	0.00728749977069578\\
272.01	0.00728727610096022\\
273.01	0.00728704738044919\\
274.01	0.00728681349330705\\
275.01	0.00728657432098675\\
276.01	0.0072863297421869\\
277.01	0.00728607963278766\\
278.01	0.00728582386578517\\
279.01	0.00728556231122452\\
280.01	0.00728529483613123\\
281.01	0.00728502130444085\\
282.01	0.00728474157692739\\
283.01	0.00728445551113003\\
284.01	0.00728416296127768\\
285.01	0.00728386377821308\\
286.01	0.00728355780931356\\
287.01	0.00728324489841133\\
288.01	0.00728292488571118\\
289.01	0.00728259760770695\\
290.01	0.00728226289709558\\
291.01	0.00728192058268982\\
292.01	0.00728157048932837\\
293.01	0.00728121243778467\\
294.01	0.00728084624467267\\
295.01	0.00728047172235189\\
296.01	0.00728008867882943\\
297.01	0.00727969691765975\\
298.01	0.00727929623784288\\
299.01	0.00727888643371967\\
300.01	0.00727846729486543\\
301.01	0.00727803860598041\\
302.01	0.00727760014677891\\
303.01	0.00727715169187461\\
304.01	0.00727669301066443\\
305.01	0.00727622386720973\\
306.01	0.007275744020114\\
307.01	0.00727525322239907\\
308.01	0.00727475122137744\\
309.01	0.00727423775852274\\
310.01	0.0072737125693362\\
311.01	0.00727317538321151\\
312.01	0.00727262592329506\\
313.01	0.00727206390634452\\
314.01	0.00727148904258305\\
315.01	0.00727090103555079\\
316.01	0.00727029958195293\\
317.01	0.00726968437150379\\
318.01	0.00726905508676791\\
319.01	0.00726841140299698\\
320.01	0.00726775298796305\\
321.01	0.00726707950178749\\
322.01	0.00726639059676636\\
323.01	0.00726568591719115\\
324.01	0.00726496509916505\\
325.01	0.00726422777041479\\
326.01	0.00726347355009767\\
327.01	0.00726270204860363\\
328.01	0.00726191286735221\\
329.01	0.00726110559858428\\
330.01	0.00726027982514809\\
331.01	0.0072594351202791\\
332.01	0.00725857104737481\\
333.01	0.00725768715976247\\
334.01	0.00725678300046007\\
335.01	0.00725585810193118\\
336.01	0.00725491198583203\\
337.01	0.00725394416275126\\
338.01	0.00725295413194201\\
339.01	0.00725194138104491\\
340.01	0.00725090538580397\\
341.01	0.00724984560977174\\
342.01	0.0072487615040063\\
343.01	0.00724765250675767\\
344.01	0.00724651804314418\\
345.01	0.00724535752481752\\
346.01	0.00724417034961681\\
347.01	0.00724295590121094\\
348.01	0.00724171354872714\\
349.01	0.00724044264636811\\
350.01	0.00723914253301424\\
351.01	0.00723781253181164\\
352.01	0.00723645194974541\\
353.01	0.0072350600771969\\
354.01	0.00723363618748432\\
355.01	0.00723217953638594\\
356.01	0.00723068936164578\\
357.01	0.00722916488245954\\
358.01	0.00722760529894151\\
359.01	0.00722600979157003\\
360.01	0.00722437752061259\\
361.01	0.00722270762552784\\
362.01	0.00722099922434488\\
363.01	0.00721925141301872\\
364.01	0.00721746326476159\\
365.01	0.00721563382934946\\
366.01	0.00721376213240344\\
367.01	0.00721184717464625\\
368.01	0.00720988793113445\\
369.01	0.00720788335046698\\
370.01	0.00720583235397099\\
371.01	0.00720373383486799\\
372.01	0.00720158665742248\\
373.01	0.00719938965607631\\
374.01	0.00719714163457338\\
375.01	0.00719484136507993\\
376.01	0.00719248758730472\\
377.01	0.00719007900762558\\
378.01	0.00718761429822667\\
379.01	0.00718509209624926\\
380.01	0.00718251100295767\\
381.01	0.00717986958291612\\
382.01	0.00717716636316688\\
383.01	0.0071743998323949\\
384.01	0.00717156844005421\\
385.01	0.00716867059543121\\
386.01	0.00716570466662249\\
387.01	0.00716266897943273\\
388.01	0.00715956181622413\\
389.01	0.00715638141473258\\
390.01	0.00715312596685178\\
391.01	0.0071497936173832\\
392.01	0.00714638246275295\\
393.01	0.00714289054969291\\
394.01	0.00713931587388703\\
395.01	0.00713565637857985\\
396.01	0.00713190995314948\\
397.01	0.00712807443164127\\
398.01	0.00712414759126293\\
399.01	0.00712012715084104\\
400.01	0.00711601076923545\\
401.01	0.00711179604371409\\
402.01	0.00710748050828523\\
403.01	0.00710306163198771\\
404.01	0.00709853681713742\\
405.01	0.00709390339753045\\
406.01	0.00708915863660106\\
407.01	0.00708429972553548\\
408.01	0.00707932378133915\\
409.01	0.00707422784485851\\
410.01	0.0070690088787558\\
411.01	0.00706366376543657\\
412.01	0.00705818930493069\\
413.01	0.00705258221272485\\
414.01	0.00704683911754739\\
415.01	0.00704095655910548\\
416.01	0.00703493098577409\\
417.01	0.0070287587522365\\
418.01	0.00702243611707767\\
419.01	0.0070159592403298\\
420.01	0.00700932418097047\\
421.01	0.00700252689437422\\
422.01	0.00699556322971849\\
423.01	0.00698842892734394\\
424.01	0.00698111961607107\\
425.01	0.00697363081047384\\
426.01	0.00696595790811191\\
427.01	0.00695809618672318\\
428.01	0.00695004080137968\\
429.01	0.00694178678160729\\
430.01	0.00693332902847468\\
431.01	0.0069246623116534\\
432.01	0.00691578126645409\\
433.01	0.00690668039084431\\
434.01	0.0068973540424542\\
435.01	0.00688779643557772\\
436.01	0.00687800163817969\\
437.01	0.00686796356892092\\
438.01	0.00685767599421733\\
439.01	0.00684713252535202\\
440.01	0.00683632661566616\\
441.01	0.00682525155786101\\
442.01	0.00681390048145097\\
443.01	0.00680226635042226\\
444.01	0.0067903419611648\\
445.01	0.00677811994076563\\
446.01	0.00676559274577735\\
447.01	0.00675275266160967\\
448.01	0.00673959180273363\\
449.01	0.0067261021139434\\
450.01	0.00671227537299304\\
451.01	0.00669810319501526\\
452.01	0.00668357703924631\\
453.01	0.00666868821873034\\
454.01	0.00665342791386457\\
455.01	0.00663778719088858\\
456.01	0.00662175702671926\\
457.01	0.00660532834191707\\
458.01	0.00658849204403995\\
459.01	0.00657123908423049\\
460.01	0.00655356053060433\\
461.01	0.00653544766288437\\
462.01	0.00651689209377258\\
463.01	0.00649788592375032\\
464.01	0.00647842193727471\\
465.01	0.00645849384938605\\
466.01	0.00643809661159538\\
467.01	0.00641722678149182\\
468.01	0.00639588294174248\\
469.01	0.00637406615110894\\
470.01	0.00635178065127839\\
471.01	0.00632903481355111\\
472.01	0.00630584226079985\\
473.01	0.00628222328826783\\
474.01	0.00625820667405755\\
475.01	0.00623383199391051\\
476.01	0.00620915259197159\\
477.01	0.00618423941211728\\
478.01	0.00615918597730179\\
479.01	0.00613411491311489\\
480.01	0.00610918623765697\\
481.01	0.00608439604413744\\
482.01	0.00605925887006073\\
483.01	0.00603379493791583\\
484.01	0.00600808471650222\\
485.01	0.00598223538922471\\
486.01	0.00595638857752813\\
487.01	0.00593073036687392\\
488.01	0.00590550435137352\\
489.01	0.0058810286444121\\
490.01	0.00585771811093029\\
491.01	0.00583492745131773\\
492.01	0.00581153101441279\\
493.01	0.00578751257433942\\
494.01	0.00576285618546294\\
495.01	0.00573754499475068\\
496.01	0.00571156110089086\\
497.01	0.00568488546783704\\
498.01	0.00565749796584674\\
499.01	0.00562937771148849\\
500.01	0.00560050357407001\\
501.01	0.00557085452658569\\
502.01	0.00554041010518881\\
503.01	0.00550915111070708\\
504.01	0.00547706066367205\\
505.01	0.00544412572887732\\
506.01	0.00541033797620297\\
507.01	0.0053756930186474\\
508.01	0.00534019071220625\\
509.01	0.00530383582615455\\
510.01	0.0052666386784107\\
511.01	0.00522861565502374\\
512.01	0.0051897894890629\\
513.01	0.00515018910921669\\
514.01	0.00510984877493633\\
515.01	0.00506880608133519\\
516.01	0.00502709822701115\\
517.01	0.00498475526112614\\
518.01	0.00494178415792848\\
519.01	0.00489816705544328\\
520.01	0.00485386328916163\\
521.01	0.00480879788007793\\
522.01	0.00476284469585961\\
523.01	0.00471580276867963\\
524.01	0.0046673739575744\\
525.01	0.00461732068128018\\
526.01	0.00456552036522299\\
527.01	0.00451190405559132\\
528.01	0.00445643718448044\\
529.01	0.00439912066840682\\
530.01	0.0043400101504628\\
531.01	0.00427924337089882\\
532.01	0.00421731809571334\\
533.01	0.00415501977916613\\
534.01	0.00409250425311458\\
535.01	0.00402992051392232\\
536.01	0.00396743628341949\\
537.01	0.00390523746308485\\
538.01	0.00384352588757257\\
539.01	0.0037825146019863\\
540.01	0.00372241963365504\\
541.01	0.00366344652277587\\
542.01	0.0036057689618982\\
543.01	0.00354949655435368\\
544.01	0.00349450365672057\\
545.01	0.00343991588066112\\
546.01	0.00338564234923884\\
547.01	0.00333174561849863\\
548.01	0.00327827103727041\\
549.01	0.0032252416502087\\
550.01	0.00317265336847283\\
551.01	0.00312046826653471\\
552.01	0.00306860796740712\\
553.01	0.00301695220746892\\
554.01	0.00296534440928589\\
555.01	0.00291361933827177\\
556.01	0.0028617036131168\\
557.01	0.00280956496612162\\
558.01	0.00275716146401953\\
559.01	0.00270444228312854\\
560.01	0.00265134944665799\\
561.01	0.002597819944586\\
562.01	0.00254378938571621\\
563.01	0.00248919751402807\\
564.01	0.00243399416831436\\
565.01	0.0023781426409719\\
566.01	0.00232161204526715\\
567.01	0.00226437095798489\\
568.01	0.00220638810077902\\
569.01	0.00214763356692056\\
570.01	0.00208808010357186\\
571.01	0.0020277041292811\\
572.01	0.00196648618517145\\
573.01	0.00190441059245309\\
574.01	0.00184146435238803\\
575.01	0.00177763626689189\\
576.01	0.00171291713759237\\
577.01	0.00164730022332831\\
578.01	0.00158078160588281\\
579.01	0.00151336043079601\\
580.01	0.00144503902502086\\
581.01	0.00137582293874799\\
582.01	0.00130572101420638\\
583.01	0.00123474560536223\\
584.01	0.00116291293976363\\
585.01	0.00109024350277839\\
586.01	0.00101676239875393\\
587.01	0.000942499671248528\\
588.01	0.00086749056337539\\
589.01	0.000791775698818989\\
590.01	0.000715401150971457\\
591.01	0.000638418341011752\\
592.01	0.000560883678765979\\
593.01	0.000482857857459612\\
594.01	0.000404404717382021\\
595.01	0.000325589587232982\\
596.01	0.000246477000692876\\
597.01	0.000167127667889054\\
598.01	9.13379822583427e-05\\
599.01	2.91271958431007e-05\\
599.02	2.86192783682145e-05\\
599.03	2.81144237470974e-05\\
599.04	2.76126618026333e-05\\
599.05	2.71140226516964e-05\\
599.06	2.6618536708066e-05\\
599.07	2.61262346853714e-05\\
599.08	2.56371476000505e-05\\
599.09	2.51513067743691e-05\\
599.1	2.46687438394382e-05\\
599.11	2.41894907382703e-05\\
599.12	2.37135797288915e-05\\
599.13	2.32410433874414e-05\\
599.14	2.27719146113341e-05\\
599.15	2.23062266224462e-05\\
599.16	2.18440129703245e-05\\
599.17	2.13853075354369e-05\\
599.18	2.09301445324543e-05\\
599.19	2.04785585135574e-05\\
599.2	2.00305843717826e-05\\
599.21	1.95862573444047e-05\\
599.22	1.9145613016339e-05\\
599.23	1.87086873236e-05\\
599.24	1.82755165567643e-05\\
599.25	1.78461373645019e-05\\
599.26	1.74205874172526e-05\\
599.27	1.69989072024649e-05\\
599.28	1.65811376108427e-05\\
599.29	1.61673199403477e-05\\
599.3	1.57574959002166e-05\\
599.31	1.53517076150536e-05\\
599.32	1.49499976289241e-05\\
599.33	1.45524089095253e-05\\
599.34	1.41589848523631e-05\\
599.35	1.37697692849988e-05\\
599.36	1.33848064713289e-05\\
599.37	1.30041411158955e-05\\
599.38	1.2627818368258e-05\\
599.39	1.22558838273942e-05\\
599.4	1.18883835461461e-05\\
599.41	1.15253640357253e-05\\
599.42	1.11668722702386e-05\\
599.43	1.08129556912751e-05\\
599.44	1.04636622125405e-05\\
599.45	1.01190402245176e-05\\
599.46	9.77913859919952e-06\\
599.47	9.44400669484861e-06\\
599.48	9.11369436081e-06\\
599.49	8.78825194237758e-06\\
599.5	8.46773028570155e-06\\
599.51	8.15218074274454e-06\\
599.52	7.84165517629144e-06\\
599.53	7.53620596500443e-06\\
599.54	7.23588600852476e-06\\
599.55	6.94074873264228e-06\\
599.56	6.65084809449087e-06\\
599.57	6.36623858781861e-06\\
599.58	6.08697524828207e-06\\
599.59	5.81311365883443e-06\\
599.6	5.54470995512042e-06\\
599.61	5.2818208309633e-06\\
599.62	5.02450354387951e-06\\
599.63	4.77281592066101e-06\\
599.64	4.52681636300793e-06\\
599.65	4.28656385322544e-06\\
599.66	4.05211795996216e-06\\
599.67	3.82353884401804e-06\\
599.68	3.60088726420252e-06\\
599.69	3.38422458325514e-06\\
599.7	3.17361277382341e-06\\
599.71	2.96911442449442e-06\\
599.72	2.77079274589413e-06\\
599.73	2.5787115768474e-06\\
599.74	2.3929353905848e-06\\
599.75	2.213529301031e-06\\
599.76	2.04055906913823e-06\\
599.77	1.87409110929959e-06\\
599.78	1.71419249580043e-06\\
599.79	1.56093096936177e-06\\
599.8	1.41437494372877e-06\\
599.81	1.27459351233379e-06\\
599.82	1.14165645502366e-06\\
599.83	1.01563424484592e-06\\
599.84	8.96598054920053e-07\\
599.85	7.84619765348618e-07\\
599.86	6.79771970232487e-07\\
599.87	5.82127984717282e-07\\
599.88	4.91761852147374e-07\\
599.89	4.08748351252847e-07\\
599.9	3.33163003437068e-07\\
599.91	2.65082080131915e-07\\
599.92	2.04582610201579e-07\\
599.93	1.51742387450443e-07\\
599.94	1.06639978189951e-07\\
599.95	6.93547288800611e-08\\
599.96	3.99667738487652e-08\\
599.97	1.85570430896731e-08\\
599.98	5.20727013418598e-09\\
599.99	0\\
600	0\\
};
\addplot [color=blue!75!mycolor7,solid,forget plot]
  table[row sep=crcr]{%
0.01	0.00902214729946002\\
1.01	0.00902214636557786\\
2.01	0.0090221454118588\\
3.01	0.00902214443787868\\
4.01	0.00902214344320444\\
5.01	0.00902214242739349\\
6.01	0.00902214138999379\\
7.01	0.00902214033054357\\
8.01	0.00902213924857109\\
9.01	0.00902213814359446\\
10.01	0.00902213701512133\\
11.01	0.00902213586264874\\
12.01	0.00902213468566294\\
13.01	0.00902213348363904\\
14.01	0.00902213225604088\\
15.01	0.00902213100232068\\
16.01	0.0090221297219188\\
17.01	0.00902212841426364\\
18.01	0.0090221270787711\\
19.01	0.00902212571484459\\
20.01	0.0090221243218745\\
21.01	0.00902212289923817\\
22.01	0.00902212144629944\\
23.01	0.00902211996240837\\
24.01	0.00902211844690088\\
25.01	0.00902211689909873\\
26.01	0.0090221153183088\\
27.01	0.00902211370382313\\
28.01	0.00902211205491836\\
29.01	0.00902211037085547\\
30.01	0.00902210865087943\\
31.01	0.00902210689421885\\
32.01	0.00902210510008571\\
33.01	0.00902210326767487\\
34.01	0.0090221013961636\\
35.01	0.00902209948471151\\
36.01	0.00902209753245992\\
37.01	0.00902209553853158\\
38.01	0.00902209350203016\\
39.01	0.00902209142203981\\
40.01	0.00902208929762494\\
41.01	0.00902208712782956\\
42.01	0.00902208491167694\\
43.01	0.00902208264816917\\
44.01	0.00902208033628665\\
45.01	0.00902207797498759\\
46.01	0.00902207556320759\\
47.01	0.00902207309985908\\
48.01	0.00902207058383085\\
49.01	0.00902206801398751\\
50.01	0.00902206538916903\\
51.01	0.00902206270819004\\
52.01	0.00902205996983952\\
53.01	0.00902205717287987\\
54.01	0.00902205431604674\\
55.01	0.00902205139804815\\
56.01	0.00902204841756396\\
57.01	0.00902204537324524\\
58.01	0.00902204226371373\\
59.01	0.00902203908756107\\
60.01	0.00902203584334826\\
61.01	0.00902203252960478\\
62.01	0.00902202914482812\\
63.01	0.0090220256874829\\
64.01	0.00902202215600024\\
65.01	0.00902201854877709\\
66.01	0.00902201486417526\\
67.01	0.00902201110052087\\
68.01	0.00902200725610342\\
69.01	0.00902200332917496\\
70.01	0.00902199931794933\\
71.01	0.00902199522060141\\
72.01	0.00902199103526603\\
73.01	0.00902198676003726\\
74.01	0.00902198239296747\\
75.01	0.00902197793206631\\
76.01	0.00902197337529987\\
77.01	0.00902196872058973\\
78.01	0.00902196396581186\\
79.01	0.00902195910879569\\
80.01	0.00902195414732309\\
81.01	0.00902194907912724\\
82.01	0.00902194390189153\\
83.01	0.00902193861324868\\
84.01	0.00902193321077929\\
85.01	0.00902192769201089\\
86.01	0.00902192205441676\\
87.01	0.00902191629541448\\
88.01	0.0090219104123652\\
89.01	0.00902190440257177\\
90.01	0.00902189826327783\\
91.01	0.00902189199166643\\
92.01	0.00902188558485854\\
93.01	0.00902187903991186\\
94.01	0.00902187235381927\\
95.01	0.00902186552350747\\
96.01	0.00902185854583548\\
97.01	0.00902185141759291\\
98.01	0.00902184413549887\\
99.01	0.00902183669620003\\
100.01	0.00902182909626898\\
101.01	0.0090218213322029\\
102.01	0.00902181340042147\\
103.01	0.00902180529726546\\
104.01	0.0090217970189948\\
105.01	0.00902178856178685\\
106.01	0.00902177992173439\\
107.01	0.00902177109484386\\
108.01	0.00902176207703357\\
109.01	0.00902175286413132\\
110.01	0.00902174345187275\\
111.01	0.00902173383589908\\
112.01	0.00902172401175507\\
113.01	0.00902171397488669\\
114.01	0.0090217037206392\\
115.01	0.00902169324425448\\
116.01	0.00902168254086924\\
117.01	0.00902167160551206\\
118.01	0.00902166043310138\\
119.01	0.00902164901844296\\
120.01	0.00902163735622711\\
121.01	0.00902162544102633\\
122.01	0.00902161326729258\\
123.01	0.00902160082935447\\
124.01	0.00902158812141463\\
125.01	0.00902157513754676\\
126.01	0.00902156187169281\\
127.01	0.00902154831765991\\
128.01	0.00902153446911727\\
129.01	0.0090215203195933\\
130.01	0.00902150586247225\\
131.01	0.00902149109099095\\
132.01	0.00902147599823548\\
133.01	0.00902146057713779\\
134.01	0.00902144482047227\\
135.01	0.00902142872085208\\
136.01	0.00902141227072544\\
137.01	0.00902139546237215\\
138.01	0.0090213782878994\\
139.01	0.00902136073923823\\
140.01	0.00902134280813927\\
141.01	0.00902132448616875\\
142.01	0.00902130576470425\\
143.01	0.00902128663493054\\
144.01	0.00902126708783508\\
145.01	0.00902124711420357\\
146.01	0.00902122670461531\\
147.01	0.00902120584943863\\
148.01	0.00902118453882594\\
149.01	0.0090211627627089\\
150.01	0.00902114051079341\\
151.01	0.00902111777255441\\
152.01	0.00902109453723066\\
153.01	0.0090210707938193\\
154.01	0.00902104653107041\\
155.01	0.0090210217374813\\
156.01	0.00902099640129076\\
157.01	0.00902097051047316\\
158.01	0.00902094405273254\\
159.01	0.0090209170154961\\
160.01	0.00902088938590829\\
161.01	0.0090208611508241\\
162.01	0.00902083229680238\\
163.01	0.00902080281009923\\
164.01	0.00902077267666104\\
165.01	0.00902074188211725\\
166.01	0.00902071041177333\\
167.01	0.00902067825060315\\
168.01	0.00902064538324151\\
169.01	0.00902061179397631\\
170.01	0.00902057746674059\\
171.01	0.00902054238510434\\
172.01	0.00902050653226642\\
173.01	0.00902046989104571\\
174.01	0.00902043244387255\\
175.01	0.00902039417277981\\
176.01	0.0090203550593938\\
177.01	0.00902031508492479\\
178.01	0.00902027423015762\\
179.01	0.00902023247544173\\
180.01	0.0090201898006814\\
181.01	0.00902014618532521\\
182.01	0.00902010160835585\\
183.01	0.00902005604827914\\
184.01	0.00902000948311318\\
185.01	0.00901996189037716\\
186.01	0.00901991324707981\\
187.01	0.00901986352970765\\
188.01	0.00901981271421294\\
189.01	0.00901976077600127\\
190.01	0.00901970768991928\\
191.01	0.00901965343024152\\
192.01	0.00901959797065711\\
193.01	0.00901954128425669\\
194.01	0.00901948334351816\\
195.01	0.00901942412029288\\
196.01	0.00901936358579104\\
197.01	0.0090193017105668\\
198.01	0.00901923846450342\\
199.01	0.00901917381679741\\
200.01	0.00901910773594297\\
201.01	0.00901904018971561\\
202.01	0.00901897114515559\\
203.01	0.00901890056855102\\
204.01	0.00901882842542045\\
205.01	0.00901875468049491\\
206.01	0.00901867929769993\\
207.01	0.00901860224013693\\
208.01	0.00901852347006404\\
209.01	0.00901844294887678\\
210.01	0.00901836063708794\\
211.01	0.00901827649430733\\
212.01	0.00901819047922082\\
213.01	0.00901810254956911\\
214.01	0.00901801266212574\\
215.01	0.00901792077267485\\
216.01	0.00901782683598827\\
217.01	0.00901773080580212\\
218.01	0.00901763263479289\\
219.01	0.00901753227455295\\
220.01	0.00901742967556563\\
221.01	0.00901732478717937\\
222.01	0.00901721755758161\\
223.01	0.00901710793377213\\
224.01	0.0090169958615353\\
225.01	0.00901688128541232\\
226.01	0.00901676414867232\\
227.01	0.0090166443932831\\
228.01	0.00901652195988103\\
229.01	0.00901639678774042\\
230.01	0.00901626881474201\\
231.01	0.00901613797734075\\
232.01	0.00901600421053311\\
233.01	0.00901586744782334\\
234.01	0.00901572762118908\\
235.01	0.00901558466104616\\
236.01	0.00901543849621265\\
237.01	0.00901528905387199\\
238.01	0.00901513625953546\\
239.01	0.0090149800370036\\
240.01	0.00901482030832685\\
241.01	0.00901465699376532\\
242.01	0.00901449001174758\\
243.01	0.00901431927882856\\
244.01	0.00901414470964644\\
245.01	0.00901396621687866\\
246.01	0.00901378371119698\\
247.01	0.00901359710122127\\
248.01	0.00901340629347245\\
249.01	0.00901321119232442\\
250.01	0.00901301169995473\\
251.01	0.00901280771629435\\
252.01	0.00901259913897617\\
253.01	0.00901238586328219\\
254.01	0.00901216778208983\\
255.01	0.00901194478581698\\
256.01	0.00901171676236574\\
257.01	0.00901148359706464\\
258.01	0.00901124517261017\\
259.01	0.00901100136900639\\
260.01	0.00901075206350364\\
261.01	0.00901049713053553\\
262.01	0.00901023644165471\\
263.01	0.00900996986546737\\
264.01	0.00900969726756579\\
265.01	0.00900941851045983\\
266.01	0.00900913345350662\\
267.01	0.00900884195283892\\
268.01	0.00900854386129167\\
269.01	0.00900823902832682\\
270.01	0.00900792729995694\\
271.01	0.00900760851866644\\
272.01	0.00900728252333164\\
273.01	0.00900694914913861\\
274.01	0.00900660822749957\\
275.01	0.0090062595859669\\
276.01	0.00900590304814578\\
277.01	0.00900553843360438\\
278.01	0.0090051655577822\\
279.01	0.00900478423189671\\
280.01	0.00900439426284703\\
281.01	0.00900399545311629\\
282.01	0.00900358760067128\\
283.01	0.00900317049885994\\
284.01	0.00900274393630672\\
285.01	0.00900230769680487\\
286.01	0.00900186155920724\\
287.01	0.00900140529731369\\
288.01	0.00900093867975641\\
289.01	0.00900046146988245\\
290.01	0.00899997342563339\\
291.01	0.00899947429942243\\
292.01	0.00899896383800842\\
293.01	0.00899844178236686\\
294.01	0.00899790786755827\\
295.01	0.00899736182259278\\
296.01	0.00899680337029201\\
297.01	0.00899623222714744\\
298.01	0.00899564810317535\\
299.01	0.00899505070176827\\
300.01	0.0089944397195428\\
301.01	0.00899381484618379\\
302.01	0.00899317576428434\\
303.01	0.00899252214918232\\
304.01	0.00899185366879247\\
305.01	0.00899116998343422\\
306.01	0.0089904707456555\\
307.01	0.00898975560005174\\
308.01	0.00898902418308034\\
309.01	0.00898827612287043\\
310.01	0.00898751103902754\\
311.01	0.00898672854243331\\
312.01	0.00898592823503992\\
313.01	0.0089851097096588\\
314.01	0.00898427254974418\\
315.01	0.00898341632917067\\
316.01	0.00898254061200452\\
317.01	0.0089816449522693\\
318.01	0.00898072889370453\\
319.01	0.00897979196951827\\
320.01	0.00897883370213262\\
321.01	0.00897785360292217\\
322.01	0.00897685117194539\\
323.01	0.00897582589766829\\
324.01	0.00897477725668059\\
325.01	0.00897370471340389\\
326.01	0.00897260771979137\\
327.01	0.0089714857150195\\
328.01	0.00897033812517077\\
329.01	0.00896916436290742\\
330.01	0.00896796382713621\\
331.01	0.00896673590266359\\
332.01	0.00896547995984101\\
333.01	0.00896419535420055\\
334.01	0.00896288142608007\\
335.01	0.00896153750023806\\
336.01	0.00896016288545745\\
337.01	0.00895875687413877\\
338.01	0.00895731874188163\\
339.01	0.00895584774705512\\
340.01	0.00895434313035604\\
341.01	0.00895280411435532\\
342.01	0.00895122990303212\\
343.01	0.00894961968129542\\
344.01	0.00894797261449294\\
345.01	0.00894628784790683\\
346.01	0.00894456450623659\\
347.01	0.00894280169306797\\
348.01	0.00894099849032874\\
349.01	0.00893915395773019\\
350.01	0.00893726713219465\\
351.01	0.00893533702726874\\
352.01	0.0089333626325217\\
353.01	0.00893134291292928\\
354.01	0.00892927680824225\\
355.01	0.00892716323233978\\
356.01	0.00892500107256717\\
357.01	0.00892278918905757\\
358.01	0.00892052641403791\\
359.01	0.00891821155111816\\
360.01	0.00891584337456376\\
361.01	0.00891342062855118\\
362.01	0.00891094202640624\\
363.01	0.00890840624982432\\
364.01	0.0089058119480727\\
365.01	0.00890315773717442\\
366.01	0.00890044219907318\\
367.01	0.00889766388077904\\
368.01	0.00889482129349435\\
369.01	0.00889191291171938\\
370.01	0.00888893717233722\\
371.01	0.00888589247367732\\
372.01	0.008882777174557\\
373.01	0.00887958959330008\\
374.01	0.00887632800673178\\
375.01	0.0088729906491486\\
376.01	0.00886957571126188\\
377.01	0.00886608133911376\\
378.01	0.00886250563296292\\
379.01	0.00885884664613867\\
380.01	0.00885510238386041\\
381.01	0.00885127080202006\\
382.01	0.00884734980592478\\
383.01	0.00884333724899739\\
384.01	0.00883923093143278\\
385.01	0.00883502859880898\\
386.01	0.00883072794065253\\
387.01	0.0088263265889584\\
388.01	0.00882182211666361\\
389.01	0.00881721203607317\\
390.01	0.00881249379723585\\
391.01	0.00880766478626766\\
392.01	0.00880272232362082\\
393.01	0.00879766366229554\\
394.01	0.00879248598599183\\
395.01	0.0087871864071987\\
396.01	0.00878176196521772\\
397.01	0.00877620962411707\\
398.01	0.00877052627061375\\
399.01	0.00876470871187895\\
400.01	0.00875875367326385\\
401.01	0.00875265779594081\\
402.01	0.00874641763445592\\
403.01	0.0087400296541879\\
404.01	0.0087334902287085\\
405.01	0.00872679563703865\\
406.01	0.00871994206079482\\
407.01	0.00871292558121942\\
408.01	0.00870574217608834\\
409.01	0.00869838771648888\\
410.01	0.00869085796346007\\
411.01	0.00868314856448815\\
412.01	0.00867525504984742\\
413.01	0.00866717282877817\\
414.01	0.00865889718549119\\
415.01	0.00865042327498873\\
416.01	0.00864174611869009\\
417.01	0.00863286059984992\\
418.01	0.00862376145875623\\
419.01	0.00861444328769367\\
420.01	0.00860490052565739\\
421.01	0.00859512745280099\\
422.01	0.00858511818460098\\
423.01	0.00857486666571896\\
424.01	0.00856436666354078\\
425.01	0.00855361176137078\\
426.01	0.00854259535125666\\
427.01	0.00853131062641902\\
428.01	0.0085197505732563\\
429.01	0.00850790796289468\\
430.01	0.00849577534224728\\
431.01	0.00848334502454542\\
432.01	0.00847060907929983\\
433.01	0.00845755932164421\\
434.01	0.00844418730100998\\
435.01	0.00843048428907305\\
436.01	0.00841644126690722\\
437.01	0.00840204891126913\\
438.01	0.00838729757993086\\
439.01	0.00837217729596297\\
440.01	0.00835667773085732\\
441.01	0.00834078818636127\\
442.01	0.00832449757487602\\
443.01	0.00830779439824565\\
444.01	0.00829066672473585\\
445.01	0.0082731021639658\\
446.01	0.00825508783951468\\
447.01	0.00823661035887356\\
448.01	0.00821765578035209\\
449.01	0.00819820957647621\\
450.01	0.00817825659332246\\
451.01	0.00815778100512831\\
452.01	0.0081367662633852\\
453.01	0.00811519503946528\\
454.01	0.00809304915964056\\
455.01	0.0080703095311234\\
456.01	0.00804695605748009\\
457.01	0.00802296754143559\\
458.01	0.0079983215726893\\
459.01	0.00797299439789086\\
460.01	0.00794696076938257\\
461.01	0.00792019376872971\\
462.01	0.00789266460054664\\
463.01	0.00786434235207658\\
464.01	0.00783519371563762\\
465.01	0.00780518267839087\\
466.01	0.00777427020982304\\
467.01	0.0077424140646549\\
468.01	0.0077095690915114\\
469.01	0.00767568632242102\\
470.01	0.00764070929011195\\
471.01	0.00760457477695843\\
472.01	0.00756721198221674\\
473.01	0.00752854110038599\\
474.01	0.00748847156388789\\
475.01	0.00744689985044509\\
476.01	0.00740370670259724\\
477.01	0.00735875348263947\\
478.01	0.00731187703838316\\
479.01	0.0072628837925216\\
480.01	0.00721154725105092\\
481.01	0.00718190514624087\\
482.01	0.00716230954357111\\
483.01	0.00714175761399098\\
484.01	0.00712013013151004\\
485.01	0.00709728208902563\\
486.01	0.0070730359896628\\
487.01	0.00704717323628954\\
488.01	0.00701942304826762\\
489.01	0.0069894481553389\\
490.01	0.00695682628098655\\
491.01	0.00692220285691665\\
492.01	0.00688667793783651\\
493.01	0.00685023293560006\\
494.01	0.00681284823645732\\
495.01	0.00677450416666226\\
496.01	0.00673518085100826\\
497.01	0.00669485795942825\\
498.01	0.00665351428663556\\
499.01	0.00661112708481027\\
500.01	0.0065676710391065\\
501.01	0.00652311675181007\\
502.01	0.00647742855102759\\
503.01	0.00643056136511647\\
504.01	0.00638245630966505\\
505.01	0.00633303927685509\\
506.01	0.00628226434297923\\
507.01	0.00623011202621469\\
508.01	0.00617656872138363\\
509.01	0.00612162812908867\\
510.01	0.00606529346721341\\
511.01	0.00600758029255231\\
512.01	0.00594852011353755\\
513.01	0.00588816503327066\\
514.01	0.0058265937402637\\
515.01	0.00576391926959179\\
516.01	0.00570029909933447\\
517.01	0.00563594834130146\\
518.01	0.00557115711633265\\
519.01	0.00550631321884673\\
520.01	0.00544193099400693\\
521.01	0.00537868871517858\\
522.01	0.0053174772958715\\
523.01	0.00525946392137587\\
524.01	0.00520531260447456\\
525.01	0.00515224219814917\\
526.01	0.00509900084438186\\
527.01	0.00504430539160467\\
528.01	0.004988157377622\\
529.01	0.00493056553609459\\
530.01	0.00487154406794097\\
531.01	0.00481110992258657\\
532.01	0.00474927598110803\\
533.01	0.00468603582532421\\
534.01	0.00462137786818744\\
535.01	0.00455529471347772\\
536.01	0.00448778347953893\\
537.01	0.00441884646204981\\
538.01	0.00434849204940306\\
539.01	0.00427673591926868\\
540.01	0.004203602664043\\
541.01	0.00412912804179831\\
542.01	0.00405336103587868\\
543.01	0.003976359182936\\
544.01	0.00389830370161603\\
545.01	0.0038200553540577\\
546.01	0.00374193698585412\\
547.01	0.0036641755998312\\
548.01	0.00358700119398903\\
549.01	0.00351062917203564\\
550.01	0.00343526607327019\\
551.01	0.00336120054192128\\
552.01	0.00328872981598254\\
553.01	0.00321810286178094\\
554.01	0.00314946895573235\\
555.01	0.00308243132699263\\
556.01	0.00301600563065513\\
557.01	0.00295019309929753\\
558.01	0.0028850146013036\\
559.01	0.00282046134724498\\
560.01	0.00275650087514034\\
561.01	0.00269307732904906\\
562.01	0.00263008779722273\\
563.01	0.00256737583515977\\
564.01	0.00250474150326022\\
565.01	0.00244200987170124\\
566.01	0.00237912602438769\\
567.01	0.00231604585404112\\
568.01	0.00225271645253628\\
569.01	0.00218907572873699\\
570.01	0.00212505471491989\\
571.01	0.00206058414873081\\
572.01	0.00199560260247596\\
573.01	0.00193006437600999\\
574.01	0.00186394204314159\\
575.01	0.00179721456938685\\
576.01	0.00172986002512933\\
577.01	0.00166185690585053\\
578.01	0.0015931860350652\\
579.01	0.00152383235677303\\
580.01	0.00145378623821854\\
581.01	0.00138304383207852\\
582.01	0.00131160612454736\\
583.01	0.00123947717929371\\
584.01	0.00116666351012589\\
585.01	0.0010931744009037\\
586.01	0.00101902232302626\\
587.01	0.000944223407204438\\
588.01	0.000868797861908435\\
589.01	0.000792770299867063\\
590.01	0.000716170051933755\\
591.01	0.000639031578739221\\
592.01	0.000561394895303546\\
593.01	0.000483305700085252\\
594.01	0.000404814972475517\\
595.01	0.000325977840673588\\
596.01	0.000246851522619258\\
597.01	0.000167492162777875\\
598.01	9.13380205773313e-05\\
599.01	2.91271962973242e-05\\
599.02	2.86192787954821e-05\\
599.03	2.81144241487431e-05\\
599.04	2.76126621799409e-05\\
599.05	2.7114023005901e-05\\
599.06	2.6618537040354e-05\\
599.07	2.61262349968722e-05\\
599.08	2.5637147891857e-05\\
599.09	2.51513070475221e-05\\
599.1	2.466874409493e-05\\
599.11	2.41894909770619e-05\\
599.12	2.37135799518919e-05\\
599.13	2.32410435955233e-05\\
599.14	2.27719148053352e-05\\
599.15	2.2306226803161e-05\\
599.16	2.18440131385146e-05\\
599.17	2.13853076918291e-05\\
599.18	2.09301446777409e-05\\
599.19	2.04785586483958e-05\\
599.2	2.00305844968024e-05\\
599.21	1.95862574602045e-05\\
599.22	1.91456131234894e-05\\
599.23	1.87086874226389e-05\\
599.24	1.82755166482085e-05\\
599.25	1.78461374488355e-05\\
599.26	1.74205874949405e-05\\
599.27	1.69989072739424e-05\\
599.28	1.65811376765281e-05\\
599.29	1.61673200006276e-05\\
599.3	1.57574959554676e-05\\
599.31	1.53517076656243e-05\\
599.32	1.49499976751493e-05\\
599.33	1.4552408951712e-05\\
599.34	1.4158984890808e-05\\
599.35	1.3769769319983e-05\\
599.36	1.33848065031108e-05\\
599.37	1.30041411447214e-05\\
599.38	1.26278183943569e-05\\
599.39	1.22558838509795e-05\\
599.4	1.18883835674242e-05\\
599.41	1.15253640548835e-05\\
599.42	1.1166872287454e-05\\
599.43	1.08129557067158e-05\\
599.44	1.04636622263576e-05\\
599.45	1.01190402368567e-05\\
599.46	9.77913861019246e-06\\
599.47	9.44400670461684e-06\\
599.48	9.11369436946974e-06\\
599.49	8.78825195003465e-06\\
599.5	8.4677302924531e-06\\
599.51	8.1521807486825e-06\\
599.52	7.84165518149561e-06\\
599.53	7.53620596955114e-06\\
599.54	7.23588601248687e-06\\
599.55	6.94074873608223e-06\\
599.56	6.65084809746766e-06\\
599.57	6.3662385903808e-06\\
599.58	6.0869752504817e-06\\
599.59	5.81311366071487e-06\\
599.6	5.54470995672157e-06\\
599.61	5.28182083231986e-06\\
599.62	5.02450354502269e-06\\
599.63	4.77281592161857e-06\\
599.64	4.52681636380764e-06\\
599.65	4.28656385388811e-06\\
599.66	4.0521179605086e-06\\
599.67	3.8235388444656e-06\\
599.68	3.60088726456508e-06\\
599.69	3.38422458354831e-06\\
599.7	3.1736127740576e-06\\
599.71	2.96911442468004e-06\\
599.72	2.77079274604158e-06\\
599.73	2.57871157696016e-06\\
599.74	2.39293539067154e-06\\
599.75	2.21352930109692e-06\\
599.76	2.04055906919028e-06\\
599.77	1.87409110933776e-06\\
599.78	1.71419249582645e-06\\
599.79	1.56093096938086e-06\\
599.8	1.41437494374265e-06\\
599.81	1.2745935123442e-06\\
599.82	1.1416564550306e-06\\
599.83	1.01563424485286e-06\\
599.84	8.96598054921788e-07\\
599.85	7.84619765352088e-07\\
599.86	6.79771970232487e-07\\
599.87	5.82127984719016e-07\\
599.88	4.91761852147374e-07\\
599.89	4.08748351251112e-07\\
599.9	3.33163003438802e-07\\
599.91	2.65082080131915e-07\\
599.92	2.04582610201579e-07\\
599.93	1.51742387452178e-07\\
599.94	1.06639978191686e-07\\
599.95	6.93547288817958e-08\\
599.96	3.99667738487652e-08\\
599.97	1.85570430914078e-08\\
599.98	5.20727013418598e-09\\
599.99	0\\
600	0\\
};
\addplot [color=blue!80!mycolor9,solid,forget plot]
  table[row sep=crcr]{%
0.01	0.01\\
1.01	0.01\\
2.01	0.01\\
3.01	0.01\\
4.01	0.01\\
5.01	0.01\\
6.01	0.01\\
7.01	0.01\\
8.01	0.01\\
9.01	0.01\\
10.01	0.01\\
11.01	0.01\\
12.01	0.01\\
13.01	0.01\\
14.01	0.01\\
15.01	0.01\\
16.01	0.01\\
17.01	0.01\\
18.01	0.01\\
19.01	0.01\\
20.01	0.01\\
21.01	0.01\\
22.01	0.01\\
23.01	0.01\\
24.01	0.01\\
25.01	0.01\\
26.01	0.01\\
27.01	0.01\\
28.01	0.01\\
29.01	0.01\\
30.01	0.01\\
31.01	0.01\\
32.01	0.01\\
33.01	0.01\\
34.01	0.01\\
35.01	0.01\\
36.01	0.01\\
37.01	0.01\\
38.01	0.01\\
39.01	0.01\\
40.01	0.01\\
41.01	0.01\\
42.01	0.01\\
43.01	0.01\\
44.01	0.01\\
45.01	0.01\\
46.01	0.01\\
47.01	0.01\\
48.01	0.01\\
49.01	0.01\\
50.01	0.01\\
51.01	0.01\\
52.01	0.01\\
53.01	0.01\\
54.01	0.01\\
55.01	0.01\\
56.01	0.01\\
57.01	0.01\\
58.01	0.01\\
59.01	0.01\\
60.01	0.01\\
61.01	0.01\\
62.01	0.01\\
63.01	0.01\\
64.01	0.01\\
65.01	0.01\\
66.01	0.01\\
67.01	0.01\\
68.01	0.01\\
69.01	0.01\\
70.01	0.01\\
71.01	0.01\\
72.01	0.01\\
73.01	0.01\\
74.01	0.01\\
75.01	0.01\\
76.01	0.01\\
77.01	0.01\\
78.01	0.01\\
79.01	0.01\\
80.01	0.01\\
81.01	0.01\\
82.01	0.01\\
83.01	0.01\\
84.01	0.01\\
85.01	0.01\\
86.01	0.01\\
87.01	0.01\\
88.01	0.01\\
89.01	0.01\\
90.01	0.01\\
91.01	0.01\\
92.01	0.01\\
93.01	0.01\\
94.01	0.01\\
95.01	0.01\\
96.01	0.01\\
97.01	0.01\\
98.01	0.01\\
99.01	0.01\\
100.01	0.01\\
101.01	0.01\\
102.01	0.01\\
103.01	0.01\\
104.01	0.01\\
105.01	0.01\\
106.01	0.01\\
107.01	0.01\\
108.01	0.01\\
109.01	0.01\\
110.01	0.01\\
111.01	0.01\\
112.01	0.01\\
113.01	0.01\\
114.01	0.01\\
115.01	0.01\\
116.01	0.01\\
117.01	0.01\\
118.01	0.01\\
119.01	0.01\\
120.01	0.01\\
121.01	0.01\\
122.01	0.01\\
123.01	0.01\\
124.01	0.01\\
125.01	0.01\\
126.01	0.01\\
127.01	0.01\\
128.01	0.01\\
129.01	0.01\\
130.01	0.01\\
131.01	0.01\\
132.01	0.01\\
133.01	0.01\\
134.01	0.01\\
135.01	0.01\\
136.01	0.01\\
137.01	0.01\\
138.01	0.01\\
139.01	0.01\\
140.01	0.01\\
141.01	0.01\\
142.01	0.01\\
143.01	0.01\\
144.01	0.01\\
145.01	0.01\\
146.01	0.01\\
147.01	0.01\\
148.01	0.01\\
149.01	0.01\\
150.01	0.01\\
151.01	0.01\\
152.01	0.01\\
153.01	0.01\\
154.01	0.01\\
155.01	0.01\\
156.01	0.01\\
157.01	0.01\\
158.01	0.01\\
159.01	0.01\\
160.01	0.01\\
161.01	0.01\\
162.01	0.01\\
163.01	0.01\\
164.01	0.01\\
165.01	0.01\\
166.01	0.01\\
167.01	0.01\\
168.01	0.01\\
169.01	0.01\\
170.01	0.01\\
171.01	0.01\\
172.01	0.01\\
173.01	0.01\\
174.01	0.01\\
175.01	0.01\\
176.01	0.01\\
177.01	0.01\\
178.01	0.01\\
179.01	0.01\\
180.01	0.01\\
181.01	0.01\\
182.01	0.01\\
183.01	0.01\\
184.01	0.01\\
185.01	0.01\\
186.01	0.01\\
187.01	0.01\\
188.01	0.01\\
189.01	0.01\\
190.01	0.01\\
191.01	0.01\\
192.01	0.01\\
193.01	0.01\\
194.01	0.01\\
195.01	0.01\\
196.01	0.01\\
197.01	0.01\\
198.01	0.01\\
199.01	0.01\\
200.01	0.01\\
201.01	0.01\\
202.01	0.01\\
203.01	0.01\\
204.01	0.01\\
205.01	0.01\\
206.01	0.01\\
207.01	0.01\\
208.01	0.01\\
209.01	0.01\\
210.01	0.01\\
211.01	0.01\\
212.01	0.01\\
213.01	0.01\\
214.01	0.01\\
215.01	0.01\\
216.01	0.01\\
217.01	0.01\\
218.01	0.01\\
219.01	0.01\\
220.01	0.01\\
221.01	0.01\\
222.01	0.01\\
223.01	0.01\\
224.01	0.01\\
225.01	0.01\\
226.01	0.01\\
227.01	0.01\\
228.01	0.01\\
229.01	0.01\\
230.01	0.01\\
231.01	0.01\\
232.01	0.01\\
233.01	0.01\\
234.01	0.01\\
235.01	0.01\\
236.01	0.01\\
237.01	0.01\\
238.01	0.01\\
239.01	0.01\\
240.01	0.01\\
241.01	0.01\\
242.01	0.01\\
243.01	0.01\\
244.01	0.01\\
245.01	0.01\\
246.01	0.01\\
247.01	0.01\\
248.01	0.01\\
249.01	0.01\\
250.01	0.01\\
251.01	0.01\\
252.01	0.01\\
253.01	0.01\\
254.01	0.01\\
255.01	0.01\\
256.01	0.01\\
257.01	0.01\\
258.01	0.01\\
259.01	0.01\\
260.01	0.01\\
261.01	0.01\\
262.01	0.01\\
263.01	0.01\\
264.01	0.01\\
265.01	0.01\\
266.01	0.01\\
267.01	0.01\\
268.01	0.01\\
269.01	0.01\\
270.01	0.01\\
271.01	0.01\\
272.01	0.01\\
273.01	0.01\\
274.01	0.01\\
275.01	0.01\\
276.01	0.01\\
277.01	0.01\\
278.01	0.01\\
279.01	0.01\\
280.01	0.01\\
281.01	0.01\\
282.01	0.01\\
283.01	0.01\\
284.01	0.01\\
285.01	0.01\\
286.01	0.01\\
287.01	0.01\\
288.01	0.01\\
289.01	0.01\\
290.01	0.01\\
291.01	0.01\\
292.01	0.01\\
293.01	0.01\\
294.01	0.01\\
295.01	0.01\\
296.01	0.01\\
297.01	0.01\\
298.01	0.01\\
299.01	0.01\\
300.01	0.01\\
301.01	0.01\\
302.01	0.01\\
303.01	0.01\\
304.01	0.01\\
305.01	0.01\\
306.01	0.01\\
307.01	0.01\\
308.01	0.01\\
309.01	0.01\\
310.01	0.01\\
311.01	0.01\\
312.01	0.01\\
313.01	0.01\\
314.01	0.01\\
315.01	0.01\\
316.01	0.01\\
317.01	0.01\\
318.01	0.01\\
319.01	0.01\\
320.01	0.01\\
321.01	0.01\\
322.01	0.01\\
323.01	0.01\\
324.01	0.01\\
325.01	0.01\\
326.01	0.01\\
327.01	0.01\\
328.01	0.01\\
329.01	0.01\\
330.01	0.01\\
331.01	0.01\\
332.01	0.01\\
333.01	0.01\\
334.01	0.01\\
335.01	0.01\\
336.01	0.01\\
337.01	0.01\\
338.01	0.01\\
339.01	0.01\\
340.01	0.01\\
341.01	0.01\\
342.01	0.01\\
343.01	0.01\\
344.01	0.01\\
345.01	0.01\\
346.01	0.01\\
347.01	0.01\\
348.01	0.01\\
349.01	0.01\\
350.01	0.01\\
351.01	0.01\\
352.01	0.01\\
353.01	0.01\\
354.01	0.01\\
355.01	0.01\\
356.01	0.01\\
357.01	0.01\\
358.01	0.01\\
359.01	0.01\\
360.01	0.01\\
361.01	0.01\\
362.01	0.01\\
363.01	0.01\\
364.01	0.01\\
365.01	0.01\\
366.01	0.01\\
367.01	0.01\\
368.01	0.01\\
369.01	0.01\\
370.01	0.01\\
371.01	0.01\\
372.01	0.01\\
373.01	0.01\\
374.01	0.01\\
375.01	0.01\\
376.01	0.01\\
377.01	0.01\\
378.01	0.01\\
379.01	0.01\\
380.01	0.01\\
381.01	0.01\\
382.01	0.01\\
383.01	0.01\\
384.01	0.01\\
385.01	0.01\\
386.01	0.01\\
387.01	0.01\\
388.01	0.01\\
389.01	0.01\\
390.01	0.01\\
391.01	0.01\\
392.01	0.01\\
393.01	0.01\\
394.01	0.01\\
395.01	0.01\\
396.01	0.01\\
397.01	0.01\\
398.01	0.01\\
399.01	0.01\\
400.01	0.01\\
401.01	0.01\\
402.01	0.01\\
403.01	0.01\\
404.01	0.01\\
405.01	0.01\\
406.01	0.01\\
407.01	0.01\\
408.01	0.01\\
409.01	0.01\\
410.01	0.01\\
411.01	0.01\\
412.01	0.01\\
413.01	0.01\\
414.01	0.01\\
415.01	0.01\\
416.01	0.01\\
417.01	0.01\\
418.01	0.01\\
419.01	0.01\\
420.01	0.01\\
421.01	0.01\\
422.01	0.01\\
423.01	0.01\\
424.01	0.01\\
425.01	0.01\\
426.01	0.01\\
427.01	0.01\\
428.01	0.01\\
429.01	0.01\\
430.01	0.01\\
431.01	0.01\\
432.01	0.01\\
433.01	0.01\\
434.01	0.01\\
435.01	0.01\\
436.01	0.01\\
437.01	0.01\\
438.01	0.01\\
439.01	0.01\\
440.01	0.01\\
441.01	0.01\\
442.01	0.01\\
443.01	0.01\\
444.01	0.01\\
445.01	0.01\\
446.01	0.01\\
447.01	0.01\\
448.01	0.01\\
449.01	0.01\\
450.01	0.01\\
451.01	0.01\\
452.01	0.01\\
453.01	0.01\\
454.01	0.01\\
455.01	0.01\\
456.01	0.01\\
457.01	0.01\\
458.01	0.01\\
459.01	0.01\\
460.01	0.01\\
461.01	0.01\\
462.01	0.01\\
463.01	0.01\\
464.01	0.01\\
465.01	0.01\\
466.01	0.01\\
467.01	0.01\\
468.01	0.01\\
469.01	0.01\\
470.01	0.01\\
471.01	0.01\\
472.01	0.01\\
473.01	0.01\\
474.01	0.01\\
475.01	0.01\\
476.01	0.01\\
477.01	0.01\\
478.01	0.01\\
479.01	0.01\\
480.01	0.01\\
481.01	0.00997590330187731\\
482.01	0.0099397710803562\\
483.01	0.009902506047415\\
484.01	0.00986406736092241\\
485.01	0.00982441532168716\\
486.01	0.00978351272117389\\
487.01	0.00974132671698551\\
488.01	0.00969783141771086\\
489.01	0.00965301141901538\\
490.01	0.0096068666127476\\
491.01	0.00955940405583456\\
492.01	0.00951058542054645\\
493.01	0.00946035531854987\\
494.01	0.00940865521854821\\
495.01	0.00935542338323276\\
496.01	0.00930059487426016\\
497.01	0.00924410165733517\\
498.01	0.00918587285084049\\
499.01	0.00912583517703891\\
500.01	0.00906391369582405\\
501.01	0.00900003292861265\\
502.01	0.00893411851633223\\
503.01	0.00886609960406335\\
504.01	0.00879591220965877\\
505.01	0.00872349919267541\\
506.01	0.00864876615224294\\
507.01	0.00857158411947091\\
508.01	0.00849181001384525\\
509.01	0.00840928475711463\\
510.01	0.00832383064150769\\
511.01	0.00823524816323673\\
512.01	0.00814331219581346\\
513.01	0.00804776734473679\\
514.01	0.00794832228261845\\
515.01	0.00784464280879699\\
516.01	0.00773634330593778\\
517.01	0.00762297617260505\\
518.01	0.00750401868706793\\
519.01	0.00737885659549818\\
520.01	0.00724676352707722\\
521.01	0.00710687507426357\\
522.01	0.00695815600761101\\
523.01	0.00679935860995764\\
524.01	0.00662982152981215\\
525.01	0.0064522538886912\\
526.01	0.00637048746293591\\
527.01	0.00628910728394081\\
528.01	0.00620543104607813\\
529.01	0.00611943051037367\\
530.01	0.00603108724159264\\
531.01	0.00594039608167631\\
532.01	0.00584736991317263\\
533.01	0.00575204627680054\\
534.01	0.00565449621512798\\
535.01	0.00555484123860636\\
536.01	0.0054532634251946\\
537.01	0.0053500142876607\\
538.01	0.00524543190603984\\
539.01	0.00513996137705213\\
540.01	0.00503418067720346\\
541.01	0.00492883311832434\\
542.01	0.00482493241400958\\
543.01	0.00472395111388705\\
544.01	0.00462778929292503\\
545.01	0.00453327746680012\\
546.01	0.00443779374764156\\
547.01	0.00434190759610771\\
548.01	0.0042464049440776\\
549.01	0.00415236225924511\\
550.01	0.00405829690783676\\
551.01	0.00396203517870896\\
552.01	0.00386376249144564\\
553.01	0.00376376412958557\\
554.01	0.00366246157832406\\
555.01	0.00356082309427307\\
556.01	0.00346030794138988\\
557.01	0.00336129707339248\\
558.01	0.0032640477506096\\
559.01	0.00316867271328949\\
560.01	0.00307507527510044\\
561.01	0.00298338436260538\\
562.01	0.00289399019343357\\
563.01	0.0028072164128482\\
564.01	0.00272322791111361\\
565.01	0.0026410043770131\\
566.01	0.00255988858995343\\
567.01	0.00247982934092541\\
568.01	0.00240074249336894\\
569.01	0.00232261423584147\\
570.01	0.00224537135654028\\
571.01	0.00216886546893075\\
572.01	0.00209288208425282\\
573.01	0.00201717448028996\\
574.01	0.00194157448049112\\
575.01	0.00186606419108644\\
576.01	0.00179062678453804\\
577.01	0.00171522496449762\\
578.01	0.00163980400119186\\
579.01	0.00156429794377999\\
580.01	0.00148863948022046\\
581.01	0.00141277325550749\\
582.01	0.00133667030299336\\
583.01	0.00126032737918952\\
584.01	0.00118374811240739\\
585.01	0.00110693731887654\\
586.01	0.00102989957727651\\
587.01	0.000952638358326865\\
588.01	0.00087515672606412\\
589.01	0.000797457530026434\\
590.01	0.000719542457697378\\
591.01	0.000641410561971532\\
592.01	0.000563058798570631\\
593.01	0.000484484304561451\\
594.01	0.000405687038562455\\
595.01	0.000326672291767633\\
596.01	0.000247452480598537\\
597.01	0.000168047438790727\\
598.01	9.13423856162294e-05\\
599.01	2.91272356096692e-05\\
599.02	2.86193161340032e-05\\
599.03	2.81144595931631e-05\\
599.04	2.76126958075537e-05\\
599.05	2.71140548916225e-05\\
599.06	2.66185672567684e-05\\
599.07	2.61262636142949e-05\\
599.08	2.56371749783636e-05\\
599.09	2.51513326690062e-05\\
599.1	2.46687683151548e-05\\
599.11	2.41895138576968e-05\\
599.12	2.37136015525663e-05\\
599.13	2.32410639738646e-05\\
599.14	2.27719340170195e-05\\
599.15	2.23062449019578e-05\\
599.16	2.18440301763267e-05\\
599.17	2.13853237187381e-05\\
599.18	2.09301597420501e-05\\
599.19	2.04785727966738e-05\\
599.2	2.00305977739214e-05\\
599.21	1.95862699093816e-05\\
599.22	1.91456247863337e-05\\
599.23	1.87086983391843e-05\\
599.24	1.8275526856959e-05\\
599.25	1.78461469867996e-05\\
599.26	1.74205963976177e-05\\
599.27	1.69989155754252e-05\\
599.28	1.65811454095281e-05\\
599.29	1.61673271965243e-05\\
599.3	1.57575026443323e-05\\
599.31	1.53517138762656e-05\\
599.32	1.49500034351425e-05\\
599.33	1.45524142874422e-05\\
599.34	1.41589898274928e-05\\
599.35	1.37697738817107e-05\\
599.36	1.33848107128787e-05\\
599.37	1.30041450244651e-05\\
599.38	1.26278219649834e-05\\
599.39	1.2255887132398e-05\\
599.4	1.18883865785728e-05\\
599.41	1.15253668137704e-05\\
599.42	1.11668748111818e-05\\
599.43	1.08129580115111e-05\\
599.44	1.04636643276056e-05\\
599.45	1.01190421491221e-05\\
599.46	9.77914034725434e-06\\
599.47	9.44400827949769e-06\\
599.48	9.1136957944566e-06\\
599.49	8.78825323671374e-06\\
599.5	8.46773145173063e-06\\
599.51	8.15218179081069e-06\\
599.52	7.84165611610738e-06\\
599.53	7.53620680567398e-06\\
599.54	7.23588675857409e-06\\
599.55	6.94074940003724e-06\\
599.56	6.65084868666475e-06\\
599.57	6.36623911169296e-06\\
599.58	6.08697571029791e-06\\
599.59	5.81311406496442e-06\\
599.6	5.54471031090353e-06\\
599.61	5.28182114151003e-06\\
599.62	5.0245038139083e-06\\
599.63	4.77281615451214e-06\\
599.64	4.52681656466433e-06\\
599.65	4.28656402633523e-06\\
599.66	4.05211810785081e-06\\
599.67	3.82353896971784e-06\\
599.68	3.60088737046127e-06\\
599.69	3.38422467256044e-06\\
599.7	3.17361284841132e-06\\
599.71	2.96911448637548e-06\\
599.72	2.77079279686725e-06\\
599.73	2.57871161851199e-06\\
599.74	2.39293542435466e-06\\
599.75	2.21352932815513e-06\\
599.76	2.04055909071126e-06\\
599.77	1.87409112627039e-06\\
599.78	1.71419250899474e-06\\
599.79	1.56093097948388e-06\\
599.8	1.41437495138237e-06\\
599.81	1.27459351803062e-06\\
599.82	1.14165645918526e-06\\
599.83	1.01563424782618e-06\\
599.84	8.96598057003456e-07\\
599.85	7.84619766767622e-07\\
599.86	6.79771971167503e-07\\
599.87	5.82127985312292e-07\\
599.88	4.91761852506462e-07\\
599.89	4.08748351459279e-07\\
599.9	3.33163003549825e-07\\
599.91	2.65082080187426e-07\\
599.92	2.04582610225865e-07\\
599.93	1.51742387459117e-07\\
599.94	1.06639978193421e-07\\
599.95	6.93547288800611e-08\\
599.96	3.99667738487652e-08\\
599.97	1.85570430896731e-08\\
599.98	5.20727013418598e-09\\
599.99	0\\
600	0\\
};
\addplot [color=blue,solid,forget plot]
  table[row sep=crcr]{%
0.01	0.01\\
1.01	0.01\\
2.01	0.01\\
3.01	0.01\\
4.01	0.01\\
5.01	0.01\\
6.01	0.01\\
7.01	0.01\\
8.01	0.01\\
9.01	0.01\\
10.01	0.01\\
11.01	0.01\\
12.01	0.01\\
13.01	0.01\\
14.01	0.01\\
15.01	0.01\\
16.01	0.01\\
17.01	0.01\\
18.01	0.01\\
19.01	0.01\\
20.01	0.01\\
21.01	0.01\\
22.01	0.01\\
23.01	0.01\\
24.01	0.01\\
25.01	0.01\\
26.01	0.01\\
27.01	0.01\\
28.01	0.01\\
29.01	0.01\\
30.01	0.01\\
31.01	0.01\\
32.01	0.01\\
33.01	0.01\\
34.01	0.01\\
35.01	0.01\\
36.01	0.01\\
37.01	0.01\\
38.01	0.01\\
39.01	0.01\\
40.01	0.01\\
41.01	0.01\\
42.01	0.01\\
43.01	0.01\\
44.01	0.01\\
45.01	0.01\\
46.01	0.01\\
47.01	0.01\\
48.01	0.01\\
49.01	0.01\\
50.01	0.01\\
51.01	0.01\\
52.01	0.01\\
53.01	0.01\\
54.01	0.01\\
55.01	0.01\\
56.01	0.01\\
57.01	0.01\\
58.01	0.01\\
59.01	0.01\\
60.01	0.01\\
61.01	0.01\\
62.01	0.01\\
63.01	0.01\\
64.01	0.01\\
65.01	0.01\\
66.01	0.01\\
67.01	0.01\\
68.01	0.01\\
69.01	0.01\\
70.01	0.01\\
71.01	0.01\\
72.01	0.01\\
73.01	0.01\\
74.01	0.01\\
75.01	0.01\\
76.01	0.01\\
77.01	0.01\\
78.01	0.01\\
79.01	0.01\\
80.01	0.01\\
81.01	0.01\\
82.01	0.01\\
83.01	0.01\\
84.01	0.01\\
85.01	0.01\\
86.01	0.01\\
87.01	0.01\\
88.01	0.01\\
89.01	0.01\\
90.01	0.01\\
91.01	0.01\\
92.01	0.01\\
93.01	0.01\\
94.01	0.01\\
95.01	0.01\\
96.01	0.01\\
97.01	0.01\\
98.01	0.01\\
99.01	0.01\\
100.01	0.01\\
101.01	0.01\\
102.01	0.01\\
103.01	0.01\\
104.01	0.01\\
105.01	0.01\\
106.01	0.01\\
107.01	0.01\\
108.01	0.01\\
109.01	0.01\\
110.01	0.01\\
111.01	0.01\\
112.01	0.01\\
113.01	0.01\\
114.01	0.01\\
115.01	0.01\\
116.01	0.01\\
117.01	0.01\\
118.01	0.01\\
119.01	0.01\\
120.01	0.01\\
121.01	0.01\\
122.01	0.01\\
123.01	0.01\\
124.01	0.01\\
125.01	0.01\\
126.01	0.01\\
127.01	0.01\\
128.01	0.01\\
129.01	0.01\\
130.01	0.01\\
131.01	0.01\\
132.01	0.01\\
133.01	0.01\\
134.01	0.01\\
135.01	0.01\\
136.01	0.01\\
137.01	0.01\\
138.01	0.01\\
139.01	0.01\\
140.01	0.01\\
141.01	0.01\\
142.01	0.01\\
143.01	0.01\\
144.01	0.01\\
145.01	0.01\\
146.01	0.01\\
147.01	0.01\\
148.01	0.01\\
149.01	0.01\\
150.01	0.01\\
151.01	0.01\\
152.01	0.01\\
153.01	0.01\\
154.01	0.01\\
155.01	0.01\\
156.01	0.01\\
157.01	0.01\\
158.01	0.01\\
159.01	0.01\\
160.01	0.01\\
161.01	0.01\\
162.01	0.01\\
163.01	0.01\\
164.01	0.01\\
165.01	0.01\\
166.01	0.01\\
167.01	0.01\\
168.01	0.01\\
169.01	0.01\\
170.01	0.01\\
171.01	0.01\\
172.01	0.01\\
173.01	0.01\\
174.01	0.01\\
175.01	0.01\\
176.01	0.01\\
177.01	0.01\\
178.01	0.01\\
179.01	0.01\\
180.01	0.01\\
181.01	0.01\\
182.01	0.01\\
183.01	0.01\\
184.01	0.01\\
185.01	0.01\\
186.01	0.01\\
187.01	0.01\\
188.01	0.01\\
189.01	0.01\\
190.01	0.01\\
191.01	0.01\\
192.01	0.01\\
193.01	0.01\\
194.01	0.01\\
195.01	0.01\\
196.01	0.01\\
197.01	0.01\\
198.01	0.01\\
199.01	0.01\\
200.01	0.01\\
201.01	0.01\\
202.01	0.01\\
203.01	0.01\\
204.01	0.01\\
205.01	0.01\\
206.01	0.01\\
207.01	0.01\\
208.01	0.01\\
209.01	0.01\\
210.01	0.01\\
211.01	0.01\\
212.01	0.01\\
213.01	0.01\\
214.01	0.01\\
215.01	0.01\\
216.01	0.01\\
217.01	0.01\\
218.01	0.01\\
219.01	0.01\\
220.01	0.01\\
221.01	0.01\\
222.01	0.01\\
223.01	0.01\\
224.01	0.01\\
225.01	0.01\\
226.01	0.01\\
227.01	0.01\\
228.01	0.01\\
229.01	0.01\\
230.01	0.01\\
231.01	0.01\\
232.01	0.01\\
233.01	0.01\\
234.01	0.01\\
235.01	0.01\\
236.01	0.01\\
237.01	0.01\\
238.01	0.01\\
239.01	0.01\\
240.01	0.01\\
241.01	0.01\\
242.01	0.01\\
243.01	0.01\\
244.01	0.01\\
245.01	0.01\\
246.01	0.01\\
247.01	0.01\\
248.01	0.01\\
249.01	0.01\\
250.01	0.01\\
251.01	0.01\\
252.01	0.01\\
253.01	0.01\\
254.01	0.01\\
255.01	0.01\\
256.01	0.01\\
257.01	0.01\\
258.01	0.01\\
259.01	0.01\\
260.01	0.01\\
261.01	0.01\\
262.01	0.01\\
263.01	0.01\\
264.01	0.01\\
265.01	0.01\\
266.01	0.01\\
267.01	0.01\\
268.01	0.01\\
269.01	0.01\\
270.01	0.01\\
271.01	0.01\\
272.01	0.01\\
273.01	0.01\\
274.01	0.01\\
275.01	0.01\\
276.01	0.01\\
277.01	0.01\\
278.01	0.01\\
279.01	0.01\\
280.01	0.01\\
281.01	0.01\\
282.01	0.01\\
283.01	0.01\\
284.01	0.01\\
285.01	0.01\\
286.01	0.01\\
287.01	0.01\\
288.01	0.01\\
289.01	0.01\\
290.01	0.01\\
291.01	0.01\\
292.01	0.01\\
293.01	0.01\\
294.01	0.01\\
295.01	0.01\\
296.01	0.01\\
297.01	0.01\\
298.01	0.01\\
299.01	0.01\\
300.01	0.01\\
301.01	0.01\\
302.01	0.01\\
303.01	0.01\\
304.01	0.01\\
305.01	0.01\\
306.01	0.01\\
307.01	0.01\\
308.01	0.01\\
309.01	0.01\\
310.01	0.01\\
311.01	0.01\\
312.01	0.01\\
313.01	0.01\\
314.01	0.01\\
315.01	0.01\\
316.01	0.01\\
317.01	0.01\\
318.01	0.01\\
319.01	0.01\\
320.01	0.01\\
321.01	0.01\\
322.01	0.01\\
323.01	0.01\\
324.01	0.01\\
325.01	0.01\\
326.01	0.01\\
327.01	0.01\\
328.01	0.01\\
329.01	0.01\\
330.01	0.01\\
331.01	0.01\\
332.01	0.01\\
333.01	0.01\\
334.01	0.01\\
335.01	0.01\\
336.01	0.01\\
337.01	0.01\\
338.01	0.01\\
339.01	0.01\\
340.01	0.01\\
341.01	0.01\\
342.01	0.01\\
343.01	0.01\\
344.01	0.01\\
345.01	0.01\\
346.01	0.01\\
347.01	0.01\\
348.01	0.01\\
349.01	0.01\\
350.01	0.01\\
351.01	0.01\\
352.01	0.01\\
353.01	0.01\\
354.01	0.01\\
355.01	0.01\\
356.01	0.01\\
357.01	0.01\\
358.01	0.01\\
359.01	0.01\\
360.01	0.01\\
361.01	0.01\\
362.01	0.01\\
363.01	0.01\\
364.01	0.01\\
365.01	0.01\\
366.01	0.01\\
367.01	0.01\\
368.01	0.01\\
369.01	0.01\\
370.01	0.01\\
371.01	0.01\\
372.01	0.01\\
373.01	0.01\\
374.01	0.01\\
375.01	0.01\\
376.01	0.01\\
377.01	0.01\\
378.01	0.01\\
379.01	0.01\\
380.01	0.01\\
381.01	0.01\\
382.01	0.01\\
383.01	0.01\\
384.01	0.01\\
385.01	0.01\\
386.01	0.01\\
387.01	0.01\\
388.01	0.01\\
389.01	0.01\\
390.01	0.01\\
391.01	0.01\\
392.01	0.01\\
393.01	0.01\\
394.01	0.01\\
395.01	0.01\\
396.01	0.01\\
397.01	0.01\\
398.01	0.01\\
399.01	0.01\\
400.01	0.01\\
401.01	0.01\\
402.01	0.01\\
403.01	0.01\\
404.01	0.01\\
405.01	0.01\\
406.01	0.01\\
407.01	0.01\\
408.01	0.01\\
409.01	0.01\\
410.01	0.01\\
411.01	0.01\\
412.01	0.01\\
413.01	0.01\\
414.01	0.01\\
415.01	0.01\\
416.01	0.01\\
417.01	0.01\\
418.01	0.01\\
419.01	0.01\\
420.01	0.01\\
421.01	0.01\\
422.01	0.01\\
423.01	0.01\\
424.01	0.01\\
425.01	0.01\\
426.01	0.01\\
427.01	0.01\\
428.01	0.01\\
429.01	0.01\\
430.01	0.01\\
431.01	0.01\\
432.01	0.01\\
433.01	0.01\\
434.01	0.01\\
435.01	0.01\\
436.01	0.01\\
437.01	0.01\\
438.01	0.01\\
439.01	0.01\\
440.01	0.01\\
441.01	0.01\\
442.01	0.01\\
443.01	0.01\\
444.01	0.01\\
445.01	0.01\\
446.01	0.01\\
447.01	0.01\\
448.01	0.01\\
449.01	0.01\\
450.01	0.01\\
451.01	0.01\\
452.01	0.01\\
453.01	0.01\\
454.01	0.01\\
455.01	0.01\\
456.01	0.01\\
457.01	0.01\\
458.01	0.01\\
459.01	0.01\\
460.01	0.01\\
461.01	0.01\\
462.01	0.01\\
463.01	0.01\\
464.01	0.01\\
465.01	0.01\\
466.01	0.01\\
467.01	0.01\\
468.01	0.01\\
469.01	0.01\\
470.01	0.01\\
471.01	0.01\\
472.01	0.01\\
473.01	0.01\\
474.01	0.01\\
475.01	0.01\\
476.01	0.01\\
477.01	0.01\\
478.01	0.01\\
479.01	0.01\\
480.01	0.01\\
481.01	0.01\\
482.01	0.01\\
483.01	0.01\\
484.01	0.01\\
485.01	0.01\\
486.01	0.01\\
487.01	0.01\\
488.01	0.01\\
489.01	0.01\\
490.01	0.01\\
491.01	0.01\\
492.01	0.01\\
493.01	0.01\\
494.01	0.01\\
495.01	0.01\\
496.01	0.01\\
497.01	0.01\\
498.01	0.01\\
499.01	0.01\\
500.01	0.01\\
501.01	0.01\\
502.01	0.01\\
503.01	0.01\\
504.01	0.01\\
505.01	0.01\\
506.01	0.01\\
507.01	0.01\\
508.01	0.01\\
509.01	0.01\\
510.01	0.01\\
511.01	0.01\\
512.01	0.01\\
513.01	0.01\\
514.01	0.01\\
515.01	0.01\\
516.01	0.01\\
517.01	0.01\\
518.01	0.01\\
519.01	0.01\\
520.01	0.01\\
521.01	0.01\\
522.01	0.01\\
523.01	0.01\\
524.01	0.01\\
525.01	0.01\\
526.01	0.00989722854349804\\
527.01	0.00978814273751665\\
528.01	0.00967516165466474\\
529.01	0.00955803374497832\\
530.01	0.00943648091405857\\
531.01	0.00931019467488656\\
532.01	0.00917883187684529\\
533.01	0.00904201077393832\\
534.01	0.00889930832443303\\
535.01	0.00875023292005354\\
536.01	0.00859420773776338\\
537.01	0.00843056963164173\\
538.01	0.00825857320407405\\
539.01	0.00807737494716381\\
540.01	0.00788601007179719\\
541.01	0.00768337231581051\\
542.01	0.00746812865996539\\
543.01	0.00723855993773447\\
544.01	0.00699263227818423\\
545.01	0.00673338432713554\\
546.01	0.00646300161813907\\
547.01	0.00618042366892763\\
548.01	0.00588439316956648\\
549.01	0.00557471857017588\\
550.01	0.0054365780548863\\
551.01	0.00529498022819429\\
552.01	0.00515020347795563\\
553.01	0.00500264559133276\\
554.01	0.00485285565777818\\
555.01	0.00470157180593755\\
556.01	0.00454976142488678\\
557.01	0.00439872524951088\\
558.01	0.00425022313750997\\
559.01	0.00410662240815378\\
560.01	0.00396855697861401\\
561.01	0.00382883467193446\\
562.01	0.00368720830181002\\
563.01	0.00354449059715707\\
564.01	0.00340185782150943\\
565.01	0.00326208932254077\\
566.01	0.0031280109045733\\
567.01	0.00300169631967474\\
568.01	0.00287904197142182\\
569.01	0.00275913443257296\\
570.01	0.00264254980737055\\
571.01	0.00252970762472733\\
572.01	0.00242071734339785\\
573.01	0.00231515549731677\\
574.01	0.00221140419527129\\
575.01	0.0021090280458826\\
576.01	0.00200828344457156\\
577.01	0.00190936925280396\\
578.01	0.00181239939701331\\
579.01	0.0017173717974502\\
580.01	0.00162413610956826\\
581.01	0.00153236543046933\\
582.01	0.00144157354335622\\
583.01	0.0013514547773948\\
584.01	0.0012619843991491\\
585.01	0.00117317775709303\\
586.01	0.00108508625059967\\
587.01	0.000997759365490835\\
588.01	0.00091123592620997\\
589.01	0.000825550488607604\\
590.01	0.000740739252561627\\
591.01	0.000656829007057548\\
592.01	0.000573812567071902\\
593.01	0.000491642606866438\\
594.01	0.000410231936064402\\
595.01	0.000329457487875126\\
596.01	0.000249168838092057\\
597.01	0.000169201630515214\\
598.01	9.14271566706763e-05\\
599.01	2.9130630291278e-05\\
599.02	2.86225745231034e-05\\
599.03	2.81175858083405e-05\\
599.04	2.76156938848176e-05\\
599.05	2.71169287851831e-05\\
599.06	2.66213208398331e-05\\
599.07	2.61289006798487e-05\\
599.08	2.56396992399759e-05\\
599.09	2.51537477616182e-05\\
599.1	2.46710777958743e-05\\
599.11	2.4191721206596e-05\\
599.12	2.37157101734831e-05\\
599.13	2.32430771952095e-05\\
599.14	2.27738550925647e-05\\
599.15	2.23080770116509e-05\\
599.16	2.18457764270957e-05\\
599.17	2.13869871452978e-05\\
599.18	2.09317433077106e-05\\
599.19	2.04800793941525e-05\\
599.2	2.00320302261563e-05\\
599.21	1.95876309703451e-05\\
599.22	1.91469171418445e-05\\
599.23	1.87099246077346e-05\\
599.24	1.82766895905177e-05\\
599.25	1.78472486716558e-05\\
599.26	1.74216394527651e-05\\
599.27	1.69999023562576e-05\\
599.28	1.65820782086351e-05\\
599.29	1.61682082444846e-05\\
599.3	1.57583341105112e-05\\
599.31	1.53524978696098e-05\\
599.32	1.49507420049849e-05\\
599.33	1.45531094242948e-05\\
599.34	1.41596434638531e-05\\
599.35	1.37703878928669e-05\\
599.36	1.33853869177126e-05\\
599.37	1.30046851862537e-05\\
599.38	1.26283277922107e-05\\
599.39	1.22563602795653e-05\\
599.4	1.18888286470102e-05\\
599.41	1.1525779352442e-05\\
599.42	1.11672593175044e-05\\
599.43	1.0813315932166e-05\\
599.44	1.04639970593626e-05\\
599.45	1.01193510396499e-05\\
599.46	9.7794266959491e-06\\
599.47	9.44427333829899e-06\\
599.48	9.11394076868154e-06\\
599.49	8.78847928588097e-06\\
599.5	8.46793969039999e-06\\
599.51	8.15237328941759e-06\\
599.52	7.8418319017972e-06\\
599.53	7.53636786315036e-06\\
599.54	7.23603403094721e-06\\
599.55	6.94088378966513e-06\\
599.56	6.65097105601725e-06\\
599.57	6.36635028419819e-06\\
599.58	6.08707647120968e-06\\
599.59	5.81320516222436e-06\\
599.6	5.54479245599981e-06\\
599.61	5.28189501036898e-06\\
599.62	5.02457004774616e-06\\
599.63	4.77287536072325e-06\\
599.64	4.52686931771279e-06\\
599.65	4.28661086862223e-06\\
599.66	4.05215955062363e-06\\
599.67	3.82357549394419e-06\\
599.68	3.60091942774518e-06\\
599.69	3.38425268603219e-06\\
599.7	3.17363721363643e-06\\
599.71	2.96913557226545e-06\\
599.72	2.77081094658509e-06\\
599.73	2.57872715039159e-06\\
599.74	2.39294863282891e-06\\
599.75	2.21354048466835e-06\\
599.76	2.04056844465075e-06\\
599.77	1.87409890589456e-06\\
599.78	1.71419892236284e-06\\
599.79	1.56093621539802e-06\\
599.8	1.41437918031902e-06\\
599.81	1.27459689307913e-06\\
599.82	1.14165911700367e-06\\
599.83	1.01563630956411e-06\\
599.84	8.9659962925967e-07\\
599.85	7.84620942530234e-07\\
599.86	6.79772830764619e-07\\
599.87	5.82128597343551e-07\\
599.88	4.91762274797136e-07\\
599.89	4.0874863199182e-07\\
599.9	3.33163181409288e-07\\
599.91	2.65082186487811e-07\\
599.92	2.04582669039929e-07\\
599.93	1.51742416741249e-07\\
599.94	1.06639990685164e-07\\
599.95	6.93547330260502e-08\\
599.96	3.99667746744936e-08\\
599.97	1.85570430896731e-08\\
599.98	5.20727013418598e-09\\
599.99	0\\
600	0\\
};
\addplot [color=mycolor10,solid,forget plot]
  table[row sep=crcr]{%
0.01	0.01\\
1.01	0.01\\
2.01	0.01\\
3.01	0.01\\
4.01	0.01\\
5.01	0.01\\
6.01	0.01\\
7.01	0.01\\
8.01	0.01\\
9.01	0.01\\
10.01	0.01\\
11.01	0.01\\
12.01	0.01\\
13.01	0.01\\
14.01	0.01\\
15.01	0.01\\
16.01	0.01\\
17.01	0.01\\
18.01	0.01\\
19.01	0.01\\
20.01	0.01\\
21.01	0.01\\
22.01	0.01\\
23.01	0.01\\
24.01	0.01\\
25.01	0.01\\
26.01	0.01\\
27.01	0.01\\
28.01	0.01\\
29.01	0.01\\
30.01	0.01\\
31.01	0.01\\
32.01	0.01\\
33.01	0.01\\
34.01	0.01\\
35.01	0.01\\
36.01	0.01\\
37.01	0.01\\
38.01	0.01\\
39.01	0.01\\
40.01	0.01\\
41.01	0.01\\
42.01	0.01\\
43.01	0.01\\
44.01	0.01\\
45.01	0.01\\
46.01	0.01\\
47.01	0.01\\
48.01	0.01\\
49.01	0.01\\
50.01	0.01\\
51.01	0.01\\
52.01	0.01\\
53.01	0.01\\
54.01	0.01\\
55.01	0.01\\
56.01	0.01\\
57.01	0.01\\
58.01	0.01\\
59.01	0.01\\
60.01	0.01\\
61.01	0.01\\
62.01	0.01\\
63.01	0.01\\
64.01	0.01\\
65.01	0.01\\
66.01	0.01\\
67.01	0.01\\
68.01	0.01\\
69.01	0.01\\
70.01	0.01\\
71.01	0.01\\
72.01	0.01\\
73.01	0.01\\
74.01	0.01\\
75.01	0.01\\
76.01	0.01\\
77.01	0.01\\
78.01	0.01\\
79.01	0.01\\
80.01	0.01\\
81.01	0.01\\
82.01	0.01\\
83.01	0.01\\
84.01	0.01\\
85.01	0.01\\
86.01	0.01\\
87.01	0.01\\
88.01	0.01\\
89.01	0.01\\
90.01	0.01\\
91.01	0.01\\
92.01	0.01\\
93.01	0.01\\
94.01	0.01\\
95.01	0.01\\
96.01	0.01\\
97.01	0.01\\
98.01	0.01\\
99.01	0.01\\
100.01	0.01\\
101.01	0.01\\
102.01	0.01\\
103.01	0.01\\
104.01	0.01\\
105.01	0.01\\
106.01	0.01\\
107.01	0.01\\
108.01	0.01\\
109.01	0.01\\
110.01	0.01\\
111.01	0.01\\
112.01	0.01\\
113.01	0.01\\
114.01	0.01\\
115.01	0.01\\
116.01	0.01\\
117.01	0.01\\
118.01	0.01\\
119.01	0.01\\
120.01	0.01\\
121.01	0.01\\
122.01	0.01\\
123.01	0.01\\
124.01	0.01\\
125.01	0.01\\
126.01	0.01\\
127.01	0.01\\
128.01	0.01\\
129.01	0.01\\
130.01	0.01\\
131.01	0.01\\
132.01	0.01\\
133.01	0.01\\
134.01	0.01\\
135.01	0.01\\
136.01	0.01\\
137.01	0.01\\
138.01	0.01\\
139.01	0.01\\
140.01	0.01\\
141.01	0.01\\
142.01	0.01\\
143.01	0.01\\
144.01	0.01\\
145.01	0.01\\
146.01	0.01\\
147.01	0.01\\
148.01	0.01\\
149.01	0.01\\
150.01	0.01\\
151.01	0.01\\
152.01	0.01\\
153.01	0.01\\
154.01	0.01\\
155.01	0.01\\
156.01	0.01\\
157.01	0.01\\
158.01	0.01\\
159.01	0.01\\
160.01	0.01\\
161.01	0.01\\
162.01	0.01\\
163.01	0.01\\
164.01	0.01\\
165.01	0.01\\
166.01	0.01\\
167.01	0.01\\
168.01	0.01\\
169.01	0.01\\
170.01	0.01\\
171.01	0.01\\
172.01	0.01\\
173.01	0.01\\
174.01	0.01\\
175.01	0.01\\
176.01	0.01\\
177.01	0.01\\
178.01	0.01\\
179.01	0.01\\
180.01	0.01\\
181.01	0.01\\
182.01	0.01\\
183.01	0.01\\
184.01	0.01\\
185.01	0.01\\
186.01	0.01\\
187.01	0.01\\
188.01	0.01\\
189.01	0.01\\
190.01	0.01\\
191.01	0.01\\
192.01	0.01\\
193.01	0.01\\
194.01	0.01\\
195.01	0.01\\
196.01	0.01\\
197.01	0.01\\
198.01	0.01\\
199.01	0.01\\
200.01	0.01\\
201.01	0.01\\
202.01	0.01\\
203.01	0.01\\
204.01	0.01\\
205.01	0.01\\
206.01	0.01\\
207.01	0.01\\
208.01	0.01\\
209.01	0.01\\
210.01	0.01\\
211.01	0.01\\
212.01	0.01\\
213.01	0.01\\
214.01	0.01\\
215.01	0.01\\
216.01	0.01\\
217.01	0.01\\
218.01	0.01\\
219.01	0.01\\
220.01	0.01\\
221.01	0.01\\
222.01	0.01\\
223.01	0.01\\
224.01	0.01\\
225.01	0.01\\
226.01	0.01\\
227.01	0.01\\
228.01	0.01\\
229.01	0.01\\
230.01	0.01\\
231.01	0.01\\
232.01	0.01\\
233.01	0.01\\
234.01	0.01\\
235.01	0.01\\
236.01	0.01\\
237.01	0.01\\
238.01	0.01\\
239.01	0.01\\
240.01	0.01\\
241.01	0.01\\
242.01	0.01\\
243.01	0.01\\
244.01	0.01\\
245.01	0.01\\
246.01	0.01\\
247.01	0.01\\
248.01	0.01\\
249.01	0.01\\
250.01	0.01\\
251.01	0.01\\
252.01	0.01\\
253.01	0.01\\
254.01	0.01\\
255.01	0.01\\
256.01	0.01\\
257.01	0.01\\
258.01	0.01\\
259.01	0.01\\
260.01	0.01\\
261.01	0.01\\
262.01	0.01\\
263.01	0.01\\
264.01	0.01\\
265.01	0.01\\
266.01	0.01\\
267.01	0.01\\
268.01	0.01\\
269.01	0.01\\
270.01	0.01\\
271.01	0.01\\
272.01	0.01\\
273.01	0.01\\
274.01	0.01\\
275.01	0.01\\
276.01	0.01\\
277.01	0.01\\
278.01	0.01\\
279.01	0.01\\
280.01	0.01\\
281.01	0.01\\
282.01	0.01\\
283.01	0.01\\
284.01	0.01\\
285.01	0.01\\
286.01	0.01\\
287.01	0.01\\
288.01	0.01\\
289.01	0.01\\
290.01	0.01\\
291.01	0.01\\
292.01	0.01\\
293.01	0.01\\
294.01	0.01\\
295.01	0.01\\
296.01	0.01\\
297.01	0.01\\
298.01	0.01\\
299.01	0.01\\
300.01	0.01\\
301.01	0.01\\
302.01	0.01\\
303.01	0.01\\
304.01	0.01\\
305.01	0.01\\
306.01	0.01\\
307.01	0.01\\
308.01	0.01\\
309.01	0.01\\
310.01	0.01\\
311.01	0.01\\
312.01	0.01\\
313.01	0.01\\
314.01	0.01\\
315.01	0.01\\
316.01	0.01\\
317.01	0.01\\
318.01	0.01\\
319.01	0.01\\
320.01	0.01\\
321.01	0.01\\
322.01	0.01\\
323.01	0.01\\
324.01	0.01\\
325.01	0.01\\
326.01	0.01\\
327.01	0.01\\
328.01	0.01\\
329.01	0.01\\
330.01	0.01\\
331.01	0.01\\
332.01	0.01\\
333.01	0.01\\
334.01	0.01\\
335.01	0.01\\
336.01	0.01\\
337.01	0.01\\
338.01	0.01\\
339.01	0.01\\
340.01	0.01\\
341.01	0.01\\
342.01	0.01\\
343.01	0.01\\
344.01	0.01\\
345.01	0.01\\
346.01	0.01\\
347.01	0.01\\
348.01	0.01\\
349.01	0.01\\
350.01	0.01\\
351.01	0.01\\
352.01	0.01\\
353.01	0.01\\
354.01	0.01\\
355.01	0.01\\
356.01	0.01\\
357.01	0.01\\
358.01	0.01\\
359.01	0.01\\
360.01	0.01\\
361.01	0.01\\
362.01	0.01\\
363.01	0.01\\
364.01	0.01\\
365.01	0.01\\
366.01	0.01\\
367.01	0.01\\
368.01	0.01\\
369.01	0.01\\
370.01	0.01\\
371.01	0.01\\
372.01	0.01\\
373.01	0.01\\
374.01	0.01\\
375.01	0.01\\
376.01	0.01\\
377.01	0.01\\
378.01	0.01\\
379.01	0.01\\
380.01	0.01\\
381.01	0.01\\
382.01	0.01\\
383.01	0.01\\
384.01	0.01\\
385.01	0.01\\
386.01	0.01\\
387.01	0.01\\
388.01	0.01\\
389.01	0.01\\
390.01	0.01\\
391.01	0.01\\
392.01	0.01\\
393.01	0.01\\
394.01	0.01\\
395.01	0.01\\
396.01	0.01\\
397.01	0.01\\
398.01	0.01\\
399.01	0.01\\
400.01	0.01\\
401.01	0.01\\
402.01	0.01\\
403.01	0.01\\
404.01	0.01\\
405.01	0.01\\
406.01	0.01\\
407.01	0.01\\
408.01	0.01\\
409.01	0.01\\
410.01	0.01\\
411.01	0.01\\
412.01	0.01\\
413.01	0.01\\
414.01	0.01\\
415.01	0.01\\
416.01	0.01\\
417.01	0.01\\
418.01	0.01\\
419.01	0.01\\
420.01	0.01\\
421.01	0.01\\
422.01	0.01\\
423.01	0.01\\
424.01	0.01\\
425.01	0.01\\
426.01	0.01\\
427.01	0.01\\
428.01	0.01\\
429.01	0.01\\
430.01	0.01\\
431.01	0.01\\
432.01	0.01\\
433.01	0.01\\
434.01	0.01\\
435.01	0.01\\
436.01	0.01\\
437.01	0.01\\
438.01	0.01\\
439.01	0.01\\
440.01	0.01\\
441.01	0.01\\
442.01	0.01\\
443.01	0.01\\
444.01	0.01\\
445.01	0.01\\
446.01	0.01\\
447.01	0.01\\
448.01	0.01\\
449.01	0.01\\
450.01	0.01\\
451.01	0.01\\
452.01	0.01\\
453.01	0.01\\
454.01	0.01\\
455.01	0.01\\
456.01	0.01\\
457.01	0.01\\
458.01	0.01\\
459.01	0.01\\
460.01	0.01\\
461.01	0.01\\
462.01	0.01\\
463.01	0.01\\
464.01	0.01\\
465.01	0.01\\
466.01	0.01\\
467.01	0.01\\
468.01	0.01\\
469.01	0.01\\
470.01	0.01\\
471.01	0.01\\
472.01	0.01\\
473.01	0.01\\
474.01	0.01\\
475.01	0.01\\
476.01	0.01\\
477.01	0.01\\
478.01	0.01\\
479.01	0.01\\
480.01	0.01\\
481.01	0.01\\
482.01	0.01\\
483.01	0.01\\
484.01	0.01\\
485.01	0.01\\
486.01	0.01\\
487.01	0.01\\
488.01	0.01\\
489.01	0.01\\
490.01	0.01\\
491.01	0.01\\
492.01	0.01\\
493.01	0.01\\
494.01	0.01\\
495.01	0.01\\
496.01	0.01\\
497.01	0.01\\
498.01	0.01\\
499.01	0.01\\
500.01	0.01\\
501.01	0.01\\
502.01	0.01\\
503.01	0.01\\
504.01	0.01\\
505.01	0.01\\
506.01	0.01\\
507.01	0.01\\
508.01	0.01\\
509.01	0.01\\
510.01	0.01\\
511.01	0.01\\
512.01	0.01\\
513.01	0.01\\
514.01	0.01\\
515.01	0.01\\
516.01	0.01\\
517.01	0.01\\
518.01	0.01\\
519.01	0.01\\
520.01	0.01\\
521.01	0.01\\
522.01	0.01\\
523.01	0.01\\
524.01	0.01\\
525.01	0.01\\
526.01	0.01\\
527.01	0.01\\
528.01	0.01\\
529.01	0.01\\
530.01	0.01\\
531.01	0.01\\
532.01	0.01\\
533.01	0.01\\
534.01	0.01\\
535.01	0.01\\
536.01	0.01\\
537.01	0.01\\
538.01	0.01\\
539.01	0.01\\
540.01	0.01\\
541.01	0.01\\
542.01	0.01\\
543.01	0.01\\
544.01	0.01\\
545.01	0.01\\
546.01	0.01\\
547.01	0.01\\
548.01	0.01\\
549.01	0.0099986849723122\\
550.01	0.00981196787112131\\
551.01	0.00961657413274832\\
552.01	0.00941162887606914\\
553.01	0.00919612090986087\\
554.01	0.0089688763992542\\
555.01	0.00872852663770928\\
556.01	0.00847346866094748\\
557.01	0.00820181618375634\\
558.01	0.00791133722938716\\
559.01	0.00759937552712021\\
560.01	0.0072652506873956\\
561.01	0.00691592078390264\\
562.01	0.00655121878740571\\
563.01	0.0061701041694901\\
564.01	0.00577130240260382\\
565.01	0.00535294336937349\\
566.01	0.00491261245564841\\
567.01	0.00455026635944565\\
568.01	0.00435240083757039\\
569.01	0.00415256776986028\\
570.01	0.00395251099631614\\
571.01	0.00375468014392657\\
572.01	0.00356247354745911\\
573.01	0.00337782507188291\\
574.01	0.00319314443084913\\
575.01	0.00300871330112231\\
576.01	0.00282572334554877\\
577.01	0.00264561290078165\\
578.01	0.00247010123583664\\
579.01	0.00230122116256501\\
580.01	0.0021413447694239\\
581.01	0.00199319231655826\\
582.01	0.00185835342257616\\
583.01	0.00172871168230353\\
584.01	0.0016014443999877\\
585.01	0.0014759666797958\\
586.01	0.00135209058275517\\
587.01	0.00123011324476261\\
588.01	0.00111029688784535\\
589.01	0.00099285380763359\\
590.01	0.000877986990115272\\
591.01	0.000766096817003731\\
592.01	0.000657656957495178\\
593.01	0.000553079177615124\\
594.01	0.000452675589256735\\
595.01	0.000356615005476036\\
596.01	0.000264873949180146\\
597.01	0.000177185395016351\\
598.01	9.36647801022521e-05\\
599.01	2.93891165479357e-05\\
599.02	2.88733427738237e-05\\
599.03	2.83607887843138e-05\\
599.04	2.78514829045555e-05\\
599.05	2.73454537503409e-05\\
599.06	2.68427302310189e-05\\
599.07	2.6343341552432e-05\\
599.08	2.58473172198743e-05\\
599.09	2.53546870411028e-05\\
599.1	2.4865481129354e-05\\
599.11	2.43797299064024e-05\\
599.12	2.38974641056534e-05\\
599.13	2.34187147752501e-05\\
599.14	2.29435132812292e-05\\
599.15	2.24718913107037e-05\\
599.16	2.20038808750617e-05\\
599.17	2.15395143132284e-05\\
599.18	2.10788242949178e-05\\
599.19	2.06218438239709e-05\\
599.2	2.01686062416655e-05\\
599.21	1.97191452301106e-05\\
599.22	1.92734948156557e-05\\
599.23	1.88316893723148e-05\\
599.24	1.83937636252657e-05\\
599.25	1.79597526543396e-05\\
599.26	1.75296921197578e-05\\
599.27	1.71036209698078e-05\\
599.28	1.66815785538049e-05\\
599.29	1.62636046261155e-05\\
599.3	1.58497393501957e-05\\
599.31	1.54400233026864e-05\\
599.32	1.50344974775372e-05\\
599.33	1.46332032901937e-05\\
599.34	1.42361825817925e-05\\
599.35	1.38434776234316e-05\\
599.36	1.3455131120459e-05\\
599.37	1.30711862168214e-05\\
599.38	1.26916864994445e-05\\
599.39	1.23166760026671e-05\\
599.4	1.19461992127128e-05\\
599.41	1.1580301072206e-05\\
599.42	1.12190269847406e-05\\
599.43	1.08624228194933e-05\\
599.44	1.05105349158703e-05\\
599.45	1.01634100882209e-05\\
599.46	9.82109562895975e-06\\
599.47	9.48363931251028e-06\\
599.48	9.15108940013724e-06\\
599.49	8.82349464480583e-06\\
599.5	8.50090429611351e-06\\
599.51	8.18336810525129e-06\\
599.52	7.8709363300327e-06\\
599.53	7.56365973996094e-06\\
599.54	7.26158962135841e-06\\
599.55	6.9647777825501e-06\\
599.56	6.67327655908337e-06\\
599.57	6.38713881902925e-06\\
599.58	6.10641796831846e-06\\
599.59	5.83116795613466e-06\\
599.6	5.56144328037883e-06\\
599.61	5.2972989931753e-06\\
599.62	5.03879070643844e-06\\
599.63	4.78597459751748e-06\\
599.64	4.53890741486211e-06\\
599.65	4.29764648379564e-06\\
599.66	4.06224971230724e-06\\
599.67	3.83277559694133e-06\\
599.68	3.60928322872336e-06\\
599.69	3.39183229917245e-06\\
599.7	3.18048310636067e-06\\
599.71	2.9752965610471e-06\\
599.72	2.77633419288811e-06\\
599.73	2.58365815670146e-06\\
599.74	2.39733123880842e-06\\
599.75	2.21741686344014e-06\\
599.76	2.04397909923763e-06\\
599.77	1.8770826657917e-06\\
599.78	1.71679294028865e-06\\
599.79	1.56317596421672e-06\\
599.8	1.41629845015583e-06\\
599.81	1.27622778864009e-06\\
599.82	1.14303205511236e-06\\
599.83	1.01678001696026e-06\\
599.84	8.97541140626804e-07\\
599.85	7.8538559881644e-07\\
599.86	6.80384277782636e-07\\
599.87	5.82608784729932e-07\\
599.88	4.92131455266318e-07\\
599.89	4.09025360990911e-07\\
599.9	3.33364317157969e-07\\
599.91	2.65222890446712e-07\\
599.92	2.04676406836968e-07\\
599.93	1.51800959578494e-07\\
599.94	1.06673417288664e-07\\
599.95	6.93714321420985e-08\\
599.96	3.99734481886654e-08\\
599.97	1.85587097807638e-08\\
599.98	5.20727013245126e-09\\
599.99	0\\
600	0\\
};
\addplot [color=mycolor11,solid,forget plot]
  table[row sep=crcr]{%
0.01	0.01\\
1.01	0.01\\
2.01	0.01\\
3.01	0.01\\
4.01	0.01\\
5.01	0.01\\
6.01	0.01\\
7.01	0.01\\
8.01	0.01\\
9.01	0.01\\
10.01	0.01\\
11.01	0.01\\
12.01	0.01\\
13.01	0.01\\
14.01	0.01\\
15.01	0.01\\
16.01	0.01\\
17.01	0.01\\
18.01	0.01\\
19.01	0.01\\
20.01	0.01\\
21.01	0.01\\
22.01	0.01\\
23.01	0.01\\
24.01	0.01\\
25.01	0.01\\
26.01	0.01\\
27.01	0.01\\
28.01	0.01\\
29.01	0.01\\
30.01	0.01\\
31.01	0.01\\
32.01	0.01\\
33.01	0.01\\
34.01	0.01\\
35.01	0.01\\
36.01	0.01\\
37.01	0.01\\
38.01	0.01\\
39.01	0.01\\
40.01	0.01\\
41.01	0.01\\
42.01	0.01\\
43.01	0.01\\
44.01	0.01\\
45.01	0.01\\
46.01	0.01\\
47.01	0.01\\
48.01	0.01\\
49.01	0.01\\
50.01	0.01\\
51.01	0.01\\
52.01	0.01\\
53.01	0.01\\
54.01	0.01\\
55.01	0.01\\
56.01	0.01\\
57.01	0.01\\
58.01	0.01\\
59.01	0.01\\
60.01	0.01\\
61.01	0.01\\
62.01	0.01\\
63.01	0.01\\
64.01	0.01\\
65.01	0.01\\
66.01	0.01\\
67.01	0.01\\
68.01	0.01\\
69.01	0.01\\
70.01	0.01\\
71.01	0.01\\
72.01	0.01\\
73.01	0.01\\
74.01	0.01\\
75.01	0.01\\
76.01	0.01\\
77.01	0.01\\
78.01	0.01\\
79.01	0.01\\
80.01	0.01\\
81.01	0.01\\
82.01	0.01\\
83.01	0.01\\
84.01	0.01\\
85.01	0.01\\
86.01	0.01\\
87.01	0.01\\
88.01	0.01\\
89.01	0.01\\
90.01	0.01\\
91.01	0.01\\
92.01	0.01\\
93.01	0.01\\
94.01	0.01\\
95.01	0.01\\
96.01	0.01\\
97.01	0.01\\
98.01	0.01\\
99.01	0.01\\
100.01	0.01\\
101.01	0.01\\
102.01	0.01\\
103.01	0.01\\
104.01	0.01\\
105.01	0.01\\
106.01	0.01\\
107.01	0.01\\
108.01	0.01\\
109.01	0.01\\
110.01	0.01\\
111.01	0.01\\
112.01	0.01\\
113.01	0.01\\
114.01	0.01\\
115.01	0.01\\
116.01	0.01\\
117.01	0.01\\
118.01	0.01\\
119.01	0.01\\
120.01	0.01\\
121.01	0.01\\
122.01	0.01\\
123.01	0.01\\
124.01	0.01\\
125.01	0.01\\
126.01	0.01\\
127.01	0.01\\
128.01	0.01\\
129.01	0.01\\
130.01	0.01\\
131.01	0.01\\
132.01	0.01\\
133.01	0.01\\
134.01	0.01\\
135.01	0.01\\
136.01	0.01\\
137.01	0.01\\
138.01	0.01\\
139.01	0.01\\
140.01	0.01\\
141.01	0.01\\
142.01	0.01\\
143.01	0.01\\
144.01	0.01\\
145.01	0.01\\
146.01	0.01\\
147.01	0.01\\
148.01	0.01\\
149.01	0.01\\
150.01	0.01\\
151.01	0.01\\
152.01	0.01\\
153.01	0.01\\
154.01	0.01\\
155.01	0.01\\
156.01	0.01\\
157.01	0.01\\
158.01	0.01\\
159.01	0.01\\
160.01	0.01\\
161.01	0.01\\
162.01	0.01\\
163.01	0.01\\
164.01	0.01\\
165.01	0.01\\
166.01	0.01\\
167.01	0.01\\
168.01	0.01\\
169.01	0.01\\
170.01	0.01\\
171.01	0.01\\
172.01	0.01\\
173.01	0.01\\
174.01	0.01\\
175.01	0.01\\
176.01	0.01\\
177.01	0.01\\
178.01	0.01\\
179.01	0.01\\
180.01	0.01\\
181.01	0.01\\
182.01	0.01\\
183.01	0.01\\
184.01	0.01\\
185.01	0.01\\
186.01	0.01\\
187.01	0.01\\
188.01	0.01\\
189.01	0.01\\
190.01	0.01\\
191.01	0.01\\
192.01	0.01\\
193.01	0.01\\
194.01	0.01\\
195.01	0.01\\
196.01	0.01\\
197.01	0.01\\
198.01	0.01\\
199.01	0.01\\
200.01	0.01\\
201.01	0.01\\
202.01	0.01\\
203.01	0.01\\
204.01	0.01\\
205.01	0.01\\
206.01	0.01\\
207.01	0.01\\
208.01	0.01\\
209.01	0.01\\
210.01	0.01\\
211.01	0.01\\
212.01	0.01\\
213.01	0.01\\
214.01	0.01\\
215.01	0.01\\
216.01	0.01\\
217.01	0.01\\
218.01	0.01\\
219.01	0.01\\
220.01	0.01\\
221.01	0.01\\
222.01	0.01\\
223.01	0.01\\
224.01	0.01\\
225.01	0.01\\
226.01	0.01\\
227.01	0.01\\
228.01	0.01\\
229.01	0.01\\
230.01	0.01\\
231.01	0.01\\
232.01	0.01\\
233.01	0.01\\
234.01	0.01\\
235.01	0.01\\
236.01	0.01\\
237.01	0.01\\
238.01	0.01\\
239.01	0.01\\
240.01	0.01\\
241.01	0.01\\
242.01	0.01\\
243.01	0.01\\
244.01	0.01\\
245.01	0.01\\
246.01	0.01\\
247.01	0.01\\
248.01	0.01\\
249.01	0.01\\
250.01	0.01\\
251.01	0.01\\
252.01	0.01\\
253.01	0.01\\
254.01	0.01\\
255.01	0.01\\
256.01	0.01\\
257.01	0.01\\
258.01	0.01\\
259.01	0.01\\
260.01	0.01\\
261.01	0.01\\
262.01	0.01\\
263.01	0.01\\
264.01	0.01\\
265.01	0.01\\
266.01	0.01\\
267.01	0.01\\
268.01	0.01\\
269.01	0.01\\
270.01	0.01\\
271.01	0.01\\
272.01	0.01\\
273.01	0.01\\
274.01	0.01\\
275.01	0.01\\
276.01	0.01\\
277.01	0.01\\
278.01	0.01\\
279.01	0.01\\
280.01	0.01\\
281.01	0.01\\
282.01	0.01\\
283.01	0.01\\
284.01	0.01\\
285.01	0.01\\
286.01	0.01\\
287.01	0.01\\
288.01	0.01\\
289.01	0.01\\
290.01	0.01\\
291.01	0.01\\
292.01	0.01\\
293.01	0.01\\
294.01	0.01\\
295.01	0.01\\
296.01	0.01\\
297.01	0.01\\
298.01	0.01\\
299.01	0.01\\
300.01	0.01\\
301.01	0.01\\
302.01	0.01\\
303.01	0.01\\
304.01	0.01\\
305.01	0.01\\
306.01	0.01\\
307.01	0.01\\
308.01	0.01\\
309.01	0.01\\
310.01	0.01\\
311.01	0.01\\
312.01	0.01\\
313.01	0.01\\
314.01	0.01\\
315.01	0.01\\
316.01	0.01\\
317.01	0.01\\
318.01	0.01\\
319.01	0.01\\
320.01	0.01\\
321.01	0.01\\
322.01	0.01\\
323.01	0.01\\
324.01	0.01\\
325.01	0.01\\
326.01	0.01\\
327.01	0.01\\
328.01	0.01\\
329.01	0.01\\
330.01	0.01\\
331.01	0.01\\
332.01	0.01\\
333.01	0.01\\
334.01	0.01\\
335.01	0.01\\
336.01	0.01\\
337.01	0.01\\
338.01	0.01\\
339.01	0.01\\
340.01	0.01\\
341.01	0.01\\
342.01	0.01\\
343.01	0.01\\
344.01	0.01\\
345.01	0.01\\
346.01	0.01\\
347.01	0.01\\
348.01	0.01\\
349.01	0.01\\
350.01	0.01\\
351.01	0.01\\
352.01	0.01\\
353.01	0.01\\
354.01	0.01\\
355.01	0.01\\
356.01	0.01\\
357.01	0.01\\
358.01	0.01\\
359.01	0.01\\
360.01	0.01\\
361.01	0.01\\
362.01	0.01\\
363.01	0.01\\
364.01	0.01\\
365.01	0.01\\
366.01	0.01\\
367.01	0.01\\
368.01	0.01\\
369.01	0.01\\
370.01	0.01\\
371.01	0.01\\
372.01	0.01\\
373.01	0.01\\
374.01	0.01\\
375.01	0.01\\
376.01	0.01\\
377.01	0.01\\
378.01	0.01\\
379.01	0.01\\
380.01	0.01\\
381.01	0.01\\
382.01	0.01\\
383.01	0.01\\
384.01	0.01\\
385.01	0.01\\
386.01	0.01\\
387.01	0.01\\
388.01	0.01\\
389.01	0.01\\
390.01	0.01\\
391.01	0.01\\
392.01	0.01\\
393.01	0.01\\
394.01	0.01\\
395.01	0.01\\
396.01	0.01\\
397.01	0.01\\
398.01	0.01\\
399.01	0.01\\
400.01	0.01\\
401.01	0.01\\
402.01	0.01\\
403.01	0.01\\
404.01	0.01\\
405.01	0.01\\
406.01	0.01\\
407.01	0.01\\
408.01	0.01\\
409.01	0.01\\
410.01	0.01\\
411.01	0.01\\
412.01	0.01\\
413.01	0.01\\
414.01	0.01\\
415.01	0.01\\
416.01	0.01\\
417.01	0.01\\
418.01	0.01\\
419.01	0.01\\
420.01	0.01\\
421.01	0.01\\
422.01	0.01\\
423.01	0.01\\
424.01	0.01\\
425.01	0.01\\
426.01	0.01\\
427.01	0.01\\
428.01	0.01\\
429.01	0.01\\
430.01	0.01\\
431.01	0.01\\
432.01	0.01\\
433.01	0.01\\
434.01	0.01\\
435.01	0.01\\
436.01	0.01\\
437.01	0.01\\
438.01	0.01\\
439.01	0.01\\
440.01	0.01\\
441.01	0.01\\
442.01	0.01\\
443.01	0.01\\
444.01	0.01\\
445.01	0.01\\
446.01	0.01\\
447.01	0.01\\
448.01	0.01\\
449.01	0.01\\
450.01	0.01\\
451.01	0.01\\
452.01	0.01\\
453.01	0.01\\
454.01	0.01\\
455.01	0.01\\
456.01	0.01\\
457.01	0.01\\
458.01	0.01\\
459.01	0.01\\
460.01	0.01\\
461.01	0.01\\
462.01	0.01\\
463.01	0.01\\
464.01	0.01\\
465.01	0.01\\
466.01	0.01\\
467.01	0.01\\
468.01	0.01\\
469.01	0.01\\
470.01	0.01\\
471.01	0.01\\
472.01	0.01\\
473.01	0.01\\
474.01	0.01\\
475.01	0.01\\
476.01	0.01\\
477.01	0.01\\
478.01	0.01\\
479.01	0.01\\
480.01	0.01\\
481.01	0.01\\
482.01	0.01\\
483.01	0.01\\
484.01	0.01\\
485.01	0.01\\
486.01	0.01\\
487.01	0.01\\
488.01	0.01\\
489.01	0.01\\
490.01	0.01\\
491.01	0.01\\
492.01	0.01\\
493.01	0.01\\
494.01	0.01\\
495.01	0.01\\
496.01	0.01\\
497.01	0.01\\
498.01	0.01\\
499.01	0.01\\
500.01	0.01\\
501.01	0.01\\
502.01	0.01\\
503.01	0.01\\
504.01	0.01\\
505.01	0.01\\
506.01	0.01\\
507.01	0.01\\
508.01	0.01\\
509.01	0.01\\
510.01	0.01\\
511.01	0.01\\
512.01	0.01\\
513.01	0.01\\
514.01	0.01\\
515.01	0.01\\
516.01	0.01\\
517.01	0.01\\
518.01	0.01\\
519.01	0.01\\
520.01	0.01\\
521.01	0.01\\
522.01	0.01\\
523.01	0.01\\
524.01	0.01\\
525.01	0.01\\
526.01	0.01\\
527.01	0.01\\
528.01	0.01\\
529.01	0.01\\
530.01	0.01\\
531.01	0.01\\
532.01	0.01\\
533.01	0.01\\
534.01	0.01\\
535.01	0.01\\
536.01	0.01\\
537.01	0.01\\
538.01	0.01\\
539.01	0.01\\
540.01	0.01\\
541.01	0.01\\
542.01	0.01\\
543.01	0.01\\
544.01	0.01\\
545.01	0.01\\
546.01	0.01\\
547.01	0.01\\
548.01	0.01\\
549.01	0.01\\
550.01	0.01\\
551.01	0.01\\
552.01	0.01\\
553.01	0.01\\
554.01	0.01\\
555.01	0.01\\
556.01	0.01\\
557.01	0.01\\
558.01	0.01\\
559.01	0.01\\
560.01	0.01\\
561.01	0.01\\
562.01	0.01\\
563.01	0.01\\
564.01	0.01\\
565.01	0.01\\
566.01	0.01\\
567.01	0.00989781364243563\\
568.01	0.0096109957314396\\
569.01	0.00930678972726892\\
570.01	0.00898281944686265\\
571.01	0.00863625634746471\\
572.01	0.0082637197398952\\
573.01	0.00786385895341286\\
574.01	0.00744596504207943\\
575.01	0.00701014204382437\\
576.01	0.00655491977496119\\
577.01	0.00607869128502517\\
578.01	0.00557970906787935\\
579.01	0.00505609315699457\\
580.01	0.00450587264837774\\
581.01	0.00392712360941451\\
582.01	0.00344097622170438\\
583.01	0.00320184843714926\\
584.01	0.00297373171794813\\
585.01	0.00275715000119383\\
586.01	0.00254044740799736\\
587.01	0.00232322729506633\\
588.01	0.00210635469391665\\
589.01	0.00189073040228258\\
590.01	0.00167646442262719\\
591.01	0.00146289296784564\\
592.01	0.00125144610456633\\
593.01	0.00104393128886474\\
594.01	0.000842494517544453\\
595.01	0.000649663713740501\\
596.01	0.000468392415796496\\
597.01	0.000302100291734208\\
598.01	0.000154705486090628\\
599.01	4.48417650300119e-05\\
599.02	4.40230021324511e-05\\
599.03	4.3210269553345e-05\\
599.04	4.24036022396417e-05\\
599.05	4.1603035432599e-05\\
599.06	4.08086046704923e-05\\
599.07	4.00203457913435e-05\\
599.08	3.92382949356684e-05\\
599.09	3.84624885492527e-05\\
599.1	3.76929633859428e-05\\
599.11	3.69297565104582e-05\\
599.12	3.61729053012273e-05\\
599.13	3.54224474532448e-05\\
599.14	3.46784209809495e-05\\
599.15	3.39408642211197e-05\\
599.16	3.32098158357941e-05\\
599.17	3.24853148152126e-05\\
599.18	3.1767400480779e-05\\
599.19	3.1056112488036e-05\\
599.2	3.03514908296752e-05\\
599.21	2.96535758385584e-05\\
599.22	2.89624081907525e-05\\
599.23	2.8278028908606e-05\\
599.24	2.76004793638204e-05\\
599.25	2.69298012805595e-05\\
599.26	2.62660367384721e-05\\
599.27	2.5609228174503e-05\\
599.28	2.49594183860512e-05\\
599.29	2.43166505341143e-05\\
599.3	2.36809681464778e-05\\
599.31	2.30524151209142e-05\\
599.32	2.24310357283923e-05\\
599.33	2.18168746163074e-05\\
599.34	2.12099768117386e-05\\
599.35	2.06103877247021e-05\\
599.36	2.00181531514448e-05\\
599.37	1.9433319277732e-05\\
599.38	1.88559326821588e-05\\
599.39	1.82860403394793e-05\\
599.4	1.77236896239403e-05\\
599.41	1.71689283126365e-05\\
599.42	1.66218045888709e-05\\
599.43	1.60823670455352e-05\\
599.44	1.55506646884879e-05\\
599.45	1.50267469399591e-05\\
599.46	1.45106669384547e-05\\
599.47	1.40024799546081e-05\\
599.48	1.35022417351478e-05\\
599.49	1.3010008506769e-05\\
599.5	1.25258369800228e-05\\
599.51	1.20497843531988e-05\\
599.52	1.15819083162261e-05\\
599.53	1.11222670545735e-05\\
599.54	1.06709192531571e-05\\
599.55	1.02279241002319e-05\\
599.56	9.79334129132085e-06\\
599.57	9.36723103309167e-06\\
599.58	8.94965404728103e-06\\
599.59	8.54067157457679e-06\\
599.6	8.14034537850901e-06\\
599.61	7.7487377493253e-06\\
599.62	7.36591150786621e-06\\
599.63	6.99193000940589e-06\\
599.64	6.62685714749797e-06\\
599.65	6.27075735779026e-06\\
599.66	5.92369562180985e-06\\
599.67	5.58573747075006e-06\\
599.68	5.25694898919141e-06\\
599.69	4.93739681882779e-06\\
599.7	4.62714816213197e-06\\
599.71	4.32627078599673e-06\\
599.72	4.03483302531885e-06\\
599.73	3.75290378655697e-06\\
599.74	3.480552551215e-06\\
599.75	3.21784937928893e-06\\
599.76	2.96486491263744e-06\\
599.77	2.72167037830041e-06\\
599.78	2.48833759172902e-06\\
599.79	2.26493895996022e-06\\
599.8	2.05154748467509e-06\\
599.81	1.84823676521034e-06\\
599.82	1.65508100143097e-06\\
599.83	1.47215499651972e-06\\
599.84	1.29953415965375e-06\\
599.85	1.13729450855297e-06\\
599.86	9.8551267190862e-07\\
599.87	8.44265891657495e-07\\
599.88	7.13632025122618e-07\\
599.89	5.93689546994278e-07\\
599.9	4.84517551132407e-07\\
599.91	3.86195752185084e-07\\
599.92	2.98804487031817e-07\\
599.93	2.22424715994388e-07\\
599.94	1.57138023849923e-07\\
599.95	1.0302662059071e-07\\
599.96	6.0173341939404e-08\\
599.97	2.86616496005671e-08\\
599.98	8.57563121035854e-09\\
599.99	0\\
600	0\\
};
\addplot [color=mycolor12,solid,forget plot]
  table[row sep=crcr]{%
0.01	0.01\\
1.01	0.01\\
2.01	0.01\\
3.01	0.01\\
4.01	0.01\\
5.01	0.01\\
6.01	0.01\\
7.01	0.01\\
8.01	0.01\\
9.01	0.01\\
10.01	0.01\\
11.01	0.01\\
12.01	0.01\\
13.01	0.01\\
14.01	0.01\\
15.01	0.01\\
16.01	0.01\\
17.01	0.01\\
18.01	0.01\\
19.01	0.01\\
20.01	0.01\\
21.01	0.01\\
22.01	0.01\\
23.01	0.01\\
24.01	0.01\\
25.01	0.01\\
26.01	0.01\\
27.01	0.01\\
28.01	0.01\\
29.01	0.01\\
30.01	0.01\\
31.01	0.01\\
32.01	0.01\\
33.01	0.01\\
34.01	0.01\\
35.01	0.01\\
36.01	0.01\\
37.01	0.01\\
38.01	0.01\\
39.01	0.01\\
40.01	0.01\\
41.01	0.01\\
42.01	0.01\\
43.01	0.01\\
44.01	0.01\\
45.01	0.01\\
46.01	0.01\\
47.01	0.01\\
48.01	0.01\\
49.01	0.01\\
50.01	0.01\\
51.01	0.01\\
52.01	0.01\\
53.01	0.01\\
54.01	0.01\\
55.01	0.01\\
56.01	0.01\\
57.01	0.01\\
58.01	0.01\\
59.01	0.01\\
60.01	0.01\\
61.01	0.01\\
62.01	0.01\\
63.01	0.01\\
64.01	0.01\\
65.01	0.01\\
66.01	0.01\\
67.01	0.01\\
68.01	0.01\\
69.01	0.01\\
70.01	0.01\\
71.01	0.01\\
72.01	0.01\\
73.01	0.01\\
74.01	0.01\\
75.01	0.01\\
76.01	0.01\\
77.01	0.01\\
78.01	0.01\\
79.01	0.01\\
80.01	0.01\\
81.01	0.01\\
82.01	0.01\\
83.01	0.01\\
84.01	0.01\\
85.01	0.01\\
86.01	0.01\\
87.01	0.01\\
88.01	0.01\\
89.01	0.01\\
90.01	0.01\\
91.01	0.01\\
92.01	0.01\\
93.01	0.01\\
94.01	0.01\\
95.01	0.01\\
96.01	0.01\\
97.01	0.01\\
98.01	0.01\\
99.01	0.01\\
100.01	0.01\\
101.01	0.01\\
102.01	0.01\\
103.01	0.01\\
104.01	0.01\\
105.01	0.01\\
106.01	0.01\\
107.01	0.01\\
108.01	0.01\\
109.01	0.01\\
110.01	0.01\\
111.01	0.01\\
112.01	0.01\\
113.01	0.01\\
114.01	0.01\\
115.01	0.01\\
116.01	0.01\\
117.01	0.01\\
118.01	0.01\\
119.01	0.01\\
120.01	0.01\\
121.01	0.01\\
122.01	0.01\\
123.01	0.01\\
124.01	0.01\\
125.01	0.01\\
126.01	0.01\\
127.01	0.01\\
128.01	0.01\\
129.01	0.01\\
130.01	0.01\\
131.01	0.01\\
132.01	0.01\\
133.01	0.01\\
134.01	0.01\\
135.01	0.01\\
136.01	0.01\\
137.01	0.01\\
138.01	0.01\\
139.01	0.01\\
140.01	0.01\\
141.01	0.01\\
142.01	0.01\\
143.01	0.01\\
144.01	0.01\\
145.01	0.01\\
146.01	0.01\\
147.01	0.01\\
148.01	0.01\\
149.01	0.01\\
150.01	0.01\\
151.01	0.01\\
152.01	0.01\\
153.01	0.01\\
154.01	0.01\\
155.01	0.01\\
156.01	0.01\\
157.01	0.01\\
158.01	0.01\\
159.01	0.01\\
160.01	0.01\\
161.01	0.01\\
162.01	0.01\\
163.01	0.01\\
164.01	0.01\\
165.01	0.01\\
166.01	0.01\\
167.01	0.01\\
168.01	0.01\\
169.01	0.01\\
170.01	0.01\\
171.01	0.01\\
172.01	0.01\\
173.01	0.01\\
174.01	0.01\\
175.01	0.01\\
176.01	0.01\\
177.01	0.01\\
178.01	0.01\\
179.01	0.01\\
180.01	0.01\\
181.01	0.01\\
182.01	0.01\\
183.01	0.01\\
184.01	0.01\\
185.01	0.01\\
186.01	0.01\\
187.01	0.01\\
188.01	0.01\\
189.01	0.01\\
190.01	0.01\\
191.01	0.01\\
192.01	0.01\\
193.01	0.01\\
194.01	0.01\\
195.01	0.01\\
196.01	0.01\\
197.01	0.01\\
198.01	0.01\\
199.01	0.01\\
200.01	0.01\\
201.01	0.01\\
202.01	0.01\\
203.01	0.01\\
204.01	0.01\\
205.01	0.01\\
206.01	0.01\\
207.01	0.01\\
208.01	0.01\\
209.01	0.01\\
210.01	0.01\\
211.01	0.01\\
212.01	0.01\\
213.01	0.01\\
214.01	0.01\\
215.01	0.01\\
216.01	0.01\\
217.01	0.01\\
218.01	0.01\\
219.01	0.01\\
220.01	0.01\\
221.01	0.01\\
222.01	0.01\\
223.01	0.01\\
224.01	0.01\\
225.01	0.01\\
226.01	0.01\\
227.01	0.01\\
228.01	0.01\\
229.01	0.01\\
230.01	0.01\\
231.01	0.01\\
232.01	0.01\\
233.01	0.01\\
234.01	0.01\\
235.01	0.01\\
236.01	0.01\\
237.01	0.01\\
238.01	0.01\\
239.01	0.01\\
240.01	0.01\\
241.01	0.01\\
242.01	0.01\\
243.01	0.01\\
244.01	0.01\\
245.01	0.01\\
246.01	0.01\\
247.01	0.01\\
248.01	0.01\\
249.01	0.01\\
250.01	0.01\\
251.01	0.01\\
252.01	0.01\\
253.01	0.01\\
254.01	0.01\\
255.01	0.01\\
256.01	0.01\\
257.01	0.01\\
258.01	0.01\\
259.01	0.01\\
260.01	0.01\\
261.01	0.01\\
262.01	0.01\\
263.01	0.01\\
264.01	0.01\\
265.01	0.01\\
266.01	0.01\\
267.01	0.01\\
268.01	0.01\\
269.01	0.01\\
270.01	0.01\\
271.01	0.01\\
272.01	0.01\\
273.01	0.01\\
274.01	0.01\\
275.01	0.01\\
276.01	0.01\\
277.01	0.01\\
278.01	0.01\\
279.01	0.01\\
280.01	0.01\\
281.01	0.01\\
282.01	0.01\\
283.01	0.01\\
284.01	0.01\\
285.01	0.01\\
286.01	0.01\\
287.01	0.01\\
288.01	0.01\\
289.01	0.01\\
290.01	0.01\\
291.01	0.01\\
292.01	0.01\\
293.01	0.01\\
294.01	0.01\\
295.01	0.01\\
296.01	0.01\\
297.01	0.01\\
298.01	0.01\\
299.01	0.01\\
300.01	0.01\\
301.01	0.01\\
302.01	0.01\\
303.01	0.01\\
304.01	0.01\\
305.01	0.01\\
306.01	0.01\\
307.01	0.01\\
308.01	0.01\\
309.01	0.01\\
310.01	0.01\\
311.01	0.01\\
312.01	0.01\\
313.01	0.01\\
314.01	0.01\\
315.01	0.01\\
316.01	0.01\\
317.01	0.01\\
318.01	0.01\\
319.01	0.01\\
320.01	0.01\\
321.01	0.01\\
322.01	0.01\\
323.01	0.01\\
324.01	0.01\\
325.01	0.01\\
326.01	0.01\\
327.01	0.01\\
328.01	0.01\\
329.01	0.01\\
330.01	0.01\\
331.01	0.01\\
332.01	0.01\\
333.01	0.01\\
334.01	0.01\\
335.01	0.01\\
336.01	0.01\\
337.01	0.01\\
338.01	0.01\\
339.01	0.01\\
340.01	0.01\\
341.01	0.01\\
342.01	0.01\\
343.01	0.01\\
344.01	0.01\\
345.01	0.01\\
346.01	0.01\\
347.01	0.01\\
348.01	0.01\\
349.01	0.01\\
350.01	0.01\\
351.01	0.01\\
352.01	0.01\\
353.01	0.01\\
354.01	0.01\\
355.01	0.01\\
356.01	0.01\\
357.01	0.01\\
358.01	0.01\\
359.01	0.01\\
360.01	0.01\\
361.01	0.01\\
362.01	0.01\\
363.01	0.01\\
364.01	0.01\\
365.01	0.01\\
366.01	0.01\\
367.01	0.01\\
368.01	0.01\\
369.01	0.01\\
370.01	0.01\\
371.01	0.01\\
372.01	0.01\\
373.01	0.01\\
374.01	0.01\\
375.01	0.01\\
376.01	0.01\\
377.01	0.01\\
378.01	0.01\\
379.01	0.01\\
380.01	0.01\\
381.01	0.01\\
382.01	0.01\\
383.01	0.01\\
384.01	0.01\\
385.01	0.01\\
386.01	0.01\\
387.01	0.01\\
388.01	0.01\\
389.01	0.01\\
390.01	0.01\\
391.01	0.01\\
392.01	0.01\\
393.01	0.01\\
394.01	0.01\\
395.01	0.01\\
396.01	0.01\\
397.01	0.01\\
398.01	0.01\\
399.01	0.01\\
400.01	0.01\\
401.01	0.01\\
402.01	0.01\\
403.01	0.01\\
404.01	0.01\\
405.01	0.01\\
406.01	0.01\\
407.01	0.01\\
408.01	0.01\\
409.01	0.01\\
410.01	0.01\\
411.01	0.01\\
412.01	0.01\\
413.01	0.01\\
414.01	0.01\\
415.01	0.01\\
416.01	0.01\\
417.01	0.01\\
418.01	0.01\\
419.01	0.01\\
420.01	0.01\\
421.01	0.01\\
422.01	0.01\\
423.01	0.01\\
424.01	0.01\\
425.01	0.01\\
426.01	0.01\\
427.01	0.01\\
428.01	0.01\\
429.01	0.01\\
430.01	0.01\\
431.01	0.01\\
432.01	0.01\\
433.01	0.01\\
434.01	0.01\\
435.01	0.01\\
436.01	0.01\\
437.01	0.01\\
438.01	0.01\\
439.01	0.01\\
440.01	0.01\\
441.01	0.01\\
442.01	0.01\\
443.01	0.01\\
444.01	0.01\\
445.01	0.01\\
446.01	0.01\\
447.01	0.01\\
448.01	0.01\\
449.01	0.01\\
450.01	0.01\\
451.01	0.01\\
452.01	0.01\\
453.01	0.01\\
454.01	0.01\\
455.01	0.01\\
456.01	0.01\\
457.01	0.01\\
458.01	0.01\\
459.01	0.01\\
460.01	0.01\\
461.01	0.01\\
462.01	0.01\\
463.01	0.01\\
464.01	0.01\\
465.01	0.01\\
466.01	0.01\\
467.01	0.01\\
468.01	0.01\\
469.01	0.01\\
470.01	0.01\\
471.01	0.01\\
472.01	0.01\\
473.01	0.01\\
474.01	0.01\\
475.01	0.01\\
476.01	0.01\\
477.01	0.01\\
478.01	0.01\\
479.01	0.01\\
480.01	0.01\\
481.01	0.01\\
482.01	0.01\\
483.01	0.01\\
484.01	0.01\\
485.01	0.01\\
486.01	0.01\\
487.01	0.01\\
488.01	0.01\\
489.01	0.01\\
490.01	0.01\\
491.01	0.01\\
492.01	0.01\\
493.01	0.01\\
494.01	0.01\\
495.01	0.01\\
496.01	0.01\\
497.01	0.01\\
498.01	0.01\\
499.01	0.01\\
500.01	0.01\\
501.01	0.01\\
502.01	0.01\\
503.01	0.01\\
504.01	0.01\\
505.01	0.01\\
506.01	0.01\\
507.01	0.01\\
508.01	0.01\\
509.01	0.01\\
510.01	0.01\\
511.01	0.01\\
512.01	0.01\\
513.01	0.01\\
514.01	0.01\\
515.01	0.01\\
516.01	0.01\\
517.01	0.01\\
518.01	0.01\\
519.01	0.01\\
520.01	0.01\\
521.01	0.01\\
522.01	0.01\\
523.01	0.01\\
524.01	0.01\\
525.01	0.01\\
526.01	0.01\\
527.01	0.01\\
528.01	0.01\\
529.01	0.01\\
530.01	0.01\\
531.01	0.01\\
532.01	0.01\\
533.01	0.01\\
534.01	0.01\\
535.01	0.01\\
536.01	0.01\\
537.01	0.01\\
538.01	0.01\\
539.01	0.01\\
540.01	0.01\\
541.01	0.01\\
542.01	0.01\\
543.01	0.01\\
544.01	0.01\\
545.01	0.01\\
546.01	0.01\\
547.01	0.01\\
548.01	0.01\\
549.01	0.01\\
550.01	0.01\\
551.01	0.01\\
552.01	0.01\\
553.01	0.01\\
554.01	0.01\\
555.01	0.01\\
556.01	0.01\\
557.01	0.01\\
558.01	0.01\\
559.01	0.01\\
560.01	0.01\\
561.01	0.01\\
562.01	0.01\\
563.01	0.01\\
564.01	0.01\\
565.01	0.01\\
566.01	0.01\\
567.01	0.01\\
568.01	0.01\\
569.01	0.01\\
570.01	0.01\\
571.01	0.01\\
572.01	0.01\\
573.01	0.01\\
574.01	0.01\\
575.01	0.01\\
576.01	0.01\\
577.01	0.01\\
578.01	0.01\\
579.01	0.01\\
580.01	0.01\\
581.01	0.01\\
582.01	0.00987863839394233\\
583.01	0.00949082414842157\\
584.01	0.00907610149477576\\
585.01	0.00863516110198265\\
586.01	0.00818029576360015\\
587.01	0.00771162108760121\\
588.01	0.00722758388451855\\
589.01	0.00672698722609763\\
590.01	0.00620999025475\\
591.01	0.0056775054428497\\
592.01	0.00512831674646623\\
593.01	0.0045611266017166\\
594.01	0.00397455252111775\\
595.01	0.00336704583775623\\
596.01	0.00273686338392287\\
597.01	0.00208203186321775\\
598.01	0.00140030197562107\\
599.01	0.000693147751672453\\
599.02	0.000686044700991121\\
599.03	0.000678942362288212\\
599.04	0.000671840766349614\\
599.05	0.000664739944319878\\
599.06	0.000657639927706705\\
599.07	0.000650540748385492\\
599.08	0.000643442438603979\\
599.09	0.000636345030986921\\
599.1	0.000629248558540869\\
599.11	0.000622153054659002\\
599.12	0.00061505855312604\\
599.13	0.000607965088123233\\
599.14	0.000600872694233425\\
599.15	0.000593781406446192\\
599.16	0.000586691260163077\\
599.17	0.000579602291202875\\
599.18	0.000572514535807034\\
599.19	0.00056542803064512\\
599.2	0.000558342812820379\\
599.21	0.000551258919875373\\
599.22	0.000544176389797724\\
599.23	0.000537095261025937\\
599.24	0.000530015572455318\\
599.25	0.00052293736344399\\
599.26	0.000515860673819007\\
599.27	0.000508785543882569\\
599.28	0.000501712014418322\\
599.29	0.000494640126697794\\
599.3	0.000487569922486903\\
599.31	0.000480501444052593\\
599.32	0.000473434734169572\\
599.33	0.000466369836127174\\
599.34	0.000459306793736311\\
599.35	0.000452245651336575\\
599.36	0.000445186453803434\\
599.37	0.000438129246555563\\
599.38	0.000431074075562301\\
599.39	0.000424020987351227\\
599.4	0.000416970029015879\\
599.41	0.000409921248223602\\
599.42	0.000402874693223528\\
599.43	0.000395830412854702\\
599.44	0.000388788456554356\\
599.45	0.000381748874366322\\
599.46	0.000374711716948105\\
599.47	0.000367677035578842\\
599.48	0.000360644882168144\\
599.49	0.000353615309265124\\
599.5	0.000346588370067558\\
599.51	0.000339564118431257\\
599.52	0.00033254260887957\\
599.53	0.000325523896613097\\
599.54	0.000318508037519561\\
599.55	0.000311495088183889\\
599.56	0.000304485105898459\\
599.57	0.00029747814867357\\
599.58	0.000290474275248089\\
599.59	0.000283473545100324\\
599.6	0.000276476018459097\\
599.61	0.00026948175631504\\
599.62	0.000262490820432117\\
599.63	0.000255503273359365\\
599.64	0.000248519178442886\\
599.65	0.000241538599838066\\
599.66	0.000234561602522058\\
599.67	0.000227588252306508\\
599.68	0.000220618615850559\\
599.69	0.0002136527606741\\
599.7	0.000206690755171332\\
599.71	0.00019973266862458\\
599.72	0.000192778571218435\\
599.73	0.000185828534054166\\
599.74	0.000178882629164484\\
599.75	0.000171940929528592\\
599.76	0.000165003509087581\\
599.77	0.000158070442760182\\
599.78	0.000151141806458848\\
599.79	0.000144217677106206\\
599.8	0.000137298132651904\\
599.81	0.00013038325208981\\
599.82	0.000123473115475644\\
599.83	0.000116567803945002\\
599.84	0.000109667399731816\\
599.85	0.000102771986187248\\
599.86	9.58816477990444e-05\\
599.87	8.89964702113586e-05\\
599.88	8.21165402450641e-05\\
599.89	7.52419459185572e-05\\
599.9	6.83727764690997e-05\\
599.91	6.15091223746932e-05\\
599.92	5.46510753764995e-05\\
599.93	4.77987285018664e-05\\
599.94	4.09521760879206e-05\\
599.95	3.4111513805812e-05\\
599.96	2.72768386855859e-05\\
599.97	2.04482491417222e-05\\
599.98	1.36258449993844e-05\\
599.99	6.80972752137125e-06\\
600	0\\
};
\addplot [color=mycolor13,solid,forget plot]
  table[row sep=crcr]{%
0.01	0\\
1.01	0\\
2.01	0\\
3.01	0\\
4.01	0\\
5.01	0\\
6.01	0\\
7.01	0\\
8.01	0\\
9.01	0\\
10.01	0\\
11.01	0\\
12.01	0\\
13.01	0\\
14.01	0\\
15.01	0\\
16.01	0\\
17.01	0\\
18.01	0\\
19.01	0\\
20.01	0\\
21.01	0\\
22.01	0\\
23.01	0\\
24.01	0\\
25.01	0\\
26.01	0\\
27.01	0\\
28.01	0\\
29.01	0\\
30.01	0\\
31.01	0\\
32.01	0\\
33.01	0\\
34.01	0\\
35.01	0\\
36.01	0\\
37.01	0\\
38.01	0\\
39.01	0\\
40.01	0\\
41.01	0\\
42.01	0\\
43.01	0\\
44.01	0\\
45.01	0\\
46.01	0\\
47.01	0\\
48.01	0\\
49.01	0\\
50.01	0\\
51.01	0\\
52.01	0\\
53.01	0\\
54.01	0\\
55.01	0\\
56.01	0\\
57.01	0\\
58.01	0\\
59.01	0\\
60.01	0\\
61.01	0\\
62.01	0\\
63.01	0\\
64.01	0\\
65.01	0\\
66.01	0\\
67.01	0\\
68.01	0\\
69.01	0\\
70.01	0\\
71.01	0\\
72.01	0\\
73.01	0\\
74.01	0\\
75.01	0\\
76.01	0\\
77.01	0\\
78.01	0\\
79.01	0\\
80.01	0\\
81.01	0\\
82.01	0\\
83.01	0\\
84.01	0\\
85.01	0\\
86.01	0\\
87.01	0\\
88.01	0\\
89.01	0\\
90.01	0\\
91.01	0\\
92.01	0\\
93.01	0\\
94.01	0\\
95.01	0\\
96.01	0\\
97.01	0\\
98.01	0\\
99.01	0\\
100.01	0\\
101.01	0\\
102.01	0\\
103.01	0\\
104.01	0\\
105.01	0\\
106.01	0\\
107.01	0\\
108.01	0\\
109.01	0\\
110.01	0\\
111.01	0\\
112.01	0\\
113.01	0\\
114.01	0\\
115.01	0\\
116.01	0\\
117.01	0\\
118.01	0\\
119.01	0\\
120.01	0\\
121.01	0\\
122.01	0\\
123.01	0\\
124.01	0\\
125.01	0\\
126.01	0\\
127.01	0\\
128.01	0\\
129.01	0\\
130.01	0\\
131.01	0\\
132.01	0\\
133.01	0\\
134.01	0\\
135.01	0\\
136.01	0\\
137.01	0\\
138.01	0\\
139.01	0\\
140.01	0\\
141.01	0\\
142.01	0\\
143.01	0\\
144.01	0\\
145.01	0\\
146.01	0\\
147.01	0\\
148.01	0\\
149.01	0\\
150.01	0\\
151.01	0\\
152.01	0\\
153.01	0\\
154.01	0\\
155.01	0\\
156.01	0\\
157.01	0\\
158.01	0\\
159.01	0\\
160.01	0\\
161.01	0\\
162.01	0\\
163.01	0\\
164.01	0\\
165.01	0\\
166.01	0\\
167.01	0\\
168.01	0\\
169.01	0\\
170.01	0\\
171.01	0\\
172.01	0\\
173.01	0\\
174.01	0\\
175.01	0\\
176.01	0\\
177.01	0\\
178.01	0\\
179.01	0\\
180.01	0\\
181.01	0\\
182.01	0\\
183.01	0\\
184.01	0\\
185.01	0\\
186.01	0\\
187.01	0\\
188.01	0\\
189.01	0\\
190.01	0\\
191.01	0\\
192.01	0\\
193.01	0\\
194.01	0\\
195.01	0\\
196.01	0\\
197.01	0\\
198.01	0\\
199.01	0\\
200.01	0\\
201.01	0\\
202.01	0\\
203.01	0\\
204.01	0\\
205.01	0\\
206.01	0\\
207.01	0\\
208.01	0\\
209.01	0\\
210.01	0\\
211.01	0\\
212.01	0\\
213.01	0\\
214.01	0\\
215.01	0\\
216.01	0\\
217.01	0\\
218.01	0\\
219.01	0\\
220.01	0\\
221.01	0\\
222.01	0\\
223.01	0\\
224.01	0\\
225.01	0\\
226.01	0\\
227.01	0\\
228.01	0\\
229.01	0\\
230.01	0\\
231.01	0\\
232.01	0\\
233.01	0\\
234.01	0\\
235.01	0\\
236.01	0\\
237.01	0\\
238.01	0\\
239.01	0\\
240.01	0\\
241.01	0\\
242.01	0\\
243.01	0\\
244.01	0\\
245.01	0\\
246.01	0\\
247.01	0\\
248.01	0\\
249.01	0\\
250.01	0\\
251.01	0\\
252.01	0\\
253.01	0\\
254.01	0\\
255.01	0\\
256.01	0\\
257.01	0\\
258.01	0\\
259.01	0\\
260.01	0\\
261.01	0\\
262.01	0\\
263.01	0\\
264.01	0\\
265.01	0\\
266.01	0\\
267.01	0\\
268.01	0\\
269.01	0\\
270.01	0\\
271.01	0\\
272.01	0\\
273.01	0\\
274.01	0\\
275.01	0\\
276.01	0\\
277.01	0\\
278.01	0\\
279.01	0\\
280.01	0\\
281.01	0\\
282.01	0\\
283.01	0\\
284.01	0\\
285.01	0\\
286.01	0\\
287.01	0\\
288.01	0\\
289.01	0\\
290.01	0\\
291.01	0\\
292.01	0\\
293.01	0\\
294.01	0\\
295.01	0\\
296.01	0\\
297.01	0\\
298.01	0\\
299.01	0\\
300.01	0\\
301.01	0\\
302.01	0\\
303.01	0\\
304.01	0\\
305.01	0\\
306.01	0\\
307.01	0\\
308.01	0\\
309.01	0\\
310.01	0\\
311.01	0\\
312.01	0\\
313.01	0\\
314.01	0\\
315.01	0\\
316.01	0\\
317.01	0\\
318.01	0\\
319.01	0\\
320.01	0\\
321.01	0\\
322.01	0\\
323.01	0\\
324.01	0\\
325.01	0\\
326.01	0\\
327.01	0\\
328.01	0\\
329.01	0\\
330.01	0\\
331.01	0\\
332.01	0\\
333.01	0\\
334.01	0\\
335.01	0\\
336.01	0\\
337.01	0\\
338.01	0\\
339.01	0\\
340.01	0\\
341.01	0\\
342.01	0\\
343.01	0\\
344.01	0\\
345.01	0\\
346.01	0\\
347.01	0\\
348.01	0\\
349.01	0\\
350.01	0\\
351.01	0\\
352.01	0\\
353.01	0\\
354.01	0\\
355.01	0\\
356.01	0\\
357.01	0\\
358.01	0\\
359.01	0\\
360.01	0\\
361.01	0\\
362.01	0\\
363.01	0\\
364.01	0\\
365.01	0\\
366.01	0\\
367.01	0\\
368.01	0\\
369.01	0\\
370.01	0\\
371.01	0\\
372.01	0\\
373.01	0\\
374.01	0\\
375.01	0\\
376.01	0\\
377.01	0\\
378.01	0\\
379.01	0\\
380.01	0\\
381.01	0\\
382.01	0\\
383.01	0\\
384.01	0\\
385.01	0\\
386.01	0\\
387.01	0\\
388.01	0\\
389.01	0\\
390.01	0\\
391.01	0\\
392.01	0\\
393.01	0\\
394.01	0\\
395.01	0\\
396.01	0\\
397.01	0\\
398.01	0\\
399.01	0\\
400.01	0\\
401.01	0\\
402.01	0\\
403.01	0\\
404.01	0\\
405.01	0\\
406.01	0\\
407.01	0\\
408.01	0\\
409.01	0\\
410.01	0\\
411.01	0\\
412.01	0\\
413.01	0\\
414.01	0\\
415.01	0\\
416.01	0\\
417.01	0\\
418.01	0\\
419.01	0\\
420.01	0\\
421.01	0\\
422.01	0\\
423.01	0\\
424.01	0\\
425.01	0\\
426.01	0\\
427.01	0\\
428.01	0\\
429.01	0\\
430.01	0\\
431.01	0\\
432.01	0\\
433.01	0\\
434.01	0\\
435.01	0\\
436.01	0\\
437.01	0\\
438.01	0\\
439.01	0\\
440.01	0\\
441.01	0\\
442.01	0\\
443.01	0\\
444.01	0\\
445.01	0\\
446.01	0\\
447.01	0\\
448.01	0\\
449.01	0\\
450.01	0\\
451.01	0\\
452.01	0\\
453.01	0\\
454.01	0\\
455.01	0\\
456.01	0\\
457.01	0\\
458.01	0\\
459.01	0\\
460.01	0\\
461.01	0\\
462.01	0\\
463.01	0\\
464.01	0\\
465.01	0\\
466.01	0\\
467.01	0\\
468.01	0\\
469.01	0\\
470.01	0\\
471.01	0\\
472.01	0\\
473.01	0\\
474.01	0\\
475.01	0\\
476.01	0\\
477.01	0\\
478.01	0\\
479.01	0\\
480.01	0\\
481.01	0\\
482.01	0\\
483.01	0\\
484.01	0\\
485.01	0\\
486.01	0\\
487.01	0\\
488.01	0\\
489.01	0\\
490.01	0\\
491.01	0\\
492.01	0\\
493.01	0\\
494.01	0\\
495.01	0\\
496.01	0\\
497.01	0\\
498.01	0\\
499.01	0\\
500.01	0\\
501.01	0\\
502.01	0\\
503.01	0\\
504.01	0\\
505.01	0\\
506.01	0\\
507.01	0\\
508.01	0\\
509.01	0\\
510.01	0\\
511.01	0\\
512.01	0\\
513.01	0\\
514.01	0\\
515.01	0\\
516.01	0\\
517.01	0\\
518.01	0\\
519.01	0\\
520.01	0\\
521.01	0\\
522.01	0\\
523.01	0\\
524.01	0\\
525.01	0\\
526.01	0\\
527.01	0\\
528.01	0\\
529.01	0\\
530.01	0\\
531.01	0\\
532.01	0\\
533.01	0\\
534.01	0\\
535.01	0\\
536.01	0\\
537.01	0\\
538.01	0\\
539.01	0\\
540.01	0\\
541.01	0\\
542.01	0\\
543.01	0\\
544.01	0\\
545.01	0\\
546.01	0\\
547.01	0\\
548.01	0\\
549.01	0\\
550.01	0\\
551.01	0\\
552.01	0\\
553.01	0\\
554.01	0\\
555.01	0\\
556.01	0\\
557.01	0\\
558.01	0\\
559.01	0\\
560.01	0\\
561.01	0\\
562.01	0\\
563.01	0\\
564.01	0\\
565.01	0\\
566.01	0\\
567.01	0\\
568.01	0\\
569.01	0\\
570.01	0\\
571.01	0\\
572.01	0\\
573.01	0\\
574.01	0\\
575.01	0\\
576.01	0\\
577.01	0\\
578.01	0\\
579.01	0\\
580.01	0\\
581.01	0\\
582.01	7.0103522963872e-05\\
583.01	0.000458226270178982\\
584.01	0.00087452273602337\\
585.01	0.00131560487810147\\
586.01	0.00177015318791887\\
587.01	0.00223901543772628\\
588.01	0.00272381225844904\\
589.01	0.00322575052278649\\
590.01	0.00374460846515789\\
591.01	0.00427944286429531\\
592.01	0.00483141274336544\\
593.01	0.00540178309593597\\
594.01	0.00599196228625841\\
595.01	0.00660353070603266\\
596.01	0.00723827207202806\\
597.01	0.00789821384338649\\
598.01	0.00858568062740413\\
599.01	0.00929935695921623\\
599.02	0.00930653081533247\\
599.03	0.00931370403649347\\
599.04	0.00932087659239074\\
599.05	0.00932804845236433\\
599.06	0.00933521958539841\\
599.07	0.00934238996011689\\
599.08	0.00934955954477884\\
599.09	0.00935672830727394\\
599.1	0.0093638962151178\\
599.11	0.00937106323544725\\
599.12	0.00937822933501552\\
599.13	0.00938539448018738\\
599.14	0.00939255863693421\\
599.15	0.00939972177082893\\
599.16	0.00940688384704095\\
599.17	0.00941404483033095\\
599.18	0.00942120468504563\\
599.19	0.00942836337511238\\
599.2	0.00943552086403382\\
599.21	0.00944267711488231\\
599.22	0.00944983209029436\\
599.23	0.0094569857524649\\
599.24	0.00946413806314154\\
599.25	0.00947128898361864\\
599.26	0.00947843847473141\\
599.27	0.00948558649684979\\
599.28	0.00949273300987232\\
599.29	0.00949987797321982\\
599.3	0.0095070213458291\\
599.31	0.00951416308614641\\
599.32	0.00952130315212091\\
599.33	0.00952844150119796\\
599.34	0.00953557809031229\\
599.35	0.00954271287588114\\
599.36	0.00954984581379719\\
599.37	0.0095569768594214\\
599.38	0.00956410596757575\\
599.39	0.00957123309253585\\
599.4	0.00957835818802338\\
599.41	0.00958548120719844\\
599.42	0.00959260210265181\\
599.43	0.00959972082639694\\
599.44	0.00960683732986194\\
599.45	0.00961395156388134\\
599.46	0.00962106347868781\\
599.47	0.00962817302390588\\
599.48	0.00963528014854335\\
599.49	0.00964238480098245\\
599.5	0.0096494869289709\\
599.51	0.00965658647961283\\
599.52	0.00966368339935945\\
599.53	0.00967077763399964\\
599.54	0.00967786912865027\\
599.55	0.00968495782774643\\
599.56	0.0096920436750314\\
599.57	0.00969912661354648\\
599.58	0.0097062065856206\\
599.59	0.00971328353285974\\
599.6	0.00972035739613614\\
599.61	0.00972742811557727\\
599.62	0.00973449563055469\\
599.63	0.00974155987967251\\
599.64	0.0097486208007558\\
599.65	0.00975567833083868\\
599.66	0.00976273240615213\\
599.67	0.00976978296211165\\
599.68	0.0097768299333046\\
599.69	0.00978387325347729\\
599.7	0.0097909128555218\\
599.71	0.00979794867146256\\
599.72	0.00980498063244257\\
599.73	0.00981200866870942\\
599.74	0.0098190327096009\\
599.75	0.00982605268353043\\
599.76	0.00983306851797204\\
599.77	0.0098400801394451\\
599.78	0.0098470874734987\\
599.79	0.00985409044469565\\
599.8	0.00986108897659616\\
599.81	0.0098680829917411\\
599.82	0.00987507241163493\\
599.83	0.00988205715672815\\
599.84	0.00988903714639948\\
599.85	0.00989601229893744\\
599.86	0.00990298253152164\\
599.87	0.00990994776020354\\
599.88	0.00991690789988676\\
599.89	0.00992386286430693\\
599.9	0.009930812566011\\
599.91	0.00993775691633607\\
599.92	0.00994469582538763\\
599.93	0.00995162920201733\\
599.94	0.00995855695380011\\
599.95	0.00996547898701074\\
599.96	0.00997239520659974\\
599.97	0.00997930551616874\\
599.98	0.00998620981794502\\
599.99	0.00999310801275551\\
600	0.01\\
};
\addplot [color=mycolor14,solid,forget plot]
  table[row sep=crcr]{%
0.01	0\\
1.01	0\\
2.01	0\\
3.01	0\\
4.01	0\\
5.01	0\\
6.01	0\\
7.01	0\\
8.01	0\\
9.01	0\\
10.01	0\\
11.01	0\\
12.01	0\\
13.01	0\\
14.01	0\\
15.01	0\\
16.01	0\\
17.01	0\\
18.01	0\\
19.01	0\\
20.01	0\\
21.01	0\\
22.01	0\\
23.01	0\\
24.01	0\\
25.01	0\\
26.01	0\\
27.01	0\\
28.01	0\\
29.01	0\\
30.01	0\\
31.01	0\\
32.01	0\\
33.01	0\\
34.01	0\\
35.01	0\\
36.01	0\\
37.01	0\\
38.01	0\\
39.01	0\\
40.01	0\\
41.01	0\\
42.01	0\\
43.01	0\\
44.01	0\\
45.01	0\\
46.01	0\\
47.01	0\\
48.01	0\\
49.01	0\\
50.01	0\\
51.01	0\\
52.01	0\\
53.01	0\\
54.01	0\\
55.01	0\\
56.01	0\\
57.01	0\\
58.01	0\\
59.01	0\\
60.01	0\\
61.01	0\\
62.01	0\\
63.01	0\\
64.01	0\\
65.01	0\\
66.01	0\\
67.01	0\\
68.01	0\\
69.01	0\\
70.01	0\\
71.01	0\\
72.01	0\\
73.01	0\\
74.01	0\\
75.01	0\\
76.01	0\\
77.01	0\\
78.01	0\\
79.01	0\\
80.01	0\\
81.01	0\\
82.01	0\\
83.01	0\\
84.01	0\\
85.01	0\\
86.01	0\\
87.01	0\\
88.01	0\\
89.01	0\\
90.01	0\\
91.01	0\\
92.01	0\\
93.01	0\\
94.01	0\\
95.01	0\\
96.01	0\\
97.01	0\\
98.01	0\\
99.01	0\\
100.01	0\\
101.01	0\\
102.01	0\\
103.01	0\\
104.01	0\\
105.01	0\\
106.01	0\\
107.01	0\\
108.01	0\\
109.01	0\\
110.01	0\\
111.01	0\\
112.01	0\\
113.01	0\\
114.01	0\\
115.01	0\\
116.01	0\\
117.01	0\\
118.01	0\\
119.01	0\\
120.01	0\\
121.01	0\\
122.01	0\\
123.01	0\\
124.01	0\\
125.01	0\\
126.01	0\\
127.01	0\\
128.01	0\\
129.01	0\\
130.01	0\\
131.01	0\\
132.01	0\\
133.01	0\\
134.01	0\\
135.01	0\\
136.01	0\\
137.01	0\\
138.01	0\\
139.01	0\\
140.01	0\\
141.01	0\\
142.01	0\\
143.01	0\\
144.01	0\\
145.01	0\\
146.01	0\\
147.01	0\\
148.01	0\\
149.01	0\\
150.01	0\\
151.01	0\\
152.01	0\\
153.01	0\\
154.01	0\\
155.01	0\\
156.01	0\\
157.01	0\\
158.01	0\\
159.01	0\\
160.01	0\\
161.01	0\\
162.01	0\\
163.01	0\\
164.01	0\\
165.01	0\\
166.01	0\\
167.01	0\\
168.01	0\\
169.01	0\\
170.01	0\\
171.01	0\\
172.01	0\\
173.01	0\\
174.01	0\\
175.01	0\\
176.01	0\\
177.01	0\\
178.01	0\\
179.01	0\\
180.01	0\\
181.01	0\\
182.01	0\\
183.01	0\\
184.01	0\\
185.01	0\\
186.01	0\\
187.01	0\\
188.01	0\\
189.01	0\\
190.01	0\\
191.01	0\\
192.01	0\\
193.01	0\\
194.01	0\\
195.01	0\\
196.01	0\\
197.01	0\\
198.01	0\\
199.01	0\\
200.01	0\\
201.01	0\\
202.01	0\\
203.01	0\\
204.01	0\\
205.01	0\\
206.01	0\\
207.01	0\\
208.01	0\\
209.01	0\\
210.01	0\\
211.01	0\\
212.01	0\\
213.01	0\\
214.01	0\\
215.01	0\\
216.01	0\\
217.01	0\\
218.01	0\\
219.01	0\\
220.01	0\\
221.01	0\\
222.01	0\\
223.01	0\\
224.01	0\\
225.01	0\\
226.01	0\\
227.01	0\\
228.01	0\\
229.01	0\\
230.01	0\\
231.01	0\\
232.01	0\\
233.01	0\\
234.01	0\\
235.01	0\\
236.01	0\\
237.01	0\\
238.01	0\\
239.01	0\\
240.01	0\\
241.01	0\\
242.01	0\\
243.01	0\\
244.01	0\\
245.01	0\\
246.01	0\\
247.01	0\\
248.01	0\\
249.01	0\\
250.01	0\\
251.01	0\\
252.01	0\\
253.01	0\\
254.01	0\\
255.01	0\\
256.01	0\\
257.01	0\\
258.01	0\\
259.01	0\\
260.01	0\\
261.01	0\\
262.01	0\\
263.01	0\\
264.01	0\\
265.01	0\\
266.01	0\\
267.01	0\\
268.01	0\\
269.01	0\\
270.01	0\\
271.01	0\\
272.01	0\\
273.01	0\\
274.01	0\\
275.01	0\\
276.01	0\\
277.01	0\\
278.01	0\\
279.01	0\\
280.01	0\\
281.01	0\\
282.01	0\\
283.01	0\\
284.01	0\\
285.01	0\\
286.01	0\\
287.01	0\\
288.01	0\\
289.01	0\\
290.01	0\\
291.01	0\\
292.01	0\\
293.01	0\\
294.01	0\\
295.01	0\\
296.01	0\\
297.01	0\\
298.01	0\\
299.01	0\\
300.01	0\\
301.01	0\\
302.01	0\\
303.01	0\\
304.01	0\\
305.01	0\\
306.01	0\\
307.01	0\\
308.01	0\\
309.01	0\\
310.01	0\\
311.01	0\\
312.01	0\\
313.01	0\\
314.01	0\\
315.01	0\\
316.01	0\\
317.01	0\\
318.01	0\\
319.01	0\\
320.01	0\\
321.01	0\\
322.01	0\\
323.01	0\\
324.01	0\\
325.01	0\\
326.01	0\\
327.01	0\\
328.01	0\\
329.01	0\\
330.01	0\\
331.01	0\\
332.01	0\\
333.01	0\\
334.01	0\\
335.01	0\\
336.01	0\\
337.01	0\\
338.01	0\\
339.01	0\\
340.01	0\\
341.01	0\\
342.01	0\\
343.01	0\\
344.01	0\\
345.01	0\\
346.01	0\\
347.01	0\\
348.01	0\\
349.01	0\\
350.01	0\\
351.01	0\\
352.01	0\\
353.01	0\\
354.01	0\\
355.01	0\\
356.01	0\\
357.01	0\\
358.01	0\\
359.01	0\\
360.01	0\\
361.01	0\\
362.01	0\\
363.01	0\\
364.01	0\\
365.01	0\\
366.01	0\\
367.01	0\\
368.01	0\\
369.01	0\\
370.01	0\\
371.01	0\\
372.01	0\\
373.01	0\\
374.01	0\\
375.01	0\\
376.01	0\\
377.01	0\\
378.01	0\\
379.01	0\\
380.01	0\\
381.01	0\\
382.01	0\\
383.01	0\\
384.01	0\\
385.01	0\\
386.01	0\\
387.01	0\\
388.01	0\\
389.01	0\\
390.01	0\\
391.01	0\\
392.01	0\\
393.01	0\\
394.01	0\\
395.01	0\\
396.01	0\\
397.01	0\\
398.01	0\\
399.01	0\\
400.01	0\\
401.01	0\\
402.01	0\\
403.01	0\\
404.01	0\\
405.01	0\\
406.01	0\\
407.01	0\\
408.01	0\\
409.01	0\\
410.01	0\\
411.01	0\\
412.01	0\\
413.01	0\\
414.01	0\\
415.01	0\\
416.01	0\\
417.01	0\\
418.01	0\\
419.01	0\\
420.01	0\\
421.01	0\\
422.01	0\\
423.01	0\\
424.01	0\\
425.01	0\\
426.01	0\\
427.01	0\\
428.01	0\\
429.01	0\\
430.01	0\\
431.01	0\\
432.01	0\\
433.01	0\\
434.01	0\\
435.01	0\\
436.01	0\\
437.01	0\\
438.01	0\\
439.01	0\\
440.01	0\\
441.01	0\\
442.01	0\\
443.01	0\\
444.01	0\\
445.01	0\\
446.01	0\\
447.01	0\\
448.01	0\\
449.01	0\\
450.01	0\\
451.01	0\\
452.01	0\\
453.01	0\\
454.01	0\\
455.01	0\\
456.01	0\\
457.01	0\\
458.01	0\\
459.01	0\\
460.01	0\\
461.01	0\\
462.01	0\\
463.01	0\\
464.01	0\\
465.01	0\\
466.01	0\\
467.01	0\\
468.01	0\\
469.01	0\\
470.01	0\\
471.01	0\\
472.01	0\\
473.01	0\\
474.01	0\\
475.01	0\\
476.01	0\\
477.01	0\\
478.01	0\\
479.01	0\\
480.01	0\\
481.01	0\\
482.01	0\\
483.01	0\\
484.01	0\\
485.01	0\\
486.01	0\\
487.01	0\\
488.01	0\\
489.01	0\\
490.01	0\\
491.01	0\\
492.01	0\\
493.01	0\\
494.01	0\\
495.01	0\\
496.01	0\\
497.01	0\\
498.01	0\\
499.01	0\\
500.01	0\\
501.01	0\\
502.01	0\\
503.01	0\\
504.01	0\\
505.01	0\\
506.01	0\\
507.01	0\\
508.01	0\\
509.01	0\\
510.01	0\\
511.01	0\\
512.01	0\\
513.01	0\\
514.01	0\\
515.01	0\\
516.01	0\\
517.01	0\\
518.01	0\\
519.01	0\\
520.01	0\\
521.01	0\\
522.01	0\\
523.01	0\\
524.01	0\\
525.01	0\\
526.01	0\\
527.01	0\\
528.01	0\\
529.01	0\\
530.01	0\\
531.01	0\\
532.01	0\\
533.01	0\\
534.01	0\\
535.01	0\\
536.01	0\\
537.01	0\\
538.01	0\\
539.01	0\\
540.01	0\\
541.01	0\\
542.01	0\\
543.01	0\\
544.01	0\\
545.01	0\\
546.01	0\\
547.01	0\\
548.01	0\\
549.01	0\\
550.01	0\\
551.01	0\\
552.01	0\\
553.01	0\\
554.01	0\\
555.01	0\\
556.01	0\\
557.01	0\\
558.01	0\\
559.01	0\\
560.01	0\\
561.01	0\\
562.01	0\\
563.01	0\\
564.01	0\\
565.01	0\\
566.01	0\\
567.01	0\\
568.01	0.000272550836893013\\
569.01	0.000573431998096324\\
570.01	0.000894661826268021\\
571.01	0.00123916709588926\\
572.01	0.00161046588219712\\
573.01	0.00200914876660706\\
574.01	0.00242577240038586\\
575.01	0.00286070485950432\\
576.01	0.00331543550699241\\
577.01	0.00379158961836269\\
578.01	0.00429093233054111\\
579.01	0.00481536241245629\\
580.01	0.00536687839479479\\
581.01	0.00594746706158842\\
582.01	0.00648755091140849\\
583.01	0.00673068723289618\\
584.01	0.00696177471066208\\
585.01	0.00718292114201277\\
586.01	0.00740501470221952\\
587.01	0.00762757491051881\\
588.01	0.00784970065629068\\
589.01	0.00807045080115363\\
590.01	0.00828974304449905\\
591.01	0.00850824026166992\\
592.01	0.00872445348147146\\
593.01	0.00893652110766042\\
594.01	0.00914223590543076\\
595.01	0.00933900026637371\\
596.01	0.00952378122840184\\
597.01	0.00969306778608631\\
598.01	0.00984283008036189\\
599.01	0.00995428645146456\\
599.02	0.00995511722778644\\
599.03	0.00995594190291016\\
599.04	0.00995676044198501\\
599.05	0.00995757280986887\\
599.06	0.0099583789711255\\
599.07	0.00995917889002186\\
599.08	0.00995997253052542\\
599.09	0.00996075985630137\\
599.1	0.0099615408307099\\
599.11	0.00996231541680342\\
599.12	0.00996308357732374\\
599.13	0.00996384527469924\\
599.14	0.00996460047104208\\
599.15	0.00996534912814527\\
599.16	0.00996609120747984\\
599.17	0.00996682667019189\\
599.18	0.0099675554770997\\
599.19	0.00996827758869077\\
599.2	0.00996899296511884\\
599.21	0.00996970156620092\\
599.22	0.00997040335141429\\
599.23	0.00997109827989345\\
599.24	0.00997178631042707\\
599.25	0.00997246740145495\\
599.26	0.00997314151106556\\
599.27	0.00997380859699371\\
599.28	0.00997446861661746\\
599.29	0.00997512152695501\\
599.3	0.00997576728466153\\
599.31	0.00997640584602604\\
599.32	0.00997703716696816\\
599.33	0.00997766120303499\\
599.34	0.00997827790939785\\
599.35	0.00997888724084906\\
599.36	0.00997948915179871\\
599.37	0.00998008359627136\\
599.38	0.00998067052790281\\
599.39	0.00998124989993678\\
599.4	0.00998182166522161\\
599.41	0.00998238577620696\\
599.42	0.00998294218494043\\
599.43	0.00998349084306428\\
599.44	0.00998403170181203\\
599.45	0.00998456471200512\\
599.46	0.00998508982393612\\
599.47	0.00998560698245901\\
599.48	0.00998611613195496\\
599.49	0.00998661721632843\\
599.5	0.0099871101790033\\
599.51	0.00998759496291904\\
599.52	0.00998807151052681\\
599.53	0.00998853976378555\\
599.54	0.00998899966415814\\
599.55	0.00998945115260745\\
599.56	0.00998989416959251\\
599.57	0.00999032865506456\\
599.58	0.00999075454846319\\
599.59	0.00999117178871244\\
599.6	0.00999158031421693\\
599.61	0.00999198006285796\\
599.62	0.00999237097198967\\
599.63	0.00999275297843516\\
599.64	0.00999312601848268\\
599.65	0.00999349002788177\\
599.66	0.00999384494183951\\
599.67	0.00999419069501666\\
599.68	0.00999452722152399\\
599.69	0.00999485445491849\\
599.7	0.00999517232819969\\
599.71	0.00999548077380601\\
599.72	0.00999577972361112\\
599.73	0.00999606910892039\\
599.74	0.00999634886046729\\
599.75	0.00999661890841001\\
599.76	0.00999687918232793\\
599.77	0.00999712961121834\\
599.78	0.00999737012349307\\
599.79	0.00999760064697533\\
599.8	0.0099978211088965\\
599.81	0.00999803143589311\\
599.82	0.00999823155400384\\
599.83	0.00999842138866665\\
599.84	0.009998600864716\\
599.85	0.0099987699063802\\
599.86	0.00999892843727885\\
599.87	0.00999907638042046\\
599.88	0.00999921365820013\\
599.89	0.00999934019239746\\
599.9	0.00999945590417459\\
599.91	0.00999956071407437\\
599.92	0.00999965454201877\\
599.93	0.00999973730730745\\
599.94	0.00999980892861654\\
599.95	0.00999986932399766\\
599.96	0.00999991841087715\\
599.97	0.00999995610605555\\
599.98	0.0099999823257074\\
599.99	0.00999999698538124\\
600	0.01\\
};
\addplot [color=mycolor15,solid,forget plot]
  table[row sep=crcr]{%
0.01	0\\
1.01	0\\
2.01	0\\
3.01	0\\
4.01	0\\
5.01	0\\
6.01	0\\
7.01	0\\
8.01	0\\
9.01	0\\
10.01	0\\
11.01	0\\
12.01	0\\
13.01	0\\
14.01	0\\
15.01	0\\
16.01	0\\
17.01	0\\
18.01	0\\
19.01	0\\
20.01	0\\
21.01	0\\
22.01	0\\
23.01	0\\
24.01	0\\
25.01	0\\
26.01	0\\
27.01	0\\
28.01	0\\
29.01	0\\
30.01	0\\
31.01	0\\
32.01	0\\
33.01	0\\
34.01	0\\
35.01	0\\
36.01	0\\
37.01	0\\
38.01	0\\
39.01	0\\
40.01	0\\
41.01	0\\
42.01	0\\
43.01	0\\
44.01	0\\
45.01	0\\
46.01	0\\
47.01	0\\
48.01	0\\
49.01	0\\
50.01	0\\
51.01	0\\
52.01	0\\
53.01	0\\
54.01	0\\
55.01	0\\
56.01	0\\
57.01	0\\
58.01	0\\
59.01	0\\
60.01	0\\
61.01	0\\
62.01	0\\
63.01	0\\
64.01	0\\
65.01	0\\
66.01	0\\
67.01	0\\
68.01	0\\
69.01	0\\
70.01	0\\
71.01	0\\
72.01	0\\
73.01	0\\
74.01	0\\
75.01	0\\
76.01	0\\
77.01	0\\
78.01	0\\
79.01	0\\
80.01	0\\
81.01	0\\
82.01	0\\
83.01	0\\
84.01	0\\
85.01	0\\
86.01	0\\
87.01	0\\
88.01	0\\
89.01	0\\
90.01	0\\
91.01	0\\
92.01	0\\
93.01	0\\
94.01	0\\
95.01	0\\
96.01	0\\
97.01	0\\
98.01	0\\
99.01	0\\
100.01	0\\
101.01	0\\
102.01	0\\
103.01	0\\
104.01	0\\
105.01	0\\
106.01	0\\
107.01	0\\
108.01	0\\
109.01	0\\
110.01	0\\
111.01	0\\
112.01	0\\
113.01	0\\
114.01	0\\
115.01	0\\
116.01	0\\
117.01	0\\
118.01	0\\
119.01	0\\
120.01	0\\
121.01	0\\
122.01	0\\
123.01	0\\
124.01	0\\
125.01	0\\
126.01	0\\
127.01	0\\
128.01	0\\
129.01	0\\
130.01	0\\
131.01	0\\
132.01	0\\
133.01	0\\
134.01	0\\
135.01	0\\
136.01	0\\
137.01	0\\
138.01	0\\
139.01	0\\
140.01	0\\
141.01	0\\
142.01	0\\
143.01	0\\
144.01	0\\
145.01	0\\
146.01	0\\
147.01	0\\
148.01	0\\
149.01	0\\
150.01	0\\
151.01	0\\
152.01	0\\
153.01	0\\
154.01	0\\
155.01	0\\
156.01	0\\
157.01	0\\
158.01	0\\
159.01	0\\
160.01	0\\
161.01	0\\
162.01	0\\
163.01	0\\
164.01	0\\
165.01	0\\
166.01	0\\
167.01	0\\
168.01	0\\
169.01	0\\
170.01	0\\
171.01	0\\
172.01	0\\
173.01	0\\
174.01	0\\
175.01	0\\
176.01	0\\
177.01	0\\
178.01	0\\
179.01	0\\
180.01	0\\
181.01	0\\
182.01	0\\
183.01	0\\
184.01	0\\
185.01	0\\
186.01	0\\
187.01	0\\
188.01	0\\
189.01	0\\
190.01	0\\
191.01	0\\
192.01	0\\
193.01	0\\
194.01	0\\
195.01	0\\
196.01	0\\
197.01	0\\
198.01	0\\
199.01	0\\
200.01	0\\
201.01	0\\
202.01	0\\
203.01	0\\
204.01	0\\
205.01	0\\
206.01	0\\
207.01	0\\
208.01	0\\
209.01	0\\
210.01	0\\
211.01	0\\
212.01	0\\
213.01	0\\
214.01	0\\
215.01	0\\
216.01	0\\
217.01	0\\
218.01	0\\
219.01	0\\
220.01	0\\
221.01	0\\
222.01	0\\
223.01	0\\
224.01	0\\
225.01	0\\
226.01	0\\
227.01	0\\
228.01	0\\
229.01	0\\
230.01	0\\
231.01	0\\
232.01	0\\
233.01	0\\
234.01	0\\
235.01	0\\
236.01	0\\
237.01	0\\
238.01	0\\
239.01	0\\
240.01	0\\
241.01	0\\
242.01	0\\
243.01	0\\
244.01	0\\
245.01	0\\
246.01	0\\
247.01	0\\
248.01	0\\
249.01	0\\
250.01	0\\
251.01	0\\
252.01	0\\
253.01	0\\
254.01	0\\
255.01	0\\
256.01	0\\
257.01	0\\
258.01	0\\
259.01	0\\
260.01	0\\
261.01	0\\
262.01	0\\
263.01	0\\
264.01	0\\
265.01	0\\
266.01	0\\
267.01	0\\
268.01	0\\
269.01	0\\
270.01	0\\
271.01	0\\
272.01	0\\
273.01	0\\
274.01	0\\
275.01	0\\
276.01	0\\
277.01	0\\
278.01	0\\
279.01	0\\
280.01	0\\
281.01	0\\
282.01	0\\
283.01	0\\
284.01	0\\
285.01	0\\
286.01	0\\
287.01	0\\
288.01	0\\
289.01	0\\
290.01	0\\
291.01	0\\
292.01	0\\
293.01	0\\
294.01	0\\
295.01	0\\
296.01	0\\
297.01	0\\
298.01	0\\
299.01	0\\
300.01	0\\
301.01	0\\
302.01	0\\
303.01	0\\
304.01	0\\
305.01	0\\
306.01	0\\
307.01	0\\
308.01	0\\
309.01	0\\
310.01	0\\
311.01	0\\
312.01	0\\
313.01	0\\
314.01	0\\
315.01	0\\
316.01	0\\
317.01	0\\
318.01	0\\
319.01	0\\
320.01	0\\
321.01	0\\
322.01	0\\
323.01	0\\
324.01	0\\
325.01	0\\
326.01	0\\
327.01	0\\
328.01	0\\
329.01	0\\
330.01	0\\
331.01	0\\
332.01	0\\
333.01	0\\
334.01	0\\
335.01	0\\
336.01	0\\
337.01	0\\
338.01	0\\
339.01	0\\
340.01	0\\
341.01	0\\
342.01	0\\
343.01	0\\
344.01	0\\
345.01	0\\
346.01	0\\
347.01	0\\
348.01	0\\
349.01	0\\
350.01	0\\
351.01	0\\
352.01	0\\
353.01	0\\
354.01	0\\
355.01	0\\
356.01	0\\
357.01	0\\
358.01	0\\
359.01	0\\
360.01	0\\
361.01	0\\
362.01	0\\
363.01	0\\
364.01	0\\
365.01	0\\
366.01	0\\
367.01	0\\
368.01	0\\
369.01	0\\
370.01	0\\
371.01	0\\
372.01	0\\
373.01	0\\
374.01	0\\
375.01	0\\
376.01	0\\
377.01	0\\
378.01	0\\
379.01	0\\
380.01	0\\
381.01	0\\
382.01	0\\
383.01	0\\
384.01	0\\
385.01	0\\
386.01	0\\
387.01	0\\
388.01	0\\
389.01	0\\
390.01	0\\
391.01	0\\
392.01	0\\
393.01	0\\
394.01	0\\
395.01	0\\
396.01	0\\
397.01	0\\
398.01	0\\
399.01	0\\
400.01	0\\
401.01	0\\
402.01	0\\
403.01	0\\
404.01	0\\
405.01	0\\
406.01	0\\
407.01	0\\
408.01	0\\
409.01	0\\
410.01	0\\
411.01	0\\
412.01	0\\
413.01	0\\
414.01	0\\
415.01	0\\
416.01	0\\
417.01	0\\
418.01	0\\
419.01	0\\
420.01	0\\
421.01	0\\
422.01	0\\
423.01	0\\
424.01	0\\
425.01	0\\
426.01	0\\
427.01	0\\
428.01	0\\
429.01	0\\
430.01	0\\
431.01	0\\
432.01	0\\
433.01	0\\
434.01	0\\
435.01	0\\
436.01	0\\
437.01	0\\
438.01	0\\
439.01	0\\
440.01	0\\
441.01	0\\
442.01	0\\
443.01	0\\
444.01	0\\
445.01	0\\
446.01	0\\
447.01	0\\
448.01	0\\
449.01	0\\
450.01	0\\
451.01	0\\
452.01	0\\
453.01	0\\
454.01	0\\
455.01	0\\
456.01	0\\
457.01	0\\
458.01	0\\
459.01	0\\
460.01	0\\
461.01	0\\
462.01	0\\
463.01	0\\
464.01	0\\
465.01	0\\
466.01	0\\
467.01	0\\
468.01	0\\
469.01	0\\
470.01	0\\
471.01	0\\
472.01	0\\
473.01	0\\
474.01	0\\
475.01	0\\
476.01	0\\
477.01	0\\
478.01	0\\
479.01	0\\
480.01	0\\
481.01	0\\
482.01	0\\
483.01	0\\
484.01	0\\
485.01	0\\
486.01	0\\
487.01	0\\
488.01	0\\
489.01	0\\
490.01	0\\
491.01	0\\
492.01	0\\
493.01	0\\
494.01	0\\
495.01	0\\
496.01	0\\
497.01	0\\
498.01	0\\
499.01	0\\
500.01	0\\
501.01	0\\
502.01	0\\
503.01	0\\
504.01	0\\
505.01	0\\
506.01	0\\
507.01	0\\
508.01	0\\
509.01	0\\
510.01	0\\
511.01	0\\
512.01	0\\
513.01	0\\
514.01	0\\
515.01	0\\
516.01	0\\
517.01	0\\
518.01	0\\
519.01	0\\
520.01	0\\
521.01	0\\
522.01	0\\
523.01	0\\
524.01	0\\
525.01	0\\
526.01	0\\
527.01	0\\
528.01	0\\
529.01	0\\
530.01	0\\
531.01	0\\
532.01	0\\
533.01	0\\
534.01	0\\
535.01	0\\
536.01	0\\
537.01	0\\
538.01	0\\
539.01	0\\
540.01	0\\
541.01	0\\
542.01	0\\
543.01	0\\
544.01	0\\
545.01	0\\
546.01	0\\
547.01	0\\
548.01	0\\
549.01	0\\
550.01	4.18278501244616e-05\\
551.01	0.000230792793490968\\
552.01	0.000429318385053151\\
553.01	0.000638423513595891\\
554.01	0.000859291226111458\\
555.01	0.00109330128463499\\
556.01	0.00134206999046035\\
557.01	0.00160749963610692\\
558.01	0.00189184148236909\\
559.01	0.00219777549602514\\
560.01	0.00252698266125781\\
561.01	0.00287236341131009\\
562.01	0.00323336009928089\\
563.01	0.00361112252905767\\
564.01	0.00400699194756734\\
565.01	0.00442260638993207\\
566.01	0.0048602106788841\\
567.01	0.00532249736082619\\
568.01	0.00553562477064916\\
569.01	0.00574040657268343\\
570.01	0.00594543331123816\\
571.01	0.00614813023233589\\
572.01	0.00634492779261982\\
573.01	0.00653458863917867\\
574.01	0.00672477895943008\\
575.01	0.00691480367421593\\
576.01	0.00710345125817615\\
577.01	0.00728925973806523\\
578.01	0.00747048244186695\\
579.01	0.00764505567993344\\
580.01	0.00781057365081174\\
581.01	0.00796428037686027\\
582.01	0.00810430631923353\\
583.01	0.00823778064650377\\
584.01	0.00836882604993697\\
585.01	0.00849805350251816\\
586.01	0.00862548745291904\\
587.01	0.00875080170555895\\
588.01	0.008873717407561\\
589.01	0.00899400983054572\\
590.01	0.00911146508396814\\
591.01	0.00922566189803017\\
592.01	0.00933610883364893\\
593.01	0.0094423826989696\\
594.01	0.00954416814589645\\
595.01	0.00964130354156367\\
596.01	0.00973383226636715\\
597.01	0.00982205623179395\\
598.01	0.00990570153168505\\
599.01	0.00997025669700849\\
599.02	0.00997077449256594\\
599.03	0.00997128908704524\\
599.04	0.00997180045248797\\
599.05	0.00997230856064844\\
599.06	0.00997281338299083\\
599.07	0.00997331489068629\\
599.08	0.00997381305460998\\
599.09	0.00997430784533816\\
599.1	0.00997479923314515\\
599.11	0.00997528718800035\\
599.12	0.00997577167956518\\
599.13	0.00997625267719001\\
599.14	0.00997673014991101\\
599.15	0.00997720406644708\\
599.16	0.00997767439519661\\
599.17	0.00997814110423434\\
599.18	0.00997860416130809\\
599.19	0.00997906353383548\\
599.2	0.00997951918890067\\
599.21	0.00997997109325102\\
599.22	0.00998041921329368\\
599.23	0.00998086351509228\\
599.24	0.0099813039643634\\
599.25	0.00998174052647318\\
599.26	0.00998217316503718\\
599.27	0.00998260184151771\\
599.28	0.00998302651697996\\
599.29	0.00998344715208793\\
599.3	0.00998386370710054\\
599.31	0.00998427614186751\\
599.32	0.00998468441582528\\
599.33	0.00998508848799292\\
599.34	0.00998548831696792\\
599.35	0.00998588386092202\\
599.36	0.00998627507759691\\
599.37	0.00998666192429999\\
599.38	0.00998704435789998\\
599.39	0.00998742233482257\\
599.4	0.00998779581104599\\
599.41	0.00998816474209656\\
599.42	0.0099885290830441\\
599.43	0.00998888878849749\\
599.44	0.00998924381259994\\
599.45	0.00998959410902443\\
599.46	0.00998993963096902\\
599.47	0.00999028033115466\\
599.48	0.00999061616182043\\
599.49	0.00999094707471871\\
599.5	0.00999127302111033\\
599.51	0.00999159395175964\\
599.52	0.00999190981692958\\
599.53	0.00999222056637663\\
599.54	0.00999252614934575\\
599.55	0.00999282651456528\\
599.56	0.00999312161024174\\
599.57	0.00999341138405464\\
599.58	0.00999369578315118\\
599.59	0.00999397475414089\\
599.6	0.00999424824309031\\
599.61	0.00999451619551744\\
599.62	0.00999477855638636\\
599.63	0.00999503527010156\\
599.64	0.00999528628050238\\
599.65	0.00999553153085729\\
599.66	0.0099957709638582\\
599.67	0.0099960045216146\\
599.68	0.00999623214564774\\
599.69	0.00999645377688465\\
599.7	0.00999666935565223\\
599.71	0.00999687882167107\\
599.72	0.00999708211404946\\
599.73	0.00999727917127708\\
599.74	0.00999746993121882\\
599.75	0.0099976543311084\\
599.76	0.00999783230754198\\
599.77	0.00999800379647171\\
599.78	0.00999816873319912\\
599.79	0.00999832705236858\\
599.8	0.00999847868796051\\
599.81	0.00999862357328467\\
599.82	0.00999876164097328\\
599.83	0.00999889282297404\\
599.84	0.00999901705054318\\
599.85	0.0099991342542383\\
599.86	0.00999924436391118\\
599.87	0.0099993473087005\\
599.88	0.0099994430170245\\
599.89	0.00999953141657346\\
599.9	0.00999961243430216\\
599.91	0.00999968599642223\\
599.92	0.00999975202839438\\
599.93	0.00999981045492053\\
599.94	0.00999986119993585\\
599.95	0.0099999041866007\\
599.96	0.00999993933729238\\
599.97	0.00999996657359692\\
599.98	0.00999998581630055\\
599.99	0.00999999698538124\\
600	0.01\\
};
\addplot [color=mycolor16,solid,forget plot]
  table[row sep=crcr]{%
0.01	0\\
1.01	0\\
2.01	0\\
3.01	0\\
4.01	0\\
5.01	0\\
6.01	0\\
7.01	0\\
8.01	0\\
9.01	0\\
10.01	0\\
11.01	0\\
12.01	0\\
13.01	0\\
14.01	0\\
15.01	0\\
16.01	0\\
17.01	0\\
18.01	0\\
19.01	0\\
20.01	0\\
21.01	0\\
22.01	0\\
23.01	0\\
24.01	0\\
25.01	0\\
26.01	0\\
27.01	0\\
28.01	0\\
29.01	0\\
30.01	0\\
31.01	0\\
32.01	0\\
33.01	0\\
34.01	0\\
35.01	0\\
36.01	0\\
37.01	0\\
38.01	0\\
39.01	0\\
40.01	0\\
41.01	0\\
42.01	0\\
43.01	0\\
44.01	0\\
45.01	0\\
46.01	0\\
47.01	0\\
48.01	0\\
49.01	0\\
50.01	0\\
51.01	0\\
52.01	0\\
53.01	0\\
54.01	0\\
55.01	0\\
56.01	0\\
57.01	0\\
58.01	0\\
59.01	0\\
60.01	0\\
61.01	0\\
62.01	0\\
63.01	0\\
64.01	0\\
65.01	0\\
66.01	0\\
67.01	0\\
68.01	0\\
69.01	0\\
70.01	0\\
71.01	0\\
72.01	0\\
73.01	0\\
74.01	0\\
75.01	0\\
76.01	0\\
77.01	0\\
78.01	0\\
79.01	0\\
80.01	0\\
81.01	0\\
82.01	0\\
83.01	0\\
84.01	0\\
85.01	0\\
86.01	0\\
87.01	0\\
88.01	0\\
89.01	0\\
90.01	0\\
91.01	0\\
92.01	0\\
93.01	0\\
94.01	0\\
95.01	0\\
96.01	0\\
97.01	0\\
98.01	0\\
99.01	0\\
100.01	0\\
101.01	0\\
102.01	0\\
103.01	0\\
104.01	0\\
105.01	0\\
106.01	0\\
107.01	0\\
108.01	0\\
109.01	0\\
110.01	0\\
111.01	0\\
112.01	0\\
113.01	0\\
114.01	0\\
115.01	0\\
116.01	0\\
117.01	0\\
118.01	0\\
119.01	0\\
120.01	0\\
121.01	0\\
122.01	0\\
123.01	0\\
124.01	0\\
125.01	0\\
126.01	0\\
127.01	0\\
128.01	0\\
129.01	0\\
130.01	0\\
131.01	0\\
132.01	0\\
133.01	0\\
134.01	0\\
135.01	0\\
136.01	0\\
137.01	0\\
138.01	0\\
139.01	0\\
140.01	0\\
141.01	0\\
142.01	0\\
143.01	0\\
144.01	0\\
145.01	0\\
146.01	0\\
147.01	0\\
148.01	0\\
149.01	0\\
150.01	0\\
151.01	0\\
152.01	0\\
153.01	0\\
154.01	0\\
155.01	0\\
156.01	0\\
157.01	0\\
158.01	0\\
159.01	0\\
160.01	0\\
161.01	0\\
162.01	0\\
163.01	0\\
164.01	0\\
165.01	0\\
166.01	0\\
167.01	0\\
168.01	0\\
169.01	0\\
170.01	0\\
171.01	0\\
172.01	0\\
173.01	0\\
174.01	0\\
175.01	0\\
176.01	0\\
177.01	0\\
178.01	0\\
179.01	0\\
180.01	0\\
181.01	0\\
182.01	0\\
183.01	0\\
184.01	0\\
185.01	0\\
186.01	0\\
187.01	0\\
188.01	0\\
189.01	0\\
190.01	0\\
191.01	0\\
192.01	0\\
193.01	0\\
194.01	0\\
195.01	0\\
196.01	0\\
197.01	0\\
198.01	0\\
199.01	0\\
200.01	0\\
201.01	0\\
202.01	0\\
203.01	0\\
204.01	0\\
205.01	0\\
206.01	0\\
207.01	0\\
208.01	0\\
209.01	0\\
210.01	0\\
211.01	0\\
212.01	0\\
213.01	0\\
214.01	0\\
215.01	0\\
216.01	0\\
217.01	0\\
218.01	0\\
219.01	0\\
220.01	0\\
221.01	0\\
222.01	0\\
223.01	0\\
224.01	0\\
225.01	0\\
226.01	0\\
227.01	0\\
228.01	0\\
229.01	0\\
230.01	0\\
231.01	0\\
232.01	0\\
233.01	0\\
234.01	0\\
235.01	0\\
236.01	0\\
237.01	0\\
238.01	0\\
239.01	0\\
240.01	0\\
241.01	0\\
242.01	0\\
243.01	0\\
244.01	0\\
245.01	0\\
246.01	0\\
247.01	0\\
248.01	0\\
249.01	0\\
250.01	0\\
251.01	0\\
252.01	0\\
253.01	0\\
254.01	0\\
255.01	0\\
256.01	0\\
257.01	0\\
258.01	0\\
259.01	0\\
260.01	0\\
261.01	0\\
262.01	0\\
263.01	0\\
264.01	0\\
265.01	0\\
266.01	0\\
267.01	0\\
268.01	0\\
269.01	0\\
270.01	0\\
271.01	0\\
272.01	0\\
273.01	0\\
274.01	0\\
275.01	0\\
276.01	0\\
277.01	0\\
278.01	0\\
279.01	0\\
280.01	0\\
281.01	0\\
282.01	0\\
283.01	0\\
284.01	0\\
285.01	0\\
286.01	0\\
287.01	0\\
288.01	0\\
289.01	0\\
290.01	0\\
291.01	0\\
292.01	0\\
293.01	0\\
294.01	0\\
295.01	0\\
296.01	0\\
297.01	0\\
298.01	0\\
299.01	0\\
300.01	0\\
301.01	0\\
302.01	0\\
303.01	0\\
304.01	0\\
305.01	0\\
306.01	0\\
307.01	0\\
308.01	0\\
309.01	0\\
310.01	0\\
311.01	0\\
312.01	0\\
313.01	0\\
314.01	0\\
315.01	0\\
316.01	0\\
317.01	0\\
318.01	0\\
319.01	0\\
320.01	0\\
321.01	0\\
322.01	0\\
323.01	0\\
324.01	0\\
325.01	0\\
326.01	0\\
327.01	0\\
328.01	0\\
329.01	0\\
330.01	0\\
331.01	0\\
332.01	0\\
333.01	0\\
334.01	0\\
335.01	0\\
336.01	0\\
337.01	0\\
338.01	0\\
339.01	0\\
340.01	0\\
341.01	0\\
342.01	0\\
343.01	0\\
344.01	0\\
345.01	0\\
346.01	0\\
347.01	0\\
348.01	0\\
349.01	0\\
350.01	0\\
351.01	0\\
352.01	0\\
353.01	0\\
354.01	0\\
355.01	0\\
356.01	0\\
357.01	0\\
358.01	0\\
359.01	0\\
360.01	0\\
361.01	0\\
362.01	0\\
363.01	0\\
364.01	0\\
365.01	0\\
366.01	0\\
367.01	0\\
368.01	0\\
369.01	0\\
370.01	0\\
371.01	0\\
372.01	0\\
373.01	0\\
374.01	0\\
375.01	0\\
376.01	0\\
377.01	0\\
378.01	0\\
379.01	0\\
380.01	0\\
381.01	0\\
382.01	0\\
383.01	0\\
384.01	0\\
385.01	0\\
386.01	0\\
387.01	0\\
388.01	0\\
389.01	0\\
390.01	0\\
391.01	0\\
392.01	0\\
393.01	0\\
394.01	0\\
395.01	0\\
396.01	0\\
397.01	0\\
398.01	0\\
399.01	0\\
400.01	0\\
401.01	0\\
402.01	0\\
403.01	0\\
404.01	0\\
405.01	0\\
406.01	0\\
407.01	0\\
408.01	0\\
409.01	0\\
410.01	0\\
411.01	0\\
412.01	0\\
413.01	0\\
414.01	0\\
415.01	0\\
416.01	0\\
417.01	0\\
418.01	0\\
419.01	0\\
420.01	0\\
421.01	0\\
422.01	0\\
423.01	0\\
424.01	0\\
425.01	0\\
426.01	0\\
427.01	0\\
428.01	0\\
429.01	0\\
430.01	0\\
431.01	0\\
432.01	0\\
433.01	0\\
434.01	0\\
435.01	0\\
436.01	0\\
437.01	0\\
438.01	0\\
439.01	0\\
440.01	0\\
441.01	0\\
442.01	0\\
443.01	0\\
444.01	0\\
445.01	0\\
446.01	0\\
447.01	0\\
448.01	0\\
449.01	0\\
450.01	0\\
451.01	0\\
452.01	0\\
453.01	0\\
454.01	0\\
455.01	0\\
456.01	0\\
457.01	0\\
458.01	0\\
459.01	0\\
460.01	0\\
461.01	0\\
462.01	0\\
463.01	0\\
464.01	0\\
465.01	0\\
466.01	0\\
467.01	0\\
468.01	0\\
469.01	0\\
470.01	0\\
471.01	0\\
472.01	0\\
473.01	0\\
474.01	0\\
475.01	0\\
476.01	0\\
477.01	0\\
478.01	0\\
479.01	0\\
480.01	0\\
481.01	0\\
482.01	0\\
483.01	0\\
484.01	0\\
485.01	0\\
486.01	0\\
487.01	0\\
488.01	0\\
489.01	0\\
490.01	0\\
491.01	0\\
492.01	0\\
493.01	0\\
494.01	0\\
495.01	0\\
496.01	0\\
497.01	0\\
498.01	0\\
499.01	0\\
500.01	0\\
501.01	0\\
502.01	0\\
503.01	0\\
504.01	0\\
505.01	0\\
506.01	0\\
507.01	0\\
508.01	0\\
509.01	0\\
510.01	0\\
511.01	0\\
512.01	0\\
513.01	0\\
514.01	0\\
515.01	0\\
516.01	0\\
517.01	0\\
518.01	0\\
519.01	0\\
520.01	0\\
521.01	0\\
522.01	0\\
523.01	0\\
524.01	0\\
525.01	0\\
526.01	0\\
527.01	7.58709029509003e-05\\
528.01	0.000182474483497404\\
529.01	0.000293106910623106\\
530.01	0.000408037296038929\\
531.01	0.000527564848188972\\
532.01	0.000652024036411215\\
533.01	0.000781791922589287\\
534.01	0.00091730227808122\\
535.01	0.00105904894748098\\
536.01	0.00120759049179552\\
537.01	0.00136356032051525\\
538.01	0.00152767859230953\\
539.01	0.00170076616388885\\
540.01	0.00188376095582964\\
541.01	0.00207773717383749\\
542.01	0.00228393819189821\\
543.01	0.00250393549168844\\
544.01	0.00273969443204838\\
545.01	0.0029920480527937\\
546.01	0.00325611513039836\\
547.01	0.00353212677023015\\
548.01	0.00382123777963479\\
549.01	0.00412481857923038\\
550.01	0.00440253967046936\\
551.01	0.00454701284676821\\
552.01	0.00469499207454682\\
553.01	0.00484609988606597\\
554.01	0.00499980750463021\\
555.01	0.00515539572510951\\
556.01	0.00531191602469303\\
557.01	0.00546809119478294\\
558.01	0.00562218531161904\\
559.01	0.00577185501385423\\
560.01	0.00591549938797714\\
561.01	0.00606040602741116\\
562.01	0.00620758878906431\\
563.01	0.00635622936558284\\
564.01	0.00650515733755577\\
565.01	0.00665171599421486\\
566.01	0.00679316347567889\\
567.01	0.00692694660728829\\
568.01	0.00705436387442827\\
569.01	0.00717888869534497\\
570.01	0.00729992529312984\\
571.01	0.00741704362151842\\
572.01	0.00753014340890209\\
573.01	0.00763968735015476\\
574.01	0.00774735512661497\\
575.01	0.00785350951050682\\
576.01	0.00795787532128202\\
577.01	0.00806023678096092\\
578.01	0.00816046589172386\\
579.01	0.00825855451912221\\
580.01	0.00835464755040728\\
581.01	0.00844907172503963\\
582.01	0.00854231050026635\\
583.01	0.00863470051440009\\
584.01	0.00872627653456776\\
585.01	0.0088170082496113\\
586.01	0.00890683476173956\\
587.01	0.00899570431881243\\
588.01	0.00908357815033255\\
589.01	0.0091704238016257\\
590.01	0.00925620895122939\\
591.01	0.00934091361666743\\
592.01	0.00942455622952417\\
593.01	0.00950719997922862\\
594.01	0.0095889524410984\\
595.01	0.00966996121364661\\
596.01	0.00975040472227273\\
597.01	0.00983047783702982\\
598.01	0.00990810332595271\\
599.01	0.00997052938090576\\
599.02	0.00997103903864904\\
599.03	0.00997154565629264\\
599.04	0.00997204920438894\\
599.05	0.00997254965319863\\
599.06	0.00997304697268781\\
599.07	0.00997354113252509\\
599.08	0.0099740321020787\\
599.09	0.00997451985041345\\
599.1	0.0099750043462878\\
599.11	0.00997548555815081\\
599.12	0.00997596345413907\\
599.13	0.00997643800207369\\
599.14	0.00997690916945707\\
599.15	0.00997737692346987\\
599.16	0.00997784123096778\\
599.17	0.00997830205847831\\
599.18	0.00997875937219755\\
599.19	0.00997921313798693\\
599.2	0.0099796633213699\\
599.21	0.0099801098875286\\
599.22	0.00998055280130045\\
599.23	0.00998099202717483\\
599.24	0.00998142752928957\\
599.25	0.00998185927142751\\
599.26	0.00998228721513154\\
599.27	0.00998271132031303\\
599.28	0.0099831315464829\\
599.29	0.00998354785274771\\
599.3	0.00998396019780564\\
599.31	0.00998436853994246\\
599.32	0.00998477283702748\\
599.33	0.00998517304650944\\
599.34	0.00998556912541235\\
599.35	0.0099859610303313\\
599.36	0.00998634871742824\\
599.37	0.00998673214242771\\
599.38	0.00998711126061251\\
599.39	0.00998748602681935\\
599.4	0.00998785639543445\\
599.41	0.0099882223203891\\
599.42	0.00998858375515517\\
599.43	0.00998894065274057\\
599.44	0.00998929296568469\\
599.45	0.00998964064605377\\
599.46	0.00998998364543622\\
599.47	0.00999032191493792\\
599.48	0.00999065540517746\\
599.49	0.00999098406628132\\
599.5	0.00999130784787903\\
599.51	0.00999162669909824\\
599.52	0.00999194056855979\\
599.53	0.00999224940437273\\
599.54	0.00999255315412921\\
599.55	0.00999285176489941\\
599.56	0.00999314518322642\\
599.57	0.009993433355121\\
599.58	0.00999371622605632\\
599.59	0.0099939937409627\\
599.6	0.0099942658442222\\
599.61	0.00999453247966325\\
599.62	0.00999479359055517\\
599.63	0.00999504911960265\\
599.64	0.00999529900894021\\
599.65	0.00999554320012651\\
599.66	0.00999578163413872\\
599.67	0.00999601425136677\\
599.68	0.00999624099160756\\
599.69	0.00999646179405908\\
599.7	0.00999667659731453\\
599.71	0.00999688533935636\\
599.72	0.0099970879575502\\
599.73	0.00999728438863882\\
599.74	0.00999747456873595\\
599.75	0.0099976584333201\\
599.76	0.00999783591722828\\
599.77	0.00999800695464968\\
599.78	0.00999817147911925\\
599.79	0.00999832942351129\\
599.8	0.00999848072003288\\
599.81	0.00999862530021737\\
599.82	0.00999876309491766\\
599.83	0.00999889403429953\\
599.84	0.00999901804783486\\
599.85	0.00999913506429477\\
599.86	0.00999924501174274\\
599.87	0.00999934781752758\\
599.88	0.00999944340827644\\
599.89	0.00999953170988768\\
599.9	0.00999961264752365\\
599.91	0.00999968614560349\\
599.92	0.00999975212779576\\
599.93	0.00999981051701107\\
599.94	0.0099998612353946\\
599.95	0.00999990420431857\\
599.96	0.00999993934437464\\
599.97	0.00999996657536618\\
599.98	0.00999998581630055\\
599.99	0.00999999698538124\\
600	0.01\\
};
\addplot [color=mycolor17,solid,forget plot]
  table[row sep=crcr]{%
0.01	0\\
1.01	0\\
2.01	0\\
3.01	0\\
4.01	0\\
5.01	0\\
6.01	0\\
7.01	0\\
8.01	0\\
9.01	0\\
10.01	0\\
11.01	0\\
12.01	0\\
13.01	0\\
14.01	0\\
15.01	0\\
16.01	0\\
17.01	0\\
18.01	0\\
19.01	0\\
20.01	0\\
21.01	0\\
22.01	0\\
23.01	0\\
24.01	0\\
25.01	0\\
26.01	0\\
27.01	0\\
28.01	0\\
29.01	0\\
30.01	0\\
31.01	0\\
32.01	0\\
33.01	0\\
34.01	0\\
35.01	0\\
36.01	0\\
37.01	0\\
38.01	0\\
39.01	0\\
40.01	0\\
41.01	0\\
42.01	0\\
43.01	0\\
44.01	0\\
45.01	0\\
46.01	0\\
47.01	0\\
48.01	0\\
49.01	0\\
50.01	0\\
51.01	0\\
52.01	0\\
53.01	0\\
54.01	0\\
55.01	0\\
56.01	0\\
57.01	0\\
58.01	0\\
59.01	0\\
60.01	0\\
61.01	0\\
62.01	0\\
63.01	0\\
64.01	0\\
65.01	0\\
66.01	0\\
67.01	0\\
68.01	0\\
69.01	0\\
70.01	0\\
71.01	0\\
72.01	0\\
73.01	0\\
74.01	0\\
75.01	0\\
76.01	0\\
77.01	0\\
78.01	0\\
79.01	0\\
80.01	0\\
81.01	0\\
82.01	0\\
83.01	0\\
84.01	0\\
85.01	0\\
86.01	0\\
87.01	0\\
88.01	0\\
89.01	0\\
90.01	0\\
91.01	0\\
92.01	0\\
93.01	0\\
94.01	0\\
95.01	0\\
96.01	0\\
97.01	0\\
98.01	0\\
99.01	0\\
100.01	0\\
101.01	0\\
102.01	0\\
103.01	0\\
104.01	0\\
105.01	0\\
106.01	0\\
107.01	0\\
108.01	0\\
109.01	0\\
110.01	0\\
111.01	0\\
112.01	0\\
113.01	0\\
114.01	0\\
115.01	0\\
116.01	0\\
117.01	0\\
118.01	0\\
119.01	0\\
120.01	0\\
121.01	0\\
122.01	0\\
123.01	0\\
124.01	0\\
125.01	0\\
126.01	0\\
127.01	0\\
128.01	0\\
129.01	0\\
130.01	0\\
131.01	0\\
132.01	0\\
133.01	0\\
134.01	0\\
135.01	0\\
136.01	0\\
137.01	0\\
138.01	0\\
139.01	0\\
140.01	0\\
141.01	0\\
142.01	0\\
143.01	0\\
144.01	0\\
145.01	0\\
146.01	0\\
147.01	0\\
148.01	0\\
149.01	0\\
150.01	0\\
151.01	0\\
152.01	0\\
153.01	0\\
154.01	0\\
155.01	0\\
156.01	0\\
157.01	0\\
158.01	0\\
159.01	0\\
160.01	0\\
161.01	0\\
162.01	0\\
163.01	0\\
164.01	0\\
165.01	0\\
166.01	0\\
167.01	0\\
168.01	0\\
169.01	0\\
170.01	0\\
171.01	0\\
172.01	0\\
173.01	0\\
174.01	0\\
175.01	0\\
176.01	0\\
177.01	0\\
178.01	0\\
179.01	0\\
180.01	0\\
181.01	0\\
182.01	0\\
183.01	0\\
184.01	0\\
185.01	0\\
186.01	0\\
187.01	0\\
188.01	0\\
189.01	0\\
190.01	0\\
191.01	0\\
192.01	0\\
193.01	0\\
194.01	0\\
195.01	0\\
196.01	0\\
197.01	0\\
198.01	0\\
199.01	0\\
200.01	0\\
201.01	0\\
202.01	0\\
203.01	0\\
204.01	0\\
205.01	0\\
206.01	0\\
207.01	0\\
208.01	0\\
209.01	0\\
210.01	0\\
211.01	0\\
212.01	0\\
213.01	0\\
214.01	0\\
215.01	0\\
216.01	0\\
217.01	0\\
218.01	0\\
219.01	0\\
220.01	0\\
221.01	0\\
222.01	0\\
223.01	0\\
224.01	0\\
225.01	0\\
226.01	0\\
227.01	0\\
228.01	0\\
229.01	0\\
230.01	0\\
231.01	0\\
232.01	0\\
233.01	0\\
234.01	0\\
235.01	0\\
236.01	0\\
237.01	0\\
238.01	0\\
239.01	0\\
240.01	0\\
241.01	0\\
242.01	0\\
243.01	0\\
244.01	0\\
245.01	0\\
246.01	0\\
247.01	0\\
248.01	0\\
249.01	0\\
250.01	0\\
251.01	0\\
252.01	0\\
253.01	0\\
254.01	0\\
255.01	0\\
256.01	0\\
257.01	0\\
258.01	0\\
259.01	0\\
260.01	0\\
261.01	0\\
262.01	0\\
263.01	0\\
264.01	0\\
265.01	0\\
266.01	0\\
267.01	0\\
268.01	0\\
269.01	0\\
270.01	0\\
271.01	0\\
272.01	0\\
273.01	0\\
274.01	0\\
275.01	0\\
276.01	0\\
277.01	0\\
278.01	0\\
279.01	0\\
280.01	0\\
281.01	0\\
282.01	0\\
283.01	0\\
284.01	0\\
285.01	0\\
286.01	0\\
287.01	0\\
288.01	0\\
289.01	0\\
290.01	0\\
291.01	0\\
292.01	0\\
293.01	0\\
294.01	0\\
295.01	0\\
296.01	0\\
297.01	0\\
298.01	0\\
299.01	0\\
300.01	0\\
301.01	0\\
302.01	0\\
303.01	0\\
304.01	0\\
305.01	0\\
306.01	0\\
307.01	0\\
308.01	0\\
309.01	0\\
310.01	0\\
311.01	0\\
312.01	0\\
313.01	0\\
314.01	0\\
315.01	0\\
316.01	0\\
317.01	0\\
318.01	0\\
319.01	0\\
320.01	0\\
321.01	0\\
322.01	0\\
323.01	0\\
324.01	0\\
325.01	0\\
326.01	0\\
327.01	0\\
328.01	0\\
329.01	0\\
330.01	0\\
331.01	0\\
332.01	0\\
333.01	0\\
334.01	0\\
335.01	0\\
336.01	0\\
337.01	0\\
338.01	0\\
339.01	0\\
340.01	0\\
341.01	0\\
342.01	0\\
343.01	0\\
344.01	0\\
345.01	0\\
346.01	0\\
347.01	0\\
348.01	0\\
349.01	0\\
350.01	0\\
351.01	0\\
352.01	0\\
353.01	0\\
354.01	0\\
355.01	0\\
356.01	0\\
357.01	0\\
358.01	0\\
359.01	0\\
360.01	0\\
361.01	0\\
362.01	0\\
363.01	0\\
364.01	0\\
365.01	0\\
366.01	0\\
367.01	0\\
368.01	0\\
369.01	0\\
370.01	0\\
371.01	0\\
372.01	0\\
373.01	0\\
374.01	0\\
375.01	0\\
376.01	0\\
377.01	0\\
378.01	0\\
379.01	0\\
380.01	0\\
381.01	0\\
382.01	0\\
383.01	0\\
384.01	0\\
385.01	0\\
386.01	0\\
387.01	0\\
388.01	0\\
389.01	0\\
390.01	0\\
391.01	0\\
392.01	0\\
393.01	0\\
394.01	0\\
395.01	0\\
396.01	0\\
397.01	0\\
398.01	0\\
399.01	0\\
400.01	0\\
401.01	0\\
402.01	0\\
403.01	0\\
404.01	0\\
405.01	0\\
406.01	0\\
407.01	0\\
408.01	0\\
409.01	0\\
410.01	0\\
411.01	0\\
412.01	0\\
413.01	0\\
414.01	0\\
415.01	0\\
416.01	0\\
417.01	0\\
418.01	0\\
419.01	0\\
420.01	0\\
421.01	0\\
422.01	0\\
423.01	0\\
424.01	0\\
425.01	0\\
426.01	0\\
427.01	0\\
428.01	0\\
429.01	0\\
430.01	0\\
431.01	0\\
432.01	0\\
433.01	0\\
434.01	0\\
435.01	0\\
436.01	0\\
437.01	0\\
438.01	0\\
439.01	0\\
440.01	0\\
441.01	0\\
442.01	0\\
443.01	0\\
444.01	0\\
445.01	0\\
446.01	0\\
447.01	0\\
448.01	0\\
449.01	0\\
450.01	0\\
451.01	0\\
452.01	0\\
453.01	0\\
454.01	0\\
455.01	0\\
456.01	0\\
457.01	0\\
458.01	0\\
459.01	0\\
460.01	0\\
461.01	0\\
462.01	0\\
463.01	0\\
464.01	0\\
465.01	0\\
466.01	0\\
467.01	0\\
468.01	0\\
469.01	0\\
470.01	0\\
471.01	0\\
472.01	0\\
473.01	0\\
474.01	0\\
475.01	0\\
476.01	0\\
477.01	0\\
478.01	0\\
479.01	0\\
480.01	0\\
481.01	0\\
482.01	2.41235054160918e-05\\
483.01	5.71856317362687e-05\\
484.01	9.13442313131146e-05\\
485.01	0.000126644284644054\\
486.01	0.000163131355588451\\
487.01	0.000200850837936584\\
488.01	0.000239846872422192\\
489.01	0.000280160817670599\\
490.01	0.0003218291193606\\
491.01	0.000364880369701101\\
492.01	0.000409331279972655\\
493.01	0.000455181451220144\\
494.01	0.000502437693257939\\
495.01	0.00055115312847636\\
496.01	0.000601389121154525\\
497.01	0.000653209377263549\\
498.01	0.000706679475544134\\
499.01	0.000761866126084601\\
500.01	0.00081883605400275\\
501.01	0.000877654370643646\\
502.01	0.000938382247661296\\
503.01	0.00100107364646979\\
504.01	0.00106577597635273\\
505.01	0.00113257078001797\\
506.01	0.00120156743382579\\
507.01	0.00127288609169367\\
508.01	0.00134665908291764\\
509.01	0.00142303276495134\\
510.01	0.00150216972989477\\
511.01	0.00158425144337775\\
512.01	0.00166948141423347\\
513.01	0.00175808901851725\\
514.01	0.00185033413367008\\
515.01	0.00194651278013633\\
516.01	0.00204696402140443\\
517.01	0.00215207844313003\\
518.01	0.00226230862293083\\
519.01	0.00237818212183268\\
520.01	0.00250031768790579\\
521.01	0.00262944556163042\\
522.01	0.00276643301954448\\
523.01	0.00291231665078933\\
524.01	0.00306834333907414\\
525.01	0.00323576254272315\\
526.01	0.00341209916038765\\
527.01	0.00352028318888959\\
528.01	0.00360406428927963\\
529.01	0.00369034534720328\\
530.01	0.00377916342956684\\
531.01	0.0038705441192735\\
532.01	0.00396449684460019\\
533.01	0.00406100742768074\\
534.01	0.00416002261598433\\
535.01	0.00426144479503196\\
536.01	0.00436512542922912\\
537.01	0.0044708521261581\\
538.01	0.00457833263506474\\
539.01	0.00468717510863339\\
540.01	0.00479686376382003\\
541.01	0.00490672887868641\\
542.01	0.00501589943169712\\
543.01	0.0051231210642415\\
544.01	0.00522665268863226\\
545.01	0.00532577361565408\\
546.01	0.0054256685235166\\
547.01	0.00552662808831839\\
548.01	0.00562801372395479\\
549.01	0.00572894003628174\\
550.01	0.00582833573384003\\
551.01	0.00592867321793919\\
552.01	0.00603119774121374\\
553.01	0.00613561888154726\\
554.01	0.00624150777552676\\
555.01	0.00634796511694036\\
556.01	0.00645334557089851\\
557.01	0.00655717610827899\\
558.01	0.00665915981026362\\
559.01	0.00675913613247077\\
560.01	0.00685716507348205\\
561.01	0.00695316601339255\\
562.01	0.00704673033753052\\
563.01	0.00713750430682598\\
564.01	0.00722529375586857\\
565.01	0.00731113838958158\\
566.01	0.00739571234032634\\
567.01	0.00747902290501914\\
568.01	0.00756119384135522\\
569.01	0.00764227563154391\\
570.01	0.00772234228954321\\
571.01	0.00780154653142491\\
572.01	0.00788010948001083\\
573.01	0.0079582842939276\\
574.01	0.00803623522961562\\
575.01	0.0081139736515744\\
576.01	0.00819151365356962\\
577.01	0.00826889123290443\\
578.01	0.00834616108441969\\
579.01	0.00842339004022776\\
580.01	0.00850064667423554\\
581.01	0.00857798729576122\\
582.01	0.00865544109675605\\
583.01	0.00873301147036204\\
584.01	0.00881069437075346\\
585.01	0.00888848570938689\\
586.01	0.00896638332143378\\
587.01	0.00904438745879397\\
588.01	0.0091224999717733\\
589.01	0.00920072415476087\\
590.01	0.00927906581763692\\
591.01	0.00935753490711693\\
592.01	0.00943614492328837\\
593.01	0.00951491048379248\\
594.01	0.00959384452110459\\
595.01	0.00967295564797865\\
596.01	0.00975224633779213\\
597.01	0.00983171277036777\\
598.01	0.00990824635186025\\
599.01	0.0099705330205288\\
599.02	0.00997104253216608\\
599.03	0.00997154900812005\\
599.04	0.0099720524188544\\
599.05	0.00997255273454198\\
599.06	0.00997304992506192\\
599.07	0.00997354395999672\\
599.08	0.0099740348086293\\
599.09	0.00997452243994009\\
599.1	0.00997500682260398\\
599.11	0.00997548792498733\\
599.12	0.00997596571514491\\
599.13	0.00997644016081682\\
599.14	0.00997691122942541\\
599.15	0.00997737888807207\\
599.16	0.00997784310353411\\
599.17	0.00997830384226156\\
599.18	0.00997876107037387\\
599.19	0.00997921475365675\\
599.2	0.00997966485755874\\
599.21	0.00998011134718799\\
599.22	0.00998055418730882\\
599.23	0.00998099334233836\\
599.24	0.0099814287763431\\
599.25	0.00998186045303541\\
599.26	0.00998228833388642\\
599.27	0.00998271237873929\\
599.28	0.00998313254703763\\
599.29	0.00998354879782151\\
599.3	0.00998396108972346\\
599.31	0.0099843693809645\\
599.32	0.00998477362935002\\
599.33	0.00998517379226572\\
599.34	0.0099855698266734\\
599.35	0.00998596168910684\\
599.36	0.00998634933566753\\
599.37	0.00998673272202041\\
599.38	0.00998711180338954\\
599.39	0.0099874865345538\\
599.4	0.00998785686984241\\
599.41	0.00998822276313055\\
599.42	0.00998858416783486\\
599.43	0.00998894103690888\\
599.44	0.00998929332283851\\
599.45	0.0099896409776374\\
599.46	0.00998998395284222\\
599.47	0.00999032219950803\\
599.48	0.00999065566820346\\
599.49	0.00999098430900594\\
599.5	0.00999130807149682\\
599.51	0.0099916269047565\\
599.52	0.00999194075735945\\
599.53	0.00999224957736924\\
599.54	0.00999255331233348\\
599.55	0.00999285190927873\\
599.56	0.00999314531470532\\
599.57	0.0099934334745822\\
599.58	0.00999371633434168\\
599.59	0.00999399383887409\\
599.6	0.00999426593252248\\
599.61	0.00999453255907718\\
599.62	0.00999479366177034\\
599.63	0.00999504918327044\\
599.64	0.00999529906567672\\
599.65	0.0099955432505135\\
599.66	0.00999578167872461\\
599.67	0.00999601429066754\\
599.68	0.00999624102610774\\
599.69	0.0099964618242127\\
599.7	0.00999667662354612\\
599.71	0.00999688536206187\\
599.72	0.00999708797709802\\
599.73	0.00999728440537074\\
599.74	0.00999747458296815\\
599.75	0.00999765844534413\\
599.76	0.00999783592731206\\
599.77	0.00999800696303847\\
599.78	0.00999817148603667\\
599.79	0.0099983294291603\\
599.8	0.00999848072459683\\
599.81	0.00999862530386092\\
599.82	0.00999876309778785\\
599.83	0.00999889403652677\\
599.84	0.00999901804953394\\
599.85	0.00999913506556586\\
599.86	0.00999924501267239\\
599.87	0.00999934781818976\\
599.88	0.00999944340873354\\
599.89	0.00999953171019148\\
599.9	0.00999961264771636\\
599.91	0.00999968614571872\\
599.92	0.00999975212785955\\
599.93	0.00999981051704284\\
599.94	0.00999986123540816\\
599.95	0.00999990420432308\\
599.96	0.00999993934437554\\
599.97	0.00999996657536618\\
599.98	0.00999998581630055\\
599.99	0.00999999698538124\\
600	0.01\\
};
\addplot [color=mycolor18,solid,forget plot]
  table[row sep=crcr]{%
0.01	0.00105848678225644\\
1.01	0.00105848738535112\\
2.01	0.00105848800148826\\
3.01	0.00105848863095175\\
4.01	0.00105848927403169\\
5.01	0.00105848993102456\\
6.01	0.00105849060223337\\
7.01	0.0010584912879677\\
8.01	0.00105849198854398\\
9.01	0.00105849270428564\\
10.01	0.00105849343552312\\
11.01	0.0010584941825942\\
12.01	0.00105849494584402\\
13.01	0.00105849572562536\\
14.01	0.00105849652229869\\
15.01	0.00105849733623244\\
16.01	0.00105849816780314\\
17.01	0.00105849901739562\\
18.01	0.00105849988540314\\
19.01	0.00105850077222761\\
20.01	0.00105850167827988\\
21.01	0.00105850260397972\\
22.01	0.00105850354975626\\
23.01	0.00105850451604794\\
24.01	0.00105850550330294\\
25.01	0.00105850651197934\\
26.01	0.00105850754254529\\
27.01	0.00105850859547925\\
28.01	0.00105850967127027\\
29.01	0.00105851077041814\\
30.01	0.00105851189343364\\
31.01	0.00105851304083886\\
32.01	0.00105851421316747\\
33.01	0.00105851541096488\\
34.01	0.00105851663478848\\
35.01	0.00105851788520805\\
36.01	0.00105851916280597\\
37.01	0.00105852046817738\\
38.01	0.00105852180193065\\
39.01	0.00105852316468762\\
40.01	0.00105852455708381\\
41.01	0.00105852597976887\\
42.01	0.00105852743340677\\
43.01	0.0010585289186762\\
44.01	0.00105853043627081\\
45.01	0.00105853198689978\\
46.01	0.00105853357128772\\
47.01	0.00105853519017553\\
48.01	0.00105853684432037\\
49.01	0.00105853853449629\\
50.01	0.00105854026149447\\
51.01	0.00105854202612354\\
52.01	0.00105854382921019\\
53.01	0.0010585456715994\\
54.01	0.00105854755415486\\
55.01	0.0010585494777595\\
56.01	0.00105855144331582\\
57.01	0.00105855345174632\\
58.01	0.00105855550399404\\
59.01	0.00105855760102301\\
60.01	0.00105855974381854\\
61.01	0.00105856193338804\\
62.01	0.00105856417076119\\
63.01	0.00105856645699056\\
64.01	0.00105856879315229\\
65.01	0.00105857118034624\\
66.01	0.00105857361969698\\
67.01	0.00105857611235399\\
68.01	0.00105857865949237\\
69.01	0.00105858126231338\\
70.01	0.00105858392204514\\
71.01	0.00105858663994311\\
72.01	0.0010585894172907\\
73.01	0.00105859225539998\\
74.01	0.00105859515561232\\
75.01	0.00105859811929897\\
76.01	0.00105860114786185\\
77.01	0.00105860424273414\\
78.01	0.00105860740538106\\
79.01	0.00105861063730053\\
80.01	0.00105861394002404\\
81.01	0.00105861731511729\\
82.01	0.00105862076418094\\
83.01	0.0010586242888515\\
84.01	0.00105862789080213\\
85.01	0.00105863157174338\\
86.01	0.00105863533342422\\
87.01	0.00105863917763265\\
88.01	0.00105864310619679\\
89.01	0.00105864712098581\\
90.01	0.00105865122391072\\
91.01	0.00105865541692534\\
92.01	0.0010586597020274\\
93.01	0.0010586640812595\\
94.01	0.00105866855671\\
95.01	0.00105867313051422\\
96.01	0.00105867780485542\\
97.01	0.00105868258196597\\
98.01	0.00105868746412843\\
99.01	0.00105869245367658\\
100.01	0.00105869755299677\\
101.01	0.00105870276452912\\
102.01	0.00105870809076852\\
103.01	0.00105871353426618\\
104.01	0.00105871909763068\\
105.01	0.00105872478352945\\
106.01	0.00105873059469002\\
107.01	0.00105873653390139\\
108.01	0.00105874260401557\\
109.01	0.0010587488079488\\
110.01	0.00105875514868323\\
111.01	0.00105876162926835\\
112.01	0.00105876825282249\\
113.01	0.00105877502253445\\
114.01	0.0010587819416652\\
115.01	0.0010587890135493\\
116.01	0.00105879624159688\\
117.01	0.00105880362929518\\
118.01	0.00105881118021041\\
119.01	0.00105881889798953\\
120.01	0.00105882678636213\\
121.01	0.00105883484914234\\
122.01	0.00105884309023076\\
123.01	0.00105885151361645\\
124.01	0.00105886012337895\\
125.01	0.0010588689236905\\
126.01	0.00105887791881799\\
127.01	0.00105888711312527\\
128.01	0.00105889651107538\\
129.01	0.0010589061172328\\
130.01	0.00105891593626581\\
131.01	0.00105892597294894\\
132.01	0.0010589362321654\\
133.01	0.00105894671890951\\
134.01	0.00105895743828946\\
135.01	0.00105896839552988\\
136.01	0.00105897959597436\\
137.01	0.00105899104508845\\
138.01	0.00105900274846239\\
139.01	0.00105901471181413\\
140.01	0.00105902694099205\\
141.01	0.00105903944197819\\
142.01	0.00105905222089137\\
143.01	0.00105906528399023\\
144.01	0.00105907863767659\\
145.01	0.00105909228849883\\
146.01	0.00105910624315519\\
147.01	0.00105912050849735\\
148.01	0.00105913509153394\\
149.01	0.00105914999943434\\
150.01	0.00105916523953226\\
151.01	0.00105918081932976\\
152.01	0.00105919674650106\\
153.01	0.00105921302889656\\
154.01	0.00105922967454707\\
155.01	0.00105924669166803\\
156.01	0.00105926408866368\\
157.01	0.0010592818741317\\
158.01	0.00105930005686757\\
159.01	0.00105931864586935\\
160.01	0.00105933765034238\\
161.01	0.00105935707970407\\
162.01	0.00105937694358912\\
163.01	0.00105939725185433\\
164.01	0.00105941801458413\\
165.01	0.00105943924209567\\
166.01	0.00105946094494445\\
167.01	0.00105948313393003\\
168.01	0.00105950582010154\\
169.01	0.00105952901476369\\
170.01	0.00105955272948303\\
171.01	0.00105957697609368\\
172.01	0.00105960176670405\\
173.01	0.00105962711370317\\
174.01	0.0010596530297673\\
175.01	0.00105967952786677\\
176.01	0.00105970662127308\\
177.01	0.00105973432356576\\
178.01	0.00105976264863988\\
179.01	0.00105979161071355\\
180.01	0.00105982122433548\\
181.01	0.00105985150439283\\
182.01	0.00105988246611936\\
183.01	0.00105991412510368\\
184.01	0.0010599464972975\\
185.01	0.00105997959902451\\
186.01	0.00106001344698909\\
187.01	0.00106004805828544\\
188.01	0.00106008345040687\\
189.01	0.00106011964125534\\
190.01	0.00106015664915116\\
191.01	0.00106019449284302\\
192.01	0.00106023319151823\\
193.01	0.00106027276481314\\
194.01	0.00106031323282407\\
195.01	0.0010603546161181\\
196.01	0.00106039693574442\\
197.01	0.00106044021324591\\
198.01	0.00106048447067094\\
199.01	0.00106052973058548\\
200.01	0.00106057601608568\\
201.01	0.00106062335081033\\
202.01	0.00106067175895409\\
203.01	0.00106072126528075\\
204.01	0.00106077189513696\\
205.01	0.00106082367446634\\
206.01	0.00106087662982359\\
207.01	0.00106093078838949\\
208.01	0.00106098617798584\\
209.01	0.00106104282709084\\
210.01	0.0010611007648551\\
211.01	0.00106116002111779\\
212.01	0.00106122062642309\\
213.01	0.00106128261203758\\
214.01	0.00106134600996736\\
215.01	0.00106141085297604\\
216.01	0.00106147717460314\\
217.01	0.00106154500918266\\
218.01	0.00106161439186259\\
219.01	0.00106168535862437\\
220.01	0.00106175794630319\\
221.01	0.00106183219260864\\
222.01	0.00106190813614611\\
223.01	0.00106198581643813\\
224.01	0.00106206527394699\\
225.01	0.0010621465500974\\
226.01	0.00106222968729979\\
227.01	0.00106231472897445\\
228.01	0.00106240171957593\\
229.01	0.00106249070461822\\
230.01	0.00106258173070062\\
231.01	0.00106267484553389\\
232.01	0.00106277009796765\\
233.01	0.00106286753801781\\
234.01	0.00106296721689526\\
235.01	0.00106306918703472\\
236.01	0.00106317350212472\\
237.01	0.00106328021713815\\
238.01	0.00106338938836358\\
239.01	0.00106350107343741\\
240.01	0.0010636153313767\\
241.01	0.00106373222261286\\
242.01	0.0010638518090263\\
243.01	0.00106397415398172\\
244.01	0.0010640993223644\\
245.01	0.00106422738061739\\
246.01	0.00106435839677956\\
247.01	0.00106449244052472\\
248.01	0.00106462958320148\\
249.01	0.00106476989787419\\
250.01	0.00106491345936512\\
251.01	0.00106506034429726\\
252.01	0.00106521063113856\\
253.01	0.00106536440024697\\
254.01	0.00106552173391688\\
255.01	0.00106568271642637\\
256.01	0.00106584743408583\\
257.01	0.0010660159752879\\
258.01	0.00106618843055824\\
259.01	0.00106636489260803\\
260.01	0.00106654545638736\\
261.01	0.00106673021914015\\
262.01	0.00106691928046022\\
263.01	0.00106711274234932\\
264.01	0.00106731070927565\\
265.01	0.00106751328823452\\
266.01	0.00106772058881043\\
267.01	0.00106793272324031\\
268.01	0.00106814980647867\\
269.01	0.00106837195626415\\
270.01	0.0010685992931878\\
271.01	0.00106883194076299\\
272.01	0.00106907002549712\\
273.01	0.00106931367696482\\
274.01	0.00106956302788325\\
275.01	0.00106981821418908\\
276.01	0.00107007937511739\\
277.01	0.00107034665328251\\
278.01	0.00107062019476071\\
279.01	0.00107090014917514\\
280.01	0.00107118666978256\\
281.01	0.00107147991356245\\
282.01	0.00107178004130811\\
283.01	0.00107208721772004\\
284.01	0.00107240161150159\\
285.01	0.00107272339545691\\
286.01	0.00107305274659131\\
287.01	0.00107338984621411\\
288.01	0.00107373488004374\\
289.01	0.0010740880383158\\
290.01	0.00107444951589341\\
291.01	0.00107481951238035\\
292.01	0.00107519823223692\\
293.01	0.00107558588489869\\
294.01	0.00107598268489788\\
295.01	0.00107638885198819\\
296.01	0.00107680461127202\\
297.01	0.00107723019333135\\
298.01	0.00107766583436113\\
299.01	0.00107811177630678\\
300.01	0.00107856826700423\\
301.01	0.00107903556032364\\
302.01	0.00107951391631688\\
303.01	0.00108000360136799\\
304.01	0.00108050488834801\\
305.01	0.00108101805677283\\
306.01	0.0010815433929655\\
307.01	0.00108208119022206\\
308.01	0.0010826317489817\\
309.01	0.00108319537700081\\
310.01	0.00108377238953147\\
311.01	0.00108436310950421\\
312.01	0.00108496786771517\\
313.01	0.00108558700301802\\
314.01	0.00108622086252048\\
315.01	0.0010868698017856\\
316.01	0.0010875341850381\\
317.01	0.00108821438537576\\
318.01	0.00108891078498597\\
319.01	0.00108962377536789\\
320.01	0.00109035375755977\\
321.01	0.00109110114237215\\
322.01	0.00109186635062686\\
323.01	0.00109264981340199\\
324.01	0.00109345197228301\\
325.01	0.00109427327962022\\
326.01	0.00109511419879276\\
327.01	0.0010959752044792\\
328.01	0.00109685678293532\\
329.01	0.00109775943227858\\
330.01	0.00109868366278041\\
331.01	0.00109962999716548\\
332.01	0.00110059897091945\\
333.01	0.00110159113260404\\
334.01	0.0011026070441808\\
335.01	0.00110364728134346\\
336.01	0.00110471243385864\\
337.01	0.00110580310591621\\
338.01	0.00110691991648883\\
339.01	0.00110806349970088\\
340.01	0.00110923450520787\\
341.01	0.00111043359858607\\
342.01	0.00111166146173262\\
343.01	0.00111291879327696\\
344.01	0.00111420630900358\\
345.01	0.00111552474228666\\
346.01	0.00111687484453691\\
347.01	0.00111825738566099\\
348.01	0.0011196731545343\\
349.01	0.00112112295948693\\
350.01	0.00112260762880405\\
351.01	0.00112412801124041\\
352.01	0.00112568497655043\\
353.01	0.0011272794160332\\
354.01	0.00112891224309403\\
355.01	0.0011305843938226\\
356.01	0.0011322968275882\\
357.01	0.00113405052765306\\
358.01	0.001135846501804\\
359.01	0.00113768578300327\\
360.01	0.00113956943005924\\
361.01	0.00114149852831735\\
362.01	0.00114347419037252\\
363.01	0.00114549755680328\\
364.01	0.00114756979692848\\
365.01	0.00114969210958765\\
366.01	0.00115186572394509\\
367.01	0.00115409190031938\\
368.01	0.00115637193103841\\
369.01	0.00115870714132105\\
370.01	0.00116109889018636\\
371.01	0.001163548571391\\
372.01	0.00116605761439588\\
373.01	0.00116862748536304\\
374.01	0.00117125968818357\\
375.01	0.00117395576553731\\
376.01	0.00117671729998578\\
377.01	0.00117954591509881\\
378.01	0.00118244327661627\\
379.01	0.00118541109364588\\
380.01	0.00118845111989776\\
381.01	0.00119156515495732\\
382.01	0.00119475504559751\\
383.01	0.00119802268713113\\
384.01	0.0012013700248051\\
385.01	0.00120479905523727\\
386.01	0.00120831182789741\\
387.01	0.00121191044663376\\
388.01	0.00121559707124597\\
389.01	0.0012193739191066\\
390.01	0.00122324326683203\\
391.01	0.00122720745200462\\
392.01	0.00123126887494768\\
393.01	0.00123543000055477\\
394.01	0.00123969336017508\\
395.01	0.00124406155355684\\
396.01	0.00124853725085029\\
397.01	0.00125312319467234\\
398.01	0.00125782220223461\\
399.01	0.0012626371675372\\
400.01	0.00126757106362953\\
401.01	0.00127262694494131\\
402.01	0.00127780794968466\\
403.01	0.00128311730233054\\
404.01	0.00128855831616062\\
405.01	0.0012941343958978\\
406.01	0.0012998490404172\\
407.01	0.00130570584553994\\
408.01	0.00131170850691277\\
409.01	0.00131786082297575\\
410.01	0.00132416669802152\\
411.01	0.00133063014534897\\
412.01	0.00133725529051567\\
413.01	0.00134404637469249\\
414.01	0.00135100775812517\\
415.01	0.00135814392370794\\
416.01	0.00136545948067388\\
417.01	0.00137295916840727\\
418.01	0.00138064786038253\\
419.01	0.0013885305682341\\
420.01	0.00139661244596099\\
421.01	0.00140489879426959\\
422.01	0.00141339506506206\\
423.01	0.00142210686608133\\
424.01	0.0014310399657305\\
425.01	0.00144020029808806\\
426.01	0.00144959396814871\\
427.01	0.00145922725732696\\
428.01	0.00146910662927189\\
429.01	0.00147923873605477\\
430.01	0.00148963042480637\\
431.01	0.00150028874489973\\
432.01	0.0015112209557965\\
433.01	0.00152243453569886\\
434.01	0.00153393719117675\\
435.01	0.00154573686796874\\
436.01	0.00155784176318209\\
437.01	0.00157026033914148\\
438.01	0.00158300133914879\\
439.01	0.00159607380541275\\
440.01	0.00160948709936898\\
441.01	0.00162325092452457\\
442.01	0.00163737535179515\\
443.01	0.00165187084701578\\
444.01	0.00166674829984293\\
445.01	0.00168201905254234\\
446.01	0.00169769492508298\\
447.01	0.00171378822820518\\
448.01	0.00173031178236495\\
449.01	0.00174727895075731\\
450.01	0.00176470367668922\\
451.01	0.00178260052490536\\
452.01	0.00180098472742406\\
453.01	0.00181987223453527\\
454.01	0.00183927977172944\\
455.01	0.00185922490346439\\
456.01	0.00187972610484628\\
457.01	0.00190080284250637\\
458.01	0.00192247566620689\\
459.01	0.00194476631301662\\
460.01	0.00196769782627855\\
461.01	0.00199129469206012\\
462.01	0.00201558299636367\\
463.01	0.00204059060710265\\
464.01	0.00206634738576364\\
465.01	0.00209288543482385\\
466.01	0.00212023938844311\\
467.01	0.00214844675578658\\
468.01	0.00217754832866662\\
469.01	0.00220758866816539\\
470.01	0.00223861668869814\\
471.01	0.0022706863628443\\
472.01	0.00230385757652669\\
473.01	0.00233819717216527\\
474.01	0.00237378022781636\\
475.01	0.00241069163372452\\
476.01	0.00244902804508352\\
477.01	0.00248890031230211\\
478.01	0.00253043651942755\\
479.01	0.00257378580007774\\
480.01	0.00261912315093559\\
481.01	0.00266665552933484\\
482.01	0.00269227161590402\\
483.01	0.00271078460624117\\
484.01	0.00273013395473023\\
485.01	0.00275041047004711\\
486.01	0.00277172327321201\\
487.01	0.00279420440781401\\
488.01	0.00281801472867966\\
489.01	0.00284335144662574\\
490.01	0.00287045782404049\\
491.01	0.00289963567094933\\
492.01	0.00293126149660711\\
493.01	0.00296573006204356\\
494.01	0.00300177950119776\\
495.01	0.00303880186890301\\
496.01	0.00307682062314735\\
497.01	0.00311586050323926\\
498.01	0.00315594821145552\\
499.01	0.00319711341922895\\
500.01	0.00323939023031811\\
501.01	0.00328281928464718\\
502.01	0.00332745075815011\\
503.01	0.00337334860782597\\
504.01	0.00342059126416831\\
505.01	0.00346923151883183\\
506.01	0.00351930199092616\\
507.01	0.00357083237919477\\
508.01	0.00362384856478\\
509.01	0.00367837126556257\\
510.01	0.00373441432713721\\
511.01	0.00379198254778269\\
512.01	0.00385106890334631\\
513.01	0.00391165099619443\\
514.01	0.00397368649671831\\
515.01	0.00403710727145986\\
516.01	0.00410181179213596\\
517.01	0.00416765528565727\\
518.01	0.00423443690432335\\
519.01	0.00430188292873504\\
520.01	0.00436962463512348\\
521.01	0.00443716985579608\\
522.01	0.00450386647884781\\
523.01	0.0045688549468685\\
524.01	0.00463100624357699\\
525.01	0.00468910256065676\\
526.01	0.00474566599357769\\
527.01	0.00480209616197203\\
528.01	0.00486003753646892\\
529.01	0.00491957866578477\\
530.01	0.00498071188703882\\
531.01	0.00504342381267531\\
532.01	0.00510769845015802\\
533.01	0.00517352687775602\\
534.01	0.00524091998050024\\
535.01	0.00530988917763227\\
536.01	0.00538044046120933\\
537.01	0.0054525732359726\\
538.01	0.00552627887232867\\
539.01	0.00560153885915037\\
540.01	0.0056783223867673\\
541.01	0.00575658310551631\\
542.01	0.00583625478120799\\
543.01	0.00591724929042586\\
544.01	0.00599945128942453\\
545.01	0.00608206031341499\\
546.01	0.00616454974127394\\
547.01	0.00624667176592535\\
548.01	0.00632815937212857\\
549.01	0.0064087471286963\\
550.01	0.00648819414034955\\
551.01	0.0065662572997745\\
552.01	0.00664262514675434\\
553.01	0.00671701960745354\\
554.01	0.00678925865178991\\
555.01	0.00685962124728082\\
556.01	0.0069292250010685\\
557.01	0.00699811736386689\\
558.01	0.00706626972803542\\
559.01	0.00713368620788777\\
560.01	0.0072003987304087\\
561.01	0.00726646369715596\\
562.01	0.0073319849738537\\
563.01	0.00739712337039811\\
564.01	0.00746208685838925\\
565.01	0.00752705934569325\\
566.01	0.00759209712118145\\
567.01	0.00765724568092983\\
568.01	0.00772256084138674\\
569.01	0.00778810616998565\\
570.01	0.00785395204076872\\
571.01	0.00792016901483149\\
572.01	0.00798681920062948\\
573.01	0.00805394793018242\\
574.01	0.0081215814501862\\
575.01	0.00818974004445938\\
576.01	0.00825844526606624\\
577.01	0.0083277184552464\\
578.01	0.00839757873349423\\
579.01	0.00846804113095645\\
580.01	0.00853911525761656\\
581.01	0.00861080500418132\\
582.01	0.0086831096670022\\
583.01	0.00875602586849257\\
584.01	0.00882954825853327\\
585.01	0.0089036692345867\\
586.01	0.00897837850677948\\
587.01	0.00905366261691402\\
588.01	0.00912950454200949\\
589.01	0.00920588340671642\\
590.01	0.00928277421719717\\
591.01	0.00936014749648086\\
592.01	0.00943796891956704\\
593.01	0.00951619928974091\\
594.01	0.00959479511364908\\
595.01	0.00967370999015509\\
596.01	0.00975289702425711\\
597.01	0.00983231245145954\\
598.01	0.00990826323208284\\
599.01	0.00997053306307254\\
599.02	0.00997104257257093\\
599.03	0.00997154904647258\\
599.04	0.00997205245523854\\
599.05	0.00997255276903906\\
599.06	0.00997304995775075\\
599.07	0.00997354399095361\\
599.08	0.00997403483792814\\
599.09	0.00997452246765239\\
599.1	0.00997500684879892\\
599.11	0.00997548794973182\\
599.12	0.00997596573850364\\
599.13	0.00997644018285231\\
599.14	0.00997691125019803\\
599.15	0.00997737890764013\\
599.16	0.0099778431219539\\
599.17	0.00997830385958736\\
599.18	0.00997876108665806\\
599.19	0.00997921476894979\\
599.2	0.00997966487190928\\
599.21	0.00998011136064285\\
599.22	0.00998055419991309\\
599.23	0.00998099335413542\\
599.24	0.00998142878737464\\
599.25	0.00998186046334153\\
599.26	0.00998228834350557\\
599.27	0.0099827123877084\\
599.28	0.00998313255539213\\
599.29	0.00998354880559535\\
599.3	0.00998396109694922\\
599.31	0.00998436938767336\\
599.32	0.00998477363557184\\
599.33	0.00998517379802903\\
599.34	0.0099855698320055\\
599.35	0.0099859616940338\\
599.36	0.00998634934021422\\
599.37	0.00998673272621056\\
599.38	0.00998711180724576\\
599.39	0.00998748653809762\\
599.4	0.00998785687309431\\
599.41	0.00998822276610999\\
599.42	0.00998858417056033\\
599.43	0.00998894103939792\\
599.44	0.00998929332510775\\
599.45	0.00998964097970257\\
599.46	0.00998998395471822\\
599.47	0.00999032220120892\\
599.48	0.00999065566974251\\
599.49	0.00999098431039566\\
599.5	0.00999130807274899\\
599.51	0.0099916269058822\\
599.52	0.00999194075836907\\
599.53	0.00999224957827254\\
599.54	0.00999255331313958\\
599.55	0.00999285190999615\\
599.56	0.00999314531534203\\
599.57	0.00999343347514561\\
599.58	0.00999371633483869\\
599.59	0.0099939938393111\\
599.6	0.00999426593290542\\
599.61	0.00999453255941153\\
599.62	0.00999479366206116\\
599.63	0.00999504918352239\\
599.64	0.00999529906589405\\
599.65	0.00999554325070014\\
599.66	0.00999578167888412\\
599.67	0.00999601429080317\\
599.68	0.00999624102622244\\
599.69	0.00999646182430915\\
599.7	0.00999667662362671\\
599.71	0.00999688536212877\\
599.72	0.00999708797715316\\
599.73	0.00999728440541583\\
599.74	0.00999747458300472\\
599.75	0.00999765844537353\\
599.76	0.00999783592733545\\
599.77	0.00999800696305688\\
599.78	0.00999817148605099\\
599.79	0.0099983294291713\\
599.8	0.00999848072460515\\
599.81	0.00999862530386712\\
599.82	0.00999876309779239\\
599.83	0.00999889403653002\\
599.84	0.00999901804953621\\
599.85	0.00999913506556741\\
599.86	0.00999924501267341\\
599.87	0.00999934781819042\\
599.88	0.00999944340873394\\
599.89	0.00999953171019171\\
599.9	0.00999961264771648\\
599.91	0.00999968614571878\\
599.92	0.00999975212785957\\
599.93	0.00999981051704285\\
599.94	0.00999986123540817\\
599.95	0.00999990420432308\\
599.96	0.00999993934437554\\
599.97	0.00999996657536618\\
599.98	0.00999998581630055\\
599.99	0.00999999698538124\\
600	0.01\\
};
\addplot [color=red!25!mycolor17,solid,forget plot]
  table[row sep=crcr]{%
0.01	0.00260539218361255\\
1.01	0.00260539264278432\\
2.01	0.00260539311185838\\
3.01	0.00260539359104946\\
4.01	0.00260539408057708\\
5.01	0.00260539458066539\\
6.01	0.00260539509154378\\
7.01	0.00260539561344637\\
8.01	0.00260539614661254\\
9.01	0.00260539669128666\\
10.01	0.00260539724771883\\
11.01	0.00260539781616441\\
12.01	0.00260539839688419\\
13.01	0.00260539899014492\\
14.01	0.0026053995962191\\
15.01	0.00260540021538518\\
16.01	0.00260540084792776\\
17.01	0.00260540149413763\\
18.01	0.00260540215431197\\
19.01	0.00260540282875445\\
20.01	0.00260540351777529\\
21.01	0.00260540422169167\\
22.01	0.00260540494082767\\
23.01	0.00260540567551457\\
24.01	0.00260540642609078\\
25.01	0.00260540719290203\\
26.01	0.00260540797630193\\
27.01	0.00260540877665164\\
28.01	0.00260540959432021\\
29.01	0.00260541042968503\\
30.01	0.00260541128313146\\
31.01	0.00260541215505362\\
32.01	0.00260541304585414\\
33.01	0.00260541395594439\\
34.01	0.00260541488574507\\
35.01	0.00260541583568576\\
36.01	0.00260541680620577\\
37.01	0.00260541779775401\\
38.01	0.00260541881078921\\
39.01	0.00260541984578019\\
40.01	0.00260542090320616\\
41.01	0.00260542198355669\\
42.01	0.00260542308733244\\
43.01	0.00260542421504464\\
44.01	0.00260542536721622\\
45.01	0.00260542654438137\\
46.01	0.00260542774708618\\
47.01	0.00260542897588868\\
48.01	0.00260543023135934\\
49.01	0.00260543151408094\\
50.01	0.00260543282464934\\
51.01	0.00260543416367354\\
52.01	0.00260543553177579\\
53.01	0.00260543692959229\\
54.01	0.0026054383577731\\
55.01	0.00260543981698275\\
56.01	0.00260544130790036\\
57.01	0.00260544283122015\\
58.01	0.00260544438765167\\
59.01	0.00260544597791984\\
60.01	0.00260544760276615\\
61.01	0.00260544926294792\\
62.01	0.0026054509592396\\
63.01	0.0026054526924327\\
64.01	0.00260545446333606\\
65.01	0.00260545627277671\\
66.01	0.00260545812159967\\
67.01	0.0026054600106689\\
68.01	0.00260546194086735\\
69.01	0.00260546391309763\\
70.01	0.00260546592828229\\
71.01	0.00260546798736432\\
72.01	0.00260547009130767\\
73.01	0.00260547224109756\\
74.01	0.00260547443774112\\
75.01	0.00260547668226792\\
76.01	0.00260547897573006\\
77.01	0.00260548131920335\\
78.01	0.00260548371378731\\
79.01	0.002605486160606\\
80.01	0.00260548866080827\\
81.01	0.00260549121556854\\
82.01	0.00260549382608743\\
83.01	0.00260549649359222\\
84.01	0.00260549921933741\\
85.01	0.00260550200460542\\
86.01	0.00260550485070714\\
87.01	0.00260550775898287\\
88.01	0.0026055107308026\\
89.01	0.00260551376756671\\
90.01	0.00260551687070691\\
91.01	0.00260552004168674\\
92.01	0.00260552328200254\\
93.01	0.00260552659318374\\
94.01	0.00260552997679393\\
95.01	0.00260553343443186\\
96.01	0.00260553696773154\\
97.01	0.00260554057836374\\
98.01	0.00260554426803632\\
99.01	0.00260554803849547\\
100.01	0.00260555189152625\\
101.01	0.00260555582895361\\
102.01	0.00260555985264348\\
103.01	0.00260556396450329\\
104.01	0.00260556816648316\\
105.01	0.00260557246057697\\
106.01	0.00260557684882302\\
107.01	0.00260558133330541\\
108.01	0.00260558591615481\\
109.01	0.00260559059954978\\
110.01	0.00260559538571746\\
111.01	0.00260560027693499\\
112.01	0.00260560527553061\\
113.01	0.00260561038388478\\
114.01	0.00260561560443133\\
115.01	0.00260562093965878\\
116.01	0.00260562639211136\\
117.01	0.00260563196439052\\
118.01	0.00260563765915636\\
119.01	0.00260564347912827\\
120.01	0.00260564942708721\\
121.01	0.00260565550587662\\
122.01	0.00260566171840369\\
123.01	0.00260566806764113\\
124.01	0.00260567455662879\\
125.01	0.00260568118847448\\
126.01	0.00260568796635635\\
127.01	0.00260569489352419\\
128.01	0.00260570197330107\\
129.01	0.00260570920908465\\
130.01	0.00260571660434966\\
131.01	0.00260572416264891\\
132.01	0.00260573188761569\\
133.01	0.00260573978296506\\
134.01	0.00260574785249615\\
135.01	0.002605756100094\\
136.01	0.00260576452973134\\
137.01	0.00260577314547086\\
138.01	0.00260578195146702\\
139.01	0.00260579095196832\\
140.01	0.00260580015131968\\
141.01	0.00260580955396407\\
142.01	0.00260581916444541\\
143.01	0.00260582898741055\\
144.01	0.00260583902761151\\
145.01	0.00260584928990837\\
146.01	0.0026058597792713\\
147.01	0.00260587050078337\\
148.01	0.00260588145964312\\
149.01	0.00260589266116713\\
150.01	0.00260590411079301\\
151.01	0.00260591581408147\\
152.01	0.00260592777672031\\
153.01	0.00260594000452657\\
154.01	0.00260595250344964\\
155.01	0.00260596527957435\\
156.01	0.00260597833912422\\
157.01	0.00260599168846465\\
158.01	0.00260600533410618\\
159.01	0.00260601928270789\\
160.01	0.00260603354108081\\
161.01	0.00260604811619151\\
162.01	0.0026060630151654\\
163.01	0.00260607824529115\\
164.01	0.00260609381402381\\
165.01	0.00260610972898892\\
166.01	0.00260612599798661\\
167.01	0.0026061426289954\\
168.01	0.00260615963017667\\
169.01	0.00260617700987879\\
170.01	0.00260619477664135\\
171.01	0.00260621293919987\\
172.01	0.00260623150649023\\
173.01	0.00260625048765341\\
174.01	0.00260626989204034\\
175.01	0.0026062897292168\\
176.01	0.00260631000896819\\
177.01	0.00260633074130517\\
178.01	0.00260635193646866\\
179.01	0.00260637360493511\\
180.01	0.00260639575742241\\
181.01	0.00260641840489532\\
182.01	0.00260644155857133\\
183.01	0.00260646522992639\\
184.01	0.00260648943070152\\
185.01	0.00260651417290868\\
186.01	0.00260653946883697\\
187.01	0.00260656533105974\\
188.01	0.00260659177244078\\
189.01	0.00260661880614148\\
190.01	0.00260664644562769\\
191.01	0.0026066747046771\\
192.01	0.00260670359738631\\
193.01	0.002606733138179\\
194.01	0.00260676334181278\\
195.01	0.00260679422338786\\
196.01	0.00260682579835496\\
197.01	0.00260685808252338\\
198.01	0.00260689109206985\\
199.01	0.00260692484354698\\
200.01	0.00260695935389256\\
201.01	0.0026069946404383\\
202.01	0.00260703072091945\\
203.01	0.00260706761348461\\
204.01	0.00260710533670494\\
205.01	0.00260714390958494\\
206.01	0.00260718335157251\\
207.01	0.0026072236825693\\
208.01	0.00260726492294198\\
209.01	0.00260730709353309\\
210.01	0.00260735021567236\\
211.01	0.00260739431118871\\
212.01	0.00260743940242193\\
213.01	0.0026074855122349\\
214.01	0.00260753266402636\\
215.01	0.0026075808817435\\
216.01	0.00260763018989545\\
217.01	0.00260768061356641\\
218.01	0.00260773217842991\\
219.01	0.00260778491076252\\
220.01	0.0026078388374586\\
221.01	0.00260789398604541\\
222.01	0.00260795038469781\\
223.01	0.00260800806225447\\
224.01	0.00260806704823355\\
225.01	0.00260812737284906\\
226.01	0.00260818906702792\\
227.01	0.00260825216242715\\
228.01	0.00260831669145122\\
229.01	0.00260838268727069\\
230.01	0.00260845018384015\\
231.01	0.00260851921591781\\
232.01	0.00260858981908457\\
233.01	0.00260866202976429\\
234.01	0.00260873588524395\\
235.01	0.00260881142369492\\
236.01	0.00260888868419415\\
237.01	0.00260896770674672\\
238.01	0.00260904853230784\\
239.01	0.00260913120280634\\
240.01	0.00260921576116828\\
241.01	0.0026093022513415\\
242.01	0.00260939071832027\\
243.01	0.00260948120817091\\
244.01	0.00260957376805804\\
245.01	0.00260966844627167\\
246.01	0.00260976529225425\\
247.01	0.00260986435662919\\
248.01	0.00260996569122962\\
249.01	0.00261006934912843\\
250.01	0.00261017538466818\\
251.01	0.00261028385349261\\
252.01	0.00261039481257819\\
253.01	0.00261050832026741\\
254.01	0.00261062443630211\\
255.01	0.00261074322185756\\
256.01	0.00261086473957832\\
257.01	0.00261098905361386\\
258.01	0.0026111162296563\\
259.01	0.00261124633497781\\
260.01	0.00261137943846972\\
261.01	0.00261151561068285\\
262.01	0.00261165492386814\\
263.01	0.00261179745201874\\
264.01	0.0026119432709131\\
265.01	0.00261209245815903\\
266.01	0.00261224509323914\\
267.01	0.00261240125755702\\
268.01	0.00261256103448493\\
269.01	0.00261272450941283\\
270.01	0.00261289176979809\\
271.01	0.00261306290521685\\
272.01	0.00261323800741661\\
273.01	0.00261341717037032\\
274.01	0.00261360049033131\\
275.01	0.00261378806589024\\
276.01	0.00261397999803324\\
277.01	0.00261417639020134\\
278.01	0.00261437734835157\\
279.01	0.00261458298101974\\
280.01	0.00261479339938464\\
281.01	0.00261500871733376\\
282.01	0.00261522905153085\\
283.01	0.00261545452148525\\
284.01	0.00261568524962291\\
285.01	0.00261592136135881\\
286.01	0.00261616298517209\\
287.01	0.00261641025268221\\
288.01	0.00261666329872784\\
289.01	0.00261692226144654\\
290.01	0.00261718728235789\\
291.01	0.00261745850644764\\
292.01	0.0026177360822548\\
293.01	0.00261802016196015\\
294.01	0.00261831090147771\\
295.01	0.002618608460548\\
296.01	0.00261891300283368\\
297.01	0.00261922469601805\\
298.01	0.00261954371190569\\
299.01	0.00261987022652566\\
300.01	0.0026202044202371\\
301.01	0.0026205464778384\\
302.01	0.0026208965886776\\
303.01	0.00262125494676742\\
304.01	0.00262162175090123\\
305.01	0.00262199720477367\\
306.01	0.00262238151710291\\
307.01	0.00262277490175722\\
308.01	0.00262317757788366\\
309.01	0.0026235897700408\\
310.01	0.00262401170833447\\
311.01	0.00262444362855645\\
312.01	0.00262488577232747\\
313.01	0.00262533838724324\\
314.01	0.00262580172702445\\
315.01	0.00262627605167021\\
316.01	0.00262676162761559\\
317.01	0.002627258727893\\
318.01	0.00262776763229758\\
319.01	0.00262828862755671\\
320.01	0.00262882200750381\\
321.01	0.00262936807325656\\
322.01	0.00262992713339941\\
323.01	0.00263049950417052\\
324.01	0.00263108550965396\\
325.01	0.00263168548197555\\
326.01	0.00263229976150509\\
327.01	0.00263292869706222\\
328.01	0.00263357264612828\\
329.01	0.00263423197506316\\
330.01	0.0026349070593274\\
331.01	0.00263559828371032\\
332.01	0.00263630604256286\\
333.01	0.00263703074003734\\
334.01	0.00263777279033259\\
335.01	0.00263853261794493\\
336.01	0.00263931065792652\\
337.01	0.00264010735614821\\
338.01	0.00264092316957112\\
339.01	0.00264175856652375\\
340.01	0.00264261402698605\\
341.01	0.00264349004288121\\
342.01	0.00264438711837414\\
343.01	0.00264530577017809\\
344.01	0.00264624652786823\\
345.01	0.00264720993420414\\
346.01	0.00264819654545972\\
347.01	0.00264920693176177\\
348.01	0.00265024167743746\\
349.01	0.00265130138137022\\
350.01	0.00265238665736549\\
351.01	0.00265349813452539\\
352.01	0.00265463645763316\\
353.01	0.00265580228754836\\
354.01	0.00265699630161169\\
355.01	0.00265821919406109\\
356.01	0.0026594716764589\\
357.01	0.00266075447813029\\
358.01	0.00266206834661378\\
359.01	0.00266341404812427\\
360.01	0.0026647923680284\\
361.01	0.00266620411133403\\
362.01	0.00266765010319234\\
363.01	0.00266913118941507\\
364.01	0.00267064823700659\\
365.01	0.00267220213471042\\
366.01	0.00267379379357369\\
367.01	0.0026754241475262\\
368.01	0.00267709415397809\\
369.01	0.00267880479443545\\
370.01	0.00268055707513433\\
371.01	0.00268235202769494\\
372.01	0.00268419070979604\\
373.01	0.00268607420587073\\
374.01	0.00268800362782412\\
375.01	0.0026899801157749\\
376.01	0.00269200483882065\\
377.01	0.00269407899582866\\
378.01	0.00269620381625282\\
379.01	0.00269838056097833\\
380.01	0.00270061052319469\\
381.01	0.00270289502929927\\
382.01	0.0027052354398312\\
383.01	0.00270763315043887\\
384.01	0.00271008959288041\\
385.01	0.00271260623606003\\
386.01	0.0027151845871016\\
387.01	0.00271782619245971\\
388.01	0.00272053263907178\\
389.01	0.00272330555555127\\
390.01	0.00272614661342452\\
391.01	0.0027290575284125\\
392.01	0.00273204006175974\\
393.01	0.00273509602161143\\
394.01	0.00273822726444185\\
395.01	0.00274143569653397\\
396.01	0.00274472327551371\\
397.01	0.00274809201193918\\
398.01	0.00275154397094765\\
399.01	0.00275508127395972\\
400.01	0.0027587061004439\\
401.01	0.00276242068974007\\
402.01	0.00276622734294342\\
403.01	0.00277012842484678\\
404.01	0.00277412636594257\\
405.01	0.00277822366447975\\
406.01	0.00278242288857405\\
407.01	0.00278672667836767\\
408.01	0.00279113774823274\\
409.01	0.00279565888901398\\
410.01	0.00280029297030408\\
411.01	0.00280504294274656\\
412.01	0.00280991184036238\\
413.01	0.00281490278289938\\
414.01	0.00282001897820912\\
415.01	0.00282526372466241\\
416.01	0.00283064041362675\\
417.01	0.00283615253204045\\
418.01	0.00284180366513195\\
419.01	0.0028475974993442\\
420.01	0.00285353782552022\\
421.01	0.00285962854237615\\
422.01	0.00286587366022172\\
423.01	0.00287227730489154\\
424.01	0.00287884372188733\\
425.01	0.00288557728073854\\
426.01	0.00289248247958418\\
427.01	0.00289956394997337\\
428.01	0.00290682646187324\\
429.01	0.00291427492886374\\
430.01	0.00292191441348548\\
431.01	0.00292975013268902\\
432.01	0.00293778746331484\\
433.01	0.00294603194750792\\
434.01	0.00295448929794276\\
435.01	0.00296316540270304\\
436.01	0.00297206632962868\\
437.01	0.00298119832991059\\
438.01	0.00299056784069227\\
439.01	0.00300018148642835\\
440.01	0.00301004607877679\\
441.01	0.0030201686148775\\
442.01	0.00303055627403\\
443.01	0.00304121641308117\\
444.01	0.00305215656133123\\
445.01	0.00306338441657139\\
446.01	0.00307490784610943\\
447.01	0.00308673490179431\\
448.01	0.00309887383173713\\
449.01	0.00311133308019725\\
450.01	0.00312412128530121\\
451.01	0.00313724727516976\\
452.01	0.00315072006209792\\
453.01	0.0031645488343562\\
454.01	0.00317874294508019\\
455.01	0.00319331189759522\\
456.01	0.00320826532636696\\
457.01	0.00322361297257633\\
458.01	0.00323936465306914\\
459.01	0.00325553022112308\\
460.01	0.00327211951707591\\
461.01	0.003289142306356\\
462.01	0.00330660820181151\\
463.01	0.00332452656640618\\
464.01	0.00334290639129109\\
465.01	0.00336175614289581\\
466.01	0.00338108357092422\\
467.01	0.0034008954668763\\
468.01	0.00342119735979278\\
469.01	0.00344199313214273\\
470.01	0.00346328453390141\\
471.01	0.00348507056656741\\
472.01	0.00350734670073966\\
473.01	0.00353010388038894\\
474.01	0.00355332725343989\\
475.01	0.00357699455088265\\
476.01	0.00360107401448095\\
477.01	0.00362552174478718\\
478.01	0.00365027829404276\\
479.01	0.00367526426636504\\
480.01	0.00370037461680556\\
481.01	0.00372547124262447\\
482.01	0.00375060948748071\\
483.01	0.00377623935259928\\
484.01	0.00380234984261363\\
485.01	0.0038288932514617\\
486.01	0.00385580378298964\\
487.01	0.00388299235541427\\
488.01	0.00391033988664548\\
489.01	0.0039376885964225\\
490.01	0.00396483071355128\\
491.01	0.00399149378017413\\
492.01	0.00401732148341034\\
493.01	0.00404192614285756\\
494.01	0.00406657881939221\\
495.01	0.0040919252318317\\
496.01	0.00411798501236087\\
497.01	0.00414477881907857\\
498.01	0.00417232844923256\\
499.01	0.00420065688942448\\
500.01	0.00422978823437099\\
501.01	0.00425974736012985\\
502.01	0.00429055910883907\\
503.01	0.00432224704114471\\
504.01	0.0043548320494991\\
505.01	0.00438833173951764\\
506.01	0.00442276120155938\\
507.01	0.00445813274145353\\
508.01	0.00449445526498889\\
509.01	0.00453173361332232\\
510.01	0.00456996787378437\\
511.01	0.00460915270816233\\
512.01	0.0046492767673382\\
513.01	0.00469032230037527\\
514.01	0.00473226512295134\\
515.01	0.00477507519167798\\
516.01	0.00481871814741117\\
517.01	0.00486315835623537\\
518.01	0.0049083642110119\\
519.01	0.00495431973710523\\
520.01	0.00500104011690757\\
521.01	0.00504856498876362\\
522.01	0.00509696564029012\\
523.01	0.00514636071447201\\
524.01	0.00519693833483995\\
525.01	0.00524898527419951\\
526.01	0.00530278659125299\\
527.01	0.00535847817748133\\
528.01	0.0054161337440972\\
529.01	0.00547577716078581\\
530.01	0.0055373886186283\\
531.01	0.00560088258551496\\
532.01	0.00566607636916837\\
533.01	0.0057322112561036\\
534.01	0.00579862118574934\\
535.01	0.00586514881663358\\
536.01	0.00593161618896077\\
537.01	0.00599782500740147\\
538.01	0.00606355839620054\\
539.01	0.0061285849129024\\
540.01	0.00619266594452595\\
541.01	0.00625556808300582\\
542.01	0.00631708273741376\\
543.01	0.00637705609183324\\
544.01	0.00643544340692038\\
545.01	0.00649303522913385\\
546.01	0.00655024829454778\\
547.01	0.00660701048841029\\
548.01	0.00666326784594137\\
549.01	0.00671899048338381\\
550.01	0.00677417826709985\\
551.01	0.00682886587750969\\
552.01	0.00688312947996752\\
553.01	0.00693709029534157\\
554.01	0.00699091008589604\\
555.01	0.00704476798888722\\
556.01	0.00709875436978568\\
557.01	0.00715290192790601\\
558.01	0.00720725297367087\\
559.01	0.00726185953489205\\
560.01	0.00731678141212418\\
561.01	0.00737208380128448\\
562.01	0.00742783350425905\\
563.01	0.00748409308545202\\
564.01	0.00754091443505246\\
565.01	0.00759833497221399\\
566.01	0.00765638572821262\\
567.01	0.00771509830781651\\
568.01	0.00777450411224081\\
569.01	0.00783463303892859\\
570.01	0.0078955121620811\\
571.01	0.00795716466522816\\
572.01	0.00801960939307402\\
573.01	0.00808286126938456\\
574.01	0.00814693252999731\\
575.01	0.00821183366184495\\
576.01	0.00827757317985894\\
577.01	0.00834415715868387\\
578.01	0.008411588870687\\
579.01	0.00847986856416004\\
580.01	0.00854899337845039\\
581.01	0.00861895734225392\\
582.01	0.00868975134008659\\
583.01	0.0087613629125818\\
584.01	0.00883377590927962\\
585.01	0.00890697012437737\\
586.01	0.008980920965974\\
587.01	0.00905559918081232\\
588.01	0.00913097065499155\\
589.01	0.00920699631211276\\
590.01	0.00928363214520442\\
591.01	0.00936082944810129\\
592.01	0.00943853534123744\\
593.01	0.0095166936882987\\
594.01	0.00959524649431259\\
595.01	0.00967413588099248\\
596.01	0.00975330674558366\\
597.01	0.00983271022700933\\
598.01	0.00990826330213188\\
599.01	0.00997053306356642\\
599.02	0.0099710425730354\\
599.03	0.0099715490469091\\
599.04	0.00997205245564851\\
599.05	0.00997255276942385\\
599.06	0.00997304995811164\\
599.07	0.00997354399129185\\
599.08	0.00997403483824492\\
599.09	0.00997452246794885\\
599.1	0.00997500684907616\\
599.11	0.00997548794999088\\
599.12	0.00997596573874551\\
599.13	0.00997644018307794\\
599.14	0.00997691125040834\\
599.15	0.009977378907836\\
599.16	0.00997784312213615\\
599.17	0.00997830385975678\\
599.18	0.00997876108681542\\
599.19	0.0099792147690958\\
599.2	0.00997966487204462\\
599.21	0.00998011136076818\\
599.22	0.00998055420002904\\
599.23	0.00998099335424256\\
599.24	0.00998142878747354\\
599.25	0.00998186046343272\\
599.26	0.00998228834358955\\
599.27	0.00998271238778565\\
599.28	0.00998313255546309\\
599.29	0.00998354880566047\\
599.3	0.00998396109700889\\
599.31	0.00998436938772796\\
599.32	0.00998477363562173\\
599.33	0.00998517379807456\\
599.34	0.00998556983204699\\
599.35	0.00998596169407154\\
599.36	0.0099863493402485\\
599.37	0.00998673272624164\\
599.38	0.0099871118072739\\
599.39	0.00998748653812304\\
599.4	0.00998785687311724\\
599.41	0.00998822276613064\\
599.42	0.00998858417057887\\
599.43	0.00998894103941455\\
599.44	0.00998929332512263\\
599.45	0.00998964097971585\\
599.46	0.00998998395473005\\
599.47	0.00999032220121943\\
599.48	0.00999065566975183\\
599.49	0.0099909843104039\\
599.5	0.00999130807275626\\
599.51	0.00999162690588858\\
599.52	0.00999194075837467\\
599.53	0.00999224957827743\\
599.54	0.00999255331314384\\
599.55	0.00999285190999985\\
599.56	0.00999314531534523\\
599.57	0.00999343347514837\\
599.58	0.00999371633484106\\
599.59	0.00999399383931313\\
599.6	0.00999426593290714\\
599.61	0.00999453255941299\\
599.62	0.00999479366206239\\
599.63	0.00999504918352342\\
599.64	0.00999529906589491\\
599.65	0.00999554325070085\\
599.66	0.00999578167888471\\
599.67	0.00999601429080365\\
599.68	0.00999624102622283\\
599.69	0.00999646182430947\\
599.7	0.00999667662362697\\
599.71	0.00999688536212897\\
599.72	0.00999708797715331\\
599.73	0.00999728440541595\\
599.74	0.00999747458300482\\
599.75	0.0099976584453736\\
599.76	0.00999783592733551\\
599.77	0.00999800696305692\\
599.78	0.00999817148605102\\
599.79	0.00999832942917133\\
599.8	0.00999848072460517\\
599.81	0.00999862530386713\\
599.82	0.00999876309779239\\
599.83	0.00999889403653003\\
599.84	0.00999901804953622\\
599.85	0.00999913506556741\\
599.86	0.00999924501267341\\
599.87	0.00999934781819042\\
599.88	0.00999944340873394\\
599.89	0.0099995317101917\\
599.9	0.00999961264771648\\
599.91	0.00999968614571878\\
599.92	0.00999975212785958\\
599.93	0.00999981051704285\\
599.94	0.00999986123540817\\
599.95	0.00999990420432308\\
599.96	0.00999993934437554\\
599.97	0.00999996657536618\\
599.98	0.00999998581630055\\
599.99	0.00999999698538124\\
600	0.01\\
};
\addplot [color=mycolor19,solid,forget plot]
  table[row sep=crcr]{%
0.01	0.00343559417631374\\
1.01	0.00343559453459977\\
2.01	0.00343559490056855\\
3.01	0.00343559527438557\\
4.01	0.00343559565621998\\
5.01	0.00343559604624476\\
6.01	0.00343559644463601\\
7.01	0.00343559685157423\\
8.01	0.00343559726724352\\
9.01	0.00343559769183228\\
10.01	0.0034355981255328\\
11.01	0.00343559856854134\\
12.01	0.0034355990210589\\
13.01	0.00343559948329042\\
14.01	0.00343559995544546\\
15.01	0.00343560043773815\\
16.01	0.00343560093038708\\
17.01	0.00343560143361579\\
18.01	0.00343560194765262\\
19.01	0.00343560247273081\\
20.01	0.00343560300908884\\
21.01	0.00343560355697012\\
22.01	0.00343560411662333\\
23.01	0.00343560468830276\\
24.01	0.00343560527226813\\
25.01	0.0034356058687849\\
26.01	0.00343560647812413\\
27.01	0.00343560710056288\\
28.01	0.00343560773638417\\
29.01	0.00343560838587707\\
30.01	0.00343560904933735\\
31.01	0.00343560972706672\\
32.01	0.00343561041937369\\
33.01	0.00343561112657348\\
34.01	0.00343561184898813\\
35.01	0.00343561258694682\\
36.01	0.00343561334078553\\
37.01	0.00343561411084795\\
38.01	0.00343561489748496\\
39.01	0.00343561570105522\\
40.01	0.00343561652192528\\
41.01	0.00343561736046957\\
42.01	0.00343561821707056\\
43.01	0.00343561909211944\\
44.01	0.0034356199860154\\
45.01	0.00343562089916669\\
46.01	0.00343562183199032\\
47.01	0.00343562278491255\\
48.01	0.0034356237583687\\
49.01	0.00343562475280382\\
50.01	0.0034356257686723\\
51.01	0.00343562680643879\\
52.01	0.00343562786657808\\
53.01	0.00343562894957501\\
54.01	0.00343563005592538\\
55.01	0.00343563118613566\\
56.01	0.00343563234072308\\
57.01	0.0034356335202166\\
58.01	0.00343563472515658\\
59.01	0.00343563595609524\\
60.01	0.00343563721359651\\
61.01	0.00343563849823729\\
62.01	0.00343563981060649\\
63.01	0.00343564115130619\\
64.01	0.00343564252095148\\
65.01	0.00343564392017094\\
66.01	0.00343564534960719\\
67.01	0.00343564680991645\\
68.01	0.00343564830176954\\
69.01	0.00343564982585194\\
70.01	0.00343565138286406\\
71.01	0.00343565297352166\\
72.01	0.0034356545985562\\
73.01	0.00343565625871528\\
74.01	0.0034356579547626\\
75.01	0.0034356596874788\\
76.01	0.00343566145766162\\
77.01	0.0034356632661261\\
78.01	0.00343566511370523\\
79.01	0.00343566700125036\\
80.01	0.0034356689296314\\
81.01	0.00343567089973741\\
82.01	0.00343567291247676\\
83.01	0.00343567496877807\\
84.01	0.00343567706959001\\
85.01	0.0034356792158823\\
86.01	0.00343568140864589\\
87.01	0.00343568364889331\\
88.01	0.00343568593765953\\
89.01	0.0034356882760023\\
90.01	0.00343569066500228\\
91.01	0.00343569310576429\\
92.01	0.00343569559941705\\
93.01	0.00343569814711428\\
94.01	0.00343570075003523\\
95.01	0.00343570340938464\\
96.01	0.00343570612639409\\
97.01	0.00343570890232213\\
98.01	0.00343571173845511\\
99.01	0.00343571463610742\\
100.01	0.0034357175966228\\
101.01	0.00343572062137415\\
102.01	0.00343572371176479\\
103.01	0.00343572686922904\\
104.01	0.00343573009523267\\
105.01	0.00343573339127373\\
106.01	0.00343573675888344\\
107.01	0.00343574019962674\\
108.01	0.00343574371510288\\
109.01	0.00343574730694651\\
110.01	0.00343575097682841\\
111.01	0.0034357547264561\\
112.01	0.00343575855757483\\
113.01	0.00343576247196846\\
114.01	0.00343576647145993\\
115.01	0.00343577055791282\\
116.01	0.00343577473323166\\
117.01	0.00343577899936305\\
118.01	0.00343578335829654\\
119.01	0.00343578781206608\\
120.01	0.00343579236274995\\
121.01	0.00343579701247274\\
122.01	0.00343580176340601\\
123.01	0.00343580661776942\\
124.01	0.00343581157783149\\
125.01	0.00343581664591136\\
126.01	0.00343582182437925\\
127.01	0.00343582711565779\\
128.01	0.00343583252222322\\
129.01	0.0034358380466071\\
130.01	0.00343584369139657\\
131.01	0.00343584945923642\\
132.01	0.00343585535282986\\
133.01	0.00343586137494031\\
134.01	0.00343586752839228\\
135.01	0.00343587381607297\\
136.01	0.00343588024093384\\
137.01	0.00343588680599172\\
138.01	0.00343589351433037\\
139.01	0.00343590036910227\\
140.01	0.00343590737352961\\
141.01	0.00343591453090657\\
142.01	0.00343592184460027\\
143.01	0.00343592931805282\\
144.01	0.00343593695478288\\
145.01	0.00343594475838728\\
146.01	0.00343595273254306\\
147.01	0.00343596088100898\\
148.01	0.00343596920762754\\
149.01	0.00343597771632702\\
150.01	0.00343598641112262\\
151.01	0.00343599529611985\\
152.01	0.0034360043755151\\
153.01	0.00343601365359857\\
154.01	0.00343602313475617\\
155.01	0.00343603282347156\\
156.01	0.00343604272432865\\
157.01	0.00343605284201348\\
158.01	0.00343606318131694\\
159.01	0.00343607374713666\\
160.01	0.00343608454447993\\
161.01	0.00343609557846589\\
162.01	0.00343610685432839\\
163.01	0.00343611837741792\\
164.01	0.00343613015320478\\
165.01	0.00343614218728207\\
166.01	0.00343615448536774\\
167.01	0.00343616705330792\\
168.01	0.00343617989707969\\
169.01	0.00343619302279408\\
170.01	0.00343620643669909\\
171.01	0.00343622014518301\\
172.01	0.00343623415477737\\
173.01	0.00343624847215995\\
174.01	0.0034362631041589\\
175.01	0.00343627805775558\\
176.01	0.0034362933400881\\
177.01	0.00343630895845508\\
178.01	0.00343632492031914\\
179.01	0.00343634123331095\\
180.01	0.00343635790523251\\
181.01	0.00343637494406185\\
182.01	0.00343639235795604\\
183.01	0.0034364101552566\\
184.01	0.0034364283444925\\
185.01	0.00343644693438507\\
186.01	0.00343646593385247\\
187.01	0.00343648535201395\\
188.01	0.00343650519819466\\
189.01	0.00343652548193003\\
190.01	0.00343654621297124\\
191.01	0.00343656740128961\\
192.01	0.00343658905708197\\
193.01	0.00343661119077565\\
194.01	0.003436633813034\\
195.01	0.00343665693476168\\
196.01	0.0034366805671102\\
197.01	0.00343670472148404\\
198.01	0.00343672940954601\\
199.01	0.00343675464322341\\
200.01	0.00343678043471383\\
201.01	0.00343680679649225\\
202.01	0.00343683374131652\\
203.01	0.00343686128223443\\
204.01	0.00343688943259063\\
205.01	0.00343691820603274\\
206.01	0.00343694761651915\\
207.01	0.0034369776783261\\
208.01	0.00343700840605471\\
209.01	0.00343703981463908\\
210.01	0.00343707191935368\\
211.01	0.00343710473582111\\
212.01	0.00343713828002092\\
213.01	0.00343717256829719\\
214.01	0.00343720761736765\\
215.01	0.00343724344433219\\
216.01	0.00343728006668149\\
217.01	0.00343731750230686\\
218.01	0.00343735576950878\\
219.01	0.00343739488700738\\
220.01	0.00343743487395204\\
221.01	0.00343747574993084\\
222.01	0.0034375175349818\\
223.01	0.00343756024960291\\
224.01	0.00343760391476328\\
225.01	0.003437648551914\\
226.01	0.00343769418299976\\
227.01	0.00343774083047033\\
228.01	0.00343778851729246\\
229.01	0.00343783726696252\\
230.01	0.00343788710351885\\
231.01	0.00343793805155436\\
232.01	0.0034379901362299\\
233.01	0.00343804338328796\\
234.01	0.00343809781906636\\
235.01	0.00343815347051218\\
236.01	0.00343821036519684\\
237.01	0.00343826853133047\\
238.01	0.00343832799777757\\
239.01	0.00343838879407235\\
240.01	0.00343845095043513\\
241.01	0.00343851449778838\\
242.01	0.00343857946777387\\
243.01	0.00343864589276993\\
244.01	0.00343871380590909\\
245.01	0.00343878324109616\\
246.01	0.00343885423302703\\
247.01	0.00343892681720771\\
248.01	0.00343900102997403\\
249.01	0.00343907690851143\\
250.01	0.00343915449087556\\
251.01	0.00343923381601378\\
252.01	0.00343931492378666\\
253.01	0.00343939785498987\\
254.01	0.00343948265137684\\
255.01	0.00343956935568286\\
256.01	0.00343965801164872\\
257.01	0.00343974866404484\\
258.01	0.00343984135869671\\
259.01	0.00343993614251079\\
260.01	0.00344003306350138\\
261.01	0.00344013217081733\\
262.01	0.00344023351476999\\
263.01	0.0034403371468618\\
264.01	0.00344044311981602\\
265.01	0.00344055148760665\\
266.01	0.00344066230548902\\
267.01	0.00344077563003226\\
268.01	0.00344089151915118\\
269.01	0.0034410100321399\\
270.01	0.00344113122970625\\
271.01	0.00344125517400698\\
272.01	0.00344138192868342\\
273.01	0.0034415115588991\\
274.01	0.00344164413137751\\
275.01	0.00344177971444133\\
276.01	0.00344191837805223\\
277.01	0.00344206019385215\\
278.01	0.00344220523520564\\
279.01	0.00344235357724308\\
280.01	0.00344250529690504\\
281.01	0.00344266047298834\\
282.01	0.00344281918619285\\
283.01	0.00344298151916927\\
284.01	0.00344314755656906\\
285.01	0.00344331738509535\\
286.01	0.0034434910935543\\
287.01	0.00344366877290943\\
288.01	0.00344385051633594\\
289.01	0.00344403641927789\\
290.01	0.00344422657950545\\
291.01	0.0034444210971753\\
292.01	0.00344462007489048\\
293.01	0.00344482361776452\\
294.01	0.00344503183348516\\
295.01	0.00344524483238066\\
296.01	0.00344546272748816\\
297.01	0.00344568563462302\\
298.01	0.00344591367245109\\
299.01	0.00344614696256205\\
300.01	0.00344638562954548\\
301.01	0.00344662980106772\\
302.01	0.00344687960795297\\
303.01	0.00344713518426423\\
304.01	0.00344739666738807\\
305.01	0.00344766419812098\\
306.01	0.00344793792075825\\
307.01	0.0034482179831853\\
308.01	0.00344850453697115\\
309.01	0.00344879773746484\\
310.01	0.00344909774389428\\
311.01	0.00344940471946746\\
312.01	0.00344971883147739\\
313.01	0.00345004025140842\\
314.01	0.00345036915504633\\
315.01	0.00345070572259158\\
316.01	0.00345105013877459\\
317.01	0.00345140259297511\\
318.01	0.00345176327934427\\
319.01	0.00345213239692985\\
320.01	0.00345251014980524\\
321.01	0.00345289674720113\\
322.01	0.00345329240364159\\
323.01	0.00345369733908277\\
324.01	0.00345411177905587\\
325.01	0.00345453595481371\\
326.01	0.00345497010348066\\
327.01	0.00345541446820728\\
328.01	0.00345586929832845\\
329.01	0.00345633484952479\\
330.01	0.00345681138398987\\
331.01	0.00345729917060003\\
332.01	0.00345779848508969\\
333.01	0.00345830961023015\\
334.01	0.00345883283601314\\
335.01	0.00345936845983914\\
336.01	0.00345991678670948\\
337.01	0.00346047812942478\\
338.01	0.00346105280878561\\
339.01	0.00346164115380063\\
340.01	0.00346224350189756\\
341.01	0.00346286019913965\\
342.01	0.00346349160044822\\
343.01	0.00346413806982857\\
344.01	0.00346479998060209\\
345.01	0.00346547771564318\\
346.01	0.00346617166762116\\
347.01	0.00346688223924807\\
348.01	0.00346760984353063\\
349.01	0.00346835490402887\\
350.01	0.00346911785511868\\
351.01	0.00346989914226076\\
352.01	0.00347069922227436\\
353.01	0.00347151856361652\\
354.01	0.00347235764666654\\
355.01	0.00347321696401559\\
356.01	0.00347409702076235\\
357.01	0.00347499833481291\\
358.01	0.00347592143718652\\
359.01	0.00347686687232664\\
360.01	0.00347783519841638\\
361.01	0.0034788269876997\\
362.01	0.00347984282680747\\
363.01	0.0034808833170881\\
364.01	0.00348194907494316\\
365.01	0.00348304073216839\\
366.01	0.0034841589362978\\
367.01	0.00348530435095379\\
368.01	0.00348647765620154\\
369.01	0.00348767954890673\\
370.01	0.00348891074309854\\
371.01	0.00349017197033631\\
372.01	0.0034914639800805\\
373.01	0.0034927875400674\\
374.01	0.00349414343668847\\
375.01	0.00349553247537289\\
376.01	0.00349695548097463\\
377.01	0.00349841329816351\\
378.01	0.00349990679182058\\
379.01	0.00350143684743776\\
380.01	0.00350300437152321\\
381.01	0.00350461029201103\\
382.01	0.00350625555867759\\
383.01	0.00350794114356474\\
384.01	0.00350966804141054\\
385.01	0.00351143727008986\\
386.01	0.00351324987106556\\
387.01	0.00351510690985322\\
388.01	0.00351700947650166\\
389.01	0.00351895868609191\\
390.01	0.00352095567926007\\
391.01	0.00352300162274778\\
392.01	0.00352509770998558\\
393.01	0.0035272451617181\\
394.01	0.00352944522667605\\
395.01	0.00353169918230826\\
396.01	0.00353400833558194\\
397.01	0.00353637402386713\\
398.01	0.00353879761591694\\
399.01	0.00354128051296511\\
400.01	0.00354382414995573\\
401.01	0.00354642999692976\\
402.01	0.00354909956058921\\
403.01	0.00355183438606478\\
404.01	0.00355463605890932\\
405.01	0.00355750620734187\\
406.01	0.0035604465047627\\
407.01	0.00356345867255292\\
408.01	0.00356654448316425\\
409.01	0.00356970576349025\\
410.01	0.00357294439849051\\
411.01	0.00357626233501153\\
412.01	0.00357966158571654\\
413.01	0.00358314423299301\\
414.01	0.00358671243266216\\
415.01	0.0035903684172703\\
416.01	0.00359411449870714\\
417.01	0.00359795306989642\\
418.01	0.0036018866053745\\
419.01	0.00360591766076593\\
420.01	0.00361004887161692\\
421.01	0.00361428295307557\\
422.01	0.00361862270160335\\
423.01	0.003623070997386\\
424.01	0.00362763080677886\\
425.01	0.00363230518474876\\
426.01	0.00363709727729497\\
427.01	0.00364201032382824\\
428.01	0.00364704765948635\\
429.01	0.00365221271735601\\
430.01	0.00365750903057391\\
431.01	0.00366294023427299\\
432.01	0.00366851006733962\\
433.01	0.00367422237394438\\
434.01	0.00368008110481122\\
435.01	0.00368609031818995\\
436.01	0.00369225418050002\\
437.01	0.00369857696662226\\
438.01	0.00370506305982138\\
439.01	0.0037117169512967\\
440.01	0.00371854323936845\\
441.01	0.00372554662832292\\
442.01	0.00373273192694601\\
443.01	0.00374010404676978\\
444.01	0.00374766800002781\\
445.01	0.00375542889723789\\
446.01	0.0037633919441596\\
447.01	0.00377156243746551\\
448.01	0.00377994575851832\\
449.01	0.00378854736558161\\
450.01	0.00379737278454688\\
451.01	0.00380642759803404\\
452.01	0.0038157174327139\\
453.01	0.00382524794470275\\
454.01	0.00383502480289286\\
455.01	0.00384505367009871\\
456.01	0.00385534018193391\\
457.01	0.00386588992337877\\
458.01	0.00387670840307017\\
459.01	0.00388780102543614\\
460.01	0.00389917306092506\\
461.01	0.0039108296147447\\
462.01	0.00392277559473228\\
463.01	0.00393501567924152\\
464.01	0.00394755428624646\\
465.01	0.00396039554523857\\
466.01	0.0039735432739177\\
467.01	0.00398700096213324\\
468.01	0.00400077176597986\\
469.01	0.00401485851532114\\
470.01	0.00402926373817213\\
471.01	0.00404398970512304\\
472.01	0.00405903849599316\\
473.01	0.00407441208866829\\
474.01	0.00409011246579707\\
475.01	0.00410614172744046\\
476.01	0.00412250216221274\\
477.01	0.0041391962609163\\
478.01	0.00415622700906697\\
479.01	0.00417359849123737\\
480.01	0.00419131681237889\\
481.01	0.0042093914679963\\
482.01	0.00422783567027035\\
483.01	0.00424665417779234\\
484.01	0.0042658457506006\\
485.01	0.00428540920979736\\
486.01	0.00430534427978185\\
487.01	0.00432565281531641\\
488.01	0.00434634063993837\\
489.01	0.0043674202444339\\
490.01	0.00438891468839978\\
491.01	0.00441086317693595\\
492.01	0.00443332896078841\\
493.01	0.00445641002823711\\
494.01	0.00448020119439861\\
495.01	0.00450472674675936\\
496.01	0.00453000020260878\\
497.01	0.00455603353798351\\
498.01	0.00458283761352188\\
499.01	0.00461042315757104\\
500.01	0.00463880262288641\\
501.01	0.0046679941228801\\
502.01	0.00469803096466226\\
503.01	0.00472895550970411\\
504.01	0.00476081357674989\\
505.01	0.0047936546479168\\
506.01	0.00482753206190936\\
507.01	0.00486250307105099\\
508.01	0.00489862872544714\\
509.01	0.00493597349382256\\
510.01	0.00497460448799456\\
511.01	0.00501459010252792\\
512.01	0.00505599780398509\\
513.01	0.00509889069695039\\
514.01	0.0051433223451022\\
515.01	0.00518932911893609\\
516.01	0.00523691905500377\\
517.01	0.00528605581371308\\
518.01	0.00533663577069891\\
519.01	0.00538813206587271\\
520.01	0.0054400953075425\\
521.01	0.00549245788702683\\
522.01	0.00554514035176473\\
523.01	0.00559804894147087\\
524.01	0.00565107219355912\\
525.01	0.00570407617318674\\
526.01	0.00575690025895429\\
527.01	0.00580936340913263\\
528.01	0.00586127294259233\\
529.01	0.0059124345303605\\
530.01	0.00596266719498285\\
531.01	0.00601182666056869\\
532.01	0.00605984110978693\\
533.01	0.00610721093348789\\
534.01	0.00615438034487321\\
535.01	0.00620128224286311\\
536.01	0.00624785773615075\\
537.01	0.00629406015322152\\
538.01	0.00633985961947704\\
539.01	0.00638524801153224\\
540.01	0.0064302438979217\\
541.01	0.00647489676076576\\
542.01	0.00651928931073945\\
543.01	0.00656353598216361\\
544.01	0.00660777457638503\\
545.01	0.00665211780239825\\
546.01	0.00669659427891765\\
547.01	0.00674122708944817\\
548.01	0.00678604903754087\\
549.01	0.00683110222486148\\
550.01	0.00687643684249011\\
551.01	0.00692210903677819\\
552.01	0.00696817772137494\\
553.01	0.00701470032773292\\
554.01	0.00706172796307548\\
555.01	0.00710930116119387\\
556.01	0.00715745173033026\\
557.01	0.00720621117745391\\
558.01	0.00725561207670931\\
559.01	0.00730568713866519\\
560.01	0.00735646823249806\\
561.01	0.0074079854563134\\
562.01	0.00746026637928039\\
563.01	0.00751333567174559\\
564.01	0.00756721532443995\\
565.01	0.00762192549083276\\
566.01	0.00767748529274446\\
567.01	0.00773391277025212\\
568.01	0.00779122457082556\\
569.01	0.00784943570452103\\
570.01	0.0079085593937317\\
571.01	0.0079686070252471\\
572.01	0.00802958818111597\\
573.01	0.00809151068745984\\
574.01	0.00815438059004539\\
575.01	0.00821820199788514\\
576.01	0.00828297686270888\\
577.01	0.00834870476305525\\
578.01	0.00841538269761691\\
579.01	0.00848300487906297\\
580.01	0.00855156251639029\\
581.01	0.00862104357475559\\
582.01	0.00869143250950521\\
583.01	0.00876270998652133\\
584.01	0.00883485261300233\\
585.01	0.0089078326983209\\
586.01	0.00898161806023668\\
587.01	0.00905617189448366\\
588.01	0.00913145273077657\\
589.01	0.00920741450519178\\
590.01	0.00928400678726883\\
591.01	0.00936117520881615\\
592.01	0.0094388621492115\\
593.01	0.00951700774080956\\
594.01	0.00959555127130736\\
595.01	0.00967443307848172\\
596.01	0.00975359705744108\\
597.01	0.00983299394040848\\
598.01	0.00990826330297363\\
599.01	0.00997053306357279\\
599.02	0.00997104257304134\\
599.03	0.00997154904691464\\
599.04	0.00997205245565367\\
599.05	0.00997255276942865\\
599.06	0.00997304995811611\\
599.07	0.009973543991296\\
599.08	0.00997403483824878\\
599.09	0.00997452246795243\\
599.1	0.00997500684907947\\
599.11	0.00997548794999394\\
599.12	0.00997596573874834\\
599.13	0.00997644018308057\\
599.14	0.00997691125041076\\
599.15	0.00997737890783823\\
599.16	0.0099778431221382\\
599.17	0.00997830385975868\\
599.18	0.00997876108681716\\
599.19	0.0099792147690974\\
599.2	0.00997966487204609\\
599.21	0.00998011136076953\\
599.22	0.00998055420003026\\
599.23	0.00998099335424368\\
599.24	0.00998142878747457\\
599.25	0.00998186046343366\\
599.26	0.0099822883435904\\
599.27	0.00998271238778643\\
599.28	0.0099831325554638\\
599.29	0.00998354880566111\\
599.3	0.00998396109700947\\
599.31	0.00998436938772849\\
599.32	0.0099847736356222\\
599.33	0.00998517379807499\\
599.34	0.00998556983204737\\
599.35	0.00998596169407188\\
599.36	0.00998634934024881\\
599.37	0.00998673272624191\\
599.38	0.00998711180727415\\
599.39	0.00998748653812326\\
599.4	0.00998785687311743\\
599.41	0.00998822276613081\\
599.42	0.00998858417057903\\
599.43	0.00998894103941468\\
599.44	0.00998929332512275\\
599.45	0.00998964097971596\\
599.46	0.00998998395473014\\
599.47	0.00999032220121951\\
599.48	0.0099906556697519\\
599.49	0.00999098431040396\\
599.5	0.00999130807275631\\
599.51	0.00999162690588863\\
599.52	0.00999194075837471\\
599.53	0.00999224957827746\\
599.54	0.00999255331314386\\
599.55	0.00999285190999987\\
599.56	0.00999314531534524\\
599.57	0.00999343347514839\\
599.58	0.00999371633484107\\
599.59	0.00999399383931314\\
599.6	0.00999426593290715\\
599.61	0.009994532559413\\
599.62	0.0099947936620624\\
599.63	0.00999504918352342\\
599.64	0.00999529906589491\\
599.65	0.00999554325070086\\
599.66	0.00999578167888471\\
599.67	0.00999601429080366\\
599.68	0.00999624102622283\\
599.69	0.00999646182430947\\
599.7	0.00999667662362697\\
599.71	0.00999688536212897\\
599.72	0.00999708797715332\\
599.73	0.00999728440541596\\
599.74	0.00999747458300482\\
599.75	0.0099976584453736\\
599.76	0.0099978359273355\\
599.77	0.00999800696305692\\
599.78	0.00999817148605102\\
599.79	0.00999832942917133\\
599.8	0.00999848072460517\\
599.81	0.00999862530386713\\
599.82	0.00999876309779239\\
599.83	0.00999889403653003\\
599.84	0.00999901804953622\\
599.85	0.00999913506556741\\
599.86	0.00999924501267341\\
599.87	0.00999934781819042\\
599.88	0.00999944340873394\\
599.89	0.0099995317101917\\
599.9	0.00999961264771648\\
599.91	0.00999968614571878\\
599.92	0.00999975212785957\\
599.93	0.00999981051704285\\
599.94	0.00999986123540817\\
599.95	0.00999990420432308\\
599.96	0.00999993934437554\\
599.97	0.00999996657536618\\
599.98	0.00999998581630055\\
599.99	0.00999999698538124\\
600	0.01\\
};
\addplot [color=red!50!mycolor17,solid,forget plot]
  table[row sep=crcr]{%
0.01	0.00372498017920368\\
1.01	0.00372498055307331\\
2.01	0.00372498093491974\\
3.01	0.00372498132491364\\
4.01	0.00372498172322953\\
5.01	0.00372498213004546\\
6.01	0.00372498254554356\\
7.01	0.00372498296990988\\
8.01	0.00372498340333439\\
9.01	0.00372498384601103\\
10.01	0.0037249842981379\\
11.01	0.00372498475991775\\
12.01	0.0037249852315572\\
13.01	0.00372498571326757\\
14.01	0.00372498620526469\\
15.01	0.00372498670776908\\
16.01	0.00372498722100591\\
17.01	0.0037249877452052\\
18.01	0.00372498828060196\\
19.01	0.00372498882743637\\
20.01	0.0037249893859535\\
21.01	0.00372498995640398\\
22.01	0.00372499053904392\\
23.01	0.00372499113413442\\
24.01	0.00372499174194264\\
25.01	0.00372499236274153\\
26.01	0.00372499299680955\\
27.01	0.00372499364443153\\
28.01	0.00372499430589837\\
29.01	0.00372499498150711\\
30.01	0.00372499567156116\\
31.01	0.00372499637637076\\
32.01	0.00372499709625244\\
33.01	0.00372499783152987\\
34.01	0.00372499858253345\\
35.01	0.00372499934960088\\
36.01	0.00372500013307728\\
37.01	0.00372500093331488\\
38.01	0.00372500175067365\\
39.01	0.00372500258552136\\
40.01	0.00372500343823362\\
41.01	0.00372500430919441\\
42.01	0.00372500519879559\\
43.01	0.00372500610743773\\
44.01	0.00372500703553016\\
45.01	0.00372500798349075\\
46.01	0.00372500895174664\\
47.01	0.00372500994073388\\
48.01	0.00372501095089831\\
49.01	0.00372501198269508\\
50.01	0.00372501303658947\\
51.01	0.00372501411305672\\
52.01	0.00372501521258209\\
53.01	0.00372501633566162\\
54.01	0.00372501748280195\\
55.01	0.00372501865452052\\
56.01	0.00372501985134646\\
57.01	0.00372502107381967\\
58.01	0.00372502232249201\\
59.01	0.00372502359792756\\
60.01	0.00372502490070223\\
61.01	0.0037250262314043\\
62.01	0.00372502759063517\\
63.01	0.00372502897900885\\
64.01	0.00372503039715306\\
65.01	0.0037250318457088\\
66.01	0.00372503332533074\\
67.01	0.00372503483668811\\
68.01	0.00372503638046465\\
69.01	0.00372503795735857\\
70.01	0.00372503956808343\\
71.01	0.00372504121336796\\
72.01	0.0037250428939572\\
73.01	0.00372504461061168\\
74.01	0.00372504636410882\\
75.01	0.00372504815524272\\
76.01	0.00372504998482476\\
77.01	0.00372505185368396\\
78.01	0.00372505376266724\\
79.01	0.00372505571263956\\
80.01	0.00372505770448507\\
81.01	0.00372505973910685\\
82.01	0.00372506181742771\\
83.01	0.00372506394039017\\
84.01	0.00372506610895751\\
85.01	0.00372506832411379\\
86.01	0.00372507058686426\\
87.01	0.00372507289823623\\
88.01	0.00372507525927906\\
89.01	0.00372507767106492\\
90.01	0.00372508013468935\\
91.01	0.00372508265127157\\
92.01	0.00372508522195517\\
93.01	0.0037250878479084\\
94.01	0.00372509053032499\\
95.01	0.0037250932704246\\
96.01	0.0037250960694533\\
97.01	0.00372509892868417\\
98.01	0.00372510184941803\\
99.01	0.00372510483298405\\
100.01	0.00372510788073996\\
101.01	0.0037251109940733\\
102.01	0.00372511417440156\\
103.01	0.00372511742317305\\
104.01	0.0037251207418677\\
105.01	0.00372512413199746\\
106.01	0.00372512759510714\\
107.01	0.00372513113277498\\
108.01	0.00372513474661397\\
109.01	0.0037251384382718\\
110.01	0.00372514220943214\\
111.01	0.0037251460618153\\
112.01	0.00372514999717892\\
113.01	0.00372515401731874\\
114.01	0.0037251581240699\\
115.01	0.00372516231930727\\
116.01	0.00372516660494658\\
117.01	0.00372517098294538\\
118.01	0.00372517545530357\\
119.01	0.00372518002406449\\
120.01	0.00372518469131648\\
121.01	0.00372518945919284\\
122.01	0.00372519432987377\\
123.01	0.00372519930558657\\
124.01	0.00372520438860723\\
125.01	0.00372520958126132\\
126.01	0.00372521488592517\\
127.01	0.00372522030502681\\
128.01	0.00372522584104701\\
129.01	0.00372523149652088\\
130.01	0.00372523727403887\\
131.01	0.00372524317624758\\
132.01	0.0037252492058516\\
133.01	0.00372525536561445\\
134.01	0.00372526165835989\\
135.01	0.0037252680869732\\
136.01	0.00372527465440264\\
137.01	0.00372528136366093\\
138.01	0.00372528821782658\\
139.01	0.00372529522004507\\
140.01	0.00372530237353076\\
141.01	0.00372530968156805\\
142.01	0.00372531714751304\\
143.01	0.00372532477479516\\
144.01	0.00372533256691898\\
145.01	0.00372534052746537\\
146.01	0.00372534866009359\\
147.01	0.00372535696854284\\
148.01	0.00372536545663396\\
149.01	0.00372537412827131\\
150.01	0.0037253829874446\\
151.01	0.00372539203823097\\
152.01	0.00372540128479661\\
153.01	0.00372541073139868\\
154.01	0.00372542038238768\\
155.01	0.00372543024220925\\
156.01	0.00372544031540625\\
157.01	0.00372545060662118\\
158.01	0.00372546112059782\\
159.01	0.00372547186218419\\
160.01	0.00372548283633418\\
161.01	0.00372549404811039\\
162.01	0.00372550550268611\\
163.01	0.00372551720534807\\
164.01	0.00372552916149904\\
165.01	0.00372554137665974\\
166.01	0.00372555385647234\\
167.01	0.00372556660670259\\
168.01	0.0037255796332426\\
169.01	0.0037255929421138\\
170.01	0.00372560653946968\\
171.01	0.00372562043159876\\
172.01	0.00372563462492725\\
173.01	0.00372564912602319\\
174.01	0.00372566394159801\\
175.01	0.00372567907851083\\
176.01	0.00372569454377142\\
177.01	0.00372571034454358\\
178.01	0.00372572648814846\\
179.01	0.00372574298206808\\
180.01	0.00372575983394896\\
181.01	0.00372577705160555\\
182.01	0.00372579464302471\\
183.01	0.00372581261636813\\
184.01	0.00372583097997777\\
185.01	0.00372584974237876\\
186.01	0.00372586891228406\\
187.01	0.00372588849859841\\
188.01	0.00372590851042257\\
189.01	0.00372592895705792\\
190.01	0.00372594984801061\\
191.01	0.00372597119299634\\
192.01	0.00372599300194502\\
193.01	0.00372601528500528\\
194.01	0.00372603805254988\\
195.01	0.00372606131518032\\
196.01	0.00372608508373203\\
197.01	0.00372610936927953\\
198.01	0.00372613418314181\\
199.01	0.00372615953688818\\
200.01	0.00372618544234314\\
201.01	0.00372621191159263\\
202.01	0.00372623895698999\\
203.01	0.00372626659116139\\
204.01	0.00372629482701235\\
205.01	0.00372632367773415\\
206.01	0.00372635315680996\\
207.01	0.00372638327802134\\
208.01	0.00372641405545511\\
209.01	0.0037264455035102\\
210.01	0.00372647763690498\\
211.01	0.00372651047068392\\
212.01	0.00372654402022504\\
213.01	0.00372657830124797\\
214.01	0.00372661332982088\\
215.01	0.00372664912236892\\
216.01	0.00372668569568202\\
217.01	0.00372672306692338\\
218.01	0.00372676125363779\\
219.01	0.00372680027376023\\
220.01	0.0037268401456247\\
221.01	0.00372688088797352\\
222.01	0.00372692251996626\\
223.01	0.00372696506118946\\
224.01	0.00372700853166619\\
225.01	0.00372705295186623\\
226.01	0.00372709834271619\\
227.01	0.00372714472560948\\
228.01	0.00372719212241801\\
229.01	0.00372724055550186\\
230.01	0.00372729004772124\\
231.01	0.00372734062244794\\
232.01	0.0037273923035768\\
233.01	0.00372744511553777\\
234.01	0.00372749908330823\\
235.01	0.00372755423242579\\
236.01	0.00372761058900085\\
237.01	0.00372766817973001\\
238.01	0.00372772703190941\\
239.01	0.00372778717344917\\
240.01	0.00372784863288678\\
241.01	0.00372791143940218\\
242.01	0.00372797562283254\\
243.01	0.00372804121368737\\
244.01	0.00372810824316441\\
245.01	0.0037281767431655\\
246.01	0.0037282467463131\\
247.01	0.00372831828596691\\
248.01	0.00372839139624145\\
249.01	0.00372846611202339\\
250.01	0.00372854246899006\\
251.01	0.00372862050362743\\
252.01	0.00372870025324983\\
253.01	0.00372878175601873\\
254.01	0.00372886505096376\\
255.01	0.00372895017800218\\
256.01	0.00372903717796023\\
257.01	0.00372912609259508\\
258.01	0.00372921696461667\\
259.01	0.00372930983771033\\
260.01	0.00372940475655993\\
261.01	0.00372950176687185\\
262.01	0.0037296009153998\\
263.01	0.00372970224996946\\
264.01	0.00372980581950426\\
265.01	0.00372991167405207\\
266.01	0.00373001986481211\\
267.01	0.00373013044416289\\
268.01	0.00373024346569066\\
269.01	0.00373035898421877\\
270.01	0.00373047705583764\\
271.01	0.00373059773793565\\
272.01	0.00373072108923136\\
273.01	0.00373084716980503\\
274.01	0.00373097604113301\\
275.01	0.0037311077661214\\
276.01	0.00373124240914142\\
277.01	0.00373138003606595\\
278.01	0.0037315207143062\\
279.01	0.00373166451285015\\
280.01	0.00373181150230198\\
281.01	0.0037319617549217\\
282.01	0.00373211534466722\\
283.01	0.00373227234723672\\
284.01	0.00373243284011204\\
285.01	0.00373259690260373\\
286.01	0.00373276461589765\\
287.01	0.00373293606310184\\
288.01	0.00373311132929612\\
289.01	0.00373329050158126\\
290.01	0.00373347366913161\\
291.01	0.00373366092324715\\
292.01	0.00373385235740929\\
293.01	0.00373404806733561\\
294.01	0.00373424815103884\\
295.01	0.0037344527088856\\
296.01	0.00373466184365764\\
297.01	0.00373487566061501\\
298.01	0.00373509426756033\\
299.01	0.00373531777490574\\
300.01	0.00373554629574148\\
301.01	0.00373577994590657\\
302.01	0.00373601884406109\\
303.01	0.00373626311176116\\
304.01	0.00373651287353641\\
305.01	0.00373676825696875\\
306.01	0.00373702939277445\\
307.01	0.00373729641488817\\
308.01	0.00373756946054972\\
309.01	0.00373784867039348\\
310.01	0.00373813418853994\\
311.01	0.00373842616269159\\
312.01	0.00373872474422919\\
313.01	0.00373903008831394\\
314.01	0.00373934235399027\\
315.01	0.00373966170429316\\
316.01	0.00373998830635869\\
317.01	0.00374032233153752\\
318.01	0.00374066395551208\\
319.01	0.0037410133584176\\
320.01	0.00374137072496661\\
321.01	0.0037417362445779\\
322.01	0.00374211011150859\\
323.01	0.0037424925249914\\
324.01	0.00374288368937518\\
325.01	0.00374328381427115\\
326.01	0.00374369311470253\\
327.01	0.0037441118112595\\
328.01	0.00374454013025884\\
329.01	0.00374497830390957\\
330.01	0.00374542657048217\\
331.01	0.00374588517448465\\
332.01	0.00374635436684358\\
333.01	0.00374683440509112\\
334.01	0.0037473255535577\\
335.01	0.00374782808357142\\
336.01	0.00374834227366341\\
337.01	0.00374886840977949\\
338.01	0.00374940678549973\\
339.01	0.00374995770226307\\
340.01	0.00375052146960118\\
341.01	0.00375109840537878\\
342.01	0.00375168883604102\\
343.01	0.00375229309686972\\
344.01	0.00375291153224748\\
345.01	0.00375354449592946\\
346.01	0.00375419235132436\\
347.01	0.00375485547178386\\
348.01	0.00375553424090107\\
349.01	0.00375622905281801\\
350.01	0.00375694031254284\\
351.01	0.00375766843627626\\
352.01	0.00375841385174804\\
353.01	0.00375917699856304\\
354.01	0.00375995832855833\\
355.01	0.0037607583061693\\
356.01	0.00376157740880652\\
357.01	0.00376241612724397\\
358.01	0.00376327496601676\\
359.01	0.00376415444382948\\
360.01	0.00376505509397547\\
361.01	0.00376597746476599\\
362.01	0.00376692211996997\\
363.01	0.00376788963926346\\
364.01	0.00376888061868925\\
365.01	0.00376989567112455\\
366.01	0.00377093542675871\\
367.01	0.00377200053357817\\
368.01	0.00377309165785811\\
369.01	0.00377420948466127\\
370.01	0.00377535471834049\\
371.01	0.00377652808304577\\
372.01	0.00377773032323167\\
373.01	0.00377896220416556\\
374.01	0.00378022451243198\\
375.01	0.0037815180564322\\
376.01	0.00378284366687458\\
377.01	0.00378420219725253\\
378.01	0.00378559452430612\\
379.01	0.00378702154846132\\
380.01	0.0037884841942422\\
381.01	0.00378998341065021\\
382.01	0.00379152017150261\\
383.01	0.00379309547572136\\
384.01	0.00379471034756595\\
385.01	0.00379636583679713\\
386.01	0.00379806301876236\\
387.01	0.00379980299439042\\
388.01	0.00380158689008128\\
389.01	0.00380341585747766\\
390.01	0.00380529107310197\\
391.01	0.00380721373784297\\
392.01	0.00380918507627581\\
393.01	0.00381120633579773\\
394.01	0.00381327878556506\\
395.01	0.0038154037152149\\
396.01	0.00381758243336124\\
397.01	0.00381981626585658\\
398.01	0.00382210655382005\\
399.01	0.00382445465143698\\
400.01	0.00382686192355398\\
401.01	0.00382932974310531\\
402.01	0.00383185948843176\\
403.01	0.00383445254058133\\
404.01	0.00383711028071845\\
405.01	0.00383983408781334\\
406.01	0.00384262533683892\\
407.01	0.0038454853977703\\
408.01	0.00384841563575423\\
409.01	0.00385141741289995\\
410.01	0.00385449209222417\\
411.01	0.00385764104435753\\
412.01	0.003860865657658\\
413.01	0.00386416735236141\\
414.01	0.00386754759926033\\
415.01	0.00387100794307384\\
416.01	0.00387455003001864\\
417.01	0.00387817563792722\\
418.01	0.00388188670528924\\
419.01	0.00388568535240304\\
420.01	0.00388957388001282\\
421.01	0.00389355471654147\\
422.01	0.00389763037732452\\
423.01	0.00390180346490945\\
424.01	0.00390607667301784\\
425.01	0.00391045279072186\\
426.01	0.00391493470684755\\
427.01	0.00391952541461714\\
428.01	0.00392422801654253\\
429.01	0.00392904572958684\\
430.01	0.0039339818906044\\
431.01	0.00393903996207619\\
432.01	0.00394422353815439\\
433.01	0.00394953635103058\\
434.01	0.0039549822776398\\
435.01	0.00396056534671387\\
436.01	0.0039662897461922\\
437.01	0.00397215983099526\\
438.01	0.00397818013116025\\
439.01	0.00398435536032973\\
440.01	0.00399069042457371\\
441.01	0.00399719043150867\\
442.01	0.00400386069965954\\
443.01	0.004010706767985\\
444.01	0.00401773440545669\\
445.01	0.00402494962054829\\
446.01	0.00403235867045577\\
447.01	0.00403996806983509\\
448.01	0.00404778459881801\\
449.01	0.00405581531000204\\
450.01	0.00406406753401853\\
451.01	0.00407254888318878\\
452.01	0.00408126725267054\\
453.01	0.00409023081837311\\
454.01	0.00409944803077297\\
455.01	0.0041089276036044\\
456.01	0.00411867849621471\\
457.01	0.00412870988818528\\
458.01	0.00413903114462237\\
459.01	0.00414965177033383\\
460.01	0.00416058135095958\\
461.01	0.00417182947904102\\
462.01	0.00418340566307071\\
463.01	0.00419531921783382\\
464.01	0.0042075791349757\\
465.01	0.00422019393389371\\
466.01	0.00423317149503469\\
467.01	0.00424651888088312\\
468.01	0.00426024215492449\\
469.01	0.00427434621647897\\
470.01	0.00428883468069133\\
471.01	0.0043037098497742\\
472.01	0.00431897284616421\\
473.01	0.00433462401382316\\
474.01	0.00435066374511892\\
475.01	0.00436709400149751\\
476.01	0.00438392296299855\\
477.01	0.00440116621040738\\
478.01	0.00441884136915584\\
479.01	0.00443696778367654\\
480.01	0.00445556670597234\\
481.01	0.00447466146957008\\
482.01	0.0044942776317114\\
483.01	0.004514443281654\\
484.01	0.00453518962365206\\
485.01	0.00455655140357669\\
486.01	0.00457856728399696\\
487.01	0.00460128018570108\\
488.01	0.00462473754685566\\
489.01	0.00464899142450229\\
490.01	0.00467409832483183\\
491.01	0.00470011859366921\\
492.01	0.00472711511991687\\
493.01	0.00475515099407547\\
494.01	0.00478428645556545\\
495.01	0.0048145785588599\\
496.01	0.00484607977893972\\
497.01	0.00487883345753433\\
498.01	0.00491286704141813\\
499.01	0.00494818233849838\\
500.01	0.00498474161227278\\
501.01	0.00502233835144018\\
502.01	0.0050605647173563\\
503.01	0.0050993730901166\\
504.01	0.0051387323839106\\
505.01	0.0051786037412119\\
506.01	0.00521893946779375\\
507.01	0.00525968193544965\\
508.01	0.00530076250996607\\
509.01	0.00534210059584102\\
510.01	0.00538360293910421\\
511.01	0.00542516340129653\\
512.01	0.00546666352000906\\
513.01	0.00550797431673349\\
514.01	0.00554896001829935\\
515.01	0.00558948464782177\\
516.01	0.00562942284808851\\
517.01	0.00566867687090637\\
518.01	0.00570720246416827\\
519.01	0.00574537905056015\\
520.01	0.00578353611736662\\
521.01	0.00582161977278742\\
522.01	0.00585957714925499\\
523.01	0.00589735821902541\\
524.01	0.00593491811364376\\
525.01	0.00597222004792183\\
526.01	0.00600923893205159\\
527.01	0.00604596537409513\\
528.01	0.00608240949177337\\
529.01	0.00611860415938913\\
530.01	0.00615460689620581\\
531.01	0.00619049903407586\\
532.01	0.00622638000393134\\
533.01	0.00626233778218718\\
534.01	0.00629839323853544\\
535.01	0.00633455660955361\\
536.01	0.00637084532387994\\
537.01	0.00640728423694983\\
538.01	0.00644390545520038\\
539.01	0.00648074761806494\\
540.01	0.00651785451344932\\
541.01	0.00655527294240604\\
542.01	0.00659304984595892\\
543.01	0.0066312288998787\\
544.01	0.0066698471308591\\
545.01	0.0067089334512704\\
546.01	0.00674851419094998\\
547.01	0.00678861730368158\\
548.01	0.00682927204376869\\
549.01	0.00687050825608435\\
550.01	0.00691235560656272\\
551.01	0.00695484283077342\\
552.01	0.00699799710791779\\
553.01	0.00704184369372662\\
554.01	0.00708640594671313\\
555.01	0.00713170582569912\\
556.01	0.00717776466084599\\
557.01	0.0072246034337525\\
558.01	0.00727224258703751\\
559.01	0.00732070182243218\\
560.01	0.00736999995604683\\
561.01	0.00742015484321033\\
562.01	0.00747118337583044\\
563.01	0.00752310153781753\\
564.01	0.00757592447950262\\
565.01	0.00762966655047324\\
566.01	0.00768434124365575\\
567.01	0.00773996108862188\\
568.01	0.0077965375481144\\
569.01	0.00785408092385981\\
570.01	0.00791260026552221\\
571.01	0.00797210327360572\\
572.01	0.00803259618583545\\
573.01	0.00809408363856894\\
574.01	0.00815656850118068\\
575.01	0.00822005169027631\\
576.01	0.00828453197232036\\
577.01	0.00835000575654852\\
578.01	0.00841646687724602\\
579.01	0.00848390636516268\\
580.01	0.00855231220950171\\
581.01	0.0086216691142773\\
582.01	0.00869195825546762\\
583.01	0.00876315704751246\\
584.01	0.00883523892879311\\
585.01	0.00890817317693382\\
586.01	0.00898192476731992\\
587.01	0.00905645429185747\\
588.01	0.00913171795951237\\
589.01	0.00920766770565841\\
590.01	0.00928425144388741\\
591.01	0.00936141350201614\\
592.01	0.00943909529417743\\
593.01	0.00951723629392387\\
594.01	0.00959577538991279\\
595.01	0.00967465272665585\\
596.01	0.00975381215885204\\
597.01	0.00983319200642706\\
598.01	0.00990826330299313\\
599.01	0.00997053306357288\\
599.02	0.00997104257304143\\
599.03	0.00997154904691472\\
599.04	0.00997205245565375\\
599.05	0.00997255276942872\\
599.06	0.00997304995811617\\
599.07	0.00997354399129606\\
599.08	0.00997403483824883\\
599.09	0.00997452246795248\\
599.1	0.00997500684907952\\
599.11	0.00997548794999398\\
599.12	0.00997596573874838\\
599.13	0.0099764401830806\\
599.14	0.0099769112504108\\
599.15	0.00997737890783826\\
599.16	0.00997784312213823\\
599.17	0.0099783038597587\\
599.18	0.00997876108681718\\
599.19	0.00997921476909742\\
599.2	0.00997966487204611\\
599.21	0.00998011136076955\\
599.22	0.00998055420003028\\
599.23	0.0099809933542437\\
599.24	0.00998142878747458\\
599.25	0.00998186046343367\\
599.26	0.00998228834359041\\
599.27	0.00998271238778643\\
599.28	0.00998313255546381\\
599.29	0.00998354880566111\\
599.3	0.00998396109700947\\
599.31	0.00998436938772849\\
599.32	0.00998477363562221\\
599.33	0.00998517379807499\\
599.34	0.00998556983204737\\
599.35	0.00998596169407188\\
599.36	0.00998634934024881\\
599.37	0.00998673272624192\\
599.38	0.00998711180727415\\
599.39	0.00998748653812326\\
599.4	0.00998785687311743\\
599.41	0.00998822276613081\\
599.42	0.00998858417057903\\
599.43	0.00998894103941468\\
599.44	0.00998929332512275\\
599.45	0.00998964097971596\\
599.46	0.00998998395473014\\
599.47	0.00999032220121951\\
599.48	0.0099906556697519\\
599.49	0.00999098431040396\\
599.5	0.00999130807275631\\
599.51	0.00999162690588863\\
599.52	0.00999194075837471\\
599.53	0.00999224957827746\\
599.54	0.00999255331314387\\
599.55	0.00999285190999987\\
599.56	0.00999314531534524\\
599.57	0.00999343347514839\\
599.58	0.00999371633484107\\
599.59	0.00999399383931314\\
599.6	0.00999426593290715\\
599.61	0.009994532559413\\
599.62	0.0099947936620624\\
599.63	0.00999504918352342\\
599.64	0.00999529906589491\\
599.65	0.00999554325070086\\
599.66	0.00999578167888471\\
599.67	0.00999601429080366\\
599.68	0.00999624102622283\\
599.69	0.00999646182430947\\
599.7	0.00999667662362697\\
599.71	0.00999688536212897\\
599.72	0.00999708797715332\\
599.73	0.00999728440541595\\
599.74	0.00999747458300482\\
599.75	0.0099976584453736\\
599.76	0.00999783592733551\\
599.77	0.00999800696305692\\
599.78	0.00999817148605102\\
599.79	0.00999832942917133\\
599.8	0.00999848072460517\\
599.81	0.00999862530386713\\
599.82	0.00999876309779239\\
599.83	0.00999889403653003\\
599.84	0.00999901804953622\\
599.85	0.00999913506556741\\
599.86	0.00999924501267341\\
599.87	0.00999934781819042\\
599.88	0.00999944340873394\\
599.89	0.0099995317101917\\
599.9	0.00999961264771648\\
599.91	0.00999968614571878\\
599.92	0.00999975212785958\\
599.93	0.00999981051704285\\
599.94	0.00999986123540817\\
599.95	0.00999990420432308\\
599.96	0.00999993934437554\\
599.97	0.00999996657536618\\
599.98	0.00999998581630055\\
599.99	0.00999999698538124\\
600	0.01\\
};
\addplot [color=red!40!mycolor19,solid,forget plot]
  table[row sep=crcr]{%
0.01	0.00384875486878031\\
1.01	0.00384875533045602\\
2.01	0.00384875580195604\\
3.01	0.00384875628349013\\
4.01	0.00384875677527215\\
5.01	0.00384875727752121\\
6.01	0.00384875779046065\\
7.01	0.00384875831431862\\
8.01	0.00384875884932827\\
9.01	0.00384875939572767\\
10.01	0.00384875995376011\\
11.01	0.00384876052367375\\
12.01	0.00384876110572258\\
13.01	0.00384876170016588\\
14.01	0.00384876230726797\\
15.01	0.00384876292729951\\
16.01	0.0038487635605368\\
17.01	0.00384876420726182\\
18.01	0.0038487648677628\\
19.01	0.00384876554233396\\
20.01	0.00384876623127613\\
21.01	0.00384876693489627\\
22.01	0.00384876765350804\\
23.01	0.0038487683874321\\
24.01	0.00384876913699551\\
25.01	0.0038487699025327\\
26.01	0.00384877068438512\\
27.01	0.00384877148290162\\
28.01	0.00384877229843843\\
29.01	0.00384877313135953\\
30.01	0.00384877398203673\\
31.01	0.00384877485084956\\
32.01	0.00384877573818605\\
33.01	0.00384877664444244\\
34.01	0.00384877757002339\\
35.01	0.00384877851534236\\
36.01	0.00384877948082141\\
37.01	0.00384878046689194\\
38.01	0.00384878147399474\\
39.01	0.0038487825025799\\
40.01	0.00384878355310701\\
41.01	0.00384878462604556\\
42.01	0.00384878572187557\\
43.01	0.00384878684108672\\
44.01	0.00384878798417967\\
45.01	0.0038487891516657\\
46.01	0.0038487903440671\\
47.01	0.00384879156191735\\
48.01	0.00384879280576131\\
49.01	0.00384879407615581\\
50.01	0.00384879537366947\\
51.01	0.00384879669888306\\
52.01	0.0038487980523899\\
53.01	0.00384879943479619\\
54.01	0.00384880084672094\\
55.01	0.00384880228879672\\
56.01	0.0038488037616695\\
57.01	0.00384880526599934\\
58.01	0.00384880680246034\\
59.01	0.00384880837174098\\
60.01	0.00384880997454495\\
61.01	0.00384881161159073\\
62.01	0.00384881328361245\\
63.01	0.00384881499135992\\
64.01	0.00384881673559906\\
65.01	0.00384881851711236\\
66.01	0.00384882033669913\\
67.01	0.00384882219517596\\
68.01	0.00384882409337686\\
69.01	0.00384882603215397\\
70.01	0.00384882801237758\\
71.01	0.00384883003493711\\
72.01	0.00384883210074047\\
73.01	0.0038488342107157\\
74.01	0.00384883636581072\\
75.01	0.00384883856699361\\
76.01	0.00384884081525362\\
77.01	0.00384884311160097\\
78.01	0.00384884545706804\\
79.01	0.00384884785270923\\
80.01	0.00384885029960157\\
81.01	0.0038488527988456\\
82.01	0.00384885535156525\\
83.01	0.00384885795890889\\
84.01	0.0038488606220496\\
85.01	0.00384886334218554\\
86.01	0.00384886612054111\\
87.01	0.00384886895836679\\
88.01	0.00384887185694011\\
89.01	0.00384887481756617\\
90.01	0.00384887784157831\\
91.01	0.00384888093033844\\
92.01	0.00384888408523801\\
93.01	0.00384888730769828\\
94.01	0.00384889059917125\\
95.01	0.00384889396114044\\
96.01	0.00384889739512118\\
97.01	0.00384890090266159\\
98.01	0.00384890448534314\\
99.01	0.00384890814478146\\
100.01	0.00384891188262695\\
101.01	0.00384891570056578\\
102.01	0.00384891960032059\\
103.01	0.00384892358365092\\
104.01	0.00384892765235449\\
105.01	0.00384893180826783\\
106.01	0.00384893605326712\\
107.01	0.00384894038926914\\
108.01	0.00384894481823191\\
109.01	0.00384894934215577\\
110.01	0.00384895396308424\\
111.01	0.00384895868310507\\
112.01	0.00384896350435124\\
113.01	0.00384896842900159\\
114.01	0.00384897345928227\\
115.01	0.00384897859746712\\
116.01	0.00384898384587952\\
117.01	0.00384898920689283\\
118.01	0.0038489946829318\\
119.01	0.00384900027647395\\
120.01	0.00384900599004959\\
121.01	0.00384901182624441\\
122.01	0.00384901778769957\\
123.01	0.00384902387711361\\
124.01	0.00384903009724338\\
125.01	0.0038490364509052\\
126.01	0.00384904294097643\\
127.01	0.00384904957039653\\
128.01	0.00384905634216895\\
129.01	0.00384906325936164\\
130.01	0.00384907032510892\\
131.01	0.00384907754261352\\
132.01	0.0038490849151466\\
133.01	0.00384909244605075\\
134.01	0.00384910013874072\\
135.01	0.00384910799670515\\
136.01	0.00384911602350821\\
137.01	0.00384912422279104\\
138.01	0.00384913259827395\\
139.01	0.00384914115375749\\
140.01	0.00384914989312453\\
141.01	0.00384915882034226\\
142.01	0.00384916793946358\\
143.01	0.00384917725462916\\
144.01	0.00384918677006937\\
145.01	0.00384919649010618\\
146.01	0.00384920641915507\\
147.01	0.00384921656172746\\
148.01	0.00384922692243228\\
149.01	0.0038492375059784\\
150.01	0.00384924831717694\\
151.01	0.00384925936094281\\
152.01	0.00384927064229761\\
153.01	0.003849282166372\\
154.01	0.00384929393840754\\
155.01	0.00384930596375969\\
156.01	0.00384931824789976\\
157.01	0.0038493307964178\\
158.01	0.0038493436150249\\
159.01	0.00384935670955626\\
160.01	0.00384937008597344\\
161.01	0.0038493837503672\\
162.01	0.00384939770896062\\
163.01	0.0038494119681118\\
164.01	0.00384942653431637\\
165.01	0.00384944141421127\\
166.01	0.0038494566145771\\
167.01	0.00384947214234196\\
168.01	0.00384948800458407\\
169.01	0.00384950420853521\\
170.01	0.00384952076158425\\
171.01	0.00384953767128018\\
172.01	0.0038495549453364\\
173.01	0.00384957259163313\\
174.01	0.00384959061822199\\
175.01	0.00384960903332954\\
176.01	0.00384962784536078\\
177.01	0.00384964706290317\\
178.01	0.00384966669473094\\
179.01	0.00384968674980868\\
180.01	0.00384970723729584\\
181.01	0.00384972816655102\\
182.01	0.00384974954713578\\
183.01	0.00384977138881974\\
184.01	0.00384979370158474\\
185.01	0.00384981649562967\\
186.01	0.00384983978137504\\
187.01	0.00384986356946779\\
188.01	0.00384988787078654\\
189.01	0.00384991269644636\\
190.01	0.00384993805780408\\
191.01	0.0038499639664635\\
192.01	0.00384999043428073\\
193.01	0.0038500174733701\\
194.01	0.00385004509610926\\
195.01	0.00385007331514527\\
196.01	0.00385010214340037\\
197.01	0.00385013159407825\\
198.01	0.00385016168066972\\
199.01	0.00385019241695959\\
200.01	0.00385022381703296\\
201.01	0.00385025589528125\\
202.01	0.00385028866640959\\
203.01	0.0038503221454434\\
204.01	0.00385035634773567\\
205.01	0.00385039128897369\\
206.01	0.0038504269851868\\
207.01	0.00385046345275357\\
208.01	0.00385050070841\\
209.01	0.00385053876925687\\
210.01	0.00385057765276775\\
211.01	0.00385061737679758\\
212.01	0.00385065795959111\\
213.01	0.00385069941979058\\
214.01	0.00385074177644565\\
215.01	0.00385078504902181\\
216.01	0.00385082925740939\\
217.01	0.00385087442193333\\
218.01	0.00385092056336243\\
219.01	0.0038509677029194\\
220.01	0.00385101586229071\\
221.01	0.00385106506363686\\
222.01	0.00385111532960304\\
223.01	0.00385116668332969\\
224.01	0.00385121914846331\\
225.01	0.00385127274916822\\
226.01	0.00385132751013714\\
227.01	0.00385138345660398\\
228.01	0.00385144061435491\\
229.01	0.00385149900974111\\
230.01	0.00385155866969135\\
231.01	0.00385161962172451\\
232.01	0.00385168189396309\\
233.01	0.00385174551514626\\
234.01	0.00385181051464401\\
235.01	0.00385187692247087\\
236.01	0.00385194476930062\\
237.01	0.00385201408648062\\
238.01	0.0038520849060474\\
239.01	0.00385215726074126\\
240.01	0.00385223118402295\\
241.01	0.00385230671008926\\
242.01	0.00385238387388961\\
243.01	0.00385246271114327\\
244.01	0.0038525432583562\\
245.01	0.00385262555283911\\
246.01	0.00385270963272513\\
247.01	0.00385279553698928\\
248.01	0.00385288330546632\\
249.01	0.00385297297887111\\
250.01	0.00385306459881798\\
251.01	0.00385315820784145\\
252.01	0.00385325384941676\\
253.01	0.00385335156798175\\
254.01	0.00385345140895808\\
255.01	0.00385355341877441\\
256.01	0.00385365764488861\\
257.01	0.00385376413581204\\
258.01	0.00385387294113282\\
259.01	0.00385398411154134\\
260.01	0.00385409769885486\\
261.01	0.00385421375604393\\
262.01	0.00385433233725853\\
263.01	0.00385445349785557\\
264.01	0.00385457729442669\\
265.01	0.00385470378482684\\
266.01	0.00385483302820358\\
267.01	0.00385496508502703\\
268.01	0.00385510001712074\\
269.01	0.00385523788769319\\
270.01	0.00385537876137027\\
271.01	0.00385552270422845\\
272.01	0.00385566978382858\\
273.01	0.0038558200692513\\
274.01	0.00385597363113238\\
275.01	0.00385613054169973\\
276.01	0.00385629087481112\\
277.01	0.00385645470599272\\
278.01	0.00385662211247893\\
279.01	0.00385679317325323\\
280.01	0.00385696796908966\\
281.01	0.00385714658259634\\
282.01	0.00385732909825905\\
283.01	0.00385751560248686\\
284.01	0.00385770618365856\\
285.01	0.00385790093217053\\
286.01	0.0038580999404859\\
287.01	0.00385830330318483\\
288.01	0.003858511117016\\
289.01	0.003858723480951\\
290.01	0.00385894049623794\\
291.01	0.00385916226645865\\
292.01	0.00385938889758582\\
293.01	0.00385962049804375\\
294.01	0.00385985717876804\\
295.01	0.00386009905327003\\
296.01	0.00386034623770097\\
297.01	0.00386059885091891\\
298.01	0.00386085701455754\\
299.01	0.00386112085309692\\
300.01	0.00386139049393619\\
301.01	0.00386166606746884\\
302.01	0.00386194770716014\\
303.01	0.00386223554962652\\
304.01	0.00386252973471762\\
305.01	0.00386283040560144\\
306.01	0.00386313770885108\\
307.01	0.00386345179453493\\
308.01	0.00386377281630966\\
309.01	0.00386410093151575\\
310.01	0.00386443630127654\\
311.01	0.00386477909060014\\
312.01	0.00386512946848498\\
313.01	0.00386548760802831\\
314.01	0.00386585368653898\\
315.01	0.00386622788565294\\
316.01	0.00386661039145403\\
317.01	0.00386700139459747\\
318.01	0.00386740109043877\\
319.01	0.00386780967916585\\
320.01	0.00386822736593687\\
321.01	0.00386865436102218\\
322.01	0.00386909087995135\\
323.01	0.003869537143666\\
324.01	0.00386999337867754\\
325.01	0.00387045981723068\\
326.01	0.00387093669747325\\
327.01	0.00387142426363196\\
328.01	0.00387192276619501\\
329.01	0.00387243246210124\\
330.01	0.00387295361493713\\
331.01	0.00387348649514033\\
332.01	0.00387403138021206\\
333.01	0.00387458855493709\\
334.01	0.00387515831161349\\
335.01	0.00387574095028989\\
336.01	0.003876336779014\\
337.01	0.00387694611409026\\
338.01	0.0038775692803483\\
339.01	0.0038782066114227\\
340.01	0.00387885845004482\\
341.01	0.00387952514834604\\
342.01	0.00388020706817502\\
343.01	0.00388090458142858\\
344.01	0.00388161807039665\\
345.01	0.00388234792812257\\
346.01	0.00388309455878016\\
347.01	0.00388385837806733\\
348.01	0.00388463981361746\\
349.01	0.00388543930543053\\
350.01	0.00388625730632405\\
351.01	0.00388709428240512\\
352.01	0.00388795071356546\\
353.01	0.00388882709399971\\
354.01	0.00388972393274964\\
355.01	0.00389064175427477\\
356.01	0.00389158109905192\\
357.01	0.00389254252420378\\
358.01	0.00389352660416029\\
359.01	0.0038945339313536\\
360.01	0.0038955651169489\\
361.01	0.0038966207916132\\
362.01	0.0038977016063247\\
363.01	0.00389880823322475\\
364.01	0.00389994136651454\\
365.01	0.00390110172340126\\
366.01	0.00390229004509375\\
367.01	0.00390350709785288\\
368.01	0.00390475367409801\\
369.01	0.00390603059357446\\
370.01	0.00390733870458411\\
371.01	0.00390867888528189\\
372.01	0.00391005204504401\\
373.01	0.00391145912590796\\
374.01	0.00391290110409093\\
375.01	0.00391437899158689\\
376.01	0.00391589383784735\\
377.01	0.0039174467315468\\
378.01	0.0039190388024356\\
379.01	0.00392067122328169\\
380.01	0.00392234521190106\\
381.01	0.00392406203327494\\
382.01	0.00392582300175241\\
383.01	0.0039276294833331\\
384.01	0.00392948289802104\\
385.01	0.00393138472224056\\
386.01	0.00393333649129618\\
387.01	0.00393533980185767\\
388.01	0.00393739631444119\\
389.01	0.00393950775585072\\
390.01	0.0039416759215347\\
391.01	0.0039439026777978\\
392.01	0.00394618996379574\\
393.01	0.00394853979322038\\
394.01	0.00395095425556306\\
395.01	0.00395343551681587\\
396.01	0.00395598581944177\\
397.01	0.00395860748140703\\
398.01	0.00396130289402878\\
399.01	0.00396407451834235\\
400.01	0.00396692487963876\\
401.01	0.00396985655976391\\
402.01	0.00397287218670914\\
403.01	0.00397597442095758\\
404.01	0.00397916593799423\\
405.01	0.00398244940634153\\
406.01	0.003985827460465\\
407.01	0.00398930266792481\\
408.01	0.0039928774902608\\
409.01	0.00399655423733657\\
410.01	0.00400033501530679\\
411.01	0.00400422166910225\\
412.01	0.00400821572150827\\
413.01	0.00401231831273878\\
414.01	0.00401653014719362\\
415.01	0.00402085145824729\\
416.01	0.00402528200806303\\
417.01	0.00402982114841728\\
418.01	0.00403446798156182\\
419.01	0.00403922167896034\\
420.01	0.00404408237476168\\
421.01	0.00404905231270609\\
422.01	0.00405413425232351\\
423.01	0.00405933104602202\\
424.01	0.00406464564393058\\
425.01	0.00407008109911691\\
426.01	0.00407564057321982\\
427.01	0.00408132734253863\\
428.01	0.00408714480462842\\
429.01	0.00409309648545557\\
430.01	0.0040991860471756\\
431.01	0.00410541729660299\\
432.01	0.00411179419445057\\
433.01	0.00411832086542861\\
434.01	0.0041250016093043\\
435.01	0.00413184091303385\\
436.01	0.00413884346409908\\
437.01	0.00414601416519249\\
438.01	0.00415335815041885\\
439.01	0.00416088080320058\\
440.01	0.00416858777610178\\
441.01	0.00417648501281281\\
442.01	0.00418457877256984\\
443.01	0.00419287565732265\\
444.01	0.00420138264199804\\
445.01	0.00421010710825831\\
446.01	0.00421905688219473\\
447.01	0.00422824027645134\\
448.01	0.00423766613732213\\
449.01	0.00424734389741394\\
450.01	0.00425728363451118\\
451.01	0.00426749613731328\\
452.01	0.00427799297872741\\
453.01	0.00428878659738625\\
454.01	0.00429989038799372\\
455.01	0.00431131880096315\\
456.01	0.00432308745156734\\
457.01	0.00433521323841857\\
458.01	0.00434771447046317\\
459.01	0.00436061100073\\
460.01	0.00437392436366178\\
461.01	0.0043876779108203\\
462.01	0.00440189693681349\\
463.01	0.00441660878311331\\
464.01	0.00443184290152039\\
465.01	0.00444763085071449\\
466.01	0.00446400618770507\\
467.01	0.00448100419977486\\
468.01	0.00449866139996727\\
469.01	0.00451701467790344\\
470.01	0.00453609995443392\\
471.01	0.0045559501288019\\
472.01	0.00457659202439141\\
473.01	0.00459804192517578\\
474.01	0.00462029913787318\\
475.01	0.00464332508813105\\
476.01	0.00466689683504538\\
477.01	0.00469094932399357\\
478.01	0.00471548675428907\\
479.01	0.00474051214457602\\
480.01	0.00476602707176335\\
481.01	0.00479203136697413\\
482.01	0.00481852276470723\\
483.01	0.00484549650011952\\
484.01	0.00487294484167969\\
485.01	0.0049008565537592\\
486.01	0.00492921629070482\\
487.01	0.00495800392499192\\
488.01	0.0049871938186043\\
489.01	0.00501675405767654\\
490.01	0.00504664568720932\\
491.01	0.0050768220080112\\
492.01	0.00510722803578761\\
493.01	0.00513780027796394\\
494.01	0.0051684670553882\\
495.01	0.0051991495838186\\
496.01	0.00522976409131098\\
497.01	0.00526022558976409\\
498.01	0.00529045412589123\\
499.01	0.00532038464015382\\
500.01	0.00534998202242072\\
501.01	0.00537937490757551\\
502.01	0.00540890750955372\\
503.01	0.00543856765434311\\
504.01	0.00546831797970659\\
505.01	0.00549811956810894\\
506.01	0.00552793260853647\\
507.01	0.0055577172825164\\
508.01	0.00558743491008898\\
509.01	0.0056170493872629\\
510.01	0.00564652893542456\\
511.01	0.00567584816120002\\
512.01	0.00570499038638812\\
513.01	0.00573395014301281\\
514.01	0.00576273562554443\\
515.01	0.00579137073254395\\
516.01	0.00581989608708264\\
517.01	0.00584836806176718\\
518.01	0.00587685429683296\\
519.01	0.00590541214361032\\
520.01	0.00593404905157574\\
521.01	0.00596276379842327\\
522.01	0.00599155949860458\\
523.01	0.00602044410378867\\
524.01	0.00604943077707996\\
525.01	0.00607853806625773\\
526.01	0.00610778978314085\\
527.01	0.00613721448690835\\
528.01	0.00616684448893449\\
529.01	0.00619671432899642\\
530.01	0.00622685873489252\\
531.01	0.00625731020827809\\
532.01	0.00628809661523301\\
533.01	0.0063192399069386\\
534.01	0.00635075976253935\\
535.01	0.00638267741778237\\
536.01	0.00641501570232442\\
537.01	0.00644779862787694\\
538.01	0.00648105089573116\\
539.01	0.00651479735401082\\
540.01	0.00654906245383268\\
541.01	0.00658386977391781\\
542.01	0.0066192417007872\\
543.01	0.00665519935786468\\
544.01	0.00669176285587489\\
545.01	0.00672895184647044\\
546.01	0.00676678598891822\\
547.01	0.00680528492509266\\
548.01	0.00684446810936269\\
549.01	0.00688435466246006\\
550.01	0.00692496326702412\\
551.01	0.00696631211540255\\
552.01	0.0070084189134092\\
553.01	0.00705130093263077\\
554.01	0.00709497508902708\\
555.01	0.00713945801015897\\
556.01	0.00718476604845364\\
557.01	0.00723091523913168\\
558.01	0.0072779212495796\\
559.01	0.00732579933887666\\
560.01	0.00737456432605899\\
561.01	0.00742423056265521\\
562.01	0.00747481190325003\\
563.01	0.00752632166704584\\
564.01	0.00757877258459237\\
565.01	0.00763217672773581\\
566.01	0.007686545426493\\
567.01	0.00774188917822005\\
568.01	0.00779821754982424\\
569.01	0.00785553907106167\\
570.01	0.00791386111679867\\
571.01	0.00797318977653028\\
572.01	0.00803352971022384\\
573.01	0.00809488399052311\\
574.01	0.0081572539321533\\
575.01	0.00822063890954425\\
576.01	0.00828503616326107\\
577.01	0.00835044059570835\\
578.01	0.00841684455706406\\
579.01	0.00848423762317128\\
580.01	0.00855260636803562\\
581.01	0.00862193413461367\\
582.01	0.0086922008087244\\
583.01	0.00876338260223467\\
584.01	0.0088354518533399\\
585.01	0.00890837685398345\\
586.01	0.00898212171733074\\
587.01	0.00905664630181658\\
588.01	0.00913190621277292\\
589.01	0.009207852908237\\
590.01	0.00928443394251917\\
591.01	0.00936159338983446\\
592.01	0.00943927250119964\\
593.01	0.00951741066136708\\
594.01	0.00959594672938032\\
595.01	0.00967482086709944\\
596.01	0.0097539769856271\\
597.01	0.00983333321553398\\
598.01	0.00990826330299361\\
599.01	0.00997053306357289\\
599.02	0.00997104257304143\\
599.03	0.00997154904691472\\
599.04	0.00997205245565375\\
599.05	0.00997255276942872\\
599.06	0.00997304995811617\\
599.07	0.00997354399129606\\
599.08	0.00997403483824883\\
599.09	0.00997452246795248\\
599.1	0.00997500684907951\\
599.11	0.00997548794999398\\
599.12	0.00997596573874838\\
599.13	0.0099764401830806\\
599.14	0.0099769112504108\\
599.15	0.00997737890783826\\
599.16	0.00997784312213823\\
599.17	0.0099783038597587\\
599.18	0.00997876108681718\\
599.19	0.00997921476909742\\
599.2	0.00997966487204611\\
599.21	0.00998011136076955\\
599.22	0.00998055420003028\\
599.23	0.00998099335424369\\
599.24	0.00998142878747458\\
599.25	0.00998186046343367\\
599.26	0.00998228834359041\\
599.27	0.00998271238778643\\
599.28	0.00998313255546381\\
599.29	0.00998354880566111\\
599.3	0.00998396109700947\\
599.31	0.00998436938772849\\
599.32	0.00998477363562221\\
599.33	0.00998517379807499\\
599.34	0.00998556983204737\\
599.35	0.00998596169407188\\
599.36	0.00998634934024881\\
599.37	0.00998673272624192\\
599.38	0.00998711180727415\\
599.39	0.00998748653812326\\
599.4	0.00998785687311743\\
599.41	0.00998822276613081\\
599.42	0.00998858417057903\\
599.43	0.00998894103941468\\
599.44	0.00998929332512275\\
599.45	0.00998964097971596\\
599.46	0.00998998395473014\\
599.47	0.00999032220121951\\
599.48	0.0099906556697519\\
599.49	0.00999098431040396\\
599.5	0.00999130807275631\\
599.51	0.00999162690588862\\
599.52	0.00999194075837471\\
599.53	0.00999224957827746\\
599.54	0.00999255331314386\\
599.55	0.00999285190999987\\
599.56	0.00999314531534524\\
599.57	0.00999343347514839\\
599.58	0.00999371633484107\\
599.59	0.00999399383931314\\
599.6	0.00999426593290715\\
599.61	0.009994532559413\\
599.62	0.0099947936620624\\
599.63	0.00999504918352342\\
599.64	0.00999529906589491\\
599.65	0.00999554325070086\\
599.66	0.00999578167888471\\
599.67	0.00999601429080366\\
599.68	0.00999624102622284\\
599.69	0.00999646182430947\\
599.7	0.00999667662362697\\
599.71	0.00999688536212897\\
599.72	0.00999708797715331\\
599.73	0.00999728440541595\\
599.74	0.00999747458300482\\
599.75	0.0099976584453736\\
599.76	0.00999783592733551\\
599.77	0.00999800696305692\\
599.78	0.00999817148605102\\
599.79	0.00999832942917133\\
599.8	0.00999848072460517\\
599.81	0.00999862530386713\\
599.82	0.00999876309779239\\
599.83	0.00999889403653003\\
599.84	0.00999901804953622\\
599.85	0.00999913506556741\\
599.86	0.00999924501267341\\
599.87	0.00999934781819042\\
599.88	0.00999944340873394\\
599.89	0.0099995317101917\\
599.9	0.00999961264771648\\
599.91	0.00999968614571878\\
599.92	0.00999975212785957\\
599.93	0.00999981051704285\\
599.94	0.00999986123540817\\
599.95	0.00999990420432308\\
599.96	0.00999993934437554\\
599.97	0.00999996657536618\\
599.98	0.00999998581630055\\
599.99	0.00999999698538124\\
600	0.01\\
};
\addplot [color=red!75!mycolor17,solid,forget plot]
  table[row sep=crcr]{%
0.01	0.00396769113260109\\
1.01	0.00396769177020938\\
2.01	0.00396769242137629\\
3.01	0.00396769308639122\\
4.01	0.00396769376554953\\
5.01	0.00396769445915275\\
6.01	0.00396769516750901\\
7.01	0.00396769589093288\\
8.01	0.0039676966297459\\
9.01	0.00396769738427642\\
10.01	0.0039676981548598\\
11.01	0.0039676989418386\\
12.01	0.00396769974556255\\
13.01	0.00396770056638894\\
14.01	0.00396770140468322\\
15.01	0.00396770226081775\\
16.01	0.0039677031351734\\
17.01	0.00396770402813927\\
18.01	0.00396770494011262\\
19.01	0.00396770587149925\\
20.01	0.00396770682271361\\
21.01	0.00396770779417913\\
22.01	0.00396770878632824\\
23.01	0.00396770979960269\\
24.01	0.00396771083445369\\
25.01	0.00396771189134193\\
26.01	0.00396771297073839\\
27.01	0.0039677140731237\\
28.01	0.00396771519898899\\
29.01	0.00396771634883605\\
30.01	0.00396771752317712\\
31.01	0.00396771872253589\\
32.01	0.0039677199474468\\
33.01	0.0039677211984559\\
34.01	0.00396772247612104\\
35.01	0.00396772378101196\\
36.01	0.00396772511371082\\
37.01	0.00396772647481196\\
38.01	0.00396772786492264\\
39.01	0.00396772928466314\\
40.01	0.00396773073466704\\
41.01	0.0039677322155816\\
42.01	0.00396773372806764\\
43.01	0.0039677352728007\\
44.01	0.00396773685047039\\
45.01	0.00396773846178129\\
46.01	0.00396774010745304\\
47.01	0.0039677417882208\\
48.01	0.0039677435048356\\
49.01	0.00396774525806447\\
50.01	0.00396774704869077\\
51.01	0.00396774887751485\\
52.01	0.00396775074535444\\
53.01	0.00396775265304471\\
54.01	0.00396775460143854\\
55.01	0.00396775659140745\\
56.01	0.00396775862384134\\
57.01	0.0039677606996496\\
58.01	0.00396776281976091\\
59.01	0.00396776498512429\\
60.01	0.00396776719670855\\
61.01	0.00396776945550387\\
62.01	0.0039677717625213\\
63.01	0.00396777411879395\\
64.01	0.00396777652537718\\
65.01	0.00396777898334862\\
66.01	0.00396778149380962\\
67.01	0.00396778405788478\\
68.01	0.00396778667672308\\
69.01	0.0039677893514983\\
70.01	0.00396779208340937\\
71.01	0.00396779487368093\\
72.01	0.00396779772356412\\
73.01	0.00396780063433684\\
74.01	0.00396780360730464\\
75.01	0.00396780664380129\\
76.01	0.00396780974518895\\
77.01	0.00396781291285943\\
78.01	0.00396781614823408\\
79.01	0.00396781945276524\\
80.01	0.00396782282793649\\
81.01	0.00396782627526315\\
82.01	0.00396782979629334\\
83.01	0.00396783339260852\\
84.01	0.00396783706582401\\
85.01	0.00396784081759033\\
86.01	0.00396784464959313\\
87.01	0.00396784856355457\\
88.01	0.00396785256123405\\
89.01	0.00396785664442856\\
90.01	0.00396786081497409\\
91.01	0.00396786507474611\\
92.01	0.00396786942566049\\
93.01	0.00396787386967452\\
94.01	0.00396787840878758\\
95.01	0.0039678830450421\\
96.01	0.00396788778052475\\
97.01	0.00396789261736694\\
98.01	0.00396789755774617\\
99.01	0.00396790260388696\\
100.01	0.00396790775806193\\
101.01	0.00396791302259243\\
102.01	0.00396791839984992\\
103.01	0.00396792389225731\\
104.01	0.00396792950228941\\
105.01	0.00396793523247471\\
106.01	0.00396794108539625\\
107.01	0.00396794706369274\\
108.01	0.00396795317005978\\
109.01	0.00396795940725127\\
110.01	0.00396796577808052\\
111.01	0.00396797228542163\\
112.01	0.00396797893221054\\
113.01	0.00396798572144694\\
114.01	0.00396799265619489\\
115.01	0.00396799973958503\\
116.01	0.00396800697481547\\
117.01	0.00396801436515318\\
118.01	0.00396802191393611\\
119.01	0.00396802962457379\\
120.01	0.00396803750055007\\
121.01	0.00396804554542347\\
122.01	0.00396805376282972\\
123.01	0.0039680621564831\\
124.01	0.00396807073017803\\
125.01	0.00396807948779108\\
126.01	0.00396808843328268\\
127.01	0.00396809757069876\\
128.01	0.00396810690417269\\
129.01	0.00396811643792719\\
130.01	0.00396812617627647\\
131.01	0.00396813612362767\\
132.01	0.00396814628448382\\
133.01	0.00396815666344473\\
134.01	0.0039681672652101\\
135.01	0.00396817809458115\\
136.01	0.00396818915646294\\
137.01	0.0039682004558667\\
138.01	0.00396821199791209\\
139.01	0.00396822378782956\\
140.01	0.00396823583096281\\
141.01	0.00396824813277083\\
142.01	0.00396826069883113\\
143.01	0.0039682735348419\\
144.01	0.0039682866466246\\
145.01	0.00396830004012673\\
146.01	0.00396831372142474\\
147.01	0.00396832769672614\\
148.01	0.00396834197237329\\
149.01	0.00396835655484569\\
150.01	0.00396837145076337\\
151.01	0.00396838666688958\\
152.01	0.00396840221013409\\
153.01	0.00396841808755638\\
154.01	0.00396843430636869\\
155.01	0.00396845087393959\\
156.01	0.00396846779779748\\
157.01	0.00396848508563376\\
158.01	0.00396850274530658\\
159.01	0.00396852078484405\\
160.01	0.00396853921244861\\
161.01	0.00396855803650049\\
162.01	0.00396857726556138\\
163.01	0.00396859690837859\\
164.01	0.00396861697388921\\
165.01	0.00396863747122393\\
166.01	0.0039686584097117\\
167.01	0.00396867979888334\\
168.01	0.00396870164847651\\
169.01	0.00396872396844029\\
170.01	0.00396874676893914\\
171.01	0.00396877006035841\\
172.01	0.00396879385330839\\
173.01	0.00396881815862969\\
174.01	0.0039688429873983\\
175.01	0.00396886835092998\\
176.01	0.0039688942607866\\
177.01	0.00396892072878072\\
178.01	0.00396894776698106\\
179.01	0.00396897538771862\\
180.01	0.00396900360359177\\
181.01	0.00396903242747235\\
182.01	0.00396906187251186\\
183.01	0.00396909195214723\\
184.01	0.00396912268010698\\
185.01	0.0039691540704178\\
186.01	0.00396918613741108\\
187.01	0.00396921889572945\\
188.01	0.00396925236033333\\
189.01	0.00396928654650831\\
190.01	0.00396932146987196\\
191.01	0.00396935714638096\\
192.01	0.00396939359233903\\
193.01	0.00396943082440386\\
194.01	0.00396946885959507\\
195.01	0.00396950771530226\\
196.01	0.00396954740929303\\
197.01	0.00396958795972087\\
198.01	0.00396962938513432\\
199.01	0.00396967170448466\\
200.01	0.00396971493713552\\
201.01	0.00396975910287152\\
202.01	0.00396980422190737\\
203.01	0.00396985031489741\\
204.01	0.00396989740294529\\
205.01	0.00396994550761362\\
206.01	0.00396999465093401\\
207.01	0.00397004485541752\\
208.01	0.00397009614406499\\
209.01	0.00397014854037767\\
210.01	0.00397020206836825\\
211.01	0.0039702567525722\\
212.01	0.00397031261805877\\
213.01	0.0039703696904434\\
214.01	0.00397042799589886\\
215.01	0.00397048756116784\\
216.01	0.00397054841357571\\
217.01	0.00397061058104263\\
218.01	0.00397067409209725\\
219.01	0.00397073897588941\\
220.01	0.00397080526220421\\
221.01	0.003970872981476\\
222.01	0.00397094216480219\\
223.01	0.00397101284395801\\
224.01	0.00397108505141175\\
225.01	0.00397115882033909\\
226.01	0.00397123418463971\\
227.01	0.00397131117895225\\
228.01	0.00397138983867107\\
229.01	0.0039714701999629\\
230.01	0.00397155229978335\\
231.01	0.00397163617589495\\
232.01	0.00397172186688433\\
233.01	0.00397180941218067\\
234.01	0.00397189885207395\\
235.01	0.00397199022773438\\
236.01	0.00397208358123138\\
237.01	0.00397217895555359\\
238.01	0.00397227639462915\\
239.01	0.00397237594334665\\
240.01	0.0039724776475758\\
241.01	0.00397258155418947\\
242.01	0.00397268771108618\\
243.01	0.00397279616721204\\
244.01	0.00397290697258462\\
245.01	0.00397302017831611\\
246.01	0.00397313583663846\\
247.01	0.00397325400092682\\
248.01	0.00397337472572634\\
249.01	0.00397349806677728\\
250.01	0.00397362408104181\\
251.01	0.00397375282673136\\
252.01	0.00397388436333408\\
253.01	0.00397401875164358\\
254.01	0.00397415605378797\\
255.01	0.00397429633325934\\
256.01	0.00397443965494465\\
257.01	0.00397458608515648\\
258.01	0.00397473569166543\\
259.01	0.00397488854373228\\
260.01	0.00397504471214175\\
261.01	0.00397520426923631\\
262.01	0.00397536728895163\\
263.01	0.00397553384685206\\
264.01	0.00397570402016739\\
265.01	0.00397587788783033\\
266.01	0.00397605553051503\\
267.01	0.00397623703067639\\
268.01	0.00397642247259025\\
269.01	0.00397661194239492\\
270.01	0.00397680552813294\\
271.01	0.00397700331979466\\
272.01	0.00397720540936261\\
273.01	0.00397741189085668\\
274.01	0.00397762286038064\\
275.01	0.00397783841616962\\
276.01	0.0039780586586393\\
277.01	0.0039782836904354\\
278.01	0.00397851361648541\\
279.01	0.0039787485440506\\
280.01	0.0039789885827801\\
281.01	0.00397923384476608\\
282.01	0.00397948444459989\\
283.01	0.00397974049943065\\
284.01	0.00398000212902407\\
285.01	0.00398026945582379\\
286.01	0.00398054260501357\\
287.01	0.00398082170458154\\
288.01	0.00398110688538615\\
289.01	0.00398139828122336\\
290.01	0.00398169602889592\\
291.01	0.00398200026828435\\
292.01	0.00398231114242037\\
293.01	0.00398262879756076\\
294.01	0.00398295338326518\\
295.01	0.00398328505247446\\
296.01	0.00398362396159208\\
297.01	0.00398397027056686\\
298.01	0.00398432414297918\\
299.01	0.00398468574612829\\
300.01	0.00398505525112286\\
301.01	0.00398543283297384\\
302.01	0.00398581867068952\\
303.01	0.0039862129473742\\
304.01	0.00398661585032878\\
305.01	0.00398702757115467\\
306.01	0.00398744830586066\\
307.01	0.00398787825497264\\
308.01	0.00398831762364706\\
309.01	0.00398876662178717\\
310.01	0.00398922546416321\\
311.01	0.00398969437053592\\
312.01	0.00399017356578395\\
313.01	0.00399066328003468\\
314.01	0.00399116374880031\\
315.01	0.00399167521311687\\
316.01	0.00399219791968798\\
317.01	0.00399273212103379\\
318.01	0.00399327807564361\\
319.01	0.00399383604813475\\
320.01	0.00399440630941532\\
321.01	0.00399498913685312\\
322.01	0.00399558481445038\\
323.01	0.0039961936330233\\
324.01	0.00399681589038901\\
325.01	0.00399745189155805\\
326.01	0.00399810194893385\\
327.01	0.00399876638251954\\
328.01	0.00399944552013129\\
329.01	0.00400013969762028\\
330.01	0.00400084925910212\\
331.01	0.00400157455719517\\
332.01	0.00400231595326683\\
333.01	0.00400307381769054\\
334.01	0.00400384853011121\\
335.01	0.00400464047972214\\
336.01	0.00400545006555145\\
337.01	0.00400627769676144\\
338.01	0.00400712379295835\\
339.01	0.00400798878451636\\
340.01	0.00400887311291334\\
341.01	0.00400977723108244\\
342.01	0.00401070160377703\\
343.01	0.00401164670795204\\
344.01	0.00401261303316232\\
345.01	0.00401360108197817\\
346.01	0.0040146113704197\\
347.01	0.00401564442841108\\
348.01	0.00401670080025608\\
349.01	0.00401778104513627\\
350.01	0.00401888573763207\\
351.01	0.00402001546827048\\
352.01	0.00402117084409927\\
353.01	0.00402235248929085\\
354.01	0.00402356104577592\\
355.01	0.0040247971739118\\
356.01	0.00402606155318498\\
357.01	0.00402735488295299\\
358.01	0.00402867788322644\\
359.01	0.00403003129549559\\
360.01	0.00403141588360482\\
361.01	0.00403283243467717\\
362.01	0.00403428176009619\\
363.01	0.00403576469654615\\
364.01	0.00403728210711875\\
365.01	0.00403883488248996\\
366.01	0.00404042394217372\\
367.01	0.00404205023585984\\
368.01	0.00404371474484419\\
369.01	0.00404541848355822\\
370.01	0.0040471625012093\\
371.01	0.00404894788354172\\
372.01	0.00405077575473087\\
373.01	0.00405264727942405\\
374.01	0.00405456366494364\\
375.01	0.00405652616367\\
376.01	0.00405853607562413\\
377.01	0.00406059475127134\\
378.01	0.00406270359457264\\
379.01	0.00406486406631171\\
380.01	0.00406707768772983\\
381.01	0.00406934604450744\\
382.01	0.00407167079113238\\
383.01	0.00407405365570443\\
384.01	0.00407649644523024\\
385.01	0.00407900105147033\\
386.01	0.0040815694574101\\
387.01	0.00408420374443365\\
388.01	0.00408690610029082\\
389.01	0.00408967882796181\\
390.01	0.00409252435553178\\
391.01	0.0040954452472062\\
392.01	0.0040984442156084\\
393.01	0.00410152413551731\\
394.01	0.00410468805921486\\
395.01	0.00410793923362744\\
396.01	0.00411128111944832\\
397.01	0.00411471741243497\\
398.01	0.00411825206706065\\
399.01	0.00412188932267647\\
400.01	0.00412563373228683\\
401.01	0.00412949019395649\\
402.01	0.00413346398472439\\
403.01	0.00413756079668556\\
404.01	0.0041417867745759\\
405.01	0.00414614855372214\\
406.01	0.00415065329652599\\
407.01	0.00415530872466337\\
408.01	0.00416012314277424\\
409.01	0.00416510544743282\\
410.01	0.00417026511239615\\
411.01	0.00417561213723204\\
412.01	0.00418115694098542\\
413.01	0.00418691017498596\\
414.01	0.00419288241841614\\
415.01	0.00419908370574985\\
416.01	0.00420552281511527\\
417.01	0.0042122062189507\\
418.01	0.00421913656014379\\
419.01	0.00422631046422273\\
420.01	0.00423368312743107\\
421.01	0.00424121925948537\\
422.01	0.00424892264628497\\
423.01	0.00425679716728809\\
424.01	0.00426484679749374\\
425.01	0.0042730756093373\\
426.01	0.00428148777446545\\
427.01	0.00429008756535393\\
428.01	0.00429887935672305\\
429.01	0.00430786762669753\\
430.01	0.00431705695764628\\
431.01	0.00432645203662765\\
432.01	0.0043360576553494\\
433.01	0.00434587870953865\\
434.01	0.00435592019759466\\
435.01	0.00436618721837802\\
436.01	0.00437668496796002\\
437.01	0.00438741873512619\\
438.01	0.00439839389539268\\
439.01	0.00440961590324765\\
440.01	0.00442109028228488\\
441.01	0.00443282261283391\\
442.01	0.0044448185166292\\
443.01	0.00445708363797592\\
444.01	0.00446962362078902\\
445.01	0.0044824440807707\\
446.01	0.00449555057188222\\
447.01	0.00450894854612764\\
448.01	0.00452264330552268\\
449.01	0.00453663994495723\\
450.01	0.00455094328448361\\
451.01	0.00456555778937208\\
452.01	0.004580487476087\\
453.01	0.00459573580214639\\
454.01	0.00461130553766607\\
455.01	0.00462719861627069\\
456.01	0.00464341596302448\\
457.01	0.00465995729713684\\
458.01	0.00467682090752762\\
459.01	0.00469400339999125\\
460.01	0.0047114994158568\\
461.01	0.00472930132390904\\
462.01	0.00474739889026811\\
463.01	0.00476577893533903\\
464.01	0.00478442499349607\\
465.01	0.00480331700073351\\
466.01	0.00482243104930566\\
467.01	0.00484173926811322\\
468.01	0.00486120991560975\\
469.01	0.00488080781156211\\
470.01	0.004900495289608\\
471.01	0.00492023393042901\\
472.01	0.0049399874440854\\
473.01	0.00495972622148666\\
474.01	0.00497943428541837\\
475.01	0.00499913145132538\\
476.01	0.00501902720808554\\
477.01	0.00503918180239209\\
478.01	0.00505958481623904\\
479.01	0.00508022431726177\\
480.01	0.00510108678874226\\
481.01	0.00512215707757294\\
482.01	0.00514341836753322\\
483.01	0.00516485218691057\\
484.01	0.00518643846156579\\
485.01	0.00520815562694403\\
486.01	0.00522998081483872\\
487.01	0.00525189013303434\\
488.01	0.00527385905804288\\
489.01	0.00529586296252547\\
490.01	0.00531787779894417\\
491.01	0.00533988095841141\\
492.01	0.00536185231690581\\
493.01	0.0053837754674602\\
494.01	0.00540563911290601\\
495.01	0.00542743855628496\\
496.01	0.00544917717266305\\
497.01	0.00547086765803982\\
498.01	0.00549253271345604\\
499.01	0.00551420461998016\\
500.01	0.00553592286251936\\
501.01	0.00555772596865946\\
502.01	0.00557962160141781\\
503.01	0.00560160101697943\\
504.01	0.00562365677908444\\
505.01	0.00564578336202322\\
506.01	0.0056679775498117\\
507.01	0.00569023882959235\\
508.01	0.00571256975497018\\
509.01	0.00573497624721442\\
510.01	0.005757467793804\\
511.01	0.00578005749547837\\
512.01	0.00580276190630355\\
513.01	0.00582560060892912\\
514.01	0.00584859547358037\\
515.01	0.00587176957142045\\
516.01	0.00589514576173578\\
517.01	0.00591874506485783\\
518.01	0.00594258509454382\\
519.01	0.00596667933551578\\
520.01	0.00599103986781382\\
521.01	0.00601568016865678\\
522.01	0.00604061526373739\\
523.01	0.00606586156842957\\
524.01	0.00609143666198887\\
525.01	0.00611735899771229\\
526.01	0.00614364756017159\\
527.01	0.00617032149155369\\
528.01	0.0061973997219789\\
529.01	0.00622490065177947\\
530.01	0.00625284194488335\\
531.01	0.00628124049607781\\
532.01	0.006310112620824\\
533.01	0.00633947446115379\\
534.01	0.00636934236912462\\
535.01	0.00639973293020422\\
536.01	0.00643066283958986\\
537.01	0.00646214878170193\\
538.01	0.00649420732798515\\
539.01	0.00652685486259719\\
540.01	0.00656010754382047\\
541.01	0.00659398130508659\\
542.01	0.006628491892921\\
543.01	0.00666365492999969\\
544.01	0.00669948598115028\\
545.01	0.00673600059244936\\
546.01	0.00677321428426587\\
547.01	0.00681114252189872\\
548.01	0.00684980069166331\\
549.01	0.00688920408657994\\
550.01	0.00692936790038687\\
551.01	0.00697030722712198\\
552.01	0.00701203706224635\\
553.01	0.0070545723005976\\
554.01	0.00709792772687743\\
555.01	0.00714211799637007\\
556.01	0.0071871576071451\\
557.01	0.00723306086787254\\
558.01	0.00727984186354769\\
559.01	0.00732751441838491\\
560.01	0.00737609205440055\\
561.01	0.00742558794421976\\
562.01	0.00747601485687485\\
563.01	0.00752738509579055\\
564.01	0.00757971042866334\\
565.01	0.00763300200932895\\
566.01	0.00768727029169022\\
567.01	0.00774252493532158\\
568.01	0.0077987747020062\\
569.01	0.00785602734245903\\
570.01	0.0079142894726261\\
571.01	0.0079735664391176\\
572.01	0.00803386217349965\\
573.01	0.00809517903531001\\
574.01	0.00815751764376637\\
575.01	0.00822087669824639\\
576.01	0.00828525278781392\\
577.01	0.00835064019039778\\
578.01	0.00841703066268948\\
579.01	0.008484413222397\\
580.01	0.00855277392521094\\
581.01	0.00862209563973181\\
582.01	0.00869235782474246\\
583.01	0.00876353631465262\\
584.01	0.00883560312078353\\
585.01	0.00890852625849137\\
586.01	0.00898226961305655\\
587.01	0.00905679286092972\\
588.01	0.00913205146750303\\
589.01	0.00920799678828045\\
590.01	0.00928457630742256\\
591.01	0.00936173405646149\\
592.01	0.00943941126691484\\
593.01	0.00951754732405423\\
594.01	0.00959608110579566\\
595.01	0.00967495281130476\\
596.01	0.00975410640935049\\
597.01	0.0098334186165282\\
598.01	0.00990826330299362\\
599.01	0.00997053306357288\\
599.02	0.00997104257304143\\
599.03	0.00997154904691472\\
599.04	0.00997205245565375\\
599.05	0.00997255276942872\\
599.06	0.00997304995811617\\
599.07	0.00997354399129606\\
599.08	0.00997403483824883\\
599.09	0.00997452246795248\\
599.1	0.00997500684907951\\
599.11	0.00997548794999398\\
599.12	0.00997596573874838\\
599.13	0.0099764401830806\\
599.14	0.0099769112504108\\
599.15	0.00997737890783826\\
599.16	0.00997784312213823\\
599.17	0.0099783038597587\\
599.18	0.00997876108681718\\
599.19	0.00997921476909742\\
599.2	0.00997966487204611\\
599.21	0.00998011136076955\\
599.22	0.00998055420003028\\
599.23	0.00998099335424369\\
599.24	0.00998142878747458\\
599.25	0.00998186046343367\\
599.26	0.00998228834359041\\
599.27	0.00998271238778643\\
599.28	0.00998313255546381\\
599.29	0.00998354880566111\\
599.3	0.00998396109700947\\
599.31	0.00998436938772849\\
599.32	0.00998477363562221\\
599.33	0.00998517379807499\\
599.34	0.00998556983204737\\
599.35	0.00998596169407188\\
599.36	0.00998634934024881\\
599.37	0.00998673272624191\\
599.38	0.00998711180727415\\
599.39	0.00998748653812326\\
599.4	0.00998785687311743\\
599.41	0.00998822276613081\\
599.42	0.00998858417057903\\
599.43	0.00998894103941468\\
599.44	0.00998929332512275\\
599.45	0.00998964097971596\\
599.46	0.00998998395473014\\
599.47	0.00999032220121951\\
599.48	0.0099906556697519\\
599.49	0.00999098431040396\\
599.5	0.00999130807275631\\
599.51	0.00999162690588863\\
599.52	0.00999194075837471\\
599.53	0.00999224957827746\\
599.54	0.00999255331314387\\
599.55	0.00999285190999987\\
599.56	0.00999314531534524\\
599.57	0.00999343347514839\\
599.58	0.00999371633484107\\
599.59	0.00999399383931314\\
599.6	0.00999426593290715\\
599.61	0.009994532559413\\
599.62	0.0099947936620624\\
599.63	0.00999504918352342\\
599.64	0.00999529906589491\\
599.65	0.00999554325070086\\
599.66	0.00999578167888471\\
599.67	0.00999601429080366\\
599.68	0.00999624102622283\\
599.69	0.00999646182430947\\
599.7	0.00999667662362697\\
599.71	0.00999688536212897\\
599.72	0.00999708797715332\\
599.73	0.00999728440541596\\
599.74	0.00999747458300482\\
599.75	0.0099976584453736\\
599.76	0.0099978359273355\\
599.77	0.00999800696305692\\
599.78	0.00999817148605102\\
599.79	0.00999832942917133\\
599.8	0.00999848072460517\\
599.81	0.00999862530386713\\
599.82	0.00999876309779239\\
599.83	0.00999889403653003\\
599.84	0.00999901804953622\\
599.85	0.00999913506556741\\
599.86	0.00999924501267341\\
599.87	0.00999934781819042\\
599.88	0.00999944340873394\\
599.89	0.0099995317101917\\
599.9	0.00999961264771648\\
599.91	0.00999968614571878\\
599.92	0.00999975212785957\\
599.93	0.00999981051704285\\
599.94	0.00999986123540817\\
599.95	0.00999990420432308\\
599.96	0.00999993934437554\\
599.97	0.00999996657536618\\
599.98	0.00999998581630055\\
599.99	0.00999999698538124\\
600	0.01\\
};
\addplot [color=red!80!mycolor19,solid,forget plot]
  table[row sep=crcr]{%
0.01	0.00420172588690648\\
1.01	0.00420172685462246\\
2.01	0.00420172784292355\\
3.01	0.00420172885224883\\
4.01	0.0042017298830468\\
5.01	0.00420173093577565\\
6.01	0.00420173201090362\\
7.01	0.0042017331089086\\
8.01	0.00420173423027881\\
9.01	0.00420173537551309\\
10.01	0.00420173654512063\\
11.01	0.00420173773962192\\
12.01	0.00420173895954846\\
13.01	0.00420174020544292\\
14.01	0.00420174147785983\\
15.01	0.00420174277736569\\
16.01	0.0042017441045389\\
17.01	0.00420174545997036\\
18.01	0.00420174684426371\\
19.01	0.00420174825803533\\
20.01	0.00420174970191506\\
21.01	0.00420175117654616\\
22.01	0.00420175268258559\\
23.01	0.00420175422070443\\
24.01	0.00420175579158825\\
25.01	0.00420175739593738\\
26.01	0.00420175903446679\\
27.01	0.00420176070790722\\
28.01	0.00420176241700494\\
29.01	0.00420176416252213\\
30.01	0.0042017659452375\\
31.01	0.00420176776594622\\
32.01	0.00420176962546091\\
33.01	0.00420177152461128\\
34.01	0.00420177346424515\\
35.01	0.00420177544522823\\
36.01	0.00420177746844499\\
37.01	0.00420177953479893\\
38.01	0.00420178164521279\\
39.01	0.00420178380062924\\
40.01	0.00420178600201125\\
41.01	0.00420178825034244\\
42.01	0.00420179054662748\\
43.01	0.00420179289189276\\
44.01	0.00420179528718653\\
45.01	0.0042017977335799\\
46.01	0.00420180023216656\\
47.01	0.00420180278406401\\
48.01	0.00420180539041372\\
49.01	0.00420180805238159\\
50.01	0.00420181077115887\\
51.01	0.00420181354796213\\
52.01	0.00420181638403413\\
53.01	0.00420181928064438\\
54.01	0.00420182223908996\\
55.01	0.00420182526069544\\
56.01	0.00420182834681423\\
57.01	0.00420183149882858\\
58.01	0.00420183471815069\\
59.01	0.00420183800622304\\
60.01	0.00420184136451911\\
61.01	0.00420184479454436\\
62.01	0.00420184829783626\\
63.01	0.00420185187596585\\
64.01	0.0042018555305374\\
65.01	0.0042018592631904\\
66.01	0.00420186307559901\\
67.01	0.00420186696947389\\
68.01	0.00420187094656237\\
69.01	0.0042018750086492\\
70.01	0.00420187915755808\\
71.01	0.00420188339515166\\
72.01	0.00420188772333287\\
73.01	0.00420189214404561\\
74.01	0.00420189665927558\\
75.01	0.00420190127105129\\
76.01	0.00420190598144504\\
77.01	0.00420191079257388\\
78.01	0.00420191570660039\\
79.01	0.00420192072573393\\
80.01	0.00420192585223129\\
81.01	0.00420193108839795\\
82.01	0.00420193643658928\\
83.01	0.00420194189921129\\
84.01	0.00420194747872223\\
85.01	0.00420195317763293\\
86.01	0.00420195899850874\\
87.01	0.00420196494397025\\
88.01	0.00420197101669458\\
89.01	0.00420197721941677\\
90.01	0.00420198355493073\\
91.01	0.00420199002609065\\
92.01	0.00420199663581257\\
93.01	0.00420200338707546\\
94.01	0.00420201028292251\\
95.01	0.00420201732646266\\
96.01	0.00420202452087198\\
97.01	0.0042020318693954\\
98.01	0.00420203937534767\\
99.01	0.00420204704211528\\
100.01	0.00420205487315785\\
101.01	0.00420206287200989\\
102.01	0.00420207104228214\\
103.01	0.00420207938766354\\
104.01	0.00420208791192274\\
105.01	0.00420209661890997\\
106.01	0.00420210551255851\\
107.01	0.00420211459688684\\
108.01	0.00420212387600019\\
109.01	0.00420213335409295\\
110.01	0.00420214303544998\\
111.01	0.00420215292444885\\
112.01	0.00420216302556202\\
113.01	0.00420217334335846\\
114.01	0.00420218388250633\\
115.01	0.00420219464777458\\
116.01	0.00420220564403539\\
117.01	0.00420221687626694\\
118.01	0.00420222834955447\\
119.01	0.00420224006909378\\
120.01	0.00420225204019301\\
121.01	0.00420226426827544\\
122.01	0.00420227675888177\\
123.01	0.0042022895176726\\
124.01	0.00420230255043144\\
125.01	0.00420231586306694\\
126.01	0.00420232946161561\\
127.01	0.00420234335224519\\
128.01	0.00420235754125675\\
129.01	0.00420237203508836\\
130.01	0.00420238684031733\\
131.01	0.00420240196366387\\
132.01	0.00420241741199351\\
133.01	0.00420243319232119\\
134.01	0.0042024493118135\\
135.01	0.00420246577779278\\
136.01	0.00420248259774024\\
137.01	0.0042024997792995\\
138.01	0.00420251733027965\\
139.01	0.00420253525865973\\
140.01	0.00420255357259154\\
141.01	0.00420257228040419\\
142.01	0.00420259139060747\\
143.01	0.00420261091189592\\
144.01	0.00420263085315276\\
145.01	0.0042026512234544\\
146.01	0.00420267203207395\\
147.01	0.00420269328848648\\
148.01	0.00420271500237257\\
149.01	0.00420273718362328\\
150.01	0.00420275984234433\\
151.01	0.00420278298886134\\
152.01	0.00420280663372415\\
153.01	0.00420283078771193\\
154.01	0.00420285546183846\\
155.01	0.00420288066735673\\
156.01	0.00420290641576412\\
157.01	0.00420293271880852\\
158.01	0.00420295958849301\\
159.01	0.00420298703708202\\
160.01	0.00420301507710659\\
161.01	0.00420304372137022\\
162.01	0.00420307298295526\\
163.01	0.00420310287522871\\
164.01	0.00420313341184826\\
165.01	0.0042031646067693\\
166.01	0.00420319647425037\\
167.01	0.00420322902886092\\
168.01	0.00420326228548756\\
169.01	0.00420329625934077\\
170.01	0.00420333096596262\\
171.01	0.00420336642123348\\
172.01	0.00420340264137964\\
173.01	0.00420343964298109\\
174.01	0.00420347744297875\\
175.01	0.00420351605868287\\
176.01	0.00420355550778106\\
177.01	0.00420359580834613\\
178.01	0.004203636978845\\
179.01	0.00420367903814729\\
180.01	0.00420372200553398\\
181.01	0.0042037659007062\\
182.01	0.00420381074379494\\
183.01	0.00420385655537051\\
184.01	0.00420390335645147\\
185.01	0.00420395116851523\\
186.01	0.00420400001350772\\
187.01	0.00420404991385355\\
188.01	0.00420410089246709\\
189.01	0.00420415297276246\\
190.01	0.00420420617866502\\
191.01	0.00420426053462265\\
192.01	0.00420431606561683\\
193.01	0.00420437279717452\\
194.01	0.00420443075538043\\
195.01	0.00420448996688871\\
196.01	0.00420455045893599\\
197.01	0.00420461225935385\\
198.01	0.00420467539658205\\
199.01	0.00420473989968179\\
200.01	0.00420480579834943\\
201.01	0.00420487312293046\\
202.01	0.00420494190443378\\
203.01	0.00420501217454659\\
204.01	0.00420508396564852\\
205.01	0.00420515731082764\\
206.01	0.00420523224389555\\
207.01	0.00420530879940367\\
208.01	0.00420538701265945\\
209.01	0.00420546691974286\\
210.01	0.00420554855752335\\
211.01	0.0042056319636775\\
212.01	0.00420571717670665\\
213.01	0.00420580423595522\\
214.01	0.00420589318162891\\
215.01	0.00420598405481427\\
216.01	0.00420607689749763\\
217.01	0.00420617175258493\\
218.01	0.0042062686639225\\
219.01	0.00420636767631748\\
220.01	0.0042064688355589\\
221.01	0.00420657218843963\\
222.01	0.00420667778277824\\
223.01	0.00420678566744215\\
224.01	0.00420689589237011\\
225.01	0.00420700850859657\\
226.01	0.00420712356827535\\
227.01	0.00420724112470472\\
228.01	0.00420736123235277\\
229.01	0.00420748394688271\\
230.01	0.0042076093251803\\
231.01	0.00420773742537993\\
232.01	0.00420786830689271\\
233.01	0.004208002030435\\
234.01	0.00420813865805676\\
235.01	0.00420827825317152\\
236.01	0.00420842088058652\\
237.01	0.00420856660653361\\
238.01	0.0042087154987008\\
239.01	0.00420886762626471\\
240.01	0.00420902305992332\\
241.01	0.00420918187193034\\
242.01	0.00420934413612904\\
243.01	0.00420950992798827\\
244.01	0.00420967932463812\\
245.01	0.00420985240490712\\
246.01	0.00421002924936002\\
247.01	0.00421020994033658\\
248.01	0.00421039456199077\\
249.01	0.00421058320033137\\
250.01	0.00421077594326359\\
251.01	0.00421097288063096\\
252.01	0.00421117410425897\\
253.01	0.0042113797079991\\
254.01	0.00421158978777403\\
255.01	0.00421180444162433\\
256.01	0.00421202376975553\\
257.01	0.00421224787458679\\
258.01	0.00421247686080005\\
259.01	0.00421271083539126\\
260.01	0.00421294990772209\\
261.01	0.00421319418957315\\
262.01	0.00421344379519807\\
263.01	0.00421369884137901\\
264.01	0.00421395944748389\\
265.01	0.00421422573552468\\
266.01	0.00421449783021652\\
267.01	0.00421477585903855\\
268.01	0.00421505995229708\\
269.01	0.00421535024318848\\
270.01	0.00421564686786532\\
271.01	0.00421594996550283\\
272.01	0.00421625967836696\\
273.01	0.00421657615188512\\
274.01	0.00421689953471753\\
275.01	0.00421722997883092\\
276.01	0.00421756763957329\\
277.01	0.00421791267575096\\
278.01	0.00421826524970746\\
279.01	0.00421862552740411\\
280.01	0.0042189936785028\\
281.01	0.00421936987645027\\
282.01	0.00421975429856489\\
283.01	0.00422014712612516\\
284.01	0.00422054854446124\\
285.01	0.00422095874304701\\
286.01	0.00422137791559618\\
287.01	0.00422180626015968\\
288.01	0.00422224397922555\\
289.01	0.00422269127982176\\
290.01	0.00422314837362106\\
291.01	0.00422361547704868\\
292.01	0.00422409281139248\\
293.01	0.00422458060291618\\
294.01	0.00422507908297536\\
295.01	0.0042255884881358\\
296.01	0.00422610906029544\\
297.01	0.00422664104680948\\
298.01	0.00422718470061817\\
299.01	0.00422774028037795\\
300.01	0.0042283080505961\\
301.01	0.00422888828176908\\
302.01	0.00422948125052353\\
303.01	0.00423008723976222\\
304.01	0.00423070653881283\\
305.01	0.0042313394435804\\
306.01	0.00423198625670501\\
307.01	0.00423264728772248\\
308.01	0.00423332285322914\\
309.01	0.00423401327705263\\
310.01	0.00423471889042508\\
311.01	0.00423544003216229\\
312.01	0.00423617704884756\\
313.01	0.00423693029502038\\
314.01	0.00423770013336931\\
315.01	0.00423848693493228\\
316.01	0.00423929107930002\\
317.01	0.00424011295482681\\
318.01	0.00424095295884599\\
319.01	0.00424181149789189\\
320.01	0.00424268898792847\\
321.01	0.00424358585458334\\
322.01	0.00424450253338882\\
323.01	0.00424543947003042\\
324.01	0.00424639712060161\\
325.01	0.00424737595186599\\
326.01	0.00424837644152752\\
327.01	0.00424939907850835\\
328.01	0.00425044436323427\\
329.01	0.00425151280792907\\
330.01	0.0042526049369172\\
331.01	0.00425372128693537\\
332.01	0.00425486240745378\\
333.01	0.00425602886100651\\
334.01	0.00425722122353138\\
335.01	0.00425844008472079\\
336.01	0.00425968604838278\\
337.01	0.00426095973281229\\
338.01	0.00426226177117511\\
339.01	0.00426359281190181\\
340.01	0.00426495351909451\\
341.01	0.00426634457294582\\
342.01	0.00426776667017044\\
343.01	0.00426922052444982\\
344.01	0.00427070686688991\\
345.01	0.00427222644649376\\
346.01	0.00427378003064713\\
347.01	0.00427536840561905\\
348.01	0.00427699237707777\\
349.01	0.00427865277062074\\
350.01	0.00428035043232144\\
351.01	0.00428208622929097\\
352.01	0.00428386105025584\\
353.01	0.00428567580615186\\
354.01	0.00428753143073425\\
355.01	0.0042894288812035\\
356.01	0.00429136913884787\\
357.01	0.0042933532097009\\
358.01	0.00429538212521485\\
359.01	0.00429745694294845\\
360.01	0.00429957874726847\\
361.01	0.00430174865006501\\
362.01	0.00430396779147629\\
363.01	0.00430623734062516\\
364.01	0.00430855849636115\\
365.01	0.00431093248800789\\
366.01	0.00431336057611158\\
367.01	0.00431584405318577\\
368.01	0.00431838424444871\\
369.01	0.0043209825085456\\
370.01	0.00432364023824944\\
371.01	0.00432635886113185\\
372.01	0.00432913984019159\\
373.01	0.00433198467443087\\
374.01	0.00433489489936206\\
375.01	0.00433787208742854\\
376.01	0.00434091784831726\\
377.01	0.00434403382914045\\
378.01	0.00434722171445423\\
379.01	0.00435048322608143\\
380.01	0.0043538201226959\\
381.01	0.00435723419912171\\
382.01	0.00436072728528873\\
383.01	0.00436430124477905\\
384.01	0.00436795797288421\\
385.01	0.00437169939408341\\
386.01	0.00437552745883306\\
387.01	0.00437944413954312\\
388.01	0.00438345142559465\\
389.01	0.0043875513172268\\
390.01	0.00439174581809861\\
391.01	0.00439603692629674\\
392.01	0.0044004266235303\\
393.01	0.00440491686221406\\
394.01	0.0044095095501035\\
395.01	0.00441420653210201\\
396.01	0.00441900956882048\\
397.01	0.00442392031142733\\
398.01	0.0044289402722959\\
399.01	0.00443407079093332\\
400.01	0.00443931299467878\\
401.01	0.00444466775369047\\
402.01	0.00445013562983471\\
403.01	0.00445571681925999\\
404.01	0.00446141108873049\\
405.01	0.00446721770626145\\
406.01	0.00447313536731395\\
407.01	0.00447916211888656\\
408.01	0.00448529528542208\\
409.01	0.00449153140274489\\
410.01	0.00449786616954412\\
411.01	0.00450429443060907\\
412.01	0.00451081021265944\\
413.01	0.00451740684295129\\
414.01	0.00452407719390642\\
415.01	0.00453081411525441\\
416.01	0.00453761114055384\\
417.01	0.00454446359018167\\
418.01	0.00455137024166802\\
419.01	0.00455833580572441\\
420.01	0.00456540750156501\\
421.01	0.00457262625384917\\
422.01	0.00457999472743888\\
423.01	0.00458751559386393\\
424.01	0.00459519152627431\\
425.01	0.00460302519375548\\
426.01	0.0046110192549417\\
427.01	0.00461917635085789\\
428.01	0.00462749909691135\\
429.01	0.00463599007395152\\
430.01	0.00464465181830881\\
431.01	0.00465348681071508\\
432.01	0.00466249746400581\\
433.01	0.00467168610949367\\
434.01	0.0046810549819018\\
435.01	0.0046906062027377\\
436.01	0.00470034176198623\\
437.01	0.00471026349799995\\
438.01	0.00472037307546404\\
439.01	0.00473067196132161\\
440.01	0.00474116139854958\\
441.01	0.00475184237769755\\
442.01	0.00476271560611873\\
443.01	0.00477378147486197\\
444.01	0.00478504002323578\\
445.01	0.00479649090112061\\
446.01	0.00480813332918634\\
447.01	0.00481996605728037\\
448.01	0.00483198732138755\\
449.01	0.00484419479973806\\
450.01	0.00485658556885794\\
451.01	0.00486915606062784\\
452.01	0.00488190202174907\\
453.01	0.00489481847742305\\
454.01	0.00490789970154039\\
455.01	0.00492113919625864\\
456.01	0.0049345296845308\\
457.01	0.00494806311993755\\
458.01	0.00496173071906386\\
459.01	0.00497552302263643\\
460.01	0.00498942999265438\\
461.01	0.00500344115374819\\
462.01	0.00501754578785374\\
463.01	0.00503173319181553\\
464.01	0.00504599300743304\\
465.01	0.00506031563227545\\
466.01	0.00507469271665208\\
467.01	0.00508911774641395\\
468.01	0.0051035867013233\\
469.01	0.00511809876242053\\
470.01	0.00513265701607844\\
471.01	0.00514726906286329\\
472.01	0.00516194737965422\\
473.01	0.00517670919482519\\
474.01	0.00519157550609556\\
475.01	0.00520656856541381\\
476.01	0.00522169966977432\\
477.01	0.00523696353715798\\
478.01	0.00525235315274364\\
479.01	0.00526786132161748\\
480.01	0.00528348076292864\\
481.01	0.00529920422313938\\
482.01	0.00531502460960456\\
483.01	0.00533093514519564\\
484.01	0.00534692954392317\\
485.01	0.0053630022063977\\
486.01	0.00537914843246287\\
487.01	0.00539536464632598\\
488.01	0.00541164862694314\\
489.01	0.00542799973321532\\
490.01	0.00544441910970511\\
491.01	0.00546090985416549\\
492.01	0.00547747712339409\\
493.01	0.00549412814925879\\
494.01	0.00551087213306035\\
495.01	0.00552771998516163\\
496.01	0.00554468388021921\\
497.01	0.00556177661002802\\
498.01	0.00557901074193831\\
499.01	0.00559639763997383\\
500.01	0.00561394649157437\\
501.01	0.00563166366258844\\
502.01	0.00564955376614987\\
503.01	0.00566762202633554\\
504.01	0.00568587478306898\\
505.01	0.00570431951910345\\
506.01	0.00572296485685063\\
507.01	0.00574182052090387\\
508.01	0.00576089726175706\\
509.01	0.00578020673786796\\
510.01	0.00579976135590433\\
511.01	0.0058195740729757\\
512.01	0.00583965817005484\\
513.01	0.00586002701264503\\
514.01	0.00588069382273103\\
515.01	0.00590167149422482\\
516.01	0.00592297249050802\\
517.01	0.00594460886351864\\
518.01	0.00596659242253623\\
519.01	0.00598893504221198\\
520.01	0.00601164894380868\\
521.01	0.00603474672338097\\
522.01	0.00605824127291234\\
523.01	0.00608214569417413\\
524.01	0.00610647321396675\\
525.01	0.00613123710668263\\
526.01	0.00615645063063178\\
527.01	0.00618212698434462\\
528.01	0.00620827928773983\\
529.01	0.00623492059029043\\
530.01	0.00626206390390152\\
531.01	0.00628972225221723\\
532.01	0.00631790872134252\\
533.01	0.00634663649186952\\
534.01	0.00637591883753616\\
535.01	0.00640576910431087\\
536.01	0.00643620069168729\\
537.01	0.00646722704104985\\
538.01	0.00649886163132984\\
539.01	0.0065311179812656\\
540.01	0.00656400965665378\\
541.01	0.0065975502801195\\
542.01	0.00663175354036419\\
543.01	0.0066666331978926\\
544.01	0.00670220308517147\\
545.01	0.00673847710119523\\
546.01	0.00677546920293627\\
547.01	0.00681319339652738\\
548.01	0.00685166372872077\\
549.01	0.00689089427789721\\
550.01	0.0069308991437064\\
551.01	0.00697169243442191\\
552.01	0.00701328825122415\\
553.01	0.00705570066888292\\
554.01	0.00709894371263774\\
555.01	0.0071430313313644\\
556.01	0.007187977367202\\
557.01	0.00723379552157973\\
558.01	0.00728049931723415\\
559.01	0.00732810205567089\\
560.01	0.0073766167695427\\
561.01	0.0074260561694634\\
562.01	0.00747643258482896\\
563.01	0.00752775789824955\\
564.01	0.0075800434731939\\
565.01	0.00763330007440498\\
566.01	0.00768753778057827\\
567.01	0.00774276588874375\\
568.01	0.00779899280979131\\
569.01	0.00785622595460367\\
570.01	0.00791447161029822\\
571.01	0.00797373480612668\\
572.01	0.00803401916864335\\
573.01	0.00809532676583651\\
574.01	0.00815765794003961\\
575.01	0.00822101112961452\\
576.01	0.00828538267964387\\
577.01	0.00835076664220159\\
578.01	0.00841715456719787\\
579.01	0.00848453528535402\\
580.01	0.00855289468557565\\
581.01	0.00862221548991046\\
582.01	0.00869247703044523\\
583.01	0.00876365503397906\\
584.01	0.00883572142218282\\
585.01	0.00890864413730972\\
586.01	0.00898238700647075\\
587.01	0.00905690966117702\\
588.01	0.00913216753344726\\
589.01	0.00920811195549929\\
590.01	0.00928469039714949\\
591.01	0.00936184688385573\\
592.01	0.00943952264925284\\
593.01	0.00951765708953015\\
594.01	0.00959618910368957\\
595.01	0.00967505892433464\\
596.01	0.00975421056908414\\
597.01	0.00983347437625554\\
598.01	0.00990826330299362\\
599.01	0.00997053306357289\\
599.02	0.00997104257304143\\
599.03	0.00997154904691472\\
599.04	0.00997205245565375\\
599.05	0.00997255276942872\\
599.06	0.00997304995811617\\
599.07	0.00997354399129606\\
599.08	0.00997403483824883\\
599.09	0.00997452246795248\\
599.1	0.00997500684907952\\
599.11	0.00997548794999398\\
599.12	0.00997596573874838\\
599.13	0.0099764401830806\\
599.14	0.0099769112504108\\
599.15	0.00997737890783826\\
599.16	0.00997784312213823\\
599.17	0.0099783038597587\\
599.18	0.00997876108681718\\
599.19	0.00997921476909742\\
599.2	0.00997966487204611\\
599.21	0.00998011136076955\\
599.22	0.00998055420003028\\
599.23	0.00998099335424369\\
599.24	0.00998142878747458\\
599.25	0.00998186046343367\\
599.26	0.00998228834359041\\
599.27	0.00998271238778643\\
599.28	0.00998313255546381\\
599.29	0.00998354880566111\\
599.3	0.00998396109700947\\
599.31	0.00998436938772849\\
599.32	0.00998477363562221\\
599.33	0.00998517379807499\\
599.34	0.00998556983204737\\
599.35	0.00998596169407188\\
599.36	0.00998634934024881\\
599.37	0.00998673272624192\\
599.38	0.00998711180727415\\
599.39	0.00998748653812326\\
599.4	0.00998785687311743\\
599.41	0.00998822276613081\\
599.42	0.00998858417057903\\
599.43	0.00998894103941468\\
599.44	0.00998929332512275\\
599.45	0.00998964097971596\\
599.46	0.00998998395473014\\
599.47	0.00999032220121951\\
599.48	0.0099906556697519\\
599.49	0.00999098431040396\\
599.5	0.00999130807275631\\
599.51	0.00999162690588863\\
599.52	0.00999194075837471\\
599.53	0.00999224957827746\\
599.54	0.00999255331314386\\
599.55	0.00999285190999987\\
599.56	0.00999314531534524\\
599.57	0.00999343347514839\\
599.58	0.00999371633484107\\
599.59	0.00999399383931314\\
599.6	0.00999426593290715\\
599.61	0.009994532559413\\
599.62	0.0099947936620624\\
599.63	0.00999504918352342\\
599.64	0.00999529906589492\\
599.65	0.00999554325070086\\
599.66	0.00999578167888471\\
599.67	0.00999601429080366\\
599.68	0.00999624102622283\\
599.69	0.00999646182430947\\
599.7	0.00999667662362697\\
599.71	0.00999688536212897\\
599.72	0.00999708797715332\\
599.73	0.00999728440541595\\
599.74	0.00999747458300482\\
599.75	0.0099976584453736\\
599.76	0.0099978359273355\\
599.77	0.00999800696305692\\
599.78	0.00999817148605102\\
599.79	0.00999832942917133\\
599.8	0.00999848072460517\\
599.81	0.00999862530386713\\
599.82	0.00999876309779239\\
599.83	0.00999889403653003\\
599.84	0.00999901804953622\\
599.85	0.00999913506556741\\
599.86	0.00999924501267341\\
599.87	0.00999934781819042\\
599.88	0.00999944340873394\\
599.89	0.0099995317101917\\
599.9	0.00999961264771648\\
599.91	0.00999968614571878\\
599.92	0.00999975212785958\\
599.93	0.00999981051704285\\
599.94	0.00999986123540817\\
599.95	0.00999990420432308\\
599.96	0.00999993934437554\\
599.97	0.00999996657536618\\
599.98	0.00999998581630055\\
599.99	0.00999999698538124\\
600	0.01\\
};
\addplot [color=red,solid,forget plot]
  table[row sep=crcr]{%
0.01	0.00446279362943701\\
1.01	0.00446279467437111\\
2.01	0.00446279574146974\\
3.01	0.00446279683120417\\
4.01	0.00446279794405589\\
5.01	0.00446279908051643\\
6.01	0.00446280024108764\\
7.01	0.00446280142628253\\
8.01	0.00446280263662477\\
9.01	0.00446280387264912\\
10.01	0.00446280513490203\\
11.01	0.00446280642394131\\
12.01	0.00446280774033689\\
13.01	0.00446280908467081\\
14.01	0.00446281045753742\\
15.01	0.00446281185954392\\
16.01	0.00446281329131025\\
17.01	0.00446281475346984\\
18.01	0.0044628162466694\\
19.01	0.00446281777156965\\
20.01	0.00446281932884526\\
21.01	0.00446282091918541\\
22.01	0.00446282254329389\\
23.01	0.00446282420188951\\
24.01	0.00446282589570661\\
25.01	0.00446282762549499\\
26.01	0.00446282939202075\\
27.01	0.00446283119606608\\
28.01	0.0044628330384299\\
29.01	0.00446283491992829\\
30.01	0.0044628368413948\\
31.01	0.00446283880368069\\
32.01	0.00446284080765535\\
33.01	0.00446284285420691\\
34.01	0.00446284494424239\\
35.01	0.0044628470786883\\
36.01	0.0044628492584908\\
37.01	0.00446285148461635\\
38.01	0.00446285375805211\\
39.01	0.00446285607980614\\
40.01	0.00446285845090834\\
41.01	0.00446286087241034\\
42.01	0.00446286334538661\\
43.01	0.00446286587093418\\
44.01	0.00446286845017398\\
45.01	0.00446287108425042\\
46.01	0.00446287377433295\\
47.01	0.00446287652161562\\
48.01	0.00446287932731806\\
49.01	0.00446288219268617\\
50.01	0.00446288511899234\\
51.01	0.00446288810753637\\
52.01	0.00446289115964578\\
53.01	0.00446289427667639\\
54.01	0.00446289746001305\\
55.01	0.0044629007110704\\
56.01	0.00446290403129351\\
57.01	0.00446290742215796\\
58.01	0.00446291088517126\\
59.01	0.00446291442187279\\
60.01	0.00446291803383553\\
61.01	0.00446292172266559\\
62.01	0.00446292549000384\\
63.01	0.00446292933752594\\
64.01	0.00446293326694371\\
65.01	0.00446293728000552\\
66.01	0.00446294137849734\\
67.01	0.00446294556424327\\
68.01	0.0044629498391065\\
69.01	0.00446295420499028\\
70.01	0.00446295866383831\\
71.01	0.00446296321763646\\
72.01	0.00446296786841259\\
73.01	0.00446297261823851\\
74.01	0.00446297746923029\\
75.01	0.00446298242354926\\
76.01	0.00446298748340318\\
77.01	0.00446299265104712\\
78.01	0.0044629979287845\\
79.01	0.00446300331896818\\
80.01	0.00446300882400139\\
81.01	0.00446301444633918\\
82.01	0.00446302018848894\\
83.01	0.00446302605301209\\
84.01	0.00446303204252478\\
85.01	0.00446303815969951\\
86.01	0.0044630444072659\\
87.01	0.00446305078801243\\
88.01	0.00446305730478727\\
89.01	0.0044630639604996\\
90.01	0.00446307075812129\\
91.01	0.00446307770068795\\
92.01	0.00446308479130011\\
93.01	0.00446309203312531\\
94.01	0.00446309942939881\\
95.01	0.00446310698342536\\
96.01	0.00446311469858085\\
97.01	0.00446312257831347\\
98.01	0.00446313062614562\\
99.01	0.0044631388456753\\
100.01	0.00446314724057793\\
101.01	0.0044631558146077\\
102.01	0.00446316457159972\\
103.01	0.00446317351547124\\
104.01	0.00446318265022398\\
105.01	0.00446319197994535\\
106.01	0.00446320150881091\\
107.01	0.00446321124108578\\
108.01	0.0044632211811271\\
109.01	0.00446323133338516\\
110.01	0.00446324170240645\\
111.01	0.00446325229283491\\
112.01	0.0044632631094144\\
113.01	0.00446327415699101\\
114.01	0.00446328544051491\\
115.01	0.00446329696504284\\
116.01	0.00446330873574025\\
117.01	0.00446332075788357\\
118.01	0.00446333303686298\\
119.01	0.00446334557818465\\
120.01	0.00446335838747333\\
121.01	0.00446337147047461\\
122.01	0.00446338483305785\\
123.01	0.00446339848121874\\
124.01	0.0044634124210821\\
125.01	0.00446342665890456\\
126.01	0.00446344120107753\\
127.01	0.00446345605412986\\
128.01	0.00446347122473133\\
129.01	0.004463486719695\\
130.01	0.00446350254598102\\
131.01	0.00446351871069903\\
132.01	0.00446353522111248\\
133.01	0.00446355208464071\\
134.01	0.00446356930886331\\
135.01	0.00446358690152292\\
136.01	0.00446360487052914\\
137.01	0.00446362322396197\\
138.01	0.0044636419700758\\
139.01	0.00446366111730254\\
140.01	0.00446368067425619\\
141.01	0.00446370064973658\\
142.01	0.00446372105273289\\
143.01	0.00446374189242815\\
144.01	0.00446376317820379\\
145.01	0.00446378491964319\\
146.01	0.00446380712653657\\
147.01	0.00446382980888542\\
148.01	0.00446385297690685\\
149.01	0.00446387664103824\\
150.01	0.00446390081194241\\
151.01	0.00446392550051194\\
152.01	0.00446395071787495\\
153.01	0.00446397647539928\\
154.01	0.00446400278469811\\
155.01	0.00446402965763548\\
156.01	0.00446405710633162\\
157.01	0.00446408514316785\\
158.01	0.00446411378079342\\
159.01	0.00446414303213014\\
160.01	0.00446417291037918\\
161.01	0.00446420342902669\\
162.01	0.00446423460185\\
163.01	0.00446426644292403\\
164.01	0.00446429896662766\\
165.01	0.00446433218765029\\
166.01	0.0044643661209989\\
167.01	0.00446440078200434\\
168.01	0.0044644361863289\\
169.01	0.00446447234997324\\
170.01	0.00446450928928349\\
171.01	0.00446454702095951\\
172.01	0.00446458556206192\\
173.01	0.0044646249300198\\
174.01	0.00446466514263921\\
175.01	0.0044647062181111\\
176.01	0.00446474817501948\\
177.01	0.00446479103235033\\
178.01	0.00446483480950005\\
179.01	0.00446487952628447\\
180.01	0.00446492520294781\\
181.01	0.00446497186017238\\
182.01	0.00446501951908744\\
183.01	0.00446506820127945\\
184.01	0.00446511792880173\\
185.01	0.00446516872418462\\
186.01	0.00446522061044594\\
187.01	0.00446527361110154\\
188.01	0.00446532775017572\\
189.01	0.00446538305221306\\
190.01	0.00446543954228924\\
191.01	0.00446549724602228\\
192.01	0.00446555618958482\\
193.01	0.00446561639971589\\
194.01	0.00446567790373335\\
195.01	0.00446574072954646\\
196.01	0.00446580490566843\\
197.01	0.00446587046123002\\
198.01	0.00446593742599249\\
199.01	0.00446600583036165\\
200.01	0.00446607570540163\\
201.01	0.00446614708284929\\
202.01	0.00446621999512875\\
203.01	0.00446629447536617\\
204.01	0.00446637055740541\\
205.01	0.00446644827582336\\
206.01	0.004466527665946\\
207.01	0.00446660876386459\\
208.01	0.00446669160645191\\
209.01	0.00446677623137995\\
210.01	0.00446686267713675\\
211.01	0.00446695098304456\\
212.01	0.00446704118927747\\
213.01	0.00446713333688023\\
214.01	0.00446722746778715\\
215.01	0.00446732362484133\\
216.01	0.00446742185181453\\
217.01	0.00446752219342743\\
218.01	0.00446762469536992\\
219.01	0.00446772940432243\\
220.01	0.00446783636797721\\
221.01	0.00446794563506084\\
222.01	0.00446805725535614\\
223.01	0.00446817127972544\\
224.01	0.00446828776013407\\
225.01	0.00446840674967409\\
226.01	0.00446852830258902\\
227.01	0.00446865247429892\\
228.01	0.0044687793214257\\
229.01	0.00446890890181946\\
230.01	0.00446904127458507\\
231.01	0.00446917650010961\\
232.01	0.00446931464009022\\
233.01	0.00446945575756227\\
234.01	0.00446959991692916\\
235.01	0.00446974718399143\\
236.01	0.00446989762597724\\
237.01	0.0044700513115738\\
238.01	0.00447020831095871\\
239.01	0.00447036869583235\\
240.01	0.00447053253945165\\
241.01	0.0044706999166631\\
242.01	0.00447087090393777\\
243.01	0.0044710455794067\\
244.01	0.00447122402289703\\
245.01	0.00447140631596839\\
246.01	0.00447159254195124\\
247.01	0.00447178278598523\\
248.01	0.0044719771350583\\
249.01	0.00447217567804721\\
250.01	0.00447237850575861\\
251.01	0.00447258571097083\\
252.01	0.00447279738847675\\
253.01	0.004473013635128\\
254.01	0.00447323454987954\\
255.01	0.00447346023383552\\
256.01	0.00447369079029594\\
257.01	0.00447392632480438\\
258.01	0.00447416694519714\\
259.01	0.00447441276165293\\
260.01	0.00447466388674389\\
261.01	0.00447492043548736\\
262.01	0.0044751825253998\\
263.01	0.00447545027655073\\
264.01	0.00447572381161841\\
265.01	0.00447600325594621\\
266.01	0.00447628873760142\\
267.01	0.00447658038743401\\
268.01	0.00447687833913714\\
269.01	0.00447718272930933\\
270.01	0.0044774936975174\\
271.01	0.00447781138636115\\
272.01	0.00447813594153966\\
273.01	0.00447846751191823\\
274.01	0.00447880624959775\\
275.01	0.00447915230998462\\
276.01	0.00447950585186276\\
277.01	0.00447986703746754\\
278.01	0.00448023603256008\\
279.01	0.00448061300650454\\
280.01	0.00448099813234611\\
281.01	0.00448139158689079\\
282.01	0.00448179355078797\\
283.01	0.00448220420861294\\
284.01	0.00448262374895256\\
285.01	0.00448305236449262\\
286.01	0.00448349025210623\\
287.01	0.00448393761294513\\
288.01	0.00448439465253256\\
289.01	0.00448486158085724\\
290.01	0.00448533861247102\\
291.01	0.00448582596658741\\
292.01	0.00448632386718223\\
293.01	0.00448683254309704\\
294.01	0.00448735222814395\\
295.01	0.0044878831612131\\
296.01	0.00448842558638252\\
297.01	0.00448897975302982\\
298.01	0.00448954591594614\\
299.01	0.00449012433545301\\
300.01	0.00449071527752133\\
301.01	0.00449131901389231\\
302.01	0.00449193582220168\\
303.01	0.00449256598610571\\
304.01	0.00449320979540965\\
305.01	0.00449386754619996\\
306.01	0.00449453954097724\\
307.01	0.00449522608879312\\
308.01	0.00449592750538984\\
309.01	0.00449664411334082\\
310.01	0.00449737624219611\\
311.01	0.00449812422862897\\
312.01	0.00449888841658568\\
313.01	0.00449966915743803\\
314.01	0.00450046681013858\\
315.01	0.00450128174137823\\
316.01	0.00450211432574686\\
317.01	0.00450296494589604\\
318.01	0.00450383399270537\\
319.01	0.00450472186545071\\
320.01	0.00450562897197439\\
321.01	0.00450655572885922\\
322.01	0.00450750256160418\\
323.01	0.00450846990480219\\
324.01	0.0045094582023204\\
325.01	0.00451046790748309\\
326.01	0.00451149948325541\\
327.01	0.00451255340242973\\
328.01	0.00451363014781372\\
329.01	0.00451473021241956\\
330.01	0.0045158540996538\\
331.01	0.00451700232350923\\
332.01	0.00451817540875693\\
333.01	0.00451937389113802\\
334.01	0.0045205983175562\\
335.01	0.00452184924626919\\
336.01	0.00452312724707914\\
337.01	0.00452443290152228\\
338.01	0.00452576680305498\\
339.01	0.00452712955723814\\
340.01	0.00452852178191794\\
341.01	0.00452994410740114\\
342.01	0.0045313971766261\\
343.01	0.00453288164532724\\
344.01	0.00453439818219144\\
345.01	0.00453594746900562\\
346.01	0.00453753020079468\\
347.01	0.00453914708594719\\
348.01	0.00454079884632719\\
349.01	0.00454248621737152\\
350.01	0.00454420994816808\\
351.01	0.00454597080151523\\
352.01	0.0045477695539579\\
353.01	0.00454960699579852\\
354.01	0.00455148393107861\\
355.01	0.0045534011775293\\
356.01	0.00455535956648434\\
357.01	0.00455735994275371\\
358.01	0.00455940316445122\\
359.01	0.00456149010277289\\
360.01	0.00456362164171761\\
361.01	0.00456579867774727\\
362.01	0.00456802211937603\\
363.01	0.00457029288668391\\
364.01	0.00457261191074437\\
365.01	0.00457498013295868\\
366.01	0.00457739850428496\\
367.01	0.00457986798435359\\
368.01	0.00458238954045488\\
369.01	0.0045849641463882\\
370.01	0.00458759278115758\\
371.01	0.0045902764274988\\
372.01	0.00459301607022335\\
373.01	0.00459581269436037\\
374.01	0.00459866728307932\\
375.01	0.00460158081537414\\
376.01	0.00460455426348784\\
377.01	0.00460758859005705\\
378.01	0.00461068474495481\\
379.01	0.00461384366180809\\
380.01	0.00461706625417126\\
381.01	0.00462035341132955\\
382.01	0.00462370599371752\\
383.01	0.00462712482793086\\
384.01	0.00463061070132095\\
385.01	0.00463416435616079\\
386.01	0.00463778648338296\\
387.01	0.00464147771589749\\
388.01	0.00464523862151071\\
389.01	0.00464906969548462\\
390.01	0.00465297135279585\\
391.01	0.0046569439201833\\
392.01	0.00466098762810693\\
393.01	0.00466510260278355\\
394.01	0.0046692888585225\\
395.01	0.0046735462906439\\
396.01	0.00467787466935003\\
397.01	0.00468227363501243\\
398.01	0.00468674269545274\\
399.01	0.00469128122593594\\
400.01	0.00469588847274542\\
401.01	0.00470056356139702\\
402.01	0.0047053055107453\\
403.01	0.00471011325445241\\
404.01	0.00471498567151782\\
405.01	0.00471992162777217\\
406.01	0.00472492003042431\\
407.01	0.004729979897832\\
408.01	0.00473510044661705\\
409.01	0.00474028119792783\\
410.01	0.00474552210393546\\
411.01	0.00475082369429569\\
412.01	0.00475618724000343\\
413.01	0.00476161492831731\\
414.01	0.00476711003659286\\
415.01	0.0047726770839408\\
416.01	0.00477832192627343\\
417.01	0.0047840517405757\\
418.01	0.00478987481540984\\
419.01	0.00479580002291086\\
420.01	0.00480183477499227\\
421.01	0.00480798152282351\\
422.01	0.00481424155114779\\
423.01	0.00482061609714657\\
424.01	0.00482710634457248\\
425.01	0.004833713417513\\
426.01	0.00484043837377887\\
427.01	0.00484728219791494\\
428.01	0.00485424579383584\\
429.01	0.00486132997709308\\
430.01	0.00486853546678707\\
431.01	0.00487586287714664\\
432.01	0.00488331270880467\\
433.01	0.00489088533981411\\
434.01	0.00489858101645825\\
435.01	0.00490639984392569\\
436.01	0.00491434177693975\\
437.01	0.00492240661045185\\
438.01	0.00493059397053315\\
439.01	0.00493890330562723\\
440.01	0.00494733387835797\\
441.01	0.00495588475812132\\
442.01	0.00496455481473447\\
443.01	0.00497334271345355\\
444.01	0.00498224691172661\\
445.01	0.00499126565809582\\
446.01	0.00500039699372201\\
447.01	0.00500963875705959\\
448.01	0.00501898859226985\\
449.01	0.00502844396201267\\
450.01	0.00503800216530673\\
451.01	0.00504766036118512\\
452.01	0.0050574155988901\\
453.01	0.00506726485534463\\
454.01	0.00507720508058666\\
455.01	0.00508723325175496\\
456.01	0.00509734643604093\\
457.01	0.00510754186275515\\
458.01	0.00511781700427356\\
459.01	0.00512816966509334\\
460.01	0.00513859807751232\\
461.01	0.00514910100151217\\
462.01	0.00515967782524113\\
463.01	0.00517032866103863\\
464.01	0.0051810544302055\\
465.01	0.00519185692774328\\
466.01	0.00520273885614411\\
467.01	0.00521370381520816\\
468.01	0.0052247562331483\\
469.01	0.00523590122351676\\
470.01	0.00524714435373932\\
471.01	0.00525849131582871\\
472.01	0.0052699475006005\\
473.01	0.00528151749711198\\
474.01	0.00529320457465202\\
475.01	0.005305010264441\\
476.01	0.0053169344596347\\
477.01	0.00532897666318638\\
478.01	0.00534113664366383\\
479.01	0.0053534145068606\\
480.01	0.00536581074972134\\
481.01	0.00537832631442095\\
482.01	0.00539096264107522\\
483.01	0.00540372171722552\\
484.01	0.0054166061218886\\
485.01	0.00542961906163633\\
486.01	0.00544276439587896\\
487.01	0.00545604664833761\\
488.01	0.00546947100165257\\
489.01	0.00548304327227303\\
490.01	0.0054967698633107\\
491.01	0.00551065769401794\\
492.01	0.00552471410611784\\
493.01	0.00553894674947311\\
494.01	0.00555336345262983\\
495.01	0.00556797208761162\\
496.01	0.00558278044280461\\
497.01	0.00559779612238419\\
498.01	0.00561302649450048\\
499.01	0.00562847871143375\\
500.01	0.00564415981983526\\
501.01	0.00566007696195645\\
502.01	0.00567623760351374\\
503.01	0.00569264962297497\\
504.01	0.00570932129199433\\
505.01	0.00572626123942883\\
506.01	0.00574347840841211\\
507.01	0.00576098200787976\\
508.01	0.00577878146059106\\
509.01	0.00579688635037232\\
510.01	0.00581530637193427\\
511.01	0.00583405128707744\\
512.01	0.00585313089125472\\
513.01	0.00587255499412151\\
514.01	0.00589233341665268\\
515.01	0.00591247600545589\\
516.01	0.00593299266193996\\
517.01	0.005953893380126\\
518.01	0.00597518828272067\\
519.01	0.00599688764218327\\
520.01	0.00601900187762823\\
521.01	0.00604154153722314\\
522.01	0.00606451728071976\\
523.01	0.00608793986572709\\
524.01	0.00611182013853306\\
525.01	0.00613616902991156\\
526.01	0.0061609975559043\\
527.01	0.00618631682301966\\
528.01	0.00621213803670315\\
529.01	0.00623847251140029\\
530.01	0.00626533168021377\\
531.01	0.00629272710222778\\
532.01	0.00632067046622661\\
533.01	0.00634917359086493\\
534.01	0.00637824842303027\\
535.01	0.00640790703661753\\
536.01	0.00643816163244836\\
537.01	0.0064690245390617\\
538.01	0.00650050821389488\\
539.01	0.00653262524430583\\
540.01	0.00656538834788008\\
541.01	0.00659881037154011\\
542.01	0.0066329042891283\\
543.01	0.00666768319734616\\
544.01	0.00670316031014888\\
545.01	0.00673934895181159\\
546.01	0.00677626254881891\\
547.01	0.00681391462049223\\
548.01	0.00685231876808774\\
549.01	0.0068914886620728\\
550.01	0.00693143802730413\\
551.01	0.0069721806258649\\
552.01	0.00701373023735284\\
553.01	0.00705610063643022\\
554.01	0.0070993055674506\\
555.01	0.0071433587159477\\
556.01	0.00718827367671961\\
557.01	0.0072340639181856\\
558.01	0.00728074274265835\\
559.01	0.00732832324215779\\
560.01	0.00737681824938\\
561.01	0.00742624028341765\\
562.01	0.0074766014898075\\
563.01	0.00752791357445203\\
564.01	0.00758018773093511\\
565.01	0.0076334345607198\\
566.01	0.00768766398569602\\
567.01	0.00774288515253303\\
568.01	0.0077991063282865\\
569.01	0.00785633478671762\\
570.01	0.00791457668480008\\
571.01	0.00797383692892611\\
572.01	0.0080341190303829\\
573.01	0.00809542494975993\\
574.01	0.00815775493008039\\
575.01	0.00822110731863136\\
576.01	0.00828547837771707\\
577.01	0.00835086208488767\\
578.01	0.00841724992363038\\
579.01	0.00848463066607123\\
580.01	0.00855299014995957\\
581.01	0.0086223110531343\\
582.01	0.0086925726698485\\
583.01	0.00876375069482038\\
584.01	0.00883581702275711\\
585.01	0.00890873957345649\\
586.01	0.00898248215554676\\
587.01	0.00905700438561268\\
588.01	0.0091322616840545\\
589.01	0.00920820537474489\\
590.01	0.00928478292265227\\
591.01	0.00936193835240739\\
592.01	0.00943961290170265\\
593.01	0.00951774597691674\\
594.01	0.00959627649505263\\
595.01	0.00967514471670249\\
596.01	0.00975429470021483\\
597.01	0.00983351919724493\\
598.01	0.00990826330299362\\
599.01	0.00997053306357289\\
599.02	0.00997104257304143\\
599.03	0.00997154904691472\\
599.04	0.00997205245565375\\
599.05	0.00997255276942872\\
599.06	0.00997304995811617\\
599.07	0.00997354399129606\\
599.08	0.00997403483824883\\
599.09	0.00997452246795248\\
599.1	0.00997500684907952\\
599.11	0.00997548794999398\\
599.12	0.00997596573874838\\
599.13	0.0099764401830806\\
599.14	0.0099769112504108\\
599.15	0.00997737890783826\\
599.16	0.00997784312213823\\
599.17	0.0099783038597587\\
599.18	0.00997876108681718\\
599.19	0.00997921476909742\\
599.2	0.00997966487204611\\
599.21	0.00998011136076955\\
599.22	0.00998055420003028\\
599.23	0.0099809933542437\\
599.24	0.00998142878747458\\
599.25	0.00998186046343367\\
599.26	0.00998228834359041\\
599.27	0.00998271238778643\\
599.28	0.00998313255546381\\
599.29	0.00998354880566111\\
599.3	0.00998396109700947\\
599.31	0.00998436938772849\\
599.32	0.00998477363562221\\
599.33	0.00998517379807499\\
599.34	0.00998556983204737\\
599.35	0.00998596169407188\\
599.36	0.00998634934024881\\
599.37	0.00998673272624192\\
599.38	0.00998711180727415\\
599.39	0.00998748653812326\\
599.4	0.00998785687311743\\
599.41	0.00998822276613081\\
599.42	0.00998858417057903\\
599.43	0.00998894103941468\\
599.44	0.00998929332512275\\
599.45	0.00998964097971596\\
599.46	0.00998998395473014\\
599.47	0.00999032220121951\\
599.48	0.0099906556697519\\
599.49	0.00999098431040396\\
599.5	0.00999130807275631\\
599.51	0.00999162690588863\\
599.52	0.00999194075837471\\
599.53	0.00999224957827746\\
599.54	0.00999255331314387\\
599.55	0.00999285190999987\\
599.56	0.00999314531534524\\
599.57	0.00999343347514839\\
599.58	0.00999371633484107\\
599.59	0.00999399383931314\\
599.6	0.00999426593290715\\
599.61	0.009994532559413\\
599.62	0.0099947936620624\\
599.63	0.00999504918352342\\
599.64	0.00999529906589491\\
599.65	0.00999554325070086\\
599.66	0.00999578167888471\\
599.67	0.00999601429080366\\
599.68	0.00999624102622283\\
599.69	0.00999646182430947\\
599.7	0.00999667662362697\\
599.71	0.00999688536212897\\
599.72	0.00999708797715331\\
599.73	0.00999728440541595\\
599.74	0.00999747458300482\\
599.75	0.0099976584453736\\
599.76	0.00999783592733551\\
599.77	0.00999800696305692\\
599.78	0.00999817148605102\\
599.79	0.00999832942917133\\
599.8	0.00999848072460517\\
599.81	0.00999862530386713\\
599.82	0.00999876309779239\\
599.83	0.00999889403653003\\
599.84	0.00999901804953622\\
599.85	0.00999913506556741\\
599.86	0.00999924501267341\\
599.87	0.00999934781819042\\
599.88	0.00999944340873394\\
599.89	0.0099995317101917\\
599.9	0.00999961264771648\\
599.91	0.00999968614571878\\
599.92	0.00999975212785957\\
599.93	0.00999981051704285\\
599.94	0.00999986123540817\\
599.95	0.00999990420432308\\
599.96	0.00999993934437554\\
599.97	0.00999996657536618\\
599.98	0.00999998581630055\\
599.99	0.00999999698538124\\
600	0.01\\
};
\addplot [color=mycolor20,solid,forget plot]
  table[row sep=crcr]{%
0.01	0.00464973790653378\\
1.01	0.00464973895170358\\
2.01	0.00464974001894404\\
3.01	0.00464974110872185\\
4.01	0.00464974222151349\\
5.01	0.00464974335780581\\
6.01	0.00464974451809574\\
7.01	0.00464974570289085\\
8.01	0.00464974691270935\\
9.01	0.00464974814808068\\
10.01	0.00464974940954528\\
11.01	0.00464975069765509\\
12.01	0.00464975201297374\\
13.01	0.00464975335607683\\
14.01	0.00464975472755216\\
15.01	0.00464975612799989\\
16.01	0.00464975755803317\\
17.01	0.00464975901827772\\
18.01	0.00464976050937299\\
19.01	0.00464976203197151\\
20.01	0.00464976358674003\\
21.01	0.00464976517435919\\
22.01	0.00464976679552427\\
23.01	0.0046497684509452\\
24.01	0.00464977014134682\\
25.01	0.0046497718674695\\
26.01	0.00464977363006917\\
27.01	0.00464977542991794\\
28.01	0.00464977726780437\\
29.01	0.00464977914453342\\
30.01	0.00464978106092738\\
31.01	0.00464978301782595\\
32.01	0.00464978501608659\\
33.01	0.00464978705658488\\
34.01	0.00464978914021499\\
35.01	0.00464979126789013\\
36.01	0.0046497934405428\\
37.01	0.00464979565912546\\
38.01	0.00464979792461059\\
39.01	0.00464980023799141\\
40.01	0.00464980260028218\\
41.01	0.00464980501251853\\
42.01	0.00464980747575822\\
43.01	0.0046498099910815\\
44.01	0.00464981255959152\\
45.01	0.00464981518241461\\
46.01	0.00464981786070119\\
47.01	0.00464982059562617\\
48.01	0.00464982338838932\\
49.01	0.00464982624021579\\
50.01	0.00464982915235676\\
51.01	0.00464983212609014\\
52.01	0.00464983516272083\\
53.01	0.00464983826358134\\
54.01	0.00464984143003266\\
55.01	0.00464984466346453\\
56.01	0.00464984796529611\\
57.01	0.004649851336977\\
58.01	0.00464985477998742\\
59.01	0.00464985829583896\\
60.01	0.00464986188607529\\
61.01	0.00464986555227288\\
62.01	0.00464986929604176\\
63.01	0.00464987311902611\\
64.01	0.00464987702290505\\
65.01	0.00464988100939338\\
66.01	0.00464988508024218\\
67.01	0.00464988923723986\\
68.01	0.00464989348221277\\
69.01	0.00464989781702611\\
70.01	0.00464990224358472\\
71.01	0.00464990676383367\\
72.01	0.00464991137975977\\
73.01	0.00464991609339165\\
74.01	0.00464992090680129\\
75.01	0.00464992582210448\\
76.01	0.00464993084146224\\
77.01	0.0046499359670813\\
78.01	0.00464994120121542\\
79.01	0.00464994654616613\\
80.01	0.00464995200428411\\
81.01	0.00464995757796963\\
82.01	0.00464996326967425\\
83.01	0.00464996908190156\\
84.01	0.0046499750172083\\
85.01	0.00464998107820581\\
86.01	0.00464998726756059\\
87.01	0.00464999358799618\\
88.01	0.0046500000422937\\
89.01	0.00465000663329372\\
90.01	0.00465001336389692\\
91.01	0.00465002023706589\\
92.01	0.00465002725582619\\
93.01	0.0046500344232674\\
94.01	0.00465004174254526\\
95.01	0.00465004921688221\\
96.01	0.00465005684956957\\
97.01	0.00465006464396846\\
98.01	0.00465007260351157\\
99.01	0.00465008073170467\\
100.01	0.00465008903212784\\
101.01	0.0046500975084376\\
102.01	0.00465010616436813\\
103.01	0.00465011500373314\\
104.01	0.00465012403042728\\
105.01	0.00465013324842835\\
106.01	0.00465014266179846\\
107.01	0.0046501522746864\\
108.01	0.00465016209132914\\
109.01	0.00465017211605392\\
110.01	0.00465018235327977\\
111.01	0.00465019280752009\\
112.01	0.00465020348338416\\
113.01	0.00465021438557953\\
114.01	0.00465022551891367\\
115.01	0.00465023688829648\\
116.01	0.00465024849874256\\
117.01	0.004650260355373\\
118.01	0.00465027246341811\\
119.01	0.00465028482821934\\
120.01	0.00465029745523195\\
121.01	0.00465031035002725\\
122.01	0.00465032351829534\\
123.01	0.00465033696584731\\
124.01	0.00465035069861795\\
125.01	0.00465036472266848\\
126.01	0.00465037904418932\\
127.01	0.00465039366950243\\
128.01	0.00465040860506462\\
129.01	0.00465042385747007\\
130.01	0.00465043943345331\\
131.01	0.00465045533989245\\
132.01	0.00465047158381193\\
133.01	0.004650488172386\\
134.01	0.00465050511294131\\
135.01	0.0046505224129611\\
136.01	0.00465054008008767\\
137.01	0.00465055812212613\\
138.01	0.00465057654704773\\
139.01	0.00465059536299378\\
140.01	0.00465061457827878\\
141.01	0.00465063420139443\\
142.01	0.00465065424101323\\
143.01	0.00465067470599261\\
144.01	0.00465069560537837\\
145.01	0.00465071694840936\\
146.01	0.00465073874452133\\
147.01	0.00465076100335039\\
148.01	0.00465078373473866\\
149.01	0.00465080694873762\\
150.01	0.00465083065561295\\
151.01	0.00465085486584914\\
152.01	0.00465087959015376\\
153.01	0.00465090483946284\\
154.01	0.00465093062494543\\
155.01	0.00465095695800827\\
156.01	0.00465098385030127\\
157.01	0.00465101131372267\\
158.01	0.00465103936042409\\
159.01	0.00465106800281628\\
160.01	0.00465109725357415\\
161.01	0.00465112712564295\\
162.01	0.00465115763224354\\
163.01	0.00465118878687854\\
164.01	0.00465122060333849\\
165.01	0.00465125309570741\\
166.01	0.00465128627836979\\
167.01	0.00465132016601638\\
168.01	0.00465135477365089\\
169.01	0.00465139011659707\\
170.01	0.00465142621050516\\
171.01	0.00465146307135836\\
172.01	0.0046515007154808\\
173.01	0.00465153915954444\\
174.01	0.00465157842057629\\
175.01	0.00465161851596646\\
176.01	0.00465165946347512\\
177.01	0.0046517012812413\\
178.01	0.00465174398779027\\
179.01	0.00465178760204174\\
180.01	0.00465183214331881\\
181.01	0.00465187763135624\\
182.01	0.00465192408630904\\
183.01	0.00465197152876162\\
184.01	0.00465201997973698\\
185.01	0.00465206946070575\\
186.01	0.00465211999359595\\
187.01	0.00465217160080285\\
188.01	0.00465222430519863\\
189.01	0.00465227813014253\\
190.01	0.00465233309949122\\
191.01	0.00465238923760947\\
192.01	0.00465244656938106\\
193.01	0.00465250512021954\\
194.01	0.00465256491607949\\
195.01	0.00465262598346819\\
196.01	0.00465268834945734\\
197.01	0.00465275204169491\\
198.01	0.00465281708841761\\
199.01	0.00465288351846294\\
200.01	0.00465295136128226\\
201.01	0.00465302064695397\\
202.01	0.00465309140619645\\
203.01	0.00465316367038184\\
204.01	0.00465323747154976\\
205.01	0.00465331284242161\\
206.01	0.00465338981641497\\
207.01	0.00465346842765835\\
208.01	0.00465354871100597\\
209.01	0.00465363070205359\\
210.01	0.00465371443715411\\
211.01	0.0046537999534331\\
212.01	0.00465388728880586\\
213.01	0.0046539764819936\\
214.01	0.00465406757254096\\
215.01	0.0046541606008326\\
216.01	0.00465425560811185\\
217.01	0.00465435263649835\\
218.01	0.00465445172900655\\
219.01	0.00465455292956495\\
220.01	0.00465465628303503\\
221.01	0.00465476183523107\\
222.01	0.00465486963294019\\
223.01	0.00465497972394283\\
224.01	0.00465509215703389\\
225.01	0.00465520698204382\\
226.01	0.00465532424986058\\
227.01	0.00465544401245184\\
228.01	0.00465556632288749\\
229.01	0.00465569123536354\\
230.01	0.0046558188052247\\
231.01	0.00465594908898936\\
232.01	0.00465608214437382\\
233.01	0.00465621803031775\\
234.01	0.00465635680700931\\
235.01	0.00465649853591181\\
236.01	0.00465664327979021\\
237.01	0.00465679110273851\\
238.01	0.00465694207020716\\
239.01	0.00465709624903191\\
240.01	0.00465725370746248\\
241.01	0.00465741451519225\\
242.01	0.00465757874338817\\
243.01	0.00465774646472161\\
244.01	0.00465791775339952\\
245.01	0.00465809268519705\\
246.01	0.00465827133748926\\
247.01	0.00465845378928482\\
248.01	0.00465864012126004\\
249.01	0.0046588304157934\\
250.01	0.0046590247570006\\
251.01	0.00465922323077068\\
252.01	0.00465942592480326\\
253.01	0.00465963292864495\\
254.01	0.00465984433372846\\
255.01	0.00466006023341049\\
256.01	0.00466028072301221\\
257.01	0.0046605058998596\\
258.01	0.00466073586332434\\
259.01	0.0046609707148659\\
260.01	0.0046612105580745\\
261.01	0.00466145549871461\\
262.01	0.00466170564476932\\
263.01	0.00466196110648606\\
264.01	0.00466222199642249\\
265.01	0.00466248842949307\\
266.01	0.00466276052301791\\
267.01	0.00466303839677097\\
268.01	0.00466332217302982\\
269.01	0.00466361197662639\\
270.01	0.00466390793499857\\
271.01	0.00466421017824244\\
272.01	0.00466451883916598\\
273.01	0.00466483405334356\\
274.01	0.00466515595917055\\
275.01	0.00466548469792066\\
276.01	0.00466582041380288\\
277.01	0.00466616325401943\\
278.01	0.00466651336882565\\
279.01	0.00466687091158999\\
280.01	0.00466723603885535\\
281.01	0.00466760891040184\\
282.01	0.00466798968930963\\
283.01	0.00466837854202384\\
284.01	0.00466877563841961\\
285.01	0.00466918115186886\\
286.01	0.0046695952593077\\
287.01	0.00467001814130505\\
288.01	0.00467044998213202\\
289.01	0.00467089096983278\\
290.01	0.00467134129629609\\
291.01	0.00467180115732775\\
292.01	0.00467227075272444\\
293.01	0.00467275028634806\\
294.01	0.00467323996620149\\
295.01	0.00467374000450517\\
296.01	0.00467425061777418\\
297.01	0.00467477202689687\\
298.01	0.00467530445721384\\
299.01	0.00467584813859828\\
300.01	0.00467640330553634\\
301.01	0.00467697019720896\\
302.01	0.00467754905757434\\
303.01	0.00467814013545036\\
304.01	0.00467874368459884\\
305.01	0.00467935996380875\\
306.01	0.00467998923698161\\
307.01	0.00468063177321582\\
308.01	0.00468128784689199\\
309.01	0.00468195773775858\\
310.01	0.00468264173101725\\
311.01	0.00468334011740852\\
312.01	0.00468405319329711\\
313.01	0.00468478126075703\\
314.01	0.0046855246276564\\
315.01	0.00468628360774119\\
316.01	0.0046870585207195\\
317.01	0.00468784969234311\\
318.01	0.00468865745448946\\
319.01	0.00468948214524142\\
320.01	0.00469032410896557\\
321.01	0.00469118369638891\\
322.01	0.00469206126467328\\
323.01	0.00469295717748722\\
324.01	0.00469387180507526\\
325.01	0.00469480552432427\\
326.01	0.00469575871882572\\
327.01	0.00469673177893462\\
328.01	0.00469772510182394\\
329.01	0.00469873909153376\\
330.01	0.00469977415901608\\
331.01	0.00470083072217305\\
332.01	0.00470190920588842\\
333.01	0.0047030100420528\\
334.01	0.0047041336695806\\
335.01	0.00470528053441804\\
336.01	0.00470645108954323\\
337.01	0.00470764579495417\\
338.01	0.00470886511764762\\
339.01	0.00471010953158497\\
340.01	0.00471137951764473\\
341.01	0.0047126755635621\\
342.01	0.00471399816385247\\
343.01	0.00471534781971848\\
344.01	0.00471672503893963\\
345.01	0.00471813033574209\\
346.01	0.00471956423064854\\
347.01	0.00472102725030538\\
348.01	0.00472251992728619\\
349.01	0.00472404279986911\\
350.01	0.00472559641178789\\
351.01	0.004727181311952\\
352.01	0.00472879805413681\\
353.01	0.00473044719663934\\
354.01	0.0047321293018996\\
355.01	0.00473384493608242\\
356.01	0.0047355946686205\\
357.01	0.00473737907171459\\
358.01	0.00473919871978871\\
359.01	0.00474105418889803\\
360.01	0.00474294605608805\\
361.01	0.00474487489870062\\
362.01	0.00474684129362674\\
363.01	0.00474884581650259\\
364.01	0.00475088904084812\\
365.01	0.00475297153714498\\
366.01	0.00475509387185392\\
367.01	0.00475725660637023\\
368.01	0.0047594602959166\\
369.01	0.00476170548837417\\
370.01	0.00476399272305296\\
371.01	0.0047663225294034\\
372.01	0.00476869542567324\\
373.01	0.00477111191751402\\
374.01	0.00477357249654356\\
375.01	0.00477607763887379\\
376.01	0.0047786278036154\\
377.01	0.00478122343137151\\
378.01	0.00478386494273864\\
379.01	0.00478655273683741\\
380.01	0.00478928718989483\\
381.01	0.00479206865391218\\
382.01	0.00479489745545153\\
383.01	0.00479777389458493\\
384.01	0.00480069824405516\\
385.01	0.00480367074870561\\
386.01	0.00480669162524516\\
387.01	0.00480976106242338\\
388.01	0.00481287922170244\\
389.01	0.00481604623852108\\
390.01	0.00481926222425845\\
391.01	0.00482252726901594\\
392.01	0.00482584144534431\\
393.01	0.00482920481305503\\
394.01	0.00483261742525727\\
395.01	0.0048360793357707\\
396.01	0.00483959060805844\\
397.01	0.004843151325815\\
398.01	0.00484676160532981\\
399.01	0.00485042160970917\\
400.01	0.00485413156499516\\
401.01	0.00485789177814569\\
402.01	0.00486170265674173\\
403.01	0.00486556473015579\\
404.01	0.00486947867174012\\
405.01	0.00487344532137304\\
406.01	0.0048774657074195\\
407.01	0.00488154106683368\\
408.01	0.00488567286172653\\
409.01	0.00488986279028111\\
410.01	0.00489411278941566\\
411.01	0.00489842502613223\\
412.01	0.0049028018741073\\
413.01	0.00490724587192772\\
414.01	0.00491175965964245\\
415.01	0.00491634589133311\\
416.01	0.00492100712370258\\
417.01	0.00492574568501461\\
418.01	0.00493056353624796\\
419.01	0.00493546214874704\\
420.01	0.00494044246212998\\
421.01	0.00494550512435542\\
422.01	0.00495065072569029\\
423.01	0.00495587981638243\\
424.01	0.00496119290438493\\
425.01	0.00496659045314297\\
426.01	0.0049720728794735\\
427.01	0.00497764055157077\\
428.01	0.00498329378717578\\
429.01	0.00498903285195244\\
430.01	0.0049948579581198\\
431.01	0.00500076926339145\\
432.01	0.00500676687028612\\
433.01	0.00501285082587105\\
434.01	0.00501902112201255\\
435.01	0.00502527769621157\\
436.01	0.00503162043310858\\
437.01	0.00503804916674847\\
438.01	0.00504456368370031\\
439.01	0.00505116372713159\\
440.01	0.00505784900193995\\
441.01	0.00506461918104596\\
442.01	0.00507147391294728\\
443.01	0.00507841283063389\\
444.01	0.00508543556195244\\
445.01	0.00509254174149429\\
446.01	0.00509973102406414\\
447.01	0.00510700309975812\\
448.01	0.00511435771064417\\
449.01	0.00512179466899504\\
450.01	0.00512931387696365\\
451.01	0.00513691534752551\\
452.01	0.00514459922642688\\
453.01	0.00515236581477999\\
454.01	0.00516021559183744\\
455.01	0.00516814923734875\\
456.01	0.00517616765276563\\
457.01	0.00518427198042107\\
458.01	0.00519246361966003\\
459.01	0.00520074423876754\\
460.01	0.00520911578143264\\
461.01	0.00521758046642593\\
462.01	0.00522614077917966\\
463.01	0.00523479945408369\\
464.01	0.00524355944658267\\
465.01	0.0052524238946363\\
466.01	0.00526139606983764\\
467.01	0.00527047931952553\\
468.01	0.0052796770026217\\
469.01	0.00528899242367249\\
470.01	0.00529842877161917\\
471.01	0.00530798907198014\\
472.01	0.00531767616299825\\
473.01	0.00532749270713549\\
474.01	0.00533744124778705\\
475.01	0.00534752431500228\\
476.01	0.0053577445655274\\
477.01	0.00536810488489441\\
478.01	0.00537860841506019\\
479.01	0.00538925856432541\\
480.01	0.00540005901450961\\
481.01	0.00541101372510494\\
482.01	0.00542212693396527\\
483.01	0.00543340315415176\\
484.01	0.00544484716664075\\
485.01	0.00545646400873522\\
486.01	0.00546825895820363\\
487.01	0.00548023751340282\\
488.01	0.00549240536992349\\
489.01	0.00550476839463081\\
490.01	0.00551733259832464\\
491.01	0.00553010410861765\\
492.01	0.00554308914494975\\
493.01	0.00555629399789594\\
494.01	0.00556972501498051\\
495.01	0.00558338859499675\\
496.01	0.00559729119222796\\
497.01	0.00561143933087332\\
498.01	0.00562583962832382\\
499.01	0.00564049882376123\\
500.01	0.00565542380614559\\
501.01	0.00567062163372628\\
502.01	0.00568609953794767\\
503.01	0.00570186491384957\\
504.01	0.00571792530677094\\
505.01	0.0057342883991483\\
506.01	0.00575096199816087\\
507.01	0.00576795402485959\\
508.01	0.00578527250539405\\
509.01	0.0058029255648646\\
510.01	0.00582092142417603\\
511.01	0.00583926840003969\\
512.01	0.00585797490797739\\
513.01	0.0058770494678337\\
514.01	0.00589650071095073\\
515.01	0.00591633738786673\\
516.01	0.00593656837525181\\
517.01	0.0059572026809165\\
518.01	0.00597824944621915\\
519.01	0.0059997179460891\\
520.01	0.00602161758796987\\
521.01	0.0060439579112614\\
522.01	0.00606674858788415\\
523.01	0.00608999942394906\\
524.01	0.00611372036237272\\
525.01	0.00613792148619637\\
526.01	0.00616261302230617\\
527.01	0.0061878053452192\\
528.01	0.0062135089806068\\
529.01	0.00623973460829085\\
530.01	0.0062664930645544\\
531.01	0.00629379534375352\\
532.01	0.00632165259935641\\
533.01	0.00635007614461695\\
534.01	0.00637907745305501\\
535.01	0.00640866815875527\\
536.01	0.00643886005636567\\
537.01	0.00646966510064252\\
538.01	0.00650109540539154\\
539.01	0.00653316324167511\\
540.01	0.00656588103517573\\
541.01	0.00659926136263406\\
542.01	0.0066333169472997\\
543.01	0.00666806065334119\\
544.01	0.00670350547915122\\
545.01	0.0067396645494603\\
546.01	0.00677655110613201\\
547.01	0.00681417849748861\\
548.01	0.00685256016600407\\
549.01	0.0068917096341935\\
550.01	0.00693164048852518\\
551.01	0.00697236636117179\\
552.01	0.00701390090940104\\
553.01	0.00705625779239237\\
554.01	0.00709945064524056\\
555.01	0.0071434930498821\\
556.01	0.00718839850265694\\
557.01	0.0072341803781922\\
558.01	0.0072808518892716\\
559.01	0.00732842604233042\\
560.01	0.00737691558819096\\
561.01	0.00742633296762457\\
562.01	0.00747669025130068\\
563.01	0.00752799907365417\\
564.01	0.00758027056017663\\
565.01	0.00763351524761381\\
566.01	0.00768774299653275\\
567.01	0.00774296289570682\\
568.01	0.00779918315776296\\
569.01	0.0078564110055382\\
570.01	0.00791465254861082\\
571.01	0.00797391264950997\\
572.01	0.00803419477916598\\
573.01	0.00809550086125977\\
574.01	0.00815783110525981\\
575.01	0.00822118382812191\\
576.01	0.00828555526487572\\
577.01	0.00835093936865398\\
578.01	0.00841732760115599\\
579.01	0.00848470871510037\\
580.01	0.00855306853095043\\
581.01	0.00862238971112278\\
582.01	0.00869265153606965\\
583.01	0.00876382968811866\\
584.01	0.00883589605083215\\
585.01	0.00890881853400895\\
586.01	0.00898256093740584\\
587.01	0.00905708286994538\\
588.01	0.00913233974577567\\
589.01	0.00920828288426781\\
590.01	0.00928485974814205\\
591.01	0.00936201436272632\\
592.01	0.00943968797026876\\
593.01	0.0095178199867371\\
594.01	0.0095963493452442\\
595.01	0.00967521633087718\\
596.01	0.00975436503718722\\
597.01	0.00983355571844508\\
598.01	0.00990826330299362\\
599.01	0.00997053306357289\\
599.02	0.00997104257304143\\
599.03	0.00997154904691472\\
599.04	0.00997205245565375\\
599.05	0.00997255276942872\\
599.06	0.00997304995811617\\
599.07	0.00997354399129606\\
599.08	0.00997403483824883\\
599.09	0.00997452246795248\\
599.1	0.00997500684907951\\
599.11	0.00997548794999398\\
599.12	0.00997596573874838\\
599.13	0.0099764401830806\\
599.14	0.0099769112504108\\
599.15	0.00997737890783826\\
599.16	0.00997784312213823\\
599.17	0.0099783038597587\\
599.18	0.00997876108681718\\
599.19	0.00997921476909742\\
599.2	0.00997966487204611\\
599.21	0.00998011136076955\\
599.22	0.00998055420003028\\
599.23	0.00998099335424369\\
599.24	0.00998142878747458\\
599.25	0.00998186046343367\\
599.26	0.00998228834359041\\
599.27	0.00998271238778643\\
599.28	0.00998313255546381\\
599.29	0.00998354880566111\\
599.3	0.00998396109700947\\
599.31	0.00998436938772849\\
599.32	0.00998477363562221\\
599.33	0.00998517379807499\\
599.34	0.00998556983204737\\
599.35	0.00998596169407188\\
599.36	0.00998634934024881\\
599.37	0.00998673272624192\\
599.38	0.00998711180727415\\
599.39	0.00998748653812326\\
599.4	0.00998785687311743\\
599.41	0.00998822276613081\\
599.42	0.00998858417057903\\
599.43	0.00998894103941468\\
599.44	0.00998929332512275\\
599.45	0.00998964097971596\\
599.46	0.00998998395473014\\
599.47	0.00999032220121951\\
599.48	0.0099906556697519\\
599.49	0.00999098431040396\\
599.5	0.00999130807275631\\
599.51	0.00999162690588863\\
599.52	0.00999194075837471\\
599.53	0.00999224957827746\\
599.54	0.00999255331314386\\
599.55	0.00999285190999987\\
599.56	0.00999314531534524\\
599.57	0.00999343347514839\\
599.58	0.00999371633484107\\
599.59	0.00999399383931314\\
599.6	0.00999426593290715\\
599.61	0.009994532559413\\
599.62	0.0099947936620624\\
599.63	0.00999504918352342\\
599.64	0.00999529906589491\\
599.65	0.00999554325070086\\
599.66	0.00999578167888471\\
599.67	0.00999601429080366\\
599.68	0.00999624102622284\\
599.69	0.00999646182430947\\
599.7	0.00999667662362697\\
599.71	0.00999688536212897\\
599.72	0.00999708797715332\\
599.73	0.00999728440541596\\
599.74	0.00999747458300482\\
599.75	0.0099976584453736\\
599.76	0.00999783592733551\\
599.77	0.00999800696305692\\
599.78	0.00999817148605102\\
599.79	0.00999832942917133\\
599.8	0.00999848072460517\\
599.81	0.00999862530386713\\
599.82	0.00999876309779239\\
599.83	0.00999889403653003\\
599.84	0.00999901804953622\\
599.85	0.00999913506556741\\
599.86	0.00999924501267341\\
599.87	0.00999934781819042\\
599.88	0.00999944340873394\\
599.89	0.0099995317101917\\
599.9	0.00999961264771648\\
599.91	0.00999968614571878\\
599.92	0.00999975212785958\\
599.93	0.00999981051704285\\
599.94	0.00999986123540817\\
599.95	0.00999990420432308\\
599.96	0.00999993934437554\\
599.97	0.00999996657536618\\
599.98	0.00999998581630055\\
599.99	0.00999999698538124\\
600	0.01\\
};
\addplot [color=mycolor21,solid,forget plot]
  table[row sep=crcr]{%
0.01	0.00476861616730102\\
1.01	0.00476861719411399\\
2.01	0.00476861824248907\\
3.01	0.00476861931287929\\
4.01	0.00476862040574691\\
5.01	0.00476862152156397\\
6.01	0.00476862266081259\\
7.01	0.00476862382398489\\
8.01	0.0047686250115834\\
9.01	0.00476862622412094\\
10.01	0.00476862746212145\\
11.01	0.00476862872611968\\
12.01	0.00476863001666161\\
13.01	0.00476863133430465\\
14.01	0.00476863267961813\\
15.01	0.00476863405318324\\
16.01	0.00476863545559318\\
17.01	0.00476863688745406\\
18.01	0.00476863834938408\\
19.01	0.00476863984201516\\
20.01	0.00476864136599198\\
21.01	0.00476864292197279\\
22.01	0.00476864451062983\\
23.01	0.00476864613264942\\
24.01	0.00476864778873234\\
25.01	0.00476864947959391\\
26.01	0.0047686512059648\\
27.01	0.00476865296859073\\
28.01	0.00476865476823304\\
29.01	0.00476865660566943\\
30.01	0.00476865848169362\\
31.01	0.00476866039711596\\
32.01	0.00476866235276407\\
33.01	0.00476866434948292\\
34.01	0.00476866638813504\\
35.01	0.00476866846960131\\
36.01	0.00476867059478095\\
37.01	0.00476867276459203\\
38.01	0.00476867497997197\\
39.01	0.00476867724187799\\
40.01	0.00476867955128718\\
41.01	0.00476868190919749\\
42.01	0.00476868431662754\\
43.01	0.00476868677461749\\
44.01	0.00476868928422916\\
45.01	0.00476869184654694\\
46.01	0.00476869446267791\\
47.01	0.00476869713375229\\
48.01	0.00476869986092417\\
49.01	0.0047687026453717\\
50.01	0.00476870548829811\\
51.01	0.00476870839093149\\
52.01	0.00476871135452607\\
53.01	0.00476871438036226\\
54.01	0.00476871746974734\\
55.01	0.00476872062401618\\
56.01	0.00476872384453153\\
57.01	0.00476872713268495\\
58.01	0.00476873048989689\\
59.01	0.00476873391761808\\
60.01	0.00476873741732951\\
61.01	0.00476874099054314\\
62.01	0.00476874463880295\\
63.01	0.004768748363685\\
64.01	0.00476875216679896\\
65.01	0.00476875604978781\\
66.01	0.00476876001432934\\
67.01	0.00476876406213643\\
68.01	0.00476876819495806\\
69.01	0.00476877241457976\\
70.01	0.00476877672282479\\
71.01	0.00476878112155443\\
72.01	0.00476878561266933\\
73.01	0.00476879019810998\\
74.01	0.00476879487985733\\
75.01	0.0047687996599344\\
76.01	0.00476880454040637\\
77.01	0.00476880952338192\\
78.01	0.004768814611014\\
79.01	0.00476881980550074\\
80.01	0.00476882510908637\\
81.01	0.00476883052406239\\
82.01	0.00476883605276847\\
83.01	0.00476884169759309\\
84.01	0.0047688474609754\\
85.01	0.0047688533454051\\
86.01	0.00476885935342482\\
87.01	0.00476886548763007\\
88.01	0.0047688717506712\\
89.01	0.00476887814525404\\
90.01	0.0047688846741414\\
91.01	0.0047688913401539\\
92.01	0.00476889814617147\\
93.01	0.00476890509513456\\
94.01	0.00476891219004512\\
95.01	0.00476891943396873\\
96.01	0.00476892683003471\\
97.01	0.00476893438143813\\
98.01	0.00476894209144139\\
99.01	0.00476894996337524\\
100.01	0.00476895800064057\\
101.01	0.00476896620670937\\
102.01	0.00476897458512662\\
103.01	0.00476898313951174\\
104.01	0.00476899187356015\\
105.01	0.00476900079104474\\
106.01	0.00476900989581783\\
107.01	0.00476901919181233\\
108.01	0.0047690286830437\\
109.01	0.00476903837361184\\
110.01	0.00476904826770286\\
111.01	0.00476905836959025\\
112.01	0.00476906868363763\\
113.01	0.00476907921429993\\
114.01	0.00476908996612559\\
115.01	0.00476910094375872\\
116.01	0.00476911215194051\\
117.01	0.00476912359551201\\
118.01	0.0047691352794156\\
119.01	0.00476914720869763\\
120.01	0.0047691593885099\\
121.01	0.00476917182411275\\
122.01	0.00476918452087656\\
123.01	0.00476919748428457\\
124.01	0.00476921071993493\\
125.01	0.00476922423354344\\
126.01	0.00476923803094538\\
127.01	0.00476925211809902\\
128.01	0.00476926650108682\\
129.01	0.00476928118611947\\
130.01	0.0047692961795377\\
131.01	0.00476931148781534\\
132.01	0.00476932711756173\\
133.01	0.00476934307552487\\
134.01	0.00476935936859459\\
135.01	0.00476937600380462\\
136.01	0.00476939298833644\\
137.01	0.00476941032952231\\
138.01	0.00476942803484792\\
139.01	0.00476944611195597\\
140.01	0.00476946456864935\\
141.01	0.00476948341289439\\
142.01	0.00476950265282454\\
143.01	0.00476952229674375\\
144.01	0.00476954235312999\\
145.01	0.00476956283063863\\
146.01	0.00476958373810657\\
147.01	0.00476960508455605\\
148.01	0.00476962687919795\\
149.01	0.00476964913143642\\
150.01	0.00476967185087242\\
151.01	0.00476969504730804\\
152.01	0.00476971873075087\\
153.01	0.00476974291141755\\
154.01	0.00476976759973927\\
155.01	0.00476979280636534\\
156.01	0.00476981854216808\\
157.01	0.0047698448182474\\
158.01	0.00476987164593562\\
159.01	0.00476989903680212\\
160.01	0.00476992700265862\\
161.01	0.00476995555556388\\
162.01	0.0047699847078289\\
163.01	0.00477001447202256\\
164.01	0.00477004486097647\\
165.01	0.00477007588779071\\
166.01	0.0047701075658392\\
167.01	0.00477013990877577\\
168.01	0.00477017293053987\\
169.01	0.00477020664536186\\
170.01	0.00477024106777015\\
171.01	0.00477027621259675\\
172.01	0.00477031209498327\\
173.01	0.00477034873038823\\
174.01	0.00477038613459291\\
175.01	0.00477042432370826\\
176.01	0.00477046331418203\\
177.01	0.00477050312280514\\
178.01	0.00477054376671938\\
179.01	0.00477058526342429\\
180.01	0.00477062763078501\\
181.01	0.0047706708870392\\
182.01	0.00477071505080554\\
183.01	0.00477076014109076\\
184.01	0.00477080617729847\\
185.01	0.00477085317923694\\
186.01	0.00477090116712756\\
187.01	0.00477095016161314\\
188.01	0.00477100018376705\\
189.01	0.00477105125510177\\
190.01	0.00477110339757805\\
191.01	0.00477115663361444\\
192.01	0.00477121098609609\\
193.01	0.00477126647838494\\
194.01	0.00477132313432947\\
195.01	0.00477138097827436\\
196.01	0.00477144003507117\\
197.01	0.00477150033008836\\
198.01	0.00477156188922195\\
199.01	0.00477162473890695\\
200.01	0.00477168890612775\\
201.01	0.00477175441842938\\
202.01	0.00477182130392964\\
203.01	0.00477188959133029\\
204.01	0.00477195930992912\\
205.01	0.00477203048963237\\
206.01	0.00477210316096681\\
207.01	0.00477217735509267\\
208.01	0.00477225310381692\\
209.01	0.00477233043960577\\
210.01	0.00477240939559841\\
211.01	0.00477249000562105\\
212.01	0.00477257230420047\\
213.01	0.00477265632657851\\
214.01	0.00477274210872641\\
215.01	0.00477282968736015\\
216.01	0.00477291909995488\\
217.01	0.0047730103847609\\
218.01	0.00477310358081895\\
219.01	0.00477319872797646\\
220.01	0.00477329586690389\\
221.01	0.00477339503911099\\
222.01	0.00477349628696452\\
223.01	0.0047735996537048\\
224.01	0.00477370518346353\\
225.01	0.0047738129212819\\
226.01	0.00477392291312882\\
227.01	0.00477403520591931\\
228.01	0.0047741498475338\\
229.01	0.00477426688683723\\
230.01	0.0047743863736991\\
231.01	0.00477450835901302\\
232.01	0.00477463289471761\\
233.01	0.00477476003381704\\
234.01	0.00477488983040249\\
235.01	0.00477502233967354\\
236.01	0.00477515761796042\\
237.01	0.00477529572274589\\
238.01	0.0047754367126888\\
239.01	0.00477558064764674\\
240.01	0.00477572758869982\\
241.01	0.0047758775981749\\
242.01	0.00477603073966978\\
243.01	0.00477618707807849\\
244.01	0.00477634667961647\\
245.01	0.00477650961184622\\
246.01	0.00477667594370403\\
247.01	0.00477684574552628\\
248.01	0.00477701908907684\\
249.01	0.00477719604757469\\
250.01	0.00477737669572197\\
251.01	0.00477756110973278\\
252.01	0.00477774936736183\\
253.01	0.00477794154793462\\
254.01	0.00477813773237686\\
255.01	0.00477833800324561\\
256.01	0.00477854244475999\\
257.01	0.00477875114283273\\
258.01	0.004778964185102\\
259.01	0.00477918166096468\\
260.01	0.00477940366160852\\
261.01	0.00477963028004651\\
262.01	0.00477986161115046\\
263.01	0.00478009775168553\\
264.01	0.00478033880034535\\
265.01	0.00478058485778853\\
266.01	0.00478083602667312\\
267.01	0.00478109241169455\\
268.01	0.00478135411962249\\
269.01	0.00478162125933836\\
270.01	0.00478189394187305\\
271.01	0.0047821722804466\\
272.01	0.00478245639050634\\
273.01	0.00478274638976706\\
274.01	0.00478304239825111\\
275.01	0.00478334453832885\\
276.01	0.00478365293475938\\
277.01	0.00478396771473295\\
278.01	0.0047842890079122\\
279.01	0.00478461694647471\\
280.01	0.00478495166515576\\
281.01	0.00478529330129172\\
282.01	0.00478564199486355\\
283.01	0.00478599788854078\\
284.01	0.00478636112772578\\
285.01	0.00478673186059865\\
286.01	0.00478711023816181\\
287.01	0.00478749641428546\\
288.01	0.00478789054575286\\
289.01	0.00478829279230639\\
290.01	0.00478870331669305\\
291.01	0.00478912228471053\\
292.01	0.00478954986525319\\
293.01	0.00478998623035888\\
294.01	0.0047904315552543\\
295.01	0.00479088601840179\\
296.01	0.00479134980154507\\
297.01	0.00479182308975522\\
298.01	0.00479230607147642\\
299.01	0.00479279893857149\\
300.01	0.0047933018863669\\
301.01	0.00479381511369782\\
302.01	0.00479433882295176\\
303.01	0.00479487322011277\\
304.01	0.00479541851480386\\
305.01	0.00479597492032989\\
306.01	0.00479654265371819\\
307.01	0.00479712193575974\\
308.01	0.00479771299104766\\
309.01	0.00479831604801585\\
310.01	0.00479893133897547\\
311.01	0.0047995591001504\\
312.01	0.00480019957171043\\
313.01	0.00480085299780354\\
314.01	0.00480151962658572\\
315.01	0.00480219971024873\\
316.01	0.00480289350504531\\
317.01	0.00480360127131285\\
318.01	0.00480432327349309\\
319.01	0.00480505978014947\\
320.01	0.00480581106398194\\
321.01	0.00480657740183742\\
322.01	0.0048073590747173\\
323.01	0.00480815636778088\\
324.01	0.00480896957034463\\
325.01	0.00480979897587665\\
326.01	0.00481064488198756\\
327.01	0.00481150759041511\\
328.01	0.00481238740700418\\
329.01	0.00481328464168064\\
330.01	0.00481419960841982\\
331.01	0.0048151326252079\\
332.01	0.00481608401399722\\
333.01	0.00481705410065363\\
334.01	0.00481804321489739\\
335.01	0.00481905169023573\\
336.01	0.00482007986388665\\
337.01	0.00482112807669473\\
338.01	0.00482219667303738\\
339.01	0.00482328600072193\\
340.01	0.00482439641087255\\
341.01	0.00482552825780748\\
342.01	0.00482668189890442\\
343.01	0.00482785769445584\\
344.01	0.00482905600751238\\
345.01	0.00483027720371537\\
346.01	0.00483152165111599\\
347.01	0.00483278971998301\\
348.01	0.00483408178259816\\
349.01	0.00483539821303822\\
350.01	0.00483673938694509\\
351.01	0.00483810568128304\\
352.01	0.00483949747408331\\
353.01	0.00484091514417623\\
354.01	0.00484235907091192\\
355.01	0.00484382963386931\\
356.01	0.00484532721255457\\
357.01	0.00484685218608939\\
358.01	0.00484840493289183\\
359.01	0.00484998583034842\\
360.01	0.00485159525448194\\
361.01	0.0048532335796154\\
362.01	0.00485490117803507\\
363.01	0.00485659841965547\\
364.01	0.00485832567168921\\
365.01	0.00486008329832617\\
366.01	0.00486187166042616\\
367.01	0.00486369111522957\\
368.01	0.00486554201609203\\
369.01	0.0048674247122505\\
370.01	0.0048693395486261\\
371.01	0.00487128686567313\\
372.01	0.00487326699928278\\
373.01	0.0048752802807516\\
374.01	0.00487732703682547\\
375.01	0.0048794075898316\\
376.01	0.00488152225791\\
377.01	0.00488367135536014\\
378.01	0.00488585519311742\\
379.01	0.00488807407937422\\
380.01	0.00489032832036324\\
381.01	0.00489261822132035\\
382.01	0.00489494408764431\\
383.01	0.00489730622627159\\
384.01	0.00489970494728327\\
385.01	0.00490214056576063\\
386.01	0.00490461340390473\\
387.01	0.00490712379343222\\
388.01	0.00490967207825653\\
389.01	0.00491225861745935\\
390.01	0.0049148837885504\\
391.01	0.00491754799100624\\
392.01	0.00492025165006933\\
393.01	0.00492299522077592\\
394.01	0.00492577919216772\\
395.01	0.00492860409162466\\
396.01	0.0049314704892353\\
397.01	0.00493437900210082\\
398.01	0.00493733029844079\\
399.01	0.0049403251013409\\
400.01	0.00494336419195468\\
401.01	0.00494644841193792\\
402.01	0.00494957866486805\\
403.01	0.00495275591637341\\
404.01	0.00495598119267769\\
405.01	0.00495925557725979\\
406.01	0.00496258020533975\\
407.01	0.00496595625593554\\
408.01	0.00496938494131029\\
409.01	0.00497286749374186\\
410.01	0.00497640514972274\\
411.01	0.00497999913193614\\
412.01	0.00498365062967331\\
413.01	0.00498736077875032\\
414.01	0.0049911306424406\\
415.01	0.00499496119542432\\
416.01	0.00499885331319024\\
417.01	0.0050028077695501\\
418.01	0.00500682524468592\\
419.01	0.00501090634500699\\
420.01	0.00501505163303954\\
421.01	0.00501926165440415\\
422.01	0.00502353694588678\\
423.01	0.00502787803618943\\
424.01	0.00503228544658343\\
425.01	0.0050367596917496\\
426.01	0.00504130128082537\\
427.01	0.00504591071867695\\
428.01	0.00505058850741656\\
429.01	0.00505533514818503\\
430.01	0.00506015114321863\\
431.01	0.00506503699822133\\
432.01	0.00506999322505837\\
433.01	0.00507502034479057\\
434.01	0.0050801188910639\\
435.01	0.00508528941386699\\
436.01	0.00509053248366578\\
437.01	0.00509584869592081\\
438.01	0.00510123867598589\\
439.01	0.00510670308438306\\
440.01	0.00511224262243777\\
441.01	0.00511785803825243\\
442.01	0.00512355013298382\\
443.01	0.0051293197673799\\
444.01	0.00513516786851617\\
445.01	0.00514109543666084\\
446.01	0.00514710355217816\\
447.01	0.00515319338236579\\
448.01	0.00515936618810334\\
449.01	0.00516562333017308\\
450.01	0.00517196627509768\\
451.01	0.00517839660032253\\
452.01	0.00518491599856333\\
453.01	0.00519152628112898\\
454.01	0.00519822938003053\\
455.01	0.00520502734869473\\
456.01	0.0052119223611215\\
457.01	0.0052189167093529\\
458.01	0.00522601279917589\\
459.01	0.00523321314404475\\
460.01	0.00524052035729951\\
461.01	0.00524793714286628\\
462.01	0.00525546628475967\\
463.01	0.0052631106358542\\
464.01	0.00527087310655482\\
465.01	0.00527875665415938\\
466.01	0.00528676427384443\\
467.01	0.00529489899230969\\
468.01	0.00530316386513296\\
469.01	0.00531156197878918\\
470.01	0.00532009645801362\\
471.01	0.00532877047870171\\
472.01	0.00533758728579034\\
473.01	0.00534655021458\\
474.01	0.00535566271280512\\
475.01	0.00536492835972273\\
476.01	0.00537435087817047\\
477.01	0.00538393413822291\\
478.01	0.00539368215652637\\
479.01	0.0054035990943118\\
480.01	0.00541368925441719\\
481.01	0.00542395707736736\\
482.01	0.00543440713660659\\
483.01	0.00544504413301856\\
484.01	0.00545587288891948\\
485.01	0.00546689834175822\\
486.01	0.00547812553780169\\
487.01	0.00548955962611743\\
488.01	0.00550120585319264\\
489.01	0.00551306955852603\\
490.01	0.00552515617151154\\
491.01	0.00553747120987087\\
492.01	0.00555002027980198\\
493.01	0.00556280907787551\\
494.01	0.00557584339454135\\
495.01	0.00558912911891428\\
496.01	0.00560267224430848\\
497.01	0.00561647887382405\\
498.01	0.00563055522520521\\
499.01	0.00564490763424581\\
500.01	0.00565954255627336\\
501.01	0.00567446656573335\\
502.01	0.00568968635454427\\
503.01	0.00570520873028521\\
504.01	0.00572104061487515\\
505.01	0.00573718904392297\\
506.01	0.00575366116681389\\
507.01	0.00577046424755846\\
508.01	0.00578760566638339\\
509.01	0.00580509292199594\\
510.01	0.00582293363440174\\
511.01	0.00584113554811737\\
512.01	0.00585970653558543\\
513.01	0.00587865460059492\\
514.01	0.00589798788152887\\
515.01	0.00591771465430749\\
516.01	0.00593784333497871\\
517.01	0.00595838248199877\\
518.01	0.00597934079833327\\
519.01	0.00600072713355781\\
520.01	0.00602255048610337\\
521.01	0.00604482000568882\\
522.01	0.00606754499589372\\
523.01	0.00609073491679551\\
524.01	0.00611439938759015\\
525.01	0.00613854818911833\\
526.01	0.00616319126622557\\
527.01	0.00618833872990029\\
528.01	0.00621400085915008\\
529.01	0.00624018810259756\\
530.01	0.00626691107978722\\
531.01	0.00629418058220452\\
532.01	0.00632200757400153\\
533.01	0.00635040319240993\\
534.01	0.0063793787477955\\
535.01	0.00640894572329868\\
536.01	0.00643911577399439\\
537.01	0.00646990072550448\\
538.01	0.00650131257200101\\
539.01	0.0065333634735321\\
540.01	0.00656606575260754\\
541.01	0.00659943188997036\\
542.01	0.00663347451947655\\
543.01	0.00666820642199731\\
544.01	0.00670364051824117\\
545.01	0.00673978986038594\\
546.01	0.0067766676223955\\
547.01	0.0068142870888892\\
548.01	0.00685266164241449\\
549.01	0.00689180474896772\\
550.01	0.0069317299415903\\
551.01	0.00697245080185081\\
552.01	0.00701398093901015\\
553.01	0.00705633396664328\\
554.01	0.00709952347647309\\
555.01	0.00714356300915191\\
556.01	0.00718846602169959\\
557.01	0.00723424585128786\\
558.01	0.00728091567503281\\
559.01	0.00732848846543331\\
560.01	0.00737697694106471\\
561.01	0.00742639351211209\\
562.01	0.00747675022029617\\
563.01	0.00752805867272219\\
564.01	0.00758032996915353\\
565.01	0.00763357462219004\\
566.01	0.00768780246981217\\
567.01	0.0077430225797369\\
568.01	0.00779924314502788\\
569.01	0.00785647137040364\\
570.01	0.00791471334870932\\
571.01	0.00797397392705373\\
572.01	0.00803425656217577\\
573.01	0.0080955631646969\\
574.01	0.00815789393205248\\
575.01	0.00822124717007715\\
576.01	0.00828561910347195\\
577.01	0.00835100367571266\\
578.01	0.00841739233939433\\
579.01	0.00848477383857343\\
580.01	0.00855313398539493\\
581.01	0.00862245543422146\\
582.01	0.00869271745766205\\
583.01	0.00876389573039058\\
584.01	0.00883596212852419\\
585.01	0.00890888455469373\\
586.01	0.00898262680189275\\
587.01	0.00905714847288408\\
588.01	0.00913240497654327\\
589.01	0.00920834762824205\\
590.01	0.00928492388848376\\
591.01	0.00936207778282116\\
592.01	0.00943975055701092\\
593.01	0.00951788163487973\\
594.01	0.00959640996309208\\
595.01	0.00967527584766158\\
596.01	0.00975442341254193\\
597.01	0.00983358307576999\\
598.01	0.00990826330299362\\
599.01	0.00997053306357289\\
599.02	0.00997104257304143\\
599.03	0.00997154904691472\\
599.04	0.00997205245565375\\
599.05	0.00997255276942872\\
599.06	0.00997304995811617\\
599.07	0.00997354399129606\\
599.08	0.00997403483824883\\
599.09	0.00997452246795248\\
599.1	0.00997500684907952\\
599.11	0.00997548794999398\\
599.12	0.00997596573874838\\
599.13	0.0099764401830806\\
599.14	0.0099769112504108\\
599.15	0.00997737890783826\\
599.16	0.00997784312213823\\
599.17	0.0099783038597587\\
599.18	0.00997876108681718\\
599.19	0.00997921476909742\\
599.2	0.00997966487204611\\
599.21	0.00998011136076955\\
599.22	0.00998055420003028\\
599.23	0.0099809933542437\\
599.24	0.00998142878747458\\
599.25	0.00998186046343367\\
599.26	0.00998228834359041\\
599.27	0.00998271238778643\\
599.28	0.00998313255546381\\
599.29	0.00998354880566111\\
599.3	0.00998396109700947\\
599.31	0.00998436938772849\\
599.32	0.00998477363562221\\
599.33	0.00998517379807499\\
599.34	0.00998556983204737\\
599.35	0.00998596169407188\\
599.36	0.00998634934024881\\
599.37	0.00998673272624192\\
599.38	0.00998711180727415\\
599.39	0.00998748653812326\\
599.4	0.00998785687311743\\
599.41	0.00998822276613081\\
599.42	0.00998858417057903\\
599.43	0.00998894103941468\\
599.44	0.00998929332512275\\
599.45	0.00998964097971596\\
599.46	0.00998998395473014\\
599.47	0.00999032220121951\\
599.48	0.0099906556697519\\
599.49	0.00999098431040396\\
599.5	0.00999130807275631\\
599.51	0.00999162690588863\\
599.52	0.00999194075837471\\
599.53	0.00999224957827746\\
599.54	0.00999255331314387\\
599.55	0.00999285190999987\\
599.56	0.00999314531534524\\
599.57	0.00999343347514839\\
599.58	0.00999371633484107\\
599.59	0.00999399383931314\\
599.6	0.00999426593290715\\
599.61	0.009994532559413\\
599.62	0.0099947936620624\\
599.63	0.00999504918352342\\
599.64	0.00999529906589491\\
599.65	0.00999554325070086\\
599.66	0.00999578167888471\\
599.67	0.00999601429080366\\
599.68	0.00999624102622283\\
599.69	0.00999646182430947\\
599.7	0.00999667662362697\\
599.71	0.00999688536212897\\
599.72	0.00999708797715332\\
599.73	0.00999728440541595\\
599.74	0.00999747458300482\\
599.75	0.0099976584453736\\
599.76	0.0099978359273355\\
599.77	0.00999800696305692\\
599.78	0.00999817148605102\\
599.79	0.00999832942917133\\
599.8	0.00999848072460517\\
599.81	0.00999862530386713\\
599.82	0.00999876309779239\\
599.83	0.00999889403653003\\
599.84	0.00999901804953622\\
599.85	0.00999913506556741\\
599.86	0.00999924501267341\\
599.87	0.00999934781819042\\
599.88	0.00999944340873394\\
599.89	0.00999953171019171\\
599.9	0.00999961264771648\\
599.91	0.00999968614571878\\
599.92	0.00999975212785957\\
599.93	0.00999981051704285\\
599.94	0.00999986123540817\\
599.95	0.00999990420432308\\
599.96	0.00999993934437554\\
599.97	0.00999996657536618\\
599.98	0.00999998581630055\\
599.99	0.00999999698538124\\
600	0.01\\
};
\addplot [color=black!20!mycolor21,solid,forget plot]
  table[row sep=crcr]{%
0.01	0.0048411429700902\\
1.01	0.00484114397590001\\
2.01	0.00484114500270012\\
3.01	0.00484114605092785\\
4.01	0.00484114712102995\\
5.01	0.00484114821346211\\
6.01	0.00484114932868977\\
7.01	0.00484115046718776\\
8.01	0.00484115162944103\\
9.01	0.00484115281594452\\
10.01	0.00484115402720339\\
11.01	0.00484115526373359\\
12.01	0.00484115652606144\\
13.01	0.00484115781472463\\
14.01	0.00484115913027153\\
15.01	0.00484116047326229\\
16.01	0.00484116184426858\\
17.01	0.00484116324387371\\
18.01	0.00484116467267361\\
19.01	0.00484116613127625\\
20.01	0.00484116762030226\\
21.01	0.0048411691403854\\
22.01	0.00484117069217238\\
23.01	0.00484117227632343\\
24.01	0.00484117389351243\\
25.01	0.00484117554442746\\
26.01	0.00484117722977057\\
27.01	0.00484117895025858\\
28.01	0.00484118070662343\\
29.01	0.00484118249961185\\
30.01	0.00484118432998638\\
31.01	0.0048411861985254\\
32.01	0.00484118810602319\\
33.01	0.00484119005329087\\
34.01	0.0048411920411561\\
35.01	0.00484119407046386\\
36.01	0.00484119614207673\\
37.01	0.00484119825687521\\
38.01	0.00484120041575797\\
39.01	0.00484120261964256\\
40.01	0.00484120486946539\\
41.01	0.00484120716618226\\
42.01	0.00484120951076895\\
43.01	0.00484121190422147\\
44.01	0.00484121434755651\\
45.01	0.00484121684181192\\
46.01	0.00484121938804682\\
47.01	0.00484122198734267\\
48.01	0.00484122464080317\\
49.01	0.00484122734955504\\
50.01	0.00484123011474827\\
51.01	0.00484123293755674\\
52.01	0.00484123581917857\\
53.01	0.00484123876083713\\
54.01	0.00484124176378073\\
55.01	0.00484124482928384\\
56.01	0.00484124795864718\\
57.01	0.00484125115319854\\
58.01	0.00484125441429345\\
59.01	0.00484125774331519\\
60.01	0.00484126114167586\\
61.01	0.00484126461081708\\
62.01	0.00484126815221\\
63.01	0.00484127176735651\\
64.01	0.00484127545778942\\
65.01	0.00484127922507353\\
66.01	0.004841283070806\\
67.01	0.00484128699661708\\
68.01	0.00484129100417076\\
69.01	0.00484129509516563\\
70.01	0.00484129927133535\\
71.01	0.00484130353444964\\
72.01	0.00484130788631471\\
73.01	0.00484131232877434\\
74.01	0.00484131686371049\\
75.01	0.00484132149304406\\
76.01	0.00484132621873572\\
77.01	0.00484133104278675\\
78.01	0.00484133596723996\\
79.01	0.00484134099418035\\
80.01	0.00484134612573614\\
81.01	0.00484135136407957\\
82.01	0.00484135671142785\\
83.01	0.00484136217004423\\
84.01	0.00484136774223851\\
85.01	0.00484137343036879\\
86.01	0.00484137923684153\\
87.01	0.00484138516411299\\
88.01	0.00484139121469049\\
89.01	0.00484139739113305\\
90.01	0.00484140369605274\\
91.01	0.00484141013211564\\
92.01	0.00484141670204284\\
93.01	0.00484142340861186\\
94.01	0.00484143025465771\\
95.01	0.00484143724307372\\
96.01	0.00484144437681321\\
97.01	0.0048414516588909\\
98.01	0.00484145909238325\\
99.01	0.00484146668043069\\
100.01	0.00484147442623818\\
101.01	0.00484148233307742\\
102.01	0.00484149040428737\\
103.01	0.00484149864327595\\
104.01	0.00484150705352175\\
105.01	0.00484151563857499\\
106.01	0.00484152440205915\\
107.01	0.00484153334767284\\
108.01	0.00484154247919084\\
109.01	0.00484155180046579\\
110.01	0.00484156131542998\\
111.01	0.00484157102809701\\
112.01	0.00484158094256308\\
113.01	0.00484159106300888\\
114.01	0.00484160139370177\\
115.01	0.00484161193899676\\
116.01	0.00484162270333881\\
117.01	0.00484163369126454\\
118.01	0.0048416449074042\\
119.01	0.00484165635648359\\
120.01	0.00484166804332589\\
121.01	0.00484167997285365\\
122.01	0.00484169215009089\\
123.01	0.00484170458016531\\
124.01	0.00484171726831029\\
125.01	0.00484173021986679\\
126.01	0.00484174344028601\\
127.01	0.00484175693513141\\
128.01	0.00484177071008135\\
129.01	0.00484178477093053\\
130.01	0.00484179912359343\\
131.01	0.00484181377410595\\
132.01	0.00484182872862858\\
133.01	0.00484184399344847\\
134.01	0.00484185957498212\\
135.01	0.00484187547977799\\
136.01	0.00484189171451951\\
137.01	0.0048419082860273\\
138.01	0.00484192520126239\\
139.01	0.00484194246732876\\
140.01	0.00484196009147685\\
141.01	0.0048419780811057\\
142.01	0.00484199644376682\\
143.01	0.00484201518716645\\
144.01	0.00484203431916946\\
145.01	0.00484205384780233\\
146.01	0.00484207378125599\\
147.01	0.00484209412788991\\
148.01	0.00484211489623504\\
149.01	0.00484213609499745\\
150.01	0.00484215773306196\\
151.01	0.00484217981949567\\
152.01	0.00484220236355148\\
153.01	0.00484222537467247\\
154.01	0.00484224886249482\\
155.01	0.00484227283685254\\
156.01	0.00484229730778137\\
157.01	0.00484232228552252\\
158.01	0.00484234778052693\\
159.01	0.00484237380345987\\
160.01	0.00484240036520472\\
161.01	0.00484242747686796\\
162.01	0.00484245514978336\\
163.01	0.00484248339551639\\
164.01	0.00484251222586927\\
165.01	0.00484254165288583\\
166.01	0.00484257168885596\\
167.01	0.00484260234632095\\
168.01	0.00484263363807816\\
169.01	0.00484266557718686\\
170.01	0.00484269817697271\\
171.01	0.00484273145103367\\
172.01	0.00484276541324545\\
173.01	0.00484280007776675\\
174.01	0.00484283545904539\\
175.01	0.0048428715718237\\
176.01	0.00484290843114478\\
177.01	0.00484294605235853\\
178.01	0.00484298445112744\\
179.01	0.00484302364343339\\
180.01	0.0048430636455835\\
181.01	0.00484310447421708\\
182.01	0.00484314614631209\\
183.01	0.00484318867919209\\
184.01	0.00484323209053285\\
185.01	0.00484327639836964\\
186.01	0.00484332162110407\\
187.01	0.00484336777751196\\
188.01	0.00484341488675034\\
189.01	0.00484346296836541\\
190.01	0.00484351204229985\\
191.01	0.0048435621289009\\
192.01	0.00484361324892868\\
193.01	0.004843665423564\\
194.01	0.00484371867441682\\
195.01	0.00484377302353495\\
196.01	0.00484382849341226\\
197.01	0.00484388510699811\\
198.01	0.00484394288770604\\
199.01	0.00484400185942281\\
200.01	0.00484406204651801\\
201.01	0.00484412347385346\\
202.01	0.00484418616679283\\
203.01	0.0048442501512117\\
204.01	0.00484431545350764\\
205.01	0.00484438210060993\\
206.01	0.00484445011999082\\
207.01	0.00484451953967542\\
208.01	0.00484459038825309\\
209.01	0.00484466269488806\\
210.01	0.00484473648933093\\
211.01	0.00484481180192987\\
212.01	0.00484488866364231\\
213.01	0.00484496710604693\\
214.01	0.00484504716135574\\
215.01	0.00484512886242598\\
216.01	0.00484521224277311\\
217.01	0.00484529733658319\\
218.01	0.00484538417872594\\
219.01	0.00484547280476801\\
220.01	0.00484556325098617\\
221.01	0.00484565555438125\\
222.01	0.00484574975269163\\
223.01	0.00484584588440766\\
224.01	0.00484594398878603\\
225.01	0.00484604410586413\\
226.01	0.00484614627647524\\
227.01	0.00484625054226349\\
228.01	0.00484635694569925\\
229.01	0.00484646553009466\\
230.01	0.00484657633961966\\
231.01	0.00484668941931833\\
232.01	0.00484680481512492\\
233.01	0.00484692257388074\\
234.01	0.00484704274335125\\
235.01	0.00484716537224299\\
236.01	0.00484729051022151\\
237.01	0.00484741820792921\\
238.01	0.00484754851700277\\
239.01	0.00484768149009221\\
240.01	0.00484781718087934\\
241.01	0.00484795564409644\\
242.01	0.00484809693554623\\
243.01	0.00484824111212054\\
244.01	0.00484838823182043\\
245.01	0.00484853835377692\\
246.01	0.00484869153827039\\
247.01	0.00484884784675199\\
248.01	0.00484900734186457\\
249.01	0.00484917008746368\\
250.01	0.00484933614863962\\
251.01	0.00484950559173884\\
252.01	0.00484967848438662\\
253.01	0.00484985489550946\\
254.01	0.00485003489535742\\
255.01	0.00485021855552795\\
256.01	0.00485040594898864\\
257.01	0.00485059715010121\\
258.01	0.00485079223464575\\
259.01	0.0048509912798443\\
260.01	0.00485119436438579\\
261.01	0.00485140156845081\\
262.01	0.00485161297373658\\
263.01	0.00485182866348256\\
264.01	0.00485204872249606\\
265.01	0.0048522732371775\\
266.01	0.00485250229554766\\
267.01	0.00485273598727294\\
268.01	0.00485297440369263\\
269.01	0.00485321763784555\\
270.01	0.00485346578449737\\
271.01	0.00485371894016734\\
272.01	0.00485397720315629\\
273.01	0.00485424067357403\\
274.01	0.00485450945336741\\
275.01	0.004854783646348\\
276.01	0.00485506335822079\\
277.01	0.00485534869661168\\
278.01	0.00485563977109645\\
279.01	0.00485593669322879\\
280.01	0.00485623957656975\\
281.01	0.00485654853671559\\
282.01	0.0048568636913264\\
283.01	0.00485718516015521\\
284.01	0.00485751306507655\\
285.01	0.00485784753011481\\
286.01	0.0048581886814732\\
287.01	0.00485853664756201\\
288.01	0.00485889155902688\\
289.01	0.00485925354877727\\
290.01	0.00485962275201414\\
291.01	0.00485999930625809\\
292.01	0.00486038335137674\\
293.01	0.00486077502961195\\
294.01	0.00486117448560672\\
295.01	0.00486158186643137\\
296.01	0.00486199732161045\\
297.01	0.00486242100314759\\
298.01	0.00486285306555097\\
299.01	0.00486329366585744\\
300.01	0.00486374296365687\\
301.01	0.00486420112111491\\
302.01	0.0048646683029957\\
303.01	0.00486514467668332\\
304.01	0.00486563041220268\\
305.01	0.00486612568223942\\
306.01	0.00486663066215824\\
307.01	0.00486714553002138\\
308.01	0.00486767046660457\\
309.01	0.00486820565541281\\
310.01	0.00486875128269441\\
311.01	0.00486930753745329\\
312.01	0.00486987461146118\\
313.01	0.00487045269926614\\
314.01	0.00487104199820167\\
315.01	0.00487164270839258\\
316.01	0.00487225503275988\\
317.01	0.00487287917702328\\
318.01	0.00487351534970219\\
319.01	0.0048741637621145\\
320.01	0.00487482462837312\\
321.01	0.00487549816538051\\
322.01	0.00487618459282051\\
323.01	0.00487688413314849\\
324.01	0.00487759701157832\\
325.01	0.00487832345606769\\
326.01	0.00487906369729975\\
327.01	0.00487981796866284\\
328.01	0.00488058650622751\\
329.01	0.00488136954872072\\
330.01	0.00488216733749693\\
331.01	0.00488298011650676\\
332.01	0.00488380813226291\\
333.01	0.00488465163380299\\
334.01	0.00488551087264984\\
335.01	0.00488638610276886\\
336.01	0.0048872775805229\\
337.01	0.0048881855646245\\
338.01	0.00488911031608599\\
339.01	0.00489005209816659\\
340.01	0.00489101117631845\\
341.01	0.00489198781812954\\
342.01	0.00489298229326658\\
343.01	0.00489399487341473\\
344.01	0.0048950258322186\\
345.01	0.0048960754452206\\
346.01	0.00489714398980129\\
347.01	0.00489823174511918\\
348.01	0.00489933899205165\\
349.01	0.00490046601313895\\
350.01	0.00490161309252951\\
351.01	0.0049027805159302\\
352.01	0.00490396857056039\\
353.01	0.00490517754511256\\
354.01	0.00490640772971881\\
355.01	0.00490765941592694\\
356.01	0.00490893289668548\\
357.01	0.0049102284663405\\
358.01	0.00491154642064512\\
359.01	0.00491288705678504\\
360.01	0.00491425067341986\\
361.01	0.00491563757074398\\
362.01	0.0049170480505699\\
363.01	0.0049184824164339\\
364.01	0.00491994097372978\\
365.01	0.0049214240298706\\
366.01	0.00492293189448334\\
367.01	0.00492446487963805\\
368.01	0.00492602330011557\\
369.01	0.00492760747371546\\
370.01	0.00492921772160859\\
371.01	0.00493085436873687\\
372.01	0.00493251774426251\\
373.01	0.00493420818207001\\
374.01	0.00493592602132382\\
375.01	0.00493767160708272\\
376.01	0.00493944529097344\\
377.01	0.00494124743192449\\
378.01	0.00494307839696045\\
379.01	0.0049449385620557\\
380.01	0.00494682831304756\\
381.01	0.00494874804660422\\
382.01	0.00495069817124415\\
383.01	0.00495267910840111\\
384.01	0.0049546912935258\\
385.01	0.004956735177215\\
386.01	0.00495881122635457\\
387.01	0.00496091992526212\\
388.01	0.00496306177680927\\
389.01	0.00496523730350368\\
390.01	0.00496744704850431\\
391.01	0.00496969157654296\\
392.01	0.00497197147472009\\
393.01	0.00497428735314021\\
394.01	0.00497663984534968\\
395.01	0.00497902960853638\\
396.01	0.00498145732345294\\
397.01	0.00498392369401963\\
398.01	0.00498642944657121\\
399.01	0.00498897532871037\\
400.01	0.00499156210774107\\
401.01	0.0049941905686622\\
402.01	0.00499686151171602\\
403.01	0.00499957574950577\\
404.01	0.00500233410371484\\
405.01	0.00500513740149072\\
406.01	0.00500798647158629\\
407.01	0.00501088214038635\\
408.01	0.00501382522798316\\
409.01	0.00501681654450414\\
410.01	0.00501985688692159\\
411.01	0.00502294703660079\\
412.01	0.00502608775784397\\
413.01	0.00502927979766606\\
414.01	0.00503252388697983\\
415.01	0.00503582074326243\\
416.01	0.00503917107461341\\
417.01	0.00504257558490201\\
418.01	0.00504603497944772\\
419.01	0.0050495499704387\\
420.01	0.00505312128116506\\
421.01	0.00505674964850621\\
422.01	0.00506043582437861\\
423.01	0.00506418057710659\\
424.01	0.0050679846928981\\
425.01	0.00507184897743225\\
426.01	0.00507577425755941\\
427.01	0.00507976138311518\\
428.01	0.00508381122884881\\
429.01	0.00508792469646221\\
430.01	0.00509210271675953\\
431.01	0.00509634625190054\\
432.01	0.00510065629775254\\
433.01	0.00510503388633241\\
434.01	0.00510948008832686\\
435.01	0.00511399601568081\\
436.01	0.00511858282423445\\
437.01	0.00512324171639403\\
438.01	0.00512797394381401\\
439.01	0.00513278081006661\\
440.01	0.00513766367327212\\
441.01	0.00514262394866119\\
442.01	0.00514766311103754\\
443.01	0.00515278269710621\\
444.01	0.0051579843076337\\
445.01	0.00516326960940153\\
446.01	0.00516864033691951\\
447.01	0.00517409829386191\\
448.01	0.00517964535419518\\
449.01	0.00518528346296747\\
450.01	0.00519101463673807\\
451.01	0.00519684096363207\\
452.01	0.00520276460301565\\
453.01	0.00520878778479873\\
454.01	0.00521491280838859\\
455.01	0.00522114204133339\\
456.01	0.00522747791771257\\
457.01	0.00523392293635468\\
458.01	0.00524047965897778\\
459.01	0.00524715070837192\\
460.01	0.00525393876675957\\
461.01	0.00526084657447919\\
462.01	0.00526787692914715\\
463.01	0.00527503268544641\\
464.01	0.00528231675567659\\
465.01	0.00528973211116573\\
466.01	0.00529728178459876\\
467.01	0.00530496887325003\\
468.01	0.0053127965430252\\
469.01	0.00532076803312209\\
470.01	0.0053288866610266\\
471.01	0.005337155827472\\
472.01	0.00534557902094231\\
473.01	0.0053541598213048\\
474.01	0.00536290190225683\\
475.01	0.00537180903246989\\
476.01	0.0053808850756099\\
477.01	0.00539013398969207\\
478.01	0.00539955982617882\\
479.01	0.00540916672897367\\
480.01	0.00541895893336858\\
481.01	0.00542894076500924\\
482.01	0.00543911663894388\\
483.01	0.00544949105882525\\
484.01	0.00546006861633296\\
485.01	0.00547085399088189\\
486.01	0.0054818519496684\\
487.01	0.00549306734809753\\
488.01	0.00550450513061362\\
489.01	0.00551617033193466\\
490.01	0.00552806807866621\\
491.01	0.00554020359124267\\
492.01	0.00555258218611994\\
493.01	0.00556520927811989\\
494.01	0.00557809038281561\\
495.01	0.00559123111884452\\
496.01	0.00560463721005036\\
497.01	0.00561831448738806\\
498.01	0.00563226889057196\\
499.01	0.00564650646950696\\
500.01	0.00566103338559742\\
501.01	0.00567585591306603\\
502.01	0.00569098044040994\\
503.01	0.00570641347207064\\
504.01	0.00572216163033214\\
505.01	0.0057382316574332\\
506.01	0.00575463041787029\\
507.01	0.00577136490085814\\
508.01	0.00578844222291054\\
509.01	0.00580586963049919\\
510.01	0.00582365450275156\\
511.01	0.00584180435414818\\
512.01	0.00586032683719329\\
513.01	0.00587922974503828\\
514.01	0.00589852101405182\\
515.01	0.00591820872634428\\
516.01	0.00593830111225706\\
517.01	0.00595880655283618\\
518.01	0.00597973358230162\\
519.01	0.0060010908905138\\
520.01	0.00602288732542533\\
521.01	0.00604513189549417\\
522.01	0.0060678337720324\\
523.01	0.00609100229146371\\
524.01	0.00611464695746536\\
525.01	0.00613877744297096\\
526.01	0.00616340359201505\\
527.01	0.00618853542139793\\
528.01	0.00621418312215408\\
529.01	0.00624035706080232\\
530.01	0.00626706778035532\\
531.01	0.00629432600106118\\
532.01	0.00632214262084489\\
533.01	0.00635052871541156\\
534.01	0.00637949553797108\\
535.01	0.00640905451853623\\
536.01	0.00643921726274706\\
537.01	0.00646999555016789\\
538.01	0.00650140133200039\\
539.01	0.00653344672815072\\
540.01	0.00656614402358066\\
541.01	0.0065995056638683\\
542.01	0.00663354424989565\\
543.01	0.00666827253156867\\
544.01	0.00670370340047058\\
545.01	0.00673984988133659\\
546.01	0.00677672512222907\\
547.01	0.00681434238327824\\
548.01	0.00685271502384668\\
549.01	0.0068918564879546\\
550.01	0.00693178028779452\\
551.01	0.00697249998514508\\
552.01	0.00701402917047571\\
553.01	0.00705638143951728\\
554.01	0.00709957036705249\\
555.01	0.00714360947765906\\
556.01	0.0071885122131167\\
557.01	0.00723429189616345\\
558.01	0.00728096169026451\\
559.01	0.00732853455502859\\
560.01	0.00737702319688083\\
561.01	0.00742644001457306\\
562.01	0.00747679703908753\\
563.01	0.00752810586745972\\
564.01	0.00758037759002343\\
565.01	0.00763362271055724\\
566.01	0.00768785105879237\\
567.01	0.0077430716947288\\
568.01	0.0077992928041998\\
569.01	0.00785652158513176\\
570.01	0.00791476412396291\\
571.01	0.00797402526172421\\
572.01	0.00803430844934671\\
573.01	0.00809561559185461\\
574.01	0.00815794688123565\\
575.01	0.00822130061796784\\
576.01	0.0082856730214308\\
577.01	0.00835105802976322\\
578.01	0.00841744709016409\\
579.01	0.00848482894120108\\
580.01	0.00855318938941672\\
581.01	0.00862251108345263\\
582.01	0.00869277329009313\\
583.01	0.00876395167812282\\
584.01	0.0088360181177741\\
585.01	0.00890894050590256\\
586.01	0.00898268262998571\\
587.01	0.00905720408773334\\
588.01	0.00913246028370159\\
589.01	0.0092084025300289\\
590.01	0.00928497828552531\\
591.01	0.00936213157616888\\
592.01	0.00943980365099336\\
593.01	0.00951793394087812\\
594.01	0.00959646140447543\\
595.01	0.00967532636617276\\
596.01	0.00975447297649412\\
597.01	0.00983360599464411\\
598.01	0.00990826330299362\\
599.01	0.00997053306357289\\
599.02	0.00997104257304143\\
599.03	0.00997154904691472\\
599.04	0.00997205245565375\\
599.05	0.00997255276942872\\
599.06	0.00997304995811617\\
599.07	0.00997354399129606\\
599.08	0.00997403483824883\\
599.09	0.00997452246795248\\
599.1	0.00997500684907951\\
599.11	0.00997548794999398\\
599.12	0.00997596573874838\\
599.13	0.0099764401830806\\
599.14	0.0099769112504108\\
599.15	0.00997737890783826\\
599.16	0.00997784312213823\\
599.17	0.0099783038597587\\
599.18	0.00997876108681718\\
599.19	0.00997921476909742\\
599.2	0.00997966487204611\\
599.21	0.00998011136076955\\
599.22	0.00998055420003028\\
599.23	0.0099809933542437\\
599.24	0.00998142878747458\\
599.25	0.00998186046343367\\
599.26	0.00998228834359041\\
599.27	0.00998271238778644\\
599.28	0.00998313255546381\\
599.29	0.00998354880566111\\
599.3	0.00998396109700947\\
599.31	0.00998436938772849\\
599.32	0.00998477363562221\\
599.33	0.00998517379807499\\
599.34	0.00998556983204737\\
599.35	0.00998596169407188\\
599.36	0.00998634934024881\\
599.37	0.00998673272624192\\
599.38	0.00998711180727415\\
599.39	0.00998748653812326\\
599.4	0.00998785687311743\\
599.41	0.00998822276613081\\
599.42	0.00998858417057903\\
599.43	0.00998894103941468\\
599.44	0.00998929332512275\\
599.45	0.00998964097971596\\
599.46	0.00998998395473014\\
599.47	0.00999032220121951\\
599.48	0.0099906556697519\\
599.49	0.00999098431040396\\
599.5	0.00999130807275631\\
599.51	0.00999162690588863\\
599.52	0.00999194075837471\\
599.53	0.00999224957827746\\
599.54	0.00999255331314386\\
599.55	0.00999285190999987\\
599.56	0.00999314531534524\\
599.57	0.00999343347514839\\
599.58	0.00999371633484107\\
599.59	0.00999399383931314\\
599.6	0.00999426593290715\\
599.61	0.009994532559413\\
599.62	0.0099947936620624\\
599.63	0.00999504918352342\\
599.64	0.00999529906589491\\
599.65	0.00999554325070086\\
599.66	0.00999578167888471\\
599.67	0.00999601429080366\\
599.68	0.00999624102622283\\
599.69	0.00999646182430947\\
599.7	0.00999667662362697\\
599.71	0.00999688536212897\\
599.72	0.00999708797715332\\
599.73	0.00999728440541595\\
599.74	0.00999747458300482\\
599.75	0.0099976584453736\\
599.76	0.00999783592733551\\
599.77	0.00999800696305692\\
599.78	0.00999817148605102\\
599.79	0.00999832942917133\\
599.8	0.00999848072460517\\
599.81	0.00999862530386713\\
599.82	0.00999876309779239\\
599.83	0.00999889403653003\\
599.84	0.00999901804953622\\
599.85	0.00999913506556741\\
599.86	0.00999924501267341\\
599.87	0.00999934781819042\\
599.88	0.00999944340873394\\
599.89	0.0099995317101917\\
599.9	0.00999961264771648\\
599.91	0.00999968614571878\\
599.92	0.00999975212785958\\
599.93	0.00999981051704285\\
599.94	0.00999986123540817\\
599.95	0.00999990420432308\\
599.96	0.00999993934437554\\
599.97	0.00999996657536618\\
599.98	0.00999998581630055\\
599.99	0.00999999698538124\\
600	0.01\\
};
\addplot [color=black!50!mycolor20,solid,forget plot]
  table[row sep=crcr]{%
0.01	0.00488462543491382\\
1.01	0.00488462642133425\\
2.01	0.00488462742821015\\
3.01	0.00488462845596445\\
4.01	0.0048846295050286\\
5.01	0.00488463057584322\\
6.01	0.00488463166885768\\
7.01	0.00488463278453091\\
8.01	0.00488463392333116\\
9.01	0.00488463508573644\\
10.01	0.00488463627223445\\
11.01	0.00488463748332285\\
12.01	0.00488463871950981\\
13.01	0.00488463998131372\\
14.01	0.0048846412692637\\
15.01	0.00488464258389983\\
16.01	0.00488464392577312\\
17.01	0.00488464529544633\\
18.01	0.00488464669349324\\
19.01	0.00488464812049973\\
20.01	0.00488464957706371\\
21.01	0.00488465106379535\\
22.01	0.00488465258131733\\
23.01	0.00488465413026523\\
24.01	0.00488465571128756\\
25.01	0.00488465732504612\\
26.01	0.00488465897221652\\
27.01	0.00488466065348817\\
28.01	0.00488466236956452\\
29.01	0.00488466412116355\\
30.01	0.00488466590901793\\
31.01	0.0048846677338755\\
32.01	0.00488466959649934\\
33.01	0.00488467149766832\\
34.01	0.00488467343817704\\
35.01	0.00488467541883672\\
36.01	0.00488467744047501\\
37.01	0.00488467950393669\\
38.01	0.00488468161008374\\
39.01	0.00488468375979582\\
40.01	0.00488468595397068\\
41.01	0.00488468819352453\\
42.01	0.00488469047939238\\
43.01	0.00488469281252815\\
44.01	0.00488469519390572\\
45.01	0.00488469762451858\\
46.01	0.00488470010538083\\
47.01	0.00488470263752724\\
48.01	0.00488470522201373\\
49.01	0.00488470785991796\\
50.01	0.00488471055233954\\
51.01	0.00488471330040076\\
52.01	0.00488471610524679\\
53.01	0.00488471896804608\\
54.01	0.00488472188999137\\
55.01	0.00488472487229946\\
56.01	0.00488472791621249\\
57.01	0.00488473102299765\\
58.01	0.00488473419394793\\
59.01	0.0048847374303831\\
60.01	0.00488474073364972\\
61.01	0.00488474410512205\\
62.01	0.00488474754620237\\
63.01	0.00488475105832172\\
64.01	0.00488475464294018\\
65.01	0.00488475830154779\\
66.01	0.00488476203566512\\
67.01	0.00488476584684377\\
68.01	0.00488476973666695\\
69.01	0.00488477370675029\\
70.01	0.00488477775874228\\
71.01	0.0048847818943253\\
72.01	0.00488478611521584\\
73.01	0.00488479042316561\\
74.01	0.00488479481996197\\
75.01	0.00488479930742881\\
76.01	0.00488480388742725\\
77.01	0.00488480856185623\\
78.01	0.00488481333265349\\
79.01	0.00488481820179654\\
80.01	0.00488482317130275\\
81.01	0.00488482824323097\\
82.01	0.00488483341968188\\
83.01	0.004884838702799\\
84.01	0.00488484409476954\\
85.01	0.00488484959782505\\
86.01	0.00488485521424283\\
87.01	0.00488486094634642\\
88.01	0.00488486679650672\\
89.01	0.00488487276714292\\
90.01	0.00488487886072335\\
91.01	0.00488488507976655\\
92.01	0.00488489142684233\\
93.01	0.00488489790457287\\
94.01	0.00488490451563364\\
95.01	0.00488491126275441\\
96.01	0.00488491814872058\\
97.01	0.004884925176374\\
98.01	0.00488493234861456\\
99.01	0.00488493966840066\\
100.01	0.00488494713875114\\
101.01	0.00488495476274613\\
102.01	0.00488496254352806\\
103.01	0.00488497048430354\\
104.01	0.00488497858834375\\
105.01	0.00488498685898661\\
106.01	0.0048849952996378\\
107.01	0.00488500391377188\\
108.01	0.00488501270493392\\
109.01	0.0048850216767411\\
110.01	0.00488503083288364\\
111.01	0.0048850401771266\\
112.01	0.0048850497133114\\
113.01	0.00488505944535733\\
114.01	0.00488506937726288\\
115.01	0.00488507951310754\\
116.01	0.00488508985705377\\
117.01	0.00488510041334787\\
118.01	0.00488511118632219\\
119.01	0.00488512218039647\\
120.01	0.00488513340008019\\
121.01	0.00488514484997393\\
122.01	0.00488515653477133\\
123.01	0.0048851684592606\\
124.01	0.00488518062832694\\
125.01	0.0048851930469542\\
126.01	0.00488520572022699\\
127.01	0.00488521865333234\\
128.01	0.00488523185156209\\
129.01	0.00488524532031505\\
130.01	0.00488525906509854\\
131.01	0.00488527309153144\\
132.01	0.00488528740534548\\
133.01	0.00488530201238821\\
134.01	0.00488531691862487\\
135.01	0.00488533213014113\\
136.01	0.00488534765314502\\
137.01	0.00488536349396942\\
138.01	0.00488537965907505\\
139.01	0.00488539615505265\\
140.01	0.00488541298862519\\
141.01	0.00488543016665109\\
142.01	0.00488544769612662\\
143.01	0.00488546558418873\\
144.01	0.00488548383811775\\
145.01	0.00488550246534004\\
146.01	0.00488552147343151\\
147.01	0.00488554087011993\\
148.01	0.00488556066328823\\
149.01	0.00488558086097736\\
150.01	0.00488560147138963\\
151.01	0.00488562250289179\\
152.01	0.00488564396401828\\
153.01	0.00488566586347448\\
154.01	0.00488568821014031\\
155.01	0.00488571101307322\\
156.01	0.00488573428151214\\
157.01	0.00488575802488071\\
158.01	0.00488578225279108\\
159.01	0.00488580697504766\\
160.01	0.00488583220165079\\
161.01	0.00488585794280036\\
162.01	0.00488588420890027\\
163.01	0.00488591101056195\\
164.01	0.0048859383586087\\
165.01	0.00488596626407946\\
166.01	0.0048859947382336\\
167.01	0.00488602379255464\\
168.01	0.00488605343875502\\
169.01	0.00488608368878034\\
170.01	0.00488611455481398\\
171.01	0.00488614604928185\\
172.01	0.00488617818485682\\
173.01	0.00488621097446391\\
174.01	0.00488624443128476\\
175.01	0.004886278568763\\
176.01	0.00488631340060888\\
177.01	0.00488634894080489\\
178.01	0.00488638520361084\\
179.01	0.00488642220356926\\
180.01	0.00488645995551076\\
181.01	0.00488649847455971\\
182.01	0.0048865377761399\\
183.01	0.00488657787597988\\
184.01	0.00488661879011984\\
185.01	0.00488666053491649\\
186.01	0.00488670312704981\\
187.01	0.00488674658352915\\
188.01	0.00488679092169943\\
189.01	0.00488683615924733\\
190.01	0.00488688231420836\\
191.01	0.00488692940497344\\
192.01	0.00488697745029516\\
193.01	0.00488702646929522\\
194.01	0.00488707648147119\\
195.01	0.00488712750670386\\
196.01	0.00488717956526429\\
197.01	0.00488723267782148\\
198.01	0.00488728686544965\\
199.01	0.00488734214963593\\
200.01	0.00488739855228837\\
201.01	0.00488745609574408\\
202.01	0.00488751480277642\\
203.01	0.00488757469660416\\
204.01	0.00488763580089938\\
205.01	0.00488769813979624\\
206.01	0.00488776173789938\\
207.01	0.00488782662029302\\
208.01	0.00488789281254931\\
209.01	0.00488796034073828\\
210.01	0.00488802923143664\\
211.01	0.00488809951173697\\
212.01	0.00488817120925797\\
213.01	0.00488824435215351\\
214.01	0.00488831896912283\\
215.01	0.00488839508942079\\
216.01	0.00488847274286774\\
217.01	0.00488855195986022\\
218.01	0.00488863277138125\\
219.01	0.00488871520901125\\
220.01	0.00488879930493913\\
221.01	0.00488888509197286\\
222.01	0.00488897260355128\\
223.01	0.00488906187375519\\
224.01	0.00488915293731899\\
225.01	0.00488924582964294\\
226.01	0.00488934058680437\\
227.01	0.00488943724557078\\
228.01	0.0048895358434112\\
229.01	0.00488963641850972\\
230.01	0.00488973900977757\\
231.01	0.00488984365686602\\
232.01	0.00488995040018013\\
233.01	0.00489005928089155\\
234.01	0.0048901703409521\\
235.01	0.00489028362310753\\
236.01	0.00489039917091151\\
237.01	0.00489051702873944\\
238.01	0.00489063724180322\\
239.01	0.00489075985616554\\
240.01	0.00489088491875445\\
241.01	0.00489101247737807\\
242.01	0.00489114258074019\\
243.01	0.0048912752784554\\
244.01	0.00489141062106455\\
245.01	0.00489154866005015\\
246.01	0.00489168944785268\\
247.01	0.00489183303788655\\
248.01	0.00489197948455636\\
249.01	0.00489212884327332\\
250.01	0.00489228117047229\\
251.01	0.00489243652362804\\
252.01	0.00489259496127287\\
253.01	0.00489275654301351\\
254.01	0.00489292132954898\\
255.01	0.00489308938268763\\
256.01	0.0048932607653657\\
257.01	0.00489343554166455\\
258.01	0.00489361377682952\\
259.01	0.00489379553728768\\
260.01	0.00489398089066691\\
261.01	0.00489416990581426\\
262.01	0.00489436265281511\\
263.01	0.00489455920301173\\
264.01	0.00489475962902271\\
265.01	0.00489496400476247\\
266.01	0.00489517240546028\\
267.01	0.00489538490768029\\
268.01	0.00489560158934112\\
269.01	0.00489582252973529\\
270.01	0.00489604780955007\\
271.01	0.00489627751088685\\
272.01	0.00489651171728148\\
273.01	0.00489675051372462\\
274.01	0.00489699398668199\\
275.01	0.00489724222411494\\
276.01	0.00489749531550084\\
277.01	0.00489775335185344\\
278.01	0.00489801642574396\\
279.01	0.00489828463132136\\
280.01	0.00489855806433266\\
281.01	0.00489883682214419\\
282.01	0.0048991210037619\\
283.01	0.00489941070985198\\
284.01	0.00489970604276144\\
285.01	0.00490000710653865\\
286.01	0.00490031400695382\\
287.01	0.00490062685151932\\
288.01	0.00490094574951033\\
289.01	0.00490127081198434\\
290.01	0.00490160215180185\\
291.01	0.00490193988364593\\
292.01	0.0049022841240422\\
293.01	0.00490263499137808\\
294.01	0.00490299260592256\\
295.01	0.00490335708984539\\
296.01	0.00490372856723544\\
297.01	0.00490410716412007\\
298.01	0.00490449300848338\\
299.01	0.00490488623028426\\
300.01	0.00490528696147452\\
301.01	0.00490569533601581\\
302.01	0.00490611148989772\\
303.01	0.00490653556115422\\
304.01	0.00490696768988012\\
305.01	0.00490740801824727\\
306.01	0.00490785669052059\\
307.01	0.00490831385307283\\
308.01	0.00490877965440047\\
309.01	0.00490925424513745\\
310.01	0.00490973777806981\\
311.01	0.00491023040814934\\
312.01	0.00491073229250645\\
313.01	0.00491124359046391\\
314.01	0.00491176446354842\\
315.01	0.00491229507550346\\
316.01	0.00491283559230044\\
317.01	0.00491338618215047\\
318.01	0.00491394701551483\\
319.01	0.00491451826511623\\
320.01	0.00491510010594903\\
321.01	0.00491569271528909\\
322.01	0.0049162962727043\\
323.01	0.0049169109600639\\
324.01	0.00491753696154876\\
325.01	0.0049181744636609\\
326.01	0.00491882365523369\\
327.01	0.00491948472744183\\
328.01	0.00492015787381168\\
329.01	0.00492084329023253\\
330.01	0.00492154117496778\\
331.01	0.00492225172866764\\
332.01	0.00492297515438152\\
333.01	0.00492371165757287\\
334.01	0.00492446144613423\\
335.01	0.00492522473040473\\
336.01	0.00492600172318833\\
337.01	0.00492679263977508\\
338.01	0.0049275976979638\\
339.01	0.00492841711808818\\
340.01	0.00492925112304513\\
341.01	0.0049300999383268\\
342.01	0.00493096379205611\\
343.01	0.00493184291502724\\
344.01	0.00493273754074887\\
345.01	0.00493364790549432\\
346.01	0.00493457424835596\\
347.01	0.0049355168113058\\
348.01	0.00493647583926354\\
349.01	0.00493745158017071\\
350.01	0.00493844428507326\\
351.01	0.0049394542082119\\
352.01	0.004940481607122\\
353.01	0.00494152674274304\\
354.01	0.00494258987953809\\
355.01	0.00494367128562485\\
356.01	0.00494477123291819\\
357.01	0.00494588999728481\\
358.01	0.00494702785871175\\
359.01	0.00494818510148777\\
360.01	0.0049493620143998\\
361.01	0.00495055889094456\\
362.01	0.0049517760295547\\
363.01	0.00495301373384249\\
364.01	0.00495427231285812\\
365.01	0.00495555208136646\\
366.01	0.00495685336013935\\
367.01	0.0049581764762656\\
368.01	0.0049595217634773\\
369.01	0.00496088956249255\\
370.01	0.00496228022137377\\
371.01	0.0049636940959015\\
372.01	0.00496513154996125\\
373.01	0.00496659295594318\\
374.01	0.00496807869515233\\
375.01	0.00496958915822666\\
376.01	0.00497112474556143\\
377.01	0.00497268586773587\\
378.01	0.00497427294593912\\
379.01	0.00497588641239095\\
380.01	0.00497752671075352\\
381.01	0.00497919429652868\\
382.01	0.004980889637435\\
383.01	0.00498261321375904\\
384.01	0.00498436551867423\\
385.01	0.00498614705852136\\
386.01	0.00498795835304229\\
387.01	0.00498979993556056\\
388.01	0.00499167235310256\\
389.01	0.00499357616645112\\
390.01	0.00499551195012558\\
391.01	0.00499748029228411\\
392.01	0.00499948179454097\\
393.01	0.0050015170716996\\
394.01	0.0050035867513961\\
395.01	0.00500569147365945\\
396.01	0.00500783189038718\\
397.01	0.00501000866474974\\
398.01	0.00501222247052978\\
399.01	0.0050144739914159\\
400.01	0.00501676392026873\\
401.01	0.00501909295838479\\
402.01	0.00502146181478867\\
403.01	0.0050238712055853\\
404.01	0.00502632185341148\\
405.01	0.00502881448702619\\
406.01	0.00503134984107985\\
407.01	0.00503392865610204\\
408.01	0.00503655167874308\\
409.01	0.00503921966229595\\
410.01	0.00504193336751546\\
411.01	0.00504469356373378\\
412.01	0.00504750103025557\\
413.01	0.00505035655799117\\
414.01	0.0050532609512678\\
415.01	0.0050562150297354\\
416.01	0.005059219630274\\
417.01	0.00506227560880644\\
418.01	0.0050653838419379\\
419.01	0.00506854522838324\\
420.01	0.00507176069020351\\
421.01	0.00507503117393805\\
422.01	0.00507835765173228\\
423.01	0.00508174112249618\\
424.01	0.00508518261309696\\
425.01	0.00508868317958147\\
426.01	0.00509224390842564\\
427.01	0.0050958659178061\\
428.01	0.00509955035888979\\
429.01	0.00510329841713758\\
430.01	0.00510711131361349\\
431.01	0.00511099030629548\\
432.01	0.00511493669137993\\
433.01	0.00511895180457242\\
434.01	0.00512303702235845\\
435.01	0.00512719376324269\\
436.01	0.0051314234889528\\
437.01	0.00513572770559581\\
438.01	0.00514010796476017\\
439.01	0.00514456586455563\\
440.01	0.00514910305058303\\
441.01	0.00515372121682685\\
442.01	0.00515842210646476\\
443.01	0.00516320751259082\\
444.01	0.00516807927884735\\
445.01	0.00517303929996674\\
446.01	0.00517808952222386\\
447.01	0.0051832319438026\\
448.01	0.00518846861508475\\
449.01	0.00519380163887069\\
450.01	0.00519923317054667\\
451.01	0.00520476541821373\\
452.01	0.0052104006428003\\
453.01	0.00521614115818127\\
454.01	0.00522198933132952\\
455.01	0.00522794758252718\\
456.01	0.0052340183856666\\
457.01	0.00524020426866761\\
458.01	0.00524650781403839\\
459.01	0.00525293165960437\\
460.01	0.00525947849941792\\
461.01	0.0052661510848642\\
462.01	0.00527295222595822\\
463.01	0.00527988479282504\\
464.01	0.00528695171733751\\
465.01	0.00529415599487579\\
466.01	0.00530150068616073\\
467.01	0.00530898891910096\\
468.01	0.00531662389059037\\
469.01	0.00532440886819223\\
470.01	0.00533234719165125\\
471.01	0.00534044227419347\\
472.01	0.00534869760359381\\
473.01	0.00535711674302511\\
474.01	0.00536570333172771\\
475.01	0.00537446108556734\\
476.01	0.00538339379755706\\
477.01	0.0053925053384032\\
478.01	0.00540179965711304\\
479.01	0.00541128078168127\\
480.01	0.00542095281987184\\
481.01	0.0054308199601092\\
482.01	0.00544088647248854\\
483.01	0.00545115670991292\\
484.01	0.00546163510936148\\
485.01	0.00547232619328618\\
486.01	0.00548323457113223\\
487.01	0.00549436494097295\\
488.01	0.00550572209124164\\
489.01	0.00551731090254502\\
490.01	0.00552913634953665\\
491.01	0.00554120350283076\\
492.01	0.00555351753093536\\
493.01	0.00556608370219074\\
494.01	0.00557890738670204\\
495.01	0.00559199405826316\\
496.01	0.00560534929627592\\
497.01	0.00561897878767768\\
498.01	0.0056328883288938\\
499.01	0.00564708382783537\\
500.01	0.00566157130595962\\
501.01	0.00567635690040588\\
502.01	0.00569144686620832\\
503.01	0.00570684757858236\\
504.01	0.00572256553527355\\
505.01	0.00573860735895959\\
506.01	0.00575497979969389\\
507.01	0.00577168973737842\\
508.01	0.00578874418425757\\
509.01	0.00580615028742262\\
510.01	0.00582391533132044\\
511.01	0.00584204674026063\\
512.01	0.00586055208091783\\
513.01	0.00587943906482583\\
514.01	0.00589871555086248\\
515.01	0.00591838954772095\\
516.01	0.00593846921636589\\
517.01	0.00595896287246597\\
518.01	0.00597987898879571\\
519.01	0.00600122619759545\\
520.01	0.00602301329287536\\
521.01	0.00604524923265283\\
522.01	0.0060679431411064\\
523.01	0.00609110431063371\\
524.01	0.0061147422037966\\
525.01	0.00613886645513884\\
526.01	0.00616348687285774\\
527.01	0.00618861344031034\\
528.01	0.0062142563173349\\
529.01	0.00624042584136023\\
530.01	0.00626713252828011\\
531.01	0.00629438707306058\\
532.01	0.00632220035004875\\
533.01	0.00635058341294649\\
534.01	0.00637954749440924\\
535.01	0.00640910400522809\\
536.01	0.00643926453304539\\
537.01	0.00647004084055382\\
538.01	0.00650144486311954\\
539.01	0.00653348870576834\\
540.01	0.00656618463946283\\
541.01	0.00659954509659628\\
542.01	0.00663358266561771\\
543.01	0.00666831008469636\\
544.01	0.00670374023432455\\
545.01	0.00673988612874671\\
546.01	0.00677676090609454\\
547.01	0.00681437781709397\\
548.01	0.00685275021219706\\
549.01	0.00689189152698148\\
550.01	0.00693181526564102\\
551.01	0.00697253498237767\\
552.01	0.0070140642604869\\
553.01	0.00705641668891045\\
554.01	0.00709960583601042\\
555.01	0.00714364522029591\\
556.01	0.00718854827781448\\
557.01	0.00723432832589305\\
558.01	0.00728099852289051\\
559.01	0.00732857182359682\\
560.01	0.007377060929888\\
561.01	0.00742647823621727\\
562.01	0.00747683576949768\\
563.01	0.00752814512290237\\
564.01	0.00758041738308592\\
565.01	0.00763366305030459\\
566.01	0.00768789195089741\\
567.01	0.00774311314157326\\
568.01	0.00779933480494592\\
569.01	0.00785656413576291\\
570.01	0.00791480721729327\\
571.01	0.0079740688873769\\
572.01	0.00803435259370153\\
573.01	0.008095660237966\\
574.01	0.008157992008723\\
575.01	0.00822134620288103\\
576.01	0.00828571903609403\\
577.01	0.00835110444260261\\
578.01	0.00841749386552518\\
579.01	0.00848487603916375\\
580.01	0.00855323676561756\\
581.01	0.00862255868892708\\
582.01	0.0086928210711525\\
583.01	0.00876399957628497\\
584.01	0.00883606606977153\\
585.01	0.00890898844379669\\
586.01	0.00898273048142365\\
587.01	0.00905725177639212\\
588.01	0.00913250772997538\\
589.01	0.0092084496520285\\
590.01	0.00928502500047535\\
591.01	0.00936217780230859\\
592.01	0.00943984931011133\\
593.01	0.0095179789616418\\
594.01	0.00959650572675306\\
595.01	0.00967536994658929\\
596.01	0.00975451579551667\\
597.01	0.00983362572953704\\
598.01	0.00990826330299362\\
599.01	0.00997053306357289\\
599.02	0.00997104257304143\\
599.03	0.00997154904691472\\
599.04	0.00997205245565375\\
599.05	0.00997255276942872\\
599.06	0.00997304995811617\\
599.07	0.00997354399129606\\
599.08	0.00997403483824883\\
599.09	0.00997452246795248\\
599.1	0.00997500684907952\\
599.11	0.00997548794999398\\
599.12	0.00997596573874838\\
599.13	0.0099764401830806\\
599.14	0.0099769112504108\\
599.15	0.00997737890783826\\
599.16	0.00997784312213823\\
599.17	0.0099783038597587\\
599.18	0.00997876108681718\\
599.19	0.00997921476909742\\
599.2	0.00997966487204611\\
599.21	0.00998011136076955\\
599.22	0.00998055420003028\\
599.23	0.00998099335424369\\
599.24	0.00998142878747458\\
599.25	0.00998186046343367\\
599.26	0.00998228834359041\\
599.27	0.00998271238778643\\
599.28	0.0099831325554638\\
599.29	0.00998354880566111\\
599.3	0.00998396109700947\\
599.31	0.00998436938772849\\
599.32	0.00998477363562221\\
599.33	0.00998517379807499\\
599.34	0.00998556983204737\\
599.35	0.00998596169407188\\
599.36	0.00998634934024881\\
599.37	0.00998673272624192\\
599.38	0.00998711180727415\\
599.39	0.00998748653812326\\
599.4	0.00998785687311743\\
599.41	0.00998822276613081\\
599.42	0.00998858417057903\\
599.43	0.00998894103941468\\
599.44	0.00998929332512275\\
599.45	0.00998964097971596\\
599.46	0.00998998395473014\\
599.47	0.00999032220121951\\
599.48	0.0099906556697519\\
599.49	0.00999098431040396\\
599.5	0.00999130807275631\\
599.51	0.00999162690588863\\
599.52	0.00999194075837471\\
599.53	0.00999224957827746\\
599.54	0.00999255331314387\\
599.55	0.00999285190999987\\
599.56	0.00999314531534524\\
599.57	0.00999343347514839\\
599.58	0.00999371633484107\\
599.59	0.00999399383931314\\
599.6	0.00999426593290715\\
599.61	0.009994532559413\\
599.62	0.0099947936620624\\
599.63	0.00999504918352342\\
599.64	0.00999529906589491\\
599.65	0.00999554325070086\\
599.66	0.00999578167888471\\
599.67	0.00999601429080366\\
599.68	0.00999624102622283\\
599.69	0.00999646182430947\\
599.7	0.00999667662362697\\
599.71	0.00999688536212897\\
599.72	0.00999708797715331\\
599.73	0.00999728440541596\\
599.74	0.00999747458300482\\
599.75	0.0099976584453736\\
599.76	0.00999783592733551\\
599.77	0.00999800696305692\\
599.78	0.00999817148605102\\
599.79	0.00999832942917133\\
599.8	0.00999848072460517\\
599.81	0.00999862530386713\\
599.82	0.00999876309779239\\
599.83	0.00999889403653003\\
599.84	0.00999901804953622\\
599.85	0.00999913506556741\\
599.86	0.00999924501267341\\
599.87	0.00999934781819042\\
599.88	0.00999944340873394\\
599.89	0.0099995317101917\\
599.9	0.00999961264771648\\
599.91	0.00999968614571878\\
599.92	0.00999975212785957\\
599.93	0.00999981051704285\\
599.94	0.00999986123540817\\
599.95	0.00999990420432308\\
599.96	0.00999993934437554\\
599.97	0.00999996657536618\\
599.98	0.00999998581630055\\
599.99	0.00999999698538124\\
600	0.01\\
};
\addplot [color=black!60!mycolor21,solid,forget plot]
  table[row sep=crcr]{%
0.01	0.00491031510145137\\
1.01	0.00491031607159087\\
2.01	0.00491031706172967\\
3.01	0.00491031807227832\\
4.01	0.00491031910365577\\
5.01	0.00491032015628892\\
6.01	0.004910321230614\\
7.01	0.004910322327076\\
8.01	0.00491032344612882\\
9.01	0.00491032458823537\\
10.01	0.00491032575386857\\
11.01	0.00491032694351048\\
12.01	0.0049103281576533\\
13.01	0.00491032939679925\\
14.01	0.00491033066146054\\
15.01	0.00491033195215984\\
16.01	0.00491033326943088\\
17.01	0.00491033461381757\\
18.01	0.0049103359858754\\
19.01	0.00491033738617094\\
20.01	0.00491033881528236\\
21.01	0.00491034027379961\\
22.01	0.00491034176232466\\
23.01	0.0049103432814715\\
24.01	0.00491034483186684\\
25.01	0.00491034641415014\\
26.01	0.00491034802897369\\
27.01	0.00491034967700305\\
28.01	0.00491035135891746\\
29.01	0.00491035307540957\\
30.01	0.0049103548271868\\
31.01	0.00491035661497022\\
32.01	0.00491035843949589\\
33.01	0.00491036030151454\\
34.01	0.00491036220179267\\
35.01	0.00491036414111198\\
36.01	0.00491036612026965\\
37.01	0.00491036814007967\\
38.01	0.00491037020137211\\
39.01	0.00491037230499405\\
40.01	0.0049103744518097\\
41.01	0.0049103766427008\\
42.01	0.00491037887856721\\
43.01	0.00491038116032647\\
44.01	0.0049103834889149\\
45.01	0.00491038586528794\\
46.01	0.00491038829042034\\
47.01	0.00491039076530628\\
48.01	0.00491039329096052\\
49.01	0.00491039586841802\\
50.01	0.00491039849873481\\
51.01	0.00491040118298822\\
52.01	0.00491040392227707\\
53.01	0.00491040671772291\\
54.01	0.00491040957046965\\
55.01	0.00491041248168456\\
56.01	0.00491041545255808\\
57.01	0.0049104184843049\\
58.01	0.00491042157816446\\
59.01	0.00491042473540079\\
60.01	0.00491042795730384\\
61.01	0.00491043124518914\\
62.01	0.00491043460039894\\
63.01	0.00491043802430274\\
64.01	0.00491044151829753\\
65.01	0.00491044508380858\\
66.01	0.00491044872228966\\
67.01	0.00491045243522376\\
68.01	0.00491045622412378\\
69.01	0.00491046009053339\\
70.01	0.0049104640360273\\
71.01	0.00491046806221143\\
72.01	0.00491047217072429\\
73.01	0.0049104763632376\\
74.01	0.0049104806414566\\
75.01	0.0049104850071206\\
76.01	0.00491048946200376\\
77.01	0.00491049400791638\\
78.01	0.00491049864670459\\
79.01	0.00491050338025184\\
80.01	0.00491050821047984\\
81.01	0.00491051313934815\\
82.01	0.00491051816885599\\
83.01	0.00491052330104297\\
84.01	0.00491052853798876\\
85.01	0.00491053388181567\\
86.01	0.00491053933468807\\
87.01	0.00491054489881396\\
88.01	0.00491055057644535\\
89.01	0.00491055636987944\\
90.01	0.00491056228145914\\
91.01	0.00491056831357481\\
92.01	0.00491057446866447\\
93.01	0.00491058074921474\\
94.01	0.00491058715776202\\
95.01	0.00491059369689356\\
96.01	0.00491060036924834\\
97.01	0.0049106071775179\\
98.01	0.00491061412444754\\
99.01	0.00491062121283775\\
100.01	0.00491062844554466\\
101.01	0.00491063582548121\\
102.01	0.00491064335561875\\
103.01	0.0049106510389878\\
104.01	0.00491065887867959\\
105.01	0.00491066687784653\\
106.01	0.00491067503970385\\
107.01	0.00491068336753123\\
108.01	0.00491069186467374\\
109.01	0.00491070053454246\\
110.01	0.00491070938061638\\
111.01	0.00491071840644464\\
112.01	0.00491072761564595\\
113.01	0.0049107370119115\\
114.01	0.00491074659900581\\
115.01	0.00491075638076785\\
116.01	0.00491076636111287\\
117.01	0.00491077654403418\\
118.01	0.0049107869336045\\
119.01	0.00491079753397677\\
120.01	0.0049108083493869\\
121.01	0.0049108193841543\\
122.01	0.00491083064268422\\
123.01	0.00491084212946899\\
124.01	0.00491085384909024\\
125.01	0.00491086580622008\\
126.01	0.00491087800562304\\
127.01	0.00491089045215781\\
128.01	0.00491090315077972\\
129.01	0.00491091610654132\\
130.01	0.00491092932459558\\
131.01	0.00491094281019688\\
132.01	0.00491095656870355\\
133.01	0.00491097060557918\\
134.01	0.00491098492639591\\
135.01	0.0049109995368353\\
136.01	0.00491101444269074\\
137.01	0.0049110296498705\\
138.01	0.00491104516439833\\
139.01	0.00491106099241695\\
140.01	0.00491107714018996\\
141.01	0.00491109361410449\\
142.01	0.00491111042067266\\
143.01	0.00491112756653495\\
144.01	0.00491114505846163\\
145.01	0.00491116290335708\\
146.01	0.00491118110826022\\
147.01	0.00491119968034831\\
148.01	0.00491121862693927\\
149.01	0.00491123795549456\\
150.01	0.00491125767362191\\
151.01	0.00491127778907771\\
152.01	0.00491129830977041\\
153.01	0.00491131924376292\\
154.01	0.00491134059927609\\
155.01	0.00491136238469103\\
156.01	0.00491138460855275\\
157.01	0.00491140727957315\\
158.01	0.00491143040663408\\
159.01	0.00491145399879039\\
160.01	0.00491147806527351\\
161.01	0.00491150261549497\\
162.01	0.00491152765904926\\
163.01	0.00491155320571765\\
164.01	0.00491157926547167\\
165.01	0.00491160584847666\\
166.01	0.00491163296509542\\
167.01	0.00491166062589185\\
168.01	0.00491168884163526\\
169.01	0.0049117176233032\\
170.01	0.00491174698208674\\
171.01	0.0049117769293931\\
172.01	0.00491180747685066\\
173.01	0.00491183863631256\\
174.01	0.00491187041986135\\
175.01	0.00491190283981335\\
176.01	0.00491193590872194\\
177.01	0.00491196963938337\\
178.01	0.00491200404484085\\
179.01	0.00491203913838839\\
180.01	0.00491207493357659\\
181.01	0.00491211144421663\\
182.01	0.00491214868438547\\
183.01	0.0049121866684308\\
184.01	0.00491222541097569\\
185.01	0.00491226492692406\\
186.01	0.0049123052314658\\
187.01	0.00491234634008211\\
188.01	0.00491238826855028\\
189.01	0.00491243103295011\\
190.01	0.00491247464966887\\
191.01	0.0049125191354068\\
192.01	0.00491256450718325\\
193.01	0.00491261078234234\\
194.01	0.00491265797855914\\
195.01	0.00491270611384517\\
196.01	0.00491275520655504\\
197.01	0.00491280527539256\\
198.01	0.00491285633941707\\
199.01	0.00491290841804976\\
200.01	0.00491296153108048\\
201.01	0.00491301569867408\\
202.01	0.0049130709413778\\
203.01	0.00491312728012738\\
204.01	0.00491318473625466\\
205.01	0.00491324333149432\\
206.01	0.00491330308799115\\
207.01	0.00491336402830791\\
208.01	0.00491342617543236\\
209.01	0.00491348955278451\\
210.01	0.00491355418422532\\
211.01	0.00491362009406349\\
212.01	0.00491368730706396\\
213.01	0.0049137558484561\\
214.01	0.0049138257439415\\
215.01	0.0049138970197028\\
216.01	0.00491396970241173\\
217.01	0.00491404381923788\\
218.01	0.00491411939785774\\
219.01	0.00491419646646279\\
220.01	0.00491427505376946\\
221.01	0.00491435518902755\\
222.01	0.00491443690202978\\
223.01	0.00491452022312133\\
224.01	0.00491460518320924\\
225.01	0.00491469181377202\\
226.01	0.00491478014686977\\
227.01	0.00491487021515369\\
228.01	0.00491496205187711\\
229.01	0.00491505569090489\\
230.01	0.00491515116672395\\
231.01	0.00491524851445427\\
232.01	0.00491534776985988\\
233.01	0.00491544896935836\\
234.01	0.00491555215003323\\
235.01	0.00491565734964514\\
236.01	0.00491576460664206\\
237.01	0.00491587396017187\\
238.01	0.0049159854500935\\
239.01	0.00491609911698887\\
240.01	0.00491621500217457\\
241.01	0.00491633314771489\\
242.01	0.00491645359643308\\
243.01	0.00491657639192389\\
244.01	0.00491670157856686\\
245.01	0.00491682920153858\\
246.01	0.00491695930682534\\
247.01	0.00491709194123705\\
248.01	0.0049172271524198\\
249.01	0.00491736498886938\\
250.01	0.00491750549994455\\
251.01	0.00491764873588155\\
252.01	0.00491779474780762\\
253.01	0.00491794358775449\\
254.01	0.00491809530867326\\
255.01	0.00491824996444892\\
256.01	0.00491840760991388\\
257.01	0.00491856830086376\\
258.01	0.00491873209407118\\
259.01	0.004918899047302\\
260.01	0.00491906921932911\\
261.01	0.00491924266994892\\
262.01	0.00491941945999599\\
263.01	0.00491959965135927\\
264.01	0.00491978330699741\\
265.01	0.00491997049095511\\
266.01	0.00492016126837869\\
267.01	0.00492035570553284\\
268.01	0.00492055386981688\\
269.01	0.00492075582978119\\
270.01	0.00492096165514341\\
271.01	0.00492117141680656\\
272.01	0.00492138518687485\\
273.01	0.00492160303867145\\
274.01	0.00492182504675528\\
275.01	0.00492205128693896\\
276.01	0.00492228183630566\\
277.01	0.00492251677322811\\
278.01	0.00492275617738483\\
279.01	0.00492300012977911\\
280.01	0.00492324871275682\\
281.01	0.00492350201002506\\
282.01	0.00492376010667026\\
283.01	0.00492402308917672\\
284.01	0.00492429104544514\\
285.01	0.00492456406481206\\
286.01	0.00492484223806858\\
287.01	0.00492512565747927\\
288.01	0.00492541441680122\\
289.01	0.00492570861130454\\
290.01	0.00492600833779094\\
291.01	0.00492631369461391\\
292.01	0.00492662478169853\\
293.01	0.00492694170056152\\
294.01	0.0049272645543319\\
295.01	0.00492759344777097\\
296.01	0.00492792848729334\\
297.01	0.00492826978098768\\
298.01	0.0049286174386374\\
299.01	0.00492897157174266\\
300.01	0.00492933229354131\\
301.01	0.00492969971903126\\
302.01	0.00493007396499201\\
303.01	0.00493045515000734\\
304.01	0.00493084339448832\\
305.01	0.00493123882069521\\
306.01	0.0049316415527626\\
307.01	0.00493205171672173\\
308.01	0.00493246944052533\\
309.01	0.00493289485407258\\
310.01	0.00493332808923319\\
311.01	0.00493376927987479\\
312.01	0.00493421856188775\\
313.01	0.00493467607321296\\
314.01	0.00493514195386935\\
315.01	0.00493561634598177\\
316.01	0.00493609939381048\\
317.01	0.00493659124378132\\
318.01	0.00493709204451662\\
319.01	0.0049376019468666\\
320.01	0.00493812110394312\\
321.01	0.0049386496711529\\
322.01	0.00493918780623445\\
323.01	0.00493973566929337\\
324.01	0.00494029342284161\\
325.01	0.00494086123183635\\
326.01	0.00494143926372242\\
327.01	0.00494202768847507\\
328.01	0.00494262667864553\\
329.01	0.00494323640940805\\
330.01	0.00494385705860945\\
331.01	0.00494448880682036\\
332.01	0.00494513183739081\\
333.01	0.00494578633650644\\
334.01	0.00494645249324882\\
335.01	0.00494713049965773\\
336.01	0.00494782055079787\\
337.01	0.00494852284482808\\
338.01	0.0049492375830745\\
339.01	0.00494996497010755\\
340.01	0.00495070521382246\\
341.01	0.00495145852552478\\
342.01	0.00495222512001933\\
343.01	0.00495300521570466\\
344.01	0.00495379903467205\\
345.01	0.00495460680280871\\
346.01	0.0049554287499079\\
347.01	0.00495626510978264\\
348.01	0.00495711612038646\\
349.01	0.00495798202393943\\
350.01	0.00495886306705925\\
351.01	0.00495975950090055\\
352.01	0.00496067158129826\\
353.01	0.00496159956891859\\
354.01	0.00496254372941649\\
355.01	0.00496350433359866\\
356.01	0.00496448165759408\\
357.01	0.00496547598303038\\
358.01	0.00496648759721674\\
359.01	0.0049675167933336\\
360.01	0.00496856387062783\\
361.01	0.00496962913461405\\
362.01	0.00497071289728146\\
363.01	0.00497181547730605\\
364.01	0.00497293720026756\\
365.01	0.00497407839887118\\
366.01	0.0049752394131716\\
367.01	0.00497642059080092\\
368.01	0.0049776222871997\\
369.01	0.00497884486584635\\
370.01	0.00498008869849034\\
371.01	0.00498135416538278\\
372.01	0.00498264165550477\\
373.01	0.00498395156679385\\
374.01	0.00498528430636566\\
375.01	0.00498664029072932\\
376.01	0.00498801994599571\\
377.01	0.0049894237080782\\
378.01	0.00499085202288227\\
379.01	0.00499230534648409\\
380.01	0.00499378414529664\\
381.01	0.00499528889622119\\
382.01	0.00499682008678526\\
383.01	0.00499837821526289\\
384.01	0.00499996379077947\\
385.01	0.00500157733340012\\
386.01	0.00500321937420012\\
387.01	0.00500489045532101\\
388.01	0.00500659113001053\\
389.01	0.00500832196264892\\
390.01	0.00501008352876542\\
391.01	0.0050118764150463\\
392.01	0.00501370121933842\\
393.01	0.00501555855065305\\
394.01	0.00501744902917711\\
395.01	0.00501937328629451\\
396.01	0.00502133196462709\\
397.01	0.00502332571810254\\
398.01	0.00502535521205555\\
399.01	0.00502742112337201\\
400.01	0.00502952414068444\\
401.01	0.00503166496462518\\
402.01	0.00503384430814549\\
403.01	0.00503606289690747\\
404.01	0.00503832146975167\\
405.01	0.00504062077924375\\
406.01	0.00504296159230044\\
407.01	0.00504534469089246\\
408.01	0.00504777087281655\\
409.01	0.00505024095252996\\
410.01	0.00505275576203379\\
411.01	0.00505531615179194\\
412.01	0.00505792299166824\\
413.01	0.00506057717186573\\
414.01	0.0050632796038543\\
415.01	0.00506603122127117\\
416.01	0.00506883298078781\\
417.01	0.00507168586294349\\
418.01	0.00507459087294607\\
419.01	0.00507754904145216\\
420.01	0.0050805614253398\\
421.01	0.0050836291084804\\
422.01	0.00508675320251795\\
423.01	0.00508993484765241\\
424.01	0.00509317521342523\\
425.01	0.00509647549950665\\
426.01	0.0050998369364815\\
427.01	0.00510326078663176\\
428.01	0.00510674834471328\\
429.01	0.00511030093872443\\
430.01	0.00511391993066553\\
431.01	0.00511760671728603\\
432.01	0.00512136273081784\\
433.01	0.00512518943969296\\
434.01	0.00512908834924385\\
435.01	0.00513306100238582\\
436.01	0.00513710898028007\\
437.01	0.00514123390297673\\
438.01	0.00514543743003947\\
439.01	0.00514972126114947\\
440.01	0.00515408713669316\\
441.01	0.00515853683833289\\
442.01	0.00516307218956499\\
443.01	0.00516769505626638\\
444.01	0.00517240734723573\\
445.01	0.00517721101473299\\
446.01	0.00518210805502062\\
447.01	0.00518710050891603\\
448.01	0.00519219046235775\\
449.01	0.00519738004699438\\
450.01	0.0052026714408009\\
451.01	0.00520806686873124\\
452.01	0.00521356860341134\\
453.01	0.00521917896587916\\
454.01	0.00522490032637621\\
455.01	0.00523073510519395\\
456.01	0.00523668577357699\\
457.01	0.00524275485468374\\
458.01	0.00524894492460183\\
459.01	0.00525525861341437\\
460.01	0.005261698606313\\
461.01	0.00526826764474517\\
462.01	0.00527496852759134\\
463.01	0.00528180411235646\\
464.01	0.00528877731636648\\
465.01	0.00529589111795879\\
466.01	0.00530314855765453\\
467.01	0.00531055273930738\\
468.01	0.00531810683122312\\
469.01	0.00532581406724998\\
470.01	0.00533367774784416\\
471.01	0.00534170124111938\\
472.01	0.00534988798389208\\
473.01	0.00535824148273772\\
474.01	0.0053667653150736\\
475.01	0.00537546313027987\\
476.01	0.00538433865087064\\
477.01	0.00539339567371967\\
478.01	0.00540263807134377\\
479.01	0.0054120697932465\\
480.01	0.00542169486732282\\
481.01	0.00543151740132276\\
482.01	0.00544154158437429\\
483.01	0.00545177168856248\\
484.01	0.00546221207056212\\
485.01	0.00547286717331909\\
486.01	0.00548374152777777\\
487.01	0.00549483975464801\\
488.01	0.00550616656620974\\
489.01	0.00551772676815016\\
490.01	0.00552952526143159\\
491.01	0.00554156704418657\\
492.01	0.00555385721364253\\
493.01	0.00556640096807356\\
494.01	0.00557920360878396\\
495.01	0.00559227054212341\\
496.01	0.00560560728153912\\
497.01	0.00561921944966617\\
498.01	0.00563311278045858\\
499.01	0.00564729312136132\\
500.01	0.00566176643552345\\
501.01	0.00567653880404938\\
502.01	0.0056916164282855\\
503.01	0.00570700563213769\\
504.01	0.0057227128644179\\
505.01	0.00573874470121358\\
506.01	0.00575510784827761\\
507.01	0.00577180914343571\\
508.01	0.00578885555900677\\
509.01	0.00580625420423381\\
510.01	0.00582401232772269\\
511.01	0.0058421373198836\\
512.01	0.00586063671537361\\
513.01	0.00587951819553455\\
514.01	0.00589878959082177\\
515.01	0.00591845888321874\\
516.01	0.00593853420862921\\
517.01	0.00595902385924149\\
518.01	0.0059799362858544\\
519.01	0.00600128010015649\\
520.01	0.00602306407694934\\
521.01	0.00604529715630196\\
522.01	0.00606798844562566\\
523.01	0.00609114722165505\\
524.01	0.00611478293232257\\
525.01	0.00613890519850736\\
526.01	0.00616352381564304\\
527.01	0.00618864875516505\\
528.01	0.00621429016577262\\
529.01	0.00624045837448448\\
530.01	0.00626716388746024\\
531.01	0.00629441739055772\\
532.01	0.00632222974959469\\
533.01	0.00635061201027849\\
534.01	0.00637957539776493\\
535.01	0.00640913131580215\\
536.01	0.00643929134541321\\
537.01	0.00647006724306409\\
538.01	0.00650147093825934\\
539.01	0.00653351453050154\\
540.01	0.00656621028554653\\
541.01	0.00659957063087552\\
542.01	0.00663360815030105\\
543.01	0.00666833557761387\\
544.01	0.00670376578917004\\
545.01	0.00673991179530625\\
546.01	0.00677678673046123\\
547.01	0.00681440384187\\
548.01	0.00685277647668575\\
549.01	0.00689191806736764\\
550.01	0.00693184211516124\\
551.01	0.00697256217148033\\
552.01	0.00701409181698227\\
553.01	0.00705644463811021\\
554.01	0.00709963420085635\\
555.01	0.00714367402147857\\
556.01	0.00718857753387962\\
557.01	0.00723435805333637\\
558.01	0.00728102873623945\\
559.01	0.00732860253547936\\
560.01	0.0073770921510864\\
561.01	0.00742650997570748\\
562.01	0.0074768680344723\\
563.01	0.00752817791877722\\
564.01	0.00758045071348906\\
565.01	0.00763369691704793\\
566.01	0.00768792635392892\\
567.01	0.00774314807891066\\
568.01	0.00779937027259147\\
569.01	0.0078566001275986\\
570.01	0.00791484372495724\\
571.01	0.00797410590012169\\
572.01	0.00803439009823456\\
573.01	0.00809569821827383\\
574.01	0.00815803044588143\\
575.01	0.00822138507485295\\
576.01	0.0082857583175189\\
577.01	0.00835114410458111\\
578.01	0.00841753387540446\\
579.01	0.00848491636033002\\
580.01	0.00855327735730464\\
581.01	0.00862259950605086\\
582.01	0.00869286206418521\\
583.01	0.0087640406911855\\
584.01	0.00883610724799195\\
585.01	0.00890902962239154\\
586.01	0.00898277159329311\\
587.01	0.00905729275069911\\
588.01	0.00913254849278597\\
589.01	0.00920849012723675\\
590.01	0.00928506511108912\\
591.01	0.0093622174721911\\
592.01	0.0094398884662976\\
593.01	0.00951801753737781\\
594.01	0.00959654366544385\\
595.01	0.0096754072068902\\
596.01	0.00975455235786353\\
597.01	0.00983363199586728\\
598.01	0.00990826330299362\\
599.01	0.00997053306357289\\
599.02	0.00997104257304143\\
599.03	0.00997154904691472\\
599.04	0.00997205245565375\\
599.05	0.00997255276942872\\
599.06	0.00997304995811617\\
599.07	0.00997354399129606\\
599.08	0.00997403483824883\\
599.09	0.00997452246795248\\
599.1	0.00997500684907951\\
599.11	0.00997548794999398\\
599.12	0.00997596573874838\\
599.13	0.0099764401830806\\
599.14	0.0099769112504108\\
599.15	0.00997737890783826\\
599.16	0.00997784312213823\\
599.17	0.0099783038597587\\
599.18	0.00997876108681718\\
599.19	0.00997921476909742\\
599.2	0.00997966487204611\\
599.21	0.00998011136076955\\
599.22	0.00998055420003028\\
599.23	0.00998099335424369\\
599.24	0.00998142878747458\\
599.25	0.00998186046343367\\
599.26	0.00998228834359041\\
599.27	0.00998271238778643\\
599.28	0.00998313255546381\\
599.29	0.00998354880566111\\
599.3	0.00998396109700947\\
599.31	0.00998436938772849\\
599.32	0.00998477363562221\\
599.33	0.00998517379807499\\
599.34	0.00998556983204737\\
599.35	0.00998596169407188\\
599.36	0.00998634934024881\\
599.37	0.00998673272624192\\
599.38	0.00998711180727415\\
599.39	0.00998748653812326\\
599.4	0.00998785687311743\\
599.41	0.00998822276613081\\
599.42	0.00998858417057903\\
599.43	0.00998894103941468\\
599.44	0.00998929332512275\\
599.45	0.00998964097971596\\
599.46	0.00998998395473014\\
599.47	0.00999032220121951\\
599.48	0.0099906556697519\\
599.49	0.00999098431040396\\
599.5	0.00999130807275631\\
599.51	0.00999162690588862\\
599.52	0.00999194075837471\\
599.53	0.00999224957827746\\
599.54	0.00999255331314386\\
599.55	0.00999285190999987\\
599.56	0.00999314531534524\\
599.57	0.00999343347514839\\
599.58	0.00999371633484107\\
599.59	0.00999399383931314\\
599.6	0.00999426593290715\\
599.61	0.009994532559413\\
599.62	0.0099947936620624\\
599.63	0.00999504918352342\\
599.64	0.00999529906589491\\
599.65	0.00999554325070086\\
599.66	0.00999578167888471\\
599.67	0.00999601429080366\\
599.68	0.00999624102622284\\
599.69	0.00999646182430947\\
599.7	0.00999667662362697\\
599.71	0.00999688536212897\\
599.72	0.00999708797715332\\
599.73	0.00999728440541595\\
599.74	0.00999747458300482\\
599.75	0.0099976584453736\\
599.76	0.00999783592733551\\
599.77	0.00999800696305692\\
599.78	0.00999817148605102\\
599.79	0.00999832942917133\\
599.8	0.00999848072460517\\
599.81	0.00999862530386713\\
599.82	0.00999876309779239\\
599.83	0.00999889403653003\\
599.84	0.00999901804953622\\
599.85	0.00999913506556741\\
599.86	0.00999924501267341\\
599.87	0.00999934781819042\\
599.88	0.00999944340873394\\
599.89	0.0099995317101917\\
599.9	0.00999961264771648\\
599.91	0.00999968614571878\\
599.92	0.00999975212785957\\
599.93	0.00999981051704285\\
599.94	0.00999986123540817\\
599.95	0.00999990420432308\\
599.96	0.00999993934437554\\
599.97	0.00999996657536618\\
599.98	0.00999998581630055\\
599.99	0.00999999698538124\\
600	0.01\\
};
\addplot [color=black!80!mycolor21,solid,forget plot]
  table[row sep=crcr]{%
0.01	0.00492490667902437\\
1.01	0.00492490763710551\\
2.01	0.00492490861484399\\
3.01	0.00492490961264044\\
4.01	0.00492491063090369\\
5.01	0.00492491167005141\\
6.01	0.00492491273050893\\
7.01	0.00492491381271079\\
8.01	0.0049249149170997\\
9.01	0.00492491604412815\\
10.01	0.00492491719425703\\
11.01	0.004924918367957\\
12.01	0.00492491956570795\\
13.01	0.00492492078799942\\
14.01	0.00492492203533126\\
15.01	0.00492492330821293\\
16.01	0.00492492460716443\\
17.01	0.00492492593271618\\
18.01	0.00492492728540914\\
19.01	0.00492492866579558\\
20.01	0.00492493007443867\\
21.01	0.00492493151191287\\
22.01	0.00492493297880414\\
23.01	0.00492493447571052\\
24.01	0.00492493600324184\\
25.01	0.0049249375620205\\
26.01	0.00492493915268086\\
27.01	0.004924940775871\\
28.01	0.00492494243225092\\
29.01	0.00492494412249517\\
30.01	0.00492494584729053\\
31.01	0.00492494760733822\\
32.01	0.00492494940335392\\
33.01	0.00492495123606726\\
34.01	0.00492495310622224\\
35.01	0.00492495501457851\\
36.01	0.0049249569619105\\
37.01	0.00492495894900827\\
38.01	0.00492496097667808\\
39.01	0.00492496304574183\\
40.01	0.00492496515703827\\
41.01	0.00492496731142281\\
42.01	0.00492496950976787\\
43.01	0.00492497175296369\\
44.01	0.00492497404191825\\
45.01	0.00492497637755762\\
46.01	0.00492497876082659\\
47.01	0.00492498119268858\\
48.01	0.00492498367412647\\
49.01	0.00492498620614293\\
50.01	0.00492498878976041\\
51.01	0.0049249914260221\\
52.01	0.00492499411599209\\
53.01	0.0049249968607552\\
54.01	0.00492499966141868\\
55.01	0.00492500251911115\\
56.01	0.00492500543498428\\
57.01	0.00492500841021268\\
58.01	0.00492501144599419\\
59.01	0.00492501454355068\\
60.01	0.00492501770412851\\
61.01	0.00492502092899852\\
62.01	0.00492502421945762\\
63.01	0.00492502757682772\\
64.01	0.0049250310024575\\
65.01	0.00492503449772236\\
66.01	0.00492503806402525\\
67.01	0.00492504170279701\\
68.01	0.00492504541549699\\
69.01	0.0049250492036132\\
70.01	0.00492505306866356\\
71.01	0.00492505701219631\\
72.01	0.00492506103579035\\
73.01	0.00492506514105567\\
74.01	0.00492506932963435\\
75.01	0.00492507360320107\\
76.01	0.00492507796346411\\
77.01	0.0049250824121653\\
78.01	0.00492508695108115\\
79.01	0.00492509158202337\\
80.01	0.00492509630683907\\
81.01	0.00492510112741271\\
82.01	0.00492510604566573\\
83.01	0.00492511106355731\\
84.01	0.00492511618308633\\
85.01	0.00492512140629026\\
86.01	0.00492512673524723\\
87.01	0.00492513217207638\\
88.01	0.0049251377189389\\
89.01	0.00492514337803851\\
90.01	0.00492514915162243\\
91.01	0.00492515504198224\\
92.01	0.00492516105145491\\
93.01	0.0049251671824235\\
94.01	0.00492517343731774\\
95.01	0.00492517981861551\\
96.01	0.00492518632884326\\
97.01	0.00492519297057759\\
98.01	0.00492519974644573\\
99.01	0.00492520665912602\\
100.01	0.00492521371135005\\
101.01	0.00492522090590311\\
102.01	0.00492522824562525\\
103.01	0.004925235733412\\
104.01	0.004925243372216\\
105.01	0.00492525116504766\\
106.01	0.00492525911497682\\
107.01	0.00492526722513315\\
108.01	0.00492527549870739\\
109.01	0.00492528393895361\\
110.01	0.00492529254918914\\
111.01	0.00492530133279617\\
112.01	0.00492531029322338\\
113.01	0.0049253194339865\\
114.01	0.00492532875867024\\
115.01	0.00492533827092964\\
116.01	0.00492534797449087\\
117.01	0.00492535787315251\\
118.01	0.00492536797078758\\
119.01	0.00492537827134487\\
120.01	0.00492538877884963\\
121.01	0.00492539949740589\\
122.01	0.00492541043119779\\
123.01	0.00492542158449069\\
124.01	0.00492543296163298\\
125.01	0.00492544456705721\\
126.01	0.00492545640528305\\
127.01	0.00492546848091738\\
128.01	0.00492548079865618\\
129.01	0.00492549336328715\\
130.01	0.00492550617969062\\
131.01	0.00492551925284163\\
132.01	0.00492553258781181\\
133.01	0.00492554618977081\\
134.01	0.00492556006398823\\
135.01	0.00492557421583584\\
136.01	0.00492558865078938\\
137.01	0.00492560337442985\\
138.01	0.00492561839244698\\
139.01	0.00492563371063967\\
140.01	0.00492564933491876\\
141.01	0.00492566527130894\\
142.01	0.0049256815259516\\
143.01	0.00492569810510612\\
144.01	0.00492571501515268\\
145.01	0.00492573226259324\\
146.01	0.00492574985405583\\
147.01	0.00492576779629533\\
148.01	0.00492578609619646\\
149.01	0.00492580476077628\\
150.01	0.00492582379718604\\
151.01	0.00492584321271464\\
152.01	0.0049258630147902\\
153.01	0.00492588321098314\\
154.01	0.00492590380900829\\
155.01	0.00492592481672902\\
156.01	0.00492594624215847\\
157.01	0.00492596809346252\\
158.01	0.00492599037896313\\
159.01	0.00492601310714125\\
160.01	0.0049260362866391\\
161.01	0.00492605992626372\\
162.01	0.00492608403498956\\
163.01	0.00492610862196233\\
164.01	0.00492613369650069\\
165.01	0.004926159268101\\
166.01	0.00492618534643985\\
167.01	0.00492621194137713\\
168.01	0.00492623906295953\\
169.01	0.00492626672142471\\
170.01	0.00492629492720278\\
171.01	0.00492632369092266\\
172.01	0.00492635302341332\\
173.01	0.00492638293570827\\
174.01	0.0049264134390494\\
175.01	0.00492644454489044\\
176.01	0.00492647626490114\\
177.01	0.0049265086109711\\
178.01	0.00492654159521297\\
179.01	0.00492657522996773\\
180.01	0.00492660952780785\\
181.01	0.0049266445015419\\
182.01	0.00492668016421846\\
183.01	0.00492671652913104\\
184.01	0.00492675360982156\\
185.01	0.00492679142008565\\
186.01	0.00492682997397659\\
187.01	0.00492686928581007\\
188.01	0.00492690937016915\\
189.01	0.00492695024190834\\
190.01	0.00492699191615945\\
191.01	0.0049270344083352\\
192.01	0.00492707773413564\\
193.01	0.00492712190955205\\
194.01	0.00492716695087228\\
195.01	0.00492721287468698\\
196.01	0.00492725969789357\\
197.01	0.00492730743770259\\
198.01	0.00492735611164244\\
199.01	0.00492740573756573\\
200.01	0.00492745633365438\\
201.01	0.00492750791842581\\
202.01	0.00492756051073809\\
203.01	0.00492761412979656\\
204.01	0.00492766879515991\\
205.01	0.00492772452674589\\
206.01	0.00492778134483804\\
207.01	0.00492783927009083\\
208.01	0.00492789832353788\\
209.01	0.00492795852659678\\
210.01	0.00492801990107683\\
211.01	0.00492808246918577\\
212.01	0.00492814625353585\\
213.01	0.00492821127715112\\
214.01	0.0049282775634748\\
215.01	0.00492834513637611\\
216.01	0.00492841402015756\\
217.01	0.00492848423956252\\
218.01	0.00492855581978241\\
219.01	0.00492862878646477\\
220.01	0.00492870316571992\\
221.01	0.00492877898413019\\
222.01	0.00492885626875733\\
223.01	0.00492893504715017\\
224.01	0.00492901534735302\\
225.01	0.00492909719791415\\
226.01	0.00492918062789444\\
227.01	0.00492926566687554\\
228.01	0.00492935234496852\\
229.01	0.00492944069282267\\
230.01	0.00492953074163544\\
231.01	0.00492962252315998\\
232.01	0.00492971606971512\\
233.01	0.00492981141419513\\
234.01	0.00492990859007864\\
235.01	0.00493000763143803\\
236.01	0.00493010857294976\\
237.01	0.00493021144990413\\
238.01	0.0049303162982155\\
239.01	0.0049304231544316\\
240.01	0.00493053205574517\\
241.01	0.00493064304000346\\
242.01	0.00493075614571922\\
243.01	0.00493087141208165\\
244.01	0.00493098887896681\\
245.01	0.00493110858694948\\
246.01	0.00493123057731407\\
247.01	0.00493135489206567\\
248.01	0.00493148157394202\\
249.01	0.00493161066642522\\
250.01	0.00493174221375398\\
251.01	0.00493187626093508\\
252.01	0.00493201285375597\\
253.01	0.00493215203879693\\
254.01	0.00493229386344446\\
255.01	0.00493243837590296\\
256.01	0.00493258562520849\\
257.01	0.00493273566124178\\
258.01	0.00493288853474158\\
259.01	0.00493304429731806\\
260.01	0.00493320300146705\\
261.01	0.00493336470058298\\
262.01	0.00493352944897399\\
263.01	0.00493369730187609\\
264.01	0.00493386831546727\\
265.01	0.00493404254688304\\
266.01	0.00493422005423044\\
267.01	0.00493440089660444\\
268.01	0.00493458513410247\\
269.01	0.00493477282784047\\
270.01	0.00493496403996954\\
271.01	0.00493515883369031\\
272.01	0.00493535727327132\\
273.01	0.00493555942406446\\
274.01	0.00493576535252299\\
275.01	0.00493597512621743\\
276.01	0.0049361888138548\\
277.01	0.00493640648529475\\
278.01	0.00493662821156854\\
279.01	0.00493685406489774\\
280.01	0.0049370841187128\\
281.01	0.00493731844767115\\
282.01	0.00493755712767796\\
283.01	0.00493780023590526\\
284.01	0.00493804785081245\\
285.01	0.00493830005216619\\
286.01	0.00493855692106168\\
287.01	0.0049388185399438\\
288.01	0.00493908499262932\\
289.01	0.00493935636432841\\
290.01	0.00493963274166756\\
291.01	0.0049399142127125\\
292.01	0.00494020086699201\\
293.01	0.004940492795522\\
294.01	0.00494079009082943\\
295.01	0.00494109284697816\\
296.01	0.00494140115959471\\
297.01	0.00494171512589406\\
298.01	0.00494203484470715\\
299.01	0.00494236041650806\\
300.01	0.00494269194344221\\
301.01	0.0049430295293558\\
302.01	0.00494337327982519\\
303.01	0.00494372330218756\\
304.01	0.00494407970557192\\
305.01	0.00494444260093238\\
306.01	0.00494481210107928\\
307.01	0.00494518832071475\\
308.01	0.00494557137646728\\
309.01	0.00494596138692713\\
310.01	0.00494635847268471\\
311.01	0.00494676275636632\\
312.01	0.00494717436267597\\
313.01	0.00494759341843476\\
314.01	0.00494802005262207\\
315.01	0.00494845439641964\\
316.01	0.00494889658325581\\
317.01	0.00494934674885122\\
318.01	0.00494980503126588\\
319.01	0.00495027157094885\\
320.01	0.00495074651078839\\
321.01	0.00495122999616479\\
322.01	0.00495172217500356\\
323.01	0.00495222319783234\\
324.01	0.00495273321783835\\
325.01	0.0049532523909282\\
326.01	0.00495378087578954\\
327.01	0.00495431883395647\\
328.01	0.00495486642987386\\
329.01	0.00495542383096773\\
330.01	0.00495599120771539\\
331.01	0.00495656873371975\\
332.01	0.00495715658578436\\
333.01	0.00495775494399275\\
334.01	0.00495836399179034\\
335.01	0.00495898391606756\\
336.01	0.00495961490724806\\
337.01	0.00496025715937801\\
338.01	0.00496091087021955\\
339.01	0.00496157624134719\\
340.01	0.00496225347824647\\
341.01	0.0049629427904172\\
342.01	0.00496364439147895\\
343.01	0.00496435849928029\\
344.01	0.00496508533601028\\
345.01	0.00496582512831606\\
346.01	0.00496657810741996\\
347.01	0.00496734450924372\\
348.01	0.00496812457453369\\
349.01	0.00496891854898956\\
350.01	0.00496972668339865\\
351.01	0.00497054923377022\\
352.01	0.00497138646147524\\
353.01	0.004972238633389\\
354.01	0.00497310602203585\\
355.01	0.00497398890573808\\
356.01	0.00497488756876668\\
357.01	0.00497580230149569\\
358.01	0.00497673340055799\\
359.01	0.00497768116900435\\
360.01	0.00497864591646405\\
361.01	0.00497962795930765\\
362.01	0.00498062762081122\\
363.01	0.00498164523132263\\
364.01	0.00498268112842817\\
365.01	0.00498373565711971\\
366.01	0.00498480916996449\\
367.01	0.00498590202727345\\
368.01	0.0049870145972693\\
369.01	0.0049881472562571\\
370.01	0.00498930038879045\\
371.01	0.00499047438783992\\
372.01	0.00499166965496023\\
373.01	0.00499288660045505\\
374.01	0.00499412564354196\\
375.01	0.00499538721251511\\
376.01	0.00499667174490859\\
377.01	0.00499797968765625\\
378.01	0.00499931149725183\\
379.01	0.00500066763990911\\
380.01	0.00500204859172093\\
381.01	0.00500345483881856\\
382.01	0.00500488687753237\\
383.01	0.00500634521455415\\
384.01	0.00500783036710345\\
385.01	0.0050093428630954\\
386.01	0.00501088324131688\\
387.01	0.00501245205160566\\
388.01	0.00501404985504153\\
389.01	0.00501567722414542\\
390.01	0.00501733474309138\\
391.01	0.00501902300792978\\
392.01	0.00502074262683022\\
393.01	0.00502249422033935\\
394.01	0.00502427842165828\\
395.01	0.00502609587694272\\
396.01	0.00502794724562558\\
397.01	0.0050298332007632\\
398.01	0.00503175442940865\\
399.01	0.00503371163300942\\
400.01	0.00503570552783239\\
401.01	0.0050377368454147\\
402.01	0.00503980633304035\\
403.01	0.00504191475423932\\
404.01	0.00504406288930961\\
405.01	0.00504625153585783\\
406.01	0.00504848150935616\\
407.01	0.00505075364371188\\
408.01	0.00505306879184558\\
409.01	0.00505542782627566\\
410.01	0.00505783163970333\\
411.01	0.00506028114559842\\
412.01	0.00506277727877949\\
413.01	0.00506532099599163\\
414.01	0.00506791327647585\\
415.01	0.00507055512253482\\
416.01	0.0050732475600952\\
417.01	0.00507599163926481\\
418.01	0.00507878843489198\\
419.01	0.00508163904712511\\
420.01	0.00508454460197319\\
421.01	0.00508750625187029\\
422.01	0.00509052517624065\\
423.01	0.00509360258206507\\
424.01	0.00509673970445076\\
425.01	0.00509993780720003\\
426.01	0.0051031981833782\\
427.01	0.00510652215588461\\
428.01	0.00510991107802156\\
429.01	0.00511336633406412\\
430.01	0.00511688933983065\\
431.01	0.00512048154325123\\
432.01	0.00512414442494041\\
433.01	0.0051278794987677\\
434.01	0.00513168831243274\\
435.01	0.00513557244804125\\
436.01	0.00513953352268609\\
437.01	0.00514357318903322\\
438.01	0.00514769313591301\\
439.01	0.00515189508892069\\
440.01	0.0051561808110252\\
441.01	0.00516055210319047\\
442.01	0.00516501080500903\\
443.01	0.00516955879535103\\
444.01	0.00517419799303032\\
445.01	0.00517893035748884\\
446.01	0.00518375788950279\\
447.01	0.00518868263191022\\
448.01	0.00519370667036323\\
449.01	0.0051988321341048\\
450.01	0.00520406119677222\\
451.01	0.00520939607722745\\
452.01	0.00521483904041384\\
453.01	0.00522039239824043\\
454.01	0.00522605851049271\\
455.01	0.00523183978576761\\
456.01	0.00523773868243391\\
457.01	0.00524375770961405\\
458.01	0.00524989942818671\\
459.01	0.00525616645180813\\
460.01	0.00526256144794826\\
461.01	0.00526908713894422\\
462.01	0.00527574630306341\\
463.01	0.00528254177558157\\
464.01	0.00528947644986957\\
465.01	0.00529655327849493\\
466.01	0.00530377527433369\\
467.01	0.00531114551169888\\
468.01	0.00531866712748578\\
469.01	0.00532634332233806\\
470.01	0.00533417736183868\\
471.01	0.00534217257772819\\
472.01	0.00535033236915525\\
473.01	0.00535866020396041\\
474.01	0.00536715961999752\\
475.01	0.00537583422649196\\
476.01	0.00538468770543755\\
477.01	0.0053937238130322\\
478.01	0.00540294638115267\\
479.01	0.00541235931886618\\
480.01	0.00542196661398017\\
481.01	0.00543177233462867\\
482.01	0.00544178063089473\\
483.01	0.00545199573646733\\
484.01	0.00546242197033285\\
485.01	0.00547306373849976\\
486.01	0.00548392553575595\\
487.01	0.00549501194745909\\
488.01	0.005506327651358\\
489.01	0.00551787741944686\\
490.01	0.00552966611985088\\
491.01	0.0055416987187458\\
492.01	0.00555398028230869\\
493.01	0.00556651597870308\\
494.01	0.00557931108009724\\
495.01	0.00559237096471669\\
496.01	0.00560570111892965\\
497.01	0.00561930713936619\\
498.01	0.00563319473506921\\
499.01	0.00564736972967747\\
500.01	0.0056618380636375\\
501.01	0.00567660579644481\\
502.01	0.00569167910891125\\
503.01	0.00570706430545774\\
504.01	0.00572276781642914\\
505.01	0.00573879620043109\\
506.01	0.00575515614668512\\
507.01	0.00577185447740057\\
508.01	0.00578889815016059\\
509.01	0.00580629426031899\\
510.01	0.00582405004340483\\
511.01	0.00584217287753226\\
512.01	0.00586067028580932\\
513.01	0.00587954993874287\\
514.01	0.00589881965663439\\
515.01	0.00591848741195934\\
516.01	0.00593856133172524\\
517.01	0.00595904969980052\\
518.01	0.00597996095920626\\
519.01	0.00600130371436242\\
520.01	0.00602308673327795\\
521.01	0.00604531894967439\\
522.01	0.00606800946503175\\
523.01	0.00609116755054227\\
524.01	0.00611480264895806\\
525.01	0.00613892437631648\\
526.01	0.00616354252352511\\
527.01	0.00618866705778603\\
528.01	0.00621430812383886\\
529.01	0.00624047604499726\\
530.01	0.00626718132395188\\
531.01	0.00629443464331172\\
532.01	0.0063222468658508\\
533.01	0.00635062903442334\\
534.01	0.00637959237151045\\
535.01	0.00640914827835252\\
536.01	0.00643930833362073\\
537.01	0.00647008429157556\\
538.01	0.00650148807965305\\
539.01	0.00653353179541709\\
540.01	0.00656622770280604\\
541.01	0.00659958822759805\\
542.01	0.00663362595201053\\
543.01	0.00666835360834141\\
544.01	0.0067037840715494\\
545.01	0.00673993035066367\\
546.01	0.00677680557889973\\
547.01	0.00681442300234893\\
548.01	0.00685279596709369\\
549.01	0.0068919379045899\\
550.01	0.00693186231514151\\
551.01	0.0069725827492761\\
552.01	0.00701411278681359\\
553.01	0.00705646601340192\\
554.01	0.00709965599427295\\
555.01	0.00714369624495094\\
556.01	0.00718860019862366\\
557.01	0.00723438116986311\\
558.01	0.00728105231435536\\
559.01	0.00732862658427608\\
560.01	0.00737711667892064\\
561.01	0.0074265349901686\\
562.01	0.00747689354233801\\
563.01	0.00752820392595671\\
564.01	0.00758047722495186\\
565.01	0.00763372393673838\\
566.01	0.0076879538846658\\
567.01	0.00774317612227109\\
568.01	0.00779939882877782\\
569.01	0.00785662919528888\\
570.01	0.00791487330113772\\
571.01	0.00797413597990184\\
572.01	0.00803442067464424\\
573.01	0.00809572928204298\\
574.01	0.00815806198520298\\
575.01	0.00822141707512997\\
576.01	0.00828579076109853\\
577.01	0.00835117697047761\\
578.01	0.00841756713901588\\
579.01	0.00848494999315436\\
580.01	0.00855331132666273\\
581.01	0.00862263377482674\\
582.01	0.00869289659059665\\
583.01	0.00876407542860217\\
584.01	0.00883614214482343\\
585.01	0.00890906462207175\\
586.01	0.0089828066343969\\
587.01	0.0090573277672331\\
588.01	0.00913258341470725\\
589.01	0.00920852488126394\\
590.01	0.00928509962188423\\
591.01	0.00936225166400636\\
592.01	0.00943992226519836\\
593.01	0.0095180508741716\\
594.01	0.00959657647946483\\
595.01	0.00967543945080618\\
596.01	0.00975458400368758\\
597.01	0.00983363394660122\\
598.01	0.00990826330299362\\
599.01	0.00997053306357289\\
599.02	0.00997104257304143\\
599.03	0.00997154904691472\\
599.04	0.00997205245565375\\
599.05	0.00997255276942872\\
599.06	0.00997304995811617\\
599.07	0.00997354399129606\\
599.08	0.00997403483824883\\
599.09	0.00997452246795248\\
599.1	0.00997500684907952\\
599.11	0.00997548794999398\\
599.12	0.00997596573874838\\
599.13	0.0099764401830806\\
599.14	0.0099769112504108\\
599.15	0.00997737890783826\\
599.16	0.00997784312213823\\
599.17	0.0099783038597587\\
599.18	0.00997876108681718\\
599.19	0.00997921476909742\\
599.2	0.00997966487204611\\
599.21	0.00998011136076955\\
599.22	0.00998055420003028\\
599.23	0.00998099335424369\\
599.24	0.00998142878747458\\
599.25	0.00998186046343367\\
599.26	0.00998228834359041\\
599.27	0.00998271238778643\\
599.28	0.00998313255546381\\
599.29	0.00998354880566111\\
599.3	0.00998396109700947\\
599.31	0.00998436938772849\\
599.32	0.00998477363562221\\
599.33	0.00998517379807499\\
599.34	0.00998556983204737\\
599.35	0.00998596169407188\\
599.36	0.00998634934024881\\
599.37	0.00998673272624192\\
599.38	0.00998711180727415\\
599.39	0.00998748653812326\\
599.4	0.00998785687311743\\
599.41	0.00998822276613081\\
599.42	0.00998858417057903\\
599.43	0.00998894103941468\\
599.44	0.00998929332512275\\
599.45	0.00998964097971596\\
599.46	0.00998998395473014\\
599.47	0.00999032220121951\\
599.48	0.0099906556697519\\
599.49	0.00999098431040396\\
599.5	0.00999130807275631\\
599.51	0.00999162690588863\\
599.52	0.00999194075837471\\
599.53	0.00999224957827746\\
599.54	0.00999255331314387\\
599.55	0.00999285190999987\\
599.56	0.00999314531534524\\
599.57	0.00999343347514839\\
599.58	0.00999371633484107\\
599.59	0.00999399383931314\\
599.6	0.00999426593290715\\
599.61	0.009994532559413\\
599.62	0.0099947936620624\\
599.63	0.00999504918352342\\
599.64	0.00999529906589491\\
599.65	0.00999554325070086\\
599.66	0.00999578167888471\\
599.67	0.00999601429080366\\
599.68	0.00999624102622283\\
599.69	0.00999646182430947\\
599.7	0.00999667662362697\\
599.71	0.00999688536212897\\
599.72	0.00999708797715332\\
599.73	0.00999728440541596\\
599.74	0.00999747458300482\\
599.75	0.0099976584453736\\
599.76	0.0099978359273355\\
599.77	0.00999800696305692\\
599.78	0.00999817148605102\\
599.79	0.00999832942917133\\
599.8	0.00999848072460517\\
599.81	0.00999862530386713\\
599.82	0.00999876309779239\\
599.83	0.00999889403653003\\
599.84	0.00999901804953622\\
599.85	0.00999913506556741\\
599.86	0.00999924501267341\\
599.87	0.00999934781819042\\
599.88	0.00999944340873394\\
599.89	0.0099995317101917\\
599.9	0.00999961264771648\\
599.91	0.00999968614571878\\
599.92	0.00999975212785958\\
599.93	0.00999981051704285\\
599.94	0.00999986123540817\\
599.95	0.00999990420432308\\
599.96	0.00999993934437554\\
599.97	0.00999996657536618\\
599.98	0.00999998581630055\\
599.99	0.00999999698538124\\
600	0.01\\
};
\addplot [color=black,solid,forget plot]
  table[row sep=crcr]{%
0.01	0.00493182287751867\\
1.01	0.0049318238290299\\
2.01	0.00493182480000867\\
3.01	0.00493182579085036\\
4.01	0.00493182680195881\\
5.01	0.00493182783374529\\
6.01	0.00493182888663021\\
7.01	0.00493182996104166\\
8.01	0.00493183105741671\\
9.01	0.00493183217620111\\
10.01	0.00493183331784941\\
11.01	0.00493183448282594\\
12.01	0.00493183567160338\\
13.01	0.00493183688466478\\
14.01	0.00493183812250247\\
15.01	0.00493183938561867\\
16.01	0.00493184067452582\\
17.01	0.00493184198974639\\
18.01	0.00493184333181385\\
19.01	0.00493184470127195\\
20.01	0.00493184609867525\\
21.01	0.00493184752458969\\
22.01	0.00493184897959259\\
23.01	0.00493185046427261\\
24.01	0.00493185197923021\\
25.01	0.00493185352507805\\
26.01	0.00493185510244132\\
27.01	0.00493185671195676\\
28.01	0.00493185835427515\\
29.01	0.00493186003005908\\
30.01	0.00493186173998522\\
31.01	0.00493186348474337\\
32.01	0.00493186526503714\\
33.01	0.00493186708158444\\
34.01	0.00493186893511746\\
35.01	0.00493187082638238\\
36.01	0.00493187275614124\\
37.01	0.00493187472517079\\
38.01	0.0049318767342634\\
39.01	0.00493187878422696\\
40.01	0.00493188087588575\\
41.01	0.00493188301008055\\
42.01	0.00493188518766854\\
43.01	0.00493188740952449\\
44.01	0.00493188967654021\\
45.01	0.00493189198962524\\
46.01	0.00493189434970737\\
47.01	0.0049318967577329\\
48.01	0.00493189921466686\\
49.01	0.00493190172149338\\
50.01	0.00493190427921622\\
51.01	0.00493190688885906\\
52.01	0.00493190955146586\\
53.01	0.00493191226810191\\
54.01	0.00493191503985274\\
55.01	0.00493191786782596\\
56.01	0.00493192075315094\\
57.01	0.00493192369697931\\
58.01	0.00493192670048592\\
59.01	0.00493192976486884\\
60.01	0.00493193289134933\\
61.01	0.00493193608117393\\
62.01	0.00493193933561248\\
63.01	0.00493194265596109\\
64.01	0.00493194604354083\\
65.01	0.00493194949969913\\
66.01	0.00493195302580993\\
67.01	0.00493195662327456\\
68.01	0.00493196029352166\\
69.01	0.00493196403800866\\
70.01	0.00493196785822145\\
71.01	0.00493197175567481\\
72.01	0.00493197573191395\\
73.01	0.00493197978851434\\
74.01	0.00493198392708238\\
75.01	0.00493198814925644\\
76.01	0.00493199245670695\\
77.01	0.00493199685113693\\
78.01	0.00493200133428354\\
79.01	0.00493200590791718\\
80.01	0.00493201057384435\\
81.01	0.00493201533390594\\
82.01	0.00493202018997954\\
83.01	0.0049320251439792\\
84.01	0.00493203019785679\\
85.01	0.00493203535360265\\
86.01	0.00493204061324592\\
87.01	0.00493204597885555\\
88.01	0.00493205145254083\\
89.01	0.00493205703645261\\
90.01	0.00493206273278426\\
91.01	0.00493206854377126\\
92.01	0.00493207447169297\\
93.01	0.00493208051887367\\
94.01	0.00493208668768283\\
95.01	0.00493209298053607\\
96.01	0.00493209939989656\\
97.01	0.00493210594827531\\
98.01	0.00493211262823195\\
99.01	0.00493211944237681\\
100.01	0.00493212639337046\\
101.01	0.00493213348392549\\
102.01	0.00493214071680723\\
103.01	0.00493214809483475\\
104.01	0.00493215562088206\\
105.01	0.00493216329787916\\
106.01	0.00493217112881278\\
107.01	0.00493217911672769\\
108.01	0.0049321872647278\\
109.01	0.00493219557597735\\
110.01	0.00493220405370171\\
111.01	0.00493221270118868\\
112.01	0.00493222152178996\\
113.01	0.00493223051892233\\
114.01	0.00493223969606799\\
115.01	0.00493224905677733\\
116.01	0.00493225860466878\\
117.01	0.00493226834343105\\
118.01	0.0049322782768236\\
119.01	0.00493228840867871\\
120.01	0.00493229874290274\\
121.01	0.00493230928347724\\
122.01	0.00493232003446017\\
123.01	0.00493233099998788\\
124.01	0.00493234218427618\\
125.01	0.0049323535916225\\
126.01	0.00493236522640594\\
127.01	0.00493237709309033\\
128.01	0.00493238919622524\\
129.01	0.00493240154044733\\
130.01	0.00493241413048207\\
131.01	0.00493242697114588\\
132.01	0.00493244006734664\\
133.01	0.00493245342408735\\
134.01	0.00493246704646585\\
135.01	0.00493248093967752\\
136.01	0.00493249510901711\\
137.01	0.0049325095598807\\
138.01	0.00493252429776672\\
139.01	0.00493253932827917\\
140.01	0.00493255465712827\\
141.01	0.00493257029013333\\
142.01	0.00493258623322426\\
143.01	0.00493260249244373\\
144.01	0.00493261907394901\\
145.01	0.004932635984015\\
146.01	0.00493265322903502\\
147.01	0.00493267081552407\\
148.01	0.00493268875012062\\
149.01	0.00493270703958856\\
150.01	0.00493272569082001\\
151.01	0.00493274471083699\\
152.01	0.00493276410679494\\
153.01	0.00493278388598445\\
154.01	0.00493280405583318\\
155.01	0.00493282462390918\\
156.01	0.00493284559792298\\
157.01	0.00493286698573062\\
158.01	0.00493288879533524\\
159.01	0.00493291103489127\\
160.01	0.00493293371270617\\
161.01	0.00493295683724287\\
162.01	0.00493298041712356\\
163.01	0.00493300446113168\\
164.01	0.00493302897821582\\
165.01	0.00493305397749164\\
166.01	0.00493307946824523\\
167.01	0.00493310545993636\\
168.01	0.00493313196220119\\
169.01	0.00493315898485566\\
170.01	0.00493318653789934\\
171.01	0.00493321463151686\\
172.01	0.0049332432760836\\
173.01	0.00493327248216704\\
174.01	0.00493330226053137\\
175.01	0.00493333262214005\\
176.01	0.00493336357816088\\
177.01	0.00493339513996749\\
178.01	0.00493342731914478\\
179.01	0.0049334601274918\\
180.01	0.00493349357702577\\
181.01	0.00493352767998562\\
182.01	0.00493356244883617\\
183.01	0.00493359789627182\\
184.01	0.00493363403522082\\
185.01	0.00493367087884955\\
186.01	0.00493370844056579\\
187.01	0.00493374673402433\\
188.01	0.00493378577312975\\
189.01	0.00493382557204222\\
190.01	0.00493386614518054\\
191.01	0.00493390750722789\\
192.01	0.0049339496731351\\
193.01	0.00493399265812658\\
194.01	0.00493403647770455\\
195.01	0.00493408114765315\\
196.01	0.00493412668404436\\
197.01	0.00493417310324204\\
198.01	0.004934220421908\\
199.01	0.00493426865700619\\
200.01	0.00493431782580808\\
201.01	0.00493436794589777\\
202.01	0.00493441903517816\\
203.01	0.00493447111187581\\
204.01	0.00493452419454568\\
205.01	0.00493457830207786\\
206.01	0.00493463345370304\\
207.01	0.00493468966899818\\
208.01	0.00493474696789159\\
209.01	0.00493480537067087\\
210.01	0.00493486489798639\\
211.01	0.00493492557085923\\
212.01	0.00493498741068712\\
213.01	0.00493505043925033\\
214.01	0.00493511467871835\\
215.01	0.00493518015165628\\
216.01	0.00493524688103181\\
217.01	0.00493531489022142\\
218.01	0.00493538420301784\\
219.01	0.00493545484363617\\
220.01	0.0049355268367218\\
221.01	0.00493560020735717\\
222.01	0.0049356749810687\\
223.01	0.0049357511838343\\
224.01	0.00493582884209182\\
225.01	0.00493590798274565\\
226.01	0.00493598863317379\\
227.01	0.00493607082123765\\
228.01	0.00493615457528845\\
229.01	0.00493623992417599\\
230.01	0.00493632689725595\\
231.01	0.00493641552439997\\
232.01	0.00493650583600228\\
233.01	0.00493659786298942\\
234.01	0.00493669163682811\\
235.01	0.00493678718953539\\
236.01	0.0049368845536864\\
237.01	0.00493698376242396\\
238.01	0.00493708484946778\\
239.01	0.0049371878491247\\
240.01	0.00493729279629676\\
241.01	0.00493739972649196\\
242.01	0.00493750867583373\\
243.01	0.00493761968107156\\
244.01	0.00493773277959098\\
245.01	0.00493784800942301\\
246.01	0.00493796540925598\\
247.01	0.00493808501844553\\
248.01	0.00493820687702576\\
249.01	0.00493833102572058\\
250.01	0.00493845750595384\\
251.01	0.00493858635986247\\
252.01	0.0049387176303064\\
253.01	0.00493885136088135\\
254.01	0.0049389875959312\\
255.01	0.00493912638055916\\
256.01	0.00493926776064105\\
257.01	0.0049394117828373\\
258.01	0.00493955849460703\\
259.01	0.00493970794421938\\
260.01	0.00493986018076875\\
261.01	0.00494001525418711\\
262.01	0.00494017321525833\\
263.01	0.00494033411563222\\
264.01	0.00494049800783891\\
265.01	0.00494066494530354\\
266.01	0.00494083498236114\\
267.01	0.00494100817427168\\
268.01	0.00494118457723583\\
269.01	0.00494136424841066\\
270.01	0.00494154724592571\\
271.01	0.00494173362889974\\
272.01	0.00494192345745711\\
273.01	0.00494211679274509\\
274.01	0.00494231369695136\\
275.01	0.00494251423332251\\
276.01	0.00494271846618066\\
277.01	0.00494292646094402\\
278.01	0.0049431382841448\\
279.01	0.00494335400344934\\
280.01	0.00494357368767709\\
281.01	0.0049437974068223\\
282.01	0.00494402523207391\\
283.01	0.00494425723583722\\
284.01	0.00494449349175594\\
285.01	0.0049447340747345\\
286.01	0.00494497906096072\\
287.01	0.00494522852792967\\
288.01	0.00494548255446772\\
289.01	0.00494574122075717\\
290.01	0.00494600460836153\\
291.01	0.00494627280025134\\
292.01	0.00494654588083177\\
293.01	0.00494682393596887\\
294.01	0.00494710705301846\\
295.01	0.00494739532085464\\
296.01	0.00494768882989916\\
297.01	0.00494798767215293\\
298.01	0.00494829194122615\\
299.01	0.00494860173237138\\
300.01	0.00494891714251669\\
301.01	0.00494923827029894\\
302.01	0.0049495652160996\\
303.01	0.00494989808208038\\
304.01	0.00495023697222066\\
305.01	0.00495058199235521\\
306.01	0.00495093325021403\\
307.01	0.00495129085546259\\
308.01	0.00495165491974356\\
309.01	0.00495202555671972\\
310.01	0.00495240288211798\\
311.01	0.00495278701377563\\
312.01	0.00495317807168652\\
313.01	0.00495357617804909\\
314.01	0.00495398145731751\\
315.01	0.00495439403625192\\
316.01	0.00495481404397116\\
317.01	0.00495524161200769\\
318.01	0.0049556768743641\\
319.01	0.00495611996756937\\
320.01	0.00495657103073976\\
321.01	0.00495703020563959\\
322.01	0.00495749763674413\\
323.01	0.00495797347130478\\
324.01	0.00495845785941534\\
325.01	0.0049589509540818\\
326.01	0.00495945291129252\\
327.01	0.00495996389009043\\
328.01	0.00496048405264951\\
329.01	0.00496101356434989\\
330.01	0.00496155259385811\\
331.01	0.00496210131320801\\
332.01	0.00496265989788433\\
333.01	0.00496322852690834\\
334.01	0.00496380738292523\\
335.01	0.00496439665229551\\
336.01	0.00496499652518606\\
337.01	0.0049656071956663\\
338.01	0.00496622886180392\\
339.01	0.00496686172576477\\
340.01	0.00496750599391538\\
341.01	0.00496816187692553\\
342.01	0.00496882958987546\\
343.01	0.00496950935236398\\
344.01	0.00497020138861987\\
345.01	0.00497090592761398\\
346.01	0.00497162320317568\\
347.01	0.00497235345410937\\
348.01	0.00497309692431502\\
349.01	0.00497385386290883\\
350.01	0.00497462452434724\\
351.01	0.00497540916855377\\
352.01	0.00497620806104546\\
353.01	0.00497702147306339\\
354.01	0.00497784968170376\\
355.01	0.00497869297005177\\
356.01	0.00497955162731625\\
357.01	0.00498042594896672\\
358.01	0.00498131623687226\\
359.01	0.00498222279944097\\
360.01	0.00498314595176261\\
361.01	0.00498408601575159\\
362.01	0.00498504332029251\\
363.01	0.00498601820138575\\
364.01	0.00498701100229715\\
365.01	0.00498802207370727\\
366.01	0.00498905177386316\\
367.01	0.0049901004687323\\
368.01	0.00499116853215789\\
369.01	0.00499225634601669\\
370.01	0.00499336430037978\\
371.01	0.00499449279367409\\
372.01	0.00499564223284896\\
373.01	0.00499681303354341\\
374.01	0.00499800562025859\\
375.01	0.00499922042653445\\
376.01	0.00500045789512835\\
377.01	0.00500171847820011\\
378.01	0.00500300263750234\\
379.01	0.00500431084457509\\
380.01	0.00500564358094832\\
381.01	0.00500700133834953\\
382.01	0.00500838461892072\\
383.01	0.00500979393544249\\
384.01	0.00501122981156529\\
385.01	0.00501269278205324\\
386.01	0.00501418339303464\\
387.01	0.00501570220226563\\
388.01	0.00501724977940263\\
389.01	0.00501882670628742\\
390.01	0.0050204335772433\\
391.01	0.00502207099938506\\
392.01	0.00502373959293897\\
393.01	0.0050254399915764\\
394.01	0.00502717284276052\\
395.01	0.00502893880810304\\
396.01	0.00503073856373509\\
397.01	0.00503257280068803\\
398.01	0.00503444222528482\\
399.01	0.00503634755954288\\
400.01	0.00503828954158442\\
401.01	0.0050402689260558\\
402.01	0.00504228648455377\\
403.01	0.00504434300605858\\
404.01	0.00504643929737017\\
405.01	0.00504857618354995\\
406.01	0.00505075450836626\\
407.01	0.00505297513473901\\
408.01	0.00505523894519064\\
409.01	0.00505754684229329\\
410.01	0.00505989974912169\\
411.01	0.00506229860970375\\
412.01	0.00506474438947371\\
413.01	0.00506723807572645\\
414.01	0.00506978067807337\\
415.01	0.00507237322890244\\
416.01	0.00507501678383912\\
417.01	0.0050777124222123\\
418.01	0.0050804612475229\\
419.01	0.00508326438791806\\
420.01	0.00508612299666758\\
421.01	0.00508903825264689\\
422.01	0.00509201136082305\\
423.01	0.00509504355274762\\
424.01	0.00509813608705151\\
425.01	0.00510129024994832\\
426.01	0.00510450735574314\\
427.01	0.00510778874734499\\
428.01	0.00511113579678996\\
429.01	0.00511454990576912\\
430.01	0.00511803250616452\\
431.01	0.00512158506059596\\
432.01	0.00512520906297331\\
433.01	0.00512890603906338\\
434.01	0.00513267754706359\\
435.01	0.00513652517819077\\
436.01	0.00514045055727982\\
437.01	0.00514445534339625\\
438.01	0.00514854123046399\\
439.01	0.00515270994790704\\
440.01	0.00515696326130713\\
441.01	0.00516130297307757\\
442.01	0.00516573092315437\\
443.01	0.00517024898970506\\
444.01	0.00517485908985496\\
445.01	0.00517956318043405\\
446.01	0.00518436325874022\\
447.01	0.00518926136332284\\
448.01	0.00519425957478664\\
449.01	0.00519936001661375\\
450.01	0.00520456485600515\\
451.01	0.00520987630474193\\
452.01	0.00521529662006501\\
453.01	0.00522082810557442\\
454.01	0.0052264731121468\\
455.01	0.00523223403887141\\
456.01	0.00523811333400456\\
457.01	0.00524411349594245\\
458.01	0.00525023707421238\\
459.01	0.00525648667048167\\
460.01	0.0052628649395878\\
461.01	0.0052693745905851\\
462.01	0.005276018387814\\
463.01	0.00528279915198995\\
464.01	0.00528971976131598\\
465.01	0.00529678315261656\\
466.01	0.00530399232249751\\
467.01	0.00531135032853136\\
468.01	0.00531886029046935\\
469.01	0.00532652539148226\\
470.01	0.00533434887942855\\
471.01	0.00534233406815415\\
472.01	0.00535048433882165\\
473.01	0.00535880314127121\\
474.01	0.00536729399541064\\
475.01	0.00537596049263875\\
476.01	0.00538480629729859\\
477.01	0.00539383514816266\\
478.01	0.00540305085994849\\
479.01	0.00541245732486612\\
480.01	0.00542205851419633\\
481.01	0.00543185847989999\\
482.01	0.00544186135625857\\
483.01	0.00545207136154552\\
484.01	0.00546249279972985\\
485.01	0.00547313006220942\\
486.01	0.0054839876295774\\
487.01	0.00549507007341975\\
488.01	0.00550638205814426\\
489.01	0.00551792834284205\\
490.01	0.00552971378318126\\
491.01	0.00554174333333225\\
492.01	0.00555402204792588\\
493.01	0.0055665550840433\\
494.01	0.00557934770323765\\
495.01	0.00559240527358693\\
496.01	0.00560573327177863\\
497.01	0.00561933728522382\\
498.01	0.00563322301420169\\
499.01	0.00564739627403231\\
500.01	0.00566186299727794\\
501.01	0.0056766292359703\\
502.01	0.00569170116386418\\
503.01	0.00570708507871467\\
504.01	0.00572278740457782\\
505.01	0.00573881469413159\\
506.01	0.00575517363101588\\
507.01	0.00577187103218983\\
508.01	0.00578891385030207\\
509.01	0.00580630917607387\\
510.01	0.00582406424068908\\
511.01	0.005842186418189\\
512.01	0.00586068322786752\\
513.01	0.00587956233666231\\
514.01	0.00589883156153569\\
515.01	0.00591849887184132\\
516.01	0.00593857239166883\\
517.01	0.00595906040216021\\
518.01	0.00597997134378925\\
519.01	0.00600131381859617\\
520.01	0.0060230965923669\\
521.01	0.00604532859674715\\
522.01	0.00606801893127805\\
523.01	0.00609117686534137\\
524.01	0.00611481183999849\\
525.01	0.00613893346970859\\
526.01	0.00616355154390595\\
527.01	0.00618867602841776\\
528.01	0.00621431706670075\\
529.01	0.00624048498087197\\
530.01	0.00626719027250712\\
531.01	0.00629444362317652\\
532.01	0.00632225589468721\\
533.01	0.00635063812899544\\
534.01	0.00637960154774825\\
535.01	0.00640915755141296\\
536.01	0.00643931771794654\\
537.01	0.00647009380094982\\
538.01	0.00650149772725282\\
539.01	0.0065335415938632\\
540.01	0.00656623766421105\\
541.01	0.00659959836361262\\
542.01	0.00663363627386721\\
543.01	0.00666836412689591\\
544.01	0.00670379479732074\\
545.01	0.00673994129387191\\
546.01	0.00677681674950216\\
547.01	0.00681443441007326\\
548.01	0.0068528076214695\\
549.01	0.00689194981497837\\
550.01	0.00693187449076199\\
551.01	0.00697259519923017\\
552.01	0.00701412552010712\\
553.01	0.00705647903896367\\
554.01	0.00709966932096989\\
555.01	0.00714370988160053\\
556.01	0.00718861415400322\\
557.01	0.00723439545271368\\
558.01	0.00728106693338249\\
559.01	0.00732864154814654\\
560.01	0.00737713199625294\\
561.01	0.00742655066951931\\
562.01	0.00747690959218157\\
563.01	0.00752822035465922\\
564.01	0.00758049404073814\\
565.01	0.00763374114765142\\
566.01	0.00768797149851824\\
567.01	0.0077431941465875\\
568.01	0.00779941727072736\\
569.01	0.00785664806160745\\
570.01	0.00791489259803824\\
571.01	0.00797415571297161\\
572.01	0.00803444084872872\\
573.01	0.00809574990111406\\
574.01	0.00815808305221095\\
575.01	0.0082214385918378\\
576.01	0.00828581272789679\\
577.01	0.00835119938617968\\
578.01	0.00841759000063182\\
579.01	0.00848497329564366\\
580.01	0.00855333506266526\\
581.01	0.00862265793437383\\
582.01	0.00869292116080432\\
583.01	0.008764100393351\\
584.01	0.00883616748443228\\
585.01	0.00890909031297696\\
586.01	0.00898283264885292\\
587.01	0.00905735407305973\\
588.01	0.00913260997511742\\
589.01	0.00920855165482164\\
590.01	0.00928512656266121\\
591.01	0.00936227872203362\\
592.01	0.00943994938734479\\
593.01	0.00951807800563368\\
594.01	0.00959660356611897\\
595.01	0.0096754664427708\\
596.01	0.00975461086056989\\
597.01	0.00983363553942939\\
598.01	0.00990826330299362\\
599.01	0.00997053306357288\\
599.02	0.00997104257304143\\
599.03	0.00997154904691472\\
599.04	0.00997205245565375\\
599.05	0.00997255276942872\\
599.06	0.00997304995811617\\
599.07	0.00997354399129606\\
599.08	0.00997403483824883\\
599.09	0.00997452246795248\\
599.1	0.00997500684907951\\
599.11	0.00997548794999398\\
599.12	0.00997596573874838\\
599.13	0.0099764401830806\\
599.14	0.0099769112504108\\
599.15	0.00997737890783826\\
599.16	0.00997784312213823\\
599.17	0.0099783038597587\\
599.18	0.00997876108681718\\
599.19	0.00997921476909742\\
599.2	0.00997966487204611\\
599.21	0.00998011136076955\\
599.22	0.00998055420003028\\
599.23	0.00998099335424369\\
599.24	0.00998142878747458\\
599.25	0.00998186046343367\\
599.26	0.00998228834359041\\
599.27	0.00998271238778643\\
599.28	0.00998313255546381\\
599.29	0.00998354880566111\\
599.3	0.00998396109700947\\
599.31	0.00998436938772849\\
599.32	0.00998477363562221\\
599.33	0.00998517379807499\\
599.34	0.00998556983204737\\
599.35	0.00998596169407188\\
599.36	0.00998634934024881\\
599.37	0.00998673272624192\\
599.38	0.00998711180727415\\
599.39	0.00998748653812326\\
599.4	0.00998785687311743\\
599.41	0.00998822276613081\\
599.42	0.00998858417057903\\
599.43	0.00998894103941468\\
599.44	0.00998929332512275\\
599.45	0.00998964097971596\\
599.46	0.00998998395473014\\
599.47	0.00999032220121951\\
599.48	0.0099906556697519\\
599.49	0.00999098431040396\\
599.5	0.00999130807275631\\
599.51	0.00999162690588863\\
599.52	0.00999194075837471\\
599.53	0.00999224957827746\\
599.54	0.00999255331314386\\
599.55	0.00999285190999987\\
599.56	0.00999314531534524\\
599.57	0.00999343347514839\\
599.58	0.00999371633484107\\
599.59	0.00999399383931314\\
599.6	0.00999426593290715\\
599.61	0.009994532559413\\
599.62	0.0099947936620624\\
599.63	0.00999504918352342\\
599.64	0.00999529906589491\\
599.65	0.00999554325070086\\
599.66	0.00999578167888471\\
599.67	0.00999601429080366\\
599.68	0.00999624102622283\\
599.69	0.00999646182430947\\
599.7	0.00999667662362697\\
599.71	0.00999688536212897\\
599.72	0.00999708797715331\\
599.73	0.00999728440541595\\
599.74	0.00999747458300482\\
599.75	0.0099976584453736\\
599.76	0.00999783592733551\\
599.77	0.00999800696305692\\
599.78	0.00999817148605102\\
599.79	0.00999832942917133\\
599.8	0.00999848072460517\\
599.81	0.00999862530386713\\
599.82	0.00999876309779239\\
599.83	0.00999889403653003\\
599.84	0.00999901804953622\\
599.85	0.00999913506556741\\
599.86	0.00999924501267341\\
599.87	0.00999934781819042\\
599.88	0.00999944340873394\\
599.89	0.0099995317101917\\
599.9	0.00999961264771648\\
599.91	0.00999968614571878\\
599.92	0.00999975212785957\\
599.93	0.00999981051704285\\
599.94	0.00999986123540817\\
599.95	0.00999990420432308\\
599.96	0.00999993934437554\\
599.97	0.00999996657536618\\
599.98	0.00999998581630055\\
599.99	0.00999999698538124\\
600	0.01\\
};
\end{axis}
\end{tikzpicture}%
%  \caption{Continuous Time}
%\end{subfigure}%
%\hfill%
%\begin{subfigure}{.45\linewidth}
%  \centering
%  \setlength\figureheight{\linewidth} 
%  \setlength\figurewidth{\linewidth}
%  \tikzsetnextfilename{dp_dscr_z8}
%  % This file was created by matlab2tikz.
%
%The latest updates can be retrieved from
%  http://www.mathworks.com/matlabcentral/fileexchange/22022-matlab2tikz-matlab2tikz
%where you can also make suggestions and rate matlab2tikz.
%
\definecolor{mycolor1}{rgb}{0.00000,1.00000,0.14286}%
\definecolor{mycolor2}{rgb}{0.00000,1.00000,0.28571}%
\definecolor{mycolor3}{rgb}{0.00000,1.00000,0.42857}%
\definecolor{mycolor4}{rgb}{0.00000,1.00000,0.57143}%
\definecolor{mycolor5}{rgb}{0.00000,1.00000,0.71429}%
\definecolor{mycolor6}{rgb}{0.00000,1.00000,0.85714}%
\definecolor{mycolor7}{rgb}{0.00000,1.00000,1.00000}%
\definecolor{mycolor8}{rgb}{0.00000,0.87500,1.00000}%
\definecolor{mycolor9}{rgb}{0.00000,0.62500,1.00000}%
\definecolor{mycolor10}{rgb}{0.12500,0.00000,1.00000}%
\definecolor{mycolor11}{rgb}{0.25000,0.00000,1.00000}%
\definecolor{mycolor12}{rgb}{0.37500,0.00000,1.00000}%
\definecolor{mycolor13}{rgb}{0.50000,0.00000,1.00000}%
\definecolor{mycolor14}{rgb}{0.62500,0.00000,1.00000}%
\definecolor{mycolor15}{rgb}{0.75000,0.00000,1.00000}%
\definecolor{mycolor16}{rgb}{0.87500,0.00000,1.00000}%
\definecolor{mycolor17}{rgb}{1.00000,0.00000,1.00000}%
\definecolor{mycolor18}{rgb}{1.00000,0.00000,0.87500}%
\definecolor{mycolor19}{rgb}{1.00000,0.00000,0.62500}%
\definecolor{mycolor20}{rgb}{0.85714,0.00000,0.00000}%
\definecolor{mycolor21}{rgb}{0.71429,0.00000,0.00000}%
%
\begin{tikzpicture}

\begin{axis}[%
width=4.1in,
height=3.803in,
at={(0.809in,0.513in)},
scale only axis,
point meta min=0,
point meta max=1,
every outer x axis line/.append style={black},
every x tick label/.append style={font=\color{black}},
xmin=0,
xmax=600,
every outer y axis line/.append style={black},
every y tick label/.append style={font=\color{black}},
ymin=0,
ymax=0.01,
axis background/.style={fill=white},
axis x line*=bottom,
axis y line*=left,
colormap={mymap}{[1pt] rgb(0pt)=(0,1,0); rgb(7pt)=(0,1,1); rgb(15pt)=(0,0,1); rgb(23pt)=(1,0,1); rgb(31pt)=(1,0,0); rgb(38pt)=(0,0,0)},
colorbar,
colorbar style={separate axis lines,every outer x axis line/.append style={black},every x tick label/.append style={font=\color{black}},every outer y axis line/.append style={black},every y tick label/.append style={font=\color{black}},yticklabels={{-19},{-17},{-15},{-13},{-11},{-9},{-7},{-5},{-3},{-1},{1},{3},{5},{7},{9},{11},{13},{15},{17},{19}}}
]
\addplot [color=green,solid,forget plot]
  table[row sep=crcr]{%
1	0.00547780019338487\\
2	0.00547779836203247\\
3	0.00547779649665558\\
4	0.00547779459661717\\
5	0.00547779266126814\\
6	0.00547779068994721\\
7	0.00547778868198065\\
8	0.00547778663668193\\
9	0.00547778455335156\\
10	0.00547778243127691\\
11	0.00547778026973184\\
12	0.00547777806797653\\
13	0.00547777582525721\\
14	0.0054777735408058\\
15	0.00547777121383979\\
16	0.00547776884356175\\
17	0.00547776642915928\\
18	0.00547776396980463\\
19	0.00547776146465437\\
20	0.00547775891284912\\
21	0.00547775631351337\\
22	0.00547775366575492\\
23	0.00547775096866472\\
24	0.00547774822131646\\
25	0.00547774542276647\\
26	0.00547774257205312\\
27	0.00547773966819658\\
28	0.00547773671019848\\
29	0.00547773369704161\\
30	0.00547773062768945\\
31	0.00547772750108587\\
32	0.00547772431615465\\
33	0.00547772107179931\\
34	0.00547771776690257\\
35	0.00547771440032607\\
36	0.00547771097090967\\
37	0.00547770747747142\\
38	0.00547770391880689\\
39	0.00547770029368877\\
40	0.00547769660086674\\
41	0.00547769283906642\\
42	0.00547768900698951\\
43	0.00547768510331311\\
44	0.005477681126689\\
45	0.00547767707574356\\
46	0.00547767294907713\\
47	0.00547766874526338\\
48	0.00547766446284898\\
49	0.005477660100353\\
50	0.00547765565626646\\
51	0.00547765112905163\\
52	0.00547764651714156\\
53	0.00547764181893974\\
54	0.00547763703281917\\
55	0.00547763215712194\\
56	0.00547762719015876\\
57	0.00547762213020816\\
58	0.00547761697551608\\
59	0.00547761172429486\\
60	0.00547760637472321\\
61	0.00547760092494489\\
62	0.00547759537306858\\
63	0.00547758971716693\\
64	0.00547758395527595\\
65	0.00547757808539432\\
66	0.00547757210548267\\
67	0.00547756601346271\\
68	0.0054775598072167\\
69	0.00547755348458673\\
70	0.00547754704337352\\
71	0.00547754048133619\\
72	0.00547753379619117\\
73	0.00547752698561129\\
74	0.00547752004722511\\
75	0.0054775129786161\\
76	0.00547750577732178\\
77	0.0054774984408325\\
78	0.00547749096659111\\
79	0.00547748335199146\\
80	0.00547747559437789\\
81	0.00547746769104432\\
82	0.00547745963923266\\
83	0.00547745143613261\\
84	0.00547744307888013\\
85	0.00547743456455674\\
86	0.00547742589018804\\
87	0.00547741705274312\\
88	0.00547740804913314\\
89	0.00547739887621048\\
90	0.00547738953076738\\
91	0.00547738000953478\\
92	0.00547737030918148\\
93	0.00547736042631254\\
94	0.00547735035746826\\
95	0.00547734009912296\\
96	0.00547732964768352\\
97	0.00547731899948825\\
98	0.00547730815080578\\
99	0.00547729709783312\\
100	0.00547728583669486\\
101	0.0054772743634416\\
102	0.00547726267404826\\
103	0.0054772507644131\\
104	0.00547723863035594\\
105	0.00547722626761666\\
106	0.00547721367185374\\
107	0.00547720083864259\\
108	0.00547718776347421\\
109	0.00547717444175333\\
110	0.00547716086879675\\
111	0.00547714703983176\\
112	0.00547713294999438\\
113	0.00547711859432751\\
114	0.00547710396777942\\
115	0.00547708906520158\\
116	0.00547707388134694\\
117	0.00547705841086813\\
118	0.00547704264831543\\
119	0.00547702658813468\\
120	0.00547701022466556\\
121	0.00547699355213936\\
122	0.00547697656467691\\
123	0.00547695925628634\\
124	0.00547694162086135\\
125	0.00547692365217836\\
126	0.00547690534389472\\
127	0.00547688668954638\\
128	0.00547686768254531\\
129	0.00547684831617732\\
130	0.00547682858359955\\
131	0.00547680847783804\\
132	0.00547678799178508\\
133	0.00547676711819685\\
134	0.00547674584969067\\
135	0.00547672417874215\\
136	0.00547670209768282\\
137	0.00547667959869716\\
138	0.00547665667381978\\
139	0.00547663331493242\\
140	0.00547660951376113\\
141	0.00547658526187342\\
142	0.00547656055067495\\
143	0.00547653537140648\\
144	0.00547650971514078\\
145	0.00547648357277933\\
146	0.0054764569350489\\
147	0.00547642979249845\\
148	0.00547640213549555\\
149	0.00547637395422284\\
150	0.00547634523867466\\
151	0.00547631597865313\\
152	0.00547628616376466\\
153	0.0054762557834163\\
154	0.00547622482681152\\
155	0.00547619328294673\\
156	0.00547616114060697\\
157	0.00547612838836183\\
158	0.0054760950145617\\
159	0.00547606100733317\\
160	0.00547602635457487\\
161	0.00547599104395312\\
162	0.00547595506289734\\
163	0.0054759183985957\\
164	0.00547588103799036\\
165	0.0054758429677727\\
166	0.00547580417437866\\
167	0.0054757646439838\\
168	0.00547572436249831\\
169	0.00547568331556171\\
170	0.00547564148853824\\
171	0.00547559886651095\\
172	0.00547555543427678\\
173	0.00547551117634084\\
174	0.00547546607691097\\
175	0.00547542011989205\\
176	0.00547537328888027\\
177	0.00547532556715714\\
178	0.00547527693768367\\
179	0.00547522738309414\\
180	0.00547517688568993\\
181	0.00547512542743321\\
182	0.00547507298994053\\
183	0.00547501955447631\\
184	0.00547496510194596\\
185	0.00547490961288938\\
186	0.00547485306747379\\
187	0.00547479544548693\\
188	0.00547473672632967\\
189	0.00547467688900864\\
190	0.00547461591212905\\
191	0.00547455377388681\\
192	0.0054744904520609\\
193	0.00547442592400535\\
194	0.00547436016664137\\
195	0.00547429315644886\\
196	0.00547422486945841\\
197	0.00547415528124248\\
198	0.0054740843669055\\
199	0.00547401210107523\\
200	0.00547393845789294\\
201	0.00547386341100487\\
202	0.00547378693355224\\
203	0.00547370899816192\\
204	0.00547362957693638\\
205	0.00547354864144361\\
206	0.00547346616270701\\
207	0.00547338211119487\\
208	0.0054732964568095\\
209	0.00547320916887667\\
210	0.00547312021613425\\
211	0.0054730295667209\\
212	0.00547293718816452\\
213	0.0054728430473704\\
214	0.00547274711060898\\
215	0.00547264934350355\\
216	0.00547254971101772\\
217	0.00547244817744224\\
218	0.00547234470638199\\
219	0.0054722392607422\\
220	0.00547213180271464\\
221	0.00547202229376333\\
222	0.00547191069460997\\
223	0.00547179696521914\\
224	0.0054716810647825\\
225	0.00547156295170338\\
226	0.0054714425835802\\
227	0.00547131991719004\\
228	0.00547119490847116\\
229	0.00547106751250553\\
230	0.00547093768350016\\
231	0.00547080537476862\\
232	0.00547067053871114\\
233	0.00547053312679474\\
234	0.00547039308953201\\
235	0.00547025037645972\\
236	0.00547010493611618\\
237	0.00546995671601781\\
238	0.00546980566263517\\
239	0.00546965172136738\\
240	0.00546949483651604\\
241	0.0054693349512578\\
242	0.00546917200761569\\
243	0.00546900594642939\\
244	0.00546883670732378\\
245	0.0054686642286767\\
246	0.00546848844758473\\
247	0.00546830929982718\\
248	0.0054681267198294\\
249	0.00546794064062302\\
250	0.00546775099380574\\
251	0.00546755770949818\\
252	0.00546736071629964\\
253	0.00546715994124117\\
254	0.00546695530973744\\
255	0.00546674674553633\\
256	0.00546653417066647\\
257	0.00546631750538349\\
258	0.00546609666811435\\
259	0.00546587157540045\\
260	0.00546564214183956\\
261	0.00546540828002758\\
262	0.00546516990049957\\
263	0.00546492691167163\\
264	0.00546467921978485\\
265	0.00546442672885071\\
266	0.00546416934060098\\
267	0.00546390695444281\\
268	0.00546363946742059\\
269	0.00546336677418728\\
270	0.0054630887669867\\
271	0.00546280533565021\\
272	0.00546251636760866\\
273	0.00546222174792165\\
274	0.00546192135932096\\
275	0.00546161508225876\\
276	0.0054613027949394\\
277	0.00546098437332235\\
278	0.00546065969126471\\
279	0.00546032862209576\\
280	0.00545999103929571\\
281	0.00545964681381874\\
282	0.00545929581404179\\
283	0.00545893790571285\\
284	0.00545857295189752\\
285	0.00545820081292533\\
286	0.005457821346334\\
287	0.00545743440681342\\
288	0.00545703984614783\\
289	0.00545663751315714\\
290	0.00545622725363689\\
291	0.00545580891029723\\
292	0.00545538232270017\\
293	0.005454947327196\\
294	0.005454503756858\\
295	0.00545405144141609\\
296	0.00545359020718901\\
297	0.0054531198770149\\
298	0.00545264027018071\\
299	0.0054521512023498\\
300	0.00545165248548837\\
301	0.00545114392779012\\
302	0.00545062533359921\\
303	0.00545009650333209\\
304	0.00544955723339684\\
305	0.00544900731611164\\
306	0.00544844653962099\\
307	0.00544787468781012\\
308	0.00544729154021798\\
309	0.00544669687194783\\
310	0.0054460904535762\\
311	0.00544547205105978\\
312	0.00544484142564023\\
313	0.0054441983337471\\
314	0.00544354252689835\\
315	0.00544287375159881\\
316	0.00544219174923633\\
317	0.00544149625597578\\
318	0.00544078700265033\\
319	0.00544006371465047\\
320	0.0054393261118108\\
321	0.00543857390829369\\
322	0.0054378068124707\\
323	0.00543702452680095\\
324	0.00543622674770699\\
325	0.00543541316544767\\
326	0.00543458346398793\\
327	0.00543373732086547\\
328	0.0054328744070547\\
329	0.00543199438682682\\
330	0.00543109691760711\\
331	0.0054301816498285\\
332	0.00542924822678164\\
333	0.00542829628446135\\
334	0.00542732545140942\\
335	0.00542633534855325\\
336	0.00542532558904098\\
337	0.00542429577807218\\
338	0.00542324551272441\\
339	0.00542217438177564\\
340	0.00542108196552196\\
341	0.00541996783559111\\
342	0.00541883155475096\\
343	0.00541767267671351\\
344	0.00541649074593372\\
345	0.00541528529740356\\
346	0.00541405585644107\\
347	0.00541280193847348\\
348	0.00541152304881607\\
349	0.00541021868244466\\
350	0.00540888832376318\\
351	0.00540753144636543\\
352	0.00540614751279144\\
353	0.00540473597427764\\
354	0.00540329627050202\\
355	0.00540182782932307\\
356	0.00540033006651343\\
357	0.00539880238548735\\
358	0.00539724417702318\\
359	0.00539565481897998\\
360	0.00539403367600851\\
361	0.00539238009925782\\
362	0.00539069342607582\\
363	0.00538897297970559\\
364	0.00538721806897695\\
365	0.00538542798799348\\
366	0.00538360201581567\\
367	0.00538173941614002\\
368	0.00537983943697468\\
369	0.00537790131031196\\
370	0.00537592425179702\\
371	0.00537390746039434\\
372	0.00537185011805059\\
373	0.00536975138935397\\
374	0.0053676104211907\\
375	0.00536542634239583\\
376	0.0053631982633991\\
377	0.00536092527586302\\
378	0.00535860645231078\\
379	0.0053562408457403\\
380	0.00535382748921942\\
381	0.00535136539545445\\
382	0.00534885355632346\\
383	0.00534629094236728\\
384	0.00534367650224867\\
385	0.00534100916223877\\
386	0.00533828782582163\\
387	0.00533551137291279\\
388	0.00533267865854607\\
389	0.00532978851220988\\
390	0.00532683973715953\\
391	0.00532383110970253\\
392	0.00532076137845693\\
393	0.00531762926357949\\
394	0.0053144334559635\\
395	0.00531117261640312\\
396	0.00530784537472329\\
397	0.00530445032887193\\
398	0.00530098604397334\\
399	0.00529745105133906\\
400	0.0052938438474343\\
401	0.00529016289279669\\
402	0.00528640661090413\\
403	0.00528257338698845\\
404	0.00527866156679076\\
405	0.00527466945525515\\
406	0.00527059531515573\\
407	0.00526643736565234\\
408	0.0052621937807702\\
409	0.00525786268779764\\
410	0.00525344216559556\\
411	0.00524893024281307\\
412	0.00524432489600113\\
413	0.00523962404761742\\
414	0.00523482556391416\\
415	0.0052299272526993\\
416	0.00522492686096276\\
417	0.00521982207235701\\
418	0.00521461050452102\\
419	0.00520928970623665\\
420	0.00520385715440529\\
421	0.00519831025083372\\
422	0.00519264631881968\\
423	0.00518686259952791\\
424	0.00518095624813875\\
425	0.00517492432971706\\
426	0.00516876381481588\\
427	0.00516247157484287\\
428	0.00515604437712275\\
429	0.00514947887963317\\
430	0.00514277162539182\\
431	0.00513591903646833\\
432	0.00512891740759394\\
433	0.00512176289934059\\
434	0.00511445153083856\\
435	0.005106979171999\\
436	0.00509934153520695\\
437	0.00509153416644783\\
438	0.00508355243582635\\
439	0.00507539152743719\\
440	0.0050670464285423\\
441	0.00505851191800767\\
442	0.00504978255395004\\
443	0.00504085266054141\\
444	0.00503171631391671\\
445	0.00502236732712819\\
446	0.00501279923408703\\
447	0.00500300527243186\\
448	0.00499297836526077\\
449	0.00498271110166135\\
450	0.00497219571597312\\
451	0.00496142406572608\\
452	0.00495038760818735\\
453	0.0049390773754565\\
454	0.00492748394805629\\
455	0.004915597426971\\
456	0.00490340740409602\\
457	0.00489090293107581\\
458	0.00487807248652442\\
459	0.0048649039416482\\
460	0.00485138452431932\\
461	0.00483750078168705\\
462	0.00482323854146321\\
463	0.00480858287207625\\
464	0.0047935180419613\\
465	0.00477802747833561\\
466	0.00476209372589299\\
467	0.00474569840590054\\
468	0.00472882217609919\\
469	0.00471144469131011\\
470	0.00469354456297209\\
471	0.00467509931101878\\
472	0.00465608528883763\\
473	0.0046364775315969\\
474	0.00461624941539552\\
475	0.00459537193149305\\
476	0.00457381089568\\
477	0.00455151205921459\\
478	0.00452843715885329\\
479	0.00450455081417934\\
480	0.00447981684303206\\
481	0.00445419803128292\\
482	0.00442765630872568\\
483	0.00440015300515289\\
484	0.00437164919286085\\
485	0.00434210613156405\\
486	0.00431148583748201\\
487	0.00427975180317468\\
488	0.00424686989968002\\
489	0.00421280949829499\\
490	0.00417754485596785\\
491	0.00414105681581911\\
492	0.00410333488287071\\
493	0.00406437974484788\\
494	0.00402420631952435\\
495	0.00398284742522583\\
496	0.00394035819421055\\
497	0.00389682138917866\\
498	0.00385235385275624\\
499	0.00380711438400887\\
500	0.00376131301138053\\
501	0.00371521852825999\\
502	0.00366915875495774\\
503	0.00362356040215765\\
504	0.00357912772682816\\
505	0.0035387837955858\\
506	0.00350347152895402\\
507	0.00347337562112852\\
508	0.0034477007156403\\
509	0.00342321915374529\\
510	0.0033995558606626\\
511	0.00337624383061699\\
512	0.00335267912856479\\
513	0.00332876510735656\\
514	0.00330442430717686\\
515	0.00327960483711814\\
516	0.00325427310653981\\
517	0.00322840785924007\\
518	0.00320198625777614\\
519	0.00317498308822855\\
520	0.00314738455198443\\
521	0.00311917684391942\\
522	0.00309034579133459\\
523	0.00306087673592382\\
524	0.00303075439888956\\
525	0.00299996271838372\\
526	0.00296848470340479\\
527	0.00293630226281886\\
528	0.00290339600729777\\
529	0.00286974502351109\\
530	0.00283532660238758\\
531	0.00280011576890602\\
532	0.00276406939748543\\
533	0.00272712121191253\\
534	0.0026911965166374\\
535	0.00265655106163712\\
536	0.00262214033891497\\
537	0.00258722058161971\\
538	0.00255170106648381\\
539	0.00251557462242589\\
540	0.00247883838157633\\
541	0.00244149232061441\\
542	0.00240353787403877\\
543	0.00236497792929109\\
544	0.00232581699903547\\
545	0.00228606137899959\\
546	0.00224571921278943\\
547	0.0022048002200509\\
548	0.00216331439627576\\
549	0.00212127465109859\\
550	0.0020787029667516\\
551	0.00203562649819421\\
552	0.00199207845654765\\
553	0.00194934483622062\\
554	0.0019076605974282\\
555	0.00186553102545023\\
556	0.00182295585535523\\
557	0.00177995162674704\\
558	0.00173653654153092\\
559	0.00169273037654859\\
560	0.00164855438448207\\
561	0.0016040311335613\\
562	0.00155918428001908\\
563	0.00151403828525429\\
564	0.0014686181370549\\
565	0.0014229492435685\\
566	0.00137720469839254\\
567	0.00133166599318628\\
568	0.0012858376589304\\
569	0.00123974504677254\\
570	0.00119341585869796\\
571	0.00114688025839211\\
572	0.00110017097212519\\
573	0.00105332337464662\\
574	0.00100637555373387\\
575	0.000959368345375567\\
576	0.00091234532951171\\
577	0.000865352773713771\\
578	0.000818439509057376\\
579	0.000771656718587056\\
580	0.000725057614040402\\
581	0.000678696970709246\\
582	0.000632630483283924\\
583	0.000586913897120322\\
584	0.000541601859672152\\
585	0.000496746426604043\\
586	0.000452395149029594\\
587	0.000408588670365815\\
588	0.000365357795666475\\
589	0.00032272012337526\\
590	0.000280676711614416\\
591	0.000239312817487405\\
592	0.000198788644961841\\
593	0.000159293651685586\\
594	0.000121079888727804\\
595	8.45520570083908e-05\\
596	5.05092148680373e-05\\
597	2.07908715710836e-05\\
598	0\\
599	0\\
600	0\\
};
\addplot [color=mycolor1,solid,forget plot]
  table[row sep=crcr]{%
1	0.00548095199615839\\
2	0.00548094992070687\\
3	0.00548094780715951\\
4	0.00548094565481181\\
5	0.00548094346294612\\
6	0.00548094123083137\\
7	0.00548093895772277\\
8	0.00548093664286176\\
9	0.00548093428547568\\
10	0.00548093188477729\\
11	0.00548092943996483\\
12	0.00548092695022149\\
13	0.00548092441471523\\
14	0.00548092183259856\\
15	0.00548091920300802\\
16	0.00548091652506419\\
17	0.00548091379787118\\
18	0.00548091102051634\\
19	0.00548090819207006\\
20	0.00548090531158525\\
21	0.00548090237809721\\
22	0.00548089939062314\\
23	0.005480896348162\\
24	0.00548089324969399\\
25	0.00548089009418024\\
26	0.00548088688056246\\
27	0.00548088360776255\\
28	0.00548088027468232\\
29	0.00548087688020299\\
30	0.00548087342318489\\
31	0.0054808699024671\\
32	0.00548086631686685\\
33	0.00548086266517931\\
34	0.00548085894617706\\
35	0.00548085515860977\\
36	0.0054808513012037\\
37	0.0054808473726612\\
38	0.00548084337166045\\
39	0.00548083929685487\\
40	0.00548083514687246\\
41	0.00548083092031579\\
42	0.00548082661576104\\
43	0.00548082223175787\\
44	0.00548081776682869\\
45	0.00548081321946827\\
46	0.00548080858814314\\
47	0.00548080387129106\\
48	0.00548079906732057\\
49	0.00548079417461039\\
50	0.00548078919150867\\
51	0.00548078411633274\\
52	0.00548077894736843\\
53	0.00548077368286925\\
54	0.00548076832105606\\
55	0.0054807628601163\\
56	0.00548075729820337\\
57	0.00548075163343604\\
58	0.00548074586389776\\
59	0.00548073998763612\\
60	0.00548073400266186\\
61	0.00548072790694861\\
62	0.00548072169843166\\
63	0.00548071537500777\\
64	0.00548070893453403\\
65	0.00548070237482737\\
66	0.00548069569366378\\
67	0.00548068888877726\\
68	0.00548068195785944\\
69	0.00548067489855835\\
70	0.00548066770847799\\
71	0.00548066038517733\\
72	0.00548065292616929\\
73	0.00548064532892022\\
74	0.00548063759084875\\
75	0.00548062970932491\\
76	0.00548062168166931\\
77	0.00548061350515221\\
78	0.00548060517699245\\
79	0.00548059669435663\\
80	0.00548058805435795\\
81	0.00548057925405533\\
82	0.00548057029045237\\
83	0.00548056116049621\\
84	0.00548055186107659\\
85	0.00548054238902466\\
86	0.00548053274111202\\
87	0.00548052291404934\\
88	0.0054805129044855\\
89	0.00548050270900618\\
90	0.00548049232413272\\
91	0.00548048174632115\\
92	0.00548047097196047\\
93	0.00548045999737188\\
94	0.00548044881880714\\
95	0.00548043743244753\\
96	0.00548042583440221\\
97	0.00548041402070723\\
98	0.00548040198732382\\
99	0.00548038973013709\\
100	0.0054803772449546\\
101	0.00548036452750497\\
102	0.00548035157343624\\
103	0.00548033837831449\\
104	0.00548032493762212\\
105	0.00548031124675644\\
106	0.00548029730102794\\
107	0.00548028309565878\\
108	0.00548026862578087\\
109	0.00548025388643441\\
110	0.00548023887256618\\
111	0.00548022357902757\\
112	0.0054802080005728\\
113	0.00548019213185738\\
114	0.00548017596743578\\
115	0.00548015950175985\\
116	0.00548014272917685\\
117	0.00548012564392743\\
118	0.00548010824014352\\
119	0.00548009051184649\\
120	0.00548007245294482\\
121	0.00548005405723204\\
122	0.00548003531838467\\
123	0.00548001622996009\\
124	0.00547999678539385\\
125	0.00547997697799801\\
126	0.00547995680095824\\
127	0.00547993624733152\\
128	0.00547991531004419\\
129	0.00547989398188898\\
130	0.00547987225552268\\
131	0.00547985012346352\\
132	0.00547982757808871\\
133	0.00547980461163155\\
134	0.00547978121617879\\
135	0.00547975738366809\\
136	0.00547973310588491\\
137	0.00547970837445984\\
138	0.00547968318086549\\
139	0.00547965751641371\\
140	0.00547963137225254\\
141	0.00547960473936312\\
142	0.00547957760855628\\
143	0.0054795499704698\\
144	0.00547952181556483\\
145	0.00547949313412261\\
146	0.00547946391624125\\
147	0.00547943415183217\\
148	0.00547940383061655\\
149	0.00547937294212185\\
150	0.00547934147567809\\
151	0.00547930942041437\\
152	0.00547927676525471\\
153	0.00547924349891452\\
154	0.00547920960989667\\
155	0.00547917508648734\\
156	0.00547913991675218\\
157	0.00547910408853202\\
158	0.00547906758943872\\
159	0.00547903040685087\\
160	0.0054789925279096\\
161	0.0054789539395139\\
162	0.00547891462831629\\
163	0.00547887458071825\\
164	0.00547883378286538\\
165	0.00547879222064283\\
166	0.0054787498796703\\
167	0.00547870674529745\\
168	0.00547866280259838\\
169	0.00547861803636722\\
170	0.00547857243111221\\
171	0.00547852597105114\\
172	0.0054784786401056\\
173	0.00547843042189563\\
174	0.00547838129973434\\
175	0.0054783312566222\\
176	0.00547828027524145\\
177	0.00547822833795021\\
178	0.00547817542677671\\
179	0.00547812152341338\\
180	0.00547806660921088\\
181	0.0054780106651719\\
182	0.00547795367194498\\
183	0.00547789560981841\\
184	0.00547783645871397\\
185	0.00547777619817999\\
186	0.00547771480738572\\
187	0.00547765226511421\\
188	0.00547758854975598\\
189	0.00547752363930238\\
190	0.00547745751133886\\
191	0.00547739014303824\\
192	0.00547732151115371\\
193	0.00547725159201235\\
194	0.00547718036150766\\
195	0.00547710779509384\\
196	0.00547703386778093\\
197	0.00547695855413523\\
198	0.00547688182828359\\
199	0.00547680366388242\\
200	0.00547672403410897\\
201	0.00547664291165308\\
202	0.00547656026870942\\
203	0.00547647607696921\\
204	0.00547639030761201\\
205	0.00547630293129755\\
206	0.00547621391815725\\
207	0.00547612323778595\\
208	0.00547603085923324\\
209	0.00547593675099499\\
210	0.00547584088100468\\
211	0.00547574321662483\\
212	0.00547564372463806\\
213	0.0054755423712384\\
214	0.00547543912202255\\
215	0.00547533394198096\\
216	0.00547522679548875\\
217	0.00547511764629692\\
218	0.00547500645752335\\
219	0.00547489319164394\\
220	0.00547477781048339\\
221	0.00547466027520632\\
222	0.00547454054630832\\
223	0.00547441858360674\\
224	0.00547429434623182\\
225	0.00547416779261774\\
226	0.00547403888049354\\
227	0.00547390756687405\\
228	0.00547377380805125\\
229	0.00547363755958502\\
230	0.00547349877629423\\
231	0.00547335741224781\\
232	0.005473213420756\\
233	0.00547306675436082\\
234	0.00547291736482763\\
235	0.0054727652031355\\
236	0.00547261021946829\\
237	0.00547245236320512\\
238	0.0054722915829106\\
239	0.00547212782632512\\
240	0.00547196104035452\\
241	0.00547179117105935\\
242	0.00547161816364347\\
243	0.00547144196244214\\
244	0.00547126251090899\\
245	0.00547107975160189\\
246	0.00547089362616763\\
247	0.00547070407532488\\
248	0.00547051103884505\\
249	0.00547031445553121\\
250	0.00547011426319354\\
251	0.00546991039862183\\
252	0.00546970279755407\\
253	0.00546949139464002\\
254	0.00546927612339942\\
255	0.00546905691617357\\
256	0.0054688337040694\\
257	0.00546860641689464\\
258	0.00546837498308231\\
259	0.00546813932960382\\
260	0.00546789938186764\\
261	0.00546765506360236\\
262	0.00546740629672147\\
263	0.00546715300116732\\
264	0.00546689509473118\\
265	0.00546663249284685\\
266	0.00546636510835429\\
267	0.00546609285122961\\
268	0.00546581562827825\\
269	0.00546553334278746\\
270	0.00546524589413507\\
271	0.00546495317735124\\
272	0.00546465508262966\\
273	0.0054643514947835\\
274	0.00546404229263197\\
275	0.00546372734826881\\
276	0.00546340652603965\\
277	0.0054630796806094\\
278	0.00546274665192147\\
279	0.00546240724915325\\
280	0.00546206128609614\\
281	0.00546170863564718\\
282	0.00546134916823953\\
283	0.00546098275179508\\
284	0.00546060925167596\\
285	0.00546022853063473\\
286	0.00545984044876465\\
287	0.00545944486344781\\
288	0.00545904162930304\\
289	0.00545863059813262\\
290	0.00545821161886802\\
291	0.00545778453751446\\
292	0.00545734919709457\\
293	0.00545690543759074\\
294	0.00545645309588682\\
295	0.00545599200570826\\
296	0.00545552199756122\\
297	0.00545504289867078\\
298	0.0054545545329177\\
299	0.0054540567207738\\
300	0.00545354927923665\\
301	0.00545303202176261\\
302	0.00545250475819857\\
303	0.00545196729471259\\
304	0.00545141943372308\\
305	0.00545086097382662\\
306	0.00545029170972414\\
307	0.00544971143214647\\
308	0.00544911992777732\\
309	0.00544851697917555\\
310	0.00544790236469576\\
311	0.00544727585840716\\
312	0.0054466372300107\\
313	0.00544598624475502\\
314	0.00544532266335037\\
315	0.00544464624188075\\
316	0.00544395673171442\\
317	0.00544325387941254\\
318	0.00544253742663585\\
319	0.00544180711004963\\
320	0.00544106266122603\\
321	0.00544030380654516\\
322	0.00543953026709337\\
323	0.00543874175855965\\
324	0.00543793799112967\\
325	0.00543711866937707\\
326	0.0054362834921529\\
327	0.00543543215247208\\
328	0.005434564337397\\
329	0.00543367972791881\\
330	0.00543277799883512\\
331	0.00543185881862527\\
332	0.0054309218493221\\
333	0.00542996674638014\\
334	0.00542899315854046\\
335	0.00542800072769201\\
336	0.00542698908872839\\
337	0.00542595786940121\\
338	0.0054249066901685\\
339	0.00542383516403881\\
340	0.00542274289641018\\
341	0.0054216294849042\\
342	0.00542049451919431\\
343	0.00541933758082834\\
344	0.00541815824304497\\
345	0.00541695607058338\\
346	0.00541573061948593\\
347	0.00541448143689398\\
348	0.00541320806083489\\
349	0.00541191002000126\\
350	0.00541058683352134\\
351	0.00540923801071954\\
352	0.00540786305086729\\
353	0.00540646144292367\\
354	0.00540503266526402\\
355	0.00540357618539771\\
356	0.00540209145967306\\
357	0.00540057793296957\\
358	0.0053990350383768\\
359	0.00539746219685888\\
360	0.00539585881690556\\
361	0.00539422429416723\\
362	0.00539255801107627\\
363	0.00539085933645289\\
364	0.00538912762509577\\
365	0.0053873622173592\\
366	0.00538556243871585\\
367	0.00538372759930749\\
368	0.00538185699348492\\
369	0.0053799498993387\\
370	0.00537800557822563\\
371	0.00537602327429234\\
372	0.0053740022140025\\
373	0.00537194160567334\\
374	0.00536984063902765\\
375	0.00536769848477218\\
376	0.00536551429421052\\
377	0.00536328719890381\\
378	0.00536101631039253\\
379	0.00535870071999319\\
380	0.00535633949868242\\
381	0.00535393169707442\\
382	0.00535147634547554\\
383	0.00534897245395576\\
384	0.00534641901230657\\
385	0.00534381498980157\\
386	0.00534115933575377\\
387	0.00533845098860062\\
388	0.00533568888069135\\
389	0.00533287192107836\\
390	0.00532999899488533\\
391	0.00532706896265217\\
392	0.00532408065965528\\
393	0.00532103289520269\\
394	0.00531792445190184\\
395	0.00531475408490083\\
396	0.00531152052110063\\
397	0.00530822245833811\\
398	0.00530485856453716\\
399	0.00530142747682666\\
400	0.00529792780062444\\
401	0.00529435810868445\\
402	0.00529071694010634\\
403	0.00528700279930482\\
404	0.00528321415493732\\
405	0.00527934943878715\\
406	0.00527540704460026\\
407	0.00527138532687276\\
408	0.00526728259958638\\
409	0.00526309713488923\\
410	0.00525882716171782\\
411	0.00525447086435603\\
412	0.0052500263809267\\
413	0.00524549180180995\\
414	0.00524086516798206\\
415	0.00523614446926672\\
416	0.00523132764248807\\
417	0.00522641256951514\\
418	0.00522139707518676\\
419	0.00521627892509709\\
420	0.00521105582321913\\
421	0.00520572540934116\\
422	0.00520028525629376\\
423	0.00519473286698471\\
424	0.0051890656713553\\
425	0.00518328102341414\\
426	0.00517737619710569\\
427	0.00517134838138553\\
428	0.0051651946761483\\
429	0.00515891208789345\\
430	0.00515249752510843\\
431	0.00514594779334966\\
432	0.00513925959000025\\
433	0.00513242949867982\\
434	0.00512545398328218\\
435	0.00511832938161421\\
436	0.00511105189860727\\
437	0.00510361759907029\\
438	0.00509602239995316\\
439	0.00508826206208469\\
440	0.00508033218134922\\
441	0.00507222817926275\\
442	0.00506394529290762\\
443	0.00505547856418199\\
444	0.005046822828319\\
445	0.00503797270162797\\
446	0.00502892256840949\\
447	0.00501966656699718\\
448	0.00501019857488201\\
449	0.00500051219286966\\
450	0.00499060072818567\\
451	0.0049804571764038\\
452	0.00497007420233899\\
453	0.00495944411973041\\
454	0.00494855886965451\\
455	0.00493740999761189\\
456	0.00492598862923287\\
457	0.00491428544455216\\
458	0.00490229065080857\\
459	0.00488999395373417\\
460	0.00487738452730974\\
461	0.00486445098197741\\
462	0.00485118133132173\\
463	0.00483756295725451\\
464	0.00482358257377012\\
465	0.00480922618937484\\
466	0.00479447906833696\\
467	0.00477932569094731\\
468	0.00476374971300784\\
469	0.00474773392474334\\
470	0.00473126020916734\\
471	0.00471430949936953\\
472	0.00469686173269478\\
473	0.00467889579597215\\
474	0.00466038944590183\\
475	0.0046413191586662\\
476	0.00462165983497246\\
477	0.0046013846366147\\
478	0.00458046289322046\\
479	0.00455885105141006\\
480	0.00453650300946854\\
481	0.00451338434121667\\
482	0.00448945977358611\\
483	0.00446469277837288\\
484	0.0044390455330161\\
485	0.00441247901676345\\
486	0.00438495317490672\\
487	0.00435642714567625\\
488	0.00432685956560797\\
489	0.00429620897197185\\
490	0.00426443432577805\\
491	0.00423149568450701\\
492	0.00419735506082228\\
493	0.00416197751252801\\
494	0.00412533252053638\\
495	0.00408739572641924\\
496	0.00404815112021496\\
497	0.00400759379329441\\
498	0.00396573339945054\\
499	0.00392259849595016\\
500	0.00387824197248343\\
501	0.00383274799840133\\
502	0.00378624154179356\\
503	0.00373890021273193\\
504	0.00369096392134772\\
505	0.00364273454391488\\
506	0.00359457071864176\\
507	0.00354702042024983\\
508	0.00350131809110213\\
509	0.00346061049916525\\
510	0.00342502183026241\\
511	0.00339439861452269\\
512	0.00336708322036617\\
513	0.00334086691234794\\
514	0.00331534730435857\\
515	0.00329008181911496\\
516	0.00326469884107946\\
517	0.00323892834194342\\
518	0.00321269198886212\\
519	0.00318593354574616\\
520	0.00315860607848959\\
521	0.00313068753007561\\
522	0.00310216113399003\\
523	0.00307301064759331\\
524	0.00304322031745423\\
525	0.00301277483353352\\
526	0.00298165833580482\\
527	0.00294985424466898\\
528	0.00291734506982389\\
529	0.00288411216856215\\
530	0.00285013547640853\\
531	0.00281539321418112\\
532	0.00277986195570541\\
533	0.00274351740193557\\
534	0.00270633319570252\\
535	0.00266822518666434\\
536	0.00263037485873978\\
537	0.0025936689644631\\
538	0.00255802540455848\\
539	0.00252190230147721\\
540	0.00248520925445124\\
541	0.00244790403567086\\
542	0.00240998217519735\\
543	0.00237144483485123\\
544	0.00233229481254299\\
545	0.00229253661599072\\
546	0.0022521764177109\\
547	0.00221122188822311\\
548	0.00216968132057519\\
549	0.00212756246687727\\
550	0.00208488317727384\\
551	0.00204166550220163\\
552	0.00199793669477371\\
553	0.0019537309880139\\
554	0.00190965922034648\\
555	0.00186720834870374\\
556	0.00182456308654889\\
557	0.00178148227067175\\
558	0.0017379831919425\\
559	0.0016940856717289\\
560	0.00164981119638615\\
561	0.00160518279977741\\
562	0.00156022488653992\\
563	0.00151496298230589\\
564	0.00146942342078429\\
565	0.00142363302680362\\
566	0.00137761899412085\\
567	0.00133166599321896\\
568	0.00128583765893305\\
569	0.0012397450467737\\
570	0.00119341585869848\\
571	0.00114688025839233\\
572	0.00110017097212528\\
573	0.00105332337464666\\
574	0.00100637555373387\\
575	0.000959368345375562\\
576	0.000912345329511703\\
577	0.000865352773713766\\
578	0.000818439509057374\\
579	0.00077165671858706\\
580	0.00072505761404041\\
581	0.000678696970709248\\
582	0.000632630483283932\\
583	0.000586913897120327\\
584	0.000541601859672158\\
585	0.00049674642660404\\
586	0.000452395149029591\\
587	0.000408588670365814\\
588	0.000365357795666474\\
589	0.000322720123375262\\
590	0.000280676711614417\\
591	0.000239312817487403\\
592	0.000198788644961841\\
593	0.000159293651685586\\
594	0.000121079888727805\\
595	8.45520570083909e-05\\
596	5.05092148680371e-05\\
597	2.07908715710836e-05\\
598	0\\
599	0\\
600	0\\
};
\addplot [color=mycolor2,solid,forget plot]
  table[row sep=crcr]{%
1	0.0054891968694182\\
2	0.00548919435537706\\
3	0.00548919179592685\\
4	0.00548918919024169\\
5	0.00548918653748061\\
6	0.00548918383678725\\
7	0.00548918108728961\\
8	0.00548917828809958\\
9	0.00548917543831285\\
10	0.00548917253700857\\
11	0.00548916958324894\\
12	0.00548916657607902\\
13	0.00548916351452631\\
14	0.00548916039760048\\
15	0.00548915722429305\\
16	0.00548915399357712\\
17	0.00548915070440682\\
18	0.00548914735571718\\
19	0.00548914394642359\\
20	0.0054891404754216\\
21	0.00548913694158649\\
22	0.00548913334377289\\
23	0.00548912968081434\\
24	0.00548912595152302\\
25	0.00548912215468925\\
26	0.00548911828908118\\
27	0.00548911435344429\\
28	0.00548911034650102\\
29	0.0054891062669503\\
30	0.00548910211346729\\
31	0.00548909788470254\\
32	0.005489093579282\\
33	0.00548908919580631\\
34	0.00548908473285036\\
35	0.0054890801889628\\
36	0.00548907556266564\\
37	0.00548907085245371\\
38	0.00548906605679403\\
39	0.00548906117412548\\
40	0.0054890562028583\\
41	0.00548905114137334\\
42	0.00548904598802173\\
43	0.00548904074112418\\
44	0.00548903539897048\\
45	0.00548902995981902\\
46	0.00548902442189594\\
47	0.00548901878339485\\
48	0.00548901304247607\\
49	0.00548900719726596\\
50	0.00548900124585643\\
51	0.00548899518630413\\
52	0.00548898901662991\\
53	0.0054889827348182\\
54	0.00548897633881622\\
55	0.00548896982653331\\
56	0.00548896319584039\\
57	0.00548895644456892\\
58	0.00548894957051057\\
59	0.005488942571416\\
60	0.00548893544499462\\
61	0.00548892818891339\\
62	0.00548892080079643\\
63	0.00548891327822374\\
64	0.00548890561873086\\
65	0.00548889781980776\\
66	0.00548888987889793\\
67	0.00548888179339792\\
68	0.00548887356065595\\
69	0.00548886517797149\\
70	0.00548885664259395\\
71	0.00548884795172195\\
72	0.0054888391025024\\
73	0.00548883009202948\\
74	0.00548882091734358\\
75	0.00548881157543053\\
76	0.0054888020632204\\
77	0.00548879237758641\\
78	0.00548878251534416\\
79	0.00548877247325024\\
80	0.00548876224800123\\
81	0.00548875183623264\\
82	0.00548874123451786\\
83	0.0054887304393668\\
84	0.00548871944722482\\
85	0.00548870825447156\\
86	0.00548869685741956\\
87	0.00548868525231328\\
88	0.00548867343532759\\
89	0.00548866140256658\\
90	0.00548864915006227\\
91	0.00548863667377318\\
92	0.00548862396958306\\
93	0.00548861103329949\\
94	0.00548859786065239\\
95	0.00548858444729253\\
96	0.00548857078879037\\
97	0.00548855688063409\\
98	0.00548854271822849\\
99	0.00548852829689328\\
100	0.0054885136118614\\
101	0.00548849865827747\\
102	0.00548848343119637\\
103	0.00548846792558115\\
104	0.00548845213630171\\
105	0.00548843605813288\\
106	0.0054884196857527\\
107	0.00548840301374054\\
108	0.00548838603657553\\
109	0.0054883687486344\\
110	0.00548835114418964\\
111	0.00548833321740775\\
112	0.00548831496234715\\
113	0.00548829637295601\\
114	0.00548827744307067\\
115	0.00548825816641306\\
116	0.00548823853658886\\
117	0.00548821854708523\\
118	0.00548819819126881\\
119	0.00548817746238323\\
120	0.00548815635354697\\
121	0.00548813485775122\\
122	0.00548811296785711\\
123	0.0054880906765936\\
124	0.00548806797655505\\
125	0.00548804486019857\\
126	0.00548802131984158\\
127	0.00548799734765942\\
128	0.00548797293568226\\
129	0.00548794807579285\\
130	0.00548792275972361\\
131	0.00548789697905397\\
132	0.00548787072520739\\
133	0.00548784398944858\\
134	0.00548781676288066\\
135	0.00548778903644202\\
136	0.00548776080090342\\
137	0.00548773204686483\\
138	0.00548770276475235\\
139	0.00548767294481489\\
140	0.00548764257712121\\
141	0.00548761165155607\\
142	0.00548758015781746\\
143	0.00548754808541279\\
144	0.00548751542365553\\
145	0.00548748216166155\\
146	0.00548744828834564\\
147	0.00548741379241772\\
148	0.00548737866237903\\
149	0.00548734288651851\\
150	0.00548730645290866\\
151	0.00548726934940173\\
152	0.00548723156362574\\
153	0.00548719308298007\\
154	0.00548715389463163\\
155	0.00548711398551041\\
156	0.00548707334230508\\
157	0.00548703195145868\\
158	0.00548698979916422\\
159	0.0054869468713599\\
160	0.00548690315372438\\
161	0.00548685863167226\\
162	0.0054868132903491\\
163	0.00548676711462645\\
164	0.00548672008909692\\
165	0.00548667219806894\\
166	0.00548662342556174\\
167	0.00548657375529968\\
168	0.00548652317070723\\
169	0.00548647165490311\\
170	0.00548641919069484\\
171	0.00548636576057296\\
172	0.00548631134670515\\
173	0.00548625593093042\\
174	0.00548619949475278\\
175	0.00548614201933525\\
176	0.00548608348549331\\
177	0.00548602387368881\\
178	0.00548596316402304\\
179	0.00548590133623029\\
180	0.00548583836967079\\
181	0.00548577424332394\\
182	0.00548570893578119\\
183	0.00548564242523852\\
184	0.00548557468948928\\
185	0.00548550570591667\\
186	0.00548543545148583\\
187	0.00548536390273596\\
188	0.00548529103577244\\
189	0.00548521682625827\\
190	0.00548514124940548\\
191	0.0054850642799659\\
192	0.00548498589222079\\
193	0.00548490605996795\\
194	0.00548482475650521\\
195	0.00548474195460491\\
196	0.00548465762647844\\
197	0.0054845717437398\\
198	0.00548448427744685\\
199	0.00548439519858112\\
200	0.0054843044776251\\
201	0.00548421208453498\\
202	0.00548411798873232\\
203	0.00548402215909528\\
204	0.00548392456395014\\
205	0.00548382517106224\\
206	0.00548372394762747\\
207	0.00548362086026317\\
208	0.0054835158749993\\
209	0.00548340895726937\\
210	0.00548330007190167\\
211	0.00548318918310984\\
212	0.0054830762544842\\
213	0.00548296124898257\\
214	0.00548284412892102\\
215	0.00548272485596526\\
216	0.00548260339112148\\
217	0.00548247969472741\\
218	0.0054823537264438\\
219	0.00548222544524543\\
220	0.00548209480941283\\
221	0.00548196177652399\\
222	0.00548182630344604\\
223	0.00548168834632761\\
224	0.00548154786059131\\
225	0.00548140480092645\\
226	0.00548125912128235\\
227	0.00548111077486217\\
228	0.00548095971411708\\
229	0.00548080589074135\\
230	0.00548064925566799\\
231	0.00548048975906534\\
232	0.00548032735033443\\
233	0.00548016197810786\\
234	0.00547999359024942\\
235	0.00547982213385538\\
236	0.00547964755525738\\
237	0.00547946980002722\\
238	0.00547928881298338\\
239	0.00547910453820034\\
240	0.00547891691901986\\
241	0.00547872589806595\\
242	0.00547853141726252\\
243	0.00547833341785528\\
244	0.00547813184043755\\
245	0.00547792662498099\\
246	0.00547771771087146\\
247	0.00547750503695109\\
248	0.00547728854156713\\
249	0.00547706816262813\\
250	0.00547684383766866\\
251	0.00547661550392378\\
252	0.00547638309841347\\
253	0.00547614655803969\\
254	0.00547590581969587\\
255	0.00547566082039141\\
256	0.0054754114973923\\
257	0.0054751577883796\\
258	0.00547489963162761\\
259	0.00547463696620386\\
260	0.00547436973219279\\
261	0.00547409787094485\\
262	0.00547382132535371\\
263	0.00547354004016329\\
264	0.00547325396230583\\
265	0.00547296304127366\\
266	0.00547266722952463\\
267	0.00547236648292219\\
268	0.00547206076120906\\
269	0.00547175002851286\\
270	0.00547143425387946\\
271	0.00547111341182841\\
272	0.00547078748292199\\
273	0.00547045645433792\\
274	0.00547012032043892\\
275	0.00546977908334958\\
276	0.00546943275361923\\
277	0.00546908135127342\\
278	0.00546872490828844\\
279	0.00546836347592409\\
280	0.00546799639134473\\
281	0.00546762240864155\\
282	0.00546724139884568\\
283	0.00546685323060331\\
284	0.00546645777013235\\
285	0.00546605488117901\\
286	0.00546564442497286\\
287	0.00546522626018168\\
288	0.00546480024286546\\
289	0.00546436622642938\\
290	0.00546392406157631\\
291	0.00546347359625828\\
292	0.00546301467562746\\
293	0.00546254714198608\\
294	0.0054620708347355\\
295	0.00546158559032467\\
296	0.00546109124219773\\
297	0.00546058762074055\\
298	0.00546007455322658\\
299	0.00545955186376227\\
300	0.00545901937323075\\
301	0.00545847689923528\\
302	0.00545792425604187\\
303	0.00545736125452067\\
304	0.00545678770208683\\
305	0.00545620340264023\\
306	0.00545560815650464\\
307	0.00545500176036559\\
308	0.0054543840072078\\
309	0.00545375468625155\\
310	0.0054531135828881\\
311	0.00545246047861439\\
312	0.00545179515096701\\
313	0.00545111737345481\\
314	0.00545042691549136\\
315	0.00544972354232614\\
316	0.00544900701497484\\
317	0.0054482770901493\\
318	0.00544753352018626\\
319	0.00544677605297551\\
320	0.00544600443188722\\
321	0.00544521839569857\\
322	0.00544441767851983\\
323	0.00544360200971929\\
324	0.00544277111384802\\
325	0.00544192471056379\\
326	0.00544106251455431\\
327	0.00544018423545993\\
328	0.00543928957779588\\
329	0.00543837824087373\\
330	0.00543744991872266\\
331	0.00543650430001013\\
332	0.00543554106796177\\
333	0.0054345599002812\\
334	0.00543356046906949\\
335	0.00543254244074374\\
336	0.00543150547595554\\
337	0.00543044922950901\\
338	0.00542937335027815\\
339	0.00542827748112374\\
340	0.00542716125880973\\
341	0.00542602431391878\\
342	0.00542486627076723\\
343	0.00542368674731854\\
344	0.00542248535509619\\
345	0.00542126169909461\\
346	0.00542001537768867\\
347	0.00541874598254035\\
348	0.00541745309850346\\
349	0.00541613630352446\\
350	0.00541479516853939\\
351	0.00541342925736604\\
352	0.00541203812658974\\
353	0.00541062132544212\\
354	0.00540917839567121\\
355	0.00540770887140096\\
356	0.00540621227897795\\
357	0.00540468813680368\\
358	0.00540313595514848\\
359	0.00540155523594473\\
360	0.00539994547255471\\
361	0.00539830614950943\\
362	0.00539663674221262\\
363	0.00539493671660397\\
364	0.00539320552877565\\
365	0.00539144262453276\\
366	0.00538964743889032\\
367	0.00538781939549552\\
368	0.0053859579059644\\
369	0.00538406236912007\\
370	0.00538213217011732\\
371	0.00538016667943867\\
372	0.00537816525174382\\
373	0.00537612722455306\\
374	0.00537405191674505\\
375	0.00537193862684648\\
376	0.00536978663109177\\
377	0.00536759518123094\\
378	0.00536536350206276\\
379	0.00536309078867397\\
380	0.00536077620336287\\
381	0.00535841887221443\\
382	0.00535601788124313\\
383	0.00535357227181194\\
384	0.00535108103429341\\
385	0.0053485430962889\\
386	0.00534595729230035\\
387	0.00534332226783449\\
388	0.00534063667519442\\
389	0.00533789953008658\\
390	0.00533510982741454\\
391	0.0053322665406314\\
392	0.00532936862105515\\
393	0.00532641499714516\\
394	0.00532340457373629\\
395	0.00532033623122087\\
396	0.00531720882469156\\
397	0.00531402118304865\\
398	0.00531077210806497\\
399	0.00530746037340651\\
400	0.00530408472360693\\
401	0.00530064387299334\\
402	0.0052971365045622\\
403	0.00529356126880294\\
404	0.00528991678246438\\
405	0.00528620162727035\\
406	0.00528241434857942\\
407	0.00527855345397782\\
408	0.00527461741181547\\
409	0.0052706046496879\\
410	0.00526651355286785\\
411	0.00526234246269237\\
412	0.00525808967491305\\
413	0.00525375343801949\\
414	0.00524933195154946\\
415	0.00524482336440346\\
416	0.00524022577318039\\
417	0.00523553722053481\\
418	0.00523075569352813\\
419	0.00522587912211942\\
420	0.00522090537774521\\
421	0.00521583227193822\\
422	0.00521065755485279\\
423	0.00520537891343096\\
424	0.00519999396908422\\
425	0.00519450027717447\\
426	0.00518889534523272\\
427	0.00518317663173477\\
428	0.00517734151593173\\
429	0.00517138729411216\\
430	0.00516531117563831\\
431	0.00515911027874116\\
432	0.00515278162605946\\
433	0.00514632213990524\\
434	0.00513972863723734\\
435	0.00513299782432283\\
436	0.00512612629106462\\
437	0.0051191105049721\\
438	0.00511194680474882\\
439	0.00510463139346928\\
440	0.00509716033131396\\
441	0.0050895295278284\\
442	0.00508173473366834\\
443	0.00507377153178841\\
444	0.00506563532802464\\
445	0.00505732134101474\\
446	0.00504882459138982\\
447	0.00504013989016983\\
448	0.00503126182632091\\
449	0.00502218475354018\\
450	0.00501290277652464\\
451	0.00500340973646137\\
452	0.0049936991921593\\
453	0.00498376440288637\\
454	0.00497359831009461\\
455	0.00496319351796841\\
456	0.00495254227273068\\
457	0.00494163644064148\\
458	0.00493046748462266\\
459	0.0049190264394442\\
460	0.00490730388540931\\
461	0.00489528992047967\\
462	0.00488297413078798\\
463	0.00487034555949376\\
464	0.00485739267395021\\
465	0.004844103331166\\
466	0.00483046474156964\\
467	0.00481646343111338\\
468	0.00480208520178408\\
469	0.00478731509058795\\
470	0.00477213732701297\\
471	0.00475653528923159\\
472	0.00474049145901888\\
473	0.00472398737499751\\
474	0.00470700358311135\\
475	0.00468951958190851\\
476	0.00467151375681195\\
477	0.00465296326746879\\
478	0.00463384386379001\\
479	0.00461412987626572\\
480	0.00459379379728429\\
481	0.00457280301484697\\
482	0.00455110433656644\\
483	0.00452865953063785\\
484	0.00450543286302288\\
485	0.00448138762584243\\
486	0.0044564855239955\\
487	0.00443068671594453\\
488	0.00440394987873579\\
489	0.00437623233356901\\
490	0.00434749022961062\\
491	0.00431767880121355\\
492	0.00428675271652006\\
493	0.00425466653997079\\
494	0.00422137533634864\\
495	0.00418683545021244\\
496	0.00415100550209224\\
497	0.00411384765185985\\
498	0.00407532919063057\\
499	0.00403542453652763\\
500	0.00399411772952133\\
501	0.00395140554229041\\
502	0.00390730132316084\\
503	0.00386183971978261\\
504	0.00381508261869486\\
505	0.00376712721209702\\
506	0.00371811743437225\\
507	0.0036682546917953\\
508	0.00361780571383835\\
509	0.00356710002107475\\
510	0.00351655779440234\\
511	0.00346685990014391\\
512	0.00342018933600302\\
513	0.00337865550221436\\
514	0.00334233331140309\\
515	0.00331092069354877\\
516	0.00328172789297185\\
517	0.00325356226126486\\
518	0.00322599612173834\\
519	0.00319856266008969\\
520	0.00317113359201397\\
521	0.00314330028321649\\
522	0.00311494865759424\\
523	0.00308604010602848\\
524	0.00305653561648579\\
525	0.00302639862942524\\
526	0.00299561110095096\\
527	0.00296415546625934\\
528	0.00293201443441139\\
529	0.00289917111057011\\
530	0.00286560835076743\\
531	0.00283130815884521\\
532	0.00279625143981421\\
533	0.0027604176819128\\
534	0.00272378461722965\\
535	0.00268632921308865\\
536	0.00264802742428661\\
537	0.00260881862072326\\
538	0.00256896387323768\\
539	0.00253012203665061\\
540	0.00249255684260618\\
541	0.00245517426589838\\
542	0.00241727949581605\\
543	0.00237877084369842\\
544	0.00233964426046532\\
545	0.00229989957797116\\
546	0.00225954118691839\\
547	0.00221857506323328\\
548	0.00217700799179227\\
549	0.00213484522472573\\
550	0.00209210022016971\\
551	0.00204879104734812\\
552	0.00200493976528336\\
553	0.00196057319704059\\
554	0.00191572431648802\\
555	0.00187042054078887\\
556	0.00182648374393835\\
557	0.00178330361132209\\
558	0.00173971373924418\\
559	0.00169571747444497\\
560	0.00165133624243625\\
561	0.00160659337207\\
562	0.00156151382896341\\
563	0.00151612401849032\\
564	0.00147045151570613\\
565	0.00142452473202305\\
566	0.00137837259118908\\
567	0.00133202429968801\\
568	0.00128583765932925\\
569	0.00123974504679409\\
570	0.00119341585870728\\
571	0.00114688025839642\\
572	0.0011001709721271\\
573	0.00105332337464741\\
574	0.00100637555373416\\
575	0.000959368345375665\\
576	0.000912345329511744\\
577	0.000865352773713785\\
578	0.000818439509057384\\
579	0.00077165671858706\\
580	0.000725057614040401\\
581	0.000678696970709242\\
582	0.000632630483283921\\
583	0.000586913897120319\\
584	0.000541601859672149\\
585	0.00049674642660404\\
586	0.000452395149029591\\
587	0.000408588670365815\\
588	0.000365357795666476\\
589	0.00032272012337526\\
590	0.000280676711614416\\
591	0.000239312817487405\\
592	0.000198788644961842\\
593	0.000159293651685586\\
594	0.000121079888727804\\
595	8.45520570083913e-05\\
596	5.05092148680373e-05\\
597	2.07908715710836e-05\\
598	0\\
599	0\\
600	0\\
};
\addplot [color=mycolor3,solid,forget plot]
  table[row sep=crcr]{%
1	0.00550477682330948\\
2	0.00550477425642561\\
3	0.0055047716434672\\
4	0.00550476898360203\\
5	0.00550476627598275\\
6	0.00550476351974651\\
7	0.00550476071401469\\
8	0.00550475785789298\\
9	0.00550475495047054\\
10	0.00550475199082008\\
11	0.00550474897799734\\
12	0.00550474591104091\\
13	0.00550474278897191\\
14	0.00550473961079362\\
15	0.0055047363754912\\
16	0.00550473308203119\\
17	0.0055047297293615\\
18	0.00550472631641071\\
19	0.00550472284208796\\
20	0.00550471930528255\\
21	0.00550471570486342\\
22	0.00550471203967905\\
23	0.00550470830855674\\
24	0.00550470451030261\\
25	0.00550470064370086\\
26	0.0055046967075136\\
27	0.00550469270048034\\
28	0.00550468862131753\\
29	0.00550468446871832\\
30	0.00550468024135183\\
31	0.00550467593786316\\
32	0.00550467155687248\\
33	0.00550466709697481\\
34	0.00550466255673955\\
35	0.00550465793470994\\
36	0.00550465322940267\\
37	0.00550464843930729\\
38	0.00550464356288576\\
39	0.00550463859857201\\
40	0.00550463354477119\\
41	0.00550462839985948\\
42	0.00550462316218334\\
43	0.00550461783005889\\
44	0.00550461240177163\\
45	0.00550460687557555\\
46	0.00550460124969282\\
47	0.00550459552231303\\
48	0.00550458969159265\\
49	0.00550458375565442\\
50	0.0055045777125868\\
51	0.0055045715604432\\
52	0.00550456529724142\\
53	0.005504558920963\\
54	0.00550455242955239\\
55	0.00550454582091652\\
56	0.00550453909292396\\
57	0.00550453224340419\\
58	0.00550452527014696\\
59	0.0055045181709016\\
60	0.00550451094337607\\
61	0.00550450358523637\\
62	0.00550449609410563\\
63	0.00550448846756352\\
64	0.0055044807031453\\
65	0.00550447279834096\\
66	0.00550446475059457\\
67	0.00550445655730317\\
68	0.00550444821581615\\
69	0.00550443972343412\\
70	0.00550443107740831\\
71	0.00550442227493934\\
72	0.00550441331317651\\
73	0.00550440418921677\\
74	0.00550439490010382\\
75	0.0055043854428269\\
76	0.00550437581432005\\
77	0.00550436601146093\\
78	0.00550435603106987\\
79	0.00550434586990864\\
80	0.00550433552467965\\
81	0.00550432499202459\\
82	0.00550431426852338\\
83	0.00550430335069311\\
84	0.00550429223498672\\
85	0.00550428091779194\\
86	0.00550426939543011\\
87	0.00550425766415471\\
88	0.00550424572015037\\
89	0.00550423355953152\\
90	0.00550422117834099\\
91	0.0055042085725488\\
92	0.00550419573805069\\
93	0.00550418267066679\\
94	0.00550416936614034\\
95	0.00550415582013608\\
96	0.0055041420282389\\
97	0.00550412798595239\\
98	0.00550411368869705\\
99	0.00550409913180917\\
100	0.00550408431053899\\
101	0.00550406922004924\\
102	0.0055040538554133\\
103	0.00550403821161376\\
104	0.00550402228354073\\
105	0.00550400606598996\\
106	0.00550398955366117\\
107	0.00550397274115646\\
108	0.00550395562297808\\
109	0.00550393819352694\\
110	0.00550392044710062\\
111	0.00550390237789137\\
112	0.0055038839799842\\
113	0.00550386524735506\\
114	0.00550384617386849\\
115	0.00550382675327583\\
116	0.00550380697921296\\
117	0.00550378684519831\\
118	0.0055037663446305\\
119	0.00550374547078629\\
120	0.00550372421681824\\
121	0.00550370257575223\\
122	0.00550368054048561\\
123	0.00550365810378426\\
124	0.00550363525828055\\
125	0.00550361199647067\\
126	0.00550358831071218\\
127	0.0055035641932214\\
128	0.00550353963607094\\
129	0.00550351463118682\\
130	0.005503489170346\\
131	0.00550346324517338\\
132	0.00550343684713932\\
133	0.00550340996755649\\
134	0.00550338259757716\\
135	0.00550335472819009\\
136	0.00550332635021771\\
137	0.00550329745431279\\
138	0.0055032680309557\\
139	0.00550323807045099\\
140	0.00550320756292402\\
141	0.00550317649831816\\
142	0.005503144866391\\
143	0.00550311265671116\\
144	0.00550307985865477\\
145	0.00550304646140206\\
146	0.00550301245393358\\
147	0.00550297782502669\\
148	0.00550294256325193\\
149	0.00550290665696907\\
150	0.00550287009432344\\
151	0.00550283286324184\\
152	0.0055027949514287\\
153	0.00550275634636214\\
154	0.00550271703528952\\
155	0.00550267700522361\\
156	0.00550263624293808\\
157	0.00550259473496338\\
158	0.00550255246758201\\
159	0.00550250942682436\\
160	0.00550246559846392\\
161	0.00550242096801271\\
162	0.00550237552071655\\
163	0.00550232924155014\\
164	0.00550228211521213\\
165	0.00550223412612033\\
166	0.00550218525840629\\
167	0.00550213549591032\\
168	0.00550208482217605\\
169	0.00550203322044512\\
170	0.00550198067365169\\
171	0.00550192716441669\\
172	0.00550187267504231\\
173	0.0055018171875059\\
174	0.00550176068345432\\
175	0.00550170314419761\\
176	0.00550164455070308\\
177	0.00550158488358875\\
178	0.00550152412311714\\
179	0.00550146224918863\\
180	0.00550139924133485\\
181	0.00550133507871183\\
182	0.00550126974009321\\
183	0.00550120320386315\\
184	0.00550113544800915\\
185	0.00550106645011499\\
186	0.00550099618735303\\
187	0.00550092463647722\\
188	0.00550085177381503\\
189	0.00550077757526\\
190	0.00550070201626403\\
191	0.00550062507182938\\
192	0.00550054671650088\\
193	0.00550046692435791\\
194	0.00550038566900639\\
195	0.00550030292357135\\
196	0.00550021866068941\\
197	0.00550013285250246\\
198	0.00550004547065237\\
199	0.00549995648627275\\
200	0.00549986586996504\\
201	0.00549977359178749\\
202	0.00549967962124463\\
203	0.00549958392727645\\
204	0.00549948647824725\\
205	0.00549938724193426\\
206	0.005499286185516\\
207	0.00549918327556026\\
208	0.00549907847801172\\
209	0.00549897175817939\\
210	0.00549886308072349\\
211	0.0054987524096421\\
212	0.0054986397082573\\
213	0.00549852493920084\\
214	0.0054984080643997\\
215	0.00549828904506062\\
216	0.00549816784165441\\
217	0.00549804441390017\\
218	0.00549791872074772\\
219	0.00549779072036074\\
220	0.00549766037009842\\
221	0.00549752762649653\\
222	0.00549739244524797\\
223	0.00549725478118225\\
224	0.00549711458824402\\
225	0.00549697181947082\\
226	0.00549682642696964\\
227	0.00549667836189246\\
228	0.00549652757441025\\
229	0.00549637401368581\\
230	0.00549621762784532\\
231	0.00549605836394773\\
232	0.00549589616795276\\
233	0.0054957309846868\\
234	0.00549556275780674\\
235	0.00549539142976125\\
236	0.00549521694174974\\
237	0.00549503923367794\\
238	0.00549485824411096\\
239	0.00549467391022222\\
240	0.00549448616773887\\
241	0.00549429495088275\\
242	0.00549410019230668\\
243	0.0054939018230254\\
244	0.00549369977234063\\
245	0.00549349396775976\\
246	0.00549328433490728\\
247	0.00549307079742838\\
248	0.00549285327688355\\
249	0.00549263169263335\\
250	0.00549240596171277\\
251	0.00549217599869314\\
252	0.00549194171553131\\
253	0.00549170302140385\\
254	0.00549145982252539\\
255	0.00549121202194953\\
256	0.0054909595193502\\
257	0.0054907022107821\\
258	0.00549043998841833\\
259	0.00549017274026306\\
260	0.00548990034983755\\
261	0.00548962269583773\\
262	0.00548933965176146\\
263	0.00548905108550404\\
264	0.00548875685892081\\
265	0.00548845682735624\\
266	0.0054881508391397\\
267	0.00548783873504886\\
268	0.00548752034774364\\
269	0.00548719550117455\\
270	0.00548686400997268\\
271	0.00548652567883073\\
272	0.0054861803018893\\
273	0.00548582766214855\\
274	0.00548546753093547\\
275	0.00548509966748068\\
276	0.00548472381871943\\
277	0.00548433971962728\\
278	0.00548394709507049\\
279	0.00548354566654718\\
280	0.00548313585038502\\
281	0.00548271859907244\\
282	0.00548229377981785\\
283	0.0054818612575201\\
284	0.00548142089472977\\
285	0.00548097255161\\
286	0.00548051608589664\\
287	0.00548005135285767\\
288	0.0054795782052522\\
289	0.00547909649328861\\
290	0.00547860606458215\\
291	0.00547810676411174\\
292	0.00547759843417632\\
293	0.00547708091435022\\
294	0.005476554041438\\
295	0.00547601764942866\\
296	0.00547547156944895\\
297	0.00547491562971612\\
298	0.00547434965548969\\
299	0.00547377346902287\\
300	0.00547318688951282\\
301	0.00547258973305047\\
302	0.00547198181256938\\
303	0.00547136293779389\\
304	0.00547073291518661\\
305	0.005470091547895\\
306	0.00546943863569711\\
307	0.00546877397494682\\
308	0.00546809735851799\\
309	0.00546740857574802\\
310	0.00546670741238051\\
311	0.00546599365050735\\
312	0.00546526706850985\\
313	0.00546452744099923\\
314	0.00546377453875629\\
315	0.00546300812867045\\
316	0.00546222797367823\\
317	0.00546143383270067\\
318	0.00546062546058052\\
319	0.00545980260801873\\
320	0.00545896502151018\\
321	0.00545811244327932\\
322	0.00545724461121499\\
323	0.00545636125880529\\
324	0.00545546211507171\\
325	0.00545454690450366\\
326	0.00545361534699242\\
327	0.0054526671577656\\
328	0.00545170204732146\\
329	0.00545071972136402\\
330	0.00544971988073853\\
331	0.00544870222136763\\
332	0.00544766643418897\\
333	0.00544661220509387\\
334	0.00544553921486785\\
335	0.00544444713913311\\
336	0.00544333564829348\\
337	0.00544220440748248\\
338	0.0054410530765146\\
339	0.00543988130984097\\
340	0.0054386887565096\\
341	0.00543747506013127\\
342	0.0054362398588518\\
343	0.00543498278533194\\
344	0.00543370346673545\\
345	0.00543240152472745\\
346	0.00543107657548345\\
347	0.00542972822971189\\
348	0.00542835609269072\\
349	0.00542695976432092\\
350	0.00542553883919896\\
351	0.00542409290671077\\
352	0.00542262155115011\\
353	0.00542112435186449\\
354	0.00541960088343242\\
355	0.00541805071587583\\
356	0.00541647341491245\\
357	0.00541486854225276\\
358	0.00541323565594737\\
359	0.00541157431079114\\
360	0.00540988405879056\\
361	0.0054081644497021\\
362	0.00540641503164977\\
363	0.00540463535183105\\
364	0.0054028249573207\\
365	0.00540098339598347\\
366	0.00539911021750652\\
367	0.00539720497456394\\
368	0.00539526722412573\\
369	0.00539329652892395\\
370	0.00539129245908901\\
371	0.00538925459396853\\
372	0.00538718252414036\\
373	0.00538507585362938\\
374	0.00538293420233504\\
375	0.00538075720867236\\
376	0.00537854453242384\\
377	0.00537629585779\\
378	0.00537401089661826\\
379	0.00537168939177402\\
380	0.00536933112060602\\
381	0.00536693589844766\\
382	0.00536450358211477\\
383	0.00536203407346954\\
384	0.00535952732353251\\
385	0.00535698333899241\\
386	0.00535440219739541\\
387	0.00535178409178646\\
388	0.00534912514988319\\
389	0.00534641761903743\\
390	0.00534366061596057\\
391	0.00534085324265133\\
392	0.00533799458610111\\
393	0.00533508371795533\\
394	0.00533211969415246\\
395	0.00532910155454309\\
396	0.00532602832211091\\
397	0.00532289900203223\\
398	0.00531971258080888\\
399	0.00531646802530294\\
400	0.00531316428168546\\
401	0.00530980027430732\\
402	0.00530637490449251\\
403	0.00530288704928074\\
404	0.00529933556009359\\
405	0.00529571926117679\\
406	0.00529203694800453\\
407	0.00528828738570286\\
408	0.00528446930703295\\
409	0.00528058141012162\\
410	0.00527662235590618\\
411	0.00527259076525703\\
412	0.00526848521573851\\
413	0.00526430423796721\\
414	0.00526004631153165\\
415	0.00525570986046398\\
416	0.00525129324831951\\
417	0.00524679477302037\\
418	0.00524221266141137\\
419	0.00523754506071366\\
420	0.00523279003052249\\
421	0.00522794553410434\\
422	0.0052230094281619\\
423	0.00521797944891619\\
424	0.00521285318686776\\
425	0.00520762802306626\\
426	0.00520230092936529\\
427	0.00519686899937804\\
428	0.00519132985949917\\
429	0.00518568105926974\\
430	0.00517992006754721\\
431	0.00517404426844573\\
432	0.00516805095702335\\
433	0.00516193733472785\\
434	0.00515570050460202\\
435	0.00514933746622776\\
436	0.00514284511040285\\
437	0.00513622021354724\\
438	0.00512945943183804\\
439	0.0051225592950757\\
440	0.00511551620027982\\
441	0.00510832640502191\\
442	0.00510098602050871\\
443	0.00509349100442846\\
444	0.00508583715357047\\
445	0.00507802009621341\\
446	0.00507003528423538\\
447	0.00506187798479317\\
448	0.00505354327122278\\
449	0.00504502601268046\\
450	0.00503632086333021\\
451	0.00502742226201914\\
452	0.00501832447056875\\
453	0.00500902149371753\\
454	0.00499950706396044\\
455	0.00498977462543773\\
456	0.00497981731679798\\
457	0.00496962795297015\\
458	0.00495919900577868\\
459	0.00494852258333314\\
460	0.00493759040812209\\
461	0.00492639379373232\\
462	0.00491492362011725\\
463	0.00490317030733283\\
464	0.00489112378765591\\
465	0.00487877347599587\\
466	0.00486610823851244\\
467	0.00485311635938098\\
468	0.00483978550576249\\
469	0.0048261026913033\\
470	0.00481205423849246\\
471	0.00479762573586072\\
472	0.00478280199444234\\
473	0.00476756700393987\\
474	0.00475190388728252\\
475	0.00473579485348131\\
476	0.00471922114823902\\
477	0.0047021630016523\\
478	0.00468459957115042\\
479	0.00466650885784214\\
480	0.0046478675515963\\
481	0.0046286508102488\\
482	0.00460883243742465\\
483	0.00458838400496198\\
484	0.00456726999448731\\
485	0.00454543469310754\\
486	0.00452284311214705\\
487	0.00449945871076504\\
488	0.00447524373166766\\
489	0.00445015868630056\\
490	0.00442416236753183\\
491	0.00439721188603949\\
492	0.00436926275841055\\
493	0.00434026904545963\\
494	0.00431018355398794\\
495	0.00427895811764693\\
496	0.00424654397658071\\
497	0.00421289228003259\\
498	0.00417795474171066\\
499	0.00414168448465006\\
500	0.0041040371207246\\
501	0.0040649721200045\\
502	0.00402445453844905\\
503	0.00398245719006864\\
504	0.00393896337156799\\
505	0.00389397025230831\\
506	0.00384749302944713\\
507	0.00379957011758446\\
508	0.00375026981833159\\
509	0.00369969955863891\\
510	0.0036480175997573\\
511	0.00359544297495342\\
512	0.0035422627542797\\
513	0.00348883084154451\\
514	0.0034356354815106\\
515	0.00338346436248934\\
516	0.00333529941397275\\
517	0.00329239584335947\\
518	0.00325479594604639\\
519	0.00322212145566726\\
520	0.00319081861398368\\
521	0.0031604552962268\\
522	0.00313060409131133\\
523	0.0031007791385642\\
524	0.00307090240615044\\
525	0.00304081945040074\\
526	0.00301018748031608\\
527	0.00297896731824318\\
528	0.0029471211890936\\
529	0.00291460839813878\\
530	0.00288140050348125\\
531	0.00284747823876823\\
532	0.00281282247445835\\
533	0.00277741396691225\\
534	0.00274123362774897\\
535	0.0027042612251743\\
536	0.00266647498145895\\
537	0.00262785203732332\\
538	0.00258836871834692\\
539	0.00254799039987889\\
540	0.00250663194236034\\
541	0.00246552657985651\\
542	0.00242551775773468\\
543	0.00238679174078739\\
544	0.00234758968952664\\
545	0.00230783756843862\\
546	0.00226746688232517\\
547	0.00222647633461372\\
548	0.00218486861377273\\
549	0.00214264741712259\\
550	0.00209982152218929\\
551	0.00205640582806659\\
552	0.00201241871448934\\
553	0.00196788261038474\\
554	0.001922824759329\\
555	0.00187727861552058\\
556	0.00183128530984319\\
557	0.00178571858457715\\
558	0.00174175370183844\\
559	0.00169765189645918\\
560	0.00165315928882859\\
561	0.00160829620918484\\
562	0.0015630878421284\\
563	0.00151756120705137\\
564	0.00147174485938914\\
565	0.00142566860063263\\
566	0.00137936312148276\\
567	0.00133285966646703\\
568	0.00128618971348929\\
569	0.00123974505182\\
570	0.00119341585886474\\
571	0.00114688025846023\\
572	0.00110017097215749\\
573	0.00105332337466137\\
574	0.00100637555374019\\
575	0.000959368345378089\\
576	0.00091234532951263\\
577	0.00086535277371407\\
578	0.00081843950905746\\
579	0.000771656718587084\\
580	0.000725057614040413\\
581	0.000678696970709247\\
582	0.000632630483283924\\
583	0.00058691389712032\\
584	0.000541601859672149\\
585	0.000496746426604037\\
586	0.000452395149029591\\
587	0.000408588670365813\\
588	0.000365357795666473\\
589	0.000322720123375262\\
590	0.000280676711614414\\
591	0.000239312817487402\\
592	0.000198788644961839\\
593	0.000159293651685585\\
594	0.000121079888727804\\
595	8.45520570083908e-05\\
596	5.05092148680372e-05\\
597	2.07908715710836e-05\\
598	0\\
599	0\\
600	0\\
};
\addplot [color=mycolor4,solid,forget plot]
  table[row sep=crcr]{%
1	0.00552236449378588\\
2	0.00552236161741465\\
3	0.00552235869004543\\
4	0.00552235571076934\\
5	0.00552235267866136\\
6	0.00552234959277986\\
7	0.00552234645216632\\
8	0.00552234325584501\\
9	0.0055223400028227\\
10	0.00552233669208844\\
11	0.0055223333226131\\
12	0.00552232989334907\\
13	0.00552232640323002\\
14	0.00552232285117041\\
15	0.00552231923606535\\
16	0.00552231555678995\\
17	0.00552231181219927\\
18	0.0055223080011278\\
19	0.0055223041223891\\
20	0.00552230017477544\\
21	0.0055222961570574\\
22	0.00552229206798352\\
23	0.00552228790627984\\
24	0.00552228367064954\\
25	0.00552227935977247\\
26	0.00552227497230478\\
27	0.00552227050687855\\
28	0.00552226596210119\\
29	0.00552226133655511\\
30	0.00552225662879733\\
31	0.00552225183735878\\
32	0.0055222469607441\\
33	0.00552224199743105\\
34	0.00552223694586996\\
35	0.00552223180448328\\
36	0.00552222657166512\\
37	0.00552222124578073\\
38	0.00552221582516592\\
39	0.00552221030812658\\
40	0.00552220469293806\\
41	0.00552219897784472\\
42	0.0055221931610593\\
43	0.00552218724076237\\
44	0.00552218121510173\\
45	0.00552217508219179\\
46	0.00552216884011311\\
47	0.00552216248691159\\
48	0.00552215602059788\\
49	0.00552214943914689\\
50	0.00552214274049698\\
51	0.00552213592254938\\
52	0.00552212898316735\\
53	0.00552212192017577\\
54	0.00552211473136027\\
55	0.00552210741446656\\
56	0.00552209996719957\\
57	0.00552209238722291\\
58	0.00552208467215801\\
59	0.00552207681958337\\
60	0.00552206882703378\\
61	0.0055220606919995\\
62	0.00552205241192548\\
63	0.00552204398421052\\
64	0.00552203540620642\\
65	0.00552202667521712\\
66	0.00552201778849787\\
67	0.00552200874325425\\
68	0.00552199953664139\\
69	0.00552199016576291\\
70	0.00552198062767006\\
71	0.00552197091936075\\
72	0.00552196103777858\\
73	0.00552195097981178\\
74	0.00552194074229234\\
75	0.00552193032199479\\
76	0.00552191971563531\\
77	0.00552190891987057\\
78	0.00552189793129664\\
79	0.00552188674644797\\
80	0.00552187536179611\\
81	0.00552186377374865\\
82	0.00552185197864809\\
83	0.00552183997277049\\
84	0.00552182775232437\\
85	0.00552181531344947\\
86	0.00552180265221537\\
87	0.00552178976462037\\
88	0.00552177664659001\\
89	0.00552176329397577\\
90	0.00552174970255381\\
91	0.00552173586802346\\
92	0.00552172178600583\\
93	0.00552170745204245\\
94	0.00552169286159359\\
95	0.00552167801003702\\
96	0.00552166289266629\\
97	0.0055216475046892\\
98	0.00552163184122631\\
99	0.00552161589730922\\
100	0.00552159966787898\\
101	0.00552158314778434\\
102	0.00552156633178012\\
103	0.0055215492145254\\
104	0.00552153179058173\\
105	0.00552151405441141\\
106	0.00552149600037559\\
107	0.0055214776227323\\
108	0.00552145891563467\\
109	0.00552143987312886\\
110	0.00552142048915216\\
111	0.00552140075753084\\
112	0.00552138067197822\\
113	0.00552136022609242\\
114	0.00552133941335427\\
115	0.00552131822712517\\
116	0.00552129666064476\\
117	0.00552127470702868\\
118	0.00552125235926623\\
119	0.00552122961021801\\
120	0.00552120645261356\\
121	0.00552118287904884\\
122	0.0055211588819837\\
123	0.00552113445373936\\
124	0.00552110958649586\\
125	0.00552108427228929\\
126	0.00552105850300921\\
127	0.00552103227039581\\
128	0.00552100556603709\\
129	0.00552097838136615\\
130	0.00552095070765808\\
131	0.00552092253602712\\
132	0.00552089385742359\\
133	0.0055208646626308\\
134	0.00552083494226198\\
135	0.00552080468675704\\
136	0.00552077388637924\\
137	0.00552074253121211\\
138	0.00552071061115585\\
139	0.00552067811592396\\
140	0.00552064503503989\\
141	0.00552061135783322\\
142	0.00552057707343636\\
143	0.00552054217078058\\
144	0.0055205066385925\\
145	0.00552047046539015\\
146	0.00552043363947916\\
147	0.00552039614894883\\
148	0.00552035798166811\\
149	0.00552031912528151\\
150	0.00552027956720514\\
151	0.00552023929462236\\
152	0.00552019829447949\\
153	0.00552015655348165\\
154	0.00552011405808831\\
155	0.00552007079450878\\
156	0.00552002674869782\\
157	0.00551998190635094\\
158	0.00551993625289978\\
159	0.00551988977350748\\
160	0.00551984245306379\\
161	0.00551979427618033\\
162	0.00551974522718557\\
163	0.00551969529011997\\
164	0.0055196444487308\\
165	0.00551959268646715\\
166	0.00551953998647461\\
167	0.00551948633159015\\
168	0.0055194317043366\\
169	0.0055193760869174\\
170	0.00551931946121092\\
171	0.00551926180876504\\
172	0.00551920311079126\\
173	0.00551914334815907\\
174	0.00551908250138998\\
175	0.00551902055065147\\
176	0.00551895747575083\\
177	0.00551889325612899\\
178	0.00551882787085396\\
179	0.0055187612986142\\
180	0.00551869351771199\\
181	0.00551862450605631\\
182	0.00551855424115581\\
183	0.00551848270011119\\
184	0.00551840985960772\\
185	0.00551833569590721\\
186	0.00551826018483992\\
187	0.00551818330179588\\
188	0.00551810502171628\\
189	0.00551802531908424\\
190	0.00551794416791541\\
191	0.00551786154174813\\
192	0.00551777741363352\\
193	0.00551769175612491\\
194	0.00551760454126753\\
195	0.00551751574058748\\
196	0.00551742532508085\\
197	0.00551733326520237\\
198	0.00551723953085419\\
199	0.00551714409137402\\
200	0.00551704691552371\\
201	0.00551694797147738\\
202	0.00551684722680929\\
203	0.00551674464848142\\
204	0.00551664020283073\\
205	0.00551653385555592\\
206	0.00551642557170416\\
207	0.0055163153156571\\
208	0.00551620305111673\\
209	0.00551608874109072\\
210	0.00551597234787748\\
211	0.00551585383305065\\
212	0.00551573315744316\\
213	0.00551561028113103\\
214	0.00551548516341644\\
215	0.00551535776281035\\
216	0.00551522803701485\\
217	0.00551509594290453\\
218	0.00551496143650786\\
219	0.00551482447298749\\
220	0.0055146850066202\\
221	0.00551454299077638\\
222	0.00551439837789852\\
223	0.00551425111947928\\
224	0.00551410116603904\\
225	0.00551394846710242\\
226	0.00551379297117448\\
227	0.00551363462571579\\
228	0.00551347337711732\\
229	0.00551330917067408\\
230	0.00551314195055836\\
231	0.00551297165979225\\
232	0.0055127982402193\\
233	0.00551262163247557\\
234	0.00551244177596003\\
235	0.00551225860880428\\
236	0.00551207206784177\\
237	0.00551188208857648\\
238	0.00551168860515112\\
239	0.00551149155031508\\
240	0.00551129085539215\\
241	0.00551108645024829\\
242	0.00551087826325943\\
243	0.00551066622127962\\
244	0.00551045024961022\\
245	0.00551023027196952\\
246	0.00551000621046425\\
247	0.0055097779855625\\
248	0.00550954551606927\\
249	0.00550930871910496\\
250	0.00550906751008725\\
251	0.00550882180271794\\
252	0.0055085715089747\\
253	0.00550831653910962\\
254	0.00550805680165543\\
255	0.00550779220344061\\
256	0.00550752264961524\\
257	0.00550724804368904\\
258	0.00550696828758372\\
259	0.00550668328170141\\
260	0.00550639292501186\\
261	0.00550609711516044\\
262	0.00550579574859972\\
263	0.00550548872074722\\
264	0.0055051759261724\\
265	0.00550485725881497\\
266	0.00550453261223783\\
267	0.00550420187991627\\
268	0.00550386495556571\\
269	0.00550352173350871\\
270	0.00550317210908153\\
271	0.00550281597907872\\
272	0.00550245324223288\\
273	0.00550208379972421\\
274	0.00550170755571362\\
275	0.00550132441789157\\
276	0.00550093429803821\\
277	0.00550053711259858\\
278	0.00550013278327558\\
279	0.0054997212375169\\
280	0.00549930240779239\\
281	0.00549887620663984\\
282	0.00549844250829231\\
283	0.00549800118495234\\
284	0.00549755210676195\\
285	0.00549709514177206\\
286	0.00549663015591132\\
287	0.00549615701295485\\
288	0.00549567557449205\\
289	0.00549518569989425\\
290	0.00549468724628167\\
291	0.00549418006848997\\
292	0.00549366401903615\\
293	0.00549313894808387\\
294	0.00549260470340838\\
295	0.00549206113036053\\
296	0.00549150807183033\\
297	0.00549094536820982\\
298	0.00549037285735526\\
299	0.00548979037454844\\
300	0.00548919775245742\\
301	0.00548859482109638\\
302	0.00548798140778461\\
303	0.00548735733710453\\
304	0.00548672243085905\\
305	0.00548607650802752\\
306	0.00548541938472111\\
307	0.00548475087413662\\
308	0.00548407078650947\\
309	0.00548337892906525\\
310	0.00548267510597001\\
311	0.00548195911827908\\
312	0.0054812307638845\\
313	0.0054804898374607\\
314	0.00547973613040869\\
315	0.00547896943079811\\
316	0.00547818952330758\\
317	0.0054773961891628\\
318	0.00547658920607254\\
319	0.00547576834816195\\
320	0.00547493338590362\\
321	0.00547408408604557\\
322	0.00547322021153632\\
323	0.00547234152144663\\
324	0.00547144777088796\\
325	0.00547053871092677\\
326	0.00546961408849504\\
327	0.00546867364629624\\
328	0.00546771712270641\\
329	0.00546674425167016\\
330	0.00546575476259082\\
331	0.00546474838021479\\
332	0.0054637248245088\\
333	0.0054626838105302\\
334	0.00546162504828915\\
335	0.00546054824260222\\
336	0.00545945309293683\\
337	0.00545833929324494\\
338	0.00545720653178605\\
339	0.00545605449093745\\
340	0.00545488284699143\\
341	0.00545369126993761\\
342	0.00545247942322917\\
343	0.00545124696353159\\
344	0.00544999354045185\\
345	0.00544871879624645\\
346	0.00544742236550625\\
347	0.00544610387481529\\
348	0.00544476294238176\\
349	0.00544339917763805\\
350	0.00544201218080623\\
351	0.0054406015424265\\
352	0.00543916684284406\\
353	0.00543770765165072\\
354	0.00543622352707626\\
355	0.00543471401532517\\
356	0.0054331786498525\\
357	0.00543161695057351\\
358	0.00543002842300023\\
359	0.00542841255729794\\
360	0.0054267688272538\\
361	0.00542509668914959\\
362	0.00542339558052954\\
363	0.00542166491885402\\
364	0.00541990410002906\\
365	0.00541811249680166\\
366	0.00541628945701016\\
367	0.00541443430167937\\
368	0.00541254632294991\\
369	0.00541062478183212\\
370	0.00540866890577628\\
371	0.00540667788605188\\
372	0.00540465087493226\\
373	0.00540258698268411\\
374	0.00540048527436669\\
375	0.00539834476645274\\
376	0.00539616442329166\\
377	0.00539394315344812\\
378	0.00539167980596377\\
379	0.00538937316660993\\
380	0.00538702195422511\\
381	0.00538462481726917\\
382	0.00538218033078822\\
383	0.00537968699412042\\
384	0.00537714323003581\\
385	0.00537454738714601\\
386	0.00537189775132708\\
387	0.00536919258590505\\
388	0.00536643411745236\\
389	0.00536362814878825\\
390	0.00536077380558616\\
391	0.00535787020184325\\
392	0.00535491644144934\\
393	0.00535191161956328\\
394	0.00534885482403642\\
395	0.0053457451376249\\
396	0.00534258164487129\\
397	0.00533936343356682\\
398	0.00533608959274483\\
399	0.00533275921349833\\
400	0.00532937138914247\\
401	0.00532592521468404\\
402	0.00532241978561147\\
403	0.00531885419614399\\
404	0.0053152275385854\\
405	0.0053115389049386\\
406	0.00530778738568527\\
407	0.00530397206919219\\
408	0.00530009204781425\\
409	0.00529614642026316\\
410	0.00529213429423897\\
411	0.00528805478932205\\
412	0.00528390704009951\\
413	0.00527969019945088\\
414	0.00527540344183438\\
415	0.00527104596630378\\
416	0.00526661699899993\\
417	0.00526211579574787\\
418	0.00525754165017734\\
419	0.00525289393257562\\
420	0.00524817207014708\\
421	0.00524337554790461\\
422	0.00523850391366673\\
423	0.00523355678417246\\
424	0.0052285338562323\\
425	0.00522343493609123\\
426	0.00521826003033973\\
427	0.00521299981013176\\
428	0.00520764154984799\\
429	0.00520218316507152\\
430	0.00519662250553933\\
431	0.00519095735143663\\
432	0.00518518540943985\\
433	0.00517930430786009\\
434	0.00517331159122543\\
435	0.00516720471464439\\
436	0.00516098103757801\\
437	0.00515463781694451\\
438	0.00514817219947595\\
439	0.0051415812132719\\
440	0.00513486175855821\\
441	0.00512801059742921\\
442	0.0051210243424309\\
443	0.00511389944394695\\
444	0.00510663217628548\\
445	0.00509921862234511\\
446	0.00509165465665163\\
447	0.00508393592620462\\
448	0.00507605782727588\\
449	0.00506801547165309\\
450	0.0050598036192859\\
451	0.00505141649510841\\
452	0.0050428475396761\\
453	0.00503409144145339\\
454	0.0050251426509999\\
455	0.00501599536655582\\
456	0.00500664351914084\\
457	0.00499708075683417\\
458	0.00498730042815378\\
459	0.00497729556449069\\
460	0.0049670588615872\\
461	0.00495658266011656\\
462	0.00494585892524127\\
463	0.0049348792251699\\
464	0.00492363470868343\\
465	0.00491211608156425\\
466	0.00490031358175842\\
467	0.00488821695288661\\
468	0.00487581541544252\\
469	0.00486309763559506\\
470	0.00485005169845369\\
471	0.0048366651328826\\
472	0.00482292483719772\\
473	0.00480881700809003\\
474	0.00479432709932754\\
475	0.00477943977858097\\
476	0.0047641388823978\\
477	0.00474840736927814\\
478	0.00473222727046199\\
479	0.00471557963788382\\
480	0.00469844448912965\\
481	0.00468080074667346\\
482	0.0046626261375146\\
483	0.00464389700735168\\
484	0.00462458811794882\\
485	0.0046046727571455\\
486	0.00458412118468098\\
487	0.00456289381669408\\
488	0.00454093863433154\\
489	0.00451821998828621\\
490	0.00449470064864536\\
491	0.00447034196370148\\
492	0.00444510340289893\\
493	0.00441894254081404\\
494	0.00439181506085209\\
495	0.00436367480329994\\
496	0.0043344738555302\\
497	0.00430416269614973\\
498	0.00427269040674317\\
499	0.00424000496854713\\
500	0.00420605366530414\\
501	0.00417078361847794\\
502	0.00413414248746009\\
503	0.00409607937443961\\
504	0.00405654598266416\\
505	0.00401549808831925\\
506	0.00397289740224313\\
507	0.00392871391491754\\
508	0.00388292883819276\\
509	0.00383553825499621\\
510	0.00378655763043521\\
511	0.00373602750432496\\
512	0.00368402082627437\\
513	0.00363065307189473\\
514	0.00357609408201811\\
515	0.00352058662871403\\
516	0.00346443837690163\\
517	0.00340801376574661\\
518	0.00335185318789953\\
519	0.00329683024981812\\
520	0.00324653985142069\\
521	0.00320161003127924\\
522	0.00316207202960114\\
523	0.00312749854380489\\
524	0.00309395292311824\\
525	0.00306108207870875\\
526	0.00302867931543196\\
527	0.00299627525953052\\
528	0.00296378678854382\\
529	0.00293115317034936\\
530	0.00289809271373717\\
531	0.00286441613276463\\
532	0.00283008516840576\\
533	0.00279506205124351\\
534	0.00275930312882306\\
535	0.00272278664427699\\
536	0.00268549207316304\\
537	0.00264739866875622\\
538	0.00260848525570127\\
539	0.00256873024649069\\
540	0.00252811165960377\\
541	0.00248660681497579\\
542	0.00244415586141974\\
543	0.00240076947053404\\
544	0.00235831976722801\\
545	0.00231705559205897\\
546	0.00227650183143643\\
547	0.00223546034490059\\
548	0.00219384736307803\\
549	0.00215161604661321\\
550	0.00210876510575541\\
551	0.00206530472400923\\
552	0.00202124965549764\\
553	0.00197661822767365\\
554	0.00193143270313793\\
555	0.00188571981291925\\
556	0.00183951158664442\\
557	0.00179284685067574\\
558	0.00174575367523509\\
559	0.00169999208744112\\
560	0.00165532117678849\\
561	0.00161033080979638\\
562	0.00156498746152975\\
563	0.00151931648608894\\
564	0.00147334703390769\\
565	0.00142710998207863\\
566	0.00138063755587401\\
567	0.00133396294340406\\
568	0.00128711993625959\\
569	0.00124014252333356\\
570	0.00119341592573713\\
571	0.00114688025971738\\
572	0.00110017097260895\\
573	0.0010533233748797\\
574	0.00100637555384331\\
575	0.000959368345424142\\
576	0.000912345329531775\\
577	0.000865352773721365\\
578	0.000818439509059961\\
579	0.000771656718587823\\
580	0.000725057614040593\\
581	0.000678696970709286\\
582	0.000632630483283935\\
583	0.000586913897120328\\
584	0.000541601859672155\\
585	0.000496746426604042\\
586	0.000452395149029594\\
587	0.000408588670365815\\
588	0.000365357795666473\\
589	0.000322720123375259\\
590	0.000280676711614414\\
591	0.000239312817487403\\
592	0.000198788644961842\\
593	0.000159293651685587\\
594	0.000121079888727805\\
595	8.45520570083912e-05\\
596	5.05092148680371e-05\\
597	2.07908715710836e-05\\
598	0\\
599	0\\
600	0\\
};
\addplot [color=mycolor5,solid,forget plot]
  table[row sep=crcr]{%
1	0.0055565662029982\\
2	0.0055565625234433\\
3	0.00555655877973134\\
4	0.00555655497074048\\
5	0.00555655109532921\\
6	0.00555654715233594\\
7	0.00555654314057877\\
8	0.00555653905885504\\
9	0.00555653490594107\\
10	0.0055565306805916\\
11	0.00555652638153959\\
12	0.00555652200749573\\
13	0.00555651755714814\\
14	0.00555651302916188\\
15	0.00555650842217858\\
16	0.0055565037348161\\
17	0.00555649896566798\\
18	0.00555649411330313\\
19	0.00555648917626525\\
20	0.00555648415307261\\
21	0.00555647904221737\\
22	0.00555647384216524\\
23	0.00555646855135501\\
24	0.00555646316819805\\
25	0.00555645769107788\\
26	0.00555645211834957\\
27	0.00555644644833935\\
28	0.00555644067934405\\
29	0.0055564348096306\\
30	0.00555642883743542\\
31	0.00555642276096407\\
32	0.00555641657839049\\
33	0.00555641028785659\\
34	0.00555640388747161\\
35	0.00555639737531164\\
36	0.00555639074941891\\
37	0.00555638400780123\\
38	0.00555637714843142\\
39	0.00555637016924665\\
40	0.00555636306814791\\
41	0.00555635584299926\\
42	0.00555634849162714\\
43	0.00555634101181982\\
44	0.00555633340132675\\
45	0.00555632565785775\\
46	0.00555631777908234\\
47	0.0055563097626291\\
48	0.00555630160608489\\
49	0.00555629330699411\\
50	0.00555628486285798\\
51	0.00555627627113374\\
52	0.00555626752923394\\
53	0.00555625863452551\\
54	0.00555624958432911\\
55	0.00555624037591822\\
56	0.00555623100651832\\
57	0.00555622147330605\\
58	0.00555621177340834\\
59	0.00555620190390153\\
60	0.00555619186181047\\
61	0.00555618164410758\\
62	0.005556171247712\\
63	0.00555616066948858\\
64	0.0055561499062469\\
65	0.00555613895474034\\
66	0.00555612781166499\\
67	0.00555611647365882\\
68	0.00555610493730038\\
69	0.00555609319910802\\
70	0.00555608125553862\\
71	0.00555606910298657\\
72	0.00555605673778266\\
73	0.00555604415619292\\
74	0.00555603135441743\\
75	0.00555601832858933\\
76	0.00555600507477335\\
77	0.0055559915889648\\
78	0.00555597786708824\\
79	0.00555596390499621\\
80	0.00555594969846801\\
81	0.00555593524320828\\
82	0.00555592053484576\\
83	0.00555590556893193\\
84	0.00555589034093952\\
85	0.00555587484626118\\
86	0.00555585908020805\\
87	0.00555584303800822\\
88	0.00555582671480538\\
89	0.00555581010565713\\
90	0.00555579320553354\\
91	0.00555577600931549\\
92	0.00555575851179316\\
93	0.00555574070766432\\
94	0.00555572259153268\\
95	0.00555570415790615\\
96	0.00555568540119518\\
97	0.00555566631571092\\
98	0.00555564689566349\\
99	0.00555562713516005\\
100	0.00555560702820297\\
101	0.00555558656868798\\
102	0.00555556575040211\\
103	0.0055555445670218\\
104	0.00555552301211084\\
105	0.00555550107911823\\
106	0.00555547876137623\\
107	0.00555545605209811\\
108	0.00555543294437596\\
109	0.00555540943117851\\
110	0.00555538550534878\\
111	0.00555536115960185\\
112	0.00555533638652235\\
113	0.00555531117856221\\
114	0.0055552855280381\\
115	0.00555525942712889\\
116	0.00555523286787309\\
117	0.00555520584216623\\
118	0.00555517834175824\\
119	0.00555515035825068\\
120	0.00555512188309391\\
121	0.00555509290758422\\
122	0.00555506342286103\\
123	0.00555503341990391\\
124	0.00555500288952936\\
125	0.0055549718223879\\
126	0.00555494020896089\\
127	0.00555490803955718\\
128	0.00555487530430996\\
129	0.00555484199317324\\
130	0.0055548080959185\\
131	0.00555477360213113\\
132	0.00555473850120685\\
133	0.005554702782348\\
134	0.00555466643455983\\
135	0.00555462944664663\\
136	0.00555459180720787\\
137	0.005554553504634\\
138	0.00555451452710263\\
139	0.00555447486257415\\
140	0.00555443449878744\\
141	0.00555439342325572\\
142	0.00555435162326185\\
143	0.00555430908585392\\
144	0.00555426579784046\\
145	0.00555422174578586\\
146	0.00555417691600541\\
147	0.00555413129456037\\
148	0.00555408486725289\\
149	0.00555403761962097\\
150	0.00555398953693303\\
151	0.0055539406041827\\
152	0.00555389080608334\\
153	0.00555384012706248\\
154	0.00555378855125606\\
155	0.00555373606250292\\
156	0.00555368264433869\\
157	0.0055536282799901\\
158	0.00555357295236878\\
159	0.00555351664406516\\
160	0.00555345933734259\\
161	0.00555340101413062\\
162	0.00555334165601911\\
163	0.00555328124425156\\
164	0.00555321975971893\\
165	0.00555315718295301\\
166	0.00555309349412002\\
167	0.00555302867301414\\
168	0.00555296269905097\\
169	0.00555289555126099\\
170	0.00555282720828328\\
171	0.00555275764835887\\
172	0.00555268684932458\\
173	0.00555261478860662\\
174	0.00555254144321438\\
175	0.00555246678973436\\
176	0.00555239080432414\\
177	0.00555231346270656\\
178	0.00555223474016394\\
179	0.00555215461153247\\
180	0.00555207305119675\\
181	0.00555199003308438\\
182	0.00555190553066068\\
183	0.00555181951692356\\
184	0.0055517319643983\\
185	0.00555164284513227\\
186	0.00555155213068955\\
187	0.00555145979214541\\
188	0.00555136580008011\\
189	0.00555127012457244\\
190	0.00555117273519229\\
191	0.00555107360099227\\
192	0.00555097269049799\\
193	0.00555086997169687\\
194	0.00555076541202491\\
195	0.00555065897835187\\
196	0.00555055063696405\\
197	0.00555044035354526\\
198	0.00555032809315718\\
199	0.00555021382022778\\
200	0.00555009749853958\\
201	0.00554997909121792\\
202	0.00554985856071892\\
203	0.00554973586881732\\
204	0.00554961097659401\\
205	0.00554948384442358\\
206	0.00554935443196139\\
207	0.00554922269813061\\
208	0.00554908860110916\\
209	0.00554895209831617\\
210	0.0055488131463986\\
211	0.00554867170121735\\
212	0.00554852771783352\\
213	0.00554838115049422\\
214	0.00554823195261837\\
215	0.0055480800767825\\
216	0.00554792547470603\\
217	0.00554776809723678\\
218	0.00554760789433616\\
219	0.00554744481506445\\
220	0.00554727880756594\\
221	0.00554710981905395\\
222	0.00554693779579595\\
223	0.0055467626830986\\
224	0.00554658442529286\\
225	0.00554640296571936\\
226	0.00554621824671349\\
227	0.00554603020959108\\
228	0.00554583879463392\\
229	0.00554564394107592\\
230	0.00554544558708915\\
231	0.0055452436697707\\
232	0.00554503812512965\\
233	0.00554482888807474\\
234	0.00554461589240254\\
235	0.00554439907078644\\
236	0.00554417835476615\\
237	0.00554395367473844\\
238	0.00554372495994837\\
239	0.00554349213848204\\
240	0.00554325513726026\\
241	0.0055430138820335\\
242	0.0055427682973783\\
243	0.00554251830669526\\
244	0.00554226383220837\\
245	0.00554200479496647\\
246	0.00554174111484607\\
247	0.0055414727105564\\
248	0.00554119949964632\\
249	0.00554092139851309\\
250	0.0055406383224134\\
251	0.00554035018547619\\
252	0.00554005690071743\\
253	0.00553975838005681\\
254	0.00553945453433581\\
255	0.00553914527333715\\
256	0.00553883050580524\\
257	0.00553851013946685\\
258	0.00553818408105177\\
259	0.00553785223631224\\
260	0.00553751451004054\\
261	0.00553717080608326\\
262	0.00553682102735139\\
263	0.005536465075824\\
264	0.00553610285254412\\
265	0.00553573425760475\\
266	0.00553535919012243\\
267	0.0055349775481963\\
268	0.00553458922884979\\
269	0.00553419412795238\\
270	0.00553379214011849\\
271	0.00553338315858182\\
272	0.00553296707504204\\
273	0.00553254377948372\\
274	0.00553211315996623\\
275	0.00553167510238606\\
276	0.00553122949021451\\
277	0.00553077620421378\\
278	0.00553031512213377\\
279	0.00552984611838764\\
280	0.0055293690637428\\
281	0.00552888382583964\\
282	0.00552839027033731\\
283	0.0055278882608906\\
284	0.0055273776591266\\
285	0.00552685832462144\\
286	0.00552633011487688\\
287	0.005525792885297\\
288	0.0055252464891646\\
289	0.00552469077761789\\
290	0.00552412559962685\\
291	0.00552355080196984\\
292	0.00552296622920996\\
293	0.00552237172367168\\
294	0.00552176712541721\\
295	0.00552115227222303\\
296	0.00552052699955636\\
297	0.00551989114055175\\
298	0.00551924452598748\\
299	0.00551858698426229\\
300	0.00551791834137189\\
301	0.00551723842088557\\
302	0.00551654704392292\\
303	0.0055158440291304\\
304	0.00551512919265814\\
305	0.00551440234813669\\
306	0.0055136633066536\\
307	0.00551291187673046\\
308	0.00551214786429946\\
309	0.00551137107268013\\
310	0.00551058130255629\\
311	0.0055097783519526\\
312	0.00550896201621118\\
313	0.00550813208796837\\
314	0.00550728835713096\\
315	0.00550643061085276\\
316	0.00550555863351065\\
317	0.00550467220668062\\
318	0.00550377110911339\\
319	0.00550285511670998\\
320	0.00550192400249677\\
321	0.00550097753660009\\
322	0.00550001548622049\\
323	0.00549903761560637\\
324	0.00549804368602693\\
325	0.00549703345574455\\
326	0.00549600667998613\\
327	0.00549496311091374\\
328	0.00549390249759393\\
329	0.00549282458596611\\
330	0.00549172911880951\\
331	0.0054906158357086\\
332	0.00548948447301702\\
333	0.00548833476381949\\
334	0.00548716643789195\\
335	0.00548597922165947\\
336	0.00548477283815144\\
337	0.00548354700695471\\
338	0.00548230144416329\\
339	0.00548103586232541\\
340	0.00547974997038695\\
341	0.0054784434736314\\
342	0.00547711607361598\\
343	0.00547576746810338\\
344	0.0054743973509896\\
345	0.00547300541222673\\
346	0.005471591337741\\
347	0.00547015480934577\\
348	0.00546869550464924\\
349	0.00546721309695638\\
350	0.00546570725516585\\
351	0.0054641776436606\\
352	0.00546262392219336\\
353	0.00546104574576619\\
354	0.00545944276450474\\
355	0.0054578146235275\\
356	0.00545616096281012\\
357	0.00545448141704604\\
358	0.00545277561550345\\
359	0.00545104318188011\\
360	0.00544928373415703\\
361	0.00544749688445256\\
362	0.00544568223887847\\
363	0.00544383939739993\\
364	0.00544196795370227\\
365	0.00544006749506626\\
366	0.00543813760225537\\
367	0.0054361778494178\\
368	0.00543418780400641\\
369	0.00543216702671998\\
370	0.00543011507146868\\
371	0.00542803148536672\\
372	0.00542591580875399\\
373	0.00542376757524806\\
374	0.00542158631182604\\
375	0.00541937153893374\\
376	0.00541712277061681\\
377	0.0054148395146646\\
378	0.00541252127275313\\
379	0.00541016754056764\\
380	0.00540777780787916\\
381	0.00540535155854355\\
382	0.00540288827038902\\
383	0.00540038741496609\\
384	0.00539784845716456\\
385	0.00539527085476927\\
386	0.00539265405805357\\
387	0.00538999750881576\\
388	0.00538730063262149\\
389	0.00538456269424022\\
390	0.00538178267450218\\
391	0.00537895948912136\\
392	0.00537609198820947\\
393	0.00537317895757003\\
394	0.00537021912229371\\
395	0.00536721115319308\\
396	0.00536415367653561\\
397	0.00536104528726815\\
398	0.00535788456594081\\
399	0.00535467009890918\\
400	0.00535140050032142\\
401	0.00534807443324338\\
402	0.00534469062506345\\
403	0.00534124786628992\\
404	0.00533774496506\\
405	0.00533418068466903\\
406	0.00533055374014991\\
407	0.00532686279465179\\
408	0.00532310645558164\\
409	0.00531928327028568\\
410	0.0053153917213833\\
411	0.0053114302217447\\
412	0.00530739710910915\\
413	0.00530329064035132\\
414	0.00529910898541757\\
415	0.00529485022097572\\
416	0.00529051232385152\\
417	0.00528609316436\\
418	0.00528159049964099\\
419	0.00527700196694627\\
420	0.00527232507615669\\
421	0.00526755720374208\\
422	0.00526269558811525\\
423	0.00525773732758168\\
424	0.0052526793845131\\
425	0.00524751860727432\\
426	0.00524225180987519\\
427	0.00523688478169057\\
428	0.00523142625771105\\
429	0.00522587457125705\\
430	0.00522022801988917\\
431	0.00521448486504337\\
432	0.00520864333315186\\
433	0.0052027016247223\\
434	0.00519665791541612\\
435	0.00519051035238131\\
436	0.00518425705670917\\
437	0.00517789612644616\\
438	0.005171425640055\\
439	0.00516484365994742\\
440	0.00515814823576862\\
441	0.00515133741063655\\
442	0.00514440922838081\\
443	0.00513736174019266\\
444	0.00513019301163666\\
445	0.00512290112979097\\
446	0.00511548421024411\\
447	0.00510794040379761\\
448	0.00510026790340343\\
449	0.00509246495436486\\
450	0.00508452987913126\\
451	0.00507646115502242\\
452	0.00506825492819732\\
453	0.00505988116691834\\
454	0.00505133523997187\\
455	0.00504261231016473\\
456	0.00503370731012738\\
457	0.00502461492443078\\
458	0.00501532957139311\\
459	0.00500584538340648\\
460	0.0049961561850006\\
461	0.00498625546799501\\
462	0.00497613636666769\\
463	0.0049657916299069\\
464	0.004955213591057\\
465	0.00494439413520791\\
466	0.00493332466344369\\
467	0.00492199605246461\\
468	0.00491039860409967\\
469	0.00489852196541588\\
470	0.00488635495113334\\
471	0.00487388502444875\\
472	0.00486110016872916\\
473	0.00484798846091057\\
474	0.00483453733968041\\
475	0.00482073356715673\\
476	0.00480656318852167\\
477	0.00479201148947316\\
478	0.00477706295186214\\
479	0.0047617012074535\\
480	0.00474590898969505\\
481	0.0047296680830029\\
482	0.00471295926900949\\
483	0.00469576226956938\\
484	0.00467805567959101\\
485	0.00465981685182614\\
486	0.00464102171153467\\
487	0.00462164470289859\\
488	0.00460165873227493\\
489	0.00458103299112202\\
490	0.00455972502689214\\
491	0.0045376851652642\\
492	0.00451487713810451\\
493	0.00449126302561625\\
494	0.00446680334652431\\
495	0.00444145660496162\\
496	0.00441517925139808\\
497	0.00438792565232686\\
498	0.00435964810079153\\
499	0.00433029686131218\\
500	0.0042998202602825\\
501	0.00426816483262271\\
502	0.00423527553509954\\
503	0.00420109605439452\\
504	0.0041655692281557\\
505	0.00412863760719055\\
506	0.00409024419353347\\
507	0.00405033339713849\\
508	0.00400885226367352\\
509	0.00396575203901821\\
510	0.0039209901516933\\
511	0.00387453271320261\\
512	0.00382635766291394\\
513	0.00377645868634219\\
514	0.00372485008048032\\
515	0.00367157241905168\\
516	0.0036167007092409\\
517	0.00356035948260369\\
518	0.00350273687447986\\
519	0.00344408652000935\\
520	0.00338472437097065\\
521	0.00332502096721129\\
522	0.00326555364965393\\
523	0.00320726580269645\\
524	0.00315410754475045\\
525	0.00310637335838345\\
526	0.00306411528959599\\
527	0.00302688819689068\\
528	0.00299070228713309\\
529	0.00295512081968544\\
530	0.00291990283029932\\
531	0.00288470889343574\\
532	0.00284941466422512\\
533	0.00281394805285373\\
534	0.00277826688557104\\
535	0.00274196152703712\\
536	0.00270497313438378\\
537	0.00266726288515937\\
538	0.0026287901198124\\
539	0.00258951542707469\\
540	0.0025494161112003\\
541	0.00250846903797176\\
542	0.00246665102411426\\
543	0.00242393945402507\\
544	0.00238030871636102\\
545	0.00233567897679858\\
546	0.00229069692309558\\
547	0.00224669736708306\\
548	0.00220394841017211\\
549	0.00216151286075968\\
550	0.00211858016627274\\
551	0.00207506500891316\\
552	0.00203094114340961\\
553	0.00198621792500414\\
554	0.00194091184632872\\
555	0.00189504487825128\\
556	0.00184864360533295\\
557	0.00180173986240179\\
558	0.00175437208562482\\
559	0.00170658633949408\\
560	0.00165892004054196\\
561	0.00161280548965377\\
562	0.00156727794385113\\
563	0.00152145163057755\\
564	0.00147531828604501\\
565	0.0014289079242206\\
566	0.00138225391875487\\
567	0.00133539122235288\\
568	0.00128835585617032\\
569	0.00124118449461087\\
570	0.00119391396189252\\
571	0.00114688118876282\\
572	0.00110017098334024\\
573	0.00105332337801152\\
574	0.00100637555536751\\
575	0.000959368346163701\\
576	0.000912345329873068\\
577	0.000865352773868407\\
578	0.00081843950911817\\
579	0.000771656718608598\\
580	0.000725057614047103\\
581	0.000678696970711006\\
582	0.000632630483284291\\
583	0.000586913897120376\\
584	0.000541601859672158\\
585	0.000496746426604039\\
586	0.000452395149029589\\
587	0.000408588670365811\\
588	0.000365357795666471\\
589	0.000322720123375259\\
590	0.000280676711614414\\
591	0.000239312817487403\\
592	0.00019878864496184\\
593	0.000159293651685586\\
594	0.000121079888727804\\
595	8.45520570083909e-05\\
596	5.05092148680371e-05\\
597	2.07908715710836e-05\\
598	0\\
599	0\\
600	0\\
};
\addplot [color=mycolor6,solid,forget plot]
  table[row sep=crcr]{%
1	0.00564414169708746\\
2	0.00564413625517687\\
3	0.00564413071987849\\
4	0.00564412508958927\\
5	0.00564411936267863\\
6	0.00564411353748807\\
7	0.00564410761233051\\
8	0.00564410158549004\\
9	0.00564409545522124\\
10	0.00564408921974873\\
11	0.00564408287726674\\
12	0.00564407642593847\\
13	0.0056440698638956\\
14	0.00564406318923781\\
15	0.00564405640003209\\
16	0.00564404949431236\\
17	0.00564404247007876\\
18	0.00564403532529714\\
19	0.00564402805789849\\
20	0.00564402066577821\\
21	0.00564401314679571\\
22	0.00564400549877365\\
23	0.00564399771949732\\
24	0.00564398980671406\\
25	0.00564398175813259\\
26	0.00564397357142226\\
27	0.00564396524421251\\
28	0.00564395677409218\\
29	0.00564394815860864\\
30	0.00564393939526732\\
31	0.00564393048153081\\
32	0.00564392141481821\\
33	0.00564391219250438\\
34	0.00564390281191916\\
35	0.00564389327034665\\
36	0.00564388356502433\\
37	0.00564387369314236\\
38	0.00564386365184279\\
39	0.00564385343821856\\
40	0.00564384304931287\\
41	0.00564383248211816\\
42	0.00564382173357545\\
43	0.00564381080057326\\
44	0.00564379967994678\\
45	0.00564378836847701\\
46	0.00564377686288969\\
47	0.00564376515985457\\
48	0.00564375325598425\\
49	0.00564374114783338\\
50	0.00564372883189747\\
51	0.00564371630461208\\
52	0.00564370356235165\\
53	0.00564369060142858\\
54	0.00564367741809204\\
55	0.0056436640085269\\
56	0.0056436503688528\\
57	0.00564363649512285\\
58	0.00564362238332254\\
59	0.00564360802936866\\
60	0.00564359342910799\\
61	0.00564357857831631\\
62	0.00564356347269692\\
63	0.0056435481078796\\
64	0.00564353247941936\\
65	0.00564351658279498\\
66	0.00564350041340793\\
67	0.00564348396658084\\
68	0.00564346723755637\\
69	0.00564345022149561\\
70	0.00564343291347689\\
71	0.00564341530849428\\
72	0.00564339740145613\\
73	0.00564337918718361\\
74	0.00564336066040927\\
75	0.00564334181577543\\
76	0.00564332264783279\\
77	0.0056433031510387\\
78	0.00564328331975563\\
79	0.0056432631482496\\
80	0.00564324263068842\\
81	0.00564322176114014\\
82	0.00564320053357121\\
83	0.0056431789418448\\
84	0.00564315697971908\\
85	0.00564313464084535\\
86	0.00564311191876622\\
87	0.00564308880691379\\
88	0.00564306529860771\\
89	0.00564304138705329\\
90	0.00564301706533944\\
91	0.00564299232643684\\
92	0.00564296716319575\\
93	0.00564294156834404\\
94	0.00564291553448503\\
95	0.00564288905409542\\
96	0.005642862119523\\
97	0.00564283472298454\\
98	0.00564280685656343\\
99	0.00564277851220748\\
100	0.00564274968172658\\
101	0.00564272035679014\\
102	0.00564269052892486\\
103	0.00564266018951221\\
104	0.00564262932978591\\
105	0.00564259794082932\\
106	0.00564256601357286\\
107	0.00564253353879135\\
108	0.00564250050710135\\
109	0.00564246690895835\\
110	0.00564243273465392\\
111	0.0056423979743129\\
112	0.00564236261789058\\
113	0.00564232665516947\\
114	0.00564229007575648\\
115	0.00564225286907973\\
116	0.00564221502438551\\
117	0.00564217653073491\\
118	0.00564213737700059\\
119	0.00564209755186348\\
120	0.00564205704380929\\
121	0.00564201584112511\\
122	0.00564197393189578\\
123	0.00564193130400025\\
124	0.0056418879451079\\
125	0.00564184384267478\\
126	0.00564179898393962\\
127	0.00564175335591998\\
128	0.0056417069454081\\
129	0.00564165973896684\\
130	0.00564161172292541\\
131	0.00564156288337497\\
132	0.00564151320616429\\
133	0.00564146267689518\\
134	0.00564141128091767\\
135	0.00564135900332546\\
136	0.00564130582895089\\
137	0.00564125174235987\\
138	0.00564119672784683\\
139	0.00564114076942943\\
140	0.00564108385084303\\
141	0.00564102595553514\\
142	0.00564096706665966\\
143	0.005640907167071\\
144	0.00564084623931801\\
145	0.00564078426563768\\
146	0.00564072122794868\\
147	0.00564065710784477\\
148	0.00564059188658799\\
149	0.00564052554510143\\
150	0.0056404580639621\\
151	0.0056403894233934\\
152	0.00564031960325727\\
153	0.00564024858304622\\
154	0.00564017634187505\\
155	0.00564010285847223\\
156	0.00564002811117116\\
157	0.00563995207790084\\
158	0.00563987473617662\\
159	0.00563979606309024\\
160	0.00563971603529977\\
161	0.00563963462901919\\
162	0.0056395518200075\\
163	0.00563946758355766\\
164	0.00563938189448501\\
165	0.00563929472711542\\
166	0.00563920605527308\\
167	0.00563911585226782\\
168	0.00563902409088234\\
169	0.00563893074335885\\
170	0.00563883578138554\\
171	0.00563873917608285\\
172	0.00563864089798935\\
173	0.00563854091704757\\
174	0.00563843920258979\\
175	0.00563833572332355\\
176	0.00563823044731771\\
177	0.0056381233419882\\
178	0.00563801437408465\\
179	0.00563790350967728\\
180	0.00563779071414467\\
181	0.00563767595216268\\
182	0.00563755918769461\\
183	0.00563744038398316\\
184	0.00563731950354441\\
185	0.00563719650816463\\
186	0.00563707135889997\\
187	0.00563694401607988\\
188	0.00563681443931492\\
189	0.00563668258750926\\
190	0.00563654841887897\\
191	0.0056364118909759\\
192	0.00563627296071807\\
193	0.00563613158442602\\
194	0.00563598771786444\\
195	0.00563584131628652\\
196	0.00563569233447492\\
197	0.00563554072676165\\
198	0.00563538644697248\\
199	0.00563522944814338\\
200	0.00563506968249567\\
201	0.00563490710142288\\
202	0.00563474165547733\\
203	0.00563457329435654\\
204	0.00563440196688957\\
205	0.00563422762102303\\
206	0.00563405020380734\\
207	0.00563386966138245\\
208	0.00563368593896365\\
209	0.00563349898082722\\
210	0.00563330873029588\\
211	0.00563311512972439\\
212	0.0056329181204847\\
213	0.00563271764295124\\
214	0.00563251363648603\\
215	0.00563230603942387\\
216	0.0056320947890571\\
217	0.00563187982162077\\
218	0.00563166107227743\\
219	0.00563143847510205\\
220	0.00563121196306666\\
221	0.00563098146802532\\
222	0.00563074692069891\\
223	0.00563050825065989\\
224	0.00563026538631715\\
225	0.00563001825490073\\
226	0.00562976678244683\\
227	0.0056295108937827\\
228	0.00562925051251158\\
229	0.00562898556099765\\
230	0.00562871596035131\\
231	0.0056284416304141\\
232	0.00562816248974406\\
233	0.00562787845560098\\
234	0.00562758944393172\\
235	0.0056272953693556\\
236	0.00562699614514988\\
237	0.00562669168323515\\
238	0.00562638189416079\\
239	0.0056260666870904\\
240	0.00562574596978704\\
241	0.00562541964859858\\
242	0.00562508762844261\\
243	0.0056247498127913\\
244	0.00562440610365575\\
245	0.00562405640157011\\
246	0.00562370060557519\\
247	0.00562333861320116\\
248	0.00562297032044985\\
249	0.00562259562177602\\
250	0.00562221441006732\\
251	0.00562182657662356\\
252	0.00562143201113391\\
253	0.00562103060165317\\
254	0.00562062223457563\\
255	0.00562020679460738\\
256	0.00561978416473611\\
257	0.00561935422619877\\
258	0.00561891685844633\\
259	0.0056184719391061\\
260	0.00561801934394091\\
261	0.00561755894680571\\
262	0.00561709061960082\\
263	0.00561661423222263\\
264	0.00561612965251152\\
265	0.00561563674619742\\
266	0.00561513537684337\\
267	0.00561462540578792\\
268	0.00561410669208684\\
269	0.00561357909245545\\
270	0.00561304246121241\\
271	0.00561249665022646\\
272	0.005611941508868\\
273	0.00561137688396609\\
274	0.00561080261977376\\
275	0.00561021855794219\\
276	0.00560962453750519\\
277	0.00560902039487498\\
278	0.00560840596384905\\
279	0.00560778107562823\\
280	0.00560714555884466\\
281	0.00560649923957224\\
282	0.00560584194129897\\
283	0.00560517348489965\\
284	0.00560449368860877\\
285	0.00560380236799386\\
286	0.00560309933592902\\
287	0.00560238440256894\\
288	0.0056016573753236\\
289	0.00560091805883311\\
290	0.00560016625494342\\
291	0.00559940176268257\\
292	0.00559862437823763\\
293	0.00559783389493254\\
294	0.00559703010320651\\
295	0.00559621279059369\\
296	0.0055953817417038\\
297	0.00559453673820359\\
298	0.00559367755879988\\
299	0.00559280397922369\\
300	0.00559191577221573\\
301	0.00559101270751374\\
302	0.00559009455184119\\
303	0.00558916106889795\\
304	0.00558821201935285\\
305	0.00558724716083838\\
306	0.00558626624794766\\
307	0.00558526903223386\\
308	0.00558425526221214\\
309	0.00558322468336484\\
310	0.00558217703814907\\
311	0.00558111206600827\\
312	0.00558002950338697\\
313	0.00557892908374936\\
314	0.00557781053760209\\
315	0.00557667359252137\\
316	0.00557551797318496\\
317	0.00557434340140912\\
318	0.00557314959619112\\
319	0.00557193627375752\\
320	0.00557070314761891\\
321	0.00556944992863151\\
322	0.00556817632506571\\
323	0.00556688204268277\\
324	0.0055655667848199\\
325	0.0055642302524841\\
326	0.00556287214445611\\
327	0.00556149215740428\\
328	0.00556008998601007\\
329	0.00555866532310511\\
330	0.0055572178598215\\
331	0.00555574728575555\\
332	0.00555425328914656\\
333	0.00555273555707142\\
334	0.00555119377565612\\
335	0.00554962763030557\\
336	0.00554803680595303\\
337	0.00554642098733027\\
338	0.00554477985926037\\
339	0.00554311310697427\\
340	0.00554142041645326\\
341	0.00553970147479857\\
342	0.00553795597063039\\
343	0.00553618359451826\\
344	0.00553438403944448\\
345	0.0055325570013031\\
346	0.00553070217943665\\
347	0.00552881927721283\\
348	0.00552690800264369\\
349	0.00552496806904948\\
350	0.00552299919577002\\
351	0.00552100110892575\\
352	0.00551897354223103\\
353	0.00551691623786212\\
354	0.00551482894738188\\
355	0.00551271143272327\\
356	0.00551056346723349\\
357	0.00550838483677973\\
358	0.00550617534091777\\
359	0.00550393479412282\\
360	0.00550166302708257\\
361	0.00549935988804942\\
362	0.00549702524424956\\
363	0.00549465898334295\\
364	0.00549226101492717\\
365	0.00548983127207504\\
366	0.00548736971289305\\
367	0.00548487632208314\\
368	0.00548235111248626\\
369	0.00547979412658058\\
370	0.00547720543789952\\
371	0.00547458515232831\\
372	0.00547193340922715\\
373	0.00546925038231827\\
374	0.0054665362802615\\
375	0.00546379134682676\\
376	0.00546101586055519\\
377	0.00545821013377916\\
378	0.00545537451084841\\
379	0.00545250936538243\\
380	0.00544961509633896\\
381	0.00544669212265489\\
382	0.00544374087617907\\
383	0.00544076179257658\\
384	0.0054377552998406\\
385	0.0054347218039998\\
386	0.0054316616715437\\
387	0.00542857520801726\\
388	0.00542546263241958\\
389	0.00542232405135473\\
390	0.00541915943173047\\
391	0.00541596856053574\\
392	0.00541275099852629\\
393	0.00540950602802968\\
394	0.00540623259578367\\
395	0.00540292925287511\\
396	0.00539959409575478\\
397	0.00539622471454812\\
398	0.00539281815841277\\
399	0.00538937093322477\\
400	0.00538587905482985\\
401	0.00538233819374411\\
402	0.00537874396997767\\
403	0.00537509250059577\\
404	0.00537138136169903\\
405	0.00536760963869151\\
406	0.00536377640670455\\
407	0.00535988073120857\\
408	0.00535592166875451\\
409	0.00535189826786303\\
410	0.00534780957007553\\
411	0.00534365461117891\\
412	0.0053394324226149\\
413	0.00533514203308011\\
414	0.00533078247031827\\
415	0.0053263527630979\\
416	0.00532185194335856\\
417	0.0053172790484941\\
418	0.0053126331237232\\
419	0.00530791322447705\\
420	0.00530311841874507\\
421	0.00529824778923973\\
422	0.00529330043525395\\
423	0.00528827547412149\\
424	0.00528317204227609\\
425	0.00527798929587917\\
426	0.00527272640940693\\
427	0.00526738255837269\\
428	0.00526195660210042\\
429	0.00525644697598015\\
430	0.00525085202010332\\
431	0.00524516996856736\\
432	0.00523939893776977\\
433	0.00523353691359344\\
434	0.00522758173720634\\
435	0.00522153108974944\\
436	0.00521538247611547\\
437	0.00520913320773026\\
438	0.00520278038449278\\
439	0.0051963208761216\\
440	0.00518975130328123\\
441	0.0051830680190074\\
442	0.00517626709101539\\
443	0.00516934428574013\\
444	0.00516229505523642\\
445	0.00515511452830036\\
446	0.00514779750747866\\
447	0.00514033847401999\\
448	0.00513273160350065\\
449	0.00512497079664492\\
450	0.00511704973592452\\
451	0.00510896200076135\\
452	0.00510070378862286\\
453	0.00509229594263602\\
454	0.00508373464752929\\
455	0.00507501624632246\\
456	0.00506613720024991\\
457	0.00505709388351448\\
458	0.00504788255170718\\
459	0.00503849933846256\\
460	0.00502894026034941\\
461	0.00501920122093163\\
462	0.00500927797509956\\
463	0.00499916613887364\\
464	0.00498886118326885\\
465	0.0049783584277492\\
466	0.0049676530324458\\
467	0.00495673999020768\\
468	0.00494561412153549\\
469	0.00493427008262226\\
470	0.00492270241860566\\
471	0.00491090576179196\\
472	0.00489884744474724\\
473	0.00488650547011184\\
474	0.00487386914060477\\
475	0.0048609271690671\\
476	0.00484766764446001\\
477	0.00483407799596508\\
478	0.00482014494740563\\
479	0.004805854470123\\
480	0.00479119173146963\\
481	0.00477614103689769\\
482	0.00476068576490874\\
483	0.00474480829222686\\
484	0.00472848990354356\\
485	0.00471171066811643\\
486	0.00469444922446338\\
487	0.00467668224998897\\
488	0.00465838538681553\\
489	0.00463953438165116\\
490	0.00462010347189625\\
491	0.00460006522725015\\
492	0.00457938821124362\\
493	0.00455802863850385\\
494	0.00453593713953987\\
495	0.00451307676840705\\
496	0.00448940875107804\\
497	0.00446489274774366\\
498	0.00443948628249789\\
499	0.00441314468988377\\
500	0.00438582105601528\\
501	0.00435746622138655\\
502	0.0043280288572434\\
503	0.00429745539131718\\
504	0.00426569009173831\\
505	0.00423267521069181\\
506	0.00419835120334439\\
507	0.00416265704191008\\
508	0.00412553064833676\\
509	0.00408690947531639\\
510	0.00404673127247722\\
511	0.0040049350833286\\
512	0.00396146252888379\\
513	0.00391625944696082\\
514	0.00386927797080916\\
515	0.00382047918193405\\
516	0.00376983645483958\\
517	0.00371733938793335\\
518	0.00366299856508242\\
519	0.0036068522353349\\
520	0.00354898254480159\\
521	0.00348952212899389\\
522	0.00342866087146759\\
523	0.00336665476859641\\
524	0.00330382152996528\\
525	0.00324053311458391\\
526	0.0031773849182677\\
527	0.00311536749133956\\
528	0.00305847479549083\\
529	0.00300703652234188\\
530	0.0029611558647825\\
531	0.00292045584447661\\
532	0.00288119041155718\\
533	0.00284263737455762\\
534	0.00280433019479108\\
535	0.00276613286193941\\
536	0.00272783883980301\\
537	0.00268936526453775\\
538	0.00265065306074691\\
539	0.00261155905477273\\
540	0.00257176095863499\\
541	0.0025312176445629\\
542	0.0024898888746985\\
543	0.00244772979898243\\
544	0.00240470570191919\\
545	0.00236079274626476\\
546	0.00231596758188079\\
547	0.0022701870295366\\
548	0.00222336772233065\\
549	0.0021766009736321\\
550	0.00213084111050196\\
551	0.00208636429435196\\
552	0.00204199580507713\\
553	0.00199714708603788\\
554	0.00195173779208016\\
555	0.00190574754438444\\
556	0.00185919235620512\\
557	0.0018120959308981\\
558	0.00176448917511599\\
559	0.00171640898605244\\
560	0.00166789981490248\\
561	0.00161900323277768\\
562	0.00157077130801954\\
563	0.00152412909216968\\
564	0.00147778367616564\\
565	0.00143118179454771\\
566	0.00138432726713373\\
567	0.00133725430873315\\
568	0.00129000081580688\\
569	0.00124260613446559\\
570	0.00119511043658036\\
571	0.0011475541064327\\
572	0.00110018398429605\\
573	0.00105332347969338\\
574	0.00100637557683123\\
575	0.000959368356499549\\
576	0.000912345335012991\\
577	0.000865352776319937\\
578	0.000818439510213676\\
579	0.000771656719059582\\
580	0.000725057614214845\\
581	0.000678696970765909\\
582	0.000632630483299512\\
583	0.000586913897123733\\
584	0.00054160185967268\\
585	0.000496746426604084\\
586	0.000452395149029592\\
587	0.000408588670365816\\
588	0.000365357795666476\\
589	0.000322720123375263\\
590	0.000280676711614417\\
591	0.000239312817487403\\
592	0.00019878864496184\\
593	0.000159293651685585\\
594	0.000121079888727804\\
595	8.45520570083909e-05\\
596	5.05092148680374e-05\\
597	2.07908715710837e-05\\
598	0\\
599	0\\
600	0\\
};
\addplot [color=mycolor7,solid,forget plot]
  table[row sep=crcr]{%
1	0.00588219123805133\\
2	0.00588218208726686\\
3	0.00588217278113544\\
4	0.00588216331702408\\
5	0.00588215369225527\\
6	0.00588214390410623\\
7	0.00588213394980824\\
8	0.00588212382654572\\
9	0.00588211353145555\\
10	0.00588210306162625\\
11	0.00588209241409704\\
12	0.00588208158585732\\
13	0.00588207057384544\\
14	0.00588205937494817\\
15	0.00588204798599965\\
16	0.00588203640378059\\
17	0.00588202462501728\\
18	0.00588201264638077\\
19	0.00588200046448591\\
20	0.00588198807589044\\
21	0.00588197547709392\\
22	0.00588196266453694\\
23	0.00588194963459998\\
24	0.00588193638360238\\
25	0.00588192290780144\\
26	0.00588190920339136\\
27	0.00588189526650207\\
28	0.00588188109319825\\
29	0.00588186667947823\\
30	0.00588185202127291\\
31	0.00588183711444447\\
32	0.0058818219547854\\
33	0.00588180653801729\\
34	0.00588179085978959\\
35	0.00588177491567842\\
36	0.00588175870118537\\
37	0.00588174221173632\\
38	0.00588172544267996\\
39	0.00588170838928676\\
40	0.00588169104674747\\
41	0.00588167341017188\\
42	0.00588165547458745\\
43	0.00588163723493796\\
44	0.00588161868608199\\
45	0.00588159982279169\\
46	0.00588158063975125\\
47	0.00588156113155537\\
48	0.00588154129270789\\
49	0.00588152111762016\\
50	0.00588150060060958\\
51	0.00588147973589798\\
52	0.00588145851761012\\
53	0.00588143693977191\\
54	0.00588141499630895\\
55	0.00588139268104478\\
56	0.00588136998769915\\
57	0.00588134690988632\\
58	0.0058813234411134\\
59	0.00588129957477834\\
60	0.00588127530416842\\
61	0.00588125062245813\\
62	0.00588122552270752\\
63	0.00588119999786017\\
64	0.00588117404074132\\
65	0.00588114764405584\\
66	0.00588112080038635\\
67	0.00588109350219113\\
68	0.00588106574180199\\
69	0.0058810375114224\\
70	0.00588100880312507\\
71	0.00588097960885001\\
72	0.00588094992040228\\
73	0.00588091972944976\\
74	0.00588088902752087\\
75	0.00588085780600227\\
76	0.00588082605613652\\
77	0.00588079376901975\\
78	0.00588076093559925\\
79	0.00588072754667089\\
80	0.00588069359287686\\
81	0.00588065906470292\\
82	0.00588062395247593\\
83	0.00588058824636128\\
84	0.00588055193636014\\
85	0.00588051501230688\\
86	0.00588047746386625\\
87	0.00588043928053062\\
88	0.00588040045161711\\
89	0.00588036096626489\\
90	0.00588032081343208\\
91	0.00588027998189295\\
92	0.00588023846023473\\
93	0.00588019623685472\\
94	0.00588015329995716\\
95	0.00588010963755001\\
96	0.00588006523744187\\
97	0.00588002008723857\\
98	0.00587997417434008\\
99	0.005879927485937\\
100	0.00587988000900718\\
101	0.0058798317303124\\
102	0.00587978263639473\\
103	0.00587973271357297\\
104	0.00587968194793907\\
105	0.00587963032535454\\
106	0.0058795778314466\\
107	0.00587952445160447\\
108	0.00587947017097546\\
109	0.00587941497446109\\
110	0.00587935884671325\\
111	0.00587930177212999\\
112	0.00587924373485153\\
113	0.00587918471875616\\
114	0.00587912470745592\\
115	0.00587906368429241\\
116	0.0058790016323324\\
117	0.00587893853436342\\
118	0.0058788743728894\\
119	0.00587880913012591\\
120	0.00587874278799578\\
121	0.00587867532812425\\
122	0.00587860673183428\\
123	0.00587853698014171\\
124	0.00587846605375035\\
125	0.00587839393304693\\
126	0.00587832059809621\\
127	0.00587824602863568\\
128	0.00587817020407035\\
129	0.0058780931034675\\
130	0.00587801470555129\\
131	0.0058779349886972\\
132	0.00587785393092651\\
133	0.00587777150990062\\
134	0.00587768770291526\\
135	0.00587760248689474\\
136	0.00587751583838573\\
137	0.00587742773355146\\
138	0.00587733814816538\\
139	0.00587724705760487\\
140	0.00587715443684479\\
141	0.00587706026045104\\
142	0.00587696450257372\\
143	0.00587686713694038\\
144	0.00587676813684901\\
145	0.00587666747516092\\
146	0.00587656512429337\\
147	0.00587646105621216\\
148	0.00587635524242384\\
149	0.00587624765396788\\
150	0.00587613826140854\\
151	0.00587602703482644\\
152	0.00587591394381001\\
153	0.00587579895744645\\
154	0.00587568204431256\\
155	0.00587556317246514\\
156	0.00587544230943093\\
157	0.00587531942219636\\
158	0.00587519447719651\\
159	0.00587506744030385\\
160	0.00587493827681619\\
161	0.00587480695144415\\
162	0.0058746734282978\\
163	0.0058745376708726\\
164	0.00587439964203436\\
165	0.00587425930400317\\
166	0.00587411661833645\\
167	0.00587397154591045\\
168	0.00587382404690054\\
169	0.00587367408075979\\
170	0.00587352160619589\\
171	0.00587336658114587\\
172	0.00587320896274888\\
173	0.00587304870731608\\
174	0.00587288577029793\\
175	0.00587272010624825\\
176	0.0058725516687845\\
177	0.0058723804105444\\
178	0.0058722062831377\\
179	0.00587202923709321\\
180	0.00587184922180032\\
181	0.00587166618544422\\
182	0.00587148007493482\\
183	0.005871290835828\\
184	0.0058710984122396\\
185	0.00587090274675077\\
186	0.00587070378030529\\
187	0.00587050145209808\\
188	0.00587029569945591\\
189	0.00587008645771111\\
190	0.0058698736600702\\
191	0.00586965723748165\\
192	0.00586943711850872\\
193	0.0058692132292203\\
194	0.00586898549312512\\
195	0.0058687538312082\\
196	0.00586851816221758\\
197	0.00586827840359564\\
198	0.00586803447412705\\
199	0.00586778630120221\\
200	0.00586753381139697\\
201	0.00586727693003613\\
202	0.0058670155811731\\
203	0.00586674968756943\\
204	0.00586647917067383\\
205	0.00586620395060098\\
206	0.00586592394611\\
207	0.00586563907458275\\
208	0.00586534925200168\\
209	0.00586505439292744\\
210	0.00586475441047631\\
211	0.00586444921629699\\
212	0.00586413872054743\\
213	0.00586382283187114\\
214	0.00586350145737323\\
215	0.00586317450259602\\
216	0.0058628418714946\\
217	0.00586250346641158\\
218	0.00586215918805206\\
219	0.00586180893545772\\
220	0.00586145260598089\\
221	0.00586109009525804\\
222	0.00586072129718302\\
223	0.00586034610387987\\
224	0.00585996440567515\\
225	0.00585957609106991\\
226	0.00585918104671132\\
227	0.00585877915736358\\
228	0.00585837030587866\\
229	0.00585795437316637\\
230	0.00585753123816403\\
231	0.00585710077780552\\
232	0.0058566628669898\\
233	0.00585621737854895\\
234	0.00585576418321546\\
235	0.005855303149589\\
236	0.00585483414410254\\
237	0.00585435703098766\\
238	0.00585387167223928\\
239	0.00585337792757956\\
240	0.00585287565442103\\
241	0.00585236470782888\\
242	0.00585184494048238\\
243	0.00585131620263546\\
244	0.00585077834207637\\
245	0.00585023120408629\\
246	0.00584967463139692\\
247	0.00584910846414732\\
248	0.00584853253983935\\
249	0.0058479466932923\\
250	0.00584735075659636\\
251	0.00584674455906493\\
252	0.00584612792718618\\
253	0.005845500684573\\
254	0.00584486265191258\\
255	0.0058442136469145\\
256	0.0058435534842581\\
257	0.00584288197553906\\
258	0.00584219892921499\\
259	0.00584150415055066\\
260	0.00584079744156258\\
261	0.00584007860096303\\
262	0.00583934742410421\\
263	0.00583860370292198\\
264	0.00583784722588\\
265	0.0058370777779139\\
266	0.00583629514037624\\
267	0.00583549909098182\\
268	0.00583468940375375\\
269	0.00583386584897045\\
270	0.00583302819311345\\
271	0.00583217619881607\\
272	0.00583130962481249\\
273	0.00583042822588784\\
274	0.00582953175282786\\
275	0.00582861995236874\\
276	0.00582769256714601\\
277	0.00582674933564253\\
278	0.00582578999213452\\
279	0.00582481426663568\\
280	0.00582382188483897\\
281	0.00582281256805669\\
282	0.00582178603316027\\
283	0.00582074199251977\\
284	0.00581968015394315\\
285	0.0058186002206152\\
286	0.00581750189103652\\
287	0.00581638485896217\\
288	0.00581524881334033\\
289	0.00581409343825085\\
290	0.0058129184128441\\
291	0.00581172341127963\\
292	0.00581050810266517\\
293	0.00580927215099572\\
294	0.00580801521509329\\
295	0.0058067369485468\\
296	0.00580543699965241\\
297	0.00580411501135488\\
298	0.00580277062118932\\
299	0.0058014034612239\\
300	0.00580001315800364\\
301	0.00579859933249527\\
302	0.00579716160003328\\
303	0.00579569957026767\\
304	0.00579421284711314\\
305	0.00579270102870023\\
306	0.00579116370732854\\
307	0.00578960046942194\\
308	0.00578801089548685\\
309	0.00578639456007275\\
310	0.00578475103173618\\
311	0.00578307987300787\\
312	0.00578138064036386\\
313	0.00577965288420035\\
314	0.00577789614881355\\
315	0.00577610997238415\\
316	0.00577429388696708\\
317	0.0057724474184875\\
318	0.00577057008674308\\
319	0.00576866140541335\\
320	0.00576672088207668\\
321	0.00576474801823573\\
322	0.00576274230935219\\
323	0.00576070324489125\\
324	0.00575863030837713\\
325	0.00575652297746055\\
326	0.00575438072399893\\
327	0.00575220301415095\\
328	0.00574998930848624\\
329	0.00574773906211215\\
330	0.00574545172481836\\
331	0.00574312674124184\\
332	0.00574076355105323\\
333	0.00573836158916711\\
334	0.00573592028597793\\
335	0.00573343906762434\\
336	0.00573091735628411\\
337	0.00572835457050286\\
338	0.00572575012555948\\
339	0.00572310343387171\\
340	0.00572041390544555\\
341	0.00571768094837294\\
342	0.00571490396938198\\
343	0.00571208237444478\\
344	0.00570921556944856\\
345	0.00570630296093632\\
346	0.00570334395692322\\
347	0.00570033796779702\\
348	0.00569728440731028\\
349	0.00569418269367389\\
350	0.00569103225076202\\
351	0.00568783250943989\\
352	0.00568458290902718\\
353	0.00568128289891095\\
354	0.00567793194032404\\
355	0.00567452950830646\\
356	0.00567107509386924\\
357	0.00566756820638286\\
358	0.00566400837621448\\
359	0.00566039515764186\\
360	0.00565672813207408\\
361	0.00565300691161411\\
362	0.00564923114300126\\
363	0.00564540051197693\\
364	0.00564151474812193\\
365	0.00563757363021977\\
366	0.00563357699220581\\
367	0.00562952472977064\\
368	0.00562541680769257\\
369	0.00562125326798283\\
370	0.00561703423893659\\
371	0.00561275994519176\\
372	0.0056084307189075\\
373	0.00560404701218465\\
374	0.00559960941085824\\
375	0.00559511864980179\\
376	0.00559057562988734\\
377	0.00558598143674747\\
378	0.00558133736148083\\
379	0.00557664492342907\\
380	0.00557190589512597\\
381	0.00556712232947278\\
382	0.00556229658911968\\
383	0.00555743137792129\\
384	0.00555252977416729\\
385	0.00554759526504962\\
386	0.00554263178149162\\
387	0.00553764373200467\\
388	0.00553263603363741\\
389	0.00552761413726744\\
390	0.00552258404403517\\
391	0.00551755230432516\\
392	0.00551252599174899\\
393	0.00550751264045885\\
394	0.00550252012886844\\
395	0.00549755648547518\\
396	0.0054926295808739\\
397	0.00548774667063801\\
398	0.00548291373303975\\
399	0.00547813451476601\\
400	0.00547340916938308\\
401	0.00546873232430192\\
402	0.00546409034107289\\
403	0.00545945746060361\\
404	0.00545479065452195\\
405	0.00545005260128847\\
406	0.00544524264586333\\
407	0.00544036015771497\\
408	0.00543540453319037\\
409	0.00543037519796206\\
410	0.00542527160953924\\
411	0.00542009325982682\\
412	0.00541483967771263\\
413	0.0054095104316578\\
414	0.00540410513226049\\
415	0.00539862343475789\\
416	0.0053930650414277\\
417	0.00538742970385511\\
418	0.00538171722506037\\
419	0.00537592746158062\\
420	0.00537006032535792\\
421	0.00536411578537901\\
422	0.00535809386903167\\
423	0.00535199466314406\\
424	0.00534581831466087\\
425	0.0053395650308661\\
426	0.00533323507899149\\
427	0.00532682878551841\\
428	0.00532034654460782\\
429	0.00531378882878005\\
430	0.00530715618751142\\
431	0.00530044924399701\\
432	0.00529366868970234\\
433	0.00528681527658234\\
434	0.00527988980632675\\
435	0.00527289311561445\\
436	0.00526582605648935\\
437	0.00525868947083078\\
438	0.00525148415773978\\
439	0.00524421083250835\\
440	0.0052368700757049\\
441	0.00522946227082878\\
442	0.00522198752898299\\
443	0.00521444559914102\\
444	0.00520683576294551\\
445	0.00519915671370076\\
446	0.00519140642050471\\
447	0.0051835819805921\\
448	0.00517567946634349\\
449	0.00516769377863308\\
450	0.00515961852592683\\
451	0.005151445958989\\
452	0.00514316699999214\\
453	0.0051347713274009\\
454	0.00512624707099367\\
455	0.00511758257154256\\
456	0.00510876815839775\\
457	0.00509979921378732\\
458	0.0050906715587827\\
459	0.00508138073314177\\
460	0.00507192196861216\\
461	0.00506229016033015\\
462	0.00505247983613435\\
463	0.00504248512515146\\
464	0.00503229972450469\\
465	0.00502191686506552\\
466	0.00501132927698835\\
467	0.00500052915644913\\
468	0.00498950813675828\\
469	0.00497825727254909\\
470	0.0049667670647291\\
471	0.00495502762238985\\
472	0.0049430541617577\\
473	0.00493085021822894\\
474	0.00491840729180021\\
475	0.00490571632057168\\
476	0.00489276766813617\\
477	0.00487955113536917\\
478	0.00486605606432649\\
479	0.00485227131562047\\
480	0.00483818530089091\\
481	0.00482378604549152\\
482	0.00480906125528886\\
483	0.00479399839257722\\
484	0.00477858473533605\\
485	0.00476280739141535\\
486	0.00474665320711469\\
487	0.00473010861408686\\
488	0.00471314021482741\\
489	0.00469570132323347\\
490	0.00467777053771616\\
491	0.00465932506538153\\
492	0.00464034054077538\\
493	0.00462079104619254\\
494	0.00460064903615265\\
495	0.00457988331032009\\
496	0.00455845203088011\\
497	0.00453630139290828\\
498	0.00451339335556852\\
499	0.00448968753890518\\
500	0.00446514170582074\\
501	0.00443971084097204\\
502	0.0044133469625095\\
503	0.0043860022132521\\
504	0.00435762631213689\\
505	0.00432816648945773\\
506	0.00429756746269201\\
507	0.00426577144806584\\
508	0.00423271823782212\\
509	0.0041983453407199\\
510	0.00416258819980435\\
511	0.00412538050824811\\
512	0.00408665464676958\\
513	0.0040463422770657\\
514	0.00400437516965353\\
515	0.00396068628906017\\
516	0.00391521098887378\\
517	0.00386788870705819\\
518	0.00381866515000677\\
519	0.00376749505137355\\
520	0.00371434520876362\\
521	0.0036591985190329\\
522	0.00360205971707007\\
523	0.00354297258640321\\
524	0.00348201672566389\\
525	0.00341931897600911\\
526	0.00335506276957203\\
527	0.00328949729187063\\
528	0.00322293267212333\\
529	0.00315573339194042\\
530	0.00308848525482472\\
531	0.00302213661827374\\
532	0.00296050629345285\\
533	0.00290430468982034\\
534	0.00285370670007441\\
535	0.00280842771725674\\
536	0.00276558846072333\\
537	0.00272363262543577\\
538	0.00268211738834541\\
539	0.00264064638487863\\
540	0.00259910961814965\\
541	0.00255741339571434\\
542	0.00251548061943308\\
543	0.00247326495309667\\
544	0.00243050165550552\\
545	0.00238698086892067\\
546	0.00234265994943839\\
547	0.00229749658658826\\
548	0.00225144245025375\\
549	0.0022044690067269\\
550	0.00215651952593421\\
551	0.00210750156462815\\
552	0.00205873951659417\\
553	0.0020109931608278\\
554	0.00196454992977493\\
555	0.00191823942837985\\
556	0.00187148664771676\\
557	0.00182422356319178\\
558	0.00177642326600128\\
559	0.0017281096045976\\
560	0.00167931427601956\\
561	0.00163007921572574\\
562	0.00158045429977808\\
563	0.00153047646411673\\
564	0.00148151315013013\\
565	0.00143416985390059\\
566	0.00138705937387004\\
567	0.00133975888995393\\
568	0.00129226939301679\\
569	0.0012446284113432\\
570	0.00119687824718637\\
571	0.0011490625211639\\
572	0.00110122520101772\\
573	0.0010534821435598\\
574	0.00100637667104439\\
575	0.000959368504257833\\
576	0.000912345403221071\\
577	0.000865352810900999\\
578	0.000818439527258546\\
579	0.000771656726968869\\
580	0.00072505761760655\\
581	0.000678696972083882\\
582	0.000632630483751538\\
583	0.000586913897255451\\
584	0.000541601859703357\\
585	0.000496746426609176\\
586	0.000452395149030046\\
587	0.000408588670365811\\
588	0.000365357795666471\\
589	0.000322720123375259\\
590	0.000280676711614415\\
591	0.000239312817487403\\
592	0.000198788644961841\\
593	0.000159293651685586\\
594	0.000121079888727804\\
595	8.45520570083908e-05\\
596	5.05092148680372e-05\\
597	2.07908715710836e-05\\
598	0\\
599	0\\
600	0\\
};
\addplot [color=mycolor8,solid,forget plot]
  table[row sep=crcr]{%
1	0.00647624004771694\\
2	0.00647622387146877\\
3	0.00647620742204131\\
4	0.00647619069483262\\
5	0.00647617368516355\\
6	0.00647615638827646\\
7	0.0064761387993338\\
8	0.0064761209134169\\
9	0.00647610272552456\\
10	0.00647608423057166\\
11	0.00647606542338788\\
12	0.00647604629871608\\
13	0.00647602685121104\\
14	0.00647600707543784\\
15	0.0064759869658705\\
16	0.00647596651689038\\
17	0.00647594572278466\\
18	0.00647592457774483\\
19	0.00647590307586496\\
20	0.0064758812111402\\
21	0.00647585897746516\\
22	0.00647583636863205\\
23	0.0064758133783292\\
24	0.00647579000013921\\
25	0.00647576622753725\\
26	0.00647574205388915\\
27	0.00647571747244984\\
28	0.00647569247636121\\
29	0.00647566705865046\\
30	0.00647564121222805\\
31	0.00647561492988589\\
32	0.00647558820429525\\
33	0.00647556102800488\\
34	0.00647553339343885\\
35	0.0064755052928946\\
36	0.00647547671854082\\
37	0.00647544766241516\\
38	0.00647541811642238\\
39	0.00647538807233187\\
40	0.00647535752177552\\
41	0.00647532645624541\\
42	0.00647529486709158\\
43	0.00647526274551955\\
44	0.0064752300825881\\
45	0.00647519686920668\\
46	0.00647516309613301\\
47	0.00647512875397063\\
48	0.00647509383316633\\
49	0.00647505832400752\\
50	0.0064750222166196\\
51	0.00647498550096336\\
52	0.00647494816683225\\
53	0.00647491020384958\\
54	0.00647487160146572\\
55	0.00647483234895532\\
56	0.00647479243541441\\
57	0.00647475184975742\\
58	0.00647471058071415\\
59	0.0064746686168269\\
60	0.00647462594644722\\
61	0.00647458255773297\\
62	0.00647453843864493\\
63	0.00647449357694374\\
64	0.0064744479601865\\
65	0.00647440157572358\\
66	0.00647435441069509\\
67	0.00647430645202757\\
68	0.00647425768643044\\
69	0.00647420810039235\\
70	0.00647415768017784\\
71	0.00647410641182344\\
72	0.00647405428113408\\
73	0.00647400127367924\\
74	0.00647394737478918\\
75	0.00647389256955111\\
76	0.00647383684280508\\
77	0.00647378017914003\\
78	0.00647372256288978\\
79	0.00647366397812885\\
80	0.00647360440866827\\
81	0.00647354383805118\\
82	0.00647348224954876\\
83	0.00647341962615564\\
84	0.00647335595058546\\
85	0.00647329120526639\\
86	0.00647322537233646\\
87	0.00647315843363897\\
88	0.00647309037071768\\
89	0.006473021164812\\
90	0.00647295079685217\\
91	0.0064728792474542\\
92	0.00647280649691494\\
93	0.00647273252520689\\
94	0.00647265731197301\\
95	0.00647258083652153\\
96	0.00647250307782057\\
97	0.00647242401449275\\
98	0.00647234362480961\\
99	0.00647226188668611\\
100	0.00647217877767497\\
101	0.0064720942749609\\
102	0.00647200835535479\\
103	0.00647192099528782\\
104	0.0064718321708054\\
105	0.00647174185756121\\
106	0.006471650030811\\
107	0.00647155666540625\\
108	0.00647146173578796\\
109	0.0064713652159802\\
110	0.00647126707958353\\
111	0.00647116729976853\\
112	0.00647106584926904\\
113	0.0064709627003754\\
114	0.0064708578249275\\
115	0.00647075119430803\\
116	0.00647064277943524\\
117	0.00647053255075598\\
118	0.00647042047823828\\
119	0.00647030653136422\\
120	0.00647019067912238\\
121	0.00647007289000054\\
122	0.00646995313197789\\
123	0.00646983137251749\\
124	0.00646970757855844\\
125	0.0064695817165082\\
126	0.00646945375223438\\
127	0.00646932365105696\\
128	0.00646919137774018\\
129	0.00646905689648426\\
130	0.00646892017091722\\
131	0.00646878116408651\\
132	0.00646863983845066\\
133	0.00646849615587075\\
134	0.00646835007760198\\
135	0.00646820156428492\\
136	0.00646805057593706\\
137	0.0064678970719439\\
138	0.00646774101105035\\
139	0.00646758235135188\\
140	0.00646742105028574\\
141	0.00646725706462213\\
142	0.00646709035045533\\
143	0.0064669208631949\\
144	0.00646674855755685\\
145	0.00646657338755481\\
146	0.00646639530649136\\
147	0.00646621426694923\\
148	0.00646603022078284\\
149	0.00646584311910966\\
150	0.00646565291230193\\
151	0.00646545954997842\\
152	0.00646526298099641\\
153	0.00646506315344396\\
154	0.00646486001463229\\
155	0.00646465351108868\\
156	0.0064644435885495\\
157	0.00646423019195383\\
158	0.00646401326543751\\
159	0.00646379275232764\\
160	0.00646356859513789\\
161	0.00646334073556431\\
162	0.00646310911448212\\
163	0.00646287367194317\\
164	0.00646263434717474\\
165	0.00646239107857922\\
166	0.00646214380373511\\
167	0.00646189245939954\\
168	0.00646163698151231\\
169	0.00646137730520183\\
170	0.00646111336479263\\
171	0.00646084509381557\\
172	0.00646057242501987\\
173	0.00646029529038804\\
174	0.00646001362115334\\
175	0.00645972734782021\\
176	0.00645943640018777\\
177	0.00645914070737652\\
178	0.00645884019785802\\
179	0.00645853479948794\\
180	0.00645822443954161\\
181	0.00645790904475235\\
182	0.00645758854135083\\
183	0.00645726285510531\\
184	0.00645693191136017\\
185	0.00645659563507051\\
186	0.006456253950829\\
187	0.00645590678287998\\
188	0.00645555405511283\\
189	0.00645519569102329\\
190	0.00645483161362372\\
191	0.00645446174526723\\
192	0.00645408600731315\\
193	0.00645370431946377\\
194	0.0064533165983436\\
195	0.0064529227541831\\
196	0.00645252268249495\\
197	0.00645211624207562\\
198	0.00645170319487064\\
199	0.00645128303787878\\
200	0.00645085563272166\\
201	0.0064504208551466\\
202	0.00644997857879551\\
203	0.00644952867516996\\
204	0.00644907101359553\\
205	0.00644860546118575\\
206	0.00644813188280542\\
207	0.00644765014103328\\
208	0.00644716009612396\\
209	0.00644666160596958\\
210	0.00644615452606039\\
211	0.00644563870944506\\
212	0.00644511400669005\\
213	0.0064445802658385\\
214	0.00644403733236834\\
215	0.00644348504914973\\
216	0.00644292325640165\\
217	0.00644235179164807\\
218	0.00644177048967304\\
219	0.00644117918247519\\
220	0.0064405776992215\\
221	0.00643996586620005\\
222	0.00643934350677232\\
223	0.0064387104413243\\
224	0.00643806648721708\\
225	0.00643741145873626\\
226	0.00643674516704073\\
227	0.00643606742011057\\
228	0.00643537802269375\\
229	0.00643467677625234\\
230	0.00643396347890729\\
231	0.00643323792538263\\
232	0.00643249990694855\\
233	0.00643174921136343\\
234	0.00643098562281489\\
235	0.00643020892185986\\
236	0.00642941888536348\\
237	0.00642861528643711\\
238	0.00642779789437499\\
239	0.00642696647459007\\
240	0.00642612078854862\\
241	0.00642526059370366\\
242	0.00642438564342734\\
243	0.00642349568694218\\
244	0.00642259046925103\\
245	0.00642166973106612\\
246	0.00642073320873672\\
247	0.00641978063417571\\
248	0.00641881173478507\\
249	0.00641782623338007\\
250	0.00641682384811246\\
251	0.00641580429239246\\
252	0.00641476727480949\\
253	0.00641371249905202\\
254	0.00641263966382602\\
255	0.00641154846277254\\
256	0.00641043858438417\\
257	0.00640930971192017\\
258	0.00640816152332094\\
259	0.00640699369112094\\
260	0.00640580588236107\\
261	0.00640459775849952\\
262	0.00640336897532169\\
263	0.00640211918284917\\
264	0.00640084802524742\\
265	0.00639955514073233\\
266	0.00639824016147552\\
267	0.00639690271350842\\
268	0.00639554241662507\\
269	0.00639415888428342\\
270	0.00639275172350507\\
271	0.00639132053477362\\
272	0.00638986491193117\\
273	0.00638838444207315\\
274	0.00638687870544136\\
275	0.00638534727531509\\
276	0.00638378971790053\\
277	0.00638220559221794\\
278	0.00638059444998737\\
279	0.00637895583551238\\
280	0.00637728928556208\\
281	0.00637559432925129\\
282	0.00637387048791925\\
283	0.00637211727500619\\
284	0.00637033419592853\\
285	0.00636852074795187\\
286	0.00636667642006241\\
287	0.00636480069283646\\
288	0.00636289303830807\\
289	0.0063609529198348\\
290	0.0063589797919614\\
291	0.00635697310028176\\
292	0.00635493228129887\\
293	0.00635285676228271\\
294	0.00635074596112604\\
295	0.00634859928619845\\
296	0.00634641613619815\\
297	0.00634419590000159\\
298	0.00634193795651116\\
299	0.00633964167450065\\
300	0.00633730641245856\\
301	0.00633493151842919\\
302	0.00633251632985156\\
303	0.00633006017339608\\
304	0.00632756236479882\\
305	0.00632502220869358\\
306	0.00632243899844158\\
307	0.00631981201595874\\
308	0.00631714053154054\\
309	0.00631442380368444\\
310	0.00631166107890984\\
311	0.00630885159157541\\
312	0.00630599456369383\\
313	0.00630308920474416\\
314	0.00630013471148108\\
315	0.00629713026774185\\
316	0.00629407504425036\\
317	0.00629096819841817\\
318	0.00628780887414304\\
319	0.00628459620160424\\
320	0.006281329297055\\
321	0.00627800726261194\\
322	0.00627462918604145\\
323	0.00627119414054285\\
324	0.00626770118452838\\
325	0.00626414936139994\\
326	0.00626053769932257\\
327	0.00625686521099455\\
328	0.00625313089341405\\
329	0.00624933372764253\\
330	0.00624547267856455\\
331	0.00624154669464419\\
332	0.00623755470767811\\
333	0.00623349563254495\\
334	0.00622936836695184\\
335	0.00622517179117711\\
336	0.00622090476781047\\
337	0.0062165661414898\\
338	0.00621215473863561\\
339	0.00620766936718299\\
340	0.00620310881631192\\
341	0.00619847185617595\\
342	0.00619375723763056\\
343	0.00618896369196163\\
344	0.00618408993061516\\
345	0.00617913464492978\\
346	0.00617409650587355\\
347	0.00616897416378686\\
348	0.00616376624813409\\
349	0.00615847136726692\\
350	0.00615308810820268\\
351	0.00614761503642237\\
352	0.00614205069569307\\
353	0.00613639360792149\\
354	0.0061306422730459\\
355	0.0061247951689757\\
356	0.00611885075158952\\
357	0.00611280745480499\\
358	0.00610666369073642\\
359	0.00610041784995903\\
360	0.00609406830190338\\
361	0.00608761339540712\\
362	0.00608105145945854\\
363	0.00607438080417158\\
364	0.00606759972204199\\
365	0.00606070648954351\\
366	0.00605369936913638\\
367	0.00604657661177525\\
368	0.00603933646002237\\
369	0.00603197715189596\\
370	0.00602449692561097\\
371	0.00601689402540503\\
372	0.00600916670868642\\
373	0.0060013132547934\\
374	0.00599333197572258\\
375	0.00598522122926663\\
376	0.00597697943510633\\
377	0.00596860509453244\\
378	0.0059600968146372\\
379	0.00595145333802245\\
380	0.0059426735793323\\
381	0.00593375667024947\\
382	0.00592470201501394\\
383	0.00591550935905417\\
384	0.00590617887399451\\
385	0.00589671126314147\\
386	0.00588710789256086\\
387	0.0058773709539363\\
388	0.00586750366609288\\
389	0.00585751052076736\\
390	0.00584739756964675\\
391	0.00583717286767827\\
392	0.00582684699909627\\
393	0.00581643373854259\\
394	0.00580595090300746\\
395	0.00579542146859583\\
396	0.00578487508013719\\
397	0.00577434973701437\\
398	0.00576389390029359\\
399	0.00575356936161976\\
400	0.00574345493461472\\
401	0.00573365119175744\\
402	0.00572428646084609\\
403	0.00571552405381711\\
404	0.00570756948526384\\
405	0.00570020151858036\\
406	0.00569271599057066\\
407	0.00568511155554668\\
408	0.00567738689906436\\
409	0.00566954074321656\\
410	0.00566157185251424\\
411	0.00565347904043678\\
412	0.00564526117674768\\
413	0.00563691719568569\\
414	0.00562844610515465\\
415	0.00561984699703459\\
416	0.00561111905869675\\
417	0.00560226158565433\\
418	0.00559327399484952\\
419	0.00558415583698435\\
420	0.00557490681356992\\
421	0.00556552679656875\\
422	0.00555601585045792\\
423	0.00554637425700223\\
424	0.0055366025430567\\
425	0.00552670151176883\\
426	0.00551667227767325\\
427	0.0055065163065078\\
428	0.00549623546132956\\
429	0.00548583205295491\\
430	0.0054753088941555\\
431	0.00546466935862506\\
432	0.0054539174425402\\
433	0.00544305782176041\\
434	0.00543209591680835\\
435	0.00542103797090364\\
436	0.0054098911331517\\
437	0.00539866354596811\\
438	0.00538736443515098\\
439	0.00537600420007305\\
440	0.00536459450004577\\
441	0.0053531483307078\\
442	0.00534168008187833\\
443	0.00533020556475153\\
444	0.00531874199109124\\
445	0.00530730788008513\\
446	0.00529592285891626\\
447	0.00528460730995834\\
448	0.00527338179949224\\
449	0.0052622661981414\\
450	0.00525127836917647\\
451	0.0052404322533988\\
452	0.0052297351121516\\
453	0.00521918359548272\\
454	0.00520875817253184\\
455	0.00519841518453535\\
456	0.00518807572061038\\
457	0.00517761143626688\\
458	0.0051670056819433\\
459	0.00515625832843616\\
460	0.00514536935435214\\
461	0.00513433882988236\\
462	0.00512316690726056\\
463	0.00511185380605984\\
464	0.00510039979199753\\
465	0.00508880514764454\\
466	0.00507707013315591\\
467	0.00506519493487367\\
468	0.00505317959946346\\
469	0.00504102395116322\\
470	0.00502872748930082\\
471	0.00501628925885373\\
472	0.00500370765351237\\
473	0.0049909793991094\\
474	0.00497809968549204\\
475	0.00496506227722912\\
476	0.00495185923263095\\
477	0.00493848060039775\\
478	0.00492491410648339\\
479	0.00491114485115891\\
480	0.00489715505779419\\
481	0.00488292393090441\\
482	0.00486842771334284\\
483	0.00485364008282515\\
484	0.00483853311333963\\
485	0.00482307919412716\\
486	0.00480725458589019\\
487	0.00479104299851726\\
488	0.00477444558520379\\
489	0.00475747358007265\\
490	0.00474011146481785\\
491	0.00472234262990966\\
492	0.00470414929833351\\
493	0.00468551243506386\\
494	0.00466641161320032\\
495	0.00464682485774645\\
496	0.00462672867982591\\
497	0.00460609842400275\\
498	0.00458490755158626\\
499	0.00456312446528458\\
500	0.00454069397053237\\
501	0.0045175760745551\\
502	0.00449372477696406\\
503	0.00446905426462991\\
504	0.00444351900407773\\
505	0.00441707207804545\\
506	0.00438966452483288\\
507	0.00436124514501922\\
508	0.00433175993730267\\
509	0.00430115198250383\\
510	0.00426936140119424\\
511	0.0042363253111851\\
512	0.00420197783560677\\
513	0.00416625008663321\\
514	0.00412906987623692\\
515	0.00409036252097511\\
516	0.00405005428318096\\
517	0.00400807037188265\\
518	0.00396433572973611\\
519	0.00391877605958476\\
520	0.00387131917511654\\
521	0.00382189674995734\\
522	0.00377044651661206\\
523	0.00371691446152693\\
524	0.00366125824878971\\
525	0.00360345349321673\\
526	0.00354350693164753\\
527	0.0034814518998612\\
528	0.00341735621285175\\
529	0.00335133319318523\\
530	0.00328355130589948\\
531	0.00321424345000754\\
532	0.00314370305522421\\
533	0.00307227580608966\\
534	0.00300050952196754\\
535	0.00292929211302619\\
536	0.00286178178697941\\
537	0.00279960223019647\\
538	0.00274303581167043\\
539	0.00269196434323486\\
540	0.00264490546750042\\
541	0.00259895181245598\\
542	0.00255370632164094\\
543	0.0025086849952371\\
544	0.00246359209360616\\
545	0.0024183959307389\\
546	0.00237300335466916\\
547	0.00232734162646493\\
548	0.00228137378200997\\
549	0.00223471812268854\\
550	0.00218726473464073\\
551	0.00213896836534945\\
552	0.00208977713704287\\
553	0.00203960532090274\\
554	0.00198836693274954\\
555	0.00193738925710515\\
556	0.00188742405447135\\
557	0.00183875487424147\\
558	0.00179049315303823\\
559	0.00174185184830918\\
560	0.00169278433015013\\
561	0.00164324280767594\\
562	0.0015932603719237\\
563	0.00154287867702145\\
564	0.00149215207571757\\
565	0.00144111678732867\\
566	0.00139123663602677\\
567	0.00134300099050608\\
568	0.00129517846879981\\
569	0.00124726049348067\\
570	0.00119922945498401\\
571	0.00115112545879614\\
572	0.00110299534370064\\
573	0.00105488697841206\\
574	0.00100684692524753\\
575	0.000959381521014027\\
576	0.000912346458482514\\
577	0.000865353251381383\\
578	0.000818439752416602\\
579	0.000771656841517049\\
580	0.000725057672851425\\
581	0.000678696996801308\\
582	0.000632630493807236\\
583	0.000586913900878002\\
584	0.000541601860815896\\
585	0.000496746426883291\\
586	0.000452395149078408\\
587	0.000408588670370421\\
588	0.000365357795666472\\
589	0.00032272012337526\\
590	0.000280676711614416\\
591	0.000239312817487404\\
592	0.000198788644961842\\
593	0.000159293651685588\\
594	0.000121079888727806\\
595	8.45520570083917e-05\\
596	5.05092148680373e-05\\
597	2.07908715710836e-05\\
598	0\\
599	0\\
600	0\\
};
\addplot [color=blue!25!mycolor7,solid,forget plot]
  table[row sep=crcr]{%
1	0.00682152599929207\\
2	0.00682151808048622\\
3	0.00682151002797638\\
4	0.00682150183951254\\
5	0.006821493512807\\
6	0.00682148504553382\\
7	0.00682147643532816\\
8	0.00682146767978563\\
9	0.00682145877646161\\
10	0.00682144972287067\\
11	0.00682144051648577\\
12	0.00682143115473767\\
13	0.00682142163501422\\
14	0.00682141195465958\\
15	0.00682140211097362\\
16	0.00682139210121112\\
17	0.00682138192258097\\
18	0.00682137157224552\\
19	0.00682136104731981\\
20	0.00682135034487067\\
21	0.00682133946191601\\
22	0.00682132839542409\\
23	0.00682131714231252\\
24	0.00682130569944763\\
25	0.00682129406364346\\
26	0.00682128223166108\\
27	0.00682127020020752\\
28	0.00682125796593495\\
29	0.00682124552543993\\
30	0.00682123287526229\\
31	0.00682122001188433\\
32	0.00682120693172984\\
33	0.0068211936311631\\
34	0.00682118010648805\\
35	0.00682116635394711\\
36	0.00682115236972032\\
37	0.00682113814992434\\
38	0.00682112369061126\\
39	0.00682110898776762\\
40	0.00682109403731346\\
41	0.00682107883510114\\
42	0.00682106337691419\\
43	0.00682104765846632\\
44	0.00682103167540016\\
45	0.00682101542328615\\
46	0.00682099889762152\\
47	0.00682098209382884\\
48	0.00682096500725499\\
49	0.00682094763316988\\
50	0.00682092996676526\\
51	0.00682091200315343\\
52	0.00682089373736587\\
53	0.00682087516435213\\
54	0.00682085627897833\\
55	0.00682083707602595\\
56	0.00682081755019037\\
57	0.00682079769607956\\
58	0.00682077750821269\\
59	0.00682075698101862\\
60	0.00682073610883456\\
61	0.00682071488590459\\
62	0.0068206933063781\\
63	0.00682067136430838\\
64	0.00682064905365105\\
65	0.00682062636826243\\
66	0.00682060330189814\\
67	0.00682057984821136\\
68	0.00682055600075125\\
69	0.00682053175296142\\
70	0.00682050709817799\\
71	0.00682048202962821\\
72	0.00682045654042859\\
73	0.00682043062358315\\
74	0.00682040427198172\\
75	0.00682037747839815\\
76	0.00682035023548839\\
77	0.00682032253578886\\
78	0.00682029437171437\\
79	0.00682026573555639\\
80	0.006820236619481\\
81	0.00682020701552719\\
82	0.00682017691560461\\
83	0.00682014631149174\\
84	0.0068201151948338\\
85	0.00682008355714087\\
86	0.00682005138978555\\
87	0.00682001868400113\\
88	0.00681998543087931\\
89	0.00681995162136805\\
90	0.00681991724626948\\
91	0.00681988229623762\\
92	0.00681984676177617\\
93	0.00681981063323628\\
94	0.00681977390081422\\
95	0.00681973655454912\\
96	0.0068196985843205\\
97	0.00681965997984607\\
98	0.00681962073067918\\
99	0.00681958082620655\\
100	0.0068195402556457\\
101	0.00681949900804251\\
102	0.00681945707226868\\
103	0.00681941443701926\\
104	0.00681937109081009\\
105	0.0068193270219751\\
106	0.00681928221866379\\
107	0.00681923666883861\\
108	0.00681919036027223\\
109	0.0068191432805449\\
110	0.00681909541704164\\
111	0.00681904675694962\\
112	0.00681899728725532\\
113	0.00681894699474166\\
114	0.00681889586598545\\
115	0.00681884388735419\\
116	0.00681879104500348\\
117	0.00681873732487395\\
118	0.00681868271268856\\
119	0.00681862719394947\\
120	0.00681857075393517\\
121	0.00681851337769744\\
122	0.00681845505005851\\
123	0.0068183957556079\\
124	0.00681833547869944\\
125	0.0068182742034482\\
126	0.00681821191372749\\
127	0.00681814859316567\\
128	0.00681808422514313\\
129	0.00681801879278911\\
130	0.00681795227897859\\
131	0.00681788466632918\\
132	0.00681781593719792\\
133	0.00681774607367807\\
134	0.00681767505759591\\
135	0.00681760287050757\\
136	0.00681752949369577\\
137	0.0068174549081666\\
138	0.00681737909464619\\
139	0.00681730203357736\\
140	0.00681722370511652\\
141	0.00681714408913013\\
142	0.00681706316519127\\
143	0.00681698091257642\\
144	0.00681689731026185\\
145	0.00681681233692014\\
146	0.00681672597091659\\
147	0.00681663819030566\\
148	0.00681654897282697\\
149	0.00681645829590186\\
150	0.00681636613662908\\
151	0.00681627247178097\\
152	0.00681617727779895\\
153	0.0068160805307893\\
154	0.00681598220651826\\
155	0.00681588228040731\\
156	0.00681578072752773\\
157	0.00681567752259502\\
158	0.00681557263996284\\
159	0.00681546605361657\\
160	0.00681535773716594\\
161	0.00681524766383737\\
162	0.00681513580646522\\
163	0.00681502213748237\\
164	0.00681490662890945\\
165	0.00681478925234313\\
166	0.00681466997894275\\
167	0.00681454877941545\\
168	0.00681442562399916\\
169	0.0068143004824438\\
170	0.00681417332398954\\
171	0.00681404411734251\\
172	0.00681391283064702\\
173	0.00681377943145406\\
174	0.00681364388668554\\
175	0.00681350616259372\\
176	0.00681336622471505\\
177	0.00681322403781779\\
178	0.00681307956584296\\
179	0.00681293277183706\\
180	0.00681278361787651\\
181	0.00681263206498246\\
182	0.00681247807302526\\
183	0.0068123216006177\\
184	0.00681216260499597\\
185	0.00681200104188856\\
186	0.00681183686537187\\
187	0.00681167002771319\\
188	0.00681150047920165\\
189	0.00681132816796832\\
190	0.00681115303979744\\
191	0.00681097503793067\\
192	0.00681079410286415\\
193	0.00681061017213128\\
194	0.0068104231800471\\
195	0.00681023305736157\\
196	0.00681003973076177\\
197	0.0068098431224084\\
198	0.00680964315129954\\
199	0.00680943974594813\\
200	0.00680923284747863\\
201	0.00680902239660873\\
202	0.00680880833307165\\
203	0.00680859059560025\\
204	0.00680836912191124\\
205	0.00680814384868866\\
206	0.00680791471156739\\
207	0.00680768164511646\\
208	0.00680744458282179\\
209	0.00680720345706904\\
210	0.00680695819912599\\
211	0.00680670873912447\\
212	0.00680645500604247\\
213	0.0068061969276855\\
214	0.00680593443066793\\
215	0.0068056674403939\\
216	0.0068053958810381\\
217	0.00680511967552597\\
218	0.00680483874551379\\
219	0.00680455301136853\\
220	0.00680426239214711\\
221	0.00680396680557559\\
222	0.00680366616802786\\
223	0.0068033603945041\\
224	0.00680304939860875\\
225	0.00680273309252847\\
226	0.00680241138700929\\
227	0.00680208419133379\\
228	0.00680175141329776\\
229	0.00680141295918634\\
230	0.00680106873375031\\
231	0.00680071864018141\\
232	0.0068003625800876\\
233	0.00680000045346802\\
234	0.0067996321586873\\
235	0.00679925759244981\\
236	0.00679887664977322\\
237	0.00679848922396195\\
238	0.00679809520658008\\
239	0.006797694487424\\
240	0.00679728695449443\\
241	0.00679687249396846\\
242	0.00679645099017089\\
243	0.00679602232554523\\
244	0.00679558638062453\\
245	0.00679514303400161\\
246	0.006794692162299\\
247	0.00679423364013861\\
248	0.00679376734011086\\
249	0.00679329313274358\\
250	0.00679281088647048\\
251	0.00679232046759924\\
252	0.00679182174027928\\
253	0.00679131456646923\\
254	0.00679079880590386\\
255	0.00679027431606076\\
256	0.00678974095212658\\
257	0.00678919856696295\\
258	0.00678864701107205\\
259	0.00678808613256156\\
260	0.00678751577710937\\
261	0.00678693578792792\\
262	0.00678634600572787\\
263	0.00678574626868148\\
264	0.0067851364123854\\
265	0.00678451626982306\\
266	0.00678388567132639\\
267	0.00678324444453717\\
268	0.00678259241436765\\
269	0.00678192940296065\\
270	0.00678125522964914\\
271	0.00678056971091495\\
272	0.00677987266034734\\
273	0.00677916388860037\\
274	0.00677844320335\\
275	0.00677771040925065\\
276	0.00677696530789062\\
277	0.00677620769774744\\
278	0.00677543737414224\\
279	0.00677465412919359\\
280	0.00677385775177052\\
281	0.00677304802744507\\
282	0.00677222473844398\\
283	0.00677138766359968\\
284	0.0067705365783005\\
285	0.00676967125444025\\
286	0.00676879146036689\\
287	0.00676789696083024\\
288	0.00676698751692908\\
289	0.00676606288605736\\
290	0.00676512282184922\\
291	0.00676416707412344\\
292	0.00676319538882644\\
293	0.00676220750797481\\
294	0.00676120316959627\\
295	0.00676018210766969\\
296	0.00675914405206401\\
297	0.00675808872847585\\
298	0.00675701585836574\\
299	0.00675592515889312\\
300	0.00675481634284983\\
301	0.00675368911859217\\
302	0.00675254318997126\\
303	0.00675137825626169\\
304	0.00675019401208859\\
305	0.00674899014735249\\
306	0.00674776634715239\\
307	0.00674652229170682\\
308	0.00674525765627203\\
309	0.00674397211105854\\
310	0.00674266532114452\\
311	0.00674133694638666\\
312	0.00673998664132796\\
313	0.00673861405510251\\
314	0.00673721883133672\\
315	0.00673580060804694\\
316	0.00673435901753325\\
317	0.00673289368626896\\
318	0.00673140423478562\\
319	0.0067298902775532\\
320	0.0067283514228549\\
321	0.00672678727265644\\
322	0.006725197422469\\
323	0.00672358146120589\\
324	0.00672193897103177\\
325	0.00672026952720406\\
326	0.00671857269790636\\
327	0.00671684804407221\\
328	0.00671509511919944\\
329	0.00671331346915329\\
330	0.00671150263195827\\
331	0.00670966213757684\\
332	0.00670779150767453\\
333	0.00670589025536974\\
334	0.00670395788496691\\
335	0.00670199389167166\\
336	0.00669999776128625\\
337	0.00669796896988296\\
338	0.00669590698345397\\
339	0.00669381125753483\\
340	0.00669168123679933\\
341	0.00668951635462255\\
342	0.00668731603260905\\
343	0.00668507968008221\\
344	0.0066828066935313\\
345	0.00668049645601082\\
346	0.0066781483364877\\
347	0.00667576168913012\\
348	0.00667333585253143\\
349	0.00667087014886187\\
350	0.00666836388293925\\
351	0.00666581634120933\\
352	0.00666322679062493\\
353	0.00666059447741077\\
354	0.0066579186257001\\
355	0.00665519843602668\\
356	0.00665243308365263\\
357	0.00664962171671082\\
358	0.00664676345413608\\
359	0.00664385738335635\\
360	0.00664090255770907\\
361	0.00663789799354379\\
362	0.006634842666964\\
363	0.00663173551015453\\
364	0.00662857540723024\\
365	0.00662536118953125\\
366	0.00662209163027598\\
367	0.00661876543846735\\
368	0.00661538125192753\\
369	0.00661193762931299\\
370	0.00660843304093303\\
371	0.00660486585816018\\
372	0.00660123434117727\\
373	0.00659753662475583\\
374	0.00659377070169413\\
375	0.00658993440346675\\
376	0.00658602537753923\\
377	0.0065820410606818\\
378	0.00657797864746696\\
379	0.00657383505295076\\
380	0.00656960686830603\\
381	0.00656529030789022\\
382	0.00656088114587788\\
383	0.00655637464017199\\
384	0.006551765440869\\
385	0.00654704748027803\\
386	0.00654221384207036\\
387	0.00653725661091762\\
388	0.00653216671896052\\
389	0.0065269338587564\\
390	0.00652154671652445\\
391	0.00651599095147768\\
392	0.00651024932391503\\
393	0.00650430118488542\\
394	0.00649812145499418\\
395	0.00649167893127105\\
396	0.00648493299469873\\
397	0.00647783680310319\\
398	0.00647033688183818\\
399	0.00646236779727463\\
400	0.00645384908975378\\
401	0.00644468139013384\\
402	0.00643474159157995\\
403	0.00642387719790623\\
404	0.00641190104780991\\
405	0.00639903657200411\\
406	0.00638596078425052\\
407	0.00637267027765513\\
408	0.00635916159825583\\
409	0.00634543124535949\\
410	0.00633147567191987\\
411	0.0063172912849489\\
412	0.00630287444596532\\
413	0.00628822147152613\\
414	0.00627332863398959\\
415	0.0062581921629295\\
416	0.00624280824829613\\
417	0.00622717304810649\\
418	0.00621128270763228\\
419	0.0061951334073186\\
420	0.00617872130624063\\
421	0.00616204253850911\\
422	0.00614509322036607\\
423	0.00612786945910534\\
424	0.0061103673640725\\
425	0.00609258305964295\\
426	0.00607451269871913\\
427	0.00605615247067725\\
428	0.00603749858282038\\
429	0.00601854731459214\\
430	0.00599929511417933\\
431	0.00597973868856861\\
432	0.00595987516664406\\
433	0.00593970243760798\\
434	0.00591921918218595\\
435	0.00589842481180852\\
436	0.0058773196493913\\
437	0.00585590515351625\\
438	0.00583418419697106\\
439	0.00581216141364589\\
440	0.00578984363321967\\
441	0.00576724043148619\\
442	0.00574436482077435\\
443	0.00572123411658393\\
444	0.00569787103223775\\
445	0.00567430506515952\\
446	0.00565057425709745\\
447	0.00562672743507575\\
448	0.00560282707190043\\
449	0.00557895294706662\\
450	0.00555520684395962\\
451	0.00553171859080939\\
452	0.00550865384334736\\
453	0.00548622411073188\\
454	0.00546469960050526\\
455	0.00544442531408781\\
456	0.0054258396714864\\
457	0.00540948983742446\\
458	0.00539323260504457\\
459	0.00537676311249133\\
460	0.00536008469878835\\
461	0.00534320198965536\\
462	0.00532612059192507\\
463	0.00530884720770051\\
464	0.00529138975884361\\
465	0.00527375752160495\\
466	0.00525596127074597\\
467	0.00523801343165346\\
468	0.00521992823775862\\
469	0.00520172188888132\\
470	0.00518341270376497\\
471	0.00516502125709558\\
472	0.0051465704860549\\
473	0.00512808576765448\\
474	0.0051095948999598\\
475	0.00509112794187953\\
476	0.00507271685943956\\
477	0.00505439488960137\\
478	0.00503619549586247\\
479	0.00501815076360548\\
480	0.0050002889997187\\
481	0.00498263121570112\\
482	0.00496518604990143\\
483	0.0049479424870087\\
484	0.00493085946334262\\
485	0.00491385109885935\\
486	0.00489676634709877\\
487	0.00487941046645583\\
488	0.00486177900139084\\
489	0.00484386633386189\\
490	0.00482566519217253\\
491	0.00480716726418363\\
492	0.00478836297873783\\
493	0.00476924124746257\\
494	0.00474978916359166\\
495	0.00472999165560429\\
496	0.00470983108666331\\
497	0.00468928677138484\\
498	0.0046683344028814\\
499	0.00464694545929378\\
500	0.00462508734092348\\
501	0.00460272350259559\\
502	0.00457982238022318\\
503	0.00455638069580096\\
504	0.00453233403861286\\
505	0.00450760479277492\\
506	0.00448213612080047\\
507	0.00445587381878527\\
508	0.00442877360691682\\
509	0.00440078970698855\\
510	0.00437187371198864\\
511	0.00434197524437358\\
512	0.00431104092335962\\
513	0.00427901432034292\\
514	0.00424583630966117\\
515	0.0042114290035927\\
516	0.00417567760871252\\
517	0.00413850706469992\\
518	0.00409983938562309\\
519	0.00405959407839367\\
520	0.00401768872169933\\
521	0.0039740397359373\\
522	0.00392856330490904\\
523	0.00388117643313738\\
524	0.00383179797078545\\
525	0.00378034940225419\\
526	0.0037267559301864\\
527	0.0036709496971085\\
528	0.00361287876217814\\
529	0.00355251646179829\\
530	0.00348985500560026\\
531	0.00342491096846352\\
532	0.00335773241447154\\
533	0.00328840965250917\\
534	0.00321708462727485\\
535	0.00314395991947484\\
536	0.00306929731610841\\
537	0.00299340958820004\\
538	0.0029167674290013\\
539	0.00284013624520599\\
540	0.00276555084315231\\
541	0.0026960778195408\\
542	0.00263212286836089\\
543	0.00257378503688891\\
544	0.00252085379215805\\
545	0.00247009708931051\\
546	0.00242037269702207\\
547	0.00237125755908936\\
548	0.00232224243204798\\
549	0.00227316453779023\\
550	0.00222398371594873\\
551	0.00217460623235058\\
552	0.00212495896211347\\
553	0.00207497488170052\\
554	0.00202421243490735\\
555	0.00197262615657274\\
556	0.0019201375163468\\
557	0.00186664188289284\\
558	0.00181318150960817\\
559	0.00176071773665684\\
560	0.00170951692853472\\
561	0.00165928064744219\\
562	0.00160875603182242\\
563	0.00155792315883206\\
564	0.00150671918854937\\
565	0.00145517384907741\\
566	0.00140334093225183\\
567	0.00135125632661799\\
568	0.00130024976398429\\
569	0.00125090910566476\\
570	0.00120241934549694\\
571	0.00115396094926996\\
572	0.00110548107662132\\
573	0.00105702283547423\\
574	0.00100863570447616\\
575	0.000960371942245695\\
576	0.000912495014597589\\
577	0.000865361483000015\\
578	0.000818442575994003\\
579	0.000771658262692563\\
580	0.000725058416501592\\
581	0.000678697369762489\\
582	0.000632630668187501\\
583	0.000586913975324799\\
584	0.000541601889067355\\
585	0.000496746436058626\\
586	0.000452395151479575\\
587	0.000408588670822602\\
588	0.000365357795712724\\
589	0.000322720123375259\\
590	0.000280676711614414\\
591	0.000239312817487402\\
592	0.000198788644961838\\
593	0.000159293651685583\\
594	0.000121079888727803\\
595	8.45520570083905e-05\\
596	5.05092148680371e-05\\
597	2.07908715710836e-05\\
598	0\\
599	0\\
600	0\\
};
\addplot [color=mycolor9,solid,forget plot]
  table[row sep=crcr]{%
1	0.00701212116330883\\
2	0.0070121158585532\\
3	0.00701211046433695\\
4	0.00701210497915749\\
5	0.00701209940148715\\
6	0.0070120937297728\\
7	0.00701208796243541\\
8	0.00701208209786965\\
9	0.00701207613444345\\
10	0.00701207007049751\\
11	0.00701206390434495\\
12	0.00701205763427078\\
13	0.00701205125853145\\
14	0.00701204477535449\\
15	0.00701203818293786\\
16	0.00701203147944953\\
17	0.00701202466302706\\
18	0.00701201773177704\\
19	0.00701201068377461\\
20	0.00701200351706295\\
21	0.00701199622965271\\
22	0.00701198881952148\\
23	0.00701198128461338\\
24	0.00701197362283837\\
25	0.00701196583207182\\
26	0.0070119579101537\\
27	0.00701194985488829\\
28	0.00701194166404353\\
29	0.00701193333535033\\
30	0.00701192486650201\\
31	0.00701191625515377\\
32	0.007011907498922\\
33	0.00701189859538364\\
34	0.00701188954207558\\
35	0.00701188033649402\\
36	0.00701187097609382\\
37	0.00701186145828771\\
38	0.00701185178044581\\
39	0.00701184193989485\\
40	0.00701183193391744\\
41	0.00701182175975135\\
42	0.00701181141458891\\
43	0.00701180089557619\\
44	0.00701179019981224\\
45	0.00701177932434842\\
46	0.00701176826618755\\
47	0.00701175702228318\\
48	0.00701174558953885\\
49	0.00701173396480721\\
50	0.00701172214488927\\
51	0.00701171012653347\\
52	0.0070116979064351\\
53	0.00701168548123514\\
54	0.00701167284751962\\
55	0.00701166000181868\\
56	0.00701164694060568\\
57	0.00701163366029628\\
58	0.00701162015724765\\
59	0.00701160642775744\\
60	0.00701159246806282\\
61	0.00701157827433967\\
62	0.00701156384270145\\
63	0.00701154916919833\\
64	0.00701153424981618\\
65	0.00701151908047565\\
66	0.00701150365703097\\
67	0.00701148797526907\\
68	0.00701147203090845\\
69	0.00701145581959819\\
70	0.00701143933691689\\
71	0.0070114225783714\\
72	0.00701140553939595\\
73	0.00701138821535091\\
74	0.00701137060152166\\
75	0.0070113526931174\\
76	0.00701133448527019\\
77	0.00701131597303348\\
78	0.00701129715138104\\
79	0.0070112780152059\\
80	0.00701125855931881\\
81	0.00701123877844722\\
82	0.00701121866723396\\
83	0.00701119822023592\\
84	0.00701117743192289\\
85	0.00701115629667599\\
86	0.00701113480878663\\
87	0.00701111296245492\\
88	0.00701109075178841\\
89	0.00701106817080076\\
90	0.00701104521341026\\
91	0.00701102187343835\\
92	0.00701099814460837\\
93	0.00701097402054388\\
94	0.00701094949476734\\
95	0.00701092456069852\\
96	0.00701089921165302\\
97	0.00701087344084075\\
98	0.00701084724136438\\
99	0.00701082060621768\\
100	0.00701079352828397\\
101	0.00701076600033452\\
102	0.00701073801502681\\
103	0.00701070956490305\\
104	0.00701068064238827\\
105	0.00701065123978876\\
106	0.00701062134929025\\
107	0.00701059096295618\\
108	0.00701056007272589\\
109	0.0070105286704128\\
110	0.0070104967477027\\
111	0.00701046429615158\\
112	0.00701043130718397\\
113	0.00701039777209098\\
114	0.00701036368202821\\
115	0.00701032902801386\\
116	0.00701029380092672\\
117	0.00701025799150396\\
118	0.0070102215903391\\
119	0.00701018458787989\\
120	0.00701014697442606\\
121	0.0070101087401271\\
122	0.00701006987497999\\
123	0.00701003036882673\\
124	0.00700999021135221\\
125	0.0070099493920815\\
126	0.00700990790037743\\
127	0.00700986572543801\\
128	0.00700982285629382\\
129	0.00700977928180526\\
130	0.00700973499065963\\
131	0.00700968997136834\\
132	0.00700964421226397\\
133	0.00700959770149705\\
134	0.00700955042703295\\
135	0.00700950237664855\\
136	0.00700945353792868\\
137	0.00700940389826273\\
138	0.00700935344484077\\
139	0.00700930216464976\\
140	0.00700925004446939\\
141	0.00700919707086783\\
142	0.00700914323019746\\
143	0.00700908850858996\\
144	0.00700903289195151\\
145	0.00700897636595761\\
146	0.00700891891604771\\
147	0.00700886052741928\\
148	0.00700880118502204\\
149	0.00700874087355126\\
150	0.00700867957744124\\
151	0.00700861728085784\\
152	0.0070085539676912\\
153	0.00700848962154737\\
154	0.00700842422573996\\
155	0.00700835776328083\\
156	0.0070082902168706\\
157	0.00700822156888834\\
158	0.00700815180138065\\
159	0.00700808089604997\\
160	0.00700800883424242\\
161	0.00700793559693475\\
162	0.00700786116472039\\
163	0.00700778551779494\\
164	0.00700770863594062\\
165	0.00700763049850998\\
166	0.00700755108440853\\
167	0.00700747037207685\\
168	0.00700738833947156\\
169	0.00700730496404541\\
170	0.00700722022272688\\
171	0.00700713409189867\\
172	0.0070070465473757\\
173	0.00700695756438277\\
174	0.0070068671175316\\
175	0.00700677518079791\\
176	0.0070066817274985\\
177	0.00700658673026886\\
178	0.00700649016104153\\
179	0.00700639199102593\\
180	0.00700629219068988\\
181	0.00700619072974375\\
182	0.00700608757712771\\
183	0.00700598270100331\\
184	0.00700587606875001\\
185	0.00700576764696783\\
186	0.00700565740148734\\
187	0.00700554529738814\\
188	0.007005431299027\\
189	0.00700531537007689\\
190	0.00700519747357758\\
191	0.00700507757199851\\
192	0.00700495562731353\\
193	0.00700483160108602\\
194	0.00700470545456245\\
195	0.00700457714877309\\
196	0.00700444664464774\\
197	0.0070043139031749\\
198	0.00700417888562068\\
199	0.00700404155337168\\
200	0.00700390186720293\\
201	0.00700375978723681\\
202	0.00700361527293231\\
203	0.00700346828307445\\
204	0.00700331877576297\\
205	0.00700316670840151\\
206	0.00700301203768601\\
207	0.00700285471959327\\
208	0.00700269470936933\\
209	0.0070025319615174\\
210	0.0070023664297858\\
211	0.00700219806715577\\
212	0.00700202682582877\\
213	0.00700185265721398\\
214	0.00700167551191515\\
215	0.0070014953397177\\
216	0.00700131208957524\\
217	0.00700112570959605\\
218	0.0070009361470293\\
219	0.00700074334825093\\
220	0.00700054725874964\\
221	0.00700034782311216\\
222	0.00700014498500878\\
223	0.0069999386871782\\
224	0.00699972887141246\\
225	0.00699951547854149\\
226	0.00699929844841746\\
227	0.00699907771989871\\
228	0.00699885323083372\\
229	0.00699862491804464\\
230	0.0069983927173104\\
231	0.0069981565633499\\
232	0.00699791638980476\\
233	0.00699767212922168\\
234	0.0069974237130348\\
235	0.00699717107154743\\
236	0.00699691413391379\\
237	0.00699665282812034\\
238	0.0069963870809668\\
239	0.00699611681804689\\
240	0.00699584196372884\\
241	0.0069955624411354\\
242	0.00699527817212372\\
243	0.0069949890772649\\
244	0.00699469507582308\\
245	0.00699439608573423\\
246	0.00699409202358484\\
247	0.00699378280458979\\
248	0.00699346834257033\\
249	0.00699314854993146\\
250	0.00699282333763886\\
251	0.00699249261519567\\
252	0.00699215629061865\\
253	0.006991814270414\\
254	0.00699146645955284\\
255	0.00699111276144603\\
256	0.00699075307791886\\
257	0.00699038730918494\\
258	0.00699001535381969\\
259	0.00698963710873354\\
260	0.00698925246914425\\
261	0.00698886132854892\\
262	0.00698846357869551\\
263	0.00698805910955341\\
264	0.0069876478092838\\
265	0.0069872295642091\\
266	0.00698680425878206\\
267	0.00698637177555383\\
268	0.00698593199514149\\
269	0.00698548479619506\\
270	0.00698503005536338\\
271	0.00698456764725966\\
272	0.00698409744442576\\
273	0.00698361931729606\\
274	0.00698313313416024\\
275	0.00698263876112519\\
276	0.00698213606207617\\
277	0.00698162489863678\\
278	0.00698110513012813\\
279	0.00698057661352673\\
280	0.00698003920342169\\
281	0.00697949275197032\\
282	0.00697893710885287\\
283	0.00697837212122592\\
284	0.00697779763367445\\
285	0.00697721348816269\\
286	0.00697661952398332\\
287	0.00697601557770547\\
288	0.00697540148312096\\
289	0.00697477707118888\\
290	0.00697414216997879\\
291	0.00697349660461161\\
292	0.00697284019719909\\
293	0.00697217276678098\\
294	0.00697149412926034\\
295	0.0069708040973366\\
296	0.00697010248043605\\
297	0.00696938908464039\\
298	0.00696866371261231\\
299	0.00696792616351857\\
300	0.00696717623295016\\
301	0.00696641371283952\\
302	0.00696563839137457\\
303	0.00696485005290942\\
304	0.00696404847787139\\
305	0.0069632334426646\\
306	0.00696240471956947\\
307	0.00696156207663779\\
308	0.006960705277584\\
309	0.00695983408167112\\
310	0.00695894824359223\\
311	0.00695804751334643\\
312	0.00695713163610941\\
313	0.00695620035209805\\
314	0.00695525339642884\\
315	0.00695429049896952\\
316	0.00695331138418405\\
317	0.00695231577096971\\
318	0.00695130337248641\\
319	0.00695027389597768\\
320	0.00694922704258255\\
321	0.00694816250713802\\
322	0.00694707997797123\\
323	0.00694597913668085\\
324	0.00694485965790696\\
325	0.00694372120908883\\
326	0.00694256345020909\\
327	0.00694138603352449\\
328	0.00694018860328137\\
329	0.0069389707954152\\
330	0.0069377322372331\\
331	0.00693647254707809\\
332	0.00693519133397384\\
333	0.0069338881972483\\
334	0.00693256272613508\\
335	0.00693121449935059\\
336	0.00692984308464523\\
337	0.00692844803832713\\
338	0.00692702890475551\\
339	0.00692558521580226\\
340	0.00692411649027866\\
341	0.00692262223332462\\
342	0.00692110193575778\\
343	0.00691955507337899\\
344	0.00691798110623043\\
345	0.00691637947780302\\
346	0.00691474961418832\\
347	0.00691309092317046\\
348	0.00691140279325312\\
349	0.00690968459261531\\
350	0.00690793566799061\\
351	0.00690615534346191\\
352	0.00690434291916476\\
353	0.00690249766989062\\
354	0.00690061884358072\\
355	0.00689870565969982\\
356	0.00689675730747893\\
357	0.00689477294401382\\
358	0.00689275169220508\\
359	0.00689069263852404\\
360	0.00688859483058697\\
361	0.00688645727451776\\
362	0.0068842789320776\\
363	0.00688205871753681\\
364	0.00687979549426217\\
365	0.00687748807098947\\
366	0.00687513519774751\\
367	0.00687273556139631\\
368	0.00687028778073843\\
369	0.006867790401157\\
370	0.00686524188873027\\
371	0.00686264062376682\\
372	0.00685998489370048\\
373	0.00685727288527866\\
374	0.00685450267597319\\
375	0.00685167222453705\\
376	0.00684877936062798\\
377	0.00684582177341692\\
378	0.0068427969991017\\
379	0.00683970240724956\\
380	0.0068365351859054\\
381	0.0068332923254236\\
382	0.00682997060101846\\
383	0.0068265665540891\\
384	0.00682307647247982\\
385	0.00681949637001059\\
386	0.00681582196590169\\
387	0.00681204866509814\\
388	0.00680817154044249\\
389	0.00680418531386084\\
390	0.00680008431010625\\
391	0.00679586243472893\\
392	0.00679151321162811\\
393	0.00678702981014269\\
394	0.00678240509034287\\
395	0.00677763170285273\\
396	0.00677270236258949\\
397	0.00676761019656155\\
398	0.00676234896814914\\
399	0.00675691344267549\\
400	0.00675130005788015\\
401	0.00674550787862871\\
402	0.00673953995275049\\
403	0.00673340522722306\\
404	0.00672712117799294\\
405	0.00672071625804746\\
406	0.00671421349619451\\
407	0.00670761146295683\\
408	0.00670090870046873\\
409	0.00669410372057442\\
410	0.00668719500258211\\
411	0.00668018099058624\\
412	0.00667306009025135\\
413	0.00666583066493078\\
414	0.0066584910309769\\
415	0.00665103945208354\\
416	0.00664347413248083\\
417	0.00663579320869548\\
418	0.00662799473913034\\
419	0.00662007668933002\\
420	0.00661203691626788\\
421	0.00660387315302666\\
422	0.00659558299162864\\
423	0.00658716386409457\\
424	0.00657861302397764\\
425	0.00656992753692295\\
426	0.00656110430995516\\
427	0.00655214026025386\\
428	0.00654303296443749\\
429	0.00653377909572568\\
430	0.00652437365427828\\
431	0.00651481037914996\\
432	0.00650508044441175\\
433	0.00649516897538018\\
434	0.00648506248822431\\
435	0.00647475124985164\\
436	0.0064642242365699\\
437	0.00645346889958397\\
438	0.00644247087869849\\
439	0.00643121364884752\\
440	0.00641967806504588\\
441	0.00640784181270222\\
442	0.00639567874715611\\
443	0.00638315807704225\\
444	0.00637024335089058\\
445	0.00635689119486041\\
446	0.00634304973467485\\
447	0.00632865661554167\\
448	0.00631363650875949\\
449	0.00629789796108431\\
450	0.00628132940073313\\
451	0.00626379406023011\\
452	0.00624512351136802\\
453	0.00622510944104158\\
454	0.00620349327436457\\
455	0.00617995345149095\\
456	0.00615409125076077\\
457	0.00612542075730385\\
458	0.00609601646693627\\
459	0.00606616241725746\\
460	0.00603584092572867\\
461	0.00600502899017431\\
462	0.00597372174869766\\
463	0.00594191505454013\\
464	0.00590960567751194\\
465	0.00587679155557354\\
466	0.0058434721069465\\
467	0.00580964861900482\\
468	0.00577532473337828\\
469	0.00574050705142841\\
470	0.00570520588925315\\
471	0.00566943621445993\\
472	0.0056332188542572\\
473	0.0055965820362056\\
474	0.00555956331100378\\
475	0.00552221199590847\\
476	0.00548459229280957\\
477	0.00544678728978496\\
478	0.00540890413927234\\
479	0.00537108040097315\\
480	0.00533349233870938\\
481	0.00529636564331368\\
482	0.00525998922738064\\
483	0.00522473313857186\\
484	0.00519107140812409\\
485	0.00515961030441914\\
486	0.00513111942658377\\
487	0.00510583050143825\\
488	0.00508025809679198\\
489	0.00505441758935908\\
490	0.00502832694705126\\
491	0.00500200695922378\\
492	0.00497548144923012\\
493	0.00494877744918957\\
494	0.0049219253078496\\
495	0.00489495868990997\\
496	0.00486791440817347\\
497	0.0048408320076661\\
498	0.00481375299114852\\
499	0.00478671953457748\\
500	0.00475977245909883\\
501	0.00473294813251668\\
502	0.00470627391279249\\
503	0.00467976119434337\\
504	0.00465339454787544\\
505	0.00462711890340711\\
506	0.00460081918406089\\
507	0.00457429104784732\\
508	0.00454720443578428\\
509	0.00451949920381706\\
510	0.00449112857439873\\
511	0.00446202316154678\\
512	0.00443214253187289\\
513	0.00440144139335974\\
514	0.00436986892632396\\
515	0.00433738351339268\\
516	0.00430396765942999\\
517	0.00426955412733077\\
518	0.00423406767181877\\
519	0.00419742583052031\\
520	0.00415953875905356\\
521	0.00412030917420201\\
522	0.00407963455826062\\
523	0.00403741149826289\\
524	0.00399354342326266\\
525	0.00394794794373237\\
526	0.00390054155404816\\
527	0.00385122407791419\\
528	0.0037998565790712\\
529	0.00374634929783748\\
530	0.00369061540493826\\
531	0.00363257431252558\\
532	0.00357216559047094\\
533	0.00350934272525963\\
534	0.00344407564145591\\
535	0.00337635592798696\\
536	0.00330620316315285\\
537	0.00323367434338437\\
538	0.00315887169747382\\
539	0.00308195591005351\\
540	0.00300314669175381\\
541	0.00292271428666918\\
542	0.00284102108688311\\
543	0.00275866609982855\\
544	0.00267662723432266\\
545	0.00259850455214081\\
546	0.00252561474725806\\
547	0.00245828060502498\\
548	0.00239657932141828\\
549	0.00233997269311372\\
550	0.00228472809978635\\
551	0.00223052266454992\\
552	0.00217693151171248\\
553	0.00212346363709513\\
554	0.00207004191030077\\
555	0.00201655731486821\\
556	0.00196290412541734\\
557	0.00190901167585809\\
558	0.00185475671516\\
559	0.00179972684135543\\
560	0.00174381965237555\\
561	0.00168751872449357\\
562	0.00163222103135658\\
563	0.00157815070953452\\
564	0.00152562817316902\\
565	0.00147318530183207\\
566	0.00142057936312707\\
567	0.00136778956338732\\
568	0.00131479124306664\\
569	0.0012616240709895\\
570	0.00120923369631625\\
571	0.00115852724686328\\
572	0.00110937816652107\\
573	0.00106041064718068\\
574	0.00101157345479646\\
575	0.000962869504412621\\
576	0.000914347675446341\\
577	0.000866058972727197\\
578	0.000818515558008548\\
579	0.000771676801487267\\
580	0.000725067167815385\\
581	0.000678702038941237\\
582	0.000632633102614749\\
583	0.000586915166466385\\
584	0.000541602424099719\\
585	0.000496746650654584\\
586	0.000452395225472513\\
587	0.000408588691486123\\
588	0.000365357799889246\\
589	0.000322720123837111\\
590	0.000280676711614414\\
591	0.000239312817487403\\
592	0.000198788644961841\\
593	0.000159293651685586\\
594	0.000121079888727805\\
595	8.45520570083913e-05\\
596	5.05092148680373e-05\\
597	2.07908715710836e-05\\
598	0\\
599	0\\
600	0\\
};
\addplot [color=blue!50!mycolor7,solid,forget plot]
  table[row sep=crcr]{%
1	0.0074068299566467\\
2	0.00740682441785876\\
3	0.00740681878582221\\
4	0.00740681305897183\\
5	0.00740680723571628\\
6	0.00740680131443758\\
7	0.00740679529349066\\
8	0.00740678917120309\\
9	0.00740678294587443\\
10	0.00740677661577581\\
11	0.00740677017914957\\
12	0.00740676363420853\\
13	0.0074067569791359\\
14	0.00740675021208434\\
15	0.00740674333117578\\
16	0.00740673633450079\\
17	0.00740672922011808\\
18	0.00740672198605392\\
19	0.00740671463030165\\
20	0.00740670715082115\\
21	0.00740669954553828\\
22	0.00740669181234433\\
23	0.00740668394909534\\
24	0.00740667595361162\\
25	0.00740666782367715\\
26	0.00740665955703898\\
27	0.00740665115140663\\
28	0.00740664260445139\\
29	0.00740663391380575\\
30	0.00740662507706283\\
31	0.00740661609177554\\
32	0.00740660695545612\\
33	0.00740659766557532\\
34	0.00740658821956183\\
35	0.00740657861480151\\
36	0.00740656884863661\\
37	0.00740655891836529\\
38	0.00740654882124067\\
39	0.00740653855447019\\
40	0.0074065281152148\\
41	0.0074065175005883\\
42	0.00740650670765637\\
43	0.007406495733436\\
44	0.00740648457489452\\
45	0.00740647322894892\\
46	0.00740646169246489\\
47	0.00740644996225608\\
48	0.00740643803508317\\
49	0.00740642590765303\\
50	0.00740641357661781\\
51	0.00740640103857418\\
52	0.00740638829006216\\
53	0.00740637532756442\\
54	0.00740636214750523\\
55	0.00740634874624953\\
56	0.00740633512010191\\
57	0.00740632126530572\\
58	0.00740630717804195\\
59	0.00740629285442831\\
60	0.00740627829051807\\
61	0.00740626348229912\\
62	0.00740624842569288\\
63	0.00740623311655316\\
64	0.00740621755066502\\
65	0.00740620172374375\\
66	0.00740618563143371\\
67	0.00740616926930701\\
68	0.00740615263286254\\
69	0.00740613571752465\\
70	0.00740611851864188\\
71	0.00740610103148595\\
72	0.00740608325125019\\
73	0.00740606517304846\\
74	0.00740604679191379\\
75	0.00740602810279702\\
76	0.00740600910056551\\
77	0.00740598978000174\\
78	0.00740597013580194\\
79	0.00740595016257458\\
80	0.00740592985483905\\
81	0.00740590920702419\\
82	0.00740588821346668\\
83	0.00740586686840964\\
84	0.00740584516600103\\
85	0.00740582310029207\\
86	0.00740580066523575\\
87	0.00740577785468505\\
88	0.00740575466239144\\
89	0.00740573108200305\\
90	0.0074057071070631\\
91	0.00740568273100809\\
92	0.00740565794716598\\
93	0.00740563274875449\\
94	0.00740560712887922\\
95	0.00740558108053179\\
96	0.00740555459658784\\
97	0.00740552766980521\\
98	0.00740550029282195\\
99	0.00740547245815421\\
100	0.00740544415819427\\
101	0.00740541538520844\\
102	0.00740538613133496\\
103	0.00740535638858165\\
104	0.00740532614882389\\
105	0.00740529540380227\\
106	0.00740526414512026\\
107	0.00740523236424184\\
108	0.00740520005248921\\
109	0.00740516720104009\\
110	0.00740513380092542\\
111	0.00740509984302671\\
112	0.00740506531807335\\
113	0.00740503021664007\\
114	0.00740499452914394\\
115	0.00740495824584187\\
116	0.00740492135682741\\
117	0.00740488385202798\\
118	0.00740484572120179\\
119	0.00740480695393475\\
120	0.00740476753963729\\
121	0.007404727467541\\
122	0.00740468672669541\\
123	0.00740464530596449\\
124	0.00740460319402304\\
125	0.00740456037935315\\
126	0.0074045168502405\\
127	0.00740447259477036\\
128	0.00740442760082377\\
129	0.00740438185607332\\
130	0.00740433534797922\\
131	0.00740428806378474\\
132	0.00740423999051178\\
133	0.00740419111495641\\
134	0.007404141423684\\
135	0.0074040909030245\\
136	0.00740403953906722\\
137	0.00740398731765573\\
138	0.00740393422438254\\
139	0.00740388024458343\\
140	0.00740382536333186\\
141	0.007403769565433\\
142	0.00740371283541765\\
143	0.00740365515753591\\
144	0.00740359651575079\\
145	0.00740353689373137\\
146	0.00740347627484593\\
147	0.00740341464215491\\
148	0.00740335197840341\\
149	0.00740328826601369\\
150	0.00740322348707738\\
151	0.00740315762334745\\
152	0.0074030906562299\\
153	0.00740302256677547\\
154	0.0074029533356708\\
155	0.00740288294322967\\
156	0.00740281136938397\\
157	0.00740273859367428\\
158	0.00740266459524067\\
159	0.00740258935281312\\
160	0.00740251284470192\\
161	0.00740243504878779\\
162	0.00740235594251239\\
163	0.00740227550286832\\
164	0.00740219370638955\\
165	0.00740211052914166\\
166	0.0074020259467126\\
167	0.00740193993420315\\
168	0.00740185246621825\\
169	0.00740176351685822\\
170	0.0074016730597107\\
171	0.00740158106784308\\
172	0.00740148751379556\\
173	0.00740139236957476\\
174	0.0074012956066485\\
175	0.0074011971959411\\
176	0.00740109710782996\\
177	0.00740099531214307\\
178	0.00740089177815762\\
179	0.00740078647459987\\
180	0.00740067936964604\\
181	0.00740057043092449\\
182	0.00740045962551893\\
183	0.00740034691997241\\
184	0.00740023228029238\\
185	0.00740011567195596\\
186	0.00739999705991545\\
187	0.0073998764086036\\
188	0.00739975368193796\\
189	0.00739962884332372\\
190	0.00739950185565448\\
191	0.00739937268130986\\
192	0.00739924128214912\\
193	0.00739910761950039\\
194	0.00739897165414423\\
195	0.0073988333462918\\
196	0.00739869265555768\\
197	0.00739854954092654\\
198	0.00739840396070977\\
199	0.00739825587250141\\
200	0.00739810523316437\\
201	0.00739795199881759\\
202	0.00739779612482297\\
203	0.0073976375657721\\
204	0.00739747627547275\\
205	0.00739731220693504\\
206	0.0073971453123574\\
207	0.00739697554311232\\
208	0.00739680284973175\\
209	0.00739662718189224\\
210	0.00739644848839992\\
211	0.007396266717175\\
212	0.00739608181523616\\
213	0.00739589372868453\\
214	0.0073957024026875\\
215	0.00739550778146202\\
216	0.00739530980825783\\
217	0.00739510842534015\\
218	0.00739490357397224\\
219	0.00739469519439751\\
220	0.00739448322582124\\
221	0.00739426760639218\\
222	0.00739404827318354\\
223	0.00739382516217373\\
224	0.00739359820822686\\
225	0.00739336734507254\\
226	0.00739313250528566\\
227	0.00739289362026561\\
228	0.00739265062021495\\
229	0.00739240343411795\\
230	0.00739215198971861\\
231	0.00739189621349811\\
232	0.007391636030652\\
233	0.00739137136506691\\
234	0.00739110213929665\\
235	0.00739082827453812\\
236	0.00739054969060642\\
237	0.00739026630590975\\
238	0.00738997803742359\\
239	0.00738968480066449\\
240	0.0073893865096632\\
241	0.00738908307693744\\
242	0.007388774413464\\
243	0.00738846042865016\\
244	0.00738814103030472\\
245	0.00738781612460839\\
246	0.00738748561608332\\
247	0.00738714940756228\\
248	0.00738680740015708\\
249	0.00738645949322613\\
250	0.00738610558434173\\
251	0.0073857455692561\\
252	0.00738537934186714\\
253	0.00738500679418307\\
254	0.00738462781628656\\
255	0.0073842422962979\\
256	0.0073838501203373\\
257	0.00738345117248648\\
258	0.00738304533474929\\
259	0.00738263248701129\\
260	0.00738221250699872\\
261	0.0073817852702362\\
262	0.00738135065000343\\
263	0.00738090851729121\\
264	0.00738045874075587\\
265	0.00738000118667315\\
266	0.0073795357188904\\
267	0.00737906219877818\\
268	0.00737858048518024\\
269	0.00737809043436237\\
270	0.00737759189996016\\
271	0.00737708473292503\\
272	0.00737656878146932\\
273	0.0073760438910098\\
274	0.00737550990410958\\
275	0.00737496666041871\\
276	0.00737441399661314\\
277	0.00737385174633213\\
278	0.00737327974011379\\
279	0.00737269780532916\\
280	0.00737210576611426\\
281	0.00737150344330047\\
282	0.00737089065434291\\
283	0.00737026721324675\\
284	0.00736963293049165\\
285	0.00736898761295384\\
286	0.00736833106382621\\
287	0.00736766308253593\\
288	0.00736698346465981\\
289	0.0073662920018371\\
290	0.00736558848167974\\
291	0.0073648726876801\\
292	0.00736414439911577\\
293	0.00736340339095163\\
294	0.00736264943373898\\
295	0.00736188229351134\\
296	0.00736110173167753\\
297	0.00736030750491094\\
298	0.00735949936503572\\
299	0.00735867705890934\\
300	0.00735784032830132\\
301	0.00735698890976823\\
302	0.00735612253452471\\
303	0.00735524092831032\\
304	0.00735434381125222\\
305	0.00735343089772323\\
306	0.00735250189619534\\
307	0.0073515565090885\\
308	0.00735059443261443\\
309	0.00734961535661524\\
310	0.00734861896439659\\
311	0.00734760493255559\\
312	0.00734657293080274\\
313	0.00734552262177798\\
314	0.00734445366086052\\
315	0.00734336569597255\\
316	0.00734225836737576\\
317	0.00734113130746157\\
318	0.00733998414053407\\
319	0.00733881648258547\\
320	0.00733762794106414\\
321	0.00733641811463464\\
322	0.00733518659292982\\
323	0.0073339329562943\\
324	0.00733265677551952\\
325	0.00733135761156964\\
326	0.00733003501529837\\
327	0.00732868852715621\\
328	0.00732731767688786\\
329	0.0073259219832197\\
330	0.00732450095353673\\
331	0.00732305408354902\\
332	0.00732158085694693\\
333	0.00732008074504546\\
334	0.00731855320641677\\
335	0.00731699768651116\\
336	0.00731541361726578\\
337	0.00731380041670108\\
338	0.00731215748850482\\
339	0.00731048422160312\\
340	0.00730877998971845\\
341	0.00730704415091459\\
342	0.00730527604712798\\
343	0.00730347500368556\\
344	0.00730164032880894\\
345	0.00729977131310471\\
346	0.00729786722904094\\
347	0.00729592733040972\\
348	0.00729395085177594\\
349	0.00729193700791217\\
350	0.00728988499322004\\
351	0.00728779398113814\\
352	0.00728566312353681\\
353	0.00728349155010023\\
354	0.00728127836769626\\
355	0.00727902265973467\\
356	0.00727672348551477\\
357	0.00727437987956289\\
358	0.00727199085096141\\
359	0.00726955538267042\\
360	0.00726707243084388\\
361	0.00726454092414224\\
362	0.007261959763044\\
363	0.00725932781915937\\
364	0.00725664393454936\\
365	0.00725390692105469\\
366	0.00725111555964022\\
367	0.00724826859976054\\
368	0.00724536475875515\\
369	0.00724240272128197\\
370	0.00723938113880114\\
371	0.00723629862912245\\
372	0.00723315377603356\\
373	0.0072299451290294\\
374	0.00722667120316761\\
375	0.00722333047908035\\
376	0.00721992140317944\\
377	0.00721644238809939\\
378	0.0072128918134324\\
379	0.00720926802682098\\
380	0.00720556934548708\\
381	0.0072017940582922\\
382	0.0071979404284423\\
383	0.00719400669697306\\
384	0.00718999108717504\\
385	0.00718589181014539\\
386	0.00718170707167325\\
387	0.00717743508066228\\
388	0.00717307405923942\\
389	0.00716862225468668\\
390	0.00716407795460443\\
391	0.00715943950628516\\
392	0.00715470533720593\\
393	0.00714987397793557\\
394	0.00714494408796095\\
395	0.00713991448424189\\
396	0.0071347841672599\\
397	0.00712955233752372\\
398	0.00712421840955929\\
399	0.00711878202108357\\
400	0.00711324302348959\\
401	0.00710760144058647\\
402	0.00710185737522207\\
403	0.00709601083121614\\
404	0.0070900613992831\\
405	0.00708400778027166\\
406	0.00707784782427909\\
407	0.00707157929726241\\
408	0.00706519989025344\\
409	0.00705870721664789\\
410	0.00705209880977829\\
411	0.00704537212084409\\
412	0.00703852451725753\\
413	0.00703155328144083\\
414	0.00702445561010178\\
415	0.00701722861410645\\
416	0.00700986931951129\\
417	0.00700237467173069\\
418	0.00699474154818311\\
419	0.00698696679845117\\
420	0.00697904714820445\\
421	0.00697097917369077\\
422	0.00696275928721133\\
423	0.00695438372097599\\
424	0.00694584850944636\\
425	0.0069371494704971\\
426	0.0069282821853252\\
427	0.00691924197185974\\
428	0.00691002381607948\\
429	0.00690062229619514\\
430	0.00689103157279853\\
431	0.0068812453785263\\
432	0.00687125700243555\\
433	0.00686105947035154\\
434	0.00685064572278754\\
435	0.00684000842741724\\
436	0.00682913979427269\\
437	0.00681803157463512\\
438	0.00680667509094791\\
439	0.00679506134536192\\
440	0.00678318167415435\\
441	0.00677102761914462\\
442	0.00675859057904313\\
443	0.00674586191458853\\
444	0.00673283310171506\\
445	0.00671949595028459\\
446	0.00670584291204887\\
447	0.00669186750977668\\
448	0.00667756493066131\\
449	0.00666293284232163\\
450	0.00664797251059849\\
451	0.00663269032774426\\
452	0.0066170999029803\\
453	0.00660122493712663\\
454	0.00658510323055887\\
455	0.00656879244459335\\
456	0.00655237888856633\\
457	0.00653598164334163\\
458	0.00651975971387527\\
459	0.00650381850604284\\
460	0.00648827189993001\\
461	0.00647325382423801\\
462	0.00645794555767362\\
463	0.00644233367387723\\
464	0.00642640309573541\\
465	0.00641013679445086\\
466	0.00639351544169336\\
467	0.00637651698950613\\
468	0.00635911616107912\\
469	0.00634128383119084\\
470	0.00632298626971447\\
471	0.0063041842149933\\
472	0.00628483173380949\\
473	0.00626487481204921\\
474	0.00624424960687539\\
475	0.00622288027233064\\
476	0.00620067624437277\\
477	0.00617752883802573\\
478	0.00615330696433514\\
479	0.00612785173133297\\
480	0.00610096961666709\\
481	0.00607242379360694\\
482	0.00604192306370216\\
483	0.00600910805190212\\
484	0.00597353391254416\\
485	0.00593464965252985\\
486	0.00589177709624734\\
487	0.0058447826614934\\
488	0.00579706219262404\\
489	0.00574862086705563\\
490	0.00569946786929883\\
491	0.00564961737971868\\
492	0.00559908980549789\\
493	0.00554791331535837\\
494	0.00549612575580097\\
495	0.00544377704720198\\
496	0.00539093218421757\\
497	0.00533767499801058\\
498	0.00528411287952206\\
499	0.00523038271546018\\
500	0.00517665835607147\\
501	0.0051231600238064\\
502	0.0050701661987078\\
503	0.00501802872194789\\
504	0.00496721929228177\\
505	0.00491837790758348\\
506	0.0048723283960852\\
507	0.0048301257606557\\
508	0.00479309668846519\\
509	0.00475625955239109\\
510	0.00471898918517486\\
511	0.00468131797440956\\
512	0.0046432840722822\\
513	0.00460493031991499\\
514	0.004566303879972\\
515	0.00452745570207619\\
516	0.00448844008980965\\
517	0.0044493097997433\\
518	0.00441008687171728\\
519	0.00437074144775284\\
520	0.00433122959196758\\
521	0.00429149784774394\\
522	0.00425142897253515\\
523	0.00421081080778946\\
524	0.00416929313380805\\
525	0.00412643381319458\\
526	0.00408215596336811\\
527	0.00403639275687767\\
528	0.00398910860078957\\
529	0.00394021841245417\\
530	0.00388963168405301\\
531	0.00383725090119883\\
532	0.00378297102842801\\
533	0.00372667994895513\\
534	0.00366826052558962\\
535	0.00360759499538104\\
536	0.00354458408089595\\
537	0.00347914048606964\\
538	0.00341120344786853\\
539	0.00334068582296219\\
540	0.00326757016964767\\
541	0.00319187772760867\\
542	0.0031136686903834\\
543	0.00303305075136004\\
544	0.00295018712023015\\
545	0.0028652920901408\\
546	0.00277862134303654\\
547	0.00269059274570993\\
548	0.00260190152765235\\
549	0.00251390181794879\\
550	0.00243061929734025\\
551	0.00235257346475307\\
552	0.00228003875241692\\
553	0.00221317257157155\\
554	0.00215112287025263\\
555	0.00209052503954314\\
556	0.00203108611377265\\
557	0.00197240575620483\\
558	0.00191405466630919\\
559	0.0018558391868882\\
560	0.001797667721791\\
561	0.00173945129258813\\
562	0.00168110421909231\\
563	0.00162243118677364\\
564	0.00156298187657357\\
565	0.00150437000468756\\
566	0.00144703158843678\\
567	0.00139122030448227\\
568	0.00133673591860897\\
569	0.00128228716789082\\
570	0.00122788747298109\\
571	0.00117350595556176\\
572	0.00111939032316793\\
573	0.00106700721614886\\
574	0.00101651208218248\\
575	0.000967001075013799\\
576	0.000917838202573339\\
577	0.000868968269894002\\
578	0.000820423355365856\\
579	0.000772347639745716\\
580	0.000725198081162504\\
581	0.000678755482625295\\
582	0.000632661594736207\\
583	0.000586930577994372\\
584	0.000541610321954773\\
585	0.00049675039409852\\
586	0.000452396819030694\\
587	0.000408589277753164\\
588	0.000365357975688249\\
589	0.000322720162279343\\
590	0.000280676716267942\\
591	0.000239312817487404\\
592	0.000198788644961841\\
593	0.000159293651685586\\
594	0.000121079888727805\\
595	8.45520570083912e-05\\
596	5.05092148680374e-05\\
597	2.07908715710836e-05\\
598	0\\
599	0\\
600	0\\
};
\addplot [color=blue!40!mycolor9,solid,forget plot]
  table[row sep=crcr]{%
1	0.0096074075755447\\
2	0.00960739244151761\\
3	0.00960737705294697\\
4	0.00960736140555528\\
5	0.00960734549499303\\
6	0.0096073293168376\\
7	0.00960731286659196\\
8	0.00960729613968336\\
9	0.00960727913146209\\
10	0.00960726183720023\\
11	0.00960724425209022\\
12	0.00960722637124369\\
13	0.00960720818968978\\
14	0.00960718970237404\\
15	0.0096071709041569\\
16	0.00960715178981218\\
17	0.00960713235402568\\
18	0.0096071125913937\\
19	0.00960709249642146\\
20	0.00960707206352154\\
21	0.00960705128701242\\
22	0.00960703016111673\\
23	0.00960700867995977\\
24	0.00960698683756772\\
25	0.00960696462786599\\
26	0.00960694204467757\\
27	0.00960691908172121\\
28	0.00960689573260967\\
29	0.00960687199084789\\
30	0.0096068478498312\\
31	0.00960682330284341\\
32	0.00960679834305482\\
33	0.00960677296352051\\
34	0.00960674715717811\\
35	0.00960672091684596\\
36	0.00960669423522108\\
37	0.00960666710487688\\
38	0.0096066395182613\\
39	0.00960661146769448\\
40	0.0096065829453667\\
41	0.00960655394333601\\
42	0.00960652445352608\\
43	0.00960649446772381\\
44	0.00960646397757705\\
45	0.0096064329745921\\
46	0.00960640145013138\\
47	0.00960636939541089\\
48	0.0096063368014977\\
49	0.00960630365930738\\
50	0.00960626995960139\\
51	0.00960623569298438\\
52	0.00960620084990154\\
53	0.00960616542063577\\
54	0.00960612939530495\\
55	0.00960609276385898\\
56	0.00960605551607695\\
57	0.00960601764156417\\
58	0.00960597912974904\\
59	0.0096059399698801\\
60	0.00960590015102288\\
61	0.00960585966205662\\
62	0.00960581849167114\\
63	0.00960577662836343\\
64	0.00960573406043442\\
65	0.00960569077598536\\
66	0.00960564676291445\\
67	0.00960560200891331\\
68	0.00960555650146327\\
69	0.00960551022783172\\
70	0.00960546317506837\\
71	0.00960541533000136\\
72	0.00960536667923342\\
73	0.00960531720913784\\
74	0.00960526690585453\\
75	0.00960521575528577\\
76	0.00960516374309205\\
77	0.00960511085468785\\
78	0.00960505707523722\\
79	0.00960500238964931\\
80	0.00960494678257397\\
81	0.00960489023839697\\
82	0.00960483274123542\\
83	0.00960477427493294\\
84	0.00960471482305481\\
85	0.00960465436888295\\
86	0.00960459289541087\\
87	0.00960453038533852\\
88	0.00960446682106704\\
89	0.00960440218469337\\
90	0.00960433645800474\\
91	0.0096042696224732\\
92	0.00960420165924989\\
93	0.00960413254915918\\
94	0.00960406227269294\\
95	0.0096039908100043\\
96	0.00960391814090174\\
97	0.00960384424484273\\
98	0.00960376910092729\\
99	0.00960369268789166\\
100	0.00960361498410159\\
101	0.00960353596754553\\
102	0.00960345561582784\\
103	0.00960337390616179\\
104	0.00960329081536231\\
105	0.00960320631983879\\
106	0.00960312039558761\\
107	0.00960303301818457\\
108	0.00960294416277709\\
109	0.00960285380407652\\
110	0.00960276191634992\\
111	0.00960266847341198\\
112	0.00960257344861666\\
113	0.00960247681484864\\
114	0.00960237854451474\\
115	0.00960227860953499\\
116	0.00960217698133369\\
117	0.00960207363083024\\
118	0.00960196852842973\\
119	0.00960186164401339\\
120	0.009601752946929\\
121	0.00960164240598088\\
122	0.00960152998941985\\
123	0.00960141566493296\\
124	0.00960129939963303\\
125	0.00960118116004798\\
126	0.00960106091210998\\
127	0.00960093862114444\\
128	0.00960081425185868\\
129	0.00960068776833064\\
130	0.009600559133997\\
131	0.00960042831164155\\
132	0.00960029526338298\\
133	0.00960015995066272\\
134	0.00960002233423241\\
135	0.00959988237414115\\
136	0.00959974002972281\\
137	0.00959959525958278\\
138	0.0095994480215847\\
139	0.00959929827283704\\
140	0.00959914596967927\\
141	0.00959899106766809\\
142	0.00959883352156315\\
143	0.00959867328531301\\
144	0.00959851031204037\\
145	0.00959834455402753\\
146	0.00959817596270155\\
147	0.0095980044886191\\
148	0.00959783008145125\\
149	0.00959765268996817\\
150	0.00959747226202343\\
151	0.00959728874453834\\
152	0.00959710208348609\\
153	0.0095969122238756\\
154	0.00959671910973556\\
155	0.00959652268409789\\
156	0.00959632288898141\\
157	0.00959611966537533\\
158	0.00959591295322249\\
159	0.00959570269140253\\
160	0.00959548881771502\\
161	0.00959527126886251\\
162	0.00959504998043319\\
163	0.00959482488688383\\
164	0.00959459592152232\\
165	0.00959436301649017\\
166	0.00959412610274484\\
167	0.00959388511004209\\
168	0.00959363996691783\\
169	0.00959339060067005\\
170	0.00959313693734044\\
171	0.00959287890169572\\
172	0.00959261641720862\\
173	0.00959234940603876\\
174	0.00959207778901272\\
175	0.00959180148560398\\
176	0.00959152041391221\\
177	0.00959123449064189\\
178	0.00959094363108033\\
179	0.00959064774907484\\
180	0.00959034675700905\\
181	0.0095900405657784\\
182	0.00958972908476431\\
183	0.00958941222180756\\
184	0.00958908988318017\\
185	0.00958876197355619\\
186	0.00958842839598119\\
187	0.00958808905184028\\
188	0.00958774384082494\\
189	0.00958739266089855\\
190	0.00958703540826065\\
191	0.00958667197731019\\
192	0.00958630226060791\\
193	0.00958592614883796\\
194	0.00958554353076913\\
195	0.00958515429321589\\
196	0.00958475832099956\\
197	0.00958435549690961\\
198	0.00958394570166566\\
199	0.00958352881387976\\
200	0.00958310471001807\\
201	0.00958267326436181\\
202	0.00958223434896743\\
203	0.00958178783362597\\
204	0.00958133358582178\\
205	0.00958087147069034\\
206	0.00958040135097536\\
207	0.00957992308698488\\
208	0.00957943653654678\\
209	0.00957894155496329\\
210	0.00957843799496454\\
211	0.00957792570666142\\
212	0.00957740453749741\\
213	0.00957687433219943\\
214	0.00957633493272788\\
215	0.00957578617822562\\
216	0.00957522790496593\\
217	0.00957465994629961\\
218	0.00957408213260086\\
219	0.00957349429121224\\
220	0.0095728962463885\\
221	0.00957228781923943\\
222	0.00957166882767131\\
223	0.00957103908632757\\
224	0.00957039840652806\\
225	0.00956974659620724\\
226	0.00956908345985101\\
227	0.00956840879843246\\
228	0.00956772240934639\\
229	0.00956702408634227\\
230	0.00956631361945617\\
231	0.00956559079494125\\
232	0.00956485539519678\\
233	0.00956410719869591\\
234	0.00956334597991188\\
235	0.00956257150924284\\
236	0.00956178355293522\\
237	0.0095609818730054\\
238	0.00956016622716001\\
239	0.00955933636871454\\
240	0.00955849204651045\\
241	0.00955763300483037\\
242	0.00955675898331186\\
243	0.0095558697168594\\
244	0.00955496493555446\\
245	0.00955404436456391\\
246	0.0095531077240466\\
247	0.00955215472905803\\
248	0.009551185089453\\
249	0.00955019850978655\\
250	0.00954919468921262\\
251	0.00954817332138089\\
252	0.00954713409433151\\
253	0.00954607669038755\\
254	0.00954500078604549\\
255	0.00954390605186338\\
256	0.00954279215234682\\
257	0.00954165874583254\\
258	0.00954050548436971\\
259	0.00953933201359898\\
260	0.00953813797262877\\
261	0.00953692299390935\\
262	0.00953568670310429\\
263	0.00953442871895927\\
264	0.00953314865316835\\
265	0.00953184611023756\\
266	0.00953052068734558\\
267	0.0095291719742018\\
268	0.00952779955290148\\
269	0.00952640299777804\\
270	0.00952498187525226\\
271	0.00952353574367869\\
272	0.0095220641531888\\
273	0.00952056664553099\\
274	0.00951904275390769\\
275	0.00951749200280888\\
276	0.00951591390784251\\
277	0.00951430797556143\\
278	0.00951267370328713\\
279	0.00951101057892972\\
280	0.00950931808080451\\
281	0.00950759567744505\\
282	0.00950584282741231\\
283	0.00950405897910038\\
284	0.00950224357053821\\
285	0.00950039602918757\\
286	0.00949851577173702\\
287	0.00949660220389208\\
288	0.00949465472016117\\
289	0.00949267270363763\\
290	0.00949065552577744\\
291	0.0094886025461728\\
292	0.00948651311232144\\
293	0.00948438655939151\\
294	0.00948222220998218\\
295	0.00948001937387975\\
296	0.00947777734780916\\
297	0.00947549541518103\\
298	0.00947317284583406\\
299	0.00947080889577266\\
300	0.00946840280689992\\
301	0.00946595380674576\\
302	0.00946346110819021\\
303	0.00946092390918188\\
304	0.00945834139245133\\
305	0.00945571272521959\\
306	0.00945303705890157\\
307	0.00945031352880442\\
308	0.00944754125382065\\
309	0.00944471933611625\\
310	0.0094418468608135\\
311	0.00943892289566851\\
312	0.00943594649074358\\
313	0.00943291667807412\\
314	0.0094298324713303\\
315	0.00942669286547325\\
316	0.00942349683640599\\
317	0.00942024334061878\\
318	0.00941693131482899\\
319	0.00941355967561562\\
320	0.00941012731904821\\
321	0.00940663312031021\\
322	0.00940307593331675\\
323	0.00939945459032695\\
324	0.00939576790155048\\
325	0.00939201465474842\\
326	0.00938819361482865\\
327	0.00938430352343525\\
328	0.0093803430985324\\
329	0.00937631103398228\\
330	0.00937220599911714\\
331	0.00936802663830553\\
332	0.00936377157051248\\
333	0.00935943938885354\\
334	0.00935502866014277\\
335	0.00935053792443428\\
336	0.00934596569455752\\
337	0.00934131045564591\\
338	0.00933657066465869\\
339	0.00933174474989592\\
340	0.00932683111050627\\
341	0.00932182811598741\\
342	0.00931673410567867\\
343	0.00931154738824574\\
344	0.00930626624115684\\
345	0.00930088891015026\\
346	0.00929541360869242\\
347	0.00928983851742644\\
348	0.00928416178361004\\
349	0.00927838152054283\\
350	0.00927249580698165\\
351	0.0092665026865438\\
352	0.00926040016709684\\
353	0.00925418622013451\\
354	0.00924785878013752\\
355	0.00924141574391836\\
356	0.00923485496994897\\
357	0.00922817427767003\\
358	0.00922137144678084\\
359	0.00921444421650832\\
360	0.00920739028485377\\
361	0.00920020730781615\\
362	0.00919289289859011\\
363	0.00918544462673757\\
364	0.00917786001733097\\
365	0.00917013655006702\\
366	0.0091622716583488\\
367	0.00915426272833516\\
368	0.00914610709795526\\
369	0.00913780205588702\\
370	0.00912934484049751\\
371	0.00912073263874376\\
372	0.00911196258503221\\
373	0.00910303176003477\\
374	0.00909393718945951\\
375	0.00908467584277375\\
376	0.00907524463187639\\
377	0.00906564040971642\\
378	0.00905585996885271\\
379	0.00904590003994982\\
380	0.00903575729020174\\
381	0.0090254283216738\\
382	0.00901490966954928\\
383	0.00900419780026284\\
384	0.00899328910949718\\
385	0.0089821799200117\\
386	0.00897086647926193\\
387	0.00895934495675723\\
388	0.0089476114410925\\
389	0.00893566193658549\\
390	0.00892349235939025\\
391	0.00891109853290295\\
392	0.00889847618237991\\
393	0.00888562092862454\\
394	0.00887252828055634\\
395	0.00885919362642484\\
396	0.0088456122235615\\
397	0.00883177918680417\\
398	0.00881768947542669\\
399	0.00880333787845024\\
400	0.00878871899876059\\
401	0.00877382723711889\\
402	0.00875865677849773\\
403	0.00874320158609038\\
404	0.00872745541512584\\
405	0.00871141187025567\\
406	0.00869506458123908\\
407	0.00867840695760704\\
408	0.00866143169002759\\
409	0.00864413122629998\\
410	0.00862649776265295\\
411	0.00860852323537416\\
412	0.00859019931259046\\
413	0.00857151738521099\\
414	0.00855246855384727\\
415	0.00853304360289888\\
416	0.0085132329392359\\
417	0.00849302644087801\\
418	0.00847241309164893\\
419	0.00845137969852336\\
420	0.00842991615910179\\
421	0.00840801263579948\\
422	0.00838565906355578\\
423	0.00836284514951068\\
424	0.00833956037385398\\
425	0.00831579399204311\\
426	0.00829153503852154\\
427	0.00826677233202398\\
428	0.00824149448407581\\
429	0.00821568991226852\\
430	0.00818934685571328\\
431	0.00816245339275658\\
432	0.00813499746363264\\
433	0.00810696689224114\\
434	0.00807834940349541\\
435	0.00804913265733362\\
436	0.00801930431272337\\
437	0.00798885212316119\\
438	0.00795776403013172\\
439	0.00792602794353914\\
440	0.00789361383292988\\
441	0.0078604925639844\\
442	0.00782664425161163\\
443	0.00779204802704287\\
444	0.00775668192794685\\
445	0.00772052276554722\\
446	0.00768354596203074\\
447	0.00764572534928105\\
448	0.00760703291693255\\
449	0.00756743849364348\\
450	0.0075269093399834\\
451	0.00748540962396113\\
452	0.00744289974033823\\
453	0.00739933542136051\\
454	0.00735466656663158\\
455	0.0073088356848356\\
456	0.00726177573290427\\
457	0.00721340736359842\\
458	0.00716363577891605\\
459	0.00711235074722052\\
460	0.00705942633796879\\
461	0.00700506357462199\\
462	0.0069897888830494\\
463	0.00697415177545719\\
464	0.00695813781822587\\
465	0.00694173199708212\\
466	0.0069249183178362\\
467	0.00690767973509524\\
468	0.0068899980807137\\
469	0.00687185399387683\\
470	0.00685322685424857\\
471	0.00683409471452101\\
472	0.00681443428282061\\
473	0.00679422095417364\\
474	0.0067734288573931\\
475	0.00675203096749532\\
476	0.00672999931676739\\
477	0.00670730535859358\\
478	0.00668392059335595\\
479	0.00665981715701593\\
480	0.00663496905067333\\
481	0.00660935439457848\\
482	0.00658296001510484\\
483	0.00655577770972172\\
484	0.00652780995818072\\
485	0.00649907640909796\\
486	0.00646961933516901\\
487	0.00643950390821288\\
488	0.00640880524000931\\
489	0.00637750451331813\\
490	0.00634558129860116\\
491	0.00631301334562971\\
492	0.00627977635446164\\
493	0.00624584372821354\\
494	0.00621118631284592\\
495	0.00617577213330074\\
496	0.00613956614119091\\
497	0.00610252999712457\\
498	0.00606462192026863\\
499	0.00602579664634754\\
500	0.00598600553441714\\
501	0.00594519683091755\\
502	0.00590331599908454\\
503	0.0058603058449187\\
504	0.00581513748167444\\
505	0.0057661407166878\\
506	0.00571256209853997\\
507	0.00565347260740609\\
508	0.00558774672291453\\
509	0.00552027502761627\\
510	0.00545171558099382\\
511	0.00538210725278225\\
512	0.00531150662435504\\
513	0.00523999271506372\\
514	0.00516767325099425\\
515	0.00509469267449204\\
516	0.00502124230610663\\
517	0.00494757317034403\\
518	0.004874013063796\\
519	0.00480099012756452\\
520	0.00472906273088725\\
521	0.00465895572787164\\
522	0.00459160707580159\\
523	0.00452822708277632\\
524	0.00447036487618838\\
525	0.0044184375290259\\
526	0.00436587397357472\\
527	0.0043126885610599\\
528	0.00425885476479472\\
529	0.00420427192507372\\
530	0.00414892816415185\\
531	0.00409285714575569\\
532	0.00403608410604617\\
533	0.00397861679527939\\
534	0.00392043183348241\\
535	0.0038614544566571\\
536	0.00380152811035017\\
537	0.00374037288718971\\
538	0.00367753023728315\\
539	0.00361247081341678\\
540	0.00354512799959542\\
541	0.00347543947168619\\
542	0.00340334641225652\\
543	0.0033287955821686\\
544	0.00325174240432155\\
545	0.00317215630959154\\
546	0.00309002880541974\\
547	0.00300538262148774\\
548	0.00291828382082769\\
549	0.00282884679169599\\
550	0.00273717375698181\\
551	0.00264351117793987\\
552	0.0025483095260621\\
553	0.00245231860330577\\
554	0.00235722079604287\\
555	0.00226672926843008\\
556	0.00218134599997664\\
557	0.00210146237082903\\
558	0.00202732470075\\
559	0.00195859464071399\\
560	0.00189147544706247\\
561	0.0018257439348815\\
562	0.00176104830254656\\
563	0.00169700040817372\\
564	0.0016334359172862\\
565	0.00157013355595559\\
566	0.0015069742191685\\
567	0.00144382934319225\\
568	0.00138100221627712\\
569	0.00131954993805358\\
570	0.0012596888460467\\
571	0.00120169720130309\\
572	0.00114504677832975\\
573	0.00108874011920619\\
574	0.00103277243214585\\
575	0.00097811364792387\\
576	0.000925437814226222\\
577	0.0008749039059792\\
578	0.000825214412511029\\
579	0.000776118395576129\\
580	0.000727556856375548\\
581	0.000679739350145835\\
582	0.00063299572419274\\
583	0.000587101587693234\\
584	0.000541705757972209\\
585	0.000496801587980551\\
586	0.000452422511210272\\
587	0.000408600951836586\\
588	0.000365362590765301\\
589	0.000322721662279121\\
590	0.000280677076068218\\
591	0.000239312865627478\\
592	0.00019878864496184\\
593	0.000159293651685586\\
594	0.000121079888727804\\
595	8.45520570083909e-05\\
596	5.0509214868037e-05\\
597	2.07908715710836e-05\\
598	0\\
599	0\\
600	0\\
};
\addplot [color=blue!75!mycolor7,solid,forget plot]
  table[row sep=crcr]{%
1	0.00983598579693763\\
2	0.00983598411159307\\
3	0.00983598239790594\\
4	0.00983598065539993\\
5	0.00983597888359073\\
6	0.00983597708198586\\
7	0.00983597525008457\\
8	0.00983597338737765\\
9	0.00983597149334735\\
10	0.00983596956746719\\
11	0.00983596760920181\\
12	0.00983596561800683\\
13	0.00983596359332872\\
14	0.00983596153460462\\
15	0.00983595944126217\\
16	0.00983595731271935\\
17	0.00983595514838439\\
18	0.00983595294765546\\
19	0.00983595070992064\\
20	0.00983594843455768\\
21	0.00983594612093382\\
22	0.00983594376840563\\
23	0.00983594137631882\\
24	0.00983593894400807\\
25	0.00983593647079682\\
26	0.00983593395599707\\
27	0.00983593139890923\\
28	0.00983592879882189\\
29	0.0098359261550116\\
30	0.00983592346674269\\
31	0.00983592073326708\\
32	0.00983591795382402\\
33	0.0098359151276399\\
34	0.00983591225392803\\
35	0.0098359093318884\\
36	0.00983590636070748\\
37	0.00983590333955796\\
38	0.00983590026759852\\
39	0.00983589714397359\\
40	0.00983589396781312\\
41	0.00983589073823231\\
42	0.00983588745433135\\
43	0.0098358841151952\\
44	0.00983588071989328\\
45	0.00983587726747922\\
46	0.00983587375699062\\
47	0.0098358701874487\\
48	0.0098358665578581\\
49	0.00983586286720652\\
50	0.00983585911446448\\
51	0.00983585529858498\\
52	0.00983585141850324\\
53	0.00983584747313635\\
54	0.00983584346138297\\
55	0.00983583938212302\\
56	0.00983583523421735\\
57	0.00983583101650739\\
58	0.00983582672781482\\
59	0.00983582236694124\\
60	0.0098358179326678\\
61	0.00983581342375487\\
62	0.00983580883894161\\
63	0.00983580417694569\\
64	0.00983579943646286\\
65	0.00983579461616655\\
66	0.00983578971470752\\
67	0.00983578473071342\\
68	0.00983577966278844\\
69	0.00983577450951282\\
70	0.00983576926944251\\
71	0.00983576394110868\\
72	0.00983575852301727\\
73	0.00983575301364863\\
74	0.00983574741145698\\
75	0.00983574171486998\\
76	0.00983573592228828\\
77	0.00983573003208498\\
78	0.0098357240426052\\
79	0.00983571795216557\\
80	0.00983571175905367\\
81	0.00983570546152761\\
82	0.00983569905781539\\
83	0.00983569254611445\\
84	0.00983568592459109\\
85	0.0098356791913799\\
86	0.00983567234458321\\
87	0.00983566538227052\\
88	0.00983565830247785\\
89	0.00983565110320724\\
90	0.00983564378242602\\
91	0.00983563633806631\\
92	0.00983562876802426\\
93	0.00983562107015949\\
94	0.00983561324229438\\
95	0.00983560528221342\\
96	0.00983559718766253\\
97	0.00983558895634832\\
98	0.00983558058593741\\
99	0.00983557207405573\\
100	0.00983556341828769\\
101	0.00983555461617552\\
102	0.00983554566521845\\
103	0.00983553656287193\\
104	0.00983552730654685\\
105	0.00983551789360869\\
106	0.00983550832137675\\
107	0.00983549858712324\\
108	0.00983548868807246\\
109	0.00983547862139991\\
110	0.0098354683842314\\
111	0.00983545797364213\\
112	0.00983544738665577\\
113	0.00983543662024351\\
114	0.00983542567132311\\
115	0.00983541453675792\\
116	0.00983540321335585\\
117	0.0098353916978684\\
118	0.00983537998698958\\
119	0.00983536807735489\\
120	0.00983535596554026\\
121	0.00983534364806088\\
122	0.0098353311213702\\
123	0.00983531838185871\\
124	0.00983530542585284\\
125	0.00983529224961377\\
126	0.00983527884933623\\
127	0.00983526522114734\\
128	0.00983525136110529\\
129	0.00983523726519819\\
130	0.00983522292934274\\
131	0.00983520834938293\\
132	0.00983519352108877\\
133	0.00983517844015494\\
134	0.0098351631021994\\
135	0.00983514750276208\\
136	0.00983513163730345\\
137	0.00983511550120311\\
138	0.00983509908975833\\
139	0.00983508239818266\\
140	0.00983506542160439\\
141	0.00983504815506511\\
142	0.00983503059351812\\
143	0.009835012731827\\
144	0.00983499456476397\\
145	0.00983497608700835\\
146	0.009834957293145\\
147	0.00983493817766266\\
148	0.00983491873495236\\
149	0.00983489895930575\\
150	0.0098348788449135\\
151	0.00983485838586352\\
152	0.00983483757613938\\
153	0.00983481640961852\\
154	0.00983479488007056\\
155	0.00983477298115552\\
156	0.00983475070642213\\
157	0.00983472804930598\\
158	0.00983470500312777\\
159	0.00983468156109151\\
160	0.00983465771628264\\
161	0.00983463346166626\\
162	0.00983460879008519\\
163	0.00983458369425816\\
164	0.00983455816677784\\
165	0.00983453220010894\\
166	0.00983450578658627\\
167	0.00983447891841271\\
168	0.00983445158765724\\
169	0.00983442378625285\\
170	0.00983439550599447\\
171	0.00983436673853685\\
172	0.00983433747539234\\
173	0.00983430770792866\\
174	0.00983427742736665\\
175	0.00983424662477785\\
176	0.00983421529108211\\
177	0.00983418341704506\\
178	0.00983415099327551\\
179	0.00983411801022278\\
180	0.00983408445817392\\
181	0.00983405032725079\\
182	0.0098340156074071\\
183	0.00983398028842531\\
184	0.00983394435991335\\
185	0.00983390781130135\\
186	0.00983387063183816\\
187	0.00983383281058781\\
188	0.00983379433642582\\
189	0.00983375519803545\\
190	0.00983371538390387\\
191	0.00983367488231827\\
192	0.00983363368136185\\
193	0.00983359176890989\\
194	0.00983354913262567\\
195	0.00983350575995649\\
196	0.00983346163812967\\
197	0.00983341675414844\\
198	0.00983337109478805\\
199	0.00983332464659158\\
200	0.0098332773958658\\
201	0.00983322932867695\\
202	0.00983318043084641\\
203	0.00983313068794629\\
204	0.00983308008529495\\
205	0.00983302860795242\\
206	0.00983297624071581\\
207	0.00983292296811446\\
208	0.00983286877440525\\
209	0.00983281364356758\\
210	0.00983275755929841\\
211	0.00983270050500715\\
212	0.00983264246381048\\
213	0.00983258341852703\\
214	0.00983252335167203\\
215	0.00983246224545177\\
216	0.00983240008175807\\
217	0.00983233684216253\\
218	0.00983227250791074\\
219	0.00983220705991638\\
220	0.00983214047875521\\
221	0.0098320727446589\\
222	0.00983200383750881\\
223	0.00983193373682962\\
224	0.00983186242178282\\
225	0.00983178987116015\\
226	0.00983171606337684\\
227	0.00983164097646476\\
228	0.00983156458806543\\
229	0.00983148687542292\\
230	0.00983140781537661\\
231	0.00983132738435378\\
232	0.0098312455583621\\
233	0.009831162312982\\
234	0.00983107762335881\\
235	0.00983099146419485\\
236	0.00983090380974132\\
237	0.00983081463379006\\
238	0.0098307239096651\\
239	0.00983063161021417\\
240	0.00983053770779993\\
241	0.00983044217429109\\
242	0.00983034498105338\\
243	0.00983024609894031\\
244	0.0098301454982838\\
245	0.0098300431488846\\
246	0.00982993902000258\\
247	0.00982983308034675\\
248	0.00982972529806523\\
249	0.00982961564073488\\
250	0.00982950407535088\\
251	0.00982939056831603\\
252	0.00982927508542986\\
253	0.00982915759187762\\
254	0.00982903805221893\\
255	0.00982891643037636\\
256	0.00982879268962371\\
257	0.00982866679257412\\
258	0.00982853870116795\\
259	0.00982840837666048\\
260	0.00982827577960929\\
261	0.00982814086986154\\
262	0.00982800360654092\\
263	0.00982786394803442\\
264	0.00982772185197886\\
265	0.00982757727524719\\
266	0.0098274301739345\\
267	0.00982728050334386\\
268	0.00982712821797182\\
269	0.00982697327149379\\
270	0.00982681561674903\\
271	0.00982665520572546\\
272	0.00982649198954422\\
273	0.00982632591844391\\
274	0.00982615694176462\\
275	0.00982598500793166\\
276	0.00982581006443908\\
277	0.0098256320578328\\
278	0.00982545093369359\\
279	0.00982526663661972\\
280	0.0098250791102093\\
281	0.00982488829704241\\
282	0.00982469413866293\\
283	0.00982449657556001\\
284	0.00982429554714937\\
285	0.00982409099175425\\
286	0.00982388284658605\\
287	0.00982367104772479\\
288	0.00982345553009914\\
289	0.0098232362274663\\
290	0.00982301307239145\\
291	0.00982278599622708\\
292	0.00982255492909187\\
293	0.00982231979984942\\
294	0.00982208053608657\\
295	0.00982183706409158\\
296	0.00982158930883191\\
297	0.00982133719393176\\
298	0.00982108064164936\\
299	0.00982081957285399\\
300	0.00982055390700268\\
301	0.00982028356211668\\
302	0.0098200084547577\\
303	0.00981972850000382\\
304	0.00981944361142525\\
305	0.00981915370105972\\
306	0.00981885867938778\\
307	0.00981855845530773\\
308	0.0098182529361104\\
309	0.00981794202745373\\
310	0.00981762563333706\\
311	0.00981730365607531\\
312	0.00981697599627289\\
313	0.0098166425527975\\
314	0.00981630322275366\\
315	0.00981595790145617\\
316	0.00981560648240336\\
317	0.00981524885725017\\
318	0.00981488491578119\\
319	0.00981451454588345\\
320	0.00981413763351916\\
321	0.00981375406269833\\
322	0.0098133637154513\\
323	0.00981296647180112\\
324	0.00981256220973592\\
325	0.00981215080518117\\
326	0.00981173213197181\\
327	0.00981130606182442\\
328	0.0098108724643092\\
329	0.00981043120682202\\
330	0.00980998215455628\\
331	0.00980952517047473\\
332	0.00980906011528128\\
333	0.00980858684739264\\
334	0.0098081052229099\\
335	0.00980761509558998\\
336	0.00980711631681693\\
337	0.00980660873557315\\
338	0.00980609219841025\\
339	0.00980556654941983\\
340	0.00980503163020391\\
341	0.00980448727984503\\
342	0.00980393333487598\\
343	0.00980336962924906\\
344	0.00980279599430481\\
345	0.0098022122587401\\
346	0.00980161824857549\\
347	0.00980101378712182\\
348	0.00980039869494577\\
349	0.00979977278983434\\
350	0.00979913588675813\\
351	0.00979848779783319\\
352	0.00979782833228123\\
353	0.00979715729638818\\
354	0.00979647449346058\\
355	0.00979577972377991\\
356	0.00979507278455434\\
357	0.00979435346986779\\
358	0.00979362157062594\\
359	0.009792876874499\\
360	0.00979211916586069\\
361	0.00979134822572343\\
362	0.00979056383166907\\
363	0.00978976575777495\\
364	0.00978895377453493\\
365	0.00978812764877493\\
366	0.00978728714356253\\
367	0.00978643201811032\\
368	0.0097855620276725\\
369	0.00978467692343428\\
370	0.00978377645239367\\
371	0.00978286035723521\\
372	0.00978192837619511\\
373	0.00978098024291748\\
374	0.00978001568630101\\
375	0.00977903443033571\\
376	0.00977803619392928\\
377	0.00977702069072246\\
378	0.00977598762889288\\
379	0.00977493671094685\\
380	0.00977386763349839\\
381	0.00977278008703469\\
382	0.00977167375566731\\
383	0.00977054831686783\\
384	0.0097694034411868\\
385	0.00976823879195447\\
386	0.00976705402496097\\
387	0.00976584878811385\\
388	0.00976462272106999\\
389	0.00976337545483856\\
390	0.00976210661134679\\
391	0.00976081580296672\\
392	0.00975950263199832\\
393	0.00975816669010216\\
394	0.00975680755767175\\
395	0.00975542480313566\\
396	0.009754017982191\\
397	0.00975258663695195\\
398	0.00975113029498769\\
399	0.0097496484682359\\
400	0.00974814065177603\\
401	0.0097466063224438\\
402	0.0097450449372837\\
403	0.00974345593202622\\
404	0.00974183872085087\\
405	0.00974019270298737\\
406	0.00973851726202865\\
407	0.00973681173897762\\
408	0.00973507542762574\\
409	0.00973330759298671\\
410	0.00973150746984965\\
411	0.00972967426148173\\
412	0.00972780713863298\\
413	0.00972590523909215\\
414	0.00972396766814842\\
415	0.0097219935001699\\
416	0.00971998177979321\\
417	0.00971793151220656\\
418	0.00971584159280819\\
419	0.00971371092465262\\
420	0.009711538524179\\
421	0.009709323390324\\
422	0.00970706448206493\\
423	0.00970476071462334\\
424	0.00970241095497063\\
425	0.00970001401646347\\
426	0.00969756865238058\\
427	0.00969507354811213\\
428	0.0096925273117392\\
429	0.00968992846244233\\
430	0.00968727541617079\\
431	0.00968456646808999\\
432	0.00968179977106074\\
433	0.00967897330958293\\
434	0.00967608487168244\\
435	0.00967313202846759\\
436	0.00967011215823784\\
437	0.00966702265224127\\
438	0.00966386179836363\\
439	0.00966063213953145\\
440	0.0096581438166461\\
441	0.00965618471398823\\
442	0.00965419083958201\\
443	0.0096521612180493\\
444	0.0096500947584985\\
445	0.00964799022959298\\
446	0.00964584622868649\\
447	0.00964366114348987\\
448	0.00964143310430262\\
449	0.00963915992428517\\
450	0.00963683902452495\\
451	0.00963446733971605\\
452	0.00963204119906393\\
453	0.00962955617537492\\
454	0.00962700689239675\\
455	0.00962438677223191\\
456	0.00962168769901288\\
457	0.00961889955818363\\
458	0.00961600960843108\\
459	0.00961300116654827\\
460	0.00960985036340646\\
461	0.0096061970184462\\
462	0.00956347884637286\\
463	0.00951990473209833\\
464	0.00947545229190016\\
465	0.00943009062697348\\
466	0.00938379558812048\\
467	0.00933654208400312\\
468	0.00928830404964439\\
469	0.00923905444750061\\
470	0.00918876536193582\\
471	0.00913740851543912\\
472	0.00908495400076351\\
473	0.00903136941236922\\
474	0.00897662092565435\\
475	0.00892067309233651\\
476	0.00886348839234452\\
477	0.00880502621294514\\
478	0.00874523943928828\\
479	0.00868408740739547\\
480	0.00862152976468595\\
481	0.00855752423148799\\
482	0.00849202612298176\\
483	0.00842498784328819\\
484	0.00835635834929183\\
485	0.00828608224596591\\
486	0.00821409847920065\\
487	0.0081403390213053\\
488	0.0080647285432332\\
489	0.00798718641478394\\
490	0.00790762535960176\\
491	0.0078259505817493\\
492	0.00774205873614955\\
493	0.0076558367082527\\
494	0.00756716015957302\\
495	0.00747589178508248\\
496	0.00738187921625173\\
497	0.00728495249213396\\
498	0.00718492101848958\\
499	0.00708156996690433\\
500	0.00697465620733147\\
501	0.00686390433814216\\
502	0.0067490048182002\\
503	0.00662962124945524\\
504	0.00654282147243754\\
505	0.00650416080218482\\
506	0.00646417271853324\\
507	0.00642286895962343\\
508	0.006380290798672\\
509	0.00633649164171396\\
510	0.00629140003101015\\
511	0.00624491050299828\\
512	0.00619690040501686\\
513	0.00614722597860324\\
514	0.00609571778460128\\
515	0.00604217500531807\\
516	0.0059863583248483\\
517	0.00592798101679594\\
518	0.00586669774697961\\
519	0.00580209039815476\\
520	0.00573365010006878\\
521	0.00566075463826947\\
522	0.0055826405375136\\
523	0.00549837040510512\\
524	0.00540680211042718\\
525	0.00530801300573983\\
526	0.00520877949944434\\
527	0.00510955840050037\\
528	0.0050109498653376\\
529	0.00491374327105498\\
530	0.00481743395479156\\
531	0.00472041401216693\\
532	0.0046230730277967\\
533	0.00452592907234172\\
534	0.00442966844522136\\
535	0.00433519904004857\\
536	0.00424372441285738\\
537	0.00415679393801028\\
538	0.00407632456672\\
539	0.00400303780149818\\
540	0.00392833741499244\\
541	0.00385228073333916\\
542	0.0037749313033303\\
543	0.00369635469524612\\
544	0.00361661131706036\\
545	0.00353574439772463\\
546	0.00345377407636575\\
547	0.0033706573271108\\
548	0.00328623817180589\\
549	0.00320020881435913\\
550	0.00311226168746383\\
551	0.00302174557633384\\
552	0.0029285721704536\\
553	0.00283284845112247\\
554	0.00273472165432939\\
555	0.0026343666970143\\
556	0.00253200294034252\\
557	0.00242802872893287\\
558	0.00232311566314357\\
559	0.00221851657094213\\
560	0.00211831907397233\\
561	0.00202299700719578\\
562	0.00193303203249592\\
563	0.00184882022133786\\
564	0.00177042797505062\\
565	0.00169520130901441\\
566	0.00162172309513774\\
567	0.00154972673714696\\
568	0.00147890019079677\\
569	0.00140915704425968\\
570	0.00133996649982656\\
571	0.00127121916543033\\
572	0.00120361310685485\\
573	0.00113828975820201\\
574	0.0010750352614193\\
575	0.00101406573453819\\
576	0.00095501450658787\\
577	0.000896782871369601\\
578	0.000840716170556225\\
579	0.0007870993751659\\
580	0.000735978059148706\\
581	0.000686067843775887\\
582	0.000637128970025702\\
583	0.000589268610066713\\
584	0.000542736806562125\\
585	0.000497389550407257\\
586	0.000452751633483821\\
587	0.000408776510568709\\
588	0.000365448460457315\\
589	0.000322758538660347\\
590	0.000280690270321202\\
591	0.000239316379010231\\
592	0.000198789174864552\\
593	0.000159293651685586\\
594	0.000121079888727805\\
595	8.45520570083912e-05\\
596	5.05092148680373e-05\\
597	2.07908715710836e-05\\
598	0\\
599	0\\
600	0\\
};
\addplot [color=blue!80!mycolor9,solid,forget plot]
  table[row sep=crcr]{%
1	0.00995024965342724\\
2	0.0099502494926092\\
3	0.00995024932908664\\
4	0.00995024916281409\\
5	0.0099502489937453\\
6	0.00995024882183327\\
7	0.00995024864703018\\
8	0.00995024846928742\\
9	0.00995024828855554\\
10	0.00995024810478429\\
11	0.00995024791792253\\
12	0.00995024772791829\\
13	0.00995024753471872\\
14	0.00995024733827005\\
15	0.00995024713851763\\
16	0.00995024693540588\\
17	0.00995024672887828\\
18	0.00995024651887735\\
19	0.00995024630534464\\
20	0.00995024608822072\\
21	0.00995024586744516\\
22	0.00995024564295647\\
23	0.00995024541469219\\
24	0.00995024518258872\\
25	0.00995024494658145\\
26	0.00995024470660464\\
27	0.00995024446259145\\
28	0.00995024421447391\\
29	0.0099502439621829\\
30	0.00995024370564812\\
31	0.00995024344479807\\
32	0.00995024317956006\\
33	0.00995024290986016\\
34	0.00995024263562317\\
35	0.00995024235677264\\
36	0.00995024207323081\\
37	0.00995024178491859\\
38	0.00995024149175555\\
39	0.0099502411936599\\
40	0.00995024089054847\\
41	0.00995024058233666\\
42	0.00995024026893843\\
43	0.00995023995026628\\
44	0.00995023962623122\\
45	0.00995023929674274\\
46	0.00995023896170881\\
47	0.00995023862103581\\
48	0.00995023827462851\\
49	0.0099502379223901\\
50	0.00995023756422208\\
51	0.00995023720002429\\
52	0.00995023682969484\\
53	0.00995023645313012\\
54	0.00995023607022472\\
55	0.00995023568087147\\
56	0.00995023528496134\\
57	0.00995023488238342\\
58	0.00995023447302493\\
59	0.00995023405677114\\
60	0.00995023363350536\\
61	0.0099502332031089\\
62	0.00995023276546102\\
63	0.00995023232043893\\
64	0.00995023186791771\\
65	0.00995023140777031\\
66	0.00995023093986748\\
67	0.00995023046407775\\
68	0.00995022998026739\\
69	0.00995022948830038\\
70	0.00995022898803832\\
71	0.00995022847934045\\
72	0.0099502279620636\\
73	0.00995022743606207\\
74	0.0099502269011877\\
75	0.00995022635728974\\
76	0.00995022580421483\\
77	0.00995022524180698\\
78	0.00995022466990747\\
79	0.00995022408835485\\
80	0.00995022349698486\\
81	0.00995022289563039\\
82	0.00995022228412143\\
83	0.00995022166228502\\
84	0.00995022102994519\\
85	0.00995022038692289\\
86	0.00995021973303601\\
87	0.00995021906809919\\
88	0.00995021839192389\\
89	0.00995021770431828\\
90	0.00995021700508715\\
91	0.0099502162940319\\
92	0.00995021557095047\\
93	0.00995021483563724\\
94	0.00995021408788301\\
95	0.0099502133274749\\
96	0.00995021255419632\\
97	0.00995021176782685\\
98	0.00995021096814223\\
99	0.00995021015491425\\
100	0.00995020932791068\\
101	0.00995020848689522\\
102	0.0099502076316274\\
103	0.00995020676186253\\
104	0.00995020587735159\\
105	0.00995020497784118\\
106	0.00995020406307343\\
107	0.00995020313278591\\
108	0.00995020218671155\\
109	0.00995020122457859\\
110	0.00995020024611043\\
111	0.00995019925102558\\
112	0.00995019823903758\\
113	0.00995019720985489\\
114	0.00995019616318081\\
115	0.00995019509871337\\
116	0.00995019401614525\\
117	0.00995019291516367\\
118	0.00995019179545031\\
119	0.00995019065668119\\
120	0.00995018949852657\\
121	0.00995018832065087\\
122	0.00995018712271253\\
123	0.00995018590436392\\
124	0.00995018466525123\\
125	0.00995018340501437\\
126	0.00995018212328684\\
127	0.00995018081969562\\
128	0.00995017949386106\\
129	0.00995017814539676\\
130	0.00995017677390944\\
131	0.00995017537899884\\
132	0.00995017396025759\\
133	0.00995017251727106\\
134	0.00995017104961728\\
135	0.00995016955686677\\
136	0.00995016803858243\\
137	0.00995016649431942\\
138	0.00995016492362499\\
139	0.00995016332603836\\
140	0.00995016170109063\\
141	0.00995016004830457\\
142	0.00995015836719451\\
143	0.00995015665726621\\
144	0.00995015491801671\\
145	0.00995015314893415\\
146	0.0099501513494977\\
147	0.00995014951917732\\
148	0.00995014765743366\\
149	0.00995014576371792\\
150	0.00995014383747166\\
151	0.00995014187812665\\
152	0.00995013988510474\\
153	0.00995013785781765\\
154	0.00995013579566686\\
155	0.00995013369804341\\
156	0.00995013156432773\\
157	0.00995012939388952\\
158	0.0099501271860875\\
159	0.00995012494026931\\
160	0.00995012265577127\\
161	0.00995012033191825\\
162	0.00995011796802346\\
163	0.00995011556338825\\
164	0.00995011311730196\\
165	0.00995011062904168\\
166	0.00995010809787208\\
167	0.00995010552304521\\
168	0.00995010290380026\\
169	0.00995010023936337\\
170	0.00995009752894742\\
171	0.00995009477175178\\
172	0.00995009196696209\\
173	0.00995008911375003\\
174	0.00995008621127306\\
175	0.00995008325867418\\
176	0.00995008025508167\\
177	0.00995007719960882\\
178	0.00995007409135364\\
179	0.00995007092939862\\
180	0.00995006771281036\\
181	0.00995006444063938\\
182	0.00995006111191971\\
183	0.00995005772566862\\
184	0.0099500542808863\\
185	0.00995005077655552\\
186	0.00995004721164129\\
187	0.00995004358509052\\
188	0.00995003989583166\\
189	0.00995003614277436\\
190	0.00995003232480913\\
191	0.00995002844080692\\
192	0.0099500244896188\\
193	0.0099500204700756\\
194	0.0099500163809875\\
195	0.00995001222114367\\
196	0.00995000798931191\\
197	0.00995000368423822\\
198	0.00994999930464644\\
199	0.00994999484923785\\
200	0.0099499903166907\\
201	0.00994998570565989\\
202	0.00994998101477646\\
203	0.00994997624264718\\
204	0.00994997138785415\\
205	0.00994996644895427\\
206	0.00994996142447887\\
207	0.00994995631293317\\
208	0.00994995111279585\\
209	0.00994994582251851\\
210	0.00994994044052526\\
211	0.00994993496521215\\
212	0.00994992939494666\\
213	0.0099499237280672\\
214	0.00994991796288258\\
215	0.00994991209767143\\
216	0.0099499061306817\\
217	0.00994990006013004\\
218	0.00994989388420129\\
219	0.0099498876010478\\
220	0.00994988120878893\\
221	0.0099498747055104\\
222	0.00994986808926366\\
223	0.00994986135806527\\
224	0.00994985450989627\\
225	0.00994984754270148\\
226	0.00994984045438891\\
227	0.00994983324282898\\
228	0.00994982590585392\\
229	0.00994981844125699\\
230	0.00994981084679182\\
231	0.00994980312017161\\
232	0.00994979525906845\\
233	0.00994978726111251\\
234	0.00994977912389129\\
235	0.00994977084494881\\
236	0.00994976242178481\\
237	0.00994975385185393\\
238	0.00994974513256489\\
239	0.00994973626127961\\
240	0.00994972723531233\\
241	0.0099497180519288\\
242	0.00994970870834526\\
243	0.00994969920172764\\
244	0.00994968952919054\\
245	0.00994967968779633\\
246	0.00994966967455414\\
247	0.00994965948641888\\
248	0.00994964912029027\\
249	0.00994963857301175\\
250	0.00994962784136949\\
251	0.00994961692209132\\
252	0.00994960581184558\\
253	0.00994959450724013\\
254	0.00994958300482111\\
255	0.00994957130107188\\
256	0.00994955939241183\\
257	0.00994954727519518\\
258	0.00994953494570976\\
259	0.00994952240017586\\
260	0.00994950963474487\\
261	0.00994949664549808\\
262	0.00994948342844539\\
263	0.00994946997952393\\
264	0.0099494562945968\\
265	0.00994944236945163\\
266	0.00994942819979924\\
267	0.00994941378127223\\
268	0.00994939910942352\\
269	0.0099493841797249\\
270	0.00994936898756556\\
271	0.00994935352825057\\
272	0.00994933779699935\\
273	0.00994932178894409\\
274	0.00994930549912823\\
275	0.00994928892250477\\
276	0.0099492720539347\\
277	0.00994925488818531\\
278	0.00994923741992849\\
279	0.00994921964373906\\
280	0.00994920155409301\\
281	0.00994918314536572\\
282	0.00994916441183021\\
283	0.00994914534765528\\
284	0.00994912594690369\\
285	0.0099491062035303\\
286	0.00994908611138015\\
287	0.00994906566418656\\
288	0.00994904485556918\\
289	0.00994902367903201\\
290	0.00994900212796138\\
291	0.00994898019562399\\
292	0.00994895787516478\\
293	0.0099489351596049\\
294	0.00994891204183962\\
295	0.00994888851463616\\
296	0.00994886457063157\\
297	0.00994884020233053\\
298	0.00994881540210321\\
299	0.00994879016218295\\
300	0.00994876447466411\\
301	0.00994873833149973\\
302	0.00994871172449927\\
303	0.00994868464532629\\
304	0.0099486570854961\\
305	0.00994862903637343\\
306	0.00994860048917002\\
307	0.00994857143494221\\
308	0.00994854186458856\\
309	0.00994851176884737\\
310	0.00994848113829424\\
311	0.00994844996333959\\
312	0.0099484182342261\\
313	0.00994838594102629\\
314	0.00994835307363989\\
315	0.00994831962179133\\
316	0.00994828557502712\\
317	0.0099482509227133\\
318	0.00994821565403277\\
319	0.00994817975798267\\
320	0.00994814322337174\\
321	0.00994810603881756\\
322	0.00994806819274393\\
323	0.00994802967337804\\
324	0.0099479904687478\\
325	0.00994795056667894\\
326	0.00994790995479227\\
327	0.00994786862050074\\
328	0.00994782655100661\\
329	0.00994778373329841\\
330	0.00994774015414801\\
331	0.00994769580010754\\
332	0.00994765065750626\\
333	0.0099476047124474\\
334	0.00994755795080488\\
335	0.00994751035821997\\
336	0.00994746192009786\\
337	0.00994741262160411\\
338	0.00994736244766097\\
339	0.00994731138294361\\
340	0.00994725941187616\\
341	0.00994720651862764\\
342	0.00994715268710765\\
343	0.00994709790096189\\
344	0.00994704214356749\\
345	0.00994698539802803\\
346	0.0099469276471684\\
347	0.00994686887352922\\
348	0.00994680905936108\\
349	0.00994674818661835\\
350	0.00994668623695262\\
351	0.00994662319170575\\
352	0.00994655903190239\\
353	0.00994649373824212\\
354	0.00994642729109096\\
355	0.00994635967047233\\
356	0.00994629085605737\\
357	0.00994622082715467\\
358	0.00994614956269912\\
359	0.00994607704124017\\
360	0.0099460032409291\\
361	0.00994592813950552\\
362	0.00994585171428288\\
363	0.00994577394213301\\
364	0.00994569479946957\\
365	0.00994561426223041\\
366	0.00994553230585879\\
367	0.00994544890528327\\
368	0.00994536403489642\\
369	0.00994527766853207\\
370	0.0099451897794412\\
371	0.00994510034026634\\
372	0.00994500932301441\\
373	0.0099449166990279\\
374	0.00994482243895451\\
375	0.00994472651271482\\
376	0.00994462888946838\\
377	0.00994452953757768\\
378	0.0099444284245702\\
379	0.00994432551709836\\
380	0.00994422078089722\\
381	0.00994411418073977\\
382	0.00994400568038974\\
383	0.00994389524255166\\
384	0.00994378282881784\\
385	0.00994366839961223\\
386	0.00994355191413059\\
387	0.00994343333027662\\
388	0.00994331260459367\\
389	0.00994318969219126\\
390	0.0099430645466659\\
391	0.00994293712001525\\
392	0.00994280736254448\\
393	0.00994267522276367\\
394	0.0099425406472743\\
395	0.00994240358064276\\
396	0.00994226396525818\\
397	0.00994212174117097\\
398	0.00994197684590766\\
399	0.00994182921425599\\
400	0.00994167877801249\\
401	0.00994152546568373\\
402	0.00994136920213928\\
403	0.00994120990824239\\
404	0.00994104750053987\\
405	0.00994088189096524\\
406	0.00994071298604871\\
407	0.00994054068635598\\
408	0.00994036488696237\\
409	0.00994018547708384\\
410	0.0099400023397103\\
411	0.00993981535125655\\
412	0.00993962438125417\\
413	0.00993942929211299\\
414	0.00993922993896757\\
415	0.0099390261695445\\
416	0.00993881782376216\\
417	0.00993860473254393\\
418	0.00993838671728155\\
419	0.00993816359336768\\
420	0.00993793516339354\\
421	0.00993770121470024\\
422	0.00993746151751359\\
423	0.00993721582279131\\
424	0.00993696385973073\\
425	0.00993670533287484\\
426	0.00993643991874117\\
427	0.00993616726188308\\
428	0.00993588697026948\\
429	0.00993559860983968\\
430	0.00993530169804049\\
431	0.00993499569604213\\
432	0.00993467999905065\\
433	0.00993435392334768\\
434	0.0099340166861986\\
435	0.00993366736666794\\
436	0.00993330480829898\\
437	0.00993292733222369\\
438	0.00993253180907329\\
439	0.00993211051292307\\
440	0.00993090909648561\\
441	0.00992913499893833\\
442	0.00992731780245534\\
443	0.00992545602322119\\
444	0.0099235481080655\\
445	0.00992159243265715\\
446	0.00991958730045363\\
447	0.00991753094274116\\
448	0.00991542152021376\\
449	0.00991325712668699\\
450	0.00991103579573368\\
451	0.00990875551127558\\
452	0.00990641422346764\\
453	0.00990400987154726\\
454	0.00990154041558171\\
455	0.0098990038789983\\
456	0.00989639840298146\\
457	0.00989372230993868\\
458	0.00989097416419154\\
459	0.00988815280269732\\
460	0.00988525731382974\\
461	0.00988227107081666\\
462	0.00987726586402313\\
463	0.00987217644234873\\
464	0.00986701232349152\\
465	0.00986207582144072\\
466	0.00985706019359379\\
467	0.0098519638048553\\
468	0.00984678493520374\\
469	0.00984152177448637\\
470	0.00983617244261739\\
471	0.00983073497250655\\
472	0.00982520722452214\\
473	0.00981958689007658\\
474	0.00981387154405269\\
475	0.00980805864121547\\
476	0.00980214548286862\\
477	0.00979612902455971\\
478	0.00979000614332297\\
479	0.0097837739541957\\
480	0.00977742945990819\\
481	0.00977096943423777\\
482	0.00976439038006545\\
483	0.00975768849293456\\
484	0.00975085960756883\\
485	0.00974389912357521\\
486	0.00973680192042395\\
487	0.00972956229998946\\
488	0.00972217402843945\\
489	0.00971463020511259\\
490	0.0097069231534983\\
491	0.00969904429054335\\
492	0.00969098396942824\\
493	0.00968273128975784\\
494	0.00967427386756375\\
495	0.00966559755544607\\
496	0.00965668610019438\\
497	0.00964752072027568\\
498	0.00963807957556147\\
499	0.00962833707685462\\
500	0.00961826291146477\\
501	0.00960782043746308\\
502	0.00959696336322506\\
503	0.00958562636004133\\
504	0.00953824077459935\\
505	0.0094428667253592\\
506	0.00934773776944827\\
507	0.00925021459251212\\
508	0.00915018834629194\\
509	0.00904753857441189\\
510	0.00894213426567151\\
511	0.00883382755758078\\
512	0.0087224673322712\\
513	0.00860790122505926\\
514	0.00848996516568321\\
515	0.00836848236237575\\
516	0.0082432622479476\\
517	0.00811409941774295\\
518	0.00798077260987465\\
519	0.00784304381493639\\
520	0.00770065766270751\\
521	0.00755334132295126\\
522	0.00740080524889721\\
523	0.00724274490892458\\
524	0.00707883679535215\\
525	0.00690873609418503\\
526	0.00673202972081162\\
527	0.00654799664263\\
528	0.00635577039827619\\
529	0.00615428802740013\\
530	0.00599822874844659\\
531	0.0059229785927858\\
532	0.00584393196698481\\
533	0.0057605063025082\\
534	0.00567197500609394\\
535	0.00557742595482749\\
536	0.00547570712171949\\
537	0.00536527523908269\\
538	0.00524427765188107\\
539	0.00511240149795213\\
540	0.00497826976213729\\
541	0.00484217428987451\\
542	0.00470452003423817\\
543	0.00456585911209548\\
544	0.00442693566006206\\
545	0.00428874727498958\\
546	0.00415262655058662\\
547	0.00402035366075448\\
548	0.00389417407788781\\
549	0.00377651184121573\\
550	0.00366528682586713\\
551	0.00356355257474437\\
552	0.00346211700310143\\
553	0.00335854913922197\\
554	0.00325299748522605\\
555	0.00314563583395675\\
556	0.00303666170785842\\
557	0.00292628271907932\\
558	0.00281469808724814\\
559	0.00270207149648971\\
560	0.00258855827107614\\
561	0.00247403834706246\\
562	0.00235818448456371\\
563	0.00224107956118409\\
564	0.00212364440435233\\
565	0.00200907458233774\\
566	0.00189926355595736\\
567	0.00179479801060058\\
568	0.00169626418907342\\
569	0.00160400293221254\\
570	0.00151783759986445\\
571	0.00143390318578877\\
572	0.00135203502647469\\
573	0.00127200640621873\\
574	0.00119387836966914\\
575	0.00111724688276997\\
576	0.0010422787258352\\
577	0.000970248500825921\\
578	0.000901376380147175\\
579	0.000835676581260966\\
580	0.000773155683964323\\
581	0.000713512969864073\\
582	0.000657039252267262\\
583	0.000603820761156831\\
584	0.000552908173124348\\
585	0.000503686775016135\\
586	0.0004564261874486\\
587	0.000410922180362966\\
588	0.000366669791782508\\
589	0.000323409636595769\\
590	0.000280998940489726\\
591	0.000239439664031353\\
592	0.000198826448456301\\
593	0.000159300178734201\\
594	0.000121079888727804\\
595	8.45520570083909e-05\\
596	5.05092148680373e-05\\
597	2.07908715710836e-05\\
598	0\\
599	0\\
600	0\\
};
\addplot [color=blue,solid,forget plot]
  table[row sep=crcr]{%
1	0.00996996602909277\\
2	0.00996996602247315\\
3	0.0099699660157422\\
4	0.00996996600889807\\
5	0.00996996600193883\\
6	0.00996996599486257\\
7	0.00996996598766729\\
8	0.00996996598035102\\
9	0.00996996597291171\\
10	0.00996996596534729\\
11	0.00996996595765566\\
12	0.00996996594983469\\
13	0.00996996594188218\\
14	0.00996996593379593\\
15	0.00996996592557369\\
16	0.00996996591721317\\
17	0.00996996590871205\\
18	0.00996996590006795\\
19	0.00996996589127847\\
20	0.00996996588234117\\
21	0.00996996587325356\\
22	0.0099699658640131\\
23	0.00996996585461723\\
24	0.00996996584506332\\
25	0.00996996583534872\\
26	0.00996996582547072\\
27	0.00996996581542656\\
28	0.00996996580521345\\
29	0.00996996579482855\\
30	0.00996996578426895\\
31	0.00996996577353172\\
32	0.00996996576261385\\
33	0.00996996575151232\\
34	0.00996996574022401\\
35	0.00996996572874579\\
36	0.00996996571707446\\
37	0.00996996570520674\\
38	0.00996996569313934\\
39	0.00996996568086888\\
40	0.00996996566839194\\
41	0.00996996565570504\\
42	0.00996996564280464\\
43	0.00996996562968712\\
44	0.00996996561634883\\
45	0.00996996560278604\\
46	0.00996996558899495\\
47	0.00996996557497172\\
48	0.00996996556071242\\
49	0.00996996554621307\\
50	0.0099699655314696\\
51	0.0099699655164779\\
52	0.00996996550123376\\
53	0.00996996548573293\\
54	0.00996996546997105\\
55	0.00996996545394371\\
56	0.00996996543764643\\
57	0.00996996542107463\\
58	0.00996996540422368\\
59	0.00996996538708884\\
60	0.00996996536966531\\
61	0.00996996535194819\\
62	0.00996996533393252\\
63	0.00996996531561324\\
64	0.0099699652969852\\
65	0.00996996527804317\\
66	0.00996996525878183\\
67	0.00996996523919575\\
68	0.00996996521927943\\
69	0.00996996519902727\\
70	0.00996996517843356\\
71	0.0099699651574925\\
72	0.0099699651361982\\
73	0.00996996511454465\\
74	0.00996996509252575\\
75	0.00996996507013529\\
76	0.00996996504736695\\
77	0.0099699650242143\\
78	0.0099699650006708\\
79	0.0099699649767298\\
80	0.00996996495238452\\
81	0.0099699649276281\\
82	0.00996996490245351\\
83	0.00996996487685363\\
84	0.00996996485082121\\
85	0.00996996482434888\\
86	0.00996996479742911\\
87	0.00996996477005428\\
88	0.00996996474221662\\
89	0.00996996471390822\\
90	0.00996996468512103\\
91	0.00996996465584687\\
92	0.00996996462607742\\
93	0.00996996459580418\\
94	0.00996996456501855\\
95	0.00996996453371174\\
96	0.00996996450187483\\
97	0.00996996446949874\\
98	0.00996996443657422\\
99	0.00996996440309185\\
100	0.00996996436904208\\
101	0.00996996433441515\\
102	0.00996996429920115\\
103	0.00996996426339\\
104	0.00996996422697143\\
105	0.009969964189935\\
106	0.00996996415227006\\
107	0.0099699641139658\\
108	0.00996996407501121\\
109	0.0099699640353951\\
110	0.00996996399510604\\
111	0.00996996395413243\\
112	0.00996996391246248\\
113	0.00996996387008414\\
114	0.0099699638269852\\
115	0.0099699637831532\\
116	0.00996996373857546\\
117	0.00996996369323909\\
118	0.00996996364713098\\
119	0.00996996360023775\\
120	0.00996996355254582\\
121	0.00996996350404134\\
122	0.00996996345471023\\
123	0.00996996340453816\\
124	0.00996996335351054\\
125	0.00996996330161252\\
126	0.009969963248829\\
127	0.00996996319514459\\
128	0.00996996314054364\\
129	0.00996996308501022\\
130	0.00996996302852812\\
131	0.00996996297108083\\
132	0.00996996291265156\\
133	0.00996996285322321\\
134	0.00996996279277839\\
135	0.00996996273129941\\
136	0.00996996266876823\\
137	0.00996996260516652\\
138	0.00996996254047562\\
139	0.00996996247467654\\
140	0.00996996240774995\\
141	0.00996996233967618\\
142	0.00996996227043522\\
143	0.0099699622000067\\
144	0.0099699621283699\\
145	0.00996996205550372\\
146	0.00996996198138669\\
147	0.00996996190599699\\
148	0.00996996182931239\\
149	0.00996996175131028\\
150	0.00996996167196766\\
151	0.00996996159126112\\
152	0.00996996150916683\\
153	0.00996996142566058\\
154	0.00996996134071771\\
155	0.00996996125431315\\
156	0.00996996116642136\\
157	0.00996996107701641\\
158	0.00996996098607187\\
159	0.0099699608935609\\
160	0.00996996079945615\\
161	0.00996996070372984\\
162	0.00996996060635368\\
163	0.00996996050729892\\
164	0.00996996040653629\\
165	0.00996996030403604\\
166	0.00996996019976788\\
167	0.00996996009370104\\
168	0.00996995998580419\\
169	0.00996995987604547\\
170	0.00996995976439248\\
171	0.00996995965081226\\
172	0.00996995953527128\\
173	0.00996995941773546\\
174	0.00996995929817009\\
175	0.0099699591765399\\
176	0.009969959052809\\
177	0.00996995892694088\\
178	0.0099699587988984\\
179	0.00996995866864378\\
180	0.00996995853613859\\
181	0.00996995840134372\\
182	0.00996995826421941\\
183	0.00996995812472517\\
184	0.00996995798281983\\
185	0.00996995783846149\\
186	0.00996995769160753\\
187	0.00996995754221457\\
188	0.00996995739023846\\
189	0.00996995723563431\\
190	0.00996995707835639\\
191	0.00996995691835821\\
192	0.00996995675559244\\
193	0.0099699565900109\\
194	0.0099699564215646\\
195	0.00996995625020365\\
196	0.00996995607587729\\
197	0.00996995589853388\\
198	0.00996995571812082\\
199	0.00996995553458463\\
200	0.00996995534787086\\
201	0.0099699551579241\\
202	0.00996995496468796\\
203	0.00996995476810505\\
204	0.00996995456811695\\
205	0.00996995436466423\\
206	0.00996995415768639\\
207	0.00996995394712184\\
208	0.00996995373290793\\
209	0.00996995351498086\\
210	0.00996995329327574\\
211	0.00996995306772648\\
212	0.00996995283826584\\
213	0.00996995260482538\\
214	0.00996995236733544\\
215	0.0099699521257251\\
216	0.00996995187992221\\
217	0.00996995162985329\\
218	0.0099699513754436\\
219	0.00996995111661702\\
220	0.00996995085329609\\
221	0.00996995058540196\\
222	0.00996995031285437\\
223	0.00996995003557163\\
224	0.00996994975347059\\
225	0.00996994946646659\\
226	0.00996994917447347\\
227	0.00996994887740352\\
228	0.00996994857516746\\
229	0.00996994826767441\\
230	0.00996994795483184\\
231	0.00996994763654557\\
232	0.00996994731271975\\
233	0.00996994698325675\\
234	0.00996994664805725\\
235	0.00996994630702009\\
236	0.0099699459600423\\
237	0.00996994560701908\\
238	0.00996994524784372\\
239	0.00996994488240757\\
240	0.00996994451060006\\
241	0.00996994413230859\\
242	0.00996994374741854\\
243	0.00996994335581322\\
244	0.00996994295737383\\
245	0.00996994255197941\\
246	0.00996994213950683\\
247	0.00996994171983073\\
248	0.00996994129282346\\
249	0.00996994085835508\\
250	0.0099699404162933\\
251	0.00996993996650341\\
252	0.00996993950884827\\
253	0.00996993904318826\\
254	0.00996993856938121\\
255	0.00996993808728239\\
256	0.00996993759674444\\
257	0.00996993709761731\\
258	0.00996993658974824\\
259	0.00996993607298169\\
260	0.00996993554715929\\
261	0.00996993501211981\\
262	0.00996993446769908\\
263	0.00996993391372993\\
264	0.00996993335004219\\
265	0.00996993277646257\\
266	0.00996993219281463\\
267	0.00996993159891875\\
268	0.009969930994592\\
269	0.00996993037964818\\
270	0.00996992975389767\\
271	0.00996992911714741\\
272	0.00996992846920084\\
273	0.00996992780985784\\
274	0.00996992713891465\\
275	0.00996992645616379\\
276	0.00996992576139404\\
277	0.00996992505439036\\
278	0.00996992433493378\\
279	0.00996992360280138\\
280	0.0099699228577662\\
281	0.00996992209959719\\
282	0.00996992132805907\\
283	0.00996992054291237\\
284	0.00996991974391325\\
285	0.00996991893081348\\
286	0.00996991810336036\\
287	0.00996991726129664\\
288	0.00996991640436042\\
289	0.00996991553228512\\
290	0.00996991464479934\\
291	0.00996991374162682\\
292	0.00996991282248637\\
293	0.00996991188709173\\
294	0.00996991093515156\\
295	0.0099699099663693\\
296	0.00996990898044311\\
297	0.00996990797706577\\
298	0.00996990695592461\\
299	0.00996990591670143\\
300	0.00996990485907237\\
301	0.00996990378270787\\
302	0.00996990268727253\\
303	0.00996990157242509\\
304	0.00996990043781826\\
305	0.00996989928309867\\
306	0.00996989810790679\\
307	0.0099698969118768\\
308	0.00996989569463652\\
309	0.00996989445580732\\
310	0.009969893195004\\
311	0.00996989191183472\\
312	0.00996989060590088\\
313	0.00996988927679704\\
314	0.00996988792411083\\
315	0.00996988654742281\\
316	0.00996988514630644\\
317	0.00996988372032789\\
318	0.00996988226904603\\
319	0.00996988079201227\\
320	0.00996987928877048\\
321	0.00996987775885688\\
322	0.00996987620179994\\
323	0.00996987461712028\\
324	0.00996987300433055\\
325	0.00996987136293531\\
326	0.009969869692431\\
327	0.0099698679923057\\
328	0.00996986626203913\\
329	0.0099698645011025\\
330	0.00996986270895836\\
331	0.00996986088506052\\
332	0.00996985902885392\\
333	0.0099698571397745\\
334	0.00996985521724908\\
335	0.0099698532606952\\
336	0.00996985126952101\\
337	0.00996984924312513\\
338	0.00996984718089648\\
339	0.00996984508221414\\
340	0.00996984294644718\\
341	0.0099698407729545\\
342	0.00996983856108466\\
343	0.00996983631017565\\
344	0.00996983401955474\\
345	0.00996983168853827\\
346	0.00996982931643136\\
347	0.00996982690252778\\
348	0.00996982444610959\\
349	0.00996982194644695\\
350	0.00996981940279781\\
351	0.00996981681440757\\
352	0.00996981418050881\\
353	0.0099698115003209\\
354	0.00996980877304962\\
355	0.00996980599788678\\
356	0.00996980317400977\\
357	0.00996980030058113\\
358	0.00996979737674798\\
359	0.00996979440164157\\
360	0.00996979137437667\\
361	0.00996978829405096\\
362	0.00996978515974438\\
363	0.00996978197051844\\
364	0.00996977872541548\\
365	0.00996977542345787\\
366	0.00996977206364713\\
367	0.00996976864496313\\
368	0.00996976516636301\\
369	0.00996976162678025\\
370	0.00996975802512356\\
371	0.00996975436027575\\
372	0.00996975063109251\\
373	0.00996974683640113\\
374	0.0099697429749992\\
375	0.00996973904565309\\
376	0.00996973504709655\\
377	0.00996973097802906\\
378	0.00996972683711417\\
379	0.00996972262297775\\
380	0.00996971833420612\\
381	0.00996971396934407\\
382	0.00996970952689278\\
383	0.00996970500530764\\
384	0.00996970040299582\\
385	0.00996969571831386\\
386	0.0099696909495649\\
387	0.00996968609499584\\
388	0.0099696811527942\\
389	0.00996967612108474\\
390	0.0099696709979258\\
391	0.00996966578130518\\
392	0.00996966046913576\\
393	0.00996965505925043\\
394	0.00996964954939657\\
395	0.00996964393722969\\
396	0.00996963822030625\\
397	0.00996963239607535\\
398	0.00996962646186914\\
399	0.00996962041489166\\
400	0.00996961425220605\\
401	0.00996960797072057\\
402	0.00996960156717452\\
403	0.00996959503812491\\
404	0.00996958837992792\\
405	0.00996958158870555\\
406	0.00996957466032922\\
407	0.00996956759043277\\
408	0.00996956037439588\\
409	0.00996955300732745\\
410	0.00996954548404943\\
411	0.00996953779908189\\
412	0.00996952994662935\\
413	0.0099695219205671\\
414	0.00996951371442238\\
415	0.00996950532134258\\
416	0.00996949673405511\\
417	0.00996948794487979\\
418	0.00996947894579393\\
419	0.00996946972817297\\
420	0.00996946028268595\\
421	0.00996945059920828\\
422	0.00996944066672143\\
423	0.00996943047319725\\
424	0.0099694200054641\\
425	0.00996940924905151\\
426	0.00996939818800903\\
427	0.00996938680469331\\
428	0.00996937507951417\\
429	0.00996936299062253\\
430	0.00996935051350271\\
431	0.00996933762037826\\
432	0.00996932427919697\\
433	0.00996931045158034\\
434	0.00996929608813639\\
435	0.00996928111709898\\
436	0.00996926541677184\\
437	0.00996924875236517\\
438	0.00996923065227521\\
439	0.00996921027240247\\
440	0.00996914722470959\\
441	0.00996905323791148\\
442	0.00996895700353201\\
443	0.00996885844626655\\
444	0.00996875748756669\\
445	0.00996865404562082\\
446	0.00996854803539017\\
447	0.00996843936872157\\
448	0.00996832795456464\\
449	0.0099682136993294\\
450	0.00996809650743038\\
451	0.00996797628207344\\
452	0.0099678529263493\\
453	0.00996772634469452\\
454	0.00996759644474977\\
455	0.00996746313955286\\
456	0.0099673263497839\\
457	0.00996718600542639\\
458	0.00996704204584876\\
459	0.00996689441745486\\
460	0.00996674306799945\\
461	0.0099665879168168\\
462	0.00996642858931903\\
463	0.00996626287390056\\
464	0.00996608086039723\\
465	0.00996560070924794\\
466	0.00996511141802948\\
467	0.00996461271605404\\
468	0.00996410431613691\\
469	0.00996358591377835\\
470	0.00996305718637086\\
471	0.00996251779086151\\
472	0.00996196736363268\\
473	0.00996140552254573\\
474	0.00996083186701568\\
475	0.0099602459774223\\
476	0.00995964741248944\\
477	0.0099590357087881\\
478	0.00995841039055374\\
479	0.00995777095424221\\
480	0.00995711686219497\\
481	0.00995644753924431\\
482	0.00995576236905725\\
483	0.00995506068982423\\
484	0.00995434178917141\\
485	0.00995360489833252\\
486	0.0099528491857619\\
487	0.00995207374943311\\
488	0.00995127760706633\\
489	0.00995045968546667\\
490	0.00994961880800696\\
491	0.00994875367986187\\
492	0.00994786287049847\\
493	0.00994694479278383\\
494	0.00994599767784186\\
495	0.00994501954436793\\
496	0.00994400816019456\\
497	0.00994296099166491\\
498	0.00994187513049822\\
499	0.00994074717187218\\
500	0.00993957297357205\\
501	0.00993834710585245\\
502	0.00993706147747492\\
503	0.00993570179724759\\
504	0.00993248398813303\\
505	0.00992458450554222\\
506	0.00991413653284674\\
507	0.00990352052705566\\
508	0.00989272929216558\\
509	0.00988175512175933\\
510	0.00987058941679399\\
511	0.0098592228462381\\
512	0.00984764596295923\\
513	0.00983584888235029\\
514	0.00982382077557695\\
515	0.00981154976657298\\
516	0.00979902281971108\\
517	0.00978622561948705\\
518	0.00977314244811886\\
519	0.00975975608224493\\
520	0.0097460477821495\\
521	0.00973199762945084\\
522	0.00971758611197099\\
523	0.00970280013215884\\
524	0.0096879112403344\\
525	0.00967331433680058\\
526	0.00965827364290924\\
527	0.0096427331866361\\
528	0.00962661829896217\\
529	0.00960982055197028\\
530	0.00953928920580111\\
531	0.009381629884998\\
532	0.00921843775153357\\
533	0.00904940266084434\\
534	0.00887420844912711\\
535	0.00869254825095905\\
536	0.00850414847764258\\
537	0.00831267373605276\\
538	0.00811927026652865\\
539	0.00791867309489294\\
540	0.00771042776501912\\
541	0.00749378093925573\\
542	0.00726785763693795\\
543	0.00703163999036617\\
544	0.00678393139256419\\
545	0.00652326853684657\\
546	0.00624786368936832\\
547	0.00595554554291631\\
548	0.00564387510515887\\
549	0.00532233053630014\\
550	0.00517288644773442\\
551	0.00500843477872852\\
552	0.00483831416493209\\
553	0.00466529191159948\\
554	0.00448978070172018\\
555	0.00431233796658526\\
556	0.00413367673565331\\
557	0.00395483603638486\\
558	0.00377714582892456\\
559	0.00360215714364037\\
560	0.00343164512815057\\
561	0.00326867563951183\\
562	0.00311755320915099\\
563	0.00297470414778135\\
564	0.00283388527610961\\
565	0.0026973744658845\\
566	0.00255897925763545\\
567	0.00241906878442113\\
568	0.002278315210554\\
569	0.00213766925746306\\
570	0.00199855167123961\\
571	0.00186534421375662\\
572	0.00173829019138898\\
573	0.00161722417034892\\
574	0.00150255682998357\\
575	0.00139486353128288\\
576	0.00129406798244129\\
577	0.0011966074677817\\
578	0.001102369942485\\
579	0.00101118175468484\\
580	0.00092283933127548\\
581	0.000838824294993086\\
582	0.000758927142791253\\
583	0.000683310297993498\\
584	0.00061327566441274\\
585	0.000548715386105192\\
586	0.000488688640616261\\
587	0.00043319358073894\\
588	0.000381096578785599\\
589	0.000332240260312541\\
590	0.000286210565888725\\
591	0.000242202348687757\\
592	0.000200085409992\\
593	0.000159745073714005\\
594	0.000121173935986304\\
595	8.45520570083912e-05\\
596	5.05092148680373e-05\\
597	2.07908715710836e-05\\
598	0\\
599	0\\
600	0\\
};
\addplot [color=mycolor10,solid,forget plot]
  table[row sep=crcr]{%
1	0.00997084548698741\\
2	0.00997084548691379\\
3	0.00997084548683892\\
4	0.0099708454867628\\
5	0.00997084548668539\\
6	0.00997084548660669\\
7	0.00997084548652666\\
8	0.00997084548644529\\
9	0.00997084548636254\\
10	0.00997084548627841\\
11	0.00997084548619286\\
12	0.00997084548610587\\
13	0.00997084548601742\\
14	0.00997084548592748\\
15	0.00997084548583603\\
16	0.00997084548574304\\
17	0.00997084548564849\\
18	0.00997084548555234\\
19	0.00997084548545458\\
20	0.00997084548535518\\
21	0.0099708454852541\\
22	0.00997084548515133\\
23	0.00997084548504682\\
24	0.00997084548494056\\
25	0.0099708454848325\\
26	0.00997084548472264\\
27	0.00997084548461093\\
28	0.00997084548449733\\
29	0.00997084548438182\\
30	0.00997084548426437\\
31	0.00997084548414495\\
32	0.00997084548402352\\
33	0.00997084548390004\\
34	0.00997084548377448\\
35	0.00997084548364682\\
36	0.009970845483517\\
37	0.00997084548338501\\
38	0.00997084548325079\\
39	0.00997084548311431\\
40	0.00997084548297553\\
41	0.00997084548283442\\
42	0.00997084548269094\\
43	0.00997084548254504\\
44	0.00997084548239668\\
45	0.00997084548224583\\
46	0.00997084548209243\\
47	0.00997084548193646\\
48	0.00997084548177786\\
49	0.00997084548161659\\
50	0.0099708454814526\\
51	0.00997084548128586\\
52	0.0099708454811163\\
53	0.00997084548094389\\
54	0.00997084548076858\\
55	0.00997084548059031\\
56	0.00997084548040904\\
57	0.00997084548022471\\
58	0.00997084548003728\\
59	0.00997084547984669\\
60	0.00997084547965289\\
61	0.00997084547945583\\
62	0.00997084547925545\\
63	0.00997084547905168\\
64	0.00997084547884449\\
65	0.00997084547863379\\
66	0.00997084547841955\\
67	0.0099708454782017\\
68	0.00997084547798016\\
69	0.0099708454777549\\
70	0.00997084547752584\\
71	0.00997084547729291\\
72	0.00997084547705604\\
73	0.00997084547681519\\
74	0.00997084547657027\\
75	0.00997084547632122\\
76	0.00997084547606796\\
77	0.00997084547581043\\
78	0.00997084547554855\\
79	0.00997084547528224\\
80	0.00997084547501144\\
81	0.00997084547473606\\
82	0.00997084547445604\\
83	0.00997084547417127\\
84	0.0099708454738817\\
85	0.00997084547358724\\
86	0.00997084547328779\\
87	0.00997084547298328\\
88	0.00997084547267362\\
89	0.00997084547235873\\
90	0.00997084547203851\\
91	0.00997084547171287\\
92	0.00997084547138171\\
93	0.00997084547104496\\
94	0.0099708454707025\\
95	0.00997084547035424\\
96	0.00997084547000008\\
97	0.00997084546963992\\
98	0.00997084546927367\\
99	0.0099708454689012\\
100	0.00997084546852242\\
101	0.00997084546813722\\
102	0.00997084546774548\\
103	0.0099708454673471\\
104	0.00997084546694196\\
105	0.00997084546652994\\
106	0.00997084546611093\\
107	0.00997084546568481\\
108	0.00997084546525145\\
109	0.00997084546481072\\
110	0.00997084546436251\\
111	0.00997084546390668\\
112	0.0099708454634431\\
113	0.00997084546297163\\
114	0.00997084546249215\\
115	0.00997084546200451\\
116	0.00997084546150856\\
117	0.00997084546100418\\
118	0.0099708454604912\\
119	0.00997084545996949\\
120	0.00997084545943889\\
121	0.00997084545889924\\
122	0.00997084545835039\\
123	0.00997084545779219\\
124	0.00997084545722446\\
125	0.00997084545664704\\
126	0.00997084545605977\\
127	0.00997084545546246\\
128	0.00997084545485496\\
129	0.00997084545423708\\
130	0.00997084545360864\\
131	0.00997084545296945\\
132	0.00997084545231933\\
133	0.0099708454516581\\
134	0.00997084545098555\\
135	0.00997084545030149\\
136	0.00997084544960571\\
137	0.00997084544889802\\
138	0.0099708454481782\\
139	0.00997084544744605\\
140	0.00997084544670135\\
141	0.00997084544594388\\
142	0.00997084544517342\\
143	0.00997084544438973\\
144	0.0099708454435926\\
145	0.00997084544278178\\
146	0.00997084544195704\\
147	0.00997084544111813\\
148	0.00997084544026481\\
149	0.00997084543939683\\
150	0.00997084543851392\\
151	0.00997084543761583\\
152	0.00997084543670229\\
153	0.00997084543577304\\
154	0.00997084543482779\\
155	0.00997084543386628\\
156	0.00997084543288821\\
157	0.0099708454318933\\
158	0.00997084543088126\\
159	0.00997084542985178\\
160	0.00997084542880455\\
161	0.00997084542773929\\
162	0.00997084542665566\\
163	0.00997084542555335\\
164	0.00997084542443202\\
165	0.00997084542329137\\
166	0.00997084542213103\\
167	0.00997084542095068\\
168	0.00997084541974996\\
169	0.00997084541852853\\
170	0.00997084541728601\\
171	0.00997084541602204\\
172	0.00997084541473626\\
173	0.00997084541342828\\
174	0.00997084541209771\\
175	0.00997084541074417\\
176	0.00997084540936725\\
177	0.00997084540796655\\
178	0.00997084540654165\\
179	0.00997084540509214\\
180	0.00997084540361759\\
181	0.00997084540211755\\
182	0.00997084540059161\\
183	0.00997084539903929\\
184	0.00997084539746014\\
185	0.0099708453958537\\
186	0.00997084539421949\\
187	0.00997084539255703\\
188	0.00997084539086584\\
189	0.00997084538914539\\
190	0.0099708453873952\\
191	0.00997084538561475\\
192	0.0099708453838035\\
193	0.00997084538196092\\
194	0.00997084538008647\\
195	0.00997084537817959\\
196	0.00997084537623971\\
197	0.00997084537426627\\
198	0.00997084537225868\\
199	0.00997084537021633\\
200	0.00997084536813864\\
201	0.00997084536602497\\
202	0.00997084536387471\\
203	0.00997084536168721\\
204	0.00997084535946183\\
205	0.0099708453571979\\
206	0.00997084535489475\\
207	0.0099708453525517\\
208	0.00997084535016805\\
209	0.00997084534774309\\
210	0.00997084534527609\\
211	0.00997084534276634\\
212	0.00997084534021306\\
213	0.00997084533761551\\
214	0.00997084533497291\\
215	0.00997084533228447\\
216	0.00997084532954939\\
217	0.00997084532676685\\
218	0.00997084532393602\\
219	0.00997084532105606\\
220	0.00997084531812609\\
221	0.00997084531514526\\
222	0.00997084531211265\\
223	0.00997084530902738\\
224	0.0099708453058885\\
225	0.00997084530269509\\
226	0.00997084529944617\\
227	0.00997084529614078\\
228	0.00997084529277792\\
229	0.00997084528935659\\
230	0.00997084528587576\\
231	0.00997084528233437\\
232	0.00997084527873137\\
233	0.00997084527506566\\
234	0.00997084527133614\\
235	0.0099708452675417\\
236	0.00997084526368117\\
237	0.00997084525975341\\
238	0.00997084525575722\\
239	0.0099708452516914\\
240	0.00997084524755472\\
241	0.00997084524334592\\
242	0.00997084523906373\\
243	0.00997084523470686\\
244	0.00997084523027398\\
245	0.00997084522576376\\
246	0.00997084522117482\\
247	0.00997084521650577\\
248	0.00997084521175519\\
249	0.00997084520692164\\
250	0.00997084520200365\\
251	0.00997084519699972\\
252	0.00997084519190834\\
253	0.00997084518672794\\
254	0.00997084518145695\\
255	0.00997084517609377\\
256	0.00997084517063675\\
257	0.00997084516508423\\
258	0.00997084515943451\\
259	0.00997084515368588\\
260	0.00997084514783656\\
261	0.00997084514188476\\
262	0.00997084513582868\\
263	0.00997084512966644\\
264	0.00997084512339617\\
265	0.00997084511701593\\
266	0.00997084511052378\\
267	0.00997084510391771\\
268	0.0099708450971957\\
269	0.00997084509035568\\
270	0.00997084508339555\\
271	0.00997084507631316\\
272	0.00997084506910634\\
273	0.00997084506177287\\
274	0.00997084505431048\\
275	0.00997084504671688\\
276	0.00997084503898972\\
277	0.00997084503112663\\
278	0.00997084502312516\\
279	0.00997084501498286\\
280	0.0099708450066972\\
281	0.00997084499826563\\
282	0.00997084498968554\\
283	0.00997084498095427\\
284	0.00997084497206913\\
285	0.00997084496302736\\
286	0.00997084495382617\\
287	0.00997084494446271\\
288	0.00997084493493408\\
289	0.00997084492523733\\
290	0.00997084491536945\\
291	0.00997084490532739\\
292	0.00997084489510803\\
293	0.00997084488470822\\
294	0.00997084487412471\\
295	0.00997084486335425\\
296	0.00997084485239349\\
297	0.00997084484123903\\
298	0.00997084482988741\\
299	0.00997084481833513\\
300	0.0099708448065786\\
301	0.00997084479461418\\
302	0.00997084478243817\\
303	0.00997084477004681\\
304	0.00997084475743624\\
305	0.00997084474460259\\
306	0.00997084473154187\\
307	0.00997084471825006\\
308	0.00997084470472304\\
309	0.00997084469095665\\
310	0.00997084467694664\\
311	0.00997084466268868\\
312	0.00997084464817839\\
313	0.00997084463341131\\
314	0.00997084461838288\\
315	0.00997084460308849\\
316	0.00997084458752345\\
317	0.00997084457168299\\
318	0.00997084455556225\\
319	0.00997084453915629\\
320	0.00997084452246011\\
321	0.0099708445054686\\
322	0.00997084448817659\\
323	0.00997084447057881\\
324	0.0099708444526699\\
325	0.00997084443444444\\
326	0.00997084441589689\\
327	0.00997084439702164\\
328	0.00997084437781298\\
329	0.00997084435826512\\
330	0.00997084433837215\\
331	0.00997084431812812\\
332	0.00997084429752692\\
333	0.00997084427656238\\
334	0.00997084425522824\\
335	0.00997084423351811\\
336	0.00997084421142552\\
337	0.00997084418894389\\
338	0.00997084416606655\\
339	0.00997084414278671\\
340	0.00997084411909746\\
341	0.00997084409499181\\
342	0.00997084407046265\\
343	0.00997084404550273\\
344	0.00997084402010473\\
345	0.00997084399426118\\
346	0.00997084396796449\\
347	0.00997084394120695\\
348	0.00997084391398074\\
349	0.0099708438862779\\
350	0.00997084385809033\\
351	0.0099708438294098\\
352	0.00997084380022794\\
353	0.00997084377053624\\
354	0.00997084374032603\\
355	0.00997084370958851\\
356	0.0099708436783147\\
357	0.00997084364649547\\
358	0.0099708436141215\\
359	0.00997084358118333\\
360	0.0099708435476713\\
361	0.00997084351357554\\
362	0.00997084347888602\\
363	0.00997084344359248\\
364	0.00997084340768447\\
365	0.0099708433711513\\
366	0.00997084333398205\\
367	0.00997084329616555\\
368	0.00997084325769042\\
369	0.00997084321854495\\
370	0.0099708431787172\\
371	0.00997084313819494\\
372	0.0099708430969656\\
373	0.00997084305501633\\
374	0.00997084301233393\\
375	0.00997084296890484\\
376	0.00997084292471516\\
377	0.00997084287975059\\
378	0.00997084283399642\\
379	0.00997084278743753\\
380	0.00997084274005834\\
381	0.00997084269184282\\
382	0.00997084264277444\\
383	0.00997084259283615\\
384	0.00997084254201035\\
385	0.00997084249027889\\
386	0.00997084243762299\\
387	0.00997084238402323\\
388	0.00997084232945953\\
389	0.00997084227391108\\
390	0.00997084221735631\\
391	0.00997084215977281\\
392	0.00997084210113733\\
393	0.00997084204142566\\
394	0.00997084198061258\\
395	0.00997084191867175\\
396	0.00997084185557563\\
397	0.00997084179129535\\
398	0.00997084172580054\\
399	0.0099708416590592\\
400	0.0099708415910376\\
401	0.00997084152170017\\
402	0.00997084145100952\\
403	0.00997084137892621\\
404	0.00997084130540826\\
405	0.00997084123041089\\
406	0.0099708411538868\\
407	0.00997084107578596\\
408	0.00997084099605557\\
409	0.00997084091463992\\
410	0.00997084083148032\\
411	0.00997084074651508\\
412	0.0099708406596792\\
413	0.00997084057090389\\
414	0.00997084048011545\\
415	0.00997084038723444\\
416	0.00997084029217644\\
417	0.00997084019485359\\
418	0.00997084009517133\\
419	0.00997083999302708\\
420	0.00997083988830929\\
421	0.0099708397808962\\
422	0.00997083967065452\\
423	0.00997083955743784\\
424	0.00997083944108465\\
425	0.00997083932141605\\
426	0.00997083919823265\\
427	0.00997083907131013\\
428	0.00997083894039184\\
429	0.0099708388051747\\
430	0.00997083866527911\\
431	0.00997083852018113\\
432	0.00997083836905726\\
433	0.0099708382104362\\
434	0.00997083804145571\\
435	0.00997083785641904\\
436	0.00997083764446549\\
437	0.0099708373873598\\
438	0.00997083706262313\\
439	0.00997083666448949\\
440	0.00997083623128341\\
441	0.00997083578821133\\
442	0.00997083533497104\\
443	0.00997083487124972\\
444	0.00997083439672429\\
445	0.00997083391106196\\
446	0.00997083341392111\\
447	0.0099708329049525\\
448	0.00997083238380088\\
449	0.0099708318501067\\
450	0.00997083130350745\\
451	0.00997083074363683\\
452	0.00997083017011757\\
453	0.00997082958253852\\
454	0.00997082898039505\\
455	0.00997082836294783\\
456	0.00997082772890378\\
457	0.00997082707570273\\
458	0.00997082639786802\\
459	0.00997082568288892\\
460	0.0099708249000621\\
461	0.00997082396978854\\
462	0.00997082269080592\\
463	0.00997082061716523\\
464	0.00997081682871155\\
465	0.00997079688302175\\
466	0.00997077655963949\\
467	0.00997075584701171\\
468	0.00997073473290042\\
469	0.00997071320433725\\
470	0.00997069124754055\\
471	0.00997066884793057\\
472	0.00997064599020059\\
473	0.00997062265830179\\
474	0.00997059883539591\\
475	0.00997057450379251\\
476	0.00997054964505322\\
477	0.00997052424024599\\
478	0.00997049826936458\\
479	0.00997047171110278\\
480	0.00997044454273607\\
481	0.00997041673999051\\
482	0.00997038827689002\\
483	0.0099703591255796\\
484	0.0099703292561263\\
485	0.00997029863629874\\
486	0.00997026723130749\\
487	0.00997023500349532\\
488	0.00997020191199549\\
489	0.00997016791233528\\
490	0.0099701329559731\\
491	0.00997009698975231\\
492	0.00997005995524691\\
493	0.0099700217879548\\
494	0.00996998241625147\\
495	0.00996994175991042\\
496	0.00996989972773478\\
497	0.00996985621320054\\
498	0.00996981108546361\\
499	0.00996976416950573\\
500	0.00996971520147044\\
501	0.0099696637305697\\
502	0.00996960891877847\\
503	0.00996954919283904\\
504	0.00996948186685571\\
505	0.00996929001369385\\
506	0.00996896204731497\\
507	0.0099686269049131\\
508	0.00996828423231784\\
509	0.00996793364018603\\
510	0.0099675747004568\\
511	0.00996720696758362\\
512	0.00996682997231482\\
513	0.00996644320053974\\
514	0.00996604608795879\\
515	0.00996563801379749\\
516	0.00996521829280044\\
517	0.00996478616328325\\
518	0.00996434076440996\\
519	0.00996388108095988\\
520	0.0099634057846906\\
521	0.00996291273618887\\
522	0.00996239734618991\\
523	0.0099618470340882\\
524	0.00996098584201838\\
525	0.00995942607081101\\
526	0.00995781208910305\\
527	0.00995613837293259\\
528	0.00995439742492012\\
529	0.00995257893419314\\
530	0.00994806151716236\\
531	0.00993936864397242\\
532	0.00993061374323875\\
533	0.00992179613327257\\
534	0.00991291338686796\\
535	0.00990395593179488\\
536	0.0098948917788269\\
537	0.00988209889720862\\
538	0.00986432247477629\\
539	0.00984622759948875\\
540	0.00982778327047255\\
541	0.00980895423599394\\
542	0.00978970060060886\\
543	0.00976997744200252\\
544	0.00974973268212212\\
545	0.00972890408895865\\
546	0.00970741598062481\\
547	0.00968517836419248\\
548	0.00966207204382259\\
549	0.00962617862383391\\
550	0.00940147602621652\\
551	0.00916775487436897\\
552	0.00892443714625995\\
553	0.0086706446418252\\
554	0.00840532921897203\\
555	0.00812729915250492\\
556	0.00783671321482161\\
557	0.00753063773729663\\
558	0.0072069709116426\\
559	0.00686361645109179\\
560	0.00650839030900701\\
561	0.00613192561547077\\
562	0.00572661969500788\\
563	0.0052961591784016\\
564	0.00484649400971892\\
565	0.00437548564659083\\
566	0.00415383614871475\\
567	0.00393383894687057\\
568	0.00371213006982874\\
569	0.00349020085871587\\
570	0.00326984259083614\\
571	0.00305370983431872\\
572	0.00284548539780111\\
573	0.00265016852435452\\
574	0.00245848661210091\\
575	0.00226662545412475\\
576	0.00207713061461257\\
577	0.00189587208535604\\
578	0.00172554605519975\\
579	0.00156853045100365\\
580	0.00142575417691301\\
581	0.00129100157684195\\
582	0.00116453672973894\\
583	0.00104320185244136\\
584	0.000925111851658451\\
585	0.000811581194832413\\
586	0.0007041690715661\\
587	0.000603390290013832\\
588	0.000510932493051772\\
589	0.00042719620397733\\
590	0.00035239839743971\\
591	0.000285859088514837\\
592	0.00022641344495182\\
593	0.000173745915387586\\
594	0.00012714455469262\\
595	8.61467099905492e-05\\
596	5.05092148680371e-05\\
597	2.07908715710836e-05\\
598	0\\
599	0\\
600	0\\
};
\addplot [color=mycolor11,solid,forget plot]
  table[row sep=crcr]{%
1	0.00997158250207167\\
2	0.00997158250206864\\
3	0.00997158250206557\\
4	0.00997158250206244\\
5	0.00997158250205925\\
6	0.00997158250205602\\
7	0.00997158250205273\\
8	0.00997158250204938\\
9	0.00997158250204598\\
10	0.00997158250204253\\
11	0.00997158250203901\\
12	0.00997158250203543\\
13	0.0099715825020318\\
14	0.0099715825020281\\
15	0.00997158250202434\\
16	0.00997158250202052\\
17	0.00997158250201663\\
18	0.00997158250201268\\
19	0.00997158250200866\\
20	0.00997158250200458\\
21	0.00997158250200042\\
22	0.0099715825019962\\
23	0.0099715825019919\\
24	0.00997158250198753\\
25	0.00997158250198309\\
26	0.00997158250197857\\
27	0.00997158250197398\\
28	0.00997158250196931\\
29	0.00997158250196457\\
30	0.00997158250195974\\
31	0.00997158250195483\\
32	0.00997158250194984\\
33	0.00997158250194476\\
34	0.0099715825019396\\
35	0.00997158250193435\\
36	0.00997158250192902\\
37	0.00997158250192359\\
38	0.00997158250191807\\
39	0.00997158250191246\\
40	0.00997158250190676\\
41	0.00997158250190096\\
42	0.00997158250189506\\
43	0.00997158250188906\\
44	0.00997158250188296\\
45	0.00997158250187676\\
46	0.00997158250187046\\
47	0.00997158250186404\\
48	0.00997158250185753\\
49	0.0099715825018509\\
50	0.00997158250184415\\
51	0.0099715825018373\\
52	0.00997158250183033\\
53	0.00997158250182324\\
54	0.00997158250181604\\
55	0.00997158250180871\\
56	0.00997158250180125\\
57	0.00997158250179368\\
58	0.00997158250178598\\
59	0.00997158250177814\\
60	0.00997158250177017\\
61	0.00997158250176207\\
62	0.00997158250175384\\
63	0.00997158250174546\\
64	0.00997158250173694\\
65	0.00997158250172828\\
66	0.00997158250171947\\
67	0.00997158250171052\\
68	0.00997158250170141\\
69	0.00997158250169215\\
70	0.00997158250168273\\
71	0.00997158250167316\\
72	0.00997158250166342\\
73	0.00997158250165352\\
74	0.00997158250164345\\
75	0.00997158250163321\\
76	0.0099715825016228\\
77	0.00997158250161221\\
78	0.00997158250160145\\
79	0.0099715825015905\\
80	0.00997158250157936\\
81	0.00997158250156805\\
82	0.00997158250155653\\
83	0.00997158250154483\\
84	0.00997158250153292\\
85	0.00997158250152082\\
86	0.0099715825015085\\
87	0.00997158250149599\\
88	0.00997158250148326\\
89	0.00997158250147031\\
90	0.00997158250145714\\
91	0.00997158250144376\\
92	0.00997158250143014\\
93	0.0099715825014163\\
94	0.00997158250140222\\
95	0.0099715825013879\\
96	0.00997158250137334\\
97	0.00997158250135853\\
98	0.00997158250134347\\
99	0.00997158250132816\\
100	0.00997158250131259\\
101	0.00997158250129675\\
102	0.00997158250128064\\
103	0.00997158250126427\\
104	0.00997158250124761\\
105	0.00997158250123067\\
106	0.00997158250121344\\
107	0.00997158250119592\\
108	0.0099715825011781\\
109	0.00997158250115998\\
110	0.00997158250114155\\
111	0.00997158250112281\\
112	0.00997158250110375\\
113	0.00997158250108436\\
114	0.00997158250106465\\
115	0.0099715825010446\\
116	0.0099715825010242\\
117	0.00997158250100346\\
118	0.00997158250098237\\
119	0.00997158250096092\\
120	0.0099715825009391\\
121	0.00997158250091691\\
122	0.00997158250089434\\
123	0.00997158250087139\\
124	0.00997158250084804\\
125	0.0099715825008243\\
126	0.00997158250080015\\
127	0.00997158250077559\\
128	0.0099715825007506\\
129	0.0099715825007252\\
130	0.00997158250069935\\
131	0.00997158250067307\\
132	0.00997158250064633\\
133	0.00997158250061914\\
134	0.00997158250059148\\
135	0.00997158250056335\\
136	0.00997158250053474\\
137	0.00997158250050563\\
138	0.00997158250047603\\
139	0.00997158250044592\\
140	0.0099715825004153\\
141	0.00997158250038414\\
142	0.00997158250035246\\
143	0.00997158250032023\\
144	0.00997158250028744\\
145	0.0099715825002541\\
146	0.00997158250022018\\
147	0.00997158250018568\\
148	0.00997158250015058\\
149	0.00997158250011488\\
150	0.00997158250007857\\
151	0.00997158250004163\\
152	0.00997158250000406\\
153	0.00997158249996584\\
154	0.00997158249992696\\
155	0.00997158249988742\\
156	0.00997158249984719\\
157	0.00997158249980627\\
158	0.00997158249976464\\
159	0.0099715824997223\\
160	0.00997158249967923\\
161	0.00997158249963541\\
162	0.00997158249959084\\
163	0.0099715824995455\\
164	0.00997158249949938\\
165	0.00997158249945247\\
166	0.00997158249940474\\
167	0.00997158249935619\\
168	0.0099715824993068\\
169	0.00997158249925656\\
170	0.00997158249920546\\
171	0.00997158249915346\\
172	0.00997158249910057\\
173	0.00997158249904678\\
174	0.00997158249899205\\
175	0.00997158249893637\\
176	0.00997158249887974\\
177	0.00997158249882212\\
178	0.00997158249876351\\
179	0.00997158249870389\\
180	0.00997158249864324\\
181	0.00997158249858154\\
182	0.00997158249851877\\
183	0.00997158249845492\\
184	0.00997158249838997\\
185	0.00997158249832389\\
186	0.00997158249825667\\
187	0.00997158249818829\\
188	0.00997158249811872\\
189	0.00997158249804795\\
190	0.00997158249797596\\
191	0.00997158249790273\\
192	0.00997158249782822\\
193	0.00997158249775243\\
194	0.00997158249767533\\
195	0.00997158249759689\\
196	0.00997158249751709\\
197	0.00997158249743592\\
198	0.00997158249735334\\
199	0.00997158249726933\\
200	0.00997158249718386\\
201	0.00997158249709692\\
202	0.00997158249700847\\
203	0.00997158249691849\\
204	0.00997158249682695\\
205	0.00997158249673382\\
206	0.00997158249663908\\
207	0.0099715824965427\\
208	0.00997158249644465\\
209	0.00997158249634489\\
210	0.00997158249624341\\
211	0.00997158249614017\\
212	0.00997158249603514\\
213	0.00997158249592829\\
214	0.00997158249581958\\
215	0.009971582495709\\
216	0.00997158249559648\\
217	0.00997158249548202\\
218	0.00997158249536557\\
219	0.0099715824952471\\
220	0.00997158249512657\\
221	0.00997158249500395\\
222	0.0099715824948792\\
223	0.00997158249475227\\
224	0.00997158249462315\\
225	0.00997158249449178\\
226	0.00997158249435813\\
227	0.00997158249422215\\
228	0.00997158249408381\\
229	0.00997158249394306\\
230	0.00997158249379987\\
231	0.00997158249365418\\
232	0.00997158249350596\\
233	0.00997158249335515\\
234	0.00997158249320173\\
235	0.00997158249304563\\
236	0.00997158249288681\\
237	0.00997158249272522\\
238	0.00997158249256082\\
239	0.00997158249239355\\
240	0.00997158249222337\\
241	0.00997158249205022\\
242	0.00997158249187405\\
243	0.00997158249169481\\
244	0.00997158249151243\\
245	0.00997158249132688\\
246	0.00997158249113808\\
247	0.00997158249094599\\
248	0.00997158249075055\\
249	0.00997158249055169\\
250	0.00997158249034935\\
251	0.00997158249014348\\
252	0.00997158248993401\\
253	0.00997158248972088\\
254	0.00997158248950401\\
255	0.00997158248928336\\
256	0.00997158248905884\\
257	0.00997158248883039\\
258	0.00997158248859794\\
259	0.00997158248836142\\
260	0.00997158248812076\\
261	0.00997158248787588\\
262	0.00997158248762671\\
263	0.00997158248737317\\
264	0.00997158248711518\\
265	0.00997158248685267\\
266	0.00997158248658555\\
267	0.00997158248631374\\
268	0.00997158248603716\\
269	0.00997158248575573\\
270	0.00997158248546934\\
271	0.00997158248517794\\
272	0.0099715824848814\\
273	0.00997158248457966\\
274	0.00997158248427261\\
275	0.00997158248396016\\
276	0.00997158248364221\\
277	0.00997158248331867\\
278	0.00997158248298943\\
279	0.0099715824826544\\
280	0.00997158248231347\\
281	0.00997158248196653\\
282	0.00997158248161348\\
283	0.00997158248125421\\
284	0.0099715824808886\\
285	0.00997158248051655\\
286	0.00997158248013794\\
287	0.00997158247975265\\
288	0.00997158247936056\\
289	0.00997158247896156\\
290	0.00997158247855551\\
291	0.0099715824781423\\
292	0.00997158247772179\\
293	0.00997158247729384\\
294	0.00997158247685835\\
295	0.00997158247641516\\
296	0.00997158247596413\\
297	0.00997158247550514\\
298	0.00997158247503804\\
299	0.00997158247456268\\
300	0.00997158247407891\\
301	0.00997158247358659\\
302	0.00997158247308556\\
303	0.00997158247257567\\
304	0.00997158247205676\\
305	0.00997158247152868\\
306	0.00997158247099126\\
307	0.00997158247044432\\
308	0.00997158246988771\\
309	0.00997158246932125\\
310	0.00997158246874477\\
311	0.00997158246815809\\
312	0.00997158246756103\\
313	0.00997158246695341\\
314	0.00997158246633504\\
315	0.00997158246570573\\
316	0.00997158246506528\\
317	0.00997158246441351\\
318	0.00997158246375021\\
319	0.00997158246307519\\
320	0.00997158246238822\\
321	0.00997158246168912\\
322	0.00997158246097766\\
323	0.00997158246025362\\
324	0.0099715824595168\\
325	0.00997158245876695\\
326	0.00997158245800387\\
327	0.00997158245722731\\
328	0.00997158245643705\\
329	0.00997158245563285\\
330	0.00997158245481445\\
331	0.00997158245398163\\
332	0.00997158245313413\\
333	0.0099715824522717\\
334	0.00997158245139408\\
335	0.00997158245050101\\
336	0.00997158244959222\\
337	0.00997158244866744\\
338	0.00997158244772641\\
339	0.00997158244676884\\
340	0.00997158244579445\\
341	0.00997158244480295\\
342	0.00997158244379405\\
343	0.00997158244276747\\
344	0.00997158244172288\\
345	0.00997158244066\\
346	0.0099715824395785\\
347	0.00997158243847807\\
348	0.0099715824373584\\
349	0.00997158243621915\\
350	0.00997158243505999\\
351	0.00997158243388059\\
352	0.0099715824326806\\
353	0.00997158243145967\\
354	0.00997158243021745\\
355	0.00997158242895357\\
356	0.00997158242766766\\
357	0.00997158242635936\\
358	0.00997158242502828\\
359	0.00997158242367402\\
360	0.00997158242229619\\
361	0.00997158242089439\\
362	0.00997158241946819\\
363	0.00997158241801718\\
364	0.00997158241654094\\
365	0.009971582415039\\
366	0.00997158241351092\\
367	0.00997158241195625\\
368	0.0099715824103745\\
369	0.00997158240876521\\
370	0.00997158240712786\\
371	0.00997158240546195\\
372	0.00997158240376696\\
373	0.00997158240204237\\
374	0.00997158240028761\\
375	0.00997158239850212\\
376	0.00997158239668534\\
377	0.00997158239483664\\
378	0.00997158239295544\\
379	0.00997158239104109\\
380	0.00997158238909294\\
381	0.00997158238711033\\
382	0.00997158238509255\\
383	0.00997158238303891\\
384	0.00997158238094865\\
385	0.00997158237882102\\
386	0.00997158237665522\\
387	0.00997158237445045\\
388	0.00997158237220584\\
389	0.00997158236992053\\
390	0.00997158236759361\\
391	0.00997158236522412\\
392	0.00997158236281107\\
393	0.00997158236035344\\
394	0.00997158235785015\\
395	0.00997158235530007\\
396	0.00997158235270203\\
397	0.00997158235005476\\
398	0.00997158234735696\\
399	0.00997158234460724\\
400	0.00997158234180415\\
401	0.00997158233894615\\
402	0.00997158233603162\\
403	0.00997158233305882\\
404	0.00997158233002592\\
405	0.00997158232693097\\
406	0.00997158232377192\\
407	0.00997158232054661\\
408	0.00997158231725276\\
409	0.00997158231388795\\
410	0.00997158231044967\\
411	0.00997158230693523\\
412	0.0099715823033418\\
413	0.00997158229966632\\
414	0.0099715822959055\\
415	0.0099715822920559\\
416	0.00997158228811384\\
417	0.00997158228407539\\
418	0.00997158227993627\\
419	0.00997158227569181\\
420	0.0099715822713369\\
421	0.00997158226686594\\
422	0.00997158226227277\\
423	0.00997158225755055\\
424	0.00997158225269167\\
425	0.00997158224768756\\
426	0.00997158224252837\\
427	0.00997158223720239\\
428	0.00997158223169493\\
429	0.00997158222598576\\
430	0.00997158222004395\\
431	0.00997158221381703\\
432	0.00997158220721015\\
433	0.00997158220004925\\
434	0.00997158219202694\\
435	0.0099715821826513\\
436	0.00997158217127756\\
437	0.00997158215739005\\
438	0.00997158214122857\\
439	0.00997158212401503\\
440	0.00997158210640971\\
441	0.00997158208840071\\
442	0.00997158206997567\\
443	0.00997158205112189\\
444	0.00997158203182631\\
445	0.00997158201207553\\
446	0.00997158199185592\\
447	0.00997158197115361\\
448	0.00997158194995454\\
449	0.00997158192824437\\
450	0.00997158190600811\\
451	0.00997158188322912\\
452	0.00997158185988651\\
453	0.00997158183594896\\
454	0.0099715818113606\\
455	0.00997158178600892\\
456	0.00997158175965185\\
457	0.00997158173175193\\
458	0.0099715817011006\\
459	0.00997158166499934\\
460	0.00997158161765242\\
461	0.00997158154774152\\
462	0.00997158143699461\\
463	0.00997158126622755\\
464	0.00997158104189281\\
465	0.00997158081343181\\
466	0.00997158058071974\\
467	0.00997158034362549\\
468	0.00997158010201132\\
469	0.00997157985573162\\
470	0.00997157960463319\\
471	0.00997157934855614\\
472	0.00997157908733262\\
473	0.0099715788207843\\
474	0.00997157854872028\\
475	0.0099715782709403\\
476	0.0099715779872406\\
477	0.00997157769740609\\
478	0.00997157740120733\\
479	0.00997157709839908\\
480	0.00997157678871876\\
481	0.00997157647188453\\
482	0.00997157614759318\\
483	0.00997157581551775\\
484	0.0099715754753051\\
485	0.00997157512657271\\
486	0.00997157476890483\\
487	0.00997157440184818\\
488	0.00997157402490682\\
489	0.00997157363753587\\
490	0.00997157323913343\\
491	0.00997157282902933\\
492	0.00997157240646753\\
493	0.00997157197057501\\
494	0.00997157152030013\\
495	0.00997157105428204\\
496	0.00997157057056529\\
497	0.00997157006597804\\
498	0.00997156953481639\\
499	0.00997156896621494\\
500	0.00997156833939104\\
501	0.00997156761655168\\
502	0.00997156673684214\\
503	0.00997156562584849\\
504	0.00997156425665817\\
505	0.00997156274604111\\
506	0.0099715612080005\\
507	0.00997155964131668\\
508	0.00997155804437759\\
509	0.00997155641517961\\
510	0.00997155475184757\\
511	0.00997155305262294\\
512	0.00997155131554636\\
513	0.00997154953838315\\
514	0.00997154771845011\\
515	0.00997154585217563\\
516	0.00997154393393788\\
517	0.00997154195295965\\
518	0.00997153988504464\\
519	0.00997153767096723\\
520	0.00997153516186568\\
521	0.00997153198982832\\
522	0.00997152729962883\\
523	0.00997151937369225\\
524	0.00997149318604075\\
525	0.00997142978152294\\
526	0.00997136434858027\\
527	0.00997129665617338\\
528	0.00997122642403624\\
529	0.00997115341161725\\
530	0.00997107758492663\\
531	0.00997099889826753\\
532	0.00997091691634657\\
533	0.00997083090031376\\
534	0.00997073943226779\\
535	0.00997063970133814\\
536	0.00997052669160612\\
537	0.0099702082159012\\
538	0.00996962328454001\\
539	0.00996902078936174\\
540	0.00996839922675705\\
541	0.00996775689666558\\
542	0.00996709187429563\\
543	0.00996640193383221\\
544	0.00996568437542882\\
545	0.00996493579503008\\
546	0.00996415169729546\\
547	0.00996332579684075\\
548	0.00996244843394521\\
549	0.00996092208856274\\
550	0.00995025740715425\\
551	0.00993968336057744\\
552	0.00992920641838787\\
553	0.00991882893957265\\
554	0.00990854606370902\\
555	0.00989832445380637\\
556	0.00988667401057027\\
557	0.00987483205160817\\
558	0.00986299957738175\\
559	0.00985111088440256\\
560	0.00982940086753967\\
561	0.00980373496254183\\
562	0.00977743457905058\\
563	0.00975058793269213\\
564	0.00972322653700024\\
565	0.00969510362383588\\
566	0.00940580285354155\\
567	0.00909740060809503\\
568	0.00877252040548564\\
569	0.00842896046466404\\
570	0.00806413126271723\\
571	0.00767497738991288\\
572	0.0072578893771046\\
573	0.00680862291380522\\
574	0.00633647419335779\\
575	0.00584531199636737\\
576	0.00533342520441997\\
577	0.00479868293791651\\
578	0.00423933523304903\\
579	0.00367550052531207\\
580	0.00317524304856927\\
581	0.00291649390778095\\
582	0.00266264787724582\\
583	0.0024212958895209\\
584	0.00219950504465807\\
585	0.00198101316635129\\
586	0.0017652741331012\\
587	0.00155340735392807\\
588	0.0013468096195542\\
589	0.00114664925084055\\
590	0.000951372357338311\\
591	0.000764343087165558\\
592	0.00058935041161929\\
593	0.000429546140184966\\
594	0.000289033664663948\\
595	0.000172252650723403\\
596	8.12550264159866e-05\\
597	2.07908715710836e-05\\
598	0\\
599	0\\
600	0\\
};
\addplot [color=mycolor12,solid,forget plot]
  table[row sep=crcr]{%
1	0.00998476612601138\\
2	0.00998476612601125\\
3	0.00998476612601112\\
4	0.00998476612601099\\
5	0.00998476612601085\\
6	0.00998476612601071\\
7	0.00998476612601057\\
8	0.00998476612601043\\
9	0.00998476612601028\\
10	0.00998476612601013\\
11	0.00998476612600998\\
12	0.00998476612600983\\
13	0.00998476612600967\\
14	0.00998476612600952\\
15	0.00998476612600935\\
16	0.00998476612600919\\
17	0.00998476612600903\\
18	0.00998476612600886\\
19	0.00998476612600869\\
20	0.00998476612600851\\
21	0.00998476612600833\\
22	0.00998476612600815\\
23	0.00998476612600797\\
24	0.00998476612600778\\
25	0.00998476612600759\\
26	0.0099847661260074\\
27	0.0099847661260072\\
28	0.009984766126007\\
29	0.0099847661260068\\
30	0.00998476612600659\\
31	0.00998476612600638\\
32	0.00998476612600617\\
33	0.00998476612600595\\
34	0.00998476612600573\\
35	0.00998476612600551\\
36	0.00998476612600528\\
37	0.00998476612600505\\
38	0.00998476612600481\\
39	0.00998476612600457\\
40	0.00998476612600433\\
41	0.00998476612600408\\
42	0.00998476612600383\\
43	0.00998476612600357\\
44	0.00998476612600331\\
45	0.00998476612600305\\
46	0.00998476612600278\\
47	0.0099847661260025\\
48	0.00998476612600223\\
49	0.00998476612600194\\
50	0.00998476612600165\\
51	0.00998476612600136\\
52	0.00998476612600106\\
53	0.00998476612600076\\
54	0.00998476612600045\\
55	0.00998476612600014\\
56	0.00998476612599982\\
57	0.0099847661259995\\
58	0.00998476612599917\\
59	0.00998476612599883\\
60	0.00998476612599849\\
61	0.00998476612599815\\
62	0.00998476612599779\\
63	0.00998476612599744\\
64	0.00998476612599707\\
65	0.0099847661259967\\
66	0.00998476612599633\\
67	0.00998476612599594\\
68	0.00998476612599555\\
69	0.00998476612599516\\
70	0.00998476612599476\\
71	0.00998476612599435\\
72	0.00998476612599393\\
73	0.00998476612599351\\
74	0.00998476612599307\\
75	0.00998476612599264\\
76	0.00998476612599219\\
77	0.00998476612599174\\
78	0.00998476612599128\\
79	0.00998476612599081\\
80	0.00998476612599034\\
81	0.00998476612598985\\
82	0.00998476612598936\\
83	0.00998476612598886\\
84	0.00998476612598835\\
85	0.00998476612598783\\
86	0.00998476612598731\\
87	0.00998476612598677\\
88	0.00998476612598623\\
89	0.00998476612598567\\
90	0.00998476612598511\\
91	0.00998476612598454\\
92	0.00998476612598396\\
93	0.00998476612598337\\
94	0.00998476612598276\\
95	0.00998476612598215\\
96	0.00998476612598153\\
97	0.00998476612598089\\
98	0.00998476612598025\\
99	0.0099847661259796\\
100	0.00998476612597893\\
101	0.00998476612597826\\
102	0.00998476612597757\\
103	0.00998476612597687\\
104	0.00998476612597616\\
105	0.00998476612597543\\
106	0.00998476612597469\\
107	0.00998476612597395\\
108	0.00998476612597318\\
109	0.00998476612597241\\
110	0.00998476612597162\\
111	0.00998476612597082\\
112	0.00998476612597\\
113	0.00998476612596918\\
114	0.00998476612596833\\
115	0.00998476612596747\\
116	0.0099847661259666\\
117	0.00998476612596572\\
118	0.00998476612596482\\
119	0.00998476612596389\\
120	0.00998476612596296\\
121	0.00998476612596201\\
122	0.00998476612596105\\
123	0.00998476612596007\\
124	0.00998476612595907\\
125	0.00998476612595805\\
126	0.00998476612595702\\
127	0.00998476612595597\\
128	0.0099847661259549\\
129	0.00998476612595382\\
130	0.00998476612595271\\
131	0.00998476612595159\\
132	0.00998476612595045\\
133	0.00998476612594928\\
134	0.0099847661259481\\
135	0.0099847661259469\\
136	0.00998476612594567\\
137	0.00998476612594443\\
138	0.00998476612594316\\
139	0.00998476612594187\\
140	0.00998476612594057\\
141	0.00998476612593923\\
142	0.00998476612593788\\
143	0.0099847661259365\\
144	0.0099847661259351\\
145	0.00998476612593367\\
146	0.00998476612593222\\
147	0.00998476612593075\\
148	0.00998476612592925\\
149	0.00998476612592772\\
150	0.00998476612592616\\
151	0.00998476612592459\\
152	0.00998476612592298\\
153	0.00998476612592135\\
154	0.00998476612591968\\
155	0.00998476612591799\\
156	0.00998476612591627\\
157	0.00998476612591452\\
158	0.00998476612591274\\
159	0.00998476612591093\\
160	0.00998476612590909\\
161	0.00998476612590721\\
162	0.00998476612590531\\
163	0.00998476612590336\\
164	0.0099847661259014\\
165	0.00998476612589939\\
166	0.00998476612589735\\
167	0.00998476612589527\\
168	0.00998476612589316\\
169	0.00998476612589101\\
170	0.00998476612588882\\
171	0.0099847661258866\\
172	0.00998476612588434\\
173	0.00998476612588204\\
174	0.0099847661258797\\
175	0.00998476612587731\\
176	0.00998476612587489\\
177	0.00998476612587243\\
178	0.00998476612586992\\
179	0.00998476612586737\\
180	0.00998476612586478\\
181	0.00998476612586214\\
182	0.00998476612585945\\
183	0.00998476612585672\\
184	0.00998476612585395\\
185	0.00998476612585112\\
186	0.00998476612584824\\
187	0.00998476612584532\\
188	0.00998476612584234\\
189	0.00998476612583932\\
190	0.00998476612583624\\
191	0.00998476612583311\\
192	0.00998476612582992\\
193	0.00998476612582668\\
194	0.00998476612582338\\
195	0.00998476612582003\\
196	0.00998476612581661\\
197	0.00998476612581314\\
198	0.00998476612580961\\
199	0.00998476612580602\\
200	0.00998476612580236\\
201	0.00998476612579864\\
202	0.00998476612579486\\
203	0.00998476612579101\\
204	0.00998476612578709\\
205	0.00998476612578311\\
206	0.00998476612577906\\
207	0.00998476612577493\\
208	0.00998476612577074\\
209	0.00998476612576647\\
210	0.00998476612576213\\
211	0.00998476612575772\\
212	0.00998476612575322\\
213	0.00998476612574865\\
214	0.009984766125744\\
215	0.00998476612573928\\
216	0.00998476612573446\\
217	0.00998476612572956\\
218	0.00998476612572458\\
219	0.00998476612571952\\
220	0.00998476612571436\\
221	0.00998476612570912\\
222	0.00998476612570378\\
223	0.00998476612569835\\
224	0.00998476612569283\\
225	0.00998476612568721\\
226	0.00998476612568149\\
227	0.00998476612567567\\
228	0.00998476612566975\\
229	0.00998476612566373\\
230	0.00998476612565761\\
231	0.00998476612565138\\
232	0.00998476612564504\\
233	0.00998476612563859\\
234	0.00998476612563202\\
235	0.00998476612562534\\
236	0.00998476612561855\\
237	0.00998476612561164\\
238	0.0099847661256046\\
239	0.00998476612559745\\
240	0.00998476612559017\\
241	0.00998476612558276\\
242	0.00998476612557522\\
243	0.00998476612556756\\
244	0.00998476612555975\\
245	0.00998476612555182\\
246	0.00998476612554374\\
247	0.00998476612553552\\
248	0.00998476612552716\\
249	0.00998476612551865\\
250	0.00998476612551\\
251	0.00998476612550119\\
252	0.00998476612549223\\
253	0.00998476612548311\\
254	0.00998476612547383\\
255	0.00998476612546439\\
256	0.00998476612545479\\
257	0.00998476612544501\\
258	0.00998476612543507\\
259	0.00998476612542495\\
260	0.00998476612541465\\
261	0.00998476612540418\\
262	0.00998476612539351\\
263	0.00998476612538267\\
264	0.00998476612537163\\
265	0.0099847661253604\\
266	0.00998476612534897\\
267	0.00998476612533734\\
268	0.00998476612532551\\
269	0.00998476612531347\\
270	0.00998476612530121\\
271	0.00998476612528874\\
272	0.00998476612527606\\
273	0.00998476612526314\\
274	0.00998476612525001\\
275	0.00998476612523664\\
276	0.00998476612522303\\
277	0.00998476612520919\\
278	0.00998476612519511\\
279	0.00998476612518077\\
280	0.00998476612516618\\
281	0.00998476612515134\\
282	0.00998476612513623\\
283	0.00998476612512086\\
284	0.00998476612510522\\
285	0.00998476612508929\\
286	0.0099847661250731\\
287	0.00998476612505661\\
288	0.00998476612503983\\
289	0.00998476612502276\\
290	0.00998476612500538\\
291	0.0099847661249877\\
292	0.00998476612496971\\
293	0.0099847661249514\\
294	0.00998476612493276\\
295	0.0099847661249138\\
296	0.0099847661248945\\
297	0.00998476612487486\\
298	0.00998476612485487\\
299	0.00998476612483453\\
300	0.00998476612481383\\
301	0.00998476612479277\\
302	0.00998476612477133\\
303	0.00998476612474951\\
304	0.00998476612472731\\
305	0.00998476612470471\\
306	0.00998476612468171\\
307	0.00998476612465831\\
308	0.00998476612463449\\
309	0.00998476612461025\\
310	0.00998476612458559\\
311	0.00998476612456048\\
312	0.00998476612453494\\
313	0.00998476612450894\\
314	0.00998476612448248\\
315	0.00998476612445555\\
316	0.00998476612442815\\
317	0.00998476612440026\\
318	0.00998476612437188\\
319	0.009984766124343\\
320	0.00998476612431361\\
321	0.0099847661242837\\
322	0.00998476612425326\\
323	0.00998476612422228\\
324	0.00998476612419075\\
325	0.00998476612415867\\
326	0.00998476612412602\\
327	0.00998476612409281\\
328	0.009984766124059\\
329	0.00998476612402459\\
330	0.00998476612398958\\
331	0.00998476612395395\\
332	0.0099847661239177\\
333	0.0099847661238808\\
334	0.00998476612384326\\
335	0.00998476612380506\\
336	0.00998476612376618\\
337	0.00998476612372663\\
338	0.00998476612368637\\
339	0.00998476612364541\\
340	0.00998476612360373\\
341	0.00998476612356133\\
342	0.00998476612351818\\
343	0.00998476612347427\\
344	0.00998476612342959\\
345	0.00998476612338413\\
346	0.00998476612333788\\
347	0.00998476612329082\\
348	0.00998476612324294\\
349	0.00998476612319422\\
350	0.00998476612314465\\
351	0.00998476612309421\\
352	0.0099847661230429\\
353	0.00998476612299069\\
354	0.00998476612293757\\
355	0.00998476612288353\\
356	0.00998476612282855\\
357	0.00998476612277261\\
358	0.00998476612271569\\
359	0.00998476612265779\\
360	0.00998476612259888\\
361	0.00998476612253895\\
362	0.00998476612247797\\
363	0.00998476612241593\\
364	0.00998476612235281\\
365	0.0099847661222886\\
366	0.00998476612222327\\
367	0.0099847661221568\\
368	0.00998476612208917\\
369	0.00998476612202037\\
370	0.00998476612195036\\
371	0.00998476612187914\\
372	0.00998476612180667\\
373	0.00998476612173293\\
374	0.0099847661216579\\
375	0.00998476612158155\\
376	0.00998476612150387\\
377	0.00998476612142482\\
378	0.00998476612134437\\
379	0.0099847661212625\\
380	0.00998476612117919\\
381	0.00998476612109439\\
382	0.00998476612100809\\
383	0.00998476612092025\\
384	0.00998476612083083\\
385	0.0099847661207398\\
386	0.00998476612064715\\
387	0.00998476612055281\\
388	0.00998476612045675\\
389	0.00998476612035895\\
390	0.00998476612025935\\
391	0.00998476612015792\\
392	0.00998476612005461\\
393	0.00998476611994938\\
394	0.00998476611984217\\
395	0.00998476611973294\\
396	0.00998476611962164\\
397	0.0099847661195082\\
398	0.00998476611939257\\
399	0.00998476611927468\\
400	0.00998476611915448\\
401	0.00998476611903188\\
402	0.00998476611890682\\
403	0.00998476611877922\\
404	0.00998476611864899\\
405	0.00998476611851605\\
406	0.00998476611838029\\
407	0.00998476611824164\\
408	0.00998476611809998\\
409	0.0099847661179552\\
410	0.0099847661178072\\
411	0.00998476611765584\\
412	0.00998476611750099\\
413	0.00998476611734252\\
414	0.00998476611718027\\
415	0.00998476611701409\\
416	0.0099847661168438\\
417	0.00998476611666922\\
418	0.00998476611649014\\
419	0.00998476611630634\\
420	0.00998476611611757\\
421	0.00998476611592355\\
422	0.00998476611572399\\
423	0.00998476611551854\\
424	0.0099847661153068\\
425	0.00998476611508829\\
426	0.00998476611486241\\
427	0.00998476611462834\\
428	0.00998476611438484\\
429	0.0099847661141299\\
430	0.0099847661138601\\
431	0.00998476611356959\\
432	0.00998476611324869\\
433	0.0099847661128826\\
434	0.00998476611245178\\
435	0.00998476611193696\\
436	0.00998476611133118\\
437	0.00998476611065367\\
438	0.00998476610994262\\
439	0.00998476610921538\\
440	0.00998476610847146\\
441	0.00998476610771034\\
442	0.00998476610693152\\
443	0.00998476610613445\\
444	0.00998476610531858\\
445	0.00998476610448337\\
446	0.00998476610362824\\
447	0.00998476610275261\\
448	0.00998476610185584\\
449	0.0099847661009372\\
450	0.00998476609999576\\
451	0.00998476609903\\
452	0.00998476609803725\\
453	0.00998476609701211\\
454	0.00998476609594338\\
455	0.00998476609480752\\
456	0.00998476609355561\\
457	0.00998476609208882\\
458	0.00998476609021732\\
459	0.00998476608760555\\
460	0.00998476608373925\\
461	0.00998476607801881\\
462	0.00998476607014061\\
463	0.00998476606073503\\
464	0.00998476605115606\\
465	0.00998476604139841\\
466	0.00998476603145654\\
467	0.00998476602132461\\
468	0.00998476601099649\\
469	0.00998476600046571\\
470	0.00998476598972551\\
471	0.00998476597876875\\
472	0.00998476596758783\\
473	0.00998476595617469\\
474	0.00998476594452093\\
475	0.00998476593261786\\
476	0.00998476592045632\\
477	0.0099847659080265\\
478	0.00998476589531791\\
479	0.00998476588231931\\
480	0.00998476586901861\\
481	0.0099847658554028\\
482	0.00998476584145781\\
483	0.00998476582716846\\
484	0.00998476581251822\\
485	0.00998476579748913\\
486	0.00998476578206153\\
487	0.00998476576621385\\
488	0.00998476574992225\\
489	0.00998476573316013\\
490	0.00998476571589736\\
491	0.0099847656980989\\
492	0.00998476567972207\\
493	0.00998476566071112\\
494	0.00998476564098614\\
495	0.00998476562042082\\
496	0.00998476559879951\\
497	0.0099847655757395\\
498	0.00998476555056474\\
499	0.00998476552213733\\
500	0.00998476548872632\\
501	0.00998476544816116\\
502	0.00998476539873531\\
503	0.00998476534112569\\
504	0.00998476527941104\\
505	0.00998476521657973\\
506	0.00998476515257641\\
507	0.00998476508733086\\
508	0.00998476502076274\\
509	0.00998476495279559\\
510	0.00998476488335207\\
511	0.00998476481234044\\
512	0.00998476473964256\\
513	0.00998476466508539\\
514	0.00998476458837248\\
515	0.0099847645089216\\
516	0.00998476442548878\\
517	0.00998476433532989\\
518	0.00998476423243209\\
519	0.00998476410410157\\
520	0.00998476392536108\\
521	0.00998476365277688\\
522	0.00998476322724478\\
523	0.00998476261074155\\
524	0.00998476185929397\\
525	0.00998476107951758\\
526	0.00998476026608737\\
527	0.00998475941368839\\
528	0.00998475851976169\\
529	0.00998475758590979\\
530	0.00998475660914964\\
531	0.00998475557133194\\
532	0.00998475443618455\\
533	0.00998475313530915\\
534	0.00998475155561873\\
535	0.00998474955614854\\
536	0.00998474707681101\\
537	0.00998474431322528\\
538	0.00998474148463942\\
539	0.00998473858561031\\
540	0.00998473560989778\\
541	0.00998473254958401\\
542	0.00998472939203223\\
543	0.0099847261133409\\
544	0.00998472266845293\\
545	0.00998471897603467\\
546	0.00998471489917038\\
547	0.00998471023443736\\
548	0.00998470478673389\\
549	0.009984698629892\\
550	0.0099846922355815\\
551	0.00998468539895788\\
552	0.00998467761634069\\
553	0.00998466766639784\\
554	0.00998465284881902\\
555	0.00998462839384316\\
556	0.00998451835371763\\
557	0.00998438713828728\\
558	0.00998424091014161\\
559	0.00998407048417542\\
560	0.00998337245231295\\
561	0.0099824607661268\\
562	0.00998150752370361\\
563	0.00998050497714601\\
564	0.00997944065250886\\
565	0.00997829668708648\\
566	0.00996445051854514\\
567	0.00995056165887344\\
568	0.00993686172377713\\
569	0.00992336079454392\\
570	0.00991007448540605\\
571	0.0098970254209339\\
572	0.0098842472847136\\
573	0.00987180223381704\\
574	0.00986010348666058\\
575	0.00984931761907658\\
576	0.00983942691884809\\
577	0.00983036670881545\\
578	0.00982193005125355\\
579	0.0097931048823283\\
580	0.00967661252408177\\
581	0.00930277543452092\\
582	0.00890533237643708\\
583	0.00848053657077329\\
584	0.00802397360009435\\
585	0.00755205906109305\\
586	0.00706559415643771\\
587	0.00656389548252311\\
588	0.00604646101757425\\
589	0.00551317570106527\\
590	0.00496674996040426\\
591	0.00440647859325226\\
592	0.00383129818654383\\
593	0.00323992338914098\\
594	0.0026306372037739\\
595	0.00200104994316786\\
596	0.00135174293718153\\
597	0.000683662792068022\\
598	0\\
599	0\\
600	0\\
};
\addplot [color=mycolor13,solid,forget plot]
  table[row sep=crcr]{%
1	0.000484240466903288\\
2	0.000484240466903338\\
3	0.000484240466903389\\
4	0.000484240466903441\\
5	0.000484240466903493\\
6	0.000484240466903547\\
7	0.000484240466903602\\
8	0.000484240466903656\\
9	0.000484240466903713\\
10	0.000484240466903771\\
11	0.000484240466903829\\
12	0.000484240466903889\\
13	0.00048424046690395\\
14	0.000484240466904013\\
15	0.000484240466904075\\
16	0.000484240466904139\\
17	0.000484240466904205\\
18	0.000484240466904272\\
19	0.000484240466904339\\
20	0.000484240466904408\\
21	0.000484240466904478\\
22	0.000484240466904549\\
23	0.000484240466904622\\
24	0.000484240466904695\\
25	0.000484240466904771\\
26	0.000484240466904848\\
27	0.000484240466904927\\
28	0.000484240466905006\\
29	0.000484240466905086\\
30	0.000484240466905168\\
31	0.000484240466905252\\
32	0.000484240466905337\\
33	0.000484240466905425\\
34	0.000484240466905513\\
35	0.000484240466905602\\
36	0.000484240466905693\\
37	0.000484240466905788\\
38	0.000484240466905882\\
39	0.000484240466905979\\
40	0.000484240466906078\\
41	0.000484240466906178\\
42	0.000484240466906279\\
43	0.000484240466906384\\
44	0.000484240466906488\\
45	0.000484240466906596\\
46	0.000484240466906706\\
47	0.000484240466906817\\
48	0.000484240466906931\\
49	0.000484240466907046\\
50	0.000484240466907165\\
51	0.000484240466907284\\
52	0.000484240466907405\\
53	0.000484240466907529\\
54	0.000484240466907655\\
55	0.000484240466907784\\
56	0.000484240466907914\\
57	0.000484240466908047\\
58	0.000484240466908183\\
59	0.000484240466908321\\
60	0.000484240466908462\\
61	0.000484240466908604\\
62	0.000484240466908751\\
63	0.000484240466908898\\
64	0.000484240466909049\\
65	0.000484240466909203\\
66	0.000484240466909359\\
67	0.000484240466909518\\
68	0.00048424046690968\\
69	0.000484240466909845\\
70	0.000484240466910012\\
71	0.000484240466910183\\
72	0.000484240466910357\\
73	0.000484240466910534\\
74	0.000484240466910714\\
75	0.000484240466910899\\
76	0.000484240466911085\\
77	0.000484240466911275\\
78	0.000484240466911469\\
79	0.000484240466911665\\
80	0.000484240466911866\\
81	0.00048424046691207\\
82	0.000484240466912277\\
83	0.000484240466912489\\
84	0.000484240466912703\\
85	0.000484240466912923\\
86	0.000484240466913147\\
87	0.000484240466913373\\
88	0.000484240466913604\\
89	0.00048424046691384\\
90	0.000484240466914079\\
91	0.000484240466914324\\
92	0.000484240466914572\\
93	0.000484240466914824\\
94	0.000484240466915081\\
95	0.000484240466915342\\
96	0.00048424046691561\\
97	0.00048424046691588\\
98	0.000484240466916157\\
99	0.000484240466916438\\
100	0.000484240466916723\\
101	0.000484240466917015\\
102	0.000484240466917311\\
103	0.000484240466917613\\
104	0.00048424046691792\\
105	0.000484240466918233\\
106	0.000484240466918552\\
107	0.000484240466918875\\
108	0.000484240466919205\\
109	0.00048424046691954\\
110	0.000484240466919883\\
111	0.000484240466920231\\
112	0.000484240466920584\\
113	0.000484240466920946\\
114	0.000484240466921311\\
115	0.000484240466921684\\
116	0.000484240466922066\\
117	0.000484240466922452\\
118	0.000484240466922847\\
119	0.000484240466923247\\
120	0.000484240466923655\\
121	0.000484240466924071\\
122	0.000484240466924494\\
123	0.000484240466924925\\
124	0.000484240466925363\\
125	0.000484240466925809\\
126	0.000484240466926262\\
127	0.000484240466926725\\
128	0.000484240466927195\\
129	0.000484240466927674\\
130	0.000484240466928163\\
131	0.000484240466928658\\
132	0.000484240466929163\\
133	0.000484240466929677\\
134	0.000484240466930202\\
135	0.000484240466930734\\
136	0.000484240466931277\\
137	0.000484240466931828\\
138	0.000484240466932391\\
139	0.000484240466932962\\
140	0.000484240466933544\\
141	0.000484240466934137\\
142	0.00048424046693474\\
143	0.000484240466935354\\
144	0.00048424046693598\\
145	0.000484240466936615\\
146	0.000484240466937264\\
147	0.000484240466937923\\
148	0.000484240466938593\\
149	0.000484240466939276\\
150	0.000484240466939972\\
151	0.00048424046694068\\
152	0.000484240466941401\\
153	0.000484240466942133\\
154	0.00048424046694288\\
155	0.000484240466943641\\
156	0.000484240466944414\\
157	0.000484240466945202\\
158	0.000484240466946003\\
159	0.000484240466946819\\
160	0.000484240466947649\\
161	0.000484240466948494\\
162	0.000484240466949355\\
163	0.000484240466950231\\
164	0.000484240466951122\\
165	0.00048424046695203\\
166	0.000484240466952954\\
167	0.000484240466953896\\
168	0.000484240466954853\\
169	0.000484240466955828\\
170	0.000484240466956819\\
171	0.000484240466957829\\
172	0.000484240466958857\\
173	0.000484240466959902\\
174	0.000484240466960968\\
175	0.000484240466962053\\
176	0.000484240466963156\\
177	0.000484240466964281\\
178	0.000484240466965423\\
179	0.000484240466966587\\
180	0.000484240466967772\\
181	0.000484240466968979\\
182	0.000484240466970208\\
183	0.000484240466971458\\
184	0.00048424046697273\\
185	0.000484240466974026\\
186	0.000484240466975344\\
187	0.000484240466976687\\
188	0.000484240466978053\\
189	0.000484240466979444\\
190	0.00048424046698086\\
191	0.000484240466982302\\
192	0.000484240466983769\\
193	0.000484240466985264\\
194	0.000484240466986784\\
195	0.000484240466988332\\
196	0.000484240466989907\\
197	0.000484240466991513\\
198	0.000484240466993146\\
199	0.000484240466994807\\
200	0.000484240466996501\\
201	0.000484240466998223\\
202	0.000484240466999977\\
203	0.000484240467001762\\
204	0.00048424046700358\\
205	0.00048424046700543\\
206	0.000484240467007315\\
207	0.000484240467009232\\
208	0.000484240467011185\\
209	0.000484240467013172\\
210	0.000484240467015197\\
211	0.000484240467017256\\
212	0.000484240467019352\\
213	0.000484240467021488\\
214	0.000484240467023662\\
215	0.000484240467025873\\
216	0.000484240467028126\\
217	0.00048424046703042\\
218	0.000484240467032754\\
219	0.000484240467035132\\
220	0.000484240467037552\\
221	0.000484240467040016\\
222	0.000484240467042524\\
223	0.000484240467045077\\
224	0.000484240467047678\\
225	0.000484240467050325\\
226	0.00048424046705302\\
227	0.000484240467055764\\
228	0.000484240467058556\\
229	0.0004842404670614\\
230	0.000484240467064296\\
231	0.000484240467067244\\
232	0.000484240467070247\\
233	0.000484240467073302\\
234	0.000484240467076414\\
235	0.000484240467079583\\
236	0.000484240467082807\\
237	0.000484240467086091\\
238	0.000484240467089436\\
239	0.00048424046709284\\
240	0.000484240467096307\\
241	0.000484240467099838\\
242	0.000484240467103431\\
243	0.000484240467107091\\
244	0.000484240467110817\\
245	0.000484240467114611\\
246	0.000484240467118475\\
247	0.000484240467122408\\
248	0.000484240467126413\\
249	0.000484240467130492\\
250	0.000484240467134645\\
251	0.000484240467138874\\
252	0.00048424046714318\\
253	0.000484240467147566\\
254	0.00048424046715203\\
255	0.000484240467156577\\
256	0.000484240467161208\\
257	0.000484240467165922\\
258	0.000484240467170724\\
259	0.000484240467175612\\
260	0.000484240467180592\\
261	0.000484240467185661\\
262	0.000484240467190824\\
263	0.000484240467196083\\
264	0.000484240467201438\\
265	0.00048424046720689\\
266	0.000484240467212444\\
267	0.000484240467218099\\
268	0.000484240467223858\\
269	0.000484240467229725\\
270	0.000484240467235697\\
271	0.000484240467241782\\
272	0.000484240467247977\\
273	0.000484240467254287\\
274	0.000484240467260715\\
275	0.000484240467267261\\
276	0.000484240467273926\\
277	0.000484240467280716\\
278	0.000484240467287631\\
279	0.000484240467294674\\
280	0.000484240467301846\\
281	0.000484240467309154\\
282	0.000484240467316596\\
283	0.000484240467324175\\
284	0.000484240467331895\\
285	0.000484240467339758\\
286	0.000484240467347768\\
287	0.000484240467355925\\
288	0.000484240467364235\\
289	0.000484240467372699\\
290	0.000484240467381321\\
291	0.000484240467390103\\
292	0.000484240467399049\\
293	0.000484240467408161\\
294	0.000484240467417442\\
295	0.000484240467426897\\
296	0.000484240467436528\\
297	0.000484240467446339\\
298	0.000484240467456333\\
299	0.000484240467466513\\
300	0.000484240467476884\\
301	0.000484240467487449\\
302	0.00048424046749821\\
303	0.000484240467509173\\
304	0.00048424046752034\\
305	0.000484240467531717\\
306	0.000484240467543307\\
307	0.000484240467555115\\
308	0.000484240467567143\\
309	0.000484240467579396\\
310	0.000484240467591879\\
311	0.000484240467604594\\
312	0.000484240467617549\\
313	0.000484240467630747\\
314	0.000484240467644192\\
315	0.00048424046765789\\
316	0.000484240467671845\\
317	0.00048424046768606\\
318	0.000484240467700543\\
319	0.000484240467715297\\
320	0.000484240467730329\\
321	0.000484240467745641\\
322	0.000484240467761242\\
323	0.000484240467777135\\
324	0.000484240467793326\\
325	0.000484240467809823\\
326	0.000484240467826628\\
327	0.000484240467843747\\
328	0.000484240467861189\\
329	0.000484240467878957\\
330	0.000484240467897058\\
331	0.0004842404679155\\
332	0.000484240467934286\\
333	0.000484240467953425\\
334	0.000484240467972922\\
335	0.000484240467992787\\
336	0.000484240468013021\\
337	0.000484240468033635\\
338	0.000484240468054636\\
339	0.000484240468076028\\
340	0.000484240468097822\\
341	0.000484240468120023\\
342	0.000484240468142638\\
343	0.000484240468165678\\
344	0.000484240468189149\\
345	0.000484240468213057\\
346	0.000484240468237412\\
347	0.000484240468262221\\
348	0.000484240468287495\\
349	0.000484240468313239\\
350	0.000484240468339464\\
351	0.000484240468366179\\
352	0.000484240468393391\\
353	0.000484240468421111\\
354	0.000484240468449347\\
355	0.00048424046847811\\
356	0.000484240468507407\\
357	0.000484240468537252\\
358	0.000484240468567653\\
359	0.000484240468598618\\
360	0.000484240468630161\\
361	0.000484240468662293\\
362	0.000484240468695021\\
363	0.000484240468728362\\
364	0.000484240468762321\\
365	0.000484240468796915\\
366	0.000484240468832154\\
367	0.00048424046886805\\
368	0.000484240468904617\\
369	0.000484240468941867\\
370	0.000484240468979814\\
371	0.000484240469018471\\
372	0.000484240469057852\\
373	0.000484240469097974\\
374	0.000484240469138849\\
375	0.000484240469180494\\
376	0.000484240469222925\\
377	0.00048424046926616\\
378	0.000484240469310213\\
379	0.000484240469355105\\
380	0.000484240469400852\\
381	0.000484240469447475\\
382	0.000484240469494994\\
383	0.000484240469543428\\
384	0.000484240469592801\\
385	0.000484240469643136\\
386	0.000484240469694458\\
387	0.000484240469746791\\
388	0.000484240469800161\\
389	0.000484240469854599\\
390	0.000484240469910134\\
391	0.000484240469966797\\
392	0.00048424047002462\\
393	0.00048424047008364\\
394	0.000484240470143896\\
395	0.000484240470205425\\
396	0.000484240470268271\\
397	0.000484240470332477\\
398	0.000484240470398088\\
399	0.000484240470465156\\
400	0.000484240470533727\\
401	0.000484240470603858\\
402	0.000484240470675606\\
403	0.00048424047074903\\
404	0.000484240470824197\\
405	0.000484240470901179\\
406	0.000484240470980048\\
407	0.000484240471060881\\
408	0.000484240471143761\\
409	0.000484240471228782\\
410	0.000484240471316043\\
411	0.000484240471405652\\
412	0.000484240471497727\\
413	0.000484240471592404\\
414	0.000484240471689826\\
415	0.000484240471790159\\
416	0.000484240471893589\\
417	0.000484240472000334\\
418	0.00048424047211065\\
419	0.000484240472224868\\
420	0.00048424047234344\\
421	0.000484240472467045\\
422	0.000484240472596777\\
423	0.000484240472734483\\
424	0.000484240472883296\\
425	0.000484240473048316\\
426	0.000484240473237145\\
427	0.000484240473459371\\
428	0.000484240473723402\\
429	0.000484240474029596\\
430	0.000484240474363645\\
431	0.000484240474704589\\
432	0.000484240475052085\\
433	0.000484240475405585\\
434	0.000484240475764482\\
435	0.000484240476128623\\
436	0.000484240476499017\\
437	0.000484240476877848\\
438	0.000484240477266483\\
439	0.000484240477665547\\
440	0.000484240478076063\\
441	0.000484240478499937\\
442	0.000484240478941073\\
443	0.000484240479407776\\
444	0.00048424047991768\\
445	0.000484240480507274\\
446	0.000484240481248529\\
447	0.000484240482273253\\
448	0.000484240483796312\\
449	0.000484240486103035\\
450	0.0004842404894243\\
451	0.000484240493635419\\
452	0.000484240498062313\\
453	0.000484240502573705\\
454	0.000484240507169916\\
455	0.000484240511850491\\
456	0.000484240516613272\\
457	0.000484240521452686\\
458	0.000484240526357104\\
459	0.000484240531306371\\
460	0.000484240536273769\\
461	0.000484240541240519\\
462	0.000484240546228037\\
463	0.000484240551311235\\
464	0.000484240556492951\\
465	0.000484240561776135\\
466	0.000484240567163889\\
467	0.000484240572659478\\
468	0.000484240578266315\\
469	0.000484240583988024\\
470	0.000484240589828514\\
471	0.000484240595792018\\
472	0.000484240601882934\\
473	0.000484240608105622\\
474	0.000484240614464737\\
475	0.000484240620965255\\
476	0.000484240627612496\\
477	0.000484240634412163\\
478	0.000484240641370363\\
479	0.000484240648493667\\
480	0.000484240655789144\\
481	0.000484240663264423\\
482	0.000484240670927752\\
483	0.000484240678788071\\
484	0.000484240686855094\\
485	0.00048424069513941\\
486	0.000484240703652612\\
487	0.000484240712407421\\
488	0.000484240721417891\\
489	0.000484240730699638\\
490	0.000484240740270217\\
491	0.000484240750149749\\
492	0.00048424076036209\\
493	0.000484240770937182\\
494	0.000484240781915921\\
495	0.00048424079336023\\
496	0.000484240805373368\\
497	0.000484240818138671\\
498	0.000484240831986899\\
499	0.000484240847495088\\
500	0.000484240865582196\\
501	0.000484240887462403\\
502	0.00048424091413874\\
503	0.000484240945128151\\
504	0.000484240977435586\\
505	0.000484241010360868\\
506	0.000484241043948233\\
507	0.000484241078241602\\
508	0.000484241113276388\\
509	0.000484241149094674\\
510	0.000484241185749511\\
511	0.000484241223317212\\
512	0.000484241261926852\\
513	0.000484241301830343\\
514	0.000484241343566185\\
515	0.000484241388329228\\
516	0.000484241438764512\\
517	0.000484241500542035\\
518	0.000484241585077113\\
519	0.000484241712968724\\
520	0.000484241914290421\\
521	0.000484242213187589\\
522	0.000484242580237095\\
523	0.000484242952253886\\
524	0.000484243336263891\\
525	0.000484243733975993\\
526	0.000484244147872262\\
527	0.000484244581210334\\
528	0.000484245037161741\\
529	0.000484245516419022\\
530	0.000484246016790978\\
531	0.000484246545701493\\
532	0.000484247119218224\\
533	0.000484247767406508\\
534	0.000484248541306853\\
535	0.000484249512882124\\
536	0.000484250731440822\\
537	0.000484252078883766\\
538	0.000484253458944545\\
539	0.000484254874471593\\
540	0.000484256328862317\\
541	0.000484257826378404\\
542	0.000484259373029604\\
543	0.000484260979213294\\
544	0.00048426266514276\\
545	0.000484264468980179\\
546	0.000484266455953145\\
547	0.000484268725433269\\
548	0.00048427138412627\\
549	0.000484274421921086\\
550	0.000484277630733646\\
551	0.000484281192722443\\
552	0.000484285569045597\\
553	0.000484291838485129\\
554	0.000484302055719857\\
555	0.000484318648444575\\
556	0.00048433717149597\\
557	0.000484358312049961\\
558	0.000484384433837813\\
559	0.000484419741214524\\
560	0.000484467972308084\\
561	0.000484518384319466\\
562	0.000484571833803974\\
563	0.00048463043821816\\
564	0.000484698204292027\\
565	0.000484780473870265\\
566	0.000484872232915441\\
567	0.000484965833244364\\
568	0.000485062002730881\\
569	0.00048516191722124\\
570	0.000485267787354841\\
571	0.000485383204690684\\
572	0.000485511922586414\\
573	0.000485642971766619\\
574	0.000485775812999213\\
575	0.000485914470174083\\
576	0.000486084229404065\\
577	0.000486346843596949\\
578	0.000486874060292007\\
579	0.000499332124294703\\
580	0.000549606817473229\\
581	0.00088240546943094\\
582	0.00124000461749179\\
583	0.00162646793862394\\
584	0.0020440485140991\\
585	0.00247787686417299\\
586	0.00292892988002479\\
587	0.00339798959748976\\
588	0.00388566154600113\\
589	0.00439214851677062\\
590	0.00491474520087469\\
591	0.00545443621534336\\
592	0.00601233132542992\\
593	0.00658977217077654\\
594	0.00718858827196951\\
595	0.00781118959413535\\
596	0.00845712341615153\\
597	0.00912599392411371\\
598	0.00981478197180822\\
599	0\\
600	0\\
};
\addplot [color=mycolor14,solid,forget plot]
  table[row sep=crcr]{%
1	2.808615381101e-05\\
2	2.80861538121573e-05\\
3	2.8086153813325e-05\\
4	2.80861538145132e-05\\
5	2.80861538157218e-05\\
6	2.80861538169526e-05\\
7	2.80861538182056e-05\\
8	2.80861538194807e-05\\
9	2.80861538207797e-05\\
10	2.80861538221009e-05\\
11	2.80861538234459e-05\\
12	2.8086153824813e-05\\
13	2.80861538262075e-05\\
14	2.80861538276258e-05\\
15	2.80861538290697e-05\\
16	2.80861538305392e-05\\
17	2.80861538320342e-05\\
18	2.80861538335565e-05\\
19	2.80861538351061e-05\\
20	2.80861538366846e-05\\
21	2.80861538382905e-05\\
22	2.80861538399236e-05\\
23	2.80861538415874e-05\\
24	2.80861538432801e-05\\
25	2.80861538450036e-05\\
26	2.80861538467577e-05\\
27	2.80861538485443e-05\\
28	2.80861538503615e-05\\
29	2.80861538522111e-05\\
30	2.80861538540948e-05\\
31	2.80861538560109e-05\\
32	2.80861538579611e-05\\
33	2.80861538599471e-05\\
34	2.80861538619689e-05\\
35	2.80861538640265e-05\\
36	2.80861538661198e-05\\
37	2.80861538682524e-05\\
38	2.80861538704208e-05\\
39	2.80861538726301e-05\\
40	2.80861538748769e-05\\
41	2.80861538771646e-05\\
42	2.80861538794933e-05\\
43	2.80861538818645e-05\\
44	2.80861538842767e-05\\
45	2.80861538867332e-05\\
46	2.80861538892323e-05\\
47	2.80861538917774e-05\\
48	2.80861538943669e-05\\
49	2.80861538970023e-05\\
50	2.80861538996855e-05\\
51	2.80861539024165e-05\\
52	2.80861539051969e-05\\
53	2.8086153908025e-05\\
54	2.80861539109059e-05\\
55	2.80861539138363e-05\\
56	2.80861539168213e-05\\
57	2.80861539198574e-05\\
58	2.8086153922948e-05\\
59	2.80861539260949e-05\\
60	2.80861539292981e-05\\
61	2.80861539325574e-05\\
62	2.80861539358748e-05\\
63	2.80861539392518e-05\\
64	2.80861539426902e-05\\
65	2.80861539461883e-05\\
66	2.80861539497494e-05\\
67	2.80861539533754e-05\\
68	2.80861539570643e-05\\
69	2.80861539608198e-05\\
70	2.80861539646418e-05\\
71	2.80861539685319e-05\\
72	2.80861539724919e-05\\
73	2.80861539765236e-05\\
74	2.80861539806268e-05\\
75	2.80861539848016e-05\\
76	2.80861539890532e-05\\
77	2.80861539933798e-05\\
78	2.8086153997783e-05\\
79	2.80861540022647e-05\\
80	2.80861540068265e-05\\
81	2.80861540114701e-05\\
82	2.80861540161973e-05\\
83	2.8086154021008e-05\\
84	2.80861540259039e-05\\
85	2.80861540308885e-05\\
86	2.80861540359617e-05\\
87	2.80861540411252e-05\\
88	2.80861540463809e-05\\
89	2.80861540517302e-05\\
90	2.80861540571751e-05\\
91	2.80861540627171e-05\\
92	2.8086154068358e-05\\
93	2.80861540740994e-05\\
94	2.80861540799432e-05\\
95	2.80861540858909e-05\\
96	2.8086154091946e-05\\
97	2.80861540981085e-05\\
98	2.80861541043802e-05\\
99	2.80861541107643e-05\\
100	2.80861541172626e-05\\
101	2.80861541238769e-05\\
102	2.80861541306088e-05\\
103	2.80861541374617e-05\\
104	2.80861541444357e-05\\
105	2.80861541515341e-05\\
106	2.80861541587604e-05\\
107	2.80861541661145e-05\\
108	2.80861541735999e-05\\
109	2.80861541812199e-05\\
110	2.80861541889746e-05\\
111	2.80861541968674e-05\\
112	2.80861542049017e-05\\
113	2.80861542130792e-05\\
114	2.80861542214033e-05\\
115	2.80861542298757e-05\\
116	2.80861542384981e-05\\
117	2.80861542472757e-05\\
118	2.80861542562083e-05\\
119	2.8086154265303e-05\\
120	2.80861542745578e-05\\
121	2.80861542839781e-05\\
122	2.80861542935653e-05\\
123	2.80861543033248e-05\\
124	2.80861543132598e-05\\
125	2.80861543233705e-05\\
126	2.80861543336618e-05\\
127	2.80861543441355e-05\\
128	2.80861543547967e-05\\
129	2.80861543656489e-05\\
130	2.80861543766955e-05\\
131	2.8086154387938e-05\\
132	2.808615439938e-05\\
133	2.80861544110283e-05\\
134	2.80861544228828e-05\\
135	2.80861544349505e-05\\
136	2.80861544472329e-05\\
137	2.80861544597335e-05\\
138	2.80861544724575e-05\\
139	2.80861544854098e-05\\
140	2.80861544985924e-05\\
141	2.80861545120101e-05\\
142	2.80861545256665e-05\\
143	2.80861545395684e-05\\
144	2.80861545537175e-05\\
145	2.80861545681189e-05\\
146	2.80861545827777e-05\\
147	2.80861545976973e-05\\
148	2.80861546128846e-05\\
149	2.80861546283429e-05\\
150	2.80861546440756e-05\\
151	2.80861546600896e-05\\
152	2.80861546763918e-05\\
153	2.8086154692982e-05\\
154	2.80861547098706e-05\\
155	2.80861547270591e-05\\
156	2.80861547445563e-05\\
157	2.80861547623654e-05\\
158	2.80861547804915e-05\\
159	2.80861547989416e-05\\
160	2.80861548177224e-05\\
161	2.80861548368372e-05\\
162	2.80861548562948e-05\\
163	2.80861548761001e-05\\
164	2.80861548962583e-05\\
165	2.80861549167762e-05\\
166	2.80861549376605e-05\\
167	2.808615495892e-05\\
168	2.80861549805578e-05\\
169	2.80861550025827e-05\\
170	2.80861550250013e-05\\
171	2.80861550478205e-05\\
172	2.80861550710472e-05\\
173	2.80861550946898e-05\\
174	2.80861551187552e-05\\
175	2.80861551432501e-05\\
176	2.80861551681832e-05\\
177	2.80861551935629e-05\\
178	2.80861552193961e-05\\
179	2.80861552456912e-05\\
180	2.80861552724568e-05\\
181	2.80861552997015e-05\\
182	2.80861553274337e-05\\
183	2.8086155355662e-05\\
184	2.80861553843966e-05\\
185	2.80861554136442e-05\\
186	2.80861554434153e-05\\
187	2.80861554737199e-05\\
188	2.80861555045666e-05\\
189	2.80861555359673e-05\\
190	2.80861555679289e-05\\
191	2.80861556004633e-05\\
192	2.80861556335807e-05\\
193	2.80861556672913e-05\\
194	2.80861557016054e-05\\
195	2.80861557365348e-05\\
196	2.80861557720917e-05\\
197	2.80861558082843e-05\\
198	2.80861558451265e-05\\
199	2.80861558826301e-05\\
200	2.80861559208071e-05\\
201	2.80861559596676e-05\\
202	2.80861559992254e-05\\
203	2.80861560394923e-05\\
204	2.80861560804838e-05\\
205	2.80861561222099e-05\\
206	2.80861561646862e-05\\
207	2.80861562079244e-05\\
208	2.808615625194e-05\\
209	2.80861562967465e-05\\
210	2.80861563423577e-05\\
211	2.80861563887889e-05\\
212	2.80861564360554e-05\\
213	2.80861564841725e-05\\
214	2.80861565331522e-05\\
215	2.80861565830149e-05\\
216	2.80861566337761e-05\\
217	2.80861566854492e-05\\
218	2.80861567380514e-05\\
219	2.80861567916031e-05\\
220	2.8086156846118e-05\\
221	2.80861569016166e-05\\
222	2.8086156958114e-05\\
223	2.80861570156292e-05\\
224	2.80861570741826e-05\\
225	2.80861571337911e-05\\
226	2.80861571944754e-05\\
227	2.8086157256254e-05\\
228	2.80861573191475e-05\\
229	2.80861573831781e-05\\
230	2.80861574483645e-05\\
231	2.80861575147271e-05\\
232	2.80861575822899e-05\\
233	2.80861576510749e-05\\
234	2.80861577211026e-05\\
235	2.8086157792397e-05\\
236	2.80861578649818e-05\\
237	2.80861579388793e-05\\
238	2.80861580141149e-05\\
239	2.80861580907143e-05\\
240	2.80861581686996e-05\\
241	2.80861582480999e-05\\
242	2.80861583289389e-05\\
243	2.8086158411244e-05\\
244	2.80861584950424e-05\\
245	2.80861585803614e-05\\
246	2.808615866723e-05\\
247	2.80861587556754e-05\\
248	2.80861588457284e-05\\
249	2.80861589374196e-05\\
250	2.80861590307763e-05\\
251	2.80861591258309e-05\\
252	2.80861592226158e-05\\
253	2.80861593211634e-05\\
254	2.80861594215043e-05\\
255	2.80861595236743e-05\\
256	2.80861596277059e-05\\
257	2.80861597336349e-05\\
258	2.80861598414953e-05\\
259	2.80861599513246e-05\\
260	2.80861600631604e-05\\
261	2.80861601770367e-05\\
262	2.80861602929962e-05\\
263	2.80861604110746e-05\\
264	2.80861605313111e-05\\
265	2.80861606537502e-05\\
266	2.80861607784293e-05\\
267	2.8086160905391e-05\\
268	2.80861610346796e-05\\
269	2.80861611663394e-05\\
270	2.80861613004132e-05\\
271	2.8086161436945e-05\\
272	2.80861615759846e-05\\
273	2.80861617175778e-05\\
274	2.80861618617707e-05\\
275	2.80861620086161e-05\\
276	2.80861621581618e-05\\
277	2.80861623104571e-05\\
278	2.80861624655585e-05\\
279	2.80861626235135e-05\\
280	2.80861627843802e-05\\
281	2.8086162948213e-05\\
282	2.80861631150649e-05\\
283	2.80861632849972e-05\\
284	2.80861634580662e-05\\
285	2.80861636343298e-05\\
286	2.80861638138511e-05\\
287	2.80861639966915e-05\\
288	2.80861641829124e-05\\
289	2.80861643725784e-05\\
290	2.80861645657562e-05\\
291	2.80861647625104e-05\\
292	2.80861649629093e-05\\
293	2.80861651670228e-05\\
294	2.80861653749207e-05\\
295	2.80861655866747e-05\\
296	2.80861658023598e-05\\
297	2.80861660220475e-05\\
298	2.80861662458163e-05\\
299	2.80861664737411e-05\\
300	2.80861667059039e-05\\
301	2.80861669423831e-05\\
302	2.80861671832603e-05\\
303	2.8086167428621e-05\\
304	2.80861676785469e-05\\
305	2.80861679331266e-05\\
306	2.80861681924488e-05\\
307	2.80861684566022e-05\\
308	2.80861687256771e-05\\
309	2.80861689997707e-05\\
310	2.80861692789733e-05\\
311	2.8086169563382e-05\\
312	2.80861698530975e-05\\
313	2.80861701482204e-05\\
314	2.80861704488495e-05\\
315	2.80861707550905e-05\\
316	2.80861710670491e-05\\
317	2.80861713848309e-05\\
318	2.80861717085486e-05\\
319	2.80861720383129e-05\\
320	2.80861723742346e-05\\
321	2.8086172716433e-05\\
322	2.80861730650224e-05\\
323	2.80861734201238e-05\\
324	2.80861737818582e-05\\
325	2.80861741503501e-05\\
326	2.80861745257256e-05\\
327	2.80861749081126e-05\\
328	2.80861752976423e-05\\
329	2.80861756944477e-05\\
330	2.808617609866e-05\\
331	2.80861765104226e-05\\
332	2.80861769298699e-05\\
333	2.8086177357147e-05\\
334	2.80861777923987e-05\\
335	2.80861782357716e-05\\
336	2.80861786874158e-05\\
337	2.8086179147483e-05\\
338	2.80861796161282e-05\\
339	2.80861800935101e-05\\
340	2.80861805797872e-05\\
341	2.80861810751247e-05\\
342	2.80861815796899e-05\\
343	2.8086182093648e-05\\
344	2.80861826171747e-05\\
345	2.80861831504437e-05\\
346	2.80861836936324e-05\\
347	2.8086184246925e-05\\
348	2.80861848105038e-05\\
349	2.80861853845562e-05\\
350	2.80861859692751e-05\\
351	2.80861865648563e-05\\
352	2.8086187171496e-05\\
353	2.80861877893987e-05\\
354	2.8086188418769e-05\\
355	2.80861890598182e-05\\
356	2.80861897127595e-05\\
357	2.80861903778127e-05\\
358	2.80861910551978e-05\\
359	2.80861917451431e-05\\
360	2.80861924478824e-05\\
361	2.80861931636507e-05\\
362	2.80861938926884e-05\\
363	2.80861946352462e-05\\
364	2.80861953915747e-05\\
365	2.80861961619311e-05\\
366	2.80861969465816e-05\\
367	2.80861977457971e-05\\
368	2.80861985598556e-05\\
369	2.80861993890399e-05\\
370	2.80862002336433e-05\\
371	2.80862010939642e-05\\
372	2.80862019703127e-05\\
373	2.80862028630044e-05\\
374	2.80862037723647e-05\\
375	2.80862046987312e-05\\
376	2.808620564245e-05\\
377	2.80862066038774e-05\\
378	2.80862075833849e-05\\
379	2.80862085813546e-05\\
380	2.80862095981818e-05\\
381	2.80862106342774e-05\\
382	2.8086211690071e-05\\
383	2.80862127660042e-05\\
384	2.80862138625407e-05\\
385	2.80862149801645e-05\\
386	2.80862161193821e-05\\
387	2.8086217280727e-05\\
388	2.80862184647584e-05\\
389	2.80862196720662e-05\\
390	2.80862209032726e-05\\
391	2.80862221590288e-05\\
392	2.80862234400219e-05\\
393	2.80862247469712e-05\\
394	2.80862260806475e-05\\
395	2.80862274418707e-05\\
396	2.80862288315067e-05\\
397	2.80862302504661e-05\\
398	2.80862316997153e-05\\
399	2.80862331802619e-05\\
400	2.80862346931698e-05\\
401	2.80862362395503e-05\\
402	2.80862378205697e-05\\
403	2.80862394374588e-05\\
404	2.8086241091532e-05\\
405	2.80862427842111e-05\\
406	2.80862445170288e-05\\
407	2.80862462915791e-05\\
408	2.80862481095242e-05\\
409	2.80862499726541e-05\\
410	2.80862518829092e-05\\
411	2.80862538424e-05\\
412	2.80862558534264e-05\\
413	2.8086257918505e-05\\
414	2.80862600403959e-05\\
415	2.80862622221511e-05\\
416	2.80862644671482e-05\\
417	2.80862667791621e-05\\
418	2.80862691624525e-05\\
419	2.80862716219223e-05\\
420	2.80862741634101e-05\\
421	2.80862767942748e-05\\
422	2.80862795246284e-05\\
423	2.80862823699214e-05\\
424	2.8086285356212e-05\\
425	2.8086288530267e-05\\
426	2.80862919770322e-05\\
427	2.80862958444663e-05\\
428	2.80863003639551e-05\\
429	2.80863058241207e-05\\
430	2.80863124129024e-05\\
431	2.80863198919898e-05\\
432	2.80863275459902e-05\\
433	2.80863353806416e-05\\
434	2.80863434019857e-05\\
435	2.80863516163779e-05\\
436	2.80863600305164e-05\\
437	2.80863686514935e-05\\
438	2.80863774868853e-05\\
439	2.8086386544978e-05\\
440	2.80863958353203e-05\\
441	2.80864053701281e-05\\
442	2.80864151678357e-05\\
443	2.80864252619863e-05\\
444	2.80864357231286e-05\\
445	2.8086446711237e-05\\
446	2.80864585968064e-05\\
447	2.80864722254694e-05\\
448	2.80864894474804e-05\\
449	2.80865140184409e-05\\
450	2.80865526116246e-05\\
451	2.80866141754087e-05\\
452	2.80867023239014e-05\\
453	2.80867965835178e-05\\
454	2.80868926758453e-05\\
455	2.80869906251254e-05\\
456	2.80870904593219e-05\\
457	2.80871922118918e-05\\
458	2.8087295922761e-05\\
459	2.80874016360984e-05\\
460	2.8087509391849e-05\\
461	2.80876192130461e-05\\
462	2.808773110846e-05\\
463	2.80878451184784e-05\\
464	2.80879613047944e-05\\
465	2.80880797319544e-05\\
466	2.80882004670643e-05\\
467	2.80883235804694e-05\\
468	2.80884491466583e-05\\
469	2.80885772429902e-05\\
470	2.80887079505498e-05\\
471	2.80888413562767e-05\\
472	2.80889775553677e-05\\
473	2.80891166493581e-05\\
474	2.80892587363316e-05\\
475	2.80894039207413e-05\\
476	2.80895523139946e-05\\
477	2.80897040350871e-05\\
478	2.8089859211319e-05\\
479	2.80900179791046e-05\\
480	2.80901804848662e-05\\
481	2.80903468860667e-05\\
482	2.80905173523679e-05\\
483	2.8090692066955e-05\\
484	2.80908712280712e-05\\
485	2.80910550508182e-05\\
486	2.80912437692606e-05\\
487	2.80914376388795e-05\\
488	2.80916369395449e-05\\
489	2.80918419790192e-05\\
490	2.80920530972932e-05\\
491	2.80922706722094e-05\\
492	2.80924951278461e-05\\
493	2.80927269487464e-05\\
494	2.80929667067045e-05\\
495	2.80932151172259e-05\\
496	2.80934731650202e-05\\
497	2.80937423869787e-05\\
498	2.80940254996891e-05\\
499	2.80943277284858e-05\\
500	2.80946593912327e-05\\
501	2.8095040150302e-05\\
502	2.80955034844796e-05\\
503	2.80960927022307e-05\\
504	2.80968231164462e-05\\
505	2.80975931042638e-05\\
506	2.80983776764648e-05\\
507	2.80991778365475e-05\\
508	2.80999946916829e-05\\
509	2.81008290390136e-05\\
510	2.81016817629114e-05\\
511	2.81025538510951e-05\\
512	2.8103446463201e-05\\
513	2.81043611010734e-05\\
514	2.81053000457361e-05\\
515	2.81062675292016e-05\\
516	2.81072728711379e-05\\
517	2.81083387599474e-05\\
518	2.81095225156801e-05\\
519	2.81109679733623e-05\\
520	2.81130200084217e-05\\
521	2.81164217137343e-05\\
522	2.8122372809078e-05\\
523	2.81307328451471e-05\\
524	2.81393374669202e-05\\
525	2.8148210089866e-05\\
526	2.81573853040879e-05\\
527	2.81669176615914e-05\\
528	2.81768901519144e-05\\
529	2.81874046473829e-05\\
530	2.81985071666067e-05\\
531	2.82100322058695e-05\\
532	2.82220592867267e-05\\
533	2.82348190650448e-05\\
534	2.82488249099873e-05\\
535	2.82651675017933e-05\\
536	2.82860473335031e-05\\
537	2.83145323942167e-05\\
538	2.83472442182761e-05\\
539	2.8380749358613e-05\\
540	2.84151133427996e-05\\
541	2.84504107378315e-05\\
542	2.8486726090546e-05\\
543	2.85241574117138e-05\\
544	2.85628504144372e-05\\
545	2.8603106863314e-05\\
546	2.86455858784958e-05\\
547	2.86916048197475e-05\\
548	2.87437943798303e-05\\
549	2.88064993319595e-05\\
550	2.88810108054738e-05\\
551	2.89574258384307e-05\\
552	2.90366136841843e-05\\
553	2.9121818143783e-05\\
554	2.92262971949298e-05\\
555	2.93947309519714e-05\\
556	3.11587393730253e-05\\
557	3.31624962695448e-05\\
558	3.52871923224828e-05\\
559	3.75823663834745e-05\\
560	4.1712988773412e-05\\
561	5.80622218823199e-05\\
562	7.5201694526181e-05\\
563	9.32340626589157e-05\\
564	0.000112306196431779\\
565	0.000132645131255909\\
566	0.00030164397021752\\
567	0.000607768803823587\\
568	0.000930822380175926\\
569	0.00127311620720435\\
570	0.00163736940748546\\
571	0.00202680492261195\\
572	0.00244524712103414\\
573	0.00289722057015775\\
574	0.0033697985617141\\
575	0.00386159633509131\\
576	0.00437431846692315\\
577	0.00491016184443514\\
578	0.00547110656601669\\
579	0.00604483533067061\\
580	0.00659986535839362\\
581	0.00686411121575563\\
582	0.00712289222269977\\
583	0.007367743343057\\
584	0.00759482125548611\\
585	0.00782073949169679\\
586	0.00804446638930281\\
587	0.00826483417025616\\
588	0.00848064034958728\\
589	0.00869044844652259\\
590	0.00889569740473372\\
591	0.00909380142313226\\
592	0.00928132122564964\\
593	0.00945516715155288\\
594	0.00961172798547802\\
595	0.00974741425137617\\
596	0.00986072654667908\\
597	0.00994757959173143\\
598	0.0099999191923403\\
599	0\\
600	0\\
};
\addplot [color=mycolor15,solid,forget plot]
  table[row sep=crcr]{%
1	2.90931793024638e-05\\
2	2.9093179327587e-05\\
3	2.90931793531601e-05\\
4	2.90931793791901e-05\\
5	2.90931794056854e-05\\
6	2.90931794326563e-05\\
7	2.90931794601061e-05\\
8	2.90931794880486e-05\\
9	2.90931795164905e-05\\
10	2.90931795454421e-05\\
11	2.90931795749101e-05\\
12	2.90931796049031e-05\\
13	2.90931796354348e-05\\
14	2.90931796665119e-05\\
15	2.9093179698143e-05\\
16	2.909317973034e-05\\
17	2.90931797631131e-05\\
18	2.90931797964726e-05\\
19	2.90931798304269e-05\\
20	2.90931798649881e-05\\
21	2.90931799001681e-05\\
22	2.90931799359752e-05\\
23	2.90931799724233e-05\\
24	2.90931800095226e-05\\
25	2.90931800472849e-05\\
26	2.90931800857222e-05\\
27	2.90931801248464e-05\\
28	2.90931801646695e-05\\
29	2.90931802052034e-05\\
30	2.90931802464617e-05\\
31	2.9093180288458e-05\\
32	2.90931803312044e-05\\
33	2.90931803747144e-05\\
34	2.90931804190017e-05\\
35	2.90931804640798e-05\\
36	2.90931805099642e-05\\
37	2.90931805566668e-05\\
38	2.90931806042046e-05\\
39	2.90931806525913e-05\\
40	2.90931807018439e-05\\
41	2.90931807519743e-05\\
42	2.90931808030013e-05\\
43	2.90931808549385e-05\\
44	2.90931809078047e-05\\
45	2.90931809616153e-05\\
46	2.90931810163854e-05\\
47	2.90931810721357e-05\\
48	2.90931811288797e-05\\
49	2.90931811866379e-05\\
50	2.90931812454274e-05\\
51	2.90931813052686e-05\\
52	2.90931813661768e-05\\
53	2.90931814281724e-05\\
54	2.90931814912744e-05\\
55	2.90931815555047e-05\\
56	2.90931816208822e-05\\
57	2.90931816874256e-05\\
58	2.9093181755157e-05\\
59	2.90931818240987e-05\\
60	2.90931818942728e-05\\
61	2.9093181965698e-05\\
62	2.90931820383982e-05\\
63	2.90931821123955e-05\\
64	2.90931821877155e-05\\
65	2.90931822643805e-05\\
66	2.90931823424124e-05\\
67	2.90931824218387e-05\\
68	2.90931825026832e-05\\
69	2.90931825849697e-05\\
70	2.90931826687255e-05\\
71	2.90931827539762e-05\\
72	2.9093182840749e-05\\
73	2.90931829290696e-05\\
74	2.90931830189685e-05\\
75	2.90931831104697e-05\\
76	2.90931832036056e-05\\
77	2.90931832984034e-05\\
78	2.90931833948938e-05\\
79	2.90931834931058e-05\\
80	2.909318359307e-05\\
81	2.90931836948189e-05\\
82	2.90931837983831e-05\\
83	2.90931839037968e-05\\
84	2.90931840110905e-05\\
85	2.90931841202984e-05\\
86	2.90931842314564e-05\\
87	2.90931843445984e-05\\
88	2.90931844597586e-05\\
89	2.90931845769727e-05\\
90	2.909318469628e-05\\
91	2.90931848177163e-05\\
92	2.90931849413191e-05\\
93	2.90931850671258e-05\\
94	2.90931851951791e-05\\
95	2.90931853255164e-05\\
96	2.90931854581805e-05\\
97	2.90931855932105e-05\\
98	2.90931857306489e-05\\
99	2.9093185870542e-05\\
100	2.90931860129287e-05\\
101	2.90931861578569e-05\\
102	2.90931863053709e-05\\
103	2.90931864555167e-05\\
104	2.90931866083403e-05\\
105	2.90931867638929e-05\\
106	2.90931869222187e-05\\
107	2.9093187083369e-05\\
108	2.90931872473948e-05\\
109	2.90931874143473e-05\\
110	2.90931875842777e-05\\
111	2.90931877572404e-05\\
112	2.909318793329e-05\\
113	2.90931881124794e-05\\
114	2.90931882948648e-05\\
115	2.90931884805041e-05\\
116	2.90931886694554e-05\\
117	2.90931888617782e-05\\
118	2.90931890575323e-05\\
119	2.9093189256779e-05\\
120	2.90931894595796e-05\\
121	2.9093189665999e-05\\
122	2.90931898761001e-05\\
123	2.90931900899495e-05\\
124	2.90931903076154e-05\\
125	2.90931905291642e-05\\
126	2.90931907546642e-05\\
127	2.90931909841902e-05\\
128	2.90931912178088e-05\\
129	2.90931914555968e-05\\
130	2.90931916976273e-05\\
131	2.90931919439771e-05\\
132	2.90931921947213e-05\\
133	2.90931924499398e-05\\
134	2.90931927097111e-05\\
135	2.90931929741188e-05\\
136	2.90931932432446e-05\\
137	2.90931935171721e-05\\
138	2.90931937959883e-05\\
139	2.90931940797799e-05\\
140	2.90931943686358e-05\\
141	2.90931946626461e-05\\
142	2.90931949619031e-05\\
143	2.9093195266502e-05\\
144	2.90931955765367e-05\\
145	2.90931958921044e-05\\
146	2.90931962133055e-05\\
147	2.9093196540239e-05\\
148	2.90931968730071e-05\\
149	2.90931972117172e-05\\
150	2.90931975564716e-05\\
151	2.90931979073828e-05\\
152	2.90931982645565e-05\\
153	2.90931986281085e-05\\
154	2.90931989981497e-05\\
155	2.90931993747994e-05\\
156	2.90931997581735e-05\\
157	2.9093200148393e-05\\
158	2.90932005455807e-05\\
159	2.90932009498626e-05\\
160	2.9093201361365e-05\\
161	2.90932017802156e-05\\
162	2.90932022065474e-05\\
163	2.90932026404951e-05\\
164	2.9093203082195e-05\\
165	2.90932035317853e-05\\
166	2.9093203989409e-05\\
167	2.90932044552076e-05\\
168	2.90932049293313e-05\\
169	2.90932054119247e-05\\
170	2.90932059031448e-05\\
171	2.90932064031433e-05\\
172	2.90932069120787e-05\\
173	2.90932074301129e-05\\
174	2.90932079574061e-05\\
175	2.90932084941271e-05\\
176	2.90932090404447e-05\\
177	2.90932095965309e-05\\
178	2.90932101625632e-05\\
179	2.90932107387169e-05\\
180	2.9093211325178e-05\\
181	2.90932119221289e-05\\
182	2.90932125297603e-05\\
183	2.90932131482633e-05\\
184	2.90932137778338e-05\\
185	2.90932144186731e-05\\
186	2.90932150709804e-05\\
187	2.90932157349655e-05\\
188	2.90932164108381e-05\\
189	2.90932170988112e-05\\
190	2.90932177991047e-05\\
191	2.90932185119401e-05\\
192	2.90932192375442e-05\\
193	2.90932199761454e-05\\
194	2.90932207279824e-05\\
195	2.90932214932903e-05\\
196	2.90932222723164e-05\\
197	2.90932230653042e-05\\
198	2.90932238725096e-05\\
199	2.90932246941883e-05\\
200	2.90932255306027e-05\\
201	2.90932263820186e-05\\
202	2.9093227248709e-05\\
203	2.90932281309498e-05\\
204	2.90932290290242e-05\\
205	2.90932299432148e-05\\
206	2.90932308738202e-05\\
207	2.90932318211317e-05\\
208	2.90932327854579e-05\\
209	2.9093233767104e-05\\
210	2.90932347663886e-05\\
211	2.9093235783627e-05\\
212	2.909323681915e-05\\
213	2.90932378732866e-05\\
214	2.90932389463793e-05\\
215	2.90932400387691e-05\\
216	2.90932411508105e-05\\
217	2.90932422828615e-05\\
218	2.90932434352834e-05\\
219	2.90932446084512e-05\\
220	2.90932458027433e-05\\
221	2.90932470185433e-05\\
222	2.90932482562432e-05\\
223	2.90932495162469e-05\\
224	2.90932507989602e-05\\
225	2.90932521047973e-05\\
226	2.90932534341825e-05\\
227	2.90932547875471e-05\\
228	2.90932561653293e-05\\
229	2.90932575679772e-05\\
230	2.90932589959478e-05\\
231	2.90932604497046e-05\\
232	2.90932619297197e-05\\
233	2.90932634364791e-05\\
234	2.90932649704702e-05\\
235	2.90932665321958e-05\\
236	2.90932681221656e-05\\
237	2.90932697409011e-05\\
238	2.90932713889309e-05\\
239	2.90932730667934e-05\\
240	2.9093274775041e-05\\
241	2.90932765142328e-05\\
242	2.90932782849433e-05\\
243	2.90932800877501e-05\\
244	2.90932819232483e-05\\
245	2.90932837920446e-05\\
246	2.90932856947527e-05\\
247	2.90932876320032e-05\\
248	2.90932896044337e-05\\
249	2.9093291612697e-05\\
250	2.90932936574579e-05\\
251	2.90932957393947e-05\\
252	2.90932978591995e-05\\
253	2.90933000175729e-05\\
254	2.90933022152342e-05\\
255	2.9093304452913e-05\\
256	2.90933067313574e-05\\
257	2.90933090513244e-05\\
258	2.90933114135895e-05\\
259	2.90933138189419e-05\\
260	2.90933162681879e-05\\
261	2.90933187621439e-05\\
262	2.90933213016469e-05\\
263	2.9093323887549e-05\\
264	2.90933265207198e-05\\
265	2.90933292020454e-05\\
266	2.90933319324242e-05\\
267	2.90933347127801e-05\\
268	2.90933375440488e-05\\
269	2.90933404271866e-05\\
270	2.90933433631668e-05\\
271	2.90933463529849e-05\\
272	2.90933493976515e-05\\
273	2.90933524981998e-05\\
274	2.90933556556813e-05\\
275	2.90933588711683e-05\\
276	2.9093362145755e-05\\
277	2.90933654805545e-05\\
278	2.90933688767037e-05\\
279	2.90933723353618e-05\\
280	2.90933758577065e-05\\
281	2.90933794449447e-05\\
282	2.90933830983003e-05\\
283	2.90933868190243e-05\\
284	2.90933906083918e-05\\
285	2.90933944677015e-05\\
286	2.90933983982781e-05\\
287	2.90934024014714e-05\\
288	2.90934064786569e-05\\
289	2.90934106312359e-05\\
290	2.90934148606385e-05\\
291	2.90934191683202e-05\\
292	2.90934235557693e-05\\
293	2.90934280244941e-05\\
294	2.90934325760407e-05\\
295	2.90934372119791e-05\\
296	2.90934419339097e-05\\
297	2.90934467434655e-05\\
298	2.90934516423083e-05\\
299	2.90934566321344e-05\\
300	2.9093461714672e-05\\
301	2.90934668916767e-05\\
302	2.90934721649436e-05\\
303	2.90934775362964e-05\\
304	2.90934830075999e-05\\
305	2.90934885807462e-05\\
306	2.90934942576664e-05\\
307	2.90935000403294e-05\\
308	2.90935059307362e-05\\
309	2.90935119309289e-05\\
310	2.90935180429871e-05\\
311	2.90935242690242e-05\\
312	2.90935306111966e-05\\
313	2.90935370717015e-05\\
314	2.90935436527717e-05\\
315	2.90935503566846e-05\\
316	2.90935571857566e-05\\
317	2.90935641423467e-05\\
318	2.90935712288566e-05\\
319	2.90935784477308e-05\\
320	2.90935858014594e-05\\
321	2.90935932925737e-05\\
322	2.90936009236545e-05\\
323	2.90936086973232e-05\\
324	2.90936166162525e-05\\
325	2.90936246831596e-05\\
326	2.90936329008091e-05\\
327	2.90936412720169e-05\\
328	2.90936497996484e-05\\
329	2.90936584866132e-05\\
330	2.90936673358806e-05\\
331	2.90936763504659e-05\\
332	2.90936855334373e-05\\
333	2.90936948879192e-05\\
334	2.90937044170888e-05\\
335	2.90937141241779e-05\\
336	2.90937240124763e-05\\
337	2.909373408533e-05\\
338	2.9093744346143e-05\\
339	2.90937547983787e-05\\
340	2.9093765445564e-05\\
341	2.90937762912835e-05\\
342	2.90937873391848e-05\\
343	2.90937985929839e-05\\
344	2.90938100564562e-05\\
345	2.9093821733449e-05\\
346	2.9093833627874e-05\\
347	2.90938457437149e-05\\
348	2.90938580850267e-05\\
349	2.90938706559326e-05\\
350	2.90938834606379e-05\\
351	2.90938965034156e-05\\
352	2.9093909788626e-05\\
353	2.90939233207027e-05\\
354	2.9093937104166e-05\\
355	2.90939511436199e-05\\
356	2.90939654437536e-05\\
357	2.90939800093517e-05\\
358	2.90939948452909e-05\\
359	2.909400995654e-05\\
360	2.90940253481715e-05\\
361	2.9094041025364e-05\\
362	2.90940569933997e-05\\
363	2.90940732576754e-05\\
364	2.90940898237051e-05\\
365	2.90941066971224e-05\\
366	2.90941238836903e-05\\
367	2.90941413893049e-05\\
368	2.90941592200037e-05\\
369	2.90941773819708e-05\\
370	2.90941958815436e-05\\
371	2.90942147252217e-05\\
372	2.90942339196784e-05\\
373	2.90942534717711e-05\\
374	2.90942733885413e-05\\
375	2.90942936772436e-05\\
376	2.90943143453424e-05\\
377	2.90943354005319e-05\\
378	2.90943568507555e-05\\
379	2.90943787042138e-05\\
380	2.90944009693871e-05\\
381	2.90944236550557e-05\\
382	2.90944467703202e-05\\
383	2.90944703246275e-05\\
384	2.9094494327796e-05\\
385	2.90945187900515e-05\\
386	2.9094543722063e-05\\
387	2.90945691349873e-05\\
388	2.90945950405195e-05\\
389	2.90946214509568e-05\\
390	2.9094648379261e-05\\
391	2.9094675839103e-05\\
392	2.90947038448594e-05\\
393	2.90947324115871e-05\\
394	2.9094761555059e-05\\
395	2.90947912921322e-05\\
396	2.9094821640816e-05\\
397	2.90948526201269e-05\\
398	2.90948842501369e-05\\
399	2.90949165520221e-05\\
400	2.90949495480855e-05\\
401	2.9094983261762e-05\\
402	2.90950177176112e-05\\
403	2.90950529413199e-05\\
404	2.9095088959847e-05\\
405	2.90951258017984e-05\\
406	2.90951634980899e-05\\
407	2.90952020822798e-05\\
408	2.90952415895099e-05\\
409	2.90952820557793e-05\\
410	2.90953235198916e-05\\
411	2.90953660238076e-05\\
412	2.9095409613039e-05\\
413	2.90954543371108e-05\\
414	2.90955002500981e-05\\
415	2.90955474112519e-05\\
416	2.90955958857304e-05\\
417	2.90956457454964e-05\\
418	2.90956970703915e-05\\
419	2.90957499495869e-05\\
420	2.90958044836598e-05\\
421	2.90958607880629e-05\\
422	2.90959189997387e-05\\
423	2.90959792911682e-05\\
424	2.90960419019107e-05\\
425	2.9096107210368e-05\\
426	2.90961758938051e-05\\
427	2.90962492668808e-05\\
428	2.90963299303535e-05\\
429	2.90964227872562e-05\\
430	2.90965358853274e-05\\
431	2.90966784828055e-05\\
432	2.90968501766455e-05\\
433	2.90970258792738e-05\\
434	2.90972057215506e-05\\
435	2.90973898409771e-05\\
436	2.90975783821051e-05\\
437	2.90977714969531e-05\\
438	2.90979693455173e-05\\
439	2.9098172096481e-05\\
440	2.90983799285325e-05\\
441	2.90985930334534e-05\\
442	2.90988116241307e-05\\
443	2.90990359564183e-05\\
444	2.9099266389285e-05\\
445	2.9099503550225e-05\\
446	2.90997487864562e-05\\
447	2.91000053791739e-05\\
448	2.91002817408142e-05\\
449	2.91005995366793e-05\\
450	2.91010130430726e-05\\
451	2.9101649662234e-05\\
452	2.91027682805687e-05\\
453	2.91047020870107e-05\\
454	2.91068190737854e-05\\
455	2.91089775935504e-05\\
456	2.91111781752481e-05\\
457	2.91134214173341e-05\\
458	2.91157080402251e-05\\
459	2.9118038945718e-05\\
460	2.91204152310904e-05\\
461	2.91228380245918e-05\\
462	2.91253079910998e-05\\
463	2.91278249038759e-05\\
464	2.91303894186689e-05\\
465	2.91330029185693e-05\\
466	2.91356668570307e-05\\
467	2.91383827440669e-05\\
468	2.91411521584032e-05\\
469	2.91439767852071e-05\\
470	2.91468583702275e-05\\
471	2.91497987253e-05\\
472	2.91527997675997e-05\\
473	2.91558636058589e-05\\
474	2.91589925727011e-05\\
475	2.91621888539417e-05\\
476	2.91654547756575e-05\\
477	2.91687928170124e-05\\
478	2.91722056241405e-05\\
479	2.91756960256899e-05\\
480	2.91792670502333e-05\\
481	2.91829219459032e-05\\
482	2.9186664202492e-05\\
483	2.91904975764392e-05\\
484	2.91944261191551e-05\\
485	2.91984542096033e-05\\
486	2.92025865924738e-05\\
487	2.92068284229079e-05\\
488	2.92111853178327e-05\\
489	2.92156634186546e-05\\
490	2.92202694634806e-05\\
491	2.92250108729358e-05\\
492	2.92298958493688e-05\\
493	2.92349335061599e-05\\
494	2.9240134058643e-05\\
495	2.92455091202278e-05\\
496	2.92510722606742e-05\\
497	2.92568402214188e-05\\
498	2.92628358171017e-05\\
499	2.92690951603317e-05\\
500	2.92756857450048e-05\\
501	2.92827506305816e-05\\
502	2.92906101766628e-05\\
503	2.92999676784904e-05\\
504	2.93121927531563e-05\\
505	2.93289709976244e-05\\
506	2.9346987131976e-05\\
507	2.93653466747014e-05\\
508	2.93840716575043e-05\\
509	2.94031900535272e-05\\
510	2.94227202101169e-05\\
511	2.94426823864227e-05\\
512	2.94630985001884e-05\\
513	2.9483992550852e-05\\
514	2.95053915277041e-05\\
515	2.95273272121389e-05\\
516	2.95498412891083e-05\\
517	2.95730008941682e-05\\
518	2.95969477534526e-05\\
519	2.96220575361834e-05\\
520	2.96494673391088e-05\\
521	2.96828537393044e-05\\
522	2.97345261078044e-05\\
523	3.01021053709156e-05\\
524	3.10823745284106e-05\\
525	3.20921802992323e-05\\
526	3.31334756897174e-05\\
527	3.42085687022956e-05\\
528	3.53203790258592e-05\\
529	3.64729006401969e-05\\
530	3.76717780476027e-05\\
531	3.89235042501472e-05\\
532	4.02246349365916e-05\\
533	4.15798328242147e-05\\
534	4.29965025234212e-05\\
535	4.44859984380255e-05\\
536	4.60685486942687e-05\\
537	4.77956073358063e-05\\
538	5.46164785367609e-05\\
539	6.53176031175611e-05\\
540	7.63862330248849e-05\\
541	8.78552200880326e-05\\
542	9.9762079827913e-05\\
543	0.000112149796931475\\
544	0.000125067565715275\\
545	0.000138572067831741\\
546	0.00015273059037222\\
547	0.000167626377004082\\
548	0.000183363584696547\\
549	0.000200089867442313\\
550	0.000259094817245577\\
551	0.000488015275408033\\
552	0.000726542020146932\\
553	0.000975629408224585\\
554	0.00123636589032539\\
555	0.00150999973544108\\
556	0.00179648290113427\\
557	0.00209887107636958\\
558	0.00241929322761919\\
559	0.00276003182189189\\
560	0.00312210450741178\\
561	0.00349818716082299\\
562	0.00390375972898349\\
563	0.00433457550024097\\
564	0.00478459859119485\\
565	0.00525596815664969\\
566	0.00560062013606284\\
567	0.00582740021304894\\
568	0.00605605899399815\\
569	0.00628510015539956\\
570	0.00651276544529824\\
571	0.00673634212359634\\
572	0.0069520380325136\\
573	0.0071547202173806\\
574	0.00735717499191341\\
575	0.00756064332953579\\
576	0.0077619994239701\\
577	0.00795658404551624\\
578	0.00814198173030071\\
579	0.00831546642819351\\
580	0.00847403078801572\\
581	0.00861655664322103\\
582	0.00875071443575307\\
583	0.00887968108874371\\
584	0.00900452361312009\\
585	0.00912444387945188\\
586	0.0092381478141785\\
587	0.00934511367152276\\
588	0.009443897894158\\
589	0.0095340619381977\\
590	0.00961580002830814\\
591	0.00968902107613021\\
592	0.00975458929242902\\
593	0.00981291410532457\\
594	0.00986442025097488\\
595	0.00990933460668555\\
596	0.00994779104554705\\
597	0.00997906286423442\\
598	0.0099999191923403\\
599	0\\
600	0\\
};
\addplot [color=mycolor16,solid,forget plot]
  table[row sep=crcr]{%
1	2.93043087117404e-05\\
2	2.93043092703793e-05\\
3	2.93043098390084e-05\\
4	2.93043104178049e-05\\
5	2.93043110069512e-05\\
6	2.93043116066297e-05\\
7	2.93043122170296e-05\\
8	2.93043128383435e-05\\
9	2.9304313470764e-05\\
10	2.93043141144905e-05\\
11	2.93043147697259e-05\\
12	2.93043154366748e-05\\
13	2.93043161155449e-05\\
14	2.93043168065513e-05\\
15	2.93043175099102e-05\\
16	2.93043182258399e-05\\
17	2.93043189545671e-05\\
18	2.93043196963202e-05\\
19	2.93043204513311e-05\\
20	2.93043212198348e-05\\
21	2.93043220020752e-05\\
22	2.93043227982943e-05\\
23	2.93043236087444e-05\\
24	2.9304324433676e-05\\
25	2.93043252733517e-05\\
26	2.93043261280323e-05\\
27	2.93043269979852e-05\\
28	2.93043278834833e-05\\
29	2.93043287848044e-05\\
30	2.93043297022315e-05\\
31	2.93043306360509e-05\\
32	2.9304331586554e-05\\
33	2.93043325540409e-05\\
34	2.93043335388133e-05\\
35	2.93043345411797e-05\\
36	2.93043355614537e-05\\
37	2.93043365999557e-05\\
38	2.93043376570097e-05\\
39	2.9304338732948e-05\\
40	2.93043398281064e-05\\
41	2.93043409428274e-05\\
42	2.93043420774606e-05\\
43	2.93043432323622e-05\\
44	2.93043444078918e-05\\
45	2.93043456044193e-05\\
46	2.93043468223162e-05\\
47	2.93043480619679e-05\\
48	2.93043493237578e-05\\
49	2.93043506080848e-05\\
50	2.93043519153494e-05\\
51	2.9304353245959e-05\\
52	2.93043546003329e-05\\
53	2.93043559788922e-05\\
54	2.93043573820698e-05\\
55	2.93043588103037e-05\\
56	2.93043602640422e-05\\
57	2.93043617437405e-05\\
58	2.93043632498588e-05\\
59	2.93043647828691e-05\\
60	2.93043663432522e-05\\
61	2.93043679314956e-05\\
62	2.9304369548097e-05\\
63	2.93043711935609e-05\\
64	2.93043728684021e-05\\
65	2.93043745731439e-05\\
66	2.93043763083197e-05\\
67	2.93043780744734e-05\\
68	2.93043798721572e-05\\
69	2.93043817019318e-05\\
70	2.930438356437e-05\\
71	2.93043854600548e-05\\
72	2.93043873895775e-05\\
73	2.93043893535434e-05\\
74	2.93043913525642e-05\\
75	2.93043933872674e-05\\
76	2.93043954582886e-05\\
77	2.93043975662756e-05\\
78	2.93043997118847e-05\\
79	2.9304401895789e-05\\
80	2.93044041186721e-05\\
81	2.93044063812278e-05\\
82	2.93044086841617e-05\\
83	2.93044110281965e-05\\
84	2.93044134140635e-05\\
85	2.93044158425092e-05\\
86	2.93044183142939e-05\\
87	2.93044208301863e-05\\
88	2.93044233909772e-05\\
89	2.93044259974645e-05\\
90	2.93044286504645e-05\\
91	2.9304431350804e-05\\
92	2.930443409933e-05\\
93	2.93044368969e-05\\
94	2.93044397443884e-05\\
95	2.93044426426867e-05\\
96	2.93044455926998e-05\\
97	2.93044485953501e-05\\
98	2.93044516515749e-05\\
99	2.93044547623304e-05\\
100	2.93044579285901e-05\\
101	2.93044611513407e-05\\
102	2.93044644315931e-05\\
103	2.93044677703717e-05\\
104	2.93044711687196e-05\\
105	2.93044746276987e-05\\
106	2.93044781483897e-05\\
107	2.93044817318955e-05\\
108	2.93044853793341e-05\\
109	2.93044890918477e-05\\
110	2.93044928705969e-05\\
111	2.93044967167614e-05\\
112	2.93045006315444e-05\\
113	2.93045046161699e-05\\
114	2.93045086718838e-05\\
115	2.93045127999542e-05\\
116	2.93045170016716e-05\\
117	2.93045212783518e-05\\
118	2.93045256313294e-05\\
119	2.93045300619681e-05\\
120	2.93045345716518e-05\\
121	2.9304539161792e-05\\
122	2.93045438338238e-05\\
123	2.93045485892098e-05\\
124	2.93045534294344e-05\\
125	2.93045583560166e-05\\
126	2.93045633704937e-05\\
127	2.93045684744355e-05\\
128	2.9304573669441e-05\\
129	2.93045789571328e-05\\
130	2.93045843391694e-05\\
131	2.93045898172315e-05\\
132	2.93045953930355e-05\\
133	2.93046010683268e-05\\
134	2.93046068448816e-05\\
135	2.93046127245083e-05\\
136	2.93046187090495e-05\\
137	2.93046248003767e-05\\
138	2.9304631000399e-05\\
139	2.93046373110579e-05\\
140	2.93046437343306e-05\\
141	2.93046502722301e-05\\
142	2.93046569268019e-05\\
143	2.93046637001356e-05\\
144	2.930467059435e-05\\
145	2.93046776116064e-05\\
146	2.9304684754107e-05\\
147	2.93046920240916e-05\\
148	2.93046994238357e-05\\
149	2.93047069556625e-05\\
150	2.93047146219363e-05\\
151	2.93047224250586e-05\\
152	2.93047303674788e-05\\
153	2.93047384516907e-05\\
154	2.93047466802305e-05\\
155	2.93047550556807e-05\\
156	2.93047635806729e-05\\
157	2.93047722578833e-05\\
158	2.93047810900391e-05\\
159	2.93047900799118e-05\\
160	2.93047992303276e-05\\
161	2.93048085441635e-05\\
162	2.93048180243463e-05\\
163	2.93048276738572e-05\\
164	2.93048374957319e-05\\
165	2.93048474930573e-05\\
166	2.93048576689833e-05\\
167	2.93048680267076e-05\\
168	2.93048785694943e-05\\
169	2.93048893006622e-05\\
170	2.93049002235912e-05\\
171	2.93049113417229e-05\\
172	2.93049226585599e-05\\
173	2.93049341776698e-05\\
174	2.93049459026849e-05\\
175	2.93049578373041e-05\\
176	2.93049699852926e-05\\
177	2.93049823504821e-05\\
178	2.93049949367794e-05\\
179	2.93050077481577e-05\\
180	2.93050207886653e-05\\
181	2.93050340624202e-05\\
182	2.93050475736206e-05\\
183	2.93050613265381e-05\\
184	2.93050753255261e-05\\
185	2.9305089575011e-05\\
186	2.93051040795048e-05\\
187	2.93051188436029e-05\\
188	2.93051338719807e-05\\
189	2.93051491694023e-05\\
190	2.93051647407171e-05\\
191	2.93051805908648e-05\\
192	2.93051967248771e-05\\
193	2.93052131478743e-05\\
194	2.93052298650723e-05\\
195	2.93052468817857e-05\\
196	2.93052642034196e-05\\
197	2.93052818354881e-05\\
198	2.93052997835972e-05\\
199	2.9305318053464e-05\\
200	2.93053366509077e-05\\
201	2.93053555818531e-05\\
202	2.93053748523359e-05\\
203	2.93053944685061e-05\\
204	2.93054144366209e-05\\
205	2.93054347630603e-05\\
206	2.93054554543167e-05\\
207	2.93054765170072e-05\\
208	2.93054979578662e-05\\
209	2.93055197837596e-05\\
210	2.93055420016758e-05\\
211	2.93055646187347e-05\\
212	2.93055876421871e-05\\
213	2.93056110794205e-05\\
214	2.93056349379604e-05\\
215	2.93056592254701e-05\\
216	2.93056839497564e-05\\
217	2.93057091187743e-05\\
218	2.93057347406236e-05\\
219	2.93057608235576e-05\\
220	2.93057873759846e-05\\
221	2.93058144064681e-05\\
222	2.93058419237318e-05\\
223	2.93058699366683e-05\\
224	2.93058984543284e-05\\
225	2.93059274859404e-05\\
226	2.9305957040903e-05\\
227	2.93059871287905e-05\\
228	2.93060177593578e-05\\
229	2.93060489425456e-05\\
230	2.93060806884822e-05\\
231	2.93061130074815e-05\\
232	2.93061459100605e-05\\
233	2.93061794069286e-05\\
234	2.93062135089981e-05\\
235	2.93062482273944e-05\\
236	2.93062835734457e-05\\
237	2.93063195586984e-05\\
238	2.93063561949187e-05\\
239	2.93063934940963e-05\\
240	2.93064314684473e-05\\
241	2.93064701304198e-05\\
242	2.93065094927023e-05\\
243	2.93065495682236e-05\\
244	2.9306590370156e-05\\
245	2.93066319119261e-05\\
246	2.93066742072177e-05\\
247	2.93067172699755e-05\\
248	2.930676111441e-05\\
249	2.93068057550044e-05\\
250	2.930685120652e-05\\
251	2.93068974840007e-05\\
252	2.93069446027787e-05\\
253	2.93069925784809e-05\\
254	2.93070414270343e-05\\
255	2.93070911646745e-05\\
256	2.93071418079455e-05\\
257	2.93071933737134e-05\\
258	2.93072458791667e-05\\
259	2.93072993418294e-05\\
260	2.9307353779558e-05\\
261	2.93074092105568e-05\\
262	2.93074656533828e-05\\
263	2.93075231269477e-05\\
264	2.93075816505345e-05\\
265	2.93076412437948e-05\\
266	2.93077019267599e-05\\
267	2.93077637198516e-05\\
268	2.93078266438872e-05\\
269	2.93078907200863e-05\\
270	2.93079559700773e-05\\
271	2.93080224159134e-05\\
272	2.93080900800703e-05\\
273	2.93081589854635e-05\\
274	2.93082291554465e-05\\
275	2.93083006138332e-05\\
276	2.93083733848958e-05\\
277	2.93084474933771e-05\\
278	2.93085229644989e-05\\
279	2.93085998239753e-05\\
280	2.93086780980152e-05\\
281	2.93087578133384e-05\\
282	2.93088389971796e-05\\
283	2.93089216773039e-05\\
284	2.93090058820111e-05\\
285	2.9309091640152e-05\\
286	2.93091789811311e-05\\
287	2.93092679349241e-05\\
288	2.93093585320843e-05\\
289	2.93094508037568e-05\\
290	2.93095447816828e-05\\
291	2.93096404982194e-05\\
292	2.93097379863436e-05\\
293	2.93098372796669e-05\\
294	2.93099384124448e-05\\
295	2.93100414195927e-05\\
296	2.93101463366906e-05\\
297	2.93102532000003e-05\\
298	2.93103620464757e-05\\
299	2.93104729137747e-05\\
300	2.93105858402726e-05\\
301	2.93107008650693e-05\\
302	2.93108180280113e-05\\
303	2.93109373696914e-05\\
304	2.93110589314767e-05\\
305	2.93111827555043e-05\\
306	2.93113088847092e-05\\
307	2.93114373628293e-05\\
308	2.93115682344172e-05\\
309	2.93117015448571e-05\\
310	2.93118373403806e-05\\
311	2.93119756680697e-05\\
312	2.93121165758844e-05\\
313	2.93122601126605e-05\\
314	2.93124063281393e-05\\
315	2.93125552729684e-05\\
316	2.93127069987215e-05\\
317	2.93128615579127e-05\\
318	2.93130190040092e-05\\
319	2.93131793914446e-05\\
320	2.93133427756377e-05\\
321	2.93135092130009e-05\\
322	2.93136787609609e-05\\
323	2.93138514779671e-05\\
324	2.93140274235173e-05\\
325	2.93142066581623e-05\\
326	2.93143892435288e-05\\
327	2.93145752423309e-05\\
328	2.93147647183921e-05\\
329	2.93149577366578e-05\\
330	2.9315154363217e-05\\
331	2.93153546653141e-05\\
332	2.93155587113752e-05\\
333	2.93157665710192e-05\\
334	2.9315978315082e-05\\
335	2.93161940156423e-05\\
336	2.93164137460263e-05\\
337	2.93166375808508e-05\\
338	2.93168655960277e-05\\
339	2.93170978687985e-05\\
340	2.93173344777514e-05\\
341	2.93175755028535e-05\\
342	2.93178210254676e-05\\
343	2.93180711283923e-05\\
344	2.93183258958727e-05\\
345	2.93185854136492e-05\\
346	2.93188497689687e-05\\
347	2.93191190506294e-05\\
348	2.93193933490078e-05\\
349	2.93196727560963e-05\\
350	2.93199573655373e-05\\
351	2.9320247272664e-05\\
352	2.93205425745397e-05\\
353	2.93208433700023e-05\\
354	2.93211497597115e-05\\
355	2.93214618461989e-05\\
356	2.93217797339203e-05\\
357	2.93221035293106e-05\\
358	2.93224333408535e-05\\
359	2.93227692791394e-05\\
360	2.93231114569356e-05\\
361	2.93234599892746e-05\\
362	2.93238149935226e-05\\
363	2.93241765894779e-05\\
364	2.93245448994635e-05\\
365	2.93249200484307e-05\\
366	2.93253021640719e-05\\
367	2.93256913769481e-05\\
368	2.93260878206101e-05\\
369	2.93264916317505e-05\\
370	2.9326902950348e-05\\
371	2.93273219198469e-05\\
372	2.9327748687329e-05\\
373	2.93281834037115e-05\\
374	2.93286262239667e-05\\
375	2.93290773073439e-05\\
376	2.93295368176299e-05\\
377	2.93300049234186e-05\\
378	2.93304817984106e-05\\
379	2.9330967621743e-05\\
380	2.93314625783413e-05\\
381	2.93319668593139e-05\\
382	2.93324806623867e-05\\
383	2.93330041923679e-05\\
384	2.93335376616909e-05\\
385	2.93340812909987e-05\\
386	2.93346353098155e-05\\
387	2.93351999573295e-05\\
388	2.93357754832963e-05\\
389	2.93363621491538e-05\\
390	2.9336960229354e-05\\
391	2.9337570012906e-05\\
392	2.93381918048516e-05\\
393	2.93388259268588e-05\\
394	2.9339472715923e-05\\
395	2.93401325229076e-05\\
396	2.93408057222732e-05\\
397	2.93414927158237e-05\\
398	2.93421939278154e-05\\
399	2.93429098063191e-05\\
400	2.9343640824428e-05\\
401	2.93443874811739e-05\\
402	2.93451503019373e-05\\
403	2.9345929838172e-05\\
404	2.93467266666941e-05\\
405	2.93475413900327e-05\\
406	2.93483746418173e-05\\
407	2.93492271025257e-05\\
408	2.93500995193379e-05\\
409	2.93509926816469e-05\\
410	2.93519073836872e-05\\
411	2.93528444802147e-05\\
412	2.93538048938463e-05\\
413	2.93547896235088e-05\\
414	2.93557997541904e-05\\
415	2.93568364682357e-05\\
416	2.93579010584453e-05\\
417	2.9358994943368e-05\\
418	2.93601196852197e-05\\
419	2.93612770111704e-05\\
420	2.93624688391245e-05\\
421	2.93636973104335e-05\\
422	2.93649648350278e-05\\
423	2.93662741629849e-05\\
424	2.93676285195353e-05\\
425	2.93690319025862e-05\\
426	2.93704898071103e-05\\
427	2.9372011068031e-05\\
428	2.93736125641227e-05\\
429	2.93753308720823e-05\\
430	2.93772491091177e-05\\
431	2.937954906549e-05\\
432	2.93825649121727e-05\\
433	2.93865647993176e-05\\
434	2.93906580715017e-05\\
435	2.93948477653989e-05\\
436	2.93991370711979e-05\\
437	2.94035293415676e-05\\
438	2.94080281007108e-05\\
439	2.94126370534127e-05\\
440	2.94173600941866e-05\\
441	2.94222013171684e-05\\
442	2.94271650294862e-05\\
443	2.94322557769839e-05\\
444	2.94374784106811e-05\\
445	2.94428382829006e-05\\
446	2.94483418539612e-05\\
447	2.94539986059908e-05\\
448	2.94598271691304e-05\\
449	2.9465875226017e-05\\
450	2.9472285212237e-05\\
451	2.94795147135854e-05\\
452	2.94890877018201e-05\\
453	2.95061946780885e-05\\
454	2.97137528650851e-05\\
455	2.99643177527541e-05\\
456	3.02198907130476e-05\\
457	3.04805329288714e-05\\
458	3.07463097120086e-05\\
459	3.10172934108331e-05\\
460	3.12935673858202e-05\\
461	3.15752306426986e-05\\
462	3.18623999270261e-05\\
463	3.21551991485585e-05\\
464	3.24537260405704e-05\\
465	3.2758119079863e-05\\
466	3.3068553554098e-05\\
467	3.33852152092756e-05\\
468	3.37083005507875e-05\\
469	3.40380176623664e-05\\
470	3.43745878109453e-05\\
471	3.47182432694875e-05\\
472	3.50692271046537e-05\\
473	3.54277931937303e-05\\
474	3.57942075380966e-05\\
475	3.61687510396599e-05\\
476	3.65517064958356e-05\\
477	3.69433747872977e-05\\
478	3.73440764709209e-05\\
479	3.77541535261776e-05\\
480	3.817397129973e-05\\
481	3.86039206799129e-05\\
482	3.90444205384928e-05\\
483	3.9495920483335e-05\\
484	3.99589039722751e-05\\
485	4.04338918478452e-05\\
486	4.09214463707476e-05\\
487	4.14221758634367e-05\\
488	4.19367400822611e-05\\
489	4.24658563771034e-05\\
490	4.30103069572825e-05\\
491	4.3570947296266e-05\\
492	4.41487160335469e-05\\
493	4.47446463784222e-05\\
494	4.53598795319635e-05\\
495	4.59956810291058e-05\\
496	4.66534600498183e-05\\
497	4.73347933366168e-05\\
498	4.80414565567155e-05\\
499	4.87754703249452e-05\\
500	4.9539181367542e-05\\
501	5.033544070285e-05\\
502	5.11680726456097e-05\\
503	5.20432562481787e-05\\
504	5.29738500783278e-05\\
505	5.39938867634016e-05\\
506	5.87673186764253e-05\\
507	6.4606815017149e-05\\
508	7.0590274701469e-05\\
509	7.67252049983633e-05\\
510	8.3019714326634e-05\\
511	8.9481993182314e-05\\
512	9.61209472369967e-05\\
513	0.000102946251576546\\
514	0.000109968443091134\\
515	0.000117199021355419\\
516	0.000124650541318546\\
517	0.000132336703579475\\
518	0.000140272419398281\\
519	0.000148473773838568\\
520	0.000156957625963504\\
521	0.000165739941223944\\
522	0.000174829676150907\\
523	0.000183950961874965\\
524	0.000192825752632742\\
525	0.000202059447372597\\
526	0.000211688618436699\\
527	0.000221755970778601\\
528	0.000232312204801424\\
529	0.000243420847026636\\
530	0.000255169619523371\\
531	0.000267702889893964\\
532	0.000412746538030032\\
533	0.000576195888250494\\
534	0.000745765199194563\\
535	0.000921788564508206\\
536	0.00110457820815963\\
537	0.0012943489400593\\
538	0.00148633721617432\\
539	0.00168191902455894\\
540	0.00188511074741496\\
541	0.00209666253199608\\
542	0.00231744598211593\\
543	0.00254847920196742\\
544	0.00279095733672176\\
545	0.00304628750794034\\
546	0.00331615021568388\\
547	0.0036026079092931\\
548	0.00390820819911889\\
549	0.00423590450669407\\
550	0.00454732922748713\\
551	0.00471530322788151\\
552	0.00489204389082255\\
553	0.00507209484026282\\
554	0.00525506600206778\\
555	0.00544042211683775\\
556	0.00562744957463521\\
557	0.00581514029179541\\
558	0.0060021940002762\\
559	0.00618708573928175\\
560	0.00636777611433338\\
561	0.00654145229614157\\
562	0.00670414805441769\\
563	0.00686023270116887\\
564	0.00701594208881994\\
565	0.00716944104891064\\
566	0.00731902377334951\\
567	0.00746707784675243\\
568	0.00761602253010009\\
569	0.00776477719611707\\
570	0.00791099048273329\\
571	0.00805137960956685\\
572	0.0081854601127217\\
573	0.00831318779882302\\
574	0.0084344350629761\\
575	0.00854874215821512\\
576	0.00865665372359101\\
577	0.00876045240599831\\
578	0.0088600806646344\\
579	0.00895569983895539\\
580	0.00904732770420615\\
581	0.00913419155627037\\
582	0.00921661474496478\\
583	0.00929486551170983\\
584	0.00936833250999321\\
585	0.00943643823988877\\
586	0.00949976142125213\\
587	0.00955824735515365\\
588	0.00961252025824673\\
589	0.00966324058718606\\
590	0.00971085603739681\\
591	0.00975602317557434\\
592	0.00979895351145169\\
593	0.00983980399718307\\
594	0.00987864202528269\\
595	0.00991535020799166\\
596	0.00994937493726701\\
597	0.00997906286423442\\
598	0.0099999191923403\\
599	0\\
600	0\\
};
\addplot [color=mycolor17,solid,forget plot]
  table[row sep=crcr]{%
1	4.85195228051361e-05\\
2	4.85195905472055e-05\\
3	4.85196595005982e-05\\
4	4.85197296869432e-05\\
5	4.85198011282528e-05\\
6	4.85198738469381e-05\\
7	4.85199478658032e-05\\
8	4.85200232080614e-05\\
9	4.85200998973416e-05\\
10	4.8520177957688e-05\\
11	4.85202574135813e-05\\
12	4.85203382899344e-05\\
13	4.85204206121018e-05\\
14	4.85205044058957e-05\\
15	4.85205896975854e-05\\
16	4.85206765139064e-05\\
17	4.8520764882073e-05\\
18	4.85208548297851e-05\\
19	4.85209463852382e-05\\
20	4.85210395771233e-05\\
21	4.85211344346477e-05\\
22	4.85212309875416e-05\\
23	4.85213292660615e-05\\
24	4.85214293010004e-05\\
25	4.85215311237068e-05\\
26	4.85216347660825e-05\\
27	4.85217402606018e-05\\
28	4.8521847640313e-05\\
29	4.85219569388589e-05\\
30	4.85220681904767e-05\\
31	4.85221814300167e-05\\
32	4.85222966929513e-05\\
33	4.85224140153793e-05\\
34	4.85225334340487e-05\\
35	4.85226549863615e-05\\
36	4.85227787103824e-05\\
37	4.85229046448591e-05\\
38	4.85230328292256e-05\\
39	4.85231633036214e-05\\
40	4.85232961088994e-05\\
41	4.85234312866416e-05\\
42	4.85235688791694e-05\\
43	4.85237089295637e-05\\
44	4.85238514816636e-05\\
45	4.85239965800966e-05\\
46	4.85241442702842e-05\\
47	4.85242945984534e-05\\
48	4.85244476116607e-05\\
49	4.85246033577975e-05\\
50	4.85247618856084e-05\\
51	4.85249232447049e-05\\
52	4.85250874855878e-05\\
53	4.85252546596556e-05\\
54	4.85254248192179e-05\\
55	4.85255980175213e-05\\
56	4.85257743087629e-05\\
57	4.85259537480985e-05\\
58	4.85261363916739e-05\\
59	4.85263222966311e-05\\
60	4.85265115211309e-05\\
61	4.85267041243712e-05\\
62	4.85269001666046e-05\\
63	4.85270997091599e-05\\
64	4.85273028144527e-05\\
65	4.85275095460144e-05\\
66	4.85277199685125e-05\\
67	4.85279341477591e-05\\
68	4.8528152150745e-05\\
69	4.85283740456517e-05\\
70	4.85285999018786e-05\\
71	4.852882979006e-05\\
72	4.85290637820906e-05\\
73	4.85293019511444e-05\\
74	4.85295443717018e-05\\
75	4.85297911195685e-05\\
76	4.85300422719043e-05\\
77	4.85302979072419e-05\\
78	4.85305581055176e-05\\
79	4.85308229480864e-05\\
80	4.85310925177616e-05\\
81	4.85313668988247e-05\\
82	4.85316461770697e-05\\
83	4.85319304398101e-05\\
84	4.85322197759229e-05\\
85	4.85325142758694e-05\\
86	4.85328140317205e-05\\
87	4.85331191371924e-05\\
88	4.85334296876743e-05\\
89	4.85337457802515e-05\\
90	4.85340675137454e-05\\
91	4.85343949887367e-05\\
92	4.85347283076049e-05\\
93	4.85350675745539e-05\\
94	4.8535412895644e-05\\
95	4.85357643788334e-05\\
96	4.85361221340029e-05\\
97	4.85364862729943e-05\\
98	4.85368569096487e-05\\
99	4.85372341598347e-05\\
100	4.85376181414916e-05\\
101	4.85380089746661e-05\\
102	4.85384067815443e-05\\
103	4.85388116864975e-05\\
104	4.85392238161132e-05\\
105	4.85396432992459e-05\\
106	4.85400702670465e-05\\
107	4.8540504853011e-05\\
108	4.85409471930187e-05\\
109	4.85413974253778e-05\\
110	4.85418556908663e-05\\
111	4.85423221327831e-05\\
112	4.85427968969806e-05\\
113	4.85432801319221e-05\\
114	4.85437719887268e-05\\
115	4.85442726212134e-05\\
116	4.854478218595e-05\\
117	4.85453008423083e-05\\
118	4.8545828752508e-05\\
119	4.85463660816713e-05\\
120	4.85469129978757e-05\\
121	4.85474696722051e-05\\
122	4.85480362788079e-05\\
123	4.85486129949458e-05\\
124	4.85492000010577e-05\\
125	4.85497974808099e-05\\
126	4.85504056211634e-05\\
127	4.85510246124225e-05\\
128	4.85516546483036e-05\\
129	4.85522959259959e-05\\
130	4.85529486462182e-05\\
131	4.85536130132935e-05\\
132	4.85542892352019e-05\\
133	4.8554977523657e-05\\
134	4.85556780941695e-05\\
135	4.85563911661148e-05\\
136	4.85571169628064e-05\\
137	4.85578557115676e-05\\
138	4.85586076437961e-05\\
139	4.85593729950541e-05\\
140	4.85601520051265e-05\\
141	4.85609449181111e-05\\
142	4.85617519824902e-05\\
143	4.85625734512119e-05\\
144	4.85634095817727e-05\\
145	4.85642606363002e-05\\
146	4.85651268816352e-05\\
147	4.85660085894219e-05\\
148	4.8566906036195e-05\\
149	4.85678195034628e-05\\
150	4.85687492778094e-05\\
151	4.85696956509753e-05\\
152	4.85706589199607e-05\\
153	4.85716393871228e-05\\
154	4.85726373602646e-05\\
155	4.85736531527433e-05\\
156	4.85746870835716e-05\\
157	4.85757394775177e-05\\
158	4.85768106652127e-05\\
159	4.85779009832611e-05\\
160	4.85790107743469e-05\\
161	4.8580140387351e-05\\
162	4.85812901774596e-05\\
163	4.85824605062826e-05\\
164	4.85836517419741e-05\\
165	4.85848642593533e-05\\
166	4.85860984400237e-05\\
167	4.85873546725047e-05\\
168	4.85886333523554e-05\\
169	4.8589934882306e-05\\
170	4.85912596723923e-05\\
171	4.85926081400904e-05\\
172	4.85939807104564e-05\\
173	4.85953778162622e-05\\
174	4.85967998981494e-05\\
175	4.85982474047637e-05\\
176	4.85997207929129e-05\\
177	4.86012205277192e-05\\
178	4.86027470827684e-05\\
179	4.86043009402722e-05\\
180	4.86058825912316e-05\\
181	4.86074925355984e-05\\
182	4.86091312824445e-05\\
183	4.86107993501316e-05\\
184	4.86124972664885e-05\\
185	4.86142255689852e-05\\
186	4.86159848049198e-05\\
187	4.86177755315993e-05\\
188	4.86195983165268e-05\\
189	4.86214537376046e-05\\
190	4.86233423833176e-05\\
191	4.86252648529382e-05\\
192	4.8627221756737e-05\\
193	4.8629213716181e-05\\
194	4.86312413641528e-05\\
195	4.8633305345169e-05\\
196	4.86354063155928e-05\\
197	4.86375449438675e-05\\
198	4.8639721910745e-05\\
199	4.86419379095203e-05\\
200	4.8644193646269e-05\\
201	4.86464898400934e-05\\
202	4.86488272233737e-05\\
203	4.86512065420214e-05\\
204	4.86536285557313e-05\\
205	4.86560940382629e-05\\
206	4.86586037776941e-05\\
207	4.86611585767051e-05\\
208	4.86637592528588e-05\\
209	4.86664066388893e-05\\
210	4.86691015829889e-05\\
211	4.86718449491142e-05\\
212	4.86746376172859e-05\\
213	4.8677480483906e-05\\
214	4.86803744620675e-05\\
215	4.86833204818831e-05\\
216	4.86863194908192e-05\\
217	4.86893724540281e-05\\
218	4.86924803546937e-05\\
219	4.86956441943839e-05\\
220	4.86988649934139e-05\\
221	4.87021437912051e-05\\
222	4.87054816466617e-05\\
223	4.87088796385576e-05\\
224	4.87123388659159e-05\\
225	4.87158604484162e-05\\
226	4.87194455267965e-05\\
227	4.87230952632638e-05\\
228	4.87268108419249e-05\\
229	4.87305934692125e-05\\
230	4.87344443743226e-05\\
231	4.87383648096746e-05\\
232	4.87423560513576e-05\\
233	4.87464193996125e-05\\
234	4.87505561792985e-05\\
235	4.87547677403838e-05\\
236	4.87590554584498e-05\\
237	4.87634207351953e-05\\
238	4.87678649989508e-05\\
239	4.87723897052186e-05\\
240	4.87769963372058e-05\\
241	4.87816864063762e-05\\
242	4.87864614530179e-05\\
243	4.87913230468166e-05\\
244	4.87962727874388e-05\\
245	4.88013123051327e-05\\
246	4.88064432613451e-05\\
247	4.88116673493306e-05\\
248	4.88169862948016e-05\\
249	4.88224018565728e-05\\
250	4.8827915827221e-05\\
251	4.88335300337648e-05\\
252	4.88392463383471e-05\\
253	4.8845066638951e-05\\
254	4.88509928701049e-05\\
255	4.88570270036215e-05\\
256	4.88631710493406e-05\\
257	4.88694270558925e-05\\
258	4.88757971114794e-05\\
259	4.88822833446537e-05\\
260	4.88888879251433e-05\\
261	4.88956130646577e-05\\
262	4.89024610177392e-05\\
263	4.89094340826172e-05\\
264	4.89165346020737e-05\\
265	4.89237649643376e-05\\
266	4.89311276039891e-05\\
267	4.89386250028848e-05\\
268	4.89462596910926e-05\\
269	4.89540342478559e-05\\
270	4.89619513025609e-05\\
271	4.89700135357422e-05\\
272	4.89782236800782e-05\\
273	4.89865845214382e-05\\
274	4.89950988999169e-05\\
275	4.90037697109065e-05\\
276	4.90125999061813e-05\\
277	4.90215924950037e-05\\
278	4.90307505452409e-05\\
279	4.90400771845188e-05\\
280	4.9049575601366e-05\\
281	4.90592490464081e-05\\
282	4.90691008335636e-05\\
283	4.9079134341253e-05\\
284	4.90893530136492e-05\\
285	4.90997603619277e-05\\
286	4.9110359965539e-05\\
287	4.91211554735117e-05\\
288	4.91321506057572e-05\\
289	4.91433491544102e-05\\
290	4.91547549851743e-05\\
291	4.91663720386933e-05\\
292	4.9178204331943e-05\\
293	4.91902559596341e-05\\
294	4.9202531095637e-05\\
295	4.92150339944272e-05\\
296	4.9227768992544e-05\\
297	4.92407405100684e-05\\
298	4.92539530521209e-05\\
299	4.92674112103702e-05\\
300	4.92811196645611e-05\\
301	4.92950831840634e-05\\
302	4.93093066294312e-05\\
303	4.93237949539777e-05\\
304	4.93385532053724e-05\\
305	4.93535865272458e-05\\
306	4.93689001608142e-05\\
307	4.93844994465169e-05\\
308	4.94003898256681e-05\\
309	4.941657684213e-05\\
310	4.94330661439883e-05\\
311	4.9449863485256e-05\\
312	4.94669747275817e-05\\
313	4.94844058419817e-05\\
314	4.95021629105794e-05\\
315	4.95202521283679e-05\\
316	4.9538679804987e-05\\
317	4.9557452366518e-05\\
318	4.95765763572965e-05\\
319	4.95960584417466e-05\\
320	4.96159054062375e-05\\
321	4.96361241609624e-05\\
322	4.9656721741848e-05\\
323	4.96777053124757e-05\\
324	4.96990821660607e-05\\
325	4.97208597274403e-05\\
326	4.97430455551105e-05\\
327	4.9765647343306e-05\\
328	4.97886729241193e-05\\
329	4.98121302696654e-05\\
330	4.98360274943216e-05\\
331	4.98603728569967e-05\\
332	4.98851747634897e-05\\
333	4.99104417689053e-05\\
334	4.99361825801509e-05\\
335	4.99624060585073e-05\\
336	4.9989121222297e-05\\
337	5.00163372496359e-05\\
338	5.00440634812877e-05\\
339	5.00723094236053e-05\\
340	5.01010847516042e-05\\
341	5.01303993121218e-05\\
342	5.01602631270978e-05\\
343	5.01906863969831e-05\\
344	5.02216795042634e-05\\
345	5.02532530171252e-05\\
346	5.02854176932574e-05\\
347	5.03181844838271e-05\\
348	5.03515645375995e-05\\
349	5.03855692052767e-05\\
350	5.04202100440215e-05\\
351	5.04554988222424e-05\\
352	5.04914475246349e-05\\
353	5.05280683575287e-05\\
354	5.05653737545819e-05\\
355	5.06033763828397e-05\\
356	5.06420891492121e-05\\
357	5.06815252073879e-05\\
358	5.07216979652491e-05\\
359	5.07626210928243e-05\\
360	5.08043085308115e-05\\
361	5.08467744997615e-05\\
362	5.08900335099431e-05\\
363	5.09341003719842e-05\\
364	5.09789902083251e-05\\
365	5.10247184655912e-05\\
366	5.10713009279394e-05\\
367	5.11187537314744e-05\\
368	5.11670933798314e-05\\
369	5.12163367610239e-05\\
370	5.12665011656658e-05\\
371	5.13176043066897e-05\\
372	5.13696643406997e-05\\
373	5.14226998910996e-05\\
374	5.14767300731561e-05\\
375	5.15317745211822e-05\\
376	5.15878534180362e-05\\
377	5.1644987527163e-05\\
378	5.17031982274314e-05\\
379	5.17625075510554e-05\\
380	5.18229382249418e-05\\
381	5.18845137158304e-05\\
382	5.19472582797052e-05\\
383	5.20111970159642e-05\\
384	5.20763559270271e-05\\
385	5.2142761984092e-05\\
386	5.22104431999679e-05\\
387	5.22794287101352e-05\\
388	5.23497488634706e-05\\
389	5.24214353245409e-05\\
390	5.24945211900758e-05\\
391	5.25690411231054e-05\\
392	5.26450315089373e-05\\
393	5.27225306351912e-05\\
394	5.28015788873556e-05\\
395	5.28822189220311e-05\\
396	5.29644957750493e-05\\
397	5.30484573727744e-05\\
398	5.31341547007723e-05\\
399	5.3221641628035e-05\\
400	5.33109751041869e-05\\
401	5.34022153600259e-05\\
402	5.34954261059868e-05\\
403	5.35906747198619e-05\\
404	5.36880324110592e-05\\
405	5.37875743466039e-05\\
406	5.38893797348472e-05\\
407	5.39935319156555e-05\\
408	5.41001186388268e-05\\
409	5.42092327617706e-05\\
410	5.4320971769912e-05\\
411	5.44354371495247e-05\\
412	5.45527380881619e-05\\
413	5.46729923971985e-05\\
414	5.47963275720986e-05\\
415	5.49228820134563e-05\\
416	5.50528064366033e-05\\
417	5.51862655029529e-05\\
418	5.53234397127437e-05\\
419	5.54645276068821e-05\\
420	5.56097483355481e-05\\
421	5.57593446642293e-05\\
422	5.5913586505215e-05\\
423	5.60727750903251e-05\\
424	5.6237247952476e-05\\
425	5.6407385006861e-05\\
426	5.65836163645633e-05\\
427	5.67664335585839e-05\\
428	5.69564092393166e-05\\
429	5.71542416273935e-05\\
430	5.73608781483743e-05\\
431	5.75779043230569e-05\\
432	5.78088449023424e-05\\
433	5.81528195365713e-05\\
434	5.944600411787e-05\\
435	6.07708016525669e-05\\
436	6.21282928284642e-05\\
437	6.35196126365089e-05\\
438	6.4945952967713e-05\\
439	6.64085650019074e-05\\
440	6.7908761246117e-05\\
441	6.94479170315175e-05\\
442	7.10274712130791e-05\\
443	7.26489257282991e-05\\
444	7.4313843546243e-05\\
445	7.60238443407499e-05\\
446	7.77805968532678e-05\\
447	7.95858060671009e-05\\
448	8.14411910082956e-05\\
449	8.33484419114667e-05\\
450	8.53091223231798e-05\\
451	8.7324403025868e-05\\
452	8.93942409883346e-05\\
453	9.15146481566733e-05\\
454	9.35035181938711e-05\\
455	9.55063189739746e-05\\
456	9.75619116404759e-05\\
457	9.96717437119328e-05\\
458	0.000101837158122837\\
459	0.000104059334890113\\
460	0.000106339220398147\\
461	0.000108677450218538\\
462	0.000111074278278899\\
463	0.000113529500974515\\
464	0.000116042112437598\\
465	0.000118607941916725\\
466	0.000121225878689891\\
467	0.000123897386685402\\
468	0.000126624013333797\\
469	0.000129407381553128\\
470	0.000132249194157849\\
471	0.000135151309566943\\
472	0.000138115684211088\\
473	0.000141144369207877\\
474	0.000144239505589599\\
475	0.000147403349520146\\
476	0.000150638493100372\\
477	0.000153947246360344\\
478	0.000157332063585502\\
479	0.000160795558711385\\
480	0.000164340522107563\\
481	0.000167969939549219\\
482	0.000171687013726675\\
483	0.00017549518871605\\
484	0.000179398177915557\\
485	0.000183399996029068\\
486	0.000187504995758643\\
487	0.000191717910133522\\
488	0.000196043902545426\\
489	0.00020048862747069\\
490	0.000205058300007985\\
491	0.000209759786026714\\
492	0.000214600706085218\\
493	0.000219589567095355\\
494	0.000224735913396954\\
495	0.000230050508790892\\
496	0.000235545583825447\\
497	0.000241235118228053\\
498	0.000247135186952318\\
499	0.00025326438473167\\
500	0.00025964434531677\\
501	0.000266300365657899\\
502	0.000273262109020108\\
503	0.000280564206036033\\
504	0.000288245964501537\\
505	0.000296341669277401\\
506	0.000308549128404662\\
507	0.000401301320030103\\
508	0.000496495979642583\\
509	0.00059424841886353\\
510	0.000694694016538126\\
511	0.000797978055842614\\
512	0.000904240215439014\\
513	0.00101363016259423\\
514	0.00112630667846547\\
515	0.00124243940537317\\
516	0.00136221071779773\\
517	0.00148581664666537\\
518	0.00161346774579206\\
519	0.0017453898123079\\
520	0.00188182432228271\\
521	0.00202302843785282\\
522	0.00216927474552162\\
523	0.00232085412302543\\
524	0.0024780776132213\\
525	0.00264125873975925\\
526	0.00281073686921825\\
527	0.00298714697226363\\
528	0.00317122809954563\\
529	0.00336384916330986\\
530	0.00356606087372385\\
531	0.00377911073125441\\
532	0.00387101444814595\\
533	0.00395496570181997\\
534	0.00404411861485679\\
535	0.0041393541562261\\
536	0.00424177554848268\\
537	0.0043527725456736\\
538	0.00447412533780575\\
539	0.00460796840269658\\
540	0.00474871272796527\\
541	0.00489195363004134\\
542	0.00503735684616973\\
543	0.00518445717205239\\
544	0.00533261860337901\\
545	0.00548098202337772\\
546	0.00562839635531924\\
547	0.00577331960216846\\
548	0.00591379893350599\\
549	0.00604742138754942\\
550	0.00617093021530702\\
551	0.00628179603343684\\
552	0.00639010919318062\\
553	0.00650075390556414\\
554	0.00661355826910363\\
555	0.00672831871640315\\
556	0.00684480418700696\\
557	0.00696276543345935\\
558	0.00708195602380279\\
559	0.00720220504478184\\
560	0.007323419518955\\
561	0.00744556801198611\\
562	0.00756874620780185\\
563	0.00769285619681815\\
564	0.00781677314052897\\
565	0.00793628932779018\\
566	0.00805088736053623\\
567	0.00816001629554882\\
568	0.00826313047911087\\
569	0.00835993460475097\\
570	0.00845129433017152\\
571	0.00853958906049813\\
572	0.00862495123378327\\
573	0.00870767807453262\\
574	0.00878809121736665\\
575	0.00886661500378345\\
576	0.00894286737083422\\
577	0.00901626113058295\\
578	0.0090866482592975\\
579	0.00915386945533433\\
580	0.00921808471957653\\
581	0.00927992511337708\\
582	0.00933845256431125\\
583	0.00939359549483182\\
584	0.0094456284309534\\
585	0.00949555685866655\\
586	0.00954333369779573\\
587	0.00958916916241128\\
588	0.00963370188398624\\
589	0.00967711629413547\\
590	0.00971958295016231\\
591	0.00976110413435579\\
592	0.00980161112056955\\
593	0.00984099058333643\\
594	0.00987905095617487\\
595	0.00991543381058343\\
596	0.00994937493726701\\
597	0.00997906286423442\\
598	0.0099999191923403\\
599	0\\
600	0\\
};
\addplot [color=mycolor18,solid,forget plot]
  table[row sep=crcr]{%
1	0.000169140775378202\\
2	0.000169141360870221\\
3	0.000169141956833748\\
4	0.000169142563455799\\
5	0.000169143180926731\\
6	0.000169143809440294\\
7	0.000169144449193693\\
8	0.00016914510038765\\
9	0.000169145763226465\\
10	0.000169146437918081\\
11	0.000169147124674152\\
12	0.0001691478237101\\
13	0.000169148535245194\\
14	0.000169149259502606\\
15	0.000169149996709493\\
16	0.000169150747097061\\
17	0.00016915151090063\\
18	0.000169152288359729\\
19	0.000169153079718147\\
20	0.000169153885224021\\
21	0.000169154705129912\\
22	0.000169155539692885\\
23	0.000169156389174587\\
24	0.000169157253841328\\
25	0.000169158133964165\\
26	0.000169159029818986\\
27	0.000169159941686601\\
28	0.00016916086985282\\
29	0.00016916181460855\\
30	0.000169162776249881\\
31	0.000169163755078186\\
32	0.000169164751400206\\
33	0.000169165765528146\\
34	0.000169166797779781\\
35	0.000169167848478545\\
36	0.000169168917953642\\
37	0.000169170006540139\\
38	0.000169171114579074\\
39	0.000169172242417566\\
40	0.000169173390408922\\
41	0.000169174558912741\\
42	0.000169175748295031\\
43	0.000169176958928325\\
44	0.000169178191191797\\
45	0.000169179445471374\\
46	0.000169180722159865\\
47	0.000169182021657074\\
48	0.000169183344369936\\
49	0.000169184690712634\\
50	0.000169186061106733\\
51	0.000169187455981313\\
52	0.000169188875773099\\
53	0.000169190320926595\\
54	0.000169191791894236\\
55	0.00016919328913651\\
56	0.00016919481312212\\
57	0.000169196364328115\\
58	0.00016919794324005\\
59	0.000169199550352132\\
60	0.000169201186167376\\
61	0.00016920285119776\\
62	0.00016920454596439\\
63	0.000169206270997654\\
64	0.000169208026837402\\
65	0.000169209814033093\\
66	0.000169211633143991\\
67	0.000169213484739319\\
68	0.000169215369398452\\
69	0.000169217287711087\\
70	0.000169219240277436\\
71	0.000169221227708407\\
72	0.000169223250625795\\
73	0.000169225309662487\\
74	0.000169227405462643\\
75	0.00016922953868191\\
76	0.000169231709987623\\
77	0.000169233920059015\\
78	0.000169236169587427\\
79	0.000169238459276524\\
80	0.00016924078984252\\
81	0.000169243162014397\\
82	0.000169245576534136\\
83	0.000169248034156946\\
84	0.000169250535651508\\
85	0.000169253081800205\\
86	0.000169255673399376\\
87	0.000169258311259563\\
88	0.000169260996205759\\
89	0.000169263729077678\\
90	0.000169266510730009\\
91	0.000169269342032687\\
92	0.000169272223871166\\
93	0.000169275157146695\\
94	0.000169278142776606\\
95	0.000169281181694597\\
96	0.000169284274851023\\
97	0.000169287423213203\\
98	0.000169290627765713\\
99	0.000169293889510708\\
100	0.000169297209468223\\
101	0.000169300588676507\\
102	0.000169304028192337\\
103	0.000169307529091361\\
104	0.000169311092468429\\
105	0.000169314719437943\\
106	0.000169318411134204\\
107	0.00016932216871177\\
108	0.000169325993345824\\
109	0.000169329886232535\\
110	0.000169333848589447\\
111	0.000169337881655853\\
112	0.000169341986693192\\
113	0.000169346164985444\\
114	0.000169350417839538\\
115	0.000169354746585762\\
116	0.000169359152578186\\
117	0.000169363637195088\\
118	0.000169368201839394\\
119	0.000169372847939113\\
120	0.000169377576947802\\
121	0.000169382390345014\\
122	0.000169387289636777\\
123	0.000169392276356063\\
124	0.000169397352063277\\
125	0.000169402518346758\\
126	0.000169407776823276\\
127	0.000169413129138547\\
128	0.00016941857696776\\
129	0.000169424122016103\\
130	0.000169429766019315\\
131	0.000169435510744231\\
132	0.000169441357989345\\
133	0.000169447309585389\\
134	0.000169453367395912\\
135	0.000169459533317877\\
136	0.000169465809282265\\
137	0.000169472197254695\\
138	0.000169478699236051\\
139	0.000169485317263119\\
140	0.000169492053409246\\
141	0.000169498909785002\\
142	0.000169505888538853\\
143	0.000169512991857853\\
144	0.000169520221968349\\
145	0.000169527581136695\\
146	0.000169535071669976\\
147	0.000169542695916759\\
148	0.000169550456267846\\
149	0.000169558355157043\\
150	0.000169566395061947\\
151	0.000169574578504747\\
152	0.000169582908053038\\
153	0.000169591386320653\\
154	0.000169600015968509\\
155	0.000169608799705471\\
156	0.000169617740289225\\
157	0.000169626840527181\\
158	0.00016963610327738\\
159	0.000169645531449434\\
160	0.000169655128005455\\
161	0.000169664895961041\\
162	0.000169674838386244\\
163	0.000169684958406587\\
164	0.000169695259204072\\
165	0.000169705744018236\\
166	0.000169716416147201\\
167	0.000169727278948765\\
168	0.000169738335841502\\
169	0.000169749590305881\\
170	0.000169761045885419\\
171	0.000169772706187853\\
172	0.000169784574886315\\
173	0.000169796655720564\\
174	0.000169808952498212\\
175	0.000169821469095988\\
176	0.00016983420946103\\
177	0.000169847177612184\\
178	0.000169860377641356\\
179	0.000169873813714861\\
180	0.000169887490074822\\
181	0.000169901411040579\\
182	0.000169915581010145\\
183	0.000169930004461665\\
184	0.000169944685954929\\
185	0.000169959630132898\\
186	0.00016997484172327\\
187	0.000169990325540069\\
188	0.000170006086485271\\
189	0.000170022129550458\\
190	0.000170038459818513\\
191	0.000170055082465341\\
192	0.000170072002761624\\
193	0.000170089226074614\\
194	0.000170106757869964\\
195	0.000170124603713592\\
196	0.000170142769273583\\
197	0.000170161260322129\\
198	0.00017018008273751\\
199	0.000170199242506112\\
200	0.000170218745724491\\
201	0.000170238598601468\\
202	0.000170258807460283\\
203	0.000170279378740776\\
204	0.000170300319001619\\
205	0.0001703216349226\\
206	0.00017034333330694\\
207	0.000170365421083671\\
208	0.00017038790531005\\
209	0.000170410793174038\\
210	0.000170434091996806\\
211	0.000170457809235325\\
212	0.000170481952484971\\
213	0.00017050652948222\\
214	0.000170531548107377\\
215	0.000170557016387361\\
216	0.000170582942498561\\
217	0.000170609334769738\\
218	0.000170636201684998\\
219	0.000170663551886816\\
220	0.000170691394179128\\
221	0.00017071973753049\\
222	0.000170748591077301\\
223	0.000170777964127089\\
224	0.000170807866161872\\
225	0.000170838306841585\\
226	0.00017086929600758\\
227	0.0001709008436862\\
228	0.000170932960092429\\
229	0.000170965655633617\\
230	0.000170998940913282\\
231	0.000171032826734991\\
232	0.000171067324106335\\
233	0.00017110244424297\\
234	0.000171138198572752\\
235	0.000171174598739968\\
236	0.000171211656609632\\
237	0.000171249384271903\\
238	0.000171287794046572\\
239	0.000171326898487654\\
240	0.000171366710388079\\
241	0.000171407242784477\\
242	0.00017144850896207\\
243	0.000171490522459664\\
244	0.000171533297074748\\
245	0.000171576846868694\\
246	0.000171621186172091\\
247	0.000171666329590157\\
248	0.000171712292008301\\
249	0.00017175908859777\\
250	0.000171806734821448\\
251	0.00017185524643975\\
252	0.000171904639516657\\
253	0.000171954930425878\\
254	0.000172006135857134\\
255	0.000172058272822577\\
256	0.000172111358663353\\
257	0.000172165411056294\\
258	0.000172220448020755\\
259	0.000172276487925589\\
260	0.000172333549496276\\
261	0.000172391651822199\\
262	0.000172450814364063\\
263	0.00017251105696149\\
264	0.000172572399840738\\
265	0.000172634863622617\\
266	0.000172698469330543\\
267	0.000172763238398767\\
268	0.000172829192680768\\
269	0.000172896354457827\\
270	0.000172964746447769\\
271	0.000173034391813874\\
272	0.00017310531417399\\
273	0.000173177537609807\\
274	0.000173251086676319\\
275	0.000173325986411495\\
276	0.000173402262346116\\
277	0.00017347994051381\\
278	0.000173559047461299\\
279	0.000173639610258825\\
280	0.000173721656510783\\
281	0.000173805214366563\\
282	0.000173890312531585\\
283	0.000173976980278556\\
284	0.000174065247458914\\
285	0.000174155144514509\\
286	0.000174246702489475\\
287	0.000174339953042325\\
288	0.000174434928458261\\
289	0.000174531661661694\\
290	0.000174630186228997\\
291	0.000174730536401446\\
292	0.000174832747098415\\
293	0.000174936853930769\\
294	0.000175042893214468\\
295	0.000175150901984418\\
296	0.000175260918008512\\
297	0.000175372979801904\\
298	0.0001754871266415\\
299	0.000175603398580655\\
300	0.000175721836464091\\
301	0.000175842481943035\\
302	0.000175965377490548\\
303	0.00017609056641709\\
304	0.000176218092886261\\
305	0.000176348001930775\\
306	0.000176480339468613\\
307	0.000176615152319388\\
308	0.000176752488220905\\
309	0.000176892395845903\\
310	0.000177034924819003\\
311	0.000177180125733827\\
312	0.000177328050170325\\
313	0.000177478750712254\\
314	0.000177632280964872\\
315	0.00017778869557278\\
316	0.000177948050237974\\
317	0.000178110401738062\\
318	0.000178275807944658\\
319	0.00017844432784198\\
320	0.000178616021545623\\
321	0.000178790950321544\\
322	0.000178969176605242\\
323	0.000179150764021163\\
324	0.000179335777402335\\
325	0.000179524282810263\\
326	0.000179716347555077\\
327	0.000179912040215995\\
328	0.00018011143066211\\
329	0.000180314590073516\\
330	0.000180521590962852\\
331	0.000180732507197275\\
332	0.000180947414020906\\
333	0.000181166388077831\\
334	0.000181389507435652\\
335	0.000181616851609706\\
336	0.000181848501587966\\
337	0.000182084539856667\\
338	0.000182325050426761\\
339	0.000182570118861175\\
340	0.00018281983230296\\
341	0.000183074279504315\\
342	0.000183333550856513\\
343	0.000183597738420683\\
344	0.000183866935959448\\
345	0.000184141238969377\\
346	0.000184420744714248\\
347	0.000184705552259115\\
348	0.000184995762505134\\
349	0.000185291478225168\\
350	0.000185592804100197\\
351	0.000185899846756542\\
352	0.000186212714804042\\
353	0.000186531518875363\\
354	0.000186856371666636\\
355	0.000187187387979784\\
356	0.000187524684766725\\
357	0.000187868381175616\\
358	0.000188218598599337\\
359	0.000188575460726427\\
360	0.000188939093594694\\
361	0.000189309625647759\\
362	0.000189687187794814\\
363	0.00019007191347385\\
364	0.000190463938718729\\
365	0.000190863402230402\\
366	0.000191270445452653\\
367	0.000191685212652803\\
368	0.000192107851007744\\
369	0.000192538510695839\\
370	0.000192977344995141\\
371	0.000193424510388485\\
372	0.000193880166676053\\
373	0.000194344477096035\\
374	0.000194817608454052\\
375	0.00019529973126214\\
376	0.000195791019888057\\
377	0.000196291652715871\\
378	0.0001968018123188\\
379	0.000197321685645456\\
380	0.000197851464220781\\
381	0.000198391344363156\\
382	0.000198941527419401\\
383	0.000199502220019748\\
384	0.000200073634355187\\
385	0.000200655988480237\\
386	0.000201249506644805\\
387	0.000201854419659851\\
388	0.000202470965303061\\
389	0.000203099388773038\\
390	0.000203739943204837\\
391	0.000204392890268078\\
392	0.000205058500885782\\
393	0.000205737056143686\\
394	0.000206428848497399\\
395	0.000207134183307596\\
396	0.000207853379823844\\
397	0.000208586765521004\\
398	0.00020933468860833\\
399	0.000210097524724835\\
400	0.000210875667369174\\
401	0.000211669529028189\\
402	0.000212479542287857\\
403	0.000213306160867271\\
404	0.000214149860468807\\
405	0.000215011139248253\\
406	0.000215890517559442\\
407	0.000216788536482469\\
408	0.000217705755060487\\
409	0.000218642749987742\\
410	0.000219600141048629\\
411	0.000220578570894087\\
412	0.000221578619178366\\
413	0.000222600889698149\\
414	0.00022364601233429\\
415	0.00022471464532483\\
416	0.000225807477920701\\
417	0.000226925233513964\\
418	0.000228068673351751\\
419	0.000229238600978757\\
420	0.000230435867590063\\
421	0.000231661378522649\\
422	0.000232916101182086\\
423	0.000234201074767163\\
424	0.000235517422258308\\
425	0.000236866365179842\\
426	0.000238249241555478\\
427	0.000239667526799369\\
428	0.000241122854473137\\
429	0.000242617023424316\\
430	0.000244151940366315\\
431	0.000245729312811729\\
432	0.000247349424765246\\
433	0.000248917206812989\\
434	0.000249596544315368\\
435	0.000250289392039495\\
436	0.000250995965316014\\
437	0.000251716480921113\\
438	0.00025245115865765\\
439	0.000253200223695467\\
440	0.00025396390992625\\
441	0.000254742464665859\\
442	0.000255536155136975\\
443	0.00025634527729162\\
444	0.000257170167690267\\
445	0.000258011219336887\\
446	0.000258868902549362\\
447	0.000259743792028783\\
448	0.000260636601032502\\
449	0.000261548222420013\\
450	0.000262479773609716\\
451	0.00026343263996428\\
452	0.000264408529489691\\
453	0.000265409712920401\\
454	0.000266440324185177\\
455	0.000267507390817175\\
456	0.000268615613600356\\
457	0.000269769795927175\\
458	0.00027097573957645\\
459	0.000272240500972844\\
460	0.00027357274626769\\
461	0.000274983305147449\\
462	0.000276486214505931\\
463	0.000278101230395547\\
464	0.000279861166060325\\
465	0.000307524637827358\\
466	0.000349969645631211\\
467	0.000393328689223382\\
468	0.000437627384730763\\
469	0.000482892213005026\\
470	0.000529150187553064\\
471	0.000576429124412446\\
472	0.0006247596323477\\
473	0.000674173667795424\\
474	0.000724704758816065\\
475	0.00077638852461577\\
476	0.000829263833541199\\
477	0.000883376879913921\\
478	0.000938764033621671\\
479	0.000995463115227499\\
480	0.00105351352101643\\
481	0.00111295636342444\\
482	0.00117383465337102\\
483	0.0012361935426771\\
484	0.00130008065241501\\
485	0.00136554652389328\\
486	0.00143264524472401\\
487	0.00150143532629143\\
488	0.00157198094668955\\
489	0.00164435372963181\\
490	0.00171863436274021\\
491	0.00179490999916231\\
492	0.00187327516151905\\
493	0.00195383234317257\\
494	0.00203669287158261\\
495	0.00212197728380178\\
496	0.00220981618775743\\
497	0.0023003531839373\\
498	0.00239374683710717\\
499	0.00249017303845386\\
500	0.00258982783074779\\
501	0.00269293074595561\\
502	0.00279972860438639\\
503	0.00291049941708504\\
504	0.00302555520514576\\
505	0.00314524552958512\\
506	0.00326263227805496\\
507	0.00330343466190086\\
508	0.00334563719849741\\
509	0.00338919372248778\\
510	0.00343401933015739\\
511	0.00348021493093214\\
512	0.00352790103705132\\
513	0.00357722775633646\\
514	0.0036283968779377\\
515	0.00368161901404434\\
516	0.00373712168333583\\
517	0.00379517532398462\\
518	0.00385610310005531\\
519	0.00392029325315597\\
520	0.0039882148043472\\
521	0.00406043735756756\\
522	0.00413765605181866\\
523	0.00422072260381219\\
524	0.00431068243548318\\
525	0.00440881296717743\\
526	0.00451553428790778\\
527	0.00462318184507296\\
528	0.00473134252956375\\
529	0.00483946652908471\\
530	0.00494682918447378\\
531	0.00505248079026667\\
532	0.00515562250187475\\
533	0.00525876642613349\\
534	0.00536161268074051\\
535	0.00546336034902462\\
536	0.00556294739333485\\
537	0.00565897620022263\\
538	0.00574971498516013\\
539	0.00583295559956373\\
540	0.00591379869531271\\
541	0.00599621185148525\\
542	0.00608011835818136\\
543	0.00616543130002583\\
544	0.00625205770730007\\
545	0.00633990646259004\\
546	0.00642888543393893\\
547	0.00651894091697697\\
548	0.00661009698930814\\
549	0.00670250472605515\\
550	0.0067965135636925\\
551	0.00689273545573893\\
552	0.00699161192885386\\
553	0.00709320002231934\\
554	0.00719740780199425\\
555	0.00730401456018862\\
556	0.0074126633149158\\
557	0.00752281634470193\\
558	0.00763362999187877\\
559	0.00774222432184838\\
560	0.00784624175801257\\
561	0.0079451173684526\\
562	0.00803836138267193\\
563	0.00812565253839744\\
564	0.00820701294573663\\
565	0.00828582198432094\\
566	0.00836215658040431\\
567	0.00843620586934715\\
568	0.00850832721502562\\
569	0.00857901013151536\\
570	0.00864906440997031\\
571	0.00871885032529967\\
572	0.00878725608766623\\
573	0.00885362539487147\\
574	0.00891781359239266\\
575	0.00897959328019709\\
576	0.00903975342437911\\
577	0.00909898426363027\\
578	0.0091567304087957\\
579	0.00921190939223783\\
580	0.00926446940119685\\
581	0.00931498012486593\\
582	0.00936431107260272\\
583	0.00941219245211477\\
584	0.00945866243584944\\
585	0.00950405575984369\\
586	0.00954865718510014\\
587	0.0095925304780904\\
588	0.00963572111751646\\
589	0.00967825788510043\\
590	0.00972017907935068\\
591	0.00976137872239597\\
592	0.00980171668827597\\
593	0.00984102099863122\\
594	0.00987905595166784\\
595	0.00991543381058343\\
596	0.00994937493726701\\
597	0.00997906286423442\\
598	0.0099999191923403\\
599	0\\
600	0\\
};
\addplot [color=red!25!mycolor17,solid,forget plot]
  table[row sep=crcr]{%
1	0.000375855361567015\\
2	0.000375864735362743\\
3	0.000375874276850767\\
4	0.000375883989026931\\
5	0.000375893874940512\\
6	0.000375903937695174\\
7	0.000375914180449942\\
8	0.000375924606420201\\
9	0.000375935218878671\\
10	0.000375946021156463\\
11	0.0003759570166441\\
12	0.000375968208792553\\
13	0.000375979601114359\\
14	0.000375991197184693\\
15	0.000376003000642513\\
16	0.000376015015191641\\
17	0.000376027244601968\\
18	0.000376039692710604\\
19	0.00037605236342309\\
20	0.000376065260714613\\
21	0.000376078388631249\\
22	0.000376091751291189\\
23	0.000376105352886097\\
24	0.000376119197682364\\
25	0.000376133290022439\\
26	0.000376147634326219\\
27	0.000376162235092381\\
28	0.000376177096899843\\
29	0.000376192224409123\\
30	0.000376207622363863\\
31	0.000376223295592248\\
32	0.000376239249008547\\
33	0.000376255487614668\\
34	0.000376272016501651\\
35	0.000376288840851316\\
36	0.000376305965937858\\
37	0.000376323397129494\\
38	0.000376341139890124\\
39	0.000376359199781053\\
40	0.000376377582462738\\
41	0.000376396293696532\\
42	0.000376415339346471\\
43	0.000376434725381129\\
44	0.000376454457875476\\
45	0.000376474543012747\\
46	0.000376494987086415\\
47	0.000376515796502134\\
48	0.000376536977779709\\
49	0.000376558537555178\\
50	0.00037658048258284\\
51	0.000376602819737422\\
52	0.000376625556016146\\
53	0.000376648698540977\\
54	0.000376672254560806\\
55	0.000376696231453725\\
56	0.000376720636729335\\
57	0.000376745478031095\\
58	0.000376770763138686\\
59	0.000376796499970441\\
60	0.000376822696585871\\
61	0.000376849361188093\\
62	0.000376876502126454\\
63	0.000376904127899108\\
64	0.000376932247155693\\
65	0.000376960868699994\\
66	0.000376990001492742\\
67	0.000377019654654379\\
68	0.000377049837467919\\
69	0.00037708055938181\\
70	0.000377111830012952\\
71	0.000377143659149655\\
72	0.000377176056754711\\
73	0.0003772090329685\\
74	0.000377242598112152\\
75	0.000377276762690801\\
76	0.000377311537396846\\
77	0.000377346933113272\\
78	0.000377382960917113\\
79	0.000377419632082857\\
80	0.000377456958085995\\
81	0.000377494950606607\\
82	0.000377533621533028\\
83	0.000377572982965524\\
84	0.000377613047220143\\
85	0.000377653826832503\\
86	0.000377695334561765\\
87	0.0003777375833946\\
88	0.000377780586549253\\
89	0.000377824357479701\\
90	0.00037786890987985\\
91	0.0003779142576878\\
92	0.000377960415090283\\
93	0.000378007396527034\\
94	0.000378055216695353\\
95	0.000378103890554693\\
96	0.000378153433331373\\
97	0.00037820386052334\\
98	0.000378255187905027\\
99	0.000378307431532289\\
100	0.00037836060774748\\
101	0.000378414733184553\\
102	0.00037846982477425\\
103	0.000378525899749498\\
104	0.000378582975650768\\
105	0.000378641070331584\\
106	0.000378700201964145\\
107	0.000378760389045049\\
108	0.000378821650401074\\
109	0.000378884005195147\\
110	0.000378947472932331\\
111	0.000379012073465995\\
112	0.000379077827004037\\
113	0.000379144754115263\\
114	0.000379212875735858\\
115	0.000379282213175986\\
116	0.000379352788126547\\
117	0.000379424622665918\\
118	0.000379497739267039\\
119	0.000379572160804407\\
120	0.000379647910561355\\
121	0.000379725012237393\\
122	0.000379803489955662\\
123	0.00037988336827062\\
124	0.000379964672175753\\
125	0.000380047427111495\\
126	0.000380131658973323\\
127	0.000380217394119883\\
128	0.000380304659381396\\
129	0.000380393482068136\\
130	0.000380483889979131\\
131	0.000380575911410879\\
132	0.000380669575166452\\
133	0.000380764910564579\\
134	0.000380861947448975\\
135	0.000380960716197793\\
136	0.000381061247733372\\
137	0.000381163573531995\\
138	0.00038126772563395\\
139	0.000381373736653725\\
140	0.000381481639790436\\
141	0.000381591468838373\\
142	0.000381703258197825\\
143	0.000381817042886077\\
144	0.00038193285854853\\
145	0.000382050741470206\\
146	0.000382170728587262\\
147	0.000382292857498925\\
148	0.000382417166479432\\
149	0.000382543694490394\\
150	0.000382672481193246\\
151	0.00038280356696203\\
152	0.000382936992896346\\
153	0.000383072800834601\\
154	0.000383211033367421\\
155	0.000383351733851479\\
156	0.00038349494642337\\
157	0.000383640716013898\\
158	0.000383789088362624\\
159	0.000383940110032628\\
160	0.000384093828425572\\
161	0.000384250291797077\\
162	0.00038440954927231\\
163	0.000384571650862024\\
164	0.000384736647478714\\
165	0.000384904590953235\\
166	0.000385075534051627\\
167	0.000385249530492313\\
168	0.000385426634963654\\
169	0.000385606903141773\\
170	0.000385790391708767\\
171	0.00038597715837124\\
172	0.000386167261879223\\
173	0.000386360762045411\\
174	0.000386557719764833\\
175	0.000386758197034831\\
176	0.00038696225697549\\
177	0.000387169963850431\\
178	0.000387381383087991\\
179	0.000387596581302871\\
180	0.000387815626318109\\
181	0.000388038587187575\\
182	0.000388265534218859\\
183	0.000388496538996624\\
184	0.000388731674406369\\
185	0.000388971014658706\\
186	0.000389214635314107\\
187	0.000389462613308146\\
188	0.000389715026977176\\
189	0.000389971956084584\\
190	0.000390233481847511\\
191	0.000390499686964138\\
192	0.00039077065564149\\
193	0.0003910464736238\\
194	0.000391327228221434\\
195	0.000391613008340391\\
196	0.000391903904512368\\
197	0.000392200008925458\\
198	0.000392501415455497\\
199	0.000392808219697883\\
200	0.000393120519000252\\
201	0.000393438412495624\\
202	0.000393762001136319\\
203	0.000394091387728522\\
204	0.000394426676967558\\
205	0.000394767975473868\\
206	0.000395115391829715\\
207	0.000395469036616637\\
208	0.000395829022453655\\
209	0.000396195464036341\\
210	0.000396568478176487\\
211	0.000396948183842786\\
212	0.000397334702202259\\
213	0.000397728156662532\\
214	0.000398128672915005\\
215	0.000398536378978847\\
216	0.000398951405246032\\
217	0.00039937388452712\\
218	0.000399803952098157\\
219	0.000400241745748425\\
220	0.000400687405829319\\
221	0.000401141075304038\\
222	0.000401602899798527\\
223	0.000402073027653355\\
224	0.000402551609976692\\
225	0.000403038800698443\\
226	0.000403534756625491\\
227	0.0004040396374981\\
228	0.000404553606047494\\
229	0.000405076828054671\\
230	0.00040560947241045\\
231	0.000406151711176854\\
232	0.000406703719649673\\
233	0.000407265676422422\\
234	0.000407837763451732\\
235	0.000408420166124007\\
236	0.000409013073323558\\
237	0.000409616677502251\\
238	0.000410231174750549\\
239	0.000410856764870171\\
240	0.000411493651448245\\
241	0.000412142041933094\\
242	0.000412802147711648\\
243	0.000413474184188537\\
244	0.000414158370866868\\
245	0.000414854931430798\\
246	0.000415564093829837\\
247	0.000416286090365026\\
248	0.000417021157777013\\
249	0.000417769537335968\\
250	0.000418531474933557\\
251	0.000419307221176814\\
252	0.000420097031484151\\
253	0.000420901166183438\\
254	0.000421719890612251\\
255	0.000422553475220229\\
256	0.000423402195673846\\
257	0.000424266332963319\\
258	0.000425146173511911\\
259	0.000426042009287706\\
260	0.000426954137917746\\
261	0.000427882862804754\\
262	0.000428828493246336\\
263	0.000429791344556945\\
264	0.000430771738192403\\
265	0.000431770001877172\\
266	0.000432786469734556\\
267	0.000433821482419583\\
268	0.000434875387254931\\
269	0.000435948538369868\\
270	0.000437041296842104\\
271	0.000438154030842877\\
272	0.000439287115785182\\
273	0.000440440934475291\\
274	0.000441615877267434\\
275	0.000442812342222102\\
276	0.00044403073526753\\
277	0.000445271470364928\\
278	0.000446534969677154\\
279	0.000447821663741041\\
280	0.000449131991643555\\
281	0.000450466401201604\\
282	0.000451825349145801\\
283	0.000453209301308104\\
284	0.000454618732813421\\
285	0.000456054128275351\\
286	0.00045751598199604\\
287	0.000459004798170102\\
288	0.000460521091092953\\
289	0.000462065385373391\\
290	0.000463638216150475\\
291	0.00046524012931502\\
292	0.000466871681735374\\
293	0.000468533441487851\\
294	0.000470225988091772\\
295	0.000471949912749025\\
296	0.000473705818588461\\
297	0.000475494320914845\\
298	0.000477316047462662\\
299	0.000479171638654685\\
300	0.000481061747865299\\
301	0.000482987041688683\\
302	0.000484948200211826\\
303	0.000486945917292301\\
304	0.000488980900840993\\
305	0.000491053873109538\\
306	0.000493165570982603\\
307	0.000495316746274937\\
308	0.000497508166033209\\
309	0.000499740612842413\\
310	0.000502014885137115\\
311	0.000504331797517111\\
312	0.000506692181067717\\
313	0.000509096883684502\\
314	0.000511546770402322\\
315	0.000514042723728773\\
316	0.000516585643981767\\
317	0.000519176449631222\\
318	0.00052181607764491\\
319	0.000524505483838112\\
320	0.00052724564322733\\
321	0.000530037550387755\\
322	0.00053288221981458\\
323	0.000535780686288141\\
324	0.000538734005242898\\
325	0.000541743253140319\\
326	0.00054480952784587\\
327	0.000547933949010139\\
328	0.000551117658454588\\
329	0.000554361820562086\\
330	0.000557667622672797\\
331	0.000561036275485951\\
332	0.000564469013468215\\
333	0.000567967095269487\\
334	0.000571531804147124\\
335	0.000575164448399749\\
336	0.000578866361811827\\
337	0.000582638904110663\\
338	0.000586483461437091\\
339	0.000590401446831726\\
340	0.000594394300738327\\
341	0.000598463491525742\\
342	0.000602610516029878\\
343	0.000606836900116855\\
344	0.000611144199267906\\
345	0.000615533999183235\\
346	0.000620007916404401\\
347	0.00062456759895429\\
348	0.0006292147269931\\
349	0.000633951013488447\\
350	0.000638778204896862\\
351	0.00064369808185397\\
352	0.000648712459870477\\
353	0.000653823190031695\\
354	0.00065903215969827\\
355	0.000664341293204633\\
356	0.000669752552572866\\
357	0.000675267938244582\\
358	0.000680889489831862\\
359	0.000686619286888755\\
360	0.000692459449704616\\
361	0.000698412140121118\\
362	0.000704479562374511\\
363	0.000710663963965251\\
364	0.000716967636556976\\
365	0.000723392916907245\\
366	0.000729942187832629\\
367	0.000736617879210691\\
368	0.000743422469022132\\
369	0.00075035848443618\\
370	0.000757428502942744\\
371	0.000764635153535262\\
372	0.000771981117948212\\
373	0.000779469131953892\\
374	0.000787101986723156\\
375	0.000794882530255428\\
376	0.000802813668883572\\
377	0.000810898368859764\\
378	0.000819139658029086\\
379	0.000827540627598236\\
380	0.000836104434007321\\
381	0.000844834300913916\\
382	0.000853733521299567\\
383	0.000862805459710351\\
384	0.000872053554644853\\
385	0.000881481321105537\\
386	0.000891092353332046\\
387	0.000900890327738918\\
388	0.000910879006084138\\
389	0.000921062238898888\\
390	0.000931443969208682\\
391	0.000942028236562785\\
392	0.000952819181333663\\
393	0.000963821049074246\\
394	0.000975038194216569\\
395	0.000986475081000435\\
396	0.00099813627578383\\
397	0.00101002641521094\\
398	0.00102215005977609\\
399	0.00103451205535606\\
400	0.00104711769728795\\
401	0.00105997245280715\\
402	0.00107308197031745\\
403	0.00108645208931192\\
404	0.00110008885116143\\
405	0.00111399851140646\\
406	0.00112818755533907\\
407	0.00114266272165957\\
408	0.00115743104644096\\
409	0.00117249995723704\\
410	0.00118787748593639\\
411	0.00120357290884931\\
412	0.00121959580838612\\
413	0.00123595364884107\\
414	0.00125265406994354\\
415	0.00126970487381734\\
416	0.0012871140242872\\
417	0.00130488964536087\\
418	0.00132304001870451\\
419	0.00134157357987641\\
420	0.00136049891310906\\
421	0.00137982474455152\\
422	0.00139955993346553\\
423	0.00141971346091733\\
424	0.00144029441416247\\
425	0.00146131196750383\\
426	0.00148277535892334\\
427	0.00150469386122702\\
428	0.00152707674509668\\
429	0.00154993322871793\\
430	0.00157327240515157\\
431	0.00159710314396872\\
432	0.00162143403281746\\
433	0.00164627419956294\\
434	0.00167164021505231\\
435	0.00169756119205294\\
436	0.00172405154377306\\
437	0.00175112621257999\\
438	0.00177880070542449\\
439	0.00180709113343772\\
440	0.00183601425639072\\
441	0.00186558753285368\\
442	0.0018958291770815\\
443	0.00192675822389243\\
444	0.00195839460311546\\
445	0.00199075922557841\\
446	0.00202387408312123\\
447	0.0020577623657961\\
448	0.002092448600366\\
449	0.00212795881569221\\
450	0.00216432074310062\\
451	0.00220156406288743\\
452	0.00223972071917512\\
453	0.0022788253353444\\
454	0.0023189156084635\\
455	0.00236003293586921\\
456	0.00240222330492044\\
457	0.00244553836795654\\
458	0.00249003681886893\\
459	0.00253578618256766\\
460	0.00258286512841324\\
461	0.00263136639903981\\
462	0.00268140027958883\\
463	0.00273309391864708\\
464	0.00278658892957238\\
465	0.0028160858344251\\
466	0.00283229364123304\\
467	0.00284891162233504\\
468	0.00286595707549019\\
469	0.00288344845123734\\
470	0.00290140543312626\\
471	0.00291984904478575\\
472	0.00293880188699185\\
473	0.00295828798435947\\
474	0.00297833269726963\\
475	0.00299896215397996\\
476	0.00302020306229267\\
477	0.00304208316074912\\
478	0.00306463106400083\\
479	0.00308787617611319\\
480	0.0031118482766772\\
481	0.0031365768954682\\
482	0.00316209039491253\\
483	0.00318841464993482\\
484	0.00321557117399093\\
485	0.00324357448262726\\
486	0.00327242840310902\\
487	0.00330212091969892\\
488	0.00333261699691251\\
489	0.00336384880815839\\
490	0.00339572914874768\\
491	0.00342827959547576\\
492	0.00346152324687747\\
493	0.00349549303595669\\
494	0.00353023096274955\\
495	0.00356579818865329\\
496	0.00360226589595257\\
497	0.00363968661879014\\
498	0.00367811966898888\\
499	0.00371763218444223\\
500	0.00375830031168834\\
501	0.00380021052783213\\
502	0.00384346111023267\\
503	0.00388816378050593\\
504	0.00393444559791164\\
505	0.00398245123033693\\
506	0.00403235920962256\\
507	0.00408472396073087\\
508	0.00414228410426581\\
509	0.00420609627069447\\
510	0.00427692741442519\\
511	0.00434901585665425\\
512	0.00442233189612585\\
513	0.00449682806344451\\
514	0.0045724331827926\\
515	0.00464904554710942\\
516	0.00472652549801599\\
517	0.00480468550780751\\
518	0.0048832768550998\\
519	0.00496197116355689\\
520	0.00504033495355125\\
521	0.00511780127935372\\
522	0.00519363221594222\\
523	0.00526686946768327\\
524	0.00533627214472673\\
525	0.00540024436774731\\
526	0.00545788363371357\\
527	0.00551627335090793\\
528	0.00557539167135386\\
529	0.00563527859893137\\
530	0.00569594513862986\\
531	0.00575739841589084\\
532	0.00581967989167963\\
533	0.00588276119286136\\
534	0.00594662385371704\\
535	0.0060112831455332\\
536	0.00607681286106642\\
537	0.00614337892796407\\
538	0.00621128615237045\\
539	0.00628104207503862\\
540	0.00635312254556396\\
541	0.0064277223001837\\
542	0.0065048984881829\\
543	0.00658470064598824\\
544	0.00666716561147777\\
545	0.00675231227196042\\
546	0.00684007090564941\\
547	0.00693038605348422\\
548	0.00702318517624303\\
549	0.00711841038944991\\
550	0.00721590978540083\\
551	0.00731534307738251\\
552	0.00741610746377254\\
553	0.00751723386195233\\
554	0.007615025649278\\
555	0.00770805129765681\\
556	0.00779580222598891\\
557	0.00787788560954736\\
558	0.00795385979339356\\
559	0.00802573665129025\\
560	0.00809530374867697\\
561	0.00816276262131606\\
562	0.00822843087997214\\
563	0.00829273829880889\\
564	0.00835656690377135\\
565	0.0084203077058799\\
566	0.0084841452589513\\
567	0.00854828119591742\\
568	0.00861187435009632\\
569	0.00867400404508977\\
570	0.00873448446449719\\
571	0.00879303057564046\\
572	0.00885059500516906\\
573	0.00890770640378388\\
574	0.00896437197501948\\
575	0.00902054698487434\\
576	0.00907467179264195\\
577	0.00912662893845292\\
578	0.00917681195253149\\
579	0.00922620839379051\\
580	0.00927487032649829\\
581	0.00932247740245366\\
582	0.0093690234246846\\
583	0.00941489058069364\\
584	0.00946020548959901\\
585	0.00950497683508797\\
586	0.009549192402668\\
587	0.00959283081692013\\
588	0.00963587774208467\\
589	0.0096783320190804\\
590	0.0097202097033969\\
591	0.00976138909231767\\
592	0.00980171925872612\\
593	0.00984102135019415\\
594	0.00987905595166784\\
595	0.00991543381058343\\
596	0.00994937493726701\\
597	0.00997906286423442\\
598	0.0099999191923403\\
599	0\\
600	0\\
};
\addplot [color=mycolor19,solid,forget plot]
  table[row sep=crcr]{%
1	0.00243322575185515\\
2	0.00243322940690112\\
3	0.002433233127433\\
4	0.00243323691462267\\
5	0.00243324076966301\\
6	0.00243324469376817\\
7	0.00243324868817403\\
8	0.00243325275413852\\
9	0.00243325689294207\\
10	0.00243326110588798\\
11	0.00243326539430283\\
12	0.00243326975953691\\
13	0.00243327420296463\\
14	0.00243327872598494\\
15	0.0024332833300218\\
16	0.0024332880165246\\
17	0.0024332927869686\\
18	0.00243329764285543\\
19	0.00243330258571356\\
20	0.00243330761709872\\
21	0.00243331273859443\\
22	0.00243331795181255\\
23	0.00243332325839364\\
24	0.00243332866000762\\
25	0.00243333415835421\\
26	0.00243333975516345\\
27	0.00243334545219633\\
28	0.00243335125124526\\
29	0.00243335715413463\\
30	0.00243336316272144\\
31	0.00243336927889581\\
32	0.00243337550458164\\
33	0.00243338184173715\\
34	0.00243338829235553\\
35	0.00243339485846553\\
36	0.00243340154213214\\
37	0.0024334083454572\\
38	0.00243341527058006\\
39	0.0024334223196783\\
40	0.00243342949496824\\
41	0.0024334367987059\\
42	0.00243344423318748\\
43	0.00243345180075018\\
44	0.0024334595037729\\
45	0.00243346734467702\\
46	0.00243347532592709\\
47	0.00243348345003162\\
48	0.00243349171954392\\
49	0.00243350013706283\\
50	0.00243350870523359\\
51	0.00243351742674857\\
52	0.00243352630434822\\
53	0.00243353534082182\\
54	0.00243354453900849\\
55	0.00243355390179791\\
56	0.00243356343213137\\
57	0.00243357313300256\\
58	0.00243358300745861\\
59	0.00243359305860097\\
60	0.00243360328958637\\
61	0.00243361370362788\\
62	0.00243362430399586\\
63	0.00243363509401895\\
64	0.00243364607708517\\
65	0.00243365725664295\\
66	0.0024336686362022\\
67	0.0024336802193354\\
68	0.00243369200967867\\
69	0.00243370401093303\\
70	0.00243371622686545\\
71	0.00243372866131001\\
72	0.00243374131816915\\
73	0.00243375420141487\\
74	0.00243376731508999\\
75	0.00243378066330931\\
76	0.00243379425026102\\
77	0.00243380808020794\\
78	0.0024338221574888\\
79	0.00243383648651969\\
80	0.00243385107179539\\
81	0.00243386591789073\\
82	0.00243388102946204\\
83	0.00243389641124867\\
84	0.00243391206807432\\
85	0.00243392800484867\\
86	0.00243394422656882\\
87	0.00243396073832091\\
88	0.00243397754528165\\
89	0.00243399465271997\\
90	0.00243401206599862\\
91	0.00243402979057587\\
92	0.00243404783200714\\
93	0.00243406619594686\\
94	0.00243408488815007\\
95	0.00243410391447433\\
96	0.00243412328088146\\
97	0.00243414299343945\\
98	0.00243416305832428\\
99	0.00243418348182193\\
100	0.00243420427033023\\
101	0.00243422543036089\\
102	0.00243424696854157\\
103	0.00243426889161785\\
104	0.00243429120645541\\
105	0.00243431392004206\\
106	0.00243433703949003\\
107	0.00243436057203805\\
108	0.00243438452505372\\
109	0.00243440890603569\\
110	0.00243443372261605\\
111	0.00243445898256268\\
112	0.00243448469378165\\
113	0.00243451086431968\\
114	0.00243453750236664\\
115	0.00243456461625812\\
116	0.00243459221447791\\
117	0.00243462030566075\\
118	0.00243464889859492\\
119	0.00243467800222503\\
120	0.00243470762565476\\
121	0.00243473777814967\\
122	0.00243476846914011\\
123	0.00243479970822408\\
124	0.00243483150517026\\
125	0.00243486386992106\\
126	0.0024348968125956\\
127	0.00243493034349291\\
128	0.00243496447309514\\
129	0.00243499921207075\\
130	0.00243503457127785\\
131	0.00243507056176759\\
132	0.00243510719478751\\
133	0.00243514448178507\\
134	0.00243518243441118\\
135	0.00243522106452384\\
136	0.00243526038419174\\
137	0.00243530040569808\\
138	0.00243534114154432\\
139	0.00243538260445403\\
140	0.00243542480737691\\
141	0.00243546776349272\\
142	0.00243551148621539\\
143	0.00243555598919714\\
144	0.0024356012863328\\
145	0.00243564739176398\\
146	0.00243569431988355\\
147	0.00243574208534001\\
148	0.00243579070304211\\
149	0.00243584018816341\\
150	0.00243589055614697\\
151	0.00243594182271013\\
152	0.00243599400384942\\
153	0.00243604711584545\\
154	0.00243610117526801\\
155	0.00243615619898108\\
156	0.00243621220414821\\
157	0.00243626920823775\\
158	0.00243632722902821\\
159	0.00243638628461381\\
160	0.0024364463934101\\
161	0.00243650757415964\\
162	0.00243656984593782\\
163	0.00243663322815867\\
164	0.002436697740581\\
165	0.00243676340331443\\
166	0.00243683023682566\\
167	0.00243689826194482\\
168	0.00243696749987184\\
169	0.00243703797218308\\
170	0.00243710970083805\\
171	0.00243718270818609\\
172	0.00243725701697337\\
173	0.00243733265035\\
174	0.00243740963187707\\
175	0.00243748798553408\\
176	0.00243756773572627\\
177	0.00243764890729223\\
178	0.00243773152551165\\
179	0.00243781561611305\\
180	0.0024379012052819\\
181	0.00243798831966862\\
182	0.00243807698639691\\
183	0.00243816723307218\\
184	0.00243825908779009\\
185	0.00243835257914535\\
186	0.00243844773624056\\
187	0.00243854458869515\\
188	0.00243864316665478\\
189	0.00243874350080061\\
190	0.00243884562235886\\
191	0.00243894956311053\\
192	0.00243905535540133\\
193	0.00243916303215172\\
194	0.00243927262686716\\
195	0.0024393841736486\\
196	0.00243949770720314\\
197	0.00243961326285478\\
198	0.00243973087655549\\
199	0.0024398505848965\\
200	0.00243997242511965\\
201	0.00244009643512912\\
202	0.00244022265350323\\
203	0.0024403511195066\\
204	0.00244048187310239\\
205	0.00244061495496486\\
206	0.00244075040649215\\
207	0.0024408882698193\\
208	0.00244102858783148\\
209	0.00244117140417743\\
210	0.00244131676328333\\
211	0.00244146471036672\\
212	0.0024416152914508\\
213	0.00244176855337893\\
214	0.00244192454382946\\
215	0.00244208331133091\\
216	0.00244224490527716\\
217	0.0024424093759433\\
218	0.0024425767745015\\
219	0.00244274715303737\\
220	0.00244292056456639\\
221	0.002443097063051\\
222	0.00244327670341765\\
223	0.00244345954157455\\
224	0.00244364563442941\\
225	0.00244383503990779\\
226	0.00244402781697168\\
227	0.00244422402563843\\
228	0.00244442372700015\\
229	0.00244462698324345\\
230	0.00244483385766953\\
231	0.00244504441471467\\
232	0.00244525871997119\\
233	0.00244547684020884\\
234	0.00244569884339648\\
235	0.00244592479872436\\
236	0.00244615477662682\\
237	0.00244638884880532\\
238	0.0024466270882522\\
239	0.00244686956927463\\
240	0.00244711636751926\\
241	0.00244736755999737\\
242	0.00244762322511045\\
243	0.00244788344267635\\
244	0.00244814829395612\\
245	0.00244841786168116\\
246	0.00244869223008125\\
247	0.00244897148491293\\
248	0.00244925571348868\\
249	0.00244954500470662\\
250	0.00244983944908089\\
251	0.0024501391387728\\
252	0.00245044416762254\\
253	0.0024507546311817\\
254	0.00245107062674643\\
255	0.00245139225339151\\
256	0.00245171961200502\\
257	0.00245205280532398\\
258	0.00245239193797073\\
259	0.00245273711649021\\
260	0.00245308844938809\\
261	0.00245344604716983\\
262	0.0024538100223808\\
263	0.00245418048964712\\
264	0.00245455756571778\\
265	0.00245494136950771\\
266	0.00245533202214183\\
267	0.00245572964700042\\
268	0.00245613436976553\\
269	0.00245654631846854\\
270	0.00245696562353919\\
271	0.00245739241785566\\
272	0.00245782683679614\\
273	0.00245826901829169\\
274	0.00245871910288073\\
275	0.00245917723376485\\
276	0.00245964355686627\\
277	0.00246011822088686\\
278	0.00246060137736899\\
279	0.002461093180758\\
280	0.00246159378846649\\
281	0.0024621033609407\\
282	0.00246262206172861\\
283	0.00246315005755033\\
284	0.00246368751837049\\
285	0.00246423461747296\\
286	0.00246479153153784\\
287	0.00246535844072099\\
288	0.00246593552873593\\
289	0.00246652298293855\\
290	0.00246712099441454\\
291	0.00246772975806956\\
292	0.00246834947272263\\
293	0.00246898034120261\\
294	0.0024696225704477\\
295	0.00247027637160877\\
296	0.00247094196015583\\
297	0.00247161955598855\\
298	0.00247230938355034\\
299	0.00247301167194677\\
300	0.00247372665506791\\
301	0.00247445457171518\\
302	0.00247519566573273\\
303	0.0024759501861435\\
304	0.00247671838729021\\
305	0.00247750052898146\\
306	0.00247829687664303\\
307	0.00247910770147479\\
308	0.00247993328061304\\
309	0.00248077389729889\\
310	0.00248162984105248\\
311	0.00248250140785344\\
312	0.00248338890032766\\
313	0.00248429262794039\\
314	0.00248521290719616\\
315	0.00248615006184504\\
316	0.00248710442309585\\
317	0.00248807632983603\\
318	0.00248906612885809\\
319	0.00249007417509303\\
320	0.0024911008318499\\
321	0.00249214647106188\\
322	0.00249321147353832\\
323	0.00249429622922244\\
324	0.0024954011374541\\
325	0.00249652660723728\\
326	0.00249767305751127\\
327	0.00249884091742482\\
328	0.00250003062661243\\
329	0.00250124263547115\\
330	0.00250247740543719\\
331	0.00250373540926026\\
332	0.00250501713127448\\
333	0.00250632306766386\\
334	0.00250765372672073\\
335	0.00250900962909522\\
336	0.0025103913080341\\
337	0.00251179930960745\\
338	0.00251323419292223\\
339	0.00251469653032239\\
340	0.00251618690757661\\
341	0.00251770592405601\\
342	0.00251925419290494\\
343	0.00252083234120518\\
344	0.0025224410101529\\
345	0.00252408085532325\\
346	0.00252575254696737\\
347	0.00252745677035158\\
348	0.00252919422614939\\
349	0.00253096563089699\\
350	0.00253277171752206\\
351	0.00253461323595275\\
352	0.00253649095380898\\
353	0.00253840565717714\\
354	0.00254035815148238\\
355	0.00254234926247657\\
356	0.00254437983683534\\
357	0.00254645074273674\\
358	0.00254856287045189\\
359	0.00255071713294689\\
360	0.00255291446649582\\
361	0.00255515583130394\\
362	0.00255744221214044\\
363	0.0025597746189799\\
364	0.00256215408765162\\
365	0.00256458168049537\\
366	0.00256705848702252\\
367	0.0025695856245811\\
368	0.00257216423902266\\
369	0.00257479550536971\\
370	0.0025774806284808\\
371	0.00258022084371129\\
372	0.00258301741756675\\
373	0.00258587164834578\\
374	0.00258878486676855\\
375	0.00259175843658691\\
376	0.00259479375517109\\
377	0.00259789225406755\\
378	0.00260105539952129\\
379	0.00260428469295505\\
380	0.00260758167139656\\
381	0.00261094790784304\\
382	0.00261438501155048\\
383	0.00261789462823249\\
384	0.00262147844015029\\
385	0.0026251381660719\\
386	0.00262887556107286\\
387	0.00263269241614548\\
388	0.00263659055757467\\
389	0.00264057184602923\\
390	0.0026446381753039\\
391	0.0026487914706317\\
392	0.00265303368646489\\
393	0.00265736680359865\\
394	0.00266179282548564\\
395	0.00266631377357758\\
396	0.00267093168157545\\
397	0.00267564858866405\\
398	0.00268046653230534\\
399	0.00268538753867355\\
400	0.00269041360363406\\
401	0.00269554667067543\\
402	0.00270078861131059\\
403	0.00270614120321678\\
404	0.00271160610671546\\
405	0.00271718484080559\\
406	0.00272287876090144\\
407	0.00272868904181218\\
408	0.00273461667146946\\
409	0.00274066246357437\\
410	0.00274682710084091\\
411	0.00275311122544019\\
412	0.00275951561264378\\
413	0.00276604151076391\\
414	0.0027726912044118\\
415	0.00277946746198407\\
416	0.00278637314424618\\
417	0.00279341120986781\\
418	0.00280058472191354\\
419	0.00280789685554711\\
420	0.00281535090508996\\
421	0.00282295028744672\\
422	0.00283069855175062\\
423	0.00283859939608647\\
424	0.00284665671753488\\
425	0.0028548746330132\\
426	0.00286325750122126\\
427	0.00287180994486809\\
428	0.00288053687188287\\
429	0.00288944349364547\\
430	0.00289853533701717\\
431	0.00290781824351838\\
432	0.00291729833865878\\
433	0.00292698192475248\\
434	0.00293687522341037\\
435	0.00294698438892954\\
436	0.00295731584817982\\
437	0.00296787631184095\\
438	0.00297867278412394\\
439	0.00298971257008401\\
440	0.00300100327931023\\
441	0.00301255282434944\\
442	0.00302436941164805\\
443	0.00303646152202934\\
444	0.00304883787669455\\
445	0.00306150738335215\\
446	0.00307447905521328\\
447	0.00308776189309536\\
448	0.00310136471771375\\
449	0.00311529593617701\\
450	0.00312956322845109\\
451	0.00314417316807397\\
452	0.0031591306794131\\
453	0.00317443806932127\\
454	0.00319009365038383\\
455	0.00320608971468185\\
456	0.00322241117922144\\
457	0.00323903284133796\\
458	0.00325591568031399\\
459	0.00327300186627504\\
460	0.00329020802013293\\
461	0.00330741618752921\\
462	0.00332446235535646\\
463	0.0033412318864999\\
464	0.00335758742685534\\
465	0.00337338760836127\\
466	0.00338918042814833\\
467	0.00340530655874068\\
468	0.00342178366260316\\
469	0.00343863177671142\\
470	0.00345587363475759\\
471	0.00347353476768834\\
472	0.0034916426578092\\
473	0.00351023430286914\\
474	0.00352935638325661\\
475	0.00354907225021517\\
476	0.00356945357161289\\
477	0.00359057112268743\\
478	0.00361250894201974\\
479	0.00363536731737758\\
480	0.00365926652319814\\
481	0.00368435151566849\\
482	0.00371079785126699\\
483	0.00373881917193254\\
484	0.00376867669766305\\
485	0.00380069128279977\\
486	0.0038352586996068\\
487	0.00387286879664353\\
488	0.0039141285910975\\
489	0.00395978649863041\\
490	0.00400999994379843\\
491	0.00406104221269004\\
492	0.00411290889303081\\
493	0.00416559058337188\\
494	0.00421907124212074\\
495	0.00427332600109312\\
496	0.00432831847470865\\
497	0.00438399836381636\\
498	0.0044402986351139\\
499	0.00449713119941296\\
500	0.00455438145906978\\
501	0.00461190142112322\\
502	0.00466950098744828\\
503	0.00472693692844685\\
504	0.00478389889952752\\
505	0.00483999165498312\\
506	0.00489471185943743\\
507	0.00494740805108051\\
508	0.00499718385826722\\
509	0.00504289057825752\\
510	0.00508359944924854\\
511	0.00512486917326703\\
512	0.0051666669189476\\
513	0.00520895418874585\\
514	0.00525168675647541\\
515	0.00529481481413926\\
516	0.00533828304376811\\
517	0.00538202974207427\\
518	0.00542599384801781\\
519	0.00547014462970504\\
520	0.00551450866905051\\
521	0.00555909764201829\\
522	0.00560393937996247\\
523	0.00564911930132595\\
524	0.00569481048431436\\
525	0.00574131614048781\\
526	0.0057890876497553\\
527	0.00583843272153866\\
528	0.00588943896141111\\
529	0.00594220010200307\\
530	0.00599681393333867\\
531	0.00605338332978803\\
532	0.00611201645377706\\
533	0.00617282830798182\\
534	0.00623593928816131\\
535	0.00630147018111279\\
536	0.00636949234990505\\
537	0.00644006104353074\\
538	0.00651325385171948\\
539	0.0065891142941634\\
540	0.00666763728431354\\
541	0.00674877479497815\\
542	0.00683243735517887\\
543	0.00691848790243981\\
544	0.00700678142595555\\
545	0.00709710460458226\\
546	0.00718914909923042\\
547	0.00728233011489576\\
548	0.00737567760376649\\
549	0.00746556707184357\\
550	0.00755055199800488\\
551	0.00763014193807071\\
552	0.00770374944381809\\
553	0.00777122664892783\\
554	0.00783558854786247\\
555	0.00789783794406934\\
556	0.00795825232183846\\
557	0.0080172196760148\\
558	0.00807556562364243\\
559	0.00813380648727646\\
560	0.00819215789604154\\
561	0.00825081058869847\\
562	0.00830995726729622\\
563	0.00836978176436801\\
564	0.00843009923217756\\
565	0.00848930666147047\\
566	0.00854719485207219\\
567	0.00860348792501928\\
568	0.00865888706570304\\
569	0.0087141507280509\\
570	0.00876926249803578\\
571	0.00882426452232611\\
572	0.00887920168507198\\
573	0.00893313527028819\\
574	0.00898517449966207\\
575	0.00903522942231062\\
576	0.00908474529495557\\
577	0.00913376135612379\\
578	0.00918233514225318\\
579	0.00923013735242427\\
580	0.00927705205247275\\
581	0.00932349070899567\\
582	0.00936953410212466\\
583	0.00941516998195261\\
584	0.00946036084630153\\
585	0.00950506346562379\\
586	0.00954923819722844\\
587	0.0095928530922261\\
588	0.00963588745565751\\
589	0.00967833566889083\\
590	0.0097202108171807\\
591	0.00976138933709939\\
592	0.00980171928775089\\
593	0.00984102135019415\\
594	0.00987905595166784\\
595	0.00991543381058343\\
596	0.00994937493726701\\
597	0.00997906286423442\\
598	0.0099999191923403\\
599	0\\
600	0\\
};
\addplot [color=red!50!mycolor17,solid,forget plot]
  table[row sep=crcr]{%
1	0.00283870092861365\\
2	0.00283870433655056\\
3	0.00283870780564254\\
4	0.00283871133698597\\
5	0.00283871493169682\\
6	0.0028387185909111\\
7	0.00283872231578512\\
8	0.00283872610749591\\
9	0.00283872996724156\\
10	0.00283873389624159\\
11	0.00283873789573735\\
12	0.00283874196699244\\
13	0.00283874611129304\\
14	0.00283875032994839\\
15	0.00283875462429111\\
16	0.00283875899567769\\
17	0.00283876344548891\\
18	0.00283876797513025\\
19	0.00283877258603231\\
20	0.00283877727965132\\
21	0.00283878205746957\\
22	0.00283878692099581\\
23	0.00283879187176586\\
24	0.00283879691134296\\
25	0.00283880204131834\\
26	0.0028388072633117\\
27	0.00283881257897167\\
28	0.00283881798997637\\
29	0.00283882349803399\\
30	0.00283882910488322\\
31	0.00283883481229385\\
32	0.00283884062206733\\
33	0.00283884653603731\\
34	0.00283885255607024\\
35	0.0028388586840659\\
36	0.0028388649219581\\
37	0.00283887127171516\\
38	0.00283887773534063\\
39	0.00283888431487382\\
40	0.00283889101239058\\
41	0.00283889783000374\\
42	0.00283890476986399\\
43	0.00283891183416043\\
44	0.00283891902512122\\
45	0.00283892634501442\\
46	0.00283893379614855\\
47	0.00283894138087341\\
48	0.00283894910158078\\
49	0.00283895696070514\\
50	0.00283896496072446\\
51	0.00283897310416097\\
52	0.00283898139358199\\
53	0.00283898983160067\\
54	0.00283899842087676\\
55	0.00283900716411757\\
56	0.00283901606407876\\
57	0.00283902512356513\\
58	0.00283903434543156\\
59	0.00283904373258396\\
60	0.00283905328798007\\
61	0.00283906301463037\\
62	0.00283907291559913\\
63	0.00283908299400531\\
64	0.00283909325302345\\
65	0.00283910369588481\\
66	0.00283911432587824\\
67	0.0028391251463513\\
68	0.00283913616071125\\
69	0.00283914737242611\\
70	0.00283915878502576\\
71	0.00283917040210304\\
72	0.00283918222731482\\
73	0.0028391942643832\\
74	0.0028392065170966\\
75	0.00283921898931104\\
76	0.00283923168495119\\
77	0.00283924460801173\\
78	0.00283925776255847\\
79	0.00283927115272973\\
80	0.00283928478273752\\
81	0.00283929865686889\\
82	0.00283931277948727\\
83	0.00283932715503378\\
84	0.00283934178802866\\
85	0.00283935668307262\\
86	0.0028393718448483\\
87	0.0028393872781217\\
88	0.00283940298774367\\
89	0.00283941897865134\\
90	0.00283943525586976\\
91	0.0028394518245134\\
92	0.00283946868978772\\
93	0.00283948585699075\\
94	0.00283950333151479\\
95	0.00283952111884801\\
96	0.0028395392245762\\
97	0.00283955765438446\\
98	0.00283957641405893\\
99	0.00283959550948862\\
100	0.00283961494666721\\
101	0.0028396347316949\\
102	0.00283965487078025\\
103	0.00283967537024211\\
104	0.00283969623651155\\
105	0.00283971747613387\\
106	0.0028397390957706\\
107	0.00283976110220152\\
108	0.00283978350232674\\
109	0.00283980630316888\\
110	0.00283982951187513\\
111	0.00283985313571945\\
112	0.00283987718210492\\
113	0.00283990165856587\\
114	0.00283992657277022\\
115	0.00283995193252184\\
116	0.0028399777457629\\
117	0.00284000402057639\\
118	0.00284003076518851\\
119	0.00284005798797113\\
120	0.00284008569744448\\
121	0.00284011390227963\\
122	0.00284014261130115\\
123	0.00284017183348989\\
124	0.00284020157798559\\
125	0.00284023185408969\\
126	0.00284026267126823\\
127	0.00284029403915462\\
128	0.00284032596755265\\
129	0.0028403584664394\\
130	0.00284039154596828\\
131	0.00284042521647216\\
132	0.00284045948846641\\
133	0.00284049437265214\\
134	0.00284052987991943\\
135	0.00284056602135055\\
136	0.00284060280822343\\
137	0.00284064025201492\\
138	0.00284067836440439\\
139	0.00284071715727713\\
140	0.00284075664272799\\
141	0.002840796833065\\
142	0.00284083774081305\\
143	0.00284087937871771\\
144	0.00284092175974898\\
145	0.00284096489710522\\
146	0.00284100880421708\\
147	0.00284105349475156\\
148	0.00284109898261603\\
149	0.00284114528196243\\
150	0.0028411924071915\\
151	0.00284124037295709\\
152	0.00284128919417041\\
153	0.00284133888600459\\
154	0.00284138946389919\\
155	0.00284144094356474\\
156	0.00284149334098739\\
157	0.00284154667243366\\
158	0.00284160095445534\\
159	0.00284165620389425\\
160	0.00284171243788735\\
161	0.00284176967387167\\
162	0.00284182792958957\\
163	0.00284188722309395\\
164	0.0028419475727535\\
165	0.00284200899725816\\
166	0.00284207151562455\\
167	0.00284213514720164\\
168	0.00284219991167635\\
169	0.00284226582907936\\
170	0.00284233291979086\\
171	0.00284240120454669\\
172	0.00284247070444422\\
173	0.00284254144094859\\
174	0.00284261343589891\\
175	0.00284268671151462\\
176	0.00284276129040196\\
177	0.0028428371955605\\
178	0.00284291445038981\\
179	0.00284299307869625\\
180	0.00284307310469975\\
181	0.00284315455304094\\
182	0.00284323744878815\\
183	0.00284332181744459\\
184	0.00284340768495583\\
185	0.00284349507771704\\
186	0.00284358402258066\\
187	0.00284367454686415\\
188	0.00284376667835761\\
189	0.00284386044533187\\
190	0.00284395587654643\\
191	0.00284405300125773\\
192	0.00284415184922738\\
193	0.00284425245073059\\
194	0.00284435483656481\\
195	0.0028444590380584\\
196	0.00284456508707942\\
197	0.00284467301604467\\
198	0.00284478285792881\\
199	0.00284489464627353\\
200	0.00284500841519708\\
201	0.00284512419940369\\
202	0.00284524203419337\\
203	0.00284536195547169\\
204	0.00284548399975977\\
205	0.00284560820420451\\
206	0.00284573460658881\\
207	0.00284586324534205\\
208	0.00284599415955079\\
209	0.00284612738896945\\
210	0.00284626297403133\\
211	0.00284640095585972\\
212	0.00284654137627915\\
213	0.00284668427782694\\
214	0.00284682970376474\\
215	0.00284697769809036\\
216	0.00284712830554981\\
217	0.00284728157164946\\
218	0.0028474375426683\\
219	0.00284759626567062\\
220	0.00284775778851866\\
221	0.00284792215988554\\
222	0.00284808942926841\\
223	0.00284825964700172\\
224	0.00284843286427077\\
225	0.00284860913312545\\
226	0.00284878850649403\\
227	0.00284897103819749\\
228	0.00284915678296373\\
229	0.00284934579644215\\
230	0.00284953813521847\\
231	0.00284973385682968\\
232	0.00284993301977929\\
233	0.00285013568355279\\
234	0.00285034190863333\\
235	0.00285055175651761\\
236	0.00285076528973209\\
237	0.00285098257184941\\
238	0.0028512036675049\\
239	0.00285142864241372\\
240	0.00285165756338779\\
241	0.00285189049835336\\
242	0.00285212751636863\\
243	0.00285236868764177\\
244	0.00285261408354902\\
245	0.00285286377665339\\
246	0.00285311784072332\\
247	0.00285337635075185\\
248	0.00285363938297592\\
249	0.00285390701489617\\
250	0.0028541793252969\\
251	0.00285445639426639\\
252	0.00285473830321754\\
253	0.00285502513490887\\
254	0.00285531697346588\\
255	0.00285561390440267\\
256	0.00285591601464397\\
257	0.0028562233925476\\
258	0.00285653612792724\\
259	0.00285685431207555\\
260	0.00285717803778778\\
261	0.00285750739938577\\
262	0.00285784249274233\\
263	0.00285818341530614\\
264	0.00285853026612701\\
265	0.00285888314588166\\
266	0.00285924215690008\\
267	0.00285960740319219\\
268	0.00285997899047519\\
269	0.00286035702620138\\
270	0.00286074161958659\\
271	0.0028611328816392\\
272	0.00286153092518973\\
273	0.00286193586492109\\
274	0.00286234781739955\\
275	0.00286276690110634\\
276	0.00286319323647004\\
277	0.00286362694589979\\
278	0.00286406815381909\\
279	0.00286451698670082\\
280	0.0028649735731028\\
281	0.0028654380437045\\
282	0.00286591053134476\\
283	0.00286639117106043\\
284	0.00286688010012632\\
285	0.00286737745809612\\
286	0.00286788338684473\\
287	0.0028683980306119\\
288	0.00286892153604727\\
289	0.00286945405225691\\
290	0.00286999573085146\\
291	0.00287054672599602\\
292	0.00287110719446185\\
293	0.00287167729567994\\
294	0.0028722571917969\\
295	0.00287284704773271\\
296	0.00287344703124124\\
297	0.00287405731297303\\
298	0.00287467806654102\\
299	0.00287530946858908\\
300	0.00287595169886375\\
301	0.00287660494028946\\
302	0.0028772693790472\\
303	0.00287794520465731\\
304	0.00287863261006642\\
305	0.00287933179173891\\
306	0.00288004294975335\\
307	0.00288076628790415\\
308	0.00288150201380895\\
309	0.00288225033902219\\
310	0.0028830114791553\\
311	0.002883785654004\\
312	0.00288457308768344\\
313	0.00288537400877162\\
314	0.00288618865046179\\
315	0.00288701725072479\\
316	0.00288786005248165\\
317	0.00288871730378771\\
318	0.00288958925802896\\
319	0.00289047617413145\\
320	0.0028913783167851\\
321	0.00289229595668272\\
322	0.00289322937077541\\
323	0.0028941788425457\\
324	0.00289514466229953\\
325	0.00289612712747834\\
326	0.0028971265429927\\
327	0.00289814322157858\\
328	0.00289917748417769\\
329	0.00290022966034285\\
330	0.00290130008866944\\
331	0.00290238911725369\\
332	0.00290349710417811\\
333	0.00290462441802381\\
334	0.00290577143840861\\
335	0.00290693855654915\\
336	0.00290812617584297\\
337	0.00290933471246505\\
338	0.00291056459596982\\
339	0.00291181626988715\\
340	0.00291309019229849\\
341	0.00291438683638664\\
342	0.0029157066909734\\
343	0.00291705026105032\\
344	0.00291841806776943\\
345	0.00291981064685911\\
346	0.0029212285484882\\
347	0.00292267233700952\\
348	0.00292414259058094\\
349	0.00292563990067231\\
350	0.00292716487147907\\
351	0.002928718119271\\
352	0.00293030027168571\\
353	0.00293191196688974\\
354	0.00293355385245548\\
355	0.0029352265849003\\
356	0.00293693084065037\\
357	0.00293866731783082\\
358	0.00294043673738679\\
359	0.0029422398442847\\
360	0.00294407740880065\\
361	0.00294595022790378\\
362	0.00294785912674309\\
363	0.00294980496024692\\
364	0.00295178861484537\\
365	0.00295381101032717\\
366	0.00295587310184337\\
367	0.00295797588207172\\
368	0.00296012038355706\\
369	0.00296230768124428\\
370	0.00296453889522294\\
371	0.00296681519370352\\
372	0.00296913779624821\\
373	0.00297150797728118\\
374	0.00297392706990552\\
375	0.0029763964700572\\
376	0.00297891764102881\\
377	0.00298149211839925\\
378	0.00298412151540846\\
379	0.00298680752881971\\
380	0.00298955194531476\\
381	0.00299235664847091\\
382	0.0029952236263704\\
383	0.00299815497989572\\
384	0.0030011529317642\\
385	0.00300421983635425\\
386	0.00300735819037249\\
387	0.00301057064440294\\
388	0.003013860015367\\
389	0.00301722929990288\\
390	0.00302068168864234\\
391	0.00302422058131805\\
392	0.00302784960257062\\
393	0.00303157261823363\\
394	0.00303539375174873\\
395	0.00303931740018923\\
396	0.00304334824912741\\
397	0.00304749128523309\\
398	0.00305175180497379\\
399	0.00305613541720487\\
400	0.00306064803684511\\
401	0.00306529586542185\\
402	0.00307008535243514\\
403	0.00307502312943827\\
404	0.00308011590570261\\
405	0.00308537031020121\\
406	0.00309079265898533\\
407	0.00309638861924857\\
408	0.0031021627305885\\
409	0.00310811772883947\\
410	0.00311425359633487\\
411	0.00312056623196952\\
412	0.00312704559457391\\
413	0.00313367314836349\\
414	0.00314041859936968\\
415	0.00314727139920554\\
416	0.00315423369614006\\
417	0.00316130776674751\\
418	0.0031684960276916\\
419	0.0031758010557658\\
420	0.00318322565622549\\
421	0.00319077296565129\\
422	0.00319844631403471\\
423	0.0032062490418238\\
424	0.003214183896715\\
425	0.00322225377701107\\
426	0.00323046174710021\\
427	0.00323881105488027\\
428	0.00324730515158386\\
429	0.00325594771461001\\
430	0.00326474267408322\\
431	0.00327369424394737\\
432	0.00328280695904933\\
433	0.00329208572344072\\
434	0.00330153588294981\\
435	0.00331116331548502\\
436	0.00332097451830581\\
437	0.00333097671219466\\
438	0.00334117796620837\\
439	0.00335158734756848\\
440	0.00336221510235997\\
441	0.00337307287410827\\
442	0.00338417396907542\\
443	0.00339553367935669\\
444	0.00340716967768397\\
445	0.0034191025013505\\
446	0.00343135614685067\\
447	0.0034439588011638\\
448	0.0034569437378115\\
449	0.00347035039750513\\
450	0.00348422562530908\\
451	0.00349862484951891\\
452	0.00351361518620998\\
453	0.00352928220505041\\
454	0.00354573489592987\\
455	0.0035631115355953\\
456	0.0035815654237004\\
457	0.00360128540371144\\
458	0.00362250512451155\\
459	0.00364551475566122\\
460	0.00367067542333025\\
461	0.00369843548734482\\
462	0.00372934297113895\\
463	0.00376095621363585\\
464	0.00379309105106504\\
465	0.00382577263716753\\
466	0.00385901484731637\\
467	0.00389282359395008\\
468	0.00392720376105065\\
469	0.00396215890131907\\
470	0.00399769086849669\\
471	0.00403379936012597\\
472	0.00407048133580928\\
473	0.00410773020443479\\
474	0.00414553463487091\\
475	0.00418387698318528\\
476	0.00422273139401959\\
477	0.00426206186955287\\
478	0.00430182020803441\\
479	0.00434194273055568\\
480	0.0043823461022077\\
481	0.00442292198487946\\
482	0.00446353017886259\\
483	0.00450398980586727\\
484	0.00454406795040861\\
485	0.00458346500642036\\
486	0.00462179578739872\\
487	0.00465856535452245\\
488	0.00469313896240186\\
489	0.00472470844453819\\
490	0.0047530155955093\\
491	0.00478167072697865\\
492	0.00481065613380342\\
493	0.00483995091478616\\
494	0.00486953070053233\\
495	0.0048993674315073\\
496	0.00492942922729178\\
497	0.00495968036195307\\
498	0.00499008137895352\\
499	0.00502058946022384\\
500	0.00505115916793818\\
501	0.00508174373103509\\
502	0.00511229716796209\\
503	0.00514277748512175\\
504	0.00517315143935459\\
505	0.00520340138261903\\
506	0.00523353332555396\\
507	0.00526359232558662\\
508	0.005293685216728\\
509	0.00532401058150215\\
510	0.00535487809816213\\
511	0.00538650320169562\\
512	0.00541896601711208\\
513	0.0054523077533292\\
514	0.00548657443382623\\
515	0.00552181785933801\\
516	0.00555809814807075\\
517	0.00559548305717446\\
518	0.00563404906956839\\
519	0.00567388228072107\\
520	0.00571507818832832\\
521	0.00575774030131697\\
522	0.00580198076408421\\
523	0.00584791935643933\\
524	0.00589567988748809\\
525	0.00594538341309121\\
526	0.00599709568295925\\
527	0.00605087853310064\\
528	0.00610682270070741\\
529	0.00616501674035226\\
530	0.00622554406078374\\
531	0.00628847479267532\\
532	0.00635385821429033\\
533	0.00642172861966806\\
534	0.00649209994009733\\
535	0.00656495654067599\\
536	0.006640281558746\\
537	0.00671802543277983\\
538	0.00679805038301329\\
539	0.00688017893851313\\
540	0.00696422364181383\\
541	0.00704985310511505\\
542	0.00713651715111697\\
543	0.0072233324844939\\
544	0.00730733886327977\\
545	0.00738638960302661\\
546	0.00745982539142606\\
547	0.00752713644055451\\
548	0.00758836051141061\\
549	0.00764653857858205\\
550	0.00770271529057265\\
551	0.00775720579591524\\
552	0.00781071095053047\\
553	0.00786393070041828\\
554	0.00791723492564974\\
555	0.00797081310434071\\
556	0.0080248179508191\\
557	0.00807939150536912\\
558	0.00813468278734686\\
559	0.00819081823622241\\
560	0.0082479087802371\\
561	0.00830477765193133\\
562	0.00836059325980605\\
563	0.00841510282450088\\
564	0.00846832031068026\\
565	0.00852161182516532\\
566	0.00857497379044075\\
567	0.00862842855953914\\
568	0.00868200623738066\\
569	0.00873571853626056\\
570	0.00878953117589831\\
571	0.0088418987105902\\
572	0.00889242540907162\\
573	0.00894187213284206\\
574	0.00899097596863721\\
575	0.00903979684709103\\
576	0.0090883736771921\\
577	0.00913644094746051\\
578	0.00918373619157009\\
579	0.00923062808335025\\
580	0.0092772493210446\\
581	0.00932358230216704\\
582	0.00936958177756908\\
583	0.00941519618323153\\
584	0.00946037485366822\\
585	0.00950507047593196\\
586	0.00954924140305753\\
587	0.00959285439625311\\
588	0.00963588790901514\\
589	0.0096783357954337\\
590	0.00972021084245027\\
591	0.00976138933978722\\
592	0.00980171928775089\\
593	0.00984102135019415\\
594	0.00987905595166784\\
595	0.00991543381058343\\
596	0.00994937493726701\\
597	0.00997906286423442\\
598	0.0099999191923403\\
599	0\\
600	0\\
};
\addplot [color=red!40!mycolor19,solid,forget plot]
  table[row sep=crcr]{%
1	0.0030112771484627\\
2	0.00301128212363723\\
3	0.00301128718813487\\
4	0.00301129234355798\\
5	0.00301129759153769\\
6	0.00301130293373427\\
7	0.0030113083718378\\
8	0.00301131390756862\\
9	0.00301131954267788\\
10	0.00301132527894816\\
11	0.00301133111819388\\
12	0.00301133706226203\\
13	0.00301134311303272\\
14	0.00301134927241964\\
15	0.00301135554237081\\
16	0.00301136192486916\\
17	0.00301136842193308\\
18	0.00301137503561719\\
19	0.00301138176801284\\
20	0.00301138862124889\\
21	0.00301139559749225\\
22	0.00301140269894872\\
23	0.00301140992786355\\
24	0.00301141728652219\\
25	0.00301142477725101\\
26	0.00301143240241807\\
27	0.00301144016443379\\
28	0.00301144806575179\\
29	0.00301145610886957\\
30	0.00301146429632934\\
31	0.00301147263071887\\
32	0.00301148111467217\\
33	0.00301148975087046\\
34	0.00301149854204294\\
35	0.00301150749096766\\
36	0.00301151660047236\\
37	0.00301152587343541\\
38	0.00301153531278668\\
39	0.00301154492150848\\
40	0.00301155470263645\\
41	0.00301156465926061\\
42	0.0030115747945262\\
43	0.00301158511163475\\
44	0.00301159561384511\\
45	0.00301160630447436\\
46	0.00301161718689893\\
47	0.00301162826455568\\
48	0.0030116395409429\\
49	0.0030116510196215\\
50	0.00301166270421601\\
51	0.00301167459841581\\
52	0.00301168670597626\\
53	0.00301169903071984\\
54	0.00301171157653739\\
55	0.00301172434738933\\
56	0.00301173734730686\\
57	0.00301175058039324\\
58	0.0030117640508251\\
59	0.00301177776285367\\
60	0.00301179172080623\\
61	0.00301180592908736\\
62	0.00301182039218036\\
63	0.0030118351146486\\
64	0.00301185010113706\\
65	0.00301186535637363\\
66	0.00301188088517075\\
67	0.00301189669242676\\
68	0.00301191278312753\\
69	0.00301192916234797\\
70	0.00301194583525363\\
71	0.0030119628071023\\
72	0.0030119800832457\\
73	0.00301199766913105\\
74	0.00301201557030285\\
75	0.00301203379240457\\
76	0.00301205234118046\\
77	0.00301207122247726\\
78	0.00301209044224609\\
79	0.00301211000654424\\
80	0.00301212992153708\\
81	0.00301215019350007\\
82	0.00301217082882056\\
83	0.00301219183399986\\
84	0.00301221321565528\\
85	0.00301223498052212\\
86	0.00301225713545579\\
87	0.003012279687434\\
88	0.00301230264355882\\
89	0.00301232601105897\\
90	0.00301234979729207\\
91	0.00301237400974681\\
92	0.0030123986560454\\
93	0.00301242374394588\\
94	0.0030124492813445\\
95	0.00301247527627825\\
96	0.00301250173692723\\
97	0.00301252867161732\\
98	0.00301255608882265\\
99	0.00301258399716825\\
100	0.00301261240543275\\
101	0.00301264132255103\\
102	0.00301267075761706\\
103	0.00301270071988666\\
104	0.00301273121878037\\
105	0.00301276226388639\\
106	0.00301279386496348\\
107	0.00301282603194397\\
108	0.00301285877493694\\
109	0.00301289210423115\\
110	0.00301292603029838\\
111	0.00301296056379656\\
112	0.00301299571557307\\
113	0.00301303149666809\\
114	0.00301306791831796\\
115	0.0030131049919587\\
116	0.00301314272922949\\
117	0.0030131811419762\\
118	0.00301322024225507\\
119	0.00301326004233647\\
120	0.00301330055470852\\
121	0.00301334179208103\\
122	0.00301338376738934\\
123	0.00301342649379836\\
124	0.00301346998470643\\
125	0.00301351425374962\\
126	0.00301355931480574\\
127	0.00301360518199869\\
128	0.00301365186970269\\
129	0.00301369939254675\\
130	0.00301374776541901\\
131	0.00301379700347138\\
132	0.00301384712212412\\
133	0.00301389813707056\\
134	0.0030139500642818\\
135	0.00301400292001168\\
136	0.00301405672080159\\
137	0.00301411148348558\\
138	0.00301416722519542\\
139	0.00301422396336584\\
140	0.00301428171573976\\
141	0.00301434050037368\\
142	0.00301440033564319\\
143	0.00301446124024844\\
144	0.00301452323321978\\
145	0.00301458633392366\\
146	0.00301465056206828\\
147	0.00301471593770963\\
148	0.0030147824812575\\
149	0.00301485021348162\\
150	0.00301491915551792\\
151	0.00301498932887487\\
152	0.00301506075543991\\
153	0.00301513345748602\\
154	0.00301520745767843\\
155	0.00301528277908132\\
156	0.00301535944516481\\
157	0.00301543747981191\\
158	0.00301551690732563\\
159	0.00301559775243631\\
160	0.00301568004030886\\
161	0.00301576379655038\\
162	0.00301584904721765\\
163	0.00301593581882494\\
164	0.00301602413835183\\
165	0.00301611403325125\\
166	0.00301620553145757\\
167	0.00301629866139481\\
168	0.00301639345198515\\
169	0.00301648993265734\\
170	0.00301658813335542\\
171	0.00301668808454757\\
172	0.00301678981723495\\
173	0.00301689336296089\\
174	0.00301699875382012\\
175	0.00301710602246815\\
176	0.00301721520213082\\
177	0.00301732632661396\\
178	0.00301743943031334\\
179	0.00301755454822457\\
180	0.00301767171595336\\
181	0.00301779096972585\\
182	0.00301791234639907\\
183	0.00301803588347162\\
184	0.00301816161909453\\
185	0.00301828959208228\\
186	0.00301841984192393\\
187	0.00301855240879453\\
188	0.00301868733356669\\
189	0.00301882465782218\\
190	0.00301896442386401\\
191	0.00301910667472834\\
192	0.00301925145419692\\
193	0.0030193988068095\\
194	0.00301954877787644\\
195	0.00301970141349166\\
196	0.00301985676054566\\
197	0.00302001486673878\\
198	0.0030201757805947\\
199	0.00302033955147407\\
200	0.00302050622958839\\
201	0.00302067586601417\\
202	0.00302084851270721\\
203	0.00302102422251701\\
204	0.00302120304920175\\
205	0.00302138504744302\\
206	0.00302157027286114\\
207	0.00302175878203055\\
208	0.00302195063249546\\
209	0.00302214588278573\\
210	0.00302234459243299\\
211	0.00302254682198701\\
212	0.00302275263303224\\
213	0.00302296208820471\\
214	0.0030231752512091\\
215	0.00302339218683602\\
216	0.0030236129609797\\
217	0.00302383764065569\\
218	0.00302406629401913\\
219	0.00302429899038295\\
220	0.00302453580023652\\
221	0.00302477679526461\\
222	0.00302502204836646\\
223	0.00302527163367521\\
224	0.0030255256265776\\
225	0.00302578410373393\\
226	0.00302604714309833\\
227	0.00302631482393923\\
228	0.00302658722686019\\
229	0.00302686443382102\\
230	0.00302714652815906\\
231	0.00302743359461096\\
232	0.00302772571933458\\
233	0.00302802298993118\\
234	0.00302832549546807\\
235	0.00302863332650143\\
236	0.0030289465750994\\
237	0.00302926533486556\\
238	0.00302958970096271\\
239	0.00302991977013685\\
240	0.00303025564074161\\
241	0.00303059741276294\\
242	0.00303094518784404\\
243	0.00303129906931069\\
244	0.00303165916219689\\
245	0.00303202557327074\\
246	0.0030323984110607\\
247	0.00303277778588221\\
248	0.00303316380986452\\
249	0.00303355659697799\\
250	0.00303395626306149\\
251	0.00303436292585038\\
252	0.0030347767050046\\
253	0.00303519772213726\\
254	0.0030356261008434\\
255	0.0030360619667292\\
256	0.00303650544744144\\
257	0.0030369566726973\\
258	0.00303741577431452\\
259	0.00303788288624183\\
260	0.00303835814458976\\
261	0.00303884168766177\\
262	0.00303933365598564\\
263	0.00303983419234523\\
264	0.00304034344181264\\
265	0.00304086155178052\\
266	0.00304138867199482\\
267	0.00304192495458787\\
268	0.00304247055411169\\
269	0.00304302562757174\\
270	0.00304359033446077\\
271	0.00304416483679328\\
272	0.00304474929914\\
273	0.00304534388866292\\
274	0.00304594877515046\\
275	0.00304656413105293\\
276	0.00304719013151849\\
277	0.00304782695442915\\
278	0.0030484747804373\\
279	0.0030491337930023\\
280	0.00304980417842765\\
281	0.00305048612589808\\
282	0.00305117982751735\\
283	0.00305188547834593\\
284	0.00305260327643924\\
285	0.00305333342288612\\
286	0.00305407612184748\\
287	0.00305483158059534\\
288	0.00305560000955211\\
289	0.00305638162233016\\
290	0.00305717663577175\\
291	0.00305798526998912\\
292	0.00305880774840501\\
293	0.00305964429779349\\
294	0.00306049514832087\\
295	0.00306136053358738\\
296	0.0030622406906688\\
297	0.0030631358601587\\
298	0.00306404628621093\\
299	0.00306497221658262\\
300	0.00306591390267762\\
301	0.00306687159959031\\
302	0.00306784556615013\\
303	0.00306883606496656\\
304	0.00306984336247477\\
305	0.00307086772898195\\
306	0.00307190943871459\\
307	0.00307296876986641\\
308	0.00307404600464744\\
309	0.0030751414293342\\
310	0.00307625533432112\\
311	0.00307738801417344\\
312	0.00307853976768165\\
313	0.00307971089791798\\
314	0.00308090171229491\\
315	0.00308211252262612\\
316	0.00308334364519038\\
317	0.00308459540079854\\
318	0.00308586811486459\\
319	0.00308716211748075\\
320	0.00308847774349799\\
321	0.00308981533261227\\
322	0.00309117522945808\\
323	0.00309255778370995\\
324	0.00309396335019373\\
325	0.00309539228900961\\
326	0.00309684496566861\\
327	0.00309832175124565\\
328	0.00309982302255223\\
329	0.00310134916233282\\
330	0.00310290055949007\\
331	0.00310447760934495\\
332	0.00310608071393943\\
333	0.00310771028239149\\
334	0.00310936673131448\\
335	0.0031110504853147\\
336	0.00311276197758505\\
337	0.00311450165061221\\
338	0.00311626995701265\\
339	0.00311806736049661\\
340	0.00311989433692556\\
341	0.00312175137539228\\
342	0.00312363897943671\\
343	0.003125557670164\\
344	0.0031275080020705\\
345	0.00312949059349016\\
346	0.00313150609232436\\
347	0.00313355517864743\\
348	0.00313563856738285\\
349	0.0031377570110202\\
350	0.00313991130239194\\
351	0.00314210227763664\\
352	0.00314433081954981\\
353	0.00314659786080227\\
354	0.00314890438113493\\
355	0.00315125136603214\\
356	0.00315363958019717\\
357	0.00315606978981859\\
358	0.00315854278111835\\
359	0.00316105936144412\\
360	0.00316362036046411\\
361	0.00316622663147663\\
362	0.00316887905284827\\
363	0.00317157852959675\\
364	0.00317432599513677\\
365	0.00317712241320975\\
366	0.00317996878002174\\
367	0.00318286612661715\\
368	0.00318581552152086\\
369	0.00318881807368557\\
370	0.00319187493578809\\
371	0.0031949873079246\\
372	0.00319815644176414\\
373	0.00320138364522874\\
374	0.0032046702877813\\
375	0.00320801780641599\\
376	0.00321142771246325\\
377	0.00321490159934192\\
378	0.00321844115141562\\
379	0.0032220481541401\\
380	0.00322572450572489\\
381	0.00322947223057536\\
382	0.003233293494836\\
383	0.00323719062441954\\
384	0.00324116612598733\\
385	0.0032452227114445\\
386	0.00324936332663442\\
387	0.00325359118506684\\
388	0.00325790980770063\\
389	0.00326232307003289\\
390	0.00326683525803616\\
391	0.00327145113484695\\
392	0.00327617602056441\\
393	0.00328101588809075\\
394	0.00328597747866928\\
395	0.00329106844169038\\
396	0.00329629750449335\\
397	0.00330167467934666\\
398	0.00330721151660432\\
399	0.00331292141582399\\
400	0.0033188200096813\\
401	0.00332492563927259\\
402	0.00333125994479976\\
403	0.00333784860241919\\
404	0.00334472224685625\\
405	0.00335191763085759\\
406	0.00335947908746879\\
407	0.00336746038041965\\
408	0.00337592705245904\\
409	0.00338495941124689\\
410	0.00339465632310242\\
411	0.00340513999615007\\
412	0.0034165618451223\\
413	0.00342910904764097\\
414	0.00344300944672684\\
415	0.00345755801444051\\
416	0.00347234515899185\\
417	0.00348737409996265\\
418	0.00350264800731524\\
419	0.00351816984114086\\
420	0.0035339418067238\\
421	0.00354996568105268\\
422	0.00356624732542498\\
423	0.00358279709431792\\
424	0.00359963373631932\\
425	0.00361676185217926\\
426	0.00363418607612658\\
427	0.00365191106702642\\
428	0.00366994149738172\\
429	0.00368828203972282\\
430	0.00370693734984093\\
431	0.00372591204624422\\
432	0.00374521068514317\\
433	0.00376483772985987\\
434	0.0037847975113347\\
435	0.00380509417336685\\
436	0.00382573160994369\\
437	0.00384671338743754\\
438	0.00386804264795385\\
439	0.00388972198915353\\
440	0.00391175331465244\\
441	0.0039341376475423\\
442	0.00395687489758588\\
443	0.00397996357008652\\
444	0.00400340040115312\\
445	0.00402717989986627\\
446	0.00405129377242333\\
447	0.00407573019637187\\
448	0.0041004729041679\\
449	0.00412550002427073\\
450	0.00415078261515364\\
451	0.00417628281527402\\
452	0.00420195147286436\\
453	0.00422772500064694\\
454	0.00425352122695221\\
455	0.00427923405601563\\
456	0.00430472705272688\\
457	0.00432982539830094\\
458	0.00435430509663297\\
459	0.00437787912462104\\
460	0.00440018001228618\\
461	0.00442073945979756\\
462	0.00443897091856598\\
463	0.00445725960360944\\
464	0.00447580055336444\\
465	0.00449458924557552\\
466	0.00451361983613378\\
467	0.00453288517536736\\
468	0.00455237665129548\\
469	0.00457208401945484\\
470	0.00459199521965731\\
471	0.00461209618082414\\
472	0.00463237061681782\\
473	0.00465279982181396\\
474	0.00467336248284928\\
475	0.00469403452956161\\
476	0.0047147890479902\\
477	0.00473559627477015\\
478	0.00475642369917354\\
479	0.00477723634335092\\
480	0.00479799731346455\\
481	0.00481866874103088\\
482	0.00483921327983798\\
483	0.00485959638845276\\
484	0.00487978971827209\\
485	0.00489977605615691\\
486	0.00491955646208567\\
487	0.00493916053055873\\
488	0.00495866108989554\\
489	0.00497819474011869\\
490	0.00499796145418809\\
491	0.0050180776608813\\
492	0.00503855184396394\\
493	0.00505939348956257\\
494	0.0050806133032376\\
495	0.00510222347856746\\
496	0.00512423801255947\\
497	0.00514667307048559\\
498	0.00516954748368211\\
499	0.00519288328983599\\
500	0.00521670633080219\\
501	0.00524104668724238\\
502	0.00526593805116734\\
503	0.00529141966021656\\
504	0.00531753775955137\\
505	0.00534435061159844\\
506	0.00537195477964907\\
507	0.0054004263733106\\
508	0.00542983977475685\\
509	0.00546026936773654\\
510	0.00549178321330305\\
511	0.00552444340442616\\
512	0.00555831555240889\\
513	0.00559346761621968\\
514	0.00562997120760527\\
515	0.00566789562440533\\
516	0.00570725572049871\\
517	0.00574812831483058\\
518	0.0057905930740293\\
519	0.0058347319712251\\
520	0.00588062854546756\\
521	0.0059283669909499\\
522	0.00597803091455968\\
523	0.00602970178255804\\
524	0.0060834574252944\\
525	0.00613936985774463\\
526	0.00619754555422584\\
527	0.00625807035375208\\
528	0.00632097800128714\\
529	0.00638628421195292\\
530	0.00645398122847025\\
531	0.00652402827496695\\
532	0.00659634488034116\\
533	0.00667080369120464\\
534	0.00674722577632755\\
535	0.00682543605429951\\
536	0.00690517067081394\\
537	0.006985979879753\\
538	0.0070671353684765\\
539	0.00714694473789928\\
540	0.00722171707236833\\
541	0.00729072305803338\\
542	0.00735364306074771\\
543	0.00741047186832885\\
544	0.00746351245621795\\
545	0.00751458096933631\\
546	0.0075642019222949\\
547	0.00761304925249949\\
548	0.00766168557552592\\
549	0.00771048316522664\\
550	0.00775963560327215\\
551	0.00780930056370565\\
552	0.00785961270762311\\
553	0.00791065670204328\\
554	0.00796248760120481\\
555	0.00801519227973258\\
556	0.00806886380586354\\
557	0.00812360765985678\\
558	0.0081776104431567\\
559	0.00823054447228591\\
560	0.00828212481938252\\
561	0.00833330018496336\\
562	0.00838468146777848\\
563	0.00843628943921433\\
564	0.0084881762700586\\
565	0.00854036375010845\\
566	0.00859280326717225\\
567	0.00864547347042305\\
568	0.00869831647947506\\
569	0.00874951291989383\\
570	0.0087989032234281\\
571	0.00884783920707862\\
572	0.00889659845102266\\
573	0.00894523117842181\\
574	0.0089937388115468\\
575	0.00904204227672353\\
576	0.00908965434686499\\
577	0.00913682916240663\\
578	0.0091838363449705\\
579	0.00923066223904429\\
580	0.00927726451044059\\
581	0.00932359040395203\\
582	0.00936958612525249\\
583	0.00941519840888499\\
584	0.00946037591227052\\
585	0.00950507093368084\\
586	0.00954924157818994\\
587	0.00959285445321423\\
588	0.00963588792379213\\
589	0.00967833579815023\\
590	0.00972021084271538\\
591	0.00976138933978722\\
592	0.00980171928775088\\
593	0.00984102135019415\\
594	0.00987905595166784\\
595	0.00991543381058343\\
596	0.00994937493726701\\
597	0.00997906286423442\\
598	0.0099999191923403\\
599	0\\
600	0\\
};
\addplot [color=red!75!mycolor17,solid,forget plot]
  table[row sep=crcr]{%
1	0.00330282272054889\\
2	0.00330283342163309\\
3	0.00330284431478508\\
4	0.00330285540344943\\
5	0.00330286669113244\\
6	0.00330287818140321\\
7	0.00330288987789482\\
8	0.0033029017843054\\
9	0.00330291390439932\\
10	0.00330292624200841\\
11	0.00330293880103316\\
12	0.00330295158544386\\
13	0.00330296459928195\\
14	0.00330297784666129\\
15	0.00330299133176942\\
16	0.00330300505886883\\
17	0.00330301903229842\\
18	0.00330303325647477\\
19	0.00330304773589353\\
20	0.00330306247513091\\
21	0.00330307747884505\\
22	0.00330309275177754\\
23	0.0033031082987548\\
24	0.00330312412468978\\
25	0.00330314023458331\\
26	0.00330315663352584\\
27	0.00330317332669894\\
28	0.00330319031937693\\
29	0.00330320761692861\\
30	0.00330322522481887\\
31	0.00330324314861042\\
32	0.00330326139396562\\
33	0.00330327996664815\\
34	0.0033032988725249\\
35	0.00330331811756777\\
36	0.00330333770785555\\
37	0.00330335764957587\\
38	0.00330337794902708\\
39	0.00330339861262027\\
40	0.0033034196468813\\
41	0.00330344105845283\\
42	0.00330346285409639\\
43	0.00330348504069449\\
44	0.00330350762525285\\
45	0.00330353061490255\\
46	0.00330355401690227\\
47	0.00330357783864058\\
48	0.00330360208763825\\
49	0.00330362677155063\\
50	0.00330365189817003\\
51	0.0033036774754282\\
52	0.00330370351139872\\
53	0.0033037300142997\\
54	0.00330375699249623\\
55	0.00330378445450304\\
56	0.00330381240898714\\
57	0.00330384086477062\\
58	0.00330386983083331\\
59	0.00330389931631573\\
60	0.0033039293305218\\
61	0.00330395988292187\\
62	0.00330399098315565\\
63	0.00330402264103528\\
64	0.00330405486654828\\
65	0.00330408766986083\\
66	0.00330412106132083\\
67	0.00330415505146126\\
68	0.00330418965100335\\
69	0.00330422487086005\\
70	0.0033042607221394\\
71	0.00330429721614801\\
72	0.00330433436439457\\
73	0.00330437217859355\\
74	0.00330441067066876\\
75	0.00330444985275714\\
76	0.00330448973721252\\
77	0.00330453033660952\\
78	0.00330457166374751\\
79	0.0033046137316545\\
80	0.00330465655359143\\
81	0.003304700143056\\
82	0.00330474451378721\\
83	0.00330478967976943\\
84	0.00330483565523691\\
85	0.00330488245467818\\
86	0.00330493009284055\\
87	0.00330497858473473\\
88	0.00330502794563957\\
89	0.00330507819110677\\
90	0.00330512933696575\\
91	0.00330518139932865\\
92	0.00330523439459526\\
93	0.00330528833945823\\
94	0.00330534325090826\\
95	0.00330539914623935\\
96	0.00330545604305425\\
97	0.00330551395926991\\
98	0.00330557291312314\\
99	0.00330563292317622\\
100	0.00330569400832267\\
101	0.00330575618779331\\
102	0.003305819481162\\
103	0.00330588390835195\\
104	0.00330594948964182\\
105	0.00330601624567204\\
106	0.0033060841974513\\
107	0.00330615336636303\\
108	0.00330622377417209\\
109	0.00330629544303151\\
110	0.00330636839548943\\
111	0.00330644265449607\\
112	0.00330651824341091\\
113	0.00330659518600989\\
114	0.00330667350649289\\
115	0.00330675322949116\\
116	0.00330683438007506\\
117	0.00330691698376181\\
118	0.00330700106652339\\
119	0.00330708665479459\\
120	0.00330717377548134\\
121	0.00330726245596888\\
122	0.00330735272413047\\
123	0.0033074446083358\\
124	0.00330753813745996\\
125	0.00330763334089236\\
126	0.0033077302485458\\
127	0.00330782889086571\\
128	0.00330792929883973\\
129	0.00330803150400712\\
130	0.00330813553846866\\
131	0.00330824143489649\\
132	0.00330834922654435\\
133	0.00330845894725767\\
134	0.00330857063148426\\
135	0.00330868431428477\\
136	0.00330880003134363\\
137	0.00330891781898009\\
138	0.00330903771415932\\
139	0.00330915975450399\\
140	0.00330928397830574\\
141	0.00330941042453704\\
142	0.00330953913286325\\
143	0.00330967014365478\\
144	0.00330980349799959\\
145	0.00330993923771584\\
146	0.00331007740536477\\
147	0.00331021804426372\\
148	0.00331036119849963\\
149	0.00331050691294246\\
150	0.00331065523325904\\
151	0.0033108062059271\\
152	0.00331095987824959\\
153	0.00331111629836915\\
154	0.00331127551528291\\
155	0.00331143757885755\\
156	0.00331160253984459\\
157	0.00331177044989592\\
158	0.00331194136157969\\
159	0.00331211532839636\\
160	0.0033122924047951\\
161	0.00331247264619046\\
162	0.00331265610897938\\
163	0.00331284285055829\\
164	0.00331303292934082\\
165	0.00331322640477553\\
166	0.00331342333736411\\
167	0.00331362378867986\\
168	0.00331382782138642\\
169	0.00331403549925693\\
170	0.00331424688719343\\
171	0.0033144620512466\\
172	0.00331468105863596\\
173	0.00331490397777025\\
174	0.00331513087826821\\
175	0.00331536183097977\\
176	0.00331559690800757\\
177	0.00331583618272889\\
178	0.00331607972981778\\
179	0.00331632762526781\\
180	0.0033165799464151\\
181	0.00331683677196163\\
182	0.00331709818199918\\
183	0.00331736425803347\\
184	0.00331763508300882\\
185	0.00331791074133321\\
186	0.00331819131890377\\
187	0.00331847690313258\\
188	0.00331876758297316\\
189	0.0033190634489472\\
190	0.00331936459317178\\
191	0.00331967110938712\\
192	0.00331998309298471\\
193	0.00332030064103599\\
194	0.00332062385232146\\
195	0.00332095282736034\\
196	0.00332128766844062\\
197	0.00332162847964977\\
198	0.0033219753669058\\
199	0.00332232843798895\\
200	0.00332268780257393\\
201	0.00332305357226255\\
202	0.00332342586061704\\
203	0.00332380478319396\\
204	0.00332419045757847\\
205	0.00332458300341937\\
206	0.00332498254246467\\
207	0.00332538919859769\\
208	0.00332580309787378\\
209	0.00332622436855776\\
210	0.00332665314116178\\
211	0.003327089548484\\
212	0.00332753372564779\\
213	0.00332798581014162\\
214	0.00332844594185967\\
215	0.0033289142631429\\
216	0.00332939091882108\\
217	0.0033298760562552\\
218	0.00333036982538093\\
219	0.00333087237875247\\
220	0.00333138387158732\\
221	0.00333190446181167\\
222	0.0033324343101066\\
223	0.00333297357995497\\
224	0.00333352243768914\\
225	0.00333408105253948\\
226	0.00333464959668346\\
227	0.00333522824529588\\
228	0.00333581717659961\\
229	0.00333641657191728\\
230	0.00333702661572385\\
231	0.0033376474956999\\
232	0.00333827940278585\\
233	0.0033389225312371\\
234	0.00333957707867991\\
235	0.00334024324616826\\
236	0.00334092123824171\\
237	0.00334161126298399\\
238	0.00334231353208262\\
239	0.00334302826088954\\
240	0.00334375566848259\\
241	0.00334449597772797\\
242	0.00334524941534377\\
243	0.0033460162119644\\
244	0.00334679660220614\\
245	0.0033475908247336\\
246	0.00334839912232735\\
247	0.00334922174195242\\
248	0.00335005893482809\\
249	0.00335091095649854\\
250	0.00335177806690484\\
251	0.00335266053045776\\
252	0.00335355861611198\\
253	0.00335447259744131\\
254	0.00335540275271499\\
255	0.00335634936497527\\
256	0.00335731272211621\\
257	0.0033582931169635\\
258	0.00335929084735563\\
259	0.00336030621622627\\
260	0.00336133953168787\\
261	0.00336239110711638\\
262	0.0033634612612376\\
263	0.00336455031821432\\
264	0.00336565860773516\\
265	0.0033667864651045\\
266	0.0033679342313338\\
267	0.00336910225323429\\
268	0.00337029088351086\\
269	0.00337150048085747\\
270	0.00337273141005392\\
271	0.00337398404206385\\
272	0.00337525875413432\\
273	0.00337655592989667\\
274	0.00337787595946882\\
275	0.00337921923955908\\
276	0.00338058617357127\\
277	0.00338197717171143\\
278	0.00338339265109577\\
279	0.00338483303586036\\
280	0.00338629875727196\\
281	0.00338779025384068\\
282	0.00338930797143373\\
283	0.00339085236339104\\
284	0.00339242389064208\\
285	0.00339402302182419\\
286	0.0033956502334026\\
287	0.00339730600979159\\
288	0.00339899084347746\\
289	0.00340070523514277\\
290	0.00340244969379195\\
291	0.00340422473687872\\
292	0.00340603089043444\\
293	0.00340786868919834\\
294	0.00340973867674884\\
295	0.0034116414056363\\
296	0.00341357743751731\\
297	0.00341554734328991\\
298	0.00341755170323048\\
299	0.00341959110713156\\
300	0.00342166615444105\\
301	0.00342377745440246\\
302	0.00342592562619619\\
303	0.0034281112990819\\
304	0.00343033511254174\\
305	0.00343259771642438\\
306	0.00343489977108982\\
307	0.00343724194755478\\
308	0.00343962492763856\\
309	0.00344204940410927\\
310	0.00344451608083008\\
311	0.00344702567290555\\
312	0.00344957890682761\\
313	0.00345217652062094\\
314	0.00345481926398756\\
315	0.00345750789845017\\
316	0.00346024319749367\\
317	0.00346302594670484\\
318	0.00346585694390895\\
319	0.00346873699930295\\
320	0.00347166693558459\\
321	0.00347464758807583\\
322	0.00347767980484001\\
323	0.00348076444679074\\
324	0.00348390238779121\\
325	0.00348709451474115\\
326	0.00349034172764914\\
327	0.00349364493968681\\
328	0.0034970050772207\\
329	0.0035004230798166\\
330	0.00350389990020994\\
331	0.00350743650423407\\
332	0.00351103387069592\\
333	0.00351469299118633\\
334	0.00351841486980706\\
335	0.00352220052279181\\
336	0.00352605097798613\\
337	0.00352996727412774\\
338	0.00353395045980457\\
339	0.00353800159178545\\
340	0.00354212173185045\\
341	0.00354631193944075\\
342	0.00355057325157021\\
343	0.00355490662209339\\
344	0.00355931272808235\\
345	0.00356379189166622\\
346	0.00356834515469216\\
347	0.00357297356679901\\
348	0.00357767818776288\\
349	0.00358246009188216\\
350	0.00358732037653211\\
351	0.00359226018023511\\
352	0.00359728072439287\\
353	0.00360238341730273\\
354	0.00360757012792604\\
355	0.00361284393236545\\
356	0.00361821119603823\\
357	0.00362367390982745\\
358	0.00362923375132823\\
359	0.00363489242754245\\
360	0.00364065167534496\\
361	0.00364651326193577\\
362	0.00365247898527317\\
363	0.00365855067448048\\
364	0.00366473019021816\\
365	0.00367101942501133\\
366	0.0036774203035199\\
367	0.00368393478273699\\
368	0.00369056485209686\\
369	0.003697312533471\\
370	0.00370417988102549\\
371	0.00371116898090746\\
372	0.00371828195072195\\
373	0.00372552093875149\\
374	0.00373288812286142\\
375	0.00374038570902078\\
376	0.00374801592935452\\
377	0.00375578103962323\\
378	0.00376368331600495\\
379	0.00377172505102493\\
380	0.00377990854844547\\
381	0.00378823611688516\\
382	0.0037967100618844\\
383	0.00380533267606928\\
384	0.00381410622698426\\
385	0.00382303294206365\\
386	0.00383211499008518\\
387	0.00384135445829135\\
388	0.00385075332416488\\
389	0.00386031342059559\\
390	0.00387003639285955\\
391	0.00387992364543375\\
392	0.0038899762761649\\
393	0.0039001949946742\\
394	0.00391058002108038\\
395	0.00392113096013745\\
396	0.00393184664472379\\
397	0.00394272494144054\\
398	0.00395376251048509\\
399	0.00396495449586735\\
400	0.00397629413257958\\
401	0.00398777225813496\\
402	0.00399937669498494\\
403	0.00401109146763229\\
404	0.00402289580756077\\
405	0.00403476288509919\\
406	0.00404665818896517\\
407	0.00405853745010709\\
408	0.00407034397486529\\
409	0.00408200521173058\\
410	0.0040934283265161\\
411	0.00410449451422482\\
412	0.00411505185245532\\
413	0.00412490688083491\\
414	0.00413381680457718\\
415	0.00414246359764626\\
416	0.00415125134065076\\
417	0.00416018176530774\\
418	0.00416925659035271\\
419	0.00417847751893866\\
420	0.00418784624035994\\
421	0.00419736444267878\\
422	0.00420703380198752\\
423	0.00421685584956833\\
424	0.00422683181202162\\
425	0.00423696240349508\\
426	0.00424724815917203\\
427	0.00425768941180764\\
428	0.00426828626551413\\
429	0.00427903856647297\\
430	0.00428994587021759\\
431	0.00430100740509318\\
432	0.00431222203145255\\
433	0.00432358819609439\\
434	0.00433510388148189\\
435	0.00434676654945613\\
436	0.00435857307901268\\
437	0.00437051969760218\\
438	0.00438260190559301\\
439	0.00439481439365558\\
440	0.00440715095302605\\
441	0.00441960437891222\\
442	0.00443216636776108\\
443	0.00444482740977402\\
444	0.00445757667901259\\
445	0.00447040192479705\\
446	0.00448328937001098\\
447	0.00449622362459359\\
448	0.00450918762620207\\
449	0.00452216262508911\\
450	0.00453512823714873\\
451	0.00454806259823822\\
452	0.00456094266652204\\
453	0.00457374474231479\\
454	0.00458644530390559\\
455	0.00459902229070082\\
456	0.00461145699868897\\
457	0.00462373681301679\\
458	0.00463585913954865\\
459	0.00464783703643459\\
460	0.00465970727157742\\
461	0.00467154172123509\\
462	0.00468346221395803\\
463	0.00469554963119461\\
464	0.00470781029851564\\
465	0.00472024469819738\\
466	0.00473285328017813\\
467	0.00474563649024712\\
468	0.00475859480719211\\
469	0.00477172879083086\\
470	0.00478503914318833\\
471	0.00479852678546889\\
472	0.00481219295390058\\
473	0.0048260393178048\\
474	0.00484006812319904\\
475	0.00485428236509176\\
476	0.0048686859910726\\
477	0.00488328413879259\\
478	0.00489808340948448\\
479	0.00491309217733719\\
480	0.00492832090057559\\
481	0.0049437824638965\\
482	0.00495949254182109\\
483	0.0049754699354219\\
484	0.00499173683140299\\
485	0.00500831890355531\\
486	0.00502524513177611\\
487	0.00504254713582935\\
488	0.00506025766691289\\
489	0.0050784076222585\\
490	0.00509702183527401\\
491	0.00511612026777347\\
492	0.00513572427144415\\
493	0.00515585814725386\\
494	0.00517654827885316\\
495	0.005197823043026\\
496	0.0052197129940454\\
497	0.00524225100489788\\
498	0.00526547092274689\\
499	0.00528940925192003\\
500	0.00531410682208839\\
501	0.00533961427475088\\
502	0.00536600662699488\\
503	0.00539333449201868\\
504	0.00542164384722157\\
505	0.00545093317709244\\
506	0.00548125584165916\\
507	0.00551266631417104\\
508	0.00554522086648375\\
509	0.00557897776063199\\
510	0.00561399747215752\\
511	0.00565034295582015\\
512	0.00568807961924574\\
513	0.00572727514130953\\
514	0.0057679989351695\\
515	0.00581032699449293\\
516	0.00585439423092452\\
517	0.00590027932527563\\
518	0.00594806052659907\\
519	0.00599781429473797\\
520	0.00604961359373404\\
521	0.00610352620502252\\
522	0.00615961228747102\\
523	0.00621792029876914\\
524	0.0062784771031797\\
525	0.0063412833477263\\
526	0.00640631260834304\\
527	0.00647349981735246\\
528	0.00654273540940123\\
529	0.00661386712290187\\
530	0.0066866853889961\\
531	0.00676099618491231\\
532	0.00683645871329546\\
533	0.00691251557173578\\
534	0.00698820960614814\\
535	0.0070600184570712\\
536	0.00712619182055303\\
537	0.00718635518837399\\
538	0.00724038537103614\\
539	0.0072891926067398\\
540	0.00733611246284593\\
541	0.00738175075357763\\
542	0.00742656905747836\\
543	0.00747111430771151\\
544	0.00751581455378255\\
545	0.00756089746622922\\
546	0.00760651824603813\\
547	0.00765279954637135\\
548	0.00769981995261393\\
549	0.00774763076508289\\
550	0.00779627233522115\\
551	0.00784578270503273\\
552	0.00789623511963474\\
553	0.00794771179525126\\
554	0.00800011671826545\\
555	0.00805165496701063\\
556	0.00810209881618632\\
557	0.0081511534873175\\
558	0.00820039187507906\\
559	0.00824993373640694\\
560	0.00829982916153586\\
561	0.00835013282549448\\
562	0.00840084161324492\\
563	0.00845191747075763\\
564	0.00850331152133759\\
565	0.0085549989209369\\
566	0.00860690126516819\\
567	0.00865718321445381\\
568	0.00870571147056739\\
569	0.00875412101410349\\
570	0.00880247830324436\\
571	0.00885082904818314\\
572	0.00889914614462142\\
573	0.00894739710985454\\
574	0.00899525183449353\\
575	0.00904253590053291\\
576	0.00908973605534923\\
577	0.0091368486117437\\
578	0.00918384152338781\\
579	0.00923066477496741\\
580	0.00927726586124061\\
581	0.00932359110588857\\
582	0.00936958646951639\\
583	0.0094151985649821\\
584	0.00946037597636478\\
585	0.00950507095687452\\
586	0.00954924158529672\\
587	0.0095928544549437\\
588	0.00963588792408897\\
589	0.00967833579817709\\
590	0.00972021084271539\\
591	0.00976138933978722\\
592	0.00980171928775089\\
593	0.00984102135019415\\
594	0.00987905595166785\\
595	0.00991543381058343\\
596	0.00994937493726701\\
597	0.00997906286423442\\
598	0.0099999191923403\\
599	0\\
600	0\\
};
\addplot [color=red!80!mycolor19,solid,forget plot]
  table[row sep=crcr]{%
1	0.00391309393964753\\
2	0.00391310035145049\\
3	0.00391310687863073\\
4	0.00391311352326403\\
5	0.0039131202874635\\
6	0.00391312717338031\\
7	0.00391313418320423\\
8	0.00391314131916451\\
9	0.00391314858353044\\
10	0.00391315597861217\\
11	0.00391316350676137\\
12	0.00391317117037203\\
13	0.00391317897188121\\
14	0.00391318691376976\\
15	0.00391319499856321\\
16	0.00391320322883247\\
17	0.00391321160719473\\
18	0.00391322013631422\\
19	0.0039132288189031\\
20	0.00391323765772234\\
21	0.00391324665558253\\
22	0.00391325581534482\\
23	0.00391326513992186\\
24	0.00391327463227862\\
25	0.00391328429543347\\
26	0.003913294132459\\
27	0.00391330414648312\\
28	0.00391331434068997\\
29	0.00391332471832095\\
30	0.00391333528267575\\
31	0.00391334603711346\\
32	0.00391335698505355\\
33	0.00391336812997696\\
34	0.00391337947542727\\
35	0.00391339102501179\\
36	0.00391340278240267\\
37	0.00391341475133813\\
38	0.00391342693562365\\
39	0.00391343933913308\\
40	0.00391345196580999\\
41	0.00391346481966886\\
42	0.00391347790479635\\
43	0.00391349122535264\\
44	0.00391350478557267\\
45	0.00391351858976759\\
46	0.0039135326423261\\
47	0.00391354694771575\\
48	0.00391356151048447\\
49	0.00391357633526195\\
50	0.00391359142676115\\
51	0.0039136067897798\\
52	0.00391362242920189\\
53	0.0039136383499992\\
54	0.00391365455723295\\
55	0.00391367105605535\\
56	0.00391368785171129\\
57	0.00391370494953988\\
58	0.00391372235497636\\
59	0.00391374007355357\\
60	0.00391375811090393\\
61	0.00391377647276107\\
62	0.00391379516496177\\
63	0.00391381419344772\\
64	0.00391383356426739\\
65	0.00391385328357808\\
66	0.00391387335764775\\
67	0.00391389379285705\\
68	0.00391391459570137\\
69	0.00391393577279287\\
70	0.00391395733086256\\
71	0.00391397927676248\\
72	0.00391400161746784\\
73	0.00391402436007925\\
74	0.00391404751182495\\
75	0.00391407108006316\\
76	0.00391409507228431\\
77	0.00391411949611354\\
78	0.00391414435931302\\
79	0.00391416966978445\\
80	0.00391419543557158\\
81	0.00391422166486273\\
82	0.00391424836599338\\
83	0.00391427554744888\\
84	0.00391430321786703\\
85	0.00391433138604094\\
86	0.00391436006092166\\
87	0.00391438925162124\\
88	0.00391441896741536\\
89	0.00391444921774651\\
90	0.00391448001222679\\
91	0.00391451136064106\\
92	0.00391454327295003\\
93	0.00391457575929336\\
94	0.00391460882999292\\
95	0.00391464249555611\\
96	0.00391467676667905\\
97	0.00391471165425006\\
98	0.0039147471693531\\
99	0.00391478332327129\\
100	0.00391482012749046\\
101	0.00391485759370267\\
102	0.00391489573381023\\
103	0.00391493455992908\\
104	0.00391497408439289\\
105	0.00391501431975687\\
106	0.00391505527880174\\
107	0.0039150969745378\\
108	0.00391513942020896\\
109	0.00391518262929707\\
110	0.00391522661552606\\
111	0.00391527139286634\\
112	0.00391531697553919\\
113	0.00391536337802129\\
114	0.00391541061504923\\
115	0.00391545870162422\\
116	0.00391550765301681\\
117	0.00391555748477168\\
118	0.00391560821271267\\
119	0.00391565985294758\\
120	0.00391571242187345\\
121	0.00391576593618165\\
122	0.00391582041286306\\
123	0.00391587586921361\\
124	0.0039159323228396\\
125	0.00391598979166334\\
126	0.00391604829392865\\
127	0.00391610784820688\\
128	0.00391616847340245\\
129	0.00391623018875904\\
130	0.00391629301386557\\
131	0.00391635696866237\\
132	0.00391642207344744\\
133	0.00391648834888292\\
134	0.0039165558160015\\
135	0.00391662449621307\\
136	0.0039166944113115\\
137	0.00391676558348142\\
138	0.0039168380353053\\
139	0.00391691178977044\\
140	0.00391698687027623\\
141	0.00391706330064156\\
142	0.0039171411051122\\
143	0.00391722030836855\\
144	0.00391730093553329\\
145	0.0039173830121792\\
146	0.00391746656433737\\
147	0.00391755161850521\\
148	0.00391763820165484\\
149	0.00391772634124151\\
150	0.00391781606521217\\
151	0.00391790740201436\\
152	0.00391800038060491\\
153	0.00391809503045919\\
154	0.00391819138158023\\
155	0.00391828946450815\\
156	0.00391838931032967\\
157	0.00391849095068785\\
158	0.00391859441779201\\
159	0.00391869974442765\\
160	0.00391880696396692\\
161	0.0039189161103788\\
162	0.0039190272182398\\
163	0.0039191403227448\\
164	0.00391925545971783\\
165	0.00391937266562344\\
166	0.0039194919775779\\
167	0.0039196134333608\\
168	0.00391973707142682\\
169	0.00391986293091758\\
170	0.003919991051674\\
171	0.00392012147424852\\
172	0.00392025423991759\\
173	0.0039203893906947\\
174	0.00392052696934325\\
175	0.00392066701938984\\
176	0.00392080958513773\\
177	0.00392095471168059\\
178	0.00392110244491633\\
179	0.00392125283156142\\
180	0.00392140591916523\\
181	0.00392156175612471\\
182	0.00392172039169928\\
183	0.00392188187602615\\
184	0.00392204626013556\\
185	0.00392221359596661\\
186	0.00392238393638313\\
187	0.00392255733519008\\
188	0.00392273384714987\\
189	0.00392291352799926\\
190	0.00392309643446646\\
191	0.0039232826242884\\
192	0.00392347215622846\\
193	0.00392366509009446\\
194	0.00392386148675686\\
195	0.00392406140816729\\
196	0.00392426491737753\\
197	0.00392447207855869\\
198	0.00392468295702064\\
199	0.00392489761923196\\
200	0.00392511613284003\\
201	0.00392533856669158\\
202	0.00392556499085358\\
203	0.00392579547663438\\
204	0.00392603009660527\\
205	0.00392626892462241\\
206	0.00392651203584904\\
207	0.00392675950677819\\
208	0.00392701141525566\\
209	0.00392726784050333\\
210	0.00392752886314305\\
211	0.00392779456522076\\
212	0.00392806503023093\\
213	0.00392834034314166\\
214	0.00392862059041988\\
215	0.00392890586005726\\
216	0.00392919624159628\\
217	0.00392949182615693\\
218	0.00392979270646361\\
219	0.00393009897687269\\
220	0.00393041073340037\\
221	0.00393072807375105\\
222	0.00393105109734605\\
223	0.00393137990535299\\
224	0.00393171460071542\\
225	0.00393205528818294\\
226	0.00393240207434204\\
227	0.00393275506764708\\
228	0.00393311437845195\\
229	0.00393348011904223\\
230	0.00393385240366776\\
231	0.00393423134857578\\
232	0.00393461707204458\\
233	0.00393500969441757\\
234	0.00393540933813807\\
235	0.00393581612778442\\
236	0.0039362301901057\\
237	0.0039366516540581\\
238	0.00393708065084169\\
239	0.00393751731393771\\
240	0.00393796177914665\\
241	0.0039384141846267\\
242	0.00393887467093278\\
243	0.00393934338105621\\
244	0.00393982046046497\\
245	0.00394030605714449\\
246	0.00394080032163905\\
247	0.00394130340709382\\
248	0.00394181546929747\\
249	0.00394233666672541\\
250	0.00394286716058364\\
251	0.00394340711485321\\
252	0.00394395669633537\\
253	0.00394451607469714\\
254	0.00394508542251787\\
255	0.00394566491533617\\
256	0.00394625473169753\\
257	0.00394685505320269\\
258	0.00394746606455656\\
259	0.00394808795361791\\
260	0.00394872091144956\\
261	0.00394936513236943\\
262	0.00395002081400208\\
263	0.00395068815733114\\
264	0.00395136736675211\\
265	0.0039520586501262\\
266	0.00395276221883456\\
267	0.00395347828783326\\
268	0.00395420707570917\\
269	0.0039549488047362\\
270	0.00395570370093242\\
271	0.00395647199411784\\
272	0.00395725391797283\\
273	0.0039580497100973\\
274	0.00395885961207044\\
275	0.00395968386951128\\
276	0.00396052273213981\\
277	0.00396137645383877\\
278	0.00396224529271628\\
279	0.00396312951116883\\
280	0.00396402937594523\\
281	0.00396494515821093\\
282	0.00396587713361319\\
283	0.00396682558234666\\
284	0.00396779078921978\\
285	0.00396877304372167\\
286	0.00396977264008943\\
287	0.00397078987737639\\
288	0.00397182505952052\\
289	0.00397287849541347\\
290	0.00397395049897024\\
291	0.00397504138919915\\
292	0.00397615149027226\\
293	0.00397728113159628\\
294	0.0039784306478837\\
295	0.00397960037922457\\
296	0.00398079067115801\\
297	0.00398200187474465\\
298	0.00398323434663867\\
299	0.00398448844916045\\
300	0.00398576455036895\\
301	0.00398706302413444\\
302	0.003988384250211\\
303	0.00398972861430908\\
304	0.00399109650816784\\
305	0.00399248832962729\\
306	0.0039939044827002\\
307	0.00399534537764357\\
308	0.00399681143102979\\
309	0.00399830306581716\\
310	0.00399982071141991\\
311	0.0040013648037776\\
312	0.00400293578542367\\
313	0.0040045341055533\\
314	0.00400616022009033\\
315	0.00400781459175329\\
316	0.00400949769012048\\
317	0.00401120999169397\\
318	0.00401295197996263\\
319	0.00401472414546422\\
320	0.00401652698584636\\
321	0.00401836100592681\\
322	0.00402022671775282\\
323	0.00402212464066014\\
324	0.00402405530133158\\
325	0.00402601923385591\\
326	0.00402801697978755\\
327	0.00403004908820754\\
328	0.00403211611578721\\
329	0.00403421862685557\\
330	0.00403635719347254\\
331	0.00403853239550991\\
332	0.00404074482074375\\
333	0.00404299506496178\\
334	0.00404528373209185\\
335	0.00404761143435832\\
336	0.00404997879247634\\
337	0.00405238643589678\\
338	0.00405483500311943\\
339	0.00405732514209878\\
340	0.00405985751078104\\
341	0.00406243277784526\\
342	0.00406505162382405\\
343	0.00406771474309292\\
344	0.00407042284814956\\
345	0.0040731766793176\\
346	0.00407597701430165\\
347	0.00407882465138177\\
348	0.00408172041067304\\
349	0.00408466513531158\\
350	0.00408765969244745\\
351	0.00409070497383421\\
352	0.00409380189560034\\
353	0.00409695139626494\\
354	0.00410015443066837\\
355	0.00410341195380954\\
356	0.00410672487967\\
357	0.00411009400125772\\
358	0.00411352010226292\\
359	0.00411700396460171\\
360	0.00412054636708436\\
361	0.00412414808393454\\
362	0.004127809883143\\
363	0.00413153252463749\\
364	0.0041353167582489\\
365	0.00413916332145065\\
366	0.0041430729368464\\
367	0.00414704630937751\\
368	0.0041510841232191\\
369	0.00415518703832901\\
370	0.00415935568661078\\
371	0.00416359066764624\\
372	0.00416789254394891\\
373	0.00417226183568325\\
374	0.00417669901478839\\
375	0.00418120449843865\\
376	0.00418577864176438\\
377	0.00419042172974946\\
378	0.00419513396821128\\
379	0.00419991547376057\\
380	0.00420476626262653\\
381	0.00420968623822302\\
382	0.00421467517731951\\
383	0.00421973271466958\\
384	0.00422485832593976\\
385	0.0042300513087723\\
386	0.0042353107618099\\
387	0.00424063556150867\\
388	0.00424602433657202\\
389	0.00425147543985365\\
390	0.00425698691761027\\
391	0.00426255647603806\\
392	0.00426818144511206\\
393	0.00427385873987608\\
394	0.00427958481952026\\
395	0.00428535564485324\\
396	0.00429116663515453\\
397	0.00429701262590292\\
398	0.00430288782952713\\
399	0.00430878580264344\\
400	0.00431469942497536\\
401	0.00432062089684156\\
402	0.00432654176479418\\
403	0.00433245298898998\\
404	0.00433834507115593\\
405	0.00434420826923577\\
406	0.00435003293470858\\
407	0.00435581002220494\\
408	0.00436153184002012\\
409	0.00436719313706548\\
410	0.00437279266132182\\
411	0.00437833538472884\\
412	0.00438383567545592\\
413	0.00438932177878079\\
414	0.00439484173526621\\
415	0.00440043693996867\\
416	0.00440611959258895\\
417	0.00441189048639354\\
418	0.00441775037617439\\
419	0.00442369997286494\\
420	0.00442973993741127\\
421	0.00443587087335271\\
422	0.00444209331871319\\
423	0.00444840774111909\\
424	0.00445481453771689\\
425	0.0044613140414695\\
426	0.00446790651697064\\
427	0.0044745921563337\\
428	0.00448137107521975\\
429	0.0044882433090829\\
430	0.00449520880974954\\
431	0.00450226744247634\\
432	0.00450941898366786\\
433	0.00451666311947818\\
434	0.00452399944557325\\
435	0.00453142746838399\\
436	0.00453894660824531\\
437	0.00454655620490727\\
438	0.00455425552600115\\
439	0.00456204377916066\\
440	0.00456992012862854\\
441	0.00457788371733072\\
442	0.00458593369556622\\
443	0.00459406925764097\\
444	0.00460228968796137\\
445	0.00461059441828207\\
446	0.00461898309795326\\
447	0.00462745567909243\\
448	0.00463601251855382\\
449	0.00464465449905457\\
450	0.00465338317019809\\
451	0.00466220090898406\\
452	0.00467111109790705\\
453	0.00468011831560007\\
454	0.00468922852994733\\
455	0.00469844927614577\\
456	0.00470778979203129\\
457	0.0047172610686496\\
458	0.00472687575038563\\
459	0.00473664778038361\\
460	0.00474659161162588\\
461	0.00475672065104738\\
462	0.00476704443458528\\
463	0.00477756856208024\\
464	0.00478829864180216\\
465	0.00479924063984959\\
466	0.00481040087052541\\
467	0.00482178602364282\\
468	0.00483340319515331\\
469	0.00484525992006689\\
470	0.00485736420771812\\
471	0.004869724579392\\
472	0.00488235010823822\\
473	0.00489525046131628\\
474	0.00490843594349121\\
475	0.00492191754271023\\
476	0.00493570697592564\\
477	0.00494981673455072\\
478	0.00496426012906006\\
479	0.00497905133959719\\
480	0.00499420641614639\\
481	0.00500974301389922\\
482	0.00502567987587271\\
483	0.00504203681804972\\
484	0.00505883469393796\\
485	0.00507609533954526\\
486	0.00509384150344459\\
487	0.0051120967723867\\
488	0.00513088551504203\\
489	0.00515023291574502\\
490	0.00517016521560303\\
491	0.00519070995746875\\
492	0.00521189574655456\\
493	0.00523370568230926\\
494	0.00525615797345135\\
495	0.00527928085048286\\
496	0.00530310526724121\\
497	0.00532766854668698\\
498	0.00535303735342734\\
499	0.00537925241866236\\
500	0.0054063556515977\\
501	0.00543439127309023\\
502	0.00546340557950195\\
503	0.00549344603574776\\
504	0.00552456889504518\\
505	0.00555688280214329\\
506	0.00559044812484436\\
507	0.00562532780659876\\
508	0.00566158726887005\\
509	0.00569929423683336\\
510	0.00573851846062438\\
511	0.00577933131156622\\
512	0.00582180523855337\\
513	0.0058660130791357\\
514	0.00591202733005102\\
515	0.00595991936941512\\
516	0.00600975518857396\\
517	0.00606159220849871\\
518	0.00611547879735186\\
519	0.00617145111727593\\
520	0.00622952827222556\\
521	0.00628970070576879\\
522	0.00635192430572301\\
523	0.00641612600430055\\
524	0.00648218785696528\\
525	0.00654993917002101\\
526	0.0066191621183003\\
527	0.00668963241587316\\
528	0.00676094353650896\\
529	0.00683211407028415\\
530	0.00690211112710542\\
531	0.00696671445942419\\
532	0.00702546965301922\\
533	0.00707812424877317\\
534	0.00712481587446659\\
535	0.00716849059922027\\
536	0.00721077696089307\\
537	0.00725210040069124\\
538	0.00729298041879735\\
539	0.00733394341899069\\
540	0.00737527202531301\\
541	0.00741711243910963\\
542	0.00745958050016931\\
543	0.00750275904121561\\
544	0.00754670141587454\\
545	0.00759144767135124\\
546	0.00763703054234583\\
547	0.00768347622976199\\
548	0.00773080823398957\\
549	0.00777909452938988\\
550	0.00782841534682607\\
551	0.00787862213957222\\
552	0.00792795218358609\\
553	0.00797616861463886\\
554	0.00802315876012582\\
555	0.00807047849253736\\
556	0.00811816536939713\\
557	0.0081662917589018\\
558	0.00821491588109375\\
559	0.00826402041714991\\
560	0.00831357679383578\\
561	0.00836354672564337\\
562	0.00841388288403202\\
563	0.00846455795573843\\
564	0.00851554903587025\\
565	0.00856510622084549\\
566	0.00861296704327435\\
567	0.00866079323060213\\
568	0.0087086623118578\\
569	0.00875662026486645\\
570	0.00880463250812337\\
571	0.0088526615516325\\
572	0.00890067654903162\\
573	0.00894813872224893\\
574	0.00899535238494104\\
575	0.00904255767963824\\
576	0.00908973840022921\\
577	0.00913684942824384\\
578	0.00918384194182773\\
579	0.00923066499370698\\
580	0.00927726597097645\\
581	0.00932359115751704\\
582	0.00936958649189013\\
583	0.00941519857373444\\
584	0.00946037597937255\\
585	0.00950507095774694\\
586	0.00954924158549703\\
587	0.00959285445497603\\
588	0.00963588792409171\\
589	0.00967833579817709\\
590	0.00972021084271539\\
591	0.00976138933978722\\
592	0.00980171928775089\\
593	0.00984102135019415\\
594	0.00987905595166784\\
595	0.00991543381058343\\
596	0.00994937493726701\\
597	0.00997906286423442\\
598	0.0099999191923403\\
599	0\\
600	0\\
};
\addplot [color=red,solid,forget plot]
  table[row sep=crcr]{%
1	0.00417141729594836\\
2	0.0041714212375612\\
3	0.00417142525044982\\
4	0.00417142933590415\\
5	0.00417143349523751\\
6	0.00417143772978697\\
7	0.00417144204091389\\
8	0.00417144643000426\\
9	0.00417145089846921\\
10	0.00417145544774538\\
11	0.00417146007929553\\
12	0.00417146479460882\\
13	0.00417146959520149\\
14	0.00417147448261719\\
15	0.00417147945842758\\
16	0.00417148452423281\\
17	0.004171489681662\\
18	0.00417149493237377\\
19	0.00417150027805687\\
20	0.0041715057204306\\
21	0.00417151126124547\\
22	0.00417151690228366\\
23	0.00417152264535971\\
24	0.00417152849232102\\
25	0.00417153444504848\\
26	0.00417154050545709\\
27	0.00417154667549652\\
28	0.0041715529571518\\
29	0.00417155935244397\\
30	0.00417156586343074\\
31	0.00417157249220702\\
32	0.00417157924090577\\
33	0.00417158611169864\\
34	0.00417159310679659\\
35	0.0041716002284507\\
36	0.0041716074789529\\
37	0.00417161486063662\\
38	0.00417162237587761\\
39	0.00417163002709472\\
40	0.00417163781675066\\
41	0.00417164574735278\\
42	0.00417165382145392\\
43	0.00417166204165321\\
44	0.00417167041059691\\
45	0.0041716789309793\\
46	0.00417168760554341\\
47	0.00417169643708219\\
48	0.00417170542843912\\
49	0.00417171458250933\\
50	0.00417172390224041\\
51	0.00417173339063343\\
52	0.00417174305074389\\
53	0.00417175288568275\\
54	0.00417176289861736\\
55	0.00417177309277255\\
56	0.00417178347143162\\
57	0.00417179403793749\\
58	0.00417180479569368\\
59	0.00417181574816551\\
60	0.00417182689888108\\
61	0.0041718382514326\\
62	0.00417184980947735\\
63	0.00417186157673903\\
64	0.00417187355700894\\
65	0.00417188575414711\\
66	0.00417189817208365\\
67	0.00417191081481996\\
68	0.00417192368643009\\
69	0.00417193679106193\\
70	0.00417195013293873\\
71	0.00417196371636035\\
72	0.0041719775457047\\
73	0.00417199162542913\\
74	0.00417200596007185\\
75	0.00417202055425349\\
76	0.00417203541267852\\
77	0.0041720505401368\\
78	0.00417206594150512\\
79	0.00417208162174879\\
80	0.00417209758592323\\
81	0.00417211383917564\\
82	0.00417213038674667\\
83	0.00417214723397208\\
84	0.00417216438628448\\
85	0.00417218184921508\\
86	0.00417219962839556\\
87	0.00417221772955976\\
88	0.00417223615854567\\
89	0.00417225492129723\\
90	0.00417227402386632\\
91	0.00417229347241466\\
92	0.00417231327321582\\
93	0.00417233343265732\\
94	0.00417235395724267\\
95	0.00417237485359332\\
96	0.00417239612845109\\
97	0.00417241778868015\\
98	0.00417243984126929\\
99	0.00417246229333417\\
100	0.00417248515211969\\
101	0.00417250842500232\\
102	0.00417253211949238\\
103	0.00417255624323666\\
104	0.00417258080402071\\
105	0.00417260580977149\\
106	0.0041726312685599\\
107	0.00417265718860336\\
108	0.00417268357826855\\
109	0.00417271044607403\\
110	0.00417273780069307\\
111	0.00417276565095637\\
112	0.00417279400585506\\
113	0.00417282287454349\\
114	0.00417285226634222\\
115	0.00417288219074112\\
116	0.00417291265740226\\
117	0.00417294367616329\\
118	0.00417297525704035\\
119	0.00417300741023158\\
120	0.00417304014612021\\
121	0.004173073475278\\
122	0.00417310740846868\\
123	0.00417314195665143\\
124	0.00417317713098439\\
125	0.00417321294282827\\
126	0.00417324940375013\\
127	0.00417328652552689\\
128	0.0041733243201494\\
129	0.00417336279982615\\
130	0.0041734019769873\\
131	0.00417344186428863\\
132	0.00417348247461566\\
133	0.00417352382108783\\
134	0.00417356591706271\\
135	0.00417360877614035\\
136	0.00417365241216765\\
137	0.00417369683924282\\
138	0.00417374207171994\\
139	0.00417378812421355\\
140	0.00417383501160351\\
141	0.00417388274903964\\
142	0.00417393135194666\\
143	0.00417398083602919\\
144	0.00417403121727675\\
145	0.00417408251196902\\
146	0.00417413473668098\\
147	0.00417418790828828\\
148	0.00417424204397271\\
149	0.00417429716122767\\
150	0.00417435327786387\\
151	0.00417441041201495\\
152	0.00417446858214343\\
153	0.0041745278070466\\
154	0.00417458810586256\\
155	0.00417464949807632\\
156	0.00417471200352611\\
157	0.00417477564240979\\
158	0.00417484043529119\\
159	0.00417490640310686\\
160	0.00417497356717269\\
161	0.00417504194919076\\
162	0.00417511157125627\\
163	0.00417518245586462\\
164	0.00417525462591869\\
165	0.00417532810473602\\
166	0.00417540291605636\\
167	0.00417547908404922\\
168	0.0041755566333216\\
169	0.00417563558892583\\
170	0.00417571597636753\\
171	0.00417579782161376\\
172	0.00417588115110138\\
173	0.00417596599174526\\
174	0.00417605237094702\\
175	0.00417614031660364\\
176	0.00417622985711639\\
177	0.00417632102139971\\
178	0.00417641383889057\\
179	0.00417650833955761\\
180	0.00417660455391075\\
181	0.00417670251301075\\
182	0.00417680224847913\\
183	0.004176903792508\\
184	0.00417700717787039\\
185	0.00417711243793043\\
186	0.00417721960665394\\
187	0.00417732871861905\\
188	0.00417743980902709\\
189	0.00417755291371368\\
190	0.00417766806915986\\
191	0.0041777853125036\\
192	0.00417790468155139\\
193	0.00417802621479001\\
194	0.00417814995139857\\
195	0.00417827593126074\\
196	0.00417840419497708\\
197	0.00417853478387773\\
198	0.00417866774003521\\
199	0.00417880310627744\\
200	0.00417894092620098\\
201	0.00417908124418459\\
202	0.00417922410540279\\
203	0.00417936955583983\\
204	0.00417951764230381\\
205	0.00417966841244109\\
206	0.00417982191475086\\
207	0.00417997819859995\\
208	0.00418013731423793\\
209	0.00418029931281241\\
210	0.00418046424638457\\
211	0.00418063216794502\\
212	0.00418080313142981\\
213	0.00418097719173671\\
214	0.00418115440474184\\
215	0.00418133482731642\\
216	0.00418151851734385\\
217	0.00418170553373708\\
218	0.00418189593645622\\
219	0.00418208978652635\\
220	0.00418228714605576\\
221	0.0041824880782543\\
222	0.0041826926474521\\
223	0.00418290091911855\\
224	0.00418311295988156\\
225	0.00418332883754708\\
226	0.00418354862111898\\
227	0.00418377238081907\\
228	0.00418400018810767\\
229	0.00418423211570418\\
230	0.00418446823760813\\
231	0.00418470862912047\\
232	0.00418495336686518\\
233	0.0041852025288112\\
234	0.00418545619429455\\
235	0.00418571444404094\\
236	0.00418597736018854\\
237	0.00418624502631107\\
238	0.00418651752744131\\
239	0.00418679495009489\\
240	0.00418707738229421\\
241	0.0041873649135929\\
242	0.00418765763510055\\
243	0.00418795563950764\\
244	0.00418825902111096\\
245	0.00418856787583913\\
246	0.00418888230127873\\
247	0.00418920239670044\\
248	0.0041895282630857\\
249	0.0041898600031536\\
250	0.00419019772138816\\
251	0.00419054152406586\\
252	0.0041908915192835\\
253	0.00419124781698645\\
254	0.00419161052899709\\
255	0.00419197976904371\\
256	0.00419235565278956\\
257	0.00419273829786239\\
258	0.00419312782388419\\
259	0.00419352435250128\\
260	0.00419392800741462\\
261	0.00419433891441072\\
262	0.00419475720139237\\
263	0.00419518299841011\\
264	0.00419561643769389\\
265	0.00419605765368487\\
266	0.00419650678306768\\
267	0.00419696396480297\\
268	0.00419742934016015\\
269	0.00419790305275047\\
270	0.0041983852485605\\
271	0.00419887607598572\\
272	0.00419937568586451\\
273	0.00419988423151231\\
274	0.00420040186875631\\
275	0.00420092875596998\\
276	0.00420146505410838\\
277	0.00420201092674335\\
278	0.00420256654009912\\
279	0.00420313206308821\\
280	0.00420370766734756\\
281	0.00420429352727493\\
282	0.00420488982006546\\
283	0.00420549672574871\\
284	0.00420611442722573\\
285	0.00420674311030644\\
286	0.00420738296374742\\
287	0.00420803417928973\\
288	0.00420869695169706\\
289	0.00420937147879415\\
290	0.00421005796150549\\
291	0.00421075660389411\\
292	0.00421146761320083\\
293	0.00421219119988342\\
294	0.00421292757765646\\
295	0.00421367696353093\\
296	0.00421443957785442\\
297	0.00421521564435124\\
298	0.00421600539016304\\
299	0.00421680904588942\\
300	0.00421762684562882\\
301	0.00421845902701961\\
302	0.00421930583128124\\
303	0.00422016750325577\\
304	0.0042210442914492\\
305	0.00422193644807332\\
306	0.00422284422908731\\
307	0.00422376789423958\\
308	0.00422470770710954\\
309	0.00422566393514946\\
310	0.00422663684972616\\
311	0.00422762672616268\\
312	0.00422863384377971\\
313	0.00422965848593693\\
314	0.0042307009400738\\
315	0.00423176149775028\\
316	0.00423284045468683\\
317	0.00423393811080408\\
318	0.00423505477026159\\
319	0.00423619074149609\\
320	0.00423734633725874\\
321	0.0042385218746514\\
322	0.00423971767516177\\
323	0.0042409340646973\\
324	0.00424217137361771\\
325	0.00424342993676598\\
326	0.00424471009349763\\
327	0.00424601218770806\\
328	0.00424733656785803\\
329	0.00424868358699681\\
330	0.00425005360278272\\
331	0.0042514469775011\\
332	0.00425286407807892\\
333	0.00425430527609568\\
334	0.00425577094779001\\
335	0.00425726147406045\\
336	0.0042587772404593\\
337	0.00426031863717721\\
338	0.00426188605901504\\
339	0.0042634799053389\\
340	0.00426510058001185\\
341	0.00426674849129432\\
342	0.00426842405170062\\
343	0.00427012767778342\\
344	0.00427185978975912\\
345	0.00427362081076799\\
346	0.00427541116576958\\
347	0.00427723128085833\\
348	0.00427908158246891\\
349	0.00428096249646347\\
350	0.0042828744470972\\
351	0.00428481785586465\\
352	0.00428679314023817\\
353	0.00428880071232879\\
354	0.00429084097756436\\
355	0.00429291433371471\\
356	0.0042950211712544\\
357	0.00429716187620316\\
358	0.00429933682967659\\
359	0.00430154640713625\\
360	0.00430379097758998\\
361	0.00430607090273958\\
362	0.00430838653607358\\
363	0.00431073822190244\\
364	0.00431312629433371\\
365	0.00431555107618505\\
366	0.00431801287783277\\
367	0.0043205119959941\\
368	0.00432304871244172\\
369	0.00432562329264953\\
370	0.00432823598436941\\
371	0.00433088701613951\\
372	0.00433357659572557\\
373	0.00433630490849874\\
374	0.00433907211575428\\
375	0.00434187835297895\\
376	0.0043447237280762\\
377	0.00434760831956328\\
378	0.0043505321747576\\
379	0.00435349530797551\\
380	0.00435649769877283\\
381	0.00435953929026478\\
382	0.00436261998757205\\
383	0.00436573965645217\\
384	0.00436889812218877\\
385	0.00437209516882886\\
386	0.00437533053887853\\
387	0.00437860393359149\\
388	0.00438191501401419\\
389	0.00438526340298479\\
390	0.0043886486883227\\
391	0.00439207042749119\\
392	0.00439552815406753\\
393	0.0043990213864128\\
394	0.00440254963899731\\
395	0.00440611243690426\\
396	0.00440970933410266\\
397	0.00441333993614392\\
398	0.00441700392799374\\
399	0.00442070110772882\\
400	0.00442443142677545\\
401	0.00442819503726806\\
402	0.00443199234687599\\
403	0.00443582408101143\\
404	0.00443969135159781\\
405	0.00444359573040631\\
406	0.00444753932316088\\
407	0.0044515248379043\\
408	0.00445555563713975\\
409	0.00445963575746238\\
410	0.0044637698717059\\
411	0.00446796315433237\\
412	0.00447222098358322\\
413	0.00447654835910707\\
414	0.00448094884530292\\
415	0.00448542420369106\\
416	0.00448997560015105\\
417	0.00449460422115946\\
418	0.00449931127504153\\
419	0.00450409799335963\\
420	0.00450896563245742\\
421	0.00451391547521329\\
422	0.00451894883305504\\
423	0.00452406704818085\\
424	0.00452927149581646\\
425	0.00453456358633038\\
426	0.00453994476757048\\
427	0.00454541652748043\\
428	0.00455098039742705\\
429	0.0045566379564885\\
430	0.00456239083633436\\
431	0.00456824072677151\\
432	0.00457418938203479\\
433	0.00458023862790652\\
434	0.00458639036975255\\
435	0.00459264660156371\\
436	0.00459900941609112\\
437	0.00460548101615734\\
438	0.00461206372721509\\
439	0.00461876001120466\\
440	0.00462557248173187\\
441	0.00463250392054477\\
442	0.00463955729522559\\
443	0.00464673577793208\\
444	0.00465404276491351\\
445	0.00466148189639597\\
446	0.0046690570762898\\
447	0.00467677249104362\\
448	0.00468463262666304\\
449	0.00469264226005057\\
450	0.00470080645466231\\
451	0.00470913056139528\\
452	0.00471762021278094\\
453	0.00472628130882881\\
454	0.00473511999293577\\
455	0.00474414261666887\\
456	0.00475335569311759\\
457	0.00476276583994915\\
458	0.00477237971546353\\
459	0.00478220395394506\\
460	0.00479224511277774\\
461	0.00480250966573844\\
462	0.0048130041410784\\
463	0.00482373531637173\\
464	0.00483471025564672\\
465	0.00484593634938176\\
466	0.00485742262810758\\
467	0.00486917868065496\\
468	0.0048812146512215\\
469	0.0048935412725203\\
470	0.00490616989892637\\
471	0.00491911253884136\\
472	0.00493238188599012\\
473	0.00494599134833053\\
474	0.00495995507289613\\
475	0.00497428796445237\\
476	0.00498900569361151\\
477	0.00500412468545673\\
478	0.00501966204991506\\
479	0.00503563534260202\\
480	0.00505203203850175\\
481	0.00506885035320858\\
482	0.00508610647732088\\
483	0.00510381737352443\\
484	0.00512200081946964\\
485	0.00514067545717838\\
486	0.00515986085050313\\
487	0.0051795775543532\\
488	0.0051998472524993\\
489	0.00522069291948455\\
490	0.00524213893576288\\
491	0.00526421036697117\\
492	0.0052869337710113\\
493	0.00531038708338049\\
494	0.00533463065977775\\
495	0.00535972170666477\\
496	0.00538570357433848\\
497	0.00541262217930969\\
498	0.00544052593988315\\
499	0.00546946488375426\\
500	0.0054994915755224\\
501	0.0055306611806234\\
502	0.00556303151660603\\
503	0.00559666327700001\\
504	0.00563162031738346\\
505	0.00566796721508761\\
506	0.00570576840772046\\
507	0.00574508942737241\\
508	0.00578599628828108\\
509	0.00582855467622107\\
510	0.00587282886678455\\
511	0.00591888013540978\\
512	0.00596676496370957\\
513	0.0060165327491921\\
514	0.00606822295732727\\
515	0.00612186193541779\\
516	0.00617745923946957\\
517	0.00623501086570649\\
518	0.0062944782534695\\
519	0.00635577778146541\\
520	0.0064187788043303\\
521	0.00648329513000825\\
522	0.00654909551745368\\
523	0.00661556809570386\\
524	0.00668225778840485\\
525	0.00674845423256209\\
526	0.00681265527495289\\
527	0.00687114274886836\\
528	0.00692366259412663\\
529	0.00697041755680642\\
530	0.00701168306378662\\
531	0.00705128720921838\\
532	0.00708973073153022\\
533	0.00712748667855347\\
534	0.00716510576662421\\
535	0.00720298425452938\\
536	0.00724131062521522\\
537	0.00728020567755491\\
538	0.0073197621304377\\
539	0.00736003760201659\\
540	0.00740107026007002\\
541	0.00744289189206673\\
542	0.00748552922793119\\
543	0.00752900542539635\\
544	0.00757334165858363\\
545	0.00761855794325576\\
546	0.0076647133488703\\
547	0.0077118861491722\\
548	0.00776011644032672\\
549	0.00780747688911366\\
550	0.00785373177371214\\
551	0.00789882269053892\\
552	0.00794427584846579\\
553	0.00799013221423805\\
554	0.00803647165748248\\
555	0.00808335531207435\\
556	0.00813077466079379\\
557	0.0081787090079221\\
558	0.00822712878473145\\
559	0.00827599893382302\\
560	0.00832527752852308\\
561	0.00837493010777871\\
562	0.00842494111645729\\
563	0.00847397024240702\\
564	0.00852129441202751\\
565	0.00856849529392683\\
566	0.00861580505555088\\
567	0.00866327323530527\\
568	0.00871086822438768\\
569	0.00875855220172137\\
570	0.0088062920664252\\
571	0.00885392879390287\\
572	0.00890103813867015\\
573	0.00894817183198183\\
574	0.00899535443329664\\
575	0.00904255798659048\\
576	0.00908973853069441\\
577	0.00913684949557439\\
578	0.0091838419761039\\
579	0.009230665010298\\
580	0.00927726597847558\\
581	0.00932359116063004\\
582	0.00936958649305349\\
583	0.00941519857411533\\
584	0.00946037597947751\\
585	0.00950507095776978\\
586	0.00954924158550052\\
587	0.00959285445497631\\
588	0.00963588792409171\\
589	0.00967833579817709\\
590	0.00972021084271539\\
591	0.00976138933978722\\
592	0.00980171928775089\\
593	0.00984102135019415\\
594	0.00987905595166784\\
595	0.00991543381058343\\
596	0.00994937493726701\\
597	0.00997906286423442\\
598	0.0099999191923403\\
599	0\\
600	0\\
};
\addplot [color=mycolor20,solid,forget plot]
  table[row sep=crcr]{%
1	0.00426663331528449\\
2	0.00426663605446198\\
3	0.00426663884350637\\
4	0.00426664168332735\\
5	0.00426664457485123\\
6	0.0042666475190213\\
7	0.00426665051679803\\
8	0.00426665356915952\\
9	0.00426665667710172\\
10	0.00426665984163881\\
11	0.00426666306380356\\
12	0.00426666634464759\\
13	0.00426666968524176\\
14	0.00426667308667659\\
15	0.00426667655006247\\
16	0.00426668007653015\\
17	0.00426668366723108\\
18	0.0042666873233378\\
19	0.00426669104604424\\
20	0.00426669483656624\\
21	0.00426669869614191\\
22	0.004266702626032\\
23	0.00426670662752032\\
24	0.0042667107019142\\
25	0.00426671485054495\\
26	0.00426671907476817\\
27	0.00426672337596439\\
28	0.00426672775553932\\
29	0.00426673221492451\\
30	0.00426673675557764\\
31	0.00426674137898315\\
32	0.00426674608665263\\
33	0.00426675088012538\\
34	0.00426675576096894\\
35	0.0042667607307795\\
36	0.00426676579118251\\
37	0.00426677094383328\\
38	0.00426677619041735\\
39	0.00426678153265121\\
40	0.00426678697228281\\
41	0.00426679251109208\\
42	0.00426679815089161\\
43	0.00426680389352723\\
44	0.0042668097408786\\
45	0.00426681569485981\\
46	0.00426682175742013\\
47	0.00426682793054448\\
48	0.00426683421625422\\
49	0.00426684061660772\\
50	0.00426684713370119\\
51	0.00426685376966924\\
52	0.00426686052668561\\
53	0.00426686740696394\\
54	0.00426687441275847\\
55	0.0042668815463648\\
56	0.00426688881012064\\
57	0.00426689620640662\\
58	0.004266903737647\\
59	0.00426691140631057\\
60	0.0042669192149114\\
61	0.00426692716600971\\
62	0.00426693526221271\\
63	0.00426694350617553\\
64	0.00426695190060191\\
65	0.00426696044824531\\
66	0.00426696915190976\\
67	0.00426697801445072\\
68	0.00426698703877609\\
69	0.00426699622784719\\
70	0.00426700558467973\\
71	0.00426701511234478\\
72	0.00426702481396984\\
73	0.00426703469273984\\
74	0.00426704475189825\\
75	0.00426705499474805\\
76	0.00426706542465298\\
77	0.00426707604503858\\
78	0.0042670868593933\\
79	0.0042670978712697\\
80	0.00426710908428565\\
81	0.00426712050212553\\
82	0.00426713212854137\\
83	0.00426714396735423\\
84	0.00426715602245535\\
85	0.00426716829780761\\
86	0.00426718079744667\\
87	0.00426719352548243\\
88	0.00426720648610035\\
89	0.00426721968356286\\
90	0.00426723312221078\\
91	0.00426724680646477\\
92	0.00426726074082681\\
93	0.00426727492988168\\
94	0.0042672893782985\\
95	0.00426730409083235\\
96	0.0042673190723257\\
97	0.00426733432771018\\
98	0.0042673498620082\\
99	0.00426736568033454\\
100	0.00426738178789813\\
101	0.00426739819000379\\
102	0.00426741489205398\\
103	0.00426743189955058\\
104	0.0042674492180968\\
105	0.00426746685339901\\
106	0.00426748481126861\\
107	0.00426750309762402\\
108	0.00426752171849266\\
109	0.00426754068001293\\
110	0.00426755998843633\\
111	0.00426757965012944\\
112	0.00426759967157611\\
113	0.00426762005937972\\
114	0.00426764082026519\\
115	0.00426766196108137\\
116	0.00426768348880335\\
117	0.00426770541053467\\
118	0.00426772773350981\\
119	0.00426775046509656\\
120	0.00426777361279847\\
121	0.0042677971842574\\
122	0.00426782118725611\\
123	0.00426784562972072\\
124	0.00426787051972356\\
125	0.00426789586548568\\
126	0.00426792167537974\\
127	0.00426794795793283\\
128	0.00426797472182915\\
129	0.00426800197591313\\
130	0.00426802972919219\\
131	0.00426805799083993\\
132	0.00426808677019914\\
133	0.00426811607678491\\
134	0.00426814592028779\\
135	0.00426817631057712\\
136	0.00426820725770426\\
137	0.00426823877190602\\
138	0.004268270863608\\
139	0.00426830354342821\\
140	0.0042683368221805\\
141	0.00426837071087827\\
142	0.00426840522073809\\
143	0.00426844036318351\\
144	0.00426847614984887\\
145	0.00426851259258323\\
146	0.00426854970345422\\
147	0.00426858749475229\\
148	0.00426862597899463\\
149	0.00426866516892941\\
150	0.00426870507754012\\
151	0.00426874571804986\\
152	0.00426878710392581\\
153	0.00426882924888365\\
154	0.00426887216689225\\
155	0.00426891587217831\\
156	0.00426896037923111\\
157	0.00426900570280735\\
158	0.00426905185793613\\
159	0.00426909885992393\\
160	0.00426914672435976\\
161	0.00426919546712036\\
162	0.00426924510437553\\
163	0.00426929565259355\\
164	0.00426934712854658\\
165	0.0042693995493164\\
166	0.00426945293230002\\
167	0.00426950729521554\\
168	0.00426956265610805\\
169	0.00426961903335571\\
170	0.00426967644567576\\
171	0.00426973491213084\\
172	0.00426979445213538\\
173	0.00426985508546202\\
174	0.00426991683224824\\
175	0.00426997971300302\\
176	0.00427004374861366\\
177	0.00427010896035288\\
178	0.00427017536988565\\
179	0.00427024299927665\\
180	0.00427031187099741\\
181	0.00427038200793391\\
182	0.00427045343339406\\
183	0.00427052617111558\\
184	0.00427060024527374\\
185	0.00427067568048948\\
186	0.00427075250183746\\
187	0.00427083073485445\\
188	0.0042709104055478\\
189	0.00427099154040393\\
190	0.00427107416639718\\
191	0.00427115831099867\\
192	0.00427124400218538\\
193	0.00427133126844935\\
194	0.00427142013880714\\
195	0.00427151064280929\\
196	0.00427160281055008\\
197	0.0042716966726774\\
198	0.00427179226040283\\
199	0.00427188960551184\\
200	0.00427198874037426\\
201	0.00427208969795479\\
202	0.00427219251182382\\
203	0.00427229721616838\\
204	0.00427240384580332\\
205	0.00427251243618258\\
206	0.00427262302341075\\
207	0.0042727356442548\\
208	0.004272850336156\\
209	0.00427296713724205\\
210	0.00427308608633941\\
211	0.00427320722298579\\
212	0.00427333058744295\\
213	0.00427345622070965\\
214	0.00427358416453478\\
215	0.00427371446143077\\
216	0.00427384715468722\\
217	0.00427398228838469\\
218	0.0042741199074088\\
219	0.00427426005746445\\
220	0.0042744027850904\\
221	0.00427454813767394\\
222	0.00427469616346598\\
223	0.00427484691159612\\
224	0.00427500043208823\\
225	0.00427515677587616\\
226	0.00427531599481954\\
227	0.00427547814172013\\
228	0.00427564327033817\\
229	0.00427581143540913\\
230	0.00427598269266059\\
231	0.00427615709882952\\
232	0.00427633471167965\\
233	0.00427651559001924\\
234	0.00427669979371907\\
235	0.00427688738373068\\
236	0.00427707842210479\\
237	0.00427727297201019\\
238	0.00427747109775273\\
239	0.00427767286479459\\
240	0.00427787833977384\\
241	0.00427808759052437\\
242	0.00427830068609593\\
243	0.00427851769677457\\
244	0.00427873869410317\\
245	0.00427896375090256\\
246	0.00427919294129249\\
247	0.00427942634071331\\
248	0.00427966402594756\\
249	0.00427990607514204\\
250	0.00428015256783011\\
251	0.00428040358495417\\
252	0.00428065920888856\\
253	0.00428091952346254\\
254	0.00428118461398376\\
255	0.00428145456726173\\
256	0.00428172947163182\\
257	0.00428200941697935\\
258	0.00428229449476391\\
259	0.00428258479804408\\
260	0.00428288042150233\\
261	0.00428318146147007\\
262	0.00428348801595315\\
263	0.00428380018465741\\
264	0.00428411806901457\\
265	0.00428444177220832\\
266	0.00428477139920069\\
267	0.00428510705675852\\
268	0.00428544885348034\\
269	0.00428579689982334\\
270	0.00428615130813052\\
271	0.00428651219265822\\
272	0.00428687966960365\\
273	0.00428725385713282\\
274	0.00428763487540848\\
275	0.00428802284661846\\
276	0.00428841789500398\\
277	0.00428882014688836\\
278	0.00428922973070582\\
279	0.00428964677703041\\
280	0.00429007141860525\\
281	0.00429050379037187\\
282	0.00429094402949989\\
283	0.00429139227541663\\
284	0.00429184866983709\\
285	0.00429231335679421\\
286	0.0042927864826692\\
287	0.00429326819622204\\
288	0.00429375864862242\\
289	0.00429425799348073\\
290	0.00429476638687946\\
291	0.00429528398740469\\
292	0.00429581095617808\\
293	0.00429634745688905\\
294	0.00429689365582736\\
295	0.00429744972191603\\
296	0.00429801582674476\\
297	0.00429859214460362\\
298	0.00429917885251751\\
299	0.00429977613028076\\
300	0.00430038416049273\\
301	0.00430100312859371\\
302	0.00430163322290172\\
303	0.00430227463464979\\
304	0.00430292755802435\\
305	0.00430359219020419\\
306	0.00430426873140046\\
307	0.0043049573848976\\
308	0.00430565835709522\\
309	0.00430637185755113\\
310	0.00430709809902548\\
311	0.00430783729752581\\
312	0.0043085896723536\\
313	0.00430935544615174\\
314	0.00431013484495313\\
315	0.00431092809823047\\
316	0.00431173543894718\\
317	0.00431255710360897\\
318	0.00431339333231663\\
319	0.00431424436881902\\
320	0.00431511046056672\\
321	0.00431599185876574\\
322	0.00431688881843094\\
323	0.00431780159843884\\
324	0.00431873046157911\\
325	0.00431967567460451\\
326	0.00432063750827823\\
327	0.0043216162374183\\
328	0.00432261214093769\\
329	0.00432362550187977\\
330	0.00432465660744775\\
331	0.00432570574902713\\
332	0.00432677322220024\\
333	0.00432785932675161\\
334	0.00432896436666306\\
335	0.00433008865009771\\
336	0.00433123248937182\\
337	0.00433239620091378\\
338	0.0043335801052101\\
339	0.00433478452673791\\
340	0.00433600979388494\\
341	0.00433725623885781\\
342	0.00433852419758035\\
343	0.00433981400958369\\
344	0.00434112601789247\\
345	0.00434246056892186\\
346	0.00434381801239764\\
347	0.00434519870128648\\
348	0.00434660299173398\\
349	0.00434803124301478\\
350	0.00434948381749918\\
351	0.00435096108064044\\
352	0.00435246340098642\\
353	0.00435399115021984\\
354	0.00435554470323253\\
355	0.00435712443823616\\
356	0.00435873073685524\\
357	0.00436036398412274\\
358	0.00436202456847492\\
359	0.00436371288175468\\
360	0.00436542931922488\\
361	0.00436717427959442\\
362	0.00436894816505918\\
363	0.00437075138136093\\
364	0.00437258433786724\\
365	0.00437444744767627\\
366	0.00437634112775016\\
367	0.00437826579908203\\
368	0.00438022188690141\\
369	0.00438220982092401\\
370	0.00438423003565191\\
371	0.00438628297073149\\
372	0.00438836907137671\\
373	0.00439048878886601\\
374	0.00439264258112255\\
375	0.00439483091338743\\
376	0.0043970542589972\\
377	0.00439931310027703\\
378	0.00440160792956237\\
379	0.00440393925036222\\
380	0.00440630757867798\\
381	0.00440871344449215\\
382	0.00441115739344192\\
383	0.00441363998869198\\
384	0.00441616181302129\\
385	0.00441872347113729\\
386	0.00442132559222974\\
387	0.00442396883277432\\
388	0.00442665387959276\\
389	0.00442938145317225\\
390	0.00443215231124075\\
391	0.00443496725258767\\
392	0.00443782712110927\\
393	0.0044407328100463\\
394	0.00444368526636549\\
395	0.00444668549521858\\
396	0.00444973456438823\\
397	0.00445283360860144\\
398	0.00445598383354885\\
399	0.00445918651955122\\
400	0.00446244302487293\\
401	0.0044657547879608\\
402	0.00446912332816642\\
403	0.00447255024442683\\
404	0.00447603721130479\\
405	0.00447958597174322\\
406	0.00448319832590475\\
407	0.004486876115589\\
408	0.0044906212040159\\
409	0.00449443545129221\\
410	0.00449832068667244\\
411	0.00450227867983526\\
412	0.00450631111553174\\
413	0.00451041958314296\\
414	0.00451460561420282\\
415	0.00451887075326674\\
416	0.00452321658618162\\
417	0.0045276447430362\\
418	0.0045321569014872\\
419	0.00453675479048509\\
420	0.00454144019441523\\
421	0.00454621495765489\\
422	0.00455108098952513\\
423	0.00455604026958924\\
424	0.00456109485322701\\
425	0.00456624687739753\\
426	0.00457149856649096\\
427	0.00457685223810688\\
428	0.00458231029759972\\
429	0.004587875226996\\
430	0.0045935495869901\\
431	0.00459933601879077\\
432	0.0046052372457871\\
433	0.00461125607500289\\
434	0.00461739539831013\\
435	0.00462365819337468\\
436	0.00463004752431581\\
437	0.00463656654207527\\
438	0.00464321848450911\\
439	0.0046500066762436\\
440	0.00465693452837336\\
441	0.00466400553812897\\
442	0.00467122328870246\\
443	0.00467859144948759\\
444	0.00468611377705478\\
445	0.00469379411721445\\
446	0.00470163640858135\\
447	0.00470964468879945\\
448	0.00471782311025046\\
449	0.00472617665871337\\
450	0.00473471089905645\\
451	0.00474343165936944\\
452	0.00475234504327954\\
453	0.00476145744292708\\
454	0.00477077555288365\\
455	0.00478030638543103\\
456	0.0047900572865475\\
457	0.00480003595306472\\
458	0.00481025045074162\\
459	0.00482070923265264\\
460	0.00483142115711013\\
461	0.00484239550330722\\
462	0.00485364197314648\\
463	0.00486517066339799\\
464	0.0048769919772247\\
465	0.00488911634640063\\
466	0.00490151423351223\\
467	0.00491419268996534\\
468	0.00492716026276567\\
469	0.0049404258593231\\
470	0.00495399877491506\\
471	0.00496788872794736\\
472	0.00498210588125886\\
473	0.00499666087416632\\
474	0.00501156486292523\\
475	0.005026829556054\\
476	0.00504246724598849\\
477	0.00505849076745665\\
478	0.00507491360227769\\
479	0.00509174979338502\\
480	0.00510904461151792\\
481	0.00512683513835379\\
482	0.00514514294413699\\
483	0.00516399086837915\\
484	0.00518340310039239\\
485	0.00520340532154351\\
486	0.00522402486784549\\
487	0.00524529086619148\\
488	0.00526723353178775\\
489	0.00528988555479908\\
490	0.00531328551978419\\
491	0.00533749217269366\\
492	0.00536255852004745\\
493	0.00538852951181075\\
494	0.00541545080092527\\
495	0.00544336957331949\\
496	0.00547233467708075\\
497	0.00550239710763532\\
498	0.00553360994472241\\
499	0.00556602827457043\\
500	0.00559970902539875\\
501	0.00563471072458662\\
502	0.00567109315732001\\
503	0.00570891688837819\\
504	0.00574824251891552\\
505	0.00578912967368312\\
506	0.00583163601149896\\
507	0.00587581585992625\\
508	0.0059217185016282\\
509	0.00596938601077104\\
510	0.00601885147197273\\
511	0.00607014279388628\\
512	0.00612327173424099\\
513	0.00617822896877258\\
514	0.00623497651937254\\
515	0.00629343208266863\\
516	0.00635345162141576\\
517	0.00641450018738264\\
518	0.00647630306501683\\
519	0.00653856759307269\\
520	0.00660083177351316\\
521	0.0066623734845899\\
522	0.00672148584669011\\
523	0.00677514287299271\\
524	0.00682314710642925\\
525	0.00686550063474744\\
526	0.00690303785427555\\
527	0.006939183141758\\
528	0.0069743533931855\\
529	0.00700904633276673\\
530	0.00704380336489818\\
531	0.00707891357147736\\
532	0.00711450782547904\\
533	0.00715069050118957\\
534	0.00718753256758812\\
535	0.00722507731585913\\
536	0.00726335660395916\\
537	0.00730239682932776\\
538	0.00734222033833203\\
539	0.00738284789789318\\
540	0.00742430017052845\\
541	0.00746659733277799\\
542	0.00750975936517875\\
543	0.00755383226995456\\
544	0.00759889240011881\\
545	0.00764503248014944\\
546	0.00769063100310164\\
547	0.00773515726105684\\
548	0.00777837421757474\\
549	0.00782197210231666\\
550	0.00786599669861675\\
551	0.00791053055057518\\
552	0.0079556338739196\\
553	0.00800129915646446\\
554	0.00804751442201892\\
555	0.00809426408080562\\
556	0.00814152538695643\\
557	0.00818926789268914\\
558	0.00823745553953724\\
559	0.00828604517572598\\
560	0.00833502635030947\\
561	0.0083836844121862\\
562	0.00843068273194334\\
563	0.00847720045237796\\
564	0.00852387037291765\\
565	0.0085707516393333\\
566	0.00861782349227564\\
567	0.00866505204116376\\
568	0.00871240258683382\\
569	0.00875984726743983\\
570	0.00880711445205011\\
571	0.00885403140777324\\
572	0.00890105211970034\\
573	0.00894817202601561\\
574	0.00899535447720905\\
575	0.00904255800719692\\
576	0.00908973854120521\\
577	0.00913684950076924\\
578	0.00918384197853101\\
579	0.00923066501135355\\
580	0.00927726597889613\\
581	0.00932359116078053\\
582	0.00936958649310055\\
583	0.00941519857412769\\
584	0.00946037597948007\\
585	0.00950507095777015\\
586	0.00954924158550054\\
587	0.00959285445497631\\
588	0.00963588792409171\\
589	0.00967833579817709\\
590	0.00972021084271539\\
591	0.00976138933978722\\
592	0.00980171928775089\\
593	0.00984102135019415\\
594	0.00987905595166784\\
595	0.00991543381058343\\
596	0.00994937493726701\\
597	0.00997906286423442\\
598	0.0099999191923403\\
599	0\\
600	0\\
};
\addplot [color=mycolor21,solid,forget plot]
  table[row sep=crcr]{%
1	0.00430148357374624\\
2	0.00430148576350371\\
3	0.00430148799339185\\
4	0.00430149026414835\\
5	0.00430149257652449\\
6	0.00430149493128541\\
7	0.00430149732921043\\
8	0.00430149977109314\\
9	0.00430150225774187\\
10	0.00430150478997984\\
11	0.00430150736864544\\
12	0.00430150999459258\\
13	0.00430151266869094\\
14	0.00430151539182618\\
15	0.00430151816490036\\
16	0.00430152098883223\\
17	0.00430152386455745\\
18	0.00430152679302899\\
19	0.00430152977521747\\
20	0.00430153281211134\\
21	0.00430153590471736\\
22	0.00430153905406089\\
23	0.00430154226118625\\
24	0.00430154552715704\\
25	0.0043015488530565\\
26	0.00430155223998792\\
27	0.00430155568907497\\
28	0.00430155920146212\\
29	0.00430156277831491\\
30	0.00430156642082051\\
31	0.00430157013018804\\
32	0.00430157390764895\\
33	0.0043015777544575\\
34	0.00430158167189106\\
35	0.00430158566125074\\
36	0.00430158972386165\\
37	0.0043015938610734\\
38	0.00430159807426065\\
39	0.00430160236482345\\
40	0.0043016067341877\\
41	0.00430161118380577\\
42	0.00430161571515693\\
43	0.00430162032974776\\
44	0.00430162502911277\\
45	0.00430162981481488\\
46	0.00430163468844592\\
47	0.00430163965162722\\
48	0.00430164470601016\\
49	0.00430164985327671\\
50	0.00430165509513993\\
51	0.00430166043334467\\
52	0.00430166586966809\\
53	0.00430167140592028\\
54	0.00430167704394488\\
55	0.00430168278561971\\
56	0.00430168863285737\\
57	0.0043016945876059\\
58	0.00430170065184949\\
59	0.00430170682760911\\
60	0.00430171311694324\\
61	0.00430171952194844\\
62	0.00430172604476025\\
63	0.00430173268755374\\
64	0.00430173945254442\\
65	0.00430174634198882\\
66	0.00430175335818536\\
67	0.00430176050347511\\
68	0.00430176778024267\\
69	0.00430177519091678\\
70	0.00430178273797132\\
71	0.00430179042392609\\
72	0.00430179825134768\\
73	0.00430180622285032\\
74	0.0043018143410968\\
75	0.00430182260879937\\
76	0.0043018310287207\\
77	0.0043018396036747\\
78	0.00430184833652761\\
79	0.00430185723019895\\
80	0.00430186628766246\\
81	0.00430187551194713\\
82	0.00430188490613833\\
83	0.00430189447337872\\
84	0.00430190421686942\\
85	0.00430191413987106\\
86	0.00430192424570493\\
87	0.00430193453775406\\
88	0.00430194501946442\\
89	0.00430195569434613\\
90	0.00430196656597458\\
91	0.00430197763799169\\
92	0.00430198891410719\\
93	0.00430200039809985\\
94	0.00430201209381875\\
95	0.00430202400518465\\
96	0.00430203613619134\\
97	0.00430204849090697\\
98	0.0043020610734754\\
99	0.00430207388811781\\
100	0.00430208693913393\\
101	0.00430210023090359\\
102	0.00430211376788829\\
103	0.00430212755463265\\
104	0.00430214159576601\\
105	0.00430215589600398\\
106	0.00430217046015014\\
107	0.00430218529309764\\
108	0.00430220039983083\\
109	0.0043022157854271\\
110	0.00430223145505849\\
111	0.00430224741399359\\
112	0.00430226366759927\\
113	0.0043022802213426\\
114	0.00430229708079265\\
115	0.00430231425162249\\
116	0.00430233173961106\\
117	0.00430234955064526\\
118	0.00430236769072188\\
119	0.00430238616594977\\
120	0.00430240498255189\\
121	0.00430242414686748\\
122	0.00430244366535419\\
123	0.00430246354459042\\
124	0.00430248379127747\\
125	0.00430250441224198\\
126	0.0043025254144382\\
127	0.00430254680495045\\
128	0.00430256859099555\\
129	0.00430259077992528\\
130	0.004302613379229\\
131	0.00430263639653623\\
132	0.00430265983961922\\
133	0.00430268371639573\\
134	0.00430270803493175\\
135	0.00430273280344428\\
136	0.00430275803030421\\
137	0.00430278372403919\\
138	0.00430280989333666\\
139	0.00430283654704676\\
140	0.00430286369418549\\
141	0.00430289134393785\\
142	0.00430291950566099\\
143	0.00430294818888749\\
144	0.00430297740332864\\
145	0.00430300715887788\\
146	0.00430303746561421\\
147	0.00430306833380574\\
148	0.00430309977391316\\
149	0.00430313179659353\\
150	0.00430316441270393\\
151	0.00430319763330525\\
152	0.00430323146966604\\
153	0.00430326593326646\\
154	0.00430330103580227\\
155	0.00430333678918899\\
156	0.00430337320556595\\
157	0.00430341029730062\\
158	0.00430344807699291\\
159	0.00430348655747963\\
160	0.00430352575183885\\
161	0.00430356567339461\\
162	0.00430360633572154\\
163	0.00430364775264963\\
164	0.00430368993826906\\
165	0.00430373290693513\\
166	0.00430377667327339\\
167	0.00430382125218463\\
168	0.00430386665885023\\
169	0.0043039129087374\\
170	0.00430396001760469\\
171	0.00430400800150752\\
172	0.00430405687680379\\
173	0.00430410666015964\\
174	0.0043041573685553\\
175	0.00430420901929114\\
176	0.00430426162999371\\
177	0.00430431521862189\\
178	0.00430436980347337\\
179	0.00430442540319092\\
180	0.00430448203676915\\
181	0.00430453972356102\\
182	0.00430459848328482\\
183	0.00430465833603102\\
184	0.00430471930226947\\
185	0.00430478140285652\\
186	0.00430484465904247\\
187	0.0043049090924791\\
188	0.00430497472522717\\
189	0.00430504157976439\\
190	0.00430510967899332\\
191	0.00430517904624945\\
192	0.00430524970530943\\
193	0.00430532168039956\\
194	0.00430539499620432\\
195	0.00430546967787514\\
196	0.00430554575103932\\
197	0.00430562324180901\\
198	0.00430570217679061\\
199	0.00430578258309413\\
200	0.00430586448834281\\
201	0.00430594792068292\\
202	0.00430603290879372\\
203	0.00430611948189766\\
204	0.00430620766977075\\
205	0.00430629750275308\\
206	0.00430638901175968\\
207	0.00430648222829146\\
208	0.00430657718444632\\
209	0.0043066739129306\\
210	0.00430677244707076\\
211	0.00430687282082509\\
212	0.0043069750687958\\
213	0.00430707922624141\\
214	0.00430718532908909\\
215	0.00430729341394761\\
216	0.00430740351812016\\
217	0.00430751567961771\\
218	0.00430762993717242\\
219	0.00430774633025148\\
220	0.004307864899071\\
221	0.00430798568461041\\
222	0.00430810872862685\\
223	0.00430823407367006\\
224	0.00430836176309745\\
225	0.00430849184108941\\
226	0.00430862435266503\\
227	0.00430875934369804\\
228	0.00430889686093293\\
229	0.00430903695200158\\
230	0.00430917966544005\\
231	0.00430932505070579\\
232	0.00430947315819503\\
233	0.00430962403926053\\
234	0.00430977774622978\\
235	0.00430993433242325\\
236	0.0043100938521734\\
237	0.00431025636084354\\
238	0.00431042191484736\\
239	0.0043105905716687\\
240	0.00431076238988176\\
241	0.00431093742917142\\
242	0.00431111575035422\\
243	0.00431129741539953\\
244	0.00431148248745111\\
245	0.00431167103084907\\
246	0.00431186311115226\\
247	0.00431205879516082\\
248	0.00431225815093942\\
249	0.00431246124784067\\
250	0.00431266815652895\\
251	0.00431287894900475\\
252	0.00431309369862925\\
253	0.00431331248014943\\
254	0.00431353536972344\\
255	0.00431376244494658\\
256	0.00431399378487743\\
257	0.00431422947006457\\
258	0.00431446958257367\\
259	0.00431471420601489\\
260	0.00431496342557085\\
261	0.00431521732802484\\
262	0.00431547600178957\\
263	0.00431573953693623\\
264	0.00431600802522396\\
265	0.00431628156012981\\
266	0.00431656023687893\\
267	0.00431684415247528\\
268	0.00431713340573277\\
269	0.00431742809730652\\
270	0.00431772832972479\\
271	0.00431803420742105\\
272	0.00431834583676664\\
273	0.00431866332610351\\
274	0.00431898678577741\\
275	0.00431931632817146\\
276	0.00431965206773999\\
277	0.00431999412104263\\
278	0.00432034260677865\\
279	0.00432069764582177\\
280	0.00432105936125497\\
281	0.00432142787840569\\
282	0.00432180332488109\\
283	0.00432218583060372\\
284	0.00432257552784727\\
285	0.00432297255127231\\
286	0.00432337703796235\\
287	0.00432378912746006\\
288	0.00432420896180337\\
289	0.00432463668556183\\
290	0.00432507244587283\\
291	0.0043255163924781\\
292	0.00432596867776004\\
293	0.00432642945677815\\
294	0.00432689888730534\\
295	0.00432737712986429\\
296	0.00432786434776395\\
297	0.00432836070713575\\
298	0.00432886637696996\\
299	0.00432938152915217\\
300	0.00432990633849959\\
301	0.00433044098279764\\
302	0.00433098564283659\\
303	0.00433154050244854\\
304	0.00433210574854453\\
305	0.00433268157115222\\
306	0.00433326816345408\\
307	0.00433386572182634\\
308	0.00433447444587875\\
309	0.00433509453849552\\
310	0.00433572620587763\\
311	0.00433636965758684\\
312	0.00433702510659156\\
313	0.00433769276931526\\
314	0.00433837286568757\\
315	0.00433906561919869\\
316	0.00433977125695748\\
317	0.00434049000975394\\
318	0.00434122211212661\\
319	0.00434196780243559\\
320	0.00434272732294176\\
321	0.004343500919893\\
322	0.00434428884361807\\
323	0.004345091348629\\
324	0.00434590869373256\\
325	0.00434674114215151\\
326	0.00434758896165619\\
327	0.00434845242470694\\
328	0.00434933180860762\\
329	0.00435022739567029\\
330	0.00435113947339081\\
331	0.00435206833463485\\
332	0.00435301427783332\\
333	0.00435397760718565\\
334	0.00435495863286911\\
335	0.00435595767125125\\
336	0.00435697504510243\\
337	0.00435801108380426\\
338	0.00435906612354941\\
339	0.00436014050752808\\
340	0.0043612345860951\\
341	0.0043623487169129\\
342	0.00436348326506536\\
343	0.00436463860313877\\
344	0.004365815111269\\
345	0.00436701317715537\\
346	0.00436823319604323\\
347	0.00436947557074711\\
348	0.0043707407118326\\
349	0.0043720290378046\\
350	0.00437334097530099\\
351	0.00437467695929117\\
352	0.00437603743327923\\
353	0.00437742284951051\\
354	0.00437883366918161\\
355	0.00438027036265104\\
356	0.00438173340965126\\
357	0.00438322329950548\\
358	0.00438474053134917\\
359	0.00438628561435644\\
360	0.00438785906797147\\
361	0.0043894614221449\\
362	0.00439109321757576\\
363	0.00439275500595847\\
364	0.00439444735023603\\
365	0.00439617082485916\\
366	0.00439792601605232\\
367	0.00439971352208733\\
368	0.00440153395356529\\
369	0.00440338793370829\\
370	0.00440527609866231\\
371	0.00440719909781317\\
372	0.0044091575941179\\
373	0.00441115226445455\\
374	0.00441318379999359\\
375	0.00441525290659536\\
376	0.00441736030523858\\
377	0.00441950673248561\\
378	0.00442169294099152\\
379	0.00442391970006501\\
380	0.00442618779629041\\
381	0.00442849803422117\\
382	0.00443085123715657\\
383	0.0044332482480148\\
384	0.00443568993031624\\
385	0.00443817716929183\\
386	0.004440710873132\\
387	0.00444329197439091\\
388	0.00444592143156016\\
389	0.00444860023082302\\
390	0.00445132938799669\\
391	0.00445410995066293\\
392	0.00445694300047828\\
393	0.00445982965564104\\
394	0.00446277107347559\\
395	0.00446576845307248\\
396	0.00446882303790564\\
397	0.004471936118343\\
398	0.00447510903394544\\
399	0.00447834317079437\\
400	0.00448163995012209\\
401	0.00448500082751085\\
402	0.00448842729181676\\
403	0.0044919208638358\\
404	0.00449548309475358\\
405	0.00449911556445268\\
406	0.00450281987978895\\
407	0.0045065976729882\\
408	0.00451045060034677\\
409	0.00451438034142587\\
410	0.00451838859889704\\
411	0.00452247709914652\\
412	0.00452664759381917\\
413	0.00453090186251676\\
414	0.0045352417147148\\
415	0.00453966898894157\\
416	0.0045441855506476\\
417	0.00454879328977864\\
418	0.0045534941180901\\
419	0.00455828996627101\\
420	0.00456318278098549\\
421	0.00456817452198913\\
422	0.00457326715952721\\
423	0.00457846267227382\\
424	0.00458376304609542\\
425	0.00458917027399014\\
426	0.00459468635791866\\
427	0.00460031331623116\\
428	0.00460605352979721\\
429	0.00461190990897305\\
430	0.00461788548949603\\
431	0.00462398344079992\\
432	0.00463020707486431\\
433	0.00463655985556917\\
434	0.0046430454085166\\
435	0.00464966753135385\\
436	0.00465643020445243\\
437	0.00466333760180588\\
438	0.00467039410194801\\
439	0.00467760429861155\\
440	0.00468497301074153\\
441	0.00469250529133314\\
442	0.00470020643436151\\
443	0.00470808197874832\\
444	0.00471613770768533\\
445	0.0047243796400989\\
446	0.00473281400643386\\
447	0.0047414471853645\\
448	0.0047502855217763\\
449	0.00475931359508267\\
450	0.00476852503652838\\
451	0.00477792411818879\\
452	0.00478751522745984\\
453	0.00479730286871586\\
454	0.00480729166239887\\
455	0.00481748633836206\\
456	0.00482789175655993\\
457	0.00483851291571316\\
458	0.00484935496444078\\
459	0.00486042321558921\\
460	0.00487172316447817\\
461	0.00488326051116053\\
462	0.00489504118541369\\
463	0.0049070713661606\\
464	0.0049193574708969\\
465	0.00493190604161025\\
466	0.00494476407015433\\
467	0.00495794401795656\\
468	0.00497145787261548\\
469	0.00498531831362211\\
470	0.00499953876316261\\
471	0.00501413343991552\\
472	0.00502911741621035\\
473	0.00504450667898729\\
474	0.00506031819431716\\
475	0.00507656997661637\\
476	0.00509328116762204\\
477	0.0051104721473742\\
478	0.0051281647361623\\
479	0.00514638269199349\\
480	0.00516515080193658\\
481	0.00518449389127852\\
482	0.0052044376264225\\
483	0.0052250091872611\\
484	0.00524623737495304\\
485	0.00526815187863189\\
486	0.00529078468564697\\
487	0.00531417357557784\\
488	0.00533837735097006\\
489	0.00536344414407026\\
490	0.0053894165439534\\
491	0.00541633916344982\\
492	0.00544425797132098\\
493	0.00547322049379098\\
494	0.00550327613013326\\
495	0.00553447607120864\\
496	0.00556687314828419\\
497	0.00560052158990553\\
498	0.00563547657178693\\
499	0.00567179360157377\\
500	0.00570952787723091\\
501	0.0057487334698313\\
502	0.00578946229353781\\
503	0.00583176282309453\\
504	0.00587568061430283\\
505	0.00592126098008407\\
506	0.00596854041180243\\
507	0.0060175431613355\\
508	0.00606827693345952\\
509	0.0061207277550409\\
510	0.0061748191704826\\
511	0.00623017318207606\\
512	0.00628662745140264\\
513	0.00634395484991725\\
514	0.00640187056429931\\
515	0.00646008917106267\\
516	0.00651816770353974\\
517	0.00657574688726379\\
518	0.00663128465556018\\
519	0.00668140030949382\\
520	0.0067258985133532\\
521	0.00676486664598173\\
522	0.00679941350434478\\
523	0.00683269133876112\\
524	0.00686511250829205\\
525	0.00689716691348905\\
526	0.00692935080768741\\
527	0.00696191476452744\\
528	0.00699497463284185\\
529	0.00702862070688934\\
530	0.00706290768641132\\
531	0.00709786900375971\\
532	0.00713353198173733\\
533	0.00716991911972827\\
534	0.00720705005775949\\
535	0.00724494357205851\\
536	0.00728361853919417\\
537	0.00732309453011616\\
538	0.00736339161534406\\
539	0.00740453024079764\\
540	0.00744653870738161\\
541	0.00748948972389208\\
542	0.00753347142049639\\
543	0.00757747381845585\\
544	0.00762044744386721\\
545	0.00766211550727995\\
546	0.00770387386088141\\
547	0.00774606208827088\\
548	0.00778876715738478\\
549	0.00783205956081914\\
550	0.00787593614332746\\
551	0.00792038862872709\\
552	0.00796540239658982\\
553	0.00801096018996022\\
554	0.00805704284211701\\
555	0.00810363228870382\\
556	0.00815069909831113\\
557	0.00819820719505273\\
558	0.00824613258604494\\
559	0.00829446918237396\\
560	0.00834131866989406\\
561	0.00838711839027356\\
562	0.00843309183235914\\
563	0.00847930989270675\\
564	0.00852577430147598\\
565	0.00857245512281394\\
566	0.00861932104340902\\
567	0.0086663403000851\\
568	0.00871348883275888\\
569	0.00876041921360522\\
570	0.00880715257596726\\
571	0.00885403364170714\\
572	0.00890105214066917\\
573	0.0089481720325389\\
574	0.00899535448039004\\
575	0.0090425580087869\\
576	0.00908973854196738\\
577	0.00913684950111323\\
578	0.00918384197867516\\
579	0.00923066501140877\\
580	0.0092772659789151\\
581	0.00932359116078622\\
582	0.00936958649310198\\
583	0.00941519857412797\\
584	0.00946037597948012\\
585	0.00950507095777015\\
586	0.00954924158550055\\
587	0.00959285445497632\\
588	0.00963588792409171\\
589	0.0096783357981771\\
590	0.00972021084271539\\
591	0.00976138933978722\\
592	0.00980171928775089\\
593	0.00984102135019415\\
594	0.00987905595166784\\
595	0.00991543381058343\\
596	0.00994937493726701\\
597	0.00997906286423442\\
598	0.0099999191923403\\
599	0\\
600	0\\
};
\addplot [color=black!20!mycolor21,solid,forget plot]
  table[row sep=crcr]{%
1	0.00431499694832056\\
2	0.00431499894077788\\
3	0.00431500096991465\\
4	0.00431500303640849\\
5	0.00431500514094959\\
6	0.00431500728424092\\
7	0.00431500946699855\\
8	0.00431501168995181\\
9	0.00431501395384356\\
10	0.00431501625943045\\
11	0.00431501860748317\\
12	0.00431502099878675\\
13	0.00431502343414068\\
14	0.00431502591435945\\
15	0.00431502844027261\\
16	0.00431503101272507\\
17	0.00431503363257744\\
18	0.00431503630070642\\
19	0.00431503901800483\\
20	0.00431504178538223\\
21	0.00431504460376497\\
22	0.00431504747409669\\
23	0.00431505039733851\\
24	0.00431505337446941\\
25	0.0043150564064866\\
26	0.00431505949440579\\
27	0.00431506263926162\\
28	0.0043150658421079\\
29	0.00431506910401811\\
30	0.00431507242608562\\
31	0.00431507580942415\\
32	0.0043150792551682\\
33	0.0043150827644733\\
34	0.0043150863385165\\
35	0.00431508997849671\\
36	0.00431509368563522\\
37	0.00431509746117602\\
38	0.0043151013063862\\
39	0.00431510522255646\\
40	0.0043151092110016\\
41	0.00431511327306083\\
42	0.00431511741009825\\
43	0.00431512162350343\\
44	0.00431512591469183\\
45	0.0043151302851052\\
46	0.0043151347362122\\
47	0.00431513926950884\\
48	0.00431514388651896\\
49	0.00431514858879484\\
50	0.0043151533779177\\
51	0.0043151582554982\\
52	0.00431516322317698\\
53	0.00431516828262534\\
54	0.00431517343554571\\
55	0.00431517868367225\\
56	0.00431518402877144\\
57	0.00431518947264272\\
58	0.00431519501711916\\
59	0.00431520066406793\\
60	0.00431520641539104\\
61	0.00431521227302606\\
62	0.00431521823894665\\
63	0.00431522431516331\\
64	0.00431523050372412\\
65	0.00431523680671531\\
66	0.00431524322626214\\
67	0.00431524976452953\\
68	0.00431525642372277\\
69	0.00431526320608839\\
70	0.00431527011391491\\
71	0.00431527714953353\\
72	0.00431528431531907\\
73	0.00431529161369067\\
74	0.00431529904711277\\
75	0.00431530661809576\\
76	0.00431531432919702\\
77	0.00431532218302178\\
78	0.00431533018222386\\
79	0.00431533832950687\\
80	0.00431534662762482\\
81	0.00431535507938334\\
82	0.00431536368764055\\
83	0.00431537245530801\\
84	0.00431538138535179\\
85	0.00431539048079349\\
86	0.00431539974471124\\
87	0.00431540918024076\\
88	0.00431541879057662\\
89	0.00431542857897303\\
90	0.00431543854874529\\
91	0.00431544870327075\\
92	0.00431545904599003\\
93	0.00431546958040823\\
94	0.00431548031009616\\
95	0.00431549123869157\\
96	0.00431550236990035\\
97	0.0043155137074979\\
98	0.00431552525533045\\
99	0.00431553701731626\\
100	0.00431554899744721\\
101	0.00431556119979001\\
102	0.0043155736284877\\
103	0.00431558628776109\\
104	0.00431559918191016\\
105	0.00431561231531569\\
106	0.00431562569244063\\
107	0.00431563931783176\\
108	0.00431565319612135\\
109	0.0043156673320286\\
110	0.00431568173036141\\
111	0.004315696396018\\
112	0.00431571133398872\\
113	0.00431572654935767\\
114	0.00431574204730453\\
115	0.00431575783310647\\
116	0.00431577391213986\\
117	0.00431579028988224\\
118	0.00431580697191418\\
119	0.00431582396392129\\
120	0.00431584127169621\\
121	0.00431585890114065\\
122	0.00431587685826738\\
123	0.0043158951492025\\
124	0.00431591378018745\\
125	0.0043159327575813\\
126	0.00431595208786295\\
127	0.0043159717776334\\
128	0.00431599183361809\\
129	0.00431601226266936\\
130	0.00431603307176863\\
131	0.00431605426802919\\
132	0.00431607585869843\\
133	0.00431609785116058\\
134	0.00431612025293926\\
135	0.00431614307170008\\
136	0.00431616631525346\\
137	0.00431618999155733\\
138	0.00431621410872\\
139	0.00431623867500296\\
140	0.00431626369882397\\
141	0.00431628918875977\\
142	0.00431631515354941\\
143	0.00431634160209725\\
144	0.00431636854347602\\
145	0.00431639598693029\\
146	0.00431642394187952\\
147	0.00431645241792153\\
148	0.00431648142483594\\
149	0.00431651097258764\\
150	0.00431654107133032\\
151	0.00431657173141015\\
152	0.00431660296336936\\
153	0.00431663477795024\\
154	0.00431666718609875\\
155	0.00431670019896855\\
156	0.00431673382792501\\
157	0.00431676808454926\\
158	0.00431680298064243\\
159	0.00431683852822968\\
160	0.00431687473956477\\
161	0.00431691162713437\\
162	0.0043169492036625\\
163	0.0043169874821152\\
164	0.00431702647570518\\
165	0.00431706619789658\\
166	0.00431710666240985\\
167	0.00431714788322675\\
168	0.00431718987459531\\
169	0.00431723265103512\\
170	0.00431727622734244\\
171	0.00431732061859577\\
172	0.00431736584016113\\
173	0.00431741190769776\\
174	0.00431745883716381\\
175	0.00431750664482215\\
176	0.00431755534724629\\
177	0.00431760496132637\\
178	0.00431765550427544\\
179	0.00431770699363569\\
180	0.00431775944728483\\
181	0.00431781288344283\\
182	0.00431786732067833\\
183	0.00431792277791573\\
184	0.00431797927444189\\
185	0.00431803682991343\\
186	0.00431809546436388\\
187	0.00431815519821102\\
188	0.00431821605226454\\
189	0.00431827804773369\\
190	0.00431834120623502\\
191	0.00431840554980058\\
192	0.00431847110088596\\
193	0.00431853788237864\\
194	0.0043186059176065\\
195	0.00431867523034657\\
196	0.00431874584483381\\
197	0.00431881778577029\\
198	0.00431889107833416\\
199	0.00431896574818935\\
200	0.004319041821495\\
201	0.0043191193249154\\
202	0.00431919828562992\\
203	0.00431927873134326\\
204	0.0043193606902959\\
205	0.0043194441912748\\
206	0.00431952926362411\\
207	0.00431961593725647\\
208	0.0043197042426642\\
209	0.00431979421093099\\
210	0.00431988587374356\\
211	0.00431997926340383\\
212	0.00432007441284126\\
213	0.00432017135562532\\
214	0.00432027012597845\\
215	0.00432037075878894\\
216	0.00432047328962461\\
217	0.00432057775474629\\
218	0.00432068419112179\\
219	0.00432079263644017\\
220	0.00432090312912635\\
221	0.00432101570835582\\
222	0.00432113041406999\\
223	0.00432124728699163\\
224	0.00432136636864066\\
225	0.00432148770135029\\
226	0.00432161132828377\\
227	0.00432173729345093\\
228	0.00432186564172562\\
229	0.00432199641886331\\
230	0.00432212967151902\\
231	0.00432226544726559\\
232	0.00432240379461269\\
233	0.00432254476302579\\
234	0.00432268840294584\\
235	0.00432283476580929\\
236	0.00432298390406842\\
237	0.0043231358712123\\
238	0.00432329072178821\\
239	0.00432344851142332\\
240	0.004323609296847\\
241	0.00432377313591361\\
242	0.00432394008762582\\
243	0.00432411021215819\\
244	0.00432428357088166\\
245	0.00432446022638823\\
246	0.00432464024251637\\
247	0.00432482368437691\\
248	0.0043250106183795\\
249	0.00432520111225972\\
250	0.00432539523510669\\
251	0.00432559305739126\\
252	0.0043257946509949\\
253	0.00432600008923925\\
254	0.00432620944691617\\
255	0.00432642280031853\\
256	0.00432664022727168\\
257	0.00432686180716552\\
258	0.00432708762098739\\
259	0.00432731775135558\\
260	0.00432755228255349\\
261	0.0043277913005647\\
262	0.00432803489310863\\
263	0.00432828314967698\\
264	0.00432853616157106\\
265	0.00432879402193971\\
266	0.00432905682581817\\
267	0.00432932467016762\\
268	0.00432959765391561\\
269	0.00432987587799733\\
270	0.00433015944539764\\
271	0.00433044846119397\\
272	0.00433074303260004\\
273	0.00433104326901036\\
274	0.00433134928204578\\
275	0.00433166118559969\\
276	0.00433197909588508\\
277	0.00433230313148254\\
278	0.00433263341338901\\
279	0.00433297006506743\\
280	0.00433331321249714\\
281	0.00433366298422504\\
282	0.00433401951141767\\
283	0.00433438292791386\\
284	0.00433475337027831\\
285	0.00433513097785564\\
286	0.00433551589282521\\
287	0.00433590826025656\\
288	0.00433630822816532\\
289	0.00433671594756963\\
290	0.00433713157254702\\
291	0.00433755526029147\\
292	0.00433798717117093\\
293	0.00433842746878474\\
294	0.00433887632002143\\
295	0.00433933389511596\\
296	0.00433980036770704\\
297	0.00434027591489401\\
298	0.00434076071729329\\
299	0.00434125495909364\\
300	0.00434175882811116\\
301	0.00434227251584266\\
302	0.00434279621751782\\
303	0.00434333013214985\\
304	0.00434387446258422\\
305	0.00434442941554523\\
306	0.0043449952016803\\
307	0.00434557203560147\\
308	0.00434616013592399\\
309	0.00434675972530172\\
310	0.00434737103045892\\
311	0.00434799428221822\\
312	0.00434862971552461\\
313	0.00434927756946509\\
314	0.00434993808728401\\
315	0.00435061151639384\\
316	0.00435129810838146\\
317	0.00435199811901004\\
318	0.00435271180821678\\
319	0.0043534394401069\\
320	0.00435418128294456\\
321	0.00435493760914164\\
322	0.00435570869524555\\
323	0.00435649482192759\\
324	0.0043572962739739\\
325	0.00435811334028154\\
326	0.00435894631386279\\
327	0.00435979549186134\\
328	0.00436066117558451\\
329	0.00436154367055733\\
330	0.0043624432866041\\
331	0.00436336033796421\\
332	0.00436429514345049\\
333	0.00436524802665853\\
334	0.00436621931623619\\
335	0.00436720934622373\\
336	0.00436821845647436\\
337	0.00436924699316567\\
338	0.00437029530941072\\
339	0.00437136376597633\\
340	0.00437245273211302\\
341	0.00437356258649627\\
342	0.00437469371827179\\
343	0.00437584652819098\\
344	0.00437702142981376\\
345	0.00437821885075589\\
346	0.00437943923394699\\
347	0.00438068303683586\\
348	0.0043819507271359\\
349	0.00438324278311871\\
350	0.00438455969391738\\
351	0.00438590195984028\\
352	0.00438727009269485\\
353	0.00438866461612229\\
354	0.00439008606594292\\
355	0.00439153499051145\\
356	0.00439301195107609\\
357	0.00439451752214569\\
358	0.00439605229186454\\
359	0.00439761686239407\\
360	0.00439921185030041\\
361	0.00440083788694632\\
362	0.00440249561888556\\
363	0.00440418570825825\\
364	0.00440590883318406\\
365	0.00440766568815095\\
366	0.00440945698439527\\
367	0.0044112834502697\\
368	0.00441314583159376\\
369	0.00441504489198145\\
370	0.00441698141313957\\
371	0.00441895619512892\\
372	0.00442097005658\\
373	0.00442302383485335\\
374	0.0044251183861332\\
375	0.00442725458544242\\
376	0.00442943332656388\\
377	0.00443165552185372\\
378	0.00443392210192886\\
379	0.00443623401521027\\
380	0.00443859222730249\\
381	0.00444099772018842\\
382	0.00444345149121776\\
383	0.00444595455186778\\
384	0.00444850792625582\\
385	0.00445111264938523\\
386	0.00445376976511048\\
387	0.00445648032381303\\
388	0.00445924537978965\\
389	0.00446206598836737\\
390	0.00446494320277788\\
391	0.00446787807084819\\
392	0.00447087163159433\\
393	0.00447392491184105\\
394	0.00447703892302631\\
395	0.00448021465837707\\
396	0.00448345309066581\\
397	0.00448675517093273\\
398	0.00449012182999356\\
399	0.00449355412269602\\
400	0.00449705347939348\\
401	0.00450062137353033\\
402	0.00450425932415012\\
403	0.00450796889867491\\
404	0.00451175171598503\\
405	0.00451560944982605\\
406	0.00451954383256431\\
407	0.00452355665930099\\
408	0.00452764979233868\\
409	0.00453182516598714\\
410	0.0045360847917572\\
411	0.00454043076433954\\
412	0.00454486526831602\\
413	0.00454939058452903\\
414	0.00455400909656978\\
415	0.00455872329750913\\
416	0.00456353579682076\\
417	0.00456844932732751\\
418	0.00457346675197657\\
419	0.00457859107016703\\
420	0.00458382542321563\\
421	0.00458917309840705\\
422	0.00459463753106349\\
423	0.00460022230356219\\
424	0.0046059311395983\\
425	0.00461176788800474\\
426	0.00461773648292344\\
427	0.00462384083473687\\
428	0.00463007454269391\\
429	0.00463642669478293\\
430	0.00464289947138564\\
431	0.00464949507804951\\
432	0.00465621574373993\\
433	0.00466306371942736\\
434	0.00467004127672492\\
435	0.00467715070359887\\
436	0.00468439430175754\\
437	0.00469177438409394\\
438	0.00469929327236011\\
439	0.0047069532953228\\
440	0.00471475678774227\\
441	0.00472270609062633\\
442	0.00473080355332531\\
443	0.00473905153808924\\
444	0.00474745242750651\\
445	0.00475600863417552\\
446	0.00476472260824388\\
447	0.00477359682704871\\
448	0.00478263371851145\\
449	0.00479185742060597\\
450	0.00480128388406291\\
451	0.00481091912788268\\
452	0.0048207694746903\\
453	0.00483084157395689\\
454	0.00484114242736373\\
455	0.00485167941703031\\
456	0.00486246033618055\\
457	0.00487349342185919\\
458	0.00488478739021973\\
459	0.00489635147453042\\
460	0.00490819546631836\\
461	0.00492032976123099\\
462	0.00493276541531882\\
463	0.00494551423176829\\
464	0.00495858894656423\\
465	0.00497200374370233\\
466	0.00498577276785185\\
467	0.00499990968924556\\
468	0.00501442887447785\\
469	0.00502934546358139\\
470	0.00504467541338836\\
471	0.00506043554226604\\
472	0.00507664357606649\\
473	0.00509331819510764\\
474	0.00511047908213691\\
475	0.00512814697166886\\
476	0.00514634370104921\\
477	0.00516509226166558\\
478	0.00518441684913845\\
479	0.00520434292710943\\
480	0.00522489733605872\\
481	0.00524610843829957\\
482	0.00526800545903931\\
483	0.00529061981117917\\
484	0.00531398865034318\\
485	0.0053381701578214\\
486	0.00536321129115987\\
487	0.00538915358114079\\
488	0.00541604042217931\\
489	0.0054439163752652\\
490	0.00547282747592509\\
491	0.00550282136238553\\
492	0.00553394694540943\\
493	0.0055662540850593\\
494	0.00559979322677082\\
495	0.00563461493277553\\
496	0.00567076928958796\\
497	0.00570830517638911\\
498	0.00574727309521228\\
499	0.00578772535756451\\
500	0.00582971021638085\\
501	0.00587326972706042\\
502	0.00591843697540275\\
503	0.00596523248465533\\
504	0.00601358359371006\\
505	0.00606322632772836\\
506	0.00611408959572197\\
507	0.00616606641138297\\
508	0.00621900369888828\\
509	0.00627268736679497\\
510	0.00632686701159178\\
511	0.00638156012032947\\
512	0.00643639486293027\\
513	0.00649077754730627\\
514	0.00654356250451543\\
515	0.00659100231749131\\
516	0.00663288736294835\\
517	0.00666934946135635\\
518	0.00670138719394604\\
519	0.00673220793885969\\
520	0.00676222399287684\\
521	0.00679192047183419\\
522	0.0068217634860095\\
523	0.00685198358162248\\
524	0.00688268742381229\\
525	0.00691395526121181\\
526	0.00694583429170407\\
527	0.0069783535437772\\
528	0.00701153670833231\\
529	0.00704540320024276\\
530	0.00707997042946935\\
531	0.00711525540894092\\
532	0.0071512752192859\\
533	0.0071880474870327\\
534	0.00722559083761262\\
535	0.0072639251102361\\
536	0.00730307067581672\\
537	0.00734304842914756\\
538	0.00738391399526787\\
539	0.00742574890892478\\
540	0.00746833575598918\\
541	0.00750995398553259\\
542	0.00755035018874908\\
543	0.00759032441223851\\
544	0.00763072324758412\\
545	0.00767162903141855\\
546	0.00771311969698628\\
547	0.00775520894382689\\
548	0.00779789355612366\\
549	0.00784116403536065\\
550	0.007885008292511\\
551	0.00792941148083906\\
552	0.00797435600912795\\
553	0.00801982170725804\\
554	0.00806578702538616\\
555	0.00811222764331581\\
556	0.0081591063939708\\
557	0.00820641693101756\\
558	0.00825328101622345\\
559	0.00829844873438657\\
560	0.00834367510390144\\
561	0.00838916204614102\\
562	0.00843494259992949\\
563	0.00848099134388042\\
564	0.00852727997864265\\
565	0.00857377962259063\\
566	0.008620461017887\\
567	0.00866730268255435\\
568	0.00871391830588996\\
569	0.00876044306491334\\
570	0.00880715274894682\\
571	0.00885403364426453\\
572	0.00890105214164188\\
573	0.00894817203301684\\
574	0.00899535448062322\\
575	0.00904255800889528\\
576	0.00908973854201466\\
577	0.00913684950113236\\
578	0.00918384197868222\\
579	0.0092306650114111\\
580	0.00927726597891577\\
581	0.00932359116078638\\
582	0.00936958649310202\\
583	0.00941519857412799\\
584	0.00946037597948012\\
585	0.00950507095777015\\
586	0.00954924158550055\\
587	0.00959285445497631\\
588	0.00963588792409171\\
589	0.00967833579817709\\
590	0.00972021084271539\\
591	0.00976138933978722\\
592	0.00980171928775089\\
593	0.00984102135019415\\
594	0.00987905595166784\\
595	0.00991543381058343\\
596	0.00994937493726701\\
597	0.00997906286423442\\
598	0.0099999191923403\\
599	0\\
600	0\\
};
\addplot [color=black!50!mycolor20,solid,forget plot]
  table[row sep=crcr]{%
1	0.00432156273776794\\
2	0.0043215647387563\\
3	0.00432156677664413\\
4	0.00432156885211437\\
5	0.00432157096586263\\
6	0.00432157311859742\\
7	0.00432157531104039\\
8	0.00432157754392669\\
9	0.00432157981800506\\
10	0.00432158213403821\\
11	0.00432158449280303\\
12	0.00432158689509076\\
13	0.00432158934170756\\
14	0.00432159183347438\\
15	0.0043215943712275\\
16	0.00432159695581882\\
17	0.00432159958811599\\
18	0.00432160226900288\\
19	0.00432160499937982\\
20	0.00432160778016382\\
21	0.00432161061228898\\
22	0.00432161349670679\\
23	0.00432161643438645\\
24	0.00432161942631524\\
25	0.00432162247349878\\
26	0.00432162557696142\\
27	0.00432162873774662\\
28	0.00432163195691715\\
29	0.00432163523555574\\
30	0.00432163857476522\\
31	0.00432164197566891\\
32	0.0043216454394111\\
33	0.00432164896715743\\
34	0.00432165256009517\\
35	0.00432165621943383\\
36	0.00432165994640539\\
37	0.00432166374226482\\
38	0.00432166760829047\\
39	0.00432167154578451\\
40	0.00432167555607341\\
41	0.00432167964050835\\
42	0.00432168380046576\\
43	0.00432168803734771\\
44	0.00432169235258235\\
45	0.0043216967476246\\
46	0.00432170122395643\\
47	0.00432170578308751\\
48	0.00432171042655564\\
49	0.0043217151559273\\
50	0.00432171997279828\\
51	0.00432172487879413\\
52	0.0043217298755707\\
53	0.00432173496481483\\
54	0.00432174014824476\\
55	0.00432174542761091\\
56	0.00432175080469635\\
57	0.00432175628131747\\
58	0.00432176185932452\\
59	0.00432176754060238\\
60	0.00432177332707118\\
61	0.00432177922068684\\
62	0.00432178522344185\\
63	0.00432179133736603\\
64	0.00432179756452703\\
65	0.00432180390703123\\
66	0.00432181036702437\\
67	0.00432181694669225\\
68	0.00432182364826164\\
69	0.00432183047400093\\
70	0.00432183742622088\\
71	0.0043218445072756\\
72	0.00432185171956308\\
73	0.00432185906552633\\
74	0.00432186654765398\\
75	0.00432187416848134\\
76	0.00432188193059106\\
77	0.00432188983661416\\
78	0.00432189788923104\\
79	0.00432190609117213\\
80	0.0043219144452191\\
81	0.00432192295420573\\
82	0.00432193162101884\\
83	0.00432194044859928\\
84	0.00432194943994308\\
85	0.00432195859810238\\
86	0.00432196792618659\\
87	0.00432197742736336\\
88	0.00432198710485967\\
89	0.00432199696196312\\
90	0.00432200700202283\\
91	0.00432201722845085\\
92	0.00432202764472316\\
93	0.00432203825438095\\
94	0.0043220490610319\\
95	0.00432206006835131\\
96	0.00432207128008351\\
97	0.00432208270004313\\
98	0.00432209433211635\\
99	0.00432210618026241\\
100	0.0043221182485148\\
101	0.00432213054098283\\
102	0.00432214306185296\\
103	0.00432215581539031\\
104	0.00432216880594014\\
105	0.00432218203792936\\
106	0.00432219551586814\\
107	0.00432220924435133\\
108	0.00432222322806028\\
109	0.00432223747176431\\
110	0.00432225198032239\\
111	0.00432226675868503\\
112	0.00432228181189574\\
113	0.00432229714509297\\
114	0.00432231276351192\\
115	0.00432232867248623\\
116	0.00432234487745005\\
117	0.0043223613839398\\
118	0.00432237819759618\\
119	0.00432239532416614\\
120	0.00432241276950487\\
121	0.00432243053957785\\
122	0.00432244864046308\\
123	0.00432246707835305\\
124	0.00432248585955702\\
125	0.00432250499050324\\
126	0.00432252447774112\\
127	0.00432254432794366\\
128	0.00432256454790976\\
129	0.00432258514456658\\
130	0.0043226061249721\\
131	0.00432262749631745\\
132	0.00432264926592956\\
133	0.00432267144127379\\
134	0.00432269402995643\\
135	0.0043227170397275\\
136	0.00432274047848354\\
137	0.00432276435427027\\
138	0.00432278867528554\\
139	0.00432281344988224\\
140	0.0043228386865711\\
141	0.00432286439402405\\
142	0.00432289058107703\\
143	0.00432291725673317\\
144	0.0043229444301661\\
145	0.00432297211072314\\
146	0.00432300030792858\\
147	0.00432302903148723\\
148	0.00432305829128777\\
149	0.00432308809740627\\
150	0.00432311846010989\\
151	0.00432314938986035\\
152	0.00432318089731803\\
153	0.00432321299334535\\
154	0.00432324568901105\\
155	0.00432327899559389\\
156	0.00432331292458686\\
157	0.00432334748770122\\
158	0.00432338269687067\\
159	0.00432341856425585\\
160	0.00432345510224846\\
161	0.00432349232347589\\
162	0.00432353024080571\\
163	0.0043235688673503\\
164	0.00432360821647165\\
165	0.00432364830178617\\
166	0.00432368913716954\\
167	0.0043237307367618\\
168	0.00432377311497254\\
169	0.00432381628648593\\
170	0.00432386026626634\\
171	0.00432390506956349\\
172	0.00432395071191814\\
173	0.00432399720916782\\
174	0.00432404457745243\\
175	0.00432409283322027\\
176	0.00432414199323388\\
177	0.00432419207457645\\
178	0.00432424309465783\\
179	0.00432429507122102\\
180	0.00432434802234867\\
181	0.00432440196646958\\
182	0.00432445692236575\\
183	0.00432451290917902\\
184	0.00432456994641837\\
185	0.00432462805396691\\
186	0.00432468725208936\\
187	0.00432474756143953\\
188	0.00432480900306796\\
189	0.00432487159842961\\
190	0.00432493536939208\\
191	0.0043250003382435\\
192	0.00432506652770101\\
193	0.00432513396091912\\
194	0.00432520266149843\\
195	0.00432527265349443\\
196	0.00432534396142653\\
197	0.00432541661028717\\
198	0.00432549062555138\\
199	0.00432556603318634\\
200	0.00432564285966097\\
201	0.00432572113195616\\
202	0.00432580087757486\\
203	0.00432588212455263\\
204	0.00432596490146808\\
205	0.00432604923745394\\
206	0.00432613516220803\\
207	0.00432622270600464\\
208	0.00432631189970619\\
209	0.00432640277477489\\
210	0.00432649536328497\\
211	0.00432658969793499\\
212	0.0043266858120604\\
213	0.00432678373964642\\
214	0.00432688351534124\\
215	0.00432698517446952\\
216	0.00432708875304603\\
217	0.00432719428778965\\
218	0.00432730181613792\\
219	0.00432741137626148\\
220	0.00432752300707917\\
221	0.00432763674827337\\
222	0.00432775264030556\\
223	0.00432787072443241\\
224	0.00432799104272207\\
225	0.00432811363807102\\
226	0.00432823855422097\\
227	0.00432836583577658\\
228	0.00432849552822319\\
229	0.00432862767794518\\
230	0.00432876233224459\\
231	0.00432889953936042\\
232	0.00432903934848804\\
233	0.00432918180979924\\
234	0.00432932697446275\\
235	0.00432947489466525\\
236	0.0043296256236327\\
237	0.0043297792156524\\
238	0.00432993572609534\\
239	0.00433009521143934\\
240	0.0043302577292924\\
241	0.004330423338417\\
242	0.00433059209875454\\
243	0.00433076407145074\\
244	0.00433093931888154\\
245	0.00433111790467953\\
246	0.00433129989376101\\
247	0.00433148535235388\\
248	0.00433167434802614\\
249	0.00433186694971496\\
250	0.00433206322775662\\
251	0.00433226325391725\\
252	0.00433246710142406\\
253	0.00433267484499761\\
254	0.00433288656088486\\
255	0.00433310232689295\\
256	0.00433332222242387\\
257	0.00433354632851018\\
258	0.00433377472785137\\
259	0.00433400750485138\\
260	0.00433424474565704\\
261	0.00433448653819752\\
262	0.00433473297222474\\
263	0.00433498413935495\\
264	0.00433524013311138\\
265	0.004335501048968\\
266	0.00433576698439459\\
267	0.00433603803890286\\
268	0.00433631431409392\\
269	0.00433659591370706\\
270	0.0043368829436698\\
271	0.00433717551214946\\
272	0.00433747372960603\\
273	0.00433777770884665\\
274	0.00433808756508153\\
275	0.00433840341598155\\
276	0.0043387253817374\\
277	0.00433905358512051\\
278	0.00433938815154574\\
279	0.00433972920913584\\
280	0.00434007688878776\\
281	0.00434043132424101\\
282	0.00434079265214799\\
283	0.00434116101214642\\
284	0.00434153654693376\\
285	0.004341919402344\\
286	0.00434230972742683\\
287	0.00434270767452879\\
288	0.00434311339937728\\
289	0.00434352706116669\\
290	0.00434394882264741\\
291	0.00434437885021715\\
292	0.00434481731401516\\
293	0.00434526438801898\\
294	0.00434572025014423\\
295	0.00434618508234692\\
296	0.00434665907072889\\
297	0.00434714240564587\\
298	0.00434763528181837\\
299	0.00434813789844575\\
300	0.00434865045932256\\
301	0.00434917317295794\\
302	0.00434970625269746\\
303	0.00435024991684725\\
304	0.00435080438880053\\
305	0.00435136989716599\\
306	0.00435194667589759\\
307	0.00435253496442565\\
308	0.00435313500778857\\
309	0.00435374705676422\\
310	0.00435437136800087\\
311	0.00435500820414609\\
312	0.00435565783397282\\
313	0.0043563205325014\\
314	0.0043569965811155\\
315	0.00435768626767073\\
316	0.00435838988659347\\
317	0.00435910773896732\\
318	0.0043598401326048\\
319	0.00436058738210031\\
320	0.0043613498088614\\
321	0.00436212774111298\\
322	0.00436292151387087\\
323	0.0043637314688778\\
324	0.00436455795449694\\
325	0.00436540132555528\\
326	0.00436626194312957\\
327	0.00436714017426647\\
328	0.00436803639162843\\
329	0.00436895097305518\\
330	0.00436988430103141\\
331	0.0043708367620511\\
332	0.00437180874586785\\
333	0.00437280064462305\\
334	0.00437381285184419\\
335	0.00437484576130809\\
336	0.0043758997657681\\
337	0.00437697525554858\\
338	0.00437807261701865\\
339	0.00437919223096592\\
340	0.00438033447090391\\
341	0.00438149970136107\\
342	0.00438268827621516\\
343	0.00438390053714678\\
344	0.00438513681229731\\
345	0.00438639741528997\\
346	0.00438768264538279\\
347	0.00438899284910283\\
348	0.00439032852186834\\
349	0.00439169016963117\\
350	0.00439307830916022\\
351	0.00439449346833624\\
352	0.00439593618645757\\
353	0.00439740701455486\\
354	0.00439890651571774\\
355	0.004400435265466\\
356	0.00440199385231873\\
357	0.00440358287824578\\
358	0.00440520295913732\\
359	0.00440685472530561\\
360	0.00440853882202268\\
361	0.00441025591009763\\
362	0.00441200666649775\\
363	0.0044137917850179\\
364	0.0044156119770036\\
365	0.00441746797213329\\
366	0.00441936051926643\\
367	0.00442129038736393\\
368	0.00442325836648904\\
369	0.00442526526889701\\
370	0.00442731193022267\\
371	0.0044293992107759\\
372	0.00443152799695614\\
373	0.00443369920279728\\
374	0.00443591377165537\\
375	0.00443817267805222\\
376	0.00444047692968832\\
377	0.0044428275696388\\
378	0.00444522567874573\\
379	0.00444767237822\\
380	0.00445016883246353\\
381	0.00445271625212137\\
382	0.00445531589736821\\
383	0.00445796908142958\\
384	0.00446067717432911\\
385	0.00446344160684308\\
386	0.00446626387462823\\
387	0.0044691455424694\\
388	0.00447208824856691\\
389	0.00447509370874885\\
390	0.00447816372044754\\
391	0.00448130016621814\\
392	0.00448450501649252\\
393	0.00448778033113468\\
394	0.00449112825913685\\
395	0.00449455103528687\\
396	0.00449805097120155\\
397	0.00450163043343578\\
398	0.00450529178476443\\
399	0.00450903312499081\\
400	0.00451284518155388\\
401	0.00451672918428431\\
402	0.00452068637426448\\
403	0.00452471800279135\\
404	0.0045288253302109\\
405	0.0045330096246486\\
406	0.00453727216070178\\
407	0.00454161421822174\\
408	0.00454603708133855\\
409	0.00455054203755669\\
410	0.00455513037477349\\
411	0.00455980336567712\\
412	0.0045645622530175\\
413	0.00456940826439893\\
414	0.00457434261038014\\
415	0.00457936647922931\\
416	0.00458448103142263\\
417	0.00458968739511137\\
418	0.00459498666169048\\
419	0.00460037988191109\\
420	0.0046058680639046\\
421	0.0046114521734614\\
422	0.00461713313071034\\
423	0.00462291180709291\\
424	0.00462878901320134\\
425	0.00463476551515811\\
426	0.00464084202686863\\
427	0.00464701916605228\\
428	0.00465330764650406\\
429	0.00465972294715053\\
430	0.00466626796949668\\
431	0.0046729457221835\\
432	0.00467975932905515\\
433	0.00468671203806247\\
434	0.00469380723103485\\
435	0.0047010484344903\\
436	0.0047084393316286\\
437	0.00471598377555301\\
438	0.00472368580381427\\
439	0.00473154965436415\\
440	0.004739579782992\\
441	0.00474778088230933\\
442	0.00475615790236735\\
443	0.00476471607315437\\
444	0.00477346092989589\\
445	0.00478239834456105\\
446	0.00479153457560125\\
447	0.00480087637727662\\
448	0.00481043130862172\\
449	0.00482020706581279\\
450	0.00483021084487188\\
451	0.00484044985114416\\
452	0.00485093163581722\\
453	0.00486166411637998\\
454	0.00487265559813321\\
455	0.00488391479673149\\
456	0.00489545086173582\\
457	0.00490727340119204\\
458	0.00491939250723813\\
459	0.00493181878277906\\
460	0.00494456336936779\\
461	0.00495763797663287\\
462	0.00497105491383159\\
463	0.0049848271235522\\
464	0.0049989682118395\\
465	0.0050134924396604\\
466	0.00502841475798565\\
467	0.00504375091022794\\
468	0.00505951747774275\\
469	0.00507573192480233\\
470	0.0050924126447223\\
471	0.00510957900694087\\
472	0.00512725140490503\\
473	0.00514545130482393\\
474	0.0051642012957185\\
475	0.0051835251411441\\
476	0.00520344785397583\\
477	0.00522399583307634\\
478	0.0052451969499694\\
479	0.00526708001994137\\
480	0.00528967582130181\\
481	0.00531302073050156\\
482	0.00533717128049394\\
483	0.00536217440954135\\
484	0.00538807060812187\\
485	0.00541490203822802\\
486	0.00544271156614922\\
487	0.00547154299835975\\
488	0.00550144111267412\\
489	0.00553245141674796\\
490	0.00556461982902692\\
491	0.00559799228696207\\
492	0.00563262021729175\\
493	0.00566855676452508\\
494	0.00570585375095233\\
495	0.00574456031946986\\
496	0.00578472113107927\\
497	0.00582637393627206\\
498	0.00586941332332261\\
499	0.00591367412563156\\
500	0.0059591312442232\\
501	0.00600573960009374\\
502	0.00605342869976447\\
503	0.00610209427571644\\
504	0.00615166353661207\\
505	0.00620218464799494\\
506	0.0062534348899334\\
507	0.00630517093441952\\
508	0.00635706597096826\\
509	0.00640860003944755\\
510	0.00645893456739127\\
511	0.0065044826325323\\
512	0.00654456380460978\\
513	0.00657922957752016\\
514	0.00660912384878171\\
515	0.00663781310127959\\
516	0.00666570035195224\\
517	0.00669325982984133\\
518	0.00672095204731527\\
519	0.0067490035419314\\
520	0.00677751497187324\\
521	0.00680656037357772\\
522	0.00683618266850091\\
523	0.00686640821868181\\
524	0.00689725841090671\\
525	0.0069287508654358\\
526	0.00696090155737811\\
527	0.00699372616758012\\
528	0.00702724047294954\\
529	0.00706146075026851\\
530	0.00709640408608181\\
531	0.00713208865138898\\
532	0.00716853402710913\\
533	0.00720576089941698\\
534	0.00724379067528484\\
535	0.00728265318316113\\
536	0.00732242355127311\\
537	0.00736319290804136\\
538	0.00740362713540518\\
539	0.00744294329917313\\
540	0.00748116645121112\\
541	0.00751979739156501\\
542	0.00755890721054429\\
543	0.00759857987564944\\
544	0.00763885345226277\\
545	0.00767972887409422\\
546	0.00772120151702209\\
547	0.00776326398074099\\
548	0.00780590660744546\\
549	0.00784911740722832\\
550	0.00789288179370365\\
551	0.00793718233108313\\
552	0.0079819985880279\\
553	0.00802730742623898\\
554	0.00807308398607837\\
555	0.00811930438610058\\
556	0.00816597124869282\\
557	0.00821131168944101\\
558	0.00825575379156772\\
559	0.00830045453103746\\
560	0.00834548332965024\\
561	0.00839082502862416\\
562	0.00843645437478884\\
563	0.00848234516615932\\
564	0.00852847072095564\\
565	0.00857480405939613\\
566	0.00862132528999987\\
567	0.00866764572460421\\
568	0.00871393573045342\\
569	0.00876044307747747\\
570	0.00880715274928313\\
571	0.0088540336444077\\
572	0.00890105214171166\\
573	0.00894817203305002\\
574	0.00899535448063817\\
575	0.00904255800890162\\
576	0.00908973854201715\\
577	0.00913684950113324\\
578	0.0091838419786825\\
579	0.00923066501141119\\
580	0.00927726597891579\\
581	0.00932359116078638\\
582	0.00936958649310201\\
583	0.00941519857412797\\
584	0.00946037597948012\\
585	0.00950507095777015\\
586	0.00954924158550055\\
587	0.00959285445497631\\
588	0.00963588792409171\\
589	0.00967833579817709\\
590	0.00972021084271538\\
591	0.00976138933978722\\
592	0.00980171928775089\\
593	0.00984102135019415\\
594	0.00987905595166784\\
595	0.00991543381058343\\
596	0.00994937493726701\\
597	0.00997906286423442\\
598	0.0099999191923403\\
599	0\\
600	0\\
};
\addplot [color=black!60!mycolor21,solid,forget plot]
  table[row sep=crcr]{%
1	0.00432695855327979\\
2	0.00432696069719906\\
3	0.00432696288063723\\
4	0.00432696510432514\\
5	0.00432696736900736\\
6	0.00432696967544223\\
7	0.0043269720244022\\
8	0.00432697441667397\\
9	0.004326976853059\\
10	0.00432697933437348\\
11	0.0043269818614489\\
12	0.00432698443513213\\
13	0.00432698705628577\\
14	0.00432698972578843\\
15	0.00432699244453511\\
16	0.0043269952134374\\
17	0.0043269980334239\\
18	0.00432700090544033\\
19	0.00432700383045009\\
20	0.00432700680943442\\
21	0.00432700984339293\\
22	0.00432701293334372\\
23	0.00432701608032385\\
24	0.00432701928538961\\
25	0.004327022549617\\
26	0.004327025874102\\
27	0.00432702925996095\\
28	0.00432703270833108\\
29	0.00432703622037062\\
30	0.0043270397972594\\
31	0.00432704344019931\\
32	0.0043270471504144\\
33	0.00432705092915171\\
34	0.00432705477768136\\
35	0.00432705869729714\\
36	0.00432706268931694\\
37	0.00432706675508314\\
38	0.00432707089596317\\
39	0.00432707511334992\\
40	0.00432707940866214\\
41	0.00432708378334511\\
42	0.00432708823887093\\
43	0.00432709277673917\\
44	0.00432709739847733\\
45	0.00432710210564138\\
46	0.00432710689981621\\
47	0.0043271117826163\\
48	0.00432711675568621\\
49	0.00432712182070122\\
50	0.00432712697936762\\
51	0.0043271322334237\\
52	0.00432713758464017\\
53	0.00432714303482059\\
54	0.00432714858580225\\
55	0.00432715423945664\\
56	0.00432715999769022\\
57	0.004327165862445\\
58	0.00432717183569916\\
59	0.00432717791946777\\
60	0.00432718411580356\\
61	0.0043271904267975\\
62	0.00432719685457959\\
63	0.00432720340131958\\
64	0.00432721006922769\\
65	0.00432721686055543\\
66	0.00432722377759633\\
67	0.0043272308226867\\
68	0.00432723799820648\\
69	0.00432724530658002\\
70	0.00432725275027702\\
71	0.00432726033181314\\
72	0.00432726805375125\\
73	0.00432727591870186\\
74	0.00432728392932442\\
75	0.00432729208832793\\
76	0.00432730039847212\\
77	0.0043273088625682\\
78	0.00432731748347992\\
79	0.0043273262641246\\
80	0.00432733520747408\\
81	0.00432734431655567\\
82	0.00432735359445329\\
83	0.00432736304430868\\
84	0.00432737266932216\\
85	0.00432738247275397\\
86	0.00432739245792525\\
87	0.00432740262821932\\
88	0.00432741298708276\\
89	0.00432742353802666\\
90	0.00432743428462778\\
91	0.00432744523052981\\
92	0.00432745637944468\\
93	0.00432746773515366\\
94	0.00432747930150892\\
95	0.00432749108243471\\
96	0.00432750308192877\\
97	0.00432751530406363\\
98	0.00432752775298816\\
99	0.00432754043292886\\
100	0.00432755334819144\\
101	0.00432756650316226\\
102	0.00432757990230986\\
103	0.00432759355018648\\
104	0.00432760745142971\\
105	0.00432762161076401\\
106	0.00432763603300237\\
107	0.00432765072304815\\
108	0.00432766568589645\\
109	0.00432768092663611\\
110	0.0043276964504514\\
111	0.00432771226262376\\
112	0.0043277283685338\\
113	0.00432774477366301\\
114	0.00432776148359568\\
115	0.00432777850402097\\
116	0.00432779584073471\\
117	0.00432781349964156\\
118	0.00432783148675699\\
119	0.00432784980820943\\
120	0.00432786847024234\\
121	0.00432788747921642\\
122	0.00432790684161185\\
123	0.00432792656403053\\
124	0.00432794665319832\\
125	0.00432796711596752\\
126	0.00432798795931918\\
127	0.00432800919036546\\
128	0.00432803081635228\\
129	0.00432805284466176\\
130	0.00432807528281481\\
131	0.00432809813847368\\
132	0.00432812141944487\\
133	0.00432814513368153\\
134	0.00432816928928657\\
135	0.00432819389451525\\
136	0.00432821895777821\\
137	0.00432824448764443\\
138	0.00432827049284404\\
139	0.00432829698227167\\
140	0.0043283239649894\\
141	0.00432835145022986\\
142	0.00432837944739962\\
143	0.00432840796608247\\
144	0.00432843701604282\\
145	0.00432846660722887\\
146	0.00432849674977659\\
147	0.00432852745401276\\
148	0.00432855873045901\\
149	0.0043285905898353\\
150	0.00432862304306379\\
151	0.00432865610127265\\
152	0.00432868977579992\\
153	0.00432872407819774\\
154	0.00432875902023616\\
155	0.00432879461390737\\
156	0.00432883087143009\\
157	0.00432886780525368\\
158	0.00432890542806266\\
159	0.00432894375278118\\
160	0.0043289827925775\\
161	0.0043290225608689\\
162	0.00432906307132606\\
163	0.00432910433787831\\
164	0.00432914637471825\\
165	0.00432918919630696\\
166	0.00432923281737907\\
167	0.00432927725294803\\
168	0.00432932251831144\\
169	0.0043293686290565\\
170	0.00432941560106546\\
171	0.0043294634505214\\
172	0.00432951219391403\\
173	0.00432956184804536\\
174	0.00432961243003595\\
175	0.00432966395733081\\
176	0.00432971644770586\\
177	0.00432976991927409\\
178	0.00432982439049204\\
179	0.00432987988016658\\
180	0.00432993640746147\\
181	0.00432999399190443\\
182	0.00433005265339393\\
183	0.00433011241220651\\
184	0.0043301732890039\\
185	0.00433023530484067\\
186	0.00433029848117162\\
187	0.00433036283985964\\
188	0.00433042840318345\\
189	0.00433049519384591\\
190	0.00433056323498191\\
191	0.00433063255016688\\
192	0.00433070316342541\\
193	0.00433077509923983\\
194	0.00433084838255915\\
195	0.00433092303880804\\
196	0.00433099909389616\\
197	0.0043310765742276\\
198	0.00433115550671049\\
199	0.00433123591876661\\
200	0.00433131783834173\\
201	0.00433140129391559\\
202	0.00433148631451246\\
203	0.00433157292971159\\
204	0.00433166116965828\\
205	0.00433175106507474\\
206	0.00433184264727161\\
207	0.00433193594815924\\
208	0.00433203100025962\\
209	0.00433212783671843\\
210	0.00433222649131709\\
211	0.00433232699848539\\
212	0.00433242939331433\\
213	0.00433253371156901\\
214	0.00433263998970192\\
215	0.00433274826486667\\
216	0.00433285857493167\\
217	0.00433297095849443\\
218	0.0043330854548959\\
219	0.00433320210423526\\
220	0.00433332094738493\\
221	0.00433344202600606\\
222	0.00433356538256405\\
223	0.00433369106034479\\
224	0.00433381910347088\\
225	0.00433394955691843\\
226	0.00433408246653407\\
227	0.00433421787905251\\
228	0.00433435584211419\\
229	0.00433449640428373\\
230	0.00433463961506829\\
231	0.00433478552493675\\
232	0.00433493418533899\\
233	0.00433508564872588\\
234	0.00433523996856943\\
235	0.00433539719938359\\
236	0.00433555739674543\\
237	0.00433572061731673\\
238	0.0043358869188662\\
239	0.00433605636029206\\
240	0.00433622900164514\\
241	0.00433640490415262\\
242	0.00433658413024219\\
243	0.00433676674356681\\
244	0.00433695280902993\\
245	0.00433714239281142\\
246	0.0043373355623941\\
247	0.00433753238659085\\
248	0.0043377329355722\\
249	0.00433793728089482\\
250	0.00433814549553049\\
251	0.00433835765389585\\
252	0.00433857383188282\\
253	0.00433879410688977\\
254	0.00433901855785339\\
255	0.00433924726528139\\
256	0.00433948031128586\\
257	0.00433971777961768\\
258	0.00433995975570167\\
259	0.00434020632667236\\
260	0.00434045758141111\\
261	0.0043407136105838\\
262	0.00434097450667962\\
263	0.00434124036405092\\
264	0.00434151127895385\\
265	0.00434178734959036\\
266	0.00434206867615101\\
267	0.00434235536085907\\
268	0.00434264750801582\\
269	0.00434294522404696\\
270	0.00434324861755036\\
271	0.00434355779934511\\
272	0.00434387288252188\\
273	0.00434419398249479\\
274	0.00434452121705471\\
275	0.00434485470642407\\
276	0.00434519457331332\\
277	0.00434554094297909\\
278	0.00434589394328401\\
279	0.00434625370475853\\
280	0.0043466203606644\\
281	0.00434699404706052\\
282	0.00434737490287053\\
283	0.00434776306995298\\
284	0.00434815869317363\\
285	0.00434856192048034\\
286	0.0043489729029806\\
287	0.00434939179502176\\
288	0.00434981875427422\\
289	0.00435025394181805\\
290	0.00435069752223251\\
291	0.00435114966368958\\
292	0.00435161053805094\\
293	0.00435208032096964\\
294	0.00435255919199538\\
295	0.00435304733468546\\
296	0.00435354493672011\\
297	0.00435405219002397\\
298	0.00435456929089321\\
299	0.00435509644012949\\
300	0.0043556338431809\\
301	0.00435618171029078\\
302	0.0043567402566551\\
303	0.00435730970258917\\
304	0.00435789027370451\\
305	0.0043584822010971\\
306	0.00435908572154804\\
307	0.00435970107773785\\
308	0.00436032851847588\\
309	0.0043609682989465\\
310	0.0043616206809735\\
311	0.00436228593330542\\
312	0.00436296433192296\\
313	0.00436365616037193\\
314	0.00436436171012386\\
315	0.00436508128096751\\
316	0.00436581518143451\\
317	0.00436656372926307\\
318	0.00436732725190336\\
319	0.00436810608706939\\
320	0.00436890058334199\\
321	0.00436971110082804\\
322	0.00437053801188148\\
323	0.0043713817018922\\
324	0.00437224257014867\\
325	0.0043731210307803\\
326	0.00437401751378662\\
327	0.00437493246615868\\
328	0.00437586635309825\\
329	0.0043768196593399\\
330	0.00437779289057866\\
331	0.00437878657500424\\
332	0.00437980126494006\\
333	0.00438083753858002\\
334	0.0043818960018099\\
335	0.00438297729009362\\
336	0.00438408207039105\\
337	0.0043852110430619\\
338	0.00438636494368967\\
339	0.00438754454473416\\
340	0.00438875065688568\\
341	0.00438998412993984\\
342	0.00439124585291623\\
343	0.00439253675292674\\
344	0.00439385779169215\\
345	0.00439520995665367\\
346	0.00439659423677513\\
347	0.00439800980650354\\
348	0.00439945288719747\\
349	0.00440092400728502\\
350	0.00440242370465192\\
351	0.00440395252680818\\
352	0.00440551103111611\\
353	0.00440709978509535\\
354	0.00440871936667319\\
355	0.00441037036353843\\
356	0.00441205336796628\\
357	0.00441376898130935\\
358	0.00441551781433659\\
359	0.00441730048719107\\
360	0.00441911762932339\\
361	0.00442096987939735\\
362	0.00442285788516469\\
363	0.00442478230330447\\
364	0.00442674379922321\\
365	0.0044287430468102\\
366	0.00443078072814295\\
367	0.00443285753313633\\
368	0.00443497415912887\\
369	0.00443713131039822\\
370	0.00443932969759767\\
371	0.00444157003710466\\
372	0.00444385305027066\\
373	0.00444617946256172\\
374	0.00444855000257832\\
375	0.00445096540094059\\
376	0.00445342638902633\\
377	0.00445593369754693\\
378	0.00445848805494721\\
379	0.00446109018561414\\
380	0.00446374080788017\\
381	0.00446644063180854\\
382	0.00446919035674984\\
383	0.00447199066866269\\
384	0.00447484223719717\\
385	0.00447774571254766\\
386	0.0044807017220934\\
387	0.00448371086686047\\
388	0.00448677371786047\\
389	0.00448989081239065\\
390	0.00449306265041508\\
391	0.00449628969119351\\
392	0.0044995723503721\\
393	0.00450291099778542\\
394	0.00450630595618257\\
395	0.00450975750081975\\
396	0.00451326585891238\\
397	0.00451683120515006\\
398	0.00452045364209266\\
399	0.00452413735820681\\
400	0.00452789394284761\\
401	0.00453172483387663\\
402	0.00453563150123215\\
403	0.00453961544838681\\
404	0.00454367821396767\\
405	0.00454782137356271\\
406	0.00455204654173728\\
407	0.00455635537427974\\
408	0.00456074957067161\\
409	0.00456523087672764\\
410	0.00456980108733492\\
411	0.0045744620497789\\
412	0.00457921566841893\\
413	0.00458406390985492\\
414	0.00458900880804196\\
415	0.00459405246996129\\
416	0.00459919708201783\\
417	0.00460444491720469\\
418	0.00460979834306404\\
419	0.00461525983049784\\
420	0.00462083196341752\\
421	0.00462651744911842\\
422	0.0046323191295337\\
423	0.00463823999399964\\
424	0.00464428319606497\\
425	0.00465045208049502\\
426	0.00465675024305133\\
427	0.00466318170211106\\
428	0.00466975068776097\\
429	0.00467646102269409\\
430	0.00468331626401164\\
431	0.00469032009822216\\
432	0.00469747634809923\\
433	0.00470478897990422\\
434	0.00471226211098159\\
435	0.00471990001772997\\
436	0.00472770714394326\\
437	0.00473568810951569\\
438	0.00474384771949964\\
439	0.00475219097350291\\
440	0.00476072307541075\\
441	0.0047694494434255\\
442	0.00477837572044345\\
443	0.00478750778485379\\
444	0.00479685176198382\\
445	0.0048064140365962\\
446	0.00481620126662683\\
447	0.00482622039523245\\
448	0.00483647864149911\\
449	0.004846983503795\\
450	0.00485774280977831\\
451	0.00486876475354069\\
452	0.00488005791751922\\
453	0.00489163129574941\\
454	0.00490349431852351\\
455	0.00491565687852088\\
456	0.00492812935847706\\
457	0.00494092266045717\\
458	0.00495404823679719\\
459	0.00496751812277545\\
460	0.00498134497106543\\
461	0.00499554208798379\\
462	0.00501012347143418\\
463	0.00502510385021868\\
464	0.00504049872440022\\
465	0.00505632440914279\\
466	0.00507259808091958\\
467	0.00508933782211824\\
468	0.0051065626661562\\
469	0.00512429264292472\\
470	0.00514254882450478\\
471	0.00516135337167488\\
472	0.00518072958309387\\
473	0.00520070195230599\\
474	0.00522129624785663\\
475	0.00524253966141403\\
476	0.00526446073178525\\
477	0.00528708936517332\\
478	0.00531046069907047\\
479	0.00533462727696618\\
480	0.00535963747295621\\
481	0.00538552963238027\\
482	0.00541234318243217\\
483	0.0054401178576874\\
484	0.00546889387848687\\
485	0.00549871271132115\\
486	0.00552962358923932\\
487	0.00556167658536844\\
488	0.00559492199409357\\
489	0.00562940946645967\\
490	0.00566518680759279\\
491	0.00570229826537744\\
492	0.00574057270663483\\
493	0.00577994483980594\\
494	0.00582041438361801\\
495	0.00586196977977204\\
496	0.00590458482482532\\
497	0.00594821370429102\\
498	0.0059929190006349\\
499	0.00603878586440403\\
500	0.00608571187789154\\
501	0.00613354908577968\\
502	0.00618209679919275\\
503	0.00623111975679493\\
504	0.0062803460795765\\
505	0.00632932117173365\\
506	0.00637730219787598\\
507	0.00642155840085762\\
508	0.00646045553143953\\
509	0.00649394778892431\\
510	0.00652243322939103\\
511	0.00654927054506817\\
512	0.00657526604504233\\
513	0.00660087923655145\\
514	0.00662658810284878\\
515	0.00665262797492567\\
516	0.00667909600829342\\
517	0.00670606353836837\\
518	0.00673357132505209\\
519	0.00676164372812885\\
520	0.0067903003794335\\
521	0.00681955743687604\\
522	0.00684942956662143\\
523	0.00687993114722263\\
524	0.00691107661826311\\
525	0.00694288081781127\\
526	0.00697535922344092\\
527	0.00700852813728366\\
528	0.00704240489215253\\
529	0.0070770080849942\\
530	0.00711235785255778\\
531	0.00714847517504492\\
532	0.0071853817036795\\
533	0.00722312753643244\\
534	0.00726179417464418\\
535	0.00730115804025518\\
536	0.00733950427705983\\
537	0.00737657424641395\\
538	0.007413491098651\\
539	0.00745084701001097\\
540	0.00748873240113763\\
541	0.00752721478980981\\
542	0.00756629878659511\\
543	0.00760598438697385\\
544	0.00764626809925852\\
545	0.00768714464315555\\
546	0.0077286069021184\\
547	0.00777064572670249\\
548	0.00781324967273179\\
549	0.00785640469595659\\
550	0.00790009381669526\\
551	0.00794429678697347\\
552	0.00798898986048007\\
553	0.00803414520393946\\
554	0.00807975463982383\\
555	0.00812542668900067\\
556	0.00816939338006928\\
557	0.00821329420027352\\
558	0.00825751363257453\\
559	0.00830208407653316\\
560	0.00834698368736043\\
561	0.00839218901342985\\
562	0.00843767581465922\\
563	0.00848341941476406\\
564	0.0085293949138881\\
565	0.00857558303702803\\
566	0.00862162837759405\\
567	0.00866765938100907\\
568	0.00871393573140663\\
569	0.00876044307752332\\
570	0.00880715274930379\\
571	0.00885403364441763\\
572	0.00890105214171626\\
573	0.00894817203305202\\
574	0.00899535448063901\\
575	0.00904255800890193\\
576	0.00908973854201726\\
577	0.00913684950113328\\
578	0.00918384197868252\\
579	0.00923066501141119\\
580	0.00927726597891579\\
581	0.00932359116078638\\
582	0.00936958649310202\\
583	0.00941519857412797\\
584	0.00946037597948011\\
585	0.00950507095777015\\
586	0.00954924158550055\\
587	0.00959285445497631\\
588	0.00963588792409171\\
589	0.00967833579817709\\
590	0.00972021084271538\\
591	0.00976138933978722\\
592	0.00980171928775089\\
593	0.00984102135019415\\
594	0.00987905595166784\\
595	0.00991543381058343\\
596	0.00994937493726701\\
597	0.00997906286423442\\
598	0.0099999191923403\\
599	0\\
600	0\\
};
\addplot [color=black!80!mycolor21,solid,forget plot]
  table[row sep=crcr]{%
1	0.00433439942612947\\
2	0.00433440181287078\\
3	0.00433440424353772\\
4	0.00433440671894131\\
5	0.00433440923990749\\
6	0.0043344118072775\\
7	0.00433441442190828\\
8	0.00433441708467256\\
9	0.00433441979645923\\
10	0.00433442255817373\\
11	0.00433442537073827\\
12	0.00433442823509211\\
13	0.00433443115219192\\
14	0.00433443412301212\\
15	0.0043344371485452\\
16	0.00433444022980203\\
17	0.00433444336781228\\
18	0.00433444656362468\\
19	0.00433444981830739\\
20	0.00433445313294842\\
21	0.00433445650865589\\
22	0.00433445994655854\\
23	0.00433446344780604\\
24	0.0043344670135694\\
25	0.0043344706450413\\
26	0.00433447434343657\\
27	0.00433447810999261\\
28	0.00433448194596979\\
29	0.00433448585265188\\
30	0.00433448983134642\\
31	0.0043344938833852\\
32	0.00433449801012487\\
33	0.00433450221294707\\
34	0.00433450649325921\\
35	0.00433451085249485\\
36	0.00433451529211415\\
37	0.00433451981360434\\
38	0.00433452441848027\\
39	0.00433452910828499\\
40	0.00433453388459013\\
41	0.00433453874899659\\
42	0.00433454370313495\\
43	0.00433454874866611\\
44	0.00433455388728183\\
45	0.00433455912070522\\
46	0.00433456445069154\\
47	0.00433456987902862\\
48	0.00433457540753754\\
49	0.00433458103807315\\
50	0.00433458677252486\\
51	0.00433459261281712\\
52	0.00433459856091015\\
53	0.0043346046188007\\
54	0.00433461078852263\\
55	0.00433461707214761\\
56	0.00433462347178577\\
57	0.00433462998958654\\
58	0.00433463662773934\\
59	0.0043346433884743\\
60	0.00433465027406297\\
61	0.00433465728681925\\
62	0.00433466442910008\\
63	0.00433467170330617\\
64	0.00433467911188301\\
65	0.00433468665732144\\
66	0.00433469434215876\\
67	0.00433470216897956\\
68	0.00433471014041645\\
69	0.00433471825915103\\
70	0.00433472652791482\\
71	0.00433473494949035\\
72	0.00433474352671167\\
73	0.00433475226246583\\
74	0.0043347611596935\\
75	0.00433477022139027\\
76	0.00433477945060735\\
77	0.00433478885045292\\
78	0.00433479842409299\\
79	0.00433480817475246\\
80	0.00433481810571649\\
81	0.00433482822033137\\
82	0.00433483852200576\\
83	0.00433484901421175\\
84	0.00433485970048622\\
85	0.00433487058443192\\
86	0.00433488166971885\\
87	0.00433489296008528\\
88	0.00433490445933922\\
89	0.00433491617135971\\
90	0.00433492810009815\\
91	0.00433494024957949\\
92	0.00433495262390377\\
93	0.00433496522724763\\
94	0.0043349780638655\\
95	0.00433499113809108\\
96	0.00433500445433906\\
97	0.00433501801710641\\
98	0.00433503183097401\\
99	0.00433504590060823\\
100	0.00433506023076251\\
101	0.00433507482627893\\
102	0.00433508969209002\\
103	0.0043351048332203\\
104	0.00433512025478811\\
105	0.00433513596200738\\
106	0.00433515196018933\\
107	0.00433516825474427\\
108	0.00433518485118366\\
109	0.00433520175512177\\
110	0.00433521897227784\\
111	0.00433523650847781\\
112	0.00433525436965644\\
113	0.00433527256185936\\
114	0.00433529109124517\\
115	0.00433530996408747\\
116	0.00433532918677717\\
117	0.00433534876582449\\
118	0.00433536870786138\\
119	0.00433538901964363\\
120	0.00433540970805338\\
121	0.00433543078010148\\
122	0.00433545224292965\\
123	0.00433547410381325\\
124	0.0043354963701636\\
125	0.00433551904953057\\
126	0.00433554214960529\\
127	0.00433556567822269\\
128	0.00433558964336422\\
129	0.00433561405316057\\
130	0.00433563891589457\\
131	0.00433566424000396\\
132	0.00433569003408434\\
133	0.00433571630689219\\
134	0.00433574306734774\\
135	0.00433577032453824\\
136	0.00433579808772094\\
137	0.00433582636632634\\
138	0.00433585516996155\\
139	0.00433588450841341\\
140	0.00433591439165201\\
141	0.00433594482983415\\
142	0.00433597583330677\\
143	0.00433600741261058\\
144	0.00433603957848358\\
145	0.00433607234186507\\
146	0.0043361057138991\\
147	0.00433613970593858\\
148	0.00433617432954894\\
149	0.00433620959651243\\
150	0.00433624551883205\\
151	0.00433628210873568\\
152	0.00433631937868039\\
153	0.00433635734135661\\
154	0.00433639600969271\\
155	0.0043364353968595\\
156	0.00433647551627451\\
157	0.0043365163816069\\
158	0.00433655800678218\\
159	0.00433660040598687\\
160	0.00433664359367367\\
161	0.00433668758456619\\
162	0.00433673239366435\\
163	0.00433677803624933\\
164	0.00433682452788904\\
165	0.00433687188444338\\
166	0.00433692012206996\\
167	0.00433696925722949\\
168	0.00433701930669161\\
169	0.00433707028754067\\
170	0.00433712221718179\\
171	0.0043371751133467\\
172	0.00433722899410006\\
173	0.00433728387784567\\
174	0.00433733978333297\\
175	0.00433739672966335\\
176	0.00433745473629706\\
177	0.00433751382305975\\
178	0.00433757401014956\\
179	0.00433763531814403\\
180	0.00433769776800725\\
181	0.00433776138109724\\
182	0.00433782617917334\\
183	0.00433789218440382\\
184	0.00433795941937356\\
185	0.00433802790709201\\
186	0.00433809767100114\\
187	0.00433816873498355\\
188	0.00433824112337095\\
189	0.00433831486095252\\
190	0.00433838997298366\\
191	0.00433846648519474\\
192	0.00433854442380009\\
193	0.00433862381550717\\
194	0.00433870468752586\\
195	0.00433878706757803\\
196	0.00433887098390719\\
197	0.00433895646528827\\
198	0.00433904354103794\\
199	0.00433913224102455\\
200	0.0043392225956787\\
201	0.00433931463600408\\
202	0.00433940839358797\\
203	0.0043395039006125\\
204	0.00433960118986595\\
205	0.00433970029475408\\
206	0.00433980124931195\\
207	0.00433990408821587\\
208	0.00434000884679544\\
209	0.004340115561046\\
210	0.00434022426764125\\
211	0.00434033500394618\\
212	0.00434044780802998\\
213	0.0043405627186797\\
214	0.00434067977541363\\
215	0.00434079901849514\\
216	0.00434092048894711\\
217	0.00434104422856615\\
218	0.00434117027993722\\
219	0.00434129868644876\\
220	0.00434142949230795\\
221	0.00434156274255614\\
222	0.00434169848308484\\
223	0.00434183676065168\\
224	0.00434197762289716\\
225	0.00434212111836126\\
226	0.00434226729650053\\
227	0.0043424162077056\\
228	0.00434256790331901\\
229	0.00434272243565309\\
230	0.00434287985800876\\
231	0.00434304022469402\\
232	0.00434320359104341\\
233	0.00434337001343736\\
234	0.00434353954932225\\
235	0.00434371225723068\\
236	0.00434388819680213\\
237	0.00434406742880416\\
238	0.00434425001515388\\
239	0.00434443601893982\\
240	0.00434462550444436\\
241	0.00434481853716655\\
242	0.00434501518384526\\
243	0.00434521551248277\\
244	0.00434541959236909\\
245	0.00434562749410651\\
246	0.00434583928963452\\
247	0.00434605505225551\\
248	0.00434627485666074\\
249	0.00434649877895702\\
250	0.00434672689669355\\
251	0.00434695928888968\\
252	0.0043471960360629\\
253	0.00434743722025764\\
254	0.00434768292507426\\
255	0.00434793323569891\\
256	0.004348188238934\\
257	0.00434844802322883\\
258	0.00434871267871118\\
259	0.00434898229721944\\
260	0.00434925697233537\\
261	0.00434953679941713\\
262	0.00434982187563352\\
263	0.00435011229999835\\
264	0.00435040817340566\\
265	0.00435070959866558\\
266	0.00435101668054085\\
267	0.00435132952578396\\
268	0.00435164824317491\\
269	0.00435197294355986\\
270	0.0043523037398901\\
271	0.00435264074726211\\
272	0.00435298408295787\\
273	0.00435333386648616\\
274	0.00435369021962445\\
275	0.00435405326646142\\
276	0.00435442313344009\\
277	0.00435479994940184\\
278	0.00435518384563063\\
279	0.00435557495589837\\
280	0.00435597341651037\\
281	0.00435637936635177\\
282	0.00435679294693444\\
283	0.00435721430244384\\
284	0.00435764357978722\\
285	0.00435808092864142\\
286	0.00435852650150141\\
287	0.00435898045372909\\
288	0.004359442943602\\
289	0.00435991413236213\\
290	0.00436039418426491\\
291	0.00436088326662764\\
292	0.0043613815498779\\
293	0.00436188920760098\\
294	0.00436240641658698\\
295	0.00436293335687661\\
296	0.0043634702118058\\
297	0.00436401716804851\\
298	0.00436457441565798\\
299	0.00436514214810526\\
300	0.00436572056231491\\
301	0.00436630985869759\\
302	0.00436691024117822\\
303	0.00436752191721976\\
304	0.00436814509784179\\
305	0.00436877999763203\\
306	0.00436942683475128\\
307	0.00437008583092947\\
308	0.00437075721145214\\
309	0.00437144120513579\\
310	0.00437213804429049\\
311	0.00437284796466764\\
312	0.0043735712053918\\
313	0.00437430800887281\\
314	0.00437505862069719\\
315	0.00437582328949509\\
316	0.00437660226677961\\
317	0.00437739580675559\\
318	0.00437820416609351\\
319	0.00437902760366431\\
320	0.00437986638023082\\
321	0.00438072075809017\\
322	0.0043815910006622\\
323	0.00438247737201782\\
324	0.00438338013634087\\
325	0.00438429955731751\\
326	0.00438523589744583\\
327	0.00438618941725928\\
328	0.00438716037445751\\
329	0.0043881490229384\\
330	0.0043891556117265\\
331	0.00439018038379473\\
332	0.00439122357477812\\
333	0.00439228541158204\\
334	0.00439336611089349\\
335	0.00439446587760887\\
336	0.00439558490320252\\
337	0.00439672336407297\\
338	0.00439788141991926\\
339	0.00439905921222226\\
340	0.00440025686292988\\
341	0.0044014744734705\\
342	0.00440271212422077\\
343	0.00440396987448398\\
344	0.00440524776275998\\
345	0.00440654580531069\\
346	0.00440786398489007\\
347	0.00440920402385365\\
348	0.00441057060935592\\
349	0.00441196426536201\\
350	0.00441338552577781\\
351	0.00441483493460724\\
352	0.00441631304610621\\
353	0.00441782042492295\\
354	0.0044193576462028\\
355	0.00442092529565036\\
356	0.00442252396978652\\
357	0.00442415427627943\\
358	0.00442581683414257\\
359	0.00442751227392775\\
360	0.00442924123792482\\
361	0.00443100438036846\\
362	0.00443280236765423\\
363	0.00443463587856507\\
364	0.00443650560451036\\
365	0.00443841224977974\\
366	0.00444035653181446\\
367	0.00444233918149978\\
368	0.00444436094348195\\
369	0.00444642257651406\\
370	0.00444852485383648\\
371	0.00445066856359745\\
372	0.00445285450932144\\
373	0.00445508351043292\\
374	0.00445735640284518\\
375	0.00445967403962522\\
376	0.00446203729174694\\
377	0.00446444704894753\\
378	0.00446690422070181\\
379	0.00446940973733432\\
380	0.00447196455128907\\
381	0.00447456963857968\\
382	0.00447722600044551\\
383	0.00447993466524188\\
384	0.0044826966905934\\
385	0.00448551316584321\\
386	0.0044883852148302\\
387	0.00449131399902813\\
388	0.00449430072107765\\
389	0.00449734662873975\\
390	0.00450045301929358\\
391	0.00450362124439519\\
392	0.00450685271541414\\
393	0.00451014890930527\\
394	0.00451351137525375\\
395	0.00451694174302952\\
396	0.0045204417364464\\
397	0.00452401320375389\\
398	0.00452765820524174\\
399	0.0045313789886601\\
400	0.00453517753936264\\
401	0.00453905557873385\\
402	0.0045430148741424\\
403	0.00454705724075805\\
404	0.00455118454346668\\
405	0.00455539869888527\\
406	0.00455970167748078\\
407	0.00456409550579306\\
408	0.00456858226876349\\
409	0.00457316411217185\\
410	0.00457784324519672\\
411	0.00458262194310348\\
412	0.00458750255001831\\
413	0.00459248748176176\\
414	0.00459757922875969\\
415	0.00460278035902185\\
416	0.00460809352116916\\
417	0.00461352144748655\\
418	0.00461906695697717\\
419	0.00462473295838905\\
420	0.00463052245318885\\
421	0.00463643853847783\\
422	0.0046424844098883\\
423	0.00464866336458066\\
424	0.00465497880455615\\
425	0.00466143424044806\\
426	0.00466803329447707\\
427	0.00467477969251494\\
428	0.00468167725361806\\
429	0.0046887299093657\\
430	0.00469594173033701\\
431	0.00470331693310338\\
432	0.00471085988765404\\
433	0.00471857512528687\\
434	0.00472646734699674\\
435	0.00473454143239654\\
436	0.00474280244921074\\
437	0.00475125566338333\\
438	0.00475990654984844\\
439	0.00476876080401437\\
440	0.00477782435402021\\
441	0.00478710337383024\\
442	0.00479660429724071\\
443	0.0048063338328793\\
444	0.00481629898027238\\
445	0.00482650704700604\\
446	0.00483696566688477\\
447	0.0048476828189708\\
448	0.0048586668488317\\
449	0.0048699264922354\\
450	0.00488147089918492\\
451	0.00489330965865071\\
452	0.00490545282471894\\
453	0.00491791094418666\\
454	0.00493069508562214\\
455	0.0049438168698928\\
456	0.00495728850214065\\
457	0.00497112280515963\\
458	0.00498533325409378\\
459	0.0049999340123333\\
460	0.00501493996843038\\
461	0.0050303667737904\\
462	0.00504623088081715\\
463	0.00506254958112901\\
464	0.00507934104342271\\
465	0.00509662435029209\\
466	0.0051144195332361\\
467	0.00513274760527605\\
468	0.00515163059078765\\
469	0.00517109155320189\\
470	0.00519115462429839\\
471	0.00521184504677305\\
472	0.00523318926021573\\
473	0.00525521509555144\\
474	0.00527795217275313\\
475	0.00530143265222818\\
476	0.00532569914663003\\
477	0.00535080864374956\\
478	0.00537679809147539\\
479	0.00540370560503476\\
480	0.00543157784115154\\
481	0.00546046220442261\\
482	0.00549040667370869\\
483	0.00552145904822377\\
484	0.00555366580937486\\
485	0.00558704762300578\\
486	0.00562138769391787\\
487	0.0056566986408806\\
488	0.00569298744985147\\
489	0.00573025396980975\\
490	0.00576848877383418\\
491	0.00580766931668448\\
492	0.00584796783373438\\
493	0.00588943448996365\\
494	0.0059320294334912\\
495	0.00597568959241816\\
496	0.00602032355118579\\
497	0.00606580644661143\\
498	0.00611196917249998\\
499	0.0061585824520033\\
500	0.00620541096103591\\
501	0.0062520924216836\\
502	0.00629802034333257\\
503	0.00634147450349103\\
504	0.00637971149189926\\
505	0.00641262592420366\\
506	0.00644048029382127\\
507	0.00646574725732371\\
508	0.00649010650081638\\
509	0.00651399622170918\\
510	0.00653789763179205\\
511	0.0065620801181893\\
512	0.00658665536465146\\
513	0.00661169510902583\\
514	0.006637239803336\\
515	0.00666331249094905\\
516	0.00668993163021992\\
517	0.00671711228438945\\
518	0.00674486803670351\\
519	0.00677321216013866\\
520	0.0068021579289145\\
521	0.00683171890503599\\
522	0.00686190912800818\\
523	0.00689274324833507\\
524	0.00692423666951791\\
525	0.00695640570141792\\
526	0.00698926773593825\\
527	0.00702284145500003\\
528	0.0070571470756616\\
529	0.00709220664141625\\
530	0.00712804440069717\\
531	0.00716472854050527\\
532	0.00720234880295647\\
533	0.00723985893619002\\
534	0.00727623466988971\\
535	0.00731151311131741\\
536	0.00734719353934899\\
537	0.0073833556106261\\
538	0.00742008300313997\\
539	0.00745740535755315\\
540	0.00749532669484875\\
541	0.00753384645106589\\
542	0.00757296262261381\\
543	0.00761267172446651\\
544	0.00765296869609351\\
545	0.00769384670138497\\
546	0.00773529689100766\\
547	0.00777730812572076\\
548	0.00781986665437991\\
549	0.00786295573820756\\
550	0.00790655521515773\\
551	0.00795064102098\\
552	0.00799518479699153\\
553	0.0080401859272667\\
554	0.00808464783813621\\
555	0.00812773859917388\\
556	0.00817111836314038\\
557	0.00821486580995575\\
558	0.00825897791658296\\
559	0.00830343402391313\\
560	0.00834821243243883\\
561	0.00839329070202066\\
562	0.00843864597211353\\
563	0.00848425526709758\\
564	0.00853009920226463\\
565	0.00857589049842109\\
566	0.00862164036385014\\
567	0.0086676593810865\\
568	0.00871393573141297\\
569	0.00876044307752625\\
570	0.00880715274930517\\
571	0.00885403364441827\\
572	0.00890105214171656\\
573	0.00894817203305216\\
574	0.00899535448063906\\
575	0.00904255800890195\\
576	0.00908973854201727\\
577	0.00913684950113329\\
578	0.00918384197868252\\
579	0.00923066501141119\\
580	0.00927726597891579\\
581	0.00932359116078638\\
582	0.00936958649310202\\
583	0.00941519857412798\\
584	0.00946037597948012\\
585	0.00950507095777016\\
586	0.00954924158550055\\
587	0.00959285445497631\\
588	0.00963588792409171\\
589	0.00967833579817709\\
590	0.00972021084271539\\
591	0.00976138933978722\\
592	0.00980171928775089\\
593	0.00984102135019415\\
594	0.00987905595166784\\
595	0.00991543381058343\\
596	0.00994937493726701\\
597	0.00997906286423442\\
598	0.0099999191923403\\
599	0\\
600	0\\
};
\addplot [color=black,solid,forget plot]
  table[row sep=crcr]{%
1	0.00434222595919898\\
2	0.00434222821860981\\
3	0.00434223051966135\\
4	0.00434223286312353\\
5	0.00434223524978059\\
6	0.00434223768043138\\
7	0.00434224015588943\\
8	0.00434224267698356\\
9	0.00434224524455786\\
10	0.00434224785947209\\
11	0.00434225052260194\\
12	0.00434225323483943\\
13	0.00434225599709309\\
14	0.00434225881028843\\
15	0.00434226167536802\\
16	0.00434226459329199\\
17	0.00434226756503826\\
18	0.00434227059160297\\
19	0.00434227367400076\\
20	0.00434227681326514\\
21	0.00434228001044872\\
22	0.00434228326662375\\
23	0.00434228658288232\\
24	0.00434228996033694\\
25	0.00434229340012076\\
26	0.00434229690338792\\
27	0.00434230047131408\\
28	0.0043423041050967\\
29	0.00434230780595548\\
30	0.0043423115751329\\
31	0.00434231541389449\\
32	0.0043423193235292\\
33	0.00434232330535016\\
34	0.00434232736069477\\
35	0.00434233149092533\\
36	0.00434233569742945\\
37	0.00434233998162055\\
38	0.00434234434493852\\
39	0.00434234878884976\\
40	0.00434235331484817\\
41	0.00434235792445522\\
42	0.00434236261922094\\
43	0.00434236740072407\\
44	0.00434237227057269\\
45	0.00434237723040493\\
46	0.00434238228188933\\
47	0.00434238742672547\\
48	0.00434239266664463\\
49	0.00434239800341021\\
50	0.00434240343881856\\
51	0.00434240897469947\\
52	0.00434241461291672\\
53	0.00434242035536886\\
54	0.00434242620398974\\
55	0.00434243216074926\\
56	0.00434243822765407\\
57	0.00434244440674809\\
58	0.00434245070011332\\
59	0.00434245710987066\\
60	0.00434246363818046\\
61	0.00434247028724327\\
62	0.00434247705930069\\
63	0.00434248395663615\\
64	0.00434249098157545\\
65	0.00434249813648803\\
66	0.00434250542378723\\
67	0.00434251284593148\\
68	0.00434252040542502\\
69	0.00434252810481881\\
70	0.00434253594671141\\
71	0.00434254393374964\\
72	0.00434255206862989\\
73	0.00434256035409873\\
74	0.00434256879295399\\
75	0.00434257738804563\\
76	0.00434258614227683\\
77	0.0043425950586049\\
78	0.00434260414004231\\
79	0.00434261338965775\\
80	0.00434262281057715\\
81	0.00434263240598475\\
82	0.00434264217912425\\
83	0.00434265213329983\\
84	0.00434266227187737\\
85	0.00434267259828563\\
86	0.00434268311601725\\
87	0.00434269382863009\\
88	0.00434270473974859\\
89	0.00434271585306465\\
90	0.00434272717233917\\
91	0.00434273870140327\\
92	0.00434275044415967\\
93	0.00434276240458394\\
94	0.00434277458672585\\
95	0.00434278699471084\\
96	0.00434279963274138\\
97	0.00434281250509854\\
98	0.00434282561614324\\
99	0.00434283897031795\\
100	0.00434285257214809\\
101	0.00434286642624372\\
102	0.00434288053730087\\
103	0.00434289491010356\\
104	0.00434290954952511\\
105	0.00434292446052985\\
106	0.00434293964817495\\
107	0.00434295511761211\\
108	0.00434297087408942\\
109	0.00434298692295295\\
110	0.00434300326964874\\
111	0.00434301991972472\\
112	0.00434303687883253\\
113	0.00434305415272947\\
114	0.00434307174728058\\
115	0.00434308966846045\\
116	0.00434310792235553\\
117	0.00434312651516604\\
118	0.00434314545320822\\
119	0.00434316474291647\\
120	0.00434318439084554\\
121	0.00434320440367275\\
122	0.00434322478820046\\
123	0.00434324555135824\\
124	0.00434326670020539\\
125	0.00434328824193327\\
126	0.00434331018386777\\
127	0.00434333253347188\\
128	0.00434335529834836\\
129	0.00434337848624228\\
130	0.0043434021050436\\
131	0.00434342616279007\\
132	0.00434345066766977\\
133	0.00434347562802419\\
134	0.00434350105235097\\
135	0.0043435269493068\\
136	0.00434355332771052\\
137	0.00434358019654606\\
138	0.0043436075649656\\
139	0.00434363544229271\\
140	0.0043436638380256\\
141	0.00434369276184034\\
142	0.00434372222359422\\
143	0.00434375223332918\\
144	0.00434378280127535\\
145	0.00434381393785427\\
146	0.00434384565368292\\
147	0.00434387795957702\\
148	0.00434391086655505\\
149	0.00434394438584184\\
150	0.00434397852887255\\
151	0.00434401330729659\\
152	0.00434404873298164\\
153	0.00434408481801783\\
154	0.00434412157472182\\
155	0.00434415901564095\\
156	0.00434419715355784\\
157	0.00434423600149462\\
158	0.00434427557271744\\
159	0.00434431588074112\\
160	0.00434435693933373\\
161	0.00434439876252149\\
162	0.00434444136459342\\
163	0.00434448476010633\\
164	0.00434452896389\\
165	0.00434457399105212\\
166	0.00434461985698352\\
167	0.00434466657736367\\
168	0.00434471416816581\\
169	0.00434476264566271\\
170	0.00434481202643204\\
171	0.0043448623273624\\
172	0.00434491356565887\\
173	0.00434496575884917\\
174	0.00434501892478946\\
175	0.00434507308167075\\
176	0.00434512824802498\\
177	0.00434518444273155\\
178	0.00434524168502374\\
179	0.00434529999449536\\
180	0.00434535939110761\\
181	0.00434541989519574\\
182	0.00434548152747634\\
183	0.00434554430905421\\
184	0.00434560826142982\\
185	0.00434567340650657\\
186	0.00434573976659853\\
187	0.00434580736443798\\
188	0.00434587622318318\\
189	0.00434594636642652\\
190	0.00434601781820248\\
191	0.00434609060299602\\
192	0.00434616474575091\\
193	0.00434624027187832\\
194	0.00434631720726564\\
195	0.00434639557828524\\
196	0.00434647541180373\\
197	0.00434655673519102\\
198	0.00434663957632959\\
199	0.00434672396362446\\
200	0.00434680992601267\\
201	0.00434689749297314\\
202	0.00434698669453694\\
203	0.00434707756129756\\
204	0.00434717012442129\\
205	0.00434726441565799\\
206	0.004347360467352\\
207	0.00434745831245296\\
208	0.00434755798452739\\
209	0.00434765951777001\\
210	0.00434776294701537\\
211	0.00434786830774987\\
212	0.00434797563612374\\
213	0.0043480849689634\\
214	0.00434819634378422\\
215	0.00434830979880302\\
216	0.00434842537295115\\
217	0.00434854310588775\\
218	0.00434866303801332\\
219	0.00434878521048335\\
220	0.0043489096652223\\
221	0.00434903644493781\\
222	0.00434916559313522\\
223	0.00434929715413229\\
224	0.00434943117307416\\
225	0.00434956769594852\\
226	0.00434970676960141\\
227	0.00434984844175275\\
228	0.0043499927610126\\
229	0.00435013977689738\\
230	0.0043502895398467\\
231	0.00435044210124015\\
232	0.00435059751341458\\
233	0.00435075582968171\\
234	0.00435091710434584\\
235	0.00435108139272219\\
236	0.00435124875115522\\
237	0.00435141923703743\\
238	0.00435159290882859\\
239	0.00435176982607499\\
240	0.00435195004942941\\
241	0.00435213364067103\\
242	0.00435232066272596\\
243	0.00435251117968812\\
244	0.0043527052568402\\
245	0.00435290296067521\\
246	0.00435310435891838\\
247	0.00435330952054942\\
248	0.00435351851582486\\
249	0.00435373141630114\\
250	0.00435394829485812\\
251	0.00435416922572236\\
252	0.0043543942844916\\
253	0.0043546235481589\\
254	0.00435485709513777\\
255	0.00435509500528721\\
256	0.00435533735993752\\
257	0.0043555842419162\\
258	0.00435583573557455\\
259	0.0043560919268145\\
260	0.00435635290311574\\
261	0.00435661875356367\\
262	0.00435688956887727\\
263	0.00435716544143785\\
264	0.0043574464653178\\
265	0.00435773273631016\\
266	0.00435802435195835\\
267	0.00435832141158653\\
268	0.00435862401633014\\
269	0.00435893226916712\\
270	0.00435924627494958\\
271	0.00435956614043557\\
272	0.00435989197432189\\
273	0.00436022388727688\\
274	0.00436056199197369\\
275	0.00436090640312437\\
276	0.00436125723751402\\
277	0.00436161461403568\\
278	0.00436197865372554\\
279	0.00436234947979873\\
280	0.00436272721768567\\
281	0.00436311199506868\\
282	0.00436350394191926\\
283	0.00436390319053608\\
284	0.00436430987558299\\
285	0.00436472413412836\\
286	0.00436514610568424\\
287	0.00436557593224635\\
288	0.00436601375833513\\
289	0.00436645973103691\\
290	0.00436691400004596\\
291	0.00436737671770755\\
292	0.00436784803906134\\
293	0.00436832812188603\\
294	0.0043688171267447\\
295	0.00436931521703131\\
296	0.00436982255901841\\
297	0.00437033932190591\\
298	0.00437086567787115\\
299	0.00437140180212078\\
300	0.00437194787294426\\
301	0.00437250407176894\\
302	0.00437307058321792\\
303	0.00437364759516981\\
304	0.00437423529882131\\
305	0.00437483388875333\\
306	0.0043754435630001\\
307	0.00437606452312268\\
308	0.00437669697428681\\
309	0.00437734112534621\\
310	0.00437799718893176\\
311	0.00437866538154727\\
312	0.0043793459236733\\
313	0.00438003903987961\\
314	0.00438074495894785\\
315	0.00438146391400548\\
316	0.00438219614267317\\
317	0.0043829418872268\\
318	0.00438370139477653\\
319	0.00438447491746537\\
320	0.00438526271268933\\
321	0.00438606504334291\\
322	0.00438688217809268\\
323	0.00438771439168233\\
324	0.0043885619652747\\
325	0.00438942518683345\\
326	0.00439030435155112\\
327	0.004391199762328\\
328	0.00439211173030835\\
329	0.00439304057548008\\
330	0.00439398662734469\\
331	0.00439495022566533\\
332	0.00439593172129862\\
333	0.00439693147711934\\
334	0.00439794986904292\\
335	0.00439898728715272\\
336	0.00440004413693626\\
337	0.00440112084063253\\
338	0.00440221783869206\\
339	0.00440333559135105\\
340	0.0044044745803329\\
341	0.00440563531073312\\
342	0.00440681831329135\\
343	0.00440802414769152\\
344	0.00440925340879721\\
345	0.00441050674127023\\
346	0.00441178487748705\\
347	0.00441308861159611\\
348	0.00441441859668612\\
349	0.00441577537239013\\
350	0.00441715949007091\\
351	0.00441857151312482\\
352	0.00442001201729709\\
353	0.00442148159100886\\
354	0.00442298083569901\\
355	0.00442451036618554\\
356	0.00442607081104137\\
357	0.00442766281297806\\
358	0.00442928702924466\\
359	0.00443094413204339\\
360	0.00443263480896246\\
361	0.00443435976342743\\
362	0.00443611971517158\\
363	0.00443791540072674\\
364	0.00443974757393486\\
365	0.00444161700648187\\
366	0.00444352448845459\\
367	0.00444547082892134\\
368	0.0044474568565372\\
369	0.0044494834201751\\
370	0.00445155138958286\\
371	0.00445366165606704\\
372	0.00445581513320393\\
373	0.00445801275757793\\
374	0.00446025548954678\\
375	0.00446254431403404\\
376	0.00446488024134691\\
377	0.00446726430801856\\
378	0.00446969757767323\\
379	0.00447218114191075\\
380	0.00447471612120721\\
381	0.00447730366582724\\
382	0.00447994495674243\\
383	0.0044826412065479\\
384	0.00448539366036974\\
385	0.00448820359675243\\
386	0.0044910723285143\\
387	0.00449400120355786\\
388	0.00449699160561887\\
389	0.00450004495493866\\
390	0.00450316270884218\\
391	0.00450634636220972\\
392	0.00450959744784129\\
393	0.00451291753673766\\
394	0.00451630823837839\\
395	0.00451977120114798\\
396	0.00452330811301538\\
397	0.00452692070170608\\
398	0.00453061072922442\\
399	0.00453437998069945\\
400	0.00453823027231605\\
401	0.00454216346841403\\
402	0.00454618148311681\\
403	0.00455028628203201\\
404	0.00455447988402687\\
405	0.00455876436308343\\
406	0.00456314185023713\\
407	0.004567614535605\\
408	0.00457218467050823\\
409	0.00457685456969654\\
410	0.00458162661367982\\
411	0.00458650325117443\\
412	0.0045914870016723\\
413	0.00459658045814428\\
414	0.00460178628988913\\
415	0.0046071072455413\\
416	0.00461254615625144\\
417	0.00461810593905845\\
418	0.00462378960047169\\
419	0.0046296002402883\\
420	0.00463554105567392\\
421	0.00464161534554071\\
422	0.0046478265152618\\
423	0.00465417808175562\\
424	0.00466067367894573\\
425	0.00466731706352769\\
426	0.00467411212094043\\
427	0.00468106287215399\\
428	0.00468817348186214\\
429	0.00469544826687676\\
430	0.00470289170423734\\
431	0.00471050843987783\\
432	0.00471830329789414\\
433	0.00472628129045722\\
434	0.00473444762842076\\
435	0.00474280773267571\\
436	0.00475136724630788\\
437	0.00476013204761824\\
438	0.00476910826406953\\
439	0.00477830228722649\\
440	0.0047877207887615\\
441	0.00479737073760044\\
442	0.00480725941828665\\
443	0.00481739445064306\\
444	0.00482778381081094\\
445	0.00483843585374646\\
446	0.00484935933727037\\
447	0.00486056344780547\\
448	0.00487205782786032\\
449	0.00488385260528553\\
450	0.00489595842442378\\
451	0.00490838647926189\\
452	0.00492114854865773\\
453	0.00493425703369915\\
454	0.00494772499723305\\
455	0.00496156620557198\\
456	0.00497579517234968\\
457	0.00499042720444371\\
458	0.00500547844981828\\
459	0.00502096594705182\\
460	0.005036907676203\\
461	0.00505332261052375\\
462	0.00507023076834851\\
463	0.00508765326425881\\
464	0.00510561235833505\\
465	0.00512413150199129\\
466	0.00514323537859757\\
467	0.0051629499370455\\
468	0.00518330241737538\\
469	0.00520432137194165\\
470	0.00522603670084335\\
471	0.0052484797701117\\
472	0.00527168384025851\\
473	0.00529568554045985\\
474	0.00532052975210367\\
475	0.00534628267470183\\
476	0.00537298517324241\\
477	0.00540067823671829\\
478	0.005429401877641\\
479	0.00545919466390937\\
480	0.00548981733005028\\
481	0.00552129955298564\\
482	0.00555367068432858\\
483	0.00558695979453451\\
484	0.00562119644582296\\
485	0.00565643318442142\\
486	0.00569293098925691\\
487	0.00573071750642869\\
488	0.00576981032505155\\
489	0.00581021341315327\\
490	0.00585191298560103\\
491	0.00589487354705255\\
492	0.00593902316472585\\
493	0.00598423743188394\\
494	0.00603033170296704\\
495	0.00607704681967084\\
496	0.00612402716256964\\
497	0.0061707902296417\\
498	0.00621667981092483\\
499	0.00626077931899491\\
500	0.00629986903605491\\
501	0.00633321627267928\\
502	0.00636114201732931\\
503	0.00638521509377619\\
504	0.00640821620985371\\
505	0.00643059378635421\\
506	0.00645283186455397\\
507	0.00647526889657669\\
508	0.00649805406722564\\
509	0.0065212637067607\\
510	0.00654494242954996\\
511	0.00656911485166403\\
512	0.00659379925237844\\
513	0.00661901035700388\\
514	0.00664476133373063\\
515	0.0066710650495984\\
516	0.00669793437851615\\
517	0.0067253824797126\\
518	0.00675342297778298\\
519	0.00678207008594943\\
520	0.00681133873692882\\
521	0.00684124472623264\\
522	0.00687180488058706\\
523	0.00690303726415001\\
524	0.00693496143280232\\
525	0.00696759874906755\\
526	0.00700097277136572\\
527	0.00703510972972449\\
528	0.00707003908897031\\
529	0.00710579415587524\\
530	0.00714241253002996\\
531	0.00717823387096147\\
532	0.00721273677147728\\
533	0.00724682109888564\\
534	0.00728132396461994\\
535	0.00731634289462122\\
536	0.00735194707806507\\
537	0.00738814406559645\\
538	0.00742493702546428\\
539	0.00746232678336369\\
540	0.00750031295800742\\
541	0.00753889396747831\\
542	0.00757806687614832\\
543	0.00761782721299181\\
544	0.00765816875730519\\
545	0.00769908328917873\\
546	0.00774056029745439\\
547	0.00778258663495929\\
548	0.00782514610529904\\
549	0.00786821895268939\\
550	0.00791178119219577\\
551	0.00795580361922691\\
552	0.00800025003764972\\
553	0.00804363618546738\\
554	0.00808620366337529\\
555	0.00812910677813416\\
556	0.00817240401823542\\
557	0.00821607813701697\\
558	0.00826010969053816\\
559	0.00830447822831884\\
560	0.0083491625665264\\
561	0.0083941411104164\\
562	0.00843939212066833\\
563	0.00848489356497313\\
564	0.00853045416626427\\
565	0.00857590279196591\\
566	0.0086216403638569\\
567	0.00866765938108735\\
568	0.00871393573141338\\
569	0.00876044307752645\\
570	0.00880715274930527\\
571	0.00885403364441831\\
572	0.00890105214171657\\
573	0.00894817203305217\\
574	0.00899535448063906\\
575	0.00904255800890197\\
576	0.00908973854201728\\
577	0.00913684950113329\\
578	0.00918384197868253\\
579	0.00923066501141121\\
580	0.00927726597891581\\
581	0.00932359116078639\\
582	0.00936958649310202\\
583	0.00941519857412798\\
584	0.00946037597948012\\
585	0.00950507095777015\\
586	0.00954924158550055\\
587	0.00959285445497631\\
588	0.00963588792409171\\
589	0.00967833579817709\\
590	0.00972021084271539\\
591	0.00976138933978722\\
592	0.00980171928775089\\
593	0.00984102135019415\\
594	0.00987905595166784\\
595	0.00991543381058343\\
596	0.00994937493726701\\
597	0.00997906286423442\\
598	0.0099999191923403\\
599	0\\
600	0\\
};
\end{axis}
\end{tikzpicture}% 
%  \caption{Discrete Time}
%\end{subfigure}\\
%\vspace{1cm}
%\begin{subfigure}{.45\linewidth}
%  \centering
%  \setlength\figureheight{\linewidth} 
%  \setlength\figurewidth{\linewidth}
%  \tikzsetnextfilename{dp_cts_nFPC_z8}
%  % This file was created by matlab2tikz.
%
%The latest updates can be retrieved from
%  http://www.mathworks.com/matlabcentral/fileexchange/22022-matlab2tikz-matlab2tikz
%where you can also make suggestions and rate matlab2tikz.
%
\definecolor{mycolor1}{rgb}{1.00000,0.00000,1.00000}%
%
\begin{tikzpicture}[trim axis left, trim axis right]

\begin{axis}[%
width=\figurewidth,
height=\figureheight,
at={(0\figurewidth,0\figureheight)},
scale only axis,
every outer x axis line/.append style={black},
every x tick label/.append style={font=\color{black}},
xmin=0,
xmax=100,
xlabel={Time},
every outer y axis line/.append style={black},
every y tick label/.append style={font=\color{black}},
ymin=0,
ymax=0.015,
%ylabel={Depth $\delta^-$},
axis background/.style={fill=white},
axis x line*=bottom,
axis y line*=left,
yticklabel style={
        /pgf/number format/fixed,
        /pgf/number format/precision=3
},
scaled y ticks=false,
legend style={legend cell align=left,align=left,draw=black,font=\footnotesize, at={(0.98,0.02)},anchor=south east},
every axis legend/.code={\renewcommand\addlegendentry[2][]{}}  %ignore legend locally
]
\addplot [color=green,dashed]
  table[row sep=crcr]{%
0.01	0.00879904401680774\\
1.01	0.00878916708833785\\
2.01	0.0087787961888384\\
3.01	0.00876790513602132\\
4.01	0.00875646626029204\\
5.01	0.00874445029188437\\
6.01	0.00873182622823632\\
7.01	0.00871856117564894\\
8.01	0.00870462015755683\\
9.01	0.00868996587971709\\
10.01	0.00867455844037774\\
11.01	0.00865835497124509\\
12.01	0.00864130919332668\\
13.01	0.00862337087144434\\
14.01	0.00860448515415128\\
15.01	0.0085845917950726\\
16.01	0.008563624272698\\
17.01	0.00854150886751429\\
18.01	0.00851816383398183\\
19.01	0.00849349895135208\\
20.01	0.00846741536914966\\
21.01	0.00843980510464505\\
22.01	0.00841055053034011\\
23.01	0.00837952414414822\\
24.01	0.00834658885548674\\
25.01	0.00831159914291629\\
26.01	0.00827439088654129\\
27.01	0.00823474865363261\\
28.01	0.00819242029290907\\
29.01	0.00814711174474293\\
30.01	0.00809847572556894\\
31.01	0.008046098121038\\
32.01	0.00798948011169459\\
33.01	0.00792801450311603\\
34.01	0.00786095412805441\\
35.01	0.00778736932341551\\
36.01	0.00770609253118292\\
37.01	0.00761831095217872\\
38.01	0.00752567132623096\\
39.01	0.00742787799461891\\
40.01	0.00732461307718008\\
41.01	0.00721553022557604\\
42.01	0.00710024419862604\\
43.01	0.00697831362580073\\
44.01	0.00684921265854122\\
45.01	0.00671228646638744\\
46.01	0.00656673627739562\\
47.01	0.00641161044007972\\
48.01	0.00624575996949978\\
49.01	0.00611007732321344\\
50.01	0.00601299750052631\\
51.01	0.00591061552152229\\
52.01	0.00580267052463306\\
53.01	0.00568894582868561\\
54.01	0.00556930583078947\\
55.01	0.00544375195560288\\
56.01	0.00531250684963002\\
57.01	0.00517614047360847\\
58.01	0.00503575843718278\\
59.01	0.0048934475279031\\
60.01	0.00475210798939968\\
61.01	0.00461462362621046\\
62.01	0.00448473547667558\\
63.01	0.00435644228817179\\
64.01	0.00422613481601493\\
65.01	0.00409451963672095\\
66.01	0.00396245028281321\\
67.01	0.00383091501052414\\
68.01	0.00370099265560547\\
69.01	0.00357375159677462\\
70.01	0.00344997114751124\\
71.01	0.00333076902272894\\
72.01	0.00321880879105804\\
73.01	0.00310933418509031\\
74.01	0.00300092191586959\\
75.01	0.00289356421969468\\
76.01	0.00278657362826304\\
77.01	0.00267869253183736\\
78.01	0.00256960840121469\\
79.01	0.00245926901255094\\
80.01	0.00234772625485614\\
81.01	0.00223519869647231\\
82.01	0.00212189178782957\\
83.01	0.00200792736130027\\
84.01	0.00189327625168117\\
85.01	0.00177765665873484\\
86.01	0.00166041252095138\\
87.01	0.00154096936677342\\
88.01	0.00141928021253589\\
89.01	0.00129545664766734\\
90.01	0.00116963756502902\\
91.01	0.00104197157482483\\
92.01	0.000912617061273989\\
93.01	0.000781744839962994\\
94.01	0.0006495451184769\\
95.01	0.000516241746188426\\
96.01	0.000382118196883106\\
97.01	0.00024756148145675\\
98.01	0.000113204229188223\\
99.01	2.04676060806437e-05\\
99.02	1.98506175098923e-05\\
99.03	1.92413562266989e-05\\
99.04	1.86398922933494e-05\\
99.05	1.804629641931e-05\\
99.06	1.74606399672653e-05\\
99.07	1.68829949592029e-05\\
99.08	1.6313434082562e-05\\
99.09	1.57520306964323e-05\\
99.1	1.51988588378213e-05\\
99.11	1.46539932279633e-05\\
99.12	1.41175092787053e-05\\
99.13	1.35894830989459e-05\\
99.14	1.30699915011235e-05\\
99.15	1.25591120077907e-05\\
99.16	1.20569228582185e-05\\
99.17	1.15635030150888e-05\\
99.18	1.10789321712389e-05\\
99.19	1.06032981187169e-05\\
99.2	1.01366899886816e-05\\
99.21	9.67919775133295e-06\\
99.22	9.23091222374516e-06\\
99.23	8.79192507778105e-06\\
99.24	8.36232884808032e-06\\
99.25	7.94221694011034e-06\\
99.26	7.53168363830208e-06\\
99.27	7.13082411427438e-06\\
99.28	6.73973443509994e-06\\
99.29	6.3585115716875e-06\\
99.3	5.98725340720911e-06\\
99.31	5.62605874563499e-06\\
99.32	5.27502732032906e-06\\
99.33	4.93425980272778e-06\\
99.34	4.60385781110052e-06\\
99.35	4.28392391940528e-06\\
99.36	3.97456166620347e-06\\
99.37	3.67587556367177e-06\\
99.38	3.38797110669906e-06\\
99.39	3.11095478206652e-06\\
99.4	2.84493407771459e-06\\
99.41	2.59001749209654e-06\\
99.42	2.34631454362755e-06\\
99.43	2.11393578021697e-06\\
99.44	1.89299278887875e-06\\
99.45	1.68359820545798e-06\\
99.46	1.48586572441996e-06\\
99.47	1.29991010874159e-06\\
99.48	1.12584719991031e-06\\
99.49	9.63793927985512e-07\\
99.5	8.13868321781347e-07\\
99.51	6.76189519131093e-07\\
99.52	5.50877777248659e-07\\
99.53	4.38054483194172e-07\\
99.54	3.37842164419358e-07\\
99.55	2.50364499436093e-07\\
99.56	1.7574632856128e-07\\
99.57	1.14113664783158e-07\\
99.58	6.55937047039368e-08\\
99.59	3.03148396090663e-08\\
99.6	8.40666662844936e-09\\
99.61	0\\
99.62	0\\
99.63	0\\
99.64	0\\
99.65	0\\
99.66	0\\
99.67	0\\
99.68	0\\
99.69	0\\
99.7	0\\
99.71	0\\
99.72	0\\
99.73	0\\
99.74	0\\
99.75	0\\
99.76	0\\
99.77	0\\
99.78	0\\
99.79	0\\
99.8	0\\
99.81	0\\
99.82	0\\
99.83	0\\
99.84	0\\
99.85	0\\
99.86	0\\
99.87	0\\
99.88	0\\
99.89	0\\
99.9	0\\
99.91	0\\
99.92	0\\
99.93	0\\
99.94	0\\
99.95	0\\
99.96	0\\
99.97	0\\
99.98	0\\
99.99	0\\
100	0\\
};
\addlegendentry{$q=-4$};

\addplot [color=mycolor1,dashed]
  table[row sep=crcr]{%
0.01	0.01\\
1.01	0.01\\
2.01	0.01\\
3.01	0.01\\
4.01	0.01\\
5.01	0.01\\
6.01	0.01\\
7.01	0.01\\
8.01	0.01\\
9.01	0.01\\
10.01	0.01\\
11.01	0.01\\
12.01	0.01\\
13.01	0.01\\
14.01	0.01\\
15.01	0.01\\
16.01	0.01\\
17.01	0.01\\
18.01	0.01\\
19.01	0.01\\
20.01	0.01\\
21.01	0.01\\
22.01	0.01\\
23.01	0.01\\
24.01	0.01\\
25.01	0.01\\
26.01	0.01\\
27.01	0.01\\
28.01	0.01\\
29.01	0.01\\
30.01	0.01\\
31.01	0.01\\
32.01	0.01\\
33.01	0.01\\
34.01	0.01\\
35.01	0.01\\
36.01	0.01\\
37.01	0.01\\
38.01	0.01\\
39.01	0.01\\
40.01	0.01\\
41.01	0.01\\
42.01	0.01\\
43.01	0.01\\
44.01	0.01\\
45.01	0.01\\
46.01	0.01\\
47.01	0.01\\
48.01	0.01\\
49.01	0.00995806603276508\\
50.01	0.00986679348671141\\
51.01	0.0097696679127076\\
52.01	0.00966610471935848\\
53.01	0.0095554243848595\\
54.01	0.00943682937641714\\
55.01	0.00930937366575347\\
56.01	0.00917192202280119\\
57.01	0.00902309507088274\\
58.01	0.00886119432678425\\
59.01	0.00868409687832849\\
60.01	0.00848910618133506\\
61.01	0.00827281314552082\\
62.01	0.00803123443720172\\
63.01	0.00777023253276142\\
64.01	0.00749272709371373\\
65.01	0.0071972554037051\\
66.01	0.0068822478259438\\
67.01	0.00654605684832105\\
68.01	0.00618702135861113\\
69.01	0.00580360955010393\\
70.01	0.00539472343790348\\
71.01	0.00495878700373076\\
72.01	0.00459234396243271\\
73.01	0.00437901367131237\\
74.01	0.00416850570978012\\
75.01	0.00396572764659518\\
76.01	0.00377778805885348\\
77.01	0.00359727605438995\\
78.01	0.00341468896082242\\
79.01	0.00323034930536865\\
80.01	0.00304353727266926\\
81.01	0.00285483719727066\\
82.01	0.00266617877883341\\
83.01	0.00248016672353796\\
84.01	0.00230031846927878\\
85.01	0.00213137185966049\\
86.01	0.0019771061048414\\
87.01	0.00182511621036325\\
88.01	0.00167085897405416\\
89.01	0.00151419826491386\\
90.01	0.00135570882384954\\
91.01	0.00119608501287654\\
92.01	0.00103612616572766\\
93.01	0.000876729625283948\\
94.01	0.000718872071917943\\
95.01	0.000563572918176667\\
96.01	0.000411831375856974\\
97.01	0.000264532650272513\\
98.01	0.000122320814127108\\
99.01	2.08868531277918e-05\\
99.02	2.02551760214512e-05\\
99.03	1.96315401111164e-05\\
99.04	1.90160129319994e-05\\
99.05	1.84086626507998e-05\\
99.06	1.78095580716676e-05\\
99.07	1.7218768642243e-05\\
99.08	1.66363644597527e-05\\
99.09	1.60624162771558e-05\\
99.1	1.54969955093523e-05\\
99.11	1.49401742394594e-05\\
99.12	1.43920252251226e-05\\
99.13	1.38526219049216e-05\\
99.14	1.33220384048085e-05\\
99.15	1.28003495446158e-05\\
99.16	1.22876308446349e-05\\
99.17	1.1783958532248e-05\\
99.18	1.12894095486218e-05\\
99.19	1.08040673361641e-05\\
99.2	1.03280183761673e-05\\
99.21	9.86134997679744e-06\\
99.22	9.40415028095366e-06\\
99.23	8.95650827419971e-06\\
99.24	8.51851379275563e-06\\
99.25	8.09025753160421e-06\\
99.26	7.67183105262158e-06\\
99.27	7.26332679283626e-06\\
99.28	6.86483807273673e-06\\
99.29	6.47645910466059e-06\\
99.3	6.09828500128598e-06\\
99.31	5.73041178416993e-06\\
99.32	5.37293639239766e-06\\
99.33	5.02595669130655e-06\\
99.34	4.68957148128807e-06\\
99.35	4.36388050667827e-06\\
99.36	4.04898446473498e-06\\
99.37	3.74498501470345e-06\\
99.38	3.45198478696705e-06\\
99.39	3.17008739210423e-06\\
99.4	2.89939742239079e-06\\
99.41	2.64002046114487e-06\\
99.42	2.39206309212044e-06\\
99.43	2.15563290903095e-06\\
99.44	1.93083852514583e-06\\
99.45	1.71778958298931e-06\\
99.46	1.51659676413639e-06\\
99.47	1.32737179910601e-06\\
99.48	1.15022747734263e-06\\
99.49	9.85277657314029e-07\\
99.5	8.32637276701118e-07\\
99.51	6.92422362690015e-07\\
99.52	5.64750042368264e-07\\
99.53	4.49738553228579e-07\\
99.54	3.47507253785351e-07\\
99.55	2.58176634277893e-07\\
99.56	1.81868327514198e-07\\
99.57	1.18705119791021e-07\\
99.58	6.88109619700894e-08\\
99.59	3.23109806184968e-08\\
99.6	9.33148930348793e-09\\
99.61	0\\
99.62	0\\
99.63	0\\
99.64	0\\
99.65	0\\
99.66	0\\
99.67	0\\
99.68	0\\
99.69	0\\
99.7	0\\
99.71	0\\
99.72	0\\
99.73	0\\
99.74	0\\
99.75	0\\
99.76	0\\
99.77	0\\
99.78	0\\
99.79	0\\
99.8	0\\
99.81	0\\
99.82	0\\
99.83	0\\
99.84	0\\
99.85	0\\
99.86	0\\
99.87	0\\
99.88	0\\
99.89	0\\
99.9	0\\
99.91	0\\
99.92	0\\
99.93	0\\
99.94	0\\
99.95	0\\
99.96	0\\
99.97	0\\
99.98	0\\
99.99	0\\
100	0\\
};
\addlegendentry{$q=-3$};

\addplot [color=red,dashed]
  table[row sep=crcr]{%
0.01	0.01\\
1.01	0.01\\
2.01	0.01\\
3.01	0.01\\
4.01	0.01\\
5.01	0.01\\
6.01	0.01\\
7.01	0.01\\
8.01	0.01\\
9.01	0.01\\
10.01	0.01\\
11.01	0.01\\
12.01	0.01\\
13.01	0.01\\
14.01	0.01\\
15.01	0.01\\
16.01	0.01\\
17.01	0.01\\
18.01	0.01\\
19.01	0.01\\
20.01	0.01\\
21.01	0.01\\
22.01	0.01\\
23.01	0.01\\
24.01	0.01\\
25.01	0.01\\
26.01	0.01\\
27.01	0.01\\
28.01	0.01\\
29.01	0.01\\
30.01	0.01\\
31.01	0.01\\
32.01	0.01\\
33.01	0.01\\
34.01	0.01\\
35.01	0.01\\
36.01	0.01\\
37.01	0.01\\
38.01	0.01\\
39.01	0.01\\
40.01	0.01\\
41.01	0.01\\
42.01	0.01\\
43.01	0.01\\
44.01	0.01\\
45.01	0.01\\
46.01	0.01\\
47.01	0.01\\
48.01	0.01\\
49.01	0.01\\
50.01	0.01\\
51.01	0.01\\
52.01	0.01\\
53.01	0.01\\
54.01	0.01\\
55.01	0.01\\
56.01	0.01\\
57.01	0.01\\
58.01	0.01\\
59.01	0.01\\
60.01	0.01\\
61.01	0.01\\
62.01	0.01\\
63.01	0.01\\
64.01	0.01\\
65.01	0.01\\
66.01	0.01\\
67.01	0.01\\
68.01	0.01\\
69.01	0.01\\
70.01	0.01\\
71.01	0.01\\
72.01	0.00990057041713054\\
73.01	0.00962270905049018\\
74.01	0.00931729974228617\\
75.01	0.0089779304749674\\
76.01	0.00859677687383986\\
77.01	0.00818162924635808\\
78.01	0.00774136108451931\\
79.01	0.00727472771568591\\
80.01	0.00678140382444221\\
81.01	0.0062596052458434\\
82.01	0.00570627218954784\\
83.01	0.00511790658247763\\
84.01	0.00449042694295134\\
85.01	0.00381895042717371\\
86.01	0.00332077846995116\\
87.01	0.00307372686830099\\
88.01	0.00283493893904919\\
89.01	0.00258799776351812\\
90.01	0.00233284257025338\\
91.01	0.00207030359555413\\
92.01	0.00180161100528257\\
93.01	0.00152852324252531\\
94.01	0.00125350033934508\\
95.01	0.000979957596255637\\
96.01	0.000712585247351627\\
97.01	0.000457646608820146\\
98.01	0.00022337038221871\\
99.01	4.39236936352807e-05\\
99.02	4.27894660898227e-05\\
99.03	4.16677340412432e-05\\
99.04	4.05585809066453e-05\\
99.05	3.94620907495404e-05\\
99.06	3.83783482850279e-05\\
99.07	3.73074388850345e-05\\
99.08	3.62494485835439e-05\\
99.09	3.52044640818897e-05\\
99.1	3.41725727540567e-05\\
99.11	3.31538626520134e-05\\
99.12	3.21484225110687e-05\\
99.13	3.11563417552743e-05\\
99.14	3.01777105028277e-05\\
99.15	2.92126195715094e-05\\
99.16	2.82611604841623e-05\\
99.17	2.73234254741638e-05\\
99.18	2.63995074909502e-05\\
99.19	2.54895002014359e-05\\
99.2	2.45934979939289e-05\\
99.21	2.37115959836436e-05\\
99.22	2.28438900182325e-05\\
99.23	2.19904766833416e-05\\
99.24	2.11514533081832e-05\\
99.25	2.03269179711221e-05\\
99.26	1.95169695052736e-05\\
99.27	1.87217075041362e-05\\
99.28	1.79412323272118e-05\\
99.29	1.71756451056593e-05\\
99.3	1.64250477479484e-05\\
99.31	1.56895429455264e-05\\
99.32	1.49692341784944e-05\\
99.33	1.42642257212871e-05\\
99.34	1.35746226483605e-05\\
99.35	1.29005308398839e-05\\
99.36	1.22420569874312e-05\\
99.37	1.15993085996729e-05\\
99.38	1.09723940080689e-05\\
99.39	1.03614226602163e-05\\
99.4	9.76651714497574e-06\\
99.41	9.18780102560947e-06\\
99.42	8.62539884664143e-06\\
99.43	8.07943614066463e-06\\
99.44	7.55003943518281e-06\\
99.45	7.03733625940541e-06\\
99.46	6.5414551510528e-06\\
99.47	6.06252566313741e-06\\
99.48	5.60067837072735e-06\\
99.49	5.15604487768065e-06\\
99.5	4.72875782336381e-06\\
99.51	4.3189508893253e-06\\
99.52	3.92675880595342e-06\\
99.53	3.5523173590752e-06\\
99.54	3.19576339652752e-06\\
99.55	2.85723483467448e-06\\
99.56	2.53687066486923e-06\\
99.57	2.23481095987417e-06\\
99.58	1.95119688019101e-06\\
99.59	1.68617068034213e-06\\
99.6	1.4398757150879e-06\\
99.61	1.2124564455207e-06\\
99.62	1.00405963195105e-06\\
99.63	8.14835682361875e-07\\
99.64	6.449361741271e-07\\
99.65	4.94513860235801e-07\\
99.66	3.63722675407116e-07\\
99.67	2.52717742073305e-07\\
99.68	1.61655376246586e-07\\
99.69	9.069309325066e-08\\
99.7	3.99896132857042e-08\\
99.71	9.70486685111793e-09\\
99.72	0\\
99.73	0\\
99.74	0\\
99.75	0\\
99.76	0\\
99.77	0\\
99.78	0\\
99.79	0\\
99.8	0\\
99.81	0\\
99.82	0\\
99.83	0\\
99.84	0\\
99.85	0\\
99.86	0\\
99.87	0\\
99.88	0\\
99.89	0\\
99.9	0\\
99.91	0\\
99.92	0\\
99.93	0\\
99.94	0\\
99.95	0\\
99.96	0\\
99.97	0\\
99.98	0\\
99.99	0\\
100	0\\
};
\addlegendentry{$q=-2$};

\addplot [color=blue,dashed]
  table[row sep=crcr]{%
0.01	0.01\\
1.01	0.01\\
2.01	0.01\\
3.01	0.01\\
4.01	0.01\\
5.01	0.01\\
6.01	0.01\\
7.01	0.01\\
8.01	0.01\\
9.01	0.01\\
10.01	0.01\\
11.01	0.01\\
12.01	0.01\\
13.01	0.01\\
14.01	0.01\\
15.01	0.01\\
16.01	0.01\\
17.01	0.01\\
18.01	0.01\\
19.01	0.01\\
20.01	0.01\\
21.01	0.01\\
22.01	0.01\\
23.01	0.01\\
24.01	0.01\\
25.01	0.01\\
26.01	0.01\\
27.01	0.01\\
28.01	0.01\\
29.01	0.01\\
30.01	0.01\\
31.01	0.01\\
32.01	0.01\\
33.01	0.01\\
34.01	0.01\\
35.01	0.01\\
36.01	0.01\\
37.01	0.01\\
38.01	0.01\\
39.01	0.01\\
40.01	0.01\\
41.01	0.01\\
42.01	0.01\\
43.01	0.01\\
44.01	0.01\\
45.01	0.01\\
46.01	0.01\\
47.01	0.01\\
48.01	0.01\\
49.01	0.01\\
50.01	0.01\\
51.01	0.01\\
52.01	0.01\\
53.01	0.01\\
54.01	0.01\\
55.01	0.01\\
56.01	0.01\\
57.01	0.01\\
58.01	0.01\\
59.01	0.01\\
60.01	0.01\\
61.01	0.01\\
62.01	0.01\\
63.01	0.01\\
64.01	0.01\\
65.01	0.01\\
66.01	0.01\\
67.01	0.01\\
68.01	0.01\\
69.01	0.01\\
70.01	0.01\\
71.01	0.01\\
72.01	0.01\\
73.01	0.01\\
74.01	0.01\\
75.01	0.01\\
76.01	0.01\\
77.01	0.01\\
78.01	0.01\\
79.01	0.01\\
80.01	0.01\\
81.01	0.01\\
82.01	0.01\\
83.01	0.01\\
84.01	0.01\\
85.01	0.01\\
86.01	0.00977996835059351\\
87.01	0.0092759272733409\\
88.01	0.00873565013865457\\
89.01	0.00817534486985543\\
90.01	0.00759394957702919\\
91.01	0.00698952267376253\\
92.01	0.00635975923032467\\
93.01	0.00570178861138295\\
94.01	0.00501174575093118\\
95.01	0.00428383487608855\\
96.01	0.00351239596837763\\
97.01	0.00269284670310579\\
98.01	0.00182013337470615\\
99.01	0.000891311399442173\\
99.02	0.000881920709532172\\
99.03	0.0008725346602031\\
99.04	0.000863153327663143\\
99.05	0.000853776788903485\\
99.06	0.000844405121708091\\
99.07	0.000835038404663648\\
99.08	0.000825676717169661\\
99.09	0.000816320139448719\\
99.1	0.000806968752556916\\
99.11	0.000797622638394458\\
99.12	0.000788281879716437\\
99.13	0.000778946560143766\\
99.14	0.000769616764174317\\
99.15	0.000760292577194233\\
99.16	0.000750974085489424\\
99.17	0.000741661376257259\\
99.18	0.000732354537618451\\
99.19	0.000723053658629137\\
99.2	0.000713758829293178\\
99.21	0.000704470140574644\\
99.22	0.000695187684410533\\
99.23	0.000685911553723684\\
99.24	0.000676641842435923\\
99.25	0.000667378645481434\\
99.26	0.00065812205882035\\
99.27	0.000648872179452586\\
99.28	0.000639629105431912\\
99.29	0.000630392935880257\\
99.3	0.000621163771002279\\
99.31	0.000611941712100177\\
99.32	0.000602726861588772\\
99.33	0.000593519323010844\\
99.34	0.000584319201052751\\
99.35	0.000575126601560311\\
99.36	0.000565941631554984\\
99.37	0.00055676439925033\\
99.38	0.00054759501406877\\
99.39	0.000538433586658444\\
99.4	0.000529280228901823\\
99.41	0.000520135053933318\\
99.42	0.000510998176157183\\
99.43	0.000501869711265774\\
99.44	0.000492749776258126\\
99.45	0.000483638489458892\\
99.46	0.000474535970537619\\
99.47	0.000465442340528399\\
99.48	0.000456357721849881\\
99.49	0.000447282238325672\\
99.5	0.000438216015205108\\
99.51	0.000429159179184449\\
99.52	0.000420111858428454\\
99.53	0.000411074182592397\\
99.54	0.000402046282844485\\
99.55	0.000393028291888746\\
99.56	0.000384020343988333\\
99.57	0.000375022574989311\\
99.58	0.000366035122344904\\
99.59	0.000357058125140244\\
99.6	0.000348091724117582\\
99.61	0.000339136061702052\\
99.62	0.000330191282027605\\
99.63	0.000321257530963142\\
99.64	0.000312334956139852\\
99.65	0.000303423706979116\\
99.66	0.000294523934720993\\
99.67	0.000285635792453304\\
99.68	0.000276759435141329\\
99.69	0.000267895019658133\\
99.7	0.000259042704815544\\
99.71	0.000250202651395793\\
99.72	0.000241375022183844\\
99.73	0.000232559984946454\\
99.74	0.000223757712161201\\
99.75	0.000214968378467508\\
99.76	0.000206192160702477\\
99.77	0.000197429237937504\\
99.78	0.000188679791515737\\
99.79	0.00017994400509036\\
99.8	0.000171222064663754\\
99.81	0.000162514158627553\\
99.82	0.000153820477803632\\
99.83	0.000145141215486051\\
99.84	0.000136476567483968\\
99.85	0.00012782673216559\\
99.86	0.000119191910503162\\
99.87	0.000110572306119041\\
99.88	0.000101968125332889\\
99.89	9.33795772100343e-05\\
99.9	8.4806873611008e-05\\
99.91	7.62502292423421e-05\\
99.92	6.77098617086289e-05\\
99.93	5.91859915659125e-05\\
99.94	5.06788423764483e-05\\
99.95	4.21886407648911e-05\\
99.96	3.37156164759312e-05\\
99.97	2.52600024334866e-05\\
99.98	1.68220348014392e-05\\
99.99	8.40195304604129e-06\\
100	0\\
};
\addlegendentry{$q=-1$};

\addplot [color=black,solid]
  table[row sep=crcr]{%
0.01	0\\
1.01	0\\
2.01	0\\
3.01	0\\
4.01	0\\
5.01	0\\
6.01	0\\
7.01	0\\
8.01	0\\
9.01	0\\
10.01	0\\
11.01	0\\
12.01	0\\
13.01	0\\
14.01	0\\
15.01	0\\
16.01	0\\
17.01	0\\
18.01	0\\
19.01	0\\
20.01	0\\
21.01	0\\
22.01	0\\
23.01	0\\
24.01	0\\
25.01	0\\
26.01	0\\
27.01	0\\
28.01	0\\
29.01	0\\
30.01	0\\
31.01	0\\
32.01	0\\
33.01	0\\
34.01	0\\
35.01	0\\
36.01	0\\
37.01	0\\
38.01	0\\
39.01	0\\
40.01	0\\
41.01	0\\
42.01	0\\
43.01	0\\
44.01	0\\
45.01	0\\
46.01	0\\
47.01	0\\
48.01	0\\
49.01	0\\
50.01	0\\
51.01	0\\
52.01	0\\
53.01	0\\
54.01	0\\
55.01	0\\
56.01	0\\
57.01	0\\
58.01	0\\
59.01	0\\
60.01	0\\
61.01	0\\
62.01	0\\
63.01	0\\
64.01	0\\
65.01	0\\
66.01	0\\
67.01	0\\
68.01	0\\
69.01	0\\
70.01	0\\
71.01	0\\
72.01	0\\
73.01	0\\
74.01	0\\
75.01	0\\
76.01	0\\
77.01	0\\
78.01	0\\
79.01	0\\
80.01	0\\
81.01	0\\
82.01	0\\
83.01	0\\
84.01	0\\
85.01	0\\
86.01	0\\
87.01	9.33319065741164e-05\\
88.01	0.00059435419498903\\
89.01	0.00114736171693115\\
90.01	0.00174962640650023\\
91.01	0.00238295213405392\\
92.01	0.00304664764442438\\
93.01	0.00374309473172899\\
94.01	0.00447679439357722\\
95.01	0.00525497275216101\\
96.01	0.00608691999137679\\
97.01	0.00697905984819611\\
98.01	0.00793791965569993\\
99.01	0.00896782660176875\\
99.02	0.00897835004419935\\
99.03	0.00898887365066564\\
99.04	0.00899939737136215\\
99.05	0.00900992115593992\\
99.06	0.00902044495349986\\
99.07	0.00903096871258595\\
99.08	0.00904149238117846\\
99.09	0.00905201590668712\\
99.1	0.009062539235944\\
99.11	0.00907306231519623\\
99.12	0.0090835850903821\\
99.13	0.00909410750691204\\
99.14	0.00910462950959158\\
99.15	0.00911515104261371\\
99.16	0.00912567204955099\\
99.17	0.00913619247334767\\
99.18	0.00914671225631155\\
99.19	0.00915723134010582\\
99.2	0.00916774966574069\\
99.21	0.00917826717356491\\
99.22	0.00918878380325717\\
99.23	0.00919929949381728\\
99.24	0.00920981418355733\\
99.25	0.00922032781009259\\
99.26	0.0092308403103323\\
99.27	0.00924135162047032\\
99.28	0.00925186167597559\\
99.29	0.00926237041158246\\
99.3	0.00927287776128084\\
99.31	0.00928338365830616\\
99.32	0.00929388803512921\\
99.33	0.00930439082344577\\
99.34	0.00931489195416604\\
99.35	0.00932539135740396\\
99.36	0.00933588896246625\\
99.37	0.00934638469784135\\
99.38	0.00935687849118809\\
99.39	0.00936737026932421\\
99.4	0.00937785995821467\\
99.41	0.00938834748295972\\
99.42	0.00939883276778278\\
99.43	0.00940931573601812\\
99.44	0.00941979631009832\\
99.45	0.00943027441154143\\
99.46	0.00944074996093798\\
99.47	0.00945122287793775\\
99.48	0.00946169308123622\\
99.49	0.00947216048856088\\
99.5	0.0094826250166572\\
99.51	0.00949308658127434\\
99.52	0.00950354509715069\\
99.53	0.00951400047799902\\
99.54	0.00952445263649142\\
99.55	0.00953490148424391\\
99.56	0.00954534693180085\\
99.57	0.00955578888861889\\
99.58	0.00956622726305078\\
99.59	0.00957666196232875\\
99.6	0.00958709289254763\\
99.61	0.0095975199586476\\
99.62	0.00960794306439661\\
99.63	0.00961836211237247\\
99.64	0.00962877700394458\\
99.65	0.00963918763925525\\
99.66	0.00964959391720069\\
99.67	0.00965999573541164\\
99.68	0.00967039299023348\\
99.69	0.0096807855767061\\
99.7	0.00969117338854325\\
99.71	0.00970155631812073\\
99.72	0.0097119342564558\\
99.73	0.00972230709318531\\
99.74	0.00973267471654341\\
99.75	0.00974303701333878\\
99.76	0.00975339386893135\\
99.77	0.00976374516720854\\
99.78	0.00977409079056095\\
99.79	0.00978443061985756\\
99.8	0.0097947645344203\\
99.81	0.00980509241199814\\
99.82	0.00981541412874054\\
99.83	0.00982572955917033\\
99.84	0.00983603857615593\\
99.85	0.00984634105088294\\
99.86	0.00985663685282512\\
99.87	0.00986692584971462\\
99.88	0.00987720790751152\\
99.89	0.0098874828903727\\
99.9	0.00989775066061986\\
99.91	0.00990801107870688\\
99.92	0.00991826400318625\\
99.93	0.00992850929067483\\
99.94	0.00993874679581858\\
99.95	0.00994897637125655\\
99.96	0.00995919786758387\\
99.97	0.0099694111333138\\
99.98	0.00997961601483888\\
99.99	0.0099898123563909\\
100	0.01\\
};
\addlegendentry{$q=0$};

\addplot [color=blue,solid]
  table[row sep=crcr]{%
0.01	0\\
1.01	0\\
2.01	0\\
3.01	0\\
4.01	0\\
5.01	0\\
6.01	0\\
7.01	0\\
8.01	0\\
9.01	0\\
10.01	0\\
11.01	0\\
12.01	0\\
13.01	0\\
14.01	0\\
15.01	0\\
16.01	0\\
17.01	0\\
18.01	0\\
19.01	0\\
20.01	0\\
21.01	0\\
22.01	0\\
23.01	0\\
24.01	0\\
25.01	0\\
26.01	0\\
27.01	0\\
28.01	0\\
29.01	0\\
30.01	0\\
31.01	0\\
32.01	0\\
33.01	0\\
34.01	0\\
35.01	0\\
36.01	0\\
37.01	0\\
38.01	0\\
39.01	0\\
40.01	0\\
41.01	0\\
42.01	0\\
43.01	0\\
44.01	0\\
45.01	0\\
46.01	0\\
47.01	0\\
48.01	0\\
49.01	0\\
50.01	0\\
51.01	0\\
52.01	0\\
53.01	0\\
54.01	0\\
55.01	0\\
56.01	0\\
57.01	0\\
58.01	0\\
59.01	0\\
60.01	0\\
61.01	0\\
62.01	0\\
63.01	0\\
64.01	0\\
65.01	0\\
66.01	0\\
67.01	0\\
68.01	0\\
69.01	0\\
70.01	0\\
71.01	0\\
72.01	0\\
73.01	0\\
74.01	0\\
75.01	0\\
76.01	0.000157971794062944\\
77.01	0.000497402840502613\\
78.01	0.000868732816274718\\
79.01	0.00127798997886026\\
80.01	0.00173267500751677\\
81.01	0.00223731746694362\\
82.01	0.00277948353736682\\
83.01	0.00336076721073516\\
84.01	0.00398508455809201\\
85.01	0.00465541316784714\\
86.01	0.0053750082596734\\
87.01	0.00605767964090931\\
88.01	0.00638058741292995\\
89.01	0.00669288600691645\\
90.01	0.0069952637881693\\
91.01	0.00730567053966473\\
92.01	0.00762610637409276\\
93.01	0.00795595548448326\\
94.01	0.00829271951313568\\
95.01	0.0086326909036133\\
96.01	0.00897044149421249\\
97.01	0.00929838445085943\\
98.01	0.0096063332737912\\
99.01	0.0098623535996167\\
99.02	0.00986437411420948\\
99.03	0.00986638419216177\\
99.04	0.0098683837718627\\
99.05	0.00987037279127147\\
99.06	0.00987235118791395\\
99.07	0.00987431889887921\\
99.08	0.00987627586053859\\
99.09	0.00987822200859097\\
99.1	0.00988015727829076\\
99.11	0.00988208160444435\\
99.12	0.00988399492112197\\
99.13	0.00988589716186539\\
99.14	0.00988778825975373\\
99.15	0.00988966814739988\\
99.16	0.00989153675694688\\
99.17	0.00989339402006423\\
99.18	0.00989523986794429\\
99.19	0.00989707423129856\\
99.2	0.00989889704035396\\
99.21	0.00990070822484916\\
99.22	0.00990250771403083\\
99.23	0.00990429543664984\\
99.24	0.00990607132095756\\
99.25	0.00990783529470205\\
99.26	0.00990958728512422\\
99.27	0.00991132721895409\\
99.28	0.0099130550224069\\
99.29	0.00991477062117929\\
99.3	0.00991647394044544\\
99.31	0.00991816490485323\\
99.32	0.00991984343852032\\
99.33	0.00992150946503027\\
99.34	0.00992316290742868\\
99.35	0.00992480368821925\\
99.36	0.00992643172935984\\
99.37	0.00992804695225863\\
99.38	0.00992964927777009\\
99.39	0.00993123862619115\\
99.4	0.00993281491725719\\
99.41	0.00993437807013812\\
99.42	0.00993592800343448\\
99.43	0.00993746463517346\\
99.44	0.00993898788280498\\
99.45	0.00994049766319776\\
99.46	0.00994199389263541\\
99.47	0.00994347648681246\\
99.48	0.00994494536083052\\
99.49	0.00994640042919431\\
99.5	0.00994784160580784\\
99.51	0.00994926880397047\\
99.52	0.00995068193637311\\
99.53	0.00995208091509435\\
99.54	0.00995346565159667\\
99.55	0.00995483605672261\\
99.56	0.00995619204069105\\
99.57	0.00995753351309345\\
99.58	0.00995886038289013\\
99.59	0.00996017255840666\\
99.6	0.00996146994733017\\
99.61	0.00996275245671045\\
99.62	0.00996401999295883\\
99.63	0.00996527246184481\\
99.64	0.00996650976849268\\
99.65	0.00996773181737827\\
99.66	0.0099689385123257\\
99.67	0.00997012975650425\\
99.68	0.00997130545242531\\
99.69	0.00997246550193929\\
99.7	0.00997360980623282\\
99.71	0.009974738248558\\
99.72	0.00997585070992424\\
99.73	0.00997694707046292\\
99.74	0.00997802720942307\\
99.75	0.00997909100516722\\
99.76	0.00998013833516726\\
99.77	0.00998116907600054\\
99.78	0.00998218310334597\\
99.79	0.00998318029198038\\
99.8	0.00998416051577496\\
99.81	0.00998512364769184\\
99.82	0.00998606955978089\\
99.83	0.00998699812317667\\
99.84	0.00998790920809558\\
99.85	0.00998880268383318\\
99.86	0.0099896784187618\\
99.87	0.00999053628032826\\
99.88	0.00999137613505194\\
99.89	0.00999219784852308\\
99.9	0.00999300128540131\\
99.91	0.0099937863094145\\
99.92	0.00999455278335794\\
99.93	0.00999530056909381\\
99.94	0.009996029527551\\
99.95	0.00999673951872531\\
99.96	0.00999743040168006\\
99.97	0.00999810203454702\\
99.98	0.00999875427452794\\
99.99	0.00999938697789635\\
100	0.01\\
};
\addlegendentry{$q=1$};

\addplot [color=red,solid]
  table[row sep=crcr]{%
0.01	0\\
1.01	0\\
2.01	0\\
3.01	0\\
4.01	0\\
5.01	0\\
6.01	0\\
7.01	0\\
8.01	0\\
9.01	0\\
10.01	0\\
11.01	0\\
12.01	0\\
13.01	0\\
14.01	0\\
15.01	0\\
16.01	0\\
17.01	0\\
18.01	0\\
19.01	0\\
20.01	0\\
21.01	0\\
22.01	0\\
23.01	0\\
24.01	0\\
25.01	0\\
26.01	0\\
27.01	0\\
28.01	0\\
29.01	0\\
30.01	0\\
31.01	0\\
32.01	0\\
33.01	0\\
34.01	0\\
35.01	0\\
36.01	0\\
37.01	0\\
38.01	0\\
39.01	0\\
40.01	0\\
41.01	0\\
42.01	0\\
43.01	0\\
44.01	0\\
45.01	0\\
46.01	0\\
47.01	0\\
48.01	0\\
49.01	0\\
50.01	0\\
51.01	0\\
52.01	0\\
53.01	0\\
54.01	0\\
55.01	0\\
56.01	0\\
57.01	0\\
58.01	0\\
59.01	0\\
60.01	0\\
61.01	8.12348871340116e-05\\
62.01	0.000266803100728522\\
63.01	0.000465190390940062\\
64.01	0.00067797855544325\\
65.01	0.000907073550808404\\
66.01	0.0011547455705739\\
67.01	0.00142371516634133\\
68.01	0.00171724410815382\\
69.01	0.00203920083310826\\
70.01	0.00239432008939969\\
71.01	0.00277840888262248\\
72.01	0.003187457880224\\
73.01	0.00362456692764628\\
74.01	0.00409352079991442\\
75.01	0.00459918226766583\\
76.01	0.00498833685124197\\
77.01	0.00523196975578501\\
78.01	0.00547844433181472\\
79.01	0.00572303242577908\\
80.01	0.00595956126638655\\
81.01	0.00618383648916674\\
82.01	0.00640903628688978\\
83.01	0.00663500495559349\\
84.01	0.00685912686680056\\
85.01	0.00707947183678679\\
86.01	0.00729337640664342\\
87.01	0.00749391939642775\\
88.01	0.00768901014755165\\
89.01	0.00788544157167353\\
90.01	0.00808501675136655\\
91.01	0.0082882727431725\\
92.01	0.00849454843487404\\
93.01	0.00870271239060721\\
94.01	0.00891144151874779\\
95.01	0.00911932188514614\\
96.01	0.00932498131322101\\
97.01	0.00952730884353645\\
98.01	0.00972576718884031\\
99.01	0.00989322850663147\\
99.02	0.00989465349182635\\
99.03	0.00989607320763906\\
99.04	0.00989748761156016\\
99.05	0.00989889666066474\\
99.06	0.00990030031160832\\
99.07	0.00990169852062259\\
99.08	0.00990309124351133\\
99.09	0.0099044784356463\\
99.1	0.00990586005196288\\
99.11	0.00990723604695566\\
99.12	0.00990860637467526\\
99.13	0.00990997098872422\\
99.14	0.00991132984225253\\
99.15	0.00991268288795305\\
99.16	0.009914030078057\\
99.17	0.00991537136432928\\
99.18	0.00991670669806387\\
99.19	0.00991803603007907\\
99.2	0.00991935931071277\\
99.21	0.00992067648981765\\
99.22	0.0099219875167563\\
99.23	0.00992329234039635\\
99.24	0.00992459090910551\\
99.25	0.00992588317074659\\
99.26	0.00992716907267241\\
99.27	0.00992844856172076\\
99.28	0.0099297215842092\\
99.29	0.0099309880859299\\
99.3	0.00993224801214435\\
99.31	0.0099335013075781\\
99.32	0.00993474791641538\\
99.33	0.00993598778229369\\
99.34	0.00993722084829831\\
99.35	0.00993844705695684\\
99.36	0.00993966635023357\\
99.37	0.00994087866952389\\
99.38	0.00994208395564854\\
99.39	0.00994328214884795\\
99.4	0.00994447318877637\\
99.41	0.00994565701449602\\
99.42	0.00994683356447118\\
99.43	0.00994800277656223\\
99.44	0.00994916458801953\\
99.45	0.00995031893547741\\
99.46	0.00995146575494794\\
99.47	0.00995260498181472\\
99.48	0.00995373655082658\\
99.49	0.00995486039609122\\
99.5	0.0099559764510688\\
99.51	0.00995708464856543\\
99.52	0.00995818492072661\\
99.53	0.00995927719903064\\
99.54	0.00996036141428188\\
99.55	0.00996143749660399\\
99.56	0.00996250537543313\\
99.57	0.00996356497951101\\
99.58	0.00996461623687793\\
99.59	0.00996565907486572\\
99.6	0.00996669342009061\\
99.61	0.00996771918980673\\
99.62	0.00996873629602636\\
99.63	0.00996974464986599\\
99.64	0.00997074416153733\\
99.65	0.00997173474033815\\
99.66	0.00997271629464304\\
99.67	0.00997368873189412\\
99.68	0.00997465195859159\\
99.69	0.00997560588028419\\
99.7	0.00997655040155958\\
99.71	0.00997748542604711\\
99.72	0.00997841085640911\\
99.73	0.00997932659433115\\
99.74	0.00998023254051213\\
99.75	0.00998112859465432\\
99.76	0.0099820146554532\\
99.77	0.00998289062058732\\
99.78	0.00998375638670789\\
99.79	0.00998461184942835\\
99.8	0.00998545690331381\\
99.81	0.00998629144187034\\
99.82	0.00998711535753411\\
99.83	0.00998792854166048\\
99.84	0.00998873088451288\\
99.85	0.00998952227525159\\
99.86	0.00999030260192242\\
99.87	0.00999107175144514\\
99.88	0.00999182960960189\\
99.89	0.00999257606102539\\
99.9	0.00999331098918694\\
99.91	0.00999403427638442\\
99.92	0.00999474580372994\\
99.93	0.0099954454511375\\
99.94	0.00999613309731035\\
99.95	0.0099968086197283\\
99.96	0.00999747189463473\\
99.97	0.00999812279702354\\
99.98	0.00999876120062581\\
99.99	0.00999938697789635\\
100	0.01\\
};
\addlegendentry{$q=2$};

\addplot [color=mycolor1,solid]
  table[row sep=crcr]{%
0.01	0\\
1.01	0\\
2.01	0\\
3.01	0\\
4.01	0\\
5.01	0\\
6.01	0\\
7.01	0\\
8.01	0\\
9.01	0\\
10.01	0\\
11.01	0\\
12.01	0\\
13.01	0\\
14.01	0\\
15.01	0\\
16.01	0\\
17.01	0\\
18.01	0\\
19.01	0\\
20.01	0\\
21.01	0\\
22.01	0\\
23.01	0\\
24.01	0\\
25.01	0\\
26.01	0\\
27.01	0\\
28.01	0\\
29.01	0\\
30.01	0\\
31.01	0\\
32.01	0\\
33.01	5.09644672509399e-05\\
34.01	0.000109794444065885\\
35.01	0.00017151533121116\\
36.01	0.000236298073143903\\
37.01	0.000304316035931007\\
38.01	0.000375738239777754\\
39.01	0.000450718319040512\\
40.01	0.00052938742339783\\
41.01	0.000611950742902587\\
42.01	0.000698694363709187\\
43.01	0.000789941291687961\\
44.01	0.000886057053765751\\
45.01	0.000987458085999988\\
46.01	0.00109462229690098\\
47.01	0.0012081024668774\\
48.01	0.00132854339560494\\
49.01	0.00145670404379396\\
50.01	0.00159348638652697\\
51.01	0.00173997336230275\\
52.01	0.00189747925447799\\
53.01	0.00206761721379348\\
54.01	0.00225239062531717\\
55.01	0.00245431795358011\\
56.01	0.00267590725002329\\
57.01	0.00291282995301112\\
58.01	0.00316479531533496\\
59.01	0.00343368670551544\\
60.01	0.00372185027368629\\
61.01	0.00395022031197872\\
62.01	0.00409293588869374\\
63.01	0.00424183573526733\\
64.01	0.00439676674185464\\
65.01	0.00455733023107847\\
66.01	0.00472276548092922\\
67.01	0.00489177668395449\\
68.01	0.00506227999851957\\
69.01	0.00523040716100169\\
70.01	0.00539063353106456\\
71.01	0.00554740124099169\\
72.01	0.00570587964539781\\
73.01	0.0058642310340034\\
74.01	0.00601985027425704\\
75.01	0.0061690628622802\\
76.01	0.00630852010183723\\
77.01	0.0064449175239508\\
78.01	0.00658079677745805\\
79.01	0.00671645879885539\\
80.01	0.00685206299611372\\
81.01	0.0069887554420211\\
82.01	0.00712734193524032\\
83.01	0.00726779158239517\\
84.01	0.00741019680706147\\
85.01	0.00755477469004441\\
86.01	0.00770168193423851\\
87.01	0.00785138131254818\\
88.01	0.00800463383062428\\
89.01	0.00816174771874713\\
90.01	0.00832282605401112\\
91.01	0.0084877615335843\\
92.01	0.00865636593366449\\
93.01	0.00882843564353477\\
94.01	0.00900378484830787\\
95.01	0.00918226226366763\\
96.01	0.00936376041375512\\
97.01	0.00954820718766945\\
98.01	0.00973490290657201\\
99.01	0.00989397488312613\\
99.02	0.00989537779255572\\
99.03	0.00989677585830473\\
99.04	0.00989816903435277\\
99.05	0.00989955727424447\\
99.06	0.00990094053108541\\
99.07	0.00990231875753784\\
99.08	0.00990369190581653\\
99.09	0.00990505992768447\\
99.1	0.00990642277444855\\
99.11	0.00990778039695523\\
99.12	0.00990913274558614\\
99.13	0.00991047977025362\\
99.14	0.0099118214203963\\
99.15	0.00991315764497448\\
99.16	0.00991448839246565\\
99.17	0.00991581361085984\\
99.18	0.00991713324765496\\
99.19	0.00991844724985212\\
99.2	0.00991975556395086\\
99.21	0.00992105813594439\\
99.22	0.00992235491131472\\
99.23	0.0099236458350278\\
99.24	0.0099249308515286\\
99.25	0.00992620990473608\\
99.26	0.00992748293803825\\
99.27	0.00992874989428702\\
99.28	0.00993001071579311\\
99.29	0.00993126534432089\\
99.3	0.00993251372108315\\
99.31	0.00993375578673581\\
99.32	0.00993499148137264\\
99.33	0.00993622074451984\\
99.34	0.00993744351513067\\
99.35	0.00993865973157995\\
99.36	0.00993986933165849\\
99.37	0.0099410722525676\\
99.38	0.00994226843091338\\
99.39	0.00994345780270105\\
99.4	0.00994464030332924\\
99.41	0.00994581586758415\\
99.42	0.00994698442963372\\
99.43	0.00994814592302171\\
99.44	0.00994930028066174\\
99.45	0.00995044743483127\\
99.46	0.00995158731716552\\
99.47	0.0099527198586513\\
99.48	0.00995384498962087\\
99.49	0.00995496263974562\\
99.5	0.0099560727380298\\
99.51	0.00995717521280411\\
99.52	0.00995826999171929\\
99.53	0.00995935700173958\\
99.54	0.0099604361691362\\
99.55	0.00996150741948069\\
99.56	0.00996257067763823\\
99.57	0.00996362586776092\\
99.58	0.00996467291328088\\
99.59	0.00996571173690344\\
99.6	0.00996674226060019\\
99.61	0.00996776439644508\\
99.62	0.00996877805189459\\
99.63	0.00996978313349925\\
99.64	0.0099707795468949\\
99.65	0.00997176719679398\\
99.66	0.00997274598697662\\
99.67	0.00997371582028172\\
99.68	0.00997467659859792\\
99.69	0.00997562822285446\\
99.7	0.00997657059301199\\
99.71	0.0099775036080533\\
99.72	0.00997842716597387\\
99.73	0.00997934116377243\\
99.74	0.0099802454974414\\
99.75	0.00998114006195719\\
99.76	0.00998202475127047\\
99.77	0.00998289945829632\\
99.78	0.00998376407490432\\
99.79	0.00998461849190844\\
99.8	0.00998546259905699\\
99.81	0.00998629628502236\\
99.82	0.00998711943739068\\
99.83	0.00998793194265142\\
99.84	0.00998873368618689\\
99.85	0.00998952455226156\\
99.86	0.0099903044240114\\
99.87	0.00999107318343302\\
99.88	0.00999183071137277\\
99.89	0.00999257688751569\\
99.9	0.00999331159037439\\
99.91	0.00999403469727783\\
99.92	0.00999474608435993\\
99.93	0.00999544562654817\\
99.94	0.00999613319755202\\
99.95	0.00999680866985125\\
99.96	0.0099974719146842\\
99.97	0.00999812280203583\\
99.98	0.00999876120062581\\
99.99	0.00999938697789635\\
100	0.01\\
};
\addlegendentry{$q=3$};

\addplot [color=green,solid]
  table[row sep=crcr]{%
0.01	0.00157538227128272\\
1.01	0.00159330354651341\\
2.01	0.00161198537576968\\
3.01	0.00163146474644173\\
4.01	0.00165178130380222\\
5.01	0.00167297744204932\\
6.01	0.00169509857820879\\
7.01	0.00171819349773444\\
8.01	0.00174231475891689\\
9.01	0.0017675191685306\\
10.01	0.00179386834469968\\
11.01	0.00182142938721429\\
12.01	0.00185027568107371\\
13.01	0.00188048786631688\\
14.01	0.00191215501683038\\
15.01	0.00194537608363874\\
16.01	0.00198026167534923\\
17.01	0.00201693627157198\\
18.01	0.00205554099655589\\
19.01	0.00209623712319749\\
20.01	0.00213921053658141\\
21.01	0.002184677467822\\
22.01	0.00223289192256133\\
23.01	0.0022841553875158\\
24.01	0.00233882962246737\\
25.01	0.00239735366247747\\
26.01	0.00246026660746542\\
27.01	0.00252823842498746\\
28.01	0.00260211192820974\\
29.01	0.00268242605954877\\
30.01	0.00276757166883638\\
31.01	0.00285772953510862\\
32.01	0.00295348971141785\\
33.01	0.00300372734832802\\
34.01	0.00305074524799789\\
35.01	0.003099746021628\\
36.01	0.00315080568957947\\
37.01	0.00320401012188151\\
38.01	0.00325946384392634\\
39.01	0.00331730508574213\\
40.01	0.00337772026057272\\
41.01	0.00344085589109503\\
42.01	0.00350682458589178\\
43.01	0.00357575608053452\\
44.01	0.00364777979727941\\
45.01	0.0037230187089124\\
46.01	0.00380158242461726\\
47.01	0.00388355697266633\\
48.01	0.00396898979951969\\
49.01	0.00405786783034105\\
50.01	0.00415008544349357\\
51.01	0.00424539774580612\\
52.01	0.0043433523618149\\
53.01	0.00444318971346495\\
54.01	0.00454369692583007\\
55.01	0.00464299322119982\\
56.01	0.00473892097650941\\
57.01	0.00483613780119834\\
58.01	0.00493510252095684\\
59.01	0.00503411480971749\\
60.01	0.00513164758415562\\
61.01	0.00522627916595729\\
62.01	0.00532151384207943\\
63.01	0.00541782343389308\\
64.01	0.0055146740077758\\
65.01	0.00561145400197507\\
66.01	0.00570750901710846\\
67.01	0.00580221735540401\\
68.01	0.0058951352324198\\
69.01	0.00598689973384306\\
70.01	0.00607893169971058\\
71.01	0.0061719996330892\\
72.01	0.0062661112646304\\
73.01	0.00636111892574771\\
74.01	0.00645698629557117\\
75.01	0.00655387851398136\\
76.01	0.00665228205931369\\
77.01	0.00675276780062926\\
78.01	0.00685569246110767\\
79.01	0.00696131053188288\\
80.01	0.00706992105889258\\
81.01	0.0071818163679376\\
82.01	0.0072972050433229\\
83.01	0.00741628152993219\\
84.01	0.0075392471567647\\
85.01	0.00766629502929004\\
86.01	0.00779761327606906\\
87.01	0.00793338849602983\\
88.01	0.00807377339546965\\
89.01	0.00821887633982523\\
90.01	0.00836876794752141\\
91.01	0.00852348913886508\\
92.01	0.00868305782989654\\
93.01	0.00884746537033053\\
94.01	0.0090166661308453\\
95.01	0.00919056134976411\\
96.01	0.0093689767617065\\
97.01	0.00955163360046357\\
98.01	0.00973566740022747\\
99.01	0.00989398858246139\\
99.02	0.00989539094476144\\
99.03	0.00989678847973188\\
99.04	0.00989818114103547\\
99.05	0.00989956888190258\\
99.06	0.00990095165512703\\
99.07	0.00990232941306191\\
99.08	0.00990370210761538\\
99.09	0.00990506969024643\\
99.1	0.00990643211196056\\
99.11	0.00990778932330546\\
99.12	0.00990914127436665\\
99.13	0.00991048791476302\\
99.14	0.0099118291936424\\
99.15	0.00991316505967706\\
99.16	0.00991449546105915\\
99.17	0.00991582034549609\\
99.18	0.00991713966020598\\
99.19	0.00991845335191288\\
99.2	0.00991976136684211\\
99.21	0.00992106365071546\\
99.22	0.00992236014874638\\
99.23	0.00992365080563514\\
99.24	0.00992493556556388\\
99.25	0.00992621437219167\\
99.26	0.00992748716864954\\
99.27	0.00992875389753535\\
99.28	0.00993001450090877\\
99.29	0.00993126892028608\\
99.3	0.00993251709663497\\
99.31	0.00993375897036932\\
99.32	0.00993499448134388\\
99.33	0.00993622356884892\\
99.34	0.00993744617160479\\
99.35	0.00993866222775656\\
99.36	0.0099398716748684\\
99.37	0.00994107444991809\\
99.38	0.0099422704892914\\
99.39	0.00994345972877637\\
99.4	0.00994464210355766\\
99.41	0.00994581754821071\\
99.42	0.00994698599669593\\
99.43	0.00994814738235283\\
99.44	0.00994930163789402\\
99.45	0.00995044869539926\\
99.46	0.00995158848630938\\
99.47	0.00995272094142012\\
99.48	0.00995384599087603\\
99.49	0.00995496356416414\\
99.5	0.00995607359010774\\
99.51	0.00995717599685997\\
99.52	0.00995827071189741\\
99.53	0.0099593576620136\\
99.54	0.00996043677331251\\
99.55	0.0099615079712019\\
99.56	0.00996257118038667\\
99.57	0.00996362632486211\\
99.58	0.00996467332790712\\
99.59	0.00996571211207732\\
99.6	0.00996674259919812\\
99.61	0.00996776470120001\\
99.62	0.00996877832540117\\
99.63	0.00996978337821723\\
99.64	0.00997077976515262\\
99.65	0.00997176739079183\\
99.66	0.0099727461587905\\
99.67	0.00997371597186654\\
99.68	0.00997467673179107\\
99.69	0.00997562833937934\\
99.7	0.00997657069448153\\
99.71	0.00997750369597346\\
99.72	0.00997842724174722\\
99.73	0.0099793412287017\\
99.74	0.00998024555273308\\
99.75	0.00998114010872511\\
99.76	0.00998202479053945\\
99.77	0.00998289949100579\\
99.78	0.00998376410191195\\
99.79	0.00998461851399383\\
99.8	0.00998546261692534\\
99.81	0.00998629629930815\\
99.82	0.0099871194486614\\
99.83	0.00998793195141125\\
99.84	0.00998873369288044\\
99.85	0.00998952455727761\\
99.86	0.00999030442768662\\
99.87	0.00999107318605574\\
99.88	0.0099918307131867\\
99.89	0.00999257688872371\\
99.9	0.00999331159114228\\
99.91	0.00999403469773802\\
99.92	0.00999474608461527\\
99.93	0.00999544562667568\\
99.94	0.00999613319760659\\
99.95	0.00999680866986942\\
99.96	0.00999747191468782\\
99.97	0.00999812280203583\\
99.98	0.00999876120062581\\
99.99	0.00999938697789635\\
100	0.01\\
};
\addlegendentry{$q=4$};

\end{axis}
\end{tikzpicture}%
 
%  \caption{Continuous Time w/ nFPC}
%\end{subfigure}%
%\hfill%
%\begin{subfigure}{.45\linewidth}
%  \centering
%  \setlength\figureheight{\linewidth} 
%  \setlength\figurewidth{\linewidth}
%  \tikzsetnextfilename{dp_dscr_nFPC_z8}
%  % This file was created by matlab2tikz.
%
%The latest updates can be retrieved from
%  http://www.mathworks.com/matlabcentral/fileexchange/22022-matlab2tikz-matlab2tikz
%where you can also make suggestions and rate matlab2tikz.
%
\definecolor{mycolor1}{rgb}{0.00000,1.00000,0.14286}%
\definecolor{mycolor2}{rgb}{0.00000,1.00000,0.28571}%
\definecolor{mycolor3}{rgb}{0.00000,1.00000,0.42857}%
\definecolor{mycolor4}{rgb}{0.00000,1.00000,0.57143}%
\definecolor{mycolor5}{rgb}{0.00000,1.00000,0.71429}%
\definecolor{mycolor6}{rgb}{0.00000,1.00000,0.85714}%
\definecolor{mycolor7}{rgb}{0.00000,1.00000,1.00000}%
\definecolor{mycolor8}{rgb}{0.00000,0.87500,1.00000}%
\definecolor{mycolor9}{rgb}{0.00000,0.62500,1.00000}%
\definecolor{mycolor10}{rgb}{0.12500,0.00000,1.00000}%
\definecolor{mycolor11}{rgb}{0.25000,0.00000,1.00000}%
\definecolor{mycolor12}{rgb}{0.37500,0.00000,1.00000}%
\definecolor{mycolor13}{rgb}{0.50000,0.00000,1.00000}%
\definecolor{mycolor14}{rgb}{0.62500,0.00000,1.00000}%
\definecolor{mycolor15}{rgb}{0.75000,0.00000,1.00000}%
\definecolor{mycolor16}{rgb}{0.87500,0.00000,1.00000}%
\definecolor{mycolor17}{rgb}{1.00000,0.00000,1.00000}%
\definecolor{mycolor18}{rgb}{1.00000,0.00000,0.87500}%
\definecolor{mycolor19}{rgb}{1.00000,0.00000,0.62500}%
\definecolor{mycolor20}{rgb}{0.85714,0.00000,0.00000}%
\definecolor{mycolor21}{rgb}{0.71429,0.00000,0.00000}%
%
\begin{tikzpicture}[trim axis left, trim axis right]

\begin{axis}[%
width=\figurewidth,
height=\figureheight,
at={(0\figurewidth,0\figureheight)},
scale only axis,
every outer x axis line/.append style={black},
every x tick label/.append style={font=\color{black}},
xmin=0,
xmax=600,
every outer y axis line/.append style={black},
every y tick label/.append style={font=\color{black}},
ymin=0,
ymax=0.014,
axis background/.style={fill=white},
axis x line*=bottom,
axis y line*=left,
yticklabel style={
        /pgf/number format/fixed,
        /pgf/number format/precision=3
},
scaled y ticks=false
]
\addplot [color=green,solid,forget plot]
  table[row sep=crcr]{%
1	0.00590226758762085\\
2	0.00590219569436802\\
3	0.00590212217829792\\
4	0.00590204700301902\\
5	0.00590197013133593\\
6	0.0059018915252321\\
7	0.00590181114585226\\
8	0.00590172895348463\\
9	0.00590164490754284\\
10	0.00590155896654729\\
11	0.00590147108810626\\
12	0.00590138122889673\\
13	0.00590128934464471\\
14	0.00590119539010542\\
15	0.00590109931904272\\
16	0.00590100108420855\\
17	0.00590090063732162\\
18	0.00590079792904614\\
19	0.00590069290896984\\
20	0.00590058552558166\\
21	0.00590047572624894\\
22	0.00590036345719459\\
23	0.00590024866347334\\
24	0.00590013128894799\\
25	0.00590001127626487\\
26	0.00589988856682939\\
27	0.00589976310078054\\
28	0.00589963481696547\\
29	0.00589950365291344\\
30	0.00589936954480913\\
31	0.0058992324274661\\
32	0.00589909223429924\\
33	0.00589894889729688\\
34	0.00589880234699279\\
35	0.00589865251243731\\
36	0.00589849932116842\\
37	0.00589834269918213\\
38	0.00589818257090226\\
39	0.00589801885915011\\
40	0.00589785148511377\\
41	0.00589768036831639\\
42	0.00589750542658482\\
43	0.0058973265760169\\
44	0.00589714373094918\\
45	0.00589695680392344\\
46	0.00589676570565338\\
47	0.00589657034499055\\
48	0.00589637062888976\\
49	0.00589616646237447\\
50	0.00589595774850121\\
51	0.00589574438832412\\
52	0.00589552628085854\\
53	0.00589530332304463\\
54	0.00589507540971039\\
55	0.00589484243353415\\
56	0.00589460428500702\\
57	0.00589436085239466\\
58	0.00589411202169841\\
59	0.00589385767661683\\
60	0.00589359769850622\\
61	0.00589333196634099\\
62	0.00589306035667342\\
63	0.00589278274359378\\
64	0.0058924989986892\\
65	0.00589220899100278\\
66	0.00589191258699223\\
67	0.00589160965048836\\
68	0.00589130004265269\\
69	0.00589098362193565\\
70	0.00589066024403374\\
71	0.00589032976184663\\
72	0.00588999202543418\\
73	0.00588964688197296\\
74	0.00588929417571254\\
75	0.00588893374793158\\
76	0.00588856543689365\\
77	0.00588818907780258\\
78	0.0058878045027581\\
79	0.00588741154071082\\
80	0.00588701001741674\\
81	0.00588659975539223\\
82	0.00588618057386828\\
83	0.00588575228874461\\
84	0.00588531471254356\\
85	0.00588486765436407\\
86	0.00588441091983501\\
87	0.00588394431106872\\
88	0.00588346762661411\\
89	0.00588298066140999\\
90	0.00588248320673816\\
91	0.00588197505017642\\
92	0.0058814559755516\\
93	0.00588092576289306\\
94	0.00588038418838609\\
95	0.00587983102432578\\
96	0.00587926603907173\\
97	0.00587868899700308\\
98	0.00587809965847495\\
99	0.00587749777977601\\
100	0.00587688311308808\\
101	0.00587625540644743\\
102	0.00587561440370964\\
103	0.00587495984451722\\
104	0.00587429146427208\\
105	0.00587360899411308\\
106	0.00587291216090031\\
107	0.00587220068720716\\
108	0.00587147429132234\\
109	0.0058707326872633\\
110	0.00586997558480418\\
111	0.00586920268952053\\
112	0.00586841370285426\\
113	0.00586760832220309\\
114	0.00586678624103878\\
115	0.00586594714905966\\
116	0.0058650907323843\\
117	0.00586421667379264\\
118	0.00586332465302427\\
119	0.00586241434714379\\
120	0.00586148543098398\\
121	0.00586053757768146\\
122	0.00585957045931951\\
123	0.0058585837476963\\
124	0.0058575771152385\\
125	0.00585655023608422\\
126	0.00585550278736069\\
127	0.00585443445068751\\
128	0.00585334491393736\\
129	0.00585223387329249\\
130	0.00585110103563626\\
131	0.00584994612132499\\
132	0.00584876886738457\\
133	0.005847569031183\\
134	0.00584634639462503\\
135	0.00584510076891578\\
136	0.00584383199993321\\
137	0.00584253997423657\\
138	0.00584122462571889\\
139	0.00583988594288035\\
140	0.00583852397665295\\
141	0.00583713884863892\\
142	0.00583573075952868\\
143	0.00583429999732783\\
144	0.00583284694483269\\
145	0.00583137208553207\\
146	0.00582987600675632\\
147	0.00582835939840925\\
148	0.00582682304496935\\
149	0.00582526780758558\\
150	0.00582369459197112\\
151	0.00582210429627641\\
152	0.0058204977296022\\
153	0.00581887545880713\\
154	0.00581723739739895\\
155	0.00581558340849058\\
156	0.00581391335407126\\
157	0.00581222709496928\\
158	0.00581052449081065\\
159	0.00580880539997442\\
160	0.00580706967954358\\
161	0.0058053171852519\\
162	0.00580354777142536\\
163	0.00580176129091901\\
164	0.00579995759504783\\
165	0.00579813653351179\\
166	0.00579629795431418\\
167	0.00579444170367297\\
168	0.00579256762592458\\
169	0.00579067556341947\\
170	0.00578876535640884\\
171	0.00578683684292163\\
172	0.00578488985863155\\
173	0.00578292423671281\\
174	0.00578093980768388\\
175	0.00577893639923868\\
176	0.00577691383606331\\
177	0.00577487193963837\\
178	0.00577281052802476\\
179	0.00577072941563249\\
180	0.00576862841297023\\
181	0.00576650732637528\\
182	0.00576436595772126\\
183	0.00576220410410249\\
184	0.00576002155749306\\
185	0.00575781810437872\\
186	0.00575559352535908\\
187	0.00575334759471802\\
188	0.00575108007995981\\
189	0.00574879074130821\\
190	0.00574647933116541\\
191	0.00574414559352756\\
192	0.00574178926335378\\
193	0.00573941006588399\\
194	0.00573700771590237\\
195	0.00573458191694115\\
196	0.0057321323604202\\
197	0.00572965872471653\\
198	0.00572716067415846\\
199	0.00572463785793738\\
200	0.00572208990893019\\
201	0.00571951644242463\\
202	0.00571691705473904\\
203	0.00571429132172704\\
204	0.00571163879715717\\
205	0.00570895901095526\\
206	0.00570625146729825\\
207	0.00570351564254499\\
208	0.00570075098298861\\
209	0.00569795690241481\\
210	0.0056951327794465\\
211	0.00569227795465521\\
212	0.00568939172741659\\
213	0.00568647335248552\\
214	0.00568352203626303\\
215	0.00568053693272564\\
216	0.00567751713898401\\
217	0.00567446169043394\\
218	0.00567136955546088\\
219	0.00566823962965385\\
220	0.00566507072948129\\
221	0.0056618615853776\\
222	0.00565861083418448\\
223	0.00565531701088754\\
224	0.00565197853958499\\
225	0.00564859372362192\\
226	0.00564516073482158\\
227	0.00564167760174593\\
228	0.00563814219692024\\
229	0.00563455222296846\\
230	0.00563090519763434\\
231	0.00562719843773486\\
232	0.00562342904227852\\
233	0.00561959387539725\\
234	0.00561568954947589\\
235	0.00561171238065216\\
236	0.00560765838085181\\
237	0.00560352323743865\\
238	0.00559930228887502\\
239	0.0055949904999797\\
240	0.00559058243743392\\
241	0.00558607224639563\\
242	0.00558145362942611\\
243	0.00557671982937909\\
244	0.00557186361849583\\
245	0.00556687729672331\\
246	0.00556175270328639\\
247	0.00555648124686407\\
248	0.0055510539614342\\
249	0.005545461597087\\
250	0.00553969475807989\\
251	0.00553374410485552\\
252	0.00552760064720504\\
253	0.00552125616713517\\
254	0.00551470375091368\\
255	0.00550793853492297\\
256	0.00550095871514837\\
257	0.00549376689797336\\
258	0.00548637190179406\\
259	0.00547880412473265\\
260	0.0054711615979969\\
261	0.00546344384532699\\
262	0.00545565041652165\\
263	0.00544778089136691\\
264	0.00543983488411992\\
265	0.00543181204862253\\
266	0.00542371208001076\\
267	0.00541553463770604\\
268	0.00540727939736451\\
269	0.00539894607785779\\
270	0.0053905344489206\\
271	0.00538204433742959\\
272	0.00537347563429518\\
273	0.00536482830200602\\
274	0.0053561023828637\\
275	0.00534729800793862\\
276	0.00533841540677069\\
277	0.00532945491783\\
278	0.00532041699977937\\
279	0.00531130224385107\\
280	0.00530211138967355\\
281	0.00529284536486983\\
282	0.00528350523427824\\
283	0.00527409219089737\\
284	0.00526460765797136\\
285	0.00525505332659508\\
286	0.00524543121741848\\
287	0.00523574385418239\\
288	0.00522599423971317\\
289	0.00521618584396532\\
290	0.00520632265207895\\
291	0.00519640921234668\\
292	0.00518645069507146\\
293	0.00517645328602592\\
294	0.00516642397989371\\
295	0.00515637066445414\\
296	0.00514630221035993\\
297	0.00513622856570469\\
298	0.00512616085401361\\
299	0.00511611147354042\\
300	0.00510609419480906\\
301	0.00509612425244915\\
302	0.00508621842822103\\
303	0.00507639513606668\\
304	0.00506667464224506\\
305	0.00505707780951913\\
306	0.0050476264979348\\
307	0.00503834342573031\\
308	0.00502925168468738\\
309	0.00502037401474721\\
310	0.00501173175224369\\
311	0.00500334333618516\\
312	0.00499522220916615\\
313	0.00498737383552014\\
314	0.00497979285788452\\
315	0.00497245932733467\\
316	0.00496517032215807\\
317	0.00495789415233938\\
318	0.00495063539936947\\
319	0.00494339888326918\\
320	0.00493618966568305\\
321	0.00492901305487639\\
322	0.00492187460699136\\
323	0.00491478012219509\\
324	0.00490773563673702\\
325	0.00490074740982068\\
326	0.00489382190401122\\
327	0.00488696575826997\\
328	0.00488018574497965\\
329	0.00487348871923138\\
330	0.00486688156124533\\
331	0.0048603711051543\\
332	0.00485396405196309\\
333	0.00484766686390582\\
334	0.00484148564706391\\
335	0.004835426041604\\
336	0.00482949307007449\\
337	0.00482369096615793\\
338	0.00481802298469154\\
339	0.00481249119501329\\
340	0.00480709626105165\\
341	0.0048018371760404\\
342	0.00479671096199709\\
343	0.0047917124319064\\
344	0.00478683399505208\\
345	0.00478206554455709\\
346	0.00477739448361568\\
347	0.00477280597085663\\
348	0.00476828349815441\\
349	0.00476380995973181\\
350	0.00475937041137946\\
351	0.00475496584128219\\
352	0.00475059704889023\\
353	0.00474626461898813\\
354	0.00474196889476297\\
355	0.00473770995058193\\
356	0.00473348756479409\\
357	0.00472930119310447\\
358	0.00472514994262298\\
359	0.00472103254748688\\
360	0.00471694734743485\\
361	0.00471289227064709\\
362	0.00470886482249196\\
363	0.00470486208307431\\
364	0.00470088071707579\\
365	0.00469691699727165\\
366	0.00469296684452168\\
367	0.00468902588702934\\
368	0.0046850895413815\\
369	0.00468115311715026\\
370	0.00467721194541925\\
371	0.00467326152914154\\
372	0.00466929770959023\\
373	0.00466531683699681\\
374	0.00466131592437641\\
375	0.00465729275009499\\
376	0.00465324578873874\\
377	0.00464917342082965\\
378	0.00464507393782156\\
379	0.00464094554882346\\
380	0.00463678638918679\\
381	0.00463259453106235\\
382	0.00462836799593486\\
383	0.00462410476907713\\
384	0.00461980281578399\\
385	0.00461546009911589\\
386	0.00461107459872934\\
387	0.0046066443301807\\
388	0.0046021673638745\\
389	0.00459764184259331\\
390	0.00459306599629965\\
391	0.00458843815268068\\
392	0.00458375674176487\\
393	0.00457902029295402\\
394	0.00457422742312289\\
395	0.00456937681524134\\
396	0.00456446718854771\\
397	0.00455949726408454\\
398	0.00455446576244348\\
399	0.00454937140691227\\
400	0.00454421292645538\\
401	0.00453898905842823\\
402	0.00453369855091595\\
403	0.00452834016458548\\
404	0.00452291267394101\\
405	0.00451741486788391\\
406	0.00451184554949909\\
407	0.00450620353502427\\
408	0.00450048765201002\\
409	0.00449469673674516\\
410	0.004488829631108\\
411	0.00448288517909952\\
412	0.00447686222341574\\
413	0.00447075960249672\\
414	0.00446457614851732\\
415	0.00445831068669633\\
416	0.00445196203510796\\
417	0.00444552900437213\\
418	0.00443901039722519\\
419	0.00443240500797667\\
420	0.00442571162186341\\
421	0.00441892901431833\\
422	0.0044120559501766\\
423	0.00440509118284664\\
424	0.00439803345347797\\
425	0.00439088149015869\\
426	0.00438363400717377\\
427	0.0043762897043491\\
428	0.0043688472664947\\
429	0.00436130536294375\\
430	0.00435366264716264\\
431	0.00434591775638611\\
432	0.0043380693112426\\
433	0.00433011591537352\\
434	0.00432205615505058\\
435	0.00431388859879552\\
436	0.00430561179700604\\
437	0.00429722428159108\\
438	0.0042887245656184\\
439	0.00428011114297546\\
440	0.00427138248804422\\
441	0.00426253705538868\\
442	0.00425357327945279\\
443	0.00424448957426615\\
444	0.00423528433315402\\
445	0.00422595592844986\\
446	0.00421650271120965\\
447	0.00420692301092892\\
448	0.00419721513526305\\
449	0.0041873773697511\\
450	0.00417740797754377\\
451	0.00416730519913552\\
452	0.00415706725210115\\
453	0.0041466923308371\\
454	0.00413617860630727\\
455	0.00412552422579394\\
456	0.00411472731265397\\
457	0.00410378596608072\\
458	0.00409269826087246\\
459	0.00408146224720821\\
460	0.00407007595043173\\
461	0.00405853737084471\\
462	0.00404684448351028\\
463	0.00403499523806797\\
464	0.00402298755856135\\
465	0.00401081934327958\\
466	0.00399848846461431\\
467	0.00398599276893367\\
468	0.00397333007647451\\
469	0.00396049818125501\\
470	0.00394749485100917\\
471	0.00393431782714499\\
472	0.00392096482472816\\
473	0.00390743353249293\\
474	0.00389372161288206\\
475	0.0038798267021169\\
476	0.00386574641029963\\
477	0.00385147832154774\\
478	0.00383701999416199\\
479	0.00382236896082727\\
480	0.00380752272884562\\
481	0.00379247878039903\\
482	0.00377723457283913\\
483	0.00376178753899777\\
484	0.00374613508751146\\
485	0.00373027460314929\\
486	0.00371420344713001\\
487	0.0036979189574103\\
488	0.00368141844892043\\
489	0.00366469921371667\\
490	0.00364775852101219\\
491	0.0036305936170375\\
492	0.00361320172467033\\
493	0.00359558004275919\\
494	0.00357772574504836\\
495	0.00355963597859005\\
496	0.00354130786150495\\
497	0.00352273847992189\\
498	0.00350392488389229\\
499	0.00348486408203332\\
500	0.00346555303460428\\
501	0.00344598864466502\\
502	0.00342616774689879\\
503	0.0034060870936082\\
504	0.00338574333730855\\
505	0.00336513300924976\\
506	0.00334425249309668\\
507	0.0033230979928903\\
508	0.00330166549430296\\
509	0.00327995071809673\\
510	0.00325794906454704\\
511	0.00323565558805765\\
512	0.00321306503419404\\
513	0.00319017544354297\\
514	0.00316698274648975\\
515	0.00314347361015554\\
516	0.00311963450968671\\
517	0.00309544530579971\\
518	0.00307089148905672\\
519	0.00304596426033527\\
520	0.00302065473786325\\
521	0.00299495569260285\\
522	0.00296886135966797\\
523	0.00294236719344355\\
524	0.00291547045767632\\
525	0.00288817079499245\\
526	0.00286047088236643\\
527	0.00283237770487081\\
528	0.00280390829254135\\
529	0.00277516730460763\\
530	0.00274609835995322\\
531	0.00271660490486484\\
532	0.00268653048771086\\
533	0.00265582621920599\\
534	0.00262444011296189\\
535	0.00259232902272056\\
536	0.0025594459900875\\
537	0.00252574097051207\\
538	0.00249115886685006\\
539	0.00245563912095192\\
540	0.00241911485691729\\
541	0.00238151650599522\\
542	0.00234276913476679\\
543	0.00230278422282756\\
544	0.00226144060937377\\
545	0.00222094444173137\\
546	0.00218284040994203\\
547	0.00214486939655888\\
548	0.00210633730549874\\
549	0.00206719310578365\\
550	0.00202744380742546\\
551	0.0019871081064355\\
552	0.00194621198469961\\
553	0.00190478906376779\\
554	0.00186288216891816\\
555	0.0018205457624277\\
556	0.0017778495907487\\
557	0.00173488030913016\\
558	0.00169240265022286\\
559	0.00165074091200184\\
560	0.0016087876624426\\
561	0.00156654726965046\\
562	0.00152404814044937\\
563	0.00148131883357367\\
564	0.00143838736743784\\
565	0.00139539549877649\\
566	0.00135261018668069\\
567	0.00130946073240556\\
568	0.001265963685663\\
569	0.001222137707383\\
570	0.00117800354514696\\
571	0.0011335841768572\\
572	0.00108890495847541\\
573	0.00104399377313653\\
574	0.000998881179097747\\
575	0.000953600553169377\\
576	0.00090818822522013\\
577	0.000862683598018209\\
578	0.000817129244997436\\
579	0.000771570976441667\\
580	0.000726057861961392\\
581	0.000680642193874463\\
582	0.000635379372066651\\
583	0.000590327685971717\\
584	0.000545547963424768\\
585	0.000501103049522702\\
586	0.000457057072276927\\
587	0.000413474448977722\\
588	0.000370418597003136\\
589	0.000327950360562704\\
590	0.000286126318718799\\
591	0.000244997582319803\\
592	0.000204610904245313\\
593	0.000165017211836692\\
594	0.000126301460681321\\
595	8.88161203105196e-05\\
596	5.34134895574392e-05\\
597	2.21100055488405e-05\\
598	0\\
599	0\\
600	0\\
};
\addplot [color=mycolor1,solid,forget plot]
  table[row sep=crcr]{%
1	0.00590251934398063\\
2	0.00590245528621135\\
3	0.00590238985445012\\
4	0.0059023230203767\\
5	0.00590225475513416\\
6	0.00590218502932082\\
7	0.00590211381298212\\
8	0.00590204107560234\\
9	0.00590196678609648\\
10	0.00590189091280194\\
11	0.00590181342347044\\
12	0.00590173428525967\\
13	0.00590165346472526\\
14	0.00590157092781229\\
15	0.00590148663984745\\
16	0.00590140056553063\\
17	0.005901312668927\\
18	0.00590122291345906\\
19	0.00590113126189837\\
20	0.0059010376763578\\
21	0.00590094211828385\\
22	0.00590084454844871\\
23	0.00590074492694275\\
24	0.00590064321316698\\
25	0.00590053936582584\\
26	0.00590043334291956\\
27	0.00590032510173767\\
28	0.00590021459885176\\
29	0.00590010179010885\\
30	0.00589998663062511\\
31	0.00589986907477924\\
32	0.00589974907620682\\
33	0.00589962658779437\\
34	0.00589950156167406\\
35	0.00589937394921843\\
36	0.00589924370103549\\
37	0.00589911076696443\\
38	0.00589897509607151\\
39	0.00589883663664628\\
40	0.0058986953361982\\
41	0.00589855114145387\\
42	0.00589840399835472\\
43	0.00589825385205506\\
44	0.00589810064692058\\
45	0.00589794432652779\\
46	0.00589778483366375\\
47	0.00589762211032613\\
48	0.00589745609772473\\
49	0.00589728673628285\\
50	0.00589711396563975\\
51	0.00589693772465387\\
52	0.00589675795140676\\
53	0.00589657458320775\\
54	0.00589638755659954\\
55	0.00589619680736474\\
56	0.00589600227053305\\
57	0.00589580388038979\\
58	0.00589560157048523\\
59	0.0058953952736447\\
60	0.00589518492198015\\
61	0.00589497044690265\\
62	0.00589475177913608\\
63	0.0058945288487318\\
64	0.00589430158508479\\
65	0.00589406991695089\\
66	0.00589383377246554\\
67	0.00589359307916327\\
68	0.0058933477639994\\
69	0.00589309775337223\\
70	0.00589284297314741\\
71	0.00589258334868332\\
72	0.00589231880485801\\
73	0.00589204926609763\\
74	0.00589177465640655\\
75	0.00589149489939882\\
76	0.00589120991833141\\
77	0.00589091963613875\\
78	0.00589062397546922\\
79	0.00589032285872279\\
80	0.00589001620809103\\
81	0.00588970394559791\\
82	0.00588938599314266\\
83	0.0058890622725441\\
84	0.00588873270558701\\
85	0.00588839721406927\\
86	0.00588805571985134\\
87	0.00588770814490695\\
88	0.00588735441137535\\
89	0.00588699444161509\\
90	0.0058866281582593\\
91	0.00588625548427224\\
92	0.0058858763430074\\
93	0.00588549065826699\\
94	0.0058850983543624\\
95	0.00588469935617632\\
96	0.00588429358922564\\
97	0.00588388097972579\\
98	0.00588346145465596\\
99	0.00588303494182551\\
100	0.00588260136994127\\
101	0.00588216066867578\\
102	0.00588171276873639\\
103	0.00588125760193554\\
104	0.00588079510126159\\
105	0.0058803252009507\\
106	0.00587984783655952\\
107	0.0058793629450391\\
108	0.00587887046480937\\
109	0.00587837033583518\\
110	0.00587786249970313\\
111	0.00587734689969997\\
112	0.00587682348089286\\
113	0.00587629219021066\\
114	0.00587575297652769\\
115	0.00587520579074966\\
116	0.00587465058590212\\
117	0.00587408731722169\\
118	0.0058735159422506\\
119	0.00587293642093432\\
120	0.00587234871572358\\
121	0.00587175279167982\\
122	0.00587114861658496\\
123	0.0058705361610553\\
124	0.00586991539865931\\
125	0.00586928630603887\\
126	0.00586864886303308\\
127	0.00586800305280335\\
128	0.00586734886195822\\
129	0.00586668628067415\\
130	0.0058660153028101\\
131	0.00586533592600947\\
132	0.00586464815178444\\
133	0.00586395198557304\\
134	0.00586324743675978\\
135	0.00586253451864659\\
136	0.00586181324835835\\
137	0.005861083646664\\
138	0.00586034573769071\\
139	0.00585959954850546\\
140	0.00585884510853362\\
141	0.00585808244878297\\
142	0.00585731160083764\\
143	0.00585653259558932\\
144	0.00585574546167559\\
145	0.0058549502236067\\
146	0.00585414689958255\\
147	0.00585333549903554\\
148	0.00585251601999084\\
149	0.00585168844641513\\
150	0.00585085274580252\\
151	0.00585000886715439\\
152	0.00584915673875567\\
153	0.00584829627335727\\
154	0.00584742738093528\\
155	0.00584654996957622\\
156	0.00584566394539258\\
157	0.00584476921243401\\
158	0.0058438656725945\\
159	0.00584295322551465\\
160	0.00584203176847918\\
161	0.00584110119630936\\
162	0.00584016140125032\\
163	0.00583921227285307\\
164	0.00583825369785029\\
165	0.00583728556002674\\
166	0.00583630774008337\\
167	0.00583532011549491\\
168	0.00583432256036083\\
169	0.00583331494524959\\
170	0.0058322971370354\\
171	0.00583126899872801\\
172	0.00583023038929416\\
173	0.00582918116347169\\
174	0.00582812117157468\\
175	0.0058270502592907\\
176	0.00582596826746862\\
177	0.00582487503189767\\
178	0.00582377038307673\\
179	0.00582265414597404\\
180	0.00582152613977678\\
181	0.00582038617763003\\
182	0.00581923406636559\\
183	0.00581806960621902\\
184	0.00581689259053609\\
185	0.00581570280546712\\
186	0.00581450002964959\\
187	0.0058132840338785\\
188	0.0058120545807642\\
189	0.00581081142437762\\
190	0.00580955430988192\\
191	0.00580828297315126\\
192	0.00580699714037604\\
193	0.00580569652765392\\
194	0.00580438084056727\\
195	0.00580304977374607\\
196	0.00580170301041663\\
197	0.00580034022193596\\
198	0.00579896106731123\\
199	0.00579756519270495\\
200	0.00579615223092535\\
201	0.00579472180090243\\
202	0.00579327350714958\\
203	0.0057918069392109\\
204	0.0057903216710948\\
205	0.0057888172606946\\
206	0.00578729324919572\\
207	0.00578574916047149\\
208	0.00578418450046743\\
209	0.00578259875657595\\
210	0.00578099139700239\\
211	0.00577936187012468\\
212	0.00577770960384813\\
213	0.00577603400495866\\
214	0.00577433445847713\\
215	0.00577261032701884\\
216	0.00577086095016265\\
217	0.00576908564383538\\
218	0.00576728369971782\\
219	0.00576545438468046\\
220	0.00576359694025803\\
221	0.00576171058217476\\
222	0.00575979449993266\\
223	0.00575784785647946\\
224	0.00575586978797509\\
225	0.00575385940367807\\
226	0.00575181578597957\\
227	0.00574973799061557\\
228	0.00574762504709556\\
229	0.00574547595939447\\
230	0.00574328970696595\\
231	0.00574106524615232\\
232	0.00573880151205814\\
233	0.00573649742074274\\
234	0.00573415187039852\\
235	0.00573176374606769\\
236	0.00572933192429559\\
237	0.00572685527840376\\
238	0.00572433268484067\\
239	0.00572176303077739\\
240	0.00571914522312208\\
241	0.00571647819914063\\
242	0.00571376093887343\\
243	0.0057109924795355\\
244	0.00570817193206658\\
245	0.00570529849995605\\
246	0.00570237150039545\\
247	0.00569939038768694\\
248	0.00569635477865656\\
249	0.00569326447956258\\
250	0.00569011951368892\\
251	0.00568692014856976\\
252	0.00568366692062209\\
253	0.00568036064986266\\
254	0.00567700244381534\\
255	0.0056735936822851\\
256	0.00567013597217538\\
257	0.00566663105804775\\
258	0.00566308066957925\\
259	0.00565948625958952\\
260	0.00565584841263152\\
261	0.00565216609102806\\
262	0.00564843820419253\\
263	0.00564466360485588\\
264	0.00564084108475608\\
265	0.00563696936879612\\
266	0.00563304710399896\\
267	0.00562907285885337\\
268	0.0056250451197385\\
269	0.00562096228485869\\
270	0.00561682265740823\\
271	0.00561262443812102\\
272	0.00560836571714789\\
273	0.00560404446520052\\
274	0.00559965852389711\\
275	0.00559520559524481\\
276	0.00559068323019928\\
277	0.00558608881626838\\
278	0.00558141956421164\\
279	0.00557667249411617\\
280	0.00557184442119394\\
281	0.00556693193155663\\
282	0.0055619313597477\\
283	0.00555683877510902\\
284	0.00555164996184468\\
285	0.00554636039859688\\
286	0.00554096524159194\\
287	0.0055354592881318\\
288	0.00552983693876181\\
289	0.00552409216144583\\
290	0.00551821845284976\\
291	0.00551220880162227\\
292	0.00550605567548976\\
293	0.0054997509562657\\
294	0.00549328589104347\\
295	0.00548665104689526\\
296	0.00547983626585418\\
297	0.00547283062171438\\
298	0.00546562238080637\\
299	0.00545819896974385\\
300	0.00545054695429415\\
301	0.00544265203515902\\
302	0.00543449906859606\\
303	0.0054260721179611\\
304	0.00541735445952625\\
305	0.00540832879420024\\
306	0.00539897752471909\\
307	0.00538928312778538\\
308	0.00537922868237661\\
309	0.00536879860955287\\
310	0.00535797969309182\\
311	0.00534676247257293\\
312	0.00533514313645609\\
313	0.00532312615580551\\
314	0.00531072776264848\\
315	0.00529798033230431\\
316	0.00528509108377234\\
317	0.00527209809328364\\
318	0.00525900434016779\\
319	0.00524581325619099\\
320	0.00523252881885486\\
321	0.00521915545066693\\
322	0.00520569811781198\\
323	0.00519216247741685\\
324	0.00517855496125716\\
325	0.00516488286843877\\
326	0.00515115446023739\\
327	0.00513737901426608\\
328	0.00512356725396317\\
329	0.00510973143323508\\
330	0.00509588529630315\\
331	0.0050820442322894\\
332	0.0050682254535027\\
333	0.00505444827216694\\
334	0.00504073406369509\\
335	0.00502710510947419\\
336	0.00501358570886181\\
337	0.00500020219050444\\
338	0.00498698284779547\\
339	0.00497395774552717\\
340	0.00496115823720337\\
341	0.0049486180502011\\
342	0.00493637445751942\\
343	0.00492446539907476\\
344	0.00491292833816373\\
345	0.00490179860145864\\
346	0.00489110702351452\\
347	0.00488087665843246\\
348	0.0048711182416854\\
349	0.00486182396433311\\
350	0.00485293806156923\\
351	0.00484412957686245\\
352	0.00483540736418802\\
353	0.00482678051127734\\
354	0.00481825828917079\\
355	0.00480985007568382\\
356	0.0048015652690991\\
357	0.00479341318551467\\
358	0.00478540296815216\\
359	0.00477754347686151\\
360	0.00476984314266119\\
361	0.00476230980045775\\
362	0.00475495049790426\\
363	0.0047477712358786\\
364	0.00474077662442931\\
365	0.00473396956968372\\
366	0.00472735093479879\\
367	0.00472091918377157\\
368	0.00471467002319526\\
369	0.00470859606602662\\
370	0.00470268655422674\\
371	0.00469692719506652\\
372	0.0046913001824281\\
373	0.00468578451720068\\
374	0.00468035678206487\\
375	0.00467499258685562\\
376	0.00466967071569014\\
377	0.00466439094468745\\
378	0.00465915263878804\\
379	0.0046539547154853\\
380	0.00464879561007354\\
381	0.00464367324380538\\
382	0.00463858499881921\\
383	0.00463352770164772\\
384	0.0046284976171737\\
385	0.00462349045605455\\
386	0.00461850139899356\\
387	0.00461352514149374\\
388	0.00460855596279575\\
389	0.00460358782241647\\
390	0.00459861448713568\\
391	0.00459362968977339\\
392	0.00458862731836186\\
393	0.00458360162990402\\
394	0.00457854747576893\\
395	0.00457346051477442\\
396	0.0045683373742081\\
397	0.00456317569562472\\
398	0.00455797318520075\\
399	0.0045527274656\\
400	0.00454743609094299\\
401	0.0045420965642578\\
402	0.00453670635728405\\
403	0.00453126293235265\\
404	0.00452576376587907\\
405	0.00452020637278369\\
406	0.00451458833088573\\
407	0.00450890730401451\\
408	0.004503161062266\\
409	0.00449734749752988\\
410	0.00449146463220514\\
411	0.00448551061897748\\
412	0.00447948372980187\\
413	0.00447338233305419\\
414	0.00446720485950733\\
415	0.00446094976085365\\
416	0.00445461549458166\\
417	0.00444820052744019\\
418	0.00444170333848689\\
419	0.00443512242158188\\
420	0.00442845628718682\\
421	0.0044217034633394\\
422	0.00441486249569465\\
423	0.00440793194656331\\
424	0.00440091039293376\\
425	0.0043937964235418\\
426	0.00438658863515213\\
427	0.00437928562833206\\
428	0.00437188600312198\\
429	0.00436438835511852\\
430	0.00435679127254411\\
431	0.00434909333481376\\
432	0.0043412931121523\\
433	0.00433338916506646\\
434	0.00432538004367972\\
435	0.00431726428694371\\
436	0.00430904042174816\\
437	0.00430070696195715\\
438	0.00429226240740638\\
439	0.0042837052429001\\
440	0.00427503393724809\\
441	0.00426624694238005\\
442	0.00425734269256671\\
443	0.0042483196037618\\
444	0.00423917607305786\\
445	0.00422991047822245\\
446	0.00422052117725603\\
447	0.00421100650794284\\
448	0.00420136478739952\\
449	0.00419159431162678\\
450	0.00418169335506834\\
451	0.00417166017018127\\
452	0.00416149298702058\\
453	0.0041511900128395\\
454	0.00414074943170557\\
455	0.00413016940413091\\
456	0.00411944806671382\\
457	0.00410858353178745\\
458	0.00409757388707241\\
459	0.00408641719532977\\
460	0.00407511149401484\\
461	0.00406365479493196\\
462	0.00405204508389014\\
463	0.00404028032035976\\
464	0.00402835843713015\\
465	0.0040162773399675\\
466	0.00400403490727281\\
467	0.00399162898973887\\
468	0.00397905741000597\\
469	0.00396631796231549\\
470	0.00395340841216087\\
471	0.00394032649593539\\
472	0.00392706992057615\\
473	0.00391363636320374\\
474	0.0039000234707567\\
475	0.00388622885961998\\
476	0.00387225011524586\\
477	0.00385808479176625\\
478	0.00384373041159462\\
479	0.00382918446501518\\
480	0.00381444440975717\\
481	0.00379950767055129\\
482	0.0037843716386642\\
483	0.00376903367140757\\
484	0.00375349109161576\\
485	0.00373774118708615\\
486	0.00372178120997454\\
487	0.00370560837613651\\
488	0.0036892198644035\\
489	0.00367261281578044\\
490	0.00365578433254883\\
491	0.00363873147725569\\
492	0.00362145127156468\\
493	0.00360394069494109\\
494	0.00358619668313547\\
495	0.00356821612642407\\
496	0.00354999586755376\\
497	0.00353153269932913\\
498	0.00351282336176382\\
499	0.00349386453870145\\
500	0.00347465285378928\\
501	0.00345518486566005\\
502	0.00343545706214343\\
503	0.00341546585328494\\
504	0.0033952075628955\\
505	0.00337467841828577\\
506	0.00335387453775167\\
507	0.0033327919152645\\
508	0.00331142640167784\\
509	0.00328977368157769\\
510	0.00326782924466955\\
511	0.00324558834792906\\
512	0.00322304596628121\\
513	0.00320019673530399\\
514	0.0031770349151287\\
515	0.00315355442308624\\
516	0.0031297488046761\\
517	0.00310561647545263\\
518	0.00308114887875852\\
519	0.00305633122473592\\
520	0.00303114847792327\\
521	0.00300557837230024\\
522	0.00297960563559337\\
523	0.00295321995832703\\
524	0.00292641113572341\\
525	0.00289917060189006\\
526	0.00287149176898304\\
527	0.00284336974872904\\
528	0.00281480202683791\\
529	0.00278578925927946\\
530	0.00275633684383654\\
531	0.00272647973443768\\
532	0.0026962839166467\\
533	0.00266568859564292\\
534	0.00263458148689847\\
535	0.00260281655666955\\
536	0.00257033981782914\\
537	0.00253709333153114\\
538	0.00250302847963458\\
539	0.00246809247793383\\
540	0.00243223002778302\\
541	0.00239538281627027\\
542	0.00235749417763278\\
543	0.00231849814622264\\
544	0.00227831752274385\\
545	0.00223686152525975\\
546	0.00219399064469931\\
547	0.00215238099906579\\
548	0.00211316779448829\\
549	0.00207391789185168\\
550	0.00203412739027735\\
551	0.00199374047816936\\
552	0.0019527641731985\\
553	0.00191122226322775\\
554	0.00186914804010064\\
555	0.00182658423727254\\
556	0.00178358468438323\\
557	0.00174021795680351\\
558	0.00169656963512825\\
559	0.00165305721170727\\
560	0.0016107056490028\\
561	0.00156824435218013\\
562	0.00152551367306318\\
563	0.0014825422894469\\
564	0.00143936045003831\\
565	0.0013959977150654\\
566	0.00135261153787214\\
567	0.00130946074191574\\
568	0.00126596368633984\\
569	0.00122213770752465\\
570	0.00117800354519282\\
571	0.0011335841768772\\
572	0.00108890495848566\\
573	0.00104399377314213\\
574	0.000998881179100837\\
575	0.000953600553171082\\
576	0.000908188225221049\\
577	0.00086268359801868\\
578	0.000817129244997649\\
579	0.000771570976441746\\
580	0.000726057861961415\\
581	0.000680642193874474\\
582	0.000635379372066665\\
583	0.000590327685971726\\
584	0.000545547963424773\\
585	0.000501103049522709\\
586	0.000457057072276937\\
587	0.000413474448977734\\
588	0.000370418597003141\\
589	0.000327950360562709\\
590	0.000286126318718803\\
591	0.000244997582319807\\
592	0.000204610904245316\\
593	0.000165017211836693\\
594	0.000126301460681324\\
595	8.8816120310521e-05\\
596	5.341348955744e-05\\
597	2.21100055488408e-05\\
598	0\\
599	0\\
600	0\\
};
\addplot [color=mycolor2,solid,forget plot]
  table[row sep=crcr]{%
1	0.00590289298809553\\
2	0.00590283980942159\\
3	0.00590278557316541\\
4	0.00590273026010499\\
5	0.00590267385073519\\
6	0.00590261632526622\\
7	0.00590255766362219\\
8	0.0059024978454396\\
9	0.00590243685006619\\
10	0.00590237465655978\\
11	0.00590231124368723\\
12	0.00590224658992364\\
13	0.00590218067345159\\
14	0.00590211347216063\\
15	0.00590204496364685\\
16	0.00590197512521264\\
17	0.00590190393386664\\
18	0.0059018313663239\\
19	0.00590175739900615\\
20	0.00590168200804236\\
21	0.00590160516926949\\
22	0.00590152685823341\\
23	0.00590144705018995\\
24	0.00590136572010657\\
25	0.00590128284266371\\
26	0.00590119839225697\\
27	0.00590111234299898\\
28	0.00590102466872212\\
29	0.00590093534298097\\
30	0.00590084433905542\\
31	0.00590075162995403\\
32	0.00590065718841745\\
33	0.00590056098692247\\
34	0.00590046299768619\\
35	0.00590036319267056\\
36	0.00590026154358736\\
37	0.0059001580219034\\
38	0.00590005259884612\\
39	0.00589994524540961\\
40	0.00589983593236088\\
41	0.00589972463024672\\
42	0.00589961130940069\\
43	0.0058994959399507\\
44	0.00589937849182696\\
45	0.00589925893477018\\
46	0.00589913723834041\\
47	0.00589901337192616\\
48	0.00589888730475396\\
49	0.0058987590058984\\
50	0.00589862844429249\\
51	0.00589849558873845\\
52	0.00589836040791912\\
53	0.00589822287040956\\
54	0.0058980829446892\\
55	0.00589794059915432\\
56	0.00589779580213108\\
57	0.00589764852188881\\
58	0.00589749872665372\\
59	0.00589734638462316\\
60	0.0058971914639801\\
61	0.00589703393290788\\
62	0.00589687375960556\\
63	0.00589671091230352\\
64	0.00589654535927913\\
65	0.00589637706887303\\
66	0.0058962060095055\\
67	0.00589603214969311\\
68	0.00589585545806554\\
69	0.00589567590338246\\
70	0.00589549345455087\\
71	0.00589530808064215\\
72	0.00589511975090933\\
73	0.00589492843480444\\
74	0.00589473410199556\\
75	0.00589453672238405\\
76	0.00589433626612137\\
77	0.00589413270362589\\
78	0.0058939260055991\\
79	0.00589371614304192\\
80	0.00589350308727014\\
81	0.00589328680992961\\
82	0.00589306728301068\\
83	0.00589284447886212\\
84	0.0058926183702043\\
85	0.00589238893014128\\
86	0.00589215613217215\\
87	0.00589191995020146\\
88	0.00589168035854821\\
89	0.00589143733195379\\
90	0.00589119084558862\\
91	0.00589094087505732\\
92	0.00589068739640216\\
93	0.00589043038610515\\
94	0.00589016982108816\\
95	0.00588990567871135\\
96	0.00588963793676946\\
97	0.0058893665734861\\
98	0.00588909156750609\\
99	0.00588881289788516\\
100	0.00588853054407754\\
101	0.00588824448592126\\
102	0.00588795470362059\\
103	0.00588766117772615\\
104	0.00588736388911231\\
105	0.00588706281895193\\
106	0.0058867579486883\\
107	0.00588644926000442\\
108	0.00588613673478923\\
109	0.00588582035510137\\
110	0.00588550010312994\\
111	0.0058851759611525\\
112	0.00588484791149027\\
113	0.00588451593646074\\
114	0.00588418001832737\\
115	0.00588384013924701\\
116	0.00588349628121432\\
117	0.00588314842600411\\
118	0.00588279655511109\\
119	0.00588244064968738\\
120	0.00588208069047766\\
121	0.00588171665775221\\
122	0.00588134853123793\\
123	0.00588097629004714\\
124	0.00588059991260446\\
125	0.00588021937657165\\
126	0.00587983465877019\\
127	0.00587944573510217\\
128	0.00587905258046838\\
129	0.00587865516868465\\
130	0.00587825347239518\\
131	0.00587784746298339\\
132	0.00587743711047974\\
133	0.00587702238346651\\
134	0.00587660324897912\\
135	0.00587617967240413\\
136	0.00587575161737366\\
137	0.00587531904565667\\
138	0.00587488191704753\\
139	0.00587444018925267\\
140	0.0058739938177769\\
141	0.00587354275581134\\
142	0.00587308695412598\\
143	0.00587262636097098\\
144	0.00587216092199153\\
145	0.00587169058016306\\
146	0.00587121527575362\\
147	0.00587073494632138\\
148	0.00587024952675232\\
149	0.00586975894933288\\
150	0.00586926314382588\\
151	0.00586876203750937\\
152	0.00586825555561693\\
153	0.00586774362158243\\
154	0.00586722615700598\\
155	0.00586670308159826\\
156	0.0058661743131236\\
157	0.00586563976734193\\
158	0.00586509935794921\\
159	0.00586455299651673\\
160	0.00586400059242919\\
161	0.0058634420528215\\
162	0.00586287728251447\\
163	0.00586230618394927\\
164	0.00586172865712087\\
165	0.00586114459951034\\
166	0.00586055390601624\\
167	0.00585995646888486\\
168	0.00585935217763975\\
169	0.00585874091901054\\
170	0.00585812257686064\\
171	0.00585749703211461\\
172	0.0058568641626848\\
173	0.00585622384339763\\
174	0.00585557594591947\\
175	0.00585492033868218\\
176	0.00585425688680876\\
177	0.00585358545203869\\
178	0.00585290589265376\\
179	0.00585221806340392\\
180	0.00585152181543385\\
181	0.0058508169962099\\
182	0.005850103449448\\
183	0.00584938101504254\\
184	0.00584864952899625\\
185	0.00584790882335182\\
186	0.00584715872612471\\
187	0.00584639906123796\\
188	0.00584562964845907\\
189	0.00584485030333891\\
190	0.00584406083715339\\
191	0.0058432610568476\\
192	0.00584245076498301\\
193	0.00584162975968797\\
194	0.00584079783461143\\
195	0.00583995477888071\\
196	0.00583910037706289\\
197	0.00583823440913059\\
198	0.00583735665043217\\
199	0.00583646687166668\\
200	0.00583556483886362\\
201	0.0058346503133681\\
202	0.00583372305183121\\
203	0.0058327828062062\\
204	0.00583182932375047\\
205	0.00583086234703337\\
206	0.00582988161395089\\
207	0.00582888685774623\\
208	0.00582787780703734\\
209	0.00582685418585128\\
210	0.00582581571366544\\
211	0.00582476210545613\\
212	0.0058236930717545\\
213	0.00582260831870985\\
214	0.00582150754816091\\
215	0.00582039045771465\\
216	0.00581925674083349\\
217	0.00581810608693054\\
218	0.00581693818147348\\
219	0.00581575270609708\\
220	0.00581454933872472\\
221	0.00581332775369906\\
222	0.00581208762192244\\
223	0.00581082861100702\\
224	0.00580955038543486\\
225	0.00580825260672872\\
226	0.00580693493363345\\
227	0.0058055970223081\\
228	0.00580423852652999\\
229	0.00580285909790977\\
230	0.00580145838611824\\
231	0.00580003603911752\\
232	0.00579859170336768\\
233	0.00579712502394831\\
234	0.0057956356448991\\
235	0.00579412320947861\\
236	0.00579258736038246\\
237	0.0057910277399373\\
238	0.00578944399025673\\
239	0.00578783575334226\\
240	0.00578620267110513\\
241	0.00578454438528075\\
242	0.00578286053719846\\
243	0.005781150767361\\
244	0.00577941471477828\\
245	0.00577765201598826\\
246	0.00577586230368574\\
247	0.00577404520486917\\
248	0.00577220033840756\\
249	0.00577032731193253\\
250	0.00576842571797302\\
251	0.00576649512919416\\
252	0.00576453509242091\\
253	0.00576254512176232\\
254	0.00576052469084514\\
255	0.00575847322438624\\
256	0.00575639008963429\\
257	0.0057542745884814\\
258	0.00575212595030449\\
259	0.00574994332611573\\
260	0.00574772580515928\\
261	0.00574547244434873\\
262	0.00574318226705595\\
263	0.00574085426181224\\
264	0.00573848738082318\\
265	0.00573608053819416\\
266	0.0057336326090271\\
267	0.00573114242822554\\
268	0.00572860878907956\\
269	0.00572603044179799\\
270	0.00572340609201073\\
271	0.00572073439924577\\
272	0.00571801397538682\\
273	0.00571524338311896\\
274	0.00571242113437221\\
275	0.00570954568877621\\
276	0.00570661545214432\\
277	0.0057036287750148\\
278	0.00570058395127895\\
279	0.00569747921684793\\
280	0.00569431274780722\\
281	0.00569108265911777\\
282	0.00568778700431337\\
283	0.00568442377512202\\
284	0.00568099090152077\\
285	0.00567748625235599\\
286	0.005673907634917\\
287	0.00567025279523492\\
288	0.00566651941947826\\
289	0.00566270513625548\\
290	0.00565880752048726\\
291	0.00565482409980961\\
292	0.00565075235857844\\
293	0.00564658974480492\\
294	0.00564233367934105\\
295	0.00563798156718042\\
296	0.00563353081119124\\
297	0.00562897882862025\\
298	0.00562432307071964\\
299	0.00561956104584457\\
300	0.0056146903463208\\
301	0.00560970867915591\\
302	0.00560461389963621\\
303	0.00559940404330536\\
304	0.0055940773734678\\
305	0.00558863242917835\\
306	0.00558306807195979\\
307	0.00557738353058685\\
308	0.00557157843999121\\
309	0.00556565286800116\\
310	0.00555960732095007\\
311	0.00555344271627002\\
312	0.00554716030791843\\
313	0.00554076152678147\\
314	0.00553424768685379\\
315	0.00552761954265008\\
316	0.00552087636392247\\
317	0.00551401375131407\\
318	0.0055070263749434\\
319	0.00549990849730939\\
320	0.00549265392265208\\
321	0.00548525595894545\\
322	0.00547770738267522\\
323	0.00547000039348945\\
324	0.00546212656341064\\
325	0.00545407678091484\\
326	0.00544584119101884\\
327	0.00543740916390555\\
328	0.00542876923151705\\
329	0.00541990900561854\\
330	0.00541081511062264\\
331	0.00540147312116253\\
332	0.00539186750342838\\
333	0.00538198152648837\\
334	0.00537179712695583\\
335	0.00536129497651668\\
336	0.0053504545039735\\
337	0.00533925394283664\\
338	0.00532767043787667\\
339	0.00531568024517913\\
340	0.00530325917823541\\
341	0.00529038316818771\\
342	0.00527702871586357\\
343	0.00526317376585702\\
344	0.0052487989607292\\
345	0.00523388935754593\\
346	0.0052184367554935\\
347	0.00520244283120165\\
348	0.00518592334981263\\
349	0.00516891387837378\\
350	0.00515149700577763\\
351	0.00513401777571863\\
352	0.00511648800722797\\
353	0.00509892097993325\\
354	0.0050813315129457\\
355	0.00506373666185014\\
356	0.00504615572162247\\
357	0.00502861053340402\\
358	0.00501112428391625\\
359	0.00499372163377343\\
360	0.0049764294647856\\
361	0.00495927686281922\\
362	0.00494229485614734\\
363	0.00492551797085755\\
364	0.00490898678737586\\
365	0.00489274506618564\\
366	0.00487683956578182\\
367	0.00486131966195579\\
368	0.00484623668923569\\
369	0.0048316428977439\\
370	0.00481758987942017\\
371	0.00480412626590053\\
372	0.00479129461440321\\
373	0.00477912694617862\\
374	0.00476763859951386\\
375	0.00475681984758731\\
376	0.00474658792067708\\
377	0.00473650845189604\\
378	0.00472659321758804\\
379	0.00471685387017981\\
380	0.00470730179998657\\
381	0.00469794797234422\\
382	0.00468880263451389\\
383	0.00467987500589663\\
384	0.00467117296451996\\
385	0.00466270269153025\\
386	0.00465446827488359\\
387	0.00464647127685829\\
388	0.00463871027524485\\
389	0.00463118039627614\\
390	0.00462387286196124\\
391	0.00461677459336521\\
392	0.00460986793314097\\
393	0.00460313057695795\\
394	0.00459653584704915\\
395	0.00459005349898099\\
396	0.00458365131539413\\
397	0.00457729784580636\\
398	0.00457098698920859\\
399	0.004564716294561\\
400	0.00455848264931132\\
401	0.00455228225476963\\
402	0.00454611061339167\\
403	0.00453996253161096\\
404	0.00453383214232365\\
405	0.00452771295147087\\
406	0.00452159791349731\\
407	0.00451547954042655\\
408	0.00450935004856175\\
409	0.00450320154540282\\
410	0.00449702625590228\\
411	0.00449081678246951\\
412	0.00448456638528331\\
413	0.00447826925709325\\
414	0.00447192074868736\\
415	0.00446551747447454\\
416	0.00445905650867842\\
417	0.00445253485727306\\
418	0.00444594948123969\\
419	0.0044392973225393\\
420	0.00443257533232113\\
421	0.0044257805006135\\
422	0.00441890988641957\\
423	0.00441196064676609\\
424	0.00440493006287965\\
425	0.00439781556128897\\
426	0.00439061472735234\\
427	0.00438332530857979\\
428	0.00437594520532298\\
429	0.00436847244719166\\
430	0.004360905155297\\
431	0.00435324149367675\\
432	0.00434547963732502\\
433	0.00433761777577844\\
434	0.00432965411601971\\
435	0.00432158688452897\\
436	0.00431341432832667\\
437	0.0043051347148801\\
438	0.00429674633079376\\
439	0.0042882474792748\\
440	0.00427963647646285\\
441	0.00427091164683658\\
442	0.00426207131805377\\
443	0.00425311381573304\\
444	0.00424403745881409\\
445	0.00423484055619014\\
446	0.00422552140519882\\
447	0.00421607829096292\\
448	0.00420650948556624\\
449	0.00419681324708218\\
450	0.00418698781848135\\
451	0.00417703142645265\\
452	0.00416694228017938\\
453	0.00415671857011792\\
454	0.00414635846682665\\
455	0.00413586011988973\\
456	0.00412522165696883\\
457	0.00411444118299726\\
458	0.00410351677950335\\
459	0.0040924465040181\\
460	0.00408122838949152\\
461	0.00406986044370007\\
462	0.00405834064865096\\
463	0.00404666695998803\\
464	0.00403483730640359\\
465	0.00402284958905925\\
466	0.00401070168101698\\
467	0.00399839142668\\
468	0.00398591664124104\\
469	0.00397327511013314\\
470	0.00396046458847795\\
471	0.00394748280052504\\
472	0.00393432743907793\\
473	0.00392099616490454\\
474	0.00390748660613059\\
475	0.00389379635761438\\
476	0.0038799229803005\\
477	0.00386586400054992\\
478	0.00385161690944312\\
479	0.00383717916205258\\
480	0.00382254817668082\\
481	0.00380772133405904\\
482	0.00379269597650179\\
483	0.0037774694070118\\
484	0.00376203888832935\\
485	0.00374640164191922\\
486	0.00373055484688748\\
487	0.00371449563881903\\
488	0.00369822110852578\\
489	0.00368172830069326\\
490	0.00366501421241198\\
491	0.00364807579157742\\
492	0.00363090993514002\\
493	0.00361351348718326\\
494	0.00359588323680422\\
495	0.00357801591576661\\
496	0.00355990819589096\\
497	0.00354155668614011\\
498	0.00352295792935061\\
499	0.00350410839855148\\
500	0.00348500449280061\\
501	0.00346564253245645\\
502	0.00344601875378601\\
503	0.00342612930279188\\
504	0.00340597022811835\\
505	0.00338553747286873\\
506	0.00336482686513437\\
507	0.00334383410699678\\
508	0.00332255476171738\\
509	0.00330098423877584\\
510	0.00327911777635182\\
511	0.00325695042085306\\
512	0.00323447700299123\\
513	0.00321169210960435\\
514	0.00318859004860879\\
515	0.00316516480378879\\
516	0.00314140998102817\\
517	0.00311731874960481\\
518	0.00309288381940531\\
519	0.00306809749335543\\
520	0.00304295159572964\\
521	0.00301744315427394\\
522	0.00299156088140972\\
523	0.00296528798680148\\
524	0.00293860735335825\\
525	0.00291149560615242\\
526	0.00288393397029197\\
527	0.00285591009287589\\
528	0.00282741204123086\\
529	0.00279842919924932\\
530	0.00276895377662112\\
531	0.0027389801810752\\
532	0.00270850589336545\\
533	0.00267753399434044\\
534	0.00264610521054587\\
535	0.00261426666192251\\
536	0.00258195392813423\\
537	0.00254905409497617\\
538	0.00251541422986857\\
539	0.00248097774510772\\
540	0.00244568021648018\\
541	0.00240947031427079\\
542	0.00237230079634654\\
543	0.00233412157856374\\
544	0.0022948750819944\\
545	0.00225449432082486\\
546	0.00221290115191247\\
547	0.00217000353342318\\
548	0.00212565426290639\\
549	0.0020826812330069\\
550	0.00204203122193233\\
551	0.00200145631970456\\
552	0.00196040402445154\\
553	0.00191878027993334\\
554	0.00187658805460734\\
555	0.00183385596587102\\
556	0.00179062510621794\\
557	0.00174694704334947\\
558	0.00170288693239941\\
559	0.00165852630810456\\
560	0.001613963685285\\
561	0.0015704372970884\\
562	0.00152743720717162\\
563	0.00148420001088167\\
564	0.00144074240204645\\
565	0.00139709605779737\\
566	0.00135329267834897\\
567	0.00130947115127387\\
568	0.00126596376036952\\
569	0.00122213771256506\\
570	0.00117800354620471\\
571	0.00113358417719133\\
572	0.00108890495861866\\
573	0.00104399377320935\\
574	0.000998881179137393\\
575	0.000953600553191194\\
576	0.000908188225232209\\
577	0.000862683598024717\\
578	0.000817129245000729\\
579	0.000771570976443156\\
580	0.000726057861961947\\
581	0.000680642193874616\\
582	0.000635379372066682\\
583	0.00059032768597172\\
584	0.00054554796342477\\
585	0.000501103049522705\\
586	0.00045705707227693\\
587	0.000413474448977723\\
588	0.000370418597003136\\
589	0.000327950360562706\\
590	0.000286126318718799\\
591	0.000244997582319804\\
592	0.000204610904245314\\
593	0.000165017211836693\\
594	0.000126301460681322\\
595	8.88161203105204e-05\\
596	5.34134895574396e-05\\
597	2.21100055488407e-05\\
598	0\\
599	0\\
600	0\\
};
\addplot [color=mycolor3,solid,forget plot]
  table[row sep=crcr]{%
1	0.00590333806059412\\
2	0.00590329678128906\\
3	0.00590325475384485\\
4	0.00590321196662225\\
5	0.00590316840786256\\
6	0.00590312406568831\\
7	0.00590307892810431\\
8	0.00590303298299859\\
9	0.00590298621814341\\
10	0.00590293862119661\\
11	0.00590289017970282\\
12	0.00590284088109476\\
13	0.00590279071269491\\
14	0.00590273966171686\\
15	0.00590268771526706\\
16	0.00590263486034665\\
17	0.00590258108385331\\
18	0.00590252637258319\\
19	0.00590247071323301\\
20	0.0059024140924023\\
21	0.00590235649659557\\
22	0.00590229791222486\\
23	0.00590223832561214\\
24	0.00590217772299197\\
25	0.00590211609051422\\
26	0.00590205341424684\\
27	0.00590198968017892\\
28	0.00590192487422354\\
29	0.00590185898222115\\
30	0.00590179198994261\\
31	0.00590172388309263\\
32	0.00590165464731318\\
33	0.00590158426818708\\
34	0.00590151273124164\\
35	0.00590144002195226\\
36	0.00590136612574644\\
37	0.0059012910280075\\
38	0.00590121471407871\\
39	0.00590113716926716\\
40	0.00590105837884802\\
41	0.00590097832806862\\
42	0.00590089700215268\\
43	0.00590081438630474\\
44	0.00590073046571417\\
45	0.00590064522555982\\
46	0.00590055865101416\\
47	0.0059004707272478\\
48	0.00590038143943372\\
49	0.00590029077275173\\
50	0.00590019871239278\\
51	0.00590010524356324\\
52	0.00590001035148916\\
53	0.00589991402142042\\
54	0.0058998162386348\\
55	0.00589971698844211\\
56	0.0058996162561879\\
57	0.0058995140272573\\
58	0.00589941028707861\\
59	0.00589930502112669\\
60	0.0058991982149262\\
61	0.00589908985405471\\
62	0.00589897992414526\\
63	0.00589886841088904\\
64	0.0058987553000376\\
65	0.00589864057740471\\
66	0.00589852422886803\\
67	0.00589840624037018\\
68	0.00589828659791972\\
69	0.00589816528759147\\
70	0.00589804229552654\\
71	0.00589791760793174\\
72	0.0058977912110786\\
73	0.00589766309130188\\
74	0.00589753323499729\\
75	0.00589740162861898\\
76	0.00589726825867608\\
77	0.00589713311172873\\
78	0.00589699617438341\\
79	0.00589685743328763\\
80	0.00589671687512365\\
81	0.00589657448660168\\
82	0.00589643025445222\\
83	0.00589628416541746\\
84	0.00589613620624197\\
85	0.00589598636366242\\
86	0.00589583462439656\\
87	0.00589568097513107\\
88	0.00589552540250864\\
89	0.00589536789311421\\
90	0.00589520843345995\\
91	0.00589504700996954\\
92	0.0058948836089615\\
93	0.00589471821663132\\
94	0.00589455081903294\\
95	0.00589438140205905\\
96	0.00589420995142053\\
97	0.00589403645262513\\
98	0.00589386089095484\\
99	0.00589368325144289\\
100	0.00589350351884961\\
101	0.00589332167763755\\
102	0.00589313771194584\\
103	0.0058929516055639\\
104	0.00589276334190443\\
105	0.00589257290397592\\
106	0.00589238027435446\\
107	0.00589218543515516\\
108	0.00589198836800336\\
109	0.00589178905400529\\
110	0.00589158747371858\\
111	0.00589138360712279\\
112	0.00589117743358958\\
113	0.0058909689318533\\
114	0.0058907580799815\\
115	0.00589054485534555\\
116	0.00589032923459194\\
117	0.00589011119361365\\
118	0.00588989070752221\\
119	0.00588966775062016\\
120	0.00588944229637443\\
121	0.00588921431739025\\
122	0.00588898378538588\\
123	0.00588875067116826\\
124	0.00588851494460939\\
125	0.00588827657462387\\
126	0.00588803552914733\\
127	0.00588779177511562\\
128	0.00588754527844537\\
129	0.00588729600401512\\
130	0.00588704391564782\\
131	0.00588678897609389\\
132	0.00588653114701554\\
133	0.0058862703889718\\
134	0.00588600666140452\\
135	0.00588573992262503\\
136	0.00588547012980185\\
137	0.00588519723894899\\
138	0.00588492120491518\\
139	0.0058846419813737\\
140	0.00588435952081299\\
141	0.00588407377452813\\
142	0.00588378469261278\\
143	0.00588349222395189\\
144	0.00588319631621476\\
145	0.00588289691584855\\
146	0.0058825939680713\\
147	0.00588228741686388\\
148	0.00588197720495874\\
149	0.00588166327382249\\
150	0.00588134556363449\\
151	0.0058810240132927\\
152	0.00588069856039914\\
153	0.00588036914123539\\
154	0.00588003569073719\\
155	0.00587969814246919\\
156	0.0058793564285995\\
157	0.00587901047987424\\
158	0.00587866022559217\\
159	0.00587830559357929\\
160	0.0058779465101634\\
161	0.00587758290014899\\
162	0.00587721468679199\\
163	0.00587684179177486\\
164	0.00587646413518159\\
165	0.00587608163547312\\
166	0.00587569420946294\\
167	0.00587530177229281\\
168	0.00587490423740891\\
169	0.00587450151653802\\
170	0.00587409351966432\\
171	0.00587368015500627\\
172	0.00587326132899405\\
173	0.00587283694624714\\
174	0.00587240690955239\\
175	0.00587197111984255\\
176	0.00587152947617501\\
177	0.00587108187571106\\
178	0.00587062821369553\\
179	0.00587016838343667\\
180	0.00586970227628657\\
181	0.00586922978162183\\
182	0.00586875078682447\\
183	0.00586826517726333\\
184	0.00586777283627547\\
185	0.0058672736451478\\
186	0.00586676748309892\\
187	0.00586625422726093\\
188	0.00586573375266123\\
189	0.00586520593220404\\
190	0.00586467063665185\\
191	0.00586412773460643\\
192	0.00586357709248929\\
193	0.00586301857452158\\
194	0.00586245204270305\\
195	0.00586187735679009\\
196	0.00586129437427256\\
197	0.00586070295034923\\
198	0.00586010293790159\\
199	0.0058594941874658\\
200	0.00585887654720249\\
201	0.00585824986286415\\
202	0.0058576139777599\\
203	0.0058569687327172\\
204	0.00585631396604036\\
205	0.00585564951346543\\
206	0.00585497520811081\\
207	0.00585429088042399\\
208	0.00585359635812338\\
209	0.00585289146613499\\
210	0.00585217602652403\\
211	0.00585144985842028\\
212	0.00585071277793774\\
213	0.00584996459808745\\
214	0.00584920512868351\\
215	0.00584843417624176\\
216	0.00584765154387117\\
217	0.0058468570311569\\
218	0.0058460504340353\\
219	0.00584523154466043\\
220	0.00584440015126168\\
221	0.00584355603799238\\
222	0.00584269898476903\\
223	0.00584182876710115\\
224	0.00584094515591158\\
225	0.00584004791734668\\
226	0.00583913681257685\\
227	0.00583821159758714\\
228	0.00583727202295753\\
229	0.00583631783363322\\
230	0.00583534876868337\\
231	0.00583436456104625\\
232	0.00583336493725985\\
233	0.00583234961719841\\
234	0.00583131831379131\\
235	0.00583027073272855\\
236	0.00582920657215461\\
237	0.00582812552235046\\
238	0.00582702726540339\\
239	0.00582591147486398\\
240	0.00582477781539051\\
241	0.00582362594237958\\
242	0.00582245550158399\\
243	0.00582126612871744\\
244	0.00582005744904719\\
245	0.00581882907697657\\
246	0.00581758061562036\\
247	0.00581631165637793\\
248	0.00581502177851218\\
249	0.00581371054874442\\
250	0.00581237752087128\\
251	0.00581102223540446\\
252	0.00580964421927844\\
253	0.00580824298564178\\
254	0.00580681803375392\\
255	0.00580536884900539\\
256	0.00580389490304745\\
257	0.00580239565391754\\
258	0.00580087054616907\\
259	0.00579931901198088\\
260	0.00579774047163374\\
261	0.00579613433334658\\
262	0.00579449999310545\\
263	0.00579283683447723\\
264	0.00579114422841492\\
265	0.00578942153313777\\
266	0.00578766809398388\\
267	0.00578588324324393\\
268	0.00578406629999092\\
269	0.00578221656990779\\
270	0.00578033334511233\\
271	0.00577841590397929\\
272	0.00577646351095954\\
273	0.00577447541639589\\
274	0.00577245085633587\\
275	0.00577038905234164\\
276	0.0057682892112973\\
277	0.00576615052521205\\
278	0.00576397217100841\\
279	0.00576175331026504\\
280	0.00575949308900137\\
281	0.00575719063752106\\
282	0.0057548450702292\\
283	0.0057524554854554\\
284	0.00575002096526562\\
285	0.00574754057514018\\
286	0.00574501336366976\\
287	0.0057424383622854\\
288	0.00573981458499559\\
289	0.00573714102816268\\
290	0.00573441667030899\\
291	0.00573164047160095\\
292	0.00572881137339478\\
293	0.00572592829774257\\
294	0.0057229901467965\\
295	0.00571999580207755\\
296	0.00571694412356491\\
297	0.0057138339485516\\
298	0.00571066409019515\\
299	0.00570743333566944\\
300	0.00570414044377602\\
301	0.00570078414177516\\
302	0.00569736312114746\\
303	0.00569387603342288\\
304	0.00569032148463817\\
305	0.0056866980282371\\
306	0.00568300415638837\\
307	0.00567923828954731\\
308	0.0056753987641211\\
309	0.00567148381823785\\
310	0.00566749157589968\\
311	0.00566342003013474\\
312	0.00565926702490346\\
313	0.00565503023655459\\
314	0.0056507071588506\\
315	0.00564629508481615\\
316	0.00564179112620737\\
317	0.00563719227535485\\
318	0.00563249541323885\\
319	0.0056276973060651\\
320	0.00562279460355457\\
321	0.00561778383818147\\
322	0.00561266142449403\\
323	0.00560742365907418\\
324	0.00560206672146222\\
325	0.00559658667665739\\
326	0.00559097948114829\\
327	0.00558524098817996\\
328	0.00557936695390648\\
329	0.00557335304724402\\
330	0.00556719486278295\\
331	0.00556088793634252\\
332	0.00555442776079585\\
333	0.0055478098039952\\
334	0.00554102954609717\\
335	0.00553408251539891\\
336	0.00552696432800334\\
337	0.00551967073358674\\
338	0.00551219766920138\\
339	0.00550454132719009\\
340	0.00549669821790575\\
341	0.00548866520831701\\
342	0.00548043956988406\\
343	0.00547201901752715\\
344	0.00546340172581629\\
345	0.00545458630710211\\
346	0.00544557172993664\\
347	0.00543635714778947\\
348	0.00542694159604296\\
349	0.00541732346631383\\
350	0.00540749972686067\\
351	0.00539746421660027\\
352	0.00538720469915132\\
353	0.0053767079213199\\
354	0.00536595958104545\\
355	0.00535494423409109\\
356	0.00534364519367905\\
357	0.00533204433172414\\
358	0.00532012201216823\\
359	0.00530785709942542\\
360	0.00529522692650257\\
361	0.0052822072972082\\
362	0.00526877268711841\\
363	0.00525489651118935\\
364	0.0052405510446857\\
365	0.00522570766390407\\
366	0.00521033727861089\\
367	0.00519441096669814\\
368	0.00517790088003339\\
369	0.00516078151427541\\
370	0.00514303147787896\\
371	0.00512463595368377\\
372	0.00510558951866255\\
373	0.0050859006581032\\
374	0.00506559748035046\\
375	0.00504473529900209\\
376	0.00502344156360004\\
377	0.00500218022872259\\
378	0.0049809760460786\\
379	0.00495985578832219\\
380	0.00493884842349805\\
381	0.00491798468283812\\
382	0.00489730159774379\\
383	0.004876841736052\\
384	0.0048566516451108\\
385	0.0048367819424855\\
386	0.00481728729511061\\
387	0.00479822623413121\\
388	0.0047796607313178\\
389	0.00476165542284863\\
390	0.00474427648300055\\
391	0.00472758988880617\\
392	0.00471165883337791\\
393	0.00469654004973934\\
394	0.00468227857998984\\
395	0.00466890043389443\\
396	0.00465640265863999\\
397	0.00464473984308537\\
398	0.00463338476898287\\
399	0.00462226273154278\\
400	0.00461138633557096\\
401	0.00460076702303838\\
402	0.00459041470085884\\
403	0.00458033731668045\\
404	0.00457054038341459\\
405	0.00456102645771964\\
406	0.00455179457840555\\
407	0.00454283968002161\\
408	0.00453415201224889\\
409	0.00452571660673242\\
410	0.00451751287505113\\
411	0.0045095144341123\\
412	0.00450168930621192\\
413	0.00449400071203626\\
414	0.00448640874644793\\
415	0.00447887334794408\\
416	0.00447137650171153\\
417	0.0044639138489833\\
418	0.00445648019262798\\
419	0.00444906949342142\\
420	0.00444167488844161\\
421	0.00443428873727616\\
422	0.00442690270202452\\
423	0.00441950786725293\\
424	0.0044120949047246\\
425	0.00440465428644279\\
426	0.00439717654633382\\
427	0.00438965258573034\\
428	0.00438207400929819\\
429	0.0043744334648178\\
430	0.00436672494026864\\
431	0.00435894394214415\\
432	0.00435108693653489\\
433	0.00434315033914831\\
434	0.00433513054650948\\
435	0.00432702396978867\\
436	0.00431882707029175\\
437	0.00431053639527444\\
438	0.00430214861230272\\
439	0.00429366053991516\\
440	0.00428506917189402\\
441	0.00427637169210207\\
442	0.00426756547672204\\
443	0.00425864808105266\\
444	0.00424961720908999\\
445	0.00424047066643495\\
446	0.00423120630132512\\
447	0.00422182197827159\\
448	0.00421231558137371\\
449	0.00420268501660574\\
450	0.00419292821288889\\
451	0.00418304312179867\\
452	0.00417302771581684\\
453	0.00416287998512681\\
454	0.00415259793307238\\
455	0.00414217957055166\\
456	0.00413162290979472\\
457	0.00412092595815515\\
458	0.00411008671269693\\
459	0.00409910315640782\\
460	0.00408797325670467\\
461	0.0040766949644316\\
462	0.0040652662126725\\
463	0.00405368491540735\\
464	0.00404194896605094\\
465	0.00403005623592155\\
466	0.00401800457269473\\
467	0.00400579179889832\\
468	0.00399341571050184\\
469	0.00398087407564079\\
470	0.00396816463349411\\
471	0.00395528509330173\\
472	0.00394223313347\\
473	0.00392900640067599\\
474	0.00391560250894252\\
475	0.00390201903868778\\
476	0.00388825353575286\\
477	0.00387430351040873\\
478	0.00386016643634252\\
479	0.00384583974961992\\
480	0.00383132084761834\\
481	0.00381660708792198\\
482	0.00380169578716801\\
483	0.00378658421983103\\
484	0.00377126961693291\\
485	0.00375574916466674\\
486	0.00374002000292527\\
487	0.00372407922372343\\
488	0.0037079238695024\\
489	0.00369155093130099\\
490	0.00367495734677861\\
491	0.00365813999807156\\
492	0.00364109570946233\\
493	0.00362382124483922\\
494	0.00360631330492075\\
495	0.0035885685242167\\
496	0.00357058346769363\\
497	0.00355235462710943\\
498	0.00353387841697602\\
499	0.00351515117010466\\
500	0.00349616913268174\\
501	0.00347692845881597\\
502	0.00345742520449019\\
503	0.00343765532084158\\
504	0.00341761464668331\\
505	0.00339729890016923\\
506	0.00337670366948799\\
507	0.00335582440245784\\
508	0.00333465639487335\\
509	0.00331319477743365\\
510	0.00329143450105606\\
511	0.00326937032034516\\
512	0.00324699677495018\\
513	0.00322430816850468\\
514	0.00320129854485634\\
515	0.00317796166130206\\
516	0.00315429095836913\\
517	0.00313027952542767\\
518	0.0031059200590184\\
519	0.00308120481164158\\
520	0.00305612553426673\\
521	0.00303067341577268\\
522	0.00300483906420901\\
523	0.00297861256684646\\
524	0.00295198341446589\\
525	0.00292494538011159\\
526	0.00289748602609065\\
527	0.00286958592024448\\
528	0.00284122506151952\\
529	0.00281237967021063\\
530	0.00278302467603092\\
531	0.00275314503776596\\
532	0.00272272663175835\\
533	0.00269175613661856\\
534	0.00266022413796544\\
535	0.00262812442108877\\
536	0.00259545648095494\\
537	0.00256225561626944\\
538	0.0025285702644702\\
539	0.00249433480316095\\
540	0.00245945308195854\\
541	0.00242374914526642\\
542	0.00238717222854202\\
543	0.00234966450884324\\
544	0.00231117872502218\\
545	0.00227166332173057\\
546	0.00223105961431705\\
547	0.00218930071425861\\
548	0.00214630848721728\\
549	0.00210199113162199\\
550	0.00205619994680202\\
551	0.00201158132110446\\
552	0.0019691256828508\\
553	0.00192715271475019\\
554	0.00188480931366883\\
555	0.0018419273240984\\
556	0.00179850167657468\\
557	0.00175457044864958\\
558	0.00171018229188805\\
559	0.0016654007279087\\
560	0.00162030670009296\\
561	0.00157499906801893\\
562	0.00153015667117478\\
563	0.00148647803572806\\
564	0.00144270425365799\\
565	0.00139873860085343\\
566	0.00135460699404303\\
567	0.00131034339611248\\
568	0.00126603637621005\\
569	0.00122213828276059\\
570	0.00117800358348419\\
571	0.00113358418438274\\
572	0.00108890496075918\\
573	0.00104399377408744\\
574	0.000998881179574082\\
575	0.000953600553427397\\
576	0.000908188225361615\\
577	0.000862683598096681\\
578	0.000817129245039902\\
579	0.000771570976463344\\
580	0.00072605786197129\\
581	0.000680642193878206\\
582	0.000635379372067672\\
583	0.000590327685971876\\
584	0.000545547963424773\\
585	0.000501103049522706\\
586	0.000457057072276932\\
587	0.00041347444897773\\
588	0.000370418597003138\\
589	0.000327950360562707\\
590	0.0002861263187188\\
591	0.000244997582319805\\
592	0.000204610904245315\\
593	0.000165017211836693\\
594	0.000126301460681323\\
595	8.88161203105201e-05\\
596	5.34134895574397e-05\\
597	2.21100055488406e-05\\
598	0\\
599	0\\
600	0\\
};
\addplot [color=mycolor4,solid,forget plot]
  table[row sep=crcr]{%
1	0.00590379526347467\\
2	0.00590376498787609\\
3	0.00590373421525883\\
4	0.00590370293902819\\
5	0.00590367115254342\\
6	0.00590363884911854\\
7	0.00590360602202297\\
8	0.00590357266448247\\
9	0.00590353876967977\\
10	0.00590350433075548\\
11	0.00590346934080889\\
12	0.00590343379289877\\
13	0.00590339768004422\\
14	0.00590336099522564\\
15	0.00590332373138558\\
16	0.00590328588142959\\
17	0.00590324743822709\\
18	0.0059032083946124\\
19	0.00590316874338557\\
20	0.00590312847731336\\
21	0.00590308758913008\\
22	0.00590304607153859\\
23	0.00590300391721115\\
24	0.00590296111879034\\
25	0.00590291766889001\\
26	0.00590287356009605\\
27	0.00590282878496737\\
28	0.00590278333603661\\
29	0.00590273720581107\\
30	0.00590269038677339\\
31	0.00590264287138242\\
32	0.00590259465207385\\
33	0.00590254572126088\\
34	0.00590249607133481\\
35	0.00590244569466576\\
36	0.00590239458360299\\
37	0.00590234273047549\\
38	0.00590229012759227\\
39	0.00590223676724273\\
40	0.00590218264169687\\
41	0.00590212774320538\\
42	0.00590207206399979\\
43	0.00590201559629226\\
44	0.00590195833227557\\
45	0.00590190026412281\\
46	0.00590184138398693\\
47	0.00590178168400025\\
48	0.00590172115627383\\
49	0.00590165979289661\\
50	0.00590159758593454\\
51	0.00590153452742941\\
52	0.00590147060939759\\
53	0.00590140582382857\\
54	0.0059013401626834\\
55	0.00590127361789277\\
56	0.00590120618135503\\
57	0.00590113784493395\\
58	0.00590106860045635\\
59	0.00590099843970934\\
60	0.00590092735443741\\
61	0.0059008553363394\\
62	0.00590078237706507\\
63	0.00590070846821142\\
64	0.00590063360131881\\
65	0.00590055776786686\\
66	0.00590048095926989\\
67	0.00590040316687229\\
68	0.00590032438194349\\
69	0.00590024459567266\\
70	0.00590016379916304\\
71	0.00590008198342608\\
72	0.0058999991393753\\
73	0.00589991525781958\\
74	0.00589983032945659\\
75	0.00589974434486537\\
76	0.00589965729449901\\
77	0.00589956916867689\\
78	0.0058994799575765\\
79	0.00589938965122506\\
80	0.00589929823949085\\
81	0.00589920571207409\\
82	0.00589911205849771\\
83	0.00589901726809768\\
84	0.00589892133001309\\
85	0.00589882423317598\\
86	0.00589872596630087\\
87	0.00589862651787407\\
88	0.00589852587614271\\
89	0.00589842402910346\\
90	0.0058983209644912\\
91	0.00589821666976742\\
92	0.00589811113210831\\
93	0.00589800433839292\\
94	0.00589789627519096\\
95	0.00589778692875054\\
96	0.00589767628498594\\
97	0.005897564329465\\
98	0.00589745104739679\\
99	0.00589733642361903\\
100	0.00589722044258555\\
101	0.00589710308835377\\
102	0.00589698434457225\\
103	0.00589686419446828\\
104	0.00589674262083548\\
105	0.00589661960602164\\
106	0.00589649513191666\\
107	0.0058963691799406\\
108	0.00589624173103197\\
109	0.00589611276563616\\
110	0.0058959822636941\\
111	0.00589585020463123\\
112	0.00589571656734662\\
113	0.0058955813302024\\
114	0.00589544447101334\\
115	0.00589530596703697\\
116	0.00589516579496371\\
117	0.0058950239309073\\
118	0.00589488035039563\\
119	0.00589473502836173\\
120	0.00589458793913506\\
121	0.00589443905643291\\
122	0.00589428835335214\\
123	0.00589413580236104\\
124	0.00589398137529138\\
125	0.00589382504333054\\
126	0.00589366677701375\\
127	0.00589350654621647\\
128	0.00589334432014663\\
129	0.00589318006733699\\
130	0.00589301375563747\\
131	0.00589284535220734\\
132	0.00589267482350716\\
133	0.00589250213529082\\
134	0.00589232725259709\\
135	0.00589215013974123\\
136	0.00589197076030609\\
137	0.0058917890771331\\
138	0.00589160505231279\\
139	0.00589141864717509\\
140	0.00589122982227927\\
141	0.00589103853740335\\
142	0.00589084475153345\\
143	0.00589064842285231\\
144	0.0058904495087279\\
145	0.00589024796570103\\
146	0.00589004374947293\\
147	0.00588983681489199\\
148	0.00588962711593984\\
149	0.00588941460571741\\
150	0.00588919923643322\\
151	0.0058889809593899\\
152	0.00588875972496993\\
153	0.00588853548262112\\
154	0.00588830818084222\\
155	0.00588807776716816\\
156	0.00588784418815529\\
157	0.00588760738936659\\
158	0.00588736731535644\\
159	0.00588712390965559\\
160	0.00588687711475577\\
161	0.00588662687209412\\
162	0.00588637312203766\\
163	0.00588611580386718\\
164	0.00588585485576133\\
165	0.00588559021478023\\
166	0.00588532181684884\\
167	0.00588504959674015\\
168	0.00588477348805809\\
169	0.00588449342321997\\
170	0.00588420933343877\\
171	0.00588392114870485\\
172	0.00588362879776743\\
173	0.00588333220811543\\
174	0.0058830313059581\\
175	0.00588272601620475\\
176	0.00588241626244428\\
177	0.00588210196692384\\
178	0.00588178305052682\\
179	0.00588145943275045\\
180	0.00588113103168215\\
181	0.00588079776397541\\
182	0.00588045954482466\\
183	0.00588011628793912\\
184	0.0058797679055157\\
185	0.00587941430821089\\
186	0.00587905540511135\\
187	0.00587869110370336\\
188	0.00587832130984095\\
189	0.00587794592771265\\
190	0.0058775648598068\\
191	0.00587717800687543\\
192	0.00587678526789627\\
193	0.00587638654003338\\
194	0.00587598171859583\\
195	0.00587557069699474\\
196	0.00587515336669815\\
197	0.00587472961718414\\
198	0.00587429933589181\\
199	0.00587386240817019\\
200	0.00587341871722503\\
201	0.00587296814406326\\
202	0.00587251056743539\\
203	0.00587204586377532\\
204	0.0058715739071381\\
205	0.00587109456913512\\
206	0.00587060771886728\\
207	0.00587011322285538\\
208	0.00586961094496848\\
209	0.00586910074634974\\
210	0.00586858248534013\\
211	0.00586805601739993\\
212	0.00586752119502793\\
213	0.00586697786767869\\
214	0.00586642588167781\\
215	0.00586586508013543\\
216	0.00586529530285779\\
217	0.0058647163862575\\
218	0.00586412816326234\\
219	0.00586353046322277\\
220	0.00586292311181854\\
221	0.00586230593096449\\
222	0.00586167873871565\\
223	0.00586104134917222\\
224	0.0058603935723842\\
225	0.00585973521425634\\
226	0.0058590660764534\\
227	0.00585838595630602\\
228	0.00585769464671747\\
229	0.00585699193607131\\
230	0.00585627760814004\\
231	0.0058555514419952\\
232	0.00585481321192058\\
233	0.00585406268732634\\
234	0.00585329963266529\\
235	0.00585252380735102\\
236	0.00585173496567821\\
237	0.00585093285674526\\
238	0.00585011722437906\\
239	0.00584928780706206\\
240	0.00584844433786135\\
241	0.00584758654436012\\
242	0.00584671414859089\\
243	0.00584582686697076\\
244	0.00584492441023848\\
245	0.00584400648339321\\
246	0.00584307278563476\\
247	0.00584212301030537\\
248	0.00584115684483277\\
249	0.00584017397067318\\
250	0.0058391740632536\\
251	0.00583815679191384\\
252	0.00583712181984754\\
253	0.00583606880403983\\
254	0.00583499739519842\\
255	0.00583390723767195\\
256	0.00583279796934581\\
257	0.00583166922152828\\
258	0.00583052061889368\\
259	0.00582935177936182\\
260	0.00582816231394888\\
261	0.00582695182660875\\
262	0.00582571991406411\\
263	0.00582446616562789\\
264	0.00582319016302086\\
265	0.00582189148017541\\
266	0.00582056968302588\\
267	0.00581922432928647\\
268	0.00581785496821623\\
269	0.00581646114037047\\
270	0.0058150423773379\\
271	0.00581359820146302\\
272	0.0058121281255533\\
273	0.00581063165257026\\
274	0.00580910827530425\\
275	0.00580755747603235\\
276	0.00580597872615834\\
277	0.00580437148583363\\
278	0.00580273520355715\\
279	0.00580106931576133\\
280	0.00579937324638386\\
281	0.00579764640641807\\
282	0.00579588819344376\\
283	0.00579409799113489\\
284	0.00579227516873575\\
285	0.00579041908051904\\
286	0.00578852906522668\\
287	0.00578660444549092\\
288	0.00578464452723698\\
289	0.00578264859906228\\
290	0.0057806159315693\\
291	0.00577854577668241\\
292	0.00577643736693986\\
293	0.00577428991475673\\
294	0.00577210261165694\\
295	0.00576987462747252\\
296	0.0057676051095085\\
297	0.00576529318167107\\
298	0.0057629379435557\\
299	0.00576053846948955\\
300	0.00575809380752012\\
301	0.0057556029783565\\
302	0.0057530649743625\\
303	0.00575047875849766\\
304	0.00574784326322206\\
305	0.00574515738939738\\
306	0.00574242000520803\\
307	0.00573962994513303\\
308	0.00573678600901041\\
309	0.00573388696124936\\
310	0.00573093153023273\\
311	0.00572791840784602\\
312	0.00572484624910994\\
313	0.00572171367197512\\
314	0.00571851925676098\\
315	0.00571526154757381\\
316	0.00571193905303247\\
317	0.00570855024551227\\
318	0.00570509356005693\\
319	0.0057015673933592\\
320	0.00569797010272115\\
321	0.00569430000491495\\
322	0.00569055537497388\\
323	0.00568673444492674\\
324	0.00568283540250686\\
325	0.00567885638989284\\
326	0.00567479550211841\\
327	0.00567065078526745\\
328	0.00566642023460038\\
329	0.00566210179246317\\
330	0.00565769334581952\\
331	0.00565319272315991\\
332	0.00564859769103376\\
333	0.00564390595123254\\
334	0.00563911513671801\\
335	0.00563422280663334\\
336	0.0056292264404622\\
337	0.00562412343135971\\
338	0.00561891107868596\\
339	0.00561358657779683\\
340	0.00560814700602255\\
341	0.00560258930768038\\
342	0.00559691027684483\\
343	0.00559110653741519\\
344	0.00558517452050841\\
345	0.00557911043953141\\
346	0.00557291026378581\\
347	0.00556656969196654\\
348	0.00556008412633157\\
349	0.00555344865423694\\
350	0.00554665802886981\\
351	0.00553970671752919\\
352	0.00553258900139072\\
353	0.00552529898234968\\
354	0.00551783058723264\\
355	0.00551017757296908\\
356	0.00550233352881645\\
357	0.00549429189436004\\
358	0.00548604598673111\\
359	0.00547758902805796\\
360	0.00546891418086784\\
361	0.0054600146012915\\
362	0.00545088349244723\\
363	0.00544151413085002\\
364	0.00543189991725684\\
365	0.00542203443641144\\
366	0.0054119115189281\\
367	0.00540152530146267\\
368	0.00539087027925888\\
369	0.00537994134238283\\
370	0.00536873377960162\\
371	0.00535724320129957\\
372	0.00534546542353583\\
373	0.00533339621655065\\
374	0.00532103087005047\\
375	0.00530836350864288\\
376	0.00529538607474241\\
377	0.00528208573941149\\
378	0.00526844138642053\\
379	0.00525443023495017\\
380	0.00524002780960136\\
381	0.00522520825949498\\
382	0.00520994425757827\\
383	0.0051942068560803\\
384	0.00517796557513657\\
385	0.00516118861285438\\
386	0.00514384318728325\\
387	0.00512589605889187\\
388	0.00510731430512192\\
389	0.00508806649590293\\
390	0.00506812393535674\\
391	0.00504746272422302\\
392	0.00502606664883736\\
393	0.00500393098676999\\
394	0.00498106767928082\\
395	0.00495751007937781\\
396	0.00493332304891941\\
397	0.00490861433986502\\
398	0.0048839535690663\\
399	0.00485946909780619\\
400	0.00483520667819876\\
401	0.00481121659496473\\
402	0.00478755387758255\\
403	0.00476427843156136\\
404	0.00474145504198391\\
405	0.00471915317202083\\
406	0.0046974465831148\\
407	0.0046764126419548\\
408	0.00465613109236378\\
409	0.00463668225776951\\
410	0.00461814400985736\\
411	0.00460058760661798\\
412	0.00458407183862374\\
413	0.00456863464115532\\
414	0.00455428182232229\\
415	0.00454097170777687\\
416	0.00452823578507634\\
417	0.0045157922743315\\
418	0.00450365324477773\\
419	0.00449182858417429\\
420	0.00448032544899743\\
421	0.00446914764735966\\
422	0.00445829496546078\\
423	0.00444776245068988\\
424	0.0044375397014027\\
425	0.00442761020495439\\
426	0.00441795080391462\\
427	0.00440853139863911\\
428	0.00439931504829813\\
429	0.00439025870190537\\
430	0.00438131488528355\\
431	0.00437243479842817\\
432	0.00436358607575379\\
433	0.00435476249912737\\
434	0.00434595686158091\\
435	0.00433716099199814\\
436	0.00432836581422454\\
437	0.00431956144712783\\
438	0.00431073735238494\\
439	0.00430188253598267\\
440	0.00429298580733041\\
441	0.00428403609592639\\
442	0.00427502281877005\\
443	0.00426593628086591\\
444	0.00425676807438704\\
445	0.00424751141678241\\
446	0.00423816133103309\\
447	0.00422871365154483\\
448	0.00421916419305247\\
449	0.00420950879043587\\
450	0.00419974334040273\\
451	0.00418986384340222\\
452	0.00417986644360157\\
453	0.00416974746419458\\
454	0.00415950343478122\\
455	0.00414913110715767\\
456	0.00413862745575603\\
457	0.00412798965942506\\
458	0.00411721506266893\\
459	0.0041063011174757\\
460	0.00409524531237804\\
461	0.00408404515595559\\
462	0.00407269817964037\\
463	0.00406120193911939\\
464	0.00404955401414548\\
465	0.00403775200663431\\
466	0.0040257935370263\\
467	0.00401367623903292\\
468	0.00400139775306435\\
469	0.00398895571884476\\
470	0.00397634776793954\\
471	0.00396357151710422\\
472	0.00395062456343933\\
473	0.00393750448216406\\
474	0.00392420882510719\\
475	0.003910735118964\\
476	0.00389708086335479\\
477	0.00388324352873227\\
478	0.00386922055419525\\
479	0.00385500934527123\\
480	0.0038406072717301\\
481	0.00382601166548202\\
482	0.00381121981859191\\
483	0.00379622898141109\\
484	0.00378103636078207\\
485	0.00376563911822554\\
486	0.00375003436800754\\
487	0.00373421917508312\\
488	0.00371819055291034\\
489	0.00370194546112614\\
490	0.00368548080307145\\
491	0.00366879342314951\\
492	0.00365188010399561\\
493	0.00363473756343218\\
494	0.00361736245117754\\
495	0.00359975134527311\\
496	0.00358190074819062\\
497	0.00356380708258075\\
498	0.0035454666866246\\
499	0.00352687580894453\\
500	0.00350803060302674\\
501	0.0034889271211014\\
502	0.00346956130742128\\
503	0.00344992899087267\\
504	0.0034300258768456\\
505	0.00340984753828324\\
506	0.00338938940582154\\
507	0.00336864675692231\\
508	0.0033476147038928\\
509	0.00332628818067455\\
510	0.0033046619282728\\
511	0.00328273047868595\\
512	0.00326048813718142\\
513	0.00323792896275126\\
514	0.00321504674656414\\
515	0.00319183498821309\\
516	0.00316828686954246\\
517	0.00314439522582985\\
518	0.0031201525141719\\
519	0.00309555077894304\\
520	0.00307058161405028\\
521	0.00304523612154709\\
522	0.00301950486370087\\
523	0.0029933778069748\\
524	0.00296684426210323\\
525	0.00293989282401477\\
526	0.00291251135176726\\
527	0.0028846870297126\\
528	0.00285640633169834\\
529	0.00282765792198794\\
530	0.0027984296333008\\
531	0.00276869876328051\\
532	0.00273844180218386\\
533	0.00270763530767341\\
534	0.00267624608035157\\
535	0.00264425550756493\\
536	0.00261164692964258\\
537	0.00257840551553682\\
538	0.00254451994288678\\
539	0.00250998594321005\\
540	0.00247482286706217\\
541	0.00243910249701945\\
542	0.00240276234870822\\
543	0.0023657389254302\\
544	0.00232785832055229\\
545	0.00228905529707634\\
546	0.00224927220259275\\
547	0.00220844791110207\\
548	0.00216652471415998\\
549	0.00212343499341002\\
550	0.00207910199776249\\
551	0.00203343572283267\\
552	0.00198629221434771\\
553	0.00193978333544741\\
554	0.0018951895419994\\
555	0.00185179939928115\\
556	0.00180818798665882\\
557	0.00176406420818124\\
558	0.00171944480456451\\
559	0.00167435555765107\\
560	0.00162885181046918\\
561	0.00158300584892994\\
562	0.00153690735164017\\
563	0.00149066339943845\\
564	0.00144551975349939\\
565	0.00140111675669786\\
566	0.0013566170163797\\
567	0.00131197910736022\\
568	0.00126723396538844\\
569	0.00122241508511204\\
570	0.00117800790590505\\
571	0.00113358445788582\\
572	0.00108890501157333\\
573	0.00104399378859552\\
574	0.000998881185332254\\
575	0.000953600556240819\\
576	0.000908188226875188\\
577	0.000862683598921914\\
578	0.00081712924549955\\
579	0.000771570976715098\\
580	0.000726057862102571\\
581	0.000680642193939919\\
582	0.0006353793720919\\
583	0.000590327685978706\\
584	0.000545547963425862\\
585	0.000501103049522703\\
586	0.00045705707227693\\
587	0.000413474448977723\\
588	0.000370418597003137\\
589	0.000327950360562706\\
590	0.000286126318718802\\
591	0.000244997582319806\\
592	0.000204610904245315\\
593	0.000165017211836693\\
594	0.000126301460681323\\
595	8.88161203105208e-05\\
596	5.34134895574398e-05\\
597	2.21100055488407e-05\\
598	0\\
599	0\\
600	0\\
};
\addplot [color=mycolor5,solid,forget plot]
  table[row sep=crcr]{%
1	0.00590422946128678\\
2	0.00590420816105343\\
3	0.00590418654129558\\
4	0.00590416459823672\\
5	0.00590414232807503\\
6	0.00590411972698365\\
7	0.00590409679111047\\
8	0.00590407351657813\\
9	0.00590404989948405\\
10	0.00590402593590015\\
11	0.00590400162187299\\
12	0.00590397695342349\\
13	0.00590395192654684\\
14	0.00590392653721233\\
15	0.00590390078136317\\
16	0.00590387465491617\\
17	0.00590384815376174\\
18	0.00590382127376325\\
19	0.00590379401075699\\
20	0.00590376636055167\\
21	0.00590373831892813\\
22	0.00590370988163874\\
23	0.00590368104440714\\
24	0.00590365180292757\\
25	0.00590362215286433\\
26	0.0059035920898512\\
27	0.00590356160949076\\
28	0.00590353070735368\\
29	0.00590349937897794\\
30	0.00590346761986798\\
31	0.00590343542549384\\
32	0.00590340279129017\\
33	0.00590336971265528\\
34	0.00590333618494998\\
35	0.00590330220349647\\
36	0.00590326776357712\\
37	0.00590323286043313\\
38	0.00590319748926315\\
39	0.00590316164522192\\
40	0.00590312532341855\\
41	0.00590308851891512\\
42	0.00590305122672476\\
43	0.00590301344180997\\
44	0.00590297515908073\\
45	0.00590293637339243\\
46	0.00590289707954391\\
47	0.00590285727227515\\
48	0.00590281694626508\\
49	0.00590277609612918\\
50	0.00590273471641702\\
51	0.00590269280160959\\
52	0.00590265034611664\\
53	0.00590260734427394\\
54	0.00590256379034027\\
55	0.00590251967849441\\
56	0.00590247500283201\\
57	0.0059024297573624\\
58	0.00590238393600504\\
59	0.00590233753258622\\
60	0.00590229054083537\\
61	0.00590224295438132\\
62	0.00590219476674854\\
63	0.00590214597135311\\
64	0.00590209656149873\\
65	0.00590204653037243\\
66	0.00590199587104044\\
67	0.00590194457644361\\
68	0.00590189263939302\\
69	0.00590184005256525\\
70	0.00590178680849777\\
71	0.005901732899584\\
72	0.00590167831806837\\
73	0.00590162305604139\\
74	0.00590156710543432\\
75	0.00590151045801415\\
76	0.00590145310537817\\
77	0.00590139503894853\\
78	0.0059013362499669\\
79	0.00590127672948873\\
80	0.00590121646837771\\
81	0.00590115545730012\\
82	0.00590109368671895\\
83	0.00590103114688817\\
84	0.00590096782784679\\
85	0.00590090371941303\\
86	0.00590083881117837\\
87	0.00590077309250141\\
88	0.005900706552502\\
89	0.0059006391800552\\
90	0.00590057096378514\\
91	0.00590050189205903\\
92	0.00590043195298099\\
93	0.00590036113438613\\
94	0.00590028942383433\\
95	0.00590021680860435\\
96	0.00590014327568756\\
97	0.00590006881178215\\
98	0.00589999340328701\\
99	0.00589991703629577\\
100	0.00589983969659083\\
101	0.00589976136963753\\
102	0.00589968204057814\\
103	0.00589960169422615\\
104	0.00589952031506033\\
105	0.005899437887219\\
106	0.00589935439449429\\
107	0.00589926982032635\\
108	0.00589918414779768\\
109	0.00589909735962746\\
110	0.00589900943816588\\
111	0.0058989203653884\\
112	0.00589883012289022\\
113	0.00589873869188047\\
114	0.00589864605317668\\
115	0.00589855218719898\\
116	0.00589845707396431\\
117	0.00589836069308083\\
118	0.0058982630237419\\
119	0.00589816404472036\\
120	0.00589806373436234\\
121	0.00589796207058135\\
122	0.00589785903085215\\
123	0.00589775459220429\\
124	0.00589764873121586\\
125	0.00589754142400693\\
126	0.00589743264623284\\
127	0.00589732237307732\\
128	0.00589721057924555\\
129	0.00589709723895697\\
130	0.00589698232593792\\
131	0.00589686581341395\\
132	0.00589674767410229\\
133	0.00589662788020372\\
134	0.00589650640339444\\
135	0.00589638321481786\\
136	0.00589625828507575\\
137	0.00589613158421963\\
138	0.00589600308174176\\
139	0.00589587274656581\\
140	0.00589574054703761\\
141	0.00589560645091531\\
142	0.00589547042535966\\
143	0.00589533243692397\\
144	0.00589519245154357\\
145	0.00589505043452572\\
146	0.00589490635053855\\
147	0.00589476016360021\\
148	0.00589461183706771\\
149	0.00589446133362589\\
150	0.00589430861527592\\
151	0.00589415364332347\\
152	0.00589399637836669\\
153	0.00589383678028406\\
154	0.00589367480822181\\
155	0.00589351042058109\\
156	0.00589334357500514\\
157	0.00589317422836573\\
158	0.00589300233674966\\
159	0.00589282785544471\\
160	0.0058926507389255\\
161	0.00589247094083871\\
162	0.00589228841398823\\
163	0.00589210311031981\\
164	0.00589191498090522\\
165	0.00589172397592632\\
166	0.0058915300446583\\
167	0.00589133313545283\\
168	0.00589113319572062\\
169	0.00589093017191344\\
170	0.00589072400950573\\
171	0.00589051465297586\\
172	0.00589030204578648\\
173	0.00589008613036476\\
174	0.0058898668480817\\
175	0.00588964413923115\\
176	0.00588941794300805\\
177	0.00588918819748602\\
178	0.0058889548395947\\
179	0.00588871780509582\\
180	0.00588847702855905\\
181	0.0058882324433371\\
182	0.00588798398153998\\
183	0.00588773157400866\\
184	0.0058874751502879\\
185	0.0058872146385985\\
186	0.00588694996580856\\
187	0.00588668105740421\\
188	0.00588640783745937\\
189	0.00588613022860486\\
190	0.00588584815199671\\
191	0.00588556152728348\\
192	0.00588527027257313\\
193	0.00588497430439884\\
194	0.00588467353768407\\
195	0.00588436788570691\\
196	0.00588405726006361\\
197	0.00588374157063139\\
198	0.00588342072553046\\
199	0.00588309463108524\\
200	0.00588276319178499\\
201	0.00588242631024372\\
202	0.00588208388715927\\
203	0.00588173582127211\\
204	0.00588138200932304\\
205	0.0058810223460108\\
206	0.00588065672394878\\
207	0.0058802850336214\\
208	0.00587990716334001\\
209	0.00587952299919832\\
210	0.00587913242502751\\
211	0.00587873532235092\\
212	0.00587833157033847\\
213	0.00587792104576093\\
214	0.00587750362294375\\
215	0.00587707917372098\\
216	0.00587664756738887\\
217	0.00587620867065931\\
218	0.00587576234761343\\
219	0.00587530845965489\\
220	0.00587484686546326\\
221	0.00587437742094727\\
222	0.0058738999791982\\
223	0.00587341439044296\\
224	0.00587292050199739\\
225	0.00587241815821919\\
226	0.00587190720046095\\
227	0.00587138746702281\\
228	0.00587085879310497\\
229	0.00587032101075973\\
230	0.00586977394884321\\
231	0.00586921743296657\\
232	0.00586865128544643\\
233	0.00586807532525476\\
234	0.00586748936796734\\
235	0.00586689322571157\\
236	0.00586628670711273\\
237	0.00586566961723889\\
238	0.00586504175754402\\
239	0.0058644029258093\\
240	0.00586375291608242\\
241	0.00586309151861443\\
242	0.00586241851979424\\
243	0.00586173370208021\\
244	0.00586103684392907\\
245	0.00586032771972136\\
246	0.00585960609968369\\
247	0.00585887174980757\\
248	0.00585812443176398\\
249	0.00585736390281433\\
250	0.00585658991571682\\
251	0.0058558022186289\\
252	0.00585500055500476\\
253	0.00585418466348795\\
254	0.0058533542777985\\
255	0.00585250912661443\\
256	0.00585164893344955\\
257	0.00585077341653201\\
258	0.00584988228867367\\
259	0.00584897525713348\\
260	0.00584805202347618\\
261	0.00584711228342581\\
262	0.00584615572671486\\
263	0.00584518203692854\\
264	0.00584419089134428\\
265	0.00584318196076571\\
266	0.00584215490935187\\
267	0.00584110939444136\\
268	0.00584004506637146\\
269	0.00583896156829252\\
270	0.00583785853597729\\
271	0.00583673559762553\\
272	0.00583559237366397\\
273	0.00583442847654169\\
274	0.00583324351052077\\
275	0.00583203707146267\\
276	0.00583080874661024\\
277	0.00582955811436523\\
278	0.0058282847440624\\
279	0.00582698819574003\\
280	0.00582566801990627\\
281	0.00582432375730181\\
282	0.00582295493865824\\
283	0.00582156108445193\\
284	0.00582014170465437\\
285	0.00581869629847926\\
286	0.005817224354126\\
287	0.00581572534851926\\
288	0.00581419874704411\\
289	0.0058126440032749\\
290	0.00581106055869989\\
291	0.00580944784244093\\
292	0.00580780527096708\\
293	0.00580613224780166\\
294	0.00580442816322209\\
295	0.00580269239395184\\
296	0.00580092430284353\\
297	0.00579912323855193\\
298	0.00579728853519571\\
299	0.00579541951200662\\
300	0.00579351547296668\\
301	0.00579157570643923\\
302	0.00578959948478311\\
303	0.00578758606394963\\
304	0.00578553468306317\\
305	0.00578344456398434\\
306	0.00578131491085511\\
307	0.00577914490962455\\
308	0.00577693372755404\\
309	0.00577468051269665\\
310	0.0057723843933358\\
311	0.00577004447737454\\
312	0.00576765985167197\\
313	0.00576522958131156\\
314	0.00576275270896784\\
315	0.00576022825415838\\
316	0.00575765521239794\\
317	0.00575503255430145\\
318	0.00575235922465439\\
319	0.00574963414144133\\
320	0.00574685619482496\\
321	0.00574402424607719\\
322	0.00574113712646314\\
323	0.0057381936360782\\
324	0.00573519254263726\\
325	0.00573213258018739\\
326	0.0057290124477552\\
327	0.00572583080793719\\
328	0.00572258628541859\\
329	0.00571927746540623\\
330	0.00571590289196394\\
331	0.00571246106628896\\
332	0.00570895044499687\\
333	0.00570536943826348\\
334	0.00570171640787011\\
335	0.00569798966517718\\
336	0.00569418746904691\\
337	0.00569030802371435\\
338	0.00568634947646598\\
339	0.00568230991512193\\
340	0.00567818736560189\\
341	0.00567397978950855\\
342	0.00566968508174158\\
343	0.0056653010681927\\
344	0.00566082550358261\\
345	0.00565625606949576\\
346	0.00565159037262873\\
347	0.0056468259431447\\
348	0.00564196023320115\\
349	0.00563699061497836\\
350	0.00563191438172003\\
351	0.00562672874685184\\
352	0.00562143084105162\\
353	0.0056160177086346\\
354	0.00561048630343991\\
355	0.00560483348405053\\
356	0.0055990560097217\\
357	0.0055931505361707\\
358	0.0055871136104212\\
359	0.00558094166527853\\
360	0.00557463101373626\\
361	0.00556817784129923\\
362	0.00556157819448539\\
363	0.00555482796952004\\
364	0.00554792289961393\\
365	0.00554085854011185\\
366	0.0055336302511991\\
367	0.00552623317787879\\
368	0.00551866222696334\\
369	0.00551091204059506\\
370	0.00550297696512918\\
371	0.00549485102078616\\
372	0.00548652786910693\\
373	0.0054780007812667\\
374	0.00546926261160376\\
375	0.00546030577965059\\
376	0.00545112223515048\\
377	0.0054417035439922\\
378	0.00543204103315766\\
379	0.00542212581691445\\
380	0.00541194884783419\\
381	0.00540150093385621\\
382	0.00539077275646738\\
383	0.00537975491110892\\
384	0.0053684379573355\\
385	0.00535681247564071\\
386	0.00534486912991896\\
387	0.00533259873379562\\
388	0.00531999231713941\\
389	0.00530704116034601\\
390	0.00529373682989795\\
391	0.00528007118827396\\
392	0.00526603633846951\\
393	0.00525162446004344\\
394	0.00523682738499222\\
395	0.00522163623423634\\
396	0.00520604069801403\\
397	0.00519002813110791\\
398	0.00517358147230424\\
399	0.00515667352214649\\
400	0.0051392739056539\\
401	0.00512135042052818\\
402	0.00510286916788613\\
403	0.00508379479328046\\
404	0.00506409088727603\\
405	0.0050437206611807\\
406	0.00502264760897073\\
407	0.00500083663684043\\
408	0.00497825615655347\\
409	0.00495487846400973\\
410	0.00493068335478271\\
411	0.00490566247336895\\
412	0.00487982473353056\\
413	0.00485321070322321\\
414	0.00482589774349329\\
415	0.00479801353163928\\
416	0.0047700960633323\\
417	0.00474249402177076\\
418	0.00471527022582038\\
419	0.00468849313252759\\
420	0.00466223684002734\\
421	0.00463658110994893\\
422	0.00461161107985043\\
423	0.00458741670597206\\
424	0.00456409123097715\\
425	0.00454172905936182\\
426	0.00452042247842255\\
427	0.00450025711351624\\
428	0.00448130550552234\\
429	0.00446361815966903\\
430	0.0044472112158671\\
431	0.0044320496150116\\
432	0.00441776627010051\\
433	0.00440382560375862\\
434	0.00439023849119318\\
435	0.00437701253035674\\
436	0.00436415129386796\\
437	0.00435165353474997\\
438	0.00433951235928766\\
439	0.0043277144036239\\
440	0.00431623907618547\\
441	0.00430505796026616\\
442	0.00429413451603094\\
443	0.00428342428335162\\
444	0.00427287587232464\\
445	0.00426243314551698\\
446	0.00425203914870832\\
447	0.00424166417643199\\
448	0.00423129980799707\\
449	0.00422093651362871\\
450	0.00421056372715538\\
451	0.0042001699674827\\
452	0.00418974301691701\\
453	0.00417927016324874\\
454	0.00416873850982556\\
455	0.00415813535282318\\
456	0.00414744861648833\\
457	0.0041366673238222\\
458	0.00412578205906265\\
459	0.00411478534664403\\
460	0.00410367182457272\\
461	0.00409243667364076\\
462	0.00408107508771231\\
463	0.00406958232175505\\
464	0.00405795374048849\\
465	0.00404618486518619\\
466	0.00403427141549785\\
467	0.00402220934250697\\
468	0.00400999484873307\\
469	0.00399762439062231\\
470	0.00398509465953596\\
471	0.00397240253880971\\
472	0.00395954503782476\\
473	0.0039465192102554\\
474	0.00393332212963109\\
475	0.00391995089169573\\
476	0.00390640261499791\\
477	0.00389267443951113\\
478	0.0038787635231776\\
479	0.00386466703641068\\
480	0.00385038215477614\\
481	0.00383590605030284\\
482	0.00382123588213187\\
483	0.00380636878746327\\
484	0.00379130187393502\\
485	0.00377603221454406\\
486	0.003760556845066\\
487	0.00374487276109667\\
488	0.00372897691474455\\
489	0.00371286621101696\\
490	0.00369653750395411\\
491	0.00367998759257086\\
492	0.00366321321666703\\
493	0.00364621105255499\\
494	0.00362897770872998\\
495	0.00361150972146927\\
496	0.00359380355029271\\
497	0.00357585557315651\\
498	0.00355766208122395\\
499	0.00353921927317872\\
500	0.00352052324904254\\
501	0.00350157000345198\\
502	0.00348235541834172\\
503	0.00346287525497366\\
504	0.00344312514524119\\
505	0.00342310058216884\\
506	0.00340279690951775\\
507	0.00338220931039916\\
508	0.00336133279479212\\
509	0.00334016218585757\\
510	0.00331869210493753\\
511	0.00329691695511973\\
512	0.00327483090324128\\
513	0.00325242786019738\\
514	0.00322970145941672\\
515	0.00320664503336107\\
516	0.00318325158790684\\
517	0.00315951377446843\\
518	0.00313542385972952\\
519	0.00311097369285948\\
520	0.00308615467011548\\
521	0.00306095769677305\\
522	0.00303537314646925\\
523	0.00300939081813563\\
524	0.00298299989065061\\
525	0.00295618887532174\\
526	0.00292894556434193\\
527	0.00290125697427681\\
528	0.00287310928924201\\
529	0.00284448780932446\\
530	0.00281537693173909\\
531	0.00278576021769529\\
532	0.00275562044994857\\
533	0.00272493954434936\\
534	0.00269370660143718\\
535	0.00266189593130707\\
536	0.00262948053655158\\
537	0.00259643339538549\\
538	0.0025627243467191\\
539	0.00252832388409115\\
540	0.00249321388509144\\
541	0.00245737974754387\\
542	0.00242081245681518\\
543	0.00238351439691289\\
544	0.00234558287630616\\
545	0.00230699193288679\\
546	0.00226768390732465\\
547	0.00222753786807174\\
548	0.00218642148282379\\
549	0.00214427776531739\\
550	0.00210103341167732\\
551	0.00205662112099617\\
552	0.0020109660400507\\
553	0.00196397980144884\\
554	0.0019155303457963\\
555	0.00186682514613946\\
556	0.00181969807274834\\
557	0.00177480133178205\\
558	0.00172970914963858\\
559	0.00168431053725827\\
560	0.00163846944513358\\
561	0.00159221208723224\\
562	0.00154559597045618\\
563	0.00149870412838683\\
564	0.00145164000108015\\
565	0.00140490722431574\\
566	0.00135932596872896\\
567	0.00131422382038051\\
568	0.00126906825494275\\
569	0.00122384286451314\\
570	0.00117857875099172\\
571	0.00113361617673813\\
572	0.00108890699726297\\
573	0.00104399414514463\\
574	0.000998881283084228\\
575	0.000953600593730909\\
576	0.000908188244841367\\
577	0.000862683608535594\\
578	0.000817129250711461\\
579	0.000771570979621272\\
580	0.000726057863703377\\
581	0.000680642194784237\\
582	0.000635379372495318\\
583	0.000590327686140574\\
584	0.000545547963472517\\
585	0.000501103049530357\\
586	0.000457057072276932\\
587	0.00041347444897773\\
588	0.000370418597003137\\
589	0.000327950360562706\\
590	0.000286126318718801\\
591	0.000244997582319805\\
592	0.000204610904245315\\
593	0.000165017211836693\\
594	0.000126301460681322\\
595	8.88161203105204e-05\\
596	5.34134895574397e-05\\
597	2.21100055488407e-05\\
598	0\\
599	0\\
600	0\\
};
\addplot [color=mycolor6,solid,forget plot]
  table[row sep=crcr]{%
1	0.00590468340290409\\
2	0.00590466875793651\\
3	0.00590465390703613\\
4	0.00590463884782199\\
5	0.00590462357788963\\
6	0.0059046080948106\\
7	0.00590459239613193\\
8	0.00590457647937567\\
9	0.00590456034203833\\
10	0.00590454398159037\\
11	0.00590452739547542\\
12	0.00590451058110989\\
13	0.00590449353588219\\
14	0.0059044762571521\\
15	0.00590445874225004\\
16	0.00590444098847636\\
17	0.00590442299310058\\
18	0.0059044047533605\\
19	0.00590438626646154\\
20	0.00590436752957581\\
21	0.00590434853984111\\
22	0.00590432929436019\\
23	0.00590430979019961\\
24	0.0059042900243888\\
25	0.0059042699939191\\
26	0.00590424969574257\\
27	0.00590422912677091\\
28	0.00590420828387435\\
29	0.00590418716388034\\
30	0.00590416576357244\\
31	0.00590414407968894\\
32	0.00590412210892162\\
33	0.00590409984791431\\
34	0.00590407729326157\\
35	0.00590405444150706\\
36	0.00590403128914223\\
37	0.00590400783260469\\
38	0.00590398406827658\\
39	0.00590395999248299\\
40	0.00590393560149027\\
41	0.00590391089150424\\
42	0.00590388585866846\\
43	0.00590386049906233\\
44	0.00590383480869931\\
45	0.0059038087835249\\
46	0.00590378241941467\\
47	0.00590375571217231\\
48	0.00590372865752741\\
49	0.00590370125113351\\
50	0.00590367348856571\\
51	0.00590364536531862\\
52	0.00590361687680403\\
53	0.00590358801834854\\
54	0.00590355878519121\\
55	0.00590352917248113\\
56	0.00590349917527505\\
57	0.00590346878853472\\
58	0.00590343800712437\\
59	0.00590340682580811\\
60	0.0059033752392473\\
61	0.00590334324199781\\
62	0.00590331082850727\\
63	0.00590327799311232\\
64	0.00590324473003571\\
65	0.00590321103338355\\
66	0.00590317689714224\\
67	0.00590314231517563\\
68	0.00590310728122199\\
69	0.00590307178889096\\
70	0.00590303583166055\\
71	0.00590299940287396\\
72	0.00590296249573647\\
73	0.00590292510331232\\
74	0.00590288721852144\\
75	0.00590284883413628\\
76	0.00590280994277845\\
77	0.00590277053691555\\
78	0.00590273060885776\\
79	0.0059026901507546\\
80	0.00590264915459143\\
81	0.00590260761218613\\
82	0.00590256551518568\\
83	0.00590252285506267\\
84	0.00590247962311196\\
85	0.00590243581044699\\
86	0.00590239140799638\\
87	0.00590234640650043\\
88	0.00590230079650753\\
89	0.00590225456837046\\
90	0.00590220771224303\\
91	0.00590216021807623\\
92	0.00590211207561474\\
93	0.00590206327439308\\
94	0.00590201380373216\\
95	0.00590196365273521\\
96	0.00590191281028439\\
97	0.00590186126503672\\
98	0.00590180900542042\\
99	0.00590175601963092\\
100	0.00590170229562716\\
101	0.00590164782112746\\
102	0.00590159258360568\\
103	0.00590153657028705\\
104	0.00590147976814426\\
105	0.00590142216389328\\
106	0.00590136374398914\\
107	0.00590130449462175\\
108	0.00590124440171157\\
109	0.00590118345090519\\
110	0.00590112162757103\\
111	0.00590105891679464\\
112	0.0059009953033743\\
113	0.00590093077181617\\
114	0.00590086530632965\\
115	0.00590079889082242\\
116	0.00590073150889556\\
117	0.00590066314383845\\
118	0.00590059377862367\\
119	0.00590052339590161\\
120	0.00590045197799523\\
121	0.00590037950689453\\
122	0.00590030596425081\\
123	0.00590023133137115\\
124	0.00590015558921236\\
125	0.00590007871837515\\
126	0.00590000069909783\\
127	0.00589992151125026\\
128	0.00589984113432733\\
129	0.00589975954744251\\
130	0.00589967672932106\\
131	0.00589959265829342\\
132	0.00589950731228811\\
133	0.00589942066882459\\
134	0.00589933270500623\\
135	0.00589924339751263\\
136	0.00589915272259224\\
137	0.00589906065605457\\
138	0.00589896717326231\\
139	0.00589887224912327\\
140	0.00589877585808217\\
141	0.00589867797411222\\
142	0.00589857857070645\\
143	0.00589847762086901\\
144	0.00589837509710623\\
145	0.00589827097141713\\
146	0.0058981652152841\\
147	0.00589805779966319\\
148	0.00589794869497398\\
149	0.00589783787108906\\
150	0.00589772529732377\\
151	0.00589761094242572\\
152	0.0058974947745641\\
153	0.00589737676131866\\
154	0.00589725686966849\\
155	0.00589713506598066\\
156	0.00589701131599841\\
157	0.00589688558482921\\
158	0.0058967578369325\\
159	0.00589662803610724\\
160	0.00589649614547907\\
161	0.0058963621274872\\
162	0.00589622594387109\\
163	0.0058960875556567\\
164	0.00589594692314268\\
165	0.00589580400588577\\
166	0.00589565876268648\\
167	0.00589551115157401\\
168	0.00589536112979095\\
169	0.00589520865377784\\
170	0.00589505367915706\\
171	0.00589489616071657\\
172	0.00589473605239332\\
173	0.0058945733072562\\
174	0.00589440787748867\\
175	0.0058942397143711\\
176	0.00589406876826257\\
177	0.00589389498858256\\
178	0.0058937183237918\\
179	0.00589353872137333\\
180	0.00589335612781283\\
181	0.00589317048857853\\
182	0.00589298174810084\\
183	0.00589278984975164\\
184	0.00589259473582314\\
185	0.00589239634750617\\
186	0.00589219462486845\\
187	0.00589198950683209\\
188	0.00589178093115097\\
189	0.00589156883438758\\
190	0.00589135315188949\\
191	0.00589113381776557\\
192	0.00589091076486166\\
193	0.00589068392473583\\
194	0.0058904532276335\\
195	0.00589021860246186\\
196	0.00588997997676424\\
197	0.00588973727669382\\
198	0.0058894904269871\\
199	0.00588923935093704\\
200	0.00588898397036569\\
201	0.00588872420559656\\
202	0.00588845997542659\\
203	0.00588819119709763\\
204	0.0058879177862678\\
205	0.00588763965698221\\
206	0.0058873567216434\\
207	0.0058870688909814\\
208	0.00588677607402335\\
209	0.00588647817806277\\
210	0.00588617510862836\\
211	0.00588586676945228\\
212	0.00588555306243819\\
213	0.00588523388762864\\
214	0.00588490914317202\\
215	0.00588457872528902\\
216	0.00588424252823844\\
217	0.0058839004442827\\
218	0.00588355236365237\\
219	0.00588319817451027\\
220	0.00588283776291499\\
221	0.00588247101278344\\
222	0.00588209780585283\\
223	0.00588171802164172\\
224	0.00588133153741025\\
225	0.00588093822811953\\
226	0.00588053796638988\\
227	0.00588013062245819\\
228	0.00587971606413418\\
229	0.0058792941567554\\
230	0.00587886476314115\\
231	0.00587842774354511\\
232	0.0058779829556065\\
233	0.00587753025430005\\
234	0.00587706949188437\\
235	0.00587660051784882\\
236	0.00587612317885884\\
237	0.00587563731869965\\
238	0.00587514277821815\\
239	0.00587463939526331\\
240	0.00587412700462452\\
241	0.00587360543796824\\
242	0.00587307452377284\\
243	0.0058725340872614\\
244	0.00587198395033262\\
245	0.00587142393148978\\
246	0.00587085384576781\\
247	0.00587027350465801\\
248	0.00586968271603117\\
249	0.00586908128405838\\
250	0.00586846900912979\\
251	0.00586784568777148\\
252	0.00586721111255999\\
253	0.00586656507203495\\
254	0.00586590735060949\\
255	0.00586523772847881\\
256	0.00586455598152726\\
257	0.00586386188123273\\
258	0.00586315519456908\\
259	0.0058624356839065\\
260	0.00586170310690962\\
261	0.00586095721643395\\
262	0.00586019776041985\\
263	0.0058594244817849\\
264	0.00585863711831371\\
265	0.00585783540254617\\
266	0.00585701906166307\\
267	0.00585618781737013\\
268	0.00585534138577965\\
269	0.0058544794772901\\
270	0.00585360179646345\\
271	0.00585270804190059\\
272	0.00585179790611422\\
273	0.0058508710753995\\
274	0.0058499272297025\\
275	0.00584896604248601\\
276	0.00584798718059275\\
277	0.00584699030410621\\
278	0.00584597506620858\\
279	0.00584494111303573\\
280	0.00584388808352895\\
281	0.00584281560928368\\
282	0.00584172331439418\\
283	0.00584061081529492\\
284	0.00583947772059793\\
285	0.00583832363092568\\
286	0.00583714813873986\\
287	0.00583595082816481\\
288	0.00583473127480586\\
289	0.0058334890455621\\
290	0.00583222369843335\\
291	0.00583093478232062\\
292	0.00582962183681989\\
293	0.00582828439200871\\
294	0.00582692196822549\\
295	0.00582553407584027\\
296	0.00582412021501751\\
297	0.00582267987546966\\
298	0.00582121253620148\\
299	0.00581971766524496\\
300	0.0058181947193846\\
301	0.00581664314387221\\
302	0.00581506237213053\\
303	0.00581345182544603\\
304	0.00581181091264998\\
305	0.00581013902978819\\
306	0.00580843555977847\\
307	0.00580669987205628\\
308	0.00580493132220713\\
309	0.00580312925158503\\
310	0.00580129298691616\\
311	0.00579942183988817\\
312	0.00579751510672676\\
313	0.00579557206777081\\
314	0.00579359198702791\\
315	0.00579157411171272\\
316	0.00578951767177127\\
317	0.00578742187939196\\
318	0.00578528592850206\\
319	0.00578310899424838\\
320	0.00578089023246223\\
321	0.00577862877910772\\
322	0.0057763237497129\\
323	0.00577397423878248\\
324	0.00577157931918913\\
325	0.00576913804154369\\
326	0.00576664943354335\\
327	0.00576411249929563\\
328	0.00576152621861568\\
329	0.00575888954629525\\
330	0.00575620141134653\\
331	0.00575346071622257\\
332	0.00575066633600049\\
333	0.00574781711752994\\
334	0.0057449118785474\\
335	0.0057419494067554\\
336	0.00573892845886111\\
337	0.00573584775955973\\
338	0.00573270600046262\\
339	0.00572950183898877\\
340	0.00572623389720769\\
341	0.00572290076062817\\
342	0.00571950097692899\\
343	0.0057160330546261\\
344	0.0057124954616689\\
345	0.00570888662395184\\
346	0.00570520492372011\\
347	0.00570144869786388\\
348	0.00569761623608626\\
349	0.00569370577918557\\
350	0.00568971551707695\\
351	0.00568564358665135\\
352	0.00568148806948192\\
353	0.00567724698939166\\
354	0.00567291830988646\\
355	0.00566849993155864\\
356	0.00566398968937646\\
357	0.00565938534980228\\
358	0.00565468460779411\\
359	0.00564988508369076\\
360	0.00564498431980329\\
361	0.00563997977664507\\
362	0.0056348688291516\\
363	0.00562964876276981\\
364	0.0056243167693902\\
365	0.00561886994313852\\
366	0.00561330527605117\\
367	0.00560761965366104\\
368	0.00560180985050265\\
369	0.00559587252554999\\
370	0.00558980421797721\\
371	0.00558360134298044\\
372	0.00557726018776682\\
373	0.00557077690767884\\
374	0.00556414752193276\\
375	0.00555736790745322\\
376	0.00555043379876151\\
377	0.00554334078460109\\
378	0.00553608430173897\\
379	0.00552865962902432\\
380	0.00552106187709049\\
381	0.00551328597684666\\
382	0.00550532666814631\\
383	0.0054971784872554\\
384	0.00548883575262141\\
385	0.00548029254866431\\
386	0.00547154270728018\\
387	0.00546257978652851\\
388	0.005453397044808\\
389	0.00544398741351953\\
390	0.00543434346693309\\
391	0.0054244573877434\\
392	0.00541432092740871\\
393	0.00540392535923411\\
394	0.00539326145515441\\
395	0.00538231946491431\\
396	0.0053710891204171\\
397	0.00535955963133878\\
398	0.00534771972949598\\
399	0.0053355578458907\\
400	0.00532306216854998\\
401	0.00531022068392572\\
402	0.00529702122647642\\
403	0.00528345153681775\\
404	0.00526949932806628\\
405	0.00525515233608147\\
406	0.00524039837866732\\
407	0.00522522542145117\\
408	0.00520962147480793\\
409	0.0051935746538616\\
410	0.00517707320074113\\
411	0.00516010544766586\\
412	0.00514266003806422\\
413	0.00512472503489213\\
414	0.005106287077587\\
415	0.00508733024375371\\
416	0.00506783371395862\\
417	0.00504776645706047\\
418	0.00502709125794974\\
419	0.00500576929221536\\
420	0.00498376072912799\\
421	0.00496102502375408\\
422	0.00493752090317557\\
423	0.00491320597827768\\
424	0.00488804049162593\\
425	0.00486199133725965\\
426	0.00483503887956531\\
427	0.00480717787557458\\
428	0.00477842379352288\\
429	0.0047488212357013\\
430	0.00471845512139913\\
431	0.0046874654866577\\
432	0.00465631307862017\\
433	0.00462561156414618\\
434	0.00459543983872804\\
435	0.00456588365269817\\
436	0.00453703556819766\\
437	0.00450899410422867\\
438	0.00448186264312013\\
439	0.00445574775678373\\
440	0.00443075661614078\\
441	0.00440699314951797\\
442	0.00438455250599638\\
443	0.00436351323469214\\
444	0.00434392639820229\\
445	0.00432580057556547\\
446	0.00430908150370947\\
447	0.00429318860155691\\
448	0.00427768535355789\\
449	0.00426257968591244\\
450	0.00424787480931541\\
451	0.0042335682976457\\
452	0.00421965112901356\\
453	0.00420610673759469\\
454	0.00419291015256491\\
455	0.00418002734009607\\
456	0.00416741491750858\\
457	0.00415502047892401\\
458	0.00414278388644718\\
459	0.00413064002251907\\
460	0.00411852369359921\\
461	0.00410641085200688\\
462	0.00409429084603178\\
463	0.00408215182083046\\
464	0.00406998084925039\\
465	0.00405776412744179\\
466	0.00404548724370351\\
467	0.00403313552625417\\
468	0.00402069446996158\\
469	0.00400815023234441\\
470	0.00399549017400645\\
471	0.0039827033951195\\
472	0.00396978118393515\\
473	0.00395671724057395\\
474	0.00394350602644476\\
475	0.00393014203339979\\
476	0.00391661983939307\\
477	0.00390293416378062\\
478	0.00388907991902475\\
479	0.00387505225478129\\
480	0.00386084658963687\\
481	0.00384645862530778\\
482	0.00383188433818874\\
483	0.00381711994419265\\
484	0.00380216183552631\\
485	0.00378700649341192\\
486	0.0037716504093325\\
487	0.00375609008721532\\
488	0.00374032204350607\\
489	0.00372434280486503\\
490	0.00370814890332141\\
491	0.00369173686887879\\
492	0.00367510321977642\\
493	0.00365824445087879\\
494	0.00364115702097106\\
495	0.00362383734004319\\
496	0.00360628175787332\\
497	0.00358848655523246\\
498	0.00357044793801671\\
499	0.00355216203063861\\
500	0.00353362486868388\\
501	0.00351483239085247\\
502	0.00349578043021364\\
503	0.00347646470480874\\
504	0.00345688080763063\\
505	0.00343702419598986\\
506	0.00341689018024293\\
507	0.00339647391180406\\
508	0.00337577037029195\\
509	0.0033547743495868\\
510	0.00333348044258833\\
511	0.00331188302457041\\
512	0.00328997623502093\\
513	0.00326775395784681\\
514	0.00324520979981679\\
515	0.00322233706710684\\
516	0.00319912873980865\\
517	0.00317557744425976\\
518	0.00315167542305859\\
519	0.00312741450263957\\
520	0.00310278605830833\\
521	0.00307778097667462\\
522	0.00305238961546403\\
523	0.00302660176074863\\
524	0.00300040658172168\\
525	0.00297379258326178\\
526	0.00294674755676089\\
527	0.00291925852992653\\
528	0.00289131171640068\\
529	0.00286289246622052\\
530	0.00283398521723053\\
531	0.00280457344714429\\
532	0.00277463963122367\\
533	0.00274416521425818\\
534	0.00271313060469573\\
535	0.0026815152615177\\
536	0.00264929790275776\\
537	0.00261645655312388\\
538	0.00258297183329461\\
539	0.0025488234470146\\
540	0.00251398192579926\\
541	0.00247841834093031\\
542	0.00244210564234407\\
543	0.00240501362073586\\
544	0.00236712119477741\\
545	0.00232842771539922\\
546	0.00228894492744627\\
547	0.00224873067773624\\
548	0.00220782828421262\\
549	0.00216618570376261\\
550	0.00212373698039233\\
551	0.00208030553153823\\
552	0.00203581006666402\\
553	0.00199018245213893\\
554	0.00194333659588346\\
555	0.00189519383286964\\
556	0.00184564302198585\\
557	0.00179463488713487\\
558	0.00174495241332814\\
559	0.00169716065825053\\
560	0.0016505243091927\\
561	0.00160375639864616\\
562	0.00155667487689093\\
563	0.00150924837142255\\
564	0.00146152330798457\\
565	0.00141358649116827\\
566	0.00136554815600325\\
567	0.00131817961727847\\
568	0.00127200212309758\\
569	0.00122621830059046\\
570	0.00118045788475918\\
571	0.00113470471685007\\
572	0.00108911748476318\\
573	0.00104400833163835\\
574	0.000998883762037553\\
575	0.00095360124786219\\
576	0.000908188487090748\\
577	0.000862683722203797\\
578	0.000817129311225428\\
579	0.000771571012213757\\
580	0.000726057881889793\\
581	0.000680642204855582\\
582	0.000635379377867654\\
583	0.000590327688750045\\
584	0.00054554796454357\\
585	0.000501103049846204\\
586	0.000457057072330234\\
587	0.000413474448977726\\
588	0.000370418597003137\\
589	0.000327950360562707\\
590	0.000286126318718801\\
591	0.000244997582319805\\
592	0.000204610904245315\\
593	0.000165017211836693\\
594	0.000126301460681323\\
595	8.88161203105205e-05\\
596	5.34134895574399e-05\\
597	2.21100055488407e-05\\
598	0\\
599	0\\
600	0\\
};
\addplot [color=mycolor7,solid,forget plot]
  table[row sep=crcr]{%
1	0.00590546841052787\\
2	0.00590545833457366\\
3	0.00590544812051456\\
4	0.00590543776662942\\
5	0.00590542727117264\\
6	0.00590541663237348\\
7	0.00590540584843553\\
8	0.00590539491753609\\
9	0.00590538383782547\\
10	0.00590537260742631\\
11	0.00590536122443303\\
12	0.00590534968691108\\
13	0.00590533799289611\\
14	0.00590532614039341\\
15	0.00590531412737702\\
16	0.00590530195178901\\
17	0.00590528961153871\\
18	0.00590527710450184\\
19	0.0059052644285197\\
20	0.00590525158139828\\
21	0.00590523856090745\\
22	0.00590522536477998\\
23	0.00590521199071064\\
24	0.00590519843635526\\
25	0.00590518469932979\\
26	0.00590517077720924\\
27	0.0059051566675267\\
28	0.0059051423677723\\
29	0.00590512787539216\\
30	0.00590511318778731\\
31	0.00590509830231249\\
32	0.00590508321627515\\
33	0.00590506792693418\\
34	0.00590505243149878\\
35	0.00590503672712734\\
36	0.005905020810926\\
37	0.00590500467994756\\
38	0.00590498833119019\\
39	0.00590497176159597\\
40	0.00590495496804984\\
41	0.00590493794737794\\
42	0.00590492069634647\\
43	0.00590490321166016\\
44	0.00590488548996088\\
45	0.00590486752782616\\
46	0.00590484932176777\\
47	0.00590483086823016\\
48	0.00590481216358893\\
49	0.00590479320414936\\
50	0.00590477398614471\\
51	0.00590475450573475\\
52	0.00590473475900402\\
53	0.00590471474196023\\
54	0.00590469445053262\\
55	0.00590467388057021\\
56	0.00590465302784006\\
57	0.00590463188802554\\
58	0.00590461045672465\\
59	0.0059045887294481\\
60	0.00590456670161748\\
61	0.00590454436856355\\
62	0.00590452172552423\\
63	0.00590449876764281\\
64	0.00590447548996593\\
65	0.00590445188744177\\
66	0.00590442795491788\\
67	0.00590440368713947\\
68	0.00590437907874711\\
69	0.00590435412427487\\
70	0.00590432881814813\\
71	0.00590430315468164\\
72	0.00590427712807726\\
73	0.00590425073242188\\
74	0.00590422396168519\\
75	0.00590419680971755\\
76	0.00590416927024773\\
77	0.00590414133688066\\
78	0.00590411300309512\\
79	0.00590408426224143\\
80	0.00590405510753915\\
81	0.00590402553207467\\
82	0.00590399552879873\\
83	0.00590396509052408\\
84	0.00590393420992294\\
85	0.00590390287952453\\
86	0.0059038710917125\\
87	0.00590383883872226\\
88	0.00590380611263852\\
89	0.00590377290539246\\
90	0.00590373920875905\\
91	0.00590370501435438\\
92	0.00590367031363274\\
93	0.00590363509788389\\
94	0.00590359935822993\\
95	0.0059035630856227\\
96	0.00590352627084043\\
97	0.00590348890448488\\
98	0.00590345097697815\\
99	0.00590341247855951\\
100	0.00590337339928215\\
101	0.00590333372900982\\
102	0.00590329345741355\\
103	0.00590325257396822\\
104	0.00590321106794894\\
105	0.00590316892842748\\
106	0.00590312614426865\\
107	0.00590308270412653\\
108	0.00590303859644063\\
109	0.00590299380943207\\
110	0.00590294833109937\\
111	0.00590290214921464\\
112	0.00590285525131915\\
113	0.00590280762471929\\
114	0.00590275925648192\\
115	0.00590271013343025\\
116	0.00590266024213901\\
117	0.00590260956892993\\
118	0.00590255809986686\\
119	0.0059025058207511\\
120	0.00590245271711619\\
121	0.00590239877422297\\
122	0.0059023439770544\\
123	0.00590228831031021\\
124	0.00590223175840156\\
125	0.00590217430544544\\
126	0.0059021159352592\\
127	0.0059020566313547\\
128	0.00590199637693254\\
129	0.00590193515487609\\
130	0.00590187294774557\\
131	0.00590180973777175\\
132	0.00590174550684978\\
133	0.00590168023653298\\
134	0.00590161390802616\\
135	0.00590154650217936\\
136	0.00590147799948121\\
137	0.00590140838005219\\
138	0.00590133762363802\\
139	0.0059012657096031\\
140	0.00590119261692367\\
141	0.00590111832418158\\
142	0.00590104280955797\\
143	0.00590096605082757\\
144	0.00590088802535353\\
145	0.00590080871008369\\
146	0.00590072808154783\\
147	0.00590064611585666\\
148	0.00590056278870112\\
149	0.00590047807534445\\
150	0.00590039195058673\\
151	0.00590030438875595\\
152	0.00590021536369892\\
153	0.00590012484877209\\
154	0.00590003281683201\\
155	0.00589993924022577\\
156	0.00589984409078112\\
157	0.00589974733979652\\
158	0.00589964895803088\\
159	0.00589954891569313\\
160	0.00589944718243163\\
161	0.00589934372732335\\
162	0.00589923851886275\\
163	0.00589913152495061\\
164	0.00589902271288244\\
165	0.00589891204933697\\
166	0.0058987995003641\\
167	0.00589868503137265\\
168	0.00589856860711831\\
169	0.00589845019169064\\
170	0.00589832974850041\\
171	0.00589820724026663\\
172	0.00589808262900302\\
173	0.00589795587600448\\
174	0.00589782694183337\\
175	0.00589769578630525\\
176	0.00589756236847475\\
177	0.00589742664662077\\
178	0.00589728857823184\\
179	0.00589714811999084\\
180	0.00589700522775965\\
181	0.0058968598565635\\
182	0.00589671196057508\\
183	0.00589656149309826\\
184	0.00589640840655157\\
185	0.00589625265245151\\
186	0.00589609418139532\\
187	0.00589593294304377\\
188	0.00589576888610337\\
189	0.00589560195830838\\
190	0.00589543210640261\\
191	0.00589525927612065\\
192	0.00589508341216907\\
193	0.00589490445820704\\
194	0.00589472235682681\\
195	0.00589453704953374\\
196	0.00589434847672596\\
197	0.00589415657767367\\
198	0.00589396129049823\\
199	0.00589376255215059\\
200	0.00589356029838965\\
201	0.00589335446375991\\
202	0.00589314498156892\\
203	0.0058929317838643\\
204	0.00589271480141018\\
205	0.00589249396366328\\
206	0.0058922691987486\\
207	0.00589204043343456\\
208	0.00589180759310755\\
209	0.00589157060174617\\
210	0.00589132938189475\\
211	0.00589108385463654\\
212	0.00589083393956618\\
213	0.00589057955476151\\
214	0.00589032061675508\\
215	0.00589005704050472\\
216	0.00588978873936371\\
217	0.0058895156250501\\
218	0.00588923760761547\\
219	0.00588895459541305\\
220	0.00588866649506475\\
221	0.005888373211428\\
222	0.00588807464756129\\
223	0.00588777070468935\\
224	0.00588746128216713\\
225	0.00588714627744318\\
226	0.00588682558602215\\
227	0.00588649910142634\\
228	0.00588616671515635\\
229	0.00588582831665084\\
230	0.00588548379324532\\
231	0.00588513303013002\\
232	0.00588477591030671\\
233	0.0058844123145445\\
234	0.00588404212133474\\
235	0.00588366520684475\\
236	0.00588328144487065\\
237	0.00588289070678877\\
238	0.00588249286150653\\
239	0.00588208777541169\\
240	0.0058816753123208\\
241	0.00588125533342637\\
242	0.00588082769724298\\
243	0.00588039225955217\\
244	0.00587994887334623\\
245	0.0058794973887707\\
246	0.00587903765306576\\
247	0.00587856951050643\\
248	0.00587809280234142\\
249	0.00587760736673078\\
250	0.00587711303868243\\
251	0.00587660964998708\\
252	0.00587609702915229\\
253	0.00587557500133476\\
254	0.00587504338827168\\
255	0.00587450200821038\\
256	0.00587395067583698\\
257	0.00587338920220323\\
258	0.00587281739465211\\
259	0.00587223505674193\\
260	0.00587164198816886\\
261	0.00587103798468795\\
262	0.00587042283803267\\
263	0.00586979633583266\\
264	0.00586915826152999\\
265	0.00586850839429349\\
266	0.00586784650893159\\
267	0.00586717237580318\\
268	0.00586648576072665\\
269	0.00586578642488698\\
270	0.00586507412474107\\
271	0.00586434861192062\\
272	0.00586360963313332\\
273	0.00586285693006161\\
274	0.00586209023925931\\
275	0.00586130929204601\\
276	0.00586051381439918\\
277	0.00585970352684377\\
278	0.00585887814433981\\
279	0.00585803737616813\\
280	0.00585718092581171\\
281	0.00585630849083357\\
282	0.00585541976275199\\
283	0.00585451442691236\\
284	0.00585359216235555\\
285	0.00585265264168327\\
286	0.00585169553091936\\
287	0.00585072048936791\\
288	0.00584972716946724\\
289	0.00584871521664013\\
290	0.00584768426914011\\
291	0.00584663395789347\\
292	0.00584556390633723\\
293	0.00584447373025259\\
294	0.00584336303759399\\
295	0.0058422314283138\\
296	0.00584107849418192\\
297	0.00583990381860107\\
298	0.00583870697641697\\
299	0.00583748753372358\\
300	0.00583624504766346\\
301	0.00583497906622257\\
302	0.00583368912802002\\
303	0.0058323747620922\\
304	0.00583103548767137\\
305	0.00582967081395859\\
306	0.00582828023989054\\
307	0.0058268632539002\\
308	0.00582541933367116\\
309	0.00582394794588494\\
310	0.00582244854596126\\
311	0.0058209205777915\\
312	0.00581936347346473\\
313	0.00581777665298484\\
314	0.005816159523978\\
315	0.00581451148138946\\
316	0.00581283190716716\\
317	0.00581112016994243\\
318	0.00580937562470352\\
319	0.00580759761245936\\
320	0.00580578545989234\\
321	0.00580393847899998\\
322	0.00580205596672471\\
323	0.00580013720457074\\
324	0.00579818145820796\\
325	0.00579618797706165\\
326	0.00579415599388751\\
327	0.00579208472433111\\
328	0.00578997336647088\\
329	0.00578782110034475\\
330	0.00578562708745906\\
331	0.00578339047027857\\
332	0.00578111037169665\\
333	0.00577878589448524\\
334	0.00577641612072356\\
335	0.00577400011120417\\
336	0.00577153690481451\\
337	0.00576902551789336\\
338	0.0057664649435628\\
339	0.00576385415103406\\
340	0.00576119208488562\\
341	0.00575847766431253\\
342	0.00575570978234539\\
343	0.00575288730503758\\
344	0.00575000907061851\\
345	0.00574707388861067\\
346	0.00574408053890996\\
347	0.00574102777083051\\
348	0.00573791430212861\\
349	0.00573473881797358\\
350	0.0057314999698727\\
351	0.00572819637454793\\
352	0.00572482661276381\\
353	0.00572138922810651\\
354	0.00571788272571942\\
355	0.00571430557098448\\
356	0.00571065618814261\\
357	0.00570693295885431\\
358	0.00570313422069467\\
359	0.00569925826556355\\
360	0.00569530333800685\\
361	0.00569126763347323\\
362	0.00568714929649077\\
363	0.00568294641875502\\
364	0.00567865703712379\\
365	0.00567427913151291\\
366	0.00566981062268538\\
367	0.00566524936992456\\
368	0.00566059316858385\\
369	0.00565583974752377\\
370	0.00565098676639879\\
371	0.00564603181277991\\
372	0.00564097239908317\\
373	0.00563580595924434\\
374	0.00563052984511503\\
375	0.005625141323137\\
376	0.00561963757057203\\
377	0.00561401567146891\\
378	0.00560827261242815\\
379	0.00560240527782194\\
380	0.00559641044477364\\
381	0.00559028477800437\\
382	0.00558402482443303\\
383	0.00557762700750355\\
384	0.00557108762123924\\
385	0.00556440282402394\\
386	0.00555756863209419\\
387	0.00555058091269631\\
388	0.00554343537716995\\
389	0.00553612757386892\\
390	0.00552865288080679\\
391	0.00552100649796893\\
392	0.00551318343945385\\
393	0.00550517852756039\\
394	0.00549698638669302\\
395	0.00548860143780738\\
396	0.00548001789009351\\
397	0.00547122973279356\\
398	0.00546223072678853\\
399	0.0054530143922605\\
400	0.0054435739945392\\
401	0.00543390252840922\\
402	0.00542399270063053\\
403	0.00541383691020215\\
404	0.0054034272247367\\
405	0.00539275535468412\\
406	0.00538181262392792\\
407	0.00537058992780932\\
408	0.00535907770488959\\
409	0.00534726590909281\\
410	0.00533514398475938\\
411	0.00532270085996921\\
412	0.00530992487134049\\
413	0.00529680372629114\\
414	0.00528332448794985\\
415	0.00526947355874959\\
416	0.00525523674381165\\
417	0.00524059944589341\\
418	0.005225546794567\\
419	0.00521006370752857\\
420	0.00519413491090905\\
421	0.00517774491376843\\
422	0.00516087799672983\\
423	0.00514351850418792\\
424	0.00512565111900571\\
425	0.00510726111113186\\
426	0.00508833397563259\\
427	0.00506885526078299\\
428	0.00504881031382298\\
429	0.00502818381973569\\
430	0.00500695904268403\\
431	0.00498511668385148\\
432	0.0049626327749291\\
433	0.00493947399325718\\
434	0.00491559676440567\\
435	0.00489095299498545\\
436	0.00486549206433014\\
437	0.0048391676680193\\
438	0.00481193842453155\\
439	0.00478376779435718\\
440	0.00475462720371412\\
441	0.00472450027553666\\
442	0.0046933885180738\\
443	0.00466131892900443\\
444	0.0046283541132556\\
445	0.004594605694368\\
446	0.00456025212193193\\
447	0.00452597762992022\\
448	0.00449231447466387\\
449	0.00445936026029413\\
450	0.0044272197346658\\
451	0.00439600413019116\\
452	0.00436582992860276\\
453	0.00433681682754361\\
454	0.00430908461065308\\
455	0.00428274851459142\\
456	0.00425791259423312\\
457	0.00423466049179356\\
458	0.00421304246245616\\
459	0.00419305744549281\\
460	0.00417462853232605\\
461	0.00415690944873209\\
462	0.00413962019997972\\
463	0.00412276501290849\\
464	0.00410634185751169\\
465	0.00409034135471902\\
466	0.00407474568594373\\
467	0.00405952758531734\\
468	0.00404464954627945\\
469	0.00403006343814548\\
470	0.00401571080411634\\
471	0.00400152422913765\\
472	0.00398743032716485\\
473	0.00397335511442664\\
474	0.00395926769111637\\
475	0.00394515546851621\\
476	0.00393100453794348\\
477	0.0039167998492336\\
478	0.00390252546881891\\
479	0.00388816492575132\\
480	0.00387370164967247\\
481	0.00385911949642118\\
482	0.00384440334324048\\
483	0.00382953971373353\\
484	0.00381451735932588\\
485	0.00379932767405619\\
486	0.00378396425980192\\
487	0.00376842072626709\\
488	0.00375269075365844\\
489	0.00373676815550082\\
490	0.00372064693795249\\
491	0.00370432135103731\\
492	0.00368778592633672\\
493	0.00367103549507012\\
494	0.0036540651804571\\
495	0.00363687035928918\\
496	0.00361944659051906\\
497	0.00360178951453377\\
498	0.00358389475219438\\
499	0.00356575790433221\\
500	0.00354737454845765\\
501	0.00352874023234101\\
502	0.00350985046425289\\
503	0.00349070069984416\\
504	0.00347128632590631\\
505	0.00345160264157714\\
506	0.003431644837923\\
507	0.00341140797718071\\
508	0.00339088697317599\\
509	0.0033700765743615\\
510	0.00334897134842124\\
511	0.00332756566518767\\
512	0.00330585367781566\\
513	0.00328382930217281\\
514	0.0032614861944159\\
515	0.00323881772672662\\
516	0.003215816961173\\
517	0.00319247662163953\\
518	0.00316878906372772\\
519	0.00314474624246813\\
520	0.00312033967761277\\
521	0.0030955604162116\\
522	0.00307039899235162\\
523	0.0030448453840286\\
524	0.00301888896717868\\
525	0.00299251846697522\\
526	0.00296572190659903\\
527	0.00293848655382617\\
528	0.00291079886596397\\
529	0.00288264443391518\\
530	0.00285400792652315\\
531	0.00282487303684166\\
532	0.0027952224324324\\
533	0.00276503771232443\\
534	0.00273429937400198\\
535	0.00270298679154721\\
536	0.00267107821079603\\
537	0.00263855077512195\\
538	0.00260538059463947\\
539	0.00257154289936966\\
540	0.00253701234040634\\
541	0.0025017633546709\\
542	0.00246577049818809\\
543	0.00242901399438544\\
544	0.00239146970466376\\
545	0.00235310987813035\\
546	0.00231391248249973\\
547	0.00227387357085289\\
548	0.00223298399921975\\
549	0.00219124395363196\\
550	0.00214866854507098\\
551	0.00210535359027717\\
552	0.00206127734338405\\
553	0.00201637958665321\\
554	0.00197057891757341\\
555	0.00192367985441363\\
556	0.00187561454417354\\
557	0.00182630117890887\\
558	0.00177563615587418\\
559	0.0017234803562672\\
560	0.0016710561887658\\
561	0.00162005618808052\\
562	0.00157114435097566\\
563	0.00152278495823474\\
564	0.00147437765394238\\
565	0.00142573906631173\\
566	0.00137688342679055\\
567	0.00132788753056811\\
568	0.00127885728545426\\
569	0.00123068072321142\\
570	0.00118373042443368\\
571	0.00113727730782157\\
572	0.0010909595635236\\
573	0.00104473750684888\\
574	0.000998981829455881\\
575	0.00095361824078903\\
576	0.000908192826501344\\
577	0.000862685274585634\\
578	0.000817130023225066\\
579	0.000771571389571413\\
580	0.000726058083632912\\
581	0.000680642317466521\\
582	0.000635379440543511\\
583	0.000590327722563027\\
584	0.000545547981242051\\
585	0.000501103056863606\\
586	0.000457057074448636\\
587	0.000413474449346032\\
588	0.000370418597003137\\
589	0.000327950360562705\\
590	0.0002861263187188\\
591	0.000244997582319805\\
592	0.000204610904245314\\
593	0.000165017211836693\\
594	0.000126301460681323\\
595	8.88161203105205e-05\\
596	5.34134895574397e-05\\
597	2.21100055488407e-05\\
598	0\\
599	0\\
600	0\\
};
\addplot [color=mycolor8,solid,forget plot]
  table[row sep=crcr]{%
1	0.00590802353428763\\
2	0.00590801630497429\\
3	0.00590800897473085\\
4	0.00590800154213922\\
5	0.00590799400575707\\
6	0.00590798636411732\\
7	0.00590797861572751\\
8	0.00590797075906941\\
9	0.00590796279259819\\
10	0.00590795471474214\\
11	0.00590794652390183\\
12	0.00590793821844956\\
13	0.00590792979672887\\
14	0.00590792125705373\\
15	0.00590791259770804\\
16	0.0059079038169449\\
17	0.00590789491298587\\
18	0.00590788588402048\\
19	0.00590787672820538\\
20	0.00590786744366364\\
21	0.00590785802848411\\
22	0.00590784848072059\\
23	0.0059078387983912\\
24	0.00590782897947751\\
25	0.00590781902192376\\
26	0.00590780892363614\\
27	0.00590779868248203\\
28	0.00590778829628906\\
29	0.00590777776284428\\
30	0.00590776707989345\\
31	0.0059077562451401\\
32	0.00590774525624463\\
33	0.00590773411082346\\
34	0.0059077228064481\\
35	0.00590771134064428\\
36	0.005907699710891\\
37	0.00590768791461956\\
38	0.00590767594921261\\
39	0.00590766381200326\\
40	0.00590765150027391\\
41	0.00590763901125545\\
42	0.00590762634212604\\
43	0.00590761349001027\\
44	0.00590760045197793\\
45	0.00590758722504313\\
46	0.00590757380616302\\
47	0.00590756019223683\\
48	0.00590754638010476\\
49	0.00590753236654675\\
50	0.00590751814828145\\
51	0.00590750372196487\\
52	0.00590748908418953\\
53	0.00590747423148292\\
54	0.0059074591603065\\
55	0.00590744386705438\\
56	0.00590742834805213\\
57	0.00590741259955545\\
58	0.00590739661774893\\
59	0.00590738039874481\\
60	0.00590736393858155\\
61	0.0059073472332225\\
62	0.00590733027855464\\
63	0.00590731307038715\\
64	0.00590729560445\\
65	0.00590727787639254\\
66	0.00590725988178212\\
67	0.00590724161610255\\
68	0.00590722307475266\\
69	0.00590720425304483\\
70	0.00590718514620346\\
71	0.00590716574936337\\
72	0.00590714605756831\\
73	0.00590712606576926\\
74	0.00590710576882298\\
75	0.00590708516149026\\
76	0.00590706423843425\\
77	0.00590704299421882\\
78	0.00590702142330676\\
79	0.00590699952005819\\
80	0.00590697727872856\\
81	0.00590695469346705\\
82	0.00590693175831464\\
83	0.00590690846720231\\
84	0.00590688481394906\\
85	0.00590686079225999\\
86	0.00590683639572437\\
87	0.00590681161781369\\
88	0.00590678645187956\\
89	0.00590676089115165\\
90	0.00590673492873556\\
91	0.00590670855761073\\
92	0.00590668177062822\\
93	0.00590665456050847\\
94	0.00590662691983911\\
95	0.00590659884107245\\
96	0.00590657031652335\\
97	0.00590654133836662\\
98	0.00590651189863468\\
99	0.00590648198921498\\
100	0.00590645160184743\\
101	0.00590642072812184\\
102	0.00590638935947516\\
103	0.00590635748718875\\
104	0.00590632510238557\\
105	0.00590629219602741\\
106	0.00590625875891178\\
107	0.00590622478166902\\
108	0.00590619025475909\\
109	0.00590615516846852\\
110	0.00590611951290706\\
111	0.00590608327800438\\
112	0.00590604645350658\\
113	0.00590600902897263\\
114	0.00590597099377086\\
115	0.00590593233707497\\
116	0.00590589304786028\\
117	0.00590585311489977\\
118	0.00590581252675971\\
119	0.00590577127179557\\
120	0.00590572933814746\\
121	0.0059056867137355\\
122	0.005905643386255\\
123	0.00590559934317145\\
124	0.00590555457171525\\
125	0.0059055090588763\\
126	0.00590546279139811\\
127	0.00590541575577188\\
128	0.00590536793822994\\
129	0.00590531932473914\\
130	0.00590526990099337\\
131	0.00590521965240594\\
132	0.00590516856410093\\
133	0.00590511662090402\\
134	0.00590506380733209\\
135	0.00590501010758167\\
136	0.00590495550551558\\
137	0.00590489998464754\\
138	0.005904843528124\\
139	0.00590478611870195\\
140	0.00590472773872228\\
141	0.00590466837007603\\
142	0.00590460799416253\\
143	0.00590454659183592\\
144	0.00590448414333835\\
145	0.00590442062821777\\
146	0.00590435602523644\\
147	0.00590429031230012\\
148	0.005904223466523\\
149	0.00590415546486037\\
150	0.00590408628607652\\
151	0.0059040159085373\\
152	0.00590394431020254\\
153	0.00590387146861833\\
154	0.00590379736090922\\
155	0.00590372196377013\\
156	0.00590364525345832\\
157	0.00590356720578494\\
158	0.00590348779610667\\
159	0.00590340699931706\\
160	0.00590332478983781\\
161	0.0059032411416097\\
162	0.00590315602808361\\
163	0.00590306942221118\\
164	0.00590298129643536\\
165	0.0059028916226808\\
166	0.00590280037234397\\
167	0.00590270751628335\\
168	0.00590261302480897\\
169	0.00590251686767241\\
170	0.00590241901405598\\
171	0.005902319432562\\
172	0.00590221809120199\\
173	0.0059021149573854\\
174	0.00590200999790831\\
175	0.00590190317894187\\
176	0.0059017944660205\\
177	0.00590168382402991\\
178	0.00590157121719488\\
179	0.00590145660906684\\
180	0.00590133996251117\\
181	0.00590122123969432\\
182	0.00590110040207058\\
183	0.00590097741036876\\
184	0.00590085222457861\\
185	0.00590072480393673\\
186	0.00590059510691262\\
187	0.00590046309119406\\
188	0.00590032871367252\\
189	0.00590019193042812\\
190	0.00590005269671445\\
191	0.00589991096694289\\
192	0.00589976669466679\\
193	0.00589961983256537\\
194	0.00589947033242719\\
195	0.00589931814513329\\
196	0.00589916322064012\\
197	0.00589900550796218\\
198	0.00589884495515392\\
199	0.0058986815092919\\
200	0.00589851511645607\\
201	0.00589834572171095\\
202	0.00589817326908631\\
203	0.00589799770155759\\
204	0.00589781896102576\\
205	0.00589763698829691\\
206	0.00589745172306129\\
207	0.00589726310387215\\
208	0.00589707106812389\\
209	0.00589687555202988\\
210	0.00589667649059989\\
211	0.00589647381761683\\
212	0.00589626746561332\\
213	0.00589605736584751\\
214	0.00589584344827859\\
215	0.00589562564154159\\
216	0.0058954038729219\\
217	0.00589517806832906\\
218	0.0058949481522701\\
219	0.0058947140478223\\
220	0.00589447567660545\\
221	0.00589423295875337\\
222	0.00589398581288512\\
223	0.00589373415607526\\
224	0.00589347790382388\\
225	0.00589321697002569\\
226	0.00589295126693872\\
227	0.00589268070515217\\
228	0.00589240519355378\\
229	0.0058921246392965\\
230	0.00589183894776449\\
231	0.00589154802253821\\
232	0.00589125176535926\\
233	0.00589095007609415\\
234	0.00589064285269744\\
235	0.00589032999117428\\
236	0.00589001138554192\\
237	0.00588968692779095\\
238	0.0058893565078451\\
239	0.0058890200135209\\
240	0.00588867733048607\\
241	0.00588832834221746\\
242	0.00588797292995786\\
243	0.00588761097267214\\
244	0.00588724234700246\\
245	0.00588686692722262\\
246	0.00588648458519144\\
247	0.00588609519030535\\
248	0.0058856986094498\\
249	0.00588529470695005\\
250	0.00588488334452054\\
251	0.00588446438121353\\
252	0.00588403767336654\\
253	0.00588360307454875\\
254	0.0058831604355063\\
255	0.0058827096041064\\
256	0.00588225042528007\\
257	0.00588178274096392\\
258	0.00588130639004054\\
259	0.0058808212082773\\
260	0.00588032702826426\\
261	0.00587982367935019\\
262	0.00587931098757728\\
263	0.0058787887756143\\
264	0.0058782568626881\\
265	0.00587771506451347\\
266	0.00587716319322116\\
267	0.00587660105728395\\
268	0.00587602846144095\\
269	0.00587544520661932\\
270	0.00587485108985398\\
271	0.00587424590420423\\
272	0.00587362943866753\\
273	0.00587300147808908\\
274	0.00587236180306697\\
275	0.0058717101898511\\
276	0.00587104641023446\\
277	0.00587037023143491\\
278	0.0058696814159676\\
279	0.00586897972152364\\
280	0.00586826490097686\\
281	0.00586753670228778\\
282	0.00586679486839854\\
283	0.00586603913712516\\
284	0.00586526924104771\\
285	0.00586448490739712\\
286	0.00586368585793983\\
287	0.00586287180885939\\
288	0.00586204247063553\\
289	0.00586119754792022\\
290	0.00586033673941081\\
291	0.00585945973772049\\
292	0.00585856622924569\\
293	0.00585765589403042\\
294	0.00585672840562798\\
295	0.00585578343095918\\
296	0.00585482063016807\\
297	0.0058538396564742\\
298	0.00585284015602224\\
299	0.0058518217677283\\
300	0.00585078412312331\\
301	0.00584972684619354\\
302	0.00584864955321813\\
303	0.00584755185260379\\
304	0.00584643334471714\\
305	0.00584529362171465\\
306	0.00584413226737116\\
307	0.00584294885690771\\
308	0.00584174295682031\\
309	0.0058405141247117\\
310	0.00583926190912996\\
311	0.00583798584941846\\
312	0.005836685475584\\
313	0.00583536030819091\\
314	0.00583400985828665\\
315	0.0058326336273461\\
316	0.00583123110714031\\
317	0.00582980177906359\\
318	0.00582834511371602\\
319	0.00582686057065276\\
320	0.00582534759812624\\
321	0.00582380563282054\\
322	0.00582223409957815\\
323	0.00582063241111845\\
324	0.00581899996774759\\
325	0.00581733615705988\\
326	0.00581564035362977\\
327	0.00581391191869461\\
328	0.00581215019982756\\
329	0.00581035453060042\\
330	0.00580852423023593\\
331	0.00580665860324909\\
332	0.00580475693907734\\
333	0.00580281851169891\\
334	0.00580084257923887\\
335	0.00579882838356263\\
336	0.00579677514985606\\
337	0.00579468208619208\\
338	0.00579254838308299\\
339	0.00579037321301806\\
340	0.00578815572998549\\
341	0.00578589506897827\\
342	0.00578359034548299\\
343	0.00578124065495101\\
344	0.00577884507225076\\
345	0.00577640265110077\\
346	0.00577391242348277\\
347	0.00577137339903472\\
348	0.00576878456442045\\
349	0.00576614488267504\\
350	0.0057634532925253\\
351	0.00576070870768393\\
352	0.00575791001611674\\
353	0.00575505607928187\\
354	0.00575214573133892\\
355	0.00574917777832662\\
356	0.00574615099730748\\
357	0.00574306413547791\\
358	0.00573991590924122\\
359	0.00573670500324364\\
360	0.00573343006937158\\
361	0.00573008972569926\\
362	0.00572668255539077\\
363	0.00572320710555429\\
364	0.0057196618860459\\
365	0.00571604536821996\\
366	0.00571235598362319\\
367	0.00570859212262963\\
368	0.00570475213301496\\
369	0.00570083431846369\\
370	0.00569683693700552\\
371	0.00569275819937571\\
372	0.00568859626729262\\
373	0.00568434925165601\\
374	0.0056800152106995\\
375	0.00567559214803389\\
376	0.00567107801059165\\
377	0.00566647068646577\\
378	0.00566176800261187\\
379	0.00565696772243323\\
380	0.00565206754324283\\
381	0.00564706509362413\\
382	0.00564195793067386\\
383	0.00563674353710629\\
384	0.00563141931820979\\
385	0.0056259825986447\\
386	0.00562043061907116\\
387	0.0056147605326124\\
388	0.0056089694011292\\
389	0.00560305419127803\\
390	0.00559701177033551\\
391	0.00559083890180946\\
392	0.00558453224096251\\
393	0.00557808833003779\\
394	0.00557150359318984\\
395	0.00556477433087966\\
396	0.00555789671398568\\
397	0.00555086677762797\\
398	0.00554368041457364\\
399	0.00553633336827129\\
400	0.00552882122551482\\
401	0.00552113940871941\\
402	0.00551328316777066\\
403	0.00550524757136001\\
404	0.00549702749793597\\
405	0.00548861762611267\\
406	0.00548001242419854\\
407	0.0054712061408935\\
408	0.00546219279608239\\
409	0.00545296617182449\\
410	0.00544351980363581\\
411	0.00543384696606244\\
412	0.00542394066082811\\
413	0.00541379360523075\\
414	0.00540339821817247\\
415	0.00539274660862177\\
416	0.00538183056359896\\
417	0.0053706415307633\\
418	0.00535917059686725\\
419	0.00534740845962554\\
420	0.00533534539380005\\
421	0.00532297122059882\\
422	0.00531027529883268\\
423	0.00529724651062267\\
424	0.00528387323452439\\
425	0.00527014327336349\\
426	0.0052560438045496\\
427	0.00524156133578352\\
428	0.0052266816635043\\
429	0.00521138983961713\\
430	0.00519567015480811\\
431	0.00517950613184211\\
432	0.00516288056946737\\
433	0.00514577563265416\\
434	0.00512817288665004\\
435	0.0051100535863273\\
436	0.00509139925137241\\
437	0.00507219164355963\\
438	0.00505241258518674\\
439	0.00503204397804265\\
440	0.00501106781927507\\
441	0.00498946615169442\\
442	0.00496722091004136\\
443	0.00494431361570609\\
444	0.00492072489886094\\
445	0.00489643408364169\\
446	0.00487141414822884\\
447	0.00484562979888623\\
448	0.00481903642213972\\
449	0.00479158537430365\\
450	0.004763228102476\\
451	0.00473391756542243\\
452	0.00470361022057111\\
453	0.0046722687614305\\
454	0.00463986584641357\\
455	0.0046063891318761\\
456	0.00457184809079168\\
457	0.00453628306388062\\
458	0.00449977717874501\\
459	0.00446247211174984\\
460	0.00442458897286291\\
461	0.00438708476217384\\
462	0.00435035239775994\\
463	0.00431450735850626\\
464	0.00427967241906488\\
465	0.00424597631785417\\
466	0.00421355160993508\\
467	0.00418253137301872\\
468	0.00415304418914572\\
469	0.00412520677216004\\
470	0.00409911365408149\\
471	0.00407482289056936\\
472	0.0040523363874517\\
473	0.0040315730449379\\
474	0.00401164881165539\\
475	0.00399218739683764\\
476	0.00397319059741844\\
477	0.00395465264206548\\
478	0.00393655893484832\\
479	0.00391888484667759\\
480	0.00390159466504802\\
481	0.00388464087834956\\
482	0.00386796405063881\\
483	0.00385149365669067\\
484	0.00383515040115759\\
485	0.00381885075552387\\
486	0.00380252636992823\\
487	0.00378616337875535\\
488	0.00376974638153455\\
489	0.00375325862944356\\
490	0.00373668230125755\\
491	0.0037199988792966\\
492	0.00370318963112153\\
493	0.00368623619386584\\
494	0.00366912124307526\\
495	0.00365182920404782\\
496	0.00363434692661208\\
497	0.00361666418912785\\
498	0.00359877344348116\\
499	0.00358066712062533\\
500	0.00356233769956089\\
501	0.00354377777586562\\
502	0.00352498012534427\\
503	0.00350593775725361\\
504	0.00348664395054972\\
505	0.00346709226595765\\
506	0.00344727652677869\\
507	0.00342719076289304\\
508	0.00340682911637355\\
509	0.00338618571499653\\
510	0.00336525457896681\\
511	0.00334402961160912\\
512	0.00332250458591081\\
513	0.00330067312646816\\
514	0.00327852868656125\\
515	0.00325606452033717\\
516	0.00323327365042335\\
517	0.00321014883171252\\
518	0.00318668251252003\\
519	0.00316286679473023\\
520	0.00313869339476105\\
521	0.00311415360690661\\
522	0.00308923826480123\\
523	0.00306393769958115\\
524	0.0030382416947482\\
525	0.00301213943781063\\
526	0.00298561946885924\\
527	0.00295866962633007\\
528	0.00293127699030934\\
529	0.00290342782385294\\
530	0.00287510751292521\\
531	0.00284630050572302\\
532	0.00281699025237423\\
533	0.00278715914642115\\
534	0.00275678847024927\\
535	0.0027258583474368\\
536	0.00269434770585814\\
537	0.00266223425631255\\
538	0.00262949449273195\\
539	0.00259610372036823\\
540	0.00256203611983489\\
541	0.00252726486215598\\
542	0.00249176229581887\\
543	0.00245550022948785\\
544	0.00241845037920276\\
545	0.00238058505390565\\
546	0.00234187788129203\\
547	0.00230230436425532\\
548	0.00226185053906073\\
549	0.00222051365330463\\
550	0.00217828765766402\\
551	0.00213516904957736\\
552	0.00209115827729499\\
553	0.00204625694083962\\
554	0.00200047918608743\\
555	0.00195392585632444\\
556	0.00190654874064295\\
557	0.00185828393241978\\
558	0.00180903905475592\\
559	0.00175861688358186\\
560	0.0017069408381786\\
561	0.00165390077486336\\
562	0.00159933339332958\\
563	0.00154519250351498\\
564	0.00149251652556966\\
565	0.00144199233608514\\
566	0.00139193664277865\\
567	0.00134193923371883\\
568	0.00129189696277558\\
569	0.00124179721745673\\
570	0.0011917551532044\\
571	0.00114258430627697\\
572	0.00109468008503176\\
573	0.0010475786939303\\
574	0.0010007642291051\\
575	0.000954230162850314\\
576	0.000908306034116644\\
577	0.000862713691105962\\
578	0.000817139873524682\\
579	0.000771575799886298\\
580	0.00072606041348764\\
581	0.000680643552963479\\
582	0.000635380130251366\\
583	0.000590328108249356\\
584	0.000545548191687311\\
585	0.000501103162543998\\
586	0.000457057119965888\\
587	0.000413474463421802\\
588	0.000370418599527697\\
589	0.000327950360562708\\
590	0.000286126318718801\\
591	0.000244997582319806\\
592	0.000204610904245315\\
593	0.000165017211836693\\
594	0.000126301460681323\\
595	8.88161203105207e-05\\
596	5.34134895574399e-05\\
597	2.21100055488407e-05\\
598	0\\
599	0\\
600	0\\
};
\addplot [color=blue!25!mycolor7,solid,forget plot]
  table[row sep=crcr]{%
1	0.0059188651418757\\
2	0.00591885908155997\\
3	0.00591885293182187\\
4	0.00591884669126303\\
5	0.0059188403584599\\
6	0.00591883393196333\\
7	0.00591882741029797\\
8	0.00591882079196169\\
9	0.00591881407542529\\
10	0.00591880725913178\\
11	0.00591880034149585\\
12	0.00591879332090345\\
13	0.0059187861957111\\
14	0.00591877896424534\\
15	0.00591877162480226\\
16	0.00591876417564675\\
17	0.0059187566150121\\
18	0.00591874894109924\\
19	0.00591874115207617\\
20	0.00591873324607738\\
21	0.00591872522120314\\
22	0.00591871707551892\\
23	0.00591870880705471\\
24	0.00591870041380436\\
25	0.0059186918937249\\
26	0.00591868324473587\\
27	0.00591867446471858\\
28	0.00591866555151549\\
29	0.00591865650292945\\
30	0.00591864731672289\\
31	0.00591863799061722\\
32	0.00591862852229202\\
33	0.00591861890938424\\
34	0.00591860914948753\\
35	0.00591859924015131\\
36	0.00591858917888011\\
37	0.00591857896313272\\
38	0.00591856859032132\\
39	0.00591855805781074\\
40	0.00591854736291759\\
41	0.00591853650290934\\
42	0.00591852547500355\\
43	0.00591851427636694\\
44	0.0059185029041145\\
45	0.00591849135530861\\
46	0.00591847962695814\\
47	0.00591846771601745\\
48	0.00591845561938551\\
49	0.00591844333390496\\
50	0.00591843085636111\\
51	0.00591841818348097\\
52	0.00591840531193227\\
53	0.00591839223832242\\
54	0.00591837895919755\\
55	0.00591836547104141\\
56	0.00591835177027443\\
57	0.00591833785325253\\
58	0.00591832371626611\\
59	0.005918309355539\\
60	0.00591829476722731\\
61	0.00591827994741838\\
62	0.00591826489212959\\
63	0.00591824959730721\\
64	0.00591823405882531\\
65	0.00591821827248458\\
66	0.00591820223401108\\
67	0.00591818593905516\\
68	0.00591816938319019\\
69	0.00591815256191135\\
70	0.00591813547063441\\
71	0.00591811810469449\\
72	0.00591810045934478\\
73	0.00591808252975533\\
74	0.00591806431101174\\
75	0.00591804579811384\\
76	0.00591802698597446\\
77	0.00591800786941809\\
78	0.00591798844317966\\
79	0.00591796870190301\\
80	0.0059179486401398\\
81	0.00591792825234804\\
82	0.00591790753289084\\
83	0.00591788647603501\\
84	0.00591786507594974\\
85	0.00591784332670531\\
86	0.00591782122227177\\
87	0.0059177987565175\\
88	0.00591777592320803\\
89	0.00591775271600468\\
90	0.0059177291284633\\
91	0.00591770515403296\\
92	0.00591768078605472\\
93	0.00591765601776043\\
94	0.00591763084227152\\
95	0.00591760525259787\\
96	0.00591757924163657\\
97	0.00591755280217107\\
98	0.00591752592687001\\
99	0.00591749860828638\\
100	0.00591747083885655\\
101	0.00591744261089953\\
102	0.00591741391661631\\
103	0.0059173847480892\\
104	0.00591735509728145\\
105	0.00591732495603683\\
106	0.00591729431607956\\
107	0.00591726316901414\\
108	0.00591723150632576\\
109	0.00591719931938059\\
110	0.00591716659942644\\
111	0.00591713333759381\\
112	0.00591709952489709\\
113	0.00591706515223625\\
114	0.00591703021039891\\
115	0.00591699469006286\\
116	0.00591695858179918\\
117	0.0059169218760759\\
118	0.0059168845632626\\
119	0.00591684663363541\\
120	0.00591680807738351\\
121	0.00591676888461626\\
122	0.00591672904537201\\
123	0.0059166885496281\\
124	0.00591664738731295\\
125	0.00591660554832\\
126	0.00591656302252438\\
127	0.00591651979980229\\
128	0.00591647587005437\\
129	0.00591643122323283\\
130	0.00591638584937416\\
131	0.00591633973863793\\
132	0.00591629288135331\\
133	0.00591624526807472\\
134	0.00591619688964922\\
135	0.00591614773729764\\
136	0.00591609780271336\\
137	0.00591604707818248\\
138	0.00591599555673046\\
139	0.005915943232302\\
140	0.00591589009998083\\
141	0.00591583615625805\\
142	0.00591578139935393\\
143	0.0059157258295878\\
144	0.00591566944975112\\
145	0.00591561226531382\\
146	0.00591555428390163\\
147	0.00591549551221862\\
148	0.00591543594429587\\
149	0.00591537551860708\\
150	0.00591531402421091\\
151	0.00591525144149972\\
152	0.00591518775050186\\
153	0.005915122930875\\
154	0.00591505696189918\\
155	0.0059149898224698\\
156	0.00591492149109051\\
157	0.00591485194586587\\
158	0.00591478116449396\\
159	0.00591470912425885\\
160	0.00591463580202286\\
161	0.0059145611742188\\
162	0.00591448521684186\\
163	0.00591440790544167\\
164	0.0059143292151139\\
165	0.00591424912049185\\
166	0.00591416759573793\\
167	0.00591408461453486\\
168	0.00591400015007682\\
169	0.0059139141750603\\
170	0.00591382666167498\\
171	0.00591373758159426\\
172	0.00591364690596571\\
173	0.00591355460540129\\
174	0.00591346064996747\\
175	0.00591336500917504\\
176	0.00591326765196887\\
177	0.00591316854671734\\
178	0.00591306766120182\\
179	0.00591296496260548\\
180	0.00591286041750253\\
181	0.00591275399184665\\
182	0.0059126456509597\\
183	0.00591253535951979\\
184	0.00591242308154943\\
185	0.00591230878040346\\
186	0.00591219241875638\\
187	0.00591207395859003\\
188	0.00591195336118055\\
189	0.00591183058708528\\
190	0.00591170559612939\\
191	0.00591157834739234\\
192	0.00591144879919404\\
193	0.00591131690908058\\
194	0.00591118263380986\\
195	0.0059110459293371\\
196	0.00591090675079973\\
197	0.00591076505250216\\
198	0.0059106207879004\\
199	0.00591047390958608\\
200	0.00591032436927049\\
201	0.00591017211776811\\
202	0.00591001710497983\\
203	0.00590985927987601\\
204	0.00590969859047909\\
205	0.00590953498384588\\
206	0.00590936840604958\\
207	0.00590919880216138\\
208	0.0059090261162317\\
209	0.00590885029127128\\
210	0.00590867126923168\\
211	0.00590848899098555\\
212	0.00590830339630636\\
213	0.00590811442384809\\
214	0.0059079220111241\\
215	0.00590772609448603\\
216	0.005907526609102\\
217	0.00590732348893462\\
218	0.00590711666671837\\
219	0.00590690607393682\\
220	0.00590669164079936\\
221	0.0059064732962173\\
222	0.00590625096777978\\
223	0.00590602458172922\\
224	0.0059057940629362\\
225	0.00590555933487407\\
226	0.00590532031959288\\
227	0.00590507693769307\\
228	0.00590482910829859\\
229	0.0059045767490296\\
230	0.00590431977597459\\
231	0.00590405810366222\\
232	0.00590379164503247\\
233	0.00590352031140745\\
234	0.00590324401246165\\
235	0.00590296265619171\\
236	0.00590267614888578\\
237	0.00590238439509216\\
238	0.00590208729758769\\
239	0.00590178475734554\\
240	0.00590147667350251\\
241	0.00590116294332576\\
242	0.00590084346217916\\
243	0.00590051812348901\\
244	0.00590018681870945\\
245	0.00589984943728723\\
246	0.00589950586662604\\
247	0.00589915599205045\\
248	0.00589879969676935\\
249	0.00589843686183872\\
250	0.00589806736612446\\
251	0.00589769108626435\\
252	0.00589730789662985\\
253	0.00589691766928763\\
254	0.00589652027396045\\
255	0.00589611557798811\\
256	0.00589570344628814\\
257	0.00589528374131621\\
258	0.00589485632302661\\
259	0.00589442104883306\\
260	0.00589397777356926\\
261	0.00589352634945068\\
262	0.00589306662603661\\
263	0.00589259845019399\\
264	0.00589212166606282\\
265	0.0058916361150245\\
266	0.00589114163567389\\
267	0.00589063806379673\\
268	0.00589012523235447\\
269	0.00588960297147957\\
270	0.00588907110848436\\
271	0.00588852946788946\\
272	0.00588797787147783\\
273	0.00588741613838466\\
274	0.00588684408523326\\
275	0.00588626152633054\\
276	0.00588566827392798\\
277	0.00588506413851412\\
278	0.00588444892891336\\
279	0.00588382245100376\\
280	0.00588318449807008\\
281	0.0058825348590128\\
282	0.00588187331880002\\
283	0.00588119965838552\\
284	0.00588051365462389\\
285	0.00587981508018288\\
286	0.00587910370345234\\
287	0.00587837928844987\\
288	0.00587764159472272\\
289	0.00587689037724563\\
290	0.00587612538631439\\
291	0.00587534636743452\\
292	0.00587455306120481\\
293	0.00587374520319499\\
294	0.0058729225238168\\
295	0.00587208474818797\\
296	0.00587123159598753\\
297	0.00587036278130146\\
298	0.00586947801245687\\
299	0.00586857699184229\\
300	0.00586765941571126\\
301	0.00586672497396502\\
302	0.00586577334990901\\
303	0.0058648042199761\\
304	0.00586381725340672\\
305	0.00586281211187248\\
306	0.00586178844902646\\
307	0.00586074590995554\\
308	0.0058596841305036\\
309	0.00585860273642446\\
310	0.00585750134231237\\
311	0.00585637955025261\\
312	0.00585523694814955\\
313	0.00585407310778303\\
314	0.00585288758299598\\
315	0.00585167990964557\\
316	0.00585044961334749\\
317	0.00584919624883876\\
318	0.00584791937657553\\
319	0.00584661854835003\\
320	0.00584529330710308\\
321	0.00584394318673143\\
322	0.00584256771188981\\
323	0.00584116639778782\\
324	0.00583973874998108\\
325	0.00583828426415636\\
326	0.00583680242591091\\
327	0.00583529271052531\\
328	0.00583375458272986\\
329	0.00583218749646402\\
330	0.00583059089462881\\
331	0.0058289642088318\\
332	0.0058273068591242\\
333	0.00582561825372994\\
334	0.00582389778876628\\
335	0.00582214484795558\\
336	0.00582035880232746\\
337	0.00581853900991181\\
338	0.00581668481542106\\
339	0.00581479554992205\\
340	0.00581287053049673\\
341	0.0058109090598908\\
342	0.00580891042615028\\
343	0.00580687390224495\\
344	0.00580479874567809\\
345	0.00580268419808228\\
346	0.00580052948479973\\
347	0.00579833381444694\\
348	0.00579609637846199\\
349	0.0057938163506341\\
350	0.00579149288661333\\
351	0.00578912512339917\\
352	0.00578671217880552\\
353	0.00578425315089892\\
354	0.00578174711740561\\
355	0.00577919313508162\\
356	0.0057765902390379\\
357	0.00577393744201062\\
358	0.00577123373357035\\
359	0.00576847807929635\\
360	0.00576566942017733\\
361	0.00576280667229116\\
362	0.00575988872605111\\
363	0.00575691444542336\\
364	0.00575388266711421\\
365	0.00575079219972624\\
366	0.00574764182288094\\
367	0.00574443028630436\\
368	0.00574115630888115\\
369	0.00573781857767396\\
370	0.00573441574690804\\
371	0.00573094643692255\\
372	0.00572740923309333\\
373	0.00572380268473599\\
374	0.005720125303995\\
375	0.00571637556473634\\
376	0.00571255190146616\\
377	0.00570865270830226\\
378	0.00570467633802012\\
379	0.00570062110112619\\
380	0.00569648526460439\\
381	0.00569226704858611\\
382	0.00568796462368904\\
383	0.00568357610907727\\
384	0.00567909957042097\\
385	0.00567453301774997\\
386	0.00566987440319761\\
387	0.00566512161862799\\
388	0.00566027249314091\\
389	0.00565532479045079\\
390	0.00565027620614079\\
391	0.00564512436479847\\
392	0.00563986681701453\\
393	0.00563450103624374\\
394	0.00562902441551655\\
395	0.00562343426402935\\
396	0.00561772780362139\\
397	0.0056119021651399\\
398	0.00560595438471072\\
399	0.00559988139993504\\
400	0.00559368004603577\\
401	0.00558734705198143\\
402	0.00558087903661795\\
403	0.00557427250485708\\
404	0.00556752384394927\\
405	0.00556062931988138\\
406	0.00555358507408224\\
407	0.00554638712035304\\
408	0.00553903134196755\\
409	0.00553151348872566\\
410	0.00552382917336607\\
411	0.00551597386761887\\
412	0.00550794289708719\\
413	0.00549973143413929\\
414	0.00549133448834178\\
415	0.00548274688664091\\
416	0.00547396325814046\\
417	0.00546497802373885\\
418	0.00545578538484463\\
419	0.00544637931141403\\
420	0.00543675353033366\\
421	0.00542690151522444\\
422	0.00541681647523832\\
423	0.00540649134188126\\
424	0.00539591875216336\\
425	0.0053850910335577\\
426	0.00537400018914585\\
427	0.00536263788242526\\
428	0.00535099542177299\\
429	0.00533906374426414\\
430	0.00532683339727622\\
431	0.00531429451844637\\
432	0.00530143680895152\\
433	0.00528824950213971\\
434	0.00527472136143819\\
435	0.00526084068756722\\
436	0.00524659526987646\\
437	0.00523197232446155\\
438	0.0052169584486447\\
439	0.00520153957555673\\
440	0.00518570092578128\\
441	0.00516942695750495\\
442	0.00515270131816908\\
443	0.00513550680251643\\
444	0.00511782530865607\\
445	0.0050996375612257\\
446	0.005080923273698\\
447	0.00506166172231842\\
448	0.00504183203136729\\
449	0.00502141304891374\\
450	0.00500038337249701\\
451	0.00497872142081816\\
452	0.00495640549556105\\
453	0.00493341382583834\\
454	0.00490972461720514\\
455	0.00488531639942946\\
456	0.00486016471433154\\
457	0.0048342424059789\\
458	0.00480752277340168\\
459	0.00477997995509787\\
460	0.00475158503908517\\
461	0.0047223031056269\\
462	0.00469208532795192\\
463	0.00466087903083142\\
464	0.00462863329802819\\
465	0.00459530120078491\\
466	0.00456084286403987\\
467	0.00452522958610005\\
468	0.00448844925421451\\
469	0.00445051372491344\\
470	0.00441146866877086\\
471	0.00437140664415404\\
472	0.00433048439716472\\
473	0.0042889456962915\\
474	0.00424780188765614\\
475	0.00420755420416029\\
476	0.0041683333579541\\
477	0.00413027769285024\\
478	0.00409353132043151\\
479	0.00405824096408036\\
480	0.00402455137992824\\
481	0.00399259863899142\\
482	0.00396250056459678\\
483	0.00393434330753264\\
484	0.00390816275628304\\
485	0.00388391906252844\\
486	0.00386123616913274\\
487	0.0038390417191476\\
488	0.00381733894561227\\
489	0.00379612284227517\\
490	0.00377537874906163\\
491	0.00375508096906127\\
492	0.00373519154191179\\
493	0.00371565936210906\\
494	0.00369641992156969\\
495	0.00367739607739938\\
496	0.00365850042077436\\
497	0.00363964004349435\\
498	0.00362073356826921\\
499	0.00360176517069187\\
500	0.00358271728286732\\
501	0.00356357081159216\\
502	0.00354430546077433\\
503	0.00352490016954723\\
504	0.00350533367146163\\
505	0.00348558516873753\\
506	0.00346563509676915\\
507	0.00344546592364189\\
508	0.00342506288277991\\
509	0.00340441446709452\\
510	0.00338351130775128\\
511	0.00336234394263013\\
512	0.00334090288803782\\
513	0.00331917870814738\\
514	0.00329716207651345\\
515	0.00327484382265145\\
516	0.00325221495551864\\
517	0.00322926665511696\\
518	0.00320599022398332\\
519	0.00318237699293373\\
520	0.00315841818145577\\
521	0.00313410472472062\\
522	0.00310942721126966\\
523	0.00308437585158505\\
524	0.00305894044027125\\
525	0.00303311031139261\\
526	0.00300687428679977\\
527	0.00298022061768556\\
528	0.00295313692014422\\
529	0.00292561010614763\\
530	0.00289762631203977\\
531	0.00286917082725164\\
532	0.00284022802620434\\
533	0.00281078130393891\\
534	0.00278081301155708\\
535	0.00275030439324347\\
536	0.00271923552725015\\
537	0.00268758527399107\\
538	0.00265533123535152\\
539	0.00262244973055008\\
540	0.0025889157954173\\
541	0.00255470321368202\\
542	0.00251978459095741\\
543	0.0024841314847975\\
544	0.00244771460558413\\
545	0.00241050410851801\\
546	0.00237247000965317\\
547	0.0023335827737848\\
548	0.00229381404395241\\
549	0.00225313741932586\\
550	0.00221154059211673\\
551	0.00216901289169036\\
552	0.002125545741278\\
553	0.00208113833035888\\
554	0.00203579361568159\\
555	0.00198950880324453\\
556	0.00194228488943636\\
557	0.00189413021836552\\
558	0.00184507109150061\\
559	0.0017951892598509\\
560	0.00174443657204966\\
561	0.00169274885086732\\
562	0.00164004493249296\\
563	0.0015861209178538\\
564	0.0015308512570822\\
565	0.00147408828014922\\
566	0.00141791177863419\\
567	0.00136317271247085\\
568	0.00131050219331458\\
569	0.00125878059438045\\
570	0.00120724494964626\\
571	0.00115595954309104\\
572	0.00110479684501904\\
573	0.00105438977026886\\
574	0.00100530531250217\\
575	0.000957529082998364\\
576	0.000910184678218123\\
577	0.000863406211270714\\
578	0.000817321342927877\\
579	0.000771637462967062\\
580	0.000726087369012586\\
581	0.000680657779075544\\
582	0.000635387610447342\\
583	0.000590332284282012\\
584	0.000545550537540452\\
585	0.000501104457276143\\
586	0.000457057781190827\\
587	0.000413474755637532\\
588	0.000370418692160746\\
589	0.000327950377726591\\
590	0.000286126318718799\\
591	0.000244997582319803\\
592	0.000204610904245313\\
593	0.000165017211836692\\
594	0.000126301460681322\\
595	8.88161203105201e-05\\
596	5.34134895574395e-05\\
597	2.21100055488407e-05\\
598	0\\
599	0\\
600	0\\
};
\addplot [color=mycolor9,solid,forget plot]
  table[row sep=crcr]{%
1	0.00596698032876224\\
2	0.00596697428696311\\
3	0.0059669681495499\\
4	0.00596696191492659\\
5	0.00596695558146875\\
6	0.00596694914752286\\
7	0.00596694261140592\\
8	0.00596693597140488\\
9	0.00596692922577601\\
10	0.00596692237274445\\
11	0.00596691541050369\\
12	0.00596690833721486\\
13	0.00596690115100621\\
14	0.0059668938499726\\
15	0.00596688643217482\\
16	0.00596687889563902\\
17	0.00596687123835603\\
18	0.00596686345828079\\
19	0.00596685555333174\\
20	0.00596684752139008\\
21	0.0059668393602992\\
22	0.00596683106786396\\
23	0.00596682264185003\\
24	0.00596681407998321\\
25	0.0059668053799487\\
26	0.00596679653939043\\
27	0.00596678755591023\\
28	0.0059667784270672\\
29	0.00596676915037696\\
30	0.00596675972331082\\
31	0.00596675014329509\\
32	0.00596674040771014\\
33	0.00596673051388984\\
34	0.00596672045912049\\
35	0.00596671024064024\\
36	0.00596669985563803\\
37	0.00596668930125289\\
38	0.00596667857457304\\
39	0.00596666767263494\\
40	0.00596665659242247\\
41	0.005966645330866\\
42	0.00596663388484146\\
43	0.0059666222511694\\
44	0.00596661042661409\\
45	0.00596659840788246\\
46	0.00596658619162313\\
47	0.00596657377442547\\
48	0.00596656115281855\\
49	0.00596654832327008\\
50	0.00596653528218533\\
51	0.0059665220259062\\
52	0.00596650855070997\\
53	0.00596649485280829\\
54	0.005966480928346\\
55	0.00596646677340007\\
56	0.00596645238397834\\
57	0.00596643775601838\\
58	0.00596642288538634\\
59	0.00596640776787563\\
60	0.0059663923992058\\
61	0.00596637677502112\\
62	0.00596636089088945\\
63	0.00596634474230088\\
64	0.00596632832466638\\
65	0.00596631163331648\\
66	0.0059662946634999\\
67	0.00596627741038214\\
68	0.00596625986904409\\
69	0.00596624203448054\\
70	0.00596622390159879\\
71	0.00596620546521708\\
72	0.00596618672006311\\
73	0.00596616766077254\\
74	0.00596614828188739\\
75	0.00596612857785442\\
76	0.00596610854302357\\
77	0.00596608817164626\\
78	0.00596606745787369\\
79	0.0059660463957553\\
80	0.00596602497923682\\
81	0.00596600320215869\\
82	0.00596598105825415\\
83	0.00596595854114744\\
84	0.00596593564435203\\
85	0.00596591236126865\\
86	0.00596588868518346\\
87	0.00596586460926599\\
88	0.00596584012656725\\
89	0.00596581523001769\\
90	0.00596578991242516\\
91	0.00596576416647284\\
92	0.00596573798471713\\
93	0.00596571135958542\\
94	0.005965684283374\\
95	0.00596565674824579\\
96	0.00596562874622822\\
97	0.00596560026921068\\
98	0.00596557130894236\\
99	0.00596554185702984\\
100	0.00596551190493465\\
101	0.00596548144397089\\
102	0.00596545046530268\\
103	0.00596541895994166\\
104	0.0059653869187444\\
105	0.00596535433240978\\
106	0.00596532119147637\\
107	0.00596528748631974\\
108	0.00596525320714972\\
109	0.00596521834400758\\
110	0.00596518288676329\\
111	0.00596514682511265\\
112	0.00596511014857434\\
113	0.00596507284648711\\
114	0.00596503490800669\\
115	0.0059649963221029\\
116	0.0059649570775566\\
117	0.00596491716295662\\
118	0.0059648765666967\\
119	0.00596483527697249\\
120	0.00596479328177839\\
121	0.0059647505689045\\
122	0.00596470712593359\\
123	0.00596466294023813\\
124	0.00596461799897737\\
125	0.0059645722890943\\
126	0.0059645257973131\\
127	0.00596447851013654\\
128	0.00596443041384336\\
129	0.00596438149448628\\
130	0.00596433173789011\\
131	0.00596428112965005\\
132	0.00596422965513067\\
133	0.00596417729946522\\
134	0.00596412404755542\\
135	0.00596406988407196\\
136	0.00596401479345549\\
137	0.00596395875991829\\
138	0.00596390176744615\\
139	0.00596384379980049\\
140	0.00596378484051931\\
141	0.00596372487291455\\
142	0.00596366388005987\\
143	0.00596360184475242\\
144	0.00596353874940599\\
145	0.00596347457576359\\
146	0.00596340930414213\\
147	0.00596334291153906\\
148	0.00596327536756221\\
149	0.0059632066316546\\
150	0.00596313667910292\\
151	0.00596306548798833\\
152	0.00596299303599557\\
153	0.00596291930040575\\
154	0.00596284425808905\\
155	0.00596276788549745\\
156	0.005962690158657\\
157	0.00596261105316029\\
158	0.0059625305441587\\
159	0.0059624486063543\\
160	0.00596236521399185\\
161	0.00596228034085064\\
162	0.005962193960236\\
163	0.00596210604497084\\
164	0.00596201656738697\\
165	0.00596192549931628\\
166	0.00596183281208175\\
167	0.00596173847648834\\
168	0.00596164246281363\\
169	0.00596154474079839\\
170	0.00596144527963697\\
171	0.00596134404796742\\
172	0.00596124101386157\\
173	0.00596113614481485\\
174	0.00596102940773601\\
175	0.00596092076893647\\
176	0.00596081019411975\\
177	0.00596069764837057\\
178	0.00596058309614354\\
179	0.00596046650125228\\
180	0.00596034782685749\\
181	0.00596022703545563\\
182	0.00596010408886674\\
183	0.00595997894822261\\
184	0.00595985157395424\\
185	0.00595972192577939\\
186	0.00595958996268988\\
187	0.00595945564293847\\
188	0.00595931892402574\\
189	0.00595917976268663\\
190	0.00595903811487684\\
191	0.00595889393575869\\
192	0.00595874717968713\\
193	0.00595859780019526\\
194	0.00595844574997971\\
195	0.00595829098088561\\
196	0.00595813344389141\\
197	0.00595797308909345\\
198	0.00595780986569022\\
199	0.00595764372196629\\
200	0.00595747460527606\\
201	0.00595730246202706\\
202	0.00595712723766321\\
203	0.00595694887664747\\
204	0.00595676732244446\\
205	0.00595658251750257\\
206	0.00595639440323586\\
207	0.00595620292000564\\
208	0.00595600800710164\\
209	0.00595580960272282\\
210	0.00595560764395797\\
211	0.00595540206676586\\
212	0.00595519280595509\\
213	0.00595497979516345\\
214	0.00595476296683703\\
215	0.00595454225220883\\
216	0.00595431758127702\\
217	0.00595408888278293\\
218	0.00595385608418831\\
219	0.00595361911165241\\
220	0.00595337789000854\\
221	0.00595313234274014\\
222	0.00595288239195639\\
223	0.00595262795836741\\
224	0.00595236896125876\\
225	0.00595210531846558\\
226	0.00595183694634617\\
227	0.005951563759755\\
228	0.00595128567201496\\
229	0.00595100259488926\\
230	0.00595071443855242\\
231	0.0059504211115608\\
232	0.00595012252082237\\
233	0.00594981857156558\\
234	0.00594950916730758\\
235	0.00594919420982151\\
236	0.00594887359910305\\
237	0.00594854723333576\\
238	0.00594821500885565\\
239	0.00594787682011443\\
240	0.00594753255964174\\
241	0.00594718211800609\\
242	0.00594682538377413\\
243	0.00594646224346886\\
244	0.00594609258152561\\
245	0.00594571628024663\\
246	0.00594533321975339\\
247	0.00594494327793646\\
248	0.00594454633040312\\
249	0.00594414225042176\\
250	0.00594373090886309\\
251	0.00594331217413756\\
252	0.00594288591212829\\
253	0.00594245198611909\\
254	0.00594201025671653\\
255	0.00594156058176518\\
256	0.00594110281625461\\
257	0.00594063681221694\\
258	0.00594016241861275\\
259	0.00593967948120345\\
260	0.00593918784240694\\
261	0.00593868734113341\\
262	0.00593817781259664\\
263	0.00593765908809572\\
264	0.00593713099476015\\
265	0.0059365933552501\\
266	0.00593604598740111\\
267	0.00593548870379985\\
268	0.00593492131127443\\
269	0.00593434361027776\\
270	0.00593375539413795\\
271	0.00593315644814377\\
272	0.00593254654842749\\
273	0.00593192546061117\\
274	0.00593129293821519\\
275	0.00593064872098129\\
276	0.00592999253387784\\
277	0.00592932409005896\\
278	0.00592864311207371\\
279	0.00592794944143\\
280	0.00592724366041662\\
281	0.00592652562069577\\
282	0.00592579512551166\\
283	0.00592505197612543\\
284	0.00592429597190641\\
285	0.00592352691043863\\
286	0.00592274458764527\\
287	0.00592194879793301\\
288	0.00592113933436002\\
289	0.00592031598883094\\
290	0.00591947855232387\\
291	0.00591862681515412\\
292	0.00591776056728166\\
293	0.00591687959867023\\
294	0.00591598369970732\\
295	0.00591507266169724\\
296	0.0059141462774414\\
297	0.00591320434192362\\
298	0.00591224665312267\\
299	0.00591127301297869\\
300	0.00591028322854776\\
301	0.00590927711338638\\
302	0.00590825448921893\\
303	0.00590721518795306\\
304	0.00590615905412585\\
305	0.00590508594788322\\
306	0.00590399574861847\\
307	0.00590288835942188\\
308	0.00590176371251024\\
309	0.00590062177578279\\
310	0.00589946256049717\\
311	0.00589828612948728\\
312	0.00589709260348951\\
313	0.00589588215729965\\
314	0.00589465497888111\\
315	0.00589341110283245\\
316	0.00589214980670914\\
317	0.00589086730518115\\
318	0.00588956197479014\\
319	0.00588823342016629\\
320	0.00588688123950195\\
321	0.00588550502445978\\
322	0.00588410436008018\\
323	0.00588267882468772\\
324	0.00588122798979675\\
325	0.00587975142001631\\
326	0.00587824867295422\\
327	0.00587671929912034\\
328	0.00587516284182927\\
329	0.00587357883710227\\
330	0.00587196681356858\\
331	0.00587032629236615\\
332	0.00586865678704194\\
333	0.00586695780345159\\
334	0.00586522883965904\\
335	0.00586346938583569\\
336	0.00586167892415979\\
337	0.00585985692871575\\
338	0.0058580028653944\\
339	0.00585611619179378\\
340	0.00585419635712175\\
341	0.00585224280210114\\
342	0.00585025495887806\\
343	0.00584823225093598\\
344	0.0058461740930169\\
345	0.00584407989105306\\
346	0.00584194904211389\\
347	0.00583978093437328\\
348	0.00583757494710622\\
349	0.00583533045072542\\
350	0.00583304680687375\\
351	0.00583072336859305\\
352	0.00582835948059781\\
353	0.00582595447969194\\
354	0.0058235076953776\\
355	0.0058210184507158\\
356	0.00581848606349089\\
357	0.00581590984764366\\
358	0.00581328911447787\\
359	0.00581062317086183\\
360	0.00580791129842352\\
361	0.00580515273253551\\
362	0.00580234669481404\\
363	0.00579949239258684\\
364	0.00579658901829688\\
365	0.00579363574883313\\
366	0.00579063174477445\\
367	0.00578757614948704\\
368	0.00578446808787763\\
369	0.0057813066650461\\
370	0.00577809096464617\\
371	0.00577482004685582\\
372	0.00577149294583371\\
373	0.00576810866649863\\
374	0.00576466618042918\\
375	0.00576116442065325\\
376	0.00575760227514969\\
377	0.00575397857922289\\
378	0.00575029210821432\\
379	0.00574654157658063\\
380	0.00574272566568746\\
381	0.00573884316894899\\
382	0.00573489292066993\\
383	0.00573087372860639\\
384	0.00572678437235225\\
385	0.00572262360151242\\
386	0.00571839013363018\\
387	0.00571408265182969\\
388	0.00570969980212954\\
389	0.00570524019037783\\
390	0.00570070237875122\\
391	0.00569608488175161\\
392	0.00569138616162681\\
393	0.00568660462313372\\
394	0.00568173860755651\\
395	0.00567678638588469\\
396	0.00567174615104976\\
397	0.00566661600911928\\
398	0.00566139396935263\\
399	0.00565607793303559\\
400	0.00565066568103991\\
401	0.00564515486010003\\
402	0.00563954296787925\\
403	0.00563382733701319\\
404	0.00562800511850132\\
405	0.00562207326509037\\
406	0.00561602851566418\\
407	0.00560986738226857\\
408	0.00560358614274056\\
409	0.00559718084344612\\
410	0.00559064731396248\\
411	0.00558398120219288\\
412	0.00557717803971416\\
413	0.00557023335131169\\
414	0.00556314283044125\\
415	0.00555590262676456\\
416	0.00554850901419348\\
417	0.00554095814152592\\
418	0.0055332460268363\\
419	0.00552536855176885\\
420	0.00551732145581221\\
421	0.00550910033037316\\
422	0.00550070061258381\\
423	0.00549211757871309\\
424	0.00548334633706524\\
425	0.00547438181743129\\
426	0.0054652187589276\\
427	0.00545585170022723\\
428	0.0054462749691268\\
429	0.00543648267129829\\
430	0.00542646867823506\\
431	0.00541622661415981\\
432	0.00540574984263258\\
433	0.00539503145523869\\
434	0.00538406425960559\\
435	0.00537284076209295\\
436	0.00536135314921188\\
437	0.00534959327003797\\
438	0.00533755261815991\\
439	0.00532522231287002\\
440	0.00531259307962802\\
441	0.00529965522977213\\
442	0.00528639863899811\\
443	0.00527281272170294\\
444	0.00525888639097232\\
445	0.00524460805509669\\
446	0.00522996563264719\\
447	0.00521494652217509\\
448	0.00519953754023312\\
449	0.00518372487686434\\
450	0.00516749405216731\\
451	0.00515082986910345\\
452	0.00513371636310139\\
453	0.00511613674955181\\
454	0.00509807335976727\\
455	0.00507950736007831\\
456	0.00506041883433673\\
457	0.00504078711048057\\
458	0.00502059082109013\\
459	0.00499980766035087\\
460	0.00497841434972291\\
461	0.00495638673588621\\
462	0.00493370007258682\\
463	0.00491032918825705\\
464	0.00488624881909362\\
465	0.00486143208115949\\
466	0.00483584970052833\\
467	0.00480947088826041\\
468	0.00478226938885918\\
469	0.00475421917490449\\
470	0.00472529398634746\\
471	0.00469546662444491\\
472	0.00466470802727958\\
473	0.00463298578218709\\
474	0.0046002615361796\\
475	0.00456648303841523\\
476	0.00453159402478002\\
477	0.00449554146674696\\
478	0.00445827840448508\\
479	0.00441976782313848\\
480	0.00437998790630702\\
481	0.00433893908938266\\
482	0.00429665347640327\\
483	0.00425320735615181\\
484	0.00420873777828003\\
485	0.00416346444776063\\
486	0.00411793338147825\\
487	0.00407334441145914\\
488	0.00402983908360743\\
489	0.00398756754637643\\
490	0.00394668653099172\\
491	0.00390735617353304\\
492	0.00386973517480984\\
493	0.00383397370225649\\
494	0.00380020320070834\\
495	0.00376852208426059\\
496	0.00373897579237152\\
497	0.00371152955435554\\
498	0.00368586419896074\\
499	0.00366070892853893\\
500	0.00363606621761742\\
501	0.0036119291710621\\
502	0.00358827992768946\\
503	0.00356508812130673\\
504	0.00354230954779571\\
505	0.00351988526989542\\
506	0.00349774149255958\\
507	0.0034757907019673\\
508	0.00345393476507764\\
509	0.00343207096241981\\
510	0.00341012764353683\\
511	0.00338808604648802\\
512	0.00336592541463887\\
513	0.003343623263393\\
514	0.00332115577071465\\
515	0.00329849830394357\\
516	0.00327562608630737\\
517	0.00325251499140129\\
518	0.00322914242719214\\
519	0.00320548822984379\\
520	0.00318153542522883\\
521	0.00315727062184529\\
522	0.00313268099904592\\
523	0.00310775343241039\\
524	0.00308247455671109\\
525	0.00305683082287246\\
526	0.00303080854165051\\
527	0.00300439390522109\\
528	0.00297757297671185\\
529	0.00295033163747356\\
530	0.00292265548344591\\
531	0.00289452966666671\\
532	0.00286593868799494\\
533	0.00283686621062073\\
534	0.00280729500462495\\
535	0.00277720688756282\\
536	0.00274658266198832\\
537	0.00271540205159736\\
538	0.00268364363870369\\
539	0.00265128480712363\\
540	0.00261830169627242\\
541	0.00258466917438718\\
542	0.00255036084124864\\
543	0.00251534907343426\\
544	0.00247960512775518\\
545	0.0024430993137522\\
546	0.00240580125358572\\
547	0.00236768025517169\\
548	0.00232870583345605\\
549	0.00228884842711842\\
550	0.00224808016815168\\
551	0.0022063863306714\\
552	0.00216375294517026\\
553	0.00212016698020766\\
554	0.00207561654051804\\
555	0.00203009119300053\\
556	0.00198358244425289\\
557	0.0019360841537035\\
558	0.00188759332586823\\
559	0.00183812199659996\\
560	0.00178767181241054\\
561	0.00173625130079578\\
562	0.00168387822104933\\
563	0.00163065515449775\\
564	0.00157654741904633\\
565	0.0015214872420157\\
566	0.00146539784434777\\
567	0.00140810574272265\\
568	0.00134939730533792\\
569	0.00129085071884086\\
570	0.00123365690412139\\
571	0.0011782816837452\\
572	0.0011249450961202\\
573	0.00107192215633457\\
574	0.00101936903435207\\
575	0.000967409323524144\\
576	0.000916845628028934\\
577	0.000867784394304618\\
578	0.000819851503148485\\
579	0.000772715621978611\\
580	0.000726464313101914\\
581	0.000680819564535681\\
582	0.000635473298127815\\
583	0.000590376990638551\\
584	0.000545575509395754\\
585	0.000501118550186597\\
586	0.000457065651499505\\
587	0.000413478843923962\\
588	0.000370420548387954\\
589	0.000327950981361874\\
590	0.000286126434448213\\
591	0.000244997582319806\\
592	0.000204610904245315\\
593	0.000165017211836694\\
594	0.000126301460681323\\
595	8.88161203105206e-05\\
596	5.34134895574398e-05\\
597	2.21100055488407e-05\\
598	0\\
599	0\\
600	0\\
};
\addplot [color=blue!50!mycolor7,solid,forget plot]
  table[row sep=crcr]{%
1	0.00616796378288477\\
2	0.00616795498629279\\
3	0.00616794604287643\\
4	0.00616793695012169\\
5	0.00616792770547062\\
6	0.00616791830632054\\
7	0.00616790875002333\\
8	0.00616789903388457\\
9	0.00616788915516275\\
10	0.00616787911106843\\
11	0.00616786889876348\\
12	0.00616785851536016\\
13	0.00616784795792029\\
14	0.00616783722345434\\
15	0.0061678263089205\\
16	0.00616781521122393\\
17	0.00616780392721567\\
18	0.00616779245369177\\
19	0.00616778078739232\\
20	0.00616776892500048\\
21	0.00616775686314153\\
22	0.00616774459838177\\
23	0.00616773212722757\\
24	0.00616771944612435\\
25	0.0061677065514554\\
26	0.00616769343954091\\
27	0.00616768010663687\\
28	0.00616766654893387\\
29	0.00616765276255607\\
30	0.00616763874355997\\
31	0.00616762448793324\\
32	0.00616760999159356\\
33	0.00616759525038742\\
34	0.00616758026008878\\
35	0.00616756501639784\\
36	0.00616754951493993\\
37	0.00616753375126397\\
38	0.00616751772084123\\
39	0.00616750141906402\\
40	0.00616748484124429\\
41	0.0061674679826122\\
42	0.00616745083831474\\
43	0.00616743340341421\\
44	0.00616741567288683\\
45	0.00616739764162118\\
46	0.00616737930441671\\
47	0.00616736065598214\\
48	0.00616734169093393\\
49	0.0061673224037946\\
50	0.00616730278899121\\
51	0.00616728284085353\\
52	0.00616726255361243\\
53	0.00616724192139813\\
54	0.0061672209382385\\
55	0.00616719959805714\\
56	0.00616717789467163\\
57	0.00616715582179169\\
58	0.00616713337301722\\
59	0.0061671105418365\\
60	0.00616708732162401\\
61	0.00616706370563864\\
62	0.00616703968702152\\
63	0.00616701525879401\\
64	0.0061669904138556\\
65	0.00616696514498168\\
66	0.00616693944482146\\
67	0.00616691330589566\\
68	0.00616688672059423\\
69	0.00616685968117411\\
70	0.00616683217975678\\
71	0.00616680420832595\\
72	0.00616677575872509\\
73	0.00616674682265486\\
74	0.00616671739167068\\
75	0.00616668745718011\\
76	0.00616665701044015\\
77	0.0061666260425547\\
78	0.00616659454447171\\
79	0.0061665625069804\\
80	0.00616652992070855\\
81	0.00616649677611941\\
82	0.00616646306350893\\
83	0.00616642877300271\\
84	0.00616639389455295\\
85	0.00616635841793529\\
86	0.00616632233274569\\
87	0.0061662856283972\\
88	0.0061662482941167\\
89	0.0061662103189415\\
90	0.00616617169171596\\
91	0.00616613240108794\\
92	0.0061660924355054\\
93	0.00616605178321268\\
94	0.00616601043224692\\
95	0.00616596837043422\\
96	0.00616592558538583\\
97	0.00616588206449441\\
98	0.00616583779492999\\
99	0.00616579276363588\\
100	0.00616574695732473\\
101	0.00616570036247421\\
102	0.00616565296532289\\
103	0.0061656047518658\\
104	0.00616555570785015\\
105	0.0061655058187708\\
106	0.0061654550698657\\
107	0.00616540344611128\\
108	0.00616535093221767\\
109	0.006165297512624\\
110	0.00616524317149349\\
111	0.00616518789270843\\
112	0.00616513165986524\\
113	0.00616507445626923\\
114	0.00616501626492945\\
115	0.00616495706855343\\
116	0.00616489684954168\\
117	0.00616483558998232\\
118	0.00616477327164549\\
119	0.00616470987597776\\
120	0.00616464538409635\\
121	0.00616457977678329\\
122	0.00616451303447963\\
123	0.00616444513727944\\
124	0.00616437606492364\\
125	0.0061643057967941\\
126	0.00616423431190723\\
127	0.0061641615889078\\
128	0.00616408760606264\\
129	0.0061640123412541\\
130	0.00616393577197366\\
131	0.00616385787531536\\
132	0.00616377862796928\\
133	0.00616369800621478\\
134	0.00616361598591388\\
135	0.00616353254250459\\
136	0.00616344765099389\\
137	0.00616336128595101\\
138	0.00616327342150043\\
139	0.0061631840313144\\
140	0.00616309308860505\\
141	0.00616300056611451\\
142	0.00616290643610077\\
143	0.00616281067031201\\
144	0.00616271323993481\\
145	0.00616261411548337\\
146	0.00616251326658348\\
147	0.00616241066167364\\
148	0.0061623062681566\\
149	0.00616220005437358\\
150	0.00616209198842411\\
151	0.00616198203784909\\
152	0.00616187016962098\\
153	0.00616175635013395\\
154	0.00616164054519381\\
155	0.00616152272000782\\
156	0.00616140283917421\\
157	0.00616128086667171\\
158	0.00616115676584856\\
159	0.00616103049941183\\
160	0.00616090202941606\\
161	0.00616077131725206\\
162	0.0061606383236354\\
163	0.0061605030085946\\
164	0.00616036533145928\\
165	0.00616022525084805\\
166	0.00616008272465619\\
167	0.00615993771004306\\
168	0.00615979016341949\\
169	0.00615964004043468\\
170	0.00615948729596306\\
171	0.00615933188409106\\
172	0.00615917375810327\\
173	0.0061590128704687\\
174	0.00615884917282671\\
175	0.00615868261597266\\
176	0.00615851314984341\\
177	0.00615834072350244\\
178	0.00615816528512489\\
179	0.00615798678198224\\
180	0.00615780516042676\\
181	0.00615762036587572\\
182	0.00615743234279532\\
183	0.0061572410346844\\
184	0.00615704638405779\\
185	0.00615684833242945\\
186	0.00615664682029542\\
187	0.00615644178711623\\
188	0.00615623317129941\\
189	0.00615602091018129\\
190	0.0061558049400088\\
191	0.00615558519592096\\
192	0.00615536161192984\\
193	0.00615513412090156\\
194	0.00615490265453664\\
195	0.00615466714335028\\
196	0.00615442751665224\\
197	0.00615418370252641\\
198	0.00615393562780995\\
199	0.00615368321807235\\
200	0.00615342639759385\\
201	0.00615316508934384\\
202	0.00615289921495866\\
203	0.00615262869471919\\
204	0.00615235344752803\\
205	0.00615207339088641\\
206	0.00615178844087065\\
207	0.00615149851210829\\
208	0.00615120351775392\\
209	0.00615090336946461\\
210	0.00615059797737486\\
211	0.00615028725007127\\
212	0.00614997109456685\\
213	0.00614964941627499\\
214	0.00614932211898281\\
215	0.00614898910482444\\
216	0.00614865027425364\\
217	0.00614830552601614\\
218	0.00614795475712158\\
219	0.0061475978628151\\
220	0.00614723473654822\\
221	0.00614686526994981\\
222	0.00614648935279618\\
223	0.00614610687298098\\
224	0.00614571771648471\\
225	0.00614532176734357\\
226	0.00614491890761819\\
227	0.00614450901736168\\
228	0.00614409197458738\\
229	0.0061436676552361\\
230	0.00614323593314302\\
231	0.00614279668000399\\
232	0.00614234976534144\\
233	0.00614189505646999\\
234	0.00614143241846134\\
235	0.00614096171410899\\
236	0.00614048280389222\\
237	0.00613999554593982\\
238	0.00613949979599332\\
239	0.00613899540736961\\
240	0.00613848223092319\\
241	0.006137960115008\\
242	0.00613742890543874\\
243	0.00613688844545148\\
244	0.00613633857566415\\
245	0.00613577913403619\\
246	0.00613520995582788\\
247	0.00613463087355897\\
248	0.00613404171696693\\
249	0.0061334423129645\\
250	0.00613283248559666\\
251	0.00613221205599688\\
252	0.00613158084234291\\
253	0.00613093865981151\\
254	0.00613028532053265\\
255	0.00612962063354266\\
256	0.00612894440473639\\
257	0.00612825643681849\\
258	0.00612755652925316\\
259	0.00612684447821274\\
260	0.00612612007652472\\
261	0.00612538311361694\\
262	0.00612463337546085\\
263	0.00612387064451269\\
264	0.00612309469965206\\
265	0.00612230531611807\\
266	0.00612150226544255\\
267	0.00612068531538084\\
268	0.00611985422983949\\
269	0.00611900876880259\\
270	0.00611814868825768\\
271	0.00611727374012557\\
272	0.0061163836722045\\
273	0.00611547822815719\\
274	0.00611455714762405\\
275	0.00611362016671604\\
276	0.00611266701966769\\
277	0.00611169744400973\\
278	0.00611071119554222\\
279	0.00610970807905205\\
280	0.00610868783412906\\
281	0.00610765017968261\\
282	0.00610659482966852\\
283	0.00610552149380714\\
284	0.00610442987754155\\
285	0.00610331968199689\\
286	0.00610219060394079\\
287	0.00610104233574528\\
288	0.00609987456535038\\
289	0.00609868697623001\\
290	0.00609747924736\\
291	0.00609625105318948\\
292	0.00609500206361547\\
293	0.00609373194396177\\
294	0.00609244035496291\\
295	0.00609112695275381\\
296	0.00608979138886644\\
297	0.00608843331023451\\
298	0.00608705235920753\\
299	0.00608564817357558\\
300	0.00608422038660648\\
301	0.00608276862709649\\
302	0.00608129251943599\\
303	0.00607979168369105\\
304	0.00607826573570055\\
305	0.00607671428718683\\
306	0.0060751369458734\\
307	0.00607353331559405\\
308	0.00607190299635369\\
309	0.00607024558424393\\
310	0.00606856067096013\\
311	0.00606684784225487\\
312	0.00606510667356318\\
313	0.00606333671819089\\
314	0.00606153747673036\\
315	0.00605970832521404\\
316	0.00605784840435603\\
317	0.00605595704761991\\
318	0.00605403373875352\\
319	0.00605207797540534\\
320	0.00605008924888612\\
321	0.00604806704414971\\
322	0.00604601083977554\\
323	0.00604392010795301\\
324	0.00604179431446763\\
325	0.0060396329186885\\
326	0.00603743537355697\\
327	0.00603520112557593\\
328	0.00603292961479949\\
329	0.00603062027482197\\
330	0.00602827253276559\\
331	0.00602588580926559\\
332	0.00602345951845144\\
333	0.00602099306792202\\
334	0.0060184858587127\\
335	0.00601593728525108\\
336	0.00601334673529779\\
337	0.00601071358986723\\
338	0.00600803722312264\\
339	0.00600531700223735\\
340	0.0060025522872127\\
341	0.00599974243064031\\
342	0.00599688677739309\\
343	0.00599398466422511\\
344	0.00599103541925553\\
345	0.00598803836130458\\
346	0.00598499279904132\\
347	0.00598189802989213\\
348	0.0059787533386445\\
349	0.00597555799566378\\
350	0.00597231125461864\\
351	0.00596901234958503\\
352	0.00596566049137276\\
353	0.00596225486290824\\
354	0.00595879461357244\\
355	0.00595527885276398\\
356	0.00595170664450247\\
357	0.0059480770111792\\
358	0.00594438898170422\\
359	0.0059406418518164\\
360	0.00593683662340024\\
361	0.00593297577655792\\
362	0.00592905894473821\\
363	0.00592508580091501\\
364	0.00592105606328772\\
365	0.00591696950153508\\
366	0.00591282594413886\\
367	0.00590862528963745\\
368	0.0059043675281043\\
369	0.00590005274550033\\
370	0.00589568113934188\\
371	0.00589125303809417\\
372	0.00588676892447935\\
373	0.0058822294632954\\
374	0.00587763553379345\\
375	0.00587298826458306\\
376	0.00586828906221889\\
377	0.0058635396030339\\
378	0.0058587416888934\\
379	0.00585389663900076\\
380	0.00584900306679337\\
381	0.00584404835726201\\
382	0.00583902671542171\\
383	0.00583393813557483\\
384	0.00582878269393236\\
385	0.0058235605564739\\
386	0.00581827198738759\\
387	0.00581291735809316\\
388	0.00580749715683386\\
389	0.00580201199879914\\
390	0.00579646263670784\\
391	0.00579084997173673\\
392	0.00578517506461892\\
393	0.0057794391466551\\
394	0.00577364363027226\\
395	0.00576779011861874\\
396	0.0057618804134934\\
397	0.00575591652065193\\
398	0.00574990065119917\\
399	0.00574383521733511\\
400	0.00573772282014266\\
401	0.00573156622634407\\
402	0.00572536832994877\\
403	0.00571913209338799\\
404	0.00571286046093951\\
405	0.00570655623470178\\
406	0.00570022189923676\\
407	0.00569385937178882\\
408	0.00568746962131479\\
409	0.00568105210058922\\
410	0.0056746042595674\\
411	0.00566812073197445\\
412	0.00566159225447525\\
413	0.00565500423479394\\
414	0.00564833486260923\\
415	0.00564155264144871\\
416	0.00563464346015069\\
417	0.00562760468649372\\
418	0.00562043359053898\\
419	0.00561312733264331\\
420	0.00560568294873898\\
421	0.00559809733294775\\
422	0.00559036721966646\\
423	0.00558248917463631\\
424	0.00557445963074975\\
425	0.00556627510944276\\
426	0.0055579322708286\\
427	0.00554942767253612\\
428	0.00554075776446712\\
429	0.00553191888318161\\
430	0.00552290724588343\\
431	0.00551371894406478\\
432	0.00550434993691797\\
433	0.00549479604417364\\
434	0.00548505293799717\\
435	0.00547511613427119\\
436	0.00546498098337248\\
437	0.00545464266024167\\
438	0.00544409615363794\\
439	0.00543333625448267\\
440	0.00542235754316521\\
441	0.00541115437560315\\
442	0.00539972086767413\\
443	0.00538805087773602\\
444	0.00537613799088753\\
445	0.00536397550227631\\
446	0.00535155639435993\\
447	0.00533887331012354\\
448	0.00532591852573199\\
449	0.00531268392138899\\
450	0.00529916095340206\\
451	0.00528534062188356\\
452	0.00527121343455982\\
453	0.00525676936553943\\
454	0.00524199780072244\\
455	0.00522688751487832\\
456	0.00521142666218862\\
457	0.00519560273943266\\
458	0.00517940252820087\\
459	0.00516281197690765\\
460	0.00514581611704362\\
461	0.00512839899095527\\
462	0.00511054357688084\\
463	0.00509223171262275\\
464	0.00507344390265162\\
465	0.00505415925670797\\
466	0.00503435567621611\\
467	0.00501401036798015\\
468	0.00499309963100182\\
469	0.0049715990315815\\
470	0.00494948384062078\\
471	0.00492672976637657\\
472	0.00490331133257622\\
473	0.00487920215532924\\
474	0.00485437358386235\\
475	0.00482879420512406\\
476	0.00480242978742587\\
477	0.00477525017568199\\
478	0.00474722552552484\\
479	0.00471832626772328\\
480	0.00468852275574725\\
481	0.00465778522425869\\
482	0.00462608361112403\\
483	0.00459338717082247\\
484	0.00455966382745061\\
485	0.00452487901905876\\
486	0.00448899374288225\\
487	0.00445195974803545\\
488	0.00441371527226019\\
489	0.00437420191016548\\
490	0.0043333676522765\\
491	0.0042911710405434\\
492	0.00424758678823924\\
493	0.00420261332515753\\
494	0.00415628287011242\\
495	0.00410867481589535\\
496	0.00405993345448968\\
497	0.00401029138671468\\
498	0.00396025901850412\\
499	0.0039112447521658\\
500	0.00386340321922209\\
501	0.00381689842882749\\
502	0.00377190122476033\\
503	0.00372858540682304\\
504	0.00368712197668675\\
505	0.00364767067695465\\
506	0.00361036806645511\\
507	0.00357531064274824\\
508	0.00354253139428384\\
509	0.00351196768246299\\
510	0.00348294412958198\\
511	0.0034544445612314\\
512	0.00342646859927395\\
513	0.00339900494372885\\
514	0.00337202956824401\\
515	0.00334550400835951\\
516	0.00331937394815773\\
517	0.00329356839018511\\
518	0.00326799986285606\\
519	0.00324256629797655\\
520	0.0032171554602406\\
521	0.00319165318127545\\
522	0.00316601664016738\\
523	0.00314022231224942\\
524	0.00311424430564812\\
525	0.003088054691934\\
526	0.00306162398333896\\
527	0.00303492176756812\\
528	0.00300791749839272\\
529	0.00298058141801307\\
530	0.00295288555025977\\
531	0.00292480464640869\\
532	0.00289631687292284\\
533	0.00286740285777628\\
534	0.00283804236074501\\
535	0.00280821432287764\\
536	0.00277789691726086\\
537	0.00274706759525826\\
538	0.00271570312083671\\
539	0.0026837795843368\\
540	0.00265127238661691\\
541	0.00261815618565458\\
542	0.00258440480192056\\
543	0.00254999108823549\\
544	0.00251488678752385\\
545	0.00247906258232332\\
546	0.00244248826276903\\
547	0.0024051329724945\\
548	0.0023669655562731\\
549	0.00232795503983998\\
550	0.00228807128655577\\
551	0.00224728557924605\\
552	0.00220558181446146\\
553	0.00216294491653439\\
554	0.00211936041432799\\
555	0.00207481457162041\\
556	0.00202929454834635\\
557	0.00198278858773779\\
558	0.00193528623440087\\
559	0.00188677858576229\\
560	0.00183725850665937\\
561	0.00178672112391201\\
562	0.001735164201167\\
563	0.00168258854553136\\
564	0.001629003756701\\
565	0.00157442816337441\\
566	0.00151887965724378\\
567	0.00146242148369835\\
568	0.00140510286482678\\
569	0.00134687888471046\\
570	0.00128766019442617\\
571	0.00122731519619431\\
572	0.0011661227516964\\
573	0.0011061623275438\\
574	0.00104779478575355\\
575	0.000991600938932895\\
576	0.000937139387801049\\
577	0.000883303309683172\\
578	0.000830704566292631\\
579	0.00077988188673798\\
580	0.000730735365002536\\
581	0.000682984289216812\\
582	0.000636415229577655\\
583	0.000590882705834239\\
584	0.000545838398608977\\
585	0.000501265719758254\\
586	0.000457149164974275\\
587	0.000413526070867918\\
588	0.000370445508788355\\
589	0.000327962642665688\\
590	0.000286130327584013\\
591	0.00024499835594388\\
592	0.000204610904245314\\
593	0.000165017211836693\\
594	0.000126301460681323\\
595	8.88161203105204e-05\\
596	5.34134895574397e-05\\
597	2.21100055488407e-05\\
598	0\\
599	0\\
600	0\\
};
\addplot [color=blue!40!mycolor9,solid,forget plot]
  table[row sep=crcr]{%
1	0.00694684913621286\\
2	0.00694683074131371\\
3	0.00694681203309307\\
4	0.0069467930061662\\
5	0.00694677365505543\\
6	0.00694675397418873\\
7	0.00694673395789778\\
8	0.00694671360041654\\
9	0.0069466928958795\\
10	0.00694667183832005\\
11	0.00694665042166856\\
12	0.00694662863975075\\
13	0.00694660648628583\\
14	0.00694658395488472\\
15	0.00694656103904817\\
16	0.00694653773216479\\
17	0.00694651402750923\\
18	0.00694648991824014\\
19	0.00694646539739823\\
20	0.00694644045790419\\
21	0.00694641509255667\\
22	0.0069463892940301\\
23	0.00694636305487271\\
24	0.00694633636750413\\
25	0.00694630922421341\\
26	0.00694628161715664\\
27	0.00694625353835466\\
28	0.00694622497969081\\
29	0.00694619593290846\\
30	0.00694616638960869\\
31	0.00694613634124777\\
32	0.00694610577913478\\
33	0.00694607469442889\\
34	0.00694604307813698\\
35	0.0069460109211109\\
36	0.00694597821404477\\
37	0.00694594494747236\\
38	0.00694591111176434\\
39	0.00694587669712533\\
40	0.00694584169359116\\
41	0.00694580609102596\\
42	0.00694576987911914\\
43	0.0069457330473824\\
44	0.00694569558514672\\
45	0.0069456574815592\\
46	0.00694561872557987\\
47	0.00694557930597856\\
48	0.00694553921133148\\
49	0.00694549843001798\\
50	0.00694545695021716\\
51	0.00694541475990437\\
52	0.00694537184684776\\
53	0.00694532819860465\\
54	0.00694528380251787\\
55	0.00694523864571218\\
56	0.00694519271509042\\
57	0.00694514599732968\\
58	0.00694509847887745\\
59	0.00694505014594761\\
60	0.00694500098451646\\
61	0.00694495098031857\\
62	0.00694490011884265\\
63	0.00694484838532735\\
64	0.00694479576475676\\
65	0.00694474224185629\\
66	0.00694468780108797\\
67	0.00694463242664606\\
68	0.00694457610245239\\
69	0.00694451881215163\\
70	0.00694446053910653\\
71	0.00694440126639309\\
72	0.00694434097679557\\
73	0.0069442796528015\\
74	0.00694421727659651\\
75	0.0069441538300592\\
76	0.00694408929475578\\
77	0.00694402365193476\\
78	0.00694395688252141\\
79	0.00694388896711222\\
80	0.00694381988596922\\
81	0.00694374961901428\\
82	0.00694367814582314\\
83	0.00694360544561959\\
84	0.00694353149726931\\
85	0.00694345627927373\\
86	0.00694337976976384\\
87	0.00694330194649369\\
88	0.00694322278683403\\
89	0.00694314226776572\\
90	0.00694306036587285\\
91	0.00694297705733624\\
92	0.00694289231792625\\
93	0.00694280612299587\\
94	0.0069427184474735\\
95	0.00694262926585578\\
96	0.00694253855220005\\
97	0.00694244628011693\\
98	0.00694235242276263\\
99	0.00694225695283123\\
100	0.00694215984254669\\
101	0.00694206106365491\\
102	0.00694196058741549\\
103	0.00694185838459349\\
104	0.00694175442545098\\
105	0.00694164867973846\\
106	0.0069415411166861\\
107	0.00694143170499497\\
108	0.00694132041282801\\
109	0.0069412072078008\\
110	0.00694109205697233\\
111	0.00694097492683563\\
112	0.00694085578330795\\
113	0.00694073459172116\\
114	0.00694061131681176\\
115	0.00694048592271083\\
116	0.00694035837293372\\
117	0.00694022863036972\\
118	0.00694009665727136\\
119	0.00693996241524365\\
120	0.00693982586523321\\
121	0.00693968696751714\\
122	0.00693954568169164\\
123	0.00693940196666051\\
124	0.00693925578062355\\
125	0.00693910708106462\\
126	0.0069389558247396\\
127	0.00693880196766406\\
128	0.00693864546510077\\
129	0.00693848627154717\\
130	0.00693832434072222\\
131	0.00693815962555352\\
132	0.00693799207816383\\
133	0.00693782164985762\\
134	0.00693764829110722\\
135	0.00693747195153883\\
136	0.00693729257991836\\
137	0.00693711012413684\\
138	0.00693692453119565\\
139	0.00693673574719145\\
140	0.00693654371730047\\
141	0.00693634838576177\\
142	0.00693614969585849\\
143	0.00693594758989551\\
144	0.00693574200916975\\
145	0.00693553289393244\\
146	0.00693532018335594\\
147	0.00693510381555999\\
148	0.00693488372774939\\
149	0.00693465985608184\\
150	0.0069344321356242\\
151	0.0069342005003339\\
152	0.00693396488304016\\
153	0.00693372521542484\\
154	0.00693348142800299\\
155	0.00693323345010296\\
156	0.00693298120984649\\
157	0.00693272463412804\\
158	0.00693246364859417\\
159	0.00693219817762225\\
160	0.00693192814429906\\
161	0.00693165347039893\\
162	0.00693137407636147\\
163	0.00693108988126901\\
164	0.00693080080282362\\
165	0.00693050675732383\\
166	0.00693020765964077\\
167	0.00692990342319414\\
168	0.00692959395992757\\
169	0.00692927918028385\\
170	0.00692895899317939\\
171	0.00692863330597857\\
172	0.00692830202446747\\
173	0.00692796505282729\\
174	0.0069276222936072\\
175	0.00692727364769687\\
176	0.00692691901429847\\
177	0.00692655829089819\\
178	0.00692619137323748\\
179	0.00692581815528343\\
180	0.00692543852919908\\
181	0.00692505238531296\\
182	0.00692465961208832\\
183	0.00692426009609168\\
184	0.00692385372196094\\
185	0.00692344037237305\\
186	0.00692301992801094\\
187	0.00692259226753029\\
188	0.00692215726752522\\
189	0.00692171480249389\\
190	0.0069212647448033\\
191	0.00692080696465362\\
192	0.00692034133004186\\
193	0.00691986770672502\\
194	0.00691938595818254\\
195	0.00691889594557832\\
196	0.00691839752772193\\
197	0.00691789056102931\\
198	0.00691737489948283\\
199	0.00691685039459065\\
200	0.00691631689534553\\
201	0.00691577424818281\\
202	0.00691522229693786\\
203	0.00691466088280287\\
204	0.00691408984428273\\
205	0.00691350901715058\\
206	0.00691291823440226\\
207	0.00691231732621025\\
208	0.0069117061198769\\
209	0.00691108443978675\\
210	0.00691045210735832\\
211	0.00690980894099496\\
212	0.00690915475603499\\
213	0.0069084893647011\\
214	0.00690781257604885\\
215	0.00690712419591445\\
216	0.00690642402686185\\
217	0.00690571186812849\\
218	0.00690498751557098\\
219	0.00690425076160927\\
220	0.00690350139517041\\
221	0.00690273920163103\\
222	0.00690196396275935\\
223	0.00690117545665597\\
224	0.00690037345769399\\
225	0.00689955773645808\\
226	0.00689872805968258\\
227	0.00689788419018891\\
228	0.0068970258868217\\
229	0.00689615290438418\\
230	0.00689526499357249\\
231	0.00689436190090919\\
232	0.00689344336867545\\
233	0.00689250913484253\\
234	0.00689155893300214\\
235	0.0068905924922957\\
236	0.00688960953734268\\
237	0.0068886097881678\\
238	0.00688759296012722\\
239	0.0068865587638336\\
240	0.0068855069050802\\
241	0.00688443708476369\\
242	0.00688334899880603\\
243	0.00688224233807519\\
244	0.0068811167883046\\
245	0.00687997203001178\\
246	0.00687880773841547\\
247	0.00687762358335181\\
248	0.00687641922918933\\
249	0.00687519433474276\\
250	0.00687394855318545\\
251	0.00687268153196102\\
252	0.00687139291269337\\
253	0.00687008233109573\\
254	0.00686874941687837\\
255	0.00686739379365499\\
256	0.00686601507884814\\
257	0.00686461288359297\\
258	0.00686318681264006\\
259	0.00686173646425668\\
260	0.00686026143012689\\
261	0.00685876129525027\\
262	0.00685723563783935\\
263	0.00685568402921559\\
264	0.00685410603370409\\
265	0.0068525012085269\\
266	0.00685086910369517\\
267	0.00684920926189953\\
268	0.00684752121839974\\
269	0.00684580450091296\\
270	0.00684405862950203\\
271	0.00684228311646515\\
272	0.00684047746623336\\
273	0.00683864117529172\\
274	0.00683677373216894\\
275	0.00683487461760879\\
276	0.00683294330516728\\
277	0.0068309792625111\\
278	0.00682898195220097\\
279	0.00682695082218309\\
280	0.00682488531004993\\
281	0.00682278484442637\\
282	0.00682064884492553\\
283	0.00681847672203204\\
284	0.00681626787698549\\
285	0.00681402170166375\\
286	0.00681173757846667\\
287	0.00680941488020006\\
288	0.00680705296996019\\
289	0.00680465120101892\\
290	0.00680220891670961\\
291	0.00679972545031397\\
292	0.00679720012495015\\
293	0.00679463225346215\\
294	0.00679202113831082\\
295	0.00678936607146664\\
296	0.00678666633430457\\
297	0.00678392119750127\\
298	0.00678112992093475\\
299	0.00677829175358691\\
300	0.0067754059334494\\
301	0.0067724716874326\\
302	0.00676948823127832\\
303	0.0067664547694763\\
304	0.00676337049518445\\
305	0.00676023459015205\\
306	0.0067570462246441\\
307	0.00675380455736031\\
308	0.00675050873533282\\
309	0.00674715789375852\\
310	0.00674375115565688\\
311	0.00674028763108845\\
312	0.00673676641535544\\
313	0.00673318658517247\\
314	0.00672954719226556\\
315	0.00672584726090572\\
316	0.00672208582459313\\
317	0.00671826191934218\\
318	0.00671437457069753\\
319	0.0067104227917692\\
320	0.00670640558318577\\
321	0.00670232193305088\\
322	0.00669817081690297\\
323	0.00669395119767772\\
324	0.00668966202567304\\
325	0.00668530223851645\\
326	0.00668087076113417\\
327	0.00667636650572151\\
328	0.00667178837171383\\
329	0.00666713524575737\\
330	0.00666240600167895\\
331	0.00665759950045361\\
332	0.00665271459016837\\
333	0.00664775010598122\\
334	0.00664270487007293\\
335	0.00663757769158991\\
336	0.00663236736657532\\
337	0.00662707267788572\\
338	0.00662169239508965\\
339	0.00661622527434456\\
340	0.00661067005824696\\
341	0.00660502547565139\\
342	0.00659929024145186\\
343	0.00659346305631933\\
344	0.00658754260638799\\
345	0.00658152756288226\\
346	0.00657541658167648\\
347	0.00656920830277777\\
348	0.00656290134972417\\
349	0.00655649432889036\\
350	0.00654998582869663\\
351	0.00654337441872457\\
352	0.00653665864876187\\
353	0.0065298370478523\\
354	0.00652290812358438\\
355	0.00651587036233241\\
356	0.00650872223266671\\
357	0.00650146219875987\\
358	0.00649408876322378\\
359	0.00648660057352709\\
360	0.00647899633611881\\
361	0.00647127446191862\\
362	0.00646343328816032\\
363	0.00645547112456574\\
364	0.00644738625466931\\
365	0.00643917693731825\\
366	0.00643084140625058\\
367	0.00642237786411884\\
368	0.00641378448435023\\
369	0.00640505963384136\\
370	0.0063962016968313\\
371	0.00638720907496289\\
372	0.00637808019165857\\
373	0.00636881349640315\\
374	0.00635940746804279\\
375	0.00634986061467753\\
376	0.00634017146360102\\
377	0.0063303385239127\\
378	0.00632036017812918\\
379	0.00631023441153605\\
380	0.00629995835099525\\
381	0.00628952983903434\\
382	0.00627894748838681\\
383	0.00626821011798545\\
384	0.00625731668427991\\
385	0.00624626630671213\\
386	0.00623505829751204\\
387	0.00622369219660639\\
388	0.00621216781259234\\
389	0.00620048527092107\\
390	0.00618864507067233\\
391	0.00617664815158981\\
392	0.00616449597340108\\
393	0.00615219060988033\\
394	0.00613973486064801\\
395	0.00612713238436226\\
396	0.00611438785777488\\
397	0.00610150716613895\\
398	0.00608849763171694\\
399	0.00607536828871431\\
400	0.00606213021494645\\
401	0.00604879693308199\\
402	0.0060353848976741\\
403	0.00602191408910535\\
404	0.00600840874430738\\
405	0.00599489827521582\\
406	0.00598141849593739\\
407	0.00596801358005025\\
408	0.00595474082519926\\
409	0.00594167761009698\\
410	0.00592890464599032\\
411	0.00591651994719634\\
412	0.00590464274849586\\
413	0.0058934183122294\\
414	0.00588302352550035\\
415	0.005873672292017\\
416	0.00586463262703322\\
417	0.00585546218744932\\
418	0.00584616175105737\\
419	0.0058367325413083\\
420	0.00582717628869936\\
421	0.00581749523685951\\
422	0.00580769193367492\\
423	0.00579776827942576\\
424	0.00578772198825286\\
425	0.00577753309913231\\
426	0.00576717950020707\\
427	0.00575665945030968\\
428	0.00574597124855693\\
429	0.00573511323967497\\
430	0.00572408381979775\\
431	0.00571288144278806\\
432	0.00570150462709614\\
433	0.00568995196318385\\
434	0.00567822212160286\\
435	0.00566631386179493\\
436	0.00565422604166051\\
437	0.00564195762795269\\
438	0.00562950770755671\\
439	0.00561687549971017\\
440	0.00560406036921005\\
441	0.00559106184064171\\
442	0.00557787961370345\\
443	0.00556451357991259\\
444	0.00555096384037302\\
445	0.00553723072412042\\
446	0.00552331480693291\\
447	0.00550921692951282\\
448	0.00549493821106663\\
449	0.00548048001355052\\
450	0.00546584365931318\\
451	0.00545103054659524\\
452	0.00543604216524697\\
453	0.00542088000505086\\
454	0.00540554542385716\\
455	0.00539003945985792\\
456	0.00537436257255373\\
457	0.00535851431654713\\
458	0.00534249302310106\\
459	0.005326300620607\\
460	0.00530993977572947\\
461	0.00529341246413642\\
462	0.00527671948422658\\
463	0.00525985977606376\\
464	0.0052428295268324\\
465	0.00522562099366093\\
466	0.00520822093174952\\
467	0.00519060842437592\\
468	0.00517275203084588\\
469	0.00515460601916424\\
470	0.00513610538144768\\
471	0.00511716187763846\\
472	0.00509776069838698\\
473	0.00507788617190998\\
474	0.00505752168482359\\
475	0.00503664974569963\\
476	0.00501525245848585\\
477	0.00499331115423009\\
478	0.00497080623605254\\
479	0.00494771706670877\\
480	0.00492402188656566\\
481	0.00489969775395363\\
482	0.00487472058900362\\
483	0.0048490652129133\\
484	0.00482270245235811\\
485	0.0047956005757536\\
486	0.00476772442343782\\
487	0.00473904035393458\\
488	0.00470951624803794\\
489	0.0046791196018706\\
490	0.00464781759877594\\
491	0.0046155769116861\\
492	0.00458236369881093\\
493	0.00454814351465856\\
494	0.00451288108072529\\
495	0.00447653991846318\\
496	0.00443908164873674\\
497	0.00440046483465721\\
498	0.00436064307353843\\
499	0.00431956066780517\\
500	0.00427714610387846\\
501	0.00423333184872459\\
502	0.00418805823184039\\
503	0.00414127884382855\\
504	0.00409296795400067\\
505	0.00404313061070491\\
506	0.00399181187508489\\
507	0.00393911226723767\\
508	0.00388520834453271\\
509	0.00383037974367361\\
510	0.00377549415595658\\
511	0.00372172196201362\\
512	0.00366923277375448\\
513	0.00361820554168301\\
514	0.0035688256063494\\
515	0.00352128004985227\\
516	0.00347575048911292\\
517	0.00343240284597587\\
518	0.00339137233319415\\
519	0.00335274239791009\\
520	0.00331651583374602\\
521	0.00328257512187733\\
522	0.00324954687571363\\
523	0.00321703873827858\\
524	0.0031850442592495\\
525	0.00315354400458046\\
526	0.00312250357931197\\
527	0.00309187183781254\\
528	0.00306157953849304\\
529	0.00303153882359366\\
530	0.00300164409444535\\
531	0.00297177508822339\\
532	0.00294180349620389\\
533	0.00291162737945269\\
534	0.00288121816240077\\
535	0.00285054396232949\\
536	0.0028195698273891\\
537	0.00278825813558\\
538	0.00275656917596266\\
539	0.00272446192340949\\
540	0.00269189499921097\\
541	0.00265882778302623\\
542	0.00262522158590738\\
543	0.00259104071058111\\
544	0.00255625310260784\\
545	0.00252082680627605\\
546	0.00248472875678594\\
547	0.0024479250553087\\
548	0.00241038132282204\\
549	0.00237206314687042\\
550	0.00233293663899378\\
551	0.00229296913515159\\
552	0.00225212977229066\\
553	0.00221039882459138\\
554	0.0021677601511439\\
555	0.00212419835623349\\
556	0.00207969864940423\\
557	0.00203424697558643\\
558	0.00198783015632776\\
559	0.00194043604283886\\
560	0.0018920536825513\\
561	0.00184267349774045\\
562	0.0017922874812342\\
563	0.00174088941369864\\
564	0.00168847509589551\\
565	0.00163504253595368\\
566	0.00158059220452231\\
567	0.00152512740094253\\
568	0.00146865479755459\\
569	0.00141118627857556\\
570	0.00135274699387529\\
571	0.0012933730569172\\
572	0.00123315339091021\\
573	0.00117210960961848\\
574	0.0011101693316119\\
575	0.0010472300279469\\
576	0.000984276860620152\\
577	0.00092279293671995\\
578	0.000863072420869007\\
579	0.000805675803180638\\
580	0.000750592684941863\\
581	0.000696982038873612\\
582	0.000645396014974021\\
583	0.000595986134186247\\
584	0.000548695373864638\\
585	0.000502773121282149\\
586	0.000457999343816736\\
587	0.000414012697924258\\
588	0.000370724689282775\\
589	0.0003281128921034\\
590	0.00028620269797633\\
591	0.00024502317858971\\
592	0.00020461602593791\\
593	0.000165017211836693\\
594	0.000126301460681323\\
595	8.88161203105206e-05\\
596	5.34134895574398e-05\\
597	2.21100055488407e-05\\
598	0\\
599	0\\
600	0\\
};
\addplot [color=blue!75!mycolor7,solid,forget plot]
  table[row sep=crcr]{%
1	0.0074924042369101\\
2	0.00749239506213014\\
3	0.00749238573072458\\
4	0.00749237624000059\\
5	0.00749236658721893\\
6	0.00749235676959312\\
7	0.00749234678428867\\
8	0.00749233662842226\\
9	0.00749232629906093\\
10	0.00749231579322112\\
11	0.00749230510786791\\
12	0.00749229423991404\\
13	0.00749228318621916\\
14	0.00749227194358871\\
15	0.00749226050877319\\
16	0.00749224887846705\\
17	0.00749223704930785\\
18	0.00749222501787518\\
19	0.00749221278068979\\
20	0.00749220033421243\\
21	0.00749218767484292\\
22	0.00749217479891907\\
23	0.00749216170271557\\
24	0.00749214838244303\\
25	0.00749213483424673\\
26	0.00749212105420562\\
27	0.00749210703833106\\
28	0.00749209278256576\\
29	0.00749207828278255\\
30	0.00749206353478315\\
31	0.00749204853429706\\
32	0.00749203327698016\\
33	0.00749201775841356\\
34	0.00749200197410226\\
35	0.00749198591947388\\
36	0.0074919695898773\\
37	0.00749195298058131\\
38	0.0074919360867732\\
39	0.00749191890355744\\
40	0.00749190142595414\\
41	0.00749188364889773\\
42	0.00749186556723538\\
43	0.00749184717572557\\
44	0.00749182846903647\\
45	0.00749180944174449\\
46	0.00749179008833266\\
47	0.007491770403189\\
48	0.00749175038060492\\
49	0.00749173001477353\\
50	0.007491709299788\\
51	0.00749168822963974\\
52	0.00749166679821676\\
53	0.00749164499930182\\
54	0.00749162282657067\\
55	0.00749160027359014\\
56	0.00749157733381634\\
57	0.00749155400059271\\
58	0.0074915302671481\\
59	0.00749150612659481\\
60	0.00749148157192654\\
61	0.00749145659601643\\
62	0.00749143119161491\\
63	0.0074914053513476\\
64	0.00749137906771326\\
65	0.00749135233308147\\
66	0.00749132513969053\\
67	0.00749129747964512\\
68	0.00749126934491408\\
69	0.00749124072732802\\
70	0.00749121161857695\\
71	0.0074911820102079\\
72	0.00749115189362249\\
73	0.00749112126007436\\
74	0.00749109010066671\\
75	0.00749105840634964\\
76	0.00749102616791765\\
77	0.00749099337600681\\
78	0.0074909600210922\\
79	0.00749092609348508\\
80	0.00749089158333009\\
81	0.00749085648060244\\
82	0.00749082077510497\\
83	0.00749078445646523\\
84	0.00749074751413248\\
85	0.00749070993737465\\
86	0.00749067171527522\\
87	0.00749063283673014\\
88	0.00749059329044456\\
89	0.00749055306492962\\
90	0.00749051214849914\\
91	0.00749047052926625\\
92	0.00749042819513999\\
93	0.00749038513382188\\
94	0.00749034133280228\\
95	0.00749029677935688\\
96	0.00749025146054308\\
97	0.00749020536319627\\
98	0.00749015847392597\\
99	0.00749011077911215\\
100	0.00749006226490123\\
101	0.00749001291720217\\
102	0.00748996272168245\\
103	0.00748991166376396\\
104	0.00748985972861885\\
105	0.00748980690116533\\
106	0.00748975316606336\\
107	0.00748969850771025\\
108	0.00748964291023628\\
109	0.00748958635750016\\
110	0.00748952883308447\\
111	0.00748947032029089\\
112	0.00748941080213562\\
113	0.00748935026134442\\
114	0.0074892886803478\\
115	0.00748922604127593\\
116	0.0074891623259537\\
117	0.00748909751589544\\
118	0.00748903159229973\\
119	0.00748896453604408\\
120	0.00748889632767947\\
121	0.00748882694742483\\
122	0.00748875637516146\\
123	0.00748868459042729\\
124	0.00748861157241107\\
125	0.00748853729994645\\
126	0.00748846175150601\\
127	0.00748838490519511\\
128	0.00748830673874567\\
129	0.00748822722950986\\
130	0.00748814635445365\\
131	0.00748806409015029\\
132	0.00748798041277362\\
133	0.00748789529809124\\
134	0.00748780872145778\\
135	0.00748772065780771\\
136	0.00748763108164834\\
137	0.0074875399670525\\
138	0.00748744728765113\\
139	0.00748735301662569\\
140	0.0074872571267003\\
141	0.00748715959013352\\
142	0.00748706037870942\\
143	0.00748695946372785\\
144	0.00748685681599415\\
145	0.00748675240581058\\
146	0.00748664620297454\\
147	0.00748653817678317\\
148	0.00748642829601296\\
149	0.00748631652890841\\
150	0.00748620284317294\\
151	0.00748608720595981\\
152	0.00748596958386282\\
153	0.00748584994290679\\
154	0.00748572824853804\\
155	0.00748560446561454\\
156	0.00748547855839603\\
157	0.00748535049053385\\
158	0.00748522022506078\\
159	0.00748508772438051\\
160	0.00748495295025705\\
161	0.00748481586380396\\
162	0.00748467642547336\\
163	0.00748453459504479\\
164	0.0074843903316139\\
165	0.00748424359358089\\
166	0.00748409433863882\\
167	0.00748394252376171\\
168	0.0074837881051924\\
169	0.00748363103843031\\
170	0.00748347127821889\\
171	0.00748330877853291\\
172	0.00748314349256557\\
173	0.00748297537271532\\
174	0.00748280437057253\\
175	0.00748263043690597\\
176	0.00748245352164894\\
177	0.0074822735738853\\
178	0.00748209054183523\\
179	0.00748190437284069\\
180	0.00748171501335078\\
181	0.00748152240890672\\
182	0.00748132650412671\\
183	0.00748112724269037\\
184	0.00748092456732317\\
185	0.00748071841978038\\
186	0.00748050874083089\\
187	0.00748029547024069\\
188	0.00748007854675619\\
189	0.00747985790808712\\
190	0.00747963349088927\\
191	0.00747940523074695\\
192	0.00747917306215502\\
193	0.00747893691850082\\
194	0.00747869673204577\\
195	0.00747845243390651\\
196	0.00747820395403595\\
197	0.00747795122120389\\
198	0.00747769416297732\\
199	0.00747743270570058\\
200	0.00747716677447484\\
201	0.00747689629313772\\
202	0.00747662118424218\\
203	0.0074763413690353\\
204	0.00747605676743666\\
205	0.00747576729801631\\
206	0.00747547287797256\\
207	0.00747517342310925\\
208	0.0074748688478127\\
209	0.00747455906502837\\
210	0.00747424398623706\\
211	0.00747392352143078\\
212	0.00747359757908827\\
213	0.00747326606615006\\
214	0.00747292888799319\\
215	0.00747258594840554\\
216	0.00747223714955969\\
217	0.00747188239198657\\
218	0.00747152157454836\\
219	0.00747115459441131\\
220	0.00747078134701797\\
221	0.007470401726059\\
222	0.0074700156234446\\
223	0.00746962292927545\\
224	0.0074692235318132\\
225	0.00746881731745064\\
226	0.00746840417068127\\
227	0.0074679839740684\\
228	0.00746755660821392\\
229	0.00746712195172656\\
230	0.0074666798811896\\
231	0.00746623027112818\\
232	0.00746577299397611\\
233	0.00746530792004223\\
234	0.00746483491747617\\
235	0.00746435385223381\\
236	0.00746386458804207\\
237	0.00746336698636328\\
238	0.00746286090635916\\
239	0.00746234620485403\\
240	0.0074618227362978\\
241	0.00746129035272837\\
242	0.00746074890373346\\
243	0.00746019823641203\\
244	0.00745963819533521\\
245	0.00745906862250666\\
246	0.00745848935732256\\
247	0.00745790023653101\\
248	0.00745730109419103\\
249	0.00745669176163109\\
250	0.00745607206740713\\
251	0.00745544183726021\\
252	0.00745480089407363\\
253	0.00745414905782975\\
254	0.00745348614556628\\
255	0.00745281197133225\\
256	0.00745212634614359\\
257	0.00745142907793836\\
258	0.00745071997153162\\
259	0.00744999882857008\\
260	0.00744926544748631\\
261	0.00744851962345285\\
262	0.00744776114833602\\
263	0.00744698981064949\\
264	0.00744620539550787\\
265	0.00744540768457994\\
266	0.00744459645604195\\
267	0.00744377148453098\\
268	0.00744293254109817\\
269	0.00744207939316252\\
270	0.0074412118044653\\
271	0.00744032953502675\\
272	0.00743943234110788\\
273	0.00743851997518487\\
274	0.00743759218595036\\
275	0.00743664871836117\\
276	0.00743568931371634\\
277	0.00743471370954964\\
278	0.00743372163875567\\
279	0.00743271283000344\\
280	0.00743168700784525\\
281	0.00743064389268382\\
282	0.00742958320073343\\
283	0.00742850464398215\\
284	0.00742740793015531\\
285	0.00742629276268059\\
286	0.00742515884065443\\
287	0.0074240058588103\\
288	0.0074228335074887\\
289	0.00742164147260929\\
290	0.00742042943564502\\
291	0.00741919707359877\\
292	0.00741794405898229\\
293	0.00741667005979812\\
294	0.00741537473952427\\
295	0.00741405775710208\\
296	0.0074127187669274\\
297	0.00741135741884556\\
298	0.00740997335815006\\
299	0.00740856622558549\\
300	0.00740713565735493\\
301	0.00740568128513199\\
302	0.0074042027360781\\
303	0.00740269963286488\\
304	0.00740117159370228\\
305	0.00739961823237191\\
306	0.00739803915826482\\
307	0.0073964339764209\\
308	0.00739480228756206\\
309	0.00739314368810112\\
310	0.00739145777008878\\
311	0.00738974412103208\\
312	0.0073880023235208\\
313	0.00738623195480151\\
314	0.00738443258727678\\
315	0.0073826037919998\\
316	0.00738074513790797\\
317	0.00737885619067737\\
318	0.00737693651269443\\
319	0.00737498566322692\\
320	0.00737300319861045\\
321	0.0073709886724509\\
322	0.00736894163584407\\
323	0.00736686163761355\\
324	0.00736474822456766\\
325	0.00736260094177692\\
326	0.00736041933287318\\
327	0.00735820294037166\\
328	0.00735595130601743\\
329	0.00735366397115779\\
330	0.00735134047714198\\
331	0.00734898036575014\\
332	0.00734658317965315\\
333	0.00734414846290497\\
334	0.00734167576146975\\
335	0.00733916462378496\\
336	0.00733661460136281\\
337	0.00733402524943113\\
338	0.007331396127615\\
339	0.00732872680066001\\
340	0.00732601683919667\\
341	0.00732326582054489\\
342	0.00732047332955525\\
343	0.00731763895948154\\
344	0.00731476231287552\\
345	0.00731184300249034\\
346	0.00730888065217183\\
347	0.00730587489770941\\
348	0.00730282538760534\\
349	0.00729973178370637\\
350	0.00729659376162193\\
351	0.00729341101082845\\
352	0.00729018323433446\\
353	0.00728691014776416\\
354	0.00728359147774521\\
355	0.00728022695963563\\
356	0.00727681633502597\\
357	0.00727335934990114\\
358	0.00726985575162433\\
359	0.00726630526080623\\
360	0.0072627075219497\\
361	0.00725906209845442\\
362	0.00725536844057738\\
363	0.00725162583561816\\
364	0.00724783333997737\\
365	0.00724398968401482\\
366	0.00724009312976099\\
367	0.00723614122745122\\
368	0.00723213030331952\\
369	0.00722805028703685\\
370	0.00722389889421287\\
371	0.0072196743740852\\
372	0.00721537490020484\\
373	0.0072109985683676\\
374	0.00720654339513614\\
375	0.00720200731676553\\
376	0.00719738818823209\\
377	0.00719268378442597\\
378	0.00718789182452313\\
379	0.0071830101526967\\
380	0.00717803786455369\\
381	0.00717297302174104\\
382	0.00716781263677195\\
383	0.00716255349826023\\
384	0.0071571921445855\\
385	0.00715172483369407\\
386	0.00714614750836237\\
387	0.00714045575611459\\
388	0.00713464476282308\\
389	0.00712870925881882\\
390	0.00712264345609405\\
391	0.00711644097487716\\
392	0.00711009475748809\\
393	0.00710359696692507\\
394	0.00709693886706633\\
395	0.00709011068066894\\
396	0.00708310142047591\\
397	0.00707589868765737\\
398	0.00706848843045399\\
399	0.00706085465417936\\
400	0.00705297907154527\\
401	0.00704484067937505\\
402	0.00703641524368279\\
403	0.00702767466862863\\
404	0.00701858621263323\\
405	0.0070091114860898\\
406	0.00699920507810039\\
407	0.00698881230796572\\
408	0.00697786340322368\\
409	0.00696628449370698\\
410	0.00695399082909338\\
411	0.00694088181989411\\
412	0.00692683759102407\\
413	0.00691171474897144\\
414	0.00689534123322774\\
415	0.00687751107176523\\
416	0.0068589445373444\\
417	0.0068400736467418\\
418	0.00682089451339999\\
419	0.00680140327978297\\
420	0.00678159610050093\\
421	0.00676146908036924\\
422	0.00674101809379129\\
423	0.00672023832728233\\
424	0.00669912345187745\\
425	0.00667766734831997\\
426	0.00665586601704147\\
427	0.00663371588571305\\
428	0.00661121352443312\\
429	0.0065883556680858\\
430	0.00656513924150641\\
431	0.00654156138785944\\
432	0.00651761950070601\\
433	0.00649331126032545\\
434	0.00646863467495198\\
435	0.00644358812770502\\
436	0.006418170430136\\
437	0.00639238088348852\\
438	0.00636621934897878\\
439	0.00633968632865795\\
440	0.00631278305872948\\
441	0.00628551161758354\\
442	0.00625787505132119\\
443	0.00622987752022658\\
444	0.00620152447109456\\
445	0.00617282284439145\\
446	0.00614378133980875\\
447	0.0061144108289088\\
448	0.00608472516573098\\
449	0.00605474503763231\\
450	0.00602451632089729\\
451	0.0059940757788918\\
452	0.00596345983494581\\
453	0.00593271253265712\\
454	0.00590188708135116\\
455	0.00587104757514936\\
456	0.00584027051878397\\
457	0.00580964478849715\\
458	0.00577926518114491\\
459	0.00574884459712267\\
460	0.00571834681854514\\
461	0.00568784185090121\\
462	0.00565741516602163\\
463	0.00562717144134524\\
464	0.00559723828658416\\
465	0.00556777100765139\\
466	0.00553895881724486\\
467	0.00551103263774442\\
468	0.00548427489164653\\
469	0.00545903143612402\\
470	0.00543572631584556\\
471	0.0054147965624325\\
472	0.00539349444727599\\
473	0.00537181420825687\\
474	0.00534975006497606\\
475	0.00532729626326827\\
476	0.00530444705284925\\
477	0.00528119668588837\\
478	0.00525753942186825\\
479	0.00523346952832216\\
480	0.00520898123931054\\
481	0.005184068476006\\
482	0.00515872469670289\\
483	0.00513294295240333\\
484	0.00510671600976015\\
485	0.00508003638604027\\
486	0.00505289675152728\\
487	0.00502528991797591\\
488	0.00499720876771142\\
489	0.00496864706955647\\
490	0.00493960006460045\\
491	0.00491006342328762\\
492	0.00488003257272219\\
493	0.00484950249627489\\
494	0.00481846789260173\\
495	0.00478691941683505\\
496	0.00475484403321968\\
497	0.00472222377036806\\
498	0.00468903406153541\\
499	0.00465524054947011\\
500	0.00462080662780665\\
501	0.00458568465853478\\
502	0.00454981067668868\\
503	0.00451309773425715\\
504	0.00447542720897012\\
505	0.00443663751685369\\
506	0.00439667946359437\\
507	0.00435550655755052\\
508	0.00431307046188243\\
509	0.00426932728172256\\
510	0.00422422752017184\\
511	0.00417770960408217\\
512	0.00412970135996104\\
513	0.00408013656419233\\
514	0.00402895935181766\\
515	0.00397613019795216\\
516	0.00392163360410665\\
517	0.00386548864943913\\
518	0.00380776307374029\\
519	0.00374859197637785\\
520	0.00368820257472482\\
521	0.00362694688926678\\
522	0.00356637012747609\\
523	0.00350703592601809\\
524	0.00344912621670985\\
525	0.00339283064665058\\
526	0.0033383426009943\\
527	0.00328585325766154\\
528	0.00323554292911896\\
529	0.00318756877619778\\
530	0.0031420473407322\\
531	0.00309903033854259\\
532	0.00305846404358267\\
533	0.00301973398722696\\
534	0.00298147907057372\\
535	0.00294369779238631\\
536	0.0029063754048515\\
537	0.00286948159293979\\
538	0.0028329682403081\\
539	0.00279676752725694\\
540	0.00276079073611847\\
541	0.00272492799658533\\
542	0.00268904999554552\\
543	0.00265301287973426\\
544	0.00261666780847322\\
545	0.00257994640352295\\
546	0.00254280720659853\\
547	0.00250520494078379\\
548	0.0024670910246756\\
549	0.00242841434684367\\
550	0.00238912233411181\\
551	0.00234916233066769\\
552	0.00230848328248133\\
553	0.00226703747237023\\
554	0.00222478701931318\\
555	0.0021817075326122\\
556	0.00213778081363613\\
557	0.00209298897830381\\
558	0.00204731468235649\\
559	0.00200074137081413\\
560	0.00195325354663635\\
561	0.00190483705770766\\
562	0.00185547938081325\\
563	0.00180516988557633\\
564	0.00175390005863065\\
565	0.0017016636684699\\
566	0.00164845685395718\\
567	0.00159427812763442\\
568	0.00153912832808868\\
569	0.00148301086881223\\
570	0.00142593197227525\\
571	0.00136790084687\\
572	0.00130892989353384\\
573	0.00124903539994602\\
574	0.00118823985767079\\
575	0.00112657419000161\\
576	0.00106408186145155\\
577	0.00100084678722425\\
578	0.000936906460728108\\
579	0.00087219575385531\\
580	0.000807381068522308\\
581	0.000744253919298644\\
582	0.000683149208921055\\
583	0.000624409451839995\\
584	0.00056899987841953\\
585	0.000516346813304098\\
586	0.000466156175058029\\
587	0.000418738807028305\\
588	0.000373486291036665\\
589	0.000329728904611942\\
590	0.000287090770953892\\
591	0.000245465410968095\\
592	0.000204771950375222\\
593	0.000165050678943705\\
594	0.000126301460681323\\
595	8.88161203105205e-05\\
596	5.34134895574398e-05\\
597	2.21100055488407e-05\\
598	0\\
599	0\\
600	0\\
};
\addplot [color=blue!80!mycolor9,solid,forget plot]
  table[row sep=crcr]{%
1	0.00965497018695193\\
2	0.00965495204258113\\
3	0.00965493358761741\\
4	0.00965491481673022\\
5	0.00965489572449764\\
6	0.00965487630540482\\
7	0.0096548565538424\\
8	0.0096548364641049\\
9	0.00965481603038905\\
10	0.00965479524679217\\
11	0.00965477410731042\\
12	0.00965475260583716\\
13	0.00965473073616105\\
14	0.00965470849196445\\
15	0.00965468586682145\\
16	0.0096546628541961\\
17	0.00965463944744055\\
18	0.00965461563979307\\
19	0.00965459142437615\\
20	0.00965456679419457\\
21	0.00965454174213332\\
22	0.00965451626095559\\
23	0.00965449034330074\\
24	0.00965446398168209\\
25	0.00965443716848485\\
26	0.00965440989596393\\
27	0.0096543821562417\\
28	0.00965435394130576\\
29	0.00965432524300659\\
30	0.00965429605305531\\
31	0.0096542663630212\\
32	0.00965423616432937\\
33	0.00965420544825828\\
34	0.00965417420593723\\
35	0.00965414242834384\\
36	0.00965411010630147\\
37	0.00965407723047657\\
38	0.00965404379137607\\
39	0.00965400977934462\\
40	0.00965397518456185\\
41	0.00965393999703954\\
42	0.00965390420661882\\
43	0.00965386780296721\\
44	0.00965383077557574\\
45	0.00965379311375591\\
46	0.00965375480663662\\
47	0.0096537158431611\\
48	0.00965367621208379\\
49	0.0096536359019671\\
50	0.0096535949011781\\
51	0.00965355319788532\\
52	0.00965351078005532\\
53	0.00965346763544924\\
54	0.00965342375161937\\
55	0.00965337911590559\\
56	0.00965333371543175\\
57	0.00965328753710205\\
58	0.0096532405675973\\
59	0.00965319279337109\\
60	0.00965314420064604\\
61	0.0096530947754098\\
62	0.00965304450341113\\
63	0.00965299337015583\\
64	0.00965294136090262\\
65	0.00965288846065902\\
66	0.009652834654177\\
67	0.00965277992594876\\
68	0.00965272426020226\\
69	0.0096526676408968\\
70	0.0096526100517185\\
71	0.00965255147607562\\
72	0.00965249189709388\\
73	0.00965243129761175\\
74	0.00965236966017552\\
75	0.00965230696703442\\
76	0.00965224320013557\\
77	0.00965217834111894\\
78	0.00965211237131208\\
79	0.00965204527172493\\
80	0.00965197702304439\\
81	0.00965190760562893\\
82	0.00965183699950305\\
83	0.00965176518435158\\
84	0.00965169213951404\\
85	0.00965161784397877\\
86	0.00965154227637701\\
87	0.00965146541497694\\
88	0.00965138723767745\\
89	0.00965130772200203\\
90	0.00965122684509239\\
91	0.00965114458370204\\
92	0.00965106091418975\\
93	0.00965097581251287\\
94	0.00965088925422066\\
95	0.00965080121444731\\
96	0.00965071166790502\\
97	0.00965062058887685\\
98	0.00965052795120954\\
99	0.00965043372830612\\
100	0.00965033789311843\\
101	0.00965024041813962\\
102	0.00965014127539626\\
103	0.00965004043644066\\
104	0.00964993787234274\\
105	0.00964983355368198\\
106	0.00964972745053914\\
107	0.00964961953248789\\
108	0.0096495097685862\\
109	0.00964939812736771\\
110	0.00964928457683289\\
111	0.00964916908444004\\
112	0.00964905161709613\\
113	0.0096489321411476\\
114	0.0096488106223708\\
115	0.00964868702596246\\
116	0.00964856131652987\\
117	0.009648433458081\\
118	0.00964830341401434\\
119	0.00964817114710863\\
120	0.00964803661951248\\
121	0.00964789979273363\\
122	0.00964776062762822\\
123	0.00964761908438977\\
124	0.00964747512253805\\
125	0.00964732870090764\\
126	0.00964717977763642\\
127	0.0096470283101538\\
128	0.0096468742551688\\
129	0.00964671756865782\\
130	0.00964655820585237\\
131	0.00964639612122644\\
132	0.00964623126848375\\
133	0.00964606360054474\\
134	0.00964589306953336\\
135	0.00964571962676362\\
136	0.00964554322272591\\
137	0.00964536380707311\\
138	0.00964518132860641\\
139	0.00964499573526083\\
140	0.00964480697409056\\
141	0.00964461499125392\\
142	0.00964441973199811\\
143	0.00964422114064373\\
144	0.00964401916056945\\
145	0.00964381373419684\\
146	0.0096436048029754\\
147	0.00964339230736474\\
148	0.00964317618681762\\
149	0.00964295637976299\\
150	0.00964273282358866\\
151	0.00964250545462378\\
152	0.00964227420812089\\
153	0.00964203901823771\\
154	0.00964179981801877\\
155	0.00964155653937645\\
156	0.00964130911307192\\
157	0.00964105746869568\\
158	0.00964080153464767\\
159	0.00964054123811728\\
160	0.00964027650506273\\
161	0.00964000726019028\\
162	0.00963973342693307\\
163	0.00963945492742951\\
164	0.00963917168250134\\
165	0.00963888361163138\\
166	0.00963859063294074\\
167	0.00963829266316583\\
168	0.0096379896176348\\
169	0.00963768141024367\\
170	0.00963736795343198\\
171	0.00963704915815813\\
172	0.00963672493387414\\
173	0.00963639518850007\\
174	0.00963605982839801\\
175	0.00963571875834549\\
176	0.00963537188150858\\
177	0.00963501909941442\\
178	0.0096346603119233\\
179	0.00963429541720022\\
180	0.00963392431168603\\
181	0.00963354689006795\\
182	0.00963316304524962\\
183	0.00963277266832066\\
184	0.00963237564852566\\
185	0.00963197187323258\\
186	0.0096315612279006\\
187	0.00963114359604752\\
188	0.00963071885921635\\
189	0.00963028689694149\\
190	0.00962984758671429\\
191	0.00962940080394782\\
192	0.00962894642194125\\
193	0.00962848431184342\\
194	0.00962801434261575\\
195	0.00962753638099462\\
196	0.00962705029145289\\
197	0.00962655593616085\\
198	0.00962605317494643\\
199	0.00962554186525456\\
200	0.00962502186210601\\
201	0.0096244930180553\\
202	0.00962395518314787\\
203	0.00962340820487649\\
204	0.00962285192813687\\
205	0.00962228619518239\\
206	0.00962171084557805\\
207	0.00962112571615347\\
208	0.00962053064095517\\
209	0.00961992545119777\\
210	0.00961930997521443\\
211	0.00961868403840623\\
212	0.00961804746319063\\
213	0.009617400068949\\
214	0.00961674167197312\\
215	0.00961607208541063\\
216	0.00961539111920946\\
217	0.00961469858006117\\
218	0.00961399427134329\\
219	0.00961327799306033\\
220	0.00961254954178392\\
221	0.00961180871059156\\
222	0.00961105528900429\\
223	0.0096102890629231\\
224	0.00960950981456411\\
225	0.00960871732239239\\
226	0.00960791136105457\\
227	0.00960709170131012\\
228	0.0096062581099611\\
229	0.00960541034978067\\
230	0.00960454817944007\\
231	0.00960367135343416\\
232	0.00960277962200546\\
233	0.00960187273106658\\
234	0.00960095042212111\\
235	0.00960001243218291\\
236	0.00959905849369366\\
237	0.00959808833443883\\
238	0.00959710167746167\\
239	0.00959609824097572\\
240	0.00959507773827518\\
241	0.00959403987764359\\
242	0.00959298436226045\\
243	0.009591910890106\\
244	0.00959081915386375\\
245	0.00958970884082113\\
246	0.00958857963276779\\
247	0.00958743120589187\\
248	0.0095862632306739\\
249	0.00958507537177835\\
250	0.00958386728794284\\
251	0.00958263863186483\\
252	0.00958138905008581\\
253	0.00958011818287291\\
254	0.00957882566409769\\
255	0.0095775111211123\\
256	0.00957617417462274\\
257	0.0095748144385592\\
258	0.00957343151994328\\
259	0.00957202501875225\\
260	0.00957059452777998\\
261	0.00956913963249458\\
262	0.00956765991089259\\
263	0.00956615493334974\\
264	0.00956462426246793\\
265	0.00956306745291855\\
266	0.00956148405128192\\
267	0.0095598735958828\\
268	0.00955823561662191\\
269	0.00955656963480352\\
270	0.00955487516295943\\
271	0.00955315170466955\\
272	0.00955139875437975\\
273	0.0095496157972168\\
274	0.0095478023087985\\
275	0.00954595775503347\\
276	0.00954408159189874\\
277	0.00954217326514716\\
278	0.00954023221014498\\
279	0.00953825785166094\\
280	0.00953624960363585\\
281	0.00953420686894513\\
282	0.00953212903915469\\
283	0.00953001549426992\\
284	0.00952786560247765\\
285	0.00952567871988064\\
286	0.00952345419022464\\
287	0.00952119134461741\\
288	0.00951888950123982\\
289	0.00951654796504829\\
290	0.00951416602746885\\
291	0.00951174296608198\\
292	0.00950927804429836\\
293	0.00950677051102498\\
294	0.00950421960032135\\
295	0.00950162453104562\\
296	0.0094989845064902\\
297	0.0094962987140064\\
298	0.00949356632461807\\
299	0.00949078649262375\\
300	0.00948795835518685\\
301	0.00948508103191377\\
302	0.00948215362441926\\
303	0.00947917521587892\\
304	0.00947614487056771\\
305	0.00947306163338395\\
306	0.00946992452935717\\
307	0.00946673256313739\\
308	0.00946348471846284\\
309	0.0094601799576027\\
310	0.00945681722077536\\
311	0.00945339542555257\\
312	0.00944991346627829\\
313	0.00944637021356014\\
314	0.00944276451398482\\
315	0.00943909518935328\\
316	0.0094353610358999\\
317	0.00943156082361616\\
318	0.00942769329557353\\
319	0.00942375716722943\\
320	0.00941975112571598\\
321	0.00941567382911174\\
322	0.00941152390569652\\
323	0.00940729995318932\\
324	0.00940300053796974\\
325	0.00939862419428304\\
326	0.00939416942342937\\
327	0.00938963469293767\\
328	0.00938501843572465\\
329	0.00938031904923997\\
330	0.00937553489459816\\
331	0.00937066429569856\\
332	0.00936570553833459\\
333	0.0093606568692938\\
334	0.00935551649545057\\
335	0.00935028258285381\\
336	0.00934495325581231\\
337	0.00933952659598115\\
338	0.00933400064145329\\
339	0.00932837338586146\\
340	0.00932264277749711\\
341	0.00931680671845429\\
342	0.00931086306380939\\
343	0.00930480962084981\\
344	0.00929864414836947\\
345	0.00929236435605318\\
346	0.00928596790397956\\
347	0.0092794524022805\\
348	0.00927281541100742\\
349	0.00926605444027035\\
350	0.00925916695073759\\
351	0.00925215035461403\\
352	0.00924500201725672\\
353	0.0092377192596408\\
354	0.00923029936194912\\
355	0.00922273956860483\\
356	0.0092150370950777\\
357	0.00920718913680887\\
358	0.00919919287924184\\
359	0.00919104551082314\\
360	0.00918274424628183\\
361	0.0091742863547925\\
362	0.00916566919767199\\
363	0.00915689028159032\\
364	0.00914794733950183\\
365	0.00913883847019818\\
366	0.0091295624239009\\
367	0.00912011931536049\\
368	0.00911051298327061\\
369	0.00910117833255604\\
370	0.00909171561842355\\
371	0.00908209058493244\\
372	0.00907230016985511\\
373	0.00906234124237296\\
374	0.00905221060126851\\
375	0.00904190497370628\\
376	0.00903142101565248\\
377	0.00902075531438558\\
378	0.00900990438881151\\
379	0.00899886475236882\\
380	0.00898763268450551\\
381	0.00897620422810316\\
382	0.00896457528579671\\
383	0.00895274162548049\\
384	0.00894069887176215\\
385	0.00892844249664058\\
386	0.00891596780934495\\
387	0.00890326994527436\\
388	0.008890343853981\\
389	0.00887718428614845\\
390	0.00886378577953019\\
391	0.00885014264383539\\
392	0.00883624894458003\\
393	0.00882209848596725\\
394	0.00880768479292399\\
395	0.0087930010925074\\
396	0.00877804029500976\\
397	0.00876279497523994\\
398	0.00874725735464402\\
399	0.00873141928513815\\
400	0.0087152722357321\\
401	0.00869880728314632\\
402	0.00868201510755678\\
403	0.00866488599434183\\
404	0.00864740984332204\\
405	0.00862957619448917\\
406	0.008611374305736\\
407	0.00859279306337955\\
408	0.00857382229199742\\
409	0.00855445255551324\\
410	0.00853467508333412\\
411	0.00851448206690299\\
412	0.00849386721740041\\
413	0.00847282553754913\\
414	0.00845135629138582\\
415	0.00842946837858887\\
416	0.00840717303111885\\
417	0.00838446855188986\\
418	0.00836134607464309\\
419	0.00833779644702568\\
420	0.00831381020081693\\
421	0.00828937751239539\\
422	0.00826448815811504\\
423	0.00823913146378406\\
424	0.00821329646866346\\
425	0.00818697205617378\\
426	0.00816014675232015\\
427	0.00813280867112936\\
428	0.00810494548958678\\
429	0.00807654442026407\\
430	0.00804759218136667\\
431	0.00801807496389312\\
432	0.00798797839555728\\
433	0.00795728750107564\\
434	0.00792598665836684\\
435	0.00789405955014713\\
436	0.00786148911033843\\
437	0.0078282574646376\\
438	0.00779434586455267\\
439	0.0077597346142501\\
440	0.00772440298984031\\
441	0.00768832915167115\\
442	0.00765149005284808\\
443	0.00761386135416194\\
444	0.00757541737269823\\
445	0.00753613112674139\\
446	0.00749597454924612\\
447	0.00745491813766635\\
448	0.00741293710877575\\
449	0.00736999563629485\\
450	0.00732605222905261\\
451	0.00728105983545202\\
452	0.00723496581642318\\
453	0.00718771099863757\\
454	0.00713922838691379\\
455	0.00708944162447838\\
456	0.00703826340693689\\
457	0.00698559495438312\\
458	0.00693133047916984\\
459	0.00690529055976151\\
460	0.0068841207630141\\
461	0.00686208024413438\\
462	0.00683905710526833\\
463	0.0068149054819794\\
464	0.0067894708922581\\
465	0.00676258301758642\\
466	0.00673403986817408\\
467	0.00670360081697532\\
468	0.00667097525108228\\
469	0.00663582092278856\\
470	0.00659773011524339\\
471	0.00655629509722116\\
472	0.00651421305901012\\
473	0.00647147989328616\\
474	0.00642809231747627\\
475	0.00638404796618308\\
476	0.00633934549180825\\
477	0.00629398469173378\\
478	0.00624796680193247\\
479	0.00620129537540108\\
480	0.0061539799181788\\
481	0.00610605450934546\\
482	0.00605756163724939\\
483	0.00600852154552289\\
484	0.00595895779064099\\
485	0.00590889731784291\\
486	0.00585836996046816\\
487	0.00580740626716363\\
488	0.00575603018181687\\
489	0.00570421180804988\\
490	0.00565187157863738\\
491	0.00559901569161241\\
492	0.00554570427623523\\
493	0.00549201187304566\\
494	0.00543803029105876\\
495	0.00538387254887218\\
496	0.00532967764954309\\
497	0.00527561646363064\\
498	0.00522189901646822\\
499	0.00516878428684203\\
500	0.00511659104976579\\
501	0.00506571200740636\\
502	0.00501663098046658\\
503	0.00496994308152165\\
504	0.0049263810243382\\
505	0.00488684674767849\\
506	0.00484726967063035\\
507	0.00480753330695101\\
508	0.00476753172025539\\
509	0.00472663807761456\\
510	0.00468482920714398\\
511	0.00464208086863117\\
512	0.00459836793622333\\
513	0.00455366403367335\\
514	0.0045079407329001\\
515	0.00446116550987426\\
516	0.00441331715078829\\
517	0.00436437751689232\\
518	0.00431432977304112\\
519	0.00426315727662655\\
520	0.0042108415062387\\
521	0.00415735964308127\\
522	0.0041026801451013\\
523	0.00404675126036695\\
524	0.00398951582042739\\
525	0.00393091740233218\\
526	0.00387090184799251\\
527	0.00380941952584333\\
528	0.00374642829474921\\
529	0.00368189691490911\\
530	0.00361580972351305\\
531	0.00354817300613794\\
532	0.00347928394531365\\
533	0.00340998619139048\\
534	0.00334181294001288\\
535	0.00327495586372995\\
536	0.00320961667718388\\
537	0.00314600321252046\\
538	0.00308432322384668\\
539	0.00302477504282014\\
540	0.00296753390652657\\
541	0.00291275062540001\\
542	0.00286052641625685\\
543	0.00281087172974648\\
544	0.00276366019386304\\
545	0.00271737865858413\\
546	0.00267149001838159\\
547	0.00262597823294146\\
548	0.00258081087012544\\
549	0.0025359370152983\\
550	0.00249128575749629\\
551	0.00244676577299991\\
552	0.00240226677489789\\
553	0.00235766189430851\\
554	0.00231280486886916\\
555	0.0022675368226262\\
556	0.00222170449377727\\
557	0.00217528619547\\
558	0.00212825770808596\\
559	0.00208059251779431\\
560	0.00203226220776163\\
561	0.00198323669429928\\
562	0.00193348516259038\\
563	0.00188297732214686\\
564	0.00183168498165631\\
565	0.00177958389218695\\
566	0.00172665571247692\\
567	0.00167288999768657\\
568	0.00161828474516008\\
569	0.00156284015785779\\
570	0.00150655873420948\\
571	0.00144944562230146\\
572	0.00139150887414687\\
573	0.00133275960045463\\
574	0.001273212338925\\
575	0.00121288536990015\\
576	0.00115180094160388\\
577	0.00108998546890569\\
578	0.00102746988043389\\
579	0.000964291042797229\\
580	0.000900493476418601\\
581	0.000836121682920783\\
582	0.000771207498915577\\
583	0.000705846288818108\\
584	0.000640030929677252\\
585	0.000575449812676461\\
586	0.000512984914594378\\
587	0.000453002036723499\\
588	0.000396493011046006\\
589	0.000344400514679134\\
590	0.000296031958987211\\
591	0.000250550941554444\\
592	0.000207406150564994\\
593	0.000166007309561553\\
594	0.000126517918855203\\
595	8.88161203105205e-05\\
596	5.34134895574398e-05\\
597	2.21100055488407e-05\\
598	0\\
599	0\\
600	0\\
};
\addplot [color=blue,solid,forget plot]
  table[row sep=crcr]{%
1	0.00994728939738958\\
2	0.0099472887672835\\
3	0.00994728812639287\\
4	0.00994728747453268\\
5	0.00994728681151475\\
6	0.00994728613714766\\
7	0.00994728545123673\\
8	0.00994728475358395\\
9	0.0099472840439879\\
10	0.00994728332224372\\
11	0.00994728258814306\\
12	0.00994728184147398\\
13	0.00994728108202093\\
14	0.00994728030956467\\
15	0.00994727952388218\\
16	0.00994727872474666\\
17	0.00994727791192743\\
18	0.00994727708518983\\
19	0.00994727624429521\\
20	0.00994727538900085\\
21	0.00994727451905984\\
22	0.0099472736342211\\
23	0.0099472727342292\\
24	0.00994727181882439\\
25	0.00994727088774246\\
26	0.00994726994071468\\
27	0.00994726897746774\\
28	0.00994726799772364\\
29	0.00994726700119965\\
30	0.00994726598760819\\
31	0.00994726495665679\\
32	0.00994726390804796\\
33	0.00994726284147914\\
34	0.00994726175664261\\
35	0.00994726065322539\\
36	0.00994725953090913\\
37	0.0099472583893701\\
38	0.00994725722827899\\
39	0.0099472560473009\\
40	0.0099472548460952\\
41	0.00994725362431546\\
42	0.00994725238160932\\
43	0.00994725111761844\\
44	0.00994724983197833\\
45	0.0099472485243183\\
46	0.00994724719426134\\
47	0.00994724584142402\\
48	0.00994724446541636\\
49	0.00994724306584174\\
50	0.00994724164229676\\
51	0.00994724019437117\\
52	0.00994723872164771\\
53	0.00994723722370203\\
54	0.00994723570010253\\
55	0.00994723415041029\\
56	0.00994723257417887\\
57	0.00994723097095426\\
58	0.00994722934027473\\
59	0.00994722768167065\\
60	0.00994722599466443\\
61	0.00994722427877035\\
62	0.0099472225334944\\
63	0.0099472207583342\\
64	0.00994721895277881\\
65	0.00994721711630858\\
66	0.00994721524839507\\
67	0.00994721334850081\\
68	0.00994721141607922\\
69	0.00994720945057442\\
70	0.0099472074514211\\
71	0.00994720541804432\\
72	0.0099472033498594\\
73	0.00994720124627171\\
74	0.00994719910667653\\
75	0.00994719693045889\\
76	0.00994719471699334\\
77	0.00994719246564387\\
78	0.00994719017576362\\
79	0.00994718784669481\\
80	0.00994718547776846\\
81	0.00994718306830426\\
82	0.00994718061761038\\
83	0.00994717812498324\\
84	0.00994717558970735\\
85	0.00994717301105508\\
86	0.00994717038828649\\
87	0.00994716772064909\\
88	0.00994716500737764\\
89	0.00994716224769397\\
90	0.00994715944080671\\
91	0.0099471565859111\\
92	0.00994715368218877\\
93	0.00994715072880748\\
94	0.00994714772492094\\
95	0.00994714466966854\\
96	0.00994714156217508\\
97	0.00994713840155061\\
98	0.00994713518689011\\
99	0.00994713191727327\\
100	0.00994712859176422\\
101	0.00994712520941129\\
102	0.00994712176924673\\
103	0.00994711827028644\\
104	0.00994711471152971\\
105	0.00994711109195893\\
106	0.0099471074105393\\
107	0.00994710366621859\\
108	0.00994709985792678\\
109	0.00994709598457581\\
110	0.00994709204505926\\
111	0.00994708803825206\\
112	0.00994708396301017\\
113	0.00994707981817023\\
114	0.0099470756025493\\
115	0.0099470713149445\\
116	0.00994706695413266\\
117	0.00994706251887001\\
118	0.00994705800789183\\
119	0.0099470534199121\\
120	0.00994704875362312\\
121	0.0099470440076952\\
122	0.00994703918077621\\
123	0.00994703427149133\\
124	0.00994702927844254\\
125	0.00994702420020832\\
126	0.00994701903534321\\
127	0.00994701378237747\\
128	0.00994700843981659\\
129	0.00994700300614096\\
130	0.00994699747980538\\
131	0.0099469918592387\\
132	0.00994698614284335\\
133	0.00994698032899489\\
134	0.00994697441604159\\
135	0.00994696840230396\\
136	0.00994696228607432\\
137	0.00994695606561627\\
138	0.00994694973916428\\
139	0.00994694330492319\\
140	0.00994693676106768\\
141	0.00994693010574182\\
142	0.00994692333705845\\
143	0.00994691645309862\\
144	0.00994690945191107\\
145	0.00994690233151202\\
146	0.00994689508988443\\
147	0.00994688772497744\\
148	0.00994688023470578\\
149	0.00994687261694925\\
150	0.00994686486955207\\
151	0.00994685699032234\\
152	0.00994684897703144\\
153	0.00994684082741339\\
154	0.00994683253916425\\
155	0.0099468241099415\\
156	0.00994681553736337\\
157	0.00994680681900821\\
158	0.00994679795241383\\
159	0.00994678893507682\\
160	0.00994677976445188\\
161	0.00994677043795109\\
162	0.00994676095294323\\
163	0.0099467513067531\\
164	0.00994674149666068\\
165	0.00994673151990052\\
166	0.00994672137366089\\
167	0.00994671105508305\\
168	0.00994670056126047\\
169	0.00994668988923804\\
170	0.00994667903601124\\
171	0.00994666799852535\\
172	0.00994665677367461\\
173	0.00994664535830136\\
174	0.00994663374919521\\
175	0.00994662194309211\\
176	0.00994660993667353\\
177	0.00994659772656549\\
178	0.0099465853093377\\
179	0.00994657268150257\\
180	0.0099465598395143\\
181	0.0099465467797679\\
182	0.00994653349859821\\
183	0.00994651999227888\\
184	0.00994650625702139\\
185	0.009946492288974\\
186	0.00994647808422069\\
187	0.00994646363878009\\
188	0.00994644894860445\\
189	0.00994643400957845\\
190	0.00994641881751813\\
191	0.00994640336816973\\
192	0.00994638765720855\\
193	0.00994637168023776\\
194	0.00994635543278717\\
195	0.00994633891031205\\
196	0.00994632210819187\\
197	0.00994630502172903\\
198	0.00994628764614757\\
199	0.00994626997659191\\
200	0.00994625200812545\\
201	0.00994623373572928\\
202	0.00994621515430077\\
203	0.00994619625865216\\
204	0.00994617704350918\\
205	0.00994615750350956\\
206	0.0099461376332016\\
207	0.00994611742704261\\
208	0.00994609687939744\\
209	0.00994607598453692\\
210	0.00994605473663625\\
211	0.00994603312977342\\
212	0.00994601115792758\\
213	0.00994598881497736\\
214	0.00994596609469918\\
215	0.00994594299076553\\
216	0.00994591949674325\\
217	0.00994589560609167\\
218	0.00994587131216087\\
219	0.00994584660818978\\
220	0.00994582148730435\\
221	0.00994579594251557\\
222	0.00994576996671757\\
223	0.00994574355268562\\
224	0.0099457166930741\\
225	0.00994568938041447\\
226	0.00994566160711314\\
227	0.00994563336544935\\
228	0.009945604647573\\
229	0.00994557544550243\\
230	0.00994554575112222\\
231	0.00994551555618077\\
232	0.00994548485228811\\
233	0.00994545363091341\\
234	0.0099454218833826\\
235	0.0099453896008759\\
236	0.00994535677442528\\
237	0.00994532339491192\\
238	0.00994528945306356\\
239	0.00994525493945187\\
240	0.00994521984448973\\
241	0.00994518415842843\\
242	0.00994514787135489\\
243	0.00994511097318873\\
244	0.0099450734536794\\
245	0.00994503530240314\\
246	0.00994499650875996\\
247	0.00994495706197049\\
248	0.00994491695107285\\
249	0.00994487616491938\\
250	0.00994483469217337\\
251	0.00994479252130564\\
252	0.00994474964059116\\
253	0.00994470603810546\\
254	0.00994466170172113\\
255	0.0099446166191041\\
256	0.00994457077770993\\
257	0.00994452416477997\\
258	0.00994447676733747\\
259	0.00994442857218362\\
260	0.0099443795658934\\
261	0.00994432973481152\\
262	0.00994427906504804\\
263	0.00994422754247412\\
264	0.00994417515271752\\
265	0.00994412188115801\\
266	0.00994406771292274\\
267	0.00994401263288139\\
268	0.00994395662564126\\
269	0.0099438996755421\\
270	0.0099438417666508\\
271	0.00994378288275576\\
272	0.00994372300736148\\
273	0.00994366212368418\\
274	0.00994360021465019\\
275	0.00994353726289344\\
276	0.00994347325073415\\
277	0.00994340816017812\\
278	0.00994334197291253\\
279	0.00994327467029953\\
280	0.00994320623336957\\
281	0.00994313664281463\\
282	0.00994306587898105\\
283	0.00994299392186234\\
284	0.00994292075109171\\
285	0.00994284634593433\\
286	0.00994277068527944\\
287	0.00994269374763214\\
288	0.00994261551110497\\
289	0.00994253595340922\\
290	0.00994245505184591\\
291	0.00994237278329653\\
292	0.00994228912421344\\
293	0.00994220405060995\\
294	0.00994211753805004\\
295	0.00994202956163778\\
296	0.00994194009600626\\
297	0.00994184911530623\\
298	0.00994175659319425\\
299	0.00994166250282041\\
300	0.00994156681681557\\
301	0.00994146950727814\\
302	0.00994137054576027\\
303	0.00994126990325364\\
304	0.00994116755017456\\
305	0.00994106345634881\\
306	0.00994095759099596\\
307	0.00994084992271344\\
308	0.00994074041946005\\
309	0.00994062904853703\\
310	0.00994051577656212\\
311	0.0099404005694313\\
312	0.00994028339228341\\
313	0.00994016420953451\\
314	0.0099400429848504\\
315	0.00993991968110448\\
316	0.00993979426034843\\
317	0.00993966668378385\\
318	0.0099395369117322\\
319	0.00993940490360296\\
320	0.00993927061785956\\
321	0.00993913401198321\\
322	0.00993899504243423\\
323	0.00993885366461058\\
324	0.00993870983280351\\
325	0.00993856350014987\\
326	0.00993841461858081\\
327	0.00993826313876641\\
328	0.00993810901005589\\
329	0.00993795218041293\\
330	0.0099377925963454\\
331	0.00993763020282909\\
332	0.00993746494322466\\
333	0.009937296759187\\
334	0.00993712559056614\\
335	0.0099369513752989\\
336	0.00993677404928993\\
337	0.00993659354628105\\
338	0.00993640979770759\\
339	0.00993622273253997\\
340	0.00993603227710894\\
341	0.00993583835491259\\
342	0.00993564088640272\\
343	0.00993543978874843\\
344	0.00993523497557408\\
345	0.00993502635666861\\
346	0.00993481383766298\\
347	0.00993459731967199\\
348	0.00993437669889627\\
349	0.00993415186617975\\
350	0.00993392270651697\\
351	0.0099336890985035\\
352	0.00993345091372144\\
353	0.00993320801605457\\
354	0.0099329602609451\\
355	0.00993270749465525\\
356	0.00993244955361171\\
357	0.00993218626336552\\
358	0.00993191743696515\\
359	0.00993164287405719\\
360	0.00993136235969612\\
361	0.00993107566302746\\
362	0.00993078253599338\\
363	0.00993048271233565\\
364	0.00993017590777067\\
365	0.00992986182381064\\
366	0.00992954015840349\\
367	0.00992921057257196\\
368	0.00992887161019262\\
369	0.00992810194969619\\
370	0.00992728902741637\\
371	0.0099264631145681\\
372	0.00992562400856727\\
373	0.00992477150386479\\
374	0.0099239053917256\\
375	0.00992302545946369\\
376	0.00992213148860753\\
377	0.00992122325503199\\
378	0.00992030055431135\\
379	0.00991936317258418\\
380	0.00991841087754756\\
381	0.00991744342998338\\
382	0.00991646058560671\\
383	0.00991546209471297\\
384	0.00991444770177994\\
385	0.00991341714501988\\
386	0.00991237015587637\\
387	0.00991130645846048\\
388	0.00991022576892032\\
389	0.0099091277947384\\
390	0.00990801223395065\\
391	0.00990687877428166\\
392	0.00990572709219101\\
393	0.0099045568518263\\
394	0.00990336770388\\
395	0.00990215928434925\\
396	0.00990093121320017\\
397	0.0098996830929426\\
398	0.00989841450712726\\
399	0.00989712501879033\\
400	0.00989581416890412\\
401	0.00989448147499027\\
402	0.0098931264303108\\
403	0.00989174850451013\\
404	0.00989034714578065\\
405	0.00988892177088828\\
406	0.00988747166429707\\
407	0.00988599614891656\\
408	0.00988449457131745\\
409	0.00988296627242109\\
410	0.00988141060728983\\
411	0.00987982697629938\\
412	0.00987821450495692\\
413	0.00987657213037055\\
414	0.00987489960185651\\
415	0.00987319675954955\\
416	0.00987146305496015\\
417	0.00986969776315128\\
418	0.00986790013262936\\
419	0.00986606938492638\\
420	0.00986420471345122\\
421	0.00986230527599937\\
422	0.00986037017056468\\
423	0.00985839844667363\\
424	0.00985638913465386\\
425	0.00985434122410222\\
426	0.00985225365688332\\
427	0.00985012532347765\\
428	0.00984795505893773\\
429	0.0098457416383978\\
430	0.00984348377207433\\
431	0.00984118009968394\\
432	0.00983882918419298\\
433	0.00983642950479868\\
434	0.00983397944902411\\
435	0.00983147730378909\\
436	0.00982892124529488\\
437	0.00982630932753172\\
438	0.00982363946918497\\
439	0.00982090943867187\\
440	0.00981811683697266\\
441	0.00981525907773534\\
442	0.0098123333634594\\
443	0.00980933665384108\\
444	0.00980626561203159\\
445	0.00980311648509642\\
446	0.00979988490903435\\
447	0.00979656751367569\\
448	0.00979315984879927\\
449	0.00978965665875299\\
450	0.00978605196421807\\
451	0.00978233904989328\\
452	0.009778510358334\\
453	0.00977455735752918\\
454	0.0097704703965325\\
455	0.00976623854629632\\
456	0.00976184932712826\\
457	0.00975728762879463\\
458	0.00975253094102019\\
459	0.00971839771449009\\
460	0.00967783551590882\\
461	0.00963638307702326\\
462	0.00959401571846844\\
463	0.00955071116987698\\
464	0.009506449746934\\
465	0.00946121356433595\\
466	0.00941498989763655\\
467	0.00936777382588069\\
468	0.00931982649739658\\
469	0.00927097724594474\\
470	0.00922111249492722\\
471	0.00917024171997016\\
472	0.00911836431980632\\
473	0.00906544098017759\\
474	0.00901142986803239\\
475	0.00895628650202312\\
476	0.0088999636513926\\
477	0.00884241151725245\\
478	0.00878357579389252\\
479	0.00872339937401684\\
480	0.00866181904152994\\
481	0.0085987855990807\\
482	0.008534244330055\\
483	0.00846813739739182\\
484	0.008400404229598\\
485	0.00833098180268491\\
486	0.00825980565040134\\
487	0.00818681356724348\\
488	0.00811196041225784\\
489	0.00803757241513387\\
490	0.00796229125054936\\
491	0.00788495233018119\\
492	0.0078054261504511\\
493	0.00772356675223783\\
494	0.00763920884209486\\
495	0.00755216429112926\\
496	0.00746221784295183\\
497	0.00736912176415375\\
498	0.00727258914291985\\
499	0.00717228510848414\\
500	0.00706781684756801\\
501	0.0069587135245635\\
502	0.00684442693088627\\
503	0.00672436104868497\\
504	0.00659781673954426\\
505	0.00646397029198241\\
506	0.00632679729735914\\
507	0.00618639402032289\\
508	0.00606171878598091\\
509	0.00599096059071272\\
510	0.00591912655344071\\
511	0.00584623680915749\\
512	0.00577231931384684\\
513	0.00569741081960089\\
514	0.00562155638887639\\
515	0.00554480421508631\\
516	0.0054671773277444\\
517	0.00538855941456101\\
518	0.00530897412214534\\
519	0.00522850802198831\\
520	0.00514727042913968\\
521	0.00506539966618615\\
522	0.00498306956464646\\
523	0.00490049748841788\\
524	0.00481795446526811\\
525	0.00473577534449757\\
526	0.0046543761075988\\
527	0.00457426204420224\\
528	0.00449604712090942\\
529	0.00442050433839845\\
530	0.00434859482926174\\
531	0.00428150109757077\\
532	0.00421321273900356\\
533	0.00414348913715137\\
534	0.00407226318348304\\
535	0.0039995517841578\\
536	0.00392540198290194\\
537	0.00384990165189182\\
538	0.00377319371463011\\
539	0.00369549476055221\\
540	0.00361711914268737\\
541	0.00353742125391003\\
542	0.00345601782437688\\
543	0.00337319025389791\\
544	0.003289365302518\\
545	0.00320628170397525\\
546	0.00312468842652381\\
547	0.00304480981507298\\
548	0.00296686662900341\\
549	0.00289106424411822\\
550	0.00281757556151098\\
551	0.00274651671417479\\
552	0.00267791305917642\\
553	0.00261171652587721\\
554	0.00254802634911223\\
555	0.00248683843879565\\
556	0.0024279600872744\\
557	0.00236932352608795\\
558	0.00231094340623584\\
559	0.00225282286035417\\
560	0.00219495151987034\\
561	0.00213732035400731\\
562	0.00207989833733159\\
563	0.00202262868797722\\
564	0.00196542589818876\\
565	0.00190817449924115\\
566	0.00185073167445929\\
567	0.00179292770605968\\
568	0.00173462363221089\\
569	0.00167581051444986\\
570	0.00161648382026782\\
571	0.00155664018532771\\
572	0.00149628178551331\\
573	0.00143541952223136\\
574	0.00137406514768928\\
575	0.00131223203833818\\
576	0.00124993644289524\\
577	0.00118719902170031\\
578	0.00112404659450915\\
579	0.00106051390222132\\
580	0.000996645105392352\\
581	0.000932492781884864\\
582	0.000868111167276619\\
583	0.000803555608238753\\
584	0.000738882285574681\\
585	0.000674149269925966\\
586	0.000609397230982037\\
587	0.000544655405819526\\
588	0.000479944735464092\\
589	0.000415300425128081\\
590	0.000352022048369845\\
591	0.000291263299764878\\
592	0.000233865201392129\\
593	0.000180687335852113\\
594	0.000132203256552347\\
595	9.01789376102564e-05\\
596	5.34134895574398e-05\\
597	2.21100055488407e-05\\
598	0\\
599	0\\
600	0\\
};
\addplot [color=mycolor10,solid,forget plot]
  table[row sep=crcr]{%
1	0.00997061354577094\\
2	0.00997061352463255\\
3	0.00997061350313236\\
4	0.00997061348126415\\
5	0.00997061345902163\\
6	0.00997061343639837\\
7	0.00997061341338782\\
8	0.00997061338998336\\
9	0.00997061336617822\\
10	0.00997061334196553\\
11	0.00997061331733829\\
12	0.00997061329228941\\
13	0.00997061326681163\\
14	0.00997061324089762\\
15	0.00997061321453988\\
16	0.0099706131877308\\
17	0.00997061316046265\\
18	0.00997061313272754\\
19	0.00997061310451748\\
20	0.00997061307582432\\
21	0.00997061304663977\\
22	0.00997061301695541\\
23	0.00997061298676266\\
24	0.00997061295605282\\
25	0.00997061292481701\\
26	0.00997061289304622\\
27	0.00997061286073127\\
28	0.00997061282786284\\
29	0.00997061279443144\\
30	0.00997061276042742\\
31	0.00997061272584096\\
32	0.00997061269066209\\
33	0.00997061265488064\\
34	0.0099706126184863\\
35	0.00997061258146856\\
36	0.00997061254381673\\
37	0.00997061250551995\\
38	0.00997061246656718\\
39	0.00997061242694716\\
40	0.00997061238664848\\
41	0.0099706123456595\\
42	0.0099706123039684\\
43	0.00997061226156315\\
44	0.00997061221843151\\
45	0.00997061217456107\\
46	0.00997061212993914\\
47	0.00997061208455288\\
48	0.00997061203838918\\
49	0.00997061199143474\\
50	0.00997061194367601\\
51	0.00997061189509924\\
52	0.00997061184569041\\
53	0.00997061179543529\\
54	0.00997061174431937\\
55	0.00997061169232795\\
56	0.00997061163944602\\
57	0.00997061158565835\\
58	0.00997061153094944\\
59	0.00997061147530353\\
60	0.00997061141870458\\
61	0.0099706113611363\\
62	0.0099706113025821\\
63	0.00997061124302512\\
64	0.0099706111824482\\
65	0.0099706111208339\\
66	0.00997061105816449\\
67	0.00997061099442191\\
68	0.00997061092958783\\
69	0.00997061086364357\\
70	0.00997061079657017\\
71	0.00997061072834832\\
72	0.00997061065895839\\
73	0.00997061058838042\\
74	0.00997061051659411\\
75	0.0099706104435788\\
76	0.00997061036931351\\
77	0.00997061029377688\\
78	0.00997061021694718\\
79	0.00997061013880234\\
80	0.00997061005931989\\
81	0.00997060997847699\\
82	0.00997060989625042\\
83	0.00997060981261653\\
84	0.00997060972755132\\
85	0.00997060964103035\\
86	0.00997060955302877\\
87	0.0099706094635213\\
88	0.00997060937248227\\
89	0.00997060927988552\\
90	0.00997060918570449\\
91	0.00997060908991215\\
92	0.00997060899248102\\
93	0.00997060889338314\\
94	0.00997060879259009\\
95	0.00997060869007298\\
96	0.00997060858580239\\
97	0.00997060847974846\\
98	0.00997060837188076\\
99	0.0099706082621684\\
100	0.00997060815057993\\
101	0.00997060803708339\\
102	0.00997060792164628\\
103	0.00997060780423553\\
104	0.00997060768481753\\
105	0.0099706075633581\\
106	0.00997060743982246\\
107	0.00997060731417527\\
108	0.00997060718638058\\
109	0.00997060705640184\\
110	0.00997060692420186\\
111	0.00997060678974285\\
112	0.00997060665298638\\
113	0.00997060651389334\\
114	0.009970606372424\\
115	0.00997060622853792\\
116	0.00997060608219402\\
117	0.00997060593335048\\
118	0.00997060578196481\\
119	0.00997060562799379\\
120	0.00997060547139348\\
121	0.00997060531211917\\
122	0.00997060515012543\\
123	0.00997060498536604\\
124	0.00997060481779402\\
125	0.00997060464736158\\
126	0.00997060447402013\\
127	0.00997060429772027\\
128	0.00997060411841176\\
129	0.00997060393604351\\
130	0.00997060375056356\\
131	0.00997060356191911\\
132	0.00997060337005643\\
133	0.0099706031749209\\
134	0.00997060297645698\\
135	0.00997060277460821\\
136	0.00997060256931717\\
137	0.00997060236052546\\
138	0.00997060214817371\\
139	0.00997060193220158\\
140	0.00997060171254768\\
141	0.00997060148914957\\
142	0.00997060126194376\\
143	0.00997060103086568\\
144	0.00997060079584969\\
145	0.00997060055682905\\
146	0.00997060031373591\\
147	0.00997060006650126\\
148	0.00997059981505495\\
149	0.00997059955932563\\
150	0.00997059929924076\\
151	0.00997059903472658\\
152	0.00997059876570809\\
153	0.00997059849210903\\
154	0.00997059821385185\\
155	0.00997059793085772\\
156	0.00997059764304644\\
157	0.00997059735033651\\
158	0.00997059705264504\\
159	0.00997059674988774\\
160	0.0099705964419789\\
161	0.00997059612883142\\
162	0.00997059581035666\\
163	0.00997059548646454\\
164	0.00997059515706347\\
165	0.00997059482206029\\
166	0.00997059448136031\\
167	0.00997059413486723\\
168	0.00997059378248313\\
169	0.00997059342410845\\
170	0.00997059305964198\\
171	0.00997059268898078\\
172	0.0099705923120202\\
173	0.00997059192865382\\
174	0.00997059153877344\\
175	0.00997059114226905\\
176	0.0099705907390288\\
177	0.00997059032893893\\
178	0.00997058991188381\\
179	0.00997058948774585\\
180	0.00997058905640547\\
181	0.00997058861774113\\
182	0.00997058817162919\\
183	0.009970587717944\\
184	0.00997058725655774\\
185	0.00997058678734048\\
186	0.00997058631016011\\
187	0.00997058582488228\\
188	0.00997058533137041\\
189	0.00997058482948562\\
190	0.00997058431908669\\
191	0.00997058380003003\\
192	0.00997058327216965\\
193	0.0099705827353571\\
194	0.00997058218944143\\
195	0.00997058163426919\\
196	0.00997058106968431\\
197	0.00997058049552813\\
198	0.00997057991163929\\
199	0.00997057931785376\\
200	0.00997057871400473\\
201	0.0099705780999226\\
202	0.00997057747543489\\
203	0.00997057684036625\\
204	0.00997057619453839\\
205	0.00997057553776999\\
206	0.00997057486987669\\
207	0.00997057419067105\\
208	0.00997057349996245\\
209	0.00997057279755708\\
210	0.00997057208325785\\
211	0.00997057135686437\\
212	0.00997057061817286\\
213	0.00997056986697611\\
214	0.00997056910306343\\
215	0.00997056832622056\\
216	0.00997056753622963\\
217	0.0099705667328691\\
218	0.0099705659159137\\
219	0.00997056508513435\\
220	0.00997056424029808\\
221	0.00997056338116801\\
222	0.00997056250750326\\
223	0.00997056161905885\\
224	0.00997056071558568\\
225	0.00997055979683041\\
226	0.00997055886253545\\
227	0.00997055791243879\\
228	0.00997055694627403\\
229	0.00997055596377021\\
230	0.00997055496465181\\
231	0.0099705539486386\\
232	0.00997055291544561\\
233	0.009970551864783\\
234	0.00997055079635602\\
235	0.00997054970986491\\
236	0.00997054860500477\\
237	0.00997054748146552\\
238	0.0099705463389318\\
239	0.00997054517708285\\
240	0.00997054399559243\\
241	0.00997054279412871\\
242	0.00997054157235421\\
243	0.00997054032992564\\
244	0.00997053906649382\\
245	0.0099705377817036\\
246	0.0099705364751937\\
247	0.00997053514659665\\
248	0.00997053379553863\\
249	0.00997053242163939\\
250	0.00997053102451211\\
251	0.0099705296037633\\
252	0.00997052815899263\\
253	0.00997052668979288\\
254	0.00997052519574974\\
255	0.0099705236764417\\
256	0.00997052213143995\\
257	0.00997052056030817\\
258	0.00997051896260246\\
259	0.00997051733787115\\
260	0.00997051568565469\\
261	0.00997051400548545\\
262	0.0099705122968876\\
263	0.00997051055937694\\
264	0.00997050879246074\\
265	0.00997050699563754\\
266	0.00997050516839702\\
267	0.00997050331021974\\
268	0.00997050142057697\\
269	0.00997049949893044\\
270	0.00997049754473208\\
271	0.00997049555742399\\
272	0.00997049353643848\\
273	0.00997049148119846\\
274	0.00997048939111724\\
275	0.00997048726559648\\
276	0.00997048510402675\\
277	0.00997048290578748\\
278	0.0099704806702468\\
279	0.00997047839676128\\
280	0.00997047608467566\\
281	0.00997047373332259\\
282	0.00997047134202239\\
283	0.00997046891008274\\
284	0.0099704664367984\\
285	0.00997046392145093\\
286	0.00997046136330836\\
287	0.00997045876162489\\
288	0.00997045611564053\\
289	0.00997045342458081\\
290	0.00997045068765636\\
291	0.00997044790406261\\
292	0.00997044507297935\\
293	0.00997044219357035\\
294	0.00997043926498294\\
295	0.00997043628634762\\
296	0.00997043325677753\\
297	0.00997043017536806\\
298	0.00997042704119628\\
299	0.0099704238533205\\
300	0.00997042061077969\\
301	0.00997041731259295\\
302	0.00997041395775895\\
303	0.00997041054525537\\
304	0.00997040707403837\\
305	0.00997040354304208\\
306	0.00997039995117815\\
307	0.00997039629733503\\
308	0.00997039258037674\\
309	0.00997038879914037\\
310	0.00997038495243257\\
311	0.00997038103902861\\
312	0.00997037705767839\\
313	0.00997037300710451\\
314	0.00997036888599895\\
315	0.00997036469302148\\
316	0.00997036042679827\\
317	0.00997035608592045\\
318	0.00997035166894253\\
319	0.00997034717438066\\
320	0.00997034260071082\\
321	0.00997033794636685\\
322	0.00997033320973826\\
323	0.00997032838916791\\
324	0.00997032348294946\\
325	0.00997031848932463\\
326	0.00997031340648017\\
327	0.00997030823254455\\
328	0.00997030296558441\\
329	0.00997029760360056\\
330	0.00997029214452369\\
331	0.00997028658620959\\
332	0.00997028092643395\\
333	0.00997027516288656\\
334	0.009970269293165\\
335	0.00997026331476755\\
336	0.00997025722508549\\
337	0.00997025102139443\\
338	0.00997024470084481\\
339	0.00997023826045129\\
340	0.00997023169708096\\
341	0.00997022500744026\\
342	0.00997021818806036\\
343	0.0099702112352809\\
344	0.00997020414523177\\
345	0.00997019691381281\\
346	0.00997018953667097\\
347	0.00997018200917466\\
348	0.00997017432638478\\
349	0.00997016648302173\\
350	0.00997015847342776\\
351	0.00997015029152418\\
352	0.00997014193076433\\
353	0.0099701333840868\\
354	0.00997012464387614\\
355	0.00997011570192416\\
356	0.00997010654933114\\
357	0.00997009717635317\\
358	0.00997008757231958\\
359	0.00997007772546795\\
360	0.00997006762263018\\
361	0.00997005724853111\\
362	0.0099700465839626\\
363	0.00997003560059562\\
364	0.00997002424554486\\
365	0.00997001239440149\\
366	0.00996999970837929\\
367	0.00996998522494726\\
368	0.00996996657780003\\
369	0.00996993646474205\\
370	0.00996990507936311\\
371	0.00996987319490772\\
372	0.0099698408035988\\
373	0.00996980789731911\\
374	0.00996977446739821\\
375	0.00996974050453822\\
376	0.00996970599978222\\
377	0.0099696709475187\\
378	0.0099696353414128\\
379	0.00996959917324234\\
380	0.00996956243436619\\
381	0.00996952511598974\\
382	0.00996948720915478\\
383	0.00996944870472804\\
384	0.00996940959338796\\
385	0.00996936986560978\\
386	0.00996932951164845\\
387	0.00996928852151937\\
388	0.00996924688497654\\
389	0.00996920459148802\\
390	0.00996916163020821\\
391	0.00996911798994705\\
392	0.00996907365913566\\
393	0.00996902862578872\\
394	0.00996898287746428\\
395	0.00996893640122287\\
396	0.00996888918359111\\
397	0.00996884121054134\\
398	0.00996879246751142\\
399	0.00996874293951026\\
400	0.00996869261137801\\
401	0.00996864146824745\\
402	0.00996858949600124\\
403	0.009968536680564\\
404	0.00996848300283412\\
405	0.00996842842818678\\
406	0.0099683729291681\\
407	0.00996831648483095\\
408	0.00996825907823155\\
409	0.00996820069884713\\
410	0.00996814133719446\\
411	0.00996808093013593\\
412	0.00996801938085864\\
413	0.00996795668555509\\
414	0.00996789284271058\\
415	0.00996782782906769\\
416	0.00996776161727656\\
417	0.00996769417936158\\
418	0.00996762548679398\\
419	0.00996755551015033\\
420	0.00996748421766628\\
421	0.00996741157296322\\
422	0.00996733753749477\\
423	0.00996726207391997\\
424	0.00996718514376581\\
425	0.00996710670680078\\
426	0.00996702672090777\\
427	0.00996694514194422\\
428	0.00996686192358747\\
429	0.00996677701716359\\
430	0.00996669037145715\\
431	0.00996660193249929\\
432	0.00996651164333072\\
433	0.00996641944373556\\
434	0.00996632526994054\\
435	0.0099662290542713\\
436	0.00996613072475194\\
437	0.00996603020461965\\
438	0.00996592741169297\\
439	0.00996582225745127\\
440	0.00996571464549786\\
441	0.00996560446869608\\
442	0.0099654916036704\\
443	0.0099653759013705\\
444	0.00996525717809596\\
445	0.00996513524235298\\
446	0.00996501006455883\\
447	0.00996488153408937\\
448	0.00996474948263035\\
449	0.00996461370973543\\
450	0.00996447399127729\\
451	0.00996433007723816\\
452	0.00996418168428262\\
453	0.00996402847544085\\
454	0.00996386999537104\\
455	0.00996370546495979\\
456	0.00996353318254629\\
457	0.00996334905208016\\
458	0.00996314400097686\\
459	0.00996221523625862\\
460	0.00996113519056949\\
461	0.00996004041913434\\
462	0.00995893062995004\\
463	0.00995780564366182\\
464	0.00995666508847368\\
465	0.00995550896793464\\
466	0.00995433689561372\\
467	0.00995314343639058\\
468	0.0099516644700794\\
469	0.00995005647906279\\
470	0.00994842701693517\\
471	0.00994677631784331\\
472	0.00994510343433459\\
473	0.00994340732390194\\
474	0.00994168683820687\\
475	0.0099399407083655\\
476	0.0099381675040042\\
477	0.00993636533629464\\
478	0.00993454594009915\\
479	0.00993269413198386\\
480	0.00993080771174991\\
481	0.00992888502425512\\
482	0.00992692420088466\\
483	0.00992492320323511\\
484	0.00992287979615216\\
485	0.00992079144033754\\
486	0.00991865480706367\\
487	0.00991646356327589\\
488	0.009914197330135\\
489	0.00990953231637944\\
490	0.00990371933406626\\
491	0.00989777880307087\\
492	0.00989170464998114\\
493	0.00988549019416691\\
494	0.00987912805813988\\
495	0.00987261006145199\\
496	0.0098659271028743\\
497	0.00985906907976534\\
498	0.00985202488750755\\
499	0.00984478251775438\\
500	0.00983732599335023\\
501	0.00982963322013123\\
502	0.00982168145822714\\
503	0.00981348801526836\\
504	0.00980504427665915\\
505	0.00979627543354369\\
506	0.00978720324055375\\
507	0.00977780324281051\\
508	0.00974946762103066\\
509	0.00966496048588523\\
510	0.00957847197711245\\
511	0.00948989807925203\\
512	0.00939912432654791\\
513	0.00930602386821295\\
514	0.00921045412905488\\
515	0.00911225690659132\\
516	0.00901124405373172\\
517	0.00890721372196417\\
518	0.0087999486559805\\
519	0.00868920757694372\\
520	0.0085747290257097\\
521	0.0084562062354396\\
522	0.00833329598139413\\
523	0.00820561825086714\\
524	0.00807272727380087\\
525	0.00793418036646053\\
526	0.00778945875051984\\
527	0.00763796023128488\\
528	0.00747977264159258\\
529	0.00731359410387879\\
530	0.00713811110598407\\
531	0.00695215354600507\\
532	0.0067614980714633\\
533	0.00656930143035399\\
534	0.00637632138398905\\
535	0.0061785466664545\\
536	0.00597574754103314\\
537	0.00576767079421171\\
538	0.00555403531562184\\
539	0.00533452942614049\\
540	0.00510882425846582\\
541	0.00495120491619428\\
542	0.00483771765206894\\
543	0.00472283222744187\\
544	0.00460649821460026\\
545	0.00448896402189749\\
546	0.00437052540171803\\
547	0.00425156557335448\\
548	0.00413258864285608\\
549	0.0040142527263529\\
550	0.00389741170323932\\
551	0.00378316824563911\\
552	0.00367294001524462\\
553	0.00356681767329889\\
554	0.00345826409295106\\
555	0.0033476284425787\\
556	0.00323546794452343\\
557	0.00312412750013787\\
558	0.00301391693601337\\
559	0.00290518133434445\\
560	0.00279830352166498\\
561	0.0026937038752877\\
562	0.00259183838832754\\
563	0.00249319226631342\\
564	0.0023982743494106\\
565	0.00230757743610891\\
566	0.00222145668415517\\
567	0.00214031063684839\\
568	0.00206192599598129\\
569	0.00198419548821309\\
570	0.00190703933181228\\
571	0.00183039326121471\\
572	0.00175407930742514\\
573	0.00167786725530715\\
574	0.00160180056386005\\
575	0.00152591978290154\\
576	0.00145025069519184\\
577	0.00137480311354071\\
578	0.00129957149013983\\
579	0.00122453819501638\\
580	0.00114967844732551\\
581	0.00107500225175364\\
582	0.00100064326835064\\
583	0.000926739340948194\\
584	0.000853430677943719\\
585	0.000780857329077455\\
586	0.000709156049726809\\
587	0.000638455314416028\\
588	0.000568870748519688\\
589	0.000500499906204788\\
590	0.000433416517188362\\
591	0.000367650704584345\\
592	0.000303181785186723\\
593	0.000239933768152231\\
594	0.000177768097533602\\
595	0.000116479891379525\\
596	6.11665484105478e-05\\
597	2.21100055488407e-05\\
598	0\\
599	0\\
600	0\\
};
\addplot [color=mycolor11,solid,forget plot]
  table[row sep=crcr]{%
1	0.00999886160658231\\
2	0.00999886160551195\\
3	0.00999886160442327\\
4	0.00999886160331596\\
5	0.00999886160218969\\
6	0.00999886160104414\\
7	0.00999886159987898\\
8	0.00999886159869388\\
9	0.00999886159748848\\
10	0.00999886159626245\\
11	0.00999886159501543\\
12	0.00999886159374705\\
13	0.00999886159245696\\
14	0.00999886159114478\\
15	0.00999886158981012\\
16	0.00999886158845262\\
17	0.00999886158707186\\
18	0.00999886158566747\\
19	0.00999886158423902\\
20	0.00999886158278612\\
21	0.00999886158130832\\
22	0.00999886157980522\\
23	0.00999886157827638\\
24	0.00999886157672136\\
25	0.0099988615751397\\
26	0.00999886157353096\\
27	0.00999886157189466\\
28	0.00999886157023033\\
29	0.0099988615685375\\
30	0.00999886156681567\\
31	0.00999886156506436\\
32	0.00999886156328304\\
33	0.00999886156147122\\
34	0.00999886155962837\\
35	0.00999886155775394\\
36	0.00999886155584741\\
37	0.00999886155390823\\
38	0.00999886155193583\\
39	0.00999886154992965\\
40	0.0099988615478891\\
41	0.0099988615458136\\
42	0.00999886154370255\\
43	0.00999886154155535\\
44	0.00999886153937136\\
45	0.00999886153714997\\
46	0.00999886153489053\\
47	0.00999886153259239\\
48	0.00999886153025489\\
49	0.00999886152787735\\
50	0.00999886152545909\\
51	0.00999886152299941\\
52	0.0099988615204976\\
53	0.00999886151795295\\
54	0.00999886151536472\\
55	0.00999886151273216\\
56	0.00999886151005451\\
57	0.009998861507331\\
58	0.00999886150456086\\
59	0.00999886150174328\\
60	0.00999886149887744\\
61	0.00999886149596253\\
62	0.00999886149299771\\
63	0.00999886148998212\\
64	0.00999886148691489\\
65	0.00999886148379514\\
66	0.00999886148062197\\
67	0.00999886147739448\\
68	0.00999886147411172\\
69	0.00999886147077277\\
70	0.00999886146737665\\
71	0.00999886146392239\\
72	0.00999886146040899\\
73	0.00999886145683546\\
74	0.00999886145320075\\
75	0.00999886144950382\\
76	0.00999886144574362\\
77	0.00999886144191905\\
78	0.00999886143802903\\
79	0.00999886143407244\\
80	0.00999886143004813\\
81	0.00999886142595495\\
82	0.00999886142179173\\
83	0.00999886141755727\\
84	0.00999886141325035\\
85	0.00999886140886975\\
86	0.00999886140441419\\
87	0.00999886139988241\\
88	0.0099988613952731\\
89	0.00999886139058494\\
90	0.00999886138581659\\
91	0.00999886138096668\\
92	0.00999886137603382\\
93	0.00999886137101659\\
94	0.00999886136591356\\
95	0.00999886136072326\\
96	0.0099988613554442\\
97	0.00999886135007489\\
98	0.00999886134461377\\
99	0.00999886133905928\\
100	0.00999886133340984\\
101	0.00999886132766382\\
102	0.00999886132181958\\
103	0.00999886131587545\\
104	0.00999886130982973\\
105	0.00999886130368069\\
106	0.00999886129742657\\
107	0.00999886129106559\\
108	0.00999886128459592\\
109	0.00999886127801571\\
110	0.00999886127132309\\
111	0.00999886126451614\\
112	0.00999886125759292\\
113	0.00999886125055145\\
114	0.00999886124338973\\
115	0.0099988612361057\\
116	0.00999886122869729\\
117	0.00999886122116239\\
118	0.00999886121349884\\
119	0.00999886120570446\\
120	0.00999886119777703\\
121	0.00999886118971429\\
122	0.00999886118151393\\
123	0.00999886117317363\\
124	0.009998861164691\\
125	0.00999886115606363\\
126	0.00999886114728907\\
127	0.00999886113836481\\
128	0.00999886112928831\\
129	0.009998861120057\\
130	0.00999886111066824\\
131	0.00999886110111936\\
132	0.00999886109140766\\
133	0.00999886108153036\\
134	0.00999886107148465\\
135	0.0099988610612677\\
136	0.00999886105087658\\
137	0.00999886104030835\\
138	0.00999886102956001\\
139	0.0099988610186285\\
140	0.00999886100751073\\
141	0.00999886099620353\\
142	0.0099988609847037\\
143	0.00999886097300798\\
144	0.00999886096111304\\
145	0.00999886094901551\\
146	0.00999886093671197\\
147	0.00999886092419893\\
148	0.00999886091147283\\
149	0.00999886089853008\\
150	0.00999886088536701\\
151	0.00999886087197989\\
152	0.00999886085836493\\
153	0.00999886084451827\\
154	0.00999886083043601\\
155	0.00999886081611415\\
156	0.00999886080154864\\
157	0.00999886078673536\\
158	0.00999886077167013\\
159	0.00999886075634869\\
160	0.0099988607407667\\
161	0.00999886072491978\\
162	0.00999886070880342\\
163	0.0099988606924131\\
164	0.00999886067574417\\
165	0.00999886065879194\\
166	0.00999886064155162\\
167	0.00999886062401834\\
168	0.00999886060618717\\
169	0.00999886058805307\\
170	0.00999886056961092\\
171	0.00999886055085554\\
172	0.00999886053178163\\
173	0.00999886051238381\\
174	0.00999886049265664\\
175	0.00999886047259455\\
176	0.00999886045219188\\
177	0.00999886043144291\\
178	0.00999886041034178\\
179	0.00999886038888257\\
180	0.00999886036705923\\
181	0.00999886034486562\\
182	0.00999886032229551\\
183	0.00999886029934255\\
184	0.00999886027600029\\
185	0.00999886025226217\\
186	0.00999886022812152\\
187	0.00999886020357156\\
188	0.0099988601786054\\
189	0.00999886015321602\\
190	0.0099988601273963\\
191	0.00999886010113899\\
192	0.00999886007443672\\
193	0.00999886004728201\\
194	0.00999886001966723\\
195	0.00999885999158465\\
196	0.00999885996302637\\
197	0.00999885993398442\\
198	0.00999885990445063\\
199	0.00999885987441673\\
200	0.00999885984387432\\
201	0.00999885981281483\\
202	0.00999885978122956\\
203	0.00999885974910967\\
204	0.00999885971644617\\
205	0.00999885968322991\\
206	0.00999885964945159\\
207	0.00999885961510175\\
208	0.0099988595801708\\
209	0.00999885954464895\\
210	0.00999885950852626\\
211	0.00999885947179264\\
212	0.0099988594344378\\
213	0.0099988593964513\\
214	0.00999885935782252\\
215	0.00999885931854066\\
216	0.00999885927859473\\
217	0.00999885923797358\\
218	0.00999885919666584\\
219	0.00999885915465996\\
220	0.00999885911194422\\
221	0.00999885906850666\\
222	0.00999885902433516\\
223	0.00999885897941737\\
224	0.00999885893374073\\
225	0.00999885888729249\\
226	0.00999885884005967\\
227	0.00999885879202906\\
228	0.00999885874318726\\
229	0.0099988586935206\\
230	0.00999885864301522\\
231	0.009998858591657\\
232	0.00999885853943159\\
233	0.00999885848632439\\
234	0.00999885843232057\\
235	0.00999885837740502\\
236	0.00999885832156241\\
237	0.00999885826477711\\
238	0.00999885820703325\\
239	0.00999885814831469\\
240	0.00999885808860501\\
241	0.0099988580278875\\
242	0.00999885796614518\\
243	0.00999885790336077\\
244	0.00999885783951671\\
245	0.00999885777459512\\
246	0.00999885770857783\\
247	0.00999885764144635\\
248	0.00999885757318187\\
249	0.00999885750376527\\
250	0.00999885743317708\\
251	0.00999885736139753\\
252	0.00999885728840647\\
253	0.00999885721418343\\
254	0.00999885713870757\\
255	0.00999885706195769\\
256	0.00999885698391225\\
257	0.0099988569045493\\
258	0.00999885682384652\\
259	0.00999885674178121\\
260	0.00999885665833028\\
261	0.00999885657347022\\
262	0.00999885648717712\\
263	0.00999885639942663\\
264	0.009998856310194\\
265	0.00999885621945404\\
266	0.0099988561271811\\
267	0.00999885603334908\\
268	0.00999885593793144\\
269	0.00999885584090112\\
270	0.00999885574223063\\
271	0.00999885564189193\\
272	0.00999885553985654\\
273	0.00999885543609545\\
274	0.00999885533057917\\
275	0.00999885522327762\\
276	0.00999885511416021\\
277	0.00999885500319577\\
278	0.00999885489035256\\
279	0.00999885477559827\\
280	0.0099988546589\\
281	0.00999885454022423\\
282	0.00999885441953683\\
283	0.00999885429680303\\
284	0.00999885417198742\\
285	0.00999885404505391\\
286	0.00999885391596574\\
287	0.00999885378468545\\
288	0.00999885365117487\\
289	0.0099988535153951\\
290	0.00999885337730647\\
291	0.00999885323686857\\
292	0.00999885309404018\\
293	0.00999885294877928\\
294	0.009998852801043\\
295	0.00999885265078764\\
296	0.0099988524979686\\
297	0.00999885234254039\\
298	0.00999885218445659\\
299	0.00999885202366979\\
300	0.00999885186013163\\
301	0.00999885169379272\\
302	0.00999885152460262\\
303	0.0099988513525098\\
304	0.00999885117746163\\
305	0.00999885099940431\\
306	0.00999885081828287\\
307	0.00999885063404112\\
308	0.00999885044662156\\
309	0.00999885025596534\\
310	0.00999885006201211\\
311	0.00999884986469999\\
312	0.00999884966396568\\
313	0.00999884945974437\\
314	0.0099988492519696\\
315	0.00999884904057318\\
316	0.00999884882548513\\
317	0.00999884860663358\\
318	0.00999884838394468\\
319	0.00999884815734249\\
320	0.00999884792674886\\
321	0.00999884769208335\\
322	0.00999884745326304\\
323	0.00999884721020244\\
324	0.00999884696281328\\
325	0.0099988467110044\\
326	0.00999884645468148\\
327	0.00999884619374692\\
328	0.00999884592809954\\
329	0.00999884565763438\\
330	0.0099988453822424\\
331	0.00999884510181016\\
332	0.00999884481621952\\
333	0.00999884452534724\\
334	0.00999884422906459\\
335	0.00999884392723685\\
336	0.00999884361972287\\
337	0.00999884330637445\\
338	0.00999884298703574\\
339	0.00999884266154254\\
340	0.00999884232972153\\
341	0.00999884199138937\\
342	0.00999884164635183\\
343	0.00999884129440259\\
344	0.00999884093532213\\
345	0.00999884056887634\\
346	0.00999884019481503\\
347	0.00999883981287024\\
348	0.00999883942275431\\
349	0.00999883902415775\\
350	0.00999883861674684\\
351	0.00999883820016082\\
352	0.00999883777400891\\
353	0.00999883733786695\\
354	0.00999883689127398\\
355	0.00999883643372824\\
356	0.00999883596468079\\
357	0.00999883548352641\\
358	0.009998834989593\\
359	0.00999883448211997\\
360	0.00999883396021254\\
361	0.0099988334227364\\
362	0.00999883286806642\\
363	0.00999883229348891\\
364	0.00999883169382415\\
365	0.00999883105845305\\
366	0.00999883036571932\\
367	0.00999882957683947\\
368	0.00999882867617437\\
369	0.0099988277547941\\
370	0.00999882681921332\\
371	0.00999882586922991\\
372	0.00999882490463685\\
373	0.00999882392521911\\
374	0.00999882293074936\\
375	0.00999882192098873\\
376	0.00999882089571304\\
377	0.00999881985478002\\
378	0.00999881879801241\\
379	0.00999881772519101\\
380	0.00999881663609405\\
381	0.00999881553049709\\
382	0.00999881440817278\\
383	0.00999881326889071\\
384	0.00999881211241709\\
385	0.00999881093851453\\
386	0.00999880974694162\\
387	0.00999880853745256\\
388	0.00999880730979668\\
389	0.0099988060637179\\
390	0.00999880479895408\\
391	0.00999880351523635\\
392	0.00999880221228826\\
393	0.00999880088982494\\
394	0.00999879954755211\\
395	0.0099987981851652\\
396	0.0099987968023486\\
397	0.00999879539877554\\
398	0.00999879397410944\\
399	0.00999879252800816\\
400	0.0099987910601337\\
401	0.0099987895701702\\
402	0.00999878805784939\\
403	0.00999878652296952\\
404	0.00999878496537107\\
405	0.00999878338492933\\
406	0.00999878178253733\\
407	0.00999878016043559\\
408	0.00999877852213052\\
409	0.00999877687002016\\
410	0.00999877519870064\\
411	0.00999877349877619\\
412	0.00999877176810963\\
413	0.00999877000698754\\
414	0.00999876821525015\\
415	0.0099987663922394\\
416	0.0099987645372807\\
417	0.00999876264968584\\
418	0.00999876072875439\\
419	0.00999875877376427\\
420	0.00999875678393502\\
421	0.00999875475837485\\
422	0.00999875269615624\\
423	0.00999875059639194\\
424	0.00999874845815901\\
425	0.00999874628049661\\
426	0.00999874406240368\\
427	0.00999874180283633\\
428	0.00999873950070511\\
429	0.00999873715487195\\
430	0.0099987347641469\\
431	0.00999873232728452\\
432	0.00999872984297999\\
433	0.00999872730986472\\
434	0.00999872472650155\\
435	0.00999872209137928\\
436	0.00999871940290623\\
437	0.0099987166594021\\
438	0.00999871385908642\\
439	0.00999871100005988\\
440	0.00999870808027008\\
441	0.00999870509744464\\
442	0.00999870204896453\\
443	0.0099986989316628\\
444	0.00999869574169428\\
445	0.00999869247534774\\
446	0.00999868913309026\\
447	0.0099986857126465\\
448	0.00999868221092105\\
449	0.00999867862430996\\
450	0.00999867494887379\\
451	0.00999867118020179\\
452	0.00999866731305387\\
453	0.00999866334034507\\
454	0.00999865925026592\\
455	0.00999865501853829\\
456	0.00999865059012545\\
457	0.00999864584946905\\
458	0.00999864064092852\\
459	0.00999863528250578\\
460	0.00999862983254051\\
461	0.00999862426219808\\
462	0.0099986185225333\\
463	0.0099986125415341\\
464	0.00999860627307679\\
465	0.00999859953783219\\
466	0.00999859156724826\\
467	0.0099985804685139\\
468	0.00999856122720195\\
469	0.00999853882104121\\
470	0.00999851599136136\\
471	0.00999849271598111\\
472	0.00999846897430738\\
473	0.00999844475242065\\
474	0.00999842005282698\\
475	0.00999839490691326\\
476	0.00999836936563406\\
477	0.00999834337238239\\
478	0.0099983043124748\\
479	0.0099982634295283\\
480	0.00999822129941152\\
481	0.00999817782978874\\
482	0.00999813290841915\\
483	0.00999808638303712\\
484	0.00999803800808412\\
485	0.00999798730127267\\
486	0.00999793315762983\\
487	0.00999787285584563\\
488	0.00999780014810592\\
489	0.00999766218044758\\
490	0.00999749457900609\\
491	0.0099973232727647\\
492	0.00999714810679892\\
493	0.0099969689184178\\
494	0.00999678554375719\\
495	0.0099965978334026\\
496	0.00999640568167148\\
497	0.00999620905808177\\
498	0.00999600794429296\\
499	0.00999580187601628\\
500	0.0099955898615952\\
501	0.00999537133399394\\
502	0.00999514566352727\\
503	0.00999487198030658\\
504	0.00999453197877233\\
505	0.00999417733984137\\
506	0.00999380368710282\\
507	0.00999339824171929\\
508	0.00999249863479512\\
509	0.00999024138993689\\
510	0.009987969291116\\
511	0.00998568193103925\\
512	0.00998337888068322\\
513	0.00998105967078947\\
514	0.0099787245288255\\
515	0.00997637157505996\\
516	0.0099739989746714\\
517	0.0099716055794803\\
518	0.00996919032502216\\
519	0.00996675219624949\\
520	0.00996428303999693\\
521	0.00996178664909822\\
522	0.0099592643678395\\
523	0.00995670698165256\\
524	0.00995411392853342\\
525	0.00995148465139348\\
526	0.00994881150400438\\
527	0.0099460695764821\\
528	0.00994247452732493\\
529	0.00993846045303648\\
530	0.00993432248483253\\
531	0.00993006061103056\\
532	0.00992571852724962\\
533	0.00991822796978299\\
534	0.00990672774350531\\
535	0.00989493303834666\\
536	0.00988281219865795\\
537	0.00987032853565887\\
538	0.00985743862467483\\
539	0.00984408667533638\\
540	0.00983017945278123\\
541	0.00974272020075207\\
542	0.00960566597158402\\
543	0.00946373262367878\\
544	0.00931647391436078\\
545	0.00916338120362373\\
546	0.00900387131459727\\
547	0.00883727334291648\\
548	0.00866281236164742\\
549	0.00847958965623579\\
550	0.00828655898260977\\
551	0.00808249893165934\\
552	0.00786598465982756\\
553	0.00763699834573866\\
554	0.00740195069507757\\
555	0.00716052774043377\\
556	0.00691239029212287\\
557	0.00665715505285347\\
558	0.00639441629849354\\
559	0.00612378737998776\\
560	0.00584478903828285\\
561	0.00555693540262497\\
562	0.00525977774490818\\
563	0.00495305641811948\\
564	0.0046363667826996\\
565	0.00431107070286592\\
566	0.00398614249697624\\
567	0.00366456760228114\\
568	0.00349609316333513\\
569	0.00333187726868771\\
570	0.0031732896727137\\
571	0.00302187273812389\\
572	0.00287974362373755\\
573	0.0027468448869553\\
574	0.00261504272331021\\
575	0.00248398048505642\\
576	0.00235392470609242\\
577	0.00222512595041651\\
578	0.00209779280040771\\
579	0.00197206281766885\\
580	0.00184806598976503\\
581	0.00172537279860785\\
582	0.00160240632595472\\
583	0.00147944176473777\\
584	0.00135679239519785\\
585	0.00123481760980238\\
586	0.00111393050009632\\
587	0.000994646016777647\\
588	0.000877530773009266\\
589	0.00076320155692426\\
590	0.00065232074510787\\
591	0.000545589588426686\\
592	0.000443742268922828\\
593	0.00034753281156088\\
594	0.000257714783085832\\
595	0.000175011746328155\\
596	0.000100073878205332\\
597	3.33540724222573e-05\\
598	0\\
599	0\\
600	0\\
};
\addplot [color=mycolor12,solid,forget plot]
  table[row sep=crcr]{%
1	0.00999970439610195\\
2	0.0099997043960851\\
3	0.00999970439606796\\
4	0.00999970439605053\\
5	0.0099997043960328\\
6	0.00999970439601477\\
7	0.00999970439599642\\
8	0.00999970439597777\\
9	0.00999970439595879\\
10	0.00999970439593949\\
11	0.00999970439591986\\
12	0.00999970439589989\\
13	0.00999970439587958\\
14	0.00999970439585892\\
15	0.00999970439583791\\
16	0.00999970439581654\\
17	0.0099997043957948\\
18	0.00999970439577269\\
19	0.0099997043957502\\
20	0.00999970439572733\\
21	0.00999970439570406\\
22	0.0099997043956804\\
23	0.00999970439565633\\
24	0.00999970439563185\\
25	0.00999970439560695\\
26	0.00999970439558162\\
27	0.00999970439555586\\
28	0.00999970439552966\\
29	0.00999970439550301\\
30	0.0099997043954759\\
31	0.00999970439544833\\
32	0.00999970439542029\\
33	0.00999970439539176\\
34	0.00999970439536275\\
35	0.00999970439533324\\
36	0.00999970439530323\\
37	0.0099997043952727\\
38	0.00999970439524165\\
39	0.00999970439521006\\
40	0.00999970439517794\\
41	0.00999970439514526\\
42	0.00999970439511203\\
43	0.00999970439507823\\
44	0.00999970439504384\\
45	0.00999970439500887\\
46	0.0099997043949733\\
47	0.00999970439493712\\
48	0.00999970439490032\\
49	0.00999970439486289\\
50	0.00999970439482482\\
51	0.0099997043947861\\
52	0.00999970439474671\\
53	0.00999970439470665\\
54	0.0099997043946659\\
55	0.00999970439462446\\
56	0.00999970439458231\\
57	0.00999970439453943\\
58	0.00999970439449582\\
59	0.00999970439445146\\
60	0.00999970439440635\\
61	0.00999970439436046\\
62	0.00999970439431378\\
63	0.00999970439426631\\
64	0.00999970439421802\\
65	0.00999970439416891\\
66	0.00999970439411895\\
67	0.00999970439406814\\
68	0.00999970439401646\\
69	0.0099997043939639\\
70	0.00999970439391043\\
71	0.00999970439385606\\
72	0.00999970439380075\\
73	0.00999970439374449\\
74	0.00999970439368727\\
75	0.00999970439362907\\
76	0.00999970439356987\\
77	0.00999970439350967\\
78	0.00999970439344843\\
79	0.00999970439338614\\
80	0.00999970439332279\\
81	0.00999970439325835\\
82	0.00999970439319281\\
83	0.00999970439312615\\
84	0.00999970439305835\\
85	0.00999970439298939\\
86	0.00999970439291925\\
87	0.00999970439284791\\
88	0.00999970439277535\\
89	0.00999970439270155\\
90	0.00999970439262648\\
91	0.00999970439255013\\
92	0.00999970439247248\\
93	0.0099997043923935\\
94	0.00999970439231317\\
95	0.00999970439223146\\
96	0.00999970439214836\\
97	0.00999970439206384\\
98	0.00999970439197787\\
99	0.00999970439189043\\
100	0.0099997043918015\\
101	0.00999970439171105\\
102	0.00999970439161905\\
103	0.00999970439152548\\
104	0.00999970439143031\\
105	0.00999970439133352\\
106	0.00999970439123507\\
107	0.00999970439113494\\
108	0.00999970439103309\\
109	0.00999970439092951\\
110	0.00999970439082416\\
111	0.00999970439071701\\
112	0.00999970439060803\\
113	0.00999970439049719\\
114	0.00999970439038446\\
115	0.0099997043902698\\
116	0.00999970439015319\\
117	0.00999970439003458\\
118	0.00999970438991395\\
119	0.00999970438979126\\
120	0.00999970438966647\\
121	0.00999970438953956\\
122	0.00999970438941048\\
123	0.0099997043892792\\
124	0.00999970438914568\\
125	0.00999970438900988\\
126	0.00999970438887176\\
127	0.00999970438873129\\
128	0.00999970438858843\\
129	0.00999970438844312\\
130	0.00999970438829534\\
131	0.00999970438814504\\
132	0.00999970438799218\\
133	0.00999970438783671\\
134	0.00999970438767859\\
135	0.00999970438751778\\
136	0.00999970438735422\\
137	0.00999970438718788\\
138	0.00999970438701871\\
139	0.00999970438684665\\
140	0.00999970438667166\\
141	0.00999970438649369\\
142	0.00999970438631269\\
143	0.00999970438612861\\
144	0.00999970438594139\\
145	0.00999970438575099\\
146	0.00999970438555734\\
147	0.0099997043853604\\
148	0.0099997043851601\\
149	0.00999970438495639\\
150	0.00999970438474922\\
151	0.00999970438453853\\
152	0.00999970438432425\\
153	0.00999970438410632\\
154	0.00999970438388469\\
155	0.00999970438365928\\
156	0.00999970438343004\\
157	0.00999970438319691\\
158	0.00999970438295981\\
159	0.00999970438271868\\
160	0.00999970438247345\\
161	0.00999970438222405\\
162	0.00999970438197042\\
163	0.00999970438171247\\
164	0.00999970438145014\\
165	0.00999970438118335\\
166	0.00999970438091203\\
167	0.0099997043806361\\
168	0.00999970438035549\\
169	0.00999970438007011\\
170	0.00999970437977988\\
171	0.00999970437948472\\
172	0.00999970437918456\\
173	0.0099997043788793\\
174	0.00999970437856885\\
175	0.00999970437825314\\
176	0.00999970437793208\\
177	0.00999970437760556\\
178	0.0099997043772735\\
179	0.00999970437693581\\
180	0.0099997043765924\\
181	0.00999970437624316\\
182	0.00999970437588799\\
183	0.00999970437552681\\
184	0.0099997043751595\\
185	0.00999970437478596\\
186	0.00999970437440609\\
187	0.00999970437401979\\
188	0.00999970437362694\\
189	0.00999970437322743\\
190	0.00999970437282115\\
191	0.00999970437240799\\
192	0.00999970437198783\\
193	0.00999970437156055\\
194	0.00999970437112604\\
195	0.00999970437068418\\
196	0.00999970437023483\\
197	0.00999970436977787\\
198	0.00999970436931317\\
199	0.00999970436884062\\
200	0.00999970436836006\\
201	0.00999970436787137\\
202	0.00999970436737442\\
203	0.00999970436686905\\
204	0.00999970436635514\\
205	0.00999970436583253\\
206	0.00999970436530108\\
207	0.00999970436476065\\
208	0.00999970436421108\\
209	0.00999970436365222\\
210	0.0099997043630839\\
211	0.00999970436250598\\
212	0.0099997043619183\\
213	0.00999970436132067\\
214	0.00999970436071295\\
215	0.00999970436009496\\
216	0.00999970435946653\\
217	0.00999970435882748\\
218	0.00999970435817764\\
219	0.00999970435751681\\
220	0.00999970435684483\\
221	0.0099997043561615\\
222	0.00999970435546662\\
223	0.00999970435476001\\
224	0.00999970435404147\\
225	0.0099997043533108\\
226	0.0099997043525678\\
227	0.00999970435181225\\
228	0.00999970435104395\\
229	0.00999970435026268\\
230	0.00999970434946822\\
231	0.00999970434866035\\
232	0.00999970434783885\\
233	0.00999970434700349\\
234	0.00999970434615404\\
235	0.00999970434529025\\
236	0.00999970434441188\\
237	0.0099997043435187\\
238	0.00999970434261045\\
239	0.00999970434168687\\
240	0.00999970434074772\\
241	0.00999970433979272\\
242	0.00999970433882161\\
243	0.00999970433783413\\
244	0.00999970433682998\\
245	0.0099997043358089\\
246	0.0099997043347706\\
247	0.00999970433371478\\
248	0.00999970433264115\\
249	0.00999970433154941\\
250	0.00999970433043926\\
251	0.00999970432931038\\
252	0.00999970432816247\\
253	0.00999970432699518\\
254	0.00999970432580821\\
255	0.00999970432460121\\
256	0.00999970432337385\\
257	0.00999970432212578\\
258	0.00999970432085666\\
259	0.00999970431956612\\
260	0.0099997043182538\\
261	0.00999970431691934\\
262	0.00999970431556236\\
263	0.00999970431418247\\
264	0.00999970431277928\\
265	0.00999970431135241\\
266	0.00999970430990145\\
267	0.00999970430842599\\
268	0.0099997043069256\\
269	0.00999970430539988\\
270	0.00999970430384837\\
271	0.00999970430227065\\
272	0.00999970430066627\\
273	0.00999970429903477\\
274	0.00999970429737569\\
275	0.00999970429568854\\
276	0.00999970429397287\\
277	0.00999970429222817\\
278	0.00999970429045394\\
279	0.00999970428864967\\
280	0.00999970428681486\\
281	0.00999970428494897\\
282	0.00999970428305146\\
283	0.0099997042811218\\
284	0.00999970427915942\\
285	0.00999970427716375\\
286	0.00999970427513421\\
287	0.00999970427307023\\
288	0.00999970427097119\\
289	0.00999970426883648\\
290	0.00999970426666549\\
291	0.00999970426445756\\
292	0.00999970426221207\\
293	0.00999970425992833\\
294	0.00999970425760568\\
295	0.00999970425524343\\
296	0.00999970425284088\\
297	0.0099997042503973\\
298	0.00999970424791197\\
299	0.00999970424538413\\
300	0.00999970424281304\\
301	0.00999970424019789\\
302	0.00999970423753791\\
303	0.00999970423483226\\
304	0.00999970423208013\\
305	0.00999970422928066\\
306	0.00999970422643297\\
307	0.00999970422353618\\
308	0.00999970422058937\\
309	0.00999970421759161\\
310	0.00999970421454195\\
311	0.0099997042114394\\
312	0.00999970420828295\\
313	0.00999970420507158\\
314	0.00999970420180422\\
315	0.00999970419847978\\
316	0.00999970419509715\\
317	0.00999970419165518\\
318	0.00999970418815268\\
319	0.00999970418458844\\
320	0.0099997041809612\\
321	0.00999970417726965\\
322	0.00999970417351248\\
323	0.00999970416968828\\
324	0.00999970416579564\\
325	0.00999970416183308\\
326	0.00999970415779904\\
327	0.00999970415369196\\
328	0.00999970414951016\\
329	0.00999970414525193\\
330	0.00999970414091547\\
331	0.00999970413649891\\
332	0.00999970413200029\\
333	0.00999970412741755\\
334	0.00999970412274857\\
335	0.00999970411799107\\
336	0.00999970411314268\\
337	0.00999970410820091\\
338	0.0099997041031631\\
339	0.00999970409802647\\
340	0.00999970409278805\\
341	0.00999970408744469\\
342	0.00999970408199303\\
343	0.00999970407642951\\
344	0.0099997040707503\\
345	0.0099997040649513\\
346	0.00999970405902812\\
347	0.00999970405297602\\
348	0.00999970404678991\\
349	0.00999970404046427\\
350	0.00999970403399313\\
351	0.00999970402737001\\
352	0.00999970402058787\\
353	0.00999970401363903\\
354	0.0099997040065151\\
355	0.00999970399920691\\
356	0.00999970399170435\\
357	0.00999970398399627\\
358	0.0099997039760702\\
359	0.0099997039679119\\
360	0.00999970395950438\\
361	0.00999970395082565\\
362	0.00999970394184347\\
363	0.00999970393250351\\
364	0.00999970392270427\\
365	0.00999970391225176\\
366	0.00999970390080888\\
367	0.00999970388800135\\
368	0.00999970387491621\\
369	0.00999970386163212\\
370	0.00999970384814638\\
371	0.00999970383445628\\
372	0.00999970382055909\\
373	0.00999970380645204\\
374	0.00999970379213237\\
375	0.00999970377759739\\
376	0.00999970376284445\\
377	0.00999970374787081\\
378	0.00999970373267359\\
379	0.00999970371724991\\
380	0.00999970370159686\\
381	0.00999970368571152\\
382	0.00999970366959092\\
383	0.0099997036532321\\
384	0.00999970363663206\\
385	0.00999970361978774\\
386	0.00999970360269609\\
387	0.00999970358535398\\
388	0.00999970356775827\\
389	0.00999970354990574\\
390	0.00999970353179314\\
391	0.00999970351341711\\
392	0.00999970349477425\\
393	0.00999970347586105\\
394	0.00999970345667391\\
395	0.00999970343720909\\
396	0.00999970341746275\\
397	0.00999970339743088\\
398	0.00999970337710936\\
399	0.00999970335649391\\
400	0.00999970333558022\\
401	0.00999970331436403\\
402	0.00999970329284148\\
403	0.00999970327101007\\
404	0.00999970324887127\\
405	0.00999970322643686\\
406	0.0099997032037377\\
407	0.00999970318083527\\
408	0.00999970315782051\\
409	0.00999970313475182\\
410	0.00999970311148934\\
411	0.00999970308782666\\
412	0.00999970306375765\\
413	0.0099997030392748\\
414	0.00999970301436928\\
415	0.009999702989032\\
416	0.00999970296325354\\
417	0.00999970293702415\\
418	0.00999970291033376\\
419	0.00999970288317189\\
420	0.00999970285552771\\
421	0.00999970282738997\\
422	0.00999970279874702\\
423	0.00999970276958678\\
424	0.00999970273989666\\
425	0.00999970270966362\\
426	0.00999970267887407\\
427	0.00999970264751387\\
428	0.00999970261556831\\
429	0.00999970258302206\\
430	0.00999970254985913\\
431	0.00999970251606284\\
432	0.00999970248161578\\
433	0.00999970244649979\\
434	0.00999970241069586\\
435	0.00999970237418414\\
436	0.00999970233694383\\
437	0.00999970229895314\\
438	0.00999970226018924\\
439	0.0099997022206281\\
440	0.00999970218024436\\
441	0.00999970213901119\\
442	0.00999970209690025\\
443	0.00999970205388205\\
444	0.00999970200992762\\
445	0.00999970196501004\\
446	0.00999970191909571\\
447	0.00999970187214897\\
448	0.00999970182413208\\
449	0.00999970177500489\\
450	0.00999970172472381\\
451	0.00999970167323947\\
452	0.00999970162049073\\
453	0.00999970156639009\\
454	0.00999970151079001\\
455	0.00999970145341547\\
456	0.00999970139377821\\
457	0.00999970133134347\\
458	0.00999970126761638\\
459	0.00999970120280487\\
460	0.0099997011366592\\
461	0.00999970106863759\\
462	0.00999970099764383\\
463	0.0099997009219837\\
464	0.00999970084079433\\
465	0.00999970074964362\\
466	0.00999970062952272\\
467	0.00999970043629822\\
468	0.00999970021503002\\
469	0.00999969999009975\\
470	0.009999699761335\\
471	0.00999969952858772\\
472	0.00999969929179549\\
473	0.00999969905111619\\
474	0.00999969880718207\\
475	0.00999969856142491\\
476	0.00999969831586726\\
477	0.00999969807009733\\
478	0.00999969749713267\\
479	0.00999969689104109\\
480	0.00999969626755884\\
481	0.0099996956253699\\
482	0.00999969496275721\\
483	0.00999969427710351\\
484	0.00999969356356345\\
485	0.00999969281150664\\
486	0.00999969199506879\\
487	0.00999969104944278\\
488	0.00999968982966578\\
489	0.00999968849158797\\
490	0.00999968712942908\\
491	0.00999968574244676\\
492	0.00999968432990924\\
493	0.00999968289118196\\
494	0.00999968142593702\\
495	0.00999967993459879\\
496	0.0099996784191395\\
497	0.00999967688390242\\
498	0.00999967533397796\\
499	0.00999967376351339\\
500	0.00999967215553076\\
501	0.00999967050724057\\
502	0.00999966881438454\\
503	0.00999966602971879\\
504	0.00999966169299949\\
505	0.00999965715544551\\
506	0.00999965232498029\\
507	0.00999964689477532\\
508	0.00999964070608822\\
509	0.00999963444375094\\
510	0.00999962810684547\\
511	0.00999962169542267\\
512	0.00999961521061403\\
513	0.00999960865463481\\
514	0.00999960205048303\\
515	0.00999959537010204\\
516	0.00999958859516865\\
517	0.00999958172256433\\
518	0.00999957475061019\\
519	0.00999956768088965\\
520	0.0099995603333951\\
521	0.00999955283804126\\
522	0.00999954526454885\\
523	0.00999953741587508\\
524	0.00999952932201664\\
525	0.0099995209888636\\
526	0.00999951225116848\\
527	0.00999950251419232\\
528	0.00999947180552544\\
529	0.00999943130832168\\
530	0.00999938862521251\\
531	0.00999934381240677\\
532	0.00999929575636018\\
533	0.00999916658108661\\
534	0.00999893491907451\\
535	0.00999869511603763\\
536	0.00999844639114465\\
537	0.00999818784741408\\
538	0.00999791843179201\\
539	0.00999763679861141\\
540	0.00999734069394201\\
541	0.00999523880893332\\
542	0.00999199346653391\\
543	0.00998872650467276\\
544	0.00998543582765006\\
545	0.00998211912727778\\
546	0.00997877390987515\\
547	0.00997539751394421\\
548	0.00997198713834896\\
549	0.00996853989228927\\
550	0.00996505288167449\\
551	0.00996152336179517\\
552	0.00995794903948198\\
553	0.00995435390004079\\
554	0.00995085585706864\\
555	0.00994745821629246\\
556	0.00994416444601642\\
557	0.00994097784367608\\
558	0.00993790230813352\\
559	0.00993494256110365\\
560	0.00993210290653603\\
561	0.00992938828563122\\
562	0.00992680457965828\\
563	0.00992435954354065\\
564	0.0099220597293271\\
565	0.00991813705201656\\
566	0.0099034258963405\\
567	0.00987473672518173\\
568	0.00968562218198496\\
569	0.00948588057623802\\
570	0.00927415985470431\\
571	0.00904887865168875\\
572	0.00880817797106934\\
573	0.00855229866101873\\
574	0.00828890720014428\\
575	0.00801779203946057\\
576	0.00773867093809746\\
577	0.00745130180414575\\
578	0.00715572070649497\\
579	0.00685181162239703\\
580	0.00653965469332596\\
581	0.0062206368942548\\
582	0.00589643732241719\\
583	0.00556694102676614\\
584	0.00523207996332671\\
585	0.004891739957962\\
586	0.00454581215895712\\
587	0.00419420467885093\\
588	0.0038368720520721\\
589	0.00347386032432953\\
590	0.0031053639618755\\
591	0.00273166653538783\\
592	0.00235288756260034\\
593	0.00196919002164565\\
594	0.00158084090714511\\
595	0.00118829559758769\\
596	0.000792401457307563\\
597	0.00039492475451582\\
598	0\\
599	0\\
600	0\\
};
\addplot [color=mycolor13,solid,forget plot]
  table[row sep=crcr]{%
1	0.000241064050983103\\
2	0.000241064050983103\\
3	0.000241064050983103\\
4	0.000241064050983103\\
5	0.000241064050983103\\
6	0.000241064050983103\\
7	0.000241064050983103\\
8	0.000241064050983103\\
9	0.000241064050983103\\
10	0.000241064050983103\\
11	0.000241064050983103\\
12	0.000241064050983103\\
13	0.000241064050983103\\
14	0.000241064050983103\\
15	0.000241064050983103\\
16	0.000241064050983103\\
17	0.000241064050983103\\
18	0.000241064050983103\\
19	0.000241064050983103\\
20	0.000241064050983103\\
21	0.000241064050983103\\
22	0.000241064050983103\\
23	0.000241064050983103\\
24	0.000241064050983103\\
25	0.000241064050983103\\
26	0.000241064050983103\\
27	0.000241064050983103\\
28	0.000241064050983103\\
29	0.000241064050983103\\
30	0.000241064050983103\\
31	0.000241064050983103\\
32	0.000241064050983103\\
33	0.000241064050983103\\
34	0.000241064050983103\\
35	0.000241064050983103\\
36	0.000241064050983103\\
37	0.000241064050983103\\
38	0.000241064050983103\\
39	0.000241064050983103\\
40	0.000241064050983103\\
41	0.000241064050983103\\
42	0.000241064050983103\\
43	0.000241064050983103\\
44	0.000241064050983103\\
45	0.000241064050983103\\
46	0.000241064050983103\\
47	0.000241064050983103\\
48	0.000241064050983103\\
49	0.000241064050983103\\
50	0.000241064050983103\\
51	0.000241064050983103\\
52	0.000241064050983103\\
53	0.000241064050983103\\
54	0.000241064050983103\\
55	0.000241064050983103\\
56	0.000241064050983103\\
57	0.000241064050983103\\
58	0.000241064050983103\\
59	0.000241064050983103\\
60	0.000241064050983103\\
61	0.000241064050983103\\
62	0.000241064050983103\\
63	0.000241064050983103\\
64	0.000241064050983103\\
65	0.000241064050983103\\
66	0.000241064050983103\\
67	0.000241064050983103\\
68	0.000241064050983103\\
69	0.000241064050983103\\
70	0.000241064050983103\\
71	0.000241064050983103\\
72	0.000241064050983103\\
73	0.000241064050983103\\
74	0.000241064050983103\\
75	0.000241064050983103\\
76	0.000241064050983103\\
77	0.000241064050983103\\
78	0.000241064050983103\\
79	0.000241064050983103\\
80	0.000241064050983103\\
81	0.000241064050983103\\
82	0.000241064050983103\\
83	0.000241064050983103\\
84	0.000241064050983103\\
85	0.000241064050983103\\
86	0.000241064050983103\\
87	0.000241064050983103\\
88	0.000241064050983103\\
89	0.000241064050983103\\
90	0.000241064050983103\\
91	0.000241064050983103\\
92	0.000241064050983103\\
93	0.000241064050983103\\
94	0.000241064050983103\\
95	0.000241064050983103\\
96	0.000241064050983103\\
97	0.000241064050983103\\
98	0.000241064050983103\\
99	0.000241064050983103\\
100	0.000241064050983103\\
101	0.000241064050983103\\
102	0.000241064050983103\\
103	0.000241064050983103\\
104	0.000241064050983103\\
105	0.000241064050983103\\
106	0.000241064050983103\\
107	0.000241064050983103\\
108	0.000241064050983103\\
109	0.000241064050983103\\
110	0.000241064050983103\\
111	0.000241064050983103\\
112	0.000241064050983103\\
113	0.000241064050983103\\
114	0.000241064050983103\\
115	0.000241064050983103\\
116	0.000241064050983103\\
117	0.000241064050983103\\
118	0.000241064050983103\\
119	0.000241064050983103\\
120	0.000241064050983103\\
121	0.000241064050983103\\
122	0.000241064050983103\\
123	0.000241064050983103\\
124	0.000241064050983103\\
125	0.000241064050983103\\
126	0.000241064050983103\\
127	0.000241064050983103\\
128	0.000241064050983103\\
129	0.000241064050983103\\
130	0.000241064050983103\\
131	0.000241064050983103\\
132	0.000241064050983103\\
133	0.000241064050983103\\
134	0.000241064050983103\\
135	0.000241064050983103\\
136	0.000241064050983103\\
137	0.000241064050983103\\
138	0.000241064050983103\\
139	0.000241064050983103\\
140	0.000241064050983103\\
141	0.000241064050983103\\
142	0.000241064050983103\\
143	0.000241064050983103\\
144	0.000241064050983103\\
145	0.000241064050983103\\
146	0.000241064050983103\\
147	0.000241064050983103\\
148	0.000241064050983103\\
149	0.000241064050983103\\
150	0.000241064050983103\\
151	0.000241064050983103\\
152	0.000241064050983103\\
153	0.000241064050983103\\
154	0.000241064050983103\\
155	0.000241064050983103\\
156	0.000241064050983103\\
157	0.000241064050983103\\
158	0.000241064050983103\\
159	0.000241064050983103\\
160	0.000241064050983103\\
161	0.000241064050983103\\
162	0.000241064050983103\\
163	0.000241064050983103\\
164	0.000241064050983103\\
165	0.000241064050983103\\
166	0.000241064050983103\\
167	0.000241064050983103\\
168	0.000241064050983103\\
169	0.000241064050983103\\
170	0.000241064050983103\\
171	0.000241064050983103\\
172	0.000241064050983103\\
173	0.000241064050983103\\
174	0.000241064050983103\\
175	0.000241064050983103\\
176	0.000241064050983103\\
177	0.000241064050983103\\
178	0.000241064050983103\\
179	0.000241064050983103\\
180	0.000241064050983103\\
181	0.000241064050983103\\
182	0.000241064050983103\\
183	0.000241064050983103\\
184	0.000241064050983103\\
185	0.000241064050983103\\
186	0.000241064050983103\\
187	0.000241064050983103\\
188	0.000241064050983103\\
189	0.000241064050983103\\
190	0.000241064050983103\\
191	0.000241064050983103\\
192	0.000241064050983103\\
193	0.000241064050983103\\
194	0.000241064050983103\\
195	0.000241064050983103\\
196	0.000241064050983103\\
197	0.000241064050983103\\
198	0.000241064050983103\\
199	0.000241064050983103\\
200	0.000241064050983103\\
201	0.000241064050983103\\
202	0.000241064050983103\\
203	0.000241064050983103\\
204	0.000241064050983103\\
205	0.000241064050983103\\
206	0.000241064050983103\\
207	0.000241064050983103\\
208	0.000241064050983103\\
209	0.000241064050983103\\
210	0.000241064050983103\\
211	0.000241064050983103\\
212	0.000241064050983103\\
213	0.000241064050983103\\
214	0.000241064050983103\\
215	0.000241064050983103\\
216	0.000241064050983103\\
217	0.000241064050983103\\
218	0.000241064050983103\\
219	0.000241064050983103\\
220	0.000241064050983103\\
221	0.000241064050983103\\
222	0.000241064050983103\\
223	0.000241064050983103\\
224	0.000241064050983103\\
225	0.000241064050983103\\
226	0.000241064050983103\\
227	0.000241064050983103\\
228	0.000241064050983103\\
229	0.000241064050983103\\
230	0.000241064050983103\\
231	0.000241064050983103\\
232	0.000241064050983103\\
233	0.000241064050983103\\
234	0.000241064050983103\\
235	0.000241064050983103\\
236	0.000241064050983103\\
237	0.000241064050983103\\
238	0.000241064050983103\\
239	0.000241064050983103\\
240	0.000241064050983103\\
241	0.000241064050983103\\
242	0.000241064050983103\\
243	0.000241064050983103\\
244	0.000241064050983103\\
245	0.000241064050983103\\
246	0.000241064050983103\\
247	0.000241064050983103\\
248	0.000241064050983103\\
249	0.000241064050983103\\
250	0.000241064050983103\\
251	0.000241064050983103\\
252	0.000241064050983103\\
253	0.000241064050983103\\
254	0.000241064050983103\\
255	0.000241064050983103\\
256	0.000241064050983103\\
257	0.000241064050983103\\
258	0.000241064050983103\\
259	0.000241064050983103\\
260	0.000241064050983103\\
261	0.000241064050983103\\
262	0.000241064050983103\\
263	0.000241064050983103\\
264	0.000241064050983103\\
265	0.000241064050983103\\
266	0.000241064050983103\\
267	0.000241064050983103\\
268	0.000241064050983103\\
269	0.000241064050983103\\
270	0.000241064050983103\\
271	0.000241064050983103\\
272	0.000241064050983103\\
273	0.000241064050983103\\
274	0.000241064050983103\\
275	0.000241064050983103\\
276	0.000241064050983103\\
277	0.000241064050983103\\
278	0.000241064050983103\\
279	0.000241064050983103\\
280	0.000241064050983103\\
281	0.000241064050983103\\
282	0.000241064050983103\\
283	0.000241064050983103\\
284	0.000241064050983103\\
285	0.000241064050983103\\
286	0.000241064050983103\\
287	0.000241064050983103\\
288	0.000241064050983103\\
289	0.000241064050983103\\
290	0.000241064050983103\\
291	0.000241064050983103\\
292	0.000241064050983103\\
293	0.000241064050983103\\
294	0.000241064050983103\\
295	0.000241064050983103\\
296	0.000241064050983103\\
297	0.000241064050983103\\
298	0.000241064050983103\\
299	0.000241064050983103\\
300	0.000241064050983103\\
301	0.000241064050983103\\
302	0.000241064050983103\\
303	0.000241064050983103\\
304	0.000241064050983103\\
305	0.000241064050983103\\
306	0.000241064050983103\\
307	0.000241064050983103\\
308	0.000241064050983103\\
309	0.000241064050983103\\
310	0.000241064050983103\\
311	0.000241064050983103\\
312	0.000241064050983103\\
313	0.000241064050983103\\
314	0.000241064050983103\\
315	0.000241064050983103\\
316	0.000241064050983103\\
317	0.000241064050983103\\
318	0.000241064050983103\\
319	0.000241064050983103\\
320	0.000241064050983103\\
321	0.000241064050983103\\
322	0.000241064050983103\\
323	0.000241064050983103\\
324	0.000241064050983103\\
325	0.000241064050983103\\
326	0.000241064050983103\\
327	0.000241064050983103\\
328	0.000241064050983103\\
329	0.000241064050983103\\
330	0.000241064050983103\\
331	0.000241064050983103\\
332	0.000241064050983103\\
333	0.000241064050983103\\
334	0.000241064050983103\\
335	0.000241064050983103\\
336	0.000241064050983103\\
337	0.000241064050983103\\
338	0.000241064050983103\\
339	0.000241064050983103\\
340	0.000241064050983103\\
341	0.000241064050983103\\
342	0.000241064050983103\\
343	0.000241064050983103\\
344	0.000241064050983103\\
345	0.000241064050983103\\
346	0.000241064050983103\\
347	0.000241064050983103\\
348	0.000241064050983103\\
349	0.000241064050983103\\
350	0.000241064050983103\\
351	0.000241064050983103\\
352	0.000241064050983103\\
353	0.000241064050983103\\
354	0.000241064050983103\\
355	0.000241064050983103\\
356	0.000241064050983103\\
357	0.000241064050983103\\
358	0.000241064050983103\\
359	0.000241064050983103\\
360	0.000241064050983103\\
361	0.000241064050983103\\
362	0.000241064050983103\\
363	0.000241064050983103\\
364	0.000241064050983103\\
365	0.000241064050983103\\
366	0.000241064050983103\\
367	0.000241064050983103\\
368	0.000241064050983103\\
369	0.000241064050983103\\
370	0.000241064050983103\\
371	0.000241064050983103\\
372	0.000241064050983103\\
373	0.000241064050983103\\
374	0.000241064050983103\\
375	0.000241064050983103\\
376	0.000241064050983103\\
377	0.000241064050983103\\
378	0.000241064050983103\\
379	0.000241064050983103\\
380	0.000241064050983103\\
381	0.000241064050983103\\
382	0.000241064050983103\\
383	0.000241064050983103\\
384	0.000241064050983103\\
385	0.000241064050983103\\
386	0.000241064050983103\\
387	0.000241064050983103\\
388	0.000241064050983103\\
389	0.000241064050983103\\
390	0.000241064050983103\\
391	0.000241064050983103\\
392	0.000241064050983103\\
393	0.000241064050983103\\
394	0.000241064050983103\\
395	0.000241064050983103\\
396	0.000241064050983103\\
397	0.000241064050983103\\
398	0.000241064050983103\\
399	0.000241064050983103\\
400	0.000241064050983103\\
401	0.000241064050983103\\
402	0.000241064050983103\\
403	0.000241064050983103\\
404	0.000241064050983103\\
405	0.000241064050983103\\
406	0.000241064050983103\\
407	0.000241064050983103\\
408	0.000241064050983103\\
409	0.000241064050983103\\
410	0.000241064050983103\\
411	0.000241064050983103\\
412	0.000241064050983103\\
413	0.000241064050983103\\
414	0.000241064050983103\\
415	0.000241064050983103\\
416	0.000241064050983103\\
417	0.000241064050983103\\
418	0.000241064050983103\\
419	0.000241064050983103\\
420	0.000241064050983103\\
421	0.000241064050983103\\
422	0.000241064050983103\\
423	0.000241064050983103\\
424	0.000241064050983103\\
425	0.000241064050983103\\
426	0.000241064050983103\\
427	0.000241064050983103\\
428	0.000241064050983103\\
429	0.000241064050983103\\
430	0.000241064050983103\\
431	0.000241064050983103\\
432	0.000241064050983103\\
433	0.000241064050983103\\
434	0.000241064050983103\\
435	0.000241064050983103\\
436	0.000241064050983103\\
437	0.000241064050983103\\
438	0.000241064050983103\\
439	0.000241064050983103\\
440	0.000241064050983103\\
441	0.000241064050983103\\
442	0.000241064050983103\\
443	0.000241064050983103\\
444	0.000241064050983103\\
445	0.000241064050983103\\
446	0.000241064050983103\\
447	0.000241064050983103\\
448	0.000241064050983103\\
449	0.000241064050983103\\
450	0.000241064050983103\\
451	0.000241064050983103\\
452	0.000241064050983103\\
453	0.000241064050983103\\
454	0.000241064050983103\\
455	0.000241064050983103\\
456	0.000241064050983103\\
457	0.000241064050983103\\
458	0.000241064050983103\\
459	0.000241064050983103\\
460	0.000241064050983103\\
461	0.000241064050983103\\
462	0.000241064050983103\\
463	0.000241064050983103\\
464	0.000241064050983103\\
465	0.000241064050983103\\
466	0.000241064050983103\\
467	0.000241064050983103\\
468	0.000241064050983103\\
469	0.000241064050983103\\
470	0.000241064050983103\\
471	0.000241064050983103\\
472	0.000241064050983103\\
473	0.000241064050983103\\
474	0.000241064050983103\\
475	0.000241064050983103\\
476	0.000241064050983103\\
477	0.000241064050983103\\
478	0.000241064050983103\\
479	0.000241064050983103\\
480	0.000241064050983103\\
481	0.000241064050983103\\
482	0.000241064050983103\\
483	0.000241064050983103\\
484	0.000241064050983103\\
485	0.000241064050983103\\
486	0.000241064050983103\\
487	0.000241064050983103\\
488	0.000241064050983103\\
489	0.000241064050983103\\
490	0.000241064050983103\\
491	0.000241064050983103\\
492	0.000241064050983103\\
493	0.000241064050983103\\
494	0.000241064050983103\\
495	0.000241064050983103\\
496	0.000241064050983103\\
497	0.000241064050983103\\
498	0.000241064050983103\\
499	0.000241064050983103\\
500	0.000241064050983103\\
501	0.000241064050983103\\
502	0.000241064050983103\\
503	0.000241064050983103\\
504	0.000241064050983103\\
505	0.000241064050983103\\
506	0.000241064050983103\\
507	0.000241064050983103\\
508	0.000241064050983103\\
509	0.000241064050983103\\
510	0.000241064050983103\\
511	0.000241064050983103\\
512	0.000241064050983103\\
513	0.000241064050983103\\
514	0.000241064050983103\\
515	0.000241064050983103\\
516	0.000241064050983103\\
517	0.000241064050983103\\
518	0.000241064050983103\\
519	0.000241064050983103\\
520	0.000241064050983103\\
521	0.000241064050983103\\
522	0.000241064050983103\\
523	0.000241064050983103\\
524	0.000241064050983103\\
525	0.000241064050983103\\
526	0.000241064050983103\\
527	0.000241064050983103\\
528	0.000241064050983103\\
529	0.000241064050983103\\
530	0.000241064050983103\\
531	0.000241064050983103\\
532	0.000241064050983103\\
533	0.000241064050983103\\
534	0.000241064050983103\\
535	0.000241064050983103\\
536	0.000241064050983103\\
537	0.000241064050983103\\
538	0.000241064050983103\\
539	0.000241064050983103\\
540	0.000241064050983103\\
541	0.000241064050983103\\
542	0.000241064050983103\\
543	0.000241064050983103\\
544	0.000241064050983103\\
545	0.000241064050983103\\
546	0.000241064050983103\\
547	0.000241064050983103\\
548	0.000241064050983103\\
549	0.000241064050983103\\
550	0.000241064050983103\\
551	0.000241064050983103\\
552	0.000241064050983103\\
553	0.000241064050983103\\
554	0.000241064050983103\\
555	0.000241064050983103\\
556	0.000241064050983103\\
557	0.000241064050983103\\
558	0.000241064050983103\\
559	0.000241064050983103\\
560	0.000241064050983103\\
561	0.000241064050983103\\
562	0.000241064050983103\\
563	0.000241064050983103\\
564	0.000241064050983103\\
565	0.000241749063135778\\
566	0.000253463059233083\\
567	0.000265313220815038\\
568	0.000410305514745056\\
569	0.000594445605831459\\
570	0.000790504526718181\\
571	0.0010000730134328\\
572	0.00122502642682604\\
573	0.00146605907646788\\
574	0.0017151074599206\\
575	0.00197243134443579\\
576	0.00223836387852356\\
577	0.00251314268478639\\
578	0.00279686556808013\\
579	0.00308975724537851\\
580	0.0033913089943072\\
581	0.00370111136261089\\
582	0.00401734174175409\\
583	0.00433970225536034\\
584	0.00466828634959776\\
585	0.00500320818845317\\
586	0.00534458915269589\\
587	0.00569252505204087\\
588	0.0060470515584461\\
589	0.00640808252825402\\
590	0.00677541915308552\\
591	0.00714888978102035\\
592	0.00752838693633574\\
593	0.00791376136525452\\
594	0.00830476600218674\\
595	0.0087009814905431\\
596	0.00910162822194027\\
597	0.0095050880145213\\
598	0.0099080685081414\\
599	0\\
600	0\\
};
\addplot [color=mycolor14,solid,forget plot]
  table[row sep=crcr]{%
1	0\\
2	0\\
3	0\\
4	0\\
5	0\\
6	0\\
7	0\\
8	0\\
9	0\\
10	0\\
11	0\\
12	0\\
13	0\\
14	0\\
15	0\\
16	0\\
17	0\\
18	0\\
19	0\\
20	0\\
21	0\\
22	0\\
23	0\\
24	0\\
25	0\\
26	0\\
27	0\\
28	0\\
29	0\\
30	0\\
31	0\\
32	0\\
33	0\\
34	0\\
35	0\\
36	0\\
37	0\\
38	0\\
39	0\\
40	0\\
41	0\\
42	0\\
43	0\\
44	0\\
45	0\\
46	0\\
47	0\\
48	0\\
49	0\\
50	0\\
51	0\\
52	0\\
53	0\\
54	0\\
55	0\\
56	0\\
57	0\\
58	0\\
59	0\\
60	0\\
61	0\\
62	0\\
63	0\\
64	0\\
65	0\\
66	0\\
67	0\\
68	0\\
69	0\\
70	0\\
71	0\\
72	0\\
73	0\\
74	0\\
75	0\\
76	0\\
77	0\\
78	0\\
79	0\\
80	0\\
81	0\\
82	0\\
83	0\\
84	0\\
85	0\\
86	0\\
87	0\\
88	0\\
89	0\\
90	0\\
91	0\\
92	0\\
93	0\\
94	0\\
95	0\\
96	0\\
97	0\\
98	0\\
99	0\\
100	0\\
101	0\\
102	0\\
103	0\\
104	0\\
105	0\\
106	0\\
107	0\\
108	0\\
109	0\\
110	0\\
111	0\\
112	0\\
113	0\\
114	0\\
115	0\\
116	0\\
117	0\\
118	0\\
119	0\\
120	0\\
121	0\\
122	0\\
123	0\\
124	0\\
125	0\\
126	0\\
127	0\\
128	0\\
129	0\\
130	0\\
131	0\\
132	0\\
133	0\\
134	0\\
135	0\\
136	0\\
137	0\\
138	0\\
139	0\\
140	0\\
141	0\\
142	0\\
143	0\\
144	0\\
145	0\\
146	0\\
147	0\\
148	0\\
149	0\\
150	0\\
151	0\\
152	0\\
153	0\\
154	0\\
155	0\\
156	0\\
157	0\\
158	0\\
159	0\\
160	0\\
161	0\\
162	0\\
163	0\\
164	0\\
165	0\\
166	0\\
167	0\\
168	0\\
169	0\\
170	0\\
171	0\\
172	0\\
173	0\\
174	0\\
175	0\\
176	0\\
177	0\\
178	0\\
179	0\\
180	0\\
181	0\\
182	0\\
183	0\\
184	0\\
185	0\\
186	0\\
187	0\\
188	0\\
189	0\\
190	0\\
191	0\\
192	0\\
193	0\\
194	0\\
195	0\\
196	0\\
197	0\\
198	0\\
199	0\\
200	0\\
201	0\\
202	0\\
203	0\\
204	0\\
205	0\\
206	0\\
207	0\\
208	0\\
209	0\\
210	0\\
211	0\\
212	0\\
213	0\\
214	0\\
215	0\\
216	0\\
217	0\\
218	0\\
219	0\\
220	0\\
221	0\\
222	0\\
223	0\\
224	0\\
225	0\\
226	0\\
227	0\\
228	0\\
229	0\\
230	0\\
231	0\\
232	0\\
233	0\\
234	0\\
235	0\\
236	0\\
237	0\\
238	0\\
239	0\\
240	0\\
241	0\\
242	0\\
243	0\\
244	0\\
245	0\\
246	0\\
247	0\\
248	0\\
249	0\\
250	0\\
251	0\\
252	0\\
253	0\\
254	0\\
255	0\\
256	0\\
257	0\\
258	0\\
259	0\\
260	0\\
261	0\\
262	0\\
263	0\\
264	0\\
265	0\\
266	0\\
267	0\\
268	0\\
269	0\\
270	0\\
271	0\\
272	0\\
273	0\\
274	0\\
275	0\\
276	0\\
277	0\\
278	0\\
279	0\\
280	0\\
281	0\\
282	0\\
283	0\\
284	0\\
285	0\\
286	0\\
287	0\\
288	0\\
289	0\\
290	0\\
291	0\\
292	0\\
293	0\\
294	0\\
295	0\\
296	0\\
297	0\\
298	0\\
299	0\\
300	0\\
301	0\\
302	0\\
303	0\\
304	0\\
305	0\\
306	0\\
307	0\\
308	0\\
309	0\\
310	0\\
311	0\\
312	0\\
313	0\\
314	0\\
315	0\\
316	0\\
317	0\\
318	0\\
319	0\\
320	0\\
321	0\\
322	0\\
323	0\\
324	0\\
325	0\\
326	0\\
327	0\\
328	0\\
329	0\\
330	0\\
331	0\\
332	0\\
333	0\\
334	5.34395650947204e-10\\
335	1.15405752732983e-09\\
336	1.78490839335517e-09\\
337	2.42718702688798e-09\\
338	3.08097622702292e-09\\
339	3.7462704053938e-09\\
340	4.42325923253163e-09\\
341	5.11214007990728e-09\\
342	5.81312728863281e-09\\
343	6.52647217814512e-09\\
344	7.25250069845883e-09\\
345	7.99166527067748e-09\\
346	8.74454929752424e-09\\
347	9.51159970801252e-09\\
348	1.02923500181668e-08\\
349	1.10862998942372e-08\\
350	1.18936706872483e-08\\
351	1.27146872222528e-08\\
352	1.35495778275436e-08\\
353	1.43985743639362e-08\\
354	1.52619122280985e-08\\
355	1.61398303870155e-08\\
356	1.70325714449766e-08\\
357	1.7940381750214e-08\\
358	1.88635113449878e-08\\
359	1.98022140017196e-08\\
360	2.07567472539151e-08\\
361	2.17273724305487e-08\\
362	2.27143546956379e-08\\
363	2.37179630756574e-08\\
364	2.4738470512949e-08\\
365	2.5776153896551e-08\\
366	2.68312941138062e-08\\
367	2.79041761053957e-08\\
368	2.8995088923791e-08\\
369	3.01043257864434e-08\\
370	3.12321841618619e-08\\
371	3.23789658417251e-08\\
372	3.35449770371859e-08\\
373	3.4730528463067e-08\\
374	3.59359353516285e-08\\
375	3.71615171534724e-08\\
376	3.84075963875673e-08\\
377	3.96744958152216e-08\\
378	4.09625329188538e-08\\
379	4.22720072520436e-08\\
380	4.36031754664621e-08\\
381	4.4956208310274e-08\\
382	4.63311348769375e-08\\
383	4.77278246214462e-08\\
384	4.91461651080398e-08\\
385	5.05866681360059e-08\\
386	5.20510761902004e-08\\
387	5.35401061011209e-08\\
388	5.50541788448286e-08\\
389	5.65937301735987e-08\\
390	5.81592121148706e-08\\
391	5.97510944375333e-08\\
392	6.13698657709378e-08\\
393	6.30160336134883e-08\\
394	6.46901213915346e-08\\
395	6.63926583226211e-08\\
396	6.81241528093417e-08\\
397	6.98850304655078e-08\\
398	7.1675503009099e-08\\
399	7.34953235246137e-08\\
400	7.53434266942629e-08\\
401	7.72177101885758e-08\\
402	7.91160011730615e-08\\
403	8.10402617750092e-08\\
404	8.30013124105549e-08\\
405	8.49998533169951e-08\\
406	8.70366282523623e-08\\
407	8.91124417830267e-08\\
408	9.12281765834018e-08\\
409	9.33847579100734e-08\\
410	9.55829274308186e-08\\
411	9.78228790172858e-08\\
412	1.00105585709825e-07\\
413	1.02432054835268e-07\\
414	1.04803329834153e-07\\
415	1.07220492206593e-07\\
416	1.09684663600859e-07\\
417	1.12197008109884e-07\\
418	1.14758734959682e-07\\
419	1.17371101876897e-07\\
420	1.20035419157529e-07\\
421	1.22753053347217e-07\\
422	1.25525427992502e-07\\
423	1.2835402409624e-07\\
424	1.31240388033559e-07\\
425	1.34186136086394e-07\\
426	1.37192959643411e-07\\
427	1.40262631489629e-07\\
428	1.43397014318821e-07\\
429	1.46598074148246e-07\\
430	1.49867905271133e-07\\
431	1.5320878232588e-07\\
432	1.56623274031272e-07\\
433	1.60114486841128e-07\\
434	1.63686540787919e-07\\
435	1.67345301663662e-07\\
436	1.71098764635964e-07\\
437	1.74954381160609e-07\\
438	1.78909154900545e-07\\
439	1.8295637287902e-07\\
440	1.87099605226687e-07\\
441	1.91342669503603e-07\\
442	1.95689656232806e-07\\
443	2.00144955729809e-07\\
444	2.04713282322897e-07\\
445	2.0939968622252e-07\\
446	2.14209534980528e-07\\
447	2.1914844253657e-07\\
448	2.24222122469994e-07\\
449	2.294359902385e-07\\
450	2.34794627981946e-07\\
451	2.40301934758161e-07\\
452	2.45965557771628e-07\\
453	2.5181662750504e-07\\
454	2.57965005822826e-07\\
455	2.64673797422747e-07\\
456	2.72410190911762e-07\\
457	2.81919231545979e-07\\
458	2.95772609912789e-07\\
459	3.10830772923359e-07\\
460	3.26185236841913e-07\\
461	3.41872439971896e-07\\
462	3.57922725856886e-07\\
463	3.74252066657456e-07\\
464	3.90866982913329e-07\\
465	4.07778149527845e-07\\
466	4.24997868336965e-07\\
467	4.42540711323325e-07\\
468	4.60423884675653e-07\\
469	4.78664199523387e-07\\
470	4.97262783721133e-07\\
471	5.16231892943477e-07\\
472	5.35584126137011e-07\\
473	5.55321885639264e-07\\
474	5.75426806261948e-07\\
475	5.95852950483929e-07\\
476	6.16567409574145e-07\\
477	6.3774354115478e-07\\
478	6.59575471449224e-07\\
479	6.82115872210608e-07\\
480	7.05438046054695e-07\\
481	7.29663740069737e-07\\
482	7.55037848234536e-07\\
483	7.82129100839374e-07\\
484	8.12347945901534e-07\\
485	8.48979433404414e-07\\
486	8.9425410391506e-07\\
487	9.40380960404758e-07\\
488	9.87385880167998e-07\\
489	1.03529459935483e-06\\
490	1.08413279222023e-06\\
491	1.13392403680387e-06\\
492	1.18468621498411e-06\\
493	1.24720906048516e-06\\
494	1.37006706325445e-06\\
495	1.49530305006788e-06\\
496	1.62279503189096e-06\\
497	1.75276173524047e-06\\
498	1.88607654705186e-06\\
499	2.02293411752196e-06\\
500	2.16357235569838e-06\\
501	2.30830980253248e-06\\
502	2.45762566848134e-06\\
503	2.6123006020255e-06\\
504	2.77347904379284e-06\\
505	2.94138237846309e-06\\
506	3.11739139183146e-06\\
507	3.30697665297456e-06\\
508	3.52424931125228e-06\\
509	3.76136854442694e-06\\
510	4.00167482907179e-06\\
511	4.24520184548467e-06\\
512	4.49197310366559e-06\\
513	4.74194860801026e-06\\
514	4.99356484026411e-06\\
515	5.24824522229138e-06\\
516	5.50687831805842e-06\\
517	5.76960685637947e-06\\
518	6.03654845807816e-06\\
519	6.30776522352361e-06\\
520	6.58327641762038e-06\\
521	6.87117288010451e-06\\
522	7.1678050440652e-06\\
523	7.47354024641004e-06\\
524	7.78880422125584e-06\\
525	8.12047792029648e-06\\
526	8.4942466400466e-06\\
527	9.78409477878136e-06\\
528	1.12481044670916e-05\\
529	1.27785154645931e-05\\
530	1.43861785462902e-05\\
531	1.60869638998932e-05\\
532	1.79245359223213e-05\\
533	2.34482844773654e-05\\
534	3.20899932668833e-05\\
535	4.10594471663407e-05\\
536	5.03858487580915e-05\\
537	6.01048232126996e-05\\
538	7.02571673572263e-05\\
539	8.08905515896843e-05\\
540	9.20670474081498e-05\\
541	0.00010622148232629\\
542	0.000237379372258858\\
543	0.000373368428114856\\
544	0.000514625762215635\\
545	0.000661651087225234\\
546	0.000815016828351423\\
547	0.000975380642210555\\
548	0.00114350123968401\\
549	0.00132025756814093\\
550	0.00150667201706074\\
551	0.00170393802307672\\
552	0.00191345060473266\\
553	0.00213682969392011\\
554	0.00236798736638595\\
555	0.00260563925080853\\
556	0.00285012933485717\\
557	0.00310182473985042\\
558	0.00336110730877499\\
559	0.00362843493984692\\
560	0.00390430453510286\\
561	0.00418922119559004\\
562	0.0044836504231414\\
563	0.00478785164046071\\
564	0.00510231617182818\\
565	0.00542675629214628\\
566	0.00575057379768084\\
567	0.00608512390105982\\
568	0.00628707380955198\\
569	0.00645757355205645\\
570	0.00662281053244697\\
571	0.00678105595816712\\
572	0.00693022715913866\\
573	0.00706963889082102\\
574	0.0072085378349189\\
575	0.00734681055555072\\
576	0.00748417292309207\\
577	0.00762034995855645\\
578	0.00775509740275521\\
579	0.00788823117439556\\
580	0.00801957808260108\\
581	0.00814908754750536\\
582	0.0082784986130116\\
583	0.00840796926358845\\
584	0.00853718929971185\\
585	0.00866579861787599\\
586	0.00879338692183546\\
587	0.00891946099316878\\
588	0.00904345529322948\\
589	0.00916475724618028\\
590	0.00928271236240283\\
591	0.00939663262290929\\
592	0.00950580711476238\\
593	0.009609516829347\\
594	0.00970705750680575\\
595	0.00979777178715996\\
596	0.00988114378468959\\
597	0.00995689746872328\\
598	0.00999970795535495\\
599	0\\
600	0\\
};
\addplot [color=mycolor15,solid,forget plot]
  table[row sep=crcr]{%
1	2.97404688413887e-05\\
2	2.97404862481491e-05\\
3	2.97405039699636e-05\\
4	2.97405220125235e-05\\
5	2.97405403816232e-05\\
6	2.9740559083164e-05\\
7	2.97405781231503e-05\\
8	2.9740597507695e-05\\
9	2.97406172430227e-05\\
10	2.97406373354716e-05\\
11	2.974065779149e-05\\
12	2.97406786176433e-05\\
13	2.97406998206171e-05\\
14	2.97407214072178e-05\\
15	2.974074338437e-05\\
16	2.97407657591278e-05\\
17	2.97407885386708e-05\\
18	2.97408117303039e-05\\
19	2.97408353414699e-05\\
20	2.97408593797424e-05\\
21	2.9740883852829e-05\\
22	2.97409087685803e-05\\
23	2.97409341349844e-05\\
24	2.97409599601773e-05\\
25	2.97409862524363e-05\\
26	2.97410130201917e-05\\
27	2.97410402720233e-05\\
28	2.97410680166644e-05\\
29	2.9741096263008e-05\\
30	2.97411250201055e-05\\
31	2.97411542971701e-05\\
32	2.97411841035834e-05\\
33	2.97412144488905e-05\\
34	2.9741245342814e-05\\
35	2.97412767952481e-05\\
36	2.97413088162646e-05\\
37	2.97413414161156e-05\\
38	2.97413746052409e-05\\
39	2.97414083942646e-05\\
40	2.97414427940014e-05\\
41	2.97414778154606e-05\\
42	2.97415134698528e-05\\
43	2.97415497685847e-05\\
44	2.97415867232712e-05\\
45	2.97416243457336e-05\\
46	2.97416626480066e-05\\
47	2.97417016423399e-05\\
48	2.97417413412035e-05\\
49	2.97417817572927e-05\\
50	2.974182290353e-05\\
51	2.97418647930666e-05\\
52	2.97419074392912e-05\\
53	2.97419508558368e-05\\
54	2.97419950565738e-05\\
55	2.9742040055629e-05\\
56	2.97420858673734e-05\\
57	2.97421325064466e-05\\
58	2.97421799877391e-05\\
59	2.9742228326417e-05\\
60	2.97422775379144e-05\\
61	2.97423276379445e-05\\
62	2.97423786424973e-05\\
63	2.97424305678536e-05\\
64	2.97424834305866e-05\\
65	2.97425372475604e-05\\
66	2.9742592035947e-05\\
67	2.97426478132246e-05\\
68	2.97427045971846e-05\\
69	2.97427624059331e-05\\
70	2.97428212579068e-05\\
71	2.97428811718655e-05\\
72	2.97429421669117e-05\\
73	2.9743004262485e-05\\
74	2.97430674783741e-05\\
75	2.97431318347205e-05\\
76	2.97431973520306e-05\\
77	2.97432640511736e-05\\
78	2.97433319533922e-05\\
79	2.97434010803108e-05\\
80	2.97434714539377e-05\\
81	2.97435430966787e-05\\
82	2.97436160313366e-05\\
83	2.97436902811259e-05\\
84	2.97437658696703e-05\\
85	2.97438428210202e-05\\
86	2.97439211596561e-05\\
87	2.97440009104921e-05\\
88	2.97440820988914e-05\\
89	2.97441647506696e-05\\
90	2.97442488921e-05\\
91	2.97443345499306e-05\\
92	2.97444217513827e-05\\
93	2.97445105241662e-05\\
94	2.97446008964847e-05\\
95	2.97446928970458e-05\\
96	2.97447865550702e-05\\
97	2.97448819002994e-05\\
98	2.97449789630034e-05\\
99	2.97450777739974e-05\\
100	2.9745178364644e-05\\
101	2.97452807668645e-05\\
102	2.97453850131519e-05\\
103	2.97454911365751e-05\\
104	2.9745599170797e-05\\
105	2.97457091500787e-05\\
106	2.97458211092926e-05\\
107	2.97459350839336e-05\\
108	2.97460511101297e-05\\
109	2.97461692246509e-05\\
110	2.97462894649258e-05\\
111	2.97464118690457e-05\\
112	2.97465364757846e-05\\
113	2.97466633246085e-05\\
114	2.97467924556816e-05\\
115	2.97469239098874e-05\\
116	2.97470577288371e-05\\
117	2.97471939548834e-05\\
118	2.97473326311327e-05\\
119	2.97474738014603e-05\\
120	2.9747617510521e-05\\
121	2.97477638037643e-05\\
122	2.97479127274519e-05\\
123	2.97480643286677e-05\\
124	2.97482186553321e-05\\
125	2.97483757562218e-05\\
126	2.97485356809794e-05\\
127	2.97486984801297e-05\\
128	2.97488642050993e-05\\
129	2.97490329082298e-05\\
130	2.97492046427922e-05\\
131	2.97493794630035e-05\\
132	2.97495574240511e-05\\
133	2.97497385820997e-05\\
134	2.97499229943135e-05\\
135	2.97501107188717e-05\\
136	2.9750301814993e-05\\
137	2.97504963429452e-05\\
138	2.97506943640664e-05\\
139	2.97508959407906e-05\\
140	2.97511011366547e-05\\
141	2.97513100163311e-05\\
142	2.97515226456378e-05\\
143	2.97517390915647e-05\\
144	2.97519594222922e-05\\
145	2.97521837072102e-05\\
146	2.97524120169437e-05\\
147	2.97526444233722e-05\\
148	2.97528809996517e-05\\
149	2.97531218202386e-05\\
150	2.97533669609094e-05\\
151	2.97536164987907e-05\\
152	2.97538705123772e-05\\
153	2.97541290815569e-05\\
154	2.97543922876374e-05\\
155	2.97546602133715e-05\\
156	2.97549329429775e-05\\
157	2.97552105621761e-05\\
158	2.97554931582052e-05\\
159	2.97557808198495e-05\\
160	2.97560736374782e-05\\
161	2.9756371703057e-05\\
162	2.97566751101879e-05\\
163	2.9756983954135e-05\\
164	2.97572983318532e-05\\
165	2.97576183420199e-05\\
166	2.97579440850588e-05\\
167	2.97582756631843e-05\\
168	2.97586131804212e-05\\
169	2.97589567426381e-05\\
170	2.97593064575862e-05\\
171	2.97596624349314e-05\\
172	2.9760024786282e-05\\
173	2.97603936252319e-05\\
174	2.97607690673841e-05\\
175	2.97611512304013e-05\\
176	2.97615402340333e-05\\
177	2.97619362001507e-05\\
178	2.97623392527923e-05\\
179	2.9762749518199e-05\\
180	2.97631671248483e-05\\
181	2.9763592203504e-05\\
182	2.97640248872477e-05\\
183	2.97644653115247e-05\\
184	2.97649136141839e-05\\
185	2.97653699355256e-05\\
186	2.97658344183348e-05\\
187	2.97663072079409e-05\\
188	2.97667884522436e-05\\
189	2.97672783017785e-05\\
190	2.97677769097478e-05\\
191	2.97682844320806e-05\\
192	2.97688010274693e-05\\
193	2.97693268574293e-05\\
194	2.97698620863476e-05\\
195	2.97704068815274e-05\\
196	2.97709614132446e-05\\
197	2.97715258548053e-05\\
198	2.97721003825931e-05\\
199	2.97726851761247e-05\\
200	2.97732804181117e-05\\
201	2.9773886294514e-05\\
202	2.97745029945912e-05\\
203	2.97751307109786e-05\\
204	2.97757696397298e-05\\
205	2.97764199803913e-05\\
206	2.97770819360553e-05\\
207	2.97777557134302e-05\\
208	2.97784415229013e-05\\
209	2.97791395786011e-05\\
210	2.97798500984726e-05\\
211	2.97805733043387e-05\\
212	2.97813094219759e-05\\
213	2.97820586811778e-05\\
214	2.97828213158347e-05\\
215	2.97835975640036e-05\\
216	2.97843876679877e-05\\
217	2.97851918744051e-05\\
218	2.97860104342731e-05\\
219	2.97868436030874e-05\\
220	2.97876916408994e-05\\
221	2.97885548124024e-05\\
222	2.97894333870122e-05\\
223	2.97903276389551e-05\\
224	2.97912378473574e-05\\
225	2.97921642963259e-05\\
226	2.97931072750464e-05\\
227	2.97940670778709e-05\\
228	2.97950440044181e-05\\
229	2.97960383596589e-05\\
230	2.97970504540231e-05\\
231	2.97980806034926e-05\\
232	2.97991291297078e-05\\
233	2.9800196360064e-05\\
234	2.98012826278215e-05\\
235	2.98023882722125e-05\\
236	2.98035136385473e-05\\
237	2.98046590783267e-05\\
238	2.9805824949355e-05\\
239	2.98070116158573e-05\\
240	2.98082194485931e-05\\
241	2.98094488249783e-05\\
242	2.98107001292055e-05\\
243	2.98119737523718e-05\\
244	2.98132700926038e-05\\
245	2.98145895551836e-05\\
246	2.98159325526829e-05\\
247	2.9817299505097e-05\\
248	2.98186908399811e-05\\
249	2.98201069925909e-05\\
250	2.9821548406019e-05\\
251	2.98230155313476e-05\\
252	2.98245088277847e-05\\
253	2.98260287628275e-05\\
254	2.98275758124032e-05\\
255	2.9829150461033e-05\\
256	2.98307532019827e-05\\
257	2.98323845374322e-05\\
258	2.98340449786399e-05\\
259	2.98357350461015e-05\\
260	2.98374552697339e-05\\
261	2.98392061890384e-05\\
262	2.98409883532887e-05\\
263	2.98428023217061e-05\\
264	2.98446486636503e-05\\
265	2.98465279588142e-05\\
266	2.98484407974301e-05\\
267	2.98503877804866e-05\\
268	2.9852369519957e-05\\
269	2.98543866390391e-05\\
270	2.98564397723334e-05\\
271	2.9858529565897e-05\\
272	2.98606566771132e-05\\
273	2.98628217747086e-05\\
274	2.98650255396937e-05\\
275	2.98672686660919e-05\\
276	2.98695518605617e-05\\
277	2.9871875842632e-05\\
278	2.98742413449366e-05\\
279	2.98766491134666e-05\\
280	2.98790999078097e-05\\
281	2.98815945014001e-05\\
282	2.98841336817878e-05\\
283	2.98867182508817e-05\\
284	2.98893490252302e-05\\
285	2.98920268362769e-05\\
286	2.98947525306485e-05\\
287	2.98975269704211e-05\\
288	2.99003510334061e-05\\
289	2.99032256134422e-05\\
290	2.99061516206816e-05\\
291	2.99091299818855e-05\\
292	2.99121616407267e-05\\
293	2.99152475580896e-05\\
294	2.99183887123827e-05\\
295	2.99215860998452e-05\\
296	2.99248407348631e-05\\
297	2.99281536502918e-05\\
298	2.99315258977662e-05\\
299	2.99349585480358e-05\\
300	2.99384526912784e-05\\
301	2.99420094374227e-05\\
302	2.99456299164607e-05\\
303	2.9949315278739e-05\\
304	2.99530666952098e-05\\
305	2.99568853576494e-05\\
306	2.99607724787912e-05\\
307	2.99647292924973e-05\\
308	2.99687570541305e-05\\
309	2.99728570416238e-05\\
310	2.99770305575876e-05\\
311	2.99812789314266e-05\\
312	2.99856035174778e-05\\
313	2.99900056885582e-05\\
314	2.99944868414278e-05\\
315	2.99990483970627e-05\\
316	3.00036918009157e-05\\
317	3.00084185231493e-05\\
318	3.00132300588652e-05\\
319	3.00181279282848e-05\\
320	3.00231136769208e-05\\
321	3.00281888757031e-05\\
322	3.00333551210753e-05\\
323	3.00386140350458e-05\\
324	3.00439672651796e-05\\
325	3.00494164845499e-05\\
326	3.00549633916248e-05\\
327	3.0060609710099e-05\\
328	3.00663571886809e-05\\
329	3.00722076009072e-05\\
330	3.00781627451161e-05\\
331	3.00842244450142e-05\\
332	3.00903945518141e-05\\
333	3.00966749503805e-05\\
334	3.01030675748446e-05\\
335	3.0109574444204e-05\\
336	3.0116197730633e-05\\
337	3.01229398372392e-05\\
338	3.01298032596678e-05\\
339	3.013678958593e-05\\
340	3.01438994298522e-05\\
341	3.01511347989999e-05\\
342	3.01584977206469e-05\\
343	3.01659902579481e-05\\
344	3.01736145562055e-05\\
345	3.01813729659352e-05\\
346	3.01892683393417e-05\\
347	3.0197304563206e-05\\
348	3.02054864102969e-05\\
349	3.02138128579057e-05\\
350	3.02222814728764e-05\\
351	3.02308947007519e-05\\
352	3.02396550330103e-05\\
353	3.02485650094424e-05\\
354	3.02576272166048e-05\\
355	3.02668442751239e-05\\
356	3.02762188145479e-05\\
357	3.02857534755208e-05\\
358	3.02954509977257e-05\\
359	3.03053141661812e-05\\
360	3.03153458119551e-05\\
361	3.03255488128955e-05\\
362	3.03359260944162e-05\\
363	3.03464806303157e-05\\
364	3.03572154436405e-05\\
365	3.03681336075773e-05\\
366	3.03792382463267e-05\\
367	3.03905325358621e-05\\
368	3.04020197043202e-05\\
369	3.04137030316267e-05\\
370	3.04255858476218e-05\\
371	3.04376715279285e-05\\
372	3.04499634877806e-05\\
373	3.04624651781903e-05\\
374	3.04751800990019e-05\\
375	3.04881118558866e-05\\
376	3.0501264266483e-05\\
377	3.05146413700678e-05\\
378	3.05282470375935e-05\\
379	3.05420851967666e-05\\
380	3.05561598031099e-05\\
381	3.05704747643933e-05\\
382	3.0585033775017e-05\\
383	3.05998400019483e-05\\
384	3.06148956558714e-05\\
385	3.06302020512804e-05\\
386	3.0645762569589e-05\\
387	3.06615913132755e-05\\
388	3.06776955832804e-05\\
389	3.06940806272026e-05\\
390	3.07107518965827e-05\\
391	3.07277150663531e-05\\
392	3.07449760558192e-05\\
393	3.07625410507499e-05\\
394	3.07804165253829e-05\\
395	3.0798609261259e-05\\
396	3.08171263552236e-05\\
397	3.08359751975293e-05\\
398	3.08551633733923e-05\\
399	3.08746983763119e-05\\
400	3.08945868818401e-05\\
401	3.09148331006111e-05\\
402	3.0935435757847e-05\\
403	3.09563860515573e-05\\
404	3.09776846867058e-05\\
405	3.09994090938901e-05\\
406	3.10215667480998e-05\\
407	3.10441679631196e-05\\
408	3.10672252442965e-05\\
409	3.10907548151211e-05\\
410	3.11147761810167e-05\\
411	3.11392994784912e-05\\
412	3.11642925247853e-05\\
413	3.11897662786693e-05\\
414	3.12157320612875e-05\\
415	3.12422015485712e-05\\
416	3.12691867359848e-05\\
417	3.1296699857704e-05\\
418	3.13247532557774e-05\\
419	3.13533592720863e-05\\
420	3.13825304572305e-05\\
421	3.14122807053136e-05\\
422	3.14426273511496e-05\\
423	3.14735901304213e-05\\
424	3.15051857807041e-05\\
425	3.15374317852405e-05\\
426	3.1570346425234e-05\\
427	3.16039488375239e-05\\
428	3.16382590789578e-05\\
429	3.16732981999255e-05\\
430	3.17090883325516e-05\\
431	3.17456528069055e-05\\
432	3.17830163293211e-05\\
433	3.18212053122779e-05\\
434	3.18602485916399e-05\\
435	3.19001791325837e-05\\
436	3.19410380561905e-05\\
437	3.19828824300761e-05\\
438	3.20257874025019e-05\\
439	3.20697658936037e-05\\
440	3.21147872590803e-05\\
441	3.21608935475674e-05\\
442	3.22081304390101e-05\\
443	3.2256548233606e-05\\
444	3.23062034443293e-05\\
445	3.23571608694045e-05\\
446	3.2409494039847e-05\\
447	3.24632763419494e-05\\
448	3.25185558668801e-05\\
449	3.25754032640866e-05\\
450	3.26339064499488e-05\\
451	3.2694162978103e-05\\
452	3.27562886805595e-05\\
453	3.28204574425456e-05\\
454	3.28870757674982e-05\\
455	3.295747725203e-05\\
456	3.30356563701028e-05\\
457	3.31302861934239e-05\\
458	3.32312889433352e-05\\
459	3.35644576772812e-05\\
460	3.3961516616606e-05\\
461	3.43665591525194e-05\\
462	3.47799253877007e-05\\
463	3.52021194582698e-05\\
464	3.56328498599616e-05\\
465	3.60724779356961e-05\\
466	3.65214328788453e-05\\
467	3.69802077471269e-05\\
468	3.74493896981799e-05\\
469	3.79297032815118e-05\\
470	3.84219729875345e-05\\
471	3.89262553057394e-05\\
472	3.94431589242761e-05\\
473	3.99735557862203e-05\\
474	4.05184070659492e-05\\
475	4.10787506897888e-05\\
476	4.16556414179785e-05\\
477	4.22500802699654e-05\\
478	4.28644809064331e-05\\
479	4.35015912728323e-05\\
480	4.4162779676883e-05\\
481	4.48497831672469e-05\\
482	4.55647937682293e-05\\
483	4.63105714109174e-05\\
484	4.70913852015442e-05\\
485	4.79174489215605e-05\\
486	4.94466246545089e-05\\
487	5.32610171649985e-05\\
488	5.71682557891886e-05\\
489	6.11734633723027e-05\\
490	6.52809737951096e-05\\
491	6.94956323432319e-05\\
492	7.38234983612785e-05\\
493	7.82615429808505e-05\\
494	8.27641291275647e-05\\
495	8.73950339060839e-05\\
496	9.21611414068531e-05\\
497	9.70695443509984e-05\\
498	0.000102129174520471\\
499	0.000107354891571254\\
500	0.000112757932422414\\
501	0.000118350994752327\\
502	0.00012414856212374\\
503	0.000130167454479554\\
504	0.000136428089358144\\
505	0.000142958330389988\\
506	0.00014977873458778\\
507	0.000156913664545248\\
508	0.000164400529141022\\
509	0.000208637551692032\\
510	0.000289365828409982\\
511	0.000372128538179114\\
512	0.000457035657703233\\
513	0.000544208739511396\\
514	0.000633784648101589\\
515	0.000725912880920369\\
516	0.000820776210335949\\
517	0.000918566704049611\\
518	0.00101949032647088\\
519	0.00112377558945576\\
520	0.00123167657983085\\
521	0.00134346807944416\\
522	0.00145947086589884\\
523	0.00158004110496509\\
524	0.0017055721376809\\
525	0.00183648333108853\\
526	0.00197325147179061\\
527	0.0021155721982587\\
528	0.00226487144985893\\
529	0.00242206314622936\\
530	0.00258807701420806\\
531	0.00276399348220323\\
532	0.0029492594674755\\
533	0.00313575936505194\\
534	0.00332415734137589\\
535	0.00351740715760901\\
536	0.00371573957706861\\
537	0.00391940576669202\\
538	0.00412868429533327\\
539	0.00434388767246155\\
540	0.00456536780885139\\
541	0.00479115779066909\\
542	0.00490665714545878\\
543	0.00502388027007063\\
544	0.00514281816252538\\
545	0.00526324563336852\\
546	0.00538489858802557\\
547	0.00550743548486917\\
548	0.00563040291106781\\
549	0.00575320531854967\\
550	0.00587506655315862\\
551	0.0059949809594261\\
552	0.00611165189103131\\
553	0.00622341901143749\\
554	0.00633639651244715\\
555	0.00645190759163161\\
556	0.00656953595970948\\
557	0.00668824396759888\\
558	0.00680607065679266\\
559	0.0069226896100025\\
560	0.0070377375858648\\
561	0.00715081464281374\\
562	0.0072614836895415\\
563	0.00736927191235638\\
564	0.00747368111557365\\
565	0.0075741982814712\\
566	0.00767034259470333\\
567	0.0077615946326214\\
568	0.00784765855962678\\
569	0.00793115728065736\\
570	0.00801393498895354\\
571	0.00809600623835434\\
572	0.00817749098008832\\
573	0.0082586155742962\\
574	0.00833945963331221\\
575	0.00841996297313883\\
576	0.00850007764627636\\
577	0.0085797729284561\\
578	0.00865903629153023\\
579	0.00873787217500842\\
580	0.00881629777701114\\
581	0.00889433337966777\\
582	0.00897189890365491\\
583	0.0090488514123376\\
584	0.00912503765204509\\
585	0.00920030513112112\\
586	0.00927450589979287\\
587	0.00934750117970699\\
588	0.00941916757473226\\
589	0.0094894043972367\\
590	0.00955815482124445\\
591	0.00962541204610133\\
592	0.00969122229181295\\
593	0.00975569306111539\\
594	0.00981900263092351\\
595	0.0098813590240843\\
596	0.00993714890633491\\
597	0.00997691198549166\\
598	0.00999970795535495\\
599	0\\
600	0\\
};
\addplot [color=mycolor16,solid,forget plot]
  table[row sep=crcr]{%
1	5.65686140171787e-05\\
2	5.65692926888515e-05\\
3	5.65699836428933e-05\\
4	5.6570687101196e-05\\
5	5.65714032896419e-05\\
6	5.65721324381866e-05\\
7	5.65728747809237e-05\\
8	5.6573630556163e-05\\
9	5.65744000065106e-05\\
10	5.65751833789346e-05\\
11	5.65759809248528e-05\\
12	5.6576792900208e-05\\
13	5.65776195655527e-05\\
14	5.6578461186128e-05\\
15	5.65793180319476e-05\\
16	5.65801903778843e-05\\
17	5.65810785037623e-05\\
18	5.65819826944348e-05\\
19	5.6582903239884e-05\\
20	5.65838404353065e-05\\
21	5.65847945812099e-05\\
22	5.65857659835094e-05\\
23	5.65867549536181e-05\\
24	5.65877618085544e-05\\
25	5.6588786871031e-05\\
26	5.65898304695701e-05\\
27	5.65908929385982e-05\\
28	5.65919746185507e-05\\
29	5.65930758559855e-05\\
30	5.65941970036895e-05\\
31	5.6595338420789e-05\\
32	5.65965004728662e-05\\
33	5.6597683532071e-05\\
34	5.65988879772381e-05\\
35	5.66001141940107e-05\\
36	5.66013625749592e-05\\
37	5.66026335197047e-05\\
38	5.66039274350468e-05\\
39	5.66052447350884e-05\\
40	5.66065858413722e-05\\
41	5.66079511830075e-05\\
42	5.6609341196806e-05\\
43	5.66107563274232e-05\\
44	5.6612197027499e-05\\
45	5.66136637577918e-05\\
46	5.66151569873301e-05\\
47	5.66166771935637e-05\\
48	5.66182248625042e-05\\
49	5.66198004888857e-05\\
50	5.66214045763156e-05\\
51	5.66230376374363e-05\\
52	5.66247001940868e-05\\
53	5.66263927774575e-05\\
54	5.66281159282758e-05\\
55	5.66298701969525e-05\\
56	5.66316561437757e-05\\
57	5.66334743390744e-05\\
58	5.66353253634023e-05\\
59	5.66372098077133e-05\\
60	5.6639128273561e-05\\
61	5.66410813732722e-05\\
62	5.66430697301463e-05\\
63	5.66450939786532e-05\\
64	5.66471547646258e-05\\
65	5.66492527454679e-05\\
66	5.66513885903539e-05\\
67	5.66535629804485e-05\\
68	5.66557766091084e-05\\
69	5.66580301821122e-05\\
70	5.66603244178669e-05\\
71	5.66626600476451e-05\\
72	5.66650378158088e-05\\
73	5.66674584800427e-05\\
74	5.66699228115921e-05\\
75	5.66724315954994e-05\\
76	5.66749856308612e-05\\
77	5.66775857310647e-05\\
78	5.66802327240513e-05\\
79	5.66829274525708e-05\\
80	5.66856707744515e-05\\
81	5.66884635628547e-05\\
82	5.66913067065632e-05\\
83	5.66942011102497e-05\\
84	5.66971476947592e-05\\
85	5.67001473974056e-05\\
86	5.67032011722475e-05\\
87	5.67063099904077e-05\\
88	5.67094748403569e-05\\
89	5.67126967282335e-05\\
90	5.6715976678153e-05\\
91	5.67193157325317e-05\\
92	5.67227149524056e-05\\
93	5.67261754177648e-05\\
94	5.67296982278936e-05\\
95	5.67332845017073e-05\\
96	5.67369353781049e-05\\
97	5.67406520163216e-05\\
98	5.67444355962932e-05\\
99	5.67482873190168e-05\\
100	5.67522084069385e-05\\
101	5.67562001043188e-05\\
102	5.67602636776284e-05\\
103	5.67644004159454e-05\\
104	5.67686116313471e-05\\
105	5.67728986593281e-05\\
106	5.67772628592091e-05\\
107	5.67817056145653e-05\\
108	5.67862283336528e-05\\
109	5.67908324498519e-05\\
110	5.67955194221043e-05\\
111	5.68002907353785e-05\\
112	5.68051479011222e-05\\
113	5.68100924577353e-05\\
114	5.68151259710528e-05\\
115	5.68202500348262e-05\\
116	5.68254662712205e-05\\
117	5.68307763313213e-05\\
118	5.68361818956422e-05\\
119	5.68416846746523e-05\\
120	5.68472864093111e-05\\
121	5.68529888715999e-05\\
122	5.68587938650786e-05\\
123	5.68647032254462e-05\\
124	5.68707188211138e-05\\
125	5.68768425537746e-05\\
126	5.68830763590099e-05\\
127	5.68894222068801e-05\\
128	5.68958821025368e-05\\
129	5.69024580868522e-05\\
130	5.69091522370484e-05\\
131	5.69159666673453e-05\\
132	5.69229035296142e-05\\
133	5.69299650140502e-05\\
134	5.69371533498509e-05\\
135	5.69444708059061e-05\\
136	5.69519196915097e-05\\
137	5.69595023570693e-05\\
138	5.69672211948445e-05\\
139	5.69750786396823e-05\\
140	5.69830771697804e-05\\
141	5.69912193074509e-05\\
142	5.69995076199075e-05\\
143	5.70079447200686e-05\\
144	5.70165332673602e-05\\
145	5.70252759685443e-05\\
146	5.7034175578567e-05\\
147	5.70432349014073e-05\\
148	5.70524567909548e-05\\
149	5.70618441518888e-05\\
150	5.70713999405958e-05\\
151	5.70811271660697e-05\\
152	5.70910288908665e-05\\
153	5.71011082320461e-05\\
154	5.71113683621456e-05\\
155	5.71218125101629e-05\\
156	5.71324439625708e-05\\
157	5.71432660643297e-05\\
158	5.71542822199352e-05\\
159	5.71654958944762e-05\\
160	5.71769106147174e-05\\
161	5.71885299701938e-05\\
162	5.72003576143319e-05\\
163	5.72123972655898e-05\\
164	5.72246527086122e-05\\
165	5.72371277954205e-05\\
166	5.72498264465989e-05\\
167	5.72627526525341e-05\\
168	5.72759104746552e-05\\
169	5.7289304046695e-05\\
170	5.7302937575994e-05\\
171	5.73168153448047e-05\\
172	5.73309417116368e-05\\
173	5.734532111262e-05\\
174	5.73599580628864e-05\\
175	5.73748571579912e-05\\
176	5.73900230753458e-05\\
177	5.74054605756886e-05\\
178	5.74211745045715e-05\\
179	5.74371697938855e-05\\
180	5.74534514634026e-05\\
181	5.74700246223558e-05\\
182	5.74868944710366e-05\\
183	5.7504066302442e-05\\
184	5.7521545503925e-05\\
185	5.75393375588965e-05\\
186	5.75574480485537e-05\\
187	5.75758826536282e-05\\
188	5.75946471561863e-05\\
189	5.76137474414499e-05\\
190	5.76331894996494e-05\\
191	5.76529794279174e-05\\
192	5.76731234322205e-05\\
193	5.76936278293164e-05\\
194	5.77144990487512e-05\\
195	5.77357436348993e-05\\
196	5.77573682490412e-05\\
197	5.77793796714665e-05\\
198	5.78017848036321e-05\\
199	5.7824590670357e-05\\
200	5.78478044220528e-05\\
201	5.78714333369946e-05\\
202	5.78954848236525e-05\\
203	5.79199664230385e-05\\
204	5.7944885811122e-05\\
205	5.79702508012799e-05\\
206	5.7996069346793e-05\\
207	5.80223495433889e-05\\
208	5.80490996318398e-05\\
209	5.80763280006069e-05\\
210	5.8104043188524e-05\\
211	5.81322538875468e-05\\
212	5.81609689455523e-05\\
213	5.81901973691784e-05\\
214	5.82199483267351e-05\\
215	5.82502311511634e-05\\
216	5.82810553430406e-05\\
217	5.83124305736681e-05\\
218	5.8344366688192e-05\\
219	5.83768737088011e-05\\
220	5.84099618379744e-05\\
221	5.84436414618047e-05\\
222	5.8477923153372e-05\\
223	5.85128176761903e-05\\
224	5.85483359877188e-05\\
225	5.85844892429447e-05\\
226	5.86212887980249e-05\\
227	5.86587462140113e-05\\
228	5.86968732606373e-05\\
229	5.87356819201838e-05\\
230	5.87751843914258e-05\\
231	5.88153930936375e-05\\
232	5.88563206706936e-05\\
233	5.88979799952467e-05\\
234	5.89403841729776e-05\\
235	5.89835465469382e-05\\
236	5.90274807019734e-05\\
237	5.90722004692365e-05\\
238	5.91177199307823e-05\\
239	5.91640534242598e-05\\
240	5.9211215547695e-05\\
241	5.92592211643669e-05\\
242	5.93080854077724e-05\\
243	5.93578236867041e-05\\
244	5.94084516904164e-05\\
245	5.94599853938932e-05\\
246	5.95124410632314e-05\\
247	5.95658352611133e-05\\
248	5.96201848524043e-05\\
249	5.96755070098495e-05\\
250	5.97318192198896e-05\\
251	5.97891392885874e-05\\
252	5.98474853476729e-05\\
253	5.9906875860719e-05\\
254	5.99673296294185e-05\\
255	6.00288658000058e-05\\
256	6.0091503869797e-05\\
257	6.01552636938625e-05\\
258	6.02201654918267e-05\\
259	6.02862298548174e-05\\
260	6.03534777525321e-05\\
261	6.04219305404695e-05\\
262	6.04916099672857e-05\\
263	6.05625381823067e-05\\
264	6.0634737743187e-05\\
265	6.07082316237365e-05\\
266	6.07830432218907e-05\\
267	6.08591963678865e-05\\
268	6.09367153326126e-05\\
269	6.10156248362292e-05\\
270	6.10959500570144e-05\\
271	6.11777166403687e-05\\
272	6.12609507074481e-05\\
273	6.13456788624173e-05\\
274	6.14319281988293e-05\\
275	6.15197263128553e-05\\
276	6.16091013176736e-05\\
277	6.1700081849194e-05\\
278	6.17926970761037e-05\\
279	6.1886976710078e-05\\
280	6.19829510162074e-05\\
281	6.20806508235941e-05\\
282	6.21801075361556e-05\\
283	6.22813531436073e-05\\
284	6.23844202326519e-05\\
285	6.24893419983553e-05\\
286	6.25961522557251e-05\\
287	6.27048854514626e-05\\
288	6.28155766759293e-05\\
289	6.29282616752975e-05\\
290	6.30429768638782e-05\\
291	6.31597593366452e-05\\
292	6.32786468819367e-05\\
293	6.33996779943343e-05\\
294	6.35228918877071e-05\\
295	6.3648328508428e-05\\
296	6.37760285487269e-05\\
297	6.39060334602184e-05\\
298	6.4038385467523e-05\\
299	6.4173127582044e-05\\
300	6.43103036158209e-05\\
301	6.44499581954836e-05\\
302	6.45921367762464e-05\\
303	6.47368856559318e-05\\
304	6.48842519889479e-05\\
305	6.50342838001458e-05\\
306	6.5187029998385e-05\\
307	6.5342540389643e-05\\
308	6.55008656895513e-05\\
309	6.5662057535781e-05\\
310	6.58261685024823e-05\\
311	6.59932521214869e-05\\
312	6.61633629120389e-05\\
313	6.63365563934328e-05\\
314	6.65128890395619e-05\\
315	6.66924183253693e-05\\
316	6.68752027372904e-05\\
317	6.70613017828279e-05\\
318	6.72507759991253e-05\\
319	6.74436869603686e-05\\
320	6.76400972838365e-05\\
321	6.78400706344218e-05\\
322	6.80436717273548e-05\\
323	6.82509663289394e-05\\
324	6.8462021254971e-05\\
325	6.86769043665899e-05\\
326	6.88956845632034e-05\\
327	6.91184317721492e-05\\
328	6.93452169347089e-05\\
329	6.9576111988072e-05\\
330	6.98111898428439e-05\\
331	7.00505243556354e-05\\
332	7.02941902964806e-05\\
333	7.05422633113625e-05\\
334	7.07948198825919e-05\\
335	7.10519372980586e-05\\
336	7.13136936645921e-05\\
337	7.15801680536252e-05\\
338	7.18514408826577e-05\\
339	7.21275938905929e-05\\
340	7.24087042430501e-05\\
341	7.26948428587868e-05\\
342	7.29860910020485e-05\\
343	7.32825302042284e-05\\
344	7.35842422087068e-05\\
345	7.38913089471398e-05\\
346	7.42038126507727e-05\\
347	7.4521836511912e-05\\
348	7.48454672176253e-05\\
349	7.51748004463351e-05\\
350	7.5509902440845e-05\\
351	7.5850838820205e-05\\
352	7.61977131217195e-05\\
353	7.65506307205999e-05\\
354	7.69096988742068e-05\\
355	7.72750267650979e-05\\
356	7.7646725496567e-05\\
357	7.80249079431435e-05\\
358	7.84096885959737e-05\\
359	7.88011843126841e-05\\
360	7.91995139765483e-05\\
361	7.96047985307576e-05\\
362	8.00171610146243e-05\\
363	8.04367266021808e-05\\
364	8.08636226435882e-05\\
365	8.12979787099416e-05\\
366	8.17399266420663e-05\\
367	8.21896006039804e-05\\
368	8.26471371416283e-05\\
369	8.31126752473286e-05\\
370	8.35863564296393e-05\\
371	8.40683247869657e-05\\
372	8.45587270806473e-05\\
373	8.50577128009715e-05\\
374	8.55654342253399e-05\\
375	8.60820465095733e-05\\
376	8.66077079884137e-05\\
377	8.71425810522433e-05\\
378	8.76868332892592e-05\\
379	8.82406345271868e-05\\
380	8.88041587279447e-05\\
381	8.93775842893488e-05\\
382	8.99610943089507e-05\\
383	9.0554876694332e-05\\
384	9.11591238130058e-05\\
385	9.17740312406845e-05\\
386	9.23997965054899e-05\\
387	9.30366290780722e-05\\
388	9.36848187472196e-05\\
389	9.43446116472078e-05\\
390	9.50162470261731e-05\\
391	9.56999747748655e-05\\
392	9.63960564241728e-05\\
393	9.71047662166858e-05\\
394	9.78263922479534e-05\\
395	9.85612376699937e-05\\
396	9.9309621946302e-05\\
397	0.000100071882143825\\
398	0.000100848374240193\\
399	0.000101639474398394\\
400	0.000102445580053366\\
401	0.000103267110253841\\
402	0.00010410450340359\\
403	0.000104958207218767\\
404	0.000105828653439658\\
405	0.000106716280661696\\
406	0.00010762201554143\\
407	0.000108546250093435\\
408	0.00010948940796019\\
409	0.000110451937993203\\
410	0.000111434327441338\\
411	0.000112437128199277\\
412	0.000113460940360323\\
413	0.00011450600917254\\
414	0.000115572906888593\\
415	0.000116662229101265\\
416	0.000117774596096609\\
417	0.000118910654227204\\
418	0.000120071077201777\\
419	0.00012125656710526\\
420	0.000122467854998132\\
421	0.000123705701775496\\
422	0.000124970903507071\\
423	0.00012626431100586\\
424	0.000127586837555735\\
425	0.000128939414976839\\
426	0.00013032302322817\\
427	0.000131738694130578\\
428	0.00013318751547734\\
429	0.000134670635581798\\
430	0.00013618926831784\\
431	0.000137744698715181\\
432	0.000139338289178032\\
433	0.000140971486410228\\
434	0.000142645829191509\\
435	0.000144362957433858\\
436	0.000146124624135332\\
437	0.000147932716261748\\
438	0.000149789303711434\\
439	0.000151696737210097\\
440	0.000153657119450931\\
441	0.000155672411296715\\
442	0.000157745331427691\\
443	0.000159878840783038\\
444	0.000162076176610412\\
445	0.000164340895066859\\
446	0.000166676927524447\\
447	0.000169088654855352\\
448	0.000171580961846792\\
449	0.000174159011220003\\
450	0.00017682868879919\\
451	0.000179596682747477\\
452	0.000182470525348742\\
453	0.000185458702239571\\
454	0.000188570720172352\\
455	0.000191817131472747\\
456	0.000195210848300859\\
457	0.000199307462198322\\
458	0.000234024009405493\\
459	0.00026928460136731\\
460	0.000305276150154545\\
461	0.000342075490123044\\
462	0.000379704063110204\\
463	0.000418183341215599\\
464	0.000457534982089145\\
465	0.000497777696872637\\
466	0.000538934363692127\\
467	0.000581026754456863\\
468	0.000624074492834437\\
469	0.000668093813393791\\
470	0.00071309587762147\\
471	0.000759084974786108\\
472	0.000806052990455533\\
473	0.000854028376390557\\
474	0.000903051642598929\\
475	0.000953165761908892\\
476	0.00100441633532496\\
477	0.00105685177698614\\
478	0.00111052282832288\\
479	0.00116548595276004\\
480	0.0012217935582322\\
481	0.0012794881130283\\
482	0.0013386153471524\\
483	0.00139922231598797\\
484	0.00146135345217606\\
485	0.00152504279172324\\
486	0.00158966854592528\\
487	0.00165368135681658\\
488	0.00171933201283993\\
489	0.00178671392434228\\
490	0.00185593334775801\\
491	0.00192709610727309\\
492	0.00200031965459637\\
493	0.00207573480513079\\
494	0.00215349542970393\\
495	0.00223377358343795\\
496	0.0023167659741897\\
497	0.00240270009880341\\
498	0.00249184103455956\\
499	0.00258450361043243\\
500	0.00268106407111543\\
501	0.00278190576531213\\
502	0.0028874736431148\\
503	0.0029982859704351\\
504	0.00311494759397139\\
505	0.0032381634133137\\
506	0.00336876465412971\\
507	0.00350338215803569\\
508	0.00364131750666185\\
509	0.00374611681805963\\
510	0.00381773951732628\\
511	0.00389051142076527\\
512	0.00396441256935567\\
513	0.00403941524333922\\
514	0.00411548449939267\\
515	0.00419258349552635\\
516	0.00427069909279331\\
517	0.00434994898963162\\
518	0.0044303370723928\\
519	0.0045117966287361\\
520	0.00459424109225291\\
521	0.00467756014622477\\
522	0.00476161321976009\\
523	0.00484622259021371\\
524	0.00493116498255079\\
525	0.00501616200872132\\
526	0.00510086597846347\\
527	0.00518485278179517\\
528	0.0052675936890152\\
529	0.00534842588310246\\
530	0.00542652963648317\\
531	0.00550089386889214\\
532	0.00557211176756093\\
533	0.0056449527050025\\
534	0.00571947466363871\\
535	0.00579570983248068\\
536	0.00587364085006118\\
537	0.00595321734125755\\
538	0.00603434412927874\\
539	0.00611686570736999\\
540	0.00620054569846214\\
541	0.00628504569438276\\
542	0.00637016506052658\\
543	0.00645710221151984\\
544	0.00654554616480435\\
545	0.00663503398630233\\
546	0.00672368199808921\\
547	0.00681075256412289\\
548	0.00689600642609548\\
549	0.00697921067175644\\
550	0.00706015269502789\\
551	0.00713866037434254\\
552	0.00721463073987145\\
553	0.00728807012257132\\
554	0.00735895954660089\\
555	0.00742723438282898\\
556	0.00749292400561867\\
557	0.00755666941165881\\
558	0.00762007127688854\\
559	0.0076831066443758\\
560	0.00774576403788845\\
561	0.00780802957268453\\
562	0.00786990696145242\\
563	0.00793142454002589\\
564	0.00799263908789431\\
565	0.00805363888320774\\
566	0.00811454494091608\\
567	0.00817550922357873\\
568	0.00823671564158016\\
569	0.00829828412878018\\
570	0.00836024274084549\\
571	0.00842259141986931\\
572	0.00848532757207436\\
573	0.00854844118564728\\
574	0.00861191607274561\\
575	0.00867573559690069\\
576	0.00873988162299363\\
577	0.0088043330464253\\
578	0.0088690640387695\\
579	0.0089340421116632\\
580	0.00899922608544475\\
581	0.00906456465846054\\
582	0.00913000140892399\\
583	0.00919547944080252\\
584	0.0092609411779913\\
585	0.00932634987444189\\
586	0.00939167533016675\\
587	0.00945690191585901\\
588	0.00952202325045493\\
589	0.00958699516677063\\
590	0.00965014598496389\\
591	0.00971019945887051\\
592	0.00976660115966509\\
593	0.00981892128442635\\
594	0.00986674062690153\\
595	0.00990834109159665\\
596	0.00994554766474265\\
597	0.0099771937715668\\
598	0.00999970795535495\\
599	0\\
600	0\\
};
\addplot [color=mycolor17,solid,forget plot]
  table[row sep=crcr]{%
1	0.00060710056306544\\
2	0.000607111173053\\
3	0.000607121975141467\\
4	0.000607132972803783\\
5	0.000607144169575505\\
6	0.000607155569055962\\
7	0.000607167174909374\\
8	0.000607178990866045\\
9	0.000607191020723523\\
10	0.000607203268347803\\
11	0.000607215737674604\\
12	0.000607228432710565\\
13	0.000607241357534549\\
14	0.000607254516298919\\
15	0.000607267913230871\\
16	0.000607281552633782\\
17	0.000607295438888552\\
18	0.00060730957645503\\
19	0.000607323969873404\\
20	0.000607338623765637\\
21	0.000607353542836953\\
22	0.00060736873187732\\
23	0.000607384195762969\\
24	0.000607399939457934\\
25	0.000607415968015658\\
26	0.000607432286580546\\
27	0.000607448900389626\\
28	0.000607465814774214\\
29	0.000607483035161575\\
30	0.000607500567076671\\
31	0.000607518416143883\\
32	0.000607536588088827\\
33	0.000607555088740134\\
34	0.000607573924031305\\
35	0.000607593100002599\\
36	0.000607612622802933\\
37	0.000607632498691823\\
38	0.000607652734041384\\
39	0.000607673335338322\\
40	0.000607694309185988\\
41	0.000607715662306484\\
42	0.000607737401542763\\
43	0.000607759533860813\\
44	0.000607782066351834\\
45	0.000607805006234513\\
46	0.000607828360857252\\
47	0.000607852137700522\\
48	0.000607876344379248\\
49	0.000607900988645143\\
50	0.000607926078389216\\
51	0.000607951621644233\\
52	0.000607977626587251\\
53	0.000608004101542216\\
54	0.000608031054982551\\
55	0.000608058495533872\\
56	0.000608086431976664\\
57	0.000608114873249078\\
58	0.00060814382844972\\
59	0.000608173306840546\\
60	0.000608203317849766\\
61	0.000608233871074806\\
62	0.000608264976285318\\
63	0.000608296643426294\\
64	0.000608328882621162\\
65	0.000608361704174973\\
66	0.000608395118577665\\
67	0.000608429136507334\\
68	0.000608463768833609\\
69	0.000608499026621057\\
70	0.000608534921132678\\
71	0.00060857146383343\\
72	0.00060860866639383\\
73	0.00060864654069363\\
74	0.000608685098825536\\
75	0.000608724353099022\\
76	0.00060876431604417\\
77	0.000608805000415634\\
78	0.000608846419196619\\
79	0.000608888585602958\\
80	0.000608931513087274\\
81	0.000608975215343184\\
82	0.00060901970630959\\
83	0.000609065000175061\\
84	0.000609111111382292\\
85	0.000609158054632594\\
86	0.000609205844890553\\
87	0.000609254497388657\\
88	0.000609304027632135\\
89	0.000609354451403769\\
90	0.00060940578476883\\
91	0.000609458044080163\\
92	0.000609511245983242\\
93	0.000609565407421429\\
94	0.000609620545641255\\
95	0.000609676678197812\\
96	0.000609733822960262\\
97	0.000609791998117441\\
98	0.000609851222183499\\
99	0.000609911514003726\\
100	0.000609972892760453\\
101	0.000610035377979026\\
102	0.000610098989533918\\
103	0.000610163747654957\\
104	0.000610229672933634\\
105	0.000610296786329516\\
106	0.000610365109176851\\
107	0.000610434663191176\\
108	0.000610505470476148\\
109	0.0006105775535304\\
110	0.00061065093525462\\
111	0.000610725638958657\\
112	0.000610801688368834\\
113	0.000610879107635327\\
114	0.000610957921339723\\
115	0.000611038154502675\\
116	0.000611119832591736\\
117	0.000611202981529285\\
118	0.00061128762770063\\
119	0.000611373797962243\\
120	0.000611461519650115\\
121	0.000611550820588303\\
122	0.000611641729097622\\
123	0.000611734274004447\\
124	0.000611828484649738\\
125	0.00061192439089816\\
126	0.000612022023147421\\
127	0.000612121412337731\\
128	0.000612222589961464\\
129	0.000612325588072974\\
130	0.000612430439298598\\
131	0.000612537176846808\\
132	0.000612645834518587\\
133	0.000612756446717948\\
134	0.000612869048462679\\
135	0.000612983675395235\\
136	0.000613100363793881\\
137	0.000613219150583975\\
138	0.0006133400733495\\
139	0.000613463170344769\\
140	0.000613588480506343\\
141	0.00061371604346519\\
142	0.00061384589955904\\
143	0.000613978089844943\\
144	0.000614112656112076\\
145	0.000614249640894796\\
146	0.000614389087485876\\
147	0.000614531039950022\\
148	0.000614675543137608\\
149	0.000614822642698651\\
150	0.000614972385097093\\
151	0.000615124817625259\\
152	0.000615279988418607\\
153	0.000615437946470778\\
154	0.000615598741648863\\
155	0.000615762424708964\\
156	0.000615929047312047\\
157	0.000616098662040062\\
158	0.000616271322412393\\
159	0.000616447082902506\\
160	0.000616625998955027\\
161	0.00061680812700302\\
162	0.000616993524485646\\
163	0.000617182249866102\\
164	0.000617374362649899\\
165	0.000617569923403457\\
166	0.000617768993773077\\
167	0.000617971636504181\\
168	0.000618177915460988\\
169	0.000618387895646483\\
170	0.000618601643222742\\
171	0.000618819225531706\\
172	0.000619040711116234\\
173	0.00061926616974161\\
174	0.000619495672417392\\
175	0.000619729291419688\\
176	0.000619967100313847\\
177	0.000620209173977507\\
178	0.000620455588624137\\
179	0.000620706421826964\\
180	0.000620961752543351\\
181	0.000621221661139611\\
182	0.000621486229416293\\
183	0.000621755540633897\\
184	0.000622029679539122\\
185	0.000622308732391533\\
186	0.000622592786990749\\
187	0.000622881932704141\\
188	0.000623176260495003\\
189	0.000623475862951279\\
190	0.000623780834314821\\
191	0.000624091270511146\\
192	0.000624407269179785\\
193	0.000624728929705209\\
194	0.000625056353248261\\
195	0.000625389642778231\\
196	0.000625728903105525\\
197	0.000626074240914882\\
198	0.000626425764799321\\
199	0.000626783585294594\\
200	0.000627147814914393\\
201	0.000627518568186147\\
202	0.000627895961687519\\
203	0.00062828011408358\\
204	0.000628671146164665\\
205	0.000629069180885006\\
206	0.000629474343402009\\
207	0.000629886761116316\\
208	0.00063030656371266\\
209	0.000630733883201393\\
210	0.000631168853960897\\
211	0.000631611612780782\\
212	0.000632062298905836\\
213	0.000632521054080933\\
214	0.000632988022596688\\
215	0.000633463351336034\\
216	0.000633947189821688\\
217	0.000634439690264526\\
218	0.000634941007612866\\
219	0.000635451299602746\\
220	0.00063597072680911\\
221	0.000636499452698023\\
222	0.000637037643679888\\
223	0.000637585469163665\\
224	0.00063814310161219\\
225	0.000638710716598526\\
226	0.000639288492863424\\
227	0.000639876612373886\\
228	0.000640475260382908\\
229	0.000641084625490354\\
230	0.000641704899705041\\
231	0.000642336278508026\\
232	0.000642978960917117\\
233	0.000643633149552712\\
234	0.000644299050704883\\
235	0.00064497687440177\\
236	0.000645666834479395\\
237	0.000646369148652763\\
238	0.000647084038588451\\
239	0.000647811729978585\\
240	0.000648552452616331\\
241	0.000649306440472815\\
242	0.000650073931775659\\
243	0.000650855169089061\\
244	0.000651650399395419\\
245	0.000652459874178667\\
246	0.000653283849509245\\
247	0.000654122586130791\\
248	0.000654976349548549\\
249	0.000655845410119596\\
250	0.000656730043144877\\
251	0.000657630528963128\\
252	0.000658547153046668\\
253	0.000659480206099143\\
254	0.000660429984155319\\
255	0.000661396788682836\\
256	0.00066238092668611\\
257	0.000663382710812323\\
258	0.000664402459459641\\
259	0.000665440496887614\\
260	0.000666497153329897\\
261	0.000667572765109307\\
262	0.000668667674755246\\
263	0.000669782231123588\\
264	0.000670916789519063\\
265	0.000672071711820202\\
266	0.000673247366606806\\
267	0.000674444129290063\\
268	0.000675662382245147\\
269	0.000676902514946222\\
270	0.000678164924103469\\
271	0.000679450013801791\\
272	0.000680758195640947\\
273	0.000682089888877918\\
274	0.000683445520574564\\
275	0.000684825525757863\\
276	0.000686230347613509\\
277	0.00068766043765968\\
278	0.000689116255873479\\
279	0.00069059827085856\\
280	0.000692106960016313\\
281	0.000693642809720723\\
282	0.000695206315496843\\
283	0.000696797982203112\\
284	0.000698418324217433\\
285	0.000700067865627164\\
286	0.000701747140423026\\
287	0.000703456692697003\\
288	0.000705197076844282\\
289	0.000706968857769262\\
290	0.000708772611095645\\
291	0.000710608923380657\\
292	0.000712478392333387\\
293	0.00071438162703725\\
294	0.000716319248176551\\
295	0.000718291888267079\\
296	0.000720300191890713\\
297	0.000722344815933939\\
298	0.000724426429830121\\
299	0.000726545715805459\\
300	0.000728703369128351\\
301	0.000730900098361999\\
302	0.000733136625620009\\
303	0.000735413686824708\\
304	0.000737732031967852\\
305	0.000740092425373718\\
306	0.000742495645964528\\
307	0.000744942487528862\\
308	0.00074743375899431\\
309	0.000749970284705992\\
310	0.000752552904711658\\
311	0.000755182475048859\\
312	0.000757859868016764\\
313	0.000760585972395715\\
314	0.000763361693543721\\
315	0.000766187953351628\\
316	0.000769065690777267\\
317	0.000771995862038124\\
318	0.000774979440783901\\
319	0.000778017418244921\\
320	0.000781110803352149\\
321	0.000784260622823736\\
322	0.000787467921212758\\
323	0.000790733760909778\\
324	0.000794059222093577\\
325	0.000797445402622664\\
326	0.000800893417859552\\
327	0.000804404400419329\\
328	0.000807979499833827\\
329	0.000811619882122375\\
330	0.000815326729260301\\
331	0.000819101238536467\\
332	0.000822944621790806\\
333	0.000826858104520196\\
334	0.000830842924832522\\
335	0.000834900332203871\\
336	0.000839031585935278\\
337	0.000843237953087477\\
338	0.000847520705524725\\
339	0.000851881115892226\\
340	0.000856320454048436\\
341	0.00086083997460567\\
342	0.000865440950954022\\
343	0.000870124765520062\\
344	0.000874892813357461\\
345	0.000879746502099697\\
346	0.000884687251878435\\
347	0.000889716494637998\\
348	0.000894835671465534\\
349	0.000900046226458471\\
350	0.00090534961427422\\
351	0.000910747165660721\\
352	0.000916240365161452\\
353	0.000921831016938411\\
354	0.000927520964625549\\
355	0.000933312092390991\\
356	0.00093920632597568\\
357	0.000945205633768997\\
358	0.00095131202810527\\
359	0.000957527567481472\\
360	0.000963854362196069\\
361	0.000970294572684496\\
362	0.000976850411488681\\
363	0.000983524145387163\\
364	0.000990318097706229\\
365	0.000997234650834975\\
366	0.00100427624897077\\
367	0.00101144540112513\\
368	0.00101874468442492\\
369	0.00102617674774945\\
370	0.00103374431575295\\
371	0.00104145019333466\\
372	0.00104929727064038\\
373	0.00105728852871219\\
374	0.00106542704594637\\
375	0.00107371600554792\\
376	0.00108215870409779\\
377	0.00109075856102227\\
378	0.00109951912815803\\
379	0.00110844409773115\\
380	0.00111753729548966\\
381	0.00112680270845646\\
382	0.00113624449881478\\
383	0.0011458670197193\\
384	0.00115567483428913\\
385	0.00116567273972835\\
386	0.00117586579861875\\
387	0.00118625937439456\\
388	0.00119685915185121\\
389	0.00120767126340748\\
390	0.00121870143312448\\
391	0.0012299554664848\\
392	0.0012414394300102\\
393	0.00125315966563248\\
394	0.0012651228044699\\
395	0.00127733577938584\\
396	0.00128980583549718\\
397	0.00130254053754614\\
398	0.00131554777276393\\
399	0.00132883574759273\\
400	0.00134241297654273\\
401	0.0013562882619089\\
402	0.00137047066481572\\
403	0.00138496947217772\\
404	0.00139979416905491\\
405	0.00141495440053258\\
406	0.00143045987581483\\
407	0.00144632158239907\\
408	0.001462545299621\\
409	0.00147913585335357\\
410	0.0014960965559549\\
411	0.00151342860016098\\
412	0.00153113025192362\\
413	0.00154919585542169\\
414	0.00156761450204312\\
415	0.0015863940812741\\
416	0.00160554271232852\\
417	0.0016250687544487\\
418	0.00164498081875182\\
419	0.00166528778274928\\
420	0.00168599880951421\\
421	0.00170712337422873\\
422	0.00172867129979596\\
423	0.00175065279766935\\
424	0.00177307850796426\\
425	0.0017959594132778\\
426	0.00181930659254416\\
427	0.00184313149916327\\
428	0.00186744597859563\\
429	0.00189226228667248\\
430	0.00191759310851943\\
431	0.00194345157791757\\
432	0.00196985129679493\\
433	0.00199680635427785\\
434	0.00202433134415148\\
435	0.00205244137820251\\
436	0.00208115208955569\\
437	0.00211047961200326\\
438	0.00214044050392997\\
439	0.00217105157111254\\
440	0.0022023298338896\\
441	0.00223429036131121\\
442	0.00226694938910766\\
443	0.00230032951286536\\
444	0.00233445494323543\\
445	0.00236935172915597\\
446	0.00240504800650734\\
447	0.00244157425986033\\
448	0.00247896359019233\\
449	0.00251725205283156\\
450	0.00255647831692348\\
451	0.00259668741809526\\
452	0.00263793096407801\\
453	0.00268026763373766\\
454	0.00272376469874542\\
455	0.00276849970976649\\
456	0.0028145610197631\\
457	0.00286150848974166\\
458	0.00287941697542049\\
459	0.00289800783767426\\
460	0.00291735065292944\\
461	0.00293752028578222\\
462	0.00295860304139865\\
463	0.00298069934473722\\
464	0.0030039265272458\\
465	0.00302842225403747\\
466	0.00305434867011648\\
467	0.00308189747994569\\
468	0.00311129624208877\\
469	0.0031428161229582\\
470	0.0031767814049533\\
471	0.00321358089606086\\
472	0.00325368035722807\\
473	0.00329501424169953\\
474	0.00333702469320655\\
475	0.00337971622038544\\
476	0.00342309259553974\\
477	0.00346715681227577\\
478	0.00351191100308439\\
479	0.00355735611816516\\
480	0.0036034912120651\\
481	0.00365030944749394\\
482	0.00369777262859418\\
483	0.00374585873342885\\
484	0.00379455808611841\\
485	0.00384386220384708\\
486	0.00389376869842156\\
487	0.00394428392090048\\
488	0.0039953968683074\\
489	0.00404708303801088\\
490	0.00409935611848499\\
491	0.00415228610492657\\
492	0.0042058298844828\\
493	0.00425993285968233\\
494	0.00431452645753851\\
495	0.00436952534095501\\
496	0.00442482334969621\\
497	0.00448028878141799\\
498	0.00453575849129831\\
499	0.00459103030891929\\
500	0.00464585317456217\\
501	0.00469991774731134\\
502	0.00475284339576551\\
503	0.00480416148072935\\
504	0.00485329454992126\\
505	0.00489953057909042\\
506	0.00494199189982736\\
507	0.00498403014859703\\
508	0.00502631851116879\\
509	0.00506877894432613\\
510	0.00511180870801828\\
511	0.00515583199383242\\
512	0.0052008761733455\\
513	0.00524696989751277\\
514	0.00529414375492416\\
515	0.00534243186387451\\
516	0.00539186417563767\\
517	0.00544246409415458\\
518	0.00549425230082413\\
519	0.00554724840675153\\
520	0.00560147249534651\\
521	0.00565694642515098\\
522	0.00571369586871487\\
523	0.00577175851596171\\
524	0.00583119248510434\\
525	0.00589205603176269\\
526	0.00595440650781836\\
527	0.00601829894926148\\
528	0.00608378416842955\\
529	0.00615090640577239\\
530	0.00621970099943915\\
531	0.006290191220047\\
532	0.00636234493983808\\
533	0.00643593470847046\\
534	0.0065104055149119\\
535	0.00658372831899706\\
536	0.00665569931798794\\
537	0.00672610360775146\\
538	0.00679471924588918\\
539	0.00686132371717709\\
540	0.0069257037777008\\
541	0.00698766973345227\\
542	0.00704707286005866\\
543	0.00710380131076663\\
544	0.00715783708505831\\
545	0.00720930379890342\\
546	0.00725978899536815\\
547	0.00730978161638203\\
548	0.00735929416828949\\
549	0.00740835724334546\\
550	0.00745702181705139\\
551	0.00750536083932739\\
552	0.007553469486322\\
553	0.00760146317781643\\
554	0.00764947698034132\\
555	0.00769766562106882\\
556	0.00774618802061691\\
557	0.00779517789609317\\
558	0.00784468709298625\\
559	0.00789473898348791\\
560	0.00794535972575758\\
561	0.00799657787168881\\
562	0.0080484239854243\\
563	0.00810092960386494\\
564	0.00815412576667316\\
565	0.00820804117363287\\
566	0.00826270003308565\\
567	0.00831811978243239\\
568	0.00837430874090948\\
569	0.00843126714425778\\
570	0.008488991119373\\
571	0.0085474742880391\\
572	0.00860670740247375\\
573	0.00866667814978815\\
574	0.00872737096574959\\
575	0.00878876667075254\\
576	0.00885084194403563\\
577	0.00891356973300979\\
578	0.00897691638806212\\
579	0.00904084037040419\\
580	0.00910530852313021\\
581	0.00917028946955943\\
582	0.009235735599458\\
583	0.00930157560444774\\
584	0.0093677995709556\\
585	0.00943209978059338\\
586	0.00949409750521473\\
587	0.00955271158918268\\
588	0.00960758154467065\\
589	0.00965794928725074\\
590	0.00970496351641899\\
591	0.00974966222411227\\
592	0.00979242953925062\\
593	0.0098334977384428\\
594	0.00987276071723054\\
595	0.00991035431098038\\
596	0.00994573390547615\\
597	0.0099771937715668\\
598	0.00999970795535495\\
599	0\\
600	0\\
};
\addplot [color=mycolor18,solid,forget plot]
  table[row sep=crcr]{%
1	0.00244237844548261\\
2	0.0024423835258782\\
3	0.00244238869846641\\
4	0.00244239396492018\\
5	0.00244239932694278\\
6	0.00244240478626838\\
7	0.00244241034466266\\
8	0.00244241600392328\\
9	0.00244242176588055\\
10	0.00244242763239799\\
11	0.00244243360537293\\
12	0.0024424396867371\\
13	0.00244244587845728\\
14	0.00244245218253596\\
15	0.00244245860101195\\
16	0.00244246513596102\\
17	0.00244247178949661\\
18	0.0024424785637705\\
19	0.00244248546097347\\
20	0.00244249248333609\\
21	0.00244249963312933\\
22	0.00244250691266538\\
23	0.00244251432429838\\
24	0.00244252187042512\\
25	0.00244252955348586\\
26	0.00244253737596515\\
27	0.00244254534039255\\
28	0.00244255344934349\\
29	0.00244256170544011\\
30	0.00244257011135209\\
31	0.0024425786697975\\
32	0.00244258738354368\\
33	0.00244259625540815\\
34	0.00244260528825948\\
35	0.00244261448501826\\
36	0.002442623848658\\
37	0.0024426333822061\\
38	0.00244264308874481\\
39	0.00244265297141227\\
40	0.00244266303340347\\
41	0.00244267327797128\\
42	0.00244268370842751\\
43	0.00244269432814396\\
44	0.00244270514055357\\
45	0.00244271614915138\\
46	0.00244272735749581\\
47	0.00244273876920971\\
48	0.00244275038798153\\
49	0.00244276221756652\\
50	0.00244277426178797\\
51	0.00244278652453837\\
52	0.00244279900978069\\
53	0.00244281172154965\\
54	0.00244282466395305\\
55	0.002442837841173\\
56	0.00244285125746735\\
57	0.00244286491717099\\
58	0.00244287882469729\\
59	0.00244289298453949\\
60	0.00244290740127211\\
61	0.00244292207955247\\
62	0.00244293702412215\\
63	0.00244295223980852\\
64	0.00244296773152627\\
65	0.002442983504279\\
66	0.00244299956316078\\
67	0.00244301591335784\\
68	0.00244303256015021\\
69	0.00244304950891338\\
70	0.00244306676512002\\
71	0.00244308433434176\\
72	0.00244310222225098\\
73	0.00244312043462255\\
74	0.00244313897733575\\
75	0.00244315785637612\\
76	0.00244317707783735\\
77	0.00244319664792325\\
78	0.00244321657294974\\
79	0.00244323685934682\\
80	0.00244325751366067\\
81	0.0024432785425557\\
82	0.0024432999528167\\
83	0.002443321751351\\
84	0.00244334394519064\\
85	0.00244336654149466\\
86	0.00244338954755132\\
87	0.0024434129707805\\
88	0.00244343681873596\\
89	0.00244346109910785\\
90	0.00244348581972508\\
91	0.00244351098855784\\
92	0.00244353661372015\\
93	0.00244356270347239\\
94	0.00244358926622398\\
95	0.00244361631053604\\
96	0.00244364384512406\\
97	0.00244367187886075\\
98	0.00244370042077877\\
99	0.00244372948007369\\
100	0.00244375906610683\\
101	0.00244378918840829\\
102	0.00244381985667991\\
103	0.00244385108079841\\
104	0.00244388287081846\\
105	0.00244391523697594\\
106	0.00244394818969112\\
107	0.00244398173957197\\
108	0.00244401589741758\\
109	0.00244405067422149\\
110	0.00244408608117523\\
111	0.00244412212967185\\
112	0.0024441588313095\\
113	0.00244419619789514\\
114	0.00244423424144822\\
115	0.00244427297420454\\
116	0.00244431240862009\\
117	0.00244435255737496\\
118	0.00244439343337737\\
119	0.00244443504976774\\
120	0.00244447741992283\\
121	0.00244452055746\\
122	0.00244456447624145\\
123	0.00244460919037862\\
124	0.00244465471423664\\
125	0.00244470106243884\\
126	0.00244474824987139\\
127	0.00244479629168795\\
128	0.00244484520331446\\
129	0.00244489500045402\\
130	0.00244494569909176\\
131	0.00244499731549994\\
132	0.00244504986624304\\
133	0.00244510336818296\\
134	0.00244515783848432\\
135	0.00244521329461989\\
136	0.002445269754376\\
137	0.00244532723585818\\
138	0.00244538575749684\\
139	0.00244544533805303\\
140	0.00244550599662434\\
141	0.00244556775265087\\
142	0.00244563062592133\\
143	0.00244569463657923\\
144	0.00244575980512923\\
145	0.00244582615244347\\
146	0.00244589369976821\\
147	0.00244596246873035\\
148	0.00244603248134433\\
149	0.00244610376001889\\
150	0.00244617632756412\\
151	0.0024462502071986\\
152	0.00244632542255664\\
153	0.00244640199769559\\
154	0.00244647995710343\\
155	0.00244655932570635\\
156	0.00244664012887657\\
157	0.00244672239244016\\
158	0.00244680614268512\\
159	0.00244689140636959\\
160	0.00244697821073016\\
161	0.00244706658349031\\
162	0.00244715655286902\\
163	0.00244724814758958\\
164	0.00244734139688848\\
165	0.00244743633052448\\
166	0.00244753297878786\\
167	0.00244763137250975\\
168	0.00244773154307178\\
169	0.00244783352241572\\
170	0.00244793734305341\\
171	0.00244804303807678\\
172	0.00244815064116814\\
173	0.00244826018661054\\
174	0.00244837170929836\\
175	0.00244848524474814\\
176	0.00244860082910945\\
177	0.00244871849917614\\
178	0.0024488382923976\\
179	0.00244896024689035\\
180	0.00244908440144973\\
181	0.0024492107955619\\
182	0.00244933946941591\\
183	0.00244947046391613\\
184	0.00244960382069475\\
185	0.00244973958212462\\
186	0.00244987779133223\\
187	0.00245001849221088\\
188	0.00245016172943426\\
189	0.00245030754847002\\
190	0.00245045599559375\\
191	0.00245060711790313\\
192	0.00245076096333232\\
193	0.0024509175806666\\
194	0.00245107701955733\\
195	0.00245123933053703\\
196	0.00245140456503486\\
197	0.00245157277539225\\
198	0.00245174401487891\\
199	0.00245191833770901\\
200	0.00245209579905768\\
201	0.00245227645507782\\
202	0.00245246036291716\\
203	0.00245264758073561\\
204	0.00245283816772298\\
205	0.00245303218411688\\
206	0.00245322969122101\\
207	0.0024534307514238\\
208	0.00245363542821723\\
209	0.00245384378621613\\
210	0.00245405589117771\\
211	0.00245427181002146\\
212	0.00245449161084941\\
213	0.00245471536296667\\
214	0.0024549431369024\\
215	0.00245517500443111\\
216	0.00245541103859433\\
217	0.00245565131372263\\
218	0.00245589590545805\\
219	0.00245614489077693\\
220	0.00245639834801303\\
221	0.00245665635688126\\
222	0.00245691899850154\\
223	0.00245718635542333\\
224	0.00245745851165044\\
225	0.00245773555266628\\
226	0.00245801756545962\\
227	0.0024583046385507\\
228	0.0024585968620179\\
229	0.00245889432752477\\
230	0.00245919712834759\\
231	0.00245950535940338\\
232	0.0024598191172785\\
233	0.00246013850025754\\
234	0.00246046360835289\\
235	0.00246079454333482\\
236	0.00246113140876195\\
237	0.00246147431001243\\
238	0.0024618233543155\\
239	0.00246217865078373\\
240	0.00246254031044571\\
241	0.00246290844627947\\
242	0.00246328317324632\\
243	0.00246366460832532\\
244	0.00246405287054844\\
245	0.00246444808103628\\
246	0.00246485036303433\\
247	0.00246525984195002\\
248	0.00246567664539031\\
249	0.00246610090319997\\
250	0.00246653274750046\\
251	0.00246697231272962\\
252	0.00246741973568192\\
253	0.00246787515554949\\
254	0.00246833871396379\\
255	0.00246881055503808\\
256	0.00246929082541056\\
257	0.0024697796742883\\
258	0.00247027725349187\\
259	0.00247078371750084\\
260	0.00247129922349995\\
261	0.00247182393142617\\
262	0.00247235800401653\\
263	0.00247290160685679\\
264	0.00247345490843098\\
265	0.0024740180801717\\
266	0.00247459129651147\\
267	0.00247517473493476\\
268	0.00247576857603087\\
269	0.00247637300354733\\
270	0.00247698820444326\\
271	0.00247761436894093\\
272	0.00247825169057141\\
273	0.00247890036620627\\
274	0.00247956059606223\\
275	0.00248023258367863\\
276	0.00248091653595876\\
277	0.00248161266358066\\
278	0.00248232118125036\\
279	0.00248304230741867\\
280	0.00248377626434689\\
281	0.00248452327817387\\
282	0.00248528357898464\\
283	0.0024860574008804\\
284	0.00248684498205017\\
285	0.00248764656484421\\
286	0.00248846239584902\\
287	0.00248929272596439\\
288	0.00249013781048225\\
289	0.00249099790916781\\
290	0.00249187328634269\\
291	0.00249276421097064\\
292	0.00249367095674565\\
293	0.00249459380218278\\
294	0.00249553303071185\\
295	0.00249648893077427\\
296	0.00249746179592313\\
297	0.00249845192492681\\
298	0.00249945962187647\\
299	0.00250048519629757\\
300	0.00250152896326585\\
301	0.00250259124352806\\
302	0.00250367236362781\\
303	0.00250477265603699\\
304	0.00250589245929339\\
305	0.00250703211814503\\
306	0.00250819198370252\\
307	0.00250937241360174\\
308	0.00251057377218133\\
309	0.00251179643068557\\
310	0.00251304076751458\\
311	0.00251430716856387\\
312	0.00251559602771501\\
313	0.00251690774749291\\
314	0.00251824273957381\\
315	0.00251960142391174\\
316	0.00252098422590553\\
317	0.00252239157967583\\
318	0.00252382392834611\\
319	0.00252528172433526\\
320	0.00252676542966019\\
321	0.00252827551624632\\
322	0.00252981246624239\\
323	0.00253137677233462\\
324	0.00253296893805325\\
325	0.00253458947806148\\
326	0.00253623891841334\\
327	0.00253791779676254\\
328	0.00253962666249777\\
329	0.00254136607677239\\
330	0.00254313661238575\\
331	0.00254493885346014\\
332	0.00254677339483983\\
333	0.00254864084111551\\
334	0.00255054180514683\\
335	0.00255247690591094\\
336	0.00255444676543193\\
337	0.00255645200439617\\
338	0.00255849323565539\\
339	0.00256057105346809\\
340	0.00256268601093812\\
341	0.00256483855260878\\
342	0.00256702871621453\\
343	0.00256925630532907\\
344	0.00257152194378727\\
345	0.00257382626851024\\
346	0.0025761699304349\\
347	0.00257855359553404\\
348	0.00258097794590763\\
349	0.00258344368094375\\
350	0.00258595151874106\\
351	0.00258850219942977\\
352	0.00259109648663455\\
353	0.00259373516443518\\
354	0.00259641903576101\\
355	0.00259914892306405\\
356	0.00260192566865458\\
357	0.00260475013406767\\
358	0.00260762319724591\\
359	0.0026105457471206\\
360	0.00261351868709786\\
361	0.0026165429802981\\
362	0.00261961961851274\\
363	0.00262274962406199\\
364	0.0026259340518677\\
365	0.00262917399177813\\
366	0.00263247057118866\\
367	0.0026358249580127\\
368	0.00263923836406946\\
369	0.00264271204897125\\
370	0.00264624732461443\\
371	0.00264984556040834\\
372	0.00265350818942373\\
373	0.00265723671572618\\
374	0.00266103272332598\\
375	0.00266489788753106\\
376	0.00266883399024505\\
377	0.00267284294219539\\
378	0.00267692681683241\\
379	0.00268108789732648\\
380	0.00268532870158226\\
381	0.00268965181341293\\
382	0.00269406006841038\\
383	0.00269855660434867\\
384	0.00270314492357118\\
385	0.00270782897478867\\
386	0.00271261326797898\\
387	0.00271750305810833\\
388	0.00272250470440484\\
389	0.0027276265667844\\
390	0.00273288190946128\\
391	0.00273827968754821\\
392	0.00274382822816893\\
393	0.00274953677257988\\
394	0.00275541561043726\\
395	0.00276147623771513\\
396	0.00276773154307394\\
397	0.00277419602856395\\
398	0.00278088607189821\\
399	0.00278782023921147\\
400	0.00279501965932281\\
401	0.00280250847313758\\
402	0.00281031437506767\\
403	0.00281846926726797\\
404	0.00282701005185636\\
405	0.00283597958983118\\
406	0.00284542785300111\\
407	0.00285541327511914\\
408	0.00286600432764904\\
409	0.00287728139424408\\
410	0.00288934128852139\\
411	0.00290229948848919\\
412	0.00291629425870046\\
413	0.00293149153708745\\
414	0.00294803052623377\\
415	0.0029648603126777\\
416	0.00298198513718147\\
417	0.00299940923279649\\
418	0.00301713681878248\\
419	0.00303517209573159\\
420	0.00305351924500449\\
421	0.00307218244096976\\
422	0.0030911658988944\\
423	0.00311047401714231\\
424	0.00313011174362107\\
425	0.00315008528389045\\
426	0.00317040135010141\\
427	0.00319106397692754\\
428	0.00321207708620355\\
429	0.00323344446751591\\
430	0.00325516975644509\\
431	0.00327725641013362\\
432	0.00329970767980168\\
433	0.00332252657977064\\
434	0.00334571585248872\\
435	0.00336927792898328\\
436	0.00339321488410088\\
437	0.00341752838589096\\
438	0.00344221963877208\\
439	0.00346728932151394\\
440	0.00349273752717384\\
441	0.00351856374366612\\
442	0.00354476677042555\\
443	0.00357134447724801\\
444	0.00359829358026841\\
445	0.00362560943043688\\
446	0.00365328575182862\\
447	0.00368131429903288\\
448	0.0037096843372865\\
449	0.00373838156662635\\
450	0.00376738468966009\\
451	0.00379664872239489\\
452	0.00382613472175291\\
453	0.00385580328726611\\
454	0.00388560615171628\\
455	0.00391548438820794\\
456	0.00394536663700935\\
457	0.00397516890657617\\
458	0.00400484313749156\\
459	0.00403468393460625\\
460	0.00406469774778422\\
461	0.00409485723653956\\
462	0.00412509802986709\\
463	0.00415534158024108\\
464	0.00418549210258833\\
465	0.00421543283122459\\
466	0.00424502122529108\\
467	0.00427408315193109\\
468	0.00430240566781353\\
469	0.00432972804576949\\
470	0.00435573062331471\\
471	0.00438002101124508\\
472	0.00440211747433547\\
473	0.00442406126361349\\
474	0.00444641432147683\\
475	0.00446918347720507\\
476	0.0044923755831371\\
477	0.00451599754607169\\
478	0.00454005632998642\\
479	0.004564558948784\\
480	0.00458951245540178\\
481	0.00461492393113333\\
482	0.00464080055425519\\
483	0.00466715003796089\\
484	0.00469398055662879\\
485	0.00472130034626765\\
486	0.00474911750556943\\
487	0.00477743919031227\\
488	0.00480627190263308\\
489	0.00483562226005485\\
490	0.00486549691479821\\
491	0.0048959019463123\\
492	0.00492684204823402\\
493	0.00495832198138339\\
494	0.00499034668114623\\
495	0.00502292107957287\\
496	0.00505605489047175\\
497	0.00508976080508257\\
498	0.00512405585633382\\
499	0.00515896350551215\\
500	0.00519451530265017\\
501	0.00523074291961656\\
502	0.00526768924415036\\
503	0.00530541283468658\\
504	0.00534399391726228\\
505	0.00538354245377115\\
506	0.00542420895843863\\
507	0.00546611266267793\\
508	0.00550931551993746\\
509	0.00555387293732511\\
510	0.00559983890393659\\
511	0.00564726579385787\\
512	0.00569622118682937\\
513	0.00574678272157384\\
514	0.00579902281922955\\
515	0.0058530045542137\\
516	0.00590877612135214\\
517	0.00596636353451982\\
518	0.0060257608323764\\
519	0.0060869167639242\\
520	0.00614971762405756\\
521	0.00621396516303915\\
522	0.00627934613861281\\
523	0.00634499542158637\\
524	0.00640940816470904\\
525	0.00647239290823549\\
526	0.00653374883032473\\
527	0.00659326934269996\\
528	0.00665074765908659\\
529	0.00670598518451775\\
530	0.0067588033458027\\
531	0.0068090601230743\\
532	0.00685667403304762\\
533	0.00690166053748132\\
534	0.00694428556288202\\
535	0.00698633836920417\\
536	0.00702781416657799\\
537	0.00706872243385485\\
538	0.00710908957047251\\
539	0.00714896149944049\\
540	0.00718840596287116\\
541	0.0072275141172655\\
542	0.00726640087898807\\
543	0.00730520379680146\\
544	0.0073440783511678\\
545	0.0073831883662304\\
546	0.00742263433169856\\
547	0.0074624700724885\\
548	0.00750274440711923\\
549	0.00754350978707902\\
550	0.00758482129058975\\
551	0.0076267352212644\\
552	0.00766930755017648\\
553	0.00771259113568549\\
554	0.00775662320431986\\
555	0.00780143604039559\\
556	0.00784705550342595\\
557	0.00789350133934143\\
558	0.00794079089526178\\
559	0.00798894090630713\\
560	0.00803796719697048\\
561	0.00808788435778896\\
562	0.00813870540146843\\
563	0.00819044142189942\\
564	0.00824310127831327\\
565	0.00829669132896895\\
566	0.00835121523792377\\
567	0.00840667387413305\\
568	0.00846306536310321\\
569	0.00852038513578629\\
570	0.00857862579519724\\
571	0.00863777707688469\\
572	0.0086978253363676\\
573	0.00875875063967371\\
574	0.00882052518607372\\
575	0.00888311124128023\\
576	0.00894648323118318\\
577	0.00901053857990508\\
578	0.00907531874573529\\
579	0.00914090526783061\\
580	0.009205549681425\\
581	0.00926835372003995\\
582	0.00932896255875889\\
583	0.00938624167980904\\
584	0.00943944829201353\\
585	0.00949040788106645\\
586	0.0095387431908683\\
587	0.00958509003453771\\
588	0.00962947324275186\\
589	0.0096724489502006\\
590	0.00971437214798795\\
591	0.00975535399573538\\
592	0.00979545792277785\\
593	0.0098347066133119\\
594	0.00987312381469506\\
595	0.00991038882774525\\
596	0.00994573390547615\\
597	0.0099771937715668\\
598	0.00999970795535495\\
599	0\\
600	0\\
};
\addplot [color=red!25!mycolor17,solid,forget plot]
  table[row sep=crcr]{%
1	0.00302839379222758\\
2	0.00302840505615055\\
3	0.00302841652450542\\
4	0.00302842820100256\\
5	0.00302844008941975\\
6	0.0030284521936034\\
7	0.0030284645174697\\
8	0.00302847706500599\\
9	0.00302848984027202\\
10	0.00302850284740127\\
11	0.00302851609060222\\
12	0.00302852957415983\\
13	0.00302854330243684\\
14	0.00302855727987524\\
15	0.0030285715109976\\
16	0.00302858600040869\\
17	0.00302860075279686\\
18	0.00302861577293557\\
19	0.00302863106568497\\
20	0.00302864663599342\\
21	0.00302866248889916\\
22	0.0030286786295319\\
23	0.00302869506311444\\
24	0.00302871179496444\\
25	0.00302872883049607\\
26	0.00302874617522179\\
27	0.00302876383475419\\
28	0.00302878181480763\\
29	0.0030288001212003\\
30	0.00302881875985596\\
31	0.00302883773680588\\
32	0.00302885705819082\\
33	0.00302887673026299\\
34	0.0030288967593881\\
35	0.00302891715204735\\
36	0.00302893791483957\\
37	0.00302895905448338\\
38	0.00302898057781929\\
39	0.00302900249181194\\
40	0.00302902480355235\\
41	0.00302904752026022\\
42	0.00302907064928623\\
43	0.00302909419811444\\
44	0.00302911817436465\\
45	0.00302914258579495\\
46	0.00302916744030412\\
47	0.00302919274593427\\
48	0.00302921851087336\\
49	0.00302924474345787\\
50	0.00302927145217546\\
51	0.00302929864566777\\
52	0.00302932633273314\\
53	0.00302935452232944\\
54	0.00302938322357701\\
55	0.00302941244576156\\
56	0.00302944219833715\\
57	0.00302947249092926\\
58	0.00302950333333786\\
59	0.00302953473554059\\
60	0.00302956670769596\\
61	0.00302959926014661\\
62	0.00302963240342263\\
63	0.00302966614824497\\
64	0.00302970050552887\\
65	0.00302973548638735\\
66	0.00302977110213484\\
67	0.00302980736429077\\
68	0.00302984428458324\\
69	0.00302988187495286\\
70	0.00302992014755654\\
71	0.0030299591147714\\
72	0.00302999878919874\\
73	0.0030300391836681\\
74	0.00303008031124134\\
75	0.00303012218521686\\
76	0.00303016481913389\\
77	0.0030302082267768\\
78	0.00303025242217946\\
79	0.00303029741962987\\
80	0.00303034323367461\\
81	0.00303038987912361\\
82	0.0030304373710548\\
83	0.00303048572481903\\
84	0.00303053495604487\\
85	0.00303058508064373\\
86	0.0030306361148149\\
87	0.00303068807505071\\
88	0.00303074097814188\\
89	0.00303079484118279\\
90	0.00303084968157703\\
91	0.00303090551704294\\
92	0.00303096236561924\\
93	0.00303102024567085\\
94	0.00303107917589472\\
95	0.00303113917532584\\
96	0.00303120026334325\\
97	0.0030312624596763\\
98	0.00303132578441093\\
99	0.00303139025799603\\
100	0.00303145590125002\\
101	0.00303152273536749\\
102	0.00303159078192597\\
103	0.00303166006289272\\
104	0.00303173060063185\\
105	0.0030318024179114\\
106	0.00303187553791054\\
107	0.00303194998422709\\
108	0.00303202578088487\\
109	0.00303210295234151\\
110	0.00303218152349608\\
111	0.00303226151969716\\
112	0.00303234296675079\\
113	0.00303242589092876\\
114	0.00303251031897693\\
115	0.00303259627812372\\
116	0.00303268379608879\\
117	0.00303277290109185\\
118	0.00303286362186161\\
119	0.00303295598764493\\
120	0.00303305002821607\\
121	0.00303314577388621\\
122	0.00303324325551295\\
123	0.00303334250451026\\
124	0.00303344355285829\\
125	0.00303354643311366\\
126	0.00303365117841963\\
127	0.00303375782251677\\
128	0.0030338663997535\\
129	0.00303397694509707\\
130	0.00303408949414461\\
131	0.00303420408313441\\
132	0.00303432074895733\\
133	0.00303443952916859\\
134	0.00303456046199954\\
135	0.00303468358636983\\
136	0.00303480894189968\\
137	0.00303493656892239\\
138	0.00303506650849714\\
139	0.00303519880242191\\
140	0.0030353334932467\\
141	0.003035470624287\\
142	0.0030356102396374\\
143	0.00303575238418558\\
144	0.00303589710362637\\
145	0.00303604444447623\\
146	0.00303619445408791\\
147	0.0030363471806654\\
148	0.00303650267327904\\
149	0.00303666098188105\\
150	0.00303682215732119\\
151	0.00303698625136286\\
152	0.00303715331669929\\
153	0.00303732340697017\\
154	0.00303749657677851\\
155	0.00303767288170779\\
156	0.00303785237833944\\
157	0.0030380351242706\\
158	0.00303822117813223\\
159	0.00303841059960749\\
160	0.00303860344945046\\
161	0.00303879978950521\\
162	0.00303899968272522\\
163	0.00303920319319303\\
164	0.00303941038614044\\
165	0.0030396213279688\\
166	0.00303983608626987\\
167	0.00304005472984703\\
168	0.00304027732873667\\
169	0.00304050395423017\\
170	0.00304073467889618\\
171	0.00304096957660327\\
172	0.00304120872254299\\
173	0.00304145219325335\\
174	0.00304170006664276\\
175	0.00304195242201417\\
176	0.00304220934008994\\
177	0.00304247090303695\\
178	0.00304273719449214\\
179	0.00304300829958858\\
180	0.00304328430498196\\
181	0.00304356529887753\\
182	0.0030438513710575\\
183	0.00304414261290895\\
184	0.00304443911745221\\
185	0.00304474097936965\\
186	0.00304504829503519\\
187	0.00304536116254405\\
188	0.00304567968174319\\
189	0.00304600395426218\\
190	0.00304633408354477\\
191	0.00304667017488073\\
192	0.00304701233543851\\
193	0.00304736067429829\\
194	0.00304771530248569\\
195	0.00304807633300599\\
196	0.00304844388087902\\
197	0.00304881806317457\\
198	0.00304919899904846\\
199	0.00304958680977917\\
200	0.00304998161880524\\
201	0.0030503835517631\\
202	0.00305079273652573\\
203	0.0030512093032419\\
204	0.00305163338437605\\
205	0.00305206511474893\\
206	0.00305250463157887\\
207	0.00305295207452379\\
208	0.00305340758572393\\
209	0.00305387130984524\\
210	0.00305434339412361\\
211	0.00305482398840977\\
212	0.00305531324521497\\
213	0.00305581131975746\\
214	0.00305631837000973\\
215	0.00305683455674661\\
216	0.003057360043594\\
217	0.0030578949970787\\
218	0.00305843958667887\\
219	0.00305899398487538\\
220	0.00305955836720408\\
221	0.0030601329123089\\
222	0.0030607178019959\\
223	0.00306131322128813\\
224	0.00306191935848148\\
225	0.00306253640520146\\
226	0.00306316455646089\\
227	0.00306380401071862\\
228	0.00306445496993916\\
229	0.00306511763965336\\
230	0.00306579222902001\\
231	0.00306647895088855\\
232	0.00306717802186287\\
233	0.003067889662366\\
234	0.00306861409670598\\
235	0.00306935155314283\\
236	0.00307010226395657\\
237	0.00307086646551644\\
238	0.0030716443983511\\
239	0.00307243630722022\\
240	0.00307324244118701\\
241	0.00307406305369208\\
242	0.00307489840262847\\
243	0.00307574875041794\\
244	0.00307661436408841\\
245	0.00307749551535274\\
246	0.00307839248068879\\
247	0.00307930554142075\\
248	0.00308023498380176\\
249	0.003081181099098\\
250	0.00308214418367396\\
251	0.00308312453907921\\
252	0.00308412247213652\\
253	0.00308513829503142\\
254	0.00308617232540319\\
255	0.00308722488643737\\
256	0.00308829630695964\\
257	0.00308938692153136\\
258	0.00309049707054654\\
259	0.00309162710033039\\
260	0.00309277736323956\\
261	0.0030939482177638\\
262	0.00309514002862941\\
263	0.00309635316690439\\
264	0.00309758801010511\\
265	0.00309884494230478\\
266	0.00310012435424364\\
267	0.00310142664344084\\
268	0.00310275221430782\\
269	0.00310410147826308\\
270	0.00310547485384705\\
271	0.00310687276683415\\
272	0.00310829565033333\\
273	0.00310974394485335\\
274	0.00311121809827172\\
275	0.00311271856556988\\
276	0.00311424580814105\\
277	0.00311580029304386\\
278	0.0031173824963509\\
279	0.00311899290588784\\
280	0.00312063201780131\\
281	0.00312230033668589\\
282	0.00312399837571202\\
283	0.00312572665675453\\
284	0.00312748571052208\\
285	0.00312927607668701\\
286	0.00313109830401566\\
287	0.00313295295049911\\
288	0.00313484058348401\\
289	0.00313676177980353\\
290	0.00313871712590803\\
291	0.00314070721799548\\
292	0.00314273266214129\\
293	0.00314479407442735\\
294	0.00314689208107\\
295	0.00314902731854667\\
296	0.00315120043372098\\
297	0.00315341208396572\\
298	0.00315566293728366\\
299	0.00315795367242572\\
300	0.00316028497900588\\
301	0.00316265755761281\\
302	0.0031650721199175\\
303	0.0031675293887765\\
304	0.00317003009833068\\
305	0.00317257499409899\\
306	0.00317516483306818\\
307	0.00317780038378063\\
308	0.00318048242642767\\
309	0.00318321175296873\\
310	0.00318598916732919\\
311	0.00318881548580932\\
312	0.00319169153801165\\
313	0.00319461816890939\\
314	0.00319759624290387\\
315	0.00320062664899632\\
316	0.0032037102951036\\
317	0.00320684807037445\\
318	0.00321004087627448\\
319	0.0032132896266077\\
320	0.00321659524753004\\
321	0.00321995867755515\\
322	0.00322338086755313\\
323	0.00322686278074274\\
324	0.00323040539267859\\
325	0.00323400969123477\\
326	0.0032376766765872\\
327	0.00324140736119793\\
328	0.00324520276980519\\
329	0.00324906393942445\\
330	0.00325299191936726\\
331	0.00325698777128638\\
332	0.00326105256925825\\
333	0.0032651873999171\\
334	0.00326939336265826\\
335	0.00327367156993406\\
336	0.00327802314767201\\
337	0.00328244923585473\\
338	0.00328695098931985\\
339	0.00329152957888654\\
340	0.0032961861930979\\
341	0.00330092204169816\\
342	0.00330573836608922\\
343	0.0033106364500463\\
344	0.00331561760769122\\
345	0.0033206831751721\\
346	0.00332583451126905\\
347	0.00333107299801417\\
348	0.00333640004138324\\
349	0.00334181707225653\\
350	0.00334732554826556\\
351	0.00335292695829203\\
352	0.00335862282034505\\
353	0.00336441467612646\\
354	0.00337030409122422\\
355	0.00337629265504074\\
356	0.00338238197997807\\
357	0.00338857369842472\\
358	0.00339486945307576\\
359	0.00340127086753418\\
360	0.00340777946832079\\
361	0.00341439661982143\\
362	0.00342112404744642\\
363	0.00342796349887596\\
364	0.00343491674355327\\
365	0.00344198557200682\\
366	0.00344917179497051\\
367	0.00345647724226561\\
368	0.00346390376140449\\
369	0.00347145321587229\\
370	0.00347912748304138\\
371	0.00348692845167971\\
372	0.00349485801903882\\
373	0.00350291808758462\\
374	0.0035111105616546\\
375	0.00351943734493847\\
376	0.00352790034136671\\
377	0.00353650146654665\\
378	0.00354524268871292\\
379	0.00355412614573979\\
380	0.00356315442598909\\
381	0.00357233094853862\\
382	0.00358165751084972\\
383	0.00359113580335864\\
384	0.00360076739601845\\
385	0.00361055370673163\\
386	0.00362049596272862\\
387	0.00363059515242785\\
388	0.00364085196232736\\
389	0.00365126668323878\\
390	0.0036618390373958\\
391	0.00367256814840234\\
392	0.00368345253156515\\
393	0.00369449000001941\\
394	0.00370567753454716\\
395	0.00371701112855533\\
396	0.00372848560301563\\
397	0.00374009438497698\\
398	0.00375182924177285\\
399	0.00376367996118231\\
400	0.00377563396545737\\
401	0.00378767584415013\\
402	0.00379978678682925\\
403	0.00381194389164227\\
404	0.00382411931833971\\
405	0.00383627924249317\\
406	0.00384838254476245\\
407	0.00386037912688761\\
408	0.00387220770166266\\
409	0.00388379162842935\\
410	0.00389502823215165\\
411	0.0039058053770272\\
412	0.00391598628688462\\
413	0.00392540432342172\\
414	0.00393391703171463\\
415	0.00394256687755733\\
416	0.00395135437222172\\
417	0.00396027978814989\\
418	0.00396934311056532\\
419	0.00397854398002191\\
420	0.00398788162754897\\
421	0.00399735481235831\\
422	0.00400696179881135\\
423	0.00401670049483978\\
424	0.00402656915525881\\
425	0.00403656904293239\\
426	0.00404671444866124\\
427	0.00405703284297276\\
428	0.00406752627570502\\
429	0.00407819676557926\\
430	0.00408904629520965\\
431	0.00410007680565582\\
432	0.00411129019047086\\
433	0.00412268828918791\\
434	0.00413427288017187\\
435	0.0041460456727693\\
436	0.00415800829871378\\
437	0.00417016230271938\\
438	0.00418250913206429\\
439	0.00419505012502878\\
440	0.00420778649786106\\
441	0.00422071932930983\\
442	0.00423384954243987\\
443	0.00424717788626631\\
444	0.00426070491824429\\
445	0.00427443098595149\\
446	0.00428835620833312\\
447	0.00430248045747204\\
448	0.00431680334269483\\
449	0.00433132419952804\\
450	0.00434604209504468\\
451	0.00436095589945023\\
452	0.004376064595495\\
453	0.00439136721676721\\
454	0.00440686288610909\\
455	0.00442255099897275\\
456	0.004438431480289\\
457	0.00445450512939587\\
458	0.00447077370408318\\
459	0.00448723486789741\\
460	0.00450388630057895\\
461	0.00452072524652081\\
462	0.00453774909302376\\
463	0.00455495611894913\\
464	0.00457234607125751\\
465	0.00458992095339337\\
466	0.00460768605450424\\
467	0.00462565133655255\\
468	0.0046438333617516\\
469	0.00466225783770557\\
470	0.00468096298585217\\
471	0.00470000401649422\\
472	0.00471945906677093\\
473	0.00473938767759429\\
474	0.00475981409024856\\
475	0.00478075575623904\\
476	0.00480223100153094\\
477	0.00482425856778606\\
478	0.0048468580439876\\
479	0.00487005002816076\\
480	0.00489385620104787\\
481	0.00491829940194202\\
482	0.00494340368502003\\
483	0.00496919421239836\\
484	0.00499569775861788\\
485	0.00502294511464872\\
486	0.00505096981172665\\
487	0.0050798091614902\\
488	0.00510949778355719\\
489	0.0051400699821511\\
490	0.00517156102005021\\
491	0.00520400707139461\\
492	0.0052374451943808\\
493	0.00527191332472441\\
494	0.00530745031577766\\
495	0.00534409604514044\\
496	0.00538189161588839\\
497	0.00542087969115504\\
498	0.00546110512910798\\
499	0.0055026159982025\\
500	0.00554546521133257\\
501	0.0055897269789056\\
502	0.00563547697008804\\
503	0.00568278516608635\\
504	0.00573171148097553\\
505	0.00578229947832214\\
506	0.00583456775120688\\
507	0.00588849998078259\\
508	0.00594403332644779\\
509	0.0060010412253308\\
510	0.00605931006271971\\
511	0.00611851027633337\\
512	0.00617747791830166\\
513	0.00623520586116127\\
514	0.00629150581262778\\
515	0.00634618044132797\\
516	0.0063990274318587\\
517	0.00644984572907789\\
518	0.00649844476803919\\
519	0.00654465776229582\\
520	0.00658836033287184\\
521	0.00662949619949569\\
522	0.00666811281657877\\
523	0.00670481814336649\\
524	0.00674090081281494\\
525	0.00677635894202173\\
526	0.00681120429792292\\
527	0.00684546463234754\\
528	0.00687918591593404\\
529	0.00691243422022874\\
530	0.00694529688123025\\
531	0.00697788240868995\\
532	0.00701031833327823\\
533	0.00704274576437311\\
534	0.00707530468489793\\
535	0.00710805919650755\\
536	0.00714104422029622\\
537	0.00717429851413554\\
538	0.00720786425663497\\
539	0.00724178641245666\\
540	0.00727611185933979\\
541	0.00731088827640736\\
542	0.00734616282644195\\
543	0.00738198070843317\\
544	0.00741838376052503\\
545	0.00745540942216023\\
546	0.00749309307772818\\
547	0.00753147042725658\\
548	0.00757057761092794\\
549	0.00761045061691577\\
550	0.0076511245205872\\
551	0.00769263289885326\\
552	0.00773499993678999\\
553	0.00777824368278787\\
554	0.00782238079408664\\
555	0.00786742681565397\\
556	0.00791339629805104\\
557	0.00796030292073503\\
558	0.00800815943839331\\
559	0.00805697750780174\\
560	0.00810676749818279\\
561	0.00815753828243189\\
562	0.00820929699943151\\
563	0.0082620488236965\\
564	0.00831579673842313\\
565	0.00837054130588479\\
566	0.00842628054133109\\
567	0.00848300995809075\\
568	0.00854071929184885\\
569	0.00859939067377092\\
570	0.00865899639130592\\
571	0.00871947851713435\\
572	0.00878076518693766\\
573	0.00884291437470965\\
574	0.00890600938256045\\
575	0.00897017009872411\\
576	0.0090329788948814\\
577	0.00909416288432404\\
578	0.00915329402632274\\
579	0.00920930753183795\\
580	0.00926288552407399\\
581	0.00931447511044903\\
582	0.00936389576540479\\
583	0.00941168254089778\\
584	0.00945797801032644\\
585	0.00950288536810165\\
586	0.00954684401204475\\
587	0.00959001465926593\\
588	0.009632483040377\\
589	0.00967428778528623\\
590	0.00971542443559021\\
591	0.00975588428269145\\
592	0.00979566256029117\\
593	0.0098347658551079\\
594	0.00987312984006299\\
595	0.00991038882774525\\
596	0.00994573390547615\\
597	0.0099771937715668\\
598	0.00999970795535495\\
599	0\\
600	0\\
};
\addplot [color=mycolor19,solid,forget plot]
  table[row sep=crcr]{%
1	0.00370774725980159\\
2	0.00370775285074728\\
3	0.00370775854349141\\
4	0.0037077643398901\\
5	0.00370777024183357\\
6	0.00370777625124661\\
7	0.00370778237008943\\
8	0.00370778860035815\\
9	0.00370779494408551\\
10	0.00370780140334163\\
11	0.00370780798023459\\
12	0.00370781467691122\\
13	0.00370782149555769\\
14	0.00370782843840041\\
15	0.00370783550770665\\
16	0.00370784270578533\\
17	0.00370785003498786\\
18	0.00370785749770877\\
19	0.00370786509638668\\
20	0.00370787283350505\\
21	0.00370788071159295\\
22	0.00370788873322599\\
23	0.00370789690102714\\
24	0.00370790521766767\\
25	0.00370791368586793\\
26	0.00370792230839837\\
27	0.00370793108808039\\
28	0.00370794002778736\\
29	0.00370794913044552\\
30	0.00370795839903501\\
31	0.00370796783659081\\
32	0.0037079774462038\\
33	0.00370798723102183\\
34	0.00370799719425068\\
35	0.00370800733915524\\
36	0.00370801766906051\\
37	0.00370802818735279\\
38	0.00370803889748079\\
39	0.00370804980295677\\
40	0.00370806090735776\\
41	0.00370807221432673\\
42	0.00370808372757378\\
43	0.00370809545087748\\
44	0.00370810738808608\\
45	0.00370811954311878\\
46	0.0037081319199671\\
47	0.00370814452269614\\
48	0.00370815735544609\\
49	0.00370817042243348\\
50	0.00370818372795265\\
51	0.0037081972763772\\
52	0.00370821107216143\\
53	0.00370822511984191\\
54	0.0037082394240389\\
55	0.003708253989458\\
56	0.00370826882089166\\
57	0.0037082839232208\\
58	0.00370829930141651\\
59	0.00370831496054165\\
60	0.0037083309057526\\
61	0.00370834714230094\\
62	0.00370836367553529\\
63	0.00370838051090304\\
64	0.0037083976539522\\
65	0.00370841511033331\\
66	0.00370843288580124\\
67	0.00370845098621722\\
68	0.00370846941755074\\
69	0.00370848818588162\\
70	0.00370850729740203\\
71	0.0037085267584185\\
72	0.00370854657535414\\
73	0.00370856675475073\\
74	0.00370858730327099\\
75	0.00370860822770072\\
76	0.00370862953495113\\
77	0.00370865123206115\\
78	0.00370867332619983\\
79	0.00370869582466869\\
80	0.00370871873490419\\
81	0.00370874206448024\\
82	0.00370876582111075\\
83	0.00370879001265215\\
84	0.00370881464710616\\
85	0.00370883973262228\\
86	0.00370886527750067\\
87	0.00370889129019495\\
88	0.00370891777931489\\
89	0.00370894475362944\\
90	0.00370897222206957\\
91	0.00370900019373133\\
92	0.00370902867787887\\
93	0.00370905768394751\\
94	0.00370908722154696\\
95	0.00370911730046449\\
96	0.00370914793066823\\
97	0.0037091791223105\\
98	0.00370921088573119\\
99	0.00370924323146131\\
100	0.00370927617022635\\
101	0.00370930971294998\\
102	0.00370934387075767\\
103	0.00370937865498044\\
104	0.00370941407715861\\
105	0.00370945014904564\\
106	0.00370948688261215\\
107	0.00370952429004976\\
108	0.00370956238377534\\
109	0.003709601176435\\
110	0.0037096406809084\\
111	0.00370968091031304\\
112	0.00370972187800857\\
113	0.00370976359760137\\
114	0.00370980608294895\\
115	0.00370984934816467\\
116	0.00370989340762235\\
117	0.00370993827596115\\
118	0.00370998396809042\\
119	0.00371003049919463\\
120	0.00371007788473846\\
121	0.0037101261404719\\
122	0.00371017528243558\\
123	0.00371022532696602\\
124	0.00371027629070111\\
125	0.00371032819058558\\
126	0.00371038104387678\\
127	0.00371043486815016\\
128	0.00371048968130542\\
129	0.00371054550157219\\
130	0.00371060234751623\\
131	0.00371066023804553\\
132	0.00371071919241662\\
133	0.00371077923024095\\
134	0.00371084037149136\\
135	0.00371090263650874\\
136	0.00371096604600877\\
137	0.00371103062108875\\
138	0.00371109638323462\\
139	0.00371116335432808\\
140	0.0037112315566538\\
141	0.00371130101290682\\
142	0.00371137174620002\\
143	0.00371144378007182\\
144	0.00371151713849395\\
145	0.00371159184587932\\
146	0.00371166792709015\\
147	0.00371174540744613\\
148	0.00371182431273281\\
149	0.00371190466921006\\
150	0.00371198650362085\\
151	0.0037120698431999\\
152	0.00371215471568282\\
153	0.0037122411493151\\
154	0.00371232917286155\\
155	0.00371241881561567\\
156	0.00371251010740934\\
157	0.00371260307862261\\
158	0.00371269776019371\\
159	0.00371279418362919\\
160	0.00371289238101432\\
161	0.00371299238502354\\
162	0.00371309422893128\\
163	0.00371319794662282\\
164	0.00371330357260537\\
165	0.00371341114201949\\
166	0.00371352069065049\\
167	0.00371363225494017\\
168	0.00371374587199876\\
169	0.00371386157961711\\
170	0.0037139794162789\\
171	0.00371409942117336\\
172	0.00371422163420798\\
173	0.0037143460960215\\
174	0.00371447284799725\\
175	0.00371460193227658\\
176	0.0037147333917726\\
177	0.00371486727018411\\
178	0.00371500361200984\\
179	0.00371514246256291\\
180	0.00371528386798552\\
181	0.00371542787526392\\
182	0.00371557453224369\\
183	0.00371572388764516\\
184	0.00371587599107926\\
185	0.00371603089306352\\
186	0.00371618864503836\\
187	0.00371634929938376\\
188	0.00371651290943616\\
189	0.00371667952950557\\
190	0.0037168492148931\\
191	0.00371702202190877\\
192	0.00371719800788955\\
193	0.00371737723121779\\
194	0.00371755975133997\\
195	0.00371774562878565\\
196	0.00371793492518691\\
197	0.00371812770329801\\
198	0.00371832402701543\\
199	0.00371852396139824\\
200	0.00371872757268879\\
201	0.00371893492833376\\
202	0.00371914609700563\\
203	0.00371936114862442\\
204	0.00371958015437981\\
205	0.00371980318675368\\
206	0.00372003031954299\\
207	0.00372026162788308\\
208	0.00372049718827116\\
209	0.00372073707859057\\
210	0.00372098137813506\\
211	0.00372123016763367\\
212	0.003721483529276\\
213	0.0037217415467378\\
214	0.00372200430520705\\
215	0.0037222718914105\\
216	0.00372254439364055\\
217	0.00372282190178258\\
218	0.00372310450734271\\
219	0.00372339230347611\\
220	0.00372368538501556\\
221	0.00372398384850068\\
222	0.0037242877922074\\
223	0.00372459731617809\\
224	0.00372491252225198\\
225	0.00372523351409626\\
226	0.00372556039723741\\
227	0.00372589327909318\\
228	0.00372623226900502\\
229	0.00372657747827098\\
230	0.00372692902017913\\
231	0.00372728701004149\\
232	0.00372765156522834\\
233	0.00372802280520326\\
234	0.00372840085155847\\
235	0.00372878582805087\\
236	0.00372917786063841\\
237	0.00372957707751717\\
238	0.00372998360915887\\
239	0.00373039758834892\\
240	0.00373081915022508\\
241	0.00373124843231657\\
242	0.00373168557458385\\
243	0.00373213071945883\\
244	0.00373258401188581\\
245	0.0037330455993628\\
246	0.00373351563198372\\
247	0.00373399426248091\\
248	0.00373448164626847\\
249	0.00373497794148611\\
250	0.0037354833090438\\
251	0.00373599791266707\\
252	0.00373652191894298\\
253	0.00373705549736696\\
254	0.00373759882039051\\
255	0.00373815206346966\\
256	0.00373871540511456\\
257	0.00373928902693999\\
258	0.00373987311371715\\
259	0.00374046785342664\\
260	0.00374107343731284\\
261	0.00374169005993994\\
262	0.00374231791924943\\
263	0.00374295721661959\\
264	0.00374360815692694\\
265	0.00374427094860952\\
266	0.00374494580373236\\
267	0.0037456329380547\\
268	0.00374633257109861\\
269	0.00374704492621789\\
270	0.00374777023066485\\
271	0.00374850871564893\\
272	0.00374926061637186\\
273	0.00375002617199477\\
274	0.00375080562540618\\
275	0.00375159922239119\\
276	0.00375240720900068\\
277	0.00375322982386502\\
278	0.00375406728101007\\
279	0.00375491980008241\\
280	0.00375578764372609\\
281	0.00375667107882098\\
282	0.00375757037653577\\
283	0.00375848581238061\\
284	0.00375941766625949\\
285	0.00376036622252202\\
286	0.00376133177001474\\
287	0.0037623146021315\\
288	0.00376331501686291\\
289	0.00376433331684438\\
290	0.00376536980940285\\
291	0.00376642480660149\\
292	0.00376749862528214\\
293	0.00376859158710497\\
294	0.00376970401858495\\
295	0.0037708362511243\\
296	0.00377198862104032\\
297	0.0037731614695878\\
298	0.00377435514297501\\
299	0.00377556999237241\\
300	0.00377680637391298\\
301	0.00377806464868323\\
302	0.00377934518270398\\
303	0.00378064834690048\\
304	0.00378197451706164\\
305	0.00378332407378998\\
306	0.0037846974024458\\
307	0.00378609489309412\\
308	0.00378751694047521\\
309	0.00378896394404684\\
310	0.00379043630822041\\
311	0.00379193444310449\\
312	0.00379345876658225\\
313	0.00379500970989577\\
314	0.00379658773227283\\
315	0.00379819335722322\\
316	0.00379982724749577\\
317	0.00380149022344091\\
318	0.00380318277068881\\
319	0.00380490538161127\\
320	0.0038066585553709\\
321	0.00380844279797141\\
322	0.00381025862230934\\
323	0.00381210654822796\\
324	0.00381398710257377\\
325	0.00381590081925616\\
326	0.00381784823931141\\
327	0.00381982991097145\\
328	0.00382184638973867\\
329	0.00382389823846755\\
330	0.00382598602745455\\
331	0.00382811033453736\\
332	0.00383027174520502\\
333	0.00383247085272003\\
334	0.00383470825825402\\
335	0.00383698457103804\\
336	0.00383930040852735\\
337	0.00384165639657982\\
338	0.00384405316964468\\
339	0.00384649137096169\\
340	0.00384897165279837\\
341	0.00385149467681659\\
342	0.00385406111430279\\
343	0.00385667164583049\\
344	0.00385932696127703\\
345	0.00386202776021929\\
346	0.00386477475244102\\
347	0.00386756865858514\\
348	0.00387041021099452\\
349	0.00387330015479153\\
350	0.00387623924923149\\
351	0.00387922826935019\\
352	0.00388226800808929\\
353	0.00388535927917676\\
354	0.00388850292087259\\
355	0.00389169980062951\\
356	0.00389495082056812\\
357	0.00389825692272083\\
358	0.00390161908920078\\
359	0.00390503831660652\\
360	0.00390851547084551\\
361	0.00391205052607056\\
362	0.00391564062560171\\
363	0.0039192862141833\\
364	0.00392298769992046\\
365	0.00392674544836104\\
366	0.00393055977558642\\
367	0.00393443094010376\\
368	0.00393835913328275\\
369	0.00394234446801822\\
370	0.00394638696522411\\
371	0.00395048653767655\\
372	0.00395464297063303\\
373	0.00395885589860197\\
374	0.00396312477775013\\
375	0.00396744885411017\\
376	0.00397182713018891\\
377	0.0039762583404722\\
378	0.00398074097125978\\
379	0.00398527344009837\\
380	0.00398985481702349\\
381	0.00399448744356757\\
382	0.00399918702415652\\
383	0.00400395374011795\\
384	0.00400878755654688\\
385	0.00401368833832755\\
386	0.0040186558402524\\
387	0.00402368969664548\\
388	0.00402878941069664\\
389	0.00403395434399141\\
390	0.0040391837075818\\
391	0.00404447655377464\\
392	0.00404983176584848\\
393	0.00405524804770437\\
394	0.00406072391450447\\
395	0.00406625768498511\\
396	0.00407184747637465\\
397	0.00407749120316849\\
398	0.00408318658143646\\
399	0.00408893114089407\\
400	0.00409472224769767\\
401	0.00410055714188188\\
402	0.00410643299461499\\
403	0.00411234699210283\\
404	0.00411829645515665\\
405	0.00412427900635643\\
406	0.00413029280069661\\
407	0.00413633684087422\\
408	0.00414241140317777\\
409	0.00414851861368006\\
410	0.00415466324757146\\
411	0.00416085389820221\\
412	0.0041671041987995\\
413	0.00417343462776469\\
414	0.00417987382238871\\
415	0.0041864383085485\\
416	0.004193130669954\\
417	0.00419995356930276\\
418	0.00420690975524765\\
419	0.00421400207096304\\
420	0.00422123346479872\\
421	0.00422860700361297\\
422	0.00423612588937687\\
423	0.00424379347919162\\
424	0.0042516133069473\\
425	0.00425958909775831\\
426	0.00426772474066566\\
427	0.00427602409957658\\
428	0.00428449080175241\\
429	0.00429312858546607\\
430	0.00430194130554165\\
431	0.00431093293928129\\
432	0.00432010759281875\\
433	0.00432946950794833\\
434	0.00433902306946049\\
435	0.00434877281287252\\
436	0.00435872343253877\\
437	0.00436887979053711\\
438	0.00437924692684946\\
439	0.00438983006992397\\
440	0.0044006346481212\\
441	0.00441166630217564\\
442	0.00442293089885492\\
443	0.00443443454599071\\
444	0.00444618360914368\\
445	0.00445818473027138\\
446	0.00447044484713513\\
447	0.00448297121265312\\
448	0.00449577141647535\\
449	0.00450885341594247\\
450	0.00452222556599645\\
451	0.00453589665078687\\
452	0.00454987591430312\\
453	0.00456417309517411\\
454	0.0045787984650985\\
455	0.00459376287157062\\
456	0.00460907779050298\\
457	0.00462475533350661\\
458	0.00464080825653633\\
459	0.0046572500923446\\
460	0.00467409523580149\\
461	0.00469135902580906\\
462	0.00470905782699198\\
463	0.00472720910829467\\
464	0.00474583151982263\\
465	0.004764945007161\\
466	0.00478457105546634\\
467	0.00480473250036142\\
468	0.00482545305244929\\
469	0.00484675749753368\\
470	0.00486867165465498\\
471	0.00489122204384054\\
472	0.00491443527086963\\
473	0.00493833810209281\\
474	0.0049629581911663\\
475	0.00498832557828856\\
476	0.00501447391319709\\
477	0.00504144009062676\\
478	0.00506925741142093\\
479	0.00509795886157181\\
480	0.00512757795960136\\
481	0.00515814867236376\\
482	0.00518970535665398\\
483	0.00522228275738427\\
484	0.00525591610606085\\
485	0.00529064133492093\\
486	0.00532649549571177\\
487	0.00536351754362962\\
488	0.00540174964165694\\
489	0.00544124498503406\\
490	0.00548206991791641\\
491	0.00552428741703136\\
492	0.00556795394712316\\
493	0.00561311510864539\\
494	0.0056597999935744\\
495	0.00570801366404955\\
496	0.00575772704840013\\
497	0.00580886355070153\\
498	0.00586128135389376\\
499	0.00591475005480695\\
500	0.00596891965519479\\
501	0.00602224078756516\\
502	0.00607430541547943\\
503	0.00612493655162792\\
504	0.00617394793527647\\
505	0.00622115025590455\\
506	0.00626635772595491\\
507	0.00630939751224373\\
508	0.00635012319693875\\
509	0.0063884336455997\\
510	0.006424299242059\\
511	0.00645779751263656\\
512	0.00648985887878153\\
513	0.0065212909655835\\
514	0.00655209447491636\\
515	0.00658228317133356\\
516	0.00661188597703505\\
517	0.0066409489001781\\
518	0.00666953653677782\\
519	0.00669773275867687\\
520	0.00672564002782952\\
521	0.00675337654140902\\
522	0.00678107006430719\\
523	0.0068088302391249\\
524	0.00683670581356637\\
525	0.00686472515311536\\
526	0.00689291942367655\\
527	0.0069213221216492\\
528	0.00694996841419969\\
529	0.00697889428131792\\
530	0.00700813547176763\\
531	0.00703772631932867\\
532	0.00706769852177207\\
533	0.00709808007400404\\
534	0.0071288948681267\\
535	0.00716016583566274\\
536	0.00719191687720358\\
537	0.00722417277820445\\
538	0.00725695912297007\\
539	0.0072903022146856\\
540	0.00732422901093675\\
541	0.0073587670848767\\
542	0.00739394462109925\\
543	0.00742979045103049\\
544	0.0074663341223274\\
545	0.00750360597705983\\
546	0.00754163706134009\\
547	0.00758045881103562\\
548	0.00762010257783498\\
549	0.00766059897626249\\
550	0.00770197741878033\\
551	0.00774425545375649\\
552	0.0077874484398888\\
553	0.00783157106649422\\
554	0.00787663735344774\\
555	0.00792266052771767\\
556	0.00796965286556778\\
557	0.00801762549231267\\
558	0.00806658813049182\\
559	0.00811654885776477\\
560	0.00816751384335804\\
561	0.0082194870649378\\
562	0.00827247057381776\\
563	0.0083264621733809\\
564	0.00838145312782495\\
565	0.0084374259061988\\
566	0.00849434126395024\\
567	0.00855209945312733\\
568	0.00861074676539497\\
569	0.00867036466855077\\
570	0.0087310632366218\\
571	0.00879301394938548\\
572	0.00885417706724721\\
573	0.00891386297593653\\
574	0.00897164060929281\\
575	0.00902671138084478\\
576	0.00908054156668914\\
577	0.00913262846411269\\
578	0.00918281655212943\\
579	0.0092315474226628\\
580	0.00927941704386785\\
581	0.00932608920321522\\
582	0.00937167546725341\\
583	0.00941645926763461\\
584	0.00946071491381077\\
585	0.0095044908654121\\
586	0.00954778452593938\\
587	0.0095905700628325\\
588	0.00963281167868831\\
589	0.00967447054637906\\
590	0.00971551367588456\\
591	0.00975591786564725\\
592	0.00979567177304212\\
593	0.00983476678232329\\
594	0.00987312984006299\\
595	0.00991038882774525\\
596	0.00994573390547615\\
597	0.0099771937715668\\
598	0.00999970795535495\\
599	0\\
600	0\\
};
\addplot [color=red!50!mycolor17,solid,forget plot]
  table[row sep=crcr]{%
1	0.00389736634792342\\
2	0.0038973698205401\\
3	0.00389737335663963\\
4	0.0038973769573813\\
5	0.00389738062394577\\
6	0.0038973843575356\\
7	0.00389738815937552\\
8	0.00389739203071294\\
9	0.00389739597281837\\
10	0.0038973999869858\\
11	0.00389740407453324\\
12	0.00389740823680307\\
13	0.00389741247516264\\
14	0.00389741679100462\\
15	0.00389742118574756\\
16	0.00389742566083632\\
17	0.00389743021774259\\
18	0.00389743485796549\\
19	0.003897439583032\\
20	0.00389744439449745\\
21	0.00389744929394619\\
22	0.00389745428299206\\
23	0.003897459363279\\
24	0.00389746453648154\\
25	0.00389746980430549\\
26	0.00389747516848852\\
27	0.00389748063080071\\
28	0.00389748619304524\\
29	0.00389749185705896\\
30	0.00389749762471306\\
31	0.00389750349791381\\
32	0.0038975094786031\\
33	0.0038975155687592\\
34	0.00389752177039746\\
35	0.00389752808557101\\
36	0.0038975345163715\\
37	0.0038975410649298\\
38	0.00389754773341678\\
39	0.00389755452404413\\
40	0.00389756143906508\\
41	0.00389756848077517\\
42	0.00389757565151321\\
43	0.00389758295366194\\
44	0.00389759038964901\\
45	0.00389759796194776\\
46	0.00389760567307816\\
47	0.00389761352560769\\
48	0.00389762152215224\\
49	0.00389762966537704\\
50	0.00389763795799768\\
51	0.00389764640278098\\
52	0.00389765500254609\\
53	0.0038976637601654\\
54	0.0038976726785656\\
55	0.0038976817607288\\
56	0.00389769100969351\\
57	0.00389770042855577\\
58	0.00389771002047026\\
59	0.00389771978865146\\
60	0.0038977297363747\\
61	0.0038977398669775\\
62	0.00389775018386061\\
63	0.00389776069048938\\
64	0.0038977713903949\\
65	0.00389778228717531\\
66	0.00389779338449712\\
67	0.00389780468609649\\
68	0.00389781619578063\\
69	0.0038978279174291\\
70	0.00389783985499525\\
71	0.00389785201250772\\
72	0.00389786439407172\\
73	0.00389787700387069\\
74	0.00389788984616773\\
75	0.00389790292530709\\
76	0.0038979162457159\\
77	0.00389792981190555\\
78	0.00389794362847351\\
79	0.00389795770010493\\
80	0.0038979720315743\\
81	0.00389798662774719\\
82	0.00389800149358205\\
83	0.00389801663413194\\
84	0.00389803205454643\\
85	0.00389804776007337\\
86	0.00389806375606092\\
87	0.00389808004795931\\
88	0.00389809664132298\\
89	0.00389811354181252\\
90	0.00389813075519668\\
91	0.0038981482873545\\
92	0.0038981661442774\\
93	0.00389818433207144\\
94	0.00389820285695938\\
95	0.00389822172528308\\
96	0.00389824094350567\\
97	0.00389826051821396\\
98	0.00389828045612083\\
99	0.00389830076406753\\
100	0.00389832144902633\\
101	0.00389834251810289\\
102	0.00389836397853896\\
103	0.00389838583771488\\
104	0.00389840810315226\\
105	0.00389843078251676\\
106	0.00389845388362081\\
107	0.00389847741442639\\
108	0.00389850138304799\\
109	0.00389852579775546\\
110	0.00389855066697705\\
111	0.00389857599930235\\
112	0.00389860180348553\\
113	0.00389862808844831\\
114	0.00389865486328337\\
115	0.00389868213725741\\
116	0.0038987099198147\\
117	0.00389873822058033\\
118	0.00389876704936367\\
119	0.00389879641616201\\
120	0.00389882633116396\\
121	0.00389885680475334\\
122	0.00389888784751269\\
123	0.00389891947022726\\
124	0.0038989516838887\\
125	0.00389898449969912\\
126	0.00389901792907507\\
127	0.00389905198365164\\
128	0.00389908667528656\\
129	0.00389912201606456\\
130	0.00389915801830162\\
131	0.00389919469454937\\
132	0.00389923205759959\\
133	0.00389927012048882\\
134	0.00389930889650297\\
135	0.00389934839918209\\
136	0.00389938864232519\\
137	0.00389942963999514\\
138	0.00389947140652378\\
139	0.00389951395651692\\
140	0.0038995573048596\\
141	0.00389960146672138\\
142	0.00389964645756178\\
143	0.00389969229313573\\
144	0.00389973898949917\\
145	0.00389978656301486\\
146	0.00389983503035809\\
147	0.00389988440852261\\
148	0.00389993471482679\\
149	0.00389998596691965\\
150	0.00390003818278713\\
151	0.00390009138075858\\
152	0.00390014557951309\\
153	0.00390020079808633\\
154	0.00390025705587712\\
155	0.00390031437265439\\
156	0.00390037276856414\\
157	0.00390043226413663\\
158	0.00390049288029356\\
159	0.0039005546383556\\
160	0.00390061756004982\\
161	0.00390068166751742\\
162	0.00390074698332155\\
163	0.0039008135304553\\
164	0.00390088133234975\\
165	0.00390095041288234\\
166	0.00390102079638526\\
167	0.003901092507654\\
168	0.00390116557195617\\
169	0.00390124001504028\\
170	0.00390131586314495\\
171	0.0039013931430081\\
172	0.00390147188187634\\
173	0.00390155210751463\\
174	0.00390163384821593\\
175	0.00390171713281131\\
176	0.00390180199067997\\
177	0.00390188845175957\\
178	0.00390197654655685\\
179	0.00390206630615819\\
180	0.00390215776224066\\
181	0.00390225094708309\\
182	0.00390234589357738\\
183	0.00390244263524002\\
184	0.00390254120622388\\
185	0.00390264164133013\\
186	0.00390274397602053\\
187	0.00390284824642968\\
188	0.00390295448937779\\
189	0.0039030627423836\\
190	0.00390317304367732\\
191	0.00390328543221417\\
192	0.00390339994768793\\
193	0.00390351663054479\\
194	0.00390363552199747\\
195	0.00390375666403967\\
196	0.00390388009946066\\
197	0.00390400587186024\\
198	0.0039041340256639\\
199	0.00390426460613837\\
200	0.00390439765940735\\
201	0.00390453323246763\\
202	0.00390467137320542\\
203	0.00390481213041306\\
204	0.00390495555380594\\
205	0.00390510169403992\\
206	0.00390525060272893\\
207	0.00390540233246284\\
208	0.00390555693682597\\
209	0.00390571447041555\\
210	0.00390587498886083\\
211	0.00390603854884248\\
212	0.00390620520811224\\
213	0.00390637502551312\\
214	0.00390654806099997\\
215	0.00390672437566022\\
216	0.00390690403173535\\
217	0.00390708709264261\\
218	0.00390727362299718\\
219	0.00390746368863474\\
220	0.00390765735663467\\
221	0.00390785469534347\\
222	0.0039080557743988\\
223	0.00390826066475405\\
224	0.0039084694387034\\
225	0.00390868216990731\\
226	0.00390889893341869\\
227	0.00390911980570959\\
228	0.00390934486469853\\
229	0.0039095741897783\\
230	0.00390980786184463\\
231	0.00391004596332535\\
232	0.00391028857821035\\
233	0.00391053579208221\\
234	0.00391078769214775\\
235	0.00391104436727017\\
236	0.0039113059080023\\
237	0.00391157240662061\\
238	0.00391184395716029\\
239	0.00391212065545136\\
240	0.0039124025991559\\
241	0.00391268988780645\\
242	0.00391298262284592\\
243	0.00391328090766864\\
244	0.00391358484766326\\
245	0.00391389455025719\\
246	0.00391421012496301\\
247	0.00391453168342704\\
248	0.00391485933948022\\
249	0.0039151932091917\\
250	0.00391553341092541\\
251	0.00391588006540011\\
252	0.00391623329575343\\
253	0.00391659322761038\\
254	0.00391695998915727\\
255	0.00391733371122172\\
256	0.00391771452736018\\
257	0.00391810257395385\\
258	0.00391849799031516\\
259	0.00391890091880645\\
260	0.00391931150497363\\
261	0.00391972989769786\\
262	0.00392015624936923\\
263	0.00392059071608725\\
264	0.0039210334578942\\
265	0.00392148463904928\\
266	0.00392194442835299\\
267	0.00392241299953364\\
268	0.00392289053171176\\
269	0.00392337720996081\\
270	0.00392387322598818\\
271	0.0039243787789641\\
272	0.00392489407652749\\
273	0.00392541933598408\\
274	0.0039259547856397\\
275	0.00392650066591704\\
276	0.00392705722871356\\
277	0.0039276247283476\\
278	0.00392820337258427\\
279	0.00392879305377505\\
280	0.00392939356584512\\
281	0.0039300050948801\\
282	0.00393062782927172\\
283	0.00393126195966689\\
284	0.00393190767890577\\
285	0.00393256518194811\\
286	0.00393323466578536\\
287	0.00393391632933697\\
288	0.0039346103733282\\
289	0.00393531700014664\\
290	0.00393603641367378\\
291	0.00393676881908754\\
292	0.00393751442263075\\
293	0.00393827343133943\\
294	0.00393904605272357\\
295	0.00393983249439113\\
296	0.00394063296360454\\
297	0.00394144766675594\\
298	0.00394227680874452\\
299	0.00394312059223514\\
300	0.00394397921677281\\
301	0.00394485287772095\\
302	0.0039457417649836\\
303	0.00394664606146187\\
304	0.0039475659411825\\
305	0.00394850156702071\\
306	0.00394945308792055\\
307	0.00395042063549458\\
308	0.00395140431986339\\
309	0.00395240422458772\\
310	0.00395342040059171\\
311	0.00395445285919665\\
312	0.0039555015651512\\
313	0.0039565664330002\\
314	0.00395764733789326\\
315	0.00395874417703582\\
316	0.00395985710401096\\
317	0.00396098738962125\\
318	0.00396213886171391\\
319	0.00396331190676099\\
320	0.00396450691791733\\
321	0.00396572429512668\\
322	0.00396696444522947\\
323	0.00396822778207173\\
324	0.00396951472661552\\
325	0.00397082570705071\\
326	0.00397216115890792\\
327	0.00397352152517256\\
328	0.00397490725639991\\
329	0.00397631881083108\\
330	0.00397775665450973\\
331	0.0039792212613991\\
332	0.00398071311349925\\
333	0.00398223270096437\\
334	0.00398378052221951\\
335	0.00398535708407654\\
336	0.00398696290184869\\
337	0.00398859849946344\\
338	0.00399026440957306\\
339	0.00399196117366264\\
340	0.003993689342155\\
341	0.00399544947451154\\
342	0.00399724213932856\\
343	0.00399906791443525\\
344	0.0040009273869971\\
345	0.00400282115361412\\
346	0.00400474982041147\\
347	0.00400671400311934\\
348	0.00400871432713766\\
349	0.00401075142757921\\
350	0.00401282594928307\\
351	0.00401493854678837\\
352	0.00401708988425414\\
353	0.00401928063530098\\
354	0.00402151148274127\\
355	0.00402378311815706\\
356	0.00402609624126987\\
357	0.0040284515590339\\
358	0.00403084978438118\\
359	0.0040332916346042\\
360	0.00403577782969322\\
361	0.00403830909258915\\
362	0.00404088616161536\\
363	0.00404350983386068\\
364	0.00404618092083107\\
365	0.00404890024897001\\
366	0.0040516686602516\\
367	0.00405448701286639\\
368	0.00405735618202696\\
369	0.00406027706092961\\
370	0.00406325056192143\\
371	0.00406627761793908\\
372	0.00406935918430876\\
373	0.00407249624102802\\
374	0.00407568979568942\\
375	0.0040789408872557\\
376	0.00408225059094776\\
377	0.00408562002453609\\
378	0.00408905035623915\\
379	0.00409254281392852\\
380	0.00409609869342913\\
381	0.00409971935696407\\
382	0.00410340618832885\\
383	0.00410716038661171\\
384	0.00411098317744976\\
385	0.00411487581673965\\
386	0.00411883959294418\\
387	0.004122875829743\\
388	0.00412698588907731\\
389	0.00413117117464473\\
390	0.00413543313586004\\
391	0.00413977327232583\\
392	0.00414419313893618\\
393	0.00414869435171167\\
394	0.00415327859445866\\
395	0.00415794762635095\\
396	0.00416270329053206\\
397	0.00416754752383153\\
398	0.00417248236767648\\
399	0.0041775099802546\\
400	0.00418263264993889\\
401	0.00418785280991241\\
402	0.00419317305381311\\
403	0.0041985961520556\\
404	0.00420412506823816\\
405	0.00420976297468375\\
406	0.00421551326565563\\
407	0.00422137956604833\\
408	0.00422736573238066\\
409	0.00423347584149412\\
410	0.00423971415986722\\
411	0.00424608508279133\\
412	0.00425259303372716\\
413	0.00425924229985253\\
414	0.00426603677859521\\
415	0.00427298008943125\\
416	0.00428007596905633\\
417	0.00428732827654689\\
418	0.00429474099877965\\
419	0.00430231825611065\\
420	0.0043100643083079\\
421	0.00431798356070496\\
422	0.00432608057051945\\
423	0.00433436005331877\\
424	0.004342826889352\\
425	0.00435148612981159\\
426	0.00436034300384433\\
427	0.00436940292925008\\
428	0.00437867152652871\\
429	0.00438815462916805\\
430	0.00439785829455999\\
431	0.00440778881559047\\
432	0.00441795273296464\\
433	0.00442835684837378\\
434	0.00443900823873352\\
435	0.00444991427180255\\
436	0.00446108262154964\\
437	0.0044725212817723\\
438	0.00448423858094976\\
439	0.0044962432057614\\
440	0.00450854422092761\\
441	0.00452115109014408\\
442	0.00453407369812363\\
443	0.00454732237379214\\
444	0.00456090791487675\\
445	0.00457484161525172\\
446	0.00458913530067627\\
447	0.00460380135787383\\
448	0.00461885273974556\\
449	0.00463430295530163\\
450	0.004650166150056\\
451	0.00466645714472242\\
452	0.00468319147353427\\
453	0.00470038543059159\\
454	0.00471805611894368\\
455	0.0047362215021036\\
456	0.00475490047618871\\
457	0.00477411317125081\\
458	0.00479388076546922\\
459	0.00481422512485293\\
460	0.00483516901356313\\
461	0.00485673633658239\\
462	0.00487895217184936\\
463	0.00490184277702585\\
464	0.00492543555181921\\
465	0.00494975875678549\\
466	0.00497484138331397\\
467	0.00500071589192295\\
468	0.00502741721143315\\
469	0.00505497690788504\\
470	0.00508342542731814\\
471	0.00511279344080281\\
472	0.00514311193797709\\
473	0.00517441246313275\\
474	0.00520672755744614\\
475	0.00524009148975446\\
476	0.0052745414091518\\
477	0.0053101217940235\\
478	0.00534689398902134\\
479	0.00538491714432695\\
480	0.00542424577839099\\
481	0.00546492646755991\\
482	0.0055069932302287\\
483	0.00555046131896994\\
484	0.0055953189682578\\
485	0.00564151663495771\\
486	0.00568895261741899\\
487	0.00573745407419114\\
488	0.00578675199310947\\
489	0.00583602825905697\\
490	0.0058842010330947\\
491	0.00593110897483594\\
492	0.00597658330801327\\
493	0.00602045039887398\\
494	0.00606253484819726\\
495	0.0061026665856415\\
496	0.00614069144522305\\
497	0.00617648509989499\\
498	0.00620997250803433\\
499	0.0062411547515585\\
500	0.00627014583967686\\
501	0.00629829006743282\\
502	0.00632581759750108\\
503	0.00635273293191337\\
504	0.00637905302861716\\
505	0.00640480916376602\\
506	0.00643004854353724\\
507	0.00645483543617274\\
508	0.00647925144017759\\
509	0.00650339433073105\\
510	0.00652737468658162\\
511	0.00655130919171871\\
512	0.00657528119528521\\
513	0.00659932731884868\\
514	0.00662347217653659\\
515	0.00664774250460606\\
516	0.00667216666117621\\
517	0.00669677395535563\\
518	0.00672159380448917\\
519	0.00674665474004682\\
520	0.0067719833180942\\
521	0.00679760304546858\\
522	0.00682353351760438\\
523	0.00684979079643919\\
524	0.006876389979067\\
525	0.00690334649439269\\
526	0.00693067601823747\\
527	0.00695839439858376\\
528	0.00698651760125394\\
529	0.00701506168821314\\
530	0.00704404284182867\\
531	0.00707347744796229\\
532	0.0071033822473147\\
533	0.00713377455577878\\
534	0.00716467252906144\\
535	0.00719609530995188\\
536	0.00722806308380102\\
537	0.00726059713458046\\
538	0.00729371989872221\\
539	0.00732745501242289\\
540	0.00736182734608918\\
541	0.00739686301702372\\
542	0.0074325893682471\\
543	0.00746903489751596\\
544	0.00750622911624425\\
545	0.00754420231346191\\
546	0.00758298520129518\\
547	0.00762260841386055\\
548	0.00766310181951272\\
549	0.00770449391757828\\
550	0.00774679995318002\\
551	0.00779003428858121\\
552	0.00783421045383369\\
553	0.00787934095957066\\
554	0.00792543702246905\\
555	0.00797250829035165\\
556	0.00802056253023774\\
557	0.00806960527968463\\
558	0.00811964027290157\\
559	0.00817066577568384\\
560	0.00822267263500106\\
561	0.00827564182636987\\
562	0.00832948684670028\\
563	0.00838419951871032\\
564	0.00843983032047843\\
565	0.00849646590164287\\
566	0.0085542417435169\\
567	0.00861340416126296\\
568	0.00867301874355139\\
569	0.00873133908695524\\
570	0.00878796637590921\\
571	0.00884218788307898\\
572	0.00889563206259737\\
573	0.00894812718337587\\
574	0.00899896376434958\\
575	0.00904832370413948\\
576	0.00909665414515263\\
577	0.00914434879159263\\
578	0.00919144228304781\\
579	0.00923764611023506\\
580	0.00928301640959322\\
581	0.0093279784285072\\
582	0.00937262878471426\\
583	0.00941698222330296\\
584	0.00946100780116802\\
585	0.00950465824199754\\
586	0.00954788190473341\\
587	0.00959062658943052\\
588	0.00963284242838222\\
589	0.0096744851752674\\
590	0.00971551906172689\\
591	0.00975591929412051\\
592	0.00979567191274874\\
593	0.00983476678232329\\
594	0.00987312984006299\\
595	0.00991038882774525\\
596	0.00994573390547615\\
597	0.0099771937715668\\
598	0.00999970795535495\\
599	0\\
600	0\\
};
\addplot [color=red!40!mycolor19,solid,forget plot]
  table[row sep=crcr]{%
1	0.00394496355807239\\
2	0.00394496646483332\\
3	0.0039449694249811\\
4	0.00394497243948291\\
5	0.00394497550932361\\
6	0.00394497863550595\\
7	0.00394498181905096\\
8	0.00394498506099832\\
9	0.00394498836240667\\
10	0.00394499172435401\\
11	0.00394499514793801\\
12	0.00394499863427644\\
13	0.00394500218450758\\
14	0.00394500579979047\\
15	0.00394500948130545\\
16	0.00394501323025453\\
17	0.0039450170478618\\
18	0.0039450209353738\\
19	0.00394502489406003\\
20	0.00394502892521339\\
21	0.00394503303015056\\
22	0.0039450372102125\\
23	0.00394504146676493\\
24	0.00394504580119881\\
25	0.00394505021493081\\
26	0.00394505470940376\\
27	0.0039450592860873\\
28	0.00394506394647825\\
29	0.00394506869210123\\
30	0.00394507352450914\\
31	0.0039450784452838\\
32	0.00394508345603639\\
33	0.00394508855840812\\
34	0.00394509375407081\\
35	0.00394509904472744\\
36	0.00394510443211276\\
37	0.00394510991799403\\
38	0.00394511550417151\\
39	0.00394512119247915\\
40	0.00394512698478532\\
41	0.00394513288299344\\
42	0.00394513888904266\\
43	0.00394514500490856\\
44	0.00394515123260396\\
45	0.00394515757417947\\
46	0.00394516403172446\\
47	0.00394517060736772\\
48	0.00394517730327818\\
49	0.00394518412166581\\
50	0.00394519106478241\\
51	0.00394519813492242\\
52	0.0039452053344238\\
53	0.00394521266566886\\
54	0.00394522013108519\\
55	0.00394522773314651\\
56	0.00394523547437364\\
57	0.00394524335733541\\
58	0.00394525138464964\\
59	0.00394525955898407\\
60	0.00394526788305749\\
61	0.00394527635964063\\
62	0.00394528499155725\\
63	0.0039452937816852\\
64	0.00394530273295754\\
65	0.00394531184836359\\
66	0.00394532113095009\\
67	0.00394533058382237\\
68	0.00394534021014553\\
69	0.00394535001314561\\
70	0.00394535999611084\\
71	0.00394537016239289\\
72	0.00394538051540814\\
73	0.00394539105863905\\
74	0.00394540179563535\\
75	0.00394541273001555\\
76	0.00394542386546822\\
77	0.00394543520575351\\
78	0.00394544675470449\\
79	0.00394545851622864\\
80	0.00394547049430942\\
81	0.00394548269300778\\
82	0.00394549511646364\\
83	0.00394550776889761\\
84	0.00394552065461258\\
85	0.00394553377799536\\
86	0.00394554714351835\\
87	0.00394556075574136\\
88	0.00394557461931328\\
89	0.00394558873897392\\
90	0.00394560311955589\\
91	0.00394561776598641\\
92	0.00394563268328927\\
93	0.00394564787658675\\
94	0.0039456633511016\\
95	0.00394567911215911\\
96	0.00394569516518915\\
97	0.00394571151572828\\
98	0.00394572816942192\\
99	0.00394574513202657\\
100	0.00394576240941198\\
101	0.00394578000756347\\
102	0.00394579793258432\\
103	0.003945816190698\\
104	0.00394583478825078\\
105	0.00394585373171405\\
106	0.00394587302768688\\
107	0.00394589268289868\\
108	0.00394591270421163\\
109	0.00394593309862358\\
110	0.00394595387327058\\
111	0.00394597503542981\\
112	0.00394599659252227\\
113	0.00394601855211579\\
114	0.00394604092192787\\
115	0.00394606370982878\\
116	0.00394608692384456\\
117	0.00394611057216016\\
118	0.00394613466312262\\
119	0.00394615920524426\\
120	0.00394618420720612\\
121	0.00394620967786115\\
122	0.00394623562623785\\
123	0.00394626206154353\\
124	0.00394628899316814\\
125	0.0039463164306877\\
126	0.0039463443838681\\
127	0.00394637286266892\\
128	0.00394640187724722\\
129	0.00394643143796143\\
130	0.00394646155537548\\
131	0.0039464922402628\\
132	0.00394652350361047\\
133	0.00394655535662353\\
134	0.0039465878107292\\
135	0.00394662087758137\\
136	0.00394665456906512\\
137	0.00394668889730117\\
138	0.00394672387465066\\
139	0.00394675951371986\\
140	0.00394679582736501\\
141	0.0039468328286973\\
142	0.00394687053108786\\
143	0.00394690894817291\\
144	0.003946948093859\\
145	0.00394698798232838\\
146	0.0039470286280443\\
147	0.00394707004575672\\
148	0.00394711225050783\\
149	0.00394715525763789\\
150	0.003947199082791\\
151	0.0039472437419212\\
152	0.00394728925129846\\
153	0.00394733562751496\\
154	0.00394738288749136\\
155	0.00394743104848332\\
156	0.00394748012808804\\
157	0.00394753014425094\\
158	0.00394758111527261\\
159	0.00394763305981562\\
160	0.00394768599691171\\
161	0.00394773994596905\\
162	0.00394779492677958\\
163	0.00394785095952653\\
164	0.00394790806479215\\
165	0.00394796626356545\\
166	0.00394802557725021\\
167	0.00394808602767313\\
168	0.00394814763709206\\
169	0.0039482104282045\\
170	0.00394827442415617\\
171	0.00394833964854986\\
172	0.00394840612545423\\
173	0.00394847387941314\\
174	0.0039485429354548\\
175	0.00394861331910132\\
176	0.00394868505637834\\
177	0.00394875817382495\\
178	0.00394883269850372\\
179	0.00394890865801092\\
180	0.00394898608048703\\
181	0.00394906499462736\\
182	0.00394914542969294\\
183	0.00394922741552157\\
184	0.00394931098253916\\
185	0.00394939616177125\\
186	0.00394948298485471\\
187	0.00394957148404979\\
188	0.00394966169225226\\
189	0.00394975364300596\\
190	0.00394984737051543\\
191	0.00394994290965888\\
192	0.00395004029600142\\
193	0.00395013956580852\\
194	0.0039502407560597\\
195	0.00395034390446258\\
196	0.0039504490494672\\
197	0.00395055623028044\\
198	0.00395066548688106\\
199	0.00395077686003469\\
200	0.00395089039130935\\
201	0.0039510061230911\\
202	0.00395112409860024\\
203	0.00395124436190747\\
204	0.00395136695795083\\
205	0.0039514919325525\\
206	0.00395161933243626\\
207	0.00395174920524514\\
208	0.00395188159955945\\
209	0.0039520165649152\\
210	0.00395215415182285\\
211	0.00395229441178633\\
212	0.00395243739732263\\
213	0.00395258316198166\\
214	0.00395273176036637\\
215	0.00395288324815356\\
216	0.00395303768211484\\
217	0.00395319512013803\\
218	0.00395335562124911\\
219	0.00395351924563447\\
220	0.00395368605466358\\
221	0.00395385611091217\\
222	0.00395402947818585\\
223	0.0039542062215441\\
224	0.0039543864073248\\
225	0.00395457010316907\\
226	0.00395475737804693\\
227	0.003954948302283\\
228	0.00395514294758295\\
229	0.00395534138706046\\
230	0.00395554369526455\\
231	0.00395574994820742\\
232	0.00395596022339293\\
233	0.00395617459984554\\
234	0.00395639315813964\\
235	0.00395661598042968\\
236	0.00395684315048058\\
237	0.00395707475369883\\
238	0.00395731087716406\\
239	0.00395755160966121\\
240	0.00395779704171326\\
241	0.00395804726561444\\
242	0.00395830237546404\\
243	0.00395856246720084\\
244	0.00395882763863805\\
245	0.00395909798949871\\
246	0.00395937362145188\\
247	0.00395965463814919\\
248	0.00395994114526192\\
249	0.00396023325051887\\
250	0.00396053106374449\\
251	0.00396083469689774\\
252	0.00396114426411135\\
253	0.00396145988173162\\
254	0.00396178166835869\\
255	0.00396210974488731\\
256	0.00396244423454795\\
257	0.00396278526294831\\
258	0.00396313295811507\\
259	0.00396348745053593\\
260	0.00396384887320159\\
261	0.00396421736164758\\
262	0.00396459305399591\\
263	0.00396497609099558\\
264	0.00396536661606227\\
265	0.00396576477531575\\
266	0.00396617071761449\\
267	0.00396658459458613\\
268	0.00396700656065184\\
269	0.00396743677304213\\
270	0.00396787539180087\\
271	0.00396832257977263\\
272	0.0039687785025673\\
273	0.00396924332849335\\
274	0.00396971722844904\\
275	0.00397020037575863\\
276	0.00397069294594217\\
277	0.00397119511642956\\
278	0.00397170706633817\\
279	0.00397222897697963\\
280	0.003972761036643\\
281	0.00397330344269698\\
282	0.0039738563963747\\
283	0.00397442010285586\\
284	0.00397499477135162\\
285	0.003975580615192\\
286	0.00397617785191623\\
287	0.00397678670336624\\
288	0.00397740739578341\\
289	0.00397804015990856\\
290	0.00397868523108615\\
291	0.00397934284937203\\
292	0.00398001325964574\\
293	0.00398069671172766\\
294	0.00398139346050093\\
295	0.00398210376603946\\
296	0.00398282789374209\\
297	0.00398356611447407\\
298	0.00398431870471683\\
299	0.00398508594672751\\
300	0.00398586812871031\\
301	0.00398666554500194\\
302	0.00398747849627502\\
303	0.00398830728976378\\
304	0.00398915223951882\\
305	0.00399001366669935\\
306	0.00399089189991507\\
307	0.00399178727563383\\
308	0.00399270013867693\\
309	0.00399363084283214\\
310	0.00399457975162358\\
311	0.00399554723929041\\
312	0.00399653369203804\\
313	0.00399753950962834\\
314	0.00399856510734339\\
315	0.00399961091819241\\
316	0.00400067739463\\
317	0.00400176500691815\\
318	0.00400287422750616\\
319	0.00400400549062357\\
320	0.00400515923946962\\
321	0.00400633592641326\\
322	0.00400753601319825\\
323	0.00400875997115389\\
324	0.00401000828141145\\
325	0.00401128143512629\\
326	0.00401257993370635\\
327	0.00401390428904663\\
328	0.0040152550237706\\
329	0.00401663267147813\\
330	0.00401803777700066\\
331	0.00401947089666371\\
332	0.00402093259855716\\
333	0.0040224234628134\\
334	0.00402394408189395\\
335	0.00402549506088478\\
336	0.00402707701780082\\
337	0.00402869058389997\\
338	0.00403033640400725\\
339	0.00403201513684932\\
340	0.00403372745540015\\
341	0.00403547404723831\\
342	0.00403725561491625\\
343	0.00403907287634269\\
344	0.00404092656517789\\
345	0.00404281743124264\\
346	0.00404474624094188\\
347	0.00404671377770299\\
348	0.00404872084243032\\
349	0.00405076825397612\\
350	0.00405285684962921\\
351	0.00405498748562241\\
352	0.00405716103766\\
353	0.0040593784014669\\
354	0.00406164049336234\\
355	0.0040639482508606\\
356	0.00406630263330409\\
357	0.00406870462253483\\
358	0.00407115522361375\\
359	0.0040736554656001\\
360	0.0040762064024049\\
361	0.00407880911370457\\
362	0.00408146470571022\\
363	0.00408417431140515\\
364	0.00408693909135697\\
365	0.00408976023456167\\
366	0.00409263895932078\\
367	0.00409557651415383\\
368	0.00409857417874719\\
369	0.00410163326494096\\
370	0.00410475511775484\\
371	0.00410794111645349\\
372	0.00411119267565054\\
373	0.00411451124644887\\
374	0.00411789831761287\\
375	0.00412135541676298\\
376	0.0041248841115783\\
377	0.00412848601098433\\
378	0.0041321627662925\\
379	0.00413591607225034\\
380	0.00413974766796995\\
381	0.0041436593378207\\
382	0.00414765291309146\\
383	0.00415173027645591\\
384	0.00415589336397858\\
385	0.00416014416719945\\
386	0.00416448473532126\\
387	0.00416891717750316\\
388	0.00417344366526517\\
389	0.00417806643500328\\
390	0.00418278779061785\\
391	0.00418761010625828\\
392	0.00419253582918275\\
393	0.00419756748273102\\
394	0.00420270766940556\\
395	0.00420795907405323\\
396	0.00421332446713702\\
397	0.00421880670808039\\
398	0.00422440874865449\\
399	0.00423013363637588\\
400	0.00423598451790212\\
401	0.00424196464238815\\
402	0.00424807736478989\\
403	0.00425432614902129\\
404	0.00426071457093349\\
405	0.00426724632107828\\
406	0.004273925207205\\
407	0.00428075515657952\\
408	0.00428774021817981\\
409	0.00429488456494814\\
410	0.00430219249646694\\
411	0.00430966844256718\\
412	0.00431731696826001\\
413	0.00432514278095873\\
414	0.00433315074798983\\
415	0.00434134591320038\\
416	0.00434973350581722\\
417	0.00435831894954741\\
418	0.00436710787222604\\
419	0.00437610611605775\\
420	0.00438531974852342\\
421	0.00439475507406703\\
422	0.00440441864655328\\
423	0.00441431728231955\\
424	0.00442445807483061\\
425	0.00443484840843557\\
426	0.00444549597140659\\
427	0.00445640876951495\\
428	0.00446759514619427\\
429	0.00447906380106728\\
430	0.0044908238084109\\
431	0.00450288463640696\\
432	0.00451525616716112\\
433	0.00452794871752132\\
434	0.00454097306099056\\
435	0.00455434045246367\\
436	0.00456806266293691\\
437	0.00458215200805808\\
438	0.0045966213459003\\
439	0.00461148406469206\\
440	0.00462675417512067\\
441	0.00464244635062386\\
442	0.00465857597089339\\
443	0.00467515916830442\\
444	0.00469221287659104\\
445	0.00470975488028099\\
446	0.00472780386788214\\
447	0.00474637959009092\\
448	0.00476550297957607\\
449	0.00478519598648636\\
450	0.00480548097141639\\
451	0.00482638121590674\\
452	0.00484792084856823\\
453	0.00487012456396235\\
454	0.00489301757738932\\
455	0.00491662557462582\\
456	0.00494097461056348\\
457	0.00496609074537327\\
458	0.00499200351336739\\
459	0.00501874473018187\\
460	0.00504634372167624\\
461	0.00507482848338156\\
462	0.00510422816602658\\
463	0.00513457392687377\\
464	0.00516590026784464\\
465	0.00519825241076104\\
466	0.00523168729608508\\
467	0.00526626029038589\\
468	0.00530202299552629\\
469	0.00533902031828007\\
470	0.00537728661793321\\
471	0.00541684044183194\\
472	0.00545767738756752\\
473	0.00549976060088042\\
474	0.00554300816069849\\
475	0.00558727665285665\\
476	0.00563233935889654\\
477	0.00567766537066826\\
478	0.00572200969788633\\
479	0.00576522661355051\\
480	0.00580716209985042\\
481	0.00584765617745405\\
482	0.00588654709499107\\
483	0.00592367616574599\\
484	0.00595889481100118\\
485	0.00599207473592931\\
486	0.00602312453221166\\
487	0.0060520110078312\\
488	0.00607878851709771\\
489	0.00610406937245691\\
490	0.00612875844950554\\
491	0.00615285424514161\\
492	0.00617636559083878\\
493	0.00619931332170027\\
494	0.0062217318742031\\
495	0.00624367062664039\\
496	0.00626519468351297\\
497	0.00628638471090834\\
498	0.00630733526897909\\
499	0.00632815083990122\\
500	0.00634893842542317\\
501	0.00636975308506636\\
502	0.00639061925824366\\
503	0.00641155928072116\\
504	0.00643259712800896\\
505	0.00645375790660415\\
506	0.00647506719390883\\
507	0.00649655023111479\\
508	0.0065182309953332\\
509	0.00654013121257779\\
510	0.00656226942791659\\
511	0.00658466033084912\\
512	0.00660731580898516\\
513	0.00663024701649982\\
514	0.00665346511383199\\
515	0.00667698117480523\\
516	0.00670080610463867\\
517	0.00672495057828269\\
518	0.00674942500995607\\
519	0.00677423956536453\\
520	0.00679940422702772\\
521	0.00682492891916385\\
522	0.00685082368968336\\
523	0.00687709889966683\\
524	0.00690376531885638\\
525	0.00693083416580522\\
526	0.00695831715653323\\
527	0.00698622656210869\\
528	0.00701457527508552\\
529	0.00704337688405048\\
530	0.00707264575467542\\
531	0.0071023971146663\\
532	0.00713264713896589\\
533	0.00716341303073712\\
534	0.00719471309383995\\
535	0.00722656679893333\\
536	0.00725899484423924\\
537	0.00729201920698204\\
538	0.00732566318026771\\
539	0.00735995138860808\\
540	0.00739490977335536\\
541	0.00743056553690132\\
542	0.00746694703150045\\
543	0.00750408357484665\\
544	0.00754200516988861\\
545	0.00758074210050495\\
546	0.00762032436740506\\
547	0.00766078090917343\\
548	0.0077021386134001\\
549	0.00774441090401645\\
550	0.00778760972370493\\
551	0.00783174548080916\\
552	0.0078768267041794\\
553	0.00792285966441682\\
554	0.00796984870473029\\
555	0.00801779233817997\\
556	0.00806668120835328\\
557	0.00811649542837718\\
558	0.00816712404026765\\
559	0.00821860615966\\
560	0.00827100021602301\\
561	0.00832438113928522\\
562	0.0083789392287531\\
563	0.00843483361816912\\
564	0.00849221069336367\\
565	0.00854933671311469\\
566	0.00860506706316596\\
567	0.00865884633410658\\
568	0.0087110393692905\\
569	0.0087629231690006\\
570	0.00881433000902838\\
571	0.00886444287004998\\
572	0.00891325983660414\\
573	0.00896106468256766\\
574	0.00900845651393775\\
575	0.00905548628705952\\
576	0.00910215844577083\\
577	0.00914809200862895\\
578	0.0091933703120837\\
579	0.00923845293548966\\
580	0.00928338532988187\\
581	0.00932815453170577\\
582	0.00937272078824421\\
583	0.00941703320141838\\
584	0.00946103673208299\\
585	0.00950467492731733\\
586	0.0095478914410834\\
587	0.00959063167473112\\
588	0.00963284478933166\\
589	0.00967448602529565\\
590	0.00971551927992669\\
591	0.00975591931512916\\
592	0.00979567191274874\\
593	0.00983476678232329\\
594	0.00987312984006299\\
595	0.00991038882774525\\
596	0.00994573390547615\\
597	0.0099771937715668\\
598	0.00999970795535495\\
599	0\\
600	0\\
};
\addplot [color=red!75!mycolor17,solid,forget plot]
  table[row sep=crcr]{%
1	0.00395602082978746\\
2	0.00395602405595148\\
3	0.00395602734201359\\
4	0.00395603068905089\\
5	0.00395603409815943\\
6	0.00395603757045459\\
7	0.00395604110707146\\
8	0.00395604470916515\\
9	0.00395604837791114\\
10	0.00395605211450571\\
11	0.00395605592016628\\
12	0.00395605979613175\\
13	0.00395606374366297\\
14	0.00395606776404308\\
15	0.00395607185857799\\
16	0.0039560760285967\\
17	0.0039560802754518\\
18	0.00395608460051981\\
19	0.00395608900520178\\
20	0.00395609349092355\\
21	0.00395609805913636\\
22	0.00395610271131725\\
23	0.00395610744896952\\
24	0.00395611227362331\\
25	0.00395611718683595\\
26	0.00395612219019262\\
27	0.00395612728530677\\
28	0.0039561324738207\\
29	0.00395613775740604\\
30	0.00395614313776441\\
31	0.00395614861662786\\
32	0.00395615419575955\\
33	0.00395615987695427\\
34	0.00395616566203905\\
35	0.00395617155287377\\
36	0.00395617755135186\\
37	0.0039561836594008\\
38	0.00395618987898292\\
39	0.00395619621209593\\
40	0.0039562026607737\\
41	0.00395620922708692\\
42	0.00395621591314377\\
43	0.00395622272109076\\
44	0.00395622965311331\\
45	0.00395623671143668\\
46	0.00395624389832659\\
47	0.00395625121609007\\
48	0.00395625866707629\\
49	0.00395626625367737\\
50	0.0039562739783292\\
51	0.0039562818435123\\
52	0.00395628985175273\\
53	0.00395629800562295\\
54	0.0039563063077428\\
55	0.00395631476078034\\
56	0.00395632336745288\\
57	0.00395633213052796\\
58	0.00395634105282425\\
59	0.00395635013721278\\
60	0.00395635938661775\\
61	0.00395636880401769\\
62	0.00395637839244663\\
63	0.0039563881549951\\
64	0.00395639809481132\\
65	0.0039564082151023\\
66	0.00395641851913513\\
67	0.00395642901023816\\
68	0.00395643969180219\\
69	0.0039564505672818\\
70	0.00395646164019659\\
71	0.00395647291413252\\
72	0.00395648439274334\\
73	0.00395649607975182\\
74	0.00395650797895131\\
75	0.00395652009420712\\
76	0.00395653242945801\\
77	0.00395654498871764\\
78	0.00395655777607619\\
79	0.00395657079570189\\
80	0.00395658405184268\\
81	0.00395659754882772\\
82	0.00395661129106929\\
83	0.00395662528306431\\
84	0.00395663952939613\\
85	0.0039566540347364\\
86	0.00395666880384683\\
87	0.00395668384158114\\
88	0.00395669915288682\\
89	0.00395671474280725\\
90	0.0039567306164836\\
91	0.00395674677915684\\
92	0.00395676323616991\\
93	0.00395677999296979\\
94	0.00395679705510968\\
95	0.00395681442825122\\
96	0.00395683211816676\\
97	0.00395685013074166\\
98	0.00395686847197662\\
99	0.00395688714799018\\
100	0.00395690616502109\\
101	0.00395692552943095\\
102	0.0039569452477066\\
103	0.00395696532646291\\
104	0.00395698577244533\\
105	0.00395700659253277\\
106	0.00395702779374025\\
107	0.00395704938322182\\
108	0.00395707136827348\\
109	0.0039570937563361\\
110	0.0039571165549985\\
111	0.00395713977200056\\
112	0.00395716341523631\\
113	0.00395718749275724\\
114	0.00395721201277556\\
115	0.0039572369836675\\
116	0.00395726241397684\\
117	0.00395728831241832\\
118	0.00395731468788126\\
119	0.00395734154943326\\
120	0.00395736890632381\\
121	0.00395739676798813\\
122	0.00395742514405103\\
123	0.00395745404433089\\
124	0.00395748347884368\\
125	0.00395751345780703\\
126	0.00395754399164451\\
127	0.00395757509098976\\
128	0.00395760676669098\\
129	0.00395763902981539\\
130	0.00395767189165368\\
131	0.00395770536372469\\
132	0.00395773945778006\\
133	0.00395777418580917\\
134	0.00395780956004397\\
135	0.00395784559296401\\
136	0.00395788229730144\\
137	0.00395791968604639\\
138	0.00395795777245217\\
139	0.00395799657004068\\
140	0.00395803609260799\\
141	0.00395807635422989\\
142	0.00395811736926766\\
143	0.0039581591523739\\
144	0.00395820171849852\\
145	0.00395824508289477\\
146	0.00395828926112551\\
147	0.00395833426906939\\
148	0.00395838012292743\\
149	0.0039584268392295\\
150	0.00395847443484102\\
151	0.00395852292696975\\
152	0.0039585723331728\\
153	0.0039586226713637\\
154	0.00395867395981956\\
155	0.00395872621718849\\
156	0.00395877946249704\\
157	0.00395883371515793\\
158	0.00395888899497776\\
159	0.00395894532216502\\
160	0.00395900271733813\\
161	0.00395906120153377\\
162	0.00395912079621516\\
163	0.00395918152328077\\
164	0.00395924340507306\\
165	0.00395930646438725\\
166	0.00395937072448059\\
167	0.00395943620908148\\
168	0.003959502942399\\
169	0.00395957094913248\\
170	0.00395964025448138\\
171	0.00395971088415519\\
172	0.00395978286438382\\
173	0.00395985622192772\\
174	0.00395993098408878\\
175	0.00396000717872095\\
176	0.00396008483424138\\
177	0.00396016397964162\\
178	0.00396024464449905\\
179	0.00396032685898871\\
180	0.00396041065389509\\
181	0.00396049606062437\\
182	0.00396058311121681\\
183	0.00396067183835945\\
184	0.00396076227539898\\
185	0.00396085445635494\\
186	0.00396094841593315\\
187	0.00396104418953945\\
188	0.00396114181329363\\
189	0.00396124132404376\\
190	0.00396134275938071\\
191	0.0039614461576531\\
192	0.00396155155798237\\
193	0.00396165900027826\\
194	0.00396176852525467\\
195	0.00396188017444574\\
196	0.00396199399022223\\
197	0.00396211001580845\\
198	0.00396222829529927\\
199	0.00396234887367768\\
200	0.00396247179683259\\
201	0.00396259711157713\\
202	0.00396272486566713\\
203	0.00396285510782021\\
204	0.0039629878877351\\
205	0.00396312325611148\\
206	0.00396326126467012\\
207	0.00396340196617354\\
208	0.00396354541444703\\
209	0.00396369166440016\\
210	0.00396384077204873\\
211	0.00396399279453712\\
212	0.00396414779016123\\
213	0.00396430581839175\\
214	0.00396446693989807\\
215	0.00396463121657251\\
216	0.00396479871155532\\
217	0.00396496948925985\\
218	0.00396514361539862\\
219	0.00396532115700966\\
220	0.0039655021824835\\
221	0.00396568676159081\\
222	0.00396587496551043\\
223	0.00396606686685812\\
224	0.0039662625397159\\
225	0.00396646205966207\\
226	0.00396666550380156\\
227	0.00396687295079733\\
228	0.00396708448090212\\
229	0.003967300175991\\
230	0.00396752011959459\\
231	0.00396774439693293\\
232	0.00396797309495013\\
233	0.00396820630234957\\
234	0.00396844410963019\\
235	0.00396868660912315\\
236	0.00396893389502942\\
237	0.00396918606345832\\
238	0.0039694432124666\\
239	0.00396970544209855\\
240	0.00396997285442672\\
241	0.00397024555359395\\
242	0.00397052364585571\\
243	0.00397080723962386\\
244	0.00397109644551091\\
245	0.00397139137637569\\
246	0.00397169214736955\\
247	0.00397199887598384\\
248	0.00397231168209841\\
249	0.00397263068803103\\
250	0.00397295601858803\\
251	0.00397328780111598\\
252	0.00397362616555453\\
253	0.0039739712444905\\
254	0.00397432317321316\\
255	0.00397468208977065\\
256	0.00397504813502792\\
257	0.00397542145272575\\
258	0.0039758021895414\\
259	0.00397619049515047\\
260	0.00397658652229027\\
261	0.00397699042682486\\
262	0.00397740236781135\\
263	0.00397782250756811\\
264	0.0039782510117444\\
265	0.00397868804939187\\
266	0.00397913379303773\\
267	0.0039795884187599\\
268	0.00398005210626397\\
269	0.00398052503896246\\
270	0.00398100740405609\\
271	0.00398149939261786\\
272	0.00398200119967981\\
273	0.00398251302432341\\
274	0.00398303506977423\\
275	0.00398356754350188\\
276	0.00398411065732733\\
277	0.00398466462753955\\
278	0.00398522967502303\\
279	0.00398580602538462\\
280	0.00398639390902049\\
281	0.00398699356115706\\
282	0.00398760522195791\\
283	0.00398822913663347\\
284	0.00398886555555311\\
285	0.00398951473436007\\
286	0.00399017693408895\\
287	0.00399085242128627\\
288	0.00399154146813359\\
289	0.00399224435257379\\
290	0.00399296135844012\\
291	0.00399369277558839\\
292	0.00399443890003218\\
293	0.00399520003408121\\
294	0.00399597648648278\\
295	0.00399676857256657\\
296	0.00399757661439261\\
297	0.00399840094090254\\
298	0.00399924188807439\\
299	0.0040000997990806\\
300	0.00400097502444961\\
301	0.00400186792223084\\
302	0.00400277885816342\\
303	0.00400370820584827\\
304	0.00400465634692377\\
305	0.00400562367124485\\
306	0.00400661057706534\\
307	0.004007617471223\\
308	0.00400864476932682\\
309	0.0040096928959449\\
310	0.0040107622847916\\
311	0.00401185337891045\\
312	0.00401296663084853\\
313	0.00401410250281579\\
314	0.00401526146681993\\
315	0.00401644400476658\\
316	0.00401765060851909\\
317	0.00401888177995787\\
318	0.00402013803126919\\
319	0.00402141988576657\\
320	0.00402272787815363\\
321	0.00402406255479396\\
322	0.00402542447398887\\
323	0.00402681420626239\\
324	0.00402823233465483\\
325	0.0040296794550242\\
326	0.00403115617635655\\
327	0.00403266312108521\\
328	0.00403420092541916\\
329	0.00403577023968131\\
330	0.00403737172865651\\
331	0.00403900607195019\\
332	0.00404067396435763\\
333	0.00404237611624463\\
334	0.00404411325393971\\
335	0.00404588612013855\\
336	0.00404769547432123\\
337	0.00404954209318242\\
338	0.00405142677107538\\
339	0.00405335032047024\\
340	0.00405531357242699\\
341	0.00405731737708404\\
342	0.00405936260416318\\
343	0.00406145014349098\\
344	0.00406358090553805\\
345	0.00406575582197638\\
346	0.00406797584625572\\
347	0.00407024195419969\\
348	0.0040725551446227\\
349	0.00407491643996836\\
350	0.00407732688697043\\
351	0.00407978755733752\\
352	0.00408229954846239\\
353	0.00408486398415702\\
354	0.00408748201541441\\
355	0.00409015482119849\\
356	0.00409288360926343\\
357	0.0040956696170037\\
358	0.0040985141123362\\
359	0.0041014183946158\\
360	0.00410438379558483\\
361	0.00410741168035432\\
362	0.00411050344842638\\
363	0.00411366053476552\\
364	0.00411688441091478\\
365	0.00412017658615775\\
366	0.00412353860872746\\
367	0.00412697206706691\\
368	0.00413047859114477\\
369	0.0041340598538287\\
370	0.00413771757232021\\
371	0.0041414535096554\\
372	0.00414526947627356\\
373	0.00414916733165619\\
374	0.00415314898604155\\
375	0.00415721640221952\\
376	0.00416137159741164\\
377	0.00416561664524115\\
378	0.00416995367779931\\
379	0.00417438488781448\\
380	0.00417891253093526\\
381	0.00418353892815308\\
382	0.00418826646833857\\
383	0.00419309761086729\\
384	0.00419803488836444\\
385	0.00420308090958143\\
386	0.00420823836242274\\
387	0.00421351001712873\\
388	0.00421889872961684\\
389	0.00422440744500477\\
390	0.00423003920130374\\
391	0.00423579713330029\\
392	0.00424168447663842\\
393	0.0042477045721139\\
394	0.0042538608701944\\
395	0.00426015693578099\\
396	0.00426659645323247\\
397	0.00427318323168068\\
398	0.00427992121066044\\
399	0.00428681446598744\\
400	0.00429386721579037\\
401	0.00430108382694307\\
402	0.00430846882187104\\
403	0.00431602688620221\\
404	0.00432376287672331\\
405	0.00433168182983377\\
406	0.00433978897059622\\
407	0.0043480897219022\\
408	0.00435658971459941\\
409	0.00436529479824546\\
410	0.00437421105255016\\
411	0.00438334479965116\\
412	0.00439270261748483\\
413	0.00440229135313236\\
414	0.00441211813466922\\
415	0.00442219038346729\\
416	0.0044325158284914\\
417	0.00444310252466623\\
418	0.00445395886868408\\
419	0.00446509361532809\\
420	0.00447651589436813\\
421	0.00448823522833583\\
422	0.00450026155237366\\
423	0.00451260523649623\\
424	0.00452527710718072\\
425	0.00453828848728478\\
426	0.00455165122226101\\
427	0.0045653776909434\\
428	0.00457948080298924\\
429	0.00459397407331674\\
430	0.00460887167322002\\
431	0.00462418847383528\\
432	0.00463994009095224\\
433	0.00465614292922633\\
434	0.00467281422244718\\
435	0.00468997206427308\\
436	0.00470763542724128\\
437	0.00472582428256058\\
438	0.00474455966549337\\
439	0.0047638632012953\\
440	0.00478375629291361\\
441	0.00480426069748669\\
442	0.00482539846900143\\
443	0.004847191910015\\
444	0.00486966354231972\\
445	0.00489283611036386\\
446	0.00491673261793214\\
447	0.00494137625657943\\
448	0.00496679181892853\\
449	0.0049930078615359\\
450	0.00502005721140985\\
451	0.00504797014935873\\
452	0.00507678366299867\\
453	0.00510655036703916\\
454	0.00513732270574328\\
455	0.00516915145734446\\
456	0.00520208365406883\\
457	0.00523615972776357\\
458	0.00527140961974777\\
459	0.00530784752197285\\
460	0.00534546502349508\\
461	0.00538422192609985\\
462	0.00542403408627614\\
463	0.00546475724299661\\
464	0.00550616574427743\\
465	0.00554754372534509\\
466	0.00558802536025232\\
467	0.00562747746474042\\
468	0.00566575906656747\\
469	0.0057027233202975\\
470	0.00573822038525274\\
471	0.00577210227021862\\
472	0.00580422976681203\\
473	0.00583448158778131\\
474	0.00586276724356554\\
475	0.00588904432465679\\
476	0.00591334489936595\\
477	0.00593600660055911\\
478	0.0059581084388347\\
479	0.00597964601036323\\
480	0.00600062369861628\\
481	0.00602105616318666\\
482	0.00604096978447627\\
483	0.00606040393537579\\
484	0.00607941189302934\\
485	0.00609806108721539\\
486	0.00611643223438726\\
487	0.00613461672470033\\
488	0.00615271141358175\\
489	0.00617079307180954\\
490	0.00618889452339401\\
491	0.0062070348678562\\
492	0.00622523483041168\\
493	0.00624351637917005\\
494	0.00626190221165521\\
495	0.00628041510624132\\
496	0.00629907714893966\\
497	0.0063179088687222\\
498	0.00633692835028307\\
499	0.0063561504481978\\
500	0.00637558630873905\\
501	0.0063952452307703\\
502	0.00641513623636046\\
503	0.00643526821137997\\
504	0.00645564981709753\\
505	0.00647628941324281\\
506	0.00649719500095732\\
507	0.006518374195067\\
508	0.00653983423523254\\
509	0.00656158204397307\\
510	0.00658362433509701\\
511	0.00660596776689138\\
512	0.006628619069477\\
513	0.00665158509273979\\
514	0.00667487282270193\\
515	0.00669848940419556\\
516	0.00672244217019867\\
517	0.00674673867782084\\
518	0.00677138675043223\\
519	0.00679639452483013\\
520	0.00682177050169216\\
521	0.0068475235970019\\
522	0.00687366319189161\\
523	0.00690019918016826\\
524	0.00692714201721569\\
525	0.00695450277284056\\
526	0.00698229318789062\\
527	0.0070105257343359\\
528	0.00703921367834333\\
529	0.00706837114569149\\
530	0.00709801318866338\\
531	0.00712815585331523\\
532	0.00715881624573353\\
533	0.00719001259553259\\
534	0.00722176431433398\\
535	0.00725409204598056\\
536	0.00728701770400715\\
537	0.0073205644905423\\
538	0.00735475688910968\\
539	0.00738962062164505\\
540	0.00742518255733732\\
541	0.0074614705575005\\
542	0.00749851323640489\\
543	0.00753633961264411\\
544	0.00757497861769079\\
545	0.00761445842043134\\
546	0.00765480549641281\\
547	0.00769604373580787\\
548	0.0077381821411131\\
549	0.00778122713602009\\
550	0.00782518306720235\\
551	0.0078700487002368\\
552	0.00791581480582034\\
553	0.00796245996525486\\
554	0.00800987316217017\\
555	0.00805809948019964\\
556	0.00810719919027359\\
557	0.00815724841861093\\
558	0.00820844826237103\\
559	0.00826089347488161\\
560	0.00831470945785028\\
561	0.00837006021856648\\
562	0.00842504692476101\\
563	0.00847857196042984\\
564	0.00852998377507575\\
565	0.00858051179727193\\
566	0.00863082934811707\\
567	0.00868092347506556\\
568	0.00873049486019678\\
569	0.00877890731150697\\
570	0.00882609246461658\\
571	0.00887301158364894\\
572	0.00891969493973841\\
573	0.00896620472866063\\
574	0.00901252443128679\\
575	0.00905824660811991\\
576	0.00910340646184474\\
577	0.00914848298145622\\
578	0.00919352650797757\\
579	0.00923851799560512\\
580	0.00928341551328938\\
581	0.00932817023841046\\
582	0.00937272947451715\\
583	0.00941703811786192\\
584	0.0094610395478402\\
585	0.00950467651425775\\
586	0.00954789227157245\\
587	0.0095906320514706\\
588	0.00963284492201702\\
589	0.00967448605830613\\
590	0.00971551928305088\\
591	0.00975591931512916\\
592	0.00979567191274874\\
593	0.00983476678232329\\
594	0.00987312984006299\\
595	0.00991038882774525\\
596	0.00994573390547615\\
597	0.0099771937715668\\
598	0.00999970795535495\\
599	0\\
600	0\\
};
\addplot [color=red!80!mycolor19,solid,forget plot]
  table[row sep=crcr]{%
1	0.00395863302433954\\
2	0.00395863704254213\\
3	0.00395864113696591\\
4	0.0039586453089917\\
5	0.00395864956002412\\
6	0.00395865389149205\\
7	0.00395865830484892\\
8	0.00395866280157322\\
9	0.00395866738316888\\
10	0.00395867205116569\\
11	0.00395867680711977\\
12	0.00395868165261395\\
13	0.00395868658925829\\
14	0.0039586916186905\\
15	0.00395869674257641\\
16	0.00395870196261045\\
17	0.00395870728051606\\
18	0.00395871269804637\\
19	0.00395871821698453\\
20	0.00395872383914418\\
21	0.00395872956637022\\
22	0.0039587354005391\\
23	0.00395874134355941\\
24	0.00395874739737249\\
25	0.00395875356395299\\
26	0.00395875984530938\\
27	0.00395876624348452\\
28	0.00395877276055638\\
29	0.0039587793986385\\
30	0.00395878615988069\\
31	0.00395879304646962\\
32	0.00395880006062943\\
33	0.00395880720462245\\
34	0.00395881448074988\\
35	0.00395882189135237\\
36	0.00395882943881079\\
37	0.00395883712554691\\
38	0.00395884495402408\\
39	0.00395885292674809\\
40	0.00395886104626777\\
41	0.00395886931517584\\
42	0.00395887773610968\\
43	0.00395888631175211\\
44	0.00395889504483215\\
45	0.00395890393812599\\
46	0.00395891299445771\\
47	0.00395892221670016\\
48	0.00395893160777593\\
49	0.00395894117065815\\
50	0.00395895090837146\\
51	0.00395896082399292\\
52	0.003958970920653\\
53	0.00395898120153658\\
54	0.00395899166988384\\
55	0.00395900232899139\\
56	0.00395901318221332\\
57	0.00395902423296218\\
58	0.00395903548471019\\
59	0.00395904694099019\\
60	0.00395905860539693\\
61	0.00395907048158818\\
62	0.00395908257328592\\
63	0.0039590948842775\\
64	0.003959107418417\\
65	0.00395912017962644\\
66	0.00395913317189703\\
67	0.00395914639929055\\
68	0.00395915986594074\\
69	0.00395917357605458\\
70	0.00395918753391386\\
71	0.00395920174387651\\
72	0.00395921621037813\\
73	0.00395923093793355\\
74	0.00395924593113826\\
75	0.00395926119467016\\
76	0.00395927673329107\\
77	0.0039592925518484\\
78	0.00395930865527695\\
79	0.00395932504860054\\
80	0.0039593417369338\\
81	0.00395935872548412\\
82	0.0039593760195533\\
83	0.00395939362453965\\
84	0.00395941154593984\\
85	0.00395942978935093\\
86	0.00395944836047241\\
87	0.00395946726510826\\
88	0.00395948650916915\\
89	0.00395950609867453\\
90	0.00395952603975498\\
91	0.00395954633865443\\
92	0.00395956700173253\\
93	0.00395958803546704\\
94	0.00395960944645631\\
95	0.00395963124142173\\
96	0.00395965342721036\\
97	0.00395967601079753\\
98	0.00395969899928957\\
99	0.00395972239992647\\
100	0.00395974622008484\\
101	0.00395977046728063\\
102	0.00395979514917221\\
103	0.00395982027356331\\
104	0.00395984584840615\\
105	0.00395987188180457\\
106	0.00395989838201725\\
107	0.00395992535746104\\
108	0.00395995281671437\\
109	0.0039599807685206\\
110	0.00396000922179166\\
111	0.00396003818561157\\
112	0.00396006766924031\\
113	0.00396009768211736\\
114	0.00396012823386576\\
115	0.00396015933429598\\
116	0.00396019099341\\
117	0.00396022322140539\\
118	0.00396025602867958\\
119	0.00396028942583416\\
120	0.00396032342367933\\
121	0.00396035803323832\\
122	0.00396039326575217\\
123	0.00396042913268429\\
124	0.00396046564572532\\
125	0.00396050281679812\\
126	0.00396054065806268\\
127	0.00396057918192141\\
128	0.00396061840102427\\
129	0.00396065832827417\\
130	0.00396069897683245\\
131	0.0039607403601245\\
132	0.0039607824918455\\
133	0.00396082538596611\\
134	0.00396086905673861\\
135	0.00396091351870289\\
136	0.00396095878669272\\
137	0.003961004875842\\
138	0.0039610518015913\\
139	0.00396109957969451\\
140	0.00396114822622545\\
141	0.00396119775758495\\
142	0.00396124819050762\\
143	0.00396129954206929\\
144	0.0039613518296941\\
145	0.00396140507116206\\
146	0.00396145928461665\\
147	0.00396151448857259\\
148	0.00396157070192373\\
149	0.00396162794395106\\
150	0.00396168623433112\\
151	0.00396174559314424\\
152	0.00396180604088311\\
153	0.00396186759846154\\
154	0.00396193028722337\\
155	0.00396199412895147\\
156	0.0039620591458771\\
157	0.00396212536068919\\
158	0.00396219279654399\\
159	0.00396226147707491\\
160	0.0039623314264024\\
161	0.00396240266914417\\
162	0.00396247523042552\\
163	0.0039625491358899\\
164	0.00396262441170959\\
165	0.00396270108459674\\
166	0.00396277918181448\\
167	0.00396285873118826\\
168	0.00396293976111745\\
169	0.00396302230058718\\
170	0.00396310637918021\\
171	0.00396319202708932\\
172	0.0039632792751297\\
173	0.00396336815475165\\
174	0.00396345869805349\\
175	0.00396355093779486\\
176	0.00396364490740993\\
177	0.00396374064102128\\
178	0.00396383817345372\\
179	0.00396393754024847\\
180	0.00396403877767768\\
181	0.00396414192275912\\
182	0.0039642470132712\\
183	0.00396435408776826\\
184	0.00396446318559614\\
185	0.00396457434690808\\
186	0.00396468761268086\\
187	0.00396480302473139\\
188	0.00396492062573343\\
189	0.00396504045923474\\
190	0.00396516256967458\\
191	0.00396528700240149\\
192	0.00396541380369148\\
193	0.00396554302076652\\
194	0.00396567470181345\\
195	0.00396580889600322\\
196	0.00396594565351054\\
197	0.00396608502553389\\
198	0.00396622706431605\\
199	0.00396637182316486\\
200	0.00396651935647459\\
201	0.00396666971974754\\
202	0.00396682296961639\\
203	0.00396697916386666\\
204	0.00396713836145993\\
205	0.0039673006225573\\
206	0.00396746600854364\\
207	0.00396763458205207\\
208	0.00396780640698901\\
209	0.00396798154855998\\
210	0.0039681600732957\\
211	0.00396834204907883\\
212	0.00396852754517139\\
213	0.00396871663224261\\
214	0.00396890938239744\\
215	0.0039691058692058\\
216	0.00396930616773226\\
217	0.00396951035456654\\
218	0.00396971850785459\\
219	0.00396993070733036\\
220	0.00397014703434829\\
221	0.0039703675719165\\
222	0.00397059240473073\\
223	0.00397082161920898\\
224	0.00397105530352697\\
225	0.00397129354765433\\
226	0.00397153644339162\\
227	0.00397178408440814\\
228	0.00397203656628065\\
229	0.00397229398653273\\
230	0.00397255644467531\\
231	0.00397282404224783\\
232	0.00397309688286044\\
233	0.00397337507223707\\
234	0.00397365871825952\\
235	0.0039739479310123\\
236	0.00397424282282885\\
237	0.00397454350833825\\
238	0.0039748501045135\\
239	0.00397516273072029\\
240	0.00397548150876739\\
241	0.00397580656295768\\
242	0.0039761380201406\\
243	0.00397647600976556\\
244	0.00397682066393664\\
245	0.00397717211746841\\
246	0.00397753050794297\\
247	0.00397789597576834\\
248	0.003978268664238\\
249	0.00397864871959184\\
250	0.0039790362910784\\
251	0.00397943153101849\\
252	0.00397983459487025\\
253	0.0039802456412957\\
254	0.00398066483222859\\
255	0.00398109233294403\\
256	0.00398152831212948\\
257	0.00398197294195742\\
258	0.00398242639815966\\
259	0.00398288886010329\\
260	0.0039833605108685\\
261	0.00398384153732786\\
262	0.00398433213022782\\
263	0.00398483248427173\\
264	0.00398534279820492\\
265	0.00398586327490178\\
266	0.00398639412145473\\
267	0.00398693554926535\\
268	0.0039874877741375\\
269	0.00398805101637271\\
270	0.00398862550086776\\
271	0.00398921145721438\\
272	0.00398980911980147\\
273	0.00399041872791959\\
274	0.00399104052586768\\
275	0.00399167476306257\\
276	0.00399232169415058\\
277	0.00399298157912189\\
278	0.00399365468342665\\
279	0.00399434127809343\\
280	0.00399504163985089\\
281	0.0039957560512532\\
282	0.00399648480080847\\
283	0.0039972281831101\\
284	0.00399798649897121\\
285	0.00399876005556228\\
286	0.00399954916655213\\
287	0.00400035415225188\\
288	0.00400117533976269\\
289	0.0040020130631268\\
290	0.00400286766348202\\
291	0.00400373948922019\\
292	0.00400462889614912\\
293	0.0040055362476586\\
294	0.00400646191489025\\
295	0.00400740627691153\\
296	0.004008369720894\\
297	0.00400935264229589\\
298	0.00401035544504916\\
299	0.00401137854175129\\
300	0.00401242235386175\\
301	0.00401348731190372\\
302	0.0040145738556706\\
303	0.00401568243443828\\
304	0.00401681350718271\\
305	0.00401796754280328\\
306	0.00401914502035211\\
307	0.00402034642926983\\
308	0.00402157226962755\\
309	0.00402282305237587\\
310	0.00402409929960078\\
311	0.00402540154478704\\
312	0.00402673033308928\\
313	0.00402808622161128\\
314	0.00402946977969398\\
315	0.00403088158921293\\
316	0.00403232224488714\\
317	0.00403379235460287\\
318	0.00403529253974487\\
319	0.00403682343553007\\
320	0.0040383856913515\\
321	0.00403997997113249\\
322	0.00404160695369182\\
323	0.0040432673331202\\
324	0.00404496181916823\\
325	0.00404669113764653\\
326	0.00404845603083848\\
327	0.00405025725792615\\
328	0.0040520955954296\\
329	0.00405397183766052\\
330	0.00405588679719062\\
331	0.00405784130533546\\
332	0.00405983621265435\\
333	0.00406187238946691\\
334	0.00406395072638708\\
335	0.00406607213487544\\
336	0.00406823754781072\\
337	0.00407044792008143\\
338	0.00407270422919885\\
339	0.00407500747593193\\
340	0.00407735868496444\\
341	0.00407975890557602\\
342	0.00408220921234976\\
343	0.00408471070590678\\
344	0.00408726451366925\\
345	0.00408987179065329\\
346	0.00409253372029332\\
347	0.00409525151529924\\
348	0.00409802641854714\\
349	0.00410085970400705\\
350	0.00410375267770889\\
351	0.00410670667874872\\
352	0.00410972308033793\\
353	0.00411280329089896\\
354	0.00411594875520805\\
355	0.00411916095558662\\
356	0.00412244141314466\\
357	0.00412579168907941\\
358	0.00412921338603191\\
359	0.00413270814950452\\
360	0.0041362776693427\\
361	0.00413992368128507\\
362	0.00414364796858523\\
363	0.00414745236371024\\
364	0.00415133875012142\\
365	0.00415530906414446\\
366	0.00415936529693058\\
367	0.0041635094964848\\
368	0.00416774376979062\\
369	0.00417207028504644\\
370	0.00417649127400975\\
371	0.00418100903445991\\
372	0.00418562593280709\\
373	0.00419034440683301\\
374	0.00419516696855612\\
375	0.00420009620724449\\
376	0.00420513479258485\\
377	0.00421028547801785\\
378	0.0042155511042497\\
379	0.00422093460295191\\
380	0.00422643900066343\\
381	0.00423206742290854\\
382	0.00423782309855356\\
383	0.00424370936443308\\
384	0.0042497296702617\\
385	0.00425588758372511\\
386	0.00426218679580812\\
387	0.00426863112651516\\
388	0.00427522453098988\\
389	0.00428197110597659\\
390	0.00428887509684302\\
391	0.00429594090488185\\
392	0.00430317309500478\\
393	0.0043105764038424\\
394	0.00431815574826033\\
395	0.00432591623430028\\
396	0.0043338631665573\\
397	0.00434200205802946\\
398	0.0043503386405642\\
399	0.00435887887614446\\
400	0.00436762896824028\\
401	0.00437659537135883\\
402	0.00438578480136786\\
403	0.0043952042449959\\
404	0.00440486097482426\\
405	0.00441476256290941\\
406	0.00442491689482738\\
407	0.00443533218625601\\
408	0.00444601699533134\\
409	0.00445698024307586\\
410	0.00446823123569972\\
411	0.0044797796892213\\
412	0.00449163575809743\\
413	0.00450381007425677\\
414	0.00451631378139361\\
415	0.00452915855083714\\
416	0.00454235659688374\\
417	0.0045559207028187\\
418	0.00456986429629113\\
419	0.00458420148799741\\
420	0.00459894710593212\\
421	0.00461411671981738\\
422	0.00462972664797165\\
423	0.00464579394609775\\
424	0.00466233634044626\\
425	0.00467937197221599\\
426	0.00469691950764049\\
427	0.0047149980975147\\
428	0.00473362714628462\\
429	0.00475282570763295\\
430	0.00477261282404043\\
431	0.00479300762252036\\
432	0.00481402934237911\\
433	0.00483569743900408\\
434	0.00485803180366222\\
435	0.00488105315331135\\
436	0.00490478364174819\\
437	0.00492924758940434\\
438	0.00495447426193388\\
439	0.00498050972171603\\
440	0.0050074088097979\\
441	0.00503522025973298\\
442	0.00506399181217181\\
443	0.00509376848891395\\
444	0.00512459022536719\\
445	0.00515648863985227\\
446	0.00518948267405278\\
447	0.00522357274918004\\
448	0.00525873296887809\\
449	0.0052949008123832\\
450	0.00533196344552232\\
451	0.00536973970926647\\
452	0.00540780491526894\\
453	0.00544513164960255\\
454	0.00548160413227004\\
455	0.00551709878836009\\
456	0.00555148537765958\\
457	0.00558462892299446\\
458	0.00561639269307804\\
459	0.00564664270068478\\
460	0.00567525409529587\\
461	0.00570212134197286\\
462	0.0057271707917798\\
463	0.00575037869247606\\
464	0.00577179501858945\\
465	0.00579196631115484\\
466	0.00581161088317639\\
467	0.00583072395100217\\
468	0.00584930851683075\\
469	0.00586737670239619\\
470	0.0058849510561947\\
471	0.00590206572157465\\
472	0.00591876729326882\\
473	0.00593511512012822\\
474	0.00595118070046808\\
475	0.0059670456533217\\
476	0.00598279752866531\\
477	0.00599851500023344\\
478	0.0060142312473829\\
479	0.00602996287092893\\
480	0.0060457280007058\\
481	0.0060615459934407\\
482	0.006077437020876\\
483	0.00609342154158974\\
484	0.00610951965913373\\
485	0.00612575038568053\\
486	0.00614213085734679\\
487	0.00615867558796312\\
488	0.00617539590799603\\
489	0.00619230048075224\\
490	0.00620939706712024\\
491	0.00622669330845284\\
492	0.00624419664227171\\
493	0.00626191422403095\\
494	0.00627985286150051\\
495	0.00629801896944802\\
496	0.00631641855290834\\
497	0.00633505722697534\\
498	0.00635394027895524\\
499	0.00637307277368215\\
500	0.00639245969301985\\
501	0.00641210601372115\\
502	0.00643201671985407\\
503	0.00645219680927015\\
504	0.00647265130469233\\
505	0.0064933852695478\\
506	0.00651440382832887\\
507	0.0065357121908277\\
508	0.00655731567908339\\
509	0.00657921975536935\\
510	0.00660143004915995\\
511	0.00662395238098873\\
512	0.00664679278386983\\
513	0.00666995752547013\\
514	0.00669345313259201\\
515	0.00671728641798934\\
516	0.00674146450951709\\
517	0.00676599488160156\\
518	0.00679088538902184\\
519	0.00681614430301679\\
520	0.00684178034978294\\
521	0.00686780275150313\\
522	0.00689422127013374\\
523	0.00692104625418669\\
524	0.00694828868854222\\
525	0.00697596024712747\\
526	0.00700407334816857\\
527	0.00703264121155625\\
528	0.00706167791764577\\
529	0.00709119846652551\\
530	0.00712121883641764\\
531	0.00715175603939151\\
532	0.00718282817194962\\
533	0.00721445445724499\\
534	0.00724665527466789\\
535	0.00727945217125151\\
536	0.00731286784772672\\
537	0.00734692611002578\\
538	0.00738165177447921\\
539	0.00741707051178375\\
540	0.00745320861031404\\
541	0.00749009263365716\\
542	0.00752774894035735\\
543	0.00756620302093191\\
544	0.00760547867133597\\
545	0.00764559690202095\\
546	0.00768657544560344\\
547	0.00772841462132849\\
548	0.0077711062852949\\
549	0.00781463193735336\\
550	0.00785888938985476\\
551	0.00790390269594911\\
552	0.00794972965265113\\
553	0.00799644535724796\\
554	0.00804424767396752\\
555	0.0080932033127342\\
556	0.00814339871409729\\
557	0.00819497083172656\\
558	0.00824806852407355\\
559	0.00830122220873903\\
560	0.00835292013228987\\
561	0.00840249754434815\\
562	0.00845135966565546\\
563	0.00850010708446728\\
564	0.00854884919147689\\
565	0.00859762193092735\\
566	0.00864573461590588\\
567	0.00869275682914903\\
568	0.0087390226911887\\
569	0.00878512832909994\\
570	0.00883118017564783\\
571	0.0088771673823357\\
572	0.00892306508132109\\
573	0.00896858605394561\\
574	0.00901356557619045\\
575	0.0090584968281205\\
576	0.00910348269220726\\
577	0.00914850897773023\\
578	0.00919353728823878\\
579	0.009238523032971\\
580	0.00928341814856172\\
581	0.00932817169731824\\
582	0.00937273029909558\\
583	0.0094170385870981\\
584	0.00946103980882924\\
585	0.00950467664840494\\
586	0.00954789233109838\\
587	0.00959063207199068\\
588	0.00963284492697341\\
589	0.0096744860587673\\
590	0.00971551928305088\\
591	0.00975591931512916\\
592	0.00979567191274874\\
593	0.00983476678232329\\
594	0.00987312984006299\\
595	0.00991038882774525\\
596	0.00994573390547615\\
597	0.0099771937715668\\
598	0.00999970795535495\\
599	0\\
600	0\\
};
\addplot [color=red,solid,forget plot]
  table[row sep=crcr]{%
1	0.00395928562838554\\
2	0.0039592907711727\\
3	0.00395929601464776\\
4	0.00395930136068423\\
5	0.00395930681118812\\
6	0.00395931236809866\\
7	0.00395931803338862\\
8	0.0039593238090649\\
9	0.00395932969716917\\
10	0.00395933569977824\\
11	0.0039593418190048\\
12	0.00395934805699787\\
13	0.00395935441594341\\
14	0.00395936089806494\\
15	0.00395936750562407\\
16	0.00395937424092114\\
17	0.00395938110629588\\
18	0.0039593881041279\\
19	0.00395939523683741\\
20	0.00395940250688591\\
21	0.00395940991677666\\
22	0.00395941746905555\\
23	0.00395942516631167\\
24	0.00395943301117795\\
25	0.00395944100633191\\
26	0.00395944915449638\\
27	0.00395945745844014\\
28	0.00395946592097865\\
29	0.00395947454497488\\
30	0.00395948333333992\\
31	0.00395949228903388\\
32	0.00395950141506656\\
33	0.00395951071449829\\
34	0.00395952019044062\\
35	0.00395952984605729\\
36	0.00395953968456493\\
37	0.00395954970923394\\
38	0.00395955992338934\\
39	0.00395957033041161\\
40	0.00395958093373761\\
41	0.00395959173686141\\
42	0.00395960274333528\\
43	0.00395961395677051\\
44	0.00395962538083848\\
45	0.00395963701927149\\
46	0.0039596488758638\\
47	0.00395966095447259\\
48	0.00395967325901903\\
49	0.00395968579348925\\
50	0.00395969856193534\\
51	0.00395971156847653\\
52	0.00395972481730017\\
53	0.00395973831266285\\
54	0.00395975205889159\\
55	0.00395976606038494\\
56	0.0039597803216141\\
57	0.00395979484712418\\
58	0.0039598096415354\\
59	0.00395982470954423\\
60	0.00395984005592484\\
61	0.00395985568553015\\
62	0.0039598716032933\\
63	0.00395988781422897\\
64	0.00395990432343463\\
65	0.00395992113609207\\
66	0.00395993825746877\\
67	0.00395995569291926\\
68	0.00395997344788672\\
69	0.00395999152790442\\
70	0.00396000993859723\\
71	0.0039600286856833\\
72	0.00396004777497558\\
73	0.00396006721238343\\
74	0.00396008700391443\\
75	0.00396010715567588\\
76	0.00396012767387676\\
77	0.00396014856482946\\
78	0.00396016983495148\\
79	0.00396019149076748\\
80	0.0039602135389111\\
81	0.00396023598612686\\
82	0.00396025883927226\\
83	0.00396028210531983\\
84	0.00396030579135909\\
85	0.00396032990459887\\
86	0.00396035445236935\\
87	0.00396037944212437\\
88	0.00396040488144378\\
89	0.00396043077803566\\
90	0.00396045713973882\\
91	0.00396048397452527\\
92	0.0039605112905027\\
93	0.00396053909591705\\
94	0.00396056739915523\\
95	0.00396059620874778\\
96	0.00396062553337168\\
97	0.00396065538185319\\
98	0.00396068576317069\\
99	0.00396071668645779\\
100	0.00396074816100637\\
101	0.0039607801962697\\
102	0.00396081280186557\\
103	0.00396084598757979\\
104	0.0039608797633694\\
105	0.00396091413936617\\
106	0.00396094912588029\\
107	0.00396098473340384\\
108	0.00396102097261462\\
109	0.00396105785437995\\
110	0.00396109538976068\\
111	0.00396113359001513\\
112	0.00396117246660317\\
113	0.00396121203119068\\
114	0.00396125229565367\\
115	0.00396129327208287\\
116	0.00396133497278819\\
117	0.00396137741030352\\
118	0.00396142059739151\\
119	0.00396146454704841\\
120	0.00396150927250922\\
121	0.00396155478725283\\
122	0.00396160110500732\\
123	0.00396164823975543\\
124	0.00396169620574011\\
125	0.00396174501747029\\
126	0.00396179468972671\\
127	0.00396184523756794\\
128	0.00396189667633653\\
129	0.00396194902166531\\
130	0.00396200228948398\\
131	0.00396205649602557\\
132	0.00396211165783331\\
133	0.00396216779176769\\
134	0.00396222491501343\\
135	0.00396228304508681\\
136	0.00396234219984329\\
137	0.00396240239748507\\
138	0.0039624636565689\\
139	0.00396252599601417\\
140	0.00396258943511117\\
141	0.00396265399352939\\
142	0.00396271969132629\\
143	0.003962786548956\\
144	0.00396285458727841\\
145	0.00396292382756833\\
146	0.00396299429152501\\
147	0.00396306600128176\\
148	0.00396313897941584\\
149	0.00396321324895854\\
150	0.00396328883340551\\
151	0.00396336575672731\\
152	0.00396344404338017\\
153	0.0039635237183171\\
154	0.00396360480699906\\
155	0.00396368733540644\\
156	0.0039637713300509\\
157	0.0039638568179873\\
158	0.0039639438268259\\
159	0.00396403238474498\\
160	0.00396412252050344\\
161	0.00396421426345396\\
162	0.00396430764355618\\
163	0.0039644026913903\\
164	0.00396449943817088\\
165	0.00396459791576088\\
166	0.00396469815668607\\
167	0.00396480019414965\\
168	0.00396490406204721\\
169	0.00396500979498179\\
170	0.00396511742827964\\
171	0.0039652269980057\\
172	0.00396533854097983\\
173	0.00396545209479315\\
174	0.00396556769782472\\
175	0.00396568538925843\\
176	0.00396580520910035\\
177	0.00396592719819628\\
178	0.00396605139824954\\
179	0.00396617785183936\\
180	0.00396630660243916\\
181	0.00396643769443564\\
182	0.00396657117314766\\
183	0.00396670708484592\\
184	0.00396684547677272\\
185	0.00396698639716208\\
186	0.00396712989526034\\
187	0.0039672760213468\\
188	0.00396742482675508\\
189	0.0039675763638946\\
190	0.0039677306862725\\
191	0.00396788784851592\\
192	0.00396804790639455\\
193	0.00396821091684379\\
194	0.00396837693798812\\
195	0.0039685460291649\\
196	0.00396871825094857\\
197	0.0039688936651753\\
198	0.00396907233496811\\
199	0.00396925432476223\\
200	0.00396943970033113\\
201	0.00396962852881294\\
202	0.0039698208787372\\
203	0.00397001682005237\\
204	0.00397021642415344\\
205	0.00397041976391054\\
206	0.00397062691369768\\
207	0.00397083794942215\\
208	0.0039710529485546\\
209	0.00397127199015954\\
210	0.00397149515492657\\
211	0.00397172252520202\\
212	0.00397195418502149\\
213	0.00397219022014285\\
214	0.00397243071808008\\
215	0.0039726757681377\\
216	0.00397292546144579\\
217	0.00397317989099619\\
218	0.00397343915167897\\
219	0.00397370334032017\\
220	0.00397397255571993\\
221	0.00397424689869178\\
222	0.00397452647210274\\
223	0.00397481138091419\\
224	0.00397510173222396\\
225	0.00397539763530904\\
226	0.00397569920166957\\
227	0.00397600654507365\\
228	0.00397631978160335\\
229	0.00397663902970174\\
230	0.0039769644102209\\
231	0.00397729604647134\\
232	0.0039776340642723\\
233	0.0039779785920034\\
234	0.00397832976065741\\
235	0.00397868770389442\\
236	0.00397905255809703\\
237	0.00397942446242707\\
238	0.00397980355888353\\
239	0.00398018999236181\\
240	0.00398058391071445\\
241	0.00398098546481303\\
242	0.00398139480861178\\
243	0.00398181209921244\\
244	0.00398223749693064\\
245	0.00398267116536377\\
246	0.00398311327146042\\
247	0.00398356398559121\\
248	0.0039840234816214\\
249	0.00398449193698499\\
250	0.00398496953276036\\
251	0.00398545645374786\\
252	0.00398595288854874\\
253	0.00398645902964598\\
254	0.00398697507348706\\
255	0.00398750122056822\\
256	0.00398803767552061\\
257	0.00398858464719871\\
258	0.00398914234877018\\
259	0.00398971099780807\\
260	0.00399029081638484\\
261	0.00399088203116879\\
262	0.00399148487352214\\
263	0.00399209957960171\\
264	0.00399272639046191\\
265	0.00399336555215997\\
266	0.00399401731586366\\
267	0.00399468193796166\\
268	0.00399535968017629\\
269	0.00399605080967924\\
270	0.00399675559920986\\
271	0.00399747432719653\\
272	0.00399820727788084\\
273	0.00399895474144503\\
274	0.00399971701414254\\
275	0.00400049439843196\\
276	0.00400128720311434\\
277	0.00400209574347409\\
278	0.00400292034142352\\
279	0.00400376132565124\\
280	0.00400461903177454\\
281	0.00400549380249566\\
282	0.00400638598776239\\
283	0.00400729594493301\\
284	0.00400822403894548\\
285	0.0040091706424914\\
286	0.0040101361361946\\
287	0.00401112090879458\\
288	0.00401212535733477\\
289	0.00401314988735613\\
290	0.00401419491309591\\
291	0.00401526085769171\\
292	0.00401634815339112\\
293	0.004017457241767\\
294	0.00401858857393864\\
295	0.00401974261079879\\
296	0.0040209198232469\\
297	0.0040221206924286\\
298	0.00402334570998154\\
299	0.00402459537828786\\
300	0.00402587021073369\\
301	0.00402717073197557\\
302	0.00402849747821428\\
303	0.00402985099747633\\
304	0.00403123184990356\\
305	0.00403264060805057\\
306	0.0040340778571902\\
307	0.00403554419562737\\
308	0.00403704023502279\\
309	0.00403856660072659\\
310	0.00404012393212164\\
311	0.00404171288297786\\
312	0.00404333412181746\\
313	0.00404498833229179\\
314	0.00404667621357041\\
315	0.00404839848074283\\
316	0.00405015586523428\\
317	0.00405194911523509\\
318	0.00405377899614491\\
319	0.00405564629103263\\
320	0.00405755180111261\\
321	0.00405949634623842\\
322	0.00406148076541443\\
323	0.00406350591732685\\
324	0.00406557268089453\\
325	0.0040676819558404\\
326	0.00406983466328365\\
327	0.00407203174635561\\
328	0.00407427417083935\\
329	0.00407656292583373\\
330	0.00407889902444354\\
331	0.00408128350449685\\
332	0.00408371742929218\\
333	0.00408620188837577\\
334	0.00408873799834865\\
335	0.00409132690370618\\
336	0.0040939697777115\\
337	0.00409666782330487\\
338	0.00409942227405102\\
339	0.00410223439512729\\
340	0.00410510548435461\\
341	0.00410803687326293\\
342	0.00411102992819091\\
343	0.0041140860514452\\
344	0.00411720668250834\\
345	0.00412039329929913\\
346	0.00412364741948896\\
347	0.00412697060187794\\
348	0.00413036444783145\\
349	0.0041338306027698\\
350	0.00413737075773042\\
351	0.00414098665100229\\
352	0.00414468006983396\\
353	0.00414845285222457\\
354	0.00415230688882021\\
355	0.00415624412490494\\
356	0.00416026656247631\\
357	0.00416437626242426\\
358	0.0041685753468203\\
359	0.00417286600132388\\
360	0.00417725047771393\\
361	0.00418173109655441\\
362	0.00418631025000467\\
363	0.00419099040478617\\
364	0.00419577410532349\\
365	0.00420066397708865\\
366	0.00420566273020264\\
367	0.0042107731633396\\
368	0.00421599816758439\\
369	0.00422134073049449\\
370	0.00422680394048904\\
371	0.00423239099144796\\
372	0.00423810518753114\\
373	0.00424394994851576\\
374	0.0042499288154414\\
375	0.00425604545637169\\
376	0.00426230367243822\\
377	0.00426870740416766\\
378	0.00427526073808825\\
379	0.00428196791360697\\
380	0.00428883333014062\\
381	0.00429586155448043\\
382	0.00430305732837217\\
383	0.00431042557632768\\
384	0.00431797141379822\\
385	0.00432570015593234\\
386	0.00433361732537465\\
387	0.00434172865924067\\
388	0.00435004011691807\\
389	0.00435855788887977\\
390	0.00436728840590924\\
391	0.00437623835386574\\
392	0.00438541468866873\\
393	0.00439482465276074\\
394	0.00440447579325377\\
395	0.00441437598197943\\
396	0.00442453343766441\\
397	0.00443495675045583\\
398	0.00444565490911844\\
399	0.00445663733203324\\
400	0.00446791390676606\\
401	0.00447949503317697\\
402	0.00449139164201457\\
403	0.00450361522064092\\
404	0.00451617781124177\\
405	0.00452909206636683\\
406	0.0045423712713968\\
407	0.00455602933868359\\
408	0.00457008078322276\\
409	0.0045845404776838\\
410	0.00459942361483797\\
411	0.00461474566295675\\
412	0.00463052231443713\\
413	0.00464676943419859\\
414	0.0046635031225445\\
415	0.00468073975085366\\
416	0.00469849576919983\\
417	0.00471678747600711\\
418	0.00473563082646425\\
419	0.00475504198679849\\
420	0.00477503761567556\\
421	0.00479563534401681\\
422	0.00481685453896025\\
423	0.00483871745262209\\
424	0.00486125290232378\\
425	0.0048845032981711\\
426	0.00490851216879996\\
427	0.00493332536941689\\
428	0.00495899115142616\\
429	0.00498555865790608\\
430	0.0050130692192738\\
431	0.00504155731314359\\
432	0.00507104835382907\\
433	0.00510155417785983\\
434	0.00513306701456907\\
435	0.00516555145828589\\
436	0.00519893382830121\\
437	0.00523308810507563\\
438	0.00526781736556338\\
439	0.00530230611598739\\
440	0.00533614661794569\\
441	0.0053692354747199\\
442	0.00540146212332761\\
443	0.00543270970304172\\
444	0.00546285657450566\\
445	0.00549177886011247\\
446	0.00551935417542613\\
447	0.00554546706659329\\
448	0.00557001671615367\\
449	0.00559292758404455\\
450	0.00561416479378256\\
451	0.00563375464816682\\
452	0.00565196719975264\\
453	0.00566969520827555\\
454	0.00568693068663521\\
455	0.00570367118878301\\
456	0.00571992156845385\\
457	0.00573569516086255\\
458	0.00575101491884332\\
459	0.00576591439055096\\
460	0.00578043836880817\\
461	0.00579464296308082\\
462	0.00580859474280781\\
463	0.00582236847192784\\
464	0.0058360427354226\\
465	0.00584967793912761\\
466	0.00586329941713264\\
467	0.00587692197482909\\
468	0.0058905617956592\\
469	0.00590423618122079\\
470	0.00591796319562236\\
471	0.0059317612080084\\
472	0.00594564833497666\\
473	0.00595964179743497\\
474	0.00597375722733093\\
475	0.00598800799244652\\
476	0.0060024046573183\\
477	0.00601695506085117\\
478	0.00603166601973468\\
479	0.00604654425438506\\
480	0.0060615963131767\\
481	0.00607682850015123\\
482	0.00609224681136822\\
483	0.00610785688606976\\
484	0.006123663979571\\
485	0.00613967296488364\\
486	0.00615588836895017\\
487	0.00617231444619037\\
488	0.00618895528558699\\
489	0.00620581490958981\\
490	0.00622289730832914\\
491	0.00624020643976903\\
492	0.00625774623333333\\
493	0.00627552059720033\\
494	0.0062935334292164\\
495	0.00631178863105106\\
496	0.00633029012481586\\
497	0.00634904187092652\\
498	0.0063680478855733\\
499	0.0063873122559112\\
500	0.00640683915122753\\
501	0.0064266328321379\\
502	0.00644669766046583\\
503	0.00646703811018547\\
504	0.00648765877938356\\
505	0.0065085644031848\\
506	0.00652975986758595\\
507	0.00655125022416271\\
508	0.00657304070565595\\
509	0.00659513674251247\\
510	0.00661754398055078\\
511	0.00664026830002966\\
512	0.00666331583640131\\
513	0.00668669300292377\\
514	0.00671040651525138\\
515	0.00673446341812398\\
516	0.00675887111427401\\
517	0.00678363739566562\\
518	0.0068087704771671\\
519	0.00683427903273516\\
520	0.00686017223415273\\
521	0.00688645979230361\\
522	0.00691315200088348\\
523	0.00694025978233218\\
524	0.00696779473562672\\
525	0.00699576918539001\\
526	0.0070241962315293\\
527	0.00705308979830633\\
528	0.00708246468134048\\
529	0.00711233659053203\\
530	0.00714272218623722\\
531	0.0071736391051927\\
532	0.00720510597162972\\
533	0.00723714238767914\\
534	0.0072697688954951\\
535	0.00730300690141519\\
536	0.00733687854989586\\
537	0.00737140653083732\\
538	0.00740661379942434\\
539	0.0074425231780382\\
540	0.00747915683284826\\
541	0.00751653561730621\\
542	0.00755467826988819\\
543	0.00759360095216047\\
544	0.00763331238082942\\
545	0.00767381095643044\\
546	0.007715025844359\\
547	0.0077569285435327\\
548	0.0077995714378718\\
549	0.00784302124100872\\
550	0.00788745658537758\\
551	0.00793295460546916\\
552	0.00797958458081587\\
553	0.00802743516966814\\
554	0.00807661277614705\\
555	0.00812725401186005\\
556	0.00817887856530284\\
557	0.0082291129550715\\
558	0.00827735421505801\\
559	0.0083245490587445\\
560	0.00837172262866089\\
561	0.00841901681308049\\
562	0.0084664844422458\\
563	0.00851408246692262\\
564	0.00856102546121805\\
565	0.00860690495770123\\
566	0.00865232480719928\\
567	0.00869771432457174\\
568	0.00874313870516814\\
569	0.00878858866114348\\
570	0.0088340308653561\\
571	0.00887936447173519\\
572	0.00892418360391415\\
573	0.00896884111831802\\
574	0.0090136165177729\\
575	0.00905850904675177\\
576	0.0091034868427784\\
577	0.0091485107227293\\
578	0.00919353811604447\\
579	0.00923852346941265\\
580	0.00928341839068734\\
581	0.0093281718340145\\
582	0.00937273037639527\\
583	0.0094170386295485\\
584	0.00946103983027298\\
585	0.00950467665772169\\
586	0.0095478923342439\\
587	0.00959063207272952\\
588	0.00963284492704104\\
589	0.0096744860587673\\
590	0.00971551928305088\\
591	0.00975591931512916\\
592	0.00979567191274874\\
593	0.00983476678232329\\
594	0.00987312984006299\\
595	0.00991038882774525\\
596	0.00994573390547615\\
597	0.0099771937715668\\
598	0.00999970795535495\\
599	0\\
600	0\\
};
\addplot [color=mycolor20,solid,forget plot]
  table[row sep=crcr]{%
1	0.00395948440225352\\
2	0.00395949092226044\\
3	0.00395949757490838\\
4	0.00395950436277181\\
5	0.00395951128847175\\
6	0.00395951835467654\\
7	0.00395952556410272\\
8	0.00395953291951572\\
9	0.00395954042373065\\
10	0.00395954807961318\\
11	0.00395955589008015\\
12	0.00395956385810068\\
13	0.00395957198669679\\
14	0.00395958027894421\\
15	0.00395958873797342\\
16	0.00395959736697037\\
17	0.00395960616917737\\
18	0.00395961514789404\\
19	0.00395962430647817\\
20	0.00395963364834655\\
21	0.00395964317697609\\
22	0.00395965289590453\\
23	0.00395966280873147\\
24	0.00395967291911941\\
25	0.0039596832307946\\
26	0.00395969374754803\\
27	0.00395970447323651\\
28	0.00395971541178364\\
29	0.00395972656718072\\
30	0.00395973794348798\\
31	0.00395974954483542\\
32	0.003959761375424\\
33	0.00395977343952666\\
34	0.00395978574148945\\
35	0.00395979828573252\\
36	0.00395981107675135\\
37	0.0039598241191178\\
38	0.00395983741748127\\
39	0.00395985097656992\\
40	0.00395986480119162\\
41	0.00395987889623541\\
42	0.00395989326667255\\
43	0.0039599079175577\\
44	0.00395992285403025\\
45	0.00395993808131548\\
46	0.00395995360472583\\
47	0.00395996942966224\\
48	0.00395998556161533\\
49	0.00396000200616676\\
50	0.00396001876899063\\
51	0.00396003585585468\\
52	0.00396005327262173\\
53	0.00396007102525107\\
54	0.00396008911979978\\
55	0.00396010756242421\\
56	0.0039601263593814\\
57	0.00396014551703048\\
58	0.00396016504183415\\
59	0.00396018494036029\\
60	0.00396020521928323\\
61	0.00396022588538556\\
62	0.00396024694555945\\
63	0.00396026840680833\\
64	0.00396029027624851\\
65	0.00396031256111069\\
66	0.0039603352687417\\
67	0.00396035840660617\\
68	0.00396038198228812\\
69	0.00396040600349278\\
70	0.00396043047804829\\
71	0.00396045541390743\\
72	0.0039604808191495\\
73	0.00396050670198204\\
74	0.00396053307074279\\
75	0.00396055993390153\\
76	0.00396058730006192\\
77	0.00396061517796358\\
78	0.00396064357648399\\
79	0.00396067250464044\\
80	0.0039607019715922\\
81	0.00396073198664259\\
82	0.003960762559241\\
83	0.00396079369898509\\
84	0.00396082541562309\\
85	0.00396085771905589\\
86	0.00396089061933936\\
87	0.00396092412668679\\
88	0.00396095825147106\\
89	0.00396099300422723\\
90	0.00396102839565496\\
91	0.00396106443662097\\
92	0.00396110113816164\\
93	0.00396113851148566\\
94	0.00396117656797667\\
95	0.00396121531919602\\
96	0.00396125477688555\\
97	0.0039612949529704\\
98	0.00396133585956208\\
99	0.00396137750896132\\
100	0.00396141991366111\\
101	0.00396146308634993\\
102	0.00396150703991493\\
103	0.00396155178744514\\
104	0.00396159734223496\\
105	0.0039616437177874\\
106	0.00396169092781778\\
107	0.00396173898625735\\
108	0.00396178790725683\\
109	0.00396183770519044\\
110	0.00396188839465959\\
111	0.00396193999049702\\
112	0.00396199250777091\\
113	0.00396204596178899\\
114	0.00396210036810296\\
115	0.00396215574251297\\
116	0.00396221210107214\\
117	0.00396226946009127\\
118	0.00396232783614365\\
119	0.00396238724607009\\
120	0.00396244770698396\\
121	0.00396250923627652\\
122	0.00396257185162219\\
123	0.00396263557098424\\
124	0.00396270041262049\\
125	0.00396276639508912\\
126	0.00396283353725475\\
127	0.00396290185829468\\
128	0.00396297137770534\\
129	0.00396304211530887\\
130	0.00396311409125984\\
131	0.00396318732605236\\
132	0.00396326184052723\\
133	0.00396333765587933\\
134	0.00396341479366529\\
135	0.00396349327581143\\
136	0.00396357312462168\\
137	0.00396365436278606\\
138	0.00396373701338924\\
139	0.00396382109991931\\
140	0.003963906646277\\
141	0.00396399367678494\\
142	0.00396408221619734\\
143	0.00396417228970994\\
144	0.00396426392297022\\
145	0.00396435714208795\\
146	0.00396445197364597\\
147	0.00396454844471137\\
148	0.00396464658284689\\
149	0.0039647464161228\\
150	0.003964847973129\\
151	0.00396495128298749\\
152	0.00396505637536522\\
153	0.0039651632804872\\
154	0.00396527202915017\\
155	0.0039653826527365\\
156	0.00396549518322848\\
157	0.00396560965322306\\
158	0.00396572609594695\\
159	0.00396584454527215\\
160	0.00396596503573188\\
161	0.00396608760253688\\
162	0.00396621228159229\\
163	0.00396633910951471\\
164	0.00396646812364995\\
165	0.00396659936209115\\
166	0.00396673286369724\\
167	0.00396686866811202\\
168	0.00396700681578351\\
169	0.00396714734798408\\
170	0.00396729030683059\\
171	0.00396743573530554\\
172	0.00396758367727822\\
173	0.00396773417752669\\
174	0.00396788728176\\
175	0.00396804303664116\\
176	0.00396820148981032\\
177	0.00396836268990858\\
178	0.00396852668660243\\
179	0.00396869353060839\\
180	0.0039688632737184\\
181	0.00396903596882557\\
182	0.00396921166995054\\
183	0.00396939043226819\\
184	0.00396957231213497\\
185	0.00396975736711667\\
186	0.00396994565601665\\
187	0.00397013723890466\\
188	0.00397033217714608\\
189	0.00397053053343162\\
190	0.00397073237180753\\
191	0.00397093775770633\\
192	0.00397114675797796\\
193	0.00397135944092138\\
194	0.0039715758763167\\
195	0.00397179613545778\\
196	0.00397202029118517\\
197	0.00397224841791972\\
198	0.00397248059169632\\
199	0.00397271689019849\\
200	0.00397295739279313\\
201	0.00397320218056577\\
202	0.00397345133635636\\
203	0.00397370494479541\\
204	0.00397396309234069\\
205	0.0039742258673142\\
206	0.00397449335993978\\
207	0.00397476566238107\\
208	0.00397504286878006\\
209	0.00397532507529596\\
210	0.00397561238014456\\
211	0.00397590488363834\\
212	0.00397620268822686\\
213	0.00397650589853769\\
214	0.00397681462141795\\
215	0.00397712896597649\\
216	0.0039774490436266\\
217	0.00397777496812941\\
218	0.00397810685563781\\
219	0.00397844482474121\\
220	0.00397878899651109\\
221	0.0039791394945472\\
222	0.0039794964450246\\
223	0.00397985997674165\\
224	0.00398023022116887\\
225	0.00398060731249876\\
226	0.00398099138769676\\
227	0.00398138258655321\\
228	0.0039817810517365\\
229	0.00398218692884742\\
230	0.00398260036647489\\
231	0.00398302151625291\\
232	0.00398345053291904\\
233	0.00398388757437447\\
234	0.00398433280174537\\
235	0.00398478637944622\\
236	0.00398524847524468\\
237	0.00398571926032826\\
238	0.0039861989093728\\
239	0.00398668760061308\\
240	0.00398718551591517\\
241	0.00398769284085103\\
242	0.0039882097647751\\
243	0.0039887364809031\\
244	0.00398927318639304\\
245	0.00398982008242858\\
246	0.00399037737430452\\
247	0.00399094527151491\\
248	0.0039915239878433\\
249	0.00399211374145552\\
250	0.00399271475499502\\
251	0.00399332725568042\\
252	0.00399395147540588\\
253	0.00399458765084366\\
254	0.00399523602354945\\
255	0.00399589684007014\\
256	0.00399657035205428\\
257	0.00399725681636487\\
258	0.00399795649519504\\
259	0.00399866965618633\\
260	0.00399939657254961\\
261	0.00400013752318873\\
262	0.00400089279282669\\
263	0.00400166267213369\\
264	0.00400244745785881\\
265	0.00400324745296446\\
266	0.00400406296676403\\
267	0.00400489431506205\\
268	0.00400574182029793\\
269	0.00400660581169249\\
270	0.00400748662539804\\
271	0.00400838460465164\\
272	0.00400930009993214\\
273	0.00401023346912073\\
274	0.00401118507766548\\
275	0.00401215529874979\\
276	0.00401314451346527\\
277	0.00401415311098896\\
278	0.00401518148876539\\
279	0.00401623005269339\\
280	0.0040172992173184\\
281	0.00401838940603019\\
282	0.00401950105126649\\
283	0.00402063459472275\\
284	0.00402179048756856\\
285	0.00402296919067083\\
286	0.00402417117482429\\
287	0.00402539692098974\\
288	0.00402664692054038\\
289	0.00402792167551637\\
290	0.00402922169888853\\
291	0.00403054751483122\\
292	0.00403189965900493\\
293	0.00403327867884902\\
294	0.00403468513388493\\
295	0.00403611959603054\\
296	0.00403758264992584\\
297	0.00403907489327053\\
298	0.00404059693717357\\
299	0.0040421494065143\\
300	0.00404373294031623\\
301	0.00404534819213448\\
302	0.00404699583045704\\
303	0.00404867653911984\\
304	0.00405039101773675\\
305	0.00405213998214593\\
306	0.00405392416487165\\
307	0.00405574431559477\\
308	0.00405760120163317\\
309	0.00405949560844524\\
310	0.00406142834014913\\
311	0.00406340022005844\\
312	0.00406541209123613\\
313	0.0040674648170667\\
314	0.00406955928184599\\
315	0.00407169639139072\\
316	0.00407387707366838\\
317	0.0040761022794489\\
318	0.00407837298297942\\
319	0.00408069018268334\\
320	0.0040830549018849\\
321	0.00408546818956152\\
322	0.00408793112112511\\
323	0.00409044479923499\\
324	0.00409301035464481\\
325	0.00409562894708605\\
326	0.00409830176618714\\
327	0.00410103003241982\\
328	0.00410381499809582\\
329	0.00410665794840662\\
330	0.00410956020250733\\
331	0.00411252311464709\\
332	0.00411554807535199\\
333	0.00411863651267666\\
334	0.00412178989352366\\
335	0.00412500972501356\\
336	0.0041282975559241\\
337	0.00413165497820438\\
338	0.00413508362857238\\
339	0.00413858519020842\\
340	0.00414216139456864\\
341	0.00414581402334682\\
342	0.00414954491048962\\
343	0.00415335594420658\\
344	0.00415724906924379\\
345	0.00416122628927412\\
346	0.00416528966941353\\
347	0.00416944133887728\\
348	0.00417368349379616\\
349	0.00417801840018998\\
350	0.00418244839695065\\
351	0.00418697589901957\\
352	0.00419160340072899\\
353	0.00419633347925919\\
354	0.0042011687982348\\
355	0.00420611211169295\\
356	0.00421116626827441\\
357	0.0042163342154261\\
358	0.00422161900375006\\
359	0.00422702379147992\\
360	0.00423255184905963\\
361	0.00423820656379134\\
362	0.00424399144451153\\
363	0.00424991012625111\\
364	0.00425596637483545\\
365	0.00426216409140471\\
366	0.00426850731693583\\
367	0.00427500023721365\\
368	0.00428164718949901\\
369	0.00428845266672812\\
370	0.00429542132320961\\
371	0.00430255798251003\\
372	0.00430986764586282\\
373	0.0043173555005868\\
374	0.00432502693245215\\
375	0.00433288754011261\\
376	0.0043409431491925\\
377	0.00434919982807049\\
378	0.00435766390559816\\
379	0.0043663419909905\\
380	0.0043752409961034\\
381	0.00438436816024748\\
382	0.00439373107756318\\
383	0.00440333772678778\\
384	0.00441319650309811\\
385	0.00442331625241044\\
386	0.00443370631256225\\
387	0.0044443765401406\\
388	0.00445533731535654\\
389	0.00446659953479586\\
390	0.00447817458043723\\
391	0.00449007413241409\\
392	0.00450231015283447\\
393	0.00451489485466701\\
394	0.00452784066455827\\
395	0.00454116018000212\\
396	0.00455486612197178\\
397	0.00456897128510276\\
398	0.00458348848880526\\
399	0.00459843053440882\\
400	0.00461381017938842\\
401	0.00462964021231349\\
402	0.00464593365578756\\
403	0.00466270372799138\\
404	0.00467996405031999\\
405	0.00469772858203319\\
406	0.00471601260218599\\
407	0.00473483384719461\\
408	0.00475421481069545\\
409	0.00477419142017473\\
410	0.00479480144025236\\
411	0.0048160842116969\\
412	0.00483808022758728\\
413	0.00486083046362899\\
414	0.00488437519333277\\
415	0.00490875381440622\\
416	0.00493400459263738\\
417	0.00496016133776183\\
418	0.00498724818555258\\
419	0.00501527111677807\\
420	0.00504421592126254\\
421	0.00507404009149779\\
422	0.00510466208601866\\
423	0.00513594713979381\\
424	0.00516755735354668\\
425	0.00519871705977875\\
426	0.00522934403357216\\
427	0.00525934920615354\\
428	0.00528863680824175\\
429	0.00531710488513274\\
430	0.00534464641881114\\
431	0.00537115119851654\\
432	0.00539650857693217\\
433	0.005420611496474\\
434	0.00544336227568943\\
435	0.0054646808461695\\
436	0.00548451611396553\\
437	0.00550286155930679\\
438	0.00551977644864224\\
439	0.0055359496527542\\
440	0.00555167583479705\\
441	0.00556694691578936\\
442	0.00558176016901808\\
443	0.00559611929440776\\
444	0.00561003529505826\\
445	0.00562352682800227\\
446	0.00563662179526778\\
447	0.00564935798415351\\
448	0.00566178325469104\\
449	0.00567395498949571\\
450	0.00568593839927066\\
451	0.00569780311094088\\
452	0.00570961156569204\\
453	0.00572138941486999\\
454	0.00573314893998529\\
455	0.00574490374263054\\
456	0.00575666857816241\\
457	0.00576845910233506\\
458	0.00578029153242253\\
459	0.00579218221910404\\
460	0.00580414713343451\\
461	0.00581620128659636\\
462	0.00582835812063487\\
463	0.00584062893949173\\
464	0.00585302249683192\\
465	0.00586554548556271\\
466	0.00587820392422858\\
467	0.00589100374351042\\
468	0.00590395071910557\\
469	0.00591705040727884\\
470	0.00593030808745506\\
471	0.00594372871708371\\
472	0.00595731690465106\\
473	0.00597107690688334\\
474	0.00598501265543397\\
475	0.00599912781599312\\
476	0.00601342587775641\\
477	0.00602791025067907\\
478	0.00604258430320072\\
479	0.0060574513602477\\
480	0.00607251470399852\\
481	0.00608777757761638\\
482	0.00610324319197112\\
483	0.00611891473512403\\
484	0.00613479538403393\\
485	0.00615088831758091\\
486	0.0061671967296386\\
487	0.00618372384064446\\
488	0.00620047290607825\\
489	0.00621744722177797\\
490	0.00623465012840185\\
491	0.00625208501643772\\
492	0.00626975533169465\\
493	0.0062876645811952\\
494	0.00630581633937912\\
495	0.00632421425453326\\
496	0.00634286205538325\\
497	0.006361763557825\\
498	0.00638092267183858\\
499	0.00640034340871236\\
500	0.00642002988879402\\
501	0.00643998634992998\\
502	0.00646021715666276\\
503	0.00648072681024703\\
504	0.00650151995955475\\
505	0.0065226014129498\\
506	0.00654397615122305\\
507	0.00656564934168963\\
508	0.0065876263535596\\
509	0.00660991277469965\\
510	0.00663251442990518\\
511	0.00665543740079762\\
512	0.00667868804745377\\
513	0.00670227303186607\\
514	0.00672619934332242\\
515	0.00675047432577689\\
516	0.00677510570725678\\
517	0.00680010163131477\\
518	0.00682547069048237\\
519	0.00685122196161199\\
520	0.00687736504289977\\
521	0.00690391009225805\\
522	0.00693086786654527\\
523	0.00695824976095131\\
524	0.00698606784756793\\
525	0.007014334911828\\
526	0.0070430644850565\\
527	0.00707227087081502\\
528	0.00710196916200891\\
529	0.00713217524482474\\
530	0.00716290578442511\\
531	0.00719417818590626\\
532	0.0072260105222357\\
533	0.00725842141872715\\
534	0.00729142987987181\\
535	0.00732505504062249\\
536	0.00735931581314322\\
537	0.00739423044572161\\
538	0.00742981597050931\\
539	0.00746608786335217\\
540	0.00750305817500797\\
541	0.00754073159948061\\
542	0.00757909781928992\\
543	0.00761806654300156\\
544	0.00765769760109196\\
545	0.00769805034120224\\
546	0.00773925844101698\\
547	0.00778143801340248\\
548	0.00782464807394301\\
549	0.00786896435375635\\
550	0.00791445734509906\\
551	0.00796120660613397\\
552	0.00800933710507936\\
553	0.00805900257046526\\
554	0.0081081314115405\\
555	0.00815551517192216\\
556	0.00820105298443216\\
557	0.00824664627952422\\
558	0.00829240598531143\\
559	0.00833844424324396\\
560	0.00838475291434543\\
561	0.00843125792622228\\
562	0.00847722774953948\\
563	0.00852215806893428\\
564	0.00856674750998803\\
565	0.00861139972405399\\
566	0.00865616050071849\\
567	0.0087010135091651\\
568	0.00874592342992904\\
569	0.00879085950398949\\
570	0.00883551799252388\\
571	0.00887980405825824\\
572	0.00892423816410628\\
573	0.00896885007363452\\
574	0.00901361838983874\\
575	0.00905850969697006\\
576	0.00910348712146237\\
577	0.0091485108571932\\
578	0.00919353818752705\\
579	0.00923852350916325\\
580	0.00928341841310162\\
581	0.00932817184660867\\
582	0.00937273038322628\\
583	0.00941703863294158\\
584	0.00946103983171765\\
585	0.00950467665819966\\
586	0.00954789233435323\\
587	0.00959063207273936\\
588	0.00963284492704104\\
589	0.0096744860587673\\
590	0.00971551928305087\\
591	0.00975591931512916\\
592	0.00979567191274874\\
593	0.00983476678232329\\
594	0.00987312984006299\\
595	0.00991038882774525\\
596	0.00994573390547615\\
597	0.0099771937715668\\
598	0.00999970795535495\\
599	0\\
600	0\\
};
\addplot [color=mycolor21,solid,forget plot]
  table[row sep=crcr]{%
1	0.0039595779923929\\
2	0.00395958604364415\\
3	0.00395959426546396\\
4	0.00395960266133263\\
5	0.00395961123479755\\
6	0.00395961998947425\\
7	0.00395962892904768\\
8	0.00395963805727339\\
9	0.00395964737797872\\
10	0.00395965689506409\\
11	0.00395966661250426\\
12	0.00395967653434954\\
13	0.00395968666472715\\
14	0.00395969700784251\\
15	0.00395970756798054\\
16	0.00395971834950707\\
17	0.00395972935687016\\
18	0.0039597405946015\\
19	0.00395975206731777\\
20	0.00395976377972216\\
21	0.00395977573660569\\
22	0.00395978794284877\\
23	0.00395980040342258\\
24	0.00395981312339065\\
25	0.00395982610791038\\
26	0.00395983936223449\\
27	0.00395985289171264\\
28	0.00395986670179298\\
29	0.00395988079802378\\
30	0.00395989518605494\\
31	0.00395990987163979\\
32	0.00395992486063655\\
33	0.00395994015901015\\
34	0.00395995577283383\\
35	0.0039599717082909\\
36	0.00395998797167647\\
37	0.00396000456939915\\
38	0.00396002150798283\\
39	0.0039600387940685\\
40	0.00396005643441613\\
41	0.00396007443590634\\
42	0.0039600928055423\\
43	0.00396011155045166\\
44	0.0039601306778884\\
45	0.00396015019523475\\
46	0.00396017011000306\\
47	0.00396019042983783\\
48	0.00396021116251764\\
49	0.00396023231595714\\
50	0.00396025389820902\\
51	0.00396027591746612\\
52	0.00396029838206338\\
53	0.00396032130047999\\
54	0.00396034468134141\\
55	0.00396036853342153\\
56	0.00396039286564475\\
57	0.00396041768708813\\
58	0.00396044300698361\\
59	0.00396046883472013\\
60	0.00396049517984587\\
61	0.0039605220520704\\
62	0.00396054946126702\\
63	0.00396057741747498\\
64	0.00396060593090174\\
65	0.00396063501192532\\
66	0.00396066467109651\\
67	0.00396069491914137\\
68	0.00396072576696344\\
69	0.00396075722564619\\
70	0.00396078930645542\\
71	0.00396082202084159\\
72	0.00396085538044239\\
73	0.00396088939708511\\
74	0.00396092408278909\\
75	0.00396095944976823\\
76	0.00396099551043356\\
77	0.00396103227739571\\
78	0.00396106976346744\\
79	0.00396110798166629\\
80	0.00396114694521711\\
81	0.00396118666755466\\
82	0.0039612271623263\\
83	0.00396126844339455\\
84	0.00396131052483993\\
85	0.00396135342096341\\
86	0.00396139714628938\\
87	0.00396144171556819\\
88	0.00396148714377905\\
89	0.0039615334461327\\
90	0.00396158063807436\\
91	0.00396162873528638\\
92	0.00396167775369127\\
93	0.0039617277094545\\
94	0.00396177861898743\\
95	0.00396183049895023\\
96	0.0039618833662549\\
97	0.00396193723806822\\
98	0.00396199213181484\\
99	0.00396204806518025\\
100	0.00396210505611397\\
101	0.00396216312283263\\
102	0.00396222228382321\\
103	0.00396228255784615\\
104	0.00396234396393865\\
105	0.00396240652141803\\
106	0.00396247024988497\\
107	0.0039625351692269\\
108	0.00396260129962157\\
109	0.00396266866154032\\
110	0.00396273727575188\\
111	0.00396280716332576\\
112	0.00396287834563605\\
113	0.00396295084436504\\
114	0.00396302468150717\\
115	0.0039630998793727\\
116	0.00396317646059185\\
117	0.00396325444811873\\
118	0.00396333386523545\\
119	0.00396341473555636\\
120	0.00396349708303233\\
121	0.00396358093195515\\
122	0.00396366630696207\\
123	0.00396375323304033\\
124	0.00396384173553193\\
125	0.00396393184013854\\
126	0.00396402357292645\\
127	0.00396411696033168\\
128	0.00396421202916528\\
129	0.00396430880661874\\
130	0.00396440732026953\\
131	0.00396450759808701\\
132	0.00396460966843818\\
133	0.00396471356009394\\
134	0.00396481930223533\\
135	0.00396492692446012\\
136	0.00396503645678954\\
137	0.00396514792967523\\
138	0.00396526137400653\\
139	0.00396537682111794\\
140	0.0039654943027968\\
141	0.00396561385129142\\
142	0.00396573549931932\\
143	0.00396585928007585\\
144	0.0039659852272432\\
145	0.00396611337499961\\
146	0.00396624375802904\\
147	0.00396637641153117\\
148	0.00396651137123178\\
149	0.00396664867339348\\
150	0.00396678835482697\\
151	0.00396693045290267\\
152	0.00396707500556277\\
153	0.00396722205133386\\
154	0.00396737162933989\\
155	0.00396752377931578\\
156	0.0039676785416215\\
157	0.00396783595725674\\
158	0.00396799606787609\\
159	0.00396815891580486\\
160	0.00396832454405552\\
161	0.00396849299634472\\
162	0.00396866431711102\\
163	0.0039688385515333\\
164	0.00396901574554988\\
165	0.00396919594587829\\
166	0.00396937920003592\\
167	0.00396956555636119\\
168	0.00396975506403595\\
169	0.00396994777310817\\
170	0.00397014373451592\\
171	0.00397034300011188\\
172	0.00397054562268897\\
173	0.00397075165600669\\
174	0.00397096115481857\\
175	0.0039711741749004\\
176	0.00397139077307949\\
177	0.00397161100726493\\
178	0.00397183493647885\\
179	0.00397206262088863\\
180	0.00397229412184027\\
181	0.00397252950189264\\
182	0.00397276882485293\\
183	0.00397301215581311\\
184	0.00397325956118748\\
185	0.00397351110875133\\
186	0.00397376686768065\\
187	0.00397402690859303\\
188	0.00397429130358959\\
189	0.003974560126298\\
190	0.00397483345191671\\
191	0.00397511135726011\\
192	0.00397539392080506\\
193	0.00397568122273813\\
194	0.00397597334500428\\
195	0.00397627037135636\\
196	0.00397657238740573\\
197	0.00397687948067391\\
198	0.00397719174064521\\
199	0.00397750925882038\\
200	0.00397783212877106\\
201	0.00397816044619533\\
202	0.00397849430897405\\
203	0.003978833817228\\
204	0.00397917907337592\\
205	0.00397953018219317\\
206	0.00397988725087129\\
207	0.0039802503890781\\
208	0.00398061970901842\\
209	0.00398099532549547\\
210	0.00398137735597267\\
211	0.00398176592063605\\
212	0.003982161142457\\
213	0.00398256314725538\\
214	0.0039829720637631\\
215	0.00398338802368782\\
216	0.00398381116177707\\
217	0.00398424161588237\\
218	0.00398467952702383\\
219	0.00398512503945458\\
220	0.00398557830072549\\
221	0.00398603946175003\\
222	0.0039865086768691\\
223	0.00398698610391609\\
224	0.00398747190428195\\
225	0.00398796624298054\\
226	0.00398846928871391\\
227	0.003988981213938\\
228	0.00398950219492847\\
229	0.00399003241184696\\
230	0.00399057204880772\\
231	0.00399112129394473\\
232	0.00399168033947957\\
233	0.00399224938178988\\
234	0.00399282862147892\\
235	0.0039934182634461\\
236	0.00399401851695871\\
237	0.00399462959572526\\
238	0.00399525171797025\\
239	0.00399588510651104\\
240	0.0039965299888365\\
241	0.00399718659718832\\
242	0.00399785516864467\\
243	0.0039985359452067\\
244	0.00399922917388823\\
245	0.00399993510680861\\
246	0.0040006540012893\\
247	0.00400138611995427\\
248	0.00400213173083437\\
249	0.00400289110747621\\
250	0.00400366452905542\\
251	0.00400445228049482\\
252	0.00400525465258755\\
253	0.00400607194212532\\
254	0.00400690445203214\\
255	0.0040077524915033\\
256	0.00400861637615048\\
257	0.00400949642815204\\
258	0.00401039297640977\\
259	0.00401130635671139\\
260	0.00401223691189968\\
261	0.00401318499204848\\
262	0.00401415095464581\\
263	0.00401513516478111\\
264	0.00401613799532841\\
265	0.00401715982714415\\
266	0.00401820104927012\\
267	0.00401926205914198\\
268	0.00402034326280314\\
269	0.00402144507512384\\
270	0.0040225679200258\\
271	0.00402371223071199\\
272	0.00402487844990196\\
273	0.00402606703007211\\
274	0.00402727843370165\\
275	0.00402851313352362\\
276	0.00402977161278135\\
277	0.00403105436549025\\
278	0.00403236189670511\\
279	0.00403369472279303\\
280	0.00403505337171201\\
281	0.00403643838329548\\
282	0.00403785030954319\\
283	0.00403928971491835\\
284	0.00404075717665173\\
285	0.0040422532850531\\
286	0.00404377864383033\\
287	0.00404533387041699\\
288	0.00404691959630863\\
289	0.00404853646740905\\
290	0.00405018514438676\\
291	0.00405186630304276\\
292	0.00405358063469048\\
293	0.00405532884654885\\
294	0.00405711166214952\\
295	0.00405892982175949\\
296	0.00406078408282071\\
297	0.00406267522040868\\
298	0.00406460402771199\\
299	0.00406657131653281\\
300	0.00406857791780154\\
301	0.00407062468211074\\
302	0.00407271248028372\\
303	0.00407484220397271\\
304	0.00407701476629087\\
305	0.00407923110248637\\
306	0.00408149217067528\\
307	0.00408379895264865\\
308	0.00408615245466971\\
309	0.00408855370825715\\
310	0.00409100377110257\\
311	0.00409350372802893\\
312	0.0040960546919937\\
313	0.00409865780514698\\
314	0.00410131423994946\\
315	0.00410402520032743\\
316	0.00410679192288071\\
317	0.00410961567814609\\
318	0.00411249777191874\\
319	0.00411543954663444\\
320	0.00411844238281576\\
321	0.00412150770058498\\
322	0.00412463696124773\\
323	0.00412783166895115\\
324	0.00413109337242268\\
325	0.00413442366680018\\
326	0.00413782419556743\\
327	0.00414129665258353\\
328	0.00414484278404729\\
329	0.00414846439064309\\
330	0.00415216332975872\\
331	0.00415594151775714\\
332	0.00415980093227251\\
333	0.00416374361452242\\
334	0.00416777167176996\\
335	0.00417188727990086\\
336	0.0041760926858126\\
337	0.00418039020972506\\
338	0.00418478224736796\\
339	0.00418927127200665\\
340	0.00419385983630205\\
341	0.00419855057414037\\
342	0.00420334620293092\\
343	0.00420824952551619\\
344	0.00421326343108173\\
345	0.00421839089884848\\
346	0.00422363500205102\\
347	0.00422899891226\\
348	0.00423448590416088\\
349	0.00424009936103744\\
350	0.00424584278125957\\
351	0.00425171978374354\\
352	0.00425773411458507\\
353	0.00426388965471359\\
354	0.00427019042807163\\
355	0.00427664061022658\\
356	0.00428324454059173\\
357	0.00429000673728246\\
358	0.00429693191220291\\
359	0.00430402498825019\\
360	0.00431129111883286\\
361	0.00431873570982133\\
362	0.00432636444389655\\
363	0.00433418330700377\\
364	0.00434219861619004\\
365	0.00435041704742423\\
366	0.00435884566096149\\
367	0.00436749192047545\\
368	0.00437636370227233\\
369	0.0043854693070619\\
370	0.00439481733945904\\
371	0.00440441665298403\\
372	0.00441427632905858\\
373	0.00442440564977241\\
374	0.00443481404823458\\
375	0.0044455110846419\\
376	0.00445650643044544\\
377	0.00446780982669083\\
378	0.0044794310382552\\
379	0.00449137980585808\\
380	0.0045036657990461\\
381	0.00451629857524127\\
382	0.00452928755259884\\
383	0.00454264200807975\\
384	0.00455637111704512\\
385	0.00457048405738908\\
386	0.00458499021645212\\
387	0.00459989966492497\\
388	0.0046152237597833\\
389	0.00463097589253995\\
390	0.00464717271358021\\
391	0.00466384263704998\\
392	0.00468101605901409\\
393	0.00469872520378388\\
394	0.00471700401685769\\
395	0.0047358879622072\\
396	0.00475541368900525\\
397	0.00477561852173576\\
398	0.00479653971262268\\
399	0.00481821337350695\\
400	0.00484067296282239\\
401	0.00486394705273863\\
402	0.00488805700361895\\
403	0.00491301544051242\\
404	0.00493881990764823\\
405	0.00496544710475881\\
406	0.00499283797409538\\
407	0.00502088860658765\\
408	0.0050493821509829\\
409	0.00507758634680132\\
410	0.00510543767386323\\
411	0.0051328667390106\\
412	0.00515979817508445\\
413	0.00518615071814976\\
414	0.00521183754728799\\
415	0.00523676696628491\\
416	0.00526084355604888\\
417	0.00528397003915966\\
418	0.00530605018083721\\
419	0.00532699312582212\\
420	0.00534671949017847\\
421	0.00536516976463816\\
422	0.00538231592593325\\
423	0.0053981773002428\\
424	0.00541297614030603\\
425	0.00542738618063539\\
426	0.00544139449392937\\
427	0.00545499164732757\\
428	0.00546817248068545\\
429	0.00548093692074996\\
430	0.00549329120131213\\
431	0.00550524851055586\\
432	0.00551682983204708\\
433	0.00552806467282283\\
434	0.00553899123328039\\
435	0.00554965577016223\\
436	0.00556011217760635\\
437	0.00557041976927555\\
438	0.0055806392335274\\
439	0.0055908064169165\\
440	0.00560093578277078\\
441	0.0056110380886052\\
442	0.00562112525464462\\
443	0.00563121021799256\\
444	0.00564130672562363\\
445	0.00565142906153454\\
446	0.00566159169729224\\
447	0.00567180885636397\\
448	0.0056820940158248\\
449	0.00569245937415726\\
450	0.00570291533896705\\
451	0.00571347012593252\\
452	0.00572412983063176\\
453	0.00573489977471728\\
454	0.00574578523871036\\
455	0.00575679140386329\\
456	0.0057679232939351\\
457	0.00577918572038726\\
458	0.00579058323498188\\
459	0.00580212009446097\\
460	0.00581380024243527\\
461	0.00582562731353743\\
462	0.0058376046639026\\
463	0.00584973542950027\\
464	0.00586202260878462\\
465	0.00587446913550529\\
466	0.00588707790151843\\
467	0.00589985175451819\\
468	0.00591279349809862\\
469	0.00592590589432517\\
470	0.00593919166884342\\
471	0.00595265351834822\\
472	0.00596629411997525\\
473	0.00598011614187412\\
474	0.00599412225390815\\
475	0.00600831513716732\\
476	0.00602269749089862\\
477	0.00603727203620461\\
478	0.00605204151852332\\
479	0.00606700871045744\\
480	0.00608217641488946\\
481	0.00609754746830291\\
482	0.00611312474422012\\
483	0.00612891115666342\\
484	0.00614490966355826\\
485	0.00616112327002145\\
486	0.00617755503152311\\
487	0.00619420805697205\\
488	0.00621108551184649\\
489	0.00622819062151137\\
490	0.0062455266747875\\
491	0.0062630970277878\\
492	0.0062809051080401\\
493	0.00629895441892298\\
494	0.00631724854444748\\
495	0.0063357911544256\\
496	0.00635458601007473\\
497	0.00637363697011413\\
498	0.00639294799741503\\
499	0.00641252316626832\\
500	0.00643236667033256\\
501	0.00645248283132593\\
502	0.00647287610852873\\
503	0.00649355110916894\\
504	0.00651451259976566\\
505	0.00653576551850926\\
506	0.00655731498875853\\
507	0.00657916633373493\\
508	0.0066013250924929\\
509	0.00662379703723859\\
510	0.00664658819206253\\
511	0.00666970485313681\\
512	0.00669315361040883\\
513	0.00671694137079573\\
514	0.00674107538284555\\
515	0.00676556326277949\\
516	0.00679041302176111\\
517	0.006815633094147\\
518	0.00684123236635596\\
519	0.0068672202058387\\
520	0.00689360648943386\\
521	0.00692040163014122\\
522	0.00694761660101961\\
523	0.0069752629545051\\
524	0.0070033528349217\\
525	0.00703189898129568\\
526	0.00706091471674855\\
527	0.00709041391968895\\
528	0.0071204109707071\\
529	0.00715092066742947\\
530	0.00718195809760305\\
531	0.00721353845722821\\
532	0.00724567679700885\\
533	0.00727838766877932\\
534	0.00731168469820507\\
535	0.00734558005552928\\
536	0.00738008435104086\\
537	0.00741520385143412\\
538	0.00745093704720049\\
539	0.00748723869970754\\
540	0.00752407422169693\\
541	0.0075615108104564\\
542	0.00759963420054334\\
543	0.00763863727699673\\
544	0.00767856978708636\\
545	0.00771949303797662\\
546	0.00776147010594045\\
547	0.00780456248016654\\
548	0.00784884021700096\\
549	0.00789439433353439\\
550	0.00794135701250253\\
551	0.00798960912808038\\
552	0.00803637094793776\\
553	0.00808097007159319\\
554	0.00812496792776143\\
555	0.00816909891588888\\
556	0.00821358388140258\\
557	0.00825844490010318\\
558	0.00830363112481854\\
559	0.00834907049081818\\
560	0.00839423278867568\\
561	0.00843838050845754\\
562	0.00848214519909988\\
563	0.00852603665066133\\
564	0.00857009983934468\\
565	0.00861431729034347\\
566	0.0086586558688049\\
567	0.00870308161489387\\
568	0.0087475675617029\\
569	0.00879162789679202\\
570	0.00883561106037741\\
571	0.00887981727907396\\
572	0.00892423949543059\\
573	0.00896885035521918\\
574	0.00901361849050742\\
575	0.00905850974103567\\
576	0.00910348714308695\\
577	0.00914851086878221\\
578	0.0091935381939862\\
579	0.00923852351279987\\
580	0.00928341841513161\\
581	0.00932817184769639\\
582	0.0093727303837578\\
583	0.00941703863316352\\
584	0.00946103983178964\\
585	0.00950467665821572\\
586	0.00954789233435466\\
587	0.00959063207273936\\
588	0.00963284492704104\\
589	0.0096744860587673\\
590	0.00971551928305088\\
591	0.00975591931512916\\
592	0.00979567191274874\\
593	0.00983476678232329\\
594	0.00987312984006299\\
595	0.00991038882774525\\
596	0.00994573390547615\\
597	0.0099771937715668\\
598	0.00999970795535495\\
599	0\\
600	0\\
};
\addplot [color=black!20!mycolor21,solid,forget plot]
  table[row sep=crcr]{%
1	0.00395964375163837\\
2	0.00395965334766334\\
3	0.00395966315493915\\
4	0.00395967317799734\\
5	0.00395968342146281\\
6	0.00395969389005562\\
7	0.00395970458859279\\
8	0.00395971552199028\\
9	0.00395972669526469\\
10	0.00395973811353531\\
11	0.00395974978202612\\
12	0.00395976170606758\\
13	0.00395977389109891\\
14	0.00395978634266988\\
15	0.0039597990664431\\
16	0.00395981206819604\\
17	0.00395982535382315\\
18	0.00395983892933814\\
19	0.0039598528008762\\
20	0.0039598669746961\\
21	0.00395988145718269\\
22	0.0039598962548491\\
23	0.00395991137433913\\
24	0.00395992682242971\\
25	0.00395994260603318\\
26	0.00395995873219995\\
27	0.00395997520812093\\
28	0.00395999204113003\\
29	0.00396000923870677\\
30	0.00396002680847904\\
31	0.00396004475822548\\
32	0.00396006309587852\\
33	0.00396008182952683\\
34	0.00396010096741822\\
35	0.0039601205179625\\
36	0.00396014048973427\\
37	0.00396016089147582\\
38	0.00396018173210012\\
39	0.00396020302069375\\
40	0.00396022476651985\\
41	0.00396024697902132\\
42	0.00396026966782382\\
43	0.00396029284273896\\
44	0.00396031651376742\\
45	0.00396034069110217\\
46	0.00396036538513182\\
47	0.00396039060644379\\
48	0.00396041636582772\\
49	0.00396044267427887\\
50	0.00396046954300149\\
51	0.00396049698341223\\
52	0.0039605250071438\\
53	0.00396055362604837\\
54	0.00396058285220117\\
55	0.00396061269790415\\
56	0.00396064317568966\\
57	0.00396067429832404\\
58	0.00396070607881142\\
59	0.00396073853039752\\
60	0.00396077166657339\\
61	0.00396080550107938\\
62	0.00396084004790885\\
63	0.0039608753213123\\
64	0.00396091133580114\\
65	0.00396094810615178\\
66	0.00396098564740967\\
67	0.00396102397489332\\
68	0.00396106310419847\\
69	0.00396110305120211\\
70	0.00396114383206675\\
71	0.00396118546324461\\
72	0.00396122796148171\\
73	0.00396127134382233\\
74	0.00396131562761313\\
75	0.00396136083050761\\
76	0.0039614069704703\\
77	0.00396145406578124\\
78	0.00396150213504033\\
79	0.00396155119717176\\
80	0.00396160127142844\\
81	0.0039616523773965\\
82	0.00396170453499971\\
83	0.00396175776450402\\
84	0.00396181208652208\\
85	0.00396186752201777\\
86	0.00396192409231075\\
87	0.00396198181908104\\
88	0.00396204072437357\\
89	0.00396210083060279\\
90	0.00396216216055722\\
91	0.00396222473740417\\
92	0.00396228858469422\\
93	0.00396235372636587\\
94	0.0039624201867502\\
95	0.00396248799057547\\
96	0.00396255716297167\\
97	0.00396262772947524\\
98	0.00396269971603358\\
99	0.00396277314900974\\
100	0.00396284805518691\\
101	0.00396292446177313\\
102	0.00396300239640572\\
103	0.00396308188715596\\
104	0.00396316296253354\\
105	0.00396324565149113\\
106	0.00396332998342891\\
107	0.00396341598819903\\
108	0.00396350369610999\\
109	0.00396359313793127\\
110	0.00396368434489753\\
111	0.00396377734871314\\
112	0.00396387218155648\\
113	0.00396396887608433\\
114	0.00396406746543607\\
115	0.00396416798323804\\
116	0.00396427046360772\\
117	0.00396437494115797\\
118	0.00396448145100114\\
119	0.00396459002875327\\
120	0.00396470071053813\\
121	0.00396481353299136\\
122	0.00396492853326446\\
123	0.00396504574902881\\
124	0.00396516521847957\\
125	0.00396528698033972\\
126	0.00396541107386386\\
127	0.00396553753884221\\
128	0.00396566641560436\\
129	0.00396579774502315\\
130	0.00396593156851853\\
131	0.00396606792806125\\
132	0.00396620686617672\\
133	0.00396634842594879\\
134	0.00396649265102352\\
135	0.00396663958561298\\
136	0.003966789274499\\
137	0.00396694176303706\\
138	0.00396709709716001\\
139	0.00396725532338214\\
140	0.00396741648880294\\
141	0.00396758064111113\\
142	0.0039677478285888\\
143	0.0039679181001154\\
144	0.00396809150517207\\
145	0.0039682680938459\\
146	0.00396844791683451\\
147	0.00396863102545051\\
148	0.00396881747162637\\
149	0.00396900730791938\\
150	0.00396920058751668\\
151	0.00396939736424077\\
152	0.00396959769255505\\
153	0.00396980162756976\\
154	0.00397000922504807\\
155	0.00397022054141285\\
156	0.00397043563375329\\
157	0.00397065455983234\\
158	0.00397087737809437\\
159	0.00397110414767336\\
160	0.00397133492840156\\
161	0.00397156978081874\\
162	0.00397180876618193\\
163	0.00397205194647591\\
164	0.00397229938442428\\
165	0.00397255114350131\\
166	0.00397280728794447\\
167	0.00397306788276805\\
168	0.00397333299377718\\
169	0.00397360268758339\\
170	0.0039738770316206\\
171	0.00397415609416259\\
172	0.00397443994434141\\
173	0.00397472865216697\\
174	0.0039750222885479\\
175	0.00397532092531391\\
176	0.00397562463523926\\
177	0.00397593349206792\\
178	0.00397624757054026\\
179	0.00397656694642131\\
180	0.00397689169653071\\
181	0.00397722189877456\\
182	0.0039775576321791\\
183	0.00397789897692628\\
184	0.00397824601439143\\
185	0.00397859882718305\\
186	0.00397895749918482\\
187	0.00397932211559987\\
188	0.00397969276299737\\
189	0.00398006952936187\\
190	0.00398045250414462\\
191	0.00398084177831814\\
192	0.00398123744443297\\
193	0.00398163959667756\\
194	0.00398204833094071\\
195	0.00398246374487709\\
196	0.00398288593797575\\
197	0.00398331501163142\\
198	0.00398375106921916\\
199	0.00398419421617194\\
200	0.00398464456006148\\
201	0.00398510221068217\\
202	0.00398556728013829\\
203	0.00398603988293433\\
204	0.00398652013606856\\
205	0.00398700815912995\\
206	0.00398750407439797\\
207	0.0039880080069457\\
208	0.00398852008474601\\
209	0.00398904043878078\\
210	0.00398956920315294\\
211	0.00399010651520159\\
212	0.00399065251561945\\
213	0.00399120734857345\\
214	0.00399177116182738\\
215	0.00399234410686703\\
216	0.00399292633902742\\
217	0.00399351801762209\\
218	0.00399411930607387\\
219	0.00399473037204754\\
220	0.00399535138758358\\
221	0.00399598252923296\\
222	0.00399662397819282\\
223	0.00399727592044269\\
224	0.00399793854688094\\
225	0.00399861205346116\\
226	0.00399929664132846\\
227	0.003999992516955\\
228	0.00400069989227484\\
229	0.00400141898481762\\
230	0.00400215001784093\\
231	0.00400289322046105\\
232	0.00400364882778198\\
233	0.00400441708102244\\
234	0.00400519822764072\\
235	0.00400599252145714\\
236	0.00400680022277436\\
237	0.00400762159849498\\
238	0.00400845692223696\\
239	0.00400930647444638\\
240	0.00401017054250808\\
241	0.00401104942085394\\
242	0.00401194341106922\\
243	0.00401285282199717\\
244	0.00401377796984214\\
245	0.00401471917827151\\
246	0.00401567677851693\\
247	0.00401665110947515\\
248	0.0040176425178092\\
249	0.00401865135804993\\
250	0.004019677992699\\
251	0.00402072279233356\\
252	0.00402178613571327\\
253	0.00402286840989043\\
254	0.00402397001032393\\
255	0.00402509134099774\\
256	0.0040262328145449\\
257	0.00402739485237837\\
258	0.00402857788482955\\
259	0.00402978235129683\\
260	0.00403100870040635\\
261	0.00403225739018976\\
262	0.00403352888828749\\
263	0.00403482367218943\\
264	0.00403614222949775\\
265	0.0040374850580941\\
266	0.00403885266642762\\
267	0.00404024557381834\\
268	0.00404166431077653\\
269	0.00404310941933892\\
270	0.00404458145342213\\
271	0.00404608097919428\\
272	0.00404760857546515\\
273	0.00404916483409596\\
274	0.00405075036042868\\
275	0.00405236577373619\\
276	0.00405401170769354\\
277	0.00405568881087088\\
278	0.00405739774724882\\
279	0.00405913919675669\\
280	0.00406091385583455\\
281	0.00406272243801931\\
282	0.0040645656745559\\
283	0.00406644431503428\\
284	0.004068359128053\\
285	0.00407031090191012\\
286	0.00407230044532286\\
287	0.00407432858817622\\
288	0.00407639618230251\\
289	0.00407850410229188\\
290	0.00408065324633533\\
291	0.00408284453710049\\
292	0.00408507892264047\\
293	0.00408735737733538\\
294	0.00408968090286549\\
295	0.00409205052921337\\
296	0.00409446731569204\\
297	0.00409693235199474\\
298	0.00409944675926425\\
299	0.00410201169118099\\
300	0.00410462833505344\\
301	0.00410729791278782\\
302	0.00411002168171443\\
303	0.00411280093540496\\
304	0.00411563700435423\\
305	0.00411853125649502\\
306	0.0041214850975437\\
307	0.00412449997128679\\
308	0.0041275773601546\\
309	0.00413071878522574\\
310	0.00413392580577175\\
311	0.00413720002038836\\
312	0.00414054306816125\\
313	0.00414395662987838\\
314	0.00414744242938025\\
315	0.00415100223514791\\
316	0.00415463786182732\\
317	0.00415835117185727\\
318	0.00416214407721573\\
319	0.00416601854130468\\
320	0.00416997658099756\\
321	0.00417402026887962\\
322	0.00417815173571903\\
323	0.00418237317321583\\
324	0.00418668683708922\\
325	0.00419109505058904\\
326	0.00419560020857598\\
327	0.00420020478245296\\
328	0.00420491132624466\\
329	0.00420972248165688\\
330	0.00421464098528164\\
331	0.0042196696769845\\
332	0.00422481150958617\\
333	0.00423006955981566\\
334	0.00423544704048069\\
335	0.00424094731580934\\
336	0.0042465739205694\\
337	0.00425233057956957\\
338	0.00425822122893274\\
339	0.00426425003844867\\
340	0.00427042143368057\\
341	0.004276740115593\\
342	0.00428321107475444\\
343	0.00428983960159714\\
344	0.0042966312552854\\
345	0.0043035917806431\\
346	0.00431072710551373\\
347	0.00431804333603102\\
348	0.00432554674944444\\
349	0.00433324378435994\\
350	0.00434114102969718\\
351	0.00434924521904846\\
352	0.0043575632011128\\
353	0.00436610191015336\\
354	0.00437486833446154\\
355	0.0043838694757048\\
356	0.00439311228965902\\
357	0.00440260364595343\\
358	0.00441235030060594\\
359	0.00442235884979655\\
360	0.00443263568766444\\
361	0.00444318697423391\\
362	0.0044540186225565\\
363	0.0044651363182996\\
364	0.00447654559072606\\
365	0.00448825196180528\\
366	0.00450026121066164\\
367	0.00451257980410656\\
368	0.00452521555989052\\
369	0.00453817862473681\\
370	0.00455148762462533\\
371	0.00456516460123408\\
372	0.00457923330368143\\
373	0.0045937192018997\\
374	0.00460864948914859\\
375	0.0046240527664869\\
376	0.00463995899883539\\
377	0.00465639960751397\\
378	0.00467340722466519\\
379	0.00469101530764693\\
380	0.00470925756498926\\
381	0.00472816713042042\\
382	0.00474777540177328\\
383	0.00476811043563431\\
384	0.00478919475461603\\
385	0.00481104237623984\\
386	0.00483365479123261\\
387	0.00485701536763692\\
388	0.00488108356522784\\
389	0.00490578651630885\\
390	0.00493100584675148\\
391	0.0049561050571859\\
392	0.00498102513730994\\
393	0.00500571703987493\\
394	0.00503012688162011\\
395	0.00505419574628497\\
396	0.00507785958478666\\
397	0.00510104926199273\\
398	0.0051236908171125\\
399	0.00514570602802744\\
400	0.00516701340075003\\
401	0.00518752974652354\\
402	0.00520717254565949\\
403	0.00522586334149184\\
404	0.00524353259235539\\
405	0.00526012649631629\\
406	0.00527561655099673\\
407	0.00529001263926006\\
408	0.00530343563728342\\
409	0.00531653331693652\\
410	0.00532929093024752\\
411	0.00534169586148326\\
412	0.00535373820700891\\
413	0.00536541142763497\\
414	0.00537671304665619\\
415	0.00538764540158245\\
416	0.00539821639072226\\
417	0.00540844024765905\\
418	0.0054183381926166\\
419	0.00542793885684519\\
420	0.00543727842000505\\
421	0.00544640026552884\\
422	0.00545535349965632\\
423	0.00546419035229756\\
424	0.0054729565826893\\
425	0.00548166983600437\\
426	0.0054903379177783\\
427	0.00549896976093038\\
428	0.00550757536486095\\
429	0.00551616568808048\\
430	0.00552475248546753\\
431	0.00553334808417513\\
432	0.00554196509939006\\
433	0.00555061608625226\\
434	0.00555931313637169\\
435	0.00556806743863489\\
436	0.00557688883323977\\
437	0.00558578541253118\\
438	0.00559476327897139\\
439	0.00560382735036957\\
440	0.00561298229679523\\
441	0.00562223275505113\\
442	0.00563158327885558\\
443	0.00564103828900022\\
444	0.00565060202589308\\
445	0.00566027850746577\\
446	0.00567007149644886\\
447	0.00567998448170495\\
448	0.00569002067801832\\
449	0.00570018304808882\\
450	0.00571047434862302\\
451	0.00572089719862039\\
452	0.00573145415290164\\
453	0.00574214772961181\\
454	0.00575298040693971\\
455	0.0057639546216638\\
456	0.00577507276974429\\
457	0.00578633720907419\\
458	0.00579775026435995\\
459	0.00580931423391469\\
460	0.00582103139791401\\
461	0.00583290402740387\\
462	0.00584493439309434\\
463	0.0058571247727867\\
464	0.00586947745628239\\
465	0.00588199474784275\\
466	0.005894678967924\\
467	0.00590753245516925\\
468	0.00592055756859933\\
469	0.00593375668992994\\
470	0.0059471322259331\\
471	0.00596068661075727\\
472	0.00597442230812716\\
473	0.00598834181336401\\
474	0.00600244765520103\\
475	0.00601674239741815\\
476	0.00603122864038028\\
477	0.00604590902259925\\
478	0.00606078622237516\\
479	0.00607586295951612\\
480	0.00609114199713636\\
481	0.0061066261435377\\
482	0.00612231825418213\\
483	0.00613822123376957\\
484	0.00615433803843859\\
485	0.00617067167811424\\
486	0.00618722521902902\\
487	0.00620400178644644\\
488	0.00622100456761418\\
489	0.00623823681497238\\
490	0.00625570184964363\\
491	0.00627340306523325\\
492	0.00629134393197285\\
493	0.00630952800124148\\
494	0.00632795891050328\\
495	0.00634664038870283\\
496	0.00636557626216234\\
497	0.0063847704610277\\
498	0.00640422702631292\\
499	0.00642395011759481\\
500	0.00644394402141205\\
501	0.00646421316042427\\
502	0.00648476210338836\\
503	0.00650559557600805\\
504	0.00652671847271206\\
505	0.00654813586941164\\
506	0.00656985303728164\\
507	0.00659187545759923\\
508	0.0066142088376587\\
509	0.00663685912776054\\
510	0.00665983253924229\\
511	0.0066831355634807\\
512	0.0067067749917419\\
513	0.00673075793568732\\
514	0.00675509184825398\\
515	0.0067797845445121\\
516	0.00680484422195338\\
517	0.00683027947947309\\
518	0.00685609933406564\\
519	0.00688231323394464\\
520	0.00690893106640585\\
521	0.00693596315825738\\
522	0.0069634202660161\\
523	0.0069913135522828\\
524	0.0070196545437194\\
525	0.00704845506480828\\
526	0.00707772714003761\\
527	0.0071074828552847\\
528	0.00713773416616653\\
529	0.00716849263744683\\
530	0.00719976908686505\\
531	0.00723157315713454\\
532	0.00726391279674431\\
533	0.00729679425204214\\
534	0.00733021895576369\\
535	0.00736418027017706\\
536	0.00739861050727177\\
537	0.00743350659545982\\
538	0.00746893739763071\\
539	0.00750502999446856\\
540	0.00754192715135258\\
541	0.00757969109057501\\
542	0.00761839596233787\\
543	0.00765809882046372\\
544	0.00769885502590327\\
545	0.0077407273894827\\
546	0.00778378744306659\\
547	0.00782813502988159\\
548	0.00787391113933587\\
549	0.00792022278387469\\
550	0.00796483486832117\\
551	0.00800724731079562\\
552	0.0080497803406812\\
553	0.00809262919199778\\
554	0.00813591216228803\\
555	0.00817962687777911\\
556	0.00822371846461619\\
557	0.00826811691466122\\
558	0.00831262787810168\\
559	0.00835614799688869\\
560	0.00839908507898075\\
561	0.00844218055789354\\
562	0.00848549450963725\\
563	0.008529013494506\\
564	0.00857270633871265\\
565	0.00861654004187527\\
566	0.00866048475963239\\
567	0.00870443353843061\\
568	0.00874795159195782\\
569	0.00879165244561201\\
570	0.00883561303603307\\
571	0.0088798174691251\\
572	0.00892423953733394\\
573	0.0089688503706678\\
574	0.00901361849741638\\
575	0.00905850974448163\\
576	0.00910348714494772\\
577	0.00914851086982138\\
578	0.00919353819457018\\
579	0.00923852351312367\\
580	0.00928341841530305\\
581	0.00932817184777885\\
582	0.00937273038379159\\
583	0.00941703863317428\\
584	0.00946103983179199\\
585	0.00950467665821593\\
586	0.00954789233435466\\
587	0.00959063207273936\\
588	0.00963284492704104\\
589	0.0096744860587673\\
590	0.00971551928305088\\
591	0.00975591931512916\\
592	0.00979567191274874\\
593	0.00983476678232329\\
594	0.00987312984006299\\
595	0.00991038882774525\\
596	0.00994573390547615\\
597	0.0099771937715668\\
598	0.00999970795535495\\
599	0\\
600	0\\
};
\addplot [color=black!50!mycolor20,solid,forget plot]
  table[row sep=crcr]{%
1	0.00395969554893294\\
2	0.00395970653642755\\
3	0.00395971777385418\\
4	0.00395972926681449\\
5	0.00395974102103281\\
6	0.00395975304235881\\
7	0.00395976533677007\\
8	0.00395977791037461\\
9	0.00395979076941396\\
10	0.00395980392026571\\
11	0.00395981736944642\\
12	0.00395983112361459\\
13	0.00395984518957348\\
14	0.00395985957427428\\
15	0.00395987428481911\\
16	0.00395988932846415\\
17	0.00395990471262283\\
18	0.00395992044486914\\
19	0.00395993653294086\\
20	0.00395995298474307\\
21	0.00395996980835143\\
22	0.00395998701201582\\
23	0.00396000460416384\\
24	0.0039600225934045\\
25	0.00396004098853185\\
26	0.00396005979852891\\
27	0.00396007903257125\\
28	0.00396009870003126\\
29	0.00396011881048184\\
30	0.00396013937370056\\
31	0.00396016039967391\\
32	0.00396018189860124\\
33	0.00396020388089935\\
34	0.00396022635720663\\
35	0.00396024933838761\\
36	0.00396027283553743\\
37	0.00396029685998643\\
38	0.00396032142330488\\
39	0.00396034653730766\\
40	0.00396037221405919\\
41	0.00396039846587827\\
42	0.00396042530534317\\
43	0.00396045274529659\\
44	0.00396048079885099\\
45	0.0039605094793938\\
46	0.0039605388005927\\
47	0.00396056877640119\\
48	0.00396059942106404\\
49	0.0039606307491229\\
50	0.00396066277542204\\
51	0.00396069551511427\\
52	0.00396072898366659\\
53	0.00396076319686642\\
54	0.00396079817082748\\
55	0.0039608339219962\\
56	0.00396087046715776\\
57	0.00396090782344264\\
58	0.00396094600833301\\
59	0.00396098503966929\\
60	0.00396102493565688\\
61	0.00396106571487285\\
62	0.0039611073962728\\
63	0.00396114999919789\\
64	0.00396119354338179\\
65	0.00396123804895797\\
66	0.00396128353646685\\
67	0.00396133002686319\\
68	0.00396137754152353\\
69	0.00396142610225383\\
70	0.00396147573129697\\
71	0.0039615264513406\\
72	0.00396157828552509\\
73	0.00396163125745124\\
74	0.00396168539118849\\
75	0.00396174071128306\\
76	0.00396179724276621\\
77	0.00396185501116245\\
78	0.00396191404249807\\
79	0.00396197436330974\\
80	0.00396203600065297\\
81	0.00396209898211091\\
82	0.00396216333580316\\
83	0.00396222909039467\\
84	0.00396229627510462\\
85	0.00396236491971563\\
86	0.00396243505458278\\
87	0.00396250671064294\\
88	0.00396257991942401\\
89	0.00396265471305432\\
90	0.00396273112427213\\
91	0.00396280918643512\\
92	0.00396288893352997\\
93	0.00396297040018213\\
94	0.00396305362166542\\
95	0.00396313863391186\\
96	0.00396322547352158\\
97	0.0039633141777726\\
98	0.00396340478463087\\
99	0.00396349733276019\\
100	0.00396359186153224\\
101	0.00396368841103668\\
102	0.00396378702209112\\
103	0.00396388773625135\\
104	0.00396399059582142\\
105	0.00396409564386375\\
106	0.00396420292420921\\
107	0.00396431248146741\\
108	0.00396442436103667\\
109	0.00396453860911422\\
110	0.00396465527270628\\
111	0.00396477439963812\\
112	0.00396489603856402\\
113	0.00396502023897732\\
114	0.00396514705122035\\
115	0.00396527652649425\\
116	0.00396540871686871\\
117	0.0039655436752918\\
118	0.00396568145559948\\
119	0.0039658221125252\\
120	0.0039659657017092\\
121	0.00396611227970794\\
122	0.00396626190400309\\
123	0.00396641463301061\\
124	0.00396657052608967\\
125	0.00396672964355128\\
126	0.00396689204666676\\
127	0.00396705779767616\\
128	0.00396722695979628\\
129	0.00396739959722869\\
130	0.00396757577516734\\
131	0.00396775555980596\\
132	0.00396793901834539\\
133	0.00396812621900038\\
134	0.00396831723100634\\
135	0.00396851212462558\\
136	0.00396871097115349\\
137	0.0039689138429242\\
138	0.00396912081331608\\
139	0.00396933195675666\\
140	0.00396954734872749\\
141	0.00396976706576844\\
142	0.00396999118548161\\
143	0.00397021978653505\\
144	0.00397045294866576\\
145	0.00397069075268268\\
146	0.00397093328046883\\
147	0.00397118061498339\\
148	0.00397143284026303\\
149	0.00397169004142311\\
150	0.00397195230465819\\
151	0.00397221971724211\\
152	0.00397249236752781\\
153	0.00397277034494633\\
154	0.00397305374000582\\
155	0.00397334264428955\\
156	0.00397363715045398\\
157	0.003973937352226\\
158	0.00397424334439992\\
159	0.00397455522283397\\
160	0.00397487308444631\\
161	0.00397519702721084\\
162	0.00397552715015242\\
163	0.00397586355334175\\
164	0.00397620633789004\\
165	0.00397655560594325\\
166	0.00397691146067616\\
167	0.00397727400628604\\
168	0.00397764334798637\\
169	0.00397801959200017\\
170	0.00397840284555344\\
171	0.00397879321686846\\
172	0.00397919081515721\\
173	0.00397959575061479\\
174	0.00398000813441317\\
175	0.00398042807869503\\
176	0.00398085569656832\\
177	0.00398129110210079\\
178	0.00398173441031555\\
179	0.00398218573718711\\
180	0.00398264519963834\\
181	0.00398311291553826\\
182	0.00398358900370114\\
183	0.00398407358388667\\
184	0.00398456677680181\\
185	0.00398506870410398\\
186	0.0039855794884062\\
187	0.00398609925328423\\
188	0.00398662812328593\\
189	0.0039871662239429\\
190	0.00398771368178504\\
191	0.00398827062435772\\
192	0.00398883718024241\\
193	0.00398941347908048\\
194	0.00398999965160087\\
195	0.00399059582965163\\
196	0.00399120214623592\\
197	0.00399181873555242\\
198	0.00399244573304076\\
199	0.0039930832754322\\
200	0.00399373150080589\\
201	0.00399439054865097\\
202	0.00399506055993499\\
203	0.00399574167717908\\
204	0.00399643404453986\\
205	0.00399713780789878\\
206	0.00399785311495922\\
207	0.0039985801153516\\
208	0.00399931896074677\\
209	0.00400006980497827\\
210	0.00400083280417361\\
211	0.00400160811689487\\
212	0.00400239590428916\\
213	0.0040031963302489\\
214	0.00400400956158237\\
215	0.00400483576819493\\
216	0.00400567512328072\\
217	0.00400652780352546\\
218	0.00400739398932024\\
219	0.00400827386498645\\
220	0.00400916761901202\\
221	0.00401007544429893\\
222	0.00401099753842209\\
223	0.00401193410389937\\
224	0.00401288534847289\\
225	0.00401385148540112\\
226	0.00401483273376201\\
227	0.00401582931876653\\
228	0.00401684147208253\\
229	0.00401786943216855\\
230	0.00401891344461715\\
231	0.00401997376250759\\
232	0.00402105064676699\\
233	0.00402214436654\\
234	0.00402325519956623\\
235	0.00402438343256498\\
236	0.00402552936162713\\
237	0.00402669329261319\\
238	0.00402787554155787\\
239	0.00402907643508009\\
240	0.00403029631079869\\
241	0.00403153551775326\\
242	0.00403279441683012\\
243	0.0040340733811933\\
244	0.00403537279672018\\
245	0.00403669306244212\\
246	0.00403803459098952\\
247	0.0040393978090413\\
248	0.00404078315777801\\
249	0.004042191093338\\
250	0.00404362208727489\\
251	0.00404507662701405\\
252	0.00404655521630505\\
253	0.00404805837566494\\
254	0.00404958664280591\\
255	0.00405114057303865\\
256	0.0040527207396397\\
257	0.00405432773416886\\
258	0.00405596216671877\\
259	0.00405762466607665\\
260	0.00405931587977623\\
261	0.00406103647402001\\
262	0.00406278713346629\\
263	0.0040645685609234\\
264	0.00406638147710659\\
265	0.00406822662055874\\
266	0.00407010474613871\\
267	0.00407201662523554\\
268	0.00407396304598531\\
269	0.00407594481349066\\
270	0.00407796275004411\\
271	0.00408001769535605\\
272	0.00408211050678803\\
273	0.00408424205959241\\
274	0.0040864132471601\\
275	0.00408862498127698\\
276	0.00409087819239134\\
277	0.00409317382989399\\
278	0.00409551286241395\\
279	0.00409789627813235\\
280	0.00410032508511883\\
281	0.00410280031169531\\
282	0.00410532300683364\\
283	0.00410789424059481\\
284	0.00411051510462028\\
285	0.00411318671268872\\
286	0.00411591020135412\\
287	0.00411868673068691\\
288	0.00412151748514391\\
289	0.00412440367460032\\
290	0.00412734653558421\\
291	0.00413034733276421\\
292	0.00413340736075139\\
293	0.00413652794628979\\
294	0.00413971045092405\\
295	0.00414295627424901\\
296	0.00414626685786274\\
297	0.00414964369016378\\
298	0.00415308831215686\\
299	0.0041566023244764\\
300	0.00416018739593252\\
301	0.00416384527394439\\
302	0.00416757779579823\\
303	0.0041713869004178\\
304	0.00417527464252895\\
305	0.00417924320775439\\
306	0.00418329492776416\\
307	0.0041874322940093\\
308	0.00419165796832062\\
309	0.00419597479204264\\
310	0.00420038576183341\\
311	0.00420489398253474\\
312	0.00420950266927981\\
313	0.00421421514900673\\
314	0.00421903486116254\\
315	0.00422396535825689\\
316	0.00422901030784202\\
317	0.00423417349193308\\
318	0.00423945880541624\\
319	0.00424487025324035\\
320	0.00425041194616065\\
321	0.00425608809477158\\
322	0.00426190300153614\\
323	0.00426786105048812\\
324	0.00427396669425726\\
325	0.00428022443804854\\
326	0.00428663882023624\\
327	0.00429321438951463\\
328	0.00429995568010478\\
329	0.00430686719176091\\
330	0.00431395334176155\\
331	0.00432121842565339\\
332	0.00432866657480124\\
333	0.00433630171164908\\
334	0.00434412750283828\\
335	0.00435214730700318\\
336	0.00436036414236211\\
337	0.00436878069589284\\
338	0.00437739934436666\\
339	0.0043862222235715\\
340	0.00439525137003741\\
341	0.00440448896883744\\
342	0.00441393775264034\\
343	0.00442360160977393\\
344	0.0044334879031296\\
345	0.00444360919024763\\
346	0.00445397903540017\\
347	0.00446461208546004\\
348	0.00447552414697224\\
349	0.0044867322621642\\
350	0.00449825478041501\\
351	0.004510111427792\\
352	0.00452232351700233\\
353	0.00453491390347039\\
354	0.0045479069566887\\
355	0.00456132853268967\\
356	0.00457520586966399\\
357	0.00458956718628757\\
358	0.00460444140180532\\
359	0.00461985793956428\\
360	0.00463584612954487\\
361	0.00465243435886602\\
362	0.00466964888774392\\
363	0.0046875122230492\\
364	0.00470604090849864\\
365	0.0047252425474306\\
366	0.00474511181771898\\
367	0.00476562516388812\\
368	0.00478673374934612\\
369	0.00480835408631128\\
370	0.00483005801812664\\
371	0.00485172198632208\\
372	0.00487331742568788\\
373	0.0048948135360401\\
374	0.00491617687233518\\
375	0.00493737172833845\\
376	0.00495835663584292\\
377	0.00497908363036898\\
378	0.00499950049407906\\
379	0.00501955077361669\\
380	0.00503917396544525\\
381	0.00505830593989146\\
382	0.00507687969917407\\
383	0.00509482660604844\\
384	0.00511207823753275\\
385	0.0051285690954679\\
386	0.0051442404718571\\
387	0.00515904586356221\\
388	0.00517295842467018\\
389	0.00518598113225567\\
390	0.00519816064651042\\
391	0.00521007320582187\\
392	0.00522171956990806\\
393	0.00523308554388016\\
394	0.00524415826380914\\
395	0.00525492661785436\\
396	0.00526538172482074\\
397	0.00527551746887198\\
398	0.00528533108466823\\
399	0.00529482378052394\\
400	0.00530400137743908\\
401	0.00531287492783769\\
402	0.00532146125856364\\
403	0.00532978335668494\\
404	0.00533787047707073\\
405	0.00534575779758709\\
406	0.00535348537323425\\
407	0.00536109604768382\\
408	0.00536862988912369\\
409	0.00537610170564491\\
410	0.00538351681248603\\
411	0.00539088150686385\\
412	0.00539820306368182\\
413	0.0054054897055481\\
414	0.0054127506373611\\
415	0.00541999584889401\\
416	0.00542723601052474\\
417	0.00543448186342449\\
418	0.0054417442293014\\
419	0.00544903371352326\\
420	0.00545636033618458\\
421	0.00546373314084725\\
422	0.00547115982365393\\
423	0.00547864645414766\\
424	0.00548619757398796\\
425	0.00549381721138449\\
426	0.00550150942358831\\
427	0.00550927825615538\\
428	0.00551712769995871\\
429	0.00552506164751555\\
430	0.00553308385076977\\
431	0.00554119788289693\\
432	0.00554940710707964\\
433	0.00555771465570698\\
434	0.00556612342353803\\
435	0.00557463607806808\\
436	0.00558325508963432\\
437	0.00559198278159137\\
438	0.00560082139614385\\
439	0.005609773136161\\
440	0.00561884017156667\\
441	0.00562802463637245\\
442	0.00563732862723041\\
443	0.00564675420368448\\
444	0.00565630339023131\\
445	0.00566597818020127\\
446	0.00567578054131269\\
447	0.00568571242255137\\
448	0.00569577576180889\\
449	0.00570597249349151\\
450	0.00571630455513656\\
451	0.00572677389202858\\
452	0.00573738245935717\\
453	0.00574813222342954\\
454	0.00575902516311368\\
455	0.00577006327147047\\
456	0.0057812485575194\\
457	0.00579258304807187\\
458	0.00580406878955898\\
459	0.00581570784978016\\
460	0.00582750231950734\\
461	0.00583945431389915\\
462	0.00585156597371133\\
463	0.00586383946633356\\
464	0.00587627698673253\\
465	0.00588888075838796\\
466	0.00590165303425179\\
467	0.00591459609772419\\
468	0.00592771226364155\\
469	0.00594100387927433\\
470	0.00595447332533495\\
471	0.00596812301699974\\
472	0.00598195540495295\\
473	0.00599597297646308\\
474	0.00601017825650559\\
475	0.0060245738089458\\
476	0.00603916223779494\\
477	0.00605394618854894\\
478	0.00606892834961896\\
479	0.00608411145386366\\
480	0.00609949828023481\\
481	0.00611509165554907\\
482	0.00613089445640049\\
483	0.00614690961122951\\
484	0.00616314010256575\\
485	0.00617958896946317\\
486	0.00619625931014782\\
487	0.00621315428489966\\
488	0.00623027711919161\\
489	0.00624763110711127\\
490	0.0062652196150922\\
491	0.00628304608598458\\
492	0.00630111404349607\\
493	0.0063194270970369\\
494	0.0063379889470043\\
495	0.00635680339054377\\
496	0.00637587432782535\\
497	0.00639520576887495\\
498	0.00641480184100072\\
499	0.00643466679685369\\
500	0.00645480502316044\\
501	0.00647522105016233\\
502	0.00649591956178959\\
503	0.00651690540659079\\
504	0.00653818360942569\\
505	0.00655975938391251\\
506	0.00658163814559771\\
507	0.00660382552578562\\
508	0.00662632738592403\\
509	0.00664914983238842\\
510	0.00667229923143845\\
511	0.00669578222402991\\
512	0.00671960574005042\\
513	0.0067437770114002\\
514	0.00676830358315132\\
515	0.00679319332178079\\
516	0.00681845441917239\\
517	0.00684409539070084\\
518	0.00687012506523294\\
519	0.006896552564277\\
520	0.00692338726675418\\
521	0.00695063875491424\\
522	0.00697831673572706\\
523	0.00700643093058926\\
524	0.00703499092432012\\
525	0.00706400596220947\\
526	0.00709348467960366\\
527	0.00712343474003986\\
528	0.0071538623899978\\
529	0.00718477193429298\\
530	0.00721616568394696\\
531	0.00724804108720154\\
532	0.00728038731950989\\
533	0.00731312613157621\\
534	0.00734626297854407\\
535	0.00737986506719777\\
536	0.00741408122042639\\
537	0.00744902119843883\\
538	0.00748474690796401\\
539	0.00752132339765042\\
540	0.00755880916842859\\
541	0.0075972666895293\\
542	0.00763676226561044\\
543	0.00767735965754053\\
544	0.00771913114226334\\
545	0.0077621788010122\\
546	0.00780664666479424\\
547	0.00785119985749986\\
548	0.00789382387062527\\
549	0.00793483981570633\\
550	0.00797601076828931\\
551	0.00801762780379761\\
552	0.00805973722232243\\
553	0.00810231501629268\\
554	0.00814531085129842\\
555	0.00818865547804408\\
556	0.00823227074941743\\
557	0.00827530602472503\\
558	0.00831741047380532\\
559	0.00835967933656371\\
560	0.00840220305870095\\
561	0.00844497776753068\\
562	0.00848797514844882\\
563	0.00853116460971669\\
564	0.00857451574226307\\
565	0.00861800248723491\\
566	0.00866143203040683\\
567	0.00870455612303824\\
568	0.00874796048963319\\
569	0.00879165271721747\\
570	0.00883561306273407\\
571	0.00887981747531298\\
572	0.00892423953968708\\
573	0.00896885037174267\\
574	0.00901361849796072\\
575	0.00905850974477759\\
576	0.00910348714511327\\
577	0.0091485108699142\\
578	0.00919353819462128\\
579	0.0092385235131504\\
580	0.0092834184153157\\
581	0.00932817184778395\\
582	0.00937273038379318\\
583	0.00941703863317463\\
584	0.00946103983179202\\
585	0.00950467665821593\\
586	0.00954789233435466\\
587	0.00959063207273937\\
588	0.00963284492704105\\
589	0.0096744860587673\\
590	0.00971551928305088\\
591	0.00975591931512916\\
592	0.00979567191274874\\
593	0.00983476678232329\\
594	0.00987312984006299\\
595	0.00991038882774525\\
596	0.00994573390547615\\
597	0.0099771937715668\\
598	0.00999970795535495\\
599	0\\
600	0\\
};
\addplot [color=black!60!mycolor21,solid,forget plot]
  table[row sep=crcr]{%
1	0.00395973365081125\\
2	0.00395974572637439\\
3	0.00395975808349903\\
4	0.00395977072871211\\
5	0.00395978366869045\\
6	0.00395979691026415\\
7	0.00395981046042001\\
8	0.00395982432630529\\
9	0.00395983851523096\\
10	0.00395985303467575\\
11	0.00395986789228969\\
12	0.00395988309589811\\
13	0.00395989865350548\\
14	0.00395991457329953\\
15	0.00395993086365537\\
16	0.00395994753313958\\
17	0.00395996459051466\\
18	0.00395998204474336\\
19	0.00395999990499313\\
20	0.00396001818064074\\
21	0.00396003688127698\\
22	0.00396005601671148\\
23	0.00396007559697752\\
24	0.00396009563233706\\
25	0.00396011613328586\\
26	0.00396013711055862\\
27	0.00396015857513445\\
28	0.00396018053824212\\
29	0.00396020301136581\\
30	0.00396022600625061\\
31	0.00396024953490839\\
32	0.00396027360962373\\
33	0.00396029824295994\\
34	0.00396032344776523\\
35	0.00396034923717902\\
36	0.00396037562463832\\
37	0.00396040262388439\\
38	0.00396043024896929\\
39	0.00396045851426296\\
40	0.00396048743445999\\
41	0.00396051702458686\\
42	0.00396054730000915\\
43	0.00396057827643913\\
44	0.00396060996994314\\
45	0.00396064239694945\\
46	0.0039606755742562\\
47	0.00396070951903935\\
48	0.00396074424886096\\
49	0.00396077978167765\\
50	0.00396081613584909\\
51	0.00396085333014676\\
52	0.00396089138376294\\
53	0.00396093031631961\\
54	0.00396097014787804\\
55	0.00396101089894803\\
56	0.0039610525904976\\
57	0.0039610952439629\\
58	0.00396113888125826\\
59	0.00396118352478639\\
60	0.00396122919744883\\
61	0.00396127592265659\\
62	0.00396132372434114\\
63	0.00396137262696529\\
64	0.00396142265553464\\
65	0.00396147383560898\\
66	0.00396152619331406\\
67	0.0039615797553535\\
68	0.00396163454902111\\
69	0.00396169060221312\\
70	0.0039617479434409\\
71	0.00396180660184387\\
72	0.00396186660720257\\
73	0.00396192798995211\\
74	0.00396199078119566\\
75	0.00396205501271844\\
76	0.00396212071700173\\
77	0.00396218792723732\\
78	0.00396225667734212\\
79	0.00396232700197303\\
80	0.00396239893654209\\
81	0.00396247251723202\\
82	0.00396254778101174\\
83	0.00396262476565246\\
84	0.00396270350974392\\
85	0.0039627840527109\\
86	0.00396286643482994\\
87	0.00396295069724662\\
88	0.00396303688199272\\
89	0.00396312503200407\\
90	0.00396321519113836\\
91	0.00396330740419339\\
92	0.00396340171692572\\
93	0.00396349817606933\\
94	0.00396359682935493\\
95	0.00396369772552925\\
96	0.00396380091437469\\
97	0.00396390644672952\\
98	0.003964014374508\\
99	0.00396412475072116\\
100	0.00396423762949755\\
101	0.00396435306610452\\
102	0.00396447111696974\\
103	0.00396459183970292\\
104	0.00396471529311793\\
105	0.00396484153725517\\
106	0.0039649706334043\\
107	0.00396510264412705\\
108	0.00396523763328059\\
109	0.00396537566604099\\
110	0.00396551680892709\\
111	0.00396566112982445\\
112	0.00396580869800976\\
113	0.0039659595841755\\
114	0.00396611386045458\\
115	0.00396627160044565\\
116	0.00396643287923846\\
117	0.00396659777343926\\
118	0.00396676636119673\\
119	0.00396693872222802\\
120	0.00396711493784497\\
121	0.00396729509098044\\
122	0.00396747926621499\\
123	0.00396766754980375\\
124	0.00396786002970323\\
125	0.00396805679559842\\
126	0.0039682579389301\\
127	0.00396846355292208\\
128	0.00396867373260866\\
129	0.00396888857486212\\
130	0.0039691081784203\\
131	0.00396933264391418\\
132	0.00396956207389549\\
133	0.00396979657286437\\
134	0.00397003624729687\\
135	0.00397028120567255\\
136	0.00397053155850191\\
137	0.0039707874183537\\
138	0.0039710488998821\\
139	0.00397131611985376\\
140	0.00397158919717467\\
141	0.0039718682529167\\
142	0.00397215341034385\\
143	0.00397244479493844\\
144	0.0039727425344266\\
145	0.00397304675880361\\
146	0.00397335760035877\\
147	0.00397367519369977\\
148	0.00397399967577657\\
149	0.00397433118590468\\
150	0.00397466986578781\\
151	0.00397501585954004\\
152	0.00397536931370701\\
153	0.00397573037728663\\
154	0.00397609920174876\\
155	0.00397647594105407\\
156	0.00397686075167213\\
157	0.00397725379259818\\
158	0.00397765522536923\\
159	0.00397806521407888\\
160	0.00397848392539086\\
161	0.00397891152855152\\
162	0.00397934819540101\\
163	0.00397979410038307\\
164	0.00398024942055329\\
165	0.00398071433558605\\
166	0.00398118902777987\\
167	0.00398167368206101\\
168	0.0039821684859856\\
169	0.00398267362973997\\
170	0.00398318930613895\\
171	0.00398371571062265\\
172	0.00398425304125103\\
173	0.00398480149869675\\
174	0.00398536128623569\\
175	0.00398593260973566\\
176	0.00398651567764272\\
177	0.00398711070096562\\
178	0.00398771789325776\\
179	0.003988337470597\\
180	0.00398896965156327\\
181	0.00398961465721376\\
182	0.00399027271105594\\
183	0.00399094403901817\\
184	0.00399162886941805\\
185	0.00399232743292847\\
186	0.00399303996254159\\
187	0.00399376669353023\\
188	0.00399450786340751\\
189	0.00399526371188408\\
190	0.00399603448082356\\
191	0.00399682041419608\\
192	0.00399762175802983\\
193	0.0039984387603612\\
194	0.00399927167118327\\
195	0.00400012074239307\\
196	0.00400098622773767\\
197	0.00400186838275947\\
198	0.00400276746474071\\
199	0.00400368373264778\\
200	0.00400461744707524\\
201	0.00400556887019027\\
202	0.00400653826567762\\
203	0.00400752589868547\\
204	0.00400853203577279\\
205	0.00400955694485846\\
206	0.00401060089517248\\
207	0.00401166415721008\\
208	0.00401274700268913\\
209	0.00401384970451091\\
210	0.0040149725367256\\
211	0.00401611577450251\\
212	0.00401727969410559\\
213	0.00401846457287518\\
214	0.00401967068921618\\
215	0.00402089832259371\\
216	0.00402214775353635\\
217	0.0040234192636479\\
218	0.00402471313562826\\
219	0.00402602965330369\\
220	0.00402736910166737\\
221	0.00402873176693055\\
222	0.00403011793658471\\
223	0.00403152789947538\\
224	0.00403296194588792\\
225	0.00403442036764567\\
226	0.00403590345822042\\
227	0.00403741151285638\\
228	0.0040389448287067\\
229	0.00404050370498376\\
230	0.00404208844312312\\
231	0.00404369934696128\\
232	0.00404533672292807\\
233	0.00404700088025384\\
234	0.00404869213119245\\
235	0.00405041079126148\\
236	0.00405215717950076\\
237	0.00405393161875217\\
238	0.0040557344359636\\
239	0.00405756596252151\\
240	0.00405942653461807\\
241	0.00406131649366064\\
242	0.00406323618673343\\
243	0.00406518596712463\\
244	0.00406716619493544\\
245	0.0040691772377917\\
246	0.00407121947168417\\
247	0.00407329328196941\\
248	0.00407539906457014\\
249	0.00407753722742205\\
250	0.00407970819222244\\
251	0.00408191239654666\\
252	0.00408415029640648\\
253	0.00408642236933493\\
254	0.00408872911808953\\
255	0.00409107107506929\\
256	0.00409344880753876\\
257	0.0040958629237395\\
258	0.00409831407993879\\
259	0.00410080298840983\\
260	0.00410333042624129\\
261	0.00410589724472036\\
262	0.00410850437879744\\
263	0.00411115285581545\\
264	0.0041138438024045\\
265	0.00411657844896079\\
266	0.00411935813113533\\
267	0.00412218423674385\\
268	0.00412505820739792\\
269	0.00412798154011819\\
270	0.00413095578891538\\
271	0.00413398256632245\\
272	0.00413706354485806\\
273	0.00414020045839793\\
274	0.00414339510342495\\
275	0.00414664934012569\\
276	0.00414996509329252\\
277	0.00415334435298507\\
278	0.00415678917489552\\
279	0.0041603016803534\\
280	0.00416388405589335\\
281	0.00416753855229801\\
282	0.00417126748301288\\
283	0.0041750732218148\\
284	0.00417895819959755\\
285	0.00418292490011852\\
286	0.00418697585452994\\
287	0.00419111363449593\\
288	0.00419534084367495\\
289	0.0041996601073263\\
290	0.00420407405978126\\
291	0.00420858532950783\\
292	0.00421319652149639\\
293	0.00421791019670725\\
294	0.00422272884835977\\
295	0.00422765487491709\\
296	0.00423269054974617\\
297	0.0042378379876333\\
298	0.00424309910864331\\
299	0.00424847560029406\\
300	0.00425396887987042\\
301	0.0042595800606655\\
302	0.00426530993112059\\
303	0.00427115894010199\\
304	0.00427712718852631\\
305	0.00428321446444065\\
306	0.00428942032659151\\
307	0.00429574425911853\\
308	0.00430218592757397\\
309	0.00430874557718916\\
310	0.00431542576113523\\
311	0.00432223194998269\\
312	0.0043291700187598\\
313	0.0043362462806597\\
314	0.00434346752159857\\
315	0.00435084103257978\\
316	0.00435837464522503\\
317	0.00436607679851565\\
318	0.00437395658936828\\
319	0.00438202382640046\\
320	0.00439028908668731\\
321	0.00439876377509387\\
322	0.00440746018550648\\
323	0.0044163915629326\\
324	0.00442557216495628\\
325	0.00443501732036787\\
326	0.00444474348182515\\
327	0.00445476826794441\\
328	0.00446511048838201\\
329	0.00447579015169591\\
330	0.00448682859803789\\
331	0.0044982482850376\\
332	0.00451007264947188\\
333	0.00452232588076041\\
334	0.00453503257412249\\
335	0.00454821720993343\\
336	0.00456190327868517\\
337	0.00457611217701565\\
338	0.00459086203911056\\
339	0.00460616586057752\\
340	0.00462202897244422\\
341	0.00463844566146682\\
342	0.00465539466877314\\
343	0.00467283320530182\\
344	0.00469060059314415\\
345	0.00470844266884853\\
346	0.0047263464874706\\
347	0.00474429746913023\\
348	0.00476227922607206\\
349	0.00478027337450774\\
350	0.0047982593258573\\
351	0.00481621402500397\\
352	0.00483411139612893\\
353	0.00485192400943727\\
354	0.00486962229425243\\
355	0.00488717408487719\\
356	0.00490454469905291\\
357	0.00492169736146272\\
358	0.00493859069784189\\
359	0.00495517802660223\\
360	0.00497140977455703\\
361	0.00498723402977358\\
362	0.00500259743945583\\
363	0.00501744657798227\\
364	0.005031729956633\\
365	0.005045400896116\\
366	0.00505842155578471\\
367	0.00507076850744154\\
368	0.00508244036750977\\
369	0.00509346818626211\\
370	0.00510423335285532\\
371	0.00511481721267507\\
372	0.00512520656178234\\
373	0.00513538847250518\\
374	0.00514535047457036\\
375	0.00515508077750276\\
376	0.00516456855361846\\
377	0.00517380431640475\\
378	0.0051827803356665\\
379	0.00519149106327959\\
380	0.00519993359963097\\
381	0.0052081081912844\\
382	0.0052160187428736\\
383	0.00522367331509532\\
384	0.00523108456575669\\
385	0.00523827006945351\\
386	0.00524525242173173\\
387	0.00525205899284521\\
388	0.00525872114211429\\
389	0.00526527263128898\\
390	0.00527174687682881\\
391	0.00527815583816508\\
392	0.00528450241978055\\
393	0.00529079005792261\\
394	0.0052970229970351\\
395	0.00530320630372358\\
396	0.00530934586209489\\
397	0.00531544834562946\\
398	0.00532152116033423\\
399	0.0053275723537607\\
400	0.00533361048473719\\
401	0.00533964444962932\\
402	0.00534568326299393\\
403	0.00535173579411485\\
404	0.00535781046704782\\
405	0.0053639149414637\\
406	0.00537005580621256\\
407	0.00537623833788493\\
408	0.00538246645925789\\
409	0.00538874363358309\\
410	0.00539507350836763\\
411	0.00540145973500464\\
412	0.00540790581700941\\
413	0.00541441522440572\\
414	0.00542099135453166\\
415	0.00542763749337966\\
416	0.005434356781826\\
417	0.00544115218914853\\
418	0.00544802649591731\\
419	0.00545498228253614\\
420	0.00546202193096005\\
421	0.0054691476426214\\
422	0.00547636147363726\\
423	0.00548366538546617\\
424	0.00549106129752603\\
425	0.00549855110550918\\
426	0.00550613667730672\\
427	0.00551381984997456\\
428	0.00552160242794284\\
429	0.0055294861826372\\
430	0.00553747285361932\\
431	0.00554556415127188\\
432	0.00555376176094407\\
433	0.00556206734832942\\
434	0.00557048256568327\\
435	0.00557900905830964\\
436	0.00558764847057428\\
437	0.00559640245059426\\
438	0.00560527265282975\\
439	0.005614260739245\\
440	0.00562336838029298\\
441	0.00563259725609319\\
442	0.00564194905776808\\
443	0.00565142548889317\\
444	0.00566102826700626\\
445	0.00567075912511421\\
446	0.0056806198131339\\
447	0.00569061209920956\\
448	0.00570073777086308\\
449	0.00571099863595957\\
450	0.00572139652350465\\
451	0.00573193328433221\\
452	0.00574261079176623\\
453	0.00575343094229183\\
454	0.00576439565622845\\
455	0.00577550687839866\\
456	0.00578676657878726\\
457	0.00579817675318735\\
458	0.00580973942383269\\
459	0.00582145664001769\\
460	0.00583333047871008\\
461	0.00584536304516276\\
462	0.00585755647353328\\
463	0.00586991292751904\\
464	0.00588243460101435\\
465	0.00589512371879264\\
466	0.00590798253721685\\
467	0.00592101334498117\\
468	0.00593421846388834\\
469	0.00594760024966681\\
470	0.00596116109283389\\
471	0.00597490341961021\\
472	0.00598882969289202\\
473	0.00600294241328854\\
474	0.00601724412023184\\
475	0.00603173739316653\\
476	0.00604642485282856\\
477	0.00606130916262169\\
478	0.00607639303010246\\
479	0.00609167920858412\\
480	0.00610717049887266\\
481	0.0061228697511475\\
482	0.00613877986700211\\
483	0.00615490380165982\\
484	0.00617124456638256\\
485	0.00618780523109075\\
486	0.00620458892721452\\
487	0.0062215988507978\\
488	0.00623883826587823\\
489	0.00625631050816762\\
490	0.00627401898905844\\
491	0.00629196719998375\\
492	0.00631015871715825\\
493	0.00632859720672935\\
494	0.00634728643036652\\
495	0.00636623025131708\\
496	0.00638543264095454\\
497	0.0064048976858429\\
498	0.00642462959533532\\
499	0.00644463270971899\\
500	0.00646491150890807\\
501	0.00648547062167295\\
502	0.00650631483537603\\
503	0.00652744910615984\\
504	0.0065488785695011\\
505	0.00657060855100292\\
506	0.0065926445772432\\
507	0.00661499238642729\\
508	0.0066376579385048\\
509	0.00666064742429596\\
510	0.00668396727302896\\
511	0.00670762415750627\\
512	0.006731624995887\\
513	0.00675597694877953\\
514	0.00678068740997081\\
515	0.0068057639886559\\
516	0.00683121448044888\\
517	0.00685704682372615\\
518	0.0068832690369369\\
519	0.00690988913136679\\
520	0.00693691499240218\\
521	0.00696435422053964\\
522	0.00699221392113048\\
523	0.00702050042902386\\
524	0.00704921895073827\\
525	0.00707837309196509\\
526	0.00710796432099803\\
527	0.00713799174457201\\
528	0.00716845010365891\\
529	0.00719932589383715\\
530	0.00723054248467109\\
531	0.00726209276444554\\
532	0.00729404052716598\\
533	0.00732653921734199\\
534	0.0073596836632091\\
535	0.00739353211339375\\
536	0.0074281402754436\\
537	0.00746356041024936\\
538	0.0074998485267378\\
539	0.00753706529569083\\
540	0.00757527798571959\\
541	0.00761456132309162\\
542	0.00765499591701934\\
543	0.00769668136422686\\
544	0.00773976131539374\\
545	0.00778279111614491\\
546	0.00782374860307764\\
547	0.00786341518638422\\
548	0.0079033333892121\\
549	0.00794373252387418\\
550	0.00798465944434795\\
551	0.00802608825170205\\
552	0.00806797470045948\\
553	0.00811025901270837\\
554	0.00815287259048969\\
555	0.00819554138159801\\
556	0.00823719242952156\\
557	0.00827861275638618\\
558	0.00832031456782766\\
559	0.00836230830113041\\
560	0.00840456921017423\\
561	0.00844706889092748\\
562	0.00848977886278278\\
563	0.00853267111690916\\
564	0.0085757222039079\\
565	0.00861868515414452\\
566	0.00866146567320772\\
567	0.00870455834676335\\
568	0.0087479605259397\\
569	0.00879165272093579\\
570	0.00883561306364211\\
571	0.00887981747566901\\
572	0.00892423953985309\\
573	0.00896885037182793\\
574	0.00901361849800735\\
575	0.00905850974480371\\
576	0.00910348714512788\\
577	0.0091485108699222\\
578	0.00919353819462542\\
579	0.00923852351315233\\
580	0.00928341841531647\\
581	0.00932817184778417\\
582	0.00937273038379323\\
583	0.00941703863317462\\
584	0.00946103983179201\\
585	0.00950467665821592\\
586	0.00954789233435465\\
587	0.00959063207273936\\
588	0.00963284492704104\\
589	0.0096744860587673\\
590	0.00971551928305087\\
591	0.00975591931512916\\
592	0.00979567191274874\\
593	0.00983476678232329\\
594	0.00987312984006299\\
595	0.00991038882774525\\
596	0.00994573390547615\\
597	0.0099771937715668\\
598	0.00999970795535495\\
599	0\\
600	0\\
};
\addplot [color=black!80!mycolor21,solid,forget plot]
  table[row sep=crcr]{%
1	0.00395975731968208\\
2	0.00395977009500059\\
3	0.00395978317291841\\
4	0.00395979656060401\\
5	0.00395981026539567\\
6	0.00395982429480537\\
7	0.00395983865652317\\
8	0.00395985335842107\\
9	0.00395986840855766\\
10	0.00395988381518224\\
11	0.00395989958673951\\
12	0.00395991573187418\\
13	0.0039599322594356\\
14	0.0039599491784827\\
15	0.00395996649828889\\
16	0.00395998422834712\\
17	0.00396000237837513\\
18	0.00396002095832076\\
19	0.00396003997836725\\
20	0.00396005944893901\\
21	0.00396007938070724\\
22	0.00396009978459562\\
23	0.00396012067178656\\
24	0.003960142053727\\
25	0.00396016394213487\\
26	0.00396018634900549\\
27	0.003960209286618\\
28	0.00396023276754217\\
29	0.00396025680464509\\
30	0.00396028141109849\\
31	0.00396030660038564\\
32	0.00396033238630884\\
33	0.00396035878299698\\
34	0.00396038580491315\\
35	0.00396041346686255\\
36	0.00396044178400063\\
37	0.00396047077184126\\
38	0.00396050044626527\\
39	0.00396053082352899\\
40	0.00396056192027318\\
41	0.00396059375353214\\
42	0.00396062634074285\\
43	0.00396065969975455\\
44	0.00396069384883843\\
45	0.00396072880669763\\
46	0.00396076459247724\\
47	0.0039608012257749\\
48	0.00396083872665135\\
49	0.00396087711564128\\
50	0.00396091641376467\\
51	0.00396095664253795\\
52	0.00396099782398588\\
53	0.00396103998065347\\
54	0.0039610831356181\\
55	0.0039611273125021\\
56	0.00396117253548557\\
57	0.00396121882931947\\
58	0.00396126621933903\\
59	0.00396131473147746\\
60	0.00396136439227993\\
61	0.00396141522891801\\
62	0.00396146726920432\\
63	0.00396152054160757\\
64	0.00396157507526792\\
65	0.00396163090001279\\
66	0.00396168804637274\\
67	0.00396174654559832\\
68	0.0039618064296765\\
69	0.00396186773134815\\
70	0.00396193048412568\\
71	0.00396199472231097\\
72	0.00396206048101383\\
73	0.00396212779617093\\
74	0.00396219670456516\\
75	0.00396226724384521\\
76	0.00396233945254585\\
77	0.00396241337010868\\
78	0.00396248903690322\\
79	0.00396256649424842\\
80	0.00396264578443488\\
81	0.00396272695074746\\
82	0.00396281003748834\\
83	0.00396289509000075\\
84	0.00396298215469307\\
85	0.00396307127906364\\
86	0.00396316251172606\\
87	0.00396325590243491\\
88	0.00396335150211243\\
89	0.00396344936287535\\
90	0.00396354953806262\\
91	0.00396365208226378\\
92	0.00396375705134773\\
93	0.00396386450249235\\
94	0.00396397449421459\\
95	0.00396408708640154\\
96	0.00396420234034195\\
97	0.00396432031875841\\
98	0.00396444108584042\\
99	0.00396456470727807\\
100	0.00396469125029649\\
101	0.00396482078369114\\
102	0.00396495337786379\\
103	0.00396508910485923\\
104	0.0039652280384029\\
105	0.00396537025393936\\
106	0.00396551582867142\\
107	0.00396566484160037\\
108	0.0039658173735669\\
109	0.00396597350729287\\
110	0.00396613332742416\\
111	0.00396629692057432\\
112	0.00396646437536921\\
113	0.00396663578249238\\
114	0.00396681123473194\\
115	0.00396699082702781\\
116	0.00396717465652026\\
117	0.00396736282259961\\
118	0.00396755542695665\\
119	0.00396775257363438\\
120	0.0039679543690807\\
121	0.00396816092220223\\
122	0.00396837234441927\\
123	0.00396858874972176\\
124	0.00396881025472663\\
125	0.00396903697873607\\
126	0.00396926904379712\\
127	0.00396950657476246\\
128	0.00396974969935238\\
129	0.00396999854821802\\
130	0.00397025325500585\\
131	0.00397051395642351\\
132	0.00397078079230673\\
133	0.00397105390568787\\
134	0.00397133344286555\\
135	0.00397161955347573\\
136	0.00397191239056412\\
137	0.00397221211066002\\
138	0.0039725188738516\\
139	0.00397283284386243\\
140	0.00397315418812967\\
141	0.00397348307788353\\
142	0.00397381968822831\\
143	0.00397416419822478\\
144	0.0039745167909744\\
145	0.00397487765370463\\
146	0.00397524697785602\\
147	0.00397562495917076\\
148	0.00397601179778292\\
149	0.0039764076983101\\
150	0.00397681286994673\\
151	0.00397722752655897\\
152	0.00397765188678133\\
153	0.00397808617411459\\
154	0.00397853061702575\\
155	0.00397898544904938\\
156	0.00397945090889058\\
157	0.00397992724053004\\
158	0.00398041469332998\\
159	0.00398091352214262\\
160	0.00398142398741975\\
161	0.00398194635532434\\
162	0.00398248089784353\\
163	0.00398302789290365\\
164	0.00398358762448686\\
165	0.00398416038274931\\
166	0.00398474646414138\\
167	0.00398534617152946\\
168	0.00398595981431937\\
169	0.00398658770858174\\
170	0.00398723017717919\\
171	0.00398788754989488\\
172	0.00398856016356348\\
173	0.00398924836220324\\
174	0.0039899524971503\\
175	0.0039906729271948\\
176	0.00399141001871862\\
177	0.00399216414583495\\
178	0.00399293569053003\\
179	0.00399372504280643\\
180	0.00399453260082846\\
181	0.00399535877106935\\
182	0.00399620396846057\\
183	0.00399706861654289\\
184	0.00399795314761986\\
185	0.00399885800291297\\
186	0.00399978363271927\\
187	0.00400073049657103\\
188	0.00400169906339758\\
189	0.00400268981168967\\
190	0.00400370322966614\\
191	0.00400473981544289\\
192	0.00400580007720474\\
193	0.00400688453337968\\
194	0.00400799371281597\\
195	0.00400912815496207\\
196	0.00401028841004964\\
197	0.00401147503927936\\
198	0.00401268861501043\\
199	0.00401392972095257\\
200	0.00401519895236227\\
201	0.00401649691624196\\
202	0.00401782423154308\\
203	0.00401918152937262\\
204	0.00402056945320336\\
205	0.00402198865908801\\
206	0.00402343981587675\\
207	0.00402492360543842\\
208	0.00402644072288514\\
209	0.00402799187680004\\
210	0.0040295777894678\\
211	0.00403119919710726\\
212	0.00403285685010575\\
213	0.00403455151325396\\
214	0.0040362839659807\\
215	0.00403805500258511\\
216	0.00403986543246583\\
217	0.00404171608034403\\
218	0.00404360778647767\\
219	0.00404554140686416\\
220	0.00404751781342695\\
221	0.00404953789418143\\
222	0.00405160255337412\\
223	0.00405371271158865\\
224	0.00405586930580977\\
225	0.00405807328943585\\
226	0.00406032563222836\\
227	0.00406262732018404\\
228	0.00406497935531465\\
229	0.00406738275531439\\
230	0.00406983855309343\\
231	0.00407234779615168\\
232	0.00407491154576282\\
233	0.00407753087593371\\
234	0.00408020687209914\\
235	0.00408294062950564\\
236	0.00408573325123097\\
237	0.00408858584577917\\
238	0.00409149952418159\\
239	0.00409447539652615\\
240	0.00409751456782683\\
241	0.0041006181331355\\
242	0.004103787171788\\
243	0.00410702274066561\\
244	0.00411032586634469\\
245	0.00411369753599901\\
246	0.00411713868691669\\
247	0.00412065019449218\\
248	0.00412423285856576\\
249	0.00412788738800069\\
250	0.00413161438342674\\
251	0.00413541431813813\\
252	0.00413928751722765\\
253	0.00414323413517606\\
254	0.00414725413231579\\
255	0.00415134725087171\\
256	0.00415551299167899\\
257	0.00415975059322887\\
258	0.00416405901544871\\
259	0.00416843693165244\\
260	0.00417288273349081\\
261	0.00417739455560111\\
262	0.00418197032914141\\
263	0.00418660787665542\\
264	0.00419130506492994\\
265	0.00419606003845378\\
266	0.00420087166060245\\
267	0.00420574168753451\\
268	0.00421067200318543\\
269	0.0042156646302065\\
270	0.0042207217418873\\
271	0.00422584567515237\\
272	0.00423103894472913\\
273	0.00423630425859268\\
274	0.00424164453479921\\
275	0.00424706291982672\\
276	0.00425256280854856\\
277	0.00425814786596869\\
278	0.00426382205085089\\
279	0.00426958964137228\\
280	0.00427545526293053\\
281	0.00428142391822107\\
282	0.00428750101967601\\
283	0.00429369242432835\\
284	0.00430000447111757\\
285	0.00430644402058436\\
286	0.00431301849680697\\
287	0.00431973593129833\\
288	0.00432660500840334\\
289	0.0043336351114907\\
290	0.00434083636890846\\
291	0.00434821969823636\\
292	0.00435579684679416\\
293	0.00436358042560654\\
294	0.00437158393302906\\
295	0.00437982176293492\\
296	0.00438830919065439\\
297	0.00439706232762599\\
298	0.0044060980327956\\
299	0.00441543376495375\\
300	0.00442508735511623\\
301	0.0044350766715519\\
302	0.00444541914624127\\
303	0.00445613120437171\\
304	0.00446722741201072\\
305	0.0044787191116309\\
306	0.00449061275563124\\
307	0.00450290767796568\\
308	0.00451559312880382\\
309	0.00452864432839761\\
310	0.004541947907067\\
311	0.0045553588873206\\
312	0.0045688748456693\\
313	0.00458249292291672\\
314	0.00459620976519901\\
315	0.00461002144510762\\
316	0.00462392331109815\\
317	0.00463790976470962\\
318	0.00465197439747579\\
319	0.00466610990037951\\
320	0.00468030796267168\\
321	0.00469455916112629\\
322	0.00470885283943766\\
323	0.00472317697770581\\
324	0.00473751805229825\\
325	0.00475186088683822\\
326	0.00476618849557077\\
327	0.00478048192037877\\
328	0.00479472005920495\\
329	0.00480887945664696\\
330	0.00482293383031546\\
331	0.00483685605757499\\
332	0.0048506168205456\\
333	0.0048641847244187\\
334	0.0048775265556112\\
335	0.00489060775715463\\
336	0.00490339331893577\\
337	0.00491584724707331\\
338	0.00492793212758714\\
339	0.00493961304147446\\
340	0.00495086008862659\\
341	0.00496165189698724\\
342	0.00497198044421259\\
343	0.00498185762882403\\
344	0.00499141439933949\\
345	0.00500088028159732\\
346	0.00501024535832554\\
347	0.00501949946955782\\
348	0.00502863227167289\\
349	0.00503763331231683\\
350	0.00504649212381208\\
351	0.00505519833767343\\
352	0.00506374182204377\\
353	0.00507211283478288\\
354	0.00508030218103914\\
355	0.0050883014106292\\
356	0.00509610304984313\\
357	0.00510370086514399\\
358	0.00511109017558575\\
359	0.00511826823791135\\
360	0.00512523465220046\\
361	0.00513199174797994\\
362	0.00513854496156206\\
363	0.00514490317304922\\
364	0.00515107895563593\\
365	0.00515708866805653\\
366	0.00516295229114501\\
367	0.00516869286883802\\
368	0.00517433535983105\\
369	0.0051799046354389\\
370	0.00518541152707971\\
371	0.00519085787461229\\
372	0.00519624456272691\\
373	0.00520157302184255\\
374	0.00520684526534517\\
375	0.00521206392179455\\
376	0.00521723225921125\\
377	0.00522235419718806\\
378	0.005227434303618\\
379	0.00523247777304402\\
380	0.00523749038250713\\
381	0.00524247842062639\\
382	0.00524744858582909\\
383	0.00525240785037693\\
384	0.00525736328838141\\
385	0.00526232186879773\\
386	0.00526729021905066\\
387	0.00527227437234063\\
388	0.00527727952288001\\
389	0.00528230982886046\\
390	0.00528736831585664\\
391	0.0052924576227457\\
392	0.00529758048059644\\
393	0.00530273970726266\\
394	0.0053079381861204\\
395	0.00531317884193692\\
396	0.00531846461422133\\
397	0.00532379842863412\\
398	0.00532918316730713\\
399	0.00533462163924448\\
400	0.0053401165523255\\
401	0.00534567048879429\\
402	0.00535128588645114\\
403	0.00535696502800076\\
404	0.00536271004104953\\
405	0.00536852291091192\\
406	0.00537440550722331\\
407	0.00538035962001042\\
408	0.00538638704634926\\
409	0.00539248949611286\\
410	0.0053986685962717\\
411	0.00540492594510756\\
412	0.00541126311200561\\
413	0.00541768163795106\\
414	0.00542418303487263\\
415	0.00543076878631851\\
416	0.0054374403494222\\
417	0.00544419915799453\\
418	0.00545104662659633\\
419	0.00545798415547701\\
420	0.00546501313597188\\
421	0.00547213495578448\\
422	0.00547935100346694\\
423	0.00548666267140053\\
424	0.00549407135701656\\
425	0.00550157846336842\\
426	0.00550918539989171\\
427	0.00551689358333569\\
428	0.0055247044388393\\
429	0.00553261940111636\\
430	0.00554063991570623\\
431	0.00554876744024031\\
432	0.00555700344567168\\
433	0.00556534941741769\\
434	0.00557380685637413\\
435	0.0055823772797757\\
436	0.00559106222190345\\
437	0.00559986323467231\\
438	0.00560878188816548\\
439	0.00561781977116567\\
440	0.00562697849169054\\
441	0.00563625967752554\\
442	0.00564566497674826\\
443	0.00565519605823902\\
444	0.00566485461217416\\
445	0.00567464235050029\\
446	0.00568456100738965\\
447	0.00569461233967939\\
448	0.00570479812729925\\
449	0.00571512017369344\\
450	0.0057255803062425\\
451	0.00573618037668963\\
452	0.00574692226157277\\
453	0.00575780786266332\\
454	0.0057688391074119\\
455	0.00578001794940259\\
456	0.00579134636881688\\
457	0.00580282637290954\\
458	0.00581445999649804\\
459	0.0058262493024685\\
460	0.00583819638230009\\
461	0.00585030335661083\\
462	0.00586257237572729\\
463	0.00587500562028092\\
464	0.00588760530183387\\
465	0.00590037366353742\\
466	0.00591331298082653\\
467	0.00592642556215416\\
468	0.00593971374977001\\
469	0.00595317992054803\\
470	0.00596682648686794\\
471	0.00598065589755636\\
472	0.00599467063889399\\
473	0.0060088732356952\\
474	0.00602326625246796\\
475	0.00603785229466203\\
476	0.00605263401001434\\
477	0.00606761409000184\\
478	0.0060827952714119\\
479	0.00609818033804263\\
480	0.00611377212254541\\
481	0.0061295735084239\\
482	0.00614558743220402\\
483	0.00616181688579138\\
484	0.00617826491903308\\
485	0.00619493464250248\\
486	0.00621182923052606\\
487	0.00622895192447276\\
488	0.00624630603632682\\
489	0.0062638949525654\\
490	0.00628172213836242\\
491	0.00629979114213942\\
492	0.00631810560048313\\
493	0.00633666924344644\\
494	0.00635548590024656\\
495	0.00637455950536808\\
496	0.00639389410507143\\
497	0.00641349386429583\\
498	0.00643336307393241\\
499	0.00645350615842317\\
500	0.00647392768361698\\
501	0.00649463236478072\\
502	0.00651562507462131\\
503	0.0065369108511203\\
504	0.00655849490491298\\
505	0.00658038262585548\\
506	0.00660257958831128\\
507	0.00662509155454699\\
508	0.00664792447544836\\
509	0.00667108448754348\\
510	0.00669457790503862\\
511	0.0067184112052194\\
512	0.00674259100513055\\
513	0.00676712402689802\\
514	0.0067920170483731\\
515	0.00681727683492816\\
516	0.00684291004717574\\
517	0.0068689231180722\\
518	0.00689532209124132\\
519	0.00692211241034079\\
520	0.00694929864680671\\
521	0.00697688415023615\\
522	0.00700487060187458\\
523	0.00703325744699845\\
524	0.00706204117622466\\
525	0.0070912153098665\\
526	0.00712076579362154\\
527	0.00715063072912295\\
528	0.00718076948847036\\
529	0.00721124154541624\\
530	0.00724218981273631\\
531	0.00727371306630571\\
532	0.00730586429774683\\
533	0.00733869159646911\\
534	0.007372240221459\\
535	0.00740655837439732\\
536	0.00744169881164493\\
537	0.00747772044052915\\
538	0.00751468935496618\\
539	0.00755268009366983\\
540	0.00759177697910282\\
541	0.00763207998339044\\
542	0.00767373445957806\\
543	0.00771550707605202\\
544	0.00775514922292376\\
545	0.00779354939627914\\
546	0.00783225659391049\\
547	0.00787147428152859\\
548	0.00791124023747355\\
549	0.00795153400233813\\
550	0.00799231827609853\\
551	0.00803354342957163\\
552	0.00807514462435062\\
553	0.00811705216563942\\
554	0.00815850355717977\\
555	0.00819910905867986\\
556	0.00823996787570166\\
557	0.0082811455538887\\
558	0.00832262966246162\\
559	0.00836439422555056\\
560	0.00840641248997081\\
561	0.00844865781267785\\
562	0.00849110408425233\\
563	0.00853372943125429\\
564	0.00857626868344534\\
565	0.00861870126158591\\
566	0.00866146599263486\\
567	0.008704558351549\\
568	0.0087479605264543\\
569	0.00879165272106828\\
570	0.00883561306369562\\
571	0.00887981747569446\\
572	0.00892423953986631\\
573	0.00896885037183521\\
574	0.00901361849801143\\
575	0.00905850974480598\\
576	0.0091034871451291\\
577	0.00914851086992281\\
578	0.00919353819462571\\
579	0.00923852351315245\\
580	0.0092834184153165\\
581	0.00932817184778419\\
582	0.00937273038379324\\
583	0.00941703863317463\\
584	0.00946103983179202\\
585	0.00950467665821593\\
586	0.00954789233435466\\
587	0.00959063207273937\\
588	0.00963284492704104\\
589	0.0096744860587673\\
590	0.00971551928305088\\
591	0.00975591931512916\\
592	0.00979567191274874\\
593	0.00983476678232329\\
594	0.00987312984006299\\
595	0.00991038882774525\\
596	0.00994573390547615\\
597	0.0099771937715668\\
598	0.00999970795535495\\
599	0\\
600	0\\
};
\addplot [color=black,solid,forget plot]
  table[row sep=crcr]{%
1	0.0039597677845085\\
2	0.00395978087530682\\
3	0.00395979427839161\\
4	0.00395980800123459\\
5	0.00395982205148686\\
6	0.00395983643698346\\
7	0.00395985116574756\\
8	0.00395986624599527\\
9	0.00395988168614005\\
10	0.0039598974947977\\
11	0.00395991368079114\\
12	0.00395993025315539\\
13	0.00395994722114278\\
14	0.00395996459422822\\
15	0.00395998238211445\\
16	0.00396000059473777\\
17	0.00396001924227345\\
18	0.00396003833514168\\
19	0.00396005788401349\\
20	0.00396007789981684\\
21	0.00396009839374268\\
22	0.00396011937725156\\
23	0.00396014086207997\\
24	0.00396016286024721\\
25	0.00396018538406209\\
26	0.00396020844613001\\
27	0.0039602320593602\\
28	0.00396025623697302\\
29	0.00396028099250763\\
30	0.00396030633982961\\
31	0.00396033229313887\\
32	0.00396035886697801\\
33	0.00396038607624027\\
34	0.00396041393617837\\
35	0.00396044246241313\\
36	0.00396047167094243\\
37	0.00396050157815039\\
38	0.00396053220081675\\
39	0.00396056355612649\\
40	0.00396059566167981\\
41	0.00396062853550198\\
42	0.0039606621960541\\
43	0.00396069666224333\\
44	0.00396073195343398\\
45	0.00396076808945855\\
46	0.00396080509062931\\
47	0.00396084297774973\\
48	0.00396088177212679\\
49	0.00396092149558308\\
50	0.00396096217046938\\
51	0.00396100381967768\\
52	0.00396104646665429\\
53	0.00396109013541348\\
54	0.00396113485055132\\
55	0.0039611806372599\\
56	0.00396122752134198\\
57	0.00396127552922588\\
58	0.00396132468798085\\
59	0.00396137502533265\\
60	0.00396142656967982\\
61	0.00396147935011013\\
62	0.00396153339641734\\
63	0.00396158873911868\\
64	0.00396164540947253\\
65	0.00396170343949658\\
66	0.00396176286198673\\
67	0.00396182371053586\\
68	0.00396188601955362\\
69	0.00396194982428658\\
70	0.00396201516083861\\
71	0.0039620820661922\\
72	0.00396215057823004\\
73	0.00396222073575712\\
74	0.00396229257852349\\
75	0.00396236614724766\\
76	0.00396244148364046\\
77	0.00396251863042954\\
78	0.00396259763138431\\
79	0.00396267853134196\\
80	0.00396276137623372\\
81	0.00396284621311189\\
82	0.00396293309017775\\
83	0.00396302205680982\\
84	0.00396311316359311\\
85	0.00396320646234906\\
86	0.00396330200616623\\
87	0.00396339984943172\\
88	0.00396350004786328\\
89	0.00396360265854262\\
90	0.00396370773994926\\
91	0.00396381535199517\\
92	0.0039639255560606\\
93	0.00396403841503058\\
94	0.00396415399333251\\
95	0.00396427235697452\\
96	0.00396439357358506\\
97	0.00396451771245327\\
98	0.00396464484457065\\
99	0.00396477504267351\\
100	0.00396490838128676\\
101	0.00396504493676871\\
102	0.00396518478735701\\
103	0.0039653280132159\\
104	0.00396547469648452\\
105	0.00396562492132671\\
106	0.00396577877398183\\
107	0.00396593634281711\\
108	0.00396609771838132\\
109	0.00396626299345977\\
110	0.00396643226313084\\
111	0.00396660562482398\\
112	0.00396678317837919\\
113	0.00396696502610827\\
114	0.00396715127285725\\
115	0.00396734202607095\\
116	0.00396753739585896\\
117	0.0039677374950636\\
118	0.00396794243932951\\
119	0.00396815234717506\\
120	0.00396836734006598\\
121	0.0039685875424906\\
122	0.00396881308203743\\
123	0.00396904408947463\\
124	0.00396928069883171\\
125	0.0039695230474835\\
126	0.00396977127623636\\
127	0.00397002552941675\\
128	0.0039702859549623\\
129	0.00397055270451526\\
130	0.00397082593351874\\
131	0.00397110580131541\\
132	0.00397139247124912\\
133	0.00397168611076932\\
134	0.00397198689153836\\
135	0.00397229498954192\\
136	0.00397261058520255\\
137	0.00397293386349641\\
138	0.00397326501407326\\
139	0.00397360423138033\\
140	0.00397395171478916\\
141	0.00397430766872666\\
142	0.00397467230280983\\
143	0.00397504583198428\\
144	0.00397542847666717\\
145	0.00397582046289416\\
146	0.00397622202247093\\
147	0.00397663339312926\\
148	0.00397705481868773\\
149	0.00397748654921761\\
150	0.00397792884121358\\
151	0.00397838195776999\\
152	0.00397884616876252\\
153	0.00397932175103576\\
154	0.00397980898859666\\
155	0.00398030817281407\\
156	0.00398081960262511\\
157	0.00398134358474793\\
158	0.00398188043390201\\
159	0.00398243047303523\\
160	0.00398299403355913\\
161	0.00398357145559186\\
162	0.00398416308820985\\
163	0.00398476928970786\\
164	0.00398539042786867\\
165	0.00398602688024207\\
166	0.00398667903443408\\
167	0.00398734728840669\\
168	0.00398803205078902\\
169	0.00398873374119959\\
170	0.0039894527905814\\
171	0.00399018964154978\\
172	0.00399094474875364\\
173	0.00399171857925138\\
174	0.00399251161290171\\
175	0.00399332434277046\\
176	0.00399415727555434\\
177	0.00399501093202239\\
178	0.00399588584747633\\
179	0.00399678257223094\\
180	0.0039977016721154\\
181	0.00399864372899717\\
182	0.00399960934132978\\
183	0.00400059912472559\\
184	0.00400161371255571\\
185	0.0040026537565782\\
186	0.00400371992759688\\
187	0.00400481291615253\\
188	0.00400593343324845\\
189	0.00400708221111312\\
190	0.00400826000400188\\
191	0.00400946758904113\\
192	0.00401070576711679\\
193	0.00401197536381151\\
194	0.00401327723039289\\
195	0.00401461224485727\\
196	0.00401598131303215\\
197	0.00401738536974248\\
198	0.00401882538004461\\
199	0.0040203023405339\\
200	0.00402181728073037\\
201	0.004023371264549\\
202	0.00402496539186076\\
203	0.00402660080015172\\
204	0.00402827866628696\\
205	0.00403000020838805\\
206	0.00403176668783252\\
207	0.00403357941138521\\
208	0.00403543973347098\\
209	0.00403734905860098\\
210	0.0040393088439634\\
211	0.00404132060219247\\
212	0.00404338590432935\\
213	0.0040455063829901\\
214	0.00404768373575706\\
215	0.0040499197288111\\
216	0.00405221620082364\\
217	0.00405457506712875\\
218	0.00405699832419687\\
219	0.00405948805443328\\
220	0.0040620464313263\\
221	0.00406467572497081\\
222	0.00406737830799553\\
223	0.00407015666192263\\
224	0.00407301338399044\\
225	0.0040759511944706\\
226	0.00407897294451281\\
227	0.00408208162454912\\
228	0.00408528037329207\\
229	0.00408857248735747\\
230	0.00409196143154372\\
231	0.00409545084979412\\
232	0.00409904457686569\\
233	0.00410274665072023\\
234	0.0041065613256449\\
235	0.00411049308609495\\
236	0.00411454666123576\\
237	0.0041187270401359\\
238	0.00412303948753413\\
239	0.00412748956006253\\
240	0.00413208312275696\\
241	0.00413682636561934\\
242	0.00414172581991221\\
243	0.00414678837375734\\
244	0.00415202128647276\\
245	0.00415743220090805\\
246	0.00416302915281537\\
247	0.00416882057601468\\
248	0.00417481530175669\\
249	0.00418102255024025\\
250	0.00418745191167736\\
251	0.00419411331358745\\
252	0.00420101697010822\\
253	0.00420817330798137\\
254	0.00421559286245153\\
255	0.00422328613452883\\
256	0.00423126339881372\\
257	0.00423953444824777\\
258	0.00424810825858506\\
259	0.0042569925508814\\
260	0.00426619322463004\\
261	0.00427571362701491\\
262	0.00428555361468141\\
263	0.00429570835280999\\
264	0.00430616678094397\\
265	0.00431690965254892\\
266	0.00432790142762682\\
267	0.00433898891337839\\
268	0.00435017149105044\\
269	0.00436144839634799\\
270	0.00437281870481662\\
271	0.00438428131576857\\
272	0.00439583493461043\\
273	0.00440747805341768\\
274	0.00441920892959058\\
275	0.00443102556241362\\
276	0.00444292566732974\\
277	0.00445490664773285\\
278	0.00446696556407647\\
279	0.00447909910010651\\
280	0.00449130352585494\\
281	0.00450357465720858\\
282	0.00451590781213734\\
283	0.00452829776324319\\
284	0.00454073868650704\\
285	0.00455322410617472\\
286	0.00456574683581455\\
287	0.00457829891571173\\
288	0.00459087154694168\\
289	0.00460345502271227\\
290	0.0046160386578906\\
291	0.00462861071806991\\
292	0.00464115835011131\\
293	0.00465366751685637\\
294	0.00466612293970611\\
295	0.00467850805406123\\
296	0.0046908049843093\\
297	0.00470299454722566\\
298	0.00471505629544852\\
299	0.00472696861618528\\
300	0.00473870890436941\\
301	0.00475025383266734\\
302	0.00476157973286847\\
303	0.00477266299195848\\
304	0.00478348176758782\\
305	0.00479401819324593\\
306	0.00480425880689436\\
307	0.00481419678173815\\
308	0.00482383488528588\\
309	0.00483318939227866\\
310	0.00484236411110974\\
311	0.00485149449666169\\
312	0.00486057507628651\\
313	0.00486960039861545\\
314	0.00487856510697842\\
315	0.0048874640255473\\
316	0.00489629226121847\\
317	0.00490504469125553\\
318	0.00491371377281994\\
319	0.00492229177784234\\
320	0.00493077082428824\\
321	0.00493914291552824\\
322	0.00494739998910972\\
323	0.00495553397636532\\
324	0.00496353687442899\\
325	0.00497140083234837\\
326	0.00497911825306907\\
327	0.00498668191308824\\
328	0.00499408510146902\\
329	0.00500132177949659\\
330	0.00500838676088834\\
331	0.00501527590213421\\
332	0.00502198629166413\\
333	0.00502851647319545\\
334	0.00503486667732287\\
335	0.00504103905245901\\
336	0.00504703788330689\\
337	0.00505286978844215\\
338	0.00505854388973855\\
339	0.00506407188869534\\
340	0.00506946795207431\\
341	0.00507474835771505\\
342	0.00507993079070375\\
343	0.00508503314178086\\
344	0.00509006849295808\\
345	0.00509503944889471\\
346	0.00509994524652635\\
347	0.00510478542672716\\
348	0.00510955986527552\\
349	0.00511426880380323\\
350	0.00511891288001494\\
351	0.00512349315628538\\
352	0.0051280111454621\\
353	0.00513246883287037\\
354	0.00513686869355852\\
355	0.00514121370276509\\
356	0.00514550733748316\\
357	0.00514975356694455\\
358	0.00515395682903118\\
359	0.00515812198868018\\
360	0.00516225427555822\\
361	0.00516635919917049\\
362	0.00517044243948941\\
363	0.00517450971242414\\
364	0.00517856661157374\\
365	0.005182618431124\\
366	0.00518666998001815\\
367	0.00519072540531589\\
368	0.00519478805272838\\
369	0.0051988603984944\\
370	0.00520294453539825\\
371	0.0052070425919397\\
372	0.00521115679769662\\
373	0.00521528947165547\\
374	0.00521944300822667\\
375	0.00522361986090933\\
376	0.00522782252366814\\
377	0.00523205351024857\\
378	0.00523631533179417\\
379	0.00524061047329331\\
380	0.0052449413695931\\
381	0.00524931038196096\\
382	0.00525371977643563\\
383	0.00525817170547221\\
384	0.0052626681946173\\
385	0.00526721113609796\\
386	0.00527180229117754\\
387	0.00527644330274793\\
388	0.00528113571851391\\
389	0.00528588102298737\\
390	0.00529068068003026\\
391	0.00529553614885919\\
392	0.0053004488806253\\
393	0.00530542031478252\\
394	0.00531045187588167\\
395	0.00531554497092085\\
396	0.00532070098737936\\
397	0.00532592129205031\\
398	0.00533120723076478\\
399	0.00533656012905985\\
400	0.00534198129378797\\
401	0.0053474720155889\\
402	0.0053530335720522\\
403	0.00535866723129939\\
404	0.00536437425565266\\
405	0.00537015590522182\\
406	0.00537601344262375\\
407	0.00538194814960723\\
408	0.00538796129411446\\
409	0.00539405405128865\\
410	0.0054002275981115\\
411	0.0054064831151994\\
412	0.0054128217877218\\
413	0.00541924480637303\\
414	0.0054257533684365\\
415	0.00543234867890174\\
416	0.00543903195159316\\
417	0.00544580441027162\\
418	0.00545266728967045\\
419	0.00545962183642861\\
420	0.00546666930989435\\
421	0.00547381098279216\\
422	0.00548104814177019\\
423	0.0054883820878729\\
424	0.00549581413699891\\
425	0.00550334562036698\\
426	0.00551097788498487\\
427	0.00551871229411495\\
428	0.00552655022773119\\
429	0.00553449308296273\\
430	0.00554254227452093\\
431	0.00555069923510744\\
432	0.00555896541580314\\
433	0.00556734228643954\\
434	0.00557583133595569\\
435	0.00558443407274487\\
436	0.00559315202499602\\
437	0.00560198674103388\\
438	0.00561093978965945\\
439	0.00562001276049151\\
440	0.00562920726430852\\
441	0.00563852493339155\\
442	0.00564796742186839\\
443	0.00565753640605954\\
444	0.00566723358482715\\
445	0.00567706067992759\\
446	0.00568701943636906\\
447	0.00569711162277522\\
448	0.00570733903175618\\
449	0.00571770348028746\\
450	0.00572820681009836\\
451	0.00573885088807027\\
452	0.00574963760664607\\
453	0.00576056888425166\\
454	0.00577164666573088\\
455	0.00578287292279483\\
456	0.00579424965448764\\
457	0.00580577888766963\\
458	0.00581746267752005\\
459	0.00582930310806115\\
460	0.00584130229270553\\
461	0.00585346237482923\\
462	0.00586578552837295\\
463	0.00587827395847425\\
464	0.00589092990213355\\
465	0.00590375562891754\\
466	0.00591675344170365\\
467	0.00592992567746969\\
468	0.00594327470813304\\
469	0.00595680294144458\\
470	0.00597051282194282\\
471	0.00598440683197411\\
472	0.00599848749278587\\
473	0.00601275736570005\\
474	0.00602721905337455\\
475	0.00604187520116198\\
476	0.00605672849857493\\
477	0.00607178168086813\\
478	0.0060870375307495\\
479	0.00610249888023157\\
480	0.00611816861263728\\
481	0.00613404966477412\\
482	0.00615014502929169\\
483	0.00616645775723915\\
484	0.00618299096083938\\
485	0.00619974781649722\\
486	0.00621673156806019\\
487	0.0062339455303493\\
488	0.0062513930929778\\
489	0.00626907772447446\\
490	0.00628700297672548\\
491	0.00630517248974692\\
492	0.00632358999679346\\
493	0.006342259329804\\
494	0.00636118442517365\\
495	0.00638036932982981\\
496	0.00639981820757165\\
497	0.00641953534561057\\
498	0.00643952516121694\\
499	0.00645979220834055\\
500	0.00648034118401903\\
501	0.00650117693432292\\
502	0.00652230445949956\\
503	0.00654372891786793\\
504	0.00656545562787597\\
505	0.0065874900675518\\
506	0.00660983787035161\\
507	0.00663250481611568\\
508	0.0066554968154751\\
509	0.00667881988558296\\
510	0.00670248011445096\\
511	0.00672648361042255\\
512	0.00675083643236417\\
513	0.00677554449495819\\
514	0.00680061344196787\\
515	0.00682604847843297\\
516	0.00685185415034493\\
517	0.00687803405730904\\
518	0.0069045904798648\\
519	0.00693152389830011\\
520	0.00695883237369566\\
521	0.00698651075424847\\
522	0.00701454966022969\\
523	0.00704293418871195\\
524	0.00707164226379601\\
525	0.00710056610684932\\
526	0.00712974350113507\\
527	0.00715929329431106\\
528	0.00718934245787863\\
529	0.00721994484589615\\
530	0.0072511468419841\\
531	0.00728298780461047\\
532	0.00731550855386092\\
533	0.00734875362171091\\
534	0.00738277285511294\\
535	0.00741762248749887\\
536	0.00745336635521868\\
537	0.00749007722424768\\
538	0.00752783839423504\\
539	0.00756674560848086\\
540	0.00760690935443786\\
541	0.00764808540321859\\
542	0.00768701508418313\\
543	0.00772433072101707\\
544	0.00776188162433868\\
545	0.0077999527261911\\
546	0.00783859087838171\\
547	0.00787778165752892\\
548	0.00791749614938111\\
549	0.00795769533073896\\
550	0.00799832857317591\\
551	0.00803932989916792\\
552	0.00808060984117103\\
553	0.00812101779645035\\
554	0.00816104474617506\\
555	0.00820139887366875\\
556	0.00824209241508463\\
557	0.0082831028968625\\
558	0.00832440529286259\\
559	0.00836597422744438\\
560	0.0084077846659062\\
561	0.00844981241998907\\
562	0.00849203327218037\\
563	0.00853420740548651\\
564	0.00857628031085049\\
565	0.00861870130330938\\
566	0.00866146599325849\\
567	0.00870455835161939\\
568	0.00874796052647332\\
569	0.00879165272107617\\
570	0.00883561306369943\\
571	0.00887981747569646\\
572	0.00892423953986743\\
573	0.00896885037183584\\
574	0.0090136184980118\\
575	0.00905850974480618\\
576	0.00910348714512921\\
577	0.00914851086992285\\
578	0.00919353819462572\\
579	0.00923852351315244\\
580	0.0092834184153165\\
581	0.00932817184778418\\
582	0.00937273038379323\\
583	0.00941703863317462\\
584	0.00946103983179201\\
585	0.00950467665821592\\
586	0.00954789233435465\\
587	0.00959063207273936\\
588	0.00963284492704104\\
589	0.0096744860587673\\
590	0.00971551928305088\\
591	0.00975591931512916\\
592	0.00979567191274874\\
593	0.00983476678232329\\
594	0.00987312984006299\\
595	0.00991038882774525\\
596	0.00994573390547615\\
597	0.0099771937715668\\
598	0.00999970795535495\\
599	0\\
600	0\\
};
\end{axis}
\end{tikzpicture}%
 
%  \caption{Discrete Time w/ nFPC}
%\end{subfigure}\\
%
%\leavevmode\smash{\makebox[0pt]{\hspace{-7em}% HORIZONTAL POSITION           
%  \rotatebox[origin=l]{90}{\hspace{20em}% VERTICAL POSITION
%    Depth $\delta^+$}%
%}}\hspace{0pt plus 1filll}\null
%
%Time (s)
%
%\vspace{1cm}
%\begin{subfigure}{\linewidth}
%  \centering
%  \tikzsetnextfilename{deltalegend}
%  \definecolor{mycolor1}{rgb}{1.00000,0.00000,1.00000}%
\begin{tikzpicture}[framed]
    \begingroup
    % inits/clears the lists (which might be populated from previous
    % axes):
    \csname pgfplots@init@cleared@structures\endcsname
    \pgfplotsset{legend style={at={(0,1)},anchor=north west},legend columns=-1,legend style={draw=none,column sep=1ex},legend entries={$q=-4$,$q=-3$,$q=-2$,$q=-1$}}%
    
    \csname pgfplots@addlegendimage\endcsname{thick,green,dashed,sharp plot}
    \csname pgfplots@addlegendimage\endcsname{thick,mycolor1,dashed,sharp plot}
    \csname pgfplots@addlegendimage\endcsname{thick,red,dashed,sharp plot}
    \csname pgfplots@addlegendimage\endcsname{thick,blue,dashed,sharp plot}

    % draws the legend:
    \csname pgfplots@createlegend\endcsname
    \endgroup

    \begingroup
    % inits/clears the lists (which might be populated from previous
    % axes):
    \csname pgfplots@init@cleared@structures\endcsname
    \pgfplotsset{legend style={at={(3.75,0.5)},anchor=north west},legend columns=-1,legend style={draw=none,column sep=1ex},legend entries={$q=0$}}%

    \csname pgfplots@addlegendimage\endcsname{thick,black,sharp plot}

    % draws the legend:
    \csname pgfplots@createlegend\endcsname
    \endgroup

    \begingroup
    % inits/clears the lists (which might be populated from previous
    % axes):
    \csname pgfplots@init@cleared@structures\endcsname
    \pgfplotsset{legend style={at={(0,0)},anchor=north west},legend columns=-1,legend style={draw=none,column sep=1ex},legend entries={$q=+4$,$q=+3$,$q=+2$,$q=+1$}}%
    
    \csname pgfplots@addlegendimage\endcsname{thick,green,sharp plot}
    \csname pgfplots@addlegendimage\endcsname{thick,mycolor1,sharp plot}
    \csname pgfplots@addlegendimage\endcsname{thick,red,sharp plot}
    \csname pgfplots@addlegendimage\endcsname{thick,blue,sharp plot}

    % draws the legend:
    \csname pgfplots@createlegend\endcsname
    \endgroup
\end{tikzpicture} 
%\end{subfigure}%
%  \caption{Optimal buy depths $\delta^+$ for Markov state $Z=(\rho = 0, \Delta S = 0)$, implying neutral imbalance and no previous price change. We expect no change in midprice.}
%  \label{fig:comp_dp_z8}
%\end{figure}
%
%\begin{figure}
%\centering
%\begin{subfigure}{.45\linewidth}
%  \centering
%  \setlength\figureheight{\linewidth} 
%  \setlength\figurewidth{\linewidth}
%  \tikzsetnextfilename{dp_cts_z15}
%  % This file was created by matlab2tikz.
%
%The latest updates can be retrieved from
%  http://www.mathworks.com/matlabcentral/fileexchange/22022-matlab2tikz-matlab2tikz
%where you can also make suggestions and rate matlab2tikz.
%
\definecolor{mycolor1}{rgb}{1.00000,0.00000,1.00000}%
%
\begin{tikzpicture}

\begin{axis}[%
width=4.564in,
height=3.803in,
at={(1.067in,0.513in)},
scale only axis,
every outer x axis line/.append style={black},
every x tick label/.append style={font=\color{black}},
xmin=0,
xmax=100,
xlabel={Time},
every outer y axis line/.append style={black},
every y tick label/.append style={font=\color{black}},
ymin=0,
ymax=0.012,
ylabel={Depth $\delta$},
axis background/.style={fill=white},
title={Z=15},
axis x line*=bottom,
axis y line*=left,
legend style={legend cell align=left,align=left,draw=black}
]
\addplot [color=green,dashed,forget plot]
  table[row sep=crcr]{%
0.01	0.01\\
0.02	0.01\\
0.03	0.01\\
0.04	0.01\\
0.05	0.01\\
0.06	0.01\\
0.07	0.01\\
0.08	0.01\\
0.09	0.01\\
0.1	0.01\\
0.11	0.01\\
0.12	0.01\\
0.13	0.01\\
0.14	0.01\\
0.15	0.01\\
0.16	0.01\\
0.17	0.01\\
0.18	0.01\\
0.19	0.01\\
0.2	0.01\\
0.21	0.01\\
0.22	0.01\\
0.23	0.01\\
0.24	0.01\\
0.25	0.01\\
0.26	0.01\\
0.27	0.01\\
0.28	0.01\\
0.29	0.01\\
0.3	0.01\\
0.31	0.01\\
0.32	0.01\\
0.33	0.01\\
0.34	0.01\\
0.35	0.01\\
0.36	0.01\\
0.37	0.01\\
0.38	0.01\\
0.39	0.01\\
0.4	0.01\\
0.41	0.01\\
0.42	0.01\\
0.43	0.01\\
0.44	0.01\\
0.45	0.01\\
0.46	0.01\\
0.47	0.01\\
0.48	0.01\\
0.49	0.01\\
0.5	0.01\\
0.51	0.01\\
0.52	0.01\\
0.53	0.01\\
0.54	0.01\\
0.55	0.01\\
0.56	0.01\\
0.57	0.01\\
0.58	0.01\\
0.59	0.01\\
0.6	0.01\\
0.61	0.01\\
0.62	0.01\\
0.63	0.01\\
0.64	0.01\\
0.65	0.01\\
0.66	0.01\\
0.67	0.01\\
0.68	0.01\\
0.69	0.01\\
0.7	0.01\\
0.71	0.01\\
0.72	0.01\\
0.73	0.01\\
0.74	0.01\\
0.75	0.01\\
0.76	0.01\\
0.77	0.01\\
0.78	0.01\\
0.79	0.01\\
0.8	0.01\\
0.81	0.01\\
0.82	0.01\\
0.83	0.01\\
0.84	0.01\\
0.85	0.01\\
0.86	0.01\\
0.87	0.01\\
0.88	0.01\\
0.89	0.01\\
0.9	0.01\\
0.91	0.01\\
0.92	0.01\\
0.93	0.01\\
0.94	0.01\\
0.95	0.01\\
0.96	0.01\\
0.97	0.01\\
0.98	0.01\\
0.99	0.01\\
1	0.01\\
1.01	0.01\\
1.02	0.01\\
1.03	0.01\\
1.04	0.01\\
1.05	0.01\\
1.06	0.01\\
1.07	0.01\\
1.08	0.01\\
1.09	0.01\\
1.1	0.01\\
1.11	0.01\\
1.12	0.01\\
1.13	0.01\\
1.14	0.01\\
1.15	0.01\\
1.16	0.01\\
1.17	0.01\\
1.18	0.01\\
1.19	0.01\\
1.2	0.01\\
1.21	0.01\\
1.22	0.01\\
1.23	0.01\\
1.24	0.01\\
1.25	0.01\\
1.26	0.01\\
1.27	0.01\\
1.28	0.01\\
1.29	0.01\\
1.3	0.01\\
1.31	0.01\\
1.32	0.01\\
1.33	0.01\\
1.34	0.01\\
1.35	0.01\\
1.36	0.01\\
1.37	0.01\\
1.38	0.01\\
1.39	0.01\\
1.4	0.01\\
1.41	0.01\\
1.42	0.01\\
1.43	0.01\\
1.44	0.01\\
1.45	0.01\\
1.46	0.01\\
1.47	0.01\\
1.48	0.01\\
1.49	0.01\\
1.5	0.01\\
1.51	0.01\\
1.52	0.01\\
1.53	0.01\\
1.54	0.01\\
1.55	0.01\\
1.56	0.01\\
1.57	0.01\\
1.58	0.01\\
1.59	0.01\\
1.6	0.01\\
1.61	0.01\\
1.62	0.01\\
1.63	0.01\\
1.64	0.01\\
1.65	0.01\\
1.66	0.01\\
1.67	0.01\\
1.68	0.01\\
1.69	0.01\\
1.7	0.01\\
1.71	0.01\\
1.72	0.01\\
1.73	0.01\\
1.74	0.01\\
1.75	0.01\\
1.76	0.01\\
1.77	0.01\\
1.78	0.01\\
1.79	0.01\\
1.8	0.01\\
1.81	0.01\\
1.82	0.01\\
1.83	0.01\\
1.84	0.01\\
1.85	0.01\\
1.86	0.01\\
1.87	0.01\\
1.88	0.01\\
1.89	0.01\\
1.9	0.01\\
1.91	0.01\\
1.92	0.01\\
1.93	0.01\\
1.94	0.01\\
1.95	0.01\\
1.96	0.01\\
1.97	0.01\\
1.98	0.01\\
1.99	0.01\\
2	0.01\\
2.01	0.01\\
2.02	0.01\\
2.03	0.01\\
2.04	0.01\\
2.05	0.01\\
2.06	0.01\\
2.07	0.01\\
2.08	0.01\\
2.09	0.01\\
2.1	0.01\\
2.11	0.01\\
2.12	0.01\\
2.13	0.01\\
2.14	0.01\\
2.15	0.01\\
2.16	0.01\\
2.17	0.01\\
2.18	0.01\\
2.19	0.01\\
2.2	0.01\\
2.21	0.01\\
2.22	0.01\\
2.23	0.01\\
2.24	0.01\\
2.25	0.01\\
2.26	0.01\\
2.27	0.01\\
2.28	0.01\\
2.29	0.01\\
2.3	0.01\\
2.31	0.01\\
2.32	0.01\\
2.33	0.01\\
2.34	0.01\\
2.35	0.01\\
2.36	0.01\\
2.37	0.01\\
2.38	0.01\\
2.39	0.01\\
2.4	0.01\\
2.41	0.01\\
2.42	0.01\\
2.43	0.01\\
2.44	0.01\\
2.45	0.01\\
2.46	0.01\\
2.47	0.01\\
2.48	0.01\\
2.49	0.01\\
2.5	0.01\\
2.51	0.01\\
2.52	0.01\\
2.53	0.01\\
2.54	0.01\\
2.55	0.01\\
2.56	0.01\\
2.57	0.01\\
2.58	0.01\\
2.59	0.01\\
2.6	0.01\\
2.61	0.01\\
2.62	0.01\\
2.63	0.01\\
2.64	0.01\\
2.65	0.01\\
2.66	0.01\\
2.67	0.01\\
2.68	0.01\\
2.69	0.01\\
2.7	0.01\\
2.71	0.01\\
2.72	0.01\\
2.73	0.01\\
2.74	0.01\\
2.75	0.01\\
2.76	0.01\\
2.77	0.01\\
2.78	0.01\\
2.79	0.01\\
2.8	0.01\\
2.81	0.01\\
2.82	0.01\\
2.83	0.01\\
2.84	0.01\\
2.85	0.01\\
2.86	0.01\\
2.87	0.01\\
2.88	0.01\\
2.89	0.01\\
2.9	0.01\\
2.91	0.01\\
2.92	0.01\\
2.93	0.01\\
2.94	0.01\\
2.95	0.01\\
2.96	0.01\\
2.97	0.01\\
2.98	0.01\\
2.99	0.01\\
3	0.01\\
3.01	0.01\\
3.02	0.01\\
3.03	0.01\\
3.04	0.01\\
3.05	0.01\\
3.06	0.01\\
3.07	0.01\\
3.08	0.01\\
3.09	0.01\\
3.1	0.01\\
3.11	0.01\\
3.12	0.01\\
3.13	0.01\\
3.14	0.01\\
3.15	0.01\\
3.16	0.01\\
3.17	0.01\\
3.18	0.01\\
3.19	0.01\\
3.2	0.01\\
3.21	0.01\\
3.22	0.01\\
3.23	0.01\\
3.24	0.01\\
3.25	0.01\\
3.26	0.01\\
3.27	0.01\\
3.28	0.01\\
3.29	0.01\\
3.3	0.01\\
3.31	0.01\\
3.32	0.01\\
3.33	0.01\\
3.34	0.01\\
3.35	0.01\\
3.36	0.01\\
3.37	0.01\\
3.38	0.01\\
3.39	0.01\\
3.4	0.01\\
3.41	0.01\\
3.42	0.01\\
3.43	0.01\\
3.44	0.01\\
3.45	0.01\\
3.46	0.01\\
3.47	0.01\\
3.48	0.01\\
3.49	0.01\\
3.5	0.01\\
3.51	0.01\\
3.52	0.01\\
3.53	0.01\\
3.54	0.01\\
3.55	0.01\\
3.56	0.01\\
3.57	0.01\\
3.58	0.01\\
3.59	0.01\\
3.6	0.01\\
3.61	0.01\\
3.62	0.01\\
3.63	0.01\\
3.64	0.01\\
3.65	0.01\\
3.66	0.01\\
3.67	0.01\\
3.68	0.01\\
3.69	0.01\\
3.7	0.01\\
3.71	0.01\\
3.72	0.01\\
3.73	0.01\\
3.74	0.01\\
3.75	0.01\\
3.76	0.01\\
3.77	0.01\\
3.78	0.01\\
3.79	0.01\\
3.8	0.01\\
3.81	0.01\\
3.82	0.01\\
3.83	0.01\\
3.84	0.01\\
3.85	0.01\\
3.86	0.01\\
3.87	0.01\\
3.88	0.01\\
3.89	0.01\\
3.9	0.01\\
3.91	0.01\\
3.92	0.01\\
3.93	0.01\\
3.94	0.01\\
3.95	0.01\\
3.96	0.01\\
3.97	0.01\\
3.98	0.01\\
3.99	0.01\\
4	0.01\\
4.01	0.01\\
4.02	0.01\\
4.03	0.01\\
4.04	0.01\\
4.05	0.01\\
4.06	0.01\\
4.07	0.01\\
4.08	0.01\\
4.09	0.01\\
4.1	0.01\\
4.11	0.01\\
4.12	0.01\\
4.13	0.01\\
4.14	0.01\\
4.15	0.01\\
4.16	0.01\\
4.17	0.01\\
4.18	0.01\\
4.19	0.01\\
4.2	0.01\\
4.21	0.01\\
4.22	0.01\\
4.23	0.01\\
4.24	0.01\\
4.25	0.01\\
4.26	0.01\\
4.27	0.01\\
4.28	0.01\\
4.29	0.01\\
4.3	0.01\\
4.31	0.01\\
4.32	0.01\\
4.33	0.01\\
4.34	0.01\\
4.35	0.01\\
4.36	0.01\\
4.37	0.01\\
4.38	0.01\\
4.39	0.01\\
4.4	0.01\\
4.41	0.01\\
4.42	0.01\\
4.43	0.01\\
4.44	0.01\\
4.45	0.01\\
4.46	0.01\\
4.47	0.01\\
4.48	0.01\\
4.49	0.01\\
4.5	0.01\\
4.51	0.01\\
4.52	0.01\\
4.53	0.01\\
4.54	0.01\\
4.55	0.01\\
4.56	0.01\\
4.57	0.01\\
4.58	0.01\\
4.59	0.01\\
4.6	0.01\\
4.61	0.01\\
4.62	0.01\\
4.63	0.01\\
4.64	0.01\\
4.65	0.01\\
4.66	0.01\\
4.67	0.01\\
4.68	0.01\\
4.69	0.01\\
4.7	0.01\\
4.71	0.01\\
4.72	0.01\\
4.73	0.01\\
4.74	0.01\\
4.75	0.01\\
4.76	0.01\\
4.77	0.01\\
4.78	0.01\\
4.79	0.01\\
4.8	0.01\\
4.81	0.01\\
4.82	0.01\\
4.83	0.01\\
4.84	0.01\\
4.85	0.01\\
4.86	0.01\\
4.87	0.01\\
4.88	0.01\\
4.89	0.01\\
4.9	0.01\\
4.91	0.01\\
4.92	0.01\\
4.93	0.01\\
4.94	0.01\\
4.95	0.01\\
4.96	0.01\\
4.97	0.01\\
4.98	0.01\\
4.99	0.01\\
5	0.01\\
5.01	0.01\\
5.02	0.01\\
5.03	0.01\\
5.04	0.01\\
5.05	0.01\\
5.06	0.01\\
5.07	0.01\\
5.08	0.01\\
5.09	0.01\\
5.1	0.01\\
5.11	0.01\\
5.12	0.01\\
5.13	0.01\\
5.14	0.01\\
5.15	0.01\\
5.16	0.01\\
5.17	0.01\\
5.18	0.01\\
5.19	0.01\\
5.2	0.01\\
5.21	0.01\\
5.22	0.01\\
5.23	0.01\\
5.24	0.01\\
5.25	0.01\\
5.26	0.01\\
5.27	0.01\\
5.28	0.01\\
5.29	0.01\\
5.3	0.01\\
5.31	0.01\\
5.32	0.01\\
5.33	0.01\\
5.34	0.01\\
5.35	0.01\\
5.36	0.01\\
5.37	0.01\\
5.38	0.01\\
5.39	0.01\\
5.4	0.01\\
5.41	0.01\\
5.42	0.01\\
5.43	0.01\\
5.44	0.01\\
5.45	0.01\\
5.46	0.01\\
5.47	0.01\\
5.48	0.01\\
5.49	0.01\\
5.5	0.01\\
5.51	0.01\\
5.52	0.01\\
5.53	0.01\\
5.54	0.01\\
5.55	0.01\\
5.56	0.01\\
5.57	0.01\\
5.58	0.01\\
5.59	0.01\\
5.6	0.01\\
5.61	0.01\\
5.62	0.01\\
5.63	0.01\\
5.64	0.01\\
5.65	0.01\\
5.66	0.01\\
5.67	0.01\\
5.68	0.01\\
5.69	0.01\\
5.7	0.01\\
5.71	0.01\\
5.72	0.01\\
5.73	0.01\\
5.74	0.01\\
5.75	0.01\\
5.76	0.01\\
5.77	0.01\\
5.78	0.01\\
5.79	0.01\\
5.8	0.01\\
5.81	0.01\\
5.82	0.01\\
5.83	0.01\\
5.84	0.01\\
5.85	0.01\\
5.86	0.01\\
5.87	0.01\\
5.88	0.01\\
5.89	0.01\\
5.9	0.01\\
5.91	0.01\\
5.92	0.01\\
5.93	0.01\\
5.94	0.01\\
5.95	0.01\\
5.96	0.01\\
5.97	0.01\\
5.98	0.01\\
5.99	0.01\\
6	0.01\\
6.01	0.01\\
6.02	0.01\\
6.03	0.01\\
6.04	0.01\\
6.05	0.01\\
6.06	0.01\\
6.07	0.01\\
6.08	0.01\\
6.09	0.01\\
6.1	0.01\\
6.11	0.01\\
6.12	0.01\\
6.13	0.01\\
6.14	0.01\\
6.15	0.01\\
6.16	0.01\\
6.17	0.01\\
6.18	0.01\\
6.19	0.01\\
6.2	0.01\\
6.21	0.01\\
6.22	0.01\\
6.23	0.01\\
6.24	0.01\\
6.25	0.01\\
6.26	0.01\\
6.27	0.01\\
6.28	0.01\\
6.29	0.01\\
6.3	0.01\\
6.31	0.01\\
6.32	0.01\\
6.33	0.01\\
6.34	0.01\\
6.35	0.01\\
6.36	0.01\\
6.37	0.01\\
6.38	0.01\\
6.39	0.01\\
6.4	0.01\\
6.41	0.01\\
6.42	0.01\\
6.43	0.01\\
6.44	0.01\\
6.45	0.01\\
6.46	0.01\\
6.47	0.01\\
6.48	0.01\\
6.49	0.01\\
6.5	0.01\\
6.51	0.01\\
6.52	0.01\\
6.53	0.01\\
6.54	0.01\\
6.55	0.01\\
6.56	0.01\\
6.57	0.01\\
6.58	0.01\\
6.59	0.01\\
6.6	0.01\\
6.61	0.01\\
6.62	0.01\\
6.63	0.01\\
6.64	0.01\\
6.65	0.01\\
6.66	0.01\\
6.67	0.01\\
6.68	0.01\\
6.69	0.01\\
6.7	0.01\\
6.71	0.01\\
6.72	0.01\\
6.73	0.01\\
6.74	0.01\\
6.75	0.01\\
6.76	0.01\\
6.77	0.01\\
6.78	0.01\\
6.79	0.01\\
6.8	0.01\\
6.81	0.01\\
6.82	0.01\\
6.83	0.01\\
6.84	0.01\\
6.85	0.01\\
6.86	0.01\\
6.87	0.01\\
6.88	0.01\\
6.89	0.01\\
6.9	0.01\\
6.91	0.01\\
6.92	0.01\\
6.93	0.01\\
6.94	0.01\\
6.95	0.01\\
6.96	0.01\\
6.97	0.01\\
6.98	0.01\\
6.99	0.01\\
7	0.01\\
7.01	0.01\\
7.02	0.01\\
7.03	0.01\\
7.04	0.01\\
7.05	0.01\\
7.06	0.01\\
7.07	0.01\\
7.08	0.01\\
7.09	0.01\\
7.1	0.01\\
7.11	0.01\\
7.12	0.01\\
7.13	0.01\\
7.14	0.01\\
7.15	0.01\\
7.16	0.01\\
7.17	0.01\\
7.18	0.01\\
7.19	0.01\\
7.2	0.01\\
7.21	0.01\\
7.22	0.01\\
7.23	0.01\\
7.24	0.01\\
7.25	0.01\\
7.26	0.01\\
7.27	0.01\\
7.28	0.01\\
7.29	0.01\\
7.3	0.01\\
7.31	0.01\\
7.32	0.01\\
7.33	0.01\\
7.34	0.01\\
7.35	0.01\\
7.36	0.01\\
7.37	0.01\\
7.38	0.01\\
7.39	0.01\\
7.4	0.01\\
7.41	0.01\\
7.42	0.01\\
7.43	0.01\\
7.44	0.01\\
7.45	0.01\\
7.46	0.01\\
7.47	0.01\\
7.48	0.01\\
7.49	0.01\\
7.5	0.01\\
7.51	0.01\\
7.52	0.01\\
7.53	0.01\\
7.54	0.01\\
7.55	0.01\\
7.56	0.01\\
7.57	0.01\\
7.58	0.01\\
7.59	0.01\\
7.6	0.01\\
7.61	0.01\\
7.62	0.01\\
7.63	0.01\\
7.64	0.01\\
7.65	0.01\\
7.66	0.01\\
7.67	0.01\\
7.68	0.01\\
7.69	0.01\\
7.7	0.01\\
7.71	0.01\\
7.72	0.01\\
7.73	0.01\\
7.74	0.01\\
7.75	0.01\\
7.76	0.01\\
7.77	0.01\\
7.78	0.01\\
7.79	0.01\\
7.8	0.01\\
7.81	0.01\\
7.82	0.01\\
7.83	0.01\\
7.84	0.01\\
7.85	0.01\\
7.86	0.01\\
7.87	0.01\\
7.88	0.01\\
7.89	0.01\\
7.9	0.01\\
7.91	0.01\\
7.92	0.01\\
7.93	0.01\\
7.94	0.01\\
7.95	0.01\\
7.96	0.01\\
7.97	0.01\\
7.98	0.01\\
7.99	0.01\\
8	0.01\\
8.01	0.01\\
8.02	0.01\\
8.03	0.01\\
8.04	0.01\\
8.05	0.01\\
8.06	0.01\\
8.07	0.01\\
8.08	0.01\\
8.09	0.01\\
8.1	0.01\\
8.11	0.01\\
8.12	0.01\\
8.13	0.01\\
8.14	0.01\\
8.15	0.01\\
8.16	0.01\\
8.17	0.01\\
8.18	0.01\\
8.19	0.01\\
8.2	0.01\\
8.21	0.01\\
8.22	0.01\\
8.23	0.01\\
8.24	0.01\\
8.25	0.01\\
8.26	0.01\\
8.27	0.01\\
8.28	0.01\\
8.29	0.01\\
8.3	0.01\\
8.31	0.01\\
8.32	0.01\\
8.33	0.01\\
8.34	0.01\\
8.35	0.01\\
8.36	0.01\\
8.37	0.01\\
8.38	0.01\\
8.39	0.01\\
8.4	0.01\\
8.41	0.01\\
8.42	0.01\\
8.43	0.01\\
8.44	0.01\\
8.45	0.01\\
8.46	0.01\\
8.47	0.01\\
8.48	0.01\\
8.49	0.01\\
8.5	0.01\\
8.51	0.01\\
8.52	0.01\\
8.53	0.01\\
8.54	0.01\\
8.55	0.01\\
8.56	0.01\\
8.57	0.01\\
8.58	0.01\\
8.59	0.01\\
8.6	0.01\\
8.61	0.01\\
8.62	0.01\\
8.63	0.01\\
8.64	0.01\\
8.65	0.01\\
8.66	0.01\\
8.67	0.01\\
8.68	0.01\\
8.69	0.01\\
8.7	0.01\\
8.71	0.01\\
8.72	0.01\\
8.73	0.01\\
8.74	0.01\\
8.75	0.01\\
8.76	0.01\\
8.77	0.01\\
8.78	0.01\\
8.79	0.01\\
8.8	0.01\\
8.81	0.01\\
8.82	0.01\\
8.83	0.01\\
8.84	0.01\\
8.85	0.01\\
8.86	0.01\\
8.87	0.01\\
8.88	0.01\\
8.89	0.01\\
8.9	0.01\\
8.91	0.01\\
8.92	0.01\\
8.93	0.01\\
8.94	0.01\\
8.95	0.01\\
8.96	0.01\\
8.97	0.01\\
8.98	0.01\\
8.99	0.01\\
9	0.01\\
9.01	0.01\\
9.02	0.01\\
9.03	0.01\\
9.04	0.01\\
9.05	0.01\\
9.06	0.01\\
9.07	0.01\\
9.08	0.01\\
9.09	0.01\\
9.1	0.01\\
9.11	0.01\\
9.12	0.01\\
9.13	0.01\\
9.14	0.01\\
9.15	0.01\\
9.16	0.01\\
9.17	0.01\\
9.18	0.01\\
9.19	0.01\\
9.2	0.01\\
9.21	0.01\\
9.22	0.01\\
9.23	0.01\\
9.24	0.01\\
9.25	0.01\\
9.26	0.01\\
9.27	0.01\\
9.28	0.01\\
9.29	0.01\\
9.3	0.01\\
9.31	0.01\\
9.32	0.01\\
9.33	0.01\\
9.34	0.01\\
9.35	0.01\\
9.36	0.01\\
9.37	0.01\\
9.38	0.01\\
9.39	0.01\\
9.4	0.01\\
9.41	0.01\\
9.42	0.01\\
9.43	0.01\\
9.44	0.01\\
9.45	0.01\\
9.46	0.01\\
9.47	0.01\\
9.48	0.01\\
9.49	0.01\\
9.5	0.01\\
9.51	0.01\\
9.52	0.01\\
9.53	0.01\\
9.54	0.01\\
9.55	0.01\\
9.56	0.01\\
9.57	0.01\\
9.58	0.01\\
9.59	0.01\\
9.6	0.01\\
9.61	0.01\\
9.62	0.01\\
9.63	0.01\\
9.64	0.01\\
9.65	0.01\\
9.66	0.01\\
9.67	0.01\\
9.68	0.01\\
9.69	0.01\\
9.7	0.01\\
9.71	0.01\\
9.72	0.01\\
9.73	0.01\\
9.74	0.01\\
9.75	0.01\\
9.76	0.01\\
9.77	0.01\\
9.78	0.01\\
9.79	0.01\\
9.8	0.01\\
9.81	0.01\\
9.82	0.01\\
9.83	0.01\\
9.84	0.01\\
9.85	0.01\\
9.86	0.01\\
9.87	0.01\\
9.88	0.01\\
9.89	0.01\\
9.9	0.01\\
9.91	0.01\\
9.92	0.01\\
9.93	0.01\\
9.94	0.01\\
9.95	0.01\\
9.96	0.01\\
9.97	0.01\\
9.98	0.01\\
9.99	0.01\\
10	0.01\\
10.01	0.01\\
10.02	0.01\\
10.03	0.01\\
10.04	0.01\\
10.05	0.01\\
10.06	0.01\\
10.07	0.01\\
10.08	0.01\\
10.09	0.01\\
10.1	0.01\\
10.11	0.01\\
10.12	0.01\\
10.13	0.01\\
10.14	0.01\\
10.15	0.01\\
10.16	0.01\\
10.17	0.01\\
10.18	0.01\\
10.19	0.01\\
10.2	0.01\\
10.21	0.01\\
10.22	0.01\\
10.23	0.01\\
10.24	0.01\\
10.25	0.01\\
10.26	0.01\\
10.27	0.01\\
10.28	0.01\\
10.29	0.01\\
10.3	0.01\\
10.31	0.01\\
10.32	0.01\\
10.33	0.01\\
10.34	0.01\\
10.35	0.01\\
10.36	0.01\\
10.37	0.01\\
10.38	0.01\\
10.39	0.01\\
10.4	0.01\\
10.41	0.01\\
10.42	0.01\\
10.43	0.01\\
10.44	0.01\\
10.45	0.01\\
10.46	0.01\\
10.47	0.01\\
10.48	0.01\\
10.49	0.01\\
10.5	0.01\\
10.51	0.01\\
10.52	0.01\\
10.53	0.01\\
10.54	0.01\\
10.55	0.01\\
10.56	0.01\\
10.57	0.01\\
10.58	0.01\\
10.59	0.01\\
10.6	0.01\\
10.61	0.01\\
10.62	0.01\\
10.63	0.01\\
10.64	0.01\\
10.65	0.01\\
10.66	0.01\\
10.67	0.01\\
10.68	0.01\\
10.69	0.01\\
10.7	0.01\\
10.71	0.01\\
10.72	0.01\\
10.73	0.01\\
10.74	0.01\\
10.75	0.01\\
10.76	0.01\\
10.77	0.01\\
10.78	0.01\\
10.79	0.01\\
10.8	0.01\\
10.81	0.01\\
10.82	0.01\\
10.83	0.01\\
10.84	0.01\\
10.85	0.01\\
10.86	0.01\\
10.87	0.01\\
10.88	0.01\\
10.89	0.01\\
10.9	0.01\\
10.91	0.01\\
10.92	0.01\\
10.93	0.01\\
10.94	0.01\\
10.95	0.01\\
10.96	0.01\\
10.97	0.01\\
10.98	0.01\\
10.99	0.01\\
11	0.01\\
11.01	0.01\\
11.02	0.01\\
11.03	0.01\\
11.04	0.01\\
11.05	0.01\\
11.06	0.01\\
11.07	0.01\\
11.08	0.01\\
11.09	0.01\\
11.1	0.01\\
11.11	0.01\\
11.12	0.01\\
11.13	0.01\\
11.14	0.01\\
11.15	0.01\\
11.16	0.01\\
11.17	0.01\\
11.18	0.01\\
11.19	0.01\\
11.2	0.01\\
11.21	0.01\\
11.22	0.01\\
11.23	0.01\\
11.24	0.01\\
11.25	0.01\\
11.26	0.01\\
11.27	0.01\\
11.28	0.01\\
11.29	0.01\\
11.3	0.01\\
11.31	0.01\\
11.32	0.01\\
11.33	0.01\\
11.34	0.01\\
11.35	0.01\\
11.36	0.01\\
11.37	0.01\\
11.38	0.01\\
11.39	0.01\\
11.4	0.01\\
11.41	0.01\\
11.42	0.01\\
11.43	0.01\\
11.44	0.01\\
11.45	0.01\\
11.46	0.01\\
11.47	0.01\\
11.48	0.01\\
11.49	0.01\\
11.5	0.01\\
11.51	0.01\\
11.52	0.01\\
11.53	0.01\\
11.54	0.01\\
11.55	0.01\\
11.56	0.01\\
11.57	0.01\\
11.58	0.01\\
11.59	0.01\\
11.6	0.01\\
11.61	0.01\\
11.62	0.01\\
11.63	0.01\\
11.64	0.01\\
11.65	0.01\\
11.66	0.01\\
11.67	0.01\\
11.68	0.01\\
11.69	0.01\\
11.7	0.01\\
11.71	0.01\\
11.72	0.01\\
11.73	0.01\\
11.74	0.01\\
11.75	0.01\\
11.76	0.01\\
11.77	0.01\\
11.78	0.01\\
11.79	0.01\\
11.8	0.01\\
11.81	0.01\\
11.82	0.01\\
11.83	0.01\\
11.84	0.01\\
11.85	0.01\\
11.86	0.01\\
11.87	0.01\\
11.88	0.01\\
11.89	0.01\\
11.9	0.01\\
11.91	0.01\\
11.92	0.01\\
11.93	0.01\\
11.94	0.01\\
11.95	0.01\\
11.96	0.01\\
11.97	0.01\\
11.98	0.01\\
11.99	0.01\\
12	0.01\\
12.01	0.01\\
12.02	0.01\\
12.03	0.01\\
12.04	0.01\\
12.05	0.01\\
12.06	0.01\\
12.07	0.01\\
12.08	0.01\\
12.09	0.01\\
12.1	0.01\\
12.11	0.01\\
12.12	0.01\\
12.13	0.01\\
12.14	0.01\\
12.15	0.01\\
12.16	0.01\\
12.17	0.01\\
12.18	0.01\\
12.19	0.01\\
12.2	0.01\\
12.21	0.01\\
12.22	0.01\\
12.23	0.01\\
12.24	0.01\\
12.25	0.01\\
12.26	0.01\\
12.27	0.01\\
12.28	0.01\\
12.29	0.01\\
12.3	0.01\\
12.31	0.01\\
12.32	0.01\\
12.33	0.01\\
12.34	0.01\\
12.35	0.01\\
12.36	0.01\\
12.37	0.01\\
12.38	0.01\\
12.39	0.01\\
12.4	0.01\\
12.41	0.01\\
12.42	0.01\\
12.43	0.01\\
12.44	0.01\\
12.45	0.01\\
12.46	0.01\\
12.47	0.01\\
12.48	0.01\\
12.49	0.01\\
12.5	0.01\\
12.51	0.01\\
12.52	0.01\\
12.53	0.01\\
12.54	0.01\\
12.55	0.01\\
12.56	0.01\\
12.57	0.01\\
12.58	0.01\\
12.59	0.01\\
12.6	0.01\\
12.61	0.01\\
12.62	0.01\\
12.63	0.01\\
12.64	0.01\\
12.65	0.01\\
12.66	0.01\\
12.67	0.01\\
12.68	0.01\\
12.69	0.01\\
12.7	0.01\\
12.71	0.01\\
12.72	0.01\\
12.73	0.01\\
12.74	0.01\\
12.75	0.01\\
12.76	0.01\\
12.77	0.01\\
12.78	0.01\\
12.79	0.01\\
12.8	0.01\\
12.81	0.01\\
12.82	0.01\\
12.83	0.01\\
12.84	0.01\\
12.85	0.01\\
12.86	0.01\\
12.87	0.01\\
12.88	0.01\\
12.89	0.01\\
12.9	0.01\\
12.91	0.01\\
12.92	0.01\\
12.93	0.01\\
12.94	0.01\\
12.95	0.01\\
12.96	0.01\\
12.97	0.01\\
12.98	0.01\\
12.99	0.01\\
13	0.01\\
13.01	0.01\\
13.02	0.01\\
13.03	0.01\\
13.04	0.01\\
13.05	0.01\\
13.06	0.01\\
13.07	0.01\\
13.08	0.01\\
13.09	0.01\\
13.1	0.01\\
13.11	0.01\\
13.12	0.01\\
13.13	0.01\\
13.14	0.01\\
13.15	0.01\\
13.16	0.01\\
13.17	0.01\\
13.18	0.01\\
13.19	0.01\\
13.2	0.01\\
13.21	0.01\\
13.22	0.01\\
13.23	0.01\\
13.24	0.01\\
13.25	0.01\\
13.26	0.01\\
13.27	0.01\\
13.28	0.01\\
13.29	0.01\\
13.3	0.01\\
13.31	0.01\\
13.32	0.01\\
13.33	0.01\\
13.34	0.01\\
13.35	0.01\\
13.36	0.01\\
13.37	0.01\\
13.38	0.01\\
13.39	0.01\\
13.4	0.01\\
13.41	0.01\\
13.42	0.01\\
13.43	0.01\\
13.44	0.01\\
13.45	0.01\\
13.46	0.01\\
13.47	0.01\\
13.48	0.01\\
13.49	0.01\\
13.5	0.01\\
13.51	0.01\\
13.52	0.01\\
13.53	0.01\\
13.54	0.01\\
13.55	0.01\\
13.56	0.01\\
13.57	0.01\\
13.58	0.01\\
13.59	0.01\\
13.6	0.01\\
13.61	0.01\\
13.62	0.01\\
13.63	0.01\\
13.64	0.01\\
13.65	0.01\\
13.66	0.01\\
13.67	0.01\\
13.68	0.01\\
13.69	0.01\\
13.7	0.01\\
13.71	0.01\\
13.72	0.01\\
13.73	0.01\\
13.74	0.01\\
13.75	0.01\\
13.76	0.01\\
13.77	0.01\\
13.78	0.01\\
13.79	0.01\\
13.8	0.01\\
13.81	0.01\\
13.82	0.01\\
13.83	0.01\\
13.84	0.01\\
13.85	0.01\\
13.86	0.01\\
13.87	0.01\\
13.88	0.01\\
13.89	0.01\\
13.9	0.01\\
13.91	0.01\\
13.92	0.01\\
13.93	0.01\\
13.94	0.01\\
13.95	0.01\\
13.96	0.01\\
13.97	0.01\\
13.98	0.01\\
13.99	0.01\\
14	0.01\\
14.01	0.01\\
14.02	0.01\\
14.03	0.01\\
14.04	0.01\\
14.05	0.01\\
14.06	0.01\\
14.07	0.01\\
14.08	0.01\\
14.09	0.01\\
14.1	0.01\\
14.11	0.01\\
14.12	0.01\\
14.13	0.01\\
14.14	0.01\\
14.15	0.01\\
14.16	0.01\\
14.17	0.01\\
14.18	0.01\\
14.19	0.01\\
14.2	0.01\\
14.21	0.01\\
14.22	0.01\\
14.23	0.01\\
14.24	0.01\\
14.25	0.01\\
14.26	0.01\\
14.27	0.01\\
14.28	0.01\\
14.29	0.01\\
14.3	0.01\\
14.31	0.01\\
14.32	0.01\\
14.33	0.01\\
14.34	0.01\\
14.35	0.01\\
14.36	0.01\\
14.37	0.01\\
14.38	0.01\\
14.39	0.01\\
14.4	0.01\\
14.41	0.01\\
14.42	0.01\\
14.43	0.01\\
14.44	0.01\\
14.45	0.01\\
14.46	0.01\\
14.47	0.01\\
14.48	0.01\\
14.49	0.01\\
14.5	0.01\\
14.51	0.01\\
14.52	0.01\\
14.53	0.01\\
14.54	0.01\\
14.55	0.01\\
14.56	0.01\\
14.57	0.01\\
14.58	0.01\\
14.59	0.01\\
14.6	0.01\\
14.61	0.01\\
14.62	0.01\\
14.63	0.01\\
14.64	0.01\\
14.65	0.01\\
14.66	0.01\\
14.67	0.01\\
14.68	0.01\\
14.69	0.01\\
14.7	0.01\\
14.71	0.01\\
14.72	0.01\\
14.73	0.01\\
14.74	0.01\\
14.75	0.01\\
14.76	0.01\\
14.77	0.01\\
14.78	0.01\\
14.79	0.01\\
14.8	0.01\\
14.81	0.01\\
14.82	0.01\\
14.83	0.01\\
14.84	0.01\\
14.85	0.01\\
14.86	0.01\\
14.87	0.01\\
14.88	0.01\\
14.89	0.01\\
14.9	0.01\\
14.91	0.01\\
14.92	0.01\\
14.93	0.01\\
14.94	0.01\\
14.95	0.01\\
14.96	0.01\\
14.97	0.01\\
14.98	0.01\\
14.99	0.01\\
15	0.01\\
15.01	0.01\\
15.02	0.01\\
15.03	0.01\\
15.04	0.01\\
15.05	0.01\\
15.06	0.01\\
15.07	0.01\\
15.08	0.01\\
15.09	0.01\\
15.1	0.01\\
15.11	0.01\\
15.12	0.01\\
15.13	0.01\\
15.14	0.01\\
15.15	0.01\\
15.16	0.01\\
15.17	0.01\\
15.18	0.01\\
15.19	0.01\\
15.2	0.01\\
15.21	0.01\\
15.22	0.01\\
15.23	0.01\\
15.24	0.01\\
15.25	0.01\\
15.26	0.01\\
15.27	0.01\\
15.28	0.01\\
15.29	0.01\\
15.3	0.01\\
15.31	0.01\\
15.32	0.01\\
15.33	0.01\\
15.34	0.01\\
15.35	0.01\\
15.36	0.01\\
15.37	0.01\\
15.38	0.01\\
15.39	0.01\\
15.4	0.01\\
15.41	0.01\\
15.42	0.01\\
15.43	0.01\\
15.44	0.01\\
15.45	0.01\\
15.46	0.01\\
15.47	0.01\\
15.48	0.01\\
15.49	0.01\\
15.5	0.01\\
15.51	0.01\\
15.52	0.01\\
15.53	0.01\\
15.54	0.01\\
15.55	0.01\\
15.56	0.01\\
15.57	0.01\\
15.58	0.01\\
15.59	0.01\\
15.6	0.01\\
15.61	0.01\\
15.62	0.01\\
15.63	0.01\\
15.64	0.01\\
15.65	0.01\\
15.66	0.01\\
15.67	0.01\\
15.68	0.01\\
15.69	0.01\\
15.7	0.01\\
15.71	0.01\\
15.72	0.01\\
15.73	0.01\\
15.74	0.01\\
15.75	0.01\\
15.76	0.01\\
15.77	0.01\\
15.78	0.01\\
15.79	0.01\\
15.8	0.01\\
15.81	0.01\\
15.82	0.01\\
15.83	0.01\\
15.84	0.01\\
15.85	0.01\\
15.86	0.01\\
15.87	0.01\\
15.88	0.01\\
15.89	0.01\\
15.9	0.01\\
15.91	0.01\\
15.92	0.01\\
15.93	0.01\\
15.94	0.01\\
15.95	0.01\\
15.96	0.01\\
15.97	0.01\\
15.98	0.01\\
15.99	0.01\\
16	0.01\\
16.01	0.01\\
16.02	0.01\\
16.03	0.01\\
16.04	0.01\\
16.05	0.01\\
16.06	0.01\\
16.07	0.01\\
16.08	0.01\\
16.09	0.01\\
16.1	0.01\\
16.11	0.01\\
16.12	0.01\\
16.13	0.01\\
16.14	0.01\\
16.15	0.01\\
16.16	0.01\\
16.17	0.01\\
16.18	0.01\\
16.19	0.01\\
16.2	0.01\\
16.21	0.01\\
16.22	0.01\\
16.23	0.01\\
16.24	0.01\\
16.25	0.01\\
16.26	0.01\\
16.27	0.01\\
16.28	0.01\\
16.29	0.01\\
16.3	0.01\\
16.31	0.01\\
16.32	0.01\\
16.33	0.01\\
16.34	0.01\\
16.35	0.01\\
16.36	0.01\\
16.37	0.01\\
16.38	0.01\\
16.39	0.01\\
16.4	0.01\\
16.41	0.01\\
16.42	0.01\\
16.43	0.01\\
16.44	0.01\\
16.45	0.01\\
16.46	0.01\\
16.47	0.01\\
16.48	0.01\\
16.49	0.01\\
16.5	0.01\\
16.51	0.01\\
16.52	0.01\\
16.53	0.01\\
16.54	0.01\\
16.55	0.01\\
16.56	0.01\\
16.57	0.01\\
16.58	0.01\\
16.59	0.01\\
16.6	0.01\\
16.61	0.01\\
16.62	0.01\\
16.63	0.01\\
16.64	0.01\\
16.65	0.01\\
16.66	0.01\\
16.67	0.01\\
16.68	0.01\\
16.69	0.01\\
16.7	0.01\\
16.71	0.01\\
16.72	0.01\\
16.73	0.01\\
16.74	0.01\\
16.75	0.01\\
16.76	0.01\\
16.77	0.01\\
16.78	0.01\\
16.79	0.01\\
16.8	0.01\\
16.81	0.01\\
16.82	0.01\\
16.83	0.01\\
16.84	0.01\\
16.85	0.01\\
16.86	0.01\\
16.87	0.01\\
16.88	0.01\\
16.89	0.01\\
16.9	0.01\\
16.91	0.01\\
16.92	0.01\\
16.93	0.01\\
16.94	0.01\\
16.95	0.01\\
16.96	0.01\\
16.97	0.01\\
16.98	0.01\\
16.99	0.01\\
17	0.01\\
17.01	0.01\\
17.02	0.01\\
17.03	0.01\\
17.04	0.01\\
17.05	0.01\\
17.06	0.01\\
17.07	0.01\\
17.08	0.01\\
17.09	0.01\\
17.1	0.01\\
17.11	0.01\\
17.12	0.01\\
17.13	0.01\\
17.14	0.01\\
17.15	0.01\\
17.16	0.01\\
17.17	0.01\\
17.18	0.01\\
17.19	0.01\\
17.2	0.01\\
17.21	0.01\\
17.22	0.01\\
17.23	0.01\\
17.24	0.01\\
17.25	0.01\\
17.26	0.01\\
17.27	0.01\\
17.28	0.01\\
17.29	0.01\\
17.3	0.01\\
17.31	0.01\\
17.32	0.01\\
17.33	0.01\\
17.34	0.01\\
17.35	0.01\\
17.36	0.01\\
17.37	0.01\\
17.38	0.01\\
17.39	0.01\\
17.4	0.01\\
17.41	0.01\\
17.42	0.01\\
17.43	0.01\\
17.44	0.01\\
17.45	0.01\\
17.46	0.01\\
17.47	0.01\\
17.48	0.01\\
17.49	0.01\\
17.5	0.01\\
17.51	0.01\\
17.52	0.01\\
17.53	0.01\\
17.54	0.01\\
17.55	0.01\\
17.56	0.01\\
17.57	0.01\\
17.58	0.01\\
17.59	0.01\\
17.6	0.01\\
17.61	0.01\\
17.62	0.01\\
17.63	0.01\\
17.64	0.01\\
17.65	0.01\\
17.66	0.01\\
17.67	0.01\\
17.68	0.01\\
17.69	0.01\\
17.7	0.01\\
17.71	0.01\\
17.72	0.01\\
17.73	0.01\\
17.74	0.01\\
17.75	0.01\\
17.76	0.01\\
17.77	0.01\\
17.78	0.01\\
17.79	0.01\\
17.8	0.01\\
17.81	0.01\\
17.82	0.01\\
17.83	0.01\\
17.84	0.01\\
17.85	0.01\\
17.86	0.01\\
17.87	0.01\\
17.88	0.01\\
17.89	0.01\\
17.9	0.01\\
17.91	0.01\\
17.92	0.01\\
17.93	0.01\\
17.94	0.01\\
17.95	0.01\\
17.96	0.01\\
17.97	0.01\\
17.98	0.01\\
17.99	0.01\\
18	0.01\\
18.01	0.01\\
18.02	0.01\\
18.03	0.01\\
18.04	0.01\\
18.05	0.01\\
18.06	0.01\\
18.07	0.01\\
18.08	0.01\\
18.09	0.01\\
18.1	0.01\\
18.11	0.01\\
18.12	0.01\\
18.13	0.01\\
18.14	0.01\\
18.15	0.01\\
18.16	0.01\\
18.17	0.01\\
18.18	0.01\\
18.19	0.01\\
18.2	0.01\\
18.21	0.01\\
18.22	0.01\\
18.23	0.01\\
18.24	0.01\\
18.25	0.01\\
18.26	0.01\\
18.27	0.01\\
18.28	0.01\\
18.29	0.01\\
18.3	0.01\\
18.31	0.01\\
18.32	0.01\\
18.33	0.01\\
18.34	0.01\\
18.35	0.01\\
18.36	0.01\\
18.37	0.01\\
18.38	0.01\\
18.39	0.01\\
18.4	0.01\\
18.41	0.01\\
18.42	0.01\\
18.43	0.01\\
18.44	0.01\\
18.45	0.01\\
18.46	0.01\\
18.47	0.01\\
18.48	0.01\\
18.49	0.01\\
18.5	0.01\\
18.51	0.01\\
18.52	0.01\\
18.53	0.01\\
18.54	0.01\\
18.55	0.01\\
18.56	0.01\\
18.57	0.01\\
18.58	0.01\\
18.59	0.01\\
18.6	0.01\\
18.61	0.01\\
18.62	0.01\\
18.63	0.01\\
18.64	0.01\\
18.65	0.01\\
18.66	0.01\\
18.67	0.01\\
18.68	0.01\\
18.69	0.01\\
18.7	0.01\\
18.71	0.01\\
18.72	0.01\\
18.73	0.01\\
18.74	0.01\\
18.75	0.01\\
18.76	0.01\\
18.77	0.01\\
18.78	0.01\\
18.79	0.01\\
18.8	0.01\\
18.81	0.01\\
18.82	0.01\\
18.83	0.01\\
18.84	0.01\\
18.85	0.01\\
18.86	0.01\\
18.87	0.01\\
18.88	0.01\\
18.89	0.01\\
18.9	0.01\\
18.91	0.01\\
18.92	0.01\\
18.93	0.01\\
18.94	0.01\\
18.95	0.01\\
18.96	0.01\\
18.97	0.01\\
18.98	0.01\\
18.99	0.01\\
19	0.01\\
19.01	0.01\\
19.02	0.01\\
19.03	0.01\\
19.04	0.01\\
19.05	0.01\\
19.06	0.01\\
19.07	0.01\\
19.08	0.01\\
19.09	0.01\\
19.1	0.01\\
19.11	0.01\\
19.12	0.01\\
19.13	0.01\\
19.14	0.01\\
19.15	0.01\\
19.16	0.01\\
19.17	0.01\\
19.18	0.01\\
19.19	0.01\\
19.2	0.01\\
19.21	0.01\\
19.22	0.01\\
19.23	0.01\\
19.24	0.01\\
19.25	0.01\\
19.26	0.01\\
19.27	0.01\\
19.28	0.01\\
19.29	0.01\\
19.3	0.01\\
19.31	0.01\\
19.32	0.01\\
19.33	0.01\\
19.34	0.01\\
19.35	0.01\\
19.36	0.01\\
19.37	0.01\\
19.38	0.01\\
19.39	0.01\\
19.4	0.01\\
19.41	0.01\\
19.42	0.01\\
19.43	0.01\\
19.44	0.01\\
19.45	0.01\\
19.46	0.01\\
19.47	0.01\\
19.48	0.01\\
19.49	0.01\\
19.5	0.01\\
19.51	0.01\\
19.52	0.01\\
19.53	0.01\\
19.54	0.01\\
19.55	0.01\\
19.56	0.01\\
19.57	0.01\\
19.58	0.01\\
19.59	0.01\\
19.6	0.01\\
19.61	0.01\\
19.62	0.01\\
19.63	0.01\\
19.64	0.01\\
19.65	0.01\\
19.66	0.01\\
19.67	0.01\\
19.68	0.01\\
19.69	0.01\\
19.7	0.01\\
19.71	0.01\\
19.72	0.01\\
19.73	0.01\\
19.74	0.01\\
19.75	0.01\\
19.76	0.01\\
19.77	0.01\\
19.78	0.01\\
19.79	0.01\\
19.8	0.01\\
19.81	0.01\\
19.82	0.01\\
19.83	0.01\\
19.84	0.01\\
19.85	0.01\\
19.86	0.01\\
19.87	0.01\\
19.88	0.01\\
19.89	0.01\\
19.9	0.01\\
19.91	0.01\\
19.92	0.01\\
19.93	0.01\\
19.94	0.01\\
19.95	0.01\\
19.96	0.01\\
19.97	0.01\\
19.98	0.01\\
19.99	0.01\\
20	0.01\\
20.01	0.01\\
20.02	0.01\\
20.03	0.01\\
20.04	0.01\\
20.05	0.01\\
20.06	0.01\\
20.07	0.01\\
20.08	0.01\\
20.09	0.01\\
20.1	0.01\\
20.11	0.01\\
20.12	0.01\\
20.13	0.01\\
20.14	0.01\\
20.15	0.01\\
20.16	0.01\\
20.17	0.01\\
20.18	0.01\\
20.19	0.01\\
20.2	0.01\\
20.21	0.01\\
20.22	0.01\\
20.23	0.01\\
20.24	0.01\\
20.25	0.01\\
20.26	0.01\\
20.27	0.01\\
20.28	0.01\\
20.29	0.01\\
20.3	0.01\\
20.31	0.01\\
20.32	0.01\\
20.33	0.01\\
20.34	0.01\\
20.35	0.01\\
20.36	0.01\\
20.37	0.01\\
20.38	0.01\\
20.39	0.01\\
20.4	0.01\\
20.41	0.01\\
20.42	0.01\\
20.43	0.01\\
20.44	0.01\\
20.45	0.01\\
20.46	0.01\\
20.47	0.01\\
20.48	0.01\\
20.49	0.01\\
20.5	0.01\\
20.51	0.01\\
20.52	0.01\\
20.53	0.01\\
20.54	0.01\\
20.55	0.01\\
20.56	0.01\\
20.57	0.01\\
20.58	0.01\\
20.59	0.01\\
20.6	0.01\\
20.61	0.01\\
20.62	0.01\\
20.63	0.01\\
20.64	0.01\\
20.65	0.01\\
20.66	0.01\\
20.67	0.01\\
20.68	0.01\\
20.69	0.01\\
20.7	0.01\\
20.71	0.01\\
20.72	0.01\\
20.73	0.01\\
20.74	0.01\\
20.75	0.01\\
20.76	0.01\\
20.77	0.01\\
20.78	0.01\\
20.79	0.01\\
20.8	0.01\\
20.81	0.01\\
20.82	0.01\\
20.83	0.01\\
20.84	0.01\\
20.85	0.01\\
20.86	0.01\\
20.87	0.01\\
20.88	0.01\\
20.89	0.01\\
20.9	0.01\\
20.91	0.01\\
20.92	0.01\\
20.93	0.01\\
20.94	0.01\\
20.95	0.01\\
20.96	0.01\\
20.97	0.01\\
20.98	0.01\\
20.99	0.01\\
21	0.01\\
21.01	0.01\\
21.02	0.01\\
21.03	0.01\\
21.04	0.01\\
21.05	0.01\\
21.06	0.01\\
21.07	0.01\\
21.08	0.01\\
21.09	0.01\\
21.1	0.01\\
21.11	0.01\\
21.12	0.01\\
21.13	0.01\\
21.14	0.01\\
21.15	0.01\\
21.16	0.01\\
21.17	0.01\\
21.18	0.01\\
21.19	0.01\\
21.2	0.01\\
21.21	0.01\\
21.22	0.01\\
21.23	0.01\\
21.24	0.01\\
21.25	0.01\\
21.26	0.01\\
21.27	0.01\\
21.28	0.01\\
21.29	0.01\\
21.3	0.01\\
21.31	0.01\\
21.32	0.01\\
21.33	0.01\\
21.34	0.01\\
21.35	0.01\\
21.36	0.01\\
21.37	0.01\\
21.38	0.01\\
21.39	0.01\\
21.4	0.01\\
21.41	0.01\\
21.42	0.01\\
21.43	0.01\\
21.44	0.01\\
21.45	0.01\\
21.46	0.01\\
21.47	0.01\\
21.48	0.01\\
21.49	0.01\\
21.5	0.01\\
21.51	0.01\\
21.52	0.01\\
21.53	0.01\\
21.54	0.01\\
21.55	0.01\\
21.56	0.01\\
21.57	0.01\\
21.58	0.01\\
21.59	0.01\\
21.6	0.01\\
21.61	0.01\\
21.62	0.01\\
21.63	0.01\\
21.64	0.01\\
21.65	0.01\\
21.66	0.01\\
21.67	0.01\\
21.68	0.01\\
21.69	0.01\\
21.7	0.01\\
21.71	0.01\\
21.72	0.01\\
21.73	0.01\\
21.74	0.01\\
21.75	0.01\\
21.76	0.01\\
21.77	0.01\\
21.78	0.01\\
21.79	0.01\\
21.8	0.01\\
21.81	0.01\\
21.82	0.01\\
21.83	0.01\\
21.84	0.01\\
21.85	0.01\\
21.86	0.01\\
21.87	0.01\\
21.88	0.01\\
21.89	0.01\\
21.9	0.01\\
21.91	0.01\\
21.92	0.01\\
21.93	0.01\\
21.94	0.01\\
21.95	0.01\\
21.96	0.01\\
21.97	0.01\\
21.98	0.01\\
21.99	0.01\\
22	0.01\\
22.01	0.01\\
22.02	0.01\\
22.03	0.01\\
22.04	0.01\\
22.05	0.01\\
22.06	0.01\\
22.07	0.01\\
22.08	0.01\\
22.09	0.01\\
22.1	0.01\\
22.11	0.01\\
22.12	0.01\\
22.13	0.01\\
22.14	0.01\\
22.15	0.01\\
22.16	0.01\\
22.17	0.01\\
22.18	0.01\\
22.19	0.01\\
22.2	0.01\\
22.21	0.01\\
22.22	0.01\\
22.23	0.01\\
22.24	0.01\\
22.25	0.01\\
22.26	0.01\\
22.27	0.01\\
22.28	0.01\\
22.29	0.01\\
22.3	0.01\\
22.31	0.01\\
22.32	0.01\\
22.33	0.01\\
22.34	0.01\\
22.35	0.01\\
22.36	0.01\\
22.37	0.01\\
22.38	0.01\\
22.39	0.01\\
22.4	0.01\\
22.41	0.01\\
22.42	0.01\\
22.43	0.01\\
22.44	0.01\\
22.45	0.01\\
22.46	0.01\\
22.47	0.01\\
22.48	0.01\\
22.49	0.01\\
22.5	0.01\\
22.51	0.01\\
22.52	0.01\\
22.53	0.01\\
22.54	0.01\\
22.55	0.01\\
22.56	0.01\\
22.57	0.01\\
22.58	0.01\\
22.59	0.01\\
22.6	0.01\\
22.61	0.01\\
22.62	0.01\\
22.63	0.01\\
22.64	0.01\\
22.65	0.01\\
22.66	0.01\\
22.67	0.01\\
22.68	0.01\\
22.69	0.01\\
22.7	0.01\\
22.71	0.01\\
22.72	0.01\\
22.73	0.01\\
22.74	0.01\\
22.75	0.01\\
22.76	0.01\\
22.77	0.01\\
22.78	0.01\\
22.79	0.01\\
22.8	0.01\\
22.81	0.01\\
22.82	0.01\\
22.83	0.01\\
22.84	0.01\\
22.85	0.01\\
22.86	0.01\\
22.87	0.01\\
22.88	0.01\\
22.89	0.01\\
22.9	0.01\\
22.91	0.01\\
22.92	0.01\\
22.93	0.01\\
22.94	0.01\\
22.95	0.01\\
22.96	0.01\\
22.97	0.01\\
22.98	0.01\\
22.99	0.01\\
23	0.01\\
23.01	0.01\\
23.02	0.01\\
23.03	0.01\\
23.04	0.01\\
23.05	0.01\\
23.06	0.01\\
23.07	0.01\\
23.08	0.01\\
23.09	0.01\\
23.1	0.01\\
23.11	0.01\\
23.12	0.01\\
23.13	0.01\\
23.14	0.01\\
23.15	0.01\\
23.16	0.01\\
23.17	0.01\\
23.18	0.01\\
23.19	0.01\\
23.2	0.01\\
23.21	0.01\\
23.22	0.01\\
23.23	0.01\\
23.24	0.01\\
23.25	0.01\\
23.26	0.01\\
23.27	0.01\\
23.28	0.01\\
23.29	0.01\\
23.3	0.01\\
23.31	0.01\\
23.32	0.01\\
23.33	0.01\\
23.34	0.01\\
23.35	0.01\\
23.36	0.01\\
23.37	0.01\\
23.38	0.01\\
23.39	0.01\\
23.4	0.01\\
23.41	0.01\\
23.42	0.01\\
23.43	0.01\\
23.44	0.01\\
23.45	0.01\\
23.46	0.01\\
23.47	0.01\\
23.48	0.01\\
23.49	0.01\\
23.5	0.01\\
23.51	0.01\\
23.52	0.01\\
23.53	0.01\\
23.54	0.01\\
23.55	0.01\\
23.56	0.01\\
23.57	0.01\\
23.58	0.01\\
23.59	0.01\\
23.6	0.01\\
23.61	0.01\\
23.62	0.01\\
23.63	0.01\\
23.64	0.01\\
23.65	0.01\\
23.66	0.01\\
23.67	0.01\\
23.68	0.01\\
23.69	0.01\\
23.7	0.01\\
23.71	0.01\\
23.72	0.01\\
23.73	0.01\\
23.74	0.01\\
23.75	0.01\\
23.76	0.01\\
23.77	0.01\\
23.78	0.01\\
23.79	0.01\\
23.8	0.01\\
23.81	0.01\\
23.82	0.01\\
23.83	0.01\\
23.84	0.01\\
23.85	0.01\\
23.86	0.01\\
23.87	0.01\\
23.88	0.01\\
23.89	0.01\\
23.9	0.01\\
23.91	0.01\\
23.92	0.01\\
23.93	0.01\\
23.94	0.01\\
23.95	0.01\\
23.96	0.01\\
23.97	0.01\\
23.98	0.01\\
23.99	0.01\\
24	0.01\\
24.01	0.01\\
24.02	0.01\\
24.03	0.01\\
24.04	0.01\\
24.05	0.01\\
24.06	0.01\\
24.07	0.01\\
24.08	0.01\\
24.09	0.01\\
24.1	0.01\\
24.11	0.01\\
24.12	0.01\\
24.13	0.01\\
24.14	0.01\\
24.15	0.01\\
24.16	0.01\\
24.17	0.01\\
24.18	0.01\\
24.19	0.01\\
24.2	0.01\\
24.21	0.01\\
24.22	0.01\\
24.23	0.01\\
24.24	0.01\\
24.25	0.01\\
24.26	0.01\\
24.27	0.01\\
24.28	0.01\\
24.29	0.01\\
24.3	0.01\\
24.31	0.01\\
24.32	0.01\\
24.33	0.01\\
24.34	0.01\\
24.35	0.01\\
24.36	0.01\\
24.37	0.01\\
24.38	0.01\\
24.39	0.01\\
24.4	0.01\\
24.41	0.01\\
24.42	0.01\\
24.43	0.01\\
24.44	0.01\\
24.45	0.01\\
24.46	0.01\\
24.47	0.01\\
24.48	0.01\\
24.49	0.01\\
24.5	0.01\\
24.51	0.01\\
24.52	0.01\\
24.53	0.01\\
24.54	0.01\\
24.55	0.01\\
24.56	0.01\\
24.57	0.01\\
24.58	0.01\\
24.59	0.01\\
24.6	0.01\\
24.61	0.01\\
24.62	0.01\\
24.63	0.01\\
24.64	0.01\\
24.65	0.01\\
24.66	0.01\\
24.67	0.01\\
24.68	0.01\\
24.69	0.01\\
24.7	0.01\\
24.71	0.01\\
24.72	0.01\\
24.73	0.01\\
24.74	0.01\\
24.75	0.01\\
24.76	0.01\\
24.77	0.01\\
24.78	0.01\\
24.79	0.01\\
24.8	0.01\\
24.81	0.01\\
24.82	0.01\\
24.83	0.01\\
24.84	0.01\\
24.85	0.01\\
24.86	0.01\\
24.87	0.01\\
24.88	0.01\\
24.89	0.01\\
24.9	0.01\\
24.91	0.01\\
24.92	0.01\\
24.93	0.01\\
24.94	0.01\\
24.95	0.01\\
24.96	0.01\\
24.97	0.01\\
24.98	0.01\\
24.99	0.01\\
25	0.01\\
25.01	0.01\\
25.02	0.01\\
25.03	0.01\\
25.04	0.01\\
25.05	0.01\\
25.06	0.01\\
25.07	0.01\\
25.08	0.01\\
25.09	0.01\\
25.1	0.01\\
25.11	0.01\\
25.12	0.01\\
25.13	0.01\\
25.14	0.01\\
25.15	0.01\\
25.16	0.01\\
25.17	0.01\\
25.18	0.01\\
25.19	0.01\\
25.2	0.01\\
25.21	0.01\\
25.22	0.01\\
25.23	0.01\\
25.24	0.01\\
25.25	0.01\\
25.26	0.01\\
25.27	0.01\\
25.28	0.01\\
25.29	0.01\\
25.3	0.01\\
25.31	0.01\\
25.32	0.01\\
25.33	0.01\\
25.34	0.01\\
25.35	0.01\\
25.36	0.01\\
25.37	0.01\\
25.38	0.01\\
25.39	0.01\\
25.4	0.01\\
25.41	0.01\\
25.42	0.01\\
25.43	0.01\\
25.44	0.01\\
25.45	0.01\\
25.46	0.01\\
25.47	0.01\\
25.48	0.01\\
25.49	0.01\\
25.5	0.01\\
25.51	0.01\\
25.52	0.01\\
25.53	0.01\\
25.54	0.01\\
25.55	0.01\\
25.56	0.01\\
25.57	0.01\\
25.58	0.01\\
25.59	0.01\\
25.6	0.01\\
25.61	0.01\\
25.62	0.01\\
25.63	0.01\\
25.64	0.01\\
25.65	0.01\\
25.66	0.01\\
25.67	0.01\\
25.68	0.01\\
25.69	0.01\\
25.7	0.01\\
25.71	0.01\\
25.72	0.01\\
25.73	0.01\\
25.74	0.01\\
25.75	0.01\\
25.76	0.01\\
25.77	0.01\\
25.78	0.01\\
25.79	0.01\\
25.8	0.01\\
25.81	0.01\\
25.82	0.01\\
25.83	0.01\\
25.84	0.01\\
25.85	0.01\\
25.86	0.01\\
25.87	0.01\\
25.88	0.01\\
25.89	0.01\\
25.9	0.01\\
25.91	0.01\\
25.92	0.01\\
25.93	0.01\\
25.94	0.01\\
25.95	0.01\\
25.96	0.01\\
25.97	0.01\\
25.98	0.01\\
25.99	0.01\\
26	0.01\\
26.01	0.01\\
26.02	0.01\\
26.03	0.01\\
26.04	0.01\\
26.05	0.01\\
26.06	0.01\\
26.07	0.01\\
26.08	0.01\\
26.09	0.01\\
26.1	0.01\\
26.11	0.01\\
26.12	0.01\\
26.13	0.01\\
26.14	0.01\\
26.15	0.01\\
26.16	0.01\\
26.17	0.01\\
26.18	0.01\\
26.19	0.01\\
26.2	0.01\\
26.21	0.01\\
26.22	0.01\\
26.23	0.01\\
26.24	0.01\\
26.25	0.01\\
26.26	0.01\\
26.27	0.01\\
26.28	0.01\\
26.29	0.01\\
26.3	0.01\\
26.31	0.01\\
26.32	0.01\\
26.33	0.01\\
26.34	0.01\\
26.35	0.01\\
26.36	0.01\\
26.37	0.01\\
26.38	0.01\\
26.39	0.01\\
26.4	0.01\\
26.41	0.01\\
26.42	0.01\\
26.43	0.01\\
26.44	0.01\\
26.45	0.01\\
26.46	0.01\\
26.47	0.01\\
26.48	0.01\\
26.49	0.01\\
26.5	0.01\\
26.51	0.01\\
26.52	0.01\\
26.53	0.01\\
26.54	0.01\\
26.55	0.01\\
26.56	0.01\\
26.57	0.01\\
26.58	0.01\\
26.59	0.01\\
26.6	0.01\\
26.61	0.01\\
26.62	0.01\\
26.63	0.01\\
26.64	0.01\\
26.65	0.01\\
26.66	0.01\\
26.67	0.01\\
26.68	0.01\\
26.69	0.01\\
26.7	0.01\\
26.71	0.01\\
26.72	0.01\\
26.73	0.01\\
26.74	0.01\\
26.75	0.01\\
26.76	0.01\\
26.77	0.01\\
26.78	0.01\\
26.79	0.01\\
26.8	0.01\\
26.81	0.01\\
26.82	0.01\\
26.83	0.01\\
26.84	0.01\\
26.85	0.01\\
26.86	0.01\\
26.87	0.01\\
26.88	0.01\\
26.89	0.01\\
26.9	0.01\\
26.91	0.01\\
26.92	0.01\\
26.93	0.01\\
26.94	0.01\\
26.95	0.01\\
26.96	0.01\\
26.97	0.01\\
26.98	0.01\\
26.99	0.01\\
27	0.01\\
27.01	0.01\\
27.02	0.01\\
27.03	0.01\\
27.04	0.01\\
27.05	0.01\\
27.06	0.01\\
27.07	0.01\\
27.08	0.01\\
27.09	0.01\\
27.1	0.01\\
27.11	0.01\\
27.12	0.01\\
27.13	0.01\\
27.14	0.01\\
27.15	0.01\\
27.16	0.01\\
27.17	0.01\\
27.18	0.01\\
27.19	0.01\\
27.2	0.01\\
27.21	0.01\\
27.22	0.01\\
27.23	0.01\\
27.24	0.01\\
27.25	0.01\\
27.26	0.01\\
27.27	0.01\\
27.28	0.01\\
27.29	0.01\\
27.3	0.01\\
27.31	0.01\\
27.32	0.01\\
27.33	0.01\\
27.34	0.01\\
27.35	0.01\\
27.36	0.01\\
27.37	0.01\\
27.38	0.01\\
27.39	0.01\\
27.4	0.01\\
27.41	0.01\\
27.42	0.01\\
27.43	0.01\\
27.44	0.01\\
27.45	0.01\\
27.46	0.01\\
27.47	0.01\\
27.48	0.01\\
27.49	0.01\\
27.5	0.01\\
27.51	0.01\\
27.52	0.01\\
27.53	0.01\\
27.54	0.01\\
27.55	0.01\\
27.56	0.01\\
27.57	0.01\\
27.58	0.01\\
27.59	0.01\\
27.6	0.01\\
27.61	0.01\\
27.62	0.01\\
27.63	0.01\\
27.64	0.01\\
27.65	0.01\\
27.66	0.01\\
27.67	0.01\\
27.68	0.01\\
27.69	0.01\\
27.7	0.01\\
27.71	0.01\\
27.72	0.01\\
27.73	0.01\\
27.74	0.01\\
27.75	0.01\\
27.76	0.01\\
27.77	0.01\\
27.78	0.01\\
27.79	0.01\\
27.8	0.01\\
27.81	0.01\\
27.82	0.01\\
27.83	0.01\\
27.84	0.01\\
27.85	0.01\\
27.86	0.01\\
27.87	0.01\\
27.88	0.01\\
27.89	0.01\\
27.9	0.01\\
27.91	0.01\\
27.92	0.01\\
27.93	0.01\\
27.94	0.01\\
27.95	0.01\\
27.96	0.01\\
27.97	0.01\\
27.98	0.01\\
27.99	0.01\\
28	0.01\\
28.01	0.01\\
28.02	0.01\\
28.03	0.01\\
28.04	0.01\\
28.05	0.01\\
28.06	0.01\\
28.07	0.01\\
28.08	0.01\\
28.09	0.01\\
28.1	0.01\\
28.11	0.01\\
28.12	0.01\\
28.13	0.01\\
28.14	0.01\\
28.15	0.01\\
28.16	0.01\\
28.17	0.01\\
28.18	0.01\\
28.19	0.01\\
28.2	0.01\\
28.21	0.01\\
28.22	0.01\\
28.23	0.01\\
28.24	0.01\\
28.25	0.01\\
28.26	0.01\\
28.27	0.01\\
28.28	0.01\\
28.29	0.01\\
28.3	0.01\\
28.31	0.01\\
28.32	0.01\\
28.33	0.01\\
28.34	0.01\\
28.35	0.01\\
28.36	0.01\\
28.37	0.01\\
28.38	0.01\\
28.39	0.01\\
28.4	0.01\\
28.41	0.01\\
28.42	0.01\\
28.43	0.01\\
28.44	0.01\\
28.45	0.01\\
28.46	0.01\\
28.47	0.01\\
28.48	0.01\\
28.49	0.01\\
28.5	0.01\\
28.51	0.01\\
28.52	0.01\\
28.53	0.01\\
28.54	0.01\\
28.55	0.01\\
28.56	0.01\\
28.57	0.01\\
28.58	0.01\\
28.59	0.01\\
28.6	0.01\\
28.61	0.01\\
28.62	0.01\\
28.63	0.01\\
28.64	0.01\\
28.65	0.01\\
28.66	0.01\\
28.67	0.01\\
28.68	0.01\\
28.69	0.01\\
28.7	0.01\\
28.71	0.01\\
28.72	0.01\\
28.73	0.01\\
28.74	0.01\\
28.75	0.01\\
28.76	0.01\\
28.77	0.01\\
28.78	0.01\\
28.79	0.01\\
28.8	0.01\\
28.81	0.01\\
28.82	0.01\\
28.83	0.01\\
28.84	0.01\\
28.85	0.01\\
28.86	0.01\\
28.87	0.01\\
28.88	0.01\\
28.89	0.01\\
28.9	0.01\\
28.91	0.01\\
28.92	0.01\\
28.93	0.01\\
28.94	0.01\\
28.95	0.01\\
28.96	0.01\\
28.97	0.01\\
28.98	0.01\\
28.99	0.01\\
29	0.01\\
29.01	0.01\\
29.02	0.01\\
29.03	0.01\\
29.04	0.01\\
29.05	0.01\\
29.06	0.01\\
29.07	0.01\\
29.08	0.01\\
29.09	0.01\\
29.1	0.01\\
29.11	0.01\\
29.12	0.01\\
29.13	0.01\\
29.14	0.01\\
29.15	0.01\\
29.16	0.01\\
29.17	0.01\\
29.18	0.01\\
29.19	0.01\\
29.2	0.01\\
29.21	0.01\\
29.22	0.01\\
29.23	0.01\\
29.24	0.01\\
29.25	0.01\\
29.26	0.01\\
29.27	0.01\\
29.28	0.01\\
29.29	0.01\\
29.3	0.01\\
29.31	0.01\\
29.32	0.01\\
29.33	0.01\\
29.34	0.01\\
29.35	0.01\\
29.36	0.01\\
29.37	0.01\\
29.38	0.01\\
29.39	0.01\\
29.4	0.01\\
29.41	0.01\\
29.42	0.01\\
29.43	0.01\\
29.44	0.01\\
29.45	0.01\\
29.46	0.01\\
29.47	0.01\\
29.48	0.01\\
29.49	0.01\\
29.5	0.01\\
29.51	0.01\\
29.52	0.01\\
29.53	0.01\\
29.54	0.01\\
29.55	0.01\\
29.56	0.01\\
29.57	0.01\\
29.58	0.01\\
29.59	0.01\\
29.6	0.01\\
29.61	0.01\\
29.62	0.01\\
29.63	0.01\\
29.64	0.01\\
29.65	0.01\\
29.66	0.01\\
29.67	0.01\\
29.68	0.01\\
29.69	0.01\\
29.7	0.01\\
29.71	0.01\\
29.72	0.01\\
29.73	0.01\\
29.74	0.01\\
29.75	0.01\\
29.76	0.01\\
29.77	0.01\\
29.78	0.01\\
29.79	0.01\\
29.8	0.01\\
29.81	0.01\\
29.82	0.01\\
29.83	0.01\\
29.84	0.01\\
29.85	0.01\\
29.86	0.01\\
29.87	0.01\\
29.88	0.01\\
29.89	0.01\\
29.9	0.01\\
29.91	0.01\\
29.92	0.01\\
29.93	0.01\\
29.94	0.01\\
29.95	0.01\\
29.96	0.01\\
29.97	0.01\\
29.98	0.01\\
29.99	0.01\\
30	0.01\\
30.01	0.01\\
30.02	0.01\\
30.03	0.01\\
30.04	0.01\\
30.05	0.01\\
30.06	0.01\\
30.07	0.01\\
30.08	0.01\\
30.09	0.01\\
30.1	0.01\\
30.11	0.01\\
30.12	0.01\\
30.13	0.01\\
30.14	0.01\\
30.15	0.01\\
30.16	0.01\\
30.17	0.01\\
30.18	0.01\\
30.19	0.01\\
30.2	0.01\\
30.21	0.01\\
30.22	0.01\\
30.23	0.01\\
30.24	0.01\\
30.25	0.01\\
30.26	0.01\\
30.27	0.01\\
30.28	0.01\\
30.29	0.01\\
30.3	0.01\\
30.31	0.01\\
30.32	0.01\\
30.33	0.01\\
30.34	0.01\\
30.35	0.01\\
30.36	0.01\\
30.37	0.01\\
30.38	0.01\\
30.39	0.01\\
30.4	0.01\\
30.41	0.01\\
30.42	0.01\\
30.43	0.01\\
30.44	0.01\\
30.45	0.01\\
30.46	0.01\\
30.47	0.01\\
30.48	0.01\\
30.49	0.01\\
30.5	0.01\\
30.51	0.01\\
30.52	0.01\\
30.53	0.01\\
30.54	0.01\\
30.55	0.01\\
30.56	0.01\\
30.57	0.01\\
30.58	0.01\\
30.59	0.01\\
30.6	0.01\\
30.61	0.01\\
30.62	0.01\\
30.63	0.01\\
30.64	0.01\\
30.65	0.01\\
30.66	0.01\\
30.67	0.01\\
30.68	0.01\\
30.69	0.01\\
30.7	0.01\\
30.71	0.01\\
30.72	0.01\\
30.73	0.01\\
30.74	0.01\\
30.75	0.01\\
30.76	0.01\\
30.77	0.01\\
30.78	0.01\\
30.79	0.01\\
30.8	0.01\\
30.81	0.01\\
30.82	0.01\\
30.83	0.01\\
30.84	0.01\\
30.85	0.01\\
30.86	0.01\\
30.87	0.01\\
30.88	0.01\\
30.89	0.01\\
30.9	0.01\\
30.91	0.01\\
30.92	0.01\\
30.93	0.01\\
30.94	0.01\\
30.95	0.01\\
30.96	0.01\\
30.97	0.01\\
30.98	0.01\\
30.99	0.01\\
31	0.01\\
31.01	0.01\\
31.02	0.01\\
31.03	0.01\\
31.04	0.01\\
31.05	0.01\\
31.06	0.01\\
31.07	0.01\\
31.08	0.01\\
31.09	0.01\\
31.1	0.01\\
31.11	0.01\\
31.12	0.01\\
31.13	0.01\\
31.14	0.01\\
31.15	0.01\\
31.16	0.01\\
31.17	0.01\\
31.18	0.01\\
31.19	0.01\\
31.2	0.01\\
31.21	0.01\\
31.22	0.01\\
31.23	0.01\\
31.24	0.01\\
31.25	0.01\\
31.26	0.01\\
31.27	0.01\\
31.28	0.01\\
31.29	0.01\\
31.3	0.01\\
31.31	0.01\\
31.32	0.01\\
31.33	0.01\\
31.34	0.01\\
31.35	0.01\\
31.36	0.01\\
31.37	0.01\\
31.38	0.01\\
31.39	0.01\\
31.4	0.01\\
31.41	0.01\\
31.42	0.01\\
31.43	0.01\\
31.44	0.01\\
31.45	0.01\\
31.46	0.01\\
31.47	0.01\\
31.48	0.01\\
31.49	0.01\\
31.5	0.01\\
31.51	0.01\\
31.52	0.01\\
31.53	0.01\\
31.54	0.01\\
31.55	0.01\\
31.56	0.01\\
31.57	0.01\\
31.58	0.01\\
31.59	0.01\\
31.6	0.01\\
31.61	0.01\\
31.62	0.01\\
31.63	0.01\\
31.64	0.01\\
31.65	0.01\\
31.66	0.01\\
31.67	0.01\\
31.68	0.01\\
31.69	0.01\\
31.7	0.01\\
31.71	0.01\\
31.72	0.01\\
31.73	0.01\\
31.74	0.01\\
31.75	0.01\\
31.76	0.01\\
31.77	0.01\\
31.78	0.01\\
31.79	0.01\\
31.8	0.01\\
31.81	0.01\\
31.82	0.01\\
31.83	0.01\\
31.84	0.01\\
31.85	0.01\\
31.86	0.01\\
31.87	0.01\\
31.88	0.01\\
31.89	0.01\\
31.9	0.01\\
31.91	0.01\\
31.92	0.01\\
31.93	0.01\\
31.94	0.01\\
31.95	0.01\\
31.96	0.01\\
31.97	0.01\\
31.98	0.01\\
31.99	0.01\\
32	0.01\\
32.01	0.01\\
32.02	0.01\\
32.03	0.01\\
32.04	0.01\\
32.05	0.01\\
32.06	0.01\\
32.07	0.01\\
32.08	0.01\\
32.09	0.01\\
32.1	0.01\\
32.11	0.01\\
32.12	0.01\\
32.13	0.01\\
32.14	0.01\\
32.15	0.01\\
32.16	0.01\\
32.17	0.01\\
32.18	0.01\\
32.19	0.01\\
32.2	0.01\\
32.21	0.01\\
32.22	0.01\\
32.23	0.01\\
32.24	0.01\\
32.25	0.01\\
32.26	0.01\\
32.27	0.01\\
32.28	0.01\\
32.29	0.01\\
32.3	0.01\\
32.31	0.01\\
32.32	0.01\\
32.33	0.01\\
32.34	0.01\\
32.35	0.01\\
32.36	0.01\\
32.37	0.01\\
32.38	0.01\\
32.39	0.01\\
32.4	0.01\\
32.41	0.01\\
32.42	0.01\\
32.43	0.01\\
32.44	0.01\\
32.45	0.01\\
32.46	0.01\\
32.47	0.01\\
32.48	0.01\\
32.49	0.01\\
32.5	0.01\\
32.51	0.01\\
32.52	0.01\\
32.53	0.01\\
32.54	0.01\\
32.55	0.01\\
32.56	0.01\\
32.57	0.01\\
32.58	0.01\\
32.59	0.01\\
32.6	0.01\\
32.61	0.01\\
32.62	0.01\\
32.63	0.01\\
32.64	0.01\\
32.65	0.01\\
32.66	0.01\\
32.67	0.01\\
32.68	0.01\\
32.69	0.01\\
32.7	0.01\\
32.71	0.01\\
32.72	0.01\\
32.73	0.01\\
32.74	0.01\\
32.75	0.01\\
32.76	0.01\\
32.77	0.01\\
32.78	0.01\\
32.79	0.01\\
32.8	0.01\\
32.81	0.01\\
32.82	0.01\\
32.83	0.01\\
32.84	0.01\\
32.85	0.01\\
32.86	0.01\\
32.87	0.01\\
32.88	0.01\\
32.89	0.01\\
32.9	0.01\\
32.91	0.01\\
32.92	0.01\\
32.93	0.01\\
32.94	0.01\\
32.95	0.01\\
32.96	0.01\\
32.97	0.01\\
32.98	0.01\\
32.99	0.01\\
33	0.01\\
33.01	0.01\\
33.02	0.01\\
33.03	0.01\\
33.04	0.01\\
33.05	0.01\\
33.06	0.01\\
33.07	0.01\\
33.08	0.01\\
33.09	0.01\\
33.1	0.01\\
33.11	0.01\\
33.12	0.01\\
33.13	0.01\\
33.14	0.01\\
33.15	0.01\\
33.16	0.01\\
33.17	0.01\\
33.18	0.01\\
33.19	0.01\\
33.2	0.01\\
33.21	0.01\\
33.22	0.01\\
33.23	0.01\\
33.24	0.01\\
33.25	0.01\\
33.26	0.01\\
33.27	0.01\\
33.28	0.01\\
33.29	0.01\\
33.3	0.01\\
33.31	0.01\\
33.32	0.01\\
33.33	0.01\\
33.34	0.01\\
33.35	0.01\\
33.36	0.01\\
33.37	0.01\\
33.38	0.01\\
33.39	0.01\\
33.4	0.01\\
33.41	0.01\\
33.42	0.01\\
33.43	0.01\\
33.44	0.01\\
33.45	0.01\\
33.46	0.01\\
33.47	0.01\\
33.48	0.01\\
33.49	0.01\\
33.5	0.01\\
33.51	0.01\\
33.52	0.01\\
33.53	0.01\\
33.54	0.01\\
33.55	0.01\\
33.56	0.01\\
33.57	0.01\\
33.58	0.01\\
33.59	0.01\\
33.6	0.01\\
33.61	0.01\\
33.62	0.01\\
33.63	0.01\\
33.64	0.01\\
33.65	0.01\\
33.66	0.01\\
33.67	0.01\\
33.68	0.01\\
33.69	0.01\\
33.7	0.01\\
33.71	0.01\\
33.72	0.01\\
33.73	0.01\\
33.74	0.01\\
33.75	0.01\\
33.76	0.01\\
33.77	0.01\\
33.78	0.01\\
33.79	0.01\\
33.8	0.01\\
33.81	0.01\\
33.82	0.01\\
33.83	0.01\\
33.84	0.01\\
33.85	0.01\\
33.86	0.01\\
33.87	0.01\\
33.88	0.01\\
33.89	0.01\\
33.9	0.01\\
33.91	0.01\\
33.92	0.01\\
33.93	0.01\\
33.94	0.01\\
33.95	0.01\\
33.96	0.01\\
33.97	0.01\\
33.98	0.01\\
33.99	0.01\\
34	0.01\\
34.01	0.01\\
34.02	0.01\\
34.03	0.01\\
34.04	0.01\\
34.05	0.01\\
34.06	0.01\\
34.07	0.01\\
34.08	0.01\\
34.09	0.01\\
34.1	0.01\\
34.11	0.01\\
34.12	0.01\\
34.13	0.01\\
34.14	0.01\\
34.15	0.01\\
34.16	0.01\\
34.17	0.01\\
34.18	0.01\\
34.19	0.01\\
34.2	0.01\\
34.21	0.01\\
34.22	0.01\\
34.23	0.01\\
34.24	0.01\\
34.25	0.01\\
34.26	0.01\\
34.27	0.01\\
34.28	0.01\\
34.29	0.01\\
34.3	0.01\\
34.31	0.01\\
34.32	0.01\\
34.33	0.01\\
34.34	0.01\\
34.35	0.01\\
34.36	0.01\\
34.37	0.01\\
34.38	0.01\\
34.39	0.01\\
34.4	0.01\\
34.41	0.01\\
34.42	0.01\\
34.43	0.01\\
34.44	0.01\\
34.45	0.01\\
34.46	0.01\\
34.47	0.01\\
34.48	0.01\\
34.49	0.01\\
34.5	0.01\\
34.51	0.01\\
34.52	0.01\\
34.53	0.01\\
34.54	0.01\\
34.55	0.01\\
34.56	0.01\\
34.57	0.01\\
34.58	0.01\\
34.59	0.01\\
34.6	0.01\\
34.61	0.01\\
34.62	0.01\\
34.63	0.01\\
34.64	0.01\\
34.65	0.01\\
34.66	0.01\\
34.67	0.01\\
34.68	0.01\\
34.69	0.01\\
34.7	0.01\\
34.71	0.01\\
34.72	0.01\\
34.73	0.01\\
34.74	0.01\\
34.75	0.01\\
34.76	0.01\\
34.77	0.01\\
34.78	0.01\\
34.79	0.01\\
34.8	0.01\\
34.81	0.01\\
34.82	0.01\\
34.83	0.01\\
34.84	0.01\\
34.85	0.01\\
34.86	0.01\\
34.87	0.01\\
34.88	0.01\\
34.89	0.01\\
34.9	0.01\\
34.91	0.01\\
34.92	0.01\\
34.93	0.01\\
34.94	0.01\\
34.95	0.01\\
34.96	0.01\\
34.97	0.01\\
34.98	0.01\\
34.99	0.01\\
35	0.01\\
35.01	0.01\\
35.02	0.01\\
35.03	0.01\\
35.04	0.01\\
35.05	0.01\\
35.06	0.01\\
35.07	0.01\\
35.08	0.01\\
35.09	0.01\\
35.1	0.01\\
35.11	0.01\\
35.12	0.01\\
35.13	0.01\\
35.14	0.01\\
35.15	0.01\\
35.16	0.01\\
35.17	0.01\\
35.18	0.01\\
35.19	0.01\\
35.2	0.01\\
35.21	0.01\\
35.22	0.01\\
35.23	0.01\\
35.24	0.01\\
35.25	0.01\\
35.26	0.01\\
35.27	0.01\\
35.28	0.01\\
35.29	0.01\\
35.3	0.01\\
35.31	0.01\\
35.32	0.01\\
35.33	0.01\\
35.34	0.01\\
35.35	0.01\\
35.36	0.01\\
35.37	0.01\\
35.38	0.01\\
35.39	0.01\\
35.4	0.01\\
35.41	0.01\\
35.42	0.01\\
35.43	0.01\\
35.44	0.01\\
35.45	0.01\\
35.46	0.01\\
35.47	0.01\\
35.48	0.01\\
35.49	0.01\\
35.5	0.01\\
35.51	0.01\\
35.52	0.01\\
35.53	0.01\\
35.54	0.01\\
35.55	0.01\\
35.56	0.01\\
35.57	0.01\\
35.58	0.01\\
35.59	0.01\\
35.6	0.01\\
35.61	0.01\\
35.62	0.01\\
35.63	0.01\\
35.64	0.01\\
35.65	0.01\\
35.66	0.01\\
35.67	0.01\\
35.68	0.01\\
35.69	0.01\\
35.7	0.01\\
35.71	0.01\\
35.72	0.01\\
35.73	0.01\\
35.74	0.01\\
35.75	0.01\\
35.76	0.01\\
35.77	0.01\\
35.78	0.01\\
35.79	0.01\\
35.8	0.01\\
35.81	0.01\\
35.82	0.01\\
35.83	0.01\\
35.84	0.01\\
35.85	0.01\\
35.86	0.01\\
35.87	0.01\\
35.88	0.01\\
35.89	0.01\\
35.9	0.01\\
35.91	0.01\\
35.92	0.01\\
35.93	0.01\\
35.94	0.01\\
35.95	0.01\\
35.96	0.01\\
35.97	0.01\\
35.98	0.01\\
35.99	0.01\\
36	0.01\\
36.01	0.01\\
36.02	0.01\\
36.03	0.01\\
36.04	0.01\\
36.05	0.01\\
36.06	0.01\\
36.07	0.01\\
36.08	0.01\\
36.09	0.01\\
36.1	0.01\\
36.11	0.01\\
36.12	0.01\\
36.13	0.01\\
36.14	0.01\\
36.15	0.01\\
36.16	0.01\\
36.17	0.01\\
36.18	0.01\\
36.19	0.01\\
36.2	0.01\\
36.21	0.01\\
36.22	0.01\\
36.23	0.01\\
36.24	0.01\\
36.25	0.01\\
36.26	0.01\\
36.27	0.01\\
36.28	0.01\\
36.29	0.01\\
36.3	0.01\\
36.31	0.01\\
36.32	0.01\\
36.33	0.01\\
36.34	0.01\\
36.35	0.01\\
36.36	0.01\\
36.37	0.01\\
36.38	0.01\\
36.39	0.01\\
36.4	0.01\\
36.41	0.01\\
36.42	0.01\\
36.43	0.01\\
36.44	0.01\\
36.45	0.01\\
36.46	0.01\\
36.47	0.01\\
36.48	0.01\\
36.49	0.01\\
36.5	0.01\\
36.51	0.01\\
36.52	0.01\\
36.53	0.01\\
36.54	0.01\\
36.55	0.01\\
36.56	0.01\\
36.57	0.01\\
36.58	0.01\\
36.59	0.01\\
36.6	0.01\\
36.61	0.01\\
36.62	0.01\\
36.63	0.01\\
36.64	0.01\\
36.65	0.01\\
36.66	0.01\\
36.67	0.01\\
36.68	0.01\\
36.69	0.01\\
36.7	0.01\\
36.71	0.01\\
36.72	0.01\\
36.73	0.01\\
36.74	0.01\\
36.75	0.01\\
36.76	0.01\\
36.77	0.01\\
36.78	0.01\\
36.79	0.01\\
36.8	0.01\\
36.81	0.01\\
36.82	0.01\\
36.83	0.01\\
36.84	0.01\\
36.85	0.01\\
36.86	0.01\\
36.87	0.01\\
36.88	0.01\\
36.89	0.01\\
36.9	0.01\\
36.91	0.01\\
36.92	0.01\\
36.93	0.01\\
36.94	0.01\\
36.95	0.01\\
36.96	0.01\\
36.97	0.01\\
36.98	0.01\\
36.99	0.01\\
37	0.01\\
37.01	0.01\\
37.02	0.01\\
37.03	0.01\\
37.04	0.01\\
37.05	0.01\\
37.06	0.01\\
37.07	0.01\\
37.08	0.01\\
37.09	0.01\\
37.1	0.01\\
37.11	0.01\\
37.12	0.01\\
37.13	0.01\\
37.14	0.01\\
37.15	0.01\\
37.16	0.01\\
37.17	0.01\\
37.18	0.01\\
37.19	0.01\\
37.2	0.01\\
37.21	0.01\\
37.22	0.01\\
37.23	0.01\\
37.24	0.01\\
37.25	0.01\\
37.26	0.01\\
37.27	0.01\\
37.28	0.01\\
37.29	0.01\\
37.3	0.01\\
37.31	0.01\\
37.32	0.01\\
37.33	0.01\\
37.34	0.01\\
37.35	0.01\\
37.36	0.01\\
37.37	0.01\\
37.38	0.01\\
37.39	0.01\\
37.4	0.01\\
37.41	0.01\\
37.42	0.01\\
37.43	0.01\\
37.44	0.01\\
37.45	0.01\\
37.46	0.01\\
37.47	0.01\\
37.48	0.01\\
37.49	0.01\\
37.5	0.01\\
37.51	0.01\\
37.52	0.01\\
37.53	0.01\\
37.54	0.01\\
37.55	0.01\\
37.56	0.01\\
37.57	0.01\\
37.58	0.01\\
37.59	0.01\\
37.6	0.01\\
37.61	0.01\\
37.62	0.01\\
37.63	0.01\\
37.64	0.01\\
37.65	0.01\\
37.66	0.01\\
37.67	0.01\\
37.68	0.01\\
37.69	0.01\\
37.7	0.01\\
37.71	0.01\\
37.72	0.01\\
37.73	0.01\\
37.74	0.01\\
37.75	0.01\\
37.76	0.01\\
37.77	0.01\\
37.78	0.01\\
37.79	0.01\\
37.8	0.01\\
37.81	0.01\\
37.82	0.01\\
37.83	0.01\\
37.84	0.01\\
37.85	0.01\\
37.86	0.01\\
37.87	0.01\\
37.88	0.01\\
37.89	0.01\\
37.9	0.01\\
37.91	0.01\\
37.92	0.01\\
37.93	0.01\\
37.94	0.01\\
37.95	0.01\\
37.96	0.01\\
37.97	0.01\\
37.98	0.01\\
37.99	0.01\\
38	0.01\\
38.01	0.01\\
38.02	0.01\\
38.03	0.01\\
38.04	0.01\\
38.05	0.01\\
38.06	0.01\\
38.07	0.01\\
38.08	0.01\\
38.09	0.01\\
38.1	0.01\\
38.11	0.01\\
38.12	0.01\\
38.13	0.01\\
38.14	0.01\\
38.15	0.01\\
38.16	0.01\\
38.17	0.01\\
38.18	0.01\\
38.19	0.01\\
38.2	0.01\\
38.21	0.01\\
38.22	0.01\\
38.23	0.01\\
38.24	0.01\\
38.25	0.01\\
38.26	0.01\\
38.27	0.01\\
38.28	0.01\\
38.29	0.01\\
38.3	0.01\\
38.31	0.01\\
38.32	0.01\\
38.33	0.01\\
38.34	0.01\\
38.35	0.01\\
38.36	0.01\\
38.37	0.01\\
38.38	0.01\\
38.39	0.01\\
38.4	0.01\\
38.41	0.01\\
38.42	0.01\\
38.43	0.01\\
38.44	0.01\\
38.45	0.01\\
38.46	0.01\\
38.47	0.01\\
38.48	0.01\\
38.49	0.01\\
38.5	0.01\\
38.51	0.01\\
38.52	0.01\\
38.53	0.01\\
38.54	0.01\\
38.55	0.01\\
38.56	0.01\\
38.57	0.01\\
38.58	0.01\\
38.59	0.01\\
38.6	0.01\\
38.61	0.01\\
38.62	0.01\\
38.63	0.01\\
38.64	0.01\\
38.65	0.01\\
38.66	0.01\\
38.67	0.01\\
38.68	0.01\\
38.69	0.01\\
38.7	0.01\\
38.71	0.01\\
38.72	0.01\\
38.73	0.01\\
38.74	0.01\\
38.75	0.01\\
38.76	0.01\\
38.77	0.01\\
38.78	0.01\\
38.79	0.01\\
38.8	0.01\\
38.81	0.01\\
38.82	0.01\\
38.83	0.01\\
38.84	0.01\\
38.85	0.01\\
38.86	0.01\\
38.87	0.01\\
38.88	0.01\\
38.89	0.01\\
38.9	0.01\\
38.91	0.01\\
38.92	0.01\\
38.93	0.01\\
38.94	0.01\\
38.95	0.01\\
38.96	0.01\\
38.97	0.01\\
38.98	0.01\\
38.99	0.01\\
39	0.01\\
39.01	0.01\\
39.02	0.01\\
39.03	0.01\\
39.04	0.01\\
39.05	0.01\\
39.06	0.01\\
39.07	0.01\\
39.08	0.01\\
39.09	0.01\\
39.1	0.01\\
39.11	0.01\\
39.12	0.01\\
39.13	0.01\\
39.14	0.01\\
39.15	0.01\\
39.16	0.01\\
39.17	0.01\\
39.18	0.01\\
39.19	0.01\\
39.2	0.01\\
39.21	0.01\\
39.22	0.01\\
39.23	0.01\\
39.24	0.01\\
39.25	0.01\\
39.26	0.01\\
39.27	0.01\\
39.28	0.01\\
39.29	0.01\\
39.3	0.01\\
39.31	0.01\\
39.32	0.01\\
39.33	0.01\\
39.34	0.01\\
39.35	0.01\\
39.36	0.01\\
39.37	0.01\\
39.38	0.01\\
39.39	0.01\\
39.4	0.01\\
39.41	0.01\\
39.42	0.01\\
39.43	0.01\\
39.44	0.01\\
39.45	0.01\\
39.46	0.01\\
39.47	0.01\\
39.48	0.01\\
39.49	0.01\\
39.5	0.01\\
39.51	0.01\\
39.52	0.01\\
39.53	0.01\\
39.54	0.01\\
39.55	0.01\\
39.56	0.01\\
39.57	0.01\\
39.58	0.01\\
39.59	0.01\\
39.6	0.01\\
39.61	0.01\\
39.62	0.01\\
39.63	0.01\\
39.64	0.01\\
39.65	0.01\\
39.66	0.01\\
39.67	0.01\\
39.68	0.01\\
39.69	0.01\\
39.7	0.01\\
39.71	0.01\\
39.72	0.01\\
39.73	0.01\\
39.74	0.01\\
39.75	0.01\\
39.76	0.01\\
39.77	0.01\\
39.78	0.01\\
39.79	0.01\\
39.8	0.01\\
39.81	0.01\\
39.82	0.01\\
39.83	0.01\\
39.84	0.01\\
39.85	0.01\\
39.86	0.01\\
39.87	0.01\\
39.88	0.01\\
39.89	0.01\\
39.9	0.01\\
39.91	0.01\\
39.92	0.01\\
39.93	0.01\\
39.94	0.01\\
39.95	0.01\\
39.96	0.01\\
39.97	0.01\\
39.98	0.01\\
39.99	0.01\\
40	0.01\\
40.01	0.01\\
};
\addplot [color=green,dashed,forget plot]
  table[row sep=crcr]{%
40.01	0.01\\
40.02	0.01\\
40.03	0.01\\
40.04	0.01\\
40.05	0.01\\
40.06	0.01\\
40.07	0.01\\
40.08	0.01\\
40.09	0.01\\
40.1	0.01\\
40.11	0.01\\
40.12	0.01\\
40.13	0.01\\
40.14	0.01\\
40.15	0.01\\
40.16	0.01\\
40.17	0.01\\
40.18	0.01\\
40.19	0.01\\
40.2	0.01\\
40.21	0.01\\
40.22	0.01\\
40.23	0.01\\
40.24	0.01\\
40.25	0.01\\
40.26	0.01\\
40.27	0.01\\
40.28	0.01\\
40.29	0.01\\
40.3	0.01\\
40.31	0.01\\
40.32	0.01\\
40.33	0.01\\
40.34	0.01\\
40.35	0.01\\
40.36	0.01\\
40.37	0.01\\
40.38	0.01\\
40.39	0.01\\
40.4	0.01\\
40.41	0.01\\
40.42	0.01\\
40.43	0.01\\
40.44	0.01\\
40.45	0.01\\
40.46	0.01\\
40.47	0.01\\
40.48	0.01\\
40.49	0.01\\
40.5	0.01\\
40.51	0.01\\
40.52	0.01\\
40.53	0.01\\
40.54	0.01\\
40.55	0.01\\
40.56	0.01\\
40.57	0.01\\
40.58	0.01\\
40.59	0.01\\
40.6	0.01\\
40.61	0.01\\
40.62	0.01\\
40.63	0.01\\
40.64	0.01\\
40.65	0.01\\
40.66	0.01\\
40.67	0.01\\
40.68	0.01\\
40.69	0.01\\
40.7	0.01\\
40.71	0.01\\
40.72	0.01\\
40.73	0.01\\
40.74	0.01\\
40.75	0.01\\
40.76	0.01\\
40.77	0.01\\
40.78	0.01\\
40.79	0.01\\
40.8	0.01\\
40.81	0.01\\
40.82	0.01\\
40.83	0.01\\
40.84	0.01\\
40.85	0.01\\
40.86	0.01\\
40.87	0.01\\
40.88	0.01\\
40.89	0.01\\
40.9	0.01\\
40.91	0.01\\
40.92	0.01\\
40.93	0.01\\
40.94	0.01\\
40.95	0.01\\
40.96	0.01\\
40.97	0.01\\
40.98	0.01\\
40.99	0.01\\
41	0.01\\
41.01	0.01\\
41.02	0.01\\
41.03	0.01\\
41.04	0.01\\
41.05	0.01\\
41.06	0.01\\
41.07	0.01\\
41.08	0.01\\
41.09	0.01\\
41.1	0.01\\
41.11	0.01\\
41.12	0.01\\
41.13	0.01\\
41.14	0.01\\
41.15	0.01\\
41.16	0.01\\
41.17	0.01\\
41.18	0.01\\
41.19	0.01\\
41.2	0.01\\
41.21	0.01\\
41.22	0.01\\
41.23	0.01\\
41.24	0.01\\
41.25	0.01\\
41.26	0.01\\
41.27	0.01\\
41.28	0.01\\
41.29	0.01\\
41.3	0.01\\
41.31	0.01\\
41.32	0.01\\
41.33	0.01\\
41.34	0.01\\
41.35	0.01\\
41.36	0.01\\
41.37	0.01\\
41.38	0.01\\
41.39	0.01\\
41.4	0.01\\
41.41	0.01\\
41.42	0.01\\
41.43	0.01\\
41.44	0.01\\
41.45	0.01\\
41.46	0.01\\
41.47	0.01\\
41.48	0.01\\
41.49	0.01\\
41.5	0.01\\
41.51	0.01\\
41.52	0.01\\
41.53	0.01\\
41.54	0.01\\
41.55	0.01\\
41.56	0.01\\
41.57	0.01\\
41.58	0.01\\
41.59	0.01\\
41.6	0.01\\
41.61	0.01\\
41.62	0.01\\
41.63	0.01\\
41.64	0.01\\
41.65	0.01\\
41.66	0.01\\
41.67	0.01\\
41.68	0.01\\
41.69	0.01\\
41.7	0.01\\
41.71	0.01\\
41.72	0.01\\
41.73	0.01\\
41.74	0.01\\
41.75	0.01\\
41.76	0.01\\
41.77	0.01\\
41.78	0.01\\
41.79	0.01\\
41.8	0.01\\
41.81	0.01\\
41.82	0.01\\
41.83	0.01\\
41.84	0.01\\
41.85	0.01\\
41.86	0.01\\
41.87	0.01\\
41.88	0.01\\
41.89	0.01\\
41.9	0.01\\
41.91	0.01\\
41.92	0.01\\
41.93	0.01\\
41.94	0.01\\
41.95	0.01\\
41.96	0.01\\
41.97	0.01\\
41.98	0.01\\
41.99	0.01\\
42	0.01\\
42.01	0.01\\
42.02	0.01\\
42.03	0.01\\
42.04	0.01\\
42.05	0.01\\
42.06	0.01\\
42.07	0.01\\
42.08	0.01\\
42.09	0.01\\
42.1	0.01\\
42.11	0.01\\
42.12	0.01\\
42.13	0.01\\
42.14	0.01\\
42.15	0.01\\
42.16	0.01\\
42.17	0.01\\
42.18	0.01\\
42.19	0.01\\
42.2	0.01\\
42.21	0.01\\
42.22	0.01\\
42.23	0.01\\
42.24	0.01\\
42.25	0.01\\
42.26	0.01\\
42.27	0.01\\
42.28	0.01\\
42.29	0.01\\
42.3	0.01\\
42.31	0.01\\
42.32	0.01\\
42.33	0.01\\
42.34	0.01\\
42.35	0.01\\
42.36	0.01\\
42.37	0.01\\
42.38	0.01\\
42.39	0.01\\
42.4	0.01\\
42.41	0.01\\
42.42	0.01\\
42.43	0.01\\
42.44	0.01\\
42.45	0.01\\
42.46	0.01\\
42.47	0.01\\
42.48	0.01\\
42.49	0.01\\
42.5	0.01\\
42.51	0.01\\
42.52	0.01\\
42.53	0.01\\
42.54	0.01\\
42.55	0.01\\
42.56	0.01\\
42.57	0.01\\
42.58	0.01\\
42.59	0.01\\
42.6	0.01\\
42.61	0.01\\
42.62	0.01\\
42.63	0.01\\
42.64	0.01\\
42.65	0.01\\
42.66	0.01\\
42.67	0.01\\
42.68	0.01\\
42.69	0.01\\
42.7	0.01\\
42.71	0.01\\
42.72	0.01\\
42.73	0.01\\
42.74	0.01\\
42.75	0.01\\
42.76	0.01\\
42.77	0.01\\
42.78	0.01\\
42.79	0.01\\
42.8	0.01\\
42.81	0.01\\
42.82	0.01\\
42.83	0.01\\
42.84	0.01\\
42.85	0.01\\
42.86	0.01\\
42.87	0.01\\
42.88	0.01\\
42.89	0.01\\
42.9	0.01\\
42.91	0.01\\
42.92	0.01\\
42.93	0.01\\
42.94	0.01\\
42.95	0.01\\
42.96	0.01\\
42.97	0.01\\
42.98	0.01\\
42.99	0.01\\
43	0.01\\
43.01	0.01\\
43.02	0.01\\
43.03	0.01\\
43.04	0.01\\
43.05	0.01\\
43.06	0.01\\
43.07	0.01\\
43.08	0.01\\
43.09	0.01\\
43.1	0.01\\
43.11	0.01\\
43.12	0.01\\
43.13	0.01\\
43.14	0.01\\
43.15	0.01\\
43.16	0.01\\
43.17	0.01\\
43.18	0.01\\
43.19	0.01\\
43.2	0.01\\
43.21	0.01\\
43.22	0.01\\
43.23	0.01\\
43.24	0.01\\
43.25	0.01\\
43.26	0.01\\
43.27	0.01\\
43.28	0.01\\
43.29	0.01\\
43.3	0.01\\
43.31	0.01\\
43.32	0.01\\
43.33	0.01\\
43.34	0.01\\
43.35	0.01\\
43.36	0.01\\
43.37	0.01\\
43.38	0.01\\
43.39	0.01\\
43.4	0.01\\
43.41	0.01\\
43.42	0.01\\
43.43	0.01\\
43.44	0.01\\
43.45	0.01\\
43.46	0.01\\
43.47	0.01\\
43.48	0.01\\
43.49	0.01\\
43.5	0.01\\
43.51	0.01\\
43.52	0.01\\
43.53	0.01\\
43.54	0.01\\
43.55	0.01\\
43.56	0.01\\
43.57	0.01\\
43.58	0.01\\
43.59	0.01\\
43.6	0.01\\
43.61	0.01\\
43.62	0.01\\
43.63	0.01\\
43.64	0.01\\
43.65	0.01\\
43.66	0.01\\
43.67	0.01\\
43.68	0.01\\
43.69	0.01\\
43.7	0.01\\
43.71	0.01\\
43.72	0.01\\
43.73	0.01\\
43.74	0.01\\
43.75	0.01\\
43.76	0.01\\
43.77	0.01\\
43.78	0.01\\
43.79	0.01\\
43.8	0.01\\
43.81	0.01\\
43.82	0.01\\
43.83	0.01\\
43.84	0.01\\
43.85	0.01\\
43.86	0.01\\
43.87	0.01\\
43.88	0.01\\
43.89	0.01\\
43.9	0.01\\
43.91	0.01\\
43.92	0.01\\
43.93	0.01\\
43.94	0.01\\
43.95	0.01\\
43.96	0.01\\
43.97	0.01\\
43.98	0.01\\
43.99	0.01\\
44	0.01\\
44.01	0.01\\
44.02	0.01\\
44.03	0.01\\
44.04	0.01\\
44.05	0.01\\
44.06	0.01\\
44.07	0.01\\
44.08	0.01\\
44.09	0.01\\
44.1	0.01\\
44.11	0.01\\
44.12	0.01\\
44.13	0.01\\
44.14	0.01\\
44.15	0.01\\
44.16	0.01\\
44.17	0.01\\
44.18	0.01\\
44.19	0.01\\
44.2	0.01\\
44.21	0.01\\
44.22	0.01\\
44.23	0.01\\
44.24	0.01\\
44.25	0.01\\
44.26	0.01\\
44.27	0.01\\
44.28	0.01\\
44.29	0.01\\
44.3	0.01\\
44.31	0.01\\
44.32	0.01\\
44.33	0.01\\
44.34	0.01\\
44.35	0.01\\
44.36	0.01\\
44.37	0.01\\
44.38	0.01\\
44.39	0.01\\
44.4	0.01\\
44.41	0.01\\
44.42	0.01\\
44.43	0.01\\
44.44	0.01\\
44.45	0.01\\
44.46	0.01\\
44.47	0.01\\
44.48	0.01\\
44.49	0.01\\
44.5	0.01\\
44.51	0.01\\
44.52	0.01\\
44.53	0.01\\
44.54	0.01\\
44.55	0.01\\
44.56	0.01\\
44.57	0.01\\
44.58	0.01\\
44.59	0.01\\
44.6	0.01\\
44.61	0.01\\
44.62	0.01\\
44.63	0.01\\
44.64	0.01\\
44.65	0.01\\
44.66	0.01\\
44.67	0.01\\
44.68	0.01\\
44.69	0.01\\
44.7	0.01\\
44.71	0.01\\
44.72	0.01\\
44.73	0.01\\
44.74	0.01\\
44.75	0.01\\
44.76	0.01\\
44.77	0.01\\
44.78	0.01\\
44.79	0.01\\
44.8	0.01\\
44.81	0.01\\
44.82	0.01\\
44.83	0.01\\
44.84	0.01\\
44.85	0.01\\
44.86	0.01\\
44.87	0.01\\
44.88	0.01\\
44.89	0.01\\
44.9	0.01\\
44.91	0.01\\
44.92	0.01\\
44.93	0.01\\
44.94	0.01\\
44.95	0.01\\
44.96	0.01\\
44.97	0.01\\
44.98	0.01\\
44.99	0.01\\
45	0.01\\
45.01	0.01\\
45.02	0.01\\
45.03	0.01\\
45.04	0.01\\
45.05	0.01\\
45.06	0.01\\
45.07	0.01\\
45.08	0.01\\
45.09	0.01\\
45.1	0.01\\
45.11	0.01\\
45.12	0.01\\
45.13	0.01\\
45.14	0.01\\
45.15	0.01\\
45.16	0.01\\
45.17	0.01\\
45.18	0.01\\
45.19	0.01\\
45.2	0.01\\
45.21	0.01\\
45.22	0.01\\
45.23	0.01\\
45.24	0.01\\
45.25	0.01\\
45.26	0.01\\
45.27	0.01\\
45.28	0.01\\
45.29	0.01\\
45.3	0.01\\
45.31	0.01\\
45.32	0.01\\
45.33	0.01\\
45.34	0.01\\
45.35	0.01\\
45.36	0.01\\
45.37	0.01\\
45.38	0.01\\
45.39	0.01\\
45.4	0.01\\
45.41	0.01\\
45.42	0.01\\
45.43	0.01\\
45.44	0.01\\
45.45	0.01\\
45.46	0.01\\
45.47	0.01\\
45.48	0.01\\
45.49	0.01\\
45.5	0.01\\
45.51	0.01\\
45.52	0.01\\
45.53	0.01\\
45.54	0.01\\
45.55	0.01\\
45.56	0.01\\
45.57	0.01\\
45.58	0.01\\
45.59	0.01\\
45.6	0.01\\
45.61	0.01\\
45.62	0.01\\
45.63	0.01\\
45.64	0.01\\
45.65	0.01\\
45.66	0.01\\
45.67	0.01\\
45.68	0.01\\
45.69	0.01\\
45.7	0.01\\
45.71	0.01\\
45.72	0.01\\
45.73	0.01\\
45.74	0.01\\
45.75	0.01\\
45.76	0.01\\
45.77	0.01\\
45.78	0.01\\
45.79	0.01\\
45.8	0.01\\
45.81	0.01\\
45.82	0.01\\
45.83	0.01\\
45.84	0.01\\
45.85	0.01\\
45.86	0.01\\
45.87	0.01\\
45.88	0.01\\
45.89	0.01\\
45.9	0.01\\
45.91	0.01\\
45.92	0.01\\
45.93	0.01\\
45.94	0.01\\
45.95	0.01\\
45.96	0.01\\
45.97	0.01\\
45.98	0.01\\
45.99	0.01\\
46	0.01\\
46.01	0.01\\
46.02	0.01\\
46.03	0.01\\
46.04	0.01\\
46.05	0.01\\
46.06	0.01\\
46.07	0.01\\
46.08	0.01\\
46.09	0.01\\
46.1	0.01\\
46.11	0.01\\
46.12	0.01\\
46.13	0.01\\
46.14	0.01\\
46.15	0.01\\
46.16	0.01\\
46.17	0.01\\
46.18	0.01\\
46.19	0.01\\
46.2	0.01\\
46.21	0.01\\
46.22	0.01\\
46.23	0.01\\
46.24	0.01\\
46.25	0.01\\
46.26	0.01\\
46.27	0.01\\
46.28	0.01\\
46.29	0.01\\
46.3	0.01\\
46.31	0.01\\
46.32	0.01\\
46.33	0.01\\
46.34	0.01\\
46.35	0.01\\
46.36	0.01\\
46.37	0.01\\
46.38	0.01\\
46.39	0.01\\
46.4	0.01\\
46.41	0.01\\
46.42	0.01\\
46.43	0.01\\
46.44	0.01\\
46.45	0.01\\
46.46	0.01\\
46.47	0.01\\
46.48	0.01\\
46.49	0.01\\
46.5	0.01\\
46.51	0.01\\
46.52	0.01\\
46.53	0.01\\
46.54	0.01\\
46.55	0.01\\
46.56	0.01\\
46.57	0.01\\
46.58	0.01\\
46.59	0.01\\
46.6	0.01\\
46.61	0.01\\
46.62	0.01\\
46.63	0.01\\
46.64	0.01\\
46.65	0.01\\
46.66	0.01\\
46.67	0.01\\
46.68	0.01\\
46.69	0.01\\
46.7	0.01\\
46.71	0.01\\
46.72	0.01\\
46.73	0.01\\
46.74	0.01\\
46.75	0.01\\
46.76	0.01\\
46.77	0.01\\
46.78	0.01\\
46.79	0.01\\
46.8	0.01\\
46.81	0.01\\
46.82	0.01\\
46.83	0.01\\
46.84	0.01\\
46.85	0.01\\
46.86	0.01\\
46.87	0.01\\
46.88	0.01\\
46.89	0.01\\
46.9	0.01\\
46.91	0.01\\
46.92	0.01\\
46.93	0.01\\
46.94	0.01\\
46.95	0.01\\
46.96	0.01\\
46.97	0.01\\
46.98	0.01\\
46.99	0.01\\
47	0.01\\
47.01	0.01\\
47.02	0.01\\
47.03	0.01\\
47.04	0.01\\
47.05	0.01\\
47.06	0.01\\
47.07	0.01\\
47.08	0.01\\
47.09	0.01\\
47.1	0.01\\
47.11	0.01\\
47.12	0.01\\
47.13	0.01\\
47.14	0.01\\
47.15	0.01\\
47.16	0.01\\
47.17	0.01\\
47.18	0.01\\
47.19	0.01\\
47.2	0.01\\
47.21	0.01\\
47.22	0.01\\
47.23	0.01\\
47.24	0.01\\
47.25	0.01\\
47.26	0.01\\
47.27	0.01\\
47.28	0.01\\
47.29	0.01\\
47.3	0.01\\
47.31	0.01\\
47.32	0.01\\
47.33	0.01\\
47.34	0.01\\
47.35	0.01\\
47.36	0.01\\
47.37	0.01\\
47.38	0.01\\
47.39	0.01\\
47.4	0.01\\
47.41	0.01\\
47.42	0.01\\
47.43	0.01\\
47.44	0.01\\
47.45	0.01\\
47.46	0.01\\
47.47	0.01\\
47.48	0.01\\
47.49	0.01\\
47.5	0.01\\
47.51	0.01\\
47.52	0.01\\
47.53	0.01\\
47.54	0.01\\
47.55	0.01\\
47.56	0.01\\
47.57	0.01\\
47.58	0.01\\
47.59	0.01\\
47.6	0.01\\
47.61	0.01\\
47.62	0.01\\
47.63	0.01\\
47.64	0.01\\
47.65	0.01\\
47.66	0.01\\
47.67	0.01\\
47.68	0.01\\
47.69	0.01\\
47.7	0.01\\
47.71	0.01\\
47.72	0.01\\
47.73	0.01\\
47.74	0.01\\
47.75	0.01\\
47.76	0.01\\
47.77	0.01\\
47.78	0.01\\
47.79	0.01\\
47.8	0.01\\
47.81	0.01\\
47.82	0.01\\
47.83	0.01\\
47.84	0.01\\
47.85	0.01\\
47.86	0.01\\
47.87	0.01\\
47.88	0.01\\
47.89	0.01\\
47.9	0.01\\
47.91	0.01\\
47.92	0.01\\
47.93	0.01\\
47.94	0.01\\
47.95	0.01\\
47.96	0.01\\
47.97	0.01\\
47.98	0.01\\
47.99	0.01\\
48	0.01\\
48.01	0.01\\
48.02	0.01\\
48.03	0.01\\
48.04	0.01\\
48.05	0.01\\
48.06	0.01\\
48.07	0.01\\
48.08	0.01\\
48.09	0.01\\
48.1	0.01\\
48.11	0.01\\
48.12	0.01\\
48.13	0.01\\
48.14	0.01\\
48.15	0.01\\
48.16	0.01\\
48.17	0.01\\
48.18	0.01\\
48.19	0.01\\
48.2	0.01\\
48.21	0.01\\
48.22	0.01\\
48.23	0.01\\
48.24	0.01\\
48.25	0.01\\
48.26	0.01\\
48.27	0.01\\
48.28	0.01\\
48.29	0.01\\
48.3	0.01\\
48.31	0.01\\
48.32	0.01\\
48.33	0.01\\
48.34	0.01\\
48.35	0.01\\
48.36	0.01\\
48.37	0.01\\
48.38	0.01\\
48.39	0.01\\
48.4	0.01\\
48.41	0.01\\
48.42	0.01\\
48.43	0.01\\
48.44	0.01\\
48.45	0.01\\
48.46	0.01\\
48.47	0.01\\
48.48	0.01\\
48.49	0.01\\
48.5	0.01\\
48.51	0.01\\
48.52	0.01\\
48.53	0.01\\
48.54	0.01\\
48.55	0.01\\
48.56	0.01\\
48.57	0.01\\
48.58	0.01\\
48.59	0.01\\
48.6	0.01\\
48.61	0.01\\
48.62	0.01\\
48.63	0.01\\
48.64	0.01\\
48.65	0.01\\
48.66	0.01\\
48.67	0.01\\
48.68	0.01\\
48.69	0.01\\
48.7	0.01\\
48.71	0.01\\
48.72	0.01\\
48.73	0.01\\
48.74	0.01\\
48.75	0.01\\
48.76	0.01\\
48.77	0.01\\
48.78	0.01\\
48.79	0.01\\
48.8	0.01\\
48.81	0.01\\
48.82	0.01\\
48.83	0.01\\
48.84	0.01\\
48.85	0.01\\
48.86	0.01\\
48.87	0.01\\
48.88	0.01\\
48.89	0.01\\
48.9	0.01\\
48.91	0.01\\
48.92	0.01\\
48.93	0.01\\
48.94	0.01\\
48.95	0.01\\
48.96	0.01\\
48.97	0.01\\
48.98	0.01\\
48.99	0.01\\
49	0.01\\
49.01	0.01\\
49.02	0.01\\
49.03	0.01\\
49.04	0.01\\
49.05	0.01\\
49.06	0.01\\
49.07	0.01\\
49.08	0.01\\
49.09	0.01\\
49.1	0.01\\
49.11	0.01\\
49.12	0.01\\
49.13	0.01\\
49.14	0.01\\
49.15	0.01\\
49.16	0.01\\
49.17	0.01\\
49.18	0.01\\
49.19	0.01\\
49.2	0.01\\
49.21	0.01\\
49.22	0.01\\
49.23	0.01\\
49.24	0.01\\
49.25	0.01\\
49.26	0.01\\
49.27	0.01\\
49.28	0.01\\
49.29	0.01\\
49.3	0.01\\
49.31	0.01\\
49.32	0.01\\
49.33	0.01\\
49.34	0.01\\
49.35	0.01\\
49.36	0.01\\
49.37	0.01\\
49.38	0.01\\
49.39	0.01\\
49.4	0.01\\
49.41	0.01\\
49.42	0.01\\
49.43	0.01\\
49.44	0.01\\
49.45	0.01\\
49.46	0.01\\
49.47	0.01\\
49.48	0.01\\
49.49	0.01\\
49.5	0.01\\
49.51	0.01\\
49.52	0.01\\
49.53	0.01\\
49.54	0.01\\
49.55	0.01\\
49.56	0.01\\
49.57	0.01\\
49.58	0.01\\
49.59	0.01\\
49.6	0.01\\
49.61	0.01\\
49.62	0.01\\
49.63	0.01\\
49.64	0.01\\
49.65	0.01\\
49.66	0.01\\
49.67	0.01\\
49.68	0.01\\
49.69	0.01\\
49.7	0.01\\
49.71	0.01\\
49.72	0.01\\
49.73	0.01\\
49.74	0.01\\
49.75	0.01\\
49.76	0.01\\
49.77	0.01\\
49.78	0.01\\
49.79	0.01\\
49.8	0.01\\
49.81	0.01\\
49.82	0.01\\
49.83	0.01\\
49.84	0.01\\
49.85	0.01\\
49.86	0.01\\
49.87	0.01\\
49.88	0.01\\
49.89	0.01\\
49.9	0.01\\
49.91	0.01\\
49.92	0.01\\
49.93	0.01\\
49.94	0.01\\
49.95	0.01\\
49.96	0.01\\
49.97	0.01\\
49.98	0.01\\
49.99	0.01\\
50	0.01\\
50.01	0.01\\
50.02	0.01\\
50.03	0.01\\
50.04	0.01\\
50.05	0.01\\
50.06	0.01\\
50.07	0.01\\
50.08	0.01\\
50.09	0.01\\
50.1	0.01\\
50.11	0.01\\
50.12	0.01\\
50.13	0.01\\
50.14	0.01\\
50.15	0.01\\
50.16	0.01\\
50.17	0.01\\
50.18	0.01\\
50.19	0.01\\
50.2	0.01\\
50.21	0.01\\
50.22	0.01\\
50.23	0.01\\
50.24	0.01\\
50.25	0.01\\
50.26	0.01\\
50.27	0.01\\
50.28	0.01\\
50.29	0.01\\
50.3	0.01\\
50.31	0.01\\
50.32	0.01\\
50.33	0.01\\
50.34	0.01\\
50.35	0.01\\
50.36	0.01\\
50.37	0.01\\
50.38	0.01\\
50.39	0.01\\
50.4	0.01\\
50.41	0.01\\
50.42	0.01\\
50.43	0.01\\
50.44	0.01\\
50.45	0.01\\
50.46	0.01\\
50.47	0.01\\
50.48	0.01\\
50.49	0.01\\
50.5	0.01\\
50.51	0.01\\
50.52	0.01\\
50.53	0.01\\
50.54	0.01\\
50.55	0.01\\
50.56	0.01\\
50.57	0.01\\
50.58	0.01\\
50.59	0.01\\
50.6	0.01\\
50.61	0.01\\
50.62	0.01\\
50.63	0.01\\
50.64	0.01\\
50.65	0.01\\
50.66	0.01\\
50.67	0.01\\
50.68	0.01\\
50.69	0.01\\
50.7	0.01\\
50.71	0.01\\
50.72	0.01\\
50.73	0.01\\
50.74	0.01\\
50.75	0.01\\
50.76	0.01\\
50.77	0.01\\
50.78	0.01\\
50.79	0.01\\
50.8	0.01\\
50.81	0.01\\
50.82	0.01\\
50.83	0.01\\
50.84	0.01\\
50.85	0.01\\
50.86	0.01\\
50.87	0.01\\
50.88	0.01\\
50.89	0.01\\
50.9	0.01\\
50.91	0.01\\
50.92	0.01\\
50.93	0.01\\
50.94	0.01\\
50.95	0.01\\
50.96	0.01\\
50.97	0.01\\
50.98	0.01\\
50.99	0.01\\
51	0.01\\
51.01	0.01\\
51.02	0.01\\
51.03	0.01\\
51.04	0.01\\
51.05	0.01\\
51.06	0.01\\
51.07	0.01\\
51.08	0.01\\
51.09	0.01\\
51.1	0.01\\
51.11	0.01\\
51.12	0.01\\
51.13	0.01\\
51.14	0.01\\
51.15	0.01\\
51.16	0.01\\
51.17	0.01\\
51.18	0.01\\
51.19	0.01\\
51.2	0.01\\
51.21	0.01\\
51.22	0.01\\
51.23	0.01\\
51.24	0.01\\
51.25	0.01\\
51.26	0.01\\
51.27	0.01\\
51.28	0.01\\
51.29	0.01\\
51.3	0.01\\
51.31	0.01\\
51.32	0.01\\
51.33	0.01\\
51.34	0.01\\
51.35	0.01\\
51.36	0.01\\
51.37	0.01\\
51.38	0.01\\
51.39	0.01\\
51.4	0.01\\
51.41	0.01\\
51.42	0.01\\
51.43	0.01\\
51.44	0.01\\
51.45	0.01\\
51.46	0.01\\
51.47	0.01\\
51.48	0.01\\
51.49	0.01\\
51.5	0.01\\
51.51	0.01\\
51.52	0.01\\
51.53	0.01\\
51.54	0.01\\
51.55	0.01\\
51.56	0.01\\
51.57	0.01\\
51.58	0.01\\
51.59	0.01\\
51.6	0.01\\
51.61	0.01\\
51.62	0.01\\
51.63	0.01\\
51.64	0.01\\
51.65	0.01\\
51.66	0.01\\
51.67	0.01\\
51.68	0.01\\
51.69	0.01\\
51.7	0.01\\
51.71	0.01\\
51.72	0.01\\
51.73	0.01\\
51.74	0.01\\
51.75	0.01\\
51.76	0.01\\
51.77	0.01\\
51.78	0.01\\
51.79	0.01\\
51.8	0.01\\
51.81	0.01\\
51.82	0.01\\
51.83	0.01\\
51.84	0.01\\
51.85	0.01\\
51.86	0.01\\
51.87	0.01\\
51.88	0.01\\
51.89	0.01\\
51.9	0.01\\
51.91	0.01\\
51.92	0.01\\
51.93	0.01\\
51.94	0.01\\
51.95	0.01\\
51.96	0.01\\
51.97	0.01\\
51.98	0.01\\
51.99	0.01\\
52	0.01\\
52.01	0.01\\
52.02	0.01\\
52.03	0.01\\
52.04	0.01\\
52.05	0.01\\
52.06	0.01\\
52.07	0.01\\
52.08	0.01\\
52.09	0.01\\
52.1	0.01\\
52.11	0.01\\
52.12	0.01\\
52.13	0.01\\
52.14	0.01\\
52.15	0.01\\
52.16	0.01\\
52.17	0.01\\
52.18	0.01\\
52.19	0.01\\
52.2	0.01\\
52.21	0.01\\
52.22	0.01\\
52.23	0.01\\
52.24	0.01\\
52.25	0.01\\
52.26	0.01\\
52.27	0.01\\
52.28	0.01\\
52.29	0.01\\
52.3	0.01\\
52.31	0.01\\
52.32	0.01\\
52.33	0.01\\
52.34	0.01\\
52.35	0.01\\
52.36	0.01\\
52.37	0.01\\
52.38	0.01\\
52.39	0.01\\
52.4	0.01\\
52.41	0.01\\
52.42	0.01\\
52.43	0.01\\
52.44	0.01\\
52.45	0.01\\
52.46	0.01\\
52.47	0.01\\
52.48	0.01\\
52.49	0.01\\
52.5	0.01\\
52.51	0.01\\
52.52	0.01\\
52.53	0.01\\
52.54	0.01\\
52.55	0.01\\
52.56	0.01\\
52.57	0.01\\
52.58	0.01\\
52.59	0.01\\
52.6	0.01\\
52.61	0.01\\
52.62	0.01\\
52.63	0.01\\
52.64	0.01\\
52.65	0.01\\
52.66	0.01\\
52.67	0.01\\
52.68	0.01\\
52.69	0.01\\
52.7	0.01\\
52.71	0.01\\
52.72	0.01\\
52.73	0.01\\
52.74	0.01\\
52.75	0.01\\
52.76	0.01\\
52.77	0.01\\
52.78	0.01\\
52.79	0.01\\
52.8	0.01\\
52.81	0.01\\
52.82	0.01\\
52.83	0.01\\
52.84	0.01\\
52.85	0.01\\
52.86	0.01\\
52.87	0.01\\
52.88	0.01\\
52.89	0.01\\
52.9	0.01\\
52.91	0.01\\
52.92	0.01\\
52.93	0.01\\
52.94	0.01\\
52.95	0.01\\
52.96	0.01\\
52.97	0.01\\
52.98	0.01\\
52.99	0.01\\
53	0.01\\
53.01	0.01\\
53.02	0.01\\
53.03	0.01\\
53.04	0.01\\
53.05	0.01\\
53.06	0.01\\
53.07	0.01\\
53.08	0.01\\
53.09	0.01\\
53.1	0.01\\
53.11	0.01\\
53.12	0.01\\
53.13	0.01\\
53.14	0.01\\
53.15	0.01\\
53.16	0.01\\
53.17	0.01\\
53.18	0.01\\
53.19	0.01\\
53.2	0.01\\
53.21	0.01\\
53.22	0.01\\
53.23	0.01\\
53.24	0.01\\
53.25	0.01\\
53.26	0.01\\
53.27	0.01\\
53.28	0.01\\
53.29	0.01\\
53.3	0.01\\
53.31	0.01\\
53.32	0.01\\
53.33	0.01\\
53.34	0.01\\
53.35	0.01\\
53.36	0.01\\
53.37	0.01\\
53.38	0.01\\
53.39	0.01\\
53.4	0.01\\
53.41	0.01\\
53.42	0.01\\
53.43	0.01\\
53.44	0.01\\
53.45	0.01\\
53.46	0.01\\
53.47	0.01\\
53.48	0.01\\
53.49	0.01\\
53.5	0.01\\
53.51	0.01\\
53.52	0.01\\
53.53	0.01\\
53.54	0.01\\
53.55	0.01\\
53.56	0.01\\
53.57	0.01\\
53.58	0.01\\
53.59	0.01\\
53.6	0.01\\
53.61	0.01\\
53.62	0.01\\
53.63	0.01\\
53.64	0.01\\
53.65	0.01\\
53.66	0.01\\
53.67	0.01\\
53.68	0.01\\
53.69	0.01\\
53.7	0.01\\
53.71	0.01\\
53.72	0.01\\
53.73	0.01\\
53.74	0.01\\
53.75	0.01\\
53.76	0.01\\
53.77	0.01\\
53.78	0.01\\
53.79	0.01\\
53.8	0.01\\
53.81	0.01\\
53.82	0.01\\
53.83	0.01\\
53.84	0.01\\
53.85	0.01\\
53.86	0.01\\
53.87	0.01\\
53.88	0.01\\
53.89	0.01\\
53.9	0.01\\
53.91	0.01\\
53.92	0.01\\
53.93	0.01\\
53.94	0.01\\
53.95	0.01\\
53.96	0.01\\
53.97	0.01\\
53.98	0.01\\
53.99	0.01\\
54	0.01\\
54.01	0.01\\
54.02	0.01\\
54.03	0.01\\
54.04	0.01\\
54.05	0.01\\
54.06	0.01\\
54.07	0.01\\
54.08	0.01\\
54.09	0.01\\
54.1	0.01\\
54.11	0.01\\
54.12	0.01\\
54.13	0.01\\
54.14	0.01\\
54.15	0.01\\
54.16	0.01\\
54.17	0.01\\
54.18	0.01\\
54.19	0.01\\
54.2	0.01\\
54.21	0.01\\
54.22	0.01\\
54.23	0.01\\
54.24	0.01\\
54.25	0.01\\
54.26	0.01\\
54.27	0.01\\
54.28	0.01\\
54.29	0.01\\
54.3	0.01\\
54.31	0.01\\
54.32	0.01\\
54.33	0.01\\
54.34	0.01\\
54.35	0.01\\
54.36	0.01\\
54.37	0.01\\
54.38	0.01\\
54.39	0.01\\
54.4	0.01\\
54.41	0.01\\
54.42	0.01\\
54.43	0.01\\
54.44	0.01\\
54.45	0.01\\
54.46	0.01\\
54.47	0.01\\
54.48	0.01\\
54.49	0.01\\
54.5	0.01\\
54.51	0.01\\
54.52	0.01\\
54.53	0.01\\
54.54	0.01\\
54.55	0.01\\
54.56	0.01\\
54.57	0.01\\
54.58	0.01\\
54.59	0.01\\
54.6	0.01\\
54.61	0.01\\
54.62	0.01\\
54.63	0.01\\
54.64	0.01\\
54.65	0.01\\
54.66	0.01\\
54.67	0.01\\
54.68	0.01\\
54.69	0.01\\
54.7	0.01\\
54.71	0.01\\
54.72	0.01\\
54.73	0.01\\
54.74	0.01\\
54.75	0.01\\
54.76	0.01\\
54.77	0.01\\
54.78	0.01\\
54.79	0.01\\
54.8	0.01\\
54.81	0.01\\
54.82	0.01\\
54.83	0.01\\
54.84	0.01\\
54.85	0.01\\
54.86	0.01\\
54.87	0.01\\
54.88	0.01\\
54.89	0.01\\
54.9	0.01\\
54.91	0.01\\
54.92	0.01\\
54.93	0.01\\
54.94	0.01\\
54.95	0.01\\
54.96	0.01\\
54.97	0.01\\
54.98	0.01\\
54.99	0.01\\
55	0.01\\
55.01	0.01\\
55.02	0.01\\
55.03	0.01\\
55.04	0.01\\
55.05	0.01\\
55.06	0.01\\
55.07	0.01\\
55.08	0.01\\
55.09	0.01\\
55.1	0.01\\
55.11	0.01\\
55.12	0.01\\
55.13	0.01\\
55.14	0.01\\
55.15	0.01\\
55.16	0.01\\
55.17	0.01\\
55.18	0.01\\
55.19	0.01\\
55.2	0.01\\
55.21	0.01\\
55.22	0.01\\
55.23	0.01\\
55.24	0.01\\
55.25	0.01\\
55.26	0.01\\
55.27	0.01\\
55.28	0.01\\
55.29	0.01\\
55.3	0.01\\
55.31	0.01\\
55.32	0.01\\
55.33	0.01\\
55.34	0.01\\
55.35	0.01\\
55.36	0.01\\
55.37	0.01\\
55.38	0.01\\
55.39	0.01\\
55.4	0.01\\
55.41	0.01\\
55.42	0.01\\
55.43	0.01\\
55.44	0.01\\
55.45	0.01\\
55.46	0.01\\
55.47	0.01\\
55.48	0.01\\
55.49	0.01\\
55.5	0.01\\
55.51	0.01\\
55.52	0.01\\
55.53	0.01\\
55.54	0.01\\
55.55	0.01\\
55.56	0.01\\
55.57	0.01\\
55.58	0.01\\
55.59	0.01\\
55.6	0.01\\
55.61	0.01\\
55.62	0.01\\
55.63	0.01\\
55.64	0.01\\
55.65	0.01\\
55.66	0.01\\
55.67	0.01\\
55.68	0.01\\
55.69	0.01\\
55.7	0.01\\
55.71	0.01\\
55.72	0.01\\
55.73	0.01\\
55.74	0.01\\
55.75	0.01\\
55.76	0.01\\
55.77	0.01\\
55.78	0.01\\
55.79	0.01\\
55.8	0.01\\
55.81	0.01\\
55.82	0.01\\
55.83	0.01\\
55.84	0.01\\
55.85	0.01\\
55.86	0.01\\
55.87	0.01\\
55.88	0.01\\
55.89	0.01\\
55.9	0.01\\
55.91	0.01\\
55.92	0.01\\
55.93	0.01\\
55.94	0.01\\
55.95	0.01\\
55.96	0.01\\
55.97	0.01\\
55.98	0.01\\
55.99	0.01\\
56	0.01\\
56.01	0.01\\
56.02	0.01\\
56.03	0.01\\
56.04	0.01\\
56.05	0.01\\
56.06	0.01\\
56.07	0.01\\
56.08	0.01\\
56.09	0.01\\
56.1	0.01\\
56.11	0.01\\
56.12	0.01\\
56.13	0.01\\
56.14	0.01\\
56.15	0.01\\
56.16	0.01\\
56.17	0.01\\
56.18	0.01\\
56.19	0.01\\
56.2	0.01\\
56.21	0.01\\
56.22	0.01\\
56.23	0.01\\
56.24	0.01\\
56.25	0.01\\
56.26	0.01\\
56.27	0.01\\
56.28	0.01\\
56.29	0.01\\
56.3	0.01\\
56.31	0.01\\
56.32	0.01\\
56.33	0.01\\
56.34	0.01\\
56.35	0.01\\
56.36	0.01\\
56.37	0.01\\
56.38	0.01\\
56.39	0.01\\
56.4	0.01\\
56.41	0.01\\
56.42	0.01\\
56.43	0.01\\
56.44	0.01\\
56.45	0.01\\
56.46	0.01\\
56.47	0.01\\
56.48	0.01\\
56.49	0.01\\
56.5	0.01\\
56.51	0.01\\
56.52	0.01\\
56.53	0.01\\
56.54	0.01\\
56.55	0.01\\
56.56	0.01\\
56.57	0.01\\
56.58	0.01\\
56.59	0.01\\
56.6	0.01\\
56.61	0.01\\
56.62	0.01\\
56.63	0.01\\
56.64	0.01\\
56.65	0.01\\
56.66	0.01\\
56.67	0.01\\
56.68	0.01\\
56.69	0.01\\
56.7	0.01\\
56.71	0.01\\
56.72	0.01\\
56.73	0.01\\
56.74	0.01\\
56.75	0.01\\
56.76	0.01\\
56.77	0.01\\
56.78	0.01\\
56.79	0.01\\
56.8	0.01\\
56.81	0.01\\
56.82	0.01\\
56.83	0.01\\
56.84	0.01\\
56.85	0.01\\
56.86	0.01\\
56.87	0.01\\
56.88	0.01\\
56.89	0.01\\
56.9	0.01\\
56.91	0.01\\
56.92	0.01\\
56.93	0.01\\
56.94	0.01\\
56.95	0.01\\
56.96	0.01\\
56.97	0.01\\
56.98	0.01\\
56.99	0.01\\
57	0.01\\
57.01	0.01\\
57.02	0.01\\
57.03	0.01\\
57.04	0.01\\
57.05	0.01\\
57.06	0.01\\
57.07	0.01\\
57.08	0.01\\
57.09	0.01\\
57.1	0.01\\
57.11	0.01\\
57.12	0.01\\
57.13	0.01\\
57.14	0.01\\
57.15	0.01\\
57.16	0.01\\
57.17	0.01\\
57.18	0.01\\
57.19	0.01\\
57.2	0.01\\
57.21	0.01\\
57.22	0.01\\
57.23	0.01\\
57.24	0.01\\
57.25	0.01\\
57.26	0.01\\
57.27	0.01\\
57.28	0.01\\
57.29	0.01\\
57.3	0.01\\
57.31	0.01\\
57.32	0.01\\
57.33	0.01\\
57.34	0.01\\
57.35	0.01\\
57.36	0.01\\
57.37	0.01\\
57.38	0.01\\
57.39	0.01\\
57.4	0.01\\
57.41	0.01\\
57.42	0.01\\
57.43	0.01\\
57.44	0.01\\
57.45	0.01\\
57.46	0.01\\
57.47	0.01\\
57.48	0.01\\
57.49	0.01\\
57.5	0.01\\
57.51	0.01\\
57.52	0.01\\
57.53	0.01\\
57.54	0.01\\
57.55	0.01\\
57.56	0.01\\
57.57	0.01\\
57.58	0.01\\
57.59	0.01\\
57.6	0.01\\
57.61	0.01\\
57.62	0.01\\
57.63	0.01\\
57.64	0.01\\
57.65	0.01\\
57.66	0.01\\
57.67	0.01\\
57.68	0.01\\
57.69	0.01\\
57.7	0.01\\
57.71	0.01\\
57.72	0.01\\
57.73	0.01\\
57.74	0.01\\
57.75	0.01\\
57.76	0.01\\
57.77	0.01\\
57.78	0.01\\
57.79	0.01\\
57.8	0.01\\
57.81	0.01\\
57.82	0.01\\
57.83	0.01\\
57.84	0.01\\
57.85	0.01\\
57.86	0.01\\
57.87	0.01\\
57.88	0.01\\
57.89	0.01\\
57.9	0.01\\
57.91	0.01\\
57.92	0.01\\
57.93	0.01\\
57.94	0.01\\
57.95	0.01\\
57.96	0.01\\
57.97	0.01\\
57.98	0.01\\
57.99	0.01\\
58	0.01\\
58.01	0.01\\
58.02	0.01\\
58.03	0.01\\
58.04	0.01\\
58.05	0.01\\
58.06	0.01\\
58.07	0.01\\
58.08	0.01\\
58.09	0.01\\
58.1	0.01\\
58.11	0.01\\
58.12	0.01\\
58.13	0.01\\
58.14	0.01\\
58.15	0.01\\
58.16	0.01\\
58.17	0.01\\
58.18	0.01\\
58.19	0.01\\
58.2	0.01\\
58.21	0.01\\
58.22	0.01\\
58.23	0.01\\
58.24	0.01\\
58.25	0.01\\
58.26	0.01\\
58.27	0.01\\
58.28	0.01\\
58.29	0.01\\
58.3	0.01\\
58.31	0.01\\
58.32	0.01\\
58.33	0.01\\
58.34	0.01\\
58.35	0.01\\
58.36	0.01\\
58.37	0.01\\
58.38	0.01\\
58.39	0.01\\
58.4	0.01\\
58.41	0.01\\
58.42	0.01\\
58.43	0.01\\
58.44	0.01\\
58.45	0.01\\
58.46	0.01\\
58.47	0.01\\
58.48	0.01\\
58.49	0.01\\
58.5	0.01\\
58.51	0.01\\
58.52	0.01\\
58.53	0.01\\
58.54	0.01\\
58.55	0.01\\
58.56	0.01\\
58.57	0.01\\
58.58	0.01\\
58.59	0.01\\
58.6	0.01\\
58.61	0.01\\
58.62	0.01\\
58.63	0.01\\
58.64	0.01\\
58.65	0.01\\
58.66	0.01\\
58.67	0.01\\
58.68	0.01\\
58.69	0.01\\
58.7	0.01\\
58.71	0.01\\
58.72	0.01\\
58.73	0.01\\
58.74	0.01\\
58.75	0.01\\
58.76	0.01\\
58.77	0.01\\
58.78	0.01\\
58.79	0.01\\
58.8	0.01\\
58.81	0.01\\
58.82	0.01\\
58.83	0.01\\
58.84	0.01\\
58.85	0.01\\
58.86	0.01\\
58.87	0.01\\
58.88	0.01\\
58.89	0.01\\
58.9	0.01\\
58.91	0.01\\
58.92	0.01\\
58.93	0.01\\
58.94	0.01\\
58.95	0.01\\
58.96	0.01\\
58.97	0.01\\
58.98	0.01\\
58.99	0.01\\
59	0.01\\
59.01	0.01\\
59.02	0.01\\
59.03	0.01\\
59.04	0.01\\
59.05	0.01\\
59.06	0.01\\
59.07	0.01\\
59.08	0.01\\
59.09	0.01\\
59.1	0.01\\
59.11	0.01\\
59.12	0.01\\
59.13	0.01\\
59.14	0.01\\
59.15	0.01\\
59.16	0.01\\
59.17	0.01\\
59.18	0.01\\
59.19	0.01\\
59.2	0.01\\
59.21	0.01\\
59.22	0.01\\
59.23	0.01\\
59.24	0.01\\
59.25	0.01\\
59.26	0.01\\
59.27	0.01\\
59.28	0.01\\
59.29	0.01\\
59.3	0.01\\
59.31	0.01\\
59.32	0.01\\
59.33	0.01\\
59.34	0.01\\
59.35	0.01\\
59.36	0.01\\
59.37	0.01\\
59.38	0.01\\
59.39	0.01\\
59.4	0.01\\
59.41	0.01\\
59.42	0.01\\
59.43	0.01\\
59.44	0.01\\
59.45	0.01\\
59.46	0.01\\
59.47	0.01\\
59.48	0.01\\
59.49	0.01\\
59.5	0.01\\
59.51	0.01\\
59.52	0.01\\
59.53	0.01\\
59.54	0.01\\
59.55	0.01\\
59.56	0.01\\
59.57	0.01\\
59.58	0.01\\
59.59	0.01\\
59.6	0.01\\
59.61	0.01\\
59.62	0.01\\
59.63	0.01\\
59.64	0.01\\
59.65	0.01\\
59.66	0.01\\
59.67	0.01\\
59.68	0.01\\
59.69	0.01\\
59.7	0.01\\
59.71	0.01\\
59.72	0.01\\
59.73	0.01\\
59.74	0.01\\
59.75	0.01\\
59.76	0.01\\
59.77	0.01\\
59.78	0.01\\
59.79	0.01\\
59.8	0.01\\
59.81	0.01\\
59.82	0.01\\
59.83	0.01\\
59.84	0.01\\
59.85	0.01\\
59.86	0.01\\
59.87	0.01\\
59.88	0.01\\
59.89	0.01\\
59.9	0.01\\
59.91	0.01\\
59.92	0.01\\
59.93	0.01\\
59.94	0.01\\
59.95	0.01\\
59.96	0.01\\
59.97	0.01\\
59.98	0.01\\
59.99	0.01\\
60	0.01\\
60.01	0.01\\
60.02	0.01\\
60.03	0.01\\
60.04	0.01\\
60.05	0.01\\
60.06	0.01\\
60.07	0.01\\
60.08	0.01\\
60.09	0.01\\
60.1	0.01\\
60.11	0.01\\
60.12	0.01\\
60.13	0.01\\
60.14	0.01\\
60.15	0.01\\
60.16	0.01\\
60.17	0.01\\
60.18	0.01\\
60.19	0.01\\
60.2	0.01\\
60.21	0.01\\
60.22	0.01\\
60.23	0.01\\
60.24	0.01\\
60.25	0.01\\
60.26	0.01\\
60.27	0.01\\
60.28	0.01\\
60.29	0.01\\
60.3	0.01\\
60.31	0.01\\
60.32	0.01\\
60.33	0.01\\
60.34	0.01\\
60.35	0.01\\
60.36	0.01\\
60.37	0.01\\
60.38	0.01\\
60.39	0.01\\
60.4	0.01\\
60.41	0.01\\
60.42	0.01\\
60.43	0.01\\
60.44	0.01\\
60.45	0.01\\
60.46	0.01\\
60.47	0.01\\
60.48	0.01\\
60.49	0.01\\
60.5	0.01\\
60.51	0.01\\
60.52	0.01\\
60.53	0.01\\
60.54	0.01\\
60.55	0.01\\
60.56	0.01\\
60.57	0.01\\
60.58	0.01\\
60.59	0.01\\
60.6	0.01\\
60.61	0.01\\
60.62	0.01\\
60.63	0.01\\
60.64	0.01\\
60.65	0.01\\
60.66	0.01\\
60.67	0.01\\
60.68	0.01\\
60.69	0.01\\
60.7	0.01\\
60.71	0.01\\
60.72	0.01\\
60.73	0.01\\
60.74	0.01\\
60.75	0.01\\
60.76	0.01\\
60.77	0.01\\
60.78	0.01\\
60.79	0.01\\
60.8	0.01\\
60.81	0.01\\
60.82	0.01\\
60.83	0.01\\
60.84	0.01\\
60.85	0.01\\
60.86	0.01\\
60.87	0.01\\
60.88	0.01\\
60.89	0.01\\
60.9	0.01\\
60.91	0.01\\
60.92	0.01\\
60.93	0.01\\
60.94	0.01\\
60.95	0.01\\
60.96	0.01\\
60.97	0.01\\
60.98	0.01\\
60.99	0.01\\
61	0.01\\
61.01	0.01\\
61.02	0.01\\
61.03	0.01\\
61.04	0.01\\
61.05	0.01\\
61.06	0.01\\
61.07	0.01\\
61.08	0.01\\
61.09	0.01\\
61.1	0.01\\
61.11	0.01\\
61.12	0.01\\
61.13	0.01\\
61.14	0.01\\
61.15	0.01\\
61.16	0.01\\
61.17	0.01\\
61.18	0.01\\
61.19	0.01\\
61.2	0.01\\
61.21	0.01\\
61.22	0.01\\
61.23	0.01\\
61.24	0.01\\
61.25	0.01\\
61.26	0.01\\
61.27	0.01\\
61.28	0.01\\
61.29	0.01\\
61.3	0.01\\
61.31	0.01\\
61.32	0.01\\
61.33	0.01\\
61.34	0.01\\
61.35	0.01\\
61.36	0.01\\
61.37	0.01\\
61.38	0.01\\
61.39	0.01\\
61.4	0.01\\
61.41	0.01\\
61.42	0.01\\
61.43	0.01\\
61.44	0.01\\
61.45	0.01\\
61.46	0.01\\
61.47	0.01\\
61.48	0.01\\
61.49	0.01\\
61.5	0.01\\
61.51	0.01\\
61.52	0.01\\
61.53	0.01\\
61.54	0.01\\
61.55	0.01\\
61.56	0.01\\
61.57	0.01\\
61.58	0.01\\
61.59	0.01\\
61.6	0.01\\
61.61	0.01\\
61.62	0.01\\
61.63	0.01\\
61.64	0.01\\
61.65	0.01\\
61.66	0.01\\
61.67	0.01\\
61.68	0.01\\
61.69	0.01\\
61.7	0.01\\
61.71	0.01\\
61.72	0.01\\
61.73	0.01\\
61.74	0.01\\
61.75	0.01\\
61.76	0.01\\
61.77	0.01\\
61.78	0.01\\
61.79	0.01\\
61.8	0.01\\
61.81	0.01\\
61.82	0.01\\
61.83	0.01\\
61.84	0.01\\
61.85	0.01\\
61.86	0.01\\
61.87	0.01\\
61.88	0.01\\
61.89	0.01\\
61.9	0.01\\
61.91	0.01\\
61.92	0.01\\
61.93	0.01\\
61.94	0.01\\
61.95	0.01\\
61.96	0.01\\
61.97	0.01\\
61.98	0.01\\
61.99	0.01\\
62	0.01\\
62.01	0.01\\
62.02	0.01\\
62.03	0.01\\
62.04	0.01\\
62.05	0.01\\
62.06	0.01\\
62.07	0.01\\
62.08	0.01\\
62.09	0.01\\
62.1	0.01\\
62.11	0.01\\
62.12	0.01\\
62.13	0.01\\
62.14	0.01\\
62.15	0.01\\
62.16	0.01\\
62.17	0.01\\
62.18	0.01\\
62.19	0.01\\
62.2	0.01\\
62.21	0.01\\
62.22	0.01\\
62.23	0.01\\
62.24	0.01\\
62.25	0.01\\
62.26	0.01\\
62.27	0.01\\
62.28	0.01\\
62.29	0.01\\
62.3	0.01\\
62.31	0.01\\
62.32	0.01\\
62.33	0.01\\
62.34	0.01\\
62.35	0.01\\
62.36	0.01\\
62.37	0.01\\
62.38	0.01\\
62.39	0.01\\
62.4	0.01\\
62.41	0.01\\
62.42	0.01\\
62.43	0.01\\
62.44	0.01\\
62.45	0.01\\
62.46	0.01\\
62.47	0.01\\
62.48	0.01\\
62.49	0.01\\
62.5	0.01\\
62.51	0.01\\
62.52	0.01\\
62.53	0.01\\
62.54	0.01\\
62.55	0.01\\
62.56	0.01\\
62.57	0.01\\
62.58	0.01\\
62.59	0.01\\
62.6	0.01\\
62.61	0.01\\
62.62	0.01\\
62.63	0.01\\
62.64	0.01\\
62.65	0.01\\
62.66	0.01\\
62.67	0.01\\
62.68	0.01\\
62.69	0.01\\
62.7	0.01\\
62.71	0.01\\
62.72	0.01\\
62.73	0.01\\
62.74	0.01\\
62.75	0.01\\
62.76	0.01\\
62.77	0.01\\
62.78	0.01\\
62.79	0.01\\
62.8	0.01\\
62.81	0.01\\
62.82	0.01\\
62.83	0.01\\
62.84	0.01\\
62.85	0.01\\
62.86	0.01\\
62.87	0.01\\
62.88	0.01\\
62.89	0.01\\
62.9	0.01\\
62.91	0.01\\
62.92	0.01\\
62.93	0.01\\
62.94	0.01\\
62.95	0.01\\
62.96	0.01\\
62.97	0.01\\
62.98	0.01\\
62.99	0.01\\
63	0.01\\
63.01	0.01\\
63.02	0.01\\
63.03	0.01\\
63.04	0.01\\
63.05	0.01\\
63.06	0.01\\
63.07	0.01\\
63.08	0.01\\
63.09	0.01\\
63.1	0.01\\
63.11	0.01\\
63.12	0.01\\
63.13	0.01\\
63.14	0.01\\
63.15	0.01\\
63.16	0.01\\
63.17	0.01\\
63.18	0.01\\
63.19	0.01\\
63.2	0.01\\
63.21	0.01\\
63.22	0.01\\
63.23	0.01\\
63.24	0.01\\
63.25	0.01\\
63.26	0.01\\
63.27	0.01\\
63.28	0.01\\
63.29	0.01\\
63.3	0.01\\
63.31	0.01\\
63.32	0.01\\
63.33	0.01\\
63.34	0.01\\
63.35	0.01\\
63.36	0.01\\
63.37	0.01\\
63.38	0.01\\
63.39	0.01\\
63.4	0.01\\
63.41	0.01\\
63.42	0.01\\
63.43	0.01\\
63.44	0.01\\
63.45	0.01\\
63.46	0.01\\
63.47	0.01\\
63.48	0.01\\
63.49	0.01\\
63.5	0.01\\
63.51	0.01\\
63.52	0.01\\
63.53	0.01\\
63.54	0.01\\
63.55	0.01\\
63.56	0.01\\
63.57	0.01\\
63.58	0.01\\
63.59	0.01\\
63.6	0.01\\
63.61	0.01\\
63.62	0.01\\
63.63	0.01\\
63.64	0.01\\
63.65	0.01\\
63.66	0.01\\
63.67	0.01\\
63.68	0.01\\
63.69	0.01\\
63.7	0.01\\
63.71	0.01\\
63.72	0.01\\
63.73	0.01\\
63.74	0.01\\
63.75	0.01\\
63.76	0.01\\
63.77	0.01\\
63.78	0.01\\
63.79	0.01\\
63.8	0.01\\
63.81	0.01\\
63.82	0.01\\
63.83	0.01\\
63.84	0.01\\
63.85	0.01\\
63.86	0.01\\
63.87	0.01\\
63.88	0.01\\
63.89	0.01\\
63.9	0.01\\
63.91	0.01\\
63.92	0.01\\
63.93	0.01\\
63.94	0.01\\
63.95	0.01\\
63.96	0.01\\
63.97	0.01\\
63.98	0.01\\
63.99	0.01\\
64	0.01\\
64.01	0.01\\
64.02	0.01\\
64.03	0.01\\
64.04	0.01\\
64.05	0.01\\
64.06	0.01\\
64.07	0.01\\
64.08	0.01\\
64.09	0.01\\
64.1	0.01\\
64.11	0.01\\
64.12	0.01\\
64.13	0.01\\
64.14	0.01\\
64.15	0.01\\
64.16	0.01\\
64.17	0.01\\
64.18	0.01\\
64.19	0.01\\
64.2	0.01\\
64.21	0.01\\
64.22	0.01\\
64.23	0.01\\
64.24	0.01\\
64.25	0.01\\
64.26	0.01\\
64.27	0.01\\
64.28	0.01\\
64.29	0.01\\
64.3	0.01\\
64.31	0.01\\
64.32	0.01\\
64.33	0.01\\
64.34	0.01\\
64.35	0.01\\
64.36	0.01\\
64.37	0.01\\
64.38	0.01\\
64.39	0.01\\
64.4	0.01\\
64.41	0.01\\
64.42	0.01\\
64.43	0.01\\
64.44	0.01\\
64.45	0.01\\
64.46	0.01\\
64.47	0.01\\
64.48	0.01\\
64.49	0.01\\
64.5	0.01\\
64.51	0.01\\
64.52	0.01\\
64.53	0.01\\
64.54	0.01\\
64.55	0.01\\
64.56	0.01\\
64.57	0.01\\
64.58	0.01\\
64.59	0.01\\
64.6	0.01\\
64.61	0.01\\
64.62	0.01\\
64.63	0.01\\
64.64	0.01\\
64.65	0.01\\
64.66	0.01\\
64.67	0.01\\
64.68	0.01\\
64.69	0.01\\
64.7	0.01\\
64.71	0.01\\
64.72	0.01\\
64.73	0.01\\
64.74	0.01\\
64.75	0.01\\
64.76	0.01\\
64.77	0.01\\
64.78	0.01\\
64.79	0.01\\
64.8	0.01\\
64.81	0.01\\
64.82	0.01\\
64.83	0.01\\
64.84	0.01\\
64.85	0.01\\
64.86	0.01\\
64.87	0.01\\
64.88	0.01\\
64.89	0.01\\
64.9	0.01\\
64.91	0.01\\
64.92	0.01\\
64.93	0.01\\
64.94	0.01\\
64.95	0.01\\
64.96	0.01\\
64.97	0.01\\
64.98	0.01\\
64.99	0.01\\
65	0.01\\
65.01	0.01\\
65.02	0.01\\
65.03	0.01\\
65.04	0.01\\
65.05	0.01\\
65.06	0.01\\
65.07	0.01\\
65.08	0.01\\
65.09	0.01\\
65.1	0.01\\
65.11	0.01\\
65.12	0.01\\
65.13	0.01\\
65.14	0.01\\
65.15	0.01\\
65.16	0.01\\
65.17	0.01\\
65.18	0.01\\
65.19	0.01\\
65.2	0.01\\
65.21	0.01\\
65.22	0.01\\
65.23	0.01\\
65.24	0.01\\
65.25	0.01\\
65.26	0.01\\
65.27	0.01\\
65.28	0.01\\
65.29	0.01\\
65.3	0.01\\
65.31	0.01\\
65.32	0.01\\
65.33	0.01\\
65.34	0.01\\
65.35	0.01\\
65.36	0.01\\
65.37	0.01\\
65.38	0.01\\
65.39	0.01\\
65.4	0.01\\
65.41	0.01\\
65.42	0.01\\
65.43	0.01\\
65.44	0.01\\
65.45	0.01\\
65.46	0.01\\
65.47	0.01\\
65.48	0.01\\
65.49	0.01\\
65.5	0.01\\
65.51	0.01\\
65.52	0.01\\
65.53	0.01\\
65.54	0.01\\
65.55	0.01\\
65.56	0.01\\
65.57	0.01\\
65.58	0.01\\
65.59	0.01\\
65.6	0.01\\
65.61	0.01\\
65.62	0.01\\
65.63	0.01\\
65.64	0.01\\
65.65	0.01\\
65.66	0.01\\
65.67	0.01\\
65.68	0.01\\
65.69	0.01\\
65.7	0.01\\
65.71	0.01\\
65.72	0.01\\
65.73	0.01\\
65.74	0.01\\
65.75	0.01\\
65.76	0.01\\
65.77	0.01\\
65.78	0.01\\
65.79	0.01\\
65.8	0.01\\
65.81	0.01\\
65.82	0.01\\
65.83	0.01\\
65.84	0.01\\
65.85	0.01\\
65.86	0.01\\
65.87	0.01\\
65.88	0.01\\
65.89	0.01\\
65.9	0.01\\
65.91	0.01\\
65.92	0.01\\
65.93	0.01\\
65.94	0.01\\
65.95	0.01\\
65.96	0.01\\
65.97	0.01\\
65.98	0.01\\
65.99	0.01\\
66	0.01\\
66.01	0.01\\
66.02	0.01\\
66.03	0.01\\
66.04	0.01\\
66.05	0.01\\
66.06	0.01\\
66.07	0.01\\
66.08	0.01\\
66.09	0.01\\
66.1	0.01\\
66.11	0.01\\
66.12	0.01\\
66.13	0.01\\
66.14	0.01\\
66.15	0.01\\
66.16	0.01\\
66.17	0.01\\
66.18	0.01\\
66.19	0.01\\
66.2	0.01\\
66.21	0.01\\
66.22	0.01\\
66.23	0.01\\
66.24	0.01\\
66.25	0.01\\
66.26	0.01\\
66.27	0.01\\
66.28	0.01\\
66.29	0.01\\
66.3	0.01\\
66.31	0.01\\
66.32	0.01\\
66.33	0.01\\
66.34	0.01\\
66.35	0.01\\
66.36	0.01\\
66.37	0.01\\
66.38	0.01\\
66.39	0.01\\
66.4	0.01\\
66.41	0.01\\
66.42	0.01\\
66.43	0.01\\
66.44	0.01\\
66.45	0.01\\
66.46	0.01\\
66.47	0.01\\
66.48	0.01\\
66.49	0.01\\
66.5	0.01\\
66.51	0.01\\
66.52	0.01\\
66.53	0.01\\
66.54	0.01\\
66.55	0.01\\
66.56	0.01\\
66.57	0.01\\
66.58	0.01\\
66.59	0.01\\
66.6	0.01\\
66.61	0.01\\
66.62	0.01\\
66.63	0.01\\
66.64	0.01\\
66.65	0.01\\
66.66	0.01\\
66.67	0.01\\
66.68	0.01\\
66.69	0.01\\
66.7	0.01\\
66.71	0.01\\
66.72	0.01\\
66.73	0.01\\
66.74	0.01\\
66.75	0.01\\
66.76	0.01\\
66.77	0.01\\
66.78	0.01\\
66.79	0.01\\
66.8	0.01\\
66.81	0.01\\
66.82	0.01\\
66.83	0.01\\
66.84	0.01\\
66.85	0.01\\
66.86	0.01\\
66.87	0.01\\
66.88	0.01\\
66.89	0.01\\
66.9	0.01\\
66.91	0.01\\
66.92	0.01\\
66.93	0.01\\
66.94	0.01\\
66.95	0.01\\
66.96	0.01\\
66.97	0.01\\
66.98	0.01\\
66.99	0.01\\
67	0.01\\
67.01	0.01\\
67.02	0.01\\
67.03	0.01\\
67.04	0.01\\
67.05	0.01\\
67.06	0.01\\
67.07	0.01\\
67.08	0.01\\
67.09	0.01\\
67.1	0.01\\
67.11	0.01\\
67.12	0.01\\
67.13	0.01\\
67.14	0.01\\
67.15	0.01\\
67.16	0.01\\
67.17	0.01\\
67.18	0.01\\
67.19	0.01\\
67.2	0.01\\
67.21	0.01\\
67.22	0.01\\
67.23	0.01\\
67.24	0.01\\
67.25	0.01\\
67.26	0.01\\
67.27	0.01\\
67.28	0.01\\
67.29	0.01\\
67.3	0.01\\
67.31	0.01\\
67.32	0.01\\
67.33	0.01\\
67.34	0.01\\
67.35	0.01\\
67.36	0.01\\
67.37	0.01\\
67.38	0.01\\
67.39	0.01\\
67.4	0.01\\
67.41	0.01\\
67.42	0.01\\
67.43	0.01\\
67.44	0.01\\
67.45	0.01\\
67.46	0.01\\
67.47	0.01\\
67.48	0.01\\
67.49	0.01\\
67.5	0.01\\
67.51	0.01\\
67.52	0.01\\
67.53	0.01\\
67.54	0.01\\
67.55	0.01\\
67.56	0.01\\
67.57	0.01\\
67.58	0.01\\
67.59	0.01\\
67.6	0.01\\
67.61	0.01\\
67.62	0.01\\
67.63	0.01\\
67.64	0.01\\
67.65	0.01\\
67.66	0.01\\
67.67	0.01\\
67.68	0.01\\
67.69	0.01\\
67.7	0.01\\
67.71	0.01\\
67.72	0.01\\
67.73	0.01\\
67.74	0.01\\
67.75	0.01\\
67.76	0.01\\
67.77	0.01\\
67.78	0.01\\
67.79	0.01\\
67.8	0.01\\
67.81	0.01\\
67.82	0.01\\
67.83	0.01\\
67.84	0.01\\
67.85	0.01\\
67.86	0.01\\
67.87	0.01\\
67.88	0.01\\
67.89	0.01\\
67.9	0.01\\
67.91	0.01\\
67.92	0.01\\
67.93	0.01\\
67.94	0.01\\
67.95	0.01\\
67.96	0.01\\
67.97	0.01\\
67.98	0.01\\
67.99	0.01\\
68	0.01\\
68.01	0.01\\
68.02	0.01\\
68.03	0.01\\
68.04	0.01\\
68.05	0.01\\
68.06	0.01\\
68.07	0.01\\
68.08	0.01\\
68.09	0.01\\
68.1	0.01\\
68.11	0.01\\
68.12	0.01\\
68.13	0.01\\
68.14	0.01\\
68.15	0.01\\
68.16	0.01\\
68.17	0.01\\
68.18	0.01\\
68.19	0.01\\
68.2	0.01\\
68.21	0.01\\
68.22	0.01\\
68.23	0.01\\
68.24	0.01\\
68.25	0.01\\
68.26	0.01\\
68.27	0.01\\
68.28	0.01\\
68.29	0.01\\
68.3	0.01\\
68.31	0.01\\
68.32	0.01\\
68.33	0.01\\
68.34	0.01\\
68.35	0.01\\
68.36	0.01\\
68.37	0.01\\
68.38	0.01\\
68.39	0.01\\
68.4	0.01\\
68.41	0.01\\
68.42	0.01\\
68.43	0.01\\
68.44	0.01\\
68.45	0.01\\
68.46	0.01\\
68.47	0.01\\
68.48	0.01\\
68.49	0.01\\
68.5	0.01\\
68.51	0.01\\
68.52	0.01\\
68.53	0.01\\
68.54	0.01\\
68.55	0.01\\
68.56	0.01\\
68.57	0.01\\
68.58	0.01\\
68.59	0.01\\
68.6	0.01\\
68.61	0.01\\
68.62	0.01\\
68.63	0.01\\
68.64	0.01\\
68.65	0.01\\
68.66	0.01\\
68.67	0.01\\
68.68	0.01\\
68.69	0.01\\
68.7	0.01\\
68.71	0.01\\
68.72	0.01\\
68.73	0.01\\
68.74	0.01\\
68.75	0.01\\
68.76	0.01\\
68.77	0.01\\
68.78	0.01\\
68.79	0.01\\
68.8	0.01\\
68.81	0.01\\
68.82	0.01\\
68.83	0.01\\
68.84	0.01\\
68.85	0.01\\
68.86	0.01\\
68.87	0.01\\
68.88	0.01\\
68.89	0.01\\
68.9	0.01\\
68.91	0.01\\
68.92	0.01\\
68.93	0.01\\
68.94	0.01\\
68.95	0.01\\
68.96	0.01\\
68.97	0.01\\
68.98	0.01\\
68.99	0.01\\
69	0.01\\
69.01	0.01\\
69.02	0.01\\
69.03	0.01\\
69.04	0.01\\
69.05	0.01\\
69.06	0.01\\
69.07	0.01\\
69.08	0.01\\
69.09	0.01\\
69.1	0.01\\
69.11	0.01\\
69.12	0.01\\
69.13	0.01\\
69.14	0.01\\
69.15	0.01\\
69.16	0.01\\
69.17	0.01\\
69.18	0.01\\
69.19	0.01\\
69.2	0.01\\
69.21	0.01\\
69.22	0.01\\
69.23	0.01\\
69.24	0.01\\
69.25	0.01\\
69.26	0.01\\
69.27	0.01\\
69.28	0.01\\
69.29	0.01\\
69.3	0.01\\
69.31	0.01\\
69.32	0.01\\
69.33	0.01\\
69.34	0.01\\
69.35	0.01\\
69.36	0.01\\
69.37	0.01\\
69.38	0.01\\
69.39	0.01\\
69.4	0.01\\
69.41	0.01\\
69.42	0.01\\
69.43	0.01\\
69.44	0.01\\
69.45	0.01\\
69.46	0.01\\
69.47	0.01\\
69.48	0.01\\
69.49	0.01\\
69.5	0.01\\
69.51	0.01\\
69.52	0.01\\
69.53	0.01\\
69.54	0.01\\
69.55	0.01\\
69.56	0.01\\
69.57	0.01\\
69.58	0.01\\
69.59	0.01\\
69.6	0.01\\
69.61	0.01\\
69.62	0.01\\
69.63	0.01\\
69.64	0.01\\
69.65	0.01\\
69.66	0.01\\
69.67	0.01\\
69.68	0.01\\
69.69	0.01\\
69.7	0.01\\
69.71	0.01\\
69.72	0.01\\
69.73	0.01\\
69.74	0.01\\
69.75	0.01\\
69.76	0.01\\
69.77	0.01\\
69.78	0.01\\
69.79	0.01\\
69.8	0.01\\
69.81	0.01\\
69.82	0.01\\
69.83	0.01\\
69.84	0.01\\
69.85	0.01\\
69.86	0.01\\
69.87	0.01\\
69.88	0.01\\
69.89	0.01\\
69.9	0.01\\
69.91	0.01\\
69.92	0.01\\
69.93	0.01\\
69.94	0.01\\
69.95	0.01\\
69.96	0.01\\
69.97	0.01\\
69.98	0.01\\
69.99	0.01\\
70	0.01\\
70.01	0.01\\
70.02	0.01\\
70.03	0.01\\
70.04	0.01\\
70.05	0.01\\
70.06	0.01\\
70.07	0.01\\
70.08	0.01\\
70.09	0.01\\
70.1	0.01\\
70.11	0.01\\
70.12	0.01\\
70.13	0.01\\
70.14	0.01\\
70.15	0.01\\
70.16	0.01\\
70.17	0.01\\
70.18	0.01\\
70.19	0.01\\
70.2	0.01\\
70.21	0.01\\
70.22	0.01\\
70.23	0.01\\
70.24	0.01\\
70.25	0.01\\
70.26	0.01\\
70.27	0.01\\
70.28	0.01\\
70.29	0.01\\
70.3	0.01\\
70.31	0.01\\
70.32	0.01\\
70.33	0.01\\
70.34	0.01\\
70.35	0.01\\
70.36	0.01\\
70.37	0.01\\
70.38	0.01\\
70.39	0.01\\
70.4	0.01\\
70.41	0.01\\
70.42	0.01\\
70.43	0.01\\
70.44	0.01\\
70.45	0.01\\
70.46	0.01\\
70.47	0.01\\
70.48	0.01\\
70.49	0.01\\
70.5	0.01\\
70.51	0.01\\
70.52	0.01\\
70.53	0.01\\
70.54	0.01\\
70.55	0.01\\
70.56	0.01\\
70.57	0.01\\
70.58	0.01\\
70.59	0.01\\
70.6	0.01\\
70.61	0.01\\
70.62	0.01\\
70.63	0.01\\
70.64	0.01\\
70.65	0.01\\
70.66	0.01\\
70.67	0.01\\
70.68	0.01\\
70.69	0.01\\
70.7	0.01\\
70.71	0.01\\
70.72	0.01\\
70.73	0.01\\
70.74	0.01\\
70.75	0.01\\
70.76	0.01\\
70.77	0.01\\
70.78	0.01\\
70.79	0.01\\
70.8	0.01\\
70.81	0.01\\
70.82	0.01\\
70.83	0.01\\
70.84	0.01\\
70.85	0.01\\
70.86	0.01\\
70.87	0.01\\
70.88	0.01\\
70.89	0.01\\
70.9	0.01\\
70.91	0.01\\
70.92	0.01\\
70.93	0.01\\
70.94	0.01\\
70.95	0.01\\
70.96	0.01\\
70.97	0.01\\
70.98	0.01\\
70.99	0.01\\
71	0.01\\
71.01	0.01\\
71.02	0.01\\
71.03	0.01\\
71.04	0.01\\
71.05	0.01\\
71.06	0.01\\
71.07	0.01\\
71.08	0.01\\
71.09	0.01\\
71.1	0.01\\
71.11	0.01\\
71.12	0.01\\
71.13	0.01\\
71.14	0.01\\
71.15	0.01\\
71.16	0.01\\
71.17	0.01\\
71.18	0.01\\
71.19	0.01\\
71.2	0.01\\
71.21	0.01\\
71.22	0.01\\
71.23	0.01\\
71.24	0.01\\
71.25	0.01\\
71.26	0.01\\
71.27	0.01\\
71.28	0.01\\
71.29	0.01\\
71.3	0.01\\
71.31	0.01\\
71.32	0.01\\
71.33	0.01\\
71.34	0.01\\
71.35	0.01\\
71.36	0.01\\
71.37	0.01\\
71.38	0.01\\
71.39	0.01\\
71.4	0.01\\
71.41	0.01\\
71.42	0.01\\
71.43	0.01\\
71.44	0.01\\
71.45	0.01\\
71.46	0.01\\
71.47	0.01\\
71.48	0.01\\
71.49	0.01\\
71.5	0.01\\
71.51	0.01\\
71.52	0.01\\
71.53	0.01\\
71.54	0.01\\
71.55	0.01\\
71.56	0.01\\
71.57	0.01\\
71.58	0.01\\
71.59	0.01\\
71.6	0.01\\
71.61	0.01\\
71.62	0.01\\
71.63	0.01\\
71.64	0.01\\
71.65	0.01\\
71.66	0.01\\
71.67	0.01\\
71.68	0.01\\
71.69	0.01\\
71.7	0.01\\
71.71	0.01\\
71.72	0.01\\
71.73	0.01\\
71.74	0.01\\
71.75	0.01\\
71.76	0.01\\
71.77	0.01\\
71.78	0.01\\
71.79	0.01\\
71.8	0.01\\
71.81	0.01\\
71.82	0.01\\
71.83	0.01\\
71.84	0.01\\
71.85	0.01\\
71.86	0.01\\
71.87	0.01\\
71.88	0.01\\
71.89	0.01\\
71.9	0.01\\
71.91	0.01\\
71.92	0.01\\
71.93	0.01\\
71.94	0.01\\
71.95	0.01\\
71.96	0.01\\
71.97	0.01\\
71.98	0.01\\
71.99	0.01\\
72	0.01\\
72.01	0.01\\
72.02	0.01\\
72.03	0.01\\
72.04	0.01\\
72.05	0.01\\
72.06	0.01\\
72.07	0.01\\
72.08	0.01\\
72.09	0.01\\
72.1	0.01\\
72.11	0.01\\
72.12	0.01\\
72.13	0.01\\
72.14	0.01\\
72.15	0.01\\
72.16	0.01\\
72.17	0.01\\
72.18	0.01\\
72.19	0.01\\
72.2	0.01\\
72.21	0.01\\
72.22	0.01\\
72.23	0.01\\
72.24	0.01\\
72.25	0.01\\
72.26	0.01\\
72.27	0.01\\
72.28	0.01\\
72.29	0.01\\
72.3	0.01\\
72.31	0.01\\
72.32	0.01\\
72.33	0.01\\
72.34	0.01\\
72.35	0.01\\
72.36	0.01\\
72.37	0.01\\
72.38	0.01\\
72.39	0.01\\
72.4	0.01\\
72.41	0.01\\
72.42	0.01\\
72.43	0.01\\
72.44	0.01\\
72.45	0.01\\
72.46	0.01\\
72.47	0.01\\
72.48	0.01\\
72.49	0.01\\
72.5	0.01\\
72.51	0.01\\
72.52	0.01\\
72.53	0.01\\
72.54	0.01\\
72.55	0.01\\
72.56	0.01\\
72.57	0.01\\
72.58	0.01\\
72.59	0.01\\
72.6	0.01\\
72.61	0.01\\
72.62	0.01\\
72.63	0.01\\
72.64	0.01\\
72.65	0.01\\
72.66	0.01\\
72.67	0.01\\
72.68	0.01\\
72.69	0.01\\
72.7	0.01\\
72.71	0.01\\
72.72	0.01\\
72.73	0.01\\
72.74	0.01\\
72.75	0.01\\
72.76	0.01\\
72.77	0.01\\
72.78	0.01\\
72.79	0.01\\
72.8	0.01\\
72.81	0.01\\
72.82	0.01\\
72.83	0.01\\
72.84	0.01\\
72.85	0.01\\
72.86	0.01\\
72.87	0.01\\
72.88	0.01\\
72.89	0.01\\
72.9	0.01\\
72.91	0.01\\
72.92	0.01\\
72.93	0.01\\
72.94	0.01\\
72.95	0.01\\
72.96	0.01\\
72.97	0.01\\
72.98	0.01\\
72.99	0.01\\
73	0.01\\
73.01	0.01\\
73.02	0.01\\
73.03	0.01\\
73.04	0.01\\
73.05	0.01\\
73.06	0.01\\
73.07	0.01\\
73.08	0.01\\
73.09	0.01\\
73.1	0.01\\
73.11	0.01\\
73.12	0.01\\
73.13	0.01\\
73.14	0.01\\
73.15	0.01\\
73.16	0.01\\
73.17	0.01\\
73.18	0.01\\
73.19	0.01\\
73.2	0.01\\
73.21	0.01\\
73.22	0.01\\
73.23	0.01\\
73.24	0.01\\
73.25	0.01\\
73.26	0.01\\
73.27	0.01\\
73.28	0.01\\
73.29	0.01\\
73.3	0.01\\
73.31	0.01\\
73.32	0.01\\
73.33	0.01\\
73.34	0.01\\
73.35	0.01\\
73.36	0.01\\
73.37	0.01\\
73.38	0.01\\
73.39	0.01\\
73.4	0.01\\
73.41	0.01\\
73.42	0.01\\
73.43	0.01\\
73.44	0.01\\
73.45	0.01\\
73.46	0.01\\
73.47	0.01\\
73.48	0.01\\
73.49	0.01\\
73.5	0.01\\
73.51	0.01\\
73.52	0.01\\
73.53	0.01\\
73.54	0.01\\
73.55	0.01\\
73.56	0.01\\
73.57	0.01\\
73.58	0.01\\
73.59	0.01\\
73.6	0.01\\
73.61	0.01\\
73.62	0.01\\
73.63	0.01\\
73.64	0.01\\
73.65	0.01\\
73.66	0.01\\
73.67	0.01\\
73.68	0.01\\
73.69	0.01\\
73.7	0.01\\
73.71	0.01\\
73.72	0.01\\
73.73	0.01\\
73.74	0.01\\
73.75	0.01\\
73.76	0.01\\
73.77	0.01\\
73.78	0.01\\
73.79	0.01\\
73.8	0.01\\
73.81	0.01\\
73.82	0.01\\
73.83	0.01\\
73.84	0.01\\
73.85	0.01\\
73.86	0.01\\
73.87	0.01\\
73.88	0.01\\
73.89	0.01\\
73.9	0.01\\
73.91	0.01\\
73.92	0.01\\
73.93	0.01\\
73.94	0.01\\
73.95	0.01\\
73.96	0.01\\
73.97	0.01\\
73.98	0.01\\
73.99	0.01\\
74	0.01\\
74.01	0.01\\
74.02	0.01\\
74.03	0.01\\
74.04	0.01\\
74.05	0.01\\
74.06	0.01\\
74.07	0.01\\
74.08	0.01\\
74.09	0.01\\
74.1	0.01\\
74.11	0.01\\
74.12	0.01\\
74.13	0.01\\
74.14	0.01\\
74.15	0.01\\
74.16	0.01\\
74.17	0.01\\
74.18	0.01\\
74.19	0.01\\
74.2	0.01\\
74.21	0.01\\
74.22	0.01\\
74.23	0.01\\
74.24	0.01\\
74.25	0.01\\
74.26	0.01\\
74.27	0.01\\
74.28	0.01\\
74.29	0.01\\
74.3	0.01\\
74.31	0.01\\
74.32	0.01\\
74.33	0.01\\
74.34	0.01\\
74.35	0.01\\
74.36	0.01\\
74.37	0.01\\
74.38	0.01\\
74.39	0.01\\
74.4	0.01\\
74.41	0.01\\
74.42	0.01\\
74.43	0.01\\
74.44	0.01\\
74.45	0.01\\
74.46	0.01\\
74.47	0.01\\
74.48	0.01\\
74.49	0.01\\
74.5	0.01\\
74.51	0.01\\
74.52	0.01\\
74.53	0.01\\
74.54	0.01\\
74.55	0.01\\
74.56	0.01\\
74.57	0.01\\
74.58	0.01\\
74.59	0.01\\
74.6	0.01\\
74.61	0.01\\
74.62	0.01\\
74.63	0.01\\
74.64	0.01\\
74.65	0.01\\
74.66	0.01\\
74.67	0.01\\
74.68	0.01\\
74.69	0.01\\
74.7	0.01\\
74.71	0.01\\
74.72	0.01\\
74.73	0.01\\
74.74	0.01\\
74.75	0.01\\
74.76	0.01\\
74.77	0.01\\
74.78	0.01\\
74.79	0.01\\
74.8	0.01\\
74.81	0.01\\
74.82	0.01\\
74.83	0.01\\
74.84	0.01\\
74.85	0.01\\
74.86	0.01\\
74.87	0.01\\
74.88	0.01\\
74.89	0.01\\
74.9	0.01\\
74.91	0.01\\
74.92	0.01\\
74.93	0.01\\
74.94	0.01\\
74.95	0.01\\
74.96	0.01\\
74.97	0.01\\
74.98	0.01\\
74.99	0.01\\
75	0.01\\
75.01	0.01\\
75.02	0.01\\
75.03	0.01\\
75.04	0.01\\
75.05	0.01\\
75.06	0.01\\
75.07	0.01\\
75.08	0.01\\
75.09	0.01\\
75.1	0.01\\
75.11	0.01\\
75.12	0.01\\
75.13	0.01\\
75.14	0.01\\
75.15	0.01\\
75.16	0.01\\
75.17	0.01\\
75.18	0.01\\
75.19	0.01\\
75.2	0.01\\
75.21	0.01\\
75.22	0.01\\
75.23	0.01\\
75.24	0.01\\
75.25	0.01\\
75.26	0.01\\
75.27	0.01\\
75.28	0.01\\
75.29	0.01\\
75.3	0.01\\
75.31	0.01\\
75.32	0.01\\
75.33	0.01\\
75.34	0.01\\
75.35	0.01\\
75.36	0.01\\
75.37	0.01\\
75.38	0.01\\
75.39	0.01\\
75.4	0.01\\
75.41	0.01\\
75.42	0.01\\
75.43	0.01\\
75.44	0.01\\
75.45	0.01\\
75.46	0.01\\
75.47	0.01\\
75.48	0.01\\
75.49	0.01\\
75.5	0.01\\
75.51	0.01\\
75.52	0.01\\
75.53	0.01\\
75.54	0.01\\
75.55	0.01\\
75.56	0.01\\
75.57	0.01\\
75.58	0.01\\
75.59	0.01\\
75.6	0.01\\
75.61	0.01\\
75.62	0.01\\
75.63	0.01\\
75.64	0.01\\
75.65	0.01\\
75.66	0.01\\
75.67	0.01\\
75.68	0.01\\
75.69	0.01\\
75.7	0.01\\
75.71	0.01\\
75.72	0.01\\
75.73	0.01\\
75.74	0.01\\
75.75	0.01\\
75.76	0.01\\
75.77	0.01\\
75.78	0.01\\
75.79	0.01\\
75.8	0.01\\
75.81	0.01\\
75.82	0.01\\
75.83	0.01\\
75.84	0.01\\
75.85	0.01\\
75.86	0.01\\
75.87	0.01\\
75.88	0.01\\
75.89	0.01\\
75.9	0.01\\
75.91	0.01\\
75.92	0.01\\
75.93	0.01\\
75.94	0.01\\
75.95	0.01\\
75.96	0.01\\
75.97	0.01\\
75.98	0.01\\
75.99	0.01\\
76	0.01\\
76.01	0.01\\
76.02	0.01\\
76.03	0.01\\
76.04	0.01\\
76.05	0.01\\
76.06	0.01\\
76.07	0.01\\
76.08	0.01\\
76.09	0.01\\
76.1	0.01\\
76.11	0.01\\
76.12	0.01\\
76.13	0.01\\
76.14	0.01\\
76.15	0.01\\
76.16	0.01\\
76.17	0.01\\
76.18	0.01\\
76.19	0.01\\
76.2	0.01\\
76.21	0.01\\
76.22	0.01\\
76.23	0.01\\
76.24	0.01\\
76.25	0.01\\
76.26	0.01\\
76.27	0.01\\
76.28	0.01\\
76.29	0.01\\
76.3	0.01\\
76.31	0.01\\
76.32	0.01\\
76.33	0.01\\
76.34	0.01\\
76.35	0.01\\
76.36	0.01\\
76.37	0.01\\
76.38	0.01\\
76.39	0.01\\
76.4	0.01\\
76.41	0.01\\
76.42	0.01\\
76.43	0.01\\
76.44	0.01\\
76.45	0.01\\
76.46	0.01\\
76.47	0.01\\
76.48	0.01\\
76.49	0.01\\
76.5	0.01\\
76.51	0.01\\
76.52	0.01\\
76.53	0.01\\
76.54	0.01\\
76.55	0.01\\
76.56	0.01\\
76.57	0.01\\
76.58	0.01\\
76.59	0.01\\
76.6	0.01\\
76.61	0.01\\
76.62	0.01\\
76.63	0.01\\
76.64	0.01\\
76.65	0.01\\
76.66	0.01\\
76.67	0.01\\
76.68	0.01\\
76.69	0.01\\
76.7	0.01\\
76.71	0.01\\
76.72	0.01\\
76.73	0.01\\
76.74	0.01\\
76.75	0.01\\
76.76	0.01\\
76.77	0.01\\
76.78	0.01\\
76.79	0.01\\
76.8	0.01\\
76.81	0.01\\
76.82	0.01\\
76.83	0.01\\
76.84	0.01\\
76.85	0.01\\
76.86	0.01\\
76.87	0.01\\
76.88	0.01\\
76.89	0.01\\
76.9	0.01\\
76.91	0.01\\
76.92	0.01\\
76.93	0.01\\
76.94	0.01\\
76.95	0.01\\
76.96	0.01\\
76.97	0.01\\
76.98	0.01\\
76.99	0.01\\
77	0.01\\
77.01	0.01\\
77.02	0.01\\
77.03	0.01\\
77.04	0.01\\
77.05	0.01\\
77.06	0.01\\
77.07	0.01\\
77.08	0.01\\
77.09	0.01\\
77.1	0.01\\
77.11	0.01\\
77.12	0.01\\
77.13	0.01\\
77.14	0.01\\
77.15	0.01\\
77.16	0.01\\
77.17	0.01\\
77.18	0.01\\
77.19	0.01\\
77.2	0.01\\
77.21	0.01\\
77.22	0.01\\
77.23	0.01\\
77.24	0.01\\
77.25	0.01\\
77.26	0.01\\
77.27	0.01\\
77.28	0.01\\
77.29	0.01\\
77.3	0.01\\
77.31	0.01\\
77.32	0.01\\
77.33	0.01\\
77.34	0.01\\
77.35	0.01\\
77.36	0.01\\
77.37	0.01\\
77.38	0.01\\
77.39	0.01\\
77.4	0.01\\
77.41	0.01\\
77.42	0.01\\
77.43	0.01\\
77.44	0.01\\
77.45	0.01\\
77.46	0.01\\
77.47	0.01\\
77.48	0.01\\
77.49	0.01\\
77.5	0.01\\
77.51	0.01\\
77.52	0.01\\
77.53	0.01\\
77.54	0.01\\
77.55	0.01\\
77.56	0.01\\
77.57	0.01\\
77.58	0.01\\
77.59	0.01\\
77.6	0.01\\
77.61	0.01\\
77.62	0.01\\
77.63	0.01\\
77.64	0.01\\
77.65	0.01\\
77.66	0.01\\
77.67	0.01\\
77.68	0.01\\
77.69	0.01\\
77.7	0.01\\
77.71	0.01\\
77.72	0.01\\
77.73	0.01\\
77.74	0.01\\
77.75	0.01\\
77.76	0.01\\
77.77	0.01\\
77.78	0.01\\
77.79	0.01\\
77.8	0.01\\
77.81	0.01\\
77.82	0.01\\
77.83	0.01\\
77.84	0.01\\
77.85	0.01\\
77.86	0.01\\
77.87	0.01\\
77.88	0.01\\
77.89	0.01\\
77.9	0.01\\
77.91	0.01\\
77.92	0.01\\
77.93	0.01\\
77.94	0.01\\
77.95	0.01\\
77.96	0.01\\
77.97	0.01\\
77.98	0.01\\
77.99	0.01\\
78	0.01\\
78.01	0.01\\
78.02	0.01\\
78.03	0.01\\
78.04	0.01\\
78.05	0.01\\
78.06	0.01\\
78.07	0.01\\
78.08	0.01\\
78.09	0.01\\
78.1	0.01\\
78.11	0.01\\
78.12	0.01\\
78.13	0.01\\
78.14	0.01\\
78.15	0.01\\
78.16	0.01\\
78.17	0.01\\
78.18	0.01\\
78.19	0.01\\
78.2	0.01\\
78.21	0.01\\
78.22	0.01\\
78.23	0.01\\
78.24	0.01\\
78.25	0.01\\
78.26	0.01\\
78.27	0.01\\
78.28	0.01\\
78.29	0.01\\
78.3	0.01\\
78.31	0.01\\
78.32	0.01\\
78.33	0.01\\
78.34	0.01\\
78.35	0.01\\
78.36	0.01\\
78.37	0.01\\
78.38	0.01\\
78.39	0.01\\
78.4	0.01\\
78.41	0.01\\
78.42	0.01\\
78.43	0.01\\
78.44	0.01\\
78.45	0.01\\
78.46	0.01\\
78.47	0.01\\
78.48	0.01\\
78.49	0.01\\
78.5	0.01\\
78.51	0.01\\
78.52	0.01\\
78.53	0.01\\
78.54	0.01\\
78.55	0.01\\
78.56	0.01\\
78.57	0.01\\
78.58	0.01\\
78.59	0.01\\
78.6	0.01\\
78.61	0.01\\
78.62	0.01\\
78.63	0.01\\
78.64	0.01\\
78.65	0.01\\
78.66	0.01\\
78.67	0.01\\
78.68	0.01\\
78.69	0.01\\
78.7	0.01\\
78.71	0.01\\
78.72	0.01\\
78.73	0.01\\
78.74	0.01\\
78.75	0.01\\
78.76	0.01\\
78.77	0.01\\
78.78	0.01\\
78.79	0.01\\
78.8	0.01\\
78.81	0.01\\
78.82	0.01\\
78.83	0.01\\
78.84	0.01\\
78.85	0.01\\
78.86	0.01\\
78.87	0.01\\
78.88	0.01\\
78.89	0.01\\
78.9	0.01\\
78.91	0.01\\
78.92	0.01\\
78.93	0.01\\
78.94	0.01\\
78.95	0.01\\
78.96	0.01\\
78.97	0.01\\
78.98	0.01\\
78.99	0.01\\
79	0.01\\
79.01	0.01\\
79.02	0.01\\
79.03	0.01\\
79.04	0.01\\
79.05	0.01\\
79.06	0.01\\
79.07	0.01\\
79.08	0.01\\
79.09	0.01\\
79.1	0.01\\
79.11	0.01\\
79.12	0.01\\
79.13	0.01\\
79.14	0.01\\
79.15	0.01\\
79.16	0.01\\
79.17	0.01\\
79.18	0.01\\
79.19	0.01\\
79.2	0.01\\
79.21	0.01\\
79.22	0.01\\
79.23	0.01\\
79.24	0.01\\
79.25	0.01\\
79.26	0.01\\
79.27	0.01\\
79.28	0.01\\
79.29	0.01\\
79.3	0.01\\
79.31	0.01\\
79.32	0.01\\
79.33	0.01\\
79.34	0.01\\
79.35	0.01\\
79.36	0.01\\
79.37	0.01\\
79.38	0.01\\
79.39	0.01\\
79.4	0.01\\
79.41	0.01\\
79.42	0.01\\
79.43	0.01\\
79.44	0.01\\
79.45	0.01\\
79.46	0.01\\
79.47	0.01\\
79.48	0.01\\
79.49	0.01\\
79.5	0.01\\
79.51	0.01\\
79.52	0.01\\
79.53	0.01\\
79.54	0.01\\
79.55	0.01\\
79.56	0.01\\
79.57	0.01\\
79.58	0.01\\
79.59	0.01\\
79.6	0.01\\
79.61	0.01\\
79.62	0.01\\
79.63	0.01\\
79.64	0.01\\
79.65	0.01\\
79.66	0.01\\
79.67	0.01\\
79.68	0.01\\
79.69	0.01\\
79.7	0.01\\
79.71	0.01\\
79.72	0.01\\
79.73	0.01\\
79.74	0.01\\
79.75	0.01\\
79.76	0.01\\
79.77	0.01\\
79.78	0.01\\
79.79	0.01\\
79.8	0.01\\
79.81	0.01\\
79.82	0.01\\
79.83	0.01\\
79.84	0.01\\
79.85	0.01\\
79.86	0.01\\
79.87	0.01\\
79.88	0.01\\
79.89	0.01\\
79.9	0.01\\
79.91	0.01\\
79.92	0.01\\
79.93	0.01\\
79.94	0.01\\
79.95	0.01\\
79.96	0.01\\
79.97	0.01\\
79.98	0.01\\
79.99	0.01\\
80	0.01\\
80.01	0.01\\
};
\addplot [color=green,dashed]
  table[row sep=crcr]{%
80.01	0.01\\
80.02	0.01\\
80.03	0.01\\
80.04	0.01\\
80.05	0.01\\
80.06	0.01\\
80.07	0.01\\
80.08	0.01\\
80.09	0.01\\
80.1	0.01\\
80.11	0.01\\
80.12	0.01\\
80.13	0.01\\
80.14	0.01\\
80.15	0.01\\
80.16	0.01\\
80.17	0.01\\
80.18	0.01\\
80.19	0.01\\
80.2	0.01\\
80.21	0.01\\
80.22	0.01\\
80.23	0.01\\
80.24	0.01\\
80.25	0.01\\
80.26	0.01\\
80.27	0.01\\
80.28	0.01\\
80.29	0.01\\
80.3	0.01\\
80.31	0.01\\
80.32	0.01\\
80.33	0.01\\
80.34	0.01\\
80.35	0.01\\
80.36	0.01\\
80.37	0.01\\
80.38	0.01\\
80.39	0.01\\
80.4	0.01\\
80.41	0.01\\
80.42	0.01\\
80.43	0.01\\
80.44	0.01\\
80.45	0.01\\
80.46	0.01\\
80.47	0.01\\
80.48	0.01\\
80.49	0.01\\
80.5	0.01\\
80.51	0.01\\
80.52	0.01\\
80.53	0.01\\
80.54	0.01\\
80.55	0.01\\
80.56	0.01\\
80.57	0.01\\
80.58	0.01\\
80.59	0.01\\
80.6	0.01\\
80.61	0.01\\
80.62	0.01\\
80.63	0.01\\
80.64	0.01\\
80.65	0.01\\
80.66	0.01\\
80.67	0.01\\
80.68	0.01\\
80.69	0.01\\
80.7	0.01\\
80.71	0.01\\
80.72	0.01\\
80.73	0.01\\
80.74	0.01\\
80.75	0.01\\
80.76	0.01\\
80.77	0.01\\
80.78	0.01\\
80.79	0.01\\
80.8	0.01\\
80.81	0.01\\
80.82	0.01\\
80.83	0.01\\
80.84	0.01\\
80.85	0.01\\
80.86	0.01\\
80.87	0.01\\
80.88	0.01\\
80.89	0.01\\
80.9	0.01\\
80.91	0.01\\
80.92	0.01\\
80.93	0.01\\
80.94	0.01\\
80.95	0.01\\
80.96	0.01\\
80.97	0.01\\
80.98	0.01\\
80.99	0.01\\
81	0.01\\
81.01	0.01\\
81.02	0.01\\
81.03	0.01\\
81.04	0.01\\
81.05	0.01\\
81.06	0.01\\
81.07	0.01\\
81.08	0.01\\
81.09	0.01\\
81.1	0.01\\
81.11	0.01\\
81.12	0.01\\
81.13	0.01\\
81.14	0.01\\
81.15	0.01\\
81.16	0.01\\
81.17	0.01\\
81.18	0.01\\
81.19	0.01\\
81.2	0.01\\
81.21	0.01\\
81.22	0.01\\
81.23	0.01\\
81.24	0.01\\
81.25	0.01\\
81.26	0.01\\
81.27	0.01\\
81.28	0.01\\
81.29	0.01\\
81.3	0.01\\
81.31	0.01\\
81.32	0.01\\
81.33	0.01\\
81.34	0.01\\
81.35	0.01\\
81.36	0.01\\
81.37	0.01\\
81.38	0.01\\
81.39	0.01\\
81.4	0.01\\
81.41	0.01\\
81.42	0.01\\
81.43	0.01\\
81.44	0.01\\
81.45	0.01\\
81.46	0.01\\
81.47	0.01\\
81.48	0.01\\
81.49	0.01\\
81.5	0.01\\
81.51	0.01\\
81.52	0.01\\
81.53	0.01\\
81.54	0.01\\
81.55	0.01\\
81.56	0.01\\
81.57	0.01\\
81.58	0.01\\
81.59	0.01\\
81.6	0.01\\
81.61	0.01\\
81.62	0.01\\
81.63	0.01\\
81.64	0.01\\
81.65	0.01\\
81.66	0.01\\
81.67	0.01\\
81.68	0.01\\
81.69	0.01\\
81.7	0.01\\
81.71	0.01\\
81.72	0.01\\
81.73	0.01\\
81.74	0.01\\
81.75	0.01\\
81.76	0.01\\
81.77	0.01\\
81.78	0.01\\
81.79	0.01\\
81.8	0.01\\
81.81	0.01\\
81.82	0.01\\
81.83	0.01\\
81.84	0.01\\
81.85	0.01\\
81.86	0.01\\
81.87	0.01\\
81.88	0.01\\
81.89	0.01\\
81.9	0.01\\
81.91	0.01\\
81.92	0.01\\
81.93	0.01\\
81.94	0.01\\
81.95	0.01\\
81.96	0.01\\
81.97	0.01\\
81.98	0.01\\
81.99	0.01\\
82	0.01\\
82.01	0.01\\
82.02	0.01\\
82.03	0.01\\
82.04	0.01\\
82.05	0.01\\
82.06	0.01\\
82.07	0.01\\
82.08	0.01\\
82.09	0.01\\
82.1	0.01\\
82.11	0.01\\
82.12	0.01\\
82.13	0.01\\
82.14	0.01\\
82.15	0.01\\
82.16	0.01\\
82.17	0.01\\
82.18	0.01\\
82.19	0.01\\
82.2	0.01\\
82.21	0.01\\
82.22	0.01\\
82.23	0.01\\
82.24	0.01\\
82.25	0.01\\
82.26	0.01\\
82.27	0.01\\
82.28	0.01\\
82.29	0.01\\
82.3	0.01\\
82.31	0.01\\
82.32	0.01\\
82.33	0.01\\
82.34	0.01\\
82.35	0.01\\
82.36	0.01\\
82.37	0.01\\
82.38	0.01\\
82.39	0.01\\
82.4	0.01\\
82.41	0.01\\
82.42	0.01\\
82.43	0.01\\
82.44	0.01\\
82.45	0.01\\
82.46	0.01\\
82.47	0.01\\
82.48	0.01\\
82.49	0.01\\
82.5	0.01\\
82.51	0.01\\
82.52	0.01\\
82.53	0.01\\
82.54	0.01\\
82.55	0.01\\
82.56	0.01\\
82.57	0.01\\
82.58	0.01\\
82.59	0.01\\
82.6	0.01\\
82.61	0.01\\
82.62	0.01\\
82.63	0.01\\
82.64	0.01\\
82.65	0.01\\
82.66	0.01\\
82.67	0.01\\
82.68	0.01\\
82.69	0.01\\
82.7	0.01\\
82.71	0.01\\
82.72	0.01\\
82.73	0.01\\
82.74	0.01\\
82.75	0.01\\
82.76	0.01\\
82.77	0.01\\
82.78	0.01\\
82.79	0.01\\
82.8	0.01\\
82.81	0.01\\
82.82	0.01\\
82.83	0.01\\
82.84	0.01\\
82.85	0.01\\
82.86	0.01\\
82.87	0.01\\
82.88	0.01\\
82.89	0.01\\
82.9	0.01\\
82.91	0.01\\
82.92	0.01\\
82.93	0.01\\
82.94	0.01\\
82.95	0.01\\
82.96	0.01\\
82.97	0.01\\
82.98	0.01\\
82.99	0.01\\
83	0.01\\
83.01	0.01\\
83.02	0.01\\
83.03	0.01\\
83.04	0.01\\
83.05	0.01\\
83.06	0.01\\
83.07	0.01\\
83.08	0.01\\
83.09	0.01\\
83.1	0.01\\
83.11	0.01\\
83.12	0.01\\
83.13	0.01\\
83.14	0.01\\
83.15	0.01\\
83.16	0.01\\
83.17	0.01\\
83.18	0.01\\
83.19	0.01\\
83.2	0.01\\
83.21	0.01\\
83.22	0.01\\
83.23	0.01\\
83.24	0.01\\
83.25	0.01\\
83.26	0.01\\
83.27	0.01\\
83.28	0.01\\
83.29	0.01\\
83.3	0.01\\
83.31	0.01\\
83.32	0.01\\
83.33	0.01\\
83.34	0.01\\
83.35	0.01\\
83.36	0.01\\
83.37	0.01\\
83.38	0.01\\
83.39	0.01\\
83.4	0.01\\
83.41	0.01\\
83.42	0.01\\
83.43	0.01\\
83.44	0.01\\
83.45	0.01\\
83.46	0.01\\
83.47	0.01\\
83.48	0.01\\
83.49	0.01\\
83.5	0.01\\
83.51	0.01\\
83.52	0.01\\
83.53	0.01\\
83.54	0.01\\
83.55	0.01\\
83.56	0.01\\
83.57	0.01\\
83.58	0.01\\
83.59	0.01\\
83.6	0.01\\
83.61	0.01\\
83.62	0.01\\
83.63	0.01\\
83.64	0.01\\
83.65	0.01\\
83.66	0.01\\
83.67	0.01\\
83.68	0.01\\
83.69	0.01\\
83.7	0.01\\
83.71	0.01\\
83.72	0.01\\
83.73	0.01\\
83.74	0.01\\
83.75	0.01\\
83.76	0.01\\
83.77	0.01\\
83.78	0.01\\
83.79	0.01\\
83.8	0.01\\
83.81	0.01\\
83.82	0.01\\
83.83	0.01\\
83.84	0.01\\
83.85	0.01\\
83.86	0.01\\
83.87	0.01\\
83.88	0.01\\
83.89	0.01\\
83.9	0.01\\
83.91	0.01\\
83.92	0.01\\
83.93	0.01\\
83.94	0.01\\
83.95	0.01\\
83.96	0.01\\
83.97	0.01\\
83.98	0.01\\
83.99	0.01\\
84	0.01\\
84.01	0.01\\
84.02	0.01\\
84.03	0.01\\
84.04	0.01\\
84.05	0.01\\
84.06	0.01\\
84.07	0.01\\
84.08	0.01\\
84.09	0.01\\
84.1	0.01\\
84.11	0.01\\
84.12	0.01\\
84.13	0.01\\
84.14	0.01\\
84.15	0.01\\
84.16	0.01\\
84.17	0.01\\
84.18	0.01\\
84.19	0.01\\
84.2	0.01\\
84.21	0.01\\
84.22	0.01\\
84.23	0.01\\
84.24	0.01\\
84.25	0.01\\
84.26	0.01\\
84.27	0.01\\
84.28	0.01\\
84.29	0.01\\
84.3	0.01\\
84.31	0.01\\
84.32	0.01\\
84.33	0.01\\
84.34	0.01\\
84.35	0.01\\
84.36	0.01\\
84.37	0.01\\
84.38	0.01\\
84.39	0.01\\
84.4	0.01\\
84.41	0.01\\
84.42	0.01\\
84.43	0.01\\
84.44	0.01\\
84.45	0.01\\
84.46	0.01\\
84.47	0.01\\
84.48	0.01\\
84.49	0.01\\
84.5	0.01\\
84.51	0.01\\
84.52	0.01\\
84.53	0.01\\
84.54	0.01\\
84.55	0.01\\
84.56	0.01\\
84.57	0.01\\
84.58	0.01\\
84.59	0.01\\
84.6	0.01\\
84.61	0.01\\
84.62	0.01\\
84.63	0.01\\
84.64	0.01\\
84.65	0.01\\
84.66	0.01\\
84.67	0.01\\
84.68	0.01\\
84.69	0.01\\
84.7	0.01\\
84.71	0.01\\
84.72	0.01\\
84.73	0.01\\
84.74	0.01\\
84.75	0.01\\
84.76	0.01\\
84.77	0.01\\
84.78	0.01\\
84.79	0.01\\
84.8	0.01\\
84.81	0.01\\
84.82	0.01\\
84.83	0.01\\
84.84	0.01\\
84.85	0.01\\
84.86	0.01\\
84.87	0.01\\
84.88	0.01\\
84.89	0.01\\
84.9	0.01\\
84.91	0.01\\
84.92	0.01\\
84.93	0.01\\
84.94	0.01\\
84.95	0.01\\
84.96	0.01\\
84.97	0.01\\
84.98	0.01\\
84.99	0.01\\
85	0.01\\
85.01	0.01\\
85.02	0.01\\
85.03	0.01\\
85.04	0.01\\
85.05	0.01\\
85.06	0.01\\
85.07	0.01\\
85.08	0.01\\
85.09	0.01\\
85.1	0.01\\
85.11	0.01\\
85.12	0.01\\
85.13	0.01\\
85.14	0.01\\
85.15	0.01\\
85.16	0.01\\
85.17	0.01\\
85.18	0.01\\
85.19	0.01\\
85.2	0.01\\
85.21	0.01\\
85.22	0.01\\
85.23	0.01\\
85.24	0.01\\
85.25	0.01\\
85.26	0.01\\
85.27	0.01\\
85.28	0.01\\
85.29	0.01\\
85.3	0.01\\
85.31	0.01\\
85.32	0.01\\
85.33	0.01\\
85.34	0.01\\
85.35	0.01\\
85.36	0.01\\
85.37	0.01\\
85.38	0.01\\
85.39	0.01\\
85.4	0.01\\
85.41	0.01\\
85.42	0.01\\
85.43	0.01\\
85.44	0.01\\
85.45	0.01\\
85.46	0.01\\
85.47	0.01\\
85.48	0.01\\
85.49	0.01\\
85.5	0.01\\
85.51	0.01\\
85.52	0.01\\
85.53	0.01\\
85.54	0.01\\
85.55	0.01\\
85.56	0.01\\
85.57	0.01\\
85.58	0.01\\
85.59	0.01\\
85.6	0.01\\
85.61	0.01\\
85.62	0.01\\
85.63	0.01\\
85.64	0.01\\
85.65	0.01\\
85.66	0.01\\
85.67	0.01\\
85.68	0.01\\
85.69	0.01\\
85.7	0.01\\
85.71	0.01\\
85.72	0.01\\
85.73	0.01\\
85.74	0.01\\
85.75	0.01\\
85.76	0.01\\
85.77	0.01\\
85.78	0.01\\
85.79	0.01\\
85.8	0.01\\
85.81	0.01\\
85.82	0.01\\
85.83	0.01\\
85.84	0.01\\
85.85	0.01\\
85.86	0.01\\
85.87	0.01\\
85.88	0.01\\
85.89	0.01\\
85.9	0.01\\
85.91	0.01\\
85.92	0.01\\
85.93	0.01\\
85.94	0.01\\
85.95	0.01\\
85.96	0.01\\
85.97	0.01\\
85.98	0.01\\
85.99	0.01\\
86	0.01\\
86.01	0.01\\
86.02	0.01\\
86.03	0.01\\
86.04	0.01\\
86.05	0.01\\
86.06	0.01\\
86.07	0.01\\
86.08	0.01\\
86.09	0.01\\
86.1	0.01\\
86.11	0.01\\
86.12	0.01\\
86.13	0.01\\
86.14	0.01\\
86.15	0.01\\
86.16	0.01\\
86.17	0.01\\
86.18	0.01\\
86.19	0.01\\
86.2	0.01\\
86.21	0.01\\
86.22	0.01\\
86.23	0.01\\
86.24	0.01\\
86.25	0.01\\
86.26	0.01\\
86.27	0.01\\
86.28	0.01\\
86.29	0.01\\
86.3	0.01\\
86.31	0.01\\
86.32	0.01\\
86.33	0.01\\
86.34	0.01\\
86.35	0.01\\
86.36	0.01\\
86.37	0.01\\
86.38	0.01\\
86.39	0.01\\
86.4	0.01\\
86.41	0.01\\
86.42	0.01\\
86.43	0.01\\
86.44	0.01\\
86.45	0.01\\
86.46	0.01\\
86.47	0.01\\
86.48	0.01\\
86.49	0.01\\
86.5	0.01\\
86.51	0.01\\
86.52	0.01\\
86.53	0.01\\
86.54	0.01\\
86.55	0.01\\
86.56	0.01\\
86.57	0.01\\
86.58	0.01\\
86.59	0.01\\
86.6	0.01\\
86.61	0.01\\
86.62	0.01\\
86.63	0.01\\
86.64	0.01\\
86.65	0.01\\
86.66	0.01\\
86.67	0.01\\
86.68	0.01\\
86.69	0.01\\
86.7	0.01\\
86.71	0.01\\
86.72	0.01\\
86.73	0.01\\
86.74	0.01\\
86.75	0.01\\
86.76	0.01\\
86.77	0.01\\
86.78	0.01\\
86.79	0.01\\
86.8	0.01\\
86.81	0.01\\
86.82	0.01\\
86.83	0.01\\
86.84	0.01\\
86.85	0.01\\
86.86	0.01\\
86.87	0.01\\
86.88	0.01\\
86.89	0.01\\
86.9	0.01\\
86.91	0.01\\
86.92	0.01\\
86.93	0.01\\
86.94	0.01\\
86.95	0.01\\
86.96	0.01\\
86.97	0.01\\
86.98	0.01\\
86.99	0.01\\
87	0.01\\
87.01	0.01\\
87.02	0.01\\
87.03	0.01\\
87.04	0.01\\
87.05	0.01\\
87.06	0.01\\
87.07	0.01\\
87.08	0.01\\
87.09	0.01\\
87.1	0.01\\
87.11	0.01\\
87.12	0.01\\
87.13	0.01\\
87.14	0.01\\
87.15	0.01\\
87.16	0.01\\
87.17	0.01\\
87.18	0.01\\
87.19	0.01\\
87.2	0.01\\
87.21	0.01\\
87.22	0.01\\
87.23	0.01\\
87.24	0.01\\
87.25	0.01\\
87.26	0.01\\
87.27	0.01\\
87.28	0.01\\
87.29	0.01\\
87.3	0.01\\
87.31	0.01\\
87.32	0.01\\
87.33	0.01\\
87.34	0.01\\
87.35	0.01\\
87.36	0.01\\
87.37	0.01\\
87.38	0.01\\
87.39	0.01\\
87.4	0.01\\
87.41	0.01\\
87.42	0.01\\
87.43	0.01\\
87.44	0.01\\
87.45	0.01\\
87.46	0.01\\
87.47	0.01\\
87.48	0.01\\
87.49	0.01\\
87.5	0.01\\
87.51	0.01\\
87.52	0.01\\
87.53	0.01\\
87.54	0.01\\
87.55	0.01\\
87.56	0.01\\
87.57	0.01\\
87.58	0.01\\
87.59	0.01\\
87.6	0.01\\
87.61	0.01\\
87.62	0.01\\
87.63	0.01\\
87.64	0.01\\
87.65	0.01\\
87.66	0.01\\
87.67	0.01\\
87.68	0.01\\
87.69	0.01\\
87.7	0.01\\
87.71	0.01\\
87.72	0.01\\
87.73	0.01\\
87.74	0.01\\
87.75	0.01\\
87.76	0.01\\
87.77	0.01\\
87.78	0.01\\
87.79	0.01\\
87.8	0.01\\
87.81	0.01\\
87.82	0.01\\
87.83	0.01\\
87.84	0.01\\
87.85	0.01\\
87.86	0.01\\
87.87	0.01\\
87.88	0.01\\
87.89	0.01\\
87.9	0.01\\
87.91	0.01\\
87.92	0.01\\
87.93	0.01\\
87.94	0.01\\
87.95	0.01\\
87.96	0.01\\
87.97	0.01\\
87.98	0.01\\
87.99	0.01\\
88	0.01\\
88.01	0.01\\
88.02	0.01\\
88.03	0.01\\
88.04	0.01\\
88.05	0.01\\
88.06	0.01\\
88.07	0.01\\
88.08	0.01\\
88.09	0.01\\
88.1	0.01\\
88.11	0.01\\
88.12	0.01\\
88.13	0.01\\
88.14	0.01\\
88.15	0.01\\
88.16	0.01\\
88.17	0.01\\
88.18	0.01\\
88.19	0.01\\
88.2	0.01\\
88.21	0.01\\
88.22	0.01\\
88.23	0.01\\
88.24	0.01\\
88.25	0.01\\
88.26	0.01\\
88.27	0.01\\
88.28	0.01\\
88.29	0.01\\
88.3	0.01\\
88.31	0.01\\
88.32	0.01\\
88.33	0.01\\
88.34	0.01\\
88.35	0.01\\
88.36	0.01\\
88.37	0.01\\
88.38	0.01\\
88.39	0.01\\
88.4	0.01\\
88.41	0.01\\
88.42	0.01\\
88.43	0.01\\
88.44	0.01\\
88.45	0.01\\
88.46	0.01\\
88.47	0.01\\
88.48	0.01\\
88.49	0.01\\
88.5	0.01\\
88.51	0.01\\
88.52	0.01\\
88.53	0.01\\
88.54	0.01\\
88.55	0.01\\
88.56	0.01\\
88.57	0.01\\
88.58	0.01\\
88.59	0.01\\
88.6	0.01\\
88.61	0.01\\
88.62	0.01\\
88.63	0.01\\
88.64	0.01\\
88.65	0.01\\
88.66	0.01\\
88.67	0.01\\
88.68	0.01\\
88.69	0.01\\
88.7	0.01\\
88.71	0.01\\
88.72	0.01\\
88.73	0.01\\
88.74	0.01\\
88.75	0.01\\
88.76	0.01\\
88.77	0.01\\
88.78	0.01\\
88.79	0.01\\
88.8	0.01\\
88.81	0.01\\
88.82	0.01\\
88.83	0.01\\
88.84	0.01\\
88.85	0.01\\
88.86	0.01\\
88.87	0.01\\
88.88	0.01\\
88.89	0.01\\
88.9	0.01\\
88.91	0.01\\
88.92	0.01\\
88.93	0.01\\
88.94	0.01\\
88.95	0.01\\
88.96	0.01\\
88.97	0.01\\
88.98	0.01\\
88.99	0.01\\
89	0.01\\
89.01	0.01\\
89.02	0.01\\
89.03	0.01\\
89.04	0.01\\
89.05	0.01\\
89.06	0.01\\
89.07	0.01\\
89.08	0.01\\
89.09	0.01\\
89.1	0.01\\
89.11	0.01\\
89.12	0.01\\
89.13	0.01\\
89.14	0.01\\
89.15	0.01\\
89.16	0.01\\
89.17	0.01\\
89.18	0.01\\
89.19	0.01\\
89.2	0.01\\
89.21	0.01\\
89.22	0.01\\
89.23	0.01\\
89.24	0.01\\
89.25	0.01\\
89.26	0.01\\
89.27	0.01\\
89.28	0.01\\
89.29	0.01\\
89.3	0.01\\
89.31	0.01\\
89.32	0.01\\
89.33	0.01\\
89.34	0.01\\
89.35	0.01\\
89.36	0.01\\
89.37	0.01\\
89.38	0.01\\
89.39	0.01\\
89.4	0.01\\
89.41	0.01\\
89.42	0.01\\
89.43	0.01\\
89.44	0.01\\
89.45	0.01\\
89.46	0.01\\
89.47	0.01\\
89.48	0.01\\
89.49	0.01\\
89.5	0.01\\
89.51	0.01\\
89.52	0.01\\
89.53	0.01\\
89.54	0.01\\
89.55	0.01\\
89.56	0.01\\
89.57	0.01\\
89.58	0.01\\
89.59	0.01\\
89.6	0.01\\
89.61	0.01\\
89.62	0.01\\
89.63	0.01\\
89.64	0.01\\
89.65	0.01\\
89.66	0.01\\
89.67	0.01\\
89.68	0.01\\
89.69	0.01\\
89.7	0.01\\
89.71	0.01\\
89.72	0.01\\
89.73	0.01\\
89.74	0.01\\
89.75	0.01\\
89.76	0.01\\
89.77	0.01\\
89.78	0.01\\
89.79	0.01\\
89.8	0.01\\
89.81	0.01\\
89.82	0.01\\
89.83	0.01\\
89.84	0.01\\
89.85	0.01\\
89.86	0.01\\
89.87	0.01\\
89.88	0.01\\
89.89	0.01\\
89.9	0.01\\
89.91	0.01\\
89.92	0.01\\
89.93	0.01\\
89.94	0.01\\
89.95	0.01\\
89.96	0.01\\
89.97	0.01\\
89.98	0.01\\
89.99	0.01\\
90	0.01\\
90.01	0.01\\
90.02	0.01\\
90.03	0.01\\
90.04	0.01\\
90.05	0.01\\
90.06	0.01\\
90.07	0.01\\
90.08	0.01\\
90.09	0.01\\
90.1	0.01\\
90.11	0.01\\
90.12	0.01\\
90.13	0.01\\
90.14	0.01\\
90.15	0.01\\
90.16	0.01\\
90.17	0.01\\
90.18	0.01\\
90.19	0.01\\
90.2	0.01\\
90.21	0.01\\
90.22	0.01\\
90.23	0.01\\
90.24	0.01\\
90.25	0.01\\
90.26	0.01\\
90.27	0.01\\
90.28	0.01\\
90.29	0.01\\
90.3	0.01\\
90.31	0.01\\
90.32	0.01\\
90.33	0.01\\
90.34	0.01\\
90.35	0.01\\
90.36	0.01\\
90.37	0.01\\
90.38	0.01\\
90.39	0.01\\
90.4	0.01\\
90.41	0.01\\
90.42	0.01\\
90.43	0.01\\
90.44	0.01\\
90.45	0.01\\
90.46	0.01\\
90.47	0.01\\
90.48	0.01\\
90.49	0.01\\
90.5	0.01\\
90.51	0.01\\
90.52	0.01\\
90.53	0.01\\
90.54	0.01\\
90.55	0.01\\
90.56	0.01\\
90.57	0.01\\
90.58	0.01\\
90.59	0.01\\
90.6	0.01\\
90.61	0.01\\
90.62	0.01\\
90.63	0.01\\
90.64	0.01\\
90.65	0.01\\
90.66	0.01\\
90.67	0.01\\
90.68	0.01\\
90.69	0.01\\
90.7	0.01\\
90.71	0.01\\
90.72	0.01\\
90.73	0.01\\
90.74	0.01\\
90.75	0.01\\
90.76	0.01\\
90.77	0.01\\
90.78	0.01\\
90.79	0.01\\
90.8	0.01\\
90.81	0.01\\
90.82	0.01\\
90.83	0.01\\
90.84	0.01\\
90.85	0.01\\
90.86	0.01\\
90.87	0.01\\
90.88	0.01\\
90.89	0.01\\
90.9	0.01\\
90.91	0.01\\
90.92	0.01\\
90.93	0.01\\
90.94	0.01\\
90.95	0.01\\
90.96	0.01\\
90.97	0.01\\
90.98	0.01\\
90.99	0.01\\
91	0.01\\
91.01	0.01\\
91.02	0.01\\
91.03	0.01\\
91.04	0.01\\
91.05	0.01\\
91.06	0.01\\
91.07	0.01\\
91.08	0.01\\
91.09	0.01\\
91.1	0.01\\
91.11	0.01\\
91.12	0.01\\
91.13	0.01\\
91.14	0.01\\
91.15	0.01\\
91.16	0.01\\
91.17	0.01\\
91.18	0.01\\
91.19	0.01\\
91.2	0.01\\
91.21	0.01\\
91.22	0.01\\
91.23	0.01\\
91.24	0.01\\
91.25	0.01\\
91.26	0.01\\
91.27	0.01\\
91.28	0.01\\
91.29	0.01\\
91.3	0.01\\
91.31	0.01\\
91.32	0.01\\
91.33	0.01\\
91.34	0.01\\
91.35	0.01\\
91.36	0.01\\
91.37	0.01\\
91.38	0.01\\
91.39	0.01\\
91.4	0.01\\
91.41	0.01\\
91.42	0.01\\
91.43	0.01\\
91.44	0.01\\
91.45	0.01\\
91.46	0.01\\
91.47	0.01\\
91.48	0.01\\
91.49	0.01\\
91.5	0.01\\
91.51	0.01\\
91.52	0.01\\
91.53	0.01\\
91.54	0.01\\
91.55	0.01\\
91.56	0.01\\
91.57	0.01\\
91.58	0.01\\
91.59	0.01\\
91.6	0.01\\
91.61	0.01\\
91.62	0.01\\
91.63	0.01\\
91.64	0.01\\
91.65	0.01\\
91.66	0.01\\
91.67	0.01\\
91.68	0.01\\
91.69	0.01\\
91.7	0.01\\
91.71	0.01\\
91.72	0.01\\
91.73	0.01\\
91.74	0.01\\
91.75	0.01\\
91.76	0.01\\
91.77	0.01\\
91.78	0.01\\
91.79	0.01\\
91.8	0.01\\
91.81	0.01\\
91.82	0.01\\
91.83	0.01\\
91.84	0.01\\
91.85	0.01\\
91.86	0.01\\
91.87	0.01\\
91.88	0.01\\
91.89	0.01\\
91.9	0.01\\
91.91	0.01\\
91.92	0.01\\
91.93	0.01\\
91.94	0.01\\
91.95	0.01\\
91.96	0.01\\
91.97	0.01\\
91.98	0.01\\
91.99	0.01\\
92	0.01\\
92.01	0.01\\
92.02	0.01\\
92.03	0.01\\
92.04	0.01\\
92.05	0.01\\
92.06	0.01\\
92.07	0.01\\
92.08	0.01\\
92.09	0.01\\
92.1	0.01\\
92.11	0.01\\
92.12	0.01\\
92.13	0.01\\
92.14	0.01\\
92.15	0.01\\
92.16	0.01\\
92.17	0.01\\
92.18	0.01\\
92.19	0.01\\
92.2	0.01\\
92.21	0.01\\
92.22	0.01\\
92.23	0.01\\
92.24	0.01\\
92.25	0.01\\
92.26	0.01\\
92.27	0.01\\
92.28	0.01\\
92.29	0.01\\
92.3	0.01\\
92.31	0.01\\
92.32	0.01\\
92.33	0.01\\
92.34	0.01\\
92.35	0.01\\
92.36	0.01\\
92.37	0.01\\
92.38	0.01\\
92.39	0.01\\
92.4	0.01\\
92.41	0.01\\
92.42	0.01\\
92.43	0.01\\
92.44	0.01\\
92.45	0.01\\
92.46	0.01\\
92.47	0.01\\
92.48	0.01\\
92.49	0.01\\
92.5	0.01\\
92.51	0.01\\
92.52	0.01\\
92.53	0.01\\
92.54	0.01\\
92.55	0.01\\
92.56	0.01\\
92.57	0.01\\
92.58	0.01\\
92.59	0.01\\
92.6	0.01\\
92.61	0.01\\
92.62	0.01\\
92.63	0.01\\
92.64	0.01\\
92.65	0.01\\
92.66	0.01\\
92.67	0.01\\
92.68	0.01\\
92.69	0.01\\
92.7	0.01\\
92.71	0.01\\
92.72	0.01\\
92.73	0.01\\
92.74	0.01\\
92.75	0.01\\
92.76	0.01\\
92.77	0.01\\
92.78	0.01\\
92.79	0.01\\
92.8	0.01\\
92.81	0.01\\
92.82	0.01\\
92.83	0.01\\
92.84	0.01\\
92.85	0.01\\
92.86	0.01\\
92.87	0.01\\
92.88	0.01\\
92.89	0.01\\
92.9	0.01\\
92.91	0.01\\
92.92	0.01\\
92.93	0.01\\
92.94	0.01\\
92.95	0.01\\
92.96	0.01\\
92.97	0.01\\
92.98	0.01\\
92.99	0.01\\
93	0.01\\
93.01	0.01\\
93.02	0.01\\
93.03	0.01\\
93.04	0.01\\
93.05	0.01\\
93.06	0.01\\
93.07	0.01\\
93.08	0.01\\
93.09	0.01\\
93.1	0.01\\
93.11	0.01\\
93.12	0.01\\
93.13	0.01\\
93.14	0.01\\
93.15	0.01\\
93.16	0.01\\
93.17	0.01\\
93.18	0.01\\
93.19	0.01\\
93.2	0.01\\
93.21	0.01\\
93.22	0.01\\
93.23	0.01\\
93.24	0.01\\
93.25	0.01\\
93.26	0.01\\
93.27	0.01\\
93.28	0.01\\
93.29	0.01\\
93.3	0.01\\
93.31	0.01\\
93.32	0.01\\
93.33	0.01\\
93.34	0.01\\
93.35	0.01\\
93.36	0.01\\
93.37	0.01\\
93.38	0.01\\
93.39	0.01\\
93.4	0.01\\
93.41	0.01\\
93.42	0.01\\
93.43	0.01\\
93.44	0.01\\
93.45	0.01\\
93.46	0.01\\
93.47	0.01\\
93.48	0.01\\
93.49	0.01\\
93.5	0.01\\
93.51	0.01\\
93.52	0.01\\
93.53	0.01\\
93.54	0.01\\
93.55	0.01\\
93.56	0.01\\
93.57	0.01\\
93.58	0.01\\
93.59	0.01\\
93.6	0.01\\
93.61	0.01\\
93.62	0.01\\
93.63	0.01\\
93.64	0.01\\
93.65	0.01\\
93.66	0.01\\
93.67	0.01\\
93.68	0.01\\
93.69	0.01\\
93.7	0.01\\
93.71	0.01\\
93.72	0.01\\
93.73	0.01\\
93.74	0.01\\
93.75	0.01\\
93.76	0.01\\
93.77	0.01\\
93.78	0.01\\
93.79	0.01\\
93.8	0.01\\
93.81	0.01\\
93.82	0.01\\
93.83	0.01\\
93.84	0.01\\
93.85	0.01\\
93.86	0.01\\
93.87	0.01\\
93.88	0.01\\
93.89	0.01\\
93.9	0.01\\
93.91	0.01\\
93.92	0.01\\
93.93	0.01\\
93.94	0.01\\
93.95	0.01\\
93.96	0.01\\
93.97	0.01\\
93.98	0.01\\
93.99	0.01\\
94	0.01\\
94.01	0.01\\
94.02	0.01\\
94.03	0.01\\
94.04	0.01\\
94.05	0.01\\
94.06	0.01\\
94.07	0.01\\
94.08	0.01\\
94.09	0.01\\
94.1	0.01\\
94.11	0.01\\
94.12	0.01\\
94.13	0.01\\
94.14	0.01\\
94.15	0.01\\
94.16	0.01\\
94.17	0.01\\
94.18	0.01\\
94.19	0.01\\
94.2	0.01\\
94.21	0.01\\
94.22	0.01\\
94.23	0.01\\
94.24	0.01\\
94.25	0.01\\
94.26	0.01\\
94.27	0.01\\
94.28	0.01\\
94.29	0.01\\
94.3	0.01\\
94.31	0.01\\
94.32	0.01\\
94.33	0.01\\
94.34	0.01\\
94.35	0.01\\
94.36	0.01\\
94.37	0.01\\
94.38	0.01\\
94.39	0.01\\
94.4	0.01\\
94.41	0.01\\
94.42	0.01\\
94.43	0.01\\
94.44	0.01\\
94.45	0.01\\
94.46	0.01\\
94.47	0.01\\
94.48	0.01\\
94.49	0.01\\
94.5	0.01\\
94.51	0.01\\
94.52	0.01\\
94.53	0.01\\
94.54	0.01\\
94.55	0.01\\
94.56	0.01\\
94.57	0.01\\
94.58	0.01\\
94.59	0.01\\
94.6	0.01\\
94.61	0.01\\
94.62	0.01\\
94.63	0.01\\
94.64	0.01\\
94.65	0.01\\
94.66	0.01\\
94.67	0.01\\
94.68	0.01\\
94.69	0.01\\
94.7	0.01\\
94.71	0.01\\
94.72	0.01\\
94.73	0.01\\
94.74	0.01\\
94.75	0.01\\
94.76	0.01\\
94.77	0.01\\
94.78	0.01\\
94.79	0.01\\
94.8	0.01\\
94.81	0.01\\
94.82	0.01\\
94.83	0.01\\
94.84	0.01\\
94.85	0.01\\
94.86	0.01\\
94.87	0.01\\
94.88	0.01\\
94.89	0.01\\
94.9	0.01\\
94.91	0.01\\
94.92	0.01\\
94.93	0.01\\
94.94	0.01\\
94.95	0.01\\
94.96	0.01\\
94.97	0.01\\
94.98	0.01\\
94.99	0.01\\
95	0.01\\
95.01	0.01\\
95.02	0.01\\
95.03	0.01\\
95.04	0.01\\
95.05	0.01\\
95.06	0.01\\
95.07	0.01\\
95.08	0.01\\
95.09	0.01\\
95.1	0.01\\
95.11	0.01\\
95.12	0.01\\
95.13	0.01\\
95.14	0.01\\
95.15	0.01\\
95.16	0.01\\
95.17	0.01\\
95.18	0.01\\
95.19	0.01\\
95.2	0.01\\
95.21	0.01\\
95.22	0.01\\
95.23	0.01\\
95.24	0.01\\
95.25	0.01\\
95.26	0.01\\
95.27	0.01\\
95.28	0.01\\
95.29	0.01\\
95.3	0.01\\
95.31	0.01\\
95.32	0.01\\
95.33	0.01\\
95.34	0.01\\
95.35	0.01\\
95.36	0.01\\
95.37	0.01\\
95.38	0.01\\
95.39	0.01\\
95.4	0.01\\
95.41	0.01\\
95.42	0.01\\
95.43	0.01\\
95.44	0.01\\
95.45	0.01\\
95.46	0.01\\
95.47	0.01\\
95.48	0.01\\
95.49	0.01\\
95.5	0.01\\
95.51	0.01\\
95.52	0.01\\
95.53	0.01\\
95.54	0.01\\
95.55	0.01\\
95.56	0.01\\
95.57	0.01\\
95.58	0.01\\
95.59	0.01\\
95.6	0.01\\
95.61	0.01\\
95.62	0.01\\
95.63	0.01\\
95.64	0.01\\
95.65	0.01\\
95.66	0.01\\
95.67	0.01\\
95.68	0.01\\
95.69	0.01\\
95.7	0.01\\
95.71	0.01\\
95.72	0.01\\
95.73	0.01\\
95.74	0.01\\
95.75	0.01\\
95.76	0.01\\
95.77	0.01\\
95.78	0.01\\
95.79	0.01\\
95.8	0.01\\
95.81	0.01\\
95.82	0.01\\
95.83	0.01\\
95.84	0.01\\
95.85	0.01\\
95.86	0.01\\
95.87	0.01\\
95.88	0.01\\
95.89	0.01\\
95.9	0.01\\
95.91	0.01\\
95.92	0.01\\
95.93	0.01\\
95.94	0.01\\
95.95	0.01\\
95.96	0.01\\
95.97	0.01\\
95.98	0.01\\
95.99	0.01\\
96	0.01\\
96.01	0.01\\
96.02	0.01\\
96.03	0.01\\
96.04	0.01\\
96.05	0.01\\
96.06	0.01\\
96.07	0.01\\
96.08	0.01\\
96.09	0.01\\
96.1	0.01\\
96.11	0.01\\
96.12	0.01\\
96.13	0.01\\
96.14	0.01\\
96.15	0.01\\
96.16	0.01\\
96.17	0.01\\
96.18	0.01\\
96.19	0.01\\
96.2	0.01\\
96.21	0.01\\
96.22	0.01\\
96.23	0.01\\
96.24	0.01\\
96.25	0.01\\
96.26	0.01\\
96.27	0.01\\
96.28	0.01\\
96.29	0.01\\
96.3	0.01\\
96.31	0.01\\
96.32	0.01\\
96.33	0.01\\
96.34	0.01\\
96.35	0.01\\
96.36	0.01\\
96.37	0.01\\
96.38	0.01\\
96.39	0.01\\
96.4	0.01\\
96.41	0.01\\
96.42	0.01\\
96.43	0.01\\
96.44	0.01\\
96.45	0.01\\
96.46	0.01\\
96.47	0.01\\
96.48	0.01\\
96.49	0.01\\
96.5	0.01\\
96.51	0.01\\
96.52	0.01\\
96.53	0.01\\
96.54	0.01\\
96.55	0.01\\
96.56	0.01\\
96.57	0.01\\
96.58	0.01\\
96.59	0.01\\
96.6	0.01\\
96.61	0.01\\
96.62	0.01\\
96.63	0.01\\
96.64	0.01\\
96.65	0.01\\
96.66	0.01\\
96.67	0.01\\
96.68	0.01\\
96.69	0.01\\
96.7	0.01\\
96.71	0.01\\
96.72	0.01\\
96.73	0.01\\
96.74	0.01\\
96.75	0.01\\
96.76	0.01\\
96.77	0.01\\
96.78	0.01\\
96.79	0.01\\
96.8	0.01\\
96.81	0.01\\
96.82	0.01\\
96.83	0.01\\
96.84	0.01\\
96.85	0.01\\
96.86	0.01\\
96.87	0.01\\
96.88	0.01\\
96.89	0.01\\
96.9	0.01\\
96.91	0.01\\
96.92	0.01\\
96.93	0.01\\
96.94	0.01\\
96.95	0.01\\
96.96	0.01\\
96.97	0.01\\
96.98	0.01\\
96.99	0.01\\
97	0.01\\
97.01	0.01\\
97.02	0.01\\
97.03	0.01\\
97.04	0.01\\
97.05	0.01\\
97.06	0.01\\
97.07	0.01\\
97.08	0.01\\
97.09	0.01\\
97.1	0.01\\
97.11	0.01\\
97.12	0.01\\
97.13	0.01\\
97.14	0.01\\
97.15	0.01\\
97.16	0.01\\
97.17	0.01\\
97.18	0.01\\
97.19	0.01\\
97.2	0.01\\
97.21	0.01\\
97.22	0.01\\
97.23	0.01\\
97.24	0.01\\
97.25	0.01\\
97.26	0.01\\
97.27	0.01\\
97.28	0.01\\
97.29	0.01\\
97.3	0.01\\
97.31	0.01\\
97.32	0.01\\
97.33	0.01\\
97.34	0.01\\
97.35	0.01\\
97.36	0.01\\
97.37	0.01\\
97.38	0.01\\
97.39	0.01\\
97.4	0.01\\
97.41	0.01\\
97.42	0.01\\
97.43	0.01\\
97.44	0.01\\
97.45	0.01\\
97.46	0.01\\
97.47	0.01\\
97.48	0.01\\
97.49	0.01\\
97.5	0.01\\
97.51	0.01\\
97.52	0.01\\
97.53	0.01\\
97.54	0.01\\
97.55	0.01\\
97.56	0.01\\
97.57	0.01\\
97.58	0.01\\
97.59	0.01\\
97.6	0.01\\
97.61	0.01\\
97.62	0.01\\
97.63	0.01\\
97.64	0.01\\
97.65	0.01\\
97.66	0.01\\
97.67	0.01\\
97.68	0.01\\
97.69	0.01\\
97.7	0.01\\
97.71	0.01\\
97.72	0.01\\
97.73	0.01\\
97.74	0.01\\
97.75	0.01\\
97.76	0.01\\
97.77	0.01\\
97.78	0.01\\
97.79	0.01\\
97.8	0.01\\
97.81	0.01\\
97.82	0.01\\
97.83	0.01\\
97.84	0.01\\
97.85	0.01\\
97.86	0.01\\
97.87	0.01\\
97.88	0.01\\
97.89	0.01\\
97.9	0.01\\
97.91	0.01\\
97.92	0.01\\
97.93	0.01\\
97.94	0.01\\
97.95	0.01\\
97.96	0.01\\
97.97	0.01\\
97.98	0.01\\
97.99	0.01\\
98	0.01\\
98.01	0.01\\
98.02	0.01\\
98.03	0.01\\
98.04	0.01\\
98.05	0.01\\
98.06	0.01\\
98.07	0.01\\
98.08	0.01\\
98.09	0.01\\
98.1	0.01\\
98.11	0.01\\
98.12	0.01\\
98.13	0.01\\
98.14	0.01\\
98.15	0.01\\
98.16	0.01\\
98.17	0.01\\
98.18	0.01\\
98.19	0.01\\
98.2	0.01\\
98.21	0.01\\
98.22	0.01\\
98.23	0.01\\
98.24	0.01\\
98.25	0.01\\
98.26	0.01\\
98.27	0.01\\
98.28	0.01\\
98.29	0.01\\
98.3	0.01\\
98.31	0.01\\
98.32	0.01\\
98.33	0.01\\
98.34	0.01\\
98.35	0.01\\
98.36	0.01\\
98.37	0.01\\
98.38	0.01\\
98.39	0.01\\
98.4	0.01\\
98.41	0.01\\
98.42	0.01\\
98.43	0.01\\
98.44	0.01\\
98.45	0.01\\
98.46	0.01\\
98.47	0.01\\
98.48	0.01\\
98.49	0.01\\
98.5	0.01\\
98.51	0.01\\
98.52	0.01\\
98.53	0.01\\
98.54	0.01\\
98.55	0.01\\
98.56	0.01\\
98.57	0.01\\
98.58	0.01\\
98.59	0.01\\
98.6	0.01\\
98.61	0.01\\
98.62	0.01\\
98.63	0.01\\
98.64	0.01\\
98.65	0.01\\
98.66	0.01\\
98.67	0.01\\
98.68	0.01\\
98.69	0.01\\
98.7	0.01\\
98.71	0.01\\
98.72	0.01\\
98.73	0.01\\
98.74	0.01\\
98.75	0.01\\
98.76	0.01\\
98.77	0.01\\
98.78	0.01\\
98.79	0.01\\
98.8	0.01\\
98.81	0.01\\
98.82	0.01\\
98.83	0.0098426521778193\\
98.84	0.00962536896190331\\
98.85	0.00940640932468859\\
98.86	0.00918574768776908\\
98.87	0.00896335769774644\\
98.88	0.00873921220349199\\
98.89	0.00851328321738781\\
98.9	0.00828554188254797\\
98.91	0.00805597995566326\\
98.92	0.0078246095817055\\
98.93	0.00759140113879057\\
98.94	0.00735632404879756\\
98.95	0.00711934673653893\\
98.96	0.00693395505590613\\
98.97	0.00688987882381456\\
98.98	0.00684559155289971\\
98.99	0.0068011007461696\\
99	0.0067564144209328\\
99.01	0.00671154109295583\\
99.02	0.00666648979566538\\
99.03	0.00662127010377893\\
99.04	0.00657589218059403\\
99.05	0.00653030932509738\\
99.06	0.00648453023571555\\
99.07	0.00643856490098297\\
99.08	0.00639242392068952\\
99.09	0.00634611853474451\\
99.1	0.00629966065349513\\
99.11	0.00625306288962129\\
99.12	0.00620633859169888\\
99.13	0.00615950187952902\\
99.14	0.0061125419965859\\
99.15	0.00606516434576647\\
99.16	0.00601736505574436\\
99.17	0.00596914019869767\\
99.18	0.0059204857881827\\
99.19	0.00587139777688372\\
99.2	0.00582187205430372\\
99.21	0.00577190444433596\\
99.22	0.00572149075281801\\
99.23	0.00567062689318552\\
99.24	0.0056193087241921\\
99.25	0.00556753204802381\\
99.26	0.00551529260831485\\
99.27	0.00546258608805913\\
99.28	0.00540940810741223\\
99.29	0.0053557542213778\\
99.3	0.00530161991737217\\
99.31	0.00524700061266076\\
99.32	0.00519189165165917\\
99.33	0.00513628830309197\\
99.34	0.00508018575700118\\
99.35	0.0050235791215973\\
99.36	0.00496646341994358\\
99.37	0.00490883358644935\\
99.38	0.0048506844630757\\
99.39	0.00479201079544656\\
99.4	0.00473280722875311\\
99.41	0.00467306830343948\\
99.42	0.00461278915159383\\
99.43	0.00455196486264461\\
99.44	0.00449059048163354\\
99.45	0.00442866100885806\\
99.46	0.0043661713995154\\
99.47	0.00430311656334906\\
99.48	0.0042394913639357\\
99.49	0.00417529061784133\\
99.5	0.00411050909416695\\
99.51	0.00404514151408925\\
99.52	0.00397918255039652\\
99.53	0.00391262682701961\\
99.54	0.00384546891855804\\
99.55	0.00377770334980126\\
99.56	0.00370932459524504\\
99.57	0.00364032707860304\\
99.58	0.00357070517231364\\
99.59	0.0035004531970421\\
99.6	0.00342956542117795\\
99.61	0.00335803606032796\\
99.62	0.0032858592768046\\
99.63	0.00321302917911018\\
99.64	0.00313953982141679\\
99.65	0.00306538520304226\\
99.66	0.0029905592679222\\
99.67	0.00291505590407856\\
99.68	0.00283886894308471\\
99.69	0.00276199217558516\\
99.7	0.00268441933711856\\
99.71	0.00260614410548349\\
99.72	0.0025271601001977\\
99.73	0.0024474608819558\\
99.74	0.00236703995208607\\
99.75	0.00228589075200679\\
99.76	0.00220400666268282\\
99.77	0.00212138100408299\\
99.78	0.00203800703463929\\
99.79	0.00195387795070856\\
99.8	0.00186898688603761\\
99.81	0.00178332691123306\\
99.82	0.00169689103323675\\
99.83	0.00160967219480833\\
99.84	0.00152166327401621\\
99.85	0.00143285708373862\\
99.86	0.00134324637117649\\
99.87	0.00125282381737996\\
99.88	0.00116158203679085\\
99.89	0.00106951357680327\\
99.9	0.00097661091734502\\
99.91	0.000882866470482559\\
99.92	0.000788272580052774\\
99.93	0.000692821521324887\\
99.94	0.000596505500696339\\
99.95	0.000499316655426806\\
99.96	0.000401247053414959\\
99.97	0.000302288693022996\\
99.98	0.000202433502954543\\
99.99	0.00010167334219198\\
100	0\\
};
\addlegendentry{$q=-4$};

\addplot [color=mycolor1,dashed,forget plot]
  table[row sep=crcr]{%
0.01	0.01\\
0.02	0.01\\
0.03	0.01\\
0.04	0.01\\
0.05	0.01\\
0.06	0.01\\
0.07	0.01\\
0.08	0.01\\
0.09	0.01\\
0.1	0.01\\
0.11	0.01\\
0.12	0.01\\
0.13	0.01\\
0.14	0.01\\
0.15	0.01\\
0.16	0.01\\
0.17	0.01\\
0.18	0.01\\
0.19	0.01\\
0.2	0.01\\
0.21	0.01\\
0.22	0.01\\
0.23	0.01\\
0.24	0.01\\
0.25	0.01\\
0.26	0.01\\
0.27	0.01\\
0.28	0.01\\
0.29	0.01\\
0.3	0.01\\
0.31	0.01\\
0.32	0.01\\
0.33	0.01\\
0.34	0.01\\
0.35	0.01\\
0.36	0.01\\
0.37	0.01\\
0.38	0.01\\
0.39	0.01\\
0.4	0.01\\
0.41	0.01\\
0.42	0.01\\
0.43	0.01\\
0.44	0.01\\
0.45	0.01\\
0.46	0.01\\
0.47	0.01\\
0.48	0.01\\
0.49	0.01\\
0.5	0.01\\
0.51	0.01\\
0.52	0.01\\
0.53	0.01\\
0.54	0.01\\
0.55	0.01\\
0.56	0.01\\
0.57	0.01\\
0.58	0.01\\
0.59	0.01\\
0.6	0.01\\
0.61	0.01\\
0.62	0.01\\
0.63	0.01\\
0.64	0.01\\
0.65	0.01\\
0.66	0.01\\
0.67	0.01\\
0.68	0.01\\
0.69	0.01\\
0.7	0.01\\
0.71	0.01\\
0.72	0.01\\
0.73	0.01\\
0.74	0.01\\
0.75	0.01\\
0.76	0.01\\
0.77	0.01\\
0.78	0.01\\
0.79	0.01\\
0.8	0.01\\
0.81	0.01\\
0.82	0.01\\
0.83	0.01\\
0.84	0.01\\
0.85	0.01\\
0.86	0.01\\
0.87	0.01\\
0.88	0.01\\
0.89	0.01\\
0.9	0.01\\
0.91	0.01\\
0.92	0.01\\
0.93	0.01\\
0.94	0.01\\
0.95	0.01\\
0.96	0.01\\
0.97	0.01\\
0.98	0.01\\
0.99	0.01\\
1	0.01\\
1.01	0.01\\
1.02	0.01\\
1.03	0.01\\
1.04	0.01\\
1.05	0.01\\
1.06	0.01\\
1.07	0.01\\
1.08	0.01\\
1.09	0.01\\
1.1	0.01\\
1.11	0.01\\
1.12	0.01\\
1.13	0.01\\
1.14	0.01\\
1.15	0.01\\
1.16	0.01\\
1.17	0.01\\
1.18	0.01\\
1.19	0.01\\
1.2	0.01\\
1.21	0.01\\
1.22	0.01\\
1.23	0.01\\
1.24	0.01\\
1.25	0.01\\
1.26	0.01\\
1.27	0.01\\
1.28	0.01\\
1.29	0.01\\
1.3	0.01\\
1.31	0.01\\
1.32	0.01\\
1.33	0.01\\
1.34	0.01\\
1.35	0.01\\
1.36	0.01\\
1.37	0.01\\
1.38	0.01\\
1.39	0.01\\
1.4	0.01\\
1.41	0.01\\
1.42	0.01\\
1.43	0.01\\
1.44	0.01\\
1.45	0.01\\
1.46	0.01\\
1.47	0.01\\
1.48	0.01\\
1.49	0.01\\
1.5	0.01\\
1.51	0.01\\
1.52	0.01\\
1.53	0.01\\
1.54	0.01\\
1.55	0.01\\
1.56	0.01\\
1.57	0.01\\
1.58	0.01\\
1.59	0.01\\
1.6	0.01\\
1.61	0.01\\
1.62	0.01\\
1.63	0.01\\
1.64	0.01\\
1.65	0.01\\
1.66	0.01\\
1.67	0.01\\
1.68	0.01\\
1.69	0.01\\
1.7	0.01\\
1.71	0.01\\
1.72	0.01\\
1.73	0.01\\
1.74	0.01\\
1.75	0.01\\
1.76	0.01\\
1.77	0.01\\
1.78	0.01\\
1.79	0.01\\
1.8	0.01\\
1.81	0.01\\
1.82	0.01\\
1.83	0.01\\
1.84	0.01\\
1.85	0.01\\
1.86	0.01\\
1.87	0.01\\
1.88	0.01\\
1.89	0.01\\
1.9	0.01\\
1.91	0.01\\
1.92	0.01\\
1.93	0.01\\
1.94	0.01\\
1.95	0.01\\
1.96	0.01\\
1.97	0.01\\
1.98	0.01\\
1.99	0.01\\
2	0.01\\
2.01	0.01\\
2.02	0.01\\
2.03	0.01\\
2.04	0.01\\
2.05	0.01\\
2.06	0.01\\
2.07	0.01\\
2.08	0.01\\
2.09	0.01\\
2.1	0.01\\
2.11	0.01\\
2.12	0.01\\
2.13	0.01\\
2.14	0.01\\
2.15	0.01\\
2.16	0.01\\
2.17	0.01\\
2.18	0.01\\
2.19	0.01\\
2.2	0.01\\
2.21	0.01\\
2.22	0.01\\
2.23	0.01\\
2.24	0.01\\
2.25	0.01\\
2.26	0.01\\
2.27	0.01\\
2.28	0.01\\
2.29	0.01\\
2.3	0.01\\
2.31	0.01\\
2.32	0.01\\
2.33	0.01\\
2.34	0.01\\
2.35	0.01\\
2.36	0.01\\
2.37	0.01\\
2.38	0.01\\
2.39	0.01\\
2.4	0.01\\
2.41	0.01\\
2.42	0.01\\
2.43	0.01\\
2.44	0.01\\
2.45	0.01\\
2.46	0.01\\
2.47	0.01\\
2.48	0.01\\
2.49	0.01\\
2.5	0.01\\
2.51	0.01\\
2.52	0.01\\
2.53	0.01\\
2.54	0.01\\
2.55	0.01\\
2.56	0.01\\
2.57	0.01\\
2.58	0.01\\
2.59	0.01\\
2.6	0.01\\
2.61	0.01\\
2.62	0.01\\
2.63	0.01\\
2.64	0.01\\
2.65	0.01\\
2.66	0.01\\
2.67	0.01\\
2.68	0.01\\
2.69	0.01\\
2.7	0.01\\
2.71	0.01\\
2.72	0.01\\
2.73	0.01\\
2.74	0.01\\
2.75	0.01\\
2.76	0.01\\
2.77	0.01\\
2.78	0.01\\
2.79	0.01\\
2.8	0.01\\
2.81	0.01\\
2.82	0.01\\
2.83	0.01\\
2.84	0.01\\
2.85	0.01\\
2.86	0.01\\
2.87	0.01\\
2.88	0.01\\
2.89	0.01\\
2.9	0.01\\
2.91	0.01\\
2.92	0.01\\
2.93	0.01\\
2.94	0.01\\
2.95	0.01\\
2.96	0.01\\
2.97	0.01\\
2.98	0.01\\
2.99	0.01\\
3	0.01\\
3.01	0.01\\
3.02	0.01\\
3.03	0.01\\
3.04	0.01\\
3.05	0.01\\
3.06	0.01\\
3.07	0.01\\
3.08	0.01\\
3.09	0.01\\
3.1	0.01\\
3.11	0.01\\
3.12	0.01\\
3.13	0.01\\
3.14	0.01\\
3.15	0.01\\
3.16	0.01\\
3.17	0.01\\
3.18	0.01\\
3.19	0.01\\
3.2	0.01\\
3.21	0.01\\
3.22	0.01\\
3.23	0.01\\
3.24	0.01\\
3.25	0.01\\
3.26	0.01\\
3.27	0.01\\
3.28	0.01\\
3.29	0.01\\
3.3	0.01\\
3.31	0.01\\
3.32	0.01\\
3.33	0.01\\
3.34	0.01\\
3.35	0.01\\
3.36	0.01\\
3.37	0.01\\
3.38	0.01\\
3.39	0.01\\
3.4	0.01\\
3.41	0.01\\
3.42	0.01\\
3.43	0.01\\
3.44	0.01\\
3.45	0.01\\
3.46	0.01\\
3.47	0.01\\
3.48	0.01\\
3.49	0.01\\
3.5	0.01\\
3.51	0.01\\
3.52	0.01\\
3.53	0.01\\
3.54	0.01\\
3.55	0.01\\
3.56	0.01\\
3.57	0.01\\
3.58	0.01\\
3.59	0.01\\
3.6	0.01\\
3.61	0.01\\
3.62	0.01\\
3.63	0.01\\
3.64	0.01\\
3.65	0.01\\
3.66	0.01\\
3.67	0.01\\
3.68	0.01\\
3.69	0.01\\
3.7	0.01\\
3.71	0.01\\
3.72	0.01\\
3.73	0.01\\
3.74	0.01\\
3.75	0.01\\
3.76	0.01\\
3.77	0.01\\
3.78	0.01\\
3.79	0.01\\
3.8	0.01\\
3.81	0.01\\
3.82	0.01\\
3.83	0.01\\
3.84	0.01\\
3.85	0.01\\
3.86	0.01\\
3.87	0.01\\
3.88	0.01\\
3.89	0.01\\
3.9	0.01\\
3.91	0.01\\
3.92	0.01\\
3.93	0.01\\
3.94	0.01\\
3.95	0.01\\
3.96	0.01\\
3.97	0.01\\
3.98	0.01\\
3.99	0.01\\
4	0.01\\
4.01	0.01\\
4.02	0.01\\
4.03	0.01\\
4.04	0.01\\
4.05	0.01\\
4.06	0.01\\
4.07	0.01\\
4.08	0.01\\
4.09	0.01\\
4.1	0.01\\
4.11	0.01\\
4.12	0.01\\
4.13	0.01\\
4.14	0.01\\
4.15	0.01\\
4.16	0.01\\
4.17	0.01\\
4.18	0.01\\
4.19	0.01\\
4.2	0.01\\
4.21	0.01\\
4.22	0.01\\
4.23	0.01\\
4.24	0.01\\
4.25	0.01\\
4.26	0.01\\
4.27	0.01\\
4.28	0.01\\
4.29	0.01\\
4.3	0.01\\
4.31	0.01\\
4.32	0.01\\
4.33	0.01\\
4.34	0.01\\
4.35	0.01\\
4.36	0.01\\
4.37	0.01\\
4.38	0.01\\
4.39	0.01\\
4.4	0.01\\
4.41	0.01\\
4.42	0.01\\
4.43	0.01\\
4.44	0.01\\
4.45	0.01\\
4.46	0.01\\
4.47	0.01\\
4.48	0.01\\
4.49	0.01\\
4.5	0.01\\
4.51	0.01\\
4.52	0.01\\
4.53	0.01\\
4.54	0.01\\
4.55	0.01\\
4.56	0.01\\
4.57	0.01\\
4.58	0.01\\
4.59	0.01\\
4.6	0.01\\
4.61	0.01\\
4.62	0.01\\
4.63	0.01\\
4.64	0.01\\
4.65	0.01\\
4.66	0.01\\
4.67	0.01\\
4.68	0.01\\
4.69	0.01\\
4.7	0.01\\
4.71	0.01\\
4.72	0.01\\
4.73	0.01\\
4.74	0.01\\
4.75	0.01\\
4.76	0.01\\
4.77	0.01\\
4.78	0.01\\
4.79	0.01\\
4.8	0.01\\
4.81	0.01\\
4.82	0.01\\
4.83	0.01\\
4.84	0.01\\
4.85	0.01\\
4.86	0.01\\
4.87	0.01\\
4.88	0.01\\
4.89	0.01\\
4.9	0.01\\
4.91	0.01\\
4.92	0.01\\
4.93	0.01\\
4.94	0.01\\
4.95	0.01\\
4.96	0.01\\
4.97	0.01\\
4.98	0.01\\
4.99	0.01\\
5	0.01\\
5.01	0.01\\
5.02	0.01\\
5.03	0.01\\
5.04	0.01\\
5.05	0.01\\
5.06	0.01\\
5.07	0.01\\
5.08	0.01\\
5.09	0.01\\
5.1	0.01\\
5.11	0.01\\
5.12	0.01\\
5.13	0.01\\
5.14	0.01\\
5.15	0.01\\
5.16	0.01\\
5.17	0.01\\
5.18	0.01\\
5.19	0.01\\
5.2	0.01\\
5.21	0.01\\
5.22	0.01\\
5.23	0.01\\
5.24	0.01\\
5.25	0.01\\
5.26	0.01\\
5.27	0.01\\
5.28	0.01\\
5.29	0.01\\
5.3	0.01\\
5.31	0.01\\
5.32	0.01\\
5.33	0.01\\
5.34	0.01\\
5.35	0.01\\
5.36	0.01\\
5.37	0.01\\
5.38	0.01\\
5.39	0.01\\
5.4	0.01\\
5.41	0.01\\
5.42	0.01\\
5.43	0.01\\
5.44	0.01\\
5.45	0.01\\
5.46	0.01\\
5.47	0.01\\
5.48	0.01\\
5.49	0.01\\
5.5	0.01\\
5.51	0.01\\
5.52	0.01\\
5.53	0.01\\
5.54	0.01\\
5.55	0.01\\
5.56	0.01\\
5.57	0.01\\
5.58	0.01\\
5.59	0.01\\
5.6	0.01\\
5.61	0.01\\
5.62	0.01\\
5.63	0.01\\
5.64	0.01\\
5.65	0.01\\
5.66	0.01\\
5.67	0.01\\
5.68	0.01\\
5.69	0.01\\
5.7	0.01\\
5.71	0.01\\
5.72	0.01\\
5.73	0.01\\
5.74	0.01\\
5.75	0.01\\
5.76	0.01\\
5.77	0.01\\
5.78	0.01\\
5.79	0.01\\
5.8	0.01\\
5.81	0.01\\
5.82	0.01\\
5.83	0.01\\
5.84	0.01\\
5.85	0.01\\
5.86	0.01\\
5.87	0.01\\
5.88	0.01\\
5.89	0.01\\
5.9	0.01\\
5.91	0.01\\
5.92	0.01\\
5.93	0.01\\
5.94	0.01\\
5.95	0.01\\
5.96	0.01\\
5.97	0.01\\
5.98	0.01\\
5.99	0.01\\
6	0.01\\
6.01	0.01\\
6.02	0.01\\
6.03	0.01\\
6.04	0.01\\
6.05	0.01\\
6.06	0.01\\
6.07	0.01\\
6.08	0.01\\
6.09	0.01\\
6.1	0.01\\
6.11	0.01\\
6.12	0.01\\
6.13	0.01\\
6.14	0.01\\
6.15	0.01\\
6.16	0.01\\
6.17	0.01\\
6.18	0.01\\
6.19	0.01\\
6.2	0.01\\
6.21	0.01\\
6.22	0.01\\
6.23	0.01\\
6.24	0.01\\
6.25	0.01\\
6.26	0.01\\
6.27	0.01\\
6.28	0.01\\
6.29	0.01\\
6.3	0.01\\
6.31	0.01\\
6.32	0.01\\
6.33	0.01\\
6.34	0.01\\
6.35	0.01\\
6.36	0.01\\
6.37	0.01\\
6.38	0.01\\
6.39	0.01\\
6.4	0.01\\
6.41	0.01\\
6.42	0.01\\
6.43	0.01\\
6.44	0.01\\
6.45	0.01\\
6.46	0.01\\
6.47	0.01\\
6.48	0.01\\
6.49	0.01\\
6.5	0.01\\
6.51	0.01\\
6.52	0.01\\
6.53	0.01\\
6.54	0.01\\
6.55	0.01\\
6.56	0.01\\
6.57	0.01\\
6.58	0.01\\
6.59	0.01\\
6.6	0.01\\
6.61	0.01\\
6.62	0.01\\
6.63	0.01\\
6.64	0.01\\
6.65	0.01\\
6.66	0.01\\
6.67	0.01\\
6.68	0.01\\
6.69	0.01\\
6.7	0.01\\
6.71	0.01\\
6.72	0.01\\
6.73	0.01\\
6.74	0.01\\
6.75	0.01\\
6.76	0.01\\
6.77	0.01\\
6.78	0.01\\
6.79	0.01\\
6.8	0.01\\
6.81	0.01\\
6.82	0.01\\
6.83	0.01\\
6.84	0.01\\
6.85	0.01\\
6.86	0.01\\
6.87	0.01\\
6.88	0.01\\
6.89	0.01\\
6.9	0.01\\
6.91	0.01\\
6.92	0.01\\
6.93	0.01\\
6.94	0.01\\
6.95	0.01\\
6.96	0.01\\
6.97	0.01\\
6.98	0.01\\
6.99	0.01\\
7	0.01\\
7.01	0.01\\
7.02	0.01\\
7.03	0.01\\
7.04	0.01\\
7.05	0.01\\
7.06	0.01\\
7.07	0.01\\
7.08	0.01\\
7.09	0.01\\
7.1	0.01\\
7.11	0.01\\
7.12	0.01\\
7.13	0.01\\
7.14	0.01\\
7.15	0.01\\
7.16	0.01\\
7.17	0.01\\
7.18	0.01\\
7.19	0.01\\
7.2	0.01\\
7.21	0.01\\
7.22	0.01\\
7.23	0.01\\
7.24	0.01\\
7.25	0.01\\
7.26	0.01\\
7.27	0.01\\
7.28	0.01\\
7.29	0.01\\
7.3	0.01\\
7.31	0.01\\
7.32	0.01\\
7.33	0.01\\
7.34	0.01\\
7.35	0.01\\
7.36	0.01\\
7.37	0.01\\
7.38	0.01\\
7.39	0.01\\
7.4	0.01\\
7.41	0.01\\
7.42	0.01\\
7.43	0.01\\
7.44	0.01\\
7.45	0.01\\
7.46	0.01\\
7.47	0.01\\
7.48	0.01\\
7.49	0.01\\
7.5	0.01\\
7.51	0.01\\
7.52	0.01\\
7.53	0.01\\
7.54	0.01\\
7.55	0.01\\
7.56	0.01\\
7.57	0.01\\
7.58	0.01\\
7.59	0.01\\
7.6	0.01\\
7.61	0.01\\
7.62	0.01\\
7.63	0.01\\
7.64	0.01\\
7.65	0.01\\
7.66	0.01\\
7.67	0.01\\
7.68	0.01\\
7.69	0.01\\
7.7	0.01\\
7.71	0.01\\
7.72	0.01\\
7.73	0.01\\
7.74	0.01\\
7.75	0.01\\
7.76	0.01\\
7.77	0.01\\
7.78	0.01\\
7.79	0.01\\
7.8	0.01\\
7.81	0.01\\
7.82	0.01\\
7.83	0.01\\
7.84	0.01\\
7.85	0.01\\
7.86	0.01\\
7.87	0.01\\
7.88	0.01\\
7.89	0.01\\
7.9	0.01\\
7.91	0.01\\
7.92	0.01\\
7.93	0.01\\
7.94	0.01\\
7.95	0.01\\
7.96	0.01\\
7.97	0.01\\
7.98	0.01\\
7.99	0.01\\
8	0.01\\
8.01	0.01\\
8.02	0.01\\
8.03	0.01\\
8.04	0.01\\
8.05	0.01\\
8.06	0.01\\
8.07	0.01\\
8.08	0.01\\
8.09	0.01\\
8.1	0.01\\
8.11	0.01\\
8.12	0.01\\
8.13	0.01\\
8.14	0.01\\
8.15	0.01\\
8.16	0.01\\
8.17	0.01\\
8.18	0.01\\
8.19	0.01\\
8.2	0.01\\
8.21	0.01\\
8.22	0.01\\
8.23	0.01\\
8.24	0.01\\
8.25	0.01\\
8.26	0.01\\
8.27	0.01\\
8.28	0.01\\
8.29	0.01\\
8.3	0.01\\
8.31	0.01\\
8.32	0.01\\
8.33	0.01\\
8.34	0.01\\
8.35	0.01\\
8.36	0.01\\
8.37	0.01\\
8.38	0.01\\
8.39	0.01\\
8.4	0.01\\
8.41	0.01\\
8.42	0.01\\
8.43	0.01\\
8.44	0.01\\
8.45	0.01\\
8.46	0.01\\
8.47	0.01\\
8.48	0.01\\
8.49	0.01\\
8.5	0.01\\
8.51	0.01\\
8.52	0.01\\
8.53	0.01\\
8.54	0.01\\
8.55	0.01\\
8.56	0.01\\
8.57	0.01\\
8.58	0.01\\
8.59	0.01\\
8.6	0.01\\
8.61	0.01\\
8.62	0.01\\
8.63	0.01\\
8.64	0.01\\
8.65	0.01\\
8.66	0.01\\
8.67	0.01\\
8.68	0.01\\
8.69	0.01\\
8.7	0.01\\
8.71	0.01\\
8.72	0.01\\
8.73	0.01\\
8.74	0.01\\
8.75	0.01\\
8.76	0.01\\
8.77	0.01\\
8.78	0.01\\
8.79	0.01\\
8.8	0.01\\
8.81	0.01\\
8.82	0.01\\
8.83	0.01\\
8.84	0.01\\
8.85	0.01\\
8.86	0.01\\
8.87	0.01\\
8.88	0.01\\
8.89	0.01\\
8.9	0.01\\
8.91	0.01\\
8.92	0.01\\
8.93	0.01\\
8.94	0.01\\
8.95	0.01\\
8.96	0.01\\
8.97	0.01\\
8.98	0.01\\
8.99	0.01\\
9	0.01\\
9.01	0.01\\
9.02	0.01\\
9.03	0.01\\
9.04	0.01\\
9.05	0.01\\
9.06	0.01\\
9.07	0.01\\
9.08	0.01\\
9.09	0.01\\
9.1	0.01\\
9.11	0.01\\
9.12	0.01\\
9.13	0.01\\
9.14	0.01\\
9.15	0.01\\
9.16	0.01\\
9.17	0.01\\
9.18	0.01\\
9.19	0.01\\
9.2	0.01\\
9.21	0.01\\
9.22	0.01\\
9.23	0.01\\
9.24	0.01\\
9.25	0.01\\
9.26	0.01\\
9.27	0.01\\
9.28	0.01\\
9.29	0.01\\
9.3	0.01\\
9.31	0.01\\
9.32	0.01\\
9.33	0.01\\
9.34	0.01\\
9.35	0.01\\
9.36	0.01\\
9.37	0.01\\
9.38	0.01\\
9.39	0.01\\
9.4	0.01\\
9.41	0.01\\
9.42	0.01\\
9.43	0.01\\
9.44	0.01\\
9.45	0.01\\
9.46	0.01\\
9.47	0.01\\
9.48	0.01\\
9.49	0.01\\
9.5	0.01\\
9.51	0.01\\
9.52	0.01\\
9.53	0.01\\
9.54	0.01\\
9.55	0.01\\
9.56	0.01\\
9.57	0.01\\
9.58	0.01\\
9.59	0.01\\
9.6	0.01\\
9.61	0.01\\
9.62	0.01\\
9.63	0.01\\
9.64	0.01\\
9.65	0.01\\
9.66	0.01\\
9.67	0.01\\
9.68	0.01\\
9.69	0.01\\
9.7	0.01\\
9.71	0.01\\
9.72	0.01\\
9.73	0.01\\
9.74	0.01\\
9.75	0.01\\
9.76	0.01\\
9.77	0.01\\
9.78	0.01\\
9.79	0.01\\
9.8	0.01\\
9.81	0.01\\
9.82	0.01\\
9.83	0.01\\
9.84	0.01\\
9.85	0.01\\
9.86	0.01\\
9.87	0.01\\
9.88	0.01\\
9.89	0.01\\
9.9	0.01\\
9.91	0.01\\
9.92	0.01\\
9.93	0.01\\
9.94	0.01\\
9.95	0.01\\
9.96	0.01\\
9.97	0.01\\
9.98	0.01\\
9.99	0.01\\
10	0.01\\
10.01	0.01\\
10.02	0.01\\
10.03	0.01\\
10.04	0.01\\
10.05	0.01\\
10.06	0.01\\
10.07	0.01\\
10.08	0.01\\
10.09	0.01\\
10.1	0.01\\
10.11	0.01\\
10.12	0.01\\
10.13	0.01\\
10.14	0.01\\
10.15	0.01\\
10.16	0.01\\
10.17	0.01\\
10.18	0.01\\
10.19	0.01\\
10.2	0.01\\
10.21	0.01\\
10.22	0.01\\
10.23	0.01\\
10.24	0.01\\
10.25	0.01\\
10.26	0.01\\
10.27	0.01\\
10.28	0.01\\
10.29	0.01\\
10.3	0.01\\
10.31	0.01\\
10.32	0.01\\
10.33	0.01\\
10.34	0.01\\
10.35	0.01\\
10.36	0.01\\
10.37	0.01\\
10.38	0.01\\
10.39	0.01\\
10.4	0.01\\
10.41	0.01\\
10.42	0.01\\
10.43	0.01\\
10.44	0.01\\
10.45	0.01\\
10.46	0.01\\
10.47	0.01\\
10.48	0.01\\
10.49	0.01\\
10.5	0.01\\
10.51	0.01\\
10.52	0.01\\
10.53	0.01\\
10.54	0.01\\
10.55	0.01\\
10.56	0.01\\
10.57	0.01\\
10.58	0.01\\
10.59	0.01\\
10.6	0.01\\
10.61	0.01\\
10.62	0.01\\
10.63	0.01\\
10.64	0.01\\
10.65	0.01\\
10.66	0.01\\
10.67	0.01\\
10.68	0.01\\
10.69	0.01\\
10.7	0.01\\
10.71	0.01\\
10.72	0.01\\
10.73	0.01\\
10.74	0.01\\
10.75	0.01\\
10.76	0.01\\
10.77	0.01\\
10.78	0.01\\
10.79	0.01\\
10.8	0.01\\
10.81	0.01\\
10.82	0.01\\
10.83	0.01\\
10.84	0.01\\
10.85	0.01\\
10.86	0.01\\
10.87	0.01\\
10.88	0.01\\
10.89	0.01\\
10.9	0.01\\
10.91	0.01\\
10.92	0.01\\
10.93	0.01\\
10.94	0.01\\
10.95	0.01\\
10.96	0.01\\
10.97	0.01\\
10.98	0.01\\
10.99	0.01\\
11	0.01\\
11.01	0.01\\
11.02	0.01\\
11.03	0.01\\
11.04	0.01\\
11.05	0.01\\
11.06	0.01\\
11.07	0.01\\
11.08	0.01\\
11.09	0.01\\
11.1	0.01\\
11.11	0.01\\
11.12	0.01\\
11.13	0.01\\
11.14	0.01\\
11.15	0.01\\
11.16	0.01\\
11.17	0.01\\
11.18	0.01\\
11.19	0.01\\
11.2	0.01\\
11.21	0.01\\
11.22	0.01\\
11.23	0.01\\
11.24	0.01\\
11.25	0.01\\
11.26	0.01\\
11.27	0.01\\
11.28	0.01\\
11.29	0.01\\
11.3	0.01\\
11.31	0.01\\
11.32	0.01\\
11.33	0.01\\
11.34	0.01\\
11.35	0.01\\
11.36	0.01\\
11.37	0.01\\
11.38	0.01\\
11.39	0.01\\
11.4	0.01\\
11.41	0.01\\
11.42	0.01\\
11.43	0.01\\
11.44	0.01\\
11.45	0.01\\
11.46	0.01\\
11.47	0.01\\
11.48	0.01\\
11.49	0.01\\
11.5	0.01\\
11.51	0.01\\
11.52	0.01\\
11.53	0.01\\
11.54	0.01\\
11.55	0.01\\
11.56	0.01\\
11.57	0.01\\
11.58	0.01\\
11.59	0.01\\
11.6	0.01\\
11.61	0.01\\
11.62	0.01\\
11.63	0.01\\
11.64	0.01\\
11.65	0.01\\
11.66	0.01\\
11.67	0.01\\
11.68	0.01\\
11.69	0.01\\
11.7	0.01\\
11.71	0.01\\
11.72	0.01\\
11.73	0.01\\
11.74	0.01\\
11.75	0.01\\
11.76	0.01\\
11.77	0.01\\
11.78	0.01\\
11.79	0.01\\
11.8	0.01\\
11.81	0.01\\
11.82	0.01\\
11.83	0.01\\
11.84	0.01\\
11.85	0.01\\
11.86	0.01\\
11.87	0.01\\
11.88	0.01\\
11.89	0.01\\
11.9	0.01\\
11.91	0.01\\
11.92	0.01\\
11.93	0.01\\
11.94	0.01\\
11.95	0.01\\
11.96	0.01\\
11.97	0.01\\
11.98	0.01\\
11.99	0.01\\
12	0.01\\
12.01	0.01\\
12.02	0.01\\
12.03	0.01\\
12.04	0.01\\
12.05	0.01\\
12.06	0.01\\
12.07	0.01\\
12.08	0.01\\
12.09	0.01\\
12.1	0.01\\
12.11	0.01\\
12.12	0.01\\
12.13	0.01\\
12.14	0.01\\
12.15	0.01\\
12.16	0.01\\
12.17	0.01\\
12.18	0.01\\
12.19	0.01\\
12.2	0.01\\
12.21	0.01\\
12.22	0.01\\
12.23	0.01\\
12.24	0.01\\
12.25	0.01\\
12.26	0.01\\
12.27	0.01\\
12.28	0.01\\
12.29	0.01\\
12.3	0.01\\
12.31	0.01\\
12.32	0.01\\
12.33	0.01\\
12.34	0.01\\
12.35	0.01\\
12.36	0.01\\
12.37	0.01\\
12.38	0.01\\
12.39	0.01\\
12.4	0.01\\
12.41	0.01\\
12.42	0.01\\
12.43	0.01\\
12.44	0.01\\
12.45	0.01\\
12.46	0.01\\
12.47	0.01\\
12.48	0.01\\
12.49	0.01\\
12.5	0.01\\
12.51	0.01\\
12.52	0.01\\
12.53	0.01\\
12.54	0.01\\
12.55	0.01\\
12.56	0.01\\
12.57	0.01\\
12.58	0.01\\
12.59	0.01\\
12.6	0.01\\
12.61	0.01\\
12.62	0.01\\
12.63	0.01\\
12.64	0.01\\
12.65	0.01\\
12.66	0.01\\
12.67	0.01\\
12.68	0.01\\
12.69	0.01\\
12.7	0.01\\
12.71	0.01\\
12.72	0.01\\
12.73	0.01\\
12.74	0.01\\
12.75	0.01\\
12.76	0.01\\
12.77	0.01\\
12.78	0.01\\
12.79	0.01\\
12.8	0.01\\
12.81	0.01\\
12.82	0.01\\
12.83	0.01\\
12.84	0.01\\
12.85	0.01\\
12.86	0.01\\
12.87	0.01\\
12.88	0.01\\
12.89	0.01\\
12.9	0.01\\
12.91	0.01\\
12.92	0.01\\
12.93	0.01\\
12.94	0.01\\
12.95	0.01\\
12.96	0.01\\
12.97	0.01\\
12.98	0.01\\
12.99	0.01\\
13	0.01\\
13.01	0.01\\
13.02	0.01\\
13.03	0.01\\
13.04	0.01\\
13.05	0.01\\
13.06	0.01\\
13.07	0.01\\
13.08	0.01\\
13.09	0.01\\
13.1	0.01\\
13.11	0.01\\
13.12	0.01\\
13.13	0.01\\
13.14	0.01\\
13.15	0.01\\
13.16	0.01\\
13.17	0.01\\
13.18	0.01\\
13.19	0.01\\
13.2	0.01\\
13.21	0.01\\
13.22	0.01\\
13.23	0.01\\
13.24	0.01\\
13.25	0.01\\
13.26	0.01\\
13.27	0.01\\
13.28	0.01\\
13.29	0.01\\
13.3	0.01\\
13.31	0.01\\
13.32	0.01\\
13.33	0.01\\
13.34	0.01\\
13.35	0.01\\
13.36	0.01\\
13.37	0.01\\
13.38	0.01\\
13.39	0.01\\
13.4	0.01\\
13.41	0.01\\
13.42	0.01\\
13.43	0.01\\
13.44	0.01\\
13.45	0.01\\
13.46	0.01\\
13.47	0.01\\
13.48	0.01\\
13.49	0.01\\
13.5	0.01\\
13.51	0.01\\
13.52	0.01\\
13.53	0.01\\
13.54	0.01\\
13.55	0.01\\
13.56	0.01\\
13.57	0.01\\
13.58	0.01\\
13.59	0.01\\
13.6	0.01\\
13.61	0.01\\
13.62	0.01\\
13.63	0.01\\
13.64	0.01\\
13.65	0.01\\
13.66	0.01\\
13.67	0.01\\
13.68	0.01\\
13.69	0.01\\
13.7	0.01\\
13.71	0.01\\
13.72	0.01\\
13.73	0.01\\
13.74	0.01\\
13.75	0.01\\
13.76	0.01\\
13.77	0.01\\
13.78	0.01\\
13.79	0.01\\
13.8	0.01\\
13.81	0.01\\
13.82	0.01\\
13.83	0.01\\
13.84	0.01\\
13.85	0.01\\
13.86	0.01\\
13.87	0.01\\
13.88	0.01\\
13.89	0.01\\
13.9	0.01\\
13.91	0.01\\
13.92	0.01\\
13.93	0.01\\
13.94	0.01\\
13.95	0.01\\
13.96	0.01\\
13.97	0.01\\
13.98	0.01\\
13.99	0.01\\
14	0.01\\
14.01	0.01\\
14.02	0.01\\
14.03	0.01\\
14.04	0.01\\
14.05	0.01\\
14.06	0.01\\
14.07	0.01\\
14.08	0.01\\
14.09	0.01\\
14.1	0.01\\
14.11	0.01\\
14.12	0.01\\
14.13	0.01\\
14.14	0.01\\
14.15	0.01\\
14.16	0.01\\
14.17	0.01\\
14.18	0.01\\
14.19	0.01\\
14.2	0.01\\
14.21	0.01\\
14.22	0.01\\
14.23	0.01\\
14.24	0.01\\
14.25	0.01\\
14.26	0.01\\
14.27	0.01\\
14.28	0.01\\
14.29	0.01\\
14.3	0.01\\
14.31	0.01\\
14.32	0.01\\
14.33	0.01\\
14.34	0.01\\
14.35	0.01\\
14.36	0.01\\
14.37	0.01\\
14.38	0.01\\
14.39	0.01\\
14.4	0.01\\
14.41	0.01\\
14.42	0.01\\
14.43	0.01\\
14.44	0.01\\
14.45	0.01\\
14.46	0.01\\
14.47	0.01\\
14.48	0.01\\
14.49	0.01\\
14.5	0.01\\
14.51	0.01\\
14.52	0.01\\
14.53	0.01\\
14.54	0.01\\
14.55	0.01\\
14.56	0.01\\
14.57	0.01\\
14.58	0.01\\
14.59	0.01\\
14.6	0.01\\
14.61	0.01\\
14.62	0.01\\
14.63	0.01\\
14.64	0.01\\
14.65	0.01\\
14.66	0.01\\
14.67	0.01\\
14.68	0.01\\
14.69	0.01\\
14.7	0.01\\
14.71	0.01\\
14.72	0.01\\
14.73	0.01\\
14.74	0.01\\
14.75	0.01\\
14.76	0.01\\
14.77	0.01\\
14.78	0.01\\
14.79	0.01\\
14.8	0.01\\
14.81	0.01\\
14.82	0.01\\
14.83	0.01\\
14.84	0.01\\
14.85	0.01\\
14.86	0.01\\
14.87	0.01\\
14.88	0.01\\
14.89	0.01\\
14.9	0.01\\
14.91	0.01\\
14.92	0.01\\
14.93	0.01\\
14.94	0.01\\
14.95	0.01\\
14.96	0.01\\
14.97	0.01\\
14.98	0.01\\
14.99	0.01\\
15	0.01\\
15.01	0.01\\
15.02	0.01\\
15.03	0.01\\
15.04	0.01\\
15.05	0.01\\
15.06	0.01\\
15.07	0.01\\
15.08	0.01\\
15.09	0.01\\
15.1	0.01\\
15.11	0.01\\
15.12	0.01\\
15.13	0.01\\
15.14	0.01\\
15.15	0.01\\
15.16	0.01\\
15.17	0.01\\
15.18	0.01\\
15.19	0.01\\
15.2	0.01\\
15.21	0.01\\
15.22	0.01\\
15.23	0.01\\
15.24	0.01\\
15.25	0.01\\
15.26	0.01\\
15.27	0.01\\
15.28	0.01\\
15.29	0.01\\
15.3	0.01\\
15.31	0.01\\
15.32	0.01\\
15.33	0.01\\
15.34	0.01\\
15.35	0.01\\
15.36	0.01\\
15.37	0.01\\
15.38	0.01\\
15.39	0.01\\
15.4	0.01\\
15.41	0.01\\
15.42	0.01\\
15.43	0.01\\
15.44	0.01\\
15.45	0.01\\
15.46	0.01\\
15.47	0.01\\
15.48	0.01\\
15.49	0.01\\
15.5	0.01\\
15.51	0.01\\
15.52	0.01\\
15.53	0.01\\
15.54	0.01\\
15.55	0.01\\
15.56	0.01\\
15.57	0.01\\
15.58	0.01\\
15.59	0.01\\
15.6	0.01\\
15.61	0.01\\
15.62	0.01\\
15.63	0.01\\
15.64	0.01\\
15.65	0.01\\
15.66	0.01\\
15.67	0.01\\
15.68	0.01\\
15.69	0.01\\
15.7	0.01\\
15.71	0.01\\
15.72	0.01\\
15.73	0.01\\
15.74	0.01\\
15.75	0.01\\
15.76	0.01\\
15.77	0.01\\
15.78	0.01\\
15.79	0.01\\
15.8	0.01\\
15.81	0.01\\
15.82	0.01\\
15.83	0.01\\
15.84	0.01\\
15.85	0.01\\
15.86	0.01\\
15.87	0.01\\
15.88	0.01\\
15.89	0.01\\
15.9	0.01\\
15.91	0.01\\
15.92	0.01\\
15.93	0.01\\
15.94	0.01\\
15.95	0.01\\
15.96	0.01\\
15.97	0.01\\
15.98	0.01\\
15.99	0.01\\
16	0.01\\
16.01	0.01\\
16.02	0.01\\
16.03	0.01\\
16.04	0.01\\
16.05	0.01\\
16.06	0.01\\
16.07	0.01\\
16.08	0.01\\
16.09	0.01\\
16.1	0.01\\
16.11	0.01\\
16.12	0.01\\
16.13	0.01\\
16.14	0.01\\
16.15	0.01\\
16.16	0.01\\
16.17	0.01\\
16.18	0.01\\
16.19	0.01\\
16.2	0.01\\
16.21	0.01\\
16.22	0.01\\
16.23	0.01\\
16.24	0.01\\
16.25	0.01\\
16.26	0.01\\
16.27	0.01\\
16.28	0.01\\
16.29	0.01\\
16.3	0.01\\
16.31	0.01\\
16.32	0.01\\
16.33	0.01\\
16.34	0.01\\
16.35	0.01\\
16.36	0.01\\
16.37	0.01\\
16.38	0.01\\
16.39	0.01\\
16.4	0.01\\
16.41	0.01\\
16.42	0.01\\
16.43	0.01\\
16.44	0.01\\
16.45	0.01\\
16.46	0.01\\
16.47	0.01\\
16.48	0.01\\
16.49	0.01\\
16.5	0.01\\
16.51	0.01\\
16.52	0.01\\
16.53	0.01\\
16.54	0.01\\
16.55	0.01\\
16.56	0.01\\
16.57	0.01\\
16.58	0.01\\
16.59	0.01\\
16.6	0.01\\
16.61	0.01\\
16.62	0.01\\
16.63	0.01\\
16.64	0.01\\
16.65	0.01\\
16.66	0.01\\
16.67	0.01\\
16.68	0.01\\
16.69	0.01\\
16.7	0.01\\
16.71	0.01\\
16.72	0.01\\
16.73	0.01\\
16.74	0.01\\
16.75	0.01\\
16.76	0.01\\
16.77	0.01\\
16.78	0.01\\
16.79	0.01\\
16.8	0.01\\
16.81	0.01\\
16.82	0.01\\
16.83	0.01\\
16.84	0.01\\
16.85	0.01\\
16.86	0.01\\
16.87	0.01\\
16.88	0.01\\
16.89	0.01\\
16.9	0.01\\
16.91	0.01\\
16.92	0.01\\
16.93	0.01\\
16.94	0.01\\
16.95	0.01\\
16.96	0.01\\
16.97	0.01\\
16.98	0.01\\
16.99	0.01\\
17	0.01\\
17.01	0.01\\
17.02	0.01\\
17.03	0.01\\
17.04	0.01\\
17.05	0.01\\
17.06	0.01\\
17.07	0.01\\
17.08	0.01\\
17.09	0.01\\
17.1	0.01\\
17.11	0.01\\
17.12	0.01\\
17.13	0.01\\
17.14	0.01\\
17.15	0.01\\
17.16	0.01\\
17.17	0.01\\
17.18	0.01\\
17.19	0.01\\
17.2	0.01\\
17.21	0.01\\
17.22	0.01\\
17.23	0.01\\
17.24	0.01\\
17.25	0.01\\
17.26	0.01\\
17.27	0.01\\
17.28	0.01\\
17.29	0.01\\
17.3	0.01\\
17.31	0.01\\
17.32	0.01\\
17.33	0.01\\
17.34	0.01\\
17.35	0.01\\
17.36	0.01\\
17.37	0.01\\
17.38	0.01\\
17.39	0.01\\
17.4	0.01\\
17.41	0.01\\
17.42	0.01\\
17.43	0.01\\
17.44	0.01\\
17.45	0.01\\
17.46	0.01\\
17.47	0.01\\
17.48	0.01\\
17.49	0.01\\
17.5	0.01\\
17.51	0.01\\
17.52	0.01\\
17.53	0.01\\
17.54	0.01\\
17.55	0.01\\
17.56	0.01\\
17.57	0.01\\
17.58	0.01\\
17.59	0.01\\
17.6	0.01\\
17.61	0.01\\
17.62	0.01\\
17.63	0.01\\
17.64	0.01\\
17.65	0.01\\
17.66	0.01\\
17.67	0.01\\
17.68	0.01\\
17.69	0.01\\
17.7	0.01\\
17.71	0.01\\
17.72	0.01\\
17.73	0.01\\
17.74	0.01\\
17.75	0.01\\
17.76	0.01\\
17.77	0.01\\
17.78	0.01\\
17.79	0.01\\
17.8	0.01\\
17.81	0.01\\
17.82	0.01\\
17.83	0.01\\
17.84	0.01\\
17.85	0.01\\
17.86	0.01\\
17.87	0.01\\
17.88	0.01\\
17.89	0.01\\
17.9	0.01\\
17.91	0.01\\
17.92	0.01\\
17.93	0.01\\
17.94	0.01\\
17.95	0.01\\
17.96	0.01\\
17.97	0.01\\
17.98	0.01\\
17.99	0.01\\
18	0.01\\
18.01	0.01\\
18.02	0.01\\
18.03	0.01\\
18.04	0.01\\
18.05	0.01\\
18.06	0.01\\
18.07	0.01\\
18.08	0.01\\
18.09	0.01\\
18.1	0.01\\
18.11	0.01\\
18.12	0.01\\
18.13	0.01\\
18.14	0.01\\
18.15	0.01\\
18.16	0.01\\
18.17	0.01\\
18.18	0.01\\
18.19	0.01\\
18.2	0.01\\
18.21	0.01\\
18.22	0.01\\
18.23	0.01\\
18.24	0.01\\
18.25	0.01\\
18.26	0.01\\
18.27	0.01\\
18.28	0.01\\
18.29	0.01\\
18.3	0.01\\
18.31	0.01\\
18.32	0.01\\
18.33	0.01\\
18.34	0.01\\
18.35	0.01\\
18.36	0.01\\
18.37	0.01\\
18.38	0.01\\
18.39	0.01\\
18.4	0.01\\
18.41	0.01\\
18.42	0.01\\
18.43	0.01\\
18.44	0.01\\
18.45	0.01\\
18.46	0.01\\
18.47	0.01\\
18.48	0.01\\
18.49	0.01\\
18.5	0.01\\
18.51	0.01\\
18.52	0.01\\
18.53	0.01\\
18.54	0.01\\
18.55	0.01\\
18.56	0.01\\
18.57	0.01\\
18.58	0.01\\
18.59	0.01\\
18.6	0.01\\
18.61	0.01\\
18.62	0.01\\
18.63	0.01\\
18.64	0.01\\
18.65	0.01\\
18.66	0.01\\
18.67	0.01\\
18.68	0.01\\
18.69	0.01\\
18.7	0.01\\
18.71	0.01\\
18.72	0.01\\
18.73	0.01\\
18.74	0.01\\
18.75	0.01\\
18.76	0.01\\
18.77	0.01\\
18.78	0.01\\
18.79	0.01\\
18.8	0.01\\
18.81	0.01\\
18.82	0.01\\
18.83	0.01\\
18.84	0.01\\
18.85	0.01\\
18.86	0.01\\
18.87	0.01\\
18.88	0.01\\
18.89	0.01\\
18.9	0.01\\
18.91	0.01\\
18.92	0.01\\
18.93	0.01\\
18.94	0.01\\
18.95	0.01\\
18.96	0.01\\
18.97	0.01\\
18.98	0.01\\
18.99	0.01\\
19	0.01\\
19.01	0.01\\
19.02	0.01\\
19.03	0.01\\
19.04	0.01\\
19.05	0.01\\
19.06	0.01\\
19.07	0.01\\
19.08	0.01\\
19.09	0.01\\
19.1	0.01\\
19.11	0.01\\
19.12	0.01\\
19.13	0.01\\
19.14	0.01\\
19.15	0.01\\
19.16	0.01\\
19.17	0.01\\
19.18	0.01\\
19.19	0.01\\
19.2	0.01\\
19.21	0.01\\
19.22	0.01\\
19.23	0.01\\
19.24	0.01\\
19.25	0.01\\
19.26	0.01\\
19.27	0.01\\
19.28	0.01\\
19.29	0.01\\
19.3	0.01\\
19.31	0.01\\
19.32	0.01\\
19.33	0.01\\
19.34	0.01\\
19.35	0.01\\
19.36	0.01\\
19.37	0.01\\
19.38	0.01\\
19.39	0.01\\
19.4	0.01\\
19.41	0.01\\
19.42	0.01\\
19.43	0.01\\
19.44	0.01\\
19.45	0.01\\
19.46	0.01\\
19.47	0.01\\
19.48	0.01\\
19.49	0.01\\
19.5	0.01\\
19.51	0.01\\
19.52	0.01\\
19.53	0.01\\
19.54	0.01\\
19.55	0.01\\
19.56	0.01\\
19.57	0.01\\
19.58	0.01\\
19.59	0.01\\
19.6	0.01\\
19.61	0.01\\
19.62	0.01\\
19.63	0.01\\
19.64	0.01\\
19.65	0.01\\
19.66	0.01\\
19.67	0.01\\
19.68	0.01\\
19.69	0.01\\
19.7	0.01\\
19.71	0.01\\
19.72	0.01\\
19.73	0.01\\
19.74	0.01\\
19.75	0.01\\
19.76	0.01\\
19.77	0.01\\
19.78	0.01\\
19.79	0.01\\
19.8	0.01\\
19.81	0.01\\
19.82	0.01\\
19.83	0.01\\
19.84	0.01\\
19.85	0.01\\
19.86	0.01\\
19.87	0.01\\
19.88	0.01\\
19.89	0.01\\
19.9	0.01\\
19.91	0.01\\
19.92	0.01\\
19.93	0.01\\
19.94	0.01\\
19.95	0.01\\
19.96	0.01\\
19.97	0.01\\
19.98	0.01\\
19.99	0.01\\
20	0.01\\
20.01	0.01\\
20.02	0.01\\
20.03	0.01\\
20.04	0.01\\
20.05	0.01\\
20.06	0.01\\
20.07	0.01\\
20.08	0.01\\
20.09	0.01\\
20.1	0.01\\
20.11	0.01\\
20.12	0.01\\
20.13	0.01\\
20.14	0.01\\
20.15	0.01\\
20.16	0.01\\
20.17	0.01\\
20.18	0.01\\
20.19	0.01\\
20.2	0.01\\
20.21	0.01\\
20.22	0.01\\
20.23	0.01\\
20.24	0.01\\
20.25	0.01\\
20.26	0.01\\
20.27	0.01\\
20.28	0.01\\
20.29	0.01\\
20.3	0.01\\
20.31	0.01\\
20.32	0.01\\
20.33	0.01\\
20.34	0.01\\
20.35	0.01\\
20.36	0.01\\
20.37	0.01\\
20.38	0.01\\
20.39	0.01\\
20.4	0.01\\
20.41	0.01\\
20.42	0.01\\
20.43	0.01\\
20.44	0.01\\
20.45	0.01\\
20.46	0.01\\
20.47	0.01\\
20.48	0.01\\
20.49	0.01\\
20.5	0.01\\
20.51	0.01\\
20.52	0.01\\
20.53	0.01\\
20.54	0.01\\
20.55	0.01\\
20.56	0.01\\
20.57	0.01\\
20.58	0.01\\
20.59	0.01\\
20.6	0.01\\
20.61	0.01\\
20.62	0.01\\
20.63	0.01\\
20.64	0.01\\
20.65	0.01\\
20.66	0.01\\
20.67	0.01\\
20.68	0.01\\
20.69	0.01\\
20.7	0.01\\
20.71	0.01\\
20.72	0.01\\
20.73	0.01\\
20.74	0.01\\
20.75	0.01\\
20.76	0.01\\
20.77	0.01\\
20.78	0.01\\
20.79	0.01\\
20.8	0.01\\
20.81	0.01\\
20.82	0.01\\
20.83	0.01\\
20.84	0.01\\
20.85	0.01\\
20.86	0.01\\
20.87	0.01\\
20.88	0.01\\
20.89	0.01\\
20.9	0.01\\
20.91	0.01\\
20.92	0.01\\
20.93	0.01\\
20.94	0.01\\
20.95	0.01\\
20.96	0.01\\
20.97	0.01\\
20.98	0.01\\
20.99	0.01\\
21	0.01\\
21.01	0.01\\
21.02	0.01\\
21.03	0.01\\
21.04	0.01\\
21.05	0.01\\
21.06	0.01\\
21.07	0.01\\
21.08	0.01\\
21.09	0.01\\
21.1	0.01\\
21.11	0.01\\
21.12	0.01\\
21.13	0.01\\
21.14	0.01\\
21.15	0.01\\
21.16	0.01\\
21.17	0.01\\
21.18	0.01\\
21.19	0.01\\
21.2	0.01\\
21.21	0.01\\
21.22	0.01\\
21.23	0.01\\
21.24	0.01\\
21.25	0.01\\
21.26	0.01\\
21.27	0.01\\
21.28	0.01\\
21.29	0.01\\
21.3	0.01\\
21.31	0.01\\
21.32	0.01\\
21.33	0.01\\
21.34	0.01\\
21.35	0.01\\
21.36	0.01\\
21.37	0.01\\
21.38	0.01\\
21.39	0.01\\
21.4	0.01\\
21.41	0.01\\
21.42	0.01\\
21.43	0.01\\
21.44	0.01\\
21.45	0.01\\
21.46	0.01\\
21.47	0.01\\
21.48	0.01\\
21.49	0.01\\
21.5	0.01\\
21.51	0.01\\
21.52	0.01\\
21.53	0.01\\
21.54	0.01\\
21.55	0.01\\
21.56	0.01\\
21.57	0.01\\
21.58	0.01\\
21.59	0.01\\
21.6	0.01\\
21.61	0.01\\
21.62	0.01\\
21.63	0.01\\
21.64	0.01\\
21.65	0.01\\
21.66	0.01\\
21.67	0.01\\
21.68	0.01\\
21.69	0.01\\
21.7	0.01\\
21.71	0.01\\
21.72	0.01\\
21.73	0.01\\
21.74	0.01\\
21.75	0.01\\
21.76	0.01\\
21.77	0.01\\
21.78	0.01\\
21.79	0.01\\
21.8	0.01\\
21.81	0.01\\
21.82	0.01\\
21.83	0.01\\
21.84	0.01\\
21.85	0.01\\
21.86	0.01\\
21.87	0.01\\
21.88	0.01\\
21.89	0.01\\
21.9	0.01\\
21.91	0.01\\
21.92	0.01\\
21.93	0.01\\
21.94	0.01\\
21.95	0.01\\
21.96	0.01\\
21.97	0.01\\
21.98	0.01\\
21.99	0.01\\
22	0.01\\
22.01	0.01\\
22.02	0.01\\
22.03	0.01\\
22.04	0.01\\
22.05	0.01\\
22.06	0.01\\
22.07	0.01\\
22.08	0.01\\
22.09	0.01\\
22.1	0.01\\
22.11	0.01\\
22.12	0.01\\
22.13	0.01\\
22.14	0.01\\
22.15	0.01\\
22.16	0.01\\
22.17	0.01\\
22.18	0.01\\
22.19	0.01\\
22.2	0.01\\
22.21	0.01\\
22.22	0.01\\
22.23	0.01\\
22.24	0.01\\
22.25	0.01\\
22.26	0.01\\
22.27	0.01\\
22.28	0.01\\
22.29	0.01\\
22.3	0.01\\
22.31	0.01\\
22.32	0.01\\
22.33	0.01\\
22.34	0.01\\
22.35	0.01\\
22.36	0.01\\
22.37	0.01\\
22.38	0.01\\
22.39	0.01\\
22.4	0.01\\
22.41	0.01\\
22.42	0.01\\
22.43	0.01\\
22.44	0.01\\
22.45	0.01\\
22.46	0.01\\
22.47	0.01\\
22.48	0.01\\
22.49	0.01\\
22.5	0.01\\
22.51	0.01\\
22.52	0.01\\
22.53	0.01\\
22.54	0.01\\
22.55	0.01\\
22.56	0.01\\
22.57	0.01\\
22.58	0.01\\
22.59	0.01\\
22.6	0.01\\
22.61	0.01\\
22.62	0.01\\
22.63	0.01\\
22.64	0.01\\
22.65	0.01\\
22.66	0.01\\
22.67	0.01\\
22.68	0.01\\
22.69	0.01\\
22.7	0.01\\
22.71	0.01\\
22.72	0.01\\
22.73	0.01\\
22.74	0.01\\
22.75	0.01\\
22.76	0.01\\
22.77	0.01\\
22.78	0.01\\
22.79	0.01\\
22.8	0.01\\
22.81	0.01\\
22.82	0.01\\
22.83	0.01\\
22.84	0.01\\
22.85	0.01\\
22.86	0.01\\
22.87	0.01\\
22.88	0.01\\
22.89	0.01\\
22.9	0.01\\
22.91	0.01\\
22.92	0.01\\
22.93	0.01\\
22.94	0.01\\
22.95	0.01\\
22.96	0.01\\
22.97	0.01\\
22.98	0.01\\
22.99	0.01\\
23	0.01\\
23.01	0.01\\
23.02	0.01\\
23.03	0.01\\
23.04	0.01\\
23.05	0.01\\
23.06	0.01\\
23.07	0.01\\
23.08	0.01\\
23.09	0.01\\
23.1	0.01\\
23.11	0.01\\
23.12	0.01\\
23.13	0.01\\
23.14	0.01\\
23.15	0.01\\
23.16	0.01\\
23.17	0.01\\
23.18	0.01\\
23.19	0.01\\
23.2	0.01\\
23.21	0.01\\
23.22	0.01\\
23.23	0.01\\
23.24	0.01\\
23.25	0.01\\
23.26	0.01\\
23.27	0.01\\
23.28	0.01\\
23.29	0.01\\
23.3	0.01\\
23.31	0.01\\
23.32	0.01\\
23.33	0.01\\
23.34	0.01\\
23.35	0.01\\
23.36	0.01\\
23.37	0.01\\
23.38	0.01\\
23.39	0.01\\
23.4	0.01\\
23.41	0.01\\
23.42	0.01\\
23.43	0.01\\
23.44	0.01\\
23.45	0.01\\
23.46	0.01\\
23.47	0.01\\
23.48	0.01\\
23.49	0.01\\
23.5	0.01\\
23.51	0.01\\
23.52	0.01\\
23.53	0.01\\
23.54	0.01\\
23.55	0.01\\
23.56	0.01\\
23.57	0.01\\
23.58	0.01\\
23.59	0.01\\
23.6	0.01\\
23.61	0.01\\
23.62	0.01\\
23.63	0.01\\
23.64	0.01\\
23.65	0.01\\
23.66	0.01\\
23.67	0.01\\
23.68	0.01\\
23.69	0.01\\
23.7	0.01\\
23.71	0.01\\
23.72	0.01\\
23.73	0.01\\
23.74	0.01\\
23.75	0.01\\
23.76	0.01\\
23.77	0.01\\
23.78	0.01\\
23.79	0.01\\
23.8	0.01\\
23.81	0.01\\
23.82	0.01\\
23.83	0.01\\
23.84	0.01\\
23.85	0.01\\
23.86	0.01\\
23.87	0.01\\
23.88	0.01\\
23.89	0.01\\
23.9	0.01\\
23.91	0.01\\
23.92	0.01\\
23.93	0.01\\
23.94	0.01\\
23.95	0.01\\
23.96	0.01\\
23.97	0.01\\
23.98	0.01\\
23.99	0.01\\
24	0.01\\
24.01	0.01\\
24.02	0.01\\
24.03	0.01\\
24.04	0.01\\
24.05	0.01\\
24.06	0.01\\
24.07	0.01\\
24.08	0.01\\
24.09	0.01\\
24.1	0.01\\
24.11	0.01\\
24.12	0.01\\
24.13	0.01\\
24.14	0.01\\
24.15	0.01\\
24.16	0.01\\
24.17	0.01\\
24.18	0.01\\
24.19	0.01\\
24.2	0.01\\
24.21	0.01\\
24.22	0.01\\
24.23	0.01\\
24.24	0.01\\
24.25	0.01\\
24.26	0.01\\
24.27	0.01\\
24.28	0.01\\
24.29	0.01\\
24.3	0.01\\
24.31	0.01\\
24.32	0.01\\
24.33	0.01\\
24.34	0.01\\
24.35	0.01\\
24.36	0.01\\
24.37	0.01\\
24.38	0.01\\
24.39	0.01\\
24.4	0.01\\
24.41	0.01\\
24.42	0.01\\
24.43	0.01\\
24.44	0.01\\
24.45	0.01\\
24.46	0.01\\
24.47	0.01\\
24.48	0.01\\
24.49	0.01\\
24.5	0.01\\
24.51	0.01\\
24.52	0.01\\
24.53	0.01\\
24.54	0.01\\
24.55	0.01\\
24.56	0.01\\
24.57	0.01\\
24.58	0.01\\
24.59	0.01\\
24.6	0.01\\
24.61	0.01\\
24.62	0.01\\
24.63	0.01\\
24.64	0.01\\
24.65	0.01\\
24.66	0.01\\
24.67	0.01\\
24.68	0.01\\
24.69	0.01\\
24.7	0.01\\
24.71	0.01\\
24.72	0.01\\
24.73	0.01\\
24.74	0.01\\
24.75	0.01\\
24.76	0.01\\
24.77	0.01\\
24.78	0.01\\
24.79	0.01\\
24.8	0.01\\
24.81	0.01\\
24.82	0.01\\
24.83	0.01\\
24.84	0.01\\
24.85	0.01\\
24.86	0.01\\
24.87	0.01\\
24.88	0.01\\
24.89	0.01\\
24.9	0.01\\
24.91	0.01\\
24.92	0.01\\
24.93	0.01\\
24.94	0.01\\
24.95	0.01\\
24.96	0.01\\
24.97	0.01\\
24.98	0.01\\
24.99	0.01\\
25	0.01\\
25.01	0.01\\
25.02	0.01\\
25.03	0.01\\
25.04	0.01\\
25.05	0.01\\
25.06	0.01\\
25.07	0.01\\
25.08	0.01\\
25.09	0.01\\
25.1	0.01\\
25.11	0.01\\
25.12	0.01\\
25.13	0.01\\
25.14	0.01\\
25.15	0.01\\
25.16	0.01\\
25.17	0.01\\
25.18	0.01\\
25.19	0.01\\
25.2	0.01\\
25.21	0.01\\
25.22	0.01\\
25.23	0.01\\
25.24	0.01\\
25.25	0.01\\
25.26	0.01\\
25.27	0.01\\
25.28	0.01\\
25.29	0.01\\
25.3	0.01\\
25.31	0.01\\
25.32	0.01\\
25.33	0.01\\
25.34	0.01\\
25.35	0.01\\
25.36	0.01\\
25.37	0.01\\
25.38	0.01\\
25.39	0.01\\
25.4	0.01\\
25.41	0.01\\
25.42	0.01\\
25.43	0.01\\
25.44	0.01\\
25.45	0.01\\
25.46	0.01\\
25.47	0.01\\
25.48	0.01\\
25.49	0.01\\
25.5	0.01\\
25.51	0.01\\
25.52	0.01\\
25.53	0.01\\
25.54	0.01\\
25.55	0.01\\
25.56	0.01\\
25.57	0.01\\
25.58	0.01\\
25.59	0.01\\
25.6	0.01\\
25.61	0.01\\
25.62	0.01\\
25.63	0.01\\
25.64	0.01\\
25.65	0.01\\
25.66	0.01\\
25.67	0.01\\
25.68	0.01\\
25.69	0.01\\
25.7	0.01\\
25.71	0.01\\
25.72	0.01\\
25.73	0.01\\
25.74	0.01\\
25.75	0.01\\
25.76	0.01\\
25.77	0.01\\
25.78	0.01\\
25.79	0.01\\
25.8	0.01\\
25.81	0.01\\
25.82	0.01\\
25.83	0.01\\
25.84	0.01\\
25.85	0.01\\
25.86	0.01\\
25.87	0.01\\
25.88	0.01\\
25.89	0.01\\
25.9	0.01\\
25.91	0.01\\
25.92	0.01\\
25.93	0.01\\
25.94	0.01\\
25.95	0.01\\
25.96	0.01\\
25.97	0.01\\
25.98	0.01\\
25.99	0.01\\
26	0.01\\
26.01	0.01\\
26.02	0.01\\
26.03	0.01\\
26.04	0.01\\
26.05	0.01\\
26.06	0.01\\
26.07	0.01\\
26.08	0.01\\
26.09	0.01\\
26.1	0.01\\
26.11	0.01\\
26.12	0.01\\
26.13	0.01\\
26.14	0.01\\
26.15	0.01\\
26.16	0.01\\
26.17	0.01\\
26.18	0.01\\
26.19	0.01\\
26.2	0.01\\
26.21	0.01\\
26.22	0.01\\
26.23	0.01\\
26.24	0.01\\
26.25	0.01\\
26.26	0.01\\
26.27	0.01\\
26.28	0.01\\
26.29	0.01\\
26.3	0.01\\
26.31	0.01\\
26.32	0.01\\
26.33	0.01\\
26.34	0.01\\
26.35	0.01\\
26.36	0.01\\
26.37	0.01\\
26.38	0.01\\
26.39	0.01\\
26.4	0.01\\
26.41	0.01\\
26.42	0.01\\
26.43	0.01\\
26.44	0.01\\
26.45	0.01\\
26.46	0.01\\
26.47	0.01\\
26.48	0.01\\
26.49	0.01\\
26.5	0.01\\
26.51	0.01\\
26.52	0.01\\
26.53	0.01\\
26.54	0.01\\
26.55	0.01\\
26.56	0.01\\
26.57	0.01\\
26.58	0.01\\
26.59	0.01\\
26.6	0.01\\
26.61	0.01\\
26.62	0.01\\
26.63	0.01\\
26.64	0.01\\
26.65	0.01\\
26.66	0.01\\
26.67	0.01\\
26.68	0.01\\
26.69	0.01\\
26.7	0.01\\
26.71	0.01\\
26.72	0.01\\
26.73	0.01\\
26.74	0.01\\
26.75	0.01\\
26.76	0.01\\
26.77	0.01\\
26.78	0.01\\
26.79	0.01\\
26.8	0.01\\
26.81	0.01\\
26.82	0.01\\
26.83	0.01\\
26.84	0.01\\
26.85	0.01\\
26.86	0.01\\
26.87	0.01\\
26.88	0.01\\
26.89	0.01\\
26.9	0.01\\
26.91	0.01\\
26.92	0.01\\
26.93	0.01\\
26.94	0.01\\
26.95	0.01\\
26.96	0.01\\
26.97	0.01\\
26.98	0.01\\
26.99	0.01\\
27	0.01\\
27.01	0.01\\
27.02	0.01\\
27.03	0.01\\
27.04	0.01\\
27.05	0.01\\
27.06	0.01\\
27.07	0.01\\
27.08	0.01\\
27.09	0.01\\
27.1	0.01\\
27.11	0.01\\
27.12	0.01\\
27.13	0.01\\
27.14	0.01\\
27.15	0.01\\
27.16	0.01\\
27.17	0.01\\
27.18	0.01\\
27.19	0.01\\
27.2	0.01\\
27.21	0.01\\
27.22	0.01\\
27.23	0.01\\
27.24	0.01\\
27.25	0.01\\
27.26	0.01\\
27.27	0.01\\
27.28	0.01\\
27.29	0.01\\
27.3	0.01\\
27.31	0.01\\
27.32	0.01\\
27.33	0.01\\
27.34	0.01\\
27.35	0.01\\
27.36	0.01\\
27.37	0.01\\
27.38	0.01\\
27.39	0.01\\
27.4	0.01\\
27.41	0.01\\
27.42	0.01\\
27.43	0.01\\
27.44	0.01\\
27.45	0.01\\
27.46	0.01\\
27.47	0.01\\
27.48	0.01\\
27.49	0.01\\
27.5	0.01\\
27.51	0.01\\
27.52	0.01\\
27.53	0.01\\
27.54	0.01\\
27.55	0.01\\
27.56	0.01\\
27.57	0.01\\
27.58	0.01\\
27.59	0.01\\
27.6	0.01\\
27.61	0.01\\
27.62	0.01\\
27.63	0.01\\
27.64	0.01\\
27.65	0.01\\
27.66	0.01\\
27.67	0.01\\
27.68	0.01\\
27.69	0.01\\
27.7	0.01\\
27.71	0.01\\
27.72	0.01\\
27.73	0.01\\
27.74	0.01\\
27.75	0.01\\
27.76	0.01\\
27.77	0.01\\
27.78	0.01\\
27.79	0.01\\
27.8	0.01\\
27.81	0.01\\
27.82	0.01\\
27.83	0.01\\
27.84	0.01\\
27.85	0.01\\
27.86	0.01\\
27.87	0.01\\
27.88	0.01\\
27.89	0.01\\
27.9	0.01\\
27.91	0.01\\
27.92	0.01\\
27.93	0.01\\
27.94	0.01\\
27.95	0.01\\
27.96	0.01\\
27.97	0.01\\
27.98	0.01\\
27.99	0.01\\
28	0.01\\
28.01	0.01\\
28.02	0.01\\
28.03	0.01\\
28.04	0.01\\
28.05	0.01\\
28.06	0.01\\
28.07	0.01\\
28.08	0.01\\
28.09	0.01\\
28.1	0.01\\
28.11	0.01\\
28.12	0.01\\
28.13	0.01\\
28.14	0.01\\
28.15	0.01\\
28.16	0.01\\
28.17	0.01\\
28.18	0.01\\
28.19	0.01\\
28.2	0.01\\
28.21	0.01\\
28.22	0.01\\
28.23	0.01\\
28.24	0.01\\
28.25	0.01\\
28.26	0.01\\
28.27	0.01\\
28.28	0.01\\
28.29	0.01\\
28.3	0.01\\
28.31	0.01\\
28.32	0.01\\
28.33	0.01\\
28.34	0.01\\
28.35	0.01\\
28.36	0.01\\
28.37	0.01\\
28.38	0.01\\
28.39	0.01\\
28.4	0.01\\
28.41	0.01\\
28.42	0.01\\
28.43	0.01\\
28.44	0.01\\
28.45	0.01\\
28.46	0.01\\
28.47	0.01\\
28.48	0.01\\
28.49	0.01\\
28.5	0.01\\
28.51	0.01\\
28.52	0.01\\
28.53	0.01\\
28.54	0.01\\
28.55	0.01\\
28.56	0.01\\
28.57	0.01\\
28.58	0.01\\
28.59	0.01\\
28.6	0.01\\
28.61	0.01\\
28.62	0.01\\
28.63	0.01\\
28.64	0.01\\
28.65	0.01\\
28.66	0.01\\
28.67	0.01\\
28.68	0.01\\
28.69	0.01\\
28.7	0.01\\
28.71	0.01\\
28.72	0.01\\
28.73	0.01\\
28.74	0.01\\
28.75	0.01\\
28.76	0.01\\
28.77	0.01\\
28.78	0.01\\
28.79	0.01\\
28.8	0.01\\
28.81	0.01\\
28.82	0.01\\
28.83	0.01\\
28.84	0.01\\
28.85	0.01\\
28.86	0.01\\
28.87	0.01\\
28.88	0.01\\
28.89	0.01\\
28.9	0.01\\
28.91	0.01\\
28.92	0.01\\
28.93	0.01\\
28.94	0.01\\
28.95	0.01\\
28.96	0.01\\
28.97	0.01\\
28.98	0.01\\
28.99	0.01\\
29	0.01\\
29.01	0.01\\
29.02	0.01\\
29.03	0.01\\
29.04	0.01\\
29.05	0.01\\
29.06	0.01\\
29.07	0.01\\
29.08	0.01\\
29.09	0.01\\
29.1	0.01\\
29.11	0.01\\
29.12	0.01\\
29.13	0.01\\
29.14	0.01\\
29.15	0.01\\
29.16	0.01\\
29.17	0.01\\
29.18	0.01\\
29.19	0.01\\
29.2	0.01\\
29.21	0.01\\
29.22	0.01\\
29.23	0.01\\
29.24	0.01\\
29.25	0.01\\
29.26	0.01\\
29.27	0.01\\
29.28	0.01\\
29.29	0.01\\
29.3	0.01\\
29.31	0.01\\
29.32	0.01\\
29.33	0.01\\
29.34	0.01\\
29.35	0.01\\
29.36	0.01\\
29.37	0.01\\
29.38	0.01\\
29.39	0.01\\
29.4	0.01\\
29.41	0.01\\
29.42	0.01\\
29.43	0.01\\
29.44	0.01\\
29.45	0.01\\
29.46	0.01\\
29.47	0.01\\
29.48	0.01\\
29.49	0.01\\
29.5	0.01\\
29.51	0.01\\
29.52	0.01\\
29.53	0.01\\
29.54	0.01\\
29.55	0.01\\
29.56	0.01\\
29.57	0.01\\
29.58	0.01\\
29.59	0.01\\
29.6	0.01\\
29.61	0.01\\
29.62	0.01\\
29.63	0.01\\
29.64	0.01\\
29.65	0.01\\
29.66	0.01\\
29.67	0.01\\
29.68	0.01\\
29.69	0.01\\
29.7	0.01\\
29.71	0.01\\
29.72	0.01\\
29.73	0.01\\
29.74	0.01\\
29.75	0.01\\
29.76	0.01\\
29.77	0.01\\
29.78	0.01\\
29.79	0.01\\
29.8	0.01\\
29.81	0.01\\
29.82	0.01\\
29.83	0.01\\
29.84	0.01\\
29.85	0.01\\
29.86	0.01\\
29.87	0.01\\
29.88	0.01\\
29.89	0.01\\
29.9	0.01\\
29.91	0.01\\
29.92	0.01\\
29.93	0.01\\
29.94	0.01\\
29.95	0.01\\
29.96	0.01\\
29.97	0.01\\
29.98	0.01\\
29.99	0.01\\
30	0.01\\
30.01	0.01\\
30.02	0.01\\
30.03	0.01\\
30.04	0.01\\
30.05	0.01\\
30.06	0.01\\
30.07	0.01\\
30.08	0.01\\
30.09	0.01\\
30.1	0.01\\
30.11	0.01\\
30.12	0.01\\
30.13	0.01\\
30.14	0.01\\
30.15	0.01\\
30.16	0.01\\
30.17	0.01\\
30.18	0.01\\
30.19	0.01\\
30.2	0.01\\
30.21	0.01\\
30.22	0.01\\
30.23	0.01\\
30.24	0.01\\
30.25	0.01\\
30.26	0.01\\
30.27	0.01\\
30.28	0.01\\
30.29	0.01\\
30.3	0.01\\
30.31	0.01\\
30.32	0.01\\
30.33	0.01\\
30.34	0.01\\
30.35	0.01\\
30.36	0.01\\
30.37	0.01\\
30.38	0.01\\
30.39	0.01\\
30.4	0.01\\
30.41	0.01\\
30.42	0.01\\
30.43	0.01\\
30.44	0.01\\
30.45	0.01\\
30.46	0.01\\
30.47	0.01\\
30.48	0.01\\
30.49	0.01\\
30.5	0.01\\
30.51	0.01\\
30.52	0.01\\
30.53	0.01\\
30.54	0.01\\
30.55	0.01\\
30.56	0.01\\
30.57	0.01\\
30.58	0.01\\
30.59	0.01\\
30.6	0.01\\
30.61	0.01\\
30.62	0.01\\
30.63	0.01\\
30.64	0.01\\
30.65	0.01\\
30.66	0.01\\
30.67	0.01\\
30.68	0.01\\
30.69	0.01\\
30.7	0.01\\
30.71	0.01\\
30.72	0.01\\
30.73	0.01\\
30.74	0.01\\
30.75	0.01\\
30.76	0.01\\
30.77	0.01\\
30.78	0.01\\
30.79	0.01\\
30.8	0.01\\
30.81	0.01\\
30.82	0.01\\
30.83	0.01\\
30.84	0.01\\
30.85	0.01\\
30.86	0.01\\
30.87	0.01\\
30.88	0.01\\
30.89	0.01\\
30.9	0.01\\
30.91	0.01\\
30.92	0.01\\
30.93	0.01\\
30.94	0.01\\
30.95	0.01\\
30.96	0.01\\
30.97	0.01\\
30.98	0.01\\
30.99	0.01\\
31	0.01\\
31.01	0.01\\
31.02	0.01\\
31.03	0.01\\
31.04	0.01\\
31.05	0.01\\
31.06	0.01\\
31.07	0.01\\
31.08	0.01\\
31.09	0.01\\
31.1	0.01\\
31.11	0.01\\
31.12	0.01\\
31.13	0.01\\
31.14	0.01\\
31.15	0.01\\
31.16	0.01\\
31.17	0.01\\
31.18	0.01\\
31.19	0.01\\
31.2	0.01\\
31.21	0.01\\
31.22	0.01\\
31.23	0.01\\
31.24	0.01\\
31.25	0.01\\
31.26	0.01\\
31.27	0.01\\
31.28	0.01\\
31.29	0.01\\
31.3	0.01\\
31.31	0.01\\
31.32	0.01\\
31.33	0.01\\
31.34	0.01\\
31.35	0.01\\
31.36	0.01\\
31.37	0.01\\
31.38	0.01\\
31.39	0.01\\
31.4	0.01\\
31.41	0.01\\
31.42	0.01\\
31.43	0.01\\
31.44	0.01\\
31.45	0.01\\
31.46	0.01\\
31.47	0.01\\
31.48	0.01\\
31.49	0.01\\
31.5	0.01\\
31.51	0.01\\
31.52	0.01\\
31.53	0.01\\
31.54	0.01\\
31.55	0.01\\
31.56	0.01\\
31.57	0.01\\
31.58	0.01\\
31.59	0.01\\
31.6	0.01\\
31.61	0.01\\
31.62	0.01\\
31.63	0.01\\
31.64	0.01\\
31.65	0.01\\
31.66	0.01\\
31.67	0.01\\
31.68	0.01\\
31.69	0.01\\
31.7	0.01\\
31.71	0.01\\
31.72	0.01\\
31.73	0.01\\
31.74	0.01\\
31.75	0.01\\
31.76	0.01\\
31.77	0.01\\
31.78	0.01\\
31.79	0.01\\
31.8	0.01\\
31.81	0.01\\
31.82	0.01\\
31.83	0.01\\
31.84	0.01\\
31.85	0.01\\
31.86	0.01\\
31.87	0.01\\
31.88	0.01\\
31.89	0.01\\
31.9	0.01\\
31.91	0.01\\
31.92	0.01\\
31.93	0.01\\
31.94	0.01\\
31.95	0.01\\
31.96	0.01\\
31.97	0.01\\
31.98	0.01\\
31.99	0.01\\
32	0.01\\
32.01	0.01\\
32.02	0.01\\
32.03	0.01\\
32.04	0.01\\
32.05	0.01\\
32.06	0.01\\
32.07	0.01\\
32.08	0.01\\
32.09	0.01\\
32.1	0.01\\
32.11	0.01\\
32.12	0.01\\
32.13	0.01\\
32.14	0.01\\
32.15	0.01\\
32.16	0.01\\
32.17	0.01\\
32.18	0.01\\
32.19	0.01\\
32.2	0.01\\
32.21	0.01\\
32.22	0.01\\
32.23	0.01\\
32.24	0.01\\
32.25	0.01\\
32.26	0.01\\
32.27	0.01\\
32.28	0.01\\
32.29	0.01\\
32.3	0.01\\
32.31	0.01\\
32.32	0.01\\
32.33	0.01\\
32.34	0.01\\
32.35	0.01\\
32.36	0.01\\
32.37	0.01\\
32.38	0.01\\
32.39	0.01\\
32.4	0.01\\
32.41	0.01\\
32.42	0.01\\
32.43	0.01\\
32.44	0.01\\
32.45	0.01\\
32.46	0.01\\
32.47	0.01\\
32.48	0.01\\
32.49	0.01\\
32.5	0.01\\
32.51	0.01\\
32.52	0.01\\
32.53	0.01\\
32.54	0.01\\
32.55	0.01\\
32.56	0.01\\
32.57	0.01\\
32.58	0.01\\
32.59	0.01\\
32.6	0.01\\
32.61	0.01\\
32.62	0.01\\
32.63	0.01\\
32.64	0.01\\
32.65	0.01\\
32.66	0.01\\
32.67	0.01\\
32.68	0.01\\
32.69	0.01\\
32.7	0.01\\
32.71	0.01\\
32.72	0.01\\
32.73	0.01\\
32.74	0.01\\
32.75	0.01\\
32.76	0.01\\
32.77	0.01\\
32.78	0.01\\
32.79	0.01\\
32.8	0.01\\
32.81	0.01\\
32.82	0.01\\
32.83	0.01\\
32.84	0.01\\
32.85	0.01\\
32.86	0.01\\
32.87	0.01\\
32.88	0.01\\
32.89	0.01\\
32.9	0.01\\
32.91	0.01\\
32.92	0.01\\
32.93	0.01\\
32.94	0.01\\
32.95	0.01\\
32.96	0.01\\
32.97	0.01\\
32.98	0.01\\
32.99	0.01\\
33	0.01\\
33.01	0.01\\
33.02	0.01\\
33.03	0.01\\
33.04	0.01\\
33.05	0.01\\
33.06	0.01\\
33.07	0.01\\
33.08	0.01\\
33.09	0.01\\
33.1	0.01\\
33.11	0.01\\
33.12	0.01\\
33.13	0.01\\
33.14	0.01\\
33.15	0.01\\
33.16	0.01\\
33.17	0.01\\
33.18	0.01\\
33.19	0.01\\
33.2	0.01\\
33.21	0.01\\
33.22	0.01\\
33.23	0.01\\
33.24	0.01\\
33.25	0.01\\
33.26	0.01\\
33.27	0.01\\
33.28	0.01\\
33.29	0.01\\
33.3	0.01\\
33.31	0.01\\
33.32	0.01\\
33.33	0.01\\
33.34	0.01\\
33.35	0.01\\
33.36	0.01\\
33.37	0.01\\
33.38	0.01\\
33.39	0.01\\
33.4	0.01\\
33.41	0.01\\
33.42	0.01\\
33.43	0.01\\
33.44	0.01\\
33.45	0.01\\
33.46	0.01\\
33.47	0.01\\
33.48	0.01\\
33.49	0.01\\
33.5	0.01\\
33.51	0.01\\
33.52	0.01\\
33.53	0.01\\
33.54	0.01\\
33.55	0.01\\
33.56	0.01\\
33.57	0.01\\
33.58	0.01\\
33.59	0.01\\
33.6	0.01\\
33.61	0.01\\
33.62	0.01\\
33.63	0.01\\
33.64	0.01\\
33.65	0.01\\
33.66	0.01\\
33.67	0.01\\
33.68	0.01\\
33.69	0.01\\
33.7	0.01\\
33.71	0.01\\
33.72	0.01\\
33.73	0.01\\
33.74	0.01\\
33.75	0.01\\
33.76	0.01\\
33.77	0.01\\
33.78	0.01\\
33.79	0.01\\
33.8	0.01\\
33.81	0.01\\
33.82	0.01\\
33.83	0.01\\
33.84	0.01\\
33.85	0.01\\
33.86	0.01\\
33.87	0.01\\
33.88	0.01\\
33.89	0.01\\
33.9	0.01\\
33.91	0.01\\
33.92	0.01\\
33.93	0.01\\
33.94	0.01\\
33.95	0.01\\
33.96	0.01\\
33.97	0.01\\
33.98	0.01\\
33.99	0.01\\
34	0.01\\
34.01	0.01\\
34.02	0.01\\
34.03	0.01\\
34.04	0.01\\
34.05	0.01\\
34.06	0.01\\
34.07	0.01\\
34.08	0.01\\
34.09	0.01\\
34.1	0.01\\
34.11	0.01\\
34.12	0.01\\
34.13	0.01\\
34.14	0.01\\
34.15	0.01\\
34.16	0.01\\
34.17	0.01\\
34.18	0.01\\
34.19	0.01\\
34.2	0.01\\
34.21	0.01\\
34.22	0.01\\
34.23	0.01\\
34.24	0.01\\
34.25	0.01\\
34.26	0.01\\
34.27	0.01\\
34.28	0.01\\
34.29	0.01\\
34.3	0.01\\
34.31	0.01\\
34.32	0.01\\
34.33	0.01\\
34.34	0.01\\
34.35	0.01\\
34.36	0.01\\
34.37	0.01\\
34.38	0.01\\
34.39	0.01\\
34.4	0.01\\
34.41	0.01\\
34.42	0.01\\
34.43	0.01\\
34.44	0.01\\
34.45	0.01\\
34.46	0.01\\
34.47	0.01\\
34.48	0.01\\
34.49	0.01\\
34.5	0.01\\
34.51	0.01\\
34.52	0.01\\
34.53	0.01\\
34.54	0.01\\
34.55	0.01\\
34.56	0.01\\
34.57	0.01\\
34.58	0.01\\
34.59	0.01\\
34.6	0.01\\
34.61	0.01\\
34.62	0.01\\
34.63	0.01\\
34.64	0.01\\
34.65	0.01\\
34.66	0.01\\
34.67	0.01\\
34.68	0.01\\
34.69	0.01\\
34.7	0.01\\
34.71	0.01\\
34.72	0.01\\
34.73	0.01\\
34.74	0.01\\
34.75	0.01\\
34.76	0.01\\
34.77	0.01\\
34.78	0.01\\
34.79	0.01\\
34.8	0.01\\
34.81	0.01\\
34.82	0.01\\
34.83	0.01\\
34.84	0.01\\
34.85	0.01\\
34.86	0.01\\
34.87	0.01\\
34.88	0.01\\
34.89	0.01\\
34.9	0.01\\
34.91	0.01\\
34.92	0.01\\
34.93	0.01\\
34.94	0.01\\
34.95	0.01\\
34.96	0.01\\
34.97	0.01\\
34.98	0.01\\
34.99	0.01\\
35	0.01\\
35.01	0.01\\
35.02	0.01\\
35.03	0.01\\
35.04	0.01\\
35.05	0.01\\
35.06	0.01\\
35.07	0.01\\
35.08	0.01\\
35.09	0.01\\
35.1	0.01\\
35.11	0.01\\
35.12	0.01\\
35.13	0.01\\
35.14	0.01\\
35.15	0.01\\
35.16	0.01\\
35.17	0.01\\
35.18	0.01\\
35.19	0.01\\
35.2	0.01\\
35.21	0.01\\
35.22	0.01\\
35.23	0.01\\
35.24	0.01\\
35.25	0.01\\
35.26	0.01\\
35.27	0.01\\
35.28	0.01\\
35.29	0.01\\
35.3	0.01\\
35.31	0.01\\
35.32	0.01\\
35.33	0.01\\
35.34	0.01\\
35.35	0.01\\
35.36	0.01\\
35.37	0.01\\
35.38	0.01\\
35.39	0.01\\
35.4	0.01\\
35.41	0.01\\
35.42	0.01\\
35.43	0.01\\
35.44	0.01\\
35.45	0.01\\
35.46	0.01\\
35.47	0.01\\
35.48	0.01\\
35.49	0.01\\
35.5	0.01\\
35.51	0.01\\
35.52	0.01\\
35.53	0.01\\
35.54	0.01\\
35.55	0.01\\
35.56	0.01\\
35.57	0.01\\
35.58	0.01\\
35.59	0.01\\
35.6	0.01\\
35.61	0.01\\
35.62	0.01\\
35.63	0.01\\
35.64	0.01\\
35.65	0.01\\
35.66	0.01\\
35.67	0.01\\
35.68	0.01\\
35.69	0.01\\
35.7	0.01\\
35.71	0.01\\
35.72	0.01\\
35.73	0.01\\
35.74	0.01\\
35.75	0.01\\
35.76	0.01\\
35.77	0.01\\
35.78	0.01\\
35.79	0.01\\
35.8	0.01\\
35.81	0.01\\
35.82	0.01\\
35.83	0.01\\
35.84	0.01\\
35.85	0.01\\
35.86	0.01\\
35.87	0.01\\
35.88	0.01\\
35.89	0.01\\
35.9	0.01\\
35.91	0.01\\
35.92	0.01\\
35.93	0.01\\
35.94	0.01\\
35.95	0.01\\
35.96	0.01\\
35.97	0.01\\
35.98	0.01\\
35.99	0.01\\
36	0.01\\
36.01	0.01\\
36.02	0.01\\
36.03	0.01\\
36.04	0.01\\
36.05	0.01\\
36.06	0.01\\
36.07	0.01\\
36.08	0.01\\
36.09	0.01\\
36.1	0.01\\
36.11	0.01\\
36.12	0.01\\
36.13	0.01\\
36.14	0.01\\
36.15	0.01\\
36.16	0.01\\
36.17	0.01\\
36.18	0.01\\
36.19	0.01\\
36.2	0.01\\
36.21	0.01\\
36.22	0.01\\
36.23	0.01\\
36.24	0.01\\
36.25	0.01\\
36.26	0.01\\
36.27	0.01\\
36.28	0.01\\
36.29	0.01\\
36.3	0.01\\
36.31	0.01\\
36.32	0.01\\
36.33	0.01\\
36.34	0.01\\
36.35	0.01\\
36.36	0.01\\
36.37	0.01\\
36.38	0.01\\
36.39	0.01\\
36.4	0.01\\
36.41	0.01\\
36.42	0.01\\
36.43	0.01\\
36.44	0.01\\
36.45	0.01\\
36.46	0.01\\
36.47	0.01\\
36.48	0.01\\
36.49	0.01\\
36.5	0.01\\
36.51	0.01\\
36.52	0.01\\
36.53	0.01\\
36.54	0.01\\
36.55	0.01\\
36.56	0.01\\
36.57	0.01\\
36.58	0.01\\
36.59	0.01\\
36.6	0.01\\
36.61	0.01\\
36.62	0.01\\
36.63	0.01\\
36.64	0.01\\
36.65	0.01\\
36.66	0.01\\
36.67	0.01\\
36.68	0.01\\
36.69	0.01\\
36.7	0.01\\
36.71	0.01\\
36.72	0.01\\
36.73	0.01\\
36.74	0.01\\
36.75	0.01\\
36.76	0.01\\
36.77	0.01\\
36.78	0.01\\
36.79	0.01\\
36.8	0.01\\
36.81	0.01\\
36.82	0.01\\
36.83	0.01\\
36.84	0.01\\
36.85	0.01\\
36.86	0.01\\
36.87	0.01\\
36.88	0.01\\
36.89	0.01\\
36.9	0.01\\
36.91	0.01\\
36.92	0.01\\
36.93	0.01\\
36.94	0.01\\
36.95	0.01\\
36.96	0.01\\
36.97	0.01\\
36.98	0.01\\
36.99	0.01\\
37	0.01\\
37.01	0.01\\
37.02	0.01\\
37.03	0.01\\
37.04	0.01\\
37.05	0.01\\
37.06	0.01\\
37.07	0.01\\
37.08	0.01\\
37.09	0.01\\
37.1	0.01\\
37.11	0.01\\
37.12	0.01\\
37.13	0.01\\
37.14	0.01\\
37.15	0.01\\
37.16	0.01\\
37.17	0.01\\
37.18	0.01\\
37.19	0.01\\
37.2	0.01\\
37.21	0.01\\
37.22	0.01\\
37.23	0.01\\
37.24	0.01\\
37.25	0.01\\
37.26	0.01\\
37.27	0.01\\
37.28	0.01\\
37.29	0.01\\
37.3	0.01\\
37.31	0.01\\
37.32	0.01\\
37.33	0.01\\
37.34	0.01\\
37.35	0.01\\
37.36	0.01\\
37.37	0.01\\
37.38	0.01\\
37.39	0.01\\
37.4	0.01\\
37.41	0.01\\
37.42	0.01\\
37.43	0.01\\
37.44	0.01\\
37.45	0.01\\
37.46	0.01\\
37.47	0.01\\
37.48	0.01\\
37.49	0.01\\
37.5	0.01\\
37.51	0.01\\
37.52	0.01\\
37.53	0.01\\
37.54	0.01\\
37.55	0.01\\
37.56	0.01\\
37.57	0.01\\
37.58	0.01\\
37.59	0.01\\
37.6	0.01\\
37.61	0.01\\
37.62	0.01\\
37.63	0.01\\
37.64	0.01\\
37.65	0.01\\
37.66	0.01\\
37.67	0.01\\
37.68	0.01\\
37.69	0.01\\
37.7	0.01\\
37.71	0.01\\
37.72	0.01\\
37.73	0.01\\
37.74	0.01\\
37.75	0.01\\
37.76	0.01\\
37.77	0.01\\
37.78	0.01\\
37.79	0.01\\
37.8	0.01\\
37.81	0.01\\
37.82	0.01\\
37.83	0.01\\
37.84	0.01\\
37.85	0.01\\
37.86	0.01\\
37.87	0.01\\
37.88	0.01\\
37.89	0.01\\
37.9	0.01\\
37.91	0.01\\
37.92	0.01\\
37.93	0.01\\
37.94	0.01\\
37.95	0.01\\
37.96	0.01\\
37.97	0.01\\
37.98	0.01\\
37.99	0.01\\
38	0.01\\
38.01	0.01\\
38.02	0.01\\
38.03	0.01\\
38.04	0.01\\
38.05	0.01\\
38.06	0.01\\
38.07	0.01\\
38.08	0.01\\
38.09	0.01\\
38.1	0.01\\
38.11	0.01\\
38.12	0.01\\
38.13	0.01\\
38.14	0.01\\
38.15	0.01\\
38.16	0.01\\
38.17	0.01\\
38.18	0.01\\
38.19	0.01\\
38.2	0.01\\
38.21	0.01\\
38.22	0.01\\
38.23	0.01\\
38.24	0.01\\
38.25	0.01\\
38.26	0.01\\
38.27	0.01\\
38.28	0.01\\
38.29	0.01\\
38.3	0.01\\
38.31	0.01\\
38.32	0.01\\
38.33	0.01\\
38.34	0.01\\
38.35	0.01\\
38.36	0.01\\
38.37	0.01\\
38.38	0.01\\
38.39	0.01\\
38.4	0.01\\
38.41	0.01\\
38.42	0.01\\
38.43	0.01\\
38.44	0.01\\
38.45	0.01\\
38.46	0.01\\
38.47	0.01\\
38.48	0.01\\
38.49	0.01\\
38.5	0.01\\
38.51	0.01\\
38.52	0.01\\
38.53	0.01\\
38.54	0.01\\
38.55	0.01\\
38.56	0.01\\
38.57	0.01\\
38.58	0.01\\
38.59	0.01\\
38.6	0.01\\
38.61	0.01\\
38.62	0.01\\
38.63	0.01\\
38.64	0.01\\
38.65	0.01\\
38.66	0.01\\
38.67	0.01\\
38.68	0.01\\
38.69	0.01\\
38.7	0.01\\
38.71	0.01\\
38.72	0.01\\
38.73	0.01\\
38.74	0.01\\
38.75	0.01\\
38.76	0.01\\
38.77	0.01\\
38.78	0.01\\
38.79	0.01\\
38.8	0.01\\
38.81	0.01\\
38.82	0.01\\
38.83	0.01\\
38.84	0.01\\
38.85	0.01\\
38.86	0.01\\
38.87	0.01\\
38.88	0.01\\
38.89	0.01\\
38.9	0.01\\
38.91	0.01\\
38.92	0.01\\
38.93	0.01\\
38.94	0.01\\
38.95	0.01\\
38.96	0.01\\
38.97	0.01\\
38.98	0.01\\
38.99	0.01\\
39	0.01\\
39.01	0.01\\
39.02	0.01\\
39.03	0.01\\
39.04	0.01\\
39.05	0.01\\
39.06	0.01\\
39.07	0.01\\
39.08	0.01\\
39.09	0.01\\
39.1	0.01\\
39.11	0.01\\
39.12	0.01\\
39.13	0.01\\
39.14	0.01\\
39.15	0.01\\
39.16	0.01\\
39.17	0.01\\
39.18	0.01\\
39.19	0.01\\
39.2	0.01\\
39.21	0.01\\
39.22	0.01\\
39.23	0.01\\
39.24	0.01\\
39.25	0.01\\
39.26	0.01\\
39.27	0.01\\
39.28	0.01\\
39.29	0.01\\
39.3	0.01\\
39.31	0.01\\
39.32	0.01\\
39.33	0.01\\
39.34	0.01\\
39.35	0.01\\
39.36	0.01\\
39.37	0.01\\
39.38	0.01\\
39.39	0.01\\
39.4	0.01\\
39.41	0.01\\
39.42	0.01\\
39.43	0.01\\
39.44	0.01\\
39.45	0.01\\
39.46	0.01\\
39.47	0.01\\
39.48	0.01\\
39.49	0.01\\
39.5	0.01\\
39.51	0.01\\
39.52	0.01\\
39.53	0.01\\
39.54	0.01\\
39.55	0.01\\
39.56	0.01\\
39.57	0.01\\
39.58	0.01\\
39.59	0.01\\
39.6	0.01\\
39.61	0.01\\
39.62	0.01\\
39.63	0.01\\
39.64	0.01\\
39.65	0.01\\
39.66	0.01\\
39.67	0.01\\
39.68	0.01\\
39.69	0.01\\
39.7	0.01\\
39.71	0.01\\
39.72	0.01\\
39.73	0.01\\
39.74	0.01\\
39.75	0.01\\
39.76	0.01\\
39.77	0.01\\
39.78	0.01\\
39.79	0.01\\
39.8	0.01\\
39.81	0.01\\
39.82	0.01\\
39.83	0.01\\
39.84	0.01\\
39.85	0.01\\
39.86	0.01\\
39.87	0.01\\
39.88	0.01\\
39.89	0.01\\
39.9	0.01\\
39.91	0.01\\
39.92	0.01\\
39.93	0.01\\
39.94	0.01\\
39.95	0.01\\
39.96	0.01\\
39.97	0.01\\
39.98	0.01\\
39.99	0.01\\
40	0.01\\
40.01	0.01\\
};
\addplot [color=mycolor1,dashed,forget plot]
  table[row sep=crcr]{%
40.01	0.01\\
40.02	0.01\\
40.03	0.01\\
40.04	0.01\\
40.05	0.01\\
40.06	0.01\\
40.07	0.01\\
40.08	0.01\\
40.09	0.01\\
40.1	0.01\\
40.11	0.01\\
40.12	0.01\\
40.13	0.01\\
40.14	0.01\\
40.15	0.01\\
40.16	0.01\\
40.17	0.01\\
40.18	0.01\\
40.19	0.01\\
40.2	0.01\\
40.21	0.01\\
40.22	0.01\\
40.23	0.01\\
40.24	0.01\\
40.25	0.01\\
40.26	0.01\\
40.27	0.01\\
40.28	0.01\\
40.29	0.01\\
40.3	0.01\\
40.31	0.01\\
40.32	0.01\\
40.33	0.01\\
40.34	0.01\\
40.35	0.01\\
40.36	0.01\\
40.37	0.01\\
40.38	0.01\\
40.39	0.01\\
40.4	0.01\\
40.41	0.01\\
40.42	0.01\\
40.43	0.01\\
40.44	0.01\\
40.45	0.01\\
40.46	0.01\\
40.47	0.01\\
40.48	0.01\\
40.49	0.01\\
40.5	0.01\\
40.51	0.01\\
40.52	0.01\\
40.53	0.01\\
40.54	0.01\\
40.55	0.01\\
40.56	0.01\\
40.57	0.01\\
40.58	0.01\\
40.59	0.01\\
40.6	0.01\\
40.61	0.01\\
40.62	0.01\\
40.63	0.01\\
40.64	0.01\\
40.65	0.01\\
40.66	0.01\\
40.67	0.01\\
40.68	0.01\\
40.69	0.01\\
40.7	0.01\\
40.71	0.01\\
40.72	0.01\\
40.73	0.01\\
40.74	0.01\\
40.75	0.01\\
40.76	0.01\\
40.77	0.01\\
40.78	0.01\\
40.79	0.01\\
40.8	0.01\\
40.81	0.01\\
40.82	0.01\\
40.83	0.01\\
40.84	0.01\\
40.85	0.01\\
40.86	0.01\\
40.87	0.01\\
40.88	0.01\\
40.89	0.01\\
40.9	0.01\\
40.91	0.01\\
40.92	0.01\\
40.93	0.01\\
40.94	0.01\\
40.95	0.01\\
40.96	0.01\\
40.97	0.01\\
40.98	0.01\\
40.99	0.01\\
41	0.01\\
41.01	0.01\\
41.02	0.01\\
41.03	0.01\\
41.04	0.01\\
41.05	0.01\\
41.06	0.01\\
41.07	0.01\\
41.08	0.01\\
41.09	0.01\\
41.1	0.01\\
41.11	0.01\\
41.12	0.01\\
41.13	0.01\\
41.14	0.01\\
41.15	0.01\\
41.16	0.01\\
41.17	0.01\\
41.18	0.01\\
41.19	0.01\\
41.2	0.01\\
41.21	0.01\\
41.22	0.01\\
41.23	0.01\\
41.24	0.01\\
41.25	0.01\\
41.26	0.01\\
41.27	0.01\\
41.28	0.01\\
41.29	0.01\\
41.3	0.01\\
41.31	0.01\\
41.32	0.01\\
41.33	0.01\\
41.34	0.01\\
41.35	0.01\\
41.36	0.01\\
41.37	0.01\\
41.38	0.01\\
41.39	0.01\\
41.4	0.01\\
41.41	0.01\\
41.42	0.01\\
41.43	0.01\\
41.44	0.01\\
41.45	0.01\\
41.46	0.01\\
41.47	0.01\\
41.48	0.01\\
41.49	0.01\\
41.5	0.01\\
41.51	0.01\\
41.52	0.01\\
41.53	0.01\\
41.54	0.01\\
41.55	0.01\\
41.56	0.01\\
41.57	0.01\\
41.58	0.01\\
41.59	0.01\\
41.6	0.01\\
41.61	0.01\\
41.62	0.01\\
41.63	0.01\\
41.64	0.01\\
41.65	0.01\\
41.66	0.01\\
41.67	0.01\\
41.68	0.01\\
41.69	0.01\\
41.7	0.01\\
41.71	0.01\\
41.72	0.01\\
41.73	0.01\\
41.74	0.01\\
41.75	0.01\\
41.76	0.01\\
41.77	0.01\\
41.78	0.01\\
41.79	0.01\\
41.8	0.01\\
41.81	0.01\\
41.82	0.01\\
41.83	0.01\\
41.84	0.01\\
41.85	0.01\\
41.86	0.01\\
41.87	0.01\\
41.88	0.01\\
41.89	0.01\\
41.9	0.01\\
41.91	0.01\\
41.92	0.01\\
41.93	0.01\\
41.94	0.01\\
41.95	0.01\\
41.96	0.01\\
41.97	0.01\\
41.98	0.01\\
41.99	0.01\\
42	0.01\\
42.01	0.01\\
42.02	0.01\\
42.03	0.01\\
42.04	0.01\\
42.05	0.01\\
42.06	0.01\\
42.07	0.01\\
42.08	0.01\\
42.09	0.01\\
42.1	0.01\\
42.11	0.01\\
42.12	0.01\\
42.13	0.01\\
42.14	0.01\\
42.15	0.01\\
42.16	0.01\\
42.17	0.01\\
42.18	0.01\\
42.19	0.01\\
42.2	0.01\\
42.21	0.01\\
42.22	0.01\\
42.23	0.01\\
42.24	0.01\\
42.25	0.01\\
42.26	0.01\\
42.27	0.01\\
42.28	0.01\\
42.29	0.01\\
42.3	0.01\\
42.31	0.01\\
42.32	0.01\\
42.33	0.01\\
42.34	0.01\\
42.35	0.01\\
42.36	0.01\\
42.37	0.01\\
42.38	0.01\\
42.39	0.01\\
42.4	0.01\\
42.41	0.01\\
42.42	0.01\\
42.43	0.01\\
42.44	0.01\\
42.45	0.01\\
42.46	0.01\\
42.47	0.01\\
42.48	0.01\\
42.49	0.01\\
42.5	0.01\\
42.51	0.01\\
42.52	0.01\\
42.53	0.01\\
42.54	0.01\\
42.55	0.01\\
42.56	0.01\\
42.57	0.01\\
42.58	0.01\\
42.59	0.01\\
42.6	0.01\\
42.61	0.01\\
42.62	0.01\\
42.63	0.01\\
42.64	0.01\\
42.65	0.01\\
42.66	0.01\\
42.67	0.01\\
42.68	0.01\\
42.69	0.01\\
42.7	0.01\\
42.71	0.01\\
42.72	0.01\\
42.73	0.01\\
42.74	0.01\\
42.75	0.01\\
42.76	0.01\\
42.77	0.01\\
42.78	0.01\\
42.79	0.01\\
42.8	0.01\\
42.81	0.01\\
42.82	0.01\\
42.83	0.01\\
42.84	0.01\\
42.85	0.01\\
42.86	0.01\\
42.87	0.01\\
42.88	0.01\\
42.89	0.01\\
42.9	0.01\\
42.91	0.01\\
42.92	0.01\\
42.93	0.01\\
42.94	0.01\\
42.95	0.01\\
42.96	0.01\\
42.97	0.01\\
42.98	0.01\\
42.99	0.01\\
43	0.01\\
43.01	0.01\\
43.02	0.01\\
43.03	0.01\\
43.04	0.01\\
43.05	0.01\\
43.06	0.01\\
43.07	0.01\\
43.08	0.01\\
43.09	0.01\\
43.1	0.01\\
43.11	0.01\\
43.12	0.01\\
43.13	0.01\\
43.14	0.01\\
43.15	0.01\\
43.16	0.01\\
43.17	0.01\\
43.18	0.01\\
43.19	0.01\\
43.2	0.01\\
43.21	0.01\\
43.22	0.01\\
43.23	0.01\\
43.24	0.01\\
43.25	0.01\\
43.26	0.01\\
43.27	0.01\\
43.28	0.01\\
43.29	0.01\\
43.3	0.01\\
43.31	0.01\\
43.32	0.01\\
43.33	0.01\\
43.34	0.01\\
43.35	0.01\\
43.36	0.01\\
43.37	0.01\\
43.38	0.01\\
43.39	0.01\\
43.4	0.01\\
43.41	0.01\\
43.42	0.01\\
43.43	0.01\\
43.44	0.01\\
43.45	0.01\\
43.46	0.01\\
43.47	0.01\\
43.48	0.01\\
43.49	0.01\\
43.5	0.01\\
43.51	0.01\\
43.52	0.01\\
43.53	0.01\\
43.54	0.01\\
43.55	0.01\\
43.56	0.01\\
43.57	0.01\\
43.58	0.01\\
43.59	0.01\\
43.6	0.01\\
43.61	0.01\\
43.62	0.01\\
43.63	0.01\\
43.64	0.01\\
43.65	0.01\\
43.66	0.01\\
43.67	0.01\\
43.68	0.01\\
43.69	0.01\\
43.7	0.01\\
43.71	0.01\\
43.72	0.01\\
43.73	0.01\\
43.74	0.01\\
43.75	0.01\\
43.76	0.01\\
43.77	0.01\\
43.78	0.01\\
43.79	0.01\\
43.8	0.01\\
43.81	0.01\\
43.82	0.01\\
43.83	0.01\\
43.84	0.01\\
43.85	0.01\\
43.86	0.01\\
43.87	0.01\\
43.88	0.01\\
43.89	0.01\\
43.9	0.01\\
43.91	0.01\\
43.92	0.01\\
43.93	0.01\\
43.94	0.01\\
43.95	0.01\\
43.96	0.01\\
43.97	0.01\\
43.98	0.01\\
43.99	0.01\\
44	0.01\\
44.01	0.01\\
44.02	0.01\\
44.03	0.01\\
44.04	0.01\\
44.05	0.01\\
44.06	0.01\\
44.07	0.01\\
44.08	0.01\\
44.09	0.01\\
44.1	0.01\\
44.11	0.01\\
44.12	0.01\\
44.13	0.01\\
44.14	0.01\\
44.15	0.01\\
44.16	0.01\\
44.17	0.01\\
44.18	0.01\\
44.19	0.01\\
44.2	0.01\\
44.21	0.01\\
44.22	0.01\\
44.23	0.01\\
44.24	0.01\\
44.25	0.01\\
44.26	0.01\\
44.27	0.01\\
44.28	0.01\\
44.29	0.01\\
44.3	0.01\\
44.31	0.01\\
44.32	0.01\\
44.33	0.01\\
44.34	0.01\\
44.35	0.01\\
44.36	0.01\\
44.37	0.01\\
44.38	0.01\\
44.39	0.01\\
44.4	0.01\\
44.41	0.01\\
44.42	0.01\\
44.43	0.01\\
44.44	0.01\\
44.45	0.01\\
44.46	0.01\\
44.47	0.01\\
44.48	0.01\\
44.49	0.01\\
44.5	0.01\\
44.51	0.01\\
44.52	0.01\\
44.53	0.01\\
44.54	0.01\\
44.55	0.01\\
44.56	0.01\\
44.57	0.01\\
44.58	0.01\\
44.59	0.01\\
44.6	0.01\\
44.61	0.01\\
44.62	0.01\\
44.63	0.01\\
44.64	0.01\\
44.65	0.01\\
44.66	0.01\\
44.67	0.01\\
44.68	0.01\\
44.69	0.01\\
44.7	0.01\\
44.71	0.01\\
44.72	0.01\\
44.73	0.01\\
44.74	0.01\\
44.75	0.01\\
44.76	0.01\\
44.77	0.01\\
44.78	0.01\\
44.79	0.01\\
44.8	0.01\\
44.81	0.01\\
44.82	0.01\\
44.83	0.01\\
44.84	0.01\\
44.85	0.01\\
44.86	0.01\\
44.87	0.01\\
44.88	0.01\\
44.89	0.01\\
44.9	0.01\\
44.91	0.01\\
44.92	0.01\\
44.93	0.01\\
44.94	0.01\\
44.95	0.01\\
44.96	0.01\\
44.97	0.01\\
44.98	0.01\\
44.99	0.01\\
45	0.01\\
45.01	0.01\\
45.02	0.01\\
45.03	0.01\\
45.04	0.01\\
45.05	0.01\\
45.06	0.01\\
45.07	0.01\\
45.08	0.01\\
45.09	0.01\\
45.1	0.01\\
45.11	0.01\\
45.12	0.01\\
45.13	0.01\\
45.14	0.01\\
45.15	0.01\\
45.16	0.01\\
45.17	0.01\\
45.18	0.01\\
45.19	0.01\\
45.2	0.01\\
45.21	0.01\\
45.22	0.01\\
45.23	0.01\\
45.24	0.01\\
45.25	0.01\\
45.26	0.01\\
45.27	0.01\\
45.28	0.01\\
45.29	0.01\\
45.3	0.01\\
45.31	0.01\\
45.32	0.01\\
45.33	0.01\\
45.34	0.01\\
45.35	0.01\\
45.36	0.01\\
45.37	0.01\\
45.38	0.01\\
45.39	0.01\\
45.4	0.01\\
45.41	0.01\\
45.42	0.01\\
45.43	0.01\\
45.44	0.01\\
45.45	0.01\\
45.46	0.01\\
45.47	0.01\\
45.48	0.01\\
45.49	0.01\\
45.5	0.01\\
45.51	0.01\\
45.52	0.01\\
45.53	0.01\\
45.54	0.01\\
45.55	0.01\\
45.56	0.01\\
45.57	0.01\\
45.58	0.01\\
45.59	0.01\\
45.6	0.01\\
45.61	0.01\\
45.62	0.01\\
45.63	0.01\\
45.64	0.01\\
45.65	0.01\\
45.66	0.01\\
45.67	0.01\\
45.68	0.01\\
45.69	0.01\\
45.7	0.01\\
45.71	0.01\\
45.72	0.01\\
45.73	0.01\\
45.74	0.01\\
45.75	0.01\\
45.76	0.01\\
45.77	0.01\\
45.78	0.01\\
45.79	0.01\\
45.8	0.01\\
45.81	0.01\\
45.82	0.01\\
45.83	0.01\\
45.84	0.01\\
45.85	0.01\\
45.86	0.01\\
45.87	0.01\\
45.88	0.01\\
45.89	0.01\\
45.9	0.01\\
45.91	0.01\\
45.92	0.01\\
45.93	0.01\\
45.94	0.01\\
45.95	0.01\\
45.96	0.01\\
45.97	0.01\\
45.98	0.01\\
45.99	0.01\\
46	0.01\\
46.01	0.01\\
46.02	0.01\\
46.03	0.01\\
46.04	0.01\\
46.05	0.01\\
46.06	0.01\\
46.07	0.01\\
46.08	0.01\\
46.09	0.01\\
46.1	0.01\\
46.11	0.01\\
46.12	0.01\\
46.13	0.01\\
46.14	0.01\\
46.15	0.01\\
46.16	0.01\\
46.17	0.01\\
46.18	0.01\\
46.19	0.01\\
46.2	0.01\\
46.21	0.01\\
46.22	0.01\\
46.23	0.01\\
46.24	0.01\\
46.25	0.01\\
46.26	0.01\\
46.27	0.01\\
46.28	0.01\\
46.29	0.01\\
46.3	0.01\\
46.31	0.01\\
46.32	0.01\\
46.33	0.01\\
46.34	0.01\\
46.35	0.01\\
46.36	0.01\\
46.37	0.01\\
46.38	0.01\\
46.39	0.01\\
46.4	0.01\\
46.41	0.01\\
46.42	0.01\\
46.43	0.01\\
46.44	0.01\\
46.45	0.01\\
46.46	0.01\\
46.47	0.01\\
46.48	0.01\\
46.49	0.01\\
46.5	0.01\\
46.51	0.01\\
46.52	0.01\\
46.53	0.01\\
46.54	0.01\\
46.55	0.01\\
46.56	0.01\\
46.57	0.01\\
46.58	0.01\\
46.59	0.01\\
46.6	0.01\\
46.61	0.01\\
46.62	0.01\\
46.63	0.01\\
46.64	0.01\\
46.65	0.01\\
46.66	0.01\\
46.67	0.01\\
46.68	0.01\\
46.69	0.01\\
46.7	0.01\\
46.71	0.01\\
46.72	0.01\\
46.73	0.01\\
46.74	0.01\\
46.75	0.01\\
46.76	0.01\\
46.77	0.01\\
46.78	0.01\\
46.79	0.01\\
46.8	0.01\\
46.81	0.01\\
46.82	0.01\\
46.83	0.01\\
46.84	0.01\\
46.85	0.01\\
46.86	0.01\\
46.87	0.01\\
46.88	0.01\\
46.89	0.01\\
46.9	0.01\\
46.91	0.01\\
46.92	0.01\\
46.93	0.01\\
46.94	0.01\\
46.95	0.01\\
46.96	0.01\\
46.97	0.01\\
46.98	0.01\\
46.99	0.01\\
47	0.01\\
47.01	0.01\\
47.02	0.01\\
47.03	0.01\\
47.04	0.01\\
47.05	0.01\\
47.06	0.01\\
47.07	0.01\\
47.08	0.01\\
47.09	0.01\\
47.1	0.01\\
47.11	0.01\\
47.12	0.01\\
47.13	0.01\\
47.14	0.01\\
47.15	0.01\\
47.16	0.01\\
47.17	0.01\\
47.18	0.01\\
47.19	0.01\\
47.2	0.01\\
47.21	0.01\\
47.22	0.01\\
47.23	0.01\\
47.24	0.01\\
47.25	0.01\\
47.26	0.01\\
47.27	0.01\\
47.28	0.01\\
47.29	0.01\\
47.3	0.01\\
47.31	0.01\\
47.32	0.01\\
47.33	0.01\\
47.34	0.01\\
47.35	0.01\\
47.36	0.01\\
47.37	0.01\\
47.38	0.01\\
47.39	0.01\\
47.4	0.01\\
47.41	0.01\\
47.42	0.01\\
47.43	0.01\\
47.44	0.01\\
47.45	0.01\\
47.46	0.01\\
47.47	0.01\\
47.48	0.01\\
47.49	0.01\\
47.5	0.01\\
47.51	0.01\\
47.52	0.01\\
47.53	0.01\\
47.54	0.01\\
47.55	0.01\\
47.56	0.01\\
47.57	0.01\\
47.58	0.01\\
47.59	0.01\\
47.6	0.01\\
47.61	0.01\\
47.62	0.01\\
47.63	0.01\\
47.64	0.01\\
47.65	0.01\\
47.66	0.01\\
47.67	0.01\\
47.68	0.01\\
47.69	0.01\\
47.7	0.01\\
47.71	0.01\\
47.72	0.01\\
47.73	0.01\\
47.74	0.01\\
47.75	0.01\\
47.76	0.01\\
47.77	0.01\\
47.78	0.01\\
47.79	0.01\\
47.8	0.01\\
47.81	0.01\\
47.82	0.01\\
47.83	0.01\\
47.84	0.01\\
47.85	0.01\\
47.86	0.01\\
47.87	0.01\\
47.88	0.01\\
47.89	0.01\\
47.9	0.01\\
47.91	0.01\\
47.92	0.01\\
47.93	0.01\\
47.94	0.01\\
47.95	0.01\\
47.96	0.01\\
47.97	0.01\\
47.98	0.01\\
47.99	0.01\\
48	0.01\\
48.01	0.01\\
48.02	0.01\\
48.03	0.01\\
48.04	0.01\\
48.05	0.01\\
48.06	0.01\\
48.07	0.01\\
48.08	0.01\\
48.09	0.01\\
48.1	0.01\\
48.11	0.01\\
48.12	0.01\\
48.13	0.01\\
48.14	0.01\\
48.15	0.01\\
48.16	0.01\\
48.17	0.01\\
48.18	0.01\\
48.19	0.01\\
48.2	0.01\\
48.21	0.01\\
48.22	0.01\\
48.23	0.01\\
48.24	0.01\\
48.25	0.01\\
48.26	0.01\\
48.27	0.01\\
48.28	0.01\\
48.29	0.01\\
48.3	0.01\\
48.31	0.01\\
48.32	0.01\\
48.33	0.01\\
48.34	0.01\\
48.35	0.01\\
48.36	0.01\\
48.37	0.01\\
48.38	0.01\\
48.39	0.01\\
48.4	0.01\\
48.41	0.01\\
48.42	0.01\\
48.43	0.01\\
48.44	0.01\\
48.45	0.01\\
48.46	0.01\\
48.47	0.01\\
48.48	0.01\\
48.49	0.01\\
48.5	0.01\\
48.51	0.01\\
48.52	0.01\\
48.53	0.01\\
48.54	0.01\\
48.55	0.01\\
48.56	0.01\\
48.57	0.01\\
48.58	0.01\\
48.59	0.01\\
48.6	0.01\\
48.61	0.01\\
48.62	0.01\\
48.63	0.01\\
48.64	0.01\\
48.65	0.01\\
48.66	0.01\\
48.67	0.01\\
48.68	0.01\\
48.69	0.01\\
48.7	0.01\\
48.71	0.01\\
48.72	0.01\\
48.73	0.01\\
48.74	0.01\\
48.75	0.01\\
48.76	0.01\\
48.77	0.01\\
48.78	0.01\\
48.79	0.01\\
48.8	0.01\\
48.81	0.01\\
48.82	0.01\\
48.83	0.01\\
48.84	0.01\\
48.85	0.01\\
48.86	0.01\\
48.87	0.01\\
48.88	0.01\\
48.89	0.01\\
48.9	0.01\\
48.91	0.01\\
48.92	0.01\\
48.93	0.01\\
48.94	0.01\\
48.95	0.01\\
48.96	0.01\\
48.97	0.01\\
48.98	0.01\\
48.99	0.01\\
49	0.01\\
49.01	0.01\\
49.02	0.01\\
49.03	0.01\\
49.04	0.01\\
49.05	0.01\\
49.06	0.01\\
49.07	0.01\\
49.08	0.01\\
49.09	0.01\\
49.1	0.01\\
49.11	0.01\\
49.12	0.01\\
49.13	0.01\\
49.14	0.01\\
49.15	0.01\\
49.16	0.01\\
49.17	0.01\\
49.18	0.01\\
49.19	0.01\\
49.2	0.01\\
49.21	0.01\\
49.22	0.01\\
49.23	0.01\\
49.24	0.01\\
49.25	0.01\\
49.26	0.01\\
49.27	0.01\\
49.28	0.01\\
49.29	0.01\\
49.3	0.01\\
49.31	0.01\\
49.32	0.01\\
49.33	0.01\\
49.34	0.01\\
49.35	0.01\\
49.36	0.01\\
49.37	0.01\\
49.38	0.01\\
49.39	0.01\\
49.4	0.01\\
49.41	0.01\\
49.42	0.01\\
49.43	0.01\\
49.44	0.01\\
49.45	0.01\\
49.46	0.01\\
49.47	0.01\\
49.48	0.01\\
49.49	0.01\\
49.5	0.01\\
49.51	0.01\\
49.52	0.01\\
49.53	0.01\\
49.54	0.01\\
49.55	0.01\\
49.56	0.01\\
49.57	0.01\\
49.58	0.01\\
49.59	0.01\\
49.6	0.01\\
49.61	0.01\\
49.62	0.01\\
49.63	0.01\\
49.64	0.01\\
49.65	0.01\\
49.66	0.01\\
49.67	0.01\\
49.68	0.01\\
49.69	0.01\\
49.7	0.01\\
49.71	0.01\\
49.72	0.01\\
49.73	0.01\\
49.74	0.01\\
49.75	0.01\\
49.76	0.01\\
49.77	0.01\\
49.78	0.01\\
49.79	0.01\\
49.8	0.01\\
49.81	0.01\\
49.82	0.01\\
49.83	0.01\\
49.84	0.01\\
49.85	0.01\\
49.86	0.01\\
49.87	0.01\\
49.88	0.01\\
49.89	0.01\\
49.9	0.01\\
49.91	0.01\\
49.92	0.01\\
49.93	0.01\\
49.94	0.01\\
49.95	0.01\\
49.96	0.01\\
49.97	0.01\\
49.98	0.01\\
49.99	0.01\\
50	0.01\\
50.01	0.01\\
50.02	0.01\\
50.03	0.01\\
50.04	0.01\\
50.05	0.01\\
50.06	0.01\\
50.07	0.01\\
50.08	0.01\\
50.09	0.01\\
50.1	0.01\\
50.11	0.01\\
50.12	0.01\\
50.13	0.01\\
50.14	0.01\\
50.15	0.01\\
50.16	0.01\\
50.17	0.01\\
50.18	0.01\\
50.19	0.01\\
50.2	0.01\\
50.21	0.01\\
50.22	0.01\\
50.23	0.01\\
50.24	0.01\\
50.25	0.01\\
50.26	0.01\\
50.27	0.01\\
50.28	0.01\\
50.29	0.01\\
50.3	0.01\\
50.31	0.01\\
50.32	0.01\\
50.33	0.01\\
50.34	0.01\\
50.35	0.01\\
50.36	0.01\\
50.37	0.01\\
50.38	0.01\\
50.39	0.01\\
50.4	0.01\\
50.41	0.01\\
50.42	0.01\\
50.43	0.01\\
50.44	0.01\\
50.45	0.01\\
50.46	0.01\\
50.47	0.01\\
50.48	0.01\\
50.49	0.01\\
50.5	0.01\\
50.51	0.01\\
50.52	0.01\\
50.53	0.01\\
50.54	0.01\\
50.55	0.01\\
50.56	0.01\\
50.57	0.01\\
50.58	0.01\\
50.59	0.01\\
50.6	0.01\\
50.61	0.01\\
50.62	0.01\\
50.63	0.01\\
50.64	0.01\\
50.65	0.01\\
50.66	0.01\\
50.67	0.01\\
50.68	0.01\\
50.69	0.01\\
50.7	0.01\\
50.71	0.01\\
50.72	0.01\\
50.73	0.01\\
50.74	0.01\\
50.75	0.01\\
50.76	0.01\\
50.77	0.01\\
50.78	0.01\\
50.79	0.01\\
50.8	0.01\\
50.81	0.01\\
50.82	0.01\\
50.83	0.01\\
50.84	0.01\\
50.85	0.01\\
50.86	0.01\\
50.87	0.01\\
50.88	0.01\\
50.89	0.01\\
50.9	0.01\\
50.91	0.01\\
50.92	0.01\\
50.93	0.01\\
50.94	0.01\\
50.95	0.01\\
50.96	0.01\\
50.97	0.01\\
50.98	0.01\\
50.99	0.01\\
51	0.01\\
51.01	0.01\\
51.02	0.01\\
51.03	0.01\\
51.04	0.01\\
51.05	0.01\\
51.06	0.01\\
51.07	0.01\\
51.08	0.01\\
51.09	0.01\\
51.1	0.01\\
51.11	0.01\\
51.12	0.01\\
51.13	0.01\\
51.14	0.01\\
51.15	0.01\\
51.16	0.01\\
51.17	0.01\\
51.18	0.01\\
51.19	0.01\\
51.2	0.01\\
51.21	0.01\\
51.22	0.01\\
51.23	0.01\\
51.24	0.01\\
51.25	0.01\\
51.26	0.01\\
51.27	0.01\\
51.28	0.01\\
51.29	0.01\\
51.3	0.01\\
51.31	0.01\\
51.32	0.01\\
51.33	0.01\\
51.34	0.01\\
51.35	0.01\\
51.36	0.01\\
51.37	0.01\\
51.38	0.01\\
51.39	0.01\\
51.4	0.01\\
51.41	0.01\\
51.42	0.01\\
51.43	0.01\\
51.44	0.01\\
51.45	0.01\\
51.46	0.01\\
51.47	0.01\\
51.48	0.01\\
51.49	0.01\\
51.5	0.01\\
51.51	0.01\\
51.52	0.01\\
51.53	0.01\\
51.54	0.01\\
51.55	0.01\\
51.56	0.01\\
51.57	0.01\\
51.58	0.01\\
51.59	0.01\\
51.6	0.01\\
51.61	0.01\\
51.62	0.01\\
51.63	0.01\\
51.64	0.01\\
51.65	0.01\\
51.66	0.01\\
51.67	0.01\\
51.68	0.01\\
51.69	0.01\\
51.7	0.01\\
51.71	0.01\\
51.72	0.01\\
51.73	0.01\\
51.74	0.01\\
51.75	0.01\\
51.76	0.01\\
51.77	0.01\\
51.78	0.01\\
51.79	0.01\\
51.8	0.01\\
51.81	0.01\\
51.82	0.01\\
51.83	0.01\\
51.84	0.01\\
51.85	0.01\\
51.86	0.01\\
51.87	0.01\\
51.88	0.01\\
51.89	0.01\\
51.9	0.01\\
51.91	0.01\\
51.92	0.01\\
51.93	0.01\\
51.94	0.01\\
51.95	0.01\\
51.96	0.01\\
51.97	0.01\\
51.98	0.01\\
51.99	0.01\\
52	0.01\\
52.01	0.01\\
52.02	0.01\\
52.03	0.01\\
52.04	0.01\\
52.05	0.01\\
52.06	0.01\\
52.07	0.01\\
52.08	0.01\\
52.09	0.01\\
52.1	0.01\\
52.11	0.01\\
52.12	0.01\\
52.13	0.01\\
52.14	0.01\\
52.15	0.01\\
52.16	0.01\\
52.17	0.01\\
52.18	0.01\\
52.19	0.01\\
52.2	0.01\\
52.21	0.01\\
52.22	0.01\\
52.23	0.01\\
52.24	0.01\\
52.25	0.01\\
52.26	0.01\\
52.27	0.01\\
52.28	0.01\\
52.29	0.01\\
52.3	0.01\\
52.31	0.01\\
52.32	0.01\\
52.33	0.01\\
52.34	0.01\\
52.35	0.01\\
52.36	0.01\\
52.37	0.01\\
52.38	0.01\\
52.39	0.01\\
52.4	0.01\\
52.41	0.01\\
52.42	0.01\\
52.43	0.01\\
52.44	0.01\\
52.45	0.01\\
52.46	0.01\\
52.47	0.01\\
52.48	0.01\\
52.49	0.01\\
52.5	0.01\\
52.51	0.01\\
52.52	0.01\\
52.53	0.01\\
52.54	0.01\\
52.55	0.01\\
52.56	0.01\\
52.57	0.01\\
52.58	0.01\\
52.59	0.01\\
52.6	0.01\\
52.61	0.01\\
52.62	0.01\\
52.63	0.01\\
52.64	0.01\\
52.65	0.01\\
52.66	0.01\\
52.67	0.01\\
52.68	0.01\\
52.69	0.01\\
52.7	0.01\\
52.71	0.01\\
52.72	0.01\\
52.73	0.01\\
52.74	0.01\\
52.75	0.01\\
52.76	0.01\\
52.77	0.01\\
52.78	0.01\\
52.79	0.01\\
52.8	0.01\\
52.81	0.01\\
52.82	0.01\\
52.83	0.01\\
52.84	0.01\\
52.85	0.01\\
52.86	0.01\\
52.87	0.01\\
52.88	0.01\\
52.89	0.01\\
52.9	0.01\\
52.91	0.01\\
52.92	0.01\\
52.93	0.01\\
52.94	0.01\\
52.95	0.01\\
52.96	0.01\\
52.97	0.01\\
52.98	0.01\\
52.99	0.01\\
53	0.01\\
53.01	0.01\\
53.02	0.01\\
53.03	0.01\\
53.04	0.01\\
53.05	0.01\\
53.06	0.01\\
53.07	0.01\\
53.08	0.01\\
53.09	0.01\\
53.1	0.01\\
53.11	0.01\\
53.12	0.01\\
53.13	0.01\\
53.14	0.01\\
53.15	0.01\\
53.16	0.01\\
53.17	0.01\\
53.18	0.01\\
53.19	0.01\\
53.2	0.01\\
53.21	0.01\\
53.22	0.01\\
53.23	0.01\\
53.24	0.01\\
53.25	0.01\\
53.26	0.01\\
53.27	0.01\\
53.28	0.01\\
53.29	0.01\\
53.3	0.01\\
53.31	0.01\\
53.32	0.01\\
53.33	0.01\\
53.34	0.01\\
53.35	0.01\\
53.36	0.01\\
53.37	0.01\\
53.38	0.01\\
53.39	0.01\\
53.4	0.01\\
53.41	0.01\\
53.42	0.01\\
53.43	0.01\\
53.44	0.01\\
53.45	0.01\\
53.46	0.01\\
53.47	0.01\\
53.48	0.01\\
53.49	0.01\\
53.5	0.01\\
53.51	0.01\\
53.52	0.01\\
53.53	0.01\\
53.54	0.01\\
53.55	0.01\\
53.56	0.01\\
53.57	0.01\\
53.58	0.01\\
53.59	0.01\\
53.6	0.01\\
53.61	0.01\\
53.62	0.01\\
53.63	0.01\\
53.64	0.01\\
53.65	0.01\\
53.66	0.01\\
53.67	0.01\\
53.68	0.01\\
53.69	0.01\\
53.7	0.01\\
53.71	0.01\\
53.72	0.01\\
53.73	0.01\\
53.74	0.01\\
53.75	0.01\\
53.76	0.01\\
53.77	0.01\\
53.78	0.01\\
53.79	0.01\\
53.8	0.01\\
53.81	0.01\\
53.82	0.01\\
53.83	0.01\\
53.84	0.01\\
53.85	0.01\\
53.86	0.01\\
53.87	0.01\\
53.88	0.01\\
53.89	0.01\\
53.9	0.01\\
53.91	0.01\\
53.92	0.01\\
53.93	0.01\\
53.94	0.01\\
53.95	0.01\\
53.96	0.01\\
53.97	0.01\\
53.98	0.01\\
53.99	0.01\\
54	0.01\\
54.01	0.01\\
54.02	0.01\\
54.03	0.01\\
54.04	0.01\\
54.05	0.01\\
54.06	0.01\\
54.07	0.01\\
54.08	0.01\\
54.09	0.01\\
54.1	0.01\\
54.11	0.01\\
54.12	0.01\\
54.13	0.01\\
54.14	0.01\\
54.15	0.01\\
54.16	0.01\\
54.17	0.01\\
54.18	0.01\\
54.19	0.01\\
54.2	0.01\\
54.21	0.01\\
54.22	0.01\\
54.23	0.01\\
54.24	0.01\\
54.25	0.01\\
54.26	0.01\\
54.27	0.01\\
54.28	0.01\\
54.29	0.01\\
54.3	0.01\\
54.31	0.01\\
54.32	0.01\\
54.33	0.01\\
54.34	0.01\\
54.35	0.01\\
54.36	0.01\\
54.37	0.01\\
54.38	0.01\\
54.39	0.01\\
54.4	0.01\\
54.41	0.01\\
54.42	0.01\\
54.43	0.01\\
54.44	0.01\\
54.45	0.01\\
54.46	0.01\\
54.47	0.01\\
54.48	0.01\\
54.49	0.01\\
54.5	0.01\\
54.51	0.01\\
54.52	0.01\\
54.53	0.01\\
54.54	0.01\\
54.55	0.01\\
54.56	0.01\\
54.57	0.01\\
54.58	0.01\\
54.59	0.01\\
54.6	0.01\\
54.61	0.01\\
54.62	0.01\\
54.63	0.01\\
54.64	0.01\\
54.65	0.01\\
54.66	0.01\\
54.67	0.01\\
54.68	0.01\\
54.69	0.01\\
54.7	0.01\\
54.71	0.01\\
54.72	0.01\\
54.73	0.01\\
54.74	0.01\\
54.75	0.01\\
54.76	0.01\\
54.77	0.01\\
54.78	0.01\\
54.79	0.01\\
54.8	0.01\\
54.81	0.01\\
54.82	0.01\\
54.83	0.01\\
54.84	0.01\\
54.85	0.01\\
54.86	0.01\\
54.87	0.01\\
54.88	0.01\\
54.89	0.01\\
54.9	0.01\\
54.91	0.01\\
54.92	0.01\\
54.93	0.01\\
54.94	0.01\\
54.95	0.01\\
54.96	0.01\\
54.97	0.01\\
54.98	0.01\\
54.99	0.01\\
55	0.01\\
55.01	0.01\\
55.02	0.01\\
55.03	0.01\\
55.04	0.01\\
55.05	0.01\\
55.06	0.01\\
55.07	0.01\\
55.08	0.01\\
55.09	0.01\\
55.1	0.01\\
55.11	0.01\\
55.12	0.01\\
55.13	0.01\\
55.14	0.01\\
55.15	0.01\\
55.16	0.01\\
55.17	0.01\\
55.18	0.01\\
55.19	0.01\\
55.2	0.01\\
55.21	0.01\\
55.22	0.01\\
55.23	0.01\\
55.24	0.01\\
55.25	0.01\\
55.26	0.01\\
55.27	0.01\\
55.28	0.01\\
55.29	0.01\\
55.3	0.01\\
55.31	0.01\\
55.32	0.01\\
55.33	0.01\\
55.34	0.01\\
55.35	0.01\\
55.36	0.01\\
55.37	0.01\\
55.38	0.01\\
55.39	0.01\\
55.4	0.01\\
55.41	0.01\\
55.42	0.01\\
55.43	0.01\\
55.44	0.01\\
55.45	0.01\\
55.46	0.01\\
55.47	0.01\\
55.48	0.01\\
55.49	0.01\\
55.5	0.01\\
55.51	0.01\\
55.52	0.01\\
55.53	0.01\\
55.54	0.01\\
55.55	0.01\\
55.56	0.01\\
55.57	0.01\\
55.58	0.01\\
55.59	0.01\\
55.6	0.01\\
55.61	0.01\\
55.62	0.01\\
55.63	0.01\\
55.64	0.01\\
55.65	0.01\\
55.66	0.01\\
55.67	0.01\\
55.68	0.01\\
55.69	0.01\\
55.7	0.01\\
55.71	0.01\\
55.72	0.01\\
55.73	0.01\\
55.74	0.01\\
55.75	0.01\\
55.76	0.01\\
55.77	0.01\\
55.78	0.01\\
55.79	0.01\\
55.8	0.01\\
55.81	0.01\\
55.82	0.01\\
55.83	0.01\\
55.84	0.01\\
55.85	0.01\\
55.86	0.01\\
55.87	0.01\\
55.88	0.01\\
55.89	0.01\\
55.9	0.01\\
55.91	0.01\\
55.92	0.01\\
55.93	0.01\\
55.94	0.01\\
55.95	0.01\\
55.96	0.01\\
55.97	0.01\\
55.98	0.01\\
55.99	0.01\\
56	0.01\\
56.01	0.01\\
56.02	0.01\\
56.03	0.01\\
56.04	0.01\\
56.05	0.01\\
56.06	0.01\\
56.07	0.01\\
56.08	0.01\\
56.09	0.01\\
56.1	0.01\\
56.11	0.01\\
56.12	0.01\\
56.13	0.01\\
56.14	0.01\\
56.15	0.01\\
56.16	0.01\\
56.17	0.01\\
56.18	0.01\\
56.19	0.01\\
56.2	0.01\\
56.21	0.01\\
56.22	0.01\\
56.23	0.01\\
56.24	0.01\\
56.25	0.01\\
56.26	0.01\\
56.27	0.01\\
56.28	0.01\\
56.29	0.01\\
56.3	0.01\\
56.31	0.01\\
56.32	0.01\\
56.33	0.01\\
56.34	0.01\\
56.35	0.01\\
56.36	0.01\\
56.37	0.01\\
56.38	0.01\\
56.39	0.01\\
56.4	0.01\\
56.41	0.01\\
56.42	0.01\\
56.43	0.01\\
56.44	0.01\\
56.45	0.01\\
56.46	0.01\\
56.47	0.01\\
56.48	0.01\\
56.49	0.01\\
56.5	0.01\\
56.51	0.01\\
56.52	0.01\\
56.53	0.01\\
56.54	0.01\\
56.55	0.01\\
56.56	0.01\\
56.57	0.01\\
56.58	0.01\\
56.59	0.01\\
56.6	0.01\\
56.61	0.01\\
56.62	0.01\\
56.63	0.01\\
56.64	0.01\\
56.65	0.01\\
56.66	0.01\\
56.67	0.01\\
56.68	0.01\\
56.69	0.01\\
56.7	0.01\\
56.71	0.01\\
56.72	0.01\\
56.73	0.01\\
56.74	0.01\\
56.75	0.01\\
56.76	0.01\\
56.77	0.01\\
56.78	0.01\\
56.79	0.01\\
56.8	0.01\\
56.81	0.01\\
56.82	0.01\\
56.83	0.01\\
56.84	0.01\\
56.85	0.01\\
56.86	0.01\\
56.87	0.01\\
56.88	0.01\\
56.89	0.01\\
56.9	0.01\\
56.91	0.01\\
56.92	0.01\\
56.93	0.01\\
56.94	0.01\\
56.95	0.01\\
56.96	0.01\\
56.97	0.01\\
56.98	0.01\\
56.99	0.01\\
57	0.01\\
57.01	0.01\\
57.02	0.01\\
57.03	0.01\\
57.04	0.01\\
57.05	0.01\\
57.06	0.01\\
57.07	0.01\\
57.08	0.01\\
57.09	0.01\\
57.1	0.01\\
57.11	0.01\\
57.12	0.01\\
57.13	0.01\\
57.14	0.01\\
57.15	0.01\\
57.16	0.01\\
57.17	0.01\\
57.18	0.01\\
57.19	0.01\\
57.2	0.01\\
57.21	0.01\\
57.22	0.01\\
57.23	0.01\\
57.24	0.01\\
57.25	0.01\\
57.26	0.01\\
57.27	0.01\\
57.28	0.01\\
57.29	0.01\\
57.3	0.01\\
57.31	0.01\\
57.32	0.01\\
57.33	0.01\\
57.34	0.01\\
57.35	0.01\\
57.36	0.01\\
57.37	0.01\\
57.38	0.01\\
57.39	0.01\\
57.4	0.01\\
57.41	0.01\\
57.42	0.01\\
57.43	0.01\\
57.44	0.01\\
57.45	0.01\\
57.46	0.01\\
57.47	0.01\\
57.48	0.01\\
57.49	0.01\\
57.5	0.01\\
57.51	0.01\\
57.52	0.01\\
57.53	0.01\\
57.54	0.01\\
57.55	0.01\\
57.56	0.01\\
57.57	0.01\\
57.58	0.01\\
57.59	0.01\\
57.6	0.01\\
57.61	0.01\\
57.62	0.01\\
57.63	0.01\\
57.64	0.01\\
57.65	0.01\\
57.66	0.01\\
57.67	0.01\\
57.68	0.01\\
57.69	0.01\\
57.7	0.01\\
57.71	0.01\\
57.72	0.01\\
57.73	0.01\\
57.74	0.01\\
57.75	0.01\\
57.76	0.01\\
57.77	0.01\\
57.78	0.01\\
57.79	0.01\\
57.8	0.01\\
57.81	0.01\\
57.82	0.01\\
57.83	0.01\\
57.84	0.01\\
57.85	0.01\\
57.86	0.01\\
57.87	0.01\\
57.88	0.01\\
57.89	0.01\\
57.9	0.01\\
57.91	0.01\\
57.92	0.01\\
57.93	0.01\\
57.94	0.01\\
57.95	0.01\\
57.96	0.01\\
57.97	0.01\\
57.98	0.01\\
57.99	0.01\\
58	0.01\\
58.01	0.01\\
58.02	0.01\\
58.03	0.01\\
58.04	0.01\\
58.05	0.01\\
58.06	0.01\\
58.07	0.01\\
58.08	0.01\\
58.09	0.01\\
58.1	0.01\\
58.11	0.01\\
58.12	0.01\\
58.13	0.01\\
58.14	0.01\\
58.15	0.01\\
58.16	0.01\\
58.17	0.01\\
58.18	0.01\\
58.19	0.01\\
58.2	0.01\\
58.21	0.01\\
58.22	0.01\\
58.23	0.01\\
58.24	0.01\\
58.25	0.01\\
58.26	0.01\\
58.27	0.01\\
58.28	0.01\\
58.29	0.01\\
58.3	0.01\\
58.31	0.01\\
58.32	0.01\\
58.33	0.01\\
58.34	0.01\\
58.35	0.01\\
58.36	0.01\\
58.37	0.01\\
58.38	0.01\\
58.39	0.01\\
58.4	0.01\\
58.41	0.01\\
58.42	0.01\\
58.43	0.01\\
58.44	0.01\\
58.45	0.01\\
58.46	0.01\\
58.47	0.01\\
58.48	0.01\\
58.49	0.01\\
58.5	0.01\\
58.51	0.01\\
58.52	0.01\\
58.53	0.01\\
58.54	0.01\\
58.55	0.01\\
58.56	0.01\\
58.57	0.01\\
58.58	0.01\\
58.59	0.01\\
58.6	0.01\\
58.61	0.01\\
58.62	0.01\\
58.63	0.01\\
58.64	0.01\\
58.65	0.01\\
58.66	0.01\\
58.67	0.01\\
58.68	0.01\\
58.69	0.01\\
58.7	0.01\\
58.71	0.01\\
58.72	0.01\\
58.73	0.01\\
58.74	0.01\\
58.75	0.01\\
58.76	0.01\\
58.77	0.01\\
58.78	0.01\\
58.79	0.01\\
58.8	0.01\\
58.81	0.01\\
58.82	0.01\\
58.83	0.01\\
58.84	0.01\\
58.85	0.01\\
58.86	0.01\\
58.87	0.01\\
58.88	0.01\\
58.89	0.01\\
58.9	0.01\\
58.91	0.01\\
58.92	0.01\\
58.93	0.01\\
58.94	0.01\\
58.95	0.01\\
58.96	0.01\\
58.97	0.01\\
58.98	0.01\\
58.99	0.01\\
59	0.01\\
59.01	0.01\\
59.02	0.01\\
59.03	0.01\\
59.04	0.01\\
59.05	0.01\\
59.06	0.01\\
59.07	0.01\\
59.08	0.01\\
59.09	0.01\\
59.1	0.01\\
59.11	0.01\\
59.12	0.01\\
59.13	0.01\\
59.14	0.01\\
59.15	0.01\\
59.16	0.01\\
59.17	0.01\\
59.18	0.01\\
59.19	0.01\\
59.2	0.01\\
59.21	0.01\\
59.22	0.01\\
59.23	0.01\\
59.24	0.01\\
59.25	0.01\\
59.26	0.01\\
59.27	0.01\\
59.28	0.01\\
59.29	0.01\\
59.3	0.01\\
59.31	0.01\\
59.32	0.01\\
59.33	0.01\\
59.34	0.01\\
59.35	0.01\\
59.36	0.01\\
59.37	0.01\\
59.38	0.01\\
59.39	0.01\\
59.4	0.01\\
59.41	0.01\\
59.42	0.01\\
59.43	0.01\\
59.44	0.01\\
59.45	0.01\\
59.46	0.01\\
59.47	0.01\\
59.48	0.01\\
59.49	0.01\\
59.5	0.01\\
59.51	0.01\\
59.52	0.01\\
59.53	0.01\\
59.54	0.01\\
59.55	0.01\\
59.56	0.01\\
59.57	0.01\\
59.58	0.01\\
59.59	0.01\\
59.6	0.01\\
59.61	0.01\\
59.62	0.01\\
59.63	0.01\\
59.64	0.01\\
59.65	0.01\\
59.66	0.01\\
59.67	0.01\\
59.68	0.01\\
59.69	0.01\\
59.7	0.01\\
59.71	0.01\\
59.72	0.01\\
59.73	0.01\\
59.74	0.01\\
59.75	0.01\\
59.76	0.01\\
59.77	0.01\\
59.78	0.01\\
59.79	0.01\\
59.8	0.01\\
59.81	0.01\\
59.82	0.01\\
59.83	0.01\\
59.84	0.01\\
59.85	0.01\\
59.86	0.01\\
59.87	0.01\\
59.88	0.01\\
59.89	0.01\\
59.9	0.01\\
59.91	0.01\\
59.92	0.01\\
59.93	0.01\\
59.94	0.01\\
59.95	0.01\\
59.96	0.01\\
59.97	0.01\\
59.98	0.01\\
59.99	0.01\\
60	0.01\\
60.01	0.01\\
60.02	0.01\\
60.03	0.01\\
60.04	0.01\\
60.05	0.01\\
60.06	0.01\\
60.07	0.01\\
60.08	0.01\\
60.09	0.01\\
60.1	0.01\\
60.11	0.01\\
60.12	0.01\\
60.13	0.01\\
60.14	0.01\\
60.15	0.01\\
60.16	0.01\\
60.17	0.01\\
60.18	0.01\\
60.19	0.01\\
60.2	0.01\\
60.21	0.01\\
60.22	0.01\\
60.23	0.01\\
60.24	0.01\\
60.25	0.01\\
60.26	0.01\\
60.27	0.01\\
60.28	0.01\\
60.29	0.01\\
60.3	0.01\\
60.31	0.01\\
60.32	0.01\\
60.33	0.01\\
60.34	0.01\\
60.35	0.01\\
60.36	0.01\\
60.37	0.01\\
60.38	0.01\\
60.39	0.01\\
60.4	0.01\\
60.41	0.01\\
60.42	0.01\\
60.43	0.01\\
60.44	0.01\\
60.45	0.01\\
60.46	0.01\\
60.47	0.01\\
60.48	0.01\\
60.49	0.01\\
60.5	0.01\\
60.51	0.01\\
60.52	0.01\\
60.53	0.01\\
60.54	0.01\\
60.55	0.01\\
60.56	0.01\\
60.57	0.01\\
60.58	0.01\\
60.59	0.01\\
60.6	0.01\\
60.61	0.01\\
60.62	0.01\\
60.63	0.01\\
60.64	0.01\\
60.65	0.01\\
60.66	0.01\\
60.67	0.01\\
60.68	0.01\\
60.69	0.01\\
60.7	0.01\\
60.71	0.01\\
60.72	0.01\\
60.73	0.01\\
60.74	0.01\\
60.75	0.01\\
60.76	0.01\\
60.77	0.01\\
60.78	0.01\\
60.79	0.01\\
60.8	0.01\\
60.81	0.01\\
60.82	0.01\\
60.83	0.01\\
60.84	0.01\\
60.85	0.01\\
60.86	0.01\\
60.87	0.01\\
60.88	0.01\\
60.89	0.01\\
60.9	0.01\\
60.91	0.01\\
60.92	0.01\\
60.93	0.01\\
60.94	0.01\\
60.95	0.01\\
60.96	0.01\\
60.97	0.01\\
60.98	0.01\\
60.99	0.01\\
61	0.01\\
61.01	0.01\\
61.02	0.01\\
61.03	0.01\\
61.04	0.01\\
61.05	0.01\\
61.06	0.01\\
61.07	0.01\\
61.08	0.01\\
61.09	0.01\\
61.1	0.01\\
61.11	0.01\\
61.12	0.01\\
61.13	0.01\\
61.14	0.01\\
61.15	0.01\\
61.16	0.01\\
61.17	0.01\\
61.18	0.01\\
61.19	0.01\\
61.2	0.01\\
61.21	0.01\\
61.22	0.01\\
61.23	0.01\\
61.24	0.01\\
61.25	0.01\\
61.26	0.01\\
61.27	0.01\\
61.28	0.01\\
61.29	0.01\\
61.3	0.01\\
61.31	0.01\\
61.32	0.01\\
61.33	0.01\\
61.34	0.01\\
61.35	0.01\\
61.36	0.01\\
61.37	0.01\\
61.38	0.01\\
61.39	0.01\\
61.4	0.01\\
61.41	0.01\\
61.42	0.01\\
61.43	0.01\\
61.44	0.01\\
61.45	0.01\\
61.46	0.01\\
61.47	0.01\\
61.48	0.01\\
61.49	0.01\\
61.5	0.01\\
61.51	0.01\\
61.52	0.01\\
61.53	0.01\\
61.54	0.01\\
61.55	0.01\\
61.56	0.01\\
61.57	0.01\\
61.58	0.01\\
61.59	0.01\\
61.6	0.01\\
61.61	0.01\\
61.62	0.01\\
61.63	0.01\\
61.64	0.01\\
61.65	0.01\\
61.66	0.01\\
61.67	0.01\\
61.68	0.01\\
61.69	0.01\\
61.7	0.01\\
61.71	0.01\\
61.72	0.01\\
61.73	0.01\\
61.74	0.01\\
61.75	0.01\\
61.76	0.01\\
61.77	0.01\\
61.78	0.01\\
61.79	0.01\\
61.8	0.01\\
61.81	0.01\\
61.82	0.01\\
61.83	0.01\\
61.84	0.01\\
61.85	0.01\\
61.86	0.01\\
61.87	0.01\\
61.88	0.01\\
61.89	0.01\\
61.9	0.01\\
61.91	0.01\\
61.92	0.01\\
61.93	0.01\\
61.94	0.01\\
61.95	0.01\\
61.96	0.01\\
61.97	0.01\\
61.98	0.01\\
61.99	0.01\\
62	0.01\\
62.01	0.01\\
62.02	0.01\\
62.03	0.01\\
62.04	0.01\\
62.05	0.01\\
62.06	0.01\\
62.07	0.01\\
62.08	0.01\\
62.09	0.01\\
62.1	0.01\\
62.11	0.01\\
62.12	0.01\\
62.13	0.01\\
62.14	0.01\\
62.15	0.01\\
62.16	0.01\\
62.17	0.01\\
62.18	0.01\\
62.19	0.01\\
62.2	0.01\\
62.21	0.01\\
62.22	0.01\\
62.23	0.01\\
62.24	0.01\\
62.25	0.01\\
62.26	0.01\\
62.27	0.01\\
62.28	0.01\\
62.29	0.01\\
62.3	0.01\\
62.31	0.01\\
62.32	0.01\\
62.33	0.01\\
62.34	0.01\\
62.35	0.01\\
62.36	0.01\\
62.37	0.01\\
62.38	0.01\\
62.39	0.01\\
62.4	0.01\\
62.41	0.01\\
62.42	0.01\\
62.43	0.01\\
62.44	0.01\\
62.45	0.01\\
62.46	0.01\\
62.47	0.01\\
62.48	0.01\\
62.49	0.01\\
62.5	0.01\\
62.51	0.01\\
62.52	0.01\\
62.53	0.01\\
62.54	0.01\\
62.55	0.01\\
62.56	0.01\\
62.57	0.01\\
62.58	0.01\\
62.59	0.01\\
62.6	0.01\\
62.61	0.01\\
62.62	0.01\\
62.63	0.01\\
62.64	0.01\\
62.65	0.01\\
62.66	0.01\\
62.67	0.01\\
62.68	0.01\\
62.69	0.01\\
62.7	0.01\\
62.71	0.01\\
62.72	0.01\\
62.73	0.01\\
62.74	0.01\\
62.75	0.01\\
62.76	0.01\\
62.77	0.01\\
62.78	0.01\\
62.79	0.01\\
62.8	0.01\\
62.81	0.01\\
62.82	0.01\\
62.83	0.01\\
62.84	0.01\\
62.85	0.01\\
62.86	0.01\\
62.87	0.01\\
62.88	0.01\\
62.89	0.01\\
62.9	0.01\\
62.91	0.01\\
62.92	0.01\\
62.93	0.01\\
62.94	0.01\\
62.95	0.01\\
62.96	0.01\\
62.97	0.01\\
62.98	0.01\\
62.99	0.01\\
63	0.01\\
63.01	0.01\\
63.02	0.01\\
63.03	0.01\\
63.04	0.01\\
63.05	0.01\\
63.06	0.01\\
63.07	0.01\\
63.08	0.01\\
63.09	0.01\\
63.1	0.01\\
63.11	0.01\\
63.12	0.01\\
63.13	0.01\\
63.14	0.01\\
63.15	0.01\\
63.16	0.01\\
63.17	0.01\\
63.18	0.01\\
63.19	0.01\\
63.2	0.01\\
63.21	0.01\\
63.22	0.01\\
63.23	0.01\\
63.24	0.01\\
63.25	0.01\\
63.26	0.01\\
63.27	0.01\\
63.28	0.01\\
63.29	0.01\\
63.3	0.01\\
63.31	0.01\\
63.32	0.01\\
63.33	0.01\\
63.34	0.01\\
63.35	0.01\\
63.36	0.01\\
63.37	0.01\\
63.38	0.01\\
63.39	0.01\\
63.4	0.01\\
63.41	0.01\\
63.42	0.01\\
63.43	0.01\\
63.44	0.01\\
63.45	0.01\\
63.46	0.01\\
63.47	0.01\\
63.48	0.01\\
63.49	0.01\\
63.5	0.01\\
63.51	0.01\\
63.52	0.01\\
63.53	0.01\\
63.54	0.01\\
63.55	0.01\\
63.56	0.01\\
63.57	0.01\\
63.58	0.01\\
63.59	0.01\\
63.6	0.01\\
63.61	0.01\\
63.62	0.01\\
63.63	0.01\\
63.64	0.01\\
63.65	0.01\\
63.66	0.01\\
63.67	0.01\\
63.68	0.01\\
63.69	0.01\\
63.7	0.01\\
63.71	0.01\\
63.72	0.01\\
63.73	0.01\\
63.74	0.01\\
63.75	0.01\\
63.76	0.01\\
63.77	0.01\\
63.78	0.01\\
63.79	0.01\\
63.8	0.01\\
63.81	0.01\\
63.82	0.01\\
63.83	0.01\\
63.84	0.01\\
63.85	0.01\\
63.86	0.01\\
63.87	0.01\\
63.88	0.01\\
63.89	0.01\\
63.9	0.01\\
63.91	0.01\\
63.92	0.01\\
63.93	0.01\\
63.94	0.01\\
63.95	0.01\\
63.96	0.01\\
63.97	0.01\\
63.98	0.01\\
63.99	0.01\\
64	0.01\\
64.01	0.01\\
64.02	0.01\\
64.03	0.01\\
64.04	0.01\\
64.05	0.01\\
64.06	0.01\\
64.07	0.01\\
64.08	0.01\\
64.09	0.01\\
64.1	0.01\\
64.11	0.01\\
64.12	0.01\\
64.13	0.01\\
64.14	0.01\\
64.15	0.01\\
64.16	0.01\\
64.17	0.01\\
64.18	0.01\\
64.19	0.01\\
64.2	0.01\\
64.21	0.01\\
64.22	0.01\\
64.23	0.01\\
64.24	0.01\\
64.25	0.01\\
64.26	0.01\\
64.27	0.01\\
64.28	0.01\\
64.29	0.01\\
64.3	0.01\\
64.31	0.01\\
64.32	0.01\\
64.33	0.01\\
64.34	0.01\\
64.35	0.01\\
64.36	0.01\\
64.37	0.01\\
64.38	0.01\\
64.39	0.01\\
64.4	0.01\\
64.41	0.01\\
64.42	0.01\\
64.43	0.01\\
64.44	0.01\\
64.45	0.01\\
64.46	0.01\\
64.47	0.01\\
64.48	0.01\\
64.49	0.01\\
64.5	0.01\\
64.51	0.01\\
64.52	0.01\\
64.53	0.01\\
64.54	0.01\\
64.55	0.01\\
64.56	0.01\\
64.57	0.01\\
64.58	0.01\\
64.59	0.01\\
64.6	0.01\\
64.61	0.01\\
64.62	0.01\\
64.63	0.01\\
64.64	0.01\\
64.65	0.01\\
64.66	0.01\\
64.67	0.01\\
64.68	0.01\\
64.69	0.01\\
64.7	0.01\\
64.71	0.01\\
64.72	0.01\\
64.73	0.01\\
64.74	0.01\\
64.75	0.01\\
64.76	0.01\\
64.77	0.01\\
64.78	0.01\\
64.79	0.01\\
64.8	0.01\\
64.81	0.01\\
64.82	0.01\\
64.83	0.01\\
64.84	0.01\\
64.85	0.01\\
64.86	0.01\\
64.87	0.01\\
64.88	0.01\\
64.89	0.01\\
64.9	0.01\\
64.91	0.01\\
64.92	0.01\\
64.93	0.01\\
64.94	0.01\\
64.95	0.01\\
64.96	0.01\\
64.97	0.01\\
64.98	0.01\\
64.99	0.01\\
65	0.01\\
65.01	0.01\\
65.02	0.01\\
65.03	0.01\\
65.04	0.01\\
65.05	0.01\\
65.06	0.01\\
65.07	0.01\\
65.08	0.01\\
65.09	0.01\\
65.1	0.01\\
65.11	0.01\\
65.12	0.01\\
65.13	0.01\\
65.14	0.01\\
65.15	0.01\\
65.16	0.01\\
65.17	0.01\\
65.18	0.01\\
65.19	0.01\\
65.2	0.01\\
65.21	0.01\\
65.22	0.01\\
65.23	0.01\\
65.24	0.01\\
65.25	0.01\\
65.26	0.01\\
65.27	0.01\\
65.28	0.01\\
65.29	0.01\\
65.3	0.01\\
65.31	0.01\\
65.32	0.01\\
65.33	0.01\\
65.34	0.01\\
65.35	0.01\\
65.36	0.01\\
65.37	0.01\\
65.38	0.01\\
65.39	0.01\\
65.4	0.01\\
65.41	0.01\\
65.42	0.01\\
65.43	0.01\\
65.44	0.01\\
65.45	0.01\\
65.46	0.01\\
65.47	0.01\\
65.48	0.01\\
65.49	0.01\\
65.5	0.01\\
65.51	0.01\\
65.52	0.01\\
65.53	0.01\\
65.54	0.01\\
65.55	0.01\\
65.56	0.01\\
65.57	0.01\\
65.58	0.01\\
65.59	0.01\\
65.6	0.01\\
65.61	0.01\\
65.62	0.01\\
65.63	0.01\\
65.64	0.01\\
65.65	0.01\\
65.66	0.01\\
65.67	0.01\\
65.68	0.01\\
65.69	0.01\\
65.7	0.01\\
65.71	0.01\\
65.72	0.01\\
65.73	0.01\\
65.74	0.01\\
65.75	0.01\\
65.76	0.01\\
65.77	0.01\\
65.78	0.01\\
65.79	0.01\\
65.8	0.01\\
65.81	0.01\\
65.82	0.01\\
65.83	0.01\\
65.84	0.01\\
65.85	0.01\\
65.86	0.01\\
65.87	0.01\\
65.88	0.01\\
65.89	0.01\\
65.9	0.01\\
65.91	0.01\\
65.92	0.01\\
65.93	0.01\\
65.94	0.01\\
65.95	0.01\\
65.96	0.01\\
65.97	0.01\\
65.98	0.01\\
65.99	0.01\\
66	0.01\\
66.01	0.01\\
66.02	0.01\\
66.03	0.01\\
66.04	0.01\\
66.05	0.01\\
66.06	0.01\\
66.07	0.01\\
66.08	0.01\\
66.09	0.01\\
66.1	0.01\\
66.11	0.01\\
66.12	0.01\\
66.13	0.01\\
66.14	0.01\\
66.15	0.01\\
66.16	0.01\\
66.17	0.01\\
66.18	0.01\\
66.19	0.01\\
66.2	0.01\\
66.21	0.01\\
66.22	0.01\\
66.23	0.01\\
66.24	0.01\\
66.25	0.01\\
66.26	0.01\\
66.27	0.01\\
66.28	0.01\\
66.29	0.01\\
66.3	0.01\\
66.31	0.01\\
66.32	0.01\\
66.33	0.01\\
66.34	0.01\\
66.35	0.01\\
66.36	0.01\\
66.37	0.01\\
66.38	0.01\\
66.39	0.01\\
66.4	0.01\\
66.41	0.01\\
66.42	0.01\\
66.43	0.01\\
66.44	0.01\\
66.45	0.01\\
66.46	0.01\\
66.47	0.01\\
66.48	0.01\\
66.49	0.01\\
66.5	0.01\\
66.51	0.01\\
66.52	0.01\\
66.53	0.01\\
66.54	0.01\\
66.55	0.01\\
66.56	0.01\\
66.57	0.01\\
66.58	0.01\\
66.59	0.01\\
66.6	0.01\\
66.61	0.01\\
66.62	0.01\\
66.63	0.01\\
66.64	0.01\\
66.65	0.01\\
66.66	0.01\\
66.67	0.01\\
66.68	0.01\\
66.69	0.01\\
66.7	0.01\\
66.71	0.01\\
66.72	0.01\\
66.73	0.01\\
66.74	0.01\\
66.75	0.01\\
66.76	0.01\\
66.77	0.01\\
66.78	0.01\\
66.79	0.01\\
66.8	0.01\\
66.81	0.01\\
66.82	0.01\\
66.83	0.01\\
66.84	0.01\\
66.85	0.01\\
66.86	0.01\\
66.87	0.01\\
66.88	0.01\\
66.89	0.01\\
66.9	0.01\\
66.91	0.01\\
66.92	0.01\\
66.93	0.01\\
66.94	0.01\\
66.95	0.01\\
66.96	0.01\\
66.97	0.01\\
66.98	0.01\\
66.99	0.01\\
67	0.01\\
67.01	0.01\\
67.02	0.01\\
67.03	0.01\\
67.04	0.01\\
67.05	0.01\\
67.06	0.01\\
67.07	0.01\\
67.08	0.01\\
67.09	0.01\\
67.1	0.01\\
67.11	0.01\\
67.12	0.01\\
67.13	0.01\\
67.14	0.01\\
67.15	0.01\\
67.16	0.01\\
67.17	0.01\\
67.18	0.01\\
67.19	0.01\\
67.2	0.01\\
67.21	0.01\\
67.22	0.01\\
67.23	0.01\\
67.24	0.01\\
67.25	0.01\\
67.26	0.01\\
67.27	0.01\\
67.28	0.01\\
67.29	0.01\\
67.3	0.01\\
67.31	0.01\\
67.32	0.01\\
67.33	0.01\\
67.34	0.01\\
67.35	0.01\\
67.36	0.01\\
67.37	0.01\\
67.38	0.01\\
67.39	0.01\\
67.4	0.01\\
67.41	0.01\\
67.42	0.01\\
67.43	0.01\\
67.44	0.01\\
67.45	0.01\\
67.46	0.01\\
67.47	0.01\\
67.48	0.01\\
67.49	0.01\\
67.5	0.01\\
67.51	0.01\\
67.52	0.01\\
67.53	0.01\\
67.54	0.01\\
67.55	0.01\\
67.56	0.01\\
67.57	0.01\\
67.58	0.01\\
67.59	0.01\\
67.6	0.01\\
67.61	0.01\\
67.62	0.01\\
67.63	0.01\\
67.64	0.01\\
67.65	0.01\\
67.66	0.01\\
67.67	0.01\\
67.68	0.01\\
67.69	0.01\\
67.7	0.01\\
67.71	0.01\\
67.72	0.01\\
67.73	0.01\\
67.74	0.01\\
67.75	0.01\\
67.76	0.01\\
67.77	0.01\\
67.78	0.01\\
67.79	0.01\\
67.8	0.01\\
67.81	0.01\\
67.82	0.01\\
67.83	0.01\\
67.84	0.01\\
67.85	0.01\\
67.86	0.01\\
67.87	0.01\\
67.88	0.01\\
67.89	0.01\\
67.9	0.01\\
67.91	0.01\\
67.92	0.01\\
67.93	0.01\\
67.94	0.01\\
67.95	0.01\\
67.96	0.01\\
67.97	0.01\\
67.98	0.01\\
67.99	0.01\\
68	0.01\\
68.01	0.01\\
68.02	0.01\\
68.03	0.01\\
68.04	0.01\\
68.05	0.01\\
68.06	0.01\\
68.07	0.01\\
68.08	0.01\\
68.09	0.01\\
68.1	0.01\\
68.11	0.01\\
68.12	0.01\\
68.13	0.01\\
68.14	0.01\\
68.15	0.01\\
68.16	0.01\\
68.17	0.01\\
68.18	0.01\\
68.19	0.01\\
68.2	0.01\\
68.21	0.01\\
68.22	0.01\\
68.23	0.01\\
68.24	0.01\\
68.25	0.01\\
68.26	0.01\\
68.27	0.01\\
68.28	0.01\\
68.29	0.01\\
68.3	0.01\\
68.31	0.01\\
68.32	0.01\\
68.33	0.01\\
68.34	0.01\\
68.35	0.01\\
68.36	0.01\\
68.37	0.01\\
68.38	0.01\\
68.39	0.01\\
68.4	0.01\\
68.41	0.01\\
68.42	0.01\\
68.43	0.01\\
68.44	0.01\\
68.45	0.01\\
68.46	0.01\\
68.47	0.01\\
68.48	0.01\\
68.49	0.01\\
68.5	0.01\\
68.51	0.01\\
68.52	0.01\\
68.53	0.01\\
68.54	0.01\\
68.55	0.01\\
68.56	0.01\\
68.57	0.01\\
68.58	0.01\\
68.59	0.01\\
68.6	0.01\\
68.61	0.01\\
68.62	0.01\\
68.63	0.01\\
68.64	0.01\\
68.65	0.01\\
68.66	0.01\\
68.67	0.01\\
68.68	0.01\\
68.69	0.01\\
68.7	0.01\\
68.71	0.01\\
68.72	0.01\\
68.73	0.01\\
68.74	0.01\\
68.75	0.01\\
68.76	0.01\\
68.77	0.01\\
68.78	0.01\\
68.79	0.01\\
68.8	0.01\\
68.81	0.01\\
68.82	0.01\\
68.83	0.01\\
68.84	0.01\\
68.85	0.01\\
68.86	0.01\\
68.87	0.01\\
68.88	0.01\\
68.89	0.01\\
68.9	0.01\\
68.91	0.01\\
68.92	0.01\\
68.93	0.01\\
68.94	0.01\\
68.95	0.01\\
68.96	0.01\\
68.97	0.01\\
68.98	0.01\\
68.99	0.01\\
69	0.01\\
69.01	0.01\\
69.02	0.01\\
69.03	0.01\\
69.04	0.01\\
69.05	0.01\\
69.06	0.01\\
69.07	0.01\\
69.08	0.01\\
69.09	0.01\\
69.1	0.01\\
69.11	0.01\\
69.12	0.01\\
69.13	0.01\\
69.14	0.01\\
69.15	0.01\\
69.16	0.01\\
69.17	0.01\\
69.18	0.01\\
69.19	0.01\\
69.2	0.01\\
69.21	0.01\\
69.22	0.01\\
69.23	0.01\\
69.24	0.01\\
69.25	0.01\\
69.26	0.01\\
69.27	0.01\\
69.28	0.01\\
69.29	0.01\\
69.3	0.01\\
69.31	0.01\\
69.32	0.01\\
69.33	0.01\\
69.34	0.01\\
69.35	0.01\\
69.36	0.01\\
69.37	0.01\\
69.38	0.01\\
69.39	0.01\\
69.4	0.01\\
69.41	0.01\\
69.42	0.01\\
69.43	0.01\\
69.44	0.01\\
69.45	0.01\\
69.46	0.01\\
69.47	0.01\\
69.48	0.01\\
69.49	0.01\\
69.5	0.01\\
69.51	0.01\\
69.52	0.01\\
69.53	0.01\\
69.54	0.01\\
69.55	0.01\\
69.56	0.01\\
69.57	0.01\\
69.58	0.01\\
69.59	0.01\\
69.6	0.01\\
69.61	0.01\\
69.62	0.01\\
69.63	0.01\\
69.64	0.01\\
69.65	0.01\\
69.66	0.01\\
69.67	0.01\\
69.68	0.01\\
69.69	0.01\\
69.7	0.01\\
69.71	0.01\\
69.72	0.01\\
69.73	0.01\\
69.74	0.01\\
69.75	0.01\\
69.76	0.01\\
69.77	0.01\\
69.78	0.01\\
69.79	0.01\\
69.8	0.01\\
69.81	0.01\\
69.82	0.01\\
69.83	0.01\\
69.84	0.01\\
69.85	0.01\\
69.86	0.01\\
69.87	0.01\\
69.88	0.01\\
69.89	0.01\\
69.9	0.01\\
69.91	0.01\\
69.92	0.01\\
69.93	0.01\\
69.94	0.01\\
69.95	0.01\\
69.96	0.01\\
69.97	0.01\\
69.98	0.01\\
69.99	0.01\\
70	0.01\\
70.01	0.01\\
70.02	0.01\\
70.03	0.01\\
70.04	0.01\\
70.05	0.01\\
70.06	0.01\\
70.07	0.01\\
70.08	0.01\\
70.09	0.01\\
70.1	0.01\\
70.11	0.01\\
70.12	0.01\\
70.13	0.01\\
70.14	0.01\\
70.15	0.01\\
70.16	0.01\\
70.17	0.01\\
70.18	0.01\\
70.19	0.01\\
70.2	0.01\\
70.21	0.01\\
70.22	0.01\\
70.23	0.01\\
70.24	0.01\\
70.25	0.01\\
70.26	0.01\\
70.27	0.01\\
70.28	0.01\\
70.29	0.01\\
70.3	0.01\\
70.31	0.01\\
70.32	0.01\\
70.33	0.01\\
70.34	0.01\\
70.35	0.01\\
70.36	0.01\\
70.37	0.01\\
70.38	0.01\\
70.39	0.01\\
70.4	0.01\\
70.41	0.01\\
70.42	0.01\\
70.43	0.01\\
70.44	0.01\\
70.45	0.01\\
70.46	0.01\\
70.47	0.01\\
70.48	0.01\\
70.49	0.01\\
70.5	0.01\\
70.51	0.01\\
70.52	0.01\\
70.53	0.01\\
70.54	0.01\\
70.55	0.01\\
70.56	0.01\\
70.57	0.01\\
70.58	0.01\\
70.59	0.01\\
70.6	0.01\\
70.61	0.01\\
70.62	0.01\\
70.63	0.01\\
70.64	0.01\\
70.65	0.01\\
70.66	0.01\\
70.67	0.01\\
70.68	0.01\\
70.69	0.01\\
70.7	0.01\\
70.71	0.01\\
70.72	0.01\\
70.73	0.01\\
70.74	0.01\\
70.75	0.01\\
70.76	0.01\\
70.77	0.01\\
70.78	0.01\\
70.79	0.01\\
70.8	0.01\\
70.81	0.01\\
70.82	0.01\\
70.83	0.01\\
70.84	0.01\\
70.85	0.01\\
70.86	0.01\\
70.87	0.01\\
70.88	0.01\\
70.89	0.01\\
70.9	0.01\\
70.91	0.01\\
70.92	0.01\\
70.93	0.01\\
70.94	0.01\\
70.95	0.01\\
70.96	0.01\\
70.97	0.01\\
70.98	0.01\\
70.99	0.01\\
71	0.01\\
71.01	0.01\\
71.02	0.01\\
71.03	0.01\\
71.04	0.01\\
71.05	0.01\\
71.06	0.01\\
71.07	0.01\\
71.08	0.01\\
71.09	0.01\\
71.1	0.01\\
71.11	0.01\\
71.12	0.01\\
71.13	0.01\\
71.14	0.01\\
71.15	0.01\\
71.16	0.01\\
71.17	0.01\\
71.18	0.01\\
71.19	0.01\\
71.2	0.01\\
71.21	0.01\\
71.22	0.01\\
71.23	0.01\\
71.24	0.01\\
71.25	0.01\\
71.26	0.01\\
71.27	0.01\\
71.28	0.01\\
71.29	0.01\\
71.3	0.01\\
71.31	0.01\\
71.32	0.01\\
71.33	0.01\\
71.34	0.01\\
71.35	0.01\\
71.36	0.01\\
71.37	0.01\\
71.38	0.01\\
71.39	0.01\\
71.4	0.01\\
71.41	0.01\\
71.42	0.01\\
71.43	0.01\\
71.44	0.01\\
71.45	0.01\\
71.46	0.01\\
71.47	0.01\\
71.48	0.01\\
71.49	0.01\\
71.5	0.01\\
71.51	0.01\\
71.52	0.01\\
71.53	0.01\\
71.54	0.01\\
71.55	0.01\\
71.56	0.01\\
71.57	0.01\\
71.58	0.01\\
71.59	0.01\\
71.6	0.01\\
71.61	0.01\\
71.62	0.01\\
71.63	0.01\\
71.64	0.01\\
71.65	0.01\\
71.66	0.01\\
71.67	0.01\\
71.68	0.01\\
71.69	0.01\\
71.7	0.01\\
71.71	0.01\\
71.72	0.01\\
71.73	0.01\\
71.74	0.01\\
71.75	0.01\\
71.76	0.01\\
71.77	0.01\\
71.78	0.01\\
71.79	0.01\\
71.8	0.01\\
71.81	0.01\\
71.82	0.01\\
71.83	0.01\\
71.84	0.01\\
71.85	0.01\\
71.86	0.01\\
71.87	0.01\\
71.88	0.01\\
71.89	0.01\\
71.9	0.01\\
71.91	0.01\\
71.92	0.01\\
71.93	0.01\\
71.94	0.01\\
71.95	0.01\\
71.96	0.01\\
71.97	0.01\\
71.98	0.01\\
71.99	0.01\\
72	0.01\\
72.01	0.01\\
72.02	0.01\\
72.03	0.01\\
72.04	0.01\\
72.05	0.01\\
72.06	0.01\\
72.07	0.01\\
72.08	0.01\\
72.09	0.01\\
72.1	0.01\\
72.11	0.01\\
72.12	0.01\\
72.13	0.01\\
72.14	0.01\\
72.15	0.01\\
72.16	0.01\\
72.17	0.01\\
72.18	0.01\\
72.19	0.01\\
72.2	0.01\\
72.21	0.01\\
72.22	0.01\\
72.23	0.01\\
72.24	0.01\\
72.25	0.01\\
72.26	0.01\\
72.27	0.01\\
72.28	0.01\\
72.29	0.01\\
72.3	0.01\\
72.31	0.01\\
72.32	0.01\\
72.33	0.01\\
72.34	0.01\\
72.35	0.01\\
72.36	0.01\\
72.37	0.01\\
72.38	0.01\\
72.39	0.01\\
72.4	0.01\\
72.41	0.01\\
72.42	0.01\\
72.43	0.01\\
72.44	0.01\\
72.45	0.01\\
72.46	0.01\\
72.47	0.01\\
72.48	0.01\\
72.49	0.01\\
72.5	0.01\\
72.51	0.01\\
72.52	0.01\\
72.53	0.01\\
72.54	0.01\\
72.55	0.01\\
72.56	0.01\\
72.57	0.01\\
72.58	0.01\\
72.59	0.01\\
72.6	0.01\\
72.61	0.01\\
72.62	0.01\\
72.63	0.01\\
72.64	0.01\\
72.65	0.01\\
72.66	0.01\\
72.67	0.01\\
72.68	0.01\\
72.69	0.01\\
72.7	0.01\\
72.71	0.01\\
72.72	0.01\\
72.73	0.01\\
72.74	0.01\\
72.75	0.01\\
72.76	0.01\\
72.77	0.01\\
72.78	0.01\\
72.79	0.01\\
72.8	0.01\\
72.81	0.01\\
72.82	0.01\\
72.83	0.01\\
72.84	0.01\\
72.85	0.01\\
72.86	0.01\\
72.87	0.01\\
72.88	0.01\\
72.89	0.01\\
72.9	0.01\\
72.91	0.01\\
72.92	0.01\\
72.93	0.01\\
72.94	0.01\\
72.95	0.01\\
72.96	0.01\\
72.97	0.01\\
72.98	0.01\\
72.99	0.01\\
73	0.01\\
73.01	0.01\\
73.02	0.01\\
73.03	0.01\\
73.04	0.01\\
73.05	0.01\\
73.06	0.01\\
73.07	0.01\\
73.08	0.01\\
73.09	0.01\\
73.1	0.01\\
73.11	0.01\\
73.12	0.01\\
73.13	0.01\\
73.14	0.01\\
73.15	0.01\\
73.16	0.01\\
73.17	0.01\\
73.18	0.01\\
73.19	0.01\\
73.2	0.01\\
73.21	0.01\\
73.22	0.01\\
73.23	0.01\\
73.24	0.01\\
73.25	0.01\\
73.26	0.01\\
73.27	0.01\\
73.28	0.01\\
73.29	0.01\\
73.3	0.01\\
73.31	0.01\\
73.32	0.01\\
73.33	0.01\\
73.34	0.01\\
73.35	0.01\\
73.36	0.01\\
73.37	0.01\\
73.38	0.01\\
73.39	0.01\\
73.4	0.01\\
73.41	0.01\\
73.42	0.01\\
73.43	0.01\\
73.44	0.01\\
73.45	0.01\\
73.46	0.01\\
73.47	0.01\\
73.48	0.01\\
73.49	0.01\\
73.5	0.01\\
73.51	0.01\\
73.52	0.01\\
73.53	0.01\\
73.54	0.01\\
73.55	0.01\\
73.56	0.01\\
73.57	0.01\\
73.58	0.01\\
73.59	0.01\\
73.6	0.01\\
73.61	0.01\\
73.62	0.01\\
73.63	0.01\\
73.64	0.01\\
73.65	0.01\\
73.66	0.01\\
73.67	0.01\\
73.68	0.01\\
73.69	0.01\\
73.7	0.01\\
73.71	0.01\\
73.72	0.01\\
73.73	0.01\\
73.74	0.01\\
73.75	0.01\\
73.76	0.01\\
73.77	0.01\\
73.78	0.01\\
73.79	0.01\\
73.8	0.01\\
73.81	0.01\\
73.82	0.01\\
73.83	0.01\\
73.84	0.01\\
73.85	0.01\\
73.86	0.01\\
73.87	0.01\\
73.88	0.01\\
73.89	0.01\\
73.9	0.01\\
73.91	0.01\\
73.92	0.01\\
73.93	0.01\\
73.94	0.01\\
73.95	0.01\\
73.96	0.01\\
73.97	0.01\\
73.98	0.01\\
73.99	0.01\\
74	0.01\\
74.01	0.01\\
74.02	0.01\\
74.03	0.01\\
74.04	0.01\\
74.05	0.01\\
74.06	0.01\\
74.07	0.01\\
74.08	0.01\\
74.09	0.01\\
74.1	0.01\\
74.11	0.01\\
74.12	0.01\\
74.13	0.01\\
74.14	0.01\\
74.15	0.01\\
74.16	0.01\\
74.17	0.01\\
74.18	0.01\\
74.19	0.01\\
74.2	0.01\\
74.21	0.01\\
74.22	0.01\\
74.23	0.01\\
74.24	0.01\\
74.25	0.01\\
74.26	0.01\\
74.27	0.01\\
74.28	0.01\\
74.29	0.01\\
74.3	0.01\\
74.31	0.01\\
74.32	0.01\\
74.33	0.01\\
74.34	0.01\\
74.35	0.01\\
74.36	0.01\\
74.37	0.01\\
74.38	0.01\\
74.39	0.01\\
74.4	0.01\\
74.41	0.01\\
74.42	0.01\\
74.43	0.01\\
74.44	0.01\\
74.45	0.01\\
74.46	0.01\\
74.47	0.01\\
74.48	0.01\\
74.49	0.01\\
74.5	0.01\\
74.51	0.01\\
74.52	0.01\\
74.53	0.01\\
74.54	0.01\\
74.55	0.01\\
74.56	0.01\\
74.57	0.01\\
74.58	0.01\\
74.59	0.01\\
74.6	0.01\\
74.61	0.01\\
74.62	0.01\\
74.63	0.01\\
74.64	0.01\\
74.65	0.01\\
74.66	0.01\\
74.67	0.01\\
74.68	0.01\\
74.69	0.01\\
74.7	0.01\\
74.71	0.01\\
74.72	0.01\\
74.73	0.01\\
74.74	0.01\\
74.75	0.01\\
74.76	0.01\\
74.77	0.01\\
74.78	0.01\\
74.79	0.01\\
74.8	0.01\\
74.81	0.01\\
74.82	0.01\\
74.83	0.01\\
74.84	0.01\\
74.85	0.01\\
74.86	0.01\\
74.87	0.01\\
74.88	0.01\\
74.89	0.01\\
74.9	0.01\\
74.91	0.01\\
74.92	0.01\\
74.93	0.01\\
74.94	0.01\\
74.95	0.01\\
74.96	0.01\\
74.97	0.01\\
74.98	0.01\\
74.99	0.01\\
75	0.01\\
75.01	0.01\\
75.02	0.01\\
75.03	0.01\\
75.04	0.01\\
75.05	0.01\\
75.06	0.01\\
75.07	0.01\\
75.08	0.01\\
75.09	0.01\\
75.1	0.01\\
75.11	0.01\\
75.12	0.01\\
75.13	0.01\\
75.14	0.01\\
75.15	0.01\\
75.16	0.01\\
75.17	0.01\\
75.18	0.01\\
75.19	0.01\\
75.2	0.01\\
75.21	0.01\\
75.22	0.01\\
75.23	0.01\\
75.24	0.01\\
75.25	0.01\\
75.26	0.01\\
75.27	0.01\\
75.28	0.01\\
75.29	0.01\\
75.3	0.01\\
75.31	0.01\\
75.32	0.01\\
75.33	0.01\\
75.34	0.01\\
75.35	0.01\\
75.36	0.01\\
75.37	0.01\\
75.38	0.01\\
75.39	0.01\\
75.4	0.01\\
75.41	0.01\\
75.42	0.01\\
75.43	0.01\\
75.44	0.01\\
75.45	0.01\\
75.46	0.01\\
75.47	0.01\\
75.48	0.01\\
75.49	0.01\\
75.5	0.01\\
75.51	0.01\\
75.52	0.01\\
75.53	0.01\\
75.54	0.01\\
75.55	0.01\\
75.56	0.01\\
75.57	0.01\\
75.58	0.01\\
75.59	0.01\\
75.6	0.01\\
75.61	0.01\\
75.62	0.01\\
75.63	0.01\\
75.64	0.01\\
75.65	0.01\\
75.66	0.01\\
75.67	0.01\\
75.68	0.01\\
75.69	0.01\\
75.7	0.01\\
75.71	0.01\\
75.72	0.01\\
75.73	0.01\\
75.74	0.01\\
75.75	0.01\\
75.76	0.01\\
75.77	0.01\\
75.78	0.01\\
75.79	0.01\\
75.8	0.01\\
75.81	0.01\\
75.82	0.01\\
75.83	0.01\\
75.84	0.01\\
75.85	0.01\\
75.86	0.01\\
75.87	0.01\\
75.88	0.01\\
75.89	0.01\\
75.9	0.01\\
75.91	0.01\\
75.92	0.01\\
75.93	0.01\\
75.94	0.01\\
75.95	0.01\\
75.96	0.01\\
75.97	0.01\\
75.98	0.01\\
75.99	0.01\\
76	0.01\\
76.01	0.01\\
76.02	0.01\\
76.03	0.01\\
76.04	0.01\\
76.05	0.01\\
76.06	0.01\\
76.07	0.01\\
76.08	0.01\\
76.09	0.01\\
76.1	0.01\\
76.11	0.01\\
76.12	0.01\\
76.13	0.01\\
76.14	0.01\\
76.15	0.01\\
76.16	0.01\\
76.17	0.01\\
76.18	0.01\\
76.19	0.01\\
76.2	0.01\\
76.21	0.01\\
76.22	0.01\\
76.23	0.01\\
76.24	0.01\\
76.25	0.01\\
76.26	0.01\\
76.27	0.01\\
76.28	0.01\\
76.29	0.01\\
76.3	0.01\\
76.31	0.01\\
76.32	0.01\\
76.33	0.01\\
76.34	0.01\\
76.35	0.01\\
76.36	0.01\\
76.37	0.01\\
76.38	0.01\\
76.39	0.01\\
76.4	0.01\\
76.41	0.01\\
76.42	0.01\\
76.43	0.01\\
76.44	0.01\\
76.45	0.01\\
76.46	0.01\\
76.47	0.01\\
76.48	0.01\\
76.49	0.01\\
76.5	0.01\\
76.51	0.01\\
76.52	0.01\\
76.53	0.01\\
76.54	0.01\\
76.55	0.01\\
76.56	0.01\\
76.57	0.01\\
76.58	0.01\\
76.59	0.01\\
76.6	0.01\\
76.61	0.01\\
76.62	0.01\\
76.63	0.01\\
76.64	0.01\\
76.65	0.01\\
76.66	0.01\\
76.67	0.01\\
76.68	0.01\\
76.69	0.01\\
76.7	0.01\\
76.71	0.01\\
76.72	0.01\\
76.73	0.01\\
76.74	0.01\\
76.75	0.01\\
76.76	0.01\\
76.77	0.01\\
76.78	0.01\\
76.79	0.01\\
76.8	0.01\\
76.81	0.01\\
76.82	0.01\\
76.83	0.01\\
76.84	0.01\\
76.85	0.01\\
76.86	0.01\\
76.87	0.01\\
76.88	0.01\\
76.89	0.01\\
76.9	0.01\\
76.91	0.01\\
76.92	0.01\\
76.93	0.01\\
76.94	0.01\\
76.95	0.01\\
76.96	0.01\\
76.97	0.01\\
76.98	0.01\\
76.99	0.01\\
77	0.01\\
77.01	0.01\\
77.02	0.01\\
77.03	0.01\\
77.04	0.01\\
77.05	0.01\\
77.06	0.01\\
77.07	0.01\\
77.08	0.01\\
77.09	0.01\\
77.1	0.01\\
77.11	0.01\\
77.12	0.01\\
77.13	0.01\\
77.14	0.01\\
77.15	0.01\\
77.16	0.01\\
77.17	0.01\\
77.18	0.01\\
77.19	0.01\\
77.2	0.01\\
77.21	0.01\\
77.22	0.01\\
77.23	0.01\\
77.24	0.01\\
77.25	0.01\\
77.26	0.01\\
77.27	0.01\\
77.28	0.01\\
77.29	0.01\\
77.3	0.01\\
77.31	0.01\\
77.32	0.01\\
77.33	0.01\\
77.34	0.01\\
77.35	0.01\\
77.36	0.01\\
77.37	0.01\\
77.38	0.01\\
77.39	0.01\\
77.4	0.01\\
77.41	0.01\\
77.42	0.01\\
77.43	0.01\\
77.44	0.01\\
77.45	0.01\\
77.46	0.01\\
77.47	0.01\\
77.48	0.01\\
77.49	0.01\\
77.5	0.01\\
77.51	0.01\\
77.52	0.01\\
77.53	0.01\\
77.54	0.01\\
77.55	0.01\\
77.56	0.01\\
77.57	0.01\\
77.58	0.01\\
77.59	0.01\\
77.6	0.01\\
77.61	0.01\\
77.62	0.01\\
77.63	0.01\\
77.64	0.01\\
77.65	0.01\\
77.66	0.01\\
77.67	0.01\\
77.68	0.01\\
77.69	0.01\\
77.7	0.01\\
77.71	0.01\\
77.72	0.01\\
77.73	0.01\\
77.74	0.01\\
77.75	0.01\\
77.76	0.01\\
77.77	0.01\\
77.78	0.01\\
77.79	0.01\\
77.8	0.01\\
77.81	0.01\\
77.82	0.01\\
77.83	0.01\\
77.84	0.01\\
77.85	0.01\\
77.86	0.01\\
77.87	0.01\\
77.88	0.01\\
77.89	0.01\\
77.9	0.01\\
77.91	0.01\\
77.92	0.01\\
77.93	0.01\\
77.94	0.01\\
77.95	0.01\\
77.96	0.01\\
77.97	0.01\\
77.98	0.01\\
77.99	0.01\\
78	0.01\\
78.01	0.01\\
78.02	0.01\\
78.03	0.01\\
78.04	0.01\\
78.05	0.01\\
78.06	0.01\\
78.07	0.01\\
78.08	0.01\\
78.09	0.01\\
78.1	0.01\\
78.11	0.01\\
78.12	0.01\\
78.13	0.01\\
78.14	0.01\\
78.15	0.01\\
78.16	0.01\\
78.17	0.01\\
78.18	0.01\\
78.19	0.01\\
78.2	0.01\\
78.21	0.01\\
78.22	0.01\\
78.23	0.01\\
78.24	0.01\\
78.25	0.01\\
78.26	0.01\\
78.27	0.01\\
78.28	0.01\\
78.29	0.01\\
78.3	0.01\\
78.31	0.01\\
78.32	0.01\\
78.33	0.01\\
78.34	0.01\\
78.35	0.01\\
78.36	0.01\\
78.37	0.01\\
78.38	0.01\\
78.39	0.01\\
78.4	0.01\\
78.41	0.01\\
78.42	0.01\\
78.43	0.01\\
78.44	0.01\\
78.45	0.01\\
78.46	0.01\\
78.47	0.01\\
78.48	0.01\\
78.49	0.01\\
78.5	0.01\\
78.51	0.01\\
78.52	0.01\\
78.53	0.01\\
78.54	0.01\\
78.55	0.01\\
78.56	0.01\\
78.57	0.01\\
78.58	0.01\\
78.59	0.01\\
78.6	0.01\\
78.61	0.01\\
78.62	0.01\\
78.63	0.01\\
78.64	0.01\\
78.65	0.01\\
78.66	0.01\\
78.67	0.01\\
78.68	0.01\\
78.69	0.01\\
78.7	0.01\\
78.71	0.01\\
78.72	0.01\\
78.73	0.01\\
78.74	0.01\\
78.75	0.01\\
78.76	0.01\\
78.77	0.01\\
78.78	0.01\\
78.79	0.01\\
78.8	0.01\\
78.81	0.01\\
78.82	0.01\\
78.83	0.01\\
78.84	0.01\\
78.85	0.01\\
78.86	0.01\\
78.87	0.01\\
78.88	0.01\\
78.89	0.01\\
78.9	0.01\\
78.91	0.01\\
78.92	0.01\\
78.93	0.01\\
78.94	0.01\\
78.95	0.01\\
78.96	0.01\\
78.97	0.01\\
78.98	0.01\\
78.99	0.01\\
79	0.01\\
79.01	0.01\\
79.02	0.01\\
79.03	0.01\\
79.04	0.01\\
79.05	0.01\\
79.06	0.01\\
79.07	0.01\\
79.08	0.01\\
79.09	0.01\\
79.1	0.01\\
79.11	0.01\\
79.12	0.01\\
79.13	0.01\\
79.14	0.01\\
79.15	0.01\\
79.16	0.01\\
79.17	0.01\\
79.18	0.01\\
79.19	0.01\\
79.2	0.01\\
79.21	0.01\\
79.22	0.01\\
79.23	0.01\\
79.24	0.01\\
79.25	0.01\\
79.26	0.01\\
79.27	0.01\\
79.28	0.01\\
79.29	0.01\\
79.3	0.01\\
79.31	0.01\\
79.32	0.01\\
79.33	0.01\\
79.34	0.01\\
79.35	0.01\\
79.36	0.01\\
79.37	0.01\\
79.38	0.01\\
79.39	0.01\\
79.4	0.01\\
79.41	0.01\\
79.42	0.01\\
79.43	0.01\\
79.44	0.01\\
79.45	0.01\\
79.46	0.01\\
79.47	0.01\\
79.48	0.01\\
79.49	0.01\\
79.5	0.01\\
79.51	0.01\\
79.52	0.01\\
79.53	0.01\\
79.54	0.01\\
79.55	0.01\\
79.56	0.01\\
79.57	0.01\\
79.58	0.01\\
79.59	0.01\\
79.6	0.01\\
79.61	0.01\\
79.62	0.01\\
79.63	0.01\\
79.64	0.01\\
79.65	0.01\\
79.66	0.01\\
79.67	0.01\\
79.68	0.01\\
79.69	0.01\\
79.7	0.01\\
79.71	0.01\\
79.72	0.01\\
79.73	0.01\\
79.74	0.01\\
79.75	0.01\\
79.76	0.01\\
79.77	0.01\\
79.78	0.01\\
79.79	0.01\\
79.8	0.01\\
79.81	0.01\\
79.82	0.01\\
79.83	0.01\\
79.84	0.01\\
79.85	0.01\\
79.86	0.01\\
79.87	0.01\\
79.88	0.01\\
79.89	0.01\\
79.9	0.01\\
79.91	0.01\\
79.92	0.01\\
79.93	0.01\\
79.94	0.01\\
79.95	0.01\\
79.96	0.01\\
79.97	0.01\\
79.98	0.01\\
79.99	0.01\\
80	0.01\\
80.01	0.01\\
};
\addplot [color=mycolor1,dashed]
  table[row sep=crcr]{%
80.01	0.01\\
80.02	0.01\\
80.03	0.01\\
80.04	0.01\\
80.05	0.01\\
80.06	0.01\\
80.07	0.01\\
80.08	0.01\\
80.09	0.01\\
80.1	0.01\\
80.11	0.01\\
80.12	0.01\\
80.13	0.01\\
80.14	0.01\\
80.15	0.01\\
80.16	0.01\\
80.17	0.01\\
80.18	0.01\\
80.19	0.01\\
80.2	0.01\\
80.21	0.01\\
80.22	0.01\\
80.23	0.01\\
80.24	0.01\\
80.25	0.01\\
80.26	0.01\\
80.27	0.01\\
80.28	0.01\\
80.29	0.01\\
80.3	0.01\\
80.31	0.01\\
80.32	0.01\\
80.33	0.01\\
80.34	0.01\\
80.35	0.01\\
80.36	0.01\\
80.37	0.01\\
80.38	0.01\\
80.39	0.01\\
80.4	0.01\\
80.41	0.01\\
80.42	0.01\\
80.43	0.01\\
80.44	0.01\\
80.45	0.01\\
80.46	0.01\\
80.47	0.01\\
80.48	0.01\\
80.49	0.01\\
80.5	0.01\\
80.51	0.01\\
80.52	0.01\\
80.53	0.01\\
80.54	0.01\\
80.55	0.01\\
80.56	0.01\\
80.57	0.01\\
80.58	0.01\\
80.59	0.01\\
80.6	0.01\\
80.61	0.01\\
80.62	0.01\\
80.63	0.01\\
80.64	0.01\\
80.65	0.01\\
80.66	0.01\\
80.67	0.01\\
80.68	0.01\\
80.69	0.01\\
80.7	0.01\\
80.71	0.01\\
80.72	0.01\\
80.73	0.01\\
80.74	0.01\\
80.75	0.01\\
80.76	0.01\\
80.77	0.01\\
80.78	0.01\\
80.79	0.01\\
80.8	0.01\\
80.81	0.01\\
80.82	0.01\\
80.83	0.01\\
80.84	0.01\\
80.85	0.01\\
80.86	0.01\\
80.87	0.01\\
80.88	0.01\\
80.89	0.01\\
80.9	0.01\\
80.91	0.01\\
80.92	0.01\\
80.93	0.01\\
80.94	0.01\\
80.95	0.01\\
80.96	0.01\\
80.97	0.01\\
80.98	0.01\\
80.99	0.01\\
81	0.01\\
81.01	0.01\\
81.02	0.01\\
81.03	0.01\\
81.04	0.01\\
81.05	0.01\\
81.06	0.01\\
81.07	0.01\\
81.08	0.01\\
81.09	0.01\\
81.1	0.01\\
81.11	0.01\\
81.12	0.01\\
81.13	0.01\\
81.14	0.01\\
81.15	0.01\\
81.16	0.01\\
81.17	0.01\\
81.18	0.01\\
81.19	0.01\\
81.2	0.01\\
81.21	0.01\\
81.22	0.01\\
81.23	0.01\\
81.24	0.01\\
81.25	0.01\\
81.26	0.01\\
81.27	0.01\\
81.28	0.01\\
81.29	0.01\\
81.3	0.01\\
81.31	0.01\\
81.32	0.01\\
81.33	0.01\\
81.34	0.01\\
81.35	0.01\\
81.36	0.01\\
81.37	0.01\\
81.38	0.01\\
81.39	0.01\\
81.4	0.01\\
81.41	0.01\\
81.42	0.01\\
81.43	0.01\\
81.44	0.01\\
81.45	0.01\\
81.46	0.01\\
81.47	0.01\\
81.48	0.01\\
81.49	0.01\\
81.5	0.01\\
81.51	0.01\\
81.52	0.01\\
81.53	0.01\\
81.54	0.01\\
81.55	0.01\\
81.56	0.01\\
81.57	0.01\\
81.58	0.01\\
81.59	0.01\\
81.6	0.01\\
81.61	0.01\\
81.62	0.01\\
81.63	0.01\\
81.64	0.01\\
81.65	0.01\\
81.66	0.01\\
81.67	0.01\\
81.68	0.01\\
81.69	0.01\\
81.7	0.01\\
81.71	0.01\\
81.72	0.01\\
81.73	0.01\\
81.74	0.01\\
81.75	0.01\\
81.76	0.01\\
81.77	0.01\\
81.78	0.01\\
81.79	0.01\\
81.8	0.01\\
81.81	0.01\\
81.82	0.01\\
81.83	0.01\\
81.84	0.01\\
81.85	0.01\\
81.86	0.01\\
81.87	0.01\\
81.88	0.01\\
81.89	0.01\\
81.9	0.01\\
81.91	0.01\\
81.92	0.01\\
81.93	0.01\\
81.94	0.01\\
81.95	0.01\\
81.96	0.01\\
81.97	0.01\\
81.98	0.01\\
81.99	0.01\\
82	0.01\\
82.01	0.01\\
82.02	0.01\\
82.03	0.01\\
82.04	0.01\\
82.05	0.01\\
82.06	0.01\\
82.07	0.01\\
82.08	0.01\\
82.09	0.01\\
82.1	0.01\\
82.11	0.01\\
82.12	0.01\\
82.13	0.01\\
82.14	0.01\\
82.15	0.01\\
82.16	0.01\\
82.17	0.01\\
82.18	0.01\\
82.19	0.01\\
82.2	0.01\\
82.21	0.01\\
82.22	0.01\\
82.23	0.01\\
82.24	0.01\\
82.25	0.01\\
82.26	0.01\\
82.27	0.01\\
82.28	0.01\\
82.29	0.01\\
82.3	0.01\\
82.31	0.01\\
82.32	0.01\\
82.33	0.01\\
82.34	0.01\\
82.35	0.01\\
82.36	0.01\\
82.37	0.01\\
82.38	0.01\\
82.39	0.01\\
82.4	0.01\\
82.41	0.01\\
82.42	0.01\\
82.43	0.01\\
82.44	0.01\\
82.45	0.01\\
82.46	0.01\\
82.47	0.01\\
82.48	0.01\\
82.49	0.01\\
82.5	0.01\\
82.51	0.01\\
82.52	0.01\\
82.53	0.01\\
82.54	0.01\\
82.55	0.01\\
82.56	0.01\\
82.57	0.01\\
82.58	0.01\\
82.59	0.01\\
82.6	0.01\\
82.61	0.01\\
82.62	0.01\\
82.63	0.01\\
82.64	0.01\\
82.65	0.01\\
82.66	0.01\\
82.67	0.01\\
82.68	0.01\\
82.69	0.01\\
82.7	0.01\\
82.71	0.01\\
82.72	0.01\\
82.73	0.01\\
82.74	0.01\\
82.75	0.01\\
82.76	0.01\\
82.77	0.01\\
82.78	0.01\\
82.79	0.01\\
82.8	0.01\\
82.81	0.01\\
82.82	0.01\\
82.83	0.01\\
82.84	0.01\\
82.85	0.01\\
82.86	0.01\\
82.87	0.01\\
82.88	0.01\\
82.89	0.01\\
82.9	0.01\\
82.91	0.01\\
82.92	0.01\\
82.93	0.01\\
82.94	0.01\\
82.95	0.01\\
82.96	0.01\\
82.97	0.01\\
82.98	0.01\\
82.99	0.01\\
83	0.01\\
83.01	0.01\\
83.02	0.01\\
83.03	0.01\\
83.04	0.01\\
83.05	0.01\\
83.06	0.01\\
83.07	0.01\\
83.08	0.01\\
83.09	0.01\\
83.1	0.01\\
83.11	0.01\\
83.12	0.01\\
83.13	0.01\\
83.14	0.01\\
83.15	0.01\\
83.16	0.01\\
83.17	0.01\\
83.18	0.01\\
83.19	0.01\\
83.2	0.01\\
83.21	0.01\\
83.22	0.01\\
83.23	0.01\\
83.24	0.01\\
83.25	0.01\\
83.26	0.01\\
83.27	0.01\\
83.28	0.01\\
83.29	0.01\\
83.3	0.01\\
83.31	0.01\\
83.32	0.01\\
83.33	0.01\\
83.34	0.01\\
83.35	0.01\\
83.36	0.01\\
83.37	0.01\\
83.38	0.01\\
83.39	0.01\\
83.4	0.01\\
83.41	0.01\\
83.42	0.01\\
83.43	0.01\\
83.44	0.01\\
83.45	0.01\\
83.46	0.01\\
83.47	0.01\\
83.48	0.01\\
83.49	0.01\\
83.5	0.01\\
83.51	0.01\\
83.52	0.01\\
83.53	0.01\\
83.54	0.01\\
83.55	0.01\\
83.56	0.01\\
83.57	0.01\\
83.58	0.01\\
83.59	0.01\\
83.6	0.01\\
83.61	0.01\\
83.62	0.01\\
83.63	0.01\\
83.64	0.01\\
83.65	0.01\\
83.66	0.01\\
83.67	0.01\\
83.68	0.01\\
83.69	0.01\\
83.7	0.01\\
83.71	0.01\\
83.72	0.01\\
83.73	0.01\\
83.74	0.01\\
83.75	0.01\\
83.76	0.01\\
83.77	0.01\\
83.78	0.01\\
83.79	0.01\\
83.8	0.01\\
83.81	0.01\\
83.82	0.01\\
83.83	0.01\\
83.84	0.01\\
83.85	0.01\\
83.86	0.01\\
83.87	0.01\\
83.88	0.01\\
83.89	0.01\\
83.9	0.01\\
83.91	0.01\\
83.92	0.01\\
83.93	0.01\\
83.94	0.01\\
83.95	0.01\\
83.96	0.01\\
83.97	0.01\\
83.98	0.01\\
83.99	0.01\\
84	0.01\\
84.01	0.01\\
84.02	0.01\\
84.03	0.01\\
84.04	0.01\\
84.05	0.01\\
84.06	0.01\\
84.07	0.01\\
84.08	0.01\\
84.09	0.01\\
84.1	0.01\\
84.11	0.01\\
84.12	0.01\\
84.13	0.01\\
84.14	0.01\\
84.15	0.01\\
84.16	0.01\\
84.17	0.01\\
84.18	0.01\\
84.19	0.01\\
84.2	0.01\\
84.21	0.01\\
84.22	0.01\\
84.23	0.01\\
84.24	0.01\\
84.25	0.01\\
84.26	0.01\\
84.27	0.01\\
84.28	0.01\\
84.29	0.01\\
84.3	0.01\\
84.31	0.01\\
84.32	0.01\\
84.33	0.01\\
84.34	0.01\\
84.35	0.01\\
84.36	0.01\\
84.37	0.01\\
84.38	0.01\\
84.39	0.01\\
84.4	0.01\\
84.41	0.01\\
84.42	0.01\\
84.43	0.01\\
84.44	0.01\\
84.45	0.01\\
84.46	0.01\\
84.47	0.01\\
84.48	0.01\\
84.49	0.01\\
84.5	0.01\\
84.51	0.01\\
84.52	0.01\\
84.53	0.01\\
84.54	0.01\\
84.55	0.01\\
84.56	0.01\\
84.57	0.01\\
84.58	0.01\\
84.59	0.01\\
84.6	0.01\\
84.61	0.01\\
84.62	0.01\\
84.63	0.01\\
84.64	0.01\\
84.65	0.01\\
84.66	0.01\\
84.67	0.01\\
84.68	0.01\\
84.69	0.01\\
84.7	0.01\\
84.71	0.01\\
84.72	0.01\\
84.73	0.01\\
84.74	0.01\\
84.75	0.01\\
84.76	0.01\\
84.77	0.01\\
84.78	0.01\\
84.79	0.01\\
84.8	0.01\\
84.81	0.01\\
84.82	0.01\\
84.83	0.01\\
84.84	0.01\\
84.85	0.01\\
84.86	0.01\\
84.87	0.01\\
84.88	0.01\\
84.89	0.01\\
84.9	0.01\\
84.91	0.01\\
84.92	0.01\\
84.93	0.01\\
84.94	0.01\\
84.95	0.01\\
84.96	0.01\\
84.97	0.01\\
84.98	0.01\\
84.99	0.01\\
85	0.01\\
85.01	0.01\\
85.02	0.01\\
85.03	0.01\\
85.04	0.01\\
85.05	0.01\\
85.06	0.01\\
85.07	0.01\\
85.08	0.01\\
85.09	0.01\\
85.1	0.01\\
85.11	0.01\\
85.12	0.01\\
85.13	0.01\\
85.14	0.01\\
85.15	0.01\\
85.16	0.01\\
85.17	0.01\\
85.18	0.01\\
85.19	0.01\\
85.2	0.01\\
85.21	0.01\\
85.22	0.01\\
85.23	0.01\\
85.24	0.01\\
85.25	0.01\\
85.26	0.01\\
85.27	0.01\\
85.28	0.01\\
85.29	0.01\\
85.3	0.01\\
85.31	0.01\\
85.32	0.01\\
85.33	0.01\\
85.34	0.01\\
85.35	0.01\\
85.36	0.01\\
85.37	0.01\\
85.38	0.01\\
85.39	0.01\\
85.4	0.01\\
85.41	0.01\\
85.42	0.01\\
85.43	0.01\\
85.44	0.01\\
85.45	0.01\\
85.46	0.01\\
85.47	0.01\\
85.48	0.01\\
85.49	0.01\\
85.5	0.01\\
85.51	0.01\\
85.52	0.01\\
85.53	0.01\\
85.54	0.01\\
85.55	0.01\\
85.56	0.01\\
85.57	0.01\\
85.58	0.01\\
85.59	0.01\\
85.6	0.01\\
85.61	0.01\\
85.62	0.01\\
85.63	0.01\\
85.64	0.01\\
85.65	0.01\\
85.66	0.01\\
85.67	0.01\\
85.68	0.01\\
85.69	0.01\\
85.7	0.01\\
85.71	0.01\\
85.72	0.01\\
85.73	0.01\\
85.74	0.01\\
85.75	0.01\\
85.76	0.01\\
85.77	0.01\\
85.78	0.01\\
85.79	0.01\\
85.8	0.01\\
85.81	0.01\\
85.82	0.01\\
85.83	0.01\\
85.84	0.01\\
85.85	0.01\\
85.86	0.01\\
85.87	0.01\\
85.88	0.01\\
85.89	0.01\\
85.9	0.01\\
85.91	0.01\\
85.92	0.01\\
85.93	0.01\\
85.94	0.01\\
85.95	0.01\\
85.96	0.01\\
85.97	0.01\\
85.98	0.01\\
85.99	0.01\\
86	0.01\\
86.01	0.01\\
86.02	0.01\\
86.03	0.01\\
86.04	0.01\\
86.05	0.01\\
86.06	0.01\\
86.07	0.01\\
86.08	0.01\\
86.09	0.01\\
86.1	0.01\\
86.11	0.01\\
86.12	0.01\\
86.13	0.01\\
86.14	0.01\\
86.15	0.01\\
86.16	0.01\\
86.17	0.01\\
86.18	0.01\\
86.19	0.01\\
86.2	0.01\\
86.21	0.01\\
86.22	0.01\\
86.23	0.01\\
86.24	0.01\\
86.25	0.01\\
86.26	0.01\\
86.27	0.01\\
86.28	0.01\\
86.29	0.01\\
86.3	0.01\\
86.31	0.01\\
86.32	0.01\\
86.33	0.01\\
86.34	0.01\\
86.35	0.01\\
86.36	0.01\\
86.37	0.01\\
86.38	0.01\\
86.39	0.01\\
86.4	0.01\\
86.41	0.01\\
86.42	0.01\\
86.43	0.01\\
86.44	0.01\\
86.45	0.01\\
86.46	0.01\\
86.47	0.01\\
86.48	0.01\\
86.49	0.01\\
86.5	0.01\\
86.51	0.01\\
86.52	0.01\\
86.53	0.01\\
86.54	0.01\\
86.55	0.01\\
86.56	0.01\\
86.57	0.01\\
86.58	0.01\\
86.59	0.01\\
86.6	0.01\\
86.61	0.01\\
86.62	0.01\\
86.63	0.01\\
86.64	0.01\\
86.65	0.01\\
86.66	0.01\\
86.67	0.01\\
86.68	0.01\\
86.69	0.01\\
86.7	0.01\\
86.71	0.01\\
86.72	0.01\\
86.73	0.01\\
86.74	0.01\\
86.75	0.01\\
86.76	0.01\\
86.77	0.01\\
86.78	0.01\\
86.79	0.01\\
86.8	0.01\\
86.81	0.01\\
86.82	0.01\\
86.83	0.01\\
86.84	0.01\\
86.85	0.01\\
86.86	0.01\\
86.87	0.01\\
86.88	0.01\\
86.89	0.01\\
86.9	0.01\\
86.91	0.01\\
86.92	0.01\\
86.93	0.01\\
86.94	0.01\\
86.95	0.01\\
86.96	0.01\\
86.97	0.01\\
86.98	0.01\\
86.99	0.01\\
87	0.01\\
87.01	0.01\\
87.02	0.01\\
87.03	0.01\\
87.04	0.01\\
87.05	0.01\\
87.06	0.01\\
87.07	0.01\\
87.08	0.01\\
87.09	0.01\\
87.1	0.01\\
87.11	0.01\\
87.12	0.01\\
87.13	0.01\\
87.14	0.01\\
87.15	0.01\\
87.16	0.01\\
87.17	0.01\\
87.18	0.01\\
87.19	0.01\\
87.2	0.01\\
87.21	0.01\\
87.22	0.01\\
87.23	0.01\\
87.24	0.01\\
87.25	0.01\\
87.26	0.01\\
87.27	0.01\\
87.28	0.01\\
87.29	0.01\\
87.3	0.01\\
87.31	0.01\\
87.32	0.01\\
87.33	0.01\\
87.34	0.01\\
87.35	0.01\\
87.36	0.01\\
87.37	0.01\\
87.38	0.01\\
87.39	0.01\\
87.4	0.01\\
87.41	0.01\\
87.42	0.01\\
87.43	0.01\\
87.44	0.01\\
87.45	0.01\\
87.46	0.01\\
87.47	0.01\\
87.48	0.01\\
87.49	0.01\\
87.5	0.01\\
87.51	0.01\\
87.52	0.01\\
87.53	0.01\\
87.54	0.01\\
87.55	0.01\\
87.56	0.01\\
87.57	0.01\\
87.58	0.01\\
87.59	0.01\\
87.6	0.01\\
87.61	0.01\\
87.62	0.01\\
87.63	0.01\\
87.64	0.01\\
87.65	0.01\\
87.66	0.01\\
87.67	0.01\\
87.68	0.01\\
87.69	0.01\\
87.7	0.01\\
87.71	0.01\\
87.72	0.01\\
87.73	0.01\\
87.74	0.01\\
87.75	0.01\\
87.76	0.01\\
87.77	0.01\\
87.78	0.01\\
87.79	0.01\\
87.8	0.01\\
87.81	0.01\\
87.82	0.01\\
87.83	0.01\\
87.84	0.01\\
87.85	0.01\\
87.86	0.01\\
87.87	0.01\\
87.88	0.01\\
87.89	0.01\\
87.9	0.01\\
87.91	0.01\\
87.92	0.01\\
87.93	0.01\\
87.94	0.01\\
87.95	0.01\\
87.96	0.01\\
87.97	0.01\\
87.98	0.01\\
87.99	0.01\\
88	0.01\\
88.01	0.01\\
88.02	0.01\\
88.03	0.01\\
88.04	0.01\\
88.05	0.01\\
88.06	0.01\\
88.07	0.01\\
88.08	0.01\\
88.09	0.01\\
88.1	0.01\\
88.11	0.01\\
88.12	0.01\\
88.13	0.01\\
88.14	0.01\\
88.15	0.01\\
88.16	0.01\\
88.17	0.01\\
88.18	0.01\\
88.19	0.01\\
88.2	0.01\\
88.21	0.01\\
88.22	0.01\\
88.23	0.01\\
88.24	0.01\\
88.25	0.01\\
88.26	0.01\\
88.27	0.01\\
88.28	0.01\\
88.29	0.01\\
88.3	0.01\\
88.31	0.01\\
88.32	0.01\\
88.33	0.01\\
88.34	0.01\\
88.35	0.01\\
88.36	0.01\\
88.37	0.01\\
88.38	0.01\\
88.39	0.01\\
88.4	0.01\\
88.41	0.01\\
88.42	0.01\\
88.43	0.01\\
88.44	0.01\\
88.45	0.01\\
88.46	0.01\\
88.47	0.01\\
88.48	0.01\\
88.49	0.01\\
88.5	0.01\\
88.51	0.01\\
88.52	0.01\\
88.53	0.01\\
88.54	0.01\\
88.55	0.01\\
88.56	0.01\\
88.57	0.01\\
88.58	0.01\\
88.59	0.01\\
88.6	0.01\\
88.61	0.01\\
88.62	0.01\\
88.63	0.01\\
88.64	0.01\\
88.65	0.01\\
88.66	0.01\\
88.67	0.01\\
88.68	0.01\\
88.69	0.01\\
88.7	0.01\\
88.71	0.01\\
88.72	0.01\\
88.73	0.01\\
88.74	0.01\\
88.75	0.01\\
88.76	0.01\\
88.77	0.01\\
88.78	0.01\\
88.79	0.01\\
88.8	0.01\\
88.81	0.01\\
88.82	0.01\\
88.83	0.01\\
88.84	0.01\\
88.85	0.01\\
88.86	0.01\\
88.87	0.01\\
88.88	0.01\\
88.89	0.01\\
88.9	0.01\\
88.91	0.01\\
88.92	0.01\\
88.93	0.01\\
88.94	0.01\\
88.95	0.01\\
88.96	0.01\\
88.97	0.01\\
88.98	0.01\\
88.99	0.01\\
89	0.01\\
89.01	0.01\\
89.02	0.01\\
89.03	0.01\\
89.04	0.01\\
89.05	0.01\\
89.06	0.01\\
89.07	0.01\\
89.08	0.01\\
89.09	0.01\\
89.1	0.01\\
89.11	0.01\\
89.12	0.01\\
89.13	0.01\\
89.14	0.01\\
89.15	0.01\\
89.16	0.01\\
89.17	0.01\\
89.18	0.01\\
89.19	0.01\\
89.2	0.01\\
89.21	0.01\\
89.22	0.01\\
89.23	0.01\\
89.24	0.01\\
89.25	0.01\\
89.26	0.01\\
89.27	0.01\\
89.28	0.01\\
89.29	0.01\\
89.3	0.01\\
89.31	0.01\\
89.32	0.01\\
89.33	0.01\\
89.34	0.01\\
89.35	0.01\\
89.36	0.01\\
89.37	0.01\\
89.38	0.01\\
89.39	0.01\\
89.4	0.01\\
89.41	0.01\\
89.42	0.01\\
89.43	0.01\\
89.44	0.01\\
89.45	0.01\\
89.46	0.01\\
89.47	0.01\\
89.48	0.01\\
89.49	0.01\\
89.5	0.01\\
89.51	0.01\\
89.52	0.01\\
89.53	0.01\\
89.54	0.01\\
89.55	0.01\\
89.56	0.01\\
89.57	0.01\\
89.58	0.01\\
89.59	0.01\\
89.6	0.01\\
89.61	0.01\\
89.62	0.01\\
89.63	0.01\\
89.64	0.01\\
89.65	0.01\\
89.66	0.01\\
89.67	0.01\\
89.68	0.01\\
89.69	0.01\\
89.7	0.01\\
89.71	0.01\\
89.72	0.01\\
89.73	0.01\\
89.74	0.01\\
89.75	0.01\\
89.76	0.01\\
89.77	0.01\\
89.78	0.01\\
89.79	0.01\\
89.8	0.01\\
89.81	0.01\\
89.82	0.01\\
89.83	0.01\\
89.84	0.01\\
89.85	0.01\\
89.86	0.01\\
89.87	0.01\\
89.88	0.01\\
89.89	0.01\\
89.9	0.01\\
89.91	0.01\\
89.92	0.01\\
89.93	0.01\\
89.94	0.01\\
89.95	0.01\\
89.96	0.01\\
89.97	0.01\\
89.98	0.01\\
89.99	0.01\\
90	0.01\\
90.01	0.01\\
90.02	0.01\\
90.03	0.01\\
90.04	0.01\\
90.05	0.01\\
90.06	0.01\\
90.07	0.01\\
90.08	0.01\\
90.09	0.01\\
90.1	0.01\\
90.11	0.01\\
90.12	0.01\\
90.13	0.01\\
90.14	0.01\\
90.15	0.01\\
90.16	0.01\\
90.17	0.01\\
90.18	0.01\\
90.19	0.01\\
90.2	0.01\\
90.21	0.01\\
90.22	0.01\\
90.23	0.01\\
90.24	0.01\\
90.25	0.01\\
90.26	0.01\\
90.27	0.01\\
90.28	0.01\\
90.29	0.01\\
90.3	0.01\\
90.31	0.01\\
90.32	0.01\\
90.33	0.01\\
90.34	0.01\\
90.35	0.01\\
90.36	0.01\\
90.37	0.01\\
90.38	0.01\\
90.39	0.01\\
90.4	0.01\\
90.41	0.01\\
90.42	0.01\\
90.43	0.01\\
90.44	0.01\\
90.45	0.01\\
90.46	0.01\\
90.47	0.01\\
90.48	0.01\\
90.49	0.01\\
90.5	0.01\\
90.51	0.01\\
90.52	0.01\\
90.53	0.01\\
90.54	0.01\\
90.55	0.01\\
90.56	0.01\\
90.57	0.01\\
90.58	0.01\\
90.59	0.01\\
90.6	0.01\\
90.61	0.01\\
90.62	0.01\\
90.63	0.01\\
90.64	0.01\\
90.65	0.01\\
90.66	0.01\\
90.67	0.01\\
90.68	0.01\\
90.69	0.01\\
90.7	0.01\\
90.71	0.01\\
90.72	0.01\\
90.73	0.01\\
90.74	0.01\\
90.75	0.01\\
90.76	0.01\\
90.77	0.01\\
90.78	0.01\\
90.79	0.01\\
90.8	0.01\\
90.81	0.01\\
90.82	0.01\\
90.83	0.01\\
90.84	0.01\\
90.85	0.01\\
90.86	0.01\\
90.87	0.01\\
90.88	0.01\\
90.89	0.01\\
90.9	0.01\\
90.91	0.01\\
90.92	0.01\\
90.93	0.01\\
90.94	0.01\\
90.95	0.01\\
90.96	0.01\\
90.97	0.01\\
90.98	0.01\\
90.99	0.01\\
91	0.01\\
91.01	0.01\\
91.02	0.01\\
91.03	0.01\\
91.04	0.01\\
91.05	0.01\\
91.06	0.01\\
91.07	0.01\\
91.08	0.01\\
91.09	0.01\\
91.1	0.01\\
91.11	0.01\\
91.12	0.01\\
91.13	0.01\\
91.14	0.01\\
91.15	0.01\\
91.16	0.01\\
91.17	0.01\\
91.18	0.01\\
91.19	0.01\\
91.2	0.01\\
91.21	0.01\\
91.22	0.01\\
91.23	0.01\\
91.24	0.01\\
91.25	0.01\\
91.26	0.01\\
91.27	0.01\\
91.28	0.01\\
91.29	0.01\\
91.3	0.01\\
91.31	0.01\\
91.32	0.01\\
91.33	0.01\\
91.34	0.01\\
91.35	0.01\\
91.36	0.01\\
91.37	0.01\\
91.38	0.01\\
91.39	0.01\\
91.4	0.01\\
91.41	0.01\\
91.42	0.01\\
91.43	0.01\\
91.44	0.01\\
91.45	0.01\\
91.46	0.01\\
91.47	0.01\\
91.48	0.01\\
91.49	0.01\\
91.5	0.01\\
91.51	0.01\\
91.52	0.01\\
91.53	0.01\\
91.54	0.01\\
91.55	0.01\\
91.56	0.01\\
91.57	0.01\\
91.58	0.01\\
91.59	0.01\\
91.6	0.01\\
91.61	0.01\\
91.62	0.01\\
91.63	0.01\\
91.64	0.01\\
91.65	0.01\\
91.66	0.01\\
91.67	0.01\\
91.68	0.01\\
91.69	0.01\\
91.7	0.01\\
91.71	0.01\\
91.72	0.01\\
91.73	0.01\\
91.74	0.01\\
91.75	0.01\\
91.76	0.01\\
91.77	0.01\\
91.78	0.01\\
91.79	0.01\\
91.8	0.01\\
91.81	0.01\\
91.82	0.01\\
91.83	0.01\\
91.84	0.01\\
91.85	0.01\\
91.86	0.01\\
91.87	0.01\\
91.88	0.01\\
91.89	0.01\\
91.9	0.01\\
91.91	0.01\\
91.92	0.01\\
91.93	0.01\\
91.94	0.01\\
91.95	0.01\\
91.96	0.01\\
91.97	0.01\\
91.98	0.01\\
91.99	0.01\\
92	0.01\\
92.01	0.01\\
92.02	0.01\\
92.03	0.01\\
92.04	0.01\\
92.05	0.01\\
92.06	0.01\\
92.07	0.01\\
92.08	0.01\\
92.09	0.01\\
92.1	0.01\\
92.11	0.01\\
92.12	0.01\\
92.13	0.01\\
92.14	0.01\\
92.15	0.01\\
92.16	0.01\\
92.17	0.01\\
92.18	0.01\\
92.19	0.01\\
92.2	0.01\\
92.21	0.01\\
92.22	0.01\\
92.23	0.01\\
92.24	0.01\\
92.25	0.01\\
92.26	0.01\\
92.27	0.01\\
92.28	0.01\\
92.29	0.01\\
92.3	0.01\\
92.31	0.01\\
92.32	0.01\\
92.33	0.01\\
92.34	0.01\\
92.35	0.01\\
92.36	0.01\\
92.37	0.01\\
92.38	0.01\\
92.39	0.01\\
92.4	0.01\\
92.41	0.01\\
92.42	0.01\\
92.43	0.01\\
92.44	0.01\\
92.45	0.01\\
92.46	0.01\\
92.47	0.01\\
92.48	0.01\\
92.49	0.01\\
92.5	0.01\\
92.51	0.01\\
92.52	0.01\\
92.53	0.01\\
92.54	0.01\\
92.55	0.01\\
92.56	0.01\\
92.57	0.01\\
92.58	0.01\\
92.59	0.01\\
92.6	0.01\\
92.61	0.01\\
92.62	0.01\\
92.63	0.01\\
92.64	0.01\\
92.65	0.01\\
92.66	0.01\\
92.67	0.01\\
92.68	0.01\\
92.69	0.01\\
92.7	0.01\\
92.71	0.01\\
92.72	0.01\\
92.73	0.01\\
92.74	0.01\\
92.75	0.01\\
92.76	0.01\\
92.77	0.01\\
92.78	0.01\\
92.79	0.01\\
92.8	0.01\\
92.81	0.01\\
92.82	0.01\\
92.83	0.01\\
92.84	0.01\\
92.85	0.01\\
92.86	0.01\\
92.87	0.01\\
92.88	0.01\\
92.89	0.01\\
92.9	0.01\\
92.91	0.01\\
92.92	0.01\\
92.93	0.01\\
92.94	0.01\\
92.95	0.01\\
92.96	0.01\\
92.97	0.01\\
92.98	0.01\\
92.99	0.01\\
93	0.01\\
93.01	0.01\\
93.02	0.01\\
93.03	0.01\\
93.04	0.01\\
93.05	0.01\\
93.06	0.01\\
93.07	0.01\\
93.08	0.01\\
93.09	0.01\\
93.1	0.01\\
93.11	0.01\\
93.12	0.01\\
93.13	0.01\\
93.14	0.01\\
93.15	0.01\\
93.16	0.01\\
93.17	0.01\\
93.18	0.01\\
93.19	0.01\\
93.2	0.01\\
93.21	0.01\\
93.22	0.01\\
93.23	0.01\\
93.24	0.01\\
93.25	0.01\\
93.26	0.01\\
93.27	0.01\\
93.28	0.01\\
93.29	0.01\\
93.3	0.01\\
93.31	0.01\\
93.32	0.01\\
93.33	0.01\\
93.34	0.01\\
93.35	0.01\\
93.36	0.01\\
93.37	0.01\\
93.38	0.01\\
93.39	0.01\\
93.4	0.01\\
93.41	0.01\\
93.42	0.01\\
93.43	0.01\\
93.44	0.01\\
93.45	0.01\\
93.46	0.01\\
93.47	0.01\\
93.48	0.01\\
93.49	0.01\\
93.5	0.01\\
93.51	0.01\\
93.52	0.01\\
93.53	0.01\\
93.54	0.01\\
93.55	0.01\\
93.56	0.01\\
93.57	0.01\\
93.58	0.01\\
93.59	0.01\\
93.6	0.01\\
93.61	0.01\\
93.62	0.01\\
93.63	0.01\\
93.64	0.01\\
93.65	0.01\\
93.66	0.01\\
93.67	0.01\\
93.68	0.01\\
93.69	0.01\\
93.7	0.01\\
93.71	0.01\\
93.72	0.01\\
93.73	0.01\\
93.74	0.01\\
93.75	0.01\\
93.76	0.01\\
93.77	0.01\\
93.78	0.01\\
93.79	0.01\\
93.8	0.01\\
93.81	0.01\\
93.82	0.01\\
93.83	0.01\\
93.84	0.01\\
93.85	0.01\\
93.86	0.01\\
93.87	0.01\\
93.88	0.01\\
93.89	0.01\\
93.9	0.01\\
93.91	0.01\\
93.92	0.01\\
93.93	0.01\\
93.94	0.01\\
93.95	0.01\\
93.96	0.01\\
93.97	0.01\\
93.98	0.01\\
93.99	0.01\\
94	0.01\\
94.01	0.01\\
94.02	0.01\\
94.03	0.01\\
94.04	0.01\\
94.05	0.01\\
94.06	0.01\\
94.07	0.01\\
94.08	0.01\\
94.09	0.01\\
94.1	0.01\\
94.11	0.01\\
94.12	0.01\\
94.13	0.01\\
94.14	0.01\\
94.15	0.01\\
94.16	0.01\\
94.17	0.01\\
94.18	0.01\\
94.19	0.01\\
94.2	0.01\\
94.21	0.01\\
94.22	0.01\\
94.23	0.01\\
94.24	0.01\\
94.25	0.01\\
94.26	0.01\\
94.27	0.01\\
94.28	0.01\\
94.29	0.01\\
94.3	0.01\\
94.31	0.01\\
94.32	0.01\\
94.33	0.01\\
94.34	0.01\\
94.35	0.01\\
94.36	0.01\\
94.37	0.01\\
94.38	0.01\\
94.39	0.01\\
94.4	0.01\\
94.41	0.01\\
94.42	0.01\\
94.43	0.01\\
94.44	0.01\\
94.45	0.01\\
94.46	0.01\\
94.47	0.01\\
94.48	0.01\\
94.49	0.01\\
94.5	0.01\\
94.51	0.01\\
94.52	0.01\\
94.53	0.01\\
94.54	0.01\\
94.55	0.01\\
94.56	0.01\\
94.57	0.01\\
94.58	0.01\\
94.59	0.01\\
94.6	0.01\\
94.61	0.01\\
94.62	0.01\\
94.63	0.01\\
94.64	0.01\\
94.65	0.01\\
94.66	0.01\\
94.67	0.01\\
94.68	0.01\\
94.69	0.01\\
94.7	0.01\\
94.71	0.01\\
94.72	0.01\\
94.73	0.01\\
94.74	0.01\\
94.75	0.01\\
94.76	0.01\\
94.77	0.01\\
94.78	0.01\\
94.79	0.01\\
94.8	0.01\\
94.81	0.01\\
94.82	0.01\\
94.83	0.01\\
94.84	0.01\\
94.85	0.01\\
94.86	0.01\\
94.87	0.01\\
94.88	0.01\\
94.89	0.01\\
94.9	0.01\\
94.91	0.01\\
94.92	0.01\\
94.93	0.01\\
94.94	0.01\\
94.95	0.01\\
94.96	0.01\\
94.97	0.01\\
94.98	0.01\\
94.99	0.01\\
95	0.01\\
95.01	0.01\\
95.02	0.01\\
95.03	0.01\\
95.04	0.01\\
95.05	0.01\\
95.06	0.01\\
95.07	0.01\\
95.08	0.01\\
95.09	0.01\\
95.1	0.01\\
95.11	0.01\\
95.12	0.01\\
95.13	0.01\\
95.14	0.01\\
95.15	0.01\\
95.16	0.01\\
95.17	0.01\\
95.18	0.01\\
95.19	0.01\\
95.2	0.01\\
95.21	0.01\\
95.22	0.01\\
95.23	0.01\\
95.24	0.01\\
95.25	0.01\\
95.26	0.01\\
95.27	0.01\\
95.28	0.01\\
95.29	0.01\\
95.3	0.01\\
95.31	0.01\\
95.32	0.01\\
95.33	0.01\\
95.34	0.01\\
95.35	0.01\\
95.36	0.01\\
95.37	0.01\\
95.38	0.01\\
95.39	0.01\\
95.4	0.01\\
95.41	0.01\\
95.42	0.01\\
95.43	0.01\\
95.44	0.01\\
95.45	0.01\\
95.46	0.01\\
95.47	0.01\\
95.48	0.01\\
95.49	0.01\\
95.5	0.01\\
95.51	0.01\\
95.52	0.01\\
95.53	0.01\\
95.54	0.01\\
95.55	0.01\\
95.56	0.01\\
95.57	0.01\\
95.58	0.01\\
95.59	0.01\\
95.6	0.01\\
95.61	0.01\\
95.62	0.01\\
95.63	0.01\\
95.64	0.01\\
95.65	0.01\\
95.66	0.01\\
95.67	0.01\\
95.68	0.01\\
95.69	0.01\\
95.7	0.01\\
95.71	0.01\\
95.72	0.01\\
95.73	0.01\\
95.74	0.01\\
95.75	0.01\\
95.76	0.01\\
95.77	0.01\\
95.78	0.01\\
95.79	0.01\\
95.8	0.01\\
95.81	0.01\\
95.82	0.01\\
95.83	0.01\\
95.84	0.01\\
95.85	0.01\\
95.86	0.01\\
95.87	0.01\\
95.88	0.01\\
95.89	0.01\\
95.9	0.01\\
95.91	0.01\\
95.92	0.01\\
95.93	0.01\\
95.94	0.01\\
95.95	0.01\\
95.96	0.01\\
95.97	0.01\\
95.98	0.01\\
95.99	0.01\\
96	0.01\\
96.01	0.01\\
96.02	0.01\\
96.03	0.01\\
96.04	0.01\\
96.05	0.01\\
96.06	0.01\\
96.07	0.01\\
96.08	0.01\\
96.09	0.01\\
96.1	0.01\\
96.11	0.01\\
96.12	0.01\\
96.13	0.01\\
96.14	0.01\\
96.15	0.01\\
96.16	0.01\\
96.17	0.01\\
96.18	0.01\\
96.19	0.01\\
96.2	0.01\\
96.21	0.01\\
96.22	0.01\\
96.23	0.01\\
96.24	0.01\\
96.25	0.01\\
96.26	0.01\\
96.27	0.01\\
96.28	0.01\\
96.29	0.01\\
96.3	0.01\\
96.31	0.01\\
96.32	0.01\\
96.33	0.01\\
96.34	0.01\\
96.35	0.01\\
96.36	0.01\\
96.37	0.01\\
96.38	0.01\\
96.39	0.01\\
96.4	0.01\\
96.41	0.01\\
96.42	0.01\\
96.43	0.01\\
96.44	0.01\\
96.45	0.01\\
96.46	0.01\\
96.47	0.01\\
96.48	0.01\\
96.49	0.01\\
96.5	0.01\\
96.51	0.01\\
96.52	0.01\\
96.53	0.01\\
96.54	0.01\\
96.55	0.01\\
96.56	0.01\\
96.57	0.01\\
96.58	0.01\\
96.59	0.01\\
96.6	0.01\\
96.61	0.01\\
96.62	0.01\\
96.63	0.01\\
96.64	0.01\\
96.65	0.01\\
96.66	0.01\\
96.67	0.01\\
96.68	0.01\\
96.69	0.01\\
96.7	0.01\\
96.71	0.01\\
96.72	0.01\\
96.73	0.01\\
96.74	0.01\\
96.75	0.01\\
96.76	0.01\\
96.77	0.01\\
96.78	0.01\\
96.79	0.01\\
96.8	0.01\\
96.81	0.01\\
96.82	0.01\\
96.83	0.01\\
96.84	0.01\\
96.85	0.01\\
96.86	0.01\\
96.87	0.01\\
96.88	0.01\\
96.89	0.01\\
96.9	0.01\\
96.91	0.01\\
96.92	0.01\\
96.93	0.01\\
96.94	0.01\\
96.95	0.01\\
96.96	0.01\\
96.97	0.01\\
96.98	0.01\\
96.99	0.01\\
97	0.01\\
97.01	0.01\\
97.02	0.01\\
97.03	0.01\\
97.04	0.01\\
97.05	0.01\\
97.06	0.01\\
97.07	0.01\\
97.08	0.01\\
97.09	0.01\\
97.1	0.01\\
97.11	0.01\\
97.12	0.01\\
97.13	0.01\\
97.14	0.01\\
97.15	0.01\\
97.16	0.01\\
97.17	0.01\\
97.18	0.01\\
97.19	0.01\\
97.2	0.01\\
97.21	0.01\\
97.22	0.01\\
97.23	0.01\\
97.24	0.01\\
97.25	0.01\\
97.26	0.01\\
97.27	0.01\\
97.28	0.01\\
97.29	0.01\\
97.3	0.01\\
97.31	0.01\\
97.32	0.01\\
97.33	0.01\\
97.34	0.01\\
97.35	0.01\\
97.36	0.01\\
97.37	0.01\\
97.38	0.01\\
97.39	0.01\\
97.4	0.01\\
97.41	0.01\\
97.42	0.01\\
97.43	0.01\\
97.44	0.01\\
97.45	0.01\\
97.46	0.01\\
97.47	0.01\\
97.48	0.01\\
97.49	0.01\\
97.5	0.01\\
97.51	0.01\\
97.52	0.01\\
97.53	0.01\\
97.54	0.01\\
97.55	0.01\\
97.56	0.01\\
97.57	0.01\\
97.58	0.01\\
97.59	0.01\\
97.6	0.01\\
97.61	0.01\\
97.62	0.01\\
97.63	0.01\\
97.64	0.01\\
97.65	0.01\\
97.66	0.01\\
97.67	0.01\\
97.68	0.01\\
97.69	0.01\\
97.7	0.01\\
97.71	0.01\\
97.72	0.01\\
97.73	0.01\\
97.74	0.01\\
97.75	0.01\\
97.76	0.01\\
97.77	0.01\\
97.78	0.01\\
97.79	0.01\\
97.8	0.01\\
97.81	0.01\\
97.82	0.01\\
97.83	0.01\\
97.84	0.01\\
97.85	0.01\\
97.86	0.01\\
97.87	0.01\\
97.88	0.01\\
97.89	0.01\\
97.9	0.01\\
97.91	0.01\\
97.92	0.01\\
97.93	0.01\\
97.94	0.01\\
97.95	0.01\\
97.96	0.01\\
97.97	0.01\\
97.98	0.01\\
97.99	0.01\\
98	0.01\\
98.01	0.01\\
98.02	0.01\\
98.03	0.01\\
98.04	0.01\\
98.05	0.01\\
98.06	0.01\\
98.07	0.01\\
98.08	0.01\\
98.09	0.01\\
98.1	0.01\\
98.11	0.01\\
98.12	0.01\\
98.13	0.01\\
98.14	0.01\\
98.15	0.01\\
98.16	0.01\\
98.17	0.01\\
98.18	0.01\\
98.19	0.01\\
98.2	0.01\\
98.21	0.01\\
98.22	0.01\\
98.23	0.01\\
98.24	0.01\\
98.25	0.01\\
98.26	0.01\\
98.27	0.01\\
98.28	0.01\\
98.29	0.01\\
98.3	0.01\\
98.31	0.01\\
98.32	0.01\\
98.33	0.01\\
98.34	0.01\\
98.35	0.01\\
98.36	0.01\\
98.37	0.01\\
98.38	0.01\\
98.39	0.01\\
98.4	0.01\\
98.41	0.01\\
98.42	0.01\\
98.43	0.01\\
98.44	0.01\\
98.45	0.01\\
98.46	0.01\\
98.47	0.01\\
98.48	0.01\\
98.49	0.01\\
98.5	0.01\\
98.51	0.01\\
98.52	0.01\\
98.53	0.01\\
98.54	0.01\\
98.55	0.01\\
98.56	0.01\\
98.57	0.01\\
98.58	0.01\\
98.59	0.01\\
98.6	0.01\\
98.61	0.01\\
98.62	0.01\\
98.63	0.01\\
98.64	0.01\\
98.65	0.01\\
98.66	0.01\\
98.67	0.01\\
98.68	0.01\\
98.69	0.01\\
98.7	0.01\\
98.71	0.01\\
98.72	0.01\\
98.73	0.01\\
98.74	0.01\\
98.75	0.01\\
98.76	0.01\\
98.77	0.01\\
98.78	0.01\\
98.79	0.01\\
98.8	0.01\\
98.81	0.01\\
98.82	0.01\\
98.83	0.01\\
98.84	0.01\\
98.85	0.01\\
98.86	0.01\\
98.87	0.01\\
98.88	0.01\\
98.89	0.01\\
98.9	0.01\\
98.91	0.01\\
98.92	0.01\\
98.93	0.01\\
98.94	0.01\\
98.95	0.01\\
98.96	0.00994657662760442\\
98.97	0.00975022967615308\\
98.98	0.00955247302775352\\
98.99	0.00935328647093454\\
99	0.00915264917735695\\
99.01	0.00895053971824736\\
99.02	0.00874693604588305\\
99.03	0.00854181547068092\\
99.04	0.0083351546369417\\
99.05	0.00812698671253183\\
99.06	0.00791728924939752\\
99.07	0.00770603839157192\\
99.08	0.00749320955439558\\
99.09	0.00727877739613037\\
99.1	0.007062715780018\\
99.11	0.0068449977483514\\
99.12	0.00662559549257636\\
99.13	0.00640448031894984\\
99.14	0.00619506749001899\\
99.15	0.00614343065536976\\
99.16	0.00609144762524035\\
99.17	0.00603912153677965\\
99.18	0.00598645582119131\\
99.19	0.00593345421943499\\
99.2	0.00588012078333322\\
99.21	0.0058264598893813\\
99.22	0.00577246463597955\\
99.23	0.00571809972977372\\
99.24	0.00566336885383304\\
99.25	0.0056082760216254\\
99.26	0.00555282559207382\\
99.27	0.00549702228530884\\
99.28	0.0054408711991545\\
99.29	0.00538437782638891\\
99.3	0.00532754807282254\\
99.31	0.00527038827624041\\
99.32	0.0052129052262574\\
99.33	0.0051551061851392\\
99.34	0.00509699890964516\\
99.35	0.00503859167365097\\
99.36	0.00497989329218159\\
99.37	0.00492091315045911\\
99.38	0.0048616612557091\\
99.39	0.00480214823897229\\
99.4	0.00474238538462658\\
99.41	0.00468238466146454\\
99.42	0.00462184456317968\\
99.43	0.00456075958847703\\
99.44	0.00449912481928159\\
99.45	0.00443693528457883\\
99.46	0.00437418595912791\\
99.47	0.00431087176214988\\
99.48	0.00424698765185965\\
99.49	0.00418252865380583\\
99.5	0.00411748975146728\\
99.51	0.00405186588565974\\
99.52	0.00398565195391432\\
99.53	0.00391884280982608\\
99.54	0.00385143326237089\\
99.55	0.00378341807518831\\
99.56	0.00371479196582851\\
99.57	0.00364554960496075\\
99.58	0.00357568561554109\\
99.59	0.00350519457193654\\
99.6	0.00343407099900291\\
99.61	0.00336230937111332\\
99.62	0.00328990411113425\\
99.63	0.00321684958934553\\
99.64	0.00314314012230076\\
99.65	0.00306876997162424\\
99.66	0.00299373334274013\\
99.67	0.00291802438352953\\
99.68	0.00284163718291057\\
99.69	0.00276456578468188\\
99.7	0.00268680417387409\\
99.71	0.00260834627216765\\
99.72	0.00252918593610012\\
99.73	0.00244931695514438\\
99.74	0.00236873304971111\\
99.75	0.00228742786903057\\
99.76	0.00220539498890514\\
99.77	0.00212262790932414\\
99.78	0.00203912005193141\\
99.79	0.0019548647573356\\
99.8	0.00186985528225254\\
99.81	0.00178408479646787\\
99.82	0.00169754637960787\\
99.83	0.00161023301770489\\
99.84	0.00152213759954336\\
99.85	0.001433252912771\\
99.86	0.00134357163975866\\
99.87	0.00125308635319149\\
99.88	0.0011617895113721\\
99.89	0.00106967345321565\\
99.9	0.000976730392914901\\
99.91	0.000882952414253254\\
99.92	0.000788331464541591\\
99.93	0.000692859348149801\\
99.94	0.000596527719603999\\
99.95	0.000499328076217975\\
99.96	0.000401251750225215\\
99.97	0.000302289900375147\\
99.98	0.000202433502954543\\
99.99	0.000101673342191978\\
100	0\\
};
\addlegendentry{$q=-3$};

\addplot [color=red,dashed,forget plot]
  table[row sep=crcr]{%
0.01	0.01\\
0.02	0.01\\
0.03	0.01\\
0.04	0.01\\
0.05	0.01\\
0.06	0.01\\
0.07	0.01\\
0.08	0.01\\
0.09	0.01\\
0.1	0.01\\
0.11	0.01\\
0.12	0.01\\
0.13	0.01\\
0.14	0.01\\
0.15	0.01\\
0.16	0.01\\
0.17	0.01\\
0.18	0.01\\
0.19	0.01\\
0.2	0.01\\
0.21	0.01\\
0.22	0.01\\
0.23	0.01\\
0.24	0.01\\
0.25	0.01\\
0.26	0.01\\
0.27	0.01\\
0.28	0.01\\
0.29	0.01\\
0.3	0.01\\
0.31	0.01\\
0.32	0.01\\
0.33	0.01\\
0.34	0.01\\
0.35	0.01\\
0.36	0.01\\
0.37	0.01\\
0.38	0.01\\
0.39	0.01\\
0.4	0.01\\
0.41	0.01\\
0.42	0.01\\
0.43	0.01\\
0.44	0.01\\
0.45	0.01\\
0.46	0.01\\
0.47	0.01\\
0.48	0.01\\
0.49	0.01\\
0.5	0.01\\
0.51	0.01\\
0.52	0.01\\
0.53	0.01\\
0.54	0.01\\
0.55	0.01\\
0.56	0.01\\
0.57	0.01\\
0.58	0.01\\
0.59	0.01\\
0.6	0.01\\
0.61	0.01\\
0.62	0.01\\
0.63	0.01\\
0.64	0.01\\
0.65	0.01\\
0.66	0.01\\
0.67	0.01\\
0.68	0.01\\
0.69	0.01\\
0.7	0.01\\
0.71	0.01\\
0.72	0.01\\
0.73	0.01\\
0.74	0.01\\
0.75	0.01\\
0.76	0.01\\
0.77	0.01\\
0.78	0.01\\
0.79	0.01\\
0.8	0.01\\
0.81	0.01\\
0.82	0.01\\
0.83	0.01\\
0.84	0.01\\
0.85	0.01\\
0.86	0.01\\
0.87	0.01\\
0.88	0.01\\
0.89	0.01\\
0.9	0.01\\
0.91	0.01\\
0.92	0.01\\
0.93	0.01\\
0.94	0.01\\
0.95	0.01\\
0.96	0.01\\
0.97	0.01\\
0.98	0.01\\
0.99	0.01\\
1	0.01\\
1.01	0.01\\
1.02	0.01\\
1.03	0.01\\
1.04	0.01\\
1.05	0.01\\
1.06	0.01\\
1.07	0.01\\
1.08	0.01\\
1.09	0.01\\
1.1	0.01\\
1.11	0.01\\
1.12	0.01\\
1.13	0.01\\
1.14	0.01\\
1.15	0.01\\
1.16	0.01\\
1.17	0.01\\
1.18	0.01\\
1.19	0.01\\
1.2	0.01\\
1.21	0.01\\
1.22	0.01\\
1.23	0.01\\
1.24	0.01\\
1.25	0.01\\
1.26	0.01\\
1.27	0.01\\
1.28	0.01\\
1.29	0.01\\
1.3	0.01\\
1.31	0.01\\
1.32	0.01\\
1.33	0.01\\
1.34	0.01\\
1.35	0.01\\
1.36	0.01\\
1.37	0.01\\
1.38	0.01\\
1.39	0.01\\
1.4	0.01\\
1.41	0.01\\
1.42	0.01\\
1.43	0.01\\
1.44	0.01\\
1.45	0.01\\
1.46	0.01\\
1.47	0.01\\
1.48	0.01\\
1.49	0.01\\
1.5	0.01\\
1.51	0.01\\
1.52	0.01\\
1.53	0.01\\
1.54	0.01\\
1.55	0.01\\
1.56	0.01\\
1.57	0.01\\
1.58	0.01\\
1.59	0.01\\
1.6	0.01\\
1.61	0.01\\
1.62	0.01\\
1.63	0.01\\
1.64	0.01\\
1.65	0.01\\
1.66	0.01\\
1.67	0.01\\
1.68	0.01\\
1.69	0.01\\
1.7	0.01\\
1.71	0.01\\
1.72	0.01\\
1.73	0.01\\
1.74	0.01\\
1.75	0.01\\
1.76	0.01\\
1.77	0.01\\
1.78	0.01\\
1.79	0.01\\
1.8	0.01\\
1.81	0.01\\
1.82	0.01\\
1.83	0.01\\
1.84	0.01\\
1.85	0.01\\
1.86	0.01\\
1.87	0.01\\
1.88	0.01\\
1.89	0.01\\
1.9	0.01\\
1.91	0.01\\
1.92	0.01\\
1.93	0.01\\
1.94	0.01\\
1.95	0.01\\
1.96	0.01\\
1.97	0.01\\
1.98	0.01\\
1.99	0.01\\
2	0.01\\
2.01	0.01\\
2.02	0.01\\
2.03	0.01\\
2.04	0.01\\
2.05	0.01\\
2.06	0.01\\
2.07	0.01\\
2.08	0.01\\
2.09	0.01\\
2.1	0.01\\
2.11	0.01\\
2.12	0.01\\
2.13	0.01\\
2.14	0.01\\
2.15	0.01\\
2.16	0.01\\
2.17	0.01\\
2.18	0.01\\
2.19	0.01\\
2.2	0.01\\
2.21	0.01\\
2.22	0.01\\
2.23	0.01\\
2.24	0.01\\
2.25	0.01\\
2.26	0.01\\
2.27	0.01\\
2.28	0.01\\
2.29	0.01\\
2.3	0.01\\
2.31	0.01\\
2.32	0.01\\
2.33	0.01\\
2.34	0.01\\
2.35	0.01\\
2.36	0.01\\
2.37	0.01\\
2.38	0.01\\
2.39	0.01\\
2.4	0.01\\
2.41	0.01\\
2.42	0.01\\
2.43	0.01\\
2.44	0.01\\
2.45	0.01\\
2.46	0.01\\
2.47	0.01\\
2.48	0.01\\
2.49	0.01\\
2.5	0.01\\
2.51	0.01\\
2.52	0.01\\
2.53	0.01\\
2.54	0.01\\
2.55	0.01\\
2.56	0.01\\
2.57	0.01\\
2.58	0.01\\
2.59	0.01\\
2.6	0.01\\
2.61	0.01\\
2.62	0.01\\
2.63	0.01\\
2.64	0.01\\
2.65	0.01\\
2.66	0.01\\
2.67	0.01\\
2.68	0.01\\
2.69	0.01\\
2.7	0.01\\
2.71	0.01\\
2.72	0.01\\
2.73	0.01\\
2.74	0.01\\
2.75	0.01\\
2.76	0.01\\
2.77	0.01\\
2.78	0.01\\
2.79	0.01\\
2.8	0.01\\
2.81	0.01\\
2.82	0.01\\
2.83	0.01\\
2.84	0.01\\
2.85	0.01\\
2.86	0.01\\
2.87	0.01\\
2.88	0.01\\
2.89	0.01\\
2.9	0.01\\
2.91	0.01\\
2.92	0.01\\
2.93	0.01\\
2.94	0.01\\
2.95	0.01\\
2.96	0.01\\
2.97	0.01\\
2.98	0.01\\
2.99	0.01\\
3	0.01\\
3.01	0.01\\
3.02	0.01\\
3.03	0.01\\
3.04	0.01\\
3.05	0.01\\
3.06	0.01\\
3.07	0.01\\
3.08	0.01\\
3.09	0.01\\
3.1	0.01\\
3.11	0.01\\
3.12	0.01\\
3.13	0.01\\
3.14	0.01\\
3.15	0.01\\
3.16	0.01\\
3.17	0.01\\
3.18	0.01\\
3.19	0.01\\
3.2	0.01\\
3.21	0.01\\
3.22	0.01\\
3.23	0.01\\
3.24	0.01\\
3.25	0.01\\
3.26	0.01\\
3.27	0.01\\
3.28	0.01\\
3.29	0.01\\
3.3	0.01\\
3.31	0.01\\
3.32	0.01\\
3.33	0.01\\
3.34	0.01\\
3.35	0.01\\
3.36	0.01\\
3.37	0.01\\
3.38	0.01\\
3.39	0.01\\
3.4	0.01\\
3.41	0.01\\
3.42	0.01\\
3.43	0.01\\
3.44	0.01\\
3.45	0.01\\
3.46	0.01\\
3.47	0.01\\
3.48	0.01\\
3.49	0.01\\
3.5	0.01\\
3.51	0.01\\
3.52	0.01\\
3.53	0.01\\
3.54	0.01\\
3.55	0.01\\
3.56	0.01\\
3.57	0.01\\
3.58	0.01\\
3.59	0.01\\
3.6	0.01\\
3.61	0.01\\
3.62	0.01\\
3.63	0.01\\
3.64	0.01\\
3.65	0.01\\
3.66	0.01\\
3.67	0.01\\
3.68	0.01\\
3.69	0.01\\
3.7	0.01\\
3.71	0.01\\
3.72	0.01\\
3.73	0.01\\
3.74	0.01\\
3.75	0.01\\
3.76	0.01\\
3.77	0.01\\
3.78	0.01\\
3.79	0.01\\
3.8	0.01\\
3.81	0.01\\
3.82	0.01\\
3.83	0.01\\
3.84	0.01\\
3.85	0.01\\
3.86	0.01\\
3.87	0.01\\
3.88	0.01\\
3.89	0.01\\
3.9	0.01\\
3.91	0.01\\
3.92	0.01\\
3.93	0.01\\
3.94	0.01\\
3.95	0.01\\
3.96	0.01\\
3.97	0.01\\
3.98	0.01\\
3.99	0.01\\
4	0.01\\
4.01	0.01\\
4.02	0.01\\
4.03	0.01\\
4.04	0.01\\
4.05	0.01\\
4.06	0.01\\
4.07	0.01\\
4.08	0.01\\
4.09	0.01\\
4.1	0.01\\
4.11	0.01\\
4.12	0.01\\
4.13	0.01\\
4.14	0.01\\
4.15	0.01\\
4.16	0.01\\
4.17	0.01\\
4.18	0.01\\
4.19	0.01\\
4.2	0.01\\
4.21	0.01\\
4.22	0.01\\
4.23	0.01\\
4.24	0.01\\
4.25	0.01\\
4.26	0.01\\
4.27	0.01\\
4.28	0.01\\
4.29	0.01\\
4.3	0.01\\
4.31	0.01\\
4.32	0.01\\
4.33	0.01\\
4.34	0.01\\
4.35	0.01\\
4.36	0.01\\
4.37	0.01\\
4.38	0.01\\
4.39	0.01\\
4.4	0.01\\
4.41	0.01\\
4.42	0.01\\
4.43	0.01\\
4.44	0.01\\
4.45	0.01\\
4.46	0.01\\
4.47	0.01\\
4.48	0.01\\
4.49	0.01\\
4.5	0.01\\
4.51	0.01\\
4.52	0.01\\
4.53	0.01\\
4.54	0.01\\
4.55	0.01\\
4.56	0.01\\
4.57	0.01\\
4.58	0.01\\
4.59	0.01\\
4.6	0.01\\
4.61	0.01\\
4.62	0.01\\
4.63	0.01\\
4.64	0.01\\
4.65	0.01\\
4.66	0.01\\
4.67	0.01\\
4.68	0.01\\
4.69	0.01\\
4.7	0.01\\
4.71	0.01\\
4.72	0.01\\
4.73	0.01\\
4.74	0.01\\
4.75	0.01\\
4.76	0.01\\
4.77	0.01\\
4.78	0.01\\
4.79	0.01\\
4.8	0.01\\
4.81	0.01\\
4.82	0.01\\
4.83	0.01\\
4.84	0.01\\
4.85	0.01\\
4.86	0.01\\
4.87	0.01\\
4.88	0.01\\
4.89	0.01\\
4.9	0.01\\
4.91	0.01\\
4.92	0.01\\
4.93	0.01\\
4.94	0.01\\
4.95	0.01\\
4.96	0.01\\
4.97	0.01\\
4.98	0.01\\
4.99	0.01\\
5	0.01\\
5.01	0.01\\
5.02	0.01\\
5.03	0.01\\
5.04	0.01\\
5.05	0.01\\
5.06	0.01\\
5.07	0.01\\
5.08	0.01\\
5.09	0.01\\
5.1	0.01\\
5.11	0.01\\
5.12	0.01\\
5.13	0.01\\
5.14	0.01\\
5.15	0.01\\
5.16	0.01\\
5.17	0.01\\
5.18	0.01\\
5.19	0.01\\
5.2	0.01\\
5.21	0.01\\
5.22	0.01\\
5.23	0.01\\
5.24	0.01\\
5.25	0.01\\
5.26	0.01\\
5.27	0.01\\
5.28	0.01\\
5.29	0.01\\
5.3	0.01\\
5.31	0.01\\
5.32	0.01\\
5.33	0.01\\
5.34	0.01\\
5.35	0.01\\
5.36	0.01\\
5.37	0.01\\
5.38	0.01\\
5.39	0.01\\
5.4	0.01\\
5.41	0.01\\
5.42	0.01\\
5.43	0.01\\
5.44	0.01\\
5.45	0.01\\
5.46	0.01\\
5.47	0.01\\
5.48	0.01\\
5.49	0.01\\
5.5	0.01\\
5.51	0.01\\
5.52	0.01\\
5.53	0.01\\
5.54	0.01\\
5.55	0.01\\
5.56	0.01\\
5.57	0.01\\
5.58	0.01\\
5.59	0.01\\
5.6	0.01\\
5.61	0.01\\
5.62	0.01\\
5.63	0.01\\
5.64	0.01\\
5.65	0.01\\
5.66	0.01\\
5.67	0.01\\
5.68	0.01\\
5.69	0.01\\
5.7	0.01\\
5.71	0.01\\
5.72	0.01\\
5.73	0.01\\
5.74	0.01\\
5.75	0.01\\
5.76	0.01\\
5.77	0.01\\
5.78	0.01\\
5.79	0.01\\
5.8	0.01\\
5.81	0.01\\
5.82	0.01\\
5.83	0.01\\
5.84	0.01\\
5.85	0.01\\
5.86	0.01\\
5.87	0.01\\
5.88	0.01\\
5.89	0.01\\
5.9	0.01\\
5.91	0.01\\
5.92	0.01\\
5.93	0.01\\
5.94	0.01\\
5.95	0.01\\
5.96	0.01\\
5.97	0.01\\
5.98	0.01\\
5.99	0.01\\
6	0.01\\
6.01	0.01\\
6.02	0.01\\
6.03	0.01\\
6.04	0.01\\
6.05	0.01\\
6.06	0.01\\
6.07	0.01\\
6.08	0.01\\
6.09	0.01\\
6.1	0.01\\
6.11	0.01\\
6.12	0.01\\
6.13	0.01\\
6.14	0.01\\
6.15	0.01\\
6.16	0.01\\
6.17	0.01\\
6.18	0.01\\
6.19	0.01\\
6.2	0.01\\
6.21	0.01\\
6.22	0.01\\
6.23	0.01\\
6.24	0.01\\
6.25	0.01\\
6.26	0.01\\
6.27	0.01\\
6.28	0.01\\
6.29	0.01\\
6.3	0.01\\
6.31	0.01\\
6.32	0.01\\
6.33	0.01\\
6.34	0.01\\
6.35	0.01\\
6.36	0.01\\
6.37	0.01\\
6.38	0.01\\
6.39	0.01\\
6.4	0.01\\
6.41	0.01\\
6.42	0.01\\
6.43	0.01\\
6.44	0.01\\
6.45	0.01\\
6.46	0.01\\
6.47	0.01\\
6.48	0.01\\
6.49	0.01\\
6.5	0.01\\
6.51	0.01\\
6.52	0.01\\
6.53	0.01\\
6.54	0.01\\
6.55	0.01\\
6.56	0.01\\
6.57	0.01\\
6.58	0.01\\
6.59	0.01\\
6.6	0.01\\
6.61	0.01\\
6.62	0.01\\
6.63	0.01\\
6.64	0.01\\
6.65	0.01\\
6.66	0.01\\
6.67	0.01\\
6.68	0.01\\
6.69	0.01\\
6.7	0.01\\
6.71	0.01\\
6.72	0.01\\
6.73	0.01\\
6.74	0.01\\
6.75	0.01\\
6.76	0.01\\
6.77	0.01\\
6.78	0.01\\
6.79	0.01\\
6.8	0.01\\
6.81	0.01\\
6.82	0.01\\
6.83	0.01\\
6.84	0.01\\
6.85	0.01\\
6.86	0.01\\
6.87	0.01\\
6.88	0.01\\
6.89	0.01\\
6.9	0.01\\
6.91	0.01\\
6.92	0.01\\
6.93	0.01\\
6.94	0.01\\
6.95	0.01\\
6.96	0.01\\
6.97	0.01\\
6.98	0.01\\
6.99	0.01\\
7	0.01\\
7.01	0.01\\
7.02	0.01\\
7.03	0.01\\
7.04	0.01\\
7.05	0.01\\
7.06	0.01\\
7.07	0.01\\
7.08	0.01\\
7.09	0.01\\
7.1	0.01\\
7.11	0.01\\
7.12	0.01\\
7.13	0.01\\
7.14	0.01\\
7.15	0.01\\
7.16	0.01\\
7.17	0.01\\
7.18	0.01\\
7.19	0.01\\
7.2	0.01\\
7.21	0.01\\
7.22	0.01\\
7.23	0.01\\
7.24	0.01\\
7.25	0.01\\
7.26	0.01\\
7.27	0.01\\
7.28	0.01\\
7.29	0.01\\
7.3	0.01\\
7.31	0.01\\
7.32	0.01\\
7.33	0.01\\
7.34	0.01\\
7.35	0.01\\
7.36	0.01\\
7.37	0.01\\
7.38	0.01\\
7.39	0.01\\
7.4	0.01\\
7.41	0.01\\
7.42	0.01\\
7.43	0.01\\
7.44	0.01\\
7.45	0.01\\
7.46	0.01\\
7.47	0.01\\
7.48	0.01\\
7.49	0.01\\
7.5	0.01\\
7.51	0.01\\
7.52	0.01\\
7.53	0.01\\
7.54	0.01\\
7.55	0.01\\
7.56	0.01\\
7.57	0.01\\
7.58	0.01\\
7.59	0.01\\
7.6	0.01\\
7.61	0.01\\
7.62	0.01\\
7.63	0.01\\
7.64	0.01\\
7.65	0.01\\
7.66	0.01\\
7.67	0.01\\
7.68	0.01\\
7.69	0.01\\
7.7	0.01\\
7.71	0.01\\
7.72	0.01\\
7.73	0.01\\
7.74	0.01\\
7.75	0.01\\
7.76	0.01\\
7.77	0.01\\
7.78	0.01\\
7.79	0.01\\
7.8	0.01\\
7.81	0.01\\
7.82	0.01\\
7.83	0.01\\
7.84	0.01\\
7.85	0.01\\
7.86	0.01\\
7.87	0.01\\
7.88	0.01\\
7.89	0.01\\
7.9	0.01\\
7.91	0.01\\
7.92	0.01\\
7.93	0.01\\
7.94	0.01\\
7.95	0.01\\
7.96	0.01\\
7.97	0.01\\
7.98	0.01\\
7.99	0.01\\
8	0.01\\
8.01	0.01\\
8.02	0.01\\
8.03	0.01\\
8.04	0.01\\
8.05	0.01\\
8.06	0.01\\
8.07	0.01\\
8.08	0.01\\
8.09	0.01\\
8.1	0.01\\
8.11	0.01\\
8.12	0.01\\
8.13	0.01\\
8.14	0.01\\
8.15	0.01\\
8.16	0.01\\
8.17	0.01\\
8.18	0.01\\
8.19	0.01\\
8.2	0.01\\
8.21	0.01\\
8.22	0.01\\
8.23	0.01\\
8.24	0.01\\
8.25	0.01\\
8.26	0.01\\
8.27	0.01\\
8.28	0.01\\
8.29	0.01\\
8.3	0.01\\
8.31	0.01\\
8.32	0.01\\
8.33	0.01\\
8.34	0.01\\
8.35	0.01\\
8.36	0.01\\
8.37	0.01\\
8.38	0.01\\
8.39	0.01\\
8.4	0.01\\
8.41	0.01\\
8.42	0.01\\
8.43	0.01\\
8.44	0.01\\
8.45	0.01\\
8.46	0.01\\
8.47	0.01\\
8.48	0.01\\
8.49	0.01\\
8.5	0.01\\
8.51	0.01\\
8.52	0.01\\
8.53	0.01\\
8.54	0.01\\
8.55	0.01\\
8.56	0.01\\
8.57	0.01\\
8.58	0.01\\
8.59	0.01\\
8.6	0.01\\
8.61	0.01\\
8.62	0.01\\
8.63	0.01\\
8.64	0.01\\
8.65	0.01\\
8.66	0.01\\
8.67	0.01\\
8.68	0.01\\
8.69	0.01\\
8.7	0.01\\
8.71	0.01\\
8.72	0.01\\
8.73	0.01\\
8.74	0.01\\
8.75	0.01\\
8.76	0.01\\
8.77	0.01\\
8.78	0.01\\
8.79	0.01\\
8.8	0.01\\
8.81	0.01\\
8.82	0.01\\
8.83	0.01\\
8.84	0.01\\
8.85	0.01\\
8.86	0.01\\
8.87	0.01\\
8.88	0.01\\
8.89	0.01\\
8.9	0.01\\
8.91	0.01\\
8.92	0.01\\
8.93	0.01\\
8.94	0.01\\
8.95	0.01\\
8.96	0.01\\
8.97	0.01\\
8.98	0.01\\
8.99	0.01\\
9	0.01\\
9.01	0.01\\
9.02	0.01\\
9.03	0.01\\
9.04	0.01\\
9.05	0.01\\
9.06	0.01\\
9.07	0.01\\
9.08	0.01\\
9.09	0.01\\
9.1	0.01\\
9.11	0.01\\
9.12	0.01\\
9.13	0.01\\
9.14	0.01\\
9.15	0.01\\
9.16	0.01\\
9.17	0.01\\
9.18	0.01\\
9.19	0.01\\
9.2	0.01\\
9.21	0.01\\
9.22	0.01\\
9.23	0.01\\
9.24	0.01\\
9.25	0.01\\
9.26	0.01\\
9.27	0.01\\
9.28	0.01\\
9.29	0.01\\
9.3	0.01\\
9.31	0.01\\
9.32	0.01\\
9.33	0.01\\
9.34	0.01\\
9.35	0.01\\
9.36	0.01\\
9.37	0.01\\
9.38	0.01\\
9.39	0.01\\
9.4	0.01\\
9.41	0.01\\
9.42	0.01\\
9.43	0.01\\
9.44	0.01\\
9.45	0.01\\
9.46	0.01\\
9.47	0.01\\
9.48	0.01\\
9.49	0.01\\
9.5	0.01\\
9.51	0.01\\
9.52	0.01\\
9.53	0.01\\
9.54	0.01\\
9.55	0.01\\
9.56	0.01\\
9.57	0.01\\
9.58	0.01\\
9.59	0.01\\
9.6	0.01\\
9.61	0.01\\
9.62	0.01\\
9.63	0.01\\
9.64	0.01\\
9.65	0.01\\
9.66	0.01\\
9.67	0.01\\
9.68	0.01\\
9.69	0.01\\
9.7	0.01\\
9.71	0.01\\
9.72	0.01\\
9.73	0.01\\
9.74	0.01\\
9.75	0.01\\
9.76	0.01\\
9.77	0.01\\
9.78	0.01\\
9.79	0.01\\
9.8	0.01\\
9.81	0.01\\
9.82	0.01\\
9.83	0.01\\
9.84	0.01\\
9.85	0.01\\
9.86	0.01\\
9.87	0.01\\
9.88	0.01\\
9.89	0.01\\
9.9	0.01\\
9.91	0.01\\
9.92	0.01\\
9.93	0.01\\
9.94	0.01\\
9.95	0.01\\
9.96	0.01\\
9.97	0.01\\
9.98	0.01\\
9.99	0.01\\
10	0.01\\
10.01	0.01\\
10.02	0.01\\
10.03	0.01\\
10.04	0.01\\
10.05	0.01\\
10.06	0.01\\
10.07	0.01\\
10.08	0.01\\
10.09	0.01\\
10.1	0.01\\
10.11	0.01\\
10.12	0.01\\
10.13	0.01\\
10.14	0.01\\
10.15	0.01\\
10.16	0.01\\
10.17	0.01\\
10.18	0.01\\
10.19	0.01\\
10.2	0.01\\
10.21	0.01\\
10.22	0.01\\
10.23	0.01\\
10.24	0.01\\
10.25	0.01\\
10.26	0.01\\
10.27	0.01\\
10.28	0.01\\
10.29	0.01\\
10.3	0.01\\
10.31	0.01\\
10.32	0.01\\
10.33	0.01\\
10.34	0.01\\
10.35	0.01\\
10.36	0.01\\
10.37	0.01\\
10.38	0.01\\
10.39	0.01\\
10.4	0.01\\
10.41	0.01\\
10.42	0.01\\
10.43	0.01\\
10.44	0.01\\
10.45	0.01\\
10.46	0.01\\
10.47	0.01\\
10.48	0.01\\
10.49	0.01\\
10.5	0.01\\
10.51	0.01\\
10.52	0.01\\
10.53	0.01\\
10.54	0.01\\
10.55	0.01\\
10.56	0.01\\
10.57	0.01\\
10.58	0.01\\
10.59	0.01\\
10.6	0.01\\
10.61	0.01\\
10.62	0.01\\
10.63	0.01\\
10.64	0.01\\
10.65	0.01\\
10.66	0.01\\
10.67	0.01\\
10.68	0.01\\
10.69	0.01\\
10.7	0.01\\
10.71	0.01\\
10.72	0.01\\
10.73	0.01\\
10.74	0.01\\
10.75	0.01\\
10.76	0.01\\
10.77	0.01\\
10.78	0.01\\
10.79	0.01\\
10.8	0.01\\
10.81	0.01\\
10.82	0.01\\
10.83	0.01\\
10.84	0.01\\
10.85	0.01\\
10.86	0.01\\
10.87	0.01\\
10.88	0.01\\
10.89	0.01\\
10.9	0.01\\
10.91	0.01\\
10.92	0.01\\
10.93	0.01\\
10.94	0.01\\
10.95	0.01\\
10.96	0.01\\
10.97	0.01\\
10.98	0.01\\
10.99	0.01\\
11	0.01\\
11.01	0.01\\
11.02	0.01\\
11.03	0.01\\
11.04	0.01\\
11.05	0.01\\
11.06	0.01\\
11.07	0.01\\
11.08	0.01\\
11.09	0.01\\
11.1	0.01\\
11.11	0.01\\
11.12	0.01\\
11.13	0.01\\
11.14	0.01\\
11.15	0.01\\
11.16	0.01\\
11.17	0.01\\
11.18	0.01\\
11.19	0.01\\
11.2	0.01\\
11.21	0.01\\
11.22	0.01\\
11.23	0.01\\
11.24	0.01\\
11.25	0.01\\
11.26	0.01\\
11.27	0.01\\
11.28	0.01\\
11.29	0.01\\
11.3	0.01\\
11.31	0.01\\
11.32	0.01\\
11.33	0.01\\
11.34	0.01\\
11.35	0.01\\
11.36	0.01\\
11.37	0.01\\
11.38	0.01\\
11.39	0.01\\
11.4	0.01\\
11.41	0.01\\
11.42	0.01\\
11.43	0.01\\
11.44	0.01\\
11.45	0.01\\
11.46	0.01\\
11.47	0.01\\
11.48	0.01\\
11.49	0.01\\
11.5	0.01\\
11.51	0.01\\
11.52	0.01\\
11.53	0.01\\
11.54	0.01\\
11.55	0.01\\
11.56	0.01\\
11.57	0.01\\
11.58	0.01\\
11.59	0.01\\
11.6	0.01\\
11.61	0.01\\
11.62	0.01\\
11.63	0.01\\
11.64	0.01\\
11.65	0.01\\
11.66	0.01\\
11.67	0.01\\
11.68	0.01\\
11.69	0.01\\
11.7	0.01\\
11.71	0.01\\
11.72	0.01\\
11.73	0.01\\
11.74	0.01\\
11.75	0.01\\
11.76	0.01\\
11.77	0.01\\
11.78	0.01\\
11.79	0.01\\
11.8	0.01\\
11.81	0.01\\
11.82	0.01\\
11.83	0.01\\
11.84	0.01\\
11.85	0.01\\
11.86	0.01\\
11.87	0.01\\
11.88	0.01\\
11.89	0.01\\
11.9	0.01\\
11.91	0.01\\
11.92	0.01\\
11.93	0.01\\
11.94	0.01\\
11.95	0.01\\
11.96	0.01\\
11.97	0.01\\
11.98	0.01\\
11.99	0.01\\
12	0.01\\
12.01	0.01\\
12.02	0.01\\
12.03	0.01\\
12.04	0.01\\
12.05	0.01\\
12.06	0.01\\
12.07	0.01\\
12.08	0.01\\
12.09	0.01\\
12.1	0.01\\
12.11	0.01\\
12.12	0.01\\
12.13	0.01\\
12.14	0.01\\
12.15	0.01\\
12.16	0.01\\
12.17	0.01\\
12.18	0.01\\
12.19	0.01\\
12.2	0.01\\
12.21	0.01\\
12.22	0.01\\
12.23	0.01\\
12.24	0.01\\
12.25	0.01\\
12.26	0.01\\
12.27	0.01\\
12.28	0.01\\
12.29	0.01\\
12.3	0.01\\
12.31	0.01\\
12.32	0.01\\
12.33	0.01\\
12.34	0.01\\
12.35	0.01\\
12.36	0.01\\
12.37	0.01\\
12.38	0.01\\
12.39	0.01\\
12.4	0.01\\
12.41	0.01\\
12.42	0.01\\
12.43	0.01\\
12.44	0.01\\
12.45	0.01\\
12.46	0.01\\
12.47	0.01\\
12.48	0.01\\
12.49	0.01\\
12.5	0.01\\
12.51	0.01\\
12.52	0.01\\
12.53	0.01\\
12.54	0.01\\
12.55	0.01\\
12.56	0.01\\
12.57	0.01\\
12.58	0.01\\
12.59	0.01\\
12.6	0.01\\
12.61	0.01\\
12.62	0.01\\
12.63	0.01\\
12.64	0.01\\
12.65	0.01\\
12.66	0.01\\
12.67	0.01\\
12.68	0.01\\
12.69	0.01\\
12.7	0.01\\
12.71	0.01\\
12.72	0.01\\
12.73	0.01\\
12.74	0.01\\
12.75	0.01\\
12.76	0.01\\
12.77	0.01\\
12.78	0.01\\
12.79	0.01\\
12.8	0.01\\
12.81	0.01\\
12.82	0.01\\
12.83	0.01\\
12.84	0.01\\
12.85	0.01\\
12.86	0.01\\
12.87	0.01\\
12.88	0.01\\
12.89	0.01\\
12.9	0.01\\
12.91	0.01\\
12.92	0.01\\
12.93	0.01\\
12.94	0.01\\
12.95	0.01\\
12.96	0.01\\
12.97	0.01\\
12.98	0.01\\
12.99	0.01\\
13	0.01\\
13.01	0.01\\
13.02	0.01\\
13.03	0.01\\
13.04	0.01\\
13.05	0.01\\
13.06	0.01\\
13.07	0.01\\
13.08	0.01\\
13.09	0.01\\
13.1	0.01\\
13.11	0.01\\
13.12	0.01\\
13.13	0.01\\
13.14	0.01\\
13.15	0.01\\
13.16	0.01\\
13.17	0.01\\
13.18	0.01\\
13.19	0.01\\
13.2	0.01\\
13.21	0.01\\
13.22	0.01\\
13.23	0.01\\
13.24	0.01\\
13.25	0.01\\
13.26	0.01\\
13.27	0.01\\
13.28	0.01\\
13.29	0.01\\
13.3	0.01\\
13.31	0.01\\
13.32	0.01\\
13.33	0.01\\
13.34	0.01\\
13.35	0.01\\
13.36	0.01\\
13.37	0.01\\
13.38	0.01\\
13.39	0.01\\
13.4	0.01\\
13.41	0.01\\
13.42	0.01\\
13.43	0.01\\
13.44	0.01\\
13.45	0.01\\
13.46	0.01\\
13.47	0.01\\
13.48	0.01\\
13.49	0.01\\
13.5	0.01\\
13.51	0.01\\
13.52	0.01\\
13.53	0.01\\
13.54	0.01\\
13.55	0.01\\
13.56	0.01\\
13.57	0.01\\
13.58	0.01\\
13.59	0.01\\
13.6	0.01\\
13.61	0.01\\
13.62	0.01\\
13.63	0.01\\
13.64	0.01\\
13.65	0.01\\
13.66	0.01\\
13.67	0.01\\
13.68	0.01\\
13.69	0.01\\
13.7	0.01\\
13.71	0.01\\
13.72	0.01\\
13.73	0.01\\
13.74	0.01\\
13.75	0.01\\
13.76	0.01\\
13.77	0.01\\
13.78	0.01\\
13.79	0.01\\
13.8	0.01\\
13.81	0.01\\
13.82	0.01\\
13.83	0.01\\
13.84	0.01\\
13.85	0.01\\
13.86	0.01\\
13.87	0.01\\
13.88	0.01\\
13.89	0.01\\
13.9	0.01\\
13.91	0.01\\
13.92	0.01\\
13.93	0.01\\
13.94	0.01\\
13.95	0.01\\
13.96	0.01\\
13.97	0.01\\
13.98	0.01\\
13.99	0.01\\
14	0.01\\
14.01	0.01\\
14.02	0.01\\
14.03	0.01\\
14.04	0.01\\
14.05	0.01\\
14.06	0.01\\
14.07	0.01\\
14.08	0.01\\
14.09	0.01\\
14.1	0.01\\
14.11	0.01\\
14.12	0.01\\
14.13	0.01\\
14.14	0.01\\
14.15	0.01\\
14.16	0.01\\
14.17	0.01\\
14.18	0.01\\
14.19	0.01\\
14.2	0.01\\
14.21	0.01\\
14.22	0.01\\
14.23	0.01\\
14.24	0.01\\
14.25	0.01\\
14.26	0.01\\
14.27	0.01\\
14.28	0.01\\
14.29	0.01\\
14.3	0.01\\
14.31	0.01\\
14.32	0.01\\
14.33	0.01\\
14.34	0.01\\
14.35	0.01\\
14.36	0.01\\
14.37	0.01\\
14.38	0.01\\
14.39	0.01\\
14.4	0.01\\
14.41	0.01\\
14.42	0.01\\
14.43	0.01\\
14.44	0.01\\
14.45	0.01\\
14.46	0.01\\
14.47	0.01\\
14.48	0.01\\
14.49	0.01\\
14.5	0.01\\
14.51	0.01\\
14.52	0.01\\
14.53	0.01\\
14.54	0.01\\
14.55	0.01\\
14.56	0.01\\
14.57	0.01\\
14.58	0.01\\
14.59	0.01\\
14.6	0.01\\
14.61	0.01\\
14.62	0.01\\
14.63	0.01\\
14.64	0.01\\
14.65	0.01\\
14.66	0.01\\
14.67	0.01\\
14.68	0.01\\
14.69	0.01\\
14.7	0.01\\
14.71	0.01\\
14.72	0.01\\
14.73	0.01\\
14.74	0.01\\
14.75	0.01\\
14.76	0.01\\
14.77	0.01\\
14.78	0.01\\
14.79	0.01\\
14.8	0.01\\
14.81	0.01\\
14.82	0.01\\
14.83	0.01\\
14.84	0.01\\
14.85	0.01\\
14.86	0.01\\
14.87	0.01\\
14.88	0.01\\
14.89	0.01\\
14.9	0.01\\
14.91	0.01\\
14.92	0.01\\
14.93	0.01\\
14.94	0.01\\
14.95	0.01\\
14.96	0.01\\
14.97	0.01\\
14.98	0.01\\
14.99	0.01\\
15	0.01\\
15.01	0.01\\
15.02	0.01\\
15.03	0.01\\
15.04	0.01\\
15.05	0.01\\
15.06	0.01\\
15.07	0.01\\
15.08	0.01\\
15.09	0.01\\
15.1	0.01\\
15.11	0.01\\
15.12	0.01\\
15.13	0.01\\
15.14	0.01\\
15.15	0.01\\
15.16	0.01\\
15.17	0.01\\
15.18	0.01\\
15.19	0.01\\
15.2	0.01\\
15.21	0.01\\
15.22	0.01\\
15.23	0.01\\
15.24	0.01\\
15.25	0.01\\
15.26	0.01\\
15.27	0.01\\
15.28	0.01\\
15.29	0.01\\
15.3	0.01\\
15.31	0.01\\
15.32	0.01\\
15.33	0.01\\
15.34	0.01\\
15.35	0.01\\
15.36	0.01\\
15.37	0.01\\
15.38	0.01\\
15.39	0.01\\
15.4	0.01\\
15.41	0.01\\
15.42	0.01\\
15.43	0.01\\
15.44	0.01\\
15.45	0.01\\
15.46	0.01\\
15.47	0.01\\
15.48	0.01\\
15.49	0.01\\
15.5	0.01\\
15.51	0.01\\
15.52	0.01\\
15.53	0.01\\
15.54	0.01\\
15.55	0.01\\
15.56	0.01\\
15.57	0.01\\
15.58	0.01\\
15.59	0.01\\
15.6	0.01\\
15.61	0.01\\
15.62	0.01\\
15.63	0.01\\
15.64	0.01\\
15.65	0.01\\
15.66	0.01\\
15.67	0.01\\
15.68	0.01\\
15.69	0.01\\
15.7	0.01\\
15.71	0.01\\
15.72	0.01\\
15.73	0.01\\
15.74	0.01\\
15.75	0.01\\
15.76	0.01\\
15.77	0.01\\
15.78	0.01\\
15.79	0.01\\
15.8	0.01\\
15.81	0.01\\
15.82	0.01\\
15.83	0.01\\
15.84	0.01\\
15.85	0.01\\
15.86	0.01\\
15.87	0.01\\
15.88	0.01\\
15.89	0.01\\
15.9	0.01\\
15.91	0.01\\
15.92	0.01\\
15.93	0.01\\
15.94	0.01\\
15.95	0.01\\
15.96	0.01\\
15.97	0.01\\
15.98	0.01\\
15.99	0.01\\
16	0.01\\
16.01	0.01\\
16.02	0.01\\
16.03	0.01\\
16.04	0.01\\
16.05	0.01\\
16.06	0.01\\
16.07	0.01\\
16.08	0.01\\
16.09	0.01\\
16.1	0.01\\
16.11	0.01\\
16.12	0.01\\
16.13	0.01\\
16.14	0.01\\
16.15	0.01\\
16.16	0.01\\
16.17	0.01\\
16.18	0.01\\
16.19	0.01\\
16.2	0.01\\
16.21	0.01\\
16.22	0.01\\
16.23	0.01\\
16.24	0.01\\
16.25	0.01\\
16.26	0.01\\
16.27	0.01\\
16.28	0.01\\
16.29	0.01\\
16.3	0.01\\
16.31	0.01\\
16.32	0.01\\
16.33	0.01\\
16.34	0.01\\
16.35	0.01\\
16.36	0.01\\
16.37	0.01\\
16.38	0.01\\
16.39	0.01\\
16.4	0.01\\
16.41	0.01\\
16.42	0.01\\
16.43	0.01\\
16.44	0.01\\
16.45	0.01\\
16.46	0.01\\
16.47	0.01\\
16.48	0.01\\
16.49	0.01\\
16.5	0.01\\
16.51	0.01\\
16.52	0.01\\
16.53	0.01\\
16.54	0.01\\
16.55	0.01\\
16.56	0.01\\
16.57	0.01\\
16.58	0.01\\
16.59	0.01\\
16.6	0.01\\
16.61	0.01\\
16.62	0.01\\
16.63	0.01\\
16.64	0.01\\
16.65	0.01\\
16.66	0.01\\
16.67	0.01\\
16.68	0.01\\
16.69	0.01\\
16.7	0.01\\
16.71	0.01\\
16.72	0.01\\
16.73	0.01\\
16.74	0.01\\
16.75	0.01\\
16.76	0.01\\
16.77	0.01\\
16.78	0.01\\
16.79	0.01\\
16.8	0.01\\
16.81	0.01\\
16.82	0.01\\
16.83	0.01\\
16.84	0.01\\
16.85	0.01\\
16.86	0.01\\
16.87	0.01\\
16.88	0.01\\
16.89	0.01\\
16.9	0.01\\
16.91	0.01\\
16.92	0.01\\
16.93	0.01\\
16.94	0.01\\
16.95	0.01\\
16.96	0.01\\
16.97	0.01\\
16.98	0.01\\
16.99	0.01\\
17	0.01\\
17.01	0.01\\
17.02	0.01\\
17.03	0.01\\
17.04	0.01\\
17.05	0.01\\
17.06	0.01\\
17.07	0.01\\
17.08	0.01\\
17.09	0.01\\
17.1	0.01\\
17.11	0.01\\
17.12	0.01\\
17.13	0.01\\
17.14	0.01\\
17.15	0.01\\
17.16	0.01\\
17.17	0.01\\
17.18	0.01\\
17.19	0.01\\
17.2	0.01\\
17.21	0.01\\
17.22	0.01\\
17.23	0.01\\
17.24	0.01\\
17.25	0.01\\
17.26	0.01\\
17.27	0.01\\
17.28	0.01\\
17.29	0.01\\
17.3	0.01\\
17.31	0.01\\
17.32	0.01\\
17.33	0.01\\
17.34	0.01\\
17.35	0.01\\
17.36	0.01\\
17.37	0.01\\
17.38	0.01\\
17.39	0.01\\
17.4	0.01\\
17.41	0.01\\
17.42	0.01\\
17.43	0.01\\
17.44	0.01\\
17.45	0.01\\
17.46	0.01\\
17.47	0.01\\
17.48	0.01\\
17.49	0.01\\
17.5	0.01\\
17.51	0.01\\
17.52	0.01\\
17.53	0.01\\
17.54	0.01\\
17.55	0.01\\
17.56	0.01\\
17.57	0.01\\
17.58	0.01\\
17.59	0.01\\
17.6	0.01\\
17.61	0.01\\
17.62	0.01\\
17.63	0.01\\
17.64	0.01\\
17.65	0.01\\
17.66	0.01\\
17.67	0.01\\
17.68	0.01\\
17.69	0.01\\
17.7	0.01\\
17.71	0.01\\
17.72	0.01\\
17.73	0.01\\
17.74	0.01\\
17.75	0.01\\
17.76	0.01\\
17.77	0.01\\
17.78	0.01\\
17.79	0.01\\
17.8	0.01\\
17.81	0.01\\
17.82	0.01\\
17.83	0.01\\
17.84	0.01\\
17.85	0.01\\
17.86	0.01\\
17.87	0.01\\
17.88	0.01\\
17.89	0.01\\
17.9	0.01\\
17.91	0.01\\
17.92	0.01\\
17.93	0.01\\
17.94	0.01\\
17.95	0.01\\
17.96	0.01\\
17.97	0.01\\
17.98	0.01\\
17.99	0.01\\
18	0.01\\
18.01	0.01\\
18.02	0.01\\
18.03	0.01\\
18.04	0.01\\
18.05	0.01\\
18.06	0.01\\
18.07	0.01\\
18.08	0.01\\
18.09	0.01\\
18.1	0.01\\
18.11	0.01\\
18.12	0.01\\
18.13	0.01\\
18.14	0.01\\
18.15	0.01\\
18.16	0.01\\
18.17	0.01\\
18.18	0.01\\
18.19	0.01\\
18.2	0.01\\
18.21	0.01\\
18.22	0.01\\
18.23	0.01\\
18.24	0.01\\
18.25	0.01\\
18.26	0.01\\
18.27	0.01\\
18.28	0.01\\
18.29	0.01\\
18.3	0.01\\
18.31	0.01\\
18.32	0.01\\
18.33	0.01\\
18.34	0.01\\
18.35	0.01\\
18.36	0.01\\
18.37	0.01\\
18.38	0.01\\
18.39	0.01\\
18.4	0.01\\
18.41	0.01\\
18.42	0.01\\
18.43	0.01\\
18.44	0.01\\
18.45	0.01\\
18.46	0.01\\
18.47	0.01\\
18.48	0.01\\
18.49	0.01\\
18.5	0.01\\
18.51	0.01\\
18.52	0.01\\
18.53	0.01\\
18.54	0.01\\
18.55	0.01\\
18.56	0.01\\
18.57	0.01\\
18.58	0.01\\
18.59	0.01\\
18.6	0.01\\
18.61	0.01\\
18.62	0.01\\
18.63	0.01\\
18.64	0.01\\
18.65	0.01\\
18.66	0.01\\
18.67	0.01\\
18.68	0.01\\
18.69	0.01\\
18.7	0.01\\
18.71	0.01\\
18.72	0.01\\
18.73	0.01\\
18.74	0.01\\
18.75	0.01\\
18.76	0.01\\
18.77	0.01\\
18.78	0.01\\
18.79	0.01\\
18.8	0.01\\
18.81	0.01\\
18.82	0.01\\
18.83	0.01\\
18.84	0.01\\
18.85	0.01\\
18.86	0.01\\
18.87	0.01\\
18.88	0.01\\
18.89	0.01\\
18.9	0.01\\
18.91	0.01\\
18.92	0.01\\
18.93	0.01\\
18.94	0.01\\
18.95	0.01\\
18.96	0.01\\
18.97	0.01\\
18.98	0.01\\
18.99	0.01\\
19	0.01\\
19.01	0.01\\
19.02	0.01\\
19.03	0.01\\
19.04	0.01\\
19.05	0.01\\
19.06	0.01\\
19.07	0.01\\
19.08	0.01\\
19.09	0.01\\
19.1	0.01\\
19.11	0.01\\
19.12	0.01\\
19.13	0.01\\
19.14	0.01\\
19.15	0.01\\
19.16	0.01\\
19.17	0.01\\
19.18	0.01\\
19.19	0.01\\
19.2	0.01\\
19.21	0.01\\
19.22	0.01\\
19.23	0.01\\
19.24	0.01\\
19.25	0.01\\
19.26	0.01\\
19.27	0.01\\
19.28	0.01\\
19.29	0.01\\
19.3	0.01\\
19.31	0.01\\
19.32	0.01\\
19.33	0.01\\
19.34	0.01\\
19.35	0.01\\
19.36	0.01\\
19.37	0.01\\
19.38	0.01\\
19.39	0.01\\
19.4	0.01\\
19.41	0.01\\
19.42	0.01\\
19.43	0.01\\
19.44	0.01\\
19.45	0.01\\
19.46	0.01\\
19.47	0.01\\
19.48	0.01\\
19.49	0.01\\
19.5	0.01\\
19.51	0.01\\
19.52	0.01\\
19.53	0.01\\
19.54	0.01\\
19.55	0.01\\
19.56	0.01\\
19.57	0.01\\
19.58	0.01\\
19.59	0.01\\
19.6	0.01\\
19.61	0.01\\
19.62	0.01\\
19.63	0.01\\
19.64	0.01\\
19.65	0.01\\
19.66	0.01\\
19.67	0.01\\
19.68	0.01\\
19.69	0.01\\
19.7	0.01\\
19.71	0.01\\
19.72	0.01\\
19.73	0.01\\
19.74	0.01\\
19.75	0.01\\
19.76	0.01\\
19.77	0.01\\
19.78	0.01\\
19.79	0.01\\
19.8	0.01\\
19.81	0.01\\
19.82	0.01\\
19.83	0.01\\
19.84	0.01\\
19.85	0.01\\
19.86	0.01\\
19.87	0.01\\
19.88	0.01\\
19.89	0.01\\
19.9	0.01\\
19.91	0.01\\
19.92	0.01\\
19.93	0.01\\
19.94	0.01\\
19.95	0.01\\
19.96	0.01\\
19.97	0.01\\
19.98	0.01\\
19.99	0.01\\
20	0.01\\
20.01	0.01\\
20.02	0.01\\
20.03	0.01\\
20.04	0.01\\
20.05	0.01\\
20.06	0.01\\
20.07	0.01\\
20.08	0.01\\
20.09	0.01\\
20.1	0.01\\
20.11	0.01\\
20.12	0.01\\
20.13	0.01\\
20.14	0.01\\
20.15	0.01\\
20.16	0.01\\
20.17	0.01\\
20.18	0.01\\
20.19	0.01\\
20.2	0.01\\
20.21	0.01\\
20.22	0.01\\
20.23	0.01\\
20.24	0.01\\
20.25	0.01\\
20.26	0.01\\
20.27	0.01\\
20.28	0.01\\
20.29	0.01\\
20.3	0.01\\
20.31	0.01\\
20.32	0.01\\
20.33	0.01\\
20.34	0.01\\
20.35	0.01\\
20.36	0.01\\
20.37	0.01\\
20.38	0.01\\
20.39	0.01\\
20.4	0.01\\
20.41	0.01\\
20.42	0.01\\
20.43	0.01\\
20.44	0.01\\
20.45	0.01\\
20.46	0.01\\
20.47	0.01\\
20.48	0.01\\
20.49	0.01\\
20.5	0.01\\
20.51	0.01\\
20.52	0.01\\
20.53	0.01\\
20.54	0.01\\
20.55	0.01\\
20.56	0.01\\
20.57	0.01\\
20.58	0.01\\
20.59	0.01\\
20.6	0.01\\
20.61	0.01\\
20.62	0.01\\
20.63	0.01\\
20.64	0.01\\
20.65	0.01\\
20.66	0.01\\
20.67	0.01\\
20.68	0.01\\
20.69	0.01\\
20.7	0.01\\
20.71	0.01\\
20.72	0.01\\
20.73	0.01\\
20.74	0.01\\
20.75	0.01\\
20.76	0.01\\
20.77	0.01\\
20.78	0.01\\
20.79	0.01\\
20.8	0.01\\
20.81	0.01\\
20.82	0.01\\
20.83	0.01\\
20.84	0.01\\
20.85	0.01\\
20.86	0.01\\
20.87	0.01\\
20.88	0.01\\
20.89	0.01\\
20.9	0.01\\
20.91	0.01\\
20.92	0.01\\
20.93	0.01\\
20.94	0.01\\
20.95	0.01\\
20.96	0.01\\
20.97	0.01\\
20.98	0.01\\
20.99	0.01\\
21	0.01\\
21.01	0.01\\
21.02	0.01\\
21.03	0.01\\
21.04	0.01\\
21.05	0.01\\
21.06	0.01\\
21.07	0.01\\
21.08	0.01\\
21.09	0.01\\
21.1	0.01\\
21.11	0.01\\
21.12	0.01\\
21.13	0.01\\
21.14	0.01\\
21.15	0.01\\
21.16	0.01\\
21.17	0.01\\
21.18	0.01\\
21.19	0.01\\
21.2	0.01\\
21.21	0.01\\
21.22	0.01\\
21.23	0.01\\
21.24	0.01\\
21.25	0.01\\
21.26	0.01\\
21.27	0.01\\
21.28	0.01\\
21.29	0.01\\
21.3	0.01\\
21.31	0.01\\
21.32	0.01\\
21.33	0.01\\
21.34	0.01\\
21.35	0.01\\
21.36	0.01\\
21.37	0.01\\
21.38	0.01\\
21.39	0.01\\
21.4	0.01\\
21.41	0.01\\
21.42	0.01\\
21.43	0.01\\
21.44	0.01\\
21.45	0.01\\
21.46	0.01\\
21.47	0.01\\
21.48	0.01\\
21.49	0.01\\
21.5	0.01\\
21.51	0.01\\
21.52	0.01\\
21.53	0.01\\
21.54	0.01\\
21.55	0.01\\
21.56	0.01\\
21.57	0.01\\
21.58	0.01\\
21.59	0.01\\
21.6	0.01\\
21.61	0.01\\
21.62	0.01\\
21.63	0.01\\
21.64	0.01\\
21.65	0.01\\
21.66	0.01\\
21.67	0.01\\
21.68	0.01\\
21.69	0.01\\
21.7	0.01\\
21.71	0.01\\
21.72	0.01\\
21.73	0.01\\
21.74	0.01\\
21.75	0.01\\
21.76	0.01\\
21.77	0.01\\
21.78	0.01\\
21.79	0.01\\
21.8	0.01\\
21.81	0.01\\
21.82	0.01\\
21.83	0.01\\
21.84	0.01\\
21.85	0.01\\
21.86	0.01\\
21.87	0.01\\
21.88	0.01\\
21.89	0.01\\
21.9	0.01\\
21.91	0.01\\
21.92	0.01\\
21.93	0.01\\
21.94	0.01\\
21.95	0.01\\
21.96	0.01\\
21.97	0.01\\
21.98	0.01\\
21.99	0.01\\
22	0.01\\
22.01	0.01\\
22.02	0.01\\
22.03	0.01\\
22.04	0.01\\
22.05	0.01\\
22.06	0.01\\
22.07	0.01\\
22.08	0.01\\
22.09	0.01\\
22.1	0.01\\
22.11	0.01\\
22.12	0.01\\
22.13	0.01\\
22.14	0.01\\
22.15	0.01\\
22.16	0.01\\
22.17	0.01\\
22.18	0.01\\
22.19	0.01\\
22.2	0.01\\
22.21	0.01\\
22.22	0.01\\
22.23	0.01\\
22.24	0.01\\
22.25	0.01\\
22.26	0.01\\
22.27	0.01\\
22.28	0.01\\
22.29	0.01\\
22.3	0.01\\
22.31	0.01\\
22.32	0.01\\
22.33	0.01\\
22.34	0.01\\
22.35	0.01\\
22.36	0.01\\
22.37	0.01\\
22.38	0.01\\
22.39	0.01\\
22.4	0.01\\
22.41	0.01\\
22.42	0.01\\
22.43	0.01\\
22.44	0.01\\
22.45	0.01\\
22.46	0.01\\
22.47	0.01\\
22.48	0.01\\
22.49	0.01\\
22.5	0.01\\
22.51	0.01\\
22.52	0.01\\
22.53	0.01\\
22.54	0.01\\
22.55	0.01\\
22.56	0.01\\
22.57	0.01\\
22.58	0.01\\
22.59	0.01\\
22.6	0.01\\
22.61	0.01\\
22.62	0.01\\
22.63	0.01\\
22.64	0.01\\
22.65	0.01\\
22.66	0.01\\
22.67	0.01\\
22.68	0.01\\
22.69	0.01\\
22.7	0.01\\
22.71	0.01\\
22.72	0.01\\
22.73	0.01\\
22.74	0.01\\
22.75	0.01\\
22.76	0.01\\
22.77	0.01\\
22.78	0.01\\
22.79	0.01\\
22.8	0.01\\
22.81	0.01\\
22.82	0.01\\
22.83	0.01\\
22.84	0.01\\
22.85	0.01\\
22.86	0.01\\
22.87	0.01\\
22.88	0.01\\
22.89	0.01\\
22.9	0.01\\
22.91	0.01\\
22.92	0.01\\
22.93	0.01\\
22.94	0.01\\
22.95	0.01\\
22.96	0.01\\
22.97	0.01\\
22.98	0.01\\
22.99	0.01\\
23	0.01\\
23.01	0.01\\
23.02	0.01\\
23.03	0.01\\
23.04	0.01\\
23.05	0.01\\
23.06	0.01\\
23.07	0.01\\
23.08	0.01\\
23.09	0.01\\
23.1	0.01\\
23.11	0.01\\
23.12	0.01\\
23.13	0.01\\
23.14	0.01\\
23.15	0.01\\
23.16	0.01\\
23.17	0.01\\
23.18	0.01\\
23.19	0.01\\
23.2	0.01\\
23.21	0.01\\
23.22	0.01\\
23.23	0.01\\
23.24	0.01\\
23.25	0.01\\
23.26	0.01\\
23.27	0.01\\
23.28	0.01\\
23.29	0.01\\
23.3	0.01\\
23.31	0.01\\
23.32	0.01\\
23.33	0.01\\
23.34	0.01\\
23.35	0.01\\
23.36	0.01\\
23.37	0.01\\
23.38	0.01\\
23.39	0.01\\
23.4	0.01\\
23.41	0.01\\
23.42	0.01\\
23.43	0.01\\
23.44	0.01\\
23.45	0.01\\
23.46	0.01\\
23.47	0.01\\
23.48	0.01\\
23.49	0.01\\
23.5	0.01\\
23.51	0.01\\
23.52	0.01\\
23.53	0.01\\
23.54	0.01\\
23.55	0.01\\
23.56	0.01\\
23.57	0.01\\
23.58	0.01\\
23.59	0.01\\
23.6	0.01\\
23.61	0.01\\
23.62	0.01\\
23.63	0.01\\
23.64	0.01\\
23.65	0.01\\
23.66	0.01\\
23.67	0.01\\
23.68	0.01\\
23.69	0.01\\
23.7	0.01\\
23.71	0.01\\
23.72	0.01\\
23.73	0.01\\
23.74	0.01\\
23.75	0.01\\
23.76	0.01\\
23.77	0.01\\
23.78	0.01\\
23.79	0.01\\
23.8	0.01\\
23.81	0.01\\
23.82	0.01\\
23.83	0.01\\
23.84	0.01\\
23.85	0.01\\
23.86	0.01\\
23.87	0.01\\
23.88	0.01\\
23.89	0.01\\
23.9	0.01\\
23.91	0.01\\
23.92	0.01\\
23.93	0.01\\
23.94	0.01\\
23.95	0.01\\
23.96	0.01\\
23.97	0.01\\
23.98	0.01\\
23.99	0.01\\
24	0.01\\
24.01	0.01\\
24.02	0.01\\
24.03	0.01\\
24.04	0.01\\
24.05	0.01\\
24.06	0.01\\
24.07	0.01\\
24.08	0.01\\
24.09	0.01\\
24.1	0.01\\
24.11	0.01\\
24.12	0.01\\
24.13	0.01\\
24.14	0.01\\
24.15	0.01\\
24.16	0.01\\
24.17	0.01\\
24.18	0.01\\
24.19	0.01\\
24.2	0.01\\
24.21	0.01\\
24.22	0.01\\
24.23	0.01\\
24.24	0.01\\
24.25	0.01\\
24.26	0.01\\
24.27	0.01\\
24.28	0.01\\
24.29	0.01\\
24.3	0.01\\
24.31	0.01\\
24.32	0.01\\
24.33	0.01\\
24.34	0.01\\
24.35	0.01\\
24.36	0.01\\
24.37	0.01\\
24.38	0.01\\
24.39	0.01\\
24.4	0.01\\
24.41	0.01\\
24.42	0.01\\
24.43	0.01\\
24.44	0.01\\
24.45	0.01\\
24.46	0.01\\
24.47	0.01\\
24.48	0.01\\
24.49	0.01\\
24.5	0.01\\
24.51	0.01\\
24.52	0.01\\
24.53	0.01\\
24.54	0.01\\
24.55	0.01\\
24.56	0.01\\
24.57	0.01\\
24.58	0.01\\
24.59	0.01\\
24.6	0.01\\
24.61	0.01\\
24.62	0.01\\
24.63	0.01\\
24.64	0.01\\
24.65	0.01\\
24.66	0.01\\
24.67	0.01\\
24.68	0.01\\
24.69	0.01\\
24.7	0.01\\
24.71	0.01\\
24.72	0.01\\
24.73	0.01\\
24.74	0.01\\
24.75	0.01\\
24.76	0.01\\
24.77	0.01\\
24.78	0.01\\
24.79	0.01\\
24.8	0.01\\
24.81	0.01\\
24.82	0.01\\
24.83	0.01\\
24.84	0.01\\
24.85	0.01\\
24.86	0.01\\
24.87	0.01\\
24.88	0.01\\
24.89	0.01\\
24.9	0.01\\
24.91	0.01\\
24.92	0.01\\
24.93	0.01\\
24.94	0.01\\
24.95	0.01\\
24.96	0.01\\
24.97	0.01\\
24.98	0.01\\
24.99	0.01\\
25	0.01\\
25.01	0.01\\
25.02	0.01\\
25.03	0.01\\
25.04	0.01\\
25.05	0.01\\
25.06	0.01\\
25.07	0.01\\
25.08	0.01\\
25.09	0.01\\
25.1	0.01\\
25.11	0.01\\
25.12	0.01\\
25.13	0.01\\
25.14	0.01\\
25.15	0.01\\
25.16	0.01\\
25.17	0.01\\
25.18	0.01\\
25.19	0.01\\
25.2	0.01\\
25.21	0.01\\
25.22	0.01\\
25.23	0.01\\
25.24	0.01\\
25.25	0.01\\
25.26	0.01\\
25.27	0.01\\
25.28	0.01\\
25.29	0.01\\
25.3	0.01\\
25.31	0.01\\
25.32	0.01\\
25.33	0.01\\
25.34	0.01\\
25.35	0.01\\
25.36	0.01\\
25.37	0.01\\
25.38	0.01\\
25.39	0.01\\
25.4	0.01\\
25.41	0.01\\
25.42	0.01\\
25.43	0.01\\
25.44	0.01\\
25.45	0.01\\
25.46	0.01\\
25.47	0.01\\
25.48	0.01\\
25.49	0.01\\
25.5	0.01\\
25.51	0.01\\
25.52	0.01\\
25.53	0.01\\
25.54	0.01\\
25.55	0.01\\
25.56	0.01\\
25.57	0.01\\
25.58	0.01\\
25.59	0.01\\
25.6	0.01\\
25.61	0.01\\
25.62	0.01\\
25.63	0.01\\
25.64	0.01\\
25.65	0.01\\
25.66	0.01\\
25.67	0.01\\
25.68	0.01\\
25.69	0.01\\
25.7	0.01\\
25.71	0.01\\
25.72	0.01\\
25.73	0.01\\
25.74	0.01\\
25.75	0.01\\
25.76	0.01\\
25.77	0.01\\
25.78	0.01\\
25.79	0.01\\
25.8	0.01\\
25.81	0.01\\
25.82	0.01\\
25.83	0.01\\
25.84	0.01\\
25.85	0.01\\
25.86	0.01\\
25.87	0.01\\
25.88	0.01\\
25.89	0.01\\
25.9	0.01\\
25.91	0.01\\
25.92	0.01\\
25.93	0.01\\
25.94	0.01\\
25.95	0.01\\
25.96	0.01\\
25.97	0.01\\
25.98	0.01\\
25.99	0.01\\
26	0.01\\
26.01	0.01\\
26.02	0.01\\
26.03	0.01\\
26.04	0.01\\
26.05	0.01\\
26.06	0.01\\
26.07	0.01\\
26.08	0.01\\
26.09	0.01\\
26.1	0.01\\
26.11	0.01\\
26.12	0.01\\
26.13	0.01\\
26.14	0.01\\
26.15	0.01\\
26.16	0.01\\
26.17	0.01\\
26.18	0.01\\
26.19	0.01\\
26.2	0.01\\
26.21	0.01\\
26.22	0.01\\
26.23	0.01\\
26.24	0.01\\
26.25	0.01\\
26.26	0.01\\
26.27	0.01\\
26.28	0.01\\
26.29	0.01\\
26.3	0.01\\
26.31	0.01\\
26.32	0.01\\
26.33	0.01\\
26.34	0.01\\
26.35	0.01\\
26.36	0.01\\
26.37	0.01\\
26.38	0.01\\
26.39	0.01\\
26.4	0.01\\
26.41	0.01\\
26.42	0.01\\
26.43	0.01\\
26.44	0.01\\
26.45	0.01\\
26.46	0.01\\
26.47	0.01\\
26.48	0.01\\
26.49	0.01\\
26.5	0.01\\
26.51	0.01\\
26.52	0.01\\
26.53	0.01\\
26.54	0.01\\
26.55	0.01\\
26.56	0.01\\
26.57	0.01\\
26.58	0.01\\
26.59	0.01\\
26.6	0.01\\
26.61	0.01\\
26.62	0.01\\
26.63	0.01\\
26.64	0.01\\
26.65	0.01\\
26.66	0.01\\
26.67	0.01\\
26.68	0.01\\
26.69	0.01\\
26.7	0.01\\
26.71	0.01\\
26.72	0.01\\
26.73	0.01\\
26.74	0.01\\
26.75	0.01\\
26.76	0.01\\
26.77	0.01\\
26.78	0.01\\
26.79	0.01\\
26.8	0.01\\
26.81	0.01\\
26.82	0.01\\
26.83	0.01\\
26.84	0.01\\
26.85	0.01\\
26.86	0.01\\
26.87	0.01\\
26.88	0.01\\
26.89	0.01\\
26.9	0.01\\
26.91	0.01\\
26.92	0.01\\
26.93	0.01\\
26.94	0.01\\
26.95	0.01\\
26.96	0.01\\
26.97	0.01\\
26.98	0.01\\
26.99	0.01\\
27	0.01\\
27.01	0.01\\
27.02	0.01\\
27.03	0.01\\
27.04	0.01\\
27.05	0.01\\
27.06	0.01\\
27.07	0.01\\
27.08	0.01\\
27.09	0.01\\
27.1	0.01\\
27.11	0.01\\
27.12	0.01\\
27.13	0.01\\
27.14	0.01\\
27.15	0.01\\
27.16	0.01\\
27.17	0.01\\
27.18	0.01\\
27.19	0.01\\
27.2	0.01\\
27.21	0.01\\
27.22	0.01\\
27.23	0.01\\
27.24	0.01\\
27.25	0.01\\
27.26	0.01\\
27.27	0.01\\
27.28	0.01\\
27.29	0.01\\
27.3	0.01\\
27.31	0.01\\
27.32	0.01\\
27.33	0.01\\
27.34	0.01\\
27.35	0.01\\
27.36	0.01\\
27.37	0.01\\
27.38	0.01\\
27.39	0.01\\
27.4	0.01\\
27.41	0.01\\
27.42	0.01\\
27.43	0.01\\
27.44	0.01\\
27.45	0.01\\
27.46	0.01\\
27.47	0.01\\
27.48	0.01\\
27.49	0.01\\
27.5	0.01\\
27.51	0.01\\
27.52	0.01\\
27.53	0.01\\
27.54	0.01\\
27.55	0.01\\
27.56	0.01\\
27.57	0.01\\
27.58	0.01\\
27.59	0.01\\
27.6	0.01\\
27.61	0.01\\
27.62	0.01\\
27.63	0.01\\
27.64	0.01\\
27.65	0.01\\
27.66	0.01\\
27.67	0.01\\
27.68	0.01\\
27.69	0.01\\
27.7	0.01\\
27.71	0.01\\
27.72	0.01\\
27.73	0.01\\
27.74	0.01\\
27.75	0.01\\
27.76	0.01\\
27.77	0.01\\
27.78	0.01\\
27.79	0.01\\
27.8	0.01\\
27.81	0.01\\
27.82	0.01\\
27.83	0.01\\
27.84	0.01\\
27.85	0.01\\
27.86	0.01\\
27.87	0.01\\
27.88	0.01\\
27.89	0.01\\
27.9	0.01\\
27.91	0.01\\
27.92	0.01\\
27.93	0.01\\
27.94	0.01\\
27.95	0.01\\
27.96	0.01\\
27.97	0.01\\
27.98	0.01\\
27.99	0.01\\
28	0.01\\
28.01	0.01\\
28.02	0.01\\
28.03	0.01\\
28.04	0.01\\
28.05	0.01\\
28.06	0.01\\
28.07	0.01\\
28.08	0.01\\
28.09	0.01\\
28.1	0.01\\
28.11	0.01\\
28.12	0.01\\
28.13	0.01\\
28.14	0.01\\
28.15	0.01\\
28.16	0.01\\
28.17	0.01\\
28.18	0.01\\
28.19	0.01\\
28.2	0.01\\
28.21	0.01\\
28.22	0.01\\
28.23	0.01\\
28.24	0.01\\
28.25	0.01\\
28.26	0.01\\
28.27	0.01\\
28.28	0.01\\
28.29	0.01\\
28.3	0.01\\
28.31	0.01\\
28.32	0.01\\
28.33	0.01\\
28.34	0.01\\
28.35	0.01\\
28.36	0.01\\
28.37	0.01\\
28.38	0.01\\
28.39	0.01\\
28.4	0.01\\
28.41	0.01\\
28.42	0.01\\
28.43	0.01\\
28.44	0.01\\
28.45	0.01\\
28.46	0.01\\
28.47	0.01\\
28.48	0.01\\
28.49	0.01\\
28.5	0.01\\
28.51	0.01\\
28.52	0.01\\
28.53	0.01\\
28.54	0.01\\
28.55	0.01\\
28.56	0.01\\
28.57	0.01\\
28.58	0.01\\
28.59	0.01\\
28.6	0.01\\
28.61	0.01\\
28.62	0.01\\
28.63	0.01\\
28.64	0.01\\
28.65	0.01\\
28.66	0.01\\
28.67	0.01\\
28.68	0.01\\
28.69	0.01\\
28.7	0.01\\
28.71	0.01\\
28.72	0.01\\
28.73	0.01\\
28.74	0.01\\
28.75	0.01\\
28.76	0.01\\
28.77	0.01\\
28.78	0.01\\
28.79	0.01\\
28.8	0.01\\
28.81	0.01\\
28.82	0.01\\
28.83	0.01\\
28.84	0.01\\
28.85	0.01\\
28.86	0.01\\
28.87	0.01\\
28.88	0.01\\
28.89	0.01\\
28.9	0.01\\
28.91	0.01\\
28.92	0.01\\
28.93	0.01\\
28.94	0.01\\
28.95	0.01\\
28.96	0.01\\
28.97	0.01\\
28.98	0.01\\
28.99	0.01\\
29	0.01\\
29.01	0.01\\
29.02	0.01\\
29.03	0.01\\
29.04	0.01\\
29.05	0.01\\
29.06	0.01\\
29.07	0.01\\
29.08	0.01\\
29.09	0.01\\
29.1	0.01\\
29.11	0.01\\
29.12	0.01\\
29.13	0.01\\
29.14	0.01\\
29.15	0.01\\
29.16	0.01\\
29.17	0.01\\
29.18	0.01\\
29.19	0.01\\
29.2	0.01\\
29.21	0.01\\
29.22	0.01\\
29.23	0.01\\
29.24	0.01\\
29.25	0.01\\
29.26	0.01\\
29.27	0.01\\
29.28	0.01\\
29.29	0.01\\
29.3	0.01\\
29.31	0.01\\
29.32	0.01\\
29.33	0.01\\
29.34	0.01\\
29.35	0.01\\
29.36	0.01\\
29.37	0.01\\
29.38	0.01\\
29.39	0.01\\
29.4	0.01\\
29.41	0.01\\
29.42	0.01\\
29.43	0.01\\
29.44	0.01\\
29.45	0.01\\
29.46	0.01\\
29.47	0.01\\
29.48	0.01\\
29.49	0.01\\
29.5	0.01\\
29.51	0.01\\
29.52	0.01\\
29.53	0.01\\
29.54	0.01\\
29.55	0.01\\
29.56	0.01\\
29.57	0.01\\
29.58	0.01\\
29.59	0.01\\
29.6	0.01\\
29.61	0.01\\
29.62	0.01\\
29.63	0.01\\
29.64	0.01\\
29.65	0.01\\
29.66	0.01\\
29.67	0.01\\
29.68	0.01\\
29.69	0.01\\
29.7	0.01\\
29.71	0.01\\
29.72	0.01\\
29.73	0.01\\
29.74	0.01\\
29.75	0.01\\
29.76	0.01\\
29.77	0.01\\
29.78	0.01\\
29.79	0.01\\
29.8	0.01\\
29.81	0.01\\
29.82	0.01\\
29.83	0.01\\
29.84	0.01\\
29.85	0.01\\
29.86	0.01\\
29.87	0.01\\
29.88	0.01\\
29.89	0.01\\
29.9	0.01\\
29.91	0.01\\
29.92	0.01\\
29.93	0.01\\
29.94	0.01\\
29.95	0.01\\
29.96	0.01\\
29.97	0.01\\
29.98	0.01\\
29.99	0.01\\
30	0.01\\
30.01	0.01\\
30.02	0.01\\
30.03	0.01\\
30.04	0.01\\
30.05	0.01\\
30.06	0.01\\
30.07	0.01\\
30.08	0.01\\
30.09	0.01\\
30.1	0.01\\
30.11	0.01\\
30.12	0.01\\
30.13	0.01\\
30.14	0.01\\
30.15	0.01\\
30.16	0.01\\
30.17	0.01\\
30.18	0.01\\
30.19	0.01\\
30.2	0.01\\
30.21	0.01\\
30.22	0.01\\
30.23	0.01\\
30.24	0.01\\
30.25	0.01\\
30.26	0.01\\
30.27	0.01\\
30.28	0.01\\
30.29	0.01\\
30.3	0.01\\
30.31	0.01\\
30.32	0.01\\
30.33	0.01\\
30.34	0.01\\
30.35	0.01\\
30.36	0.01\\
30.37	0.01\\
30.38	0.01\\
30.39	0.01\\
30.4	0.01\\
30.41	0.01\\
30.42	0.01\\
30.43	0.01\\
30.44	0.01\\
30.45	0.01\\
30.46	0.01\\
30.47	0.01\\
30.48	0.01\\
30.49	0.01\\
30.5	0.01\\
30.51	0.01\\
30.52	0.01\\
30.53	0.01\\
30.54	0.01\\
30.55	0.01\\
30.56	0.01\\
30.57	0.01\\
30.58	0.01\\
30.59	0.01\\
30.6	0.01\\
30.61	0.01\\
30.62	0.01\\
30.63	0.01\\
30.64	0.01\\
30.65	0.01\\
30.66	0.01\\
30.67	0.01\\
30.68	0.01\\
30.69	0.01\\
30.7	0.01\\
30.71	0.01\\
30.72	0.01\\
30.73	0.01\\
30.74	0.01\\
30.75	0.01\\
30.76	0.01\\
30.77	0.01\\
30.78	0.01\\
30.79	0.01\\
30.8	0.01\\
30.81	0.01\\
30.82	0.01\\
30.83	0.01\\
30.84	0.01\\
30.85	0.01\\
30.86	0.01\\
30.87	0.01\\
30.88	0.01\\
30.89	0.01\\
30.9	0.01\\
30.91	0.01\\
30.92	0.01\\
30.93	0.01\\
30.94	0.01\\
30.95	0.01\\
30.96	0.01\\
30.97	0.01\\
30.98	0.01\\
30.99	0.01\\
31	0.01\\
31.01	0.01\\
31.02	0.01\\
31.03	0.01\\
31.04	0.01\\
31.05	0.01\\
31.06	0.01\\
31.07	0.01\\
31.08	0.01\\
31.09	0.01\\
31.1	0.01\\
31.11	0.01\\
31.12	0.01\\
31.13	0.01\\
31.14	0.01\\
31.15	0.01\\
31.16	0.01\\
31.17	0.01\\
31.18	0.01\\
31.19	0.01\\
31.2	0.01\\
31.21	0.01\\
31.22	0.01\\
31.23	0.01\\
31.24	0.01\\
31.25	0.01\\
31.26	0.01\\
31.27	0.01\\
31.28	0.01\\
31.29	0.01\\
31.3	0.01\\
31.31	0.01\\
31.32	0.01\\
31.33	0.01\\
31.34	0.01\\
31.35	0.01\\
31.36	0.01\\
31.37	0.01\\
31.38	0.01\\
31.39	0.01\\
31.4	0.01\\
31.41	0.01\\
31.42	0.01\\
31.43	0.01\\
31.44	0.01\\
31.45	0.01\\
31.46	0.01\\
31.47	0.01\\
31.48	0.01\\
31.49	0.01\\
31.5	0.01\\
31.51	0.01\\
31.52	0.01\\
31.53	0.01\\
31.54	0.01\\
31.55	0.01\\
31.56	0.01\\
31.57	0.01\\
31.58	0.01\\
31.59	0.01\\
31.6	0.01\\
31.61	0.01\\
31.62	0.01\\
31.63	0.01\\
31.64	0.01\\
31.65	0.01\\
31.66	0.01\\
31.67	0.01\\
31.68	0.01\\
31.69	0.01\\
31.7	0.01\\
31.71	0.01\\
31.72	0.01\\
31.73	0.01\\
31.74	0.01\\
31.75	0.01\\
31.76	0.01\\
31.77	0.01\\
31.78	0.01\\
31.79	0.01\\
31.8	0.01\\
31.81	0.01\\
31.82	0.01\\
31.83	0.01\\
31.84	0.01\\
31.85	0.01\\
31.86	0.01\\
31.87	0.01\\
31.88	0.01\\
31.89	0.01\\
31.9	0.01\\
31.91	0.01\\
31.92	0.01\\
31.93	0.01\\
31.94	0.01\\
31.95	0.01\\
31.96	0.01\\
31.97	0.01\\
31.98	0.01\\
31.99	0.01\\
32	0.01\\
32.01	0.01\\
32.02	0.01\\
32.03	0.01\\
32.04	0.01\\
32.05	0.01\\
32.06	0.01\\
32.07	0.01\\
32.08	0.01\\
32.09	0.01\\
32.1	0.01\\
32.11	0.01\\
32.12	0.01\\
32.13	0.01\\
32.14	0.01\\
32.15	0.01\\
32.16	0.01\\
32.17	0.01\\
32.18	0.01\\
32.19	0.01\\
32.2	0.01\\
32.21	0.01\\
32.22	0.01\\
32.23	0.01\\
32.24	0.01\\
32.25	0.01\\
32.26	0.01\\
32.27	0.01\\
32.28	0.01\\
32.29	0.01\\
32.3	0.01\\
32.31	0.01\\
32.32	0.01\\
32.33	0.01\\
32.34	0.01\\
32.35	0.01\\
32.36	0.01\\
32.37	0.01\\
32.38	0.01\\
32.39	0.01\\
32.4	0.01\\
32.41	0.01\\
32.42	0.01\\
32.43	0.01\\
32.44	0.01\\
32.45	0.01\\
32.46	0.01\\
32.47	0.01\\
32.48	0.01\\
32.49	0.01\\
32.5	0.01\\
32.51	0.01\\
32.52	0.01\\
32.53	0.01\\
32.54	0.01\\
32.55	0.01\\
32.56	0.01\\
32.57	0.01\\
32.58	0.01\\
32.59	0.01\\
32.6	0.01\\
32.61	0.01\\
32.62	0.01\\
32.63	0.01\\
32.64	0.01\\
32.65	0.01\\
32.66	0.01\\
32.67	0.01\\
32.68	0.01\\
32.69	0.01\\
32.7	0.01\\
32.71	0.01\\
32.72	0.01\\
32.73	0.01\\
32.74	0.01\\
32.75	0.01\\
32.76	0.01\\
32.77	0.01\\
32.78	0.01\\
32.79	0.01\\
32.8	0.01\\
32.81	0.01\\
32.82	0.01\\
32.83	0.01\\
32.84	0.01\\
32.85	0.01\\
32.86	0.01\\
32.87	0.01\\
32.88	0.01\\
32.89	0.01\\
32.9	0.01\\
32.91	0.01\\
32.92	0.01\\
32.93	0.01\\
32.94	0.01\\
32.95	0.01\\
32.96	0.01\\
32.97	0.01\\
32.98	0.01\\
32.99	0.01\\
33	0.01\\
33.01	0.01\\
33.02	0.01\\
33.03	0.01\\
33.04	0.01\\
33.05	0.01\\
33.06	0.01\\
33.07	0.01\\
33.08	0.01\\
33.09	0.01\\
33.1	0.01\\
33.11	0.01\\
33.12	0.01\\
33.13	0.01\\
33.14	0.01\\
33.15	0.01\\
33.16	0.01\\
33.17	0.01\\
33.18	0.01\\
33.19	0.01\\
33.2	0.01\\
33.21	0.01\\
33.22	0.01\\
33.23	0.01\\
33.24	0.01\\
33.25	0.01\\
33.26	0.01\\
33.27	0.01\\
33.28	0.01\\
33.29	0.01\\
33.3	0.01\\
33.31	0.01\\
33.32	0.01\\
33.33	0.01\\
33.34	0.01\\
33.35	0.01\\
33.36	0.01\\
33.37	0.01\\
33.38	0.01\\
33.39	0.01\\
33.4	0.01\\
33.41	0.01\\
33.42	0.01\\
33.43	0.01\\
33.44	0.01\\
33.45	0.01\\
33.46	0.01\\
33.47	0.01\\
33.48	0.01\\
33.49	0.01\\
33.5	0.01\\
33.51	0.01\\
33.52	0.01\\
33.53	0.01\\
33.54	0.01\\
33.55	0.01\\
33.56	0.01\\
33.57	0.01\\
33.58	0.01\\
33.59	0.01\\
33.6	0.01\\
33.61	0.01\\
33.62	0.01\\
33.63	0.01\\
33.64	0.01\\
33.65	0.01\\
33.66	0.01\\
33.67	0.01\\
33.68	0.01\\
33.69	0.01\\
33.7	0.01\\
33.71	0.01\\
33.72	0.01\\
33.73	0.01\\
33.74	0.01\\
33.75	0.01\\
33.76	0.01\\
33.77	0.01\\
33.78	0.01\\
33.79	0.01\\
33.8	0.01\\
33.81	0.01\\
33.82	0.01\\
33.83	0.01\\
33.84	0.01\\
33.85	0.01\\
33.86	0.01\\
33.87	0.01\\
33.88	0.01\\
33.89	0.01\\
33.9	0.01\\
33.91	0.01\\
33.92	0.01\\
33.93	0.01\\
33.94	0.01\\
33.95	0.01\\
33.96	0.01\\
33.97	0.01\\
33.98	0.01\\
33.99	0.01\\
34	0.01\\
34.01	0.01\\
34.02	0.01\\
34.03	0.01\\
34.04	0.01\\
34.05	0.01\\
34.06	0.01\\
34.07	0.01\\
34.08	0.01\\
34.09	0.01\\
34.1	0.01\\
34.11	0.01\\
34.12	0.01\\
34.13	0.01\\
34.14	0.01\\
34.15	0.01\\
34.16	0.01\\
34.17	0.01\\
34.18	0.01\\
34.19	0.01\\
34.2	0.01\\
34.21	0.01\\
34.22	0.01\\
34.23	0.01\\
34.24	0.01\\
34.25	0.01\\
34.26	0.01\\
34.27	0.01\\
34.28	0.01\\
34.29	0.01\\
34.3	0.01\\
34.31	0.01\\
34.32	0.01\\
34.33	0.01\\
34.34	0.01\\
34.35	0.01\\
34.36	0.01\\
34.37	0.01\\
34.38	0.01\\
34.39	0.01\\
34.4	0.01\\
34.41	0.01\\
34.42	0.01\\
34.43	0.01\\
34.44	0.01\\
34.45	0.01\\
34.46	0.01\\
34.47	0.01\\
34.48	0.01\\
34.49	0.01\\
34.5	0.01\\
34.51	0.01\\
34.52	0.01\\
34.53	0.01\\
34.54	0.01\\
34.55	0.01\\
34.56	0.01\\
34.57	0.01\\
34.58	0.01\\
34.59	0.01\\
34.6	0.01\\
34.61	0.01\\
34.62	0.01\\
34.63	0.01\\
34.64	0.01\\
34.65	0.01\\
34.66	0.01\\
34.67	0.01\\
34.68	0.01\\
34.69	0.01\\
34.7	0.01\\
34.71	0.01\\
34.72	0.01\\
34.73	0.01\\
34.74	0.01\\
34.75	0.01\\
34.76	0.01\\
34.77	0.01\\
34.78	0.01\\
34.79	0.01\\
34.8	0.01\\
34.81	0.01\\
34.82	0.01\\
34.83	0.01\\
34.84	0.01\\
34.85	0.01\\
34.86	0.01\\
34.87	0.01\\
34.88	0.01\\
34.89	0.01\\
34.9	0.01\\
34.91	0.01\\
34.92	0.01\\
34.93	0.01\\
34.94	0.01\\
34.95	0.01\\
34.96	0.01\\
34.97	0.01\\
34.98	0.01\\
34.99	0.01\\
35	0.01\\
35.01	0.01\\
35.02	0.01\\
35.03	0.01\\
35.04	0.01\\
35.05	0.01\\
35.06	0.01\\
35.07	0.01\\
35.08	0.01\\
35.09	0.01\\
35.1	0.01\\
35.11	0.01\\
35.12	0.01\\
35.13	0.01\\
35.14	0.01\\
35.15	0.01\\
35.16	0.01\\
35.17	0.01\\
35.18	0.01\\
35.19	0.01\\
35.2	0.01\\
35.21	0.01\\
35.22	0.01\\
35.23	0.01\\
35.24	0.01\\
35.25	0.01\\
35.26	0.01\\
35.27	0.01\\
35.28	0.01\\
35.29	0.01\\
35.3	0.01\\
35.31	0.01\\
35.32	0.01\\
35.33	0.01\\
35.34	0.01\\
35.35	0.01\\
35.36	0.01\\
35.37	0.01\\
35.38	0.01\\
35.39	0.01\\
35.4	0.01\\
35.41	0.01\\
35.42	0.01\\
35.43	0.01\\
35.44	0.01\\
35.45	0.01\\
35.46	0.01\\
35.47	0.01\\
35.48	0.01\\
35.49	0.01\\
35.5	0.01\\
35.51	0.01\\
35.52	0.01\\
35.53	0.01\\
35.54	0.01\\
35.55	0.01\\
35.56	0.01\\
35.57	0.01\\
35.58	0.01\\
35.59	0.01\\
35.6	0.01\\
35.61	0.01\\
35.62	0.01\\
35.63	0.01\\
35.64	0.01\\
35.65	0.01\\
35.66	0.01\\
35.67	0.01\\
35.68	0.01\\
35.69	0.01\\
35.7	0.01\\
35.71	0.01\\
35.72	0.01\\
35.73	0.01\\
35.74	0.01\\
35.75	0.01\\
35.76	0.01\\
35.77	0.01\\
35.78	0.01\\
35.79	0.01\\
35.8	0.01\\
35.81	0.01\\
35.82	0.01\\
35.83	0.01\\
35.84	0.01\\
35.85	0.01\\
35.86	0.01\\
35.87	0.01\\
35.88	0.01\\
35.89	0.01\\
35.9	0.01\\
35.91	0.01\\
35.92	0.01\\
35.93	0.01\\
35.94	0.01\\
35.95	0.01\\
35.96	0.01\\
35.97	0.01\\
35.98	0.01\\
35.99	0.01\\
36	0.01\\
36.01	0.01\\
36.02	0.01\\
36.03	0.01\\
36.04	0.01\\
36.05	0.01\\
36.06	0.01\\
36.07	0.01\\
36.08	0.01\\
36.09	0.01\\
36.1	0.01\\
36.11	0.01\\
36.12	0.01\\
36.13	0.01\\
36.14	0.01\\
36.15	0.01\\
36.16	0.01\\
36.17	0.01\\
36.18	0.01\\
36.19	0.01\\
36.2	0.01\\
36.21	0.01\\
36.22	0.01\\
36.23	0.01\\
36.24	0.01\\
36.25	0.01\\
36.26	0.01\\
36.27	0.01\\
36.28	0.01\\
36.29	0.01\\
36.3	0.01\\
36.31	0.01\\
36.32	0.01\\
36.33	0.01\\
36.34	0.01\\
36.35	0.01\\
36.36	0.01\\
36.37	0.01\\
36.38	0.01\\
36.39	0.01\\
36.4	0.01\\
36.41	0.01\\
36.42	0.01\\
36.43	0.01\\
36.44	0.01\\
36.45	0.01\\
36.46	0.01\\
36.47	0.01\\
36.48	0.01\\
36.49	0.01\\
36.5	0.01\\
36.51	0.01\\
36.52	0.01\\
36.53	0.01\\
36.54	0.01\\
36.55	0.01\\
36.56	0.01\\
36.57	0.01\\
36.58	0.01\\
36.59	0.01\\
36.6	0.01\\
36.61	0.01\\
36.62	0.01\\
36.63	0.01\\
36.64	0.01\\
36.65	0.01\\
36.66	0.01\\
36.67	0.01\\
36.68	0.01\\
36.69	0.01\\
36.7	0.01\\
36.71	0.01\\
36.72	0.01\\
36.73	0.01\\
36.74	0.01\\
36.75	0.01\\
36.76	0.01\\
36.77	0.01\\
36.78	0.01\\
36.79	0.01\\
36.8	0.01\\
36.81	0.01\\
36.82	0.01\\
36.83	0.01\\
36.84	0.01\\
36.85	0.01\\
36.86	0.01\\
36.87	0.01\\
36.88	0.01\\
36.89	0.01\\
36.9	0.01\\
36.91	0.01\\
36.92	0.01\\
36.93	0.01\\
36.94	0.01\\
36.95	0.01\\
36.96	0.01\\
36.97	0.01\\
36.98	0.01\\
36.99	0.01\\
37	0.01\\
37.01	0.01\\
37.02	0.01\\
37.03	0.01\\
37.04	0.01\\
37.05	0.01\\
37.06	0.01\\
37.07	0.01\\
37.08	0.01\\
37.09	0.01\\
37.1	0.01\\
37.11	0.01\\
37.12	0.01\\
37.13	0.01\\
37.14	0.01\\
37.15	0.01\\
37.16	0.01\\
37.17	0.01\\
37.18	0.01\\
37.19	0.01\\
37.2	0.01\\
37.21	0.01\\
37.22	0.01\\
37.23	0.01\\
37.24	0.01\\
37.25	0.01\\
37.26	0.01\\
37.27	0.01\\
37.28	0.01\\
37.29	0.01\\
37.3	0.01\\
37.31	0.01\\
37.32	0.01\\
37.33	0.01\\
37.34	0.01\\
37.35	0.01\\
37.36	0.01\\
37.37	0.01\\
37.38	0.01\\
37.39	0.01\\
37.4	0.01\\
37.41	0.01\\
37.42	0.01\\
37.43	0.01\\
37.44	0.01\\
37.45	0.01\\
37.46	0.01\\
37.47	0.01\\
37.48	0.01\\
37.49	0.01\\
37.5	0.01\\
37.51	0.01\\
37.52	0.01\\
37.53	0.01\\
37.54	0.01\\
37.55	0.01\\
37.56	0.01\\
37.57	0.01\\
37.58	0.01\\
37.59	0.01\\
37.6	0.01\\
37.61	0.01\\
37.62	0.01\\
37.63	0.01\\
37.64	0.01\\
37.65	0.01\\
37.66	0.01\\
37.67	0.01\\
37.68	0.01\\
37.69	0.01\\
37.7	0.01\\
37.71	0.01\\
37.72	0.01\\
37.73	0.01\\
37.74	0.01\\
37.75	0.01\\
37.76	0.01\\
37.77	0.01\\
37.78	0.01\\
37.79	0.01\\
37.8	0.01\\
37.81	0.01\\
37.82	0.01\\
37.83	0.01\\
37.84	0.01\\
37.85	0.01\\
37.86	0.01\\
37.87	0.01\\
37.88	0.01\\
37.89	0.01\\
37.9	0.01\\
37.91	0.01\\
37.92	0.01\\
37.93	0.01\\
37.94	0.01\\
37.95	0.01\\
37.96	0.01\\
37.97	0.01\\
37.98	0.01\\
37.99	0.01\\
38	0.01\\
38.01	0.01\\
38.02	0.01\\
38.03	0.01\\
38.04	0.01\\
38.05	0.01\\
38.06	0.01\\
38.07	0.01\\
38.08	0.01\\
38.09	0.01\\
38.1	0.01\\
38.11	0.01\\
38.12	0.01\\
38.13	0.01\\
38.14	0.01\\
38.15	0.01\\
38.16	0.01\\
38.17	0.01\\
38.18	0.01\\
38.19	0.01\\
38.2	0.01\\
38.21	0.01\\
38.22	0.01\\
38.23	0.01\\
38.24	0.01\\
38.25	0.01\\
38.26	0.01\\
38.27	0.01\\
38.28	0.01\\
38.29	0.01\\
38.3	0.01\\
38.31	0.01\\
38.32	0.01\\
38.33	0.01\\
38.34	0.01\\
38.35	0.01\\
38.36	0.01\\
38.37	0.01\\
38.38	0.01\\
38.39	0.01\\
38.4	0.01\\
38.41	0.01\\
38.42	0.01\\
38.43	0.01\\
38.44	0.01\\
38.45	0.01\\
38.46	0.01\\
38.47	0.01\\
38.48	0.01\\
38.49	0.01\\
38.5	0.01\\
38.51	0.01\\
38.52	0.01\\
38.53	0.01\\
38.54	0.01\\
38.55	0.01\\
38.56	0.01\\
38.57	0.01\\
38.58	0.01\\
38.59	0.01\\
38.6	0.01\\
38.61	0.01\\
38.62	0.01\\
38.63	0.01\\
38.64	0.01\\
38.65	0.01\\
38.66	0.01\\
38.67	0.01\\
38.68	0.01\\
38.69	0.01\\
38.7	0.01\\
38.71	0.01\\
38.72	0.01\\
38.73	0.01\\
38.74	0.01\\
38.75	0.01\\
38.76	0.01\\
38.77	0.01\\
38.78	0.01\\
38.79	0.01\\
38.8	0.01\\
38.81	0.01\\
38.82	0.01\\
38.83	0.01\\
38.84	0.01\\
38.85	0.01\\
38.86	0.01\\
38.87	0.01\\
38.88	0.01\\
38.89	0.01\\
38.9	0.01\\
38.91	0.01\\
38.92	0.01\\
38.93	0.01\\
38.94	0.01\\
38.95	0.01\\
38.96	0.01\\
38.97	0.01\\
38.98	0.01\\
38.99	0.01\\
39	0.01\\
39.01	0.01\\
39.02	0.01\\
39.03	0.01\\
39.04	0.01\\
39.05	0.01\\
39.06	0.01\\
39.07	0.01\\
39.08	0.01\\
39.09	0.01\\
39.1	0.01\\
39.11	0.01\\
39.12	0.01\\
39.13	0.01\\
39.14	0.01\\
39.15	0.01\\
39.16	0.01\\
39.17	0.01\\
39.18	0.01\\
39.19	0.01\\
39.2	0.01\\
39.21	0.01\\
39.22	0.01\\
39.23	0.01\\
39.24	0.01\\
39.25	0.01\\
39.26	0.01\\
39.27	0.01\\
39.28	0.01\\
39.29	0.01\\
39.3	0.01\\
39.31	0.01\\
39.32	0.01\\
39.33	0.01\\
39.34	0.01\\
39.35	0.01\\
39.36	0.01\\
39.37	0.01\\
39.38	0.01\\
39.39	0.01\\
39.4	0.01\\
39.41	0.01\\
39.42	0.01\\
39.43	0.01\\
39.44	0.01\\
39.45	0.01\\
39.46	0.01\\
39.47	0.01\\
39.48	0.01\\
39.49	0.01\\
39.5	0.01\\
39.51	0.01\\
39.52	0.01\\
39.53	0.01\\
39.54	0.01\\
39.55	0.01\\
39.56	0.01\\
39.57	0.01\\
39.58	0.01\\
39.59	0.01\\
39.6	0.01\\
39.61	0.01\\
39.62	0.01\\
39.63	0.01\\
39.64	0.01\\
39.65	0.01\\
39.66	0.01\\
39.67	0.01\\
39.68	0.01\\
39.69	0.01\\
39.7	0.01\\
39.71	0.01\\
39.72	0.01\\
39.73	0.01\\
39.74	0.01\\
39.75	0.01\\
39.76	0.01\\
39.77	0.01\\
39.78	0.01\\
39.79	0.01\\
39.8	0.01\\
39.81	0.01\\
39.82	0.01\\
39.83	0.01\\
39.84	0.01\\
39.85	0.01\\
39.86	0.01\\
39.87	0.01\\
39.88	0.01\\
39.89	0.01\\
39.9	0.01\\
39.91	0.01\\
39.92	0.01\\
39.93	0.01\\
39.94	0.01\\
39.95	0.01\\
39.96	0.01\\
39.97	0.01\\
39.98	0.01\\
39.99	0.01\\
40	0.01\\
40.01	0.01\\
};
\addplot [color=red,dashed,forget plot]
  table[row sep=crcr]{%
40.01	0.01\\
40.02	0.01\\
40.03	0.01\\
40.04	0.01\\
40.05	0.01\\
40.06	0.01\\
40.07	0.01\\
40.08	0.01\\
40.09	0.01\\
40.1	0.01\\
40.11	0.01\\
40.12	0.01\\
40.13	0.01\\
40.14	0.01\\
40.15	0.01\\
40.16	0.01\\
40.17	0.01\\
40.18	0.01\\
40.19	0.01\\
40.2	0.01\\
40.21	0.01\\
40.22	0.01\\
40.23	0.01\\
40.24	0.01\\
40.25	0.01\\
40.26	0.01\\
40.27	0.01\\
40.28	0.01\\
40.29	0.01\\
40.3	0.01\\
40.31	0.01\\
40.32	0.01\\
40.33	0.01\\
40.34	0.01\\
40.35	0.01\\
40.36	0.01\\
40.37	0.01\\
40.38	0.01\\
40.39	0.01\\
40.4	0.01\\
40.41	0.01\\
40.42	0.01\\
40.43	0.01\\
40.44	0.01\\
40.45	0.01\\
40.46	0.01\\
40.47	0.01\\
40.48	0.01\\
40.49	0.01\\
40.5	0.01\\
40.51	0.01\\
40.52	0.01\\
40.53	0.01\\
40.54	0.01\\
40.55	0.01\\
40.56	0.01\\
40.57	0.01\\
40.58	0.01\\
40.59	0.01\\
40.6	0.01\\
40.61	0.01\\
40.62	0.01\\
40.63	0.01\\
40.64	0.01\\
40.65	0.01\\
40.66	0.01\\
40.67	0.01\\
40.68	0.01\\
40.69	0.01\\
40.7	0.01\\
40.71	0.01\\
40.72	0.01\\
40.73	0.01\\
40.74	0.01\\
40.75	0.01\\
40.76	0.01\\
40.77	0.01\\
40.78	0.01\\
40.79	0.01\\
40.8	0.01\\
40.81	0.01\\
40.82	0.01\\
40.83	0.01\\
40.84	0.01\\
40.85	0.01\\
40.86	0.01\\
40.87	0.01\\
40.88	0.01\\
40.89	0.01\\
40.9	0.01\\
40.91	0.01\\
40.92	0.01\\
40.93	0.01\\
40.94	0.01\\
40.95	0.01\\
40.96	0.01\\
40.97	0.01\\
40.98	0.01\\
40.99	0.01\\
41	0.01\\
41.01	0.01\\
41.02	0.01\\
41.03	0.01\\
41.04	0.01\\
41.05	0.01\\
41.06	0.01\\
41.07	0.01\\
41.08	0.01\\
41.09	0.01\\
41.1	0.01\\
41.11	0.01\\
41.12	0.01\\
41.13	0.01\\
41.14	0.01\\
41.15	0.01\\
41.16	0.01\\
41.17	0.01\\
41.18	0.01\\
41.19	0.01\\
41.2	0.01\\
41.21	0.01\\
41.22	0.01\\
41.23	0.01\\
41.24	0.01\\
41.25	0.01\\
41.26	0.01\\
41.27	0.01\\
41.28	0.01\\
41.29	0.01\\
41.3	0.01\\
41.31	0.01\\
41.32	0.01\\
41.33	0.01\\
41.34	0.01\\
41.35	0.01\\
41.36	0.01\\
41.37	0.01\\
41.38	0.01\\
41.39	0.01\\
41.4	0.01\\
41.41	0.01\\
41.42	0.01\\
41.43	0.01\\
41.44	0.01\\
41.45	0.01\\
41.46	0.01\\
41.47	0.01\\
41.48	0.01\\
41.49	0.01\\
41.5	0.01\\
41.51	0.01\\
41.52	0.01\\
41.53	0.01\\
41.54	0.01\\
41.55	0.01\\
41.56	0.01\\
41.57	0.01\\
41.58	0.01\\
41.59	0.01\\
41.6	0.01\\
41.61	0.01\\
41.62	0.01\\
41.63	0.01\\
41.64	0.01\\
41.65	0.01\\
41.66	0.01\\
41.67	0.01\\
41.68	0.01\\
41.69	0.01\\
41.7	0.01\\
41.71	0.01\\
41.72	0.01\\
41.73	0.01\\
41.74	0.01\\
41.75	0.01\\
41.76	0.01\\
41.77	0.01\\
41.78	0.01\\
41.79	0.01\\
41.8	0.01\\
41.81	0.01\\
41.82	0.01\\
41.83	0.01\\
41.84	0.01\\
41.85	0.01\\
41.86	0.01\\
41.87	0.01\\
41.88	0.01\\
41.89	0.01\\
41.9	0.01\\
41.91	0.01\\
41.92	0.01\\
41.93	0.01\\
41.94	0.01\\
41.95	0.01\\
41.96	0.01\\
41.97	0.01\\
41.98	0.01\\
41.99	0.01\\
42	0.01\\
42.01	0.01\\
42.02	0.01\\
42.03	0.01\\
42.04	0.01\\
42.05	0.01\\
42.06	0.01\\
42.07	0.01\\
42.08	0.01\\
42.09	0.01\\
42.1	0.01\\
42.11	0.01\\
42.12	0.01\\
42.13	0.01\\
42.14	0.01\\
42.15	0.01\\
42.16	0.01\\
42.17	0.01\\
42.18	0.01\\
42.19	0.01\\
42.2	0.01\\
42.21	0.01\\
42.22	0.01\\
42.23	0.01\\
42.24	0.01\\
42.25	0.01\\
42.26	0.01\\
42.27	0.01\\
42.28	0.01\\
42.29	0.01\\
42.3	0.01\\
42.31	0.01\\
42.32	0.01\\
42.33	0.01\\
42.34	0.01\\
42.35	0.01\\
42.36	0.01\\
42.37	0.01\\
42.38	0.01\\
42.39	0.01\\
42.4	0.01\\
42.41	0.01\\
42.42	0.01\\
42.43	0.01\\
42.44	0.01\\
42.45	0.01\\
42.46	0.01\\
42.47	0.01\\
42.48	0.01\\
42.49	0.01\\
42.5	0.01\\
42.51	0.01\\
42.52	0.01\\
42.53	0.01\\
42.54	0.01\\
42.55	0.01\\
42.56	0.01\\
42.57	0.01\\
42.58	0.01\\
42.59	0.01\\
42.6	0.01\\
42.61	0.01\\
42.62	0.01\\
42.63	0.01\\
42.64	0.01\\
42.65	0.01\\
42.66	0.01\\
42.67	0.01\\
42.68	0.01\\
42.69	0.01\\
42.7	0.01\\
42.71	0.01\\
42.72	0.01\\
42.73	0.01\\
42.74	0.01\\
42.75	0.01\\
42.76	0.01\\
42.77	0.01\\
42.78	0.01\\
42.79	0.01\\
42.8	0.01\\
42.81	0.01\\
42.82	0.01\\
42.83	0.01\\
42.84	0.01\\
42.85	0.01\\
42.86	0.01\\
42.87	0.01\\
42.88	0.01\\
42.89	0.01\\
42.9	0.01\\
42.91	0.01\\
42.92	0.01\\
42.93	0.01\\
42.94	0.01\\
42.95	0.01\\
42.96	0.01\\
42.97	0.01\\
42.98	0.01\\
42.99	0.01\\
43	0.01\\
43.01	0.01\\
43.02	0.01\\
43.03	0.01\\
43.04	0.01\\
43.05	0.01\\
43.06	0.01\\
43.07	0.01\\
43.08	0.01\\
43.09	0.01\\
43.1	0.01\\
43.11	0.01\\
43.12	0.01\\
43.13	0.01\\
43.14	0.01\\
43.15	0.01\\
43.16	0.01\\
43.17	0.01\\
43.18	0.01\\
43.19	0.01\\
43.2	0.01\\
43.21	0.01\\
43.22	0.01\\
43.23	0.01\\
43.24	0.01\\
43.25	0.01\\
43.26	0.01\\
43.27	0.01\\
43.28	0.01\\
43.29	0.01\\
43.3	0.01\\
43.31	0.01\\
43.32	0.01\\
43.33	0.01\\
43.34	0.01\\
43.35	0.01\\
43.36	0.01\\
43.37	0.01\\
43.38	0.01\\
43.39	0.01\\
43.4	0.01\\
43.41	0.01\\
43.42	0.01\\
43.43	0.01\\
43.44	0.01\\
43.45	0.01\\
43.46	0.01\\
43.47	0.01\\
43.48	0.01\\
43.49	0.01\\
43.5	0.01\\
43.51	0.01\\
43.52	0.01\\
43.53	0.01\\
43.54	0.01\\
43.55	0.01\\
43.56	0.01\\
43.57	0.01\\
43.58	0.01\\
43.59	0.01\\
43.6	0.01\\
43.61	0.01\\
43.62	0.01\\
43.63	0.01\\
43.64	0.01\\
43.65	0.01\\
43.66	0.01\\
43.67	0.01\\
43.68	0.01\\
43.69	0.01\\
43.7	0.01\\
43.71	0.01\\
43.72	0.01\\
43.73	0.01\\
43.74	0.01\\
43.75	0.01\\
43.76	0.01\\
43.77	0.01\\
43.78	0.01\\
43.79	0.01\\
43.8	0.01\\
43.81	0.01\\
43.82	0.01\\
43.83	0.01\\
43.84	0.01\\
43.85	0.01\\
43.86	0.01\\
43.87	0.01\\
43.88	0.01\\
43.89	0.01\\
43.9	0.01\\
43.91	0.01\\
43.92	0.01\\
43.93	0.01\\
43.94	0.01\\
43.95	0.01\\
43.96	0.01\\
43.97	0.01\\
43.98	0.01\\
43.99	0.01\\
44	0.01\\
44.01	0.01\\
44.02	0.01\\
44.03	0.01\\
44.04	0.01\\
44.05	0.01\\
44.06	0.01\\
44.07	0.01\\
44.08	0.01\\
44.09	0.01\\
44.1	0.01\\
44.11	0.01\\
44.12	0.01\\
44.13	0.01\\
44.14	0.01\\
44.15	0.01\\
44.16	0.01\\
44.17	0.01\\
44.18	0.01\\
44.19	0.01\\
44.2	0.01\\
44.21	0.01\\
44.22	0.01\\
44.23	0.01\\
44.24	0.01\\
44.25	0.01\\
44.26	0.01\\
44.27	0.01\\
44.28	0.01\\
44.29	0.01\\
44.3	0.01\\
44.31	0.01\\
44.32	0.01\\
44.33	0.01\\
44.34	0.01\\
44.35	0.01\\
44.36	0.01\\
44.37	0.01\\
44.38	0.01\\
44.39	0.01\\
44.4	0.01\\
44.41	0.01\\
44.42	0.01\\
44.43	0.01\\
44.44	0.01\\
44.45	0.01\\
44.46	0.01\\
44.47	0.01\\
44.48	0.01\\
44.49	0.01\\
44.5	0.01\\
44.51	0.01\\
44.52	0.01\\
44.53	0.01\\
44.54	0.01\\
44.55	0.01\\
44.56	0.01\\
44.57	0.01\\
44.58	0.01\\
44.59	0.01\\
44.6	0.01\\
44.61	0.01\\
44.62	0.01\\
44.63	0.01\\
44.64	0.01\\
44.65	0.01\\
44.66	0.01\\
44.67	0.01\\
44.68	0.01\\
44.69	0.01\\
44.7	0.01\\
44.71	0.01\\
44.72	0.01\\
44.73	0.01\\
44.74	0.01\\
44.75	0.01\\
44.76	0.01\\
44.77	0.01\\
44.78	0.01\\
44.79	0.01\\
44.8	0.01\\
44.81	0.01\\
44.82	0.01\\
44.83	0.01\\
44.84	0.01\\
44.85	0.01\\
44.86	0.01\\
44.87	0.01\\
44.88	0.01\\
44.89	0.01\\
44.9	0.01\\
44.91	0.01\\
44.92	0.01\\
44.93	0.01\\
44.94	0.01\\
44.95	0.01\\
44.96	0.01\\
44.97	0.01\\
44.98	0.01\\
44.99	0.01\\
45	0.01\\
45.01	0.01\\
45.02	0.01\\
45.03	0.01\\
45.04	0.01\\
45.05	0.01\\
45.06	0.01\\
45.07	0.01\\
45.08	0.01\\
45.09	0.01\\
45.1	0.01\\
45.11	0.01\\
45.12	0.01\\
45.13	0.01\\
45.14	0.01\\
45.15	0.01\\
45.16	0.01\\
45.17	0.01\\
45.18	0.01\\
45.19	0.01\\
45.2	0.01\\
45.21	0.01\\
45.22	0.01\\
45.23	0.01\\
45.24	0.01\\
45.25	0.01\\
45.26	0.01\\
45.27	0.01\\
45.28	0.01\\
45.29	0.01\\
45.3	0.01\\
45.31	0.01\\
45.32	0.01\\
45.33	0.01\\
45.34	0.01\\
45.35	0.01\\
45.36	0.01\\
45.37	0.01\\
45.38	0.01\\
45.39	0.01\\
45.4	0.01\\
45.41	0.01\\
45.42	0.01\\
45.43	0.01\\
45.44	0.01\\
45.45	0.01\\
45.46	0.01\\
45.47	0.01\\
45.48	0.01\\
45.49	0.01\\
45.5	0.01\\
45.51	0.01\\
45.52	0.01\\
45.53	0.01\\
45.54	0.01\\
45.55	0.01\\
45.56	0.01\\
45.57	0.01\\
45.58	0.01\\
45.59	0.01\\
45.6	0.01\\
45.61	0.01\\
45.62	0.01\\
45.63	0.01\\
45.64	0.01\\
45.65	0.01\\
45.66	0.01\\
45.67	0.01\\
45.68	0.01\\
45.69	0.01\\
45.7	0.01\\
45.71	0.01\\
45.72	0.01\\
45.73	0.01\\
45.74	0.01\\
45.75	0.01\\
45.76	0.01\\
45.77	0.01\\
45.78	0.01\\
45.79	0.01\\
45.8	0.01\\
45.81	0.01\\
45.82	0.01\\
45.83	0.01\\
45.84	0.01\\
45.85	0.01\\
45.86	0.01\\
45.87	0.01\\
45.88	0.01\\
45.89	0.01\\
45.9	0.01\\
45.91	0.01\\
45.92	0.01\\
45.93	0.01\\
45.94	0.01\\
45.95	0.01\\
45.96	0.01\\
45.97	0.01\\
45.98	0.01\\
45.99	0.01\\
46	0.01\\
46.01	0.01\\
46.02	0.01\\
46.03	0.01\\
46.04	0.01\\
46.05	0.01\\
46.06	0.01\\
46.07	0.01\\
46.08	0.01\\
46.09	0.01\\
46.1	0.01\\
46.11	0.01\\
46.12	0.01\\
46.13	0.01\\
46.14	0.01\\
46.15	0.01\\
46.16	0.01\\
46.17	0.01\\
46.18	0.01\\
46.19	0.01\\
46.2	0.01\\
46.21	0.01\\
46.22	0.01\\
46.23	0.01\\
46.24	0.01\\
46.25	0.01\\
46.26	0.01\\
46.27	0.01\\
46.28	0.01\\
46.29	0.01\\
46.3	0.01\\
46.31	0.01\\
46.32	0.01\\
46.33	0.01\\
46.34	0.01\\
46.35	0.01\\
46.36	0.01\\
46.37	0.01\\
46.38	0.01\\
46.39	0.01\\
46.4	0.01\\
46.41	0.01\\
46.42	0.01\\
46.43	0.01\\
46.44	0.01\\
46.45	0.01\\
46.46	0.01\\
46.47	0.01\\
46.48	0.01\\
46.49	0.01\\
46.5	0.01\\
46.51	0.01\\
46.52	0.01\\
46.53	0.01\\
46.54	0.01\\
46.55	0.01\\
46.56	0.01\\
46.57	0.01\\
46.58	0.01\\
46.59	0.01\\
46.6	0.01\\
46.61	0.01\\
46.62	0.01\\
46.63	0.01\\
46.64	0.01\\
46.65	0.01\\
46.66	0.01\\
46.67	0.01\\
46.68	0.01\\
46.69	0.01\\
46.7	0.01\\
46.71	0.01\\
46.72	0.01\\
46.73	0.01\\
46.74	0.01\\
46.75	0.01\\
46.76	0.01\\
46.77	0.01\\
46.78	0.01\\
46.79	0.01\\
46.8	0.01\\
46.81	0.01\\
46.82	0.01\\
46.83	0.01\\
46.84	0.01\\
46.85	0.01\\
46.86	0.01\\
46.87	0.01\\
46.88	0.01\\
46.89	0.01\\
46.9	0.01\\
46.91	0.01\\
46.92	0.01\\
46.93	0.01\\
46.94	0.01\\
46.95	0.01\\
46.96	0.01\\
46.97	0.01\\
46.98	0.01\\
46.99	0.01\\
47	0.01\\
47.01	0.01\\
47.02	0.01\\
47.03	0.01\\
47.04	0.01\\
47.05	0.01\\
47.06	0.01\\
47.07	0.01\\
47.08	0.01\\
47.09	0.01\\
47.1	0.01\\
47.11	0.01\\
47.12	0.01\\
47.13	0.01\\
47.14	0.01\\
47.15	0.01\\
47.16	0.01\\
47.17	0.01\\
47.18	0.01\\
47.19	0.01\\
47.2	0.01\\
47.21	0.01\\
47.22	0.01\\
47.23	0.01\\
47.24	0.01\\
47.25	0.01\\
47.26	0.01\\
47.27	0.01\\
47.28	0.01\\
47.29	0.01\\
47.3	0.01\\
47.31	0.01\\
47.32	0.01\\
47.33	0.01\\
47.34	0.01\\
47.35	0.01\\
47.36	0.01\\
47.37	0.01\\
47.38	0.01\\
47.39	0.01\\
47.4	0.01\\
47.41	0.01\\
47.42	0.01\\
47.43	0.01\\
47.44	0.01\\
47.45	0.01\\
47.46	0.01\\
47.47	0.01\\
47.48	0.01\\
47.49	0.01\\
47.5	0.01\\
47.51	0.01\\
47.52	0.01\\
47.53	0.01\\
47.54	0.01\\
47.55	0.01\\
47.56	0.01\\
47.57	0.01\\
47.58	0.01\\
47.59	0.01\\
47.6	0.01\\
47.61	0.01\\
47.62	0.01\\
47.63	0.01\\
47.64	0.01\\
47.65	0.01\\
47.66	0.01\\
47.67	0.01\\
47.68	0.01\\
47.69	0.01\\
47.7	0.01\\
47.71	0.01\\
47.72	0.01\\
47.73	0.01\\
47.74	0.01\\
47.75	0.01\\
47.76	0.01\\
47.77	0.01\\
47.78	0.01\\
47.79	0.01\\
47.8	0.01\\
47.81	0.01\\
47.82	0.01\\
47.83	0.01\\
47.84	0.01\\
47.85	0.01\\
47.86	0.01\\
47.87	0.01\\
47.88	0.01\\
47.89	0.01\\
47.9	0.01\\
47.91	0.01\\
47.92	0.01\\
47.93	0.01\\
47.94	0.01\\
47.95	0.01\\
47.96	0.01\\
47.97	0.01\\
47.98	0.01\\
47.99	0.01\\
48	0.01\\
48.01	0.01\\
48.02	0.01\\
48.03	0.01\\
48.04	0.01\\
48.05	0.01\\
48.06	0.01\\
48.07	0.01\\
48.08	0.01\\
48.09	0.01\\
48.1	0.01\\
48.11	0.01\\
48.12	0.01\\
48.13	0.01\\
48.14	0.01\\
48.15	0.01\\
48.16	0.01\\
48.17	0.01\\
48.18	0.01\\
48.19	0.01\\
48.2	0.01\\
48.21	0.01\\
48.22	0.01\\
48.23	0.01\\
48.24	0.01\\
48.25	0.01\\
48.26	0.01\\
48.27	0.01\\
48.28	0.01\\
48.29	0.01\\
48.3	0.01\\
48.31	0.01\\
48.32	0.01\\
48.33	0.01\\
48.34	0.01\\
48.35	0.01\\
48.36	0.01\\
48.37	0.01\\
48.38	0.01\\
48.39	0.01\\
48.4	0.01\\
48.41	0.01\\
48.42	0.01\\
48.43	0.01\\
48.44	0.01\\
48.45	0.01\\
48.46	0.01\\
48.47	0.01\\
48.48	0.01\\
48.49	0.01\\
48.5	0.01\\
48.51	0.01\\
48.52	0.01\\
48.53	0.01\\
48.54	0.01\\
48.55	0.01\\
48.56	0.01\\
48.57	0.01\\
48.58	0.01\\
48.59	0.01\\
48.6	0.01\\
48.61	0.01\\
48.62	0.01\\
48.63	0.01\\
48.64	0.01\\
48.65	0.01\\
48.66	0.01\\
48.67	0.01\\
48.68	0.01\\
48.69	0.01\\
48.7	0.01\\
48.71	0.01\\
48.72	0.01\\
48.73	0.01\\
48.74	0.01\\
48.75	0.01\\
48.76	0.01\\
48.77	0.01\\
48.78	0.01\\
48.79	0.01\\
48.8	0.01\\
48.81	0.01\\
48.82	0.01\\
48.83	0.01\\
48.84	0.01\\
48.85	0.01\\
48.86	0.01\\
48.87	0.01\\
48.88	0.01\\
48.89	0.01\\
48.9	0.01\\
48.91	0.01\\
48.92	0.01\\
48.93	0.01\\
48.94	0.01\\
48.95	0.01\\
48.96	0.01\\
48.97	0.01\\
48.98	0.01\\
48.99	0.01\\
49	0.01\\
49.01	0.01\\
49.02	0.01\\
49.03	0.01\\
49.04	0.01\\
49.05	0.01\\
49.06	0.01\\
49.07	0.01\\
49.08	0.01\\
49.09	0.01\\
49.1	0.01\\
49.11	0.01\\
49.12	0.01\\
49.13	0.01\\
49.14	0.01\\
49.15	0.01\\
49.16	0.01\\
49.17	0.01\\
49.18	0.01\\
49.19	0.01\\
49.2	0.01\\
49.21	0.01\\
49.22	0.01\\
49.23	0.01\\
49.24	0.01\\
49.25	0.01\\
49.26	0.01\\
49.27	0.01\\
49.28	0.01\\
49.29	0.01\\
49.3	0.01\\
49.31	0.01\\
49.32	0.01\\
49.33	0.01\\
49.34	0.01\\
49.35	0.01\\
49.36	0.01\\
49.37	0.01\\
49.38	0.01\\
49.39	0.01\\
49.4	0.01\\
49.41	0.01\\
49.42	0.01\\
49.43	0.01\\
49.44	0.01\\
49.45	0.01\\
49.46	0.01\\
49.47	0.01\\
49.48	0.01\\
49.49	0.01\\
49.5	0.01\\
49.51	0.01\\
49.52	0.01\\
49.53	0.01\\
49.54	0.01\\
49.55	0.01\\
49.56	0.01\\
49.57	0.01\\
49.58	0.01\\
49.59	0.01\\
49.6	0.01\\
49.61	0.01\\
49.62	0.01\\
49.63	0.01\\
49.64	0.01\\
49.65	0.01\\
49.66	0.01\\
49.67	0.01\\
49.68	0.01\\
49.69	0.01\\
49.7	0.01\\
49.71	0.01\\
49.72	0.01\\
49.73	0.01\\
49.74	0.01\\
49.75	0.01\\
49.76	0.01\\
49.77	0.01\\
49.78	0.01\\
49.79	0.01\\
49.8	0.01\\
49.81	0.01\\
49.82	0.01\\
49.83	0.01\\
49.84	0.01\\
49.85	0.01\\
49.86	0.01\\
49.87	0.01\\
49.88	0.01\\
49.89	0.01\\
49.9	0.01\\
49.91	0.01\\
49.92	0.01\\
49.93	0.01\\
49.94	0.01\\
49.95	0.01\\
49.96	0.01\\
49.97	0.01\\
49.98	0.01\\
49.99	0.01\\
50	0.01\\
50.01	0.01\\
50.02	0.01\\
50.03	0.01\\
50.04	0.01\\
50.05	0.01\\
50.06	0.01\\
50.07	0.01\\
50.08	0.01\\
50.09	0.01\\
50.1	0.01\\
50.11	0.01\\
50.12	0.01\\
50.13	0.01\\
50.14	0.01\\
50.15	0.01\\
50.16	0.01\\
50.17	0.01\\
50.18	0.01\\
50.19	0.01\\
50.2	0.01\\
50.21	0.01\\
50.22	0.01\\
50.23	0.01\\
50.24	0.01\\
50.25	0.01\\
50.26	0.01\\
50.27	0.01\\
50.28	0.01\\
50.29	0.01\\
50.3	0.01\\
50.31	0.01\\
50.32	0.01\\
50.33	0.01\\
50.34	0.01\\
50.35	0.01\\
50.36	0.01\\
50.37	0.01\\
50.38	0.01\\
50.39	0.01\\
50.4	0.01\\
50.41	0.01\\
50.42	0.01\\
50.43	0.01\\
50.44	0.01\\
50.45	0.01\\
50.46	0.01\\
50.47	0.01\\
50.48	0.01\\
50.49	0.01\\
50.5	0.01\\
50.51	0.01\\
50.52	0.01\\
50.53	0.01\\
50.54	0.01\\
50.55	0.01\\
50.56	0.01\\
50.57	0.01\\
50.58	0.01\\
50.59	0.01\\
50.6	0.01\\
50.61	0.01\\
50.62	0.01\\
50.63	0.01\\
50.64	0.01\\
50.65	0.01\\
50.66	0.01\\
50.67	0.01\\
50.68	0.01\\
50.69	0.01\\
50.7	0.01\\
50.71	0.01\\
50.72	0.01\\
50.73	0.01\\
50.74	0.01\\
50.75	0.01\\
50.76	0.01\\
50.77	0.01\\
50.78	0.01\\
50.79	0.01\\
50.8	0.01\\
50.81	0.01\\
50.82	0.01\\
50.83	0.01\\
50.84	0.01\\
50.85	0.01\\
50.86	0.01\\
50.87	0.01\\
50.88	0.01\\
50.89	0.01\\
50.9	0.01\\
50.91	0.01\\
50.92	0.01\\
50.93	0.01\\
50.94	0.01\\
50.95	0.01\\
50.96	0.01\\
50.97	0.01\\
50.98	0.01\\
50.99	0.01\\
51	0.01\\
51.01	0.01\\
51.02	0.01\\
51.03	0.01\\
51.04	0.01\\
51.05	0.01\\
51.06	0.01\\
51.07	0.01\\
51.08	0.01\\
51.09	0.01\\
51.1	0.01\\
51.11	0.01\\
51.12	0.01\\
51.13	0.01\\
51.14	0.01\\
51.15	0.01\\
51.16	0.01\\
51.17	0.01\\
51.18	0.01\\
51.19	0.01\\
51.2	0.01\\
51.21	0.01\\
51.22	0.01\\
51.23	0.01\\
51.24	0.01\\
51.25	0.01\\
51.26	0.01\\
51.27	0.01\\
51.28	0.01\\
51.29	0.01\\
51.3	0.01\\
51.31	0.01\\
51.32	0.01\\
51.33	0.01\\
51.34	0.01\\
51.35	0.01\\
51.36	0.01\\
51.37	0.01\\
51.38	0.01\\
51.39	0.01\\
51.4	0.01\\
51.41	0.01\\
51.42	0.01\\
51.43	0.01\\
51.44	0.01\\
51.45	0.01\\
51.46	0.01\\
51.47	0.01\\
51.48	0.01\\
51.49	0.01\\
51.5	0.01\\
51.51	0.01\\
51.52	0.01\\
51.53	0.01\\
51.54	0.01\\
51.55	0.01\\
51.56	0.01\\
51.57	0.01\\
51.58	0.01\\
51.59	0.01\\
51.6	0.01\\
51.61	0.01\\
51.62	0.01\\
51.63	0.01\\
51.64	0.01\\
51.65	0.01\\
51.66	0.01\\
51.67	0.01\\
51.68	0.01\\
51.69	0.01\\
51.7	0.01\\
51.71	0.01\\
51.72	0.01\\
51.73	0.01\\
51.74	0.01\\
51.75	0.01\\
51.76	0.01\\
51.77	0.01\\
51.78	0.01\\
51.79	0.01\\
51.8	0.01\\
51.81	0.01\\
51.82	0.01\\
51.83	0.01\\
51.84	0.01\\
51.85	0.01\\
51.86	0.01\\
51.87	0.01\\
51.88	0.01\\
51.89	0.01\\
51.9	0.01\\
51.91	0.01\\
51.92	0.01\\
51.93	0.01\\
51.94	0.01\\
51.95	0.01\\
51.96	0.01\\
51.97	0.01\\
51.98	0.01\\
51.99	0.01\\
52	0.01\\
52.01	0.01\\
52.02	0.01\\
52.03	0.01\\
52.04	0.01\\
52.05	0.01\\
52.06	0.01\\
52.07	0.01\\
52.08	0.01\\
52.09	0.01\\
52.1	0.01\\
52.11	0.01\\
52.12	0.01\\
52.13	0.01\\
52.14	0.01\\
52.15	0.01\\
52.16	0.01\\
52.17	0.01\\
52.18	0.01\\
52.19	0.01\\
52.2	0.01\\
52.21	0.01\\
52.22	0.01\\
52.23	0.01\\
52.24	0.01\\
52.25	0.01\\
52.26	0.01\\
52.27	0.01\\
52.28	0.01\\
52.29	0.01\\
52.3	0.01\\
52.31	0.01\\
52.32	0.01\\
52.33	0.01\\
52.34	0.01\\
52.35	0.01\\
52.36	0.01\\
52.37	0.01\\
52.38	0.01\\
52.39	0.01\\
52.4	0.01\\
52.41	0.01\\
52.42	0.01\\
52.43	0.01\\
52.44	0.01\\
52.45	0.01\\
52.46	0.01\\
52.47	0.01\\
52.48	0.01\\
52.49	0.01\\
52.5	0.01\\
52.51	0.01\\
52.52	0.01\\
52.53	0.01\\
52.54	0.01\\
52.55	0.01\\
52.56	0.01\\
52.57	0.01\\
52.58	0.01\\
52.59	0.01\\
52.6	0.01\\
52.61	0.01\\
52.62	0.01\\
52.63	0.01\\
52.64	0.01\\
52.65	0.01\\
52.66	0.01\\
52.67	0.01\\
52.68	0.01\\
52.69	0.01\\
52.7	0.01\\
52.71	0.01\\
52.72	0.01\\
52.73	0.01\\
52.74	0.01\\
52.75	0.01\\
52.76	0.01\\
52.77	0.01\\
52.78	0.01\\
52.79	0.01\\
52.8	0.01\\
52.81	0.01\\
52.82	0.01\\
52.83	0.01\\
52.84	0.01\\
52.85	0.01\\
52.86	0.01\\
52.87	0.01\\
52.88	0.01\\
52.89	0.01\\
52.9	0.01\\
52.91	0.01\\
52.92	0.01\\
52.93	0.01\\
52.94	0.01\\
52.95	0.01\\
52.96	0.01\\
52.97	0.01\\
52.98	0.01\\
52.99	0.01\\
53	0.01\\
53.01	0.01\\
53.02	0.01\\
53.03	0.01\\
53.04	0.01\\
53.05	0.01\\
53.06	0.01\\
53.07	0.01\\
53.08	0.01\\
53.09	0.01\\
53.1	0.01\\
53.11	0.01\\
53.12	0.01\\
53.13	0.01\\
53.14	0.01\\
53.15	0.01\\
53.16	0.01\\
53.17	0.01\\
53.18	0.01\\
53.19	0.01\\
53.2	0.01\\
53.21	0.01\\
53.22	0.01\\
53.23	0.01\\
53.24	0.01\\
53.25	0.01\\
53.26	0.01\\
53.27	0.01\\
53.28	0.01\\
53.29	0.01\\
53.3	0.01\\
53.31	0.01\\
53.32	0.01\\
53.33	0.01\\
53.34	0.01\\
53.35	0.01\\
53.36	0.01\\
53.37	0.01\\
53.38	0.01\\
53.39	0.01\\
53.4	0.01\\
53.41	0.01\\
53.42	0.01\\
53.43	0.01\\
53.44	0.01\\
53.45	0.01\\
53.46	0.01\\
53.47	0.01\\
53.48	0.01\\
53.49	0.01\\
53.5	0.01\\
53.51	0.01\\
53.52	0.01\\
53.53	0.01\\
53.54	0.01\\
53.55	0.01\\
53.56	0.01\\
53.57	0.01\\
53.58	0.01\\
53.59	0.01\\
53.6	0.01\\
53.61	0.01\\
53.62	0.01\\
53.63	0.01\\
53.64	0.01\\
53.65	0.01\\
53.66	0.01\\
53.67	0.01\\
53.68	0.01\\
53.69	0.01\\
53.7	0.01\\
53.71	0.01\\
53.72	0.01\\
53.73	0.01\\
53.74	0.01\\
53.75	0.01\\
53.76	0.01\\
53.77	0.01\\
53.78	0.01\\
53.79	0.01\\
53.8	0.01\\
53.81	0.01\\
53.82	0.01\\
53.83	0.01\\
53.84	0.01\\
53.85	0.01\\
53.86	0.01\\
53.87	0.01\\
53.88	0.01\\
53.89	0.01\\
53.9	0.01\\
53.91	0.01\\
53.92	0.01\\
53.93	0.01\\
53.94	0.01\\
53.95	0.01\\
53.96	0.01\\
53.97	0.01\\
53.98	0.01\\
53.99	0.01\\
54	0.01\\
54.01	0.01\\
54.02	0.01\\
54.03	0.01\\
54.04	0.01\\
54.05	0.01\\
54.06	0.01\\
54.07	0.01\\
54.08	0.01\\
54.09	0.01\\
54.1	0.01\\
54.11	0.01\\
54.12	0.01\\
54.13	0.01\\
54.14	0.01\\
54.15	0.01\\
54.16	0.01\\
54.17	0.01\\
54.18	0.01\\
54.19	0.01\\
54.2	0.01\\
54.21	0.01\\
54.22	0.01\\
54.23	0.01\\
54.24	0.01\\
54.25	0.01\\
54.26	0.01\\
54.27	0.01\\
54.28	0.01\\
54.29	0.01\\
54.3	0.01\\
54.31	0.01\\
54.32	0.01\\
54.33	0.01\\
54.34	0.01\\
54.35	0.01\\
54.36	0.01\\
54.37	0.01\\
54.38	0.01\\
54.39	0.01\\
54.4	0.01\\
54.41	0.01\\
54.42	0.01\\
54.43	0.01\\
54.44	0.01\\
54.45	0.01\\
54.46	0.01\\
54.47	0.01\\
54.48	0.01\\
54.49	0.01\\
54.5	0.01\\
54.51	0.01\\
54.52	0.01\\
54.53	0.01\\
54.54	0.01\\
54.55	0.01\\
54.56	0.01\\
54.57	0.01\\
54.58	0.01\\
54.59	0.01\\
54.6	0.01\\
54.61	0.01\\
54.62	0.01\\
54.63	0.01\\
54.64	0.01\\
54.65	0.01\\
54.66	0.01\\
54.67	0.01\\
54.68	0.01\\
54.69	0.01\\
54.7	0.01\\
54.71	0.01\\
54.72	0.01\\
54.73	0.01\\
54.74	0.01\\
54.75	0.01\\
54.76	0.01\\
54.77	0.01\\
54.78	0.01\\
54.79	0.01\\
54.8	0.01\\
54.81	0.01\\
54.82	0.01\\
54.83	0.01\\
54.84	0.01\\
54.85	0.01\\
54.86	0.01\\
54.87	0.01\\
54.88	0.01\\
54.89	0.01\\
54.9	0.01\\
54.91	0.01\\
54.92	0.01\\
54.93	0.01\\
54.94	0.01\\
54.95	0.01\\
54.96	0.01\\
54.97	0.01\\
54.98	0.01\\
54.99	0.01\\
55	0.01\\
55.01	0.01\\
55.02	0.01\\
55.03	0.01\\
55.04	0.01\\
55.05	0.01\\
55.06	0.01\\
55.07	0.01\\
55.08	0.01\\
55.09	0.01\\
55.1	0.01\\
55.11	0.01\\
55.12	0.01\\
55.13	0.01\\
55.14	0.01\\
55.15	0.01\\
55.16	0.01\\
55.17	0.01\\
55.18	0.01\\
55.19	0.01\\
55.2	0.01\\
55.21	0.01\\
55.22	0.01\\
55.23	0.01\\
55.24	0.01\\
55.25	0.01\\
55.26	0.01\\
55.27	0.01\\
55.28	0.01\\
55.29	0.01\\
55.3	0.01\\
55.31	0.01\\
55.32	0.01\\
55.33	0.01\\
55.34	0.01\\
55.35	0.01\\
55.36	0.01\\
55.37	0.01\\
55.38	0.01\\
55.39	0.01\\
55.4	0.01\\
55.41	0.01\\
55.42	0.01\\
55.43	0.01\\
55.44	0.01\\
55.45	0.01\\
55.46	0.01\\
55.47	0.01\\
55.48	0.01\\
55.49	0.01\\
55.5	0.01\\
55.51	0.01\\
55.52	0.01\\
55.53	0.01\\
55.54	0.01\\
55.55	0.01\\
55.56	0.01\\
55.57	0.01\\
55.58	0.01\\
55.59	0.01\\
55.6	0.01\\
55.61	0.01\\
55.62	0.01\\
55.63	0.01\\
55.64	0.01\\
55.65	0.01\\
55.66	0.01\\
55.67	0.01\\
55.68	0.01\\
55.69	0.01\\
55.7	0.01\\
55.71	0.01\\
55.72	0.01\\
55.73	0.01\\
55.74	0.01\\
55.75	0.01\\
55.76	0.01\\
55.77	0.01\\
55.78	0.01\\
55.79	0.01\\
55.8	0.01\\
55.81	0.01\\
55.82	0.01\\
55.83	0.01\\
55.84	0.01\\
55.85	0.01\\
55.86	0.01\\
55.87	0.01\\
55.88	0.01\\
55.89	0.01\\
55.9	0.01\\
55.91	0.01\\
55.92	0.01\\
55.93	0.01\\
55.94	0.01\\
55.95	0.01\\
55.96	0.01\\
55.97	0.01\\
55.98	0.01\\
55.99	0.01\\
56	0.01\\
56.01	0.01\\
56.02	0.01\\
56.03	0.01\\
56.04	0.01\\
56.05	0.01\\
56.06	0.01\\
56.07	0.01\\
56.08	0.01\\
56.09	0.01\\
56.1	0.01\\
56.11	0.01\\
56.12	0.01\\
56.13	0.01\\
56.14	0.01\\
56.15	0.01\\
56.16	0.01\\
56.17	0.01\\
56.18	0.01\\
56.19	0.01\\
56.2	0.01\\
56.21	0.01\\
56.22	0.01\\
56.23	0.01\\
56.24	0.01\\
56.25	0.01\\
56.26	0.01\\
56.27	0.01\\
56.28	0.01\\
56.29	0.01\\
56.3	0.01\\
56.31	0.01\\
56.32	0.01\\
56.33	0.01\\
56.34	0.01\\
56.35	0.01\\
56.36	0.01\\
56.37	0.01\\
56.38	0.01\\
56.39	0.01\\
56.4	0.01\\
56.41	0.01\\
56.42	0.01\\
56.43	0.01\\
56.44	0.01\\
56.45	0.01\\
56.46	0.01\\
56.47	0.01\\
56.48	0.01\\
56.49	0.01\\
56.5	0.01\\
56.51	0.01\\
56.52	0.01\\
56.53	0.01\\
56.54	0.01\\
56.55	0.01\\
56.56	0.01\\
56.57	0.01\\
56.58	0.01\\
56.59	0.01\\
56.6	0.01\\
56.61	0.01\\
56.62	0.01\\
56.63	0.01\\
56.64	0.01\\
56.65	0.01\\
56.66	0.01\\
56.67	0.01\\
56.68	0.01\\
56.69	0.01\\
56.7	0.01\\
56.71	0.01\\
56.72	0.01\\
56.73	0.01\\
56.74	0.01\\
56.75	0.01\\
56.76	0.01\\
56.77	0.01\\
56.78	0.01\\
56.79	0.01\\
56.8	0.01\\
56.81	0.01\\
56.82	0.01\\
56.83	0.01\\
56.84	0.01\\
56.85	0.01\\
56.86	0.01\\
56.87	0.01\\
56.88	0.01\\
56.89	0.01\\
56.9	0.01\\
56.91	0.01\\
56.92	0.01\\
56.93	0.01\\
56.94	0.01\\
56.95	0.01\\
56.96	0.01\\
56.97	0.01\\
56.98	0.01\\
56.99	0.01\\
57	0.01\\
57.01	0.01\\
57.02	0.01\\
57.03	0.01\\
57.04	0.01\\
57.05	0.01\\
57.06	0.01\\
57.07	0.01\\
57.08	0.01\\
57.09	0.01\\
57.1	0.01\\
57.11	0.01\\
57.12	0.01\\
57.13	0.01\\
57.14	0.01\\
57.15	0.01\\
57.16	0.01\\
57.17	0.01\\
57.18	0.01\\
57.19	0.01\\
57.2	0.01\\
57.21	0.01\\
57.22	0.01\\
57.23	0.01\\
57.24	0.01\\
57.25	0.01\\
57.26	0.01\\
57.27	0.01\\
57.28	0.01\\
57.29	0.01\\
57.3	0.01\\
57.31	0.01\\
57.32	0.01\\
57.33	0.01\\
57.34	0.01\\
57.35	0.01\\
57.36	0.01\\
57.37	0.01\\
57.38	0.01\\
57.39	0.01\\
57.4	0.01\\
57.41	0.01\\
57.42	0.01\\
57.43	0.01\\
57.44	0.01\\
57.45	0.01\\
57.46	0.01\\
57.47	0.01\\
57.48	0.01\\
57.49	0.01\\
57.5	0.01\\
57.51	0.01\\
57.52	0.01\\
57.53	0.01\\
57.54	0.01\\
57.55	0.01\\
57.56	0.01\\
57.57	0.01\\
57.58	0.01\\
57.59	0.01\\
57.6	0.01\\
57.61	0.01\\
57.62	0.01\\
57.63	0.01\\
57.64	0.01\\
57.65	0.01\\
57.66	0.01\\
57.67	0.01\\
57.68	0.01\\
57.69	0.01\\
57.7	0.01\\
57.71	0.01\\
57.72	0.01\\
57.73	0.01\\
57.74	0.01\\
57.75	0.01\\
57.76	0.01\\
57.77	0.01\\
57.78	0.01\\
57.79	0.01\\
57.8	0.01\\
57.81	0.01\\
57.82	0.01\\
57.83	0.01\\
57.84	0.01\\
57.85	0.01\\
57.86	0.01\\
57.87	0.01\\
57.88	0.01\\
57.89	0.01\\
57.9	0.01\\
57.91	0.01\\
57.92	0.01\\
57.93	0.01\\
57.94	0.01\\
57.95	0.01\\
57.96	0.01\\
57.97	0.01\\
57.98	0.01\\
57.99	0.01\\
58	0.01\\
58.01	0.01\\
58.02	0.01\\
58.03	0.01\\
58.04	0.01\\
58.05	0.01\\
58.06	0.01\\
58.07	0.01\\
58.08	0.01\\
58.09	0.01\\
58.1	0.01\\
58.11	0.01\\
58.12	0.01\\
58.13	0.01\\
58.14	0.01\\
58.15	0.01\\
58.16	0.01\\
58.17	0.01\\
58.18	0.01\\
58.19	0.01\\
58.2	0.01\\
58.21	0.01\\
58.22	0.01\\
58.23	0.01\\
58.24	0.01\\
58.25	0.01\\
58.26	0.01\\
58.27	0.01\\
58.28	0.01\\
58.29	0.01\\
58.3	0.01\\
58.31	0.01\\
58.32	0.01\\
58.33	0.01\\
58.34	0.01\\
58.35	0.01\\
58.36	0.01\\
58.37	0.01\\
58.38	0.01\\
58.39	0.01\\
58.4	0.01\\
58.41	0.01\\
58.42	0.01\\
58.43	0.01\\
58.44	0.01\\
58.45	0.01\\
58.46	0.01\\
58.47	0.01\\
58.48	0.01\\
58.49	0.01\\
58.5	0.01\\
58.51	0.01\\
58.52	0.01\\
58.53	0.01\\
58.54	0.01\\
58.55	0.01\\
58.56	0.01\\
58.57	0.01\\
58.58	0.01\\
58.59	0.01\\
58.6	0.01\\
58.61	0.01\\
58.62	0.01\\
58.63	0.01\\
58.64	0.01\\
58.65	0.01\\
58.66	0.01\\
58.67	0.01\\
58.68	0.01\\
58.69	0.01\\
58.7	0.01\\
58.71	0.01\\
58.72	0.01\\
58.73	0.01\\
58.74	0.01\\
58.75	0.01\\
58.76	0.01\\
58.77	0.01\\
58.78	0.01\\
58.79	0.01\\
58.8	0.01\\
58.81	0.01\\
58.82	0.01\\
58.83	0.01\\
58.84	0.01\\
58.85	0.01\\
58.86	0.01\\
58.87	0.01\\
58.88	0.01\\
58.89	0.01\\
58.9	0.01\\
58.91	0.01\\
58.92	0.01\\
58.93	0.01\\
58.94	0.01\\
58.95	0.01\\
58.96	0.01\\
58.97	0.01\\
58.98	0.01\\
58.99	0.01\\
59	0.01\\
59.01	0.01\\
59.02	0.01\\
59.03	0.01\\
59.04	0.01\\
59.05	0.01\\
59.06	0.01\\
59.07	0.01\\
59.08	0.01\\
59.09	0.01\\
59.1	0.01\\
59.11	0.01\\
59.12	0.01\\
59.13	0.01\\
59.14	0.01\\
59.15	0.01\\
59.16	0.01\\
59.17	0.01\\
59.18	0.01\\
59.19	0.01\\
59.2	0.01\\
59.21	0.01\\
59.22	0.01\\
59.23	0.01\\
59.24	0.01\\
59.25	0.01\\
59.26	0.01\\
59.27	0.01\\
59.28	0.01\\
59.29	0.01\\
59.3	0.01\\
59.31	0.01\\
59.32	0.01\\
59.33	0.01\\
59.34	0.01\\
59.35	0.01\\
59.36	0.01\\
59.37	0.01\\
59.38	0.01\\
59.39	0.01\\
59.4	0.01\\
59.41	0.01\\
59.42	0.01\\
59.43	0.01\\
59.44	0.01\\
59.45	0.01\\
59.46	0.01\\
59.47	0.01\\
59.48	0.01\\
59.49	0.01\\
59.5	0.01\\
59.51	0.01\\
59.52	0.01\\
59.53	0.01\\
59.54	0.01\\
59.55	0.01\\
59.56	0.01\\
59.57	0.01\\
59.58	0.01\\
59.59	0.01\\
59.6	0.01\\
59.61	0.01\\
59.62	0.01\\
59.63	0.01\\
59.64	0.01\\
59.65	0.01\\
59.66	0.01\\
59.67	0.01\\
59.68	0.01\\
59.69	0.01\\
59.7	0.01\\
59.71	0.01\\
59.72	0.01\\
59.73	0.01\\
59.74	0.01\\
59.75	0.01\\
59.76	0.01\\
59.77	0.01\\
59.78	0.01\\
59.79	0.01\\
59.8	0.01\\
59.81	0.01\\
59.82	0.01\\
59.83	0.01\\
59.84	0.01\\
59.85	0.01\\
59.86	0.01\\
59.87	0.01\\
59.88	0.01\\
59.89	0.01\\
59.9	0.01\\
59.91	0.01\\
59.92	0.01\\
59.93	0.01\\
59.94	0.01\\
59.95	0.01\\
59.96	0.01\\
59.97	0.01\\
59.98	0.01\\
59.99	0.01\\
60	0.01\\
60.01	0.01\\
60.02	0.01\\
60.03	0.01\\
60.04	0.01\\
60.05	0.01\\
60.06	0.01\\
60.07	0.01\\
60.08	0.01\\
60.09	0.01\\
60.1	0.01\\
60.11	0.01\\
60.12	0.01\\
60.13	0.01\\
60.14	0.01\\
60.15	0.01\\
60.16	0.01\\
60.17	0.01\\
60.18	0.01\\
60.19	0.01\\
60.2	0.01\\
60.21	0.01\\
60.22	0.01\\
60.23	0.01\\
60.24	0.01\\
60.25	0.01\\
60.26	0.01\\
60.27	0.01\\
60.28	0.01\\
60.29	0.01\\
60.3	0.01\\
60.31	0.01\\
60.32	0.01\\
60.33	0.01\\
60.34	0.01\\
60.35	0.01\\
60.36	0.01\\
60.37	0.01\\
60.38	0.01\\
60.39	0.01\\
60.4	0.01\\
60.41	0.01\\
60.42	0.01\\
60.43	0.01\\
60.44	0.01\\
60.45	0.01\\
60.46	0.01\\
60.47	0.01\\
60.48	0.01\\
60.49	0.01\\
60.5	0.01\\
60.51	0.01\\
60.52	0.01\\
60.53	0.01\\
60.54	0.01\\
60.55	0.01\\
60.56	0.01\\
60.57	0.01\\
60.58	0.01\\
60.59	0.01\\
60.6	0.01\\
60.61	0.01\\
60.62	0.01\\
60.63	0.01\\
60.64	0.01\\
60.65	0.01\\
60.66	0.01\\
60.67	0.01\\
60.68	0.01\\
60.69	0.01\\
60.7	0.01\\
60.71	0.01\\
60.72	0.01\\
60.73	0.01\\
60.74	0.01\\
60.75	0.01\\
60.76	0.01\\
60.77	0.01\\
60.78	0.01\\
60.79	0.01\\
60.8	0.01\\
60.81	0.01\\
60.82	0.01\\
60.83	0.01\\
60.84	0.01\\
60.85	0.01\\
60.86	0.01\\
60.87	0.01\\
60.88	0.01\\
60.89	0.01\\
60.9	0.01\\
60.91	0.01\\
60.92	0.01\\
60.93	0.01\\
60.94	0.01\\
60.95	0.01\\
60.96	0.01\\
60.97	0.01\\
60.98	0.01\\
60.99	0.01\\
61	0.01\\
61.01	0.01\\
61.02	0.01\\
61.03	0.01\\
61.04	0.01\\
61.05	0.01\\
61.06	0.01\\
61.07	0.01\\
61.08	0.01\\
61.09	0.01\\
61.1	0.01\\
61.11	0.01\\
61.12	0.01\\
61.13	0.01\\
61.14	0.01\\
61.15	0.01\\
61.16	0.01\\
61.17	0.01\\
61.18	0.01\\
61.19	0.01\\
61.2	0.01\\
61.21	0.01\\
61.22	0.01\\
61.23	0.01\\
61.24	0.01\\
61.25	0.01\\
61.26	0.01\\
61.27	0.01\\
61.28	0.01\\
61.29	0.01\\
61.3	0.01\\
61.31	0.01\\
61.32	0.01\\
61.33	0.01\\
61.34	0.01\\
61.35	0.01\\
61.36	0.01\\
61.37	0.01\\
61.38	0.01\\
61.39	0.01\\
61.4	0.01\\
61.41	0.01\\
61.42	0.01\\
61.43	0.01\\
61.44	0.01\\
61.45	0.01\\
61.46	0.01\\
61.47	0.01\\
61.48	0.01\\
61.49	0.01\\
61.5	0.01\\
61.51	0.01\\
61.52	0.01\\
61.53	0.01\\
61.54	0.01\\
61.55	0.01\\
61.56	0.01\\
61.57	0.01\\
61.58	0.01\\
61.59	0.01\\
61.6	0.01\\
61.61	0.01\\
61.62	0.01\\
61.63	0.01\\
61.64	0.01\\
61.65	0.01\\
61.66	0.01\\
61.67	0.01\\
61.68	0.01\\
61.69	0.01\\
61.7	0.01\\
61.71	0.01\\
61.72	0.01\\
61.73	0.01\\
61.74	0.01\\
61.75	0.01\\
61.76	0.01\\
61.77	0.01\\
61.78	0.01\\
61.79	0.01\\
61.8	0.01\\
61.81	0.01\\
61.82	0.01\\
61.83	0.01\\
61.84	0.01\\
61.85	0.01\\
61.86	0.01\\
61.87	0.01\\
61.88	0.01\\
61.89	0.01\\
61.9	0.01\\
61.91	0.01\\
61.92	0.01\\
61.93	0.01\\
61.94	0.01\\
61.95	0.01\\
61.96	0.01\\
61.97	0.01\\
61.98	0.01\\
61.99	0.01\\
62	0.01\\
62.01	0.01\\
62.02	0.01\\
62.03	0.01\\
62.04	0.01\\
62.05	0.01\\
62.06	0.01\\
62.07	0.01\\
62.08	0.01\\
62.09	0.01\\
62.1	0.01\\
62.11	0.01\\
62.12	0.01\\
62.13	0.01\\
62.14	0.01\\
62.15	0.01\\
62.16	0.01\\
62.17	0.01\\
62.18	0.01\\
62.19	0.01\\
62.2	0.01\\
62.21	0.01\\
62.22	0.01\\
62.23	0.01\\
62.24	0.01\\
62.25	0.01\\
62.26	0.01\\
62.27	0.01\\
62.28	0.01\\
62.29	0.01\\
62.3	0.01\\
62.31	0.01\\
62.32	0.01\\
62.33	0.01\\
62.34	0.01\\
62.35	0.01\\
62.36	0.01\\
62.37	0.01\\
62.38	0.01\\
62.39	0.01\\
62.4	0.01\\
62.41	0.01\\
62.42	0.01\\
62.43	0.01\\
62.44	0.01\\
62.45	0.01\\
62.46	0.01\\
62.47	0.01\\
62.48	0.01\\
62.49	0.01\\
62.5	0.01\\
62.51	0.01\\
62.52	0.01\\
62.53	0.01\\
62.54	0.01\\
62.55	0.01\\
62.56	0.01\\
62.57	0.01\\
62.58	0.01\\
62.59	0.01\\
62.6	0.01\\
62.61	0.01\\
62.62	0.01\\
62.63	0.01\\
62.64	0.01\\
62.65	0.01\\
62.66	0.01\\
62.67	0.01\\
62.68	0.01\\
62.69	0.01\\
62.7	0.01\\
62.71	0.01\\
62.72	0.01\\
62.73	0.01\\
62.74	0.01\\
62.75	0.01\\
62.76	0.01\\
62.77	0.01\\
62.78	0.01\\
62.79	0.01\\
62.8	0.01\\
62.81	0.01\\
62.82	0.01\\
62.83	0.01\\
62.84	0.01\\
62.85	0.01\\
62.86	0.01\\
62.87	0.01\\
62.88	0.01\\
62.89	0.01\\
62.9	0.01\\
62.91	0.01\\
62.92	0.01\\
62.93	0.01\\
62.94	0.01\\
62.95	0.01\\
62.96	0.01\\
62.97	0.01\\
62.98	0.01\\
62.99	0.01\\
63	0.01\\
63.01	0.01\\
63.02	0.01\\
63.03	0.01\\
63.04	0.01\\
63.05	0.01\\
63.06	0.01\\
63.07	0.01\\
63.08	0.01\\
63.09	0.01\\
63.1	0.01\\
63.11	0.01\\
63.12	0.01\\
63.13	0.01\\
63.14	0.01\\
63.15	0.01\\
63.16	0.01\\
63.17	0.01\\
63.18	0.01\\
63.19	0.01\\
63.2	0.01\\
63.21	0.01\\
63.22	0.01\\
63.23	0.01\\
63.24	0.01\\
63.25	0.01\\
63.26	0.01\\
63.27	0.01\\
63.28	0.01\\
63.29	0.01\\
63.3	0.01\\
63.31	0.01\\
63.32	0.01\\
63.33	0.01\\
63.34	0.01\\
63.35	0.01\\
63.36	0.01\\
63.37	0.01\\
63.38	0.01\\
63.39	0.01\\
63.4	0.01\\
63.41	0.01\\
63.42	0.01\\
63.43	0.01\\
63.44	0.01\\
63.45	0.01\\
63.46	0.01\\
63.47	0.01\\
63.48	0.01\\
63.49	0.01\\
63.5	0.01\\
63.51	0.01\\
63.52	0.01\\
63.53	0.01\\
63.54	0.01\\
63.55	0.01\\
63.56	0.01\\
63.57	0.01\\
63.58	0.01\\
63.59	0.01\\
63.6	0.01\\
63.61	0.01\\
63.62	0.01\\
63.63	0.01\\
63.64	0.01\\
63.65	0.01\\
63.66	0.01\\
63.67	0.01\\
63.68	0.01\\
63.69	0.01\\
63.7	0.01\\
63.71	0.01\\
63.72	0.01\\
63.73	0.01\\
63.74	0.01\\
63.75	0.01\\
63.76	0.01\\
63.77	0.01\\
63.78	0.01\\
63.79	0.01\\
63.8	0.01\\
63.81	0.01\\
63.82	0.01\\
63.83	0.01\\
63.84	0.01\\
63.85	0.01\\
63.86	0.01\\
63.87	0.01\\
63.88	0.01\\
63.89	0.01\\
63.9	0.01\\
63.91	0.01\\
63.92	0.01\\
63.93	0.01\\
63.94	0.01\\
63.95	0.01\\
63.96	0.01\\
63.97	0.01\\
63.98	0.01\\
63.99	0.01\\
64	0.01\\
64.01	0.01\\
64.02	0.01\\
64.03	0.01\\
64.04	0.01\\
64.05	0.01\\
64.06	0.01\\
64.07	0.01\\
64.08	0.01\\
64.09	0.01\\
64.1	0.01\\
64.11	0.01\\
64.12	0.01\\
64.13	0.01\\
64.14	0.01\\
64.15	0.01\\
64.16	0.01\\
64.17	0.01\\
64.18	0.01\\
64.19	0.01\\
64.2	0.01\\
64.21	0.01\\
64.22	0.01\\
64.23	0.01\\
64.24	0.01\\
64.25	0.01\\
64.26	0.01\\
64.27	0.01\\
64.28	0.01\\
64.29	0.01\\
64.3	0.01\\
64.31	0.01\\
64.32	0.01\\
64.33	0.01\\
64.34	0.01\\
64.35	0.01\\
64.36	0.01\\
64.37	0.01\\
64.38	0.01\\
64.39	0.01\\
64.4	0.01\\
64.41	0.01\\
64.42	0.01\\
64.43	0.01\\
64.44	0.01\\
64.45	0.01\\
64.46	0.01\\
64.47	0.01\\
64.48	0.01\\
64.49	0.01\\
64.5	0.01\\
64.51	0.01\\
64.52	0.01\\
64.53	0.01\\
64.54	0.01\\
64.55	0.01\\
64.56	0.01\\
64.57	0.01\\
64.58	0.01\\
64.59	0.01\\
64.6	0.01\\
64.61	0.01\\
64.62	0.01\\
64.63	0.01\\
64.64	0.01\\
64.65	0.01\\
64.66	0.01\\
64.67	0.01\\
64.68	0.01\\
64.69	0.01\\
64.7	0.01\\
64.71	0.01\\
64.72	0.01\\
64.73	0.01\\
64.74	0.01\\
64.75	0.01\\
64.76	0.01\\
64.77	0.01\\
64.78	0.01\\
64.79	0.01\\
64.8	0.01\\
64.81	0.01\\
64.82	0.01\\
64.83	0.01\\
64.84	0.01\\
64.85	0.01\\
64.86	0.01\\
64.87	0.01\\
64.88	0.01\\
64.89	0.01\\
64.9	0.01\\
64.91	0.01\\
64.92	0.01\\
64.93	0.01\\
64.94	0.01\\
64.95	0.01\\
64.96	0.01\\
64.97	0.01\\
64.98	0.01\\
64.99	0.01\\
65	0.01\\
65.01	0.01\\
65.02	0.01\\
65.03	0.01\\
65.04	0.01\\
65.05	0.01\\
65.06	0.01\\
65.07	0.01\\
65.08	0.01\\
65.09	0.01\\
65.1	0.01\\
65.11	0.01\\
65.12	0.01\\
65.13	0.01\\
65.14	0.01\\
65.15	0.01\\
65.16	0.01\\
65.17	0.01\\
65.18	0.01\\
65.19	0.01\\
65.2	0.01\\
65.21	0.01\\
65.22	0.01\\
65.23	0.01\\
65.24	0.01\\
65.25	0.01\\
65.26	0.01\\
65.27	0.01\\
65.28	0.01\\
65.29	0.01\\
65.3	0.01\\
65.31	0.01\\
65.32	0.01\\
65.33	0.01\\
65.34	0.01\\
65.35	0.01\\
65.36	0.01\\
65.37	0.01\\
65.38	0.01\\
65.39	0.01\\
65.4	0.01\\
65.41	0.01\\
65.42	0.01\\
65.43	0.01\\
65.44	0.01\\
65.45	0.01\\
65.46	0.01\\
65.47	0.01\\
65.48	0.01\\
65.49	0.01\\
65.5	0.01\\
65.51	0.01\\
65.52	0.01\\
65.53	0.01\\
65.54	0.01\\
65.55	0.01\\
65.56	0.01\\
65.57	0.01\\
65.58	0.01\\
65.59	0.01\\
65.6	0.01\\
65.61	0.01\\
65.62	0.01\\
65.63	0.01\\
65.64	0.01\\
65.65	0.01\\
65.66	0.01\\
65.67	0.01\\
65.68	0.01\\
65.69	0.01\\
65.7	0.01\\
65.71	0.01\\
65.72	0.01\\
65.73	0.01\\
65.74	0.01\\
65.75	0.01\\
65.76	0.01\\
65.77	0.01\\
65.78	0.01\\
65.79	0.01\\
65.8	0.01\\
65.81	0.01\\
65.82	0.01\\
65.83	0.01\\
65.84	0.01\\
65.85	0.01\\
65.86	0.01\\
65.87	0.01\\
65.88	0.01\\
65.89	0.01\\
65.9	0.01\\
65.91	0.01\\
65.92	0.01\\
65.93	0.01\\
65.94	0.01\\
65.95	0.01\\
65.96	0.01\\
65.97	0.01\\
65.98	0.01\\
65.99	0.01\\
66	0.01\\
66.01	0.01\\
66.02	0.01\\
66.03	0.01\\
66.04	0.01\\
66.05	0.01\\
66.06	0.01\\
66.07	0.01\\
66.08	0.01\\
66.09	0.01\\
66.1	0.01\\
66.11	0.01\\
66.12	0.01\\
66.13	0.01\\
66.14	0.01\\
66.15	0.01\\
66.16	0.01\\
66.17	0.01\\
66.18	0.01\\
66.19	0.01\\
66.2	0.01\\
66.21	0.01\\
66.22	0.01\\
66.23	0.01\\
66.24	0.01\\
66.25	0.01\\
66.26	0.01\\
66.27	0.01\\
66.28	0.01\\
66.29	0.01\\
66.3	0.01\\
66.31	0.01\\
66.32	0.01\\
66.33	0.01\\
66.34	0.01\\
66.35	0.01\\
66.36	0.01\\
66.37	0.01\\
66.38	0.01\\
66.39	0.01\\
66.4	0.01\\
66.41	0.01\\
66.42	0.01\\
66.43	0.01\\
66.44	0.01\\
66.45	0.01\\
66.46	0.01\\
66.47	0.01\\
66.48	0.01\\
66.49	0.01\\
66.5	0.01\\
66.51	0.01\\
66.52	0.01\\
66.53	0.01\\
66.54	0.01\\
66.55	0.01\\
66.56	0.01\\
66.57	0.01\\
66.58	0.01\\
66.59	0.01\\
66.6	0.01\\
66.61	0.01\\
66.62	0.01\\
66.63	0.01\\
66.64	0.01\\
66.65	0.01\\
66.66	0.01\\
66.67	0.01\\
66.68	0.01\\
66.69	0.01\\
66.7	0.01\\
66.71	0.01\\
66.72	0.01\\
66.73	0.01\\
66.74	0.01\\
66.75	0.01\\
66.76	0.01\\
66.77	0.01\\
66.78	0.01\\
66.79	0.01\\
66.8	0.01\\
66.81	0.01\\
66.82	0.01\\
66.83	0.01\\
66.84	0.01\\
66.85	0.01\\
66.86	0.01\\
66.87	0.01\\
66.88	0.01\\
66.89	0.01\\
66.9	0.01\\
66.91	0.01\\
66.92	0.01\\
66.93	0.01\\
66.94	0.01\\
66.95	0.01\\
66.96	0.01\\
66.97	0.01\\
66.98	0.01\\
66.99	0.01\\
67	0.01\\
67.01	0.01\\
67.02	0.01\\
67.03	0.01\\
67.04	0.01\\
67.05	0.01\\
67.06	0.01\\
67.07	0.01\\
67.08	0.01\\
67.09	0.01\\
67.1	0.01\\
67.11	0.01\\
67.12	0.01\\
67.13	0.01\\
67.14	0.01\\
67.15	0.01\\
67.16	0.01\\
67.17	0.01\\
67.18	0.01\\
67.19	0.01\\
67.2	0.01\\
67.21	0.01\\
67.22	0.01\\
67.23	0.01\\
67.24	0.01\\
67.25	0.01\\
67.26	0.01\\
67.27	0.01\\
67.28	0.01\\
67.29	0.01\\
67.3	0.01\\
67.31	0.01\\
67.32	0.01\\
67.33	0.01\\
67.34	0.01\\
67.35	0.01\\
67.36	0.01\\
67.37	0.01\\
67.38	0.01\\
67.39	0.01\\
67.4	0.01\\
67.41	0.01\\
67.42	0.01\\
67.43	0.01\\
67.44	0.01\\
67.45	0.01\\
67.46	0.01\\
67.47	0.01\\
67.48	0.01\\
67.49	0.01\\
67.5	0.01\\
67.51	0.01\\
67.52	0.01\\
67.53	0.01\\
67.54	0.01\\
67.55	0.01\\
67.56	0.01\\
67.57	0.01\\
67.58	0.01\\
67.59	0.01\\
67.6	0.01\\
67.61	0.01\\
67.62	0.01\\
67.63	0.01\\
67.64	0.01\\
67.65	0.01\\
67.66	0.01\\
67.67	0.01\\
67.68	0.01\\
67.69	0.01\\
67.7	0.01\\
67.71	0.01\\
67.72	0.01\\
67.73	0.01\\
67.74	0.01\\
67.75	0.01\\
67.76	0.01\\
67.77	0.01\\
67.78	0.01\\
67.79	0.01\\
67.8	0.01\\
67.81	0.01\\
67.82	0.01\\
67.83	0.01\\
67.84	0.01\\
67.85	0.01\\
67.86	0.01\\
67.87	0.01\\
67.88	0.01\\
67.89	0.01\\
67.9	0.01\\
67.91	0.01\\
67.92	0.01\\
67.93	0.01\\
67.94	0.01\\
67.95	0.01\\
67.96	0.01\\
67.97	0.01\\
67.98	0.01\\
67.99	0.01\\
68	0.01\\
68.01	0.01\\
68.02	0.01\\
68.03	0.01\\
68.04	0.01\\
68.05	0.01\\
68.06	0.01\\
68.07	0.01\\
68.08	0.01\\
68.09	0.01\\
68.1	0.01\\
68.11	0.01\\
68.12	0.01\\
68.13	0.01\\
68.14	0.01\\
68.15	0.01\\
68.16	0.01\\
68.17	0.01\\
68.18	0.01\\
68.19	0.01\\
68.2	0.01\\
68.21	0.01\\
68.22	0.01\\
68.23	0.01\\
68.24	0.01\\
68.25	0.01\\
68.26	0.01\\
68.27	0.01\\
68.28	0.01\\
68.29	0.01\\
68.3	0.01\\
68.31	0.01\\
68.32	0.01\\
68.33	0.01\\
68.34	0.01\\
68.35	0.01\\
68.36	0.01\\
68.37	0.01\\
68.38	0.01\\
68.39	0.01\\
68.4	0.01\\
68.41	0.01\\
68.42	0.01\\
68.43	0.01\\
68.44	0.01\\
68.45	0.01\\
68.46	0.01\\
68.47	0.01\\
68.48	0.01\\
68.49	0.01\\
68.5	0.01\\
68.51	0.01\\
68.52	0.01\\
68.53	0.01\\
68.54	0.01\\
68.55	0.01\\
68.56	0.01\\
68.57	0.01\\
68.58	0.01\\
68.59	0.01\\
68.6	0.01\\
68.61	0.01\\
68.62	0.01\\
68.63	0.01\\
68.64	0.01\\
68.65	0.01\\
68.66	0.01\\
68.67	0.01\\
68.68	0.01\\
68.69	0.01\\
68.7	0.01\\
68.71	0.01\\
68.72	0.01\\
68.73	0.01\\
68.74	0.01\\
68.75	0.01\\
68.76	0.01\\
68.77	0.01\\
68.78	0.01\\
68.79	0.01\\
68.8	0.01\\
68.81	0.01\\
68.82	0.01\\
68.83	0.01\\
68.84	0.01\\
68.85	0.01\\
68.86	0.01\\
68.87	0.01\\
68.88	0.01\\
68.89	0.01\\
68.9	0.01\\
68.91	0.01\\
68.92	0.01\\
68.93	0.01\\
68.94	0.01\\
68.95	0.01\\
68.96	0.01\\
68.97	0.01\\
68.98	0.01\\
68.99	0.01\\
69	0.01\\
69.01	0.01\\
69.02	0.01\\
69.03	0.01\\
69.04	0.01\\
69.05	0.01\\
69.06	0.01\\
69.07	0.01\\
69.08	0.01\\
69.09	0.01\\
69.1	0.01\\
69.11	0.01\\
69.12	0.01\\
69.13	0.01\\
69.14	0.01\\
69.15	0.01\\
69.16	0.01\\
69.17	0.01\\
69.18	0.01\\
69.19	0.01\\
69.2	0.01\\
69.21	0.01\\
69.22	0.01\\
69.23	0.01\\
69.24	0.01\\
69.25	0.01\\
69.26	0.01\\
69.27	0.01\\
69.28	0.01\\
69.29	0.01\\
69.3	0.01\\
69.31	0.01\\
69.32	0.01\\
69.33	0.01\\
69.34	0.01\\
69.35	0.01\\
69.36	0.01\\
69.37	0.01\\
69.38	0.01\\
69.39	0.01\\
69.4	0.01\\
69.41	0.01\\
69.42	0.01\\
69.43	0.01\\
69.44	0.01\\
69.45	0.01\\
69.46	0.01\\
69.47	0.01\\
69.48	0.01\\
69.49	0.01\\
69.5	0.01\\
69.51	0.01\\
69.52	0.01\\
69.53	0.01\\
69.54	0.01\\
69.55	0.01\\
69.56	0.01\\
69.57	0.01\\
69.58	0.01\\
69.59	0.01\\
69.6	0.01\\
69.61	0.01\\
69.62	0.01\\
69.63	0.01\\
69.64	0.01\\
69.65	0.01\\
69.66	0.01\\
69.67	0.01\\
69.68	0.01\\
69.69	0.01\\
69.7	0.01\\
69.71	0.01\\
69.72	0.01\\
69.73	0.01\\
69.74	0.01\\
69.75	0.01\\
69.76	0.01\\
69.77	0.01\\
69.78	0.01\\
69.79	0.01\\
69.8	0.01\\
69.81	0.01\\
69.82	0.01\\
69.83	0.01\\
69.84	0.01\\
69.85	0.01\\
69.86	0.01\\
69.87	0.01\\
69.88	0.01\\
69.89	0.01\\
69.9	0.01\\
69.91	0.01\\
69.92	0.01\\
69.93	0.01\\
69.94	0.01\\
69.95	0.01\\
69.96	0.01\\
69.97	0.01\\
69.98	0.01\\
69.99	0.01\\
70	0.01\\
70.01	0.01\\
70.02	0.01\\
70.03	0.01\\
70.04	0.01\\
70.05	0.01\\
70.06	0.01\\
70.07	0.01\\
70.08	0.01\\
70.09	0.01\\
70.1	0.01\\
70.11	0.01\\
70.12	0.01\\
70.13	0.01\\
70.14	0.01\\
70.15	0.01\\
70.16	0.01\\
70.17	0.01\\
70.18	0.01\\
70.19	0.01\\
70.2	0.01\\
70.21	0.01\\
70.22	0.01\\
70.23	0.01\\
70.24	0.01\\
70.25	0.01\\
70.26	0.01\\
70.27	0.01\\
70.28	0.01\\
70.29	0.01\\
70.3	0.01\\
70.31	0.01\\
70.32	0.01\\
70.33	0.01\\
70.34	0.01\\
70.35	0.01\\
70.36	0.01\\
70.37	0.01\\
70.38	0.01\\
70.39	0.01\\
70.4	0.01\\
70.41	0.01\\
70.42	0.01\\
70.43	0.01\\
70.44	0.01\\
70.45	0.01\\
70.46	0.01\\
70.47	0.01\\
70.48	0.01\\
70.49	0.01\\
70.5	0.01\\
70.51	0.01\\
70.52	0.01\\
70.53	0.01\\
70.54	0.01\\
70.55	0.01\\
70.56	0.01\\
70.57	0.01\\
70.58	0.01\\
70.59	0.01\\
70.6	0.01\\
70.61	0.01\\
70.62	0.01\\
70.63	0.01\\
70.64	0.01\\
70.65	0.01\\
70.66	0.01\\
70.67	0.01\\
70.68	0.01\\
70.69	0.01\\
70.7	0.01\\
70.71	0.01\\
70.72	0.01\\
70.73	0.01\\
70.74	0.01\\
70.75	0.01\\
70.76	0.01\\
70.77	0.01\\
70.78	0.01\\
70.79	0.01\\
70.8	0.01\\
70.81	0.01\\
70.82	0.01\\
70.83	0.01\\
70.84	0.01\\
70.85	0.01\\
70.86	0.01\\
70.87	0.01\\
70.88	0.01\\
70.89	0.01\\
70.9	0.01\\
70.91	0.01\\
70.92	0.01\\
70.93	0.01\\
70.94	0.01\\
70.95	0.01\\
70.96	0.01\\
70.97	0.01\\
70.98	0.01\\
70.99	0.01\\
71	0.01\\
71.01	0.01\\
71.02	0.01\\
71.03	0.01\\
71.04	0.01\\
71.05	0.01\\
71.06	0.01\\
71.07	0.01\\
71.08	0.01\\
71.09	0.01\\
71.1	0.01\\
71.11	0.01\\
71.12	0.01\\
71.13	0.01\\
71.14	0.01\\
71.15	0.01\\
71.16	0.01\\
71.17	0.01\\
71.18	0.01\\
71.19	0.01\\
71.2	0.01\\
71.21	0.01\\
71.22	0.01\\
71.23	0.01\\
71.24	0.01\\
71.25	0.01\\
71.26	0.01\\
71.27	0.01\\
71.28	0.01\\
71.29	0.01\\
71.3	0.01\\
71.31	0.01\\
71.32	0.01\\
71.33	0.01\\
71.34	0.01\\
71.35	0.01\\
71.36	0.01\\
71.37	0.01\\
71.38	0.01\\
71.39	0.01\\
71.4	0.01\\
71.41	0.01\\
71.42	0.01\\
71.43	0.01\\
71.44	0.01\\
71.45	0.01\\
71.46	0.01\\
71.47	0.01\\
71.48	0.01\\
71.49	0.01\\
71.5	0.01\\
71.51	0.01\\
71.52	0.01\\
71.53	0.01\\
71.54	0.01\\
71.55	0.01\\
71.56	0.01\\
71.57	0.01\\
71.58	0.01\\
71.59	0.01\\
71.6	0.01\\
71.61	0.01\\
71.62	0.01\\
71.63	0.01\\
71.64	0.01\\
71.65	0.01\\
71.66	0.01\\
71.67	0.01\\
71.68	0.01\\
71.69	0.01\\
71.7	0.01\\
71.71	0.01\\
71.72	0.01\\
71.73	0.01\\
71.74	0.01\\
71.75	0.01\\
71.76	0.01\\
71.77	0.01\\
71.78	0.01\\
71.79	0.01\\
71.8	0.01\\
71.81	0.01\\
71.82	0.01\\
71.83	0.01\\
71.84	0.01\\
71.85	0.01\\
71.86	0.01\\
71.87	0.01\\
71.88	0.01\\
71.89	0.01\\
71.9	0.01\\
71.91	0.01\\
71.92	0.01\\
71.93	0.01\\
71.94	0.01\\
71.95	0.01\\
71.96	0.01\\
71.97	0.01\\
71.98	0.01\\
71.99	0.01\\
72	0.01\\
72.01	0.01\\
72.02	0.01\\
72.03	0.01\\
72.04	0.01\\
72.05	0.01\\
72.06	0.01\\
72.07	0.01\\
72.08	0.01\\
72.09	0.01\\
72.1	0.01\\
72.11	0.01\\
72.12	0.01\\
72.13	0.01\\
72.14	0.01\\
72.15	0.01\\
72.16	0.01\\
72.17	0.01\\
72.18	0.01\\
72.19	0.01\\
72.2	0.01\\
72.21	0.01\\
72.22	0.01\\
72.23	0.01\\
72.24	0.01\\
72.25	0.01\\
72.26	0.01\\
72.27	0.01\\
72.28	0.01\\
72.29	0.01\\
72.3	0.01\\
72.31	0.01\\
72.32	0.01\\
72.33	0.01\\
72.34	0.01\\
72.35	0.01\\
72.36	0.01\\
72.37	0.01\\
72.38	0.01\\
72.39	0.01\\
72.4	0.01\\
72.41	0.01\\
72.42	0.01\\
72.43	0.01\\
72.44	0.01\\
72.45	0.01\\
72.46	0.01\\
72.47	0.01\\
72.48	0.01\\
72.49	0.01\\
72.5	0.01\\
72.51	0.01\\
72.52	0.01\\
72.53	0.01\\
72.54	0.01\\
72.55	0.01\\
72.56	0.01\\
72.57	0.01\\
72.58	0.01\\
72.59	0.01\\
72.6	0.01\\
72.61	0.01\\
72.62	0.01\\
72.63	0.01\\
72.64	0.01\\
72.65	0.01\\
72.66	0.01\\
72.67	0.01\\
72.68	0.01\\
72.69	0.01\\
72.7	0.01\\
72.71	0.01\\
72.72	0.01\\
72.73	0.01\\
72.74	0.01\\
72.75	0.01\\
72.76	0.01\\
72.77	0.01\\
72.78	0.01\\
72.79	0.01\\
72.8	0.01\\
72.81	0.01\\
72.82	0.01\\
72.83	0.01\\
72.84	0.01\\
72.85	0.01\\
72.86	0.01\\
72.87	0.01\\
72.88	0.01\\
72.89	0.01\\
72.9	0.01\\
72.91	0.01\\
72.92	0.01\\
72.93	0.01\\
72.94	0.01\\
72.95	0.01\\
72.96	0.01\\
72.97	0.01\\
72.98	0.01\\
72.99	0.01\\
73	0.01\\
73.01	0.01\\
73.02	0.01\\
73.03	0.01\\
73.04	0.01\\
73.05	0.01\\
73.06	0.01\\
73.07	0.01\\
73.08	0.01\\
73.09	0.01\\
73.1	0.01\\
73.11	0.01\\
73.12	0.01\\
73.13	0.01\\
73.14	0.01\\
73.15	0.01\\
73.16	0.01\\
73.17	0.01\\
73.18	0.01\\
73.19	0.01\\
73.2	0.01\\
73.21	0.01\\
73.22	0.01\\
73.23	0.01\\
73.24	0.01\\
73.25	0.01\\
73.26	0.01\\
73.27	0.01\\
73.28	0.01\\
73.29	0.01\\
73.3	0.01\\
73.31	0.01\\
73.32	0.01\\
73.33	0.01\\
73.34	0.01\\
73.35	0.01\\
73.36	0.01\\
73.37	0.01\\
73.38	0.01\\
73.39	0.01\\
73.4	0.01\\
73.41	0.01\\
73.42	0.01\\
73.43	0.01\\
73.44	0.01\\
73.45	0.01\\
73.46	0.01\\
73.47	0.01\\
73.48	0.01\\
73.49	0.01\\
73.5	0.01\\
73.51	0.01\\
73.52	0.01\\
73.53	0.01\\
73.54	0.01\\
73.55	0.01\\
73.56	0.01\\
73.57	0.01\\
73.58	0.01\\
73.59	0.01\\
73.6	0.01\\
73.61	0.01\\
73.62	0.01\\
73.63	0.01\\
73.64	0.01\\
73.65	0.01\\
73.66	0.01\\
73.67	0.01\\
73.68	0.01\\
73.69	0.01\\
73.7	0.01\\
73.71	0.01\\
73.72	0.01\\
73.73	0.01\\
73.74	0.01\\
73.75	0.01\\
73.76	0.01\\
73.77	0.01\\
73.78	0.01\\
73.79	0.01\\
73.8	0.01\\
73.81	0.01\\
73.82	0.01\\
73.83	0.01\\
73.84	0.01\\
73.85	0.01\\
73.86	0.01\\
73.87	0.01\\
73.88	0.01\\
73.89	0.01\\
73.9	0.01\\
73.91	0.01\\
73.92	0.01\\
73.93	0.01\\
73.94	0.01\\
73.95	0.01\\
73.96	0.01\\
73.97	0.01\\
73.98	0.01\\
73.99	0.01\\
74	0.01\\
74.01	0.01\\
74.02	0.01\\
74.03	0.01\\
74.04	0.01\\
74.05	0.01\\
74.06	0.01\\
74.07	0.01\\
74.08	0.01\\
74.09	0.01\\
74.1	0.01\\
74.11	0.01\\
74.12	0.01\\
74.13	0.01\\
74.14	0.01\\
74.15	0.01\\
74.16	0.01\\
74.17	0.01\\
74.18	0.01\\
74.19	0.01\\
74.2	0.01\\
74.21	0.01\\
74.22	0.01\\
74.23	0.01\\
74.24	0.01\\
74.25	0.01\\
74.26	0.01\\
74.27	0.01\\
74.28	0.01\\
74.29	0.01\\
74.3	0.01\\
74.31	0.01\\
74.32	0.01\\
74.33	0.01\\
74.34	0.01\\
74.35	0.01\\
74.36	0.01\\
74.37	0.01\\
74.38	0.01\\
74.39	0.01\\
74.4	0.01\\
74.41	0.01\\
74.42	0.01\\
74.43	0.01\\
74.44	0.01\\
74.45	0.01\\
74.46	0.01\\
74.47	0.01\\
74.48	0.01\\
74.49	0.01\\
74.5	0.01\\
74.51	0.01\\
74.52	0.01\\
74.53	0.01\\
74.54	0.01\\
74.55	0.01\\
74.56	0.01\\
74.57	0.01\\
74.58	0.01\\
74.59	0.01\\
74.6	0.01\\
74.61	0.01\\
74.62	0.01\\
74.63	0.01\\
74.64	0.01\\
74.65	0.01\\
74.66	0.01\\
74.67	0.01\\
74.68	0.01\\
74.69	0.01\\
74.7	0.01\\
74.71	0.01\\
74.72	0.01\\
74.73	0.01\\
74.74	0.01\\
74.75	0.01\\
74.76	0.01\\
74.77	0.01\\
74.78	0.01\\
74.79	0.01\\
74.8	0.01\\
74.81	0.01\\
74.82	0.01\\
74.83	0.01\\
74.84	0.01\\
74.85	0.01\\
74.86	0.01\\
74.87	0.01\\
74.88	0.01\\
74.89	0.01\\
74.9	0.01\\
74.91	0.01\\
74.92	0.01\\
74.93	0.01\\
74.94	0.01\\
74.95	0.01\\
74.96	0.01\\
74.97	0.01\\
74.98	0.01\\
74.99	0.01\\
75	0.01\\
75.01	0.01\\
75.02	0.01\\
75.03	0.01\\
75.04	0.01\\
75.05	0.01\\
75.06	0.01\\
75.07	0.01\\
75.08	0.01\\
75.09	0.01\\
75.1	0.01\\
75.11	0.01\\
75.12	0.01\\
75.13	0.01\\
75.14	0.01\\
75.15	0.01\\
75.16	0.01\\
75.17	0.01\\
75.18	0.01\\
75.19	0.01\\
75.2	0.01\\
75.21	0.01\\
75.22	0.01\\
75.23	0.01\\
75.24	0.01\\
75.25	0.01\\
75.26	0.01\\
75.27	0.01\\
75.28	0.01\\
75.29	0.01\\
75.3	0.01\\
75.31	0.01\\
75.32	0.01\\
75.33	0.01\\
75.34	0.01\\
75.35	0.01\\
75.36	0.01\\
75.37	0.01\\
75.38	0.01\\
75.39	0.01\\
75.4	0.01\\
75.41	0.01\\
75.42	0.01\\
75.43	0.01\\
75.44	0.01\\
75.45	0.01\\
75.46	0.01\\
75.47	0.01\\
75.48	0.01\\
75.49	0.01\\
75.5	0.01\\
75.51	0.01\\
75.52	0.01\\
75.53	0.01\\
75.54	0.01\\
75.55	0.01\\
75.56	0.01\\
75.57	0.01\\
75.58	0.01\\
75.59	0.01\\
75.6	0.01\\
75.61	0.01\\
75.62	0.01\\
75.63	0.01\\
75.64	0.01\\
75.65	0.01\\
75.66	0.01\\
75.67	0.01\\
75.68	0.01\\
75.69	0.01\\
75.7	0.01\\
75.71	0.01\\
75.72	0.01\\
75.73	0.01\\
75.74	0.01\\
75.75	0.01\\
75.76	0.01\\
75.77	0.01\\
75.78	0.01\\
75.79	0.01\\
75.8	0.01\\
75.81	0.01\\
75.82	0.01\\
75.83	0.01\\
75.84	0.01\\
75.85	0.01\\
75.86	0.01\\
75.87	0.01\\
75.88	0.01\\
75.89	0.01\\
75.9	0.01\\
75.91	0.01\\
75.92	0.01\\
75.93	0.01\\
75.94	0.01\\
75.95	0.01\\
75.96	0.01\\
75.97	0.01\\
75.98	0.01\\
75.99	0.01\\
76	0.01\\
76.01	0.01\\
76.02	0.01\\
76.03	0.01\\
76.04	0.01\\
76.05	0.01\\
76.06	0.01\\
76.07	0.01\\
76.08	0.01\\
76.09	0.01\\
76.1	0.01\\
76.11	0.01\\
76.12	0.01\\
76.13	0.01\\
76.14	0.01\\
76.15	0.01\\
76.16	0.01\\
76.17	0.01\\
76.18	0.01\\
76.19	0.01\\
76.2	0.01\\
76.21	0.01\\
76.22	0.01\\
76.23	0.01\\
76.24	0.01\\
76.25	0.01\\
76.26	0.01\\
76.27	0.01\\
76.28	0.01\\
76.29	0.01\\
76.3	0.01\\
76.31	0.01\\
76.32	0.01\\
76.33	0.01\\
76.34	0.01\\
76.35	0.01\\
76.36	0.01\\
76.37	0.01\\
76.38	0.01\\
76.39	0.01\\
76.4	0.01\\
76.41	0.01\\
76.42	0.01\\
76.43	0.01\\
76.44	0.01\\
76.45	0.01\\
76.46	0.01\\
76.47	0.01\\
76.48	0.01\\
76.49	0.01\\
76.5	0.01\\
76.51	0.01\\
76.52	0.01\\
76.53	0.01\\
76.54	0.01\\
76.55	0.01\\
76.56	0.01\\
76.57	0.01\\
76.58	0.01\\
76.59	0.01\\
76.6	0.01\\
76.61	0.01\\
76.62	0.01\\
76.63	0.01\\
76.64	0.01\\
76.65	0.01\\
76.66	0.01\\
76.67	0.01\\
76.68	0.01\\
76.69	0.01\\
76.7	0.01\\
76.71	0.01\\
76.72	0.01\\
76.73	0.01\\
76.74	0.01\\
76.75	0.01\\
76.76	0.01\\
76.77	0.01\\
76.78	0.01\\
76.79	0.01\\
76.8	0.01\\
76.81	0.01\\
76.82	0.01\\
76.83	0.01\\
76.84	0.01\\
76.85	0.01\\
76.86	0.01\\
76.87	0.01\\
76.88	0.01\\
76.89	0.01\\
76.9	0.01\\
76.91	0.01\\
76.92	0.01\\
76.93	0.01\\
76.94	0.01\\
76.95	0.01\\
76.96	0.01\\
76.97	0.01\\
76.98	0.01\\
76.99	0.01\\
77	0.01\\
77.01	0.01\\
77.02	0.01\\
77.03	0.01\\
77.04	0.01\\
77.05	0.01\\
77.06	0.01\\
77.07	0.01\\
77.08	0.01\\
77.09	0.01\\
77.1	0.01\\
77.11	0.01\\
77.12	0.01\\
77.13	0.01\\
77.14	0.01\\
77.15	0.01\\
77.16	0.01\\
77.17	0.01\\
77.18	0.01\\
77.19	0.01\\
77.2	0.01\\
77.21	0.01\\
77.22	0.01\\
77.23	0.01\\
77.24	0.01\\
77.25	0.01\\
77.26	0.01\\
77.27	0.01\\
77.28	0.01\\
77.29	0.01\\
77.3	0.01\\
77.31	0.01\\
77.32	0.01\\
77.33	0.01\\
77.34	0.01\\
77.35	0.01\\
77.36	0.01\\
77.37	0.01\\
77.38	0.01\\
77.39	0.01\\
77.4	0.01\\
77.41	0.01\\
77.42	0.01\\
77.43	0.01\\
77.44	0.01\\
77.45	0.01\\
77.46	0.01\\
77.47	0.01\\
77.48	0.01\\
77.49	0.01\\
77.5	0.01\\
77.51	0.01\\
77.52	0.01\\
77.53	0.01\\
77.54	0.01\\
77.55	0.01\\
77.56	0.01\\
77.57	0.01\\
77.58	0.01\\
77.59	0.01\\
77.6	0.01\\
77.61	0.01\\
77.62	0.01\\
77.63	0.01\\
77.64	0.01\\
77.65	0.01\\
77.66	0.01\\
77.67	0.01\\
77.68	0.01\\
77.69	0.01\\
77.7	0.01\\
77.71	0.01\\
77.72	0.01\\
77.73	0.01\\
77.74	0.01\\
77.75	0.01\\
77.76	0.01\\
77.77	0.01\\
77.78	0.01\\
77.79	0.01\\
77.8	0.01\\
77.81	0.01\\
77.82	0.01\\
77.83	0.01\\
77.84	0.01\\
77.85	0.01\\
77.86	0.01\\
77.87	0.01\\
77.88	0.01\\
77.89	0.01\\
77.9	0.01\\
77.91	0.01\\
77.92	0.01\\
77.93	0.01\\
77.94	0.01\\
77.95	0.01\\
77.96	0.01\\
77.97	0.01\\
77.98	0.01\\
77.99	0.01\\
78	0.01\\
78.01	0.01\\
78.02	0.01\\
78.03	0.01\\
78.04	0.01\\
78.05	0.01\\
78.06	0.01\\
78.07	0.01\\
78.08	0.01\\
78.09	0.01\\
78.1	0.01\\
78.11	0.01\\
78.12	0.01\\
78.13	0.01\\
78.14	0.01\\
78.15	0.01\\
78.16	0.01\\
78.17	0.01\\
78.18	0.01\\
78.19	0.01\\
78.2	0.01\\
78.21	0.01\\
78.22	0.01\\
78.23	0.01\\
78.24	0.01\\
78.25	0.01\\
78.26	0.01\\
78.27	0.01\\
78.28	0.01\\
78.29	0.01\\
78.3	0.01\\
78.31	0.01\\
78.32	0.01\\
78.33	0.01\\
78.34	0.01\\
78.35	0.01\\
78.36	0.01\\
78.37	0.01\\
78.38	0.01\\
78.39	0.01\\
78.4	0.01\\
78.41	0.01\\
78.42	0.01\\
78.43	0.01\\
78.44	0.01\\
78.45	0.01\\
78.46	0.01\\
78.47	0.01\\
78.48	0.01\\
78.49	0.01\\
78.5	0.01\\
78.51	0.01\\
78.52	0.01\\
78.53	0.01\\
78.54	0.01\\
78.55	0.01\\
78.56	0.01\\
78.57	0.01\\
78.58	0.01\\
78.59	0.01\\
78.6	0.01\\
78.61	0.01\\
78.62	0.01\\
78.63	0.01\\
78.64	0.01\\
78.65	0.01\\
78.66	0.01\\
78.67	0.01\\
78.68	0.01\\
78.69	0.01\\
78.7	0.01\\
78.71	0.01\\
78.72	0.01\\
78.73	0.01\\
78.74	0.01\\
78.75	0.01\\
78.76	0.01\\
78.77	0.01\\
78.78	0.01\\
78.79	0.01\\
78.8	0.01\\
78.81	0.01\\
78.82	0.01\\
78.83	0.01\\
78.84	0.01\\
78.85	0.01\\
78.86	0.01\\
78.87	0.01\\
78.88	0.01\\
78.89	0.01\\
78.9	0.01\\
78.91	0.01\\
78.92	0.01\\
78.93	0.01\\
78.94	0.01\\
78.95	0.01\\
78.96	0.01\\
78.97	0.01\\
78.98	0.01\\
78.99	0.01\\
79	0.01\\
79.01	0.01\\
79.02	0.01\\
79.03	0.01\\
79.04	0.01\\
79.05	0.01\\
79.06	0.01\\
79.07	0.01\\
79.08	0.01\\
79.09	0.01\\
79.1	0.01\\
79.11	0.01\\
79.12	0.01\\
79.13	0.01\\
79.14	0.01\\
79.15	0.01\\
79.16	0.01\\
79.17	0.01\\
79.18	0.01\\
79.19	0.01\\
79.2	0.01\\
79.21	0.01\\
79.22	0.01\\
79.23	0.01\\
79.24	0.01\\
79.25	0.01\\
79.26	0.01\\
79.27	0.01\\
79.28	0.01\\
79.29	0.01\\
79.3	0.01\\
79.31	0.01\\
79.32	0.01\\
79.33	0.01\\
79.34	0.01\\
79.35	0.01\\
79.36	0.01\\
79.37	0.01\\
79.38	0.01\\
79.39	0.01\\
79.4	0.01\\
79.41	0.01\\
79.42	0.01\\
79.43	0.01\\
79.44	0.01\\
79.45	0.01\\
79.46	0.01\\
79.47	0.01\\
79.48	0.01\\
79.49	0.01\\
79.5	0.01\\
79.51	0.01\\
79.52	0.01\\
79.53	0.01\\
79.54	0.01\\
79.55	0.01\\
79.56	0.01\\
79.57	0.01\\
79.58	0.01\\
79.59	0.01\\
79.6	0.01\\
79.61	0.01\\
79.62	0.01\\
79.63	0.01\\
79.64	0.01\\
79.65	0.01\\
79.66	0.01\\
79.67	0.01\\
79.68	0.01\\
79.69	0.01\\
79.7	0.01\\
79.71	0.01\\
79.72	0.01\\
79.73	0.01\\
79.74	0.01\\
79.75	0.01\\
79.76	0.01\\
79.77	0.01\\
79.78	0.01\\
79.79	0.01\\
79.8	0.01\\
79.81	0.01\\
79.82	0.01\\
79.83	0.01\\
79.84	0.01\\
79.85	0.01\\
79.86	0.01\\
79.87	0.01\\
79.88	0.01\\
79.89	0.01\\
79.9	0.01\\
79.91	0.01\\
79.92	0.01\\
79.93	0.01\\
79.94	0.01\\
79.95	0.01\\
79.96	0.01\\
79.97	0.01\\
79.98	0.01\\
79.99	0.01\\
80	0.01\\
80.01	0.01\\
};
\addplot [color=red,dashed]
  table[row sep=crcr]{%
80.01	0.01\\
80.02	0.01\\
80.03	0.01\\
80.04	0.01\\
80.05	0.01\\
80.06	0.01\\
80.07	0.01\\
80.08	0.01\\
80.09	0.01\\
80.1	0.01\\
80.11	0.01\\
80.12	0.01\\
80.13	0.01\\
80.14	0.01\\
80.15	0.01\\
80.16	0.01\\
80.17	0.01\\
80.18	0.01\\
80.19	0.01\\
80.2	0.01\\
80.21	0.01\\
80.22	0.01\\
80.23	0.01\\
80.24	0.01\\
80.25	0.01\\
80.26	0.01\\
80.27	0.01\\
80.28	0.01\\
80.29	0.01\\
80.3	0.01\\
80.31	0.01\\
80.32	0.01\\
80.33	0.01\\
80.34	0.01\\
80.35	0.01\\
80.36	0.01\\
80.37	0.01\\
80.38	0.01\\
80.39	0.01\\
80.4	0.01\\
80.41	0.01\\
80.42	0.01\\
80.43	0.01\\
80.44	0.01\\
80.45	0.01\\
80.46	0.01\\
80.47	0.01\\
80.48	0.01\\
80.49	0.01\\
80.5	0.01\\
80.51	0.01\\
80.52	0.01\\
80.53	0.01\\
80.54	0.01\\
80.55	0.01\\
80.56	0.01\\
80.57	0.01\\
80.58	0.01\\
80.59	0.01\\
80.6	0.01\\
80.61	0.01\\
80.62	0.01\\
80.63	0.01\\
80.64	0.01\\
80.65	0.01\\
80.66	0.01\\
80.67	0.01\\
80.68	0.01\\
80.69	0.01\\
80.7	0.01\\
80.71	0.01\\
80.72	0.01\\
80.73	0.01\\
80.74	0.01\\
80.75	0.01\\
80.76	0.01\\
80.77	0.01\\
80.78	0.01\\
80.79	0.01\\
80.8	0.01\\
80.81	0.01\\
80.82	0.01\\
80.83	0.01\\
80.84	0.01\\
80.85	0.01\\
80.86	0.01\\
80.87	0.01\\
80.88	0.01\\
80.89	0.01\\
80.9	0.01\\
80.91	0.01\\
80.92	0.01\\
80.93	0.01\\
80.94	0.01\\
80.95	0.01\\
80.96	0.01\\
80.97	0.01\\
80.98	0.01\\
80.99	0.01\\
81	0.01\\
81.01	0.01\\
81.02	0.01\\
81.03	0.01\\
81.04	0.01\\
81.05	0.01\\
81.06	0.01\\
81.07	0.01\\
81.08	0.01\\
81.09	0.01\\
81.1	0.01\\
81.11	0.01\\
81.12	0.01\\
81.13	0.01\\
81.14	0.01\\
81.15	0.01\\
81.16	0.01\\
81.17	0.01\\
81.18	0.01\\
81.19	0.01\\
81.2	0.01\\
81.21	0.01\\
81.22	0.01\\
81.23	0.01\\
81.24	0.01\\
81.25	0.01\\
81.26	0.01\\
81.27	0.01\\
81.28	0.01\\
81.29	0.01\\
81.3	0.01\\
81.31	0.01\\
81.32	0.01\\
81.33	0.01\\
81.34	0.01\\
81.35	0.01\\
81.36	0.01\\
81.37	0.01\\
81.38	0.01\\
81.39	0.01\\
81.4	0.01\\
81.41	0.01\\
81.42	0.01\\
81.43	0.01\\
81.44	0.01\\
81.45	0.01\\
81.46	0.01\\
81.47	0.01\\
81.48	0.01\\
81.49	0.01\\
81.5	0.01\\
81.51	0.01\\
81.52	0.01\\
81.53	0.01\\
81.54	0.01\\
81.55	0.01\\
81.56	0.01\\
81.57	0.01\\
81.58	0.01\\
81.59	0.01\\
81.6	0.01\\
81.61	0.01\\
81.62	0.01\\
81.63	0.01\\
81.64	0.01\\
81.65	0.01\\
81.66	0.01\\
81.67	0.01\\
81.68	0.01\\
81.69	0.01\\
81.7	0.01\\
81.71	0.01\\
81.72	0.01\\
81.73	0.01\\
81.74	0.01\\
81.75	0.01\\
81.76	0.01\\
81.77	0.01\\
81.78	0.01\\
81.79	0.01\\
81.8	0.01\\
81.81	0.01\\
81.82	0.01\\
81.83	0.01\\
81.84	0.01\\
81.85	0.01\\
81.86	0.01\\
81.87	0.01\\
81.88	0.01\\
81.89	0.01\\
81.9	0.01\\
81.91	0.01\\
81.92	0.01\\
81.93	0.01\\
81.94	0.01\\
81.95	0.01\\
81.96	0.01\\
81.97	0.01\\
81.98	0.01\\
81.99	0.01\\
82	0.01\\
82.01	0.01\\
82.02	0.01\\
82.03	0.01\\
82.04	0.01\\
82.05	0.01\\
82.06	0.01\\
82.07	0.01\\
82.08	0.01\\
82.09	0.01\\
82.1	0.01\\
82.11	0.01\\
82.12	0.01\\
82.13	0.01\\
82.14	0.01\\
82.15	0.01\\
82.16	0.01\\
82.17	0.01\\
82.18	0.01\\
82.19	0.01\\
82.2	0.01\\
82.21	0.01\\
82.22	0.01\\
82.23	0.01\\
82.24	0.01\\
82.25	0.01\\
82.26	0.01\\
82.27	0.01\\
82.28	0.01\\
82.29	0.01\\
82.3	0.01\\
82.31	0.01\\
82.32	0.01\\
82.33	0.01\\
82.34	0.01\\
82.35	0.01\\
82.36	0.01\\
82.37	0.01\\
82.38	0.01\\
82.39	0.01\\
82.4	0.01\\
82.41	0.01\\
82.42	0.01\\
82.43	0.01\\
82.44	0.01\\
82.45	0.01\\
82.46	0.01\\
82.47	0.01\\
82.48	0.01\\
82.49	0.01\\
82.5	0.01\\
82.51	0.01\\
82.52	0.01\\
82.53	0.01\\
82.54	0.01\\
82.55	0.01\\
82.56	0.01\\
82.57	0.01\\
82.58	0.01\\
82.59	0.01\\
82.6	0.01\\
82.61	0.01\\
82.62	0.01\\
82.63	0.01\\
82.64	0.01\\
82.65	0.01\\
82.66	0.01\\
82.67	0.01\\
82.68	0.01\\
82.69	0.01\\
82.7	0.01\\
82.71	0.01\\
82.72	0.01\\
82.73	0.01\\
82.74	0.01\\
82.75	0.01\\
82.76	0.01\\
82.77	0.01\\
82.78	0.01\\
82.79	0.01\\
82.8	0.01\\
82.81	0.01\\
82.82	0.01\\
82.83	0.01\\
82.84	0.01\\
82.85	0.01\\
82.86	0.01\\
82.87	0.01\\
82.88	0.01\\
82.89	0.01\\
82.9	0.01\\
82.91	0.01\\
82.92	0.01\\
82.93	0.01\\
82.94	0.01\\
82.95	0.01\\
82.96	0.01\\
82.97	0.01\\
82.98	0.01\\
82.99	0.01\\
83	0.01\\
83.01	0.01\\
83.02	0.01\\
83.03	0.01\\
83.04	0.01\\
83.05	0.01\\
83.06	0.01\\
83.07	0.01\\
83.08	0.01\\
83.09	0.01\\
83.1	0.01\\
83.11	0.01\\
83.12	0.01\\
83.13	0.01\\
83.14	0.01\\
83.15	0.01\\
83.16	0.01\\
83.17	0.01\\
83.18	0.01\\
83.19	0.01\\
83.2	0.01\\
83.21	0.01\\
83.22	0.01\\
83.23	0.01\\
83.24	0.01\\
83.25	0.01\\
83.26	0.01\\
83.27	0.01\\
83.28	0.01\\
83.29	0.01\\
83.3	0.01\\
83.31	0.01\\
83.32	0.01\\
83.33	0.01\\
83.34	0.01\\
83.35	0.01\\
83.36	0.01\\
83.37	0.01\\
83.38	0.01\\
83.39	0.01\\
83.4	0.01\\
83.41	0.01\\
83.42	0.01\\
83.43	0.01\\
83.44	0.01\\
83.45	0.01\\
83.46	0.01\\
83.47	0.01\\
83.48	0.01\\
83.49	0.01\\
83.5	0.01\\
83.51	0.01\\
83.52	0.01\\
83.53	0.01\\
83.54	0.01\\
83.55	0.01\\
83.56	0.01\\
83.57	0.01\\
83.58	0.01\\
83.59	0.01\\
83.6	0.01\\
83.61	0.01\\
83.62	0.01\\
83.63	0.01\\
83.64	0.01\\
83.65	0.01\\
83.66	0.01\\
83.67	0.01\\
83.68	0.01\\
83.69	0.01\\
83.7	0.01\\
83.71	0.01\\
83.72	0.01\\
83.73	0.01\\
83.74	0.01\\
83.75	0.01\\
83.76	0.01\\
83.77	0.01\\
83.78	0.01\\
83.79	0.01\\
83.8	0.01\\
83.81	0.01\\
83.82	0.01\\
83.83	0.01\\
83.84	0.01\\
83.85	0.01\\
83.86	0.01\\
83.87	0.01\\
83.88	0.01\\
83.89	0.01\\
83.9	0.01\\
83.91	0.01\\
83.92	0.01\\
83.93	0.01\\
83.94	0.01\\
83.95	0.01\\
83.96	0.01\\
83.97	0.01\\
83.98	0.01\\
83.99	0.01\\
84	0.01\\
84.01	0.01\\
84.02	0.01\\
84.03	0.01\\
84.04	0.01\\
84.05	0.01\\
84.06	0.01\\
84.07	0.01\\
84.08	0.01\\
84.09	0.01\\
84.1	0.01\\
84.11	0.01\\
84.12	0.01\\
84.13	0.01\\
84.14	0.01\\
84.15	0.01\\
84.16	0.01\\
84.17	0.01\\
84.18	0.01\\
84.19	0.01\\
84.2	0.01\\
84.21	0.01\\
84.22	0.01\\
84.23	0.01\\
84.24	0.01\\
84.25	0.01\\
84.26	0.01\\
84.27	0.01\\
84.28	0.01\\
84.29	0.01\\
84.3	0.01\\
84.31	0.01\\
84.32	0.01\\
84.33	0.01\\
84.34	0.01\\
84.35	0.01\\
84.36	0.01\\
84.37	0.01\\
84.38	0.01\\
84.39	0.01\\
84.4	0.01\\
84.41	0.01\\
84.42	0.01\\
84.43	0.01\\
84.44	0.01\\
84.45	0.01\\
84.46	0.01\\
84.47	0.01\\
84.48	0.01\\
84.49	0.01\\
84.5	0.01\\
84.51	0.01\\
84.52	0.01\\
84.53	0.01\\
84.54	0.01\\
84.55	0.01\\
84.56	0.01\\
84.57	0.01\\
84.58	0.01\\
84.59	0.01\\
84.6	0.01\\
84.61	0.01\\
84.62	0.01\\
84.63	0.01\\
84.64	0.01\\
84.65	0.01\\
84.66	0.01\\
84.67	0.01\\
84.68	0.01\\
84.69	0.01\\
84.7	0.01\\
84.71	0.01\\
84.72	0.01\\
84.73	0.01\\
84.74	0.01\\
84.75	0.01\\
84.76	0.01\\
84.77	0.01\\
84.78	0.01\\
84.79	0.01\\
84.8	0.01\\
84.81	0.01\\
84.82	0.01\\
84.83	0.01\\
84.84	0.01\\
84.85	0.01\\
84.86	0.01\\
84.87	0.01\\
84.88	0.01\\
84.89	0.01\\
84.9	0.01\\
84.91	0.01\\
84.92	0.01\\
84.93	0.01\\
84.94	0.01\\
84.95	0.01\\
84.96	0.01\\
84.97	0.01\\
84.98	0.01\\
84.99	0.01\\
85	0.01\\
85.01	0.01\\
85.02	0.01\\
85.03	0.01\\
85.04	0.01\\
85.05	0.01\\
85.06	0.01\\
85.07	0.01\\
85.08	0.01\\
85.09	0.01\\
85.1	0.01\\
85.11	0.01\\
85.12	0.01\\
85.13	0.01\\
85.14	0.01\\
85.15	0.01\\
85.16	0.01\\
85.17	0.01\\
85.18	0.01\\
85.19	0.01\\
85.2	0.01\\
85.21	0.01\\
85.22	0.01\\
85.23	0.01\\
85.24	0.01\\
85.25	0.01\\
85.26	0.01\\
85.27	0.01\\
85.28	0.01\\
85.29	0.01\\
85.3	0.01\\
85.31	0.01\\
85.32	0.01\\
85.33	0.01\\
85.34	0.01\\
85.35	0.01\\
85.36	0.01\\
85.37	0.01\\
85.38	0.01\\
85.39	0.01\\
85.4	0.01\\
85.41	0.01\\
85.42	0.01\\
85.43	0.01\\
85.44	0.01\\
85.45	0.01\\
85.46	0.01\\
85.47	0.01\\
85.48	0.01\\
85.49	0.01\\
85.5	0.01\\
85.51	0.01\\
85.52	0.01\\
85.53	0.01\\
85.54	0.01\\
85.55	0.01\\
85.56	0.01\\
85.57	0.01\\
85.58	0.01\\
85.59	0.01\\
85.6	0.01\\
85.61	0.01\\
85.62	0.01\\
85.63	0.01\\
85.64	0.01\\
85.65	0.01\\
85.66	0.01\\
85.67	0.01\\
85.68	0.01\\
85.69	0.01\\
85.7	0.01\\
85.71	0.01\\
85.72	0.01\\
85.73	0.01\\
85.74	0.01\\
85.75	0.01\\
85.76	0.01\\
85.77	0.01\\
85.78	0.01\\
85.79	0.01\\
85.8	0.01\\
85.81	0.01\\
85.82	0.01\\
85.83	0.01\\
85.84	0.01\\
85.85	0.01\\
85.86	0.01\\
85.87	0.01\\
85.88	0.01\\
85.89	0.01\\
85.9	0.01\\
85.91	0.01\\
85.92	0.01\\
85.93	0.01\\
85.94	0.01\\
85.95	0.01\\
85.96	0.01\\
85.97	0.01\\
85.98	0.01\\
85.99	0.01\\
86	0.01\\
86.01	0.01\\
86.02	0.01\\
86.03	0.01\\
86.04	0.01\\
86.05	0.01\\
86.06	0.01\\
86.07	0.01\\
86.08	0.01\\
86.09	0.01\\
86.1	0.01\\
86.11	0.01\\
86.12	0.01\\
86.13	0.01\\
86.14	0.01\\
86.15	0.01\\
86.16	0.01\\
86.17	0.01\\
86.18	0.01\\
86.19	0.01\\
86.2	0.01\\
86.21	0.01\\
86.22	0.01\\
86.23	0.01\\
86.24	0.01\\
86.25	0.01\\
86.26	0.01\\
86.27	0.01\\
86.28	0.01\\
86.29	0.01\\
86.3	0.01\\
86.31	0.01\\
86.32	0.01\\
86.33	0.01\\
86.34	0.01\\
86.35	0.01\\
86.36	0.01\\
86.37	0.01\\
86.38	0.01\\
86.39	0.01\\
86.4	0.01\\
86.41	0.01\\
86.42	0.01\\
86.43	0.01\\
86.44	0.01\\
86.45	0.01\\
86.46	0.01\\
86.47	0.01\\
86.48	0.01\\
86.49	0.01\\
86.5	0.01\\
86.51	0.01\\
86.52	0.01\\
86.53	0.01\\
86.54	0.01\\
86.55	0.01\\
86.56	0.01\\
86.57	0.01\\
86.58	0.01\\
86.59	0.01\\
86.6	0.01\\
86.61	0.01\\
86.62	0.01\\
86.63	0.01\\
86.64	0.01\\
86.65	0.01\\
86.66	0.01\\
86.67	0.01\\
86.68	0.01\\
86.69	0.01\\
86.7	0.01\\
86.71	0.01\\
86.72	0.01\\
86.73	0.01\\
86.74	0.01\\
86.75	0.01\\
86.76	0.01\\
86.77	0.01\\
86.78	0.01\\
86.79	0.01\\
86.8	0.01\\
86.81	0.01\\
86.82	0.01\\
86.83	0.01\\
86.84	0.01\\
86.85	0.01\\
86.86	0.01\\
86.87	0.01\\
86.88	0.01\\
86.89	0.01\\
86.9	0.01\\
86.91	0.01\\
86.92	0.01\\
86.93	0.01\\
86.94	0.01\\
86.95	0.01\\
86.96	0.01\\
86.97	0.01\\
86.98	0.01\\
86.99	0.01\\
87	0.01\\
87.01	0.01\\
87.02	0.01\\
87.03	0.01\\
87.04	0.01\\
87.05	0.01\\
87.06	0.01\\
87.07	0.01\\
87.08	0.01\\
87.09	0.01\\
87.1	0.01\\
87.11	0.01\\
87.12	0.01\\
87.13	0.01\\
87.14	0.01\\
87.15	0.01\\
87.16	0.01\\
87.17	0.01\\
87.18	0.01\\
87.19	0.01\\
87.2	0.01\\
87.21	0.01\\
87.22	0.01\\
87.23	0.01\\
87.24	0.01\\
87.25	0.01\\
87.26	0.01\\
87.27	0.01\\
87.28	0.01\\
87.29	0.01\\
87.3	0.01\\
87.31	0.01\\
87.32	0.01\\
87.33	0.01\\
87.34	0.01\\
87.35	0.01\\
87.36	0.01\\
87.37	0.01\\
87.38	0.01\\
87.39	0.01\\
87.4	0.01\\
87.41	0.01\\
87.42	0.01\\
87.43	0.01\\
87.44	0.01\\
87.45	0.01\\
87.46	0.01\\
87.47	0.01\\
87.48	0.01\\
87.49	0.01\\
87.5	0.01\\
87.51	0.01\\
87.52	0.01\\
87.53	0.01\\
87.54	0.01\\
87.55	0.01\\
87.56	0.01\\
87.57	0.01\\
87.58	0.01\\
87.59	0.01\\
87.6	0.01\\
87.61	0.01\\
87.62	0.01\\
87.63	0.01\\
87.64	0.01\\
87.65	0.01\\
87.66	0.01\\
87.67	0.01\\
87.68	0.01\\
87.69	0.01\\
87.7	0.01\\
87.71	0.01\\
87.72	0.01\\
87.73	0.01\\
87.74	0.01\\
87.75	0.01\\
87.76	0.01\\
87.77	0.01\\
87.78	0.01\\
87.79	0.01\\
87.8	0.01\\
87.81	0.01\\
87.82	0.01\\
87.83	0.01\\
87.84	0.01\\
87.85	0.01\\
87.86	0.01\\
87.87	0.01\\
87.88	0.01\\
87.89	0.01\\
87.9	0.01\\
87.91	0.01\\
87.92	0.01\\
87.93	0.01\\
87.94	0.01\\
87.95	0.01\\
87.96	0.01\\
87.97	0.01\\
87.98	0.01\\
87.99	0.01\\
88	0.01\\
88.01	0.01\\
88.02	0.01\\
88.03	0.01\\
88.04	0.01\\
88.05	0.01\\
88.06	0.01\\
88.07	0.01\\
88.08	0.01\\
88.09	0.01\\
88.1	0.01\\
88.11	0.01\\
88.12	0.01\\
88.13	0.01\\
88.14	0.01\\
88.15	0.01\\
88.16	0.01\\
88.17	0.01\\
88.18	0.01\\
88.19	0.01\\
88.2	0.01\\
88.21	0.01\\
88.22	0.01\\
88.23	0.01\\
88.24	0.01\\
88.25	0.01\\
88.26	0.01\\
88.27	0.01\\
88.28	0.01\\
88.29	0.01\\
88.3	0.01\\
88.31	0.01\\
88.32	0.01\\
88.33	0.01\\
88.34	0.01\\
88.35	0.01\\
88.36	0.01\\
88.37	0.01\\
88.38	0.01\\
88.39	0.01\\
88.4	0.01\\
88.41	0.01\\
88.42	0.01\\
88.43	0.01\\
88.44	0.01\\
88.45	0.01\\
88.46	0.01\\
88.47	0.01\\
88.48	0.01\\
88.49	0.01\\
88.5	0.01\\
88.51	0.01\\
88.52	0.01\\
88.53	0.01\\
88.54	0.01\\
88.55	0.01\\
88.56	0.01\\
88.57	0.01\\
88.58	0.01\\
88.59	0.01\\
88.6	0.01\\
88.61	0.01\\
88.62	0.01\\
88.63	0.01\\
88.64	0.01\\
88.65	0.01\\
88.66	0.01\\
88.67	0.01\\
88.68	0.01\\
88.69	0.01\\
88.7	0.01\\
88.71	0.01\\
88.72	0.01\\
88.73	0.01\\
88.74	0.01\\
88.75	0.01\\
88.76	0.01\\
88.77	0.01\\
88.78	0.01\\
88.79	0.01\\
88.8	0.01\\
88.81	0.01\\
88.82	0.01\\
88.83	0.01\\
88.84	0.01\\
88.85	0.01\\
88.86	0.01\\
88.87	0.01\\
88.88	0.01\\
88.89	0.01\\
88.9	0.01\\
88.91	0.01\\
88.92	0.01\\
88.93	0.01\\
88.94	0.01\\
88.95	0.01\\
88.96	0.01\\
88.97	0.01\\
88.98	0.01\\
88.99	0.01\\
89	0.01\\
89.01	0.01\\
89.02	0.01\\
89.03	0.01\\
89.04	0.01\\
89.05	0.01\\
89.06	0.01\\
89.07	0.01\\
89.08	0.01\\
89.09	0.01\\
89.1	0.01\\
89.11	0.01\\
89.12	0.01\\
89.13	0.01\\
89.14	0.01\\
89.15	0.01\\
89.16	0.01\\
89.17	0.01\\
89.18	0.01\\
89.19	0.01\\
89.2	0.01\\
89.21	0.01\\
89.22	0.01\\
89.23	0.01\\
89.24	0.01\\
89.25	0.01\\
89.26	0.01\\
89.27	0.01\\
89.28	0.01\\
89.29	0.01\\
89.3	0.01\\
89.31	0.01\\
89.32	0.01\\
89.33	0.01\\
89.34	0.01\\
89.35	0.01\\
89.36	0.01\\
89.37	0.01\\
89.38	0.01\\
89.39	0.01\\
89.4	0.01\\
89.41	0.01\\
89.42	0.01\\
89.43	0.01\\
89.44	0.01\\
89.45	0.01\\
89.46	0.01\\
89.47	0.01\\
89.48	0.01\\
89.49	0.01\\
89.5	0.01\\
89.51	0.01\\
89.52	0.01\\
89.53	0.01\\
89.54	0.01\\
89.55	0.01\\
89.56	0.01\\
89.57	0.01\\
89.58	0.01\\
89.59	0.01\\
89.6	0.01\\
89.61	0.01\\
89.62	0.01\\
89.63	0.01\\
89.64	0.01\\
89.65	0.01\\
89.66	0.01\\
89.67	0.01\\
89.68	0.01\\
89.69	0.01\\
89.7	0.01\\
89.71	0.01\\
89.72	0.01\\
89.73	0.01\\
89.74	0.01\\
89.75	0.01\\
89.76	0.01\\
89.77	0.01\\
89.78	0.01\\
89.79	0.01\\
89.8	0.01\\
89.81	0.01\\
89.82	0.01\\
89.83	0.01\\
89.84	0.01\\
89.85	0.01\\
89.86	0.01\\
89.87	0.01\\
89.88	0.01\\
89.89	0.01\\
89.9	0.01\\
89.91	0.01\\
89.92	0.01\\
89.93	0.01\\
89.94	0.01\\
89.95	0.01\\
89.96	0.01\\
89.97	0.01\\
89.98	0.01\\
89.99	0.01\\
90	0.01\\
90.01	0.01\\
90.02	0.01\\
90.03	0.01\\
90.04	0.01\\
90.05	0.01\\
90.06	0.01\\
90.07	0.01\\
90.08	0.01\\
90.09	0.01\\
90.1	0.01\\
90.11	0.01\\
90.12	0.01\\
90.13	0.01\\
90.14	0.01\\
90.15	0.01\\
90.16	0.01\\
90.17	0.01\\
90.18	0.01\\
90.19	0.01\\
90.2	0.01\\
90.21	0.01\\
90.22	0.01\\
90.23	0.01\\
90.24	0.01\\
90.25	0.01\\
90.26	0.01\\
90.27	0.01\\
90.28	0.01\\
90.29	0.01\\
90.3	0.01\\
90.31	0.01\\
90.32	0.01\\
90.33	0.01\\
90.34	0.01\\
90.35	0.01\\
90.36	0.01\\
90.37	0.01\\
90.38	0.01\\
90.39	0.01\\
90.4	0.01\\
90.41	0.01\\
90.42	0.01\\
90.43	0.01\\
90.44	0.01\\
90.45	0.01\\
90.46	0.01\\
90.47	0.01\\
90.48	0.01\\
90.49	0.01\\
90.5	0.01\\
90.51	0.01\\
90.52	0.01\\
90.53	0.01\\
90.54	0.01\\
90.55	0.01\\
90.56	0.01\\
90.57	0.01\\
90.58	0.01\\
90.59	0.01\\
90.6	0.01\\
90.61	0.01\\
90.62	0.01\\
90.63	0.01\\
90.64	0.01\\
90.65	0.01\\
90.66	0.01\\
90.67	0.01\\
90.68	0.01\\
90.69	0.01\\
90.7	0.01\\
90.71	0.01\\
90.72	0.01\\
90.73	0.01\\
90.74	0.01\\
90.75	0.01\\
90.76	0.01\\
90.77	0.01\\
90.78	0.01\\
90.79	0.01\\
90.8	0.01\\
90.81	0.01\\
90.82	0.01\\
90.83	0.01\\
90.84	0.01\\
90.85	0.01\\
90.86	0.01\\
90.87	0.01\\
90.88	0.01\\
90.89	0.01\\
90.9	0.01\\
90.91	0.01\\
90.92	0.01\\
90.93	0.01\\
90.94	0.01\\
90.95	0.01\\
90.96	0.01\\
90.97	0.01\\
90.98	0.01\\
90.99	0.01\\
91	0.01\\
91.01	0.01\\
91.02	0.01\\
91.03	0.01\\
91.04	0.01\\
91.05	0.01\\
91.06	0.01\\
91.07	0.01\\
91.08	0.01\\
91.09	0.01\\
91.1	0.01\\
91.11	0.01\\
91.12	0.01\\
91.13	0.01\\
91.14	0.01\\
91.15	0.01\\
91.16	0.01\\
91.17	0.01\\
91.18	0.01\\
91.19	0.01\\
91.2	0.01\\
91.21	0.01\\
91.22	0.01\\
91.23	0.01\\
91.24	0.01\\
91.25	0.01\\
91.26	0.01\\
91.27	0.01\\
91.28	0.01\\
91.29	0.01\\
91.3	0.01\\
91.31	0.01\\
91.32	0.01\\
91.33	0.01\\
91.34	0.01\\
91.35	0.01\\
91.36	0.01\\
91.37	0.01\\
91.38	0.01\\
91.39	0.01\\
91.4	0.01\\
91.41	0.01\\
91.42	0.01\\
91.43	0.01\\
91.44	0.01\\
91.45	0.01\\
91.46	0.01\\
91.47	0.01\\
91.48	0.01\\
91.49	0.01\\
91.5	0.01\\
91.51	0.01\\
91.52	0.01\\
91.53	0.01\\
91.54	0.01\\
91.55	0.01\\
91.56	0.01\\
91.57	0.01\\
91.58	0.01\\
91.59	0.01\\
91.6	0.01\\
91.61	0.01\\
91.62	0.01\\
91.63	0.01\\
91.64	0.01\\
91.65	0.01\\
91.66	0.01\\
91.67	0.01\\
91.68	0.01\\
91.69	0.01\\
91.7	0.01\\
91.71	0.01\\
91.72	0.01\\
91.73	0.01\\
91.74	0.01\\
91.75	0.01\\
91.76	0.01\\
91.77	0.01\\
91.78	0.01\\
91.79	0.01\\
91.8	0.01\\
91.81	0.01\\
91.82	0.01\\
91.83	0.01\\
91.84	0.01\\
91.85	0.01\\
91.86	0.01\\
91.87	0.01\\
91.88	0.01\\
91.89	0.01\\
91.9	0.01\\
91.91	0.01\\
91.92	0.01\\
91.93	0.01\\
91.94	0.01\\
91.95	0.01\\
91.96	0.01\\
91.97	0.01\\
91.98	0.01\\
91.99	0.01\\
92	0.01\\
92.01	0.01\\
92.02	0.01\\
92.03	0.01\\
92.04	0.01\\
92.05	0.01\\
92.06	0.01\\
92.07	0.01\\
92.08	0.01\\
92.09	0.01\\
92.1	0.01\\
92.11	0.01\\
92.12	0.01\\
92.13	0.01\\
92.14	0.01\\
92.15	0.01\\
92.16	0.01\\
92.17	0.01\\
92.18	0.01\\
92.19	0.01\\
92.2	0.01\\
92.21	0.01\\
92.22	0.01\\
92.23	0.01\\
92.24	0.01\\
92.25	0.01\\
92.26	0.01\\
92.27	0.01\\
92.28	0.01\\
92.29	0.01\\
92.3	0.01\\
92.31	0.01\\
92.32	0.01\\
92.33	0.01\\
92.34	0.01\\
92.35	0.01\\
92.36	0.01\\
92.37	0.01\\
92.38	0.01\\
92.39	0.01\\
92.4	0.01\\
92.41	0.01\\
92.42	0.01\\
92.43	0.01\\
92.44	0.01\\
92.45	0.01\\
92.46	0.01\\
92.47	0.01\\
92.48	0.01\\
92.49	0.01\\
92.5	0.01\\
92.51	0.01\\
92.52	0.01\\
92.53	0.01\\
92.54	0.01\\
92.55	0.01\\
92.56	0.01\\
92.57	0.01\\
92.58	0.01\\
92.59	0.01\\
92.6	0.01\\
92.61	0.01\\
92.62	0.01\\
92.63	0.01\\
92.64	0.01\\
92.65	0.01\\
92.66	0.01\\
92.67	0.01\\
92.68	0.01\\
92.69	0.01\\
92.7	0.01\\
92.71	0.01\\
92.72	0.01\\
92.73	0.01\\
92.74	0.01\\
92.75	0.01\\
92.76	0.01\\
92.77	0.01\\
92.78	0.01\\
92.79	0.01\\
92.8	0.01\\
92.81	0.01\\
92.82	0.01\\
92.83	0.01\\
92.84	0.01\\
92.85	0.01\\
92.86	0.01\\
92.87	0.01\\
92.88	0.01\\
92.89	0.01\\
92.9	0.01\\
92.91	0.01\\
92.92	0.01\\
92.93	0.01\\
92.94	0.01\\
92.95	0.01\\
92.96	0.01\\
92.97	0.01\\
92.98	0.01\\
92.99	0.01\\
93	0.01\\
93.01	0.01\\
93.02	0.01\\
93.03	0.01\\
93.04	0.01\\
93.05	0.01\\
93.06	0.01\\
93.07	0.01\\
93.08	0.01\\
93.09	0.01\\
93.1	0.01\\
93.11	0.01\\
93.12	0.01\\
93.13	0.01\\
93.14	0.01\\
93.15	0.01\\
93.16	0.01\\
93.17	0.01\\
93.18	0.01\\
93.19	0.01\\
93.2	0.01\\
93.21	0.01\\
93.22	0.01\\
93.23	0.01\\
93.24	0.01\\
93.25	0.01\\
93.26	0.01\\
93.27	0.01\\
93.28	0.01\\
93.29	0.01\\
93.3	0.01\\
93.31	0.01\\
93.32	0.01\\
93.33	0.01\\
93.34	0.01\\
93.35	0.01\\
93.36	0.01\\
93.37	0.01\\
93.38	0.01\\
93.39	0.01\\
93.4	0.01\\
93.41	0.01\\
93.42	0.01\\
93.43	0.01\\
93.44	0.01\\
93.45	0.01\\
93.46	0.01\\
93.47	0.01\\
93.48	0.01\\
93.49	0.01\\
93.5	0.01\\
93.51	0.01\\
93.52	0.01\\
93.53	0.01\\
93.54	0.01\\
93.55	0.01\\
93.56	0.01\\
93.57	0.01\\
93.58	0.01\\
93.59	0.01\\
93.6	0.01\\
93.61	0.01\\
93.62	0.01\\
93.63	0.01\\
93.64	0.01\\
93.65	0.01\\
93.66	0.01\\
93.67	0.01\\
93.68	0.01\\
93.69	0.01\\
93.7	0.01\\
93.71	0.01\\
93.72	0.01\\
93.73	0.01\\
93.74	0.01\\
93.75	0.01\\
93.76	0.01\\
93.77	0.01\\
93.78	0.01\\
93.79	0.01\\
93.8	0.01\\
93.81	0.01\\
93.82	0.01\\
93.83	0.01\\
93.84	0.01\\
93.85	0.01\\
93.86	0.01\\
93.87	0.01\\
93.88	0.01\\
93.89	0.01\\
93.9	0.01\\
93.91	0.01\\
93.92	0.01\\
93.93	0.01\\
93.94	0.01\\
93.95	0.01\\
93.96	0.01\\
93.97	0.01\\
93.98	0.01\\
93.99	0.01\\
94	0.01\\
94.01	0.01\\
94.02	0.01\\
94.03	0.01\\
94.04	0.01\\
94.05	0.01\\
94.06	0.01\\
94.07	0.01\\
94.08	0.01\\
94.09	0.01\\
94.1	0.01\\
94.11	0.01\\
94.12	0.01\\
94.13	0.01\\
94.14	0.01\\
94.15	0.01\\
94.16	0.01\\
94.17	0.01\\
94.18	0.01\\
94.19	0.01\\
94.2	0.01\\
94.21	0.01\\
94.22	0.01\\
94.23	0.01\\
94.24	0.01\\
94.25	0.01\\
94.26	0.01\\
94.27	0.01\\
94.28	0.01\\
94.29	0.01\\
94.3	0.01\\
94.31	0.01\\
94.32	0.01\\
94.33	0.01\\
94.34	0.01\\
94.35	0.01\\
94.36	0.01\\
94.37	0.01\\
94.38	0.01\\
94.39	0.01\\
94.4	0.01\\
94.41	0.01\\
94.42	0.01\\
94.43	0.01\\
94.44	0.01\\
94.45	0.01\\
94.46	0.01\\
94.47	0.01\\
94.48	0.01\\
94.49	0.01\\
94.5	0.01\\
94.51	0.01\\
94.52	0.01\\
94.53	0.01\\
94.54	0.01\\
94.55	0.01\\
94.56	0.01\\
94.57	0.01\\
94.58	0.01\\
94.59	0.01\\
94.6	0.01\\
94.61	0.01\\
94.62	0.01\\
94.63	0.01\\
94.64	0.01\\
94.65	0.01\\
94.66	0.01\\
94.67	0.01\\
94.68	0.01\\
94.69	0.01\\
94.7	0.01\\
94.71	0.01\\
94.72	0.01\\
94.73	0.01\\
94.74	0.01\\
94.75	0.01\\
94.76	0.01\\
94.77	0.01\\
94.78	0.01\\
94.79	0.01\\
94.8	0.01\\
94.81	0.01\\
94.82	0.01\\
94.83	0.01\\
94.84	0.01\\
94.85	0.01\\
94.86	0.01\\
94.87	0.01\\
94.88	0.01\\
94.89	0.01\\
94.9	0.01\\
94.91	0.01\\
94.92	0.01\\
94.93	0.01\\
94.94	0.01\\
94.95	0.01\\
94.96	0.01\\
94.97	0.01\\
94.98	0.01\\
94.99	0.01\\
95	0.01\\
95.01	0.01\\
95.02	0.01\\
95.03	0.01\\
95.04	0.01\\
95.05	0.01\\
95.06	0.01\\
95.07	0.01\\
95.08	0.01\\
95.09	0.01\\
95.1	0.01\\
95.11	0.01\\
95.12	0.01\\
95.13	0.01\\
95.14	0.01\\
95.15	0.01\\
95.16	0.01\\
95.17	0.01\\
95.18	0.01\\
95.19	0.01\\
95.2	0.01\\
95.21	0.01\\
95.22	0.01\\
95.23	0.01\\
95.24	0.01\\
95.25	0.01\\
95.26	0.01\\
95.27	0.01\\
95.28	0.01\\
95.29	0.01\\
95.3	0.01\\
95.31	0.01\\
95.32	0.01\\
95.33	0.01\\
95.34	0.01\\
95.35	0.01\\
95.36	0.01\\
95.37	0.01\\
95.38	0.01\\
95.39	0.01\\
95.4	0.01\\
95.41	0.01\\
95.42	0.01\\
95.43	0.01\\
95.44	0.01\\
95.45	0.01\\
95.46	0.01\\
95.47	0.01\\
95.48	0.01\\
95.49	0.01\\
95.5	0.01\\
95.51	0.01\\
95.52	0.01\\
95.53	0.01\\
95.54	0.01\\
95.55	0.01\\
95.56	0.01\\
95.57	0.01\\
95.58	0.01\\
95.59	0.01\\
95.6	0.01\\
95.61	0.01\\
95.62	0.01\\
95.63	0.01\\
95.64	0.01\\
95.65	0.01\\
95.66	0.01\\
95.67	0.01\\
95.68	0.01\\
95.69	0.01\\
95.7	0.01\\
95.71	0.01\\
95.72	0.01\\
95.73	0.01\\
95.74	0.01\\
95.75	0.01\\
95.76	0.01\\
95.77	0.01\\
95.78	0.01\\
95.79	0.01\\
95.8	0.01\\
95.81	0.01\\
95.82	0.01\\
95.83	0.01\\
95.84	0.01\\
95.85	0.01\\
95.86	0.01\\
95.87	0.01\\
95.88	0.01\\
95.89	0.01\\
95.9	0.01\\
95.91	0.01\\
95.92	0.01\\
95.93	0.01\\
95.94	0.01\\
95.95	0.01\\
95.96	0.01\\
95.97	0.01\\
95.98	0.01\\
95.99	0.01\\
96	0.01\\
96.01	0.01\\
96.02	0.01\\
96.03	0.01\\
96.04	0.01\\
96.05	0.01\\
96.06	0.01\\
96.07	0.01\\
96.08	0.01\\
96.09	0.01\\
96.1	0.01\\
96.11	0.01\\
96.12	0.01\\
96.13	0.01\\
96.14	0.01\\
96.15	0.01\\
96.16	0.01\\
96.17	0.01\\
96.18	0.01\\
96.19	0.01\\
96.2	0.01\\
96.21	0.01\\
96.22	0.01\\
96.23	0.01\\
96.24	0.01\\
96.25	0.01\\
96.26	0.01\\
96.27	0.01\\
96.28	0.01\\
96.29	0.01\\
96.3	0.01\\
96.31	0.01\\
96.32	0.01\\
96.33	0.01\\
96.34	0.01\\
96.35	0.01\\
96.36	0.01\\
96.37	0.01\\
96.38	0.01\\
96.39	0.01\\
96.4	0.01\\
96.41	0.01\\
96.42	0.01\\
96.43	0.01\\
96.44	0.01\\
96.45	0.01\\
96.46	0.01\\
96.47	0.01\\
96.48	0.01\\
96.49	0.01\\
96.5	0.01\\
96.51	0.01\\
96.52	0.01\\
96.53	0.01\\
96.54	0.01\\
96.55	0.01\\
96.56	0.01\\
96.57	0.01\\
96.58	0.01\\
96.59	0.01\\
96.6	0.01\\
96.61	0.01\\
96.62	0.01\\
96.63	0.01\\
96.64	0.01\\
96.65	0.01\\
96.66	0.01\\
96.67	0.01\\
96.68	0.01\\
96.69	0.01\\
96.7	0.01\\
96.71	0.01\\
96.72	0.01\\
96.73	0.01\\
96.74	0.01\\
96.75	0.01\\
96.76	0.01\\
96.77	0.01\\
96.78	0.01\\
96.79	0.01\\
96.8	0.01\\
96.81	0.01\\
96.82	0.01\\
96.83	0.01\\
96.84	0.01\\
96.85	0.01\\
96.86	0.01\\
96.87	0.01\\
96.88	0.01\\
96.89	0.01\\
96.9	0.01\\
96.91	0.01\\
96.92	0.01\\
96.93	0.01\\
96.94	0.01\\
96.95	0.01\\
96.96	0.01\\
96.97	0.01\\
96.98	0.01\\
96.99	0.01\\
97	0.01\\
97.01	0.01\\
97.02	0.01\\
97.03	0.01\\
97.04	0.01\\
97.05	0.01\\
97.06	0.01\\
97.07	0.01\\
97.08	0.01\\
97.09	0.01\\
97.1	0.01\\
97.11	0.01\\
97.12	0.01\\
97.13	0.01\\
97.14	0.01\\
97.15	0.01\\
97.16	0.01\\
97.17	0.01\\
97.18	0.01\\
97.19	0.01\\
97.2	0.01\\
97.21	0.01\\
97.22	0.01\\
97.23	0.01\\
97.24	0.01\\
97.25	0.01\\
97.26	0.01\\
97.27	0.01\\
97.28	0.01\\
97.29	0.01\\
97.3	0.01\\
97.31	0.01\\
97.32	0.01\\
97.33	0.01\\
97.34	0.01\\
97.35	0.01\\
97.36	0.01\\
97.37	0.01\\
97.38	0.01\\
97.39	0.01\\
97.4	0.01\\
97.41	0.01\\
97.42	0.01\\
97.43	0.01\\
97.44	0.01\\
97.45	0.01\\
97.46	0.01\\
97.47	0.01\\
97.48	0.01\\
97.49	0.01\\
97.5	0.01\\
97.51	0.01\\
97.52	0.01\\
97.53	0.01\\
97.54	0.01\\
97.55	0.01\\
97.56	0.01\\
97.57	0.01\\
97.58	0.01\\
97.59	0.01\\
97.6	0.01\\
97.61	0.01\\
97.62	0.01\\
97.63	0.01\\
97.64	0.01\\
97.65	0.01\\
97.66	0.01\\
97.67	0.01\\
97.68	0.01\\
97.69	0.01\\
97.7	0.01\\
97.71	0.01\\
97.72	0.01\\
97.73	0.01\\
97.74	0.01\\
97.75	0.01\\
97.76	0.01\\
97.77	0.01\\
97.78	0.01\\
97.79	0.01\\
97.8	0.01\\
97.81	0.01\\
97.82	0.01\\
97.83	0.01\\
97.84	0.01\\
97.85	0.01\\
97.86	0.01\\
97.87	0.01\\
97.88	0.01\\
97.89	0.01\\
97.9	0.01\\
97.91	0.01\\
97.92	0.01\\
97.93	0.01\\
97.94	0.01\\
97.95	0.01\\
97.96	0.01\\
97.97	0.01\\
97.98	0.01\\
97.99	0.01\\
98	0.01\\
98.01	0.01\\
98.02	0.01\\
98.03	0.01\\
98.04	0.01\\
98.05	0.01\\
98.06	0.01\\
98.07	0.01\\
98.08	0.01\\
98.09	0.01\\
98.1	0.01\\
98.11	0.01\\
98.12	0.01\\
98.13	0.01\\
98.14	0.01\\
98.15	0.01\\
98.16	0.01\\
98.17	0.01\\
98.18	0.01\\
98.19	0.01\\
98.2	0.01\\
98.21	0.01\\
98.22	0.01\\
98.23	0.01\\
98.24	0.01\\
98.25	0.01\\
98.26	0.01\\
98.27	0.01\\
98.28	0.01\\
98.29	0.01\\
98.3	0.01\\
98.31	0.01\\
98.32	0.01\\
98.33	0.01\\
98.34	0.01\\
98.35	0.01\\
98.36	0.01\\
98.37	0.01\\
98.38	0.01\\
98.39	0.01\\
98.4	0.01\\
98.41	0.01\\
98.42	0.01\\
98.43	0.01\\
98.44	0.01\\
98.45	0.01\\
98.46	0.01\\
98.47	0.01\\
98.48	0.01\\
98.49	0.01\\
98.5	0.01\\
98.51	0.01\\
98.52	0.01\\
98.53	0.01\\
98.54	0.01\\
98.55	0.01\\
98.56	0.01\\
98.57	0.01\\
98.58	0.01\\
98.59	0.01\\
98.6	0.01\\
98.61	0.01\\
98.62	0.01\\
98.63	0.01\\
98.64	0.01\\
98.65	0.01\\
98.66	0.01\\
98.67	0.01\\
98.68	0.01\\
98.69	0.01\\
98.7	0.01\\
98.71	0.01\\
98.72	0.01\\
98.73	0.01\\
98.74	0.01\\
98.75	0.01\\
98.76	0.01\\
98.77	0.01\\
98.78	0.01\\
98.79	0.01\\
98.8	0.01\\
98.81	0.01\\
98.82	0.01\\
98.83	0.01\\
98.84	0.01\\
98.85	0.01\\
98.86	0.01\\
98.87	0.01\\
98.88	0.01\\
98.89	0.01\\
98.9	0.01\\
98.91	0.01\\
98.92	0.01\\
98.93	0.01\\
98.94	0.01\\
98.95	0.01\\
98.96	0.01\\
98.97	0.01\\
98.98	0.01\\
98.99	0.01\\
99	0.01\\
99.01	0.01\\
99.02	0.01\\
99.03	0.01\\
99.04	0.01\\
99.05	0.01\\
99.06	0.01\\
99.07	0.01\\
99.08	0.01\\
99.09	0.01\\
99.1	0.01\\
99.11	0.01\\
99.12	0.01\\
99.13	0.01\\
99.14	0.00998658074540667\\
99.15	0.00981394762269664\\
99.16	0.00964021100322028\\
99.17	0.00946535581611112\\
99.18	0.0092893668729081\\
99.19	0.00911222854862415\\
99.2	0.00893392479044062\\
99.21	0.00875443910369502\\
99.22	0.00857376609587646\\
99.23	0.00839192844111959\\
99.24	0.00820890970110557\\
99.25	0.00802469297024153\\
99.26	0.00783926085981311\\
99.27	0.00765259548150028\\
99.28	0.00746467843022284\\
99.29	0.00727549076627984\\
99.3	0.00708501299674452\\
99.31	0.00689322505607424\\
99.32	0.00670010628589174\\
99.33	0.00650563541389112\\
99.34	0.00630979053181893\\
99.35	0.00611254922952013\\
99.36	0.00591388841707558\\
99.37	0.00571378427464273\\
99.38	0.00551221206393733\\
99.39	0.00530914628219126\\
99.4	0.00510456061610345\\
99.41	0.00489842792758349\\
99.42	0.00483354096659049\\
99.43	0.00476839726764135\\
99.44	0.00470269466849563\\
99.45	0.00463643043392819\\
99.46	0.0045696019104531\\
99.47	0.00450220652384934\\
99.48	0.00443422230458845\\
99.49	0.0043656225605152\\
99.5	0.0042964036119007\\
99.51	0.00422656183753986\\
99.52	0.00415609367880683\\
99.53	0.00408499564389557\\
99.54	0.00401326431225524\\
99.55	0.00394089633923041\\
99.56	0.0038678884609169\\
99.57	0.00379423749924454\\
99.58	0.00371994036729902\\
99.59	0.00364499407489556\\
99.6	0.00356939573441804\\
99.61	0.00349314256693811\\
99.62	0.00341623190862955\\
99.63	0.00333866121749428\\
99.64	0.00326042808041742\\
99.65	0.00318153022056972\\
99.66	0.00310196550517733\\
99.67	0.00302173195367941\\
99.68	0.00294082774629635\\
99.69	0.00285925123288629\\
99.7	0.00277700094237731\\
99.71	0.00269407559271374\\
99.72	0.00261047410832228\\
99.73	0.00252619563924148\\
99.74	0.00244123955819906\\
99.75	0.00235560547282003\\
99.76	0.0022692932384859\\
99.77	0.00218230297188465\\
99.78	0.002094635065294\\
99.79	0.00200629020164378\\
99.8	0.00191726937040586\\
99.81	0.001827573884364\\
99.82	0.0017372053973193\\
99.83	0.00164616592279117\\
99.84	0.00155445785377785\\
99.85	0.00146208398364519\\
99.86	0.00136904752821746\\
99.87	0.00127535214914916\\
99.88	0.00118100197866295\\
99.89	0.00108600164574474\\
99.9	0.000990356303894199\\
99.91	0.000894071660039042\\
99.92	0.000797154004978352\\
99.93	0.000699610246319197\\
99.94	0.000601447943428838\\
99.95	0.000502675344543365\\
99.96	0.000403301426184725\\
99.97	0.000303335935050114\\
99.98	0.000202789432550832\\
99.99	0.00010167334219198\\
100	0\\
};
\addlegendentry{$q=-2$};

\addplot [color=blue,dashed,forget plot]
  table[row sep=crcr]{%
0.01	0.01\\
0.02	0.01\\
0.03	0.01\\
0.04	0.01\\
0.05	0.01\\
0.06	0.01\\
0.07	0.01\\
0.08	0.01\\
0.09	0.01\\
0.1	0.01\\
0.11	0.01\\
0.12	0.01\\
0.13	0.01\\
0.14	0.01\\
0.15	0.01\\
0.16	0.01\\
0.17	0.01\\
0.18	0.01\\
0.19	0.01\\
0.2	0.01\\
0.21	0.01\\
0.22	0.01\\
0.23	0.01\\
0.24	0.01\\
0.25	0.01\\
0.26	0.01\\
0.27	0.01\\
0.28	0.01\\
0.29	0.01\\
0.3	0.01\\
0.31	0.01\\
0.32	0.01\\
0.33	0.01\\
0.34	0.01\\
0.35	0.01\\
0.36	0.01\\
0.37	0.01\\
0.38	0.01\\
0.39	0.01\\
0.4	0.01\\
0.41	0.01\\
0.42	0.01\\
0.43	0.01\\
0.44	0.01\\
0.45	0.01\\
0.46	0.01\\
0.47	0.01\\
0.48	0.01\\
0.49	0.01\\
0.5	0.01\\
0.51	0.01\\
0.52	0.01\\
0.53	0.01\\
0.54	0.01\\
0.55	0.01\\
0.56	0.01\\
0.57	0.01\\
0.58	0.01\\
0.59	0.01\\
0.6	0.01\\
0.61	0.01\\
0.62	0.01\\
0.63	0.01\\
0.64	0.01\\
0.65	0.01\\
0.66	0.01\\
0.67	0.01\\
0.68	0.01\\
0.69	0.01\\
0.7	0.01\\
0.71	0.01\\
0.72	0.01\\
0.73	0.01\\
0.74	0.01\\
0.75	0.01\\
0.76	0.01\\
0.77	0.01\\
0.78	0.01\\
0.79	0.01\\
0.8	0.01\\
0.81	0.01\\
0.82	0.01\\
0.83	0.01\\
0.84	0.01\\
0.85	0.01\\
0.86	0.01\\
0.87	0.01\\
0.88	0.01\\
0.89	0.01\\
0.9	0.01\\
0.91	0.01\\
0.92	0.01\\
0.93	0.01\\
0.94	0.01\\
0.95	0.01\\
0.96	0.01\\
0.97	0.01\\
0.98	0.01\\
0.99	0.01\\
1	0.01\\
1.01	0.01\\
1.02	0.01\\
1.03	0.01\\
1.04	0.01\\
1.05	0.01\\
1.06	0.01\\
1.07	0.01\\
1.08	0.01\\
1.09	0.01\\
1.1	0.01\\
1.11	0.01\\
1.12	0.01\\
1.13	0.01\\
1.14	0.01\\
1.15	0.01\\
1.16	0.01\\
1.17	0.01\\
1.18	0.01\\
1.19	0.01\\
1.2	0.01\\
1.21	0.01\\
1.22	0.01\\
1.23	0.01\\
1.24	0.01\\
1.25	0.01\\
1.26	0.01\\
1.27	0.01\\
1.28	0.01\\
1.29	0.01\\
1.3	0.01\\
1.31	0.01\\
1.32	0.01\\
1.33	0.01\\
1.34	0.01\\
1.35	0.01\\
1.36	0.01\\
1.37	0.01\\
1.38	0.01\\
1.39	0.01\\
1.4	0.01\\
1.41	0.01\\
1.42	0.01\\
1.43	0.01\\
1.44	0.01\\
1.45	0.01\\
1.46	0.01\\
1.47	0.01\\
1.48	0.01\\
1.49	0.01\\
1.5	0.01\\
1.51	0.01\\
1.52	0.01\\
1.53	0.01\\
1.54	0.01\\
1.55	0.01\\
1.56	0.01\\
1.57	0.01\\
1.58	0.01\\
1.59	0.01\\
1.6	0.01\\
1.61	0.01\\
1.62	0.01\\
1.63	0.01\\
1.64	0.01\\
1.65	0.01\\
1.66	0.01\\
1.67	0.01\\
1.68	0.01\\
1.69	0.01\\
1.7	0.01\\
1.71	0.01\\
1.72	0.01\\
1.73	0.01\\
1.74	0.01\\
1.75	0.01\\
1.76	0.01\\
1.77	0.01\\
1.78	0.01\\
1.79	0.01\\
1.8	0.01\\
1.81	0.01\\
1.82	0.01\\
1.83	0.01\\
1.84	0.01\\
1.85	0.01\\
1.86	0.01\\
1.87	0.01\\
1.88	0.01\\
1.89	0.01\\
1.9	0.01\\
1.91	0.01\\
1.92	0.01\\
1.93	0.01\\
1.94	0.01\\
1.95	0.01\\
1.96	0.01\\
1.97	0.01\\
1.98	0.01\\
1.99	0.01\\
2	0.01\\
2.01	0.01\\
2.02	0.01\\
2.03	0.01\\
2.04	0.01\\
2.05	0.01\\
2.06	0.01\\
2.07	0.01\\
2.08	0.01\\
2.09	0.01\\
2.1	0.01\\
2.11	0.01\\
2.12	0.01\\
2.13	0.01\\
2.14	0.01\\
2.15	0.01\\
2.16	0.01\\
2.17	0.01\\
2.18	0.01\\
2.19	0.01\\
2.2	0.01\\
2.21	0.01\\
2.22	0.01\\
2.23	0.01\\
2.24	0.01\\
2.25	0.01\\
2.26	0.01\\
2.27	0.01\\
2.28	0.01\\
2.29	0.01\\
2.3	0.01\\
2.31	0.01\\
2.32	0.01\\
2.33	0.01\\
2.34	0.01\\
2.35	0.01\\
2.36	0.01\\
2.37	0.01\\
2.38	0.01\\
2.39	0.01\\
2.4	0.01\\
2.41	0.01\\
2.42	0.01\\
2.43	0.01\\
2.44	0.01\\
2.45	0.01\\
2.46	0.01\\
2.47	0.01\\
2.48	0.01\\
2.49	0.01\\
2.5	0.01\\
2.51	0.01\\
2.52	0.01\\
2.53	0.01\\
2.54	0.01\\
2.55	0.01\\
2.56	0.01\\
2.57	0.01\\
2.58	0.01\\
2.59	0.01\\
2.6	0.01\\
2.61	0.01\\
2.62	0.01\\
2.63	0.01\\
2.64	0.01\\
2.65	0.01\\
2.66	0.01\\
2.67	0.01\\
2.68	0.01\\
2.69	0.01\\
2.7	0.01\\
2.71	0.01\\
2.72	0.01\\
2.73	0.01\\
2.74	0.01\\
2.75	0.01\\
2.76	0.01\\
2.77	0.01\\
2.78	0.01\\
2.79	0.01\\
2.8	0.01\\
2.81	0.01\\
2.82	0.01\\
2.83	0.01\\
2.84	0.01\\
2.85	0.01\\
2.86	0.01\\
2.87	0.01\\
2.88	0.01\\
2.89	0.01\\
2.9	0.01\\
2.91	0.01\\
2.92	0.01\\
2.93	0.01\\
2.94	0.01\\
2.95	0.01\\
2.96	0.01\\
2.97	0.01\\
2.98	0.01\\
2.99	0.01\\
3	0.01\\
3.01	0.01\\
3.02	0.01\\
3.03	0.01\\
3.04	0.01\\
3.05	0.01\\
3.06	0.01\\
3.07	0.01\\
3.08	0.01\\
3.09	0.01\\
3.1	0.01\\
3.11	0.01\\
3.12	0.01\\
3.13	0.01\\
3.14	0.01\\
3.15	0.01\\
3.16	0.01\\
3.17	0.01\\
3.18	0.01\\
3.19	0.01\\
3.2	0.01\\
3.21	0.01\\
3.22	0.01\\
3.23	0.01\\
3.24	0.01\\
3.25	0.01\\
3.26	0.01\\
3.27	0.01\\
3.28	0.01\\
3.29	0.01\\
3.3	0.01\\
3.31	0.01\\
3.32	0.01\\
3.33	0.01\\
3.34	0.01\\
3.35	0.01\\
3.36	0.01\\
3.37	0.01\\
3.38	0.01\\
3.39	0.01\\
3.4	0.01\\
3.41	0.01\\
3.42	0.01\\
3.43	0.01\\
3.44	0.01\\
3.45	0.01\\
3.46	0.01\\
3.47	0.01\\
3.48	0.01\\
3.49	0.01\\
3.5	0.01\\
3.51	0.01\\
3.52	0.01\\
3.53	0.01\\
3.54	0.01\\
3.55	0.01\\
3.56	0.01\\
3.57	0.01\\
3.58	0.01\\
3.59	0.01\\
3.6	0.01\\
3.61	0.01\\
3.62	0.01\\
3.63	0.01\\
3.64	0.01\\
3.65	0.01\\
3.66	0.01\\
3.67	0.01\\
3.68	0.01\\
3.69	0.01\\
3.7	0.01\\
3.71	0.01\\
3.72	0.01\\
3.73	0.01\\
3.74	0.01\\
3.75	0.01\\
3.76	0.01\\
3.77	0.01\\
3.78	0.01\\
3.79	0.01\\
3.8	0.01\\
3.81	0.01\\
3.82	0.01\\
3.83	0.01\\
3.84	0.01\\
3.85	0.01\\
3.86	0.01\\
3.87	0.01\\
3.88	0.01\\
3.89	0.01\\
3.9	0.01\\
3.91	0.01\\
3.92	0.01\\
3.93	0.01\\
3.94	0.01\\
3.95	0.01\\
3.96	0.01\\
3.97	0.01\\
3.98	0.01\\
3.99	0.01\\
4	0.01\\
4.01	0.01\\
4.02	0.01\\
4.03	0.01\\
4.04	0.01\\
4.05	0.01\\
4.06	0.01\\
4.07	0.01\\
4.08	0.01\\
4.09	0.01\\
4.1	0.01\\
4.11	0.01\\
4.12	0.01\\
4.13	0.01\\
4.14	0.01\\
4.15	0.01\\
4.16	0.01\\
4.17	0.01\\
4.18	0.01\\
4.19	0.01\\
4.2	0.01\\
4.21	0.01\\
4.22	0.01\\
4.23	0.01\\
4.24	0.01\\
4.25	0.01\\
4.26	0.01\\
4.27	0.01\\
4.28	0.01\\
4.29	0.01\\
4.3	0.01\\
4.31	0.01\\
4.32	0.01\\
4.33	0.01\\
4.34	0.01\\
4.35	0.01\\
4.36	0.01\\
4.37	0.01\\
4.38	0.01\\
4.39	0.01\\
4.4	0.01\\
4.41	0.01\\
4.42	0.01\\
4.43	0.01\\
4.44	0.01\\
4.45	0.01\\
4.46	0.01\\
4.47	0.01\\
4.48	0.01\\
4.49	0.01\\
4.5	0.01\\
4.51	0.01\\
4.52	0.01\\
4.53	0.01\\
4.54	0.01\\
4.55	0.01\\
4.56	0.01\\
4.57	0.01\\
4.58	0.01\\
4.59	0.01\\
4.6	0.01\\
4.61	0.01\\
4.62	0.01\\
4.63	0.01\\
4.64	0.01\\
4.65	0.01\\
4.66	0.01\\
4.67	0.01\\
4.68	0.01\\
4.69	0.01\\
4.7	0.01\\
4.71	0.01\\
4.72	0.01\\
4.73	0.01\\
4.74	0.01\\
4.75	0.01\\
4.76	0.01\\
4.77	0.01\\
4.78	0.01\\
4.79	0.01\\
4.8	0.01\\
4.81	0.01\\
4.82	0.01\\
4.83	0.01\\
4.84	0.01\\
4.85	0.01\\
4.86	0.01\\
4.87	0.01\\
4.88	0.01\\
4.89	0.01\\
4.9	0.01\\
4.91	0.01\\
4.92	0.01\\
4.93	0.01\\
4.94	0.01\\
4.95	0.01\\
4.96	0.01\\
4.97	0.01\\
4.98	0.01\\
4.99	0.01\\
5	0.01\\
5.01	0.01\\
5.02	0.01\\
5.03	0.01\\
5.04	0.01\\
5.05	0.01\\
5.06	0.01\\
5.07	0.01\\
5.08	0.01\\
5.09	0.01\\
5.1	0.01\\
5.11	0.01\\
5.12	0.01\\
5.13	0.01\\
5.14	0.01\\
5.15	0.01\\
5.16	0.01\\
5.17	0.01\\
5.18	0.01\\
5.19	0.01\\
5.2	0.01\\
5.21	0.01\\
5.22	0.01\\
5.23	0.01\\
5.24	0.01\\
5.25	0.01\\
5.26	0.01\\
5.27	0.01\\
5.28	0.01\\
5.29	0.01\\
5.3	0.01\\
5.31	0.01\\
5.32	0.01\\
5.33	0.01\\
5.34	0.01\\
5.35	0.01\\
5.36	0.01\\
5.37	0.01\\
5.38	0.01\\
5.39	0.01\\
5.4	0.01\\
5.41	0.01\\
5.42	0.01\\
5.43	0.01\\
5.44	0.01\\
5.45	0.01\\
5.46	0.01\\
5.47	0.01\\
5.48	0.01\\
5.49	0.01\\
5.5	0.01\\
5.51	0.01\\
5.52	0.01\\
5.53	0.01\\
5.54	0.01\\
5.55	0.01\\
5.56	0.01\\
5.57	0.01\\
5.58	0.01\\
5.59	0.01\\
5.6	0.01\\
5.61	0.01\\
5.62	0.01\\
5.63	0.01\\
5.64	0.01\\
5.65	0.01\\
5.66	0.01\\
5.67	0.01\\
5.68	0.01\\
5.69	0.01\\
5.7	0.01\\
5.71	0.01\\
5.72	0.01\\
5.73	0.01\\
5.74	0.01\\
5.75	0.01\\
5.76	0.01\\
5.77	0.01\\
5.78	0.01\\
5.79	0.01\\
5.8	0.01\\
5.81	0.01\\
5.82	0.01\\
5.83	0.01\\
5.84	0.01\\
5.85	0.01\\
5.86	0.01\\
5.87	0.01\\
5.88	0.01\\
5.89	0.01\\
5.9	0.01\\
5.91	0.01\\
5.92	0.01\\
5.93	0.01\\
5.94	0.01\\
5.95	0.01\\
5.96	0.01\\
5.97	0.01\\
5.98	0.01\\
5.99	0.01\\
6	0.01\\
6.01	0.01\\
6.02	0.01\\
6.03	0.01\\
6.04	0.01\\
6.05	0.01\\
6.06	0.01\\
6.07	0.01\\
6.08	0.01\\
6.09	0.01\\
6.1	0.01\\
6.11	0.01\\
6.12	0.01\\
6.13	0.01\\
6.14	0.01\\
6.15	0.01\\
6.16	0.01\\
6.17	0.01\\
6.18	0.01\\
6.19	0.01\\
6.2	0.01\\
6.21	0.01\\
6.22	0.01\\
6.23	0.01\\
6.24	0.01\\
6.25	0.01\\
6.26	0.01\\
6.27	0.01\\
6.28	0.01\\
6.29	0.01\\
6.3	0.01\\
6.31	0.01\\
6.32	0.01\\
6.33	0.01\\
6.34	0.01\\
6.35	0.01\\
6.36	0.01\\
6.37	0.01\\
6.38	0.01\\
6.39	0.01\\
6.4	0.01\\
6.41	0.01\\
6.42	0.01\\
6.43	0.01\\
6.44	0.01\\
6.45	0.01\\
6.46	0.01\\
6.47	0.01\\
6.48	0.01\\
6.49	0.01\\
6.5	0.01\\
6.51	0.01\\
6.52	0.01\\
6.53	0.01\\
6.54	0.01\\
6.55	0.01\\
6.56	0.01\\
6.57	0.01\\
6.58	0.01\\
6.59	0.01\\
6.6	0.01\\
6.61	0.01\\
6.62	0.01\\
6.63	0.01\\
6.64	0.01\\
6.65	0.01\\
6.66	0.01\\
6.67	0.01\\
6.68	0.01\\
6.69	0.01\\
6.7	0.01\\
6.71	0.01\\
6.72	0.01\\
6.73	0.01\\
6.74	0.01\\
6.75	0.01\\
6.76	0.01\\
6.77	0.01\\
6.78	0.01\\
6.79	0.01\\
6.8	0.01\\
6.81	0.01\\
6.82	0.01\\
6.83	0.01\\
6.84	0.01\\
6.85	0.01\\
6.86	0.01\\
6.87	0.01\\
6.88	0.01\\
6.89	0.01\\
6.9	0.01\\
6.91	0.01\\
6.92	0.01\\
6.93	0.01\\
6.94	0.01\\
6.95	0.01\\
6.96	0.01\\
6.97	0.01\\
6.98	0.01\\
6.99	0.01\\
7	0.01\\
7.01	0.01\\
7.02	0.01\\
7.03	0.01\\
7.04	0.01\\
7.05	0.01\\
7.06	0.01\\
7.07	0.01\\
7.08	0.01\\
7.09	0.01\\
7.1	0.01\\
7.11	0.01\\
7.12	0.01\\
7.13	0.01\\
7.14	0.01\\
7.15	0.01\\
7.16	0.01\\
7.17	0.01\\
7.18	0.01\\
7.19	0.01\\
7.2	0.01\\
7.21	0.01\\
7.22	0.01\\
7.23	0.01\\
7.24	0.01\\
7.25	0.01\\
7.26	0.01\\
7.27	0.01\\
7.28	0.01\\
7.29	0.01\\
7.3	0.01\\
7.31	0.01\\
7.32	0.01\\
7.33	0.01\\
7.34	0.01\\
7.35	0.01\\
7.36	0.01\\
7.37	0.01\\
7.38	0.01\\
7.39	0.01\\
7.4	0.01\\
7.41	0.01\\
7.42	0.01\\
7.43	0.01\\
7.44	0.01\\
7.45	0.01\\
7.46	0.01\\
7.47	0.01\\
7.48	0.01\\
7.49	0.01\\
7.5	0.01\\
7.51	0.01\\
7.52	0.01\\
7.53	0.01\\
7.54	0.01\\
7.55	0.01\\
7.56	0.01\\
7.57	0.01\\
7.58	0.01\\
7.59	0.01\\
7.6	0.01\\
7.61	0.01\\
7.62	0.01\\
7.63	0.01\\
7.64	0.01\\
7.65	0.01\\
7.66	0.01\\
7.67	0.01\\
7.68	0.01\\
7.69	0.01\\
7.7	0.01\\
7.71	0.01\\
7.72	0.01\\
7.73	0.01\\
7.74	0.01\\
7.75	0.01\\
7.76	0.01\\
7.77	0.01\\
7.78	0.01\\
7.79	0.01\\
7.8	0.01\\
7.81	0.01\\
7.82	0.01\\
7.83	0.01\\
7.84	0.01\\
7.85	0.01\\
7.86	0.01\\
7.87	0.01\\
7.88	0.01\\
7.89	0.01\\
7.9	0.01\\
7.91	0.01\\
7.92	0.01\\
7.93	0.01\\
7.94	0.01\\
7.95	0.01\\
7.96	0.01\\
7.97	0.01\\
7.98	0.01\\
7.99	0.01\\
8	0.01\\
8.01	0.01\\
8.02	0.01\\
8.03	0.01\\
8.04	0.01\\
8.05	0.01\\
8.06	0.01\\
8.07	0.01\\
8.08	0.01\\
8.09	0.01\\
8.1	0.01\\
8.11	0.01\\
8.12	0.01\\
8.13	0.01\\
8.14	0.01\\
8.15	0.01\\
8.16	0.01\\
8.17	0.01\\
8.18	0.01\\
8.19	0.01\\
8.2	0.01\\
8.21	0.01\\
8.22	0.01\\
8.23	0.01\\
8.24	0.01\\
8.25	0.01\\
8.26	0.01\\
8.27	0.01\\
8.28	0.01\\
8.29	0.01\\
8.3	0.01\\
8.31	0.01\\
8.32	0.01\\
8.33	0.01\\
8.34	0.01\\
8.35	0.01\\
8.36	0.01\\
8.37	0.01\\
8.38	0.01\\
8.39	0.01\\
8.4	0.01\\
8.41	0.01\\
8.42	0.01\\
8.43	0.01\\
8.44	0.01\\
8.45	0.01\\
8.46	0.01\\
8.47	0.01\\
8.48	0.01\\
8.49	0.01\\
8.5	0.01\\
8.51	0.01\\
8.52	0.01\\
8.53	0.01\\
8.54	0.01\\
8.55	0.01\\
8.56	0.01\\
8.57	0.01\\
8.58	0.01\\
8.59	0.01\\
8.6	0.01\\
8.61	0.01\\
8.62	0.01\\
8.63	0.01\\
8.64	0.01\\
8.65	0.01\\
8.66	0.01\\
8.67	0.01\\
8.68	0.01\\
8.69	0.01\\
8.7	0.01\\
8.71	0.01\\
8.72	0.01\\
8.73	0.01\\
8.74	0.01\\
8.75	0.01\\
8.76	0.01\\
8.77	0.01\\
8.78	0.01\\
8.79	0.01\\
8.8	0.01\\
8.81	0.01\\
8.82	0.01\\
8.83	0.01\\
8.84	0.01\\
8.85	0.01\\
8.86	0.01\\
8.87	0.01\\
8.88	0.01\\
8.89	0.01\\
8.9	0.01\\
8.91	0.01\\
8.92	0.01\\
8.93	0.01\\
8.94	0.01\\
8.95	0.01\\
8.96	0.01\\
8.97	0.01\\
8.98	0.01\\
8.99	0.01\\
9	0.01\\
9.01	0.01\\
9.02	0.01\\
9.03	0.01\\
9.04	0.01\\
9.05	0.01\\
9.06	0.01\\
9.07	0.01\\
9.08	0.01\\
9.09	0.01\\
9.1	0.01\\
9.11	0.01\\
9.12	0.01\\
9.13	0.01\\
9.14	0.01\\
9.15	0.01\\
9.16	0.01\\
9.17	0.01\\
9.18	0.01\\
9.19	0.01\\
9.2	0.01\\
9.21	0.01\\
9.22	0.01\\
9.23	0.01\\
9.24	0.01\\
9.25	0.01\\
9.26	0.01\\
9.27	0.01\\
9.28	0.01\\
9.29	0.01\\
9.3	0.01\\
9.31	0.01\\
9.32	0.01\\
9.33	0.01\\
9.34	0.01\\
9.35	0.01\\
9.36	0.01\\
9.37	0.01\\
9.38	0.01\\
9.39	0.01\\
9.4	0.01\\
9.41	0.01\\
9.42	0.01\\
9.43	0.01\\
9.44	0.01\\
9.45	0.01\\
9.46	0.01\\
9.47	0.01\\
9.48	0.01\\
9.49	0.01\\
9.5	0.01\\
9.51	0.01\\
9.52	0.01\\
9.53	0.01\\
9.54	0.01\\
9.55	0.01\\
9.56	0.01\\
9.57	0.01\\
9.58	0.01\\
9.59	0.01\\
9.6	0.01\\
9.61	0.01\\
9.62	0.01\\
9.63	0.01\\
9.64	0.01\\
9.65	0.01\\
9.66	0.01\\
9.67	0.01\\
9.68	0.01\\
9.69	0.01\\
9.7	0.01\\
9.71	0.01\\
9.72	0.01\\
9.73	0.01\\
9.74	0.01\\
9.75	0.01\\
9.76	0.01\\
9.77	0.01\\
9.78	0.01\\
9.79	0.01\\
9.8	0.01\\
9.81	0.01\\
9.82	0.01\\
9.83	0.01\\
9.84	0.01\\
9.85	0.01\\
9.86	0.01\\
9.87	0.01\\
9.88	0.01\\
9.89	0.01\\
9.9	0.01\\
9.91	0.01\\
9.92	0.01\\
9.93	0.01\\
9.94	0.01\\
9.95	0.01\\
9.96	0.01\\
9.97	0.01\\
9.98	0.01\\
9.99	0.01\\
10	0.01\\
10.01	0.01\\
10.02	0.01\\
10.03	0.01\\
10.04	0.01\\
10.05	0.01\\
10.06	0.01\\
10.07	0.01\\
10.08	0.01\\
10.09	0.01\\
10.1	0.01\\
10.11	0.01\\
10.12	0.01\\
10.13	0.01\\
10.14	0.01\\
10.15	0.01\\
10.16	0.01\\
10.17	0.01\\
10.18	0.01\\
10.19	0.01\\
10.2	0.01\\
10.21	0.01\\
10.22	0.01\\
10.23	0.01\\
10.24	0.01\\
10.25	0.01\\
10.26	0.01\\
10.27	0.01\\
10.28	0.01\\
10.29	0.01\\
10.3	0.01\\
10.31	0.01\\
10.32	0.01\\
10.33	0.01\\
10.34	0.01\\
10.35	0.01\\
10.36	0.01\\
10.37	0.01\\
10.38	0.01\\
10.39	0.01\\
10.4	0.01\\
10.41	0.01\\
10.42	0.01\\
10.43	0.01\\
10.44	0.01\\
10.45	0.01\\
10.46	0.01\\
10.47	0.01\\
10.48	0.01\\
10.49	0.01\\
10.5	0.01\\
10.51	0.01\\
10.52	0.01\\
10.53	0.01\\
10.54	0.01\\
10.55	0.01\\
10.56	0.01\\
10.57	0.01\\
10.58	0.01\\
10.59	0.01\\
10.6	0.01\\
10.61	0.01\\
10.62	0.01\\
10.63	0.01\\
10.64	0.01\\
10.65	0.01\\
10.66	0.01\\
10.67	0.01\\
10.68	0.01\\
10.69	0.01\\
10.7	0.01\\
10.71	0.01\\
10.72	0.01\\
10.73	0.01\\
10.74	0.01\\
10.75	0.01\\
10.76	0.01\\
10.77	0.01\\
10.78	0.01\\
10.79	0.01\\
10.8	0.01\\
10.81	0.01\\
10.82	0.01\\
10.83	0.01\\
10.84	0.01\\
10.85	0.01\\
10.86	0.01\\
10.87	0.01\\
10.88	0.01\\
10.89	0.01\\
10.9	0.01\\
10.91	0.01\\
10.92	0.01\\
10.93	0.01\\
10.94	0.01\\
10.95	0.01\\
10.96	0.01\\
10.97	0.01\\
10.98	0.01\\
10.99	0.01\\
11	0.01\\
11.01	0.01\\
11.02	0.01\\
11.03	0.01\\
11.04	0.01\\
11.05	0.01\\
11.06	0.01\\
11.07	0.01\\
11.08	0.01\\
11.09	0.01\\
11.1	0.01\\
11.11	0.01\\
11.12	0.01\\
11.13	0.01\\
11.14	0.01\\
11.15	0.01\\
11.16	0.01\\
11.17	0.01\\
11.18	0.01\\
11.19	0.01\\
11.2	0.01\\
11.21	0.01\\
11.22	0.01\\
11.23	0.01\\
11.24	0.01\\
11.25	0.01\\
11.26	0.01\\
11.27	0.01\\
11.28	0.01\\
11.29	0.01\\
11.3	0.01\\
11.31	0.01\\
11.32	0.01\\
11.33	0.01\\
11.34	0.01\\
11.35	0.01\\
11.36	0.01\\
11.37	0.01\\
11.38	0.01\\
11.39	0.01\\
11.4	0.01\\
11.41	0.01\\
11.42	0.01\\
11.43	0.01\\
11.44	0.01\\
11.45	0.01\\
11.46	0.01\\
11.47	0.01\\
11.48	0.01\\
11.49	0.01\\
11.5	0.01\\
11.51	0.01\\
11.52	0.01\\
11.53	0.01\\
11.54	0.01\\
11.55	0.01\\
11.56	0.01\\
11.57	0.01\\
11.58	0.01\\
11.59	0.01\\
11.6	0.01\\
11.61	0.01\\
11.62	0.01\\
11.63	0.01\\
11.64	0.01\\
11.65	0.01\\
11.66	0.01\\
11.67	0.01\\
11.68	0.01\\
11.69	0.01\\
11.7	0.01\\
11.71	0.01\\
11.72	0.01\\
11.73	0.01\\
11.74	0.01\\
11.75	0.01\\
11.76	0.01\\
11.77	0.01\\
11.78	0.01\\
11.79	0.01\\
11.8	0.01\\
11.81	0.01\\
11.82	0.01\\
11.83	0.01\\
11.84	0.01\\
11.85	0.01\\
11.86	0.01\\
11.87	0.01\\
11.88	0.01\\
11.89	0.01\\
11.9	0.01\\
11.91	0.01\\
11.92	0.01\\
11.93	0.01\\
11.94	0.01\\
11.95	0.01\\
11.96	0.01\\
11.97	0.01\\
11.98	0.01\\
11.99	0.01\\
12	0.01\\
12.01	0.01\\
12.02	0.01\\
12.03	0.01\\
12.04	0.01\\
12.05	0.01\\
12.06	0.01\\
12.07	0.01\\
12.08	0.01\\
12.09	0.01\\
12.1	0.01\\
12.11	0.01\\
12.12	0.01\\
12.13	0.01\\
12.14	0.01\\
12.15	0.01\\
12.16	0.01\\
12.17	0.01\\
12.18	0.01\\
12.19	0.01\\
12.2	0.01\\
12.21	0.01\\
12.22	0.01\\
12.23	0.01\\
12.24	0.01\\
12.25	0.01\\
12.26	0.01\\
12.27	0.01\\
12.28	0.01\\
12.29	0.01\\
12.3	0.01\\
12.31	0.01\\
12.32	0.01\\
12.33	0.01\\
12.34	0.01\\
12.35	0.01\\
12.36	0.01\\
12.37	0.01\\
12.38	0.01\\
12.39	0.01\\
12.4	0.01\\
12.41	0.01\\
12.42	0.01\\
12.43	0.01\\
12.44	0.01\\
12.45	0.01\\
12.46	0.01\\
12.47	0.01\\
12.48	0.01\\
12.49	0.01\\
12.5	0.01\\
12.51	0.01\\
12.52	0.01\\
12.53	0.01\\
12.54	0.01\\
12.55	0.01\\
12.56	0.01\\
12.57	0.01\\
12.58	0.01\\
12.59	0.01\\
12.6	0.01\\
12.61	0.01\\
12.62	0.01\\
12.63	0.01\\
12.64	0.01\\
12.65	0.01\\
12.66	0.01\\
12.67	0.01\\
12.68	0.01\\
12.69	0.01\\
12.7	0.01\\
12.71	0.01\\
12.72	0.01\\
12.73	0.01\\
12.74	0.01\\
12.75	0.01\\
12.76	0.01\\
12.77	0.01\\
12.78	0.01\\
12.79	0.01\\
12.8	0.01\\
12.81	0.01\\
12.82	0.01\\
12.83	0.01\\
12.84	0.01\\
12.85	0.01\\
12.86	0.01\\
12.87	0.01\\
12.88	0.01\\
12.89	0.01\\
12.9	0.01\\
12.91	0.01\\
12.92	0.01\\
12.93	0.01\\
12.94	0.01\\
12.95	0.01\\
12.96	0.01\\
12.97	0.01\\
12.98	0.01\\
12.99	0.01\\
13	0.01\\
13.01	0.01\\
13.02	0.01\\
13.03	0.01\\
13.04	0.01\\
13.05	0.01\\
13.06	0.01\\
13.07	0.01\\
13.08	0.01\\
13.09	0.01\\
13.1	0.01\\
13.11	0.01\\
13.12	0.01\\
13.13	0.01\\
13.14	0.01\\
13.15	0.01\\
13.16	0.01\\
13.17	0.01\\
13.18	0.01\\
13.19	0.01\\
13.2	0.01\\
13.21	0.01\\
13.22	0.01\\
13.23	0.01\\
13.24	0.01\\
13.25	0.01\\
13.26	0.01\\
13.27	0.01\\
13.28	0.01\\
13.29	0.01\\
13.3	0.01\\
13.31	0.01\\
13.32	0.01\\
13.33	0.01\\
13.34	0.01\\
13.35	0.01\\
13.36	0.01\\
13.37	0.01\\
13.38	0.01\\
13.39	0.01\\
13.4	0.01\\
13.41	0.01\\
13.42	0.01\\
13.43	0.01\\
13.44	0.01\\
13.45	0.01\\
13.46	0.01\\
13.47	0.01\\
13.48	0.01\\
13.49	0.01\\
13.5	0.01\\
13.51	0.01\\
13.52	0.01\\
13.53	0.01\\
13.54	0.01\\
13.55	0.01\\
13.56	0.01\\
13.57	0.01\\
13.58	0.01\\
13.59	0.01\\
13.6	0.01\\
13.61	0.01\\
13.62	0.01\\
13.63	0.01\\
13.64	0.01\\
13.65	0.01\\
13.66	0.01\\
13.67	0.01\\
13.68	0.01\\
13.69	0.01\\
13.7	0.01\\
13.71	0.01\\
13.72	0.01\\
13.73	0.01\\
13.74	0.01\\
13.75	0.01\\
13.76	0.01\\
13.77	0.01\\
13.78	0.01\\
13.79	0.01\\
13.8	0.01\\
13.81	0.01\\
13.82	0.01\\
13.83	0.01\\
13.84	0.01\\
13.85	0.01\\
13.86	0.01\\
13.87	0.01\\
13.88	0.01\\
13.89	0.01\\
13.9	0.01\\
13.91	0.01\\
13.92	0.01\\
13.93	0.01\\
13.94	0.01\\
13.95	0.01\\
13.96	0.01\\
13.97	0.01\\
13.98	0.01\\
13.99	0.01\\
14	0.01\\
14.01	0.01\\
14.02	0.01\\
14.03	0.01\\
14.04	0.01\\
14.05	0.01\\
14.06	0.01\\
14.07	0.01\\
14.08	0.01\\
14.09	0.01\\
14.1	0.01\\
14.11	0.01\\
14.12	0.01\\
14.13	0.01\\
14.14	0.01\\
14.15	0.01\\
14.16	0.01\\
14.17	0.01\\
14.18	0.01\\
14.19	0.01\\
14.2	0.01\\
14.21	0.01\\
14.22	0.01\\
14.23	0.01\\
14.24	0.01\\
14.25	0.01\\
14.26	0.01\\
14.27	0.01\\
14.28	0.01\\
14.29	0.01\\
14.3	0.01\\
14.31	0.01\\
14.32	0.01\\
14.33	0.01\\
14.34	0.01\\
14.35	0.01\\
14.36	0.01\\
14.37	0.01\\
14.38	0.01\\
14.39	0.01\\
14.4	0.01\\
14.41	0.01\\
14.42	0.01\\
14.43	0.01\\
14.44	0.01\\
14.45	0.01\\
14.46	0.01\\
14.47	0.01\\
14.48	0.01\\
14.49	0.01\\
14.5	0.01\\
14.51	0.01\\
14.52	0.01\\
14.53	0.01\\
14.54	0.01\\
14.55	0.01\\
14.56	0.01\\
14.57	0.01\\
14.58	0.01\\
14.59	0.01\\
14.6	0.01\\
14.61	0.01\\
14.62	0.01\\
14.63	0.01\\
14.64	0.01\\
14.65	0.01\\
14.66	0.01\\
14.67	0.01\\
14.68	0.01\\
14.69	0.01\\
14.7	0.01\\
14.71	0.01\\
14.72	0.01\\
14.73	0.01\\
14.74	0.01\\
14.75	0.01\\
14.76	0.01\\
14.77	0.01\\
14.78	0.01\\
14.79	0.01\\
14.8	0.01\\
14.81	0.01\\
14.82	0.01\\
14.83	0.01\\
14.84	0.01\\
14.85	0.01\\
14.86	0.01\\
14.87	0.01\\
14.88	0.01\\
14.89	0.01\\
14.9	0.01\\
14.91	0.01\\
14.92	0.01\\
14.93	0.01\\
14.94	0.01\\
14.95	0.01\\
14.96	0.01\\
14.97	0.01\\
14.98	0.01\\
14.99	0.01\\
15	0.01\\
15.01	0.01\\
15.02	0.01\\
15.03	0.01\\
15.04	0.01\\
15.05	0.01\\
15.06	0.01\\
15.07	0.01\\
15.08	0.01\\
15.09	0.01\\
15.1	0.01\\
15.11	0.01\\
15.12	0.01\\
15.13	0.01\\
15.14	0.01\\
15.15	0.01\\
15.16	0.01\\
15.17	0.01\\
15.18	0.01\\
15.19	0.01\\
15.2	0.01\\
15.21	0.01\\
15.22	0.01\\
15.23	0.01\\
15.24	0.01\\
15.25	0.01\\
15.26	0.01\\
15.27	0.01\\
15.28	0.01\\
15.29	0.01\\
15.3	0.01\\
15.31	0.01\\
15.32	0.01\\
15.33	0.01\\
15.34	0.01\\
15.35	0.01\\
15.36	0.01\\
15.37	0.01\\
15.38	0.01\\
15.39	0.01\\
15.4	0.01\\
15.41	0.01\\
15.42	0.01\\
15.43	0.01\\
15.44	0.01\\
15.45	0.01\\
15.46	0.01\\
15.47	0.01\\
15.48	0.01\\
15.49	0.01\\
15.5	0.01\\
15.51	0.01\\
15.52	0.01\\
15.53	0.01\\
15.54	0.01\\
15.55	0.01\\
15.56	0.01\\
15.57	0.01\\
15.58	0.01\\
15.59	0.01\\
15.6	0.01\\
15.61	0.01\\
15.62	0.01\\
15.63	0.01\\
15.64	0.01\\
15.65	0.01\\
15.66	0.01\\
15.67	0.01\\
15.68	0.01\\
15.69	0.01\\
15.7	0.01\\
15.71	0.01\\
15.72	0.01\\
15.73	0.01\\
15.74	0.01\\
15.75	0.01\\
15.76	0.01\\
15.77	0.01\\
15.78	0.01\\
15.79	0.01\\
15.8	0.01\\
15.81	0.01\\
15.82	0.01\\
15.83	0.01\\
15.84	0.01\\
15.85	0.01\\
15.86	0.01\\
15.87	0.01\\
15.88	0.01\\
15.89	0.01\\
15.9	0.01\\
15.91	0.01\\
15.92	0.01\\
15.93	0.01\\
15.94	0.01\\
15.95	0.01\\
15.96	0.01\\
15.97	0.01\\
15.98	0.01\\
15.99	0.01\\
16	0.01\\
16.01	0.01\\
16.02	0.01\\
16.03	0.01\\
16.04	0.01\\
16.05	0.01\\
16.06	0.01\\
16.07	0.01\\
16.08	0.01\\
16.09	0.01\\
16.1	0.01\\
16.11	0.01\\
16.12	0.01\\
16.13	0.01\\
16.14	0.01\\
16.15	0.01\\
16.16	0.01\\
16.17	0.01\\
16.18	0.01\\
16.19	0.01\\
16.2	0.01\\
16.21	0.01\\
16.22	0.01\\
16.23	0.01\\
16.24	0.01\\
16.25	0.01\\
16.26	0.01\\
16.27	0.01\\
16.28	0.01\\
16.29	0.01\\
16.3	0.01\\
16.31	0.01\\
16.32	0.01\\
16.33	0.01\\
16.34	0.01\\
16.35	0.01\\
16.36	0.01\\
16.37	0.01\\
16.38	0.01\\
16.39	0.01\\
16.4	0.01\\
16.41	0.01\\
16.42	0.01\\
16.43	0.01\\
16.44	0.01\\
16.45	0.01\\
16.46	0.01\\
16.47	0.01\\
16.48	0.01\\
16.49	0.01\\
16.5	0.01\\
16.51	0.01\\
16.52	0.01\\
16.53	0.01\\
16.54	0.01\\
16.55	0.01\\
16.56	0.01\\
16.57	0.01\\
16.58	0.01\\
16.59	0.01\\
16.6	0.01\\
16.61	0.01\\
16.62	0.01\\
16.63	0.01\\
16.64	0.01\\
16.65	0.01\\
16.66	0.01\\
16.67	0.01\\
16.68	0.01\\
16.69	0.01\\
16.7	0.01\\
16.71	0.01\\
16.72	0.01\\
16.73	0.01\\
16.74	0.01\\
16.75	0.01\\
16.76	0.01\\
16.77	0.01\\
16.78	0.01\\
16.79	0.01\\
16.8	0.01\\
16.81	0.01\\
16.82	0.01\\
16.83	0.01\\
16.84	0.01\\
16.85	0.01\\
16.86	0.01\\
16.87	0.01\\
16.88	0.01\\
16.89	0.01\\
16.9	0.01\\
16.91	0.01\\
16.92	0.01\\
16.93	0.01\\
16.94	0.01\\
16.95	0.01\\
16.96	0.01\\
16.97	0.01\\
16.98	0.01\\
16.99	0.01\\
17	0.01\\
17.01	0.01\\
17.02	0.01\\
17.03	0.01\\
17.04	0.01\\
17.05	0.01\\
17.06	0.01\\
17.07	0.01\\
17.08	0.01\\
17.09	0.01\\
17.1	0.01\\
17.11	0.01\\
17.12	0.01\\
17.13	0.01\\
17.14	0.01\\
17.15	0.01\\
17.16	0.01\\
17.17	0.01\\
17.18	0.01\\
17.19	0.01\\
17.2	0.01\\
17.21	0.01\\
17.22	0.01\\
17.23	0.01\\
17.24	0.01\\
17.25	0.01\\
17.26	0.01\\
17.27	0.01\\
17.28	0.01\\
17.29	0.01\\
17.3	0.01\\
17.31	0.01\\
17.32	0.01\\
17.33	0.01\\
17.34	0.01\\
17.35	0.01\\
17.36	0.01\\
17.37	0.01\\
17.38	0.01\\
17.39	0.01\\
17.4	0.01\\
17.41	0.01\\
17.42	0.01\\
17.43	0.01\\
17.44	0.01\\
17.45	0.01\\
17.46	0.01\\
17.47	0.01\\
17.48	0.01\\
17.49	0.01\\
17.5	0.01\\
17.51	0.01\\
17.52	0.01\\
17.53	0.01\\
17.54	0.01\\
17.55	0.01\\
17.56	0.01\\
17.57	0.01\\
17.58	0.01\\
17.59	0.01\\
17.6	0.01\\
17.61	0.01\\
17.62	0.01\\
17.63	0.01\\
17.64	0.01\\
17.65	0.01\\
17.66	0.01\\
17.67	0.01\\
17.68	0.01\\
17.69	0.01\\
17.7	0.01\\
17.71	0.01\\
17.72	0.01\\
17.73	0.01\\
17.74	0.01\\
17.75	0.01\\
17.76	0.01\\
17.77	0.01\\
17.78	0.01\\
17.79	0.01\\
17.8	0.01\\
17.81	0.01\\
17.82	0.01\\
17.83	0.01\\
17.84	0.01\\
17.85	0.01\\
17.86	0.01\\
17.87	0.01\\
17.88	0.01\\
17.89	0.01\\
17.9	0.01\\
17.91	0.01\\
17.92	0.01\\
17.93	0.01\\
17.94	0.01\\
17.95	0.01\\
17.96	0.01\\
17.97	0.01\\
17.98	0.01\\
17.99	0.01\\
18	0.01\\
18.01	0.01\\
18.02	0.01\\
18.03	0.01\\
18.04	0.01\\
18.05	0.01\\
18.06	0.01\\
18.07	0.01\\
18.08	0.01\\
18.09	0.01\\
18.1	0.01\\
18.11	0.01\\
18.12	0.01\\
18.13	0.01\\
18.14	0.01\\
18.15	0.01\\
18.16	0.01\\
18.17	0.01\\
18.18	0.01\\
18.19	0.01\\
18.2	0.01\\
18.21	0.01\\
18.22	0.01\\
18.23	0.01\\
18.24	0.01\\
18.25	0.01\\
18.26	0.01\\
18.27	0.01\\
18.28	0.01\\
18.29	0.01\\
18.3	0.01\\
18.31	0.01\\
18.32	0.01\\
18.33	0.01\\
18.34	0.01\\
18.35	0.01\\
18.36	0.01\\
18.37	0.01\\
18.38	0.01\\
18.39	0.01\\
18.4	0.01\\
18.41	0.01\\
18.42	0.01\\
18.43	0.01\\
18.44	0.01\\
18.45	0.01\\
18.46	0.01\\
18.47	0.01\\
18.48	0.01\\
18.49	0.01\\
18.5	0.01\\
18.51	0.01\\
18.52	0.01\\
18.53	0.01\\
18.54	0.01\\
18.55	0.01\\
18.56	0.01\\
18.57	0.01\\
18.58	0.01\\
18.59	0.01\\
18.6	0.01\\
18.61	0.01\\
18.62	0.01\\
18.63	0.01\\
18.64	0.01\\
18.65	0.01\\
18.66	0.01\\
18.67	0.01\\
18.68	0.01\\
18.69	0.01\\
18.7	0.01\\
18.71	0.01\\
18.72	0.01\\
18.73	0.01\\
18.74	0.01\\
18.75	0.01\\
18.76	0.01\\
18.77	0.01\\
18.78	0.01\\
18.79	0.01\\
18.8	0.01\\
18.81	0.01\\
18.82	0.01\\
18.83	0.01\\
18.84	0.01\\
18.85	0.01\\
18.86	0.01\\
18.87	0.01\\
18.88	0.01\\
18.89	0.01\\
18.9	0.01\\
18.91	0.01\\
18.92	0.01\\
18.93	0.01\\
18.94	0.01\\
18.95	0.01\\
18.96	0.01\\
18.97	0.01\\
18.98	0.01\\
18.99	0.01\\
19	0.01\\
19.01	0.01\\
19.02	0.01\\
19.03	0.01\\
19.04	0.01\\
19.05	0.01\\
19.06	0.01\\
19.07	0.01\\
19.08	0.01\\
19.09	0.01\\
19.1	0.01\\
19.11	0.01\\
19.12	0.01\\
19.13	0.01\\
19.14	0.01\\
19.15	0.01\\
19.16	0.01\\
19.17	0.01\\
19.18	0.01\\
19.19	0.01\\
19.2	0.01\\
19.21	0.01\\
19.22	0.01\\
19.23	0.01\\
19.24	0.01\\
19.25	0.01\\
19.26	0.01\\
19.27	0.01\\
19.28	0.01\\
19.29	0.01\\
19.3	0.01\\
19.31	0.01\\
19.32	0.01\\
19.33	0.01\\
19.34	0.01\\
19.35	0.01\\
19.36	0.01\\
19.37	0.01\\
19.38	0.01\\
19.39	0.01\\
19.4	0.01\\
19.41	0.01\\
19.42	0.01\\
19.43	0.01\\
19.44	0.01\\
19.45	0.01\\
19.46	0.01\\
19.47	0.01\\
19.48	0.01\\
19.49	0.01\\
19.5	0.01\\
19.51	0.01\\
19.52	0.01\\
19.53	0.01\\
19.54	0.01\\
19.55	0.01\\
19.56	0.01\\
19.57	0.01\\
19.58	0.01\\
19.59	0.01\\
19.6	0.01\\
19.61	0.01\\
19.62	0.01\\
19.63	0.01\\
19.64	0.01\\
19.65	0.01\\
19.66	0.01\\
19.67	0.01\\
19.68	0.01\\
19.69	0.01\\
19.7	0.01\\
19.71	0.01\\
19.72	0.01\\
19.73	0.01\\
19.74	0.01\\
19.75	0.01\\
19.76	0.01\\
19.77	0.01\\
19.78	0.01\\
19.79	0.01\\
19.8	0.01\\
19.81	0.01\\
19.82	0.01\\
19.83	0.01\\
19.84	0.01\\
19.85	0.01\\
19.86	0.01\\
19.87	0.01\\
19.88	0.01\\
19.89	0.01\\
19.9	0.01\\
19.91	0.01\\
19.92	0.01\\
19.93	0.01\\
19.94	0.01\\
19.95	0.01\\
19.96	0.01\\
19.97	0.01\\
19.98	0.01\\
19.99	0.01\\
20	0.01\\
20.01	0.01\\
20.02	0.01\\
20.03	0.01\\
20.04	0.01\\
20.05	0.01\\
20.06	0.01\\
20.07	0.01\\
20.08	0.01\\
20.09	0.01\\
20.1	0.01\\
20.11	0.01\\
20.12	0.01\\
20.13	0.01\\
20.14	0.01\\
20.15	0.01\\
20.16	0.01\\
20.17	0.01\\
20.18	0.01\\
20.19	0.01\\
20.2	0.01\\
20.21	0.01\\
20.22	0.01\\
20.23	0.01\\
20.24	0.01\\
20.25	0.01\\
20.26	0.01\\
20.27	0.01\\
20.28	0.01\\
20.29	0.01\\
20.3	0.01\\
20.31	0.01\\
20.32	0.01\\
20.33	0.01\\
20.34	0.01\\
20.35	0.01\\
20.36	0.01\\
20.37	0.01\\
20.38	0.01\\
20.39	0.01\\
20.4	0.01\\
20.41	0.01\\
20.42	0.01\\
20.43	0.01\\
20.44	0.01\\
20.45	0.01\\
20.46	0.01\\
20.47	0.01\\
20.48	0.01\\
20.49	0.01\\
20.5	0.01\\
20.51	0.01\\
20.52	0.01\\
20.53	0.01\\
20.54	0.01\\
20.55	0.01\\
20.56	0.01\\
20.57	0.01\\
20.58	0.01\\
20.59	0.01\\
20.6	0.01\\
20.61	0.01\\
20.62	0.01\\
20.63	0.01\\
20.64	0.01\\
20.65	0.01\\
20.66	0.01\\
20.67	0.01\\
20.68	0.01\\
20.69	0.01\\
20.7	0.01\\
20.71	0.01\\
20.72	0.01\\
20.73	0.01\\
20.74	0.01\\
20.75	0.01\\
20.76	0.01\\
20.77	0.01\\
20.78	0.01\\
20.79	0.01\\
20.8	0.01\\
20.81	0.01\\
20.82	0.01\\
20.83	0.01\\
20.84	0.01\\
20.85	0.01\\
20.86	0.01\\
20.87	0.01\\
20.88	0.01\\
20.89	0.01\\
20.9	0.01\\
20.91	0.01\\
20.92	0.01\\
20.93	0.01\\
20.94	0.01\\
20.95	0.01\\
20.96	0.01\\
20.97	0.01\\
20.98	0.01\\
20.99	0.01\\
21	0.01\\
21.01	0.01\\
21.02	0.01\\
21.03	0.01\\
21.04	0.01\\
21.05	0.01\\
21.06	0.01\\
21.07	0.01\\
21.08	0.01\\
21.09	0.01\\
21.1	0.01\\
21.11	0.01\\
21.12	0.01\\
21.13	0.01\\
21.14	0.01\\
21.15	0.01\\
21.16	0.01\\
21.17	0.01\\
21.18	0.01\\
21.19	0.01\\
21.2	0.01\\
21.21	0.01\\
21.22	0.01\\
21.23	0.01\\
21.24	0.01\\
21.25	0.01\\
21.26	0.01\\
21.27	0.01\\
21.28	0.01\\
21.29	0.01\\
21.3	0.01\\
21.31	0.01\\
21.32	0.01\\
21.33	0.01\\
21.34	0.01\\
21.35	0.01\\
21.36	0.01\\
21.37	0.01\\
21.38	0.01\\
21.39	0.01\\
21.4	0.01\\
21.41	0.01\\
21.42	0.01\\
21.43	0.01\\
21.44	0.01\\
21.45	0.01\\
21.46	0.01\\
21.47	0.01\\
21.48	0.01\\
21.49	0.01\\
21.5	0.01\\
21.51	0.01\\
21.52	0.01\\
21.53	0.01\\
21.54	0.01\\
21.55	0.01\\
21.56	0.01\\
21.57	0.01\\
21.58	0.01\\
21.59	0.01\\
21.6	0.01\\
21.61	0.01\\
21.62	0.01\\
21.63	0.01\\
21.64	0.01\\
21.65	0.01\\
21.66	0.01\\
21.67	0.01\\
21.68	0.01\\
21.69	0.01\\
21.7	0.01\\
21.71	0.01\\
21.72	0.01\\
21.73	0.01\\
21.74	0.01\\
21.75	0.01\\
21.76	0.01\\
21.77	0.01\\
21.78	0.01\\
21.79	0.01\\
21.8	0.01\\
21.81	0.01\\
21.82	0.01\\
21.83	0.01\\
21.84	0.01\\
21.85	0.01\\
21.86	0.01\\
21.87	0.01\\
21.88	0.01\\
21.89	0.01\\
21.9	0.01\\
21.91	0.01\\
21.92	0.01\\
21.93	0.01\\
21.94	0.01\\
21.95	0.01\\
21.96	0.01\\
21.97	0.01\\
21.98	0.01\\
21.99	0.01\\
22	0.01\\
22.01	0.01\\
22.02	0.01\\
22.03	0.01\\
22.04	0.01\\
22.05	0.01\\
22.06	0.01\\
22.07	0.01\\
22.08	0.01\\
22.09	0.01\\
22.1	0.01\\
22.11	0.01\\
22.12	0.01\\
22.13	0.01\\
22.14	0.01\\
22.15	0.01\\
22.16	0.01\\
22.17	0.01\\
22.18	0.01\\
22.19	0.01\\
22.2	0.01\\
22.21	0.01\\
22.22	0.01\\
22.23	0.01\\
22.24	0.01\\
22.25	0.01\\
22.26	0.01\\
22.27	0.01\\
22.28	0.01\\
22.29	0.01\\
22.3	0.01\\
22.31	0.01\\
22.32	0.01\\
22.33	0.01\\
22.34	0.01\\
22.35	0.01\\
22.36	0.01\\
22.37	0.01\\
22.38	0.01\\
22.39	0.01\\
22.4	0.01\\
22.41	0.01\\
22.42	0.01\\
22.43	0.01\\
22.44	0.01\\
22.45	0.01\\
22.46	0.01\\
22.47	0.01\\
22.48	0.01\\
22.49	0.01\\
22.5	0.01\\
22.51	0.01\\
22.52	0.01\\
22.53	0.01\\
22.54	0.01\\
22.55	0.01\\
22.56	0.01\\
22.57	0.01\\
22.58	0.01\\
22.59	0.01\\
22.6	0.01\\
22.61	0.01\\
22.62	0.01\\
22.63	0.01\\
22.64	0.01\\
22.65	0.01\\
22.66	0.01\\
22.67	0.01\\
22.68	0.01\\
22.69	0.01\\
22.7	0.01\\
22.71	0.01\\
22.72	0.01\\
22.73	0.01\\
22.74	0.01\\
22.75	0.01\\
22.76	0.01\\
22.77	0.01\\
22.78	0.01\\
22.79	0.01\\
22.8	0.01\\
22.81	0.01\\
22.82	0.01\\
22.83	0.01\\
22.84	0.01\\
22.85	0.01\\
22.86	0.01\\
22.87	0.01\\
22.88	0.01\\
22.89	0.01\\
22.9	0.01\\
22.91	0.01\\
22.92	0.01\\
22.93	0.01\\
22.94	0.01\\
22.95	0.01\\
22.96	0.01\\
22.97	0.01\\
22.98	0.01\\
22.99	0.01\\
23	0.01\\
23.01	0.01\\
23.02	0.01\\
23.03	0.01\\
23.04	0.01\\
23.05	0.01\\
23.06	0.01\\
23.07	0.01\\
23.08	0.01\\
23.09	0.01\\
23.1	0.01\\
23.11	0.01\\
23.12	0.01\\
23.13	0.01\\
23.14	0.01\\
23.15	0.01\\
23.16	0.01\\
23.17	0.01\\
23.18	0.01\\
23.19	0.01\\
23.2	0.01\\
23.21	0.01\\
23.22	0.01\\
23.23	0.01\\
23.24	0.01\\
23.25	0.01\\
23.26	0.01\\
23.27	0.01\\
23.28	0.01\\
23.29	0.01\\
23.3	0.01\\
23.31	0.01\\
23.32	0.01\\
23.33	0.01\\
23.34	0.01\\
23.35	0.01\\
23.36	0.01\\
23.37	0.01\\
23.38	0.01\\
23.39	0.01\\
23.4	0.01\\
23.41	0.01\\
23.42	0.01\\
23.43	0.01\\
23.44	0.01\\
23.45	0.01\\
23.46	0.01\\
23.47	0.01\\
23.48	0.01\\
23.49	0.01\\
23.5	0.01\\
23.51	0.01\\
23.52	0.01\\
23.53	0.01\\
23.54	0.01\\
23.55	0.01\\
23.56	0.01\\
23.57	0.01\\
23.58	0.01\\
23.59	0.01\\
23.6	0.01\\
23.61	0.01\\
23.62	0.01\\
23.63	0.01\\
23.64	0.01\\
23.65	0.01\\
23.66	0.01\\
23.67	0.01\\
23.68	0.01\\
23.69	0.01\\
23.7	0.01\\
23.71	0.01\\
23.72	0.01\\
23.73	0.01\\
23.74	0.01\\
23.75	0.01\\
23.76	0.01\\
23.77	0.01\\
23.78	0.01\\
23.79	0.01\\
23.8	0.01\\
23.81	0.01\\
23.82	0.01\\
23.83	0.01\\
23.84	0.01\\
23.85	0.01\\
23.86	0.01\\
23.87	0.01\\
23.88	0.01\\
23.89	0.01\\
23.9	0.01\\
23.91	0.01\\
23.92	0.01\\
23.93	0.01\\
23.94	0.01\\
23.95	0.01\\
23.96	0.01\\
23.97	0.01\\
23.98	0.01\\
23.99	0.01\\
24	0.01\\
24.01	0.01\\
24.02	0.01\\
24.03	0.01\\
24.04	0.01\\
24.05	0.01\\
24.06	0.01\\
24.07	0.01\\
24.08	0.01\\
24.09	0.01\\
24.1	0.01\\
24.11	0.01\\
24.12	0.01\\
24.13	0.01\\
24.14	0.01\\
24.15	0.01\\
24.16	0.01\\
24.17	0.01\\
24.18	0.01\\
24.19	0.01\\
24.2	0.01\\
24.21	0.01\\
24.22	0.01\\
24.23	0.01\\
24.24	0.01\\
24.25	0.01\\
24.26	0.01\\
24.27	0.01\\
24.28	0.01\\
24.29	0.01\\
24.3	0.01\\
24.31	0.01\\
24.32	0.01\\
24.33	0.01\\
24.34	0.01\\
24.35	0.01\\
24.36	0.01\\
24.37	0.01\\
24.38	0.01\\
24.39	0.01\\
24.4	0.01\\
24.41	0.01\\
24.42	0.01\\
24.43	0.01\\
24.44	0.01\\
24.45	0.01\\
24.46	0.01\\
24.47	0.01\\
24.48	0.01\\
24.49	0.01\\
24.5	0.01\\
24.51	0.01\\
24.52	0.01\\
24.53	0.01\\
24.54	0.01\\
24.55	0.01\\
24.56	0.01\\
24.57	0.01\\
24.58	0.01\\
24.59	0.01\\
24.6	0.01\\
24.61	0.01\\
24.62	0.01\\
24.63	0.01\\
24.64	0.01\\
24.65	0.01\\
24.66	0.01\\
24.67	0.01\\
24.68	0.01\\
24.69	0.01\\
24.7	0.01\\
24.71	0.01\\
24.72	0.01\\
24.73	0.01\\
24.74	0.01\\
24.75	0.01\\
24.76	0.01\\
24.77	0.01\\
24.78	0.01\\
24.79	0.01\\
24.8	0.01\\
24.81	0.01\\
24.82	0.01\\
24.83	0.01\\
24.84	0.01\\
24.85	0.01\\
24.86	0.01\\
24.87	0.01\\
24.88	0.01\\
24.89	0.01\\
24.9	0.01\\
24.91	0.01\\
24.92	0.01\\
24.93	0.01\\
24.94	0.01\\
24.95	0.01\\
24.96	0.01\\
24.97	0.01\\
24.98	0.01\\
24.99	0.01\\
25	0.01\\
25.01	0.01\\
25.02	0.01\\
25.03	0.01\\
25.04	0.01\\
25.05	0.01\\
25.06	0.01\\
25.07	0.01\\
25.08	0.01\\
25.09	0.01\\
25.1	0.01\\
25.11	0.01\\
25.12	0.01\\
25.13	0.01\\
25.14	0.01\\
25.15	0.01\\
25.16	0.01\\
25.17	0.01\\
25.18	0.01\\
25.19	0.01\\
25.2	0.01\\
25.21	0.01\\
25.22	0.01\\
25.23	0.01\\
25.24	0.01\\
25.25	0.01\\
25.26	0.01\\
25.27	0.01\\
25.28	0.01\\
25.29	0.01\\
25.3	0.01\\
25.31	0.01\\
25.32	0.01\\
25.33	0.01\\
25.34	0.01\\
25.35	0.01\\
25.36	0.01\\
25.37	0.01\\
25.38	0.01\\
25.39	0.01\\
25.4	0.01\\
25.41	0.01\\
25.42	0.01\\
25.43	0.01\\
25.44	0.01\\
25.45	0.01\\
25.46	0.01\\
25.47	0.01\\
25.48	0.01\\
25.49	0.01\\
25.5	0.01\\
25.51	0.01\\
25.52	0.01\\
25.53	0.01\\
25.54	0.01\\
25.55	0.01\\
25.56	0.01\\
25.57	0.01\\
25.58	0.01\\
25.59	0.01\\
25.6	0.01\\
25.61	0.01\\
25.62	0.01\\
25.63	0.01\\
25.64	0.01\\
25.65	0.01\\
25.66	0.01\\
25.67	0.01\\
25.68	0.01\\
25.69	0.01\\
25.7	0.01\\
25.71	0.01\\
25.72	0.01\\
25.73	0.01\\
25.74	0.01\\
25.75	0.01\\
25.76	0.01\\
25.77	0.01\\
25.78	0.01\\
25.79	0.01\\
25.8	0.01\\
25.81	0.01\\
25.82	0.01\\
25.83	0.01\\
25.84	0.01\\
25.85	0.01\\
25.86	0.01\\
25.87	0.01\\
25.88	0.01\\
25.89	0.01\\
25.9	0.01\\
25.91	0.01\\
25.92	0.01\\
25.93	0.01\\
25.94	0.01\\
25.95	0.01\\
25.96	0.01\\
25.97	0.01\\
25.98	0.01\\
25.99	0.01\\
26	0.01\\
26.01	0.01\\
26.02	0.01\\
26.03	0.01\\
26.04	0.01\\
26.05	0.01\\
26.06	0.01\\
26.07	0.01\\
26.08	0.01\\
26.09	0.01\\
26.1	0.01\\
26.11	0.01\\
26.12	0.01\\
26.13	0.01\\
26.14	0.01\\
26.15	0.01\\
26.16	0.01\\
26.17	0.01\\
26.18	0.01\\
26.19	0.01\\
26.2	0.01\\
26.21	0.01\\
26.22	0.01\\
26.23	0.01\\
26.24	0.01\\
26.25	0.01\\
26.26	0.01\\
26.27	0.01\\
26.28	0.01\\
26.29	0.01\\
26.3	0.01\\
26.31	0.01\\
26.32	0.01\\
26.33	0.01\\
26.34	0.01\\
26.35	0.01\\
26.36	0.01\\
26.37	0.01\\
26.38	0.01\\
26.39	0.01\\
26.4	0.01\\
26.41	0.01\\
26.42	0.01\\
26.43	0.01\\
26.44	0.01\\
26.45	0.01\\
26.46	0.01\\
26.47	0.01\\
26.48	0.01\\
26.49	0.01\\
26.5	0.01\\
26.51	0.01\\
26.52	0.01\\
26.53	0.01\\
26.54	0.01\\
26.55	0.01\\
26.56	0.01\\
26.57	0.01\\
26.58	0.01\\
26.59	0.01\\
26.6	0.01\\
26.61	0.01\\
26.62	0.01\\
26.63	0.01\\
26.64	0.01\\
26.65	0.01\\
26.66	0.01\\
26.67	0.01\\
26.68	0.01\\
26.69	0.01\\
26.7	0.01\\
26.71	0.01\\
26.72	0.01\\
26.73	0.01\\
26.74	0.01\\
26.75	0.01\\
26.76	0.01\\
26.77	0.01\\
26.78	0.01\\
26.79	0.01\\
26.8	0.01\\
26.81	0.01\\
26.82	0.01\\
26.83	0.01\\
26.84	0.01\\
26.85	0.01\\
26.86	0.01\\
26.87	0.01\\
26.88	0.01\\
26.89	0.01\\
26.9	0.01\\
26.91	0.01\\
26.92	0.01\\
26.93	0.01\\
26.94	0.01\\
26.95	0.01\\
26.96	0.01\\
26.97	0.01\\
26.98	0.01\\
26.99	0.01\\
27	0.01\\
27.01	0.01\\
27.02	0.01\\
27.03	0.01\\
27.04	0.01\\
27.05	0.01\\
27.06	0.01\\
27.07	0.01\\
27.08	0.01\\
27.09	0.01\\
27.1	0.01\\
27.11	0.01\\
27.12	0.01\\
27.13	0.01\\
27.14	0.01\\
27.15	0.01\\
27.16	0.01\\
27.17	0.01\\
27.18	0.01\\
27.19	0.01\\
27.2	0.01\\
27.21	0.01\\
27.22	0.01\\
27.23	0.01\\
27.24	0.01\\
27.25	0.01\\
27.26	0.01\\
27.27	0.01\\
27.28	0.01\\
27.29	0.01\\
27.3	0.01\\
27.31	0.01\\
27.32	0.01\\
27.33	0.01\\
27.34	0.01\\
27.35	0.01\\
27.36	0.01\\
27.37	0.01\\
27.38	0.01\\
27.39	0.01\\
27.4	0.01\\
27.41	0.01\\
27.42	0.01\\
27.43	0.01\\
27.44	0.01\\
27.45	0.01\\
27.46	0.01\\
27.47	0.01\\
27.48	0.01\\
27.49	0.01\\
27.5	0.01\\
27.51	0.01\\
27.52	0.01\\
27.53	0.01\\
27.54	0.01\\
27.55	0.01\\
27.56	0.01\\
27.57	0.01\\
27.58	0.01\\
27.59	0.01\\
27.6	0.01\\
27.61	0.01\\
27.62	0.01\\
27.63	0.01\\
27.64	0.01\\
27.65	0.01\\
27.66	0.01\\
27.67	0.01\\
27.68	0.01\\
27.69	0.01\\
27.7	0.01\\
27.71	0.01\\
27.72	0.01\\
27.73	0.01\\
27.74	0.01\\
27.75	0.01\\
27.76	0.01\\
27.77	0.01\\
27.78	0.01\\
27.79	0.01\\
27.8	0.01\\
27.81	0.01\\
27.82	0.01\\
27.83	0.01\\
27.84	0.01\\
27.85	0.01\\
27.86	0.01\\
27.87	0.01\\
27.88	0.01\\
27.89	0.01\\
27.9	0.01\\
27.91	0.01\\
27.92	0.01\\
27.93	0.01\\
27.94	0.01\\
27.95	0.01\\
27.96	0.01\\
27.97	0.01\\
27.98	0.01\\
27.99	0.01\\
28	0.01\\
28.01	0.01\\
28.02	0.01\\
28.03	0.01\\
28.04	0.01\\
28.05	0.01\\
28.06	0.01\\
28.07	0.01\\
28.08	0.01\\
28.09	0.01\\
28.1	0.01\\
28.11	0.01\\
28.12	0.01\\
28.13	0.01\\
28.14	0.01\\
28.15	0.01\\
28.16	0.01\\
28.17	0.01\\
28.18	0.01\\
28.19	0.01\\
28.2	0.01\\
28.21	0.01\\
28.22	0.01\\
28.23	0.01\\
28.24	0.01\\
28.25	0.01\\
28.26	0.01\\
28.27	0.01\\
28.28	0.01\\
28.29	0.01\\
28.3	0.01\\
28.31	0.01\\
28.32	0.01\\
28.33	0.01\\
28.34	0.01\\
28.35	0.01\\
28.36	0.01\\
28.37	0.01\\
28.38	0.01\\
28.39	0.01\\
28.4	0.01\\
28.41	0.01\\
28.42	0.01\\
28.43	0.01\\
28.44	0.01\\
28.45	0.01\\
28.46	0.01\\
28.47	0.01\\
28.48	0.01\\
28.49	0.01\\
28.5	0.01\\
28.51	0.01\\
28.52	0.01\\
28.53	0.01\\
28.54	0.01\\
28.55	0.01\\
28.56	0.01\\
28.57	0.01\\
28.58	0.01\\
28.59	0.01\\
28.6	0.01\\
28.61	0.01\\
28.62	0.01\\
28.63	0.01\\
28.64	0.01\\
28.65	0.01\\
28.66	0.01\\
28.67	0.01\\
28.68	0.01\\
28.69	0.01\\
28.7	0.01\\
28.71	0.01\\
28.72	0.01\\
28.73	0.01\\
28.74	0.01\\
28.75	0.01\\
28.76	0.01\\
28.77	0.01\\
28.78	0.01\\
28.79	0.01\\
28.8	0.01\\
28.81	0.01\\
28.82	0.01\\
28.83	0.01\\
28.84	0.01\\
28.85	0.01\\
28.86	0.01\\
28.87	0.01\\
28.88	0.01\\
28.89	0.01\\
28.9	0.01\\
28.91	0.01\\
28.92	0.01\\
28.93	0.01\\
28.94	0.01\\
28.95	0.01\\
28.96	0.01\\
28.97	0.01\\
28.98	0.01\\
28.99	0.01\\
29	0.01\\
29.01	0.01\\
29.02	0.01\\
29.03	0.01\\
29.04	0.01\\
29.05	0.01\\
29.06	0.01\\
29.07	0.01\\
29.08	0.01\\
29.09	0.01\\
29.1	0.01\\
29.11	0.01\\
29.12	0.01\\
29.13	0.01\\
29.14	0.01\\
29.15	0.01\\
29.16	0.01\\
29.17	0.01\\
29.18	0.01\\
29.19	0.01\\
29.2	0.01\\
29.21	0.01\\
29.22	0.01\\
29.23	0.01\\
29.24	0.01\\
29.25	0.01\\
29.26	0.01\\
29.27	0.01\\
29.28	0.01\\
29.29	0.01\\
29.3	0.01\\
29.31	0.01\\
29.32	0.01\\
29.33	0.01\\
29.34	0.01\\
29.35	0.01\\
29.36	0.01\\
29.37	0.01\\
29.38	0.01\\
29.39	0.01\\
29.4	0.01\\
29.41	0.01\\
29.42	0.01\\
29.43	0.01\\
29.44	0.01\\
29.45	0.01\\
29.46	0.01\\
29.47	0.01\\
29.48	0.01\\
29.49	0.01\\
29.5	0.01\\
29.51	0.01\\
29.52	0.01\\
29.53	0.01\\
29.54	0.01\\
29.55	0.01\\
29.56	0.01\\
29.57	0.01\\
29.58	0.01\\
29.59	0.01\\
29.6	0.01\\
29.61	0.01\\
29.62	0.01\\
29.63	0.01\\
29.64	0.01\\
29.65	0.01\\
29.66	0.01\\
29.67	0.01\\
29.68	0.01\\
29.69	0.01\\
29.7	0.01\\
29.71	0.01\\
29.72	0.01\\
29.73	0.01\\
29.74	0.01\\
29.75	0.01\\
29.76	0.01\\
29.77	0.01\\
29.78	0.01\\
29.79	0.01\\
29.8	0.01\\
29.81	0.01\\
29.82	0.01\\
29.83	0.01\\
29.84	0.01\\
29.85	0.01\\
29.86	0.01\\
29.87	0.01\\
29.88	0.01\\
29.89	0.01\\
29.9	0.01\\
29.91	0.01\\
29.92	0.01\\
29.93	0.01\\
29.94	0.01\\
29.95	0.01\\
29.96	0.01\\
29.97	0.01\\
29.98	0.01\\
29.99	0.01\\
30	0.01\\
30.01	0.01\\
30.02	0.01\\
30.03	0.01\\
30.04	0.01\\
30.05	0.01\\
30.06	0.01\\
30.07	0.01\\
30.08	0.01\\
30.09	0.01\\
30.1	0.01\\
30.11	0.01\\
30.12	0.01\\
30.13	0.01\\
30.14	0.01\\
30.15	0.01\\
30.16	0.01\\
30.17	0.01\\
30.18	0.01\\
30.19	0.01\\
30.2	0.01\\
30.21	0.01\\
30.22	0.01\\
30.23	0.01\\
30.24	0.01\\
30.25	0.01\\
30.26	0.01\\
30.27	0.01\\
30.28	0.01\\
30.29	0.01\\
30.3	0.01\\
30.31	0.01\\
30.32	0.01\\
30.33	0.01\\
30.34	0.01\\
30.35	0.01\\
30.36	0.01\\
30.37	0.01\\
30.38	0.01\\
30.39	0.01\\
30.4	0.01\\
30.41	0.01\\
30.42	0.01\\
30.43	0.01\\
30.44	0.01\\
30.45	0.01\\
30.46	0.01\\
30.47	0.01\\
30.48	0.01\\
30.49	0.01\\
30.5	0.01\\
30.51	0.01\\
30.52	0.01\\
30.53	0.01\\
30.54	0.01\\
30.55	0.01\\
30.56	0.01\\
30.57	0.01\\
30.58	0.01\\
30.59	0.01\\
30.6	0.01\\
30.61	0.01\\
30.62	0.01\\
30.63	0.01\\
30.64	0.01\\
30.65	0.01\\
30.66	0.01\\
30.67	0.01\\
30.68	0.01\\
30.69	0.01\\
30.7	0.01\\
30.71	0.01\\
30.72	0.01\\
30.73	0.01\\
30.74	0.01\\
30.75	0.01\\
30.76	0.01\\
30.77	0.01\\
30.78	0.01\\
30.79	0.01\\
30.8	0.01\\
30.81	0.01\\
30.82	0.01\\
30.83	0.01\\
30.84	0.01\\
30.85	0.01\\
30.86	0.01\\
30.87	0.01\\
30.88	0.01\\
30.89	0.01\\
30.9	0.01\\
30.91	0.01\\
30.92	0.01\\
30.93	0.01\\
30.94	0.01\\
30.95	0.01\\
30.96	0.01\\
30.97	0.01\\
30.98	0.01\\
30.99	0.01\\
31	0.01\\
31.01	0.01\\
31.02	0.01\\
31.03	0.01\\
31.04	0.01\\
31.05	0.01\\
31.06	0.01\\
31.07	0.01\\
31.08	0.01\\
31.09	0.01\\
31.1	0.01\\
31.11	0.01\\
31.12	0.01\\
31.13	0.01\\
31.14	0.01\\
31.15	0.01\\
31.16	0.01\\
31.17	0.01\\
31.18	0.01\\
31.19	0.01\\
31.2	0.01\\
31.21	0.01\\
31.22	0.01\\
31.23	0.01\\
31.24	0.01\\
31.25	0.01\\
31.26	0.01\\
31.27	0.01\\
31.28	0.01\\
31.29	0.01\\
31.3	0.01\\
31.31	0.01\\
31.32	0.01\\
31.33	0.01\\
31.34	0.01\\
31.35	0.01\\
31.36	0.01\\
31.37	0.01\\
31.38	0.01\\
31.39	0.01\\
31.4	0.01\\
31.41	0.01\\
31.42	0.01\\
31.43	0.01\\
31.44	0.01\\
31.45	0.01\\
31.46	0.01\\
31.47	0.01\\
31.48	0.01\\
31.49	0.01\\
31.5	0.01\\
31.51	0.01\\
31.52	0.01\\
31.53	0.01\\
31.54	0.01\\
31.55	0.01\\
31.56	0.01\\
31.57	0.01\\
31.58	0.01\\
31.59	0.01\\
31.6	0.01\\
31.61	0.01\\
31.62	0.01\\
31.63	0.01\\
31.64	0.01\\
31.65	0.01\\
31.66	0.01\\
31.67	0.01\\
31.68	0.01\\
31.69	0.01\\
31.7	0.01\\
31.71	0.01\\
31.72	0.01\\
31.73	0.01\\
31.74	0.01\\
31.75	0.01\\
31.76	0.01\\
31.77	0.01\\
31.78	0.01\\
31.79	0.01\\
31.8	0.01\\
31.81	0.01\\
31.82	0.01\\
31.83	0.01\\
31.84	0.01\\
31.85	0.01\\
31.86	0.01\\
31.87	0.01\\
31.88	0.01\\
31.89	0.01\\
31.9	0.01\\
31.91	0.01\\
31.92	0.01\\
31.93	0.01\\
31.94	0.01\\
31.95	0.01\\
31.96	0.01\\
31.97	0.01\\
31.98	0.01\\
31.99	0.01\\
32	0.01\\
32.01	0.01\\
32.02	0.01\\
32.03	0.01\\
32.04	0.01\\
32.05	0.01\\
32.06	0.01\\
32.07	0.01\\
32.08	0.01\\
32.09	0.01\\
32.1	0.01\\
32.11	0.01\\
32.12	0.01\\
32.13	0.01\\
32.14	0.01\\
32.15	0.01\\
32.16	0.01\\
32.17	0.01\\
32.18	0.01\\
32.19	0.01\\
32.2	0.01\\
32.21	0.01\\
32.22	0.01\\
32.23	0.01\\
32.24	0.01\\
32.25	0.01\\
32.26	0.01\\
32.27	0.01\\
32.28	0.01\\
32.29	0.01\\
32.3	0.01\\
32.31	0.01\\
32.32	0.01\\
32.33	0.01\\
32.34	0.01\\
32.35	0.01\\
32.36	0.01\\
32.37	0.01\\
32.38	0.01\\
32.39	0.01\\
32.4	0.01\\
32.41	0.01\\
32.42	0.01\\
32.43	0.01\\
32.44	0.01\\
32.45	0.01\\
32.46	0.01\\
32.47	0.01\\
32.48	0.01\\
32.49	0.01\\
32.5	0.01\\
32.51	0.01\\
32.52	0.01\\
32.53	0.01\\
32.54	0.01\\
32.55	0.01\\
32.56	0.01\\
32.57	0.01\\
32.58	0.01\\
32.59	0.01\\
32.6	0.01\\
32.61	0.01\\
32.62	0.01\\
32.63	0.01\\
32.64	0.01\\
32.65	0.01\\
32.66	0.01\\
32.67	0.01\\
32.68	0.01\\
32.69	0.01\\
32.7	0.01\\
32.71	0.01\\
32.72	0.01\\
32.73	0.01\\
32.74	0.01\\
32.75	0.01\\
32.76	0.01\\
32.77	0.01\\
32.78	0.01\\
32.79	0.01\\
32.8	0.01\\
32.81	0.01\\
32.82	0.01\\
32.83	0.01\\
32.84	0.01\\
32.85	0.01\\
32.86	0.01\\
32.87	0.01\\
32.88	0.01\\
32.89	0.01\\
32.9	0.01\\
32.91	0.01\\
32.92	0.01\\
32.93	0.01\\
32.94	0.01\\
32.95	0.01\\
32.96	0.01\\
32.97	0.01\\
32.98	0.01\\
32.99	0.01\\
33	0.01\\
33.01	0.01\\
33.02	0.01\\
33.03	0.01\\
33.04	0.01\\
33.05	0.01\\
33.06	0.01\\
33.07	0.01\\
33.08	0.01\\
33.09	0.01\\
33.1	0.01\\
33.11	0.01\\
33.12	0.01\\
33.13	0.01\\
33.14	0.01\\
33.15	0.01\\
33.16	0.01\\
33.17	0.01\\
33.18	0.01\\
33.19	0.01\\
33.2	0.01\\
33.21	0.01\\
33.22	0.01\\
33.23	0.01\\
33.24	0.01\\
33.25	0.01\\
33.26	0.01\\
33.27	0.01\\
33.28	0.01\\
33.29	0.01\\
33.3	0.01\\
33.31	0.01\\
33.32	0.01\\
33.33	0.01\\
33.34	0.01\\
33.35	0.01\\
33.36	0.01\\
33.37	0.01\\
33.38	0.01\\
33.39	0.01\\
33.4	0.01\\
33.41	0.01\\
33.42	0.01\\
33.43	0.01\\
33.44	0.01\\
33.45	0.01\\
33.46	0.01\\
33.47	0.01\\
33.48	0.01\\
33.49	0.01\\
33.5	0.01\\
33.51	0.01\\
33.52	0.01\\
33.53	0.01\\
33.54	0.01\\
33.55	0.01\\
33.56	0.01\\
33.57	0.01\\
33.58	0.01\\
33.59	0.01\\
33.6	0.01\\
33.61	0.01\\
33.62	0.01\\
33.63	0.01\\
33.64	0.01\\
33.65	0.01\\
33.66	0.01\\
33.67	0.01\\
33.68	0.01\\
33.69	0.01\\
33.7	0.01\\
33.71	0.01\\
33.72	0.01\\
33.73	0.01\\
33.74	0.01\\
33.75	0.01\\
33.76	0.01\\
33.77	0.01\\
33.78	0.01\\
33.79	0.01\\
33.8	0.01\\
33.81	0.01\\
33.82	0.01\\
33.83	0.01\\
33.84	0.01\\
33.85	0.01\\
33.86	0.01\\
33.87	0.01\\
33.88	0.01\\
33.89	0.01\\
33.9	0.01\\
33.91	0.01\\
33.92	0.01\\
33.93	0.01\\
33.94	0.01\\
33.95	0.01\\
33.96	0.01\\
33.97	0.01\\
33.98	0.01\\
33.99	0.01\\
34	0.01\\
34.01	0.01\\
34.02	0.01\\
34.03	0.01\\
34.04	0.01\\
34.05	0.01\\
34.06	0.01\\
34.07	0.01\\
34.08	0.01\\
34.09	0.01\\
34.1	0.01\\
34.11	0.01\\
34.12	0.01\\
34.13	0.01\\
34.14	0.01\\
34.15	0.01\\
34.16	0.01\\
34.17	0.01\\
34.18	0.01\\
34.19	0.01\\
34.2	0.01\\
34.21	0.01\\
34.22	0.01\\
34.23	0.01\\
34.24	0.01\\
34.25	0.01\\
34.26	0.01\\
34.27	0.01\\
34.28	0.01\\
34.29	0.01\\
34.3	0.01\\
34.31	0.01\\
34.32	0.01\\
34.33	0.01\\
34.34	0.01\\
34.35	0.01\\
34.36	0.01\\
34.37	0.01\\
34.38	0.01\\
34.39	0.01\\
34.4	0.01\\
34.41	0.01\\
34.42	0.01\\
34.43	0.01\\
34.44	0.01\\
34.45	0.01\\
34.46	0.01\\
34.47	0.01\\
34.48	0.01\\
34.49	0.01\\
34.5	0.01\\
34.51	0.01\\
34.52	0.01\\
34.53	0.01\\
34.54	0.01\\
34.55	0.01\\
34.56	0.01\\
34.57	0.01\\
34.58	0.01\\
34.59	0.01\\
34.6	0.01\\
34.61	0.01\\
34.62	0.01\\
34.63	0.01\\
34.64	0.01\\
34.65	0.01\\
34.66	0.01\\
34.67	0.01\\
34.68	0.01\\
34.69	0.01\\
34.7	0.01\\
34.71	0.01\\
34.72	0.01\\
34.73	0.01\\
34.74	0.01\\
34.75	0.01\\
34.76	0.01\\
34.77	0.01\\
34.78	0.01\\
34.79	0.01\\
34.8	0.01\\
34.81	0.01\\
34.82	0.01\\
34.83	0.01\\
34.84	0.01\\
34.85	0.01\\
34.86	0.01\\
34.87	0.01\\
34.88	0.01\\
34.89	0.01\\
34.9	0.01\\
34.91	0.01\\
34.92	0.01\\
34.93	0.01\\
34.94	0.01\\
34.95	0.01\\
34.96	0.01\\
34.97	0.01\\
34.98	0.01\\
34.99	0.01\\
35	0.01\\
35.01	0.01\\
35.02	0.01\\
35.03	0.01\\
35.04	0.01\\
35.05	0.01\\
35.06	0.01\\
35.07	0.01\\
35.08	0.01\\
35.09	0.01\\
35.1	0.01\\
35.11	0.01\\
35.12	0.01\\
35.13	0.01\\
35.14	0.01\\
35.15	0.01\\
35.16	0.01\\
35.17	0.01\\
35.18	0.01\\
35.19	0.01\\
35.2	0.01\\
35.21	0.01\\
35.22	0.01\\
35.23	0.01\\
35.24	0.01\\
35.25	0.01\\
35.26	0.01\\
35.27	0.01\\
35.28	0.01\\
35.29	0.01\\
35.3	0.01\\
35.31	0.01\\
35.32	0.01\\
35.33	0.01\\
35.34	0.01\\
35.35	0.01\\
35.36	0.01\\
35.37	0.01\\
35.38	0.01\\
35.39	0.01\\
35.4	0.01\\
35.41	0.01\\
35.42	0.01\\
35.43	0.01\\
35.44	0.01\\
35.45	0.01\\
35.46	0.01\\
35.47	0.01\\
35.48	0.01\\
35.49	0.01\\
35.5	0.01\\
35.51	0.01\\
35.52	0.01\\
35.53	0.01\\
35.54	0.01\\
35.55	0.01\\
35.56	0.01\\
35.57	0.01\\
35.58	0.01\\
35.59	0.01\\
35.6	0.01\\
35.61	0.01\\
35.62	0.01\\
35.63	0.01\\
35.64	0.01\\
35.65	0.01\\
35.66	0.01\\
35.67	0.01\\
35.68	0.01\\
35.69	0.01\\
35.7	0.01\\
35.71	0.01\\
35.72	0.01\\
35.73	0.01\\
35.74	0.01\\
35.75	0.01\\
35.76	0.01\\
35.77	0.01\\
35.78	0.01\\
35.79	0.01\\
35.8	0.01\\
35.81	0.01\\
35.82	0.01\\
35.83	0.01\\
35.84	0.01\\
35.85	0.01\\
35.86	0.01\\
35.87	0.01\\
35.88	0.01\\
35.89	0.01\\
35.9	0.01\\
35.91	0.01\\
35.92	0.01\\
35.93	0.01\\
35.94	0.01\\
35.95	0.01\\
35.96	0.01\\
35.97	0.01\\
35.98	0.01\\
35.99	0.01\\
36	0.01\\
36.01	0.01\\
36.02	0.01\\
36.03	0.01\\
36.04	0.01\\
36.05	0.01\\
36.06	0.01\\
36.07	0.01\\
36.08	0.01\\
36.09	0.01\\
36.1	0.01\\
36.11	0.01\\
36.12	0.01\\
36.13	0.01\\
36.14	0.01\\
36.15	0.01\\
36.16	0.01\\
36.17	0.01\\
36.18	0.01\\
36.19	0.01\\
36.2	0.01\\
36.21	0.01\\
36.22	0.01\\
36.23	0.01\\
36.24	0.01\\
36.25	0.01\\
36.26	0.01\\
36.27	0.01\\
36.28	0.01\\
36.29	0.01\\
36.3	0.01\\
36.31	0.01\\
36.32	0.01\\
36.33	0.01\\
36.34	0.01\\
36.35	0.01\\
36.36	0.01\\
36.37	0.01\\
36.38	0.01\\
36.39	0.01\\
36.4	0.01\\
36.41	0.01\\
36.42	0.01\\
36.43	0.01\\
36.44	0.01\\
36.45	0.01\\
36.46	0.01\\
36.47	0.01\\
36.48	0.01\\
36.49	0.01\\
36.5	0.01\\
36.51	0.01\\
36.52	0.01\\
36.53	0.01\\
36.54	0.01\\
36.55	0.01\\
36.56	0.01\\
36.57	0.01\\
36.58	0.01\\
36.59	0.01\\
36.6	0.01\\
36.61	0.01\\
36.62	0.01\\
36.63	0.01\\
36.64	0.01\\
36.65	0.01\\
36.66	0.01\\
36.67	0.01\\
36.68	0.01\\
36.69	0.01\\
36.7	0.01\\
36.71	0.01\\
36.72	0.01\\
36.73	0.01\\
36.74	0.01\\
36.75	0.01\\
36.76	0.01\\
36.77	0.01\\
36.78	0.01\\
36.79	0.01\\
36.8	0.01\\
36.81	0.01\\
36.82	0.01\\
36.83	0.01\\
36.84	0.01\\
36.85	0.01\\
36.86	0.01\\
36.87	0.01\\
36.88	0.01\\
36.89	0.01\\
36.9	0.01\\
36.91	0.01\\
36.92	0.01\\
36.93	0.01\\
36.94	0.01\\
36.95	0.01\\
36.96	0.01\\
36.97	0.01\\
36.98	0.01\\
36.99	0.01\\
37	0.01\\
37.01	0.01\\
37.02	0.01\\
37.03	0.01\\
37.04	0.01\\
37.05	0.01\\
37.06	0.01\\
37.07	0.01\\
37.08	0.01\\
37.09	0.01\\
37.1	0.01\\
37.11	0.01\\
37.12	0.01\\
37.13	0.01\\
37.14	0.01\\
37.15	0.01\\
37.16	0.01\\
37.17	0.01\\
37.18	0.01\\
37.19	0.01\\
37.2	0.01\\
37.21	0.01\\
37.22	0.01\\
37.23	0.01\\
37.24	0.01\\
37.25	0.01\\
37.26	0.01\\
37.27	0.01\\
37.28	0.01\\
37.29	0.01\\
37.3	0.01\\
37.31	0.01\\
37.32	0.01\\
37.33	0.01\\
37.34	0.01\\
37.35	0.01\\
37.36	0.01\\
37.37	0.01\\
37.38	0.01\\
37.39	0.01\\
37.4	0.01\\
37.41	0.01\\
37.42	0.01\\
37.43	0.01\\
37.44	0.01\\
37.45	0.01\\
37.46	0.01\\
37.47	0.01\\
37.48	0.01\\
37.49	0.01\\
37.5	0.01\\
37.51	0.01\\
37.52	0.01\\
37.53	0.01\\
37.54	0.01\\
37.55	0.01\\
37.56	0.01\\
37.57	0.01\\
37.58	0.01\\
37.59	0.01\\
37.6	0.01\\
37.61	0.01\\
37.62	0.01\\
37.63	0.01\\
37.64	0.01\\
37.65	0.01\\
37.66	0.01\\
37.67	0.01\\
37.68	0.01\\
37.69	0.01\\
37.7	0.01\\
37.71	0.01\\
37.72	0.01\\
37.73	0.01\\
37.74	0.01\\
37.75	0.01\\
37.76	0.01\\
37.77	0.01\\
37.78	0.01\\
37.79	0.01\\
37.8	0.01\\
37.81	0.01\\
37.82	0.01\\
37.83	0.01\\
37.84	0.01\\
37.85	0.01\\
37.86	0.01\\
37.87	0.01\\
37.88	0.01\\
37.89	0.01\\
37.9	0.01\\
37.91	0.01\\
37.92	0.01\\
37.93	0.01\\
37.94	0.01\\
37.95	0.01\\
37.96	0.01\\
37.97	0.01\\
37.98	0.01\\
37.99	0.01\\
38	0.01\\
38.01	0.01\\
38.02	0.01\\
38.03	0.01\\
38.04	0.01\\
38.05	0.01\\
38.06	0.01\\
38.07	0.01\\
38.08	0.01\\
38.09	0.01\\
38.1	0.01\\
38.11	0.01\\
38.12	0.01\\
38.13	0.01\\
38.14	0.01\\
38.15	0.01\\
38.16	0.01\\
38.17	0.01\\
38.18	0.01\\
38.19	0.01\\
38.2	0.01\\
38.21	0.01\\
38.22	0.01\\
38.23	0.01\\
38.24	0.01\\
38.25	0.01\\
38.26	0.01\\
38.27	0.01\\
38.28	0.01\\
38.29	0.01\\
38.3	0.01\\
38.31	0.01\\
38.32	0.01\\
38.33	0.01\\
38.34	0.01\\
38.35	0.01\\
38.36	0.01\\
38.37	0.01\\
38.38	0.01\\
38.39	0.01\\
38.4	0.01\\
38.41	0.01\\
38.42	0.01\\
38.43	0.01\\
38.44	0.01\\
38.45	0.01\\
38.46	0.01\\
38.47	0.01\\
38.48	0.01\\
38.49	0.01\\
38.5	0.01\\
38.51	0.01\\
38.52	0.01\\
38.53	0.01\\
38.54	0.01\\
38.55	0.01\\
38.56	0.01\\
38.57	0.01\\
38.58	0.01\\
38.59	0.01\\
38.6	0.01\\
38.61	0.01\\
38.62	0.01\\
38.63	0.01\\
38.64	0.01\\
38.65	0.01\\
38.66	0.01\\
38.67	0.01\\
38.68	0.01\\
38.69	0.01\\
38.7	0.01\\
38.71	0.01\\
38.72	0.01\\
38.73	0.01\\
38.74	0.01\\
38.75	0.01\\
38.76	0.01\\
38.77	0.01\\
38.78	0.01\\
38.79	0.01\\
38.8	0.01\\
38.81	0.01\\
38.82	0.01\\
38.83	0.01\\
38.84	0.01\\
38.85	0.01\\
38.86	0.01\\
38.87	0.01\\
38.88	0.01\\
38.89	0.01\\
38.9	0.01\\
38.91	0.01\\
38.92	0.01\\
38.93	0.01\\
38.94	0.01\\
38.95	0.01\\
38.96	0.01\\
38.97	0.01\\
38.98	0.01\\
38.99	0.01\\
39	0.01\\
39.01	0.01\\
39.02	0.01\\
39.03	0.01\\
39.04	0.01\\
39.05	0.01\\
39.06	0.01\\
39.07	0.01\\
39.08	0.01\\
39.09	0.01\\
39.1	0.01\\
39.11	0.01\\
39.12	0.01\\
39.13	0.01\\
39.14	0.01\\
39.15	0.01\\
39.16	0.01\\
39.17	0.01\\
39.18	0.01\\
39.19	0.01\\
39.2	0.01\\
39.21	0.01\\
39.22	0.01\\
39.23	0.01\\
39.24	0.01\\
39.25	0.01\\
39.26	0.01\\
39.27	0.01\\
39.28	0.01\\
39.29	0.01\\
39.3	0.01\\
39.31	0.01\\
39.32	0.01\\
39.33	0.01\\
39.34	0.01\\
39.35	0.01\\
39.36	0.01\\
39.37	0.01\\
39.38	0.01\\
39.39	0.01\\
39.4	0.01\\
39.41	0.01\\
39.42	0.01\\
39.43	0.01\\
39.44	0.01\\
39.45	0.01\\
39.46	0.01\\
39.47	0.01\\
39.48	0.01\\
39.49	0.01\\
39.5	0.01\\
39.51	0.01\\
39.52	0.01\\
39.53	0.01\\
39.54	0.01\\
39.55	0.01\\
39.56	0.01\\
39.57	0.01\\
39.58	0.01\\
39.59	0.01\\
39.6	0.01\\
39.61	0.01\\
39.62	0.01\\
39.63	0.01\\
39.64	0.01\\
39.65	0.01\\
39.66	0.01\\
39.67	0.01\\
39.68	0.01\\
39.69	0.01\\
39.7	0.01\\
39.71	0.01\\
39.72	0.01\\
39.73	0.01\\
39.74	0.01\\
39.75	0.01\\
39.76	0.01\\
39.77	0.01\\
39.78	0.01\\
39.79	0.01\\
39.8	0.01\\
39.81	0.01\\
39.82	0.01\\
39.83	0.01\\
39.84	0.01\\
39.85	0.01\\
39.86	0.01\\
39.87	0.01\\
39.88	0.01\\
39.89	0.01\\
39.9	0.01\\
39.91	0.01\\
39.92	0.01\\
39.93	0.01\\
39.94	0.01\\
39.95	0.01\\
39.96	0.01\\
39.97	0.01\\
39.98	0.01\\
39.99	0.01\\
40	0.01\\
40.01	0.01\\
};
\addplot [color=blue,dashed,forget plot]
  table[row sep=crcr]{%
40.01	0.01\\
40.02	0.01\\
40.03	0.01\\
40.04	0.01\\
40.05	0.01\\
40.06	0.01\\
40.07	0.01\\
40.08	0.01\\
40.09	0.01\\
40.1	0.01\\
40.11	0.01\\
40.12	0.01\\
40.13	0.01\\
40.14	0.01\\
40.15	0.01\\
40.16	0.01\\
40.17	0.01\\
40.18	0.01\\
40.19	0.01\\
40.2	0.01\\
40.21	0.01\\
40.22	0.01\\
40.23	0.01\\
40.24	0.01\\
40.25	0.01\\
40.26	0.01\\
40.27	0.01\\
40.28	0.01\\
40.29	0.01\\
40.3	0.01\\
40.31	0.01\\
40.32	0.01\\
40.33	0.01\\
40.34	0.01\\
40.35	0.01\\
40.36	0.01\\
40.37	0.01\\
40.38	0.01\\
40.39	0.01\\
40.4	0.01\\
40.41	0.01\\
40.42	0.01\\
40.43	0.01\\
40.44	0.01\\
40.45	0.01\\
40.46	0.01\\
40.47	0.01\\
40.48	0.01\\
40.49	0.01\\
40.5	0.01\\
40.51	0.01\\
40.52	0.01\\
40.53	0.01\\
40.54	0.01\\
40.55	0.01\\
40.56	0.01\\
40.57	0.01\\
40.58	0.01\\
40.59	0.01\\
40.6	0.01\\
40.61	0.01\\
40.62	0.01\\
40.63	0.01\\
40.64	0.01\\
40.65	0.01\\
40.66	0.01\\
40.67	0.01\\
40.68	0.01\\
40.69	0.01\\
40.7	0.01\\
40.71	0.01\\
40.72	0.01\\
40.73	0.01\\
40.74	0.01\\
40.75	0.01\\
40.76	0.01\\
40.77	0.01\\
40.78	0.01\\
40.79	0.01\\
40.8	0.01\\
40.81	0.01\\
40.82	0.01\\
40.83	0.01\\
40.84	0.01\\
40.85	0.01\\
40.86	0.01\\
40.87	0.01\\
40.88	0.01\\
40.89	0.01\\
40.9	0.01\\
40.91	0.01\\
40.92	0.01\\
40.93	0.01\\
40.94	0.01\\
40.95	0.01\\
40.96	0.01\\
40.97	0.01\\
40.98	0.01\\
40.99	0.01\\
41	0.01\\
41.01	0.01\\
41.02	0.01\\
41.03	0.01\\
41.04	0.01\\
41.05	0.01\\
41.06	0.01\\
41.07	0.01\\
41.08	0.01\\
41.09	0.01\\
41.1	0.01\\
41.11	0.01\\
41.12	0.01\\
41.13	0.01\\
41.14	0.01\\
41.15	0.01\\
41.16	0.01\\
41.17	0.01\\
41.18	0.01\\
41.19	0.01\\
41.2	0.01\\
41.21	0.01\\
41.22	0.01\\
41.23	0.01\\
41.24	0.01\\
41.25	0.01\\
41.26	0.01\\
41.27	0.01\\
41.28	0.01\\
41.29	0.01\\
41.3	0.01\\
41.31	0.01\\
41.32	0.01\\
41.33	0.01\\
41.34	0.01\\
41.35	0.01\\
41.36	0.01\\
41.37	0.01\\
41.38	0.01\\
41.39	0.01\\
41.4	0.01\\
41.41	0.01\\
41.42	0.01\\
41.43	0.01\\
41.44	0.01\\
41.45	0.01\\
41.46	0.01\\
41.47	0.01\\
41.48	0.01\\
41.49	0.01\\
41.5	0.01\\
41.51	0.01\\
41.52	0.01\\
41.53	0.01\\
41.54	0.01\\
41.55	0.01\\
41.56	0.01\\
41.57	0.01\\
41.58	0.01\\
41.59	0.01\\
41.6	0.01\\
41.61	0.01\\
41.62	0.01\\
41.63	0.01\\
41.64	0.01\\
41.65	0.01\\
41.66	0.01\\
41.67	0.01\\
41.68	0.01\\
41.69	0.01\\
41.7	0.01\\
41.71	0.01\\
41.72	0.01\\
41.73	0.01\\
41.74	0.01\\
41.75	0.01\\
41.76	0.01\\
41.77	0.01\\
41.78	0.01\\
41.79	0.01\\
41.8	0.01\\
41.81	0.01\\
41.82	0.01\\
41.83	0.01\\
41.84	0.01\\
41.85	0.01\\
41.86	0.01\\
41.87	0.01\\
41.88	0.01\\
41.89	0.01\\
41.9	0.01\\
41.91	0.01\\
41.92	0.01\\
41.93	0.01\\
41.94	0.01\\
41.95	0.01\\
41.96	0.01\\
41.97	0.01\\
41.98	0.01\\
41.99	0.01\\
42	0.01\\
42.01	0.01\\
42.02	0.01\\
42.03	0.01\\
42.04	0.01\\
42.05	0.01\\
42.06	0.01\\
42.07	0.01\\
42.08	0.01\\
42.09	0.01\\
42.1	0.01\\
42.11	0.01\\
42.12	0.01\\
42.13	0.01\\
42.14	0.01\\
42.15	0.01\\
42.16	0.01\\
42.17	0.01\\
42.18	0.01\\
42.19	0.01\\
42.2	0.01\\
42.21	0.01\\
42.22	0.01\\
42.23	0.01\\
42.24	0.01\\
42.25	0.01\\
42.26	0.01\\
42.27	0.01\\
42.28	0.01\\
42.29	0.01\\
42.3	0.01\\
42.31	0.01\\
42.32	0.01\\
42.33	0.01\\
42.34	0.01\\
42.35	0.01\\
42.36	0.01\\
42.37	0.01\\
42.38	0.01\\
42.39	0.01\\
42.4	0.01\\
42.41	0.01\\
42.42	0.01\\
42.43	0.01\\
42.44	0.01\\
42.45	0.01\\
42.46	0.01\\
42.47	0.01\\
42.48	0.01\\
42.49	0.01\\
42.5	0.01\\
42.51	0.01\\
42.52	0.01\\
42.53	0.01\\
42.54	0.01\\
42.55	0.01\\
42.56	0.01\\
42.57	0.01\\
42.58	0.01\\
42.59	0.01\\
42.6	0.01\\
42.61	0.01\\
42.62	0.01\\
42.63	0.01\\
42.64	0.01\\
42.65	0.01\\
42.66	0.01\\
42.67	0.01\\
42.68	0.01\\
42.69	0.01\\
42.7	0.01\\
42.71	0.01\\
42.72	0.01\\
42.73	0.01\\
42.74	0.01\\
42.75	0.01\\
42.76	0.01\\
42.77	0.01\\
42.78	0.01\\
42.79	0.01\\
42.8	0.01\\
42.81	0.01\\
42.82	0.01\\
42.83	0.01\\
42.84	0.01\\
42.85	0.01\\
42.86	0.01\\
42.87	0.01\\
42.88	0.01\\
42.89	0.01\\
42.9	0.01\\
42.91	0.01\\
42.92	0.01\\
42.93	0.01\\
42.94	0.01\\
42.95	0.01\\
42.96	0.01\\
42.97	0.01\\
42.98	0.01\\
42.99	0.01\\
43	0.01\\
43.01	0.01\\
43.02	0.01\\
43.03	0.01\\
43.04	0.01\\
43.05	0.01\\
43.06	0.01\\
43.07	0.01\\
43.08	0.01\\
43.09	0.01\\
43.1	0.01\\
43.11	0.01\\
43.12	0.01\\
43.13	0.01\\
43.14	0.01\\
43.15	0.01\\
43.16	0.01\\
43.17	0.01\\
43.18	0.01\\
43.19	0.01\\
43.2	0.01\\
43.21	0.01\\
43.22	0.01\\
43.23	0.01\\
43.24	0.01\\
43.25	0.01\\
43.26	0.01\\
43.27	0.01\\
43.28	0.01\\
43.29	0.01\\
43.3	0.01\\
43.31	0.01\\
43.32	0.01\\
43.33	0.01\\
43.34	0.01\\
43.35	0.01\\
43.36	0.01\\
43.37	0.01\\
43.38	0.01\\
43.39	0.01\\
43.4	0.01\\
43.41	0.01\\
43.42	0.01\\
43.43	0.01\\
43.44	0.01\\
43.45	0.01\\
43.46	0.01\\
43.47	0.01\\
43.48	0.01\\
43.49	0.01\\
43.5	0.01\\
43.51	0.01\\
43.52	0.01\\
43.53	0.01\\
43.54	0.01\\
43.55	0.01\\
43.56	0.01\\
43.57	0.01\\
43.58	0.01\\
43.59	0.01\\
43.6	0.01\\
43.61	0.01\\
43.62	0.01\\
43.63	0.01\\
43.64	0.01\\
43.65	0.01\\
43.66	0.01\\
43.67	0.01\\
43.68	0.01\\
43.69	0.01\\
43.7	0.01\\
43.71	0.01\\
43.72	0.01\\
43.73	0.01\\
43.74	0.01\\
43.75	0.01\\
43.76	0.01\\
43.77	0.01\\
43.78	0.01\\
43.79	0.01\\
43.8	0.01\\
43.81	0.01\\
43.82	0.01\\
43.83	0.01\\
43.84	0.01\\
43.85	0.01\\
43.86	0.01\\
43.87	0.01\\
43.88	0.01\\
43.89	0.01\\
43.9	0.01\\
43.91	0.01\\
43.92	0.01\\
43.93	0.01\\
43.94	0.01\\
43.95	0.01\\
43.96	0.01\\
43.97	0.01\\
43.98	0.01\\
43.99	0.01\\
44	0.01\\
44.01	0.01\\
44.02	0.01\\
44.03	0.01\\
44.04	0.01\\
44.05	0.01\\
44.06	0.01\\
44.07	0.01\\
44.08	0.01\\
44.09	0.01\\
44.1	0.01\\
44.11	0.01\\
44.12	0.01\\
44.13	0.01\\
44.14	0.01\\
44.15	0.01\\
44.16	0.01\\
44.17	0.01\\
44.18	0.01\\
44.19	0.01\\
44.2	0.01\\
44.21	0.01\\
44.22	0.01\\
44.23	0.01\\
44.24	0.01\\
44.25	0.01\\
44.26	0.01\\
44.27	0.01\\
44.28	0.01\\
44.29	0.01\\
44.3	0.01\\
44.31	0.01\\
44.32	0.01\\
44.33	0.01\\
44.34	0.01\\
44.35	0.01\\
44.36	0.01\\
44.37	0.01\\
44.38	0.01\\
44.39	0.01\\
44.4	0.01\\
44.41	0.01\\
44.42	0.01\\
44.43	0.01\\
44.44	0.01\\
44.45	0.01\\
44.46	0.01\\
44.47	0.01\\
44.48	0.01\\
44.49	0.01\\
44.5	0.01\\
44.51	0.01\\
44.52	0.01\\
44.53	0.01\\
44.54	0.01\\
44.55	0.01\\
44.56	0.01\\
44.57	0.01\\
44.58	0.01\\
44.59	0.01\\
44.6	0.01\\
44.61	0.01\\
44.62	0.01\\
44.63	0.01\\
44.64	0.01\\
44.65	0.01\\
44.66	0.01\\
44.67	0.01\\
44.68	0.01\\
44.69	0.01\\
44.7	0.01\\
44.71	0.01\\
44.72	0.01\\
44.73	0.01\\
44.74	0.01\\
44.75	0.01\\
44.76	0.01\\
44.77	0.01\\
44.78	0.01\\
44.79	0.01\\
44.8	0.01\\
44.81	0.01\\
44.82	0.01\\
44.83	0.01\\
44.84	0.01\\
44.85	0.01\\
44.86	0.01\\
44.87	0.01\\
44.88	0.01\\
44.89	0.01\\
44.9	0.01\\
44.91	0.01\\
44.92	0.01\\
44.93	0.01\\
44.94	0.01\\
44.95	0.01\\
44.96	0.01\\
44.97	0.01\\
44.98	0.01\\
44.99	0.01\\
45	0.01\\
45.01	0.01\\
45.02	0.01\\
45.03	0.01\\
45.04	0.01\\
45.05	0.01\\
45.06	0.01\\
45.07	0.01\\
45.08	0.01\\
45.09	0.01\\
45.1	0.01\\
45.11	0.01\\
45.12	0.01\\
45.13	0.01\\
45.14	0.01\\
45.15	0.01\\
45.16	0.01\\
45.17	0.01\\
45.18	0.01\\
45.19	0.01\\
45.2	0.01\\
45.21	0.01\\
45.22	0.01\\
45.23	0.01\\
45.24	0.01\\
45.25	0.01\\
45.26	0.01\\
45.27	0.01\\
45.28	0.01\\
45.29	0.01\\
45.3	0.01\\
45.31	0.01\\
45.32	0.01\\
45.33	0.01\\
45.34	0.01\\
45.35	0.01\\
45.36	0.01\\
45.37	0.01\\
45.38	0.01\\
45.39	0.01\\
45.4	0.01\\
45.41	0.01\\
45.42	0.01\\
45.43	0.01\\
45.44	0.01\\
45.45	0.01\\
45.46	0.01\\
45.47	0.01\\
45.48	0.01\\
45.49	0.01\\
45.5	0.01\\
45.51	0.01\\
45.52	0.01\\
45.53	0.01\\
45.54	0.01\\
45.55	0.01\\
45.56	0.01\\
45.57	0.01\\
45.58	0.01\\
45.59	0.01\\
45.6	0.01\\
45.61	0.01\\
45.62	0.01\\
45.63	0.01\\
45.64	0.01\\
45.65	0.01\\
45.66	0.01\\
45.67	0.01\\
45.68	0.01\\
45.69	0.01\\
45.7	0.01\\
45.71	0.01\\
45.72	0.01\\
45.73	0.01\\
45.74	0.01\\
45.75	0.01\\
45.76	0.01\\
45.77	0.01\\
45.78	0.01\\
45.79	0.01\\
45.8	0.01\\
45.81	0.01\\
45.82	0.01\\
45.83	0.01\\
45.84	0.01\\
45.85	0.01\\
45.86	0.01\\
45.87	0.01\\
45.88	0.01\\
45.89	0.01\\
45.9	0.01\\
45.91	0.01\\
45.92	0.01\\
45.93	0.01\\
45.94	0.01\\
45.95	0.01\\
45.96	0.01\\
45.97	0.01\\
45.98	0.01\\
45.99	0.01\\
46	0.01\\
46.01	0.01\\
46.02	0.01\\
46.03	0.01\\
46.04	0.01\\
46.05	0.01\\
46.06	0.01\\
46.07	0.01\\
46.08	0.01\\
46.09	0.01\\
46.1	0.01\\
46.11	0.01\\
46.12	0.01\\
46.13	0.01\\
46.14	0.01\\
46.15	0.01\\
46.16	0.01\\
46.17	0.01\\
46.18	0.01\\
46.19	0.01\\
46.2	0.01\\
46.21	0.01\\
46.22	0.01\\
46.23	0.01\\
46.24	0.01\\
46.25	0.01\\
46.26	0.01\\
46.27	0.01\\
46.28	0.01\\
46.29	0.01\\
46.3	0.01\\
46.31	0.01\\
46.32	0.01\\
46.33	0.01\\
46.34	0.01\\
46.35	0.01\\
46.36	0.01\\
46.37	0.01\\
46.38	0.01\\
46.39	0.01\\
46.4	0.01\\
46.41	0.01\\
46.42	0.01\\
46.43	0.01\\
46.44	0.01\\
46.45	0.01\\
46.46	0.01\\
46.47	0.01\\
46.48	0.01\\
46.49	0.01\\
46.5	0.01\\
46.51	0.01\\
46.52	0.01\\
46.53	0.01\\
46.54	0.01\\
46.55	0.01\\
46.56	0.01\\
46.57	0.01\\
46.58	0.01\\
46.59	0.01\\
46.6	0.01\\
46.61	0.01\\
46.62	0.01\\
46.63	0.01\\
46.64	0.01\\
46.65	0.01\\
46.66	0.01\\
46.67	0.01\\
46.68	0.01\\
46.69	0.01\\
46.7	0.01\\
46.71	0.01\\
46.72	0.01\\
46.73	0.01\\
46.74	0.01\\
46.75	0.01\\
46.76	0.01\\
46.77	0.01\\
46.78	0.01\\
46.79	0.01\\
46.8	0.01\\
46.81	0.01\\
46.82	0.01\\
46.83	0.01\\
46.84	0.01\\
46.85	0.01\\
46.86	0.01\\
46.87	0.01\\
46.88	0.01\\
46.89	0.01\\
46.9	0.01\\
46.91	0.01\\
46.92	0.01\\
46.93	0.01\\
46.94	0.01\\
46.95	0.01\\
46.96	0.01\\
46.97	0.01\\
46.98	0.01\\
46.99	0.01\\
47	0.01\\
47.01	0.01\\
47.02	0.01\\
47.03	0.01\\
47.04	0.01\\
47.05	0.01\\
47.06	0.01\\
47.07	0.01\\
47.08	0.01\\
47.09	0.01\\
47.1	0.01\\
47.11	0.01\\
47.12	0.01\\
47.13	0.01\\
47.14	0.01\\
47.15	0.01\\
47.16	0.01\\
47.17	0.01\\
47.18	0.01\\
47.19	0.01\\
47.2	0.01\\
47.21	0.01\\
47.22	0.01\\
47.23	0.01\\
47.24	0.01\\
47.25	0.01\\
47.26	0.01\\
47.27	0.01\\
47.28	0.01\\
47.29	0.01\\
47.3	0.01\\
47.31	0.01\\
47.32	0.01\\
47.33	0.01\\
47.34	0.01\\
47.35	0.01\\
47.36	0.01\\
47.37	0.01\\
47.38	0.01\\
47.39	0.01\\
47.4	0.01\\
47.41	0.01\\
47.42	0.01\\
47.43	0.01\\
47.44	0.01\\
47.45	0.01\\
47.46	0.01\\
47.47	0.01\\
47.48	0.01\\
47.49	0.01\\
47.5	0.01\\
47.51	0.01\\
47.52	0.01\\
47.53	0.01\\
47.54	0.01\\
47.55	0.01\\
47.56	0.01\\
47.57	0.01\\
47.58	0.01\\
47.59	0.01\\
47.6	0.01\\
47.61	0.01\\
47.62	0.01\\
47.63	0.01\\
47.64	0.01\\
47.65	0.01\\
47.66	0.01\\
47.67	0.01\\
47.68	0.01\\
47.69	0.01\\
47.7	0.01\\
47.71	0.01\\
47.72	0.01\\
47.73	0.01\\
47.74	0.01\\
47.75	0.01\\
47.76	0.01\\
47.77	0.01\\
47.78	0.01\\
47.79	0.01\\
47.8	0.01\\
47.81	0.01\\
47.82	0.01\\
47.83	0.01\\
47.84	0.01\\
47.85	0.01\\
47.86	0.01\\
47.87	0.01\\
47.88	0.01\\
47.89	0.01\\
47.9	0.01\\
47.91	0.01\\
47.92	0.01\\
47.93	0.01\\
47.94	0.01\\
47.95	0.01\\
47.96	0.01\\
47.97	0.01\\
47.98	0.01\\
47.99	0.01\\
48	0.01\\
48.01	0.01\\
48.02	0.01\\
48.03	0.01\\
48.04	0.01\\
48.05	0.01\\
48.06	0.01\\
48.07	0.01\\
48.08	0.01\\
48.09	0.01\\
48.1	0.01\\
48.11	0.01\\
48.12	0.01\\
48.13	0.01\\
48.14	0.01\\
48.15	0.01\\
48.16	0.01\\
48.17	0.01\\
48.18	0.01\\
48.19	0.01\\
48.2	0.01\\
48.21	0.01\\
48.22	0.01\\
48.23	0.01\\
48.24	0.01\\
48.25	0.01\\
48.26	0.01\\
48.27	0.01\\
48.28	0.01\\
48.29	0.01\\
48.3	0.01\\
48.31	0.01\\
48.32	0.01\\
48.33	0.01\\
48.34	0.01\\
48.35	0.01\\
48.36	0.01\\
48.37	0.01\\
48.38	0.01\\
48.39	0.01\\
48.4	0.01\\
48.41	0.01\\
48.42	0.01\\
48.43	0.01\\
48.44	0.01\\
48.45	0.01\\
48.46	0.01\\
48.47	0.01\\
48.48	0.01\\
48.49	0.01\\
48.5	0.01\\
48.51	0.01\\
48.52	0.01\\
48.53	0.01\\
48.54	0.01\\
48.55	0.01\\
48.56	0.01\\
48.57	0.01\\
48.58	0.01\\
48.59	0.01\\
48.6	0.01\\
48.61	0.01\\
48.62	0.01\\
48.63	0.01\\
48.64	0.01\\
48.65	0.01\\
48.66	0.01\\
48.67	0.01\\
48.68	0.01\\
48.69	0.01\\
48.7	0.01\\
48.71	0.01\\
48.72	0.01\\
48.73	0.01\\
48.74	0.01\\
48.75	0.01\\
48.76	0.01\\
48.77	0.01\\
48.78	0.01\\
48.79	0.01\\
48.8	0.01\\
48.81	0.01\\
48.82	0.01\\
48.83	0.01\\
48.84	0.01\\
48.85	0.01\\
48.86	0.01\\
48.87	0.01\\
48.88	0.01\\
48.89	0.01\\
48.9	0.01\\
48.91	0.01\\
48.92	0.01\\
48.93	0.01\\
48.94	0.01\\
48.95	0.01\\
48.96	0.01\\
48.97	0.01\\
48.98	0.01\\
48.99	0.01\\
49	0.01\\
49.01	0.01\\
49.02	0.01\\
49.03	0.01\\
49.04	0.01\\
49.05	0.01\\
49.06	0.01\\
49.07	0.01\\
49.08	0.01\\
49.09	0.01\\
49.1	0.01\\
49.11	0.01\\
49.12	0.01\\
49.13	0.01\\
49.14	0.01\\
49.15	0.01\\
49.16	0.01\\
49.17	0.01\\
49.18	0.01\\
49.19	0.01\\
49.2	0.01\\
49.21	0.01\\
49.22	0.01\\
49.23	0.01\\
49.24	0.01\\
49.25	0.01\\
49.26	0.01\\
49.27	0.01\\
49.28	0.01\\
49.29	0.01\\
49.3	0.01\\
49.31	0.01\\
49.32	0.01\\
49.33	0.01\\
49.34	0.01\\
49.35	0.01\\
49.36	0.01\\
49.37	0.01\\
49.38	0.01\\
49.39	0.01\\
49.4	0.01\\
49.41	0.01\\
49.42	0.01\\
49.43	0.01\\
49.44	0.01\\
49.45	0.01\\
49.46	0.01\\
49.47	0.01\\
49.48	0.01\\
49.49	0.01\\
49.5	0.01\\
49.51	0.01\\
49.52	0.01\\
49.53	0.01\\
49.54	0.01\\
49.55	0.01\\
49.56	0.01\\
49.57	0.01\\
49.58	0.01\\
49.59	0.01\\
49.6	0.01\\
49.61	0.01\\
49.62	0.01\\
49.63	0.01\\
49.64	0.01\\
49.65	0.01\\
49.66	0.01\\
49.67	0.01\\
49.68	0.01\\
49.69	0.01\\
49.7	0.01\\
49.71	0.01\\
49.72	0.01\\
49.73	0.01\\
49.74	0.01\\
49.75	0.01\\
49.76	0.01\\
49.77	0.01\\
49.78	0.01\\
49.79	0.01\\
49.8	0.01\\
49.81	0.01\\
49.82	0.01\\
49.83	0.01\\
49.84	0.01\\
49.85	0.01\\
49.86	0.01\\
49.87	0.01\\
49.88	0.01\\
49.89	0.01\\
49.9	0.01\\
49.91	0.01\\
49.92	0.01\\
49.93	0.01\\
49.94	0.01\\
49.95	0.01\\
49.96	0.01\\
49.97	0.01\\
49.98	0.01\\
49.99	0.01\\
50	0.01\\
50.01	0.01\\
50.02	0.01\\
50.03	0.01\\
50.04	0.01\\
50.05	0.01\\
50.06	0.01\\
50.07	0.01\\
50.08	0.01\\
50.09	0.01\\
50.1	0.01\\
50.11	0.01\\
50.12	0.01\\
50.13	0.01\\
50.14	0.01\\
50.15	0.01\\
50.16	0.01\\
50.17	0.01\\
50.18	0.01\\
50.19	0.01\\
50.2	0.01\\
50.21	0.01\\
50.22	0.01\\
50.23	0.01\\
50.24	0.01\\
50.25	0.01\\
50.26	0.01\\
50.27	0.01\\
50.28	0.01\\
50.29	0.01\\
50.3	0.01\\
50.31	0.01\\
50.32	0.01\\
50.33	0.01\\
50.34	0.01\\
50.35	0.01\\
50.36	0.01\\
50.37	0.01\\
50.38	0.01\\
50.39	0.01\\
50.4	0.01\\
50.41	0.01\\
50.42	0.01\\
50.43	0.01\\
50.44	0.01\\
50.45	0.01\\
50.46	0.01\\
50.47	0.01\\
50.48	0.01\\
50.49	0.01\\
50.5	0.01\\
50.51	0.01\\
50.52	0.01\\
50.53	0.01\\
50.54	0.01\\
50.55	0.01\\
50.56	0.01\\
50.57	0.01\\
50.58	0.01\\
50.59	0.01\\
50.6	0.01\\
50.61	0.01\\
50.62	0.01\\
50.63	0.01\\
50.64	0.01\\
50.65	0.01\\
50.66	0.01\\
50.67	0.01\\
50.68	0.01\\
50.69	0.01\\
50.7	0.01\\
50.71	0.01\\
50.72	0.01\\
50.73	0.01\\
50.74	0.01\\
50.75	0.01\\
50.76	0.01\\
50.77	0.01\\
50.78	0.01\\
50.79	0.01\\
50.8	0.01\\
50.81	0.01\\
50.82	0.01\\
50.83	0.01\\
50.84	0.01\\
50.85	0.01\\
50.86	0.01\\
50.87	0.01\\
50.88	0.01\\
50.89	0.01\\
50.9	0.01\\
50.91	0.01\\
50.92	0.01\\
50.93	0.01\\
50.94	0.01\\
50.95	0.01\\
50.96	0.01\\
50.97	0.01\\
50.98	0.01\\
50.99	0.01\\
51	0.01\\
51.01	0.01\\
51.02	0.01\\
51.03	0.01\\
51.04	0.01\\
51.05	0.01\\
51.06	0.01\\
51.07	0.01\\
51.08	0.01\\
51.09	0.01\\
51.1	0.01\\
51.11	0.01\\
51.12	0.01\\
51.13	0.01\\
51.14	0.01\\
51.15	0.01\\
51.16	0.01\\
51.17	0.01\\
51.18	0.01\\
51.19	0.01\\
51.2	0.01\\
51.21	0.01\\
51.22	0.01\\
51.23	0.01\\
51.24	0.01\\
51.25	0.01\\
51.26	0.01\\
51.27	0.01\\
51.28	0.01\\
51.29	0.01\\
51.3	0.01\\
51.31	0.01\\
51.32	0.01\\
51.33	0.01\\
51.34	0.01\\
51.35	0.01\\
51.36	0.01\\
51.37	0.01\\
51.38	0.01\\
51.39	0.01\\
51.4	0.01\\
51.41	0.01\\
51.42	0.01\\
51.43	0.01\\
51.44	0.01\\
51.45	0.01\\
51.46	0.01\\
51.47	0.01\\
51.48	0.01\\
51.49	0.01\\
51.5	0.01\\
51.51	0.01\\
51.52	0.01\\
51.53	0.01\\
51.54	0.01\\
51.55	0.01\\
51.56	0.01\\
51.57	0.01\\
51.58	0.01\\
51.59	0.01\\
51.6	0.01\\
51.61	0.01\\
51.62	0.01\\
51.63	0.01\\
51.64	0.01\\
51.65	0.01\\
51.66	0.01\\
51.67	0.01\\
51.68	0.01\\
51.69	0.01\\
51.7	0.01\\
51.71	0.01\\
51.72	0.01\\
51.73	0.01\\
51.74	0.01\\
51.75	0.01\\
51.76	0.01\\
51.77	0.01\\
51.78	0.01\\
51.79	0.01\\
51.8	0.01\\
51.81	0.01\\
51.82	0.01\\
51.83	0.01\\
51.84	0.01\\
51.85	0.01\\
51.86	0.01\\
51.87	0.01\\
51.88	0.01\\
51.89	0.01\\
51.9	0.01\\
51.91	0.01\\
51.92	0.01\\
51.93	0.01\\
51.94	0.01\\
51.95	0.01\\
51.96	0.01\\
51.97	0.01\\
51.98	0.01\\
51.99	0.01\\
52	0.01\\
52.01	0.01\\
52.02	0.01\\
52.03	0.01\\
52.04	0.01\\
52.05	0.01\\
52.06	0.01\\
52.07	0.01\\
52.08	0.01\\
52.09	0.01\\
52.1	0.01\\
52.11	0.01\\
52.12	0.01\\
52.13	0.01\\
52.14	0.01\\
52.15	0.01\\
52.16	0.01\\
52.17	0.01\\
52.18	0.01\\
52.19	0.01\\
52.2	0.01\\
52.21	0.01\\
52.22	0.01\\
52.23	0.01\\
52.24	0.01\\
52.25	0.01\\
52.26	0.01\\
52.27	0.01\\
52.28	0.01\\
52.29	0.01\\
52.3	0.01\\
52.31	0.01\\
52.32	0.01\\
52.33	0.01\\
52.34	0.01\\
52.35	0.01\\
52.36	0.01\\
52.37	0.01\\
52.38	0.01\\
52.39	0.01\\
52.4	0.01\\
52.41	0.01\\
52.42	0.01\\
52.43	0.01\\
52.44	0.01\\
52.45	0.01\\
52.46	0.01\\
52.47	0.01\\
52.48	0.01\\
52.49	0.01\\
52.5	0.01\\
52.51	0.01\\
52.52	0.01\\
52.53	0.01\\
52.54	0.01\\
52.55	0.01\\
52.56	0.01\\
52.57	0.01\\
52.58	0.01\\
52.59	0.01\\
52.6	0.01\\
52.61	0.01\\
52.62	0.01\\
52.63	0.01\\
52.64	0.01\\
52.65	0.01\\
52.66	0.01\\
52.67	0.01\\
52.68	0.01\\
52.69	0.01\\
52.7	0.01\\
52.71	0.01\\
52.72	0.01\\
52.73	0.01\\
52.74	0.01\\
52.75	0.01\\
52.76	0.01\\
52.77	0.01\\
52.78	0.01\\
52.79	0.01\\
52.8	0.01\\
52.81	0.01\\
52.82	0.01\\
52.83	0.01\\
52.84	0.01\\
52.85	0.01\\
52.86	0.01\\
52.87	0.01\\
52.88	0.01\\
52.89	0.01\\
52.9	0.01\\
52.91	0.01\\
52.92	0.01\\
52.93	0.01\\
52.94	0.01\\
52.95	0.01\\
52.96	0.01\\
52.97	0.01\\
52.98	0.01\\
52.99	0.01\\
53	0.01\\
53.01	0.01\\
53.02	0.01\\
53.03	0.01\\
53.04	0.01\\
53.05	0.01\\
53.06	0.01\\
53.07	0.01\\
53.08	0.01\\
53.09	0.01\\
53.1	0.01\\
53.11	0.01\\
53.12	0.01\\
53.13	0.01\\
53.14	0.01\\
53.15	0.01\\
53.16	0.01\\
53.17	0.01\\
53.18	0.01\\
53.19	0.01\\
53.2	0.01\\
53.21	0.01\\
53.22	0.01\\
53.23	0.01\\
53.24	0.01\\
53.25	0.01\\
53.26	0.01\\
53.27	0.01\\
53.28	0.01\\
53.29	0.01\\
53.3	0.01\\
53.31	0.01\\
53.32	0.01\\
53.33	0.01\\
53.34	0.01\\
53.35	0.01\\
53.36	0.01\\
53.37	0.01\\
53.38	0.01\\
53.39	0.01\\
53.4	0.01\\
53.41	0.01\\
53.42	0.01\\
53.43	0.01\\
53.44	0.01\\
53.45	0.01\\
53.46	0.01\\
53.47	0.01\\
53.48	0.01\\
53.49	0.01\\
53.5	0.01\\
53.51	0.01\\
53.52	0.01\\
53.53	0.01\\
53.54	0.01\\
53.55	0.01\\
53.56	0.01\\
53.57	0.01\\
53.58	0.01\\
53.59	0.01\\
53.6	0.01\\
53.61	0.01\\
53.62	0.01\\
53.63	0.01\\
53.64	0.01\\
53.65	0.01\\
53.66	0.01\\
53.67	0.01\\
53.68	0.01\\
53.69	0.01\\
53.7	0.01\\
53.71	0.01\\
53.72	0.01\\
53.73	0.01\\
53.74	0.01\\
53.75	0.01\\
53.76	0.01\\
53.77	0.01\\
53.78	0.01\\
53.79	0.01\\
53.8	0.01\\
53.81	0.01\\
53.82	0.01\\
53.83	0.01\\
53.84	0.01\\
53.85	0.01\\
53.86	0.01\\
53.87	0.01\\
53.88	0.01\\
53.89	0.01\\
53.9	0.01\\
53.91	0.01\\
53.92	0.01\\
53.93	0.01\\
53.94	0.01\\
53.95	0.01\\
53.96	0.01\\
53.97	0.01\\
53.98	0.01\\
53.99	0.01\\
54	0.01\\
54.01	0.01\\
54.02	0.01\\
54.03	0.01\\
54.04	0.01\\
54.05	0.01\\
54.06	0.01\\
54.07	0.01\\
54.08	0.01\\
54.09	0.01\\
54.1	0.01\\
54.11	0.01\\
54.12	0.01\\
54.13	0.01\\
54.14	0.01\\
54.15	0.01\\
54.16	0.01\\
54.17	0.01\\
54.18	0.01\\
54.19	0.01\\
54.2	0.01\\
54.21	0.01\\
54.22	0.01\\
54.23	0.01\\
54.24	0.01\\
54.25	0.01\\
54.26	0.01\\
54.27	0.01\\
54.28	0.01\\
54.29	0.01\\
54.3	0.01\\
54.31	0.01\\
54.32	0.01\\
54.33	0.01\\
54.34	0.01\\
54.35	0.01\\
54.36	0.01\\
54.37	0.01\\
54.38	0.01\\
54.39	0.01\\
54.4	0.01\\
54.41	0.01\\
54.42	0.01\\
54.43	0.01\\
54.44	0.01\\
54.45	0.01\\
54.46	0.01\\
54.47	0.01\\
54.48	0.01\\
54.49	0.01\\
54.5	0.01\\
54.51	0.01\\
54.52	0.01\\
54.53	0.01\\
54.54	0.01\\
54.55	0.01\\
54.56	0.01\\
54.57	0.01\\
54.58	0.01\\
54.59	0.01\\
54.6	0.01\\
54.61	0.01\\
54.62	0.01\\
54.63	0.01\\
54.64	0.01\\
54.65	0.01\\
54.66	0.01\\
54.67	0.01\\
54.68	0.01\\
54.69	0.01\\
54.7	0.01\\
54.71	0.01\\
54.72	0.01\\
54.73	0.01\\
54.74	0.01\\
54.75	0.01\\
54.76	0.01\\
54.77	0.01\\
54.78	0.01\\
54.79	0.01\\
54.8	0.01\\
54.81	0.01\\
54.82	0.01\\
54.83	0.01\\
54.84	0.01\\
54.85	0.01\\
54.86	0.01\\
54.87	0.01\\
54.88	0.01\\
54.89	0.01\\
54.9	0.01\\
54.91	0.01\\
54.92	0.01\\
54.93	0.01\\
54.94	0.01\\
54.95	0.01\\
54.96	0.01\\
54.97	0.01\\
54.98	0.01\\
54.99	0.01\\
55	0.01\\
55.01	0.01\\
55.02	0.01\\
55.03	0.01\\
55.04	0.01\\
55.05	0.01\\
55.06	0.01\\
55.07	0.01\\
55.08	0.01\\
55.09	0.01\\
55.1	0.01\\
55.11	0.01\\
55.12	0.01\\
55.13	0.01\\
55.14	0.01\\
55.15	0.01\\
55.16	0.01\\
55.17	0.01\\
55.18	0.01\\
55.19	0.01\\
55.2	0.01\\
55.21	0.01\\
55.22	0.01\\
55.23	0.01\\
55.24	0.01\\
55.25	0.01\\
55.26	0.01\\
55.27	0.01\\
55.28	0.01\\
55.29	0.01\\
55.3	0.01\\
55.31	0.01\\
55.32	0.01\\
55.33	0.01\\
55.34	0.01\\
55.35	0.01\\
55.36	0.01\\
55.37	0.01\\
55.38	0.01\\
55.39	0.01\\
55.4	0.01\\
55.41	0.01\\
55.42	0.01\\
55.43	0.01\\
55.44	0.01\\
55.45	0.01\\
55.46	0.01\\
55.47	0.01\\
55.48	0.01\\
55.49	0.01\\
55.5	0.01\\
55.51	0.01\\
55.52	0.01\\
55.53	0.01\\
55.54	0.01\\
55.55	0.01\\
55.56	0.01\\
55.57	0.01\\
55.58	0.01\\
55.59	0.01\\
55.6	0.01\\
55.61	0.01\\
55.62	0.01\\
55.63	0.01\\
55.64	0.01\\
55.65	0.01\\
55.66	0.01\\
55.67	0.01\\
55.68	0.01\\
55.69	0.01\\
55.7	0.01\\
55.71	0.01\\
55.72	0.01\\
55.73	0.01\\
55.74	0.01\\
55.75	0.01\\
55.76	0.01\\
55.77	0.01\\
55.78	0.01\\
55.79	0.01\\
55.8	0.01\\
55.81	0.01\\
55.82	0.01\\
55.83	0.01\\
55.84	0.01\\
55.85	0.01\\
55.86	0.01\\
55.87	0.01\\
55.88	0.01\\
55.89	0.01\\
55.9	0.01\\
55.91	0.01\\
55.92	0.01\\
55.93	0.01\\
55.94	0.01\\
55.95	0.01\\
55.96	0.01\\
55.97	0.01\\
55.98	0.01\\
55.99	0.01\\
56	0.01\\
56.01	0.01\\
56.02	0.01\\
56.03	0.01\\
56.04	0.01\\
56.05	0.01\\
56.06	0.01\\
56.07	0.01\\
56.08	0.01\\
56.09	0.01\\
56.1	0.01\\
56.11	0.01\\
56.12	0.01\\
56.13	0.01\\
56.14	0.01\\
56.15	0.01\\
56.16	0.01\\
56.17	0.01\\
56.18	0.01\\
56.19	0.01\\
56.2	0.01\\
56.21	0.01\\
56.22	0.01\\
56.23	0.01\\
56.24	0.01\\
56.25	0.01\\
56.26	0.01\\
56.27	0.01\\
56.28	0.01\\
56.29	0.01\\
56.3	0.01\\
56.31	0.01\\
56.32	0.01\\
56.33	0.01\\
56.34	0.01\\
56.35	0.01\\
56.36	0.01\\
56.37	0.01\\
56.38	0.01\\
56.39	0.01\\
56.4	0.01\\
56.41	0.01\\
56.42	0.01\\
56.43	0.01\\
56.44	0.01\\
56.45	0.01\\
56.46	0.01\\
56.47	0.01\\
56.48	0.01\\
56.49	0.01\\
56.5	0.01\\
56.51	0.01\\
56.52	0.01\\
56.53	0.01\\
56.54	0.01\\
56.55	0.01\\
56.56	0.01\\
56.57	0.01\\
56.58	0.01\\
56.59	0.01\\
56.6	0.01\\
56.61	0.01\\
56.62	0.01\\
56.63	0.01\\
56.64	0.01\\
56.65	0.01\\
56.66	0.01\\
56.67	0.01\\
56.68	0.01\\
56.69	0.01\\
56.7	0.01\\
56.71	0.01\\
56.72	0.01\\
56.73	0.01\\
56.74	0.01\\
56.75	0.01\\
56.76	0.01\\
56.77	0.01\\
56.78	0.01\\
56.79	0.01\\
56.8	0.01\\
56.81	0.01\\
56.82	0.01\\
56.83	0.01\\
56.84	0.01\\
56.85	0.01\\
56.86	0.01\\
56.87	0.01\\
56.88	0.01\\
56.89	0.01\\
56.9	0.01\\
56.91	0.01\\
56.92	0.01\\
56.93	0.01\\
56.94	0.01\\
56.95	0.01\\
56.96	0.01\\
56.97	0.01\\
56.98	0.01\\
56.99	0.01\\
57	0.01\\
57.01	0.01\\
57.02	0.01\\
57.03	0.01\\
57.04	0.01\\
57.05	0.01\\
57.06	0.01\\
57.07	0.01\\
57.08	0.01\\
57.09	0.01\\
57.1	0.01\\
57.11	0.01\\
57.12	0.01\\
57.13	0.01\\
57.14	0.01\\
57.15	0.01\\
57.16	0.01\\
57.17	0.01\\
57.18	0.01\\
57.19	0.01\\
57.2	0.01\\
57.21	0.01\\
57.22	0.01\\
57.23	0.01\\
57.24	0.01\\
57.25	0.01\\
57.26	0.01\\
57.27	0.01\\
57.28	0.01\\
57.29	0.01\\
57.3	0.01\\
57.31	0.01\\
57.32	0.01\\
57.33	0.01\\
57.34	0.01\\
57.35	0.01\\
57.36	0.01\\
57.37	0.01\\
57.38	0.01\\
57.39	0.01\\
57.4	0.01\\
57.41	0.01\\
57.42	0.01\\
57.43	0.01\\
57.44	0.01\\
57.45	0.01\\
57.46	0.01\\
57.47	0.01\\
57.48	0.01\\
57.49	0.01\\
57.5	0.01\\
57.51	0.01\\
57.52	0.01\\
57.53	0.01\\
57.54	0.01\\
57.55	0.01\\
57.56	0.01\\
57.57	0.01\\
57.58	0.01\\
57.59	0.01\\
57.6	0.01\\
57.61	0.01\\
57.62	0.01\\
57.63	0.01\\
57.64	0.01\\
57.65	0.01\\
57.66	0.01\\
57.67	0.01\\
57.68	0.01\\
57.69	0.01\\
57.7	0.01\\
57.71	0.01\\
57.72	0.01\\
57.73	0.01\\
57.74	0.01\\
57.75	0.01\\
57.76	0.01\\
57.77	0.01\\
57.78	0.01\\
57.79	0.01\\
57.8	0.01\\
57.81	0.01\\
57.82	0.01\\
57.83	0.01\\
57.84	0.01\\
57.85	0.01\\
57.86	0.01\\
57.87	0.01\\
57.88	0.01\\
57.89	0.01\\
57.9	0.01\\
57.91	0.01\\
57.92	0.01\\
57.93	0.01\\
57.94	0.01\\
57.95	0.01\\
57.96	0.01\\
57.97	0.01\\
57.98	0.01\\
57.99	0.01\\
58	0.01\\
58.01	0.01\\
58.02	0.01\\
58.03	0.01\\
58.04	0.01\\
58.05	0.01\\
58.06	0.01\\
58.07	0.01\\
58.08	0.01\\
58.09	0.01\\
58.1	0.01\\
58.11	0.01\\
58.12	0.01\\
58.13	0.01\\
58.14	0.01\\
58.15	0.01\\
58.16	0.01\\
58.17	0.01\\
58.18	0.01\\
58.19	0.01\\
58.2	0.01\\
58.21	0.01\\
58.22	0.01\\
58.23	0.01\\
58.24	0.01\\
58.25	0.01\\
58.26	0.01\\
58.27	0.01\\
58.28	0.01\\
58.29	0.01\\
58.3	0.01\\
58.31	0.01\\
58.32	0.01\\
58.33	0.01\\
58.34	0.01\\
58.35	0.01\\
58.36	0.01\\
58.37	0.01\\
58.38	0.01\\
58.39	0.01\\
58.4	0.01\\
58.41	0.01\\
58.42	0.01\\
58.43	0.01\\
58.44	0.01\\
58.45	0.01\\
58.46	0.01\\
58.47	0.01\\
58.48	0.01\\
58.49	0.01\\
58.5	0.01\\
58.51	0.01\\
58.52	0.01\\
58.53	0.01\\
58.54	0.01\\
58.55	0.01\\
58.56	0.01\\
58.57	0.01\\
58.58	0.01\\
58.59	0.01\\
58.6	0.01\\
58.61	0.01\\
58.62	0.01\\
58.63	0.01\\
58.64	0.01\\
58.65	0.01\\
58.66	0.01\\
58.67	0.01\\
58.68	0.01\\
58.69	0.01\\
58.7	0.01\\
58.71	0.01\\
58.72	0.01\\
58.73	0.01\\
58.74	0.01\\
58.75	0.01\\
58.76	0.01\\
58.77	0.01\\
58.78	0.01\\
58.79	0.01\\
58.8	0.01\\
58.81	0.01\\
58.82	0.01\\
58.83	0.01\\
58.84	0.01\\
58.85	0.01\\
58.86	0.01\\
58.87	0.01\\
58.88	0.01\\
58.89	0.01\\
58.9	0.01\\
58.91	0.01\\
58.92	0.01\\
58.93	0.01\\
58.94	0.01\\
58.95	0.01\\
58.96	0.01\\
58.97	0.01\\
58.98	0.01\\
58.99	0.01\\
59	0.01\\
59.01	0.01\\
59.02	0.01\\
59.03	0.01\\
59.04	0.01\\
59.05	0.01\\
59.06	0.01\\
59.07	0.01\\
59.08	0.01\\
59.09	0.01\\
59.1	0.01\\
59.11	0.01\\
59.12	0.01\\
59.13	0.01\\
59.14	0.01\\
59.15	0.01\\
59.16	0.01\\
59.17	0.01\\
59.18	0.01\\
59.19	0.01\\
59.2	0.01\\
59.21	0.01\\
59.22	0.01\\
59.23	0.01\\
59.24	0.01\\
59.25	0.01\\
59.26	0.01\\
59.27	0.01\\
59.28	0.01\\
59.29	0.01\\
59.3	0.01\\
59.31	0.01\\
59.32	0.01\\
59.33	0.01\\
59.34	0.01\\
59.35	0.01\\
59.36	0.01\\
59.37	0.01\\
59.38	0.01\\
59.39	0.01\\
59.4	0.01\\
59.41	0.01\\
59.42	0.01\\
59.43	0.01\\
59.44	0.01\\
59.45	0.01\\
59.46	0.01\\
59.47	0.01\\
59.48	0.01\\
59.49	0.01\\
59.5	0.01\\
59.51	0.01\\
59.52	0.01\\
59.53	0.01\\
59.54	0.01\\
59.55	0.01\\
59.56	0.01\\
59.57	0.01\\
59.58	0.01\\
59.59	0.01\\
59.6	0.01\\
59.61	0.01\\
59.62	0.01\\
59.63	0.01\\
59.64	0.01\\
59.65	0.01\\
59.66	0.01\\
59.67	0.01\\
59.68	0.01\\
59.69	0.01\\
59.7	0.01\\
59.71	0.01\\
59.72	0.01\\
59.73	0.01\\
59.74	0.01\\
59.75	0.01\\
59.76	0.01\\
59.77	0.01\\
59.78	0.01\\
59.79	0.01\\
59.8	0.01\\
59.81	0.01\\
59.82	0.01\\
59.83	0.01\\
59.84	0.01\\
59.85	0.01\\
59.86	0.01\\
59.87	0.01\\
59.88	0.01\\
59.89	0.01\\
59.9	0.01\\
59.91	0.01\\
59.92	0.01\\
59.93	0.01\\
59.94	0.01\\
59.95	0.01\\
59.96	0.01\\
59.97	0.01\\
59.98	0.01\\
59.99	0.01\\
60	0.01\\
60.01	0.01\\
60.02	0.01\\
60.03	0.01\\
60.04	0.01\\
60.05	0.01\\
60.06	0.01\\
60.07	0.01\\
60.08	0.01\\
60.09	0.01\\
60.1	0.01\\
60.11	0.01\\
60.12	0.01\\
60.13	0.01\\
60.14	0.01\\
60.15	0.01\\
60.16	0.01\\
60.17	0.01\\
60.18	0.01\\
60.19	0.01\\
60.2	0.01\\
60.21	0.01\\
60.22	0.01\\
60.23	0.01\\
60.24	0.01\\
60.25	0.01\\
60.26	0.01\\
60.27	0.01\\
60.28	0.01\\
60.29	0.01\\
60.3	0.01\\
60.31	0.01\\
60.32	0.01\\
60.33	0.01\\
60.34	0.01\\
60.35	0.01\\
60.36	0.01\\
60.37	0.01\\
60.38	0.01\\
60.39	0.01\\
60.4	0.01\\
60.41	0.01\\
60.42	0.01\\
60.43	0.01\\
60.44	0.01\\
60.45	0.01\\
60.46	0.01\\
60.47	0.01\\
60.48	0.01\\
60.49	0.01\\
60.5	0.01\\
60.51	0.01\\
60.52	0.01\\
60.53	0.01\\
60.54	0.01\\
60.55	0.01\\
60.56	0.01\\
60.57	0.01\\
60.58	0.01\\
60.59	0.01\\
60.6	0.01\\
60.61	0.01\\
60.62	0.01\\
60.63	0.01\\
60.64	0.01\\
60.65	0.01\\
60.66	0.01\\
60.67	0.01\\
60.68	0.01\\
60.69	0.01\\
60.7	0.01\\
60.71	0.01\\
60.72	0.01\\
60.73	0.01\\
60.74	0.01\\
60.75	0.01\\
60.76	0.01\\
60.77	0.01\\
60.78	0.01\\
60.79	0.01\\
60.8	0.01\\
60.81	0.01\\
60.82	0.01\\
60.83	0.01\\
60.84	0.01\\
60.85	0.01\\
60.86	0.01\\
60.87	0.01\\
60.88	0.01\\
60.89	0.01\\
60.9	0.01\\
60.91	0.01\\
60.92	0.01\\
60.93	0.01\\
60.94	0.01\\
60.95	0.01\\
60.96	0.01\\
60.97	0.01\\
60.98	0.01\\
60.99	0.01\\
61	0.01\\
61.01	0.01\\
61.02	0.01\\
61.03	0.01\\
61.04	0.01\\
61.05	0.01\\
61.06	0.01\\
61.07	0.01\\
61.08	0.01\\
61.09	0.01\\
61.1	0.01\\
61.11	0.01\\
61.12	0.01\\
61.13	0.01\\
61.14	0.01\\
61.15	0.01\\
61.16	0.01\\
61.17	0.01\\
61.18	0.01\\
61.19	0.01\\
61.2	0.01\\
61.21	0.01\\
61.22	0.01\\
61.23	0.01\\
61.24	0.01\\
61.25	0.01\\
61.26	0.01\\
61.27	0.01\\
61.28	0.01\\
61.29	0.01\\
61.3	0.01\\
61.31	0.01\\
61.32	0.01\\
61.33	0.01\\
61.34	0.01\\
61.35	0.01\\
61.36	0.01\\
61.37	0.01\\
61.38	0.01\\
61.39	0.01\\
61.4	0.01\\
61.41	0.01\\
61.42	0.01\\
61.43	0.01\\
61.44	0.01\\
61.45	0.01\\
61.46	0.01\\
61.47	0.01\\
61.48	0.01\\
61.49	0.01\\
61.5	0.01\\
61.51	0.01\\
61.52	0.01\\
61.53	0.01\\
61.54	0.01\\
61.55	0.01\\
61.56	0.01\\
61.57	0.01\\
61.58	0.01\\
61.59	0.01\\
61.6	0.01\\
61.61	0.01\\
61.62	0.01\\
61.63	0.01\\
61.64	0.01\\
61.65	0.01\\
61.66	0.01\\
61.67	0.01\\
61.68	0.01\\
61.69	0.01\\
61.7	0.01\\
61.71	0.01\\
61.72	0.01\\
61.73	0.01\\
61.74	0.01\\
61.75	0.01\\
61.76	0.01\\
61.77	0.01\\
61.78	0.01\\
61.79	0.01\\
61.8	0.01\\
61.81	0.01\\
61.82	0.01\\
61.83	0.01\\
61.84	0.01\\
61.85	0.01\\
61.86	0.01\\
61.87	0.01\\
61.88	0.01\\
61.89	0.01\\
61.9	0.01\\
61.91	0.01\\
61.92	0.01\\
61.93	0.01\\
61.94	0.01\\
61.95	0.01\\
61.96	0.01\\
61.97	0.01\\
61.98	0.01\\
61.99	0.01\\
62	0.01\\
62.01	0.01\\
62.02	0.01\\
62.03	0.01\\
62.04	0.01\\
62.05	0.01\\
62.06	0.01\\
62.07	0.01\\
62.08	0.01\\
62.09	0.01\\
62.1	0.01\\
62.11	0.01\\
62.12	0.01\\
62.13	0.01\\
62.14	0.01\\
62.15	0.01\\
62.16	0.01\\
62.17	0.01\\
62.18	0.01\\
62.19	0.01\\
62.2	0.01\\
62.21	0.01\\
62.22	0.01\\
62.23	0.01\\
62.24	0.01\\
62.25	0.01\\
62.26	0.01\\
62.27	0.01\\
62.28	0.01\\
62.29	0.01\\
62.3	0.01\\
62.31	0.01\\
62.32	0.01\\
62.33	0.01\\
62.34	0.01\\
62.35	0.01\\
62.36	0.01\\
62.37	0.01\\
62.38	0.01\\
62.39	0.01\\
62.4	0.01\\
62.41	0.01\\
62.42	0.01\\
62.43	0.01\\
62.44	0.01\\
62.45	0.01\\
62.46	0.01\\
62.47	0.01\\
62.48	0.01\\
62.49	0.01\\
62.5	0.01\\
62.51	0.01\\
62.52	0.01\\
62.53	0.01\\
62.54	0.01\\
62.55	0.01\\
62.56	0.01\\
62.57	0.01\\
62.58	0.01\\
62.59	0.01\\
62.6	0.01\\
62.61	0.01\\
62.62	0.01\\
62.63	0.01\\
62.64	0.01\\
62.65	0.01\\
62.66	0.01\\
62.67	0.01\\
62.68	0.01\\
62.69	0.01\\
62.7	0.01\\
62.71	0.01\\
62.72	0.01\\
62.73	0.01\\
62.74	0.01\\
62.75	0.01\\
62.76	0.01\\
62.77	0.01\\
62.78	0.01\\
62.79	0.01\\
62.8	0.01\\
62.81	0.01\\
62.82	0.01\\
62.83	0.01\\
62.84	0.01\\
62.85	0.01\\
62.86	0.01\\
62.87	0.01\\
62.88	0.01\\
62.89	0.01\\
62.9	0.01\\
62.91	0.01\\
62.92	0.01\\
62.93	0.01\\
62.94	0.01\\
62.95	0.01\\
62.96	0.01\\
62.97	0.01\\
62.98	0.01\\
62.99	0.01\\
63	0.01\\
63.01	0.01\\
63.02	0.01\\
63.03	0.01\\
63.04	0.01\\
63.05	0.01\\
63.06	0.01\\
63.07	0.01\\
63.08	0.01\\
63.09	0.01\\
63.1	0.01\\
63.11	0.01\\
63.12	0.01\\
63.13	0.01\\
63.14	0.01\\
63.15	0.01\\
63.16	0.01\\
63.17	0.01\\
63.18	0.01\\
63.19	0.01\\
63.2	0.01\\
63.21	0.01\\
63.22	0.01\\
63.23	0.01\\
63.24	0.01\\
63.25	0.01\\
63.26	0.01\\
63.27	0.01\\
63.28	0.01\\
63.29	0.01\\
63.3	0.01\\
63.31	0.01\\
63.32	0.01\\
63.33	0.01\\
63.34	0.01\\
63.35	0.01\\
63.36	0.01\\
63.37	0.01\\
63.38	0.01\\
63.39	0.01\\
63.4	0.01\\
63.41	0.01\\
63.42	0.01\\
63.43	0.01\\
63.44	0.01\\
63.45	0.01\\
63.46	0.01\\
63.47	0.01\\
63.48	0.01\\
63.49	0.01\\
63.5	0.01\\
63.51	0.01\\
63.52	0.01\\
63.53	0.01\\
63.54	0.01\\
63.55	0.01\\
63.56	0.01\\
63.57	0.01\\
63.58	0.01\\
63.59	0.01\\
63.6	0.01\\
63.61	0.01\\
63.62	0.01\\
63.63	0.01\\
63.64	0.01\\
63.65	0.01\\
63.66	0.01\\
63.67	0.01\\
63.68	0.01\\
63.69	0.01\\
63.7	0.01\\
63.71	0.01\\
63.72	0.01\\
63.73	0.01\\
63.74	0.01\\
63.75	0.01\\
63.76	0.01\\
63.77	0.01\\
63.78	0.01\\
63.79	0.01\\
63.8	0.01\\
63.81	0.01\\
63.82	0.01\\
63.83	0.01\\
63.84	0.01\\
63.85	0.01\\
63.86	0.01\\
63.87	0.01\\
63.88	0.01\\
63.89	0.01\\
63.9	0.01\\
63.91	0.01\\
63.92	0.01\\
63.93	0.01\\
63.94	0.01\\
63.95	0.01\\
63.96	0.01\\
63.97	0.01\\
63.98	0.01\\
63.99	0.01\\
64	0.01\\
64.01	0.01\\
64.02	0.01\\
64.03	0.01\\
64.04	0.01\\
64.05	0.01\\
64.06	0.01\\
64.07	0.01\\
64.08	0.01\\
64.09	0.01\\
64.1	0.01\\
64.11	0.01\\
64.12	0.01\\
64.13	0.01\\
64.14	0.01\\
64.15	0.01\\
64.16	0.01\\
64.17	0.01\\
64.18	0.01\\
64.19	0.01\\
64.2	0.01\\
64.21	0.01\\
64.22	0.01\\
64.23	0.01\\
64.24	0.01\\
64.25	0.01\\
64.26	0.01\\
64.27	0.01\\
64.28	0.01\\
64.29	0.01\\
64.3	0.01\\
64.31	0.01\\
64.32	0.01\\
64.33	0.01\\
64.34	0.01\\
64.35	0.01\\
64.36	0.01\\
64.37	0.01\\
64.38	0.01\\
64.39	0.01\\
64.4	0.01\\
64.41	0.01\\
64.42	0.01\\
64.43	0.01\\
64.44	0.01\\
64.45	0.01\\
64.46	0.01\\
64.47	0.01\\
64.48	0.01\\
64.49	0.01\\
64.5	0.01\\
64.51	0.01\\
64.52	0.01\\
64.53	0.01\\
64.54	0.01\\
64.55	0.01\\
64.56	0.01\\
64.57	0.01\\
64.58	0.01\\
64.59	0.01\\
64.6	0.01\\
64.61	0.01\\
64.62	0.01\\
64.63	0.01\\
64.64	0.01\\
64.65	0.01\\
64.66	0.01\\
64.67	0.01\\
64.68	0.01\\
64.69	0.01\\
64.7	0.01\\
64.71	0.01\\
64.72	0.01\\
64.73	0.01\\
64.74	0.01\\
64.75	0.01\\
64.76	0.01\\
64.77	0.01\\
64.78	0.01\\
64.79	0.01\\
64.8	0.01\\
64.81	0.01\\
64.82	0.01\\
64.83	0.01\\
64.84	0.01\\
64.85	0.01\\
64.86	0.01\\
64.87	0.01\\
64.88	0.01\\
64.89	0.01\\
64.9	0.01\\
64.91	0.01\\
64.92	0.01\\
64.93	0.01\\
64.94	0.01\\
64.95	0.01\\
64.96	0.01\\
64.97	0.01\\
64.98	0.01\\
64.99	0.01\\
65	0.01\\
65.01	0.01\\
65.02	0.01\\
65.03	0.01\\
65.04	0.01\\
65.05	0.01\\
65.06	0.01\\
65.07	0.01\\
65.08	0.01\\
65.09	0.01\\
65.1	0.01\\
65.11	0.01\\
65.12	0.01\\
65.13	0.01\\
65.14	0.01\\
65.15	0.01\\
65.16	0.01\\
65.17	0.01\\
65.18	0.01\\
65.19	0.01\\
65.2	0.01\\
65.21	0.01\\
65.22	0.01\\
65.23	0.01\\
65.24	0.01\\
65.25	0.01\\
65.26	0.01\\
65.27	0.01\\
65.28	0.01\\
65.29	0.01\\
65.3	0.01\\
65.31	0.01\\
65.32	0.01\\
65.33	0.01\\
65.34	0.01\\
65.35	0.01\\
65.36	0.01\\
65.37	0.01\\
65.38	0.01\\
65.39	0.01\\
65.4	0.01\\
65.41	0.01\\
65.42	0.01\\
65.43	0.01\\
65.44	0.01\\
65.45	0.01\\
65.46	0.01\\
65.47	0.01\\
65.48	0.01\\
65.49	0.01\\
65.5	0.01\\
65.51	0.01\\
65.52	0.01\\
65.53	0.01\\
65.54	0.01\\
65.55	0.01\\
65.56	0.01\\
65.57	0.01\\
65.58	0.01\\
65.59	0.01\\
65.6	0.01\\
65.61	0.01\\
65.62	0.01\\
65.63	0.01\\
65.64	0.01\\
65.65	0.01\\
65.66	0.01\\
65.67	0.01\\
65.68	0.01\\
65.69	0.01\\
65.7	0.01\\
65.71	0.01\\
65.72	0.01\\
65.73	0.01\\
65.74	0.01\\
65.75	0.01\\
65.76	0.01\\
65.77	0.01\\
65.78	0.01\\
65.79	0.01\\
65.8	0.01\\
65.81	0.01\\
65.82	0.01\\
65.83	0.01\\
65.84	0.01\\
65.85	0.01\\
65.86	0.01\\
65.87	0.01\\
65.88	0.01\\
65.89	0.01\\
65.9	0.01\\
65.91	0.01\\
65.92	0.01\\
65.93	0.01\\
65.94	0.01\\
65.95	0.01\\
65.96	0.01\\
65.97	0.01\\
65.98	0.01\\
65.99	0.01\\
66	0.01\\
66.01	0.01\\
66.02	0.01\\
66.03	0.01\\
66.04	0.01\\
66.05	0.01\\
66.06	0.01\\
66.07	0.01\\
66.08	0.01\\
66.09	0.01\\
66.1	0.01\\
66.11	0.01\\
66.12	0.01\\
66.13	0.01\\
66.14	0.01\\
66.15	0.01\\
66.16	0.01\\
66.17	0.01\\
66.18	0.01\\
66.19	0.01\\
66.2	0.01\\
66.21	0.01\\
66.22	0.01\\
66.23	0.01\\
66.24	0.01\\
66.25	0.01\\
66.26	0.01\\
66.27	0.01\\
66.28	0.01\\
66.29	0.01\\
66.3	0.01\\
66.31	0.01\\
66.32	0.01\\
66.33	0.01\\
66.34	0.01\\
66.35	0.01\\
66.36	0.01\\
66.37	0.01\\
66.38	0.01\\
66.39	0.01\\
66.4	0.01\\
66.41	0.01\\
66.42	0.01\\
66.43	0.01\\
66.44	0.01\\
66.45	0.01\\
66.46	0.01\\
66.47	0.01\\
66.48	0.01\\
66.49	0.01\\
66.5	0.01\\
66.51	0.01\\
66.52	0.01\\
66.53	0.01\\
66.54	0.01\\
66.55	0.01\\
66.56	0.01\\
66.57	0.01\\
66.58	0.01\\
66.59	0.01\\
66.6	0.01\\
66.61	0.01\\
66.62	0.01\\
66.63	0.01\\
66.64	0.01\\
66.65	0.01\\
66.66	0.01\\
66.67	0.01\\
66.68	0.01\\
66.69	0.01\\
66.7	0.01\\
66.71	0.01\\
66.72	0.01\\
66.73	0.01\\
66.74	0.01\\
66.75	0.01\\
66.76	0.01\\
66.77	0.01\\
66.78	0.01\\
66.79	0.01\\
66.8	0.01\\
66.81	0.01\\
66.82	0.01\\
66.83	0.01\\
66.84	0.01\\
66.85	0.01\\
66.86	0.01\\
66.87	0.01\\
66.88	0.01\\
66.89	0.01\\
66.9	0.01\\
66.91	0.01\\
66.92	0.01\\
66.93	0.01\\
66.94	0.01\\
66.95	0.01\\
66.96	0.01\\
66.97	0.01\\
66.98	0.01\\
66.99	0.01\\
67	0.01\\
67.01	0.01\\
67.02	0.01\\
67.03	0.01\\
67.04	0.01\\
67.05	0.01\\
67.06	0.01\\
67.07	0.01\\
67.08	0.01\\
67.09	0.01\\
67.1	0.01\\
67.11	0.01\\
67.12	0.01\\
67.13	0.01\\
67.14	0.01\\
67.15	0.01\\
67.16	0.01\\
67.17	0.01\\
67.18	0.01\\
67.19	0.01\\
67.2	0.01\\
67.21	0.01\\
67.22	0.01\\
67.23	0.01\\
67.24	0.01\\
67.25	0.01\\
67.26	0.01\\
67.27	0.01\\
67.28	0.01\\
67.29	0.01\\
67.3	0.01\\
67.31	0.01\\
67.32	0.01\\
67.33	0.01\\
67.34	0.01\\
67.35	0.01\\
67.36	0.01\\
67.37	0.01\\
67.38	0.01\\
67.39	0.01\\
67.4	0.01\\
67.41	0.01\\
67.42	0.01\\
67.43	0.01\\
67.44	0.01\\
67.45	0.01\\
67.46	0.01\\
67.47	0.01\\
67.48	0.01\\
67.49	0.01\\
67.5	0.01\\
67.51	0.01\\
67.52	0.01\\
67.53	0.01\\
67.54	0.01\\
67.55	0.01\\
67.56	0.01\\
67.57	0.01\\
67.58	0.01\\
67.59	0.01\\
67.6	0.01\\
67.61	0.01\\
67.62	0.01\\
67.63	0.01\\
67.64	0.01\\
67.65	0.01\\
67.66	0.01\\
67.67	0.01\\
67.68	0.01\\
67.69	0.01\\
67.7	0.01\\
67.71	0.01\\
67.72	0.01\\
67.73	0.01\\
67.74	0.01\\
67.75	0.01\\
67.76	0.01\\
67.77	0.01\\
67.78	0.01\\
67.79	0.01\\
67.8	0.01\\
67.81	0.01\\
67.82	0.01\\
67.83	0.01\\
67.84	0.01\\
67.85	0.01\\
67.86	0.01\\
67.87	0.01\\
67.88	0.01\\
67.89	0.01\\
67.9	0.01\\
67.91	0.01\\
67.92	0.01\\
67.93	0.01\\
67.94	0.01\\
67.95	0.01\\
67.96	0.01\\
67.97	0.01\\
67.98	0.01\\
67.99	0.01\\
68	0.01\\
68.01	0.01\\
68.02	0.01\\
68.03	0.01\\
68.04	0.01\\
68.05	0.01\\
68.06	0.01\\
68.07	0.01\\
68.08	0.01\\
68.09	0.01\\
68.1	0.01\\
68.11	0.01\\
68.12	0.01\\
68.13	0.01\\
68.14	0.01\\
68.15	0.01\\
68.16	0.01\\
68.17	0.01\\
68.18	0.01\\
68.19	0.01\\
68.2	0.01\\
68.21	0.01\\
68.22	0.01\\
68.23	0.01\\
68.24	0.01\\
68.25	0.01\\
68.26	0.01\\
68.27	0.01\\
68.28	0.01\\
68.29	0.01\\
68.3	0.01\\
68.31	0.01\\
68.32	0.01\\
68.33	0.01\\
68.34	0.01\\
68.35	0.01\\
68.36	0.01\\
68.37	0.01\\
68.38	0.01\\
68.39	0.01\\
68.4	0.01\\
68.41	0.01\\
68.42	0.01\\
68.43	0.01\\
68.44	0.01\\
68.45	0.01\\
68.46	0.01\\
68.47	0.01\\
68.48	0.01\\
68.49	0.01\\
68.5	0.01\\
68.51	0.01\\
68.52	0.01\\
68.53	0.01\\
68.54	0.01\\
68.55	0.01\\
68.56	0.01\\
68.57	0.01\\
68.58	0.01\\
68.59	0.01\\
68.6	0.01\\
68.61	0.01\\
68.62	0.01\\
68.63	0.01\\
68.64	0.01\\
68.65	0.01\\
68.66	0.01\\
68.67	0.01\\
68.68	0.01\\
68.69	0.01\\
68.7	0.01\\
68.71	0.01\\
68.72	0.01\\
68.73	0.01\\
68.74	0.01\\
68.75	0.01\\
68.76	0.01\\
68.77	0.01\\
68.78	0.01\\
68.79	0.01\\
68.8	0.01\\
68.81	0.01\\
68.82	0.01\\
68.83	0.01\\
68.84	0.01\\
68.85	0.01\\
68.86	0.01\\
68.87	0.01\\
68.88	0.01\\
68.89	0.01\\
68.9	0.01\\
68.91	0.01\\
68.92	0.01\\
68.93	0.01\\
68.94	0.01\\
68.95	0.01\\
68.96	0.01\\
68.97	0.01\\
68.98	0.01\\
68.99	0.01\\
69	0.01\\
69.01	0.01\\
69.02	0.01\\
69.03	0.01\\
69.04	0.01\\
69.05	0.01\\
69.06	0.01\\
69.07	0.01\\
69.08	0.01\\
69.09	0.01\\
69.1	0.01\\
69.11	0.01\\
69.12	0.01\\
69.13	0.01\\
69.14	0.01\\
69.15	0.01\\
69.16	0.01\\
69.17	0.01\\
69.18	0.01\\
69.19	0.01\\
69.2	0.01\\
69.21	0.01\\
69.22	0.01\\
69.23	0.01\\
69.24	0.01\\
69.25	0.01\\
69.26	0.01\\
69.27	0.01\\
69.28	0.01\\
69.29	0.01\\
69.3	0.01\\
69.31	0.01\\
69.32	0.01\\
69.33	0.01\\
69.34	0.01\\
69.35	0.01\\
69.36	0.01\\
69.37	0.01\\
69.38	0.01\\
69.39	0.01\\
69.4	0.01\\
69.41	0.01\\
69.42	0.01\\
69.43	0.01\\
69.44	0.01\\
69.45	0.01\\
69.46	0.01\\
69.47	0.01\\
69.48	0.01\\
69.49	0.01\\
69.5	0.01\\
69.51	0.01\\
69.52	0.01\\
69.53	0.01\\
69.54	0.01\\
69.55	0.01\\
69.56	0.01\\
69.57	0.01\\
69.58	0.01\\
69.59	0.01\\
69.6	0.01\\
69.61	0.01\\
69.62	0.01\\
69.63	0.01\\
69.64	0.01\\
69.65	0.01\\
69.66	0.01\\
69.67	0.01\\
69.68	0.01\\
69.69	0.01\\
69.7	0.01\\
69.71	0.01\\
69.72	0.01\\
69.73	0.01\\
69.74	0.01\\
69.75	0.01\\
69.76	0.01\\
69.77	0.01\\
69.78	0.01\\
69.79	0.01\\
69.8	0.01\\
69.81	0.01\\
69.82	0.01\\
69.83	0.01\\
69.84	0.01\\
69.85	0.01\\
69.86	0.01\\
69.87	0.01\\
69.88	0.01\\
69.89	0.01\\
69.9	0.01\\
69.91	0.01\\
69.92	0.01\\
69.93	0.01\\
69.94	0.01\\
69.95	0.01\\
69.96	0.01\\
69.97	0.01\\
69.98	0.01\\
69.99	0.01\\
70	0.01\\
70.01	0.01\\
70.02	0.01\\
70.03	0.01\\
70.04	0.01\\
70.05	0.01\\
70.06	0.01\\
70.07	0.01\\
70.08	0.01\\
70.09	0.01\\
70.1	0.01\\
70.11	0.01\\
70.12	0.01\\
70.13	0.01\\
70.14	0.01\\
70.15	0.01\\
70.16	0.01\\
70.17	0.01\\
70.18	0.01\\
70.19	0.01\\
70.2	0.01\\
70.21	0.01\\
70.22	0.01\\
70.23	0.01\\
70.24	0.01\\
70.25	0.01\\
70.26	0.01\\
70.27	0.01\\
70.28	0.01\\
70.29	0.01\\
70.3	0.01\\
70.31	0.01\\
70.32	0.01\\
70.33	0.01\\
70.34	0.01\\
70.35	0.01\\
70.36	0.01\\
70.37	0.01\\
70.38	0.01\\
70.39	0.01\\
70.4	0.01\\
70.41	0.01\\
70.42	0.01\\
70.43	0.01\\
70.44	0.01\\
70.45	0.01\\
70.46	0.01\\
70.47	0.01\\
70.48	0.01\\
70.49	0.01\\
70.5	0.01\\
70.51	0.01\\
70.52	0.01\\
70.53	0.01\\
70.54	0.01\\
70.55	0.01\\
70.56	0.01\\
70.57	0.01\\
70.58	0.01\\
70.59	0.01\\
70.6	0.01\\
70.61	0.01\\
70.62	0.01\\
70.63	0.01\\
70.64	0.01\\
70.65	0.01\\
70.66	0.01\\
70.67	0.01\\
70.68	0.01\\
70.69	0.01\\
70.7	0.01\\
70.71	0.01\\
70.72	0.01\\
70.73	0.01\\
70.74	0.01\\
70.75	0.01\\
70.76	0.01\\
70.77	0.01\\
70.78	0.01\\
70.79	0.01\\
70.8	0.01\\
70.81	0.01\\
70.82	0.01\\
70.83	0.01\\
70.84	0.01\\
70.85	0.01\\
70.86	0.01\\
70.87	0.01\\
70.88	0.01\\
70.89	0.01\\
70.9	0.01\\
70.91	0.01\\
70.92	0.01\\
70.93	0.01\\
70.94	0.01\\
70.95	0.01\\
70.96	0.01\\
70.97	0.01\\
70.98	0.01\\
70.99	0.01\\
71	0.01\\
71.01	0.01\\
71.02	0.01\\
71.03	0.01\\
71.04	0.01\\
71.05	0.01\\
71.06	0.01\\
71.07	0.01\\
71.08	0.01\\
71.09	0.01\\
71.1	0.01\\
71.11	0.01\\
71.12	0.01\\
71.13	0.01\\
71.14	0.01\\
71.15	0.01\\
71.16	0.01\\
71.17	0.01\\
71.18	0.01\\
71.19	0.01\\
71.2	0.01\\
71.21	0.01\\
71.22	0.01\\
71.23	0.01\\
71.24	0.01\\
71.25	0.01\\
71.26	0.01\\
71.27	0.01\\
71.28	0.01\\
71.29	0.01\\
71.3	0.01\\
71.31	0.01\\
71.32	0.01\\
71.33	0.01\\
71.34	0.01\\
71.35	0.01\\
71.36	0.01\\
71.37	0.01\\
71.38	0.01\\
71.39	0.01\\
71.4	0.01\\
71.41	0.01\\
71.42	0.01\\
71.43	0.01\\
71.44	0.01\\
71.45	0.01\\
71.46	0.01\\
71.47	0.01\\
71.48	0.01\\
71.49	0.01\\
71.5	0.01\\
71.51	0.01\\
71.52	0.01\\
71.53	0.01\\
71.54	0.01\\
71.55	0.01\\
71.56	0.01\\
71.57	0.01\\
71.58	0.01\\
71.59	0.01\\
71.6	0.01\\
71.61	0.01\\
71.62	0.01\\
71.63	0.01\\
71.64	0.01\\
71.65	0.01\\
71.66	0.01\\
71.67	0.01\\
71.68	0.01\\
71.69	0.01\\
71.7	0.01\\
71.71	0.01\\
71.72	0.01\\
71.73	0.01\\
71.74	0.01\\
71.75	0.01\\
71.76	0.01\\
71.77	0.01\\
71.78	0.01\\
71.79	0.01\\
71.8	0.01\\
71.81	0.01\\
71.82	0.01\\
71.83	0.01\\
71.84	0.01\\
71.85	0.01\\
71.86	0.01\\
71.87	0.01\\
71.88	0.01\\
71.89	0.01\\
71.9	0.01\\
71.91	0.01\\
71.92	0.01\\
71.93	0.01\\
71.94	0.01\\
71.95	0.01\\
71.96	0.01\\
71.97	0.01\\
71.98	0.01\\
71.99	0.01\\
72	0.01\\
72.01	0.01\\
72.02	0.01\\
72.03	0.01\\
72.04	0.01\\
72.05	0.01\\
72.06	0.01\\
72.07	0.01\\
72.08	0.01\\
72.09	0.01\\
72.1	0.01\\
72.11	0.01\\
72.12	0.01\\
72.13	0.01\\
72.14	0.01\\
72.15	0.01\\
72.16	0.01\\
72.17	0.01\\
72.18	0.01\\
72.19	0.01\\
72.2	0.01\\
72.21	0.01\\
72.22	0.01\\
72.23	0.01\\
72.24	0.01\\
72.25	0.01\\
72.26	0.01\\
72.27	0.01\\
72.28	0.01\\
72.29	0.01\\
72.3	0.01\\
72.31	0.01\\
72.32	0.01\\
72.33	0.01\\
72.34	0.01\\
72.35	0.01\\
72.36	0.01\\
72.37	0.01\\
72.38	0.01\\
72.39	0.01\\
72.4	0.01\\
72.41	0.01\\
72.42	0.01\\
72.43	0.01\\
72.44	0.01\\
72.45	0.01\\
72.46	0.01\\
72.47	0.01\\
72.48	0.01\\
72.49	0.01\\
72.5	0.01\\
72.51	0.01\\
72.52	0.01\\
72.53	0.01\\
72.54	0.01\\
72.55	0.01\\
72.56	0.01\\
72.57	0.01\\
72.58	0.01\\
72.59	0.01\\
72.6	0.01\\
72.61	0.01\\
72.62	0.01\\
72.63	0.01\\
72.64	0.01\\
72.65	0.01\\
72.66	0.01\\
72.67	0.01\\
72.68	0.01\\
72.69	0.01\\
72.7	0.01\\
72.71	0.01\\
72.72	0.01\\
72.73	0.01\\
72.74	0.01\\
72.75	0.01\\
72.76	0.01\\
72.77	0.01\\
72.78	0.01\\
72.79	0.01\\
72.8	0.01\\
72.81	0.01\\
72.82	0.01\\
72.83	0.01\\
72.84	0.01\\
72.85	0.01\\
72.86	0.01\\
72.87	0.01\\
72.88	0.01\\
72.89	0.01\\
72.9	0.01\\
72.91	0.01\\
72.92	0.01\\
72.93	0.01\\
72.94	0.01\\
72.95	0.01\\
72.96	0.01\\
72.97	0.01\\
72.98	0.01\\
72.99	0.01\\
73	0.01\\
73.01	0.01\\
73.02	0.01\\
73.03	0.01\\
73.04	0.01\\
73.05	0.01\\
73.06	0.01\\
73.07	0.01\\
73.08	0.01\\
73.09	0.01\\
73.1	0.01\\
73.11	0.01\\
73.12	0.01\\
73.13	0.01\\
73.14	0.01\\
73.15	0.01\\
73.16	0.01\\
73.17	0.01\\
73.18	0.01\\
73.19	0.01\\
73.2	0.01\\
73.21	0.01\\
73.22	0.01\\
73.23	0.01\\
73.24	0.01\\
73.25	0.01\\
73.26	0.01\\
73.27	0.01\\
73.28	0.01\\
73.29	0.01\\
73.3	0.01\\
73.31	0.01\\
73.32	0.01\\
73.33	0.01\\
73.34	0.01\\
73.35	0.01\\
73.36	0.01\\
73.37	0.01\\
73.38	0.01\\
73.39	0.01\\
73.4	0.01\\
73.41	0.01\\
73.42	0.01\\
73.43	0.01\\
73.44	0.01\\
73.45	0.01\\
73.46	0.01\\
73.47	0.01\\
73.48	0.01\\
73.49	0.01\\
73.5	0.01\\
73.51	0.01\\
73.52	0.01\\
73.53	0.01\\
73.54	0.01\\
73.55	0.01\\
73.56	0.01\\
73.57	0.01\\
73.58	0.01\\
73.59	0.01\\
73.6	0.01\\
73.61	0.01\\
73.62	0.01\\
73.63	0.01\\
73.64	0.01\\
73.65	0.01\\
73.66	0.01\\
73.67	0.01\\
73.68	0.01\\
73.69	0.01\\
73.7	0.01\\
73.71	0.01\\
73.72	0.01\\
73.73	0.01\\
73.74	0.01\\
73.75	0.01\\
73.76	0.01\\
73.77	0.01\\
73.78	0.01\\
73.79	0.01\\
73.8	0.01\\
73.81	0.01\\
73.82	0.01\\
73.83	0.01\\
73.84	0.01\\
73.85	0.01\\
73.86	0.01\\
73.87	0.01\\
73.88	0.01\\
73.89	0.01\\
73.9	0.01\\
73.91	0.01\\
73.92	0.01\\
73.93	0.01\\
73.94	0.01\\
73.95	0.01\\
73.96	0.01\\
73.97	0.01\\
73.98	0.01\\
73.99	0.01\\
74	0.01\\
74.01	0.01\\
74.02	0.01\\
74.03	0.01\\
74.04	0.01\\
74.05	0.01\\
74.06	0.01\\
74.07	0.01\\
74.08	0.01\\
74.09	0.01\\
74.1	0.01\\
74.11	0.01\\
74.12	0.01\\
74.13	0.01\\
74.14	0.01\\
74.15	0.01\\
74.16	0.01\\
74.17	0.01\\
74.18	0.01\\
74.19	0.01\\
74.2	0.01\\
74.21	0.01\\
74.22	0.01\\
74.23	0.01\\
74.24	0.01\\
74.25	0.01\\
74.26	0.01\\
74.27	0.01\\
74.28	0.01\\
74.29	0.01\\
74.3	0.01\\
74.31	0.01\\
74.32	0.01\\
74.33	0.01\\
74.34	0.01\\
74.35	0.01\\
74.36	0.01\\
74.37	0.01\\
74.38	0.01\\
74.39	0.01\\
74.4	0.01\\
74.41	0.01\\
74.42	0.01\\
74.43	0.01\\
74.44	0.01\\
74.45	0.01\\
74.46	0.01\\
74.47	0.01\\
74.48	0.01\\
74.49	0.01\\
74.5	0.01\\
74.51	0.01\\
74.52	0.01\\
74.53	0.01\\
74.54	0.01\\
74.55	0.01\\
74.56	0.01\\
74.57	0.01\\
74.58	0.01\\
74.59	0.01\\
74.6	0.01\\
74.61	0.01\\
74.62	0.01\\
74.63	0.01\\
74.64	0.01\\
74.65	0.01\\
74.66	0.01\\
74.67	0.01\\
74.68	0.01\\
74.69	0.01\\
74.7	0.01\\
74.71	0.01\\
74.72	0.01\\
74.73	0.01\\
74.74	0.01\\
74.75	0.01\\
74.76	0.01\\
74.77	0.01\\
74.78	0.01\\
74.79	0.01\\
74.8	0.01\\
74.81	0.01\\
74.82	0.01\\
74.83	0.01\\
74.84	0.01\\
74.85	0.01\\
74.86	0.01\\
74.87	0.01\\
74.88	0.01\\
74.89	0.01\\
74.9	0.01\\
74.91	0.01\\
74.92	0.01\\
74.93	0.01\\
74.94	0.01\\
74.95	0.01\\
74.96	0.01\\
74.97	0.01\\
74.98	0.01\\
74.99	0.01\\
75	0.01\\
75.01	0.01\\
75.02	0.01\\
75.03	0.01\\
75.04	0.01\\
75.05	0.01\\
75.06	0.01\\
75.07	0.01\\
75.08	0.01\\
75.09	0.01\\
75.1	0.01\\
75.11	0.01\\
75.12	0.01\\
75.13	0.01\\
75.14	0.01\\
75.15	0.01\\
75.16	0.01\\
75.17	0.01\\
75.18	0.01\\
75.19	0.01\\
75.2	0.01\\
75.21	0.01\\
75.22	0.01\\
75.23	0.01\\
75.24	0.01\\
75.25	0.01\\
75.26	0.01\\
75.27	0.01\\
75.28	0.01\\
75.29	0.01\\
75.3	0.01\\
75.31	0.01\\
75.32	0.01\\
75.33	0.01\\
75.34	0.01\\
75.35	0.01\\
75.36	0.01\\
75.37	0.01\\
75.38	0.01\\
75.39	0.01\\
75.4	0.01\\
75.41	0.01\\
75.42	0.01\\
75.43	0.01\\
75.44	0.01\\
75.45	0.01\\
75.46	0.01\\
75.47	0.01\\
75.48	0.01\\
75.49	0.01\\
75.5	0.01\\
75.51	0.01\\
75.52	0.01\\
75.53	0.01\\
75.54	0.01\\
75.55	0.01\\
75.56	0.01\\
75.57	0.01\\
75.58	0.01\\
75.59	0.01\\
75.6	0.01\\
75.61	0.01\\
75.62	0.01\\
75.63	0.01\\
75.64	0.01\\
75.65	0.01\\
75.66	0.01\\
75.67	0.01\\
75.68	0.01\\
75.69	0.01\\
75.7	0.01\\
75.71	0.01\\
75.72	0.01\\
75.73	0.01\\
75.74	0.01\\
75.75	0.01\\
75.76	0.01\\
75.77	0.01\\
75.78	0.01\\
75.79	0.01\\
75.8	0.01\\
75.81	0.01\\
75.82	0.01\\
75.83	0.01\\
75.84	0.01\\
75.85	0.01\\
75.86	0.01\\
75.87	0.01\\
75.88	0.01\\
75.89	0.01\\
75.9	0.01\\
75.91	0.01\\
75.92	0.01\\
75.93	0.01\\
75.94	0.01\\
75.95	0.01\\
75.96	0.01\\
75.97	0.01\\
75.98	0.01\\
75.99	0.01\\
76	0.01\\
76.01	0.01\\
76.02	0.01\\
76.03	0.01\\
76.04	0.01\\
76.05	0.01\\
76.06	0.01\\
76.07	0.01\\
76.08	0.01\\
76.09	0.01\\
76.1	0.01\\
76.11	0.01\\
76.12	0.01\\
76.13	0.01\\
76.14	0.01\\
76.15	0.01\\
76.16	0.01\\
76.17	0.01\\
76.18	0.01\\
76.19	0.01\\
76.2	0.01\\
76.21	0.01\\
76.22	0.01\\
76.23	0.01\\
76.24	0.01\\
76.25	0.01\\
76.26	0.01\\
76.27	0.01\\
76.28	0.01\\
76.29	0.01\\
76.3	0.01\\
76.31	0.01\\
76.32	0.01\\
76.33	0.01\\
76.34	0.01\\
76.35	0.01\\
76.36	0.01\\
76.37	0.01\\
76.38	0.01\\
76.39	0.01\\
76.4	0.01\\
76.41	0.01\\
76.42	0.01\\
76.43	0.01\\
76.44	0.01\\
76.45	0.01\\
76.46	0.01\\
76.47	0.01\\
76.48	0.01\\
76.49	0.01\\
76.5	0.01\\
76.51	0.01\\
76.52	0.01\\
76.53	0.01\\
76.54	0.01\\
76.55	0.01\\
76.56	0.01\\
76.57	0.01\\
76.58	0.01\\
76.59	0.01\\
76.6	0.01\\
76.61	0.01\\
76.62	0.01\\
76.63	0.01\\
76.64	0.01\\
76.65	0.01\\
76.66	0.01\\
76.67	0.01\\
76.68	0.01\\
76.69	0.01\\
76.7	0.01\\
76.71	0.01\\
76.72	0.01\\
76.73	0.01\\
76.74	0.01\\
76.75	0.01\\
76.76	0.01\\
76.77	0.01\\
76.78	0.01\\
76.79	0.01\\
76.8	0.01\\
76.81	0.01\\
76.82	0.01\\
76.83	0.01\\
76.84	0.01\\
76.85	0.01\\
76.86	0.01\\
76.87	0.01\\
76.88	0.01\\
76.89	0.01\\
76.9	0.01\\
76.91	0.01\\
76.92	0.01\\
76.93	0.01\\
76.94	0.01\\
76.95	0.01\\
76.96	0.01\\
76.97	0.01\\
76.98	0.01\\
76.99	0.01\\
77	0.01\\
77.01	0.01\\
77.02	0.01\\
77.03	0.01\\
77.04	0.01\\
77.05	0.01\\
77.06	0.01\\
77.07	0.01\\
77.08	0.01\\
77.09	0.01\\
77.1	0.01\\
77.11	0.01\\
77.12	0.01\\
77.13	0.01\\
77.14	0.01\\
77.15	0.01\\
77.16	0.01\\
77.17	0.01\\
77.18	0.01\\
77.19	0.01\\
77.2	0.01\\
77.21	0.01\\
77.22	0.01\\
77.23	0.01\\
77.24	0.01\\
77.25	0.01\\
77.26	0.01\\
77.27	0.01\\
77.28	0.01\\
77.29	0.01\\
77.3	0.01\\
77.31	0.01\\
77.32	0.01\\
77.33	0.01\\
77.34	0.01\\
77.35	0.01\\
77.36	0.01\\
77.37	0.01\\
77.38	0.01\\
77.39	0.01\\
77.4	0.01\\
77.41	0.01\\
77.42	0.01\\
77.43	0.01\\
77.44	0.01\\
77.45	0.01\\
77.46	0.01\\
77.47	0.01\\
77.48	0.01\\
77.49	0.01\\
77.5	0.01\\
77.51	0.01\\
77.52	0.01\\
77.53	0.01\\
77.54	0.01\\
77.55	0.01\\
77.56	0.01\\
77.57	0.01\\
77.58	0.01\\
77.59	0.01\\
77.6	0.01\\
77.61	0.01\\
77.62	0.01\\
77.63	0.01\\
77.64	0.01\\
77.65	0.01\\
77.66	0.01\\
77.67	0.01\\
77.68	0.01\\
77.69	0.01\\
77.7	0.01\\
77.71	0.01\\
77.72	0.01\\
77.73	0.01\\
77.74	0.01\\
77.75	0.01\\
77.76	0.01\\
77.77	0.01\\
77.78	0.01\\
77.79	0.01\\
77.8	0.01\\
77.81	0.01\\
77.82	0.01\\
77.83	0.01\\
77.84	0.01\\
77.85	0.01\\
77.86	0.01\\
77.87	0.01\\
77.88	0.01\\
77.89	0.01\\
77.9	0.01\\
77.91	0.01\\
77.92	0.01\\
77.93	0.01\\
77.94	0.01\\
77.95	0.01\\
77.96	0.01\\
77.97	0.01\\
77.98	0.01\\
77.99	0.01\\
78	0.01\\
78.01	0.01\\
78.02	0.01\\
78.03	0.01\\
78.04	0.01\\
78.05	0.01\\
78.06	0.01\\
78.07	0.01\\
78.08	0.01\\
78.09	0.01\\
78.1	0.01\\
78.11	0.01\\
78.12	0.01\\
78.13	0.01\\
78.14	0.01\\
78.15	0.01\\
78.16	0.01\\
78.17	0.01\\
78.18	0.01\\
78.19	0.01\\
78.2	0.01\\
78.21	0.01\\
78.22	0.01\\
78.23	0.01\\
78.24	0.01\\
78.25	0.01\\
78.26	0.01\\
78.27	0.01\\
78.28	0.01\\
78.29	0.01\\
78.3	0.01\\
78.31	0.01\\
78.32	0.01\\
78.33	0.01\\
78.34	0.01\\
78.35	0.01\\
78.36	0.01\\
78.37	0.01\\
78.38	0.01\\
78.39	0.01\\
78.4	0.01\\
78.41	0.01\\
78.42	0.01\\
78.43	0.01\\
78.44	0.01\\
78.45	0.01\\
78.46	0.01\\
78.47	0.01\\
78.48	0.01\\
78.49	0.01\\
78.5	0.01\\
78.51	0.01\\
78.52	0.01\\
78.53	0.01\\
78.54	0.01\\
78.55	0.01\\
78.56	0.01\\
78.57	0.01\\
78.58	0.01\\
78.59	0.01\\
78.6	0.01\\
78.61	0.01\\
78.62	0.01\\
78.63	0.01\\
78.64	0.01\\
78.65	0.01\\
78.66	0.01\\
78.67	0.01\\
78.68	0.01\\
78.69	0.01\\
78.7	0.01\\
78.71	0.01\\
78.72	0.01\\
78.73	0.01\\
78.74	0.01\\
78.75	0.01\\
78.76	0.01\\
78.77	0.01\\
78.78	0.01\\
78.79	0.01\\
78.8	0.01\\
78.81	0.01\\
78.82	0.01\\
78.83	0.01\\
78.84	0.01\\
78.85	0.01\\
78.86	0.01\\
78.87	0.01\\
78.88	0.01\\
78.89	0.01\\
78.9	0.01\\
78.91	0.01\\
78.92	0.01\\
78.93	0.01\\
78.94	0.01\\
78.95	0.01\\
78.96	0.01\\
78.97	0.01\\
78.98	0.01\\
78.99	0.01\\
79	0.01\\
79.01	0.01\\
79.02	0.01\\
79.03	0.01\\
79.04	0.01\\
79.05	0.01\\
79.06	0.01\\
79.07	0.01\\
79.08	0.01\\
79.09	0.01\\
79.1	0.01\\
79.11	0.01\\
79.12	0.01\\
79.13	0.01\\
79.14	0.01\\
79.15	0.01\\
79.16	0.01\\
79.17	0.01\\
79.18	0.01\\
79.19	0.01\\
79.2	0.01\\
79.21	0.01\\
79.22	0.01\\
79.23	0.01\\
79.24	0.01\\
79.25	0.01\\
79.26	0.01\\
79.27	0.01\\
79.28	0.01\\
79.29	0.01\\
79.3	0.01\\
79.31	0.01\\
79.32	0.01\\
79.33	0.01\\
79.34	0.01\\
79.35	0.01\\
79.36	0.01\\
79.37	0.01\\
79.38	0.01\\
79.39	0.01\\
79.4	0.01\\
79.41	0.01\\
79.42	0.01\\
79.43	0.01\\
79.44	0.01\\
79.45	0.01\\
79.46	0.01\\
79.47	0.01\\
79.48	0.01\\
79.49	0.01\\
79.5	0.01\\
79.51	0.01\\
79.52	0.01\\
79.53	0.01\\
79.54	0.01\\
79.55	0.01\\
79.56	0.01\\
79.57	0.01\\
79.58	0.01\\
79.59	0.01\\
79.6	0.01\\
79.61	0.01\\
79.62	0.01\\
79.63	0.01\\
79.64	0.01\\
79.65	0.01\\
79.66	0.01\\
79.67	0.01\\
79.68	0.01\\
79.69	0.01\\
79.7	0.01\\
79.71	0.01\\
79.72	0.01\\
79.73	0.01\\
79.74	0.01\\
79.75	0.01\\
79.76	0.01\\
79.77	0.01\\
79.78	0.01\\
79.79	0.01\\
79.8	0.01\\
79.81	0.01\\
79.82	0.01\\
79.83	0.01\\
79.84	0.01\\
79.85	0.01\\
79.86	0.01\\
79.87	0.01\\
79.88	0.01\\
79.89	0.01\\
79.9	0.01\\
79.91	0.01\\
79.92	0.01\\
79.93	0.01\\
79.94	0.01\\
79.95	0.01\\
79.96	0.01\\
79.97	0.01\\
79.98	0.01\\
79.99	0.01\\
80	0.01\\
80.01	0.01\\
};
\addplot [color=blue,dashed]
  table[row sep=crcr]{%
80.01	0.01\\
80.02	0.01\\
80.03	0.01\\
80.04	0.01\\
80.05	0.01\\
80.06	0.01\\
80.07	0.01\\
80.08	0.01\\
80.09	0.01\\
80.1	0.01\\
80.11	0.01\\
80.12	0.01\\
80.13	0.01\\
80.14	0.01\\
80.15	0.01\\
80.16	0.01\\
80.17	0.01\\
80.18	0.01\\
80.19	0.01\\
80.2	0.01\\
80.21	0.01\\
80.22	0.01\\
80.23	0.01\\
80.24	0.01\\
80.25	0.01\\
80.26	0.01\\
80.27	0.01\\
80.28	0.01\\
80.29	0.01\\
80.3	0.01\\
80.31	0.01\\
80.32	0.01\\
80.33	0.01\\
80.34	0.01\\
80.35	0.01\\
80.36	0.01\\
80.37	0.01\\
80.38	0.01\\
80.39	0.01\\
80.4	0.01\\
80.41	0.01\\
80.42	0.01\\
80.43	0.01\\
80.44	0.01\\
80.45	0.01\\
80.46	0.01\\
80.47	0.01\\
80.48	0.01\\
80.49	0.01\\
80.5	0.01\\
80.51	0.01\\
80.52	0.01\\
80.53	0.01\\
80.54	0.01\\
80.55	0.01\\
80.56	0.01\\
80.57	0.01\\
80.58	0.01\\
80.59	0.01\\
80.6	0.01\\
80.61	0.01\\
80.62	0.01\\
80.63	0.01\\
80.64	0.01\\
80.65	0.01\\
80.66	0.01\\
80.67	0.01\\
80.68	0.01\\
80.69	0.01\\
80.7	0.01\\
80.71	0.01\\
80.72	0.01\\
80.73	0.01\\
80.74	0.01\\
80.75	0.01\\
80.76	0.01\\
80.77	0.01\\
80.78	0.01\\
80.79	0.01\\
80.8	0.01\\
80.81	0.01\\
80.82	0.01\\
80.83	0.01\\
80.84	0.01\\
80.85	0.01\\
80.86	0.01\\
80.87	0.01\\
80.88	0.01\\
80.89	0.01\\
80.9	0.01\\
80.91	0.01\\
80.92	0.01\\
80.93	0.01\\
80.94	0.01\\
80.95	0.01\\
80.96	0.01\\
80.97	0.01\\
80.98	0.01\\
80.99	0.01\\
81	0.01\\
81.01	0.01\\
81.02	0.01\\
81.03	0.01\\
81.04	0.01\\
81.05	0.01\\
81.06	0.01\\
81.07	0.01\\
81.08	0.01\\
81.09	0.01\\
81.1	0.01\\
81.11	0.01\\
81.12	0.01\\
81.13	0.01\\
81.14	0.01\\
81.15	0.01\\
81.16	0.01\\
81.17	0.01\\
81.18	0.01\\
81.19	0.01\\
81.2	0.01\\
81.21	0.01\\
81.22	0.01\\
81.23	0.01\\
81.24	0.01\\
81.25	0.01\\
81.26	0.01\\
81.27	0.01\\
81.28	0.01\\
81.29	0.01\\
81.3	0.01\\
81.31	0.01\\
81.32	0.01\\
81.33	0.01\\
81.34	0.01\\
81.35	0.01\\
81.36	0.01\\
81.37	0.01\\
81.38	0.01\\
81.39	0.01\\
81.4	0.01\\
81.41	0.01\\
81.42	0.01\\
81.43	0.01\\
81.44	0.01\\
81.45	0.01\\
81.46	0.01\\
81.47	0.01\\
81.48	0.01\\
81.49	0.01\\
81.5	0.01\\
81.51	0.01\\
81.52	0.01\\
81.53	0.01\\
81.54	0.01\\
81.55	0.01\\
81.56	0.01\\
81.57	0.01\\
81.58	0.01\\
81.59	0.01\\
81.6	0.01\\
81.61	0.01\\
81.62	0.01\\
81.63	0.01\\
81.64	0.01\\
81.65	0.01\\
81.66	0.01\\
81.67	0.01\\
81.68	0.01\\
81.69	0.01\\
81.7	0.01\\
81.71	0.01\\
81.72	0.01\\
81.73	0.01\\
81.74	0.01\\
81.75	0.01\\
81.76	0.01\\
81.77	0.01\\
81.78	0.01\\
81.79	0.01\\
81.8	0.01\\
81.81	0.01\\
81.82	0.01\\
81.83	0.01\\
81.84	0.01\\
81.85	0.01\\
81.86	0.01\\
81.87	0.01\\
81.88	0.01\\
81.89	0.01\\
81.9	0.01\\
81.91	0.01\\
81.92	0.01\\
81.93	0.01\\
81.94	0.01\\
81.95	0.01\\
81.96	0.01\\
81.97	0.01\\
81.98	0.01\\
81.99	0.01\\
82	0.01\\
82.01	0.01\\
82.02	0.01\\
82.03	0.01\\
82.04	0.01\\
82.05	0.01\\
82.06	0.01\\
82.07	0.01\\
82.08	0.01\\
82.09	0.01\\
82.1	0.01\\
82.11	0.01\\
82.12	0.01\\
82.13	0.01\\
82.14	0.01\\
82.15	0.01\\
82.16	0.01\\
82.17	0.01\\
82.18	0.01\\
82.19	0.01\\
82.2	0.01\\
82.21	0.01\\
82.22	0.01\\
82.23	0.01\\
82.24	0.01\\
82.25	0.01\\
82.26	0.01\\
82.27	0.01\\
82.28	0.01\\
82.29	0.01\\
82.3	0.01\\
82.31	0.01\\
82.32	0.01\\
82.33	0.01\\
82.34	0.01\\
82.35	0.01\\
82.36	0.01\\
82.37	0.01\\
82.38	0.01\\
82.39	0.01\\
82.4	0.01\\
82.41	0.01\\
82.42	0.01\\
82.43	0.01\\
82.44	0.01\\
82.45	0.01\\
82.46	0.01\\
82.47	0.01\\
82.48	0.01\\
82.49	0.01\\
82.5	0.01\\
82.51	0.01\\
82.52	0.01\\
82.53	0.01\\
82.54	0.01\\
82.55	0.01\\
82.56	0.01\\
82.57	0.01\\
82.58	0.01\\
82.59	0.01\\
82.6	0.01\\
82.61	0.01\\
82.62	0.01\\
82.63	0.01\\
82.64	0.01\\
82.65	0.01\\
82.66	0.01\\
82.67	0.01\\
82.68	0.01\\
82.69	0.01\\
82.7	0.01\\
82.71	0.01\\
82.72	0.01\\
82.73	0.01\\
82.74	0.01\\
82.75	0.01\\
82.76	0.01\\
82.77	0.01\\
82.78	0.01\\
82.79	0.01\\
82.8	0.01\\
82.81	0.01\\
82.82	0.01\\
82.83	0.01\\
82.84	0.01\\
82.85	0.01\\
82.86	0.01\\
82.87	0.01\\
82.88	0.01\\
82.89	0.01\\
82.9	0.01\\
82.91	0.01\\
82.92	0.01\\
82.93	0.01\\
82.94	0.01\\
82.95	0.01\\
82.96	0.01\\
82.97	0.01\\
82.98	0.01\\
82.99	0.01\\
83	0.01\\
83.01	0.01\\
83.02	0.01\\
83.03	0.01\\
83.04	0.01\\
83.05	0.01\\
83.06	0.01\\
83.07	0.01\\
83.08	0.01\\
83.09	0.01\\
83.1	0.01\\
83.11	0.01\\
83.12	0.01\\
83.13	0.01\\
83.14	0.01\\
83.15	0.01\\
83.16	0.01\\
83.17	0.01\\
83.18	0.01\\
83.19	0.01\\
83.2	0.01\\
83.21	0.01\\
83.22	0.01\\
83.23	0.01\\
83.24	0.01\\
83.25	0.01\\
83.26	0.01\\
83.27	0.01\\
83.28	0.01\\
83.29	0.01\\
83.3	0.01\\
83.31	0.01\\
83.32	0.01\\
83.33	0.01\\
83.34	0.01\\
83.35	0.01\\
83.36	0.01\\
83.37	0.01\\
83.38	0.01\\
83.39	0.01\\
83.4	0.01\\
83.41	0.01\\
83.42	0.01\\
83.43	0.01\\
83.44	0.01\\
83.45	0.01\\
83.46	0.01\\
83.47	0.01\\
83.48	0.01\\
83.49	0.01\\
83.5	0.01\\
83.51	0.01\\
83.52	0.01\\
83.53	0.01\\
83.54	0.01\\
83.55	0.01\\
83.56	0.01\\
83.57	0.01\\
83.58	0.01\\
83.59	0.01\\
83.6	0.01\\
83.61	0.01\\
83.62	0.01\\
83.63	0.01\\
83.64	0.01\\
83.65	0.01\\
83.66	0.01\\
83.67	0.01\\
83.68	0.01\\
83.69	0.01\\
83.7	0.01\\
83.71	0.01\\
83.72	0.01\\
83.73	0.01\\
83.74	0.01\\
83.75	0.01\\
83.76	0.01\\
83.77	0.01\\
83.78	0.01\\
83.79	0.01\\
83.8	0.01\\
83.81	0.01\\
83.82	0.01\\
83.83	0.01\\
83.84	0.01\\
83.85	0.01\\
83.86	0.01\\
83.87	0.01\\
83.88	0.01\\
83.89	0.01\\
83.9	0.01\\
83.91	0.01\\
83.92	0.01\\
83.93	0.01\\
83.94	0.01\\
83.95	0.01\\
83.96	0.01\\
83.97	0.01\\
83.98	0.01\\
83.99	0.01\\
84	0.01\\
84.01	0.01\\
84.02	0.01\\
84.03	0.01\\
84.04	0.01\\
84.05	0.01\\
84.06	0.01\\
84.07	0.01\\
84.08	0.01\\
84.09	0.01\\
84.1	0.01\\
84.11	0.01\\
84.12	0.01\\
84.13	0.01\\
84.14	0.01\\
84.15	0.01\\
84.16	0.01\\
84.17	0.01\\
84.18	0.01\\
84.19	0.01\\
84.2	0.01\\
84.21	0.01\\
84.22	0.01\\
84.23	0.01\\
84.24	0.01\\
84.25	0.01\\
84.26	0.01\\
84.27	0.01\\
84.28	0.01\\
84.29	0.01\\
84.3	0.01\\
84.31	0.01\\
84.32	0.01\\
84.33	0.01\\
84.34	0.01\\
84.35	0.01\\
84.36	0.01\\
84.37	0.01\\
84.38	0.01\\
84.39	0.01\\
84.4	0.01\\
84.41	0.01\\
84.42	0.01\\
84.43	0.01\\
84.44	0.01\\
84.45	0.01\\
84.46	0.01\\
84.47	0.01\\
84.48	0.01\\
84.49	0.01\\
84.5	0.01\\
84.51	0.01\\
84.52	0.01\\
84.53	0.01\\
84.54	0.01\\
84.55	0.01\\
84.56	0.01\\
84.57	0.01\\
84.58	0.01\\
84.59	0.01\\
84.6	0.01\\
84.61	0.01\\
84.62	0.01\\
84.63	0.01\\
84.64	0.01\\
84.65	0.01\\
84.66	0.01\\
84.67	0.01\\
84.68	0.01\\
84.69	0.01\\
84.7	0.01\\
84.71	0.01\\
84.72	0.01\\
84.73	0.01\\
84.74	0.01\\
84.75	0.01\\
84.76	0.01\\
84.77	0.01\\
84.78	0.01\\
84.79	0.01\\
84.8	0.01\\
84.81	0.01\\
84.82	0.01\\
84.83	0.01\\
84.84	0.01\\
84.85	0.01\\
84.86	0.01\\
84.87	0.01\\
84.88	0.01\\
84.89	0.01\\
84.9	0.01\\
84.91	0.01\\
84.92	0.01\\
84.93	0.01\\
84.94	0.01\\
84.95	0.01\\
84.96	0.01\\
84.97	0.01\\
84.98	0.01\\
84.99	0.01\\
85	0.01\\
85.01	0.01\\
85.02	0.01\\
85.03	0.01\\
85.04	0.01\\
85.05	0.01\\
85.06	0.01\\
85.07	0.01\\
85.08	0.01\\
85.09	0.01\\
85.1	0.01\\
85.11	0.01\\
85.12	0.01\\
85.13	0.01\\
85.14	0.01\\
85.15	0.01\\
85.16	0.01\\
85.17	0.01\\
85.18	0.01\\
85.19	0.01\\
85.2	0.01\\
85.21	0.01\\
85.22	0.01\\
85.23	0.01\\
85.24	0.01\\
85.25	0.01\\
85.26	0.01\\
85.27	0.01\\
85.28	0.01\\
85.29	0.01\\
85.3	0.01\\
85.31	0.01\\
85.32	0.01\\
85.33	0.01\\
85.34	0.01\\
85.35	0.01\\
85.36	0.01\\
85.37	0.01\\
85.38	0.01\\
85.39	0.01\\
85.4	0.01\\
85.41	0.01\\
85.42	0.01\\
85.43	0.01\\
85.44	0.01\\
85.45	0.01\\
85.46	0.01\\
85.47	0.01\\
85.48	0.01\\
85.49	0.01\\
85.5	0.01\\
85.51	0.01\\
85.52	0.01\\
85.53	0.01\\
85.54	0.01\\
85.55	0.01\\
85.56	0.01\\
85.57	0.01\\
85.58	0.01\\
85.59	0.01\\
85.6	0.01\\
85.61	0.01\\
85.62	0.01\\
85.63	0.01\\
85.64	0.01\\
85.65	0.01\\
85.66	0.01\\
85.67	0.01\\
85.68	0.01\\
85.69	0.01\\
85.7	0.01\\
85.71	0.01\\
85.72	0.01\\
85.73	0.01\\
85.74	0.01\\
85.75	0.01\\
85.76	0.01\\
85.77	0.01\\
85.78	0.01\\
85.79	0.01\\
85.8	0.01\\
85.81	0.01\\
85.82	0.01\\
85.83	0.01\\
85.84	0.01\\
85.85	0.01\\
85.86	0.01\\
85.87	0.01\\
85.88	0.01\\
85.89	0.01\\
85.9	0.01\\
85.91	0.01\\
85.92	0.01\\
85.93	0.01\\
85.94	0.01\\
85.95	0.01\\
85.96	0.01\\
85.97	0.01\\
85.98	0.01\\
85.99	0.01\\
86	0.01\\
86.01	0.01\\
86.02	0.01\\
86.03	0.01\\
86.04	0.01\\
86.05	0.01\\
86.06	0.01\\
86.07	0.01\\
86.08	0.01\\
86.09	0.01\\
86.1	0.01\\
86.11	0.01\\
86.12	0.01\\
86.13	0.01\\
86.14	0.01\\
86.15	0.01\\
86.16	0.01\\
86.17	0.01\\
86.18	0.01\\
86.19	0.01\\
86.2	0.01\\
86.21	0.01\\
86.22	0.01\\
86.23	0.01\\
86.24	0.01\\
86.25	0.01\\
86.26	0.01\\
86.27	0.01\\
86.28	0.01\\
86.29	0.01\\
86.3	0.01\\
86.31	0.01\\
86.32	0.01\\
86.33	0.01\\
86.34	0.01\\
86.35	0.01\\
86.36	0.01\\
86.37	0.01\\
86.38	0.01\\
86.39	0.01\\
86.4	0.01\\
86.41	0.01\\
86.42	0.01\\
86.43	0.01\\
86.44	0.01\\
86.45	0.01\\
86.46	0.01\\
86.47	0.01\\
86.48	0.01\\
86.49	0.01\\
86.5	0.01\\
86.51	0.01\\
86.52	0.01\\
86.53	0.01\\
86.54	0.01\\
86.55	0.01\\
86.56	0.01\\
86.57	0.01\\
86.58	0.01\\
86.59	0.01\\
86.6	0.01\\
86.61	0.01\\
86.62	0.01\\
86.63	0.01\\
86.64	0.01\\
86.65	0.01\\
86.66	0.01\\
86.67	0.01\\
86.68	0.01\\
86.69	0.01\\
86.7	0.01\\
86.71	0.01\\
86.72	0.01\\
86.73	0.01\\
86.74	0.01\\
86.75	0.01\\
86.76	0.01\\
86.77	0.01\\
86.78	0.01\\
86.79	0.01\\
86.8	0.01\\
86.81	0.01\\
86.82	0.01\\
86.83	0.01\\
86.84	0.01\\
86.85	0.01\\
86.86	0.01\\
86.87	0.01\\
86.88	0.01\\
86.89	0.01\\
86.9	0.01\\
86.91	0.01\\
86.92	0.01\\
86.93	0.01\\
86.94	0.01\\
86.95	0.01\\
86.96	0.01\\
86.97	0.01\\
86.98	0.01\\
86.99	0.01\\
87	0.01\\
87.01	0.01\\
87.02	0.01\\
87.03	0.01\\
87.04	0.01\\
87.05	0.01\\
87.06	0.01\\
87.07	0.01\\
87.08	0.01\\
87.09	0.01\\
87.1	0.01\\
87.11	0.01\\
87.12	0.01\\
87.13	0.01\\
87.14	0.01\\
87.15	0.01\\
87.16	0.01\\
87.17	0.01\\
87.18	0.01\\
87.19	0.01\\
87.2	0.01\\
87.21	0.01\\
87.22	0.01\\
87.23	0.01\\
87.24	0.01\\
87.25	0.01\\
87.26	0.01\\
87.27	0.01\\
87.28	0.01\\
87.29	0.01\\
87.3	0.01\\
87.31	0.01\\
87.32	0.01\\
87.33	0.01\\
87.34	0.01\\
87.35	0.01\\
87.36	0.01\\
87.37	0.01\\
87.38	0.01\\
87.39	0.01\\
87.4	0.01\\
87.41	0.01\\
87.42	0.01\\
87.43	0.01\\
87.44	0.01\\
87.45	0.01\\
87.46	0.01\\
87.47	0.01\\
87.48	0.01\\
87.49	0.01\\
87.5	0.01\\
87.51	0.01\\
87.52	0.01\\
87.53	0.01\\
87.54	0.01\\
87.55	0.01\\
87.56	0.01\\
87.57	0.01\\
87.58	0.01\\
87.59	0.01\\
87.6	0.01\\
87.61	0.01\\
87.62	0.01\\
87.63	0.01\\
87.64	0.01\\
87.65	0.01\\
87.66	0.01\\
87.67	0.01\\
87.68	0.01\\
87.69	0.01\\
87.7	0.01\\
87.71	0.01\\
87.72	0.01\\
87.73	0.01\\
87.74	0.01\\
87.75	0.01\\
87.76	0.01\\
87.77	0.01\\
87.78	0.01\\
87.79	0.01\\
87.8	0.01\\
87.81	0.01\\
87.82	0.01\\
87.83	0.01\\
87.84	0.01\\
87.85	0.01\\
87.86	0.01\\
87.87	0.01\\
87.88	0.01\\
87.89	0.01\\
87.9	0.01\\
87.91	0.01\\
87.92	0.01\\
87.93	0.01\\
87.94	0.01\\
87.95	0.01\\
87.96	0.01\\
87.97	0.01\\
87.98	0.01\\
87.99	0.01\\
88	0.01\\
88.01	0.01\\
88.02	0.01\\
88.03	0.01\\
88.04	0.01\\
88.05	0.01\\
88.06	0.01\\
88.07	0.01\\
88.08	0.01\\
88.09	0.01\\
88.1	0.01\\
88.11	0.01\\
88.12	0.01\\
88.13	0.01\\
88.14	0.01\\
88.15	0.01\\
88.16	0.01\\
88.17	0.01\\
88.18	0.01\\
88.19	0.01\\
88.2	0.01\\
88.21	0.01\\
88.22	0.01\\
88.23	0.01\\
88.24	0.01\\
88.25	0.01\\
88.26	0.01\\
88.27	0.01\\
88.28	0.01\\
88.29	0.01\\
88.3	0.01\\
88.31	0.01\\
88.32	0.01\\
88.33	0.01\\
88.34	0.01\\
88.35	0.01\\
88.36	0.01\\
88.37	0.01\\
88.38	0.01\\
88.39	0.01\\
88.4	0.01\\
88.41	0.01\\
88.42	0.01\\
88.43	0.01\\
88.44	0.01\\
88.45	0.01\\
88.46	0.01\\
88.47	0.01\\
88.48	0.01\\
88.49	0.01\\
88.5	0.01\\
88.51	0.01\\
88.52	0.01\\
88.53	0.01\\
88.54	0.01\\
88.55	0.01\\
88.56	0.01\\
88.57	0.01\\
88.58	0.01\\
88.59	0.01\\
88.6	0.01\\
88.61	0.01\\
88.62	0.01\\
88.63	0.01\\
88.64	0.01\\
88.65	0.01\\
88.66	0.01\\
88.67	0.01\\
88.68	0.01\\
88.69	0.01\\
88.7	0.01\\
88.71	0.01\\
88.72	0.01\\
88.73	0.01\\
88.74	0.01\\
88.75	0.01\\
88.76	0.01\\
88.77	0.01\\
88.78	0.01\\
88.79	0.01\\
88.8	0.01\\
88.81	0.01\\
88.82	0.01\\
88.83	0.01\\
88.84	0.01\\
88.85	0.01\\
88.86	0.01\\
88.87	0.01\\
88.88	0.01\\
88.89	0.01\\
88.9	0.01\\
88.91	0.01\\
88.92	0.01\\
88.93	0.01\\
88.94	0.01\\
88.95	0.01\\
88.96	0.01\\
88.97	0.01\\
88.98	0.01\\
88.99	0.01\\
89	0.01\\
89.01	0.01\\
89.02	0.01\\
89.03	0.01\\
89.04	0.01\\
89.05	0.01\\
89.06	0.01\\
89.07	0.01\\
89.08	0.01\\
89.09	0.01\\
89.1	0.01\\
89.11	0.01\\
89.12	0.01\\
89.13	0.01\\
89.14	0.01\\
89.15	0.01\\
89.16	0.01\\
89.17	0.01\\
89.18	0.01\\
89.19	0.01\\
89.2	0.01\\
89.21	0.01\\
89.22	0.01\\
89.23	0.01\\
89.24	0.01\\
89.25	0.01\\
89.26	0.01\\
89.27	0.01\\
89.28	0.01\\
89.29	0.01\\
89.3	0.01\\
89.31	0.01\\
89.32	0.01\\
89.33	0.01\\
89.34	0.01\\
89.35	0.01\\
89.36	0.01\\
89.37	0.01\\
89.38	0.01\\
89.39	0.01\\
89.4	0.01\\
89.41	0.01\\
89.42	0.01\\
89.43	0.01\\
89.44	0.01\\
89.45	0.01\\
89.46	0.01\\
89.47	0.01\\
89.48	0.01\\
89.49	0.01\\
89.5	0.01\\
89.51	0.01\\
89.52	0.01\\
89.53	0.01\\
89.54	0.01\\
89.55	0.01\\
89.56	0.01\\
89.57	0.01\\
89.58	0.01\\
89.59	0.01\\
89.6	0.01\\
89.61	0.01\\
89.62	0.01\\
89.63	0.01\\
89.64	0.01\\
89.65	0.01\\
89.66	0.01\\
89.67	0.01\\
89.68	0.01\\
89.69	0.01\\
89.7	0.01\\
89.71	0.01\\
89.72	0.01\\
89.73	0.01\\
89.74	0.01\\
89.75	0.01\\
89.76	0.01\\
89.77	0.01\\
89.78	0.01\\
89.79	0.01\\
89.8	0.01\\
89.81	0.01\\
89.82	0.01\\
89.83	0.01\\
89.84	0.01\\
89.85	0.01\\
89.86	0.01\\
89.87	0.01\\
89.88	0.01\\
89.89	0.01\\
89.9	0.01\\
89.91	0.01\\
89.92	0.01\\
89.93	0.01\\
89.94	0.01\\
89.95	0.01\\
89.96	0.01\\
89.97	0.01\\
89.98	0.01\\
89.99	0.01\\
90	0.01\\
90.01	0.01\\
90.02	0.01\\
90.03	0.01\\
90.04	0.01\\
90.05	0.01\\
90.06	0.01\\
90.07	0.01\\
90.08	0.01\\
90.09	0.01\\
90.1	0.01\\
90.11	0.01\\
90.12	0.01\\
90.13	0.01\\
90.14	0.01\\
90.15	0.01\\
90.16	0.01\\
90.17	0.01\\
90.18	0.01\\
90.19	0.01\\
90.2	0.01\\
90.21	0.01\\
90.22	0.01\\
90.23	0.01\\
90.24	0.01\\
90.25	0.01\\
90.26	0.01\\
90.27	0.01\\
90.28	0.01\\
90.29	0.01\\
90.3	0.01\\
90.31	0.01\\
90.32	0.01\\
90.33	0.01\\
90.34	0.01\\
90.35	0.01\\
90.36	0.01\\
90.37	0.01\\
90.38	0.01\\
90.39	0.01\\
90.4	0.01\\
90.41	0.01\\
90.42	0.01\\
90.43	0.01\\
90.44	0.01\\
90.45	0.01\\
90.46	0.01\\
90.47	0.01\\
90.48	0.01\\
90.49	0.01\\
90.5	0.01\\
90.51	0.01\\
90.52	0.01\\
90.53	0.01\\
90.54	0.01\\
90.55	0.01\\
90.56	0.01\\
90.57	0.01\\
90.58	0.01\\
90.59	0.01\\
90.6	0.01\\
90.61	0.01\\
90.62	0.01\\
90.63	0.01\\
90.64	0.01\\
90.65	0.01\\
90.66	0.01\\
90.67	0.01\\
90.68	0.01\\
90.69	0.01\\
90.7	0.01\\
90.71	0.01\\
90.72	0.01\\
90.73	0.01\\
90.74	0.01\\
90.75	0.01\\
90.76	0.01\\
90.77	0.01\\
90.78	0.01\\
90.79	0.01\\
90.8	0.01\\
90.81	0.01\\
90.82	0.01\\
90.83	0.01\\
90.84	0.01\\
90.85	0.01\\
90.86	0.01\\
90.87	0.01\\
90.88	0.01\\
90.89	0.01\\
90.9	0.01\\
90.91	0.01\\
90.92	0.01\\
90.93	0.01\\
90.94	0.01\\
90.95	0.01\\
90.96	0.01\\
90.97	0.01\\
90.98	0.01\\
90.99	0.01\\
91	0.01\\
91.01	0.01\\
91.02	0.01\\
91.03	0.01\\
91.04	0.01\\
91.05	0.01\\
91.06	0.01\\
91.07	0.01\\
91.08	0.01\\
91.09	0.01\\
91.1	0.01\\
91.11	0.01\\
91.12	0.01\\
91.13	0.01\\
91.14	0.01\\
91.15	0.01\\
91.16	0.01\\
91.17	0.01\\
91.18	0.01\\
91.19	0.01\\
91.2	0.01\\
91.21	0.01\\
91.22	0.01\\
91.23	0.01\\
91.24	0.01\\
91.25	0.01\\
91.26	0.01\\
91.27	0.01\\
91.28	0.01\\
91.29	0.01\\
91.3	0.01\\
91.31	0.01\\
91.32	0.01\\
91.33	0.01\\
91.34	0.01\\
91.35	0.01\\
91.36	0.01\\
91.37	0.01\\
91.38	0.01\\
91.39	0.01\\
91.4	0.01\\
91.41	0.01\\
91.42	0.01\\
91.43	0.01\\
91.44	0.01\\
91.45	0.01\\
91.46	0.01\\
91.47	0.01\\
91.48	0.01\\
91.49	0.01\\
91.5	0.01\\
91.51	0.01\\
91.52	0.01\\
91.53	0.01\\
91.54	0.01\\
91.55	0.01\\
91.56	0.01\\
91.57	0.01\\
91.58	0.01\\
91.59	0.01\\
91.6	0.01\\
91.61	0.01\\
91.62	0.01\\
91.63	0.01\\
91.64	0.01\\
91.65	0.01\\
91.66	0.01\\
91.67	0.01\\
91.68	0.01\\
91.69	0.01\\
91.7	0.01\\
91.71	0.01\\
91.72	0.01\\
91.73	0.01\\
91.74	0.01\\
91.75	0.01\\
91.76	0.01\\
91.77	0.01\\
91.78	0.01\\
91.79	0.01\\
91.8	0.01\\
91.81	0.01\\
91.82	0.01\\
91.83	0.01\\
91.84	0.01\\
91.85	0.01\\
91.86	0.01\\
91.87	0.01\\
91.88	0.01\\
91.89	0.01\\
91.9	0.01\\
91.91	0.01\\
91.92	0.01\\
91.93	0.01\\
91.94	0.01\\
91.95	0.01\\
91.96	0.01\\
91.97	0.01\\
91.98	0.01\\
91.99	0.01\\
92	0.01\\
92.01	0.01\\
92.02	0.01\\
92.03	0.01\\
92.04	0.01\\
92.05	0.01\\
92.06	0.01\\
92.07	0.01\\
92.08	0.01\\
92.09	0.01\\
92.1	0.01\\
92.11	0.01\\
92.12	0.01\\
92.13	0.01\\
92.14	0.01\\
92.15	0.01\\
92.16	0.01\\
92.17	0.01\\
92.18	0.01\\
92.19	0.01\\
92.2	0.01\\
92.21	0.01\\
92.22	0.01\\
92.23	0.01\\
92.24	0.01\\
92.25	0.01\\
92.26	0.01\\
92.27	0.01\\
92.28	0.01\\
92.29	0.01\\
92.3	0.01\\
92.31	0.01\\
92.32	0.01\\
92.33	0.01\\
92.34	0.01\\
92.35	0.01\\
92.36	0.01\\
92.37	0.01\\
92.38	0.01\\
92.39	0.01\\
92.4	0.01\\
92.41	0.01\\
92.42	0.01\\
92.43	0.01\\
92.44	0.01\\
92.45	0.01\\
92.46	0.01\\
92.47	0.01\\
92.48	0.01\\
92.49	0.01\\
92.5	0.01\\
92.51	0.01\\
92.52	0.01\\
92.53	0.01\\
92.54	0.01\\
92.55	0.01\\
92.56	0.01\\
92.57	0.01\\
92.58	0.01\\
92.59	0.01\\
92.6	0.01\\
92.61	0.01\\
92.62	0.01\\
92.63	0.01\\
92.64	0.01\\
92.65	0.01\\
92.66	0.01\\
92.67	0.01\\
92.68	0.01\\
92.69	0.01\\
92.7	0.01\\
92.71	0.01\\
92.72	0.01\\
92.73	0.01\\
92.74	0.01\\
92.75	0.01\\
92.76	0.01\\
92.77	0.01\\
92.78	0.01\\
92.79	0.01\\
92.8	0.01\\
92.81	0.01\\
92.82	0.01\\
92.83	0.01\\
92.84	0.01\\
92.85	0.01\\
92.86	0.01\\
92.87	0.01\\
92.88	0.01\\
92.89	0.01\\
92.9	0.01\\
92.91	0.01\\
92.92	0.01\\
92.93	0.01\\
92.94	0.01\\
92.95	0.01\\
92.96	0.01\\
92.97	0.01\\
92.98	0.01\\
92.99	0.01\\
93	0.01\\
93.01	0.01\\
93.02	0.01\\
93.03	0.01\\
93.04	0.01\\
93.05	0.01\\
93.06	0.01\\
93.07	0.01\\
93.08	0.01\\
93.09	0.01\\
93.1	0.01\\
93.11	0.01\\
93.12	0.01\\
93.13	0.01\\
93.14	0.01\\
93.15	0.01\\
93.16	0.01\\
93.17	0.01\\
93.18	0.01\\
93.19	0.01\\
93.2	0.01\\
93.21	0.01\\
93.22	0.01\\
93.23	0.01\\
93.24	0.01\\
93.25	0.01\\
93.26	0.01\\
93.27	0.01\\
93.28	0.01\\
93.29	0.01\\
93.3	0.01\\
93.31	0.01\\
93.32	0.01\\
93.33	0.01\\
93.34	0.01\\
93.35	0.01\\
93.36	0.01\\
93.37	0.01\\
93.38	0.01\\
93.39	0.01\\
93.4	0.01\\
93.41	0.01\\
93.42	0.01\\
93.43	0.01\\
93.44	0.01\\
93.45	0.01\\
93.46	0.01\\
93.47	0.01\\
93.48	0.01\\
93.49	0.01\\
93.5	0.01\\
93.51	0.01\\
93.52	0.01\\
93.53	0.01\\
93.54	0.01\\
93.55	0.01\\
93.56	0.01\\
93.57	0.01\\
93.58	0.01\\
93.59	0.01\\
93.6	0.01\\
93.61	0.01\\
93.62	0.01\\
93.63	0.01\\
93.64	0.01\\
93.65	0.01\\
93.66	0.01\\
93.67	0.01\\
93.68	0.01\\
93.69	0.01\\
93.7	0.01\\
93.71	0.01\\
93.72	0.01\\
93.73	0.01\\
93.74	0.01\\
93.75	0.01\\
93.76	0.01\\
93.77	0.01\\
93.78	0.01\\
93.79	0.01\\
93.8	0.01\\
93.81	0.01\\
93.82	0.01\\
93.83	0.01\\
93.84	0.01\\
93.85	0.01\\
93.86	0.01\\
93.87	0.01\\
93.88	0.01\\
93.89	0.01\\
93.9	0.01\\
93.91	0.01\\
93.92	0.01\\
93.93	0.01\\
93.94	0.01\\
93.95	0.01\\
93.96	0.01\\
93.97	0.01\\
93.98	0.01\\
93.99	0.01\\
94	0.01\\
94.01	0.01\\
94.02	0.01\\
94.03	0.01\\
94.04	0.01\\
94.05	0.01\\
94.06	0.01\\
94.07	0.01\\
94.08	0.01\\
94.09	0.01\\
94.1	0.01\\
94.11	0.01\\
94.12	0.01\\
94.13	0.01\\
94.14	0.01\\
94.15	0.01\\
94.16	0.01\\
94.17	0.01\\
94.18	0.01\\
94.19	0.01\\
94.2	0.01\\
94.21	0.01\\
94.22	0.01\\
94.23	0.01\\
94.24	0.01\\
94.25	0.01\\
94.26	0.01\\
94.27	0.01\\
94.28	0.01\\
94.29	0.01\\
94.3	0.01\\
94.31	0.01\\
94.32	0.01\\
94.33	0.01\\
94.34	0.01\\
94.35	0.01\\
94.36	0.01\\
94.37	0.01\\
94.38	0.01\\
94.39	0.01\\
94.4	0.01\\
94.41	0.01\\
94.42	0.01\\
94.43	0.01\\
94.44	0.01\\
94.45	0.01\\
94.46	0.01\\
94.47	0.01\\
94.48	0.01\\
94.49	0.01\\
94.5	0.01\\
94.51	0.01\\
94.52	0.01\\
94.53	0.01\\
94.54	0.01\\
94.55	0.01\\
94.56	0.01\\
94.57	0.01\\
94.58	0.01\\
94.59	0.01\\
94.6	0.01\\
94.61	0.01\\
94.62	0.01\\
94.63	0.01\\
94.64	0.01\\
94.65	0.01\\
94.66	0.01\\
94.67	0.01\\
94.68	0.01\\
94.69	0.01\\
94.7	0.01\\
94.71	0.01\\
94.72	0.01\\
94.73	0.01\\
94.74	0.01\\
94.75	0.01\\
94.76	0.01\\
94.77	0.01\\
94.78	0.01\\
94.79	0.01\\
94.8	0.01\\
94.81	0.01\\
94.82	0.01\\
94.83	0.01\\
94.84	0.01\\
94.85	0.01\\
94.86	0.01\\
94.87	0.01\\
94.88	0.01\\
94.89	0.01\\
94.9	0.01\\
94.91	0.01\\
94.92	0.01\\
94.93	0.01\\
94.94	0.01\\
94.95	0.01\\
94.96	0.01\\
94.97	0.01\\
94.98	0.01\\
94.99	0.01\\
95	0.01\\
95.01	0.01\\
95.02	0.01\\
95.03	0.01\\
95.04	0.01\\
95.05	0.01\\
95.06	0.01\\
95.07	0.01\\
95.08	0.01\\
95.09	0.01\\
95.1	0.01\\
95.11	0.01\\
95.12	0.01\\
95.13	0.01\\
95.14	0.01\\
95.15	0.01\\
95.16	0.01\\
95.17	0.01\\
95.18	0.01\\
95.19	0.01\\
95.2	0.01\\
95.21	0.01\\
95.22	0.01\\
95.23	0.01\\
95.24	0.01\\
95.25	0.01\\
95.26	0.01\\
95.27	0.01\\
95.28	0.01\\
95.29	0.01\\
95.3	0.01\\
95.31	0.01\\
95.32	0.01\\
95.33	0.01\\
95.34	0.01\\
95.35	0.01\\
95.36	0.01\\
95.37	0.01\\
95.38	0.01\\
95.39	0.01\\
95.4	0.01\\
95.41	0.01\\
95.42	0.01\\
95.43	0.01\\
95.44	0.01\\
95.45	0.01\\
95.46	0.01\\
95.47	0.01\\
95.48	0.01\\
95.49	0.01\\
95.5	0.01\\
95.51	0.01\\
95.52	0.01\\
95.53	0.01\\
95.54	0.01\\
95.55	0.01\\
95.56	0.01\\
95.57	0.01\\
95.58	0.01\\
95.59	0.01\\
95.6	0.01\\
95.61	0.01\\
95.62	0.01\\
95.63	0.01\\
95.64	0.01\\
95.65	0.01\\
95.66	0.01\\
95.67	0.01\\
95.68	0.01\\
95.69	0.01\\
95.7	0.01\\
95.71	0.01\\
95.72	0.01\\
95.73	0.01\\
95.74	0.01\\
95.75	0.01\\
95.76	0.01\\
95.77	0.01\\
95.78	0.01\\
95.79	0.01\\
95.8	0.01\\
95.81	0.01\\
95.82	0.01\\
95.83	0.01\\
95.84	0.01\\
95.85	0.01\\
95.86	0.01\\
95.87	0.01\\
95.88	0.01\\
95.89	0.01\\
95.9	0.01\\
95.91	0.01\\
95.92	0.01\\
95.93	0.01\\
95.94	0.01\\
95.95	0.01\\
95.96	0.01\\
95.97	0.01\\
95.98	0.01\\
95.99	0.01\\
96	0.01\\
96.01	0.01\\
96.02	0.01\\
96.03	0.01\\
96.04	0.01\\
96.05	0.01\\
96.06	0.01\\
96.07	0.01\\
96.08	0.01\\
96.09	0.01\\
96.1	0.01\\
96.11	0.01\\
96.12	0.01\\
96.13	0.01\\
96.14	0.01\\
96.15	0.01\\
96.16	0.01\\
96.17	0.01\\
96.18	0.01\\
96.19	0.01\\
96.2	0.01\\
96.21	0.01\\
96.22	0.01\\
96.23	0.01\\
96.24	0.01\\
96.25	0.01\\
96.26	0.01\\
96.27	0.01\\
96.28	0.01\\
96.29	0.01\\
96.3	0.01\\
96.31	0.01\\
96.32	0.01\\
96.33	0.01\\
96.34	0.01\\
96.35	0.01\\
96.36	0.01\\
96.37	0.01\\
96.38	0.01\\
96.39	0.01\\
96.4	0.01\\
96.41	0.01\\
96.42	0.01\\
96.43	0.01\\
96.44	0.01\\
96.45	0.01\\
96.46	0.01\\
96.47	0.01\\
96.48	0.01\\
96.49	0.01\\
96.5	0.01\\
96.51	0.01\\
96.52	0.01\\
96.53	0.01\\
96.54	0.01\\
96.55	0.01\\
96.56	0.01\\
96.57	0.01\\
96.58	0.01\\
96.59	0.01\\
96.6	0.01\\
96.61	0.01\\
96.62	0.01\\
96.63	0.01\\
96.64	0.01\\
96.65	0.01\\
96.66	0.01\\
96.67	0.01\\
96.68	0.01\\
96.69	0.01\\
96.7	0.01\\
96.71	0.01\\
96.72	0.01\\
96.73	0.01\\
96.74	0.01\\
96.75	0.01\\
96.76	0.01\\
96.77	0.01\\
96.78	0.01\\
96.79	0.01\\
96.8	0.01\\
96.81	0.01\\
96.82	0.01\\
96.83	0.01\\
96.84	0.01\\
96.85	0.01\\
96.86	0.01\\
96.87	0.01\\
96.88	0.01\\
96.89	0.01\\
96.9	0.01\\
96.91	0.01\\
96.92	0.01\\
96.93	0.01\\
96.94	0.01\\
96.95	0.01\\
96.96	0.01\\
96.97	0.01\\
96.98	0.01\\
96.99	0.01\\
97	0.01\\
97.01	0.01\\
97.02	0.01\\
97.03	0.01\\
97.04	0.01\\
97.05	0.01\\
97.06	0.01\\
97.07	0.01\\
97.08	0.01\\
97.09	0.01\\
97.1	0.01\\
97.11	0.01\\
97.12	0.01\\
97.13	0.01\\
97.14	0.01\\
97.15	0.01\\
97.16	0.01\\
97.17	0.01\\
97.18	0.01\\
97.19	0.01\\
97.2	0.01\\
97.21	0.01\\
97.22	0.01\\
97.23	0.01\\
97.24	0.01\\
97.25	0.01\\
97.26	0.01\\
97.27	0.01\\
97.28	0.01\\
97.29	0.01\\
97.3	0.01\\
97.31	0.01\\
97.32	0.01\\
97.33	0.01\\
97.34	0.01\\
97.35	0.01\\
97.36	0.01\\
97.37	0.01\\
97.38	0.01\\
97.39	0.01\\
97.4	0.01\\
97.41	0.01\\
97.42	0.01\\
97.43	0.01\\
97.44	0.01\\
97.45	0.01\\
97.46	0.01\\
97.47	0.01\\
97.48	0.01\\
97.49	0.01\\
97.5	0.01\\
97.51	0.01\\
97.52	0.01\\
97.53	0.01\\
97.54	0.01\\
97.55	0.01\\
97.56	0.01\\
97.57	0.01\\
97.58	0.01\\
97.59	0.01\\
97.6	0.01\\
97.61	0.01\\
97.62	0.01\\
97.63	0.01\\
97.64	0.01\\
97.65	0.01\\
97.66	0.01\\
97.67	0.01\\
97.68	0.01\\
97.69	0.01\\
97.7	0.01\\
97.71	0.01\\
97.72	0.01\\
97.73	0.01\\
97.74	0.01\\
97.75	0.01\\
97.76	0.01\\
97.77	0.01\\
97.78	0.01\\
97.79	0.01\\
97.8	0.01\\
97.81	0.01\\
97.82	0.01\\
97.83	0.01\\
97.84	0.01\\
97.85	0.01\\
97.86	0.01\\
97.87	0.01\\
97.88	0.01\\
97.89	0.01\\
97.9	0.01\\
97.91	0.01\\
97.92	0.01\\
97.93	0.01\\
97.94	0.01\\
97.95	0.01\\
97.96	0.01\\
97.97	0.01\\
97.98	0.01\\
97.99	0.01\\
98	0.01\\
98.01	0.01\\
98.02	0.01\\
98.03	0.01\\
98.04	0.01\\
98.05	0.01\\
98.06	0.01\\
98.07	0.01\\
98.08	0.01\\
98.09	0.01\\
98.1	0.01\\
98.11	0.01\\
98.12	0.01\\
98.13	0.01\\
98.14	0.01\\
98.15	0.01\\
98.16	0.01\\
98.17	0.01\\
98.18	0.01\\
98.19	0.01\\
98.2	0.01\\
98.21	0.01\\
98.22	0.01\\
98.23	0.01\\
98.24	0.01\\
98.25	0.01\\
98.26	0.01\\
98.27	0.01\\
98.28	0.01\\
98.29	0.01\\
98.3	0.01\\
98.31	0.01\\
98.32	0.01\\
98.33	0.01\\
98.34	0.01\\
98.35	0.01\\
98.36	0.01\\
98.37	0.01\\
98.38	0.01\\
98.39	0.01\\
98.4	0.01\\
98.41	0.01\\
98.42	0.01\\
98.43	0.01\\
98.44	0.01\\
98.45	0.01\\
98.46	0.01\\
98.47	0.01\\
98.48	0.01\\
98.49	0.01\\
98.5	0.01\\
98.51	0.01\\
98.52	0.01\\
98.53	0.01\\
98.54	0.01\\
98.55	0.01\\
98.56	0.01\\
98.57	0.01\\
98.58	0.01\\
98.59	0.01\\
98.6	0.01\\
98.61	0.01\\
98.62	0.01\\
98.63	0.01\\
98.64	0.01\\
98.65	0.01\\
98.66	0.01\\
98.67	0.01\\
98.68	0.01\\
98.69	0.01\\
98.7	0.01\\
98.71	0.01\\
98.72	0.01\\
98.73	0.01\\
98.74	0.01\\
98.75	0.01\\
98.76	0.01\\
98.77	0.01\\
98.78	0.01\\
98.79	0.01\\
98.8	0.01\\
98.81	0.01\\
98.82	0.01\\
98.83	0.01\\
98.84	0.01\\
98.85	0.01\\
98.86	0.01\\
98.87	0.01\\
98.88	0.01\\
98.89	0.01\\
98.9	0.01\\
98.91	0.01\\
98.92	0.01\\
98.93	0.01\\
98.94	0.01\\
98.95	0.01\\
98.96	0.01\\
98.97	0.01\\
98.98	0.01\\
98.99	0.01\\
99	0.01\\
99.01	0.01\\
99.02	0.01\\
99.03	0.01\\
99.04	0.01\\
99.05	0.01\\
99.06	0.01\\
99.07	0.01\\
99.08	0.01\\
99.09	0.01\\
99.1	0.01\\
99.11	0.01\\
99.12	0.01\\
99.13	0.01\\
99.14	0.01\\
99.15	0.01\\
99.16	0.01\\
99.17	0.01\\
99.18	0.01\\
99.19	0.01\\
99.2	0.01\\
99.21	0.01\\
99.22	0.01\\
99.23	0.01\\
99.24	0.01\\
99.25	0.01\\
99.26	0.01\\
99.27	0.01\\
99.28	0.01\\
99.29	0.01\\
99.3	0.01\\
99.31	0.01\\
99.32	0.01\\
99.33	0.01\\
99.34	0.01\\
99.35	0.01\\
99.36	0.01\\
99.37	0.01\\
99.38	0.01\\
99.39	0.01\\
99.4	0.01\\
99.41	0.01\\
99.42	0.0098574925152882\\
99.43	0.00971396905841077\\
99.44	0.00956972004081591\\
99.45	0.0094247362673596\\
99.46	0.0092790082849791\\
99.47	0.00913252639168261\\
99.48	0.00898529998637358\\
99.49	0.00883734285850463\\
99.5	0.00868864559125499\\
99.51	0.00853919851239646\\
99.52	0.00838899168717304\\
99.53	0.00823801491096786\\
99.54	0.00808625770174855\\
99.55	0.00793370929228211\\
99.56	0.00778035862210953\\
99.57	0.00762619432926998\\
99.58	0.00747120474176386\\
99.59	0.00731537786874324\\
99.6	0.00715870139141777\\
99.61	0.00700116265366315\\
99.62	0.00684274865231885\\
99.63	0.00668344602716063\\
99.64	0.00652324105053277\\
99.65	0.00636211961662405\\
99.66	0.00620006723037029\\
99.67	0.00603706899596548\\
99.68	0.0058731096049623\\
99.69	0.00570817332394313\\
99.7	0.00554224398173856\\
99.71	0.00537530495617044\\
99.72	0.00520733916022993\\
99.73	0.00503832902765711\\
99.74	0.00486825649810313\\
99.75	0.00469710300170934\\
99.76	0.00452484944307186\\
99.77	0.00435147618455827\\
99.78	0.00417696302894062\\
99.79	0.00400128920130679\\
99.8	0.00382443333020954\\
99.81	0.00364637342800988\\
99.82	0.00346708687036842\\
99.83	0.00328655037483507\\
99.84	0.00310473997848419\\
99.85	0.00292163101453843\\
99.86	0.00273719808792035\\
99.87	0.00255141504966703\\
99.88	0.00236425497013749\\
99.89	0.0021756901109384\\
99.9	0.00198569189548735\\
99.91	0.0017942310464843\\
99.92	0.00160127761552209\\
99.93	0.00140680072665657\\
99.94	0.00121076854085607\\
99.95	0.00101314821860146\\
99.96	0.000813905880513196\\
99.97	0.000613006565871897\\
99.98	0.000410414188888463\\
99.99	0.000206091492568098\\
100	0\\
};
\addlegendentry{$q=-1$};

\addplot [color=black,solid,forget plot]
  table[row sep=crcr]{%
0.01	0.000772572677978338\\
0.02	0.000772572809190752\\
0.03	0.000772572940773125\\
0.04	0.000772573072723257\\
0.05	0.000772573205038835\\
0.06	0.000772573337717434\\
0.07	0.000772573470756514\\
0.08	0.000772573604153424\\
0.09	0.000772573737905393\\
0.1	0.00077257387200954\\
0.11	0.000772574006462861\\
0.12	0.00077257414126224\\
0.13	0.000772574276404437\\
0.14	0.000772574411886101\\
0.15	0.000772574547703757\\
0.16	0.000772574683853814\\
0.17	0.000772574820332559\\
0.18	0.000772574957136164\\
0.19	0.00077257509426068\\
0.2	0.000772575231702044\\
0.21	0.000772575369456071\\
0.22	0.000772575507518461\\
0.23	0.000772575645884796\\
0.24	0.000772575784550549\\
0.25	0.000772575923511074\\
0.26	0.000772576062761616\\
0.27	0.00077257620229731\\
0.28	0.000772576342113182\\
0.29	0.000772576482204152\\
0.3	0.000772576622565036\\
0.31	0.000772576763190553\\
0.32	0.00077257690407532\\
0.33	0.000772577045213864\\
0.34	0.000772577186600619\\
0.35	0.000772577328229931\\
0.36	0.000772577470096068\\
0.37	0.00077257761219322\\
0.38	0.000772577754515502\\
0.39	0.000772577897056967\\
0.4	0.000772578039811603\\
0.41	0.000772578182773346\\
0.42	0.000772578325936087\\
0.43	0.000772578469293671\\
0.44	0.000772578612839919\\
0.45	0.000772578756568621\\
0.46	0.00077257890047356\\
0.47	0.000772579044548506\\
0.48	0.00077257918878724\\
0.49	0.000772579333183558\\
0.5	0.000772579477731282\\
0.51	0.000772579622424278\\
0.52	0.000772579767256459\\
0.53	0.000772579912221807\\
0.54	0.000772580057314386\\
0.55	0.000772580202528355\\
0.56	0.000772580347857987\\
0.57	0.000772580493297682\\
0.58	0.00077258063884199\\
0.59	0.000772580784485626\\
0.6	0.000772580930223494\\
0.61	0.000772581076050702\\
0.62	0.00077258122196259\\
0.63	0.000772581367954753\\
0.64	0.000772581514023063\\
0.65	0.000772581660163694\\
0.66	0.000772581806373152\\
0.67	0.000772581952648302\\
0.68	0.0007725820989864\\
0.69	0.000772582245385123\\
0.7	0.000772582391842599\\
0.71	0.000772582538357448\\
0.72	0.000772582684928812\\
0.73	0.000772582831556399\\
0.74	0.000772582978240231\\
0.75	0.00077258312498033\\
0.76	0.000772583271776721\\
0.77	0.000772583418629428\\
0.78	0.000772583565538474\\
0.79	0.000772583712503881\\
0.8	0.000772583859525675\\
0.81	0.000772584006603879\\
0.82	0.000772584153738515\\
0.83	0.000772584300929607\\
0.84	0.000772584448177181\\
0.85	0.000772584595481258\\
0.86	0.000772584742841862\\
0.87	0.000772584890259017\\
0.88	0.000772585037732747\\
0.89	0.000772585185263075\\
0.9	0.000772585332850025\\
0.91	0.00077258548049362\\
0.92	0.000772585628193885\\
0.93	0.000772585775950843\\
0.94	0.000772585923764517\\
0.95	0.000772586071634931\\
0.96	0.00077258621956211\\
0.97	0.000772586367546077\\
0.98	0.000772586515586855\\
0.99	0.000772586663684469\\
1	0.000772586811838942\\
1.01	0.000772586960050298\\
1.02	0.000772587108318561\\
1.03	0.000772587256643753\\
1.04	0.0007725874050259\\
1.05	0.000772587553465025\\
1.06	0.000772587701961152\\
1.07	0.000772587850514306\\
1.08	0.000772587999124509\\
1.09	0.000772588147791785\\
1.1	0.00077258829651616\\
1.11	0.000772588445297656\\
1.12	0.000772588594136297\\
1.13	0.000772588743032107\\
1.14	0.00077258889198511\\
1.15	0.000772589040995331\\
1.16	0.000772589190062793\\
1.17	0.000772589339187519\\
1.18	0.000772589488369535\\
1.19	0.000772589637608864\\
1.2	0.000772589786905529\\
1.21	0.000772589936259557\\
1.22	0.000772590085670969\\
1.23	0.00077259023513979\\
1.24	0.000772590384666044\\
1.25	0.000772590534249756\\
1.26	0.000772590683890949\\
1.27	0.000772590833589649\\
1.28	0.000772590983345877\\
1.29	0.00077259113315966\\
1.3	0.00077259128303102\\
1.31	0.000772591432959982\\
1.32	0.000772591582946571\\
1.33	0.00077259173299081\\
1.34	0.000772591883092724\\
1.35	0.000772592033252336\\
1.36	0.000772592183469671\\
1.37	0.000772592333744753\\
1.38	0.000772592484077607\\
1.39	0.000772592634468256\\
1.4	0.000772592784916724\\
1.41	0.000772592935423037\\
1.42	0.000772593085987218\\
1.43	0.000772593236609291\\
1.44	0.000772593387289282\\
1.45	0.000772593538027215\\
1.46	0.000772593688823112\\
1.47	0.000772593839676998\\
1.48	0.000772593990588899\\
1.49	0.000772594141558839\\
1.5	0.000772594292586842\\
1.51	0.000772594443672931\\
1.52	0.000772594594817133\\
1.53	0.000772594746019472\\
1.54	0.000772594897279971\\
1.55	0.000772595048598654\\
1.56	0.000772595199975547\\
1.57	0.000772595351410674\\
1.58	0.00077259550290406\\
1.59	0.000772595654455728\\
1.6	0.000772595806065705\\
1.61	0.000772595957734013\\
1.62	0.000772596109460677\\
1.63	0.000772596261245722\\
1.64	0.000772596413089173\\
1.65	0.000772596564991053\\
1.66	0.000772596716951388\\
1.67	0.000772596868970202\\
1.68	0.000772597021047519\\
1.69	0.000772597173183365\\
1.7	0.000772597325377765\\
1.71	0.00077259747763074\\
1.72	0.000772597629942319\\
1.73	0.000772597782312524\\
1.74	0.000772597934741381\\
1.75	0.000772598087228914\\
1.76	0.000772598239775149\\
1.77	0.000772598392380109\\
1.78	0.000772598545043818\\
1.79	0.000772598697766304\\
1.8	0.00077259885054759\\
1.81	0.0007725990033877\\
1.82	0.00077259915628666\\
1.83	0.000772599309244492\\
1.84	0.000772599462261223\\
1.85	0.000772599615336879\\
1.86	0.000772599768471483\\
1.87	0.000772599921665061\\
1.88	0.000772600074917636\\
1.89	0.000772600228229234\\
1.9	0.00077260038159988\\
1.91	0.000772600535029599\\
1.92	0.000772600688518416\\
1.93	0.000772600842066354\\
1.94	0.000772600995673441\\
1.95	0.0007726011493397\\
1.96	0.000772601303065156\\
1.97	0.000772601456849834\\
1.98	0.000772601610693759\\
1.99	0.000772601764596956\\
2	0.000772601918559452\\
2.01	0.000772602072581268\\
2.02	0.000772602226662433\\
2.03	0.00077260238080297\\
2.04	0.000772602535002904\\
2.05	0.000772602689262259\\
2.06	0.000772602843581062\\
2.07	0.000772602997959337\\
2.08	0.000772603152397111\\
2.09	0.000772603306894406\\
2.1	0.000772603461451249\\
2.11	0.000772603616067666\\
2.12	0.000772603770743681\\
2.13	0.000772603925479318\\
2.14	0.000772604080274604\\
2.15	0.000772604235129563\\
2.16	0.000772604390044221\\
2.17	0.000772604545018603\\
2.18	0.000772604700052734\\
2.19	0.000772604855146639\\
2.2	0.000772605010300343\\
2.21	0.000772605165513873\\
2.22	0.000772605320787252\\
2.23	0.000772605476120507\\
2.24	0.000772605631513662\\
2.25	0.000772605786966743\\
2.26	0.000772605942479776\\
2.27	0.000772606098052785\\
2.28	0.000772606253685797\\
2.29	0.000772606409378835\\
2.3	0.000772606565131926\\
2.31	0.000772606720945096\\
2.32	0.000772606876818368\\
2.33	0.00077260703275177\\
2.34	0.000772607188745325\\
2.35	0.000772607344799061\\
2.36	0.000772607500913002\\
2.37	0.000772607657087173\\
2.38	0.000772607813321599\\
2.39	0.000772607969616307\\
2.4	0.000772608125971323\\
2.41	0.000772608282386671\\
2.42	0.000772608438862376\\
2.43	0.000772608595398466\\
2.44	0.000772608751994966\\
2.45	0.0007726089086519\\
2.46	0.000772609065369294\\
2.47	0.000772609222147174\\
2.48	0.000772609378985566\\
2.49	0.000772609535884494\\
2.5	0.000772609692843984\\
2.51	0.000772609849864063\\
2.52	0.000772610006944757\\
2.53	0.00077261016408609\\
2.54	0.000772610321288088\\
2.55	0.000772610478550777\\
2.56	0.000772610635874183\\
2.57	0.000772610793258332\\
2.58	0.000772610950703249\\
2.59	0.000772611108208959\\
2.6	0.000772611265775488\\
2.61	0.000772611423402863\\
2.62	0.000772611581091109\\
2.63	0.000772611738840252\\
2.64	0.000772611896650319\\
2.65	0.000772612054521334\\
2.66	0.000772612212453323\\
2.67	0.000772612370446313\\
2.68	0.000772612528500329\\
2.69	0.000772612686615397\\
2.7	0.000772612844791543\\
2.71	0.000772613003028793\\
2.72	0.000772613161327172\\
2.73	0.000772613319686708\\
2.74	0.000772613478107424\\
2.75	0.000772613636589347\\
2.76	0.000772613795132505\\
2.77	0.000772613953736923\\
2.78	0.000772614112402625\\
2.79	0.000772614271129639\\
2.8	0.00077261442991799\\
2.81	0.000772614588767704\\
2.82	0.000772614747678809\\
2.83	0.000772614906651328\\
2.84	0.00077261506568529\\
2.85	0.000772615224780719\\
2.86	0.000772615383937643\\
2.87	0.000772615543156087\\
2.88	0.000772615702436075\\
2.89	0.000772615861777636\\
2.9	0.000772616021180795\\
2.91	0.00077261618064558\\
2.92	0.000772616340172016\\
2.93	0.000772616499760128\\
2.94	0.000772616659409942\\
2.95	0.000772616819121486\\
2.96	0.000772616978894787\\
2.97	0.000772617138729868\\
2.98	0.000772617298626757\\
2.99	0.00077261745858548\\
3	0.000772617618606065\\
3.01	0.000772617778688537\\
3.02	0.000772617938832921\\
3.03	0.000772618099039245\\
3.04	0.000772618259307535\\
3.05	0.000772618419637817\\
3.06	0.000772618580030117\\
3.07	0.000772618740484463\\
3.08	0.000772618901000881\\
3.09	0.000772619061579395\\
3.1	0.000772619222220034\\
3.11	0.000772619382922824\\
3.12	0.000772619543687791\\
3.13	0.000772619704514961\\
3.14	0.000772619865404361\\
3.15	0.000772620026356018\\
3.16	0.000772620187369958\\
3.17	0.000772620348446208\\
3.18	0.000772620509584793\\
3.19	0.000772620670785739\\
3.2	0.000772620832049075\\
3.21	0.000772620993374827\\
3.22	0.000772621154763021\\
3.23	0.000772621316213685\\
3.24	0.000772621477726844\\
3.25	0.000772621639302523\\
3.26	0.000772621800940752\\
3.27	0.000772621962641556\\
3.28	0.000772622124404963\\
3.29	0.000772622286230998\\
3.3	0.000772622448119688\\
3.31	0.00077262261007106\\
3.32	0.000772622772085141\\
3.33	0.000772622934161958\\
3.34	0.000772623096301538\\
3.35	0.000772623258503906\\
3.36	0.00077262342076909\\
3.37	0.000772623583097116\\
3.38	0.000772623745488012\\
3.39	0.000772623907941803\\
3.4	0.000772624070458517\\
3.41	0.00077262423303818\\
3.42	0.000772624395680821\\
3.43	0.000772624558386465\\
3.44	0.00077262472115514\\
3.45	0.000772624883986873\\
3.46	0.000772625046881689\\
3.47	0.000772625209839615\\
3.48	0.000772625372860681\\
3.49	0.000772625535944911\\
3.5	0.000772625699092333\\
3.51	0.000772625862302975\\
3.52	0.000772626025576862\\
3.53	0.000772626188914022\\
3.54	0.000772626352314482\\
3.55	0.000772626515778269\\
3.56	0.000772626679305411\\
3.57	0.000772626842895932\\
3.58	0.000772627006549863\\
3.59	0.000772627170267229\\
3.6	0.000772627334048056\\
3.61	0.000772627497892373\\
3.62	0.000772627661800206\\
3.63	0.000772627825771584\\
3.64	0.000772627989806533\\
3.65	0.000772628153905079\\
3.66	0.000772628318067251\\
3.67	0.000772628482293074\\
3.68	0.000772628646582578\\
3.69	0.000772628810935788\\
3.7	0.000772628975352734\\
3.71	0.00077262913983344\\
3.72	0.000772629304377935\\
3.73	0.000772629468986246\\
3.74	0.0007726296336584\\
3.75	0.000772629798394424\\
3.76	0.000772629963194346\\
3.77	0.000772630128058192\\
3.78	0.000772630292985993\\
3.79	0.000772630457977772\\
3.8	0.000772630623033559\\
3.81	0.000772630788153381\\
3.82	0.000772630953337265\\
3.83	0.000772631118585238\\
3.84	0.000772631283897329\\
3.85	0.000772631449273564\\
3.86	0.00077263161471397\\
3.87	0.000772631780218577\\
3.88	0.00077263194578741\\
3.89	0.000772632111420497\\
3.9	0.000772632277117866\\
3.91	0.000772632442879544\\
3.92	0.00077263260870556\\
3.93	0.00077263277459594\\
3.94	0.000772632940550712\\
3.95	0.000772633106569904\\
3.96	0.000772633272653544\\
3.97	0.000772633438801658\\
3.98	0.000772633605014275\\
3.99	0.000772633771291421\\
4	0.000772633937633126\\
4.01	0.000772634104039415\\
4.02	0.000772634270510318\\
4.03	0.000772634437045863\\
4.04	0.000772634603646075\\
4.05	0.000772634770310985\\
4.06	0.000772634937040618\\
4.07	0.000772635103835004\\
4.08	0.000772635270694169\\
4.09	0.000772635437618142\\
4.1	0.00077263560460695\\
4.11	0.000772635771660622\\
4.12	0.000772635938779184\\
4.13	0.000772636105962664\\
4.14	0.000772636273211093\\
4.15	0.000772636440524496\\
4.16	0.0007726366079029\\
4.17	0.000772636775346336\\
4.18	0.000772636942854829\\
4.19	0.000772637110428409\\
4.2	0.000772637278067104\\
4.21	0.000772637445770941\\
4.22	0.000772637613539948\\
4.23	0.000772637781374153\\
4.24	0.000772637949273583\\
4.25	0.000772638117238269\\
4.26	0.000772638285268235\\
4.27	0.000772638453363513\\
4.28	0.000772638621524128\\
4.29	0.00077263878975011\\
4.3	0.000772638958041487\\
4.31	0.000772639126398286\\
4.32	0.000772639294820536\\
4.33	0.000772639463308265\\
4.34	0.000772639631861501\\
4.35	0.000772639800480273\\
4.36	0.000772639969164607\\
4.37	0.000772640137914532\\
4.38	0.000772640306730079\\
4.39	0.000772640475611272\\
4.4	0.000772640644558143\\
4.41	0.000772640813570718\\
4.42	0.000772640982649026\\
4.43	0.000772641151793095\\
4.44	0.000772641321002953\\
4.45	0.000772641490278629\\
4.46	0.000772641659620151\\
4.47	0.000772641829027547\\
4.48	0.000772641998500847\\
4.49	0.000772642168040077\\
4.5	0.000772642337645268\\
4.51	0.000772642507316446\\
4.52	0.000772642677053641\\
4.53	0.00077264284685688\\
4.54	0.000772643016726193\\
4.55	0.000772643186661608\\
4.56	0.000772643356663152\\
4.57	0.000772643526730855\\
4.58	0.000772643696864746\\
4.59	0.000772643867064852\\
4.6	0.000772644037331202\\
4.61	0.000772644207663825\\
4.62	0.000772644378062749\\
4.63	0.000772644548528004\\
4.64	0.000772644719059617\\
4.65	0.000772644889657618\\
4.66	0.000772645060322034\\
4.67	0.000772645231052894\\
4.68	0.000772645401850229\\
4.69	0.000772645572714064\\
4.7	0.000772645743644431\\
4.71	0.000772645914641356\\
4.72	0.00077264608570487\\
4.73	0.000772646256835001\\
4.74	0.000772646428031777\\
4.75	0.000772646599295226\\
4.76	0.00077264677062538\\
4.77	0.000772646942022264\\
4.78	0.000772647113485909\\
4.79	0.000772647285016343\\
4.8	0.000772647456613595\\
4.81	0.000772647628277695\\
4.82	0.00077264780000867\\
4.83	0.00077264797180655\\
4.84	0.000772648143671364\\
4.85	0.00077264831560314\\
4.86	0.000772648487601908\\
4.87	0.000772648659667696\\
4.88	0.000772648831800534\\
4.89	0.00077264900400045\\
4.9	0.000772649176267473\\
4.91	0.000772649348601634\\
4.92	0.00077264952100296\\
4.93	0.00077264969347148\\
4.94	0.000772649866007225\\
4.95	0.000772650038610221\\
4.96	0.000772650211280498\\
4.97	0.000772650384018088\\
4.98	0.000772650556823016\\
4.99	0.000772650729695314\\
5	0.00077265090263501\\
5.01	0.000772651075642132\\
5.02	0.000772651248716713\\
5.03	0.000772651421858778\\
5.04	0.000772651595068358\\
5.05	0.000772651768345484\\
5.06	0.000772651941690182\\
5.07	0.000772652115102482\\
5.08	0.000772652288582414\\
5.09	0.000772652462130007\\
5.1	0.00077265263574529\\
5.11	0.000772652809428294\\
5.12	0.000772652983179047\\
5.13	0.000772653156997578\\
5.14	0.000772653330883917\\
5.15	0.000772653504838093\\
5.16	0.000772653678860136\\
5.17	0.000772653852950075\\
5.18	0.000772654027107939\\
5.19	0.000772654201333758\\
5.2	0.000772654375627561\\
5.21	0.000772654549989378\\
5.22	0.000772654724419237\\
5.23	0.000772654898917169\\
5.24	0.000772655073483204\\
5.25	0.000772655248117371\\
5.26	0.000772655422819698\\
5.27	0.000772655597590217\\
5.28	0.000772655772428956\\
5.29	0.000772655947335945\\
5.3	0.000772656122311213\\
5.31	0.00077265629735479\\
5.32	0.000772656472466706\\
5.33	0.000772656647646992\\
5.34	0.000772656822895675\\
5.35	0.000772656998212787\\
5.36	0.000772657173598356\\
5.37	0.000772657349052411\\
5.38	0.000772657524574984\\
5.39	0.000772657700166103\\
5.4	0.000772657875825799\\
5.41	0.0007726580515541\\
5.42	0.000772658227351038\\
5.43	0.000772658403216641\\
5.44	0.000772658579150941\\
5.45	0.000772658755153966\\
5.46	0.000772658931225747\\
5.47	0.000772659107366312\\
5.48	0.000772659283575693\\
5.49	0.000772659459853919\\
5.5	0.000772659636201021\\
5.51	0.000772659812617028\\
5.52	0.000772659989101969\\
5.53	0.000772660165655874\\
5.54	0.000772660342278774\\
5.55	0.000772660518970699\\
5.56	0.00077266069573168\\
5.57	0.000772660872561745\\
5.58	0.000772661049460924\\
5.59	0.000772661226429249\\
5.6	0.000772661403466749\\
5.61	0.000772661580573453\\
5.62	0.000772661757749394\\
5.63	0.000772661934994598\\
5.64	0.0007726621123091\\
5.65	0.000772662289692926\\
5.66	0.000772662467146109\\
5.67	0.000772662644668676\\
5.68	0.000772662822260661\\
5.69	0.000772662999922092\\
5.7	0.000772663177653\\
5.71	0.000772663355453415\\
5.72	0.000772663533323366\\
5.73	0.000772663711262885\\
5.74	0.000772663889272001\\
5.75	0.000772664067350746\\
5.76	0.00077266424549915\\
5.77	0.000772664423717241\\
5.78	0.000772664602005053\\
5.79	0.000772664780362614\\
5.8	0.000772664958789955\\
5.81	0.000772665137287106\\
5.82	0.000772665315854097\\
5.83	0.000772665494490959\\
5.84	0.000772665673197724\\
5.85	0.00077266585197442\\
5.86	0.00077266603082108\\
5.87	0.000772666209737732\\
5.88	0.000772666388724409\\
5.89	0.000772666567781138\\
5.9	0.000772666746907952\\
5.91	0.000772666926104882\\
5.92	0.000772667105371957\\
5.93	0.000772667284709209\\
5.94	0.000772667464116668\\
5.95	0.000772667643594365\\
5.96	0.00077266782314233\\
5.97	0.000772668002760594\\
5.98	0.000772668182449188\\
5.99	0.000772668362208143\\
6	0.000772668542037488\\
6.01	0.000772668721937257\\
6.02	0.000772668901907478\\
6.03	0.000772669081948183\\
6.04	0.000772669262059402\\
6.05	0.000772669442241166\\
6.06	0.000772669622493505\\
6.07	0.000772669802816451\\
6.08	0.000772669983210034\\
6.09	0.000772670163674286\\
6.1	0.000772670344209237\\
6.11	0.000772670524814919\\
6.12	0.000772670705491361\\
6.13	0.000772670886238596\\
6.14	0.000772671067056654\\
6.15	0.000772671247945566\\
6.16	0.000772671428905362\\
6.17	0.000772671609936075\\
6.18	0.000772671791037734\\
6.19	0.000772671972210372\\
6.2	0.000772672153454019\\
6.21	0.000772672334768706\\
6.22	0.000772672516154463\\
6.23	0.000772672697611323\\
6.24	0.000772672879139317\\
6.25	0.000772673060738475\\
6.26	0.000772673242408829\\
6.27	0.000772673424150409\\
6.28	0.000772673605963246\\
6.29	0.000772673787847374\\
6.3	0.000772673969802821\\
6.31	0.000772674151829621\\
6.32	0.000772674333927803\\
6.33	0.0007726745160974\\
6.34	0.000772674698338442\\
6.35	0.000772674880650961\\
6.36	0.000772675063034987\\
6.37	0.000772675245490554\\
6.38	0.000772675428017689\\
6.39	0.000772675610616428\\
6.4	0.0007726757932868\\
6.41	0.000772675976028836\\
6.42	0.000772676158842569\\
6.43	0.000772676341728029\\
6.44	0.000772676524685248\\
6.45	0.000772676707714259\\
6.46	0.00077267689081509\\
6.47	0.000772677073987775\\
6.48	0.000772677257232346\\
6.49	0.000772677440548833\\
6.5	0.000772677623937268\\
6.51	0.000772677807397683\\
6.52	0.00077267799093011\\
6.53	0.000772678174534578\\
6.54	0.000772678358211123\\
6.55	0.000772678541959773\\
6.56	0.00077267872578056\\
6.57	0.000772678909673516\\
6.58	0.000772679093638673\\
6.59	0.000772679277676065\\
6.6	0.000772679461785719\\
6.61	0.00077267964596767\\
6.62	0.000772679830221949\\
6.63	0.000772680014548588\\
6.64	0.000772680198947618\\
6.65	0.000772680383419072\\
6.66	0.000772680567962982\\
6.67	0.000772680752579377\\
6.68	0.000772680937268292\\
6.69	0.000772681122029757\\
6.7	0.000772681306863804\\
6.71	0.000772681491770467\\
6.72	0.000772681676749775\\
6.73	0.000772681861801763\\
6.74	0.000772682046926459\\
6.75	0.000772682232123898\\
6.76	0.000772682417394113\\
6.77	0.000772682602737133\\
6.78	0.000772682788152991\\
6.79	0.00077268297364172\\
6.8	0.00077268315920335\\
6.81	0.000772683344837915\\
6.82	0.000772683530545446\\
6.83	0.000772683716325976\\
6.84	0.000772683902179536\\
6.85	0.000772684088106159\\
6.86	0.000772684274105876\\
6.87	0.000772684460178721\\
6.88	0.000772684646324726\\
6.89	0.000772684832543922\\
6.9	0.000772685018836341\\
6.91	0.000772685205202017\\
6.92	0.00077268539164098\\
6.93	0.000772685578153265\\
6.94	0.000772685764738903\\
6.95	0.000772685951397924\\
6.96	0.000772686138130365\\
6.97	0.000772686324936255\\
6.98	0.000772686511815626\\
6.99	0.000772686698768512\\
7	0.000772686885794944\\
7.01	0.000772687072894957\\
7.02	0.000772687260068581\\
7.03	0.000772687447315849\\
7.04	0.000772687634636794\\
7.05	0.000772687822031447\\
7.06	0.000772688009499842\\
7.07	0.000772688197042012\\
7.08	0.000772688384657988\\
7.09	0.000772688572347803\\
7.1	0.00077268876011149\\
7.11	0.000772688947949081\\
7.12	0.000772689135860609\\
7.13	0.000772689323846106\\
7.14	0.000772689511905606\\
7.15	0.000772689700039141\\
7.16	0.000772689888246744\\
7.17	0.000772690076528446\\
7.18	0.000772690264884281\\
7.19	0.000772690453314281\\
7.2	0.00077269064181848\\
7.21	0.00077269083039691\\
7.22	0.000772691019049605\\
7.23	0.000772691207776595\\
7.24	0.000772691396577915\\
7.25	0.000772691585453598\\
7.26	0.000772691774403676\\
7.27	0.00077269196342818\\
7.28	0.000772692152527146\\
7.29	0.000772692341700605\\
7.3	0.000772692530948591\\
7.31	0.000772692720271136\\
7.32	0.000772692909668275\\
7.33	0.000772693099140038\\
7.34	0.00077269328868646\\
7.35	0.000772693478307574\\
7.36	0.000772693668003411\\
7.37	0.000772693857774006\\
7.38	0.000772694047619392\\
7.39	0.000772694237539602\\
7.4	0.000772694427534668\\
7.41	0.000772694617604625\\
7.42	0.000772694807749504\\
7.43	0.000772694997969338\\
7.44	0.000772695188264162\\
7.45	0.000772695378634008\\
7.46	0.000772695569078909\\
7.47	0.0007726957595989\\
7.48	0.000772695950194012\\
7.49	0.00077269614086428\\
7.5	0.000772696331609736\\
7.51	0.000772696522430413\\
7.52	0.000772696713326346\\
7.53	0.000772696904297565\\
7.54	0.000772697095344107\\
7.55	0.000772697286466003\\
7.56	0.000772697477663288\\
7.57	0.000772697668935994\\
7.58	0.000772697860284156\\
7.59	0.000772698051707805\\
7.6	0.000772698243206978\\
7.61	0.000772698434781704\\
7.62	0.00077269862643202\\
7.63	0.000772698818157959\\
7.64	0.000772699009959552\\
7.65	0.000772699201836835\\
7.66	0.000772699393789841\\
7.67	0.000772699585818604\\
7.68	0.000772699777923156\\
7.69	0.000772699970103532\\
7.7	0.000772700162359766\\
7.71	0.00077270035469189\\
7.72	0.000772700547099938\\
7.73	0.000772700739583944\\
7.74	0.000772700932143941\\
7.75	0.000772701124779965\\
7.76	0.000772701317492047\\
7.77	0.000772701510280221\\
7.78	0.000772701703144522\\
7.79	0.000772701896084983\\
7.8	0.000772702089101638\\
7.81	0.000772702282194521\\
7.82	0.000772702475363665\\
7.83	0.000772702668609106\\
7.84	0.000772702861930876\\
7.85	0.000772703055329008\\
7.86	0.000772703248803538\\
7.87	0.000772703442354498\\
7.88	0.000772703635981923\\
7.89	0.000772703829685847\\
7.9	0.000772704023466304\\
7.91	0.000772704217323327\\
7.92	0.000772704411256951\\
7.93	0.000772704605267209\\
7.94	0.000772704799354136\\
7.95	0.000772704993517765\\
7.96	0.000772705187758132\\
7.97	0.000772705382075269\\
7.98	0.000772705576469211\\
7.99	0.000772705770939991\\
8	0.000772705965487645\\
8.01	0.000772706160112206\\
8.02	0.000772706354813708\\
8.03	0.000772706549592186\\
8.04	0.000772706744447673\\
8.05	0.000772706939380204\\
8.06	0.000772707134389813\\
8.07	0.000772707329476534\\
8.08	0.000772707524640402\\
8.09	0.000772707719881451\\
8.1	0.000772707915199714\\
8.11	0.000772708110595226\\
8.12	0.000772708306068024\\
8.13	0.000772708501618138\\
8.14	0.000772708697245605\\
8.15	0.000772708892950459\\
8.16	0.000772709088732735\\
8.17	0.000772709284592466\\
8.18	0.000772709480529687\\
8.19	0.000772709676544431\\
8.2	0.000772709872636735\\
8.21	0.000772710068806632\\
8.22	0.000772710265054157\\
8.23	0.000772710461379344\\
8.24	0.000772710657782227\\
8.25	0.000772710854262842\\
8.26	0.000772711050821223\\
8.27	0.000772711247457404\\
8.28	0.000772711444171421\\
8.29	0.000772711640963307\\
8.3	0.000772711837833099\\
8.31	0.000772712034780828\\
8.32	0.000772712231806531\\
8.33	0.000772712428910242\\
8.34	0.000772712626091996\\
8.35	0.000772712823351828\\
8.36	0.000772713020689772\\
8.37	0.000772713218105863\\
8.38	0.000772713415600135\\
8.39	0.000772713613172625\\
8.4	0.000772713810823366\\
8.41	0.000772714008552394\\
8.42	0.000772714206359743\\
8.43	0.000772714404245447\\
8.44	0.000772714602209542\\
8.45	0.000772714800252062\\
8.46	0.000772714998373041\\
8.47	0.000772715196572517\\
8.48	0.000772715394850522\\
8.49	0.000772715593207093\\
8.5	0.000772715791642264\\
8.51	0.00077271599015607\\
8.52	0.000772716188748546\\
8.53	0.000772716387419726\\
8.54	0.000772716586169647\\
8.55	0.000772716784998342\\
8.56	0.000772716983905847\\
8.57	0.000772717182892197\\
8.58	0.000772717381957427\\
8.59	0.000772717581101572\\
8.6	0.000772717780324669\\
8.61	0.00077271797962675\\
8.62	0.000772718179007852\\
8.63	0.000772718378468009\\
8.64	0.000772718578007257\\
8.65	0.000772718777625631\\
8.66	0.000772718977323167\\
8.67	0.000772719177099899\\
8.68	0.000772719376955862\\
8.69	0.000772719576891092\\
8.7	0.000772719776905625\\
8.71	0.000772719976999495\\
8.72	0.000772720177172738\\
8.73	0.000772720377425389\\
8.74	0.000772720577757482\\
8.75	0.000772720778169055\\
8.76	0.00077272097866014\\
8.77	0.000772721179230775\\
8.78	0.000772721379880995\\
8.79	0.000772721580610836\\
8.8	0.000772721781420332\\
8.81	0.000772721982309519\\
8.82	0.000772722183278432\\
8.83	0.000772722384327107\\
8.84	0.000772722585455578\\
8.85	0.000772722786663882\\
8.86	0.000772722987952053\\
8.87	0.000772723189320128\\
8.88	0.000772723390768143\\
8.89	0.000772723592296132\\
8.9	0.000772723793904131\\
8.91	0.000772723995592176\\
8.92	0.000772724197360302\\
8.93	0.000772724399208545\\
8.94	0.000772724601136941\\
8.95	0.000772724803145525\\
8.96	0.000772725005234332\\
8.97	0.0007727252074034\\
8.98	0.000772725409652762\\
8.99	0.000772725611982454\\
9	0.000772725814392514\\
9.01	0.000772726016882976\\
9.02	0.000772726219453875\\
9.03	0.000772726422105246\\
9.04	0.000772726624837128\\
9.05	0.000772726827649554\\
9.06	0.000772727030542561\\
9.07	0.000772727233516185\\
9.08	0.000772727436570459\\
9.09	0.000772727639705423\\
9.1	0.00077272784292111\\
9.11	0.000772728046217558\\
9.12	0.0007727282495948\\
9.13	0.000772728453052873\\
9.14	0.000772728656591814\\
9.15	0.000772728860211657\\
9.16	0.000772729063912439\\
9.17	0.000772729267694196\\
9.18	0.000772729471556963\\
9.19	0.000772729675500777\\
9.2	0.000772729879525673\\
9.21	0.000772730083631688\\
9.22	0.000772730287818857\\
9.23	0.000772730492087216\\
9.24	0.0007727306964368\\
9.25	0.000772730900867647\\
9.26	0.000772731105379792\\
9.27	0.000772731309973271\\
9.28	0.000772731514648121\\
9.29	0.000772731719404377\\
9.3	0.000772731924242074\\
9.31	0.000772732129161249\\
9.32	0.000772732334161939\\
9.33	0.000772732539244179\\
9.34	0.000772732744408006\\
9.35	0.000772732949653455\\
9.36	0.000772733154980563\\
9.37	0.000772733360389365\\
9.38	0.000772733565879898\\
9.39	0.000772733771452197\\
9.4	0.000772733977106298\\
9.41	0.000772734182842238\\
9.42	0.000772734388660054\\
9.43	0.000772734594559782\\
9.44	0.000772734800541457\\
9.45	0.000772735006605116\\
9.46	0.000772735212750795\\
9.47	0.000772735418978529\\
9.48	0.000772735625288356\\
9.49	0.00077273583168031\\
9.5	0.000772736038154429\\
9.51	0.000772736244710748\\
9.52	0.000772736451349305\\
9.53	0.000772736658070134\\
9.54	0.000772736864873273\\
9.55	0.000772737071758758\\
9.56	0.000772737278726624\\
9.57	0.000772737485776909\\
9.58	0.000772737692909647\\
9.59	0.000772737900124876\\
9.6	0.000772738107422632\\
9.61	0.00077273831480295\\
9.62	0.000772738522265868\\
9.63	0.000772738729811422\\
9.64	0.000772738937439647\\
9.65	0.000772739145150581\\
9.66	0.000772739352944259\\
9.67	0.000772739560820718\\
9.68	0.000772739768779993\\
9.69	0.000772739976822121\\
9.7	0.00077274018494714\\
9.71	0.000772740393155084\\
9.72	0.000772740601445989\\
9.73	0.000772740809819894\\
9.74	0.000772741018276834\\
9.75	0.000772741226816843\\
9.76	0.000772741435439961\\
9.77	0.000772741644146223\\
9.78	0.000772741852935665\\
9.79	0.000772742061808324\\
9.8	0.000772742270764234\\
9.81	0.000772742479803433\\
9.82	0.000772742688925958\\
9.83	0.000772742898131845\\
9.84	0.000772743107421129\\
9.85	0.000772743316793847\\
9.86	0.000772743526250038\\
9.87	0.000772743735789734\\
9.88	0.000772743945412975\\
9.89	0.000772744155119794\\
9.9	0.000772744364910229\\
9.91	0.000772744574784316\\
9.92	0.000772744784742091\\
9.93	0.000772744994783593\\
9.94	0.000772745204908856\\
9.95	0.000772745415117916\\
9.96	0.000772745625410808\\
9.97	0.000772745835787571\\
9.98	0.000772746046248241\\
9.99	0.000772746256792854\\
10	0.000772746467421445\\
10.01	0.000772746678134052\\
10.02	0.00077274688893071\\
10.03	0.000772747099811455\\
10.04	0.000772747310776326\\
10.05	0.000772747521825356\\
10.06	0.000772747732958583\\
10.07	0.000772747944176044\\
10.08	0.000772748155477773\\
10.09	0.000772748366863807\\
10.1	0.000772748578334183\\
10.11	0.000772748789888936\\
10.12	0.000772749001528104\\
10.13	0.000772749213251721\\
10.14	0.000772749425059826\\
10.15	0.000772749636952452\\
10.16	0.000772749848929638\\
10.17	0.000772750060991419\\
10.18	0.000772750273137831\\
10.19	0.000772750485368909\\
10.2	0.000772750697684692\\
10.21	0.000772750910085215\\
10.22	0.000772751122570513\\
10.23	0.000772751335140622\\
10.24	0.00077275154779558\\
10.25	0.000772751760535421\\
10.26	0.000772751973360183\\
10.27	0.0007727521862699\\
10.28	0.000772752399264609\\
10.29	0.000772752612344347\\
10.3	0.000772752825509149\\
10.31	0.000772753038759051\\
10.32	0.000772753252094088\\
10.33	0.000772753465514298\\
10.34	0.000772753679019716\\
10.35	0.000772753892610379\\
10.36	0.000772754106286321\\
10.37	0.000772754320047577\\
10.38	0.000772754533894186\\
10.39	0.000772754747826183\\
10.4	0.000772754961843604\\
10.41	0.000772755175946483\\
10.42	0.000772755390134857\\
10.43	0.000772755604408761\\
10.44	0.000772755818768232\\
10.45	0.000772756033213305\\
10.46	0.000772756247744016\\
10.47	0.000772756462360401\\
10.48	0.000772756677062496\\
10.49	0.000772756891850334\\
10.5	0.000772757106723955\\
10.51	0.000772757321683391\\
10.52	0.000772757536728678\\
10.53	0.000772757751859853\\
10.54	0.00077275796707695\\
10.55	0.000772758182380007\\
10.56	0.000772758397769057\\
10.57	0.000772758613244136\\
10.58	0.00077275882880528\\
10.59	0.000772759044452523\\
10.6	0.000772759260185902\\
10.61	0.000772759476005452\\
10.62	0.000772759691911207\\
10.63	0.000772759907903204\\
10.64	0.000772760123981477\\
10.65	0.000772760340146061\\
10.66	0.000772760556396994\\
10.67	0.000772760772734308\\
10.68	0.000772760989158039\\
10.69	0.000772761205668222\\
10.7	0.000772761422264892\\
10.71	0.000772761638948084\\
10.72	0.000772761855717834\\
10.73	0.000772762072574176\\
10.74	0.000772762289517144\\
10.75	0.000772762506546775\\
10.76	0.000772762723663101\\
10.77	0.000772762940866159\\
10.78	0.000772763158155984\\
10.79	0.000772763375532609\\
10.8	0.00077276359299607\\
10.81	0.0007727638105464\\
10.82	0.000772764028183635\\
10.83	0.000772764245907808\\
10.84	0.000772764463718956\\
10.85	0.000772764681617112\\
10.86	0.00077276489960231\\
10.87	0.000772765117674585\\
10.88	0.000772765335833972\\
10.89	0.000772765554080504\\
10.9	0.000772765772414214\\
10.91	0.00077276599083514\\
10.92	0.000772766209343313\\
10.93	0.000772766427938767\\
10.94	0.000772766646621537\\
10.95	0.000772766865391657\\
10.96	0.000772767084249162\\
10.97	0.000772767303194083\\
10.98	0.000772767522226457\\
10.99	0.000772767741346315\\
11	0.000772767960553693\\
11.01	0.000772768179848622\\
11.02	0.000772768399231138\\
11.03	0.000772768618701273\\
11.04	0.000772768838259061\\
11.05	0.000772769057904537\\
11.06	0.000772769277637732\\
11.07	0.00077276949745868\\
11.08	0.000772769717367413\\
11.09	0.000772769937363967\\
11.1	0.000772770157448375\\
11.11	0.000772770377620666\\
11.12	0.000772770597880877\\
11.13	0.000772770818229041\\
11.14	0.000772771038665188\\
11.15	0.000772771259189352\\
11.16	0.000772771479801567\\
11.17	0.000772771700501863\\
11.18	0.000772771921290275\\
11.19	0.000772772142166835\\
11.2	0.000772772363131575\\
11.21	0.000772772584184527\\
11.22	0.000772772805325724\\
11.23	0.000772773026555198\\
11.24	0.000772773247872982\\
11.25	0.000772773469279106\\
11.26	0.000772773690773604\\
11.27	0.000772773912356507\\
11.28	0.000772774134027848\\
11.29	0.000772774355787658\\
11.3	0.000772774577635969\\
11.31	0.000772774799572811\\
11.32	0.000772775021598218\\
11.33	0.00077277524371222\\
11.34	0.00077277546591485\\
11.35	0.000772775688206136\\
11.36	0.000772775910586112\\
11.37	0.000772776133054808\\
11.38	0.000772776355612256\\
11.39	0.000772776578258487\\
11.4	0.000772776800993531\\
11.41	0.000772777023817419\\
11.42	0.000772777246730182\\
11.43	0.000772777469731851\\
11.44	0.000772777692822456\\
11.45	0.000772777916002028\\
11.46	0.000772778139270597\\
11.47	0.000772778362628193\\
11.48	0.000772778586074847\\
11.49	0.000772778809610587\\
11.5	0.000772779033235445\\
11.51	0.000772779256949451\\
11.52	0.000772779480752633\\
11.53	0.000772779704645024\\
11.54	0.000772779928626651\\
11.55	0.000772780152697543\\
11.56	0.000772780376857731\\
11.57	0.000772780601107245\\
11.58	0.000772780825446111\\
11.59	0.000772781049874362\\
11.6	0.000772781274392025\\
11.61	0.000772781498999129\\
11.62	0.000772781723695702\\
11.63	0.000772781948481776\\
11.64	0.000772782173357377\\
11.65	0.000772782398322534\\
11.66	0.000772782623377277\\
11.67	0.000772782848521631\\
11.68	0.000772783073755626\\
11.69	0.00077278329907929\\
11.7	0.000772783524492652\\
11.71	0.000772783749995738\\
11.72	0.000772783975588578\\
11.73	0.000772784201271198\\
11.74	0.000772784427043626\\
11.75	0.000772784652905891\\
11.76	0.000772784878858018\\
11.77	0.000772785104900035\\
11.78	0.000772785331031969\\
11.79	0.000772785557253849\\
11.8	0.000772785783565698\\
11.81	0.000772786009967547\\
11.82	0.000772786236459421\\
11.83	0.000772786463041346\\
11.84	0.000772786689713348\\
11.85	0.000772786916475456\\
11.86	0.000772787143327692\\
11.87	0.000772787370270086\\
11.88	0.000772787597302663\\
11.89	0.000772787824425449\\
11.9	0.000772788051638468\\
11.91	0.000772788278941746\\
11.92	0.00077278850633531\\
11.93	0.000772788733819186\\
11.94	0.000772788961393399\\
11.95	0.000772789189057972\\
11.96	0.000772789416812933\\
11.97	0.000772789644658304\\
11.98	0.000772789872594111\\
11.99	0.000772790100620381\\
12	0.000772790328737135\\
12.01	0.000772790556944401\\
12.02	0.000772790785242202\\
12.03	0.00077279101363056\\
12.04	0.000772791242109503\\
12.05	0.000772791470679051\\
12.06	0.000772791699339232\\
12.07	0.000772791928090068\\
12.08	0.000772792156931582\\
12.09	0.000772792385863799\\
12.1	0.00077279261488674\\
12.11	0.000772792844000432\\
12.12	0.000772793073204894\\
12.13	0.000772793302500152\\
12.14	0.00077279353188623\\
12.15	0.000772793761363149\\
12.16	0.000772793990930932\\
12.17	0.000772794220589601\\
12.18	0.000772794450339179\\
12.19	0.000772794680179689\\
12.2	0.000772794910111153\\
12.21	0.000772795140133594\\
12.22	0.000772795370247032\\
12.23	0.000772795600451492\\
12.24	0.000772795830746993\\
12.25	0.000772796061133559\\
12.26	0.000772796291611212\\
12.27	0.00077279652217997\\
12.28	0.000772796752839859\\
12.29	0.000772796983590897\\
12.3	0.000772797214433107\\
12.31	0.00077279744536651\\
12.32	0.000772797676391126\\
12.33	0.000772797907506978\\
12.34	0.000772798138714085\\
12.35	0.00077279837001247\\
12.36	0.000772798601402152\\
12.37	0.000772798832883151\\
12.38	0.000772799064455489\\
12.39	0.000772799296119186\\
12.4	0.000772799527874261\\
12.41	0.000772799759720737\\
12.42	0.000772799991658633\\
12.43	0.000772800223687968\\
12.44	0.000772800455808763\\
12.45	0.000772800688021038\\
12.46	0.000772800920324813\\
12.47	0.000772801152720107\\
12.48	0.000772801385206941\\
12.49	0.000772801617785333\\
12.5	0.000772801850455303\\
12.51	0.000772802083216872\\
12.52	0.000772802316070059\\
12.53	0.000772802549014882\\
12.54	0.000772802782051361\\
12.55	0.000772803015179516\\
12.56	0.000772803248399366\\
12.57	0.000772803481710931\\
12.58	0.000772803715114227\\
12.59	0.000772803948609277\\
12.6	0.000772804182196099\\
12.61	0.00077280441587471\\
12.62	0.000772804649645131\\
12.63	0.000772804883507382\\
12.64	0.000772805117461479\\
12.65	0.000772805351507444\\
12.66	0.000772805585645294\\
12.67	0.000772805819875049\\
12.68	0.000772806054196728\\
12.69	0.000772806288610348\\
12.7	0.00077280652311593\\
12.71	0.000772806757713492\\
12.72	0.000772806992403053\\
12.73	0.000772807227184633\\
12.74	0.00077280746205825\\
12.75	0.000772807697023923\\
12.76	0.000772807932081673\\
12.77	0.000772808167231515\\
12.78	0.000772808402473471\\
12.79	0.00077280863780756\\
12.8	0.000772808873233801\\
12.81	0.000772809108752213\\
12.82	0.000772809344362815\\
12.83	0.000772809580065627\\
12.84	0.000772809815860667\\
12.85	0.000772810051747955\\
12.86	0.000772810287727512\\
12.87	0.000772810523799357\\
12.88	0.000772810759963508\\
12.89	0.000772810996219985\\
12.9	0.000772811232568811\\
12.91	0.000772811469010004\\
12.92	0.000772811705543585\\
12.93	0.000772811942169573\\
12.94	0.000772812178887989\\
12.95	0.000772812415698853\\
12.96	0.000772812652602187\\
12.97	0.000772812889598011\\
12.98	0.000772813126686346\\
12.99	0.000772813363867215\\
13	0.000772813601140636\\
13.01	0.000772813838506633\\
13.02	0.000772814075965227\\
13.03	0.00077281431351644\\
13.04	0.000772814551160293\\
13.05	0.000772814788896809\\
13.06	0.000772815026726012\\
13.07	0.000772815264647921\\
13.08	0.000772815502662562\\
13.09	0.000772815740769957\\
13.1	0.00077281597897013\\
13.11	0.000772816217263104\\
13.12	0.000772816455648905\\
13.13	0.000772816694127555\\
13.14	0.000772816932699079\\
13.15	0.000772817171363502\\
13.16	0.000772817410120847\\
13.17	0.000772817648971143\\
13.18	0.000772817887914412\\
13.19	0.000772818126950684\\
13.2	0.000772818366079982\\
13.21	0.000772818605302334\\
13.22	0.000772818844617768\\
13.23	0.000772819084026309\\
13.24	0.000772819323527988\\
13.25	0.000772819563122831\\
13.26	0.000772819802810866\\
13.27	0.000772820042592123\\
13.28	0.000772820282466632\\
13.29	0.000772820522434421\\
13.3	0.00077282076249552\\
13.31	0.000772821002649962\\
13.32	0.000772821242897776\\
13.33	0.000772821483238996\\
13.34	0.00077282172367365\\
13.35	0.000772821964201774\\
13.36	0.000772822204823399\\
13.37	0.000772822445538559\\
13.38	0.000772822686347287\\
13.39	0.00077282292724962\\
13.4	0.000772823168245592\\
13.41	0.000772823409335236\\
13.42	0.000772823650518591\\
13.43	0.000772823891795692\\
13.44	0.000772824133166577\\
13.45	0.000772824374631283\\
13.46	0.000772824616189849\\
13.47	0.000772824857842315\\
13.48	0.000772825099588717\\
13.49	0.000772825341429098\\
13.5	0.000772825583363499\\
13.51	0.000772825825391961\\
13.52	0.000772826067514524\\
13.53	0.000772826309731233\\
13.54	0.000772826552042132\\
13.55	0.000772826794447262\\
13.56	0.00077282703694667\\
13.57	0.000772827279540401\\
13.58	0.000772827522228502\\
13.59	0.000772827765011018\\
13.6	0.000772828007887999\\
13.61	0.000772828250859491\\
13.62	0.000772828493925544\\
13.63	0.000772828737086208\\
13.64	0.000772828980341533\\
13.65	0.000772829223691571\\
13.66	0.000772829467136374\\
13.67	0.000772829710675995\\
13.68	0.000772829954310486\\
13.69	0.000772830198039903\\
13.7	0.000772830441864301\\
13.71	0.000772830685783736\\
13.72	0.000772830929798265\\
13.73	0.000772831173907945\\
13.74	0.000772831418112836\\
13.75	0.000772831662412996\\
13.76	0.000772831906808486\\
13.77	0.000772832151299366\\
13.78	0.000772832395885699\\
13.79	0.000772832640567546\\
13.8	0.000772832885344974\\
13.81	0.000772833130218043\\
13.82	0.000772833375186823\\
13.83	0.000772833620251379\\
13.84	0.000772833865411776\\
13.85	0.000772834110668082\\
13.86	0.000772834356020369\\
13.87	0.000772834601468705\\
13.88	0.000772834847013159\\
13.89	0.000772835092653806\\
13.9	0.000772835338390716\\
13.91	0.000772835584223962\\
13.92	0.000772835830153621\\
13.93	0.000772836076179765\\
13.94	0.000772836322302472\\
13.95	0.000772836568521817\\
13.96	0.00077283681483788\\
13.97	0.000772837061250739\\
13.98	0.000772837307760474\\
13.99	0.000772837554367163\\
14	0.00077283780107089\\
14.01	0.000772838047871736\\
14.02	0.000772838294769783\\
14.03	0.000772838541765117\\
14.04	0.000772838788857821\\
14.05	0.000772839036047981\\
14.06	0.000772839283335684\\
14.07	0.000772839530721017\\
14.08	0.000772839778204066\\
14.09	0.000772840025784923\\
14.1	0.000772840273463676\\
14.11	0.000772840521240414\\
14.12	0.00077284076911523\\
14.13	0.000772841017088216\\
14.14	0.000772841265159464\\
14.15	0.000772841513329067\\
14.16	0.000772841761597119\\
14.17	0.000772842009963715\\
14.18	0.000772842258428951\\
14.19	0.000772842506992923\\
14.2	0.000772842755655727\\
14.21	0.00077284300441746\\
14.22	0.000772843253278222\\
14.23	0.000772843502238109\\
14.24	0.000772843751297223\\
14.25	0.000772844000455662\\
14.26	0.000772844249713526\\
14.27	0.000772844499070915\\
14.28	0.000772844748527931\\
14.29	0.000772844998084676\\
14.3	0.000772845247741251\\
14.31	0.000772845497497759\\
14.32	0.000772845747354303\\
14.33	0.000772845997310985\\
14.34	0.00077284624736791\\
14.35	0.00077284649752518\\
14.36	0.0007728467477829\\
14.37	0.000772846998141175\\
14.38	0.000772847248600106\\
14.39	0.0007728474991598\\
14.4	0.000772847749820362\\
14.41	0.000772848000581897\\
14.42	0.000772848251444507\\
14.43	0.000772848502408299\\
14.44	0.000772848753473377\\
14.45	0.000772849004639847\\
14.46	0.000772849255907811\\
14.47	0.000772849507277377\\
14.48	0.000772849758748646\\
14.49	0.000772850010321725\\
14.5	0.000772850261996715\\
14.51	0.000772850513773722\\
14.52	0.000772850765652849\\
14.53	0.000772851017634198\\
14.54	0.000772851269717873\\
14.55	0.000772851521903975\\
14.56	0.000772851774192606\\
14.57	0.000772852026583868\\
14.58	0.000772852279077862\\
14.59	0.000772852531674688\\
14.6	0.000772852784374445\\
14.61	0.000772853037177232\\
14.62	0.000772853290083147\\
14.63	0.000772853543092289\\
14.64	0.000772853796204754\\
14.65	0.000772854049420638\\
14.66	0.000772854302740036\\
14.67	0.000772854556163044\\
14.68	0.000772854809689754\\
14.69	0.00077285506332026\\
14.7	0.000772855317054653\\
14.71	0.000772855570893023\\
14.72	0.000772855824835461\\
14.73	0.000772856078882058\\
14.74	0.000772856333032897\\
14.75	0.000772856587288068\\
14.76	0.000772856841647655\\
14.77	0.000772857096111744\\
14.78	0.000772857350680417\\
14.79	0.000772857605353757\\
14.8	0.000772857860131844\\
14.81	0.000772858115014759\\
14.82	0.000772858370002582\\
14.83	0.000772858625095388\\
14.84	0.000772858880293256\\
14.85	0.000772859135596261\\
14.86	0.000772859391004474\\
14.87	0.000772859646517972\\
14.88	0.000772859902136824\\
14.89	0.000772860157861101\\
14.9	0.000772860413690873\\
14.91	0.000772860669626209\\
14.92	0.000772860925667173\\
14.93	0.000772861181813835\\
14.94	0.000772861438066257\\
14.95	0.000772861694424504\\
14.96	0.000772861950888638\\
14.97	0.000772862207458723\\
14.98	0.000772862464134818\\
14.99	0.000772862720916985\\
15	0.000772862977805282\\
15.01	0.000772863234799767\\
15.02	0.000772863491900496\\
15.03	0.000772863749107528\\
15.04	0.000772864006420917\\
15.05	0.00077286426384072\\
15.06	0.000772864521366991\\
15.07	0.000772864778999783\\
15.08	0.000772865036739151\\
15.09	0.000772865294585147\\
15.1	0.000772865552537826\\
15.11	0.000772865810597238\\
15.12	0.000772866068763435\\
15.13	0.00077286632703647\\
15.14	0.000772866585416395\\
15.15	0.00077286684390326\\
15.16	0.000772867102497118\\
15.17	0.000772867361198019\\
15.18	0.000772867620006015\\
15.19	0.000772867878921158\\
15.2	0.000772868137943498\\
15.21	0.000772868397073087\\
15.22	0.000772868656309978\\
15.23	0.00077286891565422\\
15.24	0.000772869175105866\\
15.25	0.000772869434664966\\
15.26	0.000772869694331574\\
15.27	0.000772869954105738\\
15.28	0.000772870213987512\\
15.29	0.000772870473976948\\
15.3	0.000772870734074097\\
15.31	0.00077287099427901\\
15.32	0.000772871254591738\\
15.33	0.000772871515012334\\
15.34	0.00077287177554085\\
15.35	0.000772872036177338\\
15.36	0.000772872296921849\\
15.37	0.000772872557774435\\
15.38	0.000772872818735148\\
15.39	0.00077287307980404\\
15.4	0.000772873340981164\\
15.41	0.000772873602266569\\
15.42	0.000772873863660309\\
15.43	0.000772874125162436\\
15.44	0.000772874386773002\\
15.45	0.000772874648492058\\
15.46	0.000772874910319658\\
15.47	0.000772875172255853\\
15.48	0.000772875434300695\\
15.49	0.000772875696454237\\
15.5	0.000772875958716531\\
15.51	0.00077287622108763\\
15.52	0.000772876483567584\\
15.53	0.000772876746156448\\
15.54	0.000772877008854271\\
15.55	0.000772877271661108\\
15.56	0.000772877534577013\\
15.57	0.000772877797602035\\
15.58	0.000772878060736228\\
15.59	0.000772878323979645\\
15.6	0.000772878587332337\\
15.61	0.000772878850794358\\
15.62	0.000772879114365761\\
15.63	0.000772879378046596\\
15.64	0.000772879641836918\\
15.65	0.000772879905736778\\
15.66	0.000772880169746231\\
15.67	0.000772880433865329\\
15.68	0.000772880698094124\\
15.69	0.000772880962432668\\
15.7	0.000772881226881017\\
15.71	0.00077288149143922\\
15.72	0.000772881756107333\\
15.73	0.000772882020885407\\
15.74	0.000772882285773496\\
15.75	0.000772882550771653\\
15.76	0.00077288281587993\\
15.77	0.000772883081098382\\
15.78	0.000772883346427059\\
15.79	0.000772883611866016\\
15.8	0.000772883877415307\\
15.81	0.000772884143074984\\
15.82	0.0007728844088451\\
15.83	0.000772884674725709\\
15.84	0.000772884940716865\\
15.85	0.00077288520681862\\
15.86	0.000772885473031028\\
15.87	0.000772885739354141\\
15.88	0.000772886005788015\\
15.89	0.0007728862723327\\
15.9	0.000772886538988252\\
15.91	0.000772886805754724\\
15.92	0.000772887072632169\\
15.93	0.000772887339620641\\
15.94	0.000772887606720193\\
15.95	0.000772887873930879\\
15.96	0.000772888141252754\\
15.97	0.000772888408685869\\
15.98	0.000772888676230279\\
15.99	0.00077288894388604\\
16	0.000772889211653202\\
16.01	0.000772889479531821\\
16.02	0.000772889747521949\\
16.03	0.000772890015623643\\
16.04	0.000772890283836954\\
16.05	0.000772890552161936\\
16.06	0.000772890820598645\\
16.07	0.000772891089147134\\
16.08	0.000772891357807455\\
16.09	0.000772891626579664\\
16.1	0.000772891895463817\\
16.11	0.000772892164459965\\
16.12	0.000772892433568163\\
16.13	0.000772892702788465\\
16.14	0.000772892972120926\\
16.15	0.000772893241565601\\
16.16	0.000772893511122542\\
16.17	0.000772893780791805\\
16.18	0.000772894050573443\\
16.19	0.000772894320467512\\
16.2	0.000772894590474063\\
16.21	0.000772894860593155\\
16.22	0.00077289513082484\\
16.23	0.000772895401169172\\
16.24	0.000772895671626207\\
16.25	0.000772895942195999\\
16.26	0.000772896212878602\\
16.27	0.000772896483674072\\
16.28	0.000772896754582461\\
16.29	0.000772897025603826\\
16.3	0.000772897296738222\\
16.31	0.000772897567985703\\
16.32	0.000772897839346323\\
16.33	0.000772898110820137\\
16.34	0.000772898382407201\\
16.35	0.000772898654107568\\
16.36	0.000772898925921294\\
16.37	0.000772899197848434\\
16.38	0.000772899469889043\\
16.39	0.000772899742043176\\
16.4	0.000772900014310888\\
16.41	0.000772900286692234\\
16.42	0.000772900559187269\\
16.43	0.000772900831796049\\
16.44	0.000772901104518628\\
16.45	0.00077290137735506\\
16.46	0.000772901650305403\\
16.47	0.00077290192336971\\
16.48	0.000772902196548037\\
16.49	0.00077290246984044\\
16.5	0.000772902743246973\\
16.51	0.000772903016767693\\
16.52	0.000772903290402655\\
16.53	0.000772903564151913\\
16.54	0.000772903838015523\\
16.55	0.000772904111993543\\
16.56	0.000772904386086026\\
16.57	0.000772904660293026\\
16.58	0.000772904934614602\\
16.59	0.000772905209050809\\
16.6	0.000772905483601701\\
16.61	0.000772905758267335\\
16.62	0.000772906033047766\\
16.63	0.00077290630794305\\
16.64	0.000772906582953243\\
16.65	0.000772906858078402\\
16.66	0.00077290713331858\\
16.67	0.000772907408673834\\
16.68	0.000772907684144222\\
16.69	0.000772907959729797\\
16.7	0.000772908235430617\\
16.71	0.000772908511246737\\
16.72	0.000772908787178212\\
16.73	0.000772909063225101\\
16.74	0.000772909339387458\\
16.75	0.00077290961566534\\
16.76	0.000772909892058802\\
16.77	0.000772910168567903\\
16.78	0.000772910445192695\\
16.79	0.000772910721933237\\
16.8	0.000772910998789586\\
16.81	0.000772911275761796\\
16.82	0.000772911552849925\\
16.83	0.000772911830054028\\
16.84	0.000772912107374164\\
16.85	0.000772912384810387\\
16.86	0.000772912662362756\\
16.87	0.000772912940031324\\
16.88	0.000772913217816151\\
16.89	0.00077291349571729\\
16.9	0.0007729137737348\\
16.91	0.000772914051868738\\
16.92	0.000772914330119159\\
16.93	0.000772914608486122\\
16.94	0.000772914886969681\\
16.95	0.000772915165569895\\
16.96	0.00077291544428682\\
16.97	0.000772915723120512\\
16.98	0.000772916002071029\\
16.99	0.000772916281138429\\
17	0.000772916560322768\\
17.01	0.000772916839624101\\
17.02	0.000772917119042487\\
17.03	0.000772917398577983\\
17.04	0.000772917678230645\\
17.05	0.000772917958000531\\
17.06	0.000772918237887698\\
17.07	0.000772918517892204\\
17.08	0.000772918798014105\\
17.09	0.00077291907825346\\
17.1	0.000772919358610324\\
17.11	0.000772919639084756\\
17.12	0.000772919919676813\\
17.13	0.000772920200386551\\
17.14	0.000772920481214029\\
17.15	0.000772920762159303\\
17.16	0.000772921043222432\\
17.17	0.000772921324403472\\
17.18	0.000772921605702482\\
17.19	0.00077292188711952\\
17.2	0.000772922168654641\\
17.21	0.000772922450307906\\
17.22	0.00077292273207937\\
17.23	0.000772923013969092\\
17.24	0.000772923295977128\\
17.25	0.000772923578103539\\
17.26	0.000772923860348381\\
17.27	0.000772924142711712\\
17.28	0.000772924425193588\\
17.29	0.000772924707794069\\
17.3	0.000772924990513213\\
17.31	0.000772925273351077\\
17.32	0.000772925556307719\\
17.33	0.000772925839383199\\
17.34	0.000772926122577573\\
17.35	0.000772926405890899\\
17.36	0.000772926689323238\\
17.37	0.000772926972874644\\
17.38	0.000772927256545177\\
17.39	0.000772927540334896\\
17.4	0.00077292782424386\\
17.41	0.000772928108272125\\
17.42	0.00077292839241975\\
17.43	0.000772928676686794\\
17.44	0.000772928961073316\\
17.45	0.000772929245579373\\
17.46	0.000772929530205024\\
17.47	0.000772929814950328\\
17.48	0.000772930099815343\\
17.49	0.000772930384800129\\
17.5	0.000772930669904743\\
17.51	0.000772930955129243\\
17.52	0.000772931240473689\\
17.53	0.00077293152593814\\
17.54	0.000772931811522654\\
17.55	0.000772932097227289\\
17.56	0.000772932383052106\\
17.57	0.000772932668997162\\
17.58	0.000772932955062516\\
17.59	0.000772933241248228\\
17.6	0.000772933527554356\\
17.61	0.000772933813980959\\
17.62	0.000772934100528096\\
17.63	0.000772934387195827\\
17.64	0.00077293467398421\\
17.65	0.000772934960893305\\
17.66	0.00077293524792317\\
17.67	0.000772935535073866\\
17.68	0.000772935822345451\\
17.69	0.000772936109737984\\
17.7	0.000772936397251525\\
17.71	0.000772936684886132\\
17.72	0.000772936972641866\\
17.73	0.000772937260518785\\
17.74	0.000772937548516948\\
17.75	0.000772937836636418\\
17.76	0.00077293812487725\\
17.77	0.000772938413239506\\
17.78	0.000772938701723245\\
17.79	0.000772938990328527\\
17.8	0.000772939279055411\\
17.81	0.000772939567903957\\
17.82	0.000772939856874225\\
17.83	0.000772940145966274\\
17.84	0.000772940435180164\\
17.85	0.000772940724515956\\
17.86	0.000772941013973709\\
17.87	0.000772941303553482\\
17.88	0.000772941593255335\\
17.89	0.000772941883079329\\
17.9	0.000772942173025523\\
17.91	0.000772942463093979\\
17.92	0.000772942753284754\\
17.93	0.00077294304359791\\
17.94	0.000772943334033506\\
17.95	0.000772943624591602\\
17.96	0.000772943915272261\\
17.97	0.00077294420607554\\
17.98	0.0007729444970015\\
17.99	0.000772944788050203\\
18	0.000772945079221706\\
18.01	0.000772945370516073\\
18.02	0.000772945661933363\\
18.03	0.000772945953473636\\
18.04	0.000772946245136951\\
18.05	0.00077294653692337\\
18.06	0.000772946828832955\\
18.07	0.000772947120865765\\
18.08	0.000772947413021859\\
18.09	0.000772947705301301\\
18.1	0.00077294799770415\\
18.11	0.000772948290230465\\
18.12	0.000772948582880311\\
18.13	0.000772948875653745\\
18.14	0.000772949168550828\\
18.15	0.000772949461571623\\
18.16	0.000772949754716189\\
18.17	0.000772950047984588\\
18.18	0.00077295034137688\\
18.19	0.000772950634893126\\
18.2	0.000772950928533387\\
18.21	0.000772951222297725\\
18.22	0.000772951516186201\\
18.23	0.000772951810198874\\
18.24	0.000772952104335808\\
18.25	0.000772952398597062\\
18.26	0.000772952692982699\\
18.27	0.000772952987492779\\
18.28	0.000772953282127363\\
18.29	0.000772953576886514\\
18.3	0.000772953871770293\\
18.31	0.00077295416677876\\
18.32	0.000772954461911977\\
18.33	0.000772954757170005\\
18.34	0.000772955052552907\\
18.35	0.000772955348060744\\
18.36	0.000772955643693576\\
18.37	0.000772955939451467\\
18.38	0.000772956235334476\\
18.39	0.000772956531342666\\
18.4	0.0007729568274761\\
18.41	0.000772957123734838\\
18.42	0.000772957420118941\\
18.43	0.000772957716628473\\
18.44	0.000772958013263495\\
18.45	0.000772958310024068\\
18.46	0.000772958606910255\\
18.47	0.000772958903922118\\
18.48	0.000772959201059719\\
18.49	0.000772959498323119\\
18.5	0.000772959795712382\\
18.51	0.000772960093227568\\
18.52	0.00077296039086874\\
18.53	0.000772960688635959\\
18.54	0.000772960986529288\\
18.55	0.00077296128454879\\
18.56	0.000772961582694527\\
18.57	0.000772961880966561\\
18.58	0.000772962179364954\\
18.59	0.000772962477889768\\
18.6	0.000772962776541066\\
18.61	0.000772963075318911\\
18.62	0.000772963374223366\\
18.63	0.000772963673254492\\
18.64	0.000772963972412352\\
18.65	0.000772964271697008\\
18.66	0.000772964571108524\\
18.67	0.000772964870646962\\
18.68	0.000772965170312384\\
18.69	0.000772965470104855\\
18.7	0.000772965770024435\\
18.71	0.000772966070071188\\
18.72	0.000772966370245176\\
18.73	0.000772966670546464\\
18.74	0.000772966970975113\\
18.75	0.000772967271531186\\
18.76	0.000772967572214747\\
18.77	0.000772967873025858\\
18.78	0.000772968173964583\\
18.79	0.000772968475030984\\
18.8	0.000772968776225125\\
18.81	0.000772969077547068\\
18.82	0.000772969378996877\\
18.83	0.000772969680574616\\
18.84	0.000772969982280347\\
18.85	0.000772970284114135\\
18.86	0.000772970586076041\\
18.87	0.00077297088816613\\
18.88	0.000772971190384466\\
18.89	0.00077297149273111\\
18.9	0.000772971795206127\\
18.91	0.00077297209780958\\
18.92	0.000772972400541532\\
18.93	0.000772972703402049\\
18.94	0.000772973006391193\\
18.95	0.000772973309509027\\
18.96	0.000772973612755617\\
18.97	0.000772973916131023\\
18.98	0.000772974219635313\\
18.99	0.000772974523268547\\
19	0.000772974827030792\\
19.01	0.00077297513092211\\
19.02	0.000772975434942565\\
19.03	0.000772975739092222\\
19.04	0.000772976043371143\\
19.05	0.000772976347779394\\
19.06	0.000772976652317039\\
19.07	0.000772976956984141\\
19.08	0.000772977261780765\\
19.09	0.000772977566706974\\
19.1	0.000772977871762833\\
19.11	0.000772978176948406\\
19.12	0.000772978482263759\\
19.13	0.000772978787708954\\
19.14	0.000772979093284056\\
19.15	0.00077297939898913\\
19.16	0.000772979704824239\\
19.17	0.000772980010789449\\
19.18	0.000772980316884823\\
19.19	0.000772980623110427\\
19.2	0.000772980929466326\\
19.21	0.000772981235952583\\
19.22	0.000772981542569263\\
19.23	0.00077298184931643\\
19.24	0.000772982156194152\\
19.25	0.00077298246320249\\
19.26	0.000772982770341511\\
19.27	0.000772983077611278\\
19.28	0.000772983385011857\\
19.29	0.000772983692543314\\
19.3	0.000772984000205711\\
19.31	0.000772984307999115\\
19.32	0.000772984615923591\\
19.33	0.000772984923979203\\
19.34	0.000772985232166018\\
19.35	0.000772985540484099\\
19.36	0.000772985848933512\\
19.37	0.000772986157514323\\
19.38	0.000772986466226595\\
19.39	0.000772986775070397\\
19.4	0.00077298708404579\\
19.41	0.000772987393152843\\
19.42	0.00077298770239162\\
19.43	0.000772988011762185\\
19.44	0.000772988321264606\\
19.45	0.000772988630898946\\
19.46	0.000772988940665274\\
19.47	0.000772989250563651\\
19.48	0.000772989560594146\\
19.49	0.000772989870756823\\
19.5	0.000772990181051748\\
19.51	0.000772990491478987\\
19.52	0.000772990802038607\\
19.53	0.000772991112730671\\
19.54	0.000772991423555247\\
19.55	0.000772991734512399\\
19.56	0.000772992045602196\\
19.57	0.0007729923568247\\
19.58	0.00077299266817998\\
19.59	0.000772992979668102\\
19.6	0.00077299329128913\\
19.61	0.000772993603043131\\
19.62	0.000772993914930171\\
19.63	0.000772994226950317\\
19.64	0.000772994539103634\\
19.65	0.00077299485139019\\
19.66	0.000772995163810049\\
19.67	0.000772995476363279\\
19.68	0.000772995789049945\\
19.69	0.000772996101870116\\
19.7	0.000772996414823856\\
19.71	0.000772996727911231\\
19.72	0.00077299704113231\\
19.73	0.000772997354487159\\
19.74	0.000772997667975844\\
19.75	0.00077299798159843\\
19.76	0.000772998295354986\\
19.77	0.000772998609245576\\
19.78	0.00077299892327027\\
19.79	0.000772999237429134\\
19.8	0.000772999551722234\\
19.81	0.000772999866149636\\
19.82	0.000773000180711409\\
19.83	0.000773000495407617\\
19.84	0.000773000810238331\\
19.85	0.000773001125203614\\
19.86	0.000773001440303536\\
19.87	0.000773001755538162\\
19.88	0.000773002070907561\\
19.89	0.000773002386411798\\
19.9	0.000773002702050942\\
19.91	0.000773003017825059\\
19.92	0.000773003333734217\\
19.93	0.000773003649778484\\
19.94	0.000773003965957926\\
19.95	0.000773004282272612\\
19.96	0.000773004598722609\\
19.97	0.000773004915307983\\
19.98	0.000773005232028803\\
19.99	0.000773005548885135\\
20	0.000773005865877049\\
20.01	0.000773006183004609\\
20.02	0.000773006500267886\\
20.03	0.000773006817666946\\
20.04	0.000773007135201858\\
20.05	0.000773007452872688\\
20.06	0.000773007770679505\\
20.07	0.000773008088622379\\
20.08	0.000773008406701374\\
20.09	0.000773008724916559\\
20.1	0.000773009043268004\\
20.11	0.000773009361755774\\
20.12	0.00077300968037994\\
20.13	0.000773009999140569\\
20.14	0.000773010318037729\\
20.15	0.000773010637071489\\
20.16	0.000773010956241915\\
20.17	0.000773011275549077\\
20.18	0.000773011594993043\\
20.19	0.000773011914573881\\
20.2	0.000773012234291659\\
20.21	0.000773012554146447\\
20.22	0.000773012874138314\\
20.23	0.000773013194267326\\
20.24	0.000773013514533552\\
20.25	0.000773013834937062\\
20.26	0.000773014155477923\\
20.27	0.000773014476156205\\
20.28	0.000773014796971977\\
20.29	0.000773015117925307\\
20.3	0.000773015439016263\\
20.31	0.000773015760244914\\
20.32	0.00077301608161133\\
20.33	0.000773016403115579\\
20.34	0.000773016724757731\\
20.35	0.000773017046537854\\
20.36	0.000773017368456017\\
20.37	0.000773017690512289\\
20.38	0.00077301801270674\\
20.39	0.000773018335039439\\
20.4	0.000773018657510454\\
20.41	0.000773018980119855\\
20.42	0.000773019302867713\\
20.43	0.000773019625754096\\
20.44	0.000773019948779072\\
20.45	0.000773020271942712\\
20.46	0.000773020595245085\\
20.47	0.00077302091868626\\
20.48	0.000773021242266306\\
20.49	0.000773021565985296\\
20.5	0.000773021889843295\\
20.51	0.000773022213840376\\
20.52	0.000773022537976608\\
20.53	0.00077302286225206\\
20.54	0.000773023186666801\\
20.55	0.000773023511220902\\
20.56	0.000773023835914433\\
20.57	0.000773024160747463\\
20.58	0.000773024485720064\\
20.59	0.000773024810832303\\
20.6	0.000773025136084253\\
20.61	0.000773025461475983\\
20.62	0.000773025787007562\\
20.63	0.000773026112679062\\
20.64	0.000773026438490551\\
20.65	0.000773026764442101\\
20.66	0.000773027090533783\\
20.67	0.000773027416765664\\
20.68	0.000773027743137818\\
20.69	0.000773028069650314\\
20.7	0.000773028396303221\\
20.71	0.000773028723096612\\
20.72	0.000773029050030558\\
20.73	0.000773029377105126\\
20.74	0.00077302970432039\\
20.75	0.000773030031676419\\
20.76	0.000773030359173286\\
20.77	0.000773030686811059\\
20.78	0.000773031014589809\\
20.79	0.000773031342509608\\
20.8	0.000773031670570527\\
20.81	0.000773031998772636\\
20.82	0.000773032327116008\\
20.83	0.000773032655600712\\
20.84	0.000773032984226821\\
20.85	0.000773033312994404\\
20.86	0.000773033641903533\\
20.87	0.00077303397095428\\
20.88	0.000773034300146715\\
20.89	0.000773034629480911\\
20.9	0.000773034958956939\\
20.91	0.000773035288574869\\
20.92	0.000773035618334774\\
20.93	0.000773035948236725\\
20.94	0.000773036278280794\\
20.95	0.000773036608467051\\
20.96	0.000773036938795569\\
20.97	0.000773037269266419\\
20.98	0.000773037599879673\\
20.99	0.000773037930635403\\
21	0.00077303826153368\\
21.01	0.000773038592574578\\
21.02	0.000773038923758167\\
21.03	0.000773039255084521\\
21.04	0.00077303958655371\\
21.05	0.000773039918165807\\
21.06	0.000773040249920883\\
21.07	0.000773040581819012\\
21.08	0.000773040913860266\\
21.09	0.000773041246044716\\
21.1	0.000773041578372435\\
21.11	0.000773041910843496\\
21.12	0.000773042243457971\\
21.13	0.000773042576215931\\
21.14	0.00077304290911745\\
21.15	0.0007730432421626\\
21.16	0.000773043575351455\\
21.17	0.000773043908684087\\
21.18	0.00077304424216057\\
21.19	0.000773044575780974\\
21.2	0.000773044909545374\\
21.21	0.00077304524345384\\
21.22	0.000773045577506449\\
21.23	0.000773045911703272\\
21.24	0.000773046246044382\\
21.25	0.000773046580529853\\
21.26	0.000773046915159756\\
21.27	0.000773047249934166\\
21.28	0.000773047584853156\\
21.29	0.0007730479199168\\
21.3	0.000773048255125169\\
21.31	0.000773048590478338\\
21.32	0.000773048925976381\\
21.33	0.00077304926161937\\
21.34	0.00077304959740738\\
21.35	0.000773049933340484\\
21.36	0.000773050269418757\\
21.37	0.000773050605642272\\
21.38	0.0007730509420111\\
21.39	0.000773051278525319\\
21.4	0.000773051615185002\\
21.41	0.000773051951990221\\
21.42	0.000773052288941051\\
21.43	0.000773052626037567\\
21.44	0.000773052963279843\\
21.45	0.000773053300667952\\
21.46	0.000773053638201968\\
21.47	0.000773053975881968\\
21.48	0.000773054313708024\\
21.49	0.00077305465168021\\
21.5	0.000773054989798603\\
21.51	0.000773055328063276\\
21.52	0.000773055666474304\\
21.53	0.000773056005031763\\
21.54	0.000773056343735725\\
21.55	0.000773056682586266\\
21.56	0.000773057021583461\\
21.57	0.000773057360727384\\
21.58	0.000773057700018112\\
21.59	0.000773058039455718\\
21.6	0.000773058379040278\\
21.61	0.000773058718771868\\
21.62	0.000773059058650561\\
21.63	0.000773059398676435\\
21.64	0.000773059738849564\\
21.65	0.000773060079170024\\
21.66	0.000773060419637889\\
21.67	0.000773060760253236\\
21.68	0.00077306110101614\\
21.69	0.000773061441926676\\
21.7	0.000773061782984922\\
21.71	0.000773062124190952\\
21.72	0.000773062465544842\\
21.73	0.000773062807046669\\
21.74	0.000773063148696507\\
21.75	0.000773063490494434\\
21.76	0.000773063832440525\\
21.77	0.000773064174534857\\
21.78	0.000773064516777507\\
21.79	0.000773064859168549\\
21.8	0.000773065201708063\\
21.81	0.000773065544396124\\
21.82	0.000773065887232806\\
21.83	0.000773066230218189\\
21.84	0.000773066573352349\\
21.85	0.000773066916635362\\
21.86	0.000773067260067305\\
21.87	0.000773067603648255\\
21.88	0.000773067947378289\\
21.89	0.000773068291257485\\
21.9	0.000773068635285919\\
21.91	0.00077306897946367\\
21.92	0.000773069323790813\\
21.93	0.000773069668267427\\
21.94	0.00077307001289359\\
21.95	0.000773070357669377\\
21.96	0.000773070702594868\\
21.97	0.000773071047670141\\
21.98	0.000773071392895272\\
21.99	0.000773071738270341\\
22	0.000773072083795423\\
22.01	0.000773072429470599\\
22.02	0.000773072775295947\\
22.03	0.000773073121271544\\
22.04	0.000773073467397469\\
22.05	0.000773073813673799\\
22.06	0.000773074160100614\\
22.07	0.000773074506677992\\
22.08	0.000773074853406011\\
22.09	0.000773075200284752\\
22.1	0.000773075547314293\\
22.11	0.000773075894494711\\
22.12	0.000773076241826086\\
22.13	0.000773076589308499\\
22.14	0.000773076936942027\\
22.15	0.000773077284726751\\
22.16	0.000773077632662749\\
22.17	0.000773077980750101\\
22.18	0.000773078328988888\\
22.19	0.000773078677379187\\
22.2	0.000773079025921081\\
22.21	0.000773079374614648\\
22.22	0.000773079723459967\\
22.23	0.000773080072457119\\
22.24	0.000773080421606185\\
22.25	0.000773080770907245\\
22.26	0.00077308112036038\\
22.27	0.000773081469965669\\
22.28	0.000773081819723194\\
22.29	0.000773082169633035\\
22.3	0.000773082519695273\\
22.31	0.000773082869909987\\
22.32	0.000773083220277263\\
22.33	0.000773083570797178\\
22.34	0.000773083921469814\\
22.35	0.000773084272295254\\
22.36	0.000773084623273577\\
22.37	0.000773084974404866\\
22.38	0.000773085325689203\\
22.39	0.000773085677126669\\
22.4	0.000773086028717347\\
22.41	0.00077308638046132\\
22.42	0.000773086732358668\\
22.43	0.000773087084409475\\
22.44	0.000773087436613822\\
22.45	0.000773087788971793\\
22.46	0.000773088141483469\\
22.47	0.000773088494148933\\
22.48	0.00077308884696827\\
22.49	0.000773089199941561\\
22.5	0.00077308955306889\\
22.51	0.000773089906350341\\
22.52	0.000773090259785999\\
22.53	0.000773090613375944\\
22.54	0.000773090967120263\\
22.55	0.000773091321019038\\
22.56	0.000773091675072352\\
22.57	0.000773092029280293\\
22.58	0.000773092383642942\\
22.59	0.000773092738160385\\
22.6	0.000773093092832706\\
22.61	0.00077309344765999\\
22.62	0.000773093802642321\\
22.63	0.000773094157779786\\
22.64	0.000773094513072468\\
22.65	0.000773094868520455\\
22.66	0.000773095224123832\\
22.67	0.000773095579882683\\
22.68	0.000773095935797096\\
22.69	0.000773096291867155\\
22.7	0.000773096648092947\\
22.71	0.000773097004474558\\
22.72	0.000773097361012077\\
22.73	0.000773097717705586\\
22.74	0.000773098074555176\\
22.75	0.000773098431560933\\
22.76	0.000773098788722943\\
22.77	0.000773099146041295\\
22.78	0.000773099503516075\\
22.79	0.000773099861147373\\
22.8	0.000773100218935275\\
22.81	0.000773100576879869\\
22.82	0.000773100934981244\\
22.83	0.00077310129323949\\
22.84	0.000773101651654694\\
22.85	0.000773102010226944\\
22.86	0.00077310236895633\\
22.87	0.000773102727842943\\
22.88	0.000773103086886872\\
22.89	0.000773103446088205\\
22.9	0.000773103805447033\\
22.91	0.000773104164963446\\
22.92	0.000773104524637533\\
22.93	0.000773104884469387\\
22.94	0.000773105244459097\\
22.95	0.000773105604606755\\
22.96	0.000773105964912452\\
22.97	0.000773106325376279\\
22.98	0.000773106685998327\\
22.99	0.000773107046778689\\
23	0.000773107407717457\\
23.01	0.000773107768814724\\
23.02	0.000773108130070581\\
23.03	0.000773108491485121\\
23.04	0.00077310885305844\\
23.05	0.000773109214790629\\
23.06	0.000773109576681781\\
23.07	0.000773109938731991\\
23.08	0.000773110300941352\\
23.09	0.00077311066330996\\
23.1	0.000773111025837908\\
23.11	0.000773111388525293\\
23.12	0.000773111751372209\\
23.13	0.000773112114378751\\
23.14	0.000773112477545015\\
23.15	0.000773112840871098\\
23.16	0.000773113204357094\\
23.17	0.000773113568003102\\
23.18	0.000773113931809217\\
23.19	0.000773114295775538\\
23.2	0.000773114659902161\\
23.21	0.000773115024189185\\
23.22	0.000773115388636707\\
23.23	0.000773115753244827\\
23.24	0.000773116118013641\\
23.25	0.00077311648294325\\
23.26	0.000773116848033753\\
23.27	0.000773117213285251\\
23.28	0.000773117578697842\\
23.29	0.000773117944271626\\
23.3	0.000773118310006706\\
23.31	0.000773118675903182\\
23.32	0.000773119041961154\\
23.33	0.000773119408180726\\
23.34	0.000773119774561999\\
23.35	0.000773120141105076\\
23.36	0.00077312050781006\\
23.37	0.000773120874677053\\
23.38	0.00077312124170616\\
23.39	0.000773121608897484\\
23.4	0.00077312197625113\\
23.41	0.000773122343767202\\
23.42	0.000773122711445806\\
23.43	0.000773123079287048\\
23.44	0.000773123447291033\\
23.45	0.000773123815457867\\
23.46	0.000773124183787657\\
23.47	0.000773124552280511\\
23.48	0.000773124920936537\\
23.49	0.000773125289755844\\
23.5	0.000773125658738537\\
23.51	0.000773126027884728\\
23.52	0.000773126397194526\\
23.53	0.000773126766668041\\
23.54	0.000773127136305382\\
23.55	0.000773127506106662\\
23.56	0.000773127876071991\\
23.57	0.000773128246201482\\
23.58	0.000773128616495247\\
23.59	0.000773128986953399\\
23.6	0.00077312935757605\\
23.61	0.000773129728363315\\
23.62	0.000773130099315308\\
23.63	0.000773130470432145\\
23.64	0.00077313084171394\\
23.65	0.00077313121316081\\
23.66	0.000773131584772871\\
23.67	0.00077313195655024\\
23.68	0.000773132328493036\\
23.69	0.000773132700601375\\
23.7	0.000773133072875378\\
23.71	0.000773133445315164\\
23.72	0.000773133817920851\\
23.73	0.000773134190692562\\
23.74	0.000773134563630418\\
23.75	0.000773134936734541\\
23.76	0.000773135310005052\\
23.77	0.000773135683442077\\
23.78	0.000773136057045737\\
23.79	0.000773136430816158\\
23.8	0.000773136804753464\\
23.81	0.000773137178857781\\
23.82	0.000773137553129237\\
23.83	0.000773137927567957\\
23.84	0.00077313830217407\\
23.85	0.000773138676947704\\
23.86	0.00077313905188899\\
23.87	0.000773139426998058\\
23.88	0.000773139802275035\\
23.89	0.000773140177720058\\
23.9	0.000773140553333255\\
23.91	0.000773140929114761\\
23.92	0.000773141305064709\\
23.93	0.000773141681183236\\
23.94	0.000773142057470476\\
23.95	0.000773142433926564\\
23.96	0.000773142810551639\\
23.97	0.000773143187345839\\
23.98	0.000773143564309301\\
23.99	0.000773143941442166\\
24	0.000773144318744576\\
24.01	0.000773144696216669\\
24.02	0.000773145073858591\\
24.03	0.000773145451670483\\
24.04	0.00077314582965249\\
24.05	0.000773146207804757\\
24.06	0.00077314658612743\\
24.07	0.000773146964620657\\
24.08	0.000773147343284585\\
24.09	0.000773147722119363\\
24.1	0.000773148101125141\\
24.11	0.000773148480302072\\
24.12	0.000773148859650307\\
24.13	0.000773149239169999\\
24.14	0.000773149618861301\\
24.15	0.00077314999872437\\
24.16	0.000773150378759361\\
24.17	0.000773150758966433\\
24.18	0.000773151139345745\\
24.19	0.000773151519897455\\
24.2	0.000773151900621724\\
24.21	0.000773152281518717\\
24.22	0.000773152662588594\\
24.23	0.000773153043831522\\
24.24	0.000773153425247666\\
24.25	0.000773153806837191\\
24.26	0.000773154188600269\\
24.27	0.000773154570537067\\
24.28	0.000773154952647756\\
24.29	0.000773155334932508\\
24.3	0.000773155717391498\\
24.31	0.000773156100024899\\
24.32	0.000773156482832889\\
24.33	0.000773156865815644\\
24.34	0.000773157248973344\\
24.35	0.000773157632306169\\
24.36	0.000773158015814301\\
24.37	0.000773158399497922\\
24.38	0.000773158783357219\\
24.39	0.000773159167392377\\
24.4	0.000773159551603585\\
24.41	0.000773159935991032\\
24.42	0.000773160320554909\\
24.43	0.000773160705295407\\
24.44	0.000773161090212722\\
24.45	0.00077316147530705\\
24.46	0.000773161860578589\\
24.47	0.000773162246027536\\
24.48	0.000773162631654093\\
24.49	0.000773163017458462\\
24.5	0.000773163403440849\\
24.51	0.000773163789601459\\
24.52	0.000773164175940502\\
24.53	0.000773164562458184\\
24.54	0.00077316494915472\\
24.55	0.000773165336030323\\
24.56	0.000773165723085207\\
24.57	0.000773166110319591\\
24.58	0.000773166497733693\\
24.59	0.000773166885327737\\
24.6	0.000773167273101944\\
24.61	0.000773167661056539\\
24.62	0.000773168049191753\\
24.63	0.000773168437507812\\
24.64	0.000773168826004951\\
24.65	0.000773169214683402\\
24.66	0.000773169603543402\\
24.67	0.000773169992585188\\
24.68	0.000773170381809001\\
24.69	0.000773170771215085\\
24.7	0.000773171160803686\\
24.71	0.00077317155057505\\
24.72	0.000773171940529428\\
24.73	0.000773172330667073\\
24.74	0.000773172720988237\\
24.75	0.000773173111493181\\
24.76	0.000773173502182164\\
24.77	0.000773173893055448\\
24.78	0.000773174284113297\\
24.79	0.000773174675355981\\
24.8	0.000773175066783769\\
24.81	0.000773175458396934\\
24.82	0.000773175850195753\\
24.83	0.000773176242180505\\
24.84	0.000773176634351471\\
24.85	0.000773177026708935\\
24.86	0.000773177419253184\\
24.87	0.000773177811984509\\
24.88	0.000773178204903203\\
24.89	0.000773178598009563\\
24.9	0.000773178991303887\\
24.91	0.000773179384786479\\
24.92	0.000773179778457644\\
24.93	0.000773180172317691\\
24.94	0.000773180566366933\\
24.95	0.000773180960605684\\
24.96	0.000773181355034262\\
24.97	0.00077318174965299\\
24.98	0.000773182144462195\\
24.99	0.000773182539462205\\
25	0.000773182934653352\\
25.01	0.000773183330035973\\
25.02	0.000773183725610407\\
25.03	0.000773184121376998\\
25.04	0.000773184517336094\\
25.05	0.000773184913488045\\
25.06	0.000773185309833207\\
25.07	0.000773185706371938\\
25.08	0.0007731861031046\\
25.09	0.000773186500031562\\
25.1	0.000773186897153192\\
25.11	0.000773187294469867\\
25.12	0.000773187691981965\\
25.13	0.000773188089689869\\
25.14	0.000773188487593968\\
25.15	0.000773188885694652\\
25.16	0.00077318928399232\\
25.17	0.000773189682487371\\
25.18	0.000773190081180212\\
25.19	0.00077319048007125\\
25.2	0.000773190879160902\\
25.21	0.000773191278449588\\
25.22	0.00077319167793773\\
25.23	0.000773192077625759\\
25.24	0.000773192477514109\\
25.25	0.000773192877603218\\
25.26	0.00077319327789353\\
25.27	0.000773193678385494\\
25.28	0.000773194079079565\\
25.29	0.000773194479976204\\
25.3	0.000773194881075873\\
25.31	0.000773195282379045\\
25.32	0.000773195683886195\\
25.33	0.000773196085597805\\
25.34	0.000773196487514363\\
25.35	0.00077319688963636\\
25.36	0.000773197291964295\\
25.37	0.000773197694498674\\
25.38	0.000773198097240007\\
25.39	0.000773198500188811\\
25.4	0.000773198903345609\\
25.41	0.00077319930671093\\
25.42	0.000773199710285308\\
25.43	0.000773200114069287\\
25.44	0.000773200518063415\\
25.45	0.000773200922268246\\
25.46	0.000773201326684342\\
25.47	0.000773201731312271\\
25.48	0.000773202136152609\\
25.49	0.000773202541205937\\
25.5	0.000773202946472845\\
25.51	0.00077320335195393\\
25.52	0.000773203757649795\\
25.53	0.000773204163561051\\
25.54	0.000773204569688317\\
25.55	0.000773204976032218\\
25.56	0.000773205382593388\\
25.57	0.000773205789372469\\
25.58	0.00077320619637011\\
25.59	0.000773206603586967\\
25.6	0.000773207011023708\\
25.61	0.000773207418681005\\
25.62	0.000773207826559539\\
25.63	0.000773208234660001\\
25.64	0.00077320864298309\\
25.65	0.000773209051529514\\
25.66	0.000773209460299988\\
25.67	0.000773209869295237\\
25.68	0.000773210278515997\\
25.69	0.000773210687963009\\
25.7	0.000773211097637028\\
25.71	0.000773211507538814\\
25.72	0.00077321191766914\\
25.73	0.000773212328028785\\
25.74	0.000773212738618542\\
25.75	0.000773213149439211\\
25.76	0.000773213560491603\\
25.77	0.000773213971776538\\
25.78	0.000773214383294848\\
25.79	0.000773214795047376\\
25.8	0.000773215207034973\\
25.81	0.000773215619258503\\
25.82	0.00077321603171884\\
25.83	0.000773216444416868\\
25.84	0.000773216857353485\\
25.85	0.000773217270529598\\
25.86	0.000773217683946127\\
25.87	0.000773218097604002\\
25.88	0.000773218511504166\\
25.89	0.000773218925647573\\
25.9	0.000773219340035191\\
25.91	0.000773219754667999\\
25.92	0.000773220169546987\\
25.93	0.00077322058467316\\
25.94	0.000773221000047536\\
25.95	0.000773221415671144\\
25.96	0.000773221831545028\\
25.97	0.000773222247670244\\
25.98	0.000773222664047862\\
25.99	0.000773223080678965\\
26	0.000773223497564653\\
26.01	0.000773223914706035\\
26.02	0.000773224332104238\\
26.03	0.000773224749760403\\
26.04	0.000773225167675683\\
26.05	0.00077322558585125\\
26.06	0.000773226004288289\\
26.07	0.000773226422987997\\
26.08	0.000773226841951594\\
26.09	0.000773227261180308\\
26.1	0.000773227680675387\\
26.11	0.000773228100438096\\
26.12	0.000773228520469713\\
26.13	0.000773228940771534\\
26.14	0.000773229361344871\\
26.15	0.000773229782191055\\
26.16	0.000773230203311432\\
26.17	0.000773230624707365\\
26.18	0.000773231046380239\\
26.19	0.000773231468331453\\
26.2	0.000773231890562423\\
26.21	0.000773232313074586\\
26.22	0.000773232735869397\\
26.23	0.00077323315894833\\
26.24	0.000773233582312878\\
26.25	0.000773234005964553\\
26.26	0.000773234429904885\\
26.27	0.000773234854135428\\
26.28	0.000773235278657752\\
26.29	0.00077323570347345\\
26.3	0.000773236128584135\\
26.31	0.000773236553991438\\
26.32	0.000773236979697016\\
26.33	0.000773237405702546\\
26.34	0.000773237832009725\\
26.35	0.000773238258620274\\
26.36	0.000773238685535934\\
26.37	0.000773239112758473\\
26.38	0.000773239540289676\\
26.39	0.000773239968131357\\
26.4	0.000773240396285348\\
26.41	0.00077324082475351\\
26.42	0.000773241253537725\\
26.43	0.0007732416826399\\
26.44	0.000773242112061968\\
26.45	0.000773242541805885\\
26.46	0.000773242971873636\\
26.47	0.000773243402267226\\
26.48	0.000773243832988693\\
26.49	0.000773244264040098\\
26.5	0.000773244695423527\\
26.51	0.000773245127141097\\
26.52	0.000773245559194948\\
26.53	0.000773245991587252\\
26.54	0.000773246424320207\\
26.55	0.000773246857396041\\
26.56	0.000773247290817011\\
26.57	0.000773247724585402\\
26.58	0.000773248158703528\\
26.59	0.000773248593173737\\
26.6	0.000773249027998401\\
26.61	0.000773249463179929\\
26.62	0.000773249898720758\\
26.63	0.000773250334623358\\
26.64	0.000773250770890233\\
26.65	0.000773251207523916\\
26.66	0.000773251644526973\\
26.67	0.000773252081902005\\
26.68	0.000773252519651647\\
26.69	0.000773252957778567\\
26.7	0.000773253396285468\\
26.71	0.000773253835175087\\
26.72	0.0007732542744502\\
26.73	0.000773254714113616\\
26.74	0.000773255154168182\\
26.75	0.00077325559461678\\
26.76	0.000773256035462329\\
26.77	0.00077325647670779\\
26.78	0.000773256918356158\\
26.79	0.00077325736041047\\
26.8	0.0007732578028738\\
26.81	0.000773258245749262\\
26.82	0.000773258689040011\\
26.83	0.000773259132749244\\
26.84	0.000773259576880196\\
26.85	0.000773260021436149\\
26.86	0.000773260466420423\\
26.87	0.00077326091183638\\
26.88	0.000773261357687431\\
26.89	0.000773261803977026\\
26.9	0.000773262250708662\\
26.91	0.000773262697885881\\
26.92	0.00077326314551227\\
26.93	0.000773263593591462\\
26.94	0.000773264042127136\\
26.95	0.000773264491123022\\
26.96	0.000773264940582894\\
26.97	0.000773265390510577\\
26.98	0.000773265840909946\\
26.99	0.000773266291784922\\
27	0.000773266743139482\\
27.01	0.00077326719497765\\
27.02	0.0007732676473035\\
27.03	0.000773268100121164\\
27.04	0.000773268553434823\\
27.05	0.000773269007248714\\
27.06	0.000773269461567127\\
27.07	0.000773269916394407\\
27.08	0.000773270371734955\\
27.09	0.000773270827593228\\
27.1	0.000773271283973741\\
27.11	0.000773271740881066\\
27.12	0.000773272198319833\\
27.13	0.000773272656294732\\
27.14	0.000773273114810515\\
27.15	0.00077327357387199\\
27.16	0.000773274033484031\\
27.17	0.00077327449365157\\
27.18	0.000773274954379606\\
27.19	0.000773275415673199\\
27.2	0.000773275877537474\\
27.21	0.000773276339977622\\
27.22	0.0007732768029989\\
27.23	0.00077327726660663\\
27.24	0.000773277730806205\\
27.25	0.000773278195603083\\
27.26	0.000773278661002795\\
27.27	0.000773279127010938\\
27.28	0.000773279593633184\\
27.29	0.000773280060875275\\
27.3	0.000773280528743025\\
27.31	0.000773280997242323\\
27.32	0.000773281466379132\\
27.33	0.000773281936159491\\
27.34	0.000773282406589514\\
27.35	0.000773282877675393\\
27.36	0.000773283349423398\\
27.37	0.000773283821839879\\
27.38	0.000773284294931266\\
27.39	0.000773284768704068\\
27.4	0.000773285243164879\\
27.41	0.000773285718320372\\
27.42	0.000773286194177309\\
27.43	0.000773286670742532\\
27.44	0.000773287148022973\\
27.45	0.000773287626025647\\
27.46	0.000773288104757661\\
27.47	0.00077328858422621\\
27.48	0.000773289064438576\\
27.49	0.000773289545402137\\
27.5	0.000773290027124358\\
27.51	0.000773290509612803\\
27.52	0.000773290992875125\\
27.53	0.000773291476919078\\
27.54	0.000773291961752508\\
27.55	0.000773292447383363\\
27.56	0.000773292933819687\\
27.57	0.000773293421069623\\
27.58	0.00077329390914142\\
27.59	0.000773294398043425\\
27.6	0.000773294887784092\\
27.61	0.000773295378371976\\
27.62	0.000773295869815742\\
27.63	0.000773296362124162\\
27.64	0.000773296855306114\\
27.65	0.000773297349370589\\
27.66	0.000773297844326688\\
27.67	0.000773298340183621\\
27.68	0.000773298836950719\\
27.69	0.000773299334637422\\
27.7	0.000773299833253289\\
27.71	0.000773300332807996\\
27.72	0.000773300833311338\\
27.73	0.000773301334773232\\
27.74	0.000773301837203717\\
27.75	0.000773302340612951\\
27.76	0.000773302845011222\\
27.77	0.000773303350408943\\
27.78	0.000773303856816652\\
27.79	0.000773304364245017\\
27.8	0.00077330487270484\\
27.81	0.000773305382207051\\
27.82	0.000773305892762715\\
27.83	0.000773306404383032\\
27.84	0.000773306917079339\\
27.85	0.000773307430863112\\
27.86	0.000773307945745963\\
27.87	0.00077330846173965\\
27.88	0.000773308978856071\\
27.89	0.00077330949710727\\
27.9	0.000773310016505437\\
27.91	0.000773310537062909\\
27.92	0.000773311058792175\\
27.93	0.000773311581705871\\
27.94	0.000773312105816791\\
27.95	0.000773312631137878\\
27.96	0.000773313157682236\\
27.97	0.000773313685463127\\
27.98	0.000773314214493968\\
27.99	0.000773314744788346\\
28	0.000773315276360004\\
28.01	0.000773315809222854\\
28.02	0.000773316343390975\\
28.03	0.000773316878878614\\
28.04	0.000773317415700189\\
28.05	0.000773317953870293\\
28.06	0.000773318493403691\\
28.07	0.00077331903431533\\
28.08	0.000773319576620328\\
28.09	0.000773320120333989\\
28.1	0.000773320665471798\\
28.11	0.000773321212049427\\
28.12	0.000773321760082732\\
28.13	0.000773322309587758\\
28.14	0.000773322860580744\\
28.15	0.000773323413078118\\
28.16	0.000773323967096506\\
28.17	0.000773324522652732\\
28.18	0.000773325079763817\\
28.19	0.000773325638446985\\
28.2	0.000773326198719664\\
28.21	0.000773326760599487\\
28.22	0.000773327324104298\\
28.23	0.000773327889252148\\
28.24	0.000773328456061304\\
28.25	0.000773329024550247\\
28.26	0.000773329594737677\\
28.27	0.000773330166642511\\
28.28	0.000773330740283892\\
28.29	0.000773331315681186\\
28.3	0.000773331892853987\\
28.31	0.000773332471822116\\
28.32	0.000773333052605629\\
28.33	0.000773333635224818\\
28.34	0.000773334219700208\\
28.35	0.000773334806052567\\
28.36	0.000773335394302905\\
28.37	0.000773335984472474\\
28.38	0.000773336576582776\\
28.39	0.000773337170655563\\
28.4	0.000773337766712841\\
28.41	0.000773338364776869\\
28.42	0.000773338964870165\\
28.43	0.000773339567015509\\
28.44	0.000773340171235943\\
28.45	0.000773340777554778\\
28.46	0.000773341385995592\\
28.47	0.000773341996582235\\
28.48	0.000773342609338833\\
28.49	0.000773343224289789\\
28.5	0.000773343841459788\\
28.51	0.000773344460873798\\
28.52	0.000773345082557074\\
28.53	0.000773345706535158\\
28.54	0.000773346332833888\\
28.55	0.000773346961479396\\
28.56	0.00077334759249811\\
28.57	0.000773348225916763\\
28.58	0.000773348861762392\\
28.59	0.000773349500062342\\
28.6	0.000773350140844266\\
28.61	0.000773350784136134\\
28.62	0.000773351429966231\\
28.63	0.000773352078363165\\
28.64	0.000773352729355864\\
28.65	0.000773353382973584\\
28.66	0.000773354039245912\\
28.67	0.000773354698202769\\
28.68	0.000773355359874406\\
28.69	0.000773356024291421\\
28.7	0.000773356691484751\\
28.71	0.00077335736148568\\
28.72	0.000773358034325841\\
28.73	0.00077335871003722\\
28.74	0.000773359388652159\\
28.75	0.000773360070203358\\
28.76	0.000773360754723883\\
28.77	0.000773361442247163\\
28.78	0.000773362132806996\\
28.79	0.000773362826437555\\
28.8	0.000773363523173389\\
28.81	0.000773364223049425\\
28.82	0.000773364926100971\\
28.83	0.000773365632363724\\
28.84	0.00077336634187377\\
28.85	0.000773367054667587\\
28.86	0.00077336777078205\\
28.87	0.000773368490254431\\
28.88	0.000773369213122406\\
28.89	0.000773369939424057\\
28.9	0.000773370669197875\\
28.91	0.000773371402482763\\
28.92	0.000773372139318038\\
28.93	0.000773372879743437\\
28.94	0.000773373623799121\\
28.95	0.00077337437152567\\
28.96	0.000773375122964097\\
28.97	0.000773375878155843\\
28.98	0.000773376637142784\\
28.99	0.000773377399967232\\
29	0.000773378166671937\\
29.01	0.000773378937300093\\
29.02	0.000773379711895339\\
29.03	0.00077338049050176\\
29.04	0.000773381273163891\\
29.05	0.000773382059926719\\
29.06	0.000773382850835688\\
29.07	0.000773383645936696\\
29.08	0.000773384445276099\\
29.09	0.000773385248900719\\
29.1	0.000773386056857837\\
29.11	0.000773386869195198\\
29.12	0.000773387685961012\\
29.13	0.000773388507203963\\
29.14	0.000773389332973195\\
29.15	0.000773390163318328\\
29.16	0.00077339099828945\\
29.17	0.000773391837937122\\
29.18	0.000773392682312376\\
29.19	0.000773393531466715\\
29.2	0.000773394385452117\\
29.21	0.000773395244321028\\
29.22	0.00077339610812637\\
29.23	0.000773396976921532\\
29.24	0.000773397850760377\\
29.25	0.000773398729697233\\
29.26	0.000773399613786894\\
29.27	0.000773400503084623\\
29.28	0.000773401397646142\\
29.29	0.000773402297527632\\
29.3	0.000773403202785735\\
29.31	0.000773404113477542\\
29.32	0.000773405029660595\\
29.33	0.000773405951392877\\
29.34	0.000773406878732816\\
29.35	0.00077340781173927\\
29.36	0.000773408750471523\\
29.37	0.000773409694989286\\
29.38	0.000773410645352681\\
29.39	0.000773411601622236\\
29.4	0.000773412563858877\\
29.41	0.000773413532123918\\
29.42	0.000773414506479055\\
29.43	0.000773415486986345\\
29.44	0.000773416473708209\\
29.45	0.000773417466707406\\
29.46	0.000773418466047027\\
29.47	0.000773419471790483\\
29.48	0.000773420484001482\\
29.49	0.00077342150274402\\
29.5	0.000773422528082358\\
29.51	0.00077342356008101\\
29.52	0.000773424598804716\\
29.53	0.000773425644318428\\
29.54	0.000773426696687283\\
29.55	0.000773427755976584\\
29.56	0.000773428822251775\\
29.57	0.000773429895578428\\
29.58	0.000773430976022223\\
29.59	0.000773432063648922\\
29.6	0.000773433158524352\\
29.61	0.000773434260714386\\
29.62	0.000773435370284912\\
29.63	0.000773436487301814\\
29.64	0.000773437611830947\\
29.65	0.000773438743938109\\
29.66	0.000773439883689012\\
29.67	0.00077344103114926\\
29.68	0.000773442186384315\\
29.69	0.000773443349459465\\
29.7	0.000773444520439795\\
29.71	0.000773445699390155\\
29.72	0.000773446886375118\\
29.73	0.000773448081458957\\
29.74	0.000773449284705598\\
29.75	0.000773450496178585\\
29.76	0.000773451715941039\\
29.77	0.000773452944055618\\
29.78	0.000773454180584473\\
29.79	0.000773455425589202\\
29.8	0.000773456679130811\\
29.81	0.000773457941269656\\
29.82	0.000773459212065399\\
29.83	0.000773460491576956\\
29.84	0.000773461779862445\\
29.85	0.000773463076979128\\
29.86	0.000773464382983357\\
29.87	0.000773465697930509\\
29.88	0.000773467021874932\\
29.89	0.000773468354869879\\
29.9	0.000773469696967439\\
29.91	0.000773471048218474\\
29.92	0.000773472408672546\\
29.93	0.000773473778377845\\
29.94	0.000773475157381113\\
29.95	0.000773476545727566\\
29.96	0.000773477943460815\\
29.97	0.000773479350622783\\
29.98	0.000773480767253618\\
29.99	0.000773482193391604\\
30	0.000773483629073066\\
30.01	0.000773485074332282\\
30.02	0.000773486529201381\\
30.03	0.000773487993710239\\
30.04	0.000773489467886378\\
30.05	0.000773490951754857\\
30.06	0.000773492445338158\\
30.07	0.00077349394865607\\
30.08	0.000773495461725577\\
30.09	0.000773496984560723\\
30.1	0.000773498517172493\\
30.11	0.00077350005956868\\
30.12	0.000773501611753748\\
30.13	0.000773503173728691\\
30.14	0.000773504745490892\\
30.15	0.00077350632703397\\
30.16	0.000773507918347623\\
30.17	0.000773509519417482\\
30.18	0.000773511130224927\\
30.19	0.000773512750746939\\
30.2	0.000773514380955905\\
30.21	0.000773516020819454\\
30.22	0.000773517670300259\\
30.23	0.000773519329355854\\
30.24	0.000773520997938428\\
30.25	0.000773522675994625\\
30.26	0.000773524363465335\\
30.27	0.000773526060285473\\
30.28	0.000773527766383755\\
30.29	0.000773529481682468\\
30.3	0.000773531206097236\\
30.31	0.000773532939536761\\
30.32	0.000773534681902584\\
30.33	0.000773536433088815\\
30.34	0.000773538192981866\\
30.35	0.000773539961460174\\
30.36	0.000773541738393915\\
30.37	0.000773543523644704\\
30.38	0.000773545317065295\\
30.39	0.000773547118499265\\
30.4	0.000773548927780697\\
30.41	0.000773550744733833\\
30.42	0.000773552569172744\\
30.43	0.000773554400900968\\
30.44	0.000773556239711148\\
30.45	0.000773558085384658\\
30.46	0.000773559937691211\\
30.47	0.000773561796388459\\
30.48	0.000773563661221589\\
30.49	0.00077356553192289\\
30.5	0.000773567408211322\\
30.51	0.000773569289792064\\
30.52	0.00077357117635605\\
30.53	0.000773573067579494\\
30.54	0.000773574963123393\\
30.55	0.000773576862633022\\
30.56	0.000773578765737415\\
30.57	0.000773580672048822\\
30.58	0.000773582581162158\\
30.59	0.000773584492654432\\
30.6	0.000773586406084156\\
30.61	0.000773588320990743\\
30.62	0.000773590236893884\\
30.63	0.000773592153533174\\
30.64	0.000773594070908967\\
30.65	0.000773595989021615\\
30.66	0.000773597907871473\\
30.67	0.000773599827458898\\
30.68	0.000773601747784243\\
30.69	0.000773603668847864\\
30.7	0.000773605590650114\\
30.71	0.00077360751319135\\
30.72	0.000773609436471928\\
30.73	0.000773611360492204\\
30.74	0.000773613285252535\\
30.75	0.000773615210753276\\
30.76	0.000773617136994784\\
30.77	0.000773619063977419\\
30.78	0.000773620991701534\\
30.79	0.000773622920167489\\
30.8	0.000773624849375642\\
30.81	0.000773626779326349\\
30.82	0.000773628710019972\\
30.83	0.000773630641456868\\
30.84	0.000773632573637394\\
30.85	0.000773634506561911\\
30.86	0.000773636440230778\\
30.87	0.000773638374644356\\
30.88	0.000773640309803002\\
30.89	0.000773642245707078\\
30.9	0.000773644182356944\\
30.91	0.000773646119752962\\
30.92	0.000773648057895491\\
30.93	0.000773649996784892\\
30.94	0.000773651936421529\\
30.95	0.000773653876805761\\
30.96	0.000773655817937952\\
30.97	0.000773657759818462\\
30.98	0.000773659702447655\\
30.99	0.000773661645825895\\
31	0.000773663589953542\\
31.01	0.000773665534830962\\
31.02	0.000773667480458517\\
31.03	0.000773669426836572\\
31.04	0.000773671373965491\\
31.05	0.000773673321845636\\
31.06	0.000773675270477373\\
31.07	0.000773677219861067\\
31.08	0.000773679169997083\\
31.09	0.000773681120885788\\
31.1	0.000773683072527545\\
31.11	0.000773685024922722\\
31.12	0.000773686978071684\\
31.13	0.000773688931974797\\
31.14	0.000773690886632429\\
31.15	0.000773692842044947\\
31.16	0.000773694798212718\\
31.17	0.000773696755136109\\
31.18	0.000773698712815487\\
31.19	0.000773700671251222\\
31.2	0.000773702630443681\\
31.21	0.000773704590393233\\
31.22	0.000773706551100246\\
31.23	0.000773708512565089\\
31.24	0.000773710474788133\\
31.25	0.000773712437769747\\
31.26	0.000773714401510301\\
31.27	0.000773716366010164\\
31.28	0.000773718331269707\\
31.29	0.000773720297289301\\
31.3	0.000773722264069318\\
31.31	0.000773724231610127\\
31.32	0.000773726199912101\\
31.33	0.000773728168975613\\
31.34	0.000773730138801032\\
31.35	0.000773732109388732\\
31.36	0.000773734080739085\\
31.37	0.000773736052852466\\
31.38	0.000773738025729246\\
31.39	0.000773739999369799\\
31.4	0.0007737419737745\\
31.41	0.000773743948943721\\
31.42	0.000773745924877837\\
31.43	0.000773747901577223\\
31.44	0.000773749879042251\\
31.45	0.000773751857273299\\
31.46	0.000773753836270742\\
31.47	0.000773755816034956\\
31.48	0.000773757796566315\\
31.49	0.000773759777865196\\
31.5	0.000773761759931976\\
31.51	0.000773763742767032\\
31.52	0.000773765726370741\\
31.53	0.000773767710743478\\
31.54	0.000773769695885624\\
31.55	0.000773771681797554\\
31.56	0.000773773668479647\\
31.57	0.000773775655932282\\
31.58	0.000773777644155838\\
31.59	0.000773779633150692\\
31.6	0.000773781622917226\\
31.61	0.000773783613455818\\
31.62	0.000773785604766847\\
31.63	0.000773787596850694\\
31.64	0.000773789589707738\\
31.65	0.000773791583338361\\
31.66	0.000773793577742944\\
31.67	0.000773795572921867\\
31.68	0.000773797568875512\\
31.69	0.000773799565604262\\
31.7	0.000773801563108497\\
31.71	0.000773803561388601\\
31.72	0.000773805560444955\\
31.73	0.000773807560277943\\
31.74	0.000773809560887948\\
31.75	0.000773811562275352\\
31.76	0.000773813564440541\\
31.77	0.000773815567383897\\
31.78	0.000773817571105806\\
31.79	0.000773819575606653\\
31.8	0.00077382158088682\\
31.81	0.000773823586946694\\
31.82	0.000773825593786661\\
31.83	0.000773827601407106\\
31.84	0.000773829609808414\\
31.85	0.000773831618990972\\
31.86	0.000773833628955167\\
31.87	0.000773835639701386\\
31.88	0.000773837651230016\\
31.89	0.000773839663541444\\
31.9	0.000773841676636059\\
31.91	0.000773843690514246\\
31.92	0.000773845705176397\\
31.93	0.000773847720622898\\
31.94	0.000773849736854139\\
31.95	0.000773851753870508\\
31.96	0.000773853771672396\\
31.97	0.000773855790260193\\
31.98	0.000773857809634288\\
31.99	0.000773859829795071\\
32	0.000773861850742933\\
32.01	0.000773863872478265\\
32.02	0.000773865895001457\\
32.03	0.000773867918312902\\
32.04	0.000773869942412993\\
32.05	0.000773871967302119\\
32.06	0.000773873992980675\\
32.07	0.000773876019449051\\
32.08	0.000773878046707643\\
32.09	0.000773880074756842\\
32.1	0.000773882103597044\\
32.11	0.000773884133228641\\
32.12	0.000773886163652026\\
32.13	0.000773888194867595\\
32.14	0.000773890226875743\\
32.15	0.000773892259676865\\
32.16	0.000773894293271356\\
32.17	0.000773896327659611\\
32.18	0.000773898362842026\\
32.19	0.000773900398818998\\
32.2	0.000773902435590923\\
32.21	0.000773904473158199\\
32.22	0.000773906511521223\\
32.23	0.00077390855068039\\
32.24	0.0007739105906361\\
32.25	0.00077391263138875\\
32.26	0.000773914672938738\\
32.27	0.000773916715286463\\
32.28	0.000773918758432326\\
32.29	0.000773920802376723\\
32.3	0.000773922847120054\\
32.31	0.000773924892662722\\
32.32	0.000773926939005123\\
32.33	0.00077392898614766\\
32.34	0.000773931034090732\\
32.35	0.000773933082834742\\
32.36	0.000773935132380089\\
32.37	0.000773937182727178\\
32.38	0.000773939233876407\\
32.39	0.000773941285828181\\
32.4	0.000773943338582902\\
32.41	0.000773945392140972\\
32.42	0.000773947446502795\\
32.43	0.000773949501668775\\
32.44	0.000773951557639315\\
32.45	0.000773953614414818\\
32.46	0.00077395567199569\\
32.47	0.000773957730382335\\
32.48	0.000773959789575158\\
32.49	0.000773961849574565\\
32.5	0.000773963910380961\\
32.51	0.000773965971994752\\
32.52	0.000773968034416343\\
32.53	0.000773970097646142\\
32.54	0.000773972161684556\\
32.55	0.000773974226531991\\
32.56	0.000773976292188857\\
32.57	0.00077397835865556\\
32.58	0.000773980425932507\\
32.59	0.000773982494020107\\
32.6	0.000773984562918771\\
32.61	0.000773986632628905\\
32.62	0.000773988703150921\\
32.63	0.000773990774485226\\
32.64	0.000773992846632232\\
32.65	0.000773994919592349\\
32.66	0.000773996993365986\\
32.67	0.000773999067953555\\
32.68	0.000774001143355468\\
32.69	0.000774003219572135\\
32.7	0.00077400529660397\\
32.71	0.000774007374451383\\
32.72	0.000774009453114788\\
32.73	0.000774011532594597\\
32.74	0.000774013612891223\\
32.75	0.00077401569400508\\
32.76	0.000774017775936582\\
32.77	0.000774019858686144\\
32.78	0.000774021942254177\\
32.79	0.000774024026641099\\
32.8	0.000774026111847324\\
32.81	0.000774028197873266\\
32.82	0.000774030284719342\\
32.83	0.000774032372385968\\
32.84	0.000774034460873561\\
32.85	0.000774036550182535\\
32.86	0.000774038640313311\\
32.87	0.000774040731266303\\
32.88	0.000774042823041929\\
32.89	0.000774044915640608\\
32.9	0.000774047009062757\\
32.91	0.000774049103308796\\
32.92	0.000774051198379143\\
32.93	0.000774053294274216\\
32.94	0.000774055390994437\\
32.95	0.000774057488540226\\
32.96	0.000774059586912\\
32.97	0.000774061686110184\\
32.98	0.000774063786135195\\
32.99	0.000774065886987457\\
33	0.000774067988667389\\
33.01	0.000774070091175414\\
33.02	0.000774072194511954\\
33.03	0.000774074298677431\\
33.04	0.00077407640367227\\
33.05	0.000774078509496892\\
33.06	0.00077408061615172\\
33.07	0.000774082723637179\\
33.08	0.000774084831953693\\
33.09	0.000774086941101686\\
33.1	0.000774089051081584\\
33.11	0.00077409116189381\\
33.12	0.000774093273538792\\
33.13	0.000774095386016953\\
33.14	0.000774097499328722\\
33.15	0.000774099613474523\\
33.16	0.000774101728454784\\
33.17	0.000774103844269932\\
33.18	0.000774105960920394\\
33.19	0.000774108078406598\\
33.2	0.000774110196728973\\
33.21	0.000774112315887946\\
33.22	0.000774114435883946\\
33.23	0.000774116556717403\\
33.24	0.000774118678388747\\
33.25	0.000774120800898406\\
33.26	0.000774122924246811\\
33.27	0.000774125048434392\\
33.28	0.000774127173461581\\
33.29	0.000774129299328808\\
33.3	0.000774131426036506\\
33.31	0.000774133553585103\\
33.32	0.000774135681975034\\
33.33	0.000774137811206731\\
33.34	0.000774139941280628\\
33.35	0.000774142072197156\\
33.36	0.000774144203956752\\
33.37	0.000774146336559846\\
33.38	0.000774148470006874\\
33.39	0.000774150604298271\\
33.4	0.00077415273943447\\
33.41	0.000774154875415909\\
33.42	0.00077415701224302\\
33.43	0.000774159149916241\\
33.44	0.000774161288436008\\
33.45	0.000774163427802758\\
33.46	0.000774165568016929\\
33.47	0.000774167709078956\\
33.48	0.000774169850989277\\
33.49	0.000774171993748331\\
33.5	0.000774174137356556\\
33.51	0.000774176281814389\\
33.52	0.000774178427122272\\
33.53	0.000774180573280643\\
33.54	0.000774182720289941\\
33.55	0.000774184868150606\\
33.56	0.000774187016863079\\
33.57	0.000774189166427802\\
33.58	0.000774191316845214\\
33.59	0.000774193468115757\\
33.6	0.000774195620239875\\
33.61	0.000774197773218008\\
33.62	0.000774199927050597\\
33.63	0.000774202081738088\\
33.64	0.000774204237280923\\
33.65	0.000774206393679545\\
33.66	0.000774208550934399\\
33.67	0.000774210709045929\\
33.68	0.000774212868014578\\
33.69	0.000774215027840794\\
33.7	0.000774217188525019\\
33.71	0.0007742193500677\\
33.72	0.000774221512469284\\
33.73	0.000774223675730218\\
33.74	0.000774225839850946\\
33.75	0.000774228004831916\\
33.76	0.000774230170673577\\
33.77	0.000774232337376375\\
33.78	0.00077423450494076\\
33.79	0.000774236673367179\\
33.8	0.000774238842656082\\
33.81	0.000774241012807918\\
33.82	0.000774243183823136\\
33.83	0.000774245355702186\\
33.84	0.000774247528445519\\
33.85	0.000774249702053585\\
33.86	0.000774251876526836\\
33.87	0.000774254051865722\\
33.88	0.000774256228070695\\
33.89	0.00077425840514221\\
33.9	0.000774260583080716\\
33.91	0.000774262761886668\\
33.92	0.000774264941560517\\
33.93	0.000774267122102719\\
33.94	0.000774269303513726\\
33.95	0.000774271485793994\\
33.96	0.000774273668943976\\
33.97	0.000774275852964128\\
33.98	0.000774278037854906\\
33.99	0.000774280223616766\\
34	0.000774282410250163\\
34.01	0.000774284597755553\\
34.02	0.000774286786133395\\
34.03	0.000774288975384145\\
34.04	0.00077429116550826\\
34.05	0.000774293356506198\\
34.06	0.00077429554837842\\
34.07	0.000774297741125383\\
34.08	0.000774299934747546\\
34.09	0.000774302129245367\\
34.1	0.000774304324619309\\
34.11	0.000774306520869831\\
34.12	0.000774308717997392\\
34.13	0.000774310916002455\\
34.14	0.00077431311488548\\
34.15	0.000774315314646929\\
34.16	0.000774317515287264\\
34.17	0.00077431971680695\\
34.18	0.000774321919206445\\
34.19	0.000774324122486215\\
34.2	0.000774326326646726\\
34.21	0.000774328531688438\\
34.22	0.000774330737611817\\
34.23	0.000774332944417328\\
34.24	0.000774335152105435\\
34.25	0.000774337360676605\\
34.26	0.000774339570131302\\
34.27	0.000774341780469994\\
34.28	0.000774343991693146\\
34.29	0.000774346203801228\\
34.3	0.000774348416794704\\
34.31	0.000774350630674044\\
34.32	0.000774352845439715\\
34.33	0.000774355061092186\\
34.34	0.000774357277631926\\
34.35	0.000774359495059404\\
34.36	0.000774361713375089\\
34.37	0.000774363932579453\\
34.38	0.000774366152672965\\
34.39	0.000774368373656095\\
34.4	0.000774370595529315\\
34.41	0.000774372818293097\\
34.42	0.000774375041947913\\
34.43	0.000774377266494236\\
34.44	0.000774379491932538\\
34.45	0.000774381718263292\\
34.46	0.000774383945486972\\
34.47	0.000774386173604051\\
34.48	0.000774388402615005\\
34.49	0.000774390632520307\\
34.5	0.000774392863320433\\
34.51	0.000774395095015859\\
34.52	0.000774397327607059\\
34.53	0.000774399561094511\\
34.54	0.000774401795478693\\
34.55	0.000774404030760078\\
34.56	0.000774406266939146\\
34.57	0.000774408504016375\\
34.58	0.000774410741992242\\
34.59	0.000774412980867226\\
34.6	0.000774415220641807\\
34.61	0.000774417461316465\\
34.62	0.000774419702891679\\
34.63	0.000774421945367927\\
34.64	0.000774424188745692\\
34.65	0.000774426433025456\\
34.66	0.000774428678207697\\
34.67	0.0007744309242929\\
34.68	0.000774433171281546\\
34.69	0.000774435419174118\\
34.7	0.000774437667971098\\
34.71	0.000774439917672969\\
34.72	0.000774442168280216\\
34.73	0.000774444419793324\\
34.74	0.000774446672212777\\
34.75	0.000774448925539059\\
34.76	0.000774451179772656\\
34.77	0.000774453434914054\\
34.78	0.00077445569096374\\
34.79	0.000774457947922198\\
34.8	0.000774460205789917\\
34.81	0.000774462464567385\\
34.82	0.000774464724255087\\
34.83	0.000774466984853514\\
34.84	0.000774469246363155\\
34.85	0.000774471508784498\\
34.86	0.000774473772118032\\
34.87	0.000774476036364248\\
34.88	0.000774478301523635\\
34.89	0.000774480567596684\\
34.9	0.000774482834583886\\
34.91	0.000774485102485734\\
34.92	0.000774487371302717\\
34.93	0.00077448964103533\\
34.94	0.000774491911684065\\
34.95	0.000774494183249416\\
34.96	0.000774496455731875\\
34.97	0.000774498729131937\\
34.98	0.000774501003450095\\
34.99	0.000774503278686845\\
35	0.000774505554842682\\
35.01	0.000774507831918101\\
35.02	0.0007745101099136\\
35.03	0.000774512388829673\\
35.04	0.000774514668666819\\
35.05	0.000774516949425534\\
35.06	0.000774519231106316\\
35.07	0.000774521513709664\\
35.08	0.000774523797236075\\
35.09	0.00077452608168605\\
35.1	0.000774528367060085\\
35.11	0.000774530653358684\\
35.12	0.000774532940582344\\
35.13	0.000774535228731566\\
35.14	0.000774537517806853\\
35.15	0.000774539807808707\\
35.16	0.000774542098737627\\
35.17	0.000774544390594116\\
35.18	0.000774546683378678\\
35.19	0.000774548977091815\\
35.2	0.000774551271734033\\
35.21	0.000774553567305833\\
35.22	0.000774555863807721\\
35.23	0.000774558161240202\\
35.24	0.000774560459603781\\
35.25	0.000774562758898964\\
35.26	0.000774565059126257\\
35.27	0.000774567360286166\\
35.28	0.000774569662379199\\
35.29	0.000774571965405863\\
35.3	0.000774574269366666\\
35.31	0.000774576574262116\\
35.32	0.000774578880092722\\
35.33	0.000774581186858993\\
35.34	0.000774583494561439\\
35.35	0.00077458580320057\\
35.36	0.000774588112776897\\
35.37	0.00077459042329093\\
35.38	0.000774592734743181\\
35.39	0.000774595047134161\\
35.4	0.000774597360464382\\
35.41	0.000774599674734359\\
35.42	0.000774601989944603\\
35.43	0.000774604306095628\\
35.44	0.000774606623187948\\
35.45	0.000774608941222077\\
35.46	0.000774611260198531\\
35.47	0.000774613580117824\\
35.48	0.000774615900980472\\
35.49	0.000774618222786991\\
35.5	0.0007746205455379\\
35.51	0.000774622869233712\\
35.52	0.000774625193874947\\
35.53	0.000774627519462123\\
35.54	0.000774629845995757\\
35.55	0.000774632173476369\\
35.56	0.000774634501904478\\
35.57	0.000774636831280604\\
35.58	0.000774639161605266\\
35.59	0.000774641492878985\\
35.6	0.000774643825102284\\
35.61	0.000774646158275681\\
35.62	0.000774648492399702\\
35.63	0.000774650827474866\\
35.64	0.000774653163501698\\
35.65	0.00077465550048072\\
35.66	0.000774657838412456\\
35.67	0.00077466017729743\\
35.68	0.000774662517136168\\
35.69	0.000774664857929193\\
35.7	0.000774667199677032\\
35.71	0.000774669542380209\\
35.72	0.000774671886039253\\
35.73	0.000774674230654691\\
35.74	0.000774676576227047\\
35.75	0.000774678922756853\\
35.76	0.000774681270244633\\
35.77	0.00077468361869092\\
35.78	0.000774685968096239\\
35.79	0.000774688318461123\\
35.8	0.0007746906697861\\
35.81	0.000774693022071702\\
35.82	0.000774695375318459\\
35.83	0.000774697729526904\\
35.84	0.000774700084697566\\
35.85	0.00077470244083098\\
35.86	0.000774704797927678\\
35.87	0.000774707155988192\\
35.88	0.000774709515013058\\
35.89	0.000774711875002809\\
35.9	0.00077471423595798\\
35.91	0.000774716597879106\\
35.92	0.000774718960766722\\
35.93	0.000774721324621365\\
35.94	0.000774723689443571\\
35.95	0.000774726055233878\\
35.96	0.000774728421992821\\
35.97	0.000774730789720941\\
35.98	0.000774733158418774\\
35.99	0.000774735528086861\\
36	0.000774737898725739\\
36.01	0.000774740270335949\\
36.02	0.000774742642918031\\
36.03	0.000774745016472527\\
36.04	0.000774747390999976\\
36.05	0.000774749766500922\\
36.06	0.000774752142975906\\
36.07	0.000774754520425469\\
36.08	0.000774756898850157\\
36.09	0.000774759278250513\\
36.1	0.000774761658627081\\
36.11	0.000774764039980403\\
36.12	0.000774766422311028\\
36.13	0.000774768805619499\\
36.14	0.000774771189906362\\
36.15	0.000774773575172163\\
36.16	0.000774775961417449\\
36.17	0.000774778348642769\\
36.18	0.00077478073684867\\
36.19	0.0007747831260357\\
36.2	0.000774785516204409\\
36.21	0.000774787907355344\\
36.22	0.000774790299489058\\
36.23	0.000774792692606099\\
36.24	0.000774795086707018\\
36.25	0.000774797481792367\\
36.26	0.000774799877862697\\
36.27	0.00077480227491856\\
36.28	0.000774804672960509\\
36.29	0.000774807071989098\\
36.3	0.000774809472004879\\
36.31	0.000774811873008407\\
36.32	0.000774814275000237\\
36.33	0.000774816677980922\\
36.34	0.000774819081951022\\
36.35	0.000774821486911088\\
36.36	0.00077482389286168\\
36.37	0.000774826299803353\\
36.38	0.000774828707736666\\
36.39	0.000774831116662177\\
36.4	0.000774833526580443\\
36.41	0.000774835937492026\\
36.42	0.000774838349397485\\
36.43	0.000774840762297377\\
36.44	0.000774843176192265\\
36.45	0.000774845591082709\\
36.46	0.000774848006969271\\
36.47	0.000774850423852513\\
36.48	0.000774852841732997\\
36.49	0.000774855260611286\\
36.5	0.000774857680487944\\
36.51	0.000774860101363534\\
36.52	0.000774862523238622\\
36.53	0.000774864946113773\\
36.54	0.00077486736998955\\
36.55	0.000774869794866521\\
36.56	0.000774872220745253\\
36.57	0.000774874647626312\\
36.58	0.000774877075510265\\
36.59	0.000774879504397681\\
36.6	0.000774881934289129\\
36.61	0.000774884365185175\\
36.62	0.000774886797086392\\
36.63	0.000774889229993348\\
36.64	0.000774891663906615\\
36.65	0.000774894098826763\\
36.66	0.000774896534754363\\
36.67	0.000774898971689988\\
36.68	0.00077490140963421\\
36.69	0.000774903848587602\\
36.7	0.000774906288550738\\
36.71	0.000774908729524192\\
36.72	0.00077491117150854\\
36.73	0.000774913614504354\\
36.74	0.000774916058512211\\
36.75	0.000774918503532687\\
36.76	0.00077492094956636\\
36.77	0.000774923396613805\\
36.78	0.000774925844675601\\
36.79	0.000774928293752326\\
36.8	0.000774930743844559\\
36.81	0.000774933194952879\\
36.82	0.000774935647077864\\
36.83	0.000774938100220097\\
36.84	0.000774940554380156\\
36.85	0.000774943009558625\\
36.86	0.000774945465756084\\
36.87	0.000774947922973116\\
36.88	0.000774950381210303\\
36.89	0.000774952840468229\\
36.9	0.000774955300747478\\
36.91	0.000774957762048634\\
36.92	0.000774960224372283\\
36.93	0.000774962687719009\\
36.94	0.000774965152089399\\
36.95	0.000774967617484039\\
36.96	0.000774970083903516\\
36.97	0.000774972551348417\\
36.98	0.000774975019819331\\
36.99	0.000774977489316845\\
37	0.00077497995984155\\
37.01	0.000774982431394036\\
37.02	0.000774984903974891\\
37.03	0.000774987377584708\\
37.04	0.000774989852224076\\
37.05	0.000774992327893589\\
37.06	0.000774994804593837\\
37.07	0.000774997282325414\\
37.08	0.000774999761088914\\
37.09	0.00077500224088493\\
37.1	0.000775004721714058\\
37.11	0.000775007203576889\\
37.12	0.000775009686474022\\
37.13	0.000775012170406052\\
37.14	0.000775014655373576\\
37.15	0.000775017141377191\\
37.16	0.000775019628417493\\
37.17	0.000775022116495083\\
37.18	0.000775024605610557\\
37.19	0.000775027095764515\\
37.2	0.000775029586957559\\
37.21	0.000775032079190286\\
37.22	0.0007750345724633\\
37.23	0.000775037066777199\\
37.24	0.000775039562132587\\
37.25	0.000775042058530066\\
37.26	0.000775044555970239\\
37.27	0.000775047054453712\\
37.28	0.000775049553981085\\
37.29	0.000775052054552965\\
37.3	0.000775054556169956\\
37.31	0.000775057058832665\\
37.32	0.000775059562541699\\
37.33	0.000775062067297664\\
37.34	0.000775064573101166\\
37.35	0.000775067079952814\\
37.36	0.000775069587853217\\
37.37	0.000775072096802984\\
37.38	0.000775074606802723\\
37.39	0.000775077117853046\\
37.4	0.000775079629954564\\
37.41	0.000775082143107887\\
37.42	0.000775084657313626\\
37.43	0.000775087172572395\\
37.44	0.000775089688884806\\
37.45	0.000775092206251474\\
37.46	0.000775094724673011\\
37.47	0.000775097244150032\\
37.48	0.000775099764683153\\
37.49	0.000775102286272989\\
37.5	0.000775104808920157\\
37.51	0.000775107332625273\\
37.52	0.000775109857388954\\
37.53	0.00077511238321182\\
37.54	0.000775114910094486\\
37.55	0.000775117438037574\\
37.56	0.000775119967041703\\
37.57	0.000775122497107492\\
37.58	0.000775125028235563\\
37.59	0.000775127560426537\\
37.6	0.000775130093681036\\
37.61	0.000775132627999681\\
37.62	0.000775135163383097\\
37.63	0.000775137699831906\\
37.64	0.000775140237346733\\
37.65	0.000775142775928203\\
37.66	0.000775145315576939\\
37.67	0.000775147856293569\\
37.68	0.00077515039807872\\
37.69	0.000775152940933017\\
37.7	0.000775155484857088\\
37.71	0.000775158029851562\\
37.72	0.000775160575917066\\
37.73	0.000775163123054231\\
37.74	0.000775165671263685\\
37.75	0.000775168220546059\\
37.76	0.000775170770901983\\
37.77	0.000775173322332092\\
37.78	0.000775175874837015\\
37.79	0.000775178428417385\\
37.8	0.000775180983073836\\
37.81	0.000775183538807001\\
37.82	0.000775186095617516\\
37.83	0.000775188653506014\\
37.84	0.000775191212473133\\
37.85	0.000775193772519505\\
37.86	0.000775196333645771\\
37.87	0.000775198895852567\\
37.88	0.000775201459140529\\
37.89	0.000775204023510298\\
37.9	0.000775206588962511\\
37.91	0.000775209155497808\\
37.92	0.00077521172311683\\
37.93	0.000775214291820217\\
37.94	0.000775216861608611\\
37.95	0.000775219432482654\\
37.96	0.000775222004442988\\
37.97	0.000775224577490257\\
37.98	0.000775227151625103\\
37.99	0.000775229726848171\\
38	0.000775232303160105\\
38.01	0.000775234880561552\\
38.02	0.000775237459053158\\
38.03	0.000775240038635569\\
38.04	0.000775242619309431\\
38.05	0.000775245201075393\\
38.06	0.000775247783934105\\
38.07	0.000775250367886213\\
38.08	0.000775252952932369\\
38.09	0.000775255539073223\\
38.1	0.000775258126309424\\
38.11	0.000775260714641623\\
38.12	0.000775263304070475\\
38.13	0.00077526589459663\\
38.14	0.000775268486220742\\
38.15	0.000775271078943464\\
38.16	0.000775273672765451\\
38.17	0.000775276267687357\\
38.18	0.000775278863709839\\
38.19	0.000775281460833553\\
38.2	0.000775284059059154\\
38.21	0.000775286658387301\\
38.22	0.000775289258818651\\
38.23	0.000775291860353862\\
38.24	0.000775294462993594\\
38.25	0.000775297066738507\\
38.26	0.00077529967158926\\
38.27	0.000775302277546516\\
38.28	0.000775304884610934\\
38.29	0.000775307492783178\\
38.3	0.000775310102063911\\
38.31	0.000775312712453794\\
38.32	0.000775315323953493\\
38.33	0.000775317936563671\\
38.34	0.000775320550284996\\
38.35	0.000775323165118131\\
38.36	0.000775325781063744\\
38.37	0.000775328398122499\\
38.38	0.000775331016295067\\
38.39	0.000775333635582114\\
38.4	0.00077533625598431\\
38.41	0.000775338877502325\\
38.42	0.000775341500136827\\
38.43	0.000775344123888487\\
38.44	0.000775346748757977\\
38.45	0.00077534937474597\\
38.46	0.000775352001853137\\
38.47	0.000775354630080152\\
38.48	0.000775357259427687\\
38.49	0.000775359889896417\\
38.5	0.000775362521487017\\
38.51	0.000775365154200161\\
38.52	0.000775367788036529\\
38.53	0.000775370422996795\\
38.54	0.000775373059081636\\
38.55	0.000775375696291733\\
38.56	0.00077537833462776\\
38.57	0.000775380974090399\\
38.58	0.00077538361468033\\
38.59	0.000775386256398234\\
38.6	0.00077538889924479\\
38.61	0.000775391543220681\\
38.62	0.000775394188326589\\
38.63	0.000775396834563197\\
38.64	0.000775399481931189\\
38.65	0.000775402130431249\\
38.66	0.000775404780064062\\
38.67	0.000775407430830313\\
38.68	0.000775410082730688\\
38.69	0.000775412735765873\\
38.7	0.000775415389936559\\
38.71	0.000775418045243429\\
38.72	0.000775420701687174\\
38.73	0.000775423359268484\\
38.74	0.000775426017988048\\
38.75	0.000775428677846557\\
38.76	0.000775431338844701\\
38.77	0.000775434000983172\\
38.78	0.000775436664262663\\
38.79	0.000775439328683866\\
38.8	0.000775441994247476\\
38.81	0.000775444660954186\\
38.82	0.000775447328804692\\
38.83	0.000775449997799689\\
38.84	0.000775452667939873\\
38.85	0.00077545533922594\\
38.86	0.00077545801165859\\
38.87	0.000775460685238519\\
38.88	0.000775463359966427\\
38.89	0.000775466035843012\\
38.9	0.000775468712868974\\
38.91	0.000775471391045014\\
38.92	0.000775474070371835\\
38.93	0.000775476750850138\\
38.94	0.000775479432480623\\
38.95	0.000775482115263997\\
38.96	0.000775484799200962\\
38.97	0.000775487484292223\\
38.98	0.000775490170538485\\
38.99	0.000775492857940452\\
39	0.000775495546498833\\
39.01	0.000775498236214334\\
39.02	0.000775500927087662\\
39.03	0.000775503619119527\\
39.04	0.000775506312310638\\
39.05	0.000775509006661703\\
39.06	0.000775511702173434\\
39.07	0.000775514398846541\\
39.08	0.000775517096681735\\
39.09	0.000775519795679731\\
39.1	0.000775522495841239\\
39.11	0.000775525197166973\\
39.12	0.000775527899657649\\
39.13	0.00077553060331398\\
39.14	0.000775533308136683\\
39.15	0.000775536014126475\\
39.16	0.00077553872128407\\
39.17	0.000775541429610187\\
39.18	0.000775544139105546\\
39.19	0.000775546849770863\\
39.2	0.000775549561606859\\
39.21	0.000775552274614254\\
39.22	0.000775554988793769\\
39.23	0.000775557704146125\\
39.24	0.000775560420672044\\
39.25	0.000775563138372249\\
39.26	0.000775565857247465\\
39.27	0.000775568577298415\\
39.28	0.000775571298525823\\
39.29	0.000775574020930416\\
39.3	0.00077557674451292\\
39.31	0.000775579469274061\\
39.32	0.000775582195214567\\
39.33	0.000775584922335166\\
39.34	0.000775587650636587\\
39.35	0.00077559038011956\\
39.36	0.000775593110784815\\
39.37	0.000775595842633084\\
39.38	0.000775598575665096\\
39.39	0.000775601309881585\\
39.4	0.000775604045283283\\
39.41	0.000775606781870926\\
39.42	0.000775609519645245\\
39.43	0.000775612258606978\\
39.44	0.000775614998756859\\
39.45	0.000775617740095625\\
39.46	0.000775620482624011\\
39.47	0.000775623226342758\\
39.48	0.000775625971252603\\
39.49	0.000775628717354284\\
39.5	0.000775631464648542\\
39.51	0.000775634213136119\\
39.52	0.000775636962817754\\
39.53	0.000775639713694189\\
39.54	0.000775642465766167\\
39.55	0.000775645219034431\\
39.56	0.000775647973499726\\
39.57	0.000775650729162797\\
39.58	0.000775653486024389\\
39.59	0.000775656244085246\\
39.6	0.000775659003346118\\
39.61	0.000775661763807752\\
39.62	0.000775664525470893\\
39.63	0.000775667288336293\\
39.64	0.000775670052404701\\
39.65	0.000775672817676867\\
39.66	0.000775675584153543\\
39.67	0.000775678351835481\\
39.68	0.000775681120723432\\
39.69	0.000775683890818149\\
39.7	0.000775686662120387\\
39.71	0.000775689434630902\\
39.72	0.000775692208350448\\
39.73	0.000775694983279781\\
39.74	0.000775697759419658\\
39.75	0.000775700536770835\\
39.76	0.000775703315334072\\
39.77	0.000775706095110129\\
39.78	0.000775708876099764\\
39.79	0.000775711658303738\\
39.8	0.000775714441722813\\
39.81	0.000775717226357751\\
39.82	0.000775720012209314\\
39.83	0.000775722799278265\\
39.84	0.000775725587565369\\
39.85	0.000775728377071391\\
39.86	0.000775731167797097\\
39.87	0.000775733959743253\\
39.88	0.000775736752910625\\
39.89	0.000775739547299984\\
39.9	0.000775742342912096\\
39.91	0.000775745139747732\\
39.92	0.000775747937807662\\
39.93	0.000775750737092656\\
39.94	0.000775753537603487\\
39.95	0.000775756339340926\\
39.96	0.000775759142305748\\
39.97	0.000775761946498726\\
39.98	0.000775764751920637\\
39.99	0.000775767558572255\\
40	0.000775770366454357\\
40.01	0.000775773175567719\\
};
\addplot [color=black,solid,forget plot]
  table[row sep=crcr]{%
40.01	0.000775773175567719\\
40.02	0.00077577598591312\\
40.03	0.000775778797491338\\
40.04	0.000775781610303155\\
40.05	0.000775784424349349\\
40.06	0.000775787239630702\\
40.07	0.000775790056147996\\
40.08	0.000775792873902014\\
40.09	0.000775795692893539\\
40.1	0.000775798513123356\\
40.11	0.000775801334592249\\
40.12	0.000775804157301005\\
40.13	0.000775806981250411\\
40.14	0.000775809806441255\\
40.15	0.000775812632874325\\
40.16	0.000775815460550411\\
40.17	0.000775818289470303\\
40.18	0.000775821119634791\\
40.19	0.000775823951044667\\
40.2	0.000775826783700724\\
40.21	0.000775829617603757\\
40.22	0.000775832452754558\\
40.23	0.000775835289153924\\
40.24	0.000775838126802649\\
40.25	0.000775840965701532\\
40.26	0.00077584380585137\\
40.27	0.000775846647252961\\
40.28	0.000775849489907107\\
40.29	0.000775852333814605\\
40.3	0.000775855178976258\\
40.31	0.000775858025392869\\
40.32	0.00077586087306524\\
40.33	0.000775863721994174\\
40.34	0.000775866572180478\\
40.35	0.000775869423624957\\
40.36	0.000775872276328417\\
40.37	0.000775875130291667\\
40.38	0.000775877985515512\\
40.39	0.000775880842000766\\
40.4	0.000775883699748235\\
40.41	0.000775886558758734\\
40.42	0.000775889419033072\\
40.43	0.000775892280572064\\
40.44	0.000775895143376525\\
40.45	0.000775898007447267\\
40.46	0.000775900872785108\\
40.47	0.000775903739390862\\
40.48	0.000775906607265351\\
40.49	0.000775909476409391\\
40.5	0.000775912346823802\\
40.51	0.000775915218509407\\
40.52	0.000775918091467025\\
40.53	0.00077592096569748\\
40.54	0.000775923841201596\\
40.55	0.000775926717980196\\
40.56	0.000775929596034107\\
40.57	0.000775932475364156\\
40.58	0.000775935355971169\\
40.59	0.000775938237855977\\
40.6	0.000775941121019407\\
40.61	0.000775944005462293\\
40.62	0.000775946891185464\\
40.63	0.000775949778189754\\
40.64	0.000775952666475996\\
40.65	0.000775955556045026\\
40.66	0.000775958446897681\\
40.67	0.000775961339034795\\
40.68	0.000775964232457209\\
40.69	0.000775967127165759\\
40.7	0.00077597002316129\\
40.71	0.000775972920444639\\
40.72	0.00077597581901665\\
40.73	0.000775978718878167\\
40.74	0.000775981620030034\\
40.75	0.000775984522473098\\
40.76	0.000775987426208205\\
40.77	0.000775990331236202\\
40.78	0.000775993237557939\\
40.79	0.000775996145174268\\
40.8	0.000775999054086039\\
40.81	0.000776001964294105\\
40.82	0.000776004875799319\\
40.83	0.000776007788602536\\
40.84	0.000776010702704612\\
40.85	0.000776013618106405\\
40.86	0.000776016534808775\\
40.87	0.000776019452812578\\
40.88	0.000776022372118679\\
40.89	0.000776025292727937\\
40.9	0.000776028214641216\\
40.91	0.000776031137859382\\
40.92	0.0007760340623833\\
40.93	0.000776036988213838\\
40.94	0.000776039915351863\\
40.95	0.000776042843798244\\
40.96	0.000776045773553854\\
40.97	0.000776048704619564\\
40.98	0.000776051636996247\\
40.99	0.000776054570684779\\
41	0.000776057505686035\\
41.01	0.000776060442000893\\
41.02	0.00077606337963023\\
41.03	0.000776066318574929\\
41.04	0.000776069258835868\\
41.05	0.000776072200413933\\
41.06	0.000776075143310006\\
41.07	0.000776078087524973\\
41.08	0.00077608103305972\\
41.09	0.000776083979915135\\
41.1	0.000776086928092109\\
41.11	0.000776089877591531\\
41.12	0.000776092828414296\\
41.13	0.000776095780561297\\
41.14	0.000776098734033427\\
41.15	0.000776101688831585\\
41.16	0.000776104644956668\\
41.17	0.000776107602409575\\
41.18	0.000776110561191208\\
41.19	0.000776113521302468\\
41.2	0.00077611648274426\\
41.21	0.000776119445517488\\
41.22	0.00077612240962306\\
41.23	0.000776125375061885\\
41.24	0.000776128341834872\\
41.25	0.000776131309942932\\
41.26	0.000776134279386978\\
41.27	0.000776137250167926\\
41.28	0.000776140222286691\\
41.29	0.000776143195744188\\
41.3	0.00077614617054134\\
41.31	0.000776149146679065\\
41.32	0.000776152124158287\\
41.33	0.000776155102979927\\
41.34	0.000776158083144914\\
41.35	0.000776161064654172\\
41.36	0.00077616404750863\\
41.37	0.00077616703170922\\
41.38	0.000776170017256873\\
41.39	0.000776173004152522\\
41.4	0.000776175992397102\\
41.41	0.000776178981991551\\
41.42	0.000776181972936807\\
41.43	0.000776184965233807\\
41.44	0.000776187958883496\\
41.45	0.000776190953886816\\
41.46	0.000776193950244714\\
41.47	0.000776196947958135\\
41.48	0.000776199947028029\\
41.49	0.000776202947455343\\
41.5	0.000776205949241033\\
41.51	0.000776208952386051\\
41.52	0.000776211956891351\\
41.53	0.000776214962757892\\
41.54	0.00077621796998663\\
41.55	0.000776220978578528\\
41.56	0.000776223988534549\\
41.57	0.000776226999855655\\
41.58	0.000776230012542813\\
41.59	0.00077623302659699\\
41.6	0.000776236042019155\\
41.61	0.000776239058810281\\
41.62	0.000776242076971339\\
41.63	0.000776245096503305\\
41.64	0.000776248117407155\\
41.65	0.000776251139683867\\
41.66	0.000776254163334421\\
41.67	0.000776257188359801\\
41.68	0.000776260214760988\\
41.69	0.000776263242538967\\
41.7	0.000776266271694729\\
41.71	0.000776269302229259\\
41.72	0.000776272334143549\\
41.73	0.000776275367438595\\
41.74	0.000776278402115388\\
41.75	0.000776281438174925\\
41.76	0.000776284475618207\\
41.77	0.000776287514446229\\
41.78	0.000776290554659997\\
41.79	0.000776293596260514\\
41.8	0.000776296639248783\\
41.81	0.000776299683625815\\
41.82	0.000776302729392617\\
41.83	0.000776305776550201\\
41.84	0.000776308825099577\\
41.85	0.000776311875041763\\
41.86	0.000776314926377773\\
41.87	0.000776317979108626\\
41.88	0.000776321033235342\\
41.89	0.000776324088758943\\
41.9	0.000776327145680452\\
41.91	0.000776330204000895\\
41.92	0.000776333263721299\\
41.93	0.000776336324842692\\
41.94	0.000776339387366105\\
41.95	0.000776342451292572\\
41.96	0.000776345516623126\\
41.97	0.000776348583358802\\
41.98	0.00077635165150064\\
41.99	0.000776354721049676\\
42	0.000776357792006955\\
42.01	0.000776360864373517\\
42.02	0.000776363938150408\\
42.03	0.000776367013338674\\
42.04	0.000776370089939362\\
42.05	0.000776373167953521\\
42.06	0.000776376247382205\\
42.07	0.000776379328226466\\
42.08	0.000776382410487357\\
42.09	0.000776385494165933\\
42.1	0.000776388579263255\\
42.11	0.00077639166578038\\
42.12	0.000776394753718369\\
42.13	0.000776397843078284\\
42.14	0.000776400933861192\\
42.15	0.000776404026068154\\
42.16	0.000776407119700239\\
42.17	0.000776410214758516\\
42.18	0.000776413311244055\\
42.19	0.000776416409157925\\
42.2	0.0007764195085012\\
42.21	0.000776422609274954\\
42.22	0.000776425711480263\\
42.23	0.000776428815118201\\
42.24	0.000776431920189849\\
42.25	0.000776435026696286\\
42.26	0.000776438134638593\\
42.27	0.00077644124401785\\
42.28	0.000776444354835142\\
42.29	0.000776447467091552\\
42.3	0.000776450580788167\\
42.31	0.000776453695926072\\
42.32	0.000776456812506356\\
42.33	0.000776459930530106\\
42.34	0.000776463049998415\\
42.35	0.000776466170912373\\
42.36	0.000776469293273071\\
42.37	0.000776472417081602\\
42.38	0.000776475542339061\\
42.39	0.000776478669046543\\
42.4	0.000776481797205141\\
42.41	0.000776484926815955\\
42.42	0.00077648805788008\\
42.43	0.000776491190398616\\
42.44	0.000776494324372662\\
42.45	0.000776497459803316\\
42.46	0.00077650059669168\\
42.47	0.000776503735038854\\
42.48	0.000776506874845942\\
42.49	0.000776510016114043\\
42.5	0.000776513158844264\\
42.51	0.000776516303037706\\
42.52	0.000776519448695473\\
42.53	0.00077652259581867\\
42.54	0.000776525744408401\\
42.55	0.000776528894465773\\
42.56	0.000776532045991891\\
42.57	0.000776535198987861\\
42.58	0.00077653835345479\\
42.59	0.000776541509393784\\
42.6	0.000776544666805951\\
42.61	0.000776547825692397\\
42.62	0.00077655098605423\\
42.63	0.000776554147892558\\
42.64	0.000776557311208488\\
42.65	0.00077656047600313\\
42.66	0.00077656364227759\\
42.67	0.000776566810032976\\
42.68	0.000776569979270397\\
42.69	0.000776573149990962\\
42.7	0.000776576322195776\\
42.71	0.000776579495885952\\
42.72	0.000776582671062594\\
42.73	0.000776585847726812\\
42.74	0.000776589025879713\\
42.75	0.000776592205522406\\
42.76	0.000776595386655997\\
42.77	0.000776598569281596\\
42.78	0.000776601753400309\\
42.79	0.000776604939013244\\
42.8	0.000776608126121507\\
42.81	0.000776611314726207\\
42.82	0.00077661450482845\\
42.83	0.000776617696429343\\
42.84	0.000776620889529994\\
42.85	0.000776624084131506\\
42.86	0.000776627280234986\\
42.87	0.000776630477841543\\
42.88	0.000776633676952281\\
42.89	0.000776636877568307\\
42.9	0.000776640079690725\\
42.91	0.00077664328332064\\
42.92	0.000776646488459159\\
42.93	0.000776649695107387\\
42.94	0.000776652903266431\\
42.95	0.000776656112937394\\
42.96	0.000776659324121383\\
42.97	0.000776662536819501\\
42.98	0.000776665751032855\\
42.99	0.00077666896676255\\
43	0.00077667218400969\\
43.01	0.000776675402775383\\
43.02	0.000776678623060734\\
43.03	0.00077668184486685\\
43.04	0.000776685068194836\\
43.05	0.000776688293045799\\
43.06	0.000776691519420846\\
43.07	0.000776694747321085\\
43.08	0.000776697976747624\\
43.09	0.000776701207701569\\
43.1	0.000776704440184031\\
43.11	0.000776707674196121\\
43.12	0.000776710909738947\\
43.13	0.000776714146813622\\
43.14	0.000776717385421255\\
43.15	0.000776720625562961\\
43.16	0.000776723867239851\\
43.17	0.000776727110453042\\
43.18	0.000776730355203649\\
43.19	0.000776733601492788\\
43.2	0.000776736849321575\\
43.21	0.000776740098691131\\
43.22	0.000776743349602576\\
43.23	0.000776746602057029\\
43.24	0.000776749856055614\\
43.25	0.000776753111599452\\
43.26	0.000776756368689669\\
43.27	0.000776759627327392\\
43.28	0.000776762887513747\\
43.29	0.000776766149249862\\
43.3	0.000776769412536868\\
43.31	0.000776772677375893\\
43.32	0.000776775943768073\\
43.33	0.000776779211714538\\
43.34	0.000776782481216425\\
43.35	0.000776785752274866\\
43.36	0.000776789024891001\\
43.37	0.000776792299065968\\
43.38	0.000776795574800907\\
43.39	0.000776798852096958\\
43.4	0.000776802130955262\\
43.41	0.000776805411376962\\
43.42	0.000776808693363204\\
43.43	0.000776811976915133\\
43.44	0.000776815262033895\\
43.45	0.00077681854872064\\
43.46	0.000776821836976516\\
43.47	0.000776825126802674\\
43.48	0.000776828418200266\\
43.49	0.000776831711170446\\
43.5	0.000776835005714367\\
43.51	0.000776838301833184\\
43.52	0.000776841599528054\\
43.53	0.000776844898800136\\
43.54	0.000776848199650588\\
43.55	0.000776851502080572\\
43.56	0.00077685480609125\\
43.57	0.000776858111683782\\
43.58	0.000776861418859335\\
43.59	0.000776864727619073\\
43.6	0.000776868037964165\\
43.61	0.000776871349895776\\
43.62	0.000776874663415077\\
43.63	0.000776877978523237\\
43.64	0.000776881295221429\\
43.65	0.000776884613510826\\
43.66	0.0007768879333926\\
43.67	0.000776891254867927\\
43.68	0.000776894577937986\\
43.69	0.000776897902603952\\
43.7	0.000776901228867005\\
43.71	0.000776904556728325\\
43.72	0.000776907886189093\\
43.73	0.000776911217250492\\
43.74	0.000776914549913707\\
43.75	0.00077691788417992\\
43.76	0.000776921220050322\\
43.77	0.000776924557526096\\
43.78	0.000776927896608434\\
43.79	0.000776931237298523\\
43.8	0.000776934579597556\\
43.81	0.000776937923506726\\
43.82	0.000776941269027226\\
43.83	0.000776944616160249\\
43.84	0.000776947964906995\\
43.85	0.000776951315268658\\
43.86	0.000776954667246438\\
43.87	0.000776958020841535\\
43.88	0.000776961376055148\\
43.89	0.000776964732888481\\
43.9	0.000776968091342735\\
43.91	0.000776971451419117\\
43.92	0.000776974813118831\\
43.93	0.000776978176443085\\
43.94	0.000776981541393086\\
43.95	0.000776984907970044\\
43.96	0.000776988276175169\\
43.97	0.000776991646009672\\
43.98	0.000776995017474769\\
43.99	0.00077699839057167\\
44	0.000777001765301592\\
44.01	0.000777005141665751\\
44.02	0.000777008519665365\\
44.03	0.000777011899301653\\
44.04	0.000777015280575834\\
44.05	0.000777018663489129\\
44.06	0.000777022048042761\\
44.07	0.000777025434237954\\
44.08	0.000777028822075929\\
44.09	0.000777032211557916\\
44.1	0.000777035602685138\\
44.11	0.000777038995458825\\
44.12	0.000777042389880207\\
44.13	0.000777045785950512\\
44.14	0.000777049183670973\\
44.15	0.000777052583042823\\
44.16	0.000777055984067293\\
44.17	0.000777059386745622\\
44.18	0.000777062791079043\\
44.19	0.000777066197068793\\
44.2	0.000777069604716111\\
44.21	0.000777073014022234\\
44.22	0.000777076424988407\\
44.23	0.000777079837615867\\
44.24	0.000777083251905859\\
44.25	0.000777086667859625\\
44.26	0.000777090085478412\\
44.27	0.000777093504763465\\
44.28	0.00077709692571603\\
44.29	0.000777100348337355\\
44.3	0.00077710377262869\\
44.31	0.000777107198591285\\
44.32	0.000777110626226391\\
44.33	0.000777114055535261\\
44.34	0.000777117486519147\\
44.35	0.000777120919179305\\
44.36	0.000777124353516989\\
44.37	0.000777127789533455\\
44.38	0.000777131227229963\\
44.39	0.000777134666607769\\
44.4	0.000777138107668135\\
44.41	0.000777141550412318\\
44.42	0.000777144994841583\\
44.43	0.000777148440957191\\
44.44	0.000777151888760406\\
44.45	0.000777155338252492\\
44.46	0.000777158789434714\\
44.47	0.00077716224230834\\
44.48	0.000777165696874636\\
44.49	0.000777169153134871\\
44.5	0.000777172611090313\\
44.51	0.000777176070742236\\
44.52	0.000777179532091908\\
44.53	0.000777182995140603\\
44.54	0.000777186459889594\\
44.55	0.000777189926340155\\
44.56	0.00077719339449356\\
44.57	0.000777196864351085\\
44.58	0.000777200335914007\\
44.59	0.000777203809183604\\
44.6	0.000777207284161155\\
44.61	0.000777210760847937\\
44.62	0.000777214239245233\\
44.63	0.000777217719354323\\
44.64	0.000777221201176488\\
44.65	0.000777224684713012\\
44.66	0.000777228169965179\\
44.67	0.000777231656934272\\
44.68	0.000777235145621578\\
44.69	0.000777238636028382\\
44.7	0.000777242128155971\\
44.71	0.000777245622005632\\
44.72	0.000777249117578654\\
44.73	0.000777252614876326\\
44.74	0.000777256113899939\\
44.75	0.000777259614650781\\
44.76	0.000777263117130144\\
44.77	0.000777266621339322\\
44.78	0.000777270127279607\\
44.79	0.000777273634952292\\
44.8	0.000777277144358671\\
44.81	0.000777280655500038\\
44.82	0.00077728416837769\\
44.83	0.000777287682992922\\
44.84	0.000777291199347033\\
44.85	0.000777294717441318\\
44.86	0.000777298237277074\\
44.87	0.000777301758855603\\
44.88	0.000777305282178202\\
44.89	0.00077730880724617\\
44.9	0.000777312334060807\\
44.91	0.000777315862623416\\
44.92	0.000777319392935297\\
44.93	0.000777322924997752\\
44.94	0.000777326458812084\\
44.95	0.000777329994379595\\
44.96	0.000777333531701587\\
44.97	0.000777337070779366\\
44.98	0.000777340611614235\\
44.99	0.000777344154207498\\
45	0.000777347698560462\\
45.01	0.000777351244674431\\
45.02	0.00077735479255071\\
45.03	0.000777358342190606\\
45.04	0.000777361893595428\\
45.05	0.000777365446766478\\
45.06	0.000777369001705067\\
45.07	0.0007773725584125\\
45.08	0.000777376116890086\\
45.09	0.000777379677139131\\
45.1	0.000777383239160946\\
45.11	0.000777386802956838\\
45.12	0.000777390368528115\\
45.13	0.000777393935876086\\
45.14	0.00077739750500206\\
45.15	0.000777401075907347\\
45.16	0.000777404648593256\\
45.17	0.000777408223061096\\
45.18	0.000777411799312176\\
45.19	0.000777415377347806\\
45.2	0.000777418957169294\\
45.21	0.000777422538777951\\
45.22	0.000777426122175085\\
45.23	0.000777429707362009\\
45.24	0.000777433294340029\\
45.25	0.000777436883110456\\
45.26	0.000777440473674599\\
45.27	0.000777444066033766\\
45.28	0.000777447660189267\\
45.29	0.000777451256142411\\
45.3	0.000777454853894508\\
45.31	0.000777458453446863\\
45.32	0.000777462054800789\\
45.33	0.000777465657957589\\
45.34	0.000777469262918573\\
45.35	0.000777472869685051\\
45.36	0.000777476478258326\\
45.37	0.000777480088639707\\
45.38	0.0007774837008305\\
45.39	0.000777487314832011\\
45.4	0.000777490930645545\\
45.41	0.000777494548272408\\
45.42	0.000777498167713905\\
45.43	0.000777501788971338\\
45.44	0.000777505412046011\\
45.45	0.000777509036939229\\
45.46	0.000777512663652292\\
45.47	0.000777516292186503\\
45.48	0.000777519922543162\\
45.49	0.000777523554723571\\
45.5	0.000777527188729029\\
45.51	0.000777530824560833\\
45.52	0.000777534462220283\\
45.53	0.000777538101708676\\
45.54	0.000777541743027308\\
45.55	0.000777545386177473\\
45.56	0.000777549031160467\\
45.57	0.000777552677977585\\
45.58	0.000777556326630117\\
45.59	0.000777559977119356\\
45.6	0.000777563629446594\\
45.61	0.000777567283613119\\
45.62	0.000777570939620218\\
45.63	0.00077757459746918\\
45.64	0.00077757825716129\\
45.65	0.000777581918697832\\
45.66	0.000777585582080093\\
45.67	0.000777589247309352\\
45.68	0.000777592914386891\\
45.69	0.000777596583313989\\
45.7	0.000777600254091924\\
45.71	0.000777603926721972\\
45.72	0.000777607601205408\\
45.73	0.000777611277543507\\
45.74	0.000777614955737539\\
45.75	0.000777618635788775\\
45.76	0.000777622317698484\\
45.77	0.000777626001467933\\
45.78	0.000777629687098387\\
45.79	0.000777633374591108\\
45.8	0.000777637063947359\\
45.81	0.000777640755168398\\
45.82	0.000777644448255484\\
45.83	0.000777648143209873\\
45.84	0.000777651840032818\\
45.85	0.000777655538725571\\
45.86	0.000777659239289382\\
45.87	0.000777662941725495\\
45.88	0.000777666646035157\\
45.89	0.000777670352219612\\
45.9	0.000777674060280101\\
45.91	0.000777677770217861\\
45.92	0.000777681482034127\\
45.93	0.000777685195730134\\
45.94	0.00077768891130711\\
45.95	0.000777692628766286\\
45.96	0.000777696348108887\\
45.97	0.000777700069336135\\
45.98	0.000777703792449251\\
45.99	0.000777707517449454\\
46	0.000777711244337956\\
46.01	0.000777714973115972\\
46.02	0.000777718703784708\\
46.03	0.000777722436345371\\
46.04	0.000777726170799164\\
46.05	0.000777729907147286\\
46.06	0.000777733645390934\\
46.07	0.000777737385531302\\
46.08	0.000777741127569578\\
46.09	0.000777744871506952\\
46.1	0.000777748617344604\\
46.11	0.000777752365083716\\
46.12	0.000777756114725464\\
46.13	0.000777759866271021\\
46.14	0.000777763619721558\\
46.15	0.000777767375078238\\
46.16	0.000777771132342222\\
46.17	0.000777774891514671\\
46.18	0.000777778652596737\\
46.19	0.000777782415589571\\
46.2	0.000777786180494318\\
46.21	0.000777789947312122\\
46.22	0.00077779371604412\\
46.23	0.000777797486691446\\
46.24	0.000777801259255228\\
46.25	0.000777805033736593\\
46.26	0.000777808810136661\\
46.27	0.000777812588456547\\
46.28	0.000777816368697365\\
46.29	0.00077782015086022\\
46.3	0.000777823934946216\\
46.31	0.000777827720956449\\
46.32	0.000777831508892013\\
46.33	0.000777835298753995\\
46.34	0.000777839090543478\\
46.35	0.000777842884261542\\
46.36	0.000777846679909256\\
46.37	0.00077785047748769\\
46.38	0.000777854276997907\\
46.39	0.000777858078440963\\
46.4	0.000777861881817908\\
46.41	0.00077786568712979\\
46.42	0.000777869494377649\\
46.43	0.000777873303562519\\
46.44	0.000777877114685431\\
46.45	0.000777880927747407\\
46.46	0.000777884742749464\\
46.47	0.000777888559692616\\
46.48	0.000777892378577866\\
46.49	0.000777896199406214\\
46.5	0.000777900022178655\\
46.51	0.000777903846896173\\
46.52	0.000777907673559749\\
46.53	0.00077791150217036\\
46.54	0.00077791533272897\\
46.55	0.000777919165236542\\
46.56	0.000777922999694029\\
46.57	0.000777926836102382\\
46.58	0.000777930674462537\\
46.59	0.000777934514775432\\
46.6	0.00077793835704199\\
46.61	0.000777942201263135\\
46.62	0.000777946047439776\\
46.63	0.000777949895572821\\
46.64	0.000777953745663166\\
46.65	0.000777957597711702\\
46.66	0.000777961451719312\\
46.67	0.000777965307686871\\
46.68	0.000777969165615247\\
46.69	0.000777973025505301\\
46.7	0.000777976887357885\\
46.71	0.000777980751173841\\
46.72	0.000777984616954008\\
46.73	0.000777988484699212\\
46.74	0.000777992354410274\\
46.75	0.000777996226088004\\
46.76	0.000778000099733207\\
46.77	0.000778003975346676\\
46.78	0.000778007852929198\\
46.79	0.000778011732481551\\
46.8	0.000778015614004503\\
46.81	0.000778019497498816\\
46.82	0.000778023382965239\\
46.83	0.000778027270404514\\
46.84	0.000778031159817375\\
46.85	0.000778035051204546\\
46.86	0.000778038944566739\\
46.87	0.000778042839904663\\
46.88	0.000778046737219013\\
46.89	0.000778050636510473\\
46.9	0.000778054537779724\\
46.91	0.00077805844102743\\
46.92	0.000778062346254249\\
46.93	0.000778066253460828\\
46.94	0.000778070162647805\\
46.95	0.000778074073815809\\
46.96	0.000778077986965458\\
46.97	0.000778081902097357\\
46.98	0.000778085819212106\\
46.99	0.000778089738310289\\
47	0.000778093659392486\\
47.01	0.00077809758245926\\
47.02	0.000778101507511167\\
47.03	0.000778105434548752\\
47.04	0.00077810936357255\\
47.05	0.000778113294583082\\
47.06	0.000778117227580864\\
47.07	0.000778121162566394\\
47.08	0.000778125099540164\\
47.09	0.000778129038502653\\
47.1	0.000778132979454329\\
47.11	0.00077813692239565\\
47.12	0.000778140867327061\\
47.13	0.000778144814248996\\
47.14	0.000778148763161879\\
47.15	0.000778152714066118\\
47.16	0.000778156666962116\\
47.17	0.00077816062185026\\
47.18	0.000778164578730927\\
47.19	0.000778168537604483\\
47.2	0.000778172498471277\\
47.21	0.000778176461331654\\
47.22	0.000778180426185941\\
47.23	0.000778184393034455\\
47.24	0.000778188361877503\\
47.25	0.000778192332715377\\
47.26	0.000778196305548358\\
47.27	0.000778200280376715\\
47.28	0.000778204257200703\\
47.29	0.000778208236020568\\
47.3	0.000778212216836544\\
47.31	0.000778216199648848\\
47.32	0.000778220184457688\\
47.33	0.00077822417126326\\
47.34	0.000778228160065746\\
47.35	0.000778232150865315\\
47.36	0.000778236143662127\\
47.37	0.000778240138456327\\
47.38	0.000778244135248048\\
47.39	0.00077824813403741\\
47.4	0.000778252134824521\\
47.41	0.000778256137609477\\
47.42	0.000778260142392359\\
47.43	0.00077826414917324\\
47.44	0.000778268157952178\\
47.45	0.000778272168729215\\
47.46	0.000778276181504386\\
47.47	0.000778280196277712\\
47.48	0.0007782842130492\\
47.49	0.000778288231818847\\
47.5	0.000778292252586636\\
47.51	0.000778296275352537\\
47.52	0.000778300300116509\\
47.53	0.000778304326878499\\
47.54	0.000778308355638439\\
47.55	0.000778312386396253\\
47.56	0.00077831641915185\\
47.57	0.000778320453905128\\
47.58	0.000778324490655971\\
47.59	0.000778328529404255\\
47.6	0.000778332570149838\\
47.61	0.000778336612892574\\
47.62	0.0007783406576323\\
47.63	0.000778344704368843\\
47.64	0.000778348753102018\\
47.65	0.000778352803831629\\
47.66	0.000778356856557468\\
47.67	0.000778360911279316\\
47.68	0.000778364967996945\\
47.69	0.000778369026710113\\
47.7	0.00077837308741857\\
47.71	0.000778377150122052\\
47.72	0.000778381214820288\\
47.73	0.000778385281512995\\
47.74	0.00077838935019988\\
47.75	0.00077839342088064\\
47.76	0.000778397493554963\\
47.77	0.000778401568222524\\
47.78	0.000778405644882994\\
47.79	0.00077840972353603\\
47.8	0.000778413804181283\\
47.81	0.000778417886818393\\
47.82	0.000778421971446994\\
47.83	0.000778426058066708\\
47.84	0.000778430146677153\\
47.85	0.000778434237277937\\
47.86	0.000778438329868658\\
47.87	0.000778442424448912\\
47.88	0.000778446521018284\\
47.89	0.000778450619576355\\
47.9	0.000778454720122697\\
47.91	0.000778458822656876\\
47.92	0.000778462927178454\\
47.93	0.000778467033686987\\
47.94	0.000778471142182024\\
47.95	0.000778475252663113\\
47.96	0.000778479365129793\\
47.97	0.000778483479581602\\
47.98	0.000778487596018073\\
47.99	0.000778491714438738\\
48	0.000778495834843122\\
48.01	0.000778499957230751\\
48.02	0.000778504081601149\\
48.03	0.000778508207953835\\
48.04	0.000778512336288331\\
48.05	0.000778516466604155\\
48.06	0.000778520598900826\\
48.07	0.000778524733177865\\
48.08	0.00077852886943479\\
48.09	0.000778533007671124\\
48.1	0.00077853714788639\\
48.11	0.000778541290080113\\
48.12	0.000778545434251824\\
48.13	0.000778549580401054\\
48.14	0.000778553728527337\\
48.15	0.000778557878630216\\
48.16	0.000778562030709237\\
48.17	0.000778566184763952\\
48.18	0.000778570340793918\\
48.19	0.000778574498798703\\
48.2	0.000778578658777878\\
48.21	0.000778582820731026\\
48.22	0.000778586984657739\\
48.23	0.000778591150557615\\
48.24	0.00077859531843027\\
48.25	0.000778599488275323\\
48.26	0.000778603660092411\\
48.27	0.000778607833881182\\
48.28	0.000778612009641296\\
48.29	0.000778616187372431\\
48.3	0.000778620367074276\\
48.31	0.00077862454874654\\
48.32	0.000778628732388946\\
48.33	0.000778632918001237\\
48.34	0.000778637105583173\\
48.35	0.000778641295134535\\
48.36	0.000778645486655122\\
48.37	0.000778649680144758\\
48.38	0.000778653875603283\\
48.39	0.000778658073030567\\
48.4	0.000778662272426503\\
48.41	0.000778666473791004\\
48.42	0.000778670677124015\\
48.43	0.000778674882425506\\
48.44	0.000778679089695474\\
48.45	0.000778683298933946\\
48.46	0.00077868751014098\\
48.47	0.000778691723316663\\
48.48	0.000778695938461119\\
48.49	0.000778700155574501\\
48.5	0.000778704374656998\\
48.51	0.000778708595708836\\
48.52	0.000778712818730275\\
48.53	0.000778717043721617\\
48.54	0.000778721270683199\\
48.55	0.000778725499615402\\
48.56	0.000778729730518646\\
48.57	0.000778733963393395\\
48.58	0.000778738198240156\\
48.59	0.000778742435059482\\
48.6	0.000778746673851973\\
48.61	0.000778750914618273\\
48.62	0.000778755157359081\\
48.63	0.000778759402075143\\
48.64	0.000778763648767255\\
48.65	0.000778767897436269\\
48.66	0.000778772148083088\\
48.67	0.000778776400708671\\
48.68	0.000778780655314037\\
48.69	0.00077878491190026\\
48.7	0.000778789170468471\\
48.71	0.000778793431019869\\
48.72	0.000778797693555709\\
48.73	0.000778801958077311\\
48.74	0.00077880622458606\\
48.75	0.000778810493083407\\
48.76	0.000778814763570871\\
48.77	0.000778819036050038\\
48.78	0.000778823310522567\\
48.79	0.000778827586990187\\
48.8	0.0007788318654547\\
48.81	0.000778836145917982\\
48.82	0.000778840428381986\\
48.83	0.000778844712848741\\
48.84	0.000778848999320354\\
48.85	0.000778853287799012\\
48.86	0.000778857578286983\\
48.87	0.000778861870786619\\
48.88	0.000778866165300351\\
48.89	0.000778870461830701\\
48.9	0.000778874760380275\\
48.91	0.000778879060951764\\
48.92	0.00077888336354795\\
48.93	0.000778887668171704\\
48.94	0.000778891974825993\\
48.95	0.00077889628351387\\
48.96	0.000778900594238484\\
48.97	0.000778904907003079\\
48.98	0.000778909221810996\\
48.99	0.000778913538665674\\
49	0.000778917857570645\\
49.01	0.000778922178529547\\
49.02	0.000778926501546113\\
49.03	0.000778930826624181\\
49.04	0.000778935153767689\\
49.05	0.000778939482980679\\
49.06	0.000778943814267297\\
49.07	0.000778948147631792\\
49.08	0.000778952483078523\\
49.09	0.000778956820611951\\
49.1	0.000778961160236645\\
49.11	0.000778965501957284\\
49.12	0.000778969845778655\\
49.13	0.00077897419170565\\
49.14	0.000778978539743274\\
49.15	0.000778982889896642\\
49.16	0.000778987242170976\\
49.17	0.000778991596571613\\
49.18	0.000778995953103996\\
49.19	0.00077900031177368\\
49.2	0.000779004672586332\\
49.21	0.00077900903554773\\
49.22	0.000779013400663761\\
49.23	0.000779017767940425\\
49.24	0.000779022137383827\\
49.25	0.00077902650900019\\
49.26	0.00077903088279584\\
49.27	0.000779035258777214\\
49.28	0.000779039636950859\\
49.29	0.000779044017323427\\
49.3	0.000779048399901679\\
49.31	0.00077905278469248\\
49.32	0.000779057171702801\\
49.33	0.000779061560939716\\
49.34	0.000779065952410401\\
49.35	0.000779070346122134\\
49.36	0.000779074742082293\\
49.37	0.000779079140298351\\
49.38	0.000779083540777882\\
49.39	0.00077908794352855\\
49.4	0.000779092348558112\\
49.41	0.000779096755874415\\
49.42	0.000779101165485397\\
49.43	0.000779105577399077\\
49.44	0.000779109991623559\\
49.45	0.000779114408167027\\
49.46	0.000779118827037741\\
49.47	0.000779123248244034\\
49.48	0.000779127671794316\\
49.49	0.000779132097697059\\
49.5	0.000779136525960801\\
49.51	0.00077914095659414\\
49.52	0.000779145389605735\\
49.53	0.000779149825004292\\
49.54	0.000779154262798572\\
49.55	0.000779158702997374\\
49.56	0.000779163145609545\\
49.57	0.000779167590643962\\
49.58	0.000779172038109536\\
49.59	0.000779176488015204\\
49.6	0.000779180940369923\\
49.61	0.00077918539518267\\
49.62	0.000779189852462428\\
49.63	0.000779194312218191\\
49.64	0.000779198774458948\\
49.65	0.000779203239193688\\
49.66	0.000779207706431383\\
49.67	0.00077921217618099\\
49.68	0.000779216648451443\\
49.69	0.000779221123251647\\
49.7	0.000779225600590471\\
49.71	0.000779230080476742\\
49.72	0.000779234562919237\\
49.73	0.000779239047926679\\
49.74	0.000779243535507729\\
49.75	0.000779248025670981\\
49.76	0.000779252518424952\\
49.77	0.000779257013778081\\
49.78	0.000779261511738715\\
49.79	0.00077926601231511\\
49.8	0.000779270515515415\\
49.81	0.000779275021347678\\
49.82	0.000779279529819828\\
49.83	0.000779284040939673\\
49.84	0.000779288554714899\\
49.85	0.000779293071153051\\
49.86	0.000779297590261539\\
49.87	0.000779302112047629\\
49.88	0.000779306636518436\\
49.89	0.000779311163680917\\
49.9	0.000779315693541872\\
49.91	0.000779320226107934\\
49.92	0.000779324761385571\\
49.93	0.000779329299381071\\
49.94	0.000779333840100552\\
49.95	0.000779338383549951\\
49.96	0.000779342929735029\\
49.97	0.00077934747866136\\
49.98	0.000779352030334338\\
49.99	0.000779356584759174\\
50	0.000779361141940899\\
50.01	0.000779365701884367\\
50.02	0.000779370264594254\\
50.03	0.000779374830075064\\
50.04	0.000779379398331134\\
50.05	0.000779383969366642\\
50.06	0.000779388543185617\\
50.07	0.00077939311979194\\
50.08	0.000779397699189366\\
50.09	0.000779402281381528\\
50.1	0.000779406866371957\\
50.11	0.000779411454164096\\
50.12	0.000779416044761321\\
50.13	0.000779420638166958\\
50.14	0.000779425234384309\\
50.15	0.000779429833416679\\
50.16	0.000779434435267382\\
50.17	0.000779439039939735\\
50.18	0.00077944364743707\\
50.19	0.000779448257762723\\
50.2	0.000779452870920039\\
50.21	0.000779457486912374\\
50.22	0.000779462105743089\\
50.23	0.000779466727415558\\
50.24	0.00077947135193316\\
50.25	0.000779475979299287\\
50.26	0.000779480609517338\\
50.27	0.000779485242590723\\
50.28	0.00077948987852286\\
50.29	0.000779494517317177\\
50.3	0.000779499158977114\\
50.31	0.000779503803506118\\
50.32	0.000779508450907651\\
50.33	0.000779513101185181\\
50.34	0.000779517754342188\\
50.35	0.000779522410382165\\
50.36	0.000779527069308613\\
50.37	0.000779531731125048\\
50.38	0.000779536395834992\\
50.39	0.000779541063441984\\
50.4	0.000779545733949571\\
50.41	0.000779550407361313\\
50.42	0.000779555083680784\\
50.43	0.000779559762911568\\
50.44	0.000779564445057264\\
50.45	0.00077956913012148\\
50.46	0.000779573818107842\\
50.47	0.000779578509019984\\
50.48	0.000779583202861554\\
50.49	0.000779587899636216\\
50.5	0.000779592599347645\\
50.51	0.000779597301999531\\
50.52	0.000779602007595576\\
50.53	0.000779606716139497\\
50.54	0.000779611427635024\\
50.55	0.000779616142085902\\
50.56	0.000779620859495889\\
50.57	0.00077962557986876\\
50.58	0.000779630303208299\\
50.59	0.00077963502951831\\
50.6	0.000779639758802611\\
50.61	0.000779644491065034\\
50.62	0.000779649226309425\\
50.63	0.000779653964539646\\
50.64	0.000779658705759575\\
50.65	0.000779663449973106\\
50.66	0.000779668197184146\\
50.67	0.000779672947396623\\
50.68	0.000779677700614477\\
50.69	0.000779682456841664\\
50.7	0.000779687216082159\\
50.71	0.00077969197833995\\
50.72	0.000779696743619046\\
50.73	0.000779701511923471\\
50.74	0.000779706283257266\\
50.75	0.000779711057624489\\
50.76	0.000779715835029216\\
50.77	0.000779720615475541\\
50.78	0.000779725398967576\\
50.79	0.000779730185509451\\
50.8	0.000779734975105311\\
50.81	0.000779739767759326\\
50.82	0.00077974456347568\\
50.83	0.000779749362258575\\
50.84	0.000779754164112237\\
50.85	0.000779758969040907\\
50.86	0.000779763777048846\\
50.87	0.000779768588140339\\
50.88	0.000779773402319685\\
50.89	0.000779778219591208\\
50.9	0.000779783039959249\\
50.91	0.00077978786342817\\
50.92	0.000779792690002357\\
50.93	0.000779797519686215\\
50.94	0.00077980235248417\\
50.95	0.00077980718840067\\
50.96	0.000779812027440185\\
50.97	0.000779816869607207\\
50.98	0.000779821714906251\\
50.99	0.000779826563341853\\
51	0.000779831414918575\\
51.01	0.000779836269641\\
51.02	0.000779841127513734\\
51.03	0.000779845988541408\\
51.04	0.000779850852728676\\
51.05	0.000779855720080217\\
51.06	0.000779860590600734\\
51.07	0.000779865464294953\\
51.08	0.000779870341167629\\
51.09	0.000779875221223542\\
51.1	0.000779880104467496\\
51.11	0.000779884990904318\\
51.12	0.00077988988053887\\
51.13	0.000779894773376031\\
51.14	0.000779899669420714\\
51.15	0.000779904568677856\\
51.16	0.000779909471152421\\
51.17	0.000779914376849405\\
51.18	0.000779919285773829\\
51.19	0.000779924197930741\\
51.2	0.000779929113325223\\
51.21	0.000779934031962382\\
51.22	0.000779938953847356\\
51.23	0.000779943878985315\\
51.24	0.000779948807381456\\
51.25	0.000779953739041009\\
51.26	0.000779958673969235\\
51.27	0.000779963612171428\\
51.28	0.000779968553652909\\
51.29	0.000779973498419037\\
51.3	0.000779978446475199\\
51.31	0.00077998339782682\\
51.32	0.000779988352479355\\
51.33	0.000779993310438293\\
51.34	0.000779998271709159\\
51.35	0.000780003236297512\\
51.36	0.000780008204208947\\
51.37	0.000780013175449092\\
51.38	0.000780018150023615\\
51.39	0.000780023127938219\\
51.4	0.00078002810919864\\
51.41	0.000780033093810658\\
51.42	0.000780038081780088\\
51.43	0.000780043073112783\\
51.44	0.000780048067814634\\
51.45	0.000780053065891574\\
51.46	0.000780058067349576\\
51.47	0.000780063072194651\\
51.48	0.00078006808043285\\
51.49	0.000780073092070269\\
51.5	0.000780078107113044\\
51.51	0.000780083125567354\\
51.52	0.000780088147439422\\
51.53	0.000780093172735511\\
51.54	0.000780098201461932\\
51.55	0.000780103233625038\\
51.56	0.000780108269231229\\
51.57	0.00078011330828695\\
51.58	0.000780118350798692\\
51.59	0.000780123396772994\\
51.6	0.000780128446216442\\
51.61	0.000780133499135671\\
51.62	0.000780138555537362\\
51.63	0.000780143615428251\\
51.64	0.000780148678815117\\
51.65	0.000780153745704793\\
51.66	0.000780158816104168\\
51.67	0.000780163890020176\\
51.68	0.000780168967459806\\
51.69	0.000780174048430102\\
51.7	0.000780179132938159\\
51.71	0.000780184220991131\\
51.72	0.000780189312596226\\
51.73	0.000780194407760703\\
51.74	0.000780199506491887\\
51.75	0.000780204608797153\\
51.76	0.000780209714683937\\
51.77	0.000780214824159737\\
51.78	0.000780219937232107\\
51.79	0.000780225053908663\\
51.8	0.000780230174197084\\
51.81	0.000780235298105107\\
51.82	0.000780240425640538\\
51.83	0.000780245556811244\\
51.84	0.000780250691625154\\
51.85	0.000780255830090268\\
51.86	0.000780260972214649\\
51.87	0.000780266118006427\\
51.88	0.000780271267473804\\
51.89	0.000780276420625047\\
51.9	0.000780281577468494\\
51.91	0.000780286738012555\\
51.92	0.000780291902265712\\
51.93	0.000780297070236519\\
51.94	0.000780302241933603\\
51.95	0.000780307417365669\\
51.96	0.000780312596541493\\
51.97	0.00078031777946993\\
51.98	0.000780322966159914\\
51.99	0.000780328156620456\\
52	0.000780333350860647\\
52.01	0.00078033854888966\\
52.02	0.000780343750716748\\
52.03	0.00078034895635125\\
52.04	0.000780354165802584\\
52.05	0.000780359379080258\\
52.06	0.000780364596193864\\
52.07	0.000780369817153081\\
52.08	0.000780375041967677\\
52.09	0.000780380270647512\\
52.1	0.000780385503202535\\
52.11	0.000780390739642785\\
52.12	0.000780395979978398\\
52.13	0.000780401224219605\\
52.14	0.000780406472376727\\
52.15	0.000780411724460188\\
52.16	0.000780416980480508\\
52.17	0.000780422240448306\\
52.18	0.000780427504374305\\
52.19	0.000780432772269326\\
52.2	0.000780438044144297\\
52.21	0.00078044332001025\\
52.22	0.000780448599878322\\
52.23	0.00078045388375976\\
52.24	0.00078045917166592\\
52.25	0.000780464463608266\\
52.26	0.00078046975959838\\
52.27	0.000780475059647949\\
52.28	0.000780480363768784\\
52.29	0.000780485671972806\\
52.3	0.000780490984272058\\
52.31	0.000780496300678701\\
52.32	0.000780501621205019\\
52.33	0.000780506945863417\\
52.34	0.000780512274666426\\
52.35	0.000780517607626702\\
52.36	0.00078052294475703\\
52.37	0.000780528286070323\\
52.38	0.000780533631579627\\
52.39	0.000780538981298122\\
52.4	0.000780544335239119\\
52.41	0.000780549693416068\\
52.42	0.000780555055842557\\
52.43	0.000780560422532314\\
52.44	0.000780565793499211\\
52.45	0.000780571168757262\\
52.46	0.000780576548320625\\
52.47	0.000780581932203608\\
52.48	0.00078058732042067\\
52.49	0.00078059271298642\\
52.5	0.000780598109915619\\
52.51	0.000780603511223185\\
52.52	0.000780608916924195\\
52.53	0.000780614327033885\\
52.54	0.000780619741567653\\
52.55	0.00078062516054106\\
52.56	0.000780630583969833\\
52.57	0.000780636011869871\\
52.58	0.000780641444257237\\
52.59	0.000780646881148173\\
52.6	0.000780652322559092\\
52.61	0.000780657768506587\\
52.62	0.000780663219007431\\
52.63	0.000780668674078576\\
52.64	0.000780674133737161\\
52.65	0.00078067959800051\\
52.66	0.000780685066886139\\
52.67	0.000780690540411754\\
52.68	0.000780696018595255\\
52.69	0.000780701501454739\\
52.7	0.000780706989008503\\
52.71	0.000780712481275044\\
52.72	0.000780717978273067\\
52.73	0.000780723480021483\\
52.74	0.000780728986539412\\
52.75	0.000780734497846186\\
52.76	0.000780740013961355\\
52.77	0.000780745534904688\\
52.78	0.00078075106069617\\
52.79	0.000780756591356015\\
52.8	0.000780762126904663\\
52.81	0.000780767667362781\\
52.82	0.000780773212751273\\
52.83	0.000780778763091275\\
52.84	0.000780784318404166\\
52.85	0.000780789878711562\\
52.86	0.00078079544403533\\
52.87	0.00078080101439758\\
52.88	0.000780806589820675\\
52.89	0.000780812170327235\\
52.9	0.000780817755940138\\
52.91	0.000780823346682519\\
52.92	0.000780828942577782\\
52.93	0.000780834543649597\\
52.94	0.000780840149921906\\
52.95	0.000780845761418927\\
52.96	0.000780851378165155\\
52.97	0.000780857000185368\\
52.98	0.000780862627504632\\
52.99	0.000780868260148296\\
53	0.000780873898142008\\
53.01	0.000780879541511712\\
53.02	0.00078088519028365\\
53.03	0.000780890844484371\\
53.04	0.000780896504140729\\
53.05	0.000780902169279894\\
53.06	0.000780907839929351\\
53.07	0.000780913516116902\\
53.08	0.000780919197870678\\
53.09	0.000780924885219136\\
53.1	0.000780930578191065\\
53.11	0.000780936276815592\\
53.12	0.000780941981122184\\
53.13	0.000780947691140652\\
53.14	0.000780953406901159\\
53.15	0.000780959128434221\\
53.16	0.00078096485577071\\
53.17	0.000780970588941863\\
53.18	0.000780976327979285\\
53.19	0.000780982072914953\\
53.2	0.000780987823781217\\
53.21	0.000780993580610812\\
53.22	0.000780999343436856\\
53.23	0.000781005112292862\\
53.24	0.000781010887212734\\
53.25	0.00078101666823078\\
53.26	0.000781022455381711\\
53.27	0.000781028248700648\\
53.28	0.000781034048223132\\
53.29	0.000781039853985118\\
53.3	0.000781045666022991\\
53.31	0.000781051484373569\\
53.32	0.0007810573090741\\
53.33	0.000781063140162278\\
53.34	0.000781068977676243\\
53.35	0.000781074821654588\\
53.36	0.00078108067213636\\
53.37	0.000781086529161076\\
53.38	0.000781092392768719\\
53.39	0.000781098262999744\\
53.4	0.000781104139895092\\
53.41	0.000781110023496184\\
53.42	0.000781115913844938\\
53.43	0.000781121810983767\\
53.44	0.000781127714955591\\
53.45	0.000781133625803836\\
53.46	0.000781139543572448\\
53.47	0.000781145468305891\\
53.48	0.000781151400049163\\
53.49	0.000781157338847793\\
53.5	0.00078116328474785\\
53.51	0.000781169237795953\\
53.52	0.000781175198039273\\
53.53	0.000781181165525542\\
53.54	0.00078118714030306\\
53.55	0.000781193122420698\\
53.56	0.000781199111927909\\
53.57	0.000781205108874736\\
53.58	0.000781211113311808\\
53.59	0.000781217125290362\\
53.6	0.000781223144862238\\
53.61	0.000781229172079893\\
53.62	0.000781235206996406\\
53.63	0.000781241249665483\\
53.64	0.000781247300141468\\
53.65	0.000781253358479346\\
53.66	0.000781259424734754\\
53.67	0.000781265498963988\\
53.68	0.000781271581224006\\
53.69	0.000781277671572444\\
53.7	0.000781283770067615\\
53.71	0.000781289876768518\\
53.72	0.000781295991734855\\
53.73	0.000781302115027025\\
53.74	0.000781308246706141\\
53.75	0.000781314386834035\\
53.76	0.000781320535473267\\
53.77	0.000781326692687129\\
53.78	0.00078133285853966\\
53.79	0.000781339033095652\\
53.8	0.000781345216420651\\
53.81	0.000781351408580978\\
53.82	0.000781357609643721\\
53.83	0.000781363819676761\\
53.84	0.00078137003874877\\
53.85	0.000781376266929219\\
53.86	0.000781382504288391\\
53.87	0.000781388750897387\\
53.88	0.000781395006828136\\
53.89	0.000781401272153405\\
53.9	0.000781407546946802\\
53.91	0.000781413831282793\\
53.92	0.000781420125236703\\
53.93	0.00078142642888473\\
53.94	0.000781432742303953\\
53.95	0.000781439065572343\\
53.96	0.000781445398768765\\
53.97	0.000781451741972998\\
53.98	0.000781458095265732\\
53.99	0.000781464458728587\\
54	0.000781470832444122\\
54.01	0.000781477216495834\\
54.02	0.00078148361096818\\
54.03	0.00078149001594658\\
54.04	0.000781496431517427\\
54.05	0.000781502857768099\\
54.06	0.000781509294786965\\
54.07	0.000781515742663398\\
54.08	0.000781522201487781\\
54.09	0.000781528671351523\\
54.1	0.00078153515234706\\
54.11	0.000781541644567873\\
54.12	0.000781548148108492\\
54.13	0.00078155466306451\\
54.14	0.000781561189532589\\
54.15	0.000781567727610476\\
54.16	0.000781574277397007\\
54.17	0.000781580838992116\\
54.18	0.000781587412496853\\
54.19	0.000781593998013386\\
54.2	0.000781600595645015\\
54.21	0.000781607205496181\\
54.22	0.000781613827672478\\
54.23	0.000781620462280658\\
54.24	0.000781627109428648\\
54.25	0.000781633769225555\\
54.26	0.000781640441781677\\
54.27	0.000781647127208514\\
54.28	0.000781653825618782\\
54.29	0.000781660537126413\\
54.3	0.000781667261846575\\
54.31	0.00078167399989568\\
54.32	0.000781680751391388\\
54.33	0.000781687516452625\\
54.34	0.000781694295199588\\
54.35	0.000781701087753758\\
54.36	0.000781707894237909\\
54.37	0.000781714714776116\\
54.38	0.000781721549493767\\
54.39	0.000781728398517575\\
54.4	0.000781735261975584\\
54.41	0.000781742139997181\\
54.42	0.000781749032713103\\
54.43	0.000781755940255452\\
54.44	0.000781762862757702\\
54.45	0.000781769800354706\\
54.46	0.000781776753182709\\
54.47	0.000781783721379357\\
54.48	0.000781790705083704\\
54.49	0.000781797704436227\\
54.5	0.000781804719578828\\
54.51	0.000781811750654848\\
54.52	0.000781818797809074\\
54.53	0.000781825861187751\\
54.54	0.000781832940938587\\
54.55	0.000781840037210763\\
54.56	0.000781847150154943\\
54.57	0.000781854279923283\\
54.58	0.000781861426669434\\
54.59	0.00078186859054856\\
54.6	0.000781875771717334\\
54.61	0.000781882970333957\\
54.62	0.00078189018655816\\
54.63	0.000781897420551212\\
54.64	0.000781904672475931\\
54.65	0.000781911942496683\\
54.66	0.000781919230779402\\
54.67	0.000781926537491583\\
54.68	0.000781933862802299\\
54.69	0.000781941206882202\\
54.7	0.000781948569903535\\
54.71	0.00078195595204013\\
54.72	0.000781963353467419\\
54.73	0.000781970774362439\\
54.74	0.000781978214903837\\
54.75	0.000781985675271877\\
54.76	0.000781993155648442\\
54.77	0.000782000656217037\\
54.78	0.000782008177162801\\
54.79	0.0007820157186725\\
54.8	0.000782023280934544\\
54.81	0.000782030864138979\\
54.82	0.000782038468477493\\
54.83	0.000782046094143425\\
54.84	0.000782053741331759\\
54.85	0.000782061410239133\\
54.86	0.000782069101063836\\
54.87	0.00078207681400581\\
54.88	0.000782084549266657\\
54.89	0.00078209230704963\\
54.9	0.000782100087559643\\
54.91	0.000782107891003263\\
54.92	0.000782115717588716\\
54.93	0.00078212356752588\\
54.94	0.00078213144102629\\
54.95	0.000782139338303131\\
54.96	0.00078214725957124\\
54.97	0.000782155205047099\\
54.98	0.000782163174948836\\
54.99	0.000782171169496219\\
55	0.000782179188910654\\
55.01	0.000782187233415175\\
55.02	0.000782195303234447\\
55.03	0.000782203398594757\\
55.04	0.000782211519724001\\
55.05	0.000782219666851691\\
55.06	0.000782227840208935\\
55.07	0.000782236040028439\\
55.08	0.000782244266544489\\
55.09	0.00078225251999295\\
55.1	0.000782260800611253\\
55.11	0.000782269108638385\\
55.12	0.00078227744431488\\
55.13	0.000782285807882801\\
55.14	0.000782294199585741\\
55.15	0.000782302619668796\\
55.16	0.000782311068378559\\
55.17	0.000782319545963106\\
55.18	0.000782328052671982\\
55.19	0.000782336588756177\\
55.2	0.000782345154468123\\
55.21	0.000782353750061665\\
55.22	0.000782362375792054\\
55.23	0.00078237103191592\\
55.24	0.000782379718691259\\
55.25	0.000782388436377408\\
55.26	0.000782397185235031\\
55.27	0.00078240596552609\\
55.28	0.000782414777513826\\
55.29	0.000782423621462741\\
55.3	0.000782432497638569\\
55.31	0.000782441406308249\\
55.32	0.000782450347739909\\
55.33	0.00078245932220283\\
55.34	0.000782468329967426\\
55.35	0.000782477371305214\\
55.36	0.000782486446488786\\
55.37	0.000782495555791777\\
55.38	0.000782504699488838\\
55.39	0.000782513877855606\\
55.4	0.00078252309116867\\
55.41	0.000782532339705537\\
55.42	0.000782541623744606\\
55.43	0.000782550943565124\\
55.44	0.000782560299447158\\
55.45	0.000782569691671557\\
55.46	0.000782579120519917\\
55.47	0.000782588586274542\\
55.48	0.000782598089218406\\
55.49	0.000782607629635116\\
55.5	0.000782617207808872\\
55.51	0.000782626824024427\\
55.52	0.000782636478567043\\
55.53	0.000782646171722455\\
55.54	0.000782655903776826\\
55.55	0.000782665675016702\\
55.56	0.000782675485728974\\
55.57	0.000782685336200829\\
55.58	0.000782695226719707\\
55.59	0.000782705157573258\\
55.6	0.000782715129049292\\
55.61	0.000782725141435738\\
55.62	0.000782735195020597\\
55.63	0.000782745290091889\\
55.64	0.000782755426937615\\
55.65	0.000782765605845703\\
55.66	0.000782775827103965\\
55.67	0.000782786091000045\\
55.68	0.000782796397821375\\
55.69	0.000782806747855126\\
55.7	0.000782817141388158\\
55.71	0.000782827578706978\\
55.72	0.000782838060097686\\
55.73	0.000782848585845933\\
55.74	0.000782859156236873\\
55.75	0.000782869771555113\\
55.76	0.000782880432084673\\
55.77	0.00078289113810894\\
55.78	0.00078290188991062\\
55.79	0.000782912687771696\\
55.8	0.000782923531973389\\
55.81	0.000782934422796113\\
55.82	0.000782945360519441\\
55.83	0.000782956345422055\\
55.84	0.000782967377781723\\
55.85	0.000782978457875253\\
55.86	0.000782989585978467\\
55.87	0.000783000762366166\\
55.88	0.0007830119873121\\
55.89	0.000783023261088948\\
55.9	0.000783034583968289\\
55.91	0.00078304595622058\\
55.92	0.00078305737811514\\
55.93	0.000783068849920141\\
55.94	0.000783080371902589\\
55.95	0.00078309194432832\\
55.96	0.000783103567461998\\
55.97	0.000783115241567116\\
55.98	0.000783126966906003\\
55.99	0.000783138743739833\\
56	0.000783150572328643\\
56.01	0.000783162452931356\\
56.02	0.000783174385805809\\
56.03	0.000783186371208784\\
56.04	0.000783198409396056\\
56.05	0.000783210500622432\\
56.06	0.000783222645141819\\
56.07	0.000783234843207277\\
56.08	0.000783247095071099\\
56.09	0.000783259400984889\\
56.1	0.000783271761199656\\
56.11	0.000783284175965909\\
56.12	0.000783296645533775\\
56.13	0.00078330917015312\\
56.14	0.000783321750073684\\
56.15	0.000783334385545221\\
56.16	0.000783347076817669\\
56.17	0.000783359824141317\\
56.18	0.000783372627766992\\
56.19	0.000783385487946266\\
56.2	0.000783398404931669\\
56.21	0.000783411378976931\\
56.22	0.000783424410337226\\
56.23	0.000783437499269448\\
56.24	0.000783450646032505\\
56.25	0.000783463850887617\\
56.26	0.000783477114098659\\
56.27	0.000783490435928048\\
56.28	0.000783503816628239\\
56.29	0.000783517256441201\\
56.3	0.000783530755597889\\
56.31	0.000783544314317684\\
56.32	0.000783557932807816\\
56.33	0.00078357161126278\\
56.34	0.000783585349863716\\
56.35	0.000783599148777776\\
56.36	0.00078361300815747\\
56.37	0.000783626928139988\\
56.38	0.000783640908846495\\
56.39	0.000783654950381414\\
56.4	0.000783669052831676\\
56.41	0.000783683216265941\\
56.42	0.000783697440733807\\
56.43	0.00078371172626498\\
56.44	0.000783726072868425\\
56.45	0.000783740480531481\\
56.46	0.000783754949218955\\
56.47	0.000783769478872182\\
56.48	0.000783784069408051\\
56.49	0.000783798720718012\\
56.5	0.00078381343266703\\
56.51	0.000783828205092528\\
56.52	0.000783843037803274\\
56.53	0.000783857930578251\\
56.54	0.000783872883165476\\
56.55	0.000783887895280793\\
56.56	0.000783902966606612\\
56.57	0.000783918096790626\\
56.58	0.000783933285444472\\
56.59	0.000783948532142358\\
56.6	0.000783963836419635\\
56.61	0.000783979197771345\\
56.62	0.000783994615650695\\
56.63	0.000784010089467506\\
56.64	0.000784025618586599\\
56.65	0.000784041202326139\\
56.66	0.000784056839955916\\
56.67	0.00078407253069558\\
56.68	0.000784088273712824\\
56.69	0.000784104068121497\\
56.7	0.000784119912979667\\
56.71	0.000784135807287624\\
56.72	0.000784151749985815\\
56.73	0.000784167739952716\\
56.74	0.000784183776002643\\
56.75	0.000784199856883487\\
56.76	0.000784215981274384\\
56.77	0.000784232147783309\\
56.78	0.000784248354944598\\
56.79	0.000784264601216391\\
56.8	0.000784280884977999\\
56.81	0.000784297204527183\\
56.82	0.000784313558077359\\
56.83	0.000784329943754703\\
56.84	0.000784346359595185\\
56.85	0.000784362803541494\\
56.86	0.000784379273439875\\
56.87	0.000784395767036873\\
56.88	0.000784412281975976\\
56.89	0.000784428815794143\\
56.9	0.000784445365918245\\
56.91	0.000784461929661385\\
56.92	0.000784478504219102\\
56.93	0.000784495086665471\\
56.94	0.000784511676594733\\
56.95	0.000784528274011972\\
56.96	0.000784544878922277\\
56.97	0.000784561491330741\\
56.98	0.000784578111242464\\
56.99	0.000784594738662549\\
57	0.000784611373596106\\
57.01	0.00078462801604825\\
57.02	0.000784644666024101\\
57.03	0.000784661323528784\\
57.04	0.00078467798856743\\
57.05	0.000784694661145174\\
57.06	0.000784711341267157\\
57.07	0.000784728028938525\\
57.08	0.00078474472416443\\
57.09	0.000784761426950029\\
57.1	0.000784778137300483\\
57.11	0.00078479485522096\\
57.12	0.000784811580716633\\
57.13	0.000784828313792679\\
57.14	0.000784845054454282\\
57.15	0.000784861802706632\\
57.16	0.000784878558554922\\
57.17	0.000784895322004351\\
57.18	0.000784912093060124\\
57.19	0.000784928871727451\\
57.2	0.000784945658011548\\
57.21	0.000784962451917635\\
57.22	0.000784979253450939\\
57.23	0.000784996062616691\\
57.24	0.000785012879420127\\
57.25	0.000785029703866491\\
57.26	0.00078504653596103\\
57.27	0.000785063375708998\\
57.28	0.000785080223115653\\
57.29	0.00078509707818626\\
57.3	0.000785113940926086\\
57.31	0.000785130811340408\\
57.32	0.000785147689434508\\
57.33	0.000785164575213668\\
57.34	0.000785181468683182\\
57.35	0.000785198369848345\\
57.36	0.000785215278714462\\
57.37	0.000785232195286839\\
57.38	0.000785249119570789\\
57.39	0.000785266051571632\\
57.4	0.000785282991294692\\
57.41	0.000785299938745298\\
57.42	0.000785316893928786\\
57.43	0.000785333856850498\\
57.44	0.000785350827515779\\
57.45	0.000785367805929981\\
57.46	0.000785384792098462\\
57.47	0.000785401786026584\\
57.48	0.000785418787719717\\
57.49	0.000785435797183235\\
57.5	0.00078545281442252\\
57.51	0.000785469839442953\\
57.52	0.000785486872249929\\
57.53	0.000785503912848843\\
57.54	0.000785520961245099\\
57.55	0.000785538017444102\\
57.56	0.000785555081451268\\
57.57	0.000785572153272015\\
57.58	0.000785589232911768\\
57.59	0.000785606320375958\\
57.6	0.00078562341567002\\
57.61	0.000785640518799398\\
57.62	0.000785657629769539\\
57.63	0.000785674748585895\\
57.64	0.000785691875253926\\
57.65	0.000785709009779097\\
57.66	0.000785726152166877\\
57.67	0.000785743302422743\\
57.68	0.000785760460552178\\
57.69	0.000785777626560668\\
57.7	0.000785794800453706\\
57.71	0.000785811982236793\\
57.72	0.000785829171915433\\
57.73	0.000785846369495137\\
57.74	0.000785863574981421\\
57.75	0.000785880788379807\\
57.76	0.000785898009695822\\
57.77	0.000785915238935002\\
57.78	0.000785932476102884\\
57.79	0.000785949721205015\\
57.8	0.000785966974246946\\
57.81	0.000785984235234235\\
57.82	0.000786001504172443\\
57.83	0.000786018781067141\\
57.84	0.000786036065923902\\
57.85	0.000786053358748307\\
57.86	0.000786070659545944\\
57.87	0.000786087968322403\\
57.88	0.000786105285083282\\
57.89	0.000786122609834187\\
57.9	0.000786139942580727\\
57.91	0.000786157283328518\\
57.92	0.000786174632083182\\
57.93	0.000786191988850346\\
57.94	0.000786209353635643\\
57.95	0.000786226726444714\\
57.96	0.000786244107283205\\
57.97	0.000786261496156767\\
57.98	0.000786278893071056\\
57.99	0.000786296298031739\\
58	0.000786313711044482\\
58.01	0.000786331132114963\\
58.02	0.000786348561248863\\
58.03	0.000786365998451869\\
58.04	0.000786383443729674\\
58.05	0.00078640089708798\\
58.06	0.000786418358532491\\
58.07	0.000786435828068918\\
58.08	0.000786453305702978\\
58.09	0.000786470791440399\\
58.1	0.000786488285286907\\
58.11	0.000786505787248239\\
58.12	0.000786523297330137\\
58.13	0.000786540815538351\\
58.14	0.000786558341878633\\
58.15	0.000786575876356745\\
58.16	0.000786593418978453\\
58.17	0.000786610969749529\\
58.18	0.000786628528675754\\
58.19	0.00078664609576291\\
58.2	0.000786663671016791\\
58.21	0.000786681254443193\\
58.22	0.00078669884604792\\
58.23	0.00078671644583678\\
58.24	0.000786734053815592\\
58.25	0.000786751669990175\\
58.26	0.000786769294366359\\
58.27	0.00078678692694998\\
58.28	0.000786804567746876\\
58.29	0.000786822216762895\\
58.3	0.000786839874003891\\
58.31	0.000786857539475723\\
58.32	0.000786875213184258\\
58.33	0.000786892895135366\\
58.34	0.000786910585334928\\
58.35	0.000786928283788827\\
58.36	0.000786945990502955\\
58.37	0.000786963705483208\\
58.38	0.000786981428735491\\
58.39	0.000786999160265712\\
58.4	0.00078701690007979\\
58.41	0.000787034648183646\\
58.42	0.000787052404583211\\
58.43	0.000787070169284419\\
58.44	0.000787087942293211\\
58.45	0.000787105723615537\\
58.46	0.00078712351325735\\
58.47	0.000787141311224614\\
58.48	0.000787159117523292\\
58.49	0.000787176932159361\\
58.5	0.0007871947551388\\
58.51	0.000787212586467597\\
58.52	0.000787230426151744\\
58.53	0.00078724827419724\\
58.54	0.000787266130610094\\
58.55	0.000787283995396317\\
58.56	0.000787301868561928\\
58.57	0.000787319750112953\\
58.58	0.000787337640055423\\
58.59	0.000787355538395379\\
58.6	0.000787373445138864\\
58.61	0.00078739136029193\\
58.62	0.000787409283860637\\
58.63	0.000787427215851049\\
58.64	0.000787445156269237\\
58.65	0.000787463105121278\\
58.66	0.00078748106241326\\
58.67	0.00078749902815127\\
58.68	0.00078751700234141\\
58.69	0.000787534984989782\\
58.7	0.000787552976102496\\
58.71	0.000787570975685672\\
58.72	0.000787588983745435\\
58.73	0.000787607000287913\\
58.74	0.000787625025319245\\
58.75	0.000787643058845577\\
58.76	0.000787661100873058\\
58.77	0.000787679151407848\\
58.78	0.000787697210456108\\
58.79	0.000787715278024011\\
58.8	0.000787733354117736\\
58.81	0.000787751438743467\\
58.82	0.000787769531907394\\
58.83	0.000787787633615717\\
58.84	0.000787805743874642\\
58.85	0.000787823862690377\\
58.86	0.000787841990069144\\
58.87	0.000787860126017167\\
58.88	0.000787878270540677\\
58.89	0.000787896423645915\\
58.9	0.000787914585339125\\
58.91	0.000787932755626561\\
58.92	0.000787950934514482\\
58.93	0.000787969122009154\\
58.94	0.000787987318116851\\
58.95	0.000788005522843853\\
58.96	0.000788023736196446\\
58.97	0.000788041958180925\\
58.98	0.000788060188803589\\
58.99	0.000788078428070748\\
59	0.000788096675988717\\
59.01	0.000788114932563816\\
59.02	0.000788133197802373\\
59.03	0.000788151471710725\\
59.04	0.000788169754295215\\
59.05	0.000788188045562192\\
59.06	0.000788206345518011\\
59.07	0.000788224654169037\\
59.08	0.000788242971521641\\
59.09	0.000788261297582199\\
59.1	0.000788279632357097\\
59.11	0.000788297975852727\\
59.12	0.000788316328075488\\
59.13	0.000788334689031783\\
59.14	0.000788353058728028\\
59.15	0.000788371437170641\\
59.16	0.000788389824366052\\
59.17	0.000788408220320693\\
59.18	0.000788426625041005\\
59.19	0.000788445038533439\\
59.2	0.000788463460804448\\
59.21	0.000788481891860496\\
59.22	0.000788500331708054\\
59.23	0.000788518780353598\\
59.24	0.000788537237803612\\
59.25	0.000788555704064589\\
59.26	0.000788574179143029\\
59.27	0.000788592663045436\\
59.28	0.000788611155778324\\
59.29	0.000788629657348213\\
59.3	0.000788648167761632\\
59.31	0.000788666687025116\\
59.32	0.000788685215145208\\
59.33	0.000788703752128457\\
59.34	0.000788722297981419\\
59.35	0.00078874085271066\\
59.36	0.000788759416322752\\
59.37	0.000788777988824275\\
59.38	0.000788796570221814\\
59.39	0.000788815160521962\\
59.4	0.000788833759731322\\
59.41	0.000788852367856502\\
59.42	0.000788870984904117\\
59.43	0.000788889610880793\\
59.44	0.00078890824579316\\
59.45	0.000788926889647856\\
59.46	0.000788945542451527\\
59.47	0.000788964204210827\\
59.48	0.000788982874932415\\
59.49	0.000789001554622961\\
59.5	0.00078902024328914\\
59.51	0.000789038940937637\\
59.52	0.000789057647575142\\
59.53	0.000789076363208352\\
59.54	0.000789095087843975\\
59.55	0.000789113821488724\\
59.56	0.000789132564149319\\
59.57	0.00078915131583249\\
59.58	0.000789170076544974\\
59.59	0.000789188846293514\\
59.6	0.000789207625084862\\
59.61	0.000789226412925779\\
59.62	0.00078924520982303\\
59.63	0.00078926401578339\\
59.64	0.000789282830813641\\
59.65	0.000789301654920574\\
59.66	0.000789320488110985\\
59.67	0.000789339330391683\\
59.68	0.000789358181769478\\
59.69	0.000789377042251193\\
59.7	0.000789395911843656\\
59.71	0.000789414790553703\\
59.72	0.000789433678388178\\
59.73	0.000789452575353936\\
59.74	0.000789471481457834\\
59.75	0.000789490396706741\\
59.76	0.000789509321107534\\
59.77	0.000789528254667094\\
59.78	0.000789547197392314\\
59.79	0.000789566149290094\\
59.8	0.000789585110367341\\
59.81	0.000789604080630968\\
59.82	0.0007896230600879\\
59.83	0.000789642048745067\\
59.84	0.00078966104660941\\
59.85	0.000789680053687875\\
59.86	0.000789699069987417\\
59.87	0.000789718095514997\\
59.88	0.000789737130277589\\
59.89	0.000789756174282171\\
59.9	0.000789775227535729\\
59.91	0.00078979429004526\\
59.92	0.000789813361817767\\
59.93	0.00078983244286026\\
59.94	0.000789851533179758\\
59.95	0.000789870632783293\\
59.96	0.000789889741677896\\
59.97	0.000789908859870613\\
59.98	0.000789927987368495\\
59.99	0.000789947124178602\\
60	0.000789966270308005\\
60.01	0.000789985425763777\\
60.02	0.000790004590553004\\
60.03	0.000790023764682781\\
60.04	0.00079004294816021\\
60.05	0.000790062140992396\\
60.06	0.000790081343186462\\
60.07	0.000790100554749531\\
60.08	0.000790119775688738\\
60.09	0.000790139006011227\\
60.1	0.000790158245724149\\
60.11	0.000790177494834662\\
60.12	0.000790196753349936\\
60.13	0.000790216021277147\\
60.14	0.000790235298623479\\
60.15	0.000790254585396126\\
60.16	0.00079027388160229\\
60.17	0.000790293187249181\\
60.18	0.000790312502344017\\
60.19	0.000790331826894025\\
60.2	0.000790351160906443\\
60.21	0.000790370504388512\\
60.22	0.000790389857347487\\
60.23	0.000790409219790629\\
60.24	0.000790428591725208\\
60.25	0.000790447973158501\\
60.26	0.000790467364097797\\
60.27	0.00079048676455039\\
60.28	0.000790506174523586\\
60.29	0.000790525594024698\\
60.3	0.000790545023061046\\
60.31	0.000790564461639963\\
60.32	0.000790583909768787\\
60.33	0.000790603367454866\\
60.34	0.000790622834705556\\
60.35	0.000790642311528223\\
60.36	0.000790661797930243\\
60.37	0.000790681293918996\\
60.38	0.000790700799501876\\
60.39	0.000790720314686284\\
60.4	0.000790739839479629\\
60.41	0.000790759373889328\\
60.42	0.00079077891792281\\
60.43	0.000790798471587511\\
60.44	0.000790818034890877\\
60.45	0.000790837607840362\\
60.46	0.000790857190443429\\
60.47	0.000790876782707549\\
60.48	0.000790896384640204\\
60.49	0.000790915996248887\\
60.5	0.000790935617541094\\
60.51	0.000790955248524334\\
60.52	0.000790974889206125\\
60.53	0.000790994539593994\\
60.54	0.000791014199695477\\
60.55	0.000791033869518117\\
60.56	0.00079105354906947\\
60.57	0.000791073238357099\\
60.58	0.000791092937388577\\
60.59	0.000791112646171483\\
60.6	0.00079113236471341\\
60.61	0.00079115209302196\\
60.62	0.000791171831104739\\
60.63	0.000791191578969369\\
60.64	0.000791211336623476\\
60.65	0.000791231104074698\\
60.66	0.000791250881330684\\
60.67	0.000791270668399088\\
60.68	0.000791290465287578\\
60.69	0.000791310272003826\\
60.7	0.000791330088555519\\
60.71	0.000791349914950352\\
60.72	0.000791369751196027\\
60.73	0.000791389597300258\\
60.74	0.000791409453270768\\
60.75	0.000791429319115289\\
60.76	0.000791449194841564\\
60.77	0.000791469080457343\\
60.78	0.000791488975970389\\
60.79	0.000791508881388472\\
60.8	0.000791528796719374\\
60.81	0.000791548721970884\\
60.82	0.000791568657150802\\
60.83	0.000791588602266939\\
60.84	0.000791608557327114\\
60.85	0.000791628522339157\\
60.86	0.000791648497310907\\
60.87	0.000791668482250213\\
60.88	0.000791688477164933\\
60.89	0.000791708482062939\\
60.9	0.000791728496952106\\
60.91	0.000791748521840326\\
60.92	0.000791768556735498\\
60.93	0.00079178860164553\\
60.94	0.00079180865657834\\
60.95	0.000791828721541859\\
60.96	0.000791848796544025\\
60.97	0.000791868881592786\\
60.98	0.000791888976696103\\
60.99	0.000791909081861946\\
61	0.000791929197098295\\
61.01	0.000791949322413138\\
61.02	0.000791969457814476\\
61.03	0.00079198960331032\\
61.04	0.000792009758908691\\
61.05	0.000792029924617618\\
61.06	0.000792050100445146\\
61.07	0.000792070286399323\\
61.08	0.000792090482488215\\
61.09	0.000792110688719892\\
61.1	0.000792130905102439\\
61.11	0.000792151131643949\\
61.12	0.000792171368352525\\
61.13	0.000792191615236283\\
61.14	0.000792211872303349\\
61.15	0.000792232139561857\\
61.16	0.000792252417019955\\
61.17	0.000792272704685801\\
61.18	0.000792293002567562\\
61.19	0.000792313310673417\\
61.2	0.000792333629011556\\
61.21	0.000792353957590178\\
61.22	0.000792374296417497\\
61.23	0.000792394645501732\\
61.24	0.000792415004851117\\
61.25	0.000792435374473896\\
61.26	0.000792455754378323\\
61.27	0.000792476144572665\\
61.28	0.000792496545065198\\
61.29	0.000792516955864211\\
61.3	0.000792537376978\\
61.31	0.000792557808414878\\
61.32	0.000792578250183164\\
61.33	0.000792598702291194\\
61.34	0.000792619164747309\\
61.35	0.000792639637559864\\
61.36	0.000792660120737225\\
61.37	0.00079268061428777\\
61.38	0.000792701118219889\\
61.39	0.00079272163254198\\
61.4	0.000792742157262456\\
61.41	0.00079276269238974\\
61.42	0.000792783237932266\\
61.43	0.000792803793898482\\
61.44	0.000792824360296844\\
61.45	0.000792844937135821\\
61.46	0.000792865524423895\\
61.47	0.000792886122169559\\
61.48	0.000792906730381317\\
61.49	0.000792927349067687\\
61.5	0.000792947978237193\\
61.51	0.000792968617898378\\
61.52	0.000792989268059794\\
61.53	0.000793009928730003\\
61.54	0.000793030599917581\\
61.55	0.000793051281631115\\
61.56	0.000793071973879208\\
61.57	0.000793092676670467\\
61.58	0.00079311339001352\\
61.59	0.000793134113917\\
61.6	0.000793154848389557\\
61.61	0.000793175593439851\\
61.62	0.000793196349076555\\
61.63	0.000793217115308354\\
61.64	0.000793237892143944\\
61.65	0.000793258679592037\\
61.66	0.000793279477661354\\
61.67	0.000793300286360632\\
61.68	0.000793321105698616\\
61.69	0.000793341935684068\\
61.7	0.000793362776325759\\
61.71	0.000793383627632475\\
61.72	0.000793404489613014\\
61.73	0.000793425362276187\\
61.74	0.000793446245630818\\
61.75	0.000793467139685743\\
61.76	0.000793488044449812\\
61.77	0.000793508959931888\\
61.78	0.000793529886140844\\
61.79	0.000793550823085571\\
61.8	0.00079357177077497\\
61.81	0.000793592729217956\\
61.82	0.000793613698423457\\
61.83	0.000793634678400414\\
61.84	0.00079365566915778\\
61.85	0.000793676670704527\\
61.86	0.000793697683049634\\
61.87	0.000793718706202096\\
61.88	0.000793739740170921\\
61.89	0.000793760784965132\\
61.9	0.000793781840593765\\
61.91	0.000793802907065868\\
61.92	0.000793823984390506\\
61.93	0.000793845072576753\\
61.94	0.000793866171633703\\
61.95	0.000793887281570458\\
61.96	0.000793908402396138\\
61.97	0.000793929534119876\\
61.98	0.000793950676750817\\
61.99	0.000793971830298123\\
62	0.000793992994770969\\
62.01	0.000794014170178546\\
62.02	0.000794035356530056\\
62.03	0.000794056553834716\\
62.04	0.000794077762101759\\
62.05	0.000794098981340434\\
62.06	0.000794120211560001\\
62.07	0.000794141452769737\\
62.08	0.00079416270497893\\
62.09	0.000794183968196889\\
62.1	0.000794205242432932\\
62.11	0.000794226527696396\\
62.12	0.000794247823996629\\
62.13	0.000794269131342999\\
62.14	0.000794290449744885\\
62.15	0.000794311779211682\\
62.16	0.000794333119752801\\
62.17	0.000794354471377668\\
62.18	0.000794375834095725\\
62.19	0.000794397207916428\\
62.2	0.000794418592849249\\
62.21	0.000794439988903675\\
62.22	0.000794461396089212\\
62.23	0.000794482814415376\\
62.24	0.000794504243891705\\
62.25	0.000794525684527748\\
62.26	0.000794547136333071\\
62.27	0.000794568599317256\\
62.28	0.000794590073489905\\
62.29	0.000794611558860629\\
62.3	0.000794633055439061\\
62.31	0.000794654563234848\\
62.32	0.000794676082257655\\
62.33	0.000794697612517161\\
62.34	0.000794719154023061\\
62.35	0.00079474070678507\\
62.36	0.00079476227081292\\
62.37	0.000794783846116354\\
62.38	0.000794805432705138\\
62.39	0.000794827030589054\\
62.4	0.000794848639777896\\
62.41	0.000794870260281482\\
62.42	0.000794891892109642\\
62.43	0.000794913535272225\\
62.44	0.000794935189779101\\
62.45	0.00079495685564015\\
62.46	0.000794978532865276\\
62.47	0.000795000221464398\\
62.48	0.000795021921447451\\
62.49	0.000795043632824392\\
62.5	0.000795065355605192\\
62.51	0.000795087089799843\\
62.52	0.000795108835418354\\
62.53	0.000795130592470751\\
62.54	0.000795152360967081\\
62.55	0.000795174140917407\\
62.56	0.000795195932331811\\
62.57	0.000795217735220392\\
62.58	0.000795239549593272\\
62.59	0.000795261375460588\\
62.6	0.000795283212832497\\
62.61	0.000795305061719176\\
62.62	0.00079532692213082\\
62.63	0.000795348794077643\\
62.64	0.000795370677569879\\
62.65	0.00079539257261778\\
62.66	0.000795414479231619\\
62.67	0.000795436397421689\\
62.68	0.000795458327198302\\
62.69	0.000795480268571788\\
62.7	0.000795502221552499\\
62.71	0.000795524186150807\\
62.72	0.000795546162377106\\
62.73	0.000795568150241805\\
62.74	0.000795590149755339\\
62.75	0.000795612160928161\\
62.76	0.000795634183770743\\
62.77	0.00079565621829358\\
62.78	0.000795678264507188\\
62.79	0.000795700322422103\\
62.8	0.000795722392048882\\
62.81	0.000795744473398103\\
62.82	0.000795766566480368\\
62.83	0.000795788671306297\\
62.84	0.000795810787886535\\
62.85	0.000795832916231743\\
62.86	0.000795855056352612\\
62.87	0.000795877208259847\\
62.88	0.000795899371964181\\
62.89	0.000795921547476366\\
62.9	0.000795943734807179\\
62.91	0.000795965933967416\\
62.92	0.000795988144967898\\
62.93	0.000796010367819469\\
62.94	0.000796032602532994\\
62.95	0.000796054849119364\\
62.96	0.000796077107589492\\
62.97	0.000796099377954312\\
62.98	0.000796121660224785\\
62.99	0.000796143954411892\\
63	0.000796166260526643\\
63.01	0.000796188578580066\\
63.02	0.000796210908583216\\
63.03	0.000796233250547174\\
63.04	0.000796255604483043\\
63.05	0.00079627797040195\\
63.06	0.000796300348315048\\
63.07	0.000796322738233516\\
63.08	0.000796345140168554\\
63.09	0.000796367554131393\\
63.1	0.000796389980133283\\
63.11	0.000796412418185503\\
63.12	0.00079643486829936\\
63.13	0.000796457330486181\\
63.14	0.000796479804757323\\
63.15	0.000796502291124167\\
63.16	0.000796524789598123\\
63.17	0.000796547300190626\\
63.18	0.000796569822913136\\
63.19	0.000796592357777143\\
63.2	0.000796614904794163\\
63.21	0.000796637463975736\\
63.22	0.000796660035333433\\
63.23	0.000796682618878853\\
63.24	0.000796705214623619\\
63.25	0.000796727822579387\\
63.26	0.000796750442757836\\
63.27	0.000796773075170676\\
63.28	0.000796795719829645\\
63.29	0.00079681837674651\\
63.3	0.000796841045933066\\
63.31	0.000796863727401138\\
63.32	0.000796886421162579\\
63.33	0.000796909127229274\\
63.34	0.000796931845613134\\
63.35	0.000796954576326104\\
63.36	0.000796977319380153\\
63.37	0.000797000074787288\\
63.38	0.000797022842559541\\
63.39	0.000797045622708974\\
63.4	0.000797068415247685\\
63.41	0.000797091220187798\\
63.42	0.000797114037541471\\
63.43	0.000797136867320894\\
63.44	0.000797159709538288\\
63.45	0.000797182564205903\\
63.46	0.000797205431336025\\
63.47	0.000797228310940971\\
63.48	0.000797251203033091\\
63.49	0.000797274107624767\\
63.5	0.000797297024728416\\
63.51	0.000797319954356486\\
63.52	0.000797342896521459\\
63.53	0.000797365851235852\\
63.54	0.000797388818512216\\
63.55	0.000797411798363136\\
63.56	0.000797434790801228\\
63.57	0.000797457795839149\\
63.58	0.000797480813489586\\
63.59	0.000797503843765264\\
63.6	0.000797526886678942\\
63.61	0.000797549942243415\\
63.62	0.000797573010471515\\
63.63	0.000797596091376109\\
63.64	0.000797619184970103\\
63.65	0.000797642291266436\\
63.66	0.000797665410278085\\
63.67	0.000797688542018067\\
63.68	0.000797711686499433\\
63.69	0.000797734843735274\\
63.7	0.000797758013738719\\
63.71	0.000797781196522936\\
63.72	0.000797804392101129\\
63.73	0.000797827600486543\\
63.74	0.000797850821692463\\
63.75	0.000797874055732213\\
63.76	0.000797897302619154\\
63.77	0.00079792056236669\\
63.78	0.000797943834988265\\
63.79	0.000797967120497364\\
63.8	0.000797990418907511\\
63.81	0.000798013730232274\\
63.82	0.00079803705448526\\
63.83	0.00079806039168012\\
63.84	0.000798083741830546\\
63.85	0.000798107104950272\\
63.86	0.000798130481053075\\
63.87	0.000798153870152777\\
63.88	0.000798177272263242\\
63.89	0.000798200687398378\\
63.9	0.000798224115572137\\
63.91	0.000798247556798514\\
63.92	0.000798271011091551\\
63.93	0.000798294478465335\\
63.94	0.000798317958933994\\
63.95	0.00079834145251171\\
63.96	0.000798364959212702\\
63.97	0.000798388479051242\\
63.98	0.000798412012041646\\
63.99	0.000798435558198276\\
64	0.000798459117535545\\
64.01	0.000798482690067912\\
64.02	0.000798506275809883\\
64.03	0.000798529874776011\\
64.04	0.000798553486980904\\
64.05	0.000798577112439212\\
64.06	0.00079860075116564\\
64.07	0.000798624403174941\\
64.08	0.000798648068481917\\
64.09	0.000798671747101423\\
64.1	0.000798695439048364\\
64.11	0.000798719144337696\\
64.12	0.000798742862984427\\
64.13	0.000798766595003617\\
64.14	0.000798790340410379\\
64.15	0.000798814099219881\\
64.16	0.00079883787144734\\
64.17	0.000798861657108029\\
64.18	0.000798885456217277\\
64.19	0.000798909268790464\\
64.2	0.000798933094843029\\
64.21	0.000798956934390463\\
64.22	0.000798980787448313\\
64.23	0.000799004654032183\\
64.24	0.000799028534157736\\
64.25	0.000799052427840688\\
64.26	0.000799076335096814\\
64.27	0.000799100255941949\\
64.28	0.000799124190391982\\
64.29	0.000799148138462862\\
64.3	0.000799172100170602\\
64.31	0.000799196075531267\\
64.32	0.000799220064560988\\
64.33	0.000799244067275953\\
64.34	0.000799268083692411\\
64.35	0.000799292113826676\\
64.36	0.000799316157695119\\
64.37	0.000799340215314175\\
64.38	0.000799364286700343\\
64.39	0.000799388371870184\\
64.4	0.000799412470840323\\
64.41	0.000799436583627447\\
64.42	0.000799460710248312\\
64.43	0.000799484850719734\\
64.44	0.000799509005058596\\
64.45	0.000799533173281849\\
64.46	0.000799557355406509\\
64.47	0.000799581551449657\\
64.48	0.000799605761428445\\
64.49	0.00079962998536009\\
64.5	0.000799654223261878\\
64.51	0.000799678475151165\\
64.52	0.000799702741045372\\
64.53	0.000799727020961996\\
64.54	0.000799751314918599\\
64.55	0.000799775622932817\\
64.56	0.000799799945022356\\
64.57	0.000799824281204993\\
64.58	0.000799848631498578\\
64.59	0.000799872995921035\\
64.6	0.000799897374490358\\
64.61	0.000799921767224617\\
64.62	0.000799946174141957\\
64.63	0.000799970595260595\\
64.64	0.000799995030598825\\
64.65	0.000800019480175017\\
64.66	0.000800043944007617\\
64.67	0.000800068422115146\\
64.68	0.000800092914516205\\
64.69	0.000800117421229471\\
64.7	0.000800141942273701\\
64.71	0.000800166477667728\\
64.72	0.000800191027430468\\
64.73	0.000800215591580914\\
64.74	0.00080024017013814\\
64.75	0.000800264763121302\\
64.76	0.000800289370549637\\
64.77	0.000800313992442462\\
64.78	0.00080033862881918\\
64.79	0.000800363279699274\\
64.8	0.000800387945102314\\
64.81	0.00080041262504795\\
64.82	0.000800437319555918\\
64.83	0.00080046202864604\\
64.84	0.000800486752338226\\
64.85	0.000800511490652467\\
64.86	0.000800536243608842\\
64.87	0.000800561011227522\\
64.88	0.00080058579352876\\
64.89	0.000800610590532899\\
64.9	0.000800635402260372\\
64.91	0.0008006602287317\\
64.92	0.000800685069967496\\
64.93	0.00080070992598846\\
64.94	0.000800734796815385\\
64.95	0.000800759682469157\\
64.96	0.00080078458297075\\
64.97	0.000800809498341232\\
64.98	0.000800834428601765\\
64.99	0.000800859373773605\\
65	0.0008008843338781\\
65.01	0.000800909308936693\\
65.02	0.000800934298970921\\
65.03	0.000800959304002417\\
65.04	0.000800984324052914\\
65.05	0.000801009359144235\\
65.06	0.000801034409298303\\
65.07	0.000801059474537139\\
65.08	0.000801084554882859\\
65.09	0.00080110965035768\\
65.1	0.000801134760983918\\
65.11	0.000801159886783987\\
65.12	0.000801185027780401\\
65.13	0.000801210183995772\\
65.14	0.000801235355452816\\
65.15	0.000801260542174347\\
65.16	0.000801285744183283\\
65.17	0.000801310961502643\\
65.18	0.000801336194155546\\
65.19	0.000801361442165216\\
65.2	0.000801386705554982\\
65.21	0.000801411984348271\\
65.22	0.000801437278568618\\
65.23	0.000801462588239661\\
65.24	0.000801487913385142\\
65.25	0.000801513254028908\\
65.26	0.000801538610194911\\
65.27	0.000801563981907208\\
65.28	0.000801589369189963\\
65.29	0.000801614772067447\\
65.3	0.000801640190564034\\
65.31	0.000801665624704207\\
65.32	0.000801691074512556\\
65.33	0.000801716540013776\\
65.34	0.000801742021232671\\
65.35	0.000801767518194154\\
65.36	0.000801793030923243\\
65.37	0.000801818559445064\\
65.38	0.000801844103784853\\
65.39	0.000801869663967952\\
65.4	0.000801895240019813\\
65.41	0.000801920831965995\\
65.42	0.000801946439832169\\
65.43	0.000801972063644107\\
65.44	0.000801997703427697\\
65.45	0.000802023359208932\\
65.46	0.000802049031013916\\
65.47	0.000802074718868858\\
65.48	0.000802100422800078\\
65.49	0.000802126142834002\\
65.5	0.000802151878997166\\
65.51	0.000802177631316216\\
65.52	0.000802203399817901\\
65.53	0.000802229184529081\\
65.54	0.000802254985476723\\
65.55	0.000802280802687901\\
65.56	0.000802306636189796\\
65.57	0.000802332486009695\\
65.58	0.00080235835217499\\
65.59	0.000802384234713183\\
65.6	0.000802410133651876\\
65.61	0.000802436049018779\\
65.62	0.000802461980841704\\
65.63	0.00080248792914857\\
65.64	0.000802513893967396\\
65.65	0.000802539875326306\\
65.66	0.000802565873253523\\
65.67	0.000802591887777373\\
65.68	0.000802617918926281\\
65.69	0.000802643966728772\\
65.7	0.000802670031213469\\
65.71	0.000802696112409093\\
65.72	0.000802722210344459\\
65.73	0.000802748325048484\\
65.74	0.00080277445655017\\
65.75	0.000802800604878619\\
65.76	0.000802826770063022\\
65.77	0.000802852952132662\\
65.78	0.00080287915111691\\
65.79	0.000802905367045225\\
65.8	0.000802931599947153\\
65.81	0.00080295784985232\\
65.82	0.000802984116790443\\
65.83	0.000803010400791315\\
65.84	0.000803036701884807\\
65.85	0.000803063020100872\\
65.86	0.000803089355469536\\
65.87	0.000803115708020897\\
65.88	0.000803142077785129\\
65.89	0.000803168464792468\\
65.9	0.000803194869073225\\
65.91	0.000803221290657768\\
65.92	0.000803247729576531\\
65.93	0.000803274185860006\\
65.94	0.000803300659538739\\
65.95	0.000803327150643334\\
65.96	0.000803353659204442\\
65.97	0.000803380185252764\\
65.98	0.000803406728819043\\
65.99	0.000803433289934064\\
66	0.000803459868628651\\
66.01	0.000803486464933658\\
66.02	0.000803513078879974\\
66.03	0.000803539710498511\\
66.04	0.000803566359820205\\
66.05	0.00080359302687601\\
66.06	0.000803619711696894\\
66.07	0.000803646414313835\\
66.08	0.000803673134757816\\
66.09	0.000803699873059819\\
66.1	0.000803726629250823\\
66.11	0.000803753403361798\\
66.12	0.000803780195423699\\
66.13	0.000803807005467458\\
66.14	0.000803833833523986\\
66.15	0.000803860679624159\\
66.16	0.000803887543798815\\
66.17	0.00080391442607875\\
66.18	0.000803941326494709\\
66.19	0.000803968245077381\\
66.2	0.00080399518185739\\
66.21	0.000804022136865291\\
66.22	0.000804049110131561\\
66.23	0.00080407610168659\\
66.24	0.000804103111560678\\
66.25	0.00080413013978402\\
66.26	0.000804157186386703\\
66.27	0.000804184251398698\\
66.28	0.000804211334849848\\
66.29	0.00080423843676986\\
66.3	0.000804265557188297\\
66.31	0.000804292696134567\\
66.32	0.000804319853637912\\
66.33	0.0008043470297274\\
66.34	0.000804374224431916\\
66.35	0.000804401437780145\\
66.36	0.000804428669800569\\
66.37	0.000804455920521446\\
66.38	0.000804483189970806\\
66.39	0.000804510478176437\\
66.4	0.000804537785165867\\
66.41	0.00080456511096636\\
66.42	0.000804592455604892\\
66.43	0.000804619819108149\\
66.44	0.000804647201502501\\
66.45	0.000804674602813997\\
66.46	0.000804702023068343\\
66.47	0.000804729462290892\\
66.48	0.000804756920506623\\
66.49	0.000804784397740127\\
66.5	0.000804811894015588\\
66.51	0.000804839409356771\\
66.52	0.000804866943786998\\
66.53	0.00080489449732913\\
66.54	0.000804922070005551\\
66.55	0.000804949661838148\\
66.56	0.000804977272848286\\
66.57	0.000805004903056794\\
66.58	0.000805032552483941\\
66.59	0.000805060221149411\\
66.6	0.000805087909072282\\
66.61	0.000805115616271005\\
66.62	0.000805143342763378\\
66.63	0.000805171088566523\\
66.64	0.000805198853696855\\
66.65	0.000805226638170065\\
66.66	0.000805254442001085\\
66.67	0.000805282265204066\\
66.68	0.000805310107792346\\
66.69	0.000805337969778422\\
66.7	0.000805365851173918\\
66.71	0.000805393751989561\\
66.72	0.000805421672235138\\
66.73	0.000805449611919475\\
66.74	0.000805477571050397\\
66.75	0.000805505549634692\\
66.76	0.000805533547678079\\
66.77	0.000805561565185171\\
66.78	0.000805589602159439\\
66.79	0.000805617658603168\\
66.8	0.000805645734517423\\
66.81	0.000805673829902006\\
66.82	0.000805701944755413\\
66.83	0.000805730079074795\\
66.84	0.000805758232855908\\
66.85	0.000805786406093073\\
66.86	0.000805814598779127\\
66.87	0.000805842810905372\\
66.88	0.000805871042461532\\
66.89	0.0008058992934357\\
66.9	0.000805927563814283\\
66.91	0.00080595585358195\\
66.92	0.000805984162721582\\
66.93	0.000806012491214207\\
66.94	0.000806040839038947\\
66.95	0.00080606920617296\\
66.96	0.000806097592591379\\
66.97	0.000806125998267243\\
66.98	0.000806154423171437\\
66.99	0.000806182867272629\\
67	0.000806211330537196\\
67.01	0.000806239812929157\\
67.02	0.000806268314410099\\
67.03	0.000806296834939111\\
67.04	0.000806325374472694\\
67.05	0.000806353932964701\\
67.06	0.000806382510366244\\
67.07	0.000806411106625613\\
67.08	0.000806439721688201\\
67.09	0.000806468355496411\\
67.1	0.000806497008074848\\
67.11	0.00080652567946334\\
67.12	0.000806554369702072\\
67.13	0.000806583078831588\\
67.14	0.000806611806892798\\
67.15	0.000806640553926982\\
67.16	0.000806669319975795\\
67.17	0.000806698105081277\\
67.18	0.00080672690928585\\
67.19	0.000806755732632333\\
67.2	0.000806784575163937\\
67.21	0.000806813436924278\\
67.22	0.000806842317957381\\
67.23	0.000806871218307687\\
67.24	0.000806900138020053\\
67.25	0.000806929077139766\\
67.26	0.000806958035712544\\
67.27	0.00080698701378454\\
67.28	0.000807016011402355\\
67.29	0.00080704502861304\\
67.3	0.000807074065464101\\
67.31	0.00080710312200351\\
67.32	0.000807132198279707\\
67.33	0.000807161294341609\\
67.34	0.000807190410238615\\
67.35	0.000807219546020617\\
67.36	0.000807248701738001\\
67.37	0.00080727787744166\\
67.38	0.000807307073182993\\
67.39	0.000807336289013923\\
67.4	0.000807365524986897\\
67.41	0.000807394781154892\\
67.42	0.00080742405757143\\
67.43	0.000807453354290577\\
67.44	0.00080748267136696\\
67.45	0.000807512008855767\\
67.46	0.000807541366812756\\
67.47	0.00080757074529427\\
67.48	0.000807600144357235\\
67.49	0.000807629564059179\\
67.5	0.000807659004458229\\
67.51	0.000807688465613132\\
67.52	0.000807717947583253\\
67.53	0.00080774745042859\\
67.54	0.000807776974209778\\
67.55	0.000807806518988109\\
67.56	0.000807836084825524\\
67.57	0.000807865671784638\\
67.58	0.000807895279928741\\
67.59	0.00080792490932181\\
67.6	0.000807954560028521\\
67.61	0.000807984232114254\\
67.62	0.000808013925645107\\
67.63	0.000808043640687903\\
67.64	0.000808073377310205\\
67.65	0.000808103135580322\\
67.66	0.000808132915567321\\
67.67	0.00080816271734104\\
67.68	0.000808192540972094\\
67.69	0.000808222386531894\\
67.7	0.000808252254092647\\
67.71	0.000808282143727379\\
67.72	0.000808312055509939\\
67.73	0.000808341989515014\\
67.74	0.000808371945818141\\
67.75	0.000808401924495716\\
67.76	0.000808431925625011\\
67.77	0.000808461949284179\\
67.78	0.000808491995552277\\
67.79	0.000808522064509272\\
67.8	0.000808552156236054\\
67.81	0.000808582270814449\\
67.82	0.000808612408327239\\
67.83	0.000808642568858162\\
67.84	0.000808672752491942\\
67.85	0.000808702959314291\\
67.86	0.000808733189411928\\
67.87	0.000808763442872592\\
67.88	0.000808793719785057\\
67.89	0.000808824020239145\\
67.9	0.000808854344325746\\
67.91	0.000808884692136826\\
67.92	0.000808915063765448\\
67.93	0.000808945459305785\\
67.94	0.000808975878853136\\
67.95	0.000809006322503945\\
67.96	0.000809036790355809\\
67.97	0.000809067282507507\\
67.98	0.000809097799059005\\
67.99	0.000809128340111478\\
68	0.00080915890576733\\
68.01	0.000809189496130203\\
68.02	0.000809220111305006\\
68.03	0.000809250751397922\\
68.04	0.000809281416516433\\
68.05	0.000809312106769336\\
68.06	0.000809342822266763\\
68.07	0.000809373563120197\\
68.08	0.000809404329442496\\
68.09	0.000809435121347905\\
68.1	0.000809465938952085\\
68.11	0.000809496782372125\\
68.12	0.000809527651726568\\
68.13	0.000809558547135426\\
68.14	0.000809589468720206\\
68.15	0.000809620416603931\\
68.16	0.000809651390911154\\
68.17	0.000809682391767989\\
68.18	0.000809713419302128\\
68.19	0.000809744473642866\\
68.2	0.000809775554921118\\
68.21	0.000809806663269451\\
68.22	0.000809837798822098\\
68.23	0.000809868961714987\\
68.24	0.000809900152085766\\
68.25	0.000809931370073822\\
68.26	0.000809962615820309\\
68.27	0.000809993889468171\\
68.28	0.000810025191162172\\
68.29	0.000810056521048914\\
68.3	0.000810087879276869\\
68.31	0.000810119265996402\\
68.32	0.000810150681359798\\
68.33	0.000810182125521291\\
68.34	0.00081021359863709\\
68.35	0.000810245100865407\\
68.36	0.000810276632366482\\
68.37	0.000810308193302618\\
68.38	0.000810339783838201\\
68.39	0.000810371404139738\\
68.4	0.00081040305437588\\
68.41	0.000810434734717458\\
68.42	0.000810466445337506\\
68.43	0.000810498186411299\\
68.44	0.00081052995811638\\
68.45	0.000810561760632595\\
68.46	0.000810593594142116\\
68.47	0.000810625458829487\\
68.48	0.000810657354881646\\
68.49	0.000810689282487963\\
68.5	0.000810721241840276\\
68.51	0.000810753233132918\\
68.52	0.00081078525656276\\
68.53	0.000810817312329237\\
68.54	0.000810849400634394\\
68.55	0.000810881521682912\\
68.56	0.000810913675682151\\
68.57	0.000810945862842185\\
68.58	0.000810978083375841\\
68.59	0.000811010337498733\\
68.6	0.000811042625429302\\
68.61	0.000811074947388857\\
68.62	0.000811107303601614\\
68.63	0.000811139694294733\\
68.64	0.000811172119698363\\
68.65	0.000811204580045678\\
68.66	0.000811237075572919\\
68.67	0.000811269606519442\\
68.68	0.000811302173127755\\
68.69	0.000811334775643558\\
68.7	0.000811367414315797\\
68.71	0.000811400089396699\\
68.72	0.000811432801141817\\
68.73	0.000811465549810082\\
68.74	0.000811498335663843\\
68.75	0.000811531158968913\\
68.76	0.000811564019994621\\
68.77	0.000811596919013856\\
68.78	0.000811629856303113\\
68.79	0.000811662832142547\\
68.8	0.000811695846816018\\
68.81	0.000811728900611142\\
68.82	0.000811761993819346\\
68.83	0.000811795126735907\\
68.84	0.000811828299660018\\
68.85	0.000811861512894831\\
68.86	0.000811894766747517\\
68.87	0.000811928061529309\\
68.88	0.00081196139755557\\
68.89	0.000811994775145835\\
68.9	0.000812028194623874\\
68.91	0.000812061656317746\\
68.92	0.000812095160559858\\
68.93	0.000812128707687019\\
68.94	0.0008121622980405\\
68.95	0.000812195931966093\\
68.96	0.000812229609814171\\
68.97	0.000812263331939744\\
68.98	0.000812297098702527\\
68.99	0.000812330910466997\\
69	0.000812364767602455\\
69.01	0.00081239867048309\\
69.02	0.000812432619488044\\
69.03	0.000812466615001473\\
69.04	0.000812500657412616\\
69.05	0.000812534747115857\\
69.06	0.000812568884510796\\
69.07	0.000812603070002312\\
69.08	0.00081263730400063\\
69.09	0.000812671586921398\\
69.1	0.000812705919185747\\
69.11	0.000812740301220366\\
69.12	0.000812774733457571\\
69.13	0.000812809216335378\\
69.14	0.000812843750297575\\
69.15	0.000812878335793795\\
69.16	0.00081291297327959\\
69.17	0.000812947663216502\\
69.18	0.000812982406072148\\
69.19	0.000813017202320286\\
69.2	0.000813052052440899\\
69.21	0.000813086956920263\\
69.22	0.000813121916251039\\
69.23	0.000813156930932341\\
69.24	0.000813192001469822\\
69.25	0.000813227128375752\\
69.26	0.000813262312169102\\
69.27	0.000813297553375621\\
69.28	0.000813332852527925\\
69.29	0.000813368210165576\\
69.3	0.00081340362683517\\
69.31	0.000813439103090419\\
69.32	0.00081347463949224\\
69.33	0.000813510236608842\\
69.34	0.000813545895015807\\
69.35	0.000813581615296186\\
69.36	0.000813617398040586\\
69.37	0.000813653243847256\\
69.38	0.000813689153322181\\
69.39	0.00081372512707917\\
69.4	0.000813761165739953\\
69.41	0.000813797269934266\\
69.42	0.000813833440299953\\
69.43	0.000813869677483051\\
69.44	0.00081390598213789\\
69.45	0.00081394235492719\\
69.46	0.000813978796522147\\
69.47	0.000814015307602542\\
69.48	0.000814051888856832\\
69.49	0.000814088540982244\\
69.5	0.000814125264684884\\
69.51	0.000814162060679824\\
69.52	0.00081419892969121\\
69.53	0.000814235872452357\\
69.54	0.000814272889705854\\
69.55	0.000814309982203661\\
69.56	0.000814347150707218\\
69.57	0.000814384395987534\\
69.58	0.000814421718825304\\
69.59	0.000814459120011003\\
69.6	0.000814496600344994\\
69.61	0.00081453416063763\\
69.62	0.000814571801709359\\
69.63	0.000814609524390827\\
69.64	0.000814647329522985\\
69.65	0.000814685217957195\\
69.66	0.000814723190555334\\
69.67	0.0008147612481899\\
69.68	0.000814799391744118\\
69.69	0.000814837622112046\\
69.7	0.000814875940198686\\
69.71	0.000814914346920079\\
69.72	0.000814952843203426\\
69.73	0.000814991429987182\\
69.74	0.000815030108221171\\
69.75	0.000815068878866686\\
69.76	0.000815107742896603\\
69.77	0.000815146701295476\\
69.78	0.000815185755059655\\
69.79	0.000815224905197384\\
69.8	0.000815264152728907\\
69.81	0.000815303498686576\\
69.82	0.000815342944114953\\
69.83	0.000815382490070914\\
69.84	0.000815422137623754\\
69.85	0.000815461887855286\\
69.86	0.000815501741859948\\
69.87	0.000815541700744901\\
69.88	0.00081558176563013\\
69.89	0.000815621937648543\\
69.9	0.000815662217946073\\
69.91	0.000815702607681767\\
69.92	0.000815743108027895\\
69.93	0.00081578372017003\\
69.94	0.000815824445307152\\
69.95	0.000815865284651739\\
69.96	0.000815906239429853\\
69.97	0.00081594731088123\\
69.98	0.000815988500259371\\
69.99	0.000816029808831623\\
70	0.000816071237879263\\
70.01	0.00081611278869758\\
70.02	0.000816154462595954\\
70.03	0.000816196260897928\\
70.04	0.000816238184941288\\
70.05	0.00081628023607813\\
70.06	0.000816322415674932\\
70.07	0.000816364725112612\\
70.08	0.0008164071657866\\
70.09	0.000816449739106889\\
70.1	0.000816492446498096\\
70.11	0.00081653528939951\\
70.12	0.000816578269265143\\
70.13	0.000816621387563773\\
70.14	0.000816664645778982\\
70.15	0.000816708045409197\\
70.16	0.000816751587967714\\
70.17	0.00081679527498273\\
70.18	0.000816839107997363\\
70.19	0.000816883088569666\\
70.2	0.000816927218272643\\
70.21	0.000816971498694247\\
70.22	0.000817015931437381\\
70.23	0.000817060518119896\\
70.24	0.000817105260374569\\
70.25	0.000817150159849095\\
70.26	0.000817195218206045\\
70.27	0.000817240437122842\\
70.28	0.000817285818291713\\
70.29	0.000817331363419643\\
70.3	0.000817377074228314\\
70.31	0.000817422952454039\\
70.32	0.000817468999847686\\
70.33	0.000817515218174593\\
70.34	0.000817561609214472\\
70.35	0.000817608174761305\\
70.36	0.000817654916623231\\
70.37	0.000817701836622416\\
70.38	0.000817748936594915\\
70.39	0.000817796218390529\\
70.4	0.000817843683872639\\
70.41	0.000817891334918035\\
70.42	0.000817939173416726\\
70.43	0.000817987201271742\\
70.44	0.000818035420398917\\
70.45	0.000818083832726666\\
70.46	0.000818132440195735\\
70.47	0.000818181244758942\\
70.48	0.000818230248380904\\
70.49	0.000818279453037743\\
70.5	0.000818328860716772\\
70.51	0.000818378473416174\\
70.52	0.000818428293144649\\
70.53	0.000818478321921046\\
70.54	0.000818528561773984\\
70.55	0.000818579014741437\\
70.56	0.000818629682870308\\
70.57	0.000818680568215978\\
70.58	0.00081873167284183\\
70.59	0.000818782998818752\\
70.6	0.000818834548224612\\
70.61	0.000818886323143713\\
70.62	0.000818938325666211\\
70.63	0.00081899055788752\\
70.64	0.000819043021907672\\
70.65	0.000819095719830666\\
70.66	0.000819148653763769\\
70.67	0.000819201825816797\\
70.68	0.000819255238101361\\
70.69	0.000819308892730075\\
70.7	0.000819362791815737\\
70.71	0.000819416937470466\\
70.72	0.00081947133180481\\
70.73	0.000819525976926805\\
70.74	0.000819580874941005\\
70.75	0.000819636027947468\\
70.76	0.000819691438040691\\
70.77	0.000819747107308517\\
70.78	0.000819803037830981\\
70.79	0.00081985923167912\\
70.8	0.000819915690913732\\
70.81	0.000819972417584077\\
70.82	0.000820029413726544\\
70.83	0.000820086681363245\\
70.84	0.000820144222500565\\
70.85	0.000820202039127657\\
70.86	0.000820260133214869\\
70.87	0.000820318506712122\\
70.88	0.000820377161547214\\
70.89	0.000820436099624063\\
70.9	0.000820495322820891\\
70.91	0.000820554832988327\\
70.92	0.00082061463194745\\
70.93	0.000820674721487745\\
70.94	0.000820735103364998\\
70.95	0.000820795779299104\\
70.96	0.000820856750971792\\
70.97	0.000820918020024274\\
70.98	0.000820979588054806\\
70.99	0.000821041456616151\\
71	0.000821103627212964\\
71.01	0.000821166101299073\\
71.02	0.000821228880274666\\
71.03	0.00082129196548337\\
71.04	0.000821355358209243\\
71.05	0.000821419059673634\\
71.06	0.000821483071031956\\
71.07	0.000821547393370325\\
71.08	0.000821612027702099\\
71.09	0.000821676974964278\\
71.1	0.00082174223601379\\
71.11	0.000821807811623644\\
71.12	0.000821873702478946\\
71.13	0.000821939909172778\\
71.14	0.000822006432201944\\
71.15	0.000822073271962556\\
71.16	0.000822140428745472\\
71.17	0.000822207902731584\\
71.18	0.000822275693986936\\
71.19	0.000822343802457683\\
71.2	0.000822412227964876\\
71.21	0.000822480970199066\\
71.22	0.00082255002871473\\
71.23	0.000822619402924513\\
71.24	0.000822689092093262\\
71.25	0.000822759095331874\\
71.26	0.00082282941159093\\
71.27	0.000822900039654122\\
71.28	0.000822970978131448\\
71.29	0.000823042225452196\\
71.3	0.000823113779857687\\
71.31	0.000823185639393775\\
71.32	0.000823257801903102\\
71.33	0.000823330265017097\\
71.34	0.000823403026147709\\
71.35	0.000823476082478873\\
71.36	0.000823549430957684\\
71.37	0.000823623068285298\\
71.38	0.000823696990907516\\
71.39	0.000823771195005077\\
71.4	0.000823845676483624\\
71.41	0.000823920430963344\\
71.42	0.000823995453768274\\
71.43	0.000824070739915255\\
71.44	0.000824146284102538\\
71.45	0.000824222080697998\\
71.46	0.000824298123726995\\
71.47	0.000824374406859823\\
71.48	0.000824450923398759\\
71.49	0.000824527666264698\\
71.5	0.000824604627983352\\
71.51	0.000824681800671015\\
71.52	0.000824759176019851\\
71.53	0.000824836745282741\\
71.54	0.00082491449925761\\
71.55	0.000824992428271285\\
71.56	0.000825070522162807\\
71.57	0.000825148770266236\\
71.58	0.000825227162350679\\
71.59	0.000825305700266385\\
71.6	0.00082538438589441\\
71.61	0.000825463221147165\\
71.62	0.000825542207968964\\
71.63	0.000825621348336582\\
71.64	0.000825700644259833\\
71.65	0.000825780097782143\\
71.66	0.000825859710981151\\
71.67	0.000825939485969302\\
71.68	0.000826019424894475\\
71.69	0.000826099529940595\\
71.7	0.000826179803328284\\
71.71	0.000826260247315502\\
71.72	0.000826340864198213\\
71.73	0.00082642165631106\\
71.74	0.000826502626028049\\
71.75	0.000826583775763249\\
71.76	0.000826665107971513\\
71.77	0.000826746625149196\\
71.78	0.000826828329834899\\
71.79	0.000826910224610223\\
71.8	0.00082699231210054\\
71.81	0.000827074594975776\\
71.82	0.000827157075951206\\
71.83	0.00082723975778827\\
71.84	0.000827322643295405\\
71.85	0.000827405735328886\\
71.86	0.000827489036793689\\
71.87	0.000827572550644364\\
71.88	0.00082765627988594\\
71.89	0.000827740227574829\\
71.9	0.000827824396819754\\
71.91	0.000827908790782698\\
71.92	0.000827993412679877\\
71.93	0.000828078265782711\\
71.94	0.000828163353418836\\
71.95	0.000828248678973124\\
71.96	0.000828334245888728\\
71.97	0.000828420057668139\\
71.98	0.000828506117874278\\
71.99	0.000828592430131591\\
72	0.000828678998127183\\
72.01	0.000828765825611963\\
72.02	0.000828852916401815\\
72.03	0.000828940274378792\\
72.04	0.000829027903492336\\
72.05	0.000829115807760515\\
72.06	0.000829203991271291\\
72.07	0.000829292458183814\\
72.08	0.000829381212729732\\
72.09	0.000829470259214538\\
72.1	0.000829559602018939\\
72.11	0.000829649245600252\\
72.12	0.000829739194493832\\
72.13	0.000829829453314519\\
72.14	0.000829920026758121\\
72.15	0.000830010919602935\\
72.16	0.000830102136711273\\
72.17	0.000830193683031055\\
72.18	0.000830285563597396\\
72.19	0.000830377783534255\\
72.2	0.000830470348056101\\
72.21	0.00083056326246962\\
72.22	0.000830656532175454\\
72.23	0.000830750162669973\\
72.24	0.000830844159547088\\
72.25	0.000830938528500101\\
72.26	0.000831033275323587\\
72.27	0.000831128405915321\\
72.28	0.000831223926278244\\
72.29	0.000831319842522462\\
72.3	0.000831416160867298\\
72.31	0.000831512887643377\\
72.32	0.000831610029294759\\
72.33	0.000831707592381117\\
72.34	0.000831805583579955\\
72.35	0.000831904009688874\\
72.36	0.000832002877627896\\
72.37	0.00083210219444182\\
72.38	0.000832201967302635\\
72.39	0.00083230220351199\\
72.4	0.000832402910503707\\
72.41	0.000832504095846348\\
72.42	0.000832605767245842\\
72.43	0.000832707932548164\\
72.44	0.00083281059974207\\
72.45	0.000832913776961893\\
72.46	0.000833017472490399\\
72.47	0.000833121694761699\\
72.48	0.000833226452364231\\
72.49	0.000833331754043799\\
72.5	0.000833437608706685\\
72.51	0.00083354402542282\\
72.52	0.000833651013429033\\
72.53	0.000833758582132362\\
72.54	0.000833866741113439\\
72.55	0.000833975500129954\\
72.56	0.000834084869120191\\
72.57	0.000834194858206641\\
72.58	0.000834305477699693\\
72.59	0.000834416738101413\\
72.6	0.000834528650109398\\
72.61	0.00083464122462072\\
72.62	0.000834754472735962\\
72.63	0.000834868405763326\\
72.64	0.000834983035222862\\
72.65	0.00083509837285076\\
72.66	0.000835214430603764\\
72.67	0.000835331220663663\\
72.68	0.000835448755441904\\
72.69	0.000835567047584298\\
72.7	0.000835686109975829\\
72.71	0.00083580595574558\\
72.72	0.00083592659827177\\
72.73	0.000836048051186899\\
72.74	0.000836170328383026\\
72.75	0.000836293444017141\\
72.76	0.00083641741251664\\
72.77	0.000836542248584919\\
72.78	0.000836667967207093\\
72.79	0.000836794583655842\\
72.8	0.000836922113497391\\
72.81	0.000837050572597623\\
72.82	0.000837179977128335\\
72.83	0.00083731034357363\\
72.84	0.000837441688736456\\
72.85	0.000837574029745297\\
72.86	0.000837707384061014\\
72.87	0.000837841769483838\\
72.88	0.000837977204160532\\
72.89	0.000838113706591712\\
72.9	0.000838251295639331\\
72.91	0.000838389990534347\\
72.92	0.00083852981088456\\
72.93	0.000838670776682626\\
72.94	0.000838812908314267\\
72.95	0.000838956226566663\\
72.96	0.000839100752637042\\
72.97	0.000839246508141466\\
72.98	0.000839393515123832\\
72.99	0.000839540901050709\\
73	0.000839688399544802\\
73.01	0.000839836010926244\\
73.02	0.000839983735514726\\
73.03	0.000840131573629426\\
73.04	0.000840279525588952\\
73.05	0.000840427591711271\\
73.06	0.000840575772313652\\
73.07	0.000840724067712594\\
73.08	0.000840872478223762\\
73.09	0.000841021004161925\\
73.1	0.000841169645840882\\
73.11	0.000841318403573396\\
73.12	0.000841467277671129\\
73.13	0.000841616268444574\\
73.14	0.000841765376202981\\
73.15	0.000841914601254294\\
73.16	0.000842063943905078\\
73.17	0.000842213404460454\\
73.18	0.000842362983224029\\
73.19	0.000842512680497827\\
73.2	0.000842662496582221\\
73.21	0.000842812431775867\\
73.22	0.000842962486375637\\
73.23	0.000843112660676548\\
73.24	0.000843262954971705\\
73.25	0.00084341336955223\\
73.26	0.0008435639047072\\
73.27	0.000843714560723586\\
73.28	0.000843865337886194\\
73.29	0.000844016236477604\\
73.3	0.000844167256778114\\
73.31	0.000844318399065687\\
73.32	0.000844469663615892\\
73.33	0.000844621050701863\\
73.34	0.000844772560594247\\
73.35	0.000844924193561159\\
73.36	0.000845075949868143\\
73.37	0.000845227829778134\\
73.38	0.000845379833551424\\
73.39	0.000845531961445635\\
73.4	0.00084568421371569\\
73.41	0.000845836590613798\\
73.42	0.000845989092389433\\
73.43	0.000846141719289326\\
73.44	0.000846294471557468\\
73.45	0.000846447349435104\\
73.46	0.000846600353160748\\
73.47	0.000846753482970197\\
73.48	0.000846906739096554\\
73.49	0.000847060121770268\\
73.5	0.000847213631219163\\
73.51	0.0008473672676685\\
73.52	0.000847521031341029\\
73.53	0.000847674922457062\\
73.54	0.000847828941234556\\
73.55	0.000847983087889203\\
73.56	0.000848137362634536\\
73.57	0.000848291765682047\\
73.58	0.000848446297241319\\
73.59	0.000848600957520172\\
73.6	0.000848755746724824\\
73.61	0.000848910665060066\\
73.62	0.00084906571272946\\
73.63	0.000849220889935552\\
73.64	0.000849376196880094\\
73.65	0.000849531633764307\\
73.66	0.000849687200789141\\
73.67	0.000849842898155572\\
73.68	0.000849998726064923\\
73.69	0.000850154684719195\\
73.7	0.00085031077432144\\
73.71	0.000850466995076149\\
73.72	0.000850623347189673\\
73.73	0.000850779830870669\\
73.74	0.000850936446330582\\
73.75	0.000851093193784161\\
73.76	0.000851250073449996\\
73.77	0.000851407085551103\\
73.78	0.000851564230315544\\
73.79	0.000851721507977079\\
73.8	0.000851878918775867\\
73.81	0.000852036462959201\\
73.82	0.000852194140782296\\
73.83	0.000852351952509118\\
73.84	0.00085250989841326\\
73.85	0.000852667978778875\\
73.86	0.000852826193901658\\
73.87	0.00085298454408988\\
73.88	0.000853143029665493\\
73.89	0.000853301650965277\\
73.9	0.000853460408342069\\
73.91	0.000853619302166039\\
73.92	0.000853778332826051\\
73.93	0.000853937500731093\\
73.94	0.000854096806311771\\
73.95	0.000854256250021895\\
73.96	0.000854415832340142\\
73.97	0.000854575553771801\\
73.98	0.000854735414850609\\
73.99	0.000854895416140684\\
74	0.00085505555823855\\
74.01	0.000855215841775262\\
74.02	0.000855376267418641\\
74.03	0.000855536835875617\\
74.04	0.000855697547894683\\
74.05	0.00085585840426847\\
74.06	0.000856019405836449\\
74.07	0.000856180553487759\\
74.08	0.000856341848164167\\
74.09	0.000856503290863175\\
74.1	0.00085666488260152\\
74.11	0.000856826624411878\\
74.12	0.00085698851734318\\
74.13	0.000857150562460919\\
74.14	0.000857312760847478\\
74.15	0.000857475113602452\\
74.16	0.000857637621842995\\
74.17	0.000857800286704158\\
74.18	0.000857963109339239\\
74.19	0.000858126090920152\\
74.2	0.000858289232637789\\
74.21	0.000858452535702403\\
74.22	0.00085861600134399\\
74.23	0.000858779630812689\\
74.24	0.000858943425379181\\
74.25	0.000859107386335108\\
74.26	0.000859271514993497\\
74.27	0.000859435812689196\\
74.28	0.000859600280779312\\
74.29	0.000859764920643671\\
74.3	0.000859929733685288\\
74.31	0.000860094721330838\\
74.32	0.000860259885031144\\
74.33	0.000860425226261688\\
74.34	0.000860590746523112\\
74.35	0.000860756447341751\\
74.36	0.000860922330270169\\
74.37	0.000861088396887711\\
74.38	0.000861254648801069\\
74.39	0.000861421087644858\\
74.4	0.000861587715082213\\
74.41	0.000861754532805398\\
74.42	0.000861921542536425\\
74.43	0.000862088746027697\\
74.44	0.00086225614506266\\
74.45	0.000862423741456481\\
74.46	0.000862591537056723\\
74.47	0.000862759533744069\\
74.48	0.000862927733433031\\
74.49	0.000863096138072698\\
74.5	0.000863264749647499\\
74.51	0.00086343357017798\\
74.52	0.000863602601721608\\
74.53	0.00086377184637359\\
74.54	0.000863941306267716\\
74.55	0.000864110983577226\\
74.56	0.000864280880515694\\
74.57	0.000864450999337936\\
74.58	0.000864621342340954\\
74.59	0.000864791911864881\\
74.6	0.000864962710293975\\
74.61	0.000865133740057622\\
74.62	0.000865305003631375\\
74.63	0.000865476503538017\\
74.64	0.000865648242348652\\
74.65	0.000865820222683828\\
74.66	0.000865992447214685\\
74.67	0.00086616491866414\\
74.68	0.000866337639808097\\
74.69	0.000866510613476694\\
74.7	0.000866683842555584\\
74.71	0.00086685732998725\\
74.72	0.000867031078772347\\
74.73	0.0008672050919711\\
74.74	0.000867379372704723\\
74.75	0.000867553924156878\\
74.76	0.000867728749575185\\
74.77	0.000867903852272764\\
74.78	0.000868079235629822\\
74.79	0.00086825490309528\\
74.8	0.000868430858188456\\
74.81	0.000868607059687372\\
74.82	0.000868783397481305\\
74.83	0.000868959872187233\\
74.84	0.000869136484429952\\
74.85	0.000869313234842189\\
74.86	0.000869490124064726\\
74.87	0.000869667152746527\\
74.88	0.000869844321544862\\
74.89	0.000870021631125437\\
74.9	0.000870199082162524\\
74.91	0.000870376675339093\\
74.92	0.00087055441134695\\
74.93	0.000870732290886871\\
74.94	0.00087091031466874\\
74.95	0.000871088483411695\\
74.96	0.000871266797844269\\
74.97	0.000871445258704537\\
74.98	0.000871623866740263\\
74.99	0.000871802622709056\\
75	0.000871981527378517\\
75.01	0.000872160581526399\\
75.02	0.000872339785940765\\
75.03	0.00087251914142015\\
75.04	0.00087269864877372\\
75.05	0.000872878308821446\\
75.06	0.00087305812239427\\
75.07	0.000873238090334271\\
75.08	0.000873418213494853\\
75.09	0.000873598492740909\\
75.1	0.00087377892894901\\
75.11	0.000873959523007587\\
75.12	0.000874140275817115\\
75.13	0.000874321188290306\\
75.14	0.000874502261352302\\
75.15	0.000874683495940866\\
75.16	0.000874864893006588\\
75.17	0.000875046453513084\\
75.18	0.000875228178437201\\
75.19	0.000875410068769228\\
75.2	0.00087559212551311\\
75.21	0.000875774349686663\\
75.22	0.000875956742321791\\
75.23	0.000876139304464715\\
75.24	0.000876322037176198\\
75.25	0.000876504941531775\\
75.26	0.000876688018621986\\
75.27	0.000876871269552625\\
75.28	0.000877054695444966\\
75.29	0.000877238297436028\\
75.3	0.000877422076678814\\
75.31	0.000877606034342567\\
75.32	0.000877790171613036\\
75.33	0.000877974489692736\\
75.34	0.000878158989801215\\
75.35	0.000878343673175328\\
75.36	0.000878528541069514\\
75.37	0.000878713594756081\\
75.38	0.000878898835525487\\
75.39	0.000879084264686638\\
75.4	0.00087926988356718\\
75.41	0.000879455693513801\\
75.42	0.000879641695892542\\
75.43	0.0008798278920891\\
75.44	0.000880014283509157\\
75.45	0.000880200871578691\\
75.46	0.000880387657744309\\
75.47	0.000880574643473582\\
75.48	0.000880761830255383\\
75.49	0.000880949219600229\\
75.5	0.000881136813040638\\
75.51	0.000881324612131475\\
75.52	0.000881512618450328\\
75.53	0.00088170083359787\\
75.54	0.000881889259198233\\
75.55	0.00088207789689939\\
75.56	0.000882266748373549\\
75.57	0.00088245581531754\\
75.58	0.000882645099453223\\
75.59	0.000882834602527893\\
75.6	0.000883024326314697\\
75.61	0.000883214272613057\\
75.62	0.000883404443249098\\
75.63	0.000883594840076087\\
75.64	0.000883785464974876\\
75.65	0.00088397631985436\\
75.66	0.000884167406651931\\
75.67	0.00088435872733395\\
75.68	0.000884550283896222\\
75.69	0.000884742078364481\\
75.7	0.000884934112794883\\
75.71	0.00088512638927451\\
75.72	0.000885318909921879\\
75.73	0.000885511676887458\\
75.74	0.000885704692354199\\
75.75	0.000885897958538074\\
75.76	0.000886091477688622\\
75.77	0.000886285252089504\\
75.78	0.000886479284059074\\
75.79	0.000886673575950946\\
75.8	0.000886868130154593\\
75.81	0.000887062949095931\\
75.82	0.000887258035237934\\
75.83	0.000887453391081251\\
75.84	0.000887649019164828\\
75.85	0.000887844922066559\\
75.86	0.000888041102403925\\
75.87	0.000888237562834664\\
75.88	0.000888434306057445\\
75.89	0.00088863133481255\\
75.9	0.000888828651882575\\
75.91	0.000889026260093141\\
75.92	0.000889224162313615\\
75.93	0.000889422361457847\\
75.94	0.000889620860484922\\
75.95	0.00088981966239992\\
75.96	0.000890018770254689\\
75.97	0.000890218187148644\\
75.98	0.000890417916229562\\
75.99	0.000890617960694404\\
76	0.000890818323790151\\
76.01	0.000891019008814649\\
76.02	0.000891220019117471\\
76.03	0.000891421358100802\\
76.04	0.000891623029220327\\
76.05	0.000891825035986145\\
76.06	0.000892027381963699\\
76.07	0.000892230070774715\\
76.08	0.000892433106098166\\
76.09	0.000892636491671255\\
76.1	0.000892840231290399\\
76.11	0.000893044328812258\\
76.12	0.000893248788154758\\
76.13	0.000893453613298149\\
76.14	0.00089365880828607\\
76.15	0.000893864377226647\\
76.16	0.000894070324293596\\
76.17	0.000894276653727358\\
76.18	0.00089448336983625\\
76.19	0.000894690476997638\\
76.2	0.000894897979659129\\
76.21	0.000895105882339788\\
76.22	0.000895314189631381\\
76.23	0.00089552290619963\\
76.24	0.0008957320367855\\
76.25	0.000895941586206509\\
76.26	0.000896151559358059\\
76.27	0.000896361961214795\\
76.28	0.000896572796831984\\
76.29	0.00089678407134693\\
76.3	0.0008969957899804\\
76.31	0.000897207958038094\\
76.32	0.000897420580912127\\
76.33	0.000897633664082548\\
76.34	0.000897847213118884\\
76.35	0.000898061233681713\\
76.36	0.000898275731524272\\
76.37	0.000898490712494083\\
76.38	0.000898706182534622\\
76.39	0.000898922147687015\\
76.4	0.000899138614091761\\
76.41	0.000899355587990494\\
76.42	0.000899573075727781\\
76.43	0.000899791083752939\\
76.44	0.000900009618621911\\
76.45	0.00090022868699915\\
76.46	0.000900448295659561\\
76.47	0.000900668451490467\\
76.48	0.000900889161493619\\
76.49	0.000901110432787242\\
76.5	0.00090133227260812\\
76.51	0.000901554688313719\\
76.52	0.000901777687384361\\
76.53	0.000902001277425422\\
76.54	0.000902225466169589\\
76.55	0.00090245026147915\\
76.56	0.000902675671348333\\
76.57	0.000902901703905692\\
76.58	0.000903128367416531\\
76.59	0.000903355670285383\\
76.6	0.000903583621058534\\
76.61	0.000903812228426599\\
76.62	0.000904041501227141\\
76.63	0.000904271448447349\\
76.64	0.00090450207922676\\
76.65	0.000904733402860049\\
76.66	0.000904965428799854\\
76.67	0.00090519816665967\\
76.68	0.000905431626216795\\
76.69	0.000905665817415338\\
76.7	0.000905900750369282\\
76.71	0.000906136435365603\\
76.72	0.000906372882867467\\
76.73	0.000906610103517471\\
76.74	0.000906848108140958\\
76.75	0.0009070869077494\\
76.76	0.000907326513543841\\
76.77	0.00090756693691842\\
76.78	0.000907808189463944\\
76.79	0.000908050282971558\\
76.8	0.000908293229436467\\
76.81	0.000908537041061743\\
76.82	0.000908781730262209\\
76.83	0.000909027309668397\\
76.84	0.000909273792130582\\
76.85	0.000909521190722915\\
76.86	0.000909769518747612\\
76.87	0.000910018789739255\\
76.88	0.000910269017469158\\
76.89	0.000910520215949839\\
76.9	0.000910772399439572\\
76.91	0.000911025582447033\\
76.92	0.000911279779736042\\
76.93	0.00091153500633041\\
76.94	0.000911791277518866\\
76.95	0.000912048608860105\\
76.96	0.000912307016187929\\
76.97	0.000912566515616492\\
76.98	0.000912827123545661\\
76.99	0.000913088856666477\\
77	0.00091335173196674\\
77.01	0.000913615766736705\\
77.02	0.000913880978574887\\
77.03	0.000914147385394\\
77.04	0.00091441500542702\\
77.05	0.000914683857233353\\
77.06	0.00091495395970516\\
77.07	0.000915225332073796\\
77.08	0.000915497993916386\\
77.09	0.000915771965162545\\
77.1	0.000916047266101229\\
77.11	0.000916323917387746\\
77.12	0.000916601940050894\\
77.13	0.000916881355500268\\
77.14	0.000917162185533707\\
77.15	0.000917444452344913\\
77.16	0.000917728178531215\\
77.17	0.000918013387101511\\
77.18	0.000918300101484373\\
77.19	0.000918588345536327\\
77.2	0.000918878143550303\\
77.21	0.000919169520264279\\
77.22	0.000919462500870095\\
77.23	0.000919757111022467\\
77.24	0.00092005337684819\\
77.25	0.000920351324955535\\
77.26	0.000920650982443854\\
77.27	0.000920952376913399\\
77.28	0.000921255536475336\\
77.29	0.000921560489761988\\
77.3	0.000921867265937298\\
77.31	0.00092217589470751\\
77.32	0.00092248640633206\\
77.33	0.000922798831634711\\
77.34	0.000923113202014924\\
77.35	0.000923429549459471\\
77.36	0.000923747906554313\\
77.37	0.000924068306496726\\
77.38	0.000924390783107692\\
77.39	0.000924715370844566\\
77.4	0.000925042104814014\\
77.41	0.000925371020785236\\
77.42	0.000925702155203482\\
77.43	0.000926035545203862\\
77.44	0.000926371228625452\\
77.45	0.00092670924402573\\
77.46	0.000927049630695308\\
77.47	0.000927392428673006\\
77.48	0.000927737678761248\\
77.49	0.000928085422541807\\
77.5	0.000928435702391896\\
77.51	0.000928788561500611\\
77.52	0.000929144043885754\\
77.53	0.000929502194411014\\
77.54	0.000929863058803541\\
77.55	0.000930226683671917\\
77.56	0.000930593116524509\\
77.57	0.000930962405788261\\
77.58	0.000931334600827886\\
77.59	0.0009317097519655\\
77.6	0.000932087910500692\\
77.61	0.000932469128731054\\
77.62	0.000932853299300599\\
77.63	0.00093323764936913\\
77.64	0.000933622178979347\\
77.65	0.000934006888177354\\
77.66	0.000934391777012806\\
77.67	0.000934776845539071\\
77.68	0.000935162093813382\\
77.69	0.000935547521897008\\
77.7	0.000935933129855411\\
77.71	0.000936318917758425\\
77.72	0.000936704885680423\\
77.73	0.0009370910337005\\
77.74	0.000937477361902654\\
77.75	0.00093786387037597\\
77.76	0.000938250559214811\\
77.77	0.000938637428519016\\
77.78	0.000939024478394088\\
77.79	0.000939411708951409\\
77.8	0.000939799120308438\\
77.81	0.000940186712588924\\
77.82	0.000940574485923119\\
77.83	0.000940962440448001\\
77.84	0.000941350576307494\\
77.85	0.000941738893652696\\
77.86	0.000942127392642112\\
77.87	0.000942516073441887\\
77.88	0.000942904936226049\\
77.89	0.000943293981176754\\
77.9	0.000943683208484531\\
77.91	0.000944072618348538\\
77.92	0.000944462210976819\\
77.93	0.000944851986586564\\
77.94	0.000945241945404376\\
77.95	0.000945632087666539\\
77.96	0.000946022413619295\\
77.97	0.000946412923519114\\
77.98	0.000946803617632986\\
77.99	0.000947194496238701\\
78	0.000947585559625139\\
78.01	0.000947976808092561\\
78.02	0.000948368241952914\\
78.03	0.000948759861530122\\
78.04	0.000949151667160395\\
78.05	0.00094954365919254\\
78.06	0.000949935837988261\\
78.07	0.000950328203922487\\
78.08	0.000950720757383675\\
78.09	0.00095111349877414\\
78.1	0.000951506428510369\\
78.11	0.000951899547023355\\
78.12	0.000952292854758913\\
78.13	0.000952686352178019\\
78.14	0.000953080039757134\\
78.15	0.000953473917988537\\
78.16	0.000953867987380662\\
78.17	0.000954262248458428\\
78.18	0.000954656701763578\\
78.19	0.000955051347855011\\
78.2	0.000955446187309117\\
78.21	0.000955841220720116\\
78.22	0.00095623644870039\\
78.23	0.000956631871880814\\
78.24	0.000957027490911085\\
78.25	0.000957423306460065\\
78.26	0.000957819319216086\\
78.27	0.000958215529887294\\
78.28	0.000958611939201956\\
78.29	0.000959008547908778\\
78.3	0.000959405356777221\\
78.31	0.000959802366597796\\
78.32	0.000960199578182374\\
78.33	0.000960596992364469\\
78.34	0.000960994609999528\\
78.35	0.000961392431965202\\
78.36	0.000961790459161616\\
78.37	0.000962188692511622\\
78.38	0.000962587132961048\\
78.39	0.000962985781478925\\
78.4	0.000963384639057711\\
78.41	0.000963783706713493\\
78.42	0.000964182985486176\\
78.43	0.000964582476439662\\
78.44	0.000964982180661997\\
78.45	0.00096538209926551\\
78.46	0.000965782233386931\\
78.47	0.000966182584187481\\
78.48	0.00096658315285294\\
78.49	0.000966983940593696\\
78.5	0.000967384948644758\\
78.51	0.000967786178265748\\
78.52	0.000968187630740854\\
78.53	0.000968589307378765\\
78.54	0.000968991209512557\\
78.55	0.000969393338499545\\
78.56	0.000969795695721108\\
78.57	0.000970198282582456\\
78.58	0.000970601100512369\\
78.59	0.000971004150962879\\
78.6	0.000971407435408911\\
78.61	0.000971810955347867\\
78.62	0.00097221471229916\\
78.63	0.000972618707803689\\
78.64	0.000973022943423257\\
78.65	0.000973427420739918\\
78.66	0.000973832141355265\\
78.67	0.000974237106889646\\
78.68	0.000974642318981301\\
78.69	0.000975047779285428\\
78.7	0.000975453489473159\\
78.71	0.000975859451230462\\
78.72	0.000976265666256936\\
78.73	0.000976672136264525\\
78.74	0.000977078862976124\\
78.75	0.000977485848124083\\
78.76	0.000977893093448604\\
78.77	0.00097830060069601\\
78.78	0.000978708371616914\\
78.79	0.000979116407964237\\
78.8	0.000979524711491112\\
78.81	0.000979933283948634\\
78.82	0.00098034212708348\\
78.83	0.000980751242635358\\
78.84	0.00098116063233431\\
78.85	0.000981570297897836\\
78.86	0.000981980241027853\\
78.87	0.000982390463407458\\
78.88	0.000982800966697514\\
78.89	0.000983211752533023\\
78.9	0.000983622822519287\\
78.91	0.00098403417822786\\
78.92	0.000984445821192266\\
78.93	0.000984857752903469\\
78.94	0.000985269974805109\\
78.95	0.000985682488288451\\
78.96	0.000986095294687088\\
78.97	0.000986508395271337\\
78.98	0.000986921791242349\\
78.99	0.000987335483725902\\
79	0.000987749473837981\\
79.01	0.000988163762707569\\
79.02	0.000988578351476817\\
79.03	0.000988993241301223\\
79.04	0.000989408433349805\\
79.05	0.000989823928805285\\
79.06	0.000990239728864271\\
79.07	0.000990655834737439\\
79.08	0.000991072247649718\\
79.09	0.000991488968840479\\
79.1	0.000991905999563726\\
79.11	0.000992323341088289\\
79.12	0.000992740994698016\\
79.13	0.000993158961691968\\
79.14	0.000993577243384624\\
79.15	0.000993995841106073\\
79.16	0.000994414756202221\\
79.17	0.000994833990034999\\
79.18	0.000995253543982564\\
79.19	0.00099567341943951\\
79.2	0.00099609361781708\\
79.21	0.000996514140543381\\
79.22	0.000996934989063595\\
79.23	0.000997356164840198\\
79.24	0.000997777669353186\\
79.25	0.000998199504100286\\
79.26	0.000998621670597185\\
79.27	0.000999044170377762\\
79.28	0.000999467004994302\\
79.29	0.000999890176017744\\
79.3	0.0010003136850379\\
79.31	0.00100073753366368\\
79.32	0.00100116172352337\\
79.33	0.00100158625626481\\
79.34	0.0010020111335557\\
79.35	0.00100243635708378\\
79.36	0.00100286192855713\\
79.37	0.00100328784970436\\
79.38	0.00100371412227493\\
79.39	0.00100414074803931\\
79.4	0.00100456772878933\\
79.41	0.00100499506633835\\
79.42	0.00100542276252158\\
79.43	0.0010058508191963\\
79.44	0.00100627923824212\\
79.45	0.00100670802156126\\
79.46	0.00100713717107882\\
79.47	0.00100756668874302\\
79.48	0.00100799657652546\\
79.49	0.00100842683642144\\
79.5	0.00100885747045018\\
79.51	0.00100928848065509\\
79.52	0.0010097198691041\\
79.53	0.00101015163788986\\
79.54	0.00101058378913006\\
79.55	0.0010110163249677\\
79.56	0.00101144924757138\\
79.57	0.00101188255913555\\
79.58	0.00101231626188079\\
79.59	0.00101275035805415\\
79.6	0.00101318484992935\\
79.61	0.00101361973980713\\
79.62	0.00101405503001548\\
79.63	0.00101449072290999\\
79.64	0.00101492682087406\\
79.65	0.00101536332631925\\
79.66	0.00101580024168553\\
79.67	0.00101623756944157\\
79.68	0.00101667531208505\\
79.69	0.00101711347214291\\
79.7	0.00101755205217167\\
79.71	0.0010179910547577\\
79.72	0.00101843048251751\\
79.73	0.00101887033809804\\
79.74	0.00101931062417692\\
79.75	0.0010197513434628\\
79.76	0.00102019249869559\\
79.77	0.00102063409264676\\
79.78	0.00102107612811964\\
79.79	0.00102151860794967\\
79.8	0.00102196153500466\\
79.81	0.00102240491218515\\
79.82	0.00102284874242458\\
79.83	0.00102329302868963\\
79.84	0.00102373777398048\\
79.85	0.00102418298133103\\
79.86	0.00102462865380925\\
79.87	0.00102507479451735\\
79.88	0.0010255214065921\\
79.89	0.00102596849320504\\
79.9	0.00102641605756278\\
79.91	0.00102686410290719\\
79.92	0.00102731263251568\\
79.93	0.00102776164970142\\
79.94	0.00102821115781357\\
79.95	0.00102866116023751\\
79.96	0.00102911166039507\\
79.97	0.00102956266174472\\
79.98	0.00103001416778179\\
79.99	0.00103046618203866\\
80	0.001030918708085\\
80.01	0.00103137174952786\\
};
\addplot [color=black,solid]
  table[row sep=crcr]{%
80.01	0.00103137174952786\\
80.02	0.00103182531001195\\
80.03	0.00103227939321974\\
80.04	0.00103273400287165\\
80.05	0.00103318914272617\\
80.06	0.00103364481658007\\
80.07	0.00103410102826842\\
80.08	0.00103455778166481\\
80.09	0.00103501508068141\\
80.1	0.00103547292926907\\
80.11	0.00103593133141739\\
80.12	0.00103639029115482\\
80.13	0.0010368498125487\\
80.14	0.0010373098997053\\
80.15	0.00103777055676983\\
80.16	0.00103823178792649\\
80.17	0.00103869359739843\\
80.18	0.00103915598944773\\
80.19	0.00103961896837538\\
80.2	0.0010400825385212\\
80.21	0.00104054670426376\\
80.22	0.00104101147002031\\
80.23	0.00104147684024661\\
80.24	0.00104194281943683\\
80.25	0.00104240941212337\\
80.26	0.00104287662287665\\
80.27	0.00104334445630494\\
80.28	0.00104381291705408\\
80.29	0.00104428200980724\\
80.3	0.00104475173928458\\
80.31	0.00104522211024301\\
80.32	0.00104569312747575\\
80.33	0.00104616479581198\\
80.34	0.00104663712011644\\
80.35	0.00104711010528895\\
80.36	0.00104758375626394\\
80.37	0.00104805807800992\\
80.38	0.00104853307552892\\
80.39	0.00104900875385589\\
80.4	0.00104948511805809\\
80.41	0.00104996217323433\\
80.42	0.00105043992451434\\
80.43	0.00105091837705793\\
80.44	0.00105139753605422\\
80.45	0.00105187740672071\\
80.46	0.00105235799430243\\
80.47	0.00105283930407096\\
80.48	0.00105332134132337\\
80.49	0.00105380411138116\\
80.5	0.00105428761958914\\
80.51	0.00105477187131422\\
80.52	0.00105525687194416\\
80.53	0.00105574262688623\\
80.54	0.00105622914156584\\
80.55	0.00105671642142507\\
80.56	0.00105720447192115\\
80.57	0.00105769329852485\\
80.58	0.00105818290671882\\
80.59	0.00105867330199581\\
80.6	0.00105916448985684\\
80.61	0.00105965647580928\\
80.62	0.00106014926536485\\
80.63	0.00106064286403751\\
80.64	0.00106113727734125\\
80.65	0.00106163251078785\\
80.66	0.00106212856988445\\
80.67	0.00106262546013108\\
80.68	0.00106312318701804\\
80.69	0.00106362175602325\\
80.7	0.00106412117260936\\
80.71	0.00106462144222087\\
80.72	0.00106512257028106\\
80.73	0.00106562456218878\\
80.74	0.0010661274233152\\
80.75	0.00106663115900029\\
80.76	0.00106713577454932\\
80.77	0.0010676412752291\\
80.78	0.0010681476662641\\
80.79	0.00106865495283245\\
80.8	0.00106916314006178\\
80.81	0.00106967223302486\\
80.82	0.00107018223673511\\
80.83	0.00107069315614193\\
80.84	0.00107120499612585\\
80.85	0.00107171776149348\\
80.86	0.00107223145697233\\
80.87	0.00107274608720538\\
80.88	0.00107326165674543\\
80.89	0.00107377817004933\\
80.9	0.00107429563147193\\
80.91	0.00107481404525982\\
80.92	0.00107533341554484\\
80.93	0.00107585374633741\\
80.94	0.0010763750415195\\
80.95	0.00107689730483751\\
80.96	0.00107742053989475\\
80.97	0.00107794475014374\\
80.98	0.00107846993887819\\
80.99	0.00107899610922477\\
81	0.00107952326413447\\
81.01	0.00108005140637381\\
81.02	0.0010805805385156\\
81.03	0.00108111066292949\\
81.04	0.00108164178177212\\
81.05	0.00108217389697699\\
81.06	0.00108270701024393\\
81.07	0.00108324112302827\\
81.08	0.00108377623652955\\
81.09	0.00108431235167998\\
81.1	0.00108484946913237\\
81.11	0.00108538758924776\\
81.12	0.0010859267120826\\
81.13	0.00108646683737546\\
81.14	0.00108700796453335\\
81.15	0.0010875500926176\\
81.16	0.00108809322032918\\
81.17	0.00108863734599366\\
81.18	0.00108918246754556\\
81.19	0.00108972858251226\\
81.2	0.00109027568799734\\
81.21	0.00109082378066344\\
81.22	0.00109137285671444\\
81.23	0.00109192291187722\\
81.24	0.00109247394138267\\
81.25	0.00109302593994621\\
81.26	0.00109357890174761\\
81.27	0.00109413282041017\\
81.28	0.00109468768897928\\
81.29	0.00109524349990022\\
81.3	0.00109580024499531\\
81.31	0.00109635791544029\\
81.32	0.00109691650173997\\
81.33	0.00109747599370312\\
81.34	0.00109803638041651\\
81.35	0.00109859765021823\\
81.36	0.00109915979067001\\
81.37	0.00109972278852884\\
81.38	0.00110028662971761\\
81.39	0.00110085129929481\\
81.4	0.00110141678142332\\
81.41	0.00110198305933824\\
81.42	0.00110255011531366\\
81.43	0.00110311793062846\\
81.44	0.00110368648553098\\
81.45	0.00110425575920269\\
81.46	0.00110482572972059\\
81.47	0.00110539637401863\\
81.48	0.00110596766784778\\
81.49	0.00110653958573493\\
81.5	0.00110711210094061\\
81.51	0.00110768518541523\\
81.52	0.00110825883518859\\
81.53	0.00110883305497247\\
81.54	0.00110940784955213\\
81.55	0.00110998322378759\\
81.56	0.00111055918261489\\
81.57	0.00111113573104737\\
81.58	0.00111171287417704\\
81.59	0.00111229061717586\\
81.6	0.00111286896529715\\
81.61	0.00111344792387699\\
81.62	0.00111402749833559\\
81.63	0.00111460769417879\\
81.64	0.00111518851699948\\
81.65	0.00111576997247912\\
81.66	0.00111635206638927\\
81.67	0.00111693480459312\\
81.68	0.00111751819304706\\
81.69	0.00111810223780232\\
81.7	0.00111868694500658\\
81.71	0.00111927232090563\\
81.72	0.00111985837184508\\
81.73	0.00112044510427207\\
81.74	0.00112103252473706\\
81.75	0.00112162063989557\\
81.76	0.00112220945651002\\
81.77	0.0011227989814516\\
81.78	0.00112338922170216\\
81.79	0.00112398018435609\\
81.8	0.00112457187662233\\
81.81	0.00112516430582632\\
81.82	0.00112575747941207\\
81.83	0.00112635140494417\\
81.84	0.00112694609010998\\
81.85	0.00112754154272169\\
81.86	0.00112813777071854\\
81.87	0.00112873478216904\\
81.88	0.00112933258527326\\
81.89	0.00112993118836508\\
81.9	0.00113053059991458\\
81.91	0.00113113082853041\\
81.92	0.00113173188296225\\
81.93	0.00113233377210329\\
81.94	0.0011329365049927\\
81.95	0.00113354009081831\\
81.96	0.00113414453891912\\
81.97	0.00113474985878806\\
81.98	0.00113535606007465\\
81.99	0.0011359631525878\\
82	0.00113657114629866\\
82.01	0.00113718005134342\\
82.02	0.00113778987802633\\
82.03	0.0011384006368226\\
82.04	0.00113901233838153\\
82.05	0.00113962499352952\\
82.06	0.00114023861327332\\
82.07	0.00114085320880314\\
82.08	0.00114146879149602\\
82.09	0.00114208537291914\\
82.1	0.0011427029648332\\
82.11	0.0011433215791959\\
82.12	0.00114394122816549\\
82.13	0.00114456192410437\\
82.14	0.00114518367958272\\
82.15	0.00114580650738228\\
82.16	0.00114643042050015\\
82.17	0.00114705543215266\\
82.18	0.00114768155577934\\
82.19	0.00114830880504696\\
82.2	0.00114893719385362\\
82.21	0.00114956673633295\\
82.22	0.00115019744685838\\
82.23	0.00115082934004751\\
82.24	0.00115146243076651\\
82.25	0.00115209673413467\\
82.26	0.00115273226552901\\
82.27	0.00115336904058896\\
82.28	0.00115400707522116\\
82.29	0.00115464638560436\\
82.3	0.00115528698819437\\
82.31	0.00115592889972916\\
82.32	0.00115657213723402\\
82.33	0.00115721671802684\\
82.34	0.00115786265972349\\
82.35	0.00115850998024331\\
82.36	0.00115915869781467\\
82.37	0.00115980883098072\\
82.38	0.00116046039860519\\
82.39	0.00116111341987829\\
82.4	0.00116176791432278\\
82.41	0.00116242390180014\\
82.42	0.00116308140251685\\
82.43	0.00116374043703079\\
82.44	0.00116440102625781\\
82.45	0.00116506319147835\\
82.46	0.00116572695434431\\
82.47	0.00116639233688593\\
82.48	0.00116705936151888\\
82.49	0.00116772805105151\\
82.5	0.00116839842869217\\
82.51	0.00116907051805674\\
82.52	0.00116974434317631\\
82.53	0.00117041992850495\\
82.54	0.00117109729892773\\
82.55	0.00117177647976883\\
82.56	0.00117245749679984\\
82.57	0.00117314037624823\\
82.58	0.00117382514480598\\
82.59	0.00117451182963844\\
82.6	0.00117520045839323\\
82.61	0.00117589105920951\\
82.62	0.0011765836607273\\
82.63	0.00117727829209703\\
82.64	0.00117797498298932\\
82.65	0.00117867376360489\\
82.66	0.00117937466468476\\
82.67	0.00118007771752057\\
82.68	0.00118078295396519\\
82.69	0.00118149040644351\\
82.7	0.00118220010796344\\
82.71	0.00118291209212718\\
82.72	0.00118362639314266\\
82.73	0.00118434304583533\\
82.74	0.00118506208566005\\
82.75	0.00118578354871333\\
82.76	0.0011865074717458\\
82.77	0.00118723389217492\\
82.78	0.00118796284809799\\
82.79	0.00118869437830541\\
82.8	0.0011894285222942\\
82.81	0.00119016532028187\\
82.82	0.00119090481322048\\
82.83	0.00119164704281113\\
82.84	0.00119239205151861\\
82.85	0.00119313988258647\\
82.86	0.00119389058005238\\
82.87	0.00119464418876378\\
82.88	0.00119540075439388\\
82.89	0.00119616032345807\\
82.9	0.00119692294333056\\
82.91	0.00119768866226145\\
82.92	0.00119845752939418\\
82.93	0.00119922959478328\\
82.94	0.00120000490941258\\
82.95	0.00120078352521378\\
82.96	0.00120156549508538\\
82.97	0.00120235087291211\\
82.98	0.00120313971358466\\
82.99	0.00120393207301996\\
83	0.00120472800818183\\
83.01	0.00120552757710205\\
83.02	0.00120633083890199\\
83.03	0.0012071378538146\\
83.04	0.00120794868320696\\
83.05	0.00120876338960326\\
83.06	0.00120958203670836\\
83.07	0.00121040468943177\\
83.08	0.00121123141391226\\
83.09	0.00121206227754291\\
83.1	0.00121289734899678\\
83.11	0.00121373669825316\\
83.12	0.0012145803966243\\
83.13	0.00121542851678286\\
83.14	0.00121628113278987\\
83.15	0.00121713832012336\\
83.16	0.00121800015570763\\
83.17	0.00121886671794311\\
83.18	0.00121973808673699\\
83.19	0.00122061434353445\\
83.2	0.00122149557135061\\
83.21	0.00122238185480321\\
83.22	0.00122327328014602\\
83.23	0.001224169935303\\
83.24	0.00122507190990321\\
83.25	0.00122597929531654\\
83.26	0.00122689218469025\\
83.27	0.00122781067298629\\
83.28	0.00122873485701948\\
83.29	0.00122966483549664\\
83.3	0.00123060070905649\\
83.31	0.00123154258031051\\
83.32	0.00123249055388477\\
83.33	0.00123344473646265\\
83.34	0.00123440523682859\\
83.35	0.00123537216591278\\
83.36	0.001236345636837\\
83.37	0.00123732576496133\\
83.38	0.00123831266793213\\
83.39	0.00123930646573096\\
83.4	0.00124030728072478\\
83.41	0.00124131523771717\\
83.42	0.00124233046400085\\
83.43	0.00124335308941136\\
83.44	0.00124438324638201\\
83.45	0.00124542107000009\\
83.46	0.00124646669806443\\
83.47	0.00124752027114426\\
83.48	0.00124858193263948\\
83.49	0.00124965182884234\\
83.5	0.00125073010900052\\
83.51	0.00125181692538181\\
83.52	0.00125291243334014\\
83.53	0.00125401679138337\\
83.54	0.00125513016124248\\
83.55	0.00125625270794258\\
83.56	0.00125738459987546\\
83.57	0.00125852317353368\\
83.58	0.00125966227977767\\
83.59	0.0012608019191867\\
83.6	0.00126194209234181\\
83.61	0.00126308279982579\\
83.62	0.00126422404222321\\
83.63	0.00126536582012046\\
83.64	0.00126650813410573\\
83.65	0.00126765098476907\\
83.66	0.00126879437270237\\
83.67	0.00126993829849939\\
83.68	0.00127108276275582\\
83.69	0.00127222776606923\\
83.7	0.00127337330903915\\
83.71	0.00127451939226705\\
83.72	0.0012756660163564\\
83.73	0.00127681318191264\\
83.74	0.00127796088954326\\
83.75	0.00127910913985777\\
83.76	0.00128025793346775\\
83.77	0.00128140727098688\\
83.78	0.00128255715303094\\
83.79	0.00128370758021783\\
83.8	0.00128485855316764\\
83.81	0.00128601007250261\\
83.82	0.00128716213884719\\
83.83	0.00128831475282808\\
83.84	0.00128946791507421\\
83.85	0.00129062162621681\\
83.86	0.00129177588688941\\
83.87	0.00129293069772787\\
83.88	0.0012940860593704\\
83.89	0.00129524197245762\\
83.9	0.00129639843763255\\
83.91	0.00129755545554066\\
83.92	0.00129871302682987\\
83.93	0.00129987115215063\\
83.94	0.0013010298321559\\
83.95	0.00130218906750121\\
83.96	0.00130334885884467\\
83.97	0.00130450920684702\\
83.98	0.00130567011217164\\
83.99	0.00130683157548461\\
84	0.0013079935974547\\
84.01	0.00130915617875347\\
84.02	0.00131031932005522\\
84.03	0.00131148302203709\\
84.04	0.00131264728537907\\
84.05	0.00131381211076402\\
84.06	0.00131497749887775\\
84.07	0.001316143450409\\
84.08	0.00131730996604952\\
84.09	0.00131847704649408\\
84.1	0.00131964469244054\\
84.11	0.00132081290458985\\
84.12	0.00132198168364611\\
84.13	0.00132315103031661\\
84.14	0.00132432094531187\\
84.15	0.00132549142934568\\
84.16	0.00132666248313514\\
84.17	0.00132783410740069\\
84.18	0.0013290063028662\\
84.19	0.00133017907025894\\
84.2	0.00133135241030969\\
84.21	0.00133252632375275\\
84.22	0.001333700811326\\
84.23	0.00133487587377096\\
84.24	0.00133605151183278\\
84.25	0.00133722772626037\\
84.26	0.00133840451780638\\
84.27	0.0013395818872273\\
84.28	0.00134075983528346\\
84.29	0.00134193836273915\\
84.3	0.00134311747036259\\
84.31	0.00134429715892606\\
84.32	0.0013454774292059\\
84.33	0.0013466582819826\\
84.34	0.00134783971804082\\
84.35	0.00134902173816948\\
84.36	0.00135020434316182\\
84.37	0.00135138753381542\\
84.38	0.0013525713109323\\
84.39	0.00135375567531895\\
84.4	0.00135494062778642\\
84.41	0.00135612616915036\\
84.42	0.0013573123002311\\
84.43	0.00135849902185371\\
84.44	0.00135968633484804\\
84.45	0.00136087424004884\\
84.46	0.00136206273829578\\
84.47	0.00136325183043354\\
84.48	0.00136444151731187\\
84.49	0.0013656317997857\\
84.5	0.00136682267871515\\
84.51	0.00136801415496563\\
84.52	0.00136920622940795\\
84.53	0.00137039890291835\\
84.54	0.00137159217637858\\
84.55	0.00137278605067602\\
84.56	0.00137398052670371\\
84.57	0.00137517560536048\\
84.58	0.00137637128755097\\
84.59	0.00137756757418578\\
84.6	0.00137876446618153\\
84.61	0.00137996196446091\\
84.62	0.00138116006995283\\
84.63	0.00138235878359249\\
84.64	0.00138355810632143\\
84.65	0.00138475803908768\\
84.66	0.00138595858284583\\
84.67	0.00138715973855711\\
84.68	0.00138836150718952\\
84.69	0.00138956388971791\\
84.7	0.00139076688712408\\
84.71	0.00139197050039688\\
84.72	0.00139317473053235\\
84.73	0.00139437957853376\\
84.74	0.00139558504541177\\
84.75	0.00139679113218453\\
84.76	0.00139799783987778\\
84.77	0.00139920516952497\\
84.78	0.00140041312216736\\
84.79	0.00140162169885418\\
84.8	0.0014028309006427\\
84.81	0.00140404072859837\\
84.82	0.00140525118379495\\
84.83	0.00140646226731463\\
84.84	0.00140767398024815\\
84.85	0.00140888632369494\\
84.86	0.00141009929876325\\
84.87	0.00141131290657028\\
84.88	0.00141252714824231\\
84.89	0.00141374202491485\\
84.9	0.00141495753773278\\
84.91	0.00141617368785047\\
84.92	0.00141739047643198\\
84.93	0.00141860790465113\\
84.94	0.00141982597369172\\
84.95	0.00142104468474764\\
84.96	0.00142226403902305\\
84.97	0.00142348403773252\\
84.98	0.00142470468210121\\
84.99	0.001425925973365\\
85	0.00142714791277071\\
85.01	0.00142837050157621\\
85.02	0.00142959374105063\\
85.03	0.00143081763247452\\
85.04	0.00143204217714003\\
85.05	0.0014332673763511\\
85.06	0.00143449323142362\\
85.07	0.00143571974368564\\
85.08	0.00143694691447756\\
85.09	0.00143817474515231\\
85.1	0.00143940323707554\\
85.11	0.00144063239162585\\
85.12	0.00144186221019495\\
85.13	0.00144309269418792\\
85.14	0.00144432384502336\\
85.15	0.00144555566413367\\
85.16	0.00144678815296519\\
85.17	0.0014480213129785\\
85.18	0.00144925514564859\\
85.19	0.00145048965246511\\
85.2	0.00145172483493259\\
85.21	0.00145296069457069\\
85.22	0.00145419723291444\\
85.23	0.00145543445151448\\
85.24	0.00145667235193729\\
85.25	0.00145791093576548\\
85.26	0.00145915020459803\\
85.27	0.00146039016005053\\
85.28	0.00146163080375549\\
85.29	0.00146287213736256\\
85.3	0.00146411416253886\\
85.31	0.00146535688096921\\
85.32	0.00146660029435644\\
85.33	0.00146784440442167\\
85.34	0.00146908921290464\\
85.35	0.00147033472156394\\
85.36	0.00147158093217738\\
85.37	0.00147282784654225\\
85.38	0.00147407546647569\\
85.39	0.00147532379381497\\
85.4	0.0014765728304178\\
85.41	0.00147782257816271\\
85.42	0.00147907303894938\\
85.43	0.00148032421469892\\
85.44	0.00148157610735432\\
85.45	0.00148282871888072\\
85.46	0.00148408205126581\\
85.47	0.00148533610652019\\
85.48	0.00148659088667774\\
85.49	0.00148784639379601\\
85.5	0.0014891026299566\\
85.51	0.00149035959726556\\
85.52	0.00149161729785376\\
85.53	0.00149287573387735\\
85.54	0.00149413490751813\\
85.55	0.00149539482098397\\
85.56	0.0014966554765093\\
85.57	0.00149791687635544\\
85.58	0.00149917902281116\\
85.59	0.00150044191819303\\
85.6	0.00150170556484595\\
85.61	0.00150296996514358\\
85.62	0.00150423512148882\\
85.63	0.0015055010363143\\
85.64	0.00150676771208287\\
85.65	0.00150803515128811\\
85.66	0.00150930335645481\\
85.67	0.0015105723301395\\
85.68	0.001511842074931\\
85.69	0.00151311259345092\\
85.7	0.00151438388835423\\
85.71	0.00151565596232981\\
85.72	0.00151692881810099\\
85.73	0.00151820245842616\\
85.74	0.00151947688609934\\
85.75	0.00152075210395074\\
85.76	0.00152202811484742\\
85.77	0.00152330492169389\\
85.78	0.00152458252743269\\
85.79	0.0015258609350451\\
85.8	0.0015271401475517\\
85.81	0.00152842016801314\\
85.82	0.00152970099953068\\
85.83	0.00153098264524699\\
85.84	0.00153226510834678\\
85.85	0.0015335483920575\\
85.86	0.0015348324996501\\
85.87	0.00153611743443974\\
85.88	0.00153740319978652\\
85.89	0.00153868979909626\\
85.9	0.00153997723582124\\
85.91	0.00154126551346104\\
85.92	0.00154255463556327\\
85.93	0.00154384460572442\\
85.94	0.0015451354275907\\
85.95	0.00154642710485882\\
85.96	0.00154771964127692\\
85.97	0.00154901304064538\\
85.98	0.00155030730681776\\
85.99	0.00155160244370163\\
86	0.00155289845525956\\
86.01	0.00155419534551001\\
86.02	0.0015554931185283\\
86.03	0.00155679177844755\\
86.04	0.0015580913294597\\
86.05	0.00155939177581648\\
86.06	0.00156069312183045\\
86.07	0.00156199537187603\\
86.08	0.00156329853039054\\
86.09	0.0015646026018753\\
86.1	0.0015659075908967\\
86.11	0.00156721350208735\\
86.12	0.00156852034014716\\
86.13	0.00156982810984451\\
86.14	0.00157113681601746\\
86.15	0.0015724464635749\\
86.16	0.00157375705749778\\
86.17	0.00157506860284034\\
86.18	0.00157638110473138\\
86.19	0.00157769456837556\\
86.2	0.00157900899905464\\
86.21	0.00158032440212889\\
86.22	0.00158164078303837\\
86.23	0.00158295814730434\\
86.24	0.00158427650053067\\
86.25	0.00158559584840525\\
86.26	0.00158691619670142\\
86.27	0.0015882375512795\\
86.28	0.00158955991808825\\
86.29	0.00159088330316641\\
86.3	0.00159220771264425\\
86.31	0.0015935331527452\\
86.32	0.0015948596297874\\
86.33	0.0015961871501854\\
86.34	0.00159751572045179\\
86.35	0.00159884534719895\\
86.36	0.00160017603714073\\
86.37	0.00160150779709425\\
86.38	0.00160284063398172\\
86.39	0.0016041745548322\\
86.4	0.00160550956678355\\
86.41	0.00160684567708424\\
86.42	0.00160818289309537\\
86.43	0.00160952122229256\\
86.44	0.001610860672268\\
86.45	0.00161220125073247\\
86.46	0.00161354296551743\\
86.47	0.00161488582457713\\
86.48	0.00161622983599075\\
86.49	0.00161757500796459\\
86.5	0.00161892134883434\\
86.51	0.00162026886706732\\
86.52	0.00162161757126479\\
86.53	0.00162296747016433\\
86.54	0.00162431857264222\\
86.55	0.00162567088771593\\
86.56	0.00162702442454652\\
86.57	0.00162837919244127\\
86.58	0.00162973520085621\\
86.59	0.00163109245939875\\
86.6	0.00163245097783038\\
86.61	0.00163381076606939\\
86.62	0.00163517183419364\\
86.63	0.0016365341924434\\
86.64	0.0016378978512242\\
86.65	0.00163926282110982\\
86.66	0.00164062911284523\\
86.67	0.00164199673734967\\
86.68	0.00164336570571972\\
86.69	0.00164473602923247\\
86.7	0.00164610771934877\\
86.71	0.00164748078771645\\
86.72	0.00164885524617371\\
86.73	0.00165023110675247\\
86.74	0.00165160838168191\\
86.75	0.00165298708339191\\
86.76	0.00165436722451671\\
86.77	0.00165574881789855\\
86.78	0.0016571318765914\\
86.79	0.00165851641386478\\
86.8	0.00165990244320759\\
86.81	0.00166128997833212\\
86.82	0.001662679033178\\
86.83	0.00166406962191637\\
86.84	0.00166546175895397\\
86.85	0.00166685545893747\\
86.86	0.00166825073675775\\
86.87	0.00166964760755436\\
86.88	0.00167104608671997\\
86.89	0.00167244618990497\\
86.9	0.00167384793302217\\
86.91	0.00167525133225154\\
86.92	0.00167665640404502\\
86.93	0.00167806316513154\\
86.94	0.001679471632522\\
86.95	0.00168088182351441\\
86.96	0.00168229375569916\\
86.97	0.00168370744696431\\
86.98	0.00168512291550102\\
86.99	0.00168654017980912\\
87	0.00168795925870275\\
87.01	0.00168938017131609\\
87.02	0.00169080293710924\\
87.03	0.00169222757587418\\
87.04	0.00169365410774091\\
87.05	0.00169508255318358\\
87.06	0.0016965129330269\\
87.07	0.00169794526845254\\
87.08	0.00169937958100575\\
87.09	0.001700815892602\\
87.1	0.0017022542255339\\
87.11	0.00170369460247811\\
87.12	0.00170513704650246\\
87.13	0.00170658158107319\\
87.14	0.00170802823006235\\
87.15	0.00170947701775532\\
87.16	0.00171092796885849\\
87.17	0.00171238110850708\\
87.18	0.00171383646227312\\
87.19	0.00171529405617361\\
87.2	0.0017167539166788\\
87.21	0.00171821607072065\\
87.22	0.00171968054570147\\
87.23	0.00172114736950272\\
87.24	0.00172261657049397\\
87.25	0.00172408817754208\\
87.26	0.0017255622200205\\
87.27	0.00172703872781882\\
87.28	0.00172851773135245\\
87.29	0.00172999926157256\\
87.3	0.00173148334997613\\
87.31	0.00173297002861631\\
87.32	0.00173445933011288\\
87.33	0.00173595128766298\\
87.34	0.00173744593505208\\
87.35	0.00173894330666507\\
87.36	0.0017404434374977\\
87.37	0.00174194636316816\\
87.38	0.0017434521199289\\
87.39	0.00174496074467875\\
87.4	0.00174647227497524\\
87.41	0.00174798674904715\\
87.42	0.00174950420580735\\
87.43	0.00175102468486593\\
87.44	0.0017525482265435\\
87.45	0.00175407487188485\\
87.46	0.00175560466267288\\
87.47	0.00175713764144275\\
87.48	0.00175867385149638\\
87.49	0.00176021333691728\\
87.5	0.00176175614258554\\
87.51	0.00176330231419331\\
87.52	0.00176485189826049\\
87.53	0.00176640494215074\\
87.54	0.00176796149408787\\
87.55	0.00176952160317257\\
87.56	0.00177108531939942\\
87.57	0.00177265269367432\\
87.58	0.00177422377783227\\
87.59	0.0017757986246555\\
87.6	0.00177737728789199\\
87.61	0.00177895982227438\\
87.62	0.00178054628353926\\
87.63	0.00178213672844689\\
87.64	0.00178373121480132\\
87.65	0.00178532980147092\\
87.66	0.00178693254840939\\
87.67	0.00178853951667714\\
87.68	0.00179015076846316\\
87.69	0.00179176636710741\\
87.7	0.00179338637712356\\
87.71	0.00179501086422232\\
87.72	0.00179663989533522\\
87.73	0.00179827353863892\\
87.74	0.00179991186357999\\
87.75	0.00180155494090029\\
87.76	0.00180320284266281\\
87.77	0.00180485564227817\\
87.78	0.00180651341453159\\
87.79	0.00180817623561049\\
87.8	0.00180984418313267\\
87.81	0.00181151733617512\\
87.82	0.00181319577530343\\
87.83	0.00181487958260183\\
87.84	0.00181656884170392\\
87.85	0.00181826363782401\\
87.86	0.00181996405778924\\
87.87	0.00182167019007225\\
87.88	0.00182338212482473\\
87.89	0.00182509995391157\\
87.9	0.00182682377094585\\
87.91	0.00182855367132455\\
87.92	0.00183028975226504\\
87.93	0.00183203211284243\\
87.94	0.00183378085402768\\
87.95	0.00183553607872658\\
87.96	0.0018372978918196\\
87.97	0.00183906640020263\\
87.98	0.00184084171282858\\
87.99	0.00184262394074997\\
88	0.00184441319716245\\
88.01	0.00184620959744923\\
88.02	0.00184801325922662\\
88.03	0.00184982430239047\\
88.04	0.00185164284916378\\
88.05	0.00185346902414524\\
88.06	0.00185530295435899\\
88.07	0.00185714476930545\\
88.08	0.00185899460101327\\
88.09	0.00186085258409253\\
88.1	0.00186271885578906\\
88.11	0.0018645935560401\\
88.12	0.00186647682753113\\
88.13	0.00186836881575407\\
88.14	0.00187026966906676\\
88.15	0.00187217953875385\\
88.16	0.00187409857908901\\
88.17	0.00187602694739871\\
88.18	0.00187796480412729\\
88.19	0.00187991231290369\\
88.2	0.00188186964060964\\
88.21	0.00188383695744944\\
88.22	0.00188581443702137\\
88.23	0.00188780225639077\\
88.24	0.00188980059616478\\
88.25	0.00189180964056891\\
88.26	0.00189382957752529\\
88.27	0.00189586059873282\\
88.28	0.00189790289974923\\
88.29	0.00189995668007498\\
88.3	0.00190202214323918\\
88.31	0.0019040994968876\\
88.32	0.00190618895287265\\
88.33	0.00190829072734554\\
88.34	0.0019104050408507\\
88.35	0.00191253211842227\\
88.36	0.00191467218968309\\
88.37	0.00191682548894589\\
88.38	0.00191899225531702\\
88.39	0.00192117273280255\\
88.4	0.00192336717041704\\
88.41	0.0019255758222948\\
88.42	0.00192779894780395\\
88.43	0.00193003681166313\\
88.44	0.00193228968406108\\
88.45	0.00193455784077916\\
88.46	0.00193684156331674\\
88.47	0.00193914113901974\\
88.48	0.0019414568612122\\
88.49	0.00194378902933121\\
88.5	0.00194613794906498\\
88.51	0.00194850393249437\\
88.52	0.00195088729823793\\
88.53	0.00195328837160043\\
88.54	0.0019557074847251\\
88.55	0.00195814497674962\\
88.56	0.00196060119396597\\
88.57	0.00196307648998422\\
88.58	0.00196557122590046\\
88.59	0.00196808577046878\\
88.6	0.0019706205002777\\
88.61	0.00197316460559005\\
88.62	0.00197570969336572\\
88.63	0.00197825576441454\\
88.64	0.00198080281954738\\
88.65	0.00198335085957615\\
88.66	0.00198589988531384\\
88.67	0.00198844989757451\\
88.68	0.00199100089717326\\
88.69	0.00199355288492626\\
88.7	0.00199610586165076\\
88.71	0.00199865982816507\\
88.72	0.00200121478528856\\
88.73	0.0020037707338417\\
88.74	0.00200632767464599\\
88.75	0.00200888560852406\\
88.76	0.00201144453629957\\
88.77	0.00201400445879728\\
88.78	0.00201656537684304\\
88.79	0.00201912729126378\\
88.8	0.00202169020288749\\
88.81	0.00202425411254327\\
88.82	0.00202681902106132\\
88.83	0.0020293849292729\\
88.84	0.00203195183801038\\
88.85	0.00203451974810723\\
88.86	0.00203708866039801\\
88.87	0.00203965857571837\\
88.88	0.00204222949490507\\
88.89	0.00204480141879598\\
88.9	0.00204737434823005\\
88.91	0.00204994828404736\\
88.92	0.00205252322708909\\
88.93	0.00205509917819752\\
88.94	0.00205767613821605\\
88.95	0.0020602541079892\\
88.96	0.0020628330883626\\
88.97	0.00206541308018298\\
88.98	0.00206799408429823\\
88.99	0.00207057610155732\\
89	0.00207315913281036\\
89.01	0.0020757431789086\\
89.02	0.00207832824070439\\
89.03	0.00208091431905124\\
89.04	0.00208350141480375\\
89.05	0.00208608952881769\\
89.06	0.00208867866194995\\
89.07	0.00209126881505857\\
89.08	0.0020938599890027\\
89.09	0.00209645218464266\\
89.1	0.0020990454028399\\
89.11	0.00210163964445701\\
89.12	0.00210423491035774\\
89.13	0.00210683120140699\\
89.14	0.0021094285184708\\
89.15	0.00211202686241636\\
89.16	0.00211462623411203\\
89.17	0.00211722663442732\\
89.18	0.00211982806423289\\
89.19	0.00212243052440058\\
89.2	0.00212503401580337\\
89.21	0.00212763853931543\\
89.22	0.00213024409581207\\
89.23	0.0021328506861698\\
89.24	0.00213545831126627\\
89.25	0.00213806697198033\\
89.26	0.00214067666919199\\
89.27	0.00214328740378243\\
89.28	0.00214589917663404\\
89.29	0.00214851198863036\\
89.3	0.00215112584065612\\
89.31	0.00215374073359724\\
89.32	0.00215635666834084\\
89.33	0.00215897364577522\\
89.34	0.00216159166678985\\
89.35	0.00216421073227542\\
89.36	0.00216683084312382\\
89.37	0.00216945200022812\\
89.38	0.0021720742044826\\
89.39	0.00217469745678273\\
89.4	0.0021773217580252\\
89.41	0.00217994710910791\\
89.42	0.00218257351092994\\
89.43	0.0021852009643916\\
89.44	0.00218782947039443\\
89.45	0.00219045902984114\\
89.46	0.00219308964363571\\
89.47	0.00219572131268329\\
89.48	0.00219835403789028\\
89.49	0.00220098782016431\\
89.5	0.0022036226604142\\
89.51	0.00220625855955003\\
89.52	0.0022088955184831\\
89.53	0.00221153353812593\\
89.54	0.0022141726193923\\
89.55	0.0022168127631972\\
89.56	0.00221945397045687\\
89.57	0.00222209624208879\\
89.58	0.00222473957901169\\
89.59	0.00222738398214551\\
89.6	0.00223002945241149\\
89.61	0.00223267599073208\\
89.62	0.002235323598031\\
89.63	0.0022379722752332\\
89.64	0.00224062202326492\\
89.65	0.00224327284305362\\
89.66	0.00224592473552806\\
89.67	0.00224857770161823\\
89.68	0.00225123174225539\\
89.69	0.00225388685837208\\
89.7	0.00225654305090209\\
89.71	0.00225920032078051\\
89.72	0.00226185866894366\\
89.73	0.00226451809632917\\
89.74	0.00226717860387594\\
89.75	0.00226984019252414\\
89.76	0.00227250286321523\\
89.77	0.00227516661689195\\
89.78	0.00227783145449833\\
89.79	0.00228049737697967\\
89.8	0.0022831643852826\\
89.81	0.002285832480355\\
89.82	0.00228850166314608\\
89.83	0.00229117193460631\\
89.84	0.0022938432956875\\
89.85	0.00229651574734274\\
89.86	0.00229918929052641\\
89.87	0.00230186392619423\\
89.88	0.0023045396553032\\
89.89	0.00230721647881165\\
89.9	0.00230989439767921\\
89.91	0.00231257341286682\\
89.92	0.00231525352533676\\
89.93	0.0023179347360526\\
89.94	0.00232061704597927\\
89.95	0.00232330045608298\\
89.96	0.0023259849673313\\
89.97	0.00232867058069311\\
89.98	0.00233135729713863\\
89.99	0.00233404511763941\\
90	0.00233673404316833\\
90.01	0.00233942407469962\\
90.02	0.00234211521320885\\
90.03	0.0023448074596729\\
90.04	0.00234750081507004\\
90.05	0.00235019528037986\\
90.06	0.0023528908565833\\
90.07	0.00235558754466265\\
90.08	0.00235828534560157\\
90.09	0.00236098426038507\\
90.1	0.00236368428999949\\
90.11	0.00236638543543256\\
90.12	0.00236908769767337\\
90.13	0.00237179107771237\\
90.14	0.00237449557654136\\
90.15	0.00237720119515354\\
90.16	0.00237990793454346\\
90.17	0.00238261579570705\\
90.18	0.00238532477964162\\
90.19	0.00238803488734584\\
90.2	0.00239074611981979\\
90.21	0.0023934584780649\\
90.22	0.00239617196308401\\
90.23	0.00239888657588134\\
90.24	0.0024016023174625\\
90.25	0.00240431918883448\\
90.26	0.00240703719100568\\
90.27	0.00240975632498589\\
90.28	0.00241247659178629\\
90.29	0.00241519799241948\\
90.3	0.00241792052789944\\
90.31	0.00242064419924157\\
90.32	0.00242336900746268\\
90.33	0.00242609495358098\\
90.34	0.00242882203861609\\
90.35	0.00243155026358906\\
90.36	0.00243427962952233\\
90.37	0.00243701013743979\\
90.38	0.00243974178836673\\
90.39	0.00244247458332986\\
90.4	0.00244520852335733\\
90.41	0.00244794360947871\\
90.42	0.002450679842725\\
90.43	0.00245341722412863\\
90.44	0.00245615575472347\\
90.45	0.00245889543554482\\
90.46	0.00246163626762942\\
90.47	0.00246437825201545\\
90.48	0.00246712138974253\\
90.49	0.00246986568185174\\
90.5	0.00247261112938559\\
90.51	0.00247535773338804\\
90.52	0.00247810549490451\\
90.53	0.00248085441498188\\
90.54	0.00248360449466846\\
90.55	0.00248635573501404\\
90.56	0.00248910813706987\\
90.57	0.00249186170188865\\
90.58	0.00249461643052456\\
90.59	0.00249737232403322\\
90.6	0.00250012938347175\\
90.61	0.00250288760989871\\
90.62	0.00250564700437417\\
90.63	0.00250840756795964\\
90.64	0.00251116930171812\\
90.65	0.00251393220671409\\
90.66	0.00251669628401351\\
90.67	0.00251946153468383\\
90.68	0.00252222795979396\\
90.69	0.00252499556041434\\
90.7	0.00252776433761686\\
90.71	0.00253053429247491\\
90.72	0.00253330542606339\\
90.73	0.00253607773945868\\
90.74	0.00253885123373865\\
90.75	0.00254162590998269\\
90.76	0.00254440176927168\\
90.77	0.002547178812688\\
90.78	0.00254995704131553\\
90.79	0.00255273645623968\\
90.8	0.00255551705854734\\
90.81	0.00255829884932693\\
90.82	0.00256108182966837\\
90.83	0.0025638660006631\\
90.84	0.00256665136340408\\
90.85	0.00256943791898577\\
90.86	0.00257222566850418\\
90.87	0.00257501461305681\\
90.88	0.0025778047537427\\
90.89	0.00258059609166242\\
90.9	0.00258338862791804\\
90.91	0.00258618236361318\\
90.92	0.002588977299853\\
90.93	0.00259177343774416\\
90.94	0.00259457077839488\\
90.95	0.0025973693229149\\
90.96	0.0026001690724155\\
90.97	0.00260297002800949\\
90.98	0.00260577219081125\\
90.99	0.00260857556193666\\
91	0.00261138014250316\\
91.01	0.00261418593362975\\
91.02	0.00261699293643695\\
91.03	0.00261980115204684\\
91.04	0.00262261058158304\\
91.05	0.00262542122617072\\
91.06	0.00262823308693662\\
91.07	0.00263104616500901\\
91.08	0.00263386046151771\\
91.09	0.00263667597759412\\
91.1	0.00263949271437117\\
91.11	0.00264231067298337\\
91.12	0.00264512985456675\\
91.13	0.00264795026025895\\
91.14	0.00265077189119914\\
91.15	0.00265359474852805\\
91.16	0.00265641883338797\\
91.17	0.00265924414692278\\
91.18	0.00266207069027789\\
91.19	0.0026648984646003\\
91.2	0.00266772747103857\\
91.21	0.00267055771074281\\
91.22	0.00267338918486471\\
91.23	0.00267622189455754\\
91.24	0.00267905584097612\\
91.25	0.00268189102527685\\
91.26	0.0026847274486177\\
91.27	0.0026875651121582\\
91.28	0.00269040401705946\\
91.29	0.00269324416448415\\
91.3	0.00269608555559654\\
91.31	0.00269892819156244\\
91.32	0.00270177207354925\\
91.33	0.00270461720272594\\
91.34	0.00270746358026304\\
91.35	0.00271031120733268\\
91.36	0.00271316008510853\\
91.37	0.00271601021476586\\
91.38	0.0027188615974815\\
91.39	0.00272171423443386\\
91.4	0.00272456812680291\\
91.41	0.0027274232757702\\
91.42	0.00273027968251886\\
91.43	0.00273313734823356\\
91.44	0.00273599627410059\\
91.45	0.00273885646130778\\
91.46	0.00274171791104452\\
91.47	0.00274458062450179\\
91.48	0.00274744460287213\\
91.49	0.00275030984734965\\
91.5	0.00275317635913003\\
91.51	0.00275604413941051\\
91.52	0.00275891318938988\\
91.53	0.00276178351026853\\
91.54	0.00276465510324838\\
91.55	0.00276752796953292\\
91.56	0.0027704021103272\\
91.57	0.00277327752683783\\
91.58	0.00277615422027297\\
91.59	0.00277903219184234\\
91.6	0.00278191144275722\\
91.61	0.00278479197423041\\
91.62	0.00278767378747629\\
91.63	0.00279055688371078\\
91.64	0.00279344126415133\\
91.65	0.00279632693001694\\
91.66	0.00279921388252816\\
91.67	0.00280210212290706\\
91.68	0.00280499165237724\\
91.69	0.00280788247216386\\
91.7	0.00281077458349357\\
91.71	0.00281366798759457\\
91.72	0.00281656268569658\\
91.73	0.00281945867903083\\
91.74	0.00282235596883006\\
91.75	0.00282525455632854\\
91.76	0.00282815444276202\\
91.77	0.00283105562936779\\
91.78	0.0028339581173846\\
91.79	0.00283686190805273\\
91.8	0.00283976700261392\\
91.81	0.00284267340231141\\
91.82	0.00284558110838993\\
91.83	0.00284849012209567\\
91.84	0.00285140044467631\\
91.85	0.00285431207738098\\
91.86	0.00285722502146028\\
91.87	0.00286013927816625\\
91.88	0.00286305484875241\\
91.89	0.00286597173447369\\
91.9	0.00286888993658648\\
91.91	0.0028718094563486\\
91.92	0.00287473029501928\\
91.93	0.00287765245385918\\
91.94	0.00288057593413037\\
91.95	0.00288350073709632\\
91.96	0.0028864268640219\\
91.97	0.00288935431617337\\
91.98	0.00289228309481837\\
91.99	0.00289521320122591\\
92	0.00289814463666638\\
92.01	0.00290107740241149\\
92.02	0.00290401149973435\\
92.03	0.00290694692990936\\
92.04	0.00290988369421229\\
92.05	0.00291282179392019\\
92.06	0.00291576123031145\\
92.07	0.00291870200466576\\
92.08	0.00292164411826409\\
92.09	0.00292458757238868\\
92.1	0.00292753236832305\\
92.11	0.00293047850735199\\
92.12	0.00293342599076152\\
92.13	0.00293637481983889\\
92.14	0.00293932499587258\\
92.15	0.00294227652015229\\
92.16	0.00294522939396889\\
92.17	0.00294818361861446\\
92.18	0.00295113919538224\\
92.19	0.00295409612556662\\
92.2	0.00295705441046314\\
92.21	0.00296001405136848\\
92.22	0.00296297504958041\\
92.23	0.00296593740639782\\
92.24	0.00296890112312066\\
92.25	0.00297186620104996\\
92.26	0.00297483264148781\\
92.27	0.00297780044573732\\
92.28	0.00298076961510261\\
92.29	0.00298374015088883\\
92.3	0.00298671205440207\\
92.31	0.00298968532694941\\
92.32	0.00299265996983887\\
92.33	0.00299563598437938\\
92.34	0.0029986133718808\\
92.35	0.00300159213365386\\
92.36	0.00300457227101014\\
92.37	0.0030075537852621\\
92.38	0.00301053667772297\\
92.39	0.00301352094970683\\
92.4	0.0030165066025285\\
92.41	0.00301949363750357\\
92.42	0.00302248205594836\\
92.43	0.00302547185917989\\
92.44	0.00302846304851585\\
92.45	0.00303145562527461\\
92.46	0.00303444959077515\\
92.47	0.00303744494633704\\
92.48	0.00304044169328047\\
92.49	0.00304343983292613\\
92.5	0.00304643936659525\\
92.51	0.00304944029560956\\
92.52	0.00305244262129123\\
92.53	0.00305544634496288\\
92.54	0.0030584514679475\\
92.55	0.0030614579915685\\
92.56	0.00306446591714957\\
92.57	0.00306747524601473\\
92.58	0.00307048597948828\\
92.59	0.00307349811889473\\
92.6	0.00307651166555882\\
92.61	0.00307952662080542\\
92.62	0.00308254298595956\\
92.63	0.00308556076234634\\
92.64	0.00308857995129092\\
92.65	0.00309160055411848\\
92.66	0.00309462257215418\\
92.67	0.00309764600672309\\
92.68	0.00310067085915019\\
92.69	0.00310369713076032\\
92.7	0.00310672482287812\\
92.71	0.00310975393682799\\
92.72	0.00311278447393406\\
92.73	0.00311581643552013\\
92.74	0.00311884982290963\\
92.75	0.00312188463742558\\
92.76	0.00312492088039053\\
92.77	0.00312795855312651\\
92.78	0.003130997656955\\
92.79	0.00313403819319685\\
92.8	0.00313708016317225\\
92.81	0.00314012356820068\\
92.82	0.00314316840960084\\
92.83	0.00314621468869059\\
92.84	0.00314926240678694\\
92.85	0.00315231156520593\\
92.86	0.00315536216526262\\
92.87	0.00315841420827102\\
92.88	0.00316146769554402\\
92.89	0.00316452262839333\\
92.9	0.00316757900812943\\
92.91	0.00317063683606151\\
92.92	0.00317369611349739\\
92.93	0.00317675684174347\\
92.94	0.00317981902210464\\
92.95	0.00318288265588426\\
92.96	0.00318594774438405\\
92.97	0.00318901428890403\\
92.98	0.00319208229074247\\
92.99	0.00319515175119581\\
93	0.00319822267155855\\
93.01	0.00320129505312325\\
93.02	0.0032043688971804\\
93.03	0.00320744420501835\\
93.04	0.00321052097792328\\
93.05	0.00321359921717904\\
93.06	0.00321667892406717\\
93.07	0.00321976009986675\\
93.08	0.00322284274585433\\
93.09	0.00322592686330389\\
93.1	0.0032290124534867\\
93.11	0.00323209951767128\\
93.12	0.00323518805712331\\
93.13	0.00323827807310552\\
93.14	0.00324136956687763\\
93.15	0.00324446253969627\\
93.16	0.00324755699281485\\
93.17	0.00325065292748352\\
93.18	0.00325375034494905\\
93.19	0.00325684924645476\\
93.2	0.00325994963324042\\
93.21	0.00326305150654214\\
93.22	0.00326615486759233\\
93.23	0.00326925971761955\\
93.24	0.00327236605784845\\
93.25	0.00327547388949967\\
93.26	0.00327858321378973\\
93.27	0.00328169403193091\\
93.28	0.00328480634513113\\
93.29	0.00328792015459385\\
93.3	0.0032910354615179\\
93.31	0.00329415226709742\\
93.32	0.0032972705725217\\
93.33	0.00330039037897502\\
93.34	0.00330351168763658\\
93.35	0.00330663449968035\\
93.36	0.00330975881627488\\
93.37	0.00331288463858323\\
93.38	0.00331601196776281\\
93.39	0.00331914080496522\\
93.4	0.00332227115133611\\
93.41	0.00332540300801503\\
93.42	0.0033285363761353\\
93.43	0.00333167125682385\\
93.44	0.00333480765120101\\
93.45	0.00333794556038044\\
93.46	0.00334108498546891\\
93.47	0.00334422592756614\\
93.48	0.00334736838776464\\
93.49	0.00335051236714955\\
93.5	0.00335365786679845\\
93.51	0.0033568048877812\\
93.52	0.00335995343115974\\
93.53	0.00336310349798791\\
93.54	0.0033662550893113\\
93.55	0.003369408206167\\
93.56	0.00337256284958345\\
93.57	0.00337571902058026\\
93.58	0.00337887672016794\\
93.59	0.00338203594934779\\
93.6	0.00338519670911161\\
93.61	0.00338835900044156\\
93.62	0.00339152282430988\\
93.63	0.00339468818167875\\
93.64	0.00339785507349999\\
93.65	0.0034010235007149\\
93.66	0.003404193464254\\
93.67	0.00340736496503682\\
93.68	0.00341053800397162\\
93.69	0.00341371258195521\\
93.7	0.00341688869987268\\
93.71	0.00342006635859713\\
93.72	0.00342324555898948\\
93.73	0.00342642630189813\\
93.74	0.00342960858815878\\
93.75	0.00343279241859412\\
93.76	0.0034359777940136\\
93.77	0.0034391647152131\\
93.78	0.00344235318297471\\
93.79	0.00344554319806643\\
93.8	0.00344873476124186\\
93.81	0.00345192787323996\\
93.82	0.00345512253478471\\
93.83	0.00345831874658483\\
93.84	0.00346151650933348\\
93.85	0.00346471582370795\\
93.86	0.00346791669036934\\
93.87	0.00347111910996224\\
93.88	0.00347432308311441\\
93.89	0.00347752861043647\\
93.9	0.00348073569252154\\
93.91	0.00348394432994491\\
93.92	0.0034871545232637\\
93.93	0.00349036627301653\\
93.94	0.00349357957972313\\
93.95	0.00349679444388399\\
93.96	0.00350001086598002\\
93.97	0.00350322884647217\\
93.98	0.00350644838580103\\
93.99	0.00350966948438649\\
94	0.00351289214262733\\
94.01	0.00351611636090081\\
94.02	0.00351934213956231\\
94.03	0.00352256947894493\\
94.04	0.00352579837935904\\
94.05	0.00352902884109189\\
94.06	0.00353226086440719\\
94.07	0.0035354944495447\\
94.08	0.00353872959671976\\
94.09	0.00354196630612289\\
94.1	0.00354520457791931\\
94.11	0.00354844441224855\\
94.12	0.00355168580922391\\
94.13	0.00355492876893207\\
94.14	0.0035581732914326\\
94.15	0.00356141937675747\\
94.16	0.0035646670249106\\
94.17	0.00356791623586733\\
94.18	0.003571167009574\\
94.19	0.00357441934594738\\
94.2	0.00357767324487419\\
94.21	0.00358092870621063\\
94.22	0.00358418572978179\\
94.23	0.00358744431538119\\
94.24	0.00359070446277023\\
94.25	0.00359396617167766\\
94.26	0.00359722944179901\\
94.27	0.0036004942727961\\
94.28	0.00360376066429643\\
94.29	0.00360702861589267\\
94.3	0.00361029812714207\\
94.31	0.00361356919756589\\
94.32	0.00361684182664881\\
94.33	0.00362011601383841\\
94.34	0.0036233917585445\\
94.35	0.0036266690601386\\
94.36	0.00362994791795327\\
94.37	0.00363322833128158\\
94.38	0.00363651029937644\\
94.39	0.00363979382145003\\
94.4	0.00364307889667315\\
94.41	0.00364636552417461\\
94.42	0.0036496537030406\\
94.43	0.00365294343231403\\
94.44	0.00365623471099394\\
94.45	0.00365952753803482\\
94.46	0.00366282191234598\\
94.47	0.00366611783279086\\
94.48	0.00366941529818645\\
94.49	0.00367271430730255\\
94.5	0.00367601485886115\\
94.51	0.00367931695153576\\
94.52	0.00368262058395075\\
94.53	0.00368592575468064\\
94.54	0.00368923246224947\\
94.55	0.00369254070513008\\
94.56	0.0036958504817435\\
94.57	0.00369916179045819\\
94.58	0.00370247462958941\\
94.59	0.00370578899739853\\
94.6	0.00370910489209232\\
94.61	0.00371242231182233\\
94.62	0.00371574125468414\\
94.63	0.00371906171871672\\
94.64	0.00372238370190175\\
94.65	0.00372570720216291\\
94.66	0.00372903221736526\\
94.67	0.00373235874531453\\
94.68	0.00373568678375645\\
94.69	0.0037390163303761\\
94.7	0.00374234738279727\\
94.71	0.00374567993858176\\
94.72	0.00374901399522877\\
94.73	0.00375234955017426\\
94.74	0.00375568660079029\\
94.75	0.00375902514438443\\
94.76	0.00376236517819911\\
94.77	0.00376570669941106\\
94.78	0.00376904970513068\\
94.79	0.00377239419240149\\
94.8	0.00377574015819951\\
94.81	0.0037790875994328\\
94.82	0.00378243651294081\\
94.83	0.00378578689549397\\
94.84	0.00378913874379309\\
94.85	0.00379249205446895\\
94.86	0.00379584682408179\\
94.87	0.0037992030491209\\
94.88	0.00380256072600416\\
94.89	0.0038059198510777\\
94.9	0.00380928042061549\\
94.91	0.00381264243081901\\
94.92	0.00381600587781696\\
94.93	0.00381937075766495\\
94.94	0.00382273706634526\\
94.95	0.00382610479976667\\
94.96	0.00382947395376421\\
94.97	0.00383284452409909\\
94.98	0.00383621650645858\\
94.99	0.00383958989645595\\
95	0.00384296468963049\\
95.01	0.00384634088144753\\
95.02	0.00384971846729854\\
95.03	0.00385309744250129\\
95.04	0.00385647780230004\\
95.05	0.00385985954186578\\
95.06	0.00386324265629658\\
95.07	0.00386662714061794\\
95.08	0.00387001298978315\\
95.09	0.00387340019867388\\
95.1	0.00387678876210063\\
95.11	0.00388017867480346\\
95.12	0.00388356993145266\\
95.13	0.00388696252664955\\
95.14	0.00389035645492738\\
95.15	0.00389375171075228\\
95.16	0.00389714828852437\\
95.17	0.00390054618257889\\
95.18	0.00390394538718748\\
95.19	0.0039073458965596\\
95.2	0.00391074770484395\\
95.21	0.00391415080613014\\
95.22	0.00391755519445041\\
95.23	0.00392096086378148\\
95.24	0.00392436780804654\\
95.25	0.00392777602111741\\
95.26	0.00393118549681681\\
95.27	0.00393459622892078\\
95.28	0.0039380082111613\\
95.29	0.00394142143722901\\
95.3	0.00394483590077615\\
95.31	0.00394825159541971\\
95.32	0.00395166851474465\\
95.33	0.00395508665230744\\
95.34	0.00395850600163975\\
95.35	0.00396192655625233\\
95.36	0.00396534830963912\\
95.37	0.00396877125528166\\
95.38	0.00397219538665361\\
95.39	0.0039756206972256\\
95.4	0.00397904718047036\\
95.41	0.00398247482986804\\
95.42	0.00398590363891184\\
95.43	0.00398933360111399\\
95.44	0.0039927647100119\\
95.45	0.00399619695917478\\
95.46	0.00399963034221044\\
95.47	0.00400306485277252\\
95.48	0.00400650048456806\\
95.49	0.00400993723136539\\
95.5	0.00401337508700244\\
95.51	0.00401681404539544\\
95.52	0.00402025410054799\\
95.53	0.00402369524656064\\
95.54	0.0040271374776408\\
95.55	0.0040305807881132\\
95.56	0.00403402517243078\\
95.57	0.0040374706251861\\
95.58	0.00404091714112321\\
95.59	0.00404436471515014\\
95.6	0.00404781334235185\\
95.61	0.00405126301800383\\
95.62	0.00405471373758624\\
95.63	0.0040581654967987\\
95.64	0.00406161829157572\\
95.65	0.00406507211810276\\
95.66	0.00406852697283303\\
95.67	0.00407198285250498\\
95.68	0.00407543975416053\\
95.69	0.00407889767516405\\
95.7	0.00408235661322223\\
95.71	0.00408581656640464\\
95.72	0.00408927753316527\\
95.73	0.0040927395123649\\
95.74	0.00409620250329439\\
95.75	0.00409966650569898\\
95.76	0.00410313151980353\\
95.77	0.00410659754633881\\
95.78	0.00411006458656885\\
95.79	0.00411353264231943\\
95.8	0.00411700171600767\\
95.81	0.00412047181067283\\
95.82	0.00412394293000837\\
95.83	0.00412741507839523\\
95.84	0.00413088826093649\\
95.85	0.00413436248349338\\
95.86	0.00413783775272269\\
95.87	0.00414131407611571\\
95.88	0.00414479146203864\\
95.89	0.00414826991977459\\
95.9	0.00415174945956728\\
95.91	0.00415523009266638\\
95.92	0.0041587118313746\\
95.93	0.00416219468909666\\
95.94	0.00416567868039011\\
95.95	0.0041691638210181\\
95.96	0.00417265012800422\\
95.97	0.00417613761968942\\
95.98	0.00417962631579107\\
95.99	0.00418311623746432\\
96	0.00418660740736577\\
96.01	0.00419009984971961\\
96.02	0.00419359359038617\\
96.03	0.00419708865693313\\
96.04	0.00420058507870941\\
96.05	0.00420408288692186\\
96.06	0.00420758211471476\\
96.07	0.00421108279725236\\
96.08	0.00421458497180454\\
96.09	0.0042180886778356\\
96.1	0.0042215939570964\\
96.11	0.00422510085371993\\
96.12	0.00422860941432044\\
96.13	0.00423211968809624\\
96.14	0.00423563172693633\\
96.15	0.00423914558553093\\
96.16	0.00424266132148616\\
96.17	0.0042461789954429\\
96.18	0.00424969867120005\\
96.19	0.00425322041584236\\
96.2	0.00425674429987289\\
96.21	0.00426027039735038\\
96.22	0.00426379878603165\\
96.23	0.00426732954751923\\
96.24	0.00427086276741434\\
96.25	0.00427439853547551\\
96.26	0.00427793694578297\\
96.27	0.00428147809690903\\
96.28	0.0042850220920947\\
96.29	0.00428856898279929\\
96.3	0.00429211878638083\\
96.31	0.00429567152050528\\
96.32	0.00429922720315299\\
96.33	0.00430278585262515\\
96.34	0.00430634748755051\\
96.35	0.00430991212689213\\
96.36	0.00431347978995437\\
96.37	0.00431705049638999\\
96.38	0.00432062426620734\\
96.39	0.00432420111977782\\
96.4	0.00432778107784344\\
96.41	0.00433136416152448\\
96.42	0.00433495039232746\\
96.43	0.00433853979215311\\
96.44	0.00434213238330466\\
96.45	0.00434572818849622\\
96.46	0.00434932723086133\\
96.47	0.00435292953396178\\
96.48	0.00435653512179651\\
96.49	0.00436014401881079\\
96.5	0.00436375624990552\\
96.51	0.0043673718404468\\
96.52	0.00437099081627567\\
96.53	0.00437461320371802\\
96.54	0.00437823902959481\\
96.55	0.00438186832123239\\
96.56	0.00438550110647316\\
96.57	0.00438913741368635\\
96.58	0.0043927772717791\\
96.59	0.00439642071020778\\
96.6	0.0044000677589895\\
96.61	0.0044037184487139\\
96.62	0.00440737281055519\\
96.63	0.00441103087628448\\
96.64	0.00441469267828231\\
96.65	0.00441835824955147\\
96.66	0.00442202762373015\\
96.67	0.00442570083510529\\
96.68	0.00442937791862629\\
96.69	0.00443305890991894\\
96.7	0.00443674384529974\\
96.71	0.00444043276179043\\
96.72	0.00444412569713291\\
96.73	0.00444782268980444\\
96.74	0.00445152377903318\\
96.75	0.00445522900481407\\
96.76	0.00445893840792503\\
96.77	0.00446265202994354\\
96.78	0.00446636991326354\\
96.79	0.00447009210111275\\
96.8	0.00447381863757028\\
96.81	0.00447754956758471\\
96.82	0.0044812849369925\\
96.83	0.00448502479253681\\
96.84	0.00448876918188675\\
96.85	0.00449251815365699\\
96.86	0.00449627175742788\\
96.87	0.00450003004376591\\
96.88	0.00450379306424469\\
96.89	0.00450756087146632\\
96.9	0.00451133351908329\\
96.91	0.00451511106182076\\
96.92	0.00451889355549944\\
96.93	0.00452268105705886\\
96.94	0.0045264736245812\\
96.95	0.00453027131731562\\
96.96	0.00453407419570313\\
96.97	0.00453788232140196\\
96.98	0.00454169575731355\\
96.99	0.00454551456760903\\
97	0.0045493388177563\\
97.01	0.00455316857454773\\
97.02	0.00455700390612843\\
97.03	0.00456084488202511\\
97.04	0.00456469157317562\\
97.05	0.0045685440519591\\
97.06	0.0045724023922268\\
97.07	0.00457626666933354\\
97.08	0.00458013696016989\\
97.09	0.00458401334319503\\
97.1	0.00458789589847034\\
97.11	0.0045917847076937\\
97.12	0.00459567985423455\\
97.13	0.00459958142316972\\
97.14	0.00460348950132003\\
97.15	0.00460740417728766\\
97.16	0.00461132554149443\\
97.17	0.00461525368622079\\
97.18	0.00461918870564574\\
97.19	0.00462313069588762\\
97.2	0.00462707975504575\\
97.21	0.00463103598324301\\
97.22	0.00463499948266935\\
97.23	0.00463897035762623\\
97.24	0.00464294871457207\\
97.25	0.00464693466216862\\
97.26	0.00465092831132848\\
97.27	0.00465492977526348\\
97.28	0.0046589391695343\\
97.29	0.00466295661210103\\
97.3	0.00466698222337493\\
97.31	0.0046710161262713\\
97.32	0.0046750584462635\\
97.33	0.00467910931143814\\
97.34	0.00468316885255156\\
97.35	0.00468723720308742\\
97.36	0.00469131449931571\\
97.37	0.0046954008803529\\
97.38	0.00469949648822355\\
97.39	0.00470360146792318\\
97.4	0.0047077159674826\\
97.41	0.00471184013803356\\
97.42	0.00471597413387596\\
97.43	0.00472011811254644\\
97.44	0.00472427223488858\\
97.45	0.00472843666512455\\
97.46	0.00473261157092841\\
97.47	0.004736797123501\\
97.48	0.00474099349764651\\
97.49	0.00474520087185073\\
97.5	0.00474941942836098\\
97.51	0.00475364935326792\\
97.52	0.00475789083658906\\
97.53	0.0047621440723542\\
97.54	0.00476640925869272\\
97.55	0.0047706865979228\\
97.56	0.00477497629664268\\
97.57	0.00477927856582385\\
97.58	0.00478359362090641\\
97.59	0.00478792168189649\\
97.6	0.00479226297346583\\
97.61	0.00479661772505364\\
97.62	0.00480098617097063\\
97.63	0.00480536855050547\\
97.64	0.0048097651080335\\
97.65	0.00481417609312799\\
97.66	0.00481860176067379\\
97.67	0.00482304237098356\\
97.68	0.0048274981899166\\
97.69	0.0048319694890003\\
97.7	0.00483645654555951\\
97.71	0.00484095964284813\\
97.72	0.0048454790701796\\
97.73	0.00485001512306028\\
97.74	0.00485456810332584\\
97.75	0.00485913831928074\\
97.76	0.00486372608584086\\
97.77	0.00486833172467939\\
97.78	0.00487295556437595\\
97.79	0.00487759794056912\\
97.8	0.00488225919611246\\
97.81	0.00488693968123404\\
97.82	0.00489163975369956\\
97.83	0.00489635977897926\\
97.84	0.00490110013041851\\
97.85	0.00490586118941243\\
97.86	0.00491064334558433\\
97.87	0.00491544699696836\\
97.88	0.0049202725501962\\
97.89	0.00492512042068815\\
97.9	0.00492999103284847\\
97.91	0.00493488482026528\\
97.92	0.00493980222591496\\
97.93	0.00494474370237131\\
97.94	0.00494970971201947\\
97.95	0.00495470072727479\\
97.96	0.00495971723080667\\
97.97	0.00496475971576765\\
97.98	0.00496982868602769\\
97.99	0.00497492465641393\\
98	0.0049800481529559\\
98.01	0.00498519971313656\\
98.02	0.00499037988614894\\
98.03	0.00499558923315892\\
98.04	0.005000828327574\\
98.05	0.00500609775531837\\
98.06	0.00501139811511431\\
98.07	0.00501673001877028\\
98.08	0.00502209409147554\\
98.09	0.00502749097210186\\
98.1	0.00503292131351209\\
98.11	0.0050383857828761\\
98.12	0.00504388506199408\\
98.13	0.00504941984762739\\
98.14	0.00505499085183724\\
98.15	0.00506059880233135\\
98.16	0.00506624444281872\\
98.17	0.00507192853337287\\
98.18	0.00507765185080359\\
98.19	0.00508341518903751\\
98.2	0.00508921935950777\\
98.21	0.00509506519155282\\
98.22	0.00510095353282481\\
98.23	0.0051068852497077\\
98.24	0.00511286122774527\\
98.25	0.0051188823720795\\
98.26	0.00512494960789931\\
98.27	0.00513106388090016\\
98.28	0.00513722615775466\\
98.29	0.00514343742659451\\
98.3	0.00514969869750407\\
98.31	0.00515601100302584\\
98.32	0.00516237539867824\\
98.33	0.00516879296348582\\
98.34	0.00517526480052249\\
98.35	0.00518179203746792\\
98.36	0.00518837582717745\\
98.37	0.00519501734826607\\
98.38	0.00520171780570658\\
98.39	0.00520847843144248\\
98.4	0.00521530048501596\\
98.41	0.00522218525421132\\
98.42	0.00522913405571431\\
98.43	0.00523614823578784\\
98.44	0.00524322917096434\\
98.45	0.00525037826875549\\
98.46	0.00525759696837952\\
98.47	0.00526488674150674\\
98.48	0.00527224909302379\\
98.49	0.00527968556181697\\
98.5	0.0052871977215754\\
98.51	0.00529478718161438\\
98.52	0.00530245558771964\\
98.53	0.00531020462301297\\
98.54	0.00531803600883986\\
98.55	0.00532595150567984\\
98.56	0.00533395291408008\\
98.57	0.00534204207561288\\
98.58	0.00535022087385788\\
98.59	0.00535849123540955\\
98.6	0.00536685513091079\\
98.61	0.00537531457611329\\
98.62	0.00538387163296558\\
98.63	0.00539252841072941\\
98.64	0.00540128706712541\\
98.65	0.00541014980950888\\
98.66	0.0054191188960765\\
98.67	0.00542819663710504\\
98.68	0.00543738539622288\\
98.69	0.00544668759171546\\
98.7	0.00545610569786551\\
98.71	0.0054656422463293\\
98.72	0.00547529982754992\\
98.73	0.00548508109220868\\
98.74	0.00549498875271589\\
98.75	0.0055050255847422\\
98.76	0.00551519442879178\\
98.77	0.00552549819181854\\
98.78	0.005535939848887\\
98.79	0.0055465224448789\\
98.8	0.00555724909624725\\
98.81	0.00556812299281921\\
98.82	0.00557914739964944\\
98.83	0.00559032565892547\\
98.84	0.00560166119192689\\
98.85	0.00561315750103999\\
98.86	0.00562481817182983\\
98.87	0.00563664687517144\\
98.88	0.00564864736944234\\
98.89	0.00566082350281313\\
98.9	0.00567317921560775\\
98.91	0.00568571854269937\\
98.92	0.00569844561597666\\
98.93	0.00571136466688288\\
98.94	0.00572448002903031\\
98.95	0.00573779614089264\\
98.96	0.0057513175485781\\
98.97	0.00576504890868555\\
98.98	0.00577899499124521\\
98.99	0.00579316068275143\\
99	0.00580755098928834\\
99.01	0.00582217103975174\\
99.02	0.00583702608917074\\
99.03	0.00585212152213285\\
99.04	0.00586746285631636\\
99.05	0.00588305574613397\\
99.06	0.00589890598649193\\
99.07	0.00591501951666894\\
99.08	0.00593140242431951\\
99.09	0.00594806094960653\\
99.1	0.0059650014894658\\
99.11	0.00598223060201175\\
99.12	0.00599975501108825\\
99.13	0.00601758161096993\\
99.14	0.0060357174712204\\
99.15	0.00605416984171331\\
99.16	0.00607294615782251\\
99.17	0.00609205404578594\\
99.18	0.00611150132825341\\
99.19	0.00613129603002837\\
99.2	0.00615144638400911\\
99.21	0.00617196083733744\\
99.22	0.006192848057764\\
99.23	0.00621411694023918\\
99.24	0.0062357766137396\\
99.25	0.00625783644834031\\
99.26	0.00628030606254335\\
99.27	0.0063031953308743\\
99.28	0.00632651439175847\\
99.29	0.0063502736556895\\
99.3	0.00637448381370353\\
99.31	0.00639915586344442\\
99.32	0.00642430111417673\\
99.33	0.00644993118405764\\
99.34	0.00647605801001219\\
99.35	0.00650269385799353\\
99.36	0.00652985133364663\\
99.37	0.00655754339339478\\
99.38	0.00658574111450058\\
99.39	0.0066144340019461\\
99.4	0.00664363418934559\\
99.41	0.00667335415000014\\
99.42	0.00670360670988301\\
99.43	0.00673440507738807\\
99.44	0.00676574324691763\\
99.45	0.00679761637399577\\
99.46	0.0068300377389763\\
99.47	0.00686302100297037\\
99.48	0.00689658022086231\\
99.49	0.00693072985487318\\
99.5	0.00696548478869991\\
99.51	0.00700086034226031\\
99.52	0.00703687228707563\\
99.53	0.00707353686232452\\
99.54	0.00711087079160439\\
99.55	0.00714889130043846\\
99.56	0.00718761613456919\\
99.57	0.00722706357908156\\
99.58	0.00726725247840242\\
99.59	0.00730820225722517\\
99.6	0.00734993294241264\\
99.61	0.00739246518593405\\
99.62	0.00743582028889647\\
99.63	0.0074800202267347\\
99.64	0.00752508767562849\\
99.65	0.0075710460402206\\
99.66	0.00761791948271452\\
99.67	0.00766573295343635\\
99.68	0.00771451222295164\\
99.69	0.00776428391583597\\
99.7	0.00781507554620144\\
99.71	0.00786691555509222\\
99.72	0.00791983334987064\\
99.73	0.00797385934572439\\
99.74	0.00802902500943545\\
99.75	0.00808536290556256\\
99.76	0.00814290674520107\\
99.77	0.00820169143749746\\
99.78	0.00826175314411004\\
99.79	0.00832312933682349\\
99.8	0.00838585885854211\\
99.81	0.00844998198790592\\
99.82	0.00851554050779452\\
99.83	0.00858257777800699\\
99.84	0.00865113881243126\\
99.85	0.00872127036104468\\
99.86	0.00879302099711813\\
99.87	0.00886644121003055\\
99.88	0.00894158350413849\\
99.89	0.00901850250418713\\
99.9	0.00909725506779586\\
99.91	0.00917790040560339\\
99.92	0.00926050020971464\\
99.93	0.00934511879115603\\
99.94	0.0094318232271174\\
99.95	0.00952068351883868\\
99.96	0.00961177276108944\\
99.97	0.00970516732428955\\
99.98	0.00980094705043254\\
99.99	0.00989919546410015\\
100	0.01\\
};
\addlegendentry{$q=0$};

\addplot [color=blue,solid,forget plot]
  table[row sep=crcr]{%
0.01	0.00597637174049999\\
0.02	0.00597637265910895\\
0.03	0.00597637358204436\\
0.04	0.00597637450934478\\
0.05	0.00597637544104841\\
0.06	0.00597637637719302\\
0.07	0.00597637731781594\\
0.08	0.00597637826295398\\
0.09	0.00597637921264341\\
0.1	0.00597638016691989\\
0.11	0.00597638112581843\\
0.12	0.00597638208937334\\
0.13	0.00597638305761813\\
0.14	0.00597638403058552\\
0.15	0.00597638500830732\\
0.16	0.00597638599081439\\
0.17	0.00597638697813656\\
0.18	0.00597638797030259\\
0.19	0.00597638896734004\\
0.2	0.00597638996927523\\
0.21	0.00597639097613318\\
0.22	0.00597639198793746\\
0.23	0.00597639300471018\\
0.24	0.00597639402647183\\
0.25	0.00597639505324124\\
0.26	0.00597639608503544\\
0.27	0.00597639712186959\\
0.28	0.00597639816375683\\
0.29	0.00597639921070823\\
0.3	0.00597640026273261\\
0.31	0.00597640131983647\\
0.32	0.00597640238202383\\
0.33	0.00597640344929612\\
0.34	0.00597640452165204\\
0.35	0.00597640559908738\\
0.36	0.00597640668159496\\
0.37	0.00597640776916439\\
0.38	0.00597640886178196\\
0.39	0.00597640995943046\\
0.4	0.00597641106208899\\
0.41	0.00597641216973284\\
0.42	0.00597641328233323\\
0.43	0.00597641439985716\\
0.44	0.00597641552226723\\
0.45	0.00597641664952137\\
0.46	0.00597641778157266\\
0.47	0.00597641891836915\\
0.48	0.00597642005985354\\
0.49	0.005976421205963\\
0.5	0.00597642235662889\\
0.51	0.00597642351177652\\
0.52	0.00597642467132487\\
0.53	0.00597642583518631\\
0.54	0.00597642700326631\\
0.55	0.00597642817546314\\
0.56	0.00597642935166754\\
0.57	0.00597643053176243\\
0.58	0.00597643171562254\\
0.59	0.00597643290311408\\
0.6	0.00597643409409436\\
0.61	0.00597643528841144\\
0.62	0.00597643648590371\\
0.63	0.0059764376863995\\
0.64	0.00597643888971667\\
0.65	0.00597644009566217\\
0.66	0.00597644130403156\\
0.67	0.00597644251460858\\
0.68	0.00597644372716466\\
0.69	0.0059764449414584\\
0.7	0.00597644615723506\\
0.71	0.00597644737422602\\
0.72	0.00597644859214821\\
0.73	0.00597644981070357\\
0.74	0.00597645102972546\\
0.75	0.00597645224921406\\
0.76	0.00597645346916958\\
0.77	0.00597645468959222\\
0.78	0.00597645591048215\\
0.79	0.00597645713183958\\
0.8	0.0059764583536647\\
0.81	0.0059764595759577\\
0.82	0.00597646079871879\\
0.83	0.00597646202194815\\
0.84	0.00597646324564598\\
0.85	0.00597646446981247\\
0.86	0.00597646569444782\\
0.87	0.00597646691955223\\
0.88	0.00597646814512588\\
0.89	0.00597646937116898\\
0.9	0.00597647059768171\\
0.91	0.00597647182466428\\
0.92	0.00597647305211687\\
0.93	0.0059764742800397\\
0.94	0.00597647550843293\\
0.95	0.0059764767372968\\
0.96	0.00597647796663146\\
0.97	0.00597647919643713\\
0.98	0.00597648042671402\\
0.99	0.00597648165746229\\
1	0.00597648288868216\\
1.01	0.00597648412037382\\
1.02	0.00597648535253747\\
1.03	0.00597648658517331\\
1.04	0.00597648781828152\\
1.05	0.00597648905186231\\
1.06	0.00597649028591588\\
1.07	0.00597649152044241\\
1.08	0.00597649275544212\\
1.09	0.00597649399091518\\
1.1	0.00597649522686181\\
1.11	0.0059764964632822\\
1.12	0.00597649770017654\\
1.13	0.00597649893754503\\
1.14	0.00597650017538788\\
1.15	0.00597650141370527\\
1.16	0.00597650265249741\\
1.17	0.0059765038917645\\
1.18	0.00597650513150672\\
1.19	0.00597650637172428\\
1.2	0.00597650761241738\\
1.21	0.00597650885358622\\
1.22	0.00597651009523099\\
1.23	0.00597651133735189\\
1.24	0.00597651257994912\\
1.25	0.00597651382302288\\
1.26	0.00597651506657336\\
1.27	0.00597651631060078\\
1.28	0.00597651755510532\\
1.29	0.00597651880008718\\
1.3	0.00597652004554656\\
1.31	0.00597652129148367\\
1.32	0.00597652253789869\\
1.33	0.00597652378479184\\
1.34	0.00597652503216331\\
1.35	0.00597652628001329\\
1.36	0.005976527528342\\
1.37	0.00597652877714962\\
1.38	0.00597653002643637\\
1.39	0.00597653127620243\\
1.4	0.005976532526448\\
1.41	0.0059765337771733\\
1.42	0.00597653502837852\\
1.43	0.00597653628006385\\
1.44	0.0059765375322295\\
1.45	0.00597653878487567\\
1.46	0.00597654003800256\\
1.47	0.00597654129161037\\
1.48	0.00597654254569931\\
1.49	0.00597654380026957\\
1.5	0.00597654505532135\\
1.51	0.00597654631085485\\
1.52	0.00597654756687028\\
1.53	0.00597654882336783\\
1.54	0.00597655008034772\\
1.55	0.00597655133781013\\
1.56	0.00597655259575528\\
1.57	0.00597655385418335\\
1.58	0.00597655511309456\\
1.59	0.00597655637248911\\
1.6	0.0059765576323672\\
1.61	0.00597655889272902\\
1.62	0.00597656015357479\\
1.63	0.0059765614149047\\
1.64	0.00597656267671897\\
1.65	0.00597656393901778\\
1.66	0.00597656520180134\\
1.67	0.00597656646506986\\
1.68	0.00597656772882353\\
1.69	0.00597656899306256\\
1.7	0.00597657025778717\\
1.71	0.00597657152299753\\
1.72	0.00597657278869386\\
1.73	0.00597657405487637\\
1.74	0.00597657532154525\\
1.75	0.00597657658870072\\
1.76	0.00597657785634296\\
1.77	0.00597657912447219\\
1.78	0.00597658039308862\\
1.79	0.00597658166219243\\
1.8	0.00597658293178385\\
1.81	0.00597658420186306\\
1.82	0.00597658547243029\\
1.83	0.00597658674348573\\
1.84	0.00597658801502958\\
1.85	0.00597658928706205\\
1.86	0.00597659055958334\\
1.87	0.00597659183259366\\
1.88	0.00597659310609322\\
1.89	0.00597659438008221\\
1.9	0.00597659565456085\\
1.91	0.00597659692952934\\
1.92	0.00597659820498787\\
1.93	0.00597659948093667\\
1.94	0.00597660075737594\\
1.95	0.00597660203430587\\
1.96	0.00597660331172668\\
1.97	0.00597660458963857\\
1.98	0.00597660586804174\\
1.99	0.00597660714693641\\
2	0.00597660842632278\\
2.01	0.00597660970620105\\
2.02	0.00597661098657143\\
2.03	0.00597661226743412\\
2.04	0.00597661354878935\\
2.05	0.0059766148306373\\
2.06	0.00597661611297819\\
2.07	0.00597661739581222\\
2.08	0.0059766186791396\\
2.09	0.00597661996296054\\
2.1	0.00597662124727524\\
2.11	0.00597662253208392\\
2.12	0.00597662381738677\\
2.13	0.005976625103184\\
2.14	0.00597662638947583\\
2.15	0.00597662767626247\\
2.16	0.00597662896354411\\
2.17	0.00597663025132096\\
2.18	0.00597663153959323\\
2.19	0.00597663282836114\\
2.2	0.00597663411762489\\
2.21	0.00597663540738469\\
2.22	0.00597663669764075\\
2.23	0.00597663798839326\\
2.24	0.00597663927964245\\
2.25	0.00597664057138853\\
2.26	0.00597664186363169\\
2.27	0.00597664315637214\\
2.28	0.00597664444961011\\
2.29	0.0059766457433458\\
2.3	0.0059766470375794\\
2.31	0.00597664833231115\\
2.32	0.00597664962754124\\
2.33	0.00597665092326988\\
2.34	0.00597665221949728\\
2.35	0.00597665351622366\\
2.36	0.00597665481344921\\
2.37	0.00597665611117416\\
2.38	0.00597665740939872\\
2.39	0.00597665870812308\\
2.4	0.00597666000734747\\
2.41	0.00597666130707209\\
2.42	0.00597666260729715\\
2.43	0.00597666390802287\\
2.44	0.00597666520924944\\
2.45	0.00597666651097709\\
2.46	0.00597666781320603\\
2.47	0.00597666911593646\\
2.48	0.0059766704191686\\
2.49	0.00597667172290265\\
2.5	0.00597667302713884\\
2.51	0.00597667433187736\\
2.52	0.00597667563711844\\
2.53	0.00597667694286228\\
2.54	0.00597667824910909\\
2.55	0.00597667955585909\\
2.56	0.00597668086311249\\
2.57	0.00597668217086949\\
2.58	0.00597668347913032\\
2.59	0.00597668478789518\\
2.6	0.00597668609716429\\
2.61	0.00597668740693786\\
2.62	0.0059766887172161\\
2.63	0.00597669002799922\\
2.64	0.00597669133928744\\
2.65	0.00597669265108096\\
2.66	0.00597669396338001\\
2.67	0.00597669527618479\\
2.68	0.00597669658949552\\
2.69	0.00597669790331241\\
2.7	0.00597669921763568\\
2.71	0.00597670053246553\\
2.72	0.00597670184780218\\
2.73	0.00597670316364585\\
2.74	0.00597670447999675\\
2.75	0.00597670579685509\\
2.76	0.00597670711422108\\
2.77	0.00597670843209494\\
2.78	0.00597670975047689\\
2.79	0.00597671106936713\\
2.8	0.00597671238876589\\
2.81	0.00597671370867338\\
2.82	0.00597671502908981\\
2.83	0.00597671635001539\\
2.84	0.00597671767145035\\
2.85	0.00597671899339489\\
2.86	0.00597672031584923\\
2.87	0.00597672163881359\\
2.88	0.00597672296228818\\
2.89	0.00597672428627322\\
2.9	0.00597672561076892\\
2.91	0.00597672693577549\\
2.92	0.00597672826129316\\
2.93	0.00597672958732214\\
2.94	0.00597673091386264\\
2.95	0.00597673224091489\\
2.96	0.00597673356847909\\
2.97	0.00597673489655546\\
2.98	0.00597673622514422\\
2.99	0.00597673755424559\\
3	0.00597673888385978\\
3.01	0.00597674021398701\\
3.02	0.00597674154462749\\
3.03	0.00597674287578145\\
3.04	0.0059767442074491\\
3.05	0.00597674553963065\\
3.06	0.00597674687232633\\
3.07	0.00597674820553634\\
3.08	0.00597674953926092\\
3.09	0.00597675087350027\\
3.1	0.00597675220825461\\
3.11	0.00597675354352417\\
3.12	0.00597675487930915\\
3.13	0.00597675621560978\\
3.14	0.00597675755242627\\
3.15	0.00597675888975885\\
3.16	0.00597676022760773\\
3.17	0.00597676156597313\\
3.18	0.00597676290485526\\
3.19	0.00597676424425436\\
3.2	0.00597676558417063\\
3.21	0.00597676692460429\\
3.22	0.00597676826555556\\
3.23	0.00597676960702467\\
3.24	0.00597677094901183\\
3.25	0.00597677229151725\\
3.26	0.00597677363454117\\
3.27	0.0059767749780838\\
3.28	0.00597677632214535\\
3.29	0.00597677766672605\\
3.3	0.00597677901182612\\
3.31	0.00597678035744578\\
3.32	0.00597678170358525\\
3.33	0.00597678305024474\\
3.34	0.00597678439742448\\
3.35	0.0059767857451247\\
3.36	0.0059767870933456\\
3.37	0.00597678844208741\\
3.38	0.00597678979135035\\
3.39	0.00597679114113465\\
3.4	0.00597679249144051\\
3.41	0.00597679384226817\\
3.42	0.00597679519361784\\
3.43	0.00597679654548975\\
3.44	0.00597679789788412\\
3.45	0.00597679925080116\\
3.46	0.0059768006042411\\
3.47	0.00597680195820417\\
3.48	0.00597680331269057\\
3.49	0.00597680466770055\\
3.5	0.00597680602323431\\
3.51	0.00597680737929208\\
3.52	0.00597680873587408\\
3.53	0.00597681009298054\\
3.54	0.00597681145061167\\
3.55	0.0059768128087677\\
3.56	0.00597681416744885\\
3.57	0.00597681552665535\\
3.58	0.00597681688638742\\
3.59	0.00597681824664528\\
3.6	0.00597681960742915\\
3.61	0.00597682096873925\\
3.62	0.00597682233057583\\
3.63	0.00597682369293908\\
3.64	0.00597682505582924\\
3.65	0.00597682641924653\\
3.66	0.00597682778319118\\
3.67	0.0059768291476634\\
3.68	0.00597683051266343\\
3.69	0.00597683187819149\\
3.7	0.00597683324424779\\
3.71	0.00597683461083258\\
3.72	0.00597683597794606\\
3.73	0.00597683734558847\\
3.74	0.00597683871376002\\
3.75	0.00597684008246096\\
3.76	0.00597684145169149\\
3.77	0.00597684282145185\\
3.78	0.00597684419174226\\
3.79	0.00597684556256294\\
3.8	0.00597684693391412\\
3.81	0.00597684830579603\\
3.82	0.00597684967820889\\
3.83	0.00597685105115293\\
3.84	0.00597685242462837\\
3.85	0.00597685379863544\\
3.86	0.00597685517317437\\
3.87	0.00597685654824538\\
3.88	0.0059768579238487\\
3.89	0.00597685929998455\\
3.9	0.00597686067665316\\
3.91	0.00597686205385476\\
3.92	0.00597686343158958\\
3.93	0.00597686480985783\\
3.94	0.00597686618865976\\
3.95	0.00597686756799558\\
3.96	0.00597686894786552\\
3.97	0.00597687032826982\\
3.98	0.00597687170920869\\
3.99	0.00597687309068237\\
4	0.00597687447269108\\
4.01	0.00597687585523506\\
4.02	0.00597687723831451\\
4.03	0.00597687862192969\\
4.04	0.00597688000608082\\
4.05	0.00597688139076812\\
4.06	0.00597688277599182\\
4.07	0.00597688416175215\\
4.08	0.00597688554804934\\
4.09	0.00597688693488362\\
4.1	0.00597688832225521\\
4.11	0.00597688971016435\\
4.12	0.00597689109861127\\
4.13	0.00597689248759619\\
4.14	0.00597689387711934\\
4.15	0.00597689526718096\\
4.16	0.00597689665778127\\
4.17	0.0059768980489205\\
4.18	0.00597689944059889\\
4.19	0.00597690083281665\\
4.2	0.00597690222557403\\
4.21	0.00597690361887125\\
4.22	0.00597690501270855\\
4.23	0.00597690640708614\\
4.24	0.00597690780200427\\
4.25	0.00597690919746317\\
4.26	0.00597691059346306\\
4.27	0.00597691199000418\\
4.28	0.00597691338708675\\
4.29	0.00597691478471101\\
4.3	0.00597691618287719\\
4.31	0.00597691758158552\\
4.32	0.00597691898083623\\
4.33	0.00597692038062956\\
4.34	0.00597692178096574\\
4.35	0.00597692318184499\\
4.36	0.00597692458326755\\
4.37	0.00597692598523365\\
4.38	0.00597692738774353\\
4.39	0.00597692879079741\\
4.4	0.00597693019439553\\
4.41	0.00597693159853812\\
4.42	0.00597693300322541\\
4.43	0.00597693440845765\\
4.44	0.00597693581423504\\
4.45	0.00597693722055784\\
4.46	0.00597693862742628\\
4.47	0.00597694003484059\\
4.48	0.005976941442801\\
4.49	0.00597694285130774\\
4.5	0.00597694426036106\\
4.51	0.00597694566996117\\
4.52	0.00597694708010832\\
4.53	0.00597694849080274\\
4.54	0.00597694990204467\\
4.55	0.00597695131383433\\
4.56	0.00597695272617198\\
4.57	0.00597695413905783\\
4.58	0.00597695555249212\\
4.59	0.00597695696647509\\
4.6	0.00597695838100697\\
4.61	0.00597695979608799\\
4.62	0.0059769612117184\\
4.63	0.00597696262789843\\
4.64	0.00597696404462831\\
4.65	0.00597696546190827\\
4.66	0.00597696687973856\\
4.67	0.00597696829811941\\
4.68	0.00597696971705105\\
4.69	0.00597697113653373\\
4.7	0.00597697255656767\\
4.71	0.00597697397715312\\
4.72	0.0059769753982903\\
4.73	0.00597697681997946\\
4.74	0.00597697824222083\\
4.75	0.00597697966501465\\
4.76	0.00597698108836115\\
4.77	0.00597698251226058\\
4.78	0.00597698393671317\\
4.79	0.00597698536171916\\
4.8	0.00597698678727878\\
4.81	0.00597698821339226\\
4.82	0.00597698964005987\\
4.83	0.00597699106728181\\
4.84	0.00597699249505834\\
4.85	0.00597699392338969\\
4.86	0.0059769953522761\\
4.87	0.00597699678171781\\
4.88	0.00597699821171506\\
4.89	0.00597699964226808\\
4.9	0.00597700107337712\\
4.91	0.0059770025050424\\
4.92	0.00597700393726418\\
4.93	0.00597700537004269\\
4.94	0.00597700680337816\\
4.95	0.00597700823727084\\
4.96	0.00597700967172097\\
4.97	0.00597701110672878\\
4.98	0.00597701254229452\\
4.99	0.00597701397841842\\
5	0.00597701541510073\\
5.01	0.00597701685234169\\
5.02	0.00597701829014152\\
5.03	0.00597701972850049\\
5.04	0.00597702116741882\\
5.05	0.00597702260689675\\
5.06	0.00597702404693453\\
5.07	0.00597702548753239\\
5.08	0.00597702692869058\\
5.09	0.00597702837040934\\
5.1	0.00597702981268891\\
5.11	0.00597703125552953\\
5.12	0.00597703269893144\\
5.13	0.00597703414289488\\
5.14	0.00597703558742009\\
5.15	0.00597703703250733\\
5.16	0.00597703847815681\\
5.17	0.0059770399243688\\
5.18	0.00597704137114353\\
5.19	0.00597704281848124\\
5.2	0.00597704426638217\\
5.21	0.00597704571484657\\
5.22	0.00597704716387469\\
5.23	0.00597704861346675\\
5.24	0.00597705006362301\\
5.25	0.00597705151434371\\
5.26	0.00597705296562909\\
5.27	0.00597705441747939\\
5.28	0.00597705586989487\\
5.29	0.00597705732287574\\
5.3	0.00597705877642228\\
5.31	0.00597706023053471\\
5.32	0.00597706168521329\\
5.33	0.00597706314045825\\
5.34	0.00597706459626983\\
5.35	0.0059770660526483\\
5.36	0.00597706750959387\\
5.37	0.00597706896710682\\
5.38	0.00597707042518737\\
5.39	0.00597707188383576\\
5.4	0.00597707334305226\\
5.41	0.0059770748028371\\
5.42	0.00597707626319052\\
5.43	0.00597707772411277\\
5.44	0.0059770791856041\\
5.45	0.00597708064766475\\
5.46	0.00597708211029496\\
5.47	0.00597708357349499\\
5.48	0.00597708503726508\\
5.49	0.00597708650160547\\
5.5	0.00597708796651641\\
5.51	0.00597708943199815\\
5.52	0.00597709089805093\\
5.53	0.00597709236467499\\
5.54	0.0059770938318706\\
5.55	0.00597709529963798\\
5.56	0.0059770967679774\\
5.57	0.00597709823688909\\
5.58	0.0059770997063733\\
5.59	0.00597710117643028\\
5.6	0.00597710264706028\\
5.61	0.00597710411826355\\
5.62	0.00597710559004032\\
5.63	0.00597710706239086\\
5.64	0.0059771085353154\\
5.65	0.0059771100088142\\
5.66	0.0059771114828875\\
5.67	0.00597711295753556\\
5.68	0.00597711443275862\\
5.69	0.00597711590855692\\
5.7	0.00597711738493072\\
5.71	0.00597711886188027\\
5.72	0.00597712033940581\\
5.73	0.0059771218175076\\
5.74	0.00597712329618588\\
5.75	0.0059771247754409\\
5.76	0.00597712625527291\\
5.77	0.00597712773568216\\
5.78	0.0059771292166689\\
5.79	0.00597713069823339\\
5.8	0.00597713218037586\\
5.81	0.00597713366309657\\
5.82	0.00597713514639578\\
5.83	0.00597713663027373\\
5.84	0.00597713811473067\\
5.85	0.00597713959976684\\
5.86	0.00597714108538251\\
5.87	0.00597714257157793\\
5.88	0.00597714405835333\\
5.89	0.00597714554570899\\
5.9	0.00597714703364514\\
5.91	0.00597714852216203\\
5.92	0.00597715001125993\\
5.93	0.00597715150093908\\
5.94	0.00597715299119973\\
5.95	0.00597715448204213\\
5.96	0.00597715597346654\\
5.97	0.00597715746547321\\
5.98	0.00597715895806239\\
5.99	0.00597716045123434\\
6	0.00597716194498929\\
6.01	0.00597716343932752\\
6.02	0.00597716493424926\\
6.03	0.00597716642975478\\
6.04	0.00597716792584433\\
6.05	0.00597716942251816\\
6.06	0.00597717091977652\\
6.07	0.00597717241761967\\
6.08	0.00597717391604786\\
6.09	0.00597717541506134\\
6.1	0.00597717691466037\\
6.11	0.0059771784148452\\
6.12	0.0059771799156161\\
6.13	0.00597718141697329\\
6.14	0.00597718291891706\\
6.15	0.00597718442144764\\
6.16	0.0059771859245653\\
6.17	0.00597718742827029\\
6.18	0.00597718893256285\\
6.19	0.00597719043744326\\
6.2	0.00597719194291176\\
6.21	0.00597719344896861\\
6.22	0.00597719495561407\\
6.23	0.00597719646284839\\
6.24	0.00597719797067182\\
6.25	0.00597719947908462\\
6.26	0.00597720098808705\\
6.27	0.00597720249767936\\
6.28	0.00597720400786182\\
6.29	0.00597720551863466\\
6.3	0.00597720702999817\\
6.31	0.00597720854195258\\
6.32	0.00597721005449816\\
6.33	0.00597721156763517\\
6.34	0.00597721308136385\\
6.35	0.00597721459568447\\
6.36	0.00597721611059729\\
6.37	0.00597721762610255\\
6.38	0.00597721914220054\\
6.39	0.00597722065889148\\
6.4	0.00597722217617565\\
6.41	0.00597722369405331\\
6.42	0.00597722521252471\\
6.43	0.00597722673159011\\
6.44	0.00597722825124977\\
6.45	0.00597722977150394\\
6.46	0.00597723129235289\\
6.47	0.00597723281379688\\
6.48	0.00597723433583616\\
6.49	0.00597723585847099\\
6.5	0.00597723738170163\\
6.51	0.00597723890552835\\
6.52	0.00597724042995139\\
6.53	0.00597724195497103\\
6.54	0.00597724348058751\\
6.55	0.00597724500680111\\
6.56	0.00597724653361208\\
6.57	0.00597724806102067\\
6.58	0.00597724958902716\\
6.59	0.0059772511176318\\
6.6	0.00597725264683485\\
6.61	0.00597725417663658\\
6.62	0.00597725570703724\\
6.63	0.00597725723803709\\
6.64	0.0059772587696364\\
6.65	0.00597726030183542\\
6.66	0.00597726183463443\\
6.67	0.00597726336803368\\
6.68	0.00597726490203342\\
6.69	0.00597726643663393\\
6.7	0.00597726797183547\\
6.71	0.00597726950763829\\
6.72	0.00597727104404267\\
6.73	0.00597727258104885\\
6.74	0.00597727411865711\\
6.75	0.00597727565686771\\
6.76	0.00597727719568091\\
6.77	0.00597727873509698\\
6.78	0.00597728027511616\\
6.79	0.00597728181573874\\
6.8	0.00597728335696497\\
6.81	0.00597728489879513\\
6.82	0.00597728644122946\\
6.83	0.00597728798426823\\
6.84	0.00597728952791172\\
6.85	0.00597729107216017\\
6.86	0.00597729261701387\\
6.87	0.00597729416247306\\
6.88	0.00597729570853802\\
6.89	0.005977297255209\\
6.9	0.00597729880248629\\
6.91	0.00597730035037013\\
6.92	0.0059773018988608\\
6.93	0.00597730344795856\\
6.94	0.00597730499766367\\
6.95	0.00597730654797641\\
6.96	0.00597730809889702\\
6.97	0.0059773096504258\\
6.98	0.00597731120256299\\
6.99	0.00597731275530886\\
7	0.00597731430866369\\
7.01	0.00597731586262773\\
7.02	0.00597731741720126\\
7.03	0.00597731897238454\\
7.04	0.00597732052817783\\
7.05	0.00597732208458141\\
7.06	0.00597732364159553\\
7.07	0.00597732519922048\\
7.08	0.00597732675745651\\
7.09	0.00597732831630389\\
7.1	0.00597732987576289\\
7.11	0.00597733143583378\\
7.12	0.00597733299651683\\
7.13	0.0059773345578123\\
7.14	0.00597733611972046\\
7.15	0.00597733768224159\\
7.16	0.00597733924537594\\
7.17	0.00597734080912378\\
7.18	0.0059773423734854\\
7.19	0.00597734393846105\\
7.2	0.005977345504051\\
7.21	0.00597734707025553\\
7.22	0.0059773486370749\\
7.23	0.00597735020450938\\
7.24	0.00597735177255924\\
7.25	0.00597735334122476\\
7.26	0.0059773549105062\\
7.27	0.00597735648040382\\
7.28	0.00597735805091791\\
7.29	0.00597735962204873\\
7.3	0.00597736119379656\\
7.31	0.00597736276616165\\
7.32	0.00597736433914429\\
7.33	0.00597736591274475\\
7.34	0.00597736748696329\\
7.35	0.00597736906180018\\
7.36	0.00597737063725571\\
7.37	0.00597737221333014\\
7.38	0.00597737379002373\\
7.39	0.00597737536733677\\
7.4	0.00597737694526952\\
7.41	0.00597737852382226\\
7.42	0.00597738010299526\\
7.43	0.00597738168278879\\
7.44	0.00597738326320313\\
7.45	0.00597738484423854\\
7.46	0.0059773864258953\\
7.47	0.00597738800817369\\
7.48	0.00597738959107397\\
7.49	0.00597739117459642\\
7.5	0.0059773927587413\\
7.51	0.00597739434350891\\
7.52	0.0059773959288995\\
7.53	0.00597739751491336\\
7.54	0.00597739910155076\\
7.55	0.00597740068881196\\
7.56	0.00597740227669725\\
7.57	0.0059774038652069\\
7.58	0.00597740545434119\\
7.59	0.00597740704410038\\
7.6	0.00597740863448476\\
7.61	0.00597741022549459\\
7.62	0.00597741181713016\\
7.63	0.00597741340939174\\
7.64	0.0059774150022796\\
7.65	0.00597741659579402\\
7.66	0.00597741818993529\\
7.67	0.00597741978470365\\
7.68	0.00597742138009941\\
7.69	0.00597742297612283\\
7.7	0.00597742457277419\\
7.71	0.00597742617005377\\
7.72	0.00597742776796184\\
7.73	0.00597742936649868\\
7.74	0.00597743096566456\\
7.75	0.00597743256545977\\
7.76	0.00597743416588458\\
7.77	0.00597743576693927\\
7.78	0.00597743736862411\\
7.79	0.00597743897093939\\
7.8	0.00597744057388538\\
7.81	0.00597744217746236\\
7.82	0.0059774437816706\\
7.83	0.00597744538651039\\
7.84	0.005977446991982\\
7.85	0.00597744859808571\\
7.86	0.00597745020482181\\
7.87	0.00597745181219056\\
7.88	0.00597745342019225\\
7.89	0.00597745502882716\\
7.9	0.00597745663809557\\
7.91	0.00597745824799775\\
7.92	0.00597745985853399\\
7.93	0.00597746146970456\\
7.94	0.00597746308150975\\
7.95	0.00597746469394983\\
7.96	0.00597746630702509\\
7.97	0.0059774679207358\\
7.98	0.00597746953508225\\
7.99	0.00597747115006472\\
8	0.00597747276568349\\
8.01	0.00597747438193883\\
8.02	0.00597747599883103\\
8.03	0.00597747761636037\\
8.04	0.00597747923452713\\
8.05	0.0059774808533316\\
8.06	0.00597748247277405\\
8.07	0.00597748409285477\\
8.08	0.00597748571357403\\
8.09	0.00597748733493213\\
8.1	0.00597748895692934\\
8.11	0.00597749057956594\\
8.12	0.00597749220284223\\
8.13	0.00597749382675847\\
8.14	0.00597749545131496\\
8.15	0.00597749707651197\\
8.16	0.00597749870234979\\
8.17	0.0059775003288287\\
8.18	0.00597750195594899\\
8.19	0.00597750358371094\\
8.2	0.00597750521211484\\
8.21	0.00597750684116096\\
8.22	0.00597750847084959\\
8.23	0.00597751010118102\\
8.24	0.00597751173215553\\
8.25	0.0059775133637734\\
8.26	0.00597751499603492\\
8.27	0.00597751662894037\\
8.28	0.00597751826249004\\
8.29	0.00597751989668422\\
8.3	0.00597752153152318\\
8.31	0.00597752316700722\\
8.32	0.00597752480313662\\
8.33	0.00597752643991167\\
8.34	0.00597752807733264\\
8.35	0.00597752971539983\\
8.36	0.00597753135411353\\
8.37	0.00597753299347402\\
8.38	0.00597753463348158\\
8.39	0.00597753627413651\\
8.4	0.00597753791543908\\
8.41	0.00597753955738959\\
8.42	0.00597754119998833\\
8.43	0.00597754284323557\\
8.44	0.00597754448713161\\
8.45	0.00597754613167674\\
8.46	0.00597754777687125\\
8.47	0.00597754942271541\\
8.48	0.00597755106920952\\
8.49	0.00597755271635387\\
8.5	0.00597755436414875\\
8.51	0.00597755601259443\\
8.52	0.00597755766169122\\
8.53	0.0059775593114394\\
8.54	0.00597756096183926\\
8.55	0.00597756261289109\\
8.56	0.00597756426459518\\
8.57	0.00597756591695182\\
8.58	0.00597756756996129\\
8.59	0.00597756922362389\\
8.6	0.0059775708779399\\
8.61	0.00597757253290962\\
8.62	0.00597757418853334\\
8.63	0.00597757584481134\\
8.64	0.00597757750174392\\
8.65	0.00597757915933137\\
8.66	0.00597758081757397\\
8.67	0.00597758247647203\\
8.68	0.00597758413602582\\
8.69	0.00597758579623565\\
8.7	0.00597758745710179\\
8.71	0.00597758911862455\\
8.72	0.00597759078080422\\
8.73	0.00597759244364108\\
8.74	0.00597759410713543\\
8.75	0.00597759577128757\\
8.76	0.00597759743609777\\
8.77	0.00597759910156634\\
8.78	0.00597760076769357\\
8.79	0.00597760243447975\\
8.8	0.00597760410192518\\
8.81	0.00597760577003013\\
8.82	0.00597760743879492\\
8.83	0.00597760910821983\\
8.84	0.00597761077830516\\
8.85	0.0059776124490512\\
8.86	0.00597761412045824\\
8.87	0.00597761579252658\\
8.88	0.0059776174652565\\
8.89	0.00597761913864832\\
8.9	0.00597762081270231\\
8.91	0.00597762248741878\\
8.92	0.00597762416279802\\
8.93	0.00597762583884032\\
8.94	0.00597762751554597\\
8.95	0.00597762919291529\\
8.96	0.00597763087094855\\
8.97	0.00597763254964605\\
8.98	0.0059776342290081\\
8.99	0.00597763590903497\\
9	0.00597763758972698\\
9.01	0.00597763927108442\\
9.02	0.00597764095310758\\
9.03	0.00597764263579676\\
9.04	0.00597764431915225\\
9.05	0.00597764600317436\\
9.06	0.00597764768786337\\
9.07	0.0059776493732196\\
9.08	0.00597765105924332\\
9.09	0.00597765274593484\\
9.1	0.00597765443329446\\
9.11	0.00597765612132248\\
9.12	0.00597765781001919\\
9.13	0.00597765949938488\\
9.14	0.00597766118941987\\
9.15	0.00597766288012444\\
9.16	0.00597766457149889\\
9.17	0.00597766626354353\\
9.18	0.00597766795625865\\
9.19	0.00597766964964455\\
9.2	0.00597767134370153\\
9.21	0.00597767303842988\\
9.22	0.00597767473382992\\
9.23	0.00597767642990192\\
9.24	0.0059776781266462\\
9.25	0.00597767982406306\\
9.26	0.00597768152215279\\
9.27	0.00597768322091569\\
9.28	0.00597768492035207\\
9.29	0.00597768662046222\\
9.3	0.00597768832124645\\
9.31	0.00597769002270505\\
9.32	0.00597769172483832\\
9.33	0.00597769342764658\\
9.34	0.0059776951311301\\
9.35	0.00597769683528921\\
9.36	0.00597769854012419\\
9.37	0.00597770024563535\\
9.38	0.005977701951823\\
9.39	0.00597770365868743\\
9.4	0.00597770536622894\\
9.41	0.00597770707444784\\
9.42	0.00597770878334443\\
9.43	0.00597771049291901\\
9.44	0.00597771220317188\\
9.45	0.00597771391410335\\
9.46	0.00597771562571371\\
9.47	0.00597771733800328\\
9.48	0.00597771905097234\\
9.49	0.00597772076462122\\
9.5	0.0059777224789502\\
9.51	0.00597772419395959\\
9.52	0.0059777259096497\\
9.53	0.00597772762602083\\
9.54	0.00597772934307328\\
9.55	0.00597773106080736\\
9.56	0.00597773277922337\\
9.57	0.00597773449832162\\
9.58	0.0059777362181024\\
9.59	0.00597773793856603\\
9.6	0.0059777396597128\\
9.61	0.00597774138154303\\
9.62	0.00597774310405701\\
9.63	0.00597774482725506\\
9.64	0.00597774655113747\\
9.65	0.00597774827570455\\
9.66	0.00597775000095662\\
9.67	0.00597775172689396\\
9.68	0.00597775345351689\\
9.69	0.00597775518082572\\
9.7	0.00597775690882075\\
9.71	0.00597775863750228\\
9.72	0.00597776036687063\\
9.73	0.00597776209692608\\
9.74	0.00597776382766897\\
9.75	0.00597776555909959\\
9.76	0.00597776729121825\\
9.77	0.00597776902402525\\
9.78	0.0059777707575209\\
9.79	0.0059777724917055\\
9.8	0.00597777422657938\\
9.81	0.00597777596214282\\
9.82	0.00597777769839615\\
9.83	0.00597777943533966\\
9.84	0.00597778117297367\\
9.85	0.00597778291129848\\
9.86	0.00597778465031439\\
9.87	0.00597778639002173\\
9.88	0.0059777881304208\\
9.89	0.00597778987151189\\
9.9	0.00597779161329533\\
9.91	0.00597779335577143\\
9.92	0.00597779509894048\\
9.93	0.0059777968428028\\
9.94	0.00597779858735869\\
9.95	0.00597780033260848\\
9.96	0.00597780207855245\\
9.97	0.00597780382519093\\
9.98	0.00597780557252423\\
9.99	0.00597780732055265\\
10	0.0059778090692765\\
10.01	0.00597781081869609\\
10.02	0.00597781256881174\\
10.03	0.00597781431962374\\
10.04	0.00597781607113242\\
10.05	0.00597781782333808\\
10.06	0.00597781957624102\\
10.07	0.00597782132984157\\
10.08	0.00597782308414003\\
10.09	0.00597782483913672\\
10.1	0.00597782659483193\\
10.11	0.005977828351226\\
10.12	0.00597783010831921\\
10.13	0.00597783186611189\\
10.14	0.00597783362460434\\
10.15	0.00597783538379688\\
10.16	0.00597783714368982\\
10.17	0.00597783890428347\\
10.18	0.00597784066557814\\
10.19	0.00597784242757414\\
10.2	0.00597784419027178\\
10.21	0.00597784595367138\\
10.22	0.00597784771777325\\
10.23	0.00597784948257769\\
10.24	0.00597785124808502\\
10.25	0.00597785301429556\\
10.26	0.00597785478120961\\
10.27	0.00597785654882748\\
10.28	0.00597785831714949\\
10.29	0.00597786008617596\\
10.3	0.00597786185590718\\
10.31	0.00597786362634348\\
10.32	0.00597786539748517\\
10.33	0.00597786716933256\\
10.34	0.00597786894188596\\
10.35	0.00597787071514568\\
10.36	0.00597787248911204\\
10.37	0.00597787426378536\\
10.38	0.00597787603916593\\
10.39	0.00597787781525408\\
10.4	0.00597787959205013\\
10.41	0.00597788136955437\\
10.42	0.00597788314776713\\
10.43	0.00597788492668872\\
10.44	0.00597788670631945\\
10.45	0.00597788848665963\\
10.46	0.00597789026770958\\
10.47	0.00597789204946962\\
10.48	0.00597789383194005\\
10.49	0.00597789561512119\\
10.5	0.00597789739901334\\
10.51	0.00597789918361684\\
10.52	0.00597790096893198\\
10.53	0.00597790275495908\\
10.54	0.00597790454169847\\
10.55	0.00597790632915043\\
10.56	0.00597790811731531\\
10.57	0.0059779099061934\\
10.58	0.00597791169578502\\
10.59	0.00597791348609049\\
10.6	0.00597791527711011\\
10.61	0.00597791706884421\\
10.62	0.00597791886129309\\
10.63	0.00597792065445707\\
10.64	0.00597792244833646\\
10.65	0.00597792424293158\\
10.66	0.00597792603824274\\
10.67	0.00597792783427026\\
10.68	0.00597792963101444\\
10.69	0.00597793142847561\\
10.7	0.00597793322665407\\
10.71	0.00597793502555014\\
10.72	0.00597793682516413\\
10.73	0.00597793862549637\\
10.74	0.00597794042654715\\
10.75	0.00597794222831679\\
10.76	0.00597794403080562\\
10.77	0.00597794583401394\\
10.78	0.00597794763794206\\
10.79	0.0059779494425903\\
10.8	0.00597795124795898\\
10.81	0.0059779530540484\\
10.82	0.00597795486085888\\
10.83	0.00597795666839074\\
10.84	0.00597795847664428\\
10.85	0.00597796028561982\\
10.86	0.00597796209531767\\
10.87	0.00597796390573815\\
10.88	0.00597796571688157\\
10.89	0.00597796752874824\\
10.9	0.00597796934133847\\
10.91	0.00597797115465258\\
10.92	0.00597797296869089\\
10.93	0.00597797478345369\\
10.94	0.00597797659894131\\
10.95	0.00597797841515407\\
10.96	0.00597798023209226\\
10.97	0.0059779820497562\\
10.98	0.00597798386814622\\
10.99	0.00597798568726261\\
11	0.00597798750710569\\
11.01	0.00597798932767577\\
11.02	0.00597799114897316\\
11.03	0.00597799297099818\\
11.04	0.00597799479375114\\
11.05	0.00597799661723234\\
11.06	0.0059779984414421\\
11.07	0.00597800026638073\\
11.08	0.00597800209204855\\
11.09	0.00597800391844585\\
11.1	0.00597800574557296\\
11.11	0.00597800757343018\\
11.12	0.00597800940201782\\
11.13	0.0059780112313362\\
11.14	0.00597801306138562\\
11.15	0.00597801489216639\\
11.16	0.00597801672367882\\
11.17	0.00597801855592323\\
11.18	0.00597802038889991\\
11.19	0.00597802222260918\\
11.2	0.00597802405705136\\
11.21	0.00597802589222673\\
11.22	0.00597802772813563\\
11.23	0.00597802956477834\\
11.24	0.00597803140215519\\
11.25	0.00597803324026647\\
11.26	0.00597803507911249\\
11.27	0.00597803691869357\\
11.28	0.00597803875901001\\
11.29	0.00597804060006211\\
11.3	0.00597804244185019\\
11.31	0.00597804428437453\\
11.32	0.00597804612763547\\
11.33	0.00597804797163328\\
11.34	0.0059780498163683\\
11.35	0.0059780516618408\\
11.36	0.00597805350805111\\
11.37	0.00597805535499953\\
11.38	0.00597805720268635\\
11.39	0.00597805905111189\\
11.4	0.00597806090027644\\
11.41	0.00597806275018031\\
11.42	0.0059780646008238\\
11.43	0.00597806645220721\\
11.44	0.00597806830433084\\
11.45	0.00597807015719499\\
11.46	0.00597807201079997\\
11.47	0.00597807386514608\\
11.48	0.00597807572023361\\
11.49	0.00597807757606287\\
11.5	0.00597807943263414\\
11.51	0.00597808128994774\\
11.52	0.00597808314800396\\
11.53	0.00597808500680309\\
11.54	0.00597808686634543\\
11.55	0.00597808872663129\\
11.56	0.00597809058766096\\
11.57	0.00597809244943472\\
11.58	0.00597809431195289\\
11.59	0.00597809617521575\\
11.6	0.00597809803922358\\
11.61	0.0059780999039767\\
11.62	0.0059781017694754\\
11.63	0.00597810363571996\\
11.64	0.00597810550271067\\
11.65	0.00597810737044783\\
11.66	0.00597810923893173\\
11.67	0.00597811110816266\\
11.68	0.00597811297814091\\
11.69	0.00597811484886676\\
11.7	0.00597811672034051\\
11.71	0.00597811859256244\\
11.72	0.00597812046553284\\
11.73	0.00597812233925199\\
11.74	0.00597812421372019\\
11.75	0.00597812608893771\\
11.76	0.00597812796490484\\
11.77	0.00597812984162186\\
11.78	0.00597813171908907\\
11.79	0.00597813359730672\\
11.8	0.00597813547627512\\
11.81	0.00597813735599454\\
11.82	0.00597813923646525\\
11.83	0.00597814111768754\\
11.84	0.00597814299966169\\
11.85	0.00597814488238798\\
11.86	0.00597814676586667\\
11.87	0.00597814865009805\\
11.88	0.00597815053508238\\
11.89	0.00597815242081996\\
11.9	0.00597815430731103\\
11.91	0.00597815619455589\\
11.92	0.0059781580825548\\
11.93	0.00597815997130803\\
11.94	0.00597816186081584\\
11.95	0.00597816375107852\\
11.96	0.00597816564209632\\
11.97	0.00597816753386951\\
11.98	0.00597816942639836\\
11.99	0.00597817131968314\\
12	0.00597817321372409\\
12.01	0.00597817510852149\\
12.02	0.0059781770040756\\
12.03	0.00597817890038668\\
12.04	0.00597818079745499\\
12.05	0.00597818269528079\\
12.06	0.00597818459386432\\
12.07	0.00597818649320586\\
12.08	0.00597818839330564\\
12.09	0.00597819029416394\\
12.1	0.005978192195781\\
12.11	0.00597819409815706\\
12.12	0.00597819600129238\\
12.13	0.00597819790518722\\
12.14	0.00597819980984181\\
12.15	0.0059782017152564\\
12.16	0.00597820362143124\\
12.17	0.00597820552836657\\
12.18	0.00597820743606263\\
12.19	0.00597820934451967\\
12.2	0.00597821125373793\\
12.21	0.00597821316371763\\
12.22	0.00597821507445903\\
12.23	0.00597821698596235\\
12.24	0.00597821889822782\\
12.25	0.00597822081125569\\
12.26	0.00597822272504619\\
12.27	0.00597822463959953\\
12.28	0.00597822655491596\\
12.29	0.00597822847099569\\
12.3	0.00597823038783896\\
12.31	0.00597823230544598\\
12.32	0.00597823422381699\\
12.33	0.00597823614295219\\
12.34	0.00597823806285182\\
12.35	0.00597823998351607\\
12.36	0.00597824190494518\\
12.37	0.00597824382713936\\
12.38	0.00597824575009882\\
12.39	0.00597824767382377\\
12.4	0.00597824959831442\\
12.41	0.00597825152357097\\
12.42	0.00597825344959364\\
12.43	0.00597825537638262\\
12.44	0.00597825730393813\\
12.45	0.00597825923226035\\
12.46	0.00597826116134949\\
12.47	0.00597826309120575\\
12.48	0.00597826502182932\\
12.49	0.0059782669532204\\
12.5	0.00597826888537917\\
12.51	0.00597827081830582\\
12.52	0.00597827275200055\\
12.53	0.00597827468646354\\
12.54	0.00597827662169497\\
12.55	0.00597827855769502\\
12.56	0.00597828049446388\\
12.57	0.00597828243200172\\
12.58	0.00597828437030873\\
12.59	0.00597828630938506\\
12.6	0.0059782882492309\\
12.61	0.00597829018984641\\
12.62	0.00597829213123177\\
12.63	0.00597829407338713\\
12.64	0.00597829601631267\\
12.65	0.00597829796000854\\
12.66	0.00597829990447491\\
12.67	0.00597830184971192\\
12.68	0.00597830379571975\\
12.69	0.00597830574249853\\
12.7	0.00597830769004843\\
12.71	0.00597830963836958\\
12.72	0.00597831158746214\\
12.73	0.00597831353732626\\
12.74	0.00597831548796206\\
12.75	0.0059783174393697\\
12.76	0.00597831939154931\\
12.77	0.00597832134450103\\
12.78	0.00597832329822499\\
12.79	0.00597832525272133\\
12.8	0.00597832720799016\\
12.81	0.00597832916403162\\
12.82	0.00597833112084583\\
12.83	0.00597833307843292\\
12.84	0.005978335036793\\
12.85	0.0059783369959262\\
12.86	0.00597833895583262\\
12.87	0.00597834091651238\\
12.88	0.0059783428779656\\
12.89	0.00597834484019237\\
12.9	0.00597834680319281\\
12.91	0.00597834876696702\\
12.92	0.0059783507315151\\
12.93	0.00597835269683715\\
12.94	0.00597835466293327\\
12.95	0.00597835662980355\\
12.96	0.00597835859744808\\
12.97	0.00597836056586697\\
12.98	0.00597836253506028\\
12.99	0.00597836450502811\\
13	0.00597836647577054\\
13.01	0.00597836844728766\\
13.02	0.00597837041957953\\
13.03	0.00597837239264624\\
13.04	0.00597837436648787\\
13.05	0.00597837634110448\\
13.06	0.00597837831649615\\
13.07	0.00597838029266294\\
13.08	0.00597838226960491\\
13.09	0.00597838424732214\\
13.1	0.00597838622581468\\
13.11	0.0059783882050826\\
13.12	0.00597839018512594\\
13.13	0.00597839216594477\\
13.14	0.00597839414753914\\
13.15	0.00597839612990909\\
13.16	0.00597839811305469\\
13.17	0.00597840009697597\\
13.18	0.00597840208167298\\
13.19	0.00597840406714577\\
13.2	0.00597840605339437\\
13.21	0.00597840804041884\\
13.22	0.0059784100282192\\
13.23	0.00597841201679549\\
13.24	0.00597841400614775\\
13.25	0.005978415996276\\
13.26	0.00597841798718029\\
13.27	0.00597841997886064\\
13.28	0.00597842197131708\\
13.29	0.00597842396454963\\
13.3	0.00597842595855833\\
13.31	0.00597842795334321\\
13.32	0.00597842994890427\\
13.33	0.00597843194524154\\
13.34	0.00597843394235505\\
13.35	0.00597843594024482\\
13.36	0.00597843793891086\\
13.37	0.00597843993835319\\
13.38	0.00597844193857184\\
13.39	0.00597844393956681\\
13.4	0.00597844594133812\\
13.41	0.00597844794388579\\
13.42	0.00597844994720983\\
13.43	0.00597845195131027\\
13.44	0.00597845395618711\\
13.45	0.00597845596184036\\
13.46	0.00597845796827004\\
13.47	0.00597845997547617\\
13.48	0.00597846198345875\\
13.49	0.00597846399221781\\
13.5	0.00597846600175336\\
13.51	0.0059784680120654\\
13.52	0.00597847002315397\\
13.53	0.00597847203501907\\
13.54	0.00597847404766072\\
13.55	0.00597847606107894\\
13.56	0.00597847807527374\\
13.57	0.00597848009024516\\
13.58	0.0059784821059932\\
13.59	0.00597848412251789\\
13.6	0.00597848613981926\\
13.61	0.00597848815789733\\
13.62	0.00597849017675213\\
13.63	0.0059784921963837\\
13.64	0.00597849421679205\\
13.65	0.00597849623797723\\
13.66	0.00597849825993928\\
13.67	0.00597850028267822\\
13.68	0.00597850230619412\\
13.69	0.005978504330487\\
13.7	0.00597850635555692\\
13.71	0.00597850838140394\\
13.72	0.0059785104080281\\
13.73	0.00597851243542948\\
13.74	0.00597851446360812\\
13.75	0.0059785164925641\\
13.76	0.00597851852229749\\
13.77	0.00597852055280837\\
13.78	0.00597852258409681\\
13.79	0.00597852461616291\\
13.8	0.00597852664900675\\
13.81	0.00597852868262844\\
13.82	0.00597853071702807\\
13.83	0.00597853275220575\\
13.84	0.0059785347881616\\
13.85	0.00597853682489573\\
13.86	0.00597853886240828\\
13.87	0.00597854090069937\\
13.88	0.00597854293976915\\
13.89	0.00597854497961777\\
13.9	0.00597854702024537\\
13.91	0.00597854906165213\\
13.92	0.00597855110383821\\
13.93	0.0059785531468038\\
13.94	0.00597855519054907\\
13.95	0.00597855723507424\\
13.96	0.00597855928037949\\
13.97	0.00597856132646506\\
13.98	0.00597856337333115\\
13.99	0.00597856542097801\\
14	0.00597856746940587\\
14.01	0.005978569518615\\
14.02	0.00597857156860565\\
14.03	0.00597857361937811\\
14.04	0.00597857567093265\\
14.05	0.00597857772326959\\
14.06	0.00597857977638921\\
14.07	0.00597858183029186\\
14.08	0.00597858388497785\\
14.09	0.00597858594044754\\
14.1	0.00597858799670129\\
14.11	0.00597859005373946\\
14.12	0.00597859211156244\\
14.13	0.00597859417017063\\
14.14	0.00597859622956445\\
14.15	0.00597859828974431\\
14.16	0.00597860035071065\\
14.17	0.00597860241246394\\
14.18	0.00597860447500464\\
14.19	0.00597860653833324\\
14.2	0.00597860860245023\\
14.21	0.00597861066735613\\
14.22	0.00597861273305146\\
14.23	0.00597861479953679\\
14.24	0.00597861686681266\\
14.25	0.00597861893487966\\
14.26	0.00597862100373839\\
14.27	0.00597862307338944\\
14.28	0.00597862514383347\\
14.29	0.0059786272150711\\
14.3	0.005978629287103\\
14.31	0.00597863135992986\\
14.32	0.00597863343355236\\
14.33	0.00597863550797123\\
14.34	0.00597863758318721\\
14.35	0.00597863965920102\\
14.36	0.00597864173601346\\
14.37	0.0059786438136253\\
14.38	0.00597864589203736\\
14.39	0.00597864797125044\\
14.4	0.0059786500512654\\
14.41	0.00597865213208309\\
14.42	0.0059786542137044\\
14.43	0.0059786562961302\\
14.44	0.00597865837936143\\
14.45	0.00597866046339901\\
14.46	0.00597866254824388\\
14.47	0.00597866463389703\\
14.48	0.00597866672035942\\
14.49	0.00597866880763206\\
14.5	0.00597867089571598\\
14.51	0.00597867298461221\\
14.52	0.00597867507432179\\
14.53	0.00597867716484581\\
14.54	0.00597867925618534\\
14.55	0.0059786813483415\\
14.56	0.00597868344131538\\
14.57	0.00597868553510814\\
14.58	0.00597868762972092\\
14.59	0.00597868972515487\\
14.6	0.00597869182141117\\
14.61	0.00597869391849102\\
14.62	0.00597869601639562\\
14.63	0.00597869811512617\\
14.64	0.0059787002146839\\
14.65	0.00597870231507007\\
14.66	0.0059787044162859\\
14.67	0.00597870651833265\\
14.68	0.0059787086212116\\
14.69	0.00597871072492401\\
14.7	0.00597871282947117\\
14.71	0.00597871493485437\\
14.72	0.00597871704107488\\
14.73	0.00597871914813403\\
14.74	0.0059787212560331\\
14.75	0.0059787233647734\\
14.76	0.00597872547435624\\
14.77	0.00597872758478291\\
14.78	0.00597872969605473\\
14.79	0.005978731808173\\
14.8	0.00597873392113902\\
14.81	0.00597873603495408\\
14.82	0.00597873814961948\\
14.83	0.0059787402651365\\
14.84	0.00597874238150642\\
14.85	0.0059787444987305\\
14.86	0.00597874661681\\
14.87	0.00597874873574617\\
14.88	0.00597875085554024\\
14.89	0.00597875297619341\\
14.9	0.00597875509770691\\
14.91	0.0059787572200819\\
14.92	0.00597875934331957\\
14.93	0.00597876146742103\\
14.94	0.00597876359238744\\
14.95	0.00597876571821988\\
14.96	0.00597876784491944\\
14.97	0.00597876997248717\\
14.98	0.00597877210092409\\
14.99	0.0059787742302312\\
15	0.00597877636040949\\
15.01	0.00597877849145988\\
15.02	0.0059787806233833\\
15.03	0.00597878275618062\\
15.04	0.00597878488985269\\
15.05	0.00597878702440034\\
15.06	0.00597878915982434\\
15.07	0.00597879129612546\\
15.08	0.00597879343330441\\
15.09	0.00597879557136188\\
15.1	0.00597879771029853\\
15.11	0.00597879985011498\\
15.12	0.00597880199081183\\
15.13	0.00597880413238964\\
15.14	0.00597880627484894\\
15.15	0.00597880841819027\\
15.16	0.00597881056241409\\
15.17	0.00597881270752088\\
15.18	0.00597881485351108\\
15.19	0.00597881700038514\\
15.2	0.00597881914814348\\
15.21	0.00597882129678651\\
15.22	0.00597882344631466\\
15.23	0.00597882559672835\\
15.24	0.00597882774802799\\
15.25	0.00597882990021401\\
15.26	0.00597883205328682\\
15.27	0.00597883420724685\\
15.28	0.00597883636209451\\
15.29	0.00597883851783024\\
15.3	0.00597884067445444\\
15.31	0.00597884283196754\\
15.32	0.00597884499036997\\
15.33	0.00597884714966214\\
15.34	0.00597884930984448\\
15.35	0.00597885147091741\\
15.36	0.00597885363288135\\
15.37	0.00597885579573672\\
15.38	0.00597885795948395\\
15.39	0.00597886012412346\\
15.4	0.00597886228965567\\
15.41	0.00597886445608102\\
15.42	0.00597886662339991\\
15.43	0.00597886879161278\\
15.44	0.00597887096072005\\
15.45	0.00597887313072215\\
15.46	0.00597887530161951\\
15.47	0.00597887747341254\\
15.48	0.00597887964610166\\
15.49	0.00597888181968732\\
15.5	0.00597888399416993\\
15.51	0.00597888616954992\\
15.52	0.00597888834582772\\
15.53	0.00597889052300375\\
15.54	0.00597889270107844\\
15.55	0.00597889488005222\\
15.56	0.00597889705992552\\
15.57	0.00597889924069875\\
15.58	0.00597890142237236\\
15.59	0.00597890360494677\\
15.6	0.0059789057884224\\
15.61	0.00597890797279969\\
15.62	0.00597891015807906\\
15.63	0.00597891234426095\\
15.64	0.00597891453134578\\
15.65	0.00597891671933399\\
15.66	0.005978918908226\\
15.67	0.00597892109802224\\
15.68	0.00597892328872314\\
15.69	0.00597892548032914\\
15.7	0.00597892767284067\\
15.71	0.00597892986625815\\
15.72	0.00597893206058202\\
15.73	0.00597893425581271\\
15.74	0.00597893645195065\\
15.75	0.00597893864899627\\
15.76	0.00597894084695001\\
15.77	0.0059789430458123\\
15.78	0.00597894524558357\\
15.79	0.00597894744626425\\
15.8	0.00597894964785478\\
15.81	0.00597895185035559\\
15.82	0.00597895405376711\\
15.83	0.00597895625808979\\
15.84	0.00597895846332405\\
15.85	0.00597896066947032\\
15.86	0.00597896287652905\\
15.87	0.00597896508450067\\
15.88	0.00597896729338561\\
15.89	0.00597896950318431\\
15.9	0.0059789717138972\\
15.91	0.00597897392552473\\
15.92	0.00597897613806732\\
15.93	0.00597897835152542\\
15.94	0.00597898056589945\\
15.95	0.00597898278118986\\
15.96	0.00597898499739709\\
15.97	0.00597898721452157\\
15.98	0.00597898943256374\\
15.99	0.00597899165152403\\
16	0.0059789938714029\\
16.01	0.00597899609220077\\
16.02	0.00597899831391808\\
16.03	0.00597900053655527\\
16.04	0.00597900276011279\\
16.05	0.00597900498459107\\
16.06	0.00597900720999055\\
16.07	0.00597900943631167\\
16.08	0.00597901166355487\\
16.09	0.0059790138917206\\
16.1	0.00597901612080929\\
16.11	0.00597901835082138\\
16.12	0.00597902058175732\\
16.13	0.00597902281361754\\
16.14	0.00597902504640249\\
16.15	0.00597902728011262\\
16.16	0.00597902951474836\\
16.17	0.00597903175031014\\
16.18	0.00597903398679844\\
16.19	0.00597903622421367\\
16.2	0.00597903846255629\\
16.21	0.00597904070182673\\
16.22	0.00597904294202545\\
16.23	0.00597904518315288\\
16.24	0.00597904742520948\\
16.25	0.00597904966819568\\
16.26	0.00597905191211193\\
16.27	0.00597905415695867\\
16.28	0.00597905640273636\\
16.29	0.00597905864944543\\
16.3	0.00597906089708633\\
16.31	0.00597906314565951\\
16.32	0.00597906539516541\\
16.33	0.00597906764560449\\
16.34	0.00597906989697718\\
16.35	0.00597907214928394\\
16.36	0.00597907440252521\\
16.37	0.00597907665670143\\
16.38	0.00597907891181308\\
16.39	0.00597908116786057\\
16.4	0.00597908342484436\\
16.41	0.00597908568276491\\
16.42	0.00597908794162266\\
16.43	0.00597909020141806\\
16.44	0.00597909246215156\\
16.45	0.00597909472382361\\
16.46	0.00597909698643466\\
16.47	0.00597909924998516\\
16.48	0.00597910151447556\\
16.49	0.00597910377990631\\
16.5	0.00597910604627786\\
16.51	0.00597910831359066\\
16.52	0.00597911058184517\\
16.53	0.00597911285104184\\
16.54	0.00597911512118111\\
16.55	0.00597911739226344\\
16.56	0.00597911966428929\\
16.57	0.0059791219372591\\
16.58	0.00597912421117333\\
16.59	0.00597912648603243\\
16.6	0.00597912876183686\\
16.61	0.00597913103858706\\
16.62	0.0059791333162835\\
16.63	0.00597913559492662\\
16.64	0.00597913787451689\\
16.65	0.00597914015505475\\
16.66	0.00597914243654067\\
16.67	0.00597914471897509\\
16.68	0.00597914700235847\\
16.69	0.00597914928669127\\
16.7	0.00597915157197394\\
16.71	0.00597915385820695\\
16.72	0.00597915614539075\\
16.73	0.00597915843352578\\
16.74	0.00597916072261252\\
16.75	0.00597916301265142\\
16.76	0.00597916530364293\\
16.77	0.00597916759558752\\
16.78	0.00597916988848563\\
16.79	0.00597917218233774\\
16.8	0.0059791744771443\\
16.81	0.00597917677290576\\
16.82	0.0059791790696226\\
16.83	0.00597918136729526\\
16.84	0.0059791836659242\\
16.85	0.00597918596550989\\
16.86	0.00597918826605279\\
16.87	0.00597919056755336\\
16.88	0.00597919287001205\\
16.89	0.00597919517342934\\
16.9	0.00597919747780567\\
16.91	0.00597919978314151\\
16.92	0.00597920208943734\\
16.93	0.00597920439669359\\
16.94	0.00597920670491074\\
16.95	0.00597920901408925\\
16.96	0.00597921132422959\\
16.97	0.00597921363533221\\
16.98	0.00597921594739758\\
16.99	0.00597921826042617\\
17	0.00597922057441842\\
17.01	0.00597922288937483\\
17.02	0.00597922520529584\\
17.03	0.00597922752218192\\
17.04	0.00597922984003354\\
17.05	0.00597923215885116\\
17.06	0.00597923447863524\\
17.07	0.00597923679938626\\
17.08	0.00597923912110467\\
17.09	0.00597924144379094\\
17.1	0.00597924376744555\\
17.11	0.00597924609206896\\
17.12	0.00597924841766163\\
17.13	0.00597925074422403\\
17.14	0.00597925307175663\\
17.15	0.0059792554002599\\
17.16	0.00597925772973431\\
17.17	0.00597926006018032\\
17.18	0.0059792623915984\\
17.19	0.00597926472398902\\
17.2	0.00597926705735266\\
17.21	0.00597926939168977\\
17.22	0.00597927172700083\\
17.23	0.00597927406328632\\
17.24	0.00597927640054669\\
17.25	0.00597927873878243\\
17.26	0.00597928107799399\\
17.27	0.00597928341818186\\
17.28	0.0059792857593465\\
17.29	0.00597928810148839\\
17.3	0.005979290444608\\
17.31	0.00597929278870579\\
17.32	0.00597929513378225\\
17.33	0.00597929747983784\\
17.34	0.00597929982687303\\
17.35	0.00597930217488831\\
17.36	0.00597930452388414\\
17.37	0.005979306873861\\
17.38	0.00597930922481937\\
17.39	0.0059793115767597\\
17.4	0.00597931392968249\\
17.41	0.00597931628358821\\
17.42	0.00597931863847732\\
17.43	0.00597932099435031\\
17.44	0.00597932335120765\\
17.45	0.00597932570904982\\
17.46	0.00597932806787729\\
17.47	0.00597933042769054\\
17.48	0.00597933278849004\\
17.49	0.00597933515027628\\
17.5	0.00597933751304973\\
17.51	0.00597933987681087\\
17.52	0.00597934224156017\\
17.53	0.00597934460729812\\
17.54	0.00597934697402519\\
17.55	0.00597934934174186\\
17.56	0.00597935171044861\\
17.57	0.00597935408014591\\
17.58	0.00597935645083426\\
17.59	0.00597935882251412\\
17.6	0.00597936119518599\\
17.61	0.00597936356885032\\
17.62	0.00597936594350762\\
17.63	0.00597936831915836\\
17.64	0.00597937069580301\\
17.65	0.00597937307344207\\
17.66	0.00597937545207601\\
17.67	0.00597937783170531\\
17.68	0.00597938021233046\\
17.69	0.00597938259395194\\
17.7	0.00597938497657023\\
17.71	0.00597938736018581\\
17.72	0.00597938974479918\\
17.73	0.0059793921304108\\
17.74	0.00597939451702117\\
17.75	0.00597939690463077\\
17.76	0.00597939929324008\\
17.77	0.00597940168284959\\
17.78	0.00597940407345978\\
17.79	0.00597940646507114\\
17.8	0.00597940885768415\\
17.81	0.0059794112512993\\
17.82	0.00597941364591708\\
17.83	0.00597941604153797\\
17.84	0.00597941843816245\\
17.85	0.00597942083579101\\
17.86	0.00597942323442415\\
17.87	0.00597942563406235\\
17.88	0.00597942803470609\\
17.89	0.00597943043635587\\
17.9	0.00597943283901217\\
17.91	0.00597943524267547\\
17.92	0.00597943764734628\\
17.93	0.00597944005302507\\
17.94	0.00597944245971235\\
17.95	0.00597944486740858\\
17.96	0.00597944727611427\\
17.97	0.00597944968582992\\
17.98	0.00597945209655599\\
17.99	0.005979454508293\\
18	0.00597945692104142\\
18.01	0.00597945933480175\\
18.02	0.00597946174957448\\
18.03	0.00597946416536011\\
18.04	0.00597946658215911\\
18.05	0.005979468999972\\
18.06	0.00597947141879925\\
18.07	0.00597947383864137\\
18.08	0.00597947625949884\\
18.09	0.00597947868137215\\
18.1	0.00597948110426182\\
18.11	0.00597948352816831\\
18.12	0.00597948595309213\\
18.13	0.00597948837903379\\
18.14	0.00597949080599375\\
18.15	0.00597949323397253\\
18.16	0.00597949566297063\\
18.17	0.00597949809298853\\
18.18	0.00597950052402672\\
18.19	0.00597950295608572\\
18.2	0.00597950538916601\\
18.21	0.00597950782326809\\
18.22	0.00597951025839245\\
18.23	0.00597951269453961\\
18.24	0.00597951513171004\\
18.25	0.00597951756990426\\
18.26	0.00597952000912275\\
18.27	0.00597952244936601\\
18.28	0.00597952489063456\\
18.29	0.00597952733292887\\
18.3	0.00597952977624946\\
18.31	0.00597953222059682\\
18.32	0.00597953466597145\\
18.33	0.00597953711237386\\
18.34	0.00597953955980454\\
18.35	0.00597954200826399\\
18.36	0.00597954445775272\\
18.37	0.00597954690827122\\
18.38	0.00597954935982\\
18.39	0.00597955181239955\\
18.4	0.00597955426601039\\
18.41	0.00597955672065301\\
18.42	0.00597955917632791\\
18.43	0.00597956163303561\\
18.44	0.00597956409077659\\
18.45	0.00597956654955137\\
18.46	0.00597956900936045\\
18.47	0.00597957147020433\\
18.48	0.00597957393208352\\
18.49	0.00597957639499852\\
18.5	0.00597957885894984\\
18.51	0.00597958132393797\\
18.52	0.00597958378996344\\
18.53	0.00597958625702673\\
18.54	0.00597958872512837\\
18.55	0.00597959119426885\\
18.56	0.00597959366444868\\
18.57	0.00597959613566837\\
18.58	0.00597959860792842\\
18.59	0.00597960108122935\\
18.6	0.00597960355557166\\
18.61	0.00597960603095585\\
18.62	0.00597960850738244\\
18.63	0.00597961098485194\\
18.64	0.00597961346336485\\
18.65	0.00597961594292168\\
18.66	0.00597961842352294\\
18.67	0.00597962090516915\\
18.68	0.00597962338786081\\
18.69	0.00597962587159843\\
18.7	0.00597962835638252\\
18.71	0.0059796308422136\\
18.72	0.00597963332909216\\
18.73	0.00597963581701874\\
18.74	0.00597963830599384\\
18.75	0.00597964079601796\\
18.76	0.00597964328709162\\
18.77	0.00597964577921534\\
18.78	0.00597964827238963\\
18.79	0.00597965076661499\\
18.8	0.00597965326189195\\
18.81	0.00597965575822102\\
18.82	0.0059796582556027\\
18.83	0.00597966075403752\\
18.84	0.00597966325352599\\
18.85	0.00597966575406863\\
18.86	0.00597966825566594\\
18.87	0.00597967075831845\\
18.88	0.00597967326202668\\
18.89	0.00597967576679112\\
18.9	0.00597967827261231\\
18.91	0.00597968077949076\\
18.92	0.00597968328742699\\
18.93	0.00597968579642151\\
18.94	0.00597968830647484\\
18.95	0.00597969081758749\\
18.96	0.00597969332975999\\
18.97	0.00597969584299286\\
18.98	0.00597969835728661\\
18.99	0.00597970087264177\\
19	0.00597970338905884\\
19.01	0.00597970590653835\\
19.02	0.00597970842508083\\
19.03	0.00597971094468678\\
19.04	0.00597971346535674\\
19.05	0.00597971598709121\\
19.06	0.00597971850989073\\
19.07	0.00597972103375581\\
19.08	0.00597972355868698\\
19.09	0.00597972608468475\\
19.1	0.00597972861174965\\
19.11	0.0059797311398822\\
19.12	0.00597973366908291\\
19.13	0.00597973619935233\\
19.14	0.00597973873069097\\
19.15	0.00597974126309935\\
19.16	0.005979743796578\\
19.17	0.00597974633112743\\
19.18	0.00597974886674818\\
19.19	0.00597975140344077\\
19.2	0.00597975394120572\\
19.21	0.00597975648004356\\
19.22	0.00597975901995482\\
19.23	0.00597976156094002\\
19.24	0.00597976410299968\\
19.25	0.00597976664613433\\
19.26	0.00597976919034451\\
19.27	0.00597977173563073\\
19.28	0.00597977428199353\\
19.29	0.00597977682943343\\
19.3	0.00597977937795096\\
19.31	0.00597978192754665\\
19.32	0.00597978447822102\\
19.33	0.0059797870299746\\
19.34	0.00597978958280793\\
19.35	0.00597979213672153\\
19.36	0.00597979469171593\\
19.37	0.00597979724779167\\
19.38	0.00597979980494927\\
19.39	0.00597980236318926\\
19.4	0.00597980492251218\\
19.41	0.00597980748291855\\
19.42	0.0059798100444089\\
19.43	0.00597981260698377\\
19.44	0.0059798151706437\\
19.45	0.0059798177353892\\
19.46	0.00597982030122081\\
19.47	0.00597982286813908\\
19.48	0.00597982543614452\\
19.49	0.00597982800523768\\
19.5	0.00597983057541909\\
19.51	0.00597983314668928\\
19.52	0.00597983571904878\\
19.53	0.00597983829249813\\
19.54	0.00597984086703787\\
19.55	0.00597984344266852\\
19.56	0.00597984601939064\\
19.57	0.00597984859720474\\
19.58	0.00597985117611137\\
19.59	0.00597985375611107\\
19.6	0.00597985633720436\\
19.61	0.00597985891939179\\
19.62	0.0059798615026739\\
19.63	0.00597986408705122\\
19.64	0.00597986667252429\\
19.65	0.00597986925909365\\
19.66	0.00597987184675983\\
19.67	0.00597987443552338\\
19.68	0.00597987702538483\\
19.69	0.00597987961634473\\
19.7	0.0059798822084036\\
19.71	0.00597988480156201\\
19.72	0.00597988739582047\\
19.73	0.00597988999117954\\
19.74	0.00597989258763975\\
19.75	0.00597989518520165\\
19.76	0.00597989778386577\\
19.77	0.00597990038363266\\
19.78	0.00597990298450287\\
19.79	0.00597990558647692\\
19.8	0.00597990818955537\\
19.81	0.00597991079373876\\
19.82	0.00597991339902763\\
19.83	0.00597991600542253\\
19.84	0.00597991861292399\\
19.85	0.00597992122153257\\
19.86	0.0059799238312488\\
19.87	0.00597992644207324\\
19.88	0.00597992905400642\\
19.89	0.0059799316670489\\
19.9	0.0059799342812012\\
19.91	0.0059799368964639\\
19.92	0.00597993951283752\\
19.93	0.00597994213032262\\
19.94	0.00597994474891975\\
19.95	0.00597994736862944\\
19.96	0.00597994998945224\\
19.97	0.00597995261138872\\
19.98	0.0059799552344394\\
19.99	0.00597995785860485\\
20	0.00597996048388561\\
20.01	0.00597996311028222\\
20.02	0.00597996573779524\\
20.03	0.00597996836642522\\
20.04	0.0059799709961727\\
20.05	0.00597997362703825\\
20.06	0.0059799762590224\\
20.07	0.00597997889212571\\
20.08	0.00597998152634872\\
20.09	0.005979984161692\\
20.1	0.00597998679815608\\
20.11	0.00597998943574153\\
20.12	0.0059799920744489\\
20.13	0.00597999471427873\\
20.14	0.00597999735523158\\
20.15	0.00597999999730801\\
20.16	0.00598000264050857\\
20.17	0.0059800052848338\\
20.18	0.00598000793028427\\
20.19	0.00598001057686053\\
20.2	0.00598001322456313\\
20.21	0.00598001587339263\\
20.22	0.00598001852334959\\
20.23	0.00598002117443455\\
20.24	0.00598002382664808\\
20.25	0.00598002647999073\\
20.26	0.00598002913446306\\
20.27	0.00598003179006562\\
20.28	0.00598003444679897\\
20.29	0.00598003710466367\\
20.3	0.00598003976366028\\
20.31	0.00598004242378935\\
20.32	0.00598004508505144\\
20.33	0.00598004774744711\\
20.34	0.00598005041097693\\
20.35	0.00598005307564144\\
20.36	0.00598005574144121\\
20.37	0.00598005840837679\\
20.38	0.00598006107644876\\
20.39	0.00598006374565766\\
20.4	0.00598006641600407\\
20.41	0.00598006908748852\\
20.42	0.00598007176011161\\
20.43	0.00598007443387387\\
20.44	0.00598007710877588\\
20.45	0.0059800797848182\\
20.46	0.00598008246200138\\
20.47	0.005980085140326\\
20.48	0.00598008781979261\\
20.49	0.00598009050040178\\
20.5	0.00598009318215408\\
20.51	0.00598009586505005\\
20.52	0.00598009854909028\\
20.53	0.00598010123427533\\
20.54	0.00598010392060575\\
20.55	0.00598010660808212\\
20.56	0.005980109296705\\
20.57	0.00598011198647496\\
20.58	0.00598011467739256\\
20.59	0.00598011736945837\\
20.6	0.00598012006267295\\
20.61	0.00598012275703688\\
20.62	0.00598012545255072\\
20.63	0.00598012814921504\\
20.64	0.0059801308470304\\
20.65	0.00598013354599738\\
20.66	0.00598013624611654\\
20.67	0.00598013894738846\\
20.68	0.00598014164981369\\
20.69	0.00598014435339282\\
20.7	0.0059801470581264\\
20.71	0.00598014976401502\\
20.72	0.00598015247105924\\
20.73	0.00598015517925963\\
20.74	0.00598015788861677\\
20.75	0.00598016059913122\\
20.76	0.00598016331080355\\
20.77	0.00598016602363435\\
20.78	0.00598016873762418\\
20.79	0.00598017145277361\\
20.8	0.00598017416908321\\
20.81	0.00598017688655357\\
20.82	0.00598017960518525\\
20.83	0.00598018232497882\\
20.84	0.00598018504593487\\
20.85	0.00598018776805397\\
20.86	0.00598019049133668\\
20.87	0.00598019321578359\\
20.88	0.00598019594139527\\
20.89	0.00598019866817231\\
20.9	0.00598020139611526\\
20.91	0.00598020412522472\\
20.92	0.00598020685550124\\
20.93	0.00598020958694542\\
20.94	0.00598021231955784\\
20.95	0.00598021505333906\\
20.96	0.00598021778828967\\
20.97	0.00598022052441025\\
20.98	0.00598022326170136\\
20.99	0.0059802260001636\\
21	0.00598022873979755\\
21.01	0.00598023148060377\\
21.02	0.00598023422258285\\
21.03	0.00598023696573538\\
21.04	0.00598023971006193\\
21.05	0.00598024245556307\\
21.06	0.00598024520223941\\
21.07	0.0059802479500915\\
21.08	0.00598025069911995\\
21.09	0.00598025344932532\\
21.1	0.0059802562007082\\
21.11	0.00598025895326918\\
21.12	0.00598026170700884\\
21.13	0.00598026446192775\\
21.14	0.00598026721802651\\
21.15	0.0059802699753057\\
21.16	0.0059802727337659\\
21.17	0.0059802754934077\\
21.18	0.00598027825423167\\
21.19	0.00598028101623842\\
21.2	0.00598028377942852\\
21.21	0.00598028654380255\\
21.22	0.00598028930936111\\
21.23	0.00598029207610478\\
21.24	0.00598029484403415\\
21.25	0.00598029761314981\\
21.26	0.00598030038345233\\
21.27	0.00598030315494232\\
21.28	0.00598030592762036\\
21.29	0.00598030870148703\\
21.3	0.00598031147654293\\
21.31	0.00598031425278864\\
21.32	0.00598031703022476\\
21.33	0.00598031980885187\\
21.34	0.00598032258867057\\
21.35	0.00598032536968144\\
21.36	0.00598032815188508\\
21.37	0.00598033093528207\\
21.38	0.00598033371987301\\
21.39	0.00598033650565849\\
21.4	0.00598033929263911\\
21.41	0.00598034208081544\\
21.42	0.00598034487018809\\
21.43	0.00598034766075765\\
21.44	0.00598035045252472\\
21.45	0.00598035324548988\\
21.46	0.00598035603965373\\
21.47	0.00598035883501687\\
21.48	0.00598036163157988\\
21.49	0.00598036442934337\\
21.5	0.00598036722830793\\
21.51	0.00598037002847415\\
21.52	0.00598037282984264\\
21.53	0.00598037563241398\\
21.54	0.00598037843618878\\
21.55	0.00598038124116762\\
21.56	0.00598038404735112\\
21.57	0.00598038685473986\\
21.58	0.00598038966333444\\
21.59	0.00598039247313547\\
21.6	0.00598039528414353\\
21.61	0.00598039809635924\\
21.62	0.00598040090978319\\
21.63	0.00598040372441597\\
21.64	0.00598040654025819\\
21.65	0.00598040935731045\\
21.66	0.00598041217557334\\
21.67	0.00598041499504748\\
21.68	0.00598041781573346\\
21.69	0.00598042063763188\\
21.7	0.00598042346074335\\
21.71	0.00598042628506847\\
21.72	0.00598042911060783\\
21.73	0.00598043193736205\\
21.74	0.00598043476533172\\
21.75	0.00598043759451745\\
21.76	0.00598044042491985\\
21.77	0.00598044325653951\\
21.78	0.00598044608937705\\
21.79	0.00598044892343306\\
21.8	0.00598045175870816\\
21.81	0.00598045459520294\\
21.82	0.00598045743291802\\
21.83	0.005980460271854\\
21.84	0.00598046311201148\\
21.85	0.00598046595339109\\
21.86	0.00598046879599341\\
21.87	0.00598047163981907\\
21.88	0.00598047448486866\\
21.89	0.00598047733114279\\
21.9	0.00598048017864209\\
21.91	0.00598048302736714\\
21.92	0.00598048587731857\\
21.93	0.00598048872849699\\
21.94	0.00598049158090299\\
21.95	0.00598049443453721\\
21.96	0.00598049728940023\\
21.97	0.00598050014549268\\
21.98	0.00598050300281518\\
21.99	0.00598050586136832\\
22	0.00598050872115272\\
22.01	0.005980511582169\\
22.02	0.00598051444441776\\
22.03	0.00598051730789963\\
22.04	0.00598052017261521\\
22.05	0.00598052303856513\\
22.06	0.00598052590574998\\
22.07	0.00598052877417039\\
22.08	0.00598053164382698\\
22.09	0.00598053451472036\\
22.1	0.00598053738685114\\
22.11	0.00598054026021994\\
22.12	0.00598054313482737\\
22.13	0.00598054601067407\\
22.14	0.00598054888776064\\
22.15	0.0059805517660877\\
22.16	0.00598055464565586\\
22.17	0.00598055752646575\\
22.18	0.00598056040851799\\
22.19	0.00598056329181319\\
22.2	0.00598056617635197\\
22.21	0.00598056906213496\\
22.22	0.00598057194916277\\
22.23	0.00598057483743603\\
22.24	0.00598057772695535\\
22.25	0.00598058061772136\\
22.26	0.00598058350973468\\
22.27	0.00598058640299593\\
22.28	0.00598058929750574\\
22.29	0.00598059219326472\\
22.3	0.0059805950902735\\
22.31	0.00598059798853271\\
22.32	0.00598060088804296\\
22.33	0.00598060378880489\\
22.34	0.00598060669081912\\
22.35	0.00598060959408627\\
22.36	0.00598061249860696\\
22.37	0.00598061540438183\\
22.38	0.00598061831141151\\
22.39	0.0059806212196966\\
22.4	0.00598062412923776\\
22.41	0.0059806270400356\\
22.42	0.00598062995209075\\
22.43	0.00598063286540384\\
22.44	0.00598063577997549\\
22.45	0.00598063869580634\\
22.46	0.00598064161289702\\
22.47	0.00598064453124816\\
22.48	0.00598064745086039\\
22.49	0.00598065037173433\\
22.5	0.00598065329387062\\
22.51	0.0059806562172699\\
22.52	0.00598065914193279\\
22.53	0.00598066206785992\\
22.54	0.00598066499505193\\
22.55	0.00598066792350946\\
22.56	0.00598067085323313\\
22.57	0.00598067378422358\\
22.58	0.00598067671648144\\
22.59	0.00598067965000736\\
22.6	0.00598068258480196\\
22.61	0.00598068552086588\\
22.62	0.00598068845819976\\
22.63	0.00598069139680423\\
22.64	0.00598069433667994\\
22.65	0.00598069727782751\\
22.66	0.00598070022024759\\
22.67	0.00598070316394082\\
22.68	0.00598070610890783\\
22.69	0.00598070905514926\\
22.7	0.00598071200266576\\
22.71	0.00598071495145796\\
22.72	0.00598071790152651\\
22.73	0.00598072085287204\\
22.74	0.0059807238054952\\
22.75	0.00598072675939663\\
22.76	0.00598072971457697\\
22.77	0.00598073267103687\\
22.78	0.00598073562877696\\
22.79	0.0059807385877979\\
22.8	0.00598074154810032\\
22.81	0.00598074450968488\\
22.82	0.00598074747255222\\
22.83	0.00598075043670297\\
22.84	0.0059807534021378\\
22.85	0.00598075636885734\\
22.86	0.00598075933686225\\
22.87	0.00598076230615317\\
22.88	0.00598076527673074\\
22.89	0.00598076824859563\\
22.9	0.00598077122174847\\
22.91	0.00598077419618992\\
22.92	0.00598077717192062\\
22.93	0.00598078014894124\\
22.94	0.00598078312725241\\
22.95	0.0059807861068548\\
22.96	0.00598078908774905\\
22.97	0.00598079206993581\\
22.98	0.00598079505341575\\
22.99	0.00598079803818951\\
23	0.00598080102425775\\
23.01	0.00598080401162112\\
23.02	0.00598080700028028\\
23.03	0.00598080999023588\\
23.04	0.00598081298148859\\
23.05	0.00598081597403905\\
23.06	0.00598081896788792\\
23.07	0.00598082196303588\\
23.08	0.00598082495948356\\
23.09	0.00598082795723164\\
23.1	0.00598083095628076\\
23.11	0.0059808339566316\\
23.12	0.00598083695828481\\
23.13	0.00598083996124106\\
23.14	0.00598084296550101\\
23.15	0.00598084597106531\\
23.16	0.00598084897793463\\
23.17	0.00598085198610965\\
23.18	0.00598085499559101\\
23.19	0.00598085800637939\\
23.2	0.00598086101847545\\
23.21	0.00598086403187986\\
23.22	0.00598086704659329\\
23.23	0.00598087006261639\\
23.24	0.00598087307994986\\
23.25	0.00598087609859434\\
23.26	0.00598087911855051\\
23.27	0.00598088213981904\\
23.28	0.0059808851624006\\
23.29	0.00598088818629586\\
23.3	0.00598089121150549\\
23.31	0.00598089423803018\\
23.32	0.00598089726587057\\
23.33	0.00598090029502737\\
23.34	0.00598090332550124\\
23.35	0.00598090635729284\\
23.36	0.00598090939040287\\
23.37	0.005980912424832\\
23.38	0.0059809154605809\\
23.39	0.00598091849765025\\
23.4	0.00598092153604074\\
23.41	0.00598092457575304\\
23.42	0.00598092761678782\\
23.43	0.00598093065914579\\
23.44	0.0059809337028276\\
23.45	0.00598093674783396\\
23.46	0.00598093979416554\\
23.47	0.00598094284182302\\
23.48	0.00598094589080709\\
23.49	0.00598094894111844\\
23.5	0.00598095199275775\\
23.51	0.00598095504572571\\
23.52	0.00598095810002301\\
23.53	0.00598096115565033\\
23.54	0.00598096421260837\\
23.55	0.00598096727089782\\
23.56	0.00598097033051937\\
23.57	0.0059809733914737\\
23.58	0.00598097645376152\\
23.59	0.00598097951738351\\
23.6	0.00598098258234038\\
23.61	0.00598098564863282\\
23.62	0.00598098871626152\\
23.63	0.00598099178522717\\
23.64	0.00598099485553049\\
23.65	0.00598099792717218\\
23.66	0.00598100100015291\\
23.67	0.00598100407447341\\
23.68	0.00598100715013437\\
23.69	0.0059810102271365\\
23.7	0.00598101330548049\\
23.71	0.00598101638516706\\
23.72	0.00598101946619691\\
23.73	0.00598102254857074\\
23.74	0.00598102563228927\\
23.75	0.00598102871735321\\
23.76	0.00598103180376325\\
23.77	0.00598103489152012\\
23.78	0.00598103798062453\\
23.79	0.00598104107107718\\
23.8	0.0059810441628788\\
23.81	0.0059810472560301\\
23.82	0.00598105035053178\\
23.83	0.00598105344638459\\
23.84	0.00598105654358922\\
23.85	0.0059810596421464\\
23.86	0.00598106274205686\\
23.87	0.0059810658433213\\
23.88	0.00598106894594046\\
23.89	0.00598107204991506\\
23.9	0.00598107515524583\\
23.91	0.00598107826193349\\
23.92	0.00598108136997876\\
23.93	0.00598108447938239\\
23.94	0.0059810875901451\\
23.95	0.00598109070226761\\
23.96	0.00598109381575067\\
23.97	0.00598109693059501\\
23.98	0.00598110004680136\\
23.99	0.00598110316437046\\
24	0.00598110628330305\\
24.01	0.00598110940359987\\
24.02	0.00598111252526166\\
24.03	0.00598111564828916\\
24.04	0.00598111877268311\\
24.05	0.00598112189844426\\
24.06	0.00598112502557337\\
24.07	0.00598112815407116\\
24.08	0.0059811312839384\\
24.09	0.00598113441517584\\
24.1	0.00598113754778423\\
24.11	0.00598114068176431\\
24.12	0.00598114381711686\\
24.13	0.00598114695384262\\
24.14	0.00598115009194236\\
24.15	0.00598115323141683\\
24.16	0.0059811563722668\\
24.17	0.00598115951449303\\
24.18	0.00598116265809628\\
24.19	0.00598116580307734\\
24.2	0.00598116894943695\\
24.21	0.00598117209717589\\
24.22	0.00598117524629494\\
24.23	0.00598117839679487\\
24.24	0.00598118154867646\\
24.25	0.00598118470194048\\
24.26	0.00598118785658772\\
24.27	0.00598119101261895\\
24.28	0.00598119417003496\\
24.29	0.00598119732883653\\
24.3	0.00598120048902445\\
24.31	0.00598120365059952\\
24.32	0.00598120681356252\\
24.33	0.00598120997791425\\
24.34	0.0059812131436555\\
24.35	0.00598121631078706\\
24.36	0.00598121947930975\\
24.37	0.00598122264922436\\
24.38	0.0059812258205317\\
24.39	0.00598122899323257\\
24.4	0.00598123216732778\\
24.41	0.00598123534281814\\
24.42	0.00598123851970447\\
24.43	0.00598124169798758\\
24.44	0.00598124487766828\\
24.45	0.00598124805874741\\
24.46	0.00598125124122578\\
24.47	0.00598125442510422\\
24.48	0.00598125761038355\\
24.49	0.00598126079706461\\
24.5	0.00598126398514823\\
24.51	0.00598126717463524\\
24.52	0.00598127036552649\\
24.53	0.00598127355782281\\
24.54	0.00598127675152504\\
24.55	0.00598127994663404\\
24.56	0.00598128314315066\\
24.57	0.00598128634107574\\
24.58	0.00598128954041014\\
24.59	0.00598129274115471\\
24.6	0.00598129594331033\\
24.61	0.00598129914687785\\
24.62	0.00598130235185814\\
24.63	0.00598130555825207\\
24.64	0.00598130876606051\\
24.65	0.00598131197528434\\
24.66	0.00598131518592443\\
24.67	0.00598131839798169\\
24.68	0.00598132161145698\\
24.69	0.00598132482635119\\
24.7	0.00598132804266523\\
24.71	0.00598133126039998\\
24.72	0.00598133447955635\\
24.73	0.00598133770013524\\
24.74	0.00598134092213755\\
24.75	0.00598134414556421\\
24.76	0.00598134737041612\\
24.77	0.00598135059669419\\
24.78	0.00598135382439936\\
24.79	0.00598135705353255\\
24.8	0.00598136028409469\\
24.81	0.00598136351608672\\
24.82	0.00598136674950957\\
24.83	0.00598136998436417\\
24.84	0.00598137322065149\\
24.85	0.00598137645837247\\
24.86	0.00598137969752807\\
24.87	0.00598138293811924\\
24.88	0.00598138618014695\\
24.89	0.00598138942361217\\
24.9	0.00598139266851587\\
24.91	0.00598139591485902\\
24.92	0.00598139916264262\\
24.93	0.00598140241186764\\
24.94	0.00598140566253508\\
24.95	0.00598140891464593\\
24.96	0.00598141216820121\\
24.97	0.0059814154232019\\
24.98	0.00598141867964904\\
24.99	0.00598142193754362\\
25	0.00598142519688668\\
25.01	0.00598142845767925\\
25.02	0.00598143171992236\\
25.03	0.00598143498361703\\
25.04	0.00598143824876433\\
25.05	0.0059814415153653\\
25.06	0.005981444783421\\
25.07	0.00598144805293249\\
25.08	0.00598145132390083\\
25.09	0.0059814545963271\\
25.1	0.00598145787021239\\
25.11	0.00598146114555777\\
25.12	0.00598146442236435\\
25.13	0.0059814677006332\\
25.14	0.00598147098036546\\
25.15	0.00598147426156222\\
25.16	0.00598147754422462\\
25.17	0.00598148082835376\\
25.18	0.00598148411395079\\
25.19	0.00598148740101684\\
25.2	0.00598149068955306\\
25.21	0.00598149397956062\\
25.22	0.00598149727104065\\
25.23	0.00598150056399435\\
25.24	0.00598150385842288\\
25.25	0.00598150715432743\\
25.26	0.00598151045170919\\
25.27	0.00598151375056936\\
25.28	0.00598151705090916\\
25.29	0.00598152035272979\\
25.3	0.00598152365603248\\
25.31	0.00598152696081847\\
25.32	0.00598153026708899\\
25.33	0.0059815335748453\\
25.34	0.00598153688408866\\
25.35	0.00598154019482033\\
25.36	0.0059815435070416\\
25.37	0.00598154682075375\\
25.38	0.00598155013595807\\
25.39	0.00598155345265588\\
25.4	0.00598155677084848\\
25.41	0.00598156009053722\\
25.42	0.00598156341172341\\
25.43	0.0059815667344084\\
25.44	0.00598157005859356\\
25.45	0.00598157338428024\\
25.46	0.00598157671146983\\
25.47	0.00598158004016372\\
25.48	0.00598158337036329\\
25.49	0.00598158670206997\\
25.5	0.00598159003528517\\
25.51	0.00598159337001033\\
25.52	0.00598159670624689\\
25.53	0.0059816000439963\\
25.54	0.00598160338326004\\
25.55	0.00598160672403959\\
25.56	0.00598161006633643\\
25.57	0.00598161341015208\\
25.58	0.00598161675548805\\
25.59	0.00598162010234588\\
25.6	0.00598162345072711\\
25.61	0.00598162680063329\\
25.62	0.00598163015206601\\
25.63	0.00598163350502683\\
25.64	0.00598163685951737\\
25.65	0.00598164021553925\\
25.66	0.00598164357309408\\
25.67	0.0059816469321835\\
25.68	0.00598165029280919\\
25.69	0.0059816536549728\\
25.7	0.00598165701867603\\
25.71	0.00598166038392059\\
25.72	0.00598166375070819\\
25.73	0.00598166711904058\\
25.74	0.00598167048891949\\
25.75	0.00598167386034671\\
25.76	0.00598167723332401\\
25.77	0.0059816806078532\\
25.78	0.00598168398393611\\
25.79	0.00598168736157457\\
25.8	0.00598169074077044\\
25.81	0.00598169412152559\\
25.82	0.00598169750384191\\
25.83	0.00598170088772132\\
25.84	0.00598170427316576\\
25.85	0.00598170766017716\\
25.86	0.00598171104875751\\
25.87	0.00598171443890878\\
25.88	0.005981717830633\\
25.89	0.0059817212239322\\
25.9	0.00598172461880842\\
25.91	0.00598172801526375\\
25.92	0.00598173141330027\\
25.93	0.00598173481292011\\
25.94	0.0059817382141254\\
25.95	0.00598174161691832\\
25.96	0.00598174502130104\\
25.97	0.00598174842727577\\
25.98	0.00598175183484476\\
25.99	0.00598175524401025\\
26	0.00598175865477453\\
26.01	0.0059817620671399\\
26.02	0.0059817654811087\\
26.03	0.00598176889668329\\
26.04	0.00598177231386605\\
26.05	0.00598177573265939\\
26.06	0.00598177915306575\\
26.07	0.0059817825750876\\
26.08	0.00598178599872742\\
26.09	0.00598178942398774\\
26.1	0.00598179285087111\\
26.11	0.00598179627938012\\
26.12	0.00598179970951735\\
26.13	0.00598180314128547\\
26.14	0.00598180657468713\\
26.15	0.00598181000972504\\
26.16	0.00598181344640192\\
26.17	0.00598181688472055\\
26.18	0.00598182032468372\\
26.19	0.00598182376629426\\
26.2	0.00598182720955502\\
26.21	0.00598183065446892\\
26.22	0.00598183410103889\\
26.23	0.00598183754926787\\
26.24	0.00598184099915889\\
26.25	0.00598184445071498\\
26.26	0.00598184790393922\\
26.27	0.00598185135883471\\
26.28	0.00598185481540462\\
26.29	0.00598185827365212\\
26.3	0.00598186173358045\\
26.31	0.00598186519519288\\
26.32	0.00598186865849272\\
26.33	0.00598187212348331\\
26.34	0.00598187559016806\\
26.35	0.00598187905855038\\
26.36	0.00598188252863377\\
26.37	0.00598188600042174\\
26.38	0.00598188947391786\\
26.39	0.00598189294912573\\
26.4	0.00598189642604901\\
26.41	0.00598189990469142\\
26.42	0.00598190338505669\\
26.43	0.00598190686714862\\
26.44	0.00598191035097106\\
26.45	0.00598191383652791\\
26.46	0.00598191732382312\\
26.47	0.00598192081286067\\
26.48	0.00598192430364462\\
26.49	0.00598192779617908\\
26.5	0.00598193129046818\\
26.51	0.00598193478651616\\
26.52	0.00598193828432726\\
26.53	0.00598194178390581\\
26.54	0.0059819452852562\\
26.55	0.00598194878838284\\
26.56	0.00598195229329024\\
26.57	0.00598195579998295\\
26.58	0.00598195930846557\\
26.59	0.0059819628187428\\
26.6	0.00598196633081935\\
26.61	0.00598196984470005\\
26.62	0.00598197336038975\\
26.63	0.00598197687789338\\
26.64	0.00598198039721593\\
26.65	0.00598198391836247\\
26.66	0.00598198744133814\\
26.67	0.00598199096614812\\
26.68	0.00598199449279771\\
26.69	0.00598199802129223\\
26.7	0.00598200155163712\\
26.71	0.00598200508383784\\
26.72	0.00598200861789998\\
26.73	0.00598201215382917\\
26.74	0.00598201569163113\\
26.75	0.00598201923131166\\
26.76	0.00598202277287664\\
26.77	0.00598202631633202\\
26.78	0.00598202986168384\\
26.79	0.00598203340893822\\
26.8	0.00598203695810138\\
26.81	0.0059820405091796\\
26.82	0.00598204406217927\\
26.83	0.00598204761710686\\
26.84	0.00598205117396892\\
26.85	0.00598205473277211\\
26.86	0.00598205829352317\\
26.87	0.00598206185622895\\
26.88	0.00598206542089637\\
26.89	0.00598206898753246\\
26.9	0.00598207255614435\\
26.91	0.00598207612673927\\
26.92	0.00598207969932456\\
26.93	0.00598208327390765\\
26.94	0.00598208685049607\\
26.95	0.00598209042909747\\
26.96	0.00598209400971961\\
26.97	0.00598209759237035\\
26.98	0.00598210117705765\\
26.99	0.00598210476378961\\
27	0.00598210835257443\\
27.01	0.00598211194342043\\
27.02	0.00598211553633603\\
27.03	0.0059821191313298\\
27.04	0.00598212272841042\\
27.05	0.00598212632758669\\
27.06	0.00598212992886751\\
27.07	0.00598213353226197\\
27.08	0.00598213713777923\\
27.09	0.00598214074542861\\
27.1	0.00598214435521954\\
27.11	0.00598214796716163\\
27.12	0.00598215158126459\\
27.13	0.00598215519753827\\
27.14	0.00598215881599267\\
27.15	0.00598216243663793\\
27.16	0.00598216605948434\\
27.17	0.00598216968454234\\
27.18	0.00598217331182251\\
27.19	0.00598217694133559\\
27.2	0.00598218057309247\\
27.21	0.00598218420710419\\
27.22	0.00598218784338196\\
27.23	0.00598219148193716\\
27.24	0.00598219512278131\\
27.25	0.00598219876592611\\
27.26	0.00598220241138342\\
27.27	0.00598220605916528\\
27.28	0.00598220970928391\\
27.29	0.00598221336175168\\
27.3	0.00598221701658117\\
27.31	0.00598222067378511\\
27.32	0.00598222433337644\\
27.33	0.00598222799536827\\
27.34	0.00598223165977391\\
27.35	0.00598223532660686\\
27.36	0.00598223899588081\\
27.37	0.00598224266760964\\
27.38	0.00598224634180745\\
27.39	0.00598225001848853\\
27.4	0.00598225369766738\\
27.41	0.00598225737935872\\
27.42	0.00598226106357745\\
27.43	0.00598226475033872\\
27.44	0.00598226843965788\\
27.45	0.00598227213155051\\
27.46	0.00598227582603242\\
27.47	0.00598227952311962\\
27.48	0.00598228322282839\\
27.49	0.00598228692517522\\
27.5	0.00598229063017684\\
27.51	0.00598229433785022\\
27.52	0.00598229804821259\\
27.53	0.00598230176128142\\
27.54	0.00598230547707441\\
27.55	0.00598230919560956\\
27.56	0.00598231291690509\\
27.57	0.0059823166409795\\
27.58	0.00598232036785155\\
27.59	0.00598232409754027\\
27.6	0.00598232783006498\\
27.61	0.00598233156544526\\
27.62	0.00598233530370096\\
27.63	0.00598233904485226\\
27.64	0.00598234278891957\\
27.65	0.00598234653592365\\
27.66	0.00598235028588552\\
27.67	0.00598235403882653\\
27.68	0.00598235779476829\\
27.69	0.00598236155373278\\
27.7	0.00598236531574226\\
27.71	0.00598236908081929\\
27.72	0.00598237284898681\\
27.73	0.00598237662026805\\
27.74	0.00598238039468655\\
27.75	0.00598238417226625\\
27.76	0.00598238795303137\\
27.77	0.00598239173700652\\
27.78	0.00598239552421663\\
27.79	0.00598239931468701\\
27.8	0.00598240310844332\\
27.81	0.00598240690551157\\
27.82	0.00598241070591817\\
27.83	0.00598241450968989\\
27.84	0.00598241831685387\\
27.85	0.00598242212743766\\
27.86	0.00598242594146919\\
27.87	0.00598242975897677\\
27.88	0.00598243357998914\\
27.89	0.00598243740453543\\
27.9	0.00598244123264519\\
27.91	0.00598244506434837\\
27.92	0.00598244889967538\\
27.93	0.00598245273865701\\
27.94	0.00598245658132454\\
27.95	0.00598246042770965\\
27.96	0.00598246427784448\\
27.97	0.00598246813176162\\
27.98	0.00598247198949413\\
27.99	0.00598247585107552\\
28	0.00598247971653978\\
28.01	0.00598248358592138\\
28.02	0.00598248745925527\\
28.03	0.00598249133657688\\
28.04	0.00598249521792216\\
28.05	0.00598249910332753\\
28.06	0.00598250299282997\\
28.07	0.00598250688646692\\
28.08	0.00598251078427638\\
28.09	0.00598251468629687\\
28.1	0.00598251859256745\\
28.11	0.00598252250312772\\
28.12	0.00598252641801785\\
28.13	0.00598253033727853\\
28.14	0.00598253426095106\\
28.15	0.00598253818907728\\
28.16	0.00598254212169964\\
28.17	0.00598254605886116\\
28.18	0.00598255000060547\\
28.19	0.00598255394697679\\
28.2	0.00598255789801997\\
28.21	0.00598256185378046\\
28.22	0.00598256581430437\\
28.23	0.00598256977963842\\
28.24	0.00598257374983\\
28.25	0.00598257772492712\\
28.26	0.0059825817049785\\
28.27	0.00598258569003349\\
28.28	0.00598258968014216\\
28.29	0.00598259367535523\\
28.3	0.00598259767572416\\
28.31	0.00598260168130109\\
28.32	0.0059826056921389\\
28.33	0.00598260970829118\\
28.34	0.00598261372981228\\
28.35	0.00598261775675727\\
28.36	0.00598262178918201\\
28.37	0.00598262582714312\\
28.38	0.00598262987069798\\
28.39	0.00598263391990479\\
28.4	0.00598263797482254\\
28.41	0.00598264203551102\\
28.42	0.00598264610203086\\
28.43	0.00598265017444353\\
28.44	0.00598265425281134\\
28.45	0.00598265833719744\\
28.46	0.0059826624276659\\
28.47	0.00598266652428162\\
28.48	0.00598267062711044\\
28.49	0.00598267473621909\\
28.5	0.00598267885167522\\
28.51	0.00598268297354744\\
28.52	0.00598268710190527\\
28.53	0.00598269123681923\\
28.54	0.00598269537836082\\
28.55	0.00598269952660251\\
28.56	0.0059827036816178\\
28.57	0.0059827078434812\\
28.58	0.00598271201226827\\
28.59	0.00598271618805562\\
28.6	0.00598272037092094\\
28.61	0.00598272456094301\\
28.62	0.00598272875820171\\
28.63	0.00598273296277805\\
28.64	0.00598273717475419\\
28.65	0.00598274139421344\\
28.66	0.00598274562124029\\
28.67	0.00598274985592045\\
28.68	0.00598275409834084\\
28.69	0.00598275834858962\\
28.7	0.00598276260675621\\
28.71	0.00598276687293134\\
28.72	0.00598277114720701\\
28.73	0.00598277542967659\\
28.74	0.00598277972043478\\
28.75	0.00598278401957768\\
28.76	0.00598278832720277\\
28.77	0.005982792643409\\
28.78	0.00598279696829675\\
28.79	0.00598280130196789\\
28.8	0.00598280564452581\\
28.81	0.00598280999607546\\
28.82	0.00598281435672334\\
28.83	0.00598281872657759\\
28.84	0.00598282310574796\\
28.85	0.00598282749434588\\
28.86	0.00598283189248452\\
28.87	0.00598283630027876\\
28.88	0.00598284071784526\\
28.89	0.00598284514530255\\
28.9	0.00598284958277096\\
28.91	0.00598285403037277\\
28.92	0.00598285848823217\\
28.93	0.00598286295647537\\
28.94	0.00598286743523059\\
28.95	0.00598287192462815\\
28.96	0.00598287642480051\\
28.97	0.00598288093588228\\
28.98	0.00598288545801034\\
28.99	0.00598288999132385\\
29	0.00598289453596431\\
29.01	0.00598289909207564\\
29.02	0.00598290365980423\\
29.03	0.005982908239299\\
29.04	0.00598291283071144\\
29.05	0.00598291743419577\\
29.06	0.00598292204990889\\
29.07	0.00598292667801053\\
29.08	0.00598293131866333\\
29.09	0.00598293597203288\\
29.1	0.00598294063828783\\
29.11	0.00598294531759996\\
29.12	0.00598295001014429\\
29.13	0.00598295471609916\\
29.14	0.00598295943564633\\
29.15	0.00598296416897106\\
29.16	0.00598296891626227\\
29.17	0.00598297367771259\\
29.18	0.00598297845351851\\
29.19	0.00598298324388048\\
29.2	0.00598298804900303\\
29.21	0.00598299286909493\\
29.22	0.00598299770436928\\
29.23	0.00598300255504367\\
29.24	0.00598300742134033\\
29.25	0.00598301230348628\\
29.26	0.00598301720171345\\
29.27	0.0059830221162589\\
29.28	0.00598302704736495\\
29.29	0.00598303199527937\\
29.3	0.00598303696025556\\
29.31	0.00598304194255272\\
29.32	0.00598304694243611\\
29.33	0.00598305196017719\\
29.34	0.00598305699605386\\
29.35	0.00598306205035072\\
29.36	0.00598306712335922\\
29.37	0.00598307221537801\\
29.38	0.0059830773267131\\
29.39	0.00598308245767817\\
29.4	0.00598308760859486\\
29.41	0.00598309277979301\\
29.42	0.00598309797161099\\
29.43	0.00598310318439603\\
29.44	0.00598310841850447\\
29.45	0.00598311367430218\\
29.46	0.00598311895216487\\
29.47	0.00598312425247847\\
29.48	0.00598312957563948\\
29.49	0.00598313492205542\\
29.5	0.00598314029214519\\
29.51	0.00598314568633955\\
29.52	0.00598315110508151\\
29.53	0.00598315654882685\\
29.54	0.00598316201804461\\
29.55	0.00598316751321752\\
29.56	0.00598317303483954\\
29.57	0.00598317858341554\\
29.58	0.0059831841594616\\
29.59	0.00598318976350532\\
29.6	0.00598319539608602\\
29.61	0.00598320105775508\\
29.62	0.00598320674907619\\
29.63	0.00598321247062568\\
29.64	0.00598321822299284\\
29.65	0.00598322400678016\\
29.66	0.00598322982260373\\
29.67	0.00598323567109353\\
29.68	0.00598324155289379\\
29.69	0.00598324746866331\\
29.7	0.00598325341907583\\
29.71	0.00598325940482041\\
29.72	0.00598326542660178\\
29.73	0.00598327148514074\\
29.74	0.00598327758117454\\
29.75	0.00598328371545731\\
29.76	0.00598328988876043\\
29.77	0.00598329610187303\\
29.78	0.00598330235560235\\
29.79	0.00598330865077422\\
29.8	0.00598331498823354\\
29.81	0.00598332136884472\\
29.82	0.00598332779349218\\
29.83	0.00598333426308085\\
29.84	0.00598334077853667\\
29.85	0.00598334734080712\\
29.86	0.00598335395086175\\
29.87	0.00598336060969276\\
29.88	0.00598336731831552\\
29.89	0.00598337407776919\\
29.9	0.00598338088911731\\
29.91	0.00598338775344837\\
29.92	0.0059833946718765\\
29.93	0.0059834016455421\\
29.94	0.00598340867561245\\
29.95	0.00598341576328247\\
29.96	0.00598342290977534\\
29.97	0.00598343011634328\\
29.98	0.00598343738426826\\
29.99	0.00598344471486272\\
30	0.0059834521094704\\
30.01	0.00598345956946713\\
30.02	0.0059834670962616\\
30.03	0.00598347469129623\\
30.04	0.00598348235604806\\
30.05	0.00598349009202957\\
30.06	0.00598349790078964\\
30.07	0.00598350578391445\\
30.08	0.00598351374302845\\
30.09	0.00598352177979536\\
30.1	0.00598352989591914\\
30.11	0.00598353809314506\\
30.12	0.00598354637326074\\
30.13	0.00598355473809727\\
30.14	0.00598356318953031\\
30.15	0.00598357172948123\\
30.16	0.00598358035991833\\
30.17	0.00598358908285804\\
30.18	0.00598359790036616\\
30.19	0.00598360681455915\\
30.2	0.00598361582760543\\
30.21	0.00598362494172676\\
30.22	0.00598363415919962\\
30.23	0.00598364348235662\\
30.24	0.00598365291358797\\
30.25	0.00598366245534304\\
30.26	0.00598367211013181\\
30.27	0.00598368188052652\\
30.28	0.0059836917691633\\
30.29	0.00598370177874383\\
30.3	0.00598371191203708\\
30.31	0.00598372217188104\\
30.32	0.0059837325611846\\
30.33	0.00598374308292936\\
30.34	0.00598375374017157\\
30.35	0.0059837645360441\\
30.36	0.00598377547375846\\
30.37	0.00598378655660691\\
30.38	0.00598379778796455\\
30.39	0.00598380917129153\\
30.4	0.00598382071013533\\
30.41	0.00598383240813305\\
30.42	0.00598384426901383\\
30.43	0.00598385629660123\\
30.44	0.00598386849481583\\
30.45	0.00598388086767772\\
30.46	0.00598389341930925\\
30.47	0.00598390615393769\\
30.48	0.00598391907589807\\
30.49	0.00598393218963605\\
30.5	0.00598394549971086\\
30.51	0.00598395901079839\\
30.52	0.00598397272769429\\
30.53	0.00598398665531717\\
30.54	0.00598400079871192\\
30.55	0.00598401516305312\\
30.56	0.0059840297536485\\
30.57	0.00598404457594253\\
30.58	0.00598405963552015\\
30.59	0.00598407493811046\\
30.6	0.00598409048959074\\
30.61	0.00598410629599034\\
30.62	0.00598412236349485\\
30.63	0.0059841385742652\\
30.64	0.00598415479126208\\
30.65	0.00598417101448844\\
30.66	0.00598418724394727\\
30.67	0.00598420347964156\\
30.68	0.00598421972157427\\
30.69	0.0059842359697484\\
30.7	0.00598425222416693\\
30.71	0.00598426848483284\\
30.72	0.00598428475174914\\
30.73	0.0059843010249188\\
30.74	0.00598431730434482\\
30.75	0.00598433359003021\\
30.76	0.00598434988197796\\
30.77	0.00598436618019106\\
30.78	0.00598438248467253\\
30.79	0.00598439879542536\\
30.8	0.00598441511245256\\
30.81	0.00598443143575715\\
30.82	0.00598444776534213\\
30.83	0.00598446410121051\\
30.84	0.00598448044336532\\
30.85	0.00598449679180957\\
30.86	0.00598451314654627\\
30.87	0.00598452950757845\\
30.88	0.00598454587490914\\
30.89	0.00598456224854136\\
30.9	0.00598457862847813\\
30.91	0.00598459501472249\\
30.92	0.00598461140727747\\
30.93	0.00598462780614611\\
30.94	0.00598464421133144\\
30.95	0.0059846606228365\\
30.96	0.00598467704066433\\
30.97	0.00598469346481798\\
30.98	0.00598470989530048\\
30.99	0.00598472633211489\\
31	0.00598474277526426\\
31.01	0.00598475922475163\\
31.02	0.00598477568058007\\
31.03	0.00598479214275262\\
31.04	0.00598480861127236\\
31.05	0.00598482508614232\\
31.06	0.00598484156736559\\
31.07	0.00598485805494522\\
31.08	0.00598487454888428\\
31.09	0.00598489104918584\\
31.1	0.00598490755585297\\
31.11	0.00598492406888875\\
31.12	0.00598494058829624\\
31.13	0.00598495711407854\\
31.14	0.00598497364623871\\
31.15	0.00598499018477985\\
31.16	0.00598500672970502\\
31.17	0.00598502328101734\\
31.18	0.00598503983871986\\
31.19	0.00598505640281571\\
31.2	0.00598507297330795\\
31.21	0.00598508955019969\\
31.22	0.00598510613349404\\
31.23	0.00598512272319407\\
31.24	0.00598513931930291\\
31.25	0.00598515592182366\\
31.26	0.00598517253075941\\
31.27	0.00598518914611328\\
31.28	0.00598520576788838\\
31.29	0.00598522239608783\\
31.3	0.00598523903071473\\
31.31	0.00598525567177221\\
31.32	0.00598527231926339\\
31.33	0.00598528897319138\\
31.34	0.00598530563355932\\
31.35	0.00598532230037033\\
31.36	0.00598533897362753\\
31.37	0.00598535565333406\\
31.38	0.00598537233949306\\
31.39	0.00598538903210765\\
31.4	0.00598540573118098\\
31.41	0.00598542243671618\\
31.42	0.00598543914871641\\
31.43	0.00598545586718479\\
31.44	0.00598547259212449\\
31.45	0.00598548932353864\\
31.46	0.00598550606143041\\
31.47	0.00598552280580294\\
31.48	0.00598553955665939\\
31.49	0.00598555631400292\\
31.5	0.00598557307783668\\
31.51	0.00598558984816385\\
31.52	0.00598560662498759\\
31.53	0.00598562340831106\\
31.54	0.00598564019813743\\
31.55	0.00598565699446988\\
31.56	0.00598567379731157\\
31.57	0.0059856906066657\\
31.58	0.00598570742253543\\
31.59	0.00598572424492395\\
31.6	0.00598574107383443\\
31.61	0.00598575790927008\\
31.62	0.00598577475123406\\
31.63	0.00598579159972958\\
31.64	0.00598580845475983\\
31.65	0.00598582531632799\\
31.66	0.00598584218443728\\
31.67	0.00598585905909088\\
31.68	0.00598587594029201\\
31.69	0.00598589282804386\\
31.7	0.00598590972234964\\
31.71	0.00598592662321256\\
31.72	0.00598594353063584\\
31.73	0.00598596044462267\\
31.74	0.00598597736517629\\
31.75	0.00598599429229991\\
31.76	0.00598601122599675\\
31.77	0.00598602816627003\\
31.78	0.00598604511312298\\
31.79	0.00598606206655883\\
31.8	0.0059860790265808\\
31.81	0.00598609599319213\\
31.82	0.00598611296639606\\
31.83	0.00598612994619581\\
31.84	0.00598614693259464\\
31.85	0.00598616392559577\\
31.86	0.00598618092520246\\
31.87	0.00598619793141796\\
31.88	0.0059862149442455\\
31.89	0.00598623196368835\\
31.9	0.00598624898974975\\
31.91	0.00598626602243297\\
31.92	0.00598628306174126\\
31.93	0.00598630010767787\\
31.94	0.00598631716024609\\
31.95	0.00598633421944916\\
31.96	0.00598635128529037\\
31.97	0.00598636835777297\\
31.98	0.00598638543690024\\
31.99	0.00598640252267546\\
32	0.00598641961510191\\
32.01	0.00598643671418286\\
32.02	0.0059864538199216\\
32.03	0.00598647093232141\\
32.04	0.00598648805138558\\
32.05	0.0059865051771174\\
32.06	0.00598652230952016\\
32.07	0.00598653944859716\\
32.08	0.00598655659435168\\
32.09	0.00598657374678705\\
32.1	0.00598659090590654\\
32.11	0.00598660807171347\\
32.12	0.00598662524421115\\
32.13	0.00598664242340289\\
32.14	0.00598665960929199\\
32.15	0.00598667680188177\\
32.16	0.00598669400117554\\
32.17	0.00598671120717663\\
32.18	0.00598672841988836\\
32.19	0.00598674563931405\\
32.2	0.00598676286545703\\
32.21	0.00598678009832062\\
32.22	0.00598679733790815\\
32.23	0.00598681458422297\\
32.24	0.0059868318372684\\
32.25	0.00598684909704779\\
32.26	0.00598686636356449\\
32.27	0.00598688363682181\\
32.28	0.00598690091682313\\
32.29	0.00598691820357178\\
32.3	0.00598693549707111\\
32.31	0.00598695279732449\\
32.32	0.00598697010433526\\
32.33	0.00598698741810678\\
32.34	0.00598700473864242\\
32.35	0.00598702206594554\\
32.36	0.0059870394000195\\
32.37	0.00598705674086768\\
32.38	0.00598707408849344\\
32.39	0.00598709144290016\\
32.4	0.00598710880409122\\
32.41	0.00598712617206998\\
32.42	0.00598714354683984\\
32.43	0.00598716092840417\\
32.44	0.00598717831676637\\
32.45	0.00598719571192983\\
32.46	0.00598721311389792\\
32.47	0.00598723052267406\\
32.48	0.00598724793826163\\
32.49	0.00598726536066403\\
32.5	0.00598728278988466\\
32.51	0.00598730022592693\\
32.52	0.00598731766879425\\
32.53	0.00598733511849002\\
32.54	0.00598735257501765\\
32.55	0.00598737003838057\\
32.56	0.00598738750858218\\
32.57	0.0059874049856259\\
32.58	0.00598742246951516\\
32.59	0.00598743996025338\\
32.6	0.00598745745784399\\
32.61	0.00598747496229042\\
32.62	0.0059874924735961\\
32.63	0.00598750999176446\\
32.64	0.00598752751679894\\
32.65	0.00598754504870299\\
32.66	0.00598756258748004\\
32.67	0.00598758013313354\\
32.68	0.00598759768566694\\
32.69	0.00598761524508368\\
32.7	0.00598763281138723\\
32.71	0.00598765038458102\\
32.72	0.00598766796466854\\
32.73	0.00598768555165323\\
32.74	0.00598770314553855\\
32.75	0.00598772074632797\\
32.76	0.00598773835402497\\
32.77	0.00598775596863302\\
32.78	0.00598777359015558\\
32.79	0.00598779121859613\\
32.8	0.00598780885395815\\
32.81	0.00598782649624513\\
32.82	0.00598784414546055\\
32.83	0.00598786180160789\\
32.84	0.00598787946469065\\
32.85	0.00598789713471231\\
32.86	0.00598791481167637\\
32.87	0.00598793249558634\\
32.88	0.0059879501864457\\
32.89	0.00598796788425796\\
32.9	0.00598798558902664\\
32.91	0.00598800330075523\\
32.92	0.00598802101944725\\
32.93	0.0059880387451062\\
32.94	0.00598805647773562\\
32.95	0.005988074217339\\
32.96	0.00598809196391988\\
32.97	0.00598810971748179\\
32.98	0.00598812747802824\\
32.99	0.00598814524556277\\
33	0.00598816302008891\\
33.01	0.0059881808016102\\
33.02	0.00598819859013016\\
33.03	0.00598821638565235\\
33.04	0.0059882341881803\\
33.05	0.00598825199771757\\
33.06	0.00598826981426769\\
33.07	0.00598828763783423\\
33.08	0.00598830546842073\\
33.09	0.00598832330603075\\
33.1	0.00598834115066785\\
33.11	0.00598835900233559\\
33.12	0.00598837686103754\\
33.13	0.00598839472677727\\
33.14	0.00598841259955834\\
33.15	0.00598843047938433\\
33.16	0.00598844836625882\\
33.17	0.00598846626018537\\
33.18	0.00598848416116759\\
33.19	0.00598850206920903\\
33.2	0.00598851998431331\\
33.21	0.00598853790648399\\
33.22	0.00598855583572469\\
33.23	0.00598857377203898\\
33.24	0.00598859171543047\\
33.25	0.00598860966590276\\
33.26	0.00598862762345946\\
33.27	0.00598864558810416\\
33.28	0.00598866355984047\\
33.29	0.00598868153867202\\
33.3	0.00598869952460241\\
33.31	0.00598871751763527\\
33.32	0.0059887355177742\\
33.33	0.00598875352502283\\
33.34	0.0059887715393848\\
33.35	0.00598878956086372\\
33.36	0.00598880758946323\\
33.37	0.00598882562518697\\
33.38	0.00598884366803856\\
33.39	0.00598886171802165\\
33.4	0.00598887977513989\\
33.41	0.0059888978393969\\
33.42	0.00598891591079636\\
33.43	0.0059889339893419\\
33.44	0.00598895207503718\\
33.45	0.00598897016788586\\
33.46	0.00598898826789159\\
33.47	0.00598900637505803\\
33.48	0.00598902448938887\\
33.49	0.00598904261088775\\
33.5	0.00598906073955835\\
33.51	0.00598907887540435\\
33.52	0.00598909701842942\\
33.53	0.00598911516863724\\
33.54	0.00598913332603149\\
33.55	0.00598915149061586\\
33.56	0.00598916966239404\\
33.57	0.00598918784136972\\
33.58	0.00598920602754658\\
33.59	0.00598922422092833\\
33.6	0.00598924242151867\\
33.61	0.0059892606293213\\
33.62	0.00598927884433992\\
33.63	0.00598929706657824\\
33.64	0.00598931529603998\\
33.65	0.00598933353272884\\
33.66	0.00598935177664855\\
33.67	0.00598937002780282\\
33.68	0.00598938828619538\\
33.69	0.00598940655182995\\
33.7	0.00598942482471026\\
33.71	0.00598944310484004\\
33.72	0.00598946139222303\\
33.73	0.00598947968686296\\
33.74	0.00598949798876357\\
33.75	0.00598951629792861\\
33.76	0.00598953461436182\\
33.77	0.00598955293806696\\
33.78	0.00598957126904777\\
33.79	0.005989589607308\\
33.8	0.00598960795285143\\
33.81	0.00598962630568179\\
33.82	0.00598964466580288\\
33.83	0.00598966303321843\\
33.84	0.00598968140793224\\
33.85	0.00598969978994806\\
33.86	0.00598971817926968\\
33.87	0.00598973657590087\\
33.88	0.00598975497984542\\
33.89	0.00598977339110711\\
33.9	0.00598979180968972\\
33.91	0.00598981023559706\\
33.92	0.0059898286688329\\
33.93	0.00598984710940105\\
33.94	0.00598986555730531\\
33.95	0.00598988401254948\\
33.96	0.00598990247513737\\
33.97	0.00598992094507278\\
33.98	0.00598993942235952\\
33.99	0.00598995790700141\\
34	0.00598997639900227\\
34.01	0.00598999489836592\\
34.02	0.00599001340509618\\
34.03	0.00599003191919688\\
34.04	0.00599005044067185\\
34.05	0.00599006896952491\\
34.06	0.00599008750575992\\
34.07	0.0059901060493807\\
34.08	0.00599012460039109\\
34.09	0.00599014315879495\\
34.1	0.00599016172459612\\
34.11	0.00599018029779844\\
34.12	0.00599019887840579\\
34.13	0.005990217466422\\
34.14	0.00599023606185095\\
34.15	0.00599025466469649\\
34.16	0.0059902732749625\\
34.17	0.00599029189265283\\
34.18	0.00599031051777137\\
34.19	0.005990329150322\\
34.2	0.00599034779030857\\
34.21	0.00599036643773499\\
34.22	0.00599038509260513\\
34.23	0.00599040375492289\\
34.24	0.00599042242469215\\
34.25	0.00599044110191681\\
34.26	0.00599045978660077\\
34.27	0.00599047847874792\\
34.28	0.00599049717836217\\
34.29	0.00599051588544743\\
34.3	0.00599053460000761\\
34.31	0.00599055332204661\\
34.32	0.00599057205156836\\
34.33	0.00599059078857677\\
34.34	0.00599060953307578\\
34.35	0.00599062828506929\\
34.36	0.00599064704456125\\
34.37	0.00599066581155558\\
34.38	0.00599068458605621\\
34.39	0.0059907033680671\\
34.4	0.00599072215759218\\
34.41	0.00599074095463538\\
34.42	0.00599075975920066\\
34.43	0.00599077857129198\\
34.44	0.00599079739091328\\
34.45	0.00599081621806853\\
34.46	0.00599083505276167\\
34.47	0.00599085389499669\\
34.48	0.00599087274477753\\
34.49	0.00599089160210818\\
34.5	0.00599091046699261\\
34.51	0.00599092933943479\\
34.52	0.0059909482194387\\
34.53	0.00599096710700833\\
34.54	0.00599098600214766\\
34.55	0.00599100490486068\\
34.56	0.00599102381515139\\
34.57	0.00599104273302377\\
34.58	0.00599106165848183\\
34.59	0.00599108059152957\\
34.6	0.005991099532171\\
34.61	0.00599111848041012\\
34.62	0.00599113743625096\\
34.63	0.00599115639969751\\
34.64	0.0059911753707538\\
34.65	0.00599119434942386\\
34.66	0.0059912133357117\\
34.67	0.00599123232962135\\
34.68	0.00599125133115686\\
34.69	0.00599127034032225\\
34.7	0.00599128935712156\\
34.71	0.00599130838155883\\
34.72	0.0059913274136381\\
34.73	0.00599134645336344\\
34.74	0.00599136550073887\\
34.75	0.00599138455576847\\
34.76	0.00599140361845629\\
34.77	0.00599142268880638\\
34.78	0.00599144176682282\\
34.79	0.00599146085250967\\
34.8	0.00599147994587101\\
34.81	0.0059914990469109\\
34.82	0.00599151815563344\\
34.83	0.00599153727204268\\
34.84	0.00599155639614273\\
34.85	0.00599157552793767\\
34.86	0.00599159466743159\\
34.87	0.00599161381462858\\
34.88	0.00599163296953274\\
34.89	0.00599165213214818\\
34.9	0.005991671302479\\
34.91	0.00599169048052931\\
34.92	0.00599170966630322\\
34.93	0.00599172885980484\\
34.94	0.00599174806103828\\
34.95	0.00599176727000769\\
34.96	0.00599178648671717\\
34.97	0.00599180571117086\\
34.98	0.00599182494337288\\
34.99	0.00599184418332738\\
35	0.00599186343103849\\
35.01	0.00599188268651035\\
35.02	0.00599190194974711\\
35.03	0.00599192122075293\\
35.04	0.00599194049953194\\
35.05	0.0059919597860883\\
35.06	0.00599197908042618\\
35.07	0.00599199838254974\\
35.08	0.00599201769246313\\
35.09	0.00599203701017054\\
35.1	0.00599205633567614\\
35.11	0.00599207566898409\\
35.12	0.00599209501009859\\
35.13	0.00599211435902381\\
35.14	0.00599213371576394\\
35.15	0.00599215308032318\\
35.16	0.0059921724527057\\
35.17	0.00599219183291572\\
35.18	0.00599221122095743\\
35.19	0.00599223061683504\\
35.2	0.00599225002055275\\
35.21	0.00599226943211477\\
35.22	0.00599228885152532\\
35.23	0.00599230827878862\\
35.24	0.00599232771390889\\
35.25	0.00599234715689035\\
35.26	0.00599236660773723\\
35.27	0.00599238606645376\\
35.28	0.00599240553304419\\
35.29	0.00599242500751274\\
35.3	0.00599244448986367\\
35.31	0.00599246398010121\\
35.32	0.00599248347822961\\
35.33	0.00599250298425314\\
35.34	0.00599252249817605\\
35.35	0.00599254202000259\\
35.36	0.00599256154973703\\
35.37	0.00599258108738364\\
35.38	0.00599260063294669\\
35.39	0.00599262018643046\\
35.4	0.00599263974783922\\
35.41	0.00599265931717725\\
35.42	0.00599267889444884\\
35.43	0.00599269847965827\\
35.44	0.00599271807280985\\
35.45	0.00599273767390787\\
35.46	0.00599275728295662\\
35.47	0.00599277689996041\\
35.48	0.00599279652492355\\
35.49	0.00599281615785035\\
35.5	0.00599283579874511\\
35.51	0.00599285544761217\\
35.52	0.00599287510445583\\
35.53	0.00599289476928044\\
35.54	0.00599291444209031\\
35.55	0.00599293412288977\\
35.56	0.00599295381168317\\
35.57	0.00599297350847483\\
35.58	0.00599299321326911\\
35.59	0.00599301292607037\\
35.6	0.00599303264688292\\
35.61	0.00599305237571115\\
35.62	0.0059930721125594\\
35.63	0.00599309185743204\\
35.64	0.00599311161033343\\
35.65	0.00599313137126794\\
35.66	0.00599315114023995\\
35.67	0.00599317091725382\\
35.68	0.00599319070231395\\
35.69	0.00599321049542471\\
35.7	0.00599323029659049\\
35.71	0.00599325010581568\\
35.72	0.00599326992310468\\
35.73	0.00599328974846189\\
35.74	0.00599330958189171\\
35.75	0.00599332942339853\\
35.76	0.00599334927298679\\
35.77	0.00599336913066089\\
35.78	0.00599338899642523\\
35.79	0.00599340887028426\\
35.8	0.00599342875224239\\
35.81	0.00599344864230405\\
35.82	0.00599346854047367\\
35.83	0.00599348844675568\\
35.84	0.00599350836115454\\
35.85	0.00599352828367468\\
35.86	0.00599354821432055\\
35.87	0.00599356815309661\\
35.88	0.00599358810000729\\
35.89	0.00599360805505708\\
35.9	0.00599362801825041\\
35.91	0.00599364798959178\\
35.92	0.00599366796908564\\
35.93	0.00599368795673646\\
35.94	0.00599370795254873\\
35.95	0.00599372795652693\\
35.96	0.00599374796867555\\
35.97	0.00599376798899907\\
35.98	0.00599378801750198\\
35.99	0.00599380805418879\\
36	0.00599382809906399\\
36.01	0.00599384815213209\\
36.02	0.0059938682133976\\
36.03	0.00599388828286503\\
36.04	0.0059939083605389\\
36.05	0.00599392844642373\\
36.06	0.00599394854052403\\
36.07	0.00599396864284435\\
36.08	0.0059939887533892\\
36.09	0.00599400887216314\\
36.1	0.00599402899917069\\
36.11	0.0059940491344164\\
36.12	0.00599406927790483\\
36.13	0.0059940894296405\\
36.14	0.005994109589628\\
36.15	0.00599412975787187\\
36.16	0.00599414993437668\\
36.17	0.00599417011914699\\
36.18	0.00599419031218738\\
36.19	0.00599421051350242\\
36.2	0.00599423072309669\\
36.21	0.00599425094097477\\
36.22	0.00599427116714125\\
36.23	0.00599429140160073\\
36.24	0.00599431164435778\\
36.25	0.00599433189541703\\
36.26	0.00599435215478306\\
36.27	0.00599437242246048\\
36.28	0.00599439269845391\\
36.29	0.00599441298276796\\
36.3	0.00599443327540725\\
36.31	0.0059944535763764\\
36.32	0.00599447388568003\\
36.33	0.00599449420332279\\
36.34	0.0059945145293093\\
36.35	0.0059945348636442\\
36.36	0.00599455520633214\\
36.37	0.00599457555737776\\
36.38	0.00599459591678571\\
36.39	0.00599461628456065\\
36.4	0.00599463666070724\\
36.41	0.00599465704523014\\
36.42	0.00599467743813401\\
36.43	0.00599469783942354\\
36.44	0.00599471824910339\\
36.45	0.00599473866717824\\
36.46	0.00599475909365278\\
36.47	0.00599477952853169\\
36.48	0.00599479997181966\\
36.49	0.00599482042352141\\
36.5	0.0059948408836416\\
36.51	0.00599486135218497\\
36.52	0.00599488182915621\\
36.53	0.00599490231456003\\
36.54	0.00599492280840116\\
36.55	0.00599494331068431\\
36.56	0.0059949638214142\\
36.57	0.00599498434059557\\
36.58	0.00599500486823315\\
36.59	0.00599502540433168\\
36.6	0.00599504594889588\\
36.61	0.00599506650193052\\
36.62	0.00599508706344034\\
36.63	0.0059951076334301\\
36.64	0.00599512821190455\\
36.65	0.00599514879886845\\
36.66	0.00599516939432657\\
36.67	0.00599518999828368\\
36.68	0.00599521061074456\\
36.69	0.00599523123171398\\
36.7	0.00599525186119673\\
36.71	0.00599527249919759\\
36.72	0.00599529314572135\\
36.73	0.00599531380077282\\
36.74	0.00599533446435678\\
36.75	0.00599535513647806\\
36.76	0.00599537581714144\\
36.77	0.00599539650635175\\
36.78	0.0059954172041138\\
36.79	0.00599543791043241\\
36.8	0.00599545862531241\\
36.81	0.00599547934875863\\
36.82	0.0059955000807759\\
36.83	0.00599552082136907\\
36.84	0.00599554157054295\\
36.85	0.00599556232830243\\
36.86	0.00599558309465233\\
36.87	0.00599560386959751\\
36.88	0.00599562465314284\\
36.89	0.00599564544529318\\
36.9	0.00599566624605339\\
36.91	0.00599568705542835\\
36.92	0.00599570787342293\\
36.93	0.00599572870004203\\
36.94	0.00599574953529051\\
36.95	0.00599577037917327\\
36.96	0.00599579123169522\\
36.97	0.00599581209286123\\
36.98	0.00599583296267622\\
36.99	0.0059958538411451\\
37	0.00599587472827277\\
37.01	0.00599589562406416\\
37.02	0.00599591652852418\\
37.03	0.00599593744165775\\
37.04	0.00599595836346981\\
37.05	0.00599597929396528\\
37.06	0.00599600023314912\\
37.07	0.00599602118102625\\
37.08	0.00599604213760163\\
37.09	0.00599606310288021\\
37.1	0.00599608407686694\\
37.11	0.00599610505956678\\
37.12	0.00599612605098471\\
37.13	0.00599614705112567\\
37.14	0.00599616805999465\\
37.15	0.00599618907759663\\
37.16	0.0059962101039366\\
37.17	0.00599623113901952\\
37.18	0.00599625218285041\\
37.19	0.00599627323543424\\
37.2	0.00599629429677603\\
37.21	0.00599631536688078\\
37.22	0.00599633644575349\\
37.23	0.00599635753339918\\
37.24	0.00599637862982288\\
37.25	0.00599639973502959\\
37.26	0.00599642084902434\\
37.27	0.00599644197181218\\
37.28	0.00599646310339813\\
37.29	0.00599648424378724\\
37.3	0.00599650539298455\\
37.31	0.0059965265509951\\
37.32	0.00599654771782397\\
37.33	0.00599656889347619\\
37.34	0.00599659007795685\\
37.35	0.005996611271271\\
37.36	0.00599663247342371\\
37.37	0.00599665368442007\\
37.38	0.00599667490426516\\
37.39	0.00599669613296406\\
37.4	0.00599671737052186\\
37.41	0.00599673861694366\\
37.42	0.00599675987223456\\
37.43	0.00599678113639966\\
37.44	0.00599680240944408\\
37.45	0.00599682369137292\\
37.46	0.00599684498219131\\
37.47	0.00599686628190436\\
37.48	0.00599688759051721\\
37.49	0.00599690890803499\\
37.5	0.00599693023446283\\
37.51	0.00599695156980588\\
37.52	0.00599697291406929\\
37.53	0.00599699426725819\\
37.54	0.00599701562937776\\
37.55	0.00599703700043315\\
37.56	0.00599705838042952\\
37.57	0.00599707976937205\\
37.58	0.00599710116726591\\
37.59	0.00599712257411628\\
37.6	0.00599714398992834\\
37.61	0.00599716541470729\\
37.62	0.00599718684845831\\
37.63	0.0059972082911866\\
37.64	0.00599722974289736\\
37.65	0.00599725120359581\\
37.66	0.00599727267328716\\
37.67	0.00599729415197662\\
37.68	0.00599731563966941\\
37.69	0.00599733713637076\\
37.7	0.00599735864208591\\
37.71	0.00599738015682008\\
37.72	0.00599740168057853\\
37.73	0.00599742321336648\\
37.74	0.0059974447551892\\
37.75	0.00599746630605195\\
37.76	0.00599748786595997\\
37.77	0.00599750943491854\\
37.78	0.00599753101293292\\
37.79	0.00599755260000839\\
37.8	0.00599757419615023\\
37.81	0.00599759580136373\\
37.82	0.00599761741565416\\
37.83	0.00599763903902683\\
37.84	0.00599766067148703\\
37.85	0.00599768231304008\\
37.86	0.00599770396369126\\
37.87	0.0059977256234459\\
37.88	0.00599774729230933\\
37.89	0.00599776897028685\\
37.9	0.0059977906573838\\
37.91	0.00599781235360551\\
37.92	0.00599783405895732\\
37.93	0.00599785577344457\\
37.94	0.00599787749707261\\
37.95	0.00599789922984679\\
37.96	0.00599792097177247\\
37.97	0.00599794272285501\\
37.98	0.00599796448309978\\
37.99	0.00599798625251214\\
38	0.00599800803109748\\
38.01	0.00599802981886118\\
38.02	0.00599805161580863\\
38.03	0.00599807342194521\\
38.04	0.00599809523727633\\
38.05	0.00599811706180738\\
38.06	0.00599813889554378\\
38.07	0.00599816073849093\\
38.08	0.00599818259065426\\
38.09	0.00599820445203917\\
38.1	0.00599822632265111\\
38.11	0.0059982482024955\\
38.12	0.00599827009157777\\
38.13	0.00599829198990338\\
38.14	0.00599831389747777\\
38.15	0.00599833581430639\\
38.16	0.00599835774039469\\
38.17	0.00599837967574814\\
38.18	0.00599840162037221\\
38.19	0.00599842357427236\\
38.2	0.00599844553745408\\
38.21	0.00599846750992285\\
38.22	0.00599848949168415\\
38.23	0.00599851148274347\\
38.24	0.00599853348310633\\
38.25	0.0059985554927782\\
38.26	0.00599857751176462\\
38.27	0.00599859954007108\\
38.28	0.0059986215777031\\
38.29	0.00599864362466622\\
38.3	0.00599866568096595\\
38.31	0.00599868774660783\\
38.32	0.0059987098215974\\
38.33	0.0059987319059402\\
38.34	0.00599875399964178\\
38.35	0.00599877610270769\\
38.36	0.0059987982151435\\
38.37	0.00599882033695476\\
38.38	0.00599884246814705\\
38.39	0.00599886460872595\\
38.4	0.00599888675869702\\
38.41	0.00599890891806586\\
38.42	0.00599893108683805\\
38.43	0.0059989532650192\\
38.44	0.00599897545261489\\
38.45	0.00599899764963075\\
38.46	0.00599901985607237\\
38.47	0.00599904207194537\\
38.48	0.00599906429725538\\
38.49	0.00599908653200803\\
38.5	0.00599910877620893\\
38.51	0.00599913102986374\\
38.52	0.00599915329297809\\
38.53	0.00599917556555763\\
38.54	0.00599919784760802\\
38.55	0.00599922013913491\\
38.56	0.00599924244014396\\
38.57	0.00599926475064084\\
38.58	0.00599928707063124\\
38.59	0.00599930940012082\\
38.6	0.00599933173911528\\
38.61	0.00599935408762029\\
38.62	0.00599937644564156\\
38.63	0.00599939881318479\\
38.64	0.00599942119025569\\
38.65	0.00599944357685995\\
38.66	0.00599946597300331\\
38.67	0.00599948837869148\\
38.68	0.00599951079393019\\
38.69	0.00599953321872517\\
38.7	0.00599955565308216\\
38.71	0.00599957809700691\\
38.72	0.00599960055050515\\
38.73	0.00599962301358266\\
38.74	0.00599964548624517\\
38.75	0.00599966796849847\\
38.76	0.00599969046034832\\
38.77	0.00599971296180049\\
38.78	0.00599973547286077\\
38.79	0.00599975799353494\\
38.8	0.0059997805238288\\
38.81	0.00599980306374814\\
38.82	0.00599982561329876\\
38.83	0.00599984817248647\\
38.84	0.0059998707413171\\
38.85	0.00599989331979644\\
38.86	0.00599991590793033\\
38.87	0.00599993850572461\\
38.88	0.00599996111318509\\
38.89	0.00599998373031763\\
38.9	0.00600000635712808\\
38.91	0.00600002899362227\\
38.92	0.00600005163980608\\
38.93	0.00600007429568535\\
38.94	0.00600009696126597\\
38.95	0.0060001196365538\\
38.96	0.00600014232155472\\
38.97	0.00600016501627462\\
38.98	0.00600018772071939\\
38.99	0.00600021043489492\\
39	0.00600023315880712\\
39.01	0.00600025589246188\\
39.02	0.00600027863586514\\
39.03	0.00600030138902279\\
39.04	0.00600032415194077\\
39.05	0.006000346924625\\
39.06	0.00600036970708142\\
39.07	0.00600039249931596\\
39.08	0.00600041530133458\\
39.09	0.00600043811314323\\
39.1	0.00600046093474785\\
39.11	0.00600048376615442\\
39.12	0.0060005066073689\\
39.13	0.00600052945839726\\
39.14	0.00600055231924549\\
39.15	0.00600057518991957\\
39.16	0.00600059807042548\\
39.17	0.00600062096076923\\
39.18	0.00600064386095681\\
39.19	0.00600066677099424\\
39.2	0.00600068969088753\\
39.21	0.00600071262064268\\
39.22	0.00600073556026574\\
39.23	0.00600075850976272\\
39.24	0.00600078146913967\\
39.25	0.00600080443840262\\
39.26	0.00600082741755762\\
39.27	0.00600085040661073\\
39.28	0.00600087340556799\\
39.29	0.00600089641443548\\
39.3	0.00600091943321926\\
39.31	0.00600094246192542\\
39.32	0.00600096550056002\\
39.33	0.00600098854912915\\
39.34	0.00600101160763892\\
39.35	0.0060010346760954\\
39.36	0.00600105775450472\\
39.37	0.00600108084287298\\
39.38	0.00600110394120628\\
39.39	0.00600112704951077\\
39.4	0.00600115016779255\\
39.41	0.00600117329605776\\
39.42	0.00600119643431255\\
39.43	0.00600121958256304\\
39.44	0.0060012427408154\\
39.45	0.00600126590907577\\
39.46	0.00600128908735033\\
39.47	0.00600131227564523\\
39.48	0.00600133547396664\\
39.49	0.00600135868232075\\
39.5	0.00600138190071374\\
39.51	0.0060014051291518\\
39.52	0.00600142836764112\\
39.53	0.00600145161618791\\
39.54	0.00600147487479837\\
39.55	0.00600149814347871\\
39.56	0.00600152142223516\\
39.57	0.00600154471107393\\
39.58	0.00600156801000126\\
39.59	0.00600159131902339\\
39.6	0.00600161463814656\\
39.61	0.006001637967377\\
39.62	0.00600166130672098\\
39.63	0.00600168465618476\\
39.64	0.0060017080157746\\
39.65	0.00600173138549677\\
39.66	0.00600175476535755\\
39.67	0.00600177815536322\\
39.68	0.00600180155552008\\
39.69	0.0060018249658344\\
39.7	0.00600184838631251\\
39.71	0.0060018718169607\\
39.72	0.00600189525778527\\
39.73	0.00600191870879256\\
39.74	0.00600194216998888\\
39.75	0.00600196564138056\\
39.76	0.00600198912297395\\
39.77	0.00600201261477538\\
39.78	0.00600203611679119\\
39.79	0.00600205962902775\\
39.8	0.00600208315149141\\
39.81	0.00600210668418853\\
39.82	0.00600213022712549\\
39.83	0.00600215378030866\\
39.84	0.00600217734374443\\
39.85	0.00600220091743919\\
39.86	0.00600222450139932\\
39.87	0.00600224809563123\\
39.88	0.00600227170014134\\
39.89	0.00600229531493604\\
39.9	0.00600231894002176\\
39.91	0.00600234257540492\\
39.92	0.00600236622109195\\
39.93	0.0060023898770893\\
39.94	0.00600241354340339\\
39.95	0.00600243722004069\\
39.96	0.00600246090700764\\
39.97	0.0060024846043107\\
39.98	0.00600250831195634\\
39.99	0.00600253202995104\\
40	0.00600255575830126\\
40.01	0.00600257949701351\\
};
\addplot [color=blue,solid,forget plot]
  table[row sep=crcr]{%
40.01	0.00600257949701351\\
40.02	0.00600260324609425\\
40.03	0.00600262700555\\
40.04	0.00600265077538725\\
40.05	0.00600267455561252\\
40.06	0.00600269834623231\\
40.07	0.00600272214725314\\
40.08	0.00600274595868155\\
40.09	0.00600276978052407\\
40.1	0.00600279361278724\\
40.11	0.00600281745547758\\
40.12	0.00600284130860167\\
40.13	0.00600286517216606\\
40.14	0.00600288904617731\\
40.15	0.00600291293064198\\
40.16	0.00600293682556665\\
40.17	0.00600296073095792\\
40.18	0.00600298464682235\\
40.19	0.00600300857316654\\
40.2	0.0060030325099971\\
40.21	0.00600305645732063\\
40.22	0.00600308041514374\\
40.23	0.00600310438347305\\
40.24	0.00600312836231518\\
40.25	0.00600315235167676\\
40.26	0.00600317635156443\\
40.27	0.00600320036198484\\
40.28	0.00600322438294462\\
40.29	0.00600324841445043\\
40.3	0.00600327245650895\\
40.31	0.00600329650912682\\
40.32	0.00600332057231073\\
40.33	0.00600334464606735\\
40.34	0.00600336873040338\\
40.35	0.00600339282532549\\
40.36	0.0060034169308404\\
40.37	0.0060034410469548\\
40.38	0.0060034651736754\\
40.39	0.00600348931100893\\
40.4	0.0060035134589621\\
40.41	0.00600353761754165\\
40.42	0.0060035617867543\\
40.43	0.00600358596660681\\
40.44	0.00600361015710592\\
40.45	0.00600363435825838\\
40.46	0.00600365857007096\\
40.47	0.00600368279255042\\
40.48	0.00600370702570353\\
40.49	0.00600373126953708\\
40.5	0.00600375552405786\\
40.51	0.00600377978927265\\
40.52	0.00600380406518825\\
40.53	0.00600382835181147\\
40.54	0.00600385264914913\\
40.55	0.00600387695720803\\
40.56	0.006003901275995\\
40.57	0.00600392560551688\\
40.58	0.0060039499457805\\
40.59	0.00600397429679271\\
40.6	0.00600399865856036\\
40.61	0.0060040230310903\\
40.62	0.0060040474143894\\
40.63	0.00600407180846452\\
40.64	0.00600409621332255\\
40.65	0.00600412062897037\\
40.66	0.00600414505541486\\
40.67	0.00600416949266293\\
40.68	0.00600419394072148\\
40.69	0.00600421839959741\\
40.7	0.00600424286929764\\
40.71	0.0060042673498291\\
40.72	0.00600429184119872\\
40.73	0.00600431634341343\\
40.74	0.00600434085648017\\
40.75	0.0060043653804059\\
40.76	0.00600438991519757\\
40.77	0.00600441446086214\\
40.78	0.00600443901740659\\
40.79	0.0060044635848379\\
40.8	0.00600448816316304\\
40.81	0.006004512752389\\
40.82	0.00600453735252279\\
40.83	0.00600456196357141\\
40.84	0.00600458658554187\\
40.85	0.0060046112184412\\
40.86	0.00600463586227641\\
40.87	0.00600466051705454\\
40.88	0.00600468518278263\\
40.89	0.00600470985946772\\
40.9	0.00600473454711687\\
40.91	0.00600475924573714\\
40.92	0.00600478395533561\\
40.93	0.00600480867591933\\
40.94	0.00600483340749541\\
40.95	0.00600485815007092\\
40.96	0.00600488290365297\\
40.97	0.00600490766824866\\
40.98	0.0060049324438651\\
40.99	0.00600495723050941\\
41	0.00600498202818872\\
41.01	0.00600500683691016\\
41.02	0.00600503165668089\\
41.03	0.00600505648750803\\
41.04	0.00600508132939876\\
41.05	0.00600510618236024\\
41.06	0.00600513104639964\\
41.07	0.00600515592152414\\
41.08	0.00600518080774093\\
41.09	0.0060052057050572\\
41.1	0.00600523061348017\\
41.11	0.00600525553301704\\
41.12	0.00600528046367503\\
41.13	0.00600530540546137\\
41.14	0.00600533035838329\\
41.15	0.00600535532244804\\
41.16	0.00600538029766288\\
41.17	0.00600540528403506\\
41.18	0.00600543028157186\\
41.19	0.00600545529028054\\
41.2	0.00600548031016839\\
41.21	0.00600550534124272\\
41.22	0.00600553038351082\\
41.23	0.00600555543698\\
41.24	0.00600558050165758\\
41.25	0.0060056055775509\\
41.26	0.00600563066466728\\
41.27	0.00600565576301408\\
41.28	0.00600568087259865\\
41.29	0.00600570599342836\\
41.3	0.00600573112551057\\
41.31	0.00600575626885268\\
41.32	0.00600578142346207\\
41.33	0.00600580658934614\\
41.34	0.0060058317665123\\
41.35	0.00600585695496798\\
41.36	0.00600588215472061\\
41.37	0.00600590736577761\\
41.38	0.00600593258814644\\
41.39	0.00600595782183456\\
41.4	0.00600598306684944\\
41.41	0.00600600832319855\\
41.42	0.00600603359088938\\
41.43	0.00600605886992944\\
41.44	0.00600608416032622\\
41.45	0.00600610946208725\\
41.46	0.00600613477522005\\
41.47	0.00600616009973218\\
41.48	0.00600618543563117\\
41.49	0.00600621078292458\\
41.5	0.00600623614162\\
41.51	0.00600626151172499\\
41.52	0.00600628689324715\\
41.53	0.0060063122861941\\
41.54	0.00600633769057345\\
41.55	0.00600636310639281\\
41.56	0.00600638853365983\\
41.57	0.00600641397238217\\
41.58	0.00600643942256747\\
41.59	0.00600646488422342\\
41.6	0.0060064903573577\\
41.61	0.00600651584197801\\
41.62	0.00600654133809206\\
41.63	0.00600656684570756\\
41.64	0.00600659236483225\\
41.65	0.00600661789547388\\
41.66	0.00600664343764021\\
41.67	0.00600666899133902\\
41.68	0.00600669455657807\\
41.69	0.00600672013336517\\
41.7	0.00600674572170814\\
41.71	0.0060067713216148\\
41.72	0.00600679693309298\\
41.73	0.00600682255615053\\
41.74	0.00600684819079532\\
41.75	0.00600687383703523\\
41.76	0.00600689949487816\\
41.77	0.006006925164332\\
41.78	0.00600695084540467\\
41.79	0.00600697653810412\\
41.8	0.0060070022424383\\
41.81	0.00600702795841516\\
41.82	0.00600705368604268\\
41.83	0.00600707942532887\\
41.84	0.00600710517628172\\
41.85	0.00600713093890927\\
41.86	0.00600715671321955\\
41.87	0.00600718249922063\\
41.88	0.00600720829692056\\
41.89	0.00600723410632744\\
41.9	0.00600725992744938\\
41.91	0.00600728576029448\\
41.92	0.00600731160487088\\
41.93	0.00600733746118675\\
41.94	0.00600736332925024\\
41.95	0.00600738920906955\\
41.96	0.00600741510065287\\
41.97	0.00600744100400842\\
41.98	0.00600746691914445\\
41.99	0.0060074928460692\\
42	0.00600751878479094\\
42.01	0.00600754473531797\\
42.02	0.0060075706976586\\
42.03	0.00600759667182115\\
42.04	0.00600762265781396\\
42.05	0.00600764865564539\\
42.06	0.00600767466532383\\
42.07	0.00600770068685767\\
42.08	0.00600772672025533\\
42.09	0.00600775276552525\\
42.1	0.00600777882267587\\
42.11	0.00600780489171569\\
42.12	0.00600783097265319\\
42.13	0.00600785706549688\\
42.14	0.0060078831702553\\
42.15	0.00600790928693699\\
42.16	0.00600793541555053\\
42.17	0.00600796155610452\\
42.18	0.00600798770860756\\
42.19	0.00600801387306828\\
42.2	0.00600804004949534\\
42.21	0.00600806623789741\\
42.22	0.00600809243828318\\
42.23	0.00600811865066137\\
42.24	0.0060081448750407\\
42.25	0.00600817111142993\\
42.26	0.00600819735983785\\
42.27	0.00600822362027323\\
42.28	0.0060082498927449\\
42.29	0.00600827617726169\\
42.3	0.00600830247383246\\
42.31	0.0060083287824661\\
42.32	0.0060083551031715\\
42.33	0.00600838143595758\\
42.34	0.00600840778083327\\
42.35	0.00600843413780755\\
42.36	0.0060084605068894\\
42.37	0.00600848688808782\\
42.38	0.00600851328141182\\
42.39	0.00600853968687047\\
42.4	0.00600856610447283\\
42.41	0.00600859253422797\\
42.42	0.00600861897614502\\
42.43	0.00600864543023309\\
42.44	0.00600867189650133\\
42.45	0.00600869837495891\\
42.46	0.00600872486561503\\
42.47	0.00600875136847888\\
42.48	0.00600877788355969\\
42.49	0.00600880441086672\\
42.5	0.00600883095040922\\
42.51	0.00600885750219649\\
42.52	0.00600888406623783\\
42.53	0.00600891064254255\\
42.54	0.00600893723112001\\
42.55	0.00600896383197956\\
42.56	0.00600899044513058\\
42.57	0.00600901707058246\\
42.58	0.00600904370834461\\
42.59	0.00600907035842647\\
42.6	0.00600909702083749\\
42.61	0.00600912369558711\\
42.62	0.00600915038268482\\
42.63	0.0060091770821401\\
42.64	0.00600920379396247\\
42.65	0.00600923051816145\\
42.66	0.00600925725474657\\
42.67	0.00600928400372738\\
42.68	0.00600931076511344\\
42.69	0.00600933753891431\\
42.7	0.0060093643251396\\
42.71	0.00600939112379888\\
42.72	0.00600941793490178\\
42.73	0.00600944475845789\\
42.74	0.00600947159447685\\
42.75	0.00600949844296829\\
42.76	0.00600952530394185\\
42.77	0.00600955217740717\\
42.78	0.00600957906337391\\
42.79	0.00600960596185173\\
42.8	0.00600963287285029\\
42.81	0.00600965979637927\\
42.82	0.00600968673244832\\
42.83	0.00600971368106712\\
42.84	0.00600974064224535\\
42.85	0.00600976761599268\\
42.86	0.0060097946023188\\
42.87	0.00600982160123336\\
42.88	0.00600984861274604\\
42.89	0.00600987563686651\\
42.9	0.00600990267360444\\
42.91	0.00600992972296949\\
42.92	0.00600995678497131\\
42.93	0.00600998385961954\\
42.94	0.00601001094692384\\
42.95	0.00601003804689384\\
42.96	0.00601006515953915\\
42.97	0.0060100922848694\\
42.98	0.00601011942289419\\
42.99	0.00601014657362311\\
43	0.00601017373706574\\
43.01	0.00601020091323166\\
43.02	0.00601022810213042\\
43.03	0.00601025530377155\\
43.04	0.00601028251816459\\
43.05	0.00601030974531904\\
43.06	0.00601033698524441\\
43.07	0.00601036423795017\\
43.08	0.00601039150344579\\
43.09	0.00601041878174071\\
43.1	0.00601044607284438\\
43.11	0.00601047337676619\\
43.12	0.00601050069351556\\
43.13	0.00601052802310187\\
43.14	0.00601055536553448\\
43.15	0.00601058272082275\\
43.16	0.00601061008897601\\
43.17	0.0060106374700036\\
43.18	0.00601066486391483\\
43.19	0.00601069227071901\\
43.2	0.00601071969042541\\
43.21	0.00601074712304334\\
43.22	0.00601077456858209\\
43.23	0.00601080202705091\\
43.24	0.00601082949845911\\
43.25	0.00601085698281595\\
43.26	0.00601088448013074\\
43.27	0.00601091199041278\\
43.28	0.00601093951367136\\
43.29	0.00601096704991581\\
43.3	0.00601099459915548\\
43.31	0.0060110221613997\\
43.32	0.00601104973665783\\
43.33	0.00601107732493922\\
43.34	0.00601110492625327\\
43.35	0.00601113254060934\\
43.36	0.00601116016801685\\
43.37	0.00601118780848521\\
43.38	0.00601121546202381\\
43.39	0.00601124312864211\\
43.4	0.00601127080834954\\
43.41	0.00601129850115556\\
43.42	0.00601132620706962\\
43.43	0.0060113539261012\\
43.44	0.00601138165825979\\
43.45	0.00601140940355488\\
43.46	0.00601143716199598\\
43.47	0.0060114649335926\\
43.48	0.00601149271835428\\
43.49	0.00601152051629056\\
43.5	0.00601154832741098\\
43.51	0.00601157615172512\\
43.52	0.00601160398924253\\
43.53	0.00601163183997282\\
43.54	0.00601165970392557\\
43.55	0.00601168758111038\\
43.56	0.00601171547153688\\
43.57	0.00601174337521471\\
43.58	0.00601177129215348\\
43.59	0.00601179922236287\\
43.6	0.00601182716585253\\
43.61	0.00601185512263213\\
43.62	0.00601188309271136\\
43.63	0.00601191107609991\\
43.64	0.0060119390728075\\
43.65	0.00601196708284384\\
43.66	0.00601199510621867\\
43.67	0.00601202314294172\\
43.68	0.00601205119302274\\
43.69	0.00601207925647151\\
43.7	0.00601210733329779\\
43.71	0.00601213542351138\\
43.72	0.00601216352712207\\
43.73	0.00601219164413966\\
43.74	0.00601221977457399\\
43.75	0.00601224791843488\\
43.76	0.00601227607573219\\
43.77	0.00601230424647575\\
43.78	0.00601233243067545\\
43.79	0.00601236062834115\\
43.8	0.00601238883948276\\
43.81	0.00601241706411016\\
43.82	0.00601244530223328\\
43.83	0.00601247355386203\\
43.84	0.00601250181900635\\
43.85	0.00601253009767619\\
43.86	0.00601255838988152\\
43.87	0.00601258669563229\\
43.88	0.00601261501493849\\
43.89	0.00601264334781011\\
43.9	0.00601267169425717\\
43.91	0.00601270005428968\\
43.92	0.00601272842791765\\
43.93	0.00601275681515115\\
43.94	0.00601278521600021\\
43.95	0.00601281363047491\\
43.96	0.00601284205858531\\
43.97	0.00601287050034151\\
43.98	0.00601289895575359\\
43.99	0.00601292742483168\\
44	0.0060129559075859\\
44.01	0.00601298440402638\\
44.02	0.00601301291416326\\
44.03	0.0060130414380067\\
44.04	0.00601306997556687\\
44.05	0.00601309852685396\\
44.06	0.00601312709187815\\
44.07	0.00601315567064965\\
44.08	0.00601318426317868\\
44.09	0.00601321286947546\\
44.1	0.00601324148955025\\
44.11	0.00601327012341328\\
44.12	0.00601329877107482\\
44.13	0.00601332743254515\\
44.14	0.00601335610783455\\
44.15	0.00601338479695333\\
44.16	0.0060134134999118\\
44.17	0.00601344221672028\\
44.18	0.00601347094738911\\
44.19	0.00601349969192862\\
44.2	0.00601352845034919\\
44.21	0.00601355722266119\\
44.22	0.00601358600887499\\
44.23	0.006013614809001\\
44.24	0.00601364362304962\\
44.25	0.00601367245103127\\
44.26	0.00601370129295638\\
44.27	0.0060137301488354\\
44.28	0.00601375901867877\\
44.29	0.00601378790249698\\
44.3	0.00601381680030049\\
44.31	0.0060138457120998\\
44.32	0.00601387463790542\\
44.33	0.00601390357772785\\
44.34	0.00601393253157763\\
44.35	0.0060139614994653\\
44.36	0.00601399048140141\\
44.37	0.00601401947739652\\
44.38	0.00601404848746121\\
44.39	0.00601407751160606\\
44.4	0.00601410654984168\\
44.41	0.00601413560217868\\
44.42	0.00601416466862769\\
44.43	0.00601419374919934\\
44.44	0.00601422284390428\\
44.45	0.00601425195275317\\
44.46	0.00601428107575669\\
44.47	0.00601431021292552\\
44.48	0.00601433936427036\\
44.49	0.00601436852980191\\
44.5	0.00601439770953091\\
44.51	0.00601442690346808\\
44.52	0.00601445611162417\\
44.53	0.00601448533400994\\
44.54	0.00601451457063616\\
44.55	0.00601454382151361\\
44.56	0.00601457308665309\\
44.57	0.0060146023660654\\
44.58	0.00601463165976137\\
44.59	0.00601466096775183\\
44.6	0.00601469029004761\\
44.61	0.00601471962665958\\
44.62	0.0060147489775986\\
44.63	0.00601477834287556\\
44.64	0.00601480772250135\\
44.65	0.00601483711648686\\
44.66	0.00601486652484303\\
44.67	0.00601489594758077\\
44.68	0.00601492538471103\\
44.69	0.00601495483624476\\
44.7	0.00601498430219293\\
44.71	0.00601501378256651\\
44.72	0.0060150432773765\\
44.73	0.0060150727866339\\
44.74	0.00601510231034972\\
44.75	0.00601513184853499\\
44.76	0.00601516140120074\\
44.77	0.00601519096835803\\
44.78	0.00601522055001792\\
44.79	0.00601525014619149\\
44.8	0.00601527975688982\\
44.81	0.00601530938212401\\
44.82	0.00601533902190518\\
44.83	0.00601536867624444\\
44.84	0.00601539834515293\\
44.85	0.00601542802864181\\
44.86	0.00601545772672223\\
44.87	0.00601548743940535\\
44.88	0.00601551716670237\\
44.89	0.00601554690862448\\
44.9	0.00601557666518289\\
44.91	0.00601560643638883\\
44.92	0.00601563622225351\\
44.93	0.00601566602278818\\
44.94	0.00601569583800411\\
44.95	0.00601572566791255\\
44.96	0.0060157555125248\\
44.97	0.00601578537185213\\
44.98	0.00601581524590586\\
44.99	0.0060158451346973\\
45	0.00601587503823778\\
45.01	0.00601590495653862\\
45.02	0.0060159348896112\\
45.03	0.00601596483746687\\
45.04	0.006015994800117\\
45.05	0.00601602477757299\\
45.06	0.00601605476984622\\
45.07	0.00601608477694811\\
45.08	0.00601611479889008\\
45.09	0.00601614483568357\\
45.1	0.00601617488734002\\
45.11	0.00601620495387088\\
45.12	0.00601623503528762\\
45.13	0.00601626513160174\\
45.14	0.0060162952428247\\
45.15	0.00601632536896803\\
45.16	0.00601635551004322\\
45.17	0.00601638566606181\\
45.18	0.00601641583703534\\
45.19	0.00601644602297535\\
45.2	0.0060164762238934\\
45.21	0.00601650643980107\\
45.22	0.00601653667070994\\
45.23	0.00601656691663159\\
45.24	0.00601659717757764\\
45.25	0.00601662745355971\\
45.26	0.00601665774458941\\
45.27	0.00601668805067839\\
45.28	0.00601671837183829\\
45.29	0.00601674870808079\\
45.3	0.00601677905941754\\
45.31	0.00601680942586023\\
45.32	0.00601683980742054\\
45.33	0.0060168702041102\\
45.34	0.00601690061594091\\
45.35	0.0060169310429244\\
45.36	0.0060169614850724\\
45.37	0.00601699194239665\\
45.38	0.00601702241490893\\
45.39	0.00601705290262098\\
45.4	0.0060170834055446\\
45.41	0.00601711392369156\\
45.42	0.00601714445707368\\
45.43	0.00601717500570275\\
45.44	0.00601720556959059\\
45.45	0.00601723614874904\\
45.46	0.00601726674318994\\
45.47	0.00601729735292513\\
45.48	0.00601732797796647\\
45.49	0.00601735861832583\\
45.5	0.00601738927401509\\
45.51	0.00601741994504615\\
45.52	0.00601745063143088\\
45.53	0.00601748133318122\\
45.54	0.00601751205030905\\
45.55	0.00601754278282634\\
45.56	0.00601757353074499\\
45.57	0.00601760429407697\\
45.58	0.00601763507283421\\
45.59	0.0060176658670287\\
45.6	0.00601769667667239\\
45.61	0.00601772750177727\\
45.62	0.00601775834235533\\
45.63	0.00601778919841856\\
45.64	0.00601782006997898\\
45.65	0.0060178509570486\\
45.66	0.00601788185963944\\
45.67	0.00601791277776354\\
45.68	0.00601794371143293\\
45.69	0.00601797466065967\\
45.7	0.00601800562545581\\
45.71	0.00601803660583341\\
45.72	0.00601806760180455\\
45.73	0.00601809861338131\\
45.74	0.00601812964057577\\
45.75	0.00601816068340003\\
45.76	0.00601819174186618\\
45.77	0.00601822281598635\\
45.78	0.00601825390577263\\
45.79	0.00601828501123717\\
45.8	0.00601831613239207\\
45.81	0.00601834726924949\\
45.82	0.00601837842182156\\
45.83	0.00601840959012043\\
45.84	0.00601844077415826\\
45.85	0.0060184719739472\\
45.86	0.00601850318949942\\
45.87	0.0060185344208271\\
45.88	0.00601856566794242\\
45.89	0.00601859693085754\\
45.9	0.00601862820958467\\
45.91	0.006018659504136\\
45.92	0.00601869081452372\\
45.93	0.00601872214076004\\
45.94	0.00601875348285716\\
45.95	0.0060187848408273\\
45.96	0.00601881621468267\\
45.97	0.00601884760443549\\
45.98	0.00601887901009798\\
45.99	0.00601891043168238\\
46	0.00601894186920091\\
46.01	0.0060189733226658\\
46.02	0.00601900479208929\\
46.03	0.00601903627748363\\
46.04	0.00601906777886104\\
46.05	0.00601909929623379\\
46.06	0.0060191308296141\\
46.07	0.00601916237901423\\
46.08	0.00601919394444644\\
46.09	0.00601922552592296\\
46.1	0.00601925712345606\\
46.11	0.00601928873705799\\
46.12	0.00601932036674099\\
46.13	0.00601935201251733\\
46.14	0.00601938367439925\\
46.15	0.00601941535239901\\
46.16	0.00601944704652888\\
46.17	0.00601947875680109\\
46.18	0.0060195104832279\\
46.19	0.00601954222582156\\
46.2	0.00601957398459432\\
46.21	0.00601960575955844\\
46.22	0.00601963755072614\\
46.23	0.00601966935810969\\
46.24	0.00601970118172132\\
46.25	0.00601973302157326\\
46.26	0.00601976487767775\\
46.27	0.00601979675004702\\
46.28	0.0060198286386933\\
46.29	0.0060198605436288\\
46.3	0.00601989246486576\\
46.31	0.00601992440241637\\
46.32	0.00601995635629285\\
46.33	0.0060199883265074\\
46.34	0.00602002031307222\\
46.35	0.00602005231599948\\
46.36	0.00602008433530138\\
46.37	0.0060201163709901\\
46.38	0.00602014842307778\\
46.39	0.0060201804915766\\
46.4	0.00602021257649872\\
46.41	0.00602024467785626\\
46.42	0.00602027679566136\\
46.43	0.00602030892992615\\
46.44	0.00602034108066273\\
46.45	0.00602037324788322\\
46.46	0.00602040543159969\\
46.47	0.00602043763182423\\
46.48	0.00602046984856892\\
46.49	0.0060205020818458\\
46.5	0.00602053433166691\\
46.51	0.00602056659804429\\
46.52	0.00602059888098996\\
46.53	0.0060206311805159\\
46.54	0.00602066349663411\\
46.55	0.00602069582935657\\
46.56	0.00602072817869521\\
46.57	0.00602076054466199\\
46.58	0.00602079292726882\\
46.59	0.00602082532652761\\
46.6	0.00602085774245024\\
46.61	0.00602089017504857\\
46.62	0.00602092262433446\\
46.63	0.00602095509031973\\
46.64	0.0060209875730162\\
46.65	0.00602102007243563\\
46.66	0.00602105258858981\\
46.67	0.00602108512149046\\
46.68	0.00602111767114931\\
46.69	0.00602115023757806\\
46.7	0.00602118282078837\\
46.71	0.00602121542079189\\
46.72	0.00602124803760023\\
46.73	0.00602128067122499\\
46.74	0.00602131332167775\\
46.75	0.00602134598897002\\
46.76	0.00602137867311333\\
46.77	0.00602141137411915\\
46.78	0.00602144409199894\\
46.79	0.0060214768267641\\
46.8	0.00602150957842604\\
46.81	0.0060215423469961\\
46.82	0.00602157513248561\\
46.83	0.00602160793490584\\
46.84	0.00602164075426807\\
46.85	0.00602167359058349\\
46.86	0.0060217064438633\\
46.87	0.00602173931411864\\
46.88	0.0060217722013606\\
46.89	0.00602180510560026\\
46.9	0.00602183802684864\\
46.91	0.00602187096511673\\
46.92	0.00602190392041546\\
46.93	0.00602193689275574\\
46.94	0.00602196988214841\\
46.95	0.0060220028886043\\
46.96	0.00602203591213416\\
46.97	0.00602206895274872\\
46.98	0.00602210201045864\\
46.99	0.00602213508527454\\
47	0.00602216817720699\\
47.01	0.00602220128626651\\
47.02	0.00602223441246358\\
47.03	0.0060222675558086\\
47.04	0.00602230071631193\\
47.05	0.00602233389398389\\
47.06	0.0060223670888347\\
47.07	0.00602240030087458\\
47.08	0.00602243353011364\\
47.09	0.00602246677656196\\
47.1	0.00602250004022955\\
47.11	0.00602253332112635\\
47.12	0.00602256661926225\\
47.13	0.00602259993464706\\
47.14	0.00602263326729055\\
47.15	0.00602266661720238\\
47.16	0.00602269998439219\\
47.17	0.0060227333688695\\
47.18	0.0060227667706438\\
47.19	0.00602280018972449\\
47.2	0.00602283362612089\\
47.21	0.00602286707984224\\
47.22	0.00602290055089773\\
47.23	0.00602293403929645\\
47.24	0.00602296754504742\\
47.25	0.00602300106815955\\
47.26	0.00602303460864172\\
47.27	0.00602306816650267\\
47.28	0.00602310174175109\\
47.29	0.00602313533439557\\
47.3	0.00602316894444462\\
47.31	0.00602320257190664\\
47.32	0.00602323621678995\\
47.33	0.00602326987910278\\
47.34	0.00602330355885326\\
47.35	0.00602333725604942\\
47.36	0.00602337097069919\\
47.37	0.00602340470281041\\
47.38	0.0060234384523908\\
47.39	0.006023472219448\\
47.4	0.00602350600398952\\
47.41	0.00602353980602278\\
47.42	0.00602357362555507\\
47.43	0.0060236074625936\\
47.44	0.00602364131714543\\
47.45	0.00602367518921753\\
47.46	0.00602370907881676\\
47.47	0.00602374298594983\\
47.48	0.00602377691062336\\
47.49	0.00602381085284383\\
47.5	0.0060238448126176\\
47.51	0.00602387878995091\\
47.52	0.00602391278484986\\
47.53	0.00602394679732042\\
47.54	0.00602398082736845\\
47.55	0.00602401487499963\\
47.56	0.00602404894021957\\
47.57	0.00602408302303367\\
47.58	0.00602411712344723\\
47.59	0.00602415124146541\\
47.6	0.00602418537709321\\
47.61	0.00602421953033548\\
47.62	0.00602425370119694\\
47.63	0.00602428788968213\\
47.64	0.00602432209579547\\
47.65	0.0060243563195412\\
47.66	0.0060243905609234\\
47.67	0.00602442481994601\\
47.68	0.0060244590966128\\
47.69	0.00602449339092736\\
47.7	0.00602452770289312\\
47.71	0.00602456203251335\\
47.72	0.00602459637979115\\
47.73	0.00602463074472943\\
47.74	0.00602466512733093\\
47.75	0.00602469952759821\\
47.76	0.00602473394553367\\
47.77	0.00602476838113949\\
47.78	0.00602480283441769\\
47.79	0.0060248373053701\\
47.8	0.00602487179399834\\
47.81	0.00602490630030385\\
47.82	0.00602494082428788\\
47.83	0.00602497536595149\\
47.84	0.0060250099252955\\
47.85	0.00602504450232058\\
47.86	0.00602507909702715\\
47.87	0.00602511370941545\\
47.88	0.0060251483394855\\
47.89	0.0060251829872371\\
47.9	0.00602521765266985\\
47.91	0.00602525233578313\\
47.92	0.00602528703657609\\
47.93	0.00602532175504767\\
47.94	0.00602535649119658\\
47.95	0.00602539124502129\\
47.96	0.00602542601652007\\
47.97	0.00602546080569093\\
47.98	0.00602549561253166\\
47.99	0.00602553043703982\\
48	0.00602556527921271\\
48.01	0.00602560013904741\\
48.02	0.00602563501654074\\
48.03	0.00602566991168929\\
48.04	0.00602570482448939\\
48.05	0.00602573975493712\\
48.06	0.00602577470302832\\
48.07	0.00602580966875856\\
48.08	0.00602584465212317\\
48.09	0.0060258796531172\\
48.1	0.00602591467173545\\
48.11	0.00602594970797247\\
48.12	0.00602598476182252\\
48.13	0.00602601983327961\\
48.14	0.00602605492233748\\
48.15	0.00602609002898959\\
48.16	0.00602612515322914\\
48.17	0.00602616029504904\\
48.18	0.00602619545444194\\
48.19	0.00602623063140021\\
48.2	0.00602626582591592\\
48.21	0.00602630103798089\\
48.22	0.00602633626758665\\
48.23	0.00602637151472442\\
48.24	0.00602640677938517\\
48.25	0.00602644206155956\\
48.26	0.00602647736123797\\
48.27	0.0060265126784105\\
48.28	0.00602654801306696\\
48.29	0.00602658336519686\\
48.3	0.00602661873478941\\
48.31	0.00602665412183356\\
48.32	0.00602668952631794\\
48.33	0.0060267249482309\\
48.34	0.00602676038756049\\
48.35	0.00602679584429448\\
48.36	0.00602683131842034\\
48.37	0.00602686680992524\\
48.38	0.00602690231879605\\
48.39	0.00602693784501939\\
48.4	0.00602697338858153\\
48.41	0.0060270089494685\\
48.42	0.006027044527666\\
48.43	0.00602708012315947\\
48.44	0.00602711573593404\\
48.45	0.00602715136597457\\
48.46	0.00602718701326562\\
48.47	0.00602722267779147\\
48.48	0.00602725835953614\\
48.49	0.00602729405848333\\
48.5	0.00602732977461649\\
48.51	0.00602736550791879\\
48.52	0.00602740125837313\\
48.53	0.00602743702596213\\
48.54	0.00602747281066816\\
48.55	0.00602750861247331\\
48.56	0.00602754443135943\\
48.57	0.0060275802673081\\
48.58	0.00602761612030065\\
48.59	0.00602765199031817\\
48.6	0.00602768787734151\\
48.61	0.00602772378135129\\
48.62	0.00602775970232787\\
48.63	0.00602779564025142\\
48.64	0.00602783159510188\\
48.65	0.00602786756685896\\
48.66	0.00602790355550218\\
48.67	0.00602793956101087\\
48.68	0.00602797558336416\\
48.69	0.00602801162254098\\
48.7	0.0060280476785201\\
48.71	0.00602808375128014\\
48.72	0.00602811984079953\\
48.73	0.00602815594705656\\
48.74	0.00602819207002941\\
48.75	0.00602822820969609\\
48.76	0.00602826436603452\\
48.77	0.00602830053902251\\
48.78	0.00602833672863776\\
48.79	0.0060283729348579\\
48.8	0.00602840915766048\\
48.81	0.00602844539702301\\
48.82	0.00602848165292294\\
48.83	0.00602851792533769\\
48.84	0.00602855421424468\\
48.85	0.00602859051962131\\
48.86	0.00602862684144501\\
48.87	0.00602866317969323\\
48.88	0.00602869953434349\\
48.89	0.00602873590537334\\
48.9	0.00602877229276045\\
48.91	0.00602880869648258\\
48.92	0.00602884511651759\\
48.93	0.00602888155284352\\
48.94	0.00602891800543854\\
48.95	0.00602895447428102\\
48.96	0.00602899095934952\\
48.97	0.00602902746062285\\
48.98	0.00602906397808005\\
48.99	0.00602910051170044\\
49	0.00602913706146364\\
49.01	0.0060291736273496\\
49.02	0.00602921020933861\\
49.03	0.00602924680741133\\
49.04	0.00602928342154885\\
49.05	0.00602932005173266\\
49.06	0.00602935669794474\\
49.07	0.00602939336016754\\
49.08	0.00602943003838403\\
49.09	0.00602946673257775\\
49.1	0.00602950344273282\\
49.11	0.00602954016883395\\
49.12	0.00602957691086654\\
49.13	0.00602961366881665\\
49.14	0.00602965044267106\\
49.15	0.00602968723241731\\
49.16	0.00602972403804374\\
49.17	0.00602976085953951\\
49.18	0.00602979769689464\\
49.19	0.00602983455010009\\
49.2	0.00602987141914772\\
49.21	0.00602990830403041\\
49.22	0.00602994520474207\\
49.23	0.00602998212127766\\
49.24	0.00603001905363325\\
49.25	0.0060300560018061\\
49.26	0.00603009296579464\\
49.27	0.00603012994559857\\
49.28	0.00603016694121885\\
49.29	0.0060302039526578\\
49.3	0.00603024097991913\\
49.31	0.00603027802300797\\
49.32	0.00603031508193093\\
49.33	0.00603035215669615\\
49.34	0.00603038924731335\\
49.35	0.00603042635379388\\
49.36	0.00603046347615076\\
49.37	0.00603050061439875\\
49.38	0.00603053776855436\\
49.39	0.00603057493863597\\
49.4	0.0060306121246638\\
49.41	0.00603064932666003\\
49.42	0.00603068654464881\\
49.43	0.00603072377865632\\
49.44	0.00603076102871085\\
49.45	0.00603079829484279\\
49.46	0.00603083557708476\\
49.47	0.00603087287547158\\
49.48	0.00603091019004037\\
49.49	0.0060309475208306\\
49.5	0.00603098486788412\\
49.51	0.00603102223124521\\
49.52	0.00603105961096063\\
49.53	0.0060310970070797\\
49.54	0.00603113441965426\\
49.55	0.00603117184873882\\
49.56	0.00603120929439053\\
49.57	0.00603124675666922\\
49.58	0.00603128423563749\\
49.59	0.0060313217313607\\
49.6	0.00603135924390704\\
49.61	0.00603139677334751\\
49.62	0.006031434319756\\
49.63	0.00603147188320932\\
49.64	0.00603150946378715\\
49.65	0.00603154706157216\\
49.66	0.00603158467664995\\
49.67	0.0060316223091091\\
49.68	0.00603165995904116\\
49.69	0.00603169762654066\\
49.7	0.0060317353117051\\
49.71	0.00603177301463497\\
49.72	0.0060318107354337\\
49.73	0.00603184847420766\\
49.74	0.00603188623106612\\
49.75	0.00603192400612124\\
49.76	0.006031961799488\\
49.77	0.00603199961128419\\
49.78	0.0060320374416303\\
49.79	0.0060320752906495\\
49.8	0.00603211315846753\\
49.81	0.00603215104521264\\
49.82	0.00603218895101548\\
49.83	0.00603222687600899\\
49.84	0.00603226482032826\\
49.85	0.00603230278411044\\
49.86	0.00603234076749455\\
49.87	0.00603237877062133\\
49.88	0.00603241679363307\\
49.89	0.00603245483667341\\
49.9	0.00603249289988711\\
49.91	0.00603253098341984\\
49.92	0.00603256908741793\\
49.93	0.00603260721202809\\
49.94	0.00603264535739712\\
49.95	0.00603268352367159\\
49.96	0.00603272171099752\\
49.97	0.00603275991951999\\
49.98	0.00603279814938277\\
49.99	0.00603283640072791\\
50	0.00603287467369525\\
50.01	0.006032912968422\\
50.02	0.00603295128504218\\
50.03	0.00603298962368609\\
50.04	0.00603302798447976\\
50.05	0.00603306636754427\\
50.06	0.00603310477299516\\
50.07	0.00603314320094168\\
50.08	0.00603318165148607\\
50.09	0.00603322012472276\\
50.1	0.00603325862073754\\
50.11	0.00603329713960666\\
50.12	0.00603333568139592\\
50.13	0.00603337424615964\\
50.14	0.00603341283393959\\
50.15	0.00603345144476395\\
50.16	0.00603349007865749\\
50.17	0.00603352873564504\\
50.18	0.00603356741575143\\
50.19	0.00603360611900152\\
50.2	0.00603364484542017\\
50.21	0.00603368359503231\\
50.22	0.00603372236786284\\
50.23	0.00603376116393671\\
50.24	0.00603379998327886\\
50.25	0.00603383882591427\\
50.26	0.00603387769186792\\
50.27	0.00603391658116482\\
50.28	0.00603395549382999\\
50.29	0.00603399442988845\\
50.3	0.00603403338936526\\
50.31	0.00603407237228547\\
50.32	0.00603411137867417\\
50.33	0.00603415040855643\\
50.34	0.00603418946195736\\
50.35	0.00603422853890208\\
50.36	0.00603426763941572\\
50.37	0.00603430676352342\\
50.38	0.00603434591125035\\
50.39	0.0060343850826217\\
50.4	0.00603442427766267\\
50.41	0.00603446349639849\\
50.42	0.0060345027388544\\
50.43	0.00603454200505567\\
50.44	0.00603458129502763\\
50.45	0.00603462060879559\\
50.46	0.00603465994638494\\
50.47	0.00603469930782109\\
50.48	0.0060347386931295\\
50.49	0.00603477810233569\\
50.5	0.0060348175354652\\
50.51	0.00603485699254365\\
50.52	0.00603489647359669\\
50.53	0.00603493597865004\\
50.54	0.00603497550772944\\
50.55	0.00603501506086071\\
50.56	0.0060350546380697\\
50.57	0.00603509423938233\\
50.58	0.00603513386482455\\
50.59	0.00603517351442238\\
50.6	0.00603521318820187\\
50.61	0.00603525288618914\\
50.62	0.00603529260841037\\
50.63	0.00603533235489177\\
50.64	0.00603537212565961\\
50.65	0.00603541192074022\\
50.66	0.00603545174015998\\
50.67	0.00603549158394531\\
50.68	0.00603553145212271\\
50.69	0.00603557134471871\\
50.7	0.00603561126175991\\
50.71	0.00603565120327295\\
50.72	0.00603569116928455\\
50.73	0.00603573115982145\\
50.74	0.00603577117491048\\
50.75	0.0060358112145785\\
50.76	0.00603585127885243\\
50.77	0.00603589136775927\\
50.78	0.00603593148132605\\
50.79	0.00603597161957987\\
50.8	0.00603601178254789\\
50.81	0.0060360519702573\\
50.82	0.00603609218273539\\
50.83	0.00603613242000948\\
50.84	0.00603617268210696\\
50.85	0.00603621296905528\\
50.86	0.00603625328088193\\
50.87	0.00603629361761449\\
50.88	0.00603633397928058\\
50.89	0.00603637436590789\\
50.9	0.00603641477752416\\
50.91	0.00603645521415721\\
50.92	0.00603649567583489\\
50.93	0.00603653616258516\\
50.94	0.00603657667443598\\
50.95	0.00603661721141544\\
50.96	0.00603665777355164\\
50.97	0.00603669836087277\\
50.98	0.00603673897340708\\
50.99	0.00603677961118287\\
51	0.00603682027422854\\
51.01	0.00603686096257251\\
51.02	0.00603690167624331\\
51.03	0.0060369424152695\\
51.04	0.00603698317967972\\
51.05	0.00603702396950268\\
51.06	0.00603706478476716\\
51.07	0.006037105625502\\
51.08	0.00603714649173612\\
51.09	0.00603718738349849\\
51.1	0.00603722830081817\\
51.11	0.00603726924372427\\
51.12	0.00603731021224599\\
51.13	0.00603735120641259\\
51.14	0.0060373922262534\\
51.15	0.00603743327179783\\
51.16	0.00603747434307535\\
51.17	0.00603751544011552\\
51.18	0.00603755656294795\\
51.19	0.00603759771160236\\
51.2	0.00603763888610849\\
51.21	0.00603768008649622\\
51.22	0.00603772131279546\\
51.23	0.0060377625650362\\
51.24	0.00603780384324854\\
51.25	0.00603784514746262\\
51.26	0.00603788647770867\\
51.27	0.00603792783401701\\
51.28	0.00603796921641802\\
51.29	0.00603801062494218\\
51.3	0.00603805205962003\\
51.31	0.00603809352048222\\
51.32	0.00603813500755945\\
51.33	0.00603817652088252\\
51.34	0.00603821806048231\\
51.35	0.00603825962638979\\
51.36	0.006038301218636\\
51.37	0.00603834283725207\\
51.38	0.00603838448226924\\
51.39	0.00603842615371879\\
51.4	0.00603846785163214\\
51.41	0.00603850957604076\\
51.42	0.00603855132697621\\
51.43	0.00603859310447018\\
51.44	0.0060386349085544\\
51.45	0.00603867673926073\\
51.46	0.00603871859662109\\
51.47	0.0060387604806675\\
51.48	0.00603880239143211\\
51.49	0.00603884432894712\\
51.5	0.00603888629324486\\
51.51	0.00603892828435771\\
51.52	0.00603897030231819\\
51.53	0.00603901234715892\\
51.54	0.00603905441891258\\
51.55	0.00603909651761199\\
51.56	0.00603913864329004\\
51.57	0.00603918079597975\\
51.58	0.00603922297571422\\
51.59	0.00603926518252667\\
51.6	0.00603930741645042\\
51.61	0.00603934967751888\\
51.62	0.0060393919657656\\
51.63	0.0060394342812242\\
51.64	0.00603947662392844\\
51.65	0.00603951899391218\\
51.66	0.00603956139120937\\
51.67	0.00603960381585411\\
51.68	0.00603964626788059\\
51.69	0.00603968874732312\\
51.7	0.00603973125421612\\
51.71	0.00603977378859413\\
51.72	0.00603981635049181\\
51.73	0.00603985893994395\\
51.74	0.00603990155698543\\
51.75	0.00603994420165128\\
51.76	0.00603998687397666\\
51.77	0.00604002957399681\\
51.78	0.00604007230174714\\
51.79	0.00604011505726317\\
51.8	0.00604015784058055\\
51.81	0.00604020065173505\\
51.82	0.00604024349076258\\
51.83	0.0060402863576992\\
51.84	0.00604032925258107\\
51.85	0.0060403721754445\\
51.86	0.00604041512632595\\
51.87	0.006040458105262\\
51.88	0.00604050111228938\\
51.89	0.00604054414744496\\
51.9	0.00604058721076574\\
51.91	0.00604063030228889\\
51.92	0.00604067342205171\\
51.93	0.00604071657009164\\
51.94	0.00604075974644629\\
51.95	0.00604080295115341\\
51.96	0.00604084618425091\\
51.97	0.00604088944577683\\
51.98	0.00604093273576942\\
51.99	0.00604097605426702\\
52	0.00604101940130818\\
52.01	0.00604106277693161\\
52.02	0.00604110618117615\\
52.03	0.00604114961408084\\
52.04	0.00604119307568487\\
52.05	0.00604123656602762\\
52.06	0.00604128008514862\\
52.07	0.00604132363308757\\
52.08	0.00604136720988439\\
52.09	0.00604141081557912\\
52.1	0.00604145445021202\\
52.11	0.00604149811382351\\
52.12	0.00604154180645424\\
52.13	0.00604158552814497\\
52.14	0.00604162927893673\\
52.15	0.00604167305887069\\
52.16	0.00604171686798824\\
52.17	0.00604176070633095\\
52.18	0.0060418045739406\\
52.19	0.00604184847085917\\
52.2	0.00604189239712885\\
52.21	0.00604193635279203\\
52.22	0.00604198033789131\\
52.23	0.0060420243524695\\
52.24	0.00604206839656965\\
52.25	0.00604211247023499\\
52.26	0.006042156573509\\
52.27	0.00604220070643537\\
52.28	0.00604224486905802\\
52.29	0.00604228906142109\\
52.3	0.00604233328356899\\
52.31	0.00604237753554632\\
52.32	0.00604242181739794\\
52.33	0.00604246612916896\\
52.34	0.00604251047090472\\
52.35	0.00604255484265082\\
52.36	0.0060425992444531\\
52.37	0.00604264367635768\\
52.38	0.00604268813841091\\
52.39	0.00604273263065941\\
52.4	0.00604277715315009\\
52.41	0.0060428217059301\\
52.42	0.00604286628904687\\
52.43	0.00604291090254814\\
52.44	0.00604295554648187\\
52.45	0.00604300022089636\\
52.46	0.00604304492584017\\
52.47	0.00604308966136217\\
52.48	0.00604313442751151\\
52.49	0.00604317922433766\\
52.5	0.00604322405189037\\
52.51	0.00604326891021974\\
52.52	0.00604331379937614\\
52.53	0.00604335871941028\\
52.54	0.0060434036703732\\
52.55	0.00604344865231627\\
52.56	0.00604349366529117\\
52.57	0.00604353870934992\\
52.58	0.00604358378454492\\
52.59	0.00604362889092888\\
52.6	0.00604367402855486\\
52.61	0.00604371919747631\\
52.62	0.00604376439774701\\
52.63	0.00604380962942113\\
52.64	0.00604385489255319\\
52.65	0.00604390018719811\\
52.66	0.0060439455134112\\
52.67	0.00604399087124812\\
52.68	0.00604403626076498\\
52.69	0.00604408168201825\\
52.7	0.00604412713506483\\
52.71	0.00604417261996202\\
52.72	0.00604421813676754\\
52.73	0.00604426368553956\\
52.74	0.00604430926633664\\
52.75	0.00604435487921782\\
52.76	0.00604440052424257\\
52.77	0.0060444462014708\\
52.78	0.0060444919109629\\
52.79	0.0060445376527797\\
52.8	0.00604458342698255\\
52.81	0.00604462923363321\\
52.82	0.006044675072794\\
52.83	0.00604472094452768\\
52.84	0.00604476684889755\\
52.85	0.0060448127859674\\
52.86	0.00604485875580154\\
52.87	0.00604490475846482\\
52.88	0.0060449507940226\\
52.89	0.0060449968625408\\
52.9	0.0060450429640859\\
52.91	0.00604508909872491\\
52.92	0.00604513526652543\\
52.93	0.00604518146755564\\
52.94	0.00604522770188429\\
52.95	0.00604527396958073\\
52.96	0.00604532027071493\\
52.97	0.00604536660535746\\
52.98	0.00604541297357951\\
52.99	0.0060454593754529\\
53	0.00604550581105012\\
53.01	0.00604555228044429\\
53.02	0.0060455987837092\\
53.03	0.00604564532091931\\
53.04	0.00604569189214977\\
53.05	0.00604573849747642\\
53.06	0.00604578513697583\\
53.07	0.00604583181072526\\
53.08	0.0060458785188027\\
53.09	0.00604592526128692\\
53.1	0.00604597203825738\\
53.11	0.00604601884979437\\
53.12	0.00604606569597892\\
53.13	0.00604611257689285\\
53.14	0.0060461594926188\\
53.15	0.00604620644324021\\
53.16	0.00604625342884137\\
53.17	0.00604630044950739\\
53.18	0.00604634750532425\\
53.19	0.00604639459637878\\
53.2	0.00604644172275873\\
53.21	0.00604648888455272\\
53.22	0.00604653608185029\\
53.23	0.0060465833147419\\
53.24	0.00604663058331897\\
53.25	0.00604667788767387\\
53.26	0.00604672522789995\\
53.27	0.00604677260409154\\
53.28	0.00604682001634398\\
53.29	0.00604686746475362\\
53.3	0.00604691494941789\\
53.31	0.00604696247043524\\
53.32	0.00604701002790519\\
53.33	0.00604705762192838\\
53.34	0.00604710525260655\\
53.35	0.00604715292004255\\
53.36	0.0060472006243404\\
53.37	0.00604724836560528\\
53.38	0.00604729614394353\\
53.39	0.00604734395946275\\
53.4	0.00604739181227171\\
53.41	0.00604743970248045\\
53.42	0.00604748763020026\\
53.43	0.00604753559554375\\
53.44	0.0060475835986248\\
53.45	0.00604763163955865\\
53.46	0.00604767971846187\\
53.47	0.00604772783545242\\
53.48	0.00604777599064964\\
53.49	0.00604782418417432\\
53.5	0.00604787241614867\\
53.51	0.00604792068669637\\
53.52	0.0060479689959426\\
53.53	0.00604801734401408\\
53.54	0.00604806573103902\\
53.55	0.00604811415714726\\
53.56	0.0060481626224702\\
53.57	0.00604821112714088\\
53.58	0.00604825967129398\\
53.59	0.00604830825506586\\
53.6	0.00604835687859459\\
53.61	0.00604840554201997\\
53.62	0.00604845424548358\\
53.63	0.00604850298912878\\
53.64	0.00604855177310076\\
53.65	0.00604860059754656\\
53.66	0.00604864946261512\\
53.67	0.00604869836845729\\
53.68	0.00604874731522586\\
53.69	0.00604879630307563\\
53.7	0.0060488453321634\\
53.71	0.00604889440264803\\
53.72	0.00604894351469046\\
53.73	0.00604899266845376\\
53.74	0.00604904186410315\\
53.75	0.00604909110180605\\
53.76	0.00604914038173212\\
53.77	0.00604918970405328\\
53.78	0.00604923906894376\\
53.79	0.00604928847658013\\
53.8	0.00604933792714138\\
53.81	0.00604938742080888\\
53.82	0.00604943695776651\\
53.83	0.00604948653820065\\
53.84	0.00604953616230022\\
53.85	0.00604958583025675\\
53.86	0.00604963554226442\\
53.87	0.00604968529852009\\
53.88	0.00604973509922334\\
53.89	0.00604978494457655\\
53.9	0.00604983483478491\\
53.91	0.0060498847700565\\
53.92	0.00604993475060231\\
53.93	0.00604998477663632\\
53.94	0.00605003484837552\\
53.95	0.00605008496603997\\
53.96	0.00605013512985289\\
53.97	0.00605018534004063\\
53.98	0.00605023559683282\\
53.99	0.00605028590046237\\
54	0.00605033625116553\\
54.01	0.00605038664918195\\
54.02	0.00605043709475476\\
54.03	0.00605048758813058\\
54.04	0.00605053812955965\\
54.05	0.00605058871929582\\
54.06	0.00605063935759665\\
54.07	0.00605069004472348\\
54.08	0.00605074078094147\\
54.09	0.00605079156651968\\
54.1	0.00605084240173112\\
54.11	0.00605089328685285\\
54.12	0.00605094422216601\\
54.13	0.00605099520795593\\
54.14	0.00605104624451215\\
54.15	0.00605109733212853\\
54.16	0.00605114847110333\\
54.17	0.00605119966173925\\
54.18	0.00605125090434353\\
54.19	0.00605130219922802\\
54.2	0.00605135354670926\\
54.21	0.00605140494710856\\
54.22	0.00605145640075208\\
54.23	0.0060515079079709\\
54.24	0.00605155946910112\\
54.25	0.00605161108448396\\
54.26	0.00605166275446579\\
54.27	0.00605171447939827\\
54.28	0.00605176625963844\\
54.29	0.00605181809554877\\
54.3	0.00605186998749728\\
54.31	0.00605192193585763\\
54.32	0.00605197394100921\\
54.33	0.00605202600333725\\
54.34	0.0060520781232329\\
54.35	0.00605213030109334\\
54.36	0.00605218253732188\\
54.37	0.00605223483232807\\
54.38	0.00605228718652777\\
54.39	0.00605233960034331\\
54.4	0.00605239207420357\\
54.41	0.00605244460854406\\
54.42	0.0060524972038071\\
54.43	0.00605254986044189\\
54.44	0.0060526025789046\\
54.45	0.00605265535965853\\
54.46	0.00605270820317425\\
54.47	0.00605276110992963\\
54.48	0.00605281408041006\\
54.49	0.00605286711510851\\
54.5	0.0060529202145257\\
54.51	0.00605297337917017\\
54.52	0.0060530266095585\\
54.53	0.00605307990621534\\
54.54	0.00605313326967362\\
54.55	0.00605318670047464\\
54.56	0.00605324019916826\\
54.57	0.00605329376631298\\
54.58	0.00605334740247612\\
54.59	0.00605340110823396\\
54.6	0.00605345488417187\\
54.61	0.00605350873088448\\
54.62	0.00605356264897584\\
54.63	0.00605361663905952\\
54.64	0.00605367070175883\\
54.65	0.00605372483770696\\
54.66	0.0060537790475471\\
54.67	0.00605383333193264\\
54.68	0.00605388769152738\\
54.69	0.00605394212700557\\
54.7	0.0060539966390522\\
54.71	0.00605405122836313\\
54.72	0.00605410589564521\\
54.73	0.00605416064161652\\
54.74	0.00605421546700651\\
54.75	0.00605427037255618\\
54.76	0.00605432535901825\\
54.77	0.00605438042715737\\
54.78	0.00605443557775027\\
54.79	0.00605449081158596\\
54.8	0.00605454612946591\\
54.81	0.00605460153220425\\
54.82	0.00605465702062793\\
54.83	0.00605471259557698\\
54.84	0.0060547682579046\\
54.85	0.00605482400847747\\
54.86	0.00605487984817584\\
54.87	0.00605493577789386\\
54.88	0.00605499179853962\\
54.89	0.00605504791103551\\
54.9	0.00605510411631833\\
54.91	0.00605516041533951\\
54.92	0.00605521680906538\\
54.93	0.00605527329847729\\
54.94	0.00605532988457189\\
54.95	0.00605538656836133\\
54.96	0.00605544335087348\\
54.97	0.00605550023315211\\
54.98	0.00605555721625717\\
54.99	0.00605561430126495\\
55	0.00605567148926836\\
55.01	0.00605572878137712\\
55.02	0.00605578617871798\\
55.03	0.00605584368243497\\
55.04	0.00605590129368961\\
55.05	0.00605595901366115\\
55.06	0.00605601684354678\\
55.07	0.00605607478456188\\
55.08	0.00605613283794024\\
55.09	0.0060561910049343\\
55.1	0.00605624928681537\\
55.11	0.00605630768487387\\
55.12	0.00605636620041956\\
55.13	0.00605642483478177\\
55.14	0.00605648358930962\\
55.15	0.00605654246537231\\
55.16	0.00605660146435926\\
55.17	0.0060566605876804\\
55.18	0.00605671983676641\\
55.19	0.00605677921306891\\
55.2	0.0060568387180607\\
55.21	0.00605689835323601\\
55.22	0.0060569581201107\\
55.23	0.00605701802022249\\
55.24	0.00605707805513118\\
55.25	0.00605713822641887\\
55.26	0.00605719853569019\\
55.27	0.00605725898457249\\
55.28	0.00605731957471608\\
55.29	0.00605738030779439\\
55.3	0.00605744118550424\\
55.31	0.00605750220956597\\
55.32	0.00605756338172369\\
55.33	0.00605762470374543\\
55.34	0.00605768617742337\\
55.35	0.00605774780457397\\
55.36	0.00605780958703818\\
55.37	0.00605787152668158\\
55.38	0.00605793362539458\\
55.39	0.00605799588509253\\
55.4	0.00605805830771589\\
55.41	0.00605812089523037\\
55.42	0.00605818364962702\\
55.43	0.00605824657292243\\
55.44	0.00605830966715875\\
55.45	0.00605837293440385\\
55.46	0.00605843637675137\\
55.47	0.00605849999632084\\
55.48	0.00605856379525768\\
55.49	0.00605862777573333\\
55.5	0.00605869193994521\\
55.51	0.00605875629011679\\
55.52	0.00605882082849759\\
55.53	0.00605888555736316\\
55.54	0.00605895047901506\\
55.55	0.00605901559578081\\
55.56	0.00605908091001382\\
55.57	0.00605914642409335\\
55.58	0.00605921214042434\\
55.59	0.00605927806143735\\
55.6	0.00605934418958834\\
55.61	0.00605941052735859\\
55.62	0.0060594770772544\\
55.63	0.00605954384180694\\
55.64	0.00605961082357196\\
55.65	0.00605967802512956\\
55.66	0.00605974544908378\\
55.67	0.00605981309806237\\
55.68	0.00605988097471634\\
55.69	0.00605994908171957\\
55.7	0.00606001742176836\\
55.71	0.00606008599758097\\
55.72	0.00606015481189704\\
55.73	0.00606022386747709\\
55.74	0.00606029316710187\\
55.75	0.00606036271357169\\
55.76	0.00606043250970576\\
55.77	0.00606050255834144\\
55.78	0.00606057286233338\\
55.79	0.00606064342455274\\
55.8	0.0060607142478862\\
55.81	0.00606078533523509\\
55.82	0.00606085668951426\\
55.83	0.00606092831365104\\
55.84	0.00606100021058409\\
55.85	0.00606107238326213\\
55.86	0.00606114483464268\\
55.87	0.00606121756769069\\
55.88	0.00606129058537705\\
55.89	0.0060613638906771\\
55.9	0.00606143748656901\\
55.91	0.00606151137603205\\
55.92	0.00606158556204484\\
55.93	0.00606166004758344\\
55.94	0.00606173483561938\\
55.95	0.00606180992911754\\
55.96	0.00606188533103401\\
55.97	0.00606196104431377\\
55.98	0.00606203707188825\\
55.99	0.00606211341667283\\
56	0.00606219008156417\\
56.01	0.0060622670694374\\
56.02	0.00606234438314325\\
56.03	0.00606242202550493\\
56.04	0.006062499999315\\
56.05	0.00606257830733195\\
56.06	0.00606265695227673\\
56.07	0.00606273593682907\\
56.08	0.00606281526362368\\
56.09	0.00606289493524616\\
56.1	0.0060629749542289\\
56.11	0.00606305532304665\\
56.12	0.00606313604411197\\
56.13	0.00606321711977045\\
56.14	0.00606329855229573\\
56.15	0.00606338034388431\\
56.16	0.00606346249665011\\
56.17	0.00606354501261883\\
56.18	0.00606362789372205\\
56.19	0.00606371114179108\\
56.2	0.00606379475855055\\
56.21	0.0060638787456117\\
56.22	0.00606396310446548\\
56.23	0.00606404783647526\\
56.24	0.00606413294286928\\
56.25	0.0060642184247328\\
56.26	0.00606430428299991\\
56.27	0.0060643905207675\\
56.28	0.0060644771455524\\
56.29	0.00606456416505369\\
56.3	0.00606465158715751\\
56.31	0.00606473941994189\\
56.32	0.0060648276716818\\
56.33	0.00606491635085426\\
56.34	0.00606500546614368\\
56.35	0.00606509502644725\\
56.36	0.00606518504088055\\
56.37	0.00606527551878329\\
56.38	0.0060653664697252\\
56.39	0.00606545790351209\\
56.4	0.00606554983019208\\
56.41	0.006065642260062\\
56.42	0.00606573520367395\\
56.43	0.00606582867184208\\
56.44	0.00606592267564952\\
56.45	0.0060660172264555\\
56.46	0.00606611233590273\\
56.47	0.00606620801592488\\
56.48	0.00606630427875437\\
56.49	0.00606640113693028\\
56.5	0.00606649860330659\\
56.51	0.00606659669106056\\
56.52	0.00606669541370137\\
56.53	0.006066794785079\\
56.54	0.00606689481939336\\
56.55	0.00606699553120371\\
56.56	0.00606709693543825\\
56.57	0.0060671990474041\\
56.58	0.00606730188279741\\
56.59	0.00606740545771394\\
56.6	0.00606750978865974\\
56.61	0.00606761489256229\\
56.62	0.00606772078678183\\
56.63	0.00606782748912312\\
56.64	0.0060679350178474\\
56.65	0.00606804339168483\\
56.66	0.00606815262984716\\
56.67	0.0060682627520408\\
56.68	0.00606837377848031\\
56.69	0.00606848572990216\\
56.7	0.00606859862757898\\
56.71	0.00606871249333414\\
56.72	0.00606882734955679\\
56.73	0.00606894321921731\\
56.74	0.00606906012588315\\
56.75	0.0060691780937352\\
56.76	0.00606929714758459\\
56.77	0.00606941731288989\\
56.78	0.00606953861577494\\
56.79	0.00606966108304707\\
56.8	0.00606978474221584\\
56.81	0.00606990962151243\\
56.82	0.00607003574990941\\
56.83	0.00607016315714121\\
56.84	0.00607029187372508\\
56.85	0.00607042193098271\\
56.86	0.00607055336106242\\
56.87	0.00607068619696198\\
56.88	0.00607082047255217\\
56.89	0.00607095622260084\\
56.9	0.00607109348279781\\
56.91	0.00607123228978041\\
56.92	0.00607137268115975\\
56.93	0.00607151469554775\\
56.94	0.0060716569932977\\
56.95	0.00607179935664574\\
56.96	0.00607194178563709\\
56.97	0.00607208428031703\\
56.98	0.00607222684073088\\
56.99	0.00607236946692398\\
57	0.00607251215894176\\
57.01	0.00607265491682967\\
57.02	0.00607279774063322\\
57.03	0.00607294063039794\\
57.04	0.00607308358616944\\
57.05	0.00607322660799336\\
57.06	0.00607336969591538\\
57.07	0.00607351284998126\\
57.08	0.00607365607023677\\
57.09	0.00607379935672774\\
57.1	0.00607394270950005\\
57.11	0.00607408612859963\\
57.12	0.00607422961407245\\
57.13	0.00607437316596454\\
57.14	0.00607451678432195\\
57.15	0.00607466046919082\\
57.16	0.00607480422061729\\
57.17	0.00607494803864759\\
57.18	0.00607509192332796\\
57.19	0.00607523587470473\\
57.2	0.00607537989282424\\
57.21	0.00607552397773289\\
57.22	0.00607566812947715\\
57.23	0.0060758123481035\\
57.24	0.00607595663365849\\
57.25	0.00607610098618872\\
57.26	0.00607624540574083\\
57.27	0.00607638989236152\\
57.28	0.00607653444609752\\
57.29	0.00607667906699562\\
57.3	0.00607682375510267\\
57.31	0.00607696851046554\\
57.32	0.00607711333313117\\
57.33	0.00607725822314655\\
57.34	0.0060774031805587\\
57.35	0.00607754820541472\\
57.36	0.00607769329776171\\
57.37	0.00607783845764688\\
57.38	0.00607798368511743\\
57.39	0.00607812898022066\\
57.4	0.00607827434300389\\
57.41	0.00607841977351449\\
57.42	0.00607856527179988\\
57.43	0.00607871083790755\\
57.44	0.00607885647188501\\
57.45	0.00607900217377984\\
57.46	0.00607914794363966\\
57.47	0.00607929378151214\\
57.48	0.00607943968744501\\
57.49	0.00607958566148604\\
57.5	0.00607973170368305\\
57.51	0.00607987781408391\\
57.52	0.00608002399273653\\
57.53	0.00608017023968891\\
57.54	0.00608031655498904\\
57.55	0.00608046293868502\\
57.56	0.00608060939082496\\
57.57	0.00608075591145703\\
57.58	0.00608090250062946\\
57.59	0.00608104915839051\\
57.6	0.00608119588478852\\
57.61	0.00608134267987186\\
57.62	0.00608148954368895\\
57.63	0.00608163647628827\\
57.64	0.00608178347771834\\
57.65	0.00608193054802776\\
57.66	0.00608207768726513\\
57.67	0.00608222489547915\\
57.68	0.00608237217271853\\
57.69	0.00608251951903208\\
57.7	0.00608266693446862\\
57.71	0.00608281441907703\\
57.72	0.00608296197290624\\
57.73	0.00608310959600525\\
57.74	0.0060832572884231\\
57.75	0.00608340505020888\\
57.76	0.00608355288141172\\
57.77	0.00608370078208082\\
57.78	0.00608384875226543\\
57.79	0.00608399679201485\\
57.8	0.00608414490137842\\
57.81	0.00608429308040555\\
57.82	0.00608444132914569\\
57.83	0.00608458964764834\\
57.84	0.00608473803596307\\
57.85	0.00608488649413947\\
57.86	0.00608503502222723\\
57.87	0.00608518362027604\\
57.88	0.00608533228833568\\
57.89	0.00608548102645597\\
57.9	0.00608562983468678\\
57.91	0.00608577871307804\\
57.92	0.00608592766167972\\
57.93	0.00608607668054186\\
57.94	0.00608622576971454\\
57.95	0.00608637492924789\\
57.96	0.00608652415919212\\
57.97	0.00608667345959745\\
57.98	0.00608682283051419\\
57.99	0.00608697227199269\\
58	0.00608712178408335\\
58.01	0.00608727136683662\\
58.02	0.00608742102030303\\
58.03	0.00608757074453312\\
58.04	0.00608772053957752\\
58.05	0.0060878704054869\\
58.06	0.00608802034231199\\
58.07	0.00608817035010356\\
58.08	0.00608832042891244\\
58.09	0.00608847057878953\\
58.1	0.00608862079978577\\
58.11	0.00608877109195215\\
58.12	0.00608892145533972\\
58.13	0.00608907188999958\\
58.14	0.0060892223959829\\
58.15	0.00608937297334088\\
58.16	0.0060895236221248\\
58.17	0.00608967434238597\\
58.18	0.00608982513417577\\
58.19	0.00608997599754564\\
58.2	0.00609012693254706\\
58.21	0.00609027793923157\\
58.22	0.00609042901765076\\
58.23	0.0060905801678563\\
58.24	0.00609073138989989\\
58.25	0.00609088268383328\\
58.26	0.0060910340497083\\
58.27	0.00609118548757682\\
58.28	0.00609133699749076\\
58.29	0.00609148857950212\\
58.3	0.00609164023366294\\
58.31	0.00609179196002529\\
58.32	0.00609194375864134\\
58.33	0.0060920956295633\\
58.34	0.00609224757284343\\
58.35	0.00609239958853404\\
58.36	0.00609255167668751\\
58.37	0.00609270383735628\\
58.38	0.00609285607059282\\
58.39	0.00609300837644969\\
58.4	0.00609316075497948\\
58.41	0.00609331320623485\\
58.42	0.00609346573026852\\
58.43	0.00609361832713325\\
58.44	0.00609377099688188\\
58.45	0.00609392373956727\\
58.46	0.00609407655524238\\
58.47	0.00609422944396021\\
58.48	0.0060943824057738\\
58.49	0.00609453544073627\\
58.5	0.00609468854890078\\
58.51	0.00609484173032057\\
58.52	0.00609499498504892\\
58.53	0.00609514831313916\\
58.54	0.0060953017146447\\
58.55	0.00609545518961899\\
58.56	0.00609560873811555\\
58.57	0.00609576236018795\\
58.58	0.00609591605588982\\
58.59	0.00609606982527484\\
58.6	0.00609622366839676\\
58.61	0.00609637758530938\\
58.62	0.00609653157606657\\
58.63	0.00609668564072225\\
58.64	0.00609683977933039\\
58.65	0.00609699399194503\\
58.66	0.00609714827862027\\
58.67	0.00609730263941026\\
58.68	0.00609745707436921\\
58.69	0.00609761158355139\\
58.7	0.00609776616701114\\
58.71	0.00609792082480284\\
58.72	0.00609807555698095\\
58.73	0.00609823036359996\\
58.74	0.00609838524471445\\
58.75	0.00609854020037904\\
58.76	0.00609869523064842\\
58.77	0.00609885033557732\\
58.78	0.00609900551522056\\
58.79	0.006099160769633\\
58.8	0.00609931609886956\\
58.81	0.00609947150298523\\
58.82	0.00609962698203505\\
58.83	0.00609978253607411\\
58.84	0.00609993816515758\\
58.85	0.00610009386934069\\
58.86	0.00610024964867872\\
58.87	0.00610040550322702\\
58.88	0.00610056143304098\\
58.89	0.00610071743817607\\
58.9	0.00610087351868781\\
58.91	0.0061010296746318\\
58.92	0.00610118590606367\\
58.93	0.00610134221303913\\
58.94	0.00610149859561396\\
58.95	0.00610165505384398\\
58.96	0.00610181158778507\\
58.97	0.0061019681974932\\
58.98	0.00610212488302437\\
58.99	0.00610228164443465\\
59	0.00610243848178019\\
59.01	0.00610259539511717\\
59.02	0.00610275238450186\\
59.03	0.00610290944999057\\
59.04	0.00610306659163968\\
59.05	0.00610322380950565\\
59.06	0.00610338110364497\\
59.07	0.00610353847411421\\
59.08	0.00610369592097001\\
59.09	0.00610385344426904\\
59.1	0.00610401104406807\\
59.11	0.0061041687204239\\
59.12	0.00610432647339344\\
59.13	0.0061044843030336\\
59.14	0.00610464220940139\\
59.15	0.00610480019255389\\
59.16	0.00610495825254821\\
59.17	0.00610511638944156\\
59.18	0.00610527460329118\\
59.19	0.0061054328941544\\
59.2	0.00610559126208859\\
59.21	0.00610574970715121\\
59.22	0.00610590822939976\\
59.23	0.0061060668288918\\
59.24	0.00610622550568499\\
59.25	0.006106384259837\\
59.26	0.00610654309140562\\
59.27	0.00610670200044865\\
59.28	0.006106860987024\\
59.29	0.00610702005118962\\
59.3	0.00610717919300352\\
59.31	0.00610733841252379\\
59.32	0.00610749770980857\\
59.33	0.00610765708491608\\
59.34	0.00610781653790459\\
59.35	0.00610797606883244\\
59.36	0.00610813567775803\\
59.37	0.00610829536473984\\
59.38	0.0061084551298364\\
59.39	0.0061086149731063\\
59.4	0.00610877489460822\\
59.41	0.00610893489440087\\
59.42	0.00610909497254307\\
59.43	0.00610925512909366\\
59.44	0.00610941536411157\\
59.45	0.0061095756776558\\
59.46	0.00610973606978538\\
59.47	0.00610989654055946\\
59.48	0.00611005709003722\\
59.49	0.00611021771827792\\
59.5	0.00611037842534086\\
59.51	0.00611053921128544\\
59.52	0.0061107000761711\\
59.53	0.00611086102005738\\
59.54	0.00611102204300384\\
59.55	0.00611118314507015\\
59.56	0.00611134432631602\\
59.57	0.00611150558680123\\
59.58	0.00611166692658564\\
59.59	0.00611182834572916\\
59.6	0.00611198984429178\\
59.61	0.00611215142233355\\
59.62	0.00611231307991458\\
59.63	0.00611247481709506\\
59.64	0.00611263663393526\\
59.65	0.00611279853049549\\
59.66	0.00611296050683613\\
59.67	0.00611312256301764\\
59.68	0.00611328469910055\\
59.69	0.00611344691514544\\
59.7	0.00611360921121299\\
59.71	0.00611377158736391\\
59.72	0.006113934043659\\
59.73	0.00611409658015912\\
59.74	0.00611425919692521\\
59.75	0.00611442189401828\\
59.76	0.00611458467149937\\
59.77	0.00611474752942964\\
59.78	0.00611491046787029\\
59.79	0.0061150734868826\\
59.8	0.00611523658652791\\
59.81	0.00611539976686763\\
59.82	0.00611556302796325\\
59.83	0.00611572636987631\\
59.84	0.00611588979266844\\
59.85	0.00611605329640134\\
59.86	0.00611621688113675\\
59.87	0.00611638054693651\\
59.88	0.00611654429386252\\
59.89	0.00611670812197675\\
59.9	0.00611687203134124\\
59.91	0.00611703602201809\\
59.92	0.00611720009406949\\
59.93	0.00611736424755769\\
59.94	0.006117528482545\\
59.95	0.00611769279909381\\
59.96	0.00611785719726659\\
59.97	0.00611802167712588\\
59.98	0.00611818623873426\\
59.99	0.00611835088215443\\
60	0.00611851560744911\\
60.01	0.00611868041468113\\
60.02	0.00611884530391337\\
60.03	0.00611901027520879\\
60.04	0.00611917532863042\\
60.05	0.00611934046424137\\
60.06	0.0061195056821048\\
60.07	0.00611967098228396\\
60.08	0.00611983636484216\\
60.09	0.0061200018298428\\
60.1	0.00612016737734933\\
60.11	0.0061203330074253\\
60.12	0.00612049872013429\\
60.13	0.00612066451553999\\
60.14	0.00612083039370616\\
60.15	0.0061209963546966\\
60.16	0.00612116239857523\\
60.17	0.006121328525406\\
60.18	0.00612149473525295\\
60.19	0.00612166102818021\\
60.2	0.00612182740425196\\
60.21	0.00612199386353245\\
60.22	0.00612216040608603\\
60.23	0.00612232703197709\\
60.24	0.00612249374127012\\
60.25	0.00612266053402969\\
60.26	0.0061228274103204\\
60.27	0.00612299437020697\\
60.28	0.00612316141375418\\
60.29	0.00612332854102686\\
60.3	0.00612349575208996\\
60.31	0.00612366304700847\\
60.32	0.00612383042584745\\
60.33	0.00612399788867207\\
60.34	0.00612416543554755\\
60.35	0.00612433306653917\\
60.36	0.00612450078171231\\
60.37	0.00612466858113244\\
60.38	0.00612483646486506\\
60.39	0.00612500443297577\\
60.4	0.00612517248553025\\
60.41	0.00612534062259425\\
60.42	0.0061255088442336\\
60.43	0.00612567715051419\\
60.44	0.006125845541502\\
60.45	0.00612601401726309\\
60.46	0.00612618257786358\\
60.47	0.00612635122336968\\
60.48	0.00612651995384767\\
60.49	0.00612668876936391\\
60.5	0.00612685766998483\\
60.51	0.00612702665577695\\
60.52	0.00612719572680685\\
60.53	0.0061273648831412\\
60.54	0.00612753412484674\\
60.55	0.00612770345199029\\
60.56	0.00612787286463875\\
60.57	0.00612804236285908\\
60.58	0.00612821194671835\\
60.59	0.00612838161628367\\
60.6	0.00612855137162227\\
60.61	0.00612872121280142\\
60.62	0.00612889113988849\\
60.63	0.00612906115295091\\
60.64	0.00612923125205621\\
60.65	0.00612940143727198\\
60.66	0.0061295717086659\\
60.67	0.00612974206630573\\
60.68	0.0061299125102593\\
60.69	0.00613008304059452\\
60.7	0.00613025365737938\\
60.71	0.00613042436068195\\
60.72	0.00613059515057039\\
60.73	0.00613076602711292\\
60.74	0.00613093699037785\\
60.75	0.00613110804043357\\
60.76	0.00613127917734855\\
60.77	0.00613145040119133\\
60.78	0.00613162171203054\\
60.79	0.0061317931099349\\
60.8	0.00613196459497319\\
60.81	0.00613213616721427\\
60.82	0.0061323078267271\\
60.83	0.00613247957358071\\
60.84	0.0061326514078442\\
60.85	0.00613282332958678\\
60.86	0.0061329953388777\\
60.87	0.00613316743578633\\
60.88	0.00613333962038209\\
60.89	0.00613351189273452\\
60.9	0.0061336842529132\\
60.91	0.00613385670098781\\
60.92	0.00613402923702811\\
60.93	0.00613420186110395\\
60.94	0.00613437457328525\\
60.95	0.00613454737364202\\
60.96	0.00613472026224435\\
60.97	0.00613489323916241\\
60.98	0.00613506630446645\\
60.99	0.00613523945822682\\
61	0.00613541270051392\\
61.01	0.00613558603139828\\
61.02	0.00613575945095047\\
61.03	0.00613593295924115\\
61.04	0.00613610655634108\\
61.05	0.00613628024232112\\
61.06	0.00613645401725215\\
61.07	0.0061366278812052\\
61.08	0.00613680183425136\\
61.09	0.00613697587646178\\
61.1	0.00613715000790773\\
61.11	0.00613732422866055\\
61.12	0.00613749853879166\\
61.13	0.00613767293837258\\
61.14	0.00613784742747489\\
61.15	0.00613802200617027\\
61.16	0.00613819667453049\\
61.17	0.0061383714326274\\
61.18	0.00613854628053293\\
61.19	0.00613872121831909\\
61.2	0.00613889624605801\\
61.21	0.00613907136382185\\
61.22	0.00613924657168291\\
61.23	0.00613942186971355\\
61.24	0.00613959725798621\\
61.25	0.00613977273657343\\
61.26	0.00613994830554782\\
61.27	0.0061401239649821\\
61.28	0.00614029971494907\\
61.29	0.0061404755555216\\
61.3	0.00614065148677266\\
61.31	0.0061408275087753\\
61.32	0.00614100362160268\\
61.33	0.00614117982532802\\
61.34	0.00614135612002462\\
61.35	0.00614153250576592\\
61.36	0.00614170898262539\\
61.37	0.00614188555067662\\
61.38	0.00614206220999327\\
61.39	0.0061422389606491\\
61.4	0.00614241580271797\\
61.41	0.00614259273627379\\
61.42	0.00614276976139061\\
61.43	0.00614294687814252\\
61.44	0.00614312408660373\\
61.45	0.00614330138684853\\
61.46	0.00614347877895131\\
61.47	0.00614365626298652\\
61.48	0.00614383383902873\\
61.49	0.00614401150715258\\
61.5	0.00614418926743282\\
61.51	0.00614436711994426\\
61.52	0.00614454506476184\\
61.53	0.00614472310196056\\
61.54	0.00614490123161551\\
61.55	0.0061450794538019\\
61.56	0.00614525776859499\\
61.57	0.00614543617607017\\
61.58	0.00614561467630289\\
61.59	0.00614579326936871\\
61.6	0.00614597195534327\\
61.61	0.00614615073430232\\
61.62	0.00614632960632167\\
61.63	0.00614650857147725\\
61.64	0.00614668762984509\\
61.65	0.00614686678150127\\
61.66	0.006147046026522\\
61.67	0.00614722536498357\\
61.68	0.00614740479696236\\
61.69	0.00614758432253484\\
61.7	0.00614776394177759\\
61.71	0.00614794365476727\\
61.72	0.00614812346158063\\
61.73	0.00614830336229452\\
61.74	0.00614848335698589\\
61.75	0.00614866344573176\\
61.76	0.00614884362860928\\
61.77	0.00614902390569565\\
61.78	0.00614920427706821\\
61.79	0.00614938474280437\\
61.8	0.00614956530298164\\
61.81	0.00614974595767761\\
61.82	0.00614992670696999\\
61.83	0.00615010755093656\\
61.84	0.00615028848965522\\
61.85	0.00615046952320394\\
61.86	0.00615065065166082\\
61.87	0.00615083187510402\\
61.88	0.00615101319361181\\
61.89	0.00615119460726256\\
61.9	0.00615137611613474\\
61.91	0.0061515577203069\\
61.92	0.0061517394198577\\
61.93	0.00615192121486589\\
61.94	0.00615210310541033\\
61.95	0.00615228509156995\\
61.96	0.00615246717342382\\
61.97	0.00615264935105106\\
61.98	0.00615283162453092\\
61.99	0.00615301399394273\\
62	0.00615319645936594\\
62.01	0.00615337902088007\\
62.02	0.00615356167856475\\
62.03	0.00615374443249973\\
62.04	0.00615392728276482\\
62.05	0.00615411022943996\\
62.06	0.00615429327260517\\
62.07	0.00615447641234058\\
62.08	0.00615465964872642\\
62.09	0.00615484298184301\\
62.1	0.00615502641177078\\
62.11	0.00615520993859025\\
62.12	0.00615539356238205\\
62.13	0.0061555772832269\\
62.14	0.00615576110120563\\
62.15	0.00615594501639918\\
62.16	0.00615612902888856\\
62.17	0.0061563131387549\\
62.18	0.00615649734607944\\
62.19	0.00615668165094351\\
62.2	0.00615686605342853\\
62.21	0.00615705055361605\\
62.22	0.0061572351515877\\
62.23	0.00615741984742522\\
62.24	0.00615760464121046\\
62.25	0.00615778953302534\\
62.26	0.00615797452295193\\
62.27	0.00615815961107238\\
62.28	0.00615834479746893\\
62.29	0.00615853008222394\\
62.3	0.00615871546541986\\
62.31	0.00615890094713927\\
62.32	0.00615908652746483\\
62.33	0.00615927220647931\\
62.34	0.00615945798426558\\
62.35	0.00615964386090663\\
62.36	0.00615982983648553\\
62.37	0.00616001591108548\\
62.38	0.00616020208478977\\
62.39	0.00616038835768179\\
62.4	0.00616057472984506\\
62.41	0.00616076120136318\\
62.42	0.00616094777231987\\
62.43	0.00616113444279895\\
62.44	0.00616132121288434\\
62.45	0.00616150808266007\\
62.46	0.0061616950522103\\
62.47	0.00616188212161927\\
62.48	0.00616206929097133\\
62.49	0.00616225656035093\\
62.5	0.00616244392984266\\
62.51	0.00616263139953118\\
62.52	0.00616281896950128\\
62.53	0.00616300663983785\\
62.54	0.00616319441062588\\
62.55	0.00616338228195049\\
62.56	0.0061635702538969\\
62.57	0.00616375832655042\\
62.58	0.0061639464999965\\
62.59	0.00616413477432067\\
62.6	0.00616432314960858\\
62.61	0.006164511625946\\
62.62	0.0061647002034188\\
62.63	0.00616488888211296\\
62.64	0.00616507766211457\\
62.65	0.00616526654350983\\
62.66	0.00616545552638506\\
62.67	0.00616564461082667\\
62.68	0.00616583379692121\\
62.69	0.00616602308475531\\
62.7	0.00616621247441573\\
62.71	0.00616640196598935\\
62.72	0.00616659155956312\\
62.73	0.00616678125522417\\
62.74	0.00616697105305967\\
62.75	0.00616716095315695\\
62.76	0.00616735095560344\\
62.77	0.00616754106048668\\
62.78	0.00616773126789432\\
62.79	0.00616792157791413\\
62.8	0.006168111990634\\
62.81	0.0061683025061419\\
62.82	0.00616849312452596\\
62.83	0.00616868384587439\\
62.84	0.00616887467027554\\
62.85	0.00616906559781784\\
62.86	0.00616925662858987\\
62.87	0.00616944776268031\\
62.88	0.00616963900017795\\
62.89	0.0061698303411717\\
62.9	0.00617002178575059\\
62.91	0.00617021333400376\\
62.92	0.00617040498602046\\
62.93	0.00617059674189008\\
62.94	0.0061707886017021\\
62.95	0.00617098056554612\\
62.96	0.00617117263351187\\
62.97	0.00617136480568919\\
62.98	0.00617155708216804\\
62.99	0.00617174946303849\\
63	0.00617194194839074\\
63.01	0.00617213453831509\\
63.02	0.00617232723290196\\
63.03	0.00617252003224192\\
63.04	0.00617271293642562\\
63.05	0.00617290594554384\\
63.06	0.00617309905968749\\
63.07	0.00617329227894759\\
63.08	0.00617348560341527\\
63.09	0.0061736790331818\\
63.1	0.00617387256833855\\
63.11	0.00617406620897704\\
63.12	0.00617425995518887\\
63.13	0.00617445380706578\\
63.14	0.00617464776469963\\
63.15	0.00617484182818241\\
63.16	0.00617503599760622\\
63.17	0.00617523027306328\\
63.18	0.00617542465464593\\
63.19	0.00617561914244664\\
63.2	0.00617581373655799\\
63.21	0.0061760084370727\\
63.22	0.0061762032440836\\
63.23	0.00617639815768364\\
63.24	0.00617659317796589\\
63.25	0.00617678830502356\\
63.26	0.00617698353894997\\
63.27	0.00617717887983857\\
63.28	0.00617737432778291\\
63.29	0.0061775698828767\\
63.3	0.00617776554521374\\
63.31	0.00617796131488799\\
63.32	0.0061781571919935\\
63.33	0.00617835317662446\\
63.34	0.00617854926887519\\
63.35	0.00617874546884011\\
63.36	0.0061789417766138\\
63.37	0.00617913819229095\\
63.38	0.00617933471596636\\
63.39	0.00617953134773497\\
63.4	0.00617972808769185\\
63.41	0.00617992493593219\\
63.42	0.00618012189255131\\
63.43	0.00618031895764464\\
63.44	0.00618051613130776\\
63.45	0.00618071341363636\\
63.46	0.00618091080472626\\
63.47	0.00618110830467342\\
63.48	0.00618130591357391\\
63.49	0.00618150363152393\\
63.5	0.00618170145861982\\
63.51	0.00618189939495804\\
63.52	0.00618209744063517\\
63.53	0.00618229559574793\\
63.54	0.00618249386039315\\
63.55	0.00618269223466782\\
63.56	0.00618289071866904\\
63.57	0.00618308931249403\\
63.58	0.00618328801624015\\
63.59	0.00618348683000489\\
63.6	0.00618368575388587\\
63.61	0.00618388478798083\\
63.62	0.00618408393238764\\
63.63	0.00618428318720432\\
63.64	0.00618448255252899\\
63.65	0.00618468202845993\\
63.66	0.00618488161509552\\
63.67	0.0061850813125343\\
63.68	0.00618528112087492\\
63.69	0.00618548104021616\\
63.7	0.00618568107065695\\
63.71	0.00618588121229632\\
63.72	0.00618608146523347\\
63.73	0.00618628182956769\\
63.74	0.00618648230539844\\
63.75	0.00618668289282528\\
63.76	0.00618688359194791\\
63.77	0.00618708440286618\\
63.78	0.00618728532568006\\
63.79	0.00618748636048963\\
63.8	0.00618768750739513\\
63.81	0.00618788876649693\\
63.82	0.00618809013789552\\
63.83	0.00618829162169152\\
63.84	0.0061884932179857\\
63.85	0.00618869492687894\\
63.86	0.00618889674847229\\
63.87	0.00618909868286688\\
63.88	0.00618930073016401\\
63.89	0.0061895028904651\\
63.9	0.0061897051638717\\
63.91	0.00618990755048551\\
63.92	0.00619011005040835\\
63.93	0.00619031266374216\\
63.94	0.00619051539058904\\
63.95	0.00619071823105121\\
63.96	0.00619092118523101\\
63.97	0.00619112425323093\\
63.98	0.0061913274351536\\
63.99	0.00619153073110176\\
64	0.00619173414117829\\
64.01	0.00619193766548622\\
64.02	0.0061921413041287\\
64.03	0.00619234505720902\\
64.04	0.00619254892483058\\
64.05	0.00619275290709694\\
64.06	0.00619295700411179\\
64.07	0.00619316121597894\\
64.08	0.00619336554280233\\
64.09	0.00619356998468606\\
64.1	0.00619377454173434\\
64.11	0.00619397921405153\\
64.12	0.00619418400174208\\
64.13	0.00619438890491064\\
64.14	0.00619459392366194\\
64.15	0.00619479905810086\\
64.16	0.00619500430833241\\
64.17	0.00619520967446175\\
64.18	0.00619541515659414\\
64.19	0.006195620754835\\
64.2	0.00619582646928987\\
64.21	0.00619603230006442\\
64.22	0.00619623824726445\\
64.23	0.00619644431099591\\
64.24	0.00619665049136487\\
64.25	0.00619685678847753\\
64.26	0.0061970632024402\\
64.27	0.00619726973335937\\
64.28	0.00619747638134162\\
64.29	0.00619768314649368\\
64.3	0.0061978900289224\\
64.31	0.00619809702873477\\
64.32	0.0061983041460379\\
64.33	0.00619851138093904\\
64.34	0.00619871873354557\\
64.35	0.00619892620396498\\
64.36	0.00619913379230491\\
64.37	0.00619934149867312\\
64.38	0.0061995493231775\\
64.39	0.00619975726592608\\
64.4	0.00619996532702699\\
64.41	0.00620017350658851\\
64.42	0.00620038180471905\\
64.43	0.00620059022152713\\
64.44	0.0062007987571214\\
64.45	0.00620100741161064\\
64.46	0.00620121618510377\\
64.47	0.00620142507770982\\
64.48	0.00620163408953793\\
64.49	0.0062018432206974\\
64.5	0.00620205247129763\\
64.51	0.00620226184144814\\
64.52	0.0062024713312586\\
64.53	0.00620268094083877\\
64.54	0.00620289067029855\\
64.55	0.00620310051974797\\
64.56	0.00620331048929716\\
64.57	0.00620352057905639\\
64.58	0.00620373078913602\\
64.59	0.00620394111964657\\
64.6	0.00620415157069867\\
64.61	0.00620436214240304\\
64.62	0.00620457283487053\\
64.63	0.00620478364821213\\
64.64	0.00620499458253893\\
64.65	0.00620520563796212\\
64.66	0.00620541681459304\\
64.67	0.00620562811254311\\
64.68	0.0062058395319239\\
64.69	0.00620605107284704\\
64.7	0.00620626273542434\\
64.71	0.00620647451976766\\
64.72	0.00620668642598901\\
64.73	0.0062068984542005\\
64.74	0.00620711060451434\\
64.75	0.00620732287704286\\
64.76	0.00620753527189849\\
64.77	0.00620774778919377\\
64.78	0.00620796042904135\\
64.79	0.00620817319155397\\
64.8	0.0062083860768445\\
64.81	0.00620859908502588\\
64.82	0.00620881221621118\\
64.83	0.00620902547051355\\
64.84	0.00620923884804627\\
64.85	0.00620945234892268\\
64.86	0.00620966597325626\\
64.87	0.00620987972116054\\
64.88	0.00621009359274919\\
64.89	0.00621030758813595\\
64.9	0.00621052170743465\\
64.91	0.00621073595075924\\
64.92	0.00621095031822373\\
64.93	0.00621116480994224\\
64.94	0.00621137942602897\\
64.95	0.0062115941665982\\
64.96	0.00621180903176433\\
64.97	0.00621202402164181\\
64.98	0.00621223913634518\\
64.99	0.00621245437598907\\
65	0.00621266974068821\\
65.01	0.00621288523055736\\
65.02	0.00621310084571141\\
65.03	0.0062133165862653\\
65.04	0.00621353245233405\\
65.05	0.00621374844403276\\
65.06	0.00621396456147659\\
65.07	0.00621418080478079\\
65.08	0.00621439717406066\\
65.09	0.00621461366943158\\
65.1	0.006214830291009\\
65.11	0.00621504703890844\\
65.12	0.00621526391324544\\
65.13	0.00621548091413566\\
65.14	0.00621569804169479\\
65.15	0.00621591529603858\\
65.16	0.00621613267728286\\
65.17	0.00621635018554346\\
65.18	0.00621656782093633\\
65.19	0.00621678558357743\\
65.2	0.00621700347358278\\
65.21	0.00621722149106844\\
65.22	0.00621743963615054\\
65.23	0.00621765790894523\\
65.24	0.00621787630956871\\
65.25	0.00621809483813722\\
65.26	0.00621831349476705\\
65.27	0.00621853227957451\\
65.28	0.00621875119267595\\
65.29	0.00621897023418777\\
65.3	0.00621918940422637\\
65.31	0.00621940870290819\\
65.32	0.00621962813034972\\
65.33	0.00621984768666744\\
65.34	0.00622006737197786\\
65.35	0.00622028718639755\\
65.36	0.00622050713004304\\
65.37	0.00622072720303092\\
65.38	0.00622094740547777\\
65.39	0.00622116773750019\\
65.4	0.00622138819921479\\
65.41	0.00622160879073819\\
65.42	0.00622182951218701\\
65.43	0.00622205036367788\\
65.44	0.00622227134532743\\
65.45	0.00622249245725229\\
65.46	0.00622271369956907\\
65.47	0.0062229350723944\\
65.48	0.00622315657584489\\
65.49	0.00622337821003713\\
65.5	0.00622359997508773\\
65.51	0.00622382187111324\\
65.52	0.00622404389823024\\
65.53	0.00622426605655525\\
65.54	0.0062244883462048\\
65.55	0.00622471076729538\\
65.56	0.00622493331994346\\
65.57	0.00622515600426548\\
65.58	0.00622537882037786\\
65.59	0.00622560176839697\\
65.6	0.00622582484843916\\
65.61	0.00622604806062073\\
65.62	0.00622627140505796\\
65.63	0.00622649488186707\\
65.64	0.00622671849116425\\
65.65	0.00622694223306564\\
65.66	0.00622716610768734\\
65.67	0.00622739011514538\\
65.68	0.00622761425555577\\
65.69	0.00622783852903444\\
65.7	0.00622806293569727\\
65.71	0.0062282874756601\\
65.72	0.00622851214903869\\
65.73	0.00622873695594876\\
65.74	0.00622896189650594\\
65.75	0.00622918697082582\\
65.76	0.00622941217902391\\
65.77	0.00622963752121567\\
65.78	0.00622986299751646\\
65.79	0.0062300886080416\\
65.8	0.00623031435290633\\
65.81	0.0062305402322258\\
65.82	0.0062307662461151\\
65.83	0.00623099239468925\\
65.84	0.00623121867806318\\
65.85	0.00623144509635174\\
65.86	0.00623167164966973\\
65.87	0.00623189833813182\\
65.88	0.00623212516185264\\
65.89	0.00623235212094673\\
65.9	0.00623257921552854\\
65.91	0.00623280644571246\\
65.92	0.00623303381161275\\
65.93	0.00623326131334364\\
65.94	0.00623348895101926\\
65.95	0.00623371672475364\\
65.96	0.00623394463466075\\
65.97	0.00623417268085447\\
65.98	0.00623440086344861\\
65.99	0.00623462918255688\\
66	0.00623485763829294\\
66.01	0.00623508623077034\\
66.02	0.00623531496010257\\
66.03	0.00623554382640306\\
66.04	0.00623577282978514\\
66.05	0.00623600197036208\\
66.06	0.0062362312482471\\
66.07	0.00623646066355334\\
66.08	0.00623669021639387\\
66.09	0.0062369199068817\\
66.1	0.00623714973512982\\
66.11	0.00623737970125111\\
66.12	0.00623760980535844\\
66.13	0.00623784004756463\\
66.14	0.00623807042798246\\
66.15	0.00623830094672467\\
66.16	0.00623853160390397\\
66.17	0.00623876239963305\\
66.18	0.00623899333402458\\
66.19	0.00623922440719123\\
66.2	0.00623945561924565\\
66.21	0.00623968697030049\\
66.22	0.00623991846046843\\
66.23	0.00624015008986215\\
66.24	0.00624038185859437\\
66.25	0.00624061376677784\\
66.26	0.00624084581452535\\
66.27	0.00624107800194976\\
66.28	0.00624131032916397\\
66.29	0.006241542796281\\
66.3	0.0062417754034139\\
66.31	0.00624200815067589\\
66.32	0.00624224103818025\\
66.33	0.00624247406604042\\
66.34	0.00624270723436999\\
66.35	0.00624294054328268\\
66.36	0.00624317399289241\\
66.37	0.00624340758331332\\
66.38	0.00624364131465971\\
66.39	0.00624387518704617\\
66.4	0.00624410920058752\\
66.41	0.00624434335539885\\
66.42	0.00624457765159556\\
66.43	0.00624481208929336\\
66.44	0.00624504666860831\\
66.45	0.00624528138965686\\
66.46	0.00624551625255582\\
66.47	0.00624575125742245\\
66.48	0.00624598640437445\\
66.49	0.00624622169353001\\
66.5	0.00624645712500782\\
66.51	0.00624669269892713\\
66.52	0.00624692841540774\\
66.53	0.0062471642745701\\
66.54	0.00624740027653527\\
66.55	0.00624763642142503\\
66.56	0.00624787270936185\\
66.57	0.006248109140469\\
66.58	0.00624834571487051\\
66.59	0.0062485824326913\\
66.6	0.00624881929405717\\
66.61	0.00624905629909486\\
66.62	0.00624929344793209\\
66.63	0.00624953074069763\\
66.64	0.00624976817752136\\
66.65	0.00625000575853426\\
66.66	0.00625024348386856\\
66.67	0.00625048135365772\\
66.68	0.00625071936803653\\
66.69	0.00625095752714116\\
66.7	0.00625119583110922\\
66.71	0.00625143428007986\\
66.72	0.00625167287419377\\
66.73	0.00625191161359333\\
66.74	0.00625215049842264\\
66.75	0.00625238952882758\\
66.76	0.00625262870495595\\
66.77	0.0062528680269575\\
66.78	0.00625310749498401\\
66.79	0.00625334710918942\\
66.8	0.0062535868697299\\
66.81	0.00625382677676394\\
66.82	0.00625406683045243\\
66.83	0.0062543070309588\\
66.84	0.00625454737844908\\
66.85	0.00625478787309205\\
66.86	0.00625502851505931\\
66.87	0.00625526930452541\\
66.88	0.006255510241668\\
66.89	0.00625575132666786\\
66.9	0.00625599255970915\\
66.91	0.00625623394097942\\
66.92	0.00625647547066982\\
66.93	0.00625671714897522\\
66.94	0.00625695897609434\\
66.95	0.00625720095222989\\
66.96	0.00625744307758874\\
66.97	0.00625768535238207\\
66.98	0.00625792777682552\\
66.99	0.00625817035113938\\
67	0.00625841307554872\\
67.01	0.0062586559502836\\
67.02	0.00625889897557925\\
67.03	0.00625914215167623\\
67.04	0.00625938547882064\\
67.05	0.00625962895726434\\
67.06	0.00625987258726509\\
67.07	0.00626011636908684\\
67.08	0.00626036030299989\\
67.09	0.00626060438928112\\
67.1	0.00626084862812658\\
67.11	0.00626109301971804\\
67.12	0.00626133756423789\\
67.13	0.00626158226186913\\
67.14	0.0062618271127954\\
67.15	0.00626207211720095\\
67.16	0.00626231727527067\\
67.17	0.00626256258719013\\
67.18	0.00626280805314549\\
67.19	0.00626305367332364\\
67.2	0.00626329944791208\\
67.21	0.00626354537709901\\
67.22	0.00626379146107331\\
67.23	0.00626403770002455\\
67.24	0.00626428409414299\\
67.25	0.00626453064361959\\
67.26	0.00626477734864605\\
67.27	0.00626502420941477\\
67.28	0.00626527122611889\\
67.29	0.00626551839895227\\
67.3	0.00626576572810953\\
67.31	0.00626601321378605\\
67.32	0.00626626085617798\\
67.33	0.00626650865548222\\
67.34	0.00626675661189647\\
67.35	0.00626700472561922\\
67.36	0.00626725299684976\\
67.37	0.00626750142578819\\
67.38	0.00626775001263543\\
67.39	0.00626799875759324\\
67.4	0.0062682476608642\\
67.41	0.00626849672265178\\
67.42	0.00626874594316026\\
67.43	0.00626899532259484\\
67.44	0.00626924486116158\\
67.45	0.00626949455906744\\
67.46	0.00626974441652027\\
67.47	0.00626999443372886\\
67.48	0.00627024461090291\\
67.49	0.00627049494825307\\
67.5	0.00627074544599095\\
67.51	0.00627099610432911\\
67.52	0.00627124692348109\\
67.53	0.00627149790366142\\
67.54	0.00627174904508562\\
67.55	0.00627200034797026\\
67.56	0.00627225181253289\\
67.57	0.00627250343899214\\
67.58	0.00627275522756768\\
67.59	0.00627300717848024\\
67.6	0.00627325929195165\\
67.61	0.00627351156820482\\
67.62	0.00627376400746379\\
67.63	0.00627401660995372\\
67.64	0.0062742693759009\\
67.65	0.0062745223055328\\
67.66	0.00627477539907803\\
67.67	0.00627502865676643\\
67.68	0.006275282078829\\
67.69	0.00627553566549799\\
67.7	0.00627578941700689\\
67.71	0.00627604333359043\\
67.72	0.00627629741548463\\
67.73	0.00627655166292676\\
67.74	0.00627680607615545\\
67.75	0.00627706065541062\\
67.76	0.00627731540093354\\
67.77	0.00627757031296686\\
67.78	0.00627782539175459\\
67.79	0.00627808063754215\\
67.8	0.0062783360505764\\
67.81	0.0062785916311056\\
67.82	0.00627884737937952\\
67.83	0.00627910329564937\\
67.84	0.00627935938016789\\
67.85	0.00627961563318935\\
67.86	0.00627987205496954\\
67.87	0.00628012864576585\\
67.88	0.00628038540583725\\
67.89	0.00628064233544431\\
67.9	0.00628089943484928\\
67.91	0.00628115670431604\\
67.92	0.00628141414411017\\
67.93	0.00628167175449896\\
67.94	0.00628192953575143\\
67.95	0.00628218748813839\\
67.96	0.00628244561193242\\
67.97	0.00628270390740792\\
67.98	0.00628296237484112\\
67.99	0.00628322101451017\\
68	0.00628347982669505\\
68.01	0.00628373881167773\\
68.02	0.00628399796974211\\
68.03	0.00628425730117407\\
68.04	0.00628451680626153\\
68.05	0.00628477648529444\\
68.06	0.00628503633856484\\
68.07	0.00628529636636688\\
68.08	0.00628555656899684\\
68.09	0.0062858169467532\\
68.1	0.00628607749993663\\
68.11	0.00628633822885006\\
68.12	0.0062865991337987\\
68.13	0.00628686021509007\\
68.14	0.00628712147303404\\
68.15	0.00628738290794285\\
68.16	0.00628764452013121\\
68.17	0.00628790630991627\\
68.18	0.00628816827761767\\
68.19	0.00628843042355762\\
68.2	0.00628869274806088\\
68.21	0.00628895525145487\\
68.22	0.00628921793406966\\
68.23	0.00628948079623801\\
68.24	0.00628974383829546\\
68.25	0.00629000706058034\\
68.26	0.00629027046343382\\
68.27	0.00629053404719994\\
68.28	0.0062907978122257\\
68.29	0.00629106175886107\\
68.3	0.00629132588745904\\
68.31	0.00629159019837571\\
68.32	0.00629185469197027\\
68.33	0.00629211936860512\\
68.34	0.00629238422864589\\
68.35	0.00629264927246148\\
68.36	0.00629291450042416\\
68.37	0.00629317991290956\\
68.38	0.00629344551029679\\
68.39	0.00629371129296847\\
68.4	0.00629397726131077\\
68.41	0.00629424341571351\\
68.42	0.00629450975657017\\
68.43	0.00629477628427802\\
68.44	0.00629504299923812\\
68.45	0.0062953099018554\\
68.46	0.00629557699253876\\
68.47	0.00629584427170109\\
68.48	0.00629611173975936\\
68.49	0.00629637939713471\\
68.5	0.00629664724425248\\
68.51	0.00629691528154229\\
68.52	0.00629718350943815\\
68.53	0.00629745192837851\\
68.54	0.00629772053880632\\
68.55	0.00629798934116911\\
68.56	0.00629825833591914\\
68.57	0.00629852752351336\\
68.58	0.00629879690441359\\
68.59	0.00629906647908658\\
68.6	0.00629933624800407\\
68.61	0.00629960621164289\\
68.62	0.00629987637048507\\
68.63	0.00630014672501791\\
68.64	0.00630041727573407\\
68.65	0.00630068802313169\\
68.66	0.00630095896771445\\
68.67	0.00630123010999169\\
68.68	0.00630150145047853\\
68.69	0.00630177298969593\\
68.7	0.00630204472817081\\
68.71	0.00630231666643617\\
68.72	0.00630258880503121\\
68.73	0.00630286114450138\\
68.74	0.00630313368539857\\
68.75	0.00630340642828117\\
68.76	0.00630367937371423\\
68.77	0.00630395252226952\\
68.78	0.00630422587452573\\
68.79	0.00630449943106852\\
68.8	0.00630477319249071\\
68.81	0.00630504715939237\\
68.82	0.00630532133238097\\
68.83	0.0063055957120715\\
68.84	0.00630587029908665\\
68.85	0.00630614509405687\\
68.86	0.00630642009762062\\
68.87	0.00630669531042443\\
68.88	0.00630697073312308\\
68.89	0.00630724636637976\\
68.9	0.00630752221086624\\
68.91	0.00630779826726299\\
68.92	0.00630807453625936\\
68.93	0.00630835101855376\\
68.94	0.00630862771485383\\
68.95	0.00630890462587658\\
68.96	0.00630918175234859\\
68.97	0.00630945909500622\\
68.98	0.00630973665459574\\
68.99	0.00631001443187353\\
69	0.00631029242760633\\
69.01	0.00631057064257134\\
69.02	0.0063108490775565\\
69.03	0.00631112773336066\\
69.04	0.0063114066107938\\
69.05	0.00631168571067722\\
69.06	0.00631196503384381\\
69.07	0.00631224458113819\\
69.08	0.00631252435341704\\
69.09	0.00631280435154923\\
69.1	0.00631308457641614\\
69.11	0.00631336502891185\\
69.12	0.0063136457099434\\
69.13	0.00631392662043107\\
69.14	0.00631420776130858\\
69.15	0.00631448913352343\\
69.16	0.00631477073803709\\
69.17	0.00631505257582534\\
69.18	0.00631533464787852\\
69.19	0.00631561695520182\\
69.2	0.00631589949881557\\
69.21	0.00631618227975554\\
69.22	0.00631646529907329\\
69.23	0.00631674855783639\\
69.24	0.00631703205712885\\
69.25	0.00631731579805136\\
69.26	0.00631759978172166\\
69.27	0.0063178840092749\\
69.28	0.00631816848186393\\
69.29	0.00631845320065975\\
69.3	0.00631873816685178\\
69.31	0.00631902338164828\\
69.32	0.00631930884627674\\
69.33	0.00631959456198423\\
69.34	0.00631988053003786\\
69.35	0.00632016675172512\\
69.36	0.00632045322835434\\
69.37	0.00632073996125511\\
69.38	0.00632102695177869\\
69.39	0.0063213142012985\\
69.4	0.00632160171121053\\
69.41	0.00632188948293385\\
69.42	0.00632217751791102\\
69.43	0.00632246581760868\\
69.44	0.00632275438351793\\
69.45	0.00632304321715493\\
69.46	0.00632333232006139\\
69.47	0.00632362169380508\\
69.48	0.00632391133998042\\
69.49	0.00632420126020899\\
69.5	0.00632449145614016\\
69.51	0.0063247819294516\\
69.52	0.00632507268184994\\
69.53	0.00632536371507135\\
69.54	0.00632565503088218\\
69.55	0.00632594663107958\\
69.56	0.00632623851749217\\
69.57	0.00632653069198071\\
69.58	0.00632682315643878\\
69.59	0.00632711591279348\\
69.6	0.00632740896300613\\
69.61	0.00632770230907307\\
69.62	0.00632799595302633\\
69.63	0.00632828989693444\\
69.64	0.00632858414290321\\
69.65	0.00632887869307656\\
69.66	0.0063291735496373\\
69.67	0.00632946871480801\\
69.68	0.00632976419085187\\
69.69	0.00633005998007361\\
69.7	0.00633035608482033\\
69.71	0.0063306525074825\\
69.72	0.00633094925049486\\
69.73	0.00633124631633743\\
69.74	0.00633154370753647\\
69.75	0.00633184142666554\\
69.76	0.00633213947634651\\
69.77	0.00633243785925065\\
69.78	0.0063327365780997\\
69.79	0.00633303563566702\\
69.8	0.00633333503477873\\
69.81	0.00633363477831487\\
69.82	0.00633393486921064\\
69.83	0.00633423531045759\\
69.84	0.00633453610510491\\
69.85	0.00633483725626074\\
69.86	0.00633513876709345\\
69.87	0.00633544064083305\\
69.88	0.00633574288077256\\
69.89	0.0063360454902694\\
69.9	0.00633634847274694\\
69.91	0.00633665183169591\\
69.92	0.00633695557067598\\
69.93	0.00633725969331733\\
69.94	0.00633756420332224\\
69.95	0.00633786910446677\\
69.96	0.00633817440060242\\
69.97	0.00633848009565789\\
69.98	0.00633878619364086\\
69.99	0.00633909269863978\\
70	0.00633939961482579\\
70.01	0.00633970694645457\\
70.02	0.00634001469786835\\
70.03	0.0063403228734979\\
70.04	0.00634063147786462\\
70.05	0.00634094051558261\\
70.06	0.00634124999136085\\
70.07	0.00634155991000546\\
70.08	0.00634187027642192\\
70.09	0.00634218109561747\\
70.1	0.00634249237270343\\
70.11	0.00634280411289772\\
70.12	0.00634311632152734\\
70.13	0.00634342900403098\\
70.14	0.00634374216596165\\
70.15	0.00634405581298939\\
70.16	0.00634436995090409\\
70.17	0.00634468458561829\\
70.18	0.00634499972317017\\
70.19	0.00634531536972653\\
70.2	0.00634563153158585\\
70.21	0.00634594821518149\\
70.22	0.00634626542708494\\
70.23	0.00634658317400909\\
70.24	0.00634690146281171\\
70.25	0.00634722030049892\\
70.26	0.00634753969422879\\
70.27	0.00634785965131501\\
70.28	0.00634818017923072\\
70.29	0.00634850128561236\\
70.3	0.00634882297826364\\
70.31	0.00634914526515965\\
70.32	0.00634946815445108\\
70.33	0.00634979165446846\\
70.34	0.00635011577372663\\
70.35	0.00635044052092927\\
70.36	0.00635076590497353\\
70.37	0.00635109193495483\\
70.38	0.00635141862017175\\
70.39	0.00635174597013109\\
70.4	0.00635207399455296\\
70.41	0.0063524027033762\\
70.42	0.00635273210676373\\
70.43	0.00635306221510817\\
70.44	0.00635339303903759\\
70.45	0.00635372458942139\\
70.46	0.00635405687737633\\
70.47	0.00635438991427279\\
70.48	0.00635472371174108\\
70.49	0.00635505828167807\\
70.5	0.0063553936362538\\
70.51	0.00635572978791852\\
70.52	0.00635606674940966\\
70.53	0.00635640453375918\\
70.54	0.00635674315430102\\
70.55	0.00635708262467879\\
70.56	0.00635742295885367\\
70.57	0.00635776417111246\\
70.58	0.00635810627607597\\
70.59	0.00635844928870751\\
70.6	0.0063587932243217\\
70.61	0.00635913809859347\\
70.62	0.00635948392756734\\
70.63	0.00635983072766693\\
70.64	0.00636017851570471\\
70.65	0.0063605273088921\\
70.66	0.00636087712484974\\
70.67	0.0063612279816181\\
70.68	0.0063615798976684\\
70.69	0.00636193289191378\\
70.7	0.00636228698372078\\
70.71	0.00636264219292121\\
70.72	0.00636299853982424\\
70.73	0.00636335604522892\\
70.74	0.00636371473043697\\
70.75	0.00636407461726597\\
70.76	0.00636443572806293\\
70.77	0.00636479808571815\\
70.78	0.00636516171367959\\
70.79	0.00636552663596751\\
70.8	0.00636589287718965\\
70.81	0.00636626046255671\\
70.82	0.00636662941789833\\
70.83	0.00636699976967952\\
70.84	0.00636737154501751\\
70.85	0.00636774477169911\\
70.86	0.00636811947819851\\
70.87	0.00636849569369568\\
70.88	0.00636887344809511\\
70.89	0.00636925277204527\\
70.9	0.0063696336969585\\
70.91	0.00637001625503146\\
70.92	0.00637040047926624\\
70.93	0.00637078640349199\\
70.94	0.00637117406238717\\
70.95	0.00637156349150243\\
70.96	0.00637195472728421\\
70.97	0.00637234780709888\\
70.98	0.0063727427692577\\
70.99	0.00637313965304236\\
71	0.00637353849873135\\
71.01	0.00637393934762701\\
71.02	0.00637434224208341\\
71.03	0.00637474722553497\\
71.04	0.00637515434252595\\
71.05	0.0063755636387407\\
71.06	0.00637597516103488\\
71.07	0.00637638895746752\\
71.08	0.00637680507733394\\
71.09	0.00637722357119975\\
71.1	0.00637764449093571\\
71.11	0.00637806788975363\\
71.12	0.00637849382224333\\
71.13	0.00637892234441059\\
71.14	0.0063793535137163\\
71.15	0.00637978738911662\\
71.16	0.00638022403110442\\
71.17	0.0063806635017518\\
71.18	0.00638110586475393\\
71.19	0.00638155118547412\\
71.2	0.00638199953099018\\
71.21	0.00638245097014216\\
71.22	0.00638290557358145\\
71.23	0.00638336341382136\\
71.24	0.00638382456528903\\
71.25	0.00638428910437905\\
71.26	0.00638475710950846\\
71.27	0.00638522866117349\\
71.28	0.00638570384200784\\
71.29	0.00638618273684275\\
71.3	0.00638666543276881\\
71.31	0.00638715201919952\\
71.32	0.00638764258793675\\
71.33	0.00638813723323817\\
71.34	0.00638863605188657\\
71.35	0.00638913914326125\\
71.36	0.00638964660941148\\
71.37	0.00639015855513222\\
71.38	0.00639067508804196\\
71.39	0.00639119631866283\\
71.4	0.00639172236050321\\
71.41	0.00639225333014266\\
71.42	0.00639278934731937\\
71.43	0.00639333053502026\\
71.44	0.00639387701957365\\
71.45	0.00639442893074475\\
71.46	0.00639498640183395\\
71.47	0.00639554956977799\\
71.48	0.00639611857525419\\
71.49	0.00639669356278776\\
71.5	0.00639727468086223\\
71.51	0.00639786208203327\\
71.52	0.00639845592304583\\
71.53	0.00639905636495475\\
71.54	0.00639966357324907\\
71.55	0.00640027771797991\\
71.56	0.00640089897389225\\
71.57	0.00640152752056066\\
71.58	0.00640216302649419\\
71.59	0.00640279888908074\\
71.6	0.0064034351086195\\
71.61	0.0064040716853987\\
71.62	0.0064047086196952\\
71.63	0.00640534591177394\\
71.64	0.00640598356188743\\
71.65	0.00640662157027524\\
71.66	0.00640725993716345\\
71.67	0.00640789866276403\\
71.68	0.00640853774727432\\
71.69	0.00640917719087639\\
71.7	0.00640981699373641\\
71.71	0.00641045715600402\\
71.72	0.00641109767781164\\
71.73	0.00641173855927379\\
71.74	0.0064123798004864\\
71.75	0.00641302140152605\\
71.76	0.0064136633624492\\
71.77	0.00641430568329144\\
71.78	0.00641494836406665\\
71.79	0.00641559140476619\\
71.8	0.00641623480535802\\
71.81	0.0064168785657858\\
71.82	0.00641752268596802\\
71.83	0.00641816716579699\\
71.84	0.00641881200513788\\
71.85	0.00641945720382776\\
71.86	0.00642010276167447\\
71.87	0.00642074867845562\\
71.88	0.00642139495391743\\
71.89	0.00642204158777361\\
71.9	0.00642268857970418\\
71.91	0.0064233359293542\\
71.92	0.00642398363633258\\
71.93	0.00642463170021074\\
71.94	0.00642528012052127\\
71.95	0.00642592889675655\\
71.96	0.00642657802836732\\
71.97	0.0064272275147612\\
71.98	0.00642787735530119\\
71.99	0.00642852754930406\\
72	0.00642917809603879\\
72.01	0.00642982899472481\\
72.02	0.00643048024453038\\
72.03	0.00643113184457072\\
72.04	0.00643178379390622\\
72.05	0.00643243609154052\\
72.06	0.0064330887364186\\
72.07	0.0064337417274247\\
72.08	0.00643439506338029\\
72.09	0.00643504874304186\\
72.1	0.0064357027650988\\
72.11	0.006436357128171\\
72.12	0.00643701183080659\\
72.13	0.00643766687147941\\
72.14	0.0064383222485866\\
72.15	0.0064389779604459\\
72.16	0.00643963400529308\\
72.17	0.00644029038127908\\
72.18	0.00644094708646723\\
72.19	0.00644160411883029\\
72.2	0.00644226147624741\\
72.21	0.00644291915650102\\
72.22	0.00644357715727355\\
72.23	0.00644423547614416\\
72.24	0.00644489411058528\\
72.25	0.00644555305795904\\
72.26	0.00644621231551363\\
72.27	0.00644687188037951\\
72.28	0.00644753174956552\\
72.29	0.00644819191995485\\
72.3	0.00644885238830088\\
72.31	0.00644951315122291\\
72.32	0.00645017420520172\\
72.33	0.00645083554657504\\
72.34	0.00645149717153276\\
72.35	0.00645215907611218\\
72.36	0.00645282125619289\\
72.37	0.00645348370749165\\
72.38	0.00645414642555705\\
72.39	0.00645480940576392\\
72.4	0.00645547264330773\\
72.41	0.00645613613319867\\
72.42	0.00645679987025556\\
72.43	0.00645746384909963\\
72.44	0.00645812806414804\\
72.45	0.0064587925096072\\
72.46	0.0064594571794659\\
72.47	0.00646012206748816\\
72.48	0.00646078716720597\\
72.49	0.00646145247191163\\
72.5	0.00646211797464996\\
72.51	0.00646278366821026\\
72.52	0.00646344954511789\\
72.53	0.00646411559762577\\
72.54	0.00646478181770539\\
72.55	0.00646544819703773\\
72.56	0.00646611472700369\\
72.57	0.0064667813986744\\
72.58	0.00646744820280106\\
72.59	0.00646811512980452\\
72.6	0.00646878216976456\\
72.61	0.00646944931240868\\
72.62	0.00647011654710071\\
72.63	0.0064707838628289\\
72.64	0.00647145124819367\\
72.65	0.00647211869139498\\
72.66	0.00647278618021928\\
72.67	0.00647345370202599\\
72.68	0.00647412124373358\\
72.69	0.00647478879180518\\
72.7	0.00647545633223372\\
72.71	0.00647612385052657\\
72.72	0.00647679133168966\\
72.73	0.00647745876021116\\
72.74	0.00647812612004449\\
72.75	0.00647879339459399\\
72.76	0.00647946056671667\\
72.77	0.00648012761870557\\
72.78	0.00648079453227256\\
72.79	0.0064814612885308\\
72.8	0.00648212786797647\\
72.81	0.00648279425047009\\
72.82	0.00648346041521725\\
72.83	0.00648412634074874\\
72.84	0.00648479200490015\\
72.85	0.00648545738479075\\
72.86	0.00648612245680185\\
72.87	0.00648678719655448\\
72.88	0.00648745157888638\\
72.89	0.00648811557782831\\
72.9	0.00648877916657968\\
72.91	0.00648944231748344\\
72.92	0.00649010500200018\\
72.93	0.00649076719068153\\
72.94	0.00649142885314272\\
72.95	0.00649208995803431\\
72.96	0.00649275047301311\\
72.97	0.00649341036471221\\
72.98	0.00649406959871011\\
72.99	0.006494729046915\\
73	0.00649538897871655\\
73.01	0.0064960493968363\\
73.02	0.0064967103040254\\
73.03	0.00649737170306447\\
73.04	0.00649803359676347\\
73.05	0.0064986959879615\\
73.06	0.00649935887952661\\
73.07	0.00650002227435557\\
73.08	0.00650068617537358\\
73.09	0.00650135058553399\\
73.1	0.00650201550781791\\
73.11	0.00650268094523387\\
73.12	0.00650334690081738\\
73.13	0.00650401337763049\\
73.14	0.00650468037876121\\
73.15	0.00650534790732307\\
73.16	0.00650601596645438\\
73.17	0.0065066845593177\\
73.18	0.00650735368909904\\
73.19	0.00650802335900713\\
73.2	0.00650869357227261\\
73.21	0.00650936433214711\\
73.22	0.00651003564190234\\
73.23	0.00651070750482904\\
73.24	0.00651137992423591\\
73.25	0.00651205290344848\\
73.26	0.0065127264458078\\
73.27	0.00651340055466923\\
73.28	0.00651407523340093\\
73.29	0.00651475048538248\\
73.3	0.00651542631400323\\
73.31	0.0065161027226607\\
73.32	0.00651677971475874\\
73.33	0.00651745729370572\\
73.34	0.00651813546291254\\
73.35	0.00651881422579052\\
73.36	0.00651949358574924\\
73.37	0.00652017354619417\\
73.38	0.00652085411052426\\
73.39	0.00652153528212935\\
73.4	0.00652221706438741\\
73.41	0.00652289946066176\\
73.42	0.00652358247429801\\
73.43	0.00652426610862089\\
73.44	0.00652495036693096\\
73.45	0.00652563525250112\\
73.46	0.00652632076857289\\
73.47	0.00652700691835267\\
73.48	0.0065276937050076\\
73.49	0.00652838113166138\\
73.5	0.00652906920138987\\
73.51	0.00652975791721638\\
73.52	0.00653044728210684\\
73.53	0.00653113729896472\\
73.54	0.00653182797062567\\
73.55	0.00653251929985195\\
73.56	0.00653321128932661\\
73.57	0.00653390394164736\\
73.58	0.0065345972593202\\
73.59	0.00653529124475276\\
73.6	0.00653598590024729\\
73.61	0.00653668122799345\\
73.62	0.00653737723006062\\
73.63	0.00653807390839\\
73.64	0.00653877126478631\\
73.65	0.00653946930090915\\
73.66	0.00654016801826395\\
73.67	0.00654086741819252\\
73.68	0.00654156750186328\\
73.69	0.00654226827026096\\
73.7	0.00654296972417591\\
73.71	0.00654367186419298\\
73.72	0.00654437469067989\\
73.73	0.00654507820377511\\
73.74	0.00654578240337527\\
73.75	0.00654648728912198\\
73.76	0.00654719286038818\\
73.77	0.00654789911626385\\
73.78	0.00654860605554115\\
73.79	0.00654931367669901\\
73.8	0.00655002197788699\\
73.81	0.00655073095690853\\
73.82	0.00655144061120353\\
73.83	0.0065521509378302\\
73.84	0.00655286193344618\\
73.85	0.00655357359428887\\
73.86	0.00655428591615506\\
73.87	0.00655499889437963\\
73.88	0.00655571252381347\\
73.89	0.00655642679880048\\
73.9	0.00655714171315371\\
73.91	0.00655785726013047\\
73.92	0.00655857343240657\\
73.93	0.00655929022204942\\
73.94	0.00656000762049019\\
73.95	0.00656072561849476\\
73.96	0.00656144420613366\\
73.97	0.00656216337275074\\
73.98	0.00656288310693064\\
73.99	0.00656360339646509\\
74	0.00656432422831772\\
74.01	0.00656504558858773\\
74.02	0.00656576746247198\\
74.03	0.0065664898342257\\
74.04	0.00656721268712175\\
74.05	0.00656793600340817\\
74.06	0.00656865976426428\\
74.07	0.00656938394975496\\
74.08	0.0065701085387833\\
74.09	0.0065708335090414\\
74.1	0.00657155885873774\\
74.11	0.00657228458806359\\
74.12	0.00657301069719022\\
74.13	0.00657373718626799\\
74.14	0.00657446405542531\\
74.15	0.00657519130476771\\
74.16	0.00657591893437674\\
74.17	0.00657664694430894\\
74.18	0.00657737533459466\\
74.19	0.00657810410523693\\
74.2	0.00657883325621026\\
74.21	0.00657956278745936\\
74.22	0.00658029269889788\\
74.23	0.00658102299040704\\
74.24	0.00658175366183424\\
74.25	0.00658248471299163\\
74.26	0.00658321614365462\\
74.27	0.00658394795356028\\
74.28	0.00658468014240583\\
74.29	0.00658541270984687\\
74.3	0.00658614565549574\\
74.31	0.0065868789789197\\
74.32	0.00658761267963908\\
74.33	0.00658834675712537\\
74.34	0.00658908121079926\\
74.35	0.00658981604002853\\
74.36	0.00659055124412601\\
74.37	0.00659128682234728\\
74.38	0.00659202277388846\\
74.39	0.00659275909788385\\
74.4	0.00659349579340343\\
74.41	0.00659423285945039\\
74.42	0.00659497029495847\\
74.43	0.00659570809878927\\
74.44	0.00659644626972945\\
74.45	0.00659718480648779\\
74.46	0.00659792370769222\\
74.47	0.00659866297188672\\
74.48	0.00659940259752802\\
74.49	0.00660014258298237\\
74.5	0.00660088292652201\\
74.51	0.00660162362632165\\
74.52	0.00660236468045475\\
74.53	0.00660310608688973\\
74.54	0.00660384784348595\\
74.55	0.00660458994798971\\
74.56	0.00660533239802994\\
74.57	0.00660607519111386\\
74.58	0.00660681832462244\\
74.59	0.00660756179580571\\
74.6	0.00660830560177793\\
74.61	0.00660904973951254\\
74.62	0.00660979420583701\\
74.63	0.00661053899742745\\
74.64	0.00661128411080308\\
74.65	0.00661202954232046\\
74.66	0.00661277528816758\\
74.67	0.00661352134435772\\
74.68	0.00661426770672308\\
74.69	0.00661501437090822\\
74.7	0.00661576133236329\\
74.71	0.00661650858633698\\
74.72	0.00661725612786927\\
74.73	0.00661800395178391\\
74.74	0.00661875205268068\\
74.75	0.00661950042492734\\
74.76	0.00662024906265131\\
74.77	0.00662099795973113\\
74.78	0.00662174710978756\\
74.79	0.0066224965061744\\
74.8	0.00662324614196902\\
74.81	0.00662399605618991\\
74.82	0.00662474636015234\\
74.83	0.006625497054455\\
74.84	0.00662624813969999\\
74.85	0.00662699961649284\\
74.86	0.00662775148544255\\
74.87	0.00662850374716159\\
74.88	0.0066292564022659\\
74.89	0.00663000945137495\\
74.9	0.0066307628951117\\
74.91	0.00663151673410265\\
74.92	0.00663227096897785\\
74.93	0.00663302560037089\\
74.94	0.00663378062891896\\
74.95	0.00663453605526279\\
74.96	0.00663529188004672\\
74.97	0.0066360481039187\\
74.98	0.00663680472753027\\
74.99	0.00663756175153659\\
75	0.00663831917659645\\
75.01	0.00663907700337228\\
75.02	0.00663983523253013\\
75.03	0.00664059386473969\\
75.04	0.00664135290067431\\
75.05	0.00664211234101096\\
75.06	0.00664287218643029\\
75.07	0.00664363243761656\\
75.08	0.0066443930952577\\
75.09	0.00664515416004529\\
75.1	0.00664591563267452\\
75.11	0.00664667751384425\\
75.12	0.00664743980425692\\
75.13	0.00664820250461864\\
75.14	0.0066489656156391\\
75.15	0.00664972913803159\\
75.16	0.006650493072513\\
75.17	0.00665125741980378\\
75.18	0.00665202218062792\\
75.19	0.00665278735571298\\
75.2	0.00665355294579\\
75.21	0.00665431895159355\\
75.22	0.00665508537386164\\
75.23	0.00665585221333573\\
75.24	0.00665661947076071\\
75.25	0.00665738714688483\\
75.26	0.00665815524245971\\
75.27	0.00665892375824028\\
75.28	0.00665969269498475\\
75.29	0.00666046205345455\\
75.3	0.00666123183441432\\
75.31	0.00666200203863185\\
75.32	0.00666277266687801\\
75.33	0.00666354371992673\\
75.34	0.00666431519855492\\
75.35	0.00666508710354243\\
75.36	0.00666585943567197\\
75.37	0.00666663219572907\\
75.38	0.00666740538450199\\
75.39	0.00666817900278164\\
75.4	0.00666895305136153\\
75.41	0.00666972753103768\\
75.42	0.00667050244260853\\
75.43	0.00667127778687485\\
75.44	0.00667205356463966\\
75.45	0.00667282977670811\\
75.46	0.00667360642388743\\
75.47	0.00667438350698675\\
75.48	0.00667516102681704\\
75.49	0.00667593898419099\\
75.5	0.00667671737992287\\
75.51	0.0066774962148284\\
75.52	0.00667827548972466\\
75.53	0.0066790552054299\\
75.54	0.00667983536276342\\
75.55	0.00668061596254542\\
75.56	0.00668139700559685\\
75.57	0.00668217849273923\\
75.58	0.0066829604247945\\
75.59	0.00668374280258482\\
75.6	0.00668452562693241\\
75.61	0.00668530889865935\\
75.62	0.00668609261858737\\
75.63	0.00668687678753768\\
75.64	0.0066876614063307\\
75.65	0.0066884464757859\\
75.66	0.00668923199672152\\
75.67	0.00669001796995435\\
75.68	0.00669080439629949\\
75.69	0.00669159127657007\\
75.7	0.00669237861157702\\
75.71	0.00669316640212874\\
75.72	0.00669395464903087\\
75.73	0.00669474335308592\\
75.74	0.00669553251509306\\
75.75	0.00669632213584769\\
75.76	0.00669711221614121\\
75.77	0.00669790275676058\\
75.78	0.00669869375848807\\
75.79	0.0066994852221008\\
75.8	0.0067002771483704\\
75.81	0.00670106953806261\\
75.82	0.00670186239193688\\
75.83	0.00670265571074596\\
75.84	0.0067034494952354\\
75.85	0.00670424374614319\\
75.86	0.00670503846419921\\
75.87	0.0067058336501248\\
75.88	0.00670662930463227\\
75.89	0.00670742542842431\\
75.9	0.00670822202219355\\
75.91	0.00670901908662194\\
75.92	0.00670981662238022\\
75.93	0.00671061463012731\\
75.94	0.00671141311050972\\
75.95	0.0067122120641609\\
75.96	0.00671301149170063\\
75.97	0.0067138113937343\\
75.98	0.00671461177085226\\
75.99	0.00671541262362909\\
76	0.00671621395262288\\
76.01	0.00671701575837442\\
76.02	0.00671781804140648\\
76.03	0.00671862080222296\\
76.04	0.00671942404130809\\
76.05	0.00672022775912549\\
76.06	0.0067210319561174\\
76.07	0.00672183663270363\\
76.08	0.00672264178928074\\
76.09	0.00672344742622095\\
76.1	0.00672425354387123\\
76.11	0.00672506014255217\\
76.12	0.006725867222557\\
76.13	0.0067266747841504\\
76.14	0.00672748282756742\\
76.15	0.0067282913530123\\
76.16	0.00672910036065719\\
76.17	0.00672990985064102\\
76.18	0.00673071982306811\\
76.19	0.00673153027800688\\
76.2	0.0067323412154885\\
76.21	0.00673315263550546\\
76.22	0.00673396453801012\\
76.23	0.0067347769229132\\
76.24	0.00673558979008228\\
76.25	0.00673640313934015\\
76.26	0.00673721697046323\\
76.27	0.00673803128317985\\
76.28	0.0067388460771685\\
76.29	0.00673966135205609\\
76.3	0.00674047710741604\\
76.31	0.00674129334276642\\
76.32	0.00674211005756799\\
76.33	0.00674292725122216\\
76.34	0.00674374492306895\\
76.35	0.00674456307238481\\
76.36	0.00674538169838044\\
76.37	0.00674620080019851\\
76.38	0.00674702037691134\\
76.39	0.00674784042751847\\
76.4	0.00674866095094421\\
76.41	0.00674948194603505\\
76.42	0.00675030341155707\\
76.43	0.00675112534619322\\
76.44	0.00675194774854052\\
76.45	0.0067527706171072\\
76.46	0.00675359395030979\\
76.47	0.00675441774647001\\
76.48	0.00675524200381168\\
76.49	0.00675606672045751\\
76.5	0.00675689189442574\\
76.51	0.00675771752362679\\
76.52	0.00675854360585968\\
76.53	0.00675937013880845\\
76.54	0.00676019712003846\\
76.55	0.00676102454699249\\
76.56	0.00676185241698686\\
76.57	0.00676268072720735\\
76.58	0.00676350947470502\\
76.59	0.00676433865639192\\
76.6	0.00676516826903663\\
76.61	0.00676599830925977\\
76.62	0.00676682877352927\\
76.63	0.00676765965815555\\
76.64	0.00676849095928658\\
76.65	0.00676932267290276\\
76.66	0.00677015479481166\\
76.67	0.00677098732064263\\
76.68	0.00677182024584122\\
76.69	0.00677265356566346\\
76.7	0.00677348727517\\
76.71	0.00677432136921998\\
76.72	0.00677515584246488\\
76.73	0.00677599068934203\\
76.74	0.00677682590406806\\
76.75	0.00677766148063208\\
76.76	0.0067784974127887\\
76.77	0.00677933369405084\\
76.78	0.00678017031768233\\
76.79	0.00678100727669031\\
76.8	0.0067818445638174\\
76.81	0.00678268217153361\\
76.82	0.00678352009202812\\
76.83	0.00678435831720067\\
76.84	0.0067851968386529\\
76.85	0.00678603564767919\\
76.86	0.00678687473525749\\
76.87	0.00678771409203973\\
76.88	0.00678855370834198\\
76.89	0.00678939357413437\\
76.9	0.00679023367903067\\
76.91	0.00679107401227761\\
76.92	0.0067919145627439\\
76.93	0.00679275531890886\\
76.94	0.00679359626885079\\
76.95	0.00679443740023505\\
76.96	0.00679527870030164\\
76.97	0.00679612015585264\\
76.98	0.00679696175323906\\
76.99	0.0067978034783475\\
77	0.00679864531658634\\
77.01	0.00679948725287152\\
77.02	0.00680032927161197\\
77.03	0.0068011713566946\\
77.04	0.0068020134914688\\
77.05	0.00680285565873059\\
77.06	0.00680369784070629\\
77.07	0.00680454001903564\\
77.08	0.00680538217475455\\
77.09	0.00680622428827731\\
77.1	0.00680706633937824\\
77.11	0.00680790830717292\\
77.12	0.00680875017009877\\
77.13	0.00680959190589515\\
77.14	0.00681043349158285\\
77.15	0.00681127490344304\\
77.16	0.00681211611699552\\
77.17	0.00681295710697646\\
77.18	0.00681379784731545\\
77.19	0.00681463831111186\\
77.2	0.00681547847061055\\
77.21	0.00681631829717693\\
77.22	0.00681715776127121\\
77.23	0.00681799683242199\\
77.24	0.00681883547919906\\
77.25	0.00681967366918541\\
77.26	0.00682051136894845\\
77.27	0.0068213485440104\\
77.28	0.00682218515881784\\
77.29	0.00682302117671037\\
77.3	0.00682385655988836\\
77.31	0.00682469126938396\\
77.32	0.00682552526503408\\
77.33	0.0068263585054461\\
77.34	0.0068271909479627\\
77.35	0.00682802254862569\\
77.36	0.00682885326213877\\
77.37	0.0068296830418293\\
77.38	0.00683051183960899\\
77.39	0.00683133960593341\\
77.4	0.00683216628976048\\
77.41	0.00683299183850771\\
77.42	0.00683381619800826\\
77.43	0.00683463931246574\\
77.44	0.00683546112440781\\
77.45	0.00683628157463836\\
77.46	0.00683710060218851\\
77.47	0.00683791814426601\\
77.48	0.00683873413620344\\
77.49	0.00683954851140479\\
77.5	0.00684036120129063\\
77.51	0.00684117213524169\\
77.52	0.00684198124054086\\
77.53	0.00684278844231355\\
77.54	0.00684359366346645\\
77.55	0.00684439682462438\\
77.56	0.00684519784406561\\
77.57	0.00684599663765511\\
77.58	0.00684679311877612\\
77.59	0.00684758719825966\\
77.6	0.00684837878431211\\
77.61	0.00684916778244075\\
77.62	0.00684995425835396\\
77.63	0.00685074098203392\\
77.64	0.00685152795052877\\
77.65	0.00685231516083168\\
77.66	0.00685310260988019\\
77.67	0.00685389029455566\\
77.68	0.00685467821168268\\
77.69	0.00685546635802844\\
77.7	0.00685625473030225\\
77.71	0.00685704332515491\\
77.72	0.00685783213917821\\
77.73	0.00685862116890439\\
77.74	0.00685941041080562\\
77.75	0.00686019986129356\\
77.76	0.0068609895167188\\
77.77	0.00686177937337048\\
77.78	0.00686256942747582\\
77.79	0.0068633596751997\\
77.8	0.00686415011264432\\
77.81	0.0068649407358488\\
77.82	0.00686573154078884\\
77.83	0.00686652252337647\\
77.84	0.00686731367945973\\
77.85	0.00686810500482247\\
77.86	0.00686889649518411\\
77.87	0.00686968814619953\\
77.88	0.00687047995345892\\
77.89	0.0068712719124877\\
77.9	0.0068720640187465\\
77.91	0.00687285626763118\\
77.92	0.00687364865447292\\
77.93	0.00687444117453827\\
77.94	0.00687523382302942\\
77.95	0.00687602659508438\\
77.96	0.00687681948577732\\
77.97	0.00687761249011893\\
77.98	0.00687840560305686\\
77.99	0.00687919881947624\\
78	0.00687999213420028\\
78.01	0.00688078554199095\\
78.02	0.00688157903754973\\
78.03	0.00688237261551851\\
78.04	0.00688316627048049\\
78.05	0.00688395999696126\\
78.06	0.00688475378942997\\
78.07	0.00688554764230058\\
78.08	0.00688634154993326\\
78.09	0.0068871355066359\\
78.1	0.00688792950666574\\
78.11	0.00688872354423115\\
78.12	0.00688951761349354\\
78.13	0.00689031170856942\\
78.14	0.0068911058235326\\
78.15	0.00689189995241658\\
78.16	0.00689269408921712\\
78.17	0.00689348822789488\\
78.18	0.00689428236237841\\
78.19	0.00689507648656722\\
78.2	0.00689587059433507\\
78.21	0.00689666467953345\\
78.22	0.00689745873599542\\
78.23	0.00689825275753944\\
78.24	0.00689904673797367\\
78.25	0.00689984067110037\\
78.26	0.00690063455072065\\
78.27	0.00690142837063944\\
78.28	0.0069022221246708\\
78.29	0.00690301580664343\\
78.3	0.00690380941040663\\
78.31	0.00690460292983646\\
78.32	0.00690539635884234\\
78.33	0.00690618969137389\\
78.34	0.00690698292142827\\
78.35	0.00690777604305778\\
78.36	0.00690856905037796\\
78.37	0.00690936193757604\\
78.38	0.00691015469891985\\
78.39	0.0069109473287672\\
78.4	0.0069117398215757\\
78.41	0.0069125321719131\\
78.42	0.00691332437446809\\
78.43	0.00691411642406174\\
78.44	0.00691490831565937\\
78.45	0.00691570004438308\\
78.46	0.00691649160552485\\
78.47	0.0069172829945603\\
78.48	0.00691807420716303\\
78.49	0.00691886523921976\\
78.5	0.00691965608684601\\
78.51	0.0069204467464027\\
78.52	0.00692123721451337\\
78.53	0.00692202748808226\\
78.54	0.00692281756431323\\
78.55	0.0069236074407295\\
78.56	0.00692439711519429\\
78.57	0.00692518658593245\\
78.58	0.00692597585155299\\
78.59	0.00692676491107264\\
78.6	0.00692755376394048\\
78.61	0.00692834241006362\\
78.62	0.00692913084983403\\
78.63	0.00692991908415658\\
78.64	0.00693070711447823\\
78.65	0.00693149494281856\\
78.66	0.0069322825718016\\
78.67	0.00693307000468898\\
78.68	0.00693385724541462\\
78.69	0.00693464429862078\\
78.7	0.00693543116969574\\
78.71	0.00693621786481302\\
78.72	0.0069370043909723\\
78.73	0.00693779075604207\\
78.74	0.00693857696880402\\
78.75	0.00693936303899938\\
78.76	0.00694014897737708\\
78.77	0.00694093479574406\\
78.78	0.00694172050701755\\
78.79	0.00694250612527954\\
78.8	0.00694329166583357\\
78.81	0.00694407714526378\\
78.82	0.00694486258149639\\
78.83	0.00694564799386379\\
78.84	0.00694643340317109\\
78.85	0.00694721883176553\\
78.86	0.00694800430360865\\
78.87	0.00694878984435137\\
78.88	0.00694957548141211\\
78.89	0.00695036124405811\\
78.9	0.00695114716348997\\
78.91	0.00695193327292961\\
78.92	0.00695271960771177\\
78.93	0.00695350620537909\\
78.94	0.00695429310578112\\
78.95	0.00695508035117716\\
78.96	0.0069558679863432\\
78.97	0.00695665605868319\\
78.98	0.0069574446183446\\
78.99	0.00695823371833861\\
79	0.00695902337359642\\
79.01	0.00695981358597765\\
79.02	0.00696060435736961\\
79.03	0.00696139568968789\\
79.04	0.00696218758487702\\
79.05	0.00696298004491101\\
79.06	0.00696377307179408\\
79.07	0.0069645666675612\\
79.08	0.00696536083427885\\
79.09	0.00696615557404564\\
79.1	0.00696695088899304\\
79.11	0.00696774678128608\\
79.12	0.00696854325312407\\
79.13	0.00696934030674136\\
79.14	0.00697013794440811\\
79.15	0.00697093616843106\\
79.16	0.00697173498115435\\
79.17	0.0069725343849603\\
79.18	0.00697333438227031\\
79.19	0.00697413497554568\\
79.2	0.00697493616728847\\
79.21	0.00697573796004249\\
79.22	0.00697654035639412\\
79.23	0.00697734335897332\\
79.24	0.00697814697045459\\
79.25	0.00697895119355794\\
79.26	0.00697975603104992\\
79.27	0.00698056148574468\\
79.28	0.006981367560505\\
79.29	0.00698217425824341\\
79.3	0.00698298158192329\\
79.31	0.00698378953456001\\
79.32	0.00698459811922216\\
79.33	0.00698540733903267\\
79.34	0.00698621719717007\\
79.35	0.00698702769686978\\
79.36	0.00698783884142539\\
79.37	0.00698865063418995\\
79.38	0.00698946307857737\\
79.39	0.00699027617806375\\
79.4	0.00699108993618889\\
79.41	0.00699190435655767\\
79.42	0.00699271944284157\\
79.43	0.0069935351987802\\
79.44	0.0069943516281829\\
79.45	0.00699516873493026\\
79.46	0.00699598652297587\\
79.47	0.00699680499634792\\
79.48	0.006997624159151\\
79.49	0.00699844401556779\\
79.5	0.00699926456986097\\
79.51	0.00700008582637499\\
79.52	0.00700090778953803\\
79.53	0.00700173046386394\\
79.54	0.00700255385395423\\
79.55	0.00700337796450014\\
79.56	0.00700420280028472\\
79.57	0.007005028366185\\
79.58	0.0070058546671742\\
79.59	0.00700668170832398\\
79.6	0.00700750949480677\\
79.61	0.00700833803189816\\
79.62	0.00700916732497928\\
79.63	0.00700999737953939\\
79.64	0.00701082820117837\\
79.65	0.00701165979560935\\
79.66	0.00701249216866144\\
79.67	0.00701332532628247\\
79.68	0.0070141592745418\\
79.69	0.00701499401963324\\
79.7	0.00701582956787802\\
79.71	0.00701666592572783\\
79.72	0.00701750309976796\\
79.73	0.00701834109672047\\
79.74	0.0070191799234475\\
79.75	0.00702001958695467\\
79.76	0.00702086009439445\\
79.77	0.00702170145306977\\
79.78	0.00702254367043763\\
79.79	0.00702338675411281\\
79.8	0.0070242307118717\\
79.81	0.00702507555165621\\
79.82	0.00702592128157779\\
79.83	0.00702676790992153\\
79.84	0.00702761544515039\\
79.85	0.00702846389590955\\
79.86	0.00702931327103079\\
79.87	0.00703016357953713\\
79.88	0.0070310148306474\\
79.89	0.0070318670337811\\
79.9	0.00703272019856331\\
79.91	0.00703357433482966\\
79.92	0.00703442945263158\\
79.93	0.00703528556224154\\
79.94	0.0070361426741585\\
79.95	0.0070370007991135\\
79.96	0.00703785994807537\\
79.97	0.00703872013225658\\
79.98	0.00703958136311927\\
79.99	0.00704044365238141\\
80	0.00704130701202317\\
80.01	0.00704217145429333\\
};
\addplot [color=blue,solid]
  table[row sep=crcr]{%
80.01	0.00704217145429333\\
80.02	0.00704303699171604\\
80.03	0.00704390363709758\\
80.04	0.0070447714035334\\
80.05	0.00704564030441531\\
80.06	0.00704651035343886\\
80.07	0.0070473815646109\\
80.08	0.00704825395225734\\
80.09	0.00704912753103115\\
80.1	0.00705000231592049\\
80.11	0.00705087832225712\\
80.12	0.007051755565725\\
80.13	0.00705263406236912\\
80.14	0.00705351382860457\\
80.15	0.00705439488122577\\
80.16	0.00705527723741613\\
80.17	0.00705616091475767\\
80.18	0.00705704593124124\\
80.19	0.00705793230527665\\
80.2	0.00705882005570338\\
80.21	0.00705970920180134\\
80.22	0.00706059976330202\\
80.23	0.00706149176039996\\
80.24	0.0070623852137644\\
80.25	0.00706328014455135\\
80.26	0.00706417657441594\\
80.27	0.00706507452552506\\
80.28	0.00706597402057037\\
80.29	0.00706687508278165\\
80.3	0.00706777773594051\\
80.31	0.00706868200439438\\
80.32	0.00706958791307102\\
80.33	0.00707049548749322\\
80.34	0.00707140475379408\\
80.35	0.00707231573873257\\
80.36	0.00707322846970951\\
80.37	0.00707414297478401\\
80.38	0.00707505928269034\\
80.39	0.00707597742285523\\
80.4	0.00707689742541562\\
80.41	0.00707781932123689\\
80.42	0.00707874314193157\\
80.43	0.00707966891987857\\
80.44	0.00708059668824285\\
80.45	0.0070815264809957\\
80.46	0.0070824583329355\\
80.47	0.00708339227970904\\
80.48	0.00708432835783341\\
80.49	0.00708526660471847\\
80.5	0.00708620705868992\\
80.51	0.00708714975901298\\
80.52	0.00708809474591669\\
80.53	0.00708904206061889\\
80.54	0.00708999174535183\\
80.55	0.00709094384338842\\
80.56	0.00709189839906933\\
80.57	0.00709285545783062\\
80.58	0.00709381506623226\\
80.59	0.00709477727198731\\
80.6	0.00709574212399195\\
80.61	0.00709670967235624\\
80.62	0.00709767996843578\\
80.63	0.00709865306486414\\
80.64	0.00709962901558624\\
80.65	0.00710060787589251\\
80.66	0.0071015897024541\\
80.67	0.0071025745533589\\
80.68	0.00710356248814862\\
80.69	0.00710455356785681\\
80.7	0.00710554785504794\\
80.71	0.0071065454138575\\
80.72	0.00710754631003315\\
80.73	0.00710855061097704\\
80.74	0.00710955838578922\\
80.75	0.00711056970531221\\
80.76	0.00711158464217681\\
80.77	0.00711260327084909\\
80.78	0.00711362566767872\\
80.79	0.00711465191094846\\
80.8	0.00711568208092515\\
80.81	0.00711671625991197\\
80.82	0.0071177545323021\\
80.83	0.0071187969846339\\
80.84	0.00711984370564754\\
80.85	0.00712089478634309\\
80.86	0.00712195032004031\\
80.87	0.00712301040243994\\
80.88	0.00712407513168671\\
80.89	0.00712514460843402\\
80.9	0.0071262189359104\\
80.91	0.0071272982199877\\
80.92	0.00712838256925123\\
80.93	0.00712947209507168\\
80.94	0.00713056691167911\\
80.95	0.00713166713623883\\
80.96	0.00713277288892939\\
80.97	0.00713388429302271\\
80.98	0.00713500147496633\\
80.99	0.00713612456446794\\
81	0.00713725369458212\\
81.01	0.00713838900179958\\
81.02	0.00713953062613868\\
81.03	0.00714067871123951\\
81.04	0.00714183340446053\\
81.05	0.00714299485697778\\
81.06	0.0071441632238869\\
81.07	0.00714533866430783\\
81.08	0.00714652134149236\\
81.09	0.00714771142293468\\
81.1	0.00714890908048492\\
81.11	0.00715011449046574\\
81.12	0.00715132783379214\\
81.13	0.00715254929609456\\
81.14	0.00715377906784531\\
81.15	0.00715501734448843\\
81.16	0.00715626432657318\\
81.17	0.00715752021989111\\
81.18	0.00715878523561688\\
81.19	0.007160059590453\\
81.2	0.00716134350677849\\
81.21	0.00716263721280156\\
81.22	0.0071639409427166\\
81.23	0.00716525493686536\\
81.24	0.00716657944190262\\
81.25	0.00716791471096638\\
81.26	0.00716926100385273\\
81.27	0.00717061858719553\\
81.28	0.00717198773465103\\
81.29	0.00717336872708754\\
81.3	0.00717476185278041\\
81.31	0.00717616740761223\\
81.32	0.00717758569527868\\
81.33	0.00717901702750002\\
81.34	0.00718046172423833\\
81.35	0.00718192011392091\\
81.36	0.00718339253366964\\
81.37	0.0071848793295369\\
81.38	0.00718638085674779\\
81.39	0.00718789747994924\\
81.4	0.0071894295734659\\
81.41	0.00719097752156313\\
81.42	0.00719254171871732\\
81.43	0.00719412256989366\\
81.44	0.00719572049083166\\
81.45	0.0071973359083385\\
81.46	0.0071989692605907\\
81.47	0.00720062099744396\\
81.48	0.0072022915807518\\
81.49	0.00720398148469297\\
81.5	0.00720569119610799\\
81.51	0.00720742121484509\\
81.52	0.00720915720360016\\
81.53	0.00721089405183198\\
81.54	0.00721263176180608\\
81.55	0.00721437033579901\\
81.56	0.00721610977609802\\
81.57	0.00721785008500078\\
81.58	0.00721959126481496\\
81.59	0.00722133331785796\\
81.6	0.00722307624645643\\
81.61	0.00722482005294588\\
81.62	0.0072265647396703\\
81.63	0.00722831030898166\\
81.64	0.00723005676323942\\
81.65	0.00723180410481009\\
81.66	0.00723355233606664\\
81.67	0.00723530145938797\\
81.68	0.00723705147715835\\
81.69	0.00723880239176677\\
81.7	0.00724055420560636\\
81.71	0.00724230692107368\\
81.72	0.00724406054056803\\
81.73	0.00724581506649075\\
81.74	0.0072475705012444\\
81.75	0.00724932684723206\\
81.76	0.00725108410685639\\
81.77	0.00725284228251884\\
81.78	0.00725460137661873\\
81.79	0.00725636139155227\\
81.8	0.00725812232971163\\
81.81	0.00725988419348388\\
81.82	0.00726164698524992\\
81.83	0.00726341070738342\\
81.84	0.00726517536224956\\
81.85	0.00726694095220394\\
81.86	0.00726870747959124\\
81.87	0.00727047494674396\\
81.88	0.00727224335598106\\
81.89	0.00727401270960651\\
81.9	0.00727578300990787\\
81.91	0.00727755425915476\\
81.92	0.00727932645959724\\
81.93	0.00728109961346422\\
81.94	0.00728287372296169\\
81.95	0.007284648790271\\
81.96	0.00728642481754701\\
81.97	0.00728820180691615\\
81.98	0.00728997976047444\\
81.99	0.00729175868028551\\
82	0.00729353856837834\\
82.01	0.00729531942674517\\
82.02	0.00729710125733912\\
82.03	0.00729888406207187\\
82.04	0.00730066784281115\\
82.05	0.00730245260137824\\
82.06	0.00730423833954526\\
82.07	0.00730602505903248\\
82.08	0.00730781276150547\\
82.09	0.00730960144857214\\
82.1	0.00731139112177971\\
82.11	0.00731318178261156\\
82.12	0.00731497343248396\\
82.13	0.00731676607274268\\
82.14	0.0073185597046595\\
82.15	0.00732035432942863\\
82.16	0.00732214994816288\\
82.17	0.00732394656188987\\
82.18	0.00732574417154796\\
82.19	0.00732754277798214\\
82.2	0.00732934238193973\\
82.21	0.00733114298406592\\
82.22	0.00733294458489924\\
82.23	0.00733474718486674\\
82.24	0.00733655078427918\\
82.25	0.00733835538332585\\
82.26	0.00734016098206944\\
82.27	0.00734196758044053\\
82.28	0.00734377517823207\\
82.29	0.00734558377509353\\
82.3	0.00734739337052498\\
82.31	0.00734920396387085\\
82.32	0.00735101555431364\\
82.33	0.0073528281408672\\
82.34	0.00735464172237006\\
82.35	0.00735645629747829\\
82.36	0.0073582718646583\\
82.37	0.0073600884221793\\
82.38	0.00736190596810558\\
82.39	0.00736372450028848\\
82.4	0.00736554401635815\\
82.41	0.00736736451371503\\
82.42	0.00736918598952103\\
82.43	0.00737100844069042\\
82.44	0.00737283186388051\\
82.45	0.00737465625548189\\
82.46	0.00737648161160849\\
82.47	0.00737830792808725\\
82.48	0.00738013520044744\\
82.49	0.00738196342390972\\
82.5	0.00738379259337478\\
82.51	0.00738562270341162\\
82.52	0.00738745374824553\\
82.53	0.0073892857217456\\
82.54	0.00739111861741185\\
82.55	0.00739295242836205\\
82.56	0.00739478714731797\\
82.57	0.00739662276659132\\
82.58	0.00739845927806917\\
82.59	0.007400296673199\\
82.6	0.00740213494297315\\
82.61	0.00740397407791291\\
82.62	0.00740581406805205\\
82.63	0.00740765490291983\\
82.64	0.00740949657152353\\
82.65	0.00741133906233034\\
82.66	0.00741318236324881\\
82.67	0.00741502646160964\\
82.68	0.00741687134414588\\
82.69	0.00741871699697254\\
82.7	0.00742056340556558\\
82.71	0.00742241055474022\\
82.72	0.00742425842862855\\
82.73	0.00742610701065658\\
82.74	0.00742795628352041\\
82.75	0.00742980622916179\\
82.76	0.00743165682874292\\
82.77	0.00743350806262036\\
82.78	0.00743535991031833\\
82.79	0.00743721235050102\\
82.8	0.00743906536094416\\
82.81	0.00744091891850566\\
82.82	0.0074427729990954\\
82.83	0.00744462757764405\\
82.84	0.00744648262807101\\
82.85	0.00744833812325129\\
82.86	0.00745019403498147\\
82.87	0.00745205033394451\\
82.88	0.00745390698967366\\
82.89	0.00745576397051509\\
82.9	0.00745762124358955\\
82.91	0.00745947877475275\\
82.92	0.00746133652855462\\
82.93	0.00746319446819729\\
82.94	0.00746505255549183\\
82.95	0.0074669107508137\\
82.96	0.0074687690130568\\
82.97	0.0074706272995862\\
82.98	0.00747248556618942\\
82.99	0.00747434376702624\\
83	0.007476201854577\\
83.01	0.0074780597795894\\
83.02	0.00747991749102361\\
83.03	0.00748177493599586\\
83.04	0.00748363205972021\\
83.05	0.00748548880544866\\
83.06	0.00748734511440949\\
83.07	0.00748920092574367\\
83.08	0.00749105617643943\\
83.09	0.00749291080126491\\
83.1	0.00749476473269872\\
83.11	0.0074966179008585\\
83.12	0.00749847023342733\\
83.13	0.00750032165557793\\
83.14	0.00750217208989465\\
83.15	0.00750402145629312\\
83.16	0.00750586967193752\\
83.17	0.00750771665115538\\
83.18	0.00750956230534986\\
83.19	0.00751140654290946\\
83.2	0.00751324926911505\\
83.21	0.00751509038604407\\
83.22	0.00751692979247202\\
83.23	0.00751876738377097\\
83.24	0.00752060305180508\\
83.25	0.007522436684823\\
83.26	0.0075242681673472\\
83.27	0.00752609738005992\\
83.28	0.00752792419968583\\
83.29	0.00752974849887124\\
83.3	0.00753157014605965\\
83.31	0.00753338900536381\\
83.32	0.00753520493643386\\
83.33	0.00753701779432172\\
83.34	0.00753882742934136\\
83.35	0.00754063368692514\\
83.36	0.00754243640747577\\
83.37	0.00754423542621401\\
83.38	0.00754603057302196\\
83.39	0.00754782167228161\\
83.4	0.00754960854270881\\
83.41	0.00755139099718232\\
83.42	0.00755316884256784\\
83.43	0.00755494187953691\\
83.44	0.00755670990238059\\
83.45	0.00755847269881756\\
83.46	0.00756023004979678\\
83.47	0.00756198172929422\\
83.48	0.00756372750410383\\
83.49	0.00756546713362233\\
83.5	0.00756720036962772\\
83.51	0.00756892695605137\\
83.52	0.00757064662874353\\
83.53	0.0075723591152319\\
83.54	0.00757406413447329\\
83.55	0.00757576139659803\\
83.56	0.00757745060264696\\
83.57	0.00757913432378243\\
83.58	0.00758081871024833\\
83.59	0.00758250376257283\\
83.6	0.00758418948128547\\
83.61	0.00758587586691725\\
83.62	0.0075875629200006\\
83.63	0.00758925064106941\\
83.64	0.00759093903065902\\
83.65	0.00759262808930626\\
83.66	0.00759431781754948\\
83.67	0.00759600821592851\\
83.68	0.00759769928498473\\
83.69	0.00759939102526104\\
83.7	0.00760108343730193\\
83.71	0.00760277652165343\\
83.72	0.00760447027886319\\
83.73	0.00760616470948044\\
83.74	0.00760785981405605\\
83.75	0.00760955559314252\\
83.76	0.00761125204729401\\
83.77	0.00761294917706637\\
83.78	0.00761464698301711\\
83.79	0.00761634546570547\\
83.8	0.0076180446256924\\
83.81	0.00761974446354063\\
83.82	0.00762144497981461\\
83.83	0.00762314617508061\\
83.84	0.00762484804990668\\
83.85	0.00762655060486269\\
83.86	0.00762825384052036\\
83.87	0.00762995775745328\\
83.88	0.0076316623562369\\
83.89	0.00763336763744859\\
83.9	0.00763507360166763\\
83.91	0.00763678024947525\\
83.92	0.00763848758145466\\
83.93	0.00764019559819102\\
83.94	0.00764190430027155\\
83.95	0.00764361368828546\\
83.96	0.00764532376282403\\
83.97	0.00764703452448063\\
83.98	0.00764874597385073\\
83.99	0.0076504581115319\\
84	0.00765217093812389\\
84.01	0.00765388445422861\\
84.02	0.00765559866045019\\
84.03	0.00765731355739496\\
84.04	0.00765902914567152\\
84.05	0.00766074542589073\\
84.06	0.00766246239866578\\
84.07	0.00766418006461216\\
84.08	0.00766589842434776\\
84.09	0.00766761747849281\\
84.1	0.00766933722766999\\
84.11	0.00767105767250442\\
84.12	0.00767277881362366\\
84.13	0.00767450065165782\\
84.14	0.00767622318723951\\
84.15	0.00767794642100392\\
84.16	0.00767967035358882\\
84.17	0.00768139498563462\\
84.18	0.00768312031778437\\
84.19	0.00768484635068382\\
84.2	0.00768657308498144\\
84.21	0.00768830052132847\\
84.22	0.00769002866037891\\
84.23	0.00769175750278961\\
84.24	0.00769348704922027\\
84.25	0.00769521730033348\\
84.26	0.00769694825679477\\
84.27	0.00769867991927262\\
84.28	0.00770041228843853\\
84.29	0.00770214536496703\\
84.3	0.00770387914953575\\
84.31	0.00770561364282542\\
84.32	0.00770734884551991\\
84.33	0.00770908475830633\\
84.34	0.007710821381875\\
84.35	0.00771255871691953\\
84.36	0.00771429676413683\\
84.37	0.00771603552422722\\
84.38	0.00771777499789436\\
84.39	0.00771951518584543\\
84.4	0.00772125608879105\\
84.41	0.0077229977074454\\
84.42	0.00772474004252626\\
84.43	0.00772648309475501\\
84.44	0.00772822686485674\\
84.45	0.00772997135356025\\
84.46	0.0077317165615981\\
84.47	0.0077334624897067\\
84.48	0.00773520913862633\\
84.49	0.00773695650910117\\
84.5	0.00773870460187942\\
84.51	0.00774045341771326\\
84.52	0.00774220295735899\\
84.53	0.00774395322157703\\
84.54	0.00774570421113197\\
84.55	0.00774745592679268\\
84.56	0.00774920836933231\\
84.57	0.00775096153952838\\
84.58	0.00775271543816282\\
84.59	0.00775447006602202\\
84.6	0.00775622542389694\\
84.61	0.0077579815125831\\
84.62	0.0077597383328807\\
84.63	0.00776149588559466\\
84.64	0.00776325417153466\\
84.65	0.00776501319151524\\
84.66	0.00776677294635588\\
84.67	0.00776853343688097\\
84.68	0.00777029466392002\\
84.69	0.00777205662830761\\
84.7	0.0077738193308835\\
84.71	0.00777558277249273\\
84.72	0.00777734695398564\\
84.73	0.00777911187621798\\
84.74	0.00778087754005096\\
84.75	0.00778264394635133\\
84.76	0.00778441109599148\\
84.77	0.00778617898984946\\
84.78	0.00778794762880913\\
84.79	0.00778971701376019\\
84.8	0.00779148714559824\\
84.81	0.00779325802522495\\
84.82	0.00779502965354804\\
84.83	0.00779680203148142\\
84.84	0.00779857515994527\\
84.85	0.00780034903986612\\
84.86	0.00780212367217691\\
84.87	0.00780389905781714\\
84.88	0.00780567519773291\\
84.89	0.00780745209287702\\
84.9	0.00780922974420906\\
84.91	0.00781100815269552\\
84.92	0.00781278731930988\\
84.93	0.0078145672450327\\
84.94	0.00781634793085171\\
84.95	0.00781812937776191\\
84.96	0.0078199115867657\\
84.97	0.00782169455887296\\
84.98	0.00782347829510112\\
84.99	0.00782526279647533\\
85	0.00782704806402852\\
85.01	0.00782883409880153\\
85.02	0.00783062090184319\\
85.03	0.00783240847421048\\
85.04	0.00783419681696859\\
85.05	0.00783598593119105\\
85.06	0.00783777581795989\\
85.07	0.00783956647836567\\
85.08	0.00784135791350768\\
85.09	0.00784315012449401\\
85.1	0.00784494311244169\\
85.11	0.00784673687847683\\
85.12	0.0078485314237347\\
85.13	0.0078503267493599\\
85.14	0.00785212285650646\\
85.15	0.007853919746338\\
85.16	0.00785571742002782\\
85.17	0.00785751587875907\\
85.18	0.00785931512372487\\
85.19	0.00786111515612845\\
85.2	0.0078629159771833\\
85.21	0.00786471758811327\\
85.22	0.00786651999015277\\
85.23	0.00786832318454687\\
85.24	0.00787012717255148\\
85.25	0.00787193195543346\\
85.26	0.0078737375344708\\
85.27	0.00787554391095278\\
85.28	0.00787735108618007\\
85.29	0.00787915906146496\\
85.3	0.00788096783813146\\
85.31	0.00788277741751549\\
85.32	0.00788458780096502\\
85.33	0.00788639898984028\\
85.34	0.00788821098551386\\
85.35	0.00789002378937092\\
85.36	0.00789183740280937\\
85.37	0.00789365182724\\
85.38	0.00789546706408668\\
85.39	0.00789728311478654\\
85.4	0.00789909998079017\\
85.41	0.00790091766356173\\
85.42	0.00790273616457919\\
85.43	0.00790455548533451\\
85.44	0.00790637562733383\\
85.45	0.00790819659209763\\
85.46	0.00791001838116093\\
85.47	0.00791184099607352\\
85.48	0.0079136644384001\\
85.49	0.00791548870972055\\
85.5	0.00791731381163003\\
85.51	0.0079191397457393\\
85.52	0.00792096651367481\\
85.53	0.00792279411707899\\
85.54	0.00792462255761044\\
85.55	0.0079264518369441\\
85.56	0.00792828195677154\\
85.57	0.00793011291880108\\
85.58	0.0079319447247581\\
85.59	0.00793377737638521\\
85.6	0.00793561087544246\\
85.61	0.00793744522370761\\
85.62	0.00793928042297635\\
85.63	0.00794111647506247\\
85.64	0.00794295338179816\\
85.65	0.00794479114503423\\
85.66	0.0079466297666403\\
85.67	0.00794846924850511\\
85.68	0.00795030959253669\\
85.69	0.00795215080066265\\
85.7	0.00795399287483041\\
85.71	0.00795583581700744\\
85.72	0.00795767962918151\\
85.73	0.00795952431336096\\
85.74	0.0079613698715749\\
85.75	0.00796321630587354\\
85.76	0.0079650636183284\\
85.77	0.00796691181103255\\
85.78	0.00796876088610091\\
85.79	0.00797061084567052\\
85.8	0.00797246169190075\\
85.81	0.00797431342697362\\
85.82	0.00797616605309402\\
85.83	0.00797801957249003\\
85.84	0.00797987398741315\\
85.85	0.0079817293001386\\
85.86	0.00798358551296556\\
85.87	0.00798544262821748\\
85.88	0.00798730064824233\\
85.89	0.00798915957541292\\
85.9	0.00799101941212712\\
85.91	0.00799288016080817\\
85.92	0.00799474182390499\\
85.93	0.00799660440389238\\
85.94	0.00799846790327141\\
85.95	0.00800033232456962\\
85.96	0.00800219767034133\\
85.97	0.00800406394316796\\
85.98	0.00800593114565825\\
85.99	0.00800779928044861\\
86	0.00800966835020334\\
86.01	0.00801153835761498\\
86.02	0.00801340930540457\\
86.03	0.00801528119632193\\
86.04	0.00801715403314593\\
86.05	0.00801902781868483\\
86.06	0.0080209025557765\\
86.07	0.00802277824728876\\
86.08	0.0080246548961196\\
86.09	0.00802653250519754\\
86.1	0.00802841107748182\\
86.11	0.00803029061596276\\
86.12	0.008032171123662\\
86.13	0.00803405260363275\\
86.14	0.00803593505896011\\
86.15	0.00803781849276131\\
86.16	0.00803970290818599\\
86.17	0.00804158830841646\\
86.18	0.00804347469666795\\
86.19	0.00804536207618887\\
86.2	0.0080472504502611\\
86.21	0.00804913982220018\\
86.22	0.0080510301953556\\
86.23	0.00805292157311101\\
86.24	0.00805481395888447\\
86.25	0.00805670735612871\\
86.26	0.00805860176833127\\
86.27	0.00806049719901481\\
86.28	0.00806239365173724\\
86.29	0.008064291130092\\
86.3	0.00806618963770821\\
86.31	0.00806808917825084\\
86.32	0.00806998975542094\\
86.33	0.00807189137295579\\
86.34	0.00807379403462904\\
86.35	0.00807569774425089\\
86.36	0.00807760250566823\\
86.37	0.00807950832276475\\
86.38	0.00808141519946107\\
86.39	0.00808332313971485\\
86.4	0.00808523214752089\\
86.41	0.00808714222691117\\
86.42	0.00808905338195498\\
86.43	0.00809096561675893\\
86.44	0.00809287893546698\\
86.45	0.00809479334226049\\
86.46	0.0080967088413582\\
86.47	0.00809862543701623\\
86.48	0.00810054313352803\\
86.49	0.00810246193522434\\
86.5	0.0081043818464731\\
86.51	0.00810630287167939\\
86.52	0.00810822501528528\\
86.53	0.0081101482817697\\
86.54	0.00811207267564827\\
86.55	0.00811399820147312\\
86.56	0.00811592486383269\\
86.57	0.00811785266735144\\
86.58	0.00811978161668963\\
86.59	0.00812171171654297\\
86.6	0.00812364297164233\\
86.61	0.00812557538675337\\
86.62	0.00812750896667608\\
86.63	0.00812944371624443\\
86.64	0.00813137964032587\\
86.65	0.00813331674382079\\
86.66	0.00813525503166201\\
86.67	0.00813719450881416\\
86.68	0.00813913518027309\\
86.69	0.00814107705106516\\
86.7	0.00814302012624652\\
86.71	0.00814496441090235\\
86.72	0.00814690991014605\\
86.73	0.00814885662911832\\
86.74	0.00815080457298631\\
86.75	0.00815275374694257\\
86.76	0.00815470415620404\\
86.77	0.00815665580601095\\
86.78	0.00815860870162568\\
86.79	0.00816056284833145\\
86.8	0.00816251825143116\\
86.81	0.00816447491624588\\
86.82	0.00816643284811353\\
86.83	0.00816839205238731\\
86.84	0.00817035253443414\\
86.85	0.00817231429963293\\
86.86	0.00817427735337293\\
86.87	0.0081762417010518\\
86.88	0.00817820734807372\\
86.89	0.00818017429984736\\
86.9	0.00818214256178377\\
86.91	0.00818411213929416\\
86.92	0.00818608303778755\\
86.93	0.00818805526266838\\
86.94	0.00819002881933396\\
86.95	0.00819200371317178\\
86.96	0.00819397994955681\\
86.97	0.00819595753384852\\
86.98	0.00819793647138791\\
86.99	0.00819991676749432\\
87	0.00820189842746219\\
87.01	0.00820388145655756\\
87.02	0.00820586586001453\\
87.03	0.00820785164303152\\
87.04	0.00820983881076735\\
87.05	0.00821182736833727\\
87.06	0.00821381732080863\\
87.07	0.00821580867319661\\
87.08	0.00821780143045951\\
87.09	0.00821979559749414\\
87.1	0.00822179117913075\\
87.11	0.00822378818012796\\
87.12	0.00822578660516735\\
87.13	0.00822778645884796\\
87.14	0.00822978774568044\\
87.15	0.00823179047008106\\
87.16	0.0082337946363655\\
87.17	0.00823580024874232\\
87.18	0.00823780731130622\\
87.19	0.00823981582803108\\
87.2	0.00824182580276263\\
87.21	0.008243837239211\\
87.22	0.00824585014094279\\
87.23	0.00824786451137301\\
87.24	0.00824988035375663\\
87.25	0.00825189767117984\\
87.26	0.00825391646655098\\
87.27	0.00825593674259111\\
87.28	0.00825795850182427\\
87.29	0.00825998174656739\\
87.3	0.00826200647891974\\
87.31	0.00826403270075213\\
87.32	0.00826606041369557\\
87.33	0.00826808961912967\\
87.34	0.00827012031817045\\
87.35	0.0082721525116579\\
87.36	0.00827418620014292\\
87.37	0.00827622138387389\\
87.38	0.00827825806278274\\
87.39	0.00828029623647053\\
87.4	0.00828233590419246\\
87.41	0.00828437706484246\\
87.42	0.00828641971693711\\
87.43	0.00828846385859911\\
87.44	0.00829050948754006\\
87.45	0.00829255660104272\\
87.46	0.00829460519594262\\
87.47	0.00829665526860901\\
87.48	0.00829870681492519\\
87.49	0.00830075983026812\\
87.5	0.00830281430948732\\
87.51	0.00830487024688309\\
87.52	0.00830692763618393\\
87.53	0.00830898647052319\\
87.54	0.00831104674241496\\
87.55	0.0083131084437291\\
87.56	0.00831517156566538\\
87.57	0.00831723609872684\\
87.58	0.00831930203269214\\
87.59	0.00832136935658705\\
87.6	0.00832343805865492\\
87.61	0.00832550812632616\\
87.62	0.00832757954618674\\
87.63	0.00832965230394555\\
87.64	0.00833172638440071\\
87.65	0.00833380177140476\\
87.66	0.00833587844782867\\
87.67	0.0083379563955246\\
87.68	0.00834003559528748\\
87.69	0.0083421160268153\\
87.7	0.00834419766866804\\
87.71	0.00834628049822527\\
87.72	0.00834836449164234\\
87.73	0.00835044962380508\\
87.74	0.00835253586828306\\
87.75	0.00835462319728126\\
87.76	0.00835671158159017\\
87.77	0.00835880099053426\\
87.78	0.00836089139191867\\
87.79	0.00836298275197428\\
87.8	0.00836507503530088\\
87.81	0.0083671682048085\\
87.82	0.00836926222165684\\
87.83	0.00837135704519271\\
87.84	0.00837345263288542\\
87.85	0.00837554894026008\\
87.86	0.00837764592082871\\
87.87	0.00837974352601913\\
87.88	0.00838184170510149\\
87.89	0.00838394040511253\\
87.9	0.00838603957077724\\
87.91	0.00838813914442807\\
87.92	0.00839023906592154\\
87.93	0.00839233927255212\\
87.94	0.0083944396989633\\
87.95	0.00839654027705588\\
87.96	0.00839864093589326\\
87.97	0.00840074160160363\\
87.98	0.00840284219727913\\
87.99	0.00840494264287169\\
88	0.00840704285508552\\
88.01	0.00840914274726625\\
88.02	0.0084112422292864\\
88.03	0.00841334120742728\\
88.04	0.00841543958425706\\
88.05	0.00841753725850499\\
88.06	0.00841963412493152\\
88.07	0.00842173007419447\\
88.08	0.00842382499271069\\
88.09	0.00842591876251349\\
88.1	0.00842801126110551\\
88.11	0.00843010236130688\\
88.12	0.00843219193109856\\
88.13	0.00843427983346076\\
88.14	0.00843636592620617\\
88.15	0.00843845006180802\\
88.16	0.00844053208722252\\
88.17	0.0084426118437059\\
88.18	0.0084446891666255\\
88.19	0.008446763885265\\
88.2	0.00844883582262343\\
88.21	0.00845090479520792\\
88.22	0.00845297061281986\\
88.23	0.00845503307833434\\
88.24	0.00845709198747273\\
88.25	0.00845914712856798\\
88.26	0.00846119828232264\\
88.27	0.00846324522155927\\
88.28	0.00846528771096301\\
88.29	0.00846732550681607\\
88.3	0.00846935835672394\\
88.31	0.00847138599933299\\
88.32	0.00847340816403925\\
88.33	0.00847542457068813\\
88.34	0.00847743492926466\\
88.35	0.00847943893957416\\
88.36	0.0084814362909129\\
88.37	0.0084834266617285\\
88.38	0.00848540971926975\\
88.39	0.00848738511922551\\
88.4	0.00848935250535242\\
88.41	0.00849131150909099\\
88.42	0.00849326174916977\\
88.43	0.00849520283119726\\
88.44	0.00849713434724107\\
88.45	0.00849905587539418\\
88.46	0.00850096697932757\\
88.47	0.00850286720782918\\
88.48	0.00850475609432845\\
88.49	0.00850663315640625\\
88.5	0.0085084978952896\\
88.51	0.00851034979533088\\
88.52	0.00851218832347083\\
88.53	0.00851401292868509\\
88.54	0.00851582304141369\\
88.55	0.00851761807297285\\
88.56	0.00851939741494884\\
88.57	0.00852116043857307\\
88.58	0.00852290649407804\\
88.59	0.0085246349100335\\
88.6	0.00852634499266224\\
88.61	0.00852804741214841\\
88.62	0.00852975056275011\\
88.63	0.00853145444487682\\
88.64	0.0085331590589383\\
88.65	0.00853486440534453\\
88.66	0.00853657048450578\\
88.67	0.00853827729683258\\
88.68	0.00853998484273569\\
88.69	0.00854169312262613\\
88.7	0.0085434021369152\\
88.71	0.00854511188601443\\
88.72	0.00854682237033563\\
88.73	0.00854853359029085\\
88.74	0.0085502455462924\\
88.75	0.00855195823875285\\
88.76	0.00855367166808501\\
88.77	0.00855538583470198\\
88.78	0.00855710073901708\\
88.79	0.00855881638144391\\
88.8	0.00856053276239632\\
88.81	0.00856224988228842\\
88.82	0.00856396774153456\\
88.83	0.00856568634054937\\
88.84	0.00856740567974772\\
88.85	0.00856912575954473\\
88.86	0.00857084658035581\\
88.87	0.00857256814259658\\
88.88	0.00857429044668295\\
88.89	0.00857601349303108\\
88.9	0.00857773728205736\\
88.91	0.00857946181417848\\
88.92	0.00858118708981135\\
88.93	0.00858291310937315\\
88.94	0.00858463987328132\\
88.95	0.00858636738195355\\
88.96	0.00858809563580778\\
88.97	0.00858982463526222\\
88.98	0.00859155438073532\\
88.99	0.0085932848726458\\
89	0.00859501611141263\\
89.01	0.00859674809745502\\
89.02	0.00859848083119247\\
89.03	0.0086002143130447\\
89.04	0.00860194854343171\\
89.05	0.00860368352277375\\
89.06	0.00860541925149132\\
89.07	0.00860715573000516\\
89.08	0.00860889295873629\\
89.09	0.00861063093810598\\
89.1	0.00861236966853576\\
89.11	0.0086141091504474\\
89.12	0.00861584938426292\\
89.13	0.00861759037040461\\
89.14	0.00861933210929502\\
89.15	0.00862107460135695\\
89.16	0.00862281784701343\\
89.17	0.00862456184668779\\
89.18	0.00862630660080356\\
89.19	0.00862805210978457\\
89.2	0.00862979837405489\\
89.21	0.00863154539403883\\
89.22	0.00863329317016097\\
89.23	0.00863504170284613\\
89.24	0.00863679099251941\\
89.25	0.00863854103960614\\
89.26	0.0086402918445319\\
89.27	0.00864204340772254\\
89.28	0.00864379572960416\\
89.29	0.00864554881060311\\
89.3	0.00864730265114598\\
89.31	0.00864905725165964\\
89.32	0.0086508126125712\\
89.33	0.00865256873430801\\
89.34	0.00865432561729769\\
89.35	0.0086560832619681\\
89.36	0.00865784166874737\\
89.37	0.00865960083806387\\
89.38	0.00866136077034622\\
89.39	0.00866312146602329\\
89.4	0.00866488292552421\\
89.41	0.00866664514927836\\
89.42	0.00866840813771538\\
89.43	0.00867017189126514\\
89.44	0.00867193641035778\\
89.45	0.00867370169542367\\
89.46	0.00867546774689346\\
89.47	0.00867723456519802\\
89.48	0.00867900215076851\\
89.49	0.0086807705040363\\
89.5	0.00868253962543303\\
89.51	0.00868430951539058\\
89.52	0.0086860801743411\\
89.53	0.00868785160271697\\
89.54	0.00868962380095083\\
89.55	0.00869139676947556\\
89.56	0.0086931705087243\\
89.57	0.00869494501913042\\
89.58	0.00869672030112757\\
89.59	0.00869849635514963\\
89.6	0.00870027318163071\\
89.61	0.00870205078100521\\
89.62	0.00870382915370775\\
89.63	0.00870560830017319\\
89.64	0.00870738822083667\\
89.65	0.00870916891613354\\
89.66	0.00871095038649943\\
89.67	0.00871273263237019\\
89.68	0.00871451565418194\\
89.69	0.00871629945237103\\
89.7	0.00871808402737406\\
89.71	0.00871986937962787\\
89.72	0.00872165550956956\\
89.73	0.00872344241763647\\
89.74	0.00872523010426618\\
89.75	0.00872701856989652\\
89.76	0.00872880781496556\\
89.77	0.00873059783991162\\
89.78	0.00873238864517326\\
89.79	0.00873418023118928\\
89.8	0.00873597259839873\\
89.81	0.0087377657472409\\
89.82	0.00873955967815533\\
89.83	0.00874135439158179\\
89.84	0.0087431498879603\\
89.85	0.00874494616773112\\
89.86	0.00874674323133477\\
89.87	0.00874854107921196\\
89.88	0.0087503397118037\\
89.89	0.00875213912955121\\
89.9	0.00875393933289596\\
89.91	0.00875574032227965\\
89.92	0.00875754209814422\\
89.93	0.00875934466093186\\
89.94	0.008761148011085\\
89.95	0.00876295214904629\\
89.96	0.00876475707525864\\
89.97	0.00876656279016519\\
89.98	0.0087683692942093\\
89.99	0.0087701765878346\\
90	0.00877198467148493\\
90.01	0.00877379354560438\\
90.02	0.00877560321063727\\
90.03	0.00877741366702815\\
90.04	0.00877922491522182\\
90.05	0.0087810369556633\\
90.06	0.00878284978879786\\
90.07	0.00878466341507099\\
90.08	0.00878647783492841\\
90.09	0.00878829304881609\\
90.1	0.00879010905718021\\
90.11	0.00879192586046721\\
90.12	0.00879374345912374\\
90.13	0.00879556185359668\\
90.14	0.00879738104433315\\
90.15	0.0087992010317805\\
90.16	0.0088010218163863\\
90.17	0.00880284339859836\\
90.18	0.00880466577886472\\
90.19	0.00880648895763363\\
90.2	0.00880831293535358\\
90.21	0.0088101377124733\\
90.22	0.00881196328944171\\
90.23	0.00881378966670798\\
90.24	0.00881561684472152\\
90.25	0.00881744482393193\\
90.26	0.00881927360478906\\
90.27	0.00882110318774296\\
90.28	0.00882293357324393\\
90.29	0.00882476476174247\\
90.3	0.00882659675368931\\
90.31	0.00882842954953541\\
90.32	0.00883026314973193\\
90.33	0.00883209755473026\\
90.34	0.00883393276498201\\
90.35	0.00883576878093901\\
90.36	0.0088376056030533\\
90.37	0.00883944323177713\\
90.38	0.008841281667563\\
90.39	0.00884312091086359\\
90.4	0.0088449609621318\\
90.41	0.00884680182182076\\
90.42	0.0088486434903838\\
90.43	0.00885048596827445\\
90.44	0.00885232925594649\\
90.45	0.00885417335385387\\
90.46	0.00885601826245078\\
90.47	0.00885786398219159\\
90.48	0.00885971051353091\\
90.49	0.00886155785692353\\
90.5	0.00886340601282446\\
90.51	0.00886525498168892\\
90.52	0.00886710476397231\\
90.53	0.00886895536013028\\
90.54	0.00887080677061862\\
90.55	0.00887265899589339\\
90.56	0.00887451203641079\\
90.57	0.00887636589262726\\
90.58	0.00887822056499943\\
90.59	0.00888007605398412\\
90.6	0.00888193236003834\\
90.61	0.00888378948361933\\
90.62	0.00888564742518448\\
90.63	0.00888750618519141\\
90.64	0.00888936576409792\\
90.65	0.00889122616236199\\
90.66	0.00889308738044181\\
90.67	0.00889494941879575\\
90.68	0.00889681227788237\\
90.69	0.00889867595816042\\
90.7	0.00890054046008882\\
90.71	0.0089024057841267\\
90.72	0.00890427193073335\\
90.73	0.00890613890036827\\
90.74	0.00890800669349111\\
90.75	0.00890987531056173\\
90.76	0.00891174475204014\\
90.77	0.00891361501838654\\
90.78	0.00891548611006132\\
90.79	0.00891735802752502\\
90.8	0.00891923077123837\\
90.81	0.00892110434166227\\
90.82	0.00892297873925779\\
90.83	0.00892485396448615\\
90.84	0.00892673001780876\\
90.85	0.0089286068996872\\
90.86	0.0089304846105832\\
90.87	0.00893236315095866\\
90.88	0.00893424252127563\\
90.89	0.00893612272199634\\
90.9	0.00893800375358315\\
90.91	0.00893988561649862\\
90.92	0.00894176831120543\\
90.93	0.00894365183816641\\
90.94	0.00894553619784457\\
90.95	0.00894742139070305\\
90.96	0.00894930741720514\\
90.97	0.00895119427781428\\
90.98	0.00895308197299406\\
90.99	0.00895497050320821\\
91	0.00895685986892058\\
91.01	0.00895875007059519\\
91.02	0.00896064110869618\\
91.03	0.00896253298368783\\
91.04	0.00896442569603456\\
91.05	0.00896631924620091\\
91.06	0.00896821363465154\\
91.07	0.00897010886185126\\
91.08	0.008972004928265\\
91.09	0.0089739018343578\\
91.1	0.00897579958059483\\
91.11	0.00897769816744137\\
91.12	0.00897959759536284\\
91.13	0.00898149786482474\\
91.14	0.0089833989762927\\
91.15	0.00898530093023246\\
91.16	0.00898720372710987\\
91.17	0.00898910736739088\\
91.18	0.00899101185154153\\
91.19	0.00899291718002797\\
91.2	0.00899482335331646\\
91.21	0.00899673037187333\\
91.22	0.00899863823616501\\
91.23	0.00900054694665803\\
91.24	0.009002456503819\\
91.25	0.0090043669081146\\
91.26	0.00900627816001161\\
91.27	0.00900819025997687\\
91.28	0.00901010320847733\\
91.29	0.00901201700597996\\
91.3	0.00901393165295184\\
91.31	0.0090158471498601\\
91.32	0.00901776349717194\\
91.33	0.00901968069535461\\
91.34	0.00902159874487542\\
91.35	0.00902351764620174\\
91.36	0.00902543739980098\\
91.37	0.00902735800614061\\
91.38	0.00902927946568814\\
91.39	0.00903120177891111\\
91.4	0.0090331249462771\\
91.41	0.00903504896825373\\
91.42	0.00903697384530865\\
91.43	0.00903889957790953\\
91.44	0.00904082616652405\\
91.45	0.00904275361161994\\
91.46	0.00904468191366493\\
91.47	0.00904661107312673\\
91.48	0.00904854109047312\\
91.49	0.00905047196617182\\
91.5	0.00905240370069058\\
91.51	0.00905433629449715\\
91.52	0.00905626974805926\\
91.53	0.00905820406184461\\
91.54	0.00906013923632092\\
91.55	0.00906207527195584\\
91.56	0.00906401216921704\\
91.57	0.00906594992857213\\
91.58	0.00906788855048868\\
91.59	0.00906982803543424\\
91.6	0.00907176838387629\\
91.61	0.00907370959628228\\
91.62	0.00907565167311959\\
91.63	0.00907759461485555\\
91.64	0.00907953842195741\\
91.65	0.00908148309489236\\
91.66	0.00908342863412751\\
91.67	0.00908537504012988\\
91.68	0.00908732231336642\\
91.69	0.00908927045430398\\
91.7	0.0090912194634093\\
91.71	0.00909316934114903\\
91.72	0.0090951200879897\\
91.73	0.00909707170439773\\
91.74	0.00909902419083941\\
91.75	0.0091009775477809\\
91.76	0.00910293177568824\\
91.77	0.00910488687502733\\
91.78	0.00910684284626389\\
91.79	0.00910879968986351\\
91.8	0.00911075740629163\\
91.81	0.0091127159960135\\
91.82	0.0091146754594942\\
91.83	0.00911663579719862\\
91.84	0.0091185970095915\\
91.85	0.00912055909713732\\
91.86	0.00912252206030041\\
91.87	0.00912448589954485\\
91.88	0.00912645061533453\\
91.89	0.0091284162081331\\
91.9	0.00913038267840396\\
91.91	0.00913235002661028\\
91.92	0.00913431825321499\\
91.93	0.00913628735868072\\
91.94	0.00913825734346987\\
91.95	0.00914022820804454\\
91.96	0.00914219995286654\\
91.97	0.00914417257839739\\
91.98	0.00914614608509831\\
91.99	0.00914812047343018\\
92	0.00915009574385358\\
92.01	0.00915207189682873\\
92.02	0.00915404893281552\\
92.03	0.00915602685227347\\
92.04	0.00915800565566174\\
92.05	0.0091599853434391\\
92.06	0.00916196591606395\\
92.07	0.00916394737399427\\
92.08	0.00916592971768763\\
92.09	0.00916791294760117\\
92.1	0.00916989706419161\\
92.11	0.0091718820679152\\
92.12	0.00917386795922772\\
92.13	0.00917585473858451\\
92.14	0.00917784240644038\\
92.15	0.00917983096324966\\
92.16	0.00918182040946616\\
92.17	0.00918381074554313\\
92.18	0.00918580197193333\\
92.19	0.0091877940890889\\
92.2	0.00918978709746143\\
92.21	0.00919178099750192\\
92.22	0.00919377578966075\\
92.23	0.00919577147438767\\
92.24	0.00919776805213181\\
92.25	0.0091997655233416\\
92.26	0.00920176388846483\\
92.27	0.00920376314794858\\
92.28	0.00920576330223919\\
92.29	0.0092077643517823\\
92.3	0.00920976629702277\\
92.31	0.00921176913840468\\
92.32	0.00921377287637134\\
92.33	0.00921577751136521\\
92.34	0.00921778304382792\\
92.35	0.00921978947420024\\
92.36	0.00922179680292203\\
92.37	0.00922380503043228\\
92.38	0.00922581415716898\\
92.39	0.00922782418356922\\
92.4	0.00922983511006904\\
92.41	0.00923184693710351\\
92.42	0.00923385966510662\\
92.43	0.0092358732945113\\
92.44	0.00923788782574938\\
92.45	0.00923990325925152\\
92.46	0.00924191959544725\\
92.47	0.00924393683476488\\
92.48	0.00924595497763148\\
92.49	0.00924797402447285\\
92.5	0.0092499939757135\\
92.51	0.00925201483177656\\
92.52	0.00925403659308381\\
92.53	0.00925605926005558\\
92.54	0.00925808283311075\\
92.55	0.00926010731266668\\
92.56	0.00926213269913919\\
92.57	0.00926415899294249\\
92.58	0.00926618619448914\\
92.59	0.00926821430419001\\
92.6	0.00927024332245423\\
92.61	0.00927227324968911\\
92.62	0.00927430408630013\\
92.63	0.00927633583269085\\
92.64	0.00927836848926284\\
92.65	0.00928040205641568\\
92.66	0.00928243653454681\\
92.67	0.00928447192405156\\
92.68	0.009286508225323\\
92.69	0.00928854543875191\\
92.7	0.0092905835647267\\
92.71	0.00929262260363335\\
92.72	0.0092946625558553\\
92.73	0.00929670342177339\\
92.74	0.00929874520176577\\
92.75	0.00930078789620781\\
92.76	0.00930283150547201\\
92.77	0.00930487602992793\\
92.78	0.00930692146994204\\
92.79	0.00930896782587765\\
92.8	0.00931101509809481\\
92.81	0.00931306328695018\\
92.82	0.00931511239279693\\
92.83	0.00931716241598462\\
92.84	0.00931921335685906\\
92.85	0.0093212652157622\\
92.86	0.00932331799303198\\
92.87	0.00932537168900221\\
92.88	0.0093274263040024\\
92.89	0.00932948183835763\\
92.9	0.00933153829238837\\
92.91	0.00933359566641037\\
92.92	0.0093356539607344\\
92.93	0.00933771317566615\\
92.94	0.00933977331150602\\
92.95	0.00934183436854892\\
92.96	0.00934389634708409\\
92.97	0.00934595924739485\\
92.98	0.00934802306975844\\
92.99	0.00935008781444575\\
93	0.00935215348172111\\
93.01	0.00935422007184203\\
93.02	0.00935628758505895\\
93.03	0.00935835602161498\\
93.04	0.00936042538174561\\
93.05	0.00936249566567844\\
93.06	0.00936456687363287\\
93.07	0.00936663900581977\\
93.08	0.00936871206244118\\
93.09	0.00937078604368996\\
93.1	0.00937286094974944\\
93.11	0.00937493678079303\\
93.12	0.00937701353698386\\
93.13	0.00937909121847438\\
93.14	0.00938116982540592\\
93.15	0.00938324935790826\\
93.16	0.00938532981609921\\
93.17	0.00938741120008407\\
93.18	0.0093894935099552\\
93.19	0.00939157674579147\\
93.2	0.00939366090765776\\
93.21	0.00939574599560433\\
93.22	0.00939783200966631\\
93.23	0.00939991894986307\\
93.24	0.00940200681619755\\
93.25	0.00940409560865568\\
93.26	0.00940618532721096\\
93.27	0.00940827597182494\\
93.28	0.00941036754244687\\
93.29	0.00941246003901342\\
93.3	0.0094145534614484\\
93.31	0.00941664780966238\\
93.32	0.00941874308355246\\
93.33	0.00942083928300185\\
93.34	0.00942293640787958\\
93.35	0.00942503445804016\\
93.36	0.00942713343332319\\
93.37	0.009429233333553\\
93.38	0.00943133415853831\\
93.39	0.0094334359080718\\
93.4	0.00943553858192976\\
93.41	0.00943764217987163\\
93.42	0.00943974670163962\\
93.43	0.00944185214695828\\
93.44	0.00944395851553408\\
93.45	0.0094460658070549\\
93.46	0.00944817402118964\\
93.47	0.00945028315758769\\
93.48	0.00945239321587849\\
93.49	0.00945450419567099\\
93.5	0.00945661609655316\\
93.51	0.00945872891809147\\
93.52	0.00946084265983034\\
93.53	0.00946295732129159\\
93.54	0.00946507290197385\\
93.55	0.00946718940135202\\
93.56	0.00946930681887665\\
93.57	0.00947142515397328\\
93.58	0.0094735444060419\\
93.59	0.00947566457445622\\
93.6	0.00947778565856304\\
93.61	0.00947990765768156\\
93.62	0.00948203057110268\\
93.63	0.00948415439808829\\
93.64	0.00948627913787048\\
93.65	0.00948840478965085\\
93.66	0.00949053135259967\\
93.67	0.00949265882585511\\
93.68	0.00949478720852241\\
93.69	0.00949691649967302\\
93.7	0.00949904669834373\\
93.71	0.00950117780353581\\
93.72	0.00950330981421404\\
93.73	0.00950544272930582\\
93.74	0.00950757654770014\\
93.75	0.00950971126824664\\
93.76	0.00951184688975457\\
93.77	0.00951398341099173\\
93.78	0.00951612083068338\\
93.79	0.00951825914751115\\
93.8	0.00952039836011191\\
93.81	0.00952253846707655\\
93.82	0.00952467946694883\\
93.83	0.00952682135822413\\
93.84	0.00952896413934811\\
93.85	0.00953110780871553\\
93.86	0.00953325236466878\\
93.87	0.00953539780549657\\
93.88	0.00953754412943249\\
93.89	0.00953969133465357\\
93.9	0.00954183941927873\\
93.91	0.00954398838136728\\
93.92	0.00954613821891733\\
93.93	0.00954828892986415\\
93.94	0.00955044051207848\\
93.95	0.00955259296336482\\
93.96	0.00955474628145968\\
93.97	0.00955690046402971\\
93.98	0.00955905550866988\\
93.99	0.00956121141290149\\
94	0.00956336817417023\\
94.01	0.00956552578984417\\
94.02	0.00956768425721157\\
94.03	0.00956984357347883\\
94.04	0.0095720037357682\\
94.05	0.00957416474111552\\
94.06	0.00957632658646791\\
94.07	0.00957848926868131\\
94.08	0.00958065278451799\\
94.09	0.00958281713064407\\
94.1	0.00958498230362682\\
94.11	0.00958714829993199\\
94.12	0.00958931511592103\\
94.13	0.00959148274784822\\
94.14	0.00959365119185774\\
94.15	0.00959582044398067\\
94.16	0.0095979905001318\\
94.17	0.00960016135610651\\
94.18	0.00960233300757742\\
94.19	0.00960450545009101\\
94.2	0.00960667867906417\\
94.21	0.00960885268978054\\
94.22	0.00961102747738685\\
94.23	0.00961320303688914\\
94.24	0.00961537936314879\\
94.25	0.00961755645087854\\
94.26	0.00961973429463832\\
94.27	0.00962191288883099\\
94.28	0.00962409222769793\\
94.29	0.00962627230531455\\
94.3	0.00962845311558563\\
94.31	0.00963063465224048\\
94.32	0.0096328169088281\\
94.33	0.00963499987871202\\
94.34	0.00963718355506511\\
94.35	0.0096393679308642\\
94.36	0.00964155299888456\\
94.37	0.00964373875169415\\
94.38	0.0096459251816478\\
94.39	0.00964811228088112\\
94.4	0.00965030004130433\\
94.41	0.00965248845459579\\
94.42	0.00965467751219547\\
94.43	0.00965686720529811\\
94.44	0.00965905752484621\\
94.45	0.00966124846152291\\
94.46	0.00966344000574446\\
94.47	0.00966563214765266\\
94.48	0.00966782487710699\\
94.49	0.00967001818367644\\
94.5	0.00967221205663125\\
94.51	0.00967440648493425\\
94.52	0.00967660145723207\\
94.53	0.00967879696184596\\
94.54	0.00968099298676245\\
94.55	0.00968318951962364\\
94.56	0.00968538654771726\\
94.57	0.00968758405796641\\
94.58	0.00968978203691899\\
94.59	0.00969198047073677\\
94.6	0.00969417934518424\\
94.61	0.00969637864561698\\
94.62	0.00969857835696983\\
94.63	0.00970077846374455\\
94.64	0.00970297894999724\\
94.65	0.0097051797993253\\
94.66	0.00970738099485401\\
94.67	0.00970958251922271\\
94.68	0.00971178435457059\\
94.69	0.00971398648252196\\
94.7	0.0097161888841712\\
94.71	0.00971839154006714\\
94.72	0.00972059443019699\\
94.73	0.00972279753396986\\
94.74	0.00972500083019966\\
94.75	0.00972720429708757\\
94.76	0.00972940791220391\\
94.77	0.0097316116524695\\
94.78	0.00973381549413645\\
94.79	0.0097360194127683\\
94.8	0.00973822338321966\\
94.81	0.00974042737961512\\
94.82	0.00974263137532758\\
94.83	0.00974483534295588\\
94.84	0.00974703925430178\\
94.85	0.00974924308034618\\
94.86	0.00975144679122465\\
94.87	0.00975365035620222\\
94.88	0.00975585374364733\\
94.89	0.00975805692100504\\
94.9	0.00976025985476939\\
94.91	0.00976246251045491\\
94.92	0.00976466485256723\\
94.93	0.00976686684457283\\
94.94	0.0097690684488678\\
94.95	0.00977126962674569\\
94.96	0.00977347033836431\\
94.97	0.00977567054271154\\
94.98	0.00977787019757004\\
94.99	0.00978006925948098\\
95	0.00978226768370643\\
95.01	0.00978446542419082\\
95.02	0.00978666243352106\\
95.03	0.00978885866288544\\
95.04	0.00979105406203127\\
95.05	0.00979324857922124\\
95.06	0.00979544216119451\\
95.07	0.00979763475312522\\
95.08	0.00979982629857534\\
95.09	0.00980201673944598\\
95.1	0.0098042060159273\\
95.11	0.0098063940664468\\
95.12	0.009808580827616\\
95.13	0.00981076623417556\\
95.14	0.00981295021893856\\
95.15	0.00981513271273214\\
95.16	0.0098173136443373\\
95.17	0.00981949294042679\\
95.18	0.00982167052550111\\
95.19	0.00982384632182251\\
95.2	0.00982602024934697\\
95.21	0.00982819222565397\\
95.22	0.00983036216587423\\
95.23	0.00983252998261504\\
95.24	0.00983469558588341\\
95.25	0.00983685888300667\\
95.26	0.00983901977855079\\
95.27	0.00984117817423594\\
95.28	0.00984333396884961\\
95.29	0.0098454870581569\\
95.3	0.00984763733480804\\
95.31	0.00984978468824303\\
95.32	0.00985192900459333\\
95.33	0.00985407016658042\\
95.34	0.00985620805341122\\
95.35	0.00985834254067026\\
95.36	0.00986047350020845\\
95.37	0.00986260080002838\\
95.38	0.00986472430416601\\
95.39	0.00986684387256871\\
95.4	0.00986895936096943\\
95.41	0.00987107062075693\\
95.42	0.009873177498842\\
95.43	0.00987527983751939\\
95.44	0.00987737747432551\\
95.45	0.00987947024189163\\
95.46	0.00988155796779243\\
95.47	0.00988364047438985\\
95.48	0.00988571757867207\\
95.49	0.00988778909208732\\
95.5	0.00988985482037259\\
95.51	0.0098919145633769\\
95.52	0.00989396811487902\\
95.53	0.0098960152623995\\
95.54	0.00989805578700672\\
95.55	0.00990008946311692\\
95.56	0.00990211605828789\\
95.57	0.00990413533300623\\
95.58	0.0099061470404678\\
95.59	0.00990815092635141\\
95.6	0.00991014672858531\\
95.61	0.00991213417710632\\
95.62	0.00991411299361142\\
95.63	0.00991608289130149\\
95.64	0.00991804357461702\\
95.65	0.00991999473896542\\
95.66	0.00992193607043988\\
95.67	0.0099238672455292\\
95.68	0.00992578793081867\\
95.69	0.00992769778268137\\
95.7	0.00992959644695984\\
95.71	0.00993148355863768\\
95.72	0.00993335874150087\\
95.73	0.00993522160778832\\
95.74	0.00993707175783149\\
95.75	0.00993890877968268\\
95.76	0.0099407322487315\\
95.77	0.00994254172730939\\
95.78	0.00994433676428161\\
95.79	0.00994611689462638\\
95.8	0.00994788163900083\\
95.81	0.00994963050329322\\
95.82	0.0099513629781611\\
95.83	0.00995307853855489\\
95.84	0.00995477664322653\\
95.85	0.00995645673422267\\
95.86	0.00995811823636179\\
95.87	0.00995976055669505\\
95.88	0.00996138308395009\\
95.89	0.00996298518795734\\
95.9	0.00996456621905835\\
95.91	0.00996612550749545\\
95.92	0.00996766236278238\\
95.93	0.00996917607305495\\
95.94	0.00997066590440153\\
95.95	0.00997213110017244\\
95.96	0.0099735708802676\\
95.97	0.009974984440402\\
95.98	0.00997637095134808\\
95.99	0.00997772955815433\\
96	0.00997905937933954\\
96.01	0.00998035950606173\\
96.02	0.00998162900126117\\
96.03	0.0099828668987766\\
96.04	0.00998407220243377\\
96.05	0.00998524388510568\\
96.06	0.00998638088774336\\
96.07	0.00998748211837652\\
96.08	0.00998854645108307\\
96.09	0.00998957272492655\\
96.1	0.00999055974286051\\
96.11	0.00999150627059886\\
96.12	0.00999241103545119\\
96.13	0.00999327272512193\\
96.14	0.00999408998647227\\
96.15	0.00999486142424387\\
96.16	0.00999558559974299\\
96.17	0.00999626102948407\\
96.18	0.00999688618379141\\
96.19	0.00999745948535783\\
96.2	0.00999797930775888\\
96.21	0.00999844397392151\\
96.22	0.00999885175454561\\
96.23	0.00999920086647735\\
96.24	0.00999948947103253\\
96.25	0.00999971567226888\\
96.26	0.00999987751520559\\
96.27	0.00999997298398856\\
96.28	0.01\\
96.29	0.01\\
96.3	0.01\\
96.31	0.01\\
96.32	0.01\\
96.33	0.01\\
96.34	0.01\\
96.35	0.01\\
96.36	0.01\\
96.37	0.01\\
96.38	0.01\\
96.39	0.01\\
96.4	0.01\\
96.41	0.01\\
96.42	0.01\\
96.43	0.01\\
96.44	0.01\\
96.45	0.01\\
96.46	0.01\\
96.47	0.01\\
96.48	0.01\\
96.49	0.01\\
96.5	0.01\\
96.51	0.01\\
96.52	0.01\\
96.53	0.01\\
96.54	0.01\\
96.55	0.01\\
96.56	0.01\\
96.57	0.01\\
96.58	0.01\\
96.59	0.01\\
96.6	0.01\\
96.61	0.01\\
96.62	0.01\\
96.63	0.01\\
96.64	0.01\\
96.65	0.01\\
96.66	0.01\\
96.67	0.01\\
96.68	0.01\\
96.69	0.01\\
96.7	0.01\\
96.71	0.01\\
96.72	0.01\\
96.73	0.01\\
96.74	0.01\\
96.75	0.01\\
96.76	0.01\\
96.77	0.01\\
96.78	0.01\\
96.79	0.01\\
96.8	0.01\\
96.81	0.01\\
96.82	0.01\\
96.83	0.01\\
96.84	0.01\\
96.85	0.01\\
96.86	0.01\\
96.87	0.01\\
96.88	0.01\\
96.89	0.01\\
96.9	0.01\\
96.91	0.01\\
96.92	0.01\\
96.93	0.01\\
96.94	0.01\\
96.95	0.01\\
96.96	0.01\\
96.97	0.01\\
96.98	0.01\\
96.99	0.01\\
97	0.01\\
97.01	0.01\\
97.02	0.01\\
97.03	0.01\\
97.04	0.01\\
97.05	0.01\\
97.06	0.01\\
97.07	0.01\\
97.08	0.01\\
97.09	0.01\\
97.1	0.01\\
97.11	0.01\\
97.12	0.01\\
97.13	0.01\\
97.14	0.01\\
97.15	0.01\\
97.16	0.01\\
97.17	0.01\\
97.18	0.01\\
97.19	0.01\\
97.2	0.01\\
97.21	0.01\\
97.22	0.01\\
97.23	0.01\\
97.24	0.01\\
97.25	0.01\\
97.26	0.01\\
97.27	0.01\\
97.28	0.01\\
97.29	0.01\\
97.3	0.01\\
97.31	0.01\\
97.32	0.01\\
97.33	0.01\\
97.34	0.01\\
97.35	0.01\\
97.36	0.01\\
97.37	0.01\\
97.38	0.01\\
97.39	0.01\\
97.4	0.01\\
97.41	0.01\\
97.42	0.01\\
97.43	0.01\\
97.44	0.01\\
97.45	0.01\\
97.46	0.01\\
97.47	0.01\\
97.48	0.01\\
97.49	0.01\\
97.5	0.01\\
97.51	0.01\\
97.52	0.01\\
97.53	0.01\\
97.54	0.01\\
97.55	0.01\\
97.56	0.01\\
97.57	0.01\\
97.58	0.01\\
97.59	0.01\\
97.6	0.01\\
97.61	0.01\\
97.62	0.01\\
97.63	0.01\\
97.64	0.01\\
97.65	0.01\\
97.66	0.01\\
97.67	0.01\\
97.68	0.01\\
97.69	0.01\\
97.7	0.01\\
97.71	0.01\\
97.72	0.01\\
97.73	0.01\\
97.74	0.01\\
97.75	0.01\\
97.76	0.01\\
97.77	0.01\\
97.78	0.01\\
97.79	0.01\\
97.8	0.01\\
97.81	0.01\\
97.82	0.01\\
97.83	0.01\\
97.84	0.01\\
97.85	0.01\\
97.86	0.01\\
97.87	0.01\\
97.88	0.01\\
97.89	0.01\\
97.9	0.01\\
97.91	0.01\\
97.92	0.01\\
97.93	0.01\\
97.94	0.01\\
97.95	0.01\\
97.96	0.01\\
97.97	0.01\\
97.98	0.01\\
97.99	0.01\\
98	0.01\\
98.01	0.01\\
98.02	0.01\\
98.03	0.01\\
98.04	0.01\\
98.05	0.01\\
98.06	0.01\\
98.07	0.01\\
98.08	0.01\\
98.09	0.01\\
98.1	0.01\\
98.11	0.01\\
98.12	0.01\\
98.13	0.01\\
98.14	0.01\\
98.15	0.01\\
98.16	0.01\\
98.17	0.01\\
98.18	0.01\\
98.19	0.01\\
98.2	0.01\\
98.21	0.01\\
98.22	0.01\\
98.23	0.01\\
98.24	0.01\\
98.25	0.01\\
98.26	0.01\\
98.27	0.01\\
98.28	0.01\\
98.29	0.01\\
98.3	0.01\\
98.31	0.01\\
98.32	0.01\\
98.33	0.01\\
98.34	0.01\\
98.35	0.01\\
98.36	0.01\\
98.37	0.01\\
98.38	0.01\\
98.39	0.01\\
98.4	0.01\\
98.41	0.01\\
98.42	0.01\\
98.43	0.01\\
98.44	0.01\\
98.45	0.01\\
98.46	0.01\\
98.47	0.01\\
98.48	0.01\\
98.49	0.01\\
98.5	0.01\\
98.51	0.01\\
98.52	0.01\\
98.53	0.01\\
98.54	0.01\\
98.55	0.01\\
98.56	0.01\\
98.57	0.01\\
98.58	0.01\\
98.59	0.01\\
98.6	0.01\\
98.61	0.01\\
98.62	0.01\\
98.63	0.01\\
98.64	0.01\\
98.65	0.01\\
98.66	0.01\\
98.67	0.01\\
98.68	0.01\\
98.69	0.01\\
98.7	0.01\\
98.71	0.01\\
98.72	0.01\\
98.73	0.01\\
98.74	0.01\\
98.75	0.01\\
98.76	0.01\\
98.77	0.01\\
98.78	0.01\\
98.79	0.01\\
98.8	0.01\\
98.81	0.01\\
98.82	0.01\\
98.83	0.01\\
98.84	0.01\\
98.85	0.01\\
98.86	0.01\\
98.87	0.01\\
98.88	0.01\\
98.89	0.01\\
98.9	0.01\\
98.91	0.01\\
98.92	0.01\\
98.93	0.01\\
98.94	0.01\\
98.95	0.01\\
98.96	0.01\\
98.97	0.01\\
98.98	0.01\\
98.99	0.01\\
99	0.01\\
99.01	0.01\\
99.02	0.01\\
99.03	0.01\\
99.04	0.01\\
99.05	0.01\\
99.06	0.01\\
99.07	0.01\\
99.08	0.01\\
99.09	0.01\\
99.1	0.01\\
99.11	0.01\\
99.12	0.01\\
99.13	0.01\\
99.14	0.01\\
99.15	0.01\\
99.16	0.01\\
99.17	0.01\\
99.18	0.01\\
99.19	0.01\\
99.2	0.01\\
99.21	0.01\\
99.22	0.01\\
99.23	0.01\\
99.24	0.01\\
99.25	0.01\\
99.26	0.01\\
99.27	0.01\\
99.28	0.01\\
99.29	0.01\\
99.3	0.01\\
99.31	0.01\\
99.32	0.01\\
99.33	0.01\\
99.34	0.01\\
99.35	0.01\\
99.36	0.01\\
99.37	0.01\\
99.38	0.01\\
99.39	0.01\\
99.4	0.01\\
99.41	0.01\\
99.42	0.01\\
99.43	0.01\\
99.44	0.01\\
99.45	0.01\\
99.46	0.01\\
99.47	0.01\\
99.48	0.01\\
99.49	0.01\\
99.5	0.01\\
99.51	0.01\\
99.52	0.01\\
99.53	0.01\\
99.54	0.01\\
99.55	0.01\\
99.56	0.01\\
99.57	0.01\\
99.58	0.01\\
99.59	0.01\\
99.6	0.01\\
99.61	0.01\\
99.62	0.01\\
99.63	0.01\\
99.64	0.01\\
99.65	0.01\\
99.66	0.01\\
99.67	0.01\\
99.68	0.01\\
99.69	0.01\\
99.7	0.01\\
99.71	0.01\\
99.72	0.01\\
99.73	0.01\\
99.74	0.01\\
99.75	0.01\\
99.76	0.01\\
99.77	0.01\\
99.78	0.01\\
99.79	0.01\\
99.8	0.01\\
99.81	0.01\\
99.82	0.01\\
99.83	0.01\\
99.84	0.01\\
99.85	0.01\\
99.86	0.01\\
99.87	0.01\\
99.88	0.01\\
99.89	0.01\\
99.9	0.01\\
99.91	0.01\\
99.92	0.01\\
99.93	0.01\\
99.94	0.01\\
99.95	0.01\\
99.96	0.01\\
99.97	0.01\\
99.98	0.01\\
99.99	0.01\\
100	0.01\\
};
\addlegendentry{$q=1$};

\addplot [color=red,solid,forget plot]
  table[row sep=crcr]{%
0.01	0.00819891155222631\\
0.02	0.00819892028165661\\
0.03	0.00819892904835735\\
0.04	0.00819893785313367\\
0.05	0.00819894669681198\\
0.06	0.00819895558024053\\
0.07	0.00819896450429004\\
0.08	0.00819897346985436\\
0.09	0.00819898247785111\\
0.1	0.00819899152922234\\
0.11	0.00819900062493528\\
0.12	0.00819900976598298\\
0.13	0.00819901895338512\\
0.14	0.00819902818818874\\
0.15	0.008199037471469\\
0.16	0.00819904680433005\\
0.17	0.00819905618790579\\
0.18	0.00819906562336078\\
0.19	0.00819907511189112\\
0.2	0.00819908465472531\\
0.21	0.00819909425312523\\
0.22	0.00819910390838711\\
0.23	0.00819911362184249\\
0.24	0.00819912339485927\\
0.25	0.00819913322884277\\
0.26	0.00819914312523678\\
0.27	0.00819915308552473\\
0.28	0.00819916311123079\\
0.29	0.0081991732039211\\
0.3	0.00819918336520501\\
0.31	0.00819919359673626\\
0.32	0.00819920390021439\\
0.33	0.00819921427738599\\
0.34	0.00819922473004613\\
0.35	0.00819923526003979\\
0.36	0.00819924586926327\\
0.37	0.00819925655966577\\
0.38	0.00819926733325091\\
0.39	0.00819927819207835\\
0.4	0.00819928913826543\\
0.41	0.00819930017398889\\
0.42	0.00819931130148663\\
0.43	0.0081993225230595\\
0.44	0.00819933384107321\\
0.45	0.0081993452579602\\
0.46	0.00819935677622167\\
0.47	0.00819936839842962\\
0.48	0.00819938012722892\\
0.49	0.00819939196533953\\
0.5	0.00819940391555873\\
0.51	0.0081994159807634\\
0.52	0.00819942816391246\\
0.53	0.00819944046804924\\
0.54	0.00819945289630409\\
0.55	0.00819946545189698\\
0.56	0.00819947813814011\\
0.57	0.00819949095844079\\
0.58	0.00819950391630424\\
0.59	0.00819951701533655\\
0.6	0.0081995302592477\\
0.61	0.00819954365185475\\
0.62	0.00819955719708499\\
0.63	0.00819957089897935\\
0.64	0.00819958476169579\\
0.65	0.00819959878951284\\
0.66	0.00819961298683328\\
0.67	0.0081996273581879\\
0.68	0.00819964190823934\\
0.69	0.00819965664178616\\
0.7	0.00819967156376693\\
0.71	0.0081996866792645\\
0.72	0.00819970199351038\\
0.73	0.00819971751188928\\
0.74	0.00819973314568682\\
0.75	0.00819974878548379\\
0.76	0.00819976443128269\\
0.77	0.00819978008308601\\
0.78	0.00819979574089625\\
0.79	0.00819981140471589\\
0.8	0.00819982707454744\\
0.81	0.00819984275039339\\
0.82	0.00819985843225624\\
0.83	0.00819987412013849\\
0.84	0.00819988981404265\\
0.85	0.00819990551397121\\
0.86	0.00819992121992667\\
0.87	0.00819993693191155\\
0.88	0.00819995264992835\\
0.89	0.00819996837397957\\
0.9	0.00819998410406773\\
0.91	0.00819999984019533\\
0.92	0.00820001558236488\\
0.93	0.0082000313305789\\
0.94	0.0082000470848399\\
0.95	0.00820006284515039\\
0.96	0.00820007861151289\\
0.97	0.00820009438392992\\
0.98	0.00820011016240399\\
0.99	0.00820012594693762\\
1	0.00820014173753334\\
1.01	0.00820015753419366\\
1.02	0.00820017333692111\\
1.03	0.00820018914571821\\
1.04	0.00820020496058748\\
1.05	0.00820022078153146\\
1.06	0.00820023660855267\\
1.07	0.00820025244165365\\
1.08	0.00820026828083691\\
1.09	0.008200284126105\\
1.1	0.00820029997746043\\
1.11	0.00820031583490576\\
1.12	0.00820033169844351\\
1.13	0.00820034756807622\\
1.14	0.00820036344380643\\
1.15	0.00820037932563668\\
1.16	0.0082003952135695\\
1.17	0.00820041110760744\\
1.18	0.00820042700775304\\
1.19	0.00820044291400883\\
1.2	0.00820045882637738\\
1.21	0.00820047474486123\\
1.22	0.00820049066946291\\
1.23	0.00820050660018499\\
1.24	0.00820052253703\\
1.25	0.00820053848000051\\
1.26	0.00820055442909906\\
1.27	0.00820057038432821\\
1.28	0.00820058634569051\\
1.29	0.00820060231318851\\
1.3	0.00820061828682479\\
1.31	0.00820063426660189\\
1.32	0.00820065025252237\\
1.33	0.00820066624458879\\
1.34	0.00820068224280372\\
1.35	0.00820069824716972\\
1.36	0.00820071425768936\\
1.37	0.0082007302743652\\
1.38	0.0082007462971998\\
1.39	0.00820076232619574\\
1.4	0.00820077836135559\\
1.41	0.00820079440268192\\
1.42	0.00820081045017729\\
1.43	0.00820082650384429\\
1.44	0.00820084256368549\\
1.45	0.00820085862970345\\
1.46	0.00820087470190077\\
1.47	0.00820089078028002\\
1.48	0.00820090686484378\\
1.49	0.00820092295559463\\
1.5	0.00820093905253514\\
1.51	0.00820095515566791\\
1.52	0.00820097126499553\\
1.53	0.00820098738052057\\
1.54	0.00820100350224562\\
1.55	0.00820101963017328\\
1.56	0.00820103576430612\\
1.57	0.00820105190464675\\
1.58	0.00820106805119776\\
1.59	0.00820108420396173\\
1.6	0.00820110036294126\\
1.61	0.00820111652813896\\
1.62	0.00820113269955741\\
1.63	0.00820114887719921\\
1.64	0.00820116506106697\\
1.65	0.00820118125116329\\
1.66	0.00820119744749077\\
1.67	0.008201213650052\\
1.68	0.0082012298588496\\
1.69	0.00820124607388618\\
1.7	0.00820126229516434\\
1.71	0.00820127852268668\\
1.72	0.00820129475645583\\
1.73	0.00820131099647438\\
1.74	0.00820132724274496\\
1.75	0.00820134349527018\\
1.76	0.00820135975405265\\
1.77	0.00820137601909498\\
1.78	0.0082013922903998\\
1.79	0.00820140856796973\\
1.8	0.00820142485180738\\
1.81	0.00820144114191537\\
1.82	0.00820145743829634\\
1.83	0.0082014737409529\\
1.84	0.00820149004988768\\
1.85	0.0082015063651033\\
1.86	0.0082015226866024\\
1.87	0.0082015390143876\\
1.88	0.00820155534846153\\
1.89	0.00820157168882682\\
1.9	0.00820158803548612\\
1.91	0.00820160438844204\\
1.92	0.00820162074769723\\
1.93	0.00820163711325432\\
1.94	0.00820165348511596\\
1.95	0.00820166986328478\\
1.96	0.00820168624776342\\
1.97	0.00820170263855452\\
1.98	0.00820171903566073\\
1.99	0.0082017354390847\\
2	0.00820175184882906\\
2.01	0.00820176826489647\\
2.02	0.00820178468728957\\
2.03	0.00820180111601102\\
2.04	0.00820181755106346\\
2.05	0.00820183399244954\\
2.06	0.00820185044017193\\
2.07	0.00820186689423327\\
2.08	0.00820188335463622\\
2.09	0.00820189982138345\\
2.1	0.0082019162944776\\
2.11	0.00820193277392134\\
2.12	0.00820194925971733\\
2.13	0.00820196575186823\\
2.14	0.00820198225037671\\
2.15	0.00820199875524543\\
2.16	0.00820201526647707\\
2.17	0.00820203178407428\\
2.18	0.00820204830803973\\
2.19	0.0082020648383761\\
2.2	0.00820208137508606\\
2.21	0.00820209791817228\\
2.22	0.00820211446763744\\
2.23	0.00820213102348421\\
2.24	0.00820214758571527\\
2.25	0.0082021641543333\\
2.26	0.00820218072934097\\
2.27	0.00820219731074098\\
2.28	0.00820221389853599\\
2.29	0.0082022304927287\\
2.3	0.00820224709332179\\
2.31	0.00820226370031795\\
2.32	0.00820228031371986\\
2.33	0.00820229693353022\\
2.34	0.0082023135597517\\
2.35	0.00820233019238702\\
2.36	0.00820234683143885\\
2.37	0.0082023634769099\\
2.38	0.00820238012880285\\
2.39	0.00820239678712041\\
2.4	0.00820241345186527\\
2.41	0.00820243012304013\\
2.42	0.0082024468006477\\
2.43	0.00820246348469067\\
2.44	0.00820248017517175\\
2.45	0.00820249687209365\\
2.46	0.00820251357545907\\
2.47	0.00820253028527071\\
2.48	0.0082025470015313\\
2.49	0.00820256372424353\\
2.5	0.00820258045341012\\
2.51	0.00820259718903378\\
2.52	0.00820261393111723\\
2.53	0.00820263067966318\\
2.54	0.00820264743467435\\
2.55	0.00820266419615346\\
2.56	0.00820268096410322\\
2.57	0.00820269773852636\\
2.58	0.0082027145194256\\
2.59	0.00820273130680366\\
2.6	0.00820274810066327\\
2.61	0.00820276490100716\\
2.62	0.00820278170783805\\
2.63	0.00820279852115867\\
2.64	0.00820281534097175\\
2.65	0.00820283216728002\\
2.66	0.00820284900008622\\
2.67	0.00820286583939309\\
2.68	0.00820288268520335\\
2.69	0.00820289953751974\\
2.7	0.008202916396345\\
2.71	0.00820293326168187\\
2.72	0.0082029501335331\\
2.73	0.00820296701190142\\
2.74	0.00820298389678958\\
2.75	0.00820300078820033\\
2.76	0.00820301768613641\\
2.77	0.00820303459060056\\
2.78	0.00820305150159555\\
2.79	0.00820306841912411\\
2.8	0.008203085343189\\
2.81	0.00820310227379297\\
2.82	0.00820311921093879\\
2.83	0.0082031361546292\\
2.84	0.00820315310486695\\
2.85	0.00820317006165482\\
2.86	0.00820318702499556\\
2.87	0.00820320399489194\\
2.88	0.00820322097134671\\
2.89	0.00820323795436263\\
2.9	0.00820325494394249\\
2.91	0.00820327194008903\\
2.92	0.00820328894280504\\
2.93	0.00820330595209327\\
2.94	0.00820332296795651\\
2.95	0.00820333999039751\\
2.96	0.00820335701941907\\
2.97	0.00820337405502395\\
2.98	0.00820339109721492\\
2.99	0.00820340814599478\\
3	0.00820342520136629\\
3.01	0.00820344226333224\\
3.02	0.00820345933189541\\
3.03	0.00820347640705857\\
3.04	0.00820349348882453\\
3.05	0.00820351057719606\\
3.06	0.00820352767217595\\
3.07	0.00820354477376699\\
3.08	0.00820356188197197\\
3.09	0.00820357899679368\\
3.1	0.00820359611823492\\
3.11	0.00820361324629847\\
3.12	0.00820363038098715\\
3.13	0.00820364752230373\\
3.14	0.00820366467025103\\
3.15	0.00820368182483183\\
3.16	0.00820369898604895\\
3.17	0.00820371615390519\\
3.18	0.00820373332840334\\
3.19	0.00820375050954622\\
3.2	0.00820376769733664\\
3.21	0.00820378489177739\\
3.22	0.0082038020928713\\
3.23	0.00820381930062117\\
3.24	0.00820383651502981\\
3.25	0.00820385373610004\\
3.26	0.00820387096383467\\
3.27	0.00820388819823653\\
3.28	0.00820390543930842\\
3.29	0.00820392268705317\\
3.3	0.0082039399414736\\
3.31	0.00820395720257253\\
3.32	0.00820397447035279\\
3.33	0.0082039917448172\\
3.34	0.00820400902596858\\
3.35	0.00820402631380978\\
3.36	0.0082040436083436\\
3.37	0.00820406090957289\\
3.38	0.00820407821750048\\
3.39	0.0082040955321292\\
3.4	0.00820411285346189\\
3.41	0.00820413018150138\\
3.42	0.0082041475162505\\
3.43	0.00820416485771212\\
3.44	0.00820418220588904\\
3.45	0.00820419956078414\\
3.46	0.00820421692240024\\
3.47	0.00820423429074019\\
3.48	0.00820425166580683\\
3.49	0.00820426904760303\\
3.5	0.00820428643613161\\
3.51	0.00820430383139545\\
3.52	0.00820432123339737\\
3.53	0.00820433864214025\\
3.54	0.00820435605762694\\
3.55	0.00820437347986028\\
3.56	0.00820439090884315\\
3.57	0.00820440834457839\\
3.58	0.00820442578706888\\
3.59	0.00820444323631746\\
3.6	0.00820446069232701\\
3.61	0.00820447815510039\\
3.62	0.00820449562464046\\
3.63	0.0082045131009501\\
3.64	0.00820453058403217\\
3.65	0.00820454807388954\\
3.66	0.00820456557052509\\
3.67	0.00820458307394168\\
3.68	0.00820460058414221\\
3.69	0.00820461810112952\\
3.7	0.00820463562490652\\
3.71	0.00820465315547608\\
3.72	0.00820467069284107\\
3.73	0.00820468823700438\\
3.74	0.00820470578796889\\
3.75	0.0082047233457375\\
3.76	0.00820474091031307\\
3.77	0.00820475848169851\\
3.78	0.0082047760598967\\
3.79	0.00820479364491053\\
3.8	0.0082048112367429\\
3.81	0.00820482883539669\\
3.82	0.00820484644087481\\
3.83	0.00820486405318015\\
3.84	0.0082048816723156\\
3.85	0.00820489929828407\\
3.86	0.00820491693108846\\
3.87	0.00820493457073167\\
3.88	0.0082049522172166\\
3.89	0.00820496987054617\\
3.9	0.00820498753072327\\
3.91	0.00820500519775081\\
3.92	0.0082050228716317\\
3.93	0.00820504055236886\\
3.94	0.0082050582399652\\
3.95	0.00820507593442363\\
3.96	0.00820509363574707\\
3.97	0.00820511134393843\\
3.98	0.00820512905900063\\
3.99	0.00820514678093659\\
4	0.00820516450974924\\
4.01	0.00820518224544149\\
4.02	0.00820519998801627\\
4.03	0.00820521773747651\\
4.04	0.00820523549382513\\
4.05	0.00820525325706506\\
4.06	0.00820527102719923\\
4.07	0.00820528880423058\\
4.08	0.00820530658816204\\
4.09	0.00820532437899653\\
4.1	0.008205342176737\\
4.11	0.00820535998138638\\
4.12	0.00820537779294762\\
4.13	0.00820539561142365\\
4.14	0.00820541343681742\\
4.15	0.00820543126913186\\
4.16	0.00820544910836993\\
4.17	0.00820546695453457\\
4.18	0.00820548480762873\\
4.19	0.00820550266765535\\
4.2	0.00820552053461739\\
4.21	0.0082055384085178\\
4.22	0.00820555628935954\\
4.23	0.00820557417714556\\
4.24	0.00820559207187881\\
4.25	0.00820560997356226\\
4.26	0.00820562788219886\\
4.27	0.00820564579779157\\
4.28	0.00820566372034336\\
4.29	0.00820568164985719\\
4.3	0.00820569958633603\\
4.31	0.00820571752978285\\
4.32	0.00820573548020061\\
4.33	0.00820575343759228\\
4.34	0.00820577140196084\\
4.35	0.00820578937330925\\
4.36	0.0082058073516405\\
4.37	0.00820582533695756\\
4.38	0.0082058433292634\\
4.39	0.00820586132856101\\
4.4	0.00820587933485337\\
4.41	0.00820589734814345\\
4.42	0.00820591536843425\\
4.43	0.00820593339572875\\
4.44	0.00820595143002992\\
4.45	0.00820596947134077\\
4.46	0.00820598751966428\\
4.47	0.00820600557500345\\
4.48	0.00820602363736125\\
4.49	0.00820604170674071\\
4.5	0.00820605978314479\\
4.51	0.00820607786657651\\
4.52	0.00820609595703886\\
4.53	0.00820611405453485\\
4.54	0.00820613215906746\\
4.55	0.00820615027063972\\
4.56	0.00820616838925461\\
4.57	0.00820618651491516\\
4.58	0.00820620464762435\\
4.59	0.00820622278738522\\
4.6	0.00820624093420076\\
4.61	0.00820625908807399\\
4.62	0.00820627724900792\\
4.63	0.00820629541700557\\
4.64	0.00820631359206995\\
4.65	0.00820633177420409\\
4.66	0.008206349963411\\
4.67	0.0082063681596937\\
4.68	0.00820638636305523\\
4.69	0.0082064045734986\\
4.7	0.00820642279102683\\
4.71	0.00820644101564296\\
4.72	0.00820645924735002\\
4.73	0.00820647748615104\\
4.74	0.00820649573204904\\
4.75	0.00820651398504707\\
4.76	0.00820653224514815\\
4.77	0.00820655051235533\\
4.78	0.00820656878667164\\
4.79	0.00820658706810013\\
4.8	0.00820660535664383\\
4.81	0.00820662365230579\\
4.82	0.00820664195508905\\
4.83	0.00820666026499666\\
4.84	0.00820667858203167\\
4.85	0.00820669690619712\\
4.86	0.00820671523749607\\
4.87	0.00820673357593156\\
4.88	0.00820675192150666\\
4.89	0.00820677027422442\\
4.9	0.00820678863408789\\
4.91	0.00820680700110014\\
4.92	0.00820682537526421\\
4.93	0.00820684375658319\\
4.94	0.00820686214506012\\
4.95	0.00820688054069807\\
4.96	0.00820689894350012\\
4.97	0.00820691735346933\\
4.98	0.00820693577060876\\
4.99	0.00820695419492148\\
5	0.00820697262641058\\
5.01	0.00820699106507912\\
5.02	0.00820700951093018\\
5.03	0.00820702796396684\\
5.04	0.00820704642419218\\
5.05	0.00820706489160927\\
5.06	0.0082070833662212\\
5.07	0.00820710184803105\\
5.08	0.0082071203370419\\
5.09	0.00820713883325686\\
5.1	0.00820715733667899\\
5.11	0.0082071758473114\\
5.12	0.00820719436515716\\
5.13	0.00820721289021939\\
5.14	0.00820723142250116\\
5.15	0.00820724996200558\\
5.16	0.00820726850873574\\
5.17	0.00820728706269475\\
5.18	0.0082073056238857\\
5.19	0.00820732419231169\\
5.2	0.00820734276797584\\
5.21	0.00820736135088125\\
5.22	0.00820737994103102\\
5.23	0.00820739853842825\\
5.24	0.00820741714307607\\
5.25	0.00820743575497759\\
5.26	0.00820745437413591\\
5.27	0.00820747300055415\\
5.28	0.00820749163423544\\
5.29	0.00820751027518288\\
5.3	0.0082075289233996\\
5.31	0.00820754757888872\\
5.32	0.00820756624165336\\
5.33	0.00820758491169665\\
5.34	0.0082076035890217\\
5.35	0.00820762227363166\\
5.36	0.00820764096552965\\
5.37	0.00820765966471879\\
5.38	0.00820767837120224\\
5.39	0.00820769708498311\\
5.4	0.00820771580606454\\
5.41	0.00820773453444967\\
5.42	0.00820775327014164\\
5.43	0.0082077720131436\\
5.44	0.00820779076345867\\
5.45	0.00820780952109001\\
5.46	0.00820782828604076\\
5.47	0.00820784705831408\\
5.48	0.00820786583791309\\
5.49	0.00820788462484097\\
5.5	0.00820790341910086\\
5.51	0.00820792222069591\\
5.52	0.00820794102962928\\
5.53	0.00820795984590412\\
5.54	0.0082079786695236\\
5.55	0.00820799750049087\\
5.56	0.00820801633880909\\
5.57	0.00820803518448143\\
5.58	0.00820805403751106\\
5.59	0.00820807289790113\\
5.6	0.00820809176565482\\
5.61	0.0082081106407753\\
5.62	0.00820812952326573\\
5.63	0.0082081484131293\\
5.64	0.00820816731036917\\
5.65	0.00820818621498852\\
5.66	0.00820820512699054\\
5.67	0.00820822404637839\\
5.68	0.00820824297315526\\
5.69	0.00820826190732434\\
5.7	0.0082082808488888\\
5.71	0.00820829979785183\\
5.72	0.00820831875421663\\
5.73	0.00820833771798637\\
5.74	0.00820835668916426\\
5.75	0.00820837566775347\\
5.76	0.00820839465375722\\
5.77	0.00820841364717869\\
5.78	0.00820843264802108\\
5.79	0.00820845165628759\\
5.8	0.00820847067198142\\
5.81	0.00820848969510577\\
5.82	0.00820850872566385\\
5.83	0.00820852776365886\\
5.84	0.00820854680909401\\
5.85	0.00820856586197251\\
5.86	0.00820858492229756\\
5.87	0.00820860399007239\\
5.88	0.00820862306530021\\
5.89	0.00820864214798422\\
5.9	0.00820866123812766\\
5.91	0.00820868033573372\\
5.92	0.00820869944080565\\
5.93	0.00820871855334666\\
5.94	0.00820873767335997\\
5.95	0.00820875680084881\\
5.96	0.0082087759358164\\
5.97	0.00820879507826598\\
5.98	0.00820881422820077\\
5.99	0.00820883338562401\\
6	0.00820885255053894\\
6.01	0.00820887172294877\\
6.02	0.00820889090285676\\
6.03	0.00820891009026615\\
6.04	0.00820892928518017\\
6.05	0.00820894848760206\\
6.06	0.00820896769753507\\
6.07	0.00820898691498245\\
6.08	0.00820900613994743\\
6.09	0.00820902537243328\\
6.1	0.00820904461244324\\
6.11	0.00820906385998056\\
6.12	0.0082090831150485\\
6.13	0.00820910237765031\\
6.14	0.00820912164778925\\
6.15	0.00820914092546858\\
6.16	0.00820916021069156\\
6.17	0.00820917950346144\\
6.18	0.00820919880378151\\
6.19	0.00820921811165501\\
6.2	0.00820923742708523\\
6.21	0.00820925675007542\\
6.22	0.00820927608062886\\
6.23	0.00820929541874882\\
6.24	0.00820931476443857\\
6.25	0.00820933411770139\\
6.26	0.00820935347854056\\
6.27	0.00820937284695936\\
6.28	0.00820939222296107\\
6.29	0.00820941160654896\\
6.3	0.00820943099772633\\
6.31	0.00820945039649646\\
6.32	0.00820946980286264\\
6.33	0.00820948921682815\\
6.34	0.0082095086383963\\
6.35	0.00820952806757036\\
6.36	0.00820954750435364\\
6.37	0.00820956694874943\\
6.38	0.00820958640076104\\
6.39	0.00820960586039175\\
6.4	0.00820962532764486\\
6.41	0.0082096448025237\\
6.42	0.00820966428503155\\
6.43	0.00820968377517172\\
6.44	0.00820970327294753\\
6.45	0.00820972277836228\\
6.46	0.00820974229141929\\
6.47	0.00820976181212186\\
6.48	0.00820978134047331\\
6.49	0.00820980087647696\\
6.5	0.00820982042013613\\
6.51	0.00820983997145414\\
6.52	0.00820985953043431\\
6.53	0.00820987909707995\\
6.54	0.00820989867139441\\
6.55	0.008209918253381\\
6.56	0.00820993784304306\\
6.57	0.00820995744038391\\
6.58	0.00820997704540689\\
6.59	0.00820999665811533\\
6.6	0.00821001627851256\\
6.61	0.00821003590660192\\
6.62	0.00821005554238677\\
6.63	0.00821007518587042\\
6.64	0.00821009483705623\\
6.65	0.00821011449594754\\
6.66	0.0082101341625477\\
6.67	0.00821015383686005\\
6.68	0.00821017351888794\\
6.69	0.00821019320863474\\
6.7	0.00821021290610377\\
6.71	0.00821023261129842\\
6.72	0.00821025232422202\\
6.73	0.00821027204487794\\
6.74	0.00821029177326954\\
6.75	0.00821031150940017\\
6.76	0.00821033125327322\\
6.77	0.00821035100489202\\
6.78	0.00821037076425997\\
6.79	0.00821039053138042\\
6.8	0.00821041030625674\\
6.81	0.00821043008889232\\
6.82	0.00821044987929051\\
6.83	0.0082104696774547\\
6.84	0.00821048948338826\\
6.85	0.00821050929709458\\
6.86	0.00821052911857703\\
6.87	0.00821054894783901\\
6.88	0.00821056878488388\\
6.89	0.00821058862971504\\
6.9	0.00821060848233588\\
6.91	0.00821062834274979\\
6.92	0.00821064821096015\\
6.93	0.00821066808697037\\
6.94	0.00821068797078383\\
6.95	0.00821070786240393\\
6.96	0.00821072776183408\\
6.97	0.00821074766907768\\
6.98	0.00821076758413812\\
6.99	0.0082107875070188\\
7	0.00821080743772314\\
7.01	0.00821082737625455\\
7.02	0.00821084732261643\\
7.03	0.00821086727681219\\
7.04	0.00821088723884525\\
7.05	0.00821090720871902\\
7.06	0.00821092718643692\\
7.07	0.00821094717200236\\
7.08	0.00821096716541877\\
7.09	0.00821098716668957\\
7.1	0.00821100717581818\\
7.11	0.00821102719280802\\
7.12	0.00821104721766253\\
7.13	0.00821106725038514\\
7.14	0.00821108729097927\\
7.15	0.00821110733944835\\
7.16	0.00821112739579583\\
7.17	0.00821114746002513\\
7.18	0.0082111675321397\\
7.19	0.00821118761214298\\
7.2	0.0082112077000384\\
7.21	0.00821122779582941\\
7.22	0.00821124789951946\\
7.23	0.00821126801111199\\
7.24	0.00821128813061045\\
7.25	0.00821130825801829\\
7.26	0.00821132839333897\\
7.27	0.00821134853657594\\
7.28	0.00821136868773265\\
7.29	0.00821138884681256\\
7.3	0.00821140901381914\\
7.31	0.00821142918875584\\
7.32	0.00821144937162613\\
7.33	0.00821146956243346\\
7.34	0.00821148976118132\\
7.35	0.00821150996787316\\
7.36	0.00821153018251247\\
7.37	0.0082115504051027\\
7.38	0.00821157063564735\\
7.39	0.00821159087414986\\
7.4	0.00821161112061374\\
7.41	0.00821163137504246\\
7.42	0.0082116516374395\\
7.43	0.00821167190780834\\
7.44	0.00821169218615247\\
7.45	0.00821171247247537\\
7.46	0.00821173276678054\\
7.47	0.00821175306907146\\
7.48	0.00821177337935163\\
7.49	0.00821179369762454\\
7.5	0.00821181402389369\\
7.51	0.00821183435816258\\
7.52	0.0082118547004347\\
7.53	0.00821187505071356\\
7.54	0.00821189540900266\\
7.55	0.0082119157753055\\
7.56	0.0082119361496256\\
7.57	0.00821195653196646\\
7.58	0.00821197692233159\\
7.59	0.00821199732072451\\
7.6	0.00821201772714873\\
7.61	0.00821203814160776\\
7.62	0.00821205856410513\\
7.63	0.00821207899464436\\
7.64	0.00821209943322896\\
7.65	0.00821211987986246\\
7.66	0.00821214033454839\\
7.67	0.00821216079729027\\
7.68	0.00821218126809164\\
7.69	0.00821220174695602\\
7.7	0.00821222223388695\\
7.71	0.00821224272888796\\
7.72	0.00821226323196259\\
7.73	0.00821228374311438\\
7.74	0.00821230426234687\\
7.75	0.0082123247896636\\
7.76	0.00821234532506811\\
7.77	0.00821236586856395\\
7.78	0.00821238642015468\\
7.79	0.00821240697984383\\
7.8	0.00821242754763497\\
7.81	0.00821244812353164\\
7.82	0.0082124687075374\\
7.83	0.0082124892996558\\
7.84	0.00821250989989041\\
7.85	0.00821253050824479\\
7.86	0.0082125511247225\\
7.87	0.00821257174932711\\
7.88	0.00821259238206218\\
7.89	0.00821261302293128\\
7.9	0.00821263367193798\\
7.91	0.00821265432908585\\
7.92	0.00821267499437848\\
7.93	0.00821269566781943\\
7.94	0.00821271634941228\\
7.95	0.00821273703916061\\
7.96	0.008212757737068\\
7.97	0.00821277844313805\\
7.98	0.00821279915737433\\
7.99	0.00821281987978042\\
8	0.00821284061035993\\
8.01	0.00821286134911644\\
8.02	0.00821288209605355\\
8.03	0.00821290285117484\\
8.04	0.00821292361448392\\
8.05	0.00821294438598439\\
8.06	0.00821296516567984\\
8.07	0.00821298595357389\\
8.08	0.00821300674967012\\
8.09	0.00821302755397216\\
8.1	0.0082130483664836\\
8.11	0.00821306918720807\\
8.12	0.00821309001614916\\
8.13	0.0082131108533105\\
8.14	0.0082131316986957\\
8.15	0.00821315255230838\\
8.16	0.00821317341415217\\
8.17	0.00821319428423067\\
8.18	0.00821321516254751\\
8.19	0.00821323604910633\\
8.2	0.00821325694391075\\
8.21	0.00821327784696438\\
8.22	0.00821329875827089\\
8.23	0.00821331967783388\\
8.24	0.008213340605657\\
8.25	0.00821336154174388\\
8.26	0.00821338248609816\\
8.27	0.00821340343872349\\
8.28	0.00821342439962351\\
8.29	0.00821344536880186\\
8.3	0.00821346634626218\\
8.31	0.00821348733200814\\
8.32	0.00821350832604337\\
8.33	0.00821352932837154\\
8.34	0.00821355033899629\\
8.35	0.00821357135792128\\
8.36	0.00821359238515017\\
8.37	0.00821361342068662\\
8.38	0.00821363446453429\\
8.39	0.00821365551669685\\
8.4	0.00821367657717796\\
8.41	0.0082136976459813\\
8.42	0.00821371872311052\\
8.43	0.00821373980856931\\
8.44	0.00821376090236133\\
8.45	0.00821378200449027\\
8.46	0.00821380311495979\\
8.47	0.00821382423377359\\
8.48	0.00821384536093534\\
8.49	0.00821386649644872\\
8.5	0.00821388764031742\\
8.51	0.00821390879254514\\
8.52	0.00821392995313555\\
8.53	0.00821395112209235\\
8.54	0.00821397229941924\\
8.55	0.0082139934851199\\
8.56	0.00821401467919804\\
8.57	0.00821403588165735\\
8.58	0.00821405709250154\\
8.59	0.00821407831173431\\
8.6	0.00821409953935936\\
8.61	0.00821412077538041\\
8.62	0.00821414201980115\\
8.63	0.00821416327262531\\
8.64	0.0082141845338566\\
8.65	0.00821420580349872\\
8.66	0.0082142270815554\\
8.67	0.00821424836803036\\
8.68	0.00821426966292731\\
8.69	0.00821429096624998\\
8.7	0.0082143122780021\\
8.71	0.00821433359818739\\
8.72	0.00821435492680958\\
8.73	0.0082143762638724\\
8.74	0.00821439760937959\\
8.75	0.00821441896333487\\
8.76	0.00821444032574199\\
8.77	0.00821446169660468\\
8.78	0.00821448307592669\\
8.79	0.00821450446371175\\
8.8	0.00821452585996362\\
8.81	0.00821454726468604\\
8.82	0.00821456867788275\\
8.83	0.00821459009955751\\
8.84	0.00821461152971408\\
8.85	0.0082146329683562\\
8.86	0.00821465441548763\\
8.87	0.00821467587111213\\
8.88	0.00821469733523347\\
8.89	0.00821471880785539\\
8.9	0.00821474028898168\\
8.91	0.00821476177861609\\
8.92	0.0082147832767624\\
8.93	0.00821480478342437\\
8.94	0.00821482629860578\\
8.95	0.00821484782231039\\
8.96	0.008214869354542\\
8.97	0.00821489089530437\\
8.98	0.00821491244460129\\
8.99	0.00821493400243654\\
9	0.0082149555688139\\
9.01	0.00821497714373716\\
9.02	0.00821499872721012\\
9.03	0.00821502031923655\\
9.04	0.00821504191982026\\
9.05	0.00821506352896504\\
9.06	0.00821508514667467\\
9.07	0.00821510677295297\\
9.08	0.00821512840780374\\
9.09	0.00821515005123077\\
9.1	0.00821517170323787\\
9.11	0.00821519336382884\\
9.12	0.00821521503300751\\
9.13	0.00821523671077767\\
9.14	0.00821525839714314\\
9.15	0.00821528009210773\\
9.16	0.00821530179567527\\
9.17	0.00821532350784957\\
9.18	0.00821534522863444\\
9.19	0.00821536695803373\\
9.2	0.00821538869605123\\
9.21	0.00821541044269079\\
9.22	0.00821543219795624\\
9.23	0.0082154539618514\\
9.24	0.00821547573438011\\
9.25	0.00821549751554621\\
9.26	0.00821551930535352\\
9.27	0.00821554110380589\\
9.28	0.00821556291090716\\
9.29	0.00821558472666118\\
9.3	0.00821560655107179\\
9.31	0.00821562838414282\\
9.32	0.00821565022587815\\
9.33	0.00821567207628161\\
9.34	0.00821569393535706\\
9.35	0.00821571580310835\\
9.36	0.00821573767953935\\
9.37	0.0082157595646539\\
9.38	0.00821578145845588\\
9.39	0.00821580336094914\\
9.4	0.00821582527213756\\
9.41	0.00821584719202499\\
9.42	0.00821586912061531\\
9.43	0.00821589105791239\\
9.44	0.00821591300392011\\
9.45	0.00821593495864233\\
9.46	0.00821595692208293\\
9.47	0.00821597889424581\\
9.48	0.00821600087513483\\
9.49	0.00821602286475388\\
9.5	0.00821604486310686\\
9.51	0.00821606687019763\\
9.52	0.0082160888860301\\
9.53	0.00821611091060816\\
9.54	0.0082161329439357\\
9.55	0.00821615498601661\\
9.56	0.00821617703685481\\
9.57	0.00821619909645417\\
9.58	0.00821622116481861\\
9.59	0.00821624324195203\\
9.6	0.00821626532785835\\
9.61	0.00821628742254145\\
9.62	0.00821630952600526\\
9.63	0.00821633163825369\\
9.64	0.00821635375929065\\
9.65	0.00821637588912006\\
9.66	0.00821639802774585\\
9.67	0.00821642017517191\\
9.68	0.00821644233140218\\
9.69	0.00821646449644059\\
9.7	0.00821648667029107\\
9.71	0.00821650885295753\\
9.72	0.00821653104444391\\
9.73	0.00821655324475415\\
9.74	0.00821657545389217\\
9.75	0.00821659767186192\\
9.76	0.00821661989866733\\
9.77	0.00821664213431235\\
9.78	0.00821666437880092\\
9.79	0.00821668663213698\\
9.8	0.00821670889432447\\
9.81	0.00821673116536736\\
9.82	0.00821675344526958\\
9.83	0.0082167757340351\\
9.84	0.00821679803166786\\
9.85	0.00821682033817183\\
9.86	0.00821684265355096\\
9.87	0.00821686497780922\\
9.88	0.00821688731095057\\
9.89	0.00821690965297897\\
9.9	0.00821693200389839\\
9.91	0.0082169543637128\\
9.92	0.00821697673242618\\
9.93	0.00821699911004249\\
9.94	0.00821702149656571\\
9.95	0.00821704389199982\\
9.96	0.00821706629634879\\
9.97	0.00821708870961662\\
9.98	0.00821711113180728\\
9.99	0.00821713356292475\\
10	0.00821715600297304\\
10.01	0.00821717845195611\\
10.02	0.00821720090987798\\
10.03	0.00821722337674262\\
10.04	0.00821724585255405\\
10.05	0.00821726833731624\\
10.06	0.00821729083103322\\
10.07	0.00821731333370897\\
10.08	0.00821733584534749\\
10.09	0.0082173583659528\\
10.1	0.0082173808955289\\
10.11	0.00821740343407981\\
10.12	0.00821742598160953\\
10.13	0.00821744853812208\\
10.14	0.00821747110362148\\
10.15	0.00821749367811173\\
10.16	0.00821751626159687\\
10.17	0.0082175388540809\\
10.18	0.00821756145556787\\
10.19	0.00821758406606179\\
10.2	0.00821760668556669\\
10.21	0.0082176293140866\\
10.22	0.00821765195162555\\
10.23	0.00821767459818758\\
10.24	0.00821769725377672\\
10.25	0.008217719918397\\
10.26	0.00821774259205247\\
10.27	0.00821776527474718\\
10.28	0.00821778796648515\\
10.29	0.00821781066727045\\
10.3	0.00821783337710712\\
10.31	0.00821785609599919\\
10.32	0.00821787882395074\\
10.33	0.0082179015609658\\
10.34	0.00821792430704845\\
10.35	0.00821794706220272\\
10.36	0.00821796982643269\\
10.37	0.00821799259974241\\
10.38	0.00821801538213594\\
10.39	0.00821803817361736\\
10.4	0.00821806097419073\\
10.41	0.00821808378386011\\
10.42	0.00821810660262958\\
10.43	0.00821812943050321\\
10.44	0.00821815226748507\\
10.45	0.00821817511357924\\
10.46	0.00821819796878981\\
10.47	0.00821822083312084\\
10.48	0.00821824370657643\\
10.49	0.00821826658916065\\
10.5	0.00821828948087759\\
10.51	0.00821831238173134\\
10.52	0.00821833529172599\\
10.53	0.00821835821086563\\
10.54	0.00821838113915436\\
10.55	0.00821840407659626\\
10.56	0.00821842702319544\\
10.57	0.00821844997895599\\
10.58	0.00821847294388203\\
10.59	0.00821849591797764\\
10.6	0.00821851890124693\\
10.61	0.00821854189369401\\
10.62	0.00821856489532299\\
10.63	0.00821858790613797\\
10.64	0.00821861092614308\\
10.65	0.00821863395534241\\
10.66	0.0082186569937401\\
10.67	0.00821868004134024\\
10.68	0.00821870309814697\\
10.69	0.00821872616416441\\
10.7	0.00821874923939666\\
10.71	0.00821877232384787\\
10.72	0.00821879541752215\\
10.73	0.00821881852042364\\
10.74	0.00821884163255646\\
10.75	0.00821886475392474\\
10.76	0.00821888788453262\\
10.77	0.00821891102438422\\
10.78	0.00821893417348369\\
10.79	0.00821895733183517\\
10.8	0.00821898049944279\\
10.81	0.00821900367631069\\
10.82	0.00821902686244302\\
10.83	0.00821905005784392\\
10.84	0.00821907326251753\\
10.85	0.00821909647646801\\
10.86	0.00821911969969951\\
10.87	0.00821914293221616\\
10.88	0.00821916617402213\\
10.89	0.00821918942512157\\
10.9	0.00821921268551864\\
10.91	0.00821923595521749\\
10.92	0.00821925923422228\\
10.93	0.00821928252253717\\
10.94	0.00821930582016633\\
10.95	0.00821932912711391\\
10.96	0.00821935244338407\\
10.97	0.008219375768981\\
10.98	0.00821939910390884\\
10.99	0.00821942244817178\\
11	0.00821944580177398\\
11.01	0.00821946916471961\\
11.02	0.00821949253701286\\
11.03	0.00821951591865788\\
11.04	0.00821953930965885\\
11.05	0.00821956271001997\\
11.06	0.00821958611974539\\
11.07	0.00821960953883931\\
11.08	0.0082196329673059\\
11.09	0.00821965640514935\\
11.1	0.00821967985237384\\
11.11	0.00821970330898355\\
11.12	0.00821972677498267\\
11.13	0.00821975025037539\\
11.14	0.0082197737351659\\
11.15	0.00821979722935838\\
11.16	0.00821982073295703\\
11.17	0.00821984424596604\\
11.18	0.0082198677683896\\
11.19	0.00821989130023191\\
11.2	0.00821991484149716\\
11.21	0.00821993839218954\\
11.22	0.00821996195231327\\
11.23	0.00821998552187252\\
11.24	0.00822000910087151\\
11.25	0.00822003268931443\\
11.26	0.00822005628720548\\
11.27	0.00822007989454887\\
11.28	0.0082201035113488\\
11.29	0.00822012713760947\\
11.3	0.00822015077333509\\
11.31	0.00822017441852986\\
11.32	0.00822019807319799\\
11.33	0.00822022173734369\\
11.34	0.00822024541097115\\
11.35	0.00822026909408461\\
11.36	0.00822029278668825\\
11.37	0.00822031648878629\\
11.38	0.00822034020038294\\
11.39	0.00822036392148242\\
11.4	0.00822038765208892\\
11.41	0.00822041139220668\\
11.42	0.00822043514183988\\
11.43	0.00822045890099276\\
11.44	0.00822048266966953\\
11.45	0.00822050644787438\\
11.46	0.00822053023561155\\
11.47	0.00822055403288525\\
11.48	0.00822057783969968\\
11.49	0.00822060165605907\\
11.5	0.00822062548196763\\
11.51	0.00822064931742958\\
11.52	0.00822067316244912\\
11.53	0.00822069701703048\\
11.54	0.00822072088117786\\
11.55	0.00822074475489549\\
11.56	0.00822076863818759\\
11.57	0.00822079253105836\\
11.58	0.00822081643351202\\
11.59	0.00822084034555278\\
11.6	0.00822086426718487\\
11.61	0.0082208881984125\\
11.62	0.00822091213923987\\
11.63	0.0082209360896712\\
11.64	0.00822096004971072\\
11.65	0.00822098401936261\\
11.66	0.00822100799863112\\
11.67	0.00822103198752044\\
11.68	0.00822105598603477\\
11.69	0.00822107999417835\\
11.7	0.00822110401195537\\
11.71	0.00822112803937005\\
11.72	0.00822115207642659\\
11.73	0.00822117612312919\\
11.74	0.00822120017948208\\
11.75	0.00822122424548944\\
11.76	0.00822124832115549\\
11.77	0.00822127240648443\\
11.78	0.00822129650148046\\
11.79	0.00822132060614779\\
11.8	0.0082213447204906\\
11.81	0.00822136884451311\\
11.82	0.0082213929782195\\
11.83	0.00822141712161397\\
11.84	0.00822144127470072\\
11.85	0.00822146543748393\\
11.86	0.0082214896099678\\
11.87	0.00822151379215652\\
11.88	0.00822153798405427\\
11.89	0.00822156218566523\\
11.9	0.0082215863969936\\
11.91	0.00822161061804355\\
11.92	0.00822163484881925\\
11.93	0.00822165908932489\\
11.94	0.00822168333956463\\
11.95	0.00822170759954266\\
11.96	0.00822173186926314\\
11.97	0.00822175614873023\\
11.98	0.00822178043794809\\
11.99	0.0082218047369209\\
12	0.0082218290456528\\
12.01	0.00822185336414796\\
12.02	0.00822187769241051\\
12.03	0.00822190203044463\\
12.04	0.00822192637825443\\
12.05	0.00822195073584408\\
12.06	0.00822197510321771\\
12.07	0.00822199948037945\\
12.08	0.00822202386733343\\
12.09	0.0082220482640838\\
12.1	0.00822207267063465\\
12.11	0.00822209708699013\\
12.12	0.00822212151315434\\
12.13	0.0082221459491314\\
12.14	0.00822217039492541\\
12.15	0.00822219485054047\\
12.16	0.00822221931598069\\
12.17	0.00822224379125015\\
12.18	0.00822226827635296\\
12.19	0.00822229277129319\\
12.2	0.00822231727607491\\
12.21	0.00822234179070221\\
12.22	0.00822236631517915\\
12.23	0.00822239084950981\\
12.24	0.00822241539369822\\
12.25	0.00822243994774845\\
12.26	0.00822246451166454\\
12.27	0.00822248908545054\\
12.28	0.00822251366911047\\
12.29	0.00822253826264837\\
12.3	0.00822256286606825\\
12.31	0.00822258747937414\\
12.32	0.00822261210257003\\
12.33	0.00822263673565994\\
12.34	0.00822266137864785\\
12.35	0.00822268603153775\\
12.36	0.00822271069433363\\
12.37	0.00822273536703944\\
12.38	0.00822276004965917\\
12.39	0.00822278474219676\\
12.4	0.00822280944465617\\
12.41	0.00822283415704133\\
12.42	0.00822285887935617\\
12.43	0.00822288361160463\\
12.44	0.00822290835379061\\
12.45	0.00822293310591802\\
12.46	0.00822295786799076\\
12.47	0.00822298264001272\\
12.48	0.00822300742198777\\
12.49	0.00822303221391978\\
12.5	0.00822305701581262\\
12.51	0.00822308182767011\\
12.52	0.00822310664949612\\
12.53	0.00822313148129447\\
12.54	0.00822315632306896\\
12.55	0.00822318117482341\\
12.56	0.00822320603656161\\
12.57	0.00822323090828735\\
12.58	0.0082232557900044\\
12.59	0.00822328068171651\\
12.6	0.00822330558342744\\
12.61	0.00822333049514092\\
12.62	0.00822335541686067\\
12.63	0.0082233803485904\\
12.64	0.00822340529033382\\
12.65	0.00822343024209461\\
12.66	0.00822345520387643\\
12.67	0.00822348017568295\\
12.68	0.00822350515751779\\
12.69	0.00822353014938461\\
12.7	0.00822355515128701\\
12.71	0.00822358016322858\\
12.72	0.00822360518521292\\
12.73	0.00822363021724359\\
12.74	0.00822365525932415\\
12.75	0.00822368031145813\\
12.76	0.00822370537364906\\
12.77	0.00822373044590043\\
12.78	0.00822375552821575\\
12.79	0.00822378062059847\\
12.8	0.00822380572305207\\
12.81	0.00822383083557996\\
12.82	0.00822385595818557\\
12.83	0.0082238810908723\\
12.84	0.00822390623364353\\
12.85	0.00822393138650262\\
12.86	0.00822395654945293\\
12.87	0.00822398172249776\\
12.88	0.00822400690564044\\
12.89	0.00822403209888424\\
12.9	0.00822405730223242\\
12.91	0.00822408251568824\\
12.92	0.00822410773925491\\
12.93	0.00822413297293564\\
12.94	0.0082241582167336\\
12.95	0.00822418347065195\\
12.96	0.00822420873469382\\
12.97	0.00822423400886234\\
12.98	0.00822425929316058\\
12.99	0.00822428458759162\\
13	0.0082243098921585\\
13.01	0.00822433520686423\\
13.02	0.0082243605317118\\
13.03	0.0082243858667042\\
13.04	0.00822441121184435\\
13.05	0.00822443656713517\\
13.06	0.00822446193257957\\
13.07	0.0082244873081804\\
13.08	0.00822451269394051\\
13.09	0.0082245380898627\\
13.1	0.00822456349594975\\
13.11	0.00822458891220443\\
13.12	0.00822461433862947\\
13.13	0.00822463977522756\\
13.14	0.00822466522200137\\
13.15	0.00822469067895356\\
13.16	0.00822471614608672\\
13.17	0.00822474162340344\\
13.18	0.00822476711090627\\
13.19	0.00822479260859774\\
13.2	0.00822481811648033\\
13.21	0.00822484363455651\\
13.22	0.00822486916282869\\
13.23	0.00822489470129927\\
13.24	0.00822492024997062\\
13.25	0.00822494580884505\\
13.26	0.00822497137792488\\
13.27	0.00822499695721235\\
13.28	0.00822502254670969\\
13.29	0.00822504814641911\\
13.3	0.00822507375634275\\
13.31	0.00822509937648274\\
13.32	0.00822512500684116\\
13.33	0.00822515064742007\\
13.34	0.00822517629822149\\
13.35	0.00822520195924738\\
13.36	0.0082252276304997\\
13.37	0.00822525331198034\\
13.38	0.00822527900369118\\
13.39	0.00822530470563404\\
13.4	0.00822533041781072\\
13.41	0.00822535614022296\\
13.42	0.00822538187287248\\
13.43	0.00822540761576096\\
13.44	0.00822543336889003\\
13.45	0.00822545913226128\\
13.46	0.00822548490587629\\
13.47	0.00822551068973655\\
13.48	0.00822553648384355\\
13.49	0.00822556228819871\\
13.5	0.00822558810280345\\
13.51	0.00822561392765911\\
13.52	0.008225639762767\\
13.53	0.0082256656081284\\
13.54	0.00822569146374454\\
13.55	0.0082257173296166\\
13.56	0.00822574320574574\\
13.57	0.00822576909213305\\
13.58	0.00822579498877961\\
13.59	0.00822582089568644\\
13.6	0.00822584681285452\\
13.61	0.00822587274028477\\
13.62	0.00822589867797812\\
13.63	0.0082259246259354\\
13.64	0.00822595058415744\\
13.65	0.008225976552645\\
13.66	0.00822600253139882\\
13.67	0.00822602852041958\\
13.68	0.00822605451970794\\
13.69	0.0082260805292645\\
13.7	0.00822610654908983\\
13.71	0.00822613257918445\\
13.72	0.00822615861954885\\
13.73	0.00822618467018348\\
13.74	0.00822621073108874\\
13.75	0.00822623680226501\\
13.76	0.00822626288371261\\
13.77	0.00822628897543182\\
13.78	0.00822631507742292\\
13.79	0.00822634118968611\\
13.8	0.00822636731222158\\
13.81	0.00822639344502946\\
13.82	0.00822641958810988\\
13.83	0.0082264457414629\\
13.84	0.00822647190508858\\
13.85	0.00822649807898691\\
13.86	0.00822652426315789\\
13.87	0.00822655045760146\\
13.88	0.00822657666231754\\
13.89	0.00822660287730602\\
13.9	0.00822662910256677\\
13.91	0.00822665533809962\\
13.92	0.00822668158390439\\
13.93	0.00822670783998087\\
13.94	0.00822673410632884\\
13.95	0.00822676038294804\\
13.96	0.00822678666983819\\
13.97	0.00822681296699902\\
13.98	0.00822683927443024\\
13.99	0.00822686559213151\\
14	0.00822689192010252\\
14.01	0.00822691825834294\\
14.02	0.00822694460685243\\
14.03	0.00822697096563062\\
14.04	0.00822699733467719\\
14.05	0.00822702371399178\\
14.06	0.00822705010357405\\
14.07	0.00822707650342365\\
14.08	0.00822710291354025\\
14.09	0.00822712933392354\\
14.1	0.00822715576457318\\
14.11	0.00822718220548891\\
14.12	0.00822720865667044\\
14.13	0.00822723511811752\\
14.14	0.00822726158982993\\
14.15	0.00822728807180746\\
14.16	0.00822731456404997\\
14.17	0.00822734106655732\\
14.18	0.00822736757932943\\
14.19	0.00822739410236627\\
14.2	0.00822742063566784\\
14.21	0.0082274471792342\\
14.22	0.00822747373306548\\
14.23	0.00822750029716185\\
14.24	0.00822752687152357\\
14.25	0.00822755345615096\\
14.26	0.00822758005104441\\
14.27	0.00822760665620441\\
14.28	0.00822763327163152\\
14.29	0.00822765989732639\\
14.3	0.00822768653328978\\
14.31	0.00822771317952256\\
14.32	0.00822773983602569\\
14.33	0.00822776650280026\\
14.34	0.00822779317984745\\
14.35	0.00822781986716862\\
14.36	0.00822784656476522\\
14.37	0.00822787327263886\\
14.38	0.00822789999079129\\
14.39	0.00822792671922441\\
14.4	0.00822795345794031\\
14.41	0.0082279802069412\\
14.42	0.00822800696622952\\
14.43	0.00822803373580783\\
14.44	0.00822806051567893\\
14.45	0.0082280873058458\\
14.46	0.00822811410631163\\
14.47	0.00822814091707981\\
14.48	0.00822816773815396\\
14.49	0.00822819456953794\\
14.5	0.00822822141123584\\
14.51	0.00822824826325197\\
14.52	0.00822827512559094\\
14.53	0.0082283019982576\\
14.54	0.00822832888125705\\
14.55	0.0082283557745947\\
14.56	0.00822838267827625\\
14.57	0.00822840959230768\\
14.58	0.00822843651669527\\
14.59	0.00822846345144563\\
14.6	0.00822849039656568\\
14.61	0.00822851735206268\\
14.62	0.00822854431794423\\
14.63	0.00822857129421827\\
14.64	0.00822859828089309\\
14.65	0.00822862527797737\\
14.66	0.00822865228548013\\
14.67	0.00822867930341078\\
14.68	0.00822870633177912\\
14.69	0.00822873337059535\\
14.7	0.00822876041987007\\
14.71	0.00822878747961427\\
14.72	0.00822881454983936\\
14.73	0.00822884163055719\\
14.74	0.00822886872178001\\
14.75	0.00822889582352051\\
14.76	0.00822892293579181\\
14.77	0.00822895005860748\\
14.78	0.0082289771919815\\
14.79	0.00822900433592832\\
14.8	0.00822903149046283\\
14.81	0.00822905865560036\\
14.82	0.00822908583135669\\
14.83	0.00822911301774801\\
14.84	0.008229140214791\\
14.85	0.00822916742250275\\
14.86	0.00822919464090077\\
14.87	0.008229221870003\\
14.88	0.00822924910982782\\
14.89	0.00822927636039398\\
14.9	0.00822930362172064\\
14.91	0.00822933089382735\\
14.92	0.00822935817673402\\
14.93	0.0082293854704609\\
14.94	0.00822941277502858\\
14.95	0.00822944009045794\\
14.96	0.00822946741677015\\
14.97	0.00822949475398665\\
14.98	0.00822952210212907\\
14.99	0.00822954946121925\\
15	0.00822957683127917\\
15.01	0.00822960421233094\\
15.02	0.00822963160439672\\
15.03	0.00822965900749867\\
15.04	0.00822968642165895\\
15.05	0.00822971384689961\\
15.06	0.00822974128324252\\
15.07	0.00822976873070937\\
15.08	0.0082297961893215\\
15.09	0.00822982365909993\\
15.1	0.00822985114006517\\
15.11	0.0082298786322372\\
15.12	0.00822990613563533\\
15.13	0.00822993365027812\\
15.14	0.00822996117618327\\
15.15	0.00822998871336746\\
15.16	0.00823001626184626\\
15.17	0.00823004382163397\\
15.18	0.00823007139274346\\
15.19	0.00823009897518604\\
15.2	0.00823012656897128\\
15.21	0.0082301541741068\\
15.22	0.00823018179059811\\
15.23	0.00823020941845053\\
15.24	0.00823023705766938\\
15.25	0.00823026470825996\\
15.26	0.0082302923702276\\
15.27	0.00823032004357763\\
15.28	0.00823034772831537\\
15.29	0.00823037542444615\\
15.3	0.0082304031319753\\
15.31	0.00823043085090817\\
15.32	0.00823045858125009\\
15.33	0.00823048632300639\\
15.34	0.00823051407618243\\
15.35	0.00823054184078356\\
15.36	0.00823056961681511\\
15.37	0.00823059740428246\\
15.38	0.00823062520319094\\
15.39	0.00823065301354593\\
15.4	0.00823068083535278\\
15.41	0.00823070866861686\\
15.42	0.00823073651334353\\
15.43	0.00823076436953818\\
15.44	0.00823079223720617\\
15.45	0.00823082011635288\\
15.46	0.00823084800698368\\
15.47	0.00823087590910396\\
15.48	0.00823090382271911\\
15.49	0.00823093174783451\\
15.5	0.00823095968445557\\
15.51	0.00823098763258766\\
15.52	0.00823101559223619\\
15.53	0.00823104356340656\\
15.54	0.00823107154610417\\
15.55	0.00823109954033443\\
15.56	0.00823112754610275\\
15.57	0.00823115556341453\\
15.58	0.0082311835922752\\
15.59	0.00823121163269018\\
15.6	0.00823123968466488\\
15.61	0.00823126774820473\\
15.62	0.00823129582331515\\
15.63	0.00823132391000158\\
15.64	0.00823135200826944\\
15.65	0.00823138011812419\\
15.66	0.00823140823957124\\
15.67	0.00823143637261605\\
15.68	0.00823146451726405\\
15.69	0.00823149267352071\\
15.7	0.00823152084139146\\
15.71	0.00823154902088177\\
15.72	0.00823157721199709\\
15.73	0.00823160541474287\\
15.74	0.00823163362912458\\
15.75	0.0082316618551477\\
15.76	0.00823169009281768\\
15.77	0.00823171834213999\\
15.78	0.00823174660312013\\
15.79	0.00823177487576354\\
15.8	0.00823180316007574\\
15.81	0.00823183145606218\\
15.82	0.00823185976372837\\
15.83	0.00823188808307978\\
15.84	0.00823191641412192\\
15.85	0.00823194475686029\\
15.86	0.00823197311130036\\
15.87	0.00823200147744766\\
15.88	0.00823202985530768\\
15.89	0.00823205824488594\\
15.9	0.00823208664618794\\
15.91	0.00823211505921919\\
15.92	0.00823214348398522\\
15.93	0.00823217192049154\\
15.94	0.00823220036874368\\
15.95	0.00823222882874716\\
15.96	0.00823225730050751\\
15.97	0.00823228578403026\\
15.98	0.00823231427932095\\
15.99	0.00823234278638512\\
16	0.0082323713052283\\
16.01	0.00823239983585605\\
16.02	0.0082324283782739\\
16.03	0.0082324569324874\\
16.04	0.00823248549850212\\
16.05	0.0082325140763236\\
16.06	0.00823254266595741\\
16.07	0.00823257126740911\\
16.08	0.00823259988068426\\
16.09	0.00823262850578843\\
16.1	0.0082326571427272\\
16.11	0.00823268579150613\\
16.12	0.00823271445213081\\
16.13	0.0082327431246068\\
16.14	0.00823277180893972\\
16.15	0.00823280050513512\\
16.16	0.0082328292131986\\
16.17	0.00823285793313577\\
16.18	0.0082328866649522\\
16.19	0.00823291540865351\\
16.2	0.00823294416424529\\
16.21	0.00823297293173315\\
16.22	0.0082330017111227\\
16.23	0.00823303050241954\\
16.24	0.00823305930562929\\
16.25	0.00823308812075757\\
16.26	0.00823311694781\\
16.27	0.0082331457867922\\
16.28	0.0082331746377098\\
16.29	0.00823320350056842\\
16.3	0.00823323237537371\\
16.31	0.00823326126213129\\
16.32	0.0082332901608468\\
16.33	0.00823331907152589\\
16.34	0.00823334799417419\\
16.35	0.00823337692879737\\
16.36	0.00823340587540107\\
16.37	0.00823343483399093\\
16.38	0.00823346380457263\\
16.39	0.00823349278715181\\
16.4	0.00823352178173415\\
16.41	0.00823355078832531\\
16.42	0.00823357980693096\\
16.43	0.00823360883755678\\
16.44	0.00823363788020842\\
16.45	0.00823366693489159\\
16.46	0.00823369600161195\\
16.47	0.00823372508037519\\
16.48	0.008233754171187\\
16.49	0.00823378327405308\\
16.5	0.00823381238897911\\
16.51	0.0082338415159708\\
16.52	0.00823387065503384\\
16.53	0.00823389980617394\\
16.54	0.0082339289693968\\
16.55	0.00823395814470813\\
16.56	0.00823398733211365\\
16.57	0.00823401653161907\\
16.58	0.00823404574323012\\
16.59	0.0082340749669525\\
16.6	0.00823410420279196\\
16.61	0.00823413345075422\\
16.62	0.008234162710845\\
16.63	0.00823419198307005\\
16.64	0.0082342212674351\\
16.65	0.00823425056394589\\
16.66	0.00823427987260817\\
16.67	0.00823430919342768\\
16.68	0.00823433852641018\\
16.69	0.00823436787156142\\
16.7	0.00823439722888715\\
16.71	0.00823442659839313\\
16.72	0.00823445598008513\\
16.73	0.00823448537396892\\
16.74	0.00823451478005025\\
16.75	0.00823454419833491\\
16.76	0.00823457362882867\\
16.77	0.00823460307153731\\
16.78	0.00823463252646661\\
16.79	0.00823466199362235\\
16.8	0.00823469147301032\\
16.81	0.00823472096463632\\
16.82	0.00823475046850614\\
16.83	0.00823477998462557\\
16.84	0.00823480951300042\\
16.85	0.00823483905363649\\
16.86	0.00823486860653959\\
16.87	0.00823489817171552\\
16.88	0.0082349277491701\\
16.89	0.00823495733890915\\
16.9	0.00823498694093849\\
16.91	0.00823501655526394\\
16.92	0.00823504618189132\\
16.93	0.00823507582082647\\
16.94	0.0082351054720752\\
16.95	0.00823513513564338\\
16.96	0.00823516481153682\\
16.97	0.00823519449976138\\
16.98	0.00823522420032289\\
16.99	0.00823525391322721\\
17	0.00823528363848018\\
17.01	0.00823531337608766\\
17.02	0.00823534312605551\\
17.03	0.0082353728883896\\
17.04	0.00823540266309577\\
17.05	0.0082354324501799\\
17.06	0.00823546224964787\\
17.07	0.00823549206150554\\
17.08	0.0082355218857588\\
17.09	0.00823555172241351\\
17.1	0.00823558157147556\\
17.11	0.00823561143295085\\
17.12	0.00823564130684525\\
17.13	0.00823567119316467\\
17.14	0.00823570109191499\\
17.15	0.00823573100310213\\
17.16	0.00823576092673197\\
17.17	0.00823579086281043\\
17.18	0.00823582081134341\\
17.19	0.00823585077233683\\
17.2	0.0082358807457966\\
17.21	0.00823591073172865\\
17.22	0.00823594073013888\\
17.23	0.00823597074103322\\
17.24	0.00823600076441761\\
17.25	0.00823603080029798\\
17.26	0.00823606084868025\\
17.27	0.00823609090957037\\
17.28	0.00823612098297428\\
17.29	0.00823615106889792\\
17.3	0.00823618116734724\\
17.31	0.00823621127832819\\
17.32	0.00823624140184672\\
17.33	0.00823627153790879\\
17.34	0.00823630168652037\\
17.35	0.0082363318476874\\
17.36	0.00823636202141587\\
17.37	0.00823639220771174\\
17.38	0.00823642240658098\\
17.39	0.00823645261802956\\
17.4	0.00823648284206348\\
17.41	0.00823651307868871\\
17.42	0.00823654332791124\\
17.43	0.00823657358973705\\
17.44	0.00823660386417214\\
17.45	0.00823663415122252\\
17.46	0.00823666445089416\\
17.47	0.00823669476319308\\
17.48	0.00823672508812528\\
17.49	0.00823675542569678\\
17.5	0.00823678577591357\\
17.51	0.00823681613878168\\
17.52	0.00823684651430713\\
17.53	0.00823687690249594\\
17.54	0.00823690730335413\\
17.55	0.00823693771688774\\
17.56	0.00823696814310278\\
17.57	0.0082369985820053\\
17.58	0.00823702903360134\\
17.59	0.00823705949789694\\
17.6	0.00823708997489813\\
17.61	0.00823712046461098\\
17.62	0.00823715096704152\\
17.63	0.00823718148219582\\
17.64	0.00823721201007993\\
17.65	0.00823724255069992\\
17.66	0.00823727310406184\\
17.67	0.00823730367017177\\
17.68	0.00823733424903577\\
17.69	0.00823736484065993\\
17.7	0.0082373954450503\\
17.71	0.00823742606221299\\
17.72	0.00823745669215406\\
17.73	0.00823748733487961\\
17.74	0.00823751799039574\\
17.75	0.00823754865870852\\
17.76	0.00823757933982406\\
17.77	0.00823761003374846\\
17.78	0.00823764074048782\\
17.79	0.00823767146004825\\
17.8	0.00823770219243586\\
17.81	0.00823773293765677\\
17.82	0.00823776369571708\\
17.83	0.00823779446662293\\
17.84	0.00823782525038042\\
17.85	0.00823785604699569\\
17.86	0.00823788685647487\\
17.87	0.0082379176788241\\
17.88	0.0082379485140495\\
17.89	0.00823797936215722\\
17.9	0.0082380102231534\\
17.91	0.00823804109704419\\
17.92	0.00823807198383573\\
17.93	0.00823810288353419\\
17.94	0.00823813379614571\\
17.95	0.00823816472167646\\
17.96	0.00823819566013259\\
17.97	0.00823822661152028\\
17.98	0.0082382575758457\\
17.99	0.00823828855311501\\
18	0.0082383195433344\\
18.01	0.00823835054651003\\
18.02	0.00823838156264811\\
18.03	0.0082384125917548\\
18.04	0.00823844363383631\\
18.05	0.00823847468889882\\
18.06	0.00823850575694853\\
18.07	0.00823853683799164\\
18.08	0.00823856793203435\\
18.09	0.00823859903908287\\
18.1	0.00823863015914341\\
18.11	0.00823866129222217\\
18.12	0.00823869243832539\\
18.13	0.00823872359745927\\
18.14	0.00823875476963004\\
18.15	0.00823878595484392\\
18.16	0.00823881715310714\\
18.17	0.00823884836442595\\
18.18	0.00823887958880656\\
18.19	0.00823891082625522\\
18.2	0.00823894207677818\\
18.21	0.00823897334038168\\
18.22	0.00823900461707197\\
18.23	0.0082390359068553\\
18.24	0.00823906720973793\\
18.25	0.00823909852572612\\
18.26	0.00823912985482613\\
18.27	0.00823916119704423\\
18.28	0.00823919255238668\\
18.29	0.00823922392085977\\
18.3	0.00823925530246975\\
18.31	0.00823928669722293\\
18.32	0.00823931810512558\\
18.33	0.00823934952618397\\
18.34	0.00823938096040441\\
18.35	0.00823941240779319\\
18.36	0.0082394438683566\\
18.37	0.00823947534210094\\
18.38	0.00823950682903253\\
18.39	0.00823953832915765\\
18.4	0.00823956984248263\\
18.41	0.00823960136901377\\
18.42	0.00823963290875739\\
18.43	0.00823966446171982\\
18.44	0.00823969602790737\\
18.45	0.00823972760732638\\
18.46	0.00823975919998316\\
18.47	0.00823979080588406\\
18.48	0.00823982242503542\\
18.49	0.00823985405744357\\
18.5	0.00823988570311485\\
18.51	0.00823991736205561\\
18.52	0.00823994903427221\\
18.53	0.00823998071977099\\
18.54	0.00824001241855832\\
18.55	0.00824004413064056\\
18.56	0.00824007585602406\\
18.57	0.00824010759471519\\
18.58	0.00824013934672034\\
18.59	0.00824017111204586\\
18.6	0.00824020289069814\\
18.61	0.00824023468268356\\
18.62	0.0082402664880085\\
18.63	0.00824029830667936\\
18.64	0.00824033013870252\\
18.65	0.00824036198408436\\
18.66	0.0082403938428313\\
18.67	0.00824042571494974\\
18.68	0.00824045760044607\\
18.69	0.00824048949932672\\
18.7	0.00824052141159807\\
18.71	0.00824055333726656\\
18.72	0.0082405852763386\\
18.73	0.00824061722882062\\
18.74	0.00824064919471903\\
18.75	0.00824068117404027\\
18.76	0.00824071316679077\\
18.77	0.00824074517297695\\
18.78	0.00824077719260527\\
18.79	0.00824080922568216\\
18.8	0.00824084127221408\\
18.81	0.00824087333220746\\
18.82	0.00824090540566876\\
18.83	0.00824093749260443\\
18.84	0.00824096959302094\\
18.85	0.00824100170692475\\
18.86	0.00824103383432231\\
18.87	0.00824106597522012\\
18.88	0.00824109812962463\\
18.89	0.00824113029754231\\
18.9	0.00824116247897966\\
18.91	0.00824119467394315\\
18.92	0.00824122688243928\\
18.93	0.00824125910447452\\
18.94	0.00824129134005538\\
18.95	0.00824132358918834\\
18.96	0.00824135585187992\\
18.97	0.00824138812813661\\
18.98	0.00824142041796492\\
18.99	0.00824145272137137\\
19	0.00824148503836245\\
19.01	0.0082415173689447\\
19.02	0.00824154971312464\\
19.03	0.00824158207090878\\
19.04	0.00824161444230366\\
19.05	0.0082416468273158\\
19.06	0.00824167922595174\\
19.07	0.00824171163821802\\
19.08	0.00824174406412119\\
19.09	0.00824177650366778\\
19.1	0.00824180895686433\\
19.11	0.00824184142371742\\
19.12	0.00824187390423358\\
19.13	0.00824190639841939\\
19.14	0.00824193890628139\\
19.15	0.00824197142782616\\
19.16	0.00824200396306027\\
19.17	0.00824203651199028\\
19.18	0.00824206907462278\\
19.19	0.00824210165096434\\
19.2	0.00824213424102154\\
19.21	0.00824216684480098\\
19.22	0.00824219946230924\\
19.23	0.00824223209355291\\
19.24	0.00824226473853859\\
19.25	0.00824229739727289\\
19.26	0.0082423300697624\\
19.27	0.00824236275601373\\
19.28	0.00824239545603349\\
19.29	0.0082424281698283\\
19.3	0.00824246089740477\\
19.31	0.00824249363876953\\
19.32	0.00824252639392919\\
19.33	0.00824255916289039\\
19.34	0.00824259194565975\\
19.35	0.00824262474224392\\
19.36	0.00824265755264953\\
19.37	0.00824269037688322\\
19.38	0.00824272321495164\\
19.39	0.00824275606686143\\
19.4	0.00824278893261925\\
19.41	0.00824282181223175\\
19.42	0.0082428547057056\\
19.43	0.00824288761304745\\
19.44	0.00824292053426397\\
19.45	0.00824295346936183\\
19.46	0.0082429864183477\\
19.47	0.00824301938122826\\
19.48	0.00824305235801019\\
19.49	0.00824308534870016\\
19.5	0.00824311835330488\\
19.51	0.00824315137183103\\
19.52	0.0082431844042853\\
19.53	0.00824321745067439\\
19.54	0.008243250511005\\
19.55	0.00824328358528384\\
19.56	0.00824331667351761\\
19.57	0.00824334977571302\\
19.58	0.00824338289187679\\
19.59	0.00824341602201564\\
19.6	0.00824344916613628\\
19.61	0.00824348232424545\\
19.62	0.00824351549634987\\
19.63	0.00824354868245628\\
19.64	0.0082435818825714\\
19.65	0.00824361509670198\\
19.66	0.00824364832485477\\
19.67	0.0082436815670365\\
19.68	0.00824371482325393\\
19.69	0.00824374809351381\\
19.7	0.00824378137782289\\
19.71	0.00824381467618795\\
19.72	0.00824384798861573\\
19.73	0.00824388131511301\\
19.74	0.00824391465568656\\
19.75	0.00824394801034315\\
19.76	0.00824398137908956\\
19.77	0.00824401476193257\\
19.78	0.00824404815887897\\
19.79	0.00824408156993554\\
19.8	0.00824411499510907\\
19.81	0.00824414843440636\\
19.82	0.00824418188783422\\
19.83	0.00824421535539943\\
19.84	0.00824424883710881\\
19.85	0.00824428233296916\\
19.86	0.00824431584298731\\
19.87	0.00824434936717005\\
19.88	0.00824438290552422\\
19.89	0.00824441645805664\\
19.9	0.00824445002477413\\
19.91	0.00824448360568353\\
19.92	0.00824451720079166\\
19.93	0.00824455081010536\\
19.94	0.00824458443363148\\
19.95	0.00824461807137687\\
19.96	0.00824465172334836\\
19.97	0.0082446853895528\\
19.98	0.00824471906999706\\
19.99	0.008244752764688\\
20	0.00824478647363247\\
20.01	0.00824482019683733\\
20.02	0.00824485393430946\\
20.03	0.00824488768605574\\
20.04	0.00824492145208302\\
20.05	0.00824495523239821\\
20.06	0.00824498902700817\\
20.07	0.0082450228359198\\
20.08	0.00824505665913998\\
20.09	0.00824509049667561\\
20.1	0.00824512434853359\\
20.11	0.00824515821472081\\
20.12	0.00824519209524418\\
20.13	0.00824522599011061\\
20.14	0.00824525989932701\\
20.15	0.00824529382290029\\
20.16	0.00824532776083737\\
20.17	0.00824536171314518\\
20.18	0.00824539567983063\\
20.19	0.00824542966090066\\
20.2	0.0082454636563622\\
20.21	0.00824549766622218\\
20.22	0.00824553169048755\\
20.23	0.00824556572916525\\
20.24	0.00824559978226222\\
20.25	0.00824563384978541\\
20.26	0.00824566793174178\\
20.27	0.00824570202813829\\
20.28	0.00824573613898189\\
20.29	0.00824577026427955\\
20.3	0.00824580440403824\\
20.31	0.00824583855826492\\
20.32	0.00824587272696658\\
20.33	0.00824590691015019\\
20.34	0.00824594110782273\\
20.35	0.00824597531999119\\
20.36	0.00824600954666255\\
20.37	0.00824604378784382\\
20.38	0.00824607804354197\\
20.39	0.00824611231376403\\
20.4	0.00824614659851698\\
20.41	0.00824618089780783\\
20.42	0.0082462152116436\\
20.43	0.0082462495400313\\
20.44	0.00824628388297794\\
20.45	0.00824631824049054\\
20.46	0.00824635261257614\\
20.47	0.00824638699924175\\
20.48	0.00824642140049441\\
20.49	0.00824645581634116\\
20.5	0.00824649024678903\\
20.51	0.00824652469184506\\
20.52	0.0082465591515163\\
20.53	0.00824659362580981\\
20.54	0.00824662811473262\\
20.55	0.0082466626182918\\
20.56	0.00824669713649441\\
20.57	0.0082467316693475\\
20.58	0.00824676621685816\\
20.59	0.00824680077903344\\
20.6	0.00824683535588042\\
20.61	0.00824686994740618\\
20.62	0.0082469045536178\\
20.63	0.00824693917452237\\
20.64	0.00824697381012696\\
20.65	0.00824700846043869\\
20.66	0.00824704312546463\\
20.67	0.00824707780521189\\
20.68	0.00824711249968756\\
20.69	0.00824714720889877\\
20.7	0.00824718193285261\\
20.71	0.00824721667155619\\
20.72	0.00824725142501665\\
20.73	0.00824728619324109\\
20.74	0.00824732097623663\\
20.75	0.00824735577401042\\
20.76	0.00824739058656956\\
20.77	0.00824742541392121\\
20.78	0.0082474602560725\\
20.79	0.00824749511303056\\
20.8	0.00824752998480256\\
20.81	0.00824756487139562\\
20.82	0.00824759977281691\\
20.83	0.00824763468907358\\
20.84	0.00824766962017279\\
20.85	0.0082477045661217\\
20.86	0.00824773952692748\\
20.87	0.0082477745025973\\
20.88	0.00824780949313833\\
20.89	0.00824784449855775\\
20.9	0.00824787951886274\\
20.91	0.00824791455406048\\
20.92	0.00824794960415816\\
20.93	0.00824798466916298\\
20.94	0.00824801974908212\\
20.95	0.00824805484392278\\
20.96	0.00824808995369217\\
20.97	0.00824812507839749\\
20.98	0.00824816021804596\\
20.99	0.00824819537264478\\
21	0.00824823054220116\\
21.01	0.00824826572672234\\
21.02	0.00824830092621552\\
21.03	0.00824833614068795\\
21.04	0.00824837137014684\\
21.05	0.00824840661459944\\
21.06	0.00824844187405298\\
21.07	0.0082484771485147\\
21.08	0.00824851243799185\\
21.09	0.00824854774249167\\
21.1	0.00824858306202141\\
21.11	0.00824861839658834\\
21.12	0.0082486537461997\\
21.13	0.00824868911086277\\
21.14	0.0082487244905848\\
21.15	0.00824875988537307\\
21.16	0.00824879529523485\\
21.17	0.00824883072017741\\
21.18	0.00824886616020805\\
21.19	0.00824890161533403\\
21.2	0.00824893708556265\\
21.21	0.0082489725709012\\
21.22	0.00824900807135697\\
21.23	0.00824904358693726\\
21.24	0.00824907911764938\\
21.25	0.00824911466350061\\
21.26	0.00824915022449829\\
21.27	0.00824918580064971\\
21.28	0.0082492213919622\\
21.29	0.00824925699844307\\
21.3	0.00824929262009965\\
21.31	0.00824932825693926\\
21.32	0.00824936390896922\\
21.33	0.00824939957619689\\
21.34	0.00824943525862959\\
21.35	0.00824947095627466\\
21.36	0.00824950666913945\\
21.37	0.0082495423972313\\
21.38	0.00824957814055757\\
21.39	0.00824961389912562\\
21.4	0.00824964967294278\\
21.41	0.00824968546201645\\
21.42	0.00824972126635397\\
21.43	0.00824975708596272\\
21.44	0.00824979292085006\\
21.45	0.00824982877102338\\
21.46	0.00824986463649006\\
21.47	0.00824990051725748\\
21.48	0.00824993641333302\\
21.49	0.00824997232472408\\
21.5	0.00825000825143805\\
21.51	0.00825004419348233\\
21.52	0.00825008015086431\\
21.53	0.00825011612359141\\
21.54	0.00825015211167104\\
21.55	0.00825018811511059\\
21.56	0.0082502241339175\\
21.57	0.00825026016809918\\
21.58	0.00825029621766304\\
21.59	0.00825033228261652\\
21.6	0.00825036836296706\\
21.61	0.00825040445872207\\
21.62	0.008250440569889\\
21.63	0.00825047669647529\\
21.64	0.00825051283848838\\
21.65	0.00825054899593571\\
21.66	0.00825058516882475\\
21.67	0.00825062135716294\\
21.68	0.00825065756095775\\
21.69	0.00825069378021663\\
21.7	0.00825073001494705\\
21.71	0.00825076626515648\\
21.72	0.00825080253085239\\
21.73	0.00825083881204226\\
21.74	0.00825087510873357\\
21.75	0.00825091142093379\\
21.76	0.00825094774865042\\
21.77	0.00825098409189095\\
21.78	0.00825102045066288\\
21.79	0.00825105682497369\\
21.8	0.00825109321483089\\
21.81	0.00825112962024198\\
21.82	0.00825116604121448\\
21.83	0.00825120247775589\\
21.84	0.00825123892987372\\
21.85	0.00825127539757551\\
21.86	0.00825131188086876\\
21.87	0.00825134837976101\\
21.88	0.00825138489425979\\
21.89	0.00825142142437263\\
21.9	0.00825145797010706\\
21.91	0.00825149453147062\\
21.92	0.00825153110847086\\
21.93	0.00825156770111533\\
21.94	0.00825160430941158\\
21.95	0.00825164093336716\\
21.96	0.00825167757298962\\
21.97	0.00825171422828654\\
21.98	0.00825175089926547\\
21.99	0.00825178758593399\\
22	0.00825182428829966\\
22.01	0.00825186100637006\\
22.02	0.00825189774015278\\
22.03	0.00825193448965539\\
22.04	0.00825197125488548\\
22.05	0.00825200803585065\\
22.06	0.00825204483255847\\
22.07	0.00825208164501656\\
22.08	0.00825211847323251\\
22.09	0.00825215531721393\\
22.1	0.00825219217696843\\
22.11	0.0082522290525036\\
22.12	0.00825226594382708\\
22.13	0.00825230285094648\\
22.14	0.00825233977386942\\
22.15	0.00825237671260353\\
22.16	0.00825241366715643\\
22.17	0.00825245063753576\\
22.18	0.00825248762374916\\
22.19	0.00825252462580427\\
22.2	0.00825256164370872\\
22.21	0.00825259867747017\\
22.22	0.00825263572709626\\
22.23	0.00825267279259466\\
22.24	0.00825270987397301\\
22.25	0.00825274697123899\\
22.26	0.00825278408440024\\
22.27	0.00825282121346445\\
22.28	0.00825285835843929\\
22.29	0.00825289551933242\\
22.3	0.00825293269615154\\
22.31	0.00825296988890432\\
22.32	0.00825300709759845\\
22.33	0.00825304432224161\\
22.34	0.00825308156284152\\
22.35	0.00825311881940584\\
22.36	0.0082531560919423\\
22.37	0.00825319338045859\\
22.38	0.00825323068496243\\
22.39	0.00825326800546151\\
22.4	0.00825330534196357\\
22.41	0.00825334269447631\\
22.42	0.00825338006300745\\
22.43	0.00825341744756474\\
22.44	0.00825345484815588\\
22.45	0.00825349226478862\\
22.46	0.0082535296974707\\
22.47	0.00825356714620985\\
22.48	0.00825360461101381\\
22.49	0.00825364209189033\\
22.5	0.00825367958884717\\
22.51	0.00825371710189207\\
22.52	0.0082537546310328\\
22.53	0.00825379217627712\\
22.54	0.00825382973763278\\
22.55	0.00825386731510757\\
22.56	0.00825390490870924\\
22.57	0.00825394251844559\\
22.58	0.00825398014432439\\
22.59	0.00825401778635341\\
22.6	0.00825405544454045\\
22.61	0.00825409311889329\\
22.62	0.00825413080941974\\
22.63	0.00825416851612759\\
22.64	0.00825420623902463\\
22.65	0.00825424397811867\\
22.66	0.00825428173341753\\
22.67	0.008254319504929\\
22.68	0.00825435729266092\\
22.69	0.00825439509662109\\
22.7	0.00825443291681735\\
22.71	0.0082544707532575\\
22.72	0.00825450860594939\\
22.73	0.00825454647490085\\
22.74	0.00825458436011971\\
22.75	0.00825462226161382\\
22.76	0.00825466017939101\\
22.77	0.00825469811345914\\
22.78	0.00825473606382606\\
22.79	0.00825477403049963\\
22.8	0.00825481201348769\\
22.81	0.00825485001279812\\
22.82	0.00825488802843878\\
22.83	0.00825492606041754\\
22.84	0.00825496410874226\\
22.85	0.00825500217342084\\
22.86	0.00825504025446114\\
22.87	0.00825507835187106\\
22.88	0.00825511646565848\\
22.89	0.00825515459583128\\
22.9	0.00825519274239738\\
22.91	0.00825523090536465\\
22.92	0.00825526908474102\\
22.93	0.00825530728053438\\
22.94	0.00825534549275263\\
22.95	0.0082553837214037\\
22.96	0.00825542196649551\\
22.97	0.00825546022803596\\
22.98	0.00825549850603299\\
22.99	0.00825553680049452\\
23	0.00825557511142849\\
23.01	0.00825561343884282\\
23.02	0.00825565178274546\\
23.03	0.00825569014314436\\
23.04	0.00825572852004745\\
23.05	0.00825576691346268\\
23.06	0.00825580532339801\\
23.07	0.0082558437498614\\
23.08	0.00825588219286081\\
23.09	0.00825592065240419\\
23.1	0.00825595912849953\\
23.11	0.00825599762115479\\
23.12	0.00825603613037794\\
23.13	0.00825607465617697\\
23.14	0.00825611319855986\\
23.15	0.00825615175753459\\
23.16	0.00825619033310916\\
23.17	0.00825622892529156\\
23.18	0.00825626753408978\\
23.19	0.00825630615951183\\
23.2	0.00825634480156572\\
23.21	0.00825638346025945\\
23.22	0.00825642213560104\\
23.23	0.0082564608275985\\
23.24	0.00825649953625985\\
23.25	0.00825653826159312\\
23.26	0.00825657700360633\\
23.27	0.00825661576230752\\
23.28	0.00825665453770472\\
23.29	0.00825669332980598\\
23.3	0.00825673213861932\\
23.31	0.0082567709641528\\
23.32	0.00825680980641449\\
23.33	0.00825684866541241\\
23.34	0.00825688754115463\\
23.35	0.00825692643364923\\
23.36	0.00825696534290425\\
23.37	0.00825700426892777\\
23.38	0.00825704321172786\\
23.39	0.00825708217131261\\
23.4	0.0082571211476901\\
23.41	0.0082571601408684\\
23.42	0.00825719915085561\\
23.43	0.00825723817765981\\
23.44	0.00825727722128912\\
23.45	0.00825731628175161\\
23.46	0.00825735535905541\\
23.47	0.00825739445320863\\
23.48	0.00825743356421936\\
23.49	0.00825747269209573\\
23.5	0.00825751183684586\\
23.51	0.00825755099847787\\
23.52	0.0082575901769999\\
23.53	0.00825762937242006\\
23.54	0.00825766858474651\\
23.55	0.00825770781398737\\
23.56	0.0082577470601508\\
23.57	0.00825778632324494\\
23.58	0.00825782560327794\\
23.59	0.00825786490025796\\
23.6	0.00825790421419316\\
23.61	0.0082579435450917\\
23.62	0.00825798289296175\\
23.63	0.00825802225781149\\
23.64	0.00825806163964908\\
23.65	0.00825810103848272\\
23.66	0.00825814045432059\\
23.67	0.00825817988717086\\
23.68	0.00825821933704174\\
23.69	0.00825825880394142\\
23.7	0.00825829828787811\\
23.71	0.00825833778886001\\
23.72	0.00825837730689533\\
23.73	0.00825841684199228\\
23.74	0.00825845639415908\\
23.75	0.00825849596340395\\
23.76	0.00825853554973512\\
23.77	0.00825857515316082\\
23.78	0.00825861477368929\\
23.79	0.00825865441132876\\
23.8	0.00825869406608748\\
23.81	0.00825873373797369\\
23.82	0.00825877342699565\\
23.83	0.00825881313316161\\
23.84	0.00825885285647983\\
23.85	0.00825889259695859\\
23.86	0.00825893235460614\\
23.87	0.00825897212943077\\
23.88	0.00825901192144075\\
23.89	0.00825905173064436\\
23.9	0.0082590915570499\\
23.91	0.00825913140066564\\
23.92	0.0082591712614999\\
23.93	0.00825921113956098\\
23.94	0.00825925103485717\\
23.95	0.00825929094739678\\
23.96	0.00825933087718814\\
23.97	0.00825937082423956\\
23.98	0.00825941078855937\\
23.99	0.00825945077015589\\
24	0.00825949076903746\\
24.01	0.00825953078521241\\
24.02	0.0082595708186891\\
24.03	0.00825961086947586\\
24.04	0.00825965093758105\\
24.05	0.00825969102301303\\
24.06	0.00825973112578017\\
24.07	0.00825977124589081\\
24.08	0.00825981138335334\\
24.09	0.00825985153817614\\
24.1	0.00825989171036759\\
24.11	0.00825993189993608\\
24.12	0.00825997210688999\\
24.13	0.00826001233123772\\
24.14	0.00826005257298768\\
24.15	0.00826009283214827\\
24.16	0.00826013310872792\\
24.17	0.00826017340273502\\
24.18	0.00826021371417801\\
24.19	0.00826025404306532\\
24.2	0.00826029438940537\\
24.21	0.00826033475320661\\
24.22	0.00826037513447749\\
24.23	0.00826041553322645\\
24.24	0.00826045594946194\\
24.25	0.00826049638319242\\
24.26	0.00826053683442637\\
24.27	0.00826057730317226\\
24.28	0.00826061778943855\\
24.29	0.00826065829323374\\
24.3	0.00826069881456632\\
24.31	0.00826073935344477\\
24.32	0.00826077990987759\\
24.33	0.0082608204838733\\
24.34	0.00826086107544041\\
24.35	0.00826090168458743\\
24.36	0.00826094231132288\\
24.37	0.0082609829556553\\
24.38	0.00826102361759323\\
24.39	0.00826106429714519\\
24.4	0.00826110499431975\\
24.41	0.00826114570912545\\
24.42	0.00826118644157086\\
24.43	0.00826122719166454\\
24.44	0.00826126795941508\\
24.45	0.00826130874483103\\
24.46	0.00826134954792099\\
24.47	0.00826139036869356\\
24.48	0.00826143120715734\\
24.49	0.00826147206332093\\
24.5	0.00826151293719293\\
24.51	0.00826155382878198\\
24.52	0.0082615947380967\\
24.53	0.00826163566514572\\
24.54	0.00826167660993768\\
24.55	0.00826171757248123\\
24.56	0.00826175855278502\\
24.57	0.00826179955085772\\
24.58	0.008261840566708\\
24.59	0.00826188160034452\\
24.6	0.00826192265177598\\
24.61	0.00826196372101106\\
24.62	0.00826200480805847\\
24.63	0.00826204591292691\\
24.64	0.0082620870356251\\
24.65	0.00826212817616176\\
24.66	0.00826216933454561\\
24.67	0.0082622105107854\\
24.68	0.00826225170488986\\
24.69	0.00826229291686777\\
24.7	0.00826233414672788\\
24.71	0.00826237539447895\\
24.72	0.00826241666012978\\
24.73	0.00826245794368914\\
24.74	0.00826249924516583\\
24.75	0.00826254056456866\\
24.76	0.00826258190190645\\
24.77	0.00826262325718801\\
24.78	0.00826266463042218\\
24.79	0.0082627060216178\\
24.8	0.00826274743078372\\
24.81	0.0082627888579288\\
24.82	0.00826283030306191\\
24.83	0.00826287176619193\\
24.84	0.00826291324732774\\
24.85	0.00826295474647825\\
24.86	0.00826299626365237\\
24.87	0.00826303779885901\\
24.88	0.0082630793521071\\
24.89	0.00826312092340559\\
24.9	0.00826316251276342\\
24.91	0.00826320412018955\\
24.92	0.00826324574569296\\
24.93	0.00826328738928262\\
24.94	0.00826332905096754\\
24.95	0.00826337073075671\\
24.96	0.00826341242865914\\
24.97	0.00826345414468388\\
24.98	0.00826349587883995\\
24.99	0.00826353763113641\\
25	0.00826357940158232\\
25.01	0.00826362119018675\\
25.02	0.00826366299695879\\
25.03	0.00826370482190756\\
25.04	0.00826374666504214\\
25.05	0.00826378852637167\\
25.06	0.00826383040590529\\
25.07	0.00826387230365215\\
25.08	0.00826391421962141\\
25.09	0.00826395615382226\\
25.1	0.00826399810626388\\
25.11	0.00826404007695548\\
25.12	0.00826408206590627\\
25.13	0.00826412407312551\\
25.14	0.00826416609862242\\
25.15	0.00826420814240629\\
25.16	0.00826425020448638\\
25.17	0.00826429228487199\\
25.18	0.00826433438357243\\
25.19	0.00826437650059703\\
25.2	0.00826441863595512\\
25.21	0.00826446078965606\\
25.22	0.00826450296170923\\
25.23	0.00826454515212403\\
25.24	0.00826458736090984\\
25.25	0.0082646295880761\\
25.26	0.00826467183363225\\
25.27	0.00826471409758776\\
25.28	0.00826475637995209\\
25.29	0.00826479868073473\\
25.3	0.00826484099994522\\
25.31	0.00826488333759308\\
25.32	0.00826492569368785\\
25.33	0.00826496806823912\\
25.34	0.00826501046125648\\
25.35	0.00826505287274952\\
25.36	0.00826509530272789\\
25.37	0.00826513775120125\\
25.38	0.00826518021817924\\
25.39	0.00826522270367159\\
25.4	0.008265265207688\\
25.41	0.0082653077302382\\
25.42	0.00826535027133196\\
25.43	0.00826539283097906\\
25.44	0.00826543540918931\\
25.45	0.00826547800597254\\
25.46	0.0082655206213386\\
25.47	0.00826556325529736\\
25.48	0.00826560590785874\\
25.49	0.00826564857903266\\
25.5	0.00826569126882906\\
25.51	0.00826573397725793\\
25.52	0.00826577670432928\\
25.53	0.00826581945005312\\
25.54	0.00826586221443954\\
25.55	0.00826590499749861\\
25.56	0.00826594779924044\\
25.57	0.00826599061967517\\
25.58	0.00826603345881299\\
25.59	0.00826607631666409\\
25.6	0.0082661191932387\\
25.61	0.00826616208854709\\
25.62	0.00826620500259955\\
25.63	0.00826624793540641\\
25.64	0.00826629088697801\\
25.65	0.00826633385732476\\
25.66	0.00826637684645707\\
25.67	0.00826641985438541\\
25.68	0.00826646288112026\\
25.69	0.00826650592667214\\
25.7	0.00826654899105164\\
25.71	0.00826659207426933\\
25.72	0.00826663517633585\\
25.73	0.00826667829726189\\
25.74	0.00826672143705815\\
25.75	0.00826676459573537\\
25.76	0.00826680777330435\\
25.77	0.00826685096977593\\
25.78	0.00826689418516095\\
25.79	0.00826693741947035\\
25.8	0.00826698067271508\\
25.81	0.00826702394490612\\
25.82	0.00826706723605453\\
25.83	0.00826711054617139\\
25.84	0.00826715387526783\\
25.85	0.00826719722335504\\
25.86	0.00826724059044422\\
25.87	0.00826728397654666\\
25.88	0.00826732738167368\\
25.89	0.00826737080583665\\
25.9	0.00826741424904699\\
25.91	0.00826745771131617\\
25.92	0.00826750119265572\\
25.93	0.00826754469307723\\
25.94	0.00826758821259231\\
25.95	0.00826763175121266\\
25.96	0.00826767530895003\\
25.97	0.00826771888581621\\
25.98	0.00826776248182307\\
25.99	0.00826780609698253\\
26	0.00826784973130656\\
26.01	0.00826789338480721\\
26.02	0.00826793705749658\\
26.03	0.00826798074938684\\
26.04	0.00826802446049022\\
26.05	0.00826806819081903\\
26.06	0.00826811194038562\\
26.07	0.00826815570920244\\
26.08	0.00826819949728198\\
26.09	0.00826824330463683\\
26.1	0.00826828713127962\\
26.11	0.00826833097722309\\
26.12	0.00826837484248004\\
26.13	0.00826841872706332\\
26.14	0.00826846263098591\\
26.15	0.00826850655426082\\
26.16	0.00826855049690117\\
26.17	0.00826859445892015\\
26.18	0.00826863844033104\\
26.19	0.0082686824411472\\
26.2	0.0082687264613821\\
26.21	0.00826877050104925\\
26.22	0.0082688145601623\\
26.23	0.00826885863873495\\
26.24	0.00826890273678104\\
26.25	0.00826894685431446\\
26.26	0.00826899099134923\\
26.27	0.00826903514789945\\
26.28	0.00826907932397931\\
26.29	0.00826912351960316\\
26.3	0.00826916773478537\\
26.31	0.00826921196954049\\
26.32	0.00826925622388314\\
26.33	0.00826930049782806\\
26.34	0.00826934479139009\\
26.35	0.00826938910458421\\
26.36	0.0082694334374255\\
26.37	0.00826947778992917\\
26.38	0.00826952216211053\\
26.39	0.00826956655398504\\
26.4	0.00826961096556826\\
26.41	0.00826965539687591\\
26.42	0.00826969984792381\\
26.43	0.00826974431872793\\
26.44	0.00826978880930436\\
26.45	0.00826983331966935\\
26.46	0.00826987784983929\\
26.47	0.00826992239983069\\
26.48	0.00826996696966022\\
26.49	0.0082700115593447\\
26.5	0.0082700561689011\\
26.51	0.00827010079834655\\
26.52	0.00827014544769833\\
26.53	0.00827019011697388\\
26.54	0.00827023480619082\\
26.55	0.0082702795153669\\
26.56	0.00827032424452009\\
26.57	0.00827036899366849\\
26.58	0.0082704137628304\\
26.59	0.00827045855202428\\
26.6	0.0082705033612688\\
26.61	0.00827054819058279\\
26.62	0.00827059303998529\\
26.63	0.00827063790949552\\
26.64	0.00827068279913291\\
26.65	0.00827072770891708\\
26.66	0.00827077263886786\\
26.67	0.00827081758900529\\
26.68	0.00827086255934961\\
26.69	0.00827090754992129\\
26.7	0.00827095256074103\\
26.71	0.00827099759182973\\
26.72	0.00827104264320854\\
26.73	0.00827108771489883\\
26.74	0.00827113280692223\\
26.75	0.00827117791930057\\
26.76	0.00827122305205598\\
26.77	0.0082712682052108\\
26.78	0.00827131337878766\\
26.79	0.00827135857280941\\
26.8	0.00827140378729922\\
26.81	0.00827144902228047\\
26.82	0.00827149427777689\\
26.83	0.00827153955381242\\
26.84	0.00827158485041133\\
26.85	0.00827163016759818\\
26.86	0.00827167550539781\\
26.87	0.00827172086383538\\
26.88	0.00827176624293636\\
26.89	0.00827181164272653\\
26.9	0.00827185706323198\\
26.91	0.00827190250447916\\
26.92	0.00827194796649482\\
26.93	0.00827199344930608\\
26.94	0.00827203895294038\\
26.95	0.00827208447742553\\
26.96	0.0082721300227897\\
26.97	0.00827217558906142\\
26.98	0.0082722211762696\\
26.99	0.00827226678444353\\
27	0.00827231241361289\\
27.01	0.00827235806380775\\
27.02	0.00827240373505857\\
27.03	0.00827244942739627\\
27.04	0.00827249514085212\\
27.05	0.00827254087545788\\
27.06	0.00827258663124571\\
27.07	0.0082726324082482\\
27.08	0.00827267820649844\\
27.09	0.00827272402602993\\
27.1	0.00827276986687667\\
27.11	0.00827281572907311\\
27.12	0.00827286161265422\\
27.13	0.00827290751765543\\
27.14	0.00827295344411271\\
27.15	0.00827299939206251\\
27.16	0.00827304536154182\\
27.17	0.00827309135258818\\
27.18	0.00827313736523963\\
27.19	0.00827318339953481\\
27.2	0.0082732294555129\\
27.21	0.00827327553321365\\
27.22	0.00827332163267741\\
27.23	0.00827336775394512\\
27.24	0.00827341389705833\\
27.25	0.0082734600620592\\
27.26	0.00827350624899054\\
27.27	0.00827355245789577\\
27.28	0.008273598688819\\
27.29	0.00827364494180497\\
27.3	0.00827369121689913\\
27.31	0.00827373751414761\\
27.32	0.00827378383359724\\
27.33	0.00827383017529557\\
27.34	0.00827387653929087\\
27.35	0.00827392292563217\\
27.36	0.00827396933436925\\
27.37	0.00827401576555266\\
27.38	0.00827406221923373\\
27.39	0.0082741086954646\\
27.4	0.00827415519429823\\
27.41	0.00827420171578838\\
27.42	0.00827424825998968\\
27.43	0.00827429482695763\\
27.44	0.00827434141674857\\
27.45	0.00827438802941976\\
27.46	0.00827443466502936\\
27.47	0.00827448132363644\\
27.48	0.00827452800530103\\
27.49	0.00827457471008411\\
27.5	0.00827462143804764\\
27.51	0.00827466818925456\\
27.52	0.00827471496376884\\
27.53	0.00827476176165545\\
27.54	0.00827480858298044\\
27.55	0.0082748554278109\\
27.56	0.00827490229621502\\
27.57	0.00827494918826208\\
27.58	0.00827499610402249\\
27.59	0.00827504304356781\\
27.6	0.00827509000697076\\
27.61	0.00827513699430523\\
27.62	0.00827518400564633\\
27.63	0.0082752310410704\\
27.64	0.00827527810065501\\
27.65	0.00827532518447901\\
27.66	0.00827537229262253\\
27.67	0.00827541942516704\\
27.68	0.00827546658219531\\
27.69	0.00827551376379148\\
27.7	0.0082755609700411\\
27.71	0.00827560820103109\\
27.72	0.00827565545684983\\
27.73	0.00827570273758711\\
27.74	0.00827575004333425\\
27.75	0.00827579737418405\\
27.76	0.00827584473023082\\
27.77	0.00827589211157046\\
27.78	0.00827593951830043\\
27.79	0.0082759869505198\\
27.8	0.00827603440832927\\
27.81	0.00827608189183122\\
27.82	0.00827612940112969\\
27.83	0.00827617693633044\\
27.84	0.00827622449754099\\
27.85	0.00827627208487062\\
27.86	0.0082763196984304\\
27.87	0.00827636733833325\\
27.88	0.00827641500469394\\
27.89	0.00827646269762911\\
27.9	0.00827651041725733\\
27.91	0.00827655816369912\\
27.92	0.00827660593707698\\
27.93	0.0082766537375154\\
27.94	0.00827670156514092\\
27.95	0.00827674942008215\\
27.96	0.0082767973024698\\
27.97	0.00827684521243671\\
27.98	0.00827689315011788\\
27.99	0.00827694111565053\\
28	0.00827698910917407\\
28.01	0.00827703713083022\\
28.02	0.00827708518076295\\
28.03	0.00827713325911858\\
28.04	0.00827718136604578\\
28.05	0.00827722950169563\\
28.06	0.00827727766622162\\
28.07	0.00827732585977971\\
28.08	0.00827737408252836\\
28.09	0.00827742233462854\\
28.1	0.00827747061624379\\
28.11	0.00827751892754027\\
28.12	0.00827756726868673\\
28.13	0.00827761563985461\\
28.14	0.00827766404121804\\
28.15	0.00827771247295389\\
28.16	0.00827776093524179\\
28.17	0.00827780942826416\\
28.18	0.00827785795220626\\
28.19	0.00827790650725624\\
28.2	0.0082779550936051\\
28.21	0.00827800371144683\\
28.22	0.00827805236097834\\
28.23	0.00827810104239955\\
28.24	0.00827814975591343\\
28.25	0.00827819850172599\\
28.26	0.00827824728004634\\
28.27	0.00827829609108672\\
28.28	0.00827834493506251\\
28.29	0.00827839381219228\\
28.3	0.00827844272269782\\
28.31	0.00827849166680414\\
28.32	0.00827854064473956\\
28.33	0.00827858965673564\\
28.34	0.0082786387030273\\
28.35	0.00827868778385278\\
28.36	0.00827873689945372\\
28.37	0.00827878605007512\\
28.38	0.00827883523596541\\
28.39	0.00827888445737645\\
28.4	0.00827893371456355\\
28.41	0.00827898300778548\\
28.42	0.00827903233730451\\
28.43	0.00827908170338639\\
28.44	0.00827913110630041\\
28.45	0.00827918054631934\\
28.46	0.00827923002371952\\
28.47	0.00827927953878079\\
28.48	0.00827932909178655\\
28.49	0.00827937868302373\\
28.5	0.0082794283127828\\
28.51	0.00827947798135777\\
28.52	0.00827952768904618\\
28.53	0.00827957743614908\\
28.54	0.00827962722297102\\
28.55	0.00827967704982007\\
28.56	0.00827972691700774\\
28.57	0.00827977682484899\\
28.58	0.00827982677366223\\
28.59	0.00827987676376923\\
28.6	0.00827992679549511\\
28.61	0.00827997686916832\\
28.62	0.00828002698512058\\
28.63	0.00828007714368681\\
28.64	0.00828012734520512\\
28.65	0.0082801775900167\\
28.66	0.0082802278784658\\
28.67	0.00828027821089961\\
28.68	0.00828032858766823\\
28.69	0.00828037900912454\\
28.7	0.00828042947562416\\
28.71	0.00828047998752529\\
28.72	0.00828053054518866\\
28.73	0.00828058114897736\\
28.74	0.00828063179925676\\
28.75	0.00828068249639438\\
28.76	0.00828073324075971\\
28.77	0.00828078403272409\\
28.78	0.00828083487266055\\
28.79	0.0082808857609436\\
28.8	0.00828093669794912\\
28.81	0.00828098768405411\\
28.82	0.00828103871963648\\
28.83	0.00828108980507488\\
28.84	0.00828114094074844\\
28.85	0.00828119212703652\\
28.86	0.00828124336431845\\
28.87	0.00828129465297329\\
28.88	0.00828134599337951\\
28.89	0.0082813973859147\\
28.9	0.00828144883095524\\
28.91	0.00828150032887596\\
28.92	0.0082815518800498\\
28.93	0.00828160348484743\\
28.94	0.00828165514363682\\
28.95	0.0082817068567829\\
28.96	0.00828175862464703\\
28.97	0.00828181044758661\\
28.98	0.00828186232595457\\
28.99	0.00828191426009886\\
29	0.00828196625036193\\
29.01	0.00828201829708017\\
29.02	0.0082820704005833\\
29.03	0.00828212256119378\\
29.04	0.00828217477922618\\
29.05	0.00828222705498644\\
29.06	0.00828227938877125\\
29.07	0.00828233178086723\\
29.08	0.00828238423155021\\
29.09	0.00828243674108439\\
29.1	0.00828248930972148\\
29.11	0.00828254193769984\\
29.12	0.00828259462524352\\
29.13	0.00828264737256133\\
29.14	0.00828270017984578\\
29.15	0.00828275304727201\\
29.16	0.00828280597499673\\
29.17	0.00828285896315701\\
29.18	0.0082829120118691\\
29.19	0.00828296512122714\\
29.2	0.00828301829130184\\
29.21	0.00828307152213913\\
29.22	0.00828312481375868\\
29.23	0.00828317816615242\\
29.24	0.00828323157928295\\
29.25	0.00828328505308194\\
29.26	0.00828333858744837\\
29.27	0.00828339218224682\\
29.28	0.00828344583730552\\
29.29	0.00828349955241452\\
29.3	0.00828355332732357\\
29.31	0.00828360716174011\\
29.32	0.00828366105532701\\
29.33	0.00828371500770031\\
29.34	0.00828376901842686\\
29.35	0.00828382308702179\\
29.36	0.008283877212946\\
29.37	0.0082839313956034\\
29.38	0.00828398563433816\\
29.39	0.00828403992843177\\
29.4	0.00828409427710003\\
29.41	0.00828414867948984\\
29.42	0.00828420313467596\\
29.43	0.00828425764165757\\
29.44	0.00828431219935471\\
29.45	0.00828436680660456\\
29.46	0.0082844214621576\\
29.47	0.00828447616467358\\
29.48	0.00828453091271738\\
29.49	0.00828458570475463\\
29.5	0.00828464053914719\\
29.51	0.0082846954141485\\
29.52	0.00828475032789865\\
29.53	0.00828480527841933\\
29.54	0.00828486026360853\\
29.55	0.00828491528123504\\
29.56	0.00828497033093327\\
29.57	0.00828502541264783\\
29.58	0.00828508052631517\\
29.59	0.00828513567186332\\
29.6	0.00828519084921148\\
29.61	0.00828524605826963\\
29.62	0.00828530129893816\\
29.63	0.00828535657110746\\
29.64	0.0082854118746575\\
29.65	0.00828546720945739\\
29.66	0.00828552257536492\\
29.67	0.00828557797222612\\
29.68	0.00828563339987476\\
29.69	0.00828568885813184\\
29.7	0.00828574434680511\\
29.71	0.00828579986568847\\
29.72	0.00828585541456149\\
29.73	0.00828591099318877\\
29.74	0.00828596660131942\\
29.75	0.00828602223868635\\
29.76	0.00828607790500572\\
29.77	0.00828613359997624\\
29.78	0.00828618932327849\\
29.79	0.00828624507457424\\
29.8	0.00828630085350568\\
29.81	0.00828635665969472\\
29.82	0.00828641249274215\\
29.83	0.00828646835222689\\
29.84	0.00828652423770513\\
29.85	0.00828658014870944\\
29.86	0.00828663608474795\\
29.87	0.00828669204530334\\
29.88	0.00828674802983197\\
29.89	0.00828680403776281\\
29.9	0.00828686006849651\\
29.91	0.00828691612140427\\
29.92	0.00828697219582677\\
29.93	0.00828702829107307\\
29.94	0.00828708440641944\\
29.95	0.00828714054110808\\
29.96	0.00828719669434601\\
29.97	0.00828725286530364\\
29.98	0.00828730905311356\\
29.99	0.00828736525686907\\
30	0.0082874214756228\\
30.01	0.00828747770838528\\
30.02	0.00828753395412334\\
30.03	0.00828759021175858\\
30.04	0.00828764648016575\\
30.05	0.00828770275817109\\
30.06	0.00828775904455054\\
30.07	0.00828781533802803\\
30.08	0.00828787163727355\\
30.09	0.0082879279409013\\
30.1	0.00828798424746766\\
30.11	0.00828804055546921\\
30.12	0.00828809686334057\\
30.13	0.00828815316945224\\
30.14	0.00828820947210835\\
30.15	0.00828826576954435\\
30.16	0.00828832205992457\\
30.17	0.00828837834133977\\
30.18	0.00828843461180457\\
30.19	0.00828849086925478\\
30.2	0.00828854711154469\\
30.21	0.00828860333644423\\
30.22	0.00828865954163607\\
30.23	0.00828871572471256\\
30.24	0.00828877188317269\\
30.25	0.00828882801441878\\
30.26	0.00828888411575322\\
30.27	0.00828894018437506\\
30.28	0.00828899621737637\\
30.29	0.00828905221173869\\
30.3	0.00828910816432916\\
30.31	0.00828916407189669\\
30.32	0.00828921993106786\\
30.33	0.00828927573834282\\
30.34	0.00828933149009094\\
30.35	0.0082893871825464\\
30.36	0.00828944281180359\\
30.37	0.00828949837381238\\
30.38	0.00828955386437325\\
30.39	0.0082896092791322\\
30.4	0.00828966461357558\\
30.41	0.00828971986302467\\
30.42	0.00828977502263018\\
30.43	0.00828983008736643\\
30.44	0.00828988505202551\\
30.45	0.00828993991121111\\
30.46	0.00828999465933221\\
30.47	0.0082900492905966\\
30.48	0.0082901037990041\\
30.49	0.00829015817833961\\
30.5	0.00829021242216601\\
30.51	0.00829026652381666\\
30.52	0.00829032047638779\\
30.53	0.00829037427273063\\
30.54	0.00829042790544323\\
30.55	0.00829048136686207\\
30.56	0.00829053464905334\\
30.57	0.00829058774380402\\
30.58	0.00829064064261259\\
30.59	0.00829069333667951\\
30.6	0.00829074581689733\\
30.61	0.00829079807384052\\
30.62	0.00829085009775498\\
30.63	0.0082909020045421\\
30.64	0.00829095393125352\\
30.65	0.00829100587789864\\
30.66	0.00829105784448683\\
30.67	0.00829110983102749\\
30.68	0.00829116183753\\
30.69	0.00829121386400377\\
30.7	0.0082912659104582\\
30.71	0.0082913179769027\\
30.72	0.0082913700633467\\
30.73	0.00829142216979962\\
30.74	0.00829147429627089\\
30.75	0.00829152644276994\\
30.76	0.00829157860930624\\
30.77	0.00829163079588922\\
30.78	0.00829168300252834\\
30.79	0.00829173522923307\\
30.8	0.00829178747601288\\
30.81	0.00829183974287723\\
30.82	0.00829189202983563\\
30.83	0.00829194433689755\\
30.84	0.00829199666407249\\
30.85	0.00829204901136996\\
30.86	0.00829210137879946\\
30.87	0.00829215376637051\\
30.88	0.00829220617409262\\
30.89	0.00829225860197533\\
30.9	0.00829231105002818\\
30.91	0.0082923635182607\\
30.92	0.00829241600668243\\
30.93	0.00829246851530294\\
30.94	0.00829252104413179\\
30.95	0.00829257359317854\\
30.96	0.00829262616245276\\
30.97	0.00829267875196404\\
30.98	0.00829273136172195\\
30.99	0.00829278399173611\\
31	0.00829283664201609\\
31.01	0.00829288931257152\\
31.02	0.00829294200341199\\
31.03	0.00829299471454714\\
31.04	0.00829304744598658\\
31.05	0.00829310019773996\\
31.06	0.0082931529698169\\
31.07	0.00829320576222705\\
31.08	0.00829325857498007\\
31.09	0.00829331140808561\\
31.1	0.00829336426155334\\
31.11	0.00829341713539292\\
31.12	0.00829347002961405\\
31.13	0.0082935229442264\\
31.14	0.00829357587923966\\
31.15	0.00829362883466353\\
31.16	0.00829368181050772\\
31.17	0.00829373480678195\\
31.18	0.00829378782349591\\
31.19	0.00829384086065935\\
31.2	0.00829389391828199\\
31.21	0.00829394699637356\\
31.22	0.00829400009494383\\
31.23	0.00829405321400253\\
31.24	0.00829410635355942\\
31.25	0.00829415951362427\\
31.26	0.00829421269420685\\
31.27	0.00829426589531694\\
31.28	0.00829431911696432\\
31.29	0.00829437235915879\\
31.3	0.00829442562191014\\
31.31	0.00829447890522817\\
31.32	0.00829453220912271\\
31.33	0.00829458553360356\\
31.34	0.00829463887868056\\
31.35	0.00829469224436353\\
31.36	0.00829474563066231\\
31.37	0.00829479903758674\\
31.38	0.0082948524651467\\
31.39	0.00829490591335202\\
31.4	0.00829495938221257\\
31.41	0.00829501287173823\\
31.42	0.00829506638193888\\
31.43	0.00829511991282441\\
31.44	0.0082951734644047\\
31.45	0.00829522703668966\\
31.46	0.00829528062968918\\
31.47	0.0082953342434132\\
31.48	0.00829538787787162\\
31.49	0.00829544153307438\\
31.5	0.0082954952090314\\
31.51	0.00829554890575263\\
31.52	0.00829560262324802\\
31.53	0.00829565636152752\\
31.54	0.00829571012060109\\
31.55	0.0082957639004787\\
31.56	0.00829581770117033\\
31.57	0.00829587152268596\\
31.58	0.00829592536503557\\
31.59	0.00829597922822917\\
31.6	0.00829603311227675\\
31.61	0.00829608701718833\\
31.62	0.00829614094297392\\
31.63	0.00829619488964355\\
31.64	0.00829624885720725\\
31.65	0.00829630284567506\\
31.66	0.00829635685505701\\
31.67	0.00829641088536316\\
31.68	0.00829646493660358\\
31.69	0.00829651900878832\\
31.7	0.00829657310192746\\
31.71	0.00829662721603107\\
31.72	0.00829668135110925\\
31.73	0.00829673550717209\\
31.74	0.00829678968422969\\
31.75	0.00829684388229216\\
31.76	0.0082968981013696\\
31.77	0.00829695234147215\\
31.78	0.00829700660260993\\
31.79	0.00829706088479308\\
31.8	0.00829711518803174\\
31.81	0.00829716951233606\\
31.82	0.0082972238577162\\
31.83	0.00829727822418232\\
31.84	0.00829733261174459\\
31.85	0.00829738702041319\\
31.86	0.00829744145019831\\
31.87	0.00829749590111014\\
31.88	0.00829755037315887\\
31.89	0.00829760486635473\\
31.9	0.00829765938070791\\
31.91	0.00829771391622865\\
31.92	0.00829776847292716\\
31.93	0.00829782305081369\\
31.94	0.00829787764989848\\
31.95	0.00829793227019177\\
31.96	0.00829798691170383\\
31.97	0.00829804157444491\\
31.98	0.00829809625842529\\
31.99	0.00829815096365526\\
32	0.00829820569014508\\
32.01	0.00829826043790506\\
32.02	0.0082983152069455\\
32.03	0.0082983699972767\\
32.04	0.00829842480890899\\
32.05	0.00829847964185267\\
32.06	0.00829853449611808\\
32.07	0.00829858937171556\\
32.08	0.00829864426865545\\
32.09	0.0082986991869481\\
32.1	0.00829875412660388\\
32.11	0.00829880908763313\\
32.12	0.00829886407004624\\
32.13	0.0082989190738536\\
32.14	0.00829897409906557\\
32.15	0.00829902914569257\\
32.16	0.00829908421374499\\
32.17	0.00829913930323324\\
32.18	0.00829919441416773\\
32.19	0.0082992495465589\\
32.2	0.00829930470041718\\
32.21	0.008299359875753\\
32.22	0.0082994150725768\\
32.23	0.00829947029089906\\
32.24	0.00829952553073021\\
32.25	0.00829958079208074\\
32.26	0.00829963607496111\\
32.27	0.00829969137938182\\
32.28	0.00829974670535334\\
32.29	0.00829980205288619\\
32.3	0.00829985742199086\\
32.31	0.00829991281267787\\
32.32	0.00829996822495774\\
32.33	0.008300023658841\\
32.34	0.00830007911433817\\
32.35	0.00830013459145982\\
32.36	0.00830019009021647\\
32.37	0.0083002456106187\\
32.38	0.00830030115267707\\
32.39	0.00830035671640214\\
32.4	0.00830041230180451\\
32.41	0.00830046790889476\\
32.42	0.00830052353768348\\
32.43	0.00830057918818127\\
32.44	0.00830063486039876\\
32.45	0.00830069055434654\\
32.46	0.00830074627003526\\
32.47	0.00830080200747554\\
32.48	0.00830085776667803\\
32.49	0.00830091354765337\\
32.5	0.00830096935041221\\
32.51	0.00830102517496523\\
32.52	0.00830108102132308\\
32.53	0.00830113688949646\\
32.54	0.00830119277949605\\
32.55	0.00830124869133254\\
32.56	0.00830130462501663\\
32.57	0.00830136058055903\\
32.58	0.00830141655797046\\
32.59	0.00830147255726164\\
32.6	0.00830152857844331\\
32.61	0.00830158462152619\\
32.62	0.00830164068652106\\
32.63	0.00830169677343865\\
32.64	0.00830175288228972\\
32.65	0.00830180901308506\\
32.66	0.00830186516583544\\
32.67	0.00830192134055164\\
32.68	0.00830197753724446\\
32.69	0.0083020337559247\\
32.7	0.00830208999660316\\
32.71	0.00830214625929068\\
32.72	0.00830220254399806\\
32.73	0.00830225885073614\\
32.74	0.00830231517951577\\
32.75	0.0083023715303478\\
32.76	0.00830242790324306\\
32.77	0.00830248429821244\\
32.78	0.0083025407152668\\
32.79	0.00830259715441702\\
32.8	0.00830265361567398\\
32.81	0.0083027100990486\\
32.82	0.00830276660455175\\
32.83	0.00830282313219436\\
32.84	0.00830287968198735\\
32.85	0.00830293625394164\\
32.86	0.00830299284806816\\
32.87	0.00830304946437786\\
32.88	0.00830310610288169\\
32.89	0.0083031627635906\\
32.9	0.00830321944651557\\
32.91	0.00830327615166756\\
32.92	0.00830333287905755\\
32.93	0.00830338962869654\\
32.94	0.00830344640059552\\
32.95	0.0083035031947655\\
32.96	0.0083035600112175\\
32.97	0.00830361684996253\\
32.98	0.00830367371101162\\
32.99	0.00830373059437581\\
33	0.00830378750006614\\
33.01	0.00830384442809367\\
33.02	0.00830390137846947\\
33.03	0.00830395835120459\\
33.04	0.00830401534631012\\
33.05	0.00830407236379714\\
33.06	0.00830412940367675\\
33.07	0.00830418646596004\\
33.08	0.00830424355065813\\
33.09	0.00830430065778214\\
33.1	0.00830435778734318\\
33.11	0.00830441493935239\\
33.12	0.00830447211382092\\
33.13	0.00830452931075992\\
33.14	0.00830458653018054\\
33.15	0.00830464377209395\\
33.16	0.00830470103651132\\
33.17	0.00830475832344383\\
33.18	0.00830481563290269\\
33.19	0.00830487296489908\\
33.2	0.00830493031944421\\
33.21	0.0083049876965493\\
33.22	0.00830504509622557\\
33.23	0.00830510251848425\\
33.24	0.00830515996333658\\
33.25	0.00830521743079381\\
33.26	0.00830527492086719\\
33.27	0.008305332433568\\
33.28	0.00830538996890749\\
33.29	0.00830544752689696\\
33.3	0.00830550510754769\\
33.31	0.00830556271087098\\
33.32	0.00830562033687813\\
33.33	0.00830567798558045\\
33.34	0.00830573565698928\\
33.35	0.00830579335111594\\
33.36	0.00830585106797177\\
33.37	0.00830590880756811\\
33.38	0.00830596656991632\\
33.39	0.00830602435502777\\
33.4	0.00830608216291382\\
33.41	0.00830613999358586\\
33.42	0.00830619784705527\\
33.43	0.00830625572333346\\
33.44	0.00830631362243182\\
33.45	0.00830637154436176\\
33.46	0.00830642948913473\\
33.47	0.00830648745676213\\
33.48	0.00830654544725542\\
33.49	0.00830660346062603\\
33.5	0.00830666149688544\\
33.51	0.00830671955604508\\
33.52	0.00830677763811645\\
33.53	0.00830683574311103\\
33.54	0.00830689387104029\\
33.55	0.00830695202191574\\
33.56	0.00830701019574889\\
33.57	0.00830706839255125\\
33.58	0.00830712661233434\\
33.59	0.0083071848551097\\
33.6	0.00830724312088886\\
33.61	0.00830730140968338\\
33.62	0.00830735972150481\\
33.63	0.00830741805636471\\
33.64	0.00830747641427468\\
33.65	0.00830753479524627\\
33.66	0.0083075931992911\\
33.67	0.00830765162642075\\
33.68	0.00830771007664684\\
33.69	0.00830776854998098\\
33.7	0.0083078270464348\\
33.71	0.00830788556601993\\
33.72	0.00830794410874802\\
33.73	0.00830800267463072\\
33.74	0.00830806126367968\\
33.75	0.00830811987590659\\
33.76	0.00830817851132311\\
33.77	0.00830823716994094\\
33.78	0.00830829585177176\\
33.79	0.00830835455682728\\
33.8	0.00830841328511922\\
33.81	0.00830847203665929\\
33.82	0.00830853081145922\\
33.83	0.00830858960953077\\
33.84	0.00830864843088565\\
33.85	0.00830870727553565\\
33.86	0.00830876614349252\\
33.87	0.00830882503476803\\
33.88	0.00830888394937397\\
33.89	0.00830894288732213\\
33.9	0.0083090018486243\\
33.91	0.0083090608332923\\
33.92	0.00830911984133795\\
33.93	0.00830917887277306\\
33.94	0.00830923792760948\\
33.95	0.00830929700585906\\
33.96	0.00830935610753363\\
33.97	0.00830941523264506\\
33.98	0.00830947438120523\\
33.99	0.00830953355322602\\
34	0.0083095927487193\\
34.01	0.00830965196769699\\
34.02	0.00830971121017098\\
34.03	0.00830977047615319\\
34.04	0.00830982976565555\\
34.05	0.00830988907868998\\
34.06	0.00830994841526842\\
34.07	0.00831000777540284\\
34.08	0.00831006715910519\\
34.09	0.00831012656638743\\
34.1	0.00831018599726155\\
34.11	0.00831024545173953\\
34.12	0.00831030492983336\\
34.13	0.00831036443155506\\
34.14	0.00831042395691664\\
34.15	0.0083104835059301\\
34.16	0.0083105430786075\\
34.17	0.00831060267496087\\
34.18	0.00831066229500225\\
34.19	0.00831072193874372\\
34.2	0.00831078160619732\\
34.21	0.00831084129737516\\
34.22	0.0083109010122893\\
34.23	0.00831096075095184\\
34.24	0.0083110205133749\\
34.25	0.00831108029957057\\
34.26	0.00831114010955099\\
34.27	0.00831119994332828\\
34.28	0.00831125980091459\\
34.29	0.00831131968232207\\
34.3	0.00831137958756287\\
34.31	0.00831143951664916\\
34.32	0.00831149946959312\\
34.33	0.00831155944640694\\
34.34	0.00831161944710282\\
34.35	0.00831167947169295\\
34.36	0.00831173952018956\\
34.37	0.00831179959260486\\
34.38	0.00831185968895109\\
34.39	0.00831191980924049\\
34.4	0.00831197995348532\\
34.41	0.00831204012169783\\
34.42	0.00831210031389029\\
34.43	0.00831216053007498\\
34.44	0.0083122207702642\\
34.45	0.00831228103447023\\
34.46	0.00831234132270539\\
34.47	0.00831240163498199\\
34.48	0.00831246197131235\\
34.49	0.00831252233170883\\
34.5	0.00831258271618374\\
34.51	0.00831264312474946\\
34.52	0.00831270355741835\\
34.53	0.00831276401420277\\
34.54	0.00831282449511511\\
34.55	0.00831288500016777\\
34.56	0.00831294552937313\\
34.57	0.00831300608274362\\
34.58	0.00831306666029165\\
34.59	0.00831312726202965\\
34.6	0.00831318788797007\\
34.61	0.00831324853812535\\
34.62	0.00831330921250794\\
34.63	0.00831336991113031\\
34.64	0.00831343063400495\\
34.65	0.00831349138114434\\
34.66	0.00831355215256096\\
34.67	0.00831361294826734\\
34.68	0.00831367376827598\\
34.69	0.00831373461259941\\
34.7	0.00831379548125016\\
34.71	0.00831385637424078\\
34.72	0.00831391729158382\\
34.73	0.00831397823329184\\
34.74	0.00831403919937741\\
34.75	0.00831410018985312\\
34.76	0.00831416120473156\\
34.77	0.00831422224402533\\
34.78	0.00831428330774704\\
34.79	0.00831434439590931\\
34.8	0.00831440550852478\\
34.81	0.00831446664560607\\
34.82	0.00831452780716586\\
34.83	0.00831458899321678\\
34.84	0.00831465020377151\\
34.85	0.00831471143884274\\
34.86	0.00831477269844315\\
34.87	0.00831483398258543\\
34.88	0.00831489529128229\\
34.89	0.00831495662454646\\
34.9	0.00831501798239067\\
34.91	0.00831507936482764\\
34.92	0.00831514077187013\\
34.93	0.0083152022035309\\
34.94	0.0083152636598227\\
34.95	0.00831532514075833\\
34.96	0.00831538664635056\\
34.97	0.00831544817661219\\
34.98	0.00831550973155603\\
34.99	0.00831557131119489\\
35	0.0083156329155416\\
35.01	0.00831569454460901\\
35.02	0.00831575619840994\\
35.03	0.00831581787695727\\
35.04	0.00831587958026385\\
35.05	0.00831594130834256\\
35.06	0.00831600306120629\\
35.07	0.00831606483886793\\
35.08	0.00831612664134039\\
35.09	0.00831618846863658\\
35.1	0.00831625032076943\\
35.11	0.00831631219775189\\
35.12	0.00831637409959688\\
35.13	0.00831643602631737\\
35.14	0.00831649797792633\\
35.15	0.00831655995443672\\
35.16	0.00831662195586154\\
35.17	0.00831668398221379\\
35.18	0.00831674603350646\\
35.19	0.00831680810975258\\
35.2	0.00831687021096517\\
35.21	0.00831693233715727\\
35.22	0.00831699448834193\\
35.23	0.00831705666453219\\
35.24	0.00831711886574113\\
35.25	0.00831718109198184\\
35.26	0.00831724334326738\\
35.27	0.00831730561961087\\
35.28	0.0083173679210254\\
35.29	0.0083174302475241\\
35.3	0.0083174925991201\\
35.31	0.00831755497582653\\
35.32	0.00831761737765655\\
35.33	0.0083176798046233\\
35.34	0.00831774225673997\\
35.35	0.00831780473401973\\
35.36	0.00831786723647576\\
35.37	0.00831792976412128\\
35.38	0.00831799231696949\\
35.39	0.0083180548950336\\
35.4	0.00831811749832687\\
35.41	0.00831818012686251\\
35.42	0.0083182427806538\\
35.43	0.00831830545971398\\
35.44	0.00831836816405633\\
35.45	0.00831843089369414\\
35.46	0.00831849364864069\\
35.47	0.0083185564289093\\
35.48	0.00831861923451327\\
35.49	0.00831868206546593\\
35.5	0.00831874492178062\\
35.51	0.00831880780347068\\
35.52	0.00831887071054947\\
35.53	0.00831893364303034\\
35.54	0.00831899660092669\\
35.55	0.00831905958425191\\
35.56	0.00831912259301937\\
35.57	0.00831918562724251\\
35.58	0.00831924868693472\\
35.59	0.00831931177210946\\
35.6	0.00831937488278014\\
35.61	0.00831943801896024\\
35.62	0.0083195011806632\\
35.63	0.0083195643679025\\
35.64	0.00831962758069162\\
35.65	0.00831969081904406\\
35.66	0.00831975408297332\\
35.67	0.00831981737249291\\
35.68	0.00831988068761637\\
35.69	0.00831994402835722\\
35.7	0.00832000739472901\\
35.71	0.0083200707867453\\
35.72	0.00832013420441967\\
35.73	0.00832019764776568\\
35.74	0.00832026111679693\\
35.75	0.00832032461152701\\
35.76	0.00832038813196955\\
35.77	0.00832045167813815\\
35.78	0.00832051525004646\\
35.79	0.00832057884770812\\
35.8	0.00832064247113677\\
35.81	0.0083207061203461\\
35.82	0.00832076979534976\\
35.83	0.00832083349616146\\
35.84	0.00832089722279488\\
35.85	0.00832096097526373\\
35.86	0.00832102475358174\\
35.87	0.00832108855776264\\
35.88	0.00832115238782017\\
35.89	0.00832121624376807\\
35.9	0.00832128012562011\\
35.91	0.00832134403339006\\
35.92	0.00832140796709173\\
35.93	0.00832147192673888\\
35.94	0.00832153591234534\\
35.95	0.00832159992392492\\
35.96	0.00832166396149146\\
35.97	0.00832172802505878\\
35.98	0.00832179211464075\\
35.99	0.00832185623025122\\
36	0.00832192037190406\\
36.01	0.00832198453961317\\
36.02	0.00832204873339243\\
36.03	0.00832211295325575\\
36.04	0.00832217719921706\\
36.05	0.00832224147129027\\
36.06	0.00832230576948933\\
36.07	0.00832237009382819\\
36.08	0.00832243444432081\\
36.09	0.00832249882098117\\
36.1	0.00832256322382324\\
36.11	0.00832262765286103\\
36.12	0.00832269210810853\\
36.13	0.00832275658957978\\
36.14	0.0083228210972888\\
36.15	0.00832288563124962\\
36.16	0.0083229501914763\\
36.17	0.00832301477798291\\
36.18	0.00832307939078352\\
36.19	0.0083231440298922\\
36.2	0.00832320869532307\\
36.21	0.00832327338709022\\
36.22	0.00832333810520778\\
36.23	0.00832340284968988\\
36.24	0.00832346762055066\\
36.25	0.00832353241780427\\
36.26	0.00832359724146488\\
36.27	0.00832366209154667\\
36.28	0.00832372696806382\\
36.29	0.00832379187103052\\
36.3	0.00832385680046101\\
36.31	0.00832392175636949\\
36.32	0.00832398673877019\\
36.33	0.00832405174767737\\
36.34	0.00832411678310528\\
36.35	0.00832418184506819\\
36.36	0.00832424693358037\\
36.37	0.00832431204865612\\
36.38	0.00832437719030975\\
36.39	0.00832444235855555\\
36.4	0.00832450755340786\\
36.41	0.00832457277488103\\
36.42	0.00832463802298939\\
36.43	0.00832470329774731\\
36.44	0.00832476859916915\\
36.45	0.00832483392726931\\
36.46	0.00832489928206218\\
36.47	0.00832496466356216\\
36.48	0.00832503007178367\\
36.49	0.00832509550674115\\
36.5	0.00832516096844903\\
36.51	0.00832522645692177\\
36.52	0.00832529197217384\\
36.53	0.00832535751421971\\
36.54	0.00832542308307387\\
36.55	0.00832548867875082\\
36.56	0.00832555430126508\\
36.57	0.00832561995063116\\
36.58	0.00832568562686361\\
36.59	0.00832575132997697\\
36.6	0.00832581705998581\\
36.61	0.00832588281690468\\
36.62	0.00832594860074819\\
36.63	0.00832601441153091\\
36.64	0.00832608024926746\\
36.65	0.00832614611397247\\
36.66	0.00832621200566055\\
36.67	0.00832627792434636\\
36.68	0.00832634387004454\\
36.69	0.00832640984276977\\
36.7	0.00832647584253672\\
36.71	0.00832654186936008\\
36.72	0.00832660792325456\\
36.73	0.00832667400423487\\
36.74	0.00832674011231573\\
36.75	0.00832680624751189\\
36.76	0.0083268724098381\\
36.77	0.00832693859930912\\
36.78	0.00832700481593973\\
36.79	0.0083270710597447\\
36.8	0.00832713733073885\\
36.81	0.00832720362893697\\
36.82	0.0083272699543539\\
36.83	0.00832733630700448\\
36.84	0.00832740268690354\\
36.85	0.00832746909406596\\
36.86	0.00832753552850658\\
36.87	0.00832760199024032\\
36.88	0.00832766847928205\\
36.89	0.0083277349956467\\
36.9	0.00832780153934917\\
36.91	0.00832786811040441\\
36.92	0.00832793470882736\\
36.93	0.00832800133463297\\
36.94	0.00832806798783622\\
36.95	0.00832813466845209\\
36.96	0.00832820137649557\\
36.97	0.00832826811198167\\
36.98	0.00832833487492541\\
36.99	0.00832840166534182\\
37	0.00832846848324595\\
37.01	0.00832853532865284\\
37.02	0.00832860220157758\\
37.03	0.00832866910203523\\
37.04	0.00832873603004091\\
37.05	0.0083288029856097\\
37.06	0.00832886996875673\\
37.07	0.00832893697949714\\
37.08	0.00832900401784605\\
37.09	0.00832907108381863\\
37.1	0.00832913817743006\\
37.11	0.00832920529869551\\
37.12	0.00832927244763016\\
37.13	0.00832933962424924\\
37.14	0.00832940682856795\\
37.15	0.00832947406060153\\
37.16	0.00832954132036523\\
37.17	0.00832960860787429\\
37.18	0.00832967592314399\\
37.19	0.00832974326618962\\
37.2	0.00832981063702645\\
37.21	0.00832987803566981\\
37.22	0.00832994546213502\\
37.23	0.00833001291643739\\
37.24	0.00833008039859229\\
37.25	0.00833014790861506\\
37.26	0.00833021544652108\\
37.27	0.00833028301232574\\
37.28	0.00833035060604442\\
37.29	0.00833041822769254\\
37.3	0.00833048587728552\\
37.31	0.00833055355483879\\
37.32	0.0083306212603678\\
37.33	0.00833068899388802\\
37.34	0.0083307567554149\\
37.35	0.00833082454496396\\
37.36	0.00833089236255067\\
37.37	0.00833096020819055\\
37.38	0.00833102808189913\\
37.39	0.00833109598369194\\
37.4	0.00833116391358453\\
37.41	0.00833123187159248\\
37.42	0.00833129985773135\\
37.43	0.00833136787201674\\
37.44	0.00833143591446424\\
37.45	0.00833150398508947\\
37.46	0.00833157208390807\\
37.47	0.00833164021093567\\
37.48	0.00833170836618793\\
37.49	0.00833177654968051\\
37.5	0.0083318447614291\\
37.51	0.00833191300144939\\
37.52	0.0083319812697571\\
37.53	0.00833204956636792\\
37.54	0.00833211789129762\\
37.55	0.00833218624456192\\
37.56	0.00833225462617658\\
37.57	0.00833232303615739\\
37.58	0.00833239147452013\\
37.59	0.0083324599412806\\
37.6	0.0083325284364546\\
37.61	0.00833259696005797\\
37.62	0.00833266551210655\\
37.63	0.00833273409261618\\
37.64	0.00833280270160274\\
37.65	0.00833287133908209\\
37.66	0.00833294000507015\\
37.67	0.0083330086995828\\
37.68	0.00833307742263598\\
37.69	0.00833314617424561\\
37.7	0.00833321495442764\\
37.71	0.00833328376319802\\
37.72	0.00833335260057274\\
37.73	0.00833342146656777\\
37.74	0.00833349036119912\\
37.75	0.0083335592844828\\
37.76	0.00833362823643484\\
37.77	0.00833369721707127\\
37.78	0.00833376622640816\\
37.79	0.00833383526446156\\
37.8	0.00833390433124756\\
37.81	0.00833397342678225\\
37.82	0.00833404255108174\\
37.83	0.00833411170416215\\
37.84	0.00833418088603962\\
37.85	0.0083342500967303\\
37.86	0.00833431933625034\\
37.87	0.00833438860461592\\
37.88	0.00833445790184324\\
37.89	0.0083345272279485\\
37.9	0.00833459658294791\\
37.91	0.0083346659668577\\
37.92	0.00833473537969412\\
37.93	0.00833480482147343\\
37.94	0.00833487429221189\\
37.95	0.0083349437919258\\
37.96	0.00833501332063144\\
37.97	0.00833508287834515\\
37.98	0.00833515246508324\\
37.99	0.00833522208086205\\
38	0.00833529172569795\\
38.01	0.00833536139960729\\
38.02	0.00833543110260645\\
38.03	0.00833550083471185\\
38.04	0.00833557059593988\\
38.05	0.00833564038630697\\
38.06	0.00833571020582956\\
38.07	0.00833578005452409\\
38.08	0.00833584993240705\\
38.09	0.0083359198394949\\
38.1	0.00833598977580413\\
38.11	0.00833605974135127\\
38.12	0.00833612973615283\\
38.13	0.00833619976022534\\
38.14	0.00833626981358536\\
38.15	0.00833633989624944\\
38.16	0.00833641000823418\\
38.17	0.00833648014955615\\
38.18	0.00833655032023198\\
38.19	0.00833662052027827\\
38.2	0.00833669074971166\\
38.21	0.0083367610085488\\
38.22	0.00833683129680635\\
38.23	0.008336901614501\\
38.24	0.00833697196164942\\
38.25	0.00833704233826834\\
38.26	0.00833711274437446\\
38.27	0.00833718317998452\\
38.28	0.00833725364511527\\
38.29	0.00833732413978348\\
38.3	0.00833739466400592\\
38.31	0.00833746521779937\\
38.32	0.00833753580118065\\
38.33	0.00833760641416658\\
38.34	0.00833767705677399\\
38.35	0.00833774772901973\\
38.36	0.00833781843092066\\
38.37	0.00833788916249365\\
38.38	0.00833795992375562\\
38.39	0.00833803071472345\\
38.4	0.00833810153541406\\
38.41	0.0083381723858444\\
38.42	0.00833824326603141\\
38.43	0.00833831417599206\\
38.44	0.00833838511574333\\
38.45	0.0083384560853022\\
38.46	0.0083385270846857\\
38.47	0.00833859811391082\\
38.48	0.00833866917299463\\
38.49	0.00833874026195417\\
38.5	0.00833881138080649\\
38.51	0.00833888252956869\\
38.52	0.00833895370825786\\
38.53	0.0083390249168911\\
38.54	0.00833909615548555\\
38.55	0.00833916742405834\\
38.56	0.00833923872262663\\
38.57	0.00833931005120758\\
38.58	0.00833938140981838\\
38.59	0.00833945279847623\\
38.6	0.00833952421719833\\
38.61	0.00833959566600192\\
38.62	0.00833966714490424\\
38.63	0.00833973865392254\\
38.64	0.0083398101930741\\
38.65	0.00833988176237621\\
38.66	0.00833995336184616\\
38.67	0.00834002499150127\\
38.68	0.00834009665135887\\
38.69	0.00834016834143631\\
38.7	0.00834024006175095\\
38.71	0.00834031181232017\\
38.72	0.00834038359316135\\
38.73	0.0083404554042919\\
38.74	0.00834052724572925\\
38.75	0.00834059911749082\\
38.76	0.00834067101959407\\
38.77	0.00834074295205647\\
38.78	0.00834081491489549\\
38.79	0.00834088690812863\\
38.8	0.0083409589317734\\
38.81	0.00834103098584732\\
38.82	0.00834110307036795\\
38.83	0.00834117518535283\\
38.84	0.00834124733081952\\
38.85	0.00834131950678564\\
38.86	0.00834139171326875\\
38.87	0.0083414639502865\\
38.88	0.0083415362178565\\
38.89	0.00834160851599641\\
38.9	0.00834168084472388\\
38.91	0.00834175320405659\\
38.92	0.00834182559401225\\
38.93	0.00834189801460853\\
38.94	0.00834197046586319\\
38.95	0.00834204294779395\\
38.96	0.00834211546041856\\
38.97	0.0083421880037548\\
38.98	0.00834226057782043\\
38.99	0.00834233318263328\\
39	0.00834240581821115\\
39.01	0.00834247848457186\\
39.02	0.00834255118173327\\
39.03	0.00834262390971322\\
39.04	0.00834269666852961\\
39.05	0.00834276945820032\\
39.06	0.00834284227874325\\
39.07	0.00834291513017633\\
39.08	0.00834298801251749\\
39.09	0.00834306092578469\\
39.1	0.00834313386999589\\
39.11	0.00834320684516908\\
39.12	0.00834327985132225\\
39.13	0.00834335288847343\\
39.14	0.00834342595664063\\
39.15	0.0083434990558419\\
39.16	0.00834357218609532\\
39.17	0.00834364534741894\\
39.18	0.00834371853983087\\
39.19	0.0083437917633492\\
39.2	0.00834386501799207\\
39.21	0.00834393830377761\\
39.22	0.00834401162072398\\
39.23	0.00834408496884933\\
39.24	0.00834415834817187\\
39.25	0.00834423175870979\\
39.26	0.0083443052004813\\
39.27	0.00834437867350464\\
39.28	0.00834445217779807\\
39.29	0.00834452571337982\\
39.3	0.0083445992802682\\
39.31	0.00834467287848149\\
39.32	0.008344746508038\\
39.33	0.00834482016895606\\
39.34	0.00834489386125402\\
39.35	0.00834496758495022\\
39.36	0.00834504134006304\\
39.37	0.00834511512661088\\
39.38	0.00834518894461212\\
39.39	0.00834526279408521\\
39.4	0.00834533667504857\\
39.41	0.00834541058752064\\
39.42	0.00834548453151992\\
39.43	0.00834555850706487\\
39.44	0.00834563251417399\\
39.45	0.0083457065528658\\
39.46	0.00834578062315884\\
39.47	0.00834585472507164\\
39.48	0.00834592885862277\\
39.49	0.00834600302383082\\
39.5	0.00834607722071436\\
39.51	0.00834615144929202\\
39.52	0.00834622570958243\\
39.53	0.00834630000160422\\
39.54	0.00834637432537604\\
39.55	0.00834644868091659\\
39.56	0.00834652306824454\\
39.57	0.0083465974873786\\
39.58	0.0083466719383375\\
39.59	0.00834674642113996\\
39.6	0.00834682093580475\\
39.61	0.00834689548235064\\
39.62	0.00834697006079641\\
39.63	0.00834704467116086\\
39.64	0.00834711931346281\\
39.65	0.0083471939877211\\
39.66	0.00834726869395457\\
39.67	0.00834734343218209\\
39.68	0.00834741820242254\\
39.69	0.00834749300469482\\
39.7	0.00834756783901784\\
39.71	0.00834764270541054\\
39.72	0.00834771760389186\\
39.73	0.00834779253448076\\
39.74	0.00834786749719622\\
39.75	0.00834794249205723\\
39.76	0.00834801751908281\\
39.77	0.00834809257829197\\
39.78	0.00834816766970378\\
39.79	0.00834824279333727\\
39.8	0.00834831794921153\\
39.81	0.00834839313734564\\
39.82	0.00834846835775872\\
39.83	0.00834854361046989\\
39.84	0.00834861889549828\\
39.85	0.00834869421286305\\
39.86	0.00834876956258338\\
39.87	0.00834884494467845\\
39.88	0.00834892035916746\\
39.89	0.00834899580606964\\
39.9	0.00834907128540421\\
39.91	0.00834914679719044\\
39.92	0.00834922234144759\\
39.93	0.00834929791819494\\
39.94	0.0083493735274518\\
39.95	0.00834944916923748\\
39.96	0.00834952484357131\\
39.97	0.00834960055047265\\
39.98	0.00834967628996086\\
39.99	0.00834975206205533\\
40	0.00834982786677544\\
40.01	0.00834990370414062\\
};
\addplot [color=red,solid,forget plot]
  table[row sep=crcr]{%
40.01	0.00834990370414062\\
40.02	0.00834997957417028\\
40.03	0.00835005547688389\\
40.04	0.0083501314123009\\
40.05	0.00835020738044079\\
40.06	0.00835028338132306\\
40.07	0.00835035941496721\\
40.08	0.00835043548139278\\
40.09	0.0083505115806193\\
40.1	0.00835058771266635\\
40.11	0.00835066387755349\\
40.12	0.00835074007530031\\
40.13	0.00835081630592643\\
40.14	0.00835089256945146\\
40.15	0.00835096886589506\\
40.16	0.00835104519527687\\
40.17	0.00835112155761657\\
40.18	0.00835119795293385\\
40.19	0.00835127438124842\\
40.2	0.00835135084258\\
40.21	0.00835142733694832\\
40.22	0.00835150386437313\\
40.23	0.00835158042487422\\
40.24	0.00835165701847137\\
40.25	0.00835173364518438\\
40.26	0.00835181030503306\\
40.27	0.00835188699803726\\
40.28	0.00835196372421682\\
40.29	0.00835204048359163\\
40.3	0.00835211727618155\\
40.31	0.00835219410200648\\
40.32	0.00835227096108635\\
40.33	0.00835234785344109\\
40.34	0.00835242477909065\\
40.35	0.00835250173805499\\
40.36	0.00835257873035409\\
40.37	0.00835265575600796\\
40.38	0.00835273281503659\\
40.39	0.00835280990746003\\
40.4	0.00835288703329833\\
40.41	0.00835296419257154\\
40.42	0.00835304138529974\\
40.43	0.00835311861150302\\
40.44	0.0083531958712015\\
40.45	0.00835327316441531\\
40.46	0.00835335049116458\\
40.47	0.00835342785146948\\
40.48	0.00835350524535019\\
40.49	0.00835358267282689\\
40.5	0.00835366013391979\\
40.51	0.00835373762864912\\
40.52	0.00835381515703513\\
40.53	0.00835389271909805\\
40.54	0.00835397031485817\\
40.55	0.00835404794433579\\
40.56	0.00835412560755119\\
40.57	0.0083542033045247\\
40.58	0.00835428103527667\\
40.59	0.00835435879982745\\
40.6	0.00835443659819739\\
40.61	0.0083545144304069\\
40.62	0.00835459229647638\\
40.63	0.00835467019642623\\
40.64	0.0083547481302769\\
40.65	0.00835482609804884\\
40.66	0.00835490409976251\\
40.67	0.0083549821354384\\
40.68	0.00835506020509701\\
40.69	0.00835513830875886\\
40.7	0.00835521644644447\\
40.71	0.00835529461817438\\
40.72	0.00835537282396918\\
40.73	0.00835545106384944\\
40.74	0.00835552933783575\\
40.75	0.00835560764594873\\
40.76	0.00835568598820901\\
40.77	0.00835576436463723\\
40.78	0.00835584277525405\\
40.79	0.00835592122008016\\
40.8	0.00835599969913624\\
40.81	0.00835607821244301\\
40.82	0.00835615676002119\\
40.83	0.00835623534189152\\
40.84	0.00835631395807478\\
40.85	0.00835639260859172\\
40.86	0.00835647129346315\\
40.87	0.00835655001270987\\
40.88	0.0083566287663527\\
40.89	0.00835670755441249\\
40.9	0.00835678637691009\\
40.91	0.00835686523386639\\
40.92	0.00835694412530226\\
40.93	0.00835702305123861\\
40.94	0.00835710201169637\\
40.95	0.00835718100669648\\
40.96	0.00835726003625989\\
40.97	0.00835733910040758\\
40.98	0.00835741819916053\\
40.99	0.00835749733253976\\
41	0.00835757650056627\\
41.01	0.00835765570326112\\
41.02	0.00835773494064536\\
41.03	0.00835781421274005\\
41.04	0.0083578935195663\\
41.05	0.0083579728611452\\
41.06	0.00835805223749788\\
41.07	0.00835813164864547\\
41.08	0.00835821109460914\\
41.09	0.00835829057541005\\
41.1	0.00835837009106941\\
41.11	0.00835844964160841\\
41.12	0.00835852922704827\\
41.13	0.00835860884741026\\
41.14	0.00835868850271561\\
41.15	0.00835876819298561\\
41.16	0.00835884791824155\\
41.17	0.00835892767850474\\
41.18	0.00835900747379651\\
41.19	0.00835908730413821\\
41.2	0.00835916716955121\\
41.21	0.00835924707005687\\
41.22	0.00835932700567661\\
41.23	0.00835940697643183\\
41.24	0.00835948698234399\\
41.25	0.00835956702343452\\
41.26	0.0083596470997249\\
41.27	0.00835972721123662\\
41.28	0.00835980735799118\\
41.29	0.00835988754001013\\
41.3	0.00835996775731499\\
41.31	0.00836004800992733\\
41.32	0.00836012829786874\\
41.33	0.00836020862116083\\
41.34	0.0083602889798252\\
41.35	0.0083603693738835\\
41.36	0.00836044980335739\\
41.37	0.00836053026826854\\
41.38	0.00836061076863867\\
41.39	0.00836069130448948\\
41.4	0.00836077187584272\\
41.41	0.00836085248272014\\
41.42	0.00836093312514353\\
41.43	0.00836101380313468\\
41.44	0.00836109451671541\\
41.45	0.00836117526590757\\
41.46	0.00836125605073301\\
41.47	0.00836133687121363\\
41.48	0.00836141772737134\\
41.49	0.00836149861922804\\
41.5	0.0083615795468057\\
41.51	0.0083616605101263\\
41.52	0.00836174150921181\\
41.53	0.00836182254408427\\
41.54	0.00836190361476571\\
41.55	0.00836198472127819\\
41.56	0.00836206586364381\\
41.57	0.00836214704188468\\
41.58	0.00836222825602292\\
41.59	0.00836230950608072\\
41.6	0.00836239079208024\\
41.61	0.0083624721140437\\
41.62	0.00836255347199334\\
41.63	0.00836263486595142\\
41.64	0.00836271629594023\\
41.65	0.00836279776198209\\
41.66	0.00836287926409933\\
41.67	0.00836296080231434\\
41.68	0.00836304237664951\\
41.69	0.00836312398712727\\
41.7	0.00836320563377008\\
41.71	0.00836328731660041\\
41.72	0.0083633690356408\\
41.73	0.00836345079091379\\
41.74	0.00836353258244194\\
41.75	0.00836361441024788\\
41.76	0.00836369627435425\\
41.77	0.00836377817478371\\
41.78	0.00836386011155898\\
41.79	0.0083639420847028\\
41.8	0.00836402409423795\\
41.81	0.00836410614018723\\
41.82	0.00836418822257349\\
41.83	0.00836427034141962\\
41.84	0.00836435249674853\\
41.85	0.00836443468858319\\
41.86	0.00836451691694659\\
41.87	0.00836459918186178\\
41.88	0.00836468148335181\\
41.89	0.00836476382143982\\
41.9	0.00836484619614897\\
41.91	0.00836492860750245\\
41.92	0.00836501105552351\\
41.93	0.00836509354023545\\
41.94	0.00836517606166159\\
41.95	0.00836525861982533\\
41.96	0.00836534121475008\\
41.97	0.00836542384645933\\
41.98	0.00836550651497661\\
41.99	0.0083655892203255\\
42	0.00836567196252961\\
42.01	0.00836575474161264\\
42.02	0.00836583755759832\\
42.03	0.00836592041051044\\
42.04	0.00836600330037285\\
42.05	0.00836608622720945\\
42.06	0.00836616919104421\\
42.07	0.00836625219190114\\
42.08	0.00836633522980433\\
42.09	0.00836641830477794\\
42.1	0.00836650141684615\\
42.11	0.00836658456603327\\
42.12	0.00836666775236363\\
42.13	0.00836675097586163\\
42.14	0.00836683423655176\\
42.15	0.00836691753445857\\
42.16	0.00836700086960669\\
42.17	0.00836708424202083\\
42.18	0.00836716765172574\\
42.19	0.0083672510987463\\
42.2	0.00836733458310743\\
42.21	0.00836741810483414\\
42.22	0.00836750166395155\\
42.23	0.00836758526048482\\
42.24	0.00836766889445925\\
42.25	0.00836775256590017\\
42.26	0.00836783627483305\\
42.27	0.00836792002128343\\
42.28	0.00836800380527695\\
42.29	0.00836808762683934\\
42.3	0.00836817148599644\\
42.31	0.00836825538277417\\
42.32	0.0083683393171986\\
42.33	0.00836842328929585\\
42.34	0.00836850729909217\\
42.35	0.00836859134661393\\
42.36	0.00836867543188761\\
42.37	0.00836875955493979\\
42.38	0.00836884371579718\\
42.39	0.0083689279144866\\
42.4	0.00836901215103501\\
42.41	0.00836909642546945\\
42.42	0.00836918073781715\\
42.43	0.00836926508810542\\
42.44	0.00836934947636171\\
42.45	0.00836943390261363\\
42.46	0.00836951836688889\\
42.47	0.00836960286921537\\
42.48	0.00836968740962107\\
42.49	0.00836977198813414\\
42.5	0.00836985660478288\\
42.51	0.00836994125959575\\
42.52	0.00837002595260134\\
42.53	0.0083701106838284\\
42.54	0.00837019545330584\\
42.55	0.00837028026106274\\
42.56	0.00837036510712833\\
42.57	0.00837044999153199\\
42.58	0.0083705349143033\\
42.59	0.00837061987547199\\
42.6	0.00837070487506795\\
42.61	0.00837078991312128\\
42.62	0.00837087498966221\\
42.63	0.00837096010472118\\
42.64	0.0083710452583288\\
42.65	0.00837113045051586\\
42.66	0.00837121568131335\\
42.67	0.00837130095075243\\
42.68	0.00837138625886443\\
42.69	0.00837147160568092\\
42.7	0.00837155699123361\\
42.71	0.00837164241555444\\
42.72	0.00837172787867551\\
42.73	0.00837181338062914\\
42.74	0.00837189892144783\\
42.75	0.00837198450116428\\
42.76	0.0083720701198114\\
42.77	0.00837215577742226\\
42.78	0.00837224147403016\\
42.79	0.00837232720966859\\
42.8	0.00837241298437121\\
42.81	0.00837249879817191\\
42.82	0.00837258465110473\\
42.83	0.00837267054320393\\
42.84	0.00837275647450397\\
42.85	0.00837284244503944\\
42.86	0.00837292845484518\\
42.87	0.00837301450395616\\
42.88	0.00837310059240754\\
42.89	0.00837318672023466\\
42.9	0.008373272887473\\
42.91	0.00837335909415822\\
42.92	0.00837344534032613\\
42.93	0.00837353162601268\\
42.94	0.00837361795125396\\
42.95	0.00837370431608618\\
42.96	0.00837379072054568\\
42.97	0.0083738771646689\\
42.98	0.00837396364849238\\
42.99	0.00837405017205274\\
43	0.00837413673538668\\
43.01	0.00837422333853095\\
43.02	0.00837430998152233\\
43.03	0.00837439666439765\\
43.04	0.00837448338719372\\
43.05	0.00837457014994733\\
43.06	0.00837465695269527\\
43.07	0.00837474379547423\\
43.08	0.00837483067832083\\
43.09	0.00837491760127158\\
43.1	0.00837500456436287\\
43.11	0.00837509156763088\\
43.12	0.00837517861111162\\
43.13	0.00837526569484086\\
43.14	0.0083753528188541\\
43.15	0.00837543998318651\\
43.16	0.00837552718787292\\
43.17	0.00837561443294776\\
43.18	0.00837570171844502\\
43.19	0.0083757890443982\\
43.2	0.00837587641084023\\
43.21	0.00837596381780346\\
43.22	0.00837605126531956\\
43.23	0.00837613875341947\\
43.24	0.00837622628213336\\
43.25	0.00837631385149049\\
43.26	0.00837640146151921\\
43.27	0.00837648911224684\\
43.28	0.00837657680370064\\
43.29	0.0083766645359079\\
43.3	0.00837675230889594\\
43.31	0.0083768401226921\\
43.32	0.00837692797732375\\
43.33	0.00837701587281831\\
43.34	0.00837710380920318\\
43.35	0.00837719178650584\\
43.36	0.00837727980475376\\
43.37	0.00837736786397445\\
43.38	0.00837745596419546\\
43.39	0.00837754410544436\\
43.4	0.00837763228774872\\
43.41	0.00837772051113619\\
43.42	0.00837780877563442\\
43.43	0.00837789708127107\\
43.44	0.00837798542807387\\
43.45	0.00837807381607054\\
43.46	0.00837816224528885\\
43.47	0.00837825071575659\\
43.48	0.00837833922750158\\
43.49	0.00837842778055167\\
43.5	0.00837851637493474\\
43.51	0.00837860501067869\\
43.52	0.00837869368781146\\
43.53	0.008378782406361\\
43.54	0.00837887116635531\\
43.55	0.00837895996782241\\
43.56	0.00837904881079033\\
43.57	0.00837913769528717\\
43.58	0.00837922662134102\\
43.59	0.00837931558898001\\
43.6	0.00837940459823231\\
43.61	0.00837949364912611\\
43.62	0.00837958274168963\\
43.63	0.0083796718759511\\
43.64	0.00837976105193882\\
43.65	0.00837985026968108\\
43.66	0.00837993952920622\\
43.67	0.0083800288305426\\
43.68	0.00838011817371861\\
43.69	0.00838020755876268\\
43.7	0.00838029698570325\\
43.71	0.00838038645456879\\
43.72	0.00838047596538782\\
43.73	0.00838056551818888\\
43.74	0.00838065511300052\\
43.75	0.00838074474985135\\
43.76	0.00838083442876999\\
43.77	0.00838092414978508\\
43.78	0.00838101391292532\\
43.79	0.0083811037182194\\
43.8	0.00838119356569608\\
43.81	0.00838128345538412\\
43.82	0.00838137338731232\\
43.83	0.00838146336150952\\
43.84	0.00838155337800455\\
43.85	0.00838164343682632\\
43.86	0.00838173353800374\\
43.87	0.00838182368156575\\
43.88	0.00838191386754134\\
43.89	0.0083820040959595\\
43.9	0.00838209436684927\\
43.91	0.00838218468023971\\
43.92	0.00838227503615992\\
43.93	0.00838236543463902\\
43.94	0.00838245587570616\\
43.95	0.00838254635939053\\
43.96	0.00838263688572133\\
43.97	0.00838272745472781\\
43.98	0.00838281806643925\\
43.99	0.00838290872088493\\
44	0.0083829994180942\\
44.01	0.00838309015809641\\
44.02	0.00838318094092095\\
44.03	0.00838327176659725\\
44.04	0.00838336263515475\\
44.05	0.00838345354662295\\
44.06	0.00838354450103133\\
44.07	0.00838363549840946\\
44.08	0.00838372653878689\\
44.09	0.00838381762219323\\
44.1	0.00838390874865812\\
44.11	0.00838399991821119\\
44.12	0.00838409113088216\\
44.13	0.00838418238670074\\
44.14	0.00838427368569668\\
44.15	0.00838436502789977\\
44.16	0.00838445641333981\\
44.17	0.00838454784204664\\
44.18	0.00838463931405014\\
44.19	0.00838473082938022\\
44.2	0.0083848223880668\\
44.21	0.00838491399013984\\
44.22	0.00838500563562933\\
44.23	0.00838509732456531\\
44.24	0.00838518905697783\\
44.25	0.00838528083289697\\
44.26	0.00838537265235283\\
44.27	0.00838546451537558\\
44.28	0.00838555642199538\\
44.29	0.00838564837224244\\
44.3	0.00838574036614699\\
44.31	0.00838583240373931\\
44.32	0.00838592448504969\\
44.33	0.00838601661010845\\
44.34	0.00838610877894596\\
44.35	0.0083862009915926\\
44.36	0.00838629324807879\\
44.37	0.008386385548435\\
44.38	0.00838647789269167\\
44.39	0.00838657028087935\\
44.4	0.00838666271302856\\
44.41	0.00838675518916988\\
44.42	0.00838684770933391\\
44.43	0.00838694027355128\\
44.44	0.00838703288185266\\
44.45	0.00838712553426874\\
44.46	0.00838721823083025\\
44.47	0.00838731097156794\\
44.48	0.0083874037565126\\
44.49	0.00838749658569505\\
44.5	0.00838758945914614\\
44.51	0.00838768237689674\\
44.52	0.00838777533897775\\
44.53	0.00838786834542014\\
44.54	0.00838796139625485\\
44.55	0.0083880544915129\\
44.56	0.00838814763122532\\
44.57	0.00838824081542316\\
44.58	0.00838833404413753\\
44.59	0.00838842731739954\\
44.6	0.00838852063524035\\
44.61	0.00838861399769114\\
44.62	0.00838870740478314\\
44.63	0.00838880085654759\\
44.64	0.00838889435301575\\
44.65	0.00838898789421895\\
44.66	0.00838908148018853\\
44.67	0.00838917511095584\\
44.68	0.00838926878655228\\
44.69	0.0083893625070093\\
44.7	0.00838945627235834\\
44.71	0.00838955008263091\\
44.72	0.00838964393785852\\
44.73	0.00838973783807272\\
44.74	0.0083898317833051\\
44.75	0.00838992577358727\\
44.76	0.00839001980895087\\
44.77	0.00839011388942758\\
44.78	0.0083902080150491\\
44.79	0.00839030218584717\\
44.8	0.00839039640185355\\
44.81	0.00839049066310003\\
44.82	0.00839058496961845\\
44.83	0.00839067932144066\\
44.84	0.00839077371859854\\
44.85	0.00839086816112401\\
44.86	0.00839096264904902\\
44.87	0.00839105718240554\\
44.88	0.00839115176122559\\
44.89	0.00839124638554119\\
44.9	0.00839134105538441\\
44.91	0.00839143577078735\\
44.92	0.00839153053178215\\
44.93	0.00839162533840094\\
44.94	0.00839172019067592\\
44.95	0.0083918150886393\\
44.96	0.00839191003232334\\
44.97	0.0083920050217603\\
44.98	0.00839210005698249\\
44.99	0.00839219513802225\\
45	0.00839229026491193\\
45.01	0.00839238543768394\\
45.02	0.00839248065637068\\
45.03	0.00839257592100463\\
45.04	0.00839267123161824\\
45.05	0.00839276658824404\\
45.06	0.00839286199091457\\
45.07	0.00839295743966238\\
45.08	0.00839305293452008\\
45.09	0.0083931484755203\\
45.1	0.00839324406269568\\
45.11	0.00839333969607891\\
45.12	0.0083934353757027\\
45.13	0.00839353110159979\\
45.14	0.00839362687380295\\
45.15	0.00839372269234498\\
45.16	0.00839381855725869\\
45.17	0.00839391446857695\\
45.18	0.00839401042633262\\
45.19	0.00839410643055862\\
45.2	0.00839420248128789\\
45.21	0.00839429857855338\\
45.22	0.00839439472238808\\
45.23	0.00839449091282502\\
45.24	0.00839458714989724\\
45.25	0.00839468343363781\\
45.26	0.00839477976407983\\
45.27	0.00839487614125642\\
45.28	0.00839497256520074\\
45.29	0.00839506903594597\\
45.3	0.0083951655535253\\
45.31	0.00839526211797197\\
45.32	0.00839535872931924\\
45.33	0.00839545538760039\\
45.34	0.00839555209284873\\
45.35	0.00839564884509759\\
45.36	0.00839574564438033\\
45.37	0.00839584249073033\\
45.38	0.008395939384181\\
45.39	0.00839603632476578\\
45.4	0.00839613331251812\\
45.41	0.00839623034747151\\
45.42	0.00839632742965945\\
45.43	0.00839642455911547\\
45.44	0.00839652173587314\\
45.45	0.00839661895996601\\
45.46	0.0083967162314277\\
45.47	0.00839681355029183\\
45.48	0.00839691091659203\\
45.49	0.00839700833036199\\
45.5	0.0083971057916354\\
45.51	0.00839720330044597\\
45.52	0.00839730085682742\\
45.53	0.00839739846081352\\
45.54	0.00839749611243804\\
45.55	0.00839759381173479\\
45.56	0.00839769155873757\\
45.57	0.00839778935348023\\
45.58	0.00839788719599663\\
45.59	0.00839798508632064\\
45.6	0.00839808302448616\\
45.61	0.00839818101052711\\
45.62	0.00839827904447743\\
45.63	0.00839837712637106\\
45.64	0.00839847525624198\\
45.65	0.00839857343412417\\
45.66	0.00839867166005165\\
45.67	0.00839876993405843\\
45.68	0.00839886825617855\\
45.69	0.00839896662644607\\
45.7	0.00839906504489506\\
45.71	0.0083991635115596\\
45.72	0.0083992620264738\\
45.73	0.00839936058967177\\
45.74	0.00839945920118764\\
45.75	0.00839955786105554\\
45.76	0.00839965656930964\\
45.77	0.00839975532598411\\
45.78	0.00839985413111311\\
45.79	0.00839995298473085\\
45.8	0.00840005188687151\\
45.81	0.00840015083756933\\
45.82	0.00840024983685851\\
45.83	0.00840034888477329\\
45.84	0.00840044798134791\\
45.85	0.00840054712661662\\
45.86	0.00840064632061367\\
45.87	0.00840074556337333\\
45.88	0.00840084485492986\\
45.89	0.00840094419531754\\
45.9	0.00840104358457064\\
45.91	0.00840114302272347\\
45.92	0.00840124250981029\\
45.93	0.0084013420458654\\
45.94	0.0084014416309231\\
45.95	0.00840154126501768\\
45.96	0.00840164094818344\\
45.97	0.00840174068045467\\
45.98	0.00840184046186567\\
45.99	0.00840194029245074\\
46	0.00840204017224416\\
46.01	0.00840214010128022\\
46.02	0.00840224007959321\\
46.03	0.00840234010721741\\
46.04	0.00840244018418709\\
46.05	0.00840254031053654\\
46.06	0.00840264048629999\\
46.07	0.00840274071151171\\
46.08	0.00840284098620593\\
46.09	0.0084029413104169\\
46.1	0.00840304168417884\\
46.11	0.00840314210752595\\
46.12	0.00840324258049243\\
46.13	0.00840334310311245\\
46.14	0.0084034436754202\\
46.15	0.00840354429744981\\
46.16	0.00840364496923542\\
46.17	0.00840374569081115\\
46.18	0.00840384646221108\\
46.19	0.00840394728346928\\
46.2	0.00840404815461981\\
46.21	0.00840414907569668\\
46.22	0.00840425004673391\\
46.23	0.00840435106776546\\
46.24	0.00840445213882527\\
46.25	0.00840455325994727\\
46.26	0.00840465443116533\\
46.27	0.00840475565251332\\
46.28	0.00840485692402505\\
46.29	0.00840495824573431\\
46.3	0.00840505961767485\\
46.31	0.00840516103988036\\
46.32	0.00840526251238454\\
46.33	0.008405364035221\\
46.34	0.00840546560842334\\
46.35	0.00840556723202508\\
46.36	0.00840566890605972\\
46.37	0.00840577063056072\\
46.38	0.00840587240556146\\
46.39	0.00840597423109529\\
46.4	0.00840607610719551\\
46.41	0.00840617803389533\\
46.42	0.00840628001122794\\
46.43	0.00840638203922645\\
46.44	0.00840648411792392\\
46.45	0.00840658624735334\\
46.46	0.00840668842754762\\
46.47	0.00840679065853963\\
46.48	0.00840689294036214\\
46.49	0.00840699527304786\\
46.5	0.00840709765662943\\
46.51	0.00840720009113941\\
46.52	0.00840730257661026\\
46.53	0.00840740511307438\\
46.54	0.00840750770056409\\
46.55	0.00840761033911159\\
46.56	0.00840771302874901\\
46.57	0.0084078157695084\\
46.58	0.00840791856142168\\
46.59	0.00840802140452069\\
46.6	0.00840812429883717\\
46.61	0.00840822724440276\\
46.62	0.00840833024124896\\
46.63	0.00840843328940719\\
46.64	0.00840853638890875\\
46.65	0.00840863953978481\\
46.66	0.00840874274206643\\
46.67	0.00840884599578454\\
46.68	0.00840894930096995\\
46.69	0.00840905265765332\\
46.7	0.00840915606586519\\
46.71	0.00840925952563596\\
46.72	0.00840936303699587\\
46.73	0.00840946659997505\\
46.74	0.00840957021460344\\
46.75	0.00840967388091085\\
46.76	0.00840977759892691\\
46.77	0.0084098813686811\\
46.78	0.00840998519020275\\
46.79	0.00841008906352099\\
46.8	0.00841019298866477\\
46.81	0.0084102969656629\\
46.82	0.00841040099454395\\
46.83	0.00841050507533634\\
46.84	0.00841060920806829\\
46.85	0.00841071339276779\\
46.86	0.00841081762946266\\
46.87	0.00841092191818049\\
46.88	0.00841102625894866\\
46.89	0.00841113065179433\\
46.9	0.00841123509674444\\
46.91	0.00841133959382567\\
46.92	0.00841144414306449\\
46.93	0.00841154874448712\\
46.94	0.00841165339811952\\
46.95	0.0084117581039874\\
46.96	0.00841186286211621\\
46.97	0.00841196767253113\\
46.98	0.00841207253525705\\
46.99	0.0084121774503186\\
47	0.00841228241774011\\
47.01	0.00841238743754561\\
47.02	0.00841249250975884\\
47.03	0.0084125976344032\\
47.04	0.00841270281150182\\
47.05	0.00841280804107746\\
47.06	0.00841291332315257\\
47.07	0.00841301865774924\\
47.08	0.00841312404488924\\
47.09	0.00841322948459394\\
47.1	0.0084133349768844\\
47.11	0.00841344052178125\\
47.12	0.00841354611930477\\
47.13	0.00841365176947483\\
47.14	0.00841375747231092\\
47.15	0.00841386322783209\\
47.16	0.00841396903605699\\
47.17	0.00841407489700385\\
47.18	0.00841418081069042\\
47.19	0.00841428677713404\\
47.2	0.00841439279635156\\
47.21	0.00841449886835939\\
47.22	0.00841460499317341\\
47.23	0.00841471117080906\\
47.24	0.00841481740128125\\
47.25	0.00841492368460436\\
47.26	0.00841503002079227\\
47.27	0.0084151364098583\\
47.28	0.00841524285181523\\
47.29	0.00841534934667526\\
47.3	0.00841545589445004\\
47.31	0.00841556249515059\\
47.32	0.00841566914878737\\
47.33	0.0084157758553702\\
47.34	0.00841588261490826\\
47.35	0.00841598942741009\\
47.36	0.00841609629288359\\
47.37	0.00841620321133596\\
47.38	0.00841631018277373\\
47.39	0.00841641720720271\\
47.4	0.00841652428462799\\
47.41	0.00841663141505394\\
47.42	0.00841673859848416\\
47.43	0.00841684583492149\\
47.44	0.00841695312436798\\
47.45	0.00841706046682487\\
47.46	0.00841716786229259\\
47.47	0.00841727531077071\\
47.48	0.00841738281225798\\
47.49	0.00841749036675224\\
47.5	0.00841759797425043\\
47.51	0.00841770563474861\\
47.52	0.00841781334824186\\
47.53	0.00841792111472434\\
47.54	0.00841802893418921\\
47.55	0.00841813680662864\\
47.56	0.00841824473203379\\
47.57	0.00841835271039476\\
47.58	0.0084184607417006\\
47.59	0.00841856882593926\\
47.6	0.00841867696309758\\
47.61	0.00841878515316128\\
47.62	0.00841889339611492\\
47.63	0.00841900169194184\\
47.64	0.00841911004062421\\
47.65	0.00841921844214296\\
47.66	0.00841932689647773\\
47.67	0.0084194354036069\\
47.68	0.00841954396350753\\
47.69	0.00841965257615532\\
47.7	0.0084197612415246\\
47.71	0.00841986995958831\\
47.72	0.00841997873031795\\
47.73	0.00842008755368356\\
47.74	0.00842019642965368\\
47.75	0.00842030535819534\\
47.76	0.00842041433927398\\
47.77	0.00842052337285349\\
47.78	0.00842063245889612\\
47.79	0.00842074159736245\\
47.8	0.00842085078821139\\
47.81	0.00842096003140011\\
47.82	0.00842106932688401\\
47.83	0.0084211786746167\\
47.84	0.00842128807454996\\
47.85	0.00842139752663367\\
47.86	0.00842150703081582\\
47.87	0.00842161658704241\\
47.88	0.00842172619525749\\
47.89	0.00842183585540305\\
47.9	0.00842194556741898\\
47.91	0.00842205533124309\\
47.92	0.00842216514681099\\
47.93	0.00842227501405611\\
47.94	0.00842238493290961\\
47.95	0.00842249490330033\\
47.96	0.00842260492515481\\
47.97	0.00842271499839714\\
47.98	0.00842282512294899\\
47.99	0.00842293529872956\\
48	0.00842304552565546\\
48.01	0.00842315580364073\\
48.02	0.00842326613259677\\
48.03	0.00842337651243224\\
48.04	0.00842348694305309\\
48.05	0.00842359742436243\\
48.06	0.0084237079562605\\
48.07	0.00842381853864462\\
48.08	0.00842392917140915\\
48.09	0.00842403985444536\\
48.1	0.00842415058764146\\
48.11	0.00842426137088247\\
48.12	0.00842437220405019\\
48.13	0.00842448308702313\\
48.14	0.00842459401967645\\
48.15	0.00842470500188186\\
48.16	0.00842481603350762\\
48.17	0.0084249271144184\\
48.18	0.00842503824447527\\
48.19	0.0084251494235356\\
48.2	0.00842526065145296\\
48.21	0.00842537192807712\\
48.22	0.00842548325325391\\
48.23	0.00842559462682518\\
48.24	0.00842570604862871\\
48.25	0.00842581751849814\\
48.26	0.00842592903626288\\
48.27	0.00842604060174803\\
48.28	0.00842615221477433\\
48.29	0.00842626387515803\\
48.3	0.00842637558271083\\
48.31	0.00842648733723981\\
48.32	0.0084265991385473\\
48.33	0.00842671098643085\\
48.34	0.00842682288068308\\
48.35	0.00842693482109164\\
48.36	0.00842704680743908\\
48.37	0.0084271588395028\\
48.38	0.00842727091705489\\
48.39	0.00842738303986211\\
48.4	0.00842749520768574\\
48.41	0.00842760742028151\\
48.42	0.00842771967739946\\
48.43	0.00842783197878389\\
48.44	0.00842794432417324\\
48.45	0.00842805671329996\\
48.46	0.00842816914589045\\
48.47	0.00842828162166491\\
48.48	0.00842839414033725\\
48.49	0.00842850670161499\\
48.5	0.00842861930519915\\
48.51	0.00842873195078412\\
48.52	0.00842884463805755\\
48.53	0.00842895736670027\\
48.54	0.00842907013638611\\
48.55	0.00842918294678186\\
48.56	0.00842929579754708\\
48.57	0.00842940868833405\\
48.58	0.00842952161878758\\
48.59	0.00842963458854496\\
48.6	0.00842974759723576\\
48.61	0.00842986064448179\\
48.62	0.00842997372989691\\
48.63	0.00843008685308693\\
48.64	0.00843020001364951\\
48.65	0.00843031321117396\\
48.66	0.0084304264452412\\
48.67	0.00843053971542357\\
48.68	0.00843065302128474\\
48.69	0.00843076636237955\\
48.7	0.00843087973825388\\
48.71	0.00843099314844455\\
48.72	0.00843110659247918\\
48.73	0.00843122006987601\\
48.74	0.00843133358014386\\
48.75	0.00843144712278189\\
48.76	0.00843156069727957\\
48.77	0.00843167430311648\\
48.78	0.00843178793976221\\
48.79	0.0084319016066762\\
48.8	0.00843201530330766\\
48.81	0.00843212902909538\\
48.82	0.00843224278346767\\
48.83	0.00843235656584213\\
48.84	0.00843247037562563\\
48.85	0.00843258421221414\\
48.86	0.00843269807499257\\
48.87	0.00843281196333469\\
48.88	0.00843292587660302\\
48.89	0.00843303981414865\\
48.9	0.00843315377531118\\
48.91	0.00843326775941858\\
48.92	0.00843338176578708\\
48.93	0.00843349579372105\\
48.94	0.00843360984251293\\
48.95	0.00843372391144306\\
48.96	0.00843383799977965\\
48.97	0.00843395210677864\\
48.98	0.00843406623168363\\
48.99	0.00843418037372578\\
49	0.00843429453212372\\
49.01	0.00843440870608351\\
49.02	0.00843452289479851\\
49.03	0.00843463709744937\\
49.04	0.00843475131320395\\
49.05	0.00843486554121725\\
49.06	0.00843497978063141\\
49.07	0.0084350940305756\\
49.08	0.00843520829016608\\
49.09	0.00843532255850612\\
49.1	0.008435436834686\\
49.11	0.00843555111778301\\
49.12	0.00843566540686148\\
49.13	0.00843577970097278\\
49.14	0.00843589399915537\\
49.15	0.0084360083004348\\
49.16	0.00843612260382384\\
49.17	0.0084362369083225\\
49.18	0.00843635121291812\\
49.19	0.0084364655165855\\
49.2	0.008436579818287\\
49.21	0.00843669411697268\\
49.22	0.00843680841158044\\
49.23	0.00843692270103621\\
49.24	0.00843703698425415\\
49.25	0.00843715126013684\\
49.26	0.00843726552757552\\
49.27	0.00843737978545038\\
49.28	0.0084374940326308\\
49.29	0.00843760826797568\\
49.3	0.0084377224903338\\
49.31	0.00843783669854412\\
49.32	0.00843795089143624\\
49.33	0.0084380650678308\\
49.34	0.00843817922653992\\
49.35	0.00843829336636772\\
49.36	0.00843840748611083\\
49.37	0.00843852158455894\\
49.38	0.00843863566049547\\
49.39	0.0084387497126981\\
49.4	0.00843886373993959\\
49.41	0.0084389777409884\\
49.42	0.00843909171460953\\
49.43	0.00843920565956533\\
49.44	0.0084393195746164\\
49.45	0.00843943345852249\\
49.46	0.00843954731004351\\
49.47	0.00843966112794057\\
49.48	0.00843977491097711\\
49.49	0.00843988865792003\\
49.5	0.00844000236754099\\
49.51	0.00844011603861766\\
49.52	0.00844022966993514\\
49.53	0.00844034326028741\\
49.54	0.00844045680847887\\
49.55	0.00844057031332595\\
49.56	0.00844068377365882\\
49.57	0.00844079718832321\\
49.58	0.00844091055618229\\
49.59	0.00844102387611864\\
49.6	0.00844113714703635\\
49.61	0.00844125036786326\\
49.62	0.00844136353755321\\
49.63	0.00844147665508849\\
49.64	0.00844158971948243\\
49.65	0.00844170272978199\\
49.66	0.00844181568507065\\
49.67	0.00844192858447128\\
49.68	0.00844204142714929\\
49.69	0.00844215421231579\\
49.7	0.00844226693923104\\
49.71	0.00844237960720798\\
49.72	0.00844249221561593\\
49.73	0.00844260476388452\\
49.74	0.00844271725150768\\
49.75	0.00844282967804803\\
49.76	0.00844294204314122\\
49.77	0.00844305434650067\\
49.78	0.00844316658792242\\
49.79	0.00844327876729024\\
49.8	0.00844339088458094\\
49.81	0.00844350293987\\
49.82	0.00844361493333733\\
49.83	0.00844372686527339\\
49.84	0.00844383873608556\\
49.85	0.00844395054630473\\
49.86	0.0084440622965923\\
49.87	0.00844417398774735\\
49.88	0.00844428562071425\\
49.89	0.00844439719659051\\
49.9	0.00844450871663504\\
49.91	0.00844462018227671\\
49.92	0.00844473159512332\\
49.93	0.00844484295697096\\
49.94	0.00844495426981371\\
49.95	0.00844506553585388\\
49.96	0.00844517675751252\\
49.97	0.00844528793744054\\
49.98	0.00844539907853018\\
49.99	0.00844551018392705\\
50	0.00844562125704263\\
50.01	0.00844573230156728\\
50.02	0.00844584332148386\\
50.03	0.00844595432108189\\
50.04	0.00844606530497225\\
50.05	0.00844617627810256\\
50.06	0.00844628724577321\\
50.07	0.00844639821365394\\
50.08	0.00844650918780126\\
50.09	0.00844662017467648\\
50.1	0.0084467311811645\\
50.11	0.00844684221459339\\
50.12	0.00844695328275479\\
50.13	0.0084470643939251\\
50.14	0.00844717555688753\\
50.15	0.00844728678095511\\
50.16	0.00844739806847898\\
50.17	0.00844750941953051\\
50.18	0.00844762083418143\\
50.19	0.00844773231250382\\
50.2	0.00844784385457007\\
50.21	0.0084479554604529\\
50.22	0.0084480671302253\\
50.23	0.00844817886396051\\
50.24	0.00844829066173205\\
50.25	0.0084484025236136\\
50.26	0.00844851444967902\\
50.27	0.00844862644000236\\
50.28	0.00844873849465771\\
50.29	0.00844885061371928\\
50.3	0.00844896279726128\\
50.31	0.00844907504535789\\
50.32	0.00844918735808324\\
50.33	0.00844929973551131\\
50.34	0.00844941217771592\\
50.35	0.00844952468477064\\
50.36	0.00844963725674871\\
50.37	0.008449749893723\\
50.38	0.00844986259576593\\
50.39	0.00844997536294938\\
50.4	0.00845008819534459\\
50.41	0.00845020109302209\\
50.42	0.0084503140560516\\
50.43	0.00845042708450191\\
50.44	0.00845054017844079\\
50.45	0.00845065333793483\\
50.46	0.00845076656304939\\
50.47	0.00845087985384836\\
50.48	0.00845099321039414\\
50.49	0.00845110663274791\\
50.5	0.00845122012097097\\
50.51	0.00845133367512472\\
50.52	0.00845144729527062\\
50.53	0.00845156098147027\\
50.54	0.00845167473378534\\
50.55	0.00845178855227758\\
50.56	0.00845190243700889\\
50.57	0.00845201638804121\\
50.58	0.0084521304054366\\
50.59	0.00845224448925723\\
50.6	0.00845235863956535\\
50.61	0.0084524728564233\\
50.62	0.00845258713989355\\
50.63	0.00845270149003864\\
50.64	0.00845281590692122\\
50.65	0.00845293039060404\\
50.66	0.00845304494114995\\
50.67	0.00845315955862189\\
50.68	0.00845327424308292\\
50.69	0.00845338899459619\\
50.7	0.00845350381322496\\
50.71	0.00845361869903257\\
50.72	0.00845373365208248\\
50.73	0.00845384867243826\\
50.74	0.00845396376016357\\
50.75	0.00845407891532218\\
50.76	0.00845419413797796\\
50.77	0.00845430942819488\\
50.78	0.00845442478603704\\
50.79	0.00845454021156861\\
50.8	0.00845465570485389\\
50.81	0.00845477126595728\\
50.82	0.00845488689494329\\
50.83	0.00845500259187654\\
50.84	0.00845511835682175\\
50.85	0.00845523418984375\\
50.86	0.00845535009100749\\
50.87	0.00845546606037801\\
50.88	0.00845558209802048\\
50.89	0.00845569820400017\\
50.9	0.00845581437838246\\
50.91	0.00845593062123286\\
50.92	0.00845604693261696\\
50.93	0.0084561633126005\\
50.94	0.0084562797612493\\
50.95	0.00845639627862931\\
50.96	0.0084565128648066\\
50.97	0.00845662951984735\\
50.98	0.00845674624381785\\
50.99	0.00845686303678452\\
51	0.00845697989881388\\
51.01	0.00845709682997258\\
51.02	0.00845721383032739\\
51.03	0.00845733089994519\\
51.04	0.00845744803889298\\
51.05	0.00845756524723789\\
51.06	0.00845768252504717\\
51.07	0.00845779987238819\\
51.08	0.00845791728932842\\
51.09	0.0084580347759355\\
51.1	0.00845815233227715\\
51.11	0.00845826995842124\\
51.12	0.00845838765443575\\
51.13	0.0084585054203888\\
51.14	0.00845862325634864\\
51.15	0.00845874116238362\\
51.16	0.00845885913856224\\
51.17	0.00845897718495314\\
51.18	0.00845909530162506\\
51.19	0.00845921348864689\\
51.2	0.00845933174608767\\
51.21	0.00845945007401653\\
51.22	0.00845956847250276\\
51.23	0.00845968694161578\\
51.24	0.00845980548142514\\
51.25	0.00845992409200054\\
51.26	0.0084600427734118\\
51.27	0.0084601615257289\\
51.28	0.00846028034902192\\
51.29	0.00846039924336113\\
51.3	0.00846051820881689\\
51.31	0.00846063724545975\\
51.32	0.00846075635336036\\
51.33	0.00846087553258955\\
51.34	0.00846099478321826\\
51.35	0.00846111410531762\\
51.36	0.00846123349895886\\
51.37	0.00846135296421338\\
51.38	0.00846147250115273\\
51.39	0.00846159210984863\\
51.4	0.00846171179037289\\
51.41	0.00846183154279755\\
51.42	0.00846195136719474\\
51.43	0.00846207126363678\\
51.44	0.00846219123219614\\
51.45	0.00846231127294544\\
51.46	0.00846243138595747\\
51.47	0.00846255157130515\\
51.48	0.00846267182906161\\
51.49	0.00846279215930011\\
51.5	0.00846291256209407\\
51.51	0.00846303303751709\\
51.52	0.00846315358564293\\
51.53	0.00846327420654552\\
51.54	0.00846339490029896\\
51.55	0.00846351566697752\\
51.56	0.00846363650665564\\
51.57	0.00846375741940793\\
51.58	0.00846387840530919\\
51.59	0.00846399946443438\\
51.6	0.00846412059685865\\
51.61	0.00846424180265732\\
51.62	0.0084643630819059\\
51.63	0.00846448443468008\\
51.64	0.00846460586105573\\
51.65	0.00846472736110892\\
51.66	0.0084648489349159\\
51.67	0.0084649705825531\\
51.68	0.00846509230409716\\
51.69	0.00846521409962491\\
51.7	0.00846533596921336\\
51.71	0.00846545791293974\\
51.72	0.00846557993088146\\
51.73	0.00846570202311614\\
51.74	0.00846582418972161\\
51.75	0.00846594643077589\\
51.76	0.00846606874635723\\
51.77	0.00846619113654406\\
51.78	0.00846631360141505\\
51.79	0.00846643614104907\\
51.8	0.0084665587555252\\
51.81	0.00846668144492275\\
51.82	0.00846680420932125\\
51.83	0.00846692704880045\\
51.84	0.00846704996344032\\
51.85	0.00846717295332108\\
51.86	0.00846729601852314\\
51.87	0.00846741915912718\\
51.88	0.0084675423752141\\
51.89	0.00846766566686504\\
51.9	0.00846778903416139\\
51.91	0.00846791247718476\\
51.92	0.00846803599601703\\
51.93	0.00846815959074032\\
51.94	0.00846828326143699\\
51.95	0.00846840700818968\\
51.96	0.00846853083108127\\
51.97	0.0084686547301949\\
51.98	0.00846877870561399\\
51.99	0.0084689027574222\\
52	0.00846902688570349\\
52.01	0.00846915109054207\\
52.02	0.00846927537202244\\
52.03	0.00846939973022937\\
52.04	0.00846952416524793\\
52.05	0.00846964867716346\\
52.06	0.00846977326606159\\
52.07	0.00846989793202826\\
52.08	0.0084700226751497\\
52.09	0.00847014749551243\\
52.1	0.00847027239320329\\
52.11	0.00847039736830944\\
52.12	0.00847052242091832\\
52.13	0.00847064755111771\\
52.14	0.00847077275899571\\
52.15	0.00847089804464075\\
52.16	0.00847102340814158\\
52.17	0.00847114884958728\\
52.18	0.00847127436906729\\
52.19	0.00847139996667137\\
52.2	0.00847152564248964\\
52.21	0.00847165139661258\\
52.22	0.00847177722913101\\
52.23	0.00847190314013612\\
52.24	0.00847202912971948\\
52.25	0.00847215519797301\\
52.26	0.00847228134498903\\
52.27	0.00847240757086023\\
52.28	0.0084725338756797\\
52.29	0.0084726602595409\\
52.3	0.00847278672253771\\
52.31	0.00847291326476443\\
52.32	0.00847303988631572\\
52.33	0.00847316658728672\\
52.34	0.00847329336777294\\
52.35	0.00847342022787036\\
52.36	0.00847354716767537\\
52.37	0.00847367418728481\\
52.38	0.00847380128679597\\
52.39	0.0084739284663066\\
52.4	0.00847405572591491\\
52.41	0.00847418306571958\\
52.42	0.00847431048581974\\
52.43	0.00847443798631506\\
52.44	0.00847456556730565\\
52.45	0.00847469322889213\\
52.46	0.00847482097117565\\
52.47	0.00847494879425785\\
52.48	0.0084750766982409\\
52.49	0.00847520468322751\\
52.5	0.00847533274932091\\
52.51	0.00847546089662488\\
52.52	0.00847558912524378\\
52.53	0.0084757174352825\\
52.54	0.00847584582684653\\
52.55	0.00847597430004193\\
52.56	0.00847610285497537\\
52.57	0.00847623149175408\\
52.58	0.00847636021048595\\
52.59	0.00847648901127947\\
52.6	0.00847661789424375\\
52.61	0.00847674685948858\\
52.62	0.00847687590712435\\
52.63	0.00847700503726216\\
52.64	0.00847713425001377\\
52.65	0.0084772635454916\\
52.66	0.00847739292380881\\
52.67	0.00847752238507925\\
52.68	0.00847765192941748\\
52.69	0.0084777815569388\\
52.7	0.00847791126775927\\
52.71	0.0084780410619957\\
52.72	0.00847817093976566\\
52.73	0.00847830090118753\\
52.74	0.00847843094638044\\
52.75	0.00847856107546441\\
52.76	0.00847869128856021\\
52.77	0.00847882158578949\\
52.78	0.00847895196727474\\
52.79	0.00847908243313935\\
52.8	0.00847921298350756\\
52.81	0.00847934361850452\\
52.82	0.00847947433825631\\
52.83	0.00847960514288994\\
52.84	0.00847973603253336\\
52.85	0.00847986700731549\\
52.86	0.00847999806736625\\
52.87	0.00848012921281653\\
52.88	0.00848026044379829\\
52.89	0.00848039176044448\\
52.9	0.00848052316288913\\
52.91	0.00848065465126736\\
52.92	0.00848078622571536\\
52.93	0.00848091788637045\\
52.94	0.00848104963337108\\
52.95	0.00848118146685688\\
52.96	0.00848131338696863\\
52.97	0.00848144539384834\\
52.98	0.00848157748763922\\
52.99	0.00848170966848574\\
53	0.00848184193653363\\
53.01	0.00848197429192993\\
53.02	0.00848210673482299\\
53.03	0.00848223926536249\\
53.04	0.00848237188369949\\
53.05	0.00848250458998646\\
53.06	0.00848263738437727\\
53.07	0.00848277026702723\\
53.08	0.00848290323809314\\
53.09	0.00848303629773331\\
53.1	0.00848316944610756\\
53.11	0.00848330268337731\\
53.12	0.00848343600970551\\
53.13	0.0084835694252568\\
53.14	0.00848370293019742\\
53.15	0.00848383652469533\\
53.16	0.0084839702089202\\
53.17	0.00848410398304344\\
53.18	0.00848423784723826\\
53.19	0.00848437180167968\\
53.2	0.00848450584654459\\
53.21	0.00848463998201177\\
53.22	0.00848477420826192\\
53.23	0.00848490852547772\\
53.24	0.00848504293384385\\
53.25	0.00848517743354705\\
53.26	0.00848531202477612\\
53.27	0.00848544670772203\\
53.28	0.00848558148257789\\
53.29	0.00848571634953904\\
53.3	0.00848585130880308\\
53.31	0.0084859863605699\\
53.32	0.00848612150504176\\
53.33	0.00848625674242332\\
53.34	0.00848639207292166\\
53.35	0.0084865274967464\\
53.36	0.00848666301410967\\
53.37	0.0084867986252262\\
53.38	0.0084869343303134\\
53.39	0.00848707012959136\\
53.4	0.00848720602328295\\
53.41	0.00848734201161384\\
53.42	0.00848747809481258\\
53.43	0.00848761427311067\\
53.44	0.00848775054674257\\
53.45	0.00848788691594583\\
53.46	0.00848802338096111\\
53.47	0.00848815994203223\\
53.48	0.00848829659940629\\
53.49	0.00848843335333369\\
53.5	0.00848857020406822\\
53.51	0.00848870715186712\\
53.52	0.00848884419699116\\
53.53	0.00848898133970472\\
53.54	0.00848911858027584\\
53.55	0.00848925591897633\\
53.56	0.00848939335608181\\
53.57	0.00848953089187184\\
53.58	0.00848966852662994\\
53.59	0.00848980626064375\\
53.6	0.00848994409420503\\
53.61	0.00849008202760982\\
53.62	0.0084902200611585\\
53.63	0.00849035819515587\\
53.64	0.00849049642991127\\
53.65	0.00849063476573867\\
53.66	0.00849077320295674\\
53.67	0.00849091174188899\\
53.68	0.00849105038286384\\
53.69	0.00849118912621478\\
53.7	0.00849132797228038\\
53.71	0.00849146692140451\\
53.72	0.00849160597393636\\
53.73	0.00849174513023062\\
53.74	0.00849188439064757\\
53.75	0.0084920237555532\\
53.76	0.00849216322531935\\
53.77	0.00849230280032379\\
53.78	0.00849244248095041\\
53.79	0.00849258226758931\\
53.8	0.00849272216063697\\
53.81	0.00849286216049634\\
53.82	0.00849300226757701\\
53.83	0.00849314248229536\\
53.84	0.0084932828050747\\
53.85	0.00849342323634541\\
53.86	0.00849356377654511\\
53.87	0.00849370442611882\\
53.88	0.00849384518551909\\
53.89	0.00849398605520621\\
53.9	0.00849412703564835\\
53.91	0.00849426812732174\\
53.92	0.00849440933071086\\
53.93	0.00849455064630859\\
53.94	0.00849469207461643\\
53.95	0.00849483361614465\\
53.96	0.00849497527141255\\
53.97	0.00849511704094857\\
53.98	0.00849525892529058\\
53.99	0.008495400924986\\
54	0.00849554304059212\\
54.01	0.0084956852726762\\
54.02	0.00849582762181578\\
54.03	0.00849597008859888\\
54.04	0.00849611267362421\\
54.05	0.00849625537750144\\
54.06	0.00849639820085145\\
54.07	0.00849654114430652\\
54.08	0.00849668420851066\\
54.09	0.00849682739411983\\
54.1	0.00849697070180223\\
54.11	0.00849711413223853\\
54.12	0.0084972576861222\\
54.13	0.00849740136415978\\
54.14	0.00849754516707116\\
54.15	0.0084976890955899\\
54.16	0.00849783315046352\\
54.17	0.00849797733245381\\
54.18	0.00849812164233719\\
54.19	0.00849826608090499\\
54.2	0.00849841064896382\\
54.21	0.0084985553473359\\
54.22	0.00849870017685943\\
54.23	0.00849884513838891\\
54.24	0.00849899023279557\\
54.25	0.0084991354609677\\
54.26	0.00849928082381104\\
54.27	0.00849942632224921\\
54.28	0.0084995719572241\\
54.29	0.00849971772969626\\
54.3	0.00849986364064535\\
54.31	0.00850000969107058\\
54.32	0.00850015588199114\\
54.33	0.00850030221444666\\
54.34	0.00850044868949766\\
54.35	0.00850059530822607\\
54.36	0.00850074207173567\\
54.37	0.00850088898115264\\
54.38	0.008501036037626\\
54.39	0.00850118324232823\\
54.4	0.00850133059645572\\
54.41	0.00850147810122934\\
54.42	0.00850162575789505\\
54.43	0.00850177356772442\\
54.44	0.00850192153201522\\
54.45	0.00850206965209207\\
54.46	0.00850221792930701\\
54.47	0.00850236636504016\\
54.48	0.00850251496070033\\
54.49	0.00850266371772571\\
54.5	0.00850281263758457\\
54.51	0.00850296172177588\\
54.52	0.0085031109718301\\
54.53	0.00850326038930984\\
54.54	0.00850340997581065\\
54.55	0.00850355973296177\\
54.56	0.00850370966242687\\
54.57	0.00850385976590492\\
54.58	0.00850401004513093\\
54.59	0.00850416050187684\\
54.6	0.00850431113795234\\
54.61	0.00850446195520577\\
54.62	0.008504612955525\\
54.63	0.00850476414083836\\
54.64	0.00850491551311556\\
54.65	0.00850506707436867\\
54.66	0.0085052188266531\\
54.67	0.00850537077206859\\
54.68	0.00850552291276026\\
54.69	0.00850567525092023\\
54.7	0.00850582778878998\\
54.71	0.00850598052866161\\
54.72	0.00850613347287908\\
54.73	0.00850628662383949\\
54.74	0.00850643998399442\\
54.75	0.00850659355585124\\
54.76	0.00850674734197457\\
54.77	0.00850690134498764\\
54.78	0.00850705556757377\\
54.79	0.0085072100124779\\
54.8	0.0085073646825081\\
54.81	0.00850751958053716\\
54.82	0.00850767470950421\\
54.83	0.00850783007241637\\
54.84	0.0085079856723505\\
54.85	0.00850814151245491\\
54.86	0.00850829759595117\\
54.87	0.00850845392613597\\
54.88	0.00850861050638298\\
54.89	0.00850876734014485\\
54.9	0.00850892443095514\\
54.91	0.00850908178243041\\
54.92	0.00850923939827233\\
54.93	0.0085093972822698\\
54.94	0.00850955543830118\\
54.95	0.00850971387033658\\
54.96	0.00850987258244018\\
54.97	0.00851003157877261\\
54.98	0.00851019086359344\\
54.99	0.00851035044126368\\
55	0.00851051031624838\\
55.01	0.00851067049311928\\
55.02	0.00851083097655757\\
55.03	0.00851099177135666\\
55.04	0.00851115288242508\\
55.05	0.00851131431478941\\
55.06	0.00851147607359736\\
55.07	0.00851163816412083\\
55.08	0.00851180059175915\\
55.09	0.00851196336204235\\
55.1	0.00851212648063455\\
55.11	0.00851228995333737\\
55.12	0.00851245378609356\\
55.13	0.0085126179849906\\
55.14	0.00851278255626448\\
55.15	0.00851294750630351\\
55.16	0.00851311284165234\\
55.17	0.00851327856901597\\
55.18	0.00851344469526393\\
55.19	0.00851361122743458\\
55.2	0.00851377817273952\\
55.21	0.00851394553856806\\
55.22	0.00851411333249193\\
55.23	0.00851428156226998\\
55.24	0.00851445023585315\\
55.25	0.00851461936138944\\
55.26	0.00851478894722909\\
55.27	0.00851495900192993\\
55.28	0.00851512953426279\\
55.29	0.00851530055321709\\
55.3	0.00851547206800666\\
55.31	0.00851564408807558\\
55.32	0.0085158166231043\\
55.33	0.00851598968301586\\
55.34	0.0085161632779823\\
55.35	0.00851633741843126\\
55.36	0.00851651211505273\\
55.37	0.008516687378806\\
55.38	0.00851686322092684\\
55.39	0.0085170396529348\\
55.4	0.00851721668664076\\
55.41	0.00851739433415468\\
55.42	0.0085175726078936\\
55.43	0.00851775152058975\\
55.44	0.00851793108529904\\
55.45	0.00851811131540964\\
55.46	0.00851829222465087\\
55.47	0.00851847382710237\\
55.48	0.00851865613720344\\
55.49	0.00851883916976265\\
55.5	0.00851902293996781\\
55.51	0.00851920746339613\\
55.52	0.00851939275602467\\
55.53	0.00851957883424108\\
55.54	0.00851976571485469\\
55.55	0.00851995341510786\\
55.56	0.00852014195268765\\
55.57	0.00852033134573783\\
55.58	0.00852052161287122\\
55.59	0.00852071277318235\\
55.6	0.00852090484626057\\
55.61	0.00852109785220335\\
55.62	0.00852129181163015\\
55.63	0.00852148674569652\\
55.64	0.0085216826761087\\
55.65	0.0085218796251386\\
55.66	0.00852207761563913\\
55.67	0.0085222766710601\\
55.68	0.00852247681546441\\
55.69	0.00852267807354485\\
55.7	0.00852288047064127\\
55.71	0.00852308403275823\\
55.72	0.00852328878658321\\
55.73	0.00852349475950532\\
55.74	0.00852370197963448\\
55.75	0.00852391047582124\\
55.76	0.00852412027767705\\
55.77	0.00852433141559518\\
55.78	0.00852454392077222\\
55.79	0.00852475782523019\\
55.8	0.00852497316183922\\
55.81	0.00852518996434101\\
55.82	0.00852540826737276\\
55.83	0.00852562810649199\\
55.84	0.00852584951820189\\
55.85	0.00852607253997754\\
55.86	0.00852629721029272\\
55.87	0.00852652356864765\\
55.88	0.00852675165559738\\
55.89	0.00852698151278108\\
55.9	0.00852721318295214\\
55.91	0.00852744671000912\\
55.92	0.00852768213902761\\
55.93	0.00852791951629295\\
55.94	0.00852815888933396\\
55.95	0.00852840030695758\\
55.96	0.00852864381928454\\
55.97	0.00852888947778603\\
55.98	0.00852913733532144\\
55.99	0.00852938744617717\\
56	0.00852963986610655\\
56.01	0.00852989465237098\\
56.02	0.00853015186378211\\
56.03	0.00853041156074541\\
56.04	0.00853067380530485\\
56.05	0.00853093866118897\\
56.06	0.00853120619385824\\
56.07	0.00853147647055374\\
56.08	0.00853174956034737\\
56.09	0.00853202553419341\\
56.1	0.00853230446498159\\
56.11	0.00853258642759174\\
56.12	0.00853287149894999\\
56.13	0.00853315975808659\\
56.14	0.00853345128619546\\
56.15	0.00853374616669544\\
56.16	0.00853404448529328\\
56.17	0.00853434633004855\\
56.18	0.00853465179144035\\
56.19	0.00853496096243605\\
56.2	0.00853527393856196\\
56.21	0.00853559081797611\\
56.22	0.00853591170154313\\
56.23	0.0085362366929113\\
56.24	0.00853656589859193\\
56.25	0.00853689942804097\\
56.26	0.00853723739374306\\
56.27	0.00853757837316989\\
56.28	0.00853791948362209\\
56.29	0.00853826072010381\\
56.3	0.00853860207737868\\
56.31	0.00853894354996083\\
56.32	0.00853928513210557\\
56.33	0.00853962681779968\\
56.34	0.00853996860075147\\
56.35	0.00854031047438045\\
56.36	0.00854065243180663\\
56.37	0.00854099446583951\\
56.38	0.00854133656896664\\
56.39	0.00854167873334183\\
56.4	0.00854202095077297\\
56.41	0.0085423632127094\\
56.42	0.00854270551022885\\
56.43	0.00854304783402401\\
56.44	0.00854339017438857\\
56.45	0.00854373252120283\\
56.46	0.00854407486391883\\
56.47	0.00854441719154492\\
56.48	0.00854475949262996\\
56.49	0.00854510175524678\\
56.5	0.0085454439669753\\
56.51	0.00854578611488494\\
56.52	0.00854612818551648\\
56.53	0.00854647016486339\\
56.54	0.00854681203835244\\
56.55	0.00854715379082369\\
56.56	0.00854749540650987\\
56.57	0.00854783686901502\\
56.58	0.00854817816129245\\
56.59	0.00854851926562195\\
56.6	0.0085488601635863\\
56.61	0.00854920083604691\\
56.62	0.00854954126311877\\
56.63	0.0085498814241445\\
56.64	0.00855022129766756\\
56.65	0.00855056086140459\\
56.66	0.00855090009221686\\
56.67	0.00855123896608076\\
56.68	0.00855157745805728\\
56.69	0.00855191554226065\\
56.7	0.00855225319182572\\
56.71	0.0085525903788745\\
56.72	0.00855292707448145\\
56.73	0.00855326324863775\\
56.74	0.00855359887021433\\
56.75	0.00855393390692369\\
56.76	0.0085542683252806\\
56.77	0.00855460209056137\\
56.78	0.00855493516676187\\
56.79	0.00855526751655426\\
56.8	0.00855559910124216\\
56.81	0.0085559298807145\\
56.82	0.00855625981339784\\
56.83	0.00855658885620715\\
56.84	0.00855691696449496\\
56.85	0.00855724409199899\\
56.86	0.00855757019078793\\
56.87	0.00855789521120563\\
56.88	0.00855821910181337\\
56.89	0.00855854180933038\\
56.9	0.00855886327857238\\
56.91	0.00855918345238817\\
56.92	0.00855950227159422\\
56.93	0.00855981967490711\\
56.94	0.00856013699889897\\
56.95	0.00856045446145748\\
56.96	0.00856077206266139\\
56.97	0.00856108980258951\\
56.98	0.00856140768132069\\
56.99	0.00856172569893389\\
57	0.00856204385550809\\
57.01	0.00856236215112235\\
57.02	0.00856268058585579\\
57.03	0.00856299915978762\\
57.04	0.00856331787299706\\
57.05	0.00856363672556346\\
57.06	0.00856395571756617\\
57.07	0.00856427484908465\\
57.08	0.00856459412019841\\
57.09	0.00856491353098702\\
57.1	0.0085652330815301\\
57.11	0.00856555277190738\\
57.12	0.0085658726021986\\
57.13	0.0085661925724836\\
57.14	0.00856651268284228\\
57.15	0.00856683293335458\\
57.16	0.00856715332410054\\
57.17	0.00856747385516025\\
57.18	0.00856779452661384\\
57.19	0.00856811533854154\\
57.2	0.00856843629102363\\
57.21	0.00856875738414045\\
57.22	0.00856907861797241\\
57.23	0.00856939999259998\\
57.24	0.00856972150810371\\
57.25	0.00857004316456419\\
57.26	0.00857036496206209\\
57.27	0.00857068690067815\\
57.28	0.00857100898049315\\
57.29	0.00857133120158797\\
57.3	0.00857165356404352\\
57.31	0.0085719760679408\\
57.32	0.00857229871336085\\
57.33	0.00857262150038481\\
57.34	0.00857294442909385\\
57.35	0.00857326749956922\\
57.36	0.00857359071189223\\
57.37	0.00857391406614426\\
57.38	0.00857423756240675\\
57.39	0.00857456120076122\\
57.4	0.00857488498128922\\
57.41	0.00857520890407239\\
57.42	0.00857553296919244\\
57.43	0.00857585717673112\\
57.44	0.00857618152677028\\
57.45	0.00857650601939179\\
57.46	0.00857683065467764\\
57.47	0.00857715543270982\\
57.48	0.00857748035357045\\
57.49	0.00857780541734166\\
57.5	0.0085781306241057\\
57.51	0.00857845597394482\\
57.52	0.00857878146694141\\
57.53	0.00857910710317786\\
57.54	0.00857943288273666\\
57.55	0.00857975880570036\\
57.56	0.00858008487215157\\
57.57	0.00858041108217297\\
57.58	0.00858073743584731\\
57.59	0.00858106393325741\\
57.6	0.00858139057448612\\
57.61	0.00858171735961641\\
57.62	0.00858204428873128\\
57.63	0.00858237136191381\\
57.64	0.00858269857924714\\
57.65	0.00858302594081447\\
57.66	0.00858335344669908\\
57.67	0.00858368109698433\\
57.68	0.0085840088917536\\
57.69	0.00858433683109039\\
57.7	0.00858466491507822\\
57.71	0.00858499314380072\\
57.72	0.00858532151734155\\
57.73	0.00858565003578446\\
57.74	0.00858597869921326\\
57.75	0.00858630750771182\\
57.76	0.00858663646136409\\
57.77	0.00858696556025408\\
57.78	0.00858729480446587\\
57.79	0.0085876241940836\\
57.8	0.00858795372919149\\
57.81	0.00858828340987382\\
57.82	0.00858861323621495\\
57.83	0.00858894320829927\\
57.84	0.00858927332621128\\
57.85	0.00858960359003554\\
57.86	0.00858993399985666\\
57.87	0.00859026455575933\\
57.88	0.0085905952578283\\
57.89	0.0085909261061484\\
57.9	0.00859125710080453\\
57.91	0.00859158824188164\\
57.92	0.00859191952946477\\
57.93	0.00859225096363901\\
57.94	0.00859258254448953\\
57.95	0.00859291427210156\\
57.96	0.0085932461465604\\
57.97	0.00859357816795144\\
57.98	0.00859391033636011\\
57.99	0.00859424265187193\\
58	0.00859457511457246\\
58.01	0.00859490772454737\\
58.02	0.00859524048188236\\
58.03	0.00859557338666323\\
58.04	0.00859590643897583\\
58.05	0.00859623963890609\\
58.06	0.00859657298654\\
58.07	0.00859690648196363\\
58.08	0.00859724012526311\\
58.09	0.00859757391652465\\
58.1	0.00859790785583452\\
58.11	0.00859824194327906\\
58.12	0.00859857617894468\\
58.13	0.00859891056291789\\
58.14	0.00859924509528521\\
58.15	0.00859957977613329\\
58.16	0.0085999146055488\\
58.17	0.00860024958361853\\
58.18	0.0086005847104293\\
58.19	0.00860091998606802\\
58.2	0.00860125541062166\\
58.21	0.00860159098417728\\
58.22	0.00860192670682199\\
58.23	0.00860226257864297\\
58.24	0.0086025985997275\\
58.25	0.0086029347701629\\
58.26	0.00860327109003656\\
58.27	0.00860360755943596\\
58.28	0.00860394417844865\\
58.29	0.00860428094716224\\
58.3	0.00860461786566442\\
58.31	0.00860495493404295\\
58.32	0.00860529215238564\\
58.33	0.00860562952078042\\
58.34	0.00860596703931524\\
58.35	0.00860630470807816\\
58.36	0.00860664252715728\\
58.37	0.00860698049664081\\
58.38	0.00860731861661699\\
58.39	0.00860765688717416\\
58.4	0.00860799530840073\\
58.41	0.00860833388038518\\
58.42	0.00860867260321604\\
58.43	0.00860901147698195\\
58.44	0.00860935050177159\\
58.45	0.00860968967767374\\
58.46	0.00861002900477724\\
58.47	0.00861036848317099\\
58.48	0.00861070811294399\\
58.49	0.00861104789418528\\
58.5	0.00861138782698401\\
58.51	0.00861172791142937\\
58.52	0.00861206814761065\\
58.53	0.00861240853561718\\
58.54	0.00861274907553842\\
58.55	0.00861308976746383\\
58.56	0.008613430611483\\
58.57	0.00861377160768558\\
58.58	0.00861411275616127\\
58.59	0.00861445405699988\\
58.6	0.00861479551029127\\
58.61	0.00861513711612537\\
58.62	0.00861547887459221\\
58.63	0.00861582078578186\\
58.64	0.0086161628497845\\
58.65	0.00861650506669036\\
58.66	0.00861684743658975\\
58.67	0.00861718995957305\\
58.68	0.00861753263573073\\
58.69	0.00861787546515332\\
58.7	0.00861821844793143\\
58.71	0.00861856158415573\\
58.72	0.00861890487391701\\
58.73	0.00861924831730608\\
58.74	0.00861959191441385\\
58.75	0.00861993566533131\\
58.76	0.00862027957014953\\
58.77	0.00862062362895962\\
58.78	0.00862096784185282\\
58.79	0.00862131220892038\\
58.8	0.00862165673025369\\
58.81	0.00862200140594417\\
58.82	0.00862234623608335\\
58.83	0.0086226912207628\\
58.84	0.0086230363600742\\
58.85	0.00862338165410927\\
58.86	0.00862372710295985\\
58.87	0.00862407270671781\\
58.88	0.00862441846547513\\
58.89	0.00862476437932386\\
58.9	0.00862511044835612\\
58.91	0.0086254566726641\\
58.92	0.00862580305234008\\
58.93	0.00862614958747642\\
58.94	0.00862649627816554\\
58.95	0.00862684312449994\\
58.96	0.0086271901265722\\
58.97	0.008627537284475\\
58.98	0.00862788459830106\\
58.99	0.00862823206814319\\
59	0.0086285796940943\\
59.01	0.00862892747624734\\
59.02	0.00862927541469537\\
59.03	0.0086296235095315\\
59.04	0.00862997176084893\\
59.05	0.00863032016874096\\
59.06	0.00863066873330093\\
59.07	0.00863101745462229\\
59.08	0.00863136633279853\\
59.09	0.00863171536792326\\
59.1	0.00863206456009014\\
59.11	0.00863241390939293\\
59.12	0.00863276341592544\\
59.13	0.00863311307978159\\
59.14	0.00863346290105535\\
59.15	0.0086338128798408\\
59.16	0.00863416301623206\\
59.17	0.00863451331032336\\
59.18	0.00863486376220901\\
59.19	0.00863521437198337\\
59.2	0.00863556513974091\\
59.21	0.00863591606557615\\
59.22	0.00863626714958373\\
59.23	0.00863661839185832\\
59.24	0.00863696979249472\\
59.25	0.00863732135158776\\
59.26	0.00863767306923239\\
59.27	0.00863802494552363\\
59.28	0.00863837698055655\\
59.29	0.00863872917442635\\
59.3	0.00863908152722827\\
59.31	0.00863943403905764\\
59.32	0.00863978671000989\\
59.33	0.00864013954018051\\
59.34	0.00864049252966506\\
59.35	0.00864084567855922\\
59.36	0.00864119898695872\\
59.37	0.00864155245495936\\
59.38	0.00864190608265706\\
59.39	0.00864225987014779\\
59.4	0.00864261381752762\\
59.41	0.00864296792489268\\
59.42	0.0086433221923392\\
59.43	0.00864367661996348\\
59.44	0.0086440312078619\\
59.45	0.00864438595613095\\
59.46	0.00864474086486716\\
59.47	0.00864509593416716\\
59.48	0.00864545116412767\\
59.49	0.00864580655484549\\
59.5	0.00864616210641748\\
59.51	0.00864651781894062\\
59.52	0.00864687369251193\\
59.53	0.00864722972722855\\
59.54	0.00864758592318768\\
59.55	0.0086479422804866\\
59.56	0.0086482987992227\\
59.57	0.00864865547949342\\
59.58	0.0086490123213963\\
59.59	0.00864936932502896\\
59.6	0.0086497264904891\\
59.61	0.00865008381787451\\
59.62	0.00865044130728305\\
59.63	0.00865079895881269\\
59.64	0.00865115677256145\\
59.65	0.00865151474862745\\
59.66	0.00865187288710891\\
59.67	0.0086522311881041\\
59.68	0.00865258965171139\\
59.69	0.00865294827802924\\
59.7	0.00865330706715619\\
59.71	0.00865366601919086\\
59.72	0.00865402513423196\\
59.73	0.00865438441237828\\
59.74	0.00865474385372869\\
59.75	0.00865510345838216\\
59.76	0.00865546322643773\\
59.77	0.00865582315799453\\
59.78	0.00865618325315177\\
59.79	0.00865654351200876\\
59.8	0.00865690393466488\\
59.81	0.00865726452121959\\
59.82	0.00865762527177246\\
59.83	0.00865798618642313\\
59.84	0.00865834726527131\\
59.85	0.00865870850841683\\
59.86	0.00865906991595958\\
59.87	0.00865943148799954\\
59.88	0.00865979322463678\\
59.89	0.00866015512597146\\
59.9	0.00866051719210381\\
59.91	0.00866087942313417\\
59.92	0.00866124181916294\\
59.93	0.00866160438029064\\
59.94	0.00866196710661784\\
59.95	0.00866232999824522\\
59.96	0.00866269305527353\\
59.97	0.00866305627780363\\
59.98	0.00866341966593645\\
59.99	0.00866378321977301\\
60	0.00866414693941442\\
60.01	0.00866451082496186\\
60.02	0.00866487487651664\\
60.03	0.00866523909418011\\
60.04	0.00866560347805373\\
60.05	0.00866596802823905\\
60.06	0.00866633274483771\\
60.07	0.00866669762795141\\
60.08	0.00866706267768198\\
60.09	0.00866742789413131\\
60.1	0.00866779327740139\\
60.11	0.00866815882759429\\
60.12	0.00866852454481217\\
60.13	0.00866889042915728\\
60.14	0.00866925648073197\\
60.15	0.00866962269963866\\
60.16	0.00866998908597987\\
60.17	0.00867035563985821\\
60.18	0.00867072236137637\\
60.19	0.00867108925063713\\
60.2	0.00867145630774338\\
60.21	0.00867182353279807\\
60.22	0.00867219092590426\\
60.23	0.0086725584871651\\
60.24	0.00867292621668381\\
60.25	0.00867329411456371\\
60.26	0.00867366218090823\\
60.27	0.00867403041582086\\
60.28	0.0086743988194052\\
60.29	0.00867476739176492\\
60.3	0.00867513613300381\\
60.31	0.00867550504322573\\
60.32	0.00867587412253464\\
60.33	0.00867624337103457\\
60.34	0.00867661278882968\\
60.35	0.00867698237602418\\
60.36	0.00867735213272239\\
60.37	0.00867772205902874\\
60.38	0.00867809215504771\\
60.39	0.0086784624208839\\
60.4	0.008678832856642\\
60.41	0.00867920346242679\\
60.42	0.00867957423834313\\
60.43	0.00867994518449598\\
60.44	0.0086803163009904\\
60.45	0.00868068758793153\\
60.46	0.00868105904542461\\
60.47	0.00868143067357497\\
60.48	0.00868180247248804\\
60.49	0.00868217444226932\\
60.5	0.00868254658302442\\
60.51	0.00868291889485906\\
60.52	0.00868329137787901\\
60.53	0.00868366403219016\\
60.54	0.00868403685789851\\
60.55	0.00868440985511012\\
60.56	0.00868478302393116\\
60.57	0.00868515636446788\\
60.58	0.00868552987682665\\
60.59	0.00868590356111392\\
60.6	0.00868627741743621\\
60.61	0.00868665144590018\\
60.62	0.00868702564661255\\
60.63	0.00868740001968015\\
60.64	0.0086877745652099\\
60.65	0.00868814928330881\\
60.66	0.00868852417408399\\
60.67	0.00868889923764265\\
60.68	0.00868927447409207\\
60.69	0.00868964988353966\\
60.7	0.00869002546609292\\
60.71	0.00869040122185941\\
60.72	0.00869077715094682\\
60.73	0.00869115325346293\\
60.74	0.0086915295295156\\
60.75	0.00869190597921281\\
60.76	0.00869228260266261\\
60.77	0.00869265939997317\\
60.78	0.00869303637125275\\
60.79	0.00869341351660968\\
60.8	0.00869379083615243\\
60.81	0.00869416832998953\\
60.82	0.00869454599822963\\
60.83	0.00869492384098145\\
60.84	0.00869530185835386\\
60.85	0.00869568005045576\\
60.86	0.00869605841739619\\
60.87	0.00869643695928428\\
60.88	0.00869681567622926\\
60.89	0.00869719456834043\\
60.9	0.00869757363572723\\
60.91	0.00869795287849916\\
60.92	0.00869833229676586\\
60.93	0.00869871189063701\\
60.94	0.00869909166022245\\
60.95	0.00869947160563208\\
60.96	0.00869985172697591\\
60.97	0.00870023202436404\\
60.98	0.00870061249790668\\
60.99	0.00870099314771414\\
61	0.00870137397389682\\
61.01	0.00870175497656521\\
61.02	0.00870213615582993\\
61.03	0.00870251751180168\\
61.04	0.00870289904459125\\
61.05	0.00870328075430955\\
61.06	0.00870366264106757\\
61.07	0.00870404470497643\\
61.08	0.0087044269461473\\
61.09	0.00870480936469151\\
61.1	0.00870519196072044\\
61.11	0.00870557473434561\\
61.12	0.0087059576856786\\
61.13	0.00870634081483113\\
61.14	0.008706724121915\\
61.15	0.00870710760704211\\
61.16	0.00870749127032447\\
61.17	0.00870787511187419\\
61.18	0.00870825913180347\\
61.19	0.00870864333022462\\
61.2	0.00870902770725005\\
61.21	0.00870941226299229\\
61.22	0.00870979699756393\\
61.23	0.00871018191107771\\
61.24	0.00871056700364643\\
61.25	0.00871095227538303\\
61.26	0.00871133772640052\\
61.27	0.00871172335681204\\
61.28	0.0087121091667308\\
61.29	0.00871249515627016\\
61.3	0.00871288132554353\\
61.31	0.00871326767466447\\
61.32	0.0087136542037466\\
61.33	0.00871404091290369\\
61.34	0.00871442780224957\\
61.35	0.00871481487189819\\
61.36	0.00871520212196363\\
61.37	0.00871558955256004\\
61.38	0.00871597716380168\\
61.39	0.00871636495580292\\
61.4	0.00871675292867823\\
61.41	0.00871714108254219\\
61.42	0.0087175294175095\\
61.43	0.00871791793369493\\
61.44	0.00871830663121337\\
61.45	0.00871869551017984\\
61.46	0.00871908457070942\\
61.47	0.00871947381291733\\
61.48	0.00871986323691888\\
61.49	0.00872025284282949\\
61.5	0.0087206426307647\\
61.51	0.00872103260084012\\
61.52	0.0087214227531715\\
61.53	0.00872181308787469\\
61.54	0.00872220360506563\\
61.55	0.00872259430486038\\
61.56	0.00872298518737511\\
61.57	0.00872337625272608\\
61.58	0.00872376750102968\\
61.59	0.00872415893240239\\
61.6	0.0087245505469608\\
61.61	0.00872494234482161\\
61.62	0.00872533432610164\\
61.63	0.00872572649091779\\
61.64	0.00872611883938709\\
61.65	0.00872651137162667\\
61.66	0.00872690408775377\\
61.67	0.00872729698788574\\
61.68	0.00872769007214004\\
61.69	0.00872808334063422\\
61.7	0.00872847679348597\\
61.71	0.00872887043081307\\
61.72	0.0087292642527334\\
61.73	0.00872965825936498\\
61.74	0.0087300524508259\\
61.75	0.0087304468272344\\
61.76	0.0087308413887088\\
61.77	0.00873123613536755\\
61.78	0.00873163106732918\\
61.79	0.00873202618471236\\
61.8	0.00873242148763586\\
61.81	0.00873281697621856\\
61.82	0.00873321265057946\\
61.83	0.00873360851083765\\
61.84	0.00873400455711235\\
61.85	0.00873440078952288\\
61.86	0.00873479720818868\\
61.87	0.00873519381322929\\
61.88	0.00873559060476437\\
61.89	0.0087359875829137\\
61.9	0.00873638474779715\\
61.91	0.00873678209953471\\
61.92	0.0087371796382465\\
61.93	0.00873757736405273\\
61.94	0.00873797527707373\\
61.95	0.00873837337742994\\
61.96	0.00873877166524193\\
61.97	0.00873917014063035\\
61.98	0.00873956880371599\\
61.99	0.00873996765461976\\
62	0.00874036669346264\\
62.01	0.00874076592036578\\
62.02	0.00874116533545039\\
62.03	0.00874156493883784\\
62.04	0.00874196473064959\\
62.05	0.00874236471100721\\
62.06	0.00874276488003239\\
62.07	0.00874316523784694\\
62.08	0.00874356578457279\\
62.09	0.00874396652033196\\
62.1	0.00874436744524662\\
62.11	0.00874476855943902\\
62.12	0.00874516986303155\\
62.13	0.00874557135614671\\
62.14	0.0087459730389071\\
62.15	0.00874637491143545\\
62.16	0.00874677697385461\\
62.17	0.00874717922628754\\
62.18	0.00874758166885732\\
62.19	0.00874798430168713\\
62.2	0.00874838712490028\\
62.21	0.00874879013862021\\
62.22	0.00874919334297045\\
62.23	0.00874959673807467\\
62.24	0.00875000032405663\\
62.25	0.00875040410104024\\
62.26	0.0087508080691495\\
62.27	0.00875121222850855\\
62.28	0.00875161657924163\\
62.29	0.0087520211214731\\
62.3	0.00875242585532746\\
62.31	0.00875283078092929\\
62.32	0.00875323589840332\\
62.33	0.00875364120787439\\
62.34	0.00875404670946746\\
62.35	0.0087544524033076\\
62.36	0.00875485828951999\\
62.37	0.00875526436822997\\
62.38	0.00875567063956297\\
62.39	0.00875607710364453\\
62.4	0.00875648376060032\\
62.41	0.00875689061055615\\
62.42	0.00875729765363792\\
62.43	0.00875770488997166\\
62.44	0.00875811231968353\\
62.45	0.00875851994289981\\
62.46	0.00875892775974688\\
62.47	0.00875933577035126\\
62.48	0.00875974397483959\\
62.49	0.00876015237333862\\
62.5	0.00876056096597523\\
62.51	0.00876096975287642\\
62.52	0.00876137873416932\\
62.53	0.00876178790998115\\
62.54	0.0087621972804393\\
62.55	0.00876260684567125\\
62.56	0.00876301660580459\\
62.57	0.00876342656096707\\
62.58	0.00876383671128654\\
62.59	0.00876424705689097\\
62.6	0.00876465759790847\\
62.61	0.00876506833446725\\
62.62	0.00876547926669566\\
62.63	0.00876589039472218\\
62.64	0.00876630171867538\\
62.65	0.00876671323868399\\
62.66	0.00876712495487684\\
62.67	0.0087675368673829\\
62.68	0.00876794897633125\\
62.69	0.00876836128185111\\
62.7	0.00876877378407181\\
62.71	0.00876918648312281\\
62.72	0.00876959937913369\\
62.73	0.00877001247223417\\
62.74	0.00877042576255407\\
62.75	0.00877083925022336\\
62.76	0.00877125293537212\\
62.77	0.00877166681813055\\
62.78	0.008772080898629\\
62.79	0.00877249517699792\\
62.8	0.00877290965336791\\
62.81	0.00877332432786966\\
62.82	0.00877373920063402\\
62.83	0.00877415427179195\\
62.84	0.00877456954147455\\
62.85	0.00877498500981303\\
62.86	0.00877540067693873\\
62.87	0.00877581654298313\\
62.88	0.00877623260807781\\
62.89	0.00877664887235452\\
62.9	0.00877706533594508\\
62.91	0.0087774819989815\\
62.92	0.00877789886159586\\
62.93	0.0087783159239204\\
62.94	0.00877873318608749\\
62.95	0.00877915064822961\\
62.96	0.00877956831047937\\
62.97	0.00877998617296953\\
62.98	0.00878040423583295\\
62.99	0.00878082249920264\\
63	0.00878124096321172\\
63.01	0.00878165962799344\\
63.02	0.00878207849368121\\
63.03	0.00878249756040854\\
63.04	0.00878291682830905\\
63.05	0.00878333629751653\\
63.06	0.00878375596816489\\
63.07	0.00878417584038814\\
63.08	0.00878459591432045\\
63.09	0.00878501619009611\\
63.1	0.00878543666784954\\
63.11	0.00878585734771529\\
63.12	0.00878627822982803\\
63.13	0.00878669931432257\\
63.14	0.00878712060133385\\
63.15	0.00878754209099695\\
63.16	0.00878796378344704\\
63.17	0.00878838567881947\\
63.18	0.00878880777724969\\
63.19	0.00878923007887329\\
63.2	0.00878965258382599\\
63.21	0.00879007529224364\\
63.22	0.00879049820426222\\
63.23	0.00879092132001784\\
63.24	0.00879134463964674\\
63.25	0.00879176816328529\\
63.26	0.00879219189107\\
63.27	0.00879261582313749\\
63.28	0.00879303995962455\\
63.29	0.00879346430066806\\
63.3	0.00879388884640504\\
63.31	0.00879431359697266\\
63.32	0.00879473855250822\\
63.33	0.00879516371314912\\
63.34	0.00879558907903292\\
63.35	0.00879601465029732\\
63.36	0.00879644042708011\\
63.37	0.00879686640951925\\
63.38	0.00879729259775282\\
63.39	0.00879771899191902\\
63.4	0.00879814559215621\\
63.41	0.00879857239860285\\
63.42	0.00879899941139755\\
63.43	0.00879942663067904\\
63.44	0.0087998540565862\\
63.45	0.00880028168925802\\
63.46	0.00880070952883364\\
63.47	0.00880113757545232\\
63.48	0.00880156582925345\\
63.49	0.00880199429037657\\
63.5	0.00880242295896134\\
63.51	0.00880285183514753\\
63.52	0.00880328091907509\\
63.53	0.00880371021088405\\
63.54	0.00880413971071462\\
63.55	0.0088045694187071\\
63.56	0.00880499933500195\\
63.57	0.00880542945973975\\
63.58	0.0088058597930612\\
63.59	0.00880629033510716\\
63.6	0.00880672108601861\\
63.61	0.00880715204593664\\
63.62	0.0088075832150025\\
63.63	0.00880801459335756\\
63.64	0.00880844618114332\\
63.65	0.00880887797850142\\
63.66	0.00880930998557362\\
63.67	0.0088097422025018\\
63.68	0.00881017462942802\\
63.69	0.0088106072664944\\
63.7	0.00881104011384326\\
63.71	0.008811473171617\\
63.72	0.00881190643995818\\
63.73	0.00881233991900948\\
63.74	0.00881277360891371\\
63.75	0.0088132075098138\\
63.76	0.00881364162185284\\
63.77	0.00881407594517402\\
63.78	0.00881451047992067\\
63.79	0.00881494522623626\\
63.8	0.00881538018426438\\
63.81	0.00881581535414874\\
63.82	0.0088162507360332\\
63.83	0.00881668633006174\\
63.84	0.00881712213637847\\
63.85	0.00881755815512762\\
63.86	0.00881799438645355\\
63.87	0.00881843083050077\\
63.88	0.0088188674874139\\
63.89	0.00881930435733768\\
63.9	0.008819741440417\\
63.91	0.00882017873679686\\
63.92	0.0088206162466224\\
63.93	0.00882105397003887\\
63.94	0.00882149190719167\\
63.95	0.00882193005822631\\
63.96	0.00882236842328842\\
63.97	0.00882280700252379\\
63.98	0.0088232457960783\\
63.99	0.00882368480409798\\
64	0.00882412402672896\\
64.01	0.00882456346411751\\
64.02	0.00882500311641003\\
64.03	0.00882544298375305\\
64.04	0.0088258830662932\\
64.05	0.00882632336417725\\
64.06	0.00882676387755209\\
64.07	0.00882720460656474\\
64.08	0.00882764555136234\\
64.09	0.00882808671209213\\
64.1	0.00882852808890152\\
64.11	0.00882896968193799\\
64.12	0.00882941149134917\\
64.13	0.00882985351728282\\
64.14	0.00883029575988679\\
64.15	0.00883073821930908\\
64.16	0.0088311808956978\\
64.17	0.00883162378920116\\
64.18	0.00883206689996752\\
64.19	0.00883251022814533\\
64.2	0.0088329537738832\\
64.21	0.0088333975373298\\
64.22	0.00883384151863397\\
64.23	0.00883428571794463\\
64.24	0.00883473013541084\\
64.25	0.00883517477118176\\
64.26	0.00883561962540668\\
64.27	0.00883606469823499\\
64.28	0.0088365099898162\\
64.29	0.00883695550029995\\
64.3	0.00883740122983596\\
64.31	0.00883784717857409\\
64.32	0.0088382933466643\\
64.33	0.00883873973425667\\
64.34	0.00883918634150139\\
64.35	0.00883963316854875\\
64.36	0.00884008021554915\\
64.37	0.00884052748265312\\
64.38	0.00884097497001128\\
64.39	0.00884142267777437\\
64.4	0.00884187060609321\\
64.41	0.00884231875511876\\
64.42	0.00884276712500209\\
64.43	0.00884321571589433\\
64.44	0.00884366452794677\\
64.45	0.00884411356131076\\
64.46	0.00884456281613779\\
64.47	0.00884501229257943\\
64.48	0.00884546199078735\\
64.49	0.00884591191091333\\
64.5	0.00884636205310926\\
64.51	0.00884681241752712\\
64.52	0.00884726300431898\\
64.53	0.00884771381363704\\
64.54	0.00884816484563355\\
64.55	0.00884861610046091\\
64.56	0.00884906757827158\\
64.57	0.00884951927921813\\
64.58	0.00884997120345321\\
64.59	0.0088504233511296\\
64.6	0.00885087572240014\\
64.61	0.00885132831741777\\
64.62	0.00885178113633552\\
64.63	0.00885223417930654\\
64.64	0.00885268744648402\\
64.65	0.00885314093802129\\
64.66	0.00885359465407173\\
64.67	0.00885404859478883\\
64.68	0.00885450276032617\\
64.69	0.00885495715083739\\
64.7	0.00885541176647625\\
64.71	0.00885586660739658\\
64.72	0.00885632167375228\\
64.73	0.00885677696569735\\
64.74	0.00885723248338588\\
64.75	0.00885768822697203\\
64.76	0.00885814419661002\\
64.77	0.0088586003924542\\
64.78	0.00885905681465895\\
64.79	0.00885951346337875\\
64.8	0.00885997033876818\\
64.81	0.00886042744098185\\
64.82	0.00886088477017448\\
64.83	0.00886134232650086\\
64.84	0.00886180011011583\\
64.85	0.00886225812117435\\
64.86	0.00886271635983141\\
64.87	0.0088631748262421\\
64.88	0.00886363352056155\\
64.89	0.008864092442945\\
64.9	0.00886455159354773\\
64.91	0.00886501097252511\\
64.92	0.00886547058003256\\
64.93	0.00886593041622557\\
64.94	0.00886639048125972\\
64.95	0.00886685077529063\\
64.96	0.00886731129847399\\
64.97	0.00886777205096557\\
64.98	0.00886823303292118\\
64.99	0.00886869424449673\\
65	0.00886915568584815\\
65.01	0.00886961735713146\\
65.02	0.00887007925850274\\
65.03	0.00887054139011812\\
65.04	0.00887100375213381\\
65.05	0.00887146634470606\\
65.06	0.00887192916799119\\
65.07	0.00887239222214557\\
65.08	0.00887285550732565\\
65.09	0.00887331902368793\\
65.1	0.00887378277138895\\
65.11	0.00887424675058533\\
65.12	0.00887471096143374\\
65.13	0.00887517540409092\\
65.14	0.00887564007871364\\
65.15	0.00887610498545875\\
65.16	0.00887657012448315\\
65.17	0.0088770354959438\\
65.18	0.00887750109999772\\
65.19	0.00887796693680198\\
65.2	0.0088784330065137\\
65.21	0.00887889930929009\\
65.22	0.00887936584528838\\
65.23	0.00887983261466587\\
65.24	0.00888029961757994\\
65.25	0.00888076685418801\\
65.26	0.00888123432464754\\
65.27	0.00888170202911609\\
65.28	0.00888216996775125\\
65.29	0.00888263814071069\\
65.3	0.00888310654815214\\
65.31	0.00888357519023337\\
65.32	0.00888404406711225\\
65.33	0.00888451317894668\\
65.34	0.00888498252589465\\
65.35	0.00888545210811421\\
65.36	0.00888592192576347\\
65.37	0.00888639197900061\\
65.38	0.00888686226798391\\
65.39	0.00888733279287168\\
65.4	0.00888780355382234\\
65.41	0.00888827455099436\\
65.42	0.00888874578454631\\
65.43	0.00888921725463681\\
65.44	0.00888968896142461\\
65.45	0.0088901609050685\\
65.46	0.00889063308572738\\
65.47	0.00889110550356024\\
65.48	0.00889157815872615\\
65.49	0.00889205105138429\\
65.5	0.00889252418169392\\
65.51	0.00889299754981442\\
65.52	0.00889347115590527\\
65.53	0.00889394500012606\\
65.54	0.00889441908263648\\
65.55	0.00889489340359635\\
65.56	0.00889536796316561\\
65.57	0.00889584276150432\\
65.58	0.00889631779877267\\
65.59	0.00889679307513098\\
65.6	0.0088972685907397\\
65.61	0.00889774434575945\\
65.62	0.00889822034035099\\
65.63	0.0088986965746752\\
65.64	0.00889917304889317\\
65.65	0.00889964976316612\\
65.66	0.00890012671765546\\
65.67	0.00890060391252278\\
65.68	0.00890108134792983\\
65.69	0.00890155902403858\\
65.7	0.00890203694101119\\
65.71	0.00890251509901001\\
65.72	0.00890299349819764\\
65.73	0.00890347213873687\\
65.74	0.00890395102079073\\
65.75	0.0089044301445225\\
65.76	0.00890490951009569\\
65.77	0.00890538911767408\\
65.78	0.00890586896742171\\
65.79	0.00890634905950292\\
65.8	0.00890682939408229\\
65.81	0.00890730997132473\\
65.82	0.00890779079139547\\
65.83	0.00890827185446004\\
65.84	0.00890875316068429\\
65.85	0.00890923471023445\\
65.86	0.00890971650327707\\
65.87	0.00891019853997909\\
65.88	0.00891068082050783\\
65.89	0.008911163345031\\
65.9	0.00891164611371673\\
65.91	0.00891212912673357\\
65.92	0.0089126123842505\\
65.93	0.00891309588643696\\
65.94	0.00891357963346289\\
65.95	0.00891406362549867\\
65.96	0.00891454786271521\\
65.97	0.00891503234528395\\
65.98	0.00891551707337686\\
65.99	0.00891600204716647\\
66	0.00891648726682587\\
66.01	0.00891697273252878\\
66.02	0.00891745844444952\\
66.03	0.00891794440276305\\
66.04	0.00891843060764497\\
66.05	0.00891891705927159\\
66.06	0.00891940375781991\\
66.07	0.00891989070346765\\
66.08	0.00892037789639329\\
66.09	0.00892086533677609\\
66.1	0.00892135302479608\\
66.11	0.00892184096063414\\
66.12	0.008922329144472\\
66.13	0.00892281757649224\\
66.14	0.00892330625687839\\
66.15	0.00892379518581485\\
66.16	0.00892428436348705\\
66.17	0.00892477379008137\\
66.18	0.0089252634657852\\
66.19	0.00892575339078702\\
66.2	0.00892624356527637\\
66.21	0.00892673398944391\\
66.22	0.00892722466348144\\
66.23	0.00892771558758197\\
66.24	0.00892820676193971\\
66.25	0.00892869818675012\\
66.26	0.00892918986220996\\
66.27	0.0089296817885173\\
66.28	0.00893017396587161\\
66.29	0.00893066639447373\\
66.3	0.00893115907452595\\
66.31	0.00893165200623158\\
66.32	0.00893214518979471\\
66.33	0.00893263862542026\\
66.34	0.00893313231331395\\
66.35	0.00893362625368238\\
66.36	0.008934120446733\\
66.37	0.00893461489267416\\
66.38	0.00893510959171512\\
66.39	0.00893560454406605\\
66.4	0.00893609974993808\\
66.41	0.00893659520954331\\
66.42	0.00893709092309482\\
66.43	0.00893758689080671\\
66.44	0.00893808311289408\\
66.45	0.00893857958957309\\
66.46	0.00893907632106097\\
66.47	0.00893957330757603\\
66.48	0.00894007054933769\\
66.49	0.00894056804656647\\
66.5	0.00894106579948404\\
66.51	0.00894156380831325\\
66.52	0.00894206207327811\\
66.53	0.00894256059460381\\
66.54	0.00894305937251677\\
66.55	0.00894355840724464\\
66.56	0.0089440576990163\\
66.57	0.0089445572480619\\
66.58	0.00894505705461287\\
66.59	0.0089455571189019\\
66.6	0.00894605744116302\\
66.61	0.00894655802163154\\
66.62	0.00894705886054412\\
66.63	0.00894755995813874\\
66.64	0.00894806131465473\\
66.65	0.00894856293033279\\
66.66	0.00894906480541494\\
66.67	0.0089495669401446\\
66.68	0.00895006933476655\\
66.69	0.00895057198952693\\
66.7	0.00895107490467326\\
66.71	0.00895157808045445\\
66.72	0.00895208151712074\\
66.73	0.00895258521492377\\
66.74	0.00895308917411649\\
66.75	0.00895359339495325\\
66.76	0.00895409787768969\\
66.77	0.00895460262258279\\
66.78	0.00895510762989083\\
66.79	0.00895561289987335\\
66.8	0.00895611843279118\\
66.81	0.00895662422890635\\
66.82	0.00895713028848211\\
66.83	0.00895763661178286\\
66.84	0.00895814319907413\\
66.85	0.00895865005062256\\
66.86	0.00895915716669579\\
66.87	0.00895966454756248\\
66.88	0.0089601721934922\\
66.89	0.00896068010475539\\
66.9	0.00896118828162333\\
66.91	0.008961696724368\\
66.92	0.00896220543326205\\
66.93	0.0089627144085787\\
66.94	0.00896322365059164\\
66.95	0.00896373315957498\\
66.96	0.00896424293580309\\
66.97	0.00896475297955049\\
66.98	0.00896526329109181\\
66.99	0.00896577387070154\\
67	0.00896628471865401\\
67.01	0.00896679583522316\\
67.02	0.00896730722068244\\
67.03	0.00896781887530461\\
67.04	0.00896833079936159\\
67.05	0.00896884299312427\\
67.06	0.00896935545686231\\
67.07	0.00896986819084391\\
67.08	0.00897038119533564\\
67.09	0.00897089447060216\\
67.1	0.0089714080169085\\
67.11	0.00897192183452043\\
67.12	0.00897243592370455\\
67.13	0.00897295028472821\\
67.14	0.00897346491785957\\
67.15	0.0089739798233676\\
67.16	0.00897449500152208\\
67.17	0.00897501045259362\\
67.18	0.00897552617685365\\
67.19	0.00897604217457443\\
67.2	0.00897655844602911\\
67.21	0.00897707499149165\\
67.22	0.00897759181123691\\
67.23	0.00897810890554061\\
67.24	0.00897862627467937\\
67.25	0.00897914391893068\\
67.26	0.00897966183857297\\
67.27	0.00898018003388555\\
67.28	0.0089806985051487\\
67.29	0.00898121725264359\\
67.3	0.00898173627665236\\
67.31	0.00898225557745811\\
67.32	0.0089827751553449\\
67.33	0.00898329501059778\\
67.34	0.00898381514350276\\
67.35	0.0089843355543469\\
67.36	0.00898485624341824\\
67.37	0.00898537721100584\\
67.38	0.00898589845739983\\
67.39	0.00898641998289135\\
67.4	0.00898694178777263\\
67.41	0.00898746387233697\\
67.42	0.00898798623687874\\
67.43	0.00898850888169342\\
67.44	0.00898903180707762\\
67.45	0.00898955501332905\\
67.46	0.00899007850074658\\
67.47	0.00899060226963021\\
67.48	0.00899112632028114\\
67.49	0.00899165065300172\\
67.5	0.00899217526809553\\
67.51	0.00899270016586732\\
67.52	0.00899322534662312\\
67.53	0.00899375081067015\\
67.54	0.00899427655831692\\
67.55	0.0089948025898732\\
67.56	0.00899532890565005\\
67.57	0.00899585550595986\\
67.58	0.00899638239111629\\
67.59	0.0089969095614344\\
67.6	0.00899743701723056\\
67.61	0.00899796475882253\\
67.62	0.00899849278652947\\
67.63	0.00899902110067195\\
67.64	0.00899954970157194\\
67.65	0.0090000785895529\\
67.66	0.00900060776493972\\
67.67	0.00900113722805879\\
67.68	0.009001666979238\\
67.69	0.00900219701880677\\
67.7	0.00900272734709606\\
67.71	0.0090032579644384\\
67.72	0.00900378887116791\\
67.73	0.00900432006762029\\
67.74	0.00900485155413292\\
67.75	0.00900538333104478\\
67.76	0.00900591539869656\\
67.77	0.00900644775743062\\
67.78	0.00900698040759106\\
67.79	0.00900751334952372\\
67.8	0.00900804658357619\\
67.81	0.00900858011009787\\
67.82	0.00900911392943999\\
67.83	0.00900964804195557\\
67.84	0.00901018244799955\\
67.85	0.00901071714792875\\
67.86	0.00901125214210189\\
67.87	0.00901178743087965\\
67.88	0.00901232301462469\\
67.89	0.00901285889370167\\
67.9	0.00901339506847726\\
67.91	0.00901393153932021\\
67.92	0.00901446830660137\\
67.93	0.00901500537069368\\
67.94	0.00901554273197224\\
67.95	0.00901608039081433\\
67.96	0.00901661834759944\\
67.97	0.0090171566027093\\
67.98	0.00901769515652791\\
67.99	0.00901823400944161\\
68	0.00901877316183903\\
68.01	0.00901931261411121\\
68.02	0.00901985236665159\\
68.03	0.00902039241985606\\
68.04	0.00902093277412298\\
68.05	0.00902147342985323\\
68.06	0.00902201438745026\\
68.07	0.00902255564732008\\
68.08	0.00902309720987137\\
68.09	0.00902363907551544\\
68.1	0.00902418124466633\\
68.11	0.00902472371774083\\
68.12	0.00902526649515851\\
68.13	0.00902580957734178\\
68.14	0.00902635296471591\\
68.15	0.00902689665770909\\
68.16	0.00902744065675249\\
68.17	0.00902798496228026\\
68.18	0.0090285295747296\\
68.19	0.00902907449454082\\
68.2	0.00902961972215736\\
68.21	0.00903016525802586\\
68.22	0.00903071110259619\\
68.23	0.0090312572563215\\
68.24	0.00903180371965829\\
68.25	0.00903235049306645\\
68.26	0.00903289757700929\\
68.27	0.00903344497195362\\
68.28	0.0090339926783698\\
68.29	0.00903454069673177\\
68.3	0.00903508902751716\\
68.31	0.00903563767120726\\
68.32	0.00903618662828716\\
68.33	0.00903673589924578\\
68.34	0.00903728548457588\\
68.35	0.00903783538477421\\
68.36	0.00903838560034148\\
68.37	0.0090389361317825\\
68.38	0.00903948697960618\\
68.39	0.00904003814432564\\
68.4	0.00904058962645824\\
68.41	0.00904114142652567\\
68.42	0.00904169354505401\\
68.43	0.00904224598257376\\
68.44	0.00904279873962\\
68.45	0.00904335181673236\\
68.46	0.00904390521445516\\
68.47	0.00904445893333743\\
68.48	0.00904501297393304\\
68.49	0.00904556733680074\\
68.5	0.00904612202250421\\
68.51	0.00904667703161221\\
68.52	0.00904723236469859\\
68.53	0.00904778802234241\\
68.54	0.00904834400512797\\
68.55	0.00904890031364498\\
68.56	0.00904945694848855\\
68.57	0.00905001391025934\\
68.58	0.00905057119956361\\
68.59	0.00905112881701331\\
68.6	0.00905168676322619\\
68.61	0.00905224503882588\\
68.62	0.00905280364444194\\
68.63	0.00905336258071004\\
68.64	0.00905392184827199\\
68.65	0.00905448144777581\\
68.66	0.00905504137987592\\
68.67	0.00905560164523314\\
68.68	0.00905616224451487\\
68.69	0.00905672317839513\\
68.7	0.00905728444755468\\
68.71	0.00905784605268116\\
68.72	0.00905840799446915\\
68.73	0.00905897027362028\\
68.74	0.00905953289084339\\
68.75	0.00906009584685456\\
68.76	0.00906065914237731\\
68.77	0.00906122277814263\\
68.78	0.00906178675488914\\
68.79	0.00906235107336322\\
68.8	0.0090629157343191\\
68.81	0.00906348073851898\\
68.82	0.00906404608673316\\
68.83	0.00906461177974017\\
68.84	0.00906517781832689\\
68.85	0.00906574420328868\\
68.86	0.00906631093542947\\
68.87	0.00906687801556198\\
68.88	0.00906744544450775\\
68.89	0.00906801322309733\\
68.9	0.0090685813521704\\
68.91	0.00906914983257594\\
68.92	0.00906971866517229\\
68.93	0.00907028785082739\\
68.94	0.00907085739041883\\
68.95	0.00907142728483407\\
68.96	0.00907199753497054\\
68.97	0.00907256814173578\\
68.98	0.00907313910604763\\
68.99	0.00907371042883436\\
69	0.00907428211103483\\
69.01	0.0090748541535986\\
69.02	0.00907542655748615\\
69.03	0.00907599932366902\\
69.04	0.00907657245312994\\
69.05	0.00907714594686303\\
69.06	0.00907771980587393\\
69.07	0.00907829403117997\\
69.08	0.00907886862381038\\
69.09	0.00907944358480641\\
69.1	0.0090800189152215\\
69.11	0.00908059461612147\\
69.12	0.0090811706885847\\
69.13	0.00908174713370229\\
69.14	0.00908232395257824\\
69.15	0.00908290114632962\\
69.16	0.00908347871608675\\
69.17	0.00908405666299341\\
69.18	0.00908463498820698\\
69.19	0.00908521369289865\\
69.2	0.00908579277825359\\
69.21	0.00908637224547113\\
69.22	0.009086952095765\\
69.23	0.00908753233036344\\
69.24	0.00908811295050943\\
69.25	0.00908869395746091\\
69.26	0.00908927535249089\\
69.27	0.00908985713688774\\
69.28	0.00909043931195529\\
69.29	0.0090910218790131\\
69.3	0.00909160483939662\\
69.31	0.00909218819445736\\
69.32	0.00909277194556315\\
69.33	0.00909335609409827\\
69.34	0.0090939406414637\\
69.35	0.00909452558907728\\
69.36	0.0090951109383739\\
69.37	0.00909569669080576\\
69.38	0.00909628284784248\\
69.39	0.00909686941097135\\
69.4	0.00909745638169752\\
69.41	0.00909804376154418\\
69.42	0.00909863155205276\\
69.43	0.00909921975478311\\
69.44	0.00909980837131374\\
69.45	0.00910039740324194\\
69.46	0.00910098685218401\\
69.47	0.00910157671977547\\
69.48	0.00910216700767118\\
69.49	0.00910275771754557\\
69.5	0.00910334885109282\\
69.51	0.00910394041002702\\
69.52	0.00910453239608234\\
69.53	0.00910512481101323\\
69.54	0.00910571765659457\\
69.55	0.00910631093462182\\
69.56	0.00910690464691121\\
69.57	0.00910749879529989\\
69.58	0.00910809338164604\\
69.59	0.00910868840782909\\
69.6	0.00910928387574979\\
69.61	0.00910987978733038\\
69.62	0.00911047614451472\\
69.63	0.0091110729492684\\
69.64	0.00911167020357884\\
69.65	0.00911226790945546\\
69.66	0.00911286606892968\\
69.67	0.0091134646840551\\
69.68	0.00911406375690755\\
69.69	0.00911466328958512\\
69.7	0.00911526328420829\\
69.71	0.00911586374291995\\
69.72	0.00911646466788542\\
69.73	0.00911706606129251\\
69.74	0.00911766792535153\\
69.75	0.00911827026229525\\
69.76	0.00911887307437895\\
69.77	0.00911947636388035\\
69.78	0.00912008013309958\\
69.79	0.00912068438435912\\
69.8	0.00912128912000372\\
69.81	0.00912189434240032\\
69.82	0.0091225000539379\\
69.83	0.00912310625702739\\
69.84	0.0091237129541015\\
69.85	0.00912432014761452\\
69.86	0.00912492784004213\\
69.87	0.00912553603388123\\
69.88	0.00912614473164959\\
69.89	0.00912675393588567\\
69.9	0.00912736364914826\\
69.91	0.00912797387401615\\
69.92	0.00912858461308781\\
69.93	0.00912919586898095\\
69.94	0.0091298076443321\\
69.95	0.00913041994179617\\
69.96	0.00913103276404594\\
69.97	0.00913164611377149\\
69.98	0.0091322599936797\\
69.99	0.00913287440649354\\
70	0.00913348935495148\\
70.01	0.00913410484180677\\
70.02	0.00913472086982667\\
70.03	0.00913533744179163\\
70.04	0.0091359545604945\\
70.05	0.00913657222873958\\
70.06	0.00913719044934165\\
70.07	0.00913780922512498\\
70.08	0.00913842855892224\\
70.09	0.00913904845357335\\
70.1	0.00913966891192429\\
70.11	0.00914028993682579\\
70.12	0.00914091153113203\\
70.13	0.00914153369769919\\
70.14	0.00914215643938393\\
70.15	0.00914277975904188\\
70.16	0.00914340365952592\\
70.17	0.00914402814368445\\
70.18	0.00914465321435956\\
70.19	0.00914527887438507\\
70.2	0.00914590512658456\\
70.21	0.00914653197376915\\
70.22	0.00914715941873537\\
70.23	0.00914778746426272\\
70.24	0.00914841611311126\\
70.25	0.00914904536801901\\
70.26	0.00914967523169927\\
70.27	0.00915030570683776\\
70.28	0.0091509367960897\\
70.29	0.00915156850207664\\
70.3	0.00915220082738329\\
70.31	0.00915283377455406\\
70.32	0.00915346734608956\\
70.33	0.00915410154444286\\
70.34	0.00915473637201563\\
70.35	0.00915537183115406\\
70.36	0.00915600792414466\\
70.37	0.0091566446532098\\
70.38	0.00915728202050317\\
70.39	0.00915792002810488\\
70.4	0.00915855867801646\\
70.41	0.00915919797215566\\
70.42	0.00915983791235092\\
70.43	0.00916047850033572\\
70.44	0.00916111973774261\\
70.45	0.00916176162609699\\
70.46	0.00916240416681072\\
70.47	0.0091630473611753\\
70.48	0.00916369121035493\\
70.49	0.00916433571537919\\
70.5	0.0091649808771354\\
70.51	0.00916562669636077\\
70.52	0.00916627317363407\\
70.53	0.0091669203093671\\
70.54	0.00916756810379577\\
70.55	0.00916821655697077\\
70.56	0.00916886566874789\\
70.57	0.00916951543877803\\
70.58	0.00917016586649662\\
70.59	0.00917081695111285\\
70.6	0.00917146869159827\\
70.61	0.00917212108667506\\
70.62	0.00917277413480378\\
70.63	0.00917342783417066\\
70.64	0.00917408218267435\\
70.65	0.00917473717791224\\
70.66	0.00917539281716609\\
70.67	0.00917604909738728\\
70.68	0.00917670601518136\\
70.69	0.00917736356679201\\
70.7	0.00917802174808445\\
70.71	0.00917868055452817\\
70.72	0.00917933998117892\\
70.73	0.00918000002266016\\
70.74	0.00918066067314368\\
70.75	0.00918132192632949\\
70.76	0.00918198377542503\\
70.77	0.00918264621312346\\
70.78	0.00918330923158124\\
70.79	0.00918397282239479\\
70.8	0.00918463697657633\\
70.81	0.0091853016845287\\
70.82	0.00918596693601935\\
70.83	0.00918663272015332\\
70.84	0.00918729902534516\\
70.85	0.00918796583928984\\
70.86	0.00918863314893258\\
70.87	0.00918930094043754\\
70.88	0.00918996919915533\\
70.89	0.00919063790958938\\
70.9	0.00919130705536096\\
70.91	0.00919197661917301\\
70.92	0.00919264658277256\\
70.93	0.00919331692691182\\
70.94	0.0091939876313078\\
70.95	0.00919465867460047\\
70.96	0.00919533003430937\\
70.97	0.00919600168678863\\
70.98	0.0091966736071804\\
70.99	0.00919734576936646\\
71	0.00919801814591825\\
71.01	0.00919869070804489\\
71.02	0.00919936342553942\\
71.03	0.00920003626672311\\
71.04	0.00920070919838769\\
71.05	0.00920138218573549\\
71.06	0.0092020551923175\\
71.07	0.00920272817996909\\
71.08	0.00920340110874352\\
71.09	0.00920407393684294\\
71.1	0.00920474662054706\\
71.11	0.00920541911413912\\
71.12	0.00920609136982931\\
71.13	0.00920676333767535\\
71.14	0.00920743496550041\\
71.15	0.00920810619880785\\
71.16	0.00920877698069316\\
71.17	0.00920944725175259\\
71.18	0.00921011694998868\\
71.19	0.00921078601071227\\
71.2	0.00921145436644121\\
71.21	0.00921212194679533\\
71.22	0.00921278867838779\\
71.23	0.00921345448471254\\
71.24	0.00921411928602781\\
71.25	0.00921478299923555\\
71.26	0.00921544553775652\\
71.27	0.00921610681140103\\
71.28	0.0092167667262352\\
71.29	0.00921742518444238\\
71.3	0.00921808208417984\\
71.31	0.00921873731943035\\
71.32	0.0092193907798486\\
71.33	0.00922004235060224\\
71.34	0.00922069191220735\\
71.35	0.00922133934035815\\
71.36	0.00922198450575081\\
71.37	0.0092226272739011\\
71.38	0.00922326750495563\\
71.39	0.00922390505349664\\
71.4	0.00922453976833986\\
71.41	0.00922517149232547\\
71.42	0.00922580006210171\\
71.43	0.00922642530790104\\
71.44	0.00922704705330856\\
71.45	0.0092276651150223\\
71.46	0.00922827930260537\\
71.47	0.00922888941822942\\
71.48	0.00922949525640929\\
71.49	0.00923009660372852\\
71.5	0.00923069323855535\\
71.51	0.00923128493074893\\
71.52	0.00923187144135548\\
71.53	0.00923245252229393\\
71.54	0.00923302791603079\\
71.55	0.00923359735524382\\
71.56	0.00923416056247416\\
71.57	0.00923471724976661\\
71.58	0.00923526764281969\\
71.59	0.00923581834425938\\
71.6	0.00923636935377676\\
71.61	0.00923692067104964\\
71.62	0.00923747229574248\\
71.63	0.00923802422750636\\
71.64	0.00923857646597888\\
71.65	0.00923912901078417\\
71.66	0.00923968186153281\\
71.67	0.00924023501782188\\
71.68	0.00924078847923491\\
71.69	0.00924134224534193\\
71.7	0.00924189631569949\\
71.71	0.00924245068985068\\
71.72	0.00924300536732527\\
71.73	0.00924356034763974\\
71.74	0.00924411563029742\\
71.75	0.00924467121478857\\
71.76	0.00924522710059063\\
71.77	0.0092457832871683\\
71.78	0.00924633977397381\\
71.79	0.00924689656044713\\
71.8	0.00924745364601621\\
71.81	0.00924801103009733\\
71.82	0.00924856871209536\\
71.83	0.00924912669140415\\
71.84	0.00924968496740692\\
71.85	0.00925024353947671\\
71.86	0.00925080240697682\\
71.87	0.00925136156926132\\
71.88	0.00925192102567565\\
71.89	0.00925248077555717\\
71.9	0.00925304081823587\\
71.91	0.009253601153035\\
71.92	0.0092541617792719\\
71.93	0.00925472269625877\\
71.94	0.00925528390330353\\
71.95	0.0092558453997108\\
71.96	0.00925640718478287\\
71.97	0.00925696925782074\\
71.98	0.00925753161812532\\
71.99	0.00925809426499861\\
72	0.00925865719774497\\
72.01	0.00925922041567255\\
72.02	0.00925978391809469\\
72.03	0.00926034770433152\\
72.04	0.00926091177371159\\
72.05	0.00926147612557359\\
72.06	0.00926204075926825\\
72.07	0.00926260567416026\\
72.08	0.00926317086963034\\
72.09	0.00926373634507746\\
72.1	0.00926430209992114\\
72.11	0.00926486813360388\\
72.12	0.00926543444559378\\
72.13	0.0092660010353872\\
72.14	0.00926656790251172\\
72.15	0.00926713504652908\\
72.16	0.00926770246703845\\
72.17	0.00926827016367972\\
72.18	0.00926883813613709\\
72.19	0.00926940638414276\\
72.2	0.00926997490748089\\
72.21	0.00927054370599166\\
72.22	0.00927111277957564\\
72.23	0.00927168212819834\\
72.24	0.00927225175189497\\
72.25	0.00927282165077548\\
72.26	0.00927339182502982\\
72.27	0.00927396227493347\\
72.28	0.00927453300085327\\
72.29	0.00927510400325348\\
72.3	0.00927567528270217\\
72.31	0.00927624683987795\\
72.32	0.00927681867557694\\
72.33	0.00927739079072018\\
72.34	0.00927796318636129\\
72.35	0.00927853586369458\\
72.36	0.0092791088240635\\
72.37	0.00927968206896946\\
72.38	0.00928025560008117\\
72.39	0.00928082941924428\\
72.4	0.00928140352849156\\
72.41	0.00928197793005352\\
72.42	0.00928255262636952\\
72.43	0.00928312762009943\\
72.44	0.00928370291413572\\
72.45	0.00928427851161623\\
72.46	0.00928485441593741\\
72.47	0.00928543063076828\\
72.48	0.00928600716006485\\
72.49	0.00928658400808537\\
72.5	0.00928716117940613\\
72.51	0.00928773867893804\\
72.52	0.00928831651194389\\
72.53	0.00928889468405645\\
72.54	0.00928947320129732\\
72.55	0.00929005207009657\\
72.56	0.00929063129731336\\
72.57	0.00929121089025737\\
72.58	0.00929179085671119\\
72.59	0.00929237120495372\\
72.6	0.0092929519437845\\
72.61	0.00929353308254924\\
72.62	0.00929411463116631\\
72.63	0.00929469660015443\\
72.64	0.00929527900066161\\
72.65	0.00929586184449526\\
72.66	0.00929644514415356\\
72.67	0.00929702891285832\\
72.68	0.00929761316458905\\
72.69	0.00929819791411864\\
72.7	0.00929878317705046\\
72.71	0.0092993689698571\\
72.72	0.00929995530992066\\
72.73	0.00930054221557487\\
72.74	0.00930112970614884\\
72.75	0.00930171780094145\\
72.76	0.00930230651358761\\
72.77	0.00930289585835774\\
72.78	0.00930348585018432\\
72.79	0.00930407650468944\\
72.8	0.00930466783821347\\
72.81	0.00930525986784466\\
72.82	0.00930585261145006\\
72.83	0.0093064460877075\\
72.84	0.00930704031613882\\
72.85	0.00930763531714433\\
72.86	0.00930823111203865\\
72.87	0.00930882772308778\\
72.88	0.00930942517354772\\
72.89	0.00931002348770438\\
72.9	0.00931062269091515\\
72.91	0.00931122280965187\\
72.92	0.00931182387154558\\
72.93	0.00931242590543278\\
72.94	0.00931302894140351\\
72.95	0.00931363301085125\\
72.96	0.00931423814652455\\
72.97	0.00931484438258072\\
72.98	0.00931545175464145\\
72.99	0.00931606028722278\\
73	0.00931667000180168\\
73.01	0.00931728092044048\\
73.02	0.00931789306580369\\
73.03	0.00931850646117528\\
73.04	0.0093191211304765\\
73.05	0.00931973709828419\\
73.06	0.0093203543898497\\
73.07	0.00932097303111824\\
73.08	0.00932159304874899\\
73.09	0.00932221447013562\\
73.1	0.00932283732342754\\
73.11	0.00932346163755179\\
73.12	0.00932408744223544\\
73.13	0.00932471476802885\\
73.14	0.00932534364632948\\
73.15	0.00932597410940646\\
73.16	0.0093266061904259\\
73.17	0.00932723992347696\\
73.18	0.00932787534359865\\
73.19	0.00932851248680752\\
73.2	0.00932915139012606\\
73.21	0.00932979209161207\\
73.22	0.00933043463038884\\
73.23	0.00933107904667622\\
73.24	0.0093317253818227\\
73.25	0.00933237367833838\\
73.26	0.00933302397992894\\
73.27	0.00933367633153068\\
73.28	0.00933433077934655\\
73.29	0.00933498737088333\\
73.3	0.00933564615498984\\
73.31	0.0093363071818964\\
73.32	0.00933697050325543\\
73.33	0.00933763617218327\\
73.34	0.00933830424330327\\
73.35	0.0093389747727902\\
73.36	0.00933964781841597\\
73.37	0.00934032343959675\\
73.38	0.00934100169744157\\
73.39	0.00934168265480225\\
73.4	0.009342366376325\\
73.41	0.00934305292850351\\
73.42	0.00934374237973363\\
73.43	0.00934443480036976\\
73.44	0.00934513026278294\\
73.45	0.0093458288414207\\
73.46	0.00934653061286871\\
73.47	0.00934723565591436\\
73.48	0.0093479440516122\\
73.49	0.00934865588335147\\
73.5	0.00934937123692563\\
73.51	0.00935009020060398\\
73.52	0.00935081286520558\\
73.53	0.00935153932417531\\
73.54	0.0093522696736623\\
73.55	0.00935300401260083\\
73.56	0.00935374244279355\\
73.57	0.00935448506899743\\
73.58	0.00935523199901221\\
73.59	0.00935598334377162\\
73.6	0.00935673921743741\\
73.61	0.0093574997374962\\
73.62	0.00935826502485946\\
73.63	0.00935903520396634\\
73.64	0.00935981040288986\\
73.65	0.00936059075344633\\
73.66	0.00936137639130799\\
73.67	0.00936216745611937\\
73.68	0.00936296409161704\\
73.69	0.00936376644575315\\
73.7	0.00936457467082278\\
73.71	0.00936538892359519\\
73.72	0.00936620936544916\\
73.73	0.00936703616251252\\
73.74	0.009367869485806\\
73.75	0.00936870951139151\\
73.76	0.00936955642052507\\
73.77	0.00937041039981441\\
73.78	0.00937127164138158\\
73.79	0.00937214034303046\\
73.8	0.00937301670841966\\
73.81	0.0093739009472406\\
73.82	0.00937479327540129\\
73.83	0.00937569391521579\\
73.84	0.00937660309559953\\
73.85	0.0093775210522708\\
73.86	0.00937844802795843\\
73.87	0.00937938427261609\\
73.88	0.00938033004364318\\
73.89	0.00938128560611263\\
73.9	0.00938225123300595\\
73.91	0.00938322720545546\\
73.92	0.00938421381299429\\
73.93	0.0093852113538141\\
73.94	0.00938622013503094\\
73.95	0.00938724047295944\\
73.96	0.00938827269339564\\
73.97	0.00938931713190872\\
73.98	0.00939037413414188\\
73.99	0.00939144405612271\\
74	0.00939252726458335\\
74.01	0.00939362413729078\\
74.02	0.00939473506338749\\
74.03	0.00939586044374288\\
74.04	0.00939700069131593\\
74.05	0.00939815623152912\\
74.06	0.00939932750265439\\
74.07	0.00940051495621119\\
74.08	0.00940171905737727\\
74.09	0.0094029402854124\\
74.1	0.0094041634172478\\
74.11	0.00940538697201502\\
74.12	0.00940661094960099\\
74.13	0.00940783534988611\\
74.14	0.0094090601727443\\
74.15	0.00941028541804323\\
74.16	0.00941151108564439\\
74.17	0.00941273717540331\\
74.18	0.00941396368716972\\
74.19	0.00941519062078782\\
74.2	0.00941641797609645\\
74.21	0.00941764575292941\\
74.22	0.00941887395111572\\
74.23	0.00942010257047994\\
74.24	0.00942133161084253\\
74.25	0.00942256107202022\\
74.26	0.00942379095382639\\
74.27	0.00942502125607159\\
74.28	0.00942625197856393\\
74.29	0.00942748312110964\\
74.3	0.00942871468351361\\
74.31	0.00942994666558003\\
74.32	0.00943117906711296\\
74.33	0.00943241188791706\\
74.34	0.00943364512779832\\
74.35	0.00943487878656485\\
74.36	0.00943611286402771\\
74.37	0.00943734736000177\\
74.38	0.00943858227430675\\
74.39	0.00943981760676813\\
74.4	0.00944105335721831\\
74.41	0.0094422895254977\\
74.42	0.00944352611145598\\
74.43	0.00944476311495336\\
74.44	0.00944600053586197\\
74.45	0.00944723837406729\\
74.46	0.00944847662946969\\
74.47	0.00944971530198607\\
74.48	0.00945095439155155\\
74.49	0.00945219389812132\\
74.5	0.00945343382167255\\
74.51	0.00945467416220638\\
74.52	0.00945591491975008\\
74.53	0.00945715609435934\\
74.54	0.0094583976861206\\
74.55	0.00945963969515356\\
74.56	0.00946088212161384\\
74.57	0.00946212496569575\\
74.58	0.00946336822763523\\
74.59	0.00946461190771288\\
74.6	0.00946585600625725\\
74.61	0.00946710052364824\\
74.62	0.00946834546032065\\
74.63	0.00946959081676796\\
74.64	0.00947083659354628\\
74.65	0.00947208279127851\\
74.66	0.00947332941065867\\
74.67	0.00947457645245649\\
74.68	0.00947582391752222\\
74.69	0.00947707180679161\\
74.7	0.00947832012129128\\
74.71	0.00947956886214417\\
74.72	0.00948081803057536\\
74.73	0.0094820676279182\\
74.74	0.0094833176556206\\
74.75	0.00948456811525179\\
74.76	0.00948581900850922\\
74.77	0.00948707033722595\\
74.78	0.00948832210337828\\
74.79	0.00948957430909374\\
74.8	0.00949082695665953\\
74.81	0.00949208004788666\\
74.82	0.00949333358308714\\
74.83	0.00949458756257265\\
74.84	0.0094958419866545\\
74.85	0.00949709685564353\\
74.86	0.00949835216985011\\
74.87	0.00949960792958402\\
74.88	0.00950086413515444\\
74.89	0.00950212078686985\\
74.9	0.009503377885038\\
74.91	0.00950463542996581\\
74.92	0.00950589342195935\\
74.93	0.00950715186132368\\
74.94	0.00950841074836289\\
74.95	0.00950967008337993\\
74.96	0.0095109298666766\\
74.97	0.00951219009855341\\
74.98	0.00951345077930955\\
74.99	0.00951471190924277\\
75	0.00951597348864933\\
75.01	0.00951723551782385\\
75.02	0.00951849799705929\\
75.03	0.00951976092664679\\
75.04	0.00952102430687562\\
75.05	0.00952228813803304\\
75.06	0.00952355242040422\\
75.07	0.00952481715427215\\
75.08	0.00952608233991747\\
75.09	0.00952734797761843\\
75.1	0.00952861406765072\\
75.11	0.00952988061028737\\
75.12	0.00953114760579863\\
75.13	0.00953241505445184\\
75.14	0.00953368295651129\\
75.15	0.00953495131223811\\
75.16	0.0095362201218901\\
75.17	0.00953748938572164\\
75.18	0.00953875910398349\\
75.19	0.00954002927692264\\
75.2	0.00954129990478225\\
75.21	0.00954257098780135\\
75.22	0.00954384252621481\\
75.23	0.00954511452025308\\
75.24	0.00954638697014208\\
75.25	0.00954765987610301\\
75.26	0.00954893323835214\\
75.27	0.00955020705710067\\
75.28	0.00955148133255453\\
75.29	0.00955275606491418\\
75.3	0.00955403125437439\\
75.31	0.00955530690112409\\
75.32	0.00955658300534611\\
75.33	0.009557859567217\\
75.34	0.00955913658690681\\
75.35	0.00956041406457882\\
75.36	0.00956169200038938\\
75.37	0.00956297039448762\\
75.38	0.00956424924701523\\
75.39	0.00956552855810622\\
75.4	0.00956680832788665\\
75.41	0.00956808855647436\\
75.42	0.00956936924397874\\
75.43	0.00957065039050043\\
75.44	0.00957193199613104\\
75.45	0.00957321406095289\\
75.46	0.00957449658503866\\
75.47	0.00957577956845117\\
75.48	0.00957706301124301\\
75.49	0.00957834691345624\\
75.5	0.00957963127512208\\
75.51	0.00958091609626058\\
75.52	0.00958220137688028\\
75.53	0.00958348711697785\\
75.54	0.00958477331653776\\
75.55	0.0095860599755319\\
75.56	0.00958734709391923\\
75.57	0.0095886346716454\\
75.58	0.00958992270864232\\
75.59	0.00959121120482784\\
75.6	0.00959250016010529\\
75.61	0.00959378957436309\\
75.62	0.00959507944747433\\
75.63	0.00959636977929631\\
75.64	0.00959766056967014\\
75.65	0.00959895181842027\\
75.66	0.00960024352535403\\
75.67	0.00960153569026116\\
75.68	0.00960282831291333\\
75.69	0.00960412139306365\\
75.7	0.00960541493044619\\
75.71	0.00960670892477541\\
75.72	0.00960800337574573\\
75.73	0.0096092982830309\\
75.74	0.00961059364628352\\
75.75	0.00961188946513448\\
75.76	0.00961318573919235\\
75.77	0.00961448246804285\\
75.78	0.00961577965124825\\
75.79	0.00961707728834676\\
75.8	0.0096183753788519\\
75.81	0.00961967392225191\\
75.82	0.00962097291800908\\
75.83	0.00962227236555915\\
75.84	0.00962357226431058\\
75.85	0.00962487261364391\\
75.86	0.00962617341291109\\
75.87	0.00962747466143474\\
75.88	0.00962877635850747\\
75.89	0.00963007850339112\\
75.9	0.00963138109531604\\
75.91	0.00963268413348035\\
75.92	0.00963398761704913\\
75.93	0.00963529154515368\\
75.94	0.0096365959168907\\
75.95	0.00963790073132148\\
75.96	0.00963920598747111\\
75.97	0.00964051168432758\\
75.98	0.00964181782084098\\
75.99	0.00964312439592263\\
76	0.00964443140844419\\
76.01	0.00964573885723674\\
76.02	0.0096470467410899\\
76.03	0.00964835505875091\\
76.04	0.0096496638089237\\
76.05	0.00965097299026788\\
76.06	0.00965228260139784\\
76.07	0.00965359264088172\\
76.08	0.00965490310724046\\
76.09	0.00965621399894673\\
76.1	0.00965752531442397\\
76.11	0.00965883705204531\\
76.12	0.0096601492101325\\
76.13	0.00966146178695489\\
76.14	0.00966277478072833\\
76.15	0.00966408818961404\\
76.16	0.00966540201171755\\
76.17	0.00966671624508756\\
76.18	0.00966803088771478\\
76.19	0.00966934593753081\\
76.2	0.00967066139240695\\
76.21	0.00967197725015308\\
76.22	0.00967329350851638\\
76.23	0.00967461016518023\\
76.24	0.00967592721776291\\
76.25	0.00967724466381645\\
76.26	0.00967856250082537\\
76.27	0.0096798807262054\\
76.28	0.0096811993373023\\
76.29	0.00968251833139052\\
76.3	0.009683837705672\\
76.31	0.00968515745727482\\
76.32	0.009686477583252\\
76.33	0.00968779808058012\\
76.34	0.00968911894615806\\
76.35	0.00969044017680572\\
76.36	0.00969176176926267\\
76.37	0.00969308372018685\\
76.38	0.00969440602615328\\
76.39	0.00969572868365272\\
76.4	0.00969705168909035\\
76.41	0.00969837503878451\\
76.42	0.00969969872896529\\
76.43	0.00970102275577334\\
76.44	0.00970234711525847\\
76.45	0.00970367180337842\\
76.46	0.00970499681599753\\
76.47	0.00970632214888548\\
76.48	0.00970764779771604\\
76.49	0.0097089737580658\\
76.5	0.00971030002541291\\
76.51	0.00971162659513593\\
76.52	0.00971295346251254\\
76.53	0.00971428062271844\\
76.54	0.00971560807082616\\
76.55	0.00971693580180394\\
76.56	0.00971826381051462\\
76.57	0.0097195920917146\\
76.58	0.0097209206400528\\
76.59	0.00972224945006966\\
76.6	0.00972357851619621\\
76.61	0.00972490783275313\\
76.62	0.00972623739394993\\
76.63	0.00972756719388414\\
76.64	0.00972889722654053\\
76.65	0.00973022748579046\\
76.66	0.00973155796539124\\
76.67	0.00973288865898555\\
76.68	0.009734219560101\\
76.69	0.00973555066214972\\
76.7	0.00973688195842799\\
76.71	0.00973821344211606\\
76.72	0.00973954510627802\\
76.73	0.00974087694386169\\
76.74	0.00974220894769879\\
76.75	0.00974354111050501\\
76.76	0.00974487342488042\\
76.77	0.0097462058833098\\
76.78	0.00974753847816329\\
76.79	0.009748871201697\\
76.8	0.00975020404605394\\
76.81	0.00975153700326498\\
76.82	0.00975287006525004\\
76.83	0.00975420322381945\\
76.84	0.00975553647067543\\
76.85	0.00975686979741389\\
76.86	0.00975820319552632\\
76.87	0.00975953665640196\\
76.88	0.00976087017133012\\
76.89	0.00976220373150286\\
76.9	0.0097635373280178\\
76.91	0.00976487095188129\\
76.92	0.00976620459401174\\
76.93	0.00976753824524338\\
76.94	0.00976887189633022\\
76.95	0.00977020553795035\\
76.96	0.00977153916071058\\
76.97	0.00977287275515149\\
76.98	0.00977420631175271\\
76.99	0.00977553982093873\\
77	0.00977687327308501\\
77.01	0.00977820665852457\\
77.02	0.00977953996755498\\
77.03	0.00978087319044584\\
77.04	0.00978220631744675\\
77.05	0.00978353933879575\\
77.06	0.0097848722447283\\
77.07	0.00978620502548684\\
77.08	0.00978753767133089\\
77.09	0.0097888701725478\\
77.1	0.00979020251946405\\
77.11	0.0097915347024573\\
77.12	0.00979286671196906\\
77.13	0.00979419853851812\\
77.14	0.00979553017271469\\
77.15	0.00979686160527538\\
77.16	0.00979819282703893\\
77.17	0.00979952382898284\\
77.18	0.00980085460224085\\
77.19	0.00980218513812139\\
77.2	0.00980351542812699\\
77.21	0.00980484546397464\\
77.22	0.00980617523761727\\
77.23	0.00980750474126632\\
77.24	0.00980883396741539\\
77.25	0.00981016290886515\\
77.26	0.00981149155874948\\
77.27	0.00981281991056284\\
77.28	0.0098141479581891\\
77.29	0.00981547569593162\\
77.3	0.00981680311854496\\
77.31	0.00981813021953688\\
77.32	0.00981945699004156\\
77.33	0.00982078342151098\\
77.34	0.00982210950574259\\
77.35	0.00982343523490834\\
77.36	0.00982476060158512\\
77.37	0.00982608559878664\\
77.38	0.00982741021999688\\
77.39	0.00982873445920517\\
77.4	0.00983005831094282\\
77.41	0.00983138177032162\\
77.42	0.00983270483307404\\
77.43	0.00983402749559531\\
77.44	0.0098353497549875\\
77.45	0.00983667160910558\\
77.46	0.00983799305660562\\
77.47	0.00983931409699511\\
77.48	0.00984063473068567\\
77.49	0.00984195495904804\\
77.5	0.00984327478446954\\
77.51	0.00984459421041416\\
77.52	0.0098459132414852\\
77.53	0.00984723188349079\\
77.54	0.00984855014351221\\
77.55	0.00984986802997521\\
77.56	0.00985118555272445\\
77.57	0.00985250272310116\\
77.58	0.00985381955402416\\
77.59	0.0098551360600743\\
77.6	0.00985645225758261\\
77.61	0.00985776816472215\\
77.62	0.00985908379925234\\
77.63	0.00986039913948631\\
77.64	0.00986171416311329\\
77.65	0.00986302884718038\\
77.66	0.00986434316807359\\
77.67	0.00986565710149849\\
77.68	0.00986697062246007\\
77.69	0.00986828370524218\\
77.7	0.00986959632338625\\
77.71	0.00987090844966935\\
77.72	0.00987222005608166\\
77.73	0.00987353111380317\\
77.74	0.00987484159317973\\
77.75	0.00987615146369837\\
77.76	0.00987746069396183\\
77.77	0.00987876925166233\\
77.78	0.00988007710355458\\
77.79	0.00988138421542792\\
77.8	0.00988269055207759\\
77.81	0.00988399607727524\\
77.82	0.00988530075373837\\
77.83	0.00988660454309896\\
77.84	0.00988790740587112\\
77.85	0.00988920930141769\\
77.86	0.0098905101879159\\
77.87	0.00989181002232189\\
77.88	0.00989310876033424\\
77.89	0.00989440635635629\\
77.9	0.00989570276345735\\
77.91	0.00989699793333271\\
77.92	0.00989829181626241\\
77.93	0.00989958436106877\\
77.94	0.0099008755150726\\
77.95	0.00990216522404798\\
77.96	0.00990345343217582\\
77.97	0.00990474008199581\\
77.98	0.00990602511435701\\
77.99	0.00990730846836683\\
78	0.00990859008133851\\
78.01	0.0099098698887369\\
78.02	0.00991114782412263\\
78.03	0.00991242381909448\\
78.04	0.00991369780323006\\
78.05	0.00991496970402456\\
78.06	0.00991623944682764\\
78.07	0.00991750695477838\\
78.08	0.00991877214873819\\
78.09	0.00992003494722158\\
78.1	0.00992129526632494\\
78.11	0.00992255301965292\\
78.12	0.00992380811824262\\
78.13	0.00992506047048541\\
78.14	0.00992630998204631\\
78.15	0.00992755655578084\\
78.16	0.00992880009164932\\
78.17	0.00993004048662842\\
78.18	0.00993127763462008\\
78.19	0.0099325114263574\\
78.2	0.0099337417493078\\
78.21	0.00993496848757299\\
78.22	0.00993619152178588\\
78.23	0.00993741072900428\\
78.24	0.00993862598260124\\
78.25	0.00993983715215196\\
78.26	0.00994104410331717\\
78.27	0.00994224669772287\\
78.28	0.00994344479283625\\
78.29	0.0099446382418378\\
78.3	0.00994582689348928\\
78.31	0.0099470105919977\\
78.32	0.00994818917687491\\
78.33	0.00994936248279285\\
78.34	0.00995053033943415\\
78.35	0.00995169257133822\\
78.36	0.00995284899774228\\
78.37	0.00995399943241754\\
78.38	0.00995514368350016\\
78.39	0.00995628155331692\\
78.4	0.00995741283820537\\
78.41	0.00995853732832832\\
78.42	0.00995965480748252\\
78.43	0.0099607650529012\\
78.44	0.00996186783505056\\
78.45	0.00996296291741966\\
78.46	0.00996405005630379\\
78.47	0.00996512900058097\\
78.48	0.00996619949148137\\
78.49	0.0099672612623495\\
78.5	0.00996831403839884\\
78.51	0.00996935753645872\\
78.52	0.00997039146471322\\
78.53	0.00997141552243183\\
78.54	0.00997242939969153\\
78.55	0.00997343277709014\\
78.56	0.0099744253254506\\
78.57	0.00997540670551581\\
78.58	0.00997637656763397\\
78.59	0.00997733455143376\\
78.6	0.00997828028548945\\
78.61	0.00997921338697523\\
78.62	0.00998013346130866\\
78.63	0.00998104010178291\\
78.64	0.00998193288918719\\
78.65	0.00998281139141529\\
78.66	0.00998367516306167\\
78.67	0.00998452374500475\\
78.68	0.00998535666397702\\
78.69	0.00998617343212148\\
78.7	0.00998697354653412\\
78.71	0.00998775648879181\\
78.72	0.0099885217244653\\
78.73	0.00998926870261676\\
78.74	0.00998999685528138\\
78.75	0.00999070559693263\\
78.76	0.00999139432393051\\
78.77	0.00999206241395232\\
78.78	0.0099927092254055\\
78.79	0.00999333409682176\\
78.8	0.00999393634623212\\
78.81	0.00999451527052212\\
78.82	0.00999507014476665\\
78.83	0.00999560022154375\\
78.84	0.00999610473022666\\
78.85	0.00999658287625354\\
78.86	0.00999703384037408\\
78.87	0.0099974567778723\\
78.88	0.00999785081776483\\
78.89	0.00999821506197382\\
78.9	0.00999854858447381\\
78.91	0.00999885043041162\\
78.92	0.00999911961519857\\
78.93	0.00999935512357399\\
78.94	0.0099995559086393\\
78.95	0.00999972089086164\\
78.96	0.00999984895704617\\
78.97	0.00999993895927606\\
78.98	0.00999998971381908\\
78.99	0.01\\
79	0.01\\
79.01	0.01\\
79.02	0.01\\
79.03	0.01\\
79.04	0.01\\
79.05	0.01\\
79.06	0.01\\
79.07	0.01\\
79.08	0.01\\
79.09	0.01\\
79.1	0.01\\
79.11	0.01\\
79.12	0.01\\
79.13	0.01\\
79.14	0.01\\
79.15	0.01\\
79.16	0.01\\
79.17	0.01\\
79.18	0.01\\
79.19	0.01\\
79.2	0.01\\
79.21	0.01\\
79.22	0.01\\
79.23	0.01\\
79.24	0.01\\
79.25	0.01\\
79.26	0.01\\
79.27	0.01\\
79.28	0.01\\
79.29	0.01\\
79.3	0.01\\
79.31	0.01\\
79.32	0.01\\
79.33	0.01\\
79.34	0.01\\
79.35	0.01\\
79.36	0.01\\
79.37	0.01\\
79.38	0.01\\
79.39	0.01\\
79.4	0.01\\
79.41	0.01\\
79.42	0.01\\
79.43	0.01\\
79.44	0.01\\
79.45	0.01\\
79.46	0.01\\
79.47	0.01\\
79.48	0.01\\
79.49	0.01\\
79.5	0.01\\
79.51	0.01\\
79.52	0.01\\
79.53	0.01\\
79.54	0.01\\
79.55	0.01\\
79.56	0.01\\
79.57	0.01\\
79.58	0.01\\
79.59	0.01\\
79.6	0.01\\
79.61	0.01\\
79.62	0.01\\
79.63	0.01\\
79.64	0.01\\
79.65	0.01\\
79.66	0.01\\
79.67	0.01\\
79.68	0.01\\
79.69	0.01\\
79.7	0.01\\
79.71	0.01\\
79.72	0.01\\
79.73	0.01\\
79.74	0.01\\
79.75	0.01\\
79.76	0.01\\
79.77	0.01\\
79.78	0.01\\
79.79	0.01\\
79.8	0.01\\
79.81	0.01\\
79.82	0.01\\
79.83	0.01\\
79.84	0.01\\
79.85	0.01\\
79.86	0.01\\
79.87	0.01\\
79.88	0.01\\
79.89	0.01\\
79.9	0.01\\
79.91	0.01\\
79.92	0.01\\
79.93	0.01\\
79.94	0.01\\
79.95	0.01\\
79.96	0.01\\
79.97	0.01\\
79.98	0.01\\
79.99	0.01\\
80	0.01\\
80.01	0.01\\
};
\addplot [color=red,solid]
  table[row sep=crcr]{%
80.01	0.01\\
80.02	0.01\\
80.03	0.01\\
80.04	0.01\\
80.05	0.01\\
80.06	0.01\\
80.07	0.01\\
80.08	0.01\\
80.09	0.01\\
80.1	0.01\\
80.11	0.01\\
80.12	0.01\\
80.13	0.01\\
80.14	0.01\\
80.15	0.01\\
80.16	0.01\\
80.17	0.01\\
80.18	0.01\\
80.19	0.01\\
80.2	0.01\\
80.21	0.01\\
80.22	0.01\\
80.23	0.01\\
80.24	0.01\\
80.25	0.01\\
80.26	0.01\\
80.27	0.01\\
80.28	0.01\\
80.29	0.01\\
80.3	0.01\\
80.31	0.01\\
80.32	0.01\\
80.33	0.01\\
80.34	0.01\\
80.35	0.01\\
80.36	0.01\\
80.37	0.01\\
80.38	0.01\\
80.39	0.01\\
80.4	0.01\\
80.41	0.01\\
80.42	0.01\\
80.43	0.01\\
80.44	0.01\\
80.45	0.01\\
80.46	0.01\\
80.47	0.01\\
80.48	0.01\\
80.49	0.01\\
80.5	0.01\\
80.51	0.01\\
80.52	0.01\\
80.53	0.01\\
80.54	0.01\\
80.55	0.01\\
80.56	0.01\\
80.57	0.01\\
80.58	0.01\\
80.59	0.01\\
80.6	0.01\\
80.61	0.01\\
80.62	0.01\\
80.63	0.01\\
80.64	0.01\\
80.65	0.01\\
80.66	0.01\\
80.67	0.01\\
80.68	0.01\\
80.69	0.01\\
80.7	0.01\\
80.71	0.01\\
80.72	0.01\\
80.73	0.01\\
80.74	0.01\\
80.75	0.01\\
80.76	0.01\\
80.77	0.01\\
80.78	0.01\\
80.79	0.01\\
80.8	0.01\\
80.81	0.01\\
80.82	0.01\\
80.83	0.01\\
80.84	0.01\\
80.85	0.01\\
80.86	0.01\\
80.87	0.01\\
80.88	0.01\\
80.89	0.01\\
80.9	0.01\\
80.91	0.01\\
80.92	0.01\\
80.93	0.01\\
80.94	0.01\\
80.95	0.01\\
80.96	0.01\\
80.97	0.01\\
80.98	0.01\\
80.99	0.01\\
81	0.01\\
81.01	0.01\\
81.02	0.01\\
81.03	0.01\\
81.04	0.01\\
81.05	0.01\\
81.06	0.01\\
81.07	0.01\\
81.08	0.01\\
81.09	0.01\\
81.1	0.01\\
81.11	0.01\\
81.12	0.01\\
81.13	0.01\\
81.14	0.01\\
81.15	0.01\\
81.16	0.01\\
81.17	0.01\\
81.18	0.01\\
81.19	0.01\\
81.2	0.01\\
81.21	0.01\\
81.22	0.01\\
81.23	0.01\\
81.24	0.01\\
81.25	0.01\\
81.26	0.01\\
81.27	0.01\\
81.28	0.01\\
81.29	0.01\\
81.3	0.01\\
81.31	0.01\\
81.32	0.01\\
81.33	0.01\\
81.34	0.01\\
81.35	0.01\\
81.36	0.01\\
81.37	0.01\\
81.38	0.01\\
81.39	0.01\\
81.4	0.01\\
81.41	0.01\\
81.42	0.01\\
81.43	0.01\\
81.44	0.01\\
81.45	0.01\\
81.46	0.01\\
81.47	0.01\\
81.48	0.01\\
81.49	0.01\\
81.5	0.01\\
81.51	0.01\\
81.52	0.01\\
81.53	0.01\\
81.54	0.01\\
81.55	0.01\\
81.56	0.01\\
81.57	0.01\\
81.58	0.01\\
81.59	0.01\\
81.6	0.01\\
81.61	0.01\\
81.62	0.01\\
81.63	0.01\\
81.64	0.01\\
81.65	0.01\\
81.66	0.01\\
81.67	0.01\\
81.68	0.01\\
81.69	0.01\\
81.7	0.01\\
81.71	0.01\\
81.72	0.01\\
81.73	0.01\\
81.74	0.01\\
81.75	0.01\\
81.76	0.01\\
81.77	0.01\\
81.78	0.01\\
81.79	0.01\\
81.8	0.01\\
81.81	0.01\\
81.82	0.01\\
81.83	0.01\\
81.84	0.01\\
81.85	0.01\\
81.86	0.01\\
81.87	0.01\\
81.88	0.01\\
81.89	0.01\\
81.9	0.01\\
81.91	0.01\\
81.92	0.01\\
81.93	0.01\\
81.94	0.01\\
81.95	0.01\\
81.96	0.01\\
81.97	0.01\\
81.98	0.01\\
81.99	0.01\\
82	0.01\\
82.01	0.01\\
82.02	0.01\\
82.03	0.01\\
82.04	0.01\\
82.05	0.01\\
82.06	0.01\\
82.07	0.01\\
82.08	0.01\\
82.09	0.01\\
82.1	0.01\\
82.11	0.01\\
82.12	0.01\\
82.13	0.01\\
82.14	0.01\\
82.15	0.01\\
82.16	0.01\\
82.17	0.01\\
82.18	0.01\\
82.19	0.01\\
82.2	0.01\\
82.21	0.01\\
82.22	0.01\\
82.23	0.01\\
82.24	0.01\\
82.25	0.01\\
82.26	0.01\\
82.27	0.01\\
82.28	0.01\\
82.29	0.01\\
82.3	0.01\\
82.31	0.01\\
82.32	0.01\\
82.33	0.01\\
82.34	0.01\\
82.35	0.01\\
82.36	0.01\\
82.37	0.01\\
82.38	0.01\\
82.39	0.01\\
82.4	0.01\\
82.41	0.01\\
82.42	0.01\\
82.43	0.01\\
82.44	0.01\\
82.45	0.01\\
82.46	0.01\\
82.47	0.01\\
82.48	0.01\\
82.49	0.01\\
82.5	0.01\\
82.51	0.01\\
82.52	0.01\\
82.53	0.01\\
82.54	0.01\\
82.55	0.01\\
82.56	0.01\\
82.57	0.01\\
82.58	0.01\\
82.59	0.01\\
82.6	0.01\\
82.61	0.01\\
82.62	0.01\\
82.63	0.01\\
82.64	0.01\\
82.65	0.01\\
82.66	0.01\\
82.67	0.01\\
82.68	0.01\\
82.69	0.01\\
82.7	0.01\\
82.71	0.01\\
82.72	0.01\\
82.73	0.01\\
82.74	0.01\\
82.75	0.01\\
82.76	0.01\\
82.77	0.01\\
82.78	0.01\\
82.79	0.01\\
82.8	0.01\\
82.81	0.01\\
82.82	0.01\\
82.83	0.01\\
82.84	0.01\\
82.85	0.01\\
82.86	0.01\\
82.87	0.01\\
82.88	0.01\\
82.89	0.01\\
82.9	0.01\\
82.91	0.01\\
82.92	0.01\\
82.93	0.01\\
82.94	0.01\\
82.95	0.01\\
82.96	0.01\\
82.97	0.01\\
82.98	0.01\\
82.99	0.01\\
83	0.01\\
83.01	0.01\\
83.02	0.01\\
83.03	0.01\\
83.04	0.01\\
83.05	0.01\\
83.06	0.01\\
83.07	0.01\\
83.08	0.01\\
83.09	0.01\\
83.1	0.01\\
83.11	0.01\\
83.12	0.01\\
83.13	0.01\\
83.14	0.01\\
83.15	0.01\\
83.16	0.01\\
83.17	0.01\\
83.18	0.01\\
83.19	0.01\\
83.2	0.01\\
83.21	0.01\\
83.22	0.01\\
83.23	0.01\\
83.24	0.01\\
83.25	0.01\\
83.26	0.01\\
83.27	0.01\\
83.28	0.01\\
83.29	0.01\\
83.3	0.01\\
83.31	0.01\\
83.32	0.01\\
83.33	0.01\\
83.34	0.01\\
83.35	0.01\\
83.36	0.01\\
83.37	0.01\\
83.38	0.01\\
83.39	0.01\\
83.4	0.01\\
83.41	0.01\\
83.42	0.01\\
83.43	0.01\\
83.44	0.01\\
83.45	0.01\\
83.46	0.01\\
83.47	0.01\\
83.48	0.01\\
83.49	0.01\\
83.5	0.01\\
83.51	0.01\\
83.52	0.01\\
83.53	0.01\\
83.54	0.01\\
83.55	0.01\\
83.56	0.01\\
83.57	0.01\\
83.58	0.01\\
83.59	0.01\\
83.6	0.01\\
83.61	0.01\\
83.62	0.01\\
83.63	0.01\\
83.64	0.01\\
83.65	0.01\\
83.66	0.01\\
83.67	0.01\\
83.68	0.01\\
83.69	0.01\\
83.7	0.01\\
83.71	0.01\\
83.72	0.01\\
83.73	0.01\\
83.74	0.01\\
83.75	0.01\\
83.76	0.01\\
83.77	0.01\\
83.78	0.01\\
83.79	0.01\\
83.8	0.01\\
83.81	0.01\\
83.82	0.01\\
83.83	0.01\\
83.84	0.01\\
83.85	0.01\\
83.86	0.01\\
83.87	0.01\\
83.88	0.01\\
83.89	0.01\\
83.9	0.01\\
83.91	0.01\\
83.92	0.01\\
83.93	0.01\\
83.94	0.01\\
83.95	0.01\\
83.96	0.01\\
83.97	0.01\\
83.98	0.01\\
83.99	0.01\\
84	0.01\\
84.01	0.01\\
84.02	0.01\\
84.03	0.01\\
84.04	0.01\\
84.05	0.01\\
84.06	0.01\\
84.07	0.01\\
84.08	0.01\\
84.09	0.01\\
84.1	0.01\\
84.11	0.01\\
84.12	0.01\\
84.13	0.01\\
84.14	0.01\\
84.15	0.01\\
84.16	0.01\\
84.17	0.01\\
84.18	0.01\\
84.19	0.01\\
84.2	0.01\\
84.21	0.01\\
84.22	0.01\\
84.23	0.01\\
84.24	0.01\\
84.25	0.01\\
84.26	0.01\\
84.27	0.01\\
84.28	0.01\\
84.29	0.01\\
84.3	0.01\\
84.31	0.01\\
84.32	0.01\\
84.33	0.01\\
84.34	0.01\\
84.35	0.01\\
84.36	0.01\\
84.37	0.01\\
84.38	0.01\\
84.39	0.01\\
84.4	0.01\\
84.41	0.01\\
84.42	0.01\\
84.43	0.01\\
84.44	0.01\\
84.45	0.01\\
84.46	0.01\\
84.47	0.01\\
84.48	0.01\\
84.49	0.01\\
84.5	0.01\\
84.51	0.01\\
84.52	0.01\\
84.53	0.01\\
84.54	0.01\\
84.55	0.01\\
84.56	0.01\\
84.57	0.01\\
84.58	0.01\\
84.59	0.01\\
84.6	0.01\\
84.61	0.01\\
84.62	0.01\\
84.63	0.01\\
84.64	0.01\\
84.65	0.01\\
84.66	0.01\\
84.67	0.01\\
84.68	0.01\\
84.69	0.01\\
84.7	0.01\\
84.71	0.01\\
84.72	0.01\\
84.73	0.01\\
84.74	0.01\\
84.75	0.01\\
84.76	0.01\\
84.77	0.01\\
84.78	0.01\\
84.79	0.01\\
84.8	0.01\\
84.81	0.01\\
84.82	0.01\\
84.83	0.01\\
84.84	0.01\\
84.85	0.01\\
84.86	0.01\\
84.87	0.01\\
84.88	0.01\\
84.89	0.01\\
84.9	0.01\\
84.91	0.01\\
84.92	0.01\\
84.93	0.01\\
84.94	0.01\\
84.95	0.01\\
84.96	0.01\\
84.97	0.01\\
84.98	0.01\\
84.99	0.01\\
85	0.01\\
85.01	0.01\\
85.02	0.01\\
85.03	0.01\\
85.04	0.01\\
85.05	0.01\\
85.06	0.01\\
85.07	0.01\\
85.08	0.01\\
85.09	0.01\\
85.1	0.01\\
85.11	0.01\\
85.12	0.01\\
85.13	0.01\\
85.14	0.01\\
85.15	0.01\\
85.16	0.01\\
85.17	0.01\\
85.18	0.01\\
85.19	0.01\\
85.2	0.01\\
85.21	0.01\\
85.22	0.01\\
85.23	0.01\\
85.24	0.01\\
85.25	0.01\\
85.26	0.01\\
85.27	0.01\\
85.28	0.01\\
85.29	0.01\\
85.3	0.01\\
85.31	0.01\\
85.32	0.01\\
85.33	0.01\\
85.34	0.01\\
85.35	0.01\\
85.36	0.01\\
85.37	0.01\\
85.38	0.01\\
85.39	0.01\\
85.4	0.01\\
85.41	0.01\\
85.42	0.01\\
85.43	0.01\\
85.44	0.01\\
85.45	0.01\\
85.46	0.01\\
85.47	0.01\\
85.48	0.01\\
85.49	0.01\\
85.5	0.01\\
85.51	0.01\\
85.52	0.01\\
85.53	0.01\\
85.54	0.01\\
85.55	0.01\\
85.56	0.01\\
85.57	0.01\\
85.58	0.01\\
85.59	0.01\\
85.6	0.01\\
85.61	0.01\\
85.62	0.01\\
85.63	0.01\\
85.64	0.01\\
85.65	0.01\\
85.66	0.01\\
85.67	0.01\\
85.68	0.01\\
85.69	0.01\\
85.7	0.01\\
85.71	0.01\\
85.72	0.01\\
85.73	0.01\\
85.74	0.01\\
85.75	0.01\\
85.76	0.01\\
85.77	0.01\\
85.78	0.01\\
85.79	0.01\\
85.8	0.01\\
85.81	0.01\\
85.82	0.01\\
85.83	0.01\\
85.84	0.01\\
85.85	0.01\\
85.86	0.01\\
85.87	0.01\\
85.88	0.01\\
85.89	0.01\\
85.9	0.01\\
85.91	0.01\\
85.92	0.01\\
85.93	0.01\\
85.94	0.01\\
85.95	0.01\\
85.96	0.01\\
85.97	0.01\\
85.98	0.01\\
85.99	0.01\\
86	0.01\\
86.01	0.01\\
86.02	0.01\\
86.03	0.01\\
86.04	0.01\\
86.05	0.01\\
86.06	0.01\\
86.07	0.01\\
86.08	0.01\\
86.09	0.01\\
86.1	0.01\\
86.11	0.01\\
86.12	0.01\\
86.13	0.01\\
86.14	0.01\\
86.15	0.01\\
86.16	0.01\\
86.17	0.01\\
86.18	0.01\\
86.19	0.01\\
86.2	0.01\\
86.21	0.01\\
86.22	0.01\\
86.23	0.01\\
86.24	0.01\\
86.25	0.01\\
86.26	0.01\\
86.27	0.01\\
86.28	0.01\\
86.29	0.01\\
86.3	0.01\\
86.31	0.01\\
86.32	0.01\\
86.33	0.01\\
86.34	0.01\\
86.35	0.01\\
86.36	0.01\\
86.37	0.01\\
86.38	0.01\\
86.39	0.01\\
86.4	0.01\\
86.41	0.01\\
86.42	0.01\\
86.43	0.01\\
86.44	0.01\\
86.45	0.01\\
86.46	0.01\\
86.47	0.01\\
86.48	0.01\\
86.49	0.01\\
86.5	0.01\\
86.51	0.01\\
86.52	0.01\\
86.53	0.01\\
86.54	0.01\\
86.55	0.01\\
86.56	0.01\\
86.57	0.01\\
86.58	0.01\\
86.59	0.01\\
86.6	0.01\\
86.61	0.01\\
86.62	0.01\\
86.63	0.01\\
86.64	0.01\\
86.65	0.01\\
86.66	0.01\\
86.67	0.01\\
86.68	0.01\\
86.69	0.01\\
86.7	0.01\\
86.71	0.01\\
86.72	0.01\\
86.73	0.01\\
86.74	0.01\\
86.75	0.01\\
86.76	0.01\\
86.77	0.01\\
86.78	0.01\\
86.79	0.01\\
86.8	0.01\\
86.81	0.01\\
86.82	0.01\\
86.83	0.01\\
86.84	0.01\\
86.85	0.01\\
86.86	0.01\\
86.87	0.01\\
86.88	0.01\\
86.89	0.01\\
86.9	0.01\\
86.91	0.01\\
86.92	0.01\\
86.93	0.01\\
86.94	0.01\\
86.95	0.01\\
86.96	0.01\\
86.97	0.01\\
86.98	0.01\\
86.99	0.01\\
87	0.01\\
87.01	0.01\\
87.02	0.01\\
87.03	0.01\\
87.04	0.01\\
87.05	0.01\\
87.06	0.01\\
87.07	0.01\\
87.08	0.01\\
87.09	0.01\\
87.1	0.01\\
87.11	0.01\\
87.12	0.01\\
87.13	0.01\\
87.14	0.01\\
87.15	0.01\\
87.16	0.01\\
87.17	0.01\\
87.18	0.01\\
87.19	0.01\\
87.2	0.01\\
87.21	0.01\\
87.22	0.01\\
87.23	0.01\\
87.24	0.01\\
87.25	0.01\\
87.26	0.01\\
87.27	0.01\\
87.28	0.01\\
87.29	0.01\\
87.3	0.01\\
87.31	0.01\\
87.32	0.01\\
87.33	0.01\\
87.34	0.01\\
87.35	0.01\\
87.36	0.01\\
87.37	0.01\\
87.38	0.01\\
87.39	0.01\\
87.4	0.01\\
87.41	0.01\\
87.42	0.01\\
87.43	0.01\\
87.44	0.01\\
87.45	0.01\\
87.46	0.01\\
87.47	0.01\\
87.48	0.01\\
87.49	0.01\\
87.5	0.01\\
87.51	0.01\\
87.52	0.01\\
87.53	0.01\\
87.54	0.01\\
87.55	0.01\\
87.56	0.01\\
87.57	0.01\\
87.58	0.01\\
87.59	0.01\\
87.6	0.01\\
87.61	0.01\\
87.62	0.01\\
87.63	0.01\\
87.64	0.01\\
87.65	0.01\\
87.66	0.01\\
87.67	0.01\\
87.68	0.01\\
87.69	0.01\\
87.7	0.01\\
87.71	0.01\\
87.72	0.01\\
87.73	0.01\\
87.74	0.01\\
87.75	0.01\\
87.76	0.01\\
87.77	0.01\\
87.78	0.01\\
87.79	0.01\\
87.8	0.01\\
87.81	0.01\\
87.82	0.01\\
87.83	0.01\\
87.84	0.01\\
87.85	0.01\\
87.86	0.01\\
87.87	0.01\\
87.88	0.01\\
87.89	0.01\\
87.9	0.01\\
87.91	0.01\\
87.92	0.01\\
87.93	0.01\\
87.94	0.01\\
87.95	0.01\\
87.96	0.01\\
87.97	0.01\\
87.98	0.01\\
87.99	0.01\\
88	0.01\\
88.01	0.01\\
88.02	0.01\\
88.03	0.01\\
88.04	0.01\\
88.05	0.01\\
88.06	0.01\\
88.07	0.01\\
88.08	0.01\\
88.09	0.01\\
88.1	0.01\\
88.11	0.01\\
88.12	0.01\\
88.13	0.01\\
88.14	0.01\\
88.15	0.01\\
88.16	0.01\\
88.17	0.01\\
88.18	0.01\\
88.19	0.01\\
88.2	0.01\\
88.21	0.01\\
88.22	0.01\\
88.23	0.01\\
88.24	0.01\\
88.25	0.01\\
88.26	0.01\\
88.27	0.01\\
88.28	0.01\\
88.29	0.01\\
88.3	0.01\\
88.31	0.01\\
88.32	0.01\\
88.33	0.01\\
88.34	0.01\\
88.35	0.01\\
88.36	0.01\\
88.37	0.01\\
88.38	0.01\\
88.39	0.01\\
88.4	0.01\\
88.41	0.01\\
88.42	0.01\\
88.43	0.01\\
88.44	0.01\\
88.45	0.01\\
88.46	0.01\\
88.47	0.01\\
88.48	0.01\\
88.49	0.01\\
88.5	0.01\\
88.51	0.01\\
88.52	0.01\\
88.53	0.01\\
88.54	0.01\\
88.55	0.01\\
88.56	0.01\\
88.57	0.01\\
88.58	0.01\\
88.59	0.01\\
88.6	0.01\\
88.61	0.01\\
88.62	0.01\\
88.63	0.01\\
88.64	0.01\\
88.65	0.01\\
88.66	0.01\\
88.67	0.01\\
88.68	0.01\\
88.69	0.01\\
88.7	0.01\\
88.71	0.01\\
88.72	0.01\\
88.73	0.01\\
88.74	0.01\\
88.75	0.01\\
88.76	0.01\\
88.77	0.01\\
88.78	0.01\\
88.79	0.01\\
88.8	0.01\\
88.81	0.01\\
88.82	0.01\\
88.83	0.01\\
88.84	0.01\\
88.85	0.01\\
88.86	0.01\\
88.87	0.01\\
88.88	0.01\\
88.89	0.01\\
88.9	0.01\\
88.91	0.01\\
88.92	0.01\\
88.93	0.01\\
88.94	0.01\\
88.95	0.01\\
88.96	0.01\\
88.97	0.01\\
88.98	0.01\\
88.99	0.01\\
89	0.01\\
89.01	0.01\\
89.02	0.01\\
89.03	0.01\\
89.04	0.01\\
89.05	0.01\\
89.06	0.01\\
89.07	0.01\\
89.08	0.01\\
89.09	0.01\\
89.1	0.01\\
89.11	0.01\\
89.12	0.01\\
89.13	0.01\\
89.14	0.01\\
89.15	0.01\\
89.16	0.01\\
89.17	0.01\\
89.18	0.01\\
89.19	0.01\\
89.2	0.01\\
89.21	0.01\\
89.22	0.01\\
89.23	0.01\\
89.24	0.01\\
89.25	0.01\\
89.26	0.01\\
89.27	0.01\\
89.28	0.01\\
89.29	0.01\\
89.3	0.01\\
89.31	0.01\\
89.32	0.01\\
89.33	0.01\\
89.34	0.01\\
89.35	0.01\\
89.36	0.01\\
89.37	0.01\\
89.38	0.01\\
89.39	0.01\\
89.4	0.01\\
89.41	0.01\\
89.42	0.01\\
89.43	0.01\\
89.44	0.01\\
89.45	0.01\\
89.46	0.01\\
89.47	0.01\\
89.48	0.01\\
89.49	0.01\\
89.5	0.01\\
89.51	0.01\\
89.52	0.01\\
89.53	0.01\\
89.54	0.01\\
89.55	0.01\\
89.56	0.01\\
89.57	0.01\\
89.58	0.01\\
89.59	0.01\\
89.6	0.01\\
89.61	0.01\\
89.62	0.01\\
89.63	0.01\\
89.64	0.01\\
89.65	0.01\\
89.66	0.01\\
89.67	0.01\\
89.68	0.01\\
89.69	0.01\\
89.7	0.01\\
89.71	0.01\\
89.72	0.01\\
89.73	0.01\\
89.74	0.01\\
89.75	0.01\\
89.76	0.01\\
89.77	0.01\\
89.78	0.01\\
89.79	0.01\\
89.8	0.01\\
89.81	0.01\\
89.82	0.01\\
89.83	0.01\\
89.84	0.01\\
89.85	0.01\\
89.86	0.01\\
89.87	0.01\\
89.88	0.01\\
89.89	0.01\\
89.9	0.01\\
89.91	0.01\\
89.92	0.01\\
89.93	0.01\\
89.94	0.01\\
89.95	0.01\\
89.96	0.01\\
89.97	0.01\\
89.98	0.01\\
89.99	0.01\\
90	0.01\\
90.01	0.01\\
90.02	0.01\\
90.03	0.01\\
90.04	0.01\\
90.05	0.01\\
90.06	0.01\\
90.07	0.01\\
90.08	0.01\\
90.09	0.01\\
90.1	0.01\\
90.11	0.01\\
90.12	0.01\\
90.13	0.01\\
90.14	0.01\\
90.15	0.01\\
90.16	0.01\\
90.17	0.01\\
90.18	0.01\\
90.19	0.01\\
90.2	0.01\\
90.21	0.01\\
90.22	0.01\\
90.23	0.01\\
90.24	0.01\\
90.25	0.01\\
90.26	0.01\\
90.27	0.01\\
90.28	0.01\\
90.29	0.01\\
90.3	0.01\\
90.31	0.01\\
90.32	0.01\\
90.33	0.01\\
90.34	0.01\\
90.35	0.01\\
90.36	0.01\\
90.37	0.01\\
90.38	0.01\\
90.39	0.01\\
90.4	0.01\\
90.41	0.01\\
90.42	0.01\\
90.43	0.01\\
90.44	0.01\\
90.45	0.01\\
90.46	0.01\\
90.47	0.01\\
90.48	0.01\\
90.49	0.01\\
90.5	0.01\\
90.51	0.01\\
90.52	0.01\\
90.53	0.01\\
90.54	0.01\\
90.55	0.01\\
90.56	0.01\\
90.57	0.01\\
90.58	0.01\\
90.59	0.01\\
90.6	0.01\\
90.61	0.01\\
90.62	0.01\\
90.63	0.01\\
90.64	0.01\\
90.65	0.01\\
90.66	0.01\\
90.67	0.01\\
90.68	0.01\\
90.69	0.01\\
90.7	0.01\\
90.71	0.01\\
90.72	0.01\\
90.73	0.01\\
90.74	0.01\\
90.75	0.01\\
90.76	0.01\\
90.77	0.01\\
90.78	0.01\\
90.79	0.01\\
90.8	0.01\\
90.81	0.01\\
90.82	0.01\\
90.83	0.01\\
90.84	0.01\\
90.85	0.01\\
90.86	0.01\\
90.87	0.01\\
90.88	0.01\\
90.89	0.01\\
90.9	0.01\\
90.91	0.01\\
90.92	0.01\\
90.93	0.01\\
90.94	0.01\\
90.95	0.01\\
90.96	0.01\\
90.97	0.01\\
90.98	0.01\\
90.99	0.01\\
91	0.01\\
91.01	0.01\\
91.02	0.01\\
91.03	0.01\\
91.04	0.01\\
91.05	0.01\\
91.06	0.01\\
91.07	0.01\\
91.08	0.01\\
91.09	0.01\\
91.1	0.01\\
91.11	0.01\\
91.12	0.01\\
91.13	0.01\\
91.14	0.01\\
91.15	0.01\\
91.16	0.01\\
91.17	0.01\\
91.18	0.01\\
91.19	0.01\\
91.2	0.01\\
91.21	0.01\\
91.22	0.01\\
91.23	0.01\\
91.24	0.01\\
91.25	0.01\\
91.26	0.01\\
91.27	0.01\\
91.28	0.01\\
91.29	0.01\\
91.3	0.01\\
91.31	0.01\\
91.32	0.01\\
91.33	0.01\\
91.34	0.01\\
91.35	0.01\\
91.36	0.01\\
91.37	0.01\\
91.38	0.01\\
91.39	0.01\\
91.4	0.01\\
91.41	0.01\\
91.42	0.01\\
91.43	0.01\\
91.44	0.01\\
91.45	0.01\\
91.46	0.01\\
91.47	0.01\\
91.48	0.01\\
91.49	0.01\\
91.5	0.01\\
91.51	0.01\\
91.52	0.01\\
91.53	0.01\\
91.54	0.01\\
91.55	0.01\\
91.56	0.01\\
91.57	0.01\\
91.58	0.01\\
91.59	0.01\\
91.6	0.01\\
91.61	0.01\\
91.62	0.01\\
91.63	0.01\\
91.64	0.01\\
91.65	0.01\\
91.66	0.01\\
91.67	0.01\\
91.68	0.01\\
91.69	0.01\\
91.7	0.01\\
91.71	0.01\\
91.72	0.01\\
91.73	0.01\\
91.74	0.01\\
91.75	0.01\\
91.76	0.01\\
91.77	0.01\\
91.78	0.01\\
91.79	0.01\\
91.8	0.01\\
91.81	0.01\\
91.82	0.01\\
91.83	0.01\\
91.84	0.01\\
91.85	0.01\\
91.86	0.01\\
91.87	0.01\\
91.88	0.01\\
91.89	0.01\\
91.9	0.01\\
91.91	0.01\\
91.92	0.01\\
91.93	0.01\\
91.94	0.01\\
91.95	0.01\\
91.96	0.01\\
91.97	0.01\\
91.98	0.01\\
91.99	0.01\\
92	0.01\\
92.01	0.01\\
92.02	0.01\\
92.03	0.01\\
92.04	0.01\\
92.05	0.01\\
92.06	0.01\\
92.07	0.01\\
92.08	0.01\\
92.09	0.01\\
92.1	0.01\\
92.11	0.01\\
92.12	0.01\\
92.13	0.01\\
92.14	0.01\\
92.15	0.01\\
92.16	0.01\\
92.17	0.01\\
92.18	0.01\\
92.19	0.01\\
92.2	0.01\\
92.21	0.01\\
92.22	0.01\\
92.23	0.01\\
92.24	0.01\\
92.25	0.01\\
92.26	0.01\\
92.27	0.01\\
92.28	0.01\\
92.29	0.01\\
92.3	0.01\\
92.31	0.01\\
92.32	0.01\\
92.33	0.01\\
92.34	0.01\\
92.35	0.01\\
92.36	0.01\\
92.37	0.01\\
92.38	0.01\\
92.39	0.01\\
92.4	0.01\\
92.41	0.01\\
92.42	0.01\\
92.43	0.01\\
92.44	0.01\\
92.45	0.01\\
92.46	0.01\\
92.47	0.01\\
92.48	0.01\\
92.49	0.01\\
92.5	0.01\\
92.51	0.01\\
92.52	0.01\\
92.53	0.01\\
92.54	0.01\\
92.55	0.01\\
92.56	0.01\\
92.57	0.01\\
92.58	0.01\\
92.59	0.01\\
92.6	0.01\\
92.61	0.01\\
92.62	0.01\\
92.63	0.01\\
92.64	0.01\\
92.65	0.01\\
92.66	0.01\\
92.67	0.01\\
92.68	0.01\\
92.69	0.01\\
92.7	0.01\\
92.71	0.01\\
92.72	0.01\\
92.73	0.01\\
92.74	0.01\\
92.75	0.01\\
92.76	0.01\\
92.77	0.01\\
92.78	0.01\\
92.79	0.01\\
92.8	0.01\\
92.81	0.01\\
92.82	0.01\\
92.83	0.01\\
92.84	0.01\\
92.85	0.01\\
92.86	0.01\\
92.87	0.01\\
92.88	0.01\\
92.89	0.01\\
92.9	0.01\\
92.91	0.01\\
92.92	0.01\\
92.93	0.01\\
92.94	0.01\\
92.95	0.01\\
92.96	0.01\\
92.97	0.01\\
92.98	0.01\\
92.99	0.01\\
93	0.01\\
93.01	0.01\\
93.02	0.01\\
93.03	0.01\\
93.04	0.01\\
93.05	0.01\\
93.06	0.01\\
93.07	0.01\\
93.08	0.01\\
93.09	0.01\\
93.1	0.01\\
93.11	0.01\\
93.12	0.01\\
93.13	0.01\\
93.14	0.01\\
93.15	0.01\\
93.16	0.01\\
93.17	0.01\\
93.18	0.01\\
93.19	0.01\\
93.2	0.01\\
93.21	0.01\\
93.22	0.01\\
93.23	0.01\\
93.24	0.01\\
93.25	0.01\\
93.26	0.01\\
93.27	0.01\\
93.28	0.01\\
93.29	0.01\\
93.3	0.01\\
93.31	0.01\\
93.32	0.01\\
93.33	0.01\\
93.34	0.01\\
93.35	0.01\\
93.36	0.01\\
93.37	0.01\\
93.38	0.01\\
93.39	0.01\\
93.4	0.01\\
93.41	0.01\\
93.42	0.01\\
93.43	0.01\\
93.44	0.01\\
93.45	0.01\\
93.46	0.01\\
93.47	0.01\\
93.48	0.01\\
93.49	0.01\\
93.5	0.01\\
93.51	0.01\\
93.52	0.01\\
93.53	0.01\\
93.54	0.01\\
93.55	0.01\\
93.56	0.01\\
93.57	0.01\\
93.58	0.01\\
93.59	0.01\\
93.6	0.01\\
93.61	0.01\\
93.62	0.01\\
93.63	0.01\\
93.64	0.01\\
93.65	0.01\\
93.66	0.01\\
93.67	0.01\\
93.68	0.01\\
93.69	0.01\\
93.7	0.01\\
93.71	0.01\\
93.72	0.01\\
93.73	0.01\\
93.74	0.01\\
93.75	0.01\\
93.76	0.01\\
93.77	0.01\\
93.78	0.01\\
93.79	0.01\\
93.8	0.01\\
93.81	0.01\\
93.82	0.01\\
93.83	0.01\\
93.84	0.01\\
93.85	0.01\\
93.86	0.01\\
93.87	0.01\\
93.88	0.01\\
93.89	0.01\\
93.9	0.01\\
93.91	0.01\\
93.92	0.01\\
93.93	0.01\\
93.94	0.01\\
93.95	0.01\\
93.96	0.01\\
93.97	0.01\\
93.98	0.01\\
93.99	0.01\\
94	0.01\\
94.01	0.01\\
94.02	0.01\\
94.03	0.01\\
94.04	0.01\\
94.05	0.01\\
94.06	0.01\\
94.07	0.01\\
94.08	0.01\\
94.09	0.01\\
94.1	0.01\\
94.11	0.01\\
94.12	0.01\\
94.13	0.01\\
94.14	0.01\\
94.15	0.01\\
94.16	0.01\\
94.17	0.01\\
94.18	0.01\\
94.19	0.01\\
94.2	0.01\\
94.21	0.01\\
94.22	0.01\\
94.23	0.01\\
94.24	0.01\\
94.25	0.01\\
94.26	0.01\\
94.27	0.01\\
94.28	0.01\\
94.29	0.01\\
94.3	0.01\\
94.31	0.01\\
94.32	0.01\\
94.33	0.01\\
94.34	0.01\\
94.35	0.01\\
94.36	0.01\\
94.37	0.01\\
94.38	0.01\\
94.39	0.01\\
94.4	0.01\\
94.41	0.01\\
94.42	0.01\\
94.43	0.01\\
94.44	0.01\\
94.45	0.01\\
94.46	0.01\\
94.47	0.01\\
94.48	0.01\\
94.49	0.01\\
94.5	0.01\\
94.51	0.01\\
94.52	0.01\\
94.53	0.01\\
94.54	0.01\\
94.55	0.01\\
94.56	0.01\\
94.57	0.01\\
94.58	0.01\\
94.59	0.01\\
94.6	0.01\\
94.61	0.01\\
94.62	0.01\\
94.63	0.01\\
94.64	0.01\\
94.65	0.01\\
94.66	0.01\\
94.67	0.01\\
94.68	0.01\\
94.69	0.01\\
94.7	0.01\\
94.71	0.01\\
94.72	0.01\\
94.73	0.01\\
94.74	0.01\\
94.75	0.01\\
94.76	0.01\\
94.77	0.01\\
94.78	0.01\\
94.79	0.01\\
94.8	0.01\\
94.81	0.01\\
94.82	0.01\\
94.83	0.01\\
94.84	0.01\\
94.85	0.01\\
94.86	0.01\\
94.87	0.01\\
94.88	0.01\\
94.89	0.01\\
94.9	0.01\\
94.91	0.01\\
94.92	0.01\\
94.93	0.01\\
94.94	0.01\\
94.95	0.01\\
94.96	0.01\\
94.97	0.01\\
94.98	0.01\\
94.99	0.01\\
95	0.01\\
95.01	0.01\\
95.02	0.01\\
95.03	0.01\\
95.04	0.01\\
95.05	0.01\\
95.06	0.01\\
95.07	0.01\\
95.08	0.01\\
95.09	0.01\\
95.1	0.01\\
95.11	0.01\\
95.12	0.01\\
95.13	0.01\\
95.14	0.01\\
95.15	0.01\\
95.16	0.01\\
95.17	0.01\\
95.18	0.01\\
95.19	0.01\\
95.2	0.01\\
95.21	0.01\\
95.22	0.01\\
95.23	0.01\\
95.24	0.01\\
95.25	0.01\\
95.26	0.01\\
95.27	0.01\\
95.28	0.01\\
95.29	0.01\\
95.3	0.01\\
95.31	0.01\\
95.32	0.01\\
95.33	0.01\\
95.34	0.01\\
95.35	0.01\\
95.36	0.01\\
95.37	0.01\\
95.38	0.01\\
95.39	0.01\\
95.4	0.01\\
95.41	0.01\\
95.42	0.01\\
95.43	0.01\\
95.44	0.01\\
95.45	0.01\\
95.46	0.01\\
95.47	0.01\\
95.48	0.01\\
95.49	0.01\\
95.5	0.01\\
95.51	0.01\\
95.52	0.01\\
95.53	0.01\\
95.54	0.01\\
95.55	0.01\\
95.56	0.01\\
95.57	0.01\\
95.58	0.01\\
95.59	0.01\\
95.6	0.01\\
95.61	0.01\\
95.62	0.01\\
95.63	0.01\\
95.64	0.01\\
95.65	0.01\\
95.66	0.01\\
95.67	0.01\\
95.68	0.01\\
95.69	0.01\\
95.7	0.01\\
95.71	0.01\\
95.72	0.01\\
95.73	0.01\\
95.74	0.01\\
95.75	0.01\\
95.76	0.01\\
95.77	0.01\\
95.78	0.01\\
95.79	0.01\\
95.8	0.01\\
95.81	0.01\\
95.82	0.01\\
95.83	0.01\\
95.84	0.01\\
95.85	0.01\\
95.86	0.01\\
95.87	0.01\\
95.88	0.01\\
95.89	0.01\\
95.9	0.01\\
95.91	0.01\\
95.92	0.01\\
95.93	0.01\\
95.94	0.01\\
95.95	0.01\\
95.96	0.01\\
95.97	0.01\\
95.98	0.01\\
95.99	0.01\\
96	0.01\\
96.01	0.01\\
96.02	0.01\\
96.03	0.01\\
96.04	0.01\\
96.05	0.01\\
96.06	0.01\\
96.07	0.01\\
96.08	0.01\\
96.09	0.01\\
96.1	0.01\\
96.11	0.01\\
96.12	0.01\\
96.13	0.01\\
96.14	0.01\\
96.15	0.01\\
96.16	0.01\\
96.17	0.01\\
96.18	0.01\\
96.19	0.01\\
96.2	0.01\\
96.21	0.01\\
96.22	0.01\\
96.23	0.01\\
96.24	0.01\\
96.25	0.01\\
96.26	0.01\\
96.27	0.01\\
96.28	0.01\\
96.29	0.01\\
96.3	0.01\\
96.31	0.01\\
96.32	0.01\\
96.33	0.01\\
96.34	0.01\\
96.35	0.01\\
96.36	0.01\\
96.37	0.01\\
96.38	0.01\\
96.39	0.01\\
96.4	0.01\\
96.41	0.01\\
96.42	0.01\\
96.43	0.01\\
96.44	0.01\\
96.45	0.01\\
96.46	0.01\\
96.47	0.01\\
96.48	0.01\\
96.49	0.01\\
96.5	0.01\\
96.51	0.01\\
96.52	0.01\\
96.53	0.01\\
96.54	0.01\\
96.55	0.01\\
96.56	0.01\\
96.57	0.01\\
96.58	0.01\\
96.59	0.01\\
96.6	0.01\\
96.61	0.01\\
96.62	0.01\\
96.63	0.01\\
96.64	0.01\\
96.65	0.01\\
96.66	0.01\\
96.67	0.01\\
96.68	0.01\\
96.69	0.01\\
96.7	0.01\\
96.71	0.01\\
96.72	0.01\\
96.73	0.01\\
96.74	0.01\\
96.75	0.01\\
96.76	0.01\\
96.77	0.01\\
96.78	0.01\\
96.79	0.01\\
96.8	0.01\\
96.81	0.01\\
96.82	0.01\\
96.83	0.01\\
96.84	0.01\\
96.85	0.01\\
96.86	0.01\\
96.87	0.01\\
96.88	0.01\\
96.89	0.01\\
96.9	0.01\\
96.91	0.01\\
96.92	0.01\\
96.93	0.01\\
96.94	0.01\\
96.95	0.01\\
96.96	0.01\\
96.97	0.01\\
96.98	0.01\\
96.99	0.01\\
97	0.01\\
97.01	0.01\\
97.02	0.01\\
97.03	0.01\\
97.04	0.01\\
97.05	0.01\\
97.06	0.01\\
97.07	0.01\\
97.08	0.01\\
97.09	0.01\\
97.1	0.01\\
97.11	0.01\\
97.12	0.01\\
97.13	0.01\\
97.14	0.01\\
97.15	0.01\\
97.16	0.01\\
97.17	0.01\\
97.18	0.01\\
97.19	0.01\\
97.2	0.01\\
97.21	0.01\\
97.22	0.01\\
97.23	0.01\\
97.24	0.01\\
97.25	0.01\\
97.26	0.01\\
97.27	0.01\\
97.28	0.01\\
97.29	0.01\\
97.3	0.01\\
97.31	0.01\\
97.32	0.01\\
97.33	0.01\\
97.34	0.01\\
97.35	0.01\\
97.36	0.01\\
97.37	0.01\\
97.38	0.01\\
97.39	0.01\\
97.4	0.01\\
97.41	0.01\\
97.42	0.01\\
97.43	0.01\\
97.44	0.01\\
97.45	0.01\\
97.46	0.01\\
97.47	0.01\\
97.48	0.01\\
97.49	0.01\\
97.5	0.01\\
97.51	0.01\\
97.52	0.01\\
97.53	0.01\\
97.54	0.01\\
97.55	0.01\\
97.56	0.01\\
97.57	0.01\\
97.58	0.01\\
97.59	0.01\\
97.6	0.01\\
97.61	0.01\\
97.62	0.01\\
97.63	0.01\\
97.64	0.01\\
97.65	0.01\\
97.66	0.01\\
97.67	0.01\\
97.68	0.01\\
97.69	0.01\\
97.7	0.01\\
97.71	0.01\\
97.72	0.01\\
97.73	0.01\\
97.74	0.01\\
97.75	0.01\\
97.76	0.01\\
97.77	0.01\\
97.78	0.01\\
97.79	0.01\\
97.8	0.01\\
97.81	0.01\\
97.82	0.01\\
97.83	0.01\\
97.84	0.01\\
97.85	0.01\\
97.86	0.01\\
97.87	0.01\\
97.88	0.01\\
97.89	0.01\\
97.9	0.01\\
97.91	0.01\\
97.92	0.01\\
97.93	0.01\\
97.94	0.01\\
97.95	0.01\\
97.96	0.01\\
97.97	0.01\\
97.98	0.01\\
97.99	0.01\\
98	0.01\\
98.01	0.01\\
98.02	0.01\\
98.03	0.01\\
98.04	0.01\\
98.05	0.01\\
98.06	0.01\\
98.07	0.01\\
98.08	0.01\\
98.09	0.01\\
98.1	0.01\\
98.11	0.01\\
98.12	0.01\\
98.13	0.01\\
98.14	0.01\\
98.15	0.01\\
98.16	0.01\\
98.17	0.01\\
98.18	0.01\\
98.19	0.01\\
98.2	0.01\\
98.21	0.01\\
98.22	0.01\\
98.23	0.01\\
98.24	0.01\\
98.25	0.01\\
98.26	0.01\\
98.27	0.01\\
98.28	0.01\\
98.29	0.01\\
98.3	0.01\\
98.31	0.01\\
98.32	0.01\\
98.33	0.01\\
98.34	0.01\\
98.35	0.01\\
98.36	0.01\\
98.37	0.01\\
98.38	0.01\\
98.39	0.01\\
98.4	0.01\\
98.41	0.01\\
98.42	0.01\\
98.43	0.01\\
98.44	0.01\\
98.45	0.01\\
98.46	0.01\\
98.47	0.01\\
98.48	0.01\\
98.49	0.01\\
98.5	0.01\\
98.51	0.01\\
98.52	0.01\\
98.53	0.01\\
98.54	0.01\\
98.55	0.01\\
98.56	0.01\\
98.57	0.01\\
98.58	0.01\\
98.59	0.01\\
98.6	0.01\\
98.61	0.01\\
98.62	0.01\\
98.63	0.01\\
98.64	0.01\\
98.65	0.01\\
98.66	0.01\\
98.67	0.01\\
98.68	0.01\\
98.69	0.01\\
98.7	0.01\\
98.71	0.01\\
98.72	0.01\\
98.73	0.01\\
98.74	0.01\\
98.75	0.01\\
98.76	0.01\\
98.77	0.01\\
98.78	0.01\\
98.79	0.01\\
98.8	0.01\\
98.81	0.01\\
98.82	0.01\\
98.83	0.01\\
98.84	0.01\\
98.85	0.01\\
98.86	0.01\\
98.87	0.01\\
98.88	0.01\\
98.89	0.01\\
98.9	0.01\\
98.91	0.01\\
98.92	0.01\\
98.93	0.01\\
98.94	0.01\\
98.95	0.01\\
98.96	0.01\\
98.97	0.01\\
98.98	0.01\\
98.99	0.01\\
99	0.01\\
99.01	0.01\\
99.02	0.01\\
99.03	0.01\\
99.04	0.01\\
99.05	0.01\\
99.06	0.01\\
99.07	0.01\\
99.08	0.01\\
99.09	0.01\\
99.1	0.01\\
99.11	0.01\\
99.12	0.01\\
99.13	0.01\\
99.14	0.01\\
99.15	0.01\\
99.16	0.01\\
99.17	0.01\\
99.18	0.01\\
99.19	0.01\\
99.2	0.01\\
99.21	0.01\\
99.22	0.01\\
99.23	0.01\\
99.24	0.01\\
99.25	0.01\\
99.26	0.01\\
99.27	0.01\\
99.28	0.01\\
99.29	0.01\\
99.3	0.01\\
99.31	0.01\\
99.32	0.01\\
99.33	0.01\\
99.34	0.01\\
99.35	0.01\\
99.36	0.01\\
99.37	0.01\\
99.38	0.01\\
99.39	0.01\\
99.4	0.01\\
99.41	0.01\\
99.42	0.01\\
99.43	0.01\\
99.44	0.01\\
99.45	0.01\\
99.46	0.01\\
99.47	0.01\\
99.48	0.01\\
99.49	0.01\\
99.5	0.01\\
99.51	0.01\\
99.52	0.01\\
99.53	0.01\\
99.54	0.01\\
99.55	0.01\\
99.56	0.01\\
99.57	0.01\\
99.58	0.01\\
99.59	0.01\\
99.6	0.01\\
99.61	0.01\\
99.62	0.01\\
99.63	0.01\\
99.64	0.01\\
99.65	0.01\\
99.66	0.01\\
99.67	0.01\\
99.68	0.01\\
99.69	0.01\\
99.7	0.01\\
99.71	0.01\\
99.72	0.01\\
99.73	0.01\\
99.74	0.01\\
99.75	0.01\\
99.76	0.01\\
99.77	0.01\\
99.78	0.01\\
99.79	0.01\\
99.8	0.01\\
99.81	0.01\\
99.82	0.01\\
99.83	0.01\\
99.84	0.01\\
99.85	0.01\\
99.86	0.01\\
99.87	0.01\\
99.88	0.01\\
99.89	0.01\\
99.9	0.01\\
99.91	0.01\\
99.92	0.01\\
99.93	0.01\\
99.94	0.01\\
99.95	0.01\\
99.96	0.01\\
99.97	0.01\\
99.98	0.01\\
99.99	0.01\\
100	0.01\\
};
\addlegendentry{$q=2$};

\addplot [color=mycolor1,solid,forget plot]
  table[row sep=crcr]{%
0.01	0.00956306055999642\\
0.02	0.00956307872228478\\
0.03	0.0095630968976805\\
0.04	0.00956311508582356\\
0.05	0.00956313328633\\
0.06	0.00956315149879088\\
0.07	0.00956316972277116\\
0.08	0.00956318795780866\\
0.09	0.00956320620341279\\
0.1	0.00956322445906338\\
0.11	0.00956324272420942\\
0.12	0.00956326099826776\\
0.13	0.0095632792806217\\
0.14	0.00956329757061965\\
0.15	0.0095633158675736\\
0.16	0.00956333417075766\\
0.17	0.00956335247940644\\
0.18	0.00956337079271344\\
0.19	0.00956338910982933\\
0.2	0.00956340742986023\\
0.21	0.00956342575186585\\
0.22	0.00956344407485762\\
0.23	0.0095634623977967\\
0.24	0.00956348071959199\\
0.25	0.00956349903909798\\
0.26	0.00956351735511256\\
0.27	0.00956353566637479\\
0.28	0.00956355397156251\\
0.29	0.00956357226928993\\
0.3	0.00956359055810508\\
0.31	0.00956360883648718\\
0.32	0.00956362710284395\\
0.33	0.0095636453555088\\
0.34	0.00956366359273788\\
0.35	0.00956368181270707\\
0.36	0.00956370001350885\\
0.37	0.00956371819314906\\
0.38	0.00956373634954351\\
0.39	0.0095637544805145\\
0.4	0.00956377258378723\\
0.41	0.00956379065698601\\
0.42	0.00956380869763044\\
0.43	0.00956382670313134\\
0.44	0.0095638446707866\\
0.45	0.00956386259777691\\
0.46	0.0095638804811612\\
0.47	0.00956389831787212\\
0.48	0.00956391610471117\\
0.49	0.00956393383834377\\
0.5	0.00956395151529413\\
0.51	0.00956396913193986\\
0.52	0.00956398668450656\\
0.53	0.00956400416906202\\
0.54	0.00956402158151035\\
0.55	0.00956403891758586\\
0.56	0.00956405617284671\\
0.57	0.00956407334266835\\
0.58	0.0095640904222367\\
0.59	0.00956410740654118\\
0.6	0.00956412429036737\\
0.61	0.00956414106828947\\
0.62	0.00956415773466253\\
0.63	0.00956417428361436\\
0.64	0.00956419070903714\\
0.65	0.0095642070045788\\
0.66	0.00956422316363404\\
0.67	0.00956423917933505\\
0.68	0.00956425504454195\\
0.69	0.00956427075183281\\
0.7	0.0095642862934934\\
0.71	0.00956430166150654\\
0.72	0.0095643168475411\\
0.73	0.0095643318429406\\
0.74	0.00956434673458947\\
0.75	0.0095643616319118\\
0.76	0.00956437653491001\\
0.77	0.0095643914435865\\
0.78	0.00956440635794366\\
0.79	0.00956442127798393\\
0.8	0.00956443620370971\\
0.81	0.00956445113512341\\
0.82	0.00956446607222744\\
0.83	0.00956448101502423\\
0.84	0.00956449596351619\\
0.85	0.00956451091770574\\
0.86	0.00956452587759529\\
0.87	0.00956454084318728\\
0.88	0.00956455581448411\\
0.89	0.00956457079148822\\
0.9	0.00956458577420203\\
0.91	0.00956460076262797\\
0.92	0.00956461575676846\\
0.93	0.00956463075662593\\
0.94	0.00956464576220282\\
0.95	0.00956466077350155\\
0.96	0.00956467579052456\\
0.97	0.00956469081327427\\
0.98	0.00956470584175314\\
0.99	0.00956472087596358\\
1	0.00956473591590805\\
1.01	0.00956475096158899\\
1.02	0.00956476601300882\\
1.03	0.00956478107017\\
1.04	0.00956479613307496\\
1.05	0.00956481120172616\\
1.06	0.00956482627612604\\
1.07	0.00956484135627704\\
1.08	0.00956485644218161\\
1.09	0.00956487153384222\\
1.1	0.00956488663126129\\
1.11	0.0095649017344413\\
1.12	0.00956491684338469\\
1.13	0.00956493195809392\\
1.14	0.00956494707857144\\
1.15	0.00956496220481972\\
1.16	0.00956497733684121\\
1.17	0.00956499247463837\\
1.18	0.00956500761821367\\
1.19	0.00956502276756957\\
1.2	0.00956503792270853\\
1.21	0.00956505308363303\\
1.22	0.00956506825034552\\
1.23	0.00956508342284848\\
1.24	0.00956509860114438\\
1.25	0.00956511378523568\\
1.26	0.00956512897512487\\
1.27	0.00956514417081441\\
1.28	0.00956515937230679\\
1.29	0.00956517457960447\\
1.3	0.00956518979270995\\
1.31	0.00956520501162568\\
1.32	0.00956522023635417\\
1.33	0.00956523546689788\\
1.34	0.00956525070325931\\
1.35	0.00956526594544094\\
1.36	0.00956528119344526\\
1.37	0.00956529644727475\\
1.38	0.0095653117069319\\
1.39	0.00956532697241921\\
1.4	0.00956534224373916\\
1.41	0.00956535752089425\\
1.42	0.00956537280388698\\
1.43	0.00956538809271983\\
1.44	0.00956540338739532\\
1.45	0.00956541868791592\\
1.46	0.00956543399428416\\
1.47	0.00956544930650252\\
1.48	0.00956546462457352\\
1.49	0.00956547994849964\\
1.5	0.00956549527828341\\
1.51	0.00956551061392733\\
1.52	0.00956552595543391\\
1.53	0.00956554130280565\\
1.54	0.00956555665604507\\
1.55	0.00956557201515468\\
1.56	0.00956558738013699\\
1.57	0.00956560275099453\\
1.58	0.00956561812772979\\
1.59	0.00956563351034532\\
1.6	0.00956564889884362\\
1.61	0.00956566429322722\\
1.62	0.00956567969349863\\
1.63	0.00956569509966039\\
1.64	0.00956571051171501\\
1.65	0.00956572592966502\\
1.66	0.00956574135351296\\
1.67	0.00956575678326135\\
1.68	0.00956577221891272\\
1.69	0.00956578766046959\\
1.7	0.00956580310793452\\
1.71	0.00956581856131003\\
1.72	0.00956583402059866\\
1.73	0.00956584948580295\\
1.74	0.00956586495692542\\
1.75	0.00956588043396864\\
1.76	0.00956589591693513\\
1.77	0.00956591140582744\\
1.78	0.00956592690064813\\
1.79	0.00956594240139972\\
1.8	0.00956595790808477\\
1.81	0.00956597342070583\\
1.82	0.00956598893926545\\
1.83	0.00956600446376617\\
1.84	0.00956601999421057\\
1.85	0.00956603553060118\\
1.86	0.00956605107294057\\
1.87	0.00956606662123129\\
1.88	0.0095660821754759\\
1.89	0.00956609773567696\\
1.9	0.00956611330183704\\
1.91	0.0095661288739587\\
1.92	0.0095661444520445\\
1.93	0.009566160036097\\
1.94	0.00956617562611878\\
1.95	0.0095661912221124\\
1.96	0.00956620682408043\\
1.97	0.00956622243202546\\
1.98	0.00956623804595004\\
1.99	0.00956625366585676\\
2	0.00956626929174819\\
2.01	0.0095662849236269\\
2.02	0.00956630056149548\\
2.03	0.00956631620535651\\
2.04	0.00956633185521257\\
2.05	0.00956634751106624\\
2.06	0.00956636317292011\\
2.07	0.00956637884077676\\
2.08	0.00956639451463878\\
2.09	0.00956641019450876\\
2.1	0.0095664258803893\\
2.11	0.00956644157228298\\
2.12	0.00956645727019239\\
2.13	0.00956647297412013\\
2.14	0.0095664886840688\\
2.15	0.009566504400041\\
2.16	0.00956652012203932\\
2.17	0.00956653585006636\\
2.18	0.00956655158412473\\
2.19	0.00956656732421704\\
2.2	0.00956658307034587\\
2.21	0.00956659882251385\\
2.22	0.00956661458072358\\
2.23	0.00956663034497768\\
2.24	0.00956664611527874\\
2.25	0.00956666189162939\\
2.26	0.00956667767403223\\
2.27	0.00956669346248988\\
2.28	0.00956670925700497\\
2.29	0.0095667250575801\\
2.3	0.0095667408642179\\
2.31	0.00956675667692099\\
2.32	0.00956677249569199\\
2.33	0.00956678832053352\\
2.34	0.00956680415144822\\
2.35	0.0095668199884387\\
2.36	0.0095668358315076\\
2.37	0.00956685168065754\\
2.38	0.00956686753589117\\
2.39	0.0095668833972111\\
2.4	0.00956689926461999\\
2.41	0.00956691513812045\\
2.42	0.00956693101771514\\
2.43	0.00956694690340668\\
2.44	0.00956696279519772\\
2.45	0.00956697869309091\\
2.46	0.00956699459708887\\
2.47	0.00956701050719427\\
2.48	0.00956702642340974\\
2.49	0.00956704234573793\\
2.5	0.0095670582741815\\
2.51	0.0095670742087431\\
2.52	0.00956709014942537\\
2.53	0.00956710609623097\\
2.54	0.00956712204916256\\
2.55	0.00956713800822278\\
2.56	0.00956715397341432\\
2.57	0.00956716994473981\\
2.58	0.00956718592220192\\
2.59	0.00956720190580332\\
2.6	0.00956721789554667\\
2.61	0.00956723389143464\\
2.62	0.00956724989346989\\
2.63	0.00956726590165509\\
2.64	0.00956728191599292\\
2.65	0.00956729793648605\\
2.66	0.00956731396313714\\
2.67	0.00956732999594887\\
2.68	0.00956734603492393\\
2.69	0.00956736208006499\\
2.7	0.00956737813137472\\
2.71	0.00956739418885581\\
2.72	0.00956741025251095\\
2.73	0.00956742632234281\\
2.74	0.00956744239835409\\
2.75	0.00956745848054746\\
2.76	0.00956747456892562\\
2.77	0.00956749066349126\\
2.78	0.00956750676424708\\
2.79	0.00956752287119575\\
2.8	0.00956753898433999\\
2.81	0.00956755510368248\\
2.82	0.00956757122922592\\
2.83	0.00956758736097302\\
2.84	0.00956760349892647\\
2.85	0.00956761964308898\\
2.86	0.00956763579346325\\
2.87	0.00956765195005198\\
2.88	0.00956766811285789\\
2.89	0.00956768428188369\\
2.9	0.00956770045713207\\
2.91	0.00956771663860576\\
2.92	0.00956773282630747\\
2.93	0.00956774902023991\\
2.94	0.0095677652204058\\
2.95	0.00956778142680787\\
2.96	0.00956779763944881\\
2.97	0.00956781385833137\\
2.98	0.00956783008345826\\
2.99	0.00956784631483221\\
3	0.00956786255245593\\
3.01	0.00956787879633217\\
3.02	0.00956789504646365\\
3.03	0.0095679113028531\\
3.04	0.00956792756550324\\
3.05	0.00956794383441682\\
3.06	0.00956796010959658\\
3.07	0.00956797639104524\\
3.08	0.00956799267876554\\
3.09	0.00956800897276024\\
3.1	0.00956802527303205\\
3.11	0.00956804157958375\\
3.12	0.00956805789241805\\
3.13	0.00956807421153772\\
3.14	0.0095680905369455\\
3.15	0.00956810686864414\\
3.16	0.00956812320663639\\
3.17	0.009568139550925\\
3.18	0.00956815590151273\\
3.19	0.00956817225840233\\
3.2	0.00956818862159656\\
3.21	0.00956820499109819\\
3.22	0.00956822136690996\\
3.23	0.00956823774903464\\
3.24	0.009568254137475\\
3.25	0.0095682705322338\\
3.26	0.00956828693331382\\
3.27	0.0095683033407178\\
3.28	0.00956831975444854\\
3.29	0.0095683361745088\\
3.3	0.00956835260090134\\
3.31	0.00956836903362896\\
3.32	0.00956838547269442\\
3.33	0.0095684019181005\\
3.34	0.00956841836984998\\
3.35	0.00956843482794565\\
3.36	0.00956845129239029\\
3.37	0.00956846776318667\\
3.38	0.00956848424033759\\
3.39	0.00956850072384584\\
3.4	0.00956851721371421\\
3.41	0.00956853370994548\\
3.42	0.00956855021254244\\
3.43	0.00956856672150791\\
3.44	0.00956858323684465\\
3.45	0.00956859975855549\\
3.46	0.00956861628664321\\
3.47	0.00956863282111062\\
3.48	0.00956864936196052\\
3.49	0.00956866590919571\\
3.5	0.009568682462819\\
3.51	0.0095686990228332\\
3.52	0.00956871558924111\\
3.53	0.00956873216204555\\
3.54	0.00956874874124933\\
3.55	0.00956876532685526\\
3.56	0.00956878191886615\\
3.57	0.00956879851728483\\
3.58	0.00956881512211412\\
3.59	0.00956883173335682\\
3.6	0.00956884835101578\\
3.61	0.0095688649750938\\
3.62	0.00956888160559372\\
3.63	0.00956889824251836\\
3.64	0.00956891488587055\\
3.65	0.00956893153565313\\
3.66	0.00956894819186892\\
3.67	0.00956896485452075\\
3.68	0.00956898152361146\\
3.69	0.0095689981991439\\
3.7	0.0095690148811209\\
3.71	0.00956903156954529\\
3.72	0.00956904826441992\\
3.73	0.00956906496574763\\
3.74	0.00956908167353128\\
3.75	0.0095690983877737\\
3.76	0.00956911510847774\\
3.77	0.00956913183564626\\
3.78	0.00956914856928211\\
3.79	0.00956916530938814\\
3.8	0.0095691820559672\\
3.81	0.00956919880902216\\
3.82	0.00956921556855586\\
3.83	0.00956923233457118\\
3.84	0.00956924910707097\\
3.85	0.0095692658860581\\
3.86	0.00956928267153543\\
3.87	0.00956929946350583\\
3.88	0.00956931626197217\\
3.89	0.00956933306693732\\
3.9	0.00956934987840414\\
3.91	0.00956936669637553\\
3.92	0.00956938352085434\\
3.93	0.00956940035184346\\
3.94	0.00956941718934577\\
3.95	0.00956943403336414\\
3.96	0.00956945088390146\\
3.97	0.00956946774096062\\
3.98	0.00956948460454449\\
3.99	0.00956950147465597\\
4	0.00956951835129794\\
4.01	0.0095695352344733\\
4.02	0.00956955212418494\\
4.03	0.00956956902043575\\
4.04	0.00956958592322863\\
4.05	0.00956960283256647\\
4.06	0.00956961974845218\\
4.07	0.00956963667088866\\
4.08	0.00956965359987881\\
4.09	0.00956967053542552\\
4.1	0.00956968747753172\\
4.11	0.00956970442620031\\
4.12	0.00956972138143419\\
4.13	0.00956973834323628\\
4.14	0.00956975531160949\\
4.15	0.00956977228655674\\
4.16	0.00956978926808094\\
4.17	0.00956980625618501\\
4.18	0.00956982325087187\\
4.19	0.00956984025214445\\
4.2	0.00956985726000566\\
4.21	0.00956987427445843\\
4.22	0.0095698912955057\\
4.23	0.00956990832315037\\
4.24	0.00956992535739539\\
4.25	0.0095699423982437\\
4.26	0.00956995944569822\\
4.27	0.00956997649976188\\
4.28	0.00956999356043763\\
4.29	0.00957001062772841\\
4.3	0.00957002770163715\\
4.31	0.00957004478216681\\
4.32	0.00957006186932031\\
4.33	0.00957007896310061\\
4.34	0.00957009606351066\\
4.35	0.00957011317055341\\
4.36	0.0095701302842318\\
4.37	0.0095701474045488\\
4.38	0.00957016453150735\\
4.39	0.00957018166511041\\
4.4	0.00957019880536095\\
4.41	0.00957021595226191\\
4.42	0.00957023310581627\\
4.43	0.00957025026602699\\
4.44	0.00957026743289703\\
4.45	0.00957028460642936\\
4.46	0.00957030178662695\\
4.47	0.00957031897349276\\
4.48	0.00957033616702978\\
4.49	0.00957035336724097\\
4.5	0.00957037057412932\\
4.51	0.0095703877876978\\
4.52	0.00957040500794939\\
4.53	0.00957042223488707\\
4.54	0.00957043946851382\\
4.55	0.00957045670883264\\
4.56	0.0095704739558465\\
4.57	0.0095704912095584\\
4.58	0.00957050846997132\\
4.59	0.00957052573708826\\
4.6	0.00957054301091221\\
4.61	0.00957056029144618\\
4.62	0.00957057757869315\\
4.63	0.00957059487265612\\
4.64	0.0095706121733381\\
4.65	0.00957062948074209\\
4.66	0.00957064679487109\\
4.67	0.00957066411572811\\
4.68	0.00957068144331617\\
4.69	0.00957069877763827\\
4.7	0.00957071611869741\\
4.71	0.00957073346649662\\
4.72	0.00957075082103892\\
4.73	0.00957076818232731\\
4.74	0.00957078555036482\\
4.75	0.00957080292515446\\
4.76	0.00957082030669928\\
4.77	0.00957083769500228\\
4.78	0.00957085509006649\\
4.79	0.00957087249189495\\
4.8	0.00957088990049068\\
4.81	0.00957090731585671\\
4.82	0.00957092473799608\\
4.83	0.00957094216691183\\
4.84	0.00957095960260699\\
4.85	0.0095709770450846\\
4.86	0.00957099449434771\\
4.87	0.00957101195039935\\
4.88	0.00957102941324257\\
4.89	0.00957104688288042\\
4.9	0.00957106435931595\\
4.91	0.0095710818425522\\
4.92	0.00957109933259224\\
4.93	0.00957111682943911\\
4.94	0.00957113433309587\\
4.95	0.00957115184356557\\
4.96	0.00957116936085129\\
4.97	0.00957118688495607\\
4.98	0.00957120441588298\\
4.99	0.0095712219536351\\
5	0.00957123949821548\\
5.01	0.00957125704962719\\
5.02	0.00957127460787332\\
5.03	0.00957129217295692\\
5.04	0.00957130974488107\\
5.05	0.00957132732364885\\
5.06	0.00957134490926334\\
5.07	0.00957136250172762\\
5.08	0.00957138010104477\\
5.09	0.00957139770721788\\
5.1	0.00957141532025002\\
5.11	0.00957143294014429\\
5.12	0.00957145056690379\\
5.13	0.00957146820053159\\
5.14	0.00957148584103079\\
5.15	0.00957150348840449\\
5.16	0.00957152114265579\\
5.17	0.00957153880378779\\
5.18	0.00957155647180358\\
5.19	0.00957157414670627\\
5.2	0.00957159182849897\\
5.21	0.00957160951718477\\
5.22	0.0095716272127668\\
5.23	0.00957164491524816\\
5.24	0.00957166262463195\\
5.25	0.0095716803409213\\
5.26	0.00957169806411933\\
5.27	0.00957171579422915\\
5.28	0.00957173353125388\\
5.29	0.00957175127519664\\
5.3	0.00957176902606056\\
5.31	0.00957178678384876\\
5.32	0.00957180454856438\\
5.33	0.00957182232021053\\
5.34	0.00957184009879035\\
5.35	0.00957185788430699\\
5.36	0.00957187567676356\\
5.37	0.00957189347616321\\
5.38	0.00957191128250908\\
5.39	0.00957192909580431\\
5.4	0.00957194691605204\\
5.41	0.00957196474325541\\
5.42	0.00957198257741758\\
5.43	0.00957200041854169\\
5.44	0.0095720182666309\\
5.45	0.00957203612168835\\
5.46	0.00957205398371721\\
5.47	0.00957207185272063\\
5.48	0.00957208972870176\\
5.49	0.00957210761166376\\
5.5	0.00957212550160981\\
5.51	0.00957214339854306\\
5.52	0.00957216130246669\\
5.53	0.00957217921338385\\
5.54	0.00957219713129773\\
5.55	0.00957221505621148\\
5.56	0.00957223298812829\\
5.57	0.00957225092705134\\
5.58	0.0095722688729838\\
5.59	0.00957228682592884\\
5.6	0.00957230478588966\\
5.61	0.00957232275286944\\
5.62	0.00957234072687137\\
5.63	0.00957235870789862\\
5.64	0.0095723766959544\\
5.65	0.00957239469104189\\
5.66	0.00957241269316429\\
5.67	0.00957243070232479\\
5.68	0.0095724487185266\\
5.69	0.0095724667417729\\
5.7	0.00957248477206691\\
5.71	0.00957250280941183\\
5.72	0.00957252085381086\\
5.73	0.00957253890526721\\
5.74	0.0095725569637841\\
5.75	0.00957257502936473\\
5.76	0.00957259310201232\\
5.77	0.00957261118173008\\
5.78	0.00957262926852123\\
5.79	0.00957264736238899\\
5.8	0.00957266546333659\\
5.81	0.00957268357136724\\
5.82	0.00957270168648417\\
5.83	0.00957271980869062\\
5.84	0.00957273793798981\\
5.85	0.00957275607438497\\
5.86	0.00957277421787934\\
5.87	0.00957279236847615\\
5.88	0.00957281052617864\\
5.89	0.00957282869099005\\
5.9	0.00957284686291363\\
5.91	0.00957286504195261\\
5.92	0.00957288322811025\\
5.93	0.00957290142138979\\
5.94	0.00957291962179448\\
5.95	0.00957293782932757\\
5.96	0.00957295604399232\\
5.97	0.00957297426579199\\
5.98	0.00957299249472983\\
5.99	0.0095730107308091\\
6	0.00957302897403307\\
6.01	0.009573047224405\\
6.02	0.00957306548192816\\
6.03	0.00957308374660581\\
6.04	0.00957310201844123\\
6.05	0.00957312029743768\\
6.06	0.00957313858359845\\
6.07	0.00957315687692681\\
6.08	0.00957317517742604\\
6.09	0.00957319348509942\\
6.1	0.00957321179995024\\
6.11	0.00957323012198176\\
6.12	0.0095732484511973\\
6.13	0.00957326678760013\\
6.14	0.00957328513119354\\
6.15	0.00957330348198084\\
6.16	0.0095733218399653\\
6.17	0.00957334020515025\\
6.18	0.00957335857753895\\
6.19	0.00957337695713474\\
6.2	0.0095733953439409\\
6.21	0.00957341373796074\\
6.22	0.00957343213919758\\
6.23	0.00957345054765471\\
6.24	0.00957346896333546\\
6.25	0.00957348738624313\\
6.26	0.00957350581638105\\
6.27	0.00957352425375252\\
6.28	0.00957354269836088\\
6.29	0.00957356115020944\\
6.3	0.00957357960930153\\
6.31	0.00957359807564048\\
6.32	0.00957361654922961\\
6.33	0.00957363503007226\\
6.34	0.00957365351817175\\
6.35	0.00957367201353143\\
6.36	0.00957369051615463\\
6.37	0.00957370902604469\\
6.38	0.00957372754320495\\
6.39	0.00957374606763876\\
6.4	0.00957376459934945\\
6.41	0.00957378313834039\\
6.42	0.00957380168461492\\
6.43	0.00957382023817639\\
6.44	0.00957383879902815\\
6.45	0.00957385736717357\\
6.46	0.00957387594261599\\
6.47	0.00957389452535879\\
6.48	0.00957391311540532\\
6.49	0.00957393171275895\\
6.5	0.00957395031742305\\
6.51	0.00957396892940098\\
6.52	0.00957398754869612\\
6.53	0.00957400617531184\\
6.54	0.00957402480925152\\
6.55	0.00957404345051853\\
6.56	0.00957406209911626\\
6.57	0.00957408075504808\\
6.58	0.00957409941831738\\
6.59	0.00957411808892755\\
6.6	0.00957413676688198\\
6.61	0.00957415545218405\\
6.62	0.00957417414483716\\
6.63	0.00957419284484471\\
6.64	0.00957421155221009\\
6.65	0.0095742302669367\\
6.66	0.00957424898902794\\
6.67	0.00957426771848722\\
6.68	0.00957428645531794\\
6.69	0.00957430519952352\\
6.7	0.00957432395110736\\
6.71	0.00957434271007287\\
6.72	0.00957436147642347\\
6.73	0.00957438025016258\\
6.74	0.00957439903129361\\
6.75	0.00957441781981999\\
6.76	0.00957443661574515\\
6.77	0.00957445541907249\\
6.78	0.00957447422980546\\
6.79	0.00957449304794749\\
6.8	0.009574511873502\\
6.81	0.00957453070647244\\
6.82	0.00957454954686223\\
6.83	0.00957456839467482\\
6.84	0.00957458724991364\\
6.85	0.00957460611258216\\
6.86	0.00957462498268379\\
6.87	0.009574643860222\\
6.88	0.00957466274520024\\
6.89	0.00957468163762196\\
6.9	0.0095747005374906\\
6.91	0.00957471944480964\\
6.92	0.00957473835958252\\
6.93	0.00957475728181272\\
6.94	0.00957477621150368\\
6.95	0.00957479514865889\\
6.96	0.00957481409328179\\
6.97	0.00957483304537588\\
6.98	0.00957485200494462\\
6.99	0.00957487097199148\\
7	0.00957488994651995\\
7.01	0.00957490892853349\\
7.02	0.00957492791803559\\
7.03	0.00957494691502974\\
7.04	0.00957496591951941\\
7.05	0.00957498493150811\\
7.06	0.00957500395099932\\
7.07	0.00957502297799653\\
7.08	0.00957504201250324\\
7.09	0.00957506105452294\\
7.1	0.00957508010405914\\
7.11	0.00957509916111533\\
7.12	0.00957511822569502\\
7.13	0.00957513729780171\\
7.14	0.00957515637743893\\
7.15	0.00957517546461017\\
7.16	0.00957519455931895\\
7.17	0.00957521366156879\\
7.18	0.00957523277136319\\
7.19	0.00957525188870569\\
7.2	0.00957527101359981\\
7.21	0.00957529014604906\\
7.22	0.00957530928605698\\
7.23	0.0095753284336271\\
7.24	0.00957534758876294\\
7.25	0.00957536675146805\\
7.26	0.00957538592174596\\
7.27	0.0095754050996002\\
7.28	0.00957542428503431\\
7.29	0.00957544347805185\\
7.3	0.00957546267865635\\
7.31	0.00957548188685137\\
7.32	0.00957550110264044\\
7.33	0.00957552032602714\\
7.34	0.009575539557015\\
7.35	0.0095755587956076\\
7.36	0.00957557804180848\\
7.37	0.0095755972956212\\
7.38	0.00957561655704934\\
7.39	0.00957563582609646\\
7.4	0.00957565510276613\\
7.41	0.00957567438706192\\
7.42	0.00957569367898739\\
7.43	0.00957571297854614\\
7.44	0.00957573228574173\\
7.45	0.00957575160057774\\
7.46	0.00957577092305777\\
7.47	0.00957579025318539\\
7.48	0.00957580959096419\\
7.49	0.00957582893639776\\
7.5	0.0095758482894897\\
7.51	0.0095758676502436\\
7.52	0.00957588701866305\\
7.53	0.00957590639475166\\
7.54	0.00957592577851302\\
7.55	0.00957594516995074\\
7.56	0.00957596456906843\\
7.57	0.00957598397586969\\
7.58	0.00957600339035814\\
7.59	0.00957602281253739\\
7.6	0.00957604224241106\\
7.61	0.00957606167998276\\
7.62	0.00957608112525611\\
7.63	0.00957610057823475\\
7.64	0.00957612003892228\\
7.65	0.00957613950732236\\
7.66	0.00957615898343859\\
7.67	0.00957617846727462\\
7.68	0.00957619795883407\\
7.69	0.00957621745812059\\
7.7	0.00957623696513783\\
7.71	0.00957625647988941\\
7.72	0.00957627600237898\\
7.73	0.0095762955326102\\
7.74	0.00957631507058671\\
7.75	0.00957633461631215\\
7.76	0.0095763541697902\\
7.77	0.0095763737310245\\
7.78	0.00957639330001872\\
7.79	0.00957641287677651\\
7.8	0.00957643246130154\\
7.81	0.00957645205359747\\
7.82	0.00957647165366799\\
7.83	0.00957649126151674\\
7.84	0.00957651087714742\\
7.85	0.00957653050056369\\
7.86	0.00957655013176924\\
7.87	0.00957656977076775\\
7.88	0.00957658941756289\\
7.89	0.00957660907215836\\
7.9	0.00957662873455785\\
7.91	0.00957664840476504\\
7.92	0.00957666808278363\\
7.93	0.00957668776861731\\
7.94	0.00957670746226979\\
7.95	0.00957672716374477\\
7.96	0.00957674687304594\\
7.97	0.00957676659017701\\
7.98	0.0095767863151417\\
7.99	0.00957680604794371\\
8	0.00957682578858675\\
8.01	0.00957684553707455\\
8.02	0.00957686529341082\\
8.03	0.00957688505759927\\
8.04	0.00957690482964364\\
8.05	0.00957692460954765\\
8.06	0.00957694439731502\\
8.07	0.0095769641929495\\
8.08	0.0095769839964548\\
8.09	0.00957700380783468\\
8.1	0.00957702362709285\\
8.11	0.00957704345423308\\
8.12	0.00957706328925909\\
8.13	0.00957708313217463\\
8.14	0.00957710298298346\\
8.15	0.00957712284168932\\
8.16	0.00957714270829597\\
8.17	0.00957716258280716\\
8.18	0.00957718246522665\\
8.19	0.00957720235555821\\
8.2	0.00957722225380558\\
8.21	0.00957724215997255\\
8.22	0.00957726207406288\\
8.23	0.00957728199608034\\
8.24	0.0095773019260287\\
8.25	0.00957732186391173\\
8.26	0.00957734180973323\\
8.27	0.00957736176349696\\
8.28	0.00957738172520672\\
8.29	0.00957740169486629\\
8.3	0.00957742167247945\\
8.31	0.00957744165805\\
8.32	0.00957746165158174\\
8.33	0.00957748165307845\\
8.34	0.00957750166254394\\
8.35	0.00957752167998201\\
8.36	0.00957754170539646\\
8.37	0.0095775617387911\\
8.38	0.00957758178016974\\
8.39	0.0095776018295362\\
8.4	0.00957762188689427\\
8.41	0.00957764195224779\\
8.42	0.00957766202560057\\
8.43	0.00957768210695644\\
8.44	0.00957770219631921\\
8.45	0.00957772229369272\\
8.46	0.00957774239908079\\
8.47	0.00957776251248727\\
8.48	0.00957778263391597\\
8.49	0.00957780276337075\\
8.5	0.00957782290085543\\
8.51	0.00957784304637387\\
8.52	0.00957786319992991\\
8.53	0.0095778833615274\\
8.54	0.00957790353117018\\
8.55	0.00957792370886211\\
8.56	0.00957794389460705\\
8.57	0.00957796408840886\\
8.58	0.00957798429027139\\
8.59	0.00957800450019852\\
8.6	0.0095780247181941\\
8.61	0.009578044944262\\
8.62	0.0095780651784061\\
8.63	0.00957808542063027\\
8.64	0.0095781056709384\\
8.65	0.00957812592933434\\
8.66	0.009578146195822\\
8.67	0.00957816647040526\\
8.68	0.00957818675308799\\
8.69	0.0095782070438741\\
8.7	0.00957822734276748\\
8.71	0.00957824764977201\\
8.72	0.0095782679648916\\
8.73	0.00957828828813015\\
8.74	0.00957830861949156\\
8.75	0.00957832895897974\\
8.76	0.0095783493065986\\
8.77	0.00957836966235204\\
8.78	0.00957839002624399\\
8.79	0.00957841039827836\\
8.8	0.00957843077845906\\
8.81	0.00957845116679003\\
8.82	0.00957847156327517\\
8.83	0.00957849196791843\\
8.84	0.00957851238072374\\
8.85	0.00957853280169501\\
8.86	0.0095785532308362\\
8.87	0.00957857366815123\\
8.88	0.00957859411364406\\
8.89	0.00957861456731861\\
8.9	0.00957863502917884\\
8.91	0.0095786554992287\\
8.92	0.00957867597747214\\
8.93	0.00957869646391311\\
8.94	0.00957871695855558\\
8.95	0.00957873746140349\\
8.96	0.00957875797246081\\
8.97	0.00957877849173151\\
8.98	0.00957879901921955\\
8.99	0.00957881955492891\\
9	0.00957884009886356\\
9.01	0.00957886065102747\\
9.02	0.00957888121142462\\
9.03	0.009578901780059\\
9.04	0.00957892235693458\\
9.05	0.00957894294205536\\
9.06	0.00957896353542533\\
9.07	0.00957898413704847\\
9.08	0.00957900474692878\\
9.09	0.00957902536507026\\
9.1	0.0095790459914769\\
9.11	0.00957906662615273\\
9.12	0.00957908726910172\\
9.13	0.00957910792032791\\
9.14	0.0095791285798353\\
9.15	0.00957914924762789\\
9.16	0.00957916992370972\\
9.17	0.00957919060808479\\
9.18	0.00957921130075714\\
9.19	0.00957923200173078\\
9.2	0.00957925271100975\\
9.21	0.00957927342859807\\
9.22	0.00957929415449978\\
9.23	0.00957931488871892\\
9.24	0.00957933563125951\\
9.25	0.00957935638212562\\
9.26	0.00957937714132127\\
9.27	0.00957939790885052\\
9.28	0.00957941868471741\\
9.29	0.009579439468926\\
9.3	0.00957946026148035\\
9.31	0.00957948106238452\\
9.32	0.00957950187164255\\
9.33	0.00957952268925852\\
9.34	0.0095795435152365\\
9.35	0.00957956434958055\\
9.36	0.00957958519229475\\
9.37	0.00957960604338317\\
9.38	0.00957962690284989\\
9.39	0.00957964777069898\\
9.4	0.00957966864693454\\
9.41	0.00957968953156065\\
9.42	0.00957971042458139\\
9.43	0.00957973132600087\\
9.44	0.00957975223582317\\
9.45	0.00957977315405239\\
9.46	0.00957979408069263\\
9.47	0.00957981501574799\\
9.48	0.00957983595922258\\
9.49	0.00957985691112052\\
9.5	0.0095798778714459\\
9.51	0.00957989884020284\\
9.52	0.00957991981739546\\
9.53	0.00957994080302788\\
9.54	0.00957996179710423\\
9.55	0.00957998279962863\\
9.56	0.00958000381060519\\
9.57	0.00958002483003807\\
9.58	0.00958004585793139\\
9.59	0.00958006689428928\\
9.6	0.00958008793911588\\
9.61	0.00958010899241535\\
9.62	0.00958013005419182\\
9.63	0.00958015112444944\\
9.64	0.00958017220319236\\
9.65	0.00958019329042473\\
9.66	0.00958021438615072\\
9.67	0.00958023549037447\\
9.68	0.00958025660310016\\
9.69	0.00958027772433195\\
9.7	0.009580298854074\\
9.71	0.00958031999233048\\
9.72	0.00958034113910557\\
9.73	0.00958036229440344\\
9.74	0.00958038345822827\\
9.75	0.00958040463058425\\
9.76	0.00958042581147556\\
9.77	0.00958044700090638\\
9.78	0.0095804681988809\\
9.79	0.00958048940540332\\
9.8	0.00958051062047783\\
9.81	0.00958053184410864\\
9.82	0.00958055307629994\\
9.83	0.00958057431705593\\
9.84	0.00958059556638083\\
9.85	0.00958061682427884\\
9.86	0.00958063809075418\\
9.87	0.00958065936581106\\
9.88	0.0095806806494537\\
9.89	0.00958070194168631\\
9.9	0.00958072324251314\\
9.91	0.00958074455193839\\
9.92	0.0095807658699663\\
9.93	0.0095807871966011\\
9.94	0.00958080853184703\\
9.95	0.00958082987570833\\
9.96	0.00958085122818923\\
9.97	0.00958087258929398\\
9.98	0.00958089395902682\\
9.99	0.00958091533739201\\
10	0.00958093672439379\\
10.01	0.00958095812003643\\
10.02	0.00958097952432417\\
10.03	0.00958100093726127\\
10.04	0.00958102235885202\\
10.05	0.00958104378910065\\
10.06	0.00958106522801145\\
10.07	0.00958108667558868\\
10.08	0.00958110813183663\\
10.09	0.00958112959675956\\
10.1	0.00958115107036176\\
10.11	0.00958117255264751\\
10.12	0.00958119404362109\\
10.13	0.00958121554328679\\
10.14	0.00958123705164891\\
10.15	0.00958125856871173\\
10.16	0.00958128009447956\\
10.17	0.00958130162895668\\
10.18	0.00958132317214741\\
10.19	0.00958134472405605\\
10.2	0.0095813662846869\\
10.21	0.00958138785404427\\
10.22	0.00958140943213248\\
10.23	0.00958143101895585\\
10.24	0.00958145261451868\\
10.25	0.0095814742188253\\
10.26	0.00958149583188004\\
10.27	0.00958151745368723\\
10.28	0.00958153908425118\\
10.29	0.00958156072357623\\
10.3	0.00958158237166671\\
10.31	0.00958160402852697\\
10.32	0.00958162569416134\\
10.33	0.00958164736857417\\
10.34	0.0095816690517698\\
10.35	0.00958169074375257\\
10.36	0.00958171244452684\\
10.37	0.00958173415409696\\
10.38	0.00958175587246728\\
10.39	0.00958177759964217\\
10.4	0.00958179933562598\\
10.41	0.00958182108042309\\
10.42	0.00958184283403784\\
10.43	0.00958186459647462\\
10.44	0.00958188636773779\\
10.45	0.00958190814783173\\
10.46	0.00958192993676082\\
10.47	0.00958195173452943\\
10.48	0.00958197354114194\\
10.49	0.00958199535660275\\
10.5	0.00958201718091622\\
10.51	0.00958203901408676\\
10.52	0.00958206085611875\\
10.53	0.0095820827070166\\
10.54	0.00958210456678469\\
10.55	0.00958212643542743\\
10.56	0.00958214831294921\\
10.57	0.00958217019935445\\
10.58	0.00958219209464754\\
10.59	0.0095822139988329\\
10.6	0.00958223591191493\\
10.61	0.00958225783389806\\
10.62	0.00958227976478669\\
10.63	0.00958230170458524\\
10.64	0.00958232365329814\\
10.65	0.00958234561092981\\
10.66	0.00958236757748468\\
10.67	0.00958238955296717\\
10.68	0.0095824115373817\\
10.69	0.00958243353073272\\
10.7	0.00958245553302466\\
10.71	0.00958247754426196\\
10.72	0.00958249956444905\\
10.73	0.00958252159359038\\
10.74	0.00958254363169038\\
10.75	0.00958256567875351\\
10.76	0.00958258773478422\\
10.77	0.00958260979978694\\
10.78	0.00958263187376614\\
10.79	0.00958265395672627\\
10.8	0.00958267604867179\\
10.81	0.00958269814960714\\
10.82	0.0095827202595368\\
10.83	0.00958274237846522\\
10.84	0.00958276450639688\\
10.85	0.00958278664333623\\
10.86	0.00958280878928774\\
10.87	0.00958283094425589\\
10.88	0.00958285310824514\\
10.89	0.00958287528125997\\
10.9	0.00958289746330486\\
10.91	0.00958291965438428\\
10.92	0.00958294185450271\\
10.93	0.00958296406366463\\
10.94	0.00958298628187453\\
10.95	0.00958300850913689\\
10.96	0.00958303074545619\\
10.97	0.00958305299083692\\
10.98	0.00958307524528358\\
10.99	0.00958309750880064\\
11	0.00958311978139262\\
11.01	0.00958314206306399\\
11.02	0.00958316435381925\\
11.03	0.00958318665366291\\
11.04	0.00958320896259945\\
11.05	0.00958323128063338\\
11.06	0.00958325360776919\\
11.07	0.0095832759440114\\
11.08	0.0095832982893645\\
11.09	0.00958332064383299\\
11.1	0.00958334300742139\\
11.11	0.00958336538013419\\
11.12	0.00958338776197592\\
11.13	0.00958341015295106\\
11.14	0.00958343255306414\\
11.15	0.00958345496231966\\
11.16	0.00958347738072214\\
11.17	0.00958349980827609\\
11.18	0.00958352224498602\\
11.19	0.00958354469085643\\
11.2	0.00958356714589186\\
11.21	0.00958358961009681\\
11.22	0.0095836120834758\\
11.23	0.00958363456603334\\
11.24	0.00958365705777394\\
11.25	0.00958367955870214\\
11.26	0.00958370206882244\\
11.27	0.00958372458813936\\
11.28	0.00958374711665742\\
11.29	0.00958376965438113\\
11.3	0.00958379220131502\\
11.31	0.00958381475746359\\
11.32	0.00958383732283137\\
11.33	0.00958385989742288\\
11.34	0.00958388248124262\\
11.35	0.00958390507429512\\
11.36	0.0095839276765849\\
11.37	0.00958395028811647\\
11.38	0.00958397290889434\\
11.39	0.00958399553892304\\
11.4	0.00958401817820706\\
11.41	0.00958404082675094\\
11.42	0.00958406348455917\\
11.43	0.00958408615163627\\
11.44	0.00958410882798676\\
11.45	0.00958413151361513\\
11.46	0.00958415420852591\\
11.47	0.00958417691272358\\
11.48	0.00958419962621267\\
11.49	0.00958422234899767\\
11.5	0.00958424508108308\\
11.51	0.00958426782247342\\
11.52	0.00958429057317316\\
11.53	0.00958431333318682\\
11.54	0.00958433610251888\\
11.55	0.00958435888117385\\
11.56	0.00958438166915619\\
11.57	0.00958440446647041\\
11.58	0.00958442727312099\\
11.59	0.00958445008911242\\
11.6	0.00958447291444916\\
11.61	0.00958449574913571\\
11.62	0.00958451859317652\\
11.63	0.00958454144657607\\
11.64	0.00958456430933884\\
11.65	0.00958458718146928\\
11.66	0.00958461006297185\\
11.67	0.009584632953851\\
11.68	0.0095846558541112\\
11.69	0.00958467876375689\\
11.7	0.00958470168279251\\
11.71	0.0095847246112225\\
11.72	0.00958474754905129\\
11.73	0.00958477049628333\\
11.74	0.00958479345292303\\
11.75	0.00958481641897481\\
11.76	0.00958483939444308\\
11.77	0.00958486237933226\\
11.78	0.00958488537364675\\
11.79	0.00958490837739096\\
11.8	0.00958493139056926\\
11.81	0.00958495441318604\\
11.82	0.00958497744524569\\
11.83	0.00958500048675258\\
11.84	0.00958502353771107\\
11.85	0.00958504659812552\\
11.86	0.00958506966800029\\
11.87	0.00958509274733971\\
11.88	0.00958511583614812\\
11.89	0.00958513893442986\\
11.9	0.00958516204218923\\
11.91	0.00958518515943056\\
11.92	0.00958520828615813\\
11.93	0.00958523142237626\\
11.94	0.00958525456808921\\
11.95	0.00958527772330127\\
11.96	0.00958530088801669\\
11.97	0.00958532406223974\\
11.98	0.00958534724597466\\
11.99	0.00958537043922568\\
12	0.00958539364199702\\
12.01	0.00958541685429289\\
12.02	0.00958544007611749\\
12.03	0.00958546330747501\\
12.04	0.00958548654836962\\
12.05	0.00958550979880549\\
12.06	0.00958553305878676\\
12.07	0.00958555632831758\\
12.08	0.00958557960740205\\
12.09	0.00958560289604429\\
12.1	0.00958562619424839\\
12.11	0.00958564950201844\\
12.12	0.00958567281935848\\
12.13	0.00958569614627257\\
12.14	0.00958571948276475\\
12.15	0.00958574282883902\\
12.16	0.00958576618449938\\
12.17	0.00958578954974982\\
12.18	0.00958581292459429\\
12.19	0.00958583630903674\\
12.2	0.00958585970308109\\
12.21	0.00958588310673125\\
12.22	0.00958590651999111\\
12.23	0.00958592994286453\\
12.24	0.00958595337535535\\
12.25	0.0095859768174674\\
12.26	0.00958600026920449\\
12.27	0.00958602373057038\\
12.28	0.00958604720156884\\
12.29	0.00958607068220359\\
12.3	0.00958609417247836\\
12.31	0.00958611767239681\\
12.32	0.00958614118196261\\
12.33	0.0095861647011794\\
12.34	0.00958618823005076\\
12.35	0.00958621176858029\\
12.36	0.00958623531677154\\
12.37	0.00958625887462802\\
12.38	0.00958628244215323\\
12.39	0.00958630601935063\\
12.4	0.00958632960622366\\
12.41	0.0095863532027757\\
12.42	0.00958637680901013\\
12.43	0.00958640042493029\\
12.44	0.00958642405053946\\
12.45	0.00958644768584093\\
12.46	0.00958647133083791\\
12.47	0.00958649498553361\\
12.48	0.00958651864993117\\
12.49	0.00958654232403373\\
12.5	0.00958656600784434\\
12.51	0.00958658970136606\\
12.52	0.00958661340460188\\
12.53	0.00958663711755476\\
12.54	0.00958666084022761\\
12.55	0.0095866845726233\\
12.56	0.00958670831474466\\
12.57	0.00958673206659445\\
12.58	0.00958675582817542\\
12.59	0.00958677959949025\\
12.6	0.00958680338054157\\
12.61	0.00958682717133195\\
12.62	0.00958685097186394\\
12.63	0.00958687478214001\\
12.64	0.00958689860216259\\
12.65	0.00958692243193403\\
12.66	0.00958694627145667\\
12.67	0.00958697012073274\\
12.68	0.00958699397976444\\
12.69	0.00958701784855391\\
12.7	0.00958704172710322\\
12.71	0.00958706561541437\\
12.72	0.00958708951348931\\
12.73	0.00958711342132992\\
12.74	0.00958713733893799\\
12.75	0.00958716126631528\\
12.76	0.00958718520346344\\
12.77	0.00958720915038408\\
12.78	0.00958723310707871\\
12.79	0.00958725707354877\\
12.8	0.00958728104979564\\
12.81	0.00958730503582059\\
12.82	0.00958732903162483\\
12.83	0.00958735303720948\\
12.84	0.00958737705257558\\
12.85	0.00958740107772406\\
12.86	0.0095874251126558\\
12.87	0.00958744915737156\\
12.88	0.00958747321187202\\
12.89	0.00958749727615774\\
12.9	0.00958752135022922\\
12.91	0.00958754543408683\\
12.92	0.00958756952773086\\
12.93	0.00958759363116147\\
12.94	0.00958761774437873\\
12.95	0.0095876418673826\\
12.96	0.00958766600017294\\
12.97	0.00958769014274946\\
12.98	0.0095877142951118\\
12.99	0.00958773845725943\\
13	0.00958776262919175\\
13.01	0.009587786810908\\
13.02	0.0095878110024073\\
13.03	0.00958783520368865\\
13.04	0.00958785941475089\\
13.05	0.00958788363559278\\
13.06	0.00958790786621287\\
13.07	0.00958793210660961\\
13.08	0.00958795635678131\\
13.09	0.00958798061672611\\
13.1	0.009588004886442\\
13.11	0.00958802916592683\\
13.12	0.00958805345517828\\
13.13	0.00958807775419387\\
13.14	0.00958810206297095\\
13.15	0.00958812638150671\\
13.16	0.00958815070979817\\
13.17	0.00958817504784215\\
13.18	0.00958819939563531\\
13.19	0.00958822375317412\\
13.2	0.00958824812045488\\
13.21	0.00958827249747365\\
13.22	0.00958829688422634\\
13.23	0.00958832128070863\\
13.24	0.00958834568691602\\
13.25	0.00958837010284378\\
13.26	0.00958839452848697\\
13.27	0.00958841896384042\\
13.28	0.00958844340889877\\
13.29	0.00958846786365639\\
13.3	0.00958849232810746\\
13.31	0.00958851680224587\\
13.32	0.00958854128606531\\
13.33	0.00958856577955921\\
13.34	0.00958859028272072\\
13.35	0.00958861479554278\\
13.36	0.00958863931801801\\
13.37	0.0095886638501388\\
13.38	0.00958868839189725\\
13.39	0.00958871294328518\\
13.4	0.00958873750429411\\
13.41	0.00958876207491528\\
13.42	0.00958878665513964\\
13.43	0.00958881124495779\\
13.44	0.00958883584436006\\
13.45	0.00958886045333646\\
13.46	0.00958888507187665\\
13.47	0.00958890969996996\\
13.48	0.00958893433760541\\
13.49	0.00958895898477163\\
13.5	0.00958898364145692\\
13.51	0.00958900830764923\\
13.52	0.00958903298333613\\
13.53	0.00958905766850481\\
13.54	0.00958908236314207\\
13.55	0.00958910706723434\\
13.56	0.00958913178076764\\
13.57	0.00958915650372759\\
13.58	0.00958918123609937\\
13.59	0.00958920597786778\\
13.6	0.00958923072901714\\
13.61	0.00958925548953137\\
13.62	0.00958928025939393\\
13.63	0.0095893050385878\\
13.64	0.00958932982709552\\
13.65	0.00958935462489914\\
13.66	0.00958937943198024\\
13.67	0.00958940424831989\\
13.68	0.00958942907389867\\
13.69	0.00958945390869663\\
13.7	0.00958947875269332\\
13.71	0.00958950360586773\\
13.72	0.00958952846819833\\
13.73	0.00958955333966302\\
13.74	0.00958957822023916\\
13.75	0.00958960310990351\\
13.76	0.00958962800863227\\
13.77	0.00958965291640102\\
13.78	0.00958967783318476\\
13.79	0.00958970275895786\\
13.8	0.00958972769369405\\
13.81	0.00958975263736646\\
13.82	0.00958977758994753\\
13.83	0.00958980255140907\\
13.84	0.0095898275217222\\
13.85	0.00958985250085735\\
13.86	0.00958987748878428\\
13.87	0.00958990248547202\\
13.88	0.0095899274908889\\
13.89	0.0095899525050025\\
13.9	0.00958997752777966\\
13.91	0.00959000255918649\\
13.92	0.00959002759918831\\
13.93	0.00959005264774966\\
13.94	0.00959007770483431\\
13.95	0.00959010277040521\\
13.96	0.00959012784442449\\
13.97	0.00959015292685348\\
13.98	0.00959017801765265\\
13.99	0.00959020311678163\\
14	0.00959022822419916\\
14.01	0.00959025333986315\\
14.02	0.00959027846373059\\
14.03	0.00959030359575758\\
14.04	0.00959032873589932\\
14.05	0.00959035388411007\\
14.06	0.00959037904034316\\
14.07	0.009590404204551\\
14.08	0.00959042937668501\\
14.09	0.00959045455669567\\
14.1	0.00959047974453246\\
14.11	0.0095905049401439\\
14.12	0.00959053014347748\\
14.13	0.00959055535447971\\
14.14	0.00959058057309608\\
14.15	0.00959060579927103\\
14.16	0.00959063103294799\\
14.17	0.00959065627406934\\
14.18	0.00959068152257642\\
14.19	0.00959070677840948\\
14.2	0.00959073204150776\\
14.21	0.00959075731180936\\
14.22	0.00959078258925138\\
14.23	0.00959080787376978\\
14.24	0.00959083316529947\\
14.25	0.00959085846377426\\
14.26	0.00959088376912687\\
14.27	0.00959090908128896\\
14.28	0.00959093440019105\\
14.29	0.00959095972576262\\
14.3	0.00959098505793202\\
14.31	0.00959101039662656\\
14.32	0.00959103574177244\\
14.33	0.0095910610932948\\
14.34	0.00959108645111771\\
14.35	0.00959111181516421\\
14.36	0.00959113718535625\\
14.37	0.00959116256161479\\
14.38	0.00959118794385974\\
14.39	0.00959121333201005\\
14.4	0.00959123872598365\\
14.41	0.00959126412569752\\
14.42	0.0095912895310677\\
14.43	0.00959131494200934\\
14.44	0.00959134035843667\\
14.45	0.0095913657802631\\
14.46	0.0095913912074012\\
14.47	0.00959141663976276\\
14.48	0.00959144207725885\\
14.49	0.00959146751979981\\
14.5	0.00959149296729535\\
14.51	0.0095915184196546\\
14.52	0.00959154387678611\\
14.53	0.00959156933859796\\
14.54	0.00959159480499783\\
14.55	0.00959162027589302\\
14.56	0.00959164575119059\\
14.57	0.00959167123079737\\
14.58	0.0095916967146201\\
14.59	0.00959172220256548\\
14.6	0.00959174769454031\\
14.61	0.00959177319045156\\
14.62	0.00959179869020645\\
14.63	0.00959182419371267\\
14.64	0.00959184970087839\\
14.65	0.00959187521161247\\
14.66	0.00959190072582455\\
14.67	0.00959192624342522\\
14.68	0.00959195176432621\\
14.69	0.00959197728844048\\
14.7	0.00959200281568247\\
14.71	0.00959202834596826\\
14.72	0.00959205387921575\\
14.73	0.00959207941534492\\
14.74	0.00959210495427799\\
14.75	0.0095921304959397\\
14.76	0.00959215604025756\\
14.77	0.00959218158716205\\
14.78	0.009592207136587\\
14.79	0.00959223268846977\\
14.8	0.00959225824275165\\
14.81	0.00959228379937813\\
14.82	0.00959230935829926\\
14.83	0.00959233491947001\\
14.84	0.00959236048285063\\
14.85	0.0095923860484071\\
14.86	0.00959241161611149\\
14.87	0.00959243718594244\\
14.88	0.00959246275788561\\
14.89	0.0095924883319342\\
14.9	0.0095925139080894\\
14.91	0.009592539486361\\
14.92	0.00959256506676793\\
14.93	0.00959259064933885\\
14.94	0.00959261623411281\\
14.95	0.00959264182113987\\
14.96	0.0095926674104818\\
14.97	0.00959269300221287\\
14.98	0.00959271859642052\\
14.99	0.00959274419320623\\
15	0.00959276979268633\\
15.01	0.00959279539499291\\
15.02	0.00959282100027473\\
15.03	0.00959284660869818\\
15.04	0.00959287222044833\\
15.05	0.00959289783572998\\
15.06	0.0095929234547688\\
15.07	0.00959294907781246\\
15.08	0.00959297470513192\\
15.09	0.0095930003370227\\
15.1	0.00959302597380622\\
15.11	0.00959305161583123\\
15.12	0.00959307726347529\\
15.13	0.00959310291714635\\
15.14	0.00959312857728433\\
15.15	0.00959315424436289\\
15.16	0.00959317991889115\\
15.17	0.00959320560141562\\
15.18	0.00959323129252215\\
15.19	0.00959325699283793\\
15.2	0.0095932827030337\\
15.21	0.00959330842382596\\
15.22	0.00959333415597935\\
15.23	0.00959335989957167\\
15.24	0.00959338565460988\\
15.25	0.009593411421101\\
15.26	0.009593437199052\\
15.27	0.00959346298846989\\
15.28	0.00959348878936168\\
15.29	0.00959351460173438\\
15.3	0.00959354042559503\\
15.31	0.00959356626095064\\
15.32	0.00959359210780825\\
15.33	0.00959361796617491\\
15.34	0.00959364383605766\\
15.35	0.00959366971746355\\
15.36	0.00959369561039967\\
15.37	0.00959372151487306\\
15.38	0.00959374743089082\\
15.39	0.00959377335846002\\
15.4	0.00959379929758776\\
15.41	0.00959382524828112\\
15.42	0.00959385121054723\\
15.43	0.00959387718439318\\
15.44	0.00959390316982609\\
15.45	0.0095939291668531\\
15.46	0.00959395517548133\\
15.47	0.00959398119571792\\
15.48	0.00959400722757001\\
15.49	0.00959403327104478\\
15.5	0.00959405932614935\\
15.51	0.00959408539289092\\
15.52	0.00959411147127664\\
15.53	0.00959413756131371\\
15.54	0.00959416366300931\\
15.55	0.00959418977637064\\
15.56	0.00959421590140488\\
15.57	0.00959424203811927\\
15.58	0.00959426818652101\\
15.59	0.00959429434661732\\
15.6	0.00959432051841544\\
15.61	0.0095943467019226\\
15.62	0.00959437289714604\\
15.63	0.00959439910409302\\
15.64	0.0095944253227708\\
15.65	0.00959445155318665\\
15.66	0.00959447779534782\\
15.67	0.00959450404926162\\
15.68	0.00959453031493531\\
15.69	0.0095945565923762\\
15.7	0.00959458288159159\\
15.71	0.00959460918258879\\
15.72	0.0095946354953751\\
15.73	0.00959466181995787\\
15.74	0.00959468815634441\\
15.75	0.00959471450454206\\
15.76	0.00959474086455817\\
15.77	0.00959476723640009\\
15.78	0.00959479362007518\\
15.79	0.00959482001559081\\
15.8	0.00959484642295434\\
15.81	0.00959487284217317\\
15.82	0.00959489927325468\\
15.83	0.00959492571620626\\
15.84	0.00959495217103532\\
15.85	0.00959497863774927\\
15.86	0.00959500511635552\\
15.87	0.00959503160686151\\
15.88	0.00959505810927465\\
15.89	0.00959508462360241\\
15.9	0.00959511114985221\\
15.91	0.00959513768803151\\
15.92	0.00959516423814778\\
15.93	0.00959519080020849\\
15.94	0.00959521737422111\\
15.95	0.00959524396019313\\
15.96	0.00959527055813204\\
15.97	0.00959529716804534\\
15.98	0.00959532378994053\\
15.99	0.00959535042382513\\
16	0.00959537706970666\\
16.01	0.00959540372759264\\
16.02	0.00959543039749062\\
16.03	0.00959545707940815\\
16.04	0.00959548377335276\\
16.05	0.00959551047933203\\
16.06	0.00959553719735351\\
16.07	0.00959556392742479\\
16.08	0.00959559066955343\\
16.09	0.00959561742374704\\
16.1	0.00959564419001321\\
16.11	0.00959567096835955\\
16.12	0.00959569775879366\\
16.13	0.00959572456132317\\
16.14	0.0095957513759557\\
16.15	0.00959577820269889\\
16.16	0.00959580504156039\\
16.17	0.00959583189254784\\
16.18	0.00959585875566889\\
16.19	0.00959588563093123\\
16.2	0.00959591251834252\\
16.21	0.00959593941791043\\
16.22	0.00959596632964268\\
16.23	0.00959599325354693\\
16.24	0.00959602018963091\\
16.25	0.00959604713790232\\
16.26	0.00959607409836889\\
16.27	0.00959610107103834\\
16.28	0.0095961280559184\\
16.29	0.00959615505301683\\
16.3	0.00959618206234138\\
16.31	0.00959620908389979\\
16.32	0.00959623611769984\\
16.33	0.00959626316374931\\
16.34	0.00959629022205597\\
16.35	0.00959631729262762\\
16.36	0.00959634437547206\\
16.37	0.00959637147059708\\
16.38	0.00959639857801052\\
16.39	0.00959642569772018\\
16.4	0.0095964528297339\\
16.41	0.00959647997405952\\
16.42	0.00959650713070487\\
16.43	0.00959653429967782\\
16.44	0.00959656148098623\\
16.45	0.00959658867463797\\
16.46	0.00959661588064092\\
16.47	0.00959664309900296\\
16.48	0.00959667032973198\\
16.49	0.00959669757283589\\
16.5	0.0095967248283226\\
16.51	0.00959675209620002\\
16.52	0.00959677937647609\\
16.53	0.00959680666915873\\
16.54	0.00959683397425589\\
16.55	0.00959686129177553\\
16.56	0.00959688862172558\\
16.57	0.00959691596411404\\
16.58	0.00959694331894886\\
16.59	0.00959697068623803\\
16.6	0.00959699806598956\\
16.61	0.00959702545821142\\
16.62	0.00959705286291163\\
16.63	0.00959708028009821\\
16.64	0.00959710770977918\\
16.65	0.00959713515196257\\
16.66	0.00959716260665642\\
16.67	0.00959719007386879\\
16.68	0.00959721755360772\\
16.69	0.00959724504588129\\
16.7	0.00959727255069756\\
16.71	0.00959730006806463\\
16.72	0.00959732759799057\\
16.73	0.00959735514048349\\
16.74	0.00959738269555149\\
16.75	0.00959741026320269\\
16.76	0.00959743784344522\\
16.77	0.0095974654362872\\
16.78	0.00959749304173677\\
16.79	0.00959752065980209\\
16.8	0.00959754829049131\\
16.81	0.0095975759338126\\
16.82	0.00959760358977412\\
16.83	0.00959763125838408\\
16.84	0.00959765893965064\\
16.85	0.00959768663358201\\
16.86	0.0095977143401864\\
16.87	0.00959774205947203\\
16.88	0.00959776979144712\\
16.89	0.00959779753611991\\
16.9	0.00959782529349863\\
16.91	0.00959785306359154\\
16.92	0.00959788084640689\\
16.93	0.00959790864195296\\
16.94	0.00959793645023801\\
16.95	0.00959796427127034\\
16.96	0.00959799210505824\\
16.97	0.00959801995161\\
16.98	0.00959804781093394\\
16.99	0.00959807568303839\\
17	0.00959810356793166\\
17.01	0.00959813146562209\\
17.02	0.00959815937611803\\
17.03	0.00959818729942783\\
17.04	0.00959821523555985\\
17.05	0.00959824318452247\\
17.06	0.00959827114632407\\
17.07	0.00959829912097303\\
17.08	0.00959832710847775\\
17.09	0.00959835510884664\\
17.1	0.00959838312208812\\
17.11	0.00959841114821061\\
17.12	0.00959843918722253\\
17.13	0.00959846723913233\\
17.14	0.00959849530394847\\
17.15	0.00959852338167941\\
17.16	0.0095985514723336\\
17.17	0.00959857957591953\\
17.18	0.00959860769244568\\
17.19	0.00959863582192054\\
17.2	0.00959866396435263\\
17.21	0.00959869211975046\\
17.22	0.00959872028812254\\
17.23	0.0095987484694774\\
17.24	0.00959877666382359\\
17.25	0.00959880487116966\\
17.26	0.00959883309152416\\
17.27	0.00959886132489566\\
17.28	0.00959888957129273\\
17.29	0.00959891783072396\\
17.3	0.00959894610319794\\
17.31	0.00959897438872328\\
17.32	0.00959900268730858\\
17.33	0.00959903099896247\\
17.34	0.00959905932369359\\
17.35	0.00959908766151055\\
17.36	0.00959911601242203\\
17.37	0.00959914437643667\\
17.38	0.00959917275356313\\
17.39	0.00959920114381011\\
17.4	0.00959922954718627\\
17.41	0.00959925796370032\\
17.42	0.00959928639336095\\
17.43	0.00959931483617689\\
17.44	0.00959934329215685\\
17.45	0.00959937176130957\\
17.46	0.00959940024364377\\
17.47	0.00959942873916823\\
17.48	0.00959945724789168\\
17.49	0.0095994857698229\\
17.5	0.00959951430497068\\
17.51	0.00959954285334379\\
17.52	0.00959957141495103\\
17.53	0.00959959998980121\\
17.54	0.00959962857790314\\
17.55	0.00959965717926566\\
17.56	0.00959968579389758\\
17.57	0.00959971442180776\\
17.58	0.00959974306300504\\
17.59	0.0095997717174983\\
17.6	0.0095998003852964\\
17.61	0.00959982906640822\\
17.62	0.00959985776084266\\
17.63	0.00959988646860861\\
17.64	0.00959991518971498\\
17.65	0.00959994392417069\\
17.66	0.00959997267198468\\
17.67	0.00960000143316588\\
17.68	0.00960003020772323\\
17.69	0.0096000589956657\\
17.7	0.00960008779700226\\
17.71	0.00960011661174187\\
17.72	0.00960014543989354\\
17.73	0.00960017428146624\\
17.74	0.00960020313646899\\
17.75	0.00960023200491081\\
17.76	0.00960026088680072\\
17.77	0.00960028978214775\\
17.78	0.00960031869096094\\
17.79	0.00960034761324936\\
17.8	0.00960037654902208\\
17.81	0.00960040549828815\\
17.82	0.00960043446105666\\
17.83	0.00960046343733672\\
17.84	0.00960049242713742\\
17.85	0.00960052143046787\\
17.86	0.0096005504473372\\
17.87	0.00960057947775454\\
17.88	0.00960060852172904\\
17.89	0.00960063757926984\\
17.9	0.00960066665038612\\
17.91	0.00960069573508703\\
17.92	0.00960072483338177\\
17.93	0.00960075394527953\\
17.94	0.0096007830707895\\
17.95	0.00960081220992091\\
17.96	0.00960084136268298\\
17.97	0.00960087052908493\\
17.98	0.009600899709136\\
17.99	0.00960092890284546\\
18	0.00960095811022257\\
18.01	0.00960098733127659\\
18.02	0.00960101656601681\\
18.03	0.00960104581445252\\
18.04	0.00960107507659303\\
18.05	0.00960110435244765\\
18.06	0.0096011336420257\\
18.07	0.00960116294533651\\
18.08	0.00960119226238944\\
18.09	0.00960122159319382\\
18.1	0.00960125093775904\\
18.11	0.00960128029609445\\
18.12	0.00960130966820945\\
18.13	0.00960133905411343\\
18.14	0.00960136845381579\\
18.15	0.00960139786732596\\
18.16	0.00960142729465335\\
18.17	0.00960145673580741\\
18.18	0.00960148619079758\\
18.19	0.00960151565963331\\
18.2	0.00960154514232407\\
18.21	0.00960157463887934\\
18.22	0.00960160414930861\\
18.23	0.00960163367362138\\
18.24	0.00960166321182716\\
18.25	0.00960169276393546\\
18.26	0.00960172232995581\\
18.27	0.00960175190989776\\
18.28	0.00960178150377085\\
18.29	0.00960181111158465\\
18.3	0.00960184073334872\\
18.31	0.00960187036907266\\
18.32	0.00960190001876605\\
18.33	0.0096019296824385\\
18.34	0.00960195936009962\\
18.35	0.00960198905175903\\
18.36	0.00960201875742638\\
18.37	0.00960204847711129\\
18.38	0.00960207821082345\\
18.39	0.0096021079585725\\
18.4	0.00960213772036814\\
18.41	0.00960216749622004\\
18.42	0.00960219728613791\\
18.43	0.00960222709013145\\
18.44	0.0096022569082104\\
18.45	0.00960228674038447\\
18.46	0.00960231658666342\\
18.47	0.00960234644705699\\
18.48	0.00960237632157495\\
18.49	0.00960240621022708\\
18.5	0.00960243611302316\\
18.51	0.00960246602997298\\
18.52	0.00960249596108637\\
18.53	0.00960252590637313\\
18.54	0.00960255586584309\\
18.55	0.00960258583950609\\
18.56	0.00960261582737199\\
18.57	0.00960264582945065\\
18.58	0.00960267584575195\\
18.59	0.00960270587628576\\
18.6	0.00960273592106197\\
18.61	0.00960276598009051\\
18.62	0.00960279605338129\\
18.63	0.00960282614094423\\
18.64	0.00960285624278927\\
18.65	0.00960288635892637\\
18.66	0.0096029164893655\\
18.67	0.00960294663411661\\
18.68	0.00960297679318969\\
18.69	0.00960300696659476\\
18.7	0.00960303715434179\\
18.71	0.00960306735644083\\
18.72	0.00960309757290189\\
18.73	0.00960312780373502\\
18.74	0.00960315804895026\\
18.75	0.00960318830855769\\
18.76	0.00960321858256738\\
18.77	0.00960324887098941\\
18.78	0.00960327917383387\\
18.79	0.00960330949111089\\
18.8	0.00960333982283058\\
18.81	0.00960337016900306\\
18.82	0.00960340052963849\\
18.83	0.00960343090474701\\
18.84	0.00960346129433879\\
18.85	0.00960349169842401\\
18.86	0.00960352211701287\\
18.87	0.00960355255011555\\
18.88	0.00960358299774227\\
18.89	0.00960361345990325\\
18.9	0.00960364393660873\\
18.91	0.00960367442786896\\
18.92	0.00960370493369419\\
18.93	0.0096037354540947\\
18.94	0.00960376598908075\\
18.95	0.00960379653866266\\
18.96	0.00960382710285071\\
18.97	0.00960385768165523\\
18.98	0.00960388827508655\\
18.99	0.009603918883155\\
19	0.00960394950587094\\
19.01	0.00960398014324473\\
19.02	0.00960401079528675\\
19.03	0.00960404146200738\\
19.04	0.00960407214341702\\
19.05	0.00960410283952608\\
19.06	0.00960413355034499\\
19.07	0.00960416427588417\\
19.08	0.00960419501615408\\
19.09	0.00960422577116518\\
19.1	0.00960425654092792\\
19.11	0.0096042873254528\\
19.12	0.00960431812475031\\
19.13	0.00960434893883096\\
19.14	0.00960437976770525\\
19.15	0.00960441061138374\\
19.16	0.00960444146987694\\
19.17	0.00960447234319543\\
19.18	0.00960450323134977\\
19.19	0.00960453413435053\\
19.2	0.0096045650522083\\
19.21	0.00960459598493369\\
19.22	0.00960462693253731\\
19.23	0.00960465789502979\\
19.24	0.00960468887242177\\
19.25	0.0096047198647239\\
19.26	0.00960475087194685\\
19.27	0.00960478189410129\\
19.28	0.0096048129311979\\
19.29	0.00960484398324739\\
19.3	0.00960487505026047\\
19.31	0.00960490613224787\\
19.32	0.00960493722922033\\
19.33	0.00960496834118859\\
19.34	0.00960499946816341\\
19.35	0.00960503061015558\\
19.36	0.00960506176717588\\
19.37	0.0096050929392351\\
19.38	0.00960512412634407\\
19.39	0.0096051553285136\\
19.4	0.00960518654575454\\
19.41	0.00960521777807772\\
19.42	0.00960524902549402\\
19.43	0.00960528028801432\\
19.44	0.00960531156564948\\
19.45	0.00960534285841043\\
19.46	0.00960537416630807\\
19.47	0.00960540548935332\\
19.48	0.00960543682755713\\
19.49	0.00960546818093044\\
19.5	0.00960549954948422\\
19.51	0.00960553093322944\\
19.52	0.0096055623321771\\
19.53	0.00960559374633819\\
19.54	0.00960562517572373\\
19.55	0.00960565662034475\\
19.56	0.00960568808021229\\
19.57	0.0096057195553374\\
19.58	0.00960575104573115\\
19.59	0.00960578255140461\\
19.6	0.00960581407236888\\
19.61	0.00960584560863506\\
19.62	0.00960587716021428\\
19.63	0.00960590872711765\\
19.64	0.00960594030935633\\
19.65	0.00960597190694148\\
19.66	0.00960600351988425\\
19.67	0.00960603514819585\\
19.68	0.00960606679188745\\
19.69	0.00960609845097028\\
19.7	0.00960613012545554\\
19.71	0.00960616181535449\\
19.72	0.00960619352067836\\
19.73	0.00960622524143842\\
19.74	0.00960625697764595\\
19.75	0.00960628872931223\\
19.76	0.00960632049644857\\
19.77	0.00960635227906627\\
19.78	0.00960638407717667\\
19.79	0.00960641589079111\\
19.8	0.00960644771992095\\
19.81	0.00960647956457755\\
19.82	0.0096065114247723\\
19.83	0.00960654330051658\\
19.84	0.00960657519182181\\
19.85	0.00960660709869941\\
19.86	0.00960663902116082\\
19.87	0.00960667095921748\\
19.88	0.00960670291288086\\
19.89	0.00960673488216244\\
19.9	0.0096067668670737\\
19.91	0.00960679886762614\\
19.92	0.00960683088383129\\
19.93	0.00960686291570067\\
19.94	0.00960689496324583\\
19.95	0.00960692702647832\\
19.96	0.00960695910540973\\
19.97	0.00960699120005163\\
19.98	0.00960702331041561\\
19.99	0.00960705543651331\\
20	0.00960708757835634\\
20.01	0.00960711973595633\\
20.02	0.00960715190932495\\
20.03	0.00960718409847387\\
20.04	0.00960721630341477\\
20.05	0.00960724852415934\\
20.06	0.00960728076071929\\
20.07	0.00960731301310634\\
20.08	0.00960734528133225\\
20.09	0.00960737756540874\\
20.1	0.0096074098653476\\
20.11	0.0096074421811606\\
20.12	0.00960747451285954\\
20.13	0.00960750686045622\\
20.14	0.00960753922396246\\
20.15	0.00960757160339011\\
20.16	0.00960760399875102\\
20.17	0.00960763641005703\\
20.18	0.00960766883732005\\
20.19	0.00960770128055195\\
20.2	0.00960773373976465\\
20.21	0.00960776621497007\\
20.22	0.00960779870618015\\
20.23	0.00960783121340682\\
20.24	0.00960786373666207\\
20.25	0.00960789627595787\\
20.26	0.0096079288313062\\
20.27	0.00960796140271909\\
20.28	0.00960799399020855\\
20.29	0.00960802659378661\\
20.3	0.00960805921346534\\
20.31	0.00960809184925679\\
20.32	0.00960812450117304\\
20.33	0.0096081571692262\\
20.34	0.00960818985342836\\
20.35	0.00960822255379166\\
20.36	0.00960825527032823\\
20.37	0.00960828800305022\\
20.38	0.00960832075196981\\
20.39	0.00960835351709917\\
20.4	0.00960838629845051\\
20.41	0.00960841909603603\\
20.42	0.00960845190986797\\
20.43	0.00960848473995856\\
20.44	0.00960851758632006\\
20.45	0.00960855044896474\\
20.46	0.00960858332790489\\
20.47	0.00960861622315281\\
20.48	0.00960864913472082\\
20.49	0.00960868206262124\\
20.5	0.00960871500686643\\
20.51	0.00960874796746874\\
20.52	0.00960878094444055\\
20.53	0.00960881393779425\\
20.54	0.00960884694754225\\
20.55	0.00960887997369697\\
20.56	0.00960891301627084\\
20.57	0.00960894607527632\\
20.58	0.00960897915072588\\
20.59	0.00960901224263199\\
20.6	0.00960904535100716\\
20.61	0.00960907847586388\\
20.62	0.00960911161721471\\
20.63	0.00960914477507217\\
20.64	0.00960917794944882\\
20.65	0.00960921114035724\\
20.66	0.00960924434781002\\
20.67	0.00960927757181976\\
20.68	0.00960931081239909\\
20.69	0.00960934406956063\\
20.7	0.00960937734331704\\
20.71	0.00960941063368099\\
20.72	0.00960944394066515\\
20.73	0.00960947726428222\\
20.74	0.00960951060454493\\
20.75	0.00960954396146599\\
20.76	0.00960957733505815\\
20.77	0.00960961072533417\\
20.78	0.00960964413230683\\
20.79	0.00960967755598891\\
20.8	0.00960971099639323\\
20.81	0.00960974445353261\\
20.82	0.00960977792741988\\
20.83	0.00960981141806791\\
20.84	0.00960984492548956\\
20.85	0.00960987844969771\\
20.86	0.00960991199070528\\
20.87	0.00960994554852518\\
20.88	0.00960997912317033\\
20.89	0.00961001271465371\\
20.9	0.00961004632298826\\
20.91	0.00961007994818697\\
20.92	0.00961011359026284\\
20.93	0.00961014724922888\\
20.94	0.00961018092509813\\
20.95	0.00961021461788363\\
20.96	0.00961024832759844\\
20.97	0.00961028205425565\\
20.98	0.00961031579786834\\
20.99	0.00961034955844964\\
21	0.00961038333601265\\
21.01	0.00961041713057055\\
21.02	0.00961045094213647\\
21.03	0.0096104847707236\\
21.04	0.00961051861634513\\
21.05	0.00961055247901427\\
21.06	0.00961058635874425\\
21.07	0.00961062025554831\\
21.08	0.00961065416943971\\
21.09	0.00961068810043172\\
21.1	0.00961072204853764\\
21.11	0.00961075601377078\\
21.12	0.00961078999614445\\
21.13	0.00961082399567201\\
21.14	0.00961085801236681\\
21.15	0.00961089204624222\\
21.16	0.00961092609731165\\
21.17	0.00961096016558849\\
21.18	0.00961099425108618\\
21.19	0.00961102835381815\\
21.2	0.00961106247379787\\
21.21	0.00961109661103881\\
21.22	0.00961113076555447\\
21.23	0.00961116493735835\\
21.24	0.00961119912646398\\
21.25	0.0096112333328849\\
21.26	0.00961126755663469\\
21.27	0.00961130179772691\\
21.28	0.00961133605617516\\
21.29	0.00961137033199305\\
21.3	0.00961140462519422\\
21.31	0.00961143893579231\\
21.32	0.00961147326380097\\
21.33	0.0096115076092339\\
21.34	0.0096115419721048\\
21.35	0.00961157635242737\\
21.36	0.00961161075021535\\
21.37	0.00961164516548249\\
21.38	0.00961167959824256\\
21.39	0.00961171404850935\\
21.4	0.00961174851629665\\
21.41	0.00961178300161828\\
21.42	0.0096118175044881\\
21.43	0.00961185202491993\\
21.44	0.00961188656292767\\
21.45	0.0096119211185252\\
21.46	0.00961195569172643\\
21.47	0.00961199028254529\\
21.48	0.00961202489099571\\
21.49	0.00961205951709166\\
21.5	0.00961209416084712\\
21.51	0.00961212882227608\\
21.52	0.00961216350139256\\
21.53	0.00961219819821059\\
21.54	0.00961223291274421\\
21.55	0.00961226764500751\\
21.56	0.00961230239501456\\
21.57	0.00961233716277946\\
21.58	0.00961237194831634\\
21.59	0.00961240675163934\\
21.6	0.00961244157276261\\
21.61	0.00961247641170034\\
21.62	0.0096125112684667\\
21.63	0.00961254614307593\\
21.64	0.00961258103554224\\
21.65	0.00961261594587989\\
21.66	0.00961265087410314\\
21.67	0.00961268582022628\\
21.68	0.0096127207842636\\
21.69	0.00961275576622945\\
21.7	0.00961279076613813\\
21.71	0.00961282578400403\\
21.72	0.00961286081984152\\
21.73	0.00961289587366499\\
21.74	0.00961293094548886\\
21.75	0.00961296603532755\\
21.76	0.00961300114319553\\
21.77	0.00961303626910725\\
21.78	0.00961307141307721\\
21.79	0.00961310657511992\\
21.8	0.00961314175524989\\
21.81	0.00961317695348167\\
21.82	0.00961321216982984\\
21.83	0.00961324740430896\\
21.84	0.00961328265693364\\
21.85	0.0096133179277185\\
21.86	0.00961335321667817\\
21.87	0.00961338852382731\\
21.88	0.00961342384918061\\
21.89	0.00961345919275275\\
21.9	0.00961349455455846\\
21.91	0.00961352993461245\\
21.92	0.00961356533292949\\
21.93	0.00961360074952435\\
21.94	0.00961363618441181\\
21.95	0.00961367163760669\\
21.96	0.00961370710912383\\
21.97	0.00961374259897805\\
21.98	0.00961377810718424\\
21.99	0.00961381363375728\\
22	0.00961384917871207\\
22.01	0.00961388474206355\\
22.02	0.00961392032382666\\
22.03	0.00961395592401636\\
22.04	0.00961399154264763\\
22.05	0.00961402717973549\\
22.06	0.00961406283529496\\
22.07	0.00961409850934107\\
22.08	0.00961413420188889\\
22.09	0.00961416991295351\\
22.1	0.00961420564255001\\
22.11	0.00961424139069354\\
22.12	0.00961427715739922\\
22.13	0.00961431294268223\\
22.14	0.00961434874655773\\
22.15	0.00961438456904094\\
22.16	0.00961442041014708\\
22.17	0.00961445626989138\\
22.18	0.00961449214828911\\
22.19	0.00961452804535555\\
22.2	0.00961456396110599\\
22.21	0.00961459989555577\\
22.22	0.00961463584872022\\
22.23	0.0096146718206147\\
22.24	0.00961470781125461\\
22.25	0.00961474382065533\\
22.26	0.0096147798488323\\
22.27	0.00961481589580095\\
22.28	0.00961485196157675\\
22.29	0.00961488804617518\\
22.3	0.00961492414961176\\
22.31	0.00961496027190199\\
22.32	0.00961499641306143\\
22.33	0.00961503257310564\\
22.34	0.00961506875205021\\
22.35	0.00961510494991076\\
22.36	0.00961514116670289\\
22.37	0.00961517740244227\\
22.38	0.00961521365714457\\
22.39	0.00961524993082546\\
22.4	0.00961528622350068\\
22.41	0.00961532253518594\\
22.42	0.00961535886589699\\
22.43	0.00961539521564962\\
22.44	0.00961543158445962\\
22.45	0.0096154679723428\\
22.46	0.00961550437931501\\
22.47	0.00961554080539209\\
22.48	0.00961557725058993\\
22.49	0.00961561371492442\\
22.5	0.0096156501984115\\
22.51	0.00961568670106711\\
22.52	0.00961572322290719\\
22.53	0.00961575976394776\\
22.54	0.0096157963242048\\
22.55	0.00961583290369435\\
22.56	0.00961586950243246\\
22.57	0.00961590612043519\\
22.58	0.00961594275771865\\
22.59	0.00961597941429895\\
22.6	0.00961601609019222\\
22.61	0.00961605278541463\\
22.62	0.00961608949998234\\
22.63	0.00961612623391157\\
22.64	0.00961616298721853\\
22.65	0.00961619975991947\\
22.66	0.00961623655203067\\
22.67	0.0096162733635684\\
22.68	0.00961631019454897\\
22.69	0.00961634704498873\\
22.7	0.00961638391490402\\
22.71	0.00961642080431123\\
22.72	0.00961645771322675\\
22.73	0.009616494641667\\
22.74	0.00961653158964843\\
22.75	0.0096165685571875\\
22.76	0.00961660554430071\\
22.77	0.00961664255100456\\
22.78	0.00961667957731558\\
22.79	0.00961671662325033\\
22.8	0.00961675368882539\\
22.81	0.00961679077405737\\
22.82	0.00961682787896287\\
22.83	0.00961686500355856\\
22.84	0.00961690214786109\\
22.85	0.00961693931188716\\
22.86	0.00961697649565348\\
22.87	0.00961701369917679\\
22.88	0.00961705092247385\\
22.89	0.00961708816556145\\
22.9	0.00961712542845638\\
22.91	0.00961716271117547\\
22.92	0.00961720001373558\\
22.93	0.00961723733615359\\
22.94	0.00961727467844638\\
22.95	0.00961731204063088\\
22.96	0.00961734942272404\\
22.97	0.00961738682474283\\
22.98	0.00961742424670422\\
22.99	0.00961746168862525\\
23	0.00961749915052294\\
23.01	0.00961753663241435\\
23.02	0.00961757413431658\\
23.03	0.00961761165624673\\
23.04	0.00961764919822193\\
23.05	0.00961768676025933\\
23.06	0.00961772434237613\\
23.07	0.00961776194458951\\
23.08	0.00961779956691671\\
23.09	0.00961783720937498\\
23.1	0.00961787487198159\\
23.11	0.00961791255475384\\
23.12	0.00961795025770905\\
23.13	0.00961798798086458\\
23.14	0.00961802572423778\\
23.15	0.00961806348784606\\
23.16	0.00961810127170683\\
23.17	0.00961813907583755\\
23.18	0.00961817690025567\\
23.19	0.0096182147449787\\
23.2	0.00961825261002413\\
23.21	0.00961829049540953\\
23.22	0.00961832840115245\\
23.23	0.00961836632727048\\
23.24	0.00961840427378124\\
23.25	0.00961844224070237\\
23.26	0.00961848022805153\\
23.27	0.00961851823584642\\
23.28	0.00961855626410474\\
23.29	0.00961859431284423\\
23.3	0.00961863238208267\\
23.31	0.00961867047183783\\
23.32	0.00961870858212754\\
23.33	0.00961874671296963\\
23.34	0.00961878486438197\\
23.35	0.00961882303638244\\
23.36	0.00961886122898897\\
23.37	0.00961889944221949\\
23.38	0.00961893767609197\\
23.39	0.0096189759306244\\
23.4	0.00961901420583479\\
23.41	0.0096190525017412\\
23.42	0.00961909081836168\\
23.43	0.00961912915571435\\
23.44	0.0096191675138173\\
23.45	0.00961920589268869\\
23.46	0.00961924429234669\\
23.47	0.0096192827128095\\
23.48	0.00961932115409534\\
23.49	0.00961935961622247\\
23.5	0.00961939809920915\\
23.51	0.0096194366030737\\
23.52	0.00961947512783443\\
23.53	0.00961951367350971\\
23.54	0.0096195522401179\\
23.55	0.00961959082767744\\
23.56	0.00961962943620675\\
23.57	0.00961966806572428\\
23.58	0.00961970671624853\\
23.59	0.00961974538779801\\
23.6	0.00961978408039126\\
23.61	0.00961982279404686\\
23.62	0.00961986152878339\\
23.63	0.00961990028461948\\
23.64	0.00961993906157378\\
23.65	0.00961997785966496\\
23.66	0.00962001667891173\\
23.67	0.00962005551933283\\
23.68	0.009620094380947\\
23.69	0.00962013326377303\\
23.7	0.00962017216782974\\
23.71	0.00962021109313597\\
23.72	0.00962025003971059\\
23.73	0.00962028900757249\\
23.74	0.0096203279967406\\
23.75	0.00962036700723386\\
23.76	0.00962040603907128\\
23.77	0.00962044509227184\\
23.78	0.00962048416685458\\
23.79	0.00962052326283857\\
23.8	0.00962056238024291\\
23.81	0.00962060151908671\\
23.82	0.00962064067938912\\
23.83	0.00962067986116933\\
23.84	0.00962071906444653\\
23.85	0.00962075828923997\\
23.86	0.00962079753556891\\
23.87	0.00962083680345264\\
23.88	0.00962087609291048\\
23.89	0.00962091540396179\\
23.9	0.00962095473662595\\
23.91	0.00962099409092236\\
23.92	0.00962103346687046\\
23.93	0.00962107286448972\\
23.94	0.00962111228379965\\
23.95	0.00962115172481976\\
23.96	0.00962119118756961\\
23.97	0.00962123067206879\\
23.98	0.00962127017833692\\
23.99	0.00962130970639365\\
24	0.00962134925625864\\
24.01	0.0096213888279516\\
24.02	0.00962142842149228\\
24.03	0.00962146803690044\\
24.04	0.00962150767419587\\
24.05	0.00962154733339841\\
24.06	0.00962158701452791\\
24.07	0.00962162671760426\\
24.08	0.00962166644264739\\
24.09	0.00962170618967723\\
24.1	0.00962174595871378\\
24.11	0.00962178574977705\\
24.12	0.00962182556288708\\
24.13	0.00962186539806395\\
24.14	0.00962190525532776\\
24.15	0.00962194513469866\\
24.16	0.00962198503619681\\
24.17	0.00962202495984241\\
24.18	0.0096220649056557\\
24.19	0.00962210487365695\\
24.2	0.00962214486386645\\
24.21	0.00962218487630454\\
24.22	0.00962222491099157\\
24.23	0.00962226496794795\\
24.24	0.00962230504719409\\
24.25	0.00962234514875047\\
24.26	0.00962238527263757\\
24.27	0.00962242541887591\\
24.28	0.00962246558748607\\
24.29	0.00962250577848863\\
24.3	0.00962254599190421\\
24.31	0.00962258622775349\\
24.32	0.00962262648605715\\
24.33	0.00962266676683591\\
24.34	0.00962270707011055\\
24.35	0.00962274739590184\\
24.36	0.00962278774423063\\
24.37	0.00962282811511779\\
24.38	0.00962286850858419\\
24.39	0.00962290892465078\\
24.4	0.00962294936333852\\
24.41	0.00962298982466843\\
24.42	0.00962303030866153\\
24.43	0.0096230708153389\\
24.44	0.00962311134472164\\
24.45	0.00962315189683091\\
24.46	0.00962319247168787\\
24.47	0.00962323306931375\\
24.48	0.00962327368972979\\
24.49	0.00962331433295729\\
24.5	0.00962335499901757\\
24.51	0.00962339568793198\\
24.52	0.00962343639972194\\
24.53	0.00962347713440886\\
24.54	0.00962351789201422\\
24.55	0.00962355867255953\\
24.56	0.00962359947606634\\
24.57	0.00962364030255623\\
24.58	0.00962368115205082\\
24.59	0.00962372202457177\\
24.6	0.00962376292014078\\
24.61	0.00962380383877959\\
24.62	0.00962384478050997\\
24.63	0.00962388574535373\\
24.64	0.00962392673333273\\
24.65	0.00962396774446886\\
24.66	0.00962400877878404\\
24.67	0.00962404983630026\\
24.68	0.00962409091703952\\
24.69	0.00962413202102389\\
24.7	0.00962417314827543\\
24.71	0.00962421429881629\\
24.72	0.00962425547266865\\
24.73	0.00962429666985471\\
24.74	0.00962433789039673\\
24.75	0.00962437913431702\\
24.76	0.0096244204016379\\
24.77	0.00962446169238176\\
24.78	0.00962450300657102\\
24.79	0.00962454434422816\\
24.8	0.00962458570537567\\
24.81	0.00962462709003611\\
24.82	0.00962466849823207\\
24.83	0.0096247099299862\\
24.84	0.00962475138532118\\
24.85	0.00962479286425974\\
24.86	0.00962483436682464\\
24.87	0.00962487589303871\\
24.88	0.0096249174429248\\
24.89	0.00962495901650583\\
24.9	0.00962500061380474\\
24.91	0.00962504223484454\\
24.92	0.00962508387964827\\
24.93	0.00962512554823903\\
24.94	0.00962516724063995\\
24.95	0.00962520895687423\\
24.96	0.00962525069696509\\
24.97	0.00962529246093583\\
24.98	0.00962533424880977\\
24.99	0.00962537606061028\\
25	0.00962541789636082\\
25.01	0.00962545975608485\\
25.02	0.00962550163980589\\
25.03	0.00962554354754755\\
25.04	0.00962558547933343\\
25.05	0.00962562743518724\\
25.06	0.00962566941513269\\
25.07	0.00962571141919358\\
25.08	0.00962575344739374\\
25.09	0.00962579549975706\\
25.1	0.0096258375763075\\
25.11	0.00962587967706905\\
25.12	0.00962592180206575\\
25.13	0.00962596395132173\\
25.14	0.00962600612486114\\
25.15	0.0096260483227082\\
25.16	0.00962609054488718\\
25.17	0.00962613279142243\\
25.18	0.00962617506233832\\
25.19	0.0096262173576593\\
25.2	0.00962625967740988\\
25.21	0.00962630202161463\\
25.22	0.00962634439029817\\
25.23	0.00962638678348517\\
25.24	0.00962642920120039\\
25.25	0.00962647164346862\\
25.26	0.00962651411031475\\
25.27	0.00962655660176369\\
25.28	0.00962659911784044\\
25.29	0.00962664165857006\\
25.3	0.00962668422397766\\
25.31	0.00962672681408843\\
25.32	0.00962676942892762\\
25.33	0.00962681206852056\\
25.34	0.00962685473289261\\
25.35	0.00962689742206925\\
25.36	0.00962694013607598\\
25.37	0.0096269828749384\\
25.38	0.00962702563868216\\
25.39	0.00962706842733301\\
25.4	0.00962711124091672\\
25.41	0.0096271540794592\\
25.42	0.00962719694298638\\
25.43	0.00962723983152428\\
25.44	0.009627282745099\\
25.45	0.00962732568373672\\
25.46	0.00962736864746367\\
25.47	0.00962741163630619\\
25.48	0.00962745465029068\\
25.49	0.00962749768944362\\
25.5	0.00962754075379159\\
25.51	0.00962758384336122\\
25.52	0.00962762695817924\\
25.53	0.00962767009827246\\
25.54	0.00962771326366779\\
25.55	0.0096277564543922\\
25.56	0.00962779967047275\\
25.57	0.00962784291193661\\
25.58	0.00962788617881102\\
25.59	0.0096279294711233\\
25.6	0.00962797278890089\\
25.61	0.0096280161321713\\
25.62	0.00962805950096215\\
25.63	0.00962810289530113\\
25.64	0.00962814631521605\\
25.65	0.00962818976073481\\
25.66	0.0096282332318854\\
25.67	0.00962827672869593\\
25.68	0.00962832025119459\\
25.69	0.00962836379940968\\
25.7	0.0096284073733696\\
25.71	0.00962845097310288\\
25.72	0.00962849459863812\\
25.73	0.00962853825000407\\
25.74	0.00962858192722954\\
25.75	0.0096286256303435\\
25.76	0.009628669359375\\
25.77	0.00962871311435323\\
25.78	0.00962875689530748\\
25.79	0.00962880070226717\\
25.8	0.00962884453526183\\
25.81	0.00962888839432112\\
25.82	0.00962893227947482\\
25.83	0.00962897619075284\\
25.84	0.00962902012818521\\
25.85	0.0096290640918021\\
25.86	0.00962910808163381\\
25.87	0.00962915209771077\\
25.88	0.00962919614006355\\
25.89	0.00962924020872286\\
25.9	0.00962928430371953\\
25.91	0.00962932842508459\\
25.92	0.00962937257284913\\
25.93	0.00962941674704447\\
25.94	0.00962946094770202\\
25.95	0.00962950517485339\\
25.96	0.00962954942853029\\
25.97	0.00962959370876464\\
25.98	0.00962963801558851\\
25.99	0.0096296823490341\\
26	0.00962972670913381\\
26.01	0.00962977109592019\\
26.02	0.00962981550942598\\
26.03	0.00962985994968407\\
26.04	0.00962990441672754\\
26.05	0.00962994891058967\\
26.06	0.00962999343130388\\
26.07	0.00963003797890383\\
26.08	0.00963008255342331\\
26.09	0.00963012715489635\\
26.1	0.00963017178335717\\
26.11	0.00963021643884016\\
26.12	0.00963026112137996\\
26.13	0.00963030583101139\\
26.14	0.00963035056776947\\
26.15	0.00963039533168948\\
26.16	0.00963044012280686\\
26.17	0.00963048494115733\\
26.18	0.0096305297867768\\
26.19	0.00963057465970143\\
26.2	0.0096306195599676\\
26.21	0.00963066448761194\\
26.22	0.00963070944267135\\
26.23	0.00963075442518293\\
26.24	0.00963079943518406\\
26.25	0.00963084447271239\\
26.26	0.00963088953780583\\
26.27	0.00963093463050254\\
26.28	0.00963097975084097\\
26.29	0.00963102489885985\\
26.3	0.00963107007459818\\
26.31	0.00963111527809528\\
26.32	0.00963116050939073\\
26.33	0.00963120576852446\\
26.34	0.00963125105553665\\
26.35	0.00963129637046784\\
26.36	0.00963134171335887\\
26.37	0.00963138708425092\\
26.38	0.00963143248318549\\
26.39	0.00963147791020442\\
26.4	0.0096315233653499\\
26.41	0.00963156884866449\\
26.42	0.00963161436019109\\
26.43	0.00963165989997298\\
26.44	0.0096317054680538\\
26.45	0.00963175106447761\\
26.46	0.00963179668928882\\
26.47	0.00963184234253226\\
26.48	0.00963188802425317\\
26.49	0.00963193373449719\\
26.5	0.00963197947331041\\
26.51	0.00963202524073934\\
26.52	0.00963207103683094\\
26.53	0.00963211686163261\\
26.54	0.00963216271519221\\
26.55	0.00963220859755809\\
26.56	0.00963225450877908\\
26.57	0.00963230044890448\\
26.58	0.00963234641798412\\
26.59	0.00963239241606832\\
26.6	0.00963243844320794\\
26.61	0.00963248449945436\\
26.62	0.00963253058485953\\
26.63	0.00963257669947593\\
26.64	0.00963262284335664\\
26.65	0.00963266901655531\\
26.66	0.00963271521912618\\
26.67	0.00963276145112409\\
26.68	0.00963280771260454\\
26.69	0.00963285400362364\\
26.7	0.00963290032423814\\
26.71	0.00963294667450548\\
26.72	0.00963299305448376\\
26.73	0.00963303946423177\\
26.74	0.00963308590380904\\
26.75	0.00963313237327579\\
26.76	0.00963317887269301\\
26.77	0.00963322540212243\\
26.78	0.00963327196162655\\
26.79	0.00963331855126869\\
26.8	0.00963336517111295\\
26.81	0.00963341182122428\\
26.82	0.00963345850166846\\
26.83	0.00963350521251216\\
26.84	0.0096335519538229\\
26.85	0.00963359872566916\\
26.86	0.00963364552812029\\
26.87	0.00963369236124663\\
26.88	0.00963373922511948\\
26.89	0.00963378611981112\\
26.9	0.00963383304539485\\
26.91	0.00963388000194503\\
26.92	0.00963392698953705\\
26.93	0.00963397400824742\\
26.94	0.00963402105815373\\
26.95	0.00963406813933475\\
26.96	0.00963411525187038\\
26.97	0.00963416239584171\\
26.98	0.00963420957133109\\
26.99	0.00963425677842209\\
27	0.00963430401719956\\
27.01	0.00963435128774966\\
27.02	0.0096343985901599\\
27.03	0.00963444592451915\\
27.04	0.00963449329091769\\
27.05	0.00963454068944724\\
27.06	0.00963458812020098\\
27.07	0.00963463558327362\\
27.08	0.00963468307876138\\
27.09	0.00963473060676208\\
27.1	0.00963477816737516\\
27.11	0.00963482576070171\\
27.12	0.00963487338684452\\
27.13	0.0096349210459081\\
27.14	0.00963496873799876\\
27.15	0.00963501646322462\\
27.16	0.00963506422169568\\
27.17	0.00963511201352383\\
27.18	0.00963515983882293\\
27.19	0.00963520769770885\\
27.2	0.0096352555902995\\
27.21	0.0096353035167149\\
27.22	0.00963535147707724\\
27.23	0.0096353994715109\\
27.24	0.00963544750014253\\
27.25	0.0096354955631011\\
27.26	0.00963554366051795\\
27.27	0.00963559179252686\\
27.28	0.0096356399592641\\
27.29	0.0096356881608685\\
27.3	0.0096357363974815\\
27.31	0.00963578466924725\\
27.32	0.00963583297631261\\
27.33	0.0096358813188273\\
27.34	0.00963592969694389\\
27.35	0.00963597811081797\\
27.36	0.00963602656060809\\
27.37	0.00963607504647598\\
27.38	0.00963612356858653\\
27.39	0.00963617212710789\\
27.4	0.00963622072221159\\
27.41	0.00963626935407257\\
27.42	0.00963631802286932\\
27.43	0.00963636672878392\\
27.44	0.00963641547200218\\
27.45	0.00963646425271368\\
27.46	0.00963651307111191\\
27.47	0.00963656192739436\\
27.48	0.0096366108217626\\
27.49	0.00963665975442241\\
27.5	0.00963670872558388\\
27.51	0.00963675773546151\\
27.52	0.00963680678427433\\
27.53	0.00963685587224603\\
27.54	0.00963690499960505\\
27.55	0.00963695416658472\\
27.56	0.00963700337342341\\
27.57	0.00963705262036459\\
27.58	0.00963710190765704\\
27.59	0.00963715123555495\\
27.6	0.00963720060431805\\
27.61	0.00963725001421178\\
27.62	0.00963729946550741\\
27.63	0.00963734895848223\\
27.64	0.00963739849341966\\
27.65	0.00963744807060944\\
27.66	0.00963749769034781\\
27.67	0.00963754735293762\\
27.68	0.00963759705868857\\
27.69	0.00963764680791736\\
27.7	0.00963769660094786\\
27.71	0.00963774643811132\\
27.72	0.00963779631974656\\
27.73	0.00963784624620016\\
27.74	0.00963789621782668\\
27.75	0.00963794623498885\\
27.76	0.0096379962980578\\
27.77	0.0096380464074133\\
27.78	0.00963809656344394\\
27.79	0.00963814676654742\\
27.8	0.00963819701713075\\
27.81	0.00963824731561051\\
27.82	0.00963829766241311\\
27.83	0.00963834805797508\\
27.84	0.00963839850274326\\
27.85	0.00963844899717517\\
27.86	0.00963849954173924\\
27.87	0.00963855013691509\\
27.88	0.00963860078319389\\
27.89	0.00963865148107863\\
27.9	0.00963870223108442\\
27.91	0.00963875303373886\\
27.92	0.00963880388958237\\
27.93	0.0096388547991685\\
27.94	0.0096389057630643\\
27.95	0.00963895678185074\\
27.96	0.00963900785612298\\
27.97	0.00963905898649088\\
27.98	0.00963911017357928\\
27.99	0.00963916141802849\\
28	0.00963921272049469\\
28.01	0.00963926408165035\\
28.02	0.00963931550218467\\
28.03	0.00963936698280406\\
28.04	0.0096394185242326\\
28.05	0.00963947012721252\\
28.06	0.0096395217925047\\
28.07	0.00963957352088919\\
28.08	0.00963962531316573\\
28.09	0.00963967717015428\\
28.1	0.00963972909269559\\
28.11	0.00963978108165178\\
28.12	0.00963983313790694\\
28.13	0.00963988526236767\\
28.14	0.00963993745596379\\
28.15	0.00963998971964895\\
28.16	0.00964004205440125\\
28.17	0.00964009446122398\\
28.18	0.00964014694114631\\
28.19	0.00964019949522395\\
28.2	0.00964025212453998\\
28.21	0.00964030483020553\\
28.22	0.00964035761336063\\
28.23	0.00964041047517496\\
28.24	0.00964046341684874\\
28.25	0.00964051643961352\\
28.26	0.00964056954473311\\
28.27	0.00964062273350446\\
28.28	0.00964067600725859\\
28.29	0.00964072936736157\\
28.3	0.00964078281521548\\
28.31	0.00964083635225945\\
28.32	0.00964088997997068\\
28.33	0.00964094369986552\\
28.34	0.00964099751350061\\
28.35	0.00964105142247398\\
28.36	0.00964110542842622\\
28.37	0.00964115953304174\\
28.38	0.00964121373804994\\
28.39	0.00964126804522657\\
28.4	0.00964132245639496\\
28.41	0.00964137697342745\\
28.42	0.00964143159824675\\
28.43	0.00964148633282737\\
28.44	0.00964154117919715\\
28.45	0.0096415961394387\\
28.46	0.00964165121569107\\
28.47	0.00964170641015125\\
28.48	0.00964176172507597\\
28.49	0.00964181716278327\\
28.5	0.00964187272565438\\
28.51	0.00964192841613547\\
28.52	0.00964198423673952\\
28.53	0.00964204019004831\\
28.54	0.00964209627871433\\
28.55	0.00964215250546284\\
28.56	0.00964220887309402\\
28.57	0.00964226538448506\\
28.58	0.00964232204259247\\
28.59	0.0096423788504543\\
28.6	0.00964243581119259\\
28.61	0.00964249292801574\\
28.62	0.00964255020422105\\
28.63	0.00964260764319731\\
28.64	0.00964266524842744\\
28.65	0.00964272302349129\\
28.66	0.00964278097206838\\
28.67	0.0096428390979409\\
28.68	0.00964289740499665\\
28.69	0.00964295589723216\\
28.7	0.00964301457875588\\
28.71	0.00964307345379143\\
28.72	0.009643132526681\\
28.73	0.00964319180188882\\
28.74	0.00964325128400478\\
28.75	0.00964331097774808\\
28.76	0.00964337088797107\\
28.77	0.00964343101966317\\
28.78	0.00964349137795489\\
28.79	0.00964355196812204\\
28.8	0.00964361279558996\\
28.81	0.00964367386593802\\
28.82	0.00964373518490412\\
28.83	0.0096437967583894\\
28.84	0.00964385859246311\\
28.85	0.00964392069336755\\
28.86	0.00964398306752326\\
28.87	0.00964404572153429\\
28.88	0.00964410866219366\\
28.89	0.009644171896489\\
28.9	0.00964423543160837\\
28.91	0.00964429927494617\\
28.92	0.00964436343410939\\
28.93	0.00964442791692389\\
28.94	0.009644492731441\\
28.95	0.00964455788594422\\
28.96	0.00964462338895622\\
28.97	0.00964468924924599\\
28.98	0.00964475547583622\\
28.99	0.00964482207801095\\
29	0.00964488906532338\\
29.01	0.00964495644760401\\
29.02	0.00964502423496897\\
29.03	0.00964509243782863\\
29.04	0.00964516106689647\\
29.05	0.00964523013319823\\
29.06	0.00964529964808131\\
29.07	0.00964536962322454\\
29.08	0.00964544007064821\\
29.09	0.00964551100272432\\
29.1	0.00964558243218737\\
29.11	0.00964565437214524\\
29.12	0.00964572683609054\\
29.13	0.00964579983791235\\
29.14	0.00964587339190822\\
29.15	0.00964594751279658\\
29.16	0.00964602221572962\\
29.17	0.00964609751630642\\
29.18	0.00964617343058661\\
29.19	0.00964624997510445\\
29.2	0.00964632716688327\\
29.21	0.00964640502345041\\
29.22	0.00964648356285269\\
29.23	0.00964656280367225\\
29.24	0.00964664276504294\\
29.25	0.00964672346666727\\
29.26	0.00964680492883381\\
29.27	0.00964688717243517\\
29.28	0.0096469702189866\\
29.29	0.00964705409064506\\
29.3	0.00964713881022899\\
29.31	0.00964722440123866\\
29.32	0.00964731088787717\\
29.33	0.00964739829507212\\
29.34	0.00964748664849794\\
29.35	0.00964757597459894\\
29.36	0.00964766630061309\\
29.37	0.00964775765459655\\
29.38	0.00964785006544898\\
29.39	0.00964794356293962\\
29.4	0.00964803817773427\\
29.41	0.00964813394142303\\
29.42	0.00964823088654896\\
29.43	0.0096483290466377\\
29.44	0.00964842845622794\\
29.45	0.00964852915090294\\
29.46	0.00964863116732295\\
29.47	0.00964873454325879\\
29.48	0.00964883931762639\\
29.49	0.00964894553052248\\
29.5	0.00964905322326143\\
29.51	0.00964916243841316\\
29.52	0.00964927321984246\\
29.53	0.00964938561274933\\
29.54	0.00964949966371077\\
29.55	0.00964961542072387\\
29.56	0.0096497314568792\\
29.57	0.00964984753078794\\
29.58	0.00964996364213344\\
29.59	0.00965007979059063\\
29.6	0.00965019597582593\\
29.61	0.00965031219749714\\
29.62	0.00965042845525335\\
29.63	0.00965054474873485\\
29.64	0.00965066107757302\\
29.65	0.0096507774413903\\
29.66	0.00965089383980005\\
29.67	0.00965101027240652\\
29.68	0.00965112673880476\\
29.69	0.00965124323858059\\
29.7	0.0096513597713105\\
29.71	0.00965147633656166\\
29.72	0.00965159293389181\\
29.73	0.00965170956284932\\
29.74	0.00965182622297309\\
29.75	0.00965194291379257\\
29.76	0.00965205963482777\\
29.77	0.00965217638558923\\
29.78	0.0096522931655781\\
29.79	0.00965240997428611\\
29.8	0.00965252681119565\\
29.81	0.00965264367577984\\
29.82	0.00965276056750258\\
29.83	0.00965287748581863\\
29.84	0.00965299443017376\\
29.85	0.00965311140000484\\
29.86	0.00965322839473996\\
29.87	0.00965334541379867\\
29.88	0.00965346245659206\\
29.89	0.00965357952252304\\
29.9	0.00965369661098654\\
29.91	0.00965381372136975\\
29.92	0.00965393085305239\\
29.93	0.00965404800540706\\
29.94	0.0096541651777995\\
29.95	0.00965428236958901\\
29.96	0.00965439958012879\\
29.97	0.0096545168087664\\
29.98	0.00965463405484421\\
29.99	0.0096547513176999\\
30	0.00965486859666693\\
30.01	0.00965498589107521\\
30.02	0.00965510320025165\\
30.03	0.00965522052352082\\
30.04	0.00965533786020569\\
30.05	0.00965545520962833\\
30.06	0.00965557257111076\\
30.07	0.00965568994397577\\
30.08	0.00965580732754784\\
30.09	0.00965592472115413\\
30.1	0.00965604212412549\\
30.11	0.00965615953579757\\
30.12	0.00965627695551193\\
30.13	0.00965639438261736\\
30.14	0.00965651181647112\\
30.15	0.00965662925644033\\
30.16	0.00965674670190344\\
30.17	0.0096568641522518\\
30.18	0.00965698160689124\\
30.19	0.00965709906524384\\
30.2	0.00965721652674974\\
30.21	0.00965733399086905\\
30.22	0.00965745145708388\\
30.23	0.00965756892490049\\
30.24	0.00965768639385154\\
30.25	0.00965780386349841\\
30.26	0.00965792133343377\\
30.27	0.00965803880328415\\
30.28	0.0096581562727127\\
30.29	0.00965827374142214\\
30.3	0.00965839120915774\\
30.31	0.00965850867571057\\
30.32	0.00965862614092086\\
30.33	0.0096587436046815\\
30.34	0.00965886106694178\\
30.35	0.00965897852771125\\
30.36	0.00965909598706384\\
30.37	0.00965921344514208\\
30.38	0.00965933090216164\\
30.39	0.00965944835841597\\
30.4	0.00965956581428128\\
30.41	0.00965968327022166\\
30.42	0.00965980072679448\\
30.43	0.00965991818465607\\
30.44	0.0096600356445676\\
30.45	0.00966015310740132\\
30.46	0.00966027057414698\\
30.47	0.00966038804591866\\
30.48	0.00966050552396183\\
30.49	0.00966062300966077\\
30.5	0.0096607405045463\\
30.51	0.00966085801030392\\
30.52	0.00966097552878223\\
30.53	0.00966109306200179\\
30.54	0.00966121061216436\\
30.55	0.00966132818166253\\
30.56	0.00966144577308982\\
30.57	0.0096615633892512\\
30.58	0.00966168103317401\\
30.59	0.00966179870811953\\
30.6	0.00966191641759483\\
30.61	0.00966203416536535\\
30.62	0.00966215195546786\\
30.63	0.00966226979013222\\
30.64	0.00966238766937752\\
30.65	0.00966250559322285\\
30.66	0.00966262356168734\\
30.67	0.00966274157479011\\
30.68	0.00966285963255027\\
30.69	0.00966297773498698\\
30.7	0.00966309588211938\\
30.71	0.00966321407396663\\
30.72	0.00966333231054792\\
30.73	0.0096634505918824\\
30.74	0.00966356891798929\\
30.75	0.00966368728888777\\
30.76	0.00966380570459706\\
30.77	0.00966392416513638\\
30.78	0.00966404267052496\\
30.79	0.00966416122078203\\
30.8	0.00966427981592686\\
30.81	0.0096643984559787\\
30.82	0.00966451714095683\\
30.83	0.00966463587088052\\
30.84	0.00966475464576907\\
30.85	0.00966487346564177\\
30.86	0.00966499233051794\\
30.87	0.0096651112404169\\
30.88	0.00966523019535797\\
30.89	0.00966534919536052\\
30.9	0.00966546824044387\\
30.91	0.00966558733062741\\
30.92	0.00966570646593049\\
30.93	0.0096658256463725\\
30.94	0.00966594487197284\\
30.95	0.0096660641427509\\
30.96	0.0096661834587261\\
30.97	0.00966630281991786\\
30.98	0.00966642222634562\\
30.99	0.00966654167802881\\
31	0.00966666117498689\\
31.01	0.00966678071723932\\
31.02	0.00966690030480559\\
31.03	0.00966701993770516\\
31.04	0.00966713961595754\\
31.05	0.00966725933958223\\
31.06	0.00966737910859874\\
31.07	0.0096674989230266\\
31.08	0.00966761878288535\\
31.09	0.00966773868819453\\
31.1	0.00966785863897369\\
31.11	0.0096679786352424\\
31.12	0.00966809867702024\\
31.13	0.0096682187643268\\
31.14	0.00966833889718166\\
31.15	0.00966845907560444\\
31.16	0.00966857929961475\\
31.17	0.00966869956923223\\
31.18	0.00966881988447651\\
31.19	0.00966894024536723\\
31.2	0.00966906065192406\\
31.21	0.00966918110416666\\
31.22	0.00966930160211472\\
31.23	0.00966942214578792\\
31.24	0.00966954273520597\\
31.25	0.00966966337038856\\
31.26	0.00966978405135543\\
31.27	0.00966990477812631\\
31.28	0.00967002555072093\\
31.29	0.00967014636915905\\
31.3	0.00967026723346043\\
31.31	0.00967038814364484\\
31.32	0.00967050909973207\\
31.33	0.0096706301017419\\
31.34	0.00967075114969415\\
31.35	0.00967087224360862\\
31.36	0.00967099338350515\\
31.37	0.00967111456940356\\
31.38	0.00967123580132369\\
31.39	0.00967135707928542\\
31.4	0.0096714784033086\\
31.41	0.0096715997734131\\
31.42	0.00967172118961882\\
31.43	0.00967184265194567\\
31.44	0.00967196416041353\\
31.45	0.00967208571504233\\
31.46	0.009672207315852\\
31.47	0.00967232896286249\\
31.48	0.00967245065609373\\
31.49	0.0096725723955657\\
31.5	0.00967269418129835\\
31.51	0.00967281601331169\\
31.52	0.00967293789162569\\
31.53	0.00967305981626036\\
31.54	0.00967318178723571\\
31.55	0.00967330380457177\\
31.56	0.00967342586828857\\
31.57	0.00967354797840616\\
31.58	0.00967367013494459\\
31.59	0.00967379233792393\\
31.6	0.00967391458736426\\
31.61	0.00967403688328565\\
31.62	0.00967415922570822\\
31.63	0.00967428161465208\\
31.64	0.00967440405013733\\
31.65	0.00967452653218411\\
31.66	0.00967464906081256\\
31.67	0.00967477163604283\\
31.68	0.00967489425789508\\
31.69	0.00967501692638949\\
31.7	0.00967513964154625\\
31.71	0.00967526240338553\\
31.72	0.00967538521192756\\
31.73	0.00967550806719253\\
31.74	0.00967563096920069\\
31.75	0.00967575391797226\\
31.76	0.0096758769135275\\
31.77	0.00967599995588666\\
31.78	0.00967612304507001\\
31.79	0.00967624618109783\\
31.8	0.00967636936399041\\
31.81	0.00967649259376805\\
31.82	0.00967661587045106\\
31.83	0.00967673919405978\\
31.84	0.00967686256461452\\
31.85	0.00967698598213564\\
31.86	0.00967710944664349\\
31.87	0.00967723295815843\\
31.88	0.00967735651670085\\
31.89	0.00967748012229112\\
31.9	0.00967760377494965\\
31.91	0.00967772747469684\\
31.92	0.00967785122155312\\
31.93	0.00967797501553892\\
31.94	0.00967809885667468\\
31.95	0.00967822274498084\\
31.96	0.00967834668047788\\
31.97	0.00967847066318627\\
31.98	0.00967859469312649\\
31.99	0.00967871877031904\\
32	0.00967884289478443\\
32.01	0.00967896706654316\\
32.02	0.00967909128561578\\
32.03	0.00967921555202282\\
32.04	0.00967933986578483\\
32.05	0.00967946422692237\\
32.06	0.00967958863545602\\
32.07	0.00967971309140636\\
32.08	0.00967983759479397\\
32.09	0.00967996214563947\\
32.1	0.00968008674396348\\
32.11	0.00968021138978661\\
32.12	0.00968033608312951\\
32.13	0.00968046082401283\\
32.14	0.00968058561245722\\
32.15	0.00968071044848337\\
32.16	0.00968083533211193\\
32.17	0.00968096026336362\\
32.18	0.00968108524225914\\
32.19	0.0096812102688192\\
32.2	0.00968133534306452\\
32.21	0.00968146046501585\\
32.22	0.00968158563469393\\
32.23	0.00968171085211951\\
32.24	0.00968183611731338\\
32.25	0.00968196143029632\\
32.26	0.0096820867910891\\
32.27	0.00968221219971254\\
32.28	0.00968233765618746\\
32.29	0.00968246316053467\\
32.3	0.00968258871277502\\
32.31	0.00968271431292935\\
32.32	0.00968283996101852\\
32.33	0.0096829656570634\\
32.34	0.00968309140108487\\
32.35	0.00968321719310383\\
32.36	0.00968334303314118\\
32.37	0.00968346892121782\\
32.38	0.0096835948573547\\
32.39	0.00968372084157274\\
32.4	0.0096838468738929\\
32.41	0.00968397295433613\\
32.42	0.0096840990829234\\
32.43	0.0096842252596757\\
32.44	0.00968435148461401\\
32.45	0.00968447775775935\\
32.46	0.00968460407913273\\
32.47	0.00968473044875517\\
32.48	0.00968485686664771\\
32.49	0.0096849833328314\\
32.5	0.00968510984732731\\
32.51	0.0096852364101565\\
32.52	0.00968536302134006\\
32.53	0.00968548968089907\\
32.54	0.00968561638885465\\
32.55	0.00968574314522792\\
32.56	0.00968586995003999\\
32.57	0.00968599680331202\\
32.58	0.00968612370506514\\
32.59	0.00968625065532054\\
32.6	0.00968637765409936\\
32.61	0.00968650470142281\\
32.62	0.00968663179731208\\
32.63	0.00968675894178836\\
32.64	0.0096868861348729\\
32.65	0.00968701337658691\\
32.66	0.00968714066695163\\
32.67	0.00968726800598832\\
32.68	0.00968739539371824\\
32.69	0.00968752283016267\\
32.7	0.0096876503153429\\
32.71	0.00968777784928022\\
32.72	0.00968790543199595\\
32.73	0.0096880330635114\\
32.74	0.00968816074384791\\
32.75	0.00968828847302682\\
32.76	0.00968841625106949\\
32.77	0.00968854407799729\\
32.78	0.00968867195383159\\
32.79	0.00968879987859378\\
32.8	0.00968892785230527\\
32.81	0.00968905587498747\\
32.82	0.00968918394666181\\
32.83	0.00968931206734972\\
32.84	0.00968944023707264\\
32.85	0.00968956845585204\\
32.86	0.00968969672370939\\
32.87	0.00968982504066617\\
32.88	0.00968995340674387\\
32.89	0.009690081821964\\
32.9	0.00969021028634808\\
32.91	0.00969033879991763\\
32.92	0.0096904673626942\\
32.93	0.00969059597469934\\
32.94	0.00969072463595461\\
32.95	0.00969085334648158\\
32.96	0.00969098210630185\\
32.97	0.009691110915437\\
32.98	0.00969123977390866\\
32.99	0.00969136868173844\\
33	0.00969149763894797\\
33.01	0.0096916266455589\\
33.02	0.00969175570159289\\
33.03	0.0096918848070716\\
33.04	0.00969201396201672\\
33.05	0.00969214316644993\\
33.06	0.00969227242039293\\
33.07	0.00969240172386746\\
33.08	0.00969253107689522\\
33.09	0.00969266047949796\\
33.1	0.00969278993169743\\
33.11	0.00969291943351539\\
33.12	0.00969304898497361\\
33.13	0.00969317858609389\\
33.14	0.009693308236898\\
33.15	0.00969343793740777\\
33.16	0.00969356768764502\\
33.17	0.00969369748763157\\
33.18	0.00969382733738927\\
33.19	0.00969395723693999\\
33.2	0.00969408718630557\\
33.21	0.00969421718550791\\
33.22	0.0096943472345689\\
33.23	0.00969447733351044\\
33.24	0.00969460748235444\\
33.25	0.00969473768112282\\
33.26	0.00969486792983755\\
33.27	0.00969499822852055\\
33.28	0.00969512857719379\\
33.29	0.00969525897587926\\
33.3	0.00969538942459892\\
33.31	0.00969551992337478\\
33.32	0.00969565047222886\\
33.33	0.00969578107118317\\
33.34	0.00969591172025975\\
33.35	0.00969604241948064\\
33.36	0.0096961731688679\\
33.37	0.0096963039684436\\
33.38	0.00969643481822983\\
33.39	0.00969656571824867\\
33.4	0.00969669666852223\\
33.41	0.00969682766907263\\
33.42	0.009696958719922\\
33.43	0.00969708982109247\\
33.44	0.00969722097260622\\
33.45	0.00969735217448539\\
33.46	0.00969748342675217\\
33.47	0.00969761472942874\\
33.48	0.00969774608253731\\
33.49	0.00969787748610009\\
33.5	0.00969800894013931\\
33.51	0.0096981404446772\\
33.52	0.00969827199973602\\
33.53	0.00969840360533803\\
33.54	0.00969853526150549\\
33.55	0.0096986669682607\\
33.56	0.00969879872562596\\
33.57	0.00969893053362357\\
33.58	0.00969906239227586\\
33.59	0.00969919430160516\\
33.6	0.00969932626163382\\
33.61	0.00969945827238419\\
33.62	0.00969959033387866\\
33.63	0.0096997224461396\\
33.64	0.0096998546091894\\
33.65	0.00969998682305048\\
33.66	0.00970011908774525\\
33.67	0.00970025140329615\\
33.68	0.00970038376972562\\
33.69	0.00970051618705612\\
33.7	0.00970064865531011\\
33.71	0.00970078117451008\\
33.72	0.00970091374467851\\
33.73	0.00970104636583792\\
33.74	0.00970117903801082\\
33.75	0.00970131176121974\\
33.76	0.00970144453548723\\
33.77	0.00970157736083583\\
33.78	0.00970171023728812\\
33.79	0.00970184316486667\\
33.8	0.00970197614359407\\
33.81	0.00970210917349293\\
33.82	0.00970224225458586\\
33.83	0.0097023753868955\\
33.84	0.00970250857044448\\
33.85	0.00970264180525545\\
33.86	0.00970277509135108\\
33.87	0.00970290842875405\\
33.88	0.00970304181748705\\
33.89	0.00970317525757277\\
33.9	0.00970330874903394\\
33.91	0.00970344229189328\\
33.92	0.00970357588617354\\
33.93	0.00970370953189746\\
33.94	0.0097038432290878\\
33.95	0.00970397697776735\\
33.96	0.00970411077795888\\
33.97	0.00970424462968522\\
33.98	0.00970437853296917\\
33.99	0.00970451248783355\\
34	0.0097046464943012\\
34.01	0.00970478055239498\\
34.02	0.00970491466213775\\
34.03	0.00970504882355239\\
34.04	0.00970518303666178\\
34.05	0.00970531730148883\\
34.06	0.00970545161805645\\
34.07	0.00970558598638757\\
34.08	0.00970572040650513\\
34.09	0.00970585487843207\\
34.1	0.00970598940219137\\
34.11	0.009706123977806\\
34.12	0.00970625860529895\\
34.13	0.00970639328469322\\
34.14	0.00970652801601183\\
34.15	0.0097066627992778\\
34.16	0.00970679763451417\\
34.17	0.009706932521744\\
34.18	0.00970706746099036\\
34.19	0.0097072024522763\\
34.2	0.00970733749562495\\
34.21	0.00970747259105938\\
34.22	0.00970760773860272\\
34.23	0.0097077429382781\\
34.24	0.00970787819010865\\
34.25	0.00970801349411754\\
34.26	0.00970814885032793\\
34.27	0.00970828425876299\\
34.28	0.00970841971944592\\
34.29	0.00970855523239993\\
34.3	0.00970869079764822\\
34.31	0.00970882641521404\\
34.32	0.00970896208512063\\
34.33	0.00970909780739123\\
34.34	0.00970923358204912\\
34.35	0.00970936940911759\\
34.36	0.00970950528861991\\
34.37	0.00970964122057941\\
34.38	0.0097097772050194\\
34.39	0.00970991324196321\\
34.4	0.00971004933143419\\
34.41	0.00971018547345569\\
34.42	0.0097103216680511\\
34.43	0.00971045791524379\\
34.44	0.00971059421505715\\
34.45	0.0097107305675146\\
34.46	0.00971086697263956\\
34.47	0.00971100343045547\\
34.48	0.00971113994098577\\
34.49	0.00971127650425393\\
34.5	0.00971141312028342\\
34.51	0.00971154978909772\\
34.52	0.00971168651072034\\
34.53	0.00971182328517479\\
34.54	0.00971196011248459\\
34.55	0.00971209699267329\\
34.56	0.00971223392576443\\
34.57	0.00971237091178159\\
34.58	0.00971250795074833\\
34.59	0.00971264504268825\\
34.6	0.00971278218762495\\
34.61	0.00971291938558206\\
34.62	0.00971305663658319\\
34.63	0.009713193940652\\
34.64	0.00971333129781214\\
34.65	0.00971346870808727\\
34.66	0.00971360617150109\\
34.67	0.00971374368807728\\
34.68	0.00971388125783954\\
34.69	0.00971401888081162\\
34.7	0.00971415655701724\\
34.71	0.00971429428648014\\
34.72	0.00971443206922409\\
34.73	0.00971456990527286\\
34.74	0.00971470779465024\\
34.75	0.00971484573738002\\
34.76	0.00971498373348602\\
34.77	0.00971512178299208\\
34.78	0.00971525988592202\\
34.79	0.00971539804229971\\
34.8	0.00971553625214899\\
34.81	0.00971567451549377\\
34.82	0.00971581283235792\\
34.83	0.00971595120276535\\
34.84	0.00971608962673999\\
34.85	0.00971622810430576\\
34.86	0.00971636663548661\\
34.87	0.0097165052203065\\
34.88	0.0097166438587894\\
34.89	0.00971678255095929\\
34.9	0.00971692129684018\\
34.91	0.00971706009645607\\
34.92	0.00971719894983099\\
34.93	0.00971733785698899\\
34.94	0.0097174768179541\\
34.95	0.0097176158327504\\
34.96	0.00971775490140196\\
34.97	0.00971789402393287\\
34.98	0.00971803320036726\\
34.99	0.00971817243072921\\
35	0.00971831171504288\\
35.01	0.00971845105333241\\
35.02	0.00971859044562195\\
35.03	0.00971872989193568\\
35.04	0.00971886939229777\\
35.05	0.00971900894673245\\
35.06	0.0097191485552639\\
35.07	0.00971928821791637\\
35.08	0.00971942793471408\\
35.09	0.0097195677056813\\
35.1	0.00971970753084228\\
35.11	0.00971984741022132\\
35.12	0.00971998734384269\\
35.13	0.00972012733173072\\
35.14	0.0097202673739097\\
35.15	0.009720407470404\\
35.16	0.00972054762123793\\
35.17	0.00972068782643589\\
35.18	0.00972082808602222\\
35.19	0.00972096840002132\\
35.2	0.0097211087684576\\
35.21	0.00972124919135547\\
35.22	0.00972138966873936\\
35.23	0.00972153020063371\\
35.24	0.00972167078706297\\
35.25	0.00972181142805162\\
35.26	0.00972195212362414\\
35.27	0.00972209287380502\\
35.28	0.00972223367861878\\
35.29	0.00972237453808993\\
35.3	0.00972251545224303\\
35.31	0.00972265642110262\\
35.32	0.00972279744469326\\
35.33	0.00972293852303954\\
35.34	0.00972307965616604\\
35.35	0.00972322084409738\\
35.36	0.00972336208685817\\
35.37	0.00972350338447304\\
35.38	0.00972364473696665\\
35.39	0.00972378614436365\\
35.4	0.00972392760668873\\
35.41	0.00972406912396656\\
35.42	0.00972421069622186\\
35.43	0.00972435232347934\\
35.44	0.00972449400576372\\
35.45	0.00972463574309976\\
35.46	0.00972477753551221\\
35.47	0.00972491938302584\\
35.48	0.00972506128566544\\
35.49	0.0097252032434558\\
35.5	0.00972534525642176\\
35.51	0.00972548732458811\\
35.52	0.00972562944797972\\
35.53	0.00972577162662143\\
35.54	0.00972591386053812\\
35.55	0.00972605614975466\\
35.56	0.00972619849429595\\
35.57	0.00972634089418691\\
35.58	0.00972648334945246\\
35.59	0.00972662586011753\\
35.6	0.00972676842620709\\
35.61	0.00972691104774609\\
35.62	0.00972705372475951\\
35.63	0.00972719645727236\\
35.64	0.00972733924530964\\
35.65	0.00972748208889638\\
35.66	0.0097276249880576\\
35.67	0.00972776794281836\\
35.68	0.00972791095320373\\
35.69	0.00972805401923878\\
35.7	0.0097281971409486\\
35.71	0.00972834031835831\\
35.72	0.00972848355149302\\
35.73	0.00972862684037788\\
35.74	0.00972877018503802\\
35.75	0.00972891358549861\\
35.76	0.00972905704178483\\
35.77	0.00972920055392188\\
35.78	0.00972934412193494\\
35.79	0.00972948774584926\\
35.8	0.00972963142569006\\
35.81	0.00972977516148258\\
35.82	0.0097299189532521\\
35.83	0.00973006280102388\\
35.84	0.00973020670482323\\
35.85	0.00973035066467544\\
35.86	0.00973049468060583\\
35.87	0.00973063875263974\\
35.88	0.00973078288080251\\
35.89	0.00973092706511951\\
35.9	0.00973107130561611\\
35.91	0.00973121560231771\\
35.92	0.0097313599552497\\
35.93	0.00973150436443752\\
35.94	0.00973164882990659\\
35.95	0.00973179335168235\\
35.96	0.00973193792979028\\
35.97	0.00973208256425584\\
35.98	0.00973222725510454\\
35.99	0.00973237200236187\\
36	0.00973251680605335\\
36.01	0.00973266166620452\\
36.02	0.00973280658284093\\
36.03	0.00973295155598814\\
36.04	0.00973309658567172\\
36.05	0.00973324167191728\\
36.06	0.00973338681475041\\
36.07	0.00973353201419673\\
36.08	0.00973367727028189\\
36.09	0.00973382258303152\\
36.1	0.00973396795247129\\
36.11	0.00973411337862689\\
36.12	0.00973425886152401\\
36.13	0.00973440440118835\\
36.14	0.00973454999764563\\
36.15	0.0097346956509216\\
36.16	0.009734841361042\\
36.17	0.00973498712803259\\
36.18	0.00973513295191917\\
36.19	0.00973527883272752\\
36.2	0.00973542477048345\\
36.21	0.00973557076521279\\
36.22	0.00973571681694137\\
36.23	0.00973586292569505\\
36.24	0.0097360090914997\\
36.25	0.0097361553143812\\
36.26	0.00973630159436544\\
36.27	0.00973644793147834\\
36.28	0.00973659432574583\\
36.29	0.00973674077719384\\
36.3	0.00973688728584833\\
36.31	0.00973703385173528\\
36.32	0.00973718047488066\\
36.33	0.00973732715531047\\
36.34	0.00973747389305074\\
36.35	0.00973762068812749\\
36.36	0.00973776754056677\\
36.37	0.00973791445039463\\
36.38	0.00973806141763715\\
36.39	0.00973820844232042\\
36.4	0.00973835552447054\\
36.41	0.00973850266411362\\
36.42	0.00973864986127581\\
36.43	0.00973879711598324\\
36.44	0.00973894442826209\\
36.45	0.00973909179813852\\
36.46	0.00973923922563874\\
36.47	0.00973938671078895\\
36.48	0.00973953425361537\\
36.49	0.00973968185414423\\
36.5	0.0097398295124018\\
36.51	0.00973997722841433\\
36.52	0.00974012500220811\\
36.53	0.00974027283380944\\
36.54	0.00974042072324462\\
36.55	0.00974056867053999\\
36.56	0.00974071667572188\\
36.57	0.00974086473881665\\
36.58	0.00974101285985068\\
36.59	0.00974116103885035\\
36.6	0.00974130927584206\\
36.61	0.00974145757085222\\
36.62	0.00974160592390728\\
36.63	0.00974175433503367\\
36.64	0.00974190280425786\\
36.65	0.00974205133160633\\
36.66	0.00974219991710557\\
36.67	0.00974234856078207\\
36.68	0.00974249726266237\\
36.69	0.00974264602277301\\
36.7	0.00974279484114053\\
36.71	0.00974294371779151\\
36.72	0.00974309265275252\\
36.73	0.00974324164605016\\
36.74	0.00974339069771105\\
36.75	0.00974353980776182\\
36.76	0.0097436889762291\\
36.77	0.00974383820313956\\
36.78	0.00974398748851987\\
36.79	0.00974413683239672\\
36.8	0.00974428623479681\\
36.81	0.00974443569574686\\
36.82	0.00974458521527361\\
36.83	0.00974473479340381\\
36.84	0.00974488443016423\\
36.85	0.00974503412558163\\
36.86	0.00974518387968282\\
36.87	0.00974533369249462\\
36.88	0.00974548356404384\\
36.89	0.00974563349435733\\
36.9	0.00974578348346194\\
36.91	0.00974593353138455\\
36.92	0.00974608363815205\\
36.93	0.00974623380379134\\
36.94	0.00974638402832933\\
36.95	0.00974653431179297\\
36.96	0.0097466846542092\\
36.97	0.00974683505560498\\
36.98	0.0097469855160073\\
36.99	0.00974713603544315\\
37	0.00974728661393955\\
37.01	0.00974743725152351\\
37.02	0.00974758794822209\\
37.03	0.00974773870406234\\
37.04	0.00974788951907133\\
37.05	0.00974804039327615\\
37.06	0.00974819132670391\\
37.07	0.00974834231938172\\
37.08	0.00974849337133673\\
37.09	0.00974864448259607\\
37.1	0.00974879565318693\\
37.11	0.00974894688313647\\
37.12	0.0097490981724719\\
37.13	0.00974924952122043\\
37.14	0.0097494009294093\\
37.15	0.00974955239706573\\
37.16	0.009749703924217\\
37.17	0.00974985551089038\\
37.18	0.00975000715711316\\
37.19	0.00975015886291265\\
37.2	0.00975031062831617\\
37.21	0.00975046245335106\\
37.22	0.00975061433804467\\
37.23	0.00975076628242437\\
37.24	0.00975091828651755\\
37.25	0.00975107035035161\\
37.26	0.00975122247395397\\
37.27	0.00975137465735206\\
37.28	0.00975152690057332\\
37.29	0.00975167920364523\\
37.3	0.00975183156659526\\
37.31	0.00975198398945091\\
37.32	0.0097521364722397\\
37.33	0.00975228901498913\\
37.34	0.00975244161772677\\
37.35	0.00975259428048017\\
37.36	0.00975274700327691\\
37.37	0.00975289978614457\\
37.38	0.00975305262911076\\
37.39	0.00975320553220312\\
37.4	0.00975335849544926\\
37.41	0.00975351151887686\\
37.42	0.00975366460251357\\
37.43	0.0097538177463871\\
37.44	0.00975397095052513\\
37.45	0.00975412421495539\\
37.46	0.00975427753970561\\
37.47	0.00975443092480355\\
37.48	0.00975458437027696\\
37.49	0.00975473787615364\\
37.5	0.00975489144246137\\
37.51	0.00975504506922798\\
37.52	0.00975519875648129\\
37.53	0.00975535250424915\\
37.54	0.00975550631255943\\
37.55	0.00975566018144\\
37.56	0.00975581411091876\\
37.57	0.00975596810102361\\
37.58	0.00975612215178249\\
37.59	0.00975627626322334\\
37.6	0.00975643043537412\\
37.61	0.00975658466826279\\
37.62	0.00975673896191737\\
37.63	0.00975689331636584\\
37.64	0.00975704773163624\\
37.65	0.00975720220775661\\
37.66	0.009757356744755\\
37.67	0.00975751134265949\\
37.68	0.00975766600149815\\
37.69	0.00975782072129911\\
37.7	0.00975797550209048\\
37.71	0.00975813034390039\\
37.72	0.009758285246757\\
37.73	0.00975844021068849\\
37.74	0.00975859523572303\\
37.75	0.00975875032188884\\
37.76	0.00975890546921412\\
37.77	0.00975906067772712\\
37.78	0.00975921594745609\\
37.79	0.00975937127842928\\
37.8	0.009759526670675\\
37.81	0.00975968212422154\\
37.82	0.00975983763909721\\
37.83	0.00975999321533036\\
37.84	0.00976014885294932\\
37.85	0.00976030455198247\\
37.86	0.0097604603124582\\
37.87	0.00976061613440488\\
37.88	0.00976077201785095\\
37.89	0.00976092796282484\\
37.9	0.00976108396935499\\
37.91	0.00976124003746987\\
37.92	0.00976139616719797\\
37.93	0.00976155235856777\\
37.94	0.0097617086116078\\
37.95	0.00976186492634658\\
37.96	0.00976202130281267\\
37.97	0.00976217774103462\\
37.98	0.00976233424104103\\
37.99	0.00976249080286047\\
38	0.00976264742652158\\
38.01	0.00976280411205298\\
38.02	0.00976296085948332\\
38.03	0.00976311766884125\\
38.04	0.00976327454015547\\
38.05	0.00976343147345467\\
38.06	0.00976358846876757\\
38.07	0.00976374552612289\\
38.08	0.00976390264554937\\
38.09	0.00976405982707579\\
38.1	0.00976421707073092\\
38.11	0.00976437437654358\\
38.12	0.00976453174454255\\
38.13	0.00976468917475668\\
38.14	0.00976484666721482\\
38.15	0.00976500422194582\\
38.16	0.00976516183897858\\
38.17	0.00976531951834198\\
38.18	0.00976547726006495\\
38.19	0.0097656350641764\\
38.2	0.0097657929307053\\
38.21	0.00976595085968061\\
38.22	0.0097661088511313\\
38.23	0.00976626690508638\\
38.24	0.00976642502157486\\
38.25	0.00976658320062578\\
38.26	0.00976674144226818\\
38.27	0.00976689974653114\\
38.28	0.00976705811344373\\
38.29	0.00976721654303505\\
38.3	0.00976737503533422\\
38.31	0.00976753359037037\\
38.32	0.00976769220817266\\
38.33	0.00976785088877026\\
38.34	0.00976800963219233\\
38.35	0.00976816843846811\\
38.36	0.00976832730762678\\
38.37	0.0097684862396976\\
38.38	0.00976864523470982\\
38.39	0.0097688042926927\\
38.4	0.00976896341367554\\
38.41	0.00976912259768763\\
38.42	0.0097692818447583\\
38.43	0.00976944115491688\\
38.44	0.00976960052819274\\
38.45	0.00976975996461524\\
38.46	0.00976991946421376\\
38.47	0.00977007902701773\\
38.48	0.00977023865305655\\
38.49	0.00977039834235968\\
38.5	0.00977055809495657\\
38.51	0.00977071791087669\\
38.52	0.00977087779014954\\
38.53	0.00977103773280462\\
38.54	0.00977119773887146\\
38.55	0.00977135780837961\\
38.56	0.00977151794135862\\
38.57	0.00977167813783808\\
38.58	0.00977183839784757\\
38.59	0.00977199872141671\\
38.6	0.00977215910857514\\
38.61	0.00977231955935249\\
38.62	0.00977248007377843\\
38.63	0.00977264065188264\\
38.64	0.00977280129369482\\
38.65	0.00977296199924469\\
38.66	0.00977312276856199\\
38.67	0.00977328360167645\\
38.68	0.00977344449861785\\
38.69	0.00977360545941597\\
38.7	0.00977376648410062\\
38.71	0.00977392757270162\\
38.72	0.00977408872524879\\
38.73	0.00977424994177201\\
38.74	0.00977441122230114\\
38.75	0.00977457256686606\\
38.76	0.00977473397549669\\
38.77	0.00977489544822295\\
38.78	0.00977505698507479\\
38.79	0.00977521858608215\\
38.8	0.00977538025127501\\
38.81	0.00977554198068339\\
38.82	0.00977570377433727\\
38.83	0.00977586563226669\\
38.84	0.0097760275545017\\
38.85	0.00977618954107236\\
38.86	0.00977635159200875\\
38.87	0.00977651370734098\\
38.88	0.00977667588709915\\
38.89	0.00977683813131341\\
38.9	0.00977700044001389\\
38.91	0.00977716281323078\\
38.92	0.00977732525099425\\
38.93	0.00977748775333452\\
38.94	0.0097776503202818\\
38.95	0.00977781295186633\\
38.96	0.00977797564811837\\
38.97	0.00977813840906819\\
38.98	0.0097783012347461\\
38.99	0.00977846412518238\\
39	0.00977862708040738\\
39.01	0.00977879010045143\\
39.02	0.0097789531853449\\
39.03	0.00977911633511818\\
39.04	0.00977927954980165\\
39.05	0.00977944282942573\\
39.06	0.00977960617402085\\
39.07	0.00977976958361747\\
39.08	0.00977993305824606\\
39.09	0.0097800965979371\\
39.1	0.00978026020272109\\
39.11	0.00978042387262856\\
39.12	0.00978058760769005\\
39.13	0.0097807514079361\\
39.14	0.0097809152733973\\
39.15	0.00978107920410425\\
39.16	0.00978124320008754\\
39.17	0.00978140726137781\\
39.18	0.0097815713880057\\
39.19	0.00978173558000188\\
39.2	0.00978189983739703\\
39.21	0.00978206416022184\\
39.22	0.00978222854850704\\
39.23	0.00978239300228336\\
39.24	0.00978255752158155\\
39.25	0.00978272210643238\\
39.26	0.00978288675686664\\
39.27	0.00978305147291514\\
39.28	0.0097832162546087\\
39.29	0.00978338110197815\\
39.3	0.00978354601505438\\
39.31	0.00978371099386823\\
39.32	0.00978387603845063\\
39.33	0.00978404114883247\\
39.34	0.00978420632504469\\
39.35	0.00978437156711824\\
39.36	0.00978453687508407\\
39.37	0.0097847022489732\\
39.38	0.0097848676888166\\
39.39	0.00978503319464531\\
39.4	0.00978519876649035\\
39.41	0.0097853644043828\\
39.42	0.00978553010835371\\
39.43	0.0097856958784342\\
39.44	0.00978586171465535\\
39.45	0.00978602761704832\\
39.46	0.00978619358564423\\
39.47	0.00978635962047425\\
39.48	0.00978652572156958\\
39.49	0.0097866918889614\\
39.5	0.00978685812268094\\
39.51	0.00978702442275943\\
39.52	0.00978719078922812\\
39.53	0.0097873572221183\\
39.54	0.00978752372146124\\
39.55	0.00978769028728827\\
39.56	0.00978785691963069\\
39.57	0.00978802361851987\\
39.58	0.00978819038398716\\
39.59	0.00978835721606394\\
39.6	0.0097885241147816\\
39.61	0.00978869108017159\\
39.62	0.00978885811226531\\
39.63	0.00978902521109423\\
39.64	0.00978919237668981\\
39.65	0.00978935960908355\\
39.66	0.00978952690830696\\
39.67	0.00978969427439155\\
39.68	0.00978986170736888\\
39.69	0.00979002920727051\\
39.7	0.009790196774128\\
39.71	0.00979036440797297\\
39.72	0.00979053210883703\\
39.73	0.00979069987675181\\
39.74	0.00979086771174896\\
39.75	0.00979103561386016\\
39.76	0.0097912035831171\\
39.77	0.00979137161955148\\
39.78	0.00979153972319502\\
39.79	0.00979170789407948\\
39.8	0.00979187613223661\\
39.81	0.00979204443769818\\
39.82	0.00979221281049601\\
39.83	0.00979238125066191\\
39.84	0.0097925497582277\\
39.85	0.00979271833322525\\
39.86	0.00979288697568643\\
39.87	0.00979305568564311\\
39.88	0.00979322446312722\\
39.89	0.00979339330817068\\
39.9	0.00979356222080543\\
39.91	0.00979373120106343\\
39.92	0.00979390024897666\\
39.93	0.00979406936457713\\
39.94	0.00979423854789685\\
39.95	0.00979440779896786\\
39.96	0.00979457711782221\\
39.97	0.00979474650449196\\
39.98	0.00979491595900923\\
39.99	0.0097950854814061\\
40	0.00979525507171471\\
40.01	0.00979542472996721\\
};
\addplot [color=mycolor1,solid,forget plot]
  table[row sep=crcr]{%
40.01	0.00979542472996721\\
40.02	0.00979559445619576\\
40.03	0.00979576425043254\\
40.04	0.00979593411270975\\
40.05	0.00979610404305962\\
40.06	0.00979627404151438\\
40.07	0.00979644410810628\\
40.08	0.00979661424286761\\
40.09	0.00979678444583065\\
40.1	0.00979695471702772\\
40.11	0.00979712505649114\\
40.12	0.00979729546425327\\
40.13	0.00979746594034647\\
40.14	0.00979763648480314\\
40.15	0.00979780709765566\\
40.16	0.00979797777893647\\
40.17	0.00979814852867801\\
40.18	0.00979831934691273\\
40.19	0.00979849023367313\\
40.2	0.00979866118899168\\
40.21	0.00979883221290092\\
40.22	0.00979900330543337\\
40.23	0.00979917446662158\\
40.24	0.00979934569649814\\
40.25	0.00979951699509563\\
40.26	0.00979968836244666\\
40.27	0.00979985979858385\\
40.28	0.00980003130353986\\
40.29	0.00980020287734735\\
40.3	0.009800374520039\\
40.31	0.00980054623164753\\
40.32	0.00980071801220564\\
40.33	0.00980088986174609\\
40.34	0.00980106178030163\\
40.35	0.00980123376790504\\
40.36	0.00980140582458912\\
40.37	0.00980157795038669\\
40.38	0.00980175014533058\\
40.39	0.00980192240945364\\
40.4	0.00980209474278876\\
40.41	0.00980226714536882\\
40.42	0.00980243961722674\\
40.43	0.00980261215839544\\
40.44	0.00980278476890789\\
40.45	0.00980295744879704\\
40.46	0.0098031301980959\\
40.47	0.00980330301683746\\
40.48	0.00980347590505475\\
40.49	0.00980364886278083\\
40.5	0.00980382189004875\\
40.51	0.00980399498689161\\
40.52	0.0098041681533425\\
40.53	0.00980434138943456\\
40.54	0.00980451469520094\\
40.55	0.00980468807067478\\
40.56	0.00980486151588928\\
40.57	0.00980503503087764\\
40.58	0.00980520861567309\\
40.59	0.00980538227030887\\
40.6	0.00980555599481824\\
40.61	0.00980572978923448\\
40.62	0.0098059036535909\\
40.63	0.00980607758792082\\
40.64	0.00980625159225759\\
40.65	0.00980642566663457\\
40.66	0.00980659981108513\\
40.67	0.00980677402564269\\
40.68	0.00980694831034067\\
40.69	0.00980712266521251\\
40.7	0.00980729709029169\\
40.71	0.00980747158561168\\
40.72	0.00980764615120599\\
40.73	0.00980782078710815\\
40.74	0.0098079954933517\\
40.75	0.00980817026997023\\
40.76	0.00980834511699731\\
40.77	0.00980852003446656\\
40.78	0.00980869502241161\\
40.79	0.0098088700808661\\
40.8	0.00980904520986373\\
40.81	0.00980922040943818\\
40.82	0.00980939567962317\\
40.83	0.00980957102045245\\
40.84	0.00980974643195976\\
40.85	0.0098099219141789\\
40.86	0.00981009746714367\\
40.87	0.00981027309088789\\
40.88	0.00981044878544542\\
40.89	0.00981062455085012\\
40.9	0.0098108003871359\\
40.91	0.00981097629433667\\
40.92	0.00981115227248636\\
40.93	0.00981132832161894\\
40.94	0.00981150444176839\\
40.95	0.00981168063296872\\
40.96	0.00981185689525397\\
40.97	0.00981203322865818\\
40.98	0.00981220963321544\\
40.99	0.00981238610895984\\
41	0.00981256265592552\\
41.01	0.00981273927414662\\
41.02	0.00981291596365731\\
41.03	0.0098130927244918\\
41.04	0.00981326955668431\\
41.05	0.00981344646026909\\
41.06	0.0098136234352804\\
41.07	0.00981380048175255\\
41.08	0.00981397759971986\\
41.09	0.00981415478921668\\
41.1	0.00981433205027738\\
41.11	0.00981450938293636\\
41.12	0.00981468678722805\\
41.13	0.00981486426318691\\
41.14	0.0098150418108474\\
41.15	0.00981521943024404\\
41.16	0.00981539712141136\\
41.17	0.00981557488438392\\
41.18	0.00981575271919631\\
41.19	0.00981593062588315\\
41.2	0.00981610860447907\\
41.21	0.00981628665501876\\
41.22	0.00981646477753691\\
41.23	0.00981664297206825\\
41.24	0.00981682123864755\\
41.25	0.00981699957730958\\
41.26	0.00981717798808917\\
41.27	0.00981735647102118\\
41.28	0.00981753502614047\\
41.29	0.00981771365348195\\
41.3	0.00981789235308057\\
41.31	0.0098180711249713\\
41.32	0.00981824996918914\\
41.33	0.00981842888576913\\
41.34	0.00981860787474634\\
41.35	0.00981878693615587\\
41.36	0.00981896607003286\\
41.37	0.00981914527641246\\
41.38	0.00981932455532989\\
41.39	0.00981950390682038\\
41.4	0.0098196833309192\\
41.41	0.00981986282766166\\
41.42	0.0098200423970831\\
41.43	0.0098202220392189\\
41.44	0.00982040175410446\\
41.45	0.00982058154177526\\
41.46	0.00982076140226677\\
41.47	0.00982094133561452\\
41.48	0.00982112134185406\\
41.49	0.00982130142102103\\
41.5	0.00982148157315104\\
41.51	0.0098216617982798\\
41.52	0.00982184209644301\\
41.53	0.00982202246767644\\
41.54	0.0098222029120159\\
41.55	0.00982238342949724\\
41.56	0.00982256402015634\\
41.57	0.00982274468402915\\
41.58	0.00982292542115163\\
41.59	0.00982310623155981\\
41.6	0.00982328711528976\\
41.61	0.00982346807237757\\
41.62	0.00982364910285942\\
41.63	0.00982383020677151\\
41.64	0.00982401138415008\\
41.65	0.00982419263503143\\
41.66	0.00982437395945192\\
41.67	0.00982455535744793\\
41.68	0.00982473682905592\\
41.69	0.00982491837431238\\
41.7	0.00982509999325386\\
41.71	0.00982528168591696\\
41.72	0.00982546345233834\\
41.73	0.00982564529255469\\
41.74	0.00982582720660278\\
41.75	0.00982600919451942\\
41.76	0.00982619125634149\\
41.77	0.0098263733921059\\
41.78	0.00982655560184965\\
41.79	0.00982673788560978\\
41.8	0.00982692024342338\\
41.81	0.00982710267532762\\
41.82	0.00982728518135971\\
41.83	0.00982746776155694\\
41.84	0.00982765041595664\\
41.85	0.00982783314459622\\
41.86	0.00982801594751316\\
41.87	0.00982819882474499\\
41.88	0.00982838177632929\\
41.89	0.00982856480230374\\
41.9	0.00982874790270607\\
41.91	0.00982893107757407\\
41.92	0.00982911432694562\\
41.93	0.00982929765085864\\
41.94	0.00982948104935116\\
41.95	0.00982966452246123\\
41.96	0.00982984807022702\\
41.97	0.00983003169268674\\
41.98	0.0098302153898787\\
41.99	0.00983039916184126\\
42	0.00983058300861288\\
42.01	0.00983076693023207\\
42.02	0.00983095092673743\\
42.03	0.00983113499816766\\
42.04	0.00983131914456149\\
42.05	0.00983150336595778\\
42.06	0.00983168766239543\\
42.07	0.00983187203391346\\
42.08	0.00983205648055095\\
42.09	0.00983224100234705\\
42.1	0.00983242559934103\\
42.11	0.00983261027157222\\
42.12	0.00983279501908004\\
42.13	0.009832979841904\\
42.14	0.0098331647400837\\
42.15	0.00983334971365882\\
42.16	0.00983353476266914\\
42.17	0.00983371988715452\\
42.18	0.00983390508715492\\
42.19	0.00983409036271039\\
42.2	0.00983427571386106\\
42.21	0.00983446114064716\\
42.22	0.00983464664310903\\
42.23	0.00983483222128707\\
42.24	0.00983501787522181\\
42.25	0.00983520360495385\\
42.26	0.0098353894105239\\
42.27	0.00983557529197276\\
42.28	0.00983576124934134\\
42.29	0.00983594728267062\\
42.3	0.0098361333920017\\
42.31	0.00983631957737578\\
42.32	0.00983650583883414\\
42.33	0.00983669217641817\\
42.34	0.00983687859016937\\
42.35	0.00983706508012932\\
42.36	0.0098372516463397\\
42.37	0.00983743828884231\\
42.38	0.00983762500767903\\
42.39	0.00983781180289185\\
42.4	0.00983799867452285\\
42.41	0.00983818562261422\\
42.42	0.00983837264720823\\
42.43	0.00983855974834728\\
42.44	0.00983874692607384\\
42.45	0.00983893418043049\\
42.46	0.00983912151145991\\
42.47	0.00983930891920486\\
42.48	0.00983949640370823\\
42.49	0.00983968396501296\\
42.5	0.00983987160316212\\
42.51	0.00984005931819886\\
42.52	0.00984024711016643\\
42.53	0.00984043497910816\\
42.54	0.00984062292506748\\
42.55	0.00984081094808789\\
42.56	0.00984099904821301\\
42.57	0.00984118722548652\\
42.58	0.00984137547995219\\
42.59	0.00984156381165387\\
42.6	0.00984175222063549\\
42.61	0.00984194070694107\\
42.62	0.0098421292706147\\
42.63	0.00984231791170053\\
42.64	0.00984250663024279\\
42.65	0.00984269542628579\\
42.66	0.00984288429987389\\
42.67	0.00984307325105153\\
42.68	0.00984326227986318\\
42.69	0.00984345138635339\\
42.7	0.00984364057056676\\
42.71	0.00984382983254794\\
42.72	0.00984401917234163\\
42.73	0.00984420858999256\\
42.74	0.00984439808554551\\
42.75	0.00984458765904528\\
42.76	0.00984477731053672\\
42.77	0.00984496704006469\\
42.78	0.00984515684767408\\
42.79	0.00984534673340979\\
42.8	0.00984553669731674\\
42.81	0.00984572673943985\\
42.82	0.00984591685982405\\
42.83	0.00984610705851425\\
42.84	0.00984629733555537\\
42.85	0.00984648769099229\\
42.86	0.0098466781248699\\
42.87	0.00984686863723305\\
42.88	0.00984705922812655\\
42.89	0.00984724989759518\\
42.9	0.00984744064568367\\
42.91	0.00984763147243671\\
42.92	0.00984782237789892\\
42.93	0.00984801336211486\\
42.94	0.00984820442512902\\
42.95	0.00984839556698582\\
42.96	0.00984858678772959\\
42.97	0.00984877808740456\\
42.98	0.00984896946605489\\
42.99	0.00984916092372461\\
43	0.00984935246045765\\
43.01	0.00984954407629783\\
43.02	0.00984973577128883\\
43.03	0.00984992754547424\\
43.04	0.00985011939889745\\
43.05	0.00985031133160178\\
43.06	0.00985050334363034\\
43.07	0.00985069543502615\\
43.08	0.00985088760583201\\
43.09	0.00985107985609062\\
43.1	0.00985127218584445\\
43.11	0.00985146459513586\\
43.12	0.009851657084007\\
43.13	0.00985184965249985\\
43.14	0.00985204230065622\\
43.15	0.00985223502851774\\
43.16	0.00985242783612585\\
43.17	0.00985262072352182\\
43.18	0.00985281369074673\\
43.19	0.0098530067378415\\
43.2	0.00985319986484687\\
43.21	0.00985339307180341\\
43.22	0.00985358635875151\\
43.23	0.00985377972573144\\
43.24	0.0098539731727833\\
43.25	0.00985416669994707\\
43.26	0.0098543603072626\\
43.27	0.00985455399476962\\
43.28	0.00985474776250789\\
43.29	0.0098549416105172\\
43.3	0.00985513553883736\\
43.31	0.0098553295475082\\
43.32	0.00985552363656958\\
43.33	0.00985571780606138\\
43.34	0.00985591205602353\\
43.35	0.00985610638649594\\
43.36	0.00985630079751859\\
43.37	0.00985649528913147\\
43.38	0.00985668986137459\\
43.39	0.00985688451428799\\
43.4	0.00985707924791172\\
43.41	0.0098572740622859\\
43.42	0.00985746895745062\\
43.43	0.00985766393344603\\
43.44	0.00985785899031232\\
43.45	0.00985805412808965\\
43.46	0.00985824934681827\\
43.47	0.0098584446465384\\
43.48	0.00985864002729033\\
43.49	0.00985883548911435\\
43.5	0.00985903103205078\\
43.51	0.00985922665613999\\
43.52	0.00985942236142233\\
43.53	0.00985961814793821\\
43.54	0.00985981401572806\\
43.55	0.00986000996483232\\
43.56	0.00986020599529149\\
43.57	0.00986040210714607\\
43.58	0.00986059830043658\\
43.59	0.00986079457520358\\
43.6	0.00986099093148767\\
43.61	0.00986118736932943\\
43.62	0.00986138388876952\\
43.63	0.00986158048984859\\
43.64	0.00986177717260733\\
43.65	0.00986197393708646\\
43.66	0.0098621707833267\\
43.67	0.00986236771136883\\
43.68	0.00986256472125365\\
43.69	0.00986276181302196\\
43.7	0.00986295898671461\\
43.71	0.00986315624237248\\
43.72	0.00986335358003645\\
43.73	0.00986355099974746\\
43.74	0.00986374850154645\\
43.75	0.00986394608547439\\
43.76	0.00986414375157229\\
43.77	0.00986434149988116\\
43.78	0.00986453933044208\\
43.79	0.00986473724329612\\
43.8	0.00986493523848438\\
43.81	0.00986513331604799\\
43.82	0.00986533147602812\\
43.83	0.00986552971846594\\
43.84	0.00986572804340268\\
43.85	0.00986592645087956\\
43.86	0.00986612494093785\\
43.87	0.00986632351361885\\
43.88	0.00986652216896386\\
43.89	0.00986672090701423\\
43.9	0.00986691972781133\\
43.91	0.00986711863139655\\
43.92	0.00986731761781132\\
43.93	0.00986751668709708\\
43.94	0.0098677158392953\\
43.95	0.0098679150744475\\
43.96	0.00986811439259518\\
43.97	0.00986831379377991\\
43.98	0.00986851327804327\\
43.99	0.00986871284542686\\
44	0.00986891249597231\\
44.01	0.00986911222972129\\
44.02	0.00986931204671547\\
44.03	0.00986951194699657\\
44.04	0.00986971193060632\\
44.05	0.00986991199758648\\
44.06	0.00987011214797886\\
44.07	0.00987031238182526\\
44.08	0.00987051269916752\\
44.09	0.00987071310004751\\
44.1	0.00987091358450714\\
44.11	0.00987111415258831\\
44.12	0.00987131480433298\\
44.13	0.00987151553978313\\
44.14	0.00987171635898074\\
44.15	0.00987191726196785\\
44.16	0.00987211824878651\\
44.17	0.00987231931947879\\
44.18	0.00987252047408681\\
44.19	0.00987272171265269\\
44.2	0.00987292303521859\\
44.21	0.00987312444182669\\
44.22	0.00987332593251921\\
44.23	0.00987352750733837\\
44.24	0.00987372916632645\\
44.25	0.00987393090952572\\
44.26	0.0098741327369785\\
44.27	0.00987433464872713\\
44.28	0.00987453664481398\\
44.29	0.00987473872528143\\
44.3	0.00987494089017191\\
44.31	0.00987514313952787\\
44.32	0.00987534547339176\\
44.33	0.00987554789180608\\
44.34	0.00987575039481336\\
44.35	0.00987595298245614\\
44.36	0.009876155654777\\
44.37	0.00987635841181854\\
44.38	0.00987656125362337\\
44.39	0.00987676418023416\\
44.4	0.00987696719169356\\
44.41	0.0098771702880443\\
44.42	0.00987737346932909\\
44.43	0.00987757673559069\\
44.44	0.00987778008687188\\
44.45	0.00987798352321545\\
44.46	0.00987818704466424\\
44.47	0.00987839065126111\\
44.48	0.00987859434304893\\
44.49	0.00987879812007061\\
44.5	0.00987900198236908\\
44.51	0.00987920592998729\\
44.52	0.00987940996296823\\
44.53	0.0098796140813549\\
44.54	0.00987981828519033\\
44.55	0.00988002257451759\\
44.56	0.00988022694937974\\
44.57	0.0098804314098199\\
44.58	0.00988063595588119\\
44.59	0.00988084058760678\\
44.6	0.00988104530503984\\
44.61	0.00988125010822357\\
44.62	0.00988145499720121\\
44.63	0.00988165997201601\\
44.64	0.00988186503271124\\
44.65	0.00988207017933021\\
44.66	0.00988227541191625\\
44.67	0.00988248073051271\\
44.68	0.00988268613516295\\
44.69	0.00988289162591039\\
44.7	0.00988309720279843\\
44.71	0.00988330286587054\\
44.72	0.00988350861517017\\
44.73	0.00988371445074082\\
44.74	0.00988392037262601\\
44.75	0.00988412638086928\\
44.76	0.00988433247551419\\
44.77	0.00988453865660434\\
44.78	0.00988474492418331\\
44.79	0.00988495127829476\\
44.8	0.00988515771898234\\
44.81	0.00988536424628973\\
44.82	0.00988557086026062\\
44.83	0.00988577756093874\\
44.84	0.00988598434836783\\
44.85	0.00988619122259167\\
44.86	0.00988639818365404\\
44.87	0.00988660523159876\\
44.88	0.00988681236646965\\
44.89	0.00988701958831057\\
44.9	0.00988722689716541\\
44.91	0.00988743429307805\\
44.92	0.00988764177609242\\
44.93	0.00988784934625245\\
44.94	0.00988805700360212\\
44.95	0.00988826474818539\\
44.96	0.00988847258004626\\
44.97	0.00988868049922877\\
44.98	0.00988888850577695\\
44.99	0.00988909659973487\\
45	0.00988930478114661\\
45.01	0.00988951305005627\\
45.02	0.00988972140650796\\
45.03	0.00988992985054583\\
45.04	0.00989013838221403\\
45.05	0.00989034700155675\\
45.06	0.00989055570861818\\
45.07	0.00989076450344253\\
45.08	0.00989097338607403\\
45.09	0.00989118235655693\\
45.1	0.00989139141493549\\
45.11	0.00989160056125401\\
45.12	0.00989180979555678\\
45.13	0.00989201911788812\\
45.14	0.00989222852829236\\
45.15	0.00989243802681385\\
45.16	0.00989264761349695\\
45.17	0.00989285728838605\\
45.18	0.00989306705152553\\
45.19	0.00989327690295982\\
45.2	0.00989348684273334\\
45.21	0.00989369687089052\\
45.22	0.00989390698747581\\
45.23	0.00989411719253369\\
45.24	0.00989432748610864\\
45.25	0.00989453786824513\\
45.26	0.00989474833898769\\
45.27	0.00989495889838082\\
45.28	0.00989516954646905\\
45.29	0.00989538028329693\\
45.3	0.009895591108909\\
45.31	0.00989580202334982\\
45.32	0.00989601302666396\\
45.33	0.00989622411889601\\
45.34	0.00989643530009056\\
45.35	0.00989664657029218\\
45.36	0.00989685792954551\\
45.37	0.00989706937789514\\
45.38	0.0098972809153857\\
45.39	0.00989749254206181\\
45.4	0.00989770425796811\\
45.41	0.00989791606314924\\
45.42	0.00989812795764984\\
45.43	0.00989833994151455\\
45.44	0.00989855201478804\\
45.45	0.00989876417751494\\
45.46	0.00989897642973993\\
45.47	0.00989918877150766\\
45.48	0.00989940120286279\\
45.49	0.00989961372384999\\
45.5	0.00989982633451391\\
45.51	0.00990003903489922\\
45.52	0.00990025182505058\\
45.53	0.00990046470501264\\
45.54	0.00990067767483007\\
45.55	0.00990089073454751\\
45.56	0.00990110388420962\\
45.57	0.00990131712386102\\
45.58	0.00990153045354638\\
45.59	0.0099017438733103\\
45.6	0.00990195738319742\\
45.61	0.00990217098325236\\
45.62	0.00990238467351971\\
45.63	0.00990259845404407\\
45.64	0.00990281232487003\\
45.65	0.00990302628604217\\
45.66	0.00990324033760503\\
45.67	0.00990345447960317\\
45.68	0.00990366871208112\\
45.69	0.00990388303508339\\
45.7	0.00990409744865448\\
45.71	0.00990431195283887\\
45.72	0.00990452654768102\\
45.73	0.00990474123322538\\
45.74	0.00990495600951636\\
45.75	0.00990517087659836\\
45.76	0.00990538583451575\\
45.77	0.00990560088331288\\
45.78	0.00990581602303407\\
45.79	0.00990603125372361\\
45.8	0.00990624657542577\\
45.81	0.00990646198818478\\
45.82	0.00990667749204484\\
45.83	0.00990689308705013\\
45.84	0.00990710877324477\\
45.85	0.00990732455067286\\
45.86	0.00990754041937847\\
45.87	0.0099077563794056\\
45.88	0.00990797243079826\\
45.89	0.00990818857360035\\
45.9	0.00990840480785579\\
45.91	0.00990862113360842\\
45.92	0.00990883755090203\\
45.93	0.00990905405978037\\
45.94	0.00990927066028714\\
45.95	0.00990948735246598\\
45.96	0.00990970413636048\\
45.97	0.00990992101201417\\
45.98	0.00991013797947051\\
45.99	0.00991035503877293\\
46	0.00991057218996476\\
46.01	0.00991078943308928\\
46.02	0.00991100676818971\\
46.03	0.00991122419530918\\
46.04	0.00991144171449076\\
46.05	0.00991165932577743\\
46.06	0.00991187702921212\\
46.07	0.00991209482483766\\
46.08	0.0099123127126968\\
46.09	0.00991253069283219\\
46.1	0.00991274876528643\\
46.11	0.00991296693010198\\
46.12	0.00991318518732125\\
46.13	0.00991340353698652\\
46.14	0.00991362197913999\\
46.15	0.00991384051382375\\
46.16	0.00991405914107978\\
46.17	0.00991427786094996\\
46.18	0.00991449667347605\\
46.19	0.00991471557869971\\
46.2	0.00991493457666244\\
46.21	0.00991515366740567\\
46.22	0.00991537285097067\\
46.23	0.00991559212739859\\
46.24	0.00991581149673044\\
46.25	0.00991603095900711\\
46.26	0.00991625051426932\\
46.27	0.00991647016255768\\
46.28	0.00991668990391262\\
46.29	0.00991690973837443\\
46.3	0.00991712966598325\\
46.31	0.00991734968677904\\
46.32	0.0099175698008016\\
46.33	0.00991779000809057\\
46.34	0.00991801030868539\\
46.35	0.00991823070262533\\
46.36	0.0099184511899495\\
46.37	0.00991867177069677\\
46.38	0.00991889244490585\\
46.39	0.00991911321261524\\
46.4	0.00991933407386323\\
46.41	0.00991955502868789\\
46.42	0.00991977607712707\\
46.43	0.00991999721921843\\
46.44	0.00992021845499935\\
46.45	0.00992043978450701\\
46.46	0.00992066120777833\\
46.47	0.00992088272484998\\
46.48	0.00992110433575838\\
46.49	0.00992132604053969\\
46.5	0.00992154783922979\\
46.51	0.00992176973186429\\
46.52	0.00992199171847852\\
46.53	0.0099222137991075\\
46.54	0.00992243597378597\\
46.55	0.00992265824254835\\
46.56	0.00992288060542875\\
46.57	0.00992310306246096\\
46.58	0.00992332561367842\\
46.59	0.00992354825911426\\
46.6	0.00992377099880123\\
46.61	0.00992399383277174\\
46.62	0.00992421676105782\\
46.63	0.00992443978369112\\
46.64	0.00992466290070293\\
46.65	0.0099248861121241\\
46.66	0.0099251094179851\\
46.67	0.00992533281831598\\
46.68	0.00992555631314635\\
46.69	0.00992577990250539\\
46.7	0.00992600358642181\\
46.71	0.00992622736492388\\
46.72	0.00992645123803938\\
46.73	0.0099266752057956\\
46.74	0.00992689926821935\\
46.75	0.00992712342533689\\
46.76	0.00992734767717399\\
46.77	0.00992757202375587\\
46.78	0.00992779646510717\\
46.79	0.00992802100125201\\
46.8	0.00992824563221388\\
46.81	0.00992847035801571\\
46.82	0.00992869517867979\\
46.83	0.00992892009422779\\
46.84	0.00992914510468074\\
46.85	0.00992937021005902\\
46.86	0.0099295954103823\\
46.87	0.00992982070566958\\
46.88	0.00993004609593913\\
46.89	0.00993027158120853\\
46.9	0.00993049716149455\\
46.91	0.00993072283681322\\
46.92	0.0099309486071798\\
46.93	0.00993117447260871\\
46.94	0.00993140043311355\\
46.95	0.00993162648870707\\
46.96	0.00993185263940117\\
46.97	0.00993207888520681\\
46.98	0.00993230522613407\\
46.99	0.00993253166219207\\
47	0.00993275819338897\\
47.01	0.00993298481973196\\
47.02	0.00993321154122717\\
47.03	0.00993343835787973\\
47.04	0.00993366526969369\\
47.05	0.009933892276672\\
47.06	0.0099341193788165\\
47.07	0.00993434657612786\\
47.08	0.00993457386860558\\
47.09	0.00993480125624796\\
47.1	0.00993502873905203\\
47.11	0.00993525631701359\\
47.12	0.00993548399012708\\
47.13	0.00993571175838565\\
47.14	0.00993593962178104\\
47.15	0.00993616758030361\\
47.16	0.00993639563394225\\
47.17	0.0099366237826844\\
47.18	0.00993685202651593\\
47.19	0.00993708036542121\\
47.2	0.00993730879938297\\
47.21	0.00993753732838232\\
47.22	0.00993776595239867\\
47.23	0.00993799467140974\\
47.24	0.00993822348539144\\
47.25	0.0099384523943179\\
47.26	0.00993868139816137\\
47.27	0.00993891049689218\\
47.28	0.00993913969047875\\
47.29	0.00993936897888743\\
47.3	0.00993959836208256\\
47.31	0.00993982784002633\\
47.32	0.0099400574126788\\
47.33	0.00994028707999777\\
47.34	0.0099405168419388\\
47.35	0.00994074669845507\\
47.36	0.00994097664949739\\
47.37	0.00994120669501409\\
47.38	0.009941436834951\\
47.39	0.00994166706925133\\
47.4	0.00994189739785565\\
47.41	0.0099421278207018\\
47.42	0.00994235833772484\\
47.43	0.00994258894885692\\
47.44	0.0099428196540273\\
47.45	0.00994305045316218\\
47.46	0.00994328134618469\\
47.47	0.00994351233301475\\
47.48	0.00994374341356905\\
47.49	0.00994397458776093\\
47.5	0.00994420585550028\\
47.51	0.00994443721669348\\
47.52	0.0099446686712433\\
47.53	0.00994490021904881\\
47.54	0.00994513186000525\\
47.55	0.00994536359400397\\
47.56	0.00994559542093232\\
47.57	0.00994582734067354\\
47.58	0.00994605935310664\\
47.59	0.00994629145810633\\
47.6	0.00994652365554284\\
47.61	0.00994675594528187\\
47.62	0.00994698832718443\\
47.63	0.00994722080110676\\
47.64	0.00994745336690012\\
47.65	0.00994768602441075\\
47.66	0.00994791877347972\\
47.67	0.00994815161394273\\
47.68	0.00994838454563005\\
47.69	0.00994861756836633\\
47.7	0.00994885068197046\\
47.71	0.00994908388625543\\
47.72	0.00994931718102814\\
47.73	0.00994955056608932\\
47.74	0.00994978404123325\\
47.75	0.00995001760624768\\
47.76	0.00995025126091364\\
47.77	0.00995048500500522\\
47.78	0.00995071883828944\\
47.79	0.00995095276052601\\
47.8	0.00995118677146721\\
47.81	0.0099514208708576\\
47.82	0.00995165505843387\\
47.83	0.00995188933392465\\
47.84	0.00995212369705022\\
47.85	0.00995235814752238\\
47.86	0.00995259268504414\\
47.87	0.00995282730930955\\
47.88	0.0099530620200034\\
47.89	0.00995329681680104\\
47.9	0.00995353169936807\\
47.91	0.0099537666673601\\
47.92	0.00995400172042249\\
47.93	0.00995423685819008\\
47.94	0.00995447208028685\\
47.95	0.00995470738632572\\
47.96	0.00995494277590818\\
47.97	0.00995517824862401\\
47.98	0.00995541380405097\\
47.99	0.00995564944175445\\
48	0.00995588516128718\\
48.01	0.00995612096218882\\
48.02	0.00995635684398567\\
48.03	0.0099565928061903\\
48.04	0.00995682884830113\\
48.05	0.00995706496980211\\
48.06	0.00995730117016227\\
48.07	0.00995753744883536\\
48.08	0.00995777380525941\\
48.09	0.0099580102388563\\
48.1	0.00995824674903134\\
48.11	0.00995848333517279\\
48.12	0.00995871999665144\\
48.13	0.00995895673282007\\
48.14	0.00995919354301302\\
48.15	0.00995943042654563\\
48.16	0.00995966738271378\\
48.17	0.0099599044107933\\
48.18	0.00996014151003945\\
48.19	0.00996037867968635\\
48.2	0.00996061591894641\\
48.21	0.0099608532270097\\
48.22	0.00996109060304338\\
48.23	0.00996132804619104\\
48.24	0.00996156555557204\\
48.25	0.00996180313028088\\
48.26	0.00996204076938649\\
48.27	0.00996227847193151\\
48.28	0.0099625162369316\\
48.29	0.00996275406337465\\
48.3	0.00996299195022005\\
48.31	0.00996322989639786\\
48.32	0.00996346790080804\\
48.33	0.00996370596231957\\
48.34	0.00996394407976964\\
48.35	0.0099641822519627\\
48.36	0.00996442047766961\\
48.37	0.00996465875562666\\
48.38	0.00996489708453463\\
48.39	0.00996513546305779\\
48.4	0.00996537388982287\\
48.41	0.00996561236341805\\
48.42	0.00996585088239181\\
48.43	0.0099660894452519\\
48.44	0.00996632805046411\\
48.45	0.00996656669645118\\
48.46	0.00996680538159153\\
48.47	0.00996704410421803\\
48.48	0.00996728286261673\\
48.49	0.00996752165502551\\
48.5	0.00996776047963277\\
48.51	0.009967999334576\\
48.52	0.00996823821794035\\
48.53	0.00996847712775712\\
48.54	0.00996871606200231\\
48.55	0.009968955018595\\
48.56	0.00996919399539571\\
48.57	0.00996943299020484\\
48.58	0.00996967200076083\\
48.59	0.00996991102473854\\
48.6	0.0099701500597473\\
48.61	0.00997038910332918\\
48.62	0.00997062815295696\\
48.63	0.00997086720603223\\
48.64	0.00997110625988333\\
48.65	0.00997134531176326\\
48.66	0.00997158435884751\\
48.67	0.00997182339823188\\
48.68	0.00997206242693018\\
48.69	0.00997230144187185\\
48.7	0.0099725404398996\\
48.71	0.00997277941776687\\
48.72	0.00997301837213528\\
48.73	0.00997325729957199\\
48.74	0.00997349619654701\\
48.75	0.00997373505943035\\
48.76	0.00997397388448916\\
48.77	0.00997421266788478\\
48.78	0.00997445140566965\\
48.79	0.0099746900937842\\
48.8	0.00997492872805357\\
48.81	0.00997516730418431\\
48.82	0.00997540581776091\\
48.83	0.00997564426424229\\
48.84	0.00997588263895812\\
48.85	0.00997612093710513\\
48.86	0.00997635915374314\\
48.87	0.00997659728379118\\
48.88	0.00997683532202335\\
48.89	0.00997707326306455\\
48.9	0.00997731110138622\\
48.91	0.00997754883130177\\
48.92	0.00997778644696203\\
48.93	0.00997802394235046\\
48.94	0.00997826131127824\\
48.95	0.00997849854737927\\
48.96	0.00997873564410492\\
48.97	0.00997897259471871\\
48.98	0.0099792093922908\\
48.99	0.00997944602969226\\
49	0.00997968249958927\\
49.01	0.00997991879443703\\
49.02	0.00998015490647362\\
49.03	0.00998039082771352\\
49.04	0.00998062654994107\\
49.05	0.00998086206470364\\
49.06	0.00998109736330463\\
49.07	0.0099813324367963\\
49.08	0.00998156727597223\\
49.09	0.00998180187135978\\
49.1	0.0099820362132121\\
49.11	0.00998227029150005\\
49.12	0.00998250409590381\\
49.13	0.00998273761580418\\
49.14	0.00998297084027377\\
49.15	0.00998320375806776\\
49.16	0.00998343635761445\\
49.17	0.00998366862700555\\
49.18	0.00998390055398607\\
49.19	0.00998413212594406\\
49.2	0.00998436332989985\\
49.21	0.0099845941524951\\
49.22	0.00998482457998147\\
49.23	0.0099850545982089\\
49.24	0.00998528419261362\\
49.25	0.00998551334820569\\
49.26	0.00998574204955623\\
49.27	0.00998597028078423\\
49.28	0.00998619802554294\\
49.29	0.00998642526700585\\
49.3	0.0099866519878523\\
49.31	0.0099868781702525\\
49.32	0.00998710379585223\\
49.33	0.00998732884575699\\
49.34	0.00998755330051571\\
49.35	0.00998777714010388\\
49.36	0.00998800034390623\\
49.37	0.00998822289069883\\
49.38	0.00998844475863068\\
49.39	0.00998866592520466\\
49.4	0.00998888636725796\\
49.41	0.00998910606094183\\
49.42	0.00998932498170078\\
49.43	0.00998954310425105\\
49.44	0.00998976040255848\\
49.45	0.00998997684981565\\
49.46	0.0099901924184183\\
49.47	0.00999040707994106\\
49.48	0.00999062080511235\\
49.49	0.0099908335637886\\
49.5	0.00999104532492754\\
49.51	0.00999125605656075\\
49.52	0.0099914657257653\\
49.53	0.00999167429863454\\
49.54	0.00999188174024795\\
49.55	0.00999208801464001\\
49.56	0.00999229308476818\\
49.57	0.00999249691247979\\
49.58	0.00999269945847795\\
49.59	0.00999290068228636\\
49.6	0.00999310054221297\\
49.61	0.00999329899531263\\
49.62	0.0099934959973484\\
49.63	0.00999369150275175\\
49.64	0.00999388546458147\\
49.65	0.00999407783448128\\
49.66	0.00999426856263614\\
49.67	0.00999445759772709\\
49.68	0.00999464488688478\\
49.69	0.00999483037564144\\
49.7	0.00999501400788138\\
49.71	0.00999519572578993\\
49.72	0.00999537546980067\\
49.73	0.00999555317854114\\
49.74	0.0099957287887767\\
49.75	0.00999590223535269\\
49.76	0.0099960734511347\\
49.77	0.00999624236694697\\
49.78	0.00999640891150885\\
49.79	0.00999657301136921\\
49.8	0.00999673459083878\\
49.81	0.00999689357192033\\
49.82	0.00999704987423663\\
49.83	0.00999720341495613\\
49.84	0.00999735410871627\\
49.85	0.00999750186754432\\
49.86	0.00999764660077572\\
49.87	0.00999778821496981\\
49.88	0.00999792661382287\\
49.89	0.00999806169807839\\
49.9	0.00999819336543446\\
49.91	0.0099983215104482\\
49.92	0.00999844602443718\\
49.93	0.0099985667953776\\
49.94	0.00999868370779928\\
49.95	0.00999879664267728\\
49.96	0.00999890547732003\\
49.97	0.00999901008525389\\
49.98	0.009999110336104\\
49.99	0.00999920609547131\\
50	0.00999929722480567\\
50.01	0.00999938358127485\\
50.02	0.00999946501762936\\
50.03	0.00999954138206288\\
50.04	0.00999961251806832\\
50.05	0.00999967826428909\\
50.06	0.00999973845436575\\
50.07	0.00999979291677763\\
50.08	0.00999984147467936\\
50.09	0.00999988394573218\\
50.1	0.0099999201419298\\
50.11	0.00999994986941864\\
50.12	0.00999997292831228\\
50.13	0.00999998911249995\\
50.14	0.00999999820944884\\
50.15	0.01\\
50.16	0.01\\
50.17	0.01\\
50.18	0.01\\
50.19	0.01\\
50.2	0.01\\
50.21	0.01\\
50.22	0.01\\
50.23	0.01\\
50.24	0.01\\
50.25	0.01\\
50.26	0.01\\
50.27	0.01\\
50.28	0.01\\
50.29	0.01\\
50.3	0.01\\
50.31	0.01\\
50.32	0.01\\
50.33	0.01\\
50.34	0.01\\
50.35	0.01\\
50.36	0.01\\
50.37	0.01\\
50.38	0.01\\
50.39	0.01\\
50.4	0.01\\
50.41	0.01\\
50.42	0.01\\
50.43	0.01\\
50.44	0.01\\
50.45	0.01\\
50.46	0.01\\
50.47	0.01\\
50.48	0.01\\
50.49	0.01\\
50.5	0.01\\
50.51	0.01\\
50.52	0.01\\
50.53	0.01\\
50.54	0.01\\
50.55	0.01\\
50.56	0.01\\
50.57	0.01\\
50.58	0.01\\
50.59	0.01\\
50.6	0.01\\
50.61	0.01\\
50.62	0.01\\
50.63	0.01\\
50.64	0.01\\
50.65	0.01\\
50.66	0.01\\
50.67	0.01\\
50.68	0.01\\
50.69	0.01\\
50.7	0.01\\
50.71	0.01\\
50.72	0.01\\
50.73	0.01\\
50.74	0.01\\
50.75	0.01\\
50.76	0.01\\
50.77	0.01\\
50.78	0.01\\
50.79	0.01\\
50.8	0.01\\
50.81	0.01\\
50.82	0.01\\
50.83	0.01\\
50.84	0.01\\
50.85	0.01\\
50.86	0.01\\
50.87	0.01\\
50.88	0.01\\
50.89	0.01\\
50.9	0.01\\
50.91	0.01\\
50.92	0.01\\
50.93	0.01\\
50.94	0.01\\
50.95	0.01\\
50.96	0.01\\
50.97	0.01\\
50.98	0.01\\
50.99	0.01\\
51	0.01\\
51.01	0.01\\
51.02	0.01\\
51.03	0.01\\
51.04	0.01\\
51.05	0.01\\
51.06	0.01\\
51.07	0.01\\
51.08	0.01\\
51.09	0.01\\
51.1	0.01\\
51.11	0.01\\
51.12	0.01\\
51.13	0.01\\
51.14	0.01\\
51.15	0.01\\
51.16	0.01\\
51.17	0.01\\
51.18	0.01\\
51.19	0.01\\
51.2	0.01\\
51.21	0.01\\
51.22	0.01\\
51.23	0.01\\
51.24	0.01\\
51.25	0.01\\
51.26	0.01\\
51.27	0.01\\
51.28	0.01\\
51.29	0.01\\
51.3	0.01\\
51.31	0.01\\
51.32	0.01\\
51.33	0.01\\
51.34	0.01\\
51.35	0.01\\
51.36	0.01\\
51.37	0.01\\
51.38	0.01\\
51.39	0.01\\
51.4	0.01\\
51.41	0.01\\
51.42	0.01\\
51.43	0.01\\
51.44	0.01\\
51.45	0.01\\
51.46	0.01\\
51.47	0.01\\
51.48	0.01\\
51.49	0.01\\
51.5	0.01\\
51.51	0.01\\
51.52	0.01\\
51.53	0.01\\
51.54	0.01\\
51.55	0.01\\
51.56	0.01\\
51.57	0.01\\
51.58	0.01\\
51.59	0.01\\
51.6	0.01\\
51.61	0.01\\
51.62	0.01\\
51.63	0.01\\
51.64	0.01\\
51.65	0.01\\
51.66	0.01\\
51.67	0.01\\
51.68	0.01\\
51.69	0.01\\
51.7	0.01\\
51.71	0.01\\
51.72	0.01\\
51.73	0.01\\
51.74	0.01\\
51.75	0.01\\
51.76	0.01\\
51.77	0.01\\
51.78	0.01\\
51.79	0.01\\
51.8	0.01\\
51.81	0.01\\
51.82	0.01\\
51.83	0.01\\
51.84	0.01\\
51.85	0.01\\
51.86	0.01\\
51.87	0.01\\
51.88	0.01\\
51.89	0.01\\
51.9	0.01\\
51.91	0.01\\
51.92	0.01\\
51.93	0.01\\
51.94	0.01\\
51.95	0.01\\
51.96	0.01\\
51.97	0.01\\
51.98	0.01\\
51.99	0.01\\
52	0.01\\
52.01	0.01\\
52.02	0.01\\
52.03	0.01\\
52.04	0.01\\
52.05	0.01\\
52.06	0.01\\
52.07	0.01\\
52.08	0.01\\
52.09	0.01\\
52.1	0.01\\
52.11	0.01\\
52.12	0.01\\
52.13	0.01\\
52.14	0.01\\
52.15	0.01\\
52.16	0.01\\
52.17	0.01\\
52.18	0.01\\
52.19	0.01\\
52.2	0.01\\
52.21	0.01\\
52.22	0.01\\
52.23	0.01\\
52.24	0.01\\
52.25	0.01\\
52.26	0.01\\
52.27	0.01\\
52.28	0.01\\
52.29	0.01\\
52.3	0.01\\
52.31	0.01\\
52.32	0.01\\
52.33	0.01\\
52.34	0.01\\
52.35	0.01\\
52.36	0.01\\
52.37	0.01\\
52.38	0.01\\
52.39	0.01\\
52.4	0.01\\
52.41	0.01\\
52.42	0.01\\
52.43	0.01\\
52.44	0.01\\
52.45	0.01\\
52.46	0.01\\
52.47	0.01\\
52.48	0.01\\
52.49	0.01\\
52.5	0.01\\
52.51	0.01\\
52.52	0.01\\
52.53	0.01\\
52.54	0.01\\
52.55	0.01\\
52.56	0.01\\
52.57	0.01\\
52.58	0.01\\
52.59	0.01\\
52.6	0.01\\
52.61	0.01\\
52.62	0.01\\
52.63	0.01\\
52.64	0.01\\
52.65	0.01\\
52.66	0.01\\
52.67	0.01\\
52.68	0.01\\
52.69	0.01\\
52.7	0.01\\
52.71	0.01\\
52.72	0.01\\
52.73	0.01\\
52.74	0.01\\
52.75	0.01\\
52.76	0.01\\
52.77	0.01\\
52.78	0.01\\
52.79	0.01\\
52.8	0.01\\
52.81	0.01\\
52.82	0.01\\
52.83	0.01\\
52.84	0.01\\
52.85	0.01\\
52.86	0.01\\
52.87	0.01\\
52.88	0.01\\
52.89	0.01\\
52.9	0.01\\
52.91	0.01\\
52.92	0.01\\
52.93	0.01\\
52.94	0.01\\
52.95	0.01\\
52.96	0.01\\
52.97	0.01\\
52.98	0.01\\
52.99	0.01\\
53	0.01\\
53.01	0.01\\
53.02	0.01\\
53.03	0.01\\
53.04	0.01\\
53.05	0.01\\
53.06	0.01\\
53.07	0.01\\
53.08	0.01\\
53.09	0.01\\
53.1	0.01\\
53.11	0.01\\
53.12	0.01\\
53.13	0.01\\
53.14	0.01\\
53.15	0.01\\
53.16	0.01\\
53.17	0.01\\
53.18	0.01\\
53.19	0.01\\
53.2	0.01\\
53.21	0.01\\
53.22	0.01\\
53.23	0.01\\
53.24	0.01\\
53.25	0.01\\
53.26	0.01\\
53.27	0.01\\
53.28	0.01\\
53.29	0.01\\
53.3	0.01\\
53.31	0.01\\
53.32	0.01\\
53.33	0.01\\
53.34	0.01\\
53.35	0.01\\
53.36	0.01\\
53.37	0.01\\
53.38	0.01\\
53.39	0.01\\
53.4	0.01\\
53.41	0.01\\
53.42	0.01\\
53.43	0.01\\
53.44	0.01\\
53.45	0.01\\
53.46	0.01\\
53.47	0.01\\
53.48	0.01\\
53.49	0.01\\
53.5	0.01\\
53.51	0.01\\
53.52	0.01\\
53.53	0.01\\
53.54	0.01\\
53.55	0.01\\
53.56	0.01\\
53.57	0.01\\
53.58	0.01\\
53.59	0.01\\
53.6	0.01\\
53.61	0.01\\
53.62	0.01\\
53.63	0.01\\
53.64	0.01\\
53.65	0.01\\
53.66	0.01\\
53.67	0.01\\
53.68	0.01\\
53.69	0.01\\
53.7	0.01\\
53.71	0.01\\
53.72	0.01\\
53.73	0.01\\
53.74	0.01\\
53.75	0.01\\
53.76	0.01\\
53.77	0.01\\
53.78	0.01\\
53.79	0.01\\
53.8	0.01\\
53.81	0.01\\
53.82	0.01\\
53.83	0.01\\
53.84	0.01\\
53.85	0.01\\
53.86	0.01\\
53.87	0.01\\
53.88	0.01\\
53.89	0.01\\
53.9	0.01\\
53.91	0.01\\
53.92	0.01\\
53.93	0.01\\
53.94	0.01\\
53.95	0.01\\
53.96	0.01\\
53.97	0.01\\
53.98	0.01\\
53.99	0.01\\
54	0.01\\
54.01	0.01\\
54.02	0.01\\
54.03	0.01\\
54.04	0.01\\
54.05	0.01\\
54.06	0.01\\
54.07	0.01\\
54.08	0.01\\
54.09	0.01\\
54.1	0.01\\
54.11	0.01\\
54.12	0.01\\
54.13	0.01\\
54.14	0.01\\
54.15	0.01\\
54.16	0.01\\
54.17	0.01\\
54.18	0.01\\
54.19	0.01\\
54.2	0.01\\
54.21	0.01\\
54.22	0.01\\
54.23	0.01\\
54.24	0.01\\
54.25	0.01\\
54.26	0.01\\
54.27	0.01\\
54.28	0.01\\
54.29	0.01\\
54.3	0.01\\
54.31	0.01\\
54.32	0.01\\
54.33	0.01\\
54.34	0.01\\
54.35	0.01\\
54.36	0.01\\
54.37	0.01\\
54.38	0.01\\
54.39	0.01\\
54.4	0.01\\
54.41	0.01\\
54.42	0.01\\
54.43	0.01\\
54.44	0.01\\
54.45	0.01\\
54.46	0.01\\
54.47	0.01\\
54.48	0.01\\
54.49	0.01\\
54.5	0.01\\
54.51	0.01\\
54.52	0.01\\
54.53	0.01\\
54.54	0.01\\
54.55	0.01\\
54.56	0.01\\
54.57	0.01\\
54.58	0.01\\
54.59	0.01\\
54.6	0.01\\
54.61	0.01\\
54.62	0.01\\
54.63	0.01\\
54.64	0.01\\
54.65	0.01\\
54.66	0.01\\
54.67	0.01\\
54.68	0.01\\
54.69	0.01\\
54.7	0.01\\
54.71	0.01\\
54.72	0.01\\
54.73	0.01\\
54.74	0.01\\
54.75	0.01\\
54.76	0.01\\
54.77	0.01\\
54.78	0.01\\
54.79	0.01\\
54.8	0.01\\
54.81	0.01\\
54.82	0.01\\
54.83	0.01\\
54.84	0.01\\
54.85	0.01\\
54.86	0.01\\
54.87	0.01\\
54.88	0.01\\
54.89	0.01\\
54.9	0.01\\
54.91	0.01\\
54.92	0.01\\
54.93	0.01\\
54.94	0.01\\
54.95	0.01\\
54.96	0.01\\
54.97	0.01\\
54.98	0.01\\
54.99	0.01\\
55	0.01\\
55.01	0.01\\
55.02	0.01\\
55.03	0.01\\
55.04	0.01\\
55.05	0.01\\
55.06	0.01\\
55.07	0.01\\
55.08	0.01\\
55.09	0.01\\
55.1	0.01\\
55.11	0.01\\
55.12	0.01\\
55.13	0.01\\
55.14	0.01\\
55.15	0.01\\
55.16	0.01\\
55.17	0.01\\
55.18	0.01\\
55.19	0.01\\
55.2	0.01\\
55.21	0.01\\
55.22	0.01\\
55.23	0.01\\
55.24	0.01\\
55.25	0.01\\
55.26	0.01\\
55.27	0.01\\
55.28	0.01\\
55.29	0.01\\
55.3	0.01\\
55.31	0.01\\
55.32	0.01\\
55.33	0.01\\
55.34	0.01\\
55.35	0.01\\
55.36	0.01\\
55.37	0.01\\
55.38	0.01\\
55.39	0.01\\
55.4	0.01\\
55.41	0.01\\
55.42	0.01\\
55.43	0.01\\
55.44	0.01\\
55.45	0.01\\
55.46	0.01\\
55.47	0.01\\
55.48	0.01\\
55.49	0.01\\
55.5	0.01\\
55.51	0.01\\
55.52	0.01\\
55.53	0.01\\
55.54	0.01\\
55.55	0.01\\
55.56	0.01\\
55.57	0.01\\
55.58	0.01\\
55.59	0.01\\
55.6	0.01\\
55.61	0.01\\
55.62	0.01\\
55.63	0.01\\
55.64	0.01\\
55.65	0.01\\
55.66	0.01\\
55.67	0.01\\
55.68	0.01\\
55.69	0.01\\
55.7	0.01\\
55.71	0.01\\
55.72	0.01\\
55.73	0.01\\
55.74	0.01\\
55.75	0.01\\
55.76	0.01\\
55.77	0.01\\
55.78	0.01\\
55.79	0.01\\
55.8	0.01\\
55.81	0.01\\
55.82	0.01\\
55.83	0.01\\
55.84	0.01\\
55.85	0.01\\
55.86	0.01\\
55.87	0.01\\
55.88	0.01\\
55.89	0.01\\
55.9	0.01\\
55.91	0.01\\
55.92	0.01\\
55.93	0.01\\
55.94	0.01\\
55.95	0.01\\
55.96	0.01\\
55.97	0.01\\
55.98	0.01\\
55.99	0.01\\
56	0.01\\
56.01	0.01\\
56.02	0.01\\
56.03	0.01\\
56.04	0.01\\
56.05	0.01\\
56.06	0.01\\
56.07	0.01\\
56.08	0.01\\
56.09	0.01\\
56.1	0.01\\
56.11	0.01\\
56.12	0.01\\
56.13	0.01\\
56.14	0.01\\
56.15	0.01\\
56.16	0.01\\
56.17	0.01\\
56.18	0.01\\
56.19	0.01\\
56.2	0.01\\
56.21	0.01\\
56.22	0.01\\
56.23	0.01\\
56.24	0.01\\
56.25	0.01\\
56.26	0.01\\
56.27	0.01\\
56.28	0.01\\
56.29	0.01\\
56.3	0.01\\
56.31	0.01\\
56.32	0.01\\
56.33	0.01\\
56.34	0.01\\
56.35	0.01\\
56.36	0.01\\
56.37	0.01\\
56.38	0.01\\
56.39	0.01\\
56.4	0.01\\
56.41	0.01\\
56.42	0.01\\
56.43	0.01\\
56.44	0.01\\
56.45	0.01\\
56.46	0.01\\
56.47	0.01\\
56.48	0.01\\
56.49	0.01\\
56.5	0.01\\
56.51	0.01\\
56.52	0.01\\
56.53	0.01\\
56.54	0.01\\
56.55	0.01\\
56.56	0.01\\
56.57	0.01\\
56.58	0.01\\
56.59	0.01\\
56.6	0.01\\
56.61	0.01\\
56.62	0.01\\
56.63	0.01\\
56.64	0.01\\
56.65	0.01\\
56.66	0.01\\
56.67	0.01\\
56.68	0.01\\
56.69	0.01\\
56.7	0.01\\
56.71	0.01\\
56.72	0.01\\
56.73	0.01\\
56.74	0.01\\
56.75	0.01\\
56.76	0.01\\
56.77	0.01\\
56.78	0.01\\
56.79	0.01\\
56.8	0.01\\
56.81	0.01\\
56.82	0.01\\
56.83	0.01\\
56.84	0.01\\
56.85	0.01\\
56.86	0.01\\
56.87	0.01\\
56.88	0.01\\
56.89	0.01\\
56.9	0.01\\
56.91	0.01\\
56.92	0.01\\
56.93	0.01\\
56.94	0.01\\
56.95	0.01\\
56.96	0.01\\
56.97	0.01\\
56.98	0.01\\
56.99	0.01\\
57	0.01\\
57.01	0.01\\
57.02	0.01\\
57.03	0.01\\
57.04	0.01\\
57.05	0.01\\
57.06	0.01\\
57.07	0.01\\
57.08	0.01\\
57.09	0.01\\
57.1	0.01\\
57.11	0.01\\
57.12	0.01\\
57.13	0.01\\
57.14	0.01\\
57.15	0.01\\
57.16	0.01\\
57.17	0.01\\
57.18	0.01\\
57.19	0.01\\
57.2	0.01\\
57.21	0.01\\
57.22	0.01\\
57.23	0.01\\
57.24	0.01\\
57.25	0.01\\
57.26	0.01\\
57.27	0.01\\
57.28	0.01\\
57.29	0.01\\
57.3	0.01\\
57.31	0.01\\
57.32	0.01\\
57.33	0.01\\
57.34	0.01\\
57.35	0.01\\
57.36	0.01\\
57.37	0.01\\
57.38	0.01\\
57.39	0.01\\
57.4	0.01\\
57.41	0.01\\
57.42	0.01\\
57.43	0.01\\
57.44	0.01\\
57.45	0.01\\
57.46	0.01\\
57.47	0.01\\
57.48	0.01\\
57.49	0.01\\
57.5	0.01\\
57.51	0.01\\
57.52	0.01\\
57.53	0.01\\
57.54	0.01\\
57.55	0.01\\
57.56	0.01\\
57.57	0.01\\
57.58	0.01\\
57.59	0.01\\
57.6	0.01\\
57.61	0.01\\
57.62	0.01\\
57.63	0.01\\
57.64	0.01\\
57.65	0.01\\
57.66	0.01\\
57.67	0.01\\
57.68	0.01\\
57.69	0.01\\
57.7	0.01\\
57.71	0.01\\
57.72	0.01\\
57.73	0.01\\
57.74	0.01\\
57.75	0.01\\
57.76	0.01\\
57.77	0.01\\
57.78	0.01\\
57.79	0.01\\
57.8	0.01\\
57.81	0.01\\
57.82	0.01\\
57.83	0.01\\
57.84	0.01\\
57.85	0.01\\
57.86	0.01\\
57.87	0.01\\
57.88	0.01\\
57.89	0.01\\
57.9	0.01\\
57.91	0.01\\
57.92	0.01\\
57.93	0.01\\
57.94	0.01\\
57.95	0.01\\
57.96	0.01\\
57.97	0.01\\
57.98	0.01\\
57.99	0.01\\
58	0.01\\
58.01	0.01\\
58.02	0.01\\
58.03	0.01\\
58.04	0.01\\
58.05	0.01\\
58.06	0.01\\
58.07	0.01\\
58.08	0.01\\
58.09	0.01\\
58.1	0.01\\
58.11	0.01\\
58.12	0.01\\
58.13	0.01\\
58.14	0.01\\
58.15	0.01\\
58.16	0.01\\
58.17	0.01\\
58.18	0.01\\
58.19	0.01\\
58.2	0.01\\
58.21	0.01\\
58.22	0.01\\
58.23	0.01\\
58.24	0.01\\
58.25	0.01\\
58.26	0.01\\
58.27	0.01\\
58.28	0.01\\
58.29	0.01\\
58.3	0.01\\
58.31	0.01\\
58.32	0.01\\
58.33	0.01\\
58.34	0.01\\
58.35	0.01\\
58.36	0.01\\
58.37	0.01\\
58.38	0.01\\
58.39	0.01\\
58.4	0.01\\
58.41	0.01\\
58.42	0.01\\
58.43	0.01\\
58.44	0.01\\
58.45	0.01\\
58.46	0.01\\
58.47	0.01\\
58.48	0.01\\
58.49	0.01\\
58.5	0.01\\
58.51	0.01\\
58.52	0.01\\
58.53	0.01\\
58.54	0.01\\
58.55	0.01\\
58.56	0.01\\
58.57	0.01\\
58.58	0.01\\
58.59	0.01\\
58.6	0.01\\
58.61	0.01\\
58.62	0.01\\
58.63	0.01\\
58.64	0.01\\
58.65	0.01\\
58.66	0.01\\
58.67	0.01\\
58.68	0.01\\
58.69	0.01\\
58.7	0.01\\
58.71	0.01\\
58.72	0.01\\
58.73	0.01\\
58.74	0.01\\
58.75	0.01\\
58.76	0.01\\
58.77	0.01\\
58.78	0.01\\
58.79	0.01\\
58.8	0.01\\
58.81	0.01\\
58.82	0.01\\
58.83	0.01\\
58.84	0.01\\
58.85	0.01\\
58.86	0.01\\
58.87	0.01\\
58.88	0.01\\
58.89	0.01\\
58.9	0.01\\
58.91	0.01\\
58.92	0.01\\
58.93	0.01\\
58.94	0.01\\
58.95	0.01\\
58.96	0.01\\
58.97	0.01\\
58.98	0.01\\
58.99	0.01\\
59	0.01\\
59.01	0.01\\
59.02	0.01\\
59.03	0.01\\
59.04	0.01\\
59.05	0.01\\
59.06	0.01\\
59.07	0.01\\
59.08	0.01\\
59.09	0.01\\
59.1	0.01\\
59.11	0.01\\
59.12	0.01\\
59.13	0.01\\
59.14	0.01\\
59.15	0.01\\
59.16	0.01\\
59.17	0.01\\
59.18	0.01\\
59.19	0.01\\
59.2	0.01\\
59.21	0.01\\
59.22	0.01\\
59.23	0.01\\
59.24	0.01\\
59.25	0.01\\
59.26	0.01\\
59.27	0.01\\
59.28	0.01\\
59.29	0.01\\
59.3	0.01\\
59.31	0.01\\
59.32	0.01\\
59.33	0.01\\
59.34	0.01\\
59.35	0.01\\
59.36	0.01\\
59.37	0.01\\
59.38	0.01\\
59.39	0.01\\
59.4	0.01\\
59.41	0.01\\
59.42	0.01\\
59.43	0.01\\
59.44	0.01\\
59.45	0.01\\
59.46	0.01\\
59.47	0.01\\
59.48	0.01\\
59.49	0.01\\
59.5	0.01\\
59.51	0.01\\
59.52	0.01\\
59.53	0.01\\
59.54	0.01\\
59.55	0.01\\
59.56	0.01\\
59.57	0.01\\
59.58	0.01\\
59.59	0.01\\
59.6	0.01\\
59.61	0.01\\
59.62	0.01\\
59.63	0.01\\
59.64	0.01\\
59.65	0.01\\
59.66	0.01\\
59.67	0.01\\
59.68	0.01\\
59.69	0.01\\
59.7	0.01\\
59.71	0.01\\
59.72	0.01\\
59.73	0.01\\
59.74	0.01\\
59.75	0.01\\
59.76	0.01\\
59.77	0.01\\
59.78	0.01\\
59.79	0.01\\
59.8	0.01\\
59.81	0.01\\
59.82	0.01\\
59.83	0.01\\
59.84	0.01\\
59.85	0.01\\
59.86	0.01\\
59.87	0.01\\
59.88	0.01\\
59.89	0.01\\
59.9	0.01\\
59.91	0.01\\
59.92	0.01\\
59.93	0.01\\
59.94	0.01\\
59.95	0.01\\
59.96	0.01\\
59.97	0.01\\
59.98	0.01\\
59.99	0.01\\
60	0.01\\
60.01	0.01\\
60.02	0.01\\
60.03	0.01\\
60.04	0.01\\
60.05	0.01\\
60.06	0.01\\
60.07	0.01\\
60.08	0.01\\
60.09	0.01\\
60.1	0.01\\
60.11	0.01\\
60.12	0.01\\
60.13	0.01\\
60.14	0.01\\
60.15	0.01\\
60.16	0.01\\
60.17	0.01\\
60.18	0.01\\
60.19	0.01\\
60.2	0.01\\
60.21	0.01\\
60.22	0.01\\
60.23	0.01\\
60.24	0.01\\
60.25	0.01\\
60.26	0.01\\
60.27	0.01\\
60.28	0.01\\
60.29	0.01\\
60.3	0.01\\
60.31	0.01\\
60.32	0.01\\
60.33	0.01\\
60.34	0.01\\
60.35	0.01\\
60.36	0.01\\
60.37	0.01\\
60.38	0.01\\
60.39	0.01\\
60.4	0.01\\
60.41	0.01\\
60.42	0.01\\
60.43	0.01\\
60.44	0.01\\
60.45	0.01\\
60.46	0.01\\
60.47	0.01\\
60.48	0.01\\
60.49	0.01\\
60.5	0.01\\
60.51	0.01\\
60.52	0.01\\
60.53	0.01\\
60.54	0.01\\
60.55	0.01\\
60.56	0.01\\
60.57	0.01\\
60.58	0.01\\
60.59	0.01\\
60.6	0.01\\
60.61	0.01\\
60.62	0.01\\
60.63	0.01\\
60.64	0.01\\
60.65	0.01\\
60.66	0.01\\
60.67	0.01\\
60.68	0.01\\
60.69	0.01\\
60.7	0.01\\
60.71	0.01\\
60.72	0.01\\
60.73	0.01\\
60.74	0.01\\
60.75	0.01\\
60.76	0.01\\
60.77	0.01\\
60.78	0.01\\
60.79	0.01\\
60.8	0.01\\
60.81	0.01\\
60.82	0.01\\
60.83	0.01\\
60.84	0.01\\
60.85	0.01\\
60.86	0.01\\
60.87	0.01\\
60.88	0.01\\
60.89	0.01\\
60.9	0.01\\
60.91	0.01\\
60.92	0.01\\
60.93	0.01\\
60.94	0.01\\
60.95	0.01\\
60.96	0.01\\
60.97	0.01\\
60.98	0.01\\
60.99	0.01\\
61	0.01\\
61.01	0.01\\
61.02	0.01\\
61.03	0.01\\
61.04	0.01\\
61.05	0.01\\
61.06	0.01\\
61.07	0.01\\
61.08	0.01\\
61.09	0.01\\
61.1	0.01\\
61.11	0.01\\
61.12	0.01\\
61.13	0.01\\
61.14	0.01\\
61.15	0.01\\
61.16	0.01\\
61.17	0.01\\
61.18	0.01\\
61.19	0.01\\
61.2	0.01\\
61.21	0.01\\
61.22	0.01\\
61.23	0.01\\
61.24	0.01\\
61.25	0.01\\
61.26	0.01\\
61.27	0.01\\
61.28	0.01\\
61.29	0.01\\
61.3	0.01\\
61.31	0.01\\
61.32	0.01\\
61.33	0.01\\
61.34	0.01\\
61.35	0.01\\
61.36	0.01\\
61.37	0.01\\
61.38	0.01\\
61.39	0.01\\
61.4	0.01\\
61.41	0.01\\
61.42	0.01\\
61.43	0.01\\
61.44	0.01\\
61.45	0.01\\
61.46	0.01\\
61.47	0.01\\
61.48	0.01\\
61.49	0.01\\
61.5	0.01\\
61.51	0.01\\
61.52	0.01\\
61.53	0.01\\
61.54	0.01\\
61.55	0.01\\
61.56	0.01\\
61.57	0.01\\
61.58	0.01\\
61.59	0.01\\
61.6	0.01\\
61.61	0.01\\
61.62	0.01\\
61.63	0.01\\
61.64	0.01\\
61.65	0.01\\
61.66	0.01\\
61.67	0.01\\
61.68	0.01\\
61.69	0.01\\
61.7	0.01\\
61.71	0.01\\
61.72	0.01\\
61.73	0.01\\
61.74	0.01\\
61.75	0.01\\
61.76	0.01\\
61.77	0.01\\
61.78	0.01\\
61.79	0.01\\
61.8	0.01\\
61.81	0.01\\
61.82	0.01\\
61.83	0.01\\
61.84	0.01\\
61.85	0.01\\
61.86	0.01\\
61.87	0.01\\
61.88	0.01\\
61.89	0.01\\
61.9	0.01\\
61.91	0.01\\
61.92	0.01\\
61.93	0.01\\
61.94	0.01\\
61.95	0.01\\
61.96	0.01\\
61.97	0.01\\
61.98	0.01\\
61.99	0.01\\
62	0.01\\
62.01	0.01\\
62.02	0.01\\
62.03	0.01\\
62.04	0.01\\
62.05	0.01\\
62.06	0.01\\
62.07	0.01\\
62.08	0.01\\
62.09	0.01\\
62.1	0.01\\
62.11	0.01\\
62.12	0.01\\
62.13	0.01\\
62.14	0.01\\
62.15	0.01\\
62.16	0.01\\
62.17	0.01\\
62.18	0.01\\
62.19	0.01\\
62.2	0.01\\
62.21	0.01\\
62.22	0.01\\
62.23	0.01\\
62.24	0.01\\
62.25	0.01\\
62.26	0.01\\
62.27	0.01\\
62.28	0.01\\
62.29	0.01\\
62.3	0.01\\
62.31	0.01\\
62.32	0.01\\
62.33	0.01\\
62.34	0.01\\
62.35	0.01\\
62.36	0.01\\
62.37	0.01\\
62.38	0.01\\
62.39	0.01\\
62.4	0.01\\
62.41	0.01\\
62.42	0.01\\
62.43	0.01\\
62.44	0.01\\
62.45	0.01\\
62.46	0.01\\
62.47	0.01\\
62.48	0.01\\
62.49	0.01\\
62.5	0.01\\
62.51	0.01\\
62.52	0.01\\
62.53	0.01\\
62.54	0.01\\
62.55	0.01\\
62.56	0.01\\
62.57	0.01\\
62.58	0.01\\
62.59	0.01\\
62.6	0.01\\
62.61	0.01\\
62.62	0.01\\
62.63	0.01\\
62.64	0.01\\
62.65	0.01\\
62.66	0.01\\
62.67	0.01\\
62.68	0.01\\
62.69	0.01\\
62.7	0.01\\
62.71	0.01\\
62.72	0.01\\
62.73	0.01\\
62.74	0.01\\
62.75	0.01\\
62.76	0.01\\
62.77	0.01\\
62.78	0.01\\
62.79	0.01\\
62.8	0.01\\
62.81	0.01\\
62.82	0.01\\
62.83	0.01\\
62.84	0.01\\
62.85	0.01\\
62.86	0.01\\
62.87	0.01\\
62.88	0.01\\
62.89	0.01\\
62.9	0.01\\
62.91	0.01\\
62.92	0.01\\
62.93	0.01\\
62.94	0.01\\
62.95	0.01\\
62.96	0.01\\
62.97	0.01\\
62.98	0.01\\
62.99	0.01\\
63	0.01\\
63.01	0.01\\
63.02	0.01\\
63.03	0.01\\
63.04	0.01\\
63.05	0.01\\
63.06	0.01\\
63.07	0.01\\
63.08	0.01\\
63.09	0.01\\
63.1	0.01\\
63.11	0.01\\
63.12	0.01\\
63.13	0.01\\
63.14	0.01\\
63.15	0.01\\
63.16	0.01\\
63.17	0.01\\
63.18	0.01\\
63.19	0.01\\
63.2	0.01\\
63.21	0.01\\
63.22	0.01\\
63.23	0.01\\
63.24	0.01\\
63.25	0.01\\
63.26	0.01\\
63.27	0.01\\
63.28	0.01\\
63.29	0.01\\
63.3	0.01\\
63.31	0.01\\
63.32	0.01\\
63.33	0.01\\
63.34	0.01\\
63.35	0.01\\
63.36	0.01\\
63.37	0.01\\
63.38	0.01\\
63.39	0.01\\
63.4	0.01\\
63.41	0.01\\
63.42	0.01\\
63.43	0.01\\
63.44	0.01\\
63.45	0.01\\
63.46	0.01\\
63.47	0.01\\
63.48	0.01\\
63.49	0.01\\
63.5	0.01\\
63.51	0.01\\
63.52	0.01\\
63.53	0.01\\
63.54	0.01\\
63.55	0.01\\
63.56	0.01\\
63.57	0.01\\
63.58	0.01\\
63.59	0.01\\
63.6	0.01\\
63.61	0.01\\
63.62	0.01\\
63.63	0.01\\
63.64	0.01\\
63.65	0.01\\
63.66	0.01\\
63.67	0.01\\
63.68	0.01\\
63.69	0.01\\
63.7	0.01\\
63.71	0.01\\
63.72	0.01\\
63.73	0.01\\
63.74	0.01\\
63.75	0.01\\
63.76	0.01\\
63.77	0.01\\
63.78	0.01\\
63.79	0.01\\
63.8	0.01\\
63.81	0.01\\
63.82	0.01\\
63.83	0.01\\
63.84	0.01\\
63.85	0.01\\
63.86	0.01\\
63.87	0.01\\
63.88	0.01\\
63.89	0.01\\
63.9	0.01\\
63.91	0.01\\
63.92	0.01\\
63.93	0.01\\
63.94	0.01\\
63.95	0.01\\
63.96	0.01\\
63.97	0.01\\
63.98	0.01\\
63.99	0.01\\
64	0.01\\
64.01	0.01\\
64.02	0.01\\
64.03	0.01\\
64.04	0.01\\
64.05	0.01\\
64.06	0.01\\
64.07	0.01\\
64.08	0.01\\
64.09	0.01\\
64.1	0.01\\
64.11	0.01\\
64.12	0.01\\
64.13	0.01\\
64.14	0.01\\
64.15	0.01\\
64.16	0.01\\
64.17	0.01\\
64.18	0.01\\
64.19	0.01\\
64.2	0.01\\
64.21	0.01\\
64.22	0.01\\
64.23	0.01\\
64.24	0.01\\
64.25	0.01\\
64.26	0.01\\
64.27	0.01\\
64.28	0.01\\
64.29	0.01\\
64.3	0.01\\
64.31	0.01\\
64.32	0.01\\
64.33	0.01\\
64.34	0.01\\
64.35	0.01\\
64.36	0.01\\
64.37	0.01\\
64.38	0.01\\
64.39	0.01\\
64.4	0.01\\
64.41	0.01\\
64.42	0.01\\
64.43	0.01\\
64.44	0.01\\
64.45	0.01\\
64.46	0.01\\
64.47	0.01\\
64.48	0.01\\
64.49	0.01\\
64.5	0.01\\
64.51	0.01\\
64.52	0.01\\
64.53	0.01\\
64.54	0.01\\
64.55	0.01\\
64.56	0.01\\
64.57	0.01\\
64.58	0.01\\
64.59	0.01\\
64.6	0.01\\
64.61	0.01\\
64.62	0.01\\
64.63	0.01\\
64.64	0.01\\
64.65	0.01\\
64.66	0.01\\
64.67	0.01\\
64.68	0.01\\
64.69	0.01\\
64.7	0.01\\
64.71	0.01\\
64.72	0.01\\
64.73	0.01\\
64.74	0.01\\
64.75	0.01\\
64.76	0.01\\
64.77	0.01\\
64.78	0.01\\
64.79	0.01\\
64.8	0.01\\
64.81	0.01\\
64.82	0.01\\
64.83	0.01\\
64.84	0.01\\
64.85	0.01\\
64.86	0.01\\
64.87	0.01\\
64.88	0.01\\
64.89	0.01\\
64.9	0.01\\
64.91	0.01\\
64.92	0.01\\
64.93	0.01\\
64.94	0.01\\
64.95	0.01\\
64.96	0.01\\
64.97	0.01\\
64.98	0.01\\
64.99	0.01\\
65	0.01\\
65.01	0.01\\
65.02	0.01\\
65.03	0.01\\
65.04	0.01\\
65.05	0.01\\
65.06	0.01\\
65.07	0.01\\
65.08	0.01\\
65.09	0.01\\
65.1	0.01\\
65.11	0.01\\
65.12	0.01\\
65.13	0.01\\
65.14	0.01\\
65.15	0.01\\
65.16	0.01\\
65.17	0.01\\
65.18	0.01\\
65.19	0.01\\
65.2	0.01\\
65.21	0.01\\
65.22	0.01\\
65.23	0.01\\
65.24	0.01\\
65.25	0.01\\
65.26	0.01\\
65.27	0.01\\
65.28	0.01\\
65.29	0.01\\
65.3	0.01\\
65.31	0.01\\
65.32	0.01\\
65.33	0.01\\
65.34	0.01\\
65.35	0.01\\
65.36	0.01\\
65.37	0.01\\
65.38	0.01\\
65.39	0.01\\
65.4	0.01\\
65.41	0.01\\
65.42	0.01\\
65.43	0.01\\
65.44	0.01\\
65.45	0.01\\
65.46	0.01\\
65.47	0.01\\
65.48	0.01\\
65.49	0.01\\
65.5	0.01\\
65.51	0.01\\
65.52	0.01\\
65.53	0.01\\
65.54	0.01\\
65.55	0.01\\
65.56	0.01\\
65.57	0.01\\
65.58	0.01\\
65.59	0.01\\
65.6	0.01\\
65.61	0.01\\
65.62	0.01\\
65.63	0.01\\
65.64	0.01\\
65.65	0.01\\
65.66	0.01\\
65.67	0.01\\
65.68	0.01\\
65.69	0.01\\
65.7	0.01\\
65.71	0.01\\
65.72	0.01\\
65.73	0.01\\
65.74	0.01\\
65.75	0.01\\
65.76	0.01\\
65.77	0.01\\
65.78	0.01\\
65.79	0.01\\
65.8	0.01\\
65.81	0.01\\
65.82	0.01\\
65.83	0.01\\
65.84	0.01\\
65.85	0.01\\
65.86	0.01\\
65.87	0.01\\
65.88	0.01\\
65.89	0.01\\
65.9	0.01\\
65.91	0.01\\
65.92	0.01\\
65.93	0.01\\
65.94	0.01\\
65.95	0.01\\
65.96	0.01\\
65.97	0.01\\
65.98	0.01\\
65.99	0.01\\
66	0.01\\
66.01	0.01\\
66.02	0.01\\
66.03	0.01\\
66.04	0.01\\
66.05	0.01\\
66.06	0.01\\
66.07	0.01\\
66.08	0.01\\
66.09	0.01\\
66.1	0.01\\
66.11	0.01\\
66.12	0.01\\
66.13	0.01\\
66.14	0.01\\
66.15	0.01\\
66.16	0.01\\
66.17	0.01\\
66.18	0.01\\
66.19	0.01\\
66.2	0.01\\
66.21	0.01\\
66.22	0.01\\
66.23	0.01\\
66.24	0.01\\
66.25	0.01\\
66.26	0.01\\
66.27	0.01\\
66.28	0.01\\
66.29	0.01\\
66.3	0.01\\
66.31	0.01\\
66.32	0.01\\
66.33	0.01\\
66.34	0.01\\
66.35	0.01\\
66.36	0.01\\
66.37	0.01\\
66.38	0.01\\
66.39	0.01\\
66.4	0.01\\
66.41	0.01\\
66.42	0.01\\
66.43	0.01\\
66.44	0.01\\
66.45	0.01\\
66.46	0.01\\
66.47	0.01\\
66.48	0.01\\
66.49	0.01\\
66.5	0.01\\
66.51	0.01\\
66.52	0.01\\
66.53	0.01\\
66.54	0.01\\
66.55	0.01\\
66.56	0.01\\
66.57	0.01\\
66.58	0.01\\
66.59	0.01\\
66.6	0.01\\
66.61	0.01\\
66.62	0.01\\
66.63	0.01\\
66.64	0.01\\
66.65	0.01\\
66.66	0.01\\
66.67	0.01\\
66.68	0.01\\
66.69	0.01\\
66.7	0.01\\
66.71	0.01\\
66.72	0.01\\
66.73	0.01\\
66.74	0.01\\
66.75	0.01\\
66.76	0.01\\
66.77	0.01\\
66.78	0.01\\
66.79	0.01\\
66.8	0.01\\
66.81	0.01\\
66.82	0.01\\
66.83	0.01\\
66.84	0.01\\
66.85	0.01\\
66.86	0.01\\
66.87	0.01\\
66.88	0.01\\
66.89	0.01\\
66.9	0.01\\
66.91	0.01\\
66.92	0.01\\
66.93	0.01\\
66.94	0.01\\
66.95	0.01\\
66.96	0.01\\
66.97	0.01\\
66.98	0.01\\
66.99	0.01\\
67	0.01\\
67.01	0.01\\
67.02	0.01\\
67.03	0.01\\
67.04	0.01\\
67.05	0.01\\
67.06	0.01\\
67.07	0.01\\
67.08	0.01\\
67.09	0.01\\
67.1	0.01\\
67.11	0.01\\
67.12	0.01\\
67.13	0.01\\
67.14	0.01\\
67.15	0.01\\
67.16	0.01\\
67.17	0.01\\
67.18	0.01\\
67.19	0.01\\
67.2	0.01\\
67.21	0.01\\
67.22	0.01\\
67.23	0.01\\
67.24	0.01\\
67.25	0.01\\
67.26	0.01\\
67.27	0.01\\
67.28	0.01\\
67.29	0.01\\
67.3	0.01\\
67.31	0.01\\
67.32	0.01\\
67.33	0.01\\
67.34	0.01\\
67.35	0.01\\
67.36	0.01\\
67.37	0.01\\
67.38	0.01\\
67.39	0.01\\
67.4	0.01\\
67.41	0.01\\
67.42	0.01\\
67.43	0.01\\
67.44	0.01\\
67.45	0.01\\
67.46	0.01\\
67.47	0.01\\
67.48	0.01\\
67.49	0.01\\
67.5	0.01\\
67.51	0.01\\
67.52	0.01\\
67.53	0.01\\
67.54	0.01\\
67.55	0.01\\
67.56	0.01\\
67.57	0.01\\
67.58	0.01\\
67.59	0.01\\
67.6	0.01\\
67.61	0.01\\
67.62	0.01\\
67.63	0.01\\
67.64	0.01\\
67.65	0.01\\
67.66	0.01\\
67.67	0.01\\
67.68	0.01\\
67.69	0.01\\
67.7	0.01\\
67.71	0.01\\
67.72	0.01\\
67.73	0.01\\
67.74	0.01\\
67.75	0.01\\
67.76	0.01\\
67.77	0.01\\
67.78	0.01\\
67.79	0.01\\
67.8	0.01\\
67.81	0.01\\
67.82	0.01\\
67.83	0.01\\
67.84	0.01\\
67.85	0.01\\
67.86	0.01\\
67.87	0.01\\
67.88	0.01\\
67.89	0.01\\
67.9	0.01\\
67.91	0.01\\
67.92	0.01\\
67.93	0.01\\
67.94	0.01\\
67.95	0.01\\
67.96	0.01\\
67.97	0.01\\
67.98	0.01\\
67.99	0.01\\
68	0.01\\
68.01	0.01\\
68.02	0.01\\
68.03	0.01\\
68.04	0.01\\
68.05	0.01\\
68.06	0.01\\
68.07	0.01\\
68.08	0.01\\
68.09	0.01\\
68.1	0.01\\
68.11	0.01\\
68.12	0.01\\
68.13	0.01\\
68.14	0.01\\
68.15	0.01\\
68.16	0.01\\
68.17	0.01\\
68.18	0.01\\
68.19	0.01\\
68.2	0.01\\
68.21	0.01\\
68.22	0.01\\
68.23	0.01\\
68.24	0.01\\
68.25	0.01\\
68.26	0.01\\
68.27	0.01\\
68.28	0.01\\
68.29	0.01\\
68.3	0.01\\
68.31	0.01\\
68.32	0.01\\
68.33	0.01\\
68.34	0.01\\
68.35	0.01\\
68.36	0.01\\
68.37	0.01\\
68.38	0.01\\
68.39	0.01\\
68.4	0.01\\
68.41	0.01\\
68.42	0.01\\
68.43	0.01\\
68.44	0.01\\
68.45	0.01\\
68.46	0.01\\
68.47	0.01\\
68.48	0.01\\
68.49	0.01\\
68.5	0.01\\
68.51	0.01\\
68.52	0.01\\
68.53	0.01\\
68.54	0.01\\
68.55	0.01\\
68.56	0.01\\
68.57	0.01\\
68.58	0.01\\
68.59	0.01\\
68.6	0.01\\
68.61	0.01\\
68.62	0.01\\
68.63	0.01\\
68.64	0.01\\
68.65	0.01\\
68.66	0.01\\
68.67	0.01\\
68.68	0.01\\
68.69	0.01\\
68.7	0.01\\
68.71	0.01\\
68.72	0.01\\
68.73	0.01\\
68.74	0.01\\
68.75	0.01\\
68.76	0.01\\
68.77	0.01\\
68.78	0.01\\
68.79	0.01\\
68.8	0.01\\
68.81	0.01\\
68.82	0.01\\
68.83	0.01\\
68.84	0.01\\
68.85	0.01\\
68.86	0.01\\
68.87	0.01\\
68.88	0.01\\
68.89	0.01\\
68.9	0.01\\
68.91	0.01\\
68.92	0.01\\
68.93	0.01\\
68.94	0.01\\
68.95	0.01\\
68.96	0.01\\
68.97	0.01\\
68.98	0.01\\
68.99	0.01\\
69	0.01\\
69.01	0.01\\
69.02	0.01\\
69.03	0.01\\
69.04	0.01\\
69.05	0.01\\
69.06	0.01\\
69.07	0.01\\
69.08	0.01\\
69.09	0.01\\
69.1	0.01\\
69.11	0.01\\
69.12	0.01\\
69.13	0.01\\
69.14	0.01\\
69.15	0.01\\
69.16	0.01\\
69.17	0.01\\
69.18	0.01\\
69.19	0.01\\
69.2	0.01\\
69.21	0.01\\
69.22	0.01\\
69.23	0.01\\
69.24	0.01\\
69.25	0.01\\
69.26	0.01\\
69.27	0.01\\
69.28	0.01\\
69.29	0.01\\
69.3	0.01\\
69.31	0.01\\
69.32	0.01\\
69.33	0.01\\
69.34	0.01\\
69.35	0.01\\
69.36	0.01\\
69.37	0.01\\
69.38	0.01\\
69.39	0.01\\
69.4	0.01\\
69.41	0.01\\
69.42	0.01\\
69.43	0.01\\
69.44	0.01\\
69.45	0.01\\
69.46	0.01\\
69.47	0.01\\
69.48	0.01\\
69.49	0.01\\
69.5	0.01\\
69.51	0.01\\
69.52	0.01\\
69.53	0.01\\
69.54	0.01\\
69.55	0.01\\
69.56	0.01\\
69.57	0.01\\
69.58	0.01\\
69.59	0.01\\
69.6	0.01\\
69.61	0.01\\
69.62	0.01\\
69.63	0.01\\
69.64	0.01\\
69.65	0.01\\
69.66	0.01\\
69.67	0.01\\
69.68	0.01\\
69.69	0.01\\
69.7	0.01\\
69.71	0.01\\
69.72	0.01\\
69.73	0.01\\
69.74	0.01\\
69.75	0.01\\
69.76	0.01\\
69.77	0.01\\
69.78	0.01\\
69.79	0.01\\
69.8	0.01\\
69.81	0.01\\
69.82	0.01\\
69.83	0.01\\
69.84	0.01\\
69.85	0.01\\
69.86	0.01\\
69.87	0.01\\
69.88	0.01\\
69.89	0.01\\
69.9	0.01\\
69.91	0.01\\
69.92	0.01\\
69.93	0.01\\
69.94	0.01\\
69.95	0.01\\
69.96	0.01\\
69.97	0.01\\
69.98	0.01\\
69.99	0.01\\
70	0.01\\
70.01	0.01\\
70.02	0.01\\
70.03	0.01\\
70.04	0.01\\
70.05	0.01\\
70.06	0.01\\
70.07	0.01\\
70.08	0.01\\
70.09	0.01\\
70.1	0.01\\
70.11	0.01\\
70.12	0.01\\
70.13	0.01\\
70.14	0.01\\
70.15	0.01\\
70.16	0.01\\
70.17	0.01\\
70.18	0.01\\
70.19	0.01\\
70.2	0.01\\
70.21	0.01\\
70.22	0.01\\
70.23	0.01\\
70.24	0.01\\
70.25	0.01\\
70.26	0.01\\
70.27	0.01\\
70.28	0.01\\
70.29	0.01\\
70.3	0.01\\
70.31	0.01\\
70.32	0.01\\
70.33	0.01\\
70.34	0.01\\
70.35	0.01\\
70.36	0.01\\
70.37	0.01\\
70.38	0.01\\
70.39	0.01\\
70.4	0.01\\
70.41	0.01\\
70.42	0.01\\
70.43	0.01\\
70.44	0.01\\
70.45	0.01\\
70.46	0.01\\
70.47	0.01\\
70.48	0.01\\
70.49	0.01\\
70.5	0.01\\
70.51	0.01\\
70.52	0.01\\
70.53	0.01\\
70.54	0.01\\
70.55	0.01\\
70.56	0.01\\
70.57	0.01\\
70.58	0.01\\
70.59	0.01\\
70.6	0.01\\
70.61	0.01\\
70.62	0.01\\
70.63	0.01\\
70.64	0.01\\
70.65	0.01\\
70.66	0.01\\
70.67	0.01\\
70.68	0.01\\
70.69	0.01\\
70.7	0.01\\
70.71	0.01\\
70.72	0.01\\
70.73	0.01\\
70.74	0.01\\
70.75	0.01\\
70.76	0.01\\
70.77	0.01\\
70.78	0.01\\
70.79	0.01\\
70.8	0.01\\
70.81	0.01\\
70.82	0.01\\
70.83	0.01\\
70.84	0.01\\
70.85	0.01\\
70.86	0.01\\
70.87	0.01\\
70.88	0.01\\
70.89	0.01\\
70.9	0.01\\
70.91	0.01\\
70.92	0.01\\
70.93	0.01\\
70.94	0.01\\
70.95	0.01\\
70.96	0.01\\
70.97	0.01\\
70.98	0.01\\
70.99	0.01\\
71	0.01\\
71.01	0.01\\
71.02	0.01\\
71.03	0.01\\
71.04	0.01\\
71.05	0.01\\
71.06	0.01\\
71.07	0.01\\
71.08	0.01\\
71.09	0.01\\
71.1	0.01\\
71.11	0.01\\
71.12	0.01\\
71.13	0.01\\
71.14	0.01\\
71.15	0.01\\
71.16	0.01\\
71.17	0.01\\
71.18	0.01\\
71.19	0.01\\
71.2	0.01\\
71.21	0.01\\
71.22	0.01\\
71.23	0.01\\
71.24	0.01\\
71.25	0.01\\
71.26	0.01\\
71.27	0.01\\
71.28	0.01\\
71.29	0.01\\
71.3	0.01\\
71.31	0.01\\
71.32	0.01\\
71.33	0.01\\
71.34	0.01\\
71.35	0.01\\
71.36	0.01\\
71.37	0.01\\
71.38	0.01\\
71.39	0.01\\
71.4	0.01\\
71.41	0.01\\
71.42	0.01\\
71.43	0.01\\
71.44	0.01\\
71.45	0.01\\
71.46	0.01\\
71.47	0.01\\
71.48	0.01\\
71.49	0.01\\
71.5	0.01\\
71.51	0.01\\
71.52	0.01\\
71.53	0.01\\
71.54	0.01\\
71.55	0.01\\
71.56	0.01\\
71.57	0.01\\
71.58	0.01\\
71.59	0.01\\
71.6	0.01\\
71.61	0.01\\
71.62	0.01\\
71.63	0.01\\
71.64	0.01\\
71.65	0.01\\
71.66	0.01\\
71.67	0.01\\
71.68	0.01\\
71.69	0.01\\
71.7	0.01\\
71.71	0.01\\
71.72	0.01\\
71.73	0.01\\
71.74	0.01\\
71.75	0.01\\
71.76	0.01\\
71.77	0.01\\
71.78	0.01\\
71.79	0.01\\
71.8	0.01\\
71.81	0.01\\
71.82	0.01\\
71.83	0.01\\
71.84	0.01\\
71.85	0.01\\
71.86	0.01\\
71.87	0.01\\
71.88	0.01\\
71.89	0.01\\
71.9	0.01\\
71.91	0.01\\
71.92	0.01\\
71.93	0.01\\
71.94	0.01\\
71.95	0.01\\
71.96	0.01\\
71.97	0.01\\
71.98	0.01\\
71.99	0.01\\
72	0.01\\
72.01	0.01\\
72.02	0.01\\
72.03	0.01\\
72.04	0.01\\
72.05	0.01\\
72.06	0.01\\
72.07	0.01\\
72.08	0.01\\
72.09	0.01\\
72.1	0.01\\
72.11	0.01\\
72.12	0.01\\
72.13	0.01\\
72.14	0.01\\
72.15	0.01\\
72.16	0.01\\
72.17	0.01\\
72.18	0.01\\
72.19	0.01\\
72.2	0.01\\
72.21	0.01\\
72.22	0.01\\
72.23	0.01\\
72.24	0.01\\
72.25	0.01\\
72.26	0.01\\
72.27	0.01\\
72.28	0.01\\
72.29	0.01\\
72.3	0.01\\
72.31	0.01\\
72.32	0.01\\
72.33	0.01\\
72.34	0.01\\
72.35	0.01\\
72.36	0.01\\
72.37	0.01\\
72.38	0.01\\
72.39	0.01\\
72.4	0.01\\
72.41	0.01\\
72.42	0.01\\
72.43	0.01\\
72.44	0.01\\
72.45	0.01\\
72.46	0.01\\
72.47	0.01\\
72.48	0.01\\
72.49	0.01\\
72.5	0.01\\
72.51	0.01\\
72.52	0.01\\
72.53	0.01\\
72.54	0.01\\
72.55	0.01\\
72.56	0.01\\
72.57	0.01\\
72.58	0.01\\
72.59	0.01\\
72.6	0.01\\
72.61	0.01\\
72.62	0.01\\
72.63	0.01\\
72.64	0.01\\
72.65	0.01\\
72.66	0.01\\
72.67	0.01\\
72.68	0.01\\
72.69	0.01\\
72.7	0.01\\
72.71	0.01\\
72.72	0.01\\
72.73	0.01\\
72.74	0.01\\
72.75	0.01\\
72.76	0.01\\
72.77	0.01\\
72.78	0.01\\
72.79	0.01\\
72.8	0.01\\
72.81	0.01\\
72.82	0.01\\
72.83	0.01\\
72.84	0.01\\
72.85	0.01\\
72.86	0.01\\
72.87	0.01\\
72.88	0.01\\
72.89	0.01\\
72.9	0.01\\
72.91	0.01\\
72.92	0.01\\
72.93	0.01\\
72.94	0.01\\
72.95	0.01\\
72.96	0.01\\
72.97	0.01\\
72.98	0.01\\
72.99	0.01\\
73	0.01\\
73.01	0.01\\
73.02	0.01\\
73.03	0.01\\
73.04	0.01\\
73.05	0.01\\
73.06	0.01\\
73.07	0.01\\
73.08	0.01\\
73.09	0.01\\
73.1	0.01\\
73.11	0.01\\
73.12	0.01\\
73.13	0.01\\
73.14	0.01\\
73.15	0.01\\
73.16	0.01\\
73.17	0.01\\
73.18	0.01\\
73.19	0.01\\
73.2	0.01\\
73.21	0.01\\
73.22	0.01\\
73.23	0.01\\
73.24	0.01\\
73.25	0.01\\
73.26	0.01\\
73.27	0.01\\
73.28	0.01\\
73.29	0.01\\
73.3	0.01\\
73.31	0.01\\
73.32	0.01\\
73.33	0.01\\
73.34	0.01\\
73.35	0.01\\
73.36	0.01\\
73.37	0.01\\
73.38	0.01\\
73.39	0.01\\
73.4	0.01\\
73.41	0.01\\
73.42	0.01\\
73.43	0.01\\
73.44	0.01\\
73.45	0.01\\
73.46	0.01\\
73.47	0.01\\
73.48	0.01\\
73.49	0.01\\
73.5	0.01\\
73.51	0.01\\
73.52	0.01\\
73.53	0.01\\
73.54	0.01\\
73.55	0.01\\
73.56	0.01\\
73.57	0.01\\
73.58	0.01\\
73.59	0.01\\
73.6	0.01\\
73.61	0.01\\
73.62	0.01\\
73.63	0.01\\
73.64	0.01\\
73.65	0.01\\
73.66	0.01\\
73.67	0.01\\
73.68	0.01\\
73.69	0.01\\
73.7	0.01\\
73.71	0.01\\
73.72	0.01\\
73.73	0.01\\
73.74	0.01\\
73.75	0.01\\
73.76	0.01\\
73.77	0.01\\
73.78	0.01\\
73.79	0.01\\
73.8	0.01\\
73.81	0.01\\
73.82	0.01\\
73.83	0.01\\
73.84	0.01\\
73.85	0.01\\
73.86	0.01\\
73.87	0.01\\
73.88	0.01\\
73.89	0.01\\
73.9	0.01\\
73.91	0.01\\
73.92	0.01\\
73.93	0.01\\
73.94	0.01\\
73.95	0.01\\
73.96	0.01\\
73.97	0.01\\
73.98	0.01\\
73.99	0.01\\
74	0.01\\
74.01	0.01\\
74.02	0.01\\
74.03	0.01\\
74.04	0.01\\
74.05	0.01\\
74.06	0.01\\
74.07	0.01\\
74.08	0.01\\
74.09	0.01\\
74.1	0.01\\
74.11	0.01\\
74.12	0.01\\
74.13	0.01\\
74.14	0.01\\
74.15	0.01\\
74.16	0.01\\
74.17	0.01\\
74.18	0.01\\
74.19	0.01\\
74.2	0.01\\
74.21	0.01\\
74.22	0.01\\
74.23	0.01\\
74.24	0.01\\
74.25	0.01\\
74.26	0.01\\
74.27	0.01\\
74.28	0.01\\
74.29	0.01\\
74.3	0.01\\
74.31	0.01\\
74.32	0.01\\
74.33	0.01\\
74.34	0.01\\
74.35	0.01\\
74.36	0.01\\
74.37	0.01\\
74.38	0.01\\
74.39	0.01\\
74.4	0.01\\
74.41	0.01\\
74.42	0.01\\
74.43	0.01\\
74.44	0.01\\
74.45	0.01\\
74.46	0.01\\
74.47	0.01\\
74.48	0.01\\
74.49	0.01\\
74.5	0.01\\
74.51	0.01\\
74.52	0.01\\
74.53	0.01\\
74.54	0.01\\
74.55	0.01\\
74.56	0.01\\
74.57	0.01\\
74.58	0.01\\
74.59	0.01\\
74.6	0.01\\
74.61	0.01\\
74.62	0.01\\
74.63	0.01\\
74.64	0.01\\
74.65	0.01\\
74.66	0.01\\
74.67	0.01\\
74.68	0.01\\
74.69	0.01\\
74.7	0.01\\
74.71	0.01\\
74.72	0.01\\
74.73	0.01\\
74.74	0.01\\
74.75	0.01\\
74.76	0.01\\
74.77	0.01\\
74.78	0.01\\
74.79	0.01\\
74.8	0.01\\
74.81	0.01\\
74.82	0.01\\
74.83	0.01\\
74.84	0.01\\
74.85	0.01\\
74.86	0.01\\
74.87	0.01\\
74.88	0.01\\
74.89	0.01\\
74.9	0.01\\
74.91	0.01\\
74.92	0.01\\
74.93	0.01\\
74.94	0.01\\
74.95	0.01\\
74.96	0.01\\
74.97	0.01\\
74.98	0.01\\
74.99	0.01\\
75	0.01\\
75.01	0.01\\
75.02	0.01\\
75.03	0.01\\
75.04	0.01\\
75.05	0.01\\
75.06	0.01\\
75.07	0.01\\
75.08	0.01\\
75.09	0.01\\
75.1	0.01\\
75.11	0.01\\
75.12	0.01\\
75.13	0.01\\
75.14	0.01\\
75.15	0.01\\
75.16	0.01\\
75.17	0.01\\
75.18	0.01\\
75.19	0.01\\
75.2	0.01\\
75.21	0.01\\
75.22	0.01\\
75.23	0.01\\
75.24	0.01\\
75.25	0.01\\
75.26	0.01\\
75.27	0.01\\
75.28	0.01\\
75.29	0.01\\
75.3	0.01\\
75.31	0.01\\
75.32	0.01\\
75.33	0.01\\
75.34	0.01\\
75.35	0.01\\
75.36	0.01\\
75.37	0.01\\
75.38	0.01\\
75.39	0.01\\
75.4	0.01\\
75.41	0.01\\
75.42	0.01\\
75.43	0.01\\
75.44	0.01\\
75.45	0.01\\
75.46	0.01\\
75.47	0.01\\
75.48	0.01\\
75.49	0.01\\
75.5	0.01\\
75.51	0.01\\
75.52	0.01\\
75.53	0.01\\
75.54	0.01\\
75.55	0.01\\
75.56	0.01\\
75.57	0.01\\
75.58	0.01\\
75.59	0.01\\
75.6	0.01\\
75.61	0.01\\
75.62	0.01\\
75.63	0.01\\
75.64	0.01\\
75.65	0.01\\
75.66	0.01\\
75.67	0.01\\
75.68	0.01\\
75.69	0.01\\
75.7	0.01\\
75.71	0.01\\
75.72	0.01\\
75.73	0.01\\
75.74	0.01\\
75.75	0.01\\
75.76	0.01\\
75.77	0.01\\
75.78	0.01\\
75.79	0.01\\
75.8	0.01\\
75.81	0.01\\
75.82	0.01\\
75.83	0.01\\
75.84	0.01\\
75.85	0.01\\
75.86	0.01\\
75.87	0.01\\
75.88	0.01\\
75.89	0.01\\
75.9	0.01\\
75.91	0.01\\
75.92	0.01\\
75.93	0.01\\
75.94	0.01\\
75.95	0.01\\
75.96	0.01\\
75.97	0.01\\
75.98	0.01\\
75.99	0.01\\
76	0.01\\
76.01	0.01\\
76.02	0.01\\
76.03	0.01\\
76.04	0.01\\
76.05	0.01\\
76.06	0.01\\
76.07	0.01\\
76.08	0.01\\
76.09	0.01\\
76.1	0.01\\
76.11	0.01\\
76.12	0.01\\
76.13	0.01\\
76.14	0.01\\
76.15	0.01\\
76.16	0.01\\
76.17	0.01\\
76.18	0.01\\
76.19	0.01\\
76.2	0.01\\
76.21	0.01\\
76.22	0.01\\
76.23	0.01\\
76.24	0.01\\
76.25	0.01\\
76.26	0.01\\
76.27	0.01\\
76.28	0.01\\
76.29	0.01\\
76.3	0.01\\
76.31	0.01\\
76.32	0.01\\
76.33	0.01\\
76.34	0.01\\
76.35	0.01\\
76.36	0.01\\
76.37	0.01\\
76.38	0.01\\
76.39	0.01\\
76.4	0.01\\
76.41	0.01\\
76.42	0.01\\
76.43	0.01\\
76.44	0.01\\
76.45	0.01\\
76.46	0.01\\
76.47	0.01\\
76.48	0.01\\
76.49	0.01\\
76.5	0.01\\
76.51	0.01\\
76.52	0.01\\
76.53	0.01\\
76.54	0.01\\
76.55	0.01\\
76.56	0.01\\
76.57	0.01\\
76.58	0.01\\
76.59	0.01\\
76.6	0.01\\
76.61	0.01\\
76.62	0.01\\
76.63	0.01\\
76.64	0.01\\
76.65	0.01\\
76.66	0.01\\
76.67	0.01\\
76.68	0.01\\
76.69	0.01\\
76.7	0.01\\
76.71	0.01\\
76.72	0.01\\
76.73	0.01\\
76.74	0.01\\
76.75	0.01\\
76.76	0.01\\
76.77	0.01\\
76.78	0.01\\
76.79	0.01\\
76.8	0.01\\
76.81	0.01\\
76.82	0.01\\
76.83	0.01\\
76.84	0.01\\
76.85	0.01\\
76.86	0.01\\
76.87	0.01\\
76.88	0.01\\
76.89	0.01\\
76.9	0.01\\
76.91	0.01\\
76.92	0.01\\
76.93	0.01\\
76.94	0.01\\
76.95	0.01\\
76.96	0.01\\
76.97	0.01\\
76.98	0.01\\
76.99	0.01\\
77	0.01\\
77.01	0.01\\
77.02	0.01\\
77.03	0.01\\
77.04	0.01\\
77.05	0.01\\
77.06	0.01\\
77.07	0.01\\
77.08	0.01\\
77.09	0.01\\
77.1	0.01\\
77.11	0.01\\
77.12	0.01\\
77.13	0.01\\
77.14	0.01\\
77.15	0.01\\
77.16	0.01\\
77.17	0.01\\
77.18	0.01\\
77.19	0.01\\
77.2	0.01\\
77.21	0.01\\
77.22	0.01\\
77.23	0.01\\
77.24	0.01\\
77.25	0.01\\
77.26	0.01\\
77.27	0.01\\
77.28	0.01\\
77.29	0.01\\
77.3	0.01\\
77.31	0.01\\
77.32	0.01\\
77.33	0.01\\
77.34	0.01\\
77.35	0.01\\
77.36	0.01\\
77.37	0.01\\
77.38	0.01\\
77.39	0.01\\
77.4	0.01\\
77.41	0.01\\
77.42	0.01\\
77.43	0.01\\
77.44	0.01\\
77.45	0.01\\
77.46	0.01\\
77.47	0.01\\
77.48	0.01\\
77.49	0.01\\
77.5	0.01\\
77.51	0.01\\
77.52	0.01\\
77.53	0.01\\
77.54	0.01\\
77.55	0.01\\
77.56	0.01\\
77.57	0.01\\
77.58	0.01\\
77.59	0.01\\
77.6	0.01\\
77.61	0.01\\
77.62	0.01\\
77.63	0.01\\
77.64	0.01\\
77.65	0.01\\
77.66	0.01\\
77.67	0.01\\
77.68	0.01\\
77.69	0.01\\
77.7	0.01\\
77.71	0.01\\
77.72	0.01\\
77.73	0.01\\
77.74	0.01\\
77.75	0.01\\
77.76	0.01\\
77.77	0.01\\
77.78	0.01\\
77.79	0.01\\
77.8	0.01\\
77.81	0.01\\
77.82	0.01\\
77.83	0.01\\
77.84	0.01\\
77.85	0.01\\
77.86	0.01\\
77.87	0.01\\
77.88	0.01\\
77.89	0.01\\
77.9	0.01\\
77.91	0.01\\
77.92	0.01\\
77.93	0.01\\
77.94	0.01\\
77.95	0.01\\
77.96	0.01\\
77.97	0.01\\
77.98	0.01\\
77.99	0.01\\
78	0.01\\
78.01	0.01\\
78.02	0.01\\
78.03	0.01\\
78.04	0.01\\
78.05	0.01\\
78.06	0.01\\
78.07	0.01\\
78.08	0.01\\
78.09	0.01\\
78.1	0.01\\
78.11	0.01\\
78.12	0.01\\
78.13	0.01\\
78.14	0.01\\
78.15	0.01\\
78.16	0.01\\
78.17	0.01\\
78.18	0.01\\
78.19	0.01\\
78.2	0.01\\
78.21	0.01\\
78.22	0.01\\
78.23	0.01\\
78.24	0.01\\
78.25	0.01\\
78.26	0.01\\
78.27	0.01\\
78.28	0.01\\
78.29	0.01\\
78.3	0.01\\
78.31	0.01\\
78.32	0.01\\
78.33	0.01\\
78.34	0.01\\
78.35	0.01\\
78.36	0.01\\
78.37	0.01\\
78.38	0.01\\
78.39	0.01\\
78.4	0.01\\
78.41	0.01\\
78.42	0.01\\
78.43	0.01\\
78.44	0.01\\
78.45	0.01\\
78.46	0.01\\
78.47	0.01\\
78.48	0.01\\
78.49	0.01\\
78.5	0.01\\
78.51	0.01\\
78.52	0.01\\
78.53	0.01\\
78.54	0.01\\
78.55	0.01\\
78.56	0.01\\
78.57	0.01\\
78.58	0.01\\
78.59	0.01\\
78.6	0.01\\
78.61	0.01\\
78.62	0.01\\
78.63	0.01\\
78.64	0.01\\
78.65	0.01\\
78.66	0.01\\
78.67	0.01\\
78.68	0.01\\
78.69	0.01\\
78.7	0.01\\
78.71	0.01\\
78.72	0.01\\
78.73	0.01\\
78.74	0.01\\
78.75	0.01\\
78.76	0.01\\
78.77	0.01\\
78.78	0.01\\
78.79	0.01\\
78.8	0.01\\
78.81	0.01\\
78.82	0.01\\
78.83	0.01\\
78.84	0.01\\
78.85	0.01\\
78.86	0.01\\
78.87	0.01\\
78.88	0.01\\
78.89	0.01\\
78.9	0.01\\
78.91	0.01\\
78.92	0.01\\
78.93	0.01\\
78.94	0.01\\
78.95	0.01\\
78.96	0.01\\
78.97	0.01\\
78.98	0.01\\
78.99	0.01\\
79	0.01\\
79.01	0.01\\
79.02	0.01\\
79.03	0.01\\
79.04	0.01\\
79.05	0.01\\
79.06	0.01\\
79.07	0.01\\
79.08	0.01\\
79.09	0.01\\
79.1	0.01\\
79.11	0.01\\
79.12	0.01\\
79.13	0.01\\
79.14	0.01\\
79.15	0.01\\
79.16	0.01\\
79.17	0.01\\
79.18	0.01\\
79.19	0.01\\
79.2	0.01\\
79.21	0.01\\
79.22	0.01\\
79.23	0.01\\
79.24	0.01\\
79.25	0.01\\
79.26	0.01\\
79.27	0.01\\
79.28	0.01\\
79.29	0.01\\
79.3	0.01\\
79.31	0.01\\
79.32	0.01\\
79.33	0.01\\
79.34	0.01\\
79.35	0.01\\
79.36	0.01\\
79.37	0.01\\
79.38	0.01\\
79.39	0.01\\
79.4	0.01\\
79.41	0.01\\
79.42	0.01\\
79.43	0.01\\
79.44	0.01\\
79.45	0.01\\
79.46	0.01\\
79.47	0.01\\
79.48	0.01\\
79.49	0.01\\
79.5	0.01\\
79.51	0.01\\
79.52	0.01\\
79.53	0.01\\
79.54	0.01\\
79.55	0.01\\
79.56	0.01\\
79.57	0.01\\
79.58	0.01\\
79.59	0.01\\
79.6	0.01\\
79.61	0.01\\
79.62	0.01\\
79.63	0.01\\
79.64	0.01\\
79.65	0.01\\
79.66	0.01\\
79.67	0.01\\
79.68	0.01\\
79.69	0.01\\
79.7	0.01\\
79.71	0.01\\
79.72	0.01\\
79.73	0.01\\
79.74	0.01\\
79.75	0.01\\
79.76	0.01\\
79.77	0.01\\
79.78	0.01\\
79.79	0.01\\
79.8	0.01\\
79.81	0.01\\
79.82	0.01\\
79.83	0.01\\
79.84	0.01\\
79.85	0.01\\
79.86	0.01\\
79.87	0.01\\
79.88	0.01\\
79.89	0.01\\
79.9	0.01\\
79.91	0.01\\
79.92	0.01\\
79.93	0.01\\
79.94	0.01\\
79.95	0.01\\
79.96	0.01\\
79.97	0.01\\
79.98	0.01\\
79.99	0.01\\
80	0.01\\
80.01	0.01\\
};
\addplot [color=mycolor1,solid]
  table[row sep=crcr]{%
80.01	0.01\\
80.02	0.01\\
80.03	0.01\\
80.04	0.01\\
80.05	0.01\\
80.06	0.01\\
80.07	0.01\\
80.08	0.01\\
80.09	0.01\\
80.1	0.01\\
80.11	0.01\\
80.12	0.01\\
80.13	0.01\\
80.14	0.01\\
80.15	0.01\\
80.16	0.01\\
80.17	0.01\\
80.18	0.01\\
80.19	0.01\\
80.2	0.01\\
80.21	0.01\\
80.22	0.01\\
80.23	0.01\\
80.24	0.01\\
80.25	0.01\\
80.26	0.01\\
80.27	0.01\\
80.28	0.01\\
80.29	0.01\\
80.3	0.01\\
80.31	0.01\\
80.32	0.01\\
80.33	0.01\\
80.34	0.01\\
80.35	0.01\\
80.36	0.01\\
80.37	0.01\\
80.38	0.01\\
80.39	0.01\\
80.4	0.01\\
80.41	0.01\\
80.42	0.01\\
80.43	0.01\\
80.44	0.01\\
80.45	0.01\\
80.46	0.01\\
80.47	0.01\\
80.48	0.01\\
80.49	0.01\\
80.5	0.01\\
80.51	0.01\\
80.52	0.01\\
80.53	0.01\\
80.54	0.01\\
80.55	0.01\\
80.56	0.01\\
80.57	0.01\\
80.58	0.01\\
80.59	0.01\\
80.6	0.01\\
80.61	0.01\\
80.62	0.01\\
80.63	0.01\\
80.64	0.01\\
80.65	0.01\\
80.66	0.01\\
80.67	0.01\\
80.68	0.01\\
80.69	0.01\\
80.7	0.01\\
80.71	0.01\\
80.72	0.01\\
80.73	0.01\\
80.74	0.01\\
80.75	0.01\\
80.76	0.01\\
80.77	0.01\\
80.78	0.01\\
80.79	0.01\\
80.8	0.01\\
80.81	0.01\\
80.82	0.01\\
80.83	0.01\\
80.84	0.01\\
80.85	0.01\\
80.86	0.01\\
80.87	0.01\\
80.88	0.01\\
80.89	0.01\\
80.9	0.01\\
80.91	0.01\\
80.92	0.01\\
80.93	0.01\\
80.94	0.01\\
80.95	0.01\\
80.96	0.01\\
80.97	0.01\\
80.98	0.01\\
80.99	0.01\\
81	0.01\\
81.01	0.01\\
81.02	0.01\\
81.03	0.01\\
81.04	0.01\\
81.05	0.01\\
81.06	0.01\\
81.07	0.01\\
81.08	0.01\\
81.09	0.01\\
81.1	0.01\\
81.11	0.01\\
81.12	0.01\\
81.13	0.01\\
81.14	0.01\\
81.15	0.01\\
81.16	0.01\\
81.17	0.01\\
81.18	0.01\\
81.19	0.01\\
81.2	0.01\\
81.21	0.01\\
81.22	0.01\\
81.23	0.01\\
81.24	0.01\\
81.25	0.01\\
81.26	0.01\\
81.27	0.01\\
81.28	0.01\\
81.29	0.01\\
81.3	0.01\\
81.31	0.01\\
81.32	0.01\\
81.33	0.01\\
81.34	0.01\\
81.35	0.01\\
81.36	0.01\\
81.37	0.01\\
81.38	0.01\\
81.39	0.01\\
81.4	0.01\\
81.41	0.01\\
81.42	0.01\\
81.43	0.01\\
81.44	0.01\\
81.45	0.01\\
81.46	0.01\\
81.47	0.01\\
81.48	0.01\\
81.49	0.01\\
81.5	0.01\\
81.51	0.01\\
81.52	0.01\\
81.53	0.01\\
81.54	0.01\\
81.55	0.01\\
81.56	0.01\\
81.57	0.01\\
81.58	0.01\\
81.59	0.01\\
81.6	0.01\\
81.61	0.01\\
81.62	0.01\\
81.63	0.01\\
81.64	0.01\\
81.65	0.01\\
81.66	0.01\\
81.67	0.01\\
81.68	0.01\\
81.69	0.01\\
81.7	0.01\\
81.71	0.01\\
81.72	0.01\\
81.73	0.01\\
81.74	0.01\\
81.75	0.01\\
81.76	0.01\\
81.77	0.01\\
81.78	0.01\\
81.79	0.01\\
81.8	0.01\\
81.81	0.01\\
81.82	0.01\\
81.83	0.01\\
81.84	0.01\\
81.85	0.01\\
81.86	0.01\\
81.87	0.01\\
81.88	0.01\\
81.89	0.01\\
81.9	0.01\\
81.91	0.01\\
81.92	0.01\\
81.93	0.01\\
81.94	0.01\\
81.95	0.01\\
81.96	0.01\\
81.97	0.01\\
81.98	0.01\\
81.99	0.01\\
82	0.01\\
82.01	0.01\\
82.02	0.01\\
82.03	0.01\\
82.04	0.01\\
82.05	0.01\\
82.06	0.01\\
82.07	0.01\\
82.08	0.01\\
82.09	0.01\\
82.1	0.01\\
82.11	0.01\\
82.12	0.01\\
82.13	0.01\\
82.14	0.01\\
82.15	0.01\\
82.16	0.01\\
82.17	0.01\\
82.18	0.01\\
82.19	0.01\\
82.2	0.01\\
82.21	0.01\\
82.22	0.01\\
82.23	0.01\\
82.24	0.01\\
82.25	0.01\\
82.26	0.01\\
82.27	0.01\\
82.28	0.01\\
82.29	0.01\\
82.3	0.01\\
82.31	0.01\\
82.32	0.01\\
82.33	0.01\\
82.34	0.01\\
82.35	0.01\\
82.36	0.01\\
82.37	0.01\\
82.38	0.01\\
82.39	0.01\\
82.4	0.01\\
82.41	0.01\\
82.42	0.01\\
82.43	0.01\\
82.44	0.01\\
82.45	0.01\\
82.46	0.01\\
82.47	0.01\\
82.48	0.01\\
82.49	0.01\\
82.5	0.01\\
82.51	0.01\\
82.52	0.01\\
82.53	0.01\\
82.54	0.01\\
82.55	0.01\\
82.56	0.01\\
82.57	0.01\\
82.58	0.01\\
82.59	0.01\\
82.6	0.01\\
82.61	0.01\\
82.62	0.01\\
82.63	0.01\\
82.64	0.01\\
82.65	0.01\\
82.66	0.01\\
82.67	0.01\\
82.68	0.01\\
82.69	0.01\\
82.7	0.01\\
82.71	0.01\\
82.72	0.01\\
82.73	0.01\\
82.74	0.01\\
82.75	0.01\\
82.76	0.01\\
82.77	0.01\\
82.78	0.01\\
82.79	0.01\\
82.8	0.01\\
82.81	0.01\\
82.82	0.01\\
82.83	0.01\\
82.84	0.01\\
82.85	0.01\\
82.86	0.01\\
82.87	0.01\\
82.88	0.01\\
82.89	0.01\\
82.9	0.01\\
82.91	0.01\\
82.92	0.01\\
82.93	0.01\\
82.94	0.01\\
82.95	0.01\\
82.96	0.01\\
82.97	0.01\\
82.98	0.01\\
82.99	0.01\\
83	0.01\\
83.01	0.01\\
83.02	0.01\\
83.03	0.01\\
83.04	0.01\\
83.05	0.01\\
83.06	0.01\\
83.07	0.01\\
83.08	0.01\\
83.09	0.01\\
83.1	0.01\\
83.11	0.01\\
83.12	0.01\\
83.13	0.01\\
83.14	0.01\\
83.15	0.01\\
83.16	0.01\\
83.17	0.01\\
83.18	0.01\\
83.19	0.01\\
83.2	0.01\\
83.21	0.01\\
83.22	0.01\\
83.23	0.01\\
83.24	0.01\\
83.25	0.01\\
83.26	0.01\\
83.27	0.01\\
83.28	0.01\\
83.29	0.01\\
83.3	0.01\\
83.31	0.01\\
83.32	0.01\\
83.33	0.01\\
83.34	0.01\\
83.35	0.01\\
83.36	0.01\\
83.37	0.01\\
83.38	0.01\\
83.39	0.01\\
83.4	0.01\\
83.41	0.01\\
83.42	0.01\\
83.43	0.01\\
83.44	0.01\\
83.45	0.01\\
83.46	0.01\\
83.47	0.01\\
83.48	0.01\\
83.49	0.01\\
83.5	0.01\\
83.51	0.01\\
83.52	0.01\\
83.53	0.01\\
83.54	0.01\\
83.55	0.01\\
83.56	0.01\\
83.57	0.01\\
83.58	0.01\\
83.59	0.01\\
83.6	0.01\\
83.61	0.01\\
83.62	0.01\\
83.63	0.01\\
83.64	0.01\\
83.65	0.01\\
83.66	0.01\\
83.67	0.01\\
83.68	0.01\\
83.69	0.01\\
83.7	0.01\\
83.71	0.01\\
83.72	0.01\\
83.73	0.01\\
83.74	0.01\\
83.75	0.01\\
83.76	0.01\\
83.77	0.01\\
83.78	0.01\\
83.79	0.01\\
83.8	0.01\\
83.81	0.01\\
83.82	0.01\\
83.83	0.01\\
83.84	0.01\\
83.85	0.01\\
83.86	0.01\\
83.87	0.01\\
83.88	0.01\\
83.89	0.01\\
83.9	0.01\\
83.91	0.01\\
83.92	0.01\\
83.93	0.01\\
83.94	0.01\\
83.95	0.01\\
83.96	0.01\\
83.97	0.01\\
83.98	0.01\\
83.99	0.01\\
84	0.01\\
84.01	0.01\\
84.02	0.01\\
84.03	0.01\\
84.04	0.01\\
84.05	0.01\\
84.06	0.01\\
84.07	0.01\\
84.08	0.01\\
84.09	0.01\\
84.1	0.01\\
84.11	0.01\\
84.12	0.01\\
84.13	0.01\\
84.14	0.01\\
84.15	0.01\\
84.16	0.01\\
84.17	0.01\\
84.18	0.01\\
84.19	0.01\\
84.2	0.01\\
84.21	0.01\\
84.22	0.01\\
84.23	0.01\\
84.24	0.01\\
84.25	0.01\\
84.26	0.01\\
84.27	0.01\\
84.28	0.01\\
84.29	0.01\\
84.3	0.01\\
84.31	0.01\\
84.32	0.01\\
84.33	0.01\\
84.34	0.01\\
84.35	0.01\\
84.36	0.01\\
84.37	0.01\\
84.38	0.01\\
84.39	0.01\\
84.4	0.01\\
84.41	0.01\\
84.42	0.01\\
84.43	0.01\\
84.44	0.01\\
84.45	0.01\\
84.46	0.01\\
84.47	0.01\\
84.48	0.01\\
84.49	0.01\\
84.5	0.01\\
84.51	0.01\\
84.52	0.01\\
84.53	0.01\\
84.54	0.01\\
84.55	0.01\\
84.56	0.01\\
84.57	0.01\\
84.58	0.01\\
84.59	0.01\\
84.6	0.01\\
84.61	0.01\\
84.62	0.01\\
84.63	0.01\\
84.64	0.01\\
84.65	0.01\\
84.66	0.01\\
84.67	0.01\\
84.68	0.01\\
84.69	0.01\\
84.7	0.01\\
84.71	0.01\\
84.72	0.01\\
84.73	0.01\\
84.74	0.01\\
84.75	0.01\\
84.76	0.01\\
84.77	0.01\\
84.78	0.01\\
84.79	0.01\\
84.8	0.01\\
84.81	0.01\\
84.82	0.01\\
84.83	0.01\\
84.84	0.01\\
84.85	0.01\\
84.86	0.01\\
84.87	0.01\\
84.88	0.01\\
84.89	0.01\\
84.9	0.01\\
84.91	0.01\\
84.92	0.01\\
84.93	0.01\\
84.94	0.01\\
84.95	0.01\\
84.96	0.01\\
84.97	0.01\\
84.98	0.01\\
84.99	0.01\\
85	0.01\\
85.01	0.01\\
85.02	0.01\\
85.03	0.01\\
85.04	0.01\\
85.05	0.01\\
85.06	0.01\\
85.07	0.01\\
85.08	0.01\\
85.09	0.01\\
85.1	0.01\\
85.11	0.01\\
85.12	0.01\\
85.13	0.01\\
85.14	0.01\\
85.15	0.01\\
85.16	0.01\\
85.17	0.01\\
85.18	0.01\\
85.19	0.01\\
85.2	0.01\\
85.21	0.01\\
85.22	0.01\\
85.23	0.01\\
85.24	0.01\\
85.25	0.01\\
85.26	0.01\\
85.27	0.01\\
85.28	0.01\\
85.29	0.01\\
85.3	0.01\\
85.31	0.01\\
85.32	0.01\\
85.33	0.01\\
85.34	0.01\\
85.35	0.01\\
85.36	0.01\\
85.37	0.01\\
85.38	0.01\\
85.39	0.01\\
85.4	0.01\\
85.41	0.01\\
85.42	0.01\\
85.43	0.01\\
85.44	0.01\\
85.45	0.01\\
85.46	0.01\\
85.47	0.01\\
85.48	0.01\\
85.49	0.01\\
85.5	0.01\\
85.51	0.01\\
85.52	0.01\\
85.53	0.01\\
85.54	0.01\\
85.55	0.01\\
85.56	0.01\\
85.57	0.01\\
85.58	0.01\\
85.59	0.01\\
85.6	0.01\\
85.61	0.01\\
85.62	0.01\\
85.63	0.01\\
85.64	0.01\\
85.65	0.01\\
85.66	0.01\\
85.67	0.01\\
85.68	0.01\\
85.69	0.01\\
85.7	0.01\\
85.71	0.01\\
85.72	0.01\\
85.73	0.01\\
85.74	0.01\\
85.75	0.01\\
85.76	0.01\\
85.77	0.01\\
85.78	0.01\\
85.79	0.01\\
85.8	0.01\\
85.81	0.01\\
85.82	0.01\\
85.83	0.01\\
85.84	0.01\\
85.85	0.01\\
85.86	0.01\\
85.87	0.01\\
85.88	0.01\\
85.89	0.01\\
85.9	0.01\\
85.91	0.01\\
85.92	0.01\\
85.93	0.01\\
85.94	0.01\\
85.95	0.01\\
85.96	0.01\\
85.97	0.01\\
85.98	0.01\\
85.99	0.01\\
86	0.01\\
86.01	0.01\\
86.02	0.01\\
86.03	0.01\\
86.04	0.01\\
86.05	0.01\\
86.06	0.01\\
86.07	0.01\\
86.08	0.01\\
86.09	0.01\\
86.1	0.01\\
86.11	0.01\\
86.12	0.01\\
86.13	0.01\\
86.14	0.01\\
86.15	0.01\\
86.16	0.01\\
86.17	0.01\\
86.18	0.01\\
86.19	0.01\\
86.2	0.01\\
86.21	0.01\\
86.22	0.01\\
86.23	0.01\\
86.24	0.01\\
86.25	0.01\\
86.26	0.01\\
86.27	0.01\\
86.28	0.01\\
86.29	0.01\\
86.3	0.01\\
86.31	0.01\\
86.32	0.01\\
86.33	0.01\\
86.34	0.01\\
86.35	0.01\\
86.36	0.01\\
86.37	0.01\\
86.38	0.01\\
86.39	0.01\\
86.4	0.01\\
86.41	0.01\\
86.42	0.01\\
86.43	0.01\\
86.44	0.01\\
86.45	0.01\\
86.46	0.01\\
86.47	0.01\\
86.48	0.01\\
86.49	0.01\\
86.5	0.01\\
86.51	0.01\\
86.52	0.01\\
86.53	0.01\\
86.54	0.01\\
86.55	0.01\\
86.56	0.01\\
86.57	0.01\\
86.58	0.01\\
86.59	0.01\\
86.6	0.01\\
86.61	0.01\\
86.62	0.01\\
86.63	0.01\\
86.64	0.01\\
86.65	0.01\\
86.66	0.01\\
86.67	0.01\\
86.68	0.01\\
86.69	0.01\\
86.7	0.01\\
86.71	0.01\\
86.72	0.01\\
86.73	0.01\\
86.74	0.01\\
86.75	0.01\\
86.76	0.01\\
86.77	0.01\\
86.78	0.01\\
86.79	0.01\\
86.8	0.01\\
86.81	0.01\\
86.82	0.01\\
86.83	0.01\\
86.84	0.01\\
86.85	0.01\\
86.86	0.01\\
86.87	0.01\\
86.88	0.01\\
86.89	0.01\\
86.9	0.01\\
86.91	0.01\\
86.92	0.01\\
86.93	0.01\\
86.94	0.01\\
86.95	0.01\\
86.96	0.01\\
86.97	0.01\\
86.98	0.01\\
86.99	0.01\\
87	0.01\\
87.01	0.01\\
87.02	0.01\\
87.03	0.01\\
87.04	0.01\\
87.05	0.01\\
87.06	0.01\\
87.07	0.01\\
87.08	0.01\\
87.09	0.01\\
87.1	0.01\\
87.11	0.01\\
87.12	0.01\\
87.13	0.01\\
87.14	0.01\\
87.15	0.01\\
87.16	0.01\\
87.17	0.01\\
87.18	0.01\\
87.19	0.01\\
87.2	0.01\\
87.21	0.01\\
87.22	0.01\\
87.23	0.01\\
87.24	0.01\\
87.25	0.01\\
87.26	0.01\\
87.27	0.01\\
87.28	0.01\\
87.29	0.01\\
87.3	0.01\\
87.31	0.01\\
87.32	0.01\\
87.33	0.01\\
87.34	0.01\\
87.35	0.01\\
87.36	0.01\\
87.37	0.01\\
87.38	0.01\\
87.39	0.01\\
87.4	0.01\\
87.41	0.01\\
87.42	0.01\\
87.43	0.01\\
87.44	0.01\\
87.45	0.01\\
87.46	0.01\\
87.47	0.01\\
87.48	0.01\\
87.49	0.01\\
87.5	0.01\\
87.51	0.01\\
87.52	0.01\\
87.53	0.01\\
87.54	0.01\\
87.55	0.01\\
87.56	0.01\\
87.57	0.01\\
87.58	0.01\\
87.59	0.01\\
87.6	0.01\\
87.61	0.01\\
87.62	0.01\\
87.63	0.01\\
87.64	0.01\\
87.65	0.01\\
87.66	0.01\\
87.67	0.01\\
87.68	0.01\\
87.69	0.01\\
87.7	0.01\\
87.71	0.01\\
87.72	0.01\\
87.73	0.01\\
87.74	0.01\\
87.75	0.01\\
87.76	0.01\\
87.77	0.01\\
87.78	0.01\\
87.79	0.01\\
87.8	0.01\\
87.81	0.01\\
87.82	0.01\\
87.83	0.01\\
87.84	0.01\\
87.85	0.01\\
87.86	0.01\\
87.87	0.01\\
87.88	0.01\\
87.89	0.01\\
87.9	0.01\\
87.91	0.01\\
87.92	0.01\\
87.93	0.01\\
87.94	0.01\\
87.95	0.01\\
87.96	0.01\\
87.97	0.01\\
87.98	0.01\\
87.99	0.01\\
88	0.01\\
88.01	0.01\\
88.02	0.01\\
88.03	0.01\\
88.04	0.01\\
88.05	0.01\\
88.06	0.01\\
88.07	0.01\\
88.08	0.01\\
88.09	0.01\\
88.1	0.01\\
88.11	0.01\\
88.12	0.01\\
88.13	0.01\\
88.14	0.01\\
88.15	0.01\\
88.16	0.01\\
88.17	0.01\\
88.18	0.01\\
88.19	0.01\\
88.2	0.01\\
88.21	0.01\\
88.22	0.01\\
88.23	0.01\\
88.24	0.01\\
88.25	0.01\\
88.26	0.01\\
88.27	0.01\\
88.28	0.01\\
88.29	0.01\\
88.3	0.01\\
88.31	0.01\\
88.32	0.01\\
88.33	0.01\\
88.34	0.01\\
88.35	0.01\\
88.36	0.01\\
88.37	0.01\\
88.38	0.01\\
88.39	0.01\\
88.4	0.01\\
88.41	0.01\\
88.42	0.01\\
88.43	0.01\\
88.44	0.01\\
88.45	0.01\\
88.46	0.01\\
88.47	0.01\\
88.48	0.01\\
88.49	0.01\\
88.5	0.01\\
88.51	0.01\\
88.52	0.01\\
88.53	0.01\\
88.54	0.01\\
88.55	0.01\\
88.56	0.01\\
88.57	0.01\\
88.58	0.01\\
88.59	0.01\\
88.6	0.01\\
88.61	0.01\\
88.62	0.01\\
88.63	0.01\\
88.64	0.01\\
88.65	0.01\\
88.66	0.01\\
88.67	0.01\\
88.68	0.01\\
88.69	0.01\\
88.7	0.01\\
88.71	0.01\\
88.72	0.01\\
88.73	0.01\\
88.74	0.01\\
88.75	0.01\\
88.76	0.01\\
88.77	0.01\\
88.78	0.01\\
88.79	0.01\\
88.8	0.01\\
88.81	0.01\\
88.82	0.01\\
88.83	0.01\\
88.84	0.01\\
88.85	0.01\\
88.86	0.01\\
88.87	0.01\\
88.88	0.01\\
88.89	0.01\\
88.9	0.01\\
88.91	0.01\\
88.92	0.01\\
88.93	0.01\\
88.94	0.01\\
88.95	0.01\\
88.96	0.01\\
88.97	0.01\\
88.98	0.01\\
88.99	0.01\\
89	0.01\\
89.01	0.01\\
89.02	0.01\\
89.03	0.01\\
89.04	0.01\\
89.05	0.01\\
89.06	0.01\\
89.07	0.01\\
89.08	0.01\\
89.09	0.01\\
89.1	0.01\\
89.11	0.01\\
89.12	0.01\\
89.13	0.01\\
89.14	0.01\\
89.15	0.01\\
89.16	0.01\\
89.17	0.01\\
89.18	0.01\\
89.19	0.01\\
89.2	0.01\\
89.21	0.01\\
89.22	0.01\\
89.23	0.01\\
89.24	0.01\\
89.25	0.01\\
89.26	0.01\\
89.27	0.01\\
89.28	0.01\\
89.29	0.01\\
89.3	0.01\\
89.31	0.01\\
89.32	0.01\\
89.33	0.01\\
89.34	0.01\\
89.35	0.01\\
89.36	0.01\\
89.37	0.01\\
89.38	0.01\\
89.39	0.01\\
89.4	0.01\\
89.41	0.01\\
89.42	0.01\\
89.43	0.01\\
89.44	0.01\\
89.45	0.01\\
89.46	0.01\\
89.47	0.01\\
89.48	0.01\\
89.49	0.01\\
89.5	0.01\\
89.51	0.01\\
89.52	0.01\\
89.53	0.01\\
89.54	0.01\\
89.55	0.01\\
89.56	0.01\\
89.57	0.01\\
89.58	0.01\\
89.59	0.01\\
89.6	0.01\\
89.61	0.01\\
89.62	0.01\\
89.63	0.01\\
89.64	0.01\\
89.65	0.01\\
89.66	0.01\\
89.67	0.01\\
89.68	0.01\\
89.69	0.01\\
89.7	0.01\\
89.71	0.01\\
89.72	0.01\\
89.73	0.01\\
89.74	0.01\\
89.75	0.01\\
89.76	0.01\\
89.77	0.01\\
89.78	0.01\\
89.79	0.01\\
89.8	0.01\\
89.81	0.01\\
89.82	0.01\\
89.83	0.01\\
89.84	0.01\\
89.85	0.01\\
89.86	0.01\\
89.87	0.01\\
89.88	0.01\\
89.89	0.01\\
89.9	0.01\\
89.91	0.01\\
89.92	0.01\\
89.93	0.01\\
89.94	0.01\\
89.95	0.01\\
89.96	0.01\\
89.97	0.01\\
89.98	0.01\\
89.99	0.01\\
90	0.01\\
90.01	0.01\\
90.02	0.01\\
90.03	0.01\\
90.04	0.01\\
90.05	0.01\\
90.06	0.01\\
90.07	0.01\\
90.08	0.01\\
90.09	0.01\\
90.1	0.01\\
90.11	0.01\\
90.12	0.01\\
90.13	0.01\\
90.14	0.01\\
90.15	0.01\\
90.16	0.01\\
90.17	0.01\\
90.18	0.01\\
90.19	0.01\\
90.2	0.01\\
90.21	0.01\\
90.22	0.01\\
90.23	0.01\\
90.24	0.01\\
90.25	0.01\\
90.26	0.01\\
90.27	0.01\\
90.28	0.01\\
90.29	0.01\\
90.3	0.01\\
90.31	0.01\\
90.32	0.01\\
90.33	0.01\\
90.34	0.01\\
90.35	0.01\\
90.36	0.01\\
90.37	0.01\\
90.38	0.01\\
90.39	0.01\\
90.4	0.01\\
90.41	0.01\\
90.42	0.01\\
90.43	0.01\\
90.44	0.01\\
90.45	0.01\\
90.46	0.01\\
90.47	0.01\\
90.48	0.01\\
90.49	0.01\\
90.5	0.01\\
90.51	0.01\\
90.52	0.01\\
90.53	0.01\\
90.54	0.01\\
90.55	0.01\\
90.56	0.01\\
90.57	0.01\\
90.58	0.01\\
90.59	0.01\\
90.6	0.01\\
90.61	0.01\\
90.62	0.01\\
90.63	0.01\\
90.64	0.01\\
90.65	0.01\\
90.66	0.01\\
90.67	0.01\\
90.68	0.01\\
90.69	0.01\\
90.7	0.01\\
90.71	0.01\\
90.72	0.01\\
90.73	0.01\\
90.74	0.01\\
90.75	0.01\\
90.76	0.01\\
90.77	0.01\\
90.78	0.01\\
90.79	0.01\\
90.8	0.01\\
90.81	0.01\\
90.82	0.01\\
90.83	0.01\\
90.84	0.01\\
90.85	0.01\\
90.86	0.01\\
90.87	0.01\\
90.88	0.01\\
90.89	0.01\\
90.9	0.01\\
90.91	0.01\\
90.92	0.01\\
90.93	0.01\\
90.94	0.01\\
90.95	0.01\\
90.96	0.01\\
90.97	0.01\\
90.98	0.01\\
90.99	0.01\\
91	0.01\\
91.01	0.01\\
91.02	0.01\\
91.03	0.01\\
91.04	0.01\\
91.05	0.01\\
91.06	0.01\\
91.07	0.01\\
91.08	0.01\\
91.09	0.01\\
91.1	0.01\\
91.11	0.01\\
91.12	0.01\\
91.13	0.01\\
91.14	0.01\\
91.15	0.01\\
91.16	0.01\\
91.17	0.01\\
91.18	0.01\\
91.19	0.01\\
91.2	0.01\\
91.21	0.01\\
91.22	0.01\\
91.23	0.01\\
91.24	0.01\\
91.25	0.01\\
91.26	0.01\\
91.27	0.01\\
91.28	0.01\\
91.29	0.01\\
91.3	0.01\\
91.31	0.01\\
91.32	0.01\\
91.33	0.01\\
91.34	0.01\\
91.35	0.01\\
91.36	0.01\\
91.37	0.01\\
91.38	0.01\\
91.39	0.01\\
91.4	0.01\\
91.41	0.01\\
91.42	0.01\\
91.43	0.01\\
91.44	0.01\\
91.45	0.01\\
91.46	0.01\\
91.47	0.01\\
91.48	0.01\\
91.49	0.01\\
91.5	0.01\\
91.51	0.01\\
91.52	0.01\\
91.53	0.01\\
91.54	0.01\\
91.55	0.01\\
91.56	0.01\\
91.57	0.01\\
91.58	0.01\\
91.59	0.01\\
91.6	0.01\\
91.61	0.01\\
91.62	0.01\\
91.63	0.01\\
91.64	0.01\\
91.65	0.01\\
91.66	0.01\\
91.67	0.01\\
91.68	0.01\\
91.69	0.01\\
91.7	0.01\\
91.71	0.01\\
91.72	0.01\\
91.73	0.01\\
91.74	0.01\\
91.75	0.01\\
91.76	0.01\\
91.77	0.01\\
91.78	0.01\\
91.79	0.01\\
91.8	0.01\\
91.81	0.01\\
91.82	0.01\\
91.83	0.01\\
91.84	0.01\\
91.85	0.01\\
91.86	0.01\\
91.87	0.01\\
91.88	0.01\\
91.89	0.01\\
91.9	0.01\\
91.91	0.01\\
91.92	0.01\\
91.93	0.01\\
91.94	0.01\\
91.95	0.01\\
91.96	0.01\\
91.97	0.01\\
91.98	0.01\\
91.99	0.01\\
92	0.01\\
92.01	0.01\\
92.02	0.01\\
92.03	0.01\\
92.04	0.01\\
92.05	0.01\\
92.06	0.01\\
92.07	0.01\\
92.08	0.01\\
92.09	0.01\\
92.1	0.01\\
92.11	0.01\\
92.12	0.01\\
92.13	0.01\\
92.14	0.01\\
92.15	0.01\\
92.16	0.01\\
92.17	0.01\\
92.18	0.01\\
92.19	0.01\\
92.2	0.01\\
92.21	0.01\\
92.22	0.01\\
92.23	0.01\\
92.24	0.01\\
92.25	0.01\\
92.26	0.01\\
92.27	0.01\\
92.28	0.01\\
92.29	0.01\\
92.3	0.01\\
92.31	0.01\\
92.32	0.01\\
92.33	0.01\\
92.34	0.01\\
92.35	0.01\\
92.36	0.01\\
92.37	0.01\\
92.38	0.01\\
92.39	0.01\\
92.4	0.01\\
92.41	0.01\\
92.42	0.01\\
92.43	0.01\\
92.44	0.01\\
92.45	0.01\\
92.46	0.01\\
92.47	0.01\\
92.48	0.01\\
92.49	0.01\\
92.5	0.01\\
92.51	0.01\\
92.52	0.01\\
92.53	0.01\\
92.54	0.01\\
92.55	0.01\\
92.56	0.01\\
92.57	0.01\\
92.58	0.01\\
92.59	0.01\\
92.6	0.01\\
92.61	0.01\\
92.62	0.01\\
92.63	0.01\\
92.64	0.01\\
92.65	0.01\\
92.66	0.01\\
92.67	0.01\\
92.68	0.01\\
92.69	0.01\\
92.7	0.01\\
92.71	0.01\\
92.72	0.01\\
92.73	0.01\\
92.74	0.01\\
92.75	0.01\\
92.76	0.01\\
92.77	0.01\\
92.78	0.01\\
92.79	0.01\\
92.8	0.01\\
92.81	0.01\\
92.82	0.01\\
92.83	0.01\\
92.84	0.01\\
92.85	0.01\\
92.86	0.01\\
92.87	0.01\\
92.88	0.01\\
92.89	0.01\\
92.9	0.01\\
92.91	0.01\\
92.92	0.01\\
92.93	0.01\\
92.94	0.01\\
92.95	0.01\\
92.96	0.01\\
92.97	0.01\\
92.98	0.01\\
92.99	0.01\\
93	0.01\\
93.01	0.01\\
93.02	0.01\\
93.03	0.01\\
93.04	0.01\\
93.05	0.01\\
93.06	0.01\\
93.07	0.01\\
93.08	0.01\\
93.09	0.01\\
93.1	0.01\\
93.11	0.01\\
93.12	0.01\\
93.13	0.01\\
93.14	0.01\\
93.15	0.01\\
93.16	0.01\\
93.17	0.01\\
93.18	0.01\\
93.19	0.01\\
93.2	0.01\\
93.21	0.01\\
93.22	0.01\\
93.23	0.01\\
93.24	0.01\\
93.25	0.01\\
93.26	0.01\\
93.27	0.01\\
93.28	0.01\\
93.29	0.01\\
93.3	0.01\\
93.31	0.01\\
93.32	0.01\\
93.33	0.01\\
93.34	0.01\\
93.35	0.01\\
93.36	0.01\\
93.37	0.01\\
93.38	0.01\\
93.39	0.01\\
93.4	0.01\\
93.41	0.01\\
93.42	0.01\\
93.43	0.01\\
93.44	0.01\\
93.45	0.01\\
93.46	0.01\\
93.47	0.01\\
93.48	0.01\\
93.49	0.01\\
93.5	0.01\\
93.51	0.01\\
93.52	0.01\\
93.53	0.01\\
93.54	0.01\\
93.55	0.01\\
93.56	0.01\\
93.57	0.01\\
93.58	0.01\\
93.59	0.01\\
93.6	0.01\\
93.61	0.01\\
93.62	0.01\\
93.63	0.01\\
93.64	0.01\\
93.65	0.01\\
93.66	0.01\\
93.67	0.01\\
93.68	0.01\\
93.69	0.01\\
93.7	0.01\\
93.71	0.01\\
93.72	0.01\\
93.73	0.01\\
93.74	0.01\\
93.75	0.01\\
93.76	0.01\\
93.77	0.01\\
93.78	0.01\\
93.79	0.01\\
93.8	0.01\\
93.81	0.01\\
93.82	0.01\\
93.83	0.01\\
93.84	0.01\\
93.85	0.01\\
93.86	0.01\\
93.87	0.01\\
93.88	0.01\\
93.89	0.01\\
93.9	0.01\\
93.91	0.01\\
93.92	0.01\\
93.93	0.01\\
93.94	0.01\\
93.95	0.01\\
93.96	0.01\\
93.97	0.01\\
93.98	0.01\\
93.99	0.01\\
94	0.01\\
94.01	0.01\\
94.02	0.01\\
94.03	0.01\\
94.04	0.01\\
94.05	0.01\\
94.06	0.01\\
94.07	0.01\\
94.08	0.01\\
94.09	0.01\\
94.1	0.01\\
94.11	0.01\\
94.12	0.01\\
94.13	0.01\\
94.14	0.01\\
94.15	0.01\\
94.16	0.01\\
94.17	0.01\\
94.18	0.01\\
94.19	0.01\\
94.2	0.01\\
94.21	0.01\\
94.22	0.01\\
94.23	0.01\\
94.24	0.01\\
94.25	0.01\\
94.26	0.01\\
94.27	0.01\\
94.28	0.01\\
94.29	0.01\\
94.3	0.01\\
94.31	0.01\\
94.32	0.01\\
94.33	0.01\\
94.34	0.01\\
94.35	0.01\\
94.36	0.01\\
94.37	0.01\\
94.38	0.01\\
94.39	0.01\\
94.4	0.01\\
94.41	0.01\\
94.42	0.01\\
94.43	0.01\\
94.44	0.01\\
94.45	0.01\\
94.46	0.01\\
94.47	0.01\\
94.48	0.01\\
94.49	0.01\\
94.5	0.01\\
94.51	0.01\\
94.52	0.01\\
94.53	0.01\\
94.54	0.01\\
94.55	0.01\\
94.56	0.01\\
94.57	0.01\\
94.58	0.01\\
94.59	0.01\\
94.6	0.01\\
94.61	0.01\\
94.62	0.01\\
94.63	0.01\\
94.64	0.01\\
94.65	0.01\\
94.66	0.01\\
94.67	0.01\\
94.68	0.01\\
94.69	0.01\\
94.7	0.01\\
94.71	0.01\\
94.72	0.01\\
94.73	0.01\\
94.74	0.01\\
94.75	0.01\\
94.76	0.01\\
94.77	0.01\\
94.78	0.01\\
94.79	0.01\\
94.8	0.01\\
94.81	0.01\\
94.82	0.01\\
94.83	0.01\\
94.84	0.01\\
94.85	0.01\\
94.86	0.01\\
94.87	0.01\\
94.88	0.01\\
94.89	0.01\\
94.9	0.01\\
94.91	0.01\\
94.92	0.01\\
94.93	0.01\\
94.94	0.01\\
94.95	0.01\\
94.96	0.01\\
94.97	0.01\\
94.98	0.01\\
94.99	0.01\\
95	0.01\\
95.01	0.01\\
95.02	0.01\\
95.03	0.01\\
95.04	0.01\\
95.05	0.01\\
95.06	0.01\\
95.07	0.01\\
95.08	0.01\\
95.09	0.01\\
95.1	0.01\\
95.11	0.01\\
95.12	0.01\\
95.13	0.01\\
95.14	0.01\\
95.15	0.01\\
95.16	0.01\\
95.17	0.01\\
95.18	0.01\\
95.19	0.01\\
95.2	0.01\\
95.21	0.01\\
95.22	0.01\\
95.23	0.01\\
95.24	0.01\\
95.25	0.01\\
95.26	0.01\\
95.27	0.01\\
95.28	0.01\\
95.29	0.01\\
95.3	0.01\\
95.31	0.01\\
95.32	0.01\\
95.33	0.01\\
95.34	0.01\\
95.35	0.01\\
95.36	0.01\\
95.37	0.01\\
95.38	0.01\\
95.39	0.01\\
95.4	0.01\\
95.41	0.01\\
95.42	0.01\\
95.43	0.01\\
95.44	0.01\\
95.45	0.01\\
95.46	0.01\\
95.47	0.01\\
95.48	0.01\\
95.49	0.01\\
95.5	0.01\\
95.51	0.01\\
95.52	0.01\\
95.53	0.01\\
95.54	0.01\\
95.55	0.01\\
95.56	0.01\\
95.57	0.01\\
95.58	0.01\\
95.59	0.01\\
95.6	0.01\\
95.61	0.01\\
95.62	0.01\\
95.63	0.01\\
95.64	0.01\\
95.65	0.01\\
95.66	0.01\\
95.67	0.01\\
95.68	0.01\\
95.69	0.01\\
95.7	0.01\\
95.71	0.01\\
95.72	0.01\\
95.73	0.01\\
95.74	0.01\\
95.75	0.01\\
95.76	0.01\\
95.77	0.01\\
95.78	0.01\\
95.79	0.01\\
95.8	0.01\\
95.81	0.01\\
95.82	0.01\\
95.83	0.01\\
95.84	0.01\\
95.85	0.01\\
95.86	0.01\\
95.87	0.01\\
95.88	0.01\\
95.89	0.01\\
95.9	0.01\\
95.91	0.01\\
95.92	0.01\\
95.93	0.01\\
95.94	0.01\\
95.95	0.01\\
95.96	0.01\\
95.97	0.01\\
95.98	0.01\\
95.99	0.01\\
96	0.01\\
96.01	0.01\\
96.02	0.01\\
96.03	0.01\\
96.04	0.01\\
96.05	0.01\\
96.06	0.01\\
96.07	0.01\\
96.08	0.01\\
96.09	0.01\\
96.1	0.01\\
96.11	0.01\\
96.12	0.01\\
96.13	0.01\\
96.14	0.01\\
96.15	0.01\\
96.16	0.01\\
96.17	0.01\\
96.18	0.01\\
96.19	0.01\\
96.2	0.01\\
96.21	0.01\\
96.22	0.01\\
96.23	0.01\\
96.24	0.01\\
96.25	0.01\\
96.26	0.01\\
96.27	0.01\\
96.28	0.01\\
96.29	0.01\\
96.3	0.01\\
96.31	0.01\\
96.32	0.01\\
96.33	0.01\\
96.34	0.01\\
96.35	0.01\\
96.36	0.01\\
96.37	0.01\\
96.38	0.01\\
96.39	0.01\\
96.4	0.01\\
96.41	0.01\\
96.42	0.01\\
96.43	0.01\\
96.44	0.01\\
96.45	0.01\\
96.46	0.01\\
96.47	0.01\\
96.48	0.01\\
96.49	0.01\\
96.5	0.01\\
96.51	0.01\\
96.52	0.01\\
96.53	0.01\\
96.54	0.01\\
96.55	0.01\\
96.56	0.01\\
96.57	0.01\\
96.58	0.01\\
96.59	0.01\\
96.6	0.01\\
96.61	0.01\\
96.62	0.01\\
96.63	0.01\\
96.64	0.01\\
96.65	0.01\\
96.66	0.01\\
96.67	0.01\\
96.68	0.01\\
96.69	0.01\\
96.7	0.01\\
96.71	0.01\\
96.72	0.01\\
96.73	0.01\\
96.74	0.01\\
96.75	0.01\\
96.76	0.01\\
96.77	0.01\\
96.78	0.01\\
96.79	0.01\\
96.8	0.01\\
96.81	0.01\\
96.82	0.01\\
96.83	0.01\\
96.84	0.01\\
96.85	0.01\\
96.86	0.01\\
96.87	0.01\\
96.88	0.01\\
96.89	0.01\\
96.9	0.01\\
96.91	0.01\\
96.92	0.01\\
96.93	0.01\\
96.94	0.01\\
96.95	0.01\\
96.96	0.01\\
96.97	0.01\\
96.98	0.01\\
96.99	0.01\\
97	0.01\\
97.01	0.01\\
97.02	0.01\\
97.03	0.01\\
97.04	0.01\\
97.05	0.01\\
97.06	0.01\\
97.07	0.01\\
97.08	0.01\\
97.09	0.01\\
97.1	0.01\\
97.11	0.01\\
97.12	0.01\\
97.13	0.01\\
97.14	0.01\\
97.15	0.01\\
97.16	0.01\\
97.17	0.01\\
97.18	0.01\\
97.19	0.01\\
97.2	0.01\\
97.21	0.01\\
97.22	0.01\\
97.23	0.01\\
97.24	0.01\\
97.25	0.01\\
97.26	0.01\\
97.27	0.01\\
97.28	0.01\\
97.29	0.01\\
97.3	0.01\\
97.31	0.01\\
97.32	0.01\\
97.33	0.01\\
97.34	0.01\\
97.35	0.01\\
97.36	0.01\\
97.37	0.01\\
97.38	0.01\\
97.39	0.01\\
97.4	0.01\\
97.41	0.01\\
97.42	0.01\\
97.43	0.01\\
97.44	0.01\\
97.45	0.01\\
97.46	0.01\\
97.47	0.01\\
97.48	0.01\\
97.49	0.01\\
97.5	0.01\\
97.51	0.01\\
97.52	0.01\\
97.53	0.01\\
97.54	0.01\\
97.55	0.01\\
97.56	0.01\\
97.57	0.01\\
97.58	0.01\\
97.59	0.01\\
97.6	0.01\\
97.61	0.01\\
97.62	0.01\\
97.63	0.01\\
97.64	0.01\\
97.65	0.01\\
97.66	0.01\\
97.67	0.01\\
97.68	0.01\\
97.69	0.01\\
97.7	0.01\\
97.71	0.01\\
97.72	0.01\\
97.73	0.01\\
97.74	0.01\\
97.75	0.01\\
97.76	0.01\\
97.77	0.01\\
97.78	0.01\\
97.79	0.01\\
97.8	0.01\\
97.81	0.01\\
97.82	0.01\\
97.83	0.01\\
97.84	0.01\\
97.85	0.01\\
97.86	0.01\\
97.87	0.01\\
97.88	0.01\\
97.89	0.01\\
97.9	0.01\\
97.91	0.01\\
97.92	0.01\\
97.93	0.01\\
97.94	0.01\\
97.95	0.01\\
97.96	0.01\\
97.97	0.01\\
97.98	0.01\\
97.99	0.01\\
98	0.01\\
98.01	0.01\\
98.02	0.01\\
98.03	0.01\\
98.04	0.01\\
98.05	0.01\\
98.06	0.01\\
98.07	0.01\\
98.08	0.01\\
98.09	0.01\\
98.1	0.01\\
98.11	0.01\\
98.12	0.01\\
98.13	0.01\\
98.14	0.01\\
98.15	0.01\\
98.16	0.01\\
98.17	0.01\\
98.18	0.01\\
98.19	0.01\\
98.2	0.01\\
98.21	0.01\\
98.22	0.01\\
98.23	0.01\\
98.24	0.01\\
98.25	0.01\\
98.26	0.01\\
98.27	0.01\\
98.28	0.01\\
98.29	0.01\\
98.3	0.01\\
98.31	0.01\\
98.32	0.01\\
98.33	0.01\\
98.34	0.01\\
98.35	0.01\\
98.36	0.01\\
98.37	0.01\\
98.38	0.01\\
98.39	0.01\\
98.4	0.01\\
98.41	0.01\\
98.42	0.01\\
98.43	0.01\\
98.44	0.01\\
98.45	0.01\\
98.46	0.01\\
98.47	0.01\\
98.48	0.01\\
98.49	0.01\\
98.5	0.01\\
98.51	0.01\\
98.52	0.01\\
98.53	0.01\\
98.54	0.01\\
98.55	0.01\\
98.56	0.01\\
98.57	0.01\\
98.58	0.01\\
98.59	0.01\\
98.6	0.01\\
98.61	0.01\\
98.62	0.01\\
98.63	0.01\\
98.64	0.01\\
98.65	0.01\\
98.66	0.01\\
98.67	0.01\\
98.68	0.01\\
98.69	0.01\\
98.7	0.01\\
98.71	0.01\\
98.72	0.01\\
98.73	0.01\\
98.74	0.01\\
98.75	0.01\\
98.76	0.01\\
98.77	0.01\\
98.78	0.01\\
98.79	0.01\\
98.8	0.01\\
98.81	0.01\\
98.82	0.01\\
98.83	0.01\\
98.84	0.01\\
98.85	0.01\\
98.86	0.01\\
98.87	0.01\\
98.88	0.01\\
98.89	0.01\\
98.9	0.01\\
98.91	0.01\\
98.92	0.01\\
98.93	0.01\\
98.94	0.01\\
98.95	0.01\\
98.96	0.01\\
98.97	0.01\\
98.98	0.01\\
98.99	0.01\\
99	0.01\\
99.01	0.01\\
99.02	0.01\\
99.03	0.01\\
99.04	0.01\\
99.05	0.01\\
99.06	0.01\\
99.07	0.01\\
99.08	0.01\\
99.09	0.01\\
99.1	0.01\\
99.11	0.01\\
99.12	0.01\\
99.13	0.01\\
99.14	0.01\\
99.15	0.01\\
99.16	0.01\\
99.17	0.01\\
99.18	0.01\\
99.19	0.01\\
99.2	0.01\\
99.21	0.01\\
99.22	0.01\\
99.23	0.01\\
99.24	0.01\\
99.25	0.01\\
99.26	0.01\\
99.27	0.01\\
99.28	0.01\\
99.29	0.01\\
99.3	0.01\\
99.31	0.01\\
99.32	0.01\\
99.33	0.01\\
99.34	0.01\\
99.35	0.01\\
99.36	0.01\\
99.37	0.01\\
99.38	0.01\\
99.39	0.01\\
99.4	0.01\\
99.41	0.01\\
99.42	0.01\\
99.43	0.01\\
99.44	0.01\\
99.45	0.01\\
99.46	0.01\\
99.47	0.01\\
99.48	0.01\\
99.49	0.01\\
99.5	0.01\\
99.51	0.01\\
99.52	0.01\\
99.53	0.01\\
99.54	0.01\\
99.55	0.01\\
99.56	0.01\\
99.57	0.01\\
99.58	0.01\\
99.59	0.01\\
99.6	0.01\\
99.61	0.01\\
99.62	0.01\\
99.63	0.01\\
99.64	0.01\\
99.65	0.01\\
99.66	0.01\\
99.67	0.01\\
99.68	0.01\\
99.69	0.01\\
99.7	0.01\\
99.71	0.01\\
99.72	0.01\\
99.73	0.01\\
99.74	0.01\\
99.75	0.01\\
99.76	0.01\\
99.77	0.01\\
99.78	0.01\\
99.79	0.01\\
99.8	0.01\\
99.81	0.01\\
99.82	0.01\\
99.83	0.01\\
99.84	0.01\\
99.85	0.01\\
99.86	0.01\\
99.87	0.01\\
99.88	0.01\\
99.89	0.01\\
99.9	0.01\\
99.91	0.01\\
99.92	0.01\\
99.93	0.01\\
99.94	0.01\\
99.95	0.01\\
99.96	0.01\\
99.97	0.01\\
99.98	0.01\\
99.99	0.01\\
100	0.01\\
};
\addlegendentry{$q=3$};

\addplot [color=green,solid,forget plot]
  table[row sep=crcr]{%
0.01	0.01\\
0.02	0.01\\
0.03	0.01\\
0.04	0.01\\
0.05	0.01\\
0.06	0.01\\
0.07	0.01\\
0.08	0.01\\
0.09	0.01\\
0.1	0.01\\
0.11	0.01\\
0.12	0.01\\
0.13	0.01\\
0.14	0.01\\
0.15	0.01\\
0.16	0.01\\
0.17	0.01\\
0.18	0.01\\
0.19	0.01\\
0.2	0.01\\
0.21	0.01\\
0.22	0.01\\
0.23	0.01\\
0.24	0.01\\
0.25	0.01\\
0.26	0.01\\
0.27	0.01\\
0.28	0.01\\
0.29	0.01\\
0.3	0.01\\
0.31	0.01\\
0.32	0.01\\
0.33	0.01\\
0.34	0.01\\
0.35	0.01\\
0.36	0.01\\
0.37	0.01\\
0.38	0.01\\
0.39	0.01\\
0.4	0.01\\
0.41	0.01\\
0.42	0.01\\
0.43	0.01\\
0.44	0.01\\
0.45	0.01\\
0.46	0.01\\
0.47	0.01\\
0.48	0.01\\
0.49	0.01\\
0.5	0.01\\
0.51	0.01\\
0.52	0.01\\
0.53	0.01\\
0.54	0.01\\
0.55	0.01\\
0.56	0.01\\
0.57	0.01\\
0.58	0.01\\
0.59	0.01\\
0.6	0.01\\
0.61	0.01\\
0.62	0.01\\
0.63	0.01\\
0.64	0.01\\
0.65	0.01\\
0.66	0.01\\
0.67	0.01\\
0.68	0.01\\
0.69	0.01\\
0.7	0.01\\
0.71	0.01\\
0.72	0.01\\
0.73	0.01\\
0.74	0.01\\
0.75	0.01\\
0.76	0.01\\
0.77	0.01\\
0.78	0.01\\
0.79	0.01\\
0.8	0.01\\
0.81	0.01\\
0.82	0.01\\
0.83	0.01\\
0.84	0.01\\
0.85	0.01\\
0.86	0.01\\
0.87	0.01\\
0.88	0.01\\
0.89	0.01\\
0.9	0.01\\
0.91	0.01\\
0.92	0.01\\
0.93	0.01\\
0.94	0.01\\
0.95	0.01\\
0.96	0.01\\
0.97	0.01\\
0.98	0.01\\
0.99	0.01\\
1	0.01\\
1.01	0.01\\
1.02	0.01\\
1.03	0.01\\
1.04	0.01\\
1.05	0.01\\
1.06	0.01\\
1.07	0.01\\
1.08	0.01\\
1.09	0.01\\
1.1	0.01\\
1.11	0.01\\
1.12	0.01\\
1.13	0.01\\
1.14	0.01\\
1.15	0.01\\
1.16	0.01\\
1.17	0.01\\
1.18	0.01\\
1.19	0.01\\
1.2	0.01\\
1.21	0.01\\
1.22	0.01\\
1.23	0.01\\
1.24	0.01\\
1.25	0.01\\
1.26	0.01\\
1.27	0.01\\
1.28	0.01\\
1.29	0.01\\
1.3	0.01\\
1.31	0.01\\
1.32	0.01\\
1.33	0.01\\
1.34	0.01\\
1.35	0.01\\
1.36	0.01\\
1.37	0.01\\
1.38	0.01\\
1.39	0.01\\
1.4	0.01\\
1.41	0.01\\
1.42	0.01\\
1.43	0.01\\
1.44	0.01\\
1.45	0.01\\
1.46	0.01\\
1.47	0.01\\
1.48	0.01\\
1.49	0.01\\
1.5	0.01\\
1.51	0.01\\
1.52	0.01\\
1.53	0.01\\
1.54	0.01\\
1.55	0.01\\
1.56	0.01\\
1.57	0.01\\
1.58	0.01\\
1.59	0.01\\
1.6	0.01\\
1.61	0.01\\
1.62	0.01\\
1.63	0.01\\
1.64	0.01\\
1.65	0.01\\
1.66	0.01\\
1.67	0.01\\
1.68	0.01\\
1.69	0.01\\
1.7	0.01\\
1.71	0.01\\
1.72	0.01\\
1.73	0.01\\
1.74	0.01\\
1.75	0.01\\
1.76	0.01\\
1.77	0.01\\
1.78	0.01\\
1.79	0.01\\
1.8	0.01\\
1.81	0.01\\
1.82	0.01\\
1.83	0.01\\
1.84	0.01\\
1.85	0.01\\
1.86	0.01\\
1.87	0.01\\
1.88	0.01\\
1.89	0.01\\
1.9	0.01\\
1.91	0.01\\
1.92	0.01\\
1.93	0.01\\
1.94	0.01\\
1.95	0.01\\
1.96	0.01\\
1.97	0.01\\
1.98	0.01\\
1.99	0.01\\
2	0.01\\
2.01	0.01\\
2.02	0.01\\
2.03	0.01\\
2.04	0.01\\
2.05	0.01\\
2.06	0.01\\
2.07	0.01\\
2.08	0.01\\
2.09	0.01\\
2.1	0.01\\
2.11	0.01\\
2.12	0.01\\
2.13	0.01\\
2.14	0.01\\
2.15	0.01\\
2.16	0.01\\
2.17	0.01\\
2.18	0.01\\
2.19	0.01\\
2.2	0.01\\
2.21	0.01\\
2.22	0.01\\
2.23	0.01\\
2.24	0.01\\
2.25	0.01\\
2.26	0.01\\
2.27	0.01\\
2.28	0.01\\
2.29	0.01\\
2.3	0.01\\
2.31	0.01\\
2.32	0.01\\
2.33	0.01\\
2.34	0.01\\
2.35	0.01\\
2.36	0.01\\
2.37	0.01\\
2.38	0.01\\
2.39	0.01\\
2.4	0.01\\
2.41	0.01\\
2.42	0.01\\
2.43	0.01\\
2.44	0.01\\
2.45	0.01\\
2.46	0.01\\
2.47	0.01\\
2.48	0.01\\
2.49	0.01\\
2.5	0.01\\
2.51	0.01\\
2.52	0.01\\
2.53	0.01\\
2.54	0.01\\
2.55	0.01\\
2.56	0.01\\
2.57	0.01\\
2.58	0.01\\
2.59	0.01\\
2.6	0.01\\
2.61	0.01\\
2.62	0.01\\
2.63	0.01\\
2.64	0.01\\
2.65	0.01\\
2.66	0.01\\
2.67	0.01\\
2.68	0.01\\
2.69	0.01\\
2.7	0.01\\
2.71	0.01\\
2.72	0.01\\
2.73	0.01\\
2.74	0.01\\
2.75	0.01\\
2.76	0.01\\
2.77	0.01\\
2.78	0.01\\
2.79	0.01\\
2.8	0.01\\
2.81	0.01\\
2.82	0.01\\
2.83	0.01\\
2.84	0.01\\
2.85	0.01\\
2.86	0.01\\
2.87	0.01\\
2.88	0.01\\
2.89	0.01\\
2.9	0.01\\
2.91	0.01\\
2.92	0.01\\
2.93	0.01\\
2.94	0.01\\
2.95	0.01\\
2.96	0.01\\
2.97	0.01\\
2.98	0.01\\
2.99	0.01\\
3	0.01\\
3.01	0.01\\
3.02	0.01\\
3.03	0.01\\
3.04	0.01\\
3.05	0.01\\
3.06	0.01\\
3.07	0.01\\
3.08	0.01\\
3.09	0.01\\
3.1	0.01\\
3.11	0.01\\
3.12	0.01\\
3.13	0.01\\
3.14	0.01\\
3.15	0.01\\
3.16	0.01\\
3.17	0.01\\
3.18	0.01\\
3.19	0.01\\
3.2	0.01\\
3.21	0.01\\
3.22	0.01\\
3.23	0.01\\
3.24	0.01\\
3.25	0.01\\
3.26	0.01\\
3.27	0.01\\
3.28	0.01\\
3.29	0.01\\
3.3	0.01\\
3.31	0.01\\
3.32	0.01\\
3.33	0.01\\
3.34	0.01\\
3.35	0.01\\
3.36	0.01\\
3.37	0.01\\
3.38	0.01\\
3.39	0.01\\
3.4	0.01\\
3.41	0.01\\
3.42	0.01\\
3.43	0.01\\
3.44	0.01\\
3.45	0.01\\
3.46	0.01\\
3.47	0.01\\
3.48	0.01\\
3.49	0.01\\
3.5	0.01\\
3.51	0.01\\
3.52	0.01\\
3.53	0.01\\
3.54	0.01\\
3.55	0.01\\
3.56	0.01\\
3.57	0.01\\
3.58	0.01\\
3.59	0.01\\
3.6	0.01\\
3.61	0.01\\
3.62	0.01\\
3.63	0.01\\
3.64	0.01\\
3.65	0.01\\
3.66	0.01\\
3.67	0.01\\
3.68	0.01\\
3.69	0.01\\
3.7	0.01\\
3.71	0.01\\
3.72	0.01\\
3.73	0.01\\
3.74	0.01\\
3.75	0.01\\
3.76	0.01\\
3.77	0.01\\
3.78	0.01\\
3.79	0.01\\
3.8	0.01\\
3.81	0.01\\
3.82	0.01\\
3.83	0.01\\
3.84	0.01\\
3.85	0.01\\
3.86	0.01\\
3.87	0.01\\
3.88	0.01\\
3.89	0.01\\
3.9	0.01\\
3.91	0.01\\
3.92	0.01\\
3.93	0.01\\
3.94	0.01\\
3.95	0.01\\
3.96	0.01\\
3.97	0.01\\
3.98	0.01\\
3.99	0.01\\
4	0.01\\
4.01	0.01\\
4.02	0.01\\
4.03	0.01\\
4.04	0.01\\
4.05	0.01\\
4.06	0.01\\
4.07	0.01\\
4.08	0.01\\
4.09	0.01\\
4.1	0.01\\
4.11	0.01\\
4.12	0.01\\
4.13	0.01\\
4.14	0.01\\
4.15	0.01\\
4.16	0.01\\
4.17	0.01\\
4.18	0.01\\
4.19	0.01\\
4.2	0.01\\
4.21	0.01\\
4.22	0.01\\
4.23	0.01\\
4.24	0.01\\
4.25	0.01\\
4.26	0.01\\
4.27	0.01\\
4.28	0.01\\
4.29	0.01\\
4.3	0.01\\
4.31	0.01\\
4.32	0.01\\
4.33	0.01\\
4.34	0.01\\
4.35	0.01\\
4.36	0.01\\
4.37	0.01\\
4.38	0.01\\
4.39	0.01\\
4.4	0.01\\
4.41	0.01\\
4.42	0.01\\
4.43	0.01\\
4.44	0.01\\
4.45	0.01\\
4.46	0.01\\
4.47	0.01\\
4.48	0.01\\
4.49	0.01\\
4.5	0.01\\
4.51	0.01\\
4.52	0.01\\
4.53	0.01\\
4.54	0.01\\
4.55	0.01\\
4.56	0.01\\
4.57	0.01\\
4.58	0.01\\
4.59	0.01\\
4.6	0.01\\
4.61	0.01\\
4.62	0.01\\
4.63	0.01\\
4.64	0.01\\
4.65	0.01\\
4.66	0.01\\
4.67	0.01\\
4.68	0.01\\
4.69	0.01\\
4.7	0.01\\
4.71	0.01\\
4.72	0.01\\
4.73	0.01\\
4.74	0.01\\
4.75	0.01\\
4.76	0.01\\
4.77	0.01\\
4.78	0.01\\
4.79	0.01\\
4.8	0.01\\
4.81	0.01\\
4.82	0.01\\
4.83	0.01\\
4.84	0.01\\
4.85	0.01\\
4.86	0.01\\
4.87	0.01\\
4.88	0.01\\
4.89	0.01\\
4.9	0.01\\
4.91	0.01\\
4.92	0.01\\
4.93	0.01\\
4.94	0.01\\
4.95	0.01\\
4.96	0.01\\
4.97	0.01\\
4.98	0.01\\
4.99	0.01\\
5	0.01\\
5.01	0.01\\
5.02	0.01\\
5.03	0.01\\
5.04	0.01\\
5.05	0.01\\
5.06	0.01\\
5.07	0.01\\
5.08	0.01\\
5.09	0.01\\
5.1	0.01\\
5.11	0.01\\
5.12	0.01\\
5.13	0.01\\
5.14	0.01\\
5.15	0.01\\
5.16	0.01\\
5.17	0.01\\
5.18	0.01\\
5.19	0.01\\
5.2	0.01\\
5.21	0.01\\
5.22	0.01\\
5.23	0.01\\
5.24	0.01\\
5.25	0.01\\
5.26	0.01\\
5.27	0.01\\
5.28	0.01\\
5.29	0.01\\
5.3	0.01\\
5.31	0.01\\
5.32	0.01\\
5.33	0.01\\
5.34	0.01\\
5.35	0.01\\
5.36	0.01\\
5.37	0.01\\
5.38	0.01\\
5.39	0.01\\
5.4	0.01\\
5.41	0.01\\
5.42	0.01\\
5.43	0.01\\
5.44	0.01\\
5.45	0.01\\
5.46	0.01\\
5.47	0.01\\
5.48	0.01\\
5.49	0.01\\
5.5	0.01\\
5.51	0.01\\
5.52	0.01\\
5.53	0.01\\
5.54	0.01\\
5.55	0.01\\
5.56	0.01\\
5.57	0.01\\
5.58	0.01\\
5.59	0.01\\
5.6	0.01\\
5.61	0.01\\
5.62	0.01\\
5.63	0.01\\
5.64	0.01\\
5.65	0.01\\
5.66	0.01\\
5.67	0.01\\
5.68	0.01\\
5.69	0.01\\
5.7	0.01\\
5.71	0.01\\
5.72	0.01\\
5.73	0.01\\
5.74	0.01\\
5.75	0.01\\
5.76	0.01\\
5.77	0.01\\
5.78	0.01\\
5.79	0.01\\
5.8	0.01\\
5.81	0.01\\
5.82	0.01\\
5.83	0.01\\
5.84	0.01\\
5.85	0.01\\
5.86	0.01\\
5.87	0.01\\
5.88	0.01\\
5.89	0.01\\
5.9	0.01\\
5.91	0.01\\
5.92	0.01\\
5.93	0.01\\
5.94	0.01\\
5.95	0.01\\
5.96	0.01\\
5.97	0.01\\
5.98	0.01\\
5.99	0.01\\
6	0.01\\
6.01	0.01\\
6.02	0.01\\
6.03	0.01\\
6.04	0.01\\
6.05	0.01\\
6.06	0.01\\
6.07	0.01\\
6.08	0.01\\
6.09	0.01\\
6.1	0.01\\
6.11	0.01\\
6.12	0.01\\
6.13	0.01\\
6.14	0.01\\
6.15	0.01\\
6.16	0.01\\
6.17	0.01\\
6.18	0.01\\
6.19	0.01\\
6.2	0.01\\
6.21	0.01\\
6.22	0.01\\
6.23	0.01\\
6.24	0.01\\
6.25	0.01\\
6.26	0.01\\
6.27	0.01\\
6.28	0.01\\
6.29	0.01\\
6.3	0.01\\
6.31	0.01\\
6.32	0.01\\
6.33	0.01\\
6.34	0.01\\
6.35	0.01\\
6.36	0.01\\
6.37	0.01\\
6.38	0.01\\
6.39	0.01\\
6.4	0.01\\
6.41	0.01\\
6.42	0.01\\
6.43	0.01\\
6.44	0.01\\
6.45	0.01\\
6.46	0.01\\
6.47	0.01\\
6.48	0.01\\
6.49	0.01\\
6.5	0.01\\
6.51	0.01\\
6.52	0.01\\
6.53	0.01\\
6.54	0.01\\
6.55	0.01\\
6.56	0.01\\
6.57	0.01\\
6.58	0.01\\
6.59	0.01\\
6.6	0.01\\
6.61	0.01\\
6.62	0.01\\
6.63	0.01\\
6.64	0.01\\
6.65	0.01\\
6.66	0.01\\
6.67	0.01\\
6.68	0.01\\
6.69	0.01\\
6.7	0.01\\
6.71	0.01\\
6.72	0.01\\
6.73	0.01\\
6.74	0.01\\
6.75	0.01\\
6.76	0.01\\
6.77	0.01\\
6.78	0.01\\
6.79	0.01\\
6.8	0.01\\
6.81	0.01\\
6.82	0.01\\
6.83	0.01\\
6.84	0.01\\
6.85	0.01\\
6.86	0.01\\
6.87	0.01\\
6.88	0.01\\
6.89	0.01\\
6.9	0.01\\
6.91	0.01\\
6.92	0.01\\
6.93	0.01\\
6.94	0.01\\
6.95	0.01\\
6.96	0.01\\
6.97	0.01\\
6.98	0.01\\
6.99	0.01\\
7	0.01\\
7.01	0.01\\
7.02	0.01\\
7.03	0.01\\
7.04	0.01\\
7.05	0.01\\
7.06	0.01\\
7.07	0.01\\
7.08	0.01\\
7.09	0.01\\
7.1	0.01\\
7.11	0.01\\
7.12	0.01\\
7.13	0.01\\
7.14	0.01\\
7.15	0.01\\
7.16	0.01\\
7.17	0.01\\
7.18	0.01\\
7.19	0.01\\
7.2	0.01\\
7.21	0.01\\
7.22	0.01\\
7.23	0.01\\
7.24	0.01\\
7.25	0.01\\
7.26	0.01\\
7.27	0.01\\
7.28	0.01\\
7.29	0.01\\
7.3	0.01\\
7.31	0.01\\
7.32	0.01\\
7.33	0.01\\
7.34	0.01\\
7.35	0.01\\
7.36	0.01\\
7.37	0.01\\
7.38	0.01\\
7.39	0.01\\
7.4	0.01\\
7.41	0.01\\
7.42	0.01\\
7.43	0.01\\
7.44	0.01\\
7.45	0.01\\
7.46	0.01\\
7.47	0.01\\
7.48	0.01\\
7.49	0.01\\
7.5	0.01\\
7.51	0.01\\
7.52	0.01\\
7.53	0.01\\
7.54	0.01\\
7.55	0.01\\
7.56	0.01\\
7.57	0.01\\
7.58	0.01\\
7.59	0.01\\
7.6	0.01\\
7.61	0.01\\
7.62	0.01\\
7.63	0.01\\
7.64	0.01\\
7.65	0.01\\
7.66	0.01\\
7.67	0.01\\
7.68	0.01\\
7.69	0.01\\
7.7	0.01\\
7.71	0.01\\
7.72	0.01\\
7.73	0.01\\
7.74	0.01\\
7.75	0.01\\
7.76	0.01\\
7.77	0.01\\
7.78	0.01\\
7.79	0.01\\
7.8	0.01\\
7.81	0.01\\
7.82	0.01\\
7.83	0.01\\
7.84	0.01\\
7.85	0.01\\
7.86	0.01\\
7.87	0.01\\
7.88	0.01\\
7.89	0.01\\
7.9	0.01\\
7.91	0.01\\
7.92	0.01\\
7.93	0.01\\
7.94	0.01\\
7.95	0.01\\
7.96	0.01\\
7.97	0.01\\
7.98	0.01\\
7.99	0.01\\
8	0.01\\
8.01	0.01\\
8.02	0.01\\
8.03	0.01\\
8.04	0.01\\
8.05	0.01\\
8.06	0.01\\
8.07	0.01\\
8.08	0.01\\
8.09	0.01\\
8.1	0.01\\
8.11	0.01\\
8.12	0.01\\
8.13	0.01\\
8.14	0.01\\
8.15	0.01\\
8.16	0.01\\
8.17	0.01\\
8.18	0.01\\
8.19	0.01\\
8.2	0.01\\
8.21	0.01\\
8.22	0.01\\
8.23	0.01\\
8.24	0.01\\
8.25	0.01\\
8.26	0.01\\
8.27	0.01\\
8.28	0.01\\
8.29	0.01\\
8.3	0.01\\
8.31	0.01\\
8.32	0.01\\
8.33	0.01\\
8.34	0.01\\
8.35	0.01\\
8.36	0.01\\
8.37	0.01\\
8.38	0.01\\
8.39	0.01\\
8.4	0.01\\
8.41	0.01\\
8.42	0.01\\
8.43	0.01\\
8.44	0.01\\
8.45	0.01\\
8.46	0.01\\
8.47	0.01\\
8.48	0.01\\
8.49	0.01\\
8.5	0.01\\
8.51	0.01\\
8.52	0.01\\
8.53	0.01\\
8.54	0.01\\
8.55	0.01\\
8.56	0.01\\
8.57	0.01\\
8.58	0.01\\
8.59	0.01\\
8.6	0.01\\
8.61	0.01\\
8.62	0.01\\
8.63	0.01\\
8.64	0.01\\
8.65	0.01\\
8.66	0.01\\
8.67	0.01\\
8.68	0.01\\
8.69	0.01\\
8.7	0.01\\
8.71	0.01\\
8.72	0.01\\
8.73	0.01\\
8.74	0.01\\
8.75	0.01\\
8.76	0.01\\
8.77	0.01\\
8.78	0.01\\
8.79	0.01\\
8.8	0.01\\
8.81	0.01\\
8.82	0.01\\
8.83	0.01\\
8.84	0.01\\
8.85	0.01\\
8.86	0.01\\
8.87	0.01\\
8.88	0.01\\
8.89	0.01\\
8.9	0.01\\
8.91	0.01\\
8.92	0.01\\
8.93	0.01\\
8.94	0.01\\
8.95	0.01\\
8.96	0.01\\
8.97	0.01\\
8.98	0.01\\
8.99	0.01\\
9	0.01\\
9.01	0.01\\
9.02	0.01\\
9.03	0.01\\
9.04	0.01\\
9.05	0.01\\
9.06	0.01\\
9.07	0.01\\
9.08	0.01\\
9.09	0.01\\
9.1	0.01\\
9.11	0.01\\
9.12	0.01\\
9.13	0.01\\
9.14	0.01\\
9.15	0.01\\
9.16	0.01\\
9.17	0.01\\
9.18	0.01\\
9.19	0.01\\
9.2	0.01\\
9.21	0.01\\
9.22	0.01\\
9.23	0.01\\
9.24	0.01\\
9.25	0.01\\
9.26	0.01\\
9.27	0.01\\
9.28	0.01\\
9.29	0.01\\
9.3	0.01\\
9.31	0.01\\
9.32	0.01\\
9.33	0.01\\
9.34	0.01\\
9.35	0.01\\
9.36	0.01\\
9.37	0.01\\
9.38	0.01\\
9.39	0.01\\
9.4	0.01\\
9.41	0.01\\
9.42	0.01\\
9.43	0.01\\
9.44	0.01\\
9.45	0.01\\
9.46	0.01\\
9.47	0.01\\
9.48	0.01\\
9.49	0.01\\
9.5	0.01\\
9.51	0.01\\
9.52	0.01\\
9.53	0.01\\
9.54	0.01\\
9.55	0.01\\
9.56	0.01\\
9.57	0.01\\
9.58	0.01\\
9.59	0.01\\
9.6	0.01\\
9.61	0.01\\
9.62	0.01\\
9.63	0.01\\
9.64	0.01\\
9.65	0.01\\
9.66	0.01\\
9.67	0.01\\
9.68	0.01\\
9.69	0.01\\
9.7	0.01\\
9.71	0.01\\
9.72	0.01\\
9.73	0.01\\
9.74	0.01\\
9.75	0.01\\
9.76	0.01\\
9.77	0.01\\
9.78	0.01\\
9.79	0.01\\
9.8	0.01\\
9.81	0.01\\
9.82	0.01\\
9.83	0.01\\
9.84	0.01\\
9.85	0.01\\
9.86	0.01\\
9.87	0.01\\
9.88	0.01\\
9.89	0.01\\
9.9	0.01\\
9.91	0.01\\
9.92	0.01\\
9.93	0.01\\
9.94	0.01\\
9.95	0.01\\
9.96	0.01\\
9.97	0.01\\
9.98	0.01\\
9.99	0.01\\
10	0.01\\
10.01	0.01\\
10.02	0.01\\
10.03	0.01\\
10.04	0.01\\
10.05	0.01\\
10.06	0.01\\
10.07	0.01\\
10.08	0.01\\
10.09	0.01\\
10.1	0.01\\
10.11	0.01\\
10.12	0.01\\
10.13	0.01\\
10.14	0.01\\
10.15	0.01\\
10.16	0.01\\
10.17	0.01\\
10.18	0.01\\
10.19	0.01\\
10.2	0.01\\
10.21	0.01\\
10.22	0.01\\
10.23	0.01\\
10.24	0.01\\
10.25	0.01\\
10.26	0.01\\
10.27	0.01\\
10.28	0.01\\
10.29	0.01\\
10.3	0.01\\
10.31	0.01\\
10.32	0.01\\
10.33	0.01\\
10.34	0.01\\
10.35	0.01\\
10.36	0.01\\
10.37	0.01\\
10.38	0.01\\
10.39	0.01\\
10.4	0.01\\
10.41	0.01\\
10.42	0.01\\
10.43	0.01\\
10.44	0.01\\
10.45	0.01\\
10.46	0.01\\
10.47	0.01\\
10.48	0.01\\
10.49	0.01\\
10.5	0.01\\
10.51	0.01\\
10.52	0.01\\
10.53	0.01\\
10.54	0.01\\
10.55	0.01\\
10.56	0.01\\
10.57	0.01\\
10.58	0.01\\
10.59	0.01\\
10.6	0.01\\
10.61	0.01\\
10.62	0.01\\
10.63	0.01\\
10.64	0.01\\
10.65	0.01\\
10.66	0.01\\
10.67	0.01\\
10.68	0.01\\
10.69	0.01\\
10.7	0.01\\
10.71	0.01\\
10.72	0.01\\
10.73	0.01\\
10.74	0.01\\
10.75	0.01\\
10.76	0.01\\
10.77	0.01\\
10.78	0.01\\
10.79	0.01\\
10.8	0.01\\
10.81	0.01\\
10.82	0.01\\
10.83	0.01\\
10.84	0.01\\
10.85	0.01\\
10.86	0.01\\
10.87	0.01\\
10.88	0.01\\
10.89	0.01\\
10.9	0.01\\
10.91	0.01\\
10.92	0.01\\
10.93	0.01\\
10.94	0.01\\
10.95	0.01\\
10.96	0.01\\
10.97	0.01\\
10.98	0.01\\
10.99	0.01\\
11	0.01\\
11.01	0.01\\
11.02	0.01\\
11.03	0.01\\
11.04	0.01\\
11.05	0.01\\
11.06	0.01\\
11.07	0.01\\
11.08	0.01\\
11.09	0.01\\
11.1	0.01\\
11.11	0.01\\
11.12	0.01\\
11.13	0.01\\
11.14	0.01\\
11.15	0.01\\
11.16	0.01\\
11.17	0.01\\
11.18	0.01\\
11.19	0.01\\
11.2	0.01\\
11.21	0.01\\
11.22	0.01\\
11.23	0.01\\
11.24	0.01\\
11.25	0.01\\
11.26	0.01\\
11.27	0.01\\
11.28	0.01\\
11.29	0.01\\
11.3	0.01\\
11.31	0.01\\
11.32	0.01\\
11.33	0.01\\
11.34	0.01\\
11.35	0.01\\
11.36	0.01\\
11.37	0.01\\
11.38	0.01\\
11.39	0.01\\
11.4	0.01\\
11.41	0.01\\
11.42	0.01\\
11.43	0.01\\
11.44	0.01\\
11.45	0.01\\
11.46	0.01\\
11.47	0.01\\
11.48	0.01\\
11.49	0.01\\
11.5	0.01\\
11.51	0.01\\
11.52	0.01\\
11.53	0.01\\
11.54	0.01\\
11.55	0.01\\
11.56	0.01\\
11.57	0.01\\
11.58	0.01\\
11.59	0.01\\
11.6	0.01\\
11.61	0.01\\
11.62	0.01\\
11.63	0.01\\
11.64	0.01\\
11.65	0.01\\
11.66	0.01\\
11.67	0.01\\
11.68	0.01\\
11.69	0.01\\
11.7	0.01\\
11.71	0.01\\
11.72	0.01\\
11.73	0.01\\
11.74	0.01\\
11.75	0.01\\
11.76	0.01\\
11.77	0.01\\
11.78	0.01\\
11.79	0.01\\
11.8	0.01\\
11.81	0.01\\
11.82	0.01\\
11.83	0.01\\
11.84	0.01\\
11.85	0.01\\
11.86	0.01\\
11.87	0.01\\
11.88	0.01\\
11.89	0.01\\
11.9	0.01\\
11.91	0.01\\
11.92	0.01\\
11.93	0.01\\
11.94	0.01\\
11.95	0.01\\
11.96	0.01\\
11.97	0.01\\
11.98	0.01\\
11.99	0.01\\
12	0.01\\
12.01	0.01\\
12.02	0.01\\
12.03	0.01\\
12.04	0.01\\
12.05	0.01\\
12.06	0.01\\
12.07	0.01\\
12.08	0.01\\
12.09	0.01\\
12.1	0.01\\
12.11	0.01\\
12.12	0.01\\
12.13	0.01\\
12.14	0.01\\
12.15	0.01\\
12.16	0.01\\
12.17	0.01\\
12.18	0.01\\
12.19	0.01\\
12.2	0.01\\
12.21	0.01\\
12.22	0.01\\
12.23	0.01\\
12.24	0.01\\
12.25	0.01\\
12.26	0.01\\
12.27	0.01\\
12.28	0.01\\
12.29	0.01\\
12.3	0.01\\
12.31	0.01\\
12.32	0.01\\
12.33	0.01\\
12.34	0.01\\
12.35	0.01\\
12.36	0.01\\
12.37	0.01\\
12.38	0.01\\
12.39	0.01\\
12.4	0.01\\
12.41	0.01\\
12.42	0.01\\
12.43	0.01\\
12.44	0.01\\
12.45	0.01\\
12.46	0.01\\
12.47	0.01\\
12.48	0.01\\
12.49	0.01\\
12.5	0.01\\
12.51	0.01\\
12.52	0.01\\
12.53	0.01\\
12.54	0.01\\
12.55	0.01\\
12.56	0.01\\
12.57	0.01\\
12.58	0.01\\
12.59	0.01\\
12.6	0.01\\
12.61	0.01\\
12.62	0.01\\
12.63	0.01\\
12.64	0.01\\
12.65	0.01\\
12.66	0.01\\
12.67	0.01\\
12.68	0.01\\
12.69	0.01\\
12.7	0.01\\
12.71	0.01\\
12.72	0.01\\
12.73	0.01\\
12.74	0.01\\
12.75	0.01\\
12.76	0.01\\
12.77	0.01\\
12.78	0.01\\
12.79	0.01\\
12.8	0.01\\
12.81	0.01\\
12.82	0.01\\
12.83	0.01\\
12.84	0.01\\
12.85	0.01\\
12.86	0.01\\
12.87	0.01\\
12.88	0.01\\
12.89	0.01\\
12.9	0.01\\
12.91	0.01\\
12.92	0.01\\
12.93	0.01\\
12.94	0.01\\
12.95	0.01\\
12.96	0.01\\
12.97	0.01\\
12.98	0.01\\
12.99	0.01\\
13	0.01\\
13.01	0.01\\
13.02	0.01\\
13.03	0.01\\
13.04	0.01\\
13.05	0.01\\
13.06	0.01\\
13.07	0.01\\
13.08	0.01\\
13.09	0.01\\
13.1	0.01\\
13.11	0.01\\
13.12	0.01\\
13.13	0.01\\
13.14	0.01\\
13.15	0.01\\
13.16	0.01\\
13.17	0.01\\
13.18	0.01\\
13.19	0.01\\
13.2	0.01\\
13.21	0.01\\
13.22	0.01\\
13.23	0.01\\
13.24	0.01\\
13.25	0.01\\
13.26	0.01\\
13.27	0.01\\
13.28	0.01\\
13.29	0.01\\
13.3	0.01\\
13.31	0.01\\
13.32	0.01\\
13.33	0.01\\
13.34	0.01\\
13.35	0.01\\
13.36	0.01\\
13.37	0.01\\
13.38	0.01\\
13.39	0.01\\
13.4	0.01\\
13.41	0.01\\
13.42	0.01\\
13.43	0.01\\
13.44	0.01\\
13.45	0.01\\
13.46	0.01\\
13.47	0.01\\
13.48	0.01\\
13.49	0.01\\
13.5	0.01\\
13.51	0.01\\
13.52	0.01\\
13.53	0.01\\
13.54	0.01\\
13.55	0.01\\
13.56	0.01\\
13.57	0.01\\
13.58	0.01\\
13.59	0.01\\
13.6	0.01\\
13.61	0.01\\
13.62	0.01\\
13.63	0.01\\
13.64	0.01\\
13.65	0.01\\
13.66	0.01\\
13.67	0.01\\
13.68	0.01\\
13.69	0.01\\
13.7	0.01\\
13.71	0.01\\
13.72	0.01\\
13.73	0.01\\
13.74	0.01\\
13.75	0.01\\
13.76	0.01\\
13.77	0.01\\
13.78	0.01\\
13.79	0.01\\
13.8	0.01\\
13.81	0.01\\
13.82	0.01\\
13.83	0.01\\
13.84	0.01\\
13.85	0.01\\
13.86	0.01\\
13.87	0.01\\
13.88	0.01\\
13.89	0.01\\
13.9	0.01\\
13.91	0.01\\
13.92	0.01\\
13.93	0.01\\
13.94	0.01\\
13.95	0.01\\
13.96	0.01\\
13.97	0.01\\
13.98	0.01\\
13.99	0.01\\
14	0.01\\
14.01	0.01\\
14.02	0.01\\
14.03	0.01\\
14.04	0.01\\
14.05	0.01\\
14.06	0.01\\
14.07	0.01\\
14.08	0.01\\
14.09	0.01\\
14.1	0.01\\
14.11	0.01\\
14.12	0.01\\
14.13	0.01\\
14.14	0.01\\
14.15	0.01\\
14.16	0.01\\
14.17	0.01\\
14.18	0.01\\
14.19	0.01\\
14.2	0.01\\
14.21	0.01\\
14.22	0.01\\
14.23	0.01\\
14.24	0.01\\
14.25	0.01\\
14.26	0.01\\
14.27	0.01\\
14.28	0.01\\
14.29	0.01\\
14.3	0.01\\
14.31	0.01\\
14.32	0.01\\
14.33	0.01\\
14.34	0.01\\
14.35	0.01\\
14.36	0.01\\
14.37	0.01\\
14.38	0.01\\
14.39	0.01\\
14.4	0.01\\
14.41	0.01\\
14.42	0.01\\
14.43	0.01\\
14.44	0.01\\
14.45	0.01\\
14.46	0.01\\
14.47	0.01\\
14.48	0.01\\
14.49	0.01\\
14.5	0.01\\
14.51	0.01\\
14.52	0.01\\
14.53	0.01\\
14.54	0.01\\
14.55	0.01\\
14.56	0.01\\
14.57	0.01\\
14.58	0.01\\
14.59	0.01\\
14.6	0.01\\
14.61	0.01\\
14.62	0.01\\
14.63	0.01\\
14.64	0.01\\
14.65	0.01\\
14.66	0.01\\
14.67	0.01\\
14.68	0.01\\
14.69	0.01\\
14.7	0.01\\
14.71	0.01\\
14.72	0.01\\
14.73	0.01\\
14.74	0.01\\
14.75	0.01\\
14.76	0.01\\
14.77	0.01\\
14.78	0.01\\
14.79	0.01\\
14.8	0.01\\
14.81	0.01\\
14.82	0.01\\
14.83	0.01\\
14.84	0.01\\
14.85	0.01\\
14.86	0.01\\
14.87	0.01\\
14.88	0.01\\
14.89	0.01\\
14.9	0.01\\
14.91	0.01\\
14.92	0.01\\
14.93	0.01\\
14.94	0.01\\
14.95	0.01\\
14.96	0.01\\
14.97	0.01\\
14.98	0.01\\
14.99	0.01\\
15	0.01\\
15.01	0.01\\
15.02	0.01\\
15.03	0.01\\
15.04	0.01\\
15.05	0.01\\
15.06	0.01\\
15.07	0.01\\
15.08	0.01\\
15.09	0.01\\
15.1	0.01\\
15.11	0.01\\
15.12	0.01\\
15.13	0.01\\
15.14	0.01\\
15.15	0.01\\
15.16	0.01\\
15.17	0.01\\
15.18	0.01\\
15.19	0.01\\
15.2	0.01\\
15.21	0.01\\
15.22	0.01\\
15.23	0.01\\
15.24	0.01\\
15.25	0.01\\
15.26	0.01\\
15.27	0.01\\
15.28	0.01\\
15.29	0.01\\
15.3	0.01\\
15.31	0.01\\
15.32	0.01\\
15.33	0.01\\
15.34	0.01\\
15.35	0.01\\
15.36	0.01\\
15.37	0.01\\
15.38	0.01\\
15.39	0.01\\
15.4	0.01\\
15.41	0.01\\
15.42	0.01\\
15.43	0.01\\
15.44	0.01\\
15.45	0.01\\
15.46	0.01\\
15.47	0.01\\
15.48	0.01\\
15.49	0.01\\
15.5	0.01\\
15.51	0.01\\
15.52	0.01\\
15.53	0.01\\
15.54	0.01\\
15.55	0.01\\
15.56	0.01\\
15.57	0.01\\
15.58	0.01\\
15.59	0.01\\
15.6	0.01\\
15.61	0.01\\
15.62	0.01\\
15.63	0.01\\
15.64	0.01\\
15.65	0.01\\
15.66	0.01\\
15.67	0.01\\
15.68	0.01\\
15.69	0.01\\
15.7	0.01\\
15.71	0.01\\
15.72	0.01\\
15.73	0.01\\
15.74	0.01\\
15.75	0.01\\
15.76	0.01\\
15.77	0.01\\
15.78	0.01\\
15.79	0.01\\
15.8	0.01\\
15.81	0.01\\
15.82	0.01\\
15.83	0.01\\
15.84	0.01\\
15.85	0.01\\
15.86	0.01\\
15.87	0.01\\
15.88	0.01\\
15.89	0.01\\
15.9	0.01\\
15.91	0.01\\
15.92	0.01\\
15.93	0.01\\
15.94	0.01\\
15.95	0.01\\
15.96	0.01\\
15.97	0.01\\
15.98	0.01\\
15.99	0.01\\
16	0.01\\
16.01	0.01\\
16.02	0.01\\
16.03	0.01\\
16.04	0.01\\
16.05	0.01\\
16.06	0.01\\
16.07	0.01\\
16.08	0.01\\
16.09	0.01\\
16.1	0.01\\
16.11	0.01\\
16.12	0.01\\
16.13	0.01\\
16.14	0.01\\
16.15	0.01\\
16.16	0.01\\
16.17	0.01\\
16.18	0.01\\
16.19	0.01\\
16.2	0.01\\
16.21	0.01\\
16.22	0.01\\
16.23	0.01\\
16.24	0.01\\
16.25	0.01\\
16.26	0.01\\
16.27	0.01\\
16.28	0.01\\
16.29	0.01\\
16.3	0.01\\
16.31	0.01\\
16.32	0.01\\
16.33	0.01\\
16.34	0.01\\
16.35	0.01\\
16.36	0.01\\
16.37	0.01\\
16.38	0.01\\
16.39	0.01\\
16.4	0.01\\
16.41	0.01\\
16.42	0.01\\
16.43	0.01\\
16.44	0.01\\
16.45	0.01\\
16.46	0.01\\
16.47	0.01\\
16.48	0.01\\
16.49	0.01\\
16.5	0.01\\
16.51	0.01\\
16.52	0.01\\
16.53	0.01\\
16.54	0.01\\
16.55	0.01\\
16.56	0.01\\
16.57	0.01\\
16.58	0.01\\
16.59	0.01\\
16.6	0.01\\
16.61	0.01\\
16.62	0.01\\
16.63	0.01\\
16.64	0.01\\
16.65	0.01\\
16.66	0.01\\
16.67	0.01\\
16.68	0.01\\
16.69	0.01\\
16.7	0.01\\
16.71	0.01\\
16.72	0.01\\
16.73	0.01\\
16.74	0.01\\
16.75	0.01\\
16.76	0.01\\
16.77	0.01\\
16.78	0.01\\
16.79	0.01\\
16.8	0.01\\
16.81	0.01\\
16.82	0.01\\
16.83	0.01\\
16.84	0.01\\
16.85	0.01\\
16.86	0.01\\
16.87	0.01\\
16.88	0.01\\
16.89	0.01\\
16.9	0.01\\
16.91	0.01\\
16.92	0.01\\
16.93	0.01\\
16.94	0.01\\
16.95	0.01\\
16.96	0.01\\
16.97	0.01\\
16.98	0.01\\
16.99	0.01\\
17	0.01\\
17.01	0.01\\
17.02	0.01\\
17.03	0.01\\
17.04	0.01\\
17.05	0.01\\
17.06	0.01\\
17.07	0.01\\
17.08	0.01\\
17.09	0.01\\
17.1	0.01\\
17.11	0.01\\
17.12	0.01\\
17.13	0.01\\
17.14	0.01\\
17.15	0.01\\
17.16	0.01\\
17.17	0.01\\
17.18	0.01\\
17.19	0.01\\
17.2	0.01\\
17.21	0.01\\
17.22	0.01\\
17.23	0.01\\
17.24	0.01\\
17.25	0.01\\
17.26	0.01\\
17.27	0.01\\
17.28	0.01\\
17.29	0.01\\
17.3	0.01\\
17.31	0.01\\
17.32	0.01\\
17.33	0.01\\
17.34	0.01\\
17.35	0.01\\
17.36	0.01\\
17.37	0.01\\
17.38	0.01\\
17.39	0.01\\
17.4	0.01\\
17.41	0.01\\
17.42	0.01\\
17.43	0.01\\
17.44	0.01\\
17.45	0.01\\
17.46	0.01\\
17.47	0.01\\
17.48	0.01\\
17.49	0.01\\
17.5	0.01\\
17.51	0.01\\
17.52	0.01\\
17.53	0.01\\
17.54	0.01\\
17.55	0.01\\
17.56	0.01\\
17.57	0.01\\
17.58	0.01\\
17.59	0.01\\
17.6	0.01\\
17.61	0.01\\
17.62	0.01\\
17.63	0.01\\
17.64	0.01\\
17.65	0.01\\
17.66	0.01\\
17.67	0.01\\
17.68	0.01\\
17.69	0.01\\
17.7	0.01\\
17.71	0.01\\
17.72	0.01\\
17.73	0.01\\
17.74	0.01\\
17.75	0.01\\
17.76	0.01\\
17.77	0.01\\
17.78	0.01\\
17.79	0.01\\
17.8	0.01\\
17.81	0.01\\
17.82	0.01\\
17.83	0.01\\
17.84	0.01\\
17.85	0.01\\
17.86	0.01\\
17.87	0.01\\
17.88	0.01\\
17.89	0.01\\
17.9	0.01\\
17.91	0.01\\
17.92	0.01\\
17.93	0.01\\
17.94	0.01\\
17.95	0.01\\
17.96	0.01\\
17.97	0.01\\
17.98	0.01\\
17.99	0.01\\
18	0.01\\
18.01	0.01\\
18.02	0.01\\
18.03	0.01\\
18.04	0.01\\
18.05	0.01\\
18.06	0.01\\
18.07	0.01\\
18.08	0.01\\
18.09	0.01\\
18.1	0.01\\
18.11	0.01\\
18.12	0.01\\
18.13	0.01\\
18.14	0.01\\
18.15	0.01\\
18.16	0.01\\
18.17	0.01\\
18.18	0.01\\
18.19	0.01\\
18.2	0.01\\
18.21	0.01\\
18.22	0.01\\
18.23	0.01\\
18.24	0.01\\
18.25	0.01\\
18.26	0.01\\
18.27	0.01\\
18.28	0.01\\
18.29	0.01\\
18.3	0.01\\
18.31	0.01\\
18.32	0.01\\
18.33	0.01\\
18.34	0.01\\
18.35	0.01\\
18.36	0.01\\
18.37	0.01\\
18.38	0.01\\
18.39	0.01\\
18.4	0.01\\
18.41	0.01\\
18.42	0.01\\
18.43	0.01\\
18.44	0.01\\
18.45	0.01\\
18.46	0.01\\
18.47	0.01\\
18.48	0.01\\
18.49	0.01\\
18.5	0.01\\
18.51	0.01\\
18.52	0.01\\
18.53	0.01\\
18.54	0.01\\
18.55	0.01\\
18.56	0.01\\
18.57	0.01\\
18.58	0.01\\
18.59	0.01\\
18.6	0.01\\
18.61	0.01\\
18.62	0.01\\
18.63	0.01\\
18.64	0.01\\
18.65	0.01\\
18.66	0.01\\
18.67	0.01\\
18.68	0.01\\
18.69	0.01\\
18.7	0.01\\
18.71	0.01\\
18.72	0.01\\
18.73	0.01\\
18.74	0.01\\
18.75	0.01\\
18.76	0.01\\
18.77	0.01\\
18.78	0.01\\
18.79	0.01\\
18.8	0.01\\
18.81	0.01\\
18.82	0.01\\
18.83	0.01\\
18.84	0.01\\
18.85	0.01\\
18.86	0.01\\
18.87	0.01\\
18.88	0.01\\
18.89	0.01\\
18.9	0.01\\
18.91	0.01\\
18.92	0.01\\
18.93	0.01\\
18.94	0.01\\
18.95	0.01\\
18.96	0.01\\
18.97	0.01\\
18.98	0.01\\
18.99	0.01\\
19	0.01\\
19.01	0.01\\
19.02	0.01\\
19.03	0.01\\
19.04	0.01\\
19.05	0.01\\
19.06	0.01\\
19.07	0.01\\
19.08	0.01\\
19.09	0.01\\
19.1	0.01\\
19.11	0.01\\
19.12	0.01\\
19.13	0.01\\
19.14	0.01\\
19.15	0.01\\
19.16	0.01\\
19.17	0.01\\
19.18	0.01\\
19.19	0.01\\
19.2	0.01\\
19.21	0.01\\
19.22	0.01\\
19.23	0.01\\
19.24	0.01\\
19.25	0.01\\
19.26	0.01\\
19.27	0.01\\
19.28	0.01\\
19.29	0.01\\
19.3	0.01\\
19.31	0.01\\
19.32	0.01\\
19.33	0.01\\
19.34	0.01\\
19.35	0.01\\
19.36	0.01\\
19.37	0.01\\
19.38	0.01\\
19.39	0.01\\
19.4	0.01\\
19.41	0.01\\
19.42	0.01\\
19.43	0.01\\
19.44	0.01\\
19.45	0.01\\
19.46	0.01\\
19.47	0.01\\
19.48	0.01\\
19.49	0.01\\
19.5	0.01\\
19.51	0.01\\
19.52	0.01\\
19.53	0.01\\
19.54	0.01\\
19.55	0.01\\
19.56	0.01\\
19.57	0.01\\
19.58	0.01\\
19.59	0.01\\
19.6	0.01\\
19.61	0.01\\
19.62	0.01\\
19.63	0.01\\
19.64	0.01\\
19.65	0.01\\
19.66	0.01\\
19.67	0.01\\
19.68	0.01\\
19.69	0.01\\
19.7	0.01\\
19.71	0.01\\
19.72	0.01\\
19.73	0.01\\
19.74	0.01\\
19.75	0.01\\
19.76	0.01\\
19.77	0.01\\
19.78	0.01\\
19.79	0.01\\
19.8	0.01\\
19.81	0.01\\
19.82	0.01\\
19.83	0.01\\
19.84	0.01\\
19.85	0.01\\
19.86	0.01\\
19.87	0.01\\
19.88	0.01\\
19.89	0.01\\
19.9	0.01\\
19.91	0.01\\
19.92	0.01\\
19.93	0.01\\
19.94	0.01\\
19.95	0.01\\
19.96	0.01\\
19.97	0.01\\
19.98	0.01\\
19.99	0.01\\
20	0.01\\
20.01	0.01\\
20.02	0.01\\
20.03	0.01\\
20.04	0.01\\
20.05	0.01\\
20.06	0.01\\
20.07	0.01\\
20.08	0.01\\
20.09	0.01\\
20.1	0.01\\
20.11	0.01\\
20.12	0.01\\
20.13	0.01\\
20.14	0.01\\
20.15	0.01\\
20.16	0.01\\
20.17	0.01\\
20.18	0.01\\
20.19	0.01\\
20.2	0.01\\
20.21	0.01\\
20.22	0.01\\
20.23	0.01\\
20.24	0.01\\
20.25	0.01\\
20.26	0.01\\
20.27	0.01\\
20.28	0.01\\
20.29	0.01\\
20.3	0.01\\
20.31	0.01\\
20.32	0.01\\
20.33	0.01\\
20.34	0.01\\
20.35	0.01\\
20.36	0.01\\
20.37	0.01\\
20.38	0.01\\
20.39	0.01\\
20.4	0.01\\
20.41	0.01\\
20.42	0.01\\
20.43	0.01\\
20.44	0.01\\
20.45	0.01\\
20.46	0.01\\
20.47	0.01\\
20.48	0.01\\
20.49	0.01\\
20.5	0.01\\
20.51	0.01\\
20.52	0.01\\
20.53	0.01\\
20.54	0.01\\
20.55	0.01\\
20.56	0.01\\
20.57	0.01\\
20.58	0.01\\
20.59	0.01\\
20.6	0.01\\
20.61	0.01\\
20.62	0.01\\
20.63	0.01\\
20.64	0.01\\
20.65	0.01\\
20.66	0.01\\
20.67	0.01\\
20.68	0.01\\
20.69	0.01\\
20.7	0.01\\
20.71	0.01\\
20.72	0.01\\
20.73	0.01\\
20.74	0.01\\
20.75	0.01\\
20.76	0.01\\
20.77	0.01\\
20.78	0.01\\
20.79	0.01\\
20.8	0.01\\
20.81	0.01\\
20.82	0.01\\
20.83	0.01\\
20.84	0.01\\
20.85	0.01\\
20.86	0.01\\
20.87	0.01\\
20.88	0.01\\
20.89	0.01\\
20.9	0.01\\
20.91	0.01\\
20.92	0.01\\
20.93	0.01\\
20.94	0.01\\
20.95	0.01\\
20.96	0.01\\
20.97	0.01\\
20.98	0.01\\
20.99	0.01\\
21	0.01\\
21.01	0.01\\
21.02	0.01\\
21.03	0.01\\
21.04	0.01\\
21.05	0.01\\
21.06	0.01\\
21.07	0.01\\
21.08	0.01\\
21.09	0.01\\
21.1	0.01\\
21.11	0.01\\
21.12	0.01\\
21.13	0.01\\
21.14	0.01\\
21.15	0.01\\
21.16	0.01\\
21.17	0.01\\
21.18	0.01\\
21.19	0.01\\
21.2	0.01\\
21.21	0.01\\
21.22	0.01\\
21.23	0.01\\
21.24	0.01\\
21.25	0.01\\
21.26	0.01\\
21.27	0.01\\
21.28	0.01\\
21.29	0.01\\
21.3	0.01\\
21.31	0.01\\
21.32	0.01\\
21.33	0.01\\
21.34	0.01\\
21.35	0.01\\
21.36	0.01\\
21.37	0.01\\
21.38	0.01\\
21.39	0.01\\
21.4	0.01\\
21.41	0.01\\
21.42	0.01\\
21.43	0.01\\
21.44	0.01\\
21.45	0.01\\
21.46	0.01\\
21.47	0.01\\
21.48	0.01\\
21.49	0.01\\
21.5	0.01\\
21.51	0.01\\
21.52	0.01\\
21.53	0.01\\
21.54	0.01\\
21.55	0.01\\
21.56	0.01\\
21.57	0.01\\
21.58	0.01\\
21.59	0.01\\
21.6	0.01\\
21.61	0.01\\
21.62	0.01\\
21.63	0.01\\
21.64	0.01\\
21.65	0.01\\
21.66	0.01\\
21.67	0.01\\
21.68	0.01\\
21.69	0.01\\
21.7	0.01\\
21.71	0.01\\
21.72	0.01\\
21.73	0.01\\
21.74	0.01\\
21.75	0.01\\
21.76	0.01\\
21.77	0.01\\
21.78	0.01\\
21.79	0.01\\
21.8	0.01\\
21.81	0.01\\
21.82	0.01\\
21.83	0.01\\
21.84	0.01\\
21.85	0.01\\
21.86	0.01\\
21.87	0.01\\
21.88	0.01\\
21.89	0.01\\
21.9	0.01\\
21.91	0.01\\
21.92	0.01\\
21.93	0.01\\
21.94	0.01\\
21.95	0.01\\
21.96	0.01\\
21.97	0.01\\
21.98	0.01\\
21.99	0.01\\
22	0.01\\
22.01	0.01\\
22.02	0.01\\
22.03	0.01\\
22.04	0.01\\
22.05	0.01\\
22.06	0.01\\
22.07	0.01\\
22.08	0.01\\
22.09	0.01\\
22.1	0.01\\
22.11	0.01\\
22.12	0.01\\
22.13	0.01\\
22.14	0.01\\
22.15	0.01\\
22.16	0.01\\
22.17	0.01\\
22.18	0.01\\
22.19	0.01\\
22.2	0.01\\
22.21	0.01\\
22.22	0.01\\
22.23	0.01\\
22.24	0.01\\
22.25	0.01\\
22.26	0.01\\
22.27	0.01\\
22.28	0.01\\
22.29	0.01\\
22.3	0.01\\
22.31	0.01\\
22.32	0.01\\
22.33	0.01\\
22.34	0.01\\
22.35	0.01\\
22.36	0.01\\
22.37	0.01\\
22.38	0.01\\
22.39	0.01\\
22.4	0.01\\
22.41	0.01\\
22.42	0.01\\
22.43	0.01\\
22.44	0.01\\
22.45	0.01\\
22.46	0.01\\
22.47	0.01\\
22.48	0.01\\
22.49	0.01\\
22.5	0.01\\
22.51	0.01\\
22.52	0.01\\
22.53	0.01\\
22.54	0.01\\
22.55	0.01\\
22.56	0.01\\
22.57	0.01\\
22.58	0.01\\
22.59	0.01\\
22.6	0.01\\
22.61	0.01\\
22.62	0.01\\
22.63	0.01\\
22.64	0.01\\
22.65	0.01\\
22.66	0.01\\
22.67	0.01\\
22.68	0.01\\
22.69	0.01\\
22.7	0.01\\
22.71	0.01\\
22.72	0.01\\
22.73	0.01\\
22.74	0.01\\
22.75	0.01\\
22.76	0.01\\
22.77	0.01\\
22.78	0.01\\
22.79	0.01\\
22.8	0.01\\
22.81	0.01\\
22.82	0.01\\
22.83	0.01\\
22.84	0.01\\
22.85	0.01\\
22.86	0.01\\
22.87	0.01\\
22.88	0.01\\
22.89	0.01\\
22.9	0.01\\
22.91	0.01\\
22.92	0.01\\
22.93	0.01\\
22.94	0.01\\
22.95	0.01\\
22.96	0.01\\
22.97	0.01\\
22.98	0.01\\
22.99	0.01\\
23	0.01\\
23.01	0.01\\
23.02	0.01\\
23.03	0.01\\
23.04	0.01\\
23.05	0.01\\
23.06	0.01\\
23.07	0.01\\
23.08	0.01\\
23.09	0.01\\
23.1	0.01\\
23.11	0.01\\
23.12	0.01\\
23.13	0.01\\
23.14	0.01\\
23.15	0.01\\
23.16	0.01\\
23.17	0.01\\
23.18	0.01\\
23.19	0.01\\
23.2	0.01\\
23.21	0.01\\
23.22	0.01\\
23.23	0.01\\
23.24	0.01\\
23.25	0.01\\
23.26	0.01\\
23.27	0.01\\
23.28	0.01\\
23.29	0.01\\
23.3	0.01\\
23.31	0.01\\
23.32	0.01\\
23.33	0.01\\
23.34	0.01\\
23.35	0.01\\
23.36	0.01\\
23.37	0.01\\
23.38	0.01\\
23.39	0.01\\
23.4	0.01\\
23.41	0.01\\
23.42	0.01\\
23.43	0.01\\
23.44	0.01\\
23.45	0.01\\
23.46	0.01\\
23.47	0.01\\
23.48	0.01\\
23.49	0.01\\
23.5	0.01\\
23.51	0.01\\
23.52	0.01\\
23.53	0.01\\
23.54	0.01\\
23.55	0.01\\
23.56	0.01\\
23.57	0.01\\
23.58	0.01\\
23.59	0.01\\
23.6	0.01\\
23.61	0.01\\
23.62	0.01\\
23.63	0.01\\
23.64	0.01\\
23.65	0.01\\
23.66	0.01\\
23.67	0.01\\
23.68	0.01\\
23.69	0.01\\
23.7	0.01\\
23.71	0.01\\
23.72	0.01\\
23.73	0.01\\
23.74	0.01\\
23.75	0.01\\
23.76	0.01\\
23.77	0.01\\
23.78	0.01\\
23.79	0.01\\
23.8	0.01\\
23.81	0.01\\
23.82	0.01\\
23.83	0.01\\
23.84	0.01\\
23.85	0.01\\
23.86	0.01\\
23.87	0.01\\
23.88	0.01\\
23.89	0.01\\
23.9	0.01\\
23.91	0.01\\
23.92	0.01\\
23.93	0.01\\
23.94	0.01\\
23.95	0.01\\
23.96	0.01\\
23.97	0.01\\
23.98	0.01\\
23.99	0.01\\
24	0.01\\
24.01	0.01\\
24.02	0.01\\
24.03	0.01\\
24.04	0.01\\
24.05	0.01\\
24.06	0.01\\
24.07	0.01\\
24.08	0.01\\
24.09	0.01\\
24.1	0.01\\
24.11	0.01\\
24.12	0.01\\
24.13	0.01\\
24.14	0.01\\
24.15	0.01\\
24.16	0.01\\
24.17	0.01\\
24.18	0.01\\
24.19	0.01\\
24.2	0.01\\
24.21	0.01\\
24.22	0.01\\
24.23	0.01\\
24.24	0.01\\
24.25	0.01\\
24.26	0.01\\
24.27	0.01\\
24.28	0.01\\
24.29	0.01\\
24.3	0.01\\
24.31	0.01\\
24.32	0.01\\
24.33	0.01\\
24.34	0.01\\
24.35	0.01\\
24.36	0.01\\
24.37	0.01\\
24.38	0.01\\
24.39	0.01\\
24.4	0.01\\
24.41	0.01\\
24.42	0.01\\
24.43	0.01\\
24.44	0.01\\
24.45	0.01\\
24.46	0.01\\
24.47	0.01\\
24.48	0.01\\
24.49	0.01\\
24.5	0.01\\
24.51	0.01\\
24.52	0.01\\
24.53	0.01\\
24.54	0.01\\
24.55	0.01\\
24.56	0.01\\
24.57	0.01\\
24.58	0.01\\
24.59	0.01\\
24.6	0.01\\
24.61	0.01\\
24.62	0.01\\
24.63	0.01\\
24.64	0.01\\
24.65	0.01\\
24.66	0.01\\
24.67	0.01\\
24.68	0.01\\
24.69	0.01\\
24.7	0.01\\
24.71	0.01\\
24.72	0.01\\
24.73	0.01\\
24.74	0.01\\
24.75	0.01\\
24.76	0.01\\
24.77	0.01\\
24.78	0.01\\
24.79	0.01\\
24.8	0.01\\
24.81	0.01\\
24.82	0.01\\
24.83	0.01\\
24.84	0.01\\
24.85	0.01\\
24.86	0.01\\
24.87	0.01\\
24.88	0.01\\
24.89	0.01\\
24.9	0.01\\
24.91	0.01\\
24.92	0.01\\
24.93	0.01\\
24.94	0.01\\
24.95	0.01\\
24.96	0.01\\
24.97	0.01\\
24.98	0.01\\
24.99	0.01\\
25	0.01\\
25.01	0.01\\
25.02	0.01\\
25.03	0.01\\
25.04	0.01\\
25.05	0.01\\
25.06	0.01\\
25.07	0.01\\
25.08	0.01\\
25.09	0.01\\
25.1	0.01\\
25.11	0.01\\
25.12	0.01\\
25.13	0.01\\
25.14	0.01\\
25.15	0.01\\
25.16	0.01\\
25.17	0.01\\
25.18	0.01\\
25.19	0.01\\
25.2	0.01\\
25.21	0.01\\
25.22	0.01\\
25.23	0.01\\
25.24	0.01\\
25.25	0.01\\
25.26	0.01\\
25.27	0.01\\
25.28	0.01\\
25.29	0.01\\
25.3	0.01\\
25.31	0.01\\
25.32	0.01\\
25.33	0.01\\
25.34	0.01\\
25.35	0.01\\
25.36	0.01\\
25.37	0.01\\
25.38	0.01\\
25.39	0.01\\
25.4	0.01\\
25.41	0.01\\
25.42	0.01\\
25.43	0.01\\
25.44	0.01\\
25.45	0.01\\
25.46	0.01\\
25.47	0.01\\
25.48	0.01\\
25.49	0.01\\
25.5	0.01\\
25.51	0.01\\
25.52	0.01\\
25.53	0.01\\
25.54	0.01\\
25.55	0.01\\
25.56	0.01\\
25.57	0.01\\
25.58	0.01\\
25.59	0.01\\
25.6	0.01\\
25.61	0.01\\
25.62	0.01\\
25.63	0.01\\
25.64	0.01\\
25.65	0.01\\
25.66	0.01\\
25.67	0.01\\
25.68	0.01\\
25.69	0.01\\
25.7	0.01\\
25.71	0.01\\
25.72	0.01\\
25.73	0.01\\
25.74	0.01\\
25.75	0.01\\
25.76	0.01\\
25.77	0.01\\
25.78	0.01\\
25.79	0.01\\
25.8	0.01\\
25.81	0.01\\
25.82	0.01\\
25.83	0.01\\
25.84	0.01\\
25.85	0.01\\
25.86	0.01\\
25.87	0.01\\
25.88	0.01\\
25.89	0.01\\
25.9	0.01\\
25.91	0.01\\
25.92	0.01\\
25.93	0.01\\
25.94	0.01\\
25.95	0.01\\
25.96	0.01\\
25.97	0.01\\
25.98	0.01\\
25.99	0.01\\
26	0.01\\
26.01	0.01\\
26.02	0.01\\
26.03	0.01\\
26.04	0.01\\
26.05	0.01\\
26.06	0.01\\
26.07	0.01\\
26.08	0.01\\
26.09	0.01\\
26.1	0.01\\
26.11	0.01\\
26.12	0.01\\
26.13	0.01\\
26.14	0.01\\
26.15	0.01\\
26.16	0.01\\
26.17	0.01\\
26.18	0.01\\
26.19	0.01\\
26.2	0.01\\
26.21	0.01\\
26.22	0.01\\
26.23	0.01\\
26.24	0.01\\
26.25	0.01\\
26.26	0.01\\
26.27	0.01\\
26.28	0.01\\
26.29	0.01\\
26.3	0.01\\
26.31	0.01\\
26.32	0.01\\
26.33	0.01\\
26.34	0.01\\
26.35	0.01\\
26.36	0.01\\
26.37	0.01\\
26.38	0.01\\
26.39	0.01\\
26.4	0.01\\
26.41	0.01\\
26.42	0.01\\
26.43	0.01\\
26.44	0.01\\
26.45	0.01\\
26.46	0.01\\
26.47	0.01\\
26.48	0.01\\
26.49	0.01\\
26.5	0.01\\
26.51	0.01\\
26.52	0.01\\
26.53	0.01\\
26.54	0.01\\
26.55	0.01\\
26.56	0.01\\
26.57	0.01\\
26.58	0.01\\
26.59	0.01\\
26.6	0.01\\
26.61	0.01\\
26.62	0.01\\
26.63	0.01\\
26.64	0.01\\
26.65	0.01\\
26.66	0.01\\
26.67	0.01\\
26.68	0.01\\
26.69	0.01\\
26.7	0.01\\
26.71	0.01\\
26.72	0.01\\
26.73	0.01\\
26.74	0.01\\
26.75	0.01\\
26.76	0.01\\
26.77	0.01\\
26.78	0.01\\
26.79	0.01\\
26.8	0.01\\
26.81	0.01\\
26.82	0.01\\
26.83	0.01\\
26.84	0.01\\
26.85	0.01\\
26.86	0.01\\
26.87	0.01\\
26.88	0.01\\
26.89	0.01\\
26.9	0.01\\
26.91	0.01\\
26.92	0.01\\
26.93	0.01\\
26.94	0.01\\
26.95	0.01\\
26.96	0.01\\
26.97	0.01\\
26.98	0.01\\
26.99	0.01\\
27	0.01\\
27.01	0.01\\
27.02	0.01\\
27.03	0.01\\
27.04	0.01\\
27.05	0.01\\
27.06	0.01\\
27.07	0.01\\
27.08	0.01\\
27.09	0.01\\
27.1	0.01\\
27.11	0.01\\
27.12	0.01\\
27.13	0.01\\
27.14	0.01\\
27.15	0.01\\
27.16	0.01\\
27.17	0.01\\
27.18	0.01\\
27.19	0.01\\
27.2	0.01\\
27.21	0.01\\
27.22	0.01\\
27.23	0.01\\
27.24	0.01\\
27.25	0.01\\
27.26	0.01\\
27.27	0.01\\
27.28	0.01\\
27.29	0.01\\
27.3	0.01\\
27.31	0.01\\
27.32	0.01\\
27.33	0.01\\
27.34	0.01\\
27.35	0.01\\
27.36	0.01\\
27.37	0.01\\
27.38	0.01\\
27.39	0.01\\
27.4	0.01\\
27.41	0.01\\
27.42	0.01\\
27.43	0.01\\
27.44	0.01\\
27.45	0.01\\
27.46	0.01\\
27.47	0.01\\
27.48	0.01\\
27.49	0.01\\
27.5	0.01\\
27.51	0.01\\
27.52	0.01\\
27.53	0.01\\
27.54	0.01\\
27.55	0.01\\
27.56	0.01\\
27.57	0.01\\
27.58	0.01\\
27.59	0.01\\
27.6	0.01\\
27.61	0.01\\
27.62	0.01\\
27.63	0.01\\
27.64	0.01\\
27.65	0.01\\
27.66	0.01\\
27.67	0.01\\
27.68	0.01\\
27.69	0.01\\
27.7	0.01\\
27.71	0.01\\
27.72	0.01\\
27.73	0.01\\
27.74	0.01\\
27.75	0.01\\
27.76	0.01\\
27.77	0.01\\
27.78	0.01\\
27.79	0.01\\
27.8	0.01\\
27.81	0.01\\
27.82	0.01\\
27.83	0.01\\
27.84	0.01\\
27.85	0.01\\
27.86	0.01\\
27.87	0.01\\
27.88	0.01\\
27.89	0.01\\
27.9	0.01\\
27.91	0.01\\
27.92	0.01\\
27.93	0.01\\
27.94	0.01\\
27.95	0.01\\
27.96	0.01\\
27.97	0.01\\
27.98	0.01\\
27.99	0.01\\
28	0.01\\
28.01	0.01\\
28.02	0.01\\
28.03	0.01\\
28.04	0.01\\
28.05	0.01\\
28.06	0.01\\
28.07	0.01\\
28.08	0.01\\
28.09	0.01\\
28.1	0.01\\
28.11	0.01\\
28.12	0.01\\
28.13	0.01\\
28.14	0.01\\
28.15	0.01\\
28.16	0.01\\
28.17	0.01\\
28.18	0.01\\
28.19	0.01\\
28.2	0.01\\
28.21	0.01\\
28.22	0.01\\
28.23	0.01\\
28.24	0.01\\
28.25	0.01\\
28.26	0.01\\
28.27	0.01\\
28.28	0.01\\
28.29	0.01\\
28.3	0.01\\
28.31	0.01\\
28.32	0.01\\
28.33	0.01\\
28.34	0.01\\
28.35	0.01\\
28.36	0.01\\
28.37	0.01\\
28.38	0.01\\
28.39	0.01\\
28.4	0.01\\
28.41	0.01\\
28.42	0.01\\
28.43	0.01\\
28.44	0.01\\
28.45	0.01\\
28.46	0.01\\
28.47	0.01\\
28.48	0.01\\
28.49	0.01\\
28.5	0.01\\
28.51	0.01\\
28.52	0.01\\
28.53	0.01\\
28.54	0.01\\
28.55	0.01\\
28.56	0.01\\
28.57	0.01\\
28.58	0.01\\
28.59	0.01\\
28.6	0.01\\
28.61	0.01\\
28.62	0.01\\
28.63	0.01\\
28.64	0.01\\
28.65	0.01\\
28.66	0.01\\
28.67	0.01\\
28.68	0.01\\
28.69	0.01\\
28.7	0.01\\
28.71	0.01\\
28.72	0.01\\
28.73	0.01\\
28.74	0.01\\
28.75	0.01\\
28.76	0.01\\
28.77	0.01\\
28.78	0.01\\
28.79	0.01\\
28.8	0.01\\
28.81	0.01\\
28.82	0.01\\
28.83	0.01\\
28.84	0.01\\
28.85	0.01\\
28.86	0.01\\
28.87	0.01\\
28.88	0.01\\
28.89	0.01\\
28.9	0.01\\
28.91	0.01\\
28.92	0.01\\
28.93	0.01\\
28.94	0.01\\
28.95	0.01\\
28.96	0.01\\
28.97	0.01\\
28.98	0.01\\
28.99	0.01\\
29	0.01\\
29.01	0.01\\
29.02	0.01\\
29.03	0.01\\
29.04	0.01\\
29.05	0.01\\
29.06	0.01\\
29.07	0.01\\
29.08	0.01\\
29.09	0.01\\
29.1	0.01\\
29.11	0.01\\
29.12	0.01\\
29.13	0.01\\
29.14	0.01\\
29.15	0.01\\
29.16	0.01\\
29.17	0.01\\
29.18	0.01\\
29.19	0.01\\
29.2	0.01\\
29.21	0.01\\
29.22	0.01\\
29.23	0.01\\
29.24	0.01\\
29.25	0.01\\
29.26	0.01\\
29.27	0.01\\
29.28	0.01\\
29.29	0.01\\
29.3	0.01\\
29.31	0.01\\
29.32	0.01\\
29.33	0.01\\
29.34	0.01\\
29.35	0.01\\
29.36	0.01\\
29.37	0.01\\
29.38	0.01\\
29.39	0.01\\
29.4	0.01\\
29.41	0.01\\
29.42	0.01\\
29.43	0.01\\
29.44	0.01\\
29.45	0.01\\
29.46	0.01\\
29.47	0.01\\
29.48	0.01\\
29.49	0.01\\
29.5	0.01\\
29.51	0.01\\
29.52	0.01\\
29.53	0.01\\
29.54	0.01\\
29.55	0.01\\
29.56	0.01\\
29.57	0.01\\
29.58	0.01\\
29.59	0.01\\
29.6	0.01\\
29.61	0.01\\
29.62	0.01\\
29.63	0.01\\
29.64	0.01\\
29.65	0.01\\
29.66	0.01\\
29.67	0.01\\
29.68	0.01\\
29.69	0.01\\
29.7	0.01\\
29.71	0.01\\
29.72	0.01\\
29.73	0.01\\
29.74	0.01\\
29.75	0.01\\
29.76	0.01\\
29.77	0.01\\
29.78	0.01\\
29.79	0.01\\
29.8	0.01\\
29.81	0.01\\
29.82	0.01\\
29.83	0.01\\
29.84	0.01\\
29.85	0.01\\
29.86	0.01\\
29.87	0.01\\
29.88	0.01\\
29.89	0.01\\
29.9	0.01\\
29.91	0.01\\
29.92	0.01\\
29.93	0.01\\
29.94	0.01\\
29.95	0.01\\
29.96	0.01\\
29.97	0.01\\
29.98	0.01\\
29.99	0.01\\
30	0.01\\
30.01	0.01\\
30.02	0.01\\
30.03	0.01\\
30.04	0.01\\
30.05	0.01\\
30.06	0.01\\
30.07	0.01\\
30.08	0.01\\
30.09	0.01\\
30.1	0.01\\
30.11	0.01\\
30.12	0.01\\
30.13	0.01\\
30.14	0.01\\
30.15	0.01\\
30.16	0.01\\
30.17	0.01\\
30.18	0.01\\
30.19	0.01\\
30.2	0.01\\
30.21	0.01\\
30.22	0.01\\
30.23	0.01\\
30.24	0.01\\
30.25	0.01\\
30.26	0.01\\
30.27	0.01\\
30.28	0.01\\
30.29	0.01\\
30.3	0.01\\
30.31	0.01\\
30.32	0.01\\
30.33	0.01\\
30.34	0.01\\
30.35	0.01\\
30.36	0.01\\
30.37	0.01\\
30.38	0.01\\
30.39	0.01\\
30.4	0.01\\
30.41	0.01\\
30.42	0.01\\
30.43	0.01\\
30.44	0.01\\
30.45	0.01\\
30.46	0.01\\
30.47	0.01\\
30.48	0.01\\
30.49	0.01\\
30.5	0.01\\
30.51	0.01\\
30.52	0.01\\
30.53	0.01\\
30.54	0.01\\
30.55	0.01\\
30.56	0.01\\
30.57	0.01\\
30.58	0.01\\
30.59	0.01\\
30.6	0.01\\
30.61	0.01\\
30.62	0.01\\
30.63	0.01\\
30.64	0.01\\
30.65	0.01\\
30.66	0.01\\
30.67	0.01\\
30.68	0.01\\
30.69	0.01\\
30.7	0.01\\
30.71	0.01\\
30.72	0.01\\
30.73	0.01\\
30.74	0.01\\
30.75	0.01\\
30.76	0.01\\
30.77	0.01\\
30.78	0.01\\
30.79	0.01\\
30.8	0.01\\
30.81	0.01\\
30.82	0.01\\
30.83	0.01\\
30.84	0.01\\
30.85	0.01\\
30.86	0.01\\
30.87	0.01\\
30.88	0.01\\
30.89	0.01\\
30.9	0.01\\
30.91	0.01\\
30.92	0.01\\
30.93	0.01\\
30.94	0.01\\
30.95	0.01\\
30.96	0.01\\
30.97	0.01\\
30.98	0.01\\
30.99	0.01\\
31	0.01\\
31.01	0.01\\
31.02	0.01\\
31.03	0.01\\
31.04	0.01\\
31.05	0.01\\
31.06	0.01\\
31.07	0.01\\
31.08	0.01\\
31.09	0.01\\
31.1	0.01\\
31.11	0.01\\
31.12	0.01\\
31.13	0.01\\
31.14	0.01\\
31.15	0.01\\
31.16	0.01\\
31.17	0.01\\
31.18	0.01\\
31.19	0.01\\
31.2	0.01\\
31.21	0.01\\
31.22	0.01\\
31.23	0.01\\
31.24	0.01\\
31.25	0.01\\
31.26	0.01\\
31.27	0.01\\
31.28	0.01\\
31.29	0.01\\
31.3	0.01\\
31.31	0.01\\
31.32	0.01\\
31.33	0.01\\
31.34	0.01\\
31.35	0.01\\
31.36	0.01\\
31.37	0.01\\
31.38	0.01\\
31.39	0.01\\
31.4	0.01\\
31.41	0.01\\
31.42	0.01\\
31.43	0.01\\
31.44	0.01\\
31.45	0.01\\
31.46	0.01\\
31.47	0.01\\
31.48	0.01\\
31.49	0.01\\
31.5	0.01\\
31.51	0.01\\
31.52	0.01\\
31.53	0.01\\
31.54	0.01\\
31.55	0.01\\
31.56	0.01\\
31.57	0.01\\
31.58	0.01\\
31.59	0.01\\
31.6	0.01\\
31.61	0.01\\
31.62	0.01\\
31.63	0.01\\
31.64	0.01\\
31.65	0.01\\
31.66	0.01\\
31.67	0.01\\
31.68	0.01\\
31.69	0.01\\
31.7	0.01\\
31.71	0.01\\
31.72	0.01\\
31.73	0.01\\
31.74	0.01\\
31.75	0.01\\
31.76	0.01\\
31.77	0.01\\
31.78	0.01\\
31.79	0.01\\
31.8	0.01\\
31.81	0.01\\
31.82	0.01\\
31.83	0.01\\
31.84	0.01\\
31.85	0.01\\
31.86	0.01\\
31.87	0.01\\
31.88	0.01\\
31.89	0.01\\
31.9	0.01\\
31.91	0.01\\
31.92	0.01\\
31.93	0.01\\
31.94	0.01\\
31.95	0.01\\
31.96	0.01\\
31.97	0.01\\
31.98	0.01\\
31.99	0.01\\
32	0.01\\
32.01	0.01\\
32.02	0.01\\
32.03	0.01\\
32.04	0.01\\
32.05	0.01\\
32.06	0.01\\
32.07	0.01\\
32.08	0.01\\
32.09	0.01\\
32.1	0.01\\
32.11	0.01\\
32.12	0.01\\
32.13	0.01\\
32.14	0.01\\
32.15	0.01\\
32.16	0.01\\
32.17	0.01\\
32.18	0.01\\
32.19	0.01\\
32.2	0.01\\
32.21	0.01\\
32.22	0.01\\
32.23	0.01\\
32.24	0.01\\
32.25	0.01\\
32.26	0.01\\
32.27	0.01\\
32.28	0.01\\
32.29	0.01\\
32.3	0.01\\
32.31	0.01\\
32.32	0.01\\
32.33	0.01\\
32.34	0.01\\
32.35	0.01\\
32.36	0.01\\
32.37	0.01\\
32.38	0.01\\
32.39	0.01\\
32.4	0.01\\
32.41	0.01\\
32.42	0.01\\
32.43	0.01\\
32.44	0.01\\
32.45	0.01\\
32.46	0.01\\
32.47	0.01\\
32.48	0.01\\
32.49	0.01\\
32.5	0.01\\
32.51	0.01\\
32.52	0.01\\
32.53	0.01\\
32.54	0.01\\
32.55	0.01\\
32.56	0.01\\
32.57	0.01\\
32.58	0.01\\
32.59	0.01\\
32.6	0.01\\
32.61	0.01\\
32.62	0.01\\
32.63	0.01\\
32.64	0.01\\
32.65	0.01\\
32.66	0.01\\
32.67	0.01\\
32.68	0.01\\
32.69	0.01\\
32.7	0.01\\
32.71	0.01\\
32.72	0.01\\
32.73	0.01\\
32.74	0.01\\
32.75	0.01\\
32.76	0.01\\
32.77	0.01\\
32.78	0.01\\
32.79	0.01\\
32.8	0.01\\
32.81	0.01\\
32.82	0.01\\
32.83	0.01\\
32.84	0.01\\
32.85	0.01\\
32.86	0.01\\
32.87	0.01\\
32.88	0.01\\
32.89	0.01\\
32.9	0.01\\
32.91	0.01\\
32.92	0.01\\
32.93	0.01\\
32.94	0.01\\
32.95	0.01\\
32.96	0.01\\
32.97	0.01\\
32.98	0.01\\
32.99	0.01\\
33	0.01\\
33.01	0.01\\
33.02	0.01\\
33.03	0.01\\
33.04	0.01\\
33.05	0.01\\
33.06	0.01\\
33.07	0.01\\
33.08	0.01\\
33.09	0.01\\
33.1	0.01\\
33.11	0.01\\
33.12	0.01\\
33.13	0.01\\
33.14	0.01\\
33.15	0.01\\
33.16	0.01\\
33.17	0.01\\
33.18	0.01\\
33.19	0.01\\
33.2	0.01\\
33.21	0.01\\
33.22	0.01\\
33.23	0.01\\
33.24	0.01\\
33.25	0.01\\
33.26	0.01\\
33.27	0.01\\
33.28	0.01\\
33.29	0.01\\
33.3	0.01\\
33.31	0.01\\
33.32	0.01\\
33.33	0.01\\
33.34	0.01\\
33.35	0.01\\
33.36	0.01\\
33.37	0.01\\
33.38	0.01\\
33.39	0.01\\
33.4	0.01\\
33.41	0.01\\
33.42	0.01\\
33.43	0.01\\
33.44	0.01\\
33.45	0.01\\
33.46	0.01\\
33.47	0.01\\
33.48	0.01\\
33.49	0.01\\
33.5	0.01\\
33.51	0.01\\
33.52	0.01\\
33.53	0.01\\
33.54	0.01\\
33.55	0.01\\
33.56	0.01\\
33.57	0.01\\
33.58	0.01\\
33.59	0.01\\
33.6	0.01\\
33.61	0.01\\
33.62	0.01\\
33.63	0.01\\
33.64	0.01\\
33.65	0.01\\
33.66	0.01\\
33.67	0.01\\
33.68	0.01\\
33.69	0.01\\
33.7	0.01\\
33.71	0.01\\
33.72	0.01\\
33.73	0.01\\
33.74	0.01\\
33.75	0.01\\
33.76	0.01\\
33.77	0.01\\
33.78	0.01\\
33.79	0.01\\
33.8	0.01\\
33.81	0.01\\
33.82	0.01\\
33.83	0.01\\
33.84	0.01\\
33.85	0.01\\
33.86	0.01\\
33.87	0.01\\
33.88	0.01\\
33.89	0.01\\
33.9	0.01\\
33.91	0.01\\
33.92	0.01\\
33.93	0.01\\
33.94	0.01\\
33.95	0.01\\
33.96	0.01\\
33.97	0.01\\
33.98	0.01\\
33.99	0.01\\
34	0.01\\
34.01	0.01\\
34.02	0.01\\
34.03	0.01\\
34.04	0.01\\
34.05	0.01\\
34.06	0.01\\
34.07	0.01\\
34.08	0.01\\
34.09	0.01\\
34.1	0.01\\
34.11	0.01\\
34.12	0.01\\
34.13	0.01\\
34.14	0.01\\
34.15	0.01\\
34.16	0.01\\
34.17	0.01\\
34.18	0.01\\
34.19	0.01\\
34.2	0.01\\
34.21	0.01\\
34.22	0.01\\
34.23	0.01\\
34.24	0.01\\
34.25	0.01\\
34.26	0.01\\
34.27	0.01\\
34.28	0.01\\
34.29	0.01\\
34.3	0.01\\
34.31	0.01\\
34.32	0.01\\
34.33	0.01\\
34.34	0.01\\
34.35	0.01\\
34.36	0.01\\
34.37	0.01\\
34.38	0.01\\
34.39	0.01\\
34.4	0.01\\
34.41	0.01\\
34.42	0.01\\
34.43	0.01\\
34.44	0.01\\
34.45	0.01\\
34.46	0.01\\
34.47	0.01\\
34.48	0.01\\
34.49	0.01\\
34.5	0.01\\
34.51	0.01\\
34.52	0.01\\
34.53	0.01\\
34.54	0.01\\
34.55	0.01\\
34.56	0.01\\
34.57	0.01\\
34.58	0.01\\
34.59	0.01\\
34.6	0.01\\
34.61	0.01\\
34.62	0.01\\
34.63	0.01\\
34.64	0.01\\
34.65	0.01\\
34.66	0.01\\
34.67	0.01\\
34.68	0.01\\
34.69	0.01\\
34.7	0.01\\
34.71	0.01\\
34.72	0.01\\
34.73	0.01\\
34.74	0.01\\
34.75	0.01\\
34.76	0.01\\
34.77	0.01\\
34.78	0.01\\
34.79	0.01\\
34.8	0.01\\
34.81	0.01\\
34.82	0.01\\
34.83	0.01\\
34.84	0.01\\
34.85	0.01\\
34.86	0.01\\
34.87	0.01\\
34.88	0.01\\
34.89	0.01\\
34.9	0.01\\
34.91	0.01\\
34.92	0.01\\
34.93	0.01\\
34.94	0.01\\
34.95	0.01\\
34.96	0.01\\
34.97	0.01\\
34.98	0.01\\
34.99	0.01\\
35	0.01\\
35.01	0.01\\
35.02	0.01\\
35.03	0.01\\
35.04	0.01\\
35.05	0.01\\
35.06	0.01\\
35.07	0.01\\
35.08	0.01\\
35.09	0.01\\
35.1	0.01\\
35.11	0.01\\
35.12	0.01\\
35.13	0.01\\
35.14	0.01\\
35.15	0.01\\
35.16	0.01\\
35.17	0.01\\
35.18	0.01\\
35.19	0.01\\
35.2	0.01\\
35.21	0.01\\
35.22	0.01\\
35.23	0.01\\
35.24	0.01\\
35.25	0.01\\
35.26	0.01\\
35.27	0.01\\
35.28	0.01\\
35.29	0.01\\
35.3	0.01\\
35.31	0.01\\
35.32	0.01\\
35.33	0.01\\
35.34	0.01\\
35.35	0.01\\
35.36	0.01\\
35.37	0.01\\
35.38	0.01\\
35.39	0.01\\
35.4	0.01\\
35.41	0.01\\
35.42	0.01\\
35.43	0.01\\
35.44	0.01\\
35.45	0.01\\
35.46	0.01\\
35.47	0.01\\
35.48	0.01\\
35.49	0.01\\
35.5	0.01\\
35.51	0.01\\
35.52	0.01\\
35.53	0.01\\
35.54	0.01\\
35.55	0.01\\
35.56	0.01\\
35.57	0.01\\
35.58	0.01\\
35.59	0.01\\
35.6	0.01\\
35.61	0.01\\
35.62	0.01\\
35.63	0.01\\
35.64	0.01\\
35.65	0.01\\
35.66	0.01\\
35.67	0.01\\
35.68	0.01\\
35.69	0.01\\
35.7	0.01\\
35.71	0.01\\
35.72	0.01\\
35.73	0.01\\
35.74	0.01\\
35.75	0.01\\
35.76	0.01\\
35.77	0.01\\
35.78	0.01\\
35.79	0.01\\
35.8	0.01\\
35.81	0.01\\
35.82	0.01\\
35.83	0.01\\
35.84	0.01\\
35.85	0.01\\
35.86	0.01\\
35.87	0.01\\
35.88	0.01\\
35.89	0.01\\
35.9	0.01\\
35.91	0.01\\
35.92	0.01\\
35.93	0.01\\
35.94	0.01\\
35.95	0.01\\
35.96	0.01\\
35.97	0.01\\
35.98	0.01\\
35.99	0.01\\
36	0.01\\
36.01	0.01\\
36.02	0.01\\
36.03	0.01\\
36.04	0.01\\
36.05	0.01\\
36.06	0.01\\
36.07	0.01\\
36.08	0.01\\
36.09	0.01\\
36.1	0.01\\
36.11	0.01\\
36.12	0.01\\
36.13	0.01\\
36.14	0.01\\
36.15	0.01\\
36.16	0.01\\
36.17	0.01\\
36.18	0.01\\
36.19	0.01\\
36.2	0.01\\
36.21	0.01\\
36.22	0.01\\
36.23	0.01\\
36.24	0.01\\
36.25	0.01\\
36.26	0.01\\
36.27	0.01\\
36.28	0.01\\
36.29	0.01\\
36.3	0.01\\
36.31	0.01\\
36.32	0.01\\
36.33	0.01\\
36.34	0.01\\
36.35	0.01\\
36.36	0.01\\
36.37	0.01\\
36.38	0.01\\
36.39	0.01\\
36.4	0.01\\
36.41	0.01\\
36.42	0.01\\
36.43	0.01\\
36.44	0.01\\
36.45	0.01\\
36.46	0.01\\
36.47	0.01\\
36.48	0.01\\
36.49	0.01\\
36.5	0.01\\
36.51	0.01\\
36.52	0.01\\
36.53	0.01\\
36.54	0.01\\
36.55	0.01\\
36.56	0.01\\
36.57	0.01\\
36.58	0.01\\
36.59	0.01\\
36.6	0.01\\
36.61	0.01\\
36.62	0.01\\
36.63	0.01\\
36.64	0.01\\
36.65	0.01\\
36.66	0.01\\
36.67	0.01\\
36.68	0.01\\
36.69	0.01\\
36.7	0.01\\
36.71	0.01\\
36.72	0.01\\
36.73	0.01\\
36.74	0.01\\
36.75	0.01\\
36.76	0.01\\
36.77	0.01\\
36.78	0.01\\
36.79	0.01\\
36.8	0.01\\
36.81	0.01\\
36.82	0.01\\
36.83	0.01\\
36.84	0.01\\
36.85	0.01\\
36.86	0.01\\
36.87	0.01\\
36.88	0.01\\
36.89	0.01\\
36.9	0.01\\
36.91	0.01\\
36.92	0.01\\
36.93	0.01\\
36.94	0.01\\
36.95	0.01\\
36.96	0.01\\
36.97	0.01\\
36.98	0.01\\
36.99	0.01\\
37	0.01\\
37.01	0.01\\
37.02	0.01\\
37.03	0.01\\
37.04	0.01\\
37.05	0.01\\
37.06	0.01\\
37.07	0.01\\
37.08	0.01\\
37.09	0.01\\
37.1	0.01\\
37.11	0.01\\
37.12	0.01\\
37.13	0.01\\
37.14	0.01\\
37.15	0.01\\
37.16	0.01\\
37.17	0.01\\
37.18	0.01\\
37.19	0.01\\
37.2	0.01\\
37.21	0.01\\
37.22	0.01\\
37.23	0.01\\
37.24	0.01\\
37.25	0.01\\
37.26	0.01\\
37.27	0.01\\
37.28	0.01\\
37.29	0.01\\
37.3	0.01\\
37.31	0.01\\
37.32	0.01\\
37.33	0.01\\
37.34	0.01\\
37.35	0.01\\
37.36	0.01\\
37.37	0.01\\
37.38	0.01\\
37.39	0.01\\
37.4	0.01\\
37.41	0.01\\
37.42	0.01\\
37.43	0.01\\
37.44	0.01\\
37.45	0.01\\
37.46	0.01\\
37.47	0.01\\
37.48	0.01\\
37.49	0.01\\
37.5	0.01\\
37.51	0.01\\
37.52	0.01\\
37.53	0.01\\
37.54	0.01\\
37.55	0.01\\
37.56	0.01\\
37.57	0.01\\
37.58	0.01\\
37.59	0.01\\
37.6	0.01\\
37.61	0.01\\
37.62	0.01\\
37.63	0.01\\
37.64	0.01\\
37.65	0.01\\
37.66	0.01\\
37.67	0.01\\
37.68	0.01\\
37.69	0.01\\
37.7	0.01\\
37.71	0.01\\
37.72	0.01\\
37.73	0.01\\
37.74	0.01\\
37.75	0.01\\
37.76	0.01\\
37.77	0.01\\
37.78	0.01\\
37.79	0.01\\
37.8	0.01\\
37.81	0.01\\
37.82	0.01\\
37.83	0.01\\
37.84	0.01\\
37.85	0.01\\
37.86	0.01\\
37.87	0.01\\
37.88	0.01\\
37.89	0.01\\
37.9	0.01\\
37.91	0.01\\
37.92	0.01\\
37.93	0.01\\
37.94	0.01\\
37.95	0.01\\
37.96	0.01\\
37.97	0.01\\
37.98	0.01\\
37.99	0.01\\
38	0.01\\
38.01	0.01\\
38.02	0.01\\
38.03	0.01\\
38.04	0.01\\
38.05	0.01\\
38.06	0.01\\
38.07	0.01\\
38.08	0.01\\
38.09	0.01\\
38.1	0.01\\
38.11	0.01\\
38.12	0.01\\
38.13	0.01\\
38.14	0.01\\
38.15	0.01\\
38.16	0.01\\
38.17	0.01\\
38.18	0.01\\
38.19	0.01\\
38.2	0.01\\
38.21	0.01\\
38.22	0.01\\
38.23	0.01\\
38.24	0.01\\
38.25	0.01\\
38.26	0.01\\
38.27	0.01\\
38.28	0.01\\
38.29	0.01\\
38.3	0.01\\
38.31	0.01\\
38.32	0.01\\
38.33	0.01\\
38.34	0.01\\
38.35	0.01\\
38.36	0.01\\
38.37	0.01\\
38.38	0.01\\
38.39	0.01\\
38.4	0.01\\
38.41	0.01\\
38.42	0.01\\
38.43	0.01\\
38.44	0.01\\
38.45	0.01\\
38.46	0.01\\
38.47	0.01\\
38.48	0.01\\
38.49	0.01\\
38.5	0.01\\
38.51	0.01\\
38.52	0.01\\
38.53	0.01\\
38.54	0.01\\
38.55	0.01\\
38.56	0.01\\
38.57	0.01\\
38.58	0.01\\
38.59	0.01\\
38.6	0.01\\
38.61	0.01\\
38.62	0.01\\
38.63	0.01\\
38.64	0.01\\
38.65	0.01\\
38.66	0.01\\
38.67	0.01\\
38.68	0.01\\
38.69	0.01\\
38.7	0.01\\
38.71	0.01\\
38.72	0.01\\
38.73	0.01\\
38.74	0.01\\
38.75	0.01\\
38.76	0.01\\
38.77	0.01\\
38.78	0.01\\
38.79	0.01\\
38.8	0.01\\
38.81	0.01\\
38.82	0.01\\
38.83	0.01\\
38.84	0.01\\
38.85	0.01\\
38.86	0.01\\
38.87	0.01\\
38.88	0.01\\
38.89	0.01\\
38.9	0.01\\
38.91	0.01\\
38.92	0.01\\
38.93	0.01\\
38.94	0.01\\
38.95	0.01\\
38.96	0.01\\
38.97	0.01\\
38.98	0.01\\
38.99	0.01\\
39	0.01\\
39.01	0.01\\
39.02	0.01\\
39.03	0.01\\
39.04	0.01\\
39.05	0.01\\
39.06	0.01\\
39.07	0.01\\
39.08	0.01\\
39.09	0.01\\
39.1	0.01\\
39.11	0.01\\
39.12	0.01\\
39.13	0.01\\
39.14	0.01\\
39.15	0.01\\
39.16	0.01\\
39.17	0.01\\
39.18	0.01\\
39.19	0.01\\
39.2	0.01\\
39.21	0.01\\
39.22	0.01\\
39.23	0.01\\
39.24	0.01\\
39.25	0.01\\
39.26	0.01\\
39.27	0.01\\
39.28	0.01\\
39.29	0.01\\
39.3	0.01\\
39.31	0.01\\
39.32	0.01\\
39.33	0.01\\
39.34	0.01\\
39.35	0.01\\
39.36	0.01\\
39.37	0.01\\
39.38	0.01\\
39.39	0.01\\
39.4	0.01\\
39.41	0.01\\
39.42	0.01\\
39.43	0.01\\
39.44	0.01\\
39.45	0.01\\
39.46	0.01\\
39.47	0.01\\
39.48	0.01\\
39.49	0.01\\
39.5	0.01\\
39.51	0.01\\
39.52	0.01\\
39.53	0.01\\
39.54	0.01\\
39.55	0.01\\
39.56	0.01\\
39.57	0.01\\
39.58	0.01\\
39.59	0.01\\
39.6	0.01\\
39.61	0.01\\
39.62	0.01\\
39.63	0.01\\
39.64	0.01\\
39.65	0.01\\
39.66	0.01\\
39.67	0.01\\
39.68	0.01\\
39.69	0.01\\
39.7	0.01\\
39.71	0.01\\
39.72	0.01\\
39.73	0.01\\
39.74	0.01\\
39.75	0.01\\
39.76	0.01\\
39.77	0.01\\
39.78	0.01\\
39.79	0.01\\
39.8	0.01\\
39.81	0.01\\
39.82	0.01\\
39.83	0.01\\
39.84	0.01\\
39.85	0.01\\
39.86	0.01\\
39.87	0.01\\
39.88	0.01\\
39.89	0.01\\
39.9	0.01\\
39.91	0.01\\
39.92	0.01\\
39.93	0.01\\
39.94	0.01\\
39.95	0.01\\
39.96	0.01\\
39.97	0.01\\
39.98	0.01\\
39.99	0.01\\
40	0.01\\
40.01	0.01\\
};
\addplot [color=green,solid,forget plot]
  table[row sep=crcr]{%
40.01	0.01\\
40.02	0.01\\
40.03	0.01\\
40.04	0.01\\
40.05	0.01\\
40.06	0.01\\
40.07	0.01\\
40.08	0.01\\
40.09	0.01\\
40.1	0.01\\
40.11	0.01\\
40.12	0.01\\
40.13	0.01\\
40.14	0.01\\
40.15	0.01\\
40.16	0.01\\
40.17	0.01\\
40.18	0.01\\
40.19	0.01\\
40.2	0.01\\
40.21	0.01\\
40.22	0.01\\
40.23	0.01\\
40.24	0.01\\
40.25	0.01\\
40.26	0.01\\
40.27	0.01\\
40.28	0.01\\
40.29	0.01\\
40.3	0.01\\
40.31	0.01\\
40.32	0.01\\
40.33	0.01\\
40.34	0.01\\
40.35	0.01\\
40.36	0.01\\
40.37	0.01\\
40.38	0.01\\
40.39	0.01\\
40.4	0.01\\
40.41	0.01\\
40.42	0.01\\
40.43	0.01\\
40.44	0.01\\
40.45	0.01\\
40.46	0.01\\
40.47	0.01\\
40.48	0.01\\
40.49	0.01\\
40.5	0.01\\
40.51	0.01\\
40.52	0.01\\
40.53	0.01\\
40.54	0.01\\
40.55	0.01\\
40.56	0.01\\
40.57	0.01\\
40.58	0.01\\
40.59	0.01\\
40.6	0.01\\
40.61	0.01\\
40.62	0.01\\
40.63	0.01\\
40.64	0.01\\
40.65	0.01\\
40.66	0.01\\
40.67	0.01\\
40.68	0.01\\
40.69	0.01\\
40.7	0.01\\
40.71	0.01\\
40.72	0.01\\
40.73	0.01\\
40.74	0.01\\
40.75	0.01\\
40.76	0.01\\
40.77	0.01\\
40.78	0.01\\
40.79	0.01\\
40.8	0.01\\
40.81	0.01\\
40.82	0.01\\
40.83	0.01\\
40.84	0.01\\
40.85	0.01\\
40.86	0.01\\
40.87	0.01\\
40.88	0.01\\
40.89	0.01\\
40.9	0.01\\
40.91	0.01\\
40.92	0.01\\
40.93	0.01\\
40.94	0.01\\
40.95	0.01\\
40.96	0.01\\
40.97	0.01\\
40.98	0.01\\
40.99	0.01\\
41	0.01\\
41.01	0.01\\
41.02	0.01\\
41.03	0.01\\
41.04	0.01\\
41.05	0.01\\
41.06	0.01\\
41.07	0.01\\
41.08	0.01\\
41.09	0.01\\
41.1	0.01\\
41.11	0.01\\
41.12	0.01\\
41.13	0.01\\
41.14	0.01\\
41.15	0.01\\
41.16	0.01\\
41.17	0.01\\
41.18	0.01\\
41.19	0.01\\
41.2	0.01\\
41.21	0.01\\
41.22	0.01\\
41.23	0.01\\
41.24	0.01\\
41.25	0.01\\
41.26	0.01\\
41.27	0.01\\
41.28	0.01\\
41.29	0.01\\
41.3	0.01\\
41.31	0.01\\
41.32	0.01\\
41.33	0.01\\
41.34	0.01\\
41.35	0.01\\
41.36	0.01\\
41.37	0.01\\
41.38	0.01\\
41.39	0.01\\
41.4	0.01\\
41.41	0.01\\
41.42	0.01\\
41.43	0.01\\
41.44	0.01\\
41.45	0.01\\
41.46	0.01\\
41.47	0.01\\
41.48	0.01\\
41.49	0.01\\
41.5	0.01\\
41.51	0.01\\
41.52	0.01\\
41.53	0.01\\
41.54	0.01\\
41.55	0.01\\
41.56	0.01\\
41.57	0.01\\
41.58	0.01\\
41.59	0.01\\
41.6	0.01\\
41.61	0.01\\
41.62	0.01\\
41.63	0.01\\
41.64	0.01\\
41.65	0.01\\
41.66	0.01\\
41.67	0.01\\
41.68	0.01\\
41.69	0.01\\
41.7	0.01\\
41.71	0.01\\
41.72	0.01\\
41.73	0.01\\
41.74	0.01\\
41.75	0.01\\
41.76	0.01\\
41.77	0.01\\
41.78	0.01\\
41.79	0.01\\
41.8	0.01\\
41.81	0.01\\
41.82	0.01\\
41.83	0.01\\
41.84	0.01\\
41.85	0.01\\
41.86	0.01\\
41.87	0.01\\
41.88	0.01\\
41.89	0.01\\
41.9	0.01\\
41.91	0.01\\
41.92	0.01\\
41.93	0.01\\
41.94	0.01\\
41.95	0.01\\
41.96	0.01\\
41.97	0.01\\
41.98	0.01\\
41.99	0.01\\
42	0.01\\
42.01	0.01\\
42.02	0.01\\
42.03	0.01\\
42.04	0.01\\
42.05	0.01\\
42.06	0.01\\
42.07	0.01\\
42.08	0.01\\
42.09	0.01\\
42.1	0.01\\
42.11	0.01\\
42.12	0.01\\
42.13	0.01\\
42.14	0.01\\
42.15	0.01\\
42.16	0.01\\
42.17	0.01\\
42.18	0.01\\
42.19	0.01\\
42.2	0.01\\
42.21	0.01\\
42.22	0.01\\
42.23	0.01\\
42.24	0.01\\
42.25	0.01\\
42.26	0.01\\
42.27	0.01\\
42.28	0.01\\
42.29	0.01\\
42.3	0.01\\
42.31	0.01\\
42.32	0.01\\
42.33	0.01\\
42.34	0.01\\
42.35	0.01\\
42.36	0.01\\
42.37	0.01\\
42.38	0.01\\
42.39	0.01\\
42.4	0.01\\
42.41	0.01\\
42.42	0.01\\
42.43	0.01\\
42.44	0.01\\
42.45	0.01\\
42.46	0.01\\
42.47	0.01\\
42.48	0.01\\
42.49	0.01\\
42.5	0.01\\
42.51	0.01\\
42.52	0.01\\
42.53	0.01\\
42.54	0.01\\
42.55	0.01\\
42.56	0.01\\
42.57	0.01\\
42.58	0.01\\
42.59	0.01\\
42.6	0.01\\
42.61	0.01\\
42.62	0.01\\
42.63	0.01\\
42.64	0.01\\
42.65	0.01\\
42.66	0.01\\
42.67	0.01\\
42.68	0.01\\
42.69	0.01\\
42.7	0.01\\
42.71	0.01\\
42.72	0.01\\
42.73	0.01\\
42.74	0.01\\
42.75	0.01\\
42.76	0.01\\
42.77	0.01\\
42.78	0.01\\
42.79	0.01\\
42.8	0.01\\
42.81	0.01\\
42.82	0.01\\
42.83	0.01\\
42.84	0.01\\
42.85	0.01\\
42.86	0.01\\
42.87	0.01\\
42.88	0.01\\
42.89	0.01\\
42.9	0.01\\
42.91	0.01\\
42.92	0.01\\
42.93	0.01\\
42.94	0.01\\
42.95	0.01\\
42.96	0.01\\
42.97	0.01\\
42.98	0.01\\
42.99	0.01\\
43	0.01\\
43.01	0.01\\
43.02	0.01\\
43.03	0.01\\
43.04	0.01\\
43.05	0.01\\
43.06	0.01\\
43.07	0.01\\
43.08	0.01\\
43.09	0.01\\
43.1	0.01\\
43.11	0.01\\
43.12	0.01\\
43.13	0.01\\
43.14	0.01\\
43.15	0.01\\
43.16	0.01\\
43.17	0.01\\
43.18	0.01\\
43.19	0.01\\
43.2	0.01\\
43.21	0.01\\
43.22	0.01\\
43.23	0.01\\
43.24	0.01\\
43.25	0.01\\
43.26	0.01\\
43.27	0.01\\
43.28	0.01\\
43.29	0.01\\
43.3	0.01\\
43.31	0.01\\
43.32	0.01\\
43.33	0.01\\
43.34	0.01\\
43.35	0.01\\
43.36	0.01\\
43.37	0.01\\
43.38	0.01\\
43.39	0.01\\
43.4	0.01\\
43.41	0.01\\
43.42	0.01\\
43.43	0.01\\
43.44	0.01\\
43.45	0.01\\
43.46	0.01\\
43.47	0.01\\
43.48	0.01\\
43.49	0.01\\
43.5	0.01\\
43.51	0.01\\
43.52	0.01\\
43.53	0.01\\
43.54	0.01\\
43.55	0.01\\
43.56	0.01\\
43.57	0.01\\
43.58	0.01\\
43.59	0.01\\
43.6	0.01\\
43.61	0.01\\
43.62	0.01\\
43.63	0.01\\
43.64	0.01\\
43.65	0.01\\
43.66	0.01\\
43.67	0.01\\
43.68	0.01\\
43.69	0.01\\
43.7	0.01\\
43.71	0.01\\
43.72	0.01\\
43.73	0.01\\
43.74	0.01\\
43.75	0.01\\
43.76	0.01\\
43.77	0.01\\
43.78	0.01\\
43.79	0.01\\
43.8	0.01\\
43.81	0.01\\
43.82	0.01\\
43.83	0.01\\
43.84	0.01\\
43.85	0.01\\
43.86	0.01\\
43.87	0.01\\
43.88	0.01\\
43.89	0.01\\
43.9	0.01\\
43.91	0.01\\
43.92	0.01\\
43.93	0.01\\
43.94	0.01\\
43.95	0.01\\
43.96	0.01\\
43.97	0.01\\
43.98	0.01\\
43.99	0.01\\
44	0.01\\
44.01	0.01\\
44.02	0.01\\
44.03	0.01\\
44.04	0.01\\
44.05	0.01\\
44.06	0.01\\
44.07	0.01\\
44.08	0.01\\
44.09	0.01\\
44.1	0.01\\
44.11	0.01\\
44.12	0.01\\
44.13	0.01\\
44.14	0.01\\
44.15	0.01\\
44.16	0.01\\
44.17	0.01\\
44.18	0.01\\
44.19	0.01\\
44.2	0.01\\
44.21	0.01\\
44.22	0.01\\
44.23	0.01\\
44.24	0.01\\
44.25	0.01\\
44.26	0.01\\
44.27	0.01\\
44.28	0.01\\
44.29	0.01\\
44.3	0.01\\
44.31	0.01\\
44.32	0.01\\
44.33	0.01\\
44.34	0.01\\
44.35	0.01\\
44.36	0.01\\
44.37	0.01\\
44.38	0.01\\
44.39	0.01\\
44.4	0.01\\
44.41	0.01\\
44.42	0.01\\
44.43	0.01\\
44.44	0.01\\
44.45	0.01\\
44.46	0.01\\
44.47	0.01\\
44.48	0.01\\
44.49	0.01\\
44.5	0.01\\
44.51	0.01\\
44.52	0.01\\
44.53	0.01\\
44.54	0.01\\
44.55	0.01\\
44.56	0.01\\
44.57	0.01\\
44.58	0.01\\
44.59	0.01\\
44.6	0.01\\
44.61	0.01\\
44.62	0.01\\
44.63	0.01\\
44.64	0.01\\
44.65	0.01\\
44.66	0.01\\
44.67	0.01\\
44.68	0.01\\
44.69	0.01\\
44.7	0.01\\
44.71	0.01\\
44.72	0.01\\
44.73	0.01\\
44.74	0.01\\
44.75	0.01\\
44.76	0.01\\
44.77	0.01\\
44.78	0.01\\
44.79	0.01\\
44.8	0.01\\
44.81	0.01\\
44.82	0.01\\
44.83	0.01\\
44.84	0.01\\
44.85	0.01\\
44.86	0.01\\
44.87	0.01\\
44.88	0.01\\
44.89	0.01\\
44.9	0.01\\
44.91	0.01\\
44.92	0.01\\
44.93	0.01\\
44.94	0.01\\
44.95	0.01\\
44.96	0.01\\
44.97	0.01\\
44.98	0.01\\
44.99	0.01\\
45	0.01\\
45.01	0.01\\
45.02	0.01\\
45.03	0.01\\
45.04	0.01\\
45.05	0.01\\
45.06	0.01\\
45.07	0.01\\
45.08	0.01\\
45.09	0.01\\
45.1	0.01\\
45.11	0.01\\
45.12	0.01\\
45.13	0.01\\
45.14	0.01\\
45.15	0.01\\
45.16	0.01\\
45.17	0.01\\
45.18	0.01\\
45.19	0.01\\
45.2	0.01\\
45.21	0.01\\
45.22	0.01\\
45.23	0.01\\
45.24	0.01\\
45.25	0.01\\
45.26	0.01\\
45.27	0.01\\
45.28	0.01\\
45.29	0.01\\
45.3	0.01\\
45.31	0.01\\
45.32	0.01\\
45.33	0.01\\
45.34	0.01\\
45.35	0.01\\
45.36	0.01\\
45.37	0.01\\
45.38	0.01\\
45.39	0.01\\
45.4	0.01\\
45.41	0.01\\
45.42	0.01\\
45.43	0.01\\
45.44	0.01\\
45.45	0.01\\
45.46	0.01\\
45.47	0.01\\
45.48	0.01\\
45.49	0.01\\
45.5	0.01\\
45.51	0.01\\
45.52	0.01\\
45.53	0.01\\
45.54	0.01\\
45.55	0.01\\
45.56	0.01\\
45.57	0.01\\
45.58	0.01\\
45.59	0.01\\
45.6	0.01\\
45.61	0.01\\
45.62	0.01\\
45.63	0.01\\
45.64	0.01\\
45.65	0.01\\
45.66	0.01\\
45.67	0.01\\
45.68	0.01\\
45.69	0.01\\
45.7	0.01\\
45.71	0.01\\
45.72	0.01\\
45.73	0.01\\
45.74	0.01\\
45.75	0.01\\
45.76	0.01\\
45.77	0.01\\
45.78	0.01\\
45.79	0.01\\
45.8	0.01\\
45.81	0.01\\
45.82	0.01\\
45.83	0.01\\
45.84	0.01\\
45.85	0.01\\
45.86	0.01\\
45.87	0.01\\
45.88	0.01\\
45.89	0.01\\
45.9	0.01\\
45.91	0.01\\
45.92	0.01\\
45.93	0.01\\
45.94	0.01\\
45.95	0.01\\
45.96	0.01\\
45.97	0.01\\
45.98	0.01\\
45.99	0.01\\
46	0.01\\
46.01	0.01\\
46.02	0.01\\
46.03	0.01\\
46.04	0.01\\
46.05	0.01\\
46.06	0.01\\
46.07	0.01\\
46.08	0.01\\
46.09	0.01\\
46.1	0.01\\
46.11	0.01\\
46.12	0.01\\
46.13	0.01\\
46.14	0.01\\
46.15	0.01\\
46.16	0.01\\
46.17	0.01\\
46.18	0.01\\
46.19	0.01\\
46.2	0.01\\
46.21	0.01\\
46.22	0.01\\
46.23	0.01\\
46.24	0.01\\
46.25	0.01\\
46.26	0.01\\
46.27	0.01\\
46.28	0.01\\
46.29	0.01\\
46.3	0.01\\
46.31	0.01\\
46.32	0.01\\
46.33	0.01\\
46.34	0.01\\
46.35	0.01\\
46.36	0.01\\
46.37	0.01\\
46.38	0.01\\
46.39	0.01\\
46.4	0.01\\
46.41	0.01\\
46.42	0.01\\
46.43	0.01\\
46.44	0.01\\
46.45	0.01\\
46.46	0.01\\
46.47	0.01\\
46.48	0.01\\
46.49	0.01\\
46.5	0.01\\
46.51	0.01\\
46.52	0.01\\
46.53	0.01\\
46.54	0.01\\
46.55	0.01\\
46.56	0.01\\
46.57	0.01\\
46.58	0.01\\
46.59	0.01\\
46.6	0.01\\
46.61	0.01\\
46.62	0.01\\
46.63	0.01\\
46.64	0.01\\
46.65	0.01\\
46.66	0.01\\
46.67	0.01\\
46.68	0.01\\
46.69	0.01\\
46.7	0.01\\
46.71	0.01\\
46.72	0.01\\
46.73	0.01\\
46.74	0.01\\
46.75	0.01\\
46.76	0.01\\
46.77	0.01\\
46.78	0.01\\
46.79	0.01\\
46.8	0.01\\
46.81	0.01\\
46.82	0.01\\
46.83	0.01\\
46.84	0.01\\
46.85	0.01\\
46.86	0.01\\
46.87	0.01\\
46.88	0.01\\
46.89	0.01\\
46.9	0.01\\
46.91	0.01\\
46.92	0.01\\
46.93	0.01\\
46.94	0.01\\
46.95	0.01\\
46.96	0.01\\
46.97	0.01\\
46.98	0.01\\
46.99	0.01\\
47	0.01\\
47.01	0.01\\
47.02	0.01\\
47.03	0.01\\
47.04	0.01\\
47.05	0.01\\
47.06	0.01\\
47.07	0.01\\
47.08	0.01\\
47.09	0.01\\
47.1	0.01\\
47.11	0.01\\
47.12	0.01\\
47.13	0.01\\
47.14	0.01\\
47.15	0.01\\
47.16	0.01\\
47.17	0.01\\
47.18	0.01\\
47.19	0.01\\
47.2	0.01\\
47.21	0.01\\
47.22	0.01\\
47.23	0.01\\
47.24	0.01\\
47.25	0.01\\
47.26	0.01\\
47.27	0.01\\
47.28	0.01\\
47.29	0.01\\
47.3	0.01\\
47.31	0.01\\
47.32	0.01\\
47.33	0.01\\
47.34	0.01\\
47.35	0.01\\
47.36	0.01\\
47.37	0.01\\
47.38	0.01\\
47.39	0.01\\
47.4	0.01\\
47.41	0.01\\
47.42	0.01\\
47.43	0.01\\
47.44	0.01\\
47.45	0.01\\
47.46	0.01\\
47.47	0.01\\
47.48	0.01\\
47.49	0.01\\
47.5	0.01\\
47.51	0.01\\
47.52	0.01\\
47.53	0.01\\
47.54	0.01\\
47.55	0.01\\
47.56	0.01\\
47.57	0.01\\
47.58	0.01\\
47.59	0.01\\
47.6	0.01\\
47.61	0.01\\
47.62	0.01\\
47.63	0.01\\
47.64	0.01\\
47.65	0.01\\
47.66	0.01\\
47.67	0.01\\
47.68	0.01\\
47.69	0.01\\
47.7	0.01\\
47.71	0.01\\
47.72	0.01\\
47.73	0.01\\
47.74	0.01\\
47.75	0.01\\
47.76	0.01\\
47.77	0.01\\
47.78	0.01\\
47.79	0.01\\
47.8	0.01\\
47.81	0.01\\
47.82	0.01\\
47.83	0.01\\
47.84	0.01\\
47.85	0.01\\
47.86	0.01\\
47.87	0.01\\
47.88	0.01\\
47.89	0.01\\
47.9	0.01\\
47.91	0.01\\
47.92	0.01\\
47.93	0.01\\
47.94	0.01\\
47.95	0.01\\
47.96	0.01\\
47.97	0.01\\
47.98	0.01\\
47.99	0.01\\
48	0.01\\
48.01	0.01\\
48.02	0.01\\
48.03	0.01\\
48.04	0.01\\
48.05	0.01\\
48.06	0.01\\
48.07	0.01\\
48.08	0.01\\
48.09	0.01\\
48.1	0.01\\
48.11	0.01\\
48.12	0.01\\
48.13	0.01\\
48.14	0.01\\
48.15	0.01\\
48.16	0.01\\
48.17	0.01\\
48.18	0.01\\
48.19	0.01\\
48.2	0.01\\
48.21	0.01\\
48.22	0.01\\
48.23	0.01\\
48.24	0.01\\
48.25	0.01\\
48.26	0.01\\
48.27	0.01\\
48.28	0.01\\
48.29	0.01\\
48.3	0.01\\
48.31	0.01\\
48.32	0.01\\
48.33	0.01\\
48.34	0.01\\
48.35	0.01\\
48.36	0.01\\
48.37	0.01\\
48.38	0.01\\
48.39	0.01\\
48.4	0.01\\
48.41	0.01\\
48.42	0.01\\
48.43	0.01\\
48.44	0.01\\
48.45	0.01\\
48.46	0.01\\
48.47	0.01\\
48.48	0.01\\
48.49	0.01\\
48.5	0.01\\
48.51	0.01\\
48.52	0.01\\
48.53	0.01\\
48.54	0.01\\
48.55	0.01\\
48.56	0.01\\
48.57	0.01\\
48.58	0.01\\
48.59	0.01\\
48.6	0.01\\
48.61	0.01\\
48.62	0.01\\
48.63	0.01\\
48.64	0.01\\
48.65	0.01\\
48.66	0.01\\
48.67	0.01\\
48.68	0.01\\
48.69	0.01\\
48.7	0.01\\
48.71	0.01\\
48.72	0.01\\
48.73	0.01\\
48.74	0.01\\
48.75	0.01\\
48.76	0.01\\
48.77	0.01\\
48.78	0.01\\
48.79	0.01\\
48.8	0.01\\
48.81	0.01\\
48.82	0.01\\
48.83	0.01\\
48.84	0.01\\
48.85	0.01\\
48.86	0.01\\
48.87	0.01\\
48.88	0.01\\
48.89	0.01\\
48.9	0.01\\
48.91	0.01\\
48.92	0.01\\
48.93	0.01\\
48.94	0.01\\
48.95	0.01\\
48.96	0.01\\
48.97	0.01\\
48.98	0.01\\
48.99	0.01\\
49	0.01\\
49.01	0.01\\
49.02	0.01\\
49.03	0.01\\
49.04	0.01\\
49.05	0.01\\
49.06	0.01\\
49.07	0.01\\
49.08	0.01\\
49.09	0.01\\
49.1	0.01\\
49.11	0.01\\
49.12	0.01\\
49.13	0.01\\
49.14	0.01\\
49.15	0.01\\
49.16	0.01\\
49.17	0.01\\
49.18	0.01\\
49.19	0.01\\
49.2	0.01\\
49.21	0.01\\
49.22	0.01\\
49.23	0.01\\
49.24	0.01\\
49.25	0.01\\
49.26	0.01\\
49.27	0.01\\
49.28	0.01\\
49.29	0.01\\
49.3	0.01\\
49.31	0.01\\
49.32	0.01\\
49.33	0.01\\
49.34	0.01\\
49.35	0.01\\
49.36	0.01\\
49.37	0.01\\
49.38	0.01\\
49.39	0.01\\
49.4	0.01\\
49.41	0.01\\
49.42	0.01\\
49.43	0.01\\
49.44	0.01\\
49.45	0.01\\
49.46	0.01\\
49.47	0.01\\
49.48	0.01\\
49.49	0.01\\
49.5	0.01\\
49.51	0.01\\
49.52	0.01\\
49.53	0.01\\
49.54	0.01\\
49.55	0.01\\
49.56	0.01\\
49.57	0.01\\
49.58	0.01\\
49.59	0.01\\
49.6	0.01\\
49.61	0.01\\
49.62	0.01\\
49.63	0.01\\
49.64	0.01\\
49.65	0.01\\
49.66	0.01\\
49.67	0.01\\
49.68	0.01\\
49.69	0.01\\
49.7	0.01\\
49.71	0.01\\
49.72	0.01\\
49.73	0.01\\
49.74	0.01\\
49.75	0.01\\
49.76	0.01\\
49.77	0.01\\
49.78	0.01\\
49.79	0.01\\
49.8	0.01\\
49.81	0.01\\
49.82	0.01\\
49.83	0.01\\
49.84	0.01\\
49.85	0.01\\
49.86	0.01\\
49.87	0.01\\
49.88	0.01\\
49.89	0.01\\
49.9	0.01\\
49.91	0.01\\
49.92	0.01\\
49.93	0.01\\
49.94	0.01\\
49.95	0.01\\
49.96	0.01\\
49.97	0.01\\
49.98	0.01\\
49.99	0.01\\
50	0.01\\
50.01	0.01\\
50.02	0.01\\
50.03	0.01\\
50.04	0.01\\
50.05	0.01\\
50.06	0.01\\
50.07	0.01\\
50.08	0.01\\
50.09	0.01\\
50.1	0.01\\
50.11	0.01\\
50.12	0.01\\
50.13	0.01\\
50.14	0.01\\
50.15	0.01\\
50.16	0.01\\
50.17	0.01\\
50.18	0.01\\
50.19	0.01\\
50.2	0.01\\
50.21	0.01\\
50.22	0.01\\
50.23	0.01\\
50.24	0.01\\
50.25	0.01\\
50.26	0.01\\
50.27	0.01\\
50.28	0.01\\
50.29	0.01\\
50.3	0.01\\
50.31	0.01\\
50.32	0.01\\
50.33	0.01\\
50.34	0.01\\
50.35	0.01\\
50.36	0.01\\
50.37	0.01\\
50.38	0.01\\
50.39	0.01\\
50.4	0.01\\
50.41	0.01\\
50.42	0.01\\
50.43	0.01\\
50.44	0.01\\
50.45	0.01\\
50.46	0.01\\
50.47	0.01\\
50.48	0.01\\
50.49	0.01\\
50.5	0.01\\
50.51	0.01\\
50.52	0.01\\
50.53	0.01\\
50.54	0.01\\
50.55	0.01\\
50.56	0.01\\
50.57	0.01\\
50.58	0.01\\
50.59	0.01\\
50.6	0.01\\
50.61	0.01\\
50.62	0.01\\
50.63	0.01\\
50.64	0.01\\
50.65	0.01\\
50.66	0.01\\
50.67	0.01\\
50.68	0.01\\
50.69	0.01\\
50.7	0.01\\
50.71	0.01\\
50.72	0.01\\
50.73	0.01\\
50.74	0.01\\
50.75	0.01\\
50.76	0.01\\
50.77	0.01\\
50.78	0.01\\
50.79	0.01\\
50.8	0.01\\
50.81	0.01\\
50.82	0.01\\
50.83	0.01\\
50.84	0.01\\
50.85	0.01\\
50.86	0.01\\
50.87	0.01\\
50.88	0.01\\
50.89	0.01\\
50.9	0.01\\
50.91	0.01\\
50.92	0.01\\
50.93	0.01\\
50.94	0.01\\
50.95	0.01\\
50.96	0.01\\
50.97	0.01\\
50.98	0.01\\
50.99	0.01\\
51	0.01\\
51.01	0.01\\
51.02	0.01\\
51.03	0.01\\
51.04	0.01\\
51.05	0.01\\
51.06	0.01\\
51.07	0.01\\
51.08	0.01\\
51.09	0.01\\
51.1	0.01\\
51.11	0.01\\
51.12	0.01\\
51.13	0.01\\
51.14	0.01\\
51.15	0.01\\
51.16	0.01\\
51.17	0.01\\
51.18	0.01\\
51.19	0.01\\
51.2	0.01\\
51.21	0.01\\
51.22	0.01\\
51.23	0.01\\
51.24	0.01\\
51.25	0.01\\
51.26	0.01\\
51.27	0.01\\
51.28	0.01\\
51.29	0.01\\
51.3	0.01\\
51.31	0.01\\
51.32	0.01\\
51.33	0.01\\
51.34	0.01\\
51.35	0.01\\
51.36	0.01\\
51.37	0.01\\
51.38	0.01\\
51.39	0.01\\
51.4	0.01\\
51.41	0.01\\
51.42	0.01\\
51.43	0.01\\
51.44	0.01\\
51.45	0.01\\
51.46	0.01\\
51.47	0.01\\
51.48	0.01\\
51.49	0.01\\
51.5	0.01\\
51.51	0.01\\
51.52	0.01\\
51.53	0.01\\
51.54	0.01\\
51.55	0.01\\
51.56	0.01\\
51.57	0.01\\
51.58	0.01\\
51.59	0.01\\
51.6	0.01\\
51.61	0.01\\
51.62	0.01\\
51.63	0.01\\
51.64	0.01\\
51.65	0.01\\
51.66	0.01\\
51.67	0.01\\
51.68	0.01\\
51.69	0.01\\
51.7	0.01\\
51.71	0.01\\
51.72	0.01\\
51.73	0.01\\
51.74	0.01\\
51.75	0.01\\
51.76	0.01\\
51.77	0.01\\
51.78	0.01\\
51.79	0.01\\
51.8	0.01\\
51.81	0.01\\
51.82	0.01\\
51.83	0.01\\
51.84	0.01\\
51.85	0.01\\
51.86	0.01\\
51.87	0.01\\
51.88	0.01\\
51.89	0.01\\
51.9	0.01\\
51.91	0.01\\
51.92	0.01\\
51.93	0.01\\
51.94	0.01\\
51.95	0.01\\
51.96	0.01\\
51.97	0.01\\
51.98	0.01\\
51.99	0.01\\
52	0.01\\
52.01	0.01\\
52.02	0.01\\
52.03	0.01\\
52.04	0.01\\
52.05	0.01\\
52.06	0.01\\
52.07	0.01\\
52.08	0.01\\
52.09	0.01\\
52.1	0.01\\
52.11	0.01\\
52.12	0.01\\
52.13	0.01\\
52.14	0.01\\
52.15	0.01\\
52.16	0.01\\
52.17	0.01\\
52.18	0.01\\
52.19	0.01\\
52.2	0.01\\
52.21	0.01\\
52.22	0.01\\
52.23	0.01\\
52.24	0.01\\
52.25	0.01\\
52.26	0.01\\
52.27	0.01\\
52.28	0.01\\
52.29	0.01\\
52.3	0.01\\
52.31	0.01\\
52.32	0.01\\
52.33	0.01\\
52.34	0.01\\
52.35	0.01\\
52.36	0.01\\
52.37	0.01\\
52.38	0.01\\
52.39	0.01\\
52.4	0.01\\
52.41	0.01\\
52.42	0.01\\
52.43	0.01\\
52.44	0.01\\
52.45	0.01\\
52.46	0.01\\
52.47	0.01\\
52.48	0.01\\
52.49	0.01\\
52.5	0.01\\
52.51	0.01\\
52.52	0.01\\
52.53	0.01\\
52.54	0.01\\
52.55	0.01\\
52.56	0.01\\
52.57	0.01\\
52.58	0.01\\
52.59	0.01\\
52.6	0.01\\
52.61	0.01\\
52.62	0.01\\
52.63	0.01\\
52.64	0.01\\
52.65	0.01\\
52.66	0.01\\
52.67	0.01\\
52.68	0.01\\
52.69	0.01\\
52.7	0.01\\
52.71	0.01\\
52.72	0.01\\
52.73	0.01\\
52.74	0.01\\
52.75	0.01\\
52.76	0.01\\
52.77	0.01\\
52.78	0.01\\
52.79	0.01\\
52.8	0.01\\
52.81	0.01\\
52.82	0.01\\
52.83	0.01\\
52.84	0.01\\
52.85	0.01\\
52.86	0.01\\
52.87	0.01\\
52.88	0.01\\
52.89	0.01\\
52.9	0.01\\
52.91	0.01\\
52.92	0.01\\
52.93	0.01\\
52.94	0.01\\
52.95	0.01\\
52.96	0.01\\
52.97	0.01\\
52.98	0.01\\
52.99	0.01\\
53	0.01\\
53.01	0.01\\
53.02	0.01\\
53.03	0.01\\
53.04	0.01\\
53.05	0.01\\
53.06	0.01\\
53.07	0.01\\
53.08	0.01\\
53.09	0.01\\
53.1	0.01\\
53.11	0.01\\
53.12	0.01\\
53.13	0.01\\
53.14	0.01\\
53.15	0.01\\
53.16	0.01\\
53.17	0.01\\
53.18	0.01\\
53.19	0.01\\
53.2	0.01\\
53.21	0.01\\
53.22	0.01\\
53.23	0.01\\
53.24	0.01\\
53.25	0.01\\
53.26	0.01\\
53.27	0.01\\
53.28	0.01\\
53.29	0.01\\
53.3	0.01\\
53.31	0.01\\
53.32	0.01\\
53.33	0.01\\
53.34	0.01\\
53.35	0.01\\
53.36	0.01\\
53.37	0.01\\
53.38	0.01\\
53.39	0.01\\
53.4	0.01\\
53.41	0.01\\
53.42	0.01\\
53.43	0.01\\
53.44	0.01\\
53.45	0.01\\
53.46	0.01\\
53.47	0.01\\
53.48	0.01\\
53.49	0.01\\
53.5	0.01\\
53.51	0.01\\
53.52	0.01\\
53.53	0.01\\
53.54	0.01\\
53.55	0.01\\
53.56	0.01\\
53.57	0.01\\
53.58	0.01\\
53.59	0.01\\
53.6	0.01\\
53.61	0.01\\
53.62	0.01\\
53.63	0.01\\
53.64	0.01\\
53.65	0.01\\
53.66	0.01\\
53.67	0.01\\
53.68	0.01\\
53.69	0.01\\
53.7	0.01\\
53.71	0.01\\
53.72	0.01\\
53.73	0.01\\
53.74	0.01\\
53.75	0.01\\
53.76	0.01\\
53.77	0.01\\
53.78	0.01\\
53.79	0.01\\
53.8	0.01\\
53.81	0.01\\
53.82	0.01\\
53.83	0.01\\
53.84	0.01\\
53.85	0.01\\
53.86	0.01\\
53.87	0.01\\
53.88	0.01\\
53.89	0.01\\
53.9	0.01\\
53.91	0.01\\
53.92	0.01\\
53.93	0.01\\
53.94	0.01\\
53.95	0.01\\
53.96	0.01\\
53.97	0.01\\
53.98	0.01\\
53.99	0.01\\
54	0.01\\
54.01	0.01\\
54.02	0.01\\
54.03	0.01\\
54.04	0.01\\
54.05	0.01\\
54.06	0.01\\
54.07	0.01\\
54.08	0.01\\
54.09	0.01\\
54.1	0.01\\
54.11	0.01\\
54.12	0.01\\
54.13	0.01\\
54.14	0.01\\
54.15	0.01\\
54.16	0.01\\
54.17	0.01\\
54.18	0.01\\
54.19	0.01\\
54.2	0.01\\
54.21	0.01\\
54.22	0.01\\
54.23	0.01\\
54.24	0.01\\
54.25	0.01\\
54.26	0.01\\
54.27	0.01\\
54.28	0.01\\
54.29	0.01\\
54.3	0.01\\
54.31	0.01\\
54.32	0.01\\
54.33	0.01\\
54.34	0.01\\
54.35	0.01\\
54.36	0.01\\
54.37	0.01\\
54.38	0.01\\
54.39	0.01\\
54.4	0.01\\
54.41	0.01\\
54.42	0.01\\
54.43	0.01\\
54.44	0.01\\
54.45	0.01\\
54.46	0.01\\
54.47	0.01\\
54.48	0.01\\
54.49	0.01\\
54.5	0.01\\
54.51	0.01\\
54.52	0.01\\
54.53	0.01\\
54.54	0.01\\
54.55	0.01\\
54.56	0.01\\
54.57	0.01\\
54.58	0.01\\
54.59	0.01\\
54.6	0.01\\
54.61	0.01\\
54.62	0.01\\
54.63	0.01\\
54.64	0.01\\
54.65	0.01\\
54.66	0.01\\
54.67	0.01\\
54.68	0.01\\
54.69	0.01\\
54.7	0.01\\
54.71	0.01\\
54.72	0.01\\
54.73	0.01\\
54.74	0.01\\
54.75	0.01\\
54.76	0.01\\
54.77	0.01\\
54.78	0.01\\
54.79	0.01\\
54.8	0.01\\
54.81	0.01\\
54.82	0.01\\
54.83	0.01\\
54.84	0.01\\
54.85	0.01\\
54.86	0.01\\
54.87	0.01\\
54.88	0.01\\
54.89	0.01\\
54.9	0.01\\
54.91	0.01\\
54.92	0.01\\
54.93	0.01\\
54.94	0.01\\
54.95	0.01\\
54.96	0.01\\
54.97	0.01\\
54.98	0.01\\
54.99	0.01\\
55	0.01\\
55.01	0.01\\
55.02	0.01\\
55.03	0.01\\
55.04	0.01\\
55.05	0.01\\
55.06	0.01\\
55.07	0.01\\
55.08	0.01\\
55.09	0.01\\
55.1	0.01\\
55.11	0.01\\
55.12	0.01\\
55.13	0.01\\
55.14	0.01\\
55.15	0.01\\
55.16	0.01\\
55.17	0.01\\
55.18	0.01\\
55.19	0.01\\
55.2	0.01\\
55.21	0.01\\
55.22	0.01\\
55.23	0.01\\
55.24	0.01\\
55.25	0.01\\
55.26	0.01\\
55.27	0.01\\
55.28	0.01\\
55.29	0.01\\
55.3	0.01\\
55.31	0.01\\
55.32	0.01\\
55.33	0.01\\
55.34	0.01\\
55.35	0.01\\
55.36	0.01\\
55.37	0.01\\
55.38	0.01\\
55.39	0.01\\
55.4	0.01\\
55.41	0.01\\
55.42	0.01\\
55.43	0.01\\
55.44	0.01\\
55.45	0.01\\
55.46	0.01\\
55.47	0.01\\
55.48	0.01\\
55.49	0.01\\
55.5	0.01\\
55.51	0.01\\
55.52	0.01\\
55.53	0.01\\
55.54	0.01\\
55.55	0.01\\
55.56	0.01\\
55.57	0.01\\
55.58	0.01\\
55.59	0.01\\
55.6	0.01\\
55.61	0.01\\
55.62	0.01\\
55.63	0.01\\
55.64	0.01\\
55.65	0.01\\
55.66	0.01\\
55.67	0.01\\
55.68	0.01\\
55.69	0.01\\
55.7	0.01\\
55.71	0.01\\
55.72	0.01\\
55.73	0.01\\
55.74	0.01\\
55.75	0.01\\
55.76	0.01\\
55.77	0.01\\
55.78	0.01\\
55.79	0.01\\
55.8	0.01\\
55.81	0.01\\
55.82	0.01\\
55.83	0.01\\
55.84	0.01\\
55.85	0.01\\
55.86	0.01\\
55.87	0.01\\
55.88	0.01\\
55.89	0.01\\
55.9	0.01\\
55.91	0.01\\
55.92	0.01\\
55.93	0.01\\
55.94	0.01\\
55.95	0.01\\
55.96	0.01\\
55.97	0.01\\
55.98	0.01\\
55.99	0.01\\
56	0.01\\
56.01	0.01\\
56.02	0.01\\
56.03	0.01\\
56.04	0.01\\
56.05	0.01\\
56.06	0.01\\
56.07	0.01\\
56.08	0.01\\
56.09	0.01\\
56.1	0.01\\
56.11	0.01\\
56.12	0.01\\
56.13	0.01\\
56.14	0.01\\
56.15	0.01\\
56.16	0.01\\
56.17	0.01\\
56.18	0.01\\
56.19	0.01\\
56.2	0.01\\
56.21	0.01\\
56.22	0.01\\
56.23	0.01\\
56.24	0.01\\
56.25	0.01\\
56.26	0.01\\
56.27	0.01\\
56.28	0.01\\
56.29	0.01\\
56.3	0.01\\
56.31	0.01\\
56.32	0.01\\
56.33	0.01\\
56.34	0.01\\
56.35	0.01\\
56.36	0.01\\
56.37	0.01\\
56.38	0.01\\
56.39	0.01\\
56.4	0.01\\
56.41	0.01\\
56.42	0.01\\
56.43	0.01\\
56.44	0.01\\
56.45	0.01\\
56.46	0.01\\
56.47	0.01\\
56.48	0.01\\
56.49	0.01\\
56.5	0.01\\
56.51	0.01\\
56.52	0.01\\
56.53	0.01\\
56.54	0.01\\
56.55	0.01\\
56.56	0.01\\
56.57	0.01\\
56.58	0.01\\
56.59	0.01\\
56.6	0.01\\
56.61	0.01\\
56.62	0.01\\
56.63	0.01\\
56.64	0.01\\
56.65	0.01\\
56.66	0.01\\
56.67	0.01\\
56.68	0.01\\
56.69	0.01\\
56.7	0.01\\
56.71	0.01\\
56.72	0.01\\
56.73	0.01\\
56.74	0.01\\
56.75	0.01\\
56.76	0.01\\
56.77	0.01\\
56.78	0.01\\
56.79	0.01\\
56.8	0.01\\
56.81	0.01\\
56.82	0.01\\
56.83	0.01\\
56.84	0.01\\
56.85	0.01\\
56.86	0.01\\
56.87	0.01\\
56.88	0.01\\
56.89	0.01\\
56.9	0.01\\
56.91	0.01\\
56.92	0.01\\
56.93	0.01\\
56.94	0.01\\
56.95	0.01\\
56.96	0.01\\
56.97	0.01\\
56.98	0.01\\
56.99	0.01\\
57	0.01\\
57.01	0.01\\
57.02	0.01\\
57.03	0.01\\
57.04	0.01\\
57.05	0.01\\
57.06	0.01\\
57.07	0.01\\
57.08	0.01\\
57.09	0.01\\
57.1	0.01\\
57.11	0.01\\
57.12	0.01\\
57.13	0.01\\
57.14	0.01\\
57.15	0.01\\
57.16	0.01\\
57.17	0.01\\
57.18	0.01\\
57.19	0.01\\
57.2	0.01\\
57.21	0.01\\
57.22	0.01\\
57.23	0.01\\
57.24	0.01\\
57.25	0.01\\
57.26	0.01\\
57.27	0.01\\
57.28	0.01\\
57.29	0.01\\
57.3	0.01\\
57.31	0.01\\
57.32	0.01\\
57.33	0.01\\
57.34	0.01\\
57.35	0.01\\
57.36	0.01\\
57.37	0.01\\
57.38	0.01\\
57.39	0.01\\
57.4	0.01\\
57.41	0.01\\
57.42	0.01\\
57.43	0.01\\
57.44	0.01\\
57.45	0.01\\
57.46	0.01\\
57.47	0.01\\
57.48	0.01\\
57.49	0.01\\
57.5	0.01\\
57.51	0.01\\
57.52	0.01\\
57.53	0.01\\
57.54	0.01\\
57.55	0.01\\
57.56	0.01\\
57.57	0.01\\
57.58	0.01\\
57.59	0.01\\
57.6	0.01\\
57.61	0.01\\
57.62	0.01\\
57.63	0.01\\
57.64	0.01\\
57.65	0.01\\
57.66	0.01\\
57.67	0.01\\
57.68	0.01\\
57.69	0.01\\
57.7	0.01\\
57.71	0.01\\
57.72	0.01\\
57.73	0.01\\
57.74	0.01\\
57.75	0.01\\
57.76	0.01\\
57.77	0.01\\
57.78	0.01\\
57.79	0.01\\
57.8	0.01\\
57.81	0.01\\
57.82	0.01\\
57.83	0.01\\
57.84	0.01\\
57.85	0.01\\
57.86	0.01\\
57.87	0.01\\
57.88	0.01\\
57.89	0.01\\
57.9	0.01\\
57.91	0.01\\
57.92	0.01\\
57.93	0.01\\
57.94	0.01\\
57.95	0.01\\
57.96	0.01\\
57.97	0.01\\
57.98	0.01\\
57.99	0.01\\
58	0.01\\
58.01	0.01\\
58.02	0.01\\
58.03	0.01\\
58.04	0.01\\
58.05	0.01\\
58.06	0.01\\
58.07	0.01\\
58.08	0.01\\
58.09	0.01\\
58.1	0.01\\
58.11	0.01\\
58.12	0.01\\
58.13	0.01\\
58.14	0.01\\
58.15	0.01\\
58.16	0.01\\
58.17	0.01\\
58.18	0.01\\
58.19	0.01\\
58.2	0.01\\
58.21	0.01\\
58.22	0.01\\
58.23	0.01\\
58.24	0.01\\
58.25	0.01\\
58.26	0.01\\
58.27	0.01\\
58.28	0.01\\
58.29	0.01\\
58.3	0.01\\
58.31	0.01\\
58.32	0.01\\
58.33	0.01\\
58.34	0.01\\
58.35	0.01\\
58.36	0.01\\
58.37	0.01\\
58.38	0.01\\
58.39	0.01\\
58.4	0.01\\
58.41	0.01\\
58.42	0.01\\
58.43	0.01\\
58.44	0.01\\
58.45	0.01\\
58.46	0.01\\
58.47	0.01\\
58.48	0.01\\
58.49	0.01\\
58.5	0.01\\
58.51	0.01\\
58.52	0.01\\
58.53	0.01\\
58.54	0.01\\
58.55	0.01\\
58.56	0.01\\
58.57	0.01\\
58.58	0.01\\
58.59	0.01\\
58.6	0.01\\
58.61	0.01\\
58.62	0.01\\
58.63	0.01\\
58.64	0.01\\
58.65	0.01\\
58.66	0.01\\
58.67	0.01\\
58.68	0.01\\
58.69	0.01\\
58.7	0.01\\
58.71	0.01\\
58.72	0.01\\
58.73	0.01\\
58.74	0.01\\
58.75	0.01\\
58.76	0.01\\
58.77	0.01\\
58.78	0.01\\
58.79	0.01\\
58.8	0.01\\
58.81	0.01\\
58.82	0.01\\
58.83	0.01\\
58.84	0.01\\
58.85	0.01\\
58.86	0.01\\
58.87	0.01\\
58.88	0.01\\
58.89	0.01\\
58.9	0.01\\
58.91	0.01\\
58.92	0.01\\
58.93	0.01\\
58.94	0.01\\
58.95	0.01\\
58.96	0.01\\
58.97	0.01\\
58.98	0.01\\
58.99	0.01\\
59	0.01\\
59.01	0.01\\
59.02	0.01\\
59.03	0.01\\
59.04	0.01\\
59.05	0.01\\
59.06	0.01\\
59.07	0.01\\
59.08	0.01\\
59.09	0.01\\
59.1	0.01\\
59.11	0.01\\
59.12	0.01\\
59.13	0.01\\
59.14	0.01\\
59.15	0.01\\
59.16	0.01\\
59.17	0.01\\
59.18	0.01\\
59.19	0.01\\
59.2	0.01\\
59.21	0.01\\
59.22	0.01\\
59.23	0.01\\
59.24	0.01\\
59.25	0.01\\
59.26	0.01\\
59.27	0.01\\
59.28	0.01\\
59.29	0.01\\
59.3	0.01\\
59.31	0.01\\
59.32	0.01\\
59.33	0.01\\
59.34	0.01\\
59.35	0.01\\
59.36	0.01\\
59.37	0.01\\
59.38	0.01\\
59.39	0.01\\
59.4	0.01\\
59.41	0.01\\
59.42	0.01\\
59.43	0.01\\
59.44	0.01\\
59.45	0.01\\
59.46	0.01\\
59.47	0.01\\
59.48	0.01\\
59.49	0.01\\
59.5	0.01\\
59.51	0.01\\
59.52	0.01\\
59.53	0.01\\
59.54	0.01\\
59.55	0.01\\
59.56	0.01\\
59.57	0.01\\
59.58	0.01\\
59.59	0.01\\
59.6	0.01\\
59.61	0.01\\
59.62	0.01\\
59.63	0.01\\
59.64	0.01\\
59.65	0.01\\
59.66	0.01\\
59.67	0.01\\
59.68	0.01\\
59.69	0.01\\
59.7	0.01\\
59.71	0.01\\
59.72	0.01\\
59.73	0.01\\
59.74	0.01\\
59.75	0.01\\
59.76	0.01\\
59.77	0.01\\
59.78	0.01\\
59.79	0.01\\
59.8	0.01\\
59.81	0.01\\
59.82	0.01\\
59.83	0.01\\
59.84	0.01\\
59.85	0.01\\
59.86	0.01\\
59.87	0.01\\
59.88	0.01\\
59.89	0.01\\
59.9	0.01\\
59.91	0.01\\
59.92	0.01\\
59.93	0.01\\
59.94	0.01\\
59.95	0.01\\
59.96	0.01\\
59.97	0.01\\
59.98	0.01\\
59.99	0.01\\
60	0.01\\
60.01	0.01\\
60.02	0.01\\
60.03	0.01\\
60.04	0.01\\
60.05	0.01\\
60.06	0.01\\
60.07	0.01\\
60.08	0.01\\
60.09	0.01\\
60.1	0.01\\
60.11	0.01\\
60.12	0.01\\
60.13	0.01\\
60.14	0.01\\
60.15	0.01\\
60.16	0.01\\
60.17	0.01\\
60.18	0.01\\
60.19	0.01\\
60.2	0.01\\
60.21	0.01\\
60.22	0.01\\
60.23	0.01\\
60.24	0.01\\
60.25	0.01\\
60.26	0.01\\
60.27	0.01\\
60.28	0.01\\
60.29	0.01\\
60.3	0.01\\
60.31	0.01\\
60.32	0.01\\
60.33	0.01\\
60.34	0.01\\
60.35	0.01\\
60.36	0.01\\
60.37	0.01\\
60.38	0.01\\
60.39	0.01\\
60.4	0.01\\
60.41	0.01\\
60.42	0.01\\
60.43	0.01\\
60.44	0.01\\
60.45	0.01\\
60.46	0.01\\
60.47	0.01\\
60.48	0.01\\
60.49	0.01\\
60.5	0.01\\
60.51	0.01\\
60.52	0.01\\
60.53	0.01\\
60.54	0.01\\
60.55	0.01\\
60.56	0.01\\
60.57	0.01\\
60.58	0.01\\
60.59	0.01\\
60.6	0.01\\
60.61	0.01\\
60.62	0.01\\
60.63	0.01\\
60.64	0.01\\
60.65	0.01\\
60.66	0.01\\
60.67	0.01\\
60.68	0.01\\
60.69	0.01\\
60.7	0.01\\
60.71	0.01\\
60.72	0.01\\
60.73	0.01\\
60.74	0.01\\
60.75	0.01\\
60.76	0.01\\
60.77	0.01\\
60.78	0.01\\
60.79	0.01\\
60.8	0.01\\
60.81	0.01\\
60.82	0.01\\
60.83	0.01\\
60.84	0.01\\
60.85	0.01\\
60.86	0.01\\
60.87	0.01\\
60.88	0.01\\
60.89	0.01\\
60.9	0.01\\
60.91	0.01\\
60.92	0.01\\
60.93	0.01\\
60.94	0.01\\
60.95	0.01\\
60.96	0.01\\
60.97	0.01\\
60.98	0.01\\
60.99	0.01\\
61	0.01\\
61.01	0.01\\
61.02	0.01\\
61.03	0.01\\
61.04	0.01\\
61.05	0.01\\
61.06	0.01\\
61.07	0.01\\
61.08	0.01\\
61.09	0.01\\
61.1	0.01\\
61.11	0.01\\
61.12	0.01\\
61.13	0.01\\
61.14	0.01\\
61.15	0.01\\
61.16	0.01\\
61.17	0.01\\
61.18	0.01\\
61.19	0.01\\
61.2	0.01\\
61.21	0.01\\
61.22	0.01\\
61.23	0.01\\
61.24	0.01\\
61.25	0.01\\
61.26	0.01\\
61.27	0.01\\
61.28	0.01\\
61.29	0.01\\
61.3	0.01\\
61.31	0.01\\
61.32	0.01\\
61.33	0.01\\
61.34	0.01\\
61.35	0.01\\
61.36	0.01\\
61.37	0.01\\
61.38	0.01\\
61.39	0.01\\
61.4	0.01\\
61.41	0.01\\
61.42	0.01\\
61.43	0.01\\
61.44	0.01\\
61.45	0.01\\
61.46	0.01\\
61.47	0.01\\
61.48	0.01\\
61.49	0.01\\
61.5	0.01\\
61.51	0.01\\
61.52	0.01\\
61.53	0.01\\
61.54	0.01\\
61.55	0.01\\
61.56	0.01\\
61.57	0.01\\
61.58	0.01\\
61.59	0.01\\
61.6	0.01\\
61.61	0.01\\
61.62	0.01\\
61.63	0.01\\
61.64	0.01\\
61.65	0.01\\
61.66	0.01\\
61.67	0.01\\
61.68	0.01\\
61.69	0.01\\
61.7	0.01\\
61.71	0.01\\
61.72	0.01\\
61.73	0.01\\
61.74	0.01\\
61.75	0.01\\
61.76	0.01\\
61.77	0.01\\
61.78	0.01\\
61.79	0.01\\
61.8	0.01\\
61.81	0.01\\
61.82	0.01\\
61.83	0.01\\
61.84	0.01\\
61.85	0.01\\
61.86	0.01\\
61.87	0.01\\
61.88	0.01\\
61.89	0.01\\
61.9	0.01\\
61.91	0.01\\
61.92	0.01\\
61.93	0.01\\
61.94	0.01\\
61.95	0.01\\
61.96	0.01\\
61.97	0.01\\
61.98	0.01\\
61.99	0.01\\
62	0.01\\
62.01	0.01\\
62.02	0.01\\
62.03	0.01\\
62.04	0.01\\
62.05	0.01\\
62.06	0.01\\
62.07	0.01\\
62.08	0.01\\
62.09	0.01\\
62.1	0.01\\
62.11	0.01\\
62.12	0.01\\
62.13	0.01\\
62.14	0.01\\
62.15	0.01\\
62.16	0.01\\
62.17	0.01\\
62.18	0.01\\
62.19	0.01\\
62.2	0.01\\
62.21	0.01\\
62.22	0.01\\
62.23	0.01\\
62.24	0.01\\
62.25	0.01\\
62.26	0.01\\
62.27	0.01\\
62.28	0.01\\
62.29	0.01\\
62.3	0.01\\
62.31	0.01\\
62.32	0.01\\
62.33	0.01\\
62.34	0.01\\
62.35	0.01\\
62.36	0.01\\
62.37	0.01\\
62.38	0.01\\
62.39	0.01\\
62.4	0.01\\
62.41	0.01\\
62.42	0.01\\
62.43	0.01\\
62.44	0.01\\
62.45	0.01\\
62.46	0.01\\
62.47	0.01\\
62.48	0.01\\
62.49	0.01\\
62.5	0.01\\
62.51	0.01\\
62.52	0.01\\
62.53	0.01\\
62.54	0.01\\
62.55	0.01\\
62.56	0.01\\
62.57	0.01\\
62.58	0.01\\
62.59	0.01\\
62.6	0.01\\
62.61	0.01\\
62.62	0.01\\
62.63	0.01\\
62.64	0.01\\
62.65	0.01\\
62.66	0.01\\
62.67	0.01\\
62.68	0.01\\
62.69	0.01\\
62.7	0.01\\
62.71	0.01\\
62.72	0.01\\
62.73	0.01\\
62.74	0.01\\
62.75	0.01\\
62.76	0.01\\
62.77	0.01\\
62.78	0.01\\
62.79	0.01\\
62.8	0.01\\
62.81	0.01\\
62.82	0.01\\
62.83	0.01\\
62.84	0.01\\
62.85	0.01\\
62.86	0.01\\
62.87	0.01\\
62.88	0.01\\
62.89	0.01\\
62.9	0.01\\
62.91	0.01\\
62.92	0.01\\
62.93	0.01\\
62.94	0.01\\
62.95	0.01\\
62.96	0.01\\
62.97	0.01\\
62.98	0.01\\
62.99	0.01\\
63	0.01\\
63.01	0.01\\
63.02	0.01\\
63.03	0.01\\
63.04	0.01\\
63.05	0.01\\
63.06	0.01\\
63.07	0.01\\
63.08	0.01\\
63.09	0.01\\
63.1	0.01\\
63.11	0.01\\
63.12	0.01\\
63.13	0.01\\
63.14	0.01\\
63.15	0.01\\
63.16	0.01\\
63.17	0.01\\
63.18	0.01\\
63.19	0.01\\
63.2	0.01\\
63.21	0.01\\
63.22	0.01\\
63.23	0.01\\
63.24	0.01\\
63.25	0.01\\
63.26	0.01\\
63.27	0.01\\
63.28	0.01\\
63.29	0.01\\
63.3	0.01\\
63.31	0.01\\
63.32	0.01\\
63.33	0.01\\
63.34	0.01\\
63.35	0.01\\
63.36	0.01\\
63.37	0.01\\
63.38	0.01\\
63.39	0.01\\
63.4	0.01\\
63.41	0.01\\
63.42	0.01\\
63.43	0.01\\
63.44	0.01\\
63.45	0.01\\
63.46	0.01\\
63.47	0.01\\
63.48	0.01\\
63.49	0.01\\
63.5	0.01\\
63.51	0.01\\
63.52	0.01\\
63.53	0.01\\
63.54	0.01\\
63.55	0.01\\
63.56	0.01\\
63.57	0.01\\
63.58	0.01\\
63.59	0.01\\
63.6	0.01\\
63.61	0.01\\
63.62	0.01\\
63.63	0.01\\
63.64	0.01\\
63.65	0.01\\
63.66	0.01\\
63.67	0.01\\
63.68	0.01\\
63.69	0.01\\
63.7	0.01\\
63.71	0.01\\
63.72	0.01\\
63.73	0.01\\
63.74	0.01\\
63.75	0.01\\
63.76	0.01\\
63.77	0.01\\
63.78	0.01\\
63.79	0.01\\
63.8	0.01\\
63.81	0.01\\
63.82	0.01\\
63.83	0.01\\
63.84	0.01\\
63.85	0.01\\
63.86	0.01\\
63.87	0.01\\
63.88	0.01\\
63.89	0.01\\
63.9	0.01\\
63.91	0.01\\
63.92	0.01\\
63.93	0.01\\
63.94	0.01\\
63.95	0.01\\
63.96	0.01\\
63.97	0.01\\
63.98	0.01\\
63.99	0.01\\
64	0.01\\
64.01	0.01\\
64.02	0.01\\
64.03	0.01\\
64.04	0.01\\
64.05	0.01\\
64.06	0.01\\
64.07	0.01\\
64.08	0.01\\
64.09	0.01\\
64.1	0.01\\
64.11	0.01\\
64.12	0.01\\
64.13	0.01\\
64.14	0.01\\
64.15	0.01\\
64.16	0.01\\
64.17	0.01\\
64.18	0.01\\
64.19	0.01\\
64.2	0.01\\
64.21	0.01\\
64.22	0.01\\
64.23	0.01\\
64.24	0.01\\
64.25	0.01\\
64.26	0.01\\
64.27	0.01\\
64.28	0.01\\
64.29	0.01\\
64.3	0.01\\
64.31	0.01\\
64.32	0.01\\
64.33	0.01\\
64.34	0.01\\
64.35	0.01\\
64.36	0.01\\
64.37	0.01\\
64.38	0.01\\
64.39	0.01\\
64.4	0.01\\
64.41	0.01\\
64.42	0.01\\
64.43	0.01\\
64.44	0.01\\
64.45	0.01\\
64.46	0.01\\
64.47	0.01\\
64.48	0.01\\
64.49	0.01\\
64.5	0.01\\
64.51	0.01\\
64.52	0.01\\
64.53	0.01\\
64.54	0.01\\
64.55	0.01\\
64.56	0.01\\
64.57	0.01\\
64.58	0.01\\
64.59	0.01\\
64.6	0.01\\
64.61	0.01\\
64.62	0.01\\
64.63	0.01\\
64.64	0.01\\
64.65	0.01\\
64.66	0.01\\
64.67	0.01\\
64.68	0.01\\
64.69	0.01\\
64.7	0.01\\
64.71	0.01\\
64.72	0.01\\
64.73	0.01\\
64.74	0.01\\
64.75	0.01\\
64.76	0.01\\
64.77	0.01\\
64.78	0.01\\
64.79	0.01\\
64.8	0.01\\
64.81	0.01\\
64.82	0.01\\
64.83	0.01\\
64.84	0.01\\
64.85	0.01\\
64.86	0.01\\
64.87	0.01\\
64.88	0.01\\
64.89	0.01\\
64.9	0.01\\
64.91	0.01\\
64.92	0.01\\
64.93	0.01\\
64.94	0.01\\
64.95	0.01\\
64.96	0.01\\
64.97	0.01\\
64.98	0.01\\
64.99	0.01\\
65	0.01\\
65.01	0.01\\
65.02	0.01\\
65.03	0.01\\
65.04	0.01\\
65.05	0.01\\
65.06	0.01\\
65.07	0.01\\
65.08	0.01\\
65.09	0.01\\
65.1	0.01\\
65.11	0.01\\
65.12	0.01\\
65.13	0.01\\
65.14	0.01\\
65.15	0.01\\
65.16	0.01\\
65.17	0.01\\
65.18	0.01\\
65.19	0.01\\
65.2	0.01\\
65.21	0.01\\
65.22	0.01\\
65.23	0.01\\
65.24	0.01\\
65.25	0.01\\
65.26	0.01\\
65.27	0.01\\
65.28	0.01\\
65.29	0.01\\
65.3	0.01\\
65.31	0.01\\
65.32	0.01\\
65.33	0.01\\
65.34	0.01\\
65.35	0.01\\
65.36	0.01\\
65.37	0.01\\
65.38	0.01\\
65.39	0.01\\
65.4	0.01\\
65.41	0.01\\
65.42	0.01\\
65.43	0.01\\
65.44	0.01\\
65.45	0.01\\
65.46	0.01\\
65.47	0.01\\
65.48	0.01\\
65.49	0.01\\
65.5	0.01\\
65.51	0.01\\
65.52	0.01\\
65.53	0.01\\
65.54	0.01\\
65.55	0.01\\
65.56	0.01\\
65.57	0.01\\
65.58	0.01\\
65.59	0.01\\
65.6	0.01\\
65.61	0.01\\
65.62	0.01\\
65.63	0.01\\
65.64	0.01\\
65.65	0.01\\
65.66	0.01\\
65.67	0.01\\
65.68	0.01\\
65.69	0.01\\
65.7	0.01\\
65.71	0.01\\
65.72	0.01\\
65.73	0.01\\
65.74	0.01\\
65.75	0.01\\
65.76	0.01\\
65.77	0.01\\
65.78	0.01\\
65.79	0.01\\
65.8	0.01\\
65.81	0.01\\
65.82	0.01\\
65.83	0.01\\
65.84	0.01\\
65.85	0.01\\
65.86	0.01\\
65.87	0.01\\
65.88	0.01\\
65.89	0.01\\
65.9	0.01\\
65.91	0.01\\
65.92	0.01\\
65.93	0.01\\
65.94	0.01\\
65.95	0.01\\
65.96	0.01\\
65.97	0.01\\
65.98	0.01\\
65.99	0.01\\
66	0.01\\
66.01	0.01\\
66.02	0.01\\
66.03	0.01\\
66.04	0.01\\
66.05	0.01\\
66.06	0.01\\
66.07	0.01\\
66.08	0.01\\
66.09	0.01\\
66.1	0.01\\
66.11	0.01\\
66.12	0.01\\
66.13	0.01\\
66.14	0.01\\
66.15	0.01\\
66.16	0.01\\
66.17	0.01\\
66.18	0.01\\
66.19	0.01\\
66.2	0.01\\
66.21	0.01\\
66.22	0.01\\
66.23	0.01\\
66.24	0.01\\
66.25	0.01\\
66.26	0.01\\
66.27	0.01\\
66.28	0.01\\
66.29	0.01\\
66.3	0.01\\
66.31	0.01\\
66.32	0.01\\
66.33	0.01\\
66.34	0.01\\
66.35	0.01\\
66.36	0.01\\
66.37	0.01\\
66.38	0.01\\
66.39	0.01\\
66.4	0.01\\
66.41	0.01\\
66.42	0.01\\
66.43	0.01\\
66.44	0.01\\
66.45	0.01\\
66.46	0.01\\
66.47	0.01\\
66.48	0.01\\
66.49	0.01\\
66.5	0.01\\
66.51	0.01\\
66.52	0.01\\
66.53	0.01\\
66.54	0.01\\
66.55	0.01\\
66.56	0.01\\
66.57	0.01\\
66.58	0.01\\
66.59	0.01\\
66.6	0.01\\
66.61	0.01\\
66.62	0.01\\
66.63	0.01\\
66.64	0.01\\
66.65	0.01\\
66.66	0.01\\
66.67	0.01\\
66.68	0.01\\
66.69	0.01\\
66.7	0.01\\
66.71	0.01\\
66.72	0.01\\
66.73	0.01\\
66.74	0.01\\
66.75	0.01\\
66.76	0.01\\
66.77	0.01\\
66.78	0.01\\
66.79	0.01\\
66.8	0.01\\
66.81	0.01\\
66.82	0.01\\
66.83	0.01\\
66.84	0.01\\
66.85	0.01\\
66.86	0.01\\
66.87	0.01\\
66.88	0.01\\
66.89	0.01\\
66.9	0.01\\
66.91	0.01\\
66.92	0.01\\
66.93	0.01\\
66.94	0.01\\
66.95	0.01\\
66.96	0.01\\
66.97	0.01\\
66.98	0.01\\
66.99	0.01\\
67	0.01\\
67.01	0.01\\
67.02	0.01\\
67.03	0.01\\
67.04	0.01\\
67.05	0.01\\
67.06	0.01\\
67.07	0.01\\
67.08	0.01\\
67.09	0.01\\
67.1	0.01\\
67.11	0.01\\
67.12	0.01\\
67.13	0.01\\
67.14	0.01\\
67.15	0.01\\
67.16	0.01\\
67.17	0.01\\
67.18	0.01\\
67.19	0.01\\
67.2	0.01\\
67.21	0.01\\
67.22	0.01\\
67.23	0.01\\
67.24	0.01\\
67.25	0.01\\
67.26	0.01\\
67.27	0.01\\
67.28	0.01\\
67.29	0.01\\
67.3	0.01\\
67.31	0.01\\
67.32	0.01\\
67.33	0.01\\
67.34	0.01\\
67.35	0.01\\
67.36	0.01\\
67.37	0.01\\
67.38	0.01\\
67.39	0.01\\
67.4	0.01\\
67.41	0.01\\
67.42	0.01\\
67.43	0.01\\
67.44	0.01\\
67.45	0.01\\
67.46	0.01\\
67.47	0.01\\
67.48	0.01\\
67.49	0.01\\
67.5	0.01\\
67.51	0.01\\
67.52	0.01\\
67.53	0.01\\
67.54	0.01\\
67.55	0.01\\
67.56	0.01\\
67.57	0.01\\
67.58	0.01\\
67.59	0.01\\
67.6	0.01\\
67.61	0.01\\
67.62	0.01\\
67.63	0.01\\
67.64	0.01\\
67.65	0.01\\
67.66	0.01\\
67.67	0.01\\
67.68	0.01\\
67.69	0.01\\
67.7	0.01\\
67.71	0.01\\
67.72	0.01\\
67.73	0.01\\
67.74	0.01\\
67.75	0.01\\
67.76	0.01\\
67.77	0.01\\
67.78	0.01\\
67.79	0.01\\
67.8	0.01\\
67.81	0.01\\
67.82	0.01\\
67.83	0.01\\
67.84	0.01\\
67.85	0.01\\
67.86	0.01\\
67.87	0.01\\
67.88	0.01\\
67.89	0.01\\
67.9	0.01\\
67.91	0.01\\
67.92	0.01\\
67.93	0.01\\
67.94	0.01\\
67.95	0.01\\
67.96	0.01\\
67.97	0.01\\
67.98	0.01\\
67.99	0.01\\
68	0.01\\
68.01	0.01\\
68.02	0.01\\
68.03	0.01\\
68.04	0.01\\
68.05	0.01\\
68.06	0.01\\
68.07	0.01\\
68.08	0.01\\
68.09	0.01\\
68.1	0.01\\
68.11	0.01\\
68.12	0.01\\
68.13	0.01\\
68.14	0.01\\
68.15	0.01\\
68.16	0.01\\
68.17	0.01\\
68.18	0.01\\
68.19	0.01\\
68.2	0.01\\
68.21	0.01\\
68.22	0.01\\
68.23	0.01\\
68.24	0.01\\
68.25	0.01\\
68.26	0.01\\
68.27	0.01\\
68.28	0.01\\
68.29	0.01\\
68.3	0.01\\
68.31	0.01\\
68.32	0.01\\
68.33	0.01\\
68.34	0.01\\
68.35	0.01\\
68.36	0.01\\
68.37	0.01\\
68.38	0.01\\
68.39	0.01\\
68.4	0.01\\
68.41	0.01\\
68.42	0.01\\
68.43	0.01\\
68.44	0.01\\
68.45	0.01\\
68.46	0.01\\
68.47	0.01\\
68.48	0.01\\
68.49	0.01\\
68.5	0.01\\
68.51	0.01\\
68.52	0.01\\
68.53	0.01\\
68.54	0.01\\
68.55	0.01\\
68.56	0.01\\
68.57	0.01\\
68.58	0.01\\
68.59	0.01\\
68.6	0.01\\
68.61	0.01\\
68.62	0.01\\
68.63	0.01\\
68.64	0.01\\
68.65	0.01\\
68.66	0.01\\
68.67	0.01\\
68.68	0.01\\
68.69	0.01\\
68.7	0.01\\
68.71	0.01\\
68.72	0.01\\
68.73	0.01\\
68.74	0.01\\
68.75	0.01\\
68.76	0.01\\
68.77	0.01\\
68.78	0.01\\
68.79	0.01\\
68.8	0.01\\
68.81	0.01\\
68.82	0.01\\
68.83	0.01\\
68.84	0.01\\
68.85	0.01\\
68.86	0.01\\
68.87	0.01\\
68.88	0.01\\
68.89	0.01\\
68.9	0.01\\
68.91	0.01\\
68.92	0.01\\
68.93	0.01\\
68.94	0.01\\
68.95	0.01\\
68.96	0.01\\
68.97	0.01\\
68.98	0.01\\
68.99	0.01\\
69	0.01\\
69.01	0.01\\
69.02	0.01\\
69.03	0.01\\
69.04	0.01\\
69.05	0.01\\
69.06	0.01\\
69.07	0.01\\
69.08	0.01\\
69.09	0.01\\
69.1	0.01\\
69.11	0.01\\
69.12	0.01\\
69.13	0.01\\
69.14	0.01\\
69.15	0.01\\
69.16	0.01\\
69.17	0.01\\
69.18	0.01\\
69.19	0.01\\
69.2	0.01\\
69.21	0.01\\
69.22	0.01\\
69.23	0.01\\
69.24	0.01\\
69.25	0.01\\
69.26	0.01\\
69.27	0.01\\
69.28	0.01\\
69.29	0.01\\
69.3	0.01\\
69.31	0.01\\
69.32	0.01\\
69.33	0.01\\
69.34	0.01\\
69.35	0.01\\
69.36	0.01\\
69.37	0.01\\
69.38	0.01\\
69.39	0.01\\
69.4	0.01\\
69.41	0.01\\
69.42	0.01\\
69.43	0.01\\
69.44	0.01\\
69.45	0.01\\
69.46	0.01\\
69.47	0.01\\
69.48	0.01\\
69.49	0.01\\
69.5	0.01\\
69.51	0.01\\
69.52	0.01\\
69.53	0.01\\
69.54	0.01\\
69.55	0.01\\
69.56	0.01\\
69.57	0.01\\
69.58	0.01\\
69.59	0.01\\
69.6	0.01\\
69.61	0.01\\
69.62	0.01\\
69.63	0.01\\
69.64	0.01\\
69.65	0.01\\
69.66	0.01\\
69.67	0.01\\
69.68	0.01\\
69.69	0.01\\
69.7	0.01\\
69.71	0.01\\
69.72	0.01\\
69.73	0.01\\
69.74	0.01\\
69.75	0.01\\
69.76	0.01\\
69.77	0.01\\
69.78	0.01\\
69.79	0.01\\
69.8	0.01\\
69.81	0.01\\
69.82	0.01\\
69.83	0.01\\
69.84	0.01\\
69.85	0.01\\
69.86	0.01\\
69.87	0.01\\
69.88	0.01\\
69.89	0.01\\
69.9	0.01\\
69.91	0.01\\
69.92	0.01\\
69.93	0.01\\
69.94	0.01\\
69.95	0.01\\
69.96	0.01\\
69.97	0.01\\
69.98	0.01\\
69.99	0.01\\
70	0.01\\
70.01	0.01\\
70.02	0.01\\
70.03	0.01\\
70.04	0.01\\
70.05	0.01\\
70.06	0.01\\
70.07	0.01\\
70.08	0.01\\
70.09	0.01\\
70.1	0.01\\
70.11	0.01\\
70.12	0.01\\
70.13	0.01\\
70.14	0.01\\
70.15	0.01\\
70.16	0.01\\
70.17	0.01\\
70.18	0.01\\
70.19	0.01\\
70.2	0.01\\
70.21	0.01\\
70.22	0.01\\
70.23	0.01\\
70.24	0.01\\
70.25	0.01\\
70.26	0.01\\
70.27	0.01\\
70.28	0.01\\
70.29	0.01\\
70.3	0.01\\
70.31	0.01\\
70.32	0.01\\
70.33	0.01\\
70.34	0.01\\
70.35	0.01\\
70.36	0.01\\
70.37	0.01\\
70.38	0.01\\
70.39	0.01\\
70.4	0.01\\
70.41	0.01\\
70.42	0.01\\
70.43	0.01\\
70.44	0.01\\
70.45	0.01\\
70.46	0.01\\
70.47	0.01\\
70.48	0.01\\
70.49	0.01\\
70.5	0.01\\
70.51	0.01\\
70.52	0.01\\
70.53	0.01\\
70.54	0.01\\
70.55	0.01\\
70.56	0.01\\
70.57	0.01\\
70.58	0.01\\
70.59	0.01\\
70.6	0.01\\
70.61	0.01\\
70.62	0.01\\
70.63	0.01\\
70.64	0.01\\
70.65	0.01\\
70.66	0.01\\
70.67	0.01\\
70.68	0.01\\
70.69	0.01\\
70.7	0.01\\
70.71	0.01\\
70.72	0.01\\
70.73	0.01\\
70.74	0.01\\
70.75	0.01\\
70.76	0.01\\
70.77	0.01\\
70.78	0.01\\
70.79	0.01\\
70.8	0.01\\
70.81	0.01\\
70.82	0.01\\
70.83	0.01\\
70.84	0.01\\
70.85	0.01\\
70.86	0.01\\
70.87	0.01\\
70.88	0.01\\
70.89	0.01\\
70.9	0.01\\
70.91	0.01\\
70.92	0.01\\
70.93	0.01\\
70.94	0.01\\
70.95	0.01\\
70.96	0.01\\
70.97	0.01\\
70.98	0.01\\
70.99	0.01\\
71	0.01\\
71.01	0.01\\
71.02	0.01\\
71.03	0.01\\
71.04	0.01\\
71.05	0.01\\
71.06	0.01\\
71.07	0.01\\
71.08	0.01\\
71.09	0.01\\
71.1	0.01\\
71.11	0.01\\
71.12	0.01\\
71.13	0.01\\
71.14	0.01\\
71.15	0.01\\
71.16	0.01\\
71.17	0.01\\
71.18	0.01\\
71.19	0.01\\
71.2	0.01\\
71.21	0.01\\
71.22	0.01\\
71.23	0.01\\
71.24	0.01\\
71.25	0.01\\
71.26	0.01\\
71.27	0.01\\
71.28	0.01\\
71.29	0.01\\
71.3	0.01\\
71.31	0.01\\
71.32	0.01\\
71.33	0.01\\
71.34	0.01\\
71.35	0.01\\
71.36	0.01\\
71.37	0.01\\
71.38	0.01\\
71.39	0.01\\
71.4	0.01\\
71.41	0.01\\
71.42	0.01\\
71.43	0.01\\
71.44	0.01\\
71.45	0.01\\
71.46	0.01\\
71.47	0.01\\
71.48	0.01\\
71.49	0.01\\
71.5	0.01\\
71.51	0.01\\
71.52	0.01\\
71.53	0.01\\
71.54	0.01\\
71.55	0.01\\
71.56	0.01\\
71.57	0.01\\
71.58	0.01\\
71.59	0.01\\
71.6	0.01\\
71.61	0.01\\
71.62	0.01\\
71.63	0.01\\
71.64	0.01\\
71.65	0.01\\
71.66	0.01\\
71.67	0.01\\
71.68	0.01\\
71.69	0.01\\
71.7	0.01\\
71.71	0.01\\
71.72	0.01\\
71.73	0.01\\
71.74	0.01\\
71.75	0.01\\
71.76	0.01\\
71.77	0.01\\
71.78	0.01\\
71.79	0.01\\
71.8	0.01\\
71.81	0.01\\
71.82	0.01\\
71.83	0.01\\
71.84	0.01\\
71.85	0.01\\
71.86	0.01\\
71.87	0.01\\
71.88	0.01\\
71.89	0.01\\
71.9	0.01\\
71.91	0.01\\
71.92	0.01\\
71.93	0.01\\
71.94	0.01\\
71.95	0.01\\
71.96	0.01\\
71.97	0.01\\
71.98	0.01\\
71.99	0.01\\
72	0.01\\
72.01	0.01\\
72.02	0.01\\
72.03	0.01\\
72.04	0.01\\
72.05	0.01\\
72.06	0.01\\
72.07	0.01\\
72.08	0.01\\
72.09	0.01\\
72.1	0.01\\
72.11	0.01\\
72.12	0.01\\
72.13	0.01\\
72.14	0.01\\
72.15	0.01\\
72.16	0.01\\
72.17	0.01\\
72.18	0.01\\
72.19	0.01\\
72.2	0.01\\
72.21	0.01\\
72.22	0.01\\
72.23	0.01\\
72.24	0.01\\
72.25	0.01\\
72.26	0.01\\
72.27	0.01\\
72.28	0.01\\
72.29	0.01\\
72.3	0.01\\
72.31	0.01\\
72.32	0.01\\
72.33	0.01\\
72.34	0.01\\
72.35	0.01\\
72.36	0.01\\
72.37	0.01\\
72.38	0.01\\
72.39	0.01\\
72.4	0.01\\
72.41	0.01\\
72.42	0.01\\
72.43	0.01\\
72.44	0.01\\
72.45	0.01\\
72.46	0.01\\
72.47	0.01\\
72.48	0.01\\
72.49	0.01\\
72.5	0.01\\
72.51	0.01\\
72.52	0.01\\
72.53	0.01\\
72.54	0.01\\
72.55	0.01\\
72.56	0.01\\
72.57	0.01\\
72.58	0.01\\
72.59	0.01\\
72.6	0.01\\
72.61	0.01\\
72.62	0.01\\
72.63	0.01\\
72.64	0.01\\
72.65	0.01\\
72.66	0.01\\
72.67	0.01\\
72.68	0.01\\
72.69	0.01\\
72.7	0.01\\
72.71	0.01\\
72.72	0.01\\
72.73	0.01\\
72.74	0.01\\
72.75	0.01\\
72.76	0.01\\
72.77	0.01\\
72.78	0.01\\
72.79	0.01\\
72.8	0.01\\
72.81	0.01\\
72.82	0.01\\
72.83	0.01\\
72.84	0.01\\
72.85	0.01\\
72.86	0.01\\
72.87	0.01\\
72.88	0.01\\
72.89	0.01\\
72.9	0.01\\
72.91	0.01\\
72.92	0.01\\
72.93	0.01\\
72.94	0.01\\
72.95	0.01\\
72.96	0.01\\
72.97	0.01\\
72.98	0.01\\
72.99	0.01\\
73	0.01\\
73.01	0.01\\
73.02	0.01\\
73.03	0.01\\
73.04	0.01\\
73.05	0.01\\
73.06	0.01\\
73.07	0.01\\
73.08	0.01\\
73.09	0.01\\
73.1	0.01\\
73.11	0.01\\
73.12	0.01\\
73.13	0.01\\
73.14	0.01\\
73.15	0.01\\
73.16	0.01\\
73.17	0.01\\
73.18	0.01\\
73.19	0.01\\
73.2	0.01\\
73.21	0.01\\
73.22	0.01\\
73.23	0.01\\
73.24	0.01\\
73.25	0.01\\
73.26	0.01\\
73.27	0.01\\
73.28	0.01\\
73.29	0.01\\
73.3	0.01\\
73.31	0.01\\
73.32	0.01\\
73.33	0.01\\
73.34	0.01\\
73.35	0.01\\
73.36	0.01\\
73.37	0.01\\
73.38	0.01\\
73.39	0.01\\
73.4	0.01\\
73.41	0.01\\
73.42	0.01\\
73.43	0.01\\
73.44	0.01\\
73.45	0.01\\
73.46	0.01\\
73.47	0.01\\
73.48	0.01\\
73.49	0.01\\
73.5	0.01\\
73.51	0.01\\
73.52	0.01\\
73.53	0.01\\
73.54	0.01\\
73.55	0.01\\
73.56	0.01\\
73.57	0.01\\
73.58	0.01\\
73.59	0.01\\
73.6	0.01\\
73.61	0.01\\
73.62	0.01\\
73.63	0.01\\
73.64	0.01\\
73.65	0.01\\
73.66	0.01\\
73.67	0.01\\
73.68	0.01\\
73.69	0.01\\
73.7	0.01\\
73.71	0.01\\
73.72	0.01\\
73.73	0.01\\
73.74	0.01\\
73.75	0.01\\
73.76	0.01\\
73.77	0.01\\
73.78	0.01\\
73.79	0.01\\
73.8	0.01\\
73.81	0.01\\
73.82	0.01\\
73.83	0.01\\
73.84	0.01\\
73.85	0.01\\
73.86	0.01\\
73.87	0.01\\
73.88	0.01\\
73.89	0.01\\
73.9	0.01\\
73.91	0.01\\
73.92	0.01\\
73.93	0.01\\
73.94	0.01\\
73.95	0.01\\
73.96	0.01\\
73.97	0.01\\
73.98	0.01\\
73.99	0.01\\
74	0.01\\
74.01	0.01\\
74.02	0.01\\
74.03	0.01\\
74.04	0.01\\
74.05	0.01\\
74.06	0.01\\
74.07	0.01\\
74.08	0.01\\
74.09	0.01\\
74.1	0.01\\
74.11	0.01\\
74.12	0.01\\
74.13	0.01\\
74.14	0.01\\
74.15	0.01\\
74.16	0.01\\
74.17	0.01\\
74.18	0.01\\
74.19	0.01\\
74.2	0.01\\
74.21	0.01\\
74.22	0.01\\
74.23	0.01\\
74.24	0.01\\
74.25	0.01\\
74.26	0.01\\
74.27	0.01\\
74.28	0.01\\
74.29	0.01\\
74.3	0.01\\
74.31	0.01\\
74.32	0.01\\
74.33	0.01\\
74.34	0.01\\
74.35	0.01\\
74.36	0.01\\
74.37	0.01\\
74.38	0.01\\
74.39	0.01\\
74.4	0.01\\
74.41	0.01\\
74.42	0.01\\
74.43	0.01\\
74.44	0.01\\
74.45	0.01\\
74.46	0.01\\
74.47	0.01\\
74.48	0.01\\
74.49	0.01\\
74.5	0.01\\
74.51	0.01\\
74.52	0.01\\
74.53	0.01\\
74.54	0.01\\
74.55	0.01\\
74.56	0.01\\
74.57	0.01\\
74.58	0.01\\
74.59	0.01\\
74.6	0.01\\
74.61	0.01\\
74.62	0.01\\
74.63	0.01\\
74.64	0.01\\
74.65	0.01\\
74.66	0.01\\
74.67	0.01\\
74.68	0.01\\
74.69	0.01\\
74.7	0.01\\
74.71	0.01\\
74.72	0.01\\
74.73	0.01\\
74.74	0.01\\
74.75	0.01\\
74.76	0.01\\
74.77	0.01\\
74.78	0.01\\
74.79	0.01\\
74.8	0.01\\
74.81	0.01\\
74.82	0.01\\
74.83	0.01\\
74.84	0.01\\
74.85	0.01\\
74.86	0.01\\
74.87	0.01\\
74.88	0.01\\
74.89	0.01\\
74.9	0.01\\
74.91	0.01\\
74.92	0.01\\
74.93	0.01\\
74.94	0.01\\
74.95	0.01\\
74.96	0.01\\
74.97	0.01\\
74.98	0.01\\
74.99	0.01\\
75	0.01\\
75.01	0.01\\
75.02	0.01\\
75.03	0.01\\
75.04	0.01\\
75.05	0.01\\
75.06	0.01\\
75.07	0.01\\
75.08	0.01\\
75.09	0.01\\
75.1	0.01\\
75.11	0.01\\
75.12	0.01\\
75.13	0.01\\
75.14	0.01\\
75.15	0.01\\
75.16	0.01\\
75.17	0.01\\
75.18	0.01\\
75.19	0.01\\
75.2	0.01\\
75.21	0.01\\
75.22	0.01\\
75.23	0.01\\
75.24	0.01\\
75.25	0.01\\
75.26	0.01\\
75.27	0.01\\
75.28	0.01\\
75.29	0.01\\
75.3	0.01\\
75.31	0.01\\
75.32	0.01\\
75.33	0.01\\
75.34	0.01\\
75.35	0.01\\
75.36	0.01\\
75.37	0.01\\
75.38	0.01\\
75.39	0.01\\
75.4	0.01\\
75.41	0.01\\
75.42	0.01\\
75.43	0.01\\
75.44	0.01\\
75.45	0.01\\
75.46	0.01\\
75.47	0.01\\
75.48	0.01\\
75.49	0.01\\
75.5	0.01\\
75.51	0.01\\
75.52	0.01\\
75.53	0.01\\
75.54	0.01\\
75.55	0.01\\
75.56	0.01\\
75.57	0.01\\
75.58	0.01\\
75.59	0.01\\
75.6	0.01\\
75.61	0.01\\
75.62	0.01\\
75.63	0.01\\
75.64	0.01\\
75.65	0.01\\
75.66	0.01\\
75.67	0.01\\
75.68	0.01\\
75.69	0.01\\
75.7	0.01\\
75.71	0.01\\
75.72	0.01\\
75.73	0.01\\
75.74	0.01\\
75.75	0.01\\
75.76	0.01\\
75.77	0.01\\
75.78	0.01\\
75.79	0.01\\
75.8	0.01\\
75.81	0.01\\
75.82	0.01\\
75.83	0.01\\
75.84	0.01\\
75.85	0.01\\
75.86	0.01\\
75.87	0.01\\
75.88	0.01\\
75.89	0.01\\
75.9	0.01\\
75.91	0.01\\
75.92	0.01\\
75.93	0.01\\
75.94	0.01\\
75.95	0.01\\
75.96	0.01\\
75.97	0.01\\
75.98	0.01\\
75.99	0.01\\
76	0.01\\
76.01	0.01\\
76.02	0.01\\
76.03	0.01\\
76.04	0.01\\
76.05	0.01\\
76.06	0.01\\
76.07	0.01\\
76.08	0.01\\
76.09	0.01\\
76.1	0.01\\
76.11	0.01\\
76.12	0.01\\
76.13	0.01\\
76.14	0.01\\
76.15	0.01\\
76.16	0.01\\
76.17	0.01\\
76.18	0.01\\
76.19	0.01\\
76.2	0.01\\
76.21	0.01\\
76.22	0.01\\
76.23	0.01\\
76.24	0.01\\
76.25	0.01\\
76.26	0.01\\
76.27	0.01\\
76.28	0.01\\
76.29	0.01\\
76.3	0.01\\
76.31	0.01\\
76.32	0.01\\
76.33	0.01\\
76.34	0.01\\
76.35	0.01\\
76.36	0.01\\
76.37	0.01\\
76.38	0.01\\
76.39	0.01\\
76.4	0.01\\
76.41	0.01\\
76.42	0.01\\
76.43	0.01\\
76.44	0.01\\
76.45	0.01\\
76.46	0.01\\
76.47	0.01\\
76.48	0.01\\
76.49	0.01\\
76.5	0.01\\
76.51	0.01\\
76.52	0.01\\
76.53	0.01\\
76.54	0.01\\
76.55	0.01\\
76.56	0.01\\
76.57	0.01\\
76.58	0.01\\
76.59	0.01\\
76.6	0.01\\
76.61	0.01\\
76.62	0.01\\
76.63	0.01\\
76.64	0.01\\
76.65	0.01\\
76.66	0.01\\
76.67	0.01\\
76.68	0.01\\
76.69	0.01\\
76.7	0.01\\
76.71	0.01\\
76.72	0.01\\
76.73	0.01\\
76.74	0.01\\
76.75	0.01\\
76.76	0.01\\
76.77	0.01\\
76.78	0.01\\
76.79	0.01\\
76.8	0.01\\
76.81	0.01\\
76.82	0.01\\
76.83	0.01\\
76.84	0.01\\
76.85	0.01\\
76.86	0.01\\
76.87	0.01\\
76.88	0.01\\
76.89	0.01\\
76.9	0.01\\
76.91	0.01\\
76.92	0.01\\
76.93	0.01\\
76.94	0.01\\
76.95	0.01\\
76.96	0.01\\
76.97	0.01\\
76.98	0.01\\
76.99	0.01\\
77	0.01\\
77.01	0.01\\
77.02	0.01\\
77.03	0.01\\
77.04	0.01\\
77.05	0.01\\
77.06	0.01\\
77.07	0.01\\
77.08	0.01\\
77.09	0.01\\
77.1	0.01\\
77.11	0.01\\
77.12	0.01\\
77.13	0.01\\
77.14	0.01\\
77.15	0.01\\
77.16	0.01\\
77.17	0.01\\
77.18	0.01\\
77.19	0.01\\
77.2	0.01\\
77.21	0.01\\
77.22	0.01\\
77.23	0.01\\
77.24	0.01\\
77.25	0.01\\
77.26	0.01\\
77.27	0.01\\
77.28	0.01\\
77.29	0.01\\
77.3	0.01\\
77.31	0.01\\
77.32	0.01\\
77.33	0.01\\
77.34	0.01\\
77.35	0.01\\
77.36	0.01\\
77.37	0.01\\
77.38	0.01\\
77.39	0.01\\
77.4	0.01\\
77.41	0.01\\
77.42	0.01\\
77.43	0.01\\
77.44	0.01\\
77.45	0.01\\
77.46	0.01\\
77.47	0.01\\
77.48	0.01\\
77.49	0.01\\
77.5	0.01\\
77.51	0.01\\
77.52	0.01\\
77.53	0.01\\
77.54	0.01\\
77.55	0.01\\
77.56	0.01\\
77.57	0.01\\
77.58	0.01\\
77.59	0.01\\
77.6	0.01\\
77.61	0.01\\
77.62	0.01\\
77.63	0.01\\
77.64	0.01\\
77.65	0.01\\
77.66	0.01\\
77.67	0.01\\
77.68	0.01\\
77.69	0.01\\
77.7	0.01\\
77.71	0.01\\
77.72	0.01\\
77.73	0.01\\
77.74	0.01\\
77.75	0.01\\
77.76	0.01\\
77.77	0.01\\
77.78	0.01\\
77.79	0.01\\
77.8	0.01\\
77.81	0.01\\
77.82	0.01\\
77.83	0.01\\
77.84	0.01\\
77.85	0.01\\
77.86	0.01\\
77.87	0.01\\
77.88	0.01\\
77.89	0.01\\
77.9	0.01\\
77.91	0.01\\
77.92	0.01\\
77.93	0.01\\
77.94	0.01\\
77.95	0.01\\
77.96	0.01\\
77.97	0.01\\
77.98	0.01\\
77.99	0.01\\
78	0.01\\
78.01	0.01\\
78.02	0.01\\
78.03	0.01\\
78.04	0.01\\
78.05	0.01\\
78.06	0.01\\
78.07	0.01\\
78.08	0.01\\
78.09	0.01\\
78.1	0.01\\
78.11	0.01\\
78.12	0.01\\
78.13	0.01\\
78.14	0.01\\
78.15	0.01\\
78.16	0.01\\
78.17	0.01\\
78.18	0.01\\
78.19	0.01\\
78.2	0.01\\
78.21	0.01\\
78.22	0.01\\
78.23	0.01\\
78.24	0.01\\
78.25	0.01\\
78.26	0.01\\
78.27	0.01\\
78.28	0.01\\
78.29	0.01\\
78.3	0.01\\
78.31	0.01\\
78.32	0.01\\
78.33	0.01\\
78.34	0.01\\
78.35	0.01\\
78.36	0.01\\
78.37	0.01\\
78.38	0.01\\
78.39	0.01\\
78.4	0.01\\
78.41	0.01\\
78.42	0.01\\
78.43	0.01\\
78.44	0.01\\
78.45	0.01\\
78.46	0.01\\
78.47	0.01\\
78.48	0.01\\
78.49	0.01\\
78.5	0.01\\
78.51	0.01\\
78.52	0.01\\
78.53	0.01\\
78.54	0.01\\
78.55	0.01\\
78.56	0.01\\
78.57	0.01\\
78.58	0.01\\
78.59	0.01\\
78.6	0.01\\
78.61	0.01\\
78.62	0.01\\
78.63	0.01\\
78.64	0.01\\
78.65	0.01\\
78.66	0.01\\
78.67	0.01\\
78.68	0.01\\
78.69	0.01\\
78.7	0.01\\
78.71	0.01\\
78.72	0.01\\
78.73	0.01\\
78.74	0.01\\
78.75	0.01\\
78.76	0.01\\
78.77	0.01\\
78.78	0.01\\
78.79	0.01\\
78.8	0.01\\
78.81	0.01\\
78.82	0.01\\
78.83	0.01\\
78.84	0.01\\
78.85	0.01\\
78.86	0.01\\
78.87	0.01\\
78.88	0.01\\
78.89	0.01\\
78.9	0.01\\
78.91	0.01\\
78.92	0.01\\
78.93	0.01\\
78.94	0.01\\
78.95	0.01\\
78.96	0.01\\
78.97	0.01\\
78.98	0.01\\
78.99	0.01\\
79	0.01\\
79.01	0.01\\
79.02	0.01\\
79.03	0.01\\
79.04	0.01\\
79.05	0.01\\
79.06	0.01\\
79.07	0.01\\
79.08	0.01\\
79.09	0.01\\
79.1	0.01\\
79.11	0.01\\
79.12	0.01\\
79.13	0.01\\
79.14	0.01\\
79.15	0.01\\
79.16	0.01\\
79.17	0.01\\
79.18	0.01\\
79.19	0.01\\
79.2	0.01\\
79.21	0.01\\
79.22	0.01\\
79.23	0.01\\
79.24	0.01\\
79.25	0.01\\
79.26	0.01\\
79.27	0.01\\
79.28	0.01\\
79.29	0.01\\
79.3	0.01\\
79.31	0.01\\
79.32	0.01\\
79.33	0.01\\
79.34	0.01\\
79.35	0.01\\
79.36	0.01\\
79.37	0.01\\
79.38	0.01\\
79.39	0.01\\
79.4	0.01\\
79.41	0.01\\
79.42	0.01\\
79.43	0.01\\
79.44	0.01\\
79.45	0.01\\
79.46	0.01\\
79.47	0.01\\
79.48	0.01\\
79.49	0.01\\
79.5	0.01\\
79.51	0.01\\
79.52	0.01\\
79.53	0.01\\
79.54	0.01\\
79.55	0.01\\
79.56	0.01\\
79.57	0.01\\
79.58	0.01\\
79.59	0.01\\
79.6	0.01\\
79.61	0.01\\
79.62	0.01\\
79.63	0.01\\
79.64	0.01\\
79.65	0.01\\
79.66	0.01\\
79.67	0.01\\
79.68	0.01\\
79.69	0.01\\
79.7	0.01\\
79.71	0.01\\
79.72	0.01\\
79.73	0.01\\
79.74	0.01\\
79.75	0.01\\
79.76	0.01\\
79.77	0.01\\
79.78	0.01\\
79.79	0.01\\
79.8	0.01\\
79.81	0.01\\
79.82	0.01\\
79.83	0.01\\
79.84	0.01\\
79.85	0.01\\
79.86	0.01\\
79.87	0.01\\
79.88	0.01\\
79.89	0.01\\
79.9	0.01\\
79.91	0.01\\
79.92	0.01\\
79.93	0.01\\
79.94	0.01\\
79.95	0.01\\
79.96	0.01\\
79.97	0.01\\
79.98	0.01\\
79.99	0.01\\
80	0.01\\
80.01	0.01\\
};
\addplot [color=green,solid]
  table[row sep=crcr]{%
80.01	0.01\\
80.02	0.01\\
80.03	0.01\\
80.04	0.01\\
80.05	0.01\\
80.06	0.01\\
80.07	0.01\\
80.08	0.01\\
80.09	0.01\\
80.1	0.01\\
80.11	0.01\\
80.12	0.01\\
80.13	0.01\\
80.14	0.01\\
80.15	0.01\\
80.16	0.01\\
80.17	0.01\\
80.18	0.01\\
80.19	0.01\\
80.2	0.01\\
80.21	0.01\\
80.22	0.01\\
80.23	0.01\\
80.24	0.01\\
80.25	0.01\\
80.26	0.01\\
80.27	0.01\\
80.28	0.01\\
80.29	0.01\\
80.3	0.01\\
80.31	0.01\\
80.32	0.01\\
80.33	0.01\\
80.34	0.01\\
80.35	0.01\\
80.36	0.01\\
80.37	0.01\\
80.38	0.01\\
80.39	0.01\\
80.4	0.01\\
80.41	0.01\\
80.42	0.01\\
80.43	0.01\\
80.44	0.01\\
80.45	0.01\\
80.46	0.01\\
80.47	0.01\\
80.48	0.01\\
80.49	0.01\\
80.5	0.01\\
80.51	0.01\\
80.52	0.01\\
80.53	0.01\\
80.54	0.01\\
80.55	0.01\\
80.56	0.01\\
80.57	0.01\\
80.58	0.01\\
80.59	0.01\\
80.6	0.01\\
80.61	0.01\\
80.62	0.01\\
80.63	0.01\\
80.64	0.01\\
80.65	0.01\\
80.66	0.01\\
80.67	0.01\\
80.68	0.01\\
80.69	0.01\\
80.7	0.01\\
80.71	0.01\\
80.72	0.01\\
80.73	0.01\\
80.74	0.01\\
80.75	0.01\\
80.76	0.01\\
80.77	0.01\\
80.78	0.01\\
80.79	0.01\\
80.8	0.01\\
80.81	0.01\\
80.82	0.01\\
80.83	0.01\\
80.84	0.01\\
80.85	0.01\\
80.86	0.01\\
80.87	0.01\\
80.88	0.01\\
80.89	0.01\\
80.9	0.01\\
80.91	0.01\\
80.92	0.01\\
80.93	0.01\\
80.94	0.01\\
80.95	0.01\\
80.96	0.01\\
80.97	0.01\\
80.98	0.01\\
80.99	0.01\\
81	0.01\\
81.01	0.01\\
81.02	0.01\\
81.03	0.01\\
81.04	0.01\\
81.05	0.01\\
81.06	0.01\\
81.07	0.01\\
81.08	0.01\\
81.09	0.01\\
81.1	0.01\\
81.11	0.01\\
81.12	0.01\\
81.13	0.01\\
81.14	0.01\\
81.15	0.01\\
81.16	0.01\\
81.17	0.01\\
81.18	0.01\\
81.19	0.01\\
81.2	0.01\\
81.21	0.01\\
81.22	0.01\\
81.23	0.01\\
81.24	0.01\\
81.25	0.01\\
81.26	0.01\\
81.27	0.01\\
81.28	0.01\\
81.29	0.01\\
81.3	0.01\\
81.31	0.01\\
81.32	0.01\\
81.33	0.01\\
81.34	0.01\\
81.35	0.01\\
81.36	0.01\\
81.37	0.01\\
81.38	0.01\\
81.39	0.01\\
81.4	0.01\\
81.41	0.01\\
81.42	0.01\\
81.43	0.01\\
81.44	0.01\\
81.45	0.01\\
81.46	0.01\\
81.47	0.01\\
81.48	0.01\\
81.49	0.01\\
81.5	0.01\\
81.51	0.01\\
81.52	0.01\\
81.53	0.01\\
81.54	0.01\\
81.55	0.01\\
81.56	0.01\\
81.57	0.01\\
81.58	0.01\\
81.59	0.01\\
81.6	0.01\\
81.61	0.01\\
81.62	0.01\\
81.63	0.01\\
81.64	0.01\\
81.65	0.01\\
81.66	0.01\\
81.67	0.01\\
81.68	0.01\\
81.69	0.01\\
81.7	0.01\\
81.71	0.01\\
81.72	0.01\\
81.73	0.01\\
81.74	0.01\\
81.75	0.01\\
81.76	0.01\\
81.77	0.01\\
81.78	0.01\\
81.79	0.01\\
81.8	0.01\\
81.81	0.01\\
81.82	0.01\\
81.83	0.01\\
81.84	0.01\\
81.85	0.01\\
81.86	0.01\\
81.87	0.01\\
81.88	0.01\\
81.89	0.01\\
81.9	0.01\\
81.91	0.01\\
81.92	0.01\\
81.93	0.01\\
81.94	0.01\\
81.95	0.01\\
81.96	0.01\\
81.97	0.01\\
81.98	0.01\\
81.99	0.01\\
82	0.01\\
82.01	0.01\\
82.02	0.01\\
82.03	0.01\\
82.04	0.01\\
82.05	0.01\\
82.06	0.01\\
82.07	0.01\\
82.08	0.01\\
82.09	0.01\\
82.1	0.01\\
82.11	0.01\\
82.12	0.01\\
82.13	0.01\\
82.14	0.01\\
82.15	0.01\\
82.16	0.01\\
82.17	0.01\\
82.18	0.01\\
82.19	0.01\\
82.2	0.01\\
82.21	0.01\\
82.22	0.01\\
82.23	0.01\\
82.24	0.01\\
82.25	0.01\\
82.26	0.01\\
82.27	0.01\\
82.28	0.01\\
82.29	0.01\\
82.3	0.01\\
82.31	0.01\\
82.32	0.01\\
82.33	0.01\\
82.34	0.01\\
82.35	0.01\\
82.36	0.01\\
82.37	0.01\\
82.38	0.01\\
82.39	0.01\\
82.4	0.01\\
82.41	0.01\\
82.42	0.01\\
82.43	0.01\\
82.44	0.01\\
82.45	0.01\\
82.46	0.01\\
82.47	0.01\\
82.48	0.01\\
82.49	0.01\\
82.5	0.01\\
82.51	0.01\\
82.52	0.01\\
82.53	0.01\\
82.54	0.01\\
82.55	0.01\\
82.56	0.01\\
82.57	0.01\\
82.58	0.01\\
82.59	0.01\\
82.6	0.01\\
82.61	0.01\\
82.62	0.01\\
82.63	0.01\\
82.64	0.01\\
82.65	0.01\\
82.66	0.01\\
82.67	0.01\\
82.68	0.01\\
82.69	0.01\\
82.7	0.01\\
82.71	0.01\\
82.72	0.01\\
82.73	0.01\\
82.74	0.01\\
82.75	0.01\\
82.76	0.01\\
82.77	0.01\\
82.78	0.01\\
82.79	0.01\\
82.8	0.01\\
82.81	0.01\\
82.82	0.01\\
82.83	0.01\\
82.84	0.01\\
82.85	0.01\\
82.86	0.01\\
82.87	0.01\\
82.88	0.01\\
82.89	0.01\\
82.9	0.01\\
82.91	0.01\\
82.92	0.01\\
82.93	0.01\\
82.94	0.01\\
82.95	0.01\\
82.96	0.01\\
82.97	0.01\\
82.98	0.01\\
82.99	0.01\\
83	0.01\\
83.01	0.01\\
83.02	0.01\\
83.03	0.01\\
83.04	0.01\\
83.05	0.01\\
83.06	0.01\\
83.07	0.01\\
83.08	0.01\\
83.09	0.01\\
83.1	0.01\\
83.11	0.01\\
83.12	0.01\\
83.13	0.01\\
83.14	0.01\\
83.15	0.01\\
83.16	0.01\\
83.17	0.01\\
83.18	0.01\\
83.19	0.01\\
83.2	0.01\\
83.21	0.01\\
83.22	0.01\\
83.23	0.01\\
83.24	0.01\\
83.25	0.01\\
83.26	0.01\\
83.27	0.01\\
83.28	0.01\\
83.29	0.01\\
83.3	0.01\\
83.31	0.01\\
83.32	0.01\\
83.33	0.01\\
83.34	0.01\\
83.35	0.01\\
83.36	0.01\\
83.37	0.01\\
83.38	0.01\\
83.39	0.01\\
83.4	0.01\\
83.41	0.01\\
83.42	0.01\\
83.43	0.01\\
83.44	0.01\\
83.45	0.01\\
83.46	0.01\\
83.47	0.01\\
83.48	0.01\\
83.49	0.01\\
83.5	0.01\\
83.51	0.01\\
83.52	0.01\\
83.53	0.01\\
83.54	0.01\\
83.55	0.01\\
83.56	0.01\\
83.57	0.01\\
83.58	0.01\\
83.59	0.01\\
83.6	0.01\\
83.61	0.01\\
83.62	0.01\\
83.63	0.01\\
83.64	0.01\\
83.65	0.01\\
83.66	0.01\\
83.67	0.01\\
83.68	0.01\\
83.69	0.01\\
83.7	0.01\\
83.71	0.01\\
83.72	0.01\\
83.73	0.01\\
83.74	0.01\\
83.75	0.01\\
83.76	0.01\\
83.77	0.01\\
83.78	0.01\\
83.79	0.01\\
83.8	0.01\\
83.81	0.01\\
83.82	0.01\\
83.83	0.01\\
83.84	0.01\\
83.85	0.01\\
83.86	0.01\\
83.87	0.01\\
83.88	0.01\\
83.89	0.01\\
83.9	0.01\\
83.91	0.01\\
83.92	0.01\\
83.93	0.01\\
83.94	0.01\\
83.95	0.01\\
83.96	0.01\\
83.97	0.01\\
83.98	0.01\\
83.99	0.01\\
84	0.01\\
84.01	0.01\\
84.02	0.01\\
84.03	0.01\\
84.04	0.01\\
84.05	0.01\\
84.06	0.01\\
84.07	0.01\\
84.08	0.01\\
84.09	0.01\\
84.1	0.01\\
84.11	0.01\\
84.12	0.01\\
84.13	0.01\\
84.14	0.01\\
84.15	0.01\\
84.16	0.01\\
84.17	0.01\\
84.18	0.01\\
84.19	0.01\\
84.2	0.01\\
84.21	0.01\\
84.22	0.01\\
84.23	0.01\\
84.24	0.01\\
84.25	0.01\\
84.26	0.01\\
84.27	0.01\\
84.28	0.01\\
84.29	0.01\\
84.3	0.01\\
84.31	0.01\\
84.32	0.01\\
84.33	0.01\\
84.34	0.01\\
84.35	0.01\\
84.36	0.01\\
84.37	0.01\\
84.38	0.01\\
84.39	0.01\\
84.4	0.01\\
84.41	0.01\\
84.42	0.01\\
84.43	0.01\\
84.44	0.01\\
84.45	0.01\\
84.46	0.01\\
84.47	0.01\\
84.48	0.01\\
84.49	0.01\\
84.5	0.01\\
84.51	0.01\\
84.52	0.01\\
84.53	0.01\\
84.54	0.01\\
84.55	0.01\\
84.56	0.01\\
84.57	0.01\\
84.58	0.01\\
84.59	0.01\\
84.6	0.01\\
84.61	0.01\\
84.62	0.01\\
84.63	0.01\\
84.64	0.01\\
84.65	0.01\\
84.66	0.01\\
84.67	0.01\\
84.68	0.01\\
84.69	0.01\\
84.7	0.01\\
84.71	0.01\\
84.72	0.01\\
84.73	0.01\\
84.74	0.01\\
84.75	0.01\\
84.76	0.01\\
84.77	0.01\\
84.78	0.01\\
84.79	0.01\\
84.8	0.01\\
84.81	0.01\\
84.82	0.01\\
84.83	0.01\\
84.84	0.01\\
84.85	0.01\\
84.86	0.01\\
84.87	0.01\\
84.88	0.01\\
84.89	0.01\\
84.9	0.01\\
84.91	0.01\\
84.92	0.01\\
84.93	0.01\\
84.94	0.01\\
84.95	0.01\\
84.96	0.01\\
84.97	0.01\\
84.98	0.01\\
84.99	0.01\\
85	0.01\\
85.01	0.01\\
85.02	0.01\\
85.03	0.01\\
85.04	0.01\\
85.05	0.01\\
85.06	0.01\\
85.07	0.01\\
85.08	0.01\\
85.09	0.01\\
85.1	0.01\\
85.11	0.01\\
85.12	0.01\\
85.13	0.01\\
85.14	0.01\\
85.15	0.01\\
85.16	0.01\\
85.17	0.01\\
85.18	0.01\\
85.19	0.01\\
85.2	0.01\\
85.21	0.01\\
85.22	0.01\\
85.23	0.01\\
85.24	0.01\\
85.25	0.01\\
85.26	0.01\\
85.27	0.01\\
85.28	0.01\\
85.29	0.01\\
85.3	0.01\\
85.31	0.01\\
85.32	0.01\\
85.33	0.01\\
85.34	0.01\\
85.35	0.01\\
85.36	0.01\\
85.37	0.01\\
85.38	0.01\\
85.39	0.01\\
85.4	0.01\\
85.41	0.01\\
85.42	0.01\\
85.43	0.01\\
85.44	0.01\\
85.45	0.01\\
85.46	0.01\\
85.47	0.01\\
85.48	0.01\\
85.49	0.01\\
85.5	0.01\\
85.51	0.01\\
85.52	0.01\\
85.53	0.01\\
85.54	0.01\\
85.55	0.01\\
85.56	0.01\\
85.57	0.01\\
85.58	0.01\\
85.59	0.01\\
85.6	0.01\\
85.61	0.01\\
85.62	0.01\\
85.63	0.01\\
85.64	0.01\\
85.65	0.01\\
85.66	0.01\\
85.67	0.01\\
85.68	0.01\\
85.69	0.01\\
85.7	0.01\\
85.71	0.01\\
85.72	0.01\\
85.73	0.01\\
85.74	0.01\\
85.75	0.01\\
85.76	0.01\\
85.77	0.01\\
85.78	0.01\\
85.79	0.01\\
85.8	0.01\\
85.81	0.01\\
85.82	0.01\\
85.83	0.01\\
85.84	0.01\\
85.85	0.01\\
85.86	0.01\\
85.87	0.01\\
85.88	0.01\\
85.89	0.01\\
85.9	0.01\\
85.91	0.01\\
85.92	0.01\\
85.93	0.01\\
85.94	0.01\\
85.95	0.01\\
85.96	0.01\\
85.97	0.01\\
85.98	0.01\\
85.99	0.01\\
86	0.01\\
86.01	0.01\\
86.02	0.01\\
86.03	0.01\\
86.04	0.01\\
86.05	0.01\\
86.06	0.01\\
86.07	0.01\\
86.08	0.01\\
86.09	0.01\\
86.1	0.01\\
86.11	0.01\\
86.12	0.01\\
86.13	0.01\\
86.14	0.01\\
86.15	0.01\\
86.16	0.01\\
86.17	0.01\\
86.18	0.01\\
86.19	0.01\\
86.2	0.01\\
86.21	0.01\\
86.22	0.01\\
86.23	0.01\\
86.24	0.01\\
86.25	0.01\\
86.26	0.01\\
86.27	0.01\\
86.28	0.01\\
86.29	0.01\\
86.3	0.01\\
86.31	0.01\\
86.32	0.01\\
86.33	0.01\\
86.34	0.01\\
86.35	0.01\\
86.36	0.01\\
86.37	0.01\\
86.38	0.01\\
86.39	0.01\\
86.4	0.01\\
86.41	0.01\\
86.42	0.01\\
86.43	0.01\\
86.44	0.01\\
86.45	0.01\\
86.46	0.01\\
86.47	0.01\\
86.48	0.01\\
86.49	0.01\\
86.5	0.01\\
86.51	0.01\\
86.52	0.01\\
86.53	0.01\\
86.54	0.01\\
86.55	0.01\\
86.56	0.01\\
86.57	0.01\\
86.58	0.01\\
86.59	0.01\\
86.6	0.01\\
86.61	0.01\\
86.62	0.01\\
86.63	0.01\\
86.64	0.01\\
86.65	0.01\\
86.66	0.01\\
86.67	0.01\\
86.68	0.01\\
86.69	0.01\\
86.7	0.01\\
86.71	0.01\\
86.72	0.01\\
86.73	0.01\\
86.74	0.01\\
86.75	0.01\\
86.76	0.01\\
86.77	0.01\\
86.78	0.01\\
86.79	0.01\\
86.8	0.01\\
86.81	0.01\\
86.82	0.01\\
86.83	0.01\\
86.84	0.01\\
86.85	0.01\\
86.86	0.01\\
86.87	0.01\\
86.88	0.01\\
86.89	0.01\\
86.9	0.01\\
86.91	0.01\\
86.92	0.01\\
86.93	0.01\\
86.94	0.01\\
86.95	0.01\\
86.96	0.01\\
86.97	0.01\\
86.98	0.01\\
86.99	0.01\\
87	0.01\\
87.01	0.01\\
87.02	0.01\\
87.03	0.01\\
87.04	0.01\\
87.05	0.01\\
87.06	0.01\\
87.07	0.01\\
87.08	0.01\\
87.09	0.01\\
87.1	0.01\\
87.11	0.01\\
87.12	0.01\\
87.13	0.01\\
87.14	0.01\\
87.15	0.01\\
87.16	0.01\\
87.17	0.01\\
87.18	0.01\\
87.19	0.01\\
87.2	0.01\\
87.21	0.01\\
87.22	0.01\\
87.23	0.01\\
87.24	0.01\\
87.25	0.01\\
87.26	0.01\\
87.27	0.01\\
87.28	0.01\\
87.29	0.01\\
87.3	0.01\\
87.31	0.01\\
87.32	0.01\\
87.33	0.01\\
87.34	0.01\\
87.35	0.01\\
87.36	0.01\\
87.37	0.01\\
87.38	0.01\\
87.39	0.01\\
87.4	0.01\\
87.41	0.01\\
87.42	0.01\\
87.43	0.01\\
87.44	0.01\\
87.45	0.01\\
87.46	0.01\\
87.47	0.01\\
87.48	0.01\\
87.49	0.01\\
87.5	0.01\\
87.51	0.01\\
87.52	0.01\\
87.53	0.01\\
87.54	0.01\\
87.55	0.01\\
87.56	0.01\\
87.57	0.01\\
87.58	0.01\\
87.59	0.01\\
87.6	0.01\\
87.61	0.01\\
87.62	0.01\\
87.63	0.01\\
87.64	0.01\\
87.65	0.01\\
87.66	0.01\\
87.67	0.01\\
87.68	0.01\\
87.69	0.01\\
87.7	0.01\\
87.71	0.01\\
87.72	0.01\\
87.73	0.01\\
87.74	0.01\\
87.75	0.01\\
87.76	0.01\\
87.77	0.01\\
87.78	0.01\\
87.79	0.01\\
87.8	0.01\\
87.81	0.01\\
87.82	0.01\\
87.83	0.01\\
87.84	0.01\\
87.85	0.01\\
87.86	0.01\\
87.87	0.01\\
87.88	0.01\\
87.89	0.01\\
87.9	0.01\\
87.91	0.01\\
87.92	0.01\\
87.93	0.01\\
87.94	0.01\\
87.95	0.01\\
87.96	0.01\\
87.97	0.01\\
87.98	0.01\\
87.99	0.01\\
88	0.01\\
88.01	0.01\\
88.02	0.01\\
88.03	0.01\\
88.04	0.01\\
88.05	0.01\\
88.06	0.01\\
88.07	0.01\\
88.08	0.01\\
88.09	0.01\\
88.1	0.01\\
88.11	0.01\\
88.12	0.01\\
88.13	0.01\\
88.14	0.01\\
88.15	0.01\\
88.16	0.01\\
88.17	0.01\\
88.18	0.01\\
88.19	0.01\\
88.2	0.01\\
88.21	0.01\\
88.22	0.01\\
88.23	0.01\\
88.24	0.01\\
88.25	0.01\\
88.26	0.01\\
88.27	0.01\\
88.28	0.01\\
88.29	0.01\\
88.3	0.01\\
88.31	0.01\\
88.32	0.01\\
88.33	0.01\\
88.34	0.01\\
88.35	0.01\\
88.36	0.01\\
88.37	0.01\\
88.38	0.01\\
88.39	0.01\\
88.4	0.01\\
88.41	0.01\\
88.42	0.01\\
88.43	0.01\\
88.44	0.01\\
88.45	0.01\\
88.46	0.01\\
88.47	0.01\\
88.48	0.01\\
88.49	0.01\\
88.5	0.01\\
88.51	0.01\\
88.52	0.01\\
88.53	0.01\\
88.54	0.01\\
88.55	0.01\\
88.56	0.01\\
88.57	0.01\\
88.58	0.01\\
88.59	0.01\\
88.6	0.01\\
88.61	0.01\\
88.62	0.01\\
88.63	0.01\\
88.64	0.01\\
88.65	0.01\\
88.66	0.01\\
88.67	0.01\\
88.68	0.01\\
88.69	0.01\\
88.7	0.01\\
88.71	0.01\\
88.72	0.01\\
88.73	0.01\\
88.74	0.01\\
88.75	0.01\\
88.76	0.01\\
88.77	0.01\\
88.78	0.01\\
88.79	0.01\\
88.8	0.01\\
88.81	0.01\\
88.82	0.01\\
88.83	0.01\\
88.84	0.01\\
88.85	0.01\\
88.86	0.01\\
88.87	0.01\\
88.88	0.01\\
88.89	0.01\\
88.9	0.01\\
88.91	0.01\\
88.92	0.01\\
88.93	0.01\\
88.94	0.01\\
88.95	0.01\\
88.96	0.01\\
88.97	0.01\\
88.98	0.01\\
88.99	0.01\\
89	0.01\\
89.01	0.01\\
89.02	0.01\\
89.03	0.01\\
89.04	0.01\\
89.05	0.01\\
89.06	0.01\\
89.07	0.01\\
89.08	0.01\\
89.09	0.01\\
89.1	0.01\\
89.11	0.01\\
89.12	0.01\\
89.13	0.01\\
89.14	0.01\\
89.15	0.01\\
89.16	0.01\\
89.17	0.01\\
89.18	0.01\\
89.19	0.01\\
89.2	0.01\\
89.21	0.01\\
89.22	0.01\\
89.23	0.01\\
89.24	0.01\\
89.25	0.01\\
89.26	0.01\\
89.27	0.01\\
89.28	0.01\\
89.29	0.01\\
89.3	0.01\\
89.31	0.01\\
89.32	0.01\\
89.33	0.01\\
89.34	0.01\\
89.35	0.01\\
89.36	0.01\\
89.37	0.01\\
89.38	0.01\\
89.39	0.01\\
89.4	0.01\\
89.41	0.01\\
89.42	0.01\\
89.43	0.01\\
89.44	0.01\\
89.45	0.01\\
89.46	0.01\\
89.47	0.01\\
89.48	0.01\\
89.49	0.01\\
89.5	0.01\\
89.51	0.01\\
89.52	0.01\\
89.53	0.01\\
89.54	0.01\\
89.55	0.01\\
89.56	0.01\\
89.57	0.01\\
89.58	0.01\\
89.59	0.01\\
89.6	0.01\\
89.61	0.01\\
89.62	0.01\\
89.63	0.01\\
89.64	0.01\\
89.65	0.01\\
89.66	0.01\\
89.67	0.01\\
89.68	0.01\\
89.69	0.01\\
89.7	0.01\\
89.71	0.01\\
89.72	0.01\\
89.73	0.01\\
89.74	0.01\\
89.75	0.01\\
89.76	0.01\\
89.77	0.01\\
89.78	0.01\\
89.79	0.01\\
89.8	0.01\\
89.81	0.01\\
89.82	0.01\\
89.83	0.01\\
89.84	0.01\\
89.85	0.01\\
89.86	0.01\\
89.87	0.01\\
89.88	0.01\\
89.89	0.01\\
89.9	0.01\\
89.91	0.01\\
89.92	0.01\\
89.93	0.01\\
89.94	0.01\\
89.95	0.01\\
89.96	0.01\\
89.97	0.01\\
89.98	0.01\\
89.99	0.01\\
90	0.01\\
90.01	0.01\\
90.02	0.01\\
90.03	0.01\\
90.04	0.01\\
90.05	0.01\\
90.06	0.01\\
90.07	0.01\\
90.08	0.01\\
90.09	0.01\\
90.1	0.01\\
90.11	0.01\\
90.12	0.01\\
90.13	0.01\\
90.14	0.01\\
90.15	0.01\\
90.16	0.01\\
90.17	0.01\\
90.18	0.01\\
90.19	0.01\\
90.2	0.01\\
90.21	0.01\\
90.22	0.01\\
90.23	0.01\\
90.24	0.01\\
90.25	0.01\\
90.26	0.01\\
90.27	0.01\\
90.28	0.01\\
90.29	0.01\\
90.3	0.01\\
90.31	0.01\\
90.32	0.01\\
90.33	0.01\\
90.34	0.01\\
90.35	0.01\\
90.36	0.01\\
90.37	0.01\\
90.38	0.01\\
90.39	0.01\\
90.4	0.01\\
90.41	0.01\\
90.42	0.01\\
90.43	0.01\\
90.44	0.01\\
90.45	0.01\\
90.46	0.01\\
90.47	0.01\\
90.48	0.01\\
90.49	0.01\\
90.5	0.01\\
90.51	0.01\\
90.52	0.01\\
90.53	0.01\\
90.54	0.01\\
90.55	0.01\\
90.56	0.01\\
90.57	0.01\\
90.58	0.01\\
90.59	0.01\\
90.6	0.01\\
90.61	0.01\\
90.62	0.01\\
90.63	0.01\\
90.64	0.01\\
90.65	0.01\\
90.66	0.01\\
90.67	0.01\\
90.68	0.01\\
90.69	0.01\\
90.7	0.01\\
90.71	0.01\\
90.72	0.01\\
90.73	0.01\\
90.74	0.01\\
90.75	0.01\\
90.76	0.01\\
90.77	0.01\\
90.78	0.01\\
90.79	0.01\\
90.8	0.01\\
90.81	0.01\\
90.82	0.01\\
90.83	0.01\\
90.84	0.01\\
90.85	0.01\\
90.86	0.01\\
90.87	0.01\\
90.88	0.01\\
90.89	0.01\\
90.9	0.01\\
90.91	0.01\\
90.92	0.01\\
90.93	0.01\\
90.94	0.01\\
90.95	0.01\\
90.96	0.01\\
90.97	0.01\\
90.98	0.01\\
90.99	0.01\\
91	0.01\\
91.01	0.01\\
91.02	0.01\\
91.03	0.01\\
91.04	0.01\\
91.05	0.01\\
91.06	0.01\\
91.07	0.01\\
91.08	0.01\\
91.09	0.01\\
91.1	0.01\\
91.11	0.01\\
91.12	0.01\\
91.13	0.01\\
91.14	0.01\\
91.15	0.01\\
91.16	0.01\\
91.17	0.01\\
91.18	0.01\\
91.19	0.01\\
91.2	0.01\\
91.21	0.01\\
91.22	0.01\\
91.23	0.01\\
91.24	0.01\\
91.25	0.01\\
91.26	0.01\\
91.27	0.01\\
91.28	0.01\\
91.29	0.01\\
91.3	0.01\\
91.31	0.01\\
91.32	0.01\\
91.33	0.01\\
91.34	0.01\\
91.35	0.01\\
91.36	0.01\\
91.37	0.01\\
91.38	0.01\\
91.39	0.01\\
91.4	0.01\\
91.41	0.01\\
91.42	0.01\\
91.43	0.01\\
91.44	0.01\\
91.45	0.01\\
91.46	0.01\\
91.47	0.01\\
91.48	0.01\\
91.49	0.01\\
91.5	0.01\\
91.51	0.01\\
91.52	0.01\\
91.53	0.01\\
91.54	0.01\\
91.55	0.01\\
91.56	0.01\\
91.57	0.01\\
91.58	0.01\\
91.59	0.01\\
91.6	0.01\\
91.61	0.01\\
91.62	0.01\\
91.63	0.01\\
91.64	0.01\\
91.65	0.01\\
91.66	0.01\\
91.67	0.01\\
91.68	0.01\\
91.69	0.01\\
91.7	0.01\\
91.71	0.01\\
91.72	0.01\\
91.73	0.01\\
91.74	0.01\\
91.75	0.01\\
91.76	0.01\\
91.77	0.01\\
91.78	0.01\\
91.79	0.01\\
91.8	0.01\\
91.81	0.01\\
91.82	0.01\\
91.83	0.01\\
91.84	0.01\\
91.85	0.01\\
91.86	0.01\\
91.87	0.01\\
91.88	0.01\\
91.89	0.01\\
91.9	0.01\\
91.91	0.01\\
91.92	0.01\\
91.93	0.01\\
91.94	0.01\\
91.95	0.01\\
91.96	0.01\\
91.97	0.01\\
91.98	0.01\\
91.99	0.01\\
92	0.01\\
92.01	0.01\\
92.02	0.01\\
92.03	0.01\\
92.04	0.01\\
92.05	0.01\\
92.06	0.01\\
92.07	0.01\\
92.08	0.01\\
92.09	0.01\\
92.1	0.01\\
92.11	0.01\\
92.12	0.01\\
92.13	0.01\\
92.14	0.01\\
92.15	0.01\\
92.16	0.01\\
92.17	0.01\\
92.18	0.01\\
92.19	0.01\\
92.2	0.01\\
92.21	0.01\\
92.22	0.01\\
92.23	0.01\\
92.24	0.01\\
92.25	0.01\\
92.26	0.01\\
92.27	0.01\\
92.28	0.01\\
92.29	0.01\\
92.3	0.01\\
92.31	0.01\\
92.32	0.01\\
92.33	0.01\\
92.34	0.01\\
92.35	0.01\\
92.36	0.01\\
92.37	0.01\\
92.38	0.01\\
92.39	0.01\\
92.4	0.01\\
92.41	0.01\\
92.42	0.01\\
92.43	0.01\\
92.44	0.01\\
92.45	0.01\\
92.46	0.01\\
92.47	0.01\\
92.48	0.01\\
92.49	0.01\\
92.5	0.01\\
92.51	0.01\\
92.52	0.01\\
92.53	0.01\\
92.54	0.01\\
92.55	0.01\\
92.56	0.01\\
92.57	0.01\\
92.58	0.01\\
92.59	0.01\\
92.6	0.01\\
92.61	0.01\\
92.62	0.01\\
92.63	0.01\\
92.64	0.01\\
92.65	0.01\\
92.66	0.01\\
92.67	0.01\\
92.68	0.01\\
92.69	0.01\\
92.7	0.01\\
92.71	0.01\\
92.72	0.01\\
92.73	0.01\\
92.74	0.01\\
92.75	0.01\\
92.76	0.01\\
92.77	0.01\\
92.78	0.01\\
92.79	0.01\\
92.8	0.01\\
92.81	0.01\\
92.82	0.01\\
92.83	0.01\\
92.84	0.01\\
92.85	0.01\\
92.86	0.01\\
92.87	0.01\\
92.88	0.01\\
92.89	0.01\\
92.9	0.01\\
92.91	0.01\\
92.92	0.01\\
92.93	0.01\\
92.94	0.01\\
92.95	0.01\\
92.96	0.01\\
92.97	0.01\\
92.98	0.01\\
92.99	0.01\\
93	0.01\\
93.01	0.01\\
93.02	0.01\\
93.03	0.01\\
93.04	0.01\\
93.05	0.01\\
93.06	0.01\\
93.07	0.01\\
93.08	0.01\\
93.09	0.01\\
93.1	0.01\\
93.11	0.01\\
93.12	0.01\\
93.13	0.01\\
93.14	0.01\\
93.15	0.01\\
93.16	0.01\\
93.17	0.01\\
93.18	0.01\\
93.19	0.01\\
93.2	0.01\\
93.21	0.01\\
93.22	0.01\\
93.23	0.01\\
93.24	0.01\\
93.25	0.01\\
93.26	0.01\\
93.27	0.01\\
93.28	0.01\\
93.29	0.01\\
93.3	0.01\\
93.31	0.01\\
93.32	0.01\\
93.33	0.01\\
93.34	0.01\\
93.35	0.01\\
93.36	0.01\\
93.37	0.01\\
93.38	0.01\\
93.39	0.01\\
93.4	0.01\\
93.41	0.01\\
93.42	0.01\\
93.43	0.01\\
93.44	0.01\\
93.45	0.01\\
93.46	0.01\\
93.47	0.01\\
93.48	0.01\\
93.49	0.01\\
93.5	0.01\\
93.51	0.01\\
93.52	0.01\\
93.53	0.01\\
93.54	0.01\\
93.55	0.01\\
93.56	0.01\\
93.57	0.01\\
93.58	0.01\\
93.59	0.01\\
93.6	0.01\\
93.61	0.01\\
93.62	0.01\\
93.63	0.01\\
93.64	0.01\\
93.65	0.01\\
93.66	0.01\\
93.67	0.01\\
93.68	0.01\\
93.69	0.01\\
93.7	0.01\\
93.71	0.01\\
93.72	0.01\\
93.73	0.01\\
93.74	0.01\\
93.75	0.01\\
93.76	0.01\\
93.77	0.01\\
93.78	0.01\\
93.79	0.01\\
93.8	0.01\\
93.81	0.01\\
93.82	0.01\\
93.83	0.01\\
93.84	0.01\\
93.85	0.01\\
93.86	0.01\\
93.87	0.01\\
93.88	0.01\\
93.89	0.01\\
93.9	0.01\\
93.91	0.01\\
93.92	0.01\\
93.93	0.01\\
93.94	0.01\\
93.95	0.01\\
93.96	0.01\\
93.97	0.01\\
93.98	0.01\\
93.99	0.01\\
94	0.01\\
94.01	0.01\\
94.02	0.01\\
94.03	0.01\\
94.04	0.01\\
94.05	0.01\\
94.06	0.01\\
94.07	0.01\\
94.08	0.01\\
94.09	0.01\\
94.1	0.01\\
94.11	0.01\\
94.12	0.01\\
94.13	0.01\\
94.14	0.01\\
94.15	0.01\\
94.16	0.01\\
94.17	0.01\\
94.18	0.01\\
94.19	0.01\\
94.2	0.01\\
94.21	0.01\\
94.22	0.01\\
94.23	0.01\\
94.24	0.01\\
94.25	0.01\\
94.26	0.01\\
94.27	0.01\\
94.28	0.01\\
94.29	0.01\\
94.3	0.01\\
94.31	0.01\\
94.32	0.01\\
94.33	0.01\\
94.34	0.01\\
94.35	0.01\\
94.36	0.01\\
94.37	0.01\\
94.38	0.01\\
94.39	0.01\\
94.4	0.01\\
94.41	0.01\\
94.42	0.01\\
94.43	0.01\\
94.44	0.01\\
94.45	0.01\\
94.46	0.01\\
94.47	0.01\\
94.48	0.01\\
94.49	0.01\\
94.5	0.01\\
94.51	0.01\\
94.52	0.01\\
94.53	0.01\\
94.54	0.01\\
94.55	0.01\\
94.56	0.01\\
94.57	0.01\\
94.58	0.01\\
94.59	0.01\\
94.6	0.01\\
94.61	0.01\\
94.62	0.01\\
94.63	0.01\\
94.64	0.01\\
94.65	0.01\\
94.66	0.01\\
94.67	0.01\\
94.68	0.01\\
94.69	0.01\\
94.7	0.01\\
94.71	0.01\\
94.72	0.01\\
94.73	0.01\\
94.74	0.01\\
94.75	0.01\\
94.76	0.01\\
94.77	0.01\\
94.78	0.01\\
94.79	0.01\\
94.8	0.01\\
94.81	0.01\\
94.82	0.01\\
94.83	0.01\\
94.84	0.01\\
94.85	0.01\\
94.86	0.01\\
94.87	0.01\\
94.88	0.01\\
94.89	0.01\\
94.9	0.01\\
94.91	0.01\\
94.92	0.01\\
94.93	0.01\\
94.94	0.01\\
94.95	0.01\\
94.96	0.01\\
94.97	0.01\\
94.98	0.01\\
94.99	0.01\\
95	0.01\\
95.01	0.01\\
95.02	0.01\\
95.03	0.01\\
95.04	0.01\\
95.05	0.01\\
95.06	0.01\\
95.07	0.01\\
95.08	0.01\\
95.09	0.01\\
95.1	0.01\\
95.11	0.01\\
95.12	0.01\\
95.13	0.01\\
95.14	0.01\\
95.15	0.01\\
95.16	0.01\\
95.17	0.01\\
95.18	0.01\\
95.19	0.01\\
95.2	0.01\\
95.21	0.01\\
95.22	0.01\\
95.23	0.01\\
95.24	0.01\\
95.25	0.01\\
95.26	0.01\\
95.27	0.01\\
95.28	0.01\\
95.29	0.01\\
95.3	0.01\\
95.31	0.01\\
95.32	0.01\\
95.33	0.01\\
95.34	0.01\\
95.35	0.01\\
95.36	0.01\\
95.37	0.01\\
95.38	0.01\\
95.39	0.01\\
95.4	0.01\\
95.41	0.01\\
95.42	0.01\\
95.43	0.01\\
95.44	0.01\\
95.45	0.01\\
95.46	0.01\\
95.47	0.01\\
95.48	0.01\\
95.49	0.01\\
95.5	0.01\\
95.51	0.01\\
95.52	0.01\\
95.53	0.01\\
95.54	0.01\\
95.55	0.01\\
95.56	0.01\\
95.57	0.01\\
95.58	0.01\\
95.59	0.01\\
95.6	0.01\\
95.61	0.01\\
95.62	0.01\\
95.63	0.01\\
95.64	0.01\\
95.65	0.01\\
95.66	0.01\\
95.67	0.01\\
95.68	0.01\\
95.69	0.01\\
95.7	0.01\\
95.71	0.01\\
95.72	0.01\\
95.73	0.01\\
95.74	0.01\\
95.75	0.01\\
95.76	0.01\\
95.77	0.01\\
95.78	0.01\\
95.79	0.01\\
95.8	0.01\\
95.81	0.01\\
95.82	0.01\\
95.83	0.01\\
95.84	0.01\\
95.85	0.01\\
95.86	0.01\\
95.87	0.01\\
95.88	0.01\\
95.89	0.01\\
95.9	0.01\\
95.91	0.01\\
95.92	0.01\\
95.93	0.01\\
95.94	0.01\\
95.95	0.01\\
95.96	0.01\\
95.97	0.01\\
95.98	0.01\\
95.99	0.01\\
96	0.01\\
96.01	0.01\\
96.02	0.01\\
96.03	0.01\\
96.04	0.01\\
96.05	0.01\\
96.06	0.01\\
96.07	0.01\\
96.08	0.01\\
96.09	0.01\\
96.1	0.01\\
96.11	0.01\\
96.12	0.01\\
96.13	0.01\\
96.14	0.01\\
96.15	0.01\\
96.16	0.01\\
96.17	0.01\\
96.18	0.01\\
96.19	0.01\\
96.2	0.01\\
96.21	0.01\\
96.22	0.01\\
96.23	0.01\\
96.24	0.01\\
96.25	0.01\\
96.26	0.01\\
96.27	0.01\\
96.28	0.01\\
96.29	0.01\\
96.3	0.01\\
96.31	0.01\\
96.32	0.01\\
96.33	0.01\\
96.34	0.01\\
96.35	0.01\\
96.36	0.01\\
96.37	0.01\\
96.38	0.01\\
96.39	0.01\\
96.4	0.01\\
96.41	0.01\\
96.42	0.01\\
96.43	0.01\\
96.44	0.01\\
96.45	0.01\\
96.46	0.01\\
96.47	0.01\\
96.48	0.01\\
96.49	0.01\\
96.5	0.01\\
96.51	0.01\\
96.52	0.01\\
96.53	0.01\\
96.54	0.01\\
96.55	0.01\\
96.56	0.01\\
96.57	0.01\\
96.58	0.01\\
96.59	0.01\\
96.6	0.01\\
96.61	0.01\\
96.62	0.01\\
96.63	0.01\\
96.64	0.01\\
96.65	0.01\\
96.66	0.01\\
96.67	0.01\\
96.68	0.01\\
96.69	0.01\\
96.7	0.01\\
96.71	0.01\\
96.72	0.01\\
96.73	0.01\\
96.74	0.01\\
96.75	0.01\\
96.76	0.01\\
96.77	0.01\\
96.78	0.01\\
96.79	0.01\\
96.8	0.01\\
96.81	0.01\\
96.82	0.01\\
96.83	0.01\\
96.84	0.01\\
96.85	0.01\\
96.86	0.01\\
96.87	0.01\\
96.88	0.01\\
96.89	0.01\\
96.9	0.01\\
96.91	0.01\\
96.92	0.01\\
96.93	0.01\\
96.94	0.01\\
96.95	0.01\\
96.96	0.01\\
96.97	0.01\\
96.98	0.01\\
96.99	0.01\\
97	0.01\\
97.01	0.01\\
97.02	0.01\\
97.03	0.01\\
97.04	0.01\\
97.05	0.01\\
97.06	0.01\\
97.07	0.01\\
97.08	0.01\\
97.09	0.01\\
97.1	0.01\\
97.11	0.01\\
97.12	0.01\\
97.13	0.01\\
97.14	0.01\\
97.15	0.01\\
97.16	0.01\\
97.17	0.01\\
97.18	0.01\\
97.19	0.01\\
97.2	0.01\\
97.21	0.01\\
97.22	0.01\\
97.23	0.01\\
97.24	0.01\\
97.25	0.01\\
97.26	0.01\\
97.27	0.01\\
97.28	0.01\\
97.29	0.01\\
97.3	0.01\\
97.31	0.01\\
97.32	0.01\\
97.33	0.01\\
97.34	0.01\\
97.35	0.01\\
97.36	0.01\\
97.37	0.01\\
97.38	0.01\\
97.39	0.01\\
97.4	0.01\\
97.41	0.01\\
97.42	0.01\\
97.43	0.01\\
97.44	0.01\\
97.45	0.01\\
97.46	0.01\\
97.47	0.01\\
97.48	0.01\\
97.49	0.01\\
97.5	0.01\\
97.51	0.01\\
97.52	0.01\\
97.53	0.01\\
97.54	0.01\\
97.55	0.01\\
97.56	0.01\\
97.57	0.01\\
97.58	0.01\\
97.59	0.01\\
97.6	0.01\\
97.61	0.01\\
97.62	0.01\\
97.63	0.01\\
97.64	0.01\\
97.65	0.01\\
97.66	0.01\\
97.67	0.01\\
97.68	0.01\\
97.69	0.01\\
97.7	0.01\\
97.71	0.01\\
97.72	0.01\\
97.73	0.01\\
97.74	0.01\\
97.75	0.01\\
97.76	0.01\\
97.77	0.01\\
97.78	0.01\\
97.79	0.01\\
97.8	0.01\\
97.81	0.01\\
97.82	0.01\\
97.83	0.01\\
97.84	0.01\\
97.85	0.01\\
97.86	0.01\\
97.87	0.01\\
97.88	0.01\\
97.89	0.01\\
97.9	0.01\\
97.91	0.01\\
97.92	0.01\\
97.93	0.01\\
97.94	0.01\\
97.95	0.01\\
97.96	0.01\\
97.97	0.01\\
97.98	0.01\\
97.99	0.01\\
98	0.01\\
98.01	0.01\\
98.02	0.01\\
98.03	0.01\\
98.04	0.01\\
98.05	0.01\\
98.06	0.01\\
98.07	0.01\\
98.08	0.01\\
98.09	0.01\\
98.1	0.01\\
98.11	0.01\\
98.12	0.01\\
98.13	0.01\\
98.14	0.01\\
98.15	0.01\\
98.16	0.01\\
98.17	0.01\\
98.18	0.01\\
98.19	0.01\\
98.2	0.01\\
98.21	0.01\\
98.22	0.01\\
98.23	0.01\\
98.24	0.01\\
98.25	0.01\\
98.26	0.01\\
98.27	0.01\\
98.28	0.01\\
98.29	0.01\\
98.3	0.01\\
98.31	0.01\\
98.32	0.01\\
98.33	0.01\\
98.34	0.01\\
98.35	0.01\\
98.36	0.01\\
98.37	0.01\\
98.38	0.01\\
98.39	0.01\\
98.4	0.01\\
98.41	0.01\\
98.42	0.01\\
98.43	0.01\\
98.44	0.01\\
98.45	0.01\\
98.46	0.01\\
98.47	0.01\\
98.48	0.01\\
98.49	0.01\\
98.5	0.01\\
98.51	0.01\\
98.52	0.01\\
98.53	0.01\\
98.54	0.01\\
98.55	0.01\\
98.56	0.01\\
98.57	0.01\\
98.58	0.01\\
98.59	0.01\\
98.6	0.01\\
98.61	0.01\\
98.62	0.01\\
98.63	0.01\\
98.64	0.01\\
98.65	0.01\\
98.66	0.01\\
98.67	0.01\\
98.68	0.01\\
98.69	0.01\\
98.7	0.01\\
98.71	0.01\\
98.72	0.01\\
98.73	0.01\\
98.74	0.01\\
98.75	0.01\\
98.76	0.01\\
98.77	0.01\\
98.78	0.01\\
98.79	0.01\\
98.8	0.01\\
98.81	0.01\\
98.82	0.01\\
98.83	0.01\\
98.84	0.01\\
98.85	0.01\\
98.86	0.01\\
98.87	0.01\\
98.88	0.01\\
98.89	0.01\\
98.9	0.01\\
98.91	0.01\\
98.92	0.01\\
98.93	0.01\\
98.94	0.01\\
98.95	0.01\\
98.96	0.01\\
98.97	0.01\\
98.98	0.01\\
98.99	0.01\\
99	0.01\\
99.01	0.01\\
99.02	0.01\\
99.03	0.01\\
99.04	0.01\\
99.05	0.01\\
99.06	0.01\\
99.07	0.01\\
99.08	0.01\\
99.09	0.01\\
99.1	0.01\\
99.11	0.01\\
99.12	0.01\\
99.13	0.01\\
99.14	0.01\\
99.15	0.01\\
99.16	0.01\\
99.17	0.01\\
99.18	0.01\\
99.19	0.01\\
99.2	0.01\\
99.21	0.01\\
99.22	0.01\\
99.23	0.01\\
99.24	0.01\\
99.25	0.01\\
99.26	0.01\\
99.27	0.01\\
99.28	0.01\\
99.29	0.01\\
99.3	0.01\\
99.31	0.01\\
99.32	0.01\\
99.33	0.01\\
99.34	0.01\\
99.35	0.01\\
99.36	0.01\\
99.37	0.01\\
99.38	0.01\\
99.39	0.01\\
99.4	0.01\\
99.41	0.01\\
99.42	0.01\\
99.43	0.01\\
99.44	0.01\\
99.45	0.01\\
99.46	0.01\\
99.47	0.01\\
99.48	0.01\\
99.49	0.01\\
99.5	0.01\\
99.51	0.01\\
99.52	0.01\\
99.53	0.01\\
99.54	0.01\\
99.55	0.01\\
99.56	0.01\\
99.57	0.01\\
99.58	0.01\\
99.59	0.01\\
99.6	0.01\\
99.61	0.01\\
99.62	0.01\\
99.63	0.01\\
99.64	0.01\\
99.65	0.01\\
99.66	0.01\\
99.67	0.01\\
99.68	0.01\\
99.69	0.01\\
99.7	0.01\\
99.71	0.01\\
99.72	0.01\\
99.73	0.01\\
99.74	0.01\\
99.75	0.01\\
99.76	0.01\\
99.77	0.01\\
99.78	0.01\\
99.79	0.01\\
99.8	0.01\\
99.81	0.01\\
99.82	0.01\\
99.83	0.01\\
99.84	0.01\\
99.85	0.01\\
99.86	0.01\\
99.87	0.01\\
99.88	0.01\\
99.89	0.01\\
99.9	0.01\\
99.91	0.01\\
99.92	0.01\\
99.93	0.01\\
99.94	0.01\\
99.95	0.01\\
99.96	0.01\\
99.97	0.01\\
99.98	0.01\\
99.99	0.01\\
100	0.01\\
};
\addlegendentry{$q=4$};

\end{axis}
\end{tikzpicture}%
%  \caption{Continuous Time}
%\end{subfigure}%
%\hfill%
%\begin{subfigure}{.45\linewidth}
%  \centering
%  \setlength\figureheight{\linewidth} 
%  \setlength\figurewidth{\linewidth}
%  \tikzsetnextfilename{dp_dscr_z15}
%  % This file was created by matlab2tikz.
%
%The latest updates can be retrieved from
%  http://www.mathworks.com/matlabcentral/fileexchange/22022-matlab2tikz-matlab2tikz
%where you can also make suggestions and rate matlab2tikz.
%
\definecolor{mycolor1}{rgb}{0.00000,1.00000,0.14286}%
\definecolor{mycolor2}{rgb}{0.00000,1.00000,0.28571}%
\definecolor{mycolor3}{rgb}{0.00000,1.00000,0.42857}%
\definecolor{mycolor4}{rgb}{0.00000,1.00000,0.57143}%
\definecolor{mycolor5}{rgb}{0.00000,1.00000,0.71429}%
\definecolor{mycolor6}{rgb}{0.00000,1.00000,0.85714}%
\definecolor{mycolor7}{rgb}{0.00000,1.00000,1.00000}%
\definecolor{mycolor8}{rgb}{0.00000,0.87500,1.00000}%
\definecolor{mycolor9}{rgb}{0.00000,0.62500,1.00000}%
\definecolor{mycolor10}{rgb}{0.12500,0.00000,1.00000}%
\definecolor{mycolor11}{rgb}{0.25000,0.00000,1.00000}%
\definecolor{mycolor12}{rgb}{0.37500,0.00000,1.00000}%
\definecolor{mycolor13}{rgb}{0.50000,0.00000,1.00000}%
\definecolor{mycolor14}{rgb}{0.62500,0.00000,1.00000}%
\definecolor{mycolor15}{rgb}{0.75000,0.00000,1.00000}%
\definecolor{mycolor16}{rgb}{0.87500,0.00000,1.00000}%
\definecolor{mycolor17}{rgb}{1.00000,0.00000,1.00000}%
\definecolor{mycolor18}{rgb}{1.00000,0.00000,0.87500}%
\definecolor{mycolor19}{rgb}{1.00000,0.00000,0.62500}%
\definecolor{mycolor20}{rgb}{0.85714,0.00000,0.00000}%
\definecolor{mycolor21}{rgb}{0.71429,0.00000,0.00000}%
%
\begin{tikzpicture}[trim axis left, trim axis right]

\begin{axis}[%
width=\figurewidth,
height=\figureheight,
at={(0\figurewidth,0\figureheight)},
scale only axis,
point meta min=0,
point meta max=1,
every outer x axis line/.append style={black},
every x tick label/.append style={font=\color{black}},
xmin=0,
xmax=600,
every outer y axis line/.append style={black},
every y tick label/.append style={font=\color{black}},
ymin=0,
ymax=0.014,
axis background/.style={fill=white},
axis x line*=bottom,
axis y line*=left,
]
\addplot [color=green,solid,forget plot]
  table[row sep=crcr]{%
1	0\\
2	0\\
3	0\\
4	0\\
5	0\\
6	0\\
7	0\\
8	0\\
9	0\\
10	0\\
11	0\\
12	0\\
13	0\\
14	0\\
15	0\\
16	0\\
17	0\\
18	0\\
19	0\\
20	0\\
21	0\\
22	0\\
23	0\\
24	0\\
25	0\\
26	0\\
27	0\\
28	0\\
29	0\\
30	0\\
31	0\\
32	0\\
33	0\\
34	0\\
35	0\\
36	0\\
37	0\\
38	0\\
39	0\\
40	0\\
41	0\\
42	0\\
43	0\\
44	0\\
45	0\\
46	0\\
47	0\\
48	0\\
49	0\\
50	0\\
51	0\\
52	0\\
53	0\\
54	0\\
55	0\\
56	0\\
57	0\\
58	0\\
59	0\\
60	0\\
61	0\\
62	0\\
63	0\\
64	0\\
65	0\\
66	0\\
67	0\\
68	0\\
69	0\\
70	0\\
71	0\\
72	0\\
73	0\\
74	0\\
75	0\\
76	0\\
77	0\\
78	0\\
79	0\\
80	0\\
81	0\\
82	0\\
83	0\\
84	0\\
85	0\\
86	0\\
87	0\\
88	0\\
89	0\\
90	0\\
91	0\\
92	0\\
93	0\\
94	0\\
95	0\\
96	0\\
97	0\\
98	0\\
99	0\\
100	0\\
101	0\\
102	0\\
103	0\\
104	0\\
105	0\\
106	0\\
107	0\\
108	0\\
109	0\\
110	0\\
111	0\\
112	0\\
113	0\\
114	0\\
115	0\\
116	0\\
117	0\\
118	0\\
119	0\\
120	0\\
121	0\\
122	0\\
123	0\\
124	0\\
125	0\\
126	0\\
127	0\\
128	0\\
129	0\\
130	0\\
131	0\\
132	0\\
133	0\\
134	0\\
135	0\\
136	0\\
137	0\\
138	0\\
139	0\\
140	0\\
141	0\\
142	0\\
143	0\\
144	0\\
145	0\\
146	0\\
147	0\\
148	0\\
149	0\\
150	0\\
151	0\\
152	0\\
153	0\\
154	0\\
155	0\\
156	0\\
157	0\\
158	0\\
159	0\\
160	0\\
161	0\\
162	0\\
163	0\\
164	0\\
165	0\\
166	0\\
167	0\\
168	0\\
169	0\\
170	0\\
171	0\\
172	0\\
173	0\\
174	0\\
175	0\\
176	0\\
177	0\\
178	0\\
179	0\\
180	0\\
181	0\\
182	0\\
183	0\\
184	0\\
185	0\\
186	0\\
187	0\\
188	0\\
189	0\\
190	0\\
191	0\\
192	0\\
193	0\\
194	0\\
195	0\\
196	0\\
197	0\\
198	0\\
199	0\\
200	0\\
201	0\\
202	0\\
203	0\\
204	0\\
205	0\\
206	0\\
207	0\\
208	0\\
209	0\\
210	0\\
211	0\\
212	0\\
213	0\\
214	0\\
215	0\\
216	0\\
217	0\\
218	0\\
219	0\\
220	0\\
221	0\\
222	0\\
223	0\\
224	0\\
225	0\\
226	0\\
227	0\\
228	0\\
229	0\\
230	0\\
231	0\\
232	0\\
233	0\\
234	0\\
235	0\\
236	0\\
237	0\\
238	0\\
239	0\\
240	0\\
241	0\\
242	0\\
243	0\\
244	0\\
245	0\\
246	0\\
247	0\\
248	0\\
249	0\\
250	0\\
251	0\\
252	0\\
253	0\\
254	0\\
255	0\\
256	0\\
257	0\\
258	0\\
259	0\\
260	0\\
261	0\\
262	0\\
263	0\\
264	0\\
265	0\\
266	0\\
267	0\\
268	0\\
269	0\\
270	0\\
271	0\\
272	0\\
273	0\\
274	0\\
275	0\\
276	0\\
277	0\\
278	0\\
279	0\\
280	0\\
281	0\\
282	0\\
283	0\\
284	0\\
285	0\\
286	0\\
287	0\\
288	0\\
289	0\\
290	0\\
291	0\\
292	0\\
293	0\\
294	0\\
295	0\\
296	0\\
297	0\\
298	0\\
299	0\\
300	0\\
301	0\\
302	0\\
303	0\\
304	0\\
305	0\\
306	0\\
307	0\\
308	0\\
309	0\\
310	0\\
311	0\\
312	0\\
313	0\\
314	0\\
315	0\\
316	0\\
317	0\\
318	0\\
319	0\\
320	0\\
321	0\\
322	0\\
323	0\\
324	0\\
325	0\\
326	0\\
327	0\\
328	0\\
329	0\\
330	0\\
331	0\\
332	0\\
333	0\\
334	0\\
335	0\\
336	0\\
337	0\\
338	0\\
339	0\\
340	0\\
341	0\\
342	0\\
343	0\\
344	0\\
345	0\\
346	0\\
347	0\\
348	0\\
349	0\\
350	0\\
351	0\\
352	0\\
353	0\\
354	0\\
355	0\\
356	0\\
357	0\\
358	0\\
359	0\\
360	0\\
361	0\\
362	0\\
363	0\\
364	0\\
365	0\\
366	0\\
367	0\\
368	0\\
369	0\\
370	0\\
371	0\\
372	0\\
373	0\\
374	0\\
375	0\\
376	0\\
377	0\\
378	0\\
379	0\\
380	0\\
381	0\\
382	0\\
383	0\\
384	0\\
385	0\\
386	0\\
387	0\\
388	0\\
389	0\\
390	0\\
391	0\\
392	0\\
393	0\\
394	0\\
395	0\\
396	0\\
397	0\\
398	0\\
399	0\\
400	0\\
401	0\\
402	0\\
403	0\\
404	0\\
405	0\\
406	0\\
407	0\\
408	0\\
409	0\\
410	0\\
411	0\\
412	0\\
413	0\\
414	0\\
415	0\\
416	0\\
417	0\\
418	0\\
419	0\\
420	0\\
421	0\\
422	0\\
423	0\\
424	0\\
425	0\\
426	0\\
427	0\\
428	0\\
429	0\\
430	0\\
431	0\\
432	0\\
433	0\\
434	0\\
435	0\\
436	0\\
437	0\\
438	0\\
439	0\\
440	0\\
441	0\\
442	0\\
443	0\\
444	0\\
445	0\\
446	0\\
447	0\\
448	0\\
449	0\\
450	0\\
451	0\\
452	0\\
453	0\\
454	0\\
455	0\\
456	0\\
457	0\\
458	0\\
459	0\\
460	0\\
461	0\\
462	0\\
463	0\\
464	0\\
465	0\\
466	0\\
467	0\\
468	0\\
469	0\\
470	0\\
471	0\\
472	0\\
473	0\\
474	0\\
475	0\\
476	0\\
477	0\\
478	0\\
479	0\\
480	0\\
481	0\\
482	0\\
483	0\\
484	0\\
485	0\\
486	0\\
487	0\\
488	0\\
489	0\\
490	0\\
491	0\\
492	0\\
493	0\\
494	0\\
495	0\\
496	0\\
497	0\\
498	0\\
499	0\\
500	0\\
501	0\\
502	0\\
503	0\\
504	0\\
505	0\\
506	0\\
507	0\\
508	0\\
509	0\\
510	0\\
511	0\\
512	0\\
513	0\\
514	0\\
515	0\\
516	0\\
517	0\\
518	0\\
519	0\\
520	0\\
521	0\\
522	0\\
523	0\\
524	0\\
525	0\\
526	0\\
527	0\\
528	0\\
529	0\\
530	0\\
531	0\\
532	0\\
533	0\\
534	0\\
535	0\\
536	0\\
537	0\\
538	0\\
539	0\\
540	0\\
541	0\\
542	0\\
543	0\\
544	0\\
545	0\\
546	0\\
547	0\\
548	0\\
549	0\\
550	0\\
551	0\\
552	0\\
553	0\\
554	0\\
555	0\\
556	0\\
557	0\\
558	0\\
559	0\\
560	0\\
561	0\\
562	0\\
563	0\\
564	0\\
565	0\\
566	0\\
567	0\\
568	0\\
569	0\\
570	0\\
571	0\\
572	0\\
573	0\\
574	0\\
575	0\\
576	0\\
577	0\\
578	0\\
579	0\\
580	0\\
581	0\\
582	0\\
583	0\\
584	0\\
585	0\\
586	0\\
587	0\\
588	0\\
589	0\\
590	0\\
591	0\\
592	0\\
593	0\\
594	0\\
595	0\\
596	0\\
597	0\\
598	0\\
599	0\\
600	0\\
};
\addplot [color=mycolor1,solid,forget plot]
  table[row sep=crcr]{%
1	0\\
2	0\\
3	0\\
4	0\\
5	0\\
6	0\\
7	0\\
8	0\\
9	0\\
10	0\\
11	0\\
12	0\\
13	0\\
14	0\\
15	0\\
16	0\\
17	0\\
18	0\\
19	0\\
20	0\\
21	0\\
22	0\\
23	0\\
24	0\\
25	0\\
26	0\\
27	0\\
28	0\\
29	0\\
30	0\\
31	0\\
32	0\\
33	0\\
34	0\\
35	0\\
36	0\\
37	0\\
38	0\\
39	0\\
40	0\\
41	0\\
42	0\\
43	0\\
44	0\\
45	0\\
46	0\\
47	0\\
48	0\\
49	0\\
50	0\\
51	0\\
52	0\\
53	0\\
54	0\\
55	0\\
56	0\\
57	0\\
58	0\\
59	0\\
60	0\\
61	0\\
62	0\\
63	0\\
64	0\\
65	0\\
66	0\\
67	0\\
68	0\\
69	0\\
70	0\\
71	0\\
72	0\\
73	0\\
74	0\\
75	0\\
76	0\\
77	0\\
78	0\\
79	0\\
80	0\\
81	0\\
82	0\\
83	0\\
84	0\\
85	0\\
86	0\\
87	0\\
88	0\\
89	0\\
90	0\\
91	0\\
92	0\\
93	0\\
94	0\\
95	0\\
96	0\\
97	0\\
98	0\\
99	0\\
100	0\\
101	0\\
102	0\\
103	0\\
104	0\\
105	0\\
106	0\\
107	0\\
108	0\\
109	0\\
110	0\\
111	0\\
112	0\\
113	0\\
114	0\\
115	0\\
116	0\\
117	0\\
118	0\\
119	0\\
120	0\\
121	0\\
122	0\\
123	0\\
124	0\\
125	0\\
126	0\\
127	0\\
128	0\\
129	0\\
130	0\\
131	0\\
132	0\\
133	0\\
134	0\\
135	0\\
136	0\\
137	0\\
138	0\\
139	0\\
140	0\\
141	0\\
142	0\\
143	0\\
144	0\\
145	0\\
146	0\\
147	0\\
148	0\\
149	0\\
150	0\\
151	0\\
152	0\\
153	0\\
154	0\\
155	0\\
156	0\\
157	0\\
158	0\\
159	0\\
160	0\\
161	0\\
162	0\\
163	0\\
164	0\\
165	0\\
166	0\\
167	0\\
168	0\\
169	0\\
170	0\\
171	0\\
172	0\\
173	0\\
174	0\\
175	0\\
176	0\\
177	0\\
178	0\\
179	0\\
180	0\\
181	0\\
182	0\\
183	0\\
184	0\\
185	0\\
186	0\\
187	0\\
188	0\\
189	0\\
190	0\\
191	0\\
192	0\\
193	0\\
194	0\\
195	0\\
196	0\\
197	0\\
198	0\\
199	0\\
200	0\\
201	0\\
202	0\\
203	0\\
204	0\\
205	0\\
206	0\\
207	0\\
208	0\\
209	0\\
210	0\\
211	0\\
212	0\\
213	0\\
214	0\\
215	0\\
216	0\\
217	0\\
218	0\\
219	0\\
220	0\\
221	0\\
222	0\\
223	0\\
224	0\\
225	0\\
226	0\\
227	0\\
228	0\\
229	0\\
230	0\\
231	0\\
232	0\\
233	0\\
234	0\\
235	0\\
236	0\\
237	0\\
238	0\\
239	0\\
240	0\\
241	0\\
242	0\\
243	0\\
244	0\\
245	0\\
246	0\\
247	0\\
248	0\\
249	0\\
250	0\\
251	0\\
252	0\\
253	0\\
254	0\\
255	0\\
256	0\\
257	0\\
258	0\\
259	0\\
260	0\\
261	0\\
262	0\\
263	0\\
264	0\\
265	0\\
266	0\\
267	0\\
268	0\\
269	0\\
270	0\\
271	0\\
272	0\\
273	0\\
274	0\\
275	0\\
276	0\\
277	0\\
278	0\\
279	0\\
280	0\\
281	0\\
282	0\\
283	0\\
284	0\\
285	0\\
286	0\\
287	0\\
288	0\\
289	0\\
290	0\\
291	0\\
292	0\\
293	0\\
294	0\\
295	0\\
296	0\\
297	0\\
298	0\\
299	0\\
300	0\\
301	0\\
302	0\\
303	0\\
304	0\\
305	0\\
306	0\\
307	0\\
308	0\\
309	0\\
310	0\\
311	0\\
312	0\\
313	0\\
314	0\\
315	0\\
316	0\\
317	0\\
318	0\\
319	0\\
320	0\\
321	0\\
322	0\\
323	0\\
324	0\\
325	0\\
326	0\\
327	0\\
328	0\\
329	0\\
330	0\\
331	0\\
332	0\\
333	0\\
334	0\\
335	0\\
336	0\\
337	0\\
338	0\\
339	0\\
340	0\\
341	0\\
342	0\\
343	0\\
344	0\\
345	0\\
346	0\\
347	0\\
348	0\\
349	0\\
350	0\\
351	0\\
352	0\\
353	0\\
354	0\\
355	0\\
356	0\\
357	0\\
358	0\\
359	0\\
360	0\\
361	0\\
362	0\\
363	0\\
364	0\\
365	0\\
366	0\\
367	0\\
368	0\\
369	0\\
370	0\\
371	0\\
372	0\\
373	0\\
374	0\\
375	0\\
376	0\\
377	0\\
378	0\\
379	0\\
380	0\\
381	0\\
382	0\\
383	0\\
384	0\\
385	0\\
386	0\\
387	0\\
388	0\\
389	0\\
390	0\\
391	0\\
392	0\\
393	0\\
394	0\\
395	0\\
396	0\\
397	0\\
398	0\\
399	0\\
400	0\\
401	0\\
402	0\\
403	0\\
404	0\\
405	0\\
406	0\\
407	0\\
408	0\\
409	0\\
410	0\\
411	0\\
412	0\\
413	0\\
414	0\\
415	0\\
416	0\\
417	0\\
418	0\\
419	0\\
420	0\\
421	0\\
422	0\\
423	0\\
424	0\\
425	0\\
426	0\\
427	0\\
428	0\\
429	0\\
430	0\\
431	0\\
432	0\\
433	0\\
434	0\\
435	0\\
436	0\\
437	0\\
438	0\\
439	0\\
440	0\\
441	0\\
442	0\\
443	0\\
444	0\\
445	0\\
446	0\\
447	0\\
448	0\\
449	0\\
450	0\\
451	0\\
452	0\\
453	0\\
454	0\\
455	0\\
456	0\\
457	0\\
458	0\\
459	0\\
460	0\\
461	0\\
462	0\\
463	0\\
464	0\\
465	0\\
466	0\\
467	0\\
468	0\\
469	0\\
470	0\\
471	0\\
472	0\\
473	0\\
474	0\\
475	0\\
476	0\\
477	0\\
478	0\\
479	0\\
480	0\\
481	0\\
482	0\\
483	0\\
484	0\\
485	0\\
486	0\\
487	0\\
488	0\\
489	0\\
490	0\\
491	0\\
492	0\\
493	0\\
494	0\\
495	0\\
496	0\\
497	0\\
498	0\\
499	0\\
500	0\\
501	0\\
502	0\\
503	0\\
504	0\\
505	0\\
506	0\\
507	0\\
508	0\\
509	0\\
510	0\\
511	0\\
512	0\\
513	0\\
514	0\\
515	0\\
516	0\\
517	0\\
518	0\\
519	0\\
520	0\\
521	0\\
522	0\\
523	0\\
524	0\\
525	0\\
526	0\\
527	0\\
528	0\\
529	0\\
530	0\\
531	0\\
532	0\\
533	0\\
534	0\\
535	0\\
536	0\\
537	0\\
538	0\\
539	0\\
540	0\\
541	0\\
542	0\\
543	0\\
544	0\\
545	0\\
546	0\\
547	0\\
548	0\\
549	0\\
550	0\\
551	0\\
552	0\\
553	0\\
554	0\\
555	0\\
556	0\\
557	0\\
558	0\\
559	0\\
560	0\\
561	0\\
562	0\\
563	0\\
564	0\\
565	0\\
566	0\\
567	0\\
568	0\\
569	0\\
570	0\\
571	0\\
572	0\\
573	0\\
574	0\\
575	0\\
576	0\\
577	0\\
578	0\\
579	0\\
580	0\\
581	0\\
582	0\\
583	0\\
584	0\\
585	0\\
586	0\\
587	0\\
588	0\\
589	0\\
590	0\\
591	0\\
592	0\\
593	0\\
594	0\\
595	0\\
596	0\\
597	0\\
598	0\\
599	0\\
600	0\\
};
\addplot [color=mycolor2,solid,forget plot]
  table[row sep=crcr]{%
1	0\\
2	0\\
3	0\\
4	0\\
5	0\\
6	0\\
7	0\\
8	0\\
9	0\\
10	0\\
11	0\\
12	0\\
13	0\\
14	0\\
15	0\\
16	0\\
17	0\\
18	0\\
19	0\\
20	0\\
21	0\\
22	0\\
23	0\\
24	0\\
25	0\\
26	0\\
27	0\\
28	0\\
29	0\\
30	0\\
31	0\\
32	0\\
33	0\\
34	0\\
35	0\\
36	0\\
37	0\\
38	0\\
39	0\\
40	0\\
41	0\\
42	0\\
43	0\\
44	0\\
45	0\\
46	0\\
47	0\\
48	0\\
49	0\\
50	0\\
51	0\\
52	0\\
53	0\\
54	0\\
55	0\\
56	0\\
57	0\\
58	0\\
59	0\\
60	0\\
61	0\\
62	0\\
63	0\\
64	0\\
65	0\\
66	0\\
67	0\\
68	0\\
69	0\\
70	0\\
71	0\\
72	0\\
73	0\\
74	0\\
75	0\\
76	0\\
77	0\\
78	0\\
79	0\\
80	0\\
81	0\\
82	0\\
83	0\\
84	0\\
85	0\\
86	0\\
87	0\\
88	0\\
89	0\\
90	0\\
91	0\\
92	0\\
93	0\\
94	0\\
95	0\\
96	0\\
97	0\\
98	0\\
99	0\\
100	0\\
101	0\\
102	0\\
103	0\\
104	0\\
105	0\\
106	0\\
107	0\\
108	0\\
109	0\\
110	0\\
111	0\\
112	0\\
113	0\\
114	0\\
115	0\\
116	0\\
117	0\\
118	0\\
119	0\\
120	0\\
121	0\\
122	0\\
123	0\\
124	0\\
125	0\\
126	0\\
127	0\\
128	0\\
129	0\\
130	0\\
131	0\\
132	0\\
133	0\\
134	0\\
135	0\\
136	0\\
137	0\\
138	0\\
139	0\\
140	0\\
141	0\\
142	0\\
143	0\\
144	0\\
145	0\\
146	0\\
147	0\\
148	0\\
149	0\\
150	0\\
151	0\\
152	0\\
153	0\\
154	0\\
155	0\\
156	0\\
157	0\\
158	0\\
159	0\\
160	0\\
161	0\\
162	0\\
163	0\\
164	0\\
165	0\\
166	0\\
167	0\\
168	0\\
169	0\\
170	0\\
171	0\\
172	0\\
173	0\\
174	0\\
175	0\\
176	0\\
177	0\\
178	0\\
179	0\\
180	0\\
181	0\\
182	0\\
183	0\\
184	0\\
185	0\\
186	0\\
187	0\\
188	0\\
189	0\\
190	0\\
191	0\\
192	0\\
193	0\\
194	0\\
195	0\\
196	0\\
197	0\\
198	0\\
199	0\\
200	0\\
201	0\\
202	0\\
203	0\\
204	0\\
205	0\\
206	0\\
207	0\\
208	0\\
209	0\\
210	0\\
211	0\\
212	0\\
213	0\\
214	0\\
215	0\\
216	0\\
217	0\\
218	0\\
219	0\\
220	0\\
221	0\\
222	0\\
223	0\\
224	0\\
225	0\\
226	0\\
227	0\\
228	0\\
229	0\\
230	0\\
231	0\\
232	0\\
233	0\\
234	0\\
235	0\\
236	0\\
237	0\\
238	0\\
239	0\\
240	0\\
241	0\\
242	0\\
243	0\\
244	0\\
245	0\\
246	0\\
247	0\\
248	0\\
249	0\\
250	0\\
251	0\\
252	0\\
253	0\\
254	0\\
255	0\\
256	0\\
257	0\\
258	0\\
259	0\\
260	0\\
261	0\\
262	0\\
263	0\\
264	0\\
265	0\\
266	0\\
267	0\\
268	0\\
269	0\\
270	0\\
271	0\\
272	0\\
273	0\\
274	0\\
275	0\\
276	0\\
277	0\\
278	0\\
279	0\\
280	0\\
281	0\\
282	0\\
283	0\\
284	0\\
285	0\\
286	0\\
287	0\\
288	0\\
289	0\\
290	0\\
291	0\\
292	0\\
293	0\\
294	0\\
295	0\\
296	0\\
297	0\\
298	0\\
299	0\\
300	0\\
301	0\\
302	0\\
303	0\\
304	0\\
305	0\\
306	0\\
307	0\\
308	0\\
309	0\\
310	0\\
311	0\\
312	0\\
313	0\\
314	0\\
315	0\\
316	0\\
317	0\\
318	0\\
319	0\\
320	0\\
321	0\\
322	0\\
323	0\\
324	0\\
325	0\\
326	0\\
327	0\\
328	0\\
329	0\\
330	0\\
331	0\\
332	0\\
333	0\\
334	0\\
335	0\\
336	0\\
337	0\\
338	0\\
339	0\\
340	0\\
341	0\\
342	0\\
343	0\\
344	0\\
345	0\\
346	0\\
347	0\\
348	0\\
349	0\\
350	0\\
351	0\\
352	0\\
353	0\\
354	0\\
355	0\\
356	0\\
357	0\\
358	0\\
359	0\\
360	0\\
361	0\\
362	0\\
363	0\\
364	0\\
365	0\\
366	0\\
367	0\\
368	0\\
369	0\\
370	0\\
371	0\\
372	0\\
373	0\\
374	0\\
375	0\\
376	0\\
377	0\\
378	0\\
379	0\\
380	0\\
381	0\\
382	0\\
383	0\\
384	0\\
385	0\\
386	0\\
387	0\\
388	0\\
389	0\\
390	0\\
391	0\\
392	0\\
393	0\\
394	0\\
395	0\\
396	0\\
397	0\\
398	0\\
399	0\\
400	0\\
401	0\\
402	0\\
403	0\\
404	0\\
405	0\\
406	0\\
407	0\\
408	0\\
409	0\\
410	0\\
411	0\\
412	0\\
413	0\\
414	0\\
415	0\\
416	0\\
417	0\\
418	0\\
419	0\\
420	0\\
421	0\\
422	0\\
423	0\\
424	0\\
425	0\\
426	0\\
427	0\\
428	0\\
429	0\\
430	0\\
431	0\\
432	0\\
433	0\\
434	0\\
435	0\\
436	0\\
437	0\\
438	0\\
439	0\\
440	0\\
441	0\\
442	0\\
443	0\\
444	0\\
445	0\\
446	0\\
447	0\\
448	0\\
449	0\\
450	0\\
451	0\\
452	0\\
453	0\\
454	0\\
455	0\\
456	0\\
457	0\\
458	0\\
459	0\\
460	0\\
461	0\\
462	0\\
463	0\\
464	0\\
465	0\\
466	0\\
467	0\\
468	0\\
469	0\\
470	0\\
471	0\\
472	0\\
473	0\\
474	0\\
475	0\\
476	0\\
477	0\\
478	0\\
479	0\\
480	0\\
481	0\\
482	0\\
483	0\\
484	0\\
485	0\\
486	0\\
487	0\\
488	0\\
489	0\\
490	0\\
491	0\\
492	0\\
493	0\\
494	0\\
495	0\\
496	0\\
497	0\\
498	0\\
499	0\\
500	0\\
501	0\\
502	0\\
503	0\\
504	0\\
505	0\\
506	0\\
507	0\\
508	0\\
509	0\\
510	0\\
511	0\\
512	0\\
513	0\\
514	0\\
515	0\\
516	0\\
517	0\\
518	0\\
519	0\\
520	0\\
521	0\\
522	0\\
523	0\\
524	0\\
525	0\\
526	0\\
527	0\\
528	0\\
529	0\\
530	0\\
531	0\\
532	0\\
533	0\\
534	0\\
535	0\\
536	0\\
537	0\\
538	0\\
539	0\\
540	0\\
541	0\\
542	0\\
543	0\\
544	0\\
545	0\\
546	0\\
547	0\\
548	0\\
549	0\\
550	0\\
551	0\\
552	0\\
553	0\\
554	0\\
555	0\\
556	0\\
557	0\\
558	0\\
559	0\\
560	0\\
561	0\\
562	0\\
563	0\\
564	0\\
565	0\\
566	0\\
567	0\\
568	0\\
569	0\\
570	0\\
571	0\\
572	0\\
573	0\\
574	0\\
575	0\\
576	0\\
577	0\\
578	0\\
579	0\\
580	0\\
581	0\\
582	0\\
583	0\\
584	0\\
585	0\\
586	0\\
587	0\\
588	0\\
589	0\\
590	0\\
591	0\\
592	0\\
593	0\\
594	0\\
595	0\\
596	0\\
597	0\\
598	0\\
599	0\\
600	0\\
};
\addplot [color=mycolor3,solid,forget plot]
  table[row sep=crcr]{%
1	0\\
2	0\\
3	0\\
4	0\\
5	0\\
6	0\\
7	0\\
8	0\\
9	0\\
10	0\\
11	0\\
12	0\\
13	0\\
14	0\\
15	0\\
16	0\\
17	0\\
18	0\\
19	0\\
20	0\\
21	0\\
22	0\\
23	0\\
24	0\\
25	0\\
26	0\\
27	0\\
28	0\\
29	0\\
30	0\\
31	0\\
32	0\\
33	0\\
34	0\\
35	0\\
36	0\\
37	0\\
38	0\\
39	0\\
40	0\\
41	0\\
42	0\\
43	0\\
44	0\\
45	0\\
46	0\\
47	0\\
48	0\\
49	0\\
50	0\\
51	0\\
52	0\\
53	0\\
54	0\\
55	0\\
56	0\\
57	0\\
58	0\\
59	0\\
60	0\\
61	0\\
62	0\\
63	0\\
64	0\\
65	0\\
66	0\\
67	0\\
68	0\\
69	0\\
70	0\\
71	0\\
72	0\\
73	0\\
74	0\\
75	0\\
76	0\\
77	0\\
78	0\\
79	0\\
80	0\\
81	0\\
82	0\\
83	0\\
84	0\\
85	0\\
86	0\\
87	0\\
88	0\\
89	0\\
90	0\\
91	0\\
92	0\\
93	0\\
94	0\\
95	0\\
96	0\\
97	0\\
98	0\\
99	0\\
100	0\\
101	0\\
102	0\\
103	0\\
104	0\\
105	0\\
106	0\\
107	0\\
108	0\\
109	0\\
110	0\\
111	0\\
112	0\\
113	0\\
114	0\\
115	0\\
116	0\\
117	0\\
118	0\\
119	0\\
120	0\\
121	0\\
122	0\\
123	0\\
124	0\\
125	0\\
126	0\\
127	0\\
128	0\\
129	0\\
130	0\\
131	0\\
132	0\\
133	0\\
134	0\\
135	0\\
136	0\\
137	0\\
138	0\\
139	0\\
140	0\\
141	0\\
142	0\\
143	0\\
144	0\\
145	0\\
146	0\\
147	0\\
148	0\\
149	0\\
150	0\\
151	0\\
152	0\\
153	0\\
154	0\\
155	0\\
156	0\\
157	0\\
158	0\\
159	0\\
160	0\\
161	0\\
162	0\\
163	0\\
164	0\\
165	0\\
166	0\\
167	0\\
168	0\\
169	0\\
170	0\\
171	0\\
172	0\\
173	0\\
174	0\\
175	0\\
176	0\\
177	0\\
178	0\\
179	0\\
180	0\\
181	0\\
182	0\\
183	0\\
184	0\\
185	0\\
186	0\\
187	0\\
188	0\\
189	0\\
190	0\\
191	0\\
192	0\\
193	0\\
194	0\\
195	0\\
196	0\\
197	0\\
198	0\\
199	0\\
200	0\\
201	0\\
202	0\\
203	0\\
204	0\\
205	0\\
206	0\\
207	0\\
208	0\\
209	0\\
210	0\\
211	0\\
212	0\\
213	0\\
214	0\\
215	0\\
216	0\\
217	0\\
218	0\\
219	0\\
220	0\\
221	0\\
222	0\\
223	0\\
224	0\\
225	0\\
226	0\\
227	0\\
228	0\\
229	0\\
230	0\\
231	0\\
232	0\\
233	0\\
234	0\\
235	0\\
236	0\\
237	0\\
238	0\\
239	0\\
240	0\\
241	0\\
242	0\\
243	0\\
244	0\\
245	0\\
246	0\\
247	0\\
248	0\\
249	0\\
250	0\\
251	0\\
252	0\\
253	0\\
254	0\\
255	0\\
256	0\\
257	0\\
258	0\\
259	0\\
260	0\\
261	0\\
262	0\\
263	0\\
264	0\\
265	0\\
266	0\\
267	0\\
268	0\\
269	0\\
270	0\\
271	0\\
272	0\\
273	0\\
274	0\\
275	0\\
276	0\\
277	0\\
278	0\\
279	0\\
280	0\\
281	0\\
282	0\\
283	0\\
284	0\\
285	0\\
286	0\\
287	0\\
288	0\\
289	0\\
290	0\\
291	0\\
292	0\\
293	0\\
294	0\\
295	0\\
296	0\\
297	0\\
298	0\\
299	0\\
300	0\\
301	0\\
302	0\\
303	0\\
304	0\\
305	0\\
306	0\\
307	0\\
308	0\\
309	0\\
310	0\\
311	0\\
312	0\\
313	0\\
314	0\\
315	0\\
316	0\\
317	0\\
318	0\\
319	0\\
320	0\\
321	0\\
322	0\\
323	0\\
324	0\\
325	0\\
326	0\\
327	0\\
328	0\\
329	0\\
330	0\\
331	0\\
332	0\\
333	0\\
334	0\\
335	0\\
336	0\\
337	0\\
338	0\\
339	0\\
340	0\\
341	0\\
342	0\\
343	0\\
344	0\\
345	0\\
346	0\\
347	0\\
348	0\\
349	0\\
350	0\\
351	0\\
352	0\\
353	0\\
354	0\\
355	0\\
356	0\\
357	0\\
358	0\\
359	0\\
360	0\\
361	0\\
362	0\\
363	0\\
364	0\\
365	0\\
366	0\\
367	0\\
368	0\\
369	0\\
370	0\\
371	0\\
372	0\\
373	0\\
374	0\\
375	0\\
376	0\\
377	0\\
378	0\\
379	0\\
380	0\\
381	0\\
382	0\\
383	0\\
384	0\\
385	0\\
386	0\\
387	0\\
388	0\\
389	0\\
390	0\\
391	0\\
392	0\\
393	0\\
394	0\\
395	0\\
396	0\\
397	0\\
398	0\\
399	0\\
400	0\\
401	0\\
402	0\\
403	0\\
404	0\\
405	0\\
406	0\\
407	0\\
408	0\\
409	0\\
410	0\\
411	0\\
412	0\\
413	0\\
414	0\\
415	0\\
416	0\\
417	0\\
418	0\\
419	0\\
420	0\\
421	0\\
422	0\\
423	0\\
424	0\\
425	0\\
426	0\\
427	0\\
428	0\\
429	0\\
430	0\\
431	0\\
432	0\\
433	0\\
434	0\\
435	0\\
436	0\\
437	0\\
438	0\\
439	0\\
440	0\\
441	0\\
442	0\\
443	0\\
444	0\\
445	0\\
446	0\\
447	0\\
448	0\\
449	0\\
450	0\\
451	0\\
452	0\\
453	0\\
454	0\\
455	0\\
456	0\\
457	0\\
458	0\\
459	0\\
460	0\\
461	0\\
462	0\\
463	0\\
464	0\\
465	0\\
466	0\\
467	0\\
468	0\\
469	0\\
470	0\\
471	0\\
472	0\\
473	0\\
474	0\\
475	0\\
476	0\\
477	0\\
478	0\\
479	0\\
480	0\\
481	0\\
482	0\\
483	0\\
484	0\\
485	0\\
486	0\\
487	0\\
488	0\\
489	0\\
490	0\\
491	0\\
492	0\\
493	0\\
494	0\\
495	0\\
496	0\\
497	0\\
498	0\\
499	0\\
500	0\\
501	0\\
502	0\\
503	0\\
504	0\\
505	0\\
506	0\\
507	0\\
508	0\\
509	0\\
510	0\\
511	0\\
512	0\\
513	0\\
514	0\\
515	0\\
516	0\\
517	0\\
518	0\\
519	0\\
520	0\\
521	0\\
522	0\\
523	0\\
524	0\\
525	0\\
526	0\\
527	0\\
528	0\\
529	0\\
530	0\\
531	0\\
532	0\\
533	0\\
534	0\\
535	0\\
536	0\\
537	0\\
538	0\\
539	0\\
540	0\\
541	0\\
542	0\\
543	0\\
544	0\\
545	0\\
546	0\\
547	0\\
548	0\\
549	0\\
550	0\\
551	0\\
552	0\\
553	0\\
554	0\\
555	0\\
556	0\\
557	0\\
558	0\\
559	0\\
560	0\\
561	0\\
562	0\\
563	0\\
564	0\\
565	0\\
566	0\\
567	0\\
568	0\\
569	0\\
570	0\\
571	0\\
572	0\\
573	0\\
574	0\\
575	0\\
576	0\\
577	0\\
578	0\\
579	0\\
580	0\\
581	0\\
582	0\\
583	0\\
584	0\\
585	0\\
586	0\\
587	0\\
588	0\\
589	0\\
590	0\\
591	0\\
592	0\\
593	0\\
594	0\\
595	0\\
596	0\\
597	0\\
598	0\\
599	0\\
600	0\\
};
\addplot [color=mycolor4,solid,forget plot]
  table[row sep=crcr]{%
1	0\\
2	0\\
3	0\\
4	0\\
5	0\\
6	0\\
7	0\\
8	0\\
9	0\\
10	0\\
11	0\\
12	0\\
13	0\\
14	0\\
15	0\\
16	0\\
17	0\\
18	0\\
19	0\\
20	0\\
21	0\\
22	0\\
23	0\\
24	0\\
25	0\\
26	0\\
27	0\\
28	0\\
29	0\\
30	0\\
31	0\\
32	0\\
33	0\\
34	0\\
35	0\\
36	0\\
37	0\\
38	0\\
39	0\\
40	0\\
41	0\\
42	0\\
43	0\\
44	0\\
45	0\\
46	0\\
47	0\\
48	0\\
49	0\\
50	0\\
51	0\\
52	0\\
53	0\\
54	0\\
55	0\\
56	0\\
57	0\\
58	0\\
59	0\\
60	0\\
61	0\\
62	0\\
63	0\\
64	0\\
65	0\\
66	0\\
67	0\\
68	0\\
69	0\\
70	0\\
71	0\\
72	0\\
73	0\\
74	0\\
75	0\\
76	0\\
77	0\\
78	0\\
79	0\\
80	0\\
81	0\\
82	0\\
83	0\\
84	0\\
85	0\\
86	0\\
87	0\\
88	0\\
89	0\\
90	0\\
91	0\\
92	0\\
93	0\\
94	0\\
95	0\\
96	0\\
97	0\\
98	0\\
99	0\\
100	0\\
101	0\\
102	0\\
103	0\\
104	0\\
105	0\\
106	0\\
107	0\\
108	0\\
109	0\\
110	0\\
111	0\\
112	0\\
113	0\\
114	0\\
115	0\\
116	0\\
117	0\\
118	0\\
119	0\\
120	0\\
121	0\\
122	0\\
123	0\\
124	0\\
125	0\\
126	0\\
127	0\\
128	0\\
129	0\\
130	0\\
131	0\\
132	0\\
133	0\\
134	0\\
135	0\\
136	0\\
137	0\\
138	0\\
139	0\\
140	0\\
141	0\\
142	0\\
143	0\\
144	0\\
145	0\\
146	0\\
147	0\\
148	0\\
149	0\\
150	0\\
151	0\\
152	0\\
153	0\\
154	0\\
155	0\\
156	0\\
157	0\\
158	0\\
159	0\\
160	0\\
161	0\\
162	0\\
163	0\\
164	0\\
165	0\\
166	0\\
167	0\\
168	0\\
169	0\\
170	0\\
171	0\\
172	0\\
173	0\\
174	0\\
175	0\\
176	0\\
177	0\\
178	0\\
179	0\\
180	0\\
181	0\\
182	0\\
183	0\\
184	0\\
185	0\\
186	0\\
187	0\\
188	0\\
189	0\\
190	0\\
191	0\\
192	0\\
193	0\\
194	0\\
195	0\\
196	0\\
197	0\\
198	0\\
199	0\\
200	0\\
201	0\\
202	0\\
203	0\\
204	0\\
205	0\\
206	0\\
207	0\\
208	0\\
209	0\\
210	0\\
211	0\\
212	0\\
213	0\\
214	0\\
215	0\\
216	0\\
217	0\\
218	0\\
219	0\\
220	0\\
221	0\\
222	0\\
223	0\\
224	0\\
225	0\\
226	0\\
227	0\\
228	0\\
229	0\\
230	0\\
231	0\\
232	0\\
233	0\\
234	0\\
235	0\\
236	0\\
237	0\\
238	0\\
239	0\\
240	0\\
241	0\\
242	0\\
243	0\\
244	0\\
245	0\\
246	0\\
247	0\\
248	0\\
249	0\\
250	0\\
251	0\\
252	0\\
253	0\\
254	0\\
255	0\\
256	0\\
257	0\\
258	0\\
259	0\\
260	0\\
261	0\\
262	0\\
263	0\\
264	0\\
265	0\\
266	0\\
267	0\\
268	0\\
269	0\\
270	0\\
271	0\\
272	0\\
273	0\\
274	0\\
275	0\\
276	0\\
277	0\\
278	0\\
279	0\\
280	0\\
281	0\\
282	0\\
283	0\\
284	0\\
285	0\\
286	0\\
287	0\\
288	0\\
289	0\\
290	0\\
291	0\\
292	0\\
293	0\\
294	0\\
295	0\\
296	0\\
297	0\\
298	0\\
299	0\\
300	0\\
301	0\\
302	0\\
303	0\\
304	0\\
305	0\\
306	0\\
307	0\\
308	0\\
309	0\\
310	0\\
311	0\\
312	0\\
313	0\\
314	0\\
315	0\\
316	0\\
317	0\\
318	0\\
319	0\\
320	0\\
321	0\\
322	0\\
323	0\\
324	0\\
325	0\\
326	0\\
327	0\\
328	0\\
329	0\\
330	0\\
331	0\\
332	0\\
333	0\\
334	0\\
335	0\\
336	0\\
337	0\\
338	0\\
339	0\\
340	0\\
341	0\\
342	0\\
343	0\\
344	0\\
345	0\\
346	0\\
347	0\\
348	0\\
349	0\\
350	0\\
351	0\\
352	0\\
353	0\\
354	0\\
355	0\\
356	0\\
357	0\\
358	0\\
359	0\\
360	0\\
361	0\\
362	0\\
363	0\\
364	0\\
365	0\\
366	0\\
367	0\\
368	0\\
369	0\\
370	0\\
371	0\\
372	0\\
373	0\\
374	0\\
375	0\\
376	0\\
377	0\\
378	0\\
379	0\\
380	0\\
381	0\\
382	0\\
383	0\\
384	0\\
385	0\\
386	0\\
387	0\\
388	0\\
389	0\\
390	0\\
391	0\\
392	0\\
393	0\\
394	0\\
395	0\\
396	0\\
397	0\\
398	0\\
399	0\\
400	0\\
401	0\\
402	0\\
403	0\\
404	0\\
405	0\\
406	0\\
407	0\\
408	0\\
409	0\\
410	0\\
411	0\\
412	0\\
413	0\\
414	0\\
415	0\\
416	0\\
417	0\\
418	0\\
419	0\\
420	0\\
421	0\\
422	0\\
423	0\\
424	0\\
425	0\\
426	0\\
427	0\\
428	0\\
429	0\\
430	0\\
431	0\\
432	0\\
433	0\\
434	0\\
435	0\\
436	0\\
437	0\\
438	0\\
439	0\\
440	0\\
441	0\\
442	0\\
443	0\\
444	0\\
445	0\\
446	0\\
447	0\\
448	0\\
449	0\\
450	0\\
451	0\\
452	0\\
453	0\\
454	0\\
455	0\\
456	0\\
457	0\\
458	0\\
459	0\\
460	0\\
461	0\\
462	0\\
463	0\\
464	0\\
465	0\\
466	0\\
467	0\\
468	0\\
469	0\\
470	0\\
471	0\\
472	0\\
473	0\\
474	0\\
475	0\\
476	0\\
477	0\\
478	0\\
479	0\\
480	0\\
481	0\\
482	0\\
483	0\\
484	0\\
485	0\\
486	0\\
487	0\\
488	0\\
489	0\\
490	0\\
491	0\\
492	0\\
493	0\\
494	0\\
495	0\\
496	0\\
497	0\\
498	0\\
499	0\\
500	0\\
501	0\\
502	0\\
503	0\\
504	0\\
505	0\\
506	0\\
507	0\\
508	0\\
509	0\\
510	0\\
511	0\\
512	0\\
513	0\\
514	0\\
515	0\\
516	0\\
517	0\\
518	0\\
519	0\\
520	0\\
521	0\\
522	0\\
523	0\\
524	0\\
525	0\\
526	0\\
527	0\\
528	0\\
529	0\\
530	0\\
531	0\\
532	0\\
533	0\\
534	0\\
535	0\\
536	0\\
537	0\\
538	0\\
539	0\\
540	0\\
541	0\\
542	0\\
543	0\\
544	0\\
545	0\\
546	0\\
547	0\\
548	0\\
549	0\\
550	0\\
551	0\\
552	0\\
553	0\\
554	0\\
555	0\\
556	0\\
557	0\\
558	0\\
559	0\\
560	0\\
561	0\\
562	0\\
563	0\\
564	0\\
565	0\\
566	0\\
567	0\\
568	0\\
569	0\\
570	0\\
571	0\\
572	0\\
573	0\\
574	0\\
575	0\\
576	0\\
577	0\\
578	0\\
579	0\\
580	0\\
581	0\\
582	0\\
583	0\\
584	0\\
585	0\\
586	0\\
587	0\\
588	0\\
589	0\\
590	0\\
591	0\\
592	0\\
593	0\\
594	0\\
595	0\\
596	0\\
597	0\\
598	0\\
599	0\\
600	0\\
};
\addplot [color=mycolor5,solid,forget plot]
  table[row sep=crcr]{%
1	0\\
2	0\\
3	0\\
4	0\\
5	0\\
6	0\\
7	0\\
8	0\\
9	0\\
10	0\\
11	0\\
12	0\\
13	0\\
14	0\\
15	0\\
16	0\\
17	0\\
18	0\\
19	0\\
20	0\\
21	0\\
22	0\\
23	0\\
24	0\\
25	0\\
26	0\\
27	0\\
28	0\\
29	0\\
30	0\\
31	0\\
32	0\\
33	0\\
34	0\\
35	0\\
36	0\\
37	0\\
38	0\\
39	0\\
40	0\\
41	0\\
42	0\\
43	0\\
44	0\\
45	0\\
46	0\\
47	0\\
48	0\\
49	0\\
50	0\\
51	0\\
52	0\\
53	0\\
54	0\\
55	0\\
56	0\\
57	0\\
58	0\\
59	0\\
60	0\\
61	0\\
62	0\\
63	0\\
64	0\\
65	0\\
66	0\\
67	0\\
68	0\\
69	0\\
70	0\\
71	0\\
72	0\\
73	0\\
74	0\\
75	0\\
76	0\\
77	0\\
78	0\\
79	0\\
80	0\\
81	0\\
82	0\\
83	0\\
84	0\\
85	0\\
86	0\\
87	0\\
88	0\\
89	0\\
90	0\\
91	0\\
92	0\\
93	0\\
94	0\\
95	0\\
96	0\\
97	0\\
98	0\\
99	0\\
100	0\\
101	0\\
102	0\\
103	0\\
104	0\\
105	0\\
106	0\\
107	0\\
108	0\\
109	0\\
110	0\\
111	0\\
112	0\\
113	0\\
114	0\\
115	0\\
116	0\\
117	0\\
118	0\\
119	0\\
120	0\\
121	0\\
122	0\\
123	0\\
124	0\\
125	0\\
126	0\\
127	0\\
128	0\\
129	0\\
130	0\\
131	0\\
132	0\\
133	0\\
134	0\\
135	0\\
136	0\\
137	0\\
138	0\\
139	0\\
140	0\\
141	0\\
142	0\\
143	0\\
144	0\\
145	0\\
146	0\\
147	0\\
148	0\\
149	0\\
150	0\\
151	0\\
152	0\\
153	0\\
154	0\\
155	0\\
156	0\\
157	0\\
158	0\\
159	0\\
160	0\\
161	0\\
162	0\\
163	0\\
164	0\\
165	0\\
166	0\\
167	0\\
168	0\\
169	0\\
170	0\\
171	0\\
172	0\\
173	0\\
174	0\\
175	0\\
176	0\\
177	0\\
178	0\\
179	0\\
180	0\\
181	0\\
182	0\\
183	0\\
184	0\\
185	0\\
186	0\\
187	0\\
188	0\\
189	0\\
190	0\\
191	0\\
192	0\\
193	0\\
194	0\\
195	0\\
196	0\\
197	0\\
198	0\\
199	0\\
200	0\\
201	0\\
202	0\\
203	0\\
204	0\\
205	0\\
206	0\\
207	0\\
208	0\\
209	0\\
210	0\\
211	0\\
212	0\\
213	0\\
214	0\\
215	0\\
216	0\\
217	0\\
218	0\\
219	0\\
220	0\\
221	0\\
222	0\\
223	0\\
224	0\\
225	0\\
226	0\\
227	0\\
228	0\\
229	0\\
230	0\\
231	0\\
232	0\\
233	0\\
234	0\\
235	0\\
236	0\\
237	0\\
238	0\\
239	0\\
240	0\\
241	0\\
242	0\\
243	0\\
244	0\\
245	0\\
246	0\\
247	0\\
248	0\\
249	0\\
250	0\\
251	0\\
252	0\\
253	0\\
254	0\\
255	0\\
256	0\\
257	0\\
258	0\\
259	0\\
260	0\\
261	0\\
262	0\\
263	0\\
264	0\\
265	0\\
266	0\\
267	0\\
268	0\\
269	0\\
270	0\\
271	0\\
272	0\\
273	0\\
274	0\\
275	0\\
276	0\\
277	0\\
278	0\\
279	0\\
280	0\\
281	0\\
282	0\\
283	0\\
284	0\\
285	0\\
286	0\\
287	0\\
288	0\\
289	0\\
290	0\\
291	0\\
292	0\\
293	0\\
294	0\\
295	0\\
296	0\\
297	0\\
298	0\\
299	0\\
300	0\\
301	0\\
302	0\\
303	0\\
304	0\\
305	0\\
306	0\\
307	0\\
308	0\\
309	0\\
310	0\\
311	0\\
312	0\\
313	0\\
314	0\\
315	0\\
316	0\\
317	0\\
318	0\\
319	0\\
320	0\\
321	0\\
322	0\\
323	0\\
324	0\\
325	0\\
326	0\\
327	0\\
328	0\\
329	0\\
330	0\\
331	0\\
332	0\\
333	0\\
334	0\\
335	0\\
336	0\\
337	0\\
338	0\\
339	0\\
340	0\\
341	0\\
342	0\\
343	0\\
344	0\\
345	0\\
346	0\\
347	0\\
348	0\\
349	0\\
350	0\\
351	0\\
352	0\\
353	0\\
354	0\\
355	0\\
356	0\\
357	0\\
358	0\\
359	0\\
360	0\\
361	0\\
362	0\\
363	0\\
364	0\\
365	0\\
366	0\\
367	0\\
368	0\\
369	0\\
370	0\\
371	0\\
372	0\\
373	0\\
374	0\\
375	0\\
376	0\\
377	0\\
378	0\\
379	0\\
380	0\\
381	0\\
382	0\\
383	0\\
384	0\\
385	0\\
386	0\\
387	0\\
388	0\\
389	0\\
390	0\\
391	0\\
392	0\\
393	0\\
394	0\\
395	0\\
396	0\\
397	0\\
398	0\\
399	0\\
400	0\\
401	0\\
402	0\\
403	0\\
404	0\\
405	0\\
406	0\\
407	0\\
408	0\\
409	0\\
410	0\\
411	0\\
412	0\\
413	0\\
414	0\\
415	0\\
416	0\\
417	0\\
418	0\\
419	0\\
420	0\\
421	0\\
422	0\\
423	0\\
424	0\\
425	0\\
426	0\\
427	0\\
428	0\\
429	0\\
430	0\\
431	0\\
432	0\\
433	0\\
434	0\\
435	0\\
436	0\\
437	0\\
438	0\\
439	0\\
440	0\\
441	0\\
442	0\\
443	0\\
444	0\\
445	0\\
446	0\\
447	0\\
448	0\\
449	0\\
450	0\\
451	0\\
452	0\\
453	0\\
454	0\\
455	0\\
456	0\\
457	0\\
458	0\\
459	0\\
460	0\\
461	0\\
462	0\\
463	0\\
464	0\\
465	0\\
466	0\\
467	0\\
468	0\\
469	0\\
470	0\\
471	0\\
472	0\\
473	0\\
474	0\\
475	0\\
476	0\\
477	0\\
478	0\\
479	0\\
480	0\\
481	0\\
482	0\\
483	0\\
484	0\\
485	0\\
486	0\\
487	0\\
488	0\\
489	0\\
490	0\\
491	0\\
492	0\\
493	0\\
494	0\\
495	0\\
496	0\\
497	0\\
498	0\\
499	0\\
500	0\\
501	0\\
502	0\\
503	0\\
504	0\\
505	0\\
506	0\\
507	0\\
508	0\\
509	0\\
510	0\\
511	0\\
512	0\\
513	0\\
514	0\\
515	0\\
516	0\\
517	0\\
518	0\\
519	0\\
520	0\\
521	0\\
522	0\\
523	0\\
524	0\\
525	0\\
526	0\\
527	0\\
528	0\\
529	0\\
530	0\\
531	0\\
532	0\\
533	0\\
534	0\\
535	0\\
536	0\\
537	0\\
538	0\\
539	0\\
540	0\\
541	0\\
542	0\\
543	0\\
544	0\\
545	0\\
546	0\\
547	0\\
548	0\\
549	0\\
550	0\\
551	0\\
552	0\\
553	0\\
554	0\\
555	0\\
556	0\\
557	0\\
558	0\\
559	0\\
560	0\\
561	0\\
562	0\\
563	0\\
564	0\\
565	0\\
566	0\\
567	0\\
568	0\\
569	0\\
570	0\\
571	0\\
572	0\\
573	0\\
574	0\\
575	0\\
576	0\\
577	0\\
578	0\\
579	0\\
580	0\\
581	0\\
582	0\\
583	0\\
584	0\\
585	0\\
586	0\\
587	0\\
588	0\\
589	0\\
590	0\\
591	0\\
592	0\\
593	0\\
594	0\\
595	0\\
596	0\\
597	0\\
598	0\\
599	0\\
600	0\\
};
\addplot [color=mycolor6,solid,forget plot]
  table[row sep=crcr]{%
1	0\\
2	0\\
3	0\\
4	0\\
5	0\\
6	0\\
7	0\\
8	0\\
9	0\\
10	0\\
11	0\\
12	0\\
13	0\\
14	0\\
15	0\\
16	0\\
17	0\\
18	0\\
19	0\\
20	0\\
21	0\\
22	0\\
23	0\\
24	0\\
25	0\\
26	0\\
27	0\\
28	0\\
29	0\\
30	0\\
31	0\\
32	0\\
33	0\\
34	0\\
35	0\\
36	0\\
37	0\\
38	0\\
39	0\\
40	0\\
41	0\\
42	0\\
43	0\\
44	0\\
45	0\\
46	0\\
47	0\\
48	0\\
49	0\\
50	0\\
51	0\\
52	0\\
53	0\\
54	0\\
55	0\\
56	0\\
57	0\\
58	0\\
59	0\\
60	0\\
61	0\\
62	0\\
63	0\\
64	0\\
65	0\\
66	0\\
67	0\\
68	0\\
69	0\\
70	0\\
71	0\\
72	0\\
73	0\\
74	0\\
75	0\\
76	0\\
77	0\\
78	0\\
79	0\\
80	0\\
81	0\\
82	0\\
83	0\\
84	0\\
85	0\\
86	0\\
87	0\\
88	0\\
89	0\\
90	0\\
91	0\\
92	0\\
93	0\\
94	0\\
95	0\\
96	0\\
97	0\\
98	0\\
99	0\\
100	0\\
101	0\\
102	0\\
103	0\\
104	0\\
105	0\\
106	0\\
107	0\\
108	0\\
109	0\\
110	0\\
111	0\\
112	0\\
113	0\\
114	0\\
115	0\\
116	0\\
117	0\\
118	0\\
119	0\\
120	0\\
121	0\\
122	0\\
123	0\\
124	0\\
125	0\\
126	0\\
127	0\\
128	0\\
129	0\\
130	0\\
131	0\\
132	0\\
133	0\\
134	0\\
135	0\\
136	0\\
137	0\\
138	0\\
139	0\\
140	0\\
141	0\\
142	0\\
143	0\\
144	0\\
145	0\\
146	0\\
147	0\\
148	0\\
149	0\\
150	0\\
151	0\\
152	0\\
153	0\\
154	0\\
155	0\\
156	0\\
157	0\\
158	0\\
159	0\\
160	0\\
161	0\\
162	0\\
163	0\\
164	0\\
165	0\\
166	0\\
167	0\\
168	0\\
169	0\\
170	0\\
171	0\\
172	0\\
173	0\\
174	0\\
175	0\\
176	0\\
177	0\\
178	0\\
179	0\\
180	0\\
181	0\\
182	0\\
183	0\\
184	0\\
185	0\\
186	0\\
187	0\\
188	0\\
189	0\\
190	0\\
191	0\\
192	0\\
193	0\\
194	0\\
195	0\\
196	0\\
197	0\\
198	0\\
199	0\\
200	0\\
201	0\\
202	0\\
203	0\\
204	0\\
205	0\\
206	0\\
207	0\\
208	0\\
209	0\\
210	0\\
211	0\\
212	0\\
213	0\\
214	0\\
215	0\\
216	0\\
217	0\\
218	0\\
219	0\\
220	0\\
221	0\\
222	0\\
223	0\\
224	0\\
225	0\\
226	0\\
227	0\\
228	0\\
229	0\\
230	0\\
231	0\\
232	0\\
233	0\\
234	0\\
235	0\\
236	0\\
237	0\\
238	0\\
239	0\\
240	0\\
241	0\\
242	0\\
243	0\\
244	0\\
245	0\\
246	0\\
247	0\\
248	0\\
249	0\\
250	0\\
251	0\\
252	0\\
253	0\\
254	0\\
255	0\\
256	0\\
257	0\\
258	0\\
259	0\\
260	0\\
261	0\\
262	0\\
263	0\\
264	0\\
265	0\\
266	0\\
267	0\\
268	0\\
269	0\\
270	0\\
271	0\\
272	0\\
273	0\\
274	0\\
275	0\\
276	0\\
277	0\\
278	0\\
279	0\\
280	0\\
281	0\\
282	0\\
283	0\\
284	0\\
285	0\\
286	0\\
287	0\\
288	0\\
289	0\\
290	0\\
291	0\\
292	0\\
293	0\\
294	0\\
295	0\\
296	0\\
297	0\\
298	0\\
299	0\\
300	0\\
301	0\\
302	0\\
303	0\\
304	0\\
305	0\\
306	0\\
307	0\\
308	0\\
309	0\\
310	0\\
311	0\\
312	0\\
313	0\\
314	0\\
315	0\\
316	0\\
317	0\\
318	0\\
319	0\\
320	0\\
321	0\\
322	0\\
323	0\\
324	0\\
325	0\\
326	0\\
327	0\\
328	0\\
329	0\\
330	0\\
331	0\\
332	0\\
333	0\\
334	0\\
335	0\\
336	0\\
337	0\\
338	0\\
339	0\\
340	0\\
341	0\\
342	0\\
343	0\\
344	0\\
345	0\\
346	0\\
347	0\\
348	0\\
349	0\\
350	0\\
351	0\\
352	0\\
353	0\\
354	0\\
355	0\\
356	0\\
357	0\\
358	0\\
359	0\\
360	0\\
361	0\\
362	0\\
363	0\\
364	0\\
365	0\\
366	0\\
367	0\\
368	0\\
369	0\\
370	0\\
371	0\\
372	0\\
373	0\\
374	0\\
375	0\\
376	0\\
377	0\\
378	0\\
379	0\\
380	0\\
381	0\\
382	0\\
383	0\\
384	0\\
385	0\\
386	0\\
387	0\\
388	0\\
389	0\\
390	0\\
391	0\\
392	0\\
393	0\\
394	0\\
395	0\\
396	0\\
397	0\\
398	0\\
399	0\\
400	0\\
401	0\\
402	0\\
403	0\\
404	0\\
405	0\\
406	0\\
407	0\\
408	0\\
409	0\\
410	0\\
411	0\\
412	0\\
413	0\\
414	0\\
415	0\\
416	0\\
417	0\\
418	0\\
419	0\\
420	0\\
421	0\\
422	0\\
423	0\\
424	0\\
425	0\\
426	0\\
427	0\\
428	0\\
429	0\\
430	0\\
431	0\\
432	0\\
433	0\\
434	0\\
435	0\\
436	0\\
437	0\\
438	0\\
439	0\\
440	0\\
441	0\\
442	0\\
443	0\\
444	0\\
445	0\\
446	0\\
447	0\\
448	0\\
449	0\\
450	0\\
451	0\\
452	0\\
453	0\\
454	0\\
455	0\\
456	0\\
457	0\\
458	0\\
459	0\\
460	0\\
461	0\\
462	0\\
463	0\\
464	0\\
465	0\\
466	0\\
467	0\\
468	0\\
469	0\\
470	0\\
471	0\\
472	0\\
473	0\\
474	0\\
475	0\\
476	0\\
477	0\\
478	0\\
479	0\\
480	0\\
481	0\\
482	0\\
483	0\\
484	0\\
485	0\\
486	0\\
487	0\\
488	0\\
489	0\\
490	0\\
491	0\\
492	0\\
493	0\\
494	0\\
495	0\\
496	0\\
497	0\\
498	0\\
499	0\\
500	0\\
501	0\\
502	0\\
503	0\\
504	0\\
505	0\\
506	0\\
507	0\\
508	0\\
509	0\\
510	0\\
511	0\\
512	0\\
513	0\\
514	0\\
515	0\\
516	0\\
517	0\\
518	0\\
519	0\\
520	0\\
521	0\\
522	0\\
523	0\\
524	0\\
525	0\\
526	0\\
527	0\\
528	0\\
529	0\\
530	0\\
531	0\\
532	0\\
533	0\\
534	0\\
535	0\\
536	0\\
537	0\\
538	0\\
539	0\\
540	0\\
541	0\\
542	0\\
543	0\\
544	0\\
545	0\\
546	0\\
547	0\\
548	0\\
549	0\\
550	0\\
551	0\\
552	0\\
553	0\\
554	0\\
555	0\\
556	0\\
557	0\\
558	0\\
559	0\\
560	0\\
561	0\\
562	0\\
563	0\\
564	0\\
565	0\\
566	0\\
567	0\\
568	0\\
569	0\\
570	0\\
571	0\\
572	0\\
573	0\\
574	0\\
575	0\\
576	0\\
577	0\\
578	0\\
579	0\\
580	0\\
581	0\\
582	0\\
583	0\\
584	0\\
585	0\\
586	0\\
587	0\\
588	0\\
589	0\\
590	0\\
591	0\\
592	0\\
593	0\\
594	0\\
595	0\\
596	0\\
597	0\\
598	0\\
599	0\\
600	0\\
};
\addplot [color=mycolor7,solid,forget plot]
  table[row sep=crcr]{%
1	0\\
2	0\\
3	0\\
4	0\\
5	0\\
6	0\\
7	0\\
8	0\\
9	0\\
10	0\\
11	0\\
12	0\\
13	0\\
14	0\\
15	0\\
16	0\\
17	0\\
18	0\\
19	0\\
20	0\\
21	0\\
22	0\\
23	0\\
24	0\\
25	0\\
26	0\\
27	0\\
28	0\\
29	0\\
30	0\\
31	0\\
32	0\\
33	0\\
34	0\\
35	0\\
36	0\\
37	0\\
38	0\\
39	0\\
40	0\\
41	0\\
42	0\\
43	0\\
44	0\\
45	0\\
46	0\\
47	0\\
48	0\\
49	0\\
50	0\\
51	0\\
52	0\\
53	0\\
54	0\\
55	0\\
56	0\\
57	0\\
58	0\\
59	0\\
60	0\\
61	0\\
62	0\\
63	0\\
64	0\\
65	0\\
66	0\\
67	0\\
68	0\\
69	0\\
70	0\\
71	0\\
72	0\\
73	0\\
74	0\\
75	0\\
76	0\\
77	0\\
78	0\\
79	0\\
80	0\\
81	0\\
82	0\\
83	0\\
84	0\\
85	0\\
86	0\\
87	0\\
88	0\\
89	0\\
90	0\\
91	0\\
92	0\\
93	0\\
94	0\\
95	0\\
96	0\\
97	0\\
98	0\\
99	0\\
100	0\\
101	0\\
102	0\\
103	0\\
104	0\\
105	0\\
106	0\\
107	0\\
108	0\\
109	0\\
110	0\\
111	0\\
112	0\\
113	0\\
114	0\\
115	0\\
116	0\\
117	0\\
118	0\\
119	0\\
120	0\\
121	0\\
122	0\\
123	0\\
124	0\\
125	0\\
126	0\\
127	0\\
128	0\\
129	0\\
130	0\\
131	0\\
132	0\\
133	0\\
134	0\\
135	0\\
136	0\\
137	0\\
138	0\\
139	0\\
140	0\\
141	0\\
142	0\\
143	0\\
144	0\\
145	0\\
146	0\\
147	0\\
148	0\\
149	0\\
150	0\\
151	0\\
152	0\\
153	0\\
154	0\\
155	0\\
156	0\\
157	0\\
158	0\\
159	0\\
160	0\\
161	0\\
162	0\\
163	0\\
164	0\\
165	0\\
166	0\\
167	0\\
168	0\\
169	0\\
170	0\\
171	0\\
172	0\\
173	0\\
174	0\\
175	0\\
176	0\\
177	0\\
178	0\\
179	0\\
180	0\\
181	0\\
182	0\\
183	0\\
184	0\\
185	0\\
186	0\\
187	0\\
188	0\\
189	0\\
190	0\\
191	0\\
192	0\\
193	0\\
194	0\\
195	0\\
196	0\\
197	0\\
198	0\\
199	0\\
200	0\\
201	0\\
202	0\\
203	0\\
204	0\\
205	0\\
206	0\\
207	0\\
208	0\\
209	0\\
210	0\\
211	0\\
212	0\\
213	0\\
214	0\\
215	0\\
216	0\\
217	0\\
218	0\\
219	0\\
220	0\\
221	0\\
222	0\\
223	0\\
224	0\\
225	0\\
226	0\\
227	0\\
228	0\\
229	0\\
230	0\\
231	0\\
232	0\\
233	0\\
234	0\\
235	0\\
236	0\\
237	0\\
238	0\\
239	0\\
240	0\\
241	0\\
242	0\\
243	0\\
244	0\\
245	0\\
246	0\\
247	0\\
248	0\\
249	0\\
250	0\\
251	0\\
252	0\\
253	0\\
254	0\\
255	0\\
256	0\\
257	0\\
258	0\\
259	0\\
260	0\\
261	0\\
262	0\\
263	0\\
264	0\\
265	0\\
266	0\\
267	0\\
268	0\\
269	0\\
270	0\\
271	0\\
272	0\\
273	0\\
274	0\\
275	0\\
276	0\\
277	0\\
278	0\\
279	0\\
280	0\\
281	0\\
282	0\\
283	0\\
284	0\\
285	0\\
286	0\\
287	0\\
288	0\\
289	0\\
290	0\\
291	0\\
292	0\\
293	0\\
294	0\\
295	0\\
296	0\\
297	0\\
298	0\\
299	0\\
300	0\\
301	0\\
302	0\\
303	0\\
304	0\\
305	0\\
306	0\\
307	0\\
308	0\\
309	0\\
310	0\\
311	0\\
312	0\\
313	0\\
314	0\\
315	0\\
316	0\\
317	0\\
318	0\\
319	0\\
320	0\\
321	0\\
322	0\\
323	0\\
324	0\\
325	0\\
326	0\\
327	0\\
328	0\\
329	0\\
330	0\\
331	0\\
332	0\\
333	0\\
334	0\\
335	0\\
336	0\\
337	0\\
338	0\\
339	0\\
340	0\\
341	0\\
342	0\\
343	0\\
344	0\\
345	0\\
346	0\\
347	0\\
348	0\\
349	0\\
350	0\\
351	0\\
352	0\\
353	0\\
354	0\\
355	0\\
356	0\\
357	0\\
358	0\\
359	0\\
360	0\\
361	0\\
362	0\\
363	0\\
364	0\\
365	0\\
366	0\\
367	0\\
368	0\\
369	0\\
370	0\\
371	0\\
372	0\\
373	0\\
374	0\\
375	0\\
376	0\\
377	0\\
378	0\\
379	0\\
380	0\\
381	0\\
382	0\\
383	0\\
384	0\\
385	0\\
386	0\\
387	0\\
388	0\\
389	0\\
390	0\\
391	0\\
392	0\\
393	0\\
394	0\\
395	0\\
396	0\\
397	0\\
398	0\\
399	0\\
400	0\\
401	0\\
402	0\\
403	0\\
404	0\\
405	0\\
406	0\\
407	0\\
408	0\\
409	0\\
410	0\\
411	0\\
412	0\\
413	0\\
414	0\\
415	0\\
416	0\\
417	0\\
418	0\\
419	0\\
420	0\\
421	0\\
422	0\\
423	0\\
424	0\\
425	0\\
426	0\\
427	0\\
428	0\\
429	0\\
430	0\\
431	0\\
432	0\\
433	0\\
434	0\\
435	0\\
436	0\\
437	0\\
438	0\\
439	0\\
440	0\\
441	0\\
442	0\\
443	0\\
444	0\\
445	0\\
446	0\\
447	0\\
448	0\\
449	0\\
450	0\\
451	0\\
452	0\\
453	0\\
454	0\\
455	0\\
456	0\\
457	0\\
458	0\\
459	0\\
460	0\\
461	0\\
462	0\\
463	0\\
464	0\\
465	0\\
466	0\\
467	0\\
468	0\\
469	0\\
470	0\\
471	0\\
472	0\\
473	0\\
474	0\\
475	0\\
476	0\\
477	0\\
478	0\\
479	0\\
480	0\\
481	0\\
482	0\\
483	0\\
484	0\\
485	0\\
486	0\\
487	0\\
488	0\\
489	0\\
490	0\\
491	0\\
492	0\\
493	0\\
494	0\\
495	0\\
496	0\\
497	0\\
498	0\\
499	0\\
500	0\\
501	0\\
502	0\\
503	0\\
504	0\\
505	0\\
506	0\\
507	0\\
508	0\\
509	0\\
510	0\\
511	0\\
512	0\\
513	0\\
514	0\\
515	0\\
516	0\\
517	0\\
518	0\\
519	0\\
520	0\\
521	0\\
522	0\\
523	0\\
524	0\\
525	0\\
526	0\\
527	0\\
528	0\\
529	0\\
530	0\\
531	0\\
532	0\\
533	0\\
534	0\\
535	0\\
536	0\\
537	0\\
538	0\\
539	0\\
540	0\\
541	0\\
542	0\\
543	0\\
544	0\\
545	0\\
546	0\\
547	0\\
548	0\\
549	0\\
550	0\\
551	0\\
552	0\\
553	0\\
554	0\\
555	0\\
556	0\\
557	0\\
558	0\\
559	0\\
560	0\\
561	0\\
562	0\\
563	0\\
564	0\\
565	0\\
566	0\\
567	0\\
568	0\\
569	0\\
570	0\\
571	0\\
572	0\\
573	0\\
574	0\\
575	0\\
576	0\\
577	0\\
578	0\\
579	0\\
580	0\\
581	0\\
582	0\\
583	0\\
584	0\\
585	0\\
586	0\\
587	0\\
588	0\\
589	0\\
590	0\\
591	0\\
592	0\\
593	0\\
594	0\\
595	0\\
596	0\\
597	0\\
598	0\\
599	0\\
600	0\\
};
\addplot [color=mycolor8,solid,forget plot]
  table[row sep=crcr]{%
1	0\\
2	0\\
3	0\\
4	0\\
5	0\\
6	0\\
7	0\\
8	0\\
9	0\\
10	0\\
11	0\\
12	0\\
13	0\\
14	0\\
15	0\\
16	0\\
17	0\\
18	0\\
19	0\\
20	0\\
21	0\\
22	0\\
23	0\\
24	0\\
25	0\\
26	0\\
27	0\\
28	0\\
29	0\\
30	0\\
31	0\\
32	0\\
33	0\\
34	0\\
35	0\\
36	0\\
37	0\\
38	0\\
39	0\\
40	0\\
41	0\\
42	0\\
43	0\\
44	0\\
45	0\\
46	0\\
47	0\\
48	0\\
49	0\\
50	0\\
51	0\\
52	0\\
53	0\\
54	0\\
55	0\\
56	0\\
57	0\\
58	0\\
59	0\\
60	0\\
61	0\\
62	0\\
63	0\\
64	0\\
65	0\\
66	0\\
67	0\\
68	0\\
69	0\\
70	0\\
71	0\\
72	0\\
73	0\\
74	0\\
75	0\\
76	0\\
77	0\\
78	0\\
79	0\\
80	0\\
81	0\\
82	0\\
83	0\\
84	0\\
85	0\\
86	0\\
87	0\\
88	0\\
89	0\\
90	0\\
91	0\\
92	0\\
93	0\\
94	0\\
95	0\\
96	0\\
97	0\\
98	0\\
99	0\\
100	0\\
101	0\\
102	0\\
103	0\\
104	0\\
105	0\\
106	0\\
107	0\\
108	0\\
109	0\\
110	0\\
111	0\\
112	0\\
113	0\\
114	0\\
115	0\\
116	0\\
117	0\\
118	0\\
119	0\\
120	0\\
121	0\\
122	0\\
123	0\\
124	0\\
125	0\\
126	0\\
127	0\\
128	0\\
129	0\\
130	0\\
131	0\\
132	0\\
133	0\\
134	0\\
135	0\\
136	0\\
137	0\\
138	0\\
139	0\\
140	0\\
141	0\\
142	0\\
143	0\\
144	0\\
145	0\\
146	0\\
147	0\\
148	0\\
149	0\\
150	0\\
151	0\\
152	0\\
153	0\\
154	0\\
155	0\\
156	0\\
157	0\\
158	0\\
159	0\\
160	0\\
161	0\\
162	0\\
163	0\\
164	0\\
165	0\\
166	0\\
167	0\\
168	0\\
169	0\\
170	0\\
171	0\\
172	0\\
173	0\\
174	0\\
175	0\\
176	0\\
177	0\\
178	0\\
179	0\\
180	0\\
181	0\\
182	0\\
183	0\\
184	0\\
185	0\\
186	0\\
187	0\\
188	0\\
189	0\\
190	0\\
191	0\\
192	0\\
193	0\\
194	0\\
195	0\\
196	0\\
197	0\\
198	0\\
199	0\\
200	0\\
201	0\\
202	0\\
203	0\\
204	0\\
205	0\\
206	0\\
207	0\\
208	0\\
209	0\\
210	0\\
211	0\\
212	0\\
213	0\\
214	0\\
215	0\\
216	0\\
217	0\\
218	0\\
219	0\\
220	0\\
221	0\\
222	0\\
223	0\\
224	0\\
225	0\\
226	0\\
227	0\\
228	0\\
229	0\\
230	0\\
231	0\\
232	0\\
233	0\\
234	0\\
235	0\\
236	0\\
237	0\\
238	0\\
239	0\\
240	0\\
241	0\\
242	0\\
243	0\\
244	0\\
245	0\\
246	0\\
247	0\\
248	0\\
249	0\\
250	0\\
251	0\\
252	0\\
253	0\\
254	0\\
255	0\\
256	0\\
257	0\\
258	0\\
259	0\\
260	0\\
261	0\\
262	0\\
263	0\\
264	0\\
265	0\\
266	0\\
267	0\\
268	0\\
269	0\\
270	0\\
271	0\\
272	0\\
273	0\\
274	0\\
275	0\\
276	0\\
277	0\\
278	0\\
279	0\\
280	0\\
281	0\\
282	0\\
283	0\\
284	0\\
285	0\\
286	0\\
287	0\\
288	0\\
289	0\\
290	0\\
291	0\\
292	0\\
293	0\\
294	0\\
295	0\\
296	0\\
297	0\\
298	0\\
299	0\\
300	0\\
301	0\\
302	0\\
303	0\\
304	0\\
305	0\\
306	0\\
307	0\\
308	0\\
309	0\\
310	0\\
311	0\\
312	0\\
313	0\\
314	0\\
315	0\\
316	0\\
317	0\\
318	0\\
319	0\\
320	0\\
321	0\\
322	0\\
323	0\\
324	0\\
325	0\\
326	0\\
327	0\\
328	0\\
329	0\\
330	0\\
331	0\\
332	0\\
333	0\\
334	0\\
335	0\\
336	0\\
337	0\\
338	0\\
339	0\\
340	0\\
341	0\\
342	0\\
343	0\\
344	0\\
345	0\\
346	0\\
347	0\\
348	0\\
349	0\\
350	0\\
351	0\\
352	0\\
353	0\\
354	0\\
355	0\\
356	0\\
357	0\\
358	0\\
359	0\\
360	0\\
361	0\\
362	0\\
363	0\\
364	0\\
365	0\\
366	0\\
367	0\\
368	0\\
369	0\\
370	0\\
371	0\\
372	0\\
373	0\\
374	0\\
375	0\\
376	0\\
377	0\\
378	0\\
379	0\\
380	0\\
381	0\\
382	0\\
383	0\\
384	0\\
385	0\\
386	0\\
387	0\\
388	0\\
389	0\\
390	0\\
391	0\\
392	0\\
393	0\\
394	0\\
395	0\\
396	0\\
397	0\\
398	0\\
399	0\\
400	0\\
401	0\\
402	0\\
403	0\\
404	0\\
405	0\\
406	0\\
407	0\\
408	0\\
409	0\\
410	0\\
411	0\\
412	0\\
413	0\\
414	0\\
415	0\\
416	0\\
417	0\\
418	0\\
419	0\\
420	0\\
421	0\\
422	0\\
423	0\\
424	0\\
425	0\\
426	0\\
427	0\\
428	0\\
429	0\\
430	0\\
431	0\\
432	0\\
433	0\\
434	0\\
435	0\\
436	0\\
437	0\\
438	0\\
439	0\\
440	0\\
441	0\\
442	0\\
443	0\\
444	0\\
445	0\\
446	0\\
447	0\\
448	0\\
449	0\\
450	0\\
451	0\\
452	0\\
453	0\\
454	0\\
455	0\\
456	0\\
457	0\\
458	0\\
459	0\\
460	0\\
461	0\\
462	0\\
463	0\\
464	0\\
465	0\\
466	0\\
467	0\\
468	0\\
469	0\\
470	0\\
471	0\\
472	0\\
473	0\\
474	0\\
475	0\\
476	0\\
477	0\\
478	0\\
479	0\\
480	0\\
481	0\\
482	0\\
483	0\\
484	0\\
485	0\\
486	0\\
487	0\\
488	0\\
489	0\\
490	0\\
491	0\\
492	0\\
493	0\\
494	0\\
495	0\\
496	0\\
497	0\\
498	0\\
499	0\\
500	0\\
501	0\\
502	0\\
503	0\\
504	0\\
505	0\\
506	0\\
507	0\\
508	0\\
509	0\\
510	0\\
511	0\\
512	0\\
513	0\\
514	0\\
515	0\\
516	0\\
517	0\\
518	0\\
519	0\\
520	0\\
521	0\\
522	0\\
523	0\\
524	0\\
525	0\\
526	0\\
527	0\\
528	0\\
529	0\\
530	0\\
531	0\\
532	0\\
533	0\\
534	0\\
535	0\\
536	0\\
537	0\\
538	0\\
539	0\\
540	0\\
541	0\\
542	0\\
543	0\\
544	0\\
545	0\\
546	0\\
547	0\\
548	0\\
549	0\\
550	0\\
551	0\\
552	0\\
553	0\\
554	0\\
555	0\\
556	0\\
557	0\\
558	0\\
559	0\\
560	0\\
561	0\\
562	0\\
563	0\\
564	0\\
565	0\\
566	0\\
567	0\\
568	0\\
569	0\\
570	0\\
571	0\\
572	0\\
573	0\\
574	0\\
575	0\\
576	0\\
577	0\\
578	0\\
579	0\\
580	0\\
581	0\\
582	0\\
583	0\\
584	0\\
585	0\\
586	0\\
587	0\\
588	0\\
589	0\\
590	0\\
591	0\\
592	0\\
593	0\\
594	0\\
595	0\\
596	0\\
597	0\\
598	0\\
599	0\\
600	0\\
};
\addplot [color=blue!25!mycolor7,solid,forget plot]
  table[row sep=crcr]{%
1	0\\
2	0\\
3	0\\
4	0\\
5	0\\
6	0\\
7	0\\
8	0\\
9	0\\
10	0\\
11	0\\
12	0\\
13	0\\
14	0\\
15	0\\
16	0\\
17	0\\
18	0\\
19	0\\
20	0\\
21	0\\
22	0\\
23	0\\
24	0\\
25	0\\
26	0\\
27	0\\
28	0\\
29	0\\
30	0\\
31	0\\
32	0\\
33	0\\
34	0\\
35	0\\
36	0\\
37	0\\
38	0\\
39	0\\
40	0\\
41	0\\
42	0\\
43	0\\
44	0\\
45	0\\
46	0\\
47	0\\
48	0\\
49	0\\
50	0\\
51	0\\
52	0\\
53	0\\
54	0\\
55	0\\
56	0\\
57	0\\
58	0\\
59	0\\
60	0\\
61	0\\
62	0\\
63	0\\
64	0\\
65	0\\
66	0\\
67	0\\
68	0\\
69	0\\
70	0\\
71	0\\
72	0\\
73	0\\
74	0\\
75	0\\
76	0\\
77	0\\
78	0\\
79	0\\
80	0\\
81	0\\
82	0\\
83	0\\
84	0\\
85	0\\
86	0\\
87	0\\
88	0\\
89	0\\
90	0\\
91	0\\
92	0\\
93	0\\
94	0\\
95	0\\
96	0\\
97	0\\
98	0\\
99	0\\
100	0\\
101	0\\
102	0\\
103	0\\
104	0\\
105	0\\
106	0\\
107	0\\
108	0\\
109	0\\
110	0\\
111	0\\
112	0\\
113	0\\
114	0\\
115	0\\
116	0\\
117	0\\
118	0\\
119	0\\
120	0\\
121	0\\
122	0\\
123	0\\
124	0\\
125	0\\
126	0\\
127	0\\
128	0\\
129	0\\
130	0\\
131	0\\
132	0\\
133	0\\
134	0\\
135	0\\
136	0\\
137	0\\
138	0\\
139	0\\
140	0\\
141	0\\
142	0\\
143	0\\
144	0\\
145	0\\
146	0\\
147	0\\
148	0\\
149	0\\
150	0\\
151	0\\
152	0\\
153	0\\
154	0\\
155	0\\
156	0\\
157	0\\
158	0\\
159	0\\
160	0\\
161	0\\
162	0\\
163	0\\
164	0\\
165	0\\
166	0\\
167	0\\
168	0\\
169	0\\
170	0\\
171	0\\
172	0\\
173	0\\
174	0\\
175	0\\
176	0\\
177	0\\
178	0\\
179	0\\
180	0\\
181	0\\
182	0\\
183	0\\
184	0\\
185	0\\
186	0\\
187	0\\
188	0\\
189	0\\
190	0\\
191	0\\
192	0\\
193	0\\
194	0\\
195	0\\
196	0\\
197	0\\
198	0\\
199	0\\
200	0\\
201	0\\
202	0\\
203	0\\
204	0\\
205	0\\
206	0\\
207	0\\
208	0\\
209	0\\
210	0\\
211	0\\
212	0\\
213	0\\
214	0\\
215	0\\
216	0\\
217	0\\
218	0\\
219	0\\
220	0\\
221	0\\
222	0\\
223	0\\
224	0\\
225	0\\
226	0\\
227	0\\
228	0\\
229	0\\
230	0\\
231	0\\
232	0\\
233	0\\
234	0\\
235	0\\
236	0\\
237	0\\
238	0\\
239	0\\
240	0\\
241	0\\
242	0\\
243	0\\
244	0\\
245	0\\
246	0\\
247	0\\
248	0\\
249	0\\
250	0\\
251	0\\
252	0\\
253	0\\
254	0\\
255	0\\
256	0\\
257	0\\
258	0\\
259	0\\
260	0\\
261	0\\
262	0\\
263	0\\
264	0\\
265	0\\
266	0\\
267	0\\
268	0\\
269	0\\
270	0\\
271	0\\
272	0\\
273	0\\
274	0\\
275	0\\
276	0\\
277	0\\
278	0\\
279	0\\
280	0\\
281	0\\
282	0\\
283	0\\
284	0\\
285	0\\
286	0\\
287	0\\
288	0\\
289	0\\
290	0\\
291	0\\
292	0\\
293	0\\
294	0\\
295	0\\
296	0\\
297	0\\
298	0\\
299	0\\
300	0\\
301	0\\
302	0\\
303	0\\
304	0\\
305	0\\
306	0\\
307	0\\
308	0\\
309	0\\
310	0\\
311	0\\
312	0\\
313	0\\
314	0\\
315	0\\
316	0\\
317	0\\
318	0\\
319	0\\
320	0\\
321	0\\
322	0\\
323	0\\
324	0\\
325	0\\
326	0\\
327	0\\
328	0\\
329	0\\
330	0\\
331	0\\
332	0\\
333	0\\
334	0\\
335	0\\
336	0\\
337	0\\
338	0\\
339	0\\
340	0\\
341	0\\
342	0\\
343	0\\
344	0\\
345	0\\
346	0\\
347	0\\
348	0\\
349	0\\
350	0\\
351	0\\
352	0\\
353	0\\
354	0\\
355	0\\
356	0\\
357	0\\
358	0\\
359	0\\
360	0\\
361	0\\
362	0\\
363	0\\
364	0\\
365	0\\
366	0\\
367	0\\
368	0\\
369	0\\
370	0\\
371	0\\
372	0\\
373	0\\
374	0\\
375	0\\
376	0\\
377	0\\
378	0\\
379	0\\
380	0\\
381	0\\
382	0\\
383	0\\
384	0\\
385	0\\
386	0\\
387	0\\
388	0\\
389	0\\
390	0\\
391	0\\
392	0\\
393	0\\
394	0\\
395	0\\
396	0\\
397	0\\
398	0\\
399	0\\
400	0\\
401	0\\
402	0\\
403	0\\
404	0\\
405	0\\
406	0\\
407	0\\
408	0\\
409	0\\
410	0\\
411	0\\
412	0\\
413	0\\
414	0\\
415	0\\
416	0\\
417	0\\
418	0\\
419	0\\
420	0\\
421	0\\
422	0\\
423	0\\
424	0\\
425	0\\
426	0\\
427	0\\
428	0\\
429	0\\
430	0\\
431	0\\
432	0\\
433	0\\
434	0\\
435	0\\
436	0\\
437	0\\
438	0\\
439	0\\
440	0\\
441	0\\
442	0\\
443	0\\
444	0\\
445	0\\
446	0\\
447	0\\
448	0\\
449	0\\
450	0\\
451	0\\
452	0\\
453	0\\
454	0\\
455	0\\
456	0\\
457	0\\
458	0\\
459	0\\
460	0\\
461	0\\
462	0\\
463	0\\
464	0\\
465	0\\
466	0\\
467	0\\
468	0\\
469	0\\
470	0\\
471	0\\
472	0\\
473	0\\
474	0\\
475	0\\
476	0\\
477	0\\
478	0\\
479	0\\
480	0\\
481	0\\
482	0\\
483	0\\
484	0\\
485	0\\
486	0\\
487	0\\
488	0\\
489	0\\
490	0\\
491	0\\
492	0\\
493	0\\
494	0\\
495	0\\
496	0\\
497	0\\
498	0\\
499	0\\
500	0\\
501	0\\
502	0\\
503	0\\
504	0\\
505	0\\
506	0\\
507	0\\
508	0\\
509	0\\
510	0\\
511	0\\
512	0\\
513	0\\
514	0\\
515	0\\
516	0\\
517	0\\
518	0\\
519	0\\
520	0\\
521	0\\
522	0\\
523	0\\
524	0\\
525	0\\
526	0\\
527	0\\
528	0\\
529	0\\
530	0\\
531	0\\
532	0\\
533	0\\
534	0\\
535	0\\
536	0\\
537	0\\
538	0\\
539	0\\
540	0\\
541	0\\
542	0\\
543	0\\
544	0\\
545	0\\
546	0\\
547	0\\
548	0\\
549	0\\
550	0\\
551	0\\
552	0\\
553	0\\
554	0\\
555	0\\
556	0\\
557	0\\
558	0\\
559	0\\
560	0\\
561	0\\
562	0\\
563	0\\
564	0\\
565	0\\
566	0\\
567	0\\
568	0\\
569	0\\
570	0\\
571	0\\
572	0\\
573	0\\
574	0\\
575	0\\
576	0\\
577	0\\
578	0\\
579	0\\
580	0\\
581	0\\
582	0\\
583	0\\
584	0\\
585	0\\
586	0\\
587	0\\
588	0\\
589	0\\
590	0\\
591	0\\
592	0\\
593	0\\
594	0\\
595	0\\
596	0\\
597	0\\
598	0\\
599	0\\
600	0\\
};
\addplot [color=mycolor9,solid,forget plot]
  table[row sep=crcr]{%
1	0\\
2	0\\
3	0\\
4	0\\
5	0\\
6	0\\
7	0\\
8	0\\
9	0\\
10	0\\
11	0\\
12	0\\
13	0\\
14	0\\
15	0\\
16	0\\
17	0\\
18	0\\
19	0\\
20	0\\
21	0\\
22	0\\
23	0\\
24	0\\
25	0\\
26	0\\
27	0\\
28	0\\
29	0\\
30	0\\
31	0\\
32	0\\
33	0\\
34	0\\
35	0\\
36	0\\
37	0\\
38	0\\
39	0\\
40	0\\
41	0\\
42	0\\
43	0\\
44	0\\
45	0\\
46	0\\
47	0\\
48	0\\
49	0\\
50	0\\
51	0\\
52	0\\
53	0\\
54	0\\
55	0\\
56	0\\
57	0\\
58	0\\
59	0\\
60	0\\
61	0\\
62	0\\
63	0\\
64	0\\
65	0\\
66	0\\
67	0\\
68	0\\
69	0\\
70	0\\
71	0\\
72	0\\
73	0\\
74	0\\
75	0\\
76	0\\
77	0\\
78	0\\
79	0\\
80	0\\
81	0\\
82	0\\
83	0\\
84	0\\
85	0\\
86	0\\
87	0\\
88	0\\
89	0\\
90	0\\
91	0\\
92	0\\
93	0\\
94	0\\
95	0\\
96	0\\
97	0\\
98	0\\
99	0\\
100	0\\
101	0\\
102	0\\
103	0\\
104	0\\
105	0\\
106	0\\
107	0\\
108	0\\
109	0\\
110	0\\
111	0\\
112	0\\
113	0\\
114	0\\
115	0\\
116	0\\
117	0\\
118	0\\
119	0\\
120	0\\
121	0\\
122	0\\
123	0\\
124	0\\
125	0\\
126	0\\
127	0\\
128	0\\
129	0\\
130	0\\
131	0\\
132	0\\
133	0\\
134	0\\
135	0\\
136	0\\
137	0\\
138	0\\
139	0\\
140	0\\
141	0\\
142	0\\
143	0\\
144	0\\
145	0\\
146	0\\
147	0\\
148	0\\
149	0\\
150	0\\
151	0\\
152	0\\
153	0\\
154	0\\
155	0\\
156	0\\
157	0\\
158	0\\
159	0\\
160	0\\
161	0\\
162	0\\
163	0\\
164	0\\
165	0\\
166	0\\
167	0\\
168	0\\
169	0\\
170	0\\
171	0\\
172	0\\
173	0\\
174	0\\
175	0\\
176	0\\
177	0\\
178	0\\
179	0\\
180	0\\
181	0\\
182	0\\
183	0\\
184	0\\
185	0\\
186	0\\
187	0\\
188	0\\
189	0\\
190	0\\
191	0\\
192	0\\
193	0\\
194	0\\
195	0\\
196	0\\
197	0\\
198	0\\
199	0\\
200	0\\
201	0\\
202	0\\
203	0\\
204	0\\
205	0\\
206	0\\
207	0\\
208	0\\
209	0\\
210	0\\
211	0\\
212	0\\
213	0\\
214	0\\
215	0\\
216	0\\
217	0\\
218	0\\
219	0\\
220	0\\
221	0\\
222	0\\
223	0\\
224	0\\
225	0\\
226	0\\
227	0\\
228	0\\
229	0\\
230	0\\
231	0\\
232	0\\
233	0\\
234	0\\
235	0\\
236	0\\
237	0\\
238	0\\
239	0\\
240	0\\
241	0\\
242	0\\
243	0\\
244	0\\
245	0\\
246	0\\
247	0\\
248	0\\
249	0\\
250	0\\
251	0\\
252	0\\
253	0\\
254	0\\
255	0\\
256	0\\
257	0\\
258	0\\
259	0\\
260	0\\
261	0\\
262	0\\
263	0\\
264	0\\
265	0\\
266	0\\
267	0\\
268	0\\
269	0\\
270	0\\
271	0\\
272	0\\
273	0\\
274	0\\
275	0\\
276	0\\
277	0\\
278	0\\
279	0\\
280	0\\
281	0\\
282	0\\
283	0\\
284	0\\
285	0\\
286	0\\
287	0\\
288	0\\
289	0\\
290	0\\
291	0\\
292	0\\
293	0\\
294	0\\
295	0\\
296	0\\
297	0\\
298	0\\
299	0\\
300	0\\
301	0\\
302	0\\
303	0\\
304	0\\
305	0\\
306	0\\
307	0\\
308	0\\
309	0\\
310	0\\
311	0\\
312	0\\
313	0\\
314	0\\
315	0\\
316	0\\
317	0\\
318	0\\
319	0\\
320	0\\
321	0\\
322	0\\
323	0\\
324	0\\
325	0\\
326	0\\
327	0\\
328	0\\
329	0\\
330	0\\
331	0\\
332	0\\
333	0\\
334	0\\
335	0\\
336	0\\
337	0\\
338	0\\
339	0\\
340	0\\
341	0\\
342	0\\
343	0\\
344	0\\
345	0\\
346	0\\
347	0\\
348	0\\
349	0\\
350	0\\
351	0\\
352	0\\
353	0\\
354	0\\
355	0\\
356	0\\
357	0\\
358	0\\
359	0\\
360	0\\
361	0\\
362	0\\
363	0\\
364	0\\
365	0\\
366	0\\
367	0\\
368	0\\
369	0\\
370	0\\
371	0\\
372	0\\
373	0\\
374	0\\
375	0\\
376	0\\
377	0\\
378	0\\
379	0\\
380	0\\
381	0\\
382	0\\
383	0\\
384	0\\
385	0\\
386	0\\
387	0\\
388	0\\
389	0\\
390	0\\
391	0\\
392	0\\
393	0\\
394	0\\
395	0\\
396	0\\
397	0\\
398	0\\
399	0\\
400	0\\
401	0\\
402	0\\
403	0\\
404	0\\
405	0\\
406	0\\
407	0\\
408	0\\
409	0\\
410	0\\
411	0\\
412	0\\
413	0\\
414	0\\
415	0\\
416	0\\
417	0\\
418	0\\
419	0\\
420	0\\
421	0\\
422	0\\
423	0\\
424	0\\
425	0\\
426	0\\
427	0\\
428	0\\
429	0\\
430	0\\
431	0\\
432	0\\
433	0\\
434	0\\
435	0\\
436	0\\
437	0\\
438	0\\
439	0\\
440	0\\
441	0\\
442	0\\
443	0\\
444	0\\
445	0\\
446	0\\
447	0\\
448	0\\
449	0\\
450	0\\
451	0\\
452	0\\
453	0\\
454	0\\
455	0\\
456	0\\
457	0\\
458	0\\
459	0\\
460	0\\
461	0\\
462	0\\
463	0\\
464	0\\
465	0\\
466	0\\
467	0\\
468	0\\
469	0\\
470	0\\
471	0\\
472	0\\
473	0\\
474	0\\
475	0\\
476	0\\
477	0\\
478	0\\
479	0\\
480	0\\
481	0\\
482	0\\
483	0\\
484	0\\
485	0\\
486	0\\
487	0\\
488	0\\
489	0\\
490	0\\
491	0\\
492	0\\
493	0\\
494	0\\
495	0\\
496	0\\
497	0\\
498	0\\
499	0\\
500	0\\
501	0\\
502	0\\
503	0\\
504	0\\
505	0\\
506	0\\
507	0\\
508	0\\
509	0\\
510	0\\
511	0\\
512	0\\
513	0\\
514	0\\
515	0\\
516	0\\
517	0\\
518	0\\
519	0\\
520	0\\
521	0\\
522	0\\
523	0\\
524	0\\
525	0\\
526	0\\
527	0\\
528	0\\
529	0\\
530	0\\
531	0\\
532	0\\
533	0\\
534	0\\
535	0\\
536	0\\
537	0\\
538	0\\
539	0\\
540	0\\
541	0\\
542	0\\
543	0\\
544	0\\
545	0\\
546	0\\
547	0\\
548	0\\
549	0\\
550	0\\
551	0\\
552	0\\
553	0\\
554	0\\
555	0\\
556	0\\
557	0\\
558	0\\
559	0\\
560	0\\
561	0\\
562	0\\
563	0\\
564	0\\
565	0\\
566	0\\
567	0\\
568	0\\
569	0\\
570	0\\
571	0\\
572	0\\
573	0\\
574	0\\
575	0\\
576	0\\
577	0\\
578	0\\
579	0\\
580	0\\
581	0\\
582	0\\
583	0\\
584	0\\
585	0\\
586	0\\
587	0\\
588	0\\
589	0\\
590	0\\
591	0\\
592	0\\
593	0\\
594	0\\
595	0\\
596	0\\
597	0\\
598	0\\
599	0\\
600	0\\
};
\addplot [color=blue!50!mycolor7,solid,forget plot]
  table[row sep=crcr]{%
1	0\\
2	0\\
3	0\\
4	0\\
5	0\\
6	0\\
7	0\\
8	0\\
9	0\\
10	0\\
11	0\\
12	0\\
13	0\\
14	0\\
15	0\\
16	0\\
17	0\\
18	0\\
19	0\\
20	0\\
21	0\\
22	0\\
23	0\\
24	0\\
25	0\\
26	0\\
27	0\\
28	0\\
29	0\\
30	0\\
31	0\\
32	0\\
33	0\\
34	0\\
35	0\\
36	0\\
37	0\\
38	0\\
39	0\\
40	0\\
41	0\\
42	0\\
43	0\\
44	0\\
45	0\\
46	0\\
47	0\\
48	0\\
49	0\\
50	0\\
51	0\\
52	0\\
53	0\\
54	0\\
55	0\\
56	0\\
57	0\\
58	0\\
59	0\\
60	0\\
61	0\\
62	0\\
63	0\\
64	0\\
65	0\\
66	0\\
67	0\\
68	0\\
69	0\\
70	0\\
71	0\\
72	0\\
73	0\\
74	0\\
75	0\\
76	0\\
77	0\\
78	0\\
79	0\\
80	0\\
81	0\\
82	0\\
83	0\\
84	0\\
85	0\\
86	0\\
87	0\\
88	0\\
89	0\\
90	0\\
91	0\\
92	0\\
93	0\\
94	0\\
95	0\\
96	0\\
97	0\\
98	0\\
99	0\\
100	0\\
101	0\\
102	0\\
103	0\\
104	0\\
105	0\\
106	0\\
107	0\\
108	0\\
109	0\\
110	0\\
111	0\\
112	0\\
113	0\\
114	0\\
115	0\\
116	0\\
117	0\\
118	0\\
119	0\\
120	0\\
121	0\\
122	0\\
123	0\\
124	0\\
125	0\\
126	0\\
127	0\\
128	0\\
129	0\\
130	0\\
131	0\\
132	0\\
133	0\\
134	0\\
135	0\\
136	0\\
137	0\\
138	0\\
139	0\\
140	0\\
141	0\\
142	0\\
143	0\\
144	0\\
145	0\\
146	0\\
147	0\\
148	0\\
149	0\\
150	0\\
151	0\\
152	0\\
153	0\\
154	0\\
155	0\\
156	0\\
157	0\\
158	0\\
159	0\\
160	0\\
161	0\\
162	0\\
163	0\\
164	0\\
165	0\\
166	0\\
167	0\\
168	0\\
169	0\\
170	0\\
171	0\\
172	0\\
173	0\\
174	0\\
175	0\\
176	0\\
177	0\\
178	0\\
179	0\\
180	0\\
181	0\\
182	0\\
183	0\\
184	0\\
185	0\\
186	0\\
187	0\\
188	0\\
189	0\\
190	0\\
191	0\\
192	0\\
193	0\\
194	0\\
195	0\\
196	0\\
197	0\\
198	0\\
199	0\\
200	0\\
201	0\\
202	0\\
203	0\\
204	0\\
205	0\\
206	0\\
207	0\\
208	0\\
209	0\\
210	0\\
211	0\\
212	0\\
213	0\\
214	0\\
215	0\\
216	0\\
217	0\\
218	0\\
219	0\\
220	0\\
221	0\\
222	0\\
223	0\\
224	0\\
225	0\\
226	0\\
227	0\\
228	0\\
229	0\\
230	0\\
231	0\\
232	0\\
233	0\\
234	0\\
235	0\\
236	0\\
237	0\\
238	0\\
239	0\\
240	0\\
241	0\\
242	0\\
243	0\\
244	0\\
245	0\\
246	0\\
247	0\\
248	0\\
249	0\\
250	0\\
251	0\\
252	0\\
253	0\\
254	0\\
255	0\\
256	0\\
257	0\\
258	0\\
259	0\\
260	0\\
261	0\\
262	0\\
263	0\\
264	0\\
265	0\\
266	0\\
267	0\\
268	0\\
269	0\\
270	0\\
271	0\\
272	0\\
273	0\\
274	0\\
275	0\\
276	0\\
277	0\\
278	0\\
279	0\\
280	0\\
281	0\\
282	0\\
283	0\\
284	0\\
285	0\\
286	0\\
287	0\\
288	0\\
289	0\\
290	0\\
291	0\\
292	0\\
293	0\\
294	0\\
295	0\\
296	0\\
297	0\\
298	0\\
299	0\\
300	0\\
301	0\\
302	0\\
303	0\\
304	0\\
305	0\\
306	0\\
307	0\\
308	0\\
309	0\\
310	0\\
311	0\\
312	0\\
313	0\\
314	0\\
315	0\\
316	0\\
317	0\\
318	0\\
319	0\\
320	0\\
321	0\\
322	0\\
323	0\\
324	0\\
325	0\\
326	0\\
327	0\\
328	0\\
329	0\\
330	0\\
331	0\\
332	0\\
333	0\\
334	0\\
335	0\\
336	0\\
337	0\\
338	0\\
339	0\\
340	0\\
341	0\\
342	0\\
343	0\\
344	0\\
345	0\\
346	0\\
347	0\\
348	0\\
349	0\\
350	0\\
351	0\\
352	0\\
353	0\\
354	0\\
355	0\\
356	0\\
357	0\\
358	0\\
359	0\\
360	0\\
361	0\\
362	0\\
363	0\\
364	0\\
365	0\\
366	0\\
367	0\\
368	0\\
369	0\\
370	0\\
371	0\\
372	0\\
373	0\\
374	0\\
375	0\\
376	0\\
377	0\\
378	0\\
379	0\\
380	0\\
381	0\\
382	0\\
383	0\\
384	0\\
385	0\\
386	0\\
387	0\\
388	0\\
389	0\\
390	0\\
391	0\\
392	0\\
393	0\\
394	0\\
395	0\\
396	0\\
397	0\\
398	0\\
399	0\\
400	0\\
401	0\\
402	0\\
403	0\\
404	0\\
405	0\\
406	0\\
407	0\\
408	0\\
409	0\\
410	0\\
411	0\\
412	0\\
413	0\\
414	0\\
415	0\\
416	0\\
417	0\\
418	0\\
419	0\\
420	0\\
421	0\\
422	0\\
423	0\\
424	0\\
425	0\\
426	0\\
427	0\\
428	0\\
429	0\\
430	0\\
431	0\\
432	0\\
433	0\\
434	0\\
435	0\\
436	0\\
437	0\\
438	0\\
439	0\\
440	0\\
441	0\\
442	0\\
443	0\\
444	0\\
445	0\\
446	0\\
447	0\\
448	0\\
449	0\\
450	0\\
451	0\\
452	0\\
453	0\\
454	0\\
455	0\\
456	0\\
457	0\\
458	0\\
459	0\\
460	0\\
461	0\\
462	0\\
463	0\\
464	0\\
465	0\\
466	0\\
467	0\\
468	0\\
469	0\\
470	0\\
471	0\\
472	0\\
473	0\\
474	0\\
475	0\\
476	0\\
477	0\\
478	0\\
479	0\\
480	0\\
481	0\\
482	0\\
483	0\\
484	0\\
485	0\\
486	0\\
487	0\\
488	0\\
489	0\\
490	0\\
491	0\\
492	0\\
493	0\\
494	0\\
495	0\\
496	0\\
497	0\\
498	0\\
499	0\\
500	0\\
501	0\\
502	0\\
503	0\\
504	0\\
505	0\\
506	0\\
507	0\\
508	0\\
509	0\\
510	0\\
511	0\\
512	0\\
513	0\\
514	0\\
515	0\\
516	0\\
517	0\\
518	0\\
519	0\\
520	0\\
521	0\\
522	0\\
523	0\\
524	0\\
525	0\\
526	0\\
527	0\\
528	0\\
529	0\\
530	0\\
531	0\\
532	0\\
533	0\\
534	0\\
535	0\\
536	0\\
537	0\\
538	0\\
539	0\\
540	0\\
541	0\\
542	0\\
543	0\\
544	0\\
545	0\\
546	0\\
547	0\\
548	0\\
549	0\\
550	0\\
551	0\\
552	0\\
553	0\\
554	0\\
555	0\\
556	0\\
557	0\\
558	0\\
559	0\\
560	0\\
561	0\\
562	0\\
563	0\\
564	0\\
565	0\\
566	0\\
567	0\\
568	0\\
569	0\\
570	0\\
571	0\\
572	0\\
573	0\\
574	0\\
575	0\\
576	0\\
577	0\\
578	0\\
579	0\\
580	0\\
581	0\\
582	0\\
583	0\\
584	0\\
585	0\\
586	0\\
587	0\\
588	0\\
589	0\\
590	0\\
591	0\\
592	0\\
593	0\\
594	0\\
595	0\\
596	0\\
597	0\\
598	0\\
599	0\\
600	0\\
};
\addplot [color=blue!40!mycolor9,solid,forget plot]
  table[row sep=crcr]{%
1	0.000572289361079503\\
2	0.000572279828444312\\
3	0.000572270135439682\\
4	0.000572260279371882\\
5	0.000572250257501964\\
6	0.000572240067045005\\
7	0.000572229705169346\\
8	0.00057221916899576\\
9	0.000572208455596701\\
10	0.000572197561995472\\
11	0.000572186485165404\\
12	0.000572175222028997\\
13	0.000572163769457075\\
14	0.000572152124267915\\
15	0.000572140283226384\\
16	0.000572128243043005\\
17	0.000572116000373054\\
18	0.000572103551815636\\
19	0.000572090893912736\\
20	0.000572078023148238\\
21	0.000572064935946996\\
22	0.000572051628673754\\
23	0.000572038097632211\\
24	0.000572024339063951\\
25	0.000572010349147427\\
26	0.000571996123996853\\
27	0.000571981659661137\\
28	0.000571966952122788\\
29	0.00057195199729678\\
30	0.000571936791029434\\
31	0.000571921329097228\\
32	0.000571905607205632\\
33	0.000571889620987887\\
34	0.000571873366003836\\
35	0.000571856837738625\\
36	0.000571840031601462\\
37	0.000571822942924336\\
38	0.000571805566960702\\
39	0.000571787898884128\\
40	0.00057176993378699\\
41	0.000571751666679058\\
42	0.000571733092486115\\
43	0.000571714206048501\\
44	0.000571695002119696\\
45	0.000571675475364842\\
46	0.000571655620359213\\
47	0.000571635431586721\\
48	0.00057161490343834\\
49	0.000571594030210513\\
50	0.000571572806103606\\
51	0.000571551225220185\\
52	0.000571529281563428\\
53	0.000571506969035371\\
54	0.000571484281435221\\
55	0.000571461212457569\\
56	0.00057143775569063\\
57	0.000571413904614426\\
58	0.000571389652598912\\
59	0.000571364992902124\\
60	0.000571339918668244\\
61	0.000571314422925635\\
62	0.000571288498584887\\
63	0.000571262138436795\\
64	0.000571235335150263\\
65	0.000571208081270274\\
66	0.00057118036921569\\
67	0.000571152191277152\\
68	0.000571123539614814\\
69	0.00057109440625612\\
70	0.000571064783093518\\
71	0.000571034661882132\\
72	0.000571004034237361\\
73	0.000570972891632495\\
74	0.000570941225396272\\
75	0.000570909026710325\\
76	0.000570876286606628\\
77	0.000570842995964968\\
78	0.000570809145510212\\
79	0.0005707747258097\\
80	0.00057073972727042\\
81	0.000570704140136268\\
82	0.000570667954485155\\
83	0.000570631160226152\\
84	0.000570593747096516\\
85	0.000570555704658669\\
86	0.000570517022297149\\
87	0.000570477689215458\\
88	0.000570437694432896\\
89	0.000570397026781319\\
90	0.000570355674901806\\
91	0.000570313627241323\\
92	0.000570270872049269\\
93	0.00057022739737396\\
94	0.000570183191059074\\
95	0.000570138240739997\\
96	0.000570092533840138\\
97	0.000570046057567128\\
98	0.000569998798908942\\
99	0.00056995074462999\\
100	0.000569901881267093\\
101	0.000569852195125393\\
102	0.00056980167227418\\
103	0.00056975029854262\\
104	0.000569698059515415\\
105	0.000569644940528375\\
106	0.000569590926663907\\
107	0.000569536002746355\\
108	0.000569480153337341\\
109	0.000569423362730919\\
110	0.000569365614948678\\
111	0.000569306893734748\\
112	0.000569247182550685\\
113	0.000569186464570213\\
114	0.000569124722673949\\
115	0.000569061939443922\\
116	0.000568998097158043\\
117	0.000568933177784415\\
118	0.000568867162975534\\
119	0.000568800034062384\\
120	0.000568731772048393\\
121	0.000568662357603254\\
122	0.000568591771056625\\
123	0.00056851999239169\\
124	0.000568447001238563\\
125	0.000568372776867582\\
126	0.000568297298182457\\
127	0.000568220543713245\\
128	0.000568142491609186\\
129	0.000568063119631404\\
130	0.000567982405145427\\
131	0.000567900325113538\\
132	0.000567816856087025\\
133	0.000567731974198204\\
134	0.000567645655152298\\
135	0.000567557874219155\\
136	0.000567468606224778\\
137	0.000567377825542686\\
138	0.000567285506085131\\
139	0.000567191621294093\\
140	0.000567096144132122\\
141	0.000566999047073011\\
142	0.00056690030209229\\
143	0.000566799880657502\\
144	0.000566697753718357\\
145	0.000566593891696704\\
146	0.000566488264476299\\
147	0.000566380841392455\\
148	0.000566271591221453\\
149	0.000566160482169865\\
150	0.0005660474818637\\
151	0.000565932557337377\\
152	0.000565815675022591\\
153	0.000565696800737025\\
154	0.000565575899672972\\
155	0.000565452936385781\\
156	0.000565327874782326\\
157	0.00056520067810925\\
158	0.000565071308941251\\
159	0.000564939729169241\\
160	0.000564805899988544\\
161	0.000564669781887021\\
162	0.000564531334633243\\
163	0.00056439051726466\\
164	0.000564247288075846\\
165	0.000564101604606802\\
166	0.000563953423631374\\
167	0.000563802701145759\\
168	0.000563649392357172\\
169	0.000563493451672658\\
170	0.000563334832688151\\
171	0.000563173488177561\\
172	0.000563009370082275\\
173	0.000562842429500666\\
174	0.000562672616677942\\
175	0.000562499880996109\\
176	0.000562324170964073\\
177	0.000562145434207974\\
178	0.000561963617461486\\
179	0.00056177866655611\\
180	0.000561590526411449\\
181	0.000561399141025247\\
182	0.000561204453462971\\
183	0.000561006405846999\\
184	0.000560804939345069\\
185	0.000560599994157678\\
186	0.000560391509504379\\
187	0.000560179423608616\\
188	0.000559963673680764\\
189	0.000559744195899196\\
190	0.00055952092538907\\
191	0.000559293796198689\\
192	0.000559062741273323\\
193	0.000558827692426567\\
194	0.000558588580309725\\
195	0.000558345334379785\\
196	0.000558097882867651\\
197	0.000557846152747879\\
198	0.000557590069711211\\
199	0.000557329558141205\\
200	0.000557064541090551\\
201	0.000556794940256949\\
202	0.000556520675958512\\
203	0.000556241667108771\\
204	0.000555957831191096\\
205	0.000555669084232686\\
206	0.000555375340778087\\
207	0.000555076513862098\\
208	0.00055477251498228\\
209	0.000554463254070845\\
210	0.000554148639466027\\
211	0.000553828577882921\\
212	0.00055350297438373\\
213	0.000553171732347426\\
214	0.000552834753438846\\
215	0.000552491937577186\\
216	0.000552143182903838\\
217	0.000551788385749663\\
218	0.000551427440601567\\
219	0.000551060240068446\\
220	0.000550686674846437\\
221	0.000550306633683534\\
222	0.000549920003343464\\
223	0.000549526668568835\\
224	0.000549126512043608\\
225	0.000548719414354775\\
226	0.000548305253953295\\
227	0.000547883907114225\\
228	0.0005474552478961\\
229	0.000547019148099482\\
230	0.000546575477224634\\
231	0.000546124102428374\\
232	0.000545664888480091\\
233	0.000545197697716812\\
234	0.000544722389997394\\
235	0.000544238822655763\\
236	0.000543746850453227\\
237	0.00054324632552977\\
238	0.000542737097354379\\
239	0.000542219012674355\\
240	0.000541691915463587\\
241	0.000541155646869651\\
242	0.000540610045160008\\
243	0.000540054945666896\\
244	0.000539490180731165\\
245	0.000538915579644934\\
246	0.000538330968592991\\
247	0.000537736170593055\\
248	0.000537131005434672\\
249	0.00053651528961688\\
250	0.000535888836284555\\
251	0.000535251455163334\\
252	0.000534602952493253\\
253	0.000533943130960853\\
254	0.00053327178962992\\
255	0.000532588723870659\\
256	0.000531893725287404\\
257	0.000531186581644703\\
258	0.000530467076791844\\
259	0.000529734990585686\\
260	0.000528990098811893\\
261	0.000528232173104341\\
262	0.00052746098086278\\
263	0.000526676285168801\\
264	0.000525877844699746\\
265	0.000525065413640908\\
266	0.000524238741595681\\
267	0.000523397573493714\\
268	0.000522541649497094\\
269	0.000521670704904323\\
270	0.000520784470052275\\
271	0.000519882670215849\\
272	0.000518965025505488\\
273	0.000518031250762205\\
274	0.000517081055450443\\
275	0.000516114143548415\\
276	0.000515130213435959\\
277	0.000514128957779876\\
278	0.00051311006341665\\
279	0.000512073211232512\\
280	0.00051101807604073\\
281	0.000509944326456171\\
282	0.000508851624766938\\
283	0.000507739626803091\\
284	0.000506607981802348\\
285	0.000505456332272721\\
286	0.000504284313852017\\
287	0.000503091555164066\\
288	0.000501877677671679\\
289	0.00050064229552625\\
290	0.000499385015413872\\
291	0.000498105436397899\\
292	0.000496803149758002\\
293	0.00049547773882538\\
294	0.000494128778814315\\
295	0.000492755836649867\\
296	0.000491358470791584\\
297	0.000489936231053213\\
298	0.000488488658418356\\
299	0.000487015284851931\\
300	0.00048551563310734\\
301	0.000483989216529351\\
302	0.000482435538852573\\
303	0.000480854093995371\\
304	0.000479244365849313\\
305	0.000477605828063928\\
306	0.000475937943826749\\
307	0.000474240165638641\\
308	0.000472511935084228\\
309	0.000470752682597514\\
310	0.000468961827222439\\
311	0.000467138776368593\\
312	0.000465282925561809\\
313	0.000463393658189717\\
314	0.000461470345242214\\
315	0.000459512345046848\\
316	0.000457519002999091\\
317	0.000455489651287555\\
318	0.000453423608614047\\
319	0.000451320179908691\\
320	0.000449178656040015\\
321	0.000446998313520135\\
322	0.00044477841420512\\
323	0.000442518204990639\\
324	0.000440216917503031\\
325	0.000437873767785933\\
326	0.000435487955982651\\
327	0.000433058666014573\\
328	0.000430585065255739\\
329	0.000428066304203932\\
330	0.000425501516148543\\
331	0.000422889816835637\\
332	0.000420230304130622\\
333	0.00041752205767883\\
334	0.000414764138564654\\
335	0.000411955588969695\\
336	0.000409095431830549\\
337	0.000406182670496853\\
338	0.000403216288390314\\
339	0.000400195248665453\\
340	0.000397118493873001\\
341	0.000393984945626641\\
342	0.000390793504274319\\
343	0.000387543048574891\\
344	0.00038423243538136\\
345	0.000380860499331881\\
346	0.000377426052549601\\
347	0.000373927884352812\\
348	0.000370364760976633\\
349	0.000366735425307715\\
350	0.000363038596633317\\
351	0.000359272970406324\\
352	0.000355437218027721\\
353	0.000351529986647915\\
354	0.000347549898988649\\
355	0.000343495553186723\\
356	0.000339365522661187\\
357	0.000335158356005238\\
358	0.000330872576904045\\
359	0.000326506684079658\\
360	0.000322059151263819\\
361	0.000317528427199376\\
362	0.000312912935670473\\
363	0.000308211075561671\\
364	0.000303421220945299\\
365	0.000298541721196293\\
366	0.000293570901132558\\
367	0.000288507061178865\\
368	0.000283348477551088\\
369	0.000278093402456643\\
370	0.000272740064306439\\
371	0.000267286667932114\\
372	0.000261731394801327\\
373	0.000256072403222602\\
374	0.000250307828529804\\
375	0.000244435783234991\\
376	0.000238454357136863\\
377	0.000232361617370447\\
378	0.000226155608382433\\
379	0.000219834351815218\\
380	0.000213395846281269\\
381	0.000206838067008912\\
382	0.000200158965339931\\
383	0.000193356468059979\\
384	0.000186428476544544\\
385	0.000179372865706474\\
386	0.00017218748273451\\
387	0.000164870145606742\\
388	0.000157418641337741\\
389	0.000149830723905435\\
390	0.000142104111922943\\
391	0.00013423648565384\\
392	0.000126225483485972\\
393	0.000118068697859579\\
394	0.00010976367063733\\
395	0.000101307887735736\\
396	9.26987715723901e-05\\
397	8.39336683081326e-05\\
398	7.50098166564155e-05\\
399	6.59242533909565e-05\\
400	5.66735099699828e-05\\
401	4.72526162782195e-05\\
402	3.76517861404188e-05\\
403	2.78452662667821e-05\\
404	1.77533983572104e-05\\
405	7.11205651264132e-06\\
406	0\\
407	0\\
408	0\\
409	0\\
410	0\\
411	0\\
412	0\\
413	0\\
414	0\\
415	0\\
416	0\\
417	0\\
418	0\\
419	0\\
420	0\\
421	0\\
422	0\\
423	0\\
424	0\\
425	0\\
426	0\\
427	0\\
428	0\\
429	0\\
430	0\\
431	0\\
432	0\\
433	0\\
434	0\\
435	0\\
436	0\\
437	0\\
438	0\\
439	0\\
440	0\\
441	0\\
442	0\\
443	0\\
444	0\\
445	0\\
446	0\\
447	0\\
448	0\\
449	0\\
450	0\\
451	0\\
452	0\\
453	0\\
454	0\\
455	0\\
456	0\\
457	0\\
458	0\\
459	0\\
460	0\\
461	0\\
462	0\\
463	0\\
464	0\\
465	0\\
466	0\\
467	0\\
468	0\\
469	0\\
470	0\\
471	0\\
472	0\\
473	0\\
474	0\\
475	0\\
476	0\\
477	0\\
478	0\\
479	0\\
480	0\\
481	0\\
482	0\\
483	0\\
484	0\\
485	0\\
486	0\\
487	0\\
488	0\\
489	0\\
490	0\\
491	0\\
492	0\\
493	0\\
494	0\\
495	0\\
496	0\\
497	0\\
498	0\\
499	0\\
500	0\\
501	0\\
502	0\\
503	0\\
504	0\\
505	0\\
506	0\\
507	0\\
508	0\\
509	0\\
510	0\\
511	0\\
512	0\\
513	0\\
514	0\\
515	0\\
516	0\\
517	0\\
518	0\\
519	0\\
520	0\\
521	0\\
522	0\\
523	0\\
524	0\\
525	0\\
526	0\\
527	0\\
528	0\\
529	0\\
530	0\\
531	0\\
532	0\\
533	0\\
534	0\\
535	0\\
536	0\\
537	0\\
538	0\\
539	0\\
540	0\\
541	0\\
542	0\\
543	0\\
544	0\\
545	0\\
546	0\\
547	0\\
548	0\\
549	0\\
550	0\\
551	0\\
552	0\\
553	0\\
554	0\\
555	0\\
556	0\\
557	0\\
558	0\\
559	0\\
560	0\\
561	0\\
562	0\\
563	0\\
564	0\\
565	0\\
566	0\\
567	0\\
568	0\\
569	0\\
570	0\\
571	0\\
572	0\\
573	0\\
574	0\\
575	0\\
576	0\\
577	0\\
578	0\\
579	0\\
580	0\\
581	0\\
582	0\\
583	0\\
584	0\\
585	0\\
586	0\\
587	0\\
588	0\\
589	0\\
590	0\\
591	0\\
592	0\\
593	0\\
594	0\\
595	0\\
596	0\\
597	0\\
598	0\\
599	0\\
600	0\\
};
\addplot [color=blue!75!mycolor7,solid,forget plot]
  table[row sep=crcr]{%
1	0.00162295980517879\\
2	0.00162295046252148\\
3	0.00162294096273216\\
4	0.00162293130316988\\
5	0.00162292148114924\\
6	0.0016229114939396\\
7	0.00162290133876437\\
8	0.00162289101280022\\
9	0.00162288051317626\\
10	0.00162286983697323\\
11	0.00162285898122273\\
12	0.00162284794290632\\
13	0.00162283671895473\\
14	0.00162282530624696\\
15	0.00162281370160937\\
16	0.00162280190181487\\
17	0.00162278990358195\\
18	0.00162277770357375\\
19	0.00162276529839714\\
20	0.00162275268460179\\
21	0.00162273985867911\\
22	0.00162272681706135\\
23	0.00162271355612052\\
24	0.0016227000721674\\
25	0.00162268636145049\\
26	0.00162267242015493\\
27	0.00162265824440144\\
28	0.00162264383024521\\
29	0.00162262917367475\\
30	0.00162261427061081\\
31	0.00162259911690516\\
32	0.00162258370833943\\
33	0.00162256804062393\\
34	0.0016225521093964\\
35	0.00162253591022075\\
36	0.00162251943858585\\
37	0.00162250268990415\\
38	0.00162248565951048\\
39	0.00162246834266063\\
40	0.00162245073452999\\
41	0.00162243283021221\\
42	0.00162241462471777\\
43	0.00162239611297254\\
44	0.00162237728981633\\
45	0.00162235815000139\\
46	0.0016223386881909\\
47	0.00162231889895745\\
48	0.00162229877678144\\
49	0.00162227831604951\\
50	0.00162225751105288\\
51	0.00162223635598575\\
52	0.00162221484494356\\
53	0.00162219297192129\\
54	0.00162217073081174\\
55	0.00162214811540372\\
56	0.00162212511938027\\
57	0.00162210173631675\\
58	0.00162207795967905\\
59	0.0016220537828216\\
60	0.00162202919898548\\
61	0.00162200420129641\\
62	0.00162197878276273\\
63	0.00162195293627333\\
64	0.00162192665459562\\
65	0.00162189993037326\\
66	0.00162187275612416\\
67	0.00162184512423811\\
68	0.00162181702697464\\
69	0.00162178845646065\\
70	0.00162175940468811\\
71	0.00162172986351164\\
72	0.00162169982464612\\
73	0.00162166927966421\\
74	0.00162163821999377\\
75	0.00162160663691539\\
76	0.00162157452155972\\
77	0.00162154186490478\\
78	0.00162150865777329\\
79	0.00162147489082988\\
80	0.00162144055457829\\
81	0.00162140563935846\\
82	0.00162137013534365\\
83	0.00162133403253743\\
84	0.00162129732077062\\
85	0.00162125998969823\\
86	0.00162122202879629\\
87	0.00162118342735862\\
88	0.00162114417449357\\
89	0.00162110425912068\\
90	0.00162106366996724\\
91	0.00162102239556486\\
92	0.00162098042424589\\
93	0.00162093774413989\\
94	0.00162089434316986\\
95	0.00162085020904855\\
96	0.00162080532927466\\
97	0.00162075969112888\\
98	0.00162071328167\\
99	0.00162066608773084\\
100	0.00162061809591408\\
101	0.00162056929258815\\
102	0.0016205196638829\\
103	0.00162046919568523\\
104	0.00162041787363467\\
105	0.00162036568311883\\
106	0.00162031260926882\\
107	0.00162025863695449\\
108	0.00162020375077967\\
109	0.00162014793507727\\
110	0.0016200911739043\\
111	0.00162003345103682\\
112	0.00161997474996473\\
113	0.00161991505388653\\
114	0.00161985434570394\\
115	0.00161979260801643\\
116	0.00161972982311564\\
117	0.00161966597297978\\
118	0.00161960103926773\\
119	0.00161953500331324\\
120	0.00161946784611892\\
121	0.00161939954835011\\
122	0.00161933009032872\\
123	0.00161925945202685\\
124	0.00161918761306037\\
125	0.00161911455268241\\
126	0.00161904024977661\\
127	0.00161896468285042\\
128	0.00161888783002816\\
129	0.00161880966904399\\
130	0.00161873017723482\\
131	0.00161864933153302\\
132	0.00161856710845903\\
133	0.00161848348411394\\
134	0.00161839843417181\\
135	0.00161831193387193\\
136	0.00161822395801102\\
137	0.00161813448093525\\
138	0.00161804347653212\\
139	0.00161795091822225\\
140	0.00161785677895108\\
141	0.00161776103118042\\
142	0.00161766364687985\\
143	0.00161756459751812\\
144	0.00161746385405431\\
145	0.00161736138692892\\
146	0.00161725716605491\\
147	0.00161715116080853\\
148	0.00161704334002016\\
149	0.00161693367196489\\
150	0.00161682212435318\\
151	0.00161670866432125\\
152	0.00161659325842153\\
153	0.00161647587261285\\
154	0.00161635647225068\\
155	0.0016162350220772\\
156	0.00161611148621123\\
157	0.00161598582813821\\
158	0.00161585801069996\\
159	0.0016157279960844\\
160	0.00161559574581517\\
161	0.00161546122074116\\
162	0.00161532438102591\\
163	0.00161518518613694\\
164	0.00161504359483497\\
165	0.00161489956516297\\
166	0.00161475305443516\\
167	0.00161460401922579\\
168	0.00161445241535785\\
169	0.00161429819789152\\
170	0.00161414132111245\\
171	0.00161398173851994\\
172	0.00161381940281468\\
173	0.00161365426588638\\
174	0.00161348627880102\\
175	0.00161331539178782\\
176	0.00161314155422576\\
177	0.00161296471462976\\
178	0.00161278482063631\\
179	0.00161260181898868\\
180	0.00161241565552155\\
181	0.001612226275145\\
182	0.00161203362182788\\
183	0.00161183763858056\\
184	0.00161163826743685\\
185	0.00161143544943531\\
186	0.00161122912459969\\
187	0.00161101923191878\\
188	0.00161080570932537\\
189	0.00161058849367476\\
190	0.00161036752072246\\
191	0.00161014272510164\\
192	0.00160991404030002\\
193	0.00160968139863671\\
194	0.00160944473123896\\
195	0.00160920396801896\\
196	0.0016089590376509\\
197	0.00160870986754805\\
198	0.00160845638383983\\
199	0.00160819851134835\\
200	0.00160793617356469\\
201	0.00160766929262451\\
202	0.00160739778928338\\
203	0.00160712158289149\\
204	0.00160684059136798\\
205	0.00160655473117476\\
206	0.00160626391728973\\
207	0.00160596806317967\\
208	0.00160566708077244\\
209	0.00160536088042871\\
210	0.00160504937091316\\
211	0.00160473245936512\\
212	0.0016044100512686\\
213	0.0016040820504218\\
214	0.001603748358906\\
215	0.00160340887705384\\
216	0.00160306350341704\\
217	0.00160271213473337\\
218	0.00160235466589315\\
219	0.00160199098990493\\
220	0.00160162099786064\\
221	0.0016012445789\\
222	0.00160086162017417\\
223	0.00160047200680884\\
224	0.00160007562186647\\
225	0.00159967234630785\\
226	0.00159926205895291\\
227	0.00159884463644074\\
228	0.00159841995318881\\
229	0.00159798788135147\\
230	0.00159754829077751\\
231	0.00159710104896702\\
232	0.00159664602102728\\
233	0.00159618306962786\\
234	0.00159571205495483\\
235	0.001595232834664\\
236	0.00159474526383333\\
237	0.00159424919491431\\
238	0.00159374447768245\\
239	0.00159323095918677\\
240	0.00159270848369826\\
241	0.00159217689265735\\
242	0.00159163602462031\\
243	0.00159108571520466\\
244	0.00159052579703339\\
245	0.00158995609967814\\
246	0.00158937644960128\\
247	0.00158878667009675\\
248	0.00158818658122983\\
249	0.00158757599977561\\
250	0.00158695473915633\\
251	0.00158632260937742\\
252	0.00158567941696233\\
253	0.00158502496488597\\
254	0.00158435905250693\\
255	0.00158368147549834\\
256	0.00158299202577729\\
257	0.00158229049143296\\
258	0.00158157665665321\\
259	0.00158085030164991\\
260	0.00158011120258253\\
261	0.0015793591314805\\
262	0.00157859385616387\\
263	0.00157781514016245\\
264	0.00157702274263341\\
265	0.00157621641827724\\
266	0.00157539591725204\\
267	0.00157456098508619\\
268	0.00157371136258933\\
269	0.00157284678576154\\
270	0.00157196698570083\\
271	0.00157107168850885\\
272	0.00157016061519472\\
273	0.0015692334815771\\
274	0.00156828999818434\\
275	0.00156732987015272\\
276	0.00156635279712276\\
277	0.00156535847313365\\
278	0.00156434658651553\\
279	0.00156331681977994\\
280	0.00156226884950806\\
281	0.00156120234623699\\
282	0.00156011697434384\\
283	0.00155901239192775\\
284	0.00155788825068978\\
285	0.00155674419581046\\
286	0.00155557986582534\\
287	0.00155439489249817\\
288	0.00155318890069183\\
289	0.00155196150823705\\
290	0.00155071232579872\\
291	0.00154944095673998\\
292	0.00154814699698386\\
293	0.00154683003487266\\
294	0.00154548965102486\\
295	0.00154412541818962\\
296	0.00154273690109895\\
297	0.00154132365631739\\
298	0.00153988523208923\\
299	0.00153842116818331\\
300	0.00153693099573536\\
301	0.00153541423708787\\
302	0.00153387040562747\\
303	0.00153229900561981\\
304	0.00153069953204209\\
305	0.00152907147041296\\
306	0.00152741429662004\\
307	0.00152572747674503\\
308	0.00152401046688623\\
309	0.00152226271297875\\
310	0.00152048365061228\\
311	0.00151867270484644\\
312	0.0015168292900237\\
313	0.00151495280958008\\
314	0.0015130426558534\\
315	0.00151109820988929\\
316	0.0015091188412449\\
317	0.00150710390779045\\
318	0.00150505275550851\\
319	0.00150296471829112\\
320	0.0015008391177349\\
321	0.00149867526293396\\
322	0.00149647245027083\\
323	0.00149422996320542\\
324	0.00149194707206201\\
325	0.0014896230338143\\
326	0.00148725709186868\\
327	0.00148484847584552\\
328	0.00148239640135883\\
329	0.00147990006979397\\
330	0.0014773586680837\\
331	0.00147477136848247\\
332	0.00147213732833884\\
333	0.00146945568986628\\
334	0.00146672557991202\\
335	0.00146394610972412\\
336	0.0014611163747165\\
337	0.00145823545423205\\
338	0.00145530241130345\\
339	0.0014523162924117\\
340	0.00144927612724208\\
341	0.00144618092843732\\
342	0.00144302969134765\\
343	0.00143982139377739\\
344	0.00143655499572767\\
345	0.00143322943913474\\
346	0.00142984364760332\\
347	0.00142639652613438\\
348	0.0014228869608465\\
349	0.00141931381868992\\
350	0.00141567594715247\\
351	0.00141197217395605\\
352	0.00140820130674252\\
353	0.00140436213274755\\
354	0.00140045341846087\\
355	0.00139647390927118\\
356	0.0013924223290937\\
357	0.00138829737997833\\
358	0.001384097741696\\
359	0.00137982207130068\\
360	0.00137546900266415\\
361	0.00137103714598066\\
362	0.00136652508723811\\
363	0.0013619313876523\\
364	0.00135725458306041\\
365	0.00135249318326998\\
366	0.0013476456713591\\
367	0.00134271050292344\\
368	0.00133768610526583\\
369	0.0013325708765238\\
370	0.0013273631847304\\
371	0.0013220613668038\\
372	0.00131666372746137\\
373	0.00131116853805386\\
374	0.0013055740353159\\
375	0.00129987842002959\\
376	0.00129407985559819\\
377	0.00128817646652833\\
378	0.00128216633681967\\
379	0.00127604750826262\\
380	0.00126981797864585\\
381	0.00126347569987737\\
382	0.00125701857602496\\
383	0.00125044446128388\\
384	0.00124375115788245\\
385	0.00123693641393882\\
386	0.00122999792128435\\
387	0.00122293331327162\\
388	0.00121574016258898\\
389	0.00120841597911064\\
390	0.00120095820781001\\
391	0.00119336422679553\\
392	0.00118563134552309\\
393	0.00117775680323375\\
394	0.00116973776762585\\
395	0.00116157133362488\\
396	0.00115325452171927\\
397	0.00114478427407384\\
398	0.00113615744329277\\
399	0.0011273707596541\\
400	0.00111842073867647\\
401	0.0011093034318014\\
402	0.00110001379547988\\
403	0.00109054426602079\\
404	0.00108088232894597\\
405	0.00107101049869601\\
406	0.00106092951432877\\
407	0.00105065977031644\\
408	0.00104019699284573\\
409	0.00102953677969847\\
410	0.00101867464720291\\
411	0.00100760617591885\\
412	0.000996327407526952\\
413	0.000984835854188185\\
414	0.000973132915070548\\
415	0.000961229198796052\\
416	0.000949154471044381\\
417	0.000936968437368094\\
418	0.000924731355869571\\
419	0.000912268345570118\\
420	0.000899549957971183\\
421	0.00088657119851197\\
422	0.000873327160728552\\
423	0.000859813072293951\\
424	0.000846024350965498\\
425	0.000831956672444576\\
426	0.000817606052518265\\
427	0.000802968946306856\\
428	0.000788042368116034\\
429	0.00077282403565568\\
430	0.000757312543762736\\
431	0.000741507575906905\\
432	0.00072541017123818\\
433	0.000709023104064201\\
434	0.000692351598338899\\
435	0.000675405257300434\\
436	0.00065820468577962\\
437	0.000640806376765546\\
438	0.000623398212885161\\
439	0.000606665885673703\\
440	0.000593192831486634\\
441	0.000581579396905864\\
442	0.000569713302659357\\
443	0.000557586012442185\\
444	0.000545188168945224\\
445	0.000532509458907316\\
446	0.000519538454379433\\
447	0.000506262425739107\\
448	0.000492667121011696\\
449	0.000478736504767938\\
450	0.00046445244806613\\
451	0.000449794358060503\\
452	0.000434738730400776\\
453	0.000419258594188202\\
454	0.000403322780750657\\
455	0.000386894828016613\\
456	0.000369930972609213\\
457	0.000352375618567614\\
458	0.000334148997715952\\
459	0.000315109979488416\\
460	0.000294938204456355\\
461	0.000272749625838252\\
462	0.000245909460098321\\
463	0.000218990310996033\\
464	0.000192182069022129\\
465	0.000164840854180496\\
466	0.000136951103771785\\
467	0.000108481453583217\\
468	7.9346960826296e-05\\
469	4.92801445037634e-05\\
470	1.73859184336513e-05\\
471	0\\
472	0\\
473	0\\
474	0\\
475	0\\
476	0\\
477	0\\
478	0\\
479	0\\
480	0\\
481	0\\
482	0\\
483	0\\
484	0\\
485	0\\
486	0\\
487	0\\
488	0\\
489	0\\
490	0\\
491	0\\
492	0\\
493	0\\
494	0\\
495	0\\
496	0\\
497	0\\
498	0\\
499	0\\
500	0\\
501	0\\
502	0\\
503	0\\
504	0\\
505	0\\
506	0\\
507	0\\
508	0\\
509	0\\
510	0\\
511	0\\
512	0\\
513	0\\
514	0\\
515	0\\
516	0\\
517	0\\
518	0\\
519	0\\
520	0\\
521	0\\
522	0\\
523	0\\
524	0\\
525	0\\
526	0\\
527	0\\
528	0\\
529	0\\
530	0\\
531	0\\
532	0\\
533	0\\
534	0\\
535	0\\
536	0\\
537	0\\
538	0\\
539	0\\
540	0\\
541	0\\
542	0\\
543	0\\
544	0\\
545	0\\
546	0\\
547	0\\
548	0\\
549	0\\
550	0\\
551	0\\
552	0\\
553	0\\
554	0\\
555	0\\
556	0\\
557	0\\
558	0\\
559	0\\
560	0\\
561	0\\
562	0\\
563	0\\
564	0\\
565	0\\
566	0\\
567	0\\
568	0\\
569	0\\
570	0\\
571	0\\
572	0\\
573	0\\
574	0\\
575	0\\
576	0\\
577	0\\
578	0\\
579	0\\
580	0\\
581	0\\
582	0\\
583	0\\
584	0\\
585	0\\
586	0\\
587	0\\
588	0\\
589	0\\
590	0\\
591	0\\
592	0\\
593	0\\
594	0\\
595	0\\
596	0\\
597	0\\
598	0\\
599	0\\
600	0\\
};
\addplot [color=blue!80!mycolor9,solid,forget plot]
  table[row sep=crcr]{%
1	0.0031488158044997\\
2	0.00314881010942653\\
3	0.00314880431857838\\
4	0.00314879843034492\\
5	0.00314879244308868\\
6	0.00314878635514459\\
7	0.00314878016481948\\
8	0.00314877387039164\\
9	0.0031487674701103\\
10	0.00314876096219517\\
11	0.0031487543448359\\
12	0.00314874761619161\\
13	0.00314874077439029\\
14	0.00314873381752838\\
15	0.00314872674367014\\
16	0.00314871955084715\\
17	0.00314871223705772\\
18	0.00314870480026634\\
19	0.00314869723840309\\
20	0.00314868954936306\\
21	0.00314868173100576\\
22	0.00314867378115445\\
23	0.00314866569759563\\
24	0.00314865747807829\\
25	0.00314864912031333\\
26	0.00314864062197292\\
27	0.00314863198068978\\
28	0.00314862319405655\\
29	0.00314861425962505\\
30	0.00314860517490563\\
31	0.00314859593736642\\
32	0.00314858654443263\\
33	0.00314857699348576\\
34	0.00314856728186291\\
35	0.00314855740685594\\
36	0.00314854736571077\\
37	0.0031485371556265\\
38	0.00314852677375466\\
39	0.00314851621719838\\
40	0.0031485054830115\\
41	0.00314849456819778\\
42	0.00314848346970997\\
43	0.00314847218444898\\
44	0.00314846070926293\\
45	0.00314844904094627\\
46	0.0031484371762388\\
47	0.00314842511182474\\
48	0.00314841284433181\\
49	0.00314840037033013\\
50	0.0031483876863313\\
51	0.00314837478878737\\
52	0.00314836167408975\\
53	0.0031483483385682\\
54	0.0031483347784897\\
55	0.00314832099005738\\
56	0.00314830696940938\\
57	0.00314829271261774\\
58	0.00314827821568716\\
59	0.0031482634745539\\
60	0.00314824848508452\\
61	0.00314823324307466\\
62	0.00314821774424781\\
63	0.00314820198425396\\
64	0.0031481859586684\\
65	0.00314816966299032\\
66	0.00314815309264149\\
67	0.00314813624296486\\
68	0.00314811910922321\\
69	0.00314810168659764\\
70	0.00314808397018622\\
71	0.0031480659550024\\
72	0.00314804763597357\\
73	0.00314802900793951\\
74	0.0031480100656508\\
75	0.00314799080376724\\
76	0.00314797121685624\\
77	0.00314795129939111\\
78	0.00314793104574942\\
79	0.00314791045021127\\
80	0.00314788950695751\\
81	0.00314786821006797\\
82	0.00314784655351965\\
83	0.00314782453118485\\
84	0.00314780213682928\\
85	0.00314777936411013\\
86	0.00314775620657416\\
87	0.00314773265765558\\
88	0.00314770871067414\\
89	0.00314768435883298\\
90	0.00314765959521654\\
91	0.00314763441278841\\
92	0.00314760880438911\\
93	0.00314758276273387\\
94	0.00314755628041037\\
95	0.00314752934987637\\
96	0.00314750196345738\\
97	0.00314747411334424\\
98	0.00314744579159069\\
99	0.00314741699011081\\
100	0.00314738770067656\\
101	0.00314735791491511\\
102	0.00314732762430623\\
103	0.00314729682017962\\
104	0.00314726549371214\\
105	0.00314723363592501\\
106	0.00314720123768105\\
107	0.00314716828968168\\
108	0.00314713478246407\\
109	0.00314710070639809\\
110	0.00314706605168329\\
111	0.00314703080834579\\
112	0.00314699496623511\\
113	0.00314695851502099\\
114	0.00314692144419008\\
115	0.00314688374304268\\
116	0.0031468454006893\\
117	0.00314680640604725\\
118	0.00314676674783712\\
119	0.0031467264145793\\
120	0.00314668539459026\\
121	0.00314664367597899\\
122	0.00314660124664317\\
123	0.00314655809426546\\
124	0.0031465142063096\\
125	0.00314646957001652\\
126	0.0031464241724004\\
127	0.00314637800024457\\
128	0.00314633104009748\\
129	0.0031462832782685\\
130	0.00314623470082376\\
131	0.00314618529358183\\
132	0.0031461350421094\\
133	0.00314608393171689\\
134	0.00314603194745396\\
135	0.00314597907410506\\
136	0.00314592529618477\\
137	0.00314587059793318\\
138	0.00314581496331124\\
139	0.00314575837599591\\
140	0.00314570081937539\\
141	0.00314564227654423\\
142	0.00314558273029834\\
143	0.00314552216313004\\
144	0.0031454605572229\\
145	0.0031453978944467\\
146	0.00314533415635217\\
147	0.0031452693241657\\
148	0.00314520337878409\\
149	0.00314513630076908\\
150	0.00314506807034191\\
151	0.0031449986673778\\
152	0.00314492807140031\\
153	0.0031448562615757\\
154	0.0031447832167071\\
155	0.00314470891522876\\
156	0.00314463333520007\\
157	0.00314455645429954\\
158	0.00314447824981876\\
159	0.00314439869865611\\
160	0.00314431777731053\\
161	0.00314423546187506\\
162	0.0031441517280303\\
163	0.0031440665510378\\
164	0.00314397990573318\\
165	0.0031438917665193\\
166	0.00314380210735908\\
167	0.00314371090176827\\
168	0.00314361812280806\\
169	0.00314352374307744\\
170	0.00314342773470537\\
171	0.0031433300693428\\
172	0.0031432307181544\\
173	0.00314312965181013\\
174	0.00314302684047649\\
175	0.00314292225380755\\
176	0.00314281586093575\\
177	0.0031427076304624\\
178	0.00314259753044784\\
179	0.00314248552840148\\
180	0.00314237159127135\\
181	0.00314225568543354\\
182	0.00314213777668128\\
183	0.00314201783021375\\
184	0.00314189581062462\\
185	0.00314177168189044\\
186	0.00314164540735863\\
187	0.00314151694973542\\
188	0.00314138627107351\\
189	0.00314125333275963\\
190	0.00314111809550184\\
191	0.00314098051931684\\
192	0.00314084056351704\\
193	0.00314069818669758\\
194	0.00314055334672318\\
195	0.00314040600071485\\
196	0.00314025610503636\\
197	0.00314010361528055\\
198	0.00313994848625536\\
199	0.00313979067196957\\
200	0.00313963012561829\\
201	0.0031394667995682\\
202	0.00313930064534247\\
203	0.00313913161360541\\
204	0.00313895965414691\\
205	0.00313878471586639\\
206	0.00313860674675669\\
207	0.00313842569388749\\
208	0.00313824150338844\\
209	0.00313805412043207\\
210	0.00313786348921623\\
211	0.00313766955294633\\
212	0.00313747225381716\\
213	0.00313727153299439\\
214	0.00313706733059575\\
215	0.0031368595856718\\
216	0.00313664823618643\\
217	0.00313643321899681\\
218	0.0031362144698332\\
219	0.00313599192327818\\
220	0.0031357655127456\\
221	0.00313553517045906\\
222	0.00313530082743007\\
223	0.00313506241343567\\
224	0.00313481985699579\\
225	0.00313457308535002\\
226	0.00313432202443405\\
227	0.0031340665988556\\
228	0.00313380673186999\\
229	0.00313354234535508\\
230	0.00313327335978592\\
231	0.00313299969420882\\
232	0.00313272126621495\\
233	0.00313243799191342\\
234	0.00313214978590395\\
235	0.00313185656124885\\
236	0.00313155822944468\\
237	0.00313125470039322\\
238	0.00313094588237195\\
239	0.00313063168200402\\
240	0.00313031200422758\\
241	0.00312998675226461\\
242	0.0031296558275891\\
243	0.00312931912989473\\
244	0.00312897655706183\\
245	0.00312862800512388\\
246	0.0031282733682332\\
247	0.00312791253862621\\
248	0.00312754540658795\\
249	0.00312717186041587\\
250	0.00312679178638316\\
251	0.00312640506870121\\
252	0.00312601158948155\\
253	0.00312561122869699\\
254	0.00312520386414209\\
255	0.00312478937139297\\
256	0.00312436762376635\\
257	0.00312393849227786\\
258	0.00312350184559967\\
259	0.00312305755001726\\
260	0.0031226054693856\\
261	0.00312214546508438\\
262	0.00312167739597262\\
263	0.00312120111834239\\
264	0.00312071648587183\\
265	0.00312022334957726\\
266	0.00311972155776458\\
267	0.00311921095597983\\
268	0.00311869138695886\\
269	0.00311816269057627\\
270	0.00311762470379345\\
271	0.00311707726060576\\
272	0.00311652019198895\\
273	0.00311595332584462\\
274	0.00311537648694485\\
275	0.00311478949687602\\
276	0.00311419217398167\\
277	0.00311358433330457\\
278	0.00311296578652785\\
279	0.0031123363419153\\
280	0.00311169580425075\\
281	0.00311104397477658\\
282	0.00311038065113134\\
283	0.00310970562728657\\
284	0.00310901869348254\\
285	0.00310831963616339\\
286	0.00310760823791114\\
287	0.00310688427737895\\
288	0.00310614752922352\\
289	0.00310539776403659\\
290	0.00310463474827559\\
291	0.00310385824419343\\
292	0.00310306800976752\\
293	0.00310226379862781\\
294	0.00310144535998416\\
295	0.00310061243855285\\
296	0.0030997647744822\\
297	0.00309890210327755\\
298	0.00309802415572534\\
299	0.00309713065781651\\
300	0.00309622133066907\\
301	0.00309529589045001\\
302	0.00309435404829648\\
303	0.0030933955102362\\
304	0.00309241997710728\\
305	0.00309142714447724\\
306	0.00309041670256155\\
307	0.00308938833614131\\
308	0.00308834172448047\\
309	0.00308727654124234\\
310	0.00308619245440554\\
311	0.00308508912617928\\
312	0.00308396621291814\\
313	0.00308282336503617\\
314	0.00308166022692044\\
315	0.00308047643684404\\
316	0.00307927162687835\\
317	0.00307804542280484\\
318	0.00307679744402615\\
319	0.00307552730347649\\
320	0.00307423460753143\\
321	0.00307291895591687\\
322	0.00307157994161721\\
323	0.00307021715078276\\
324	0.00306883016263607\\
325	0.00306741854937735\\
326	0.00306598187608868\\
327	0.00306451970063701\\
328	0.0030630315735757\\
329	0.00306151703804454\\
330	0.00305997562966797\\
331	0.00305840687645137\\
332	0.0030568102986751\\
333	0.00305518540878608\\
334	0.00305353171128658\\
335	0.0030518487026199\\
336	0.00305013587105251\\
337	0.00304839269655235\\
338	0.00304661865066269\\
339	0.00304481319637114\\
340	0.00304297578797323\\
341	0.00304110587092995\\
342	0.00303920288171855\\
343	0.00303726624767596\\
344	0.00303529538683399\\
345	0.00303328970774543\\
346	0.00303124860930018\\
347	0.00302917148053046\\
348	0.00302705770040394\\
349	0.00302490663760374\\
350	0.00302271765029403\\
351	0.00302049008587002\\
352	0.00301822328069084\\
353	0.00301591655979405\\
354	0.00301356923659012\\
355	0.00301118061253527\\
356	0.00300874997678115\\
357	0.0030062766057994\\
358	0.00300375976297957\\
359	0.00300119869819815\\
360	0.00299859264735711\\
361	0.00299594083188986\\
362	0.00299324245823253\\
363	0.00299049671725858\\
364	0.00298770278367464\\
365	0.00298485981537552\\
366	0.00298196695275613\\
367	0.00297902331797829\\
368	0.00297602801419028\\
369	0.00297298012469691\\
370	0.00296987871207802\\
371	0.00296672281725328\\
372	0.00296351145849092\\
373	0.00296024363035836\\
374	0.00295691830261222\\
375	0.00295353441902534\\
376	0.00295009089614794\\
377	0.00294658662200026\\
378	0.00294302045469302\\
379	0.00293939122097195\\
380	0.00293569771468192\\
381	0.00293193869514513\\
382	0.00292811288544692\\
383	0.0029242189706215\\
384	0.00292025559572781\\
385	0.00291622136380439\\
386	0.00291211483368896\\
387	0.00290793451768638\\
388	0.00290367887906473\\
389	0.00289934632935552\\
390	0.00289493522542896\\
391	0.00289044386630764\\
392	0.00288587048967122\\
393	0.00288121326799\\
394	0.00287647030419765\\
395	0.00287163962676288\\
396	0.00286671918390948\\
397	0.00286170683650244\\
398	0.00285660034861841\\
399	0.00285139737384472\\
400	0.00284609543383834\\
401	0.00284069188478618\\
402	0.00283518387313868\\
403	0.0028295683130977\\
404	0.00282384200316826\\
405	0.00281800200718434\\
406	0.0028120454234071\\
407	0.00280596865857969\\
408	0.0027997679172769\\
409	0.00279343919198299\\
410	0.00278697825900345\\
411	0.00278038068760407\\
412	0.00277364187577775\\
413	0.00276675713162891\\
414	0.00275972180882898\\
415	0.00275253142415608\\
416	0.00274518141276298\\
417	0.0027376656857939\\
418	0.00272997422962037\\
419	0.00272209965955078\\
420	0.00271403454251588\\
421	0.00270577084354978\\
422	0.00269729985161397\\
423	0.00268861209420816\\
424	0.00267969723882085\\
425	0.00267054397888468\\
426	0.00266113990141535\\
427	0.00265147133286866\\
428	0.00264152315879101\\
429	0.00263127861120611\\
430	0.00262071901415549\\
431	0.00260982346880575\\
432	0.00259856843392222\\
433	0.00258692707942594\\
434	0.00257486804188674\\
435	0.00256235239401671\\
436	0.00254932489016489\\
437	0.00253568608287143\\
438	0.00252119866904994\\
439	0.00250516269361007\\
440	0.00248526366011957\\
441	0.00246288514601484\\
442	0.00243997915080848\\
443	0.00241652916105559\\
444	0.0023925180371453\\
445	0.00236792800599649\\
446	0.00234274065856143\\
447	0.00231693695337167\\
448	0.00229049722759709\\
449	0.00226340121732881\\
450	0.00223562808895717\\
451	0.00220715648332688\\
452	0.00217796457304159\\
453	0.00214803012875291\\
454	0.00211733057709868\\
455	0.0020858429982616\\
456	0.00205354392497408\\
457	0.00202040860346689\\
458	0.00198640900300285\\
459	0.00195150951581333\\
460	0.0019156610278376\\
461	0.00187880925939435\\
462	0.00184101668332657\\
463	0.00180306060647431\\
464	0.00176695657113213\\
465	0.00174002127743041\\
466	0.00171255806413367\\
467	0.00168454871256319\\
468	0.00165596653527007\\
469	0.0016267693496688\\
470	0.00159690408596396\\
471	0.00156638900126274\\
472	0.001535271194343\\
473	0.00150354977120314\\
474	0.00147124695810853\\
475	0.0014384333574062\\
476	0.00140525426127058\\
477	0.00137184892090271\\
478	0.00133779609282825\\
479	0.00130297800531827\\
480	0.0012673700781407\\
481	0.00123094637685959\\
482	0.00119367951199304\\
483	0.00115554057384795\\
484	0.00111649904362321\\
485	0.00107652267505202\\
486	0.00103557736814537\\
487	0.00099362708868859\\
488	0.00095063371984951\\
489	0.000906556731923931\\
490	0.000861352959695491\\
491	0.00081497635688427\\
492	0.000767377725191132\\
493	0.000718504413599808\\
494	0.0006682999778587\\
495	0.000616703772697162\\
496	0.000563650397617503\\
497	0.000509068764401486\\
498	0.000452880105573953\\
499	0.000394992927758715\\
500	0.000335289074672282\\
501	0.00027358392453668\\
502	0.000209511731993176\\
503	0.000142196249937518\\
504	6.92959835515677e-05\\
505	0\\
506	0\\
507	0\\
508	0\\
509	0\\
510	0\\
511	0\\
512	0\\
513	0\\
514	0\\
515	0\\
516	0\\
517	0\\
518	0\\
519	0\\
520	0\\
521	0\\
522	0\\
523	0\\
524	0\\
525	0\\
526	0\\
527	0\\
528	0\\
529	0\\
530	0\\
531	0\\
532	0\\
533	0\\
534	0\\
535	0\\
536	0\\
537	0\\
538	0\\
539	0\\
540	0\\
541	0\\
542	0\\
543	0\\
544	0\\
545	0\\
546	0\\
547	0\\
548	0\\
549	0\\
550	0\\
551	0\\
552	0\\
553	0\\
554	0\\
555	0\\
556	0\\
557	0\\
558	0\\
559	0\\
560	0\\
561	0\\
562	0\\
563	0\\
564	0\\
565	0\\
566	0\\
567	0\\
568	0\\
569	0\\
570	0\\
571	0\\
572	0\\
573	0\\
574	0\\
575	0\\
576	0\\
577	0\\
578	0\\
579	0\\
580	0\\
581	0\\
582	0\\
583	0\\
584	0\\
585	0\\
586	0\\
587	0\\
588	0\\
589	0\\
590	0\\
591	0\\
592	0\\
593	0\\
594	0\\
595	0\\
596	0\\
597	0\\
598	0\\
599	0\\
600	0\\
};
\addplot [color=blue,solid,forget plot]
  table[row sep=crcr]{%
1	0.00391145423208649\\
2	0.00391145359632034\\
3	0.00391145294986301\\
4	0.00391145229253474\\
5	0.00391145162415274\\
6	0.00391145094453115\\
7	0.00391145025348098\\
8	0.00391144955081003\\
9	0.00391144883632287\\
10	0.00391144810982078\\
11	0.00391144737110169\\
12	0.00391144661996008\\
13	0.00391144585618702\\
14	0.00391144507957\\
15	0.00391144428989295\\
16	0.00391144348693615\\
17	0.00391144267047614\\
18	0.00391144184028571\\
19	0.0039114409961338\\
20	0.00391144013778546\\
21	0.00391143926500173\\
22	0.00391143837753964\\
23	0.00391143747515211\\
24	0.00391143655758784\\
25	0.00391143562459133\\
26	0.00391143467590271\\
27	0.00391143371125772\\
28	0.00391143273038765\\
29	0.00391143173301919\\
30	0.00391143071887444\\
31	0.00391142968767075\\
32	0.00391142863912071\\
33	0.00391142757293201\\
34	0.00391142648880739\\
35	0.00391142538644453\\
36	0.003911424265536\\
37	0.00391142312576913\\
38	0.00391142196682593\\
39	0.00391142078838303\\
40	0.00391141959011152\\
41	0.00391141837167694\\
42	0.0039114171327391\\
43	0.00391141587295205\\
44	0.00391141459196392\\
45	0.00391141328941686\\
46	0.00391141196494692\\
47	0.00391141061818395\\
48	0.00391140924875146\\
49	0.00391140785626657\\
50	0.00391140644033982\\
51	0.00391140500057516\\
52	0.00391140353656969\\
53	0.00391140204791371\\
54	0.00391140053419044\\
55	0.003911398994976\\
56	0.00391139742983927\\
57	0.00391139583834171\\
58	0.00391139422003729\\
59	0.00391139257447231\\
60	0.00391139090118532\\
61	0.00391138919970692\\
62	0.00391138746955967\\
63	0.00391138571025791\\
64	0.00391138392130764\\
65	0.00391138210220638\\
66	0.00391138025244296\\
67	0.00391137837149746\\
68	0.00391137645884097\\
69	0.00391137451393546\\
70	0.00391137253623364\\
71	0.00391137052517878\\
72	0.00391136848020449\\
73	0.00391136640073466\\
74	0.00391136428618315\\
75	0.00391136213595374\\
76	0.00391135994943986\\
77	0.00391135772602443\\
78	0.00391135546507968\\
79	0.00391135316596697\\
80	0.00391135082803654\\
81	0.00391134845062738\\
82	0.00391134603306697\\
83	0.00391134357467112\\
84	0.00391134107474372\\
85	0.00391133853257653\\
86	0.003911335947449\\
87	0.00391133331862798\\
88	0.00391133064536757\\
89	0.00391132792690882\\
90	0.00391132516247953\\
91	0.003911322351294\\
92	0.00391131949255279\\
93	0.00391131658544247\\
94	0.00391131362913536\\
95	0.00391131062278926\\
96	0.00391130756554721\\
97	0.00391130445653723\\
98	0.00391130129487198\\
99	0.00391129807964859\\
100	0.00391129480994825\\
101	0.00391129148483605\\
102	0.00391128810336056\\
103	0.00391128466455363\\
104	0.00391128116743004\\
105	0.00391127761098719\\
106	0.00391127399420479\\
107	0.00391127031604454\\
108	0.0039112665754498\\
109	0.00391126277134525\\
110	0.00391125890263657\\
111	0.00391125496821008\\
112	0.00391125096693238\\
113	0.00391124689765002\\
114	0.00391124275918911\\
115	0.00391123855035498\\
116	0.00391123426993176\\
117	0.00391122991668204\\
118	0.00391122548934648\\
119	0.00391122098664338\\
120	0.00391121640726832\\
121	0.00391121174989372\\
122	0.00391120701316845\\
123	0.00391120219571741\\
124	0.00391119729614108\\
125	0.00391119231301512\\
126	0.0039111872448899\\
127	0.00391118209029005\\
128	0.00391117684771408\\
129	0.00391117151563381\\
130	0.003911166092494\\
131	0.0039111605767118\\
132	0.00391115496667634\\
133	0.00391114926074819\\
134	0.00391114345725893\\
135	0.00391113755451056\\
136	0.00391113155077509\\
137	0.00391112544429398\\
138	0.0039111192332776\\
139	0.00391111291590479\\
140	0.00391110649032221\\
141	0.0039110999546439\\
142	0.00391109330695071\\
143	0.0039110865452897\\
144	0.00391107966767366\\
145	0.00391107267208049\\
146	0.00391106555645263\\
147	0.00391105831869653\\
148	0.00391105095668199\\
149	0.00391104346824163\\
150	0.00391103585117025\\
151	0.00391102810322426\\
152	0.00391102022212101\\
153	0.00391101220553822\\
154	0.00391100405111328\\
155	0.00391099575644269\\
156	0.00391098731908134\\
157	0.00391097873654187\\
158	0.003910970006294\\
159	0.00391096112576384\\
160	0.00391095209233318\\
161	0.00391094290333881\\
162	0.00391093355607176\\
163	0.00391092404777658\\
164	0.0039109143756506\\
165	0.00391090453684312\\
166	0.00391089452845464\\
167	0.00391088434753605\\
168	0.00391087399108782\\
169	0.0039108634560591\\
170	0.00391085273934689\\
171	0.00391084183779514\\
172	0.00391083074819378\\
173	0.00391081946727783\\
174	0.0039108079917264\\
175	0.00391079631816164\\
176	0.0039107844431478\\
177	0.00391077236319006\\
178	0.00391076007473353\\
179	0.00391074757416205\\
180	0.00391073485779707\\
181	0.00391072192189647\\
182	0.00391070876265335\\
183	0.00391069537619475\\
184	0.00391068175858045\\
185	0.00391066790580162\\
186	0.00391065381377956\\
187	0.00391063947836435\\
188	0.00391062489533349\\
189	0.00391061006039058\\
190	0.00391059496916392\\
191	0.0039105796172051\\
192	0.00391056399998768\\
193	0.00391054811290567\\
194	0.00391053195127217\\
195	0.0039105155103179\\
196	0.00391049878518971\\
197	0.00391048177094907\\
198	0.00391046446257056\\
199	0.00391044685494027\\
200	0.00391042894285424\\
201	0.00391041072101682\\
202	0.00391039218403905\\
203	0.00391037332643693\\
204	0.00391035414262974\\
205	0.00391033462693829\\
206	0.00391031477358314\\
207	0.0039102945766828\\
208	0.00391027403025186\\
209	0.00391025312819916\\
210	0.00391023186432583\\
211	0.00391021023232338\\
212	0.00391018822577169\\
213	0.003910165838137\\
214	0.00391014306276988\\
215	0.00391011989290308\\
216	0.00391009632164944\\
217	0.00391007234199972\\
218	0.00391004794682034\\
219	0.00391002312885119\\
220	0.00390999788070329\\
221	0.00390997219485646\\
222	0.00390994606365694\\
223	0.003909919479315\\
224	0.0039098924339024\\
225	0.00390986491934995\\
226	0.00390983692744492\\
227	0.00390980844982839\\
228	0.00390977947799267\\
229	0.00390975000327855\\
230	0.00390972001687256\\
231	0.00390968950980415\\
232	0.00390965847294288\\
233	0.00390962689699544\\
234	0.00390959477250276\\
235	0.00390956208983697\\
236	0.00390952883919834\\
237	0.00390949501061212\\
238	0.00390946059392543\\
239	0.00390942557880395\\
240	0.00390938995472869\\
241	0.00390935371099259\\
242	0.00390931683669712\\
243	0.00390927932074883\\
244	0.0039092411518558\\
245	0.00390920231852401\\
246	0.00390916280905373\\
247	0.00390912261153578\\
248	0.0039090817138477\\
249	0.00390904010364997\\
250	0.00390899776838202\\
251	0.00390895469525827\\
252	0.00390891087126407\\
253	0.00390886628315158\\
254	0.00390882091743554\\
255	0.00390877476038906\\
256	0.00390872779803919\\
257	0.00390868001616263\\
258	0.00390863140028114\\
259	0.00390858193565703\\
260	0.00390853160728851\\
261	0.00390848039990502\\
262	0.0039084282979624\\
263	0.00390837528563805\\
264	0.003908321346826\\
265	0.0039082664651319\\
266	0.00390821062386793\\
267	0.00390815380604762\\
268	0.0039080959943806\\
269	0.00390803717126733\\
270	0.00390797731879361\\
271	0.00390791641872519\\
272	0.00390785445250214\\
273	0.00390779140123325\\
274	0.0039077272456903\\
275	0.00390766196630227\\
276	0.00390759554314946\\
277	0.00390752795595753\\
278	0.0039074591840915\\
279	0.00390738920654961\\
280	0.00390731800195713\\
281	0.00390724554856015\\
282	0.00390717182421921\\
283	0.00390709680640285\\
284	0.00390702047218123\\
285	0.00390694279821945\\
286	0.00390686376077104\\
287	0.00390678333567118\\
288	0.00390670149832998\\
289	0.00390661822372562\\
290	0.00390653348639748\\
291	0.00390644726043916\\
292	0.00390635951949148\\
293	0.00390627023673536\\
294	0.00390617938488474\\
295	0.00390608693617933\\
296	0.00390599286237742\\
297	0.00390589713474855\\
298	0.00390579972406615\\
299	0.00390570060060022\\
300	0.00390559973410983\\
301	0.00390549709383573\\
302	0.00390539264849274\\
303	0.00390528636626233\\
304	0.00390517821478497\\
305	0.00390506816115258\\
306	0.00390495617190089\\
307	0.00390484221300179\\
308	0.00390472624985571\\
309	0.00390460824728389\\
310	0.00390448816952074\\
311	0.00390436598020609\\
312	0.00390424164237752\\
313	0.0039041151184626\\
314	0.00390398637027116\\
315	0.00390385535898758\\
316	0.00390372204516303\\
317	0.0039035863887077\\
318	0.00390344834888305\\
319	0.00390330788429402\\
320	0.00390316495288124\\
321	0.00390301951191325\\
322	0.0039028715179786\\
323	0.00390272092697804\\
324	0.00390256769411655\\
325	0.00390241177389542\\
326	0.0039022531201042\\
327	0.00390209168581261\\
328	0.00390192742336232\\
329	0.00390176028435864\\
330	0.00390159021966208\\
331	0.00390141717937973\\
332	0.00390124111285647\\
333	0.00390106196866593\\
334	0.00390087969460124\\
335	0.00390069423766547\\
336	0.0039005055440617\\
337	0.00390031355918274\\
338	0.00390011822760038\\
339	0.00389991949305422\\
340	0.00389971729843982\\
341	0.0038995115857963\\
342	0.00389930229629321\\
343	0.00389908937021664\\
344	0.00389887274695436\\
345	0.00389865236498013\\
346	0.00389842816183678\\
347	0.0038982000741182\\
348	0.00389796803745\\
349	0.00389773198646871\\
350	0.00389749185479949\\
351	0.003897247575032\\
352	0.00389699907869457\\
353	0.00389674629622624\\
354	0.00389648915694669\\
355	0.00389622758902382\\
356	0.00389596151943881\\
357	0.00389569087394843\\
358	0.0038954155770446\\
359	0.00389513555191069\\
360	0.00389485072037473\\
361	0.00389456100285906\\
362	0.00389426631832628\\
363	0.00389396658422146\\
364	0.00389366171641021\\
365	0.00389335162911248\\
366	0.00389303623483195\\
367	0.00389271544428074\\
368	0.00389238916629931\\
369	0.00389205730777134\\
370	0.0038917197735333\\
371	0.00389137646627871\\
372	0.00389102728645676\\
373	0.00389067213216507\\
374	0.0038903108990365\\
375	0.00388994348011961\\
376	0.0038895697657526\\
377	0.00388918964343038\\
378	0.00388880299766439\\
379	0.00388840970983479\\
380	0.00388800965803425\\
381	0.003887602716903\\
382	0.00388718875745404\\
383	0.00388676764688758\\
384	0.00388633924839353\\
385	0.00388590342094045\\
386	0.0038854600190492\\
387	0.00388500889254906\\
388	0.00388454988631361\\
389	0.00388408283997319\\
390	0.00388360758759974\\
391	0.00388312395735876\\
392	0.0038826317711215\\
393	0.00388213084402801\\
394	0.00388162098398701\\
395	0.00388110199109091\\
396	0.00388057365691044\\
397	0.00388003576361041\\
398	0.0038794880827994\\
399	0.00387893037401931\\
400	0.00387836238289632\\
401	0.00387778383944593\\
402	0.00387719445811215\\
403	0.00387659394204268\\
404	0.00387598198951732\\
405	0.00387535828191595\\
406	0.00387472246678478\\
407	0.00387407417172406\\
408	0.0038734130033376\\
409	0.00387273854653858\\
410	0.00387205036454529\\
411	0.00387134799997831\\
412	0.00387063097718826\\
413	0.00386989880449444\\
414	0.00386915097106735\\
415	0.00386838692695242\\
416	0.00386760603773274\\
417	0.00386680756080642\\
418	0.00386599080346426\\
419	0.00386515503338101\\
420	0.00386429945743454\\
421	0.00386342321410664\\
422	0.00386252536473786\\
423	0.00386160488343816\\
424	0.00386066064541408\\
425	0.00385969141341615\\
426	0.00385869582192358\\
427	0.00385767235852507\\
428	0.00385661934161984\\
429	0.00385553489274932\\
430	0.00385441689976431\\
431	0.00385326296140546\\
432	0.00385207028869819\\
433	0.00385083549785069\\
434	0.00384955412228839\\
435	0.00384821940056294\\
436	0.00384681926064951\\
437	0.0038453291565604\\
438	0.00384369704008515\\
439	0.00384182192617243\\
440	0.00383957929471537\\
441	0.00383713441622716\\
442	0.00383464069763525\\
443	0.00383209695164383\\
444	0.0038295019659953\\
445	0.00382685450630891\\
446	0.00382415331976878\\
447	0.00382139713981654\\
448	0.00381858469201794\\
449	0.00381571470126403\\
450	0.00381278590039141\\
451	0.00380979704004449\\
452	0.00380674689886397\\
453	0.00380363429115159\\
454	0.00380045806428968\\
455	0.00379721706609648\\
456	0.00379391003229393\\
457	0.00379053526911188\\
458	0.00378708980565845\\
459	0.00378356710029378\\
460	0.00377995040566156\\
461	0.00377619143105491\\
462	0.0037721334462718\\
463	0.0037672220531096\\
464	0.00375953648037518\\
465	0.00374259590734272\\
466	0.00372533544350868\\
467	0.00370774507665263\\
468	0.00368981399443503\\
469	0.00367153110135167\\
470	0.00365288634541003\\
471	0.00363386936568687\\
472	0.00361446838002628\\
473	0.00359467186430756\\
474	0.00357446889131582\\
475	0.00355384847238614\\
476	0.00353279567495486\\
477	0.00351128549018164\\
478	0.00348929938958904\\
479	0.00346681983209936\\
480	0.00344382812335562\\
481	0.00342030431059674\\
482	0.00339622706474022\\
483	0.00337157354405513\\
484	0.00334631923702561\\
485	0.00332043778241085\\
486	0.00329390076330926\\
487	0.0032666774663388\\
488	0.00323873460222879\\
489	0.0032100359918576\\
490	0.00318054220155812\\
491	0.00315021011715184\\
492	0.00311899244306825\\
493	0.00308683710781825\\
494	0.00305368654728657\\
495	0.00301947681520266\\
496	0.0029841364142171\\
497	0.0029475845893136\\
498	0.00290972839773051\\
499	0.00287045663723923\\
500	0.00282962510700878\\
501	0.00278701696808144\\
502	0.00274222972852771\\
503	0.00269434170656592\\
504	0.00264090605458067\\
505	0.00257465229279033\\
506	0.00249789749396208\\
507	0.00241955633568501\\
508	0.00233969171468241\\
509	0.00225848184594896\\
510	0.00217611043500788\\
511	0.00209209122396565\\
512	0.00200591458078461\\
513	0.00191746610396971\\
514	0.00182662038391383\\
515	0.00173323967234684\\
516	0.00163717243645703\\
517	0.00153825200793115\\
518	0.00143629628630456\\
519	0.00133111244935181\\
520	0.00122252232193912\\
521	0.00111046918952441\\
522	0.000995439653286779\\
523	0.000880092985072983\\
524	0.000775499111761147\\
525	0.000689867989463739\\
526	0.000601110061332961\\
527	0.000508813764893824\\
528	0.000412124598928745\\
529	0.000308800153975222\\
530	0.000192070774563915\\
531	5.80908334167806e-05\\
532	0\\
533	0\\
534	0\\
535	0\\
536	0\\
537	0\\
538	0\\
539	0\\
540	0\\
541	0\\
542	0\\
543	0\\
544	0\\
545	0\\
546	0\\
547	0\\
548	0\\
549	0\\
550	0\\
551	0\\
552	0\\
553	0\\
554	0\\
555	0\\
556	0\\
557	0\\
558	0\\
559	0\\
560	0\\
561	0\\
562	0\\
563	0\\
564	0\\
565	0\\
566	0\\
567	0\\
568	0\\
569	0\\
570	0\\
571	0\\
572	0\\
573	0\\
574	0\\
575	0\\
576	0\\
577	0\\
578	0\\
579	0\\
580	0\\
581	0\\
582	0\\
583	0\\
584	0\\
585	0\\
586	0\\
587	0\\
588	0\\
589	0\\
590	0\\
591	0\\
592	0\\
593	0\\
594	0\\
595	0\\
596	0\\
597	0\\
598	0\\
599	0\\
600	0\\
};
\addplot [color=mycolor10,solid,forget plot]
  table[row sep=crcr]{%
1	0.00400023926843176\\
2	0.00400023923938983\\
3	0.00400023920985951\\
4	0.0040002391798326\\
5	0.00400023914930074\\
6	0.00400023911825545\\
7	0.00400023908668808\\
8	0.00400023905458986\\
9	0.00400023902195187\\
10	0.00400023898876501\\
11	0.00400023895502006\\
12	0.00400023892070763\\
13	0.00400023888581818\\
14	0.00400023885034198\\
15	0.00400023881426918\\
16	0.00400023877758973\\
17	0.00400023874029343\\
18	0.00400023870236989\\
19	0.00400023866380856\\
20	0.00400023862459869\\
21	0.00400023858472939\\
22	0.00400023854418953\\
23	0.00400023850296783\\
24	0.00400023846105281\\
25	0.0040002384184328\\
26	0.00400023837509592\\
27	0.00400023833103009\\
28	0.00400023828622305\\
29	0.00400023824066229\\
30	0.00400023819433513\\
31	0.00400023814722864\\
32	0.0040002380993297\\
33	0.00400023805062494\\
34	0.00400023800110079\\
35	0.00400023795074342\\
36	0.0040002378995388\\
37	0.00400023784747262\\
38	0.00400023779453037\\
39	0.00400023774069726\\
40	0.00400023768595826\\
41	0.00400023763029808\\
42	0.0040002375737012\\
43	0.00400023751615178\\
44	0.00400023745763376\\
45	0.00400023739813077\\
46	0.00400023733762619\\
47	0.00400023727610309\\
48	0.00400023721354428\\
49	0.00400023714993225\\
50	0.0040002370852492\\
51	0.00400023701947703\\
52	0.00400023695259733\\
53	0.00400023688459137\\
54	0.0040002368154401\\
55	0.00400023674512415\\
56	0.00400023667362381\\
57	0.00400023660091903\\
58	0.00400023652698944\\
59	0.00400023645181429\\
60	0.00400023637537249\\
61	0.00400023629764258\\
62	0.00400023621860273\\
63	0.00400023613823075\\
64	0.00400023605650405\\
65	0.00400023597339967\\
66	0.00400023588889424\\
67	0.00400023580296397\\
68	0.00400023571558471\\
69	0.00400023562673184\\
70	0.00400023553638036\\
71	0.0040002354445048\\
72	0.00400023535107926\\
73	0.00400023525607742\\
74	0.00400023515947247\\
75	0.00400023506123715\\
76	0.00400023496134373\\
77	0.00400023485976398\\
78	0.00400023475646921\\
79	0.00400023465143022\\
80	0.00400023454461729\\
81	0.00400023443600019\\
82	0.00400023432554817\\
83	0.00400023421322995\\
84	0.00400023409901369\\
85	0.004000233982867\\
86	0.00400023386475693\\
87	0.00400023374464995\\
88	0.00400023362251195\\
89	0.00400023349830821\\
90	0.00400023337200343\\
91	0.00400023324356165\\
92	0.00400023311294632\\
93	0.00400023298012024\\
94	0.00400023284504553\\
95	0.00400023270768369\\
96	0.00400023256799551\\
97	0.0040002324259411\\
98	0.00400023228147987\\
99	0.00400023213457053\\
100	0.00400023198517103\\
101	0.00400023183323862\\
102	0.00400023167872976\\
103	0.00400023152160017\\
104	0.00400023136180476\\
105	0.00400023119929767\\
106	0.00400023103403222\\
107	0.00400023086596091\\
108	0.00400023069503539\\
109	0.00400023052120645\\
110	0.00400023034442404\\
111	0.00400023016463718\\
112	0.00400022998179403\\
113	0.0040002297958418\\
114	0.00400022960672678\\
115	0.0040002294143943\\
116	0.00400022921878872\\
117	0.00400022901985342\\
118	0.00400022881753078\\
119	0.00400022861176213\\
120	0.00400022840248779\\
121	0.004000228189647\\
122	0.00400022797317793\\
123	0.00400022775301766\\
124	0.00400022752910214\\
125	0.00400022730136618\\
126	0.00400022706974344\\
127	0.00400022683416641\\
128	0.00400022659456636\\
129	0.00400022635087337\\
130	0.00400022610301626\\
131	0.00400022585092258\\
132	0.00400022559451863\\
133	0.00400022533372936\\
134	0.00400022506847843\\
135	0.00400022479868814\\
136	0.00400022452427939\\
137	0.00400022424517172\\
138	0.00400022396128322\\
139	0.00400022367253056\\
140	0.00400022337882892\\
141	0.004000223080092\\
142	0.00400022277623198\\
143	0.00400022246715948\\
144	0.00400022215278359\\
145	0.00400022183301175\\
146	0.00400022150774984\\
147	0.00400022117690204\\
148	0.0040002208403709\\
149	0.00400022049805724\\
150	0.00400022014986016\\
151	0.00400021979567702\\
152	0.00400021943540338\\
153	0.00400021906893297\\
154	0.00400021869615773\\
155	0.00400021831696768\\
156	0.00400021793125095\\
157	0.00400021753889377\\
158	0.00400021713978037\\
159	0.004000216733793\\
160	0.00400021632081189\\
161	0.0040002159007152\\
162	0.00400021547337901\\
163	0.00400021503867728\\
164	0.00400021459648178\\
165	0.00400021414666211\\
166	0.00400021368908563\\
167	0.00400021322361742\\
168	0.00400021275012026\\
169	0.00400021226845458\\
170	0.00400021177847842\\
171	0.00400021128004738\\
172	0.00400021077301461\\
173	0.0040002102572307\\
174	0.0040002097325437\\
175	0.00400020919879907\\
176	0.00400020865583955\\
177	0.00400020810350523\\
178	0.00400020754163341\\
179	0.00400020697005859\\
180	0.00400020638861238\\
181	0.00400020579712349\\
182	0.00400020519541765\\
183	0.00400020458331754\\
184	0.00400020396064278\\
185	0.0040002033272098\\
186	0.00400020268283185\\
187	0.00400020202731888\\
188	0.00400020136047753\\
189	0.00400020068211103\\
190	0.00400019999201917\\
191	0.00400019928999819\\
192	0.00400019857584077\\
193	0.00400019784933593\\
194	0.00400019711026898\\
195	0.00400019635842145\\
196	0.004000195593571\\
197	0.00400019481549139\\
198	0.0040001940239524\\
199	0.0040001932187197\\
200	0.00400019239955489\\
201	0.0040001915662153\\
202	0.004000190718454\\
203	0.0040001898560197\\
204	0.00400018897865665\\
205	0.0040001880861046\\
206	0.00400018717809867\\
207	0.00400018625436929\\
208	0.00400018531464214\\
209	0.00400018435863801\\
210	0.00400018338607277\\
211	0.00400018239665722\\
212	0.00400018139009704\\
213	0.00400018036609271\\
214	0.00400017932433935\\
215	0.0040001782645267\\
216	0.00400017718633895\\
217	0.00400017608945471\\
218	0.00400017497354685\\
219	0.00400017383828242\\
220	0.00400017268332256\\
221	0.00400017150832235\\
222	0.00400017031293074\\
223	0.00400016909679041\\
224	0.00400016785953769\\
225	0.0040001666008024\\
226	0.00400016532020775\\
227	0.00400016401737025\\
228	0.00400016269189953\\
229	0.00400016134339824\\
230	0.00400015997146196\\
231	0.00400015857567902\\
232	0.00400015715563037\\
233	0.00400015571088948\\
234	0.00400015424102217\\
235	0.00400015274558649\\
236	0.00400015122413256\\
237	0.00400014967620246\\
238	0.00400014810133003\\
239	0.00400014649904076\\
240	0.00400014486885164\\
241	0.00400014321027096\\
242	0.00400014152279822\\
243	0.00400013980592388\\
244	0.0040001380591293\\
245	0.00400013628188648\\
246	0.00400013447365795\\
247	0.00400013263389657\\
248	0.00400013076204536\\
249	0.00400012885753732\\
250	0.00400012691979526\\
251	0.00400012494823157\\
252	0.00400012294224811\\
253	0.00400012090123594\\
254	0.00400011882457518\\
255	0.00400011671163477\\
256	0.0040001145617723\\
257	0.00400011237433378\\
258	0.00400011014865345\\
259	0.00400010788405356\\
260	0.00400010557984417\\
261	0.0040001032353229\\
262	0.0040001008497747\\
263	0.00400009842247171\\
264	0.00400009595267291\\
265	0.00400009343962396\\
266	0.00400009088255696\\
267	0.00400008828069019\\
268	0.00400008563322785\\
269	0.00400008293935985\\
270	0.00400008019826155\\
271	0.00400007740909347\\
272	0.00400007457100108\\
273	0.00400007168311449\\
274	0.00400006874454821\\
275	0.00400006575440089\\
276	0.004000062711755\\
277	0.00400005961567661\\
278	0.00400005646521507\\
279	0.00400005325940272\\
280	0.00400004999725463\\
281	0.00400004667776828\\
282	0.00400004329992329\\
283	0.00400003986268108\\
284	0.00400003636498461\\
285	0.00400003280575806\\
286	0.00400002918390648\\
287	0.00400002549831555\\
288	0.00400002174785119\\
289	0.00400001793135928\\
290	0.00400001404766534\\
291	0.00400001009557416\\
292	0.00400000607386954\\
293	0.00400000198131388\\
294	0.0039999978166479\\
295	0.00399999357859029\\
296	0.00399998926583733\\
297	0.00399998487706262\\
298	0.00399998041091665\\
299	0.00399997586602652\\
300	0.00399997124099554\\
301	0.00399996653440293\\
302	0.00399996174480339\\
303	0.00399995687072683\\
304	0.00399995191067793\\
305	0.00399994686313584\\
306	0.00399994172655379\\
307	0.00399993649935871\\
308	0.00399993117995091\\
309	0.00399992576670368\\
310	0.00399992025796293\\
311	0.00399991465204682\\
312	0.00399990894724539\\
313	0.00399990314182019\\
314	0.00399989723400391\\
315	0.00399989122199999\\
316	0.00399988510398228\\
317	0.00399987887809461\\
318	0.00399987254245047\\
319	0.00399986609513257\\
320	0.00399985953419253\\
321	0.00399985285765041\\
322	0.00399984606349439\\
323	0.00399983914968035\\
324	0.00399983211413148\\
325	0.00399982495473788\\
326	0.00399981766935617\\
327	0.00399981025580903\\
328	0.00399980271188486\\
329	0.00399979503533729\\
330	0.00399978722388479\\
331	0.00399977927521018\\
332	0.00399977118696023\\
333	0.00399976295674514\\
334	0.00399975458213807\\
335	0.00399974606067465\\
336	0.00399973738985242\\
337	0.00399972856713033\\
338	0.00399971958992814\\
339	0.0039997104556258\\
340	0.00399970116156288\\
341	0.00399969170503784\\
342	0.00399968208330732\\
343	0.00399967229358544\\
344	0.00399966233304298\\
345	0.00399965219880651\\
346	0.00399964188795746\\
347	0.00399963139753122\\
348	0.00399962072451601\\
349	0.00399960986585184\\
350	0.00399959881842922\\
351	0.00399958757908794\\
352	0.00399957614461565\\
353	0.00399956451174636\\
354	0.00399955267715886\\
355	0.00399954063747494\\
356	0.00399952838925759\\
357	0.003999515929009\\
358	0.00399950325316837\\
359	0.00399949035810966\\
360	0.00399947724013911\\
361	0.00399946389549262\\
362	0.0039994503203329\\
363	0.0039994365107465\\
364	0.00399942246274054\\
365	0.00399940817223933\\
366	0.00399939363508073\\
367	0.00399937884701221\\
368	0.00399936380368681\\
369	0.00399934850065873\\
370	0.00399933293337871\\
371	0.00399931709718917\\
372	0.00399930098731901\\
373	0.00399928459887818\\
374	0.00399926792685191\\
375	0.00399925096609462\\
376	0.00399923371132354\\
377	0.00399921615711194\\
378	0.003999198297882\\
379	0.00399918012789727\\
380	0.00399916164125467\\
381	0.00399914283187607\\
382	0.00399912369349926\\
383	0.00399910421966836\\
384	0.00399908440372357\\
385	0.00399906423879013\\
386	0.0039990437177664\\
387	0.00399902283331092\\
388	0.00399900157782828\\
389	0.00399897994345366\\
390	0.0039989579220355\\
391	0.00399893550511629\\
392	0.00399891268391063\\
393	0.0039988894492798\\
394	0.00399886579170178\\
395	0.00399884170123454\\
396	0.00399881716747023\\
397	0.00399879217947742\\
398	0.00399876672573156\\
399	0.00399874079404327\\
400	0.00399871437151541\\
401	0.00399868744458336\\
402	0.00399865999916131\\
403	0.00399863202070385\\
404	0.00399860349370784\\
405	0.00399857440128768\\
406	0.00399854472560509\\
407	0.00399851444782748\\
408	0.00399848354810668\\
409	0.00399845200558938\\
410	0.00399841979846344\\
411	0.00399838690401134\\
412	0.00399835329856065\\
413	0.00399831895709304\\
414	0.00399828385224411\\
415	0.0039982479529837\\
416	0.00399821122496071\\
417	0.00399817363471804\\
418	0.00399813514658468\\
419	0.00399809572187645\\
420	0.00399805531851538\\
421	0.00399801389059089\\
422	0.00399797138785207\\
423	0.00399792775511748\\
424	0.00399788293158426\\
425	0.00399783685000886\\
426	0.00399778943571111\\
427	0.00399774060530526\\
428	0.00399769026494495\\
429	0.00399763830758216\\
430	0.0039975846080481\\
431	0.0039975290131297\\
432	0.00399747132012266\\
433	0.00399741122957038\\
434	0.00399734824351354\\
435	0.00399728146088603\\
436	0.00399720922077238\\
437	0.00399712866805858\\
438	0.00399703587301835\\
439	0.00399692855921201\\
440	0.00399681364852388\\
441	0.00399669628499214\\
442	0.00399657640237339\\
443	0.00399645393267616\\
444	0.00399632880627346\\
445	0.00399620095205716\\
446	0.00399607029764142\\
447	0.00399593676962055\\
448	0.00399580029388026\\
449	0.00399566079594179\\
450	0.00399551820126536\\
451	0.00399537243530081\\
452	0.00399522342271854\\
453	0.00399507108434469\\
454	0.00399491532797951\\
455	0.00399475602314739\\
456	0.00399459293353915\\
457	0.00399442553680086\\
458	0.00399425254035359\\
459	0.0039940705701746\\
460	0.00399387062197271\\
461	0.00399362865816175\\
462	0.00399328224082247\\
463	0.00399268179826925\\
464	0.00399152034506261\\
465	0.00398942525004241\\
466	0.003987295346337\\
467	0.00398512966627529\\
468	0.00398292721000522\\
469	0.00398068697662909\\
470	0.00397840791133193\\
471	0.00397608888287425\\
472	0.00397372874284859\\
473	0.00397132632195255\\
474	0.00396888037428475\\
475	0.00396638943872895\\
476	0.00396385175245868\\
477	0.0039612657638208\\
478	0.00395862990298715\\
479	0.00395594249982952\\
480	0.00395320177500551\\
481	0.00395040582985437\\
482	0.0039475526346765\\
483	0.00394464001520579\\
484	0.00394166563706097\\
485	0.00393862698785699\\
486	0.00393552135645141\\
487	0.00393234580918657\\
488	0.00392909716312208\\
489	0.00392577195493935\\
490	0.00392236640455913\\
491	0.00391887637213836\\
492	0.00391529730638808\\
493	0.0039116241805659\\
494	0.00390785140870817\\
495	0.0039039727252085\\
496	0.00389998098680041\\
497	0.00389586779479689\\
498	0.0038916226814718\\
499	0.00388723122648509\\
500	0.00388267058066376\\
501	0.0038778989473318\\
502	0.00387283208197042\\
503	0.00386729674842656\\
504	0.00386096597286974\\
505	0.0038534149944631\\
506	0.00384507261769542\\
507	0.00383665866359003\\
508	0.00382817739036378\\
509	0.00381962802362775\\
510	0.00381099192896018\\
511	0.00380225392818428\\
512	0.00379341008661713\\
513	0.00378445601683264\\
514	0.00377538679341138\\
515	0.0037661967812744\\
516	0.0037568792082255\\
517	0.00374742495454606\\
518	0.00373781888331623\\
519	0.00372802825804615\\
520	0.00371796508866977\\
521	0.00370736053989885\\
522	0.00369533602505041\\
523	0.00367890636217919\\
524	0.0036476543580472\\
525	0.00359447189176134\\
526	0.00353963152376423\\
527	0.0034829720449033\\
528	0.00342425368577073\\
529	0.00336310709277027\\
530	0.00329908329211729\\
531	0.00323217575810164\\
532	0.00316298165281235\\
533	0.00309166511403029\\
534	0.00301814370668786\\
535	0.00294111313161606\\
536	0.00285499501479115\\
537	0.00274701482624045\\
538	0.00261820957907621\\
539	0.0024858002176776\\
540	0.00234954247109751\\
541	0.00220916502212144\\
542	0.0020643662871162\\
543	0.00191481016471741\\
544	0.00176011232967979\\
545	0.00159980419294568\\
546	0.00143325509308531\\
547	0.00125945611117335\\
548	0.00107633464582706\\
549	0.000878403440878852\\
550	0.000652863504763817\\
551	0.00041853810151925\\
552	0.00017128870103758\\
553	0\\
554	0\\
555	0\\
556	0\\
557	0\\
558	0\\
559	0\\
560	0\\
561	0\\
562	0\\
563	0\\
564	0\\
565	0\\
566	0\\
567	0\\
568	0\\
569	0\\
570	0\\
571	0\\
572	0\\
573	0\\
574	0\\
575	0\\
576	0\\
577	0\\
578	0\\
579	0\\
580	0\\
581	0\\
582	0\\
583	0\\
584	0\\
585	0\\
586	0\\
587	0\\
588	0\\
589	0\\
590	0\\
591	0\\
592	0\\
593	0\\
594	0\\
595	0\\
596	0\\
597	0\\
598	0\\
599	0\\
600	0\\
};
\addplot [color=mycolor11,solid,forget plot]
  table[row sep=crcr]{%
1	0.00406147387421855\\
2	0.00406147387294227\\
3	0.00406147387164454\\
4	0.00406147387032498\\
5	0.00406147386898323\\
6	0.00406147386761891\\
7	0.00406147386623165\\
8	0.00406147386482107\\
9	0.00406147386338676\\
10	0.00406147386192833\\
11	0.00406147386044538\\
12	0.00406147385893749\\
13	0.00406147385740424\\
14	0.0040614738558452\\
15	0.00406147385425995\\
16	0.00406147385264803\\
17	0.004061473851009\\
18	0.00406147384934242\\
19	0.0040614738476478\\
20	0.00406147384592468\\
21	0.00406147384417258\\
22	0.00406147384239101\\
23	0.00406147384057948\\
24	0.00406147383873747\\
25	0.00406147383686449\\
26	0.00406147383495999\\
27	0.00406147383302347\\
28	0.00406147383105436\\
29	0.00406147382905214\\
30	0.00406147382701623\\
31	0.00406147382494607\\
32	0.00406147382284109\\
33	0.00406147382070069\\
34	0.00406147381852427\\
35	0.00406147381631124\\
36	0.00406147381406098\\
37	0.00406147381177285\\
38	0.00406147380944621\\
39	0.00406147380708042\\
40	0.00406147380467482\\
41	0.00406147380222874\\
42	0.00406147379974148\\
43	0.00406147379721236\\
44	0.00406147379464067\\
45	0.0040614737920257\\
46	0.0040614737893667\\
47	0.00406147378666293\\
48	0.00406147378391364\\
49	0.00406147378111807\\
50	0.00406147377827542\\
51	0.0040614737753849\\
52	0.00406147377244571\\
53	0.00406147376945701\\
54	0.00406147376641797\\
55	0.00406147376332775\\
56	0.00406147376018546\\
57	0.00406147375699024\\
58	0.00406147375374118\\
59	0.00406147375043737\\
60	0.00406147374707789\\
61	0.00406147374366179\\
62	0.00406147374018812\\
63	0.00406147373665589\\
64	0.00406147373306411\\
65	0.00406147372941178\\
66	0.00406147372569786\\
67	0.00406147372192131\\
68	0.00406147371808107\\
69	0.00406147371417606\\
70	0.00406147371020517\\
71	0.00406147370616729\\
72	0.00406147370206127\\
73	0.00406147369788597\\
74	0.00406147369364019\\
75	0.00406147368932274\\
76	0.00406147368493241\\
77	0.00406147368046794\\
78	0.00406147367592809\\
79	0.00406147367131156\\
80	0.00406147366661704\\
81	0.00406147366184321\\
82	0.00406147365698872\\
83	0.00406147365205219\\
84	0.00406147364703221\\
85	0.00406147364192737\\
86	0.00406147363673621\\
87	0.00406147363145726\\
88	0.00406147362608902\\
89	0.00406147362062996\\
90	0.00406147361507853\\
91	0.00406147360943315\\
92	0.0040614736036922\\
93	0.00406147359785406\\
94	0.00406147359191705\\
95	0.00406147358587949\\
96	0.00406147357973964\\
97	0.00406147357349575\\
98	0.00406147356714604\\
99	0.00406147356068869\\
100	0.00406147355412185\\
101	0.00406147354744364\\
102	0.00406147354065214\\
103	0.00406147353374541\\
104	0.00406147352672145\\
105	0.00406147351957827\\
106	0.00406147351231378\\
107	0.00406147350492592\\
108	0.00406147349741255\\
109	0.0040614734897715\\
110	0.00406147348200058\\
111	0.00406147347409753\\
112	0.00406147346606009\\
113	0.00406147345788592\\
114	0.00406147344957267\\
115	0.00406147344111792\\
116	0.00406147343251923\\
117	0.0040614734237741\\
118	0.00406147341488001\\
119	0.00406147340583437\\
120	0.00406147339663454\\
121	0.00406147338727787\\
122	0.00406147337776163\\
123	0.00406147336808304\\
124	0.0040614733582393\\
125	0.00406147334822752\\
126	0.0040614733380448\\
127	0.00406147332768816\\
128	0.00406147331715458\\
129	0.00406147330644098\\
130	0.00406147329554423\\
131	0.00406147328446114\\
132	0.00406147327318846\\
133	0.0040614732617229\\
134	0.00406147325006109\\
135	0.00406147323819962\\
136	0.004061473226135\\
137	0.00406147321386371\\
138	0.00406147320138212\\
139	0.00406147318868658\\
140	0.00406147317577335\\
141	0.00406147316263864\\
142	0.00406147314927858\\
143	0.00406147313568924\\
144	0.00406147312186662\\
145	0.00406147310780666\\
146	0.0040614730935052\\
147	0.00406147307895804\\
148	0.00406147306416089\\
149	0.00406147304910937\\
150	0.00406147303379907\\
151	0.00406147301822546\\
152	0.00406147300238396\\
153	0.00406147298626988\\
154	0.00406147296987847\\
155	0.0040614729532049\\
156	0.00406147293624424\\
157	0.0040614729189915\\
158	0.00406147290144159\\
159	0.00406147288358932\\
160	0.00406147286542943\\
161	0.00406147284695656\\
162	0.00406147282816527\\
163	0.00406147280905001\\
164	0.00406147278960514\\
165	0.00406147276982494\\
166	0.00406147274970356\\
167	0.00406147272923508\\
168	0.00406147270841346\\
169	0.00406147268723257\\
170	0.00406147266568616\\
171	0.00406147264376789\\
172	0.0040614726214713\\
173	0.00406147259878982\\
174	0.00406147257571676\\
175	0.00406147255224534\\
176	0.00406147252836863\\
177	0.00406147250407961\\
178	0.00406147247937111\\
179	0.00406147245423586\\
180	0.00406147242866646\\
181	0.00406147240265536\\
182	0.0040614723761949\\
183	0.00406147234927729\\
184	0.00406147232189459\\
185	0.00406147229403871\\
186	0.00406147226570145\\
187	0.00406147223687445\\
188	0.00406147220754919\\
189	0.00406147217771702\\
190	0.00406147214736913\\
191	0.00406147211649655\\
192	0.00406147208509017\\
193	0.0040614720531407\\
194	0.00406147202063869\\
195	0.00406147198757453\\
196	0.00406147195393844\\
197	0.00406147191972046\\
198	0.00406147188491047\\
199	0.00406147184949814\\
200	0.00406147181347299\\
201	0.00406147177682435\\
202	0.00406147173954134\\
203	0.00406147170161291\\
204	0.00406147166302779\\
205	0.00406147162377455\\
206	0.00406147158384153\\
207	0.00406147154321685\\
208	0.00406147150188845\\
209	0.00406147145984404\\
210	0.00406147141707111\\
211	0.00406147137355695\\
212	0.00406147132928859\\
213	0.00406147128425285\\
214	0.00406147123843631\\
215	0.00406147119182533\\
216	0.00406147114440599\\
217	0.00406147109616416\\
218	0.00406147104708543\\
219	0.00406147099715515\\
220	0.00406147094635841\\
221	0.00406147089468003\\
222	0.00406147084210454\\
223	0.00406147078861623\\
224	0.00406147073419909\\
225	0.00406147067883681\\
226	0.00406147062251282\\
227	0.00406147056521024\\
228	0.00406147050691188\\
229	0.00406147044760024\\
230	0.00406147038725754\\
231	0.00406147032586563\\
232	0.00406147026340609\\
233	0.00406147019986012\\
234	0.00406147013520862\\
235	0.00406147006943212\\
236	0.00406147000251083\\
237	0.00406146993442459\\
238	0.00406146986515287\\
239	0.00406146979467477\\
240	0.00406146972296904\\
241	0.00406146965001403\\
242	0.0040614695757877\\
243	0.00406146950026761\\
244	0.00406146942343094\\
245	0.00406146934525444\\
246	0.00406146926571443\\
247	0.00406146918478683\\
248	0.00406146910244713\\
249	0.00406146901867034\\
250	0.00406146893343106\\
251	0.00406146884670341\\
252	0.00406146875846106\\
253	0.0040614686686772\\
254	0.00406146857732452\\
255	0.00406146848437525\\
256	0.0040614683898011\\
257	0.00406146829357326\\
258	0.00406146819566243\\
259	0.00406146809603876\\
260	0.00406146799467186\\
261	0.00406146789153082\\
262	0.00406146778658415\\
263	0.00406146767979979\\
264	0.00406146757114512\\
265	0.00406146746058691\\
266	0.00406146734809136\\
267	0.00406146723362405\\
268	0.00406146711714993\\
269	0.00406146699863334\\
270	0.00406146687803795\\
271	0.00406146675532681\\
272	0.00406146663046229\\
273	0.00406146650340609\\
274	0.00406146637411921\\
275	0.00406146624256198\\
276	0.00406146610869399\\
277	0.00406146597247412\\
278	0.00406146583386051\\
279	0.00406146569281057\\
280	0.00406146554928092\\
281	0.00406146540322742\\
282	0.00406146525460517\\
283	0.00406146510336842\\
284	0.00406146494947065\\
285	0.00406146479286449\\
286	0.00406146463350175\\
287	0.00406146447133337\\
288	0.00406146430630942\\
289	0.00406146413837912\\
290	0.00406146396749075\\
291	0.00406146379359172\\
292	0.00406146361662849\\
293	0.0040614634365466\\
294	0.00406146325329061\\
295	0.00406146306680416\\
296	0.00406146287702986\\
297	0.00406146268390935\\
298	0.00406146248738325\\
299	0.00406146228739114\\
300	0.00406146208387159\\
301	0.00406146187676207\\
302	0.00406146166599901\\
303	0.00406146145151774\\
304	0.00406146123325248\\
305	0.00406146101113634\\
306	0.00406146078510128\\
307	0.00406146055507813\\
308	0.00406146032099653\\
309	0.00406146008278495\\
310	0.00406145984037066\\
311	0.00406145959367972\\
312	0.00406145934263694\\
313	0.0040614590871659\\
314	0.00406145882718891\\
315	0.00406145856262702\\
316	0.00406145829339994\\
317	0.00406145801942611\\
318	0.00406145774062263\\
319	0.00406145745690525\\
320	0.00406145716818835\\
321	0.00406145687438495\\
322	0.00406145657540666\\
323	0.00406145627116369\\
324	0.00406145596156482\\
325	0.00406145564651736\\
326	0.00406145532592719\\
327	0.00406145499969867\\
328	0.00406145466773469\\
329	0.00406145432993661\\
330	0.00406145398620423\\
331	0.00406145363643581\\
332	0.00406145328052804\\
333	0.00406145291837597\\
334	0.00406145254987306\\
335	0.00406145217491112\\
336	0.00406145179338026\\
337	0.00406145140516892\\
338	0.00406145101016383\\
339	0.00406145060824992\\
340	0.00406145019931037\\
341	0.00406144978322657\\
342	0.00406144935987801\\
343	0.00406144892914235\\
344	0.00406144849089529\\
345	0.00406144804501061\\
346	0.00406144759136004\\
347	0.00406144712981331\\
348	0.00406144666023802\\
349	0.00406144618249962\\
350	0.00406144569646135\\
351	0.00406144520198418\\
352	0.00406144469892675\\
353	0.00406144418714527\\
354	0.00406144366649348\\
355	0.00406144313682254\\
356	0.00406144259798095\\
357	0.00406144204981445\\
358	0.00406144149216595\\
359	0.00406144092487536\\
360	0.00406144034777953\\
361	0.00406143976071209\\
362	0.00406143916350335\\
363	0.00406143855598009\\
364	0.00406143793796551\\
365	0.00406143730927897\\
366	0.00406143666973589\\
367	0.00406143601914752\\
368	0.00406143535732077\\
369	0.00406143468405802\\
370	0.00406143399915687\\
371	0.00406143330240993\\
372	0.00406143259360459\\
373	0.00406143187252274\\
374	0.00406143113894053\\
375	0.00406143039262806\\
376	0.00406142963334911\\
377	0.0040614288608608\\
378	0.00406142807491329\\
379	0.00406142727524937\\
380	0.00406142646160414\\
381	0.00406142563370459\\
382	0.00406142479126918\\
383	0.00406142393400737\\
384	0.00406142306161913\\
385	0.00406142217379441\\
386	0.00406142127021255\\
387	0.00406142035054169\\
388	0.00406141941443797\\
389	0.00406141846154484\\
390	0.0040614174914921\\
391	0.00406141650389494\\
392	0.00406141549835268\\
393	0.00406141447444738\\
394	0.00406141343174207\\
395	0.0040614123697786\\
396	0.00406141128807509\\
397	0.00406141018612313\\
398	0.00406140906338551\\
399	0.00406140791929555\\
400	0.00406140675325923\\
401	0.00406140556465838\\
402	0.00406140435284709\\
403	0.00406140311713441\\
404	0.00406140185677566\\
405	0.00406140057098452\\
406	0.00406139925893173\\
407	0.00406139791974453\\
408	0.00406139655250733\\
409	0.00406139515626258\\
410	0.00406139373000972\\
411	0.00406139227269652\\
412	0.00406139078319603\\
413	0.00406138926026895\\
414	0.00406138770253677\\
415	0.00406138610852019\\
416	0.00406138447673179\\
417	0.00406138280557569\\
418	0.00406138109331847\\
419	0.00406137933807158\\
420	0.00406137753777094\\
421	0.00406137569015324\\
422	0.00406137379272801\\
423	0.00406137184274438\\
424	0.00406136983715\\
425	0.00406136777253767\\
426	0.0040613656450696\\
427	0.00406136345035738\\
428	0.00406136118324853\\
429	0.00406135883741312\\
430	0.00406135640450675\\
431	0.00406135387247213\\
432	0.0040613512222096\\
433	0.00406134842152963\\
434	0.00406134541564776\\
435	0.00406134211645094\\
436	0.0040613384023065\\
437	0.00406133415886565\\
438	0.00406132939785058\\
439	0.00406132437530529\\
440	0.00406131924489778\\
441	0.00406131400369967\\
442	0.00406130864870767\\
443	0.00406130317684924\\
444	0.00406129758499008\\
445	0.00406129186994368\\
446	0.00406128602848235\\
447	0.0040612800573483\\
448	0.00406127395325978\\
449	0.00406126771289894\\
450	0.00406126133284752\\
451	0.00406125480938392\\
452	0.00406124813792591\\
453	0.00406124131158086\\
454	0.00406123431746821\\
455	0.00406122712752537\\
456	0.00406121967581232\\
457	0.00406121180344801\\
458	0.0040612031287363\\
459	0.00406119275532929\\
460	0.00406117867059865\\
461	0.00406115669857275\\
462	0.00406111933689257\\
463	0.0040610566265805\\
464	0.00406096569839961\\
465	0.00406087317548331\\
466	0.00406077901137583\\
467	0.00406068315815308\\
468	0.0040605855668307\\
469	0.00406048618571796\\
470	0.00406038496016992\\
471	0.00406028183402784\\
472	0.00406017674864634\\
473	0.00406006964010068\\
474	0.00405996043544122\\
475	0.00405984905393745\\
476	0.00405973542005483\\
477	0.00405961945579985\\
478	0.00405950107808962\\
479	0.00405938019828523\\
480	0.00405925672166008\\
481	0.00405913054678598\\
482	0.00405900156482686\\
483	0.0040588696587283\\
484	0.00405873470228564\\
485	0.0040585965590667\\
486	0.00405845508117609\\
487	0.00405831010784321\\
488	0.00405816146376941\\
489	0.00405800895716853\\
490	0.00405785237739415\\
491	0.00405769149195838\\
492	0.00405752604254285\\
493	0.00405735573912402\\
494	0.00405718025020847\\
495	0.0040569991845618\\
496	0.00405681205392423\\
497	0.00405661819352445\\
498	0.00405641659189319\\
499	0.00405620553753419\\
500	0.00405598193594154\\
501	0.00405574016564575\\
502	0.00405547073960516\\
503	0.00405516067814581\\
504	0.00405480160717274\\
505	0.00405441381211283\\
506	0.00405402099120583\\
507	0.0040536231225162\\
508	0.00405321987985991\\
509	0.00405281039522229\\
510	0.00405239398982295\\
511	0.00405197033797043\\
512	0.00405153907375548\\
513	0.00405109977356717\\
514	0.00405065191342985\\
515	0.00405019475676502\\
516	0.00404972704919735\\
517	0.00404924617980659\\
518	0.00404874588097109\\
519	0.00404820999895577\\
520	0.00404759604850658\\
521	0.00404679381227801\\
522	0.00404553046041244\\
523	0.00404319768801735\\
524	0.00403876878224668\\
525	0.0040321999423493\\
526	0.00402547244576288\\
527	0.00401856983045933\\
528	0.00401147129796997\\
529	0.0040041569952074\\
530	0.00399662529008306\\
531	0.00398888828751701\\
532	0.00398092784527301\\
533	0.00397268828426009\\
534	0.0039640176156458\\
535	0.00395455217310666\\
536	0.00394369284046989\\
537	0.00393062769130563\\
538	0.00391601606445189\\
539	0.00390129681388448\\
540	0.00388646202356258\\
541	0.00387150286326357\\
542	0.00385640915205039\\
543	0.00384116793333786\\
544	0.0038257600908453\\
545	0.00381015356436016\\
546	0.00379428777375626\\
547	0.00377803953381254\\
548	0.00376116179861637\\
549	0.00374325944500222\\
550	0.00372418624906977\\
551	0.00370500254122745\\
552	0.00368548067614745\\
553	0.00366542135653332\\
554	0.00364300836795439\\
555	0.00360768816192235\\
556	0.00352012484187683\\
557	0.0034200905470597\\
558	0.0033114881502614\\
559	0.00318586122956869\\
560	0.00301590666812544\\
561	0.00282709200229717\\
562	0.00263094421539516\\
563	0.00242651276754218\\
564	0.00221179816707534\\
565	0.00198061034947524\\
566	0.00171394633613848\\
567	0.00143955403725653\\
568	0.00115727971134879\\
569	0.000866069879143256\\
570	0.000563740415471681\\
571	0.000244432763086153\\
572	0\\
573	0\\
574	0\\
575	0\\
576	0\\
577	0\\
578	0\\
579	0\\
580	0\\
581	0\\
582	0\\
583	0\\
584	0\\
585	0\\
586	0\\
587	0\\
588	0\\
589	0\\
590	0\\
591	0\\
592	0\\
593	0\\
594	0\\
595	0\\
596	0\\
597	0\\
598	0\\
599	0\\
600	0\\
};
\addplot [color=mycolor12,solid,forget plot]
  table[row sep=crcr]{%
1	0.00510109179991326\\
2	0.00510109179985886\\
3	0.00510109179980354\\
4	0.00510109179974729\\
5	0.0051010917996901\\
6	0.00510109179963195\\
7	0.00510109179957282\\
8	0.00510109179951269\\
9	0.00510109179945155\\
10	0.00510109179938939\\
11	0.00510109179932617\\
12	0.0051010917992619\\
13	0.00510109179919654\\
14	0.00510109179913009\\
15	0.00510109179906252\\
16	0.00510109179899381\\
17	0.00510109179892395\\
18	0.00510109179885291\\
19	0.00510109179878067\\
20	0.00510109179870723\\
21	0.00510109179863254\\
22	0.0051010917985566\\
23	0.00510109179847938\\
24	0.00510109179840087\\
25	0.00510109179832103\\
26	0.00510109179823985\\
27	0.00510109179815731\\
28	0.00510109179807337\\
29	0.00510109179798803\\
30	0.00510109179790124\\
31	0.005101091797813\\
32	0.00510109179772328\\
33	0.00510109179763204\\
34	0.00510109179753927\\
35	0.00510109179744493\\
36	0.00510109179734901\\
37	0.00510109179725148\\
38	0.00510109179715231\\
39	0.00510109179705146\\
40	0.00510109179694892\\
41	0.00510109179684465\\
42	0.00510109179673863\\
43	0.00510109179663082\\
44	0.0051010917965212\\
45	0.00510109179640974\\
46	0.00510109179629639\\
47	0.00510109179618114\\
48	0.00510109179606395\\
49	0.00510109179594478\\
50	0.00510109179582361\\
51	0.00510109179570039\\
52	0.00510109179557511\\
53	0.00510109179544771\\
54	0.00510109179531816\\
55	0.00510109179518643\\
56	0.00510109179505248\\
57	0.00510109179491628\\
58	0.00510109179477778\\
59	0.00510109179463695\\
60	0.00510109179449374\\
61	0.00510109179434812\\
62	0.00510109179420005\\
63	0.00510109179404947\\
64	0.00510109179389636\\
65	0.00510109179374067\\
66	0.00510109179358235\\
67	0.00510109179342137\\
68	0.00510109179325766\\
69	0.0051010917930912\\
70	0.00510109179292192\\
71	0.00510109179274979\\
72	0.00510109179257475\\
73	0.00510109179239676\\
74	0.00510109179221577\\
75	0.00510109179203172\\
76	0.00510109179184456\\
77	0.00510109179165424\\
78	0.0051010917914607\\
79	0.0051010917912639\\
80	0.00510109179106377\\
81	0.00510109179086026\\
82	0.00510109179065331\\
83	0.00510109179044286\\
84	0.00510109179022885\\
85	0.00510109179001123\\
86	0.00510109178978992\\
87	0.00510109178956487\\
88	0.00510109178933601\\
89	0.00510109178910328\\
90	0.00510109178886661\\
91	0.00510109178862593\\
92	0.00510109178838118\\
93	0.00510109178813228\\
94	0.00510109178787917\\
95	0.00510109178762177\\
96	0.00510109178736\\
97	0.0051010917870938\\
98	0.00510109178682309\\
99	0.00510109178654778\\
100	0.00510109178626781\\
101	0.00510109178598308\\
102	0.00510109178569353\\
103	0.00510109178539906\\
104	0.00510109178509958\\
105	0.00510109178479503\\
106	0.0051010917844853\\
107	0.0051010917841703\\
108	0.00510109178384996\\
109	0.00510109178352417\\
110	0.00510109178319283\\
111	0.00510109178285587\\
112	0.00510109178251317\\
113	0.00510109178216463\\
114	0.00510109178181017\\
115	0.00510109178144967\\
116	0.00510109178108302\\
117	0.00510109178071014\\
118	0.00510109178033089\\
119	0.00510109177994519\\
120	0.0051010917795529\\
121	0.00510109177915392\\
122	0.00510109177874814\\
123	0.00510109177833543\\
124	0.00510109177791568\\
125	0.00510109177748875\\
126	0.00510109177705454\\
127	0.0051010917766129\\
128	0.00510109177616372\\
129	0.00510109177570685\\
130	0.00510109177524217\\
131	0.00510109177476954\\
132	0.00510109177428883\\
133	0.00510109177379988\\
134	0.00510109177330256\\
135	0.00510109177279672\\
136	0.00510109177228222\\
137	0.00510109177175889\\
138	0.0051010917712266\\
139	0.00510109177068517\\
140	0.00510109177013446\\
141	0.0051010917695743\\
142	0.00510109176900452\\
143	0.00510109176842496\\
144	0.00510109176783545\\
145	0.00510109176723581\\
146	0.00510109176662587\\
147	0.00510109176600544\\
148	0.00510109176537435\\
149	0.00510109176473241\\
150	0.00510109176407942\\
151	0.0051010917634152\\
152	0.00510109176273955\\
153	0.00510109176205227\\
154	0.00510109176135316\\
155	0.00510109176064201\\
156	0.00510109175991861\\
157	0.00510109175918275\\
158	0.00510109175843421\\
159	0.00510109175767277\\
160	0.00510109175689821\\
161	0.00510109175611029\\
162	0.00510109175530879\\
163	0.00510109175449347\\
164	0.00510109175366408\\
165	0.00510109175282039\\
166	0.00510109175196214\\
167	0.00510109175108909\\
168	0.00510109175020097\\
169	0.00510109174929752\\
170	0.00510109174837848\\
171	0.00510109174744358\\
172	0.00510109174649253\\
173	0.00510109174552507\\
174	0.0051010917445409\\
175	0.00510109174353974\\
176	0.00510109174252128\\
177	0.00510109174148524\\
178	0.00510109174043131\\
179	0.00510109173935916\\
180	0.0051010917382685\\
181	0.00510109173715899\\
182	0.00510109173603032\\
183	0.00510109173488214\\
184	0.00510109173371412\\
185	0.00510109173252591\\
186	0.00510109173131717\\
187	0.00510109173008753\\
188	0.00510109172883664\\
189	0.00510109172756412\\
190	0.0051010917262696\\
191	0.0051010917249527\\
192	0.00510109172361302\\
193	0.00510109172225017\\
194	0.00510109172086375\\
195	0.00510109171945334\\
196	0.00510109171801854\\
197	0.00510109171655891\\
198	0.00510109171507402\\
199	0.00510109171356343\\
200	0.00510109171202669\\
201	0.00510109171046336\\
202	0.00510109170887296\\
203	0.00510109170725502\\
204	0.00510109170560906\\
205	0.0051010917039346\\
206	0.00510109170223113\\
207	0.00510109170049815\\
208	0.00510109169873514\\
209	0.00510109169694158\\
210	0.00510109169511695\\
211	0.00510109169326067\\
212	0.00510109169137222\\
213	0.00510109168945103\\
214	0.00510109168749652\\
215	0.00510109168550811\\
216	0.00510109168348521\\
217	0.00510109168142722\\
218	0.00510109167933351\\
219	0.00510109167720346\\
220	0.00510109167503645\\
221	0.00510109167283181\\
222	0.00510109167058888\\
223	0.00510109166830701\\
224	0.0051010916659855\\
225	0.00510109166362365\\
226	0.00510109166122077\\
227	0.00510109165877613\\
228	0.00510109165628899\\
229	0.00510109165375861\\
230	0.00510109165118423\\
231	0.00510109164856507\\
232	0.00510109164590035\\
233	0.00510109164318926\\
234	0.005101091640431\\
235	0.00510109163762472\\
236	0.00510109163476959\\
237	0.00510109163186473\\
238	0.00510109162890929\\
239	0.00510109162590235\\
240	0.00510109162284302\\
241	0.00510109161973038\\
242	0.00510109161656347\\
243	0.00510109161334134\\
244	0.00510109161006302\\
245	0.00510109160672752\\
246	0.00510109160333382\\
247	0.00510109159988089\\
248	0.00510109159636769\\
249	0.00510109159279315\\
250	0.00510109158915619\\
251	0.0051010915854557\\
252	0.00510109158169055\\
253	0.00510109157785961\\
254	0.0051010915739617\\
255	0.00510109156999565\\
256	0.00510109156596023\\
257	0.00510109156185422\\
258	0.00510109155767638\\
259	0.00510109155342541\\
260	0.00510109154910004\\
261	0.00510109154469893\\
262	0.00510109154022075\\
263	0.00510109153566412\\
264	0.00510109153102764\\
265	0.00510109152630991\\
266	0.00510109152150948\\
267	0.00510109151662488\\
268	0.0051010915116546\\
269	0.00510109150659713\\
270	0.00510109150145092\\
271	0.0051010914962144\\
272	0.00510109149088594\\
273	0.00510109148546392\\
274	0.00510109147994667\\
275	0.00510109147433249\\
276	0.00510109146861967\\
277	0.00510109146280644\\
278	0.00510109145689103\\
279	0.0051010914508716\\
280	0.00510109144474632\\
281	0.00510109143851329\\
282	0.00510109143217059\\
283	0.00510109142571628\\
284	0.00510109141914836\\
285	0.00510109141246483\\
286	0.00510109140566361\\
287	0.00510109139874261\\
288	0.00510109139169971\\
289	0.00510109138453274\\
290	0.00510109137723948\\
291	0.0051010913698177\\
292	0.00510109136226511\\
293	0.00510109135457938\\
294	0.00510109134675816\\
295	0.00510109133879903\\
296	0.00510109133069954\\
297	0.00510109132245721\\
298	0.00510109131406951\\
299	0.00510109130553385\\
300	0.00510109129684761\\
301	0.00510109128800814\\
302	0.00510109127901271\\
303	0.00510109126985857\\
304	0.00510109126054291\\
305	0.00510109125106289\\
306	0.00510109124141559\\
307	0.00510109123159808\\
308	0.00510109122160734\\
309	0.00510109121144035\\
310	0.00510109120109398\\
311	0.0051010911905651\\
312	0.00510109117985049\\
313	0.0051010911689469\\
314	0.00510109115785102\\
315	0.00510109114655949\\
316	0.00510109113506889\\
317	0.00510109112337573\\
318	0.0051010911114765\\
319	0.0051010910993676\\
320	0.00510109108704538\\
321	0.00510109107450615\\
322	0.00510109106174613\\
323	0.00510109104876151\\
324	0.00510109103554841\\
325	0.00510109102210286\\
326	0.00510109100842088\\
327	0.00510109099449839\\
328	0.00510109098033125\\
329	0.00510109096591527\\
330	0.00510109095124619\\
331	0.00510109093631967\\
332	0.00510109092113132\\
333	0.00510109090567667\\
334	0.00510109088995119\\
335	0.00510109087395028\\
336	0.00510109085766926\\
337	0.00510109084110338\\
338	0.00510109082424782\\
339	0.00510109080709769\\
340	0.00510109078964801\\
341	0.00510109077189374\\
342	0.00510109075382975\\
343	0.00510109073545083\\
344	0.00510109071675169\\
345	0.00510109069772694\\
346	0.00510109067837114\\
347	0.00510109065867872\\
348	0.00510109063864405\\
349	0.00510109061826139\\
350	0.00510109059752491\\
351	0.00510109057642869\\
352	0.00510109055496668\\
353	0.00510109053313275\\
354	0.00510109051092066\\
355	0.00510109048832405\\
356	0.00510109046533644\\
357	0.00510109044195123\\
358	0.0051010904181617\\
359	0.005101090393961\\
360	0.00510109036934214\\
361	0.00510109034429798\\
362	0.00510109031882124\\
363	0.00510109029290449\\
364	0.00510109026654014\\
365	0.0051010902397204\\
366	0.00510109021243735\\
367	0.00510109018468284\\
368	0.00510109015644857\\
369	0.00510109012772599\\
370	0.00510109009850639\\
371	0.00510109006878078\\
372	0.00510109003853998\\
373	0.00510109000777454\\
374	0.00510108997647477\\
375	0.00510108994463069\\
376	0.00510108991223205\\
377	0.0051010898792683\\
378	0.00510108984572858\\
379	0.00510108981160168\\
380	0.00510108977687607\\
381	0.00510108974153983\\
382	0.00510108970558068\\
383	0.00510108966898591\\
384	0.0051010896317424\\
385	0.00510108959383656\\
386	0.00510108955525432\\
387	0.0051010895159811\\
388	0.00510108947600176\\
389	0.00510108943530057\\
390	0.00510108939386118\\
391	0.00510108935166651\\
392	0.00510108930869876\\
393	0.00510108926493928\\
394	0.00510108922036845\\
395	0.00510108917496566\\
396	0.00510108912870914\\
397	0.00510108908157593\\
398	0.00510108903354188\\
399	0.00510108898458167\\
400	0.00510108893466886\\
401	0.00510108888377556\\
402	0.00510108883187196\\
403	0.00510108877892617\\
404	0.00510108872490455\\
405	0.00510108866977168\\
406	0.00510108861349032\\
407	0.00510108855602139\\
408	0.00510108849732398\\
409	0.00510108843735507\\
410	0.00510108837606896\\
411	0.00510108831341628\\
412	0.00510108824934293\\
413	0.00510108818378981\\
414	0.00510108811669441\\
415	0.00510108804799222\\
416	0.00510108797761368\\
417	0.00510108790548305\\
418	0.00510108783151769\\
419	0.00510108775562703\\
420	0.00510108767771156\\
421	0.00510108759766138\\
422	0.00510108751535461\\
423	0.00510108743065502\\
424	0.0051010873434088\\
425	0.00510108725343919\\
426	0.00510108716053712\\
427	0.0051010870644438\\
428	0.00510108696481774\\
429	0.00510108686117289\\
430	0.00510108675276765\\
431	0.00510108663842159\\
432	0.00510108651625388\\
433	0.00510108638341271\\
434	0.00510108623605208\\
435	0.00510108607011444\\
436	0.00510108588358932\\
437	0.00510108567976443\\
438	0.00510108546715944\\
439	0.00510108524996865\\
440	0.00510108502806737\\
441	0.0051010848013279\\
442	0.00510108456961976\\
443	0.00510108433281003\\
444	0.00510108409076368\\
445	0.0051010838433438\\
446	0.00510108359041134\\
447	0.00510108333182357\\
448	0.00510108306742894\\
449	0.00510108279705347\\
450	0.00510108252046636\\
451	0.00510108223729676\\
452	0.00510108194683727\\
453	0.00510108164759044\\
454	0.00510108133624716\\
455	0.00510108100545546\\
456	0.00510108063915062\\
457	0.00510108020337091\\
458	0.00510107962988618\\
459	0.00510107879175149\\
460	0.00510107747983606\\
461	0.00510107541719667\\
462	0.00510107239218583\\
463	0.0051010685759856\\
464	0.00510106469226201\\
465	0.00510106073903992\\
466	0.00510105671427971\\
467	0.00510105261587299\\
468	0.00510104844159795\\
469	0.00510104418911902\\
470	0.00510103985600564\\
471	0.00510103543967317\\
472	0.00510103093728759\\
473	0.00510102634570732\\
474	0.00510102166160921\\
475	0.00510101688175904\\
476	0.00510101200277714\\
477	0.00510100702105975\\
478	0.00510100193275785\\
479	0.00510099673375294\\
480	0.00510099141962911\\
481	0.00510098598564094\\
482	0.0051009804266767\\
483	0.00510097473721582\\
484	0.00510096891127982\\
485	0.0051009629423756\\
486	0.00510095682342939\\
487	0.00510095054670774\\
488	0.0051009441037201\\
489	0.00510093748509278\\
490	0.00510093068039336\\
491	0.00510092367786143\\
492	0.00510091646395133\\
493	0.00510090902248744\\
494	0.00510090133302177\\
495	0.00510089336758987\\
496	0.00510088508440327\\
497	0.00510087641614243\\
498	0.00510086725001991\\
499	0.00510085739884259\\
500	0.00510084657245365\\
501	0.00510083438676113\\
502	0.00510082049579685\\
503	0.00510080494095675\\
504	0.0051007885116239\\
505	0.0051007718624104\\
506	0.00510075498557465\\
507	0.00510073786101444\\
508	0.00510072045661552\\
509	0.00510070274822352\\
510	0.00510068472006531\\
511	0.00510066635294463\\
512	0.00510064762101038\\
513	0.00510062848380644\\
514	0.00510060886667179\\
515	0.0051005886129507\\
516	0.00510056736990364\\
517	0.00510054432491002\\
518	0.00510051762369321\\
519	0.0051004831778553\\
520	0.00510043250933198\\
521	0.00510034977287883\\
522	0.00510021062121271\\
523	0.00509999263205191\\
524	0.00509971271547763\\
525	0.00509942494706406\\
526	0.00509912851553759\\
527	0.00509882253688619\\
528	0.00509850628966137\\
529	0.00509817947339278\\
530	0.00509784152438081\\
531	0.00509748999325015\\
532	0.00509711952279453\\
533	0.00509671942295103\\
534	0.00509627145328356\\
535	0.00509575324774495\\
536	0.00509515058158533\\
537	0.00509449543104951\\
538	0.00509383028532832\\
539	0.00509315442498853\\
540	0.00509246701986117\\
541	0.00509176705844467\\
542	0.00509105318802637\\
543	0.00509032339795339\\
544	0.00508957445850129\\
545	0.00508880094436249\\
546	0.00508799382521893\\
547	0.00508713939509688\\
548	0.0050862221275849\\
549	0.00508523810584875\\
550	0.00508420975301684\\
551	0.00508308849968024\\
552	0.00508175972602962\\
553	0.00507990484665965\\
554	0.00507669461183126\\
555	0.00507044038034362\\
556	0.00505922864793289\\
557	0.00504656909230222\\
558	0.00503244725364375\\
559	0.00501578264226783\\
560	0.00499545046704825\\
561	0.00497398626098588\\
562	0.00495230936879711\\
563	0.00493026676404129\\
564	0.00490753314712123\\
565	0.00488359464059053\\
566	0.0048582284635519\\
567	0.00483327164309529\\
568	0.00480872353417642\\
569	0.00478452488381562\\
570	0.00476060339303416\\
571	0.00473706064438208\\
572	0.0047147640350247\\
573	0.00469513866116233\\
574	0.00467754660897937\\
575	0.00465821856935375\\
576	0.00463326745163829\\
577	0.00459144151574509\\
578	0.00450001461999156\\
579	0.00425903716898513\\
580	0.00396782385977364\\
581	0.00364026773511628\\
582	0.00330161831566113\\
583	0.00294712380411371\\
584	0.00256295996406943\\
585	0.0021687264087653\\
586	0.00176496329681716\\
587	0.00134883630364181\\
588	0.000913380900315357\\
589	0.000438890169453977\\
590	0\\
591	0\\
592	0\\
593	0\\
594	0\\
595	0\\
596	0\\
597	0\\
598	0\\
599	0\\
600	0\\
};
\addplot [color=mycolor13,solid,forget plot]
  table[row sep=crcr]{%
1	0\\
2	0\\
3	0\\
4	0\\
5	0\\
6	0\\
7	0\\
8	0\\
9	0\\
10	0\\
11	0\\
12	0\\
13	0\\
14	0\\
15	0\\
16	0\\
17	0\\
18	0\\
19	0\\
20	0\\
21	0\\
22	0\\
23	0\\
24	0\\
25	0\\
26	0\\
27	0\\
28	0\\
29	0\\
30	0\\
31	0\\
32	0\\
33	0\\
34	0\\
35	0\\
36	0\\
37	0\\
38	0\\
39	0\\
40	0\\
41	0\\
42	0\\
43	0\\
44	0\\
45	0\\
46	0\\
47	0\\
48	0\\
49	0\\
50	0\\
51	0\\
52	0\\
53	0\\
54	0\\
55	0\\
56	0\\
57	0\\
58	0\\
59	0\\
60	0\\
61	0\\
62	0\\
63	0\\
64	0\\
65	0\\
66	0\\
67	0\\
68	0\\
69	0\\
70	0\\
71	0\\
72	0\\
73	0\\
74	0\\
75	0\\
76	0\\
77	0\\
78	0\\
79	0\\
80	0\\
81	0\\
82	0\\
83	0\\
84	0\\
85	0\\
86	0\\
87	0\\
88	0\\
89	0\\
90	0\\
91	0\\
92	0\\
93	0\\
94	0\\
95	0\\
96	0\\
97	0\\
98	0\\
99	0\\
100	0\\
101	0\\
102	0\\
103	0\\
104	0\\
105	0\\
106	0\\
107	0\\
108	0\\
109	0\\
110	0\\
111	0\\
112	0\\
113	0\\
114	0\\
115	0\\
116	0\\
117	0\\
118	0\\
119	0\\
120	0\\
121	0\\
122	0\\
123	0\\
124	0\\
125	0\\
126	0\\
127	0\\
128	0\\
129	0\\
130	0\\
131	0\\
132	0\\
133	0\\
134	0\\
135	0\\
136	0\\
137	0\\
138	0\\
139	0\\
140	0\\
141	0\\
142	0\\
143	0\\
144	0\\
145	0\\
146	0\\
147	0\\
148	0\\
149	0\\
150	0\\
151	0\\
152	0\\
153	0\\
154	0\\
155	0\\
156	0\\
157	0\\
158	0\\
159	0\\
160	0\\
161	0\\
162	0\\
163	0\\
164	0\\
165	0\\
166	0\\
167	0\\
168	0\\
169	0\\
170	0\\
171	0\\
172	0\\
173	0\\
174	0\\
175	0\\
176	0\\
177	0\\
178	0\\
179	0\\
180	0\\
181	0\\
182	0\\
183	0\\
184	0\\
185	0\\
186	0\\
187	0\\
188	0\\
189	0\\
190	0\\
191	0\\
192	0\\
193	0\\
194	0\\
195	0\\
196	0\\
197	0\\
198	0\\
199	0\\
200	0\\
201	0\\
202	0\\
203	0\\
204	0\\
205	0\\
206	0\\
207	0\\
208	0\\
209	0\\
210	0\\
211	0\\
212	0\\
213	0\\
214	0\\
215	0\\
216	0\\
217	0\\
218	0\\
219	0\\
220	0\\
221	0\\
222	0\\
223	0\\
224	0\\
225	0\\
226	0\\
227	0\\
228	0\\
229	0\\
230	0\\
231	0\\
232	0\\
233	0\\
234	0\\
235	0\\
236	0\\
237	0\\
238	0\\
239	0\\
240	0\\
241	0\\
242	0\\
243	0\\
244	0\\
245	0\\
246	0\\
247	0\\
248	0\\
249	0\\
250	0\\
251	0\\
252	0\\
253	0\\
254	0\\
255	0\\
256	0\\
257	0\\
258	0\\
259	0\\
260	0\\
261	0\\
262	0\\
263	0\\
264	0\\
265	0\\
266	0\\
267	0\\
268	0\\
269	0\\
270	0\\
271	0\\
272	0\\
273	0\\
274	0\\
275	0\\
276	0\\
277	0\\
278	0\\
279	0\\
280	0\\
281	0\\
282	0\\
283	0\\
284	0\\
285	0\\
286	0\\
287	0\\
288	0\\
289	0\\
290	0\\
291	0\\
292	0\\
293	0\\
294	0\\
295	0\\
296	0\\
297	0\\
298	0\\
299	0\\
300	0\\
301	0\\
302	0\\
303	0\\
304	0\\
305	0\\
306	0\\
307	0\\
308	0\\
309	0\\
310	0\\
311	0\\
312	0\\
313	0\\
314	0\\
315	0\\
316	0\\
317	0\\
318	0\\
319	0\\
320	0\\
321	0\\
322	0\\
323	0\\
324	0\\
325	0\\
326	0\\
327	0\\
328	0\\
329	0\\
330	0\\
331	0\\
332	0\\
333	0\\
334	0\\
335	0\\
336	0\\
337	0\\
338	0\\
339	0\\
340	0\\
341	0\\
342	0\\
343	0\\
344	0\\
345	0\\
346	0\\
347	0\\
348	0\\
349	0\\
350	0\\
351	0\\
352	0\\
353	0\\
354	0\\
355	0\\
356	0\\
357	0\\
358	0\\
359	0\\
360	0\\
361	0\\
362	0\\
363	0\\
364	0\\
365	0\\
366	0\\
367	0\\
368	0\\
369	0\\
370	0\\
371	0\\
372	0\\
373	0\\
374	0\\
375	0\\
376	0\\
377	0\\
378	0\\
379	0\\
380	0\\
381	0\\
382	0\\
383	0\\
384	0\\
385	0\\
386	0\\
387	0\\
388	0\\
389	0\\
390	0\\
391	0\\
392	0\\
393	0\\
394	0\\
395	0\\
396	0\\
397	0\\
398	0\\
399	0\\
400	0\\
401	0\\
402	0\\
403	0\\
404	0\\
405	0\\
406	0\\
407	0\\
408	0\\
409	0\\
410	0\\
411	0\\
412	0\\
413	0\\
414	0\\
415	0\\
416	0\\
417	0\\
418	0\\
419	0\\
420	0\\
421	0\\
422	0\\
423	0\\
424	0\\
425	0\\
426	0\\
427	0\\
428	0\\
429	0\\
430	0\\
431	0\\
432	0\\
433	0\\
434	0\\
435	0\\
436	0\\
437	0\\
438	0\\
439	0\\
440	0\\
441	0\\
442	0\\
443	0\\
444	0\\
445	0\\
446	0\\
447	0\\
448	0\\
449	0\\
450	0\\
451	0\\
452	0\\
453	0\\
454	0\\
455	0\\
456	0\\
457	0\\
458	0\\
459	0\\
460	0\\
461	0\\
462	0\\
463	0\\
464	0\\
465	0\\
466	0\\
467	0\\
468	0\\
469	0\\
470	0\\
471	0\\
472	0\\
473	0\\
474	0\\
475	0\\
476	0\\
477	0\\
478	0\\
479	0\\
480	0\\
481	0\\
482	0\\
483	0\\
484	0\\
485	0\\
486	0\\
487	0\\
488	0\\
489	0\\
490	0\\
491	0\\
492	0\\
493	0\\
494	0\\
495	0\\
496	0\\
497	0\\
498	0\\
499	0\\
500	0\\
501	0\\
502	0\\
503	0\\
504	0\\
505	0\\
506	0\\
507	0\\
508	0\\
509	0\\
510	0\\
511	0\\
512	0\\
513	0\\
514	0\\
515	0\\
516	0\\
517	0\\
518	0\\
519	0\\
520	0\\
521	0\\
522	0\\
523	0\\
524	0\\
525	0\\
526	0\\
527	0\\
528	0\\
529	0\\
530	0\\
531	0\\
532	0\\
533	0\\
534	0\\
535	0\\
536	0\\
537	0\\
538	0\\
539	0\\
540	0\\
541	0\\
542	0\\
543	0\\
544	0\\
545	0\\
546	0\\
547	0\\
548	0\\
549	0\\
550	0\\
551	0\\
552	0\\
553	0\\
554	0\\
555	0\\
556	0\\
557	0\\
558	0\\
559	0\\
560	0\\
561	0\\
562	0\\
563	0\\
564	0\\
565	0\\
566	0\\
567	0\\
568	0\\
569	0\\
570	0\\
571	0\\
572	0\\
573	0\\
574	0\\
575	0\\
576	0\\
577	0\\
578	0\\
579	0\\
580	0\\
581	0\\
582	0\\
583	0\\
584	0\\
585	0\\
586	0\\
587	0\\
588	0\\
589	0\\
590	0\\
591	0\\
592	0\\
593	0\\
594	9.98598075601821e-05\\
595	0.000594592516223747\\
596	0.00167287594548724\\
597	0.00320324344438694\\
598	0.00587935965236376\\
599	0\\
600	0\\
};
\addplot [color=mycolor14,solid,forget plot]
  table[row sep=crcr]{%
1	0\\
2	0\\
3	0\\
4	0\\
5	0\\
6	0\\
7	0\\
8	0\\
9	0\\
10	0\\
11	0\\
12	0\\
13	0\\
14	0\\
15	0\\
16	0\\
17	0\\
18	0\\
19	0\\
20	0\\
21	0\\
22	0\\
23	0\\
24	0\\
25	0\\
26	0\\
27	0\\
28	0\\
29	0\\
30	0\\
31	0\\
32	0\\
33	0\\
34	0\\
35	0\\
36	0\\
37	0\\
38	0\\
39	0\\
40	0\\
41	0\\
42	0\\
43	0\\
44	0\\
45	0\\
46	0\\
47	0\\
48	0\\
49	0\\
50	0\\
51	0\\
52	0\\
53	0\\
54	0\\
55	0\\
56	0\\
57	0\\
58	0\\
59	0\\
60	0\\
61	0\\
62	0\\
63	0\\
64	0\\
65	0\\
66	0\\
67	0\\
68	0\\
69	0\\
70	0\\
71	0\\
72	0\\
73	0\\
74	0\\
75	0\\
76	0\\
77	0\\
78	0\\
79	0\\
80	0\\
81	0\\
82	0\\
83	0\\
84	0\\
85	0\\
86	0\\
87	0\\
88	0\\
89	0\\
90	0\\
91	0\\
92	0\\
93	0\\
94	0\\
95	0\\
96	0\\
97	0\\
98	0\\
99	0\\
100	0\\
101	0\\
102	0\\
103	0\\
104	0\\
105	0\\
106	0\\
107	0\\
108	0\\
109	0\\
110	0\\
111	0\\
112	0\\
113	0\\
114	0\\
115	0\\
116	0\\
117	0\\
118	0\\
119	0\\
120	0\\
121	0\\
122	0\\
123	0\\
124	0\\
125	0\\
126	0\\
127	0\\
128	0\\
129	0\\
130	0\\
131	0\\
132	0\\
133	0\\
134	0\\
135	0\\
136	0\\
137	0\\
138	0\\
139	0\\
140	0\\
141	0\\
142	0\\
143	0\\
144	0\\
145	0\\
146	0\\
147	0\\
148	0\\
149	0\\
150	0\\
151	0\\
152	0\\
153	0\\
154	0\\
155	0\\
156	0\\
157	0\\
158	0\\
159	0\\
160	0\\
161	0\\
162	0\\
163	0\\
164	0\\
165	0\\
166	0\\
167	0\\
168	0\\
169	0\\
170	0\\
171	0\\
172	0\\
173	0\\
174	0\\
175	0\\
176	0\\
177	0\\
178	0\\
179	0\\
180	0\\
181	0\\
182	0\\
183	0\\
184	0\\
185	0\\
186	0\\
187	0\\
188	0\\
189	0\\
190	0\\
191	0\\
192	0\\
193	0\\
194	0\\
195	0\\
196	0\\
197	0\\
198	0\\
199	0\\
200	0\\
201	0\\
202	0\\
203	0\\
204	0\\
205	0\\
206	0\\
207	0\\
208	0\\
209	0\\
210	0\\
211	0\\
212	0\\
213	0\\
214	0\\
215	0\\
216	0\\
217	0\\
218	0\\
219	0\\
220	0\\
221	0\\
222	0\\
223	0\\
224	0\\
225	0\\
226	0\\
227	0\\
228	0\\
229	0\\
230	0\\
231	0\\
232	0\\
233	0\\
234	0\\
235	0\\
236	0\\
237	0\\
238	0\\
239	0\\
240	0\\
241	0\\
242	0\\
243	0\\
244	0\\
245	0\\
246	0\\
247	0\\
248	0\\
249	0\\
250	0\\
251	0\\
252	0\\
253	0\\
254	0\\
255	0\\
256	0\\
257	0\\
258	0\\
259	0\\
260	0\\
261	0\\
262	0\\
263	0\\
264	0\\
265	0\\
266	0\\
267	0\\
268	0\\
269	0\\
270	0\\
271	0\\
272	0\\
273	0\\
274	0\\
275	0\\
276	0\\
277	0\\
278	0\\
279	0\\
280	0\\
281	0\\
282	0\\
283	0\\
284	0\\
285	0\\
286	0\\
287	0\\
288	0\\
289	0\\
290	0\\
291	0\\
292	0\\
293	0\\
294	0\\
295	0\\
296	0\\
297	0\\
298	0\\
299	0\\
300	0\\
301	0\\
302	0\\
303	0\\
304	0\\
305	0\\
306	0\\
307	0\\
308	0\\
309	0\\
310	0\\
311	0\\
312	0\\
313	0\\
314	0\\
315	0\\
316	0\\
317	0\\
318	0\\
319	0\\
320	0\\
321	0\\
322	0\\
323	0\\
324	0\\
325	0\\
326	0\\
327	0\\
328	0\\
329	0\\
330	0\\
331	0\\
332	0\\
333	0\\
334	0\\
335	0\\
336	0\\
337	0\\
338	0\\
339	0\\
340	0\\
341	0\\
342	0\\
343	0\\
344	0\\
345	0\\
346	0\\
347	0\\
348	0\\
349	0\\
350	0\\
351	0\\
352	0\\
353	0\\
354	0\\
355	0\\
356	0\\
357	0\\
358	0\\
359	0\\
360	0\\
361	0\\
362	0\\
363	0\\
364	0\\
365	0\\
366	0\\
367	0\\
368	0\\
369	0\\
370	0\\
371	0\\
372	0\\
373	0\\
374	0\\
375	0\\
376	0\\
377	0\\
378	0\\
379	0\\
380	0\\
381	0\\
382	0\\
383	0\\
384	0\\
385	0\\
386	0\\
387	0\\
388	0\\
389	0\\
390	0\\
391	0\\
392	0\\
393	0\\
394	0\\
395	0\\
396	0\\
397	0\\
398	0\\
399	0\\
400	0\\
401	0\\
402	0\\
403	0\\
404	0\\
405	0\\
406	0\\
407	0\\
408	0\\
409	0\\
410	0\\
411	0\\
412	0\\
413	0\\
414	0\\
415	0\\
416	0\\
417	0\\
418	0\\
419	0\\
420	0\\
421	0\\
422	0\\
423	0\\
424	0\\
425	0\\
426	0\\
427	0\\
428	0\\
429	0\\
430	0\\
431	0\\
432	0\\
433	0\\
434	0\\
435	0\\
436	0\\
437	0\\
438	0\\
439	0\\
440	0\\
441	0\\
442	0\\
443	0\\
444	0\\
445	0\\
446	0\\
447	0\\
448	0\\
449	0\\
450	0\\
451	0\\
452	0\\
453	0\\
454	0\\
455	0\\
456	0\\
457	0\\
458	0\\
459	0\\
460	0\\
461	0\\
462	0\\
463	0\\
464	0\\
465	0\\
466	0\\
467	0\\
468	0\\
469	0\\
470	0\\
471	0\\
472	0\\
473	0\\
474	0\\
475	0\\
476	0\\
477	0\\
478	0\\
479	0\\
480	0\\
481	0\\
482	0\\
483	0\\
484	0\\
485	0\\
486	0\\
487	0\\
488	0\\
489	0\\
490	0\\
491	0\\
492	0\\
493	0\\
494	0\\
495	0\\
496	0\\
497	0\\
498	0\\
499	0\\
500	0\\
501	0\\
502	0\\
503	0\\
504	0\\
505	0\\
506	0\\
507	0\\
508	0\\
509	0\\
510	0\\
511	0\\
512	0\\
513	0\\
514	0\\
515	0\\
516	0\\
517	0\\
518	0\\
519	0\\
520	0\\
521	0\\
522	0\\
523	0\\
524	0\\
525	0\\
526	0\\
527	0\\
528	0\\
529	0\\
530	0\\
531	0\\
532	0\\
533	0\\
534	0\\
535	0\\
536	0\\
537	0\\
538	0\\
539	0\\
540	0\\
541	0\\
542	0\\
543	0\\
544	0\\
545	0\\
546	0\\
547	0\\
548	0\\
549	0\\
550	0\\
551	0\\
552	0\\
553	0\\
554	0\\
555	0\\
556	0\\
557	0\\
558	0\\
559	0\\
560	0\\
561	0\\
562	0\\
563	0\\
564	0\\
565	0\\
566	0\\
567	0\\
568	0\\
569	0\\
570	0\\
571	0\\
572	0\\
573	0\\
574	0\\
575	0\\
576	0\\
577	0\\
578	0\\
579	0\\
580	0\\
581	0\\
582	0\\
583	0\\
584	0\\
585	0\\
586	0.000167079632204449\\
587	0.000456071715075539\\
588	0.000759434260115618\\
589	0.000951398265291226\\
590	0.00114744334461536\\
591	0.0013504417486169\\
592	0.00161095250100881\\
593	0.00219885314480476\\
594	0.00282841440239993\\
595	0.00353909829286836\\
596	0.00399657994176672\\
597	0.0047545345923529\\
598	0.00632942537858856\\
599	0\\
600	0\\
};
\addplot [color=mycolor15,solid,forget plot]
  table[row sep=crcr]{%
1	0\\
2	0\\
3	0\\
4	0\\
5	0\\
6	0\\
7	0\\
8	0\\
9	0\\
10	0\\
11	0\\
12	0\\
13	0\\
14	0\\
15	0\\
16	0\\
17	0\\
18	0\\
19	0\\
20	0\\
21	0\\
22	0\\
23	0\\
24	0\\
25	0\\
26	0\\
27	0\\
28	0\\
29	0\\
30	0\\
31	0\\
32	0\\
33	0\\
34	0\\
35	0\\
36	0\\
37	0\\
38	0\\
39	0\\
40	0\\
41	0\\
42	0\\
43	0\\
44	0\\
45	0\\
46	0\\
47	0\\
48	0\\
49	0\\
50	0\\
51	0\\
52	0\\
53	0\\
54	0\\
55	0\\
56	0\\
57	0\\
58	0\\
59	0\\
60	0\\
61	0\\
62	0\\
63	0\\
64	0\\
65	0\\
66	0\\
67	0\\
68	0\\
69	0\\
70	0\\
71	0\\
72	0\\
73	0\\
74	0\\
75	0\\
76	0\\
77	0\\
78	0\\
79	0\\
80	0\\
81	0\\
82	0\\
83	0\\
84	0\\
85	0\\
86	0\\
87	0\\
88	0\\
89	0\\
90	0\\
91	0\\
92	0\\
93	0\\
94	0\\
95	0\\
96	0\\
97	0\\
98	0\\
99	0\\
100	0\\
101	0\\
102	0\\
103	0\\
104	0\\
105	0\\
106	0\\
107	0\\
108	0\\
109	0\\
110	0\\
111	0\\
112	0\\
113	0\\
114	0\\
115	0\\
116	0\\
117	0\\
118	0\\
119	0\\
120	0\\
121	0\\
122	0\\
123	0\\
124	0\\
125	0\\
126	0\\
127	0\\
128	0\\
129	0\\
130	0\\
131	0\\
132	0\\
133	0\\
134	0\\
135	0\\
136	0\\
137	0\\
138	0\\
139	0\\
140	0\\
141	0\\
142	0\\
143	0\\
144	0\\
145	0\\
146	0\\
147	0\\
148	0\\
149	0\\
150	0\\
151	0\\
152	0\\
153	0\\
154	0\\
155	0\\
156	0\\
157	0\\
158	0\\
159	0\\
160	0\\
161	0\\
162	0\\
163	0\\
164	0\\
165	0\\
166	0\\
167	0\\
168	0\\
169	0\\
170	0\\
171	0\\
172	0\\
173	0\\
174	0\\
175	0\\
176	0\\
177	0\\
178	0\\
179	0\\
180	0\\
181	0\\
182	0\\
183	0\\
184	0\\
185	0\\
186	0\\
187	0\\
188	0\\
189	0\\
190	0\\
191	0\\
192	0\\
193	0\\
194	0\\
195	0\\
196	0\\
197	0\\
198	0\\
199	0\\
200	0\\
201	0\\
202	0\\
203	0\\
204	0\\
205	0\\
206	0\\
207	0\\
208	0\\
209	0\\
210	0\\
211	0\\
212	0\\
213	0\\
214	0\\
215	0\\
216	0\\
217	0\\
218	0\\
219	0\\
220	0\\
221	0\\
222	0\\
223	0\\
224	0\\
225	0\\
226	0\\
227	0\\
228	0\\
229	0\\
230	0\\
231	0\\
232	0\\
233	0\\
234	0\\
235	0\\
236	0\\
237	0\\
238	0\\
239	0\\
240	0\\
241	0\\
242	0\\
243	0\\
244	0\\
245	0\\
246	0\\
247	0\\
248	0\\
249	0\\
250	0\\
251	0\\
252	0\\
253	0\\
254	0\\
255	0\\
256	0\\
257	0\\
258	0\\
259	0\\
260	0\\
261	0\\
262	0\\
263	0\\
264	0\\
265	0\\
266	0\\
267	0\\
268	0\\
269	0\\
270	0\\
271	0\\
272	0\\
273	0\\
274	0\\
275	0\\
276	0\\
277	0\\
278	0\\
279	0\\
280	0\\
281	0\\
282	0\\
283	0\\
284	0\\
285	0\\
286	0\\
287	0\\
288	0\\
289	0\\
290	0\\
291	0\\
292	0\\
293	0\\
294	0\\
295	0\\
296	0\\
297	0\\
298	0\\
299	0\\
300	0\\
301	0\\
302	0\\
303	0\\
304	0\\
305	0\\
306	0\\
307	0\\
308	0\\
309	0\\
310	0\\
311	0\\
312	0\\
313	0\\
314	0\\
315	0\\
316	0\\
317	0\\
318	0\\
319	0\\
320	0\\
321	0\\
322	0\\
323	0\\
324	0\\
325	0\\
326	0\\
327	0\\
328	0\\
329	0\\
330	0\\
331	0\\
332	0\\
333	0\\
334	0\\
335	0\\
336	0\\
337	0\\
338	0\\
339	0\\
340	0\\
341	0\\
342	0\\
343	0\\
344	0\\
345	0\\
346	0\\
347	0\\
348	0\\
349	0\\
350	0\\
351	0\\
352	0\\
353	0\\
354	0\\
355	0\\
356	0\\
357	0\\
358	0\\
359	0\\
360	0\\
361	0\\
362	0\\
363	0\\
364	0\\
365	0\\
366	0\\
367	0\\
368	0\\
369	0\\
370	0\\
371	0\\
372	0\\
373	0\\
374	0\\
375	0\\
376	0\\
377	0\\
378	0\\
379	0\\
380	0\\
381	0\\
382	0\\
383	0\\
384	0\\
385	0\\
386	0\\
387	0\\
388	0\\
389	0\\
390	0\\
391	0\\
392	0\\
393	0\\
394	0\\
395	0\\
396	0\\
397	0\\
398	0\\
399	0\\
400	0\\
401	0\\
402	0\\
403	0\\
404	0\\
405	0\\
406	0\\
407	0\\
408	0\\
409	0\\
410	0\\
411	0\\
412	0\\
413	0\\
414	0\\
415	0\\
416	0\\
417	0\\
418	0\\
419	0\\
420	0\\
421	0\\
422	0\\
423	0\\
424	0\\
425	0\\
426	0\\
427	0\\
428	0\\
429	0\\
430	0\\
431	0\\
432	0\\
433	0\\
434	0\\
435	0\\
436	0\\
437	0\\
438	0\\
439	0\\
440	0\\
441	0\\
442	0\\
443	0\\
444	0\\
445	0\\
446	0\\
447	0\\
448	0\\
449	0\\
450	0\\
451	0\\
452	0\\
453	0\\
454	0\\
455	0\\
456	0\\
457	0\\
458	0\\
459	0\\
460	0\\
461	0\\
462	0\\
463	0\\
464	0\\
465	0\\
466	0\\
467	0\\
468	0\\
469	0\\
470	0\\
471	0\\
472	0\\
473	0\\
474	0\\
475	0\\
476	0\\
477	0\\
478	0\\
479	0\\
480	0\\
481	0\\
482	0\\
483	0\\
484	0\\
485	0\\
486	0\\
487	0\\
488	0\\
489	0\\
490	0\\
491	0\\
492	0\\
493	0\\
494	0\\
495	0\\
496	0\\
497	0\\
498	0\\
499	0\\
500	0\\
501	0\\
502	0\\
503	0\\
504	0\\
505	0\\
506	0\\
507	0\\
508	0\\
509	0\\
510	0\\
511	0\\
512	0\\
513	0\\
514	0\\
515	0\\
516	0\\
517	0\\
518	0\\
519	0\\
520	0\\
521	0\\
522	0\\
523	0\\
524	0\\
525	0\\
526	0\\
527	0\\
528	0\\
529	0\\
530	0\\
531	0\\
532	0\\
533	0\\
534	0\\
535	0\\
536	0\\
537	0\\
538	0\\
539	0\\
540	0\\
541	0\\
542	0\\
543	0\\
544	0\\
545	0\\
546	0\\
547	0\\
548	0\\
549	0\\
550	0\\
551	0\\
552	0\\
553	0\\
554	0\\
555	0\\
556	0\\
557	0\\
558	0\\
559	0\\
560	0\\
561	0\\
562	0\\
563	0\\
564	0\\
565	0\\
566	0\\
567	0\\
568	0\\
569	0\\
570	0\\
571	0\\
572	0\\
573	0\\
574	0\\
575	0\\
576	0\\
577	0\\
578	0\\
579	0.000143936771195565\\
580	0.000398036759884528\\
581	0.00066092679016297\\
582	0.000819713415261157\\
583	0.00096387575301797\\
584	0.00110653927940749\\
585	0.00124960147875673\\
586	0.00139021956195936\\
587	0.00152910028465296\\
588	0.00165893564622842\\
589	0.00185748867593825\\
590	0.00235311191256397\\
591	0.00285266479926953\\
592	0.00331714209625941\\
593	0.00350129544269832\\
594	0.00369876633599398\\
595	0.00393138688805932\\
596	0.00424489192197758\\
597	0.0048700418989439\\
598	0.00632942537858856\\
599	0\\
600	0\\
};
\addplot [color=mycolor16,solid,forget plot]
  table[row sep=crcr]{%
1	0\\
2	0\\
3	0\\
4	0\\
5	0\\
6	0\\
7	0\\
8	0\\
9	0\\
10	0\\
11	0\\
12	0\\
13	0\\
14	0\\
15	0\\
16	0\\
17	0\\
18	0\\
19	0\\
20	0\\
21	0\\
22	0\\
23	0\\
24	0\\
25	0\\
26	0\\
27	0\\
28	0\\
29	0\\
30	0\\
31	0\\
32	0\\
33	0\\
34	0\\
35	0\\
36	0\\
37	0\\
38	0\\
39	0\\
40	0\\
41	0\\
42	0\\
43	0\\
44	0\\
45	0\\
46	0\\
47	0\\
48	0\\
49	0\\
50	0\\
51	0\\
52	0\\
53	0\\
54	0\\
55	0\\
56	0\\
57	0\\
58	0\\
59	0\\
60	0\\
61	0\\
62	0\\
63	0\\
64	0\\
65	0\\
66	0\\
67	0\\
68	0\\
69	0\\
70	0\\
71	0\\
72	0\\
73	0\\
74	0\\
75	0\\
76	0\\
77	0\\
78	0\\
79	0\\
80	0\\
81	0\\
82	0\\
83	0\\
84	0\\
85	0\\
86	0\\
87	0\\
88	0\\
89	0\\
90	0\\
91	0\\
92	0\\
93	0\\
94	0\\
95	0\\
96	0\\
97	0\\
98	0\\
99	0\\
100	0\\
101	0\\
102	0\\
103	0\\
104	0\\
105	0\\
106	0\\
107	0\\
108	0\\
109	0\\
110	0\\
111	0\\
112	0\\
113	0\\
114	0\\
115	0\\
116	0\\
117	0\\
118	0\\
119	0\\
120	0\\
121	0\\
122	0\\
123	0\\
124	0\\
125	0\\
126	0\\
127	0\\
128	0\\
129	0\\
130	0\\
131	0\\
132	0\\
133	0\\
134	0\\
135	0\\
136	0\\
137	0\\
138	0\\
139	0\\
140	0\\
141	0\\
142	0\\
143	0\\
144	0\\
145	0\\
146	0\\
147	0\\
148	0\\
149	0\\
150	0\\
151	0\\
152	0\\
153	0\\
154	0\\
155	0\\
156	0\\
157	0\\
158	0\\
159	0\\
160	0\\
161	0\\
162	0\\
163	0\\
164	0\\
165	0\\
166	0\\
167	0\\
168	0\\
169	0\\
170	0\\
171	0\\
172	0\\
173	0\\
174	0\\
175	0\\
176	0\\
177	0\\
178	0\\
179	0\\
180	0\\
181	0\\
182	0\\
183	0\\
184	0\\
185	0\\
186	0\\
187	0\\
188	0\\
189	0\\
190	0\\
191	0\\
192	0\\
193	0\\
194	0\\
195	0\\
196	0\\
197	0\\
198	0\\
199	0\\
200	0\\
201	0\\
202	0\\
203	0\\
204	0\\
205	0\\
206	0\\
207	0\\
208	0\\
209	0\\
210	0\\
211	0\\
212	0\\
213	0\\
214	0\\
215	0\\
216	0\\
217	0\\
218	0\\
219	0\\
220	0\\
221	0\\
222	0\\
223	0\\
224	0\\
225	0\\
226	0\\
227	0\\
228	0\\
229	0\\
230	0\\
231	0\\
232	0\\
233	0\\
234	0\\
235	0\\
236	0\\
237	0\\
238	0\\
239	0\\
240	0\\
241	0\\
242	0\\
243	0\\
244	0\\
245	0\\
246	0\\
247	0\\
248	0\\
249	0\\
250	0\\
251	0\\
252	0\\
253	0\\
254	0\\
255	0\\
256	0\\
257	0\\
258	0\\
259	0\\
260	0\\
261	0\\
262	0\\
263	0\\
264	0\\
265	0\\
266	0\\
267	0\\
268	0\\
269	0\\
270	0\\
271	0\\
272	0\\
273	0\\
274	0\\
275	0\\
276	0\\
277	0\\
278	0\\
279	0\\
280	0\\
281	0\\
282	0\\
283	0\\
284	0\\
285	0\\
286	0\\
287	0\\
288	0\\
289	0\\
290	0\\
291	0\\
292	0\\
293	0\\
294	0\\
295	0\\
296	0\\
297	0\\
298	0\\
299	0\\
300	0\\
301	0\\
302	0\\
303	0\\
304	0\\
305	0\\
306	0\\
307	0\\
308	0\\
309	0\\
310	0\\
311	0\\
312	0\\
313	0\\
314	0\\
315	0\\
316	0\\
317	0\\
318	0\\
319	0\\
320	0\\
321	0\\
322	0\\
323	0\\
324	0\\
325	0\\
326	0\\
327	0\\
328	0\\
329	0\\
330	0\\
331	0\\
332	0\\
333	0\\
334	0\\
335	0\\
336	0\\
337	0\\
338	0\\
339	0\\
340	0\\
341	0\\
342	0\\
343	0\\
344	0\\
345	0\\
346	0\\
347	0\\
348	0\\
349	0\\
350	0\\
351	0\\
352	0\\
353	0\\
354	0\\
355	0\\
356	0\\
357	0\\
358	0\\
359	0\\
360	0\\
361	0\\
362	0\\
363	0\\
364	0\\
365	0\\
366	0\\
367	0\\
368	0\\
369	0\\
370	0\\
371	0\\
372	0\\
373	0\\
374	0\\
375	0\\
376	0\\
377	0\\
378	0\\
379	0\\
380	0\\
381	0\\
382	0\\
383	0\\
384	0\\
385	0\\
386	0\\
387	0\\
388	0\\
389	0\\
390	0\\
391	0\\
392	0\\
393	0\\
394	0\\
395	0\\
396	0\\
397	0\\
398	0\\
399	0\\
400	0\\
401	0\\
402	0\\
403	0\\
404	0\\
405	0\\
406	0\\
407	0\\
408	0\\
409	0\\
410	0\\
411	0\\
412	0\\
413	0\\
414	0\\
415	0\\
416	0\\
417	0\\
418	0\\
419	0\\
420	0\\
421	0\\
422	0\\
423	0\\
424	0\\
425	0\\
426	0\\
427	0\\
428	0\\
429	0\\
430	0\\
431	0\\
432	0\\
433	0\\
434	0\\
435	0\\
436	0\\
437	0\\
438	0\\
439	0\\
440	0\\
441	0\\
442	0\\
443	0\\
444	0\\
445	0\\
446	0\\
447	0\\
448	0\\
449	0\\
450	0\\
451	0\\
452	0\\
453	0\\
454	0\\
455	0\\
456	0\\
457	0\\
458	0\\
459	0\\
460	0\\
461	0\\
462	0\\
463	0\\
464	0\\
465	0\\
466	0\\
467	0\\
468	0\\
469	0\\
470	0\\
471	0\\
472	0\\
473	0\\
474	0\\
475	0\\
476	0\\
477	0\\
478	0\\
479	0\\
480	0\\
481	0\\
482	0\\
483	0\\
484	0\\
485	0\\
486	0\\
487	0\\
488	0\\
489	0\\
490	0\\
491	0\\
492	0\\
493	0\\
494	0\\
495	0\\
496	0\\
497	0\\
498	0\\
499	0\\
500	0\\
501	0\\
502	0\\
503	0\\
504	0\\
505	0\\
506	0\\
507	0\\
508	0\\
509	0\\
510	0\\
511	0\\
512	0\\
513	0\\
514	0\\
515	0\\
516	0\\
517	0\\
518	0\\
519	0\\
520	0\\
521	0\\
522	0\\
523	0\\
524	0\\
525	0\\
526	0\\
527	0\\
528	0\\
529	0\\
530	0\\
531	0\\
532	0\\
533	0\\
534	0\\
535	0\\
536	0\\
537	0\\
538	0\\
539	0\\
540	0\\
541	0\\
542	0\\
543	0\\
544	0\\
545	0\\
546	0\\
547	0\\
548	0\\
549	0\\
550	0\\
551	0\\
552	0\\
553	0\\
554	0\\
555	0\\
556	0\\
557	0\\
558	0\\
559	0\\
560	0\\
561	0\\
562	0\\
563	0\\
564	0\\
565	0\\
566	0\\
567	0\\
568	0\\
569	0\\
570	0\\
571	0\\
572	0\\
573	0.000161696296701996\\
574	0.000393713864803922\\
575	0.000598899566224693\\
576	0.000720797071750311\\
577	0.000841506955469324\\
578	0.000959997374964595\\
579	0.00107594147230983\\
580	0.00118817755523261\\
581	0.00129194223306109\\
582	0.00138617795658753\\
583	0.00147953228162457\\
584	0.00157379580842985\\
585	0.00166431968199408\\
586	0.00175199502013113\\
587	0.00222288556602185\\
588	0.00270274779694502\\
589	0.00311897222604471\\
590	0.00326220003215725\\
591	0.00340447372974604\\
592	0.00353900409992704\\
593	0.00365273120413177\\
594	0.00378238158822432\\
595	0.0039567480325281\\
596	0.00425297862742962\\
597	0.0048700418989439\\
598	0.00632942537858856\\
599	0\\
600	0\\
};
\addplot [color=mycolor17,solid,forget plot]
  table[row sep=crcr]{%
1	0\\
2	0\\
3	0\\
4	0\\
5	0\\
6	0\\
7	0\\
8	0\\
9	0\\
10	0\\
11	0\\
12	0\\
13	0\\
14	0\\
15	0\\
16	0\\
17	0\\
18	0\\
19	0\\
20	0\\
21	0\\
22	0\\
23	0\\
24	0\\
25	0\\
26	0\\
27	0\\
28	0\\
29	0\\
30	0\\
31	0\\
32	0\\
33	0\\
34	0\\
35	0\\
36	0\\
37	0\\
38	0\\
39	0\\
40	0\\
41	0\\
42	0\\
43	0\\
44	0\\
45	0\\
46	0\\
47	0\\
48	0\\
49	0\\
50	0\\
51	0\\
52	0\\
53	0\\
54	0\\
55	0\\
56	0\\
57	0\\
58	0\\
59	0\\
60	0\\
61	0\\
62	0\\
63	0\\
64	0\\
65	0\\
66	0\\
67	0\\
68	0\\
69	0\\
70	0\\
71	0\\
72	0\\
73	0\\
74	0\\
75	0\\
76	0\\
77	0\\
78	0\\
79	0\\
80	0\\
81	0\\
82	0\\
83	0\\
84	0\\
85	0\\
86	0\\
87	0\\
88	0\\
89	0\\
90	0\\
91	0\\
92	0\\
93	0\\
94	0\\
95	0\\
96	0\\
97	0\\
98	0\\
99	0\\
100	0\\
101	0\\
102	0\\
103	0\\
104	0\\
105	0\\
106	0\\
107	0\\
108	0\\
109	0\\
110	0\\
111	0\\
112	0\\
113	0\\
114	0\\
115	0\\
116	0\\
117	0\\
118	0\\
119	0\\
120	0\\
121	0\\
122	0\\
123	0\\
124	0\\
125	0\\
126	0\\
127	0\\
128	0\\
129	0\\
130	0\\
131	0\\
132	0\\
133	0\\
134	0\\
135	0\\
136	0\\
137	0\\
138	0\\
139	0\\
140	0\\
141	0\\
142	0\\
143	0\\
144	0\\
145	0\\
146	0\\
147	0\\
148	0\\
149	0\\
150	0\\
151	0\\
152	0\\
153	0\\
154	0\\
155	0\\
156	0\\
157	0\\
158	0\\
159	0\\
160	0\\
161	0\\
162	0\\
163	0\\
164	0\\
165	0\\
166	0\\
167	0\\
168	0\\
169	0\\
170	0\\
171	0\\
172	0\\
173	0\\
174	0\\
175	0\\
176	0\\
177	0\\
178	0\\
179	0\\
180	0\\
181	0\\
182	0\\
183	0\\
184	0\\
185	0\\
186	0\\
187	0\\
188	0\\
189	0\\
190	0\\
191	0\\
192	0\\
193	0\\
194	0\\
195	0\\
196	0\\
197	0\\
198	0\\
199	0\\
200	0\\
201	0\\
202	0\\
203	0\\
204	0\\
205	0\\
206	0\\
207	0\\
208	0\\
209	0\\
210	0\\
211	0\\
212	0\\
213	0\\
214	0\\
215	0\\
216	0\\
217	0\\
218	0\\
219	0\\
220	0\\
221	0\\
222	0\\
223	0\\
224	0\\
225	0\\
226	0\\
227	0\\
228	0\\
229	0\\
230	0\\
231	0\\
232	0\\
233	0\\
234	0\\
235	0\\
236	0\\
237	0\\
238	0\\
239	0\\
240	0\\
241	0\\
242	0\\
243	0\\
244	0\\
245	0\\
246	0\\
247	0\\
248	0\\
249	0\\
250	0\\
251	0\\
252	0\\
253	0\\
254	0\\
255	0\\
256	0\\
257	0\\
258	0\\
259	0\\
260	0\\
261	0\\
262	0\\
263	0\\
264	0\\
265	0\\
266	0\\
267	0\\
268	0\\
269	0\\
270	0\\
271	0\\
272	0\\
273	0\\
274	0\\
275	0\\
276	0\\
277	0\\
278	0\\
279	0\\
280	0\\
281	0\\
282	0\\
283	0\\
284	0\\
285	0\\
286	0\\
287	0\\
288	0\\
289	0\\
290	0\\
291	0\\
292	0\\
293	0\\
294	0\\
295	0\\
296	0\\
297	0\\
298	0\\
299	0\\
300	0\\
301	0\\
302	0\\
303	0\\
304	0\\
305	0\\
306	0\\
307	0\\
308	0\\
309	0\\
310	0\\
311	0\\
312	0\\
313	0\\
314	0\\
315	0\\
316	0\\
317	0\\
318	0\\
319	0\\
320	0\\
321	0\\
322	0\\
323	0\\
324	0\\
325	0\\
326	0\\
327	0\\
328	0\\
329	0\\
330	0\\
331	0\\
332	0\\
333	0\\
334	0\\
335	0\\
336	0\\
337	0\\
338	0\\
339	0\\
340	0\\
341	0\\
342	0\\
343	0\\
344	0\\
345	0\\
346	0\\
347	0\\
348	0\\
349	0\\
350	0\\
351	0\\
352	0\\
353	0\\
354	0\\
355	0\\
356	0\\
357	0\\
358	0\\
359	0\\
360	0\\
361	0\\
362	0\\
363	0\\
364	0\\
365	0\\
366	0\\
367	0\\
368	0\\
369	0\\
370	0\\
371	0\\
372	0\\
373	0\\
374	0\\
375	0\\
376	0\\
377	0\\
378	0\\
379	0\\
380	0\\
381	0\\
382	0\\
383	0\\
384	0\\
385	0\\
386	0\\
387	0\\
388	0\\
389	0\\
390	0\\
391	0\\
392	0\\
393	0\\
394	0\\
395	0\\
396	0\\
397	0\\
398	0\\
399	0\\
400	0\\
401	0\\
402	0\\
403	0\\
404	0\\
405	0\\
406	0\\
407	0\\
408	0\\
409	0\\
410	0\\
411	0\\
412	0\\
413	0\\
414	0\\
415	0\\
416	0\\
417	0\\
418	0\\
419	0\\
420	0\\
421	0\\
422	0\\
423	0\\
424	0\\
425	0\\
426	0\\
427	0\\
428	0\\
429	0\\
430	0\\
431	0\\
432	0\\
433	0\\
434	0\\
435	0\\
436	0\\
437	0\\
438	0\\
439	0\\
440	0\\
441	0\\
442	0\\
443	0\\
444	0\\
445	0\\
446	0\\
447	0\\
448	0\\
449	0\\
450	0\\
451	0\\
452	0\\
453	0\\
454	0\\
455	0\\
456	0\\
457	0\\
458	0\\
459	0\\
460	0\\
461	0\\
462	0\\
463	0\\
464	0\\
465	0\\
466	0\\
467	0\\
468	0\\
469	0\\
470	0\\
471	0\\
472	0\\
473	0\\
474	0\\
475	0\\
476	0\\
477	0\\
478	0\\
479	0\\
480	0\\
481	0\\
482	0\\
483	0\\
484	0\\
485	0\\
486	0\\
487	0\\
488	0\\
489	0\\
490	0\\
491	0\\
492	0\\
493	0\\
494	0\\
495	0\\
496	0\\
497	0\\
498	0\\
499	0\\
500	0\\
501	0\\
502	0\\
503	0\\
504	0\\
505	0\\
506	0\\
507	0\\
508	0\\
509	0\\
510	0\\
511	0\\
512	0\\
513	0\\
514	0\\
515	0\\
516	0\\
517	0\\
518	0\\
519	0\\
520	0\\
521	0\\
522	0\\
523	0\\
524	0\\
525	0\\
526	0\\
527	0\\
528	0\\
529	0\\
530	0\\
531	0\\
532	0\\
533	0\\
534	0\\
535	0\\
536	0\\
537	0\\
538	0\\
539	0\\
540	0\\
541	0\\
542	0\\
543	0\\
544	0\\
545	0\\
546	0\\
547	0\\
548	0\\
549	0\\
550	0\\
551	0\\
552	0\\
553	0\\
554	0\\
555	0\\
556	0\\
557	0\\
558	0\\
559	0\\
560	0\\
561	0\\
562	0\\
563	0\\
564	0\\
565	0\\
566	0\\
567	5.88949037576846e-05\\
568	0.000278923522351577\\
569	0.000465215789347868\\
570	0.000573095561531768\\
571	0.000679064755436726\\
572	0.000781407100120683\\
573	0.000879327214528265\\
574	0.000971216243397552\\
575	0.00105576846491719\\
576	0.00113450507723386\\
577	0.00121198502541605\\
578	0.00128801880280719\\
579	0.00136213507180008\\
580	0.00143518170906471\\
581	0.00150992377660234\\
582	0.00158312346660404\\
583	0.00165386180226987\\
584	0.00192128218169381\\
585	0.00239891244873237\\
586	0.00288945262252655\\
587	0.00302337868377814\\
588	0.00315170319741699\\
589	0.00326939440428176\\
590	0.00336308691806496\\
591	0.00345791823244592\\
592	0.00355546598078343\\
593	0.00366084952809565\\
594	0.00378520541414672\\
595	0.00395742119285503\\
596	0.00425297862742962\\
597	0.0048700418989439\\
598	0.00632942537858856\\
599	0\\
600	0\\
};
\addplot [color=mycolor18,solid,forget plot]
  table[row sep=crcr]{%
1	0\\
2	0\\
3	0\\
4	0\\
5	0\\
6	0\\
7	0\\
8	0\\
9	0\\
10	0\\
11	0\\
12	0\\
13	0\\
14	0\\
15	0\\
16	0\\
17	0\\
18	0\\
19	0\\
20	0\\
21	0\\
22	0\\
23	0\\
24	0\\
25	0\\
26	0\\
27	0\\
28	0\\
29	0\\
30	0\\
31	0\\
32	0\\
33	0\\
34	0\\
35	0\\
36	0\\
37	0\\
38	0\\
39	0\\
40	0\\
41	0\\
42	0\\
43	0\\
44	0\\
45	0\\
46	0\\
47	0\\
48	0\\
49	0\\
50	0\\
51	0\\
52	0\\
53	0\\
54	0\\
55	0\\
56	0\\
57	0\\
58	0\\
59	0\\
60	0\\
61	0\\
62	0\\
63	0\\
64	0\\
65	0\\
66	0\\
67	0\\
68	0\\
69	0\\
70	0\\
71	0\\
72	0\\
73	0\\
74	0\\
75	0\\
76	0\\
77	0\\
78	0\\
79	0\\
80	0\\
81	0\\
82	0\\
83	0\\
84	0\\
85	0\\
86	0\\
87	0\\
88	0\\
89	0\\
90	0\\
91	0\\
92	0\\
93	0\\
94	0\\
95	0\\
96	0\\
97	0\\
98	0\\
99	0\\
100	0\\
101	0\\
102	0\\
103	0\\
104	0\\
105	0\\
106	0\\
107	0\\
108	0\\
109	0\\
110	0\\
111	0\\
112	0\\
113	0\\
114	0\\
115	0\\
116	0\\
117	0\\
118	0\\
119	0\\
120	0\\
121	0\\
122	0\\
123	0\\
124	0\\
125	0\\
126	0\\
127	0\\
128	0\\
129	0\\
130	0\\
131	0\\
132	0\\
133	0\\
134	0\\
135	0\\
136	0\\
137	0\\
138	0\\
139	0\\
140	0\\
141	0\\
142	0\\
143	0\\
144	0\\
145	0\\
146	0\\
147	0\\
148	0\\
149	0\\
150	0\\
151	0\\
152	0\\
153	0\\
154	0\\
155	0\\
156	0\\
157	0\\
158	0\\
159	0\\
160	0\\
161	0\\
162	0\\
163	0\\
164	0\\
165	0\\
166	0\\
167	0\\
168	0\\
169	0\\
170	0\\
171	0\\
172	0\\
173	0\\
174	0\\
175	0\\
176	0\\
177	0\\
178	0\\
179	0\\
180	0\\
181	0\\
182	0\\
183	0\\
184	0\\
185	0\\
186	0\\
187	0\\
188	0\\
189	0\\
190	0\\
191	0\\
192	0\\
193	0\\
194	0\\
195	0\\
196	0\\
197	0\\
198	0\\
199	0\\
200	0\\
201	0\\
202	0\\
203	0\\
204	0\\
205	0\\
206	0\\
207	0\\
208	0\\
209	0\\
210	0\\
211	0\\
212	0\\
213	0\\
214	0\\
215	0\\
216	0\\
217	0\\
218	0\\
219	0\\
220	0\\
221	0\\
222	0\\
223	0\\
224	0\\
225	0\\
226	0\\
227	0\\
228	0\\
229	0\\
230	0\\
231	0\\
232	0\\
233	0\\
234	0\\
235	0\\
236	0\\
237	0\\
238	0\\
239	0\\
240	0\\
241	0\\
242	0\\
243	0\\
244	0\\
245	0\\
246	0\\
247	0\\
248	0\\
249	0\\
250	0\\
251	0\\
252	0\\
253	0\\
254	0\\
255	0\\
256	0\\
257	0\\
258	0\\
259	0\\
260	0\\
261	0\\
262	0\\
263	0\\
264	0\\
265	0\\
266	0\\
267	0\\
268	0\\
269	0\\
270	0\\
271	0\\
272	0\\
273	0\\
274	0\\
275	0\\
276	0\\
277	0\\
278	0\\
279	0\\
280	0\\
281	0\\
282	0\\
283	0\\
284	0\\
285	0\\
286	0\\
287	0\\
288	0\\
289	0\\
290	0\\
291	0\\
292	0\\
293	0\\
294	0\\
295	0\\
296	0\\
297	0\\
298	0\\
299	0\\
300	0\\
301	0\\
302	0\\
303	0\\
304	0\\
305	0\\
306	0\\
307	0\\
308	0\\
309	0\\
310	0\\
311	0\\
312	0\\
313	0\\
314	0\\
315	0\\
316	0\\
317	0\\
318	0\\
319	0\\
320	0\\
321	0\\
322	0\\
323	0\\
324	0\\
325	0\\
326	0\\
327	0\\
328	0\\
329	0\\
330	0\\
331	0\\
332	0\\
333	0\\
334	0\\
335	0\\
336	0\\
337	0\\
338	0\\
339	0\\
340	0\\
341	0\\
342	0\\
343	0\\
344	0\\
345	0\\
346	0\\
347	0\\
348	0\\
349	0\\
350	0\\
351	0\\
352	0\\
353	0\\
354	0\\
355	0\\
356	0\\
357	0\\
358	0\\
359	0\\
360	0\\
361	0\\
362	0\\
363	0\\
364	0\\
365	0\\
366	0\\
367	0\\
368	0\\
369	0\\
370	0\\
371	0\\
372	0\\
373	0\\
374	0\\
375	0\\
376	0\\
377	0\\
378	0\\
379	0\\
380	0\\
381	0\\
382	0\\
383	0\\
384	0\\
385	0\\
386	0\\
387	0\\
388	0\\
389	0\\
390	0\\
391	0\\
392	0\\
393	0\\
394	0\\
395	0\\
396	0\\
397	0\\
398	0\\
399	0\\
400	0\\
401	0\\
402	0\\
403	0\\
404	0\\
405	0\\
406	0\\
407	0\\
408	0\\
409	0\\
410	0\\
411	0\\
412	0\\
413	0\\
414	0\\
415	0\\
416	0\\
417	0\\
418	0\\
419	0\\
420	0\\
421	0\\
422	0\\
423	0\\
424	0\\
425	0\\
426	0\\
427	0\\
428	0\\
429	0\\
430	0\\
431	0\\
432	0\\
433	0\\
434	0\\
435	0\\
436	0\\
437	0\\
438	0\\
439	0\\
440	0\\
441	0\\
442	0\\
443	0\\
444	0\\
445	0\\
446	0\\
447	0\\
448	0\\
449	0\\
450	0\\
451	0\\
452	0\\
453	0\\
454	0\\
455	0\\
456	0\\
457	0\\
458	0\\
459	0\\
460	0\\
461	0\\
462	0\\
463	0\\
464	0\\
465	0\\
466	0\\
467	0\\
468	0\\
469	0\\
470	0\\
471	0\\
472	0\\
473	0\\
474	0\\
475	0\\
476	0\\
477	0\\
478	0\\
479	0\\
480	0\\
481	0\\
482	0\\
483	0\\
484	0\\
485	0\\
486	0\\
487	0\\
488	0\\
489	0\\
490	0\\
491	0\\
492	0\\
493	0\\
494	0\\
495	0\\
496	0\\
497	0\\
498	0\\
499	0\\
500	0\\
501	0\\
502	0\\
503	0\\
504	0\\
505	0\\
506	0\\
507	0\\
508	0\\
509	0\\
510	0\\
511	0\\
512	0\\
513	0\\
514	0\\
515	0\\
516	0\\
517	0\\
518	0\\
519	0\\
520	0\\
521	0\\
522	0\\
523	0\\
524	0\\
525	0\\
526	0\\
527	0\\
528	0\\
529	0\\
530	0\\
531	0\\
532	0\\
533	0\\
534	0\\
535	0\\
536	0\\
537	0\\
538	0\\
539	0\\
540	0\\
541	0\\
542	0\\
543	0\\
544	0\\
545	0\\
546	0\\
547	0\\
548	0\\
549	0\\
550	0\\
551	0\\
552	0\\
553	0\\
554	0\\
555	0\\
556	0\\
557	0\\
558	0\\
559	0\\
560	0\\
561	0\\
562	9.90403746872915e-05\\
563	0.000304968769744545\\
564	0.000402367207070976\\
565	0.000497931023480415\\
566	0.000590494868905439\\
567	0.000678077044756074\\
568	0.000758685659673696\\
569	0.000830189121630715\\
570	0.000895650850952982\\
571	0.000961984822474034\\
572	0.0010278398051382\\
573	0.00109225696560847\\
574	0.00115548679208059\\
575	0.00121672171450071\\
576	0.00127705683954306\\
577	0.00133948023067548\\
578	0.00140501315387362\\
579	0.00146934333816017\\
580	0.00153055864356594\\
581	0.00158857538063284\\
582	0.00197419871052425\\
583	0.00246466618649465\\
584	0.00277686403795953\\
585	0.00289976434281973\\
586	0.00301736053854785\\
587	0.00310424045129586\\
588	0.00319120167814864\\
589	0.00327881165853131\\
590	0.00336825554262852\\
591	0.00346025420288071\\
592	0.00355649290755294\\
593	0.00366118457402185\\
594	0.00378527046594782\\
595	0.00395742119285503\\
596	0.00425297862742962\\
597	0.0048700418989439\\
598	0.00632942537858856\\
599	0\\
600	0\\
};
\addplot [color=red!25!mycolor17,solid,forget plot]
  table[row sep=crcr]{%
1	0\\
2	0\\
3	0\\
4	0\\
5	0\\
6	0\\
7	0\\
8	0\\
9	0\\
10	0\\
11	0\\
12	0\\
13	0\\
14	0\\
15	0\\
16	0\\
17	0\\
18	0\\
19	0\\
20	0\\
21	0\\
22	0\\
23	0\\
24	0\\
25	0\\
26	0\\
27	0\\
28	0\\
29	0\\
30	0\\
31	0\\
32	0\\
33	0\\
34	0\\
35	0\\
36	0\\
37	0\\
38	0\\
39	0\\
40	0\\
41	0\\
42	0\\
43	0\\
44	0\\
45	0\\
46	0\\
47	0\\
48	0\\
49	0\\
50	0\\
51	0\\
52	0\\
53	0\\
54	0\\
55	0\\
56	0\\
57	0\\
58	0\\
59	0\\
60	0\\
61	0\\
62	0\\
63	0\\
64	0\\
65	0\\
66	0\\
67	0\\
68	0\\
69	0\\
70	0\\
71	0\\
72	0\\
73	0\\
74	0\\
75	0\\
76	0\\
77	0\\
78	0\\
79	0\\
80	0\\
81	0\\
82	0\\
83	0\\
84	0\\
85	0\\
86	0\\
87	0\\
88	0\\
89	0\\
90	0\\
91	0\\
92	0\\
93	0\\
94	0\\
95	0\\
96	0\\
97	0\\
98	0\\
99	0\\
100	0\\
101	0\\
102	0\\
103	0\\
104	0\\
105	0\\
106	0\\
107	0\\
108	0\\
109	0\\
110	0\\
111	0\\
112	0\\
113	0\\
114	0\\
115	0\\
116	0\\
117	0\\
118	0\\
119	0\\
120	0\\
121	0\\
122	0\\
123	0\\
124	0\\
125	0\\
126	0\\
127	0\\
128	0\\
129	0\\
130	0\\
131	0\\
132	0\\
133	0\\
134	0\\
135	0\\
136	0\\
137	0\\
138	0\\
139	0\\
140	0\\
141	0\\
142	0\\
143	0\\
144	0\\
145	0\\
146	0\\
147	0\\
148	0\\
149	0\\
150	0\\
151	0\\
152	0\\
153	0\\
154	0\\
155	0\\
156	0\\
157	0\\
158	0\\
159	0\\
160	0\\
161	0\\
162	0\\
163	0\\
164	0\\
165	0\\
166	0\\
167	0\\
168	0\\
169	0\\
170	0\\
171	0\\
172	0\\
173	0\\
174	0\\
175	0\\
176	0\\
177	0\\
178	0\\
179	0\\
180	0\\
181	0\\
182	0\\
183	0\\
184	0\\
185	0\\
186	0\\
187	0\\
188	0\\
189	0\\
190	0\\
191	0\\
192	0\\
193	0\\
194	0\\
195	0\\
196	0\\
197	0\\
198	0\\
199	0\\
200	0\\
201	0\\
202	0\\
203	0\\
204	0\\
205	0\\
206	0\\
207	0\\
208	0\\
209	0\\
210	0\\
211	0\\
212	0\\
213	0\\
214	0\\
215	0\\
216	0\\
217	0\\
218	0\\
219	0\\
220	0\\
221	0\\
222	0\\
223	0\\
224	0\\
225	0\\
226	0\\
227	0\\
228	0\\
229	0\\
230	0\\
231	0\\
232	0\\
233	0\\
234	0\\
235	0\\
236	0\\
237	0\\
238	0\\
239	0\\
240	0\\
241	0\\
242	0\\
243	0\\
244	0\\
245	0\\
246	0\\
247	0\\
248	0\\
249	0\\
250	0\\
251	0\\
252	0\\
253	0\\
254	0\\
255	0\\
256	0\\
257	0\\
258	0\\
259	0\\
260	0\\
261	0\\
262	0\\
263	0\\
264	0\\
265	0\\
266	0\\
267	0\\
268	0\\
269	0\\
270	0\\
271	0\\
272	0\\
273	0\\
274	0\\
275	0\\
276	0\\
277	0\\
278	0\\
279	0\\
280	0\\
281	0\\
282	0\\
283	0\\
284	0\\
285	0\\
286	0\\
287	0\\
288	0\\
289	0\\
290	0\\
291	0\\
292	0\\
293	0\\
294	0\\
295	0\\
296	0\\
297	0\\
298	0\\
299	0\\
300	0\\
301	0\\
302	0\\
303	0\\
304	0\\
305	0\\
306	0\\
307	0\\
308	0\\
309	0\\
310	0\\
311	0\\
312	0\\
313	0\\
314	0\\
315	0\\
316	0\\
317	0\\
318	0\\
319	0\\
320	0\\
321	0\\
322	0\\
323	0\\
324	0\\
325	0\\
326	0\\
327	0\\
328	0\\
329	0\\
330	0\\
331	0\\
332	0\\
333	0\\
334	0\\
335	0\\
336	0\\
337	0\\
338	0\\
339	0\\
340	0\\
341	0\\
342	0\\
343	0\\
344	0\\
345	0\\
346	0\\
347	0\\
348	0\\
349	0\\
350	0\\
351	0\\
352	0\\
353	0\\
354	0\\
355	0\\
356	0\\
357	0\\
358	0\\
359	0\\
360	0\\
361	0\\
362	0\\
363	0\\
364	0\\
365	0\\
366	0\\
367	0\\
368	0\\
369	0\\
370	0\\
371	0\\
372	0\\
373	0\\
374	0\\
375	0\\
376	0\\
377	0\\
378	0\\
379	0\\
380	0\\
381	0\\
382	0\\
383	0\\
384	0\\
385	0\\
386	0\\
387	0\\
388	0\\
389	0\\
390	0\\
391	0\\
392	0\\
393	0\\
394	0\\
395	0\\
396	0\\
397	0\\
398	0\\
399	0\\
400	0\\
401	0\\
402	0\\
403	0\\
404	0\\
405	0\\
406	0\\
407	0\\
408	0\\
409	0\\
410	0\\
411	0\\
412	0\\
413	0\\
414	0\\
415	0\\
416	0\\
417	0\\
418	0\\
419	0\\
420	0\\
421	0\\
422	0\\
423	0\\
424	0\\
425	0\\
426	0\\
427	0\\
428	0\\
429	0\\
430	0\\
431	0\\
432	0\\
433	0\\
434	0\\
435	0\\
436	0\\
437	0\\
438	0\\
439	0\\
440	0\\
441	0\\
442	0\\
443	0\\
444	0\\
445	0\\
446	0\\
447	0\\
448	0\\
449	0\\
450	0\\
451	0\\
452	0\\
453	0\\
454	0\\
455	0\\
456	0\\
457	0\\
458	0\\
459	0\\
460	0\\
461	0\\
462	0\\
463	0\\
464	0\\
465	0\\
466	0\\
467	0\\
468	0\\
469	0\\
470	0\\
471	0\\
472	0\\
473	0\\
474	0\\
475	0\\
476	0\\
477	0\\
478	0\\
479	0\\
480	0\\
481	0\\
482	0\\
483	0\\
484	0\\
485	0\\
486	0\\
487	0\\
488	0\\
489	0\\
490	0\\
491	0\\
492	0\\
493	0\\
494	0\\
495	0\\
496	0\\
497	0\\
498	0\\
499	0\\
500	0\\
501	0\\
502	0\\
503	0\\
504	0\\
505	0\\
506	0\\
507	0\\
508	0\\
509	0\\
510	0\\
511	0\\
512	0\\
513	0\\
514	0\\
515	0\\
516	0\\
517	0\\
518	0\\
519	0\\
520	0\\
521	0\\
522	0\\
523	0\\
524	0\\
525	0\\
526	0\\
527	0\\
528	0\\
529	0\\
530	0\\
531	0\\
532	0\\
533	0\\
534	0\\
535	0\\
536	0\\
537	0\\
538	0\\
539	0\\
540	0\\
541	0\\
542	0\\
543	0\\
544	0\\
545	0\\
546	0\\
547	0\\
548	0\\
549	0\\
550	0\\
551	0\\
552	0\\
553	0\\
554	0\\
555	0\\
556	0\\
557	8.01734108211468e-05\\
558	0.000223371265806801\\
559	0.000311803468712962\\
560	0.000397274215465461\\
561	0.000478427144538355\\
562	0.000553370163734247\\
563	0.0006186795095291\\
564	0.000675713681809035\\
565	0.00073191039909959\\
566	0.000787276194635121\\
567	0.000843166680497421\\
568	0.000899372565017849\\
569	0.000954897785455491\\
570	0.00101018526904893\\
571	0.0010637562962866\\
572	0.00111682448780514\\
573	0.00117048000700388\\
574	0.00122509244478267\\
575	0.00128514626774489\\
576	0.00134430788764564\\
577	0.00140097157268809\\
578	0.00145403611431188\\
579	0.00150752687252756\\
580	0.00193194802301185\\
581	0.00243680986819744\\
582	0.00264404234620892\\
583	0.00276021838521539\\
584	0.00285857309192273\\
585	0.00294082185579374\\
586	0.00302342706147123\\
587	0.00310731224695134\\
588	0.00319261476955509\\
589	0.00327956317486525\\
590	0.00336860380042371\\
591	0.00346039193281902\\
592	0.00355653331527142\\
593	0.00366119131364763\\
594	0.00378527046594782\\
595	0.00395742119285503\\
596	0.00425297862742962\\
597	0.0048700418989439\\
598	0.00632942537858856\\
599	0\\
600	0\\
};
\addplot [color=mycolor19,solid,forget plot]
  table[row sep=crcr]{%
1	0\\
2	0\\
3	0\\
4	0\\
5	0\\
6	0\\
7	0\\
8	0\\
9	0\\
10	0\\
11	0\\
12	0\\
13	0\\
14	0\\
15	0\\
16	0\\
17	0\\
18	0\\
19	0\\
20	0\\
21	0\\
22	0\\
23	0\\
24	0\\
25	0\\
26	0\\
27	0\\
28	0\\
29	0\\
30	0\\
31	0\\
32	0\\
33	0\\
34	0\\
35	0\\
36	0\\
37	0\\
38	0\\
39	0\\
40	0\\
41	0\\
42	0\\
43	0\\
44	0\\
45	0\\
46	0\\
47	0\\
48	0\\
49	0\\
50	0\\
51	0\\
52	0\\
53	0\\
54	0\\
55	0\\
56	0\\
57	0\\
58	0\\
59	0\\
60	0\\
61	0\\
62	0\\
63	0\\
64	0\\
65	0\\
66	0\\
67	0\\
68	0\\
69	0\\
70	0\\
71	0\\
72	0\\
73	0\\
74	0\\
75	0\\
76	0\\
77	0\\
78	0\\
79	0\\
80	0\\
81	0\\
82	0\\
83	0\\
84	0\\
85	0\\
86	0\\
87	0\\
88	0\\
89	0\\
90	0\\
91	0\\
92	0\\
93	0\\
94	0\\
95	0\\
96	0\\
97	0\\
98	0\\
99	0\\
100	0\\
101	0\\
102	0\\
103	0\\
104	0\\
105	0\\
106	0\\
107	0\\
108	0\\
109	0\\
110	0\\
111	0\\
112	0\\
113	0\\
114	0\\
115	0\\
116	0\\
117	0\\
118	0\\
119	0\\
120	0\\
121	0\\
122	0\\
123	0\\
124	0\\
125	0\\
126	0\\
127	0\\
128	0\\
129	0\\
130	0\\
131	0\\
132	0\\
133	0\\
134	0\\
135	0\\
136	0\\
137	0\\
138	0\\
139	0\\
140	0\\
141	0\\
142	0\\
143	0\\
144	0\\
145	0\\
146	0\\
147	0\\
148	0\\
149	0\\
150	0\\
151	0\\
152	0\\
153	0\\
154	0\\
155	0\\
156	0\\
157	0\\
158	0\\
159	0\\
160	0\\
161	0\\
162	0\\
163	0\\
164	0\\
165	0\\
166	0\\
167	0\\
168	0\\
169	0\\
170	0\\
171	0\\
172	0\\
173	0\\
174	0\\
175	0\\
176	0\\
177	0\\
178	0\\
179	0\\
180	0\\
181	0\\
182	0\\
183	0\\
184	0\\
185	0\\
186	0\\
187	0\\
188	0\\
189	0\\
190	0\\
191	0\\
192	0\\
193	0\\
194	0\\
195	0\\
196	0\\
197	0\\
198	0\\
199	0\\
200	0\\
201	0\\
202	0\\
203	0\\
204	0\\
205	0\\
206	0\\
207	0\\
208	0\\
209	0\\
210	0\\
211	0\\
212	0\\
213	0\\
214	0\\
215	0\\
216	0\\
217	0\\
218	0\\
219	0\\
220	0\\
221	0\\
222	0\\
223	0\\
224	0\\
225	0\\
226	0\\
227	0\\
228	0\\
229	0\\
230	0\\
231	0\\
232	0\\
233	0\\
234	0\\
235	0\\
236	0\\
237	0\\
238	0\\
239	0\\
240	0\\
241	0\\
242	0\\
243	0\\
244	0\\
245	0\\
246	0\\
247	0\\
248	0\\
249	0\\
250	0\\
251	0\\
252	0\\
253	0\\
254	0\\
255	0\\
256	0\\
257	0\\
258	0\\
259	0\\
260	0\\
261	0\\
262	0\\
263	0\\
264	0\\
265	0\\
266	0\\
267	0\\
268	0\\
269	0\\
270	0\\
271	0\\
272	0\\
273	0\\
274	0\\
275	0\\
276	0\\
277	0\\
278	0\\
279	0\\
280	0\\
281	0\\
282	0\\
283	0\\
284	0\\
285	0\\
286	0\\
287	0\\
288	0\\
289	0\\
290	0\\
291	0\\
292	0\\
293	0\\
294	0\\
295	0\\
296	0\\
297	0\\
298	0\\
299	0\\
300	0\\
301	0\\
302	0\\
303	0\\
304	0\\
305	0\\
306	0\\
307	0\\
308	0\\
309	0\\
310	0\\
311	0\\
312	0\\
313	0\\
314	0\\
315	0\\
316	0\\
317	0\\
318	0\\
319	0\\
320	0\\
321	0\\
322	0\\
323	0\\
324	0\\
325	0\\
326	0\\
327	0\\
328	0\\
329	0\\
330	0\\
331	0\\
332	0\\
333	0\\
334	0\\
335	0\\
336	0\\
337	0\\
338	0\\
339	0\\
340	0\\
341	0\\
342	0\\
343	0\\
344	0\\
345	0\\
346	0\\
347	0\\
348	0\\
349	0\\
350	0\\
351	0\\
352	0\\
353	0\\
354	0\\
355	0\\
356	0\\
357	0\\
358	0\\
359	0\\
360	0\\
361	0\\
362	0\\
363	0\\
364	0\\
365	0\\
366	0\\
367	0\\
368	0\\
369	0\\
370	0\\
371	0\\
372	0\\
373	0\\
374	0\\
375	0\\
376	0\\
377	0\\
378	0\\
379	0\\
380	0\\
381	0\\
382	0\\
383	0\\
384	0\\
385	0\\
386	0\\
387	0\\
388	0\\
389	0\\
390	0\\
391	0\\
392	0\\
393	0\\
394	0\\
395	0\\
396	0\\
397	0\\
398	0\\
399	0\\
400	0\\
401	0\\
402	0\\
403	0\\
404	0\\
405	0\\
406	0\\
407	0\\
408	0\\
409	0\\
410	0\\
411	0\\
412	0\\
413	0\\
414	0\\
415	0\\
416	0\\
417	0\\
418	0\\
419	0\\
420	0\\
421	0\\
422	0\\
423	0\\
424	0\\
425	0\\
426	0\\
427	0\\
428	0\\
429	0\\
430	0\\
431	0\\
432	0\\
433	0\\
434	0\\
435	0\\
436	0\\
437	0\\
438	0\\
439	0\\
440	0\\
441	0\\
442	0\\
443	0\\
444	0\\
445	0\\
446	0\\
447	0\\
448	0\\
449	0\\
450	0\\
451	0\\
452	0\\
453	0\\
454	0\\
455	0\\
456	0\\
457	0\\
458	0\\
459	0\\
460	0\\
461	0\\
462	0\\
463	0\\
464	0\\
465	0\\
466	0\\
467	0\\
468	0\\
469	0\\
470	0\\
471	0\\
472	0\\
473	0\\
474	0\\
475	0\\
476	0\\
477	0\\
478	0\\
479	0\\
480	0\\
481	0\\
482	0\\
483	0\\
484	0\\
485	0\\
486	0\\
487	0\\
488	0\\
489	0\\
490	0\\
491	0\\
492	0\\
493	0\\
494	0\\
495	0\\
496	0\\
497	0\\
498	0\\
499	0\\
500	0\\
501	0\\
502	0\\
503	0\\
504	0\\
505	0\\
506	0\\
507	0\\
508	0\\
509	0\\
510	0\\
511	0\\
512	0\\
513	0\\
514	0\\
515	0\\
516	0\\
517	0\\
518	0\\
519	0\\
520	0\\
521	0\\
522	0\\
523	0\\
524	0\\
525	0\\
526	0\\
527	0\\
528	0\\
529	0\\
530	0\\
531	0\\
532	0\\
533	0\\
534	0\\
535	0\\
536	0\\
537	0\\
538	0\\
539	0\\
540	0\\
541	0\\
542	0\\
543	0\\
544	0\\
545	0\\
546	0\\
547	0\\
548	0\\
549	0\\
550	0\\
551	0\\
552	1.50670897009176e-05\\
553	0.000124382787751889\\
554	0.00020503601022344\\
555	0.000281906524719583\\
556	0.000353475266029824\\
557	0.000416928686445217\\
558	0.000471622760274698\\
559	0.000521843264876268\\
560	0.000570759546781793\\
561	0.000618739159396117\\
562	0.000666121145496063\\
563	0.000713768966029745\\
564	0.000763815747326926\\
565	0.0008137163277131\\
566	0.000863511816317939\\
567	0.000912037857953922\\
568	0.000959785906419871\\
569	0.00100821347014938\\
570	0.00105737215001796\\
571	0.00110730055108194\\
572	0.00116093038778452\\
573	0.00121697978948583\\
574	0.00127198613892506\\
575	0.00132150660081605\\
576	0.00137146061891735\\
577	0.00142187466526321\\
578	0.00180758151954677\\
579	0.00232459949440059\\
580	0.00250414369719064\\
581	0.00261607079037986\\
582	0.00270221271091265\\
583	0.00278105910457229\\
584	0.00286069531223358\\
585	0.00294160817599962\\
586	0.00302387110985324\\
587	0.00310754666812213\\
588	0.00319273161070454\\
589	0.00327961396220161\\
590	0.00336862207624017\\
591	0.00346039677087353\\
592	0.00355653402929194\\
593	0.00366119131364763\\
594	0.00378527046594783\\
595	0.00395742119285503\\
596	0.00425297862742962\\
597	0.0048700418989439\\
598	0.00632942537858856\\
599	0\\
600	0\\
};
\addplot [color=red!50!mycolor17,solid,forget plot]
  table[row sep=crcr]{%
1	0\\
2	0\\
3	0\\
4	0\\
5	0\\
6	0\\
7	0\\
8	0\\
9	0\\
10	0\\
11	0\\
12	0\\
13	0\\
14	0\\
15	0\\
16	0\\
17	0\\
18	0\\
19	0\\
20	0\\
21	0\\
22	0\\
23	0\\
24	0\\
25	0\\
26	0\\
27	0\\
28	0\\
29	0\\
30	0\\
31	0\\
32	0\\
33	0\\
34	0\\
35	0\\
36	0\\
37	0\\
38	0\\
39	0\\
40	0\\
41	0\\
42	0\\
43	0\\
44	0\\
45	0\\
46	0\\
47	0\\
48	0\\
49	0\\
50	0\\
51	0\\
52	0\\
53	0\\
54	0\\
55	0\\
56	0\\
57	0\\
58	0\\
59	0\\
60	0\\
61	0\\
62	0\\
63	0\\
64	0\\
65	0\\
66	0\\
67	0\\
68	0\\
69	0\\
70	0\\
71	0\\
72	0\\
73	0\\
74	0\\
75	0\\
76	0\\
77	0\\
78	0\\
79	0\\
80	0\\
81	0\\
82	0\\
83	0\\
84	0\\
85	0\\
86	0\\
87	0\\
88	0\\
89	0\\
90	0\\
91	0\\
92	0\\
93	0\\
94	0\\
95	0\\
96	0\\
97	0\\
98	0\\
99	0\\
100	0\\
101	0\\
102	0\\
103	0\\
104	0\\
105	0\\
106	0\\
107	0\\
108	0\\
109	0\\
110	0\\
111	0\\
112	0\\
113	0\\
114	0\\
115	0\\
116	0\\
117	0\\
118	0\\
119	0\\
120	0\\
121	0\\
122	0\\
123	0\\
124	0\\
125	0\\
126	0\\
127	0\\
128	0\\
129	0\\
130	0\\
131	0\\
132	0\\
133	0\\
134	0\\
135	0\\
136	0\\
137	0\\
138	0\\
139	0\\
140	0\\
141	0\\
142	0\\
143	0\\
144	0\\
145	0\\
146	0\\
147	0\\
148	0\\
149	0\\
150	0\\
151	0\\
152	0\\
153	0\\
154	0\\
155	0\\
156	0\\
157	0\\
158	0\\
159	0\\
160	0\\
161	0\\
162	0\\
163	0\\
164	0\\
165	0\\
166	0\\
167	0\\
168	0\\
169	0\\
170	0\\
171	0\\
172	0\\
173	0\\
174	0\\
175	0\\
176	0\\
177	0\\
178	0\\
179	0\\
180	0\\
181	0\\
182	0\\
183	0\\
184	0\\
185	0\\
186	0\\
187	0\\
188	0\\
189	0\\
190	0\\
191	0\\
192	0\\
193	0\\
194	0\\
195	0\\
196	0\\
197	0\\
198	0\\
199	0\\
200	0\\
201	0\\
202	0\\
203	0\\
204	0\\
205	0\\
206	0\\
207	0\\
208	0\\
209	0\\
210	0\\
211	0\\
212	0\\
213	0\\
214	0\\
215	0\\
216	0\\
217	0\\
218	0\\
219	0\\
220	0\\
221	0\\
222	0\\
223	0\\
224	0\\
225	0\\
226	0\\
227	0\\
228	0\\
229	0\\
230	0\\
231	0\\
232	0\\
233	0\\
234	0\\
235	0\\
236	0\\
237	0\\
238	0\\
239	0\\
240	0\\
241	0\\
242	0\\
243	0\\
244	0\\
245	0\\
246	0\\
247	0\\
248	0\\
249	0\\
250	0\\
251	0\\
252	0\\
253	0\\
254	0\\
255	0\\
256	0\\
257	0\\
258	0\\
259	0\\
260	0\\
261	0\\
262	0\\
263	0\\
264	0\\
265	0\\
266	0\\
267	0\\
268	0\\
269	0\\
270	0\\
271	0\\
272	0\\
273	0\\
274	0\\
275	0\\
276	0\\
277	0\\
278	0\\
279	0\\
280	0\\
281	0\\
282	0\\
283	0\\
284	0\\
285	0\\
286	0\\
287	0\\
288	0\\
289	0\\
290	0\\
291	0\\
292	0\\
293	0\\
294	0\\
295	0\\
296	0\\
297	0\\
298	0\\
299	0\\
300	0\\
301	0\\
302	0\\
303	0\\
304	0\\
305	0\\
306	0\\
307	0\\
308	0\\
309	0\\
310	0\\
311	0\\
312	0\\
313	0\\
314	0\\
315	0\\
316	0\\
317	0\\
318	0\\
319	0\\
320	0\\
321	0\\
322	0\\
323	0\\
324	0\\
325	0\\
326	0\\
327	0\\
328	0\\
329	0\\
330	0\\
331	0\\
332	0\\
333	0\\
334	0\\
335	0\\
336	0\\
337	0\\
338	0\\
339	0\\
340	0\\
341	0\\
342	0\\
343	0\\
344	0\\
345	0\\
346	0\\
347	0\\
348	0\\
349	0\\
350	0\\
351	0\\
352	0\\
353	0\\
354	0\\
355	0\\
356	0\\
357	0\\
358	0\\
359	0\\
360	0\\
361	0\\
362	0\\
363	0\\
364	0\\
365	0\\
366	0\\
367	0\\
368	0\\
369	0\\
370	0\\
371	0\\
372	0\\
373	0\\
374	0\\
375	0\\
376	0\\
377	0\\
378	0\\
379	0\\
380	0\\
381	0\\
382	0\\
383	0\\
384	0\\
385	0\\
386	0\\
387	0\\
388	0\\
389	0\\
390	0\\
391	0\\
392	0\\
393	0\\
394	0\\
395	0\\
396	0\\
397	0\\
398	0\\
399	0\\
400	0\\
401	0\\
402	0\\
403	0\\
404	0\\
405	0\\
406	0\\
407	0\\
408	0\\
409	0\\
410	0\\
411	0\\
412	0\\
413	0\\
414	0\\
415	0\\
416	0\\
417	0\\
418	0\\
419	0\\
420	0\\
421	0\\
422	0\\
423	0\\
424	0\\
425	0\\
426	0\\
427	0\\
428	0\\
429	0\\
430	0\\
431	0\\
432	0\\
433	0\\
434	0\\
435	0\\
436	0\\
437	0\\
438	0\\
439	0\\
440	0\\
441	0\\
442	0\\
443	0\\
444	0\\
445	0\\
446	0\\
447	0\\
448	0\\
449	0\\
450	0\\
451	0\\
452	0\\
453	0\\
454	0\\
455	0\\
456	0\\
457	0\\
458	0\\
459	0\\
460	0\\
461	0\\
462	0\\
463	0\\
464	0\\
465	0\\
466	0\\
467	0\\
468	0\\
469	0\\
470	0\\
471	0\\
472	0\\
473	0\\
474	0\\
475	0\\
476	0\\
477	0\\
478	0\\
479	0\\
480	0\\
481	0\\
482	0\\
483	0\\
484	0\\
485	0\\
486	0\\
487	0\\
488	0\\
489	0\\
490	0\\
491	0\\
492	0\\
493	0\\
494	0\\
495	0\\
496	0\\
497	0\\
498	0\\
499	0\\
500	0\\
501	0\\
502	0\\
503	0\\
504	0\\
505	0\\
506	0\\
507	0\\
508	0\\
509	0\\
510	0\\
511	0\\
512	0\\
513	0\\
514	0\\
515	0\\
516	0\\
517	0\\
518	0\\
519	0\\
520	0\\
521	0\\
522	0\\
523	0\\
524	0\\
525	0\\
526	0\\
527	0\\
528	0\\
529	0\\
530	0\\
531	0\\
532	0\\
533	0\\
534	0\\
535	0\\
536	0\\
537	0\\
538	0\\
539	0\\
540	0\\
541	0\\
542	0\\
543	0\\
544	0\\
545	0\\
546	0\\
547	0\\
548	1.64327239660599e-05\\
549	9.08311762122886e-05\\
550	0.000161022938856363\\
551	0.000224806955478363\\
552	0.000280485308906282\\
553	0.000327622000208862\\
554	0.000372012735657423\\
555	0.000415398793616001\\
556	0.000457922393037952\\
557	0.000499893593165752\\
558	0.000541753390275113\\
559	0.000584113654120537\\
560	0.000628623132028616\\
561	0.000674257201963313\\
562	0.000719918679057545\\
563	0.000765234608134834\\
564	0.000808741945395788\\
565	0.000852912506720194\\
566	0.00089779277058568\\
567	0.000943386792557535\\
568	0.000989813512646882\\
569	0.00103725286168404\\
570	0.00109022682293433\\
571	0.00114341698775839\\
572	0.00119302706840499\\
573	0.00124014217818203\\
574	0.00128770080075513\\
575	0.00133565715457315\\
576	0.00160918154831211\\
577	0.00213608599489938\\
578	0.00235964274414307\\
579	0.00246903551743822\\
580	0.00255013683265288\\
581	0.00262610814362027\\
582	0.00270308299221297\\
583	0.00278130768962817\\
584	0.00286082862410346\\
585	0.00294168211648289\\
586	0.00302390908465139\\
587	0.003107564358473\\
588	0.00319273875713431\\
589	0.00327961632359304\\
590	0.00336862264580875\\
591	0.00346039684658503\\
592	0.00355653402929193\\
593	0.00366119131364763\\
594	0.00378527046594782\\
595	0.00395742119285503\\
596	0.00425297862742962\\
597	0.0048700418989439\\
598	0.00632942537858856\\
599	0\\
600	0\\
};
\addplot [color=red!40!mycolor19,solid,forget plot]
  table[row sep=crcr]{%
1	0\\
2	0\\
3	0\\
4	0\\
5	0\\
6	0\\
7	0\\
8	0\\
9	0\\
10	0\\
11	0\\
12	0\\
13	0\\
14	0\\
15	0\\
16	0\\
17	0\\
18	0\\
19	0\\
20	0\\
21	0\\
22	0\\
23	0\\
24	0\\
25	0\\
26	0\\
27	0\\
28	0\\
29	0\\
30	0\\
31	0\\
32	0\\
33	0\\
34	0\\
35	0\\
36	0\\
37	0\\
38	0\\
39	0\\
40	0\\
41	0\\
42	0\\
43	0\\
44	0\\
45	0\\
46	0\\
47	0\\
48	0\\
49	0\\
50	0\\
51	0\\
52	0\\
53	0\\
54	0\\
55	0\\
56	0\\
57	0\\
58	0\\
59	0\\
60	0\\
61	0\\
62	0\\
63	0\\
64	0\\
65	0\\
66	0\\
67	0\\
68	0\\
69	0\\
70	0\\
71	0\\
72	0\\
73	0\\
74	0\\
75	0\\
76	0\\
77	0\\
78	0\\
79	0\\
80	0\\
81	0\\
82	0\\
83	0\\
84	0\\
85	0\\
86	0\\
87	0\\
88	0\\
89	0\\
90	0\\
91	0\\
92	0\\
93	0\\
94	0\\
95	0\\
96	0\\
97	0\\
98	0\\
99	0\\
100	0\\
101	0\\
102	0\\
103	0\\
104	0\\
105	0\\
106	0\\
107	0\\
108	0\\
109	0\\
110	0\\
111	0\\
112	0\\
113	0\\
114	0\\
115	0\\
116	0\\
117	0\\
118	0\\
119	0\\
120	0\\
121	0\\
122	0\\
123	0\\
124	0\\
125	0\\
126	0\\
127	0\\
128	0\\
129	0\\
130	0\\
131	0\\
132	0\\
133	0\\
134	0\\
135	0\\
136	0\\
137	0\\
138	0\\
139	0\\
140	0\\
141	0\\
142	0\\
143	0\\
144	0\\
145	0\\
146	0\\
147	0\\
148	0\\
149	0\\
150	0\\
151	0\\
152	0\\
153	0\\
154	0\\
155	0\\
156	0\\
157	0\\
158	0\\
159	0\\
160	0\\
161	0\\
162	0\\
163	0\\
164	0\\
165	0\\
166	0\\
167	0\\
168	0\\
169	0\\
170	0\\
171	0\\
172	0\\
173	0\\
174	0\\
175	0\\
176	0\\
177	0\\
178	0\\
179	0\\
180	0\\
181	0\\
182	0\\
183	0\\
184	0\\
185	0\\
186	0\\
187	0\\
188	0\\
189	0\\
190	0\\
191	0\\
192	0\\
193	0\\
194	0\\
195	0\\
196	0\\
197	0\\
198	0\\
199	0\\
200	0\\
201	0\\
202	0\\
203	0\\
204	0\\
205	0\\
206	0\\
207	0\\
208	0\\
209	0\\
210	0\\
211	0\\
212	0\\
213	0\\
214	0\\
215	0\\
216	0\\
217	0\\
218	0\\
219	0\\
220	0\\
221	0\\
222	0\\
223	0\\
224	0\\
225	0\\
226	0\\
227	0\\
228	0\\
229	0\\
230	0\\
231	0\\
232	0\\
233	0\\
234	0\\
235	0\\
236	0\\
237	0\\
238	0\\
239	0\\
240	0\\
241	0\\
242	0\\
243	0\\
244	0\\
245	0\\
246	0\\
247	0\\
248	0\\
249	0\\
250	0\\
251	0\\
252	0\\
253	0\\
254	0\\
255	0\\
256	0\\
257	0\\
258	0\\
259	0\\
260	0\\
261	0\\
262	0\\
263	0\\
264	0\\
265	0\\
266	0\\
267	0\\
268	0\\
269	0\\
270	0\\
271	0\\
272	0\\
273	0\\
274	0\\
275	0\\
276	0\\
277	0\\
278	0\\
279	0\\
280	0\\
281	0\\
282	0\\
283	0\\
284	0\\
285	0\\
286	0\\
287	0\\
288	0\\
289	0\\
290	0\\
291	0\\
292	0\\
293	0\\
294	0\\
295	0\\
296	0\\
297	0\\
298	0\\
299	0\\
300	0\\
301	0\\
302	0\\
303	0\\
304	0\\
305	0\\
306	0\\
307	0\\
308	0\\
309	0\\
310	0\\
311	0\\
312	0\\
313	0\\
314	0\\
315	0\\
316	0\\
317	0\\
318	0\\
319	0\\
320	0\\
321	0\\
322	0\\
323	0\\
324	0\\
325	0\\
326	0\\
327	0\\
328	0\\
329	0\\
330	0\\
331	0\\
332	0\\
333	0\\
334	0\\
335	0\\
336	0\\
337	0\\
338	0\\
339	0\\
340	0\\
341	0\\
342	0\\
343	0\\
344	0\\
345	0\\
346	0\\
347	0\\
348	0\\
349	0\\
350	0\\
351	0\\
352	0\\
353	0\\
354	0\\
355	0\\
356	0\\
357	0\\
358	0\\
359	0\\
360	0\\
361	0\\
362	0\\
363	0\\
364	0\\
365	0\\
366	0\\
367	0\\
368	0\\
369	0\\
370	0\\
371	0\\
372	0\\
373	0\\
374	0\\
375	0\\
376	0\\
377	0\\
378	0\\
379	0\\
380	0\\
381	0\\
382	0\\
383	0\\
384	0\\
385	0\\
386	0\\
387	0\\
388	0\\
389	0\\
390	0\\
391	0\\
392	0\\
393	0\\
394	0\\
395	0\\
396	0\\
397	0\\
398	0\\
399	0\\
400	0\\
401	0\\
402	0\\
403	0\\
404	0\\
405	0\\
406	0\\
407	0\\
408	0\\
409	0\\
410	0\\
411	0\\
412	0\\
413	0\\
414	0\\
415	0\\
416	0\\
417	0\\
418	0\\
419	0\\
420	0\\
421	0\\
422	0\\
423	0\\
424	0\\
425	0\\
426	0\\
427	0\\
428	0\\
429	0\\
430	0\\
431	0\\
432	0\\
433	0\\
434	0\\
435	0\\
436	0\\
437	0\\
438	0\\
439	0\\
440	0\\
441	0\\
442	0\\
443	0\\
444	0\\
445	0\\
446	0\\
447	0\\
448	0\\
449	0\\
450	0\\
451	0\\
452	0\\
453	0\\
454	0\\
455	0\\
456	0\\
457	0\\
458	0\\
459	0\\
460	0\\
461	0\\
462	0\\
463	0\\
464	0\\
465	0\\
466	0\\
467	0\\
468	0\\
469	0\\
470	0\\
471	0\\
472	0\\
473	0\\
474	0\\
475	0\\
476	0\\
477	0\\
478	0\\
479	0\\
480	0\\
481	0\\
482	0\\
483	0\\
484	0\\
485	0\\
486	0\\
487	0\\
488	0\\
489	0\\
490	0\\
491	0\\
492	0\\
493	0\\
494	0\\
495	0\\
496	0\\
497	0\\
498	0\\
499	0\\
500	0\\
501	0\\
502	0\\
503	0\\
504	0\\
505	0\\
506	0\\
507	0\\
508	0\\
509	0\\
510	0\\
511	0\\
512	0\\
513	0\\
514	0\\
515	0\\
516	0\\
517	0\\
518	0\\
519	0\\
520	0\\
521	0\\
522	0\\
523	0\\
524	0\\
525	0\\
526	0\\
527	0\\
528	0\\
529	0\\
530	0\\
531	0\\
532	0\\
533	0\\
534	0\\
535	0\\
536	0\\
537	0\\
538	0\\
539	0\\
540	0\\
541	0\\
542	0\\
543	0\\
544	0\\
545	3.82391076137234e-05\\
546	9.68023725063791e-05\\
547	0.000147720982546101\\
548	0.000190142250714629\\
549	0.00022979867141449\\
550	0.000268454303467937\\
551	0.000306284037234123\\
552	0.00034374680041995\\
553	0.000381433047254094\\
554	0.000419542603462078\\
555	0.000458172315969797\\
556	0.000497380378338157\\
557	0.00053955610927823\\
558	0.000581988250460856\\
559	0.000624481094669423\\
560	0.000665506456059995\\
561	0.000706089745012563\\
562	0.000747338613311647\\
563	0.000789267713198676\\
564	0.000831888557103819\\
565	0.000875412308035493\\
566	0.000919872431348538\\
567	0.000965414080806018\\
568	0.00101726491970261\\
569	0.00106823974206828\\
570	0.00111385344432899\\
571	0.00115893121983135\\
572	0.00120437493685983\\
573	0.00125042703199772\\
574	0.00134080429263664\\
575	0.00187574822298428\\
576	0.00221187192817254\\
577	0.00232010641060596\\
578	0.00240256038064571\\
579	0.00247601921057826\\
580	0.00255050068707274\\
581	0.00262619072506652\\
582	0.00270312408738541\\
583	0.00278133051979913\\
584	0.00286084071525918\\
585	0.00294168800565563\\
586	0.00302391165742733\\
587	0.00310756532765526\\
588	0.00319273905359195\\
589	0.00327961638930458\\
590	0.00336862265377153\\
591	0.00346039684658503\\
592	0.00355653402929194\\
593	0.00366119131364763\\
594	0.00378527046594783\\
595	0.00395742119285503\\
596	0.00425297862742962\\
597	0.0048700418989439\\
598	0.00632942537858856\\
599	0\\
600	0\\
};
\addplot [color=red!75!mycolor17,solid,forget plot]
  table[row sep=crcr]{%
1	0\\
2	0\\
3	0\\
4	0\\
5	0\\
6	0\\
7	0\\
8	0\\
9	0\\
10	0\\
11	0\\
12	0\\
13	0\\
14	0\\
15	0\\
16	0\\
17	0\\
18	0\\
19	0\\
20	0\\
21	0\\
22	0\\
23	0\\
24	0\\
25	0\\
26	0\\
27	0\\
28	0\\
29	0\\
30	0\\
31	0\\
32	0\\
33	0\\
34	0\\
35	0\\
36	0\\
37	0\\
38	0\\
39	0\\
40	0\\
41	0\\
42	0\\
43	0\\
44	0\\
45	0\\
46	0\\
47	0\\
48	0\\
49	0\\
50	0\\
51	0\\
52	0\\
53	0\\
54	0\\
55	0\\
56	0\\
57	0\\
58	0\\
59	0\\
60	0\\
61	0\\
62	0\\
63	0\\
64	0\\
65	0\\
66	0\\
67	0\\
68	0\\
69	0\\
70	0\\
71	0\\
72	0\\
73	0\\
74	0\\
75	0\\
76	0\\
77	0\\
78	0\\
79	0\\
80	0\\
81	0\\
82	0\\
83	0\\
84	0\\
85	0\\
86	0\\
87	0\\
88	0\\
89	0\\
90	0\\
91	0\\
92	0\\
93	0\\
94	0\\
95	0\\
96	0\\
97	0\\
98	0\\
99	0\\
100	0\\
101	0\\
102	0\\
103	0\\
104	0\\
105	0\\
106	0\\
107	0\\
108	0\\
109	0\\
110	0\\
111	0\\
112	0\\
113	0\\
114	0\\
115	0\\
116	0\\
117	0\\
118	0\\
119	0\\
120	0\\
121	0\\
122	0\\
123	0\\
124	0\\
125	0\\
126	0\\
127	0\\
128	0\\
129	0\\
130	0\\
131	0\\
132	0\\
133	0\\
134	0\\
135	0\\
136	0\\
137	0\\
138	0\\
139	0\\
140	0\\
141	0\\
142	0\\
143	0\\
144	0\\
145	0\\
146	0\\
147	0\\
148	0\\
149	0\\
150	0\\
151	0\\
152	0\\
153	0\\
154	0\\
155	0\\
156	0\\
157	0\\
158	0\\
159	0\\
160	0\\
161	0\\
162	0\\
163	0\\
164	0\\
165	0\\
166	0\\
167	0\\
168	0\\
169	0\\
170	0\\
171	0\\
172	0\\
173	0\\
174	0\\
175	0\\
176	0\\
177	0\\
178	0\\
179	0\\
180	0\\
181	0\\
182	0\\
183	0\\
184	0\\
185	0\\
186	0\\
187	0\\
188	0\\
189	0\\
190	0\\
191	0\\
192	0\\
193	0\\
194	0\\
195	0\\
196	0\\
197	0\\
198	0\\
199	0\\
200	0\\
201	0\\
202	0\\
203	0\\
204	0\\
205	0\\
206	0\\
207	0\\
208	0\\
209	0\\
210	0\\
211	0\\
212	0\\
213	0\\
214	0\\
215	0\\
216	0\\
217	0\\
218	0\\
219	0\\
220	0\\
221	0\\
222	0\\
223	0\\
224	0\\
225	0\\
226	0\\
227	0\\
228	0\\
229	0\\
230	0\\
231	0\\
232	0\\
233	0\\
234	0\\
235	0\\
236	0\\
237	0\\
238	0\\
239	0\\
240	0\\
241	0\\
242	0\\
243	0\\
244	0\\
245	0\\
246	0\\
247	0\\
248	0\\
249	0\\
250	0\\
251	0\\
252	0\\
253	0\\
254	0\\
255	0\\
256	0\\
257	0\\
258	0\\
259	0\\
260	0\\
261	0\\
262	0\\
263	0\\
264	0\\
265	0\\
266	0\\
267	0\\
268	0\\
269	0\\
270	0\\
271	0\\
272	0\\
273	0\\
274	0\\
275	0\\
276	0\\
277	0\\
278	0\\
279	0\\
280	0\\
281	0\\
282	0\\
283	0\\
284	0\\
285	0\\
286	0\\
287	0\\
288	0\\
289	0\\
290	0\\
291	0\\
292	0\\
293	0\\
294	0\\
295	0\\
296	0\\
297	0\\
298	0\\
299	0\\
300	0\\
301	0\\
302	0\\
303	0\\
304	0\\
305	0\\
306	0\\
307	0\\
308	0\\
309	0\\
310	0\\
311	0\\
312	0\\
313	0\\
314	0\\
315	0\\
316	0\\
317	0\\
318	0\\
319	0\\
320	0\\
321	0\\
322	0\\
323	0\\
324	0\\
325	0\\
326	0\\
327	0\\
328	0\\
329	0\\
330	0\\
331	0\\
332	0\\
333	0\\
334	0\\
335	0\\
336	0\\
337	0\\
338	0\\
339	0\\
340	0\\
341	0\\
342	0\\
343	0\\
344	0\\
345	0\\
346	0\\
347	0\\
348	0\\
349	0\\
350	0\\
351	0\\
352	0\\
353	0\\
354	0\\
355	0\\
356	0\\
357	0\\
358	0\\
359	0\\
360	0\\
361	0\\
362	0\\
363	0\\
364	0\\
365	0\\
366	0\\
367	0\\
368	0\\
369	0\\
370	0\\
371	0\\
372	0\\
373	0\\
374	0\\
375	0\\
376	0\\
377	0\\
378	0\\
379	0\\
380	0\\
381	0\\
382	0\\
383	0\\
384	0\\
385	0\\
386	0\\
387	0\\
388	0\\
389	0\\
390	0\\
391	0\\
392	0\\
393	0\\
394	0\\
395	0\\
396	0\\
397	0\\
398	0\\
399	0\\
400	0\\
401	0\\
402	0\\
403	0\\
404	0\\
405	0\\
406	0\\
407	0\\
408	0\\
409	0\\
410	0\\
411	0\\
412	0\\
413	0\\
414	0\\
415	0\\
416	0\\
417	0\\
418	0\\
419	0\\
420	0\\
421	0\\
422	0\\
423	0\\
424	0\\
425	0\\
426	0\\
427	0\\
428	0\\
429	0\\
430	0\\
431	0\\
432	0\\
433	0\\
434	0\\
435	0\\
436	0\\
437	0\\
438	0\\
439	0\\
440	0\\
441	0\\
442	0\\
443	0\\
444	0\\
445	0\\
446	0\\
447	0\\
448	0\\
449	0\\
450	0\\
451	0\\
452	0\\
453	0\\
454	0\\
455	0\\
456	0\\
457	0\\
458	0\\
459	0\\
460	0\\
461	0\\
462	0\\
463	0\\
464	0\\
465	0\\
466	0\\
467	0\\
468	0\\
469	0\\
470	0\\
471	0\\
472	0\\
473	0\\
474	0\\
475	0\\
476	0\\
477	0\\
478	0\\
479	0\\
480	0\\
481	0\\
482	0\\
483	0\\
484	0\\
485	0\\
486	0\\
487	0\\
488	0\\
489	0\\
490	0\\
491	0\\
492	0\\
493	0\\
494	0\\
495	0\\
496	0\\
497	0\\
498	0\\
499	0\\
500	0\\
501	0\\
502	0\\
503	0\\
504	0\\
505	0\\
506	0\\
507	0\\
508	0\\
509	0\\
510	0\\
511	0\\
512	0\\
513	0\\
514	0\\
515	0\\
516	0\\
517	0\\
518	0\\
519	0\\
520	0\\
521	0\\
522	0\\
523	0\\
524	0\\
525	0\\
526	0\\
527	0\\
528	0\\
529	0\\
530	0\\
531	0\\
532	0\\
533	0\\
534	0\\
535	0\\
536	0\\
537	0\\
538	0\\
539	0\\
540	0\\
541	0\\
542	2.04853442762617e-05\\
543	6.06220385510295e-05\\
544	9.64802235070158e-05\\
545	0.000131320163008711\\
546	0.000165361460811259\\
547	0.000199053503989163\\
548	0.000232946229837191\\
549	0.00026724254469209\\
550	0.000302038372857046\\
551	0.000337433066059789\\
552	0.000373524405103253\\
553	0.000410607915064151\\
554	0.000450645605495247\\
555	0.000490655083798201\\
556	0.000530561602653835\\
557	0.000568219493923374\\
558	0.000606305612849557\\
559	0.000645024893172913\\
560	0.000684358053097023\\
561	0.000724443518312681\\
562	0.000765389047441491\\
563	0.000807225433839359\\
564	0.000849982974356452\\
565	0.000893867927586915\\
566	0.000944065882789967\\
567	0.000993239584238119\\
568	0.00103619979679305\\
569	0.00107953810792023\\
570	0.00112318247816187\\
571	0.00116757410108689\\
572	0.00121282583955014\\
573	0.00154610456066244\\
574	0.00206173146214074\\
575	0.00217000160705326\\
576	0.00225943058102288\\
577	0.00233065932806681\\
578	0.00240277285646767\\
579	0.00247604740747313\\
580	0.00255051340445631\\
581	0.00262619770980058\\
582	0.00270312784773939\\
583	0.00278133241566772\\
584	0.00286084158979877\\
585	0.00294168836537039\\
586	0.00302391178438079\\
587	0.00310756536383239\\
588	0.00319273906101823\\
589	0.00327961639013216\\
590	0.00336862265377154\\
591	0.00346039684658504\\
592	0.00355653402929194\\
593	0.00366119131364762\\
594	0.00378527046594782\\
595	0.00395742119285503\\
596	0.00425297862742962\\
597	0.0048700418989439\\
598	0.00632942537858856\\
599	0\\
600	0\\
};
\addplot [color=red!80!mycolor19,solid,forget plot]
  table[row sep=crcr]{%
1	0\\
2	0\\
3	0\\
4	0\\
5	0\\
6	0\\
7	0\\
8	0\\
9	0\\
10	0\\
11	0\\
12	0\\
13	0\\
14	0\\
15	0\\
16	0\\
17	0\\
18	0\\
19	0\\
20	0\\
21	0\\
22	0\\
23	0\\
24	0\\
25	0\\
26	0\\
27	0\\
28	0\\
29	0\\
30	0\\
31	0\\
32	0\\
33	0\\
34	0\\
35	0\\
36	0\\
37	0\\
38	0\\
39	0\\
40	0\\
41	0\\
42	0\\
43	0\\
44	0\\
45	0\\
46	0\\
47	0\\
48	0\\
49	0\\
50	0\\
51	0\\
52	0\\
53	0\\
54	0\\
55	0\\
56	0\\
57	0\\
58	0\\
59	0\\
60	0\\
61	0\\
62	0\\
63	0\\
64	0\\
65	0\\
66	0\\
67	0\\
68	0\\
69	0\\
70	0\\
71	0\\
72	0\\
73	0\\
74	0\\
75	0\\
76	0\\
77	0\\
78	0\\
79	0\\
80	0\\
81	0\\
82	0\\
83	0\\
84	0\\
85	0\\
86	0\\
87	0\\
88	0\\
89	0\\
90	0\\
91	0\\
92	0\\
93	0\\
94	0\\
95	0\\
96	0\\
97	0\\
98	0\\
99	0\\
100	0\\
101	0\\
102	0\\
103	0\\
104	0\\
105	0\\
106	0\\
107	0\\
108	0\\
109	0\\
110	0\\
111	0\\
112	0\\
113	0\\
114	0\\
115	0\\
116	0\\
117	0\\
118	0\\
119	0\\
120	0\\
121	0\\
122	0\\
123	0\\
124	0\\
125	0\\
126	0\\
127	0\\
128	0\\
129	0\\
130	0\\
131	0\\
132	0\\
133	0\\
134	0\\
135	0\\
136	0\\
137	0\\
138	0\\
139	0\\
140	0\\
141	0\\
142	0\\
143	0\\
144	0\\
145	0\\
146	0\\
147	0\\
148	0\\
149	0\\
150	0\\
151	0\\
152	0\\
153	0\\
154	0\\
155	0\\
156	0\\
157	0\\
158	0\\
159	0\\
160	0\\
161	0\\
162	0\\
163	0\\
164	0\\
165	0\\
166	0\\
167	0\\
168	0\\
169	0\\
170	0\\
171	0\\
172	0\\
173	0\\
174	0\\
175	0\\
176	0\\
177	0\\
178	0\\
179	0\\
180	0\\
181	0\\
182	0\\
183	0\\
184	0\\
185	0\\
186	0\\
187	0\\
188	0\\
189	0\\
190	0\\
191	0\\
192	0\\
193	0\\
194	0\\
195	0\\
196	0\\
197	0\\
198	0\\
199	0\\
200	0\\
201	0\\
202	0\\
203	0\\
204	0\\
205	0\\
206	0\\
207	0\\
208	0\\
209	0\\
210	0\\
211	0\\
212	0\\
213	0\\
214	0\\
215	0\\
216	0\\
217	0\\
218	0\\
219	0\\
220	0\\
221	0\\
222	0\\
223	0\\
224	0\\
225	0\\
226	0\\
227	0\\
228	0\\
229	0\\
230	0\\
231	0\\
232	0\\
233	0\\
234	0\\
235	0\\
236	0\\
237	0\\
238	0\\
239	0\\
240	0\\
241	0\\
242	0\\
243	0\\
244	0\\
245	0\\
246	0\\
247	0\\
248	0\\
249	0\\
250	0\\
251	0\\
252	0\\
253	0\\
254	0\\
255	0\\
256	0\\
257	0\\
258	0\\
259	0\\
260	0\\
261	0\\
262	0\\
263	0\\
264	0\\
265	0\\
266	0\\
267	0\\
268	0\\
269	0\\
270	0\\
271	0\\
272	0\\
273	0\\
274	0\\
275	0\\
276	0\\
277	0\\
278	0\\
279	0\\
280	0\\
281	0\\
282	0\\
283	0\\
284	0\\
285	0\\
286	0\\
287	0\\
288	0\\
289	0\\
290	0\\
291	0\\
292	0\\
293	0\\
294	0\\
295	0\\
296	0\\
297	0\\
298	0\\
299	0\\
300	0\\
301	0\\
302	0\\
303	0\\
304	0\\
305	0\\
306	0\\
307	0\\
308	0\\
309	0\\
310	0\\
311	0\\
312	0\\
313	0\\
314	0\\
315	0\\
316	0\\
317	0\\
318	0\\
319	0\\
320	0\\
321	0\\
322	0\\
323	0\\
324	0\\
325	0\\
326	0\\
327	0\\
328	0\\
329	0\\
330	0\\
331	0\\
332	0\\
333	0\\
334	0\\
335	0\\
336	0\\
337	0\\
338	0\\
339	0\\
340	0\\
341	0\\
342	0\\
343	0\\
344	0\\
345	0\\
346	0\\
347	0\\
348	0\\
349	0\\
350	0\\
351	0\\
352	0\\
353	0\\
354	0\\
355	0\\
356	0\\
357	0\\
358	0\\
359	0\\
360	0\\
361	0\\
362	0\\
363	0\\
364	0\\
365	0\\
366	0\\
367	0\\
368	0\\
369	0\\
370	0\\
371	0\\
372	0\\
373	0\\
374	0\\
375	0\\
376	0\\
377	0\\
378	0\\
379	0\\
380	0\\
381	0\\
382	0\\
383	0\\
384	0\\
385	0\\
386	0\\
387	0\\
388	0\\
389	0\\
390	0\\
391	0\\
392	0\\
393	0\\
394	0\\
395	0\\
396	0\\
397	0\\
398	0\\
399	0\\
400	0\\
401	0\\
402	0\\
403	0\\
404	0\\
405	0\\
406	0\\
407	0\\
408	0\\
409	0\\
410	0\\
411	0\\
412	0\\
413	0\\
414	0\\
415	0\\
416	0\\
417	0\\
418	0\\
419	0\\
420	0\\
421	0\\
422	0\\
423	0\\
424	0\\
425	0\\
426	0\\
427	0\\
428	0\\
429	0\\
430	0\\
431	0\\
432	0\\
433	0\\
434	0\\
435	0\\
436	0\\
437	0\\
438	0\\
439	0\\
440	0\\
441	0\\
442	0\\
443	0\\
444	0\\
445	0\\
446	0\\
447	0\\
448	0\\
449	0\\
450	0\\
451	0\\
452	0\\
453	0\\
454	0\\
455	0\\
456	0\\
457	0\\
458	0\\
459	0\\
460	0\\
461	0\\
462	0\\
463	0\\
464	0\\
465	0\\
466	0\\
467	0\\
468	0\\
469	0\\
470	0\\
471	0\\
472	0\\
473	0\\
474	0\\
475	0\\
476	0\\
477	0\\
478	0\\
479	0\\
480	0\\
481	0\\
482	0\\
483	0\\
484	0\\
485	0\\
486	0\\
487	0\\
488	0\\
489	0\\
490	0\\
491	0\\
492	0\\
493	0\\
494	0\\
495	0\\
496	0\\
497	0\\
498	0\\
499	0\\
500	0\\
501	0\\
502	0\\
503	0\\
504	0\\
505	0\\
506	0\\
507	0\\
508	0\\
509	0\\
510	0\\
511	0\\
512	0\\
513	0\\
514	0\\
515	0\\
516	0\\
517	0\\
518	0\\
519	0\\
520	0\\
521	0\\
522	0\\
523	0\\
524	0\\
525	0\\
526	0\\
527	0\\
528	0\\
529	0\\
530	0\\
531	0\\
532	0\\
533	0\\
534	0\\
535	0\\
536	0\\
537	0\\
538	0\\
539	0\\
540	4.09915539253185e-06\\
541	3.50855341422652e-05\\
542	6.56241650374363e-05\\
543	9.62275076458695e-05\\
544	0.000127190880966168\\
545	0.000158606315656507\\
546	0.000190566750982985\\
547	0.000223147011209739\\
548	0.000256388207954706\\
549	0.000290340966566383\\
550	0.000325350059729218\\
551	0.00036325193318367\\
552	0.000401173491610415\\
553	0.000438840853802339\\
554	0.000474293154902854\\
555	0.000510160152892572\\
556	0.000546586920712631\\
557	0.000583562655249454\\
558	0.000621307465471656\\
559	0.00065986095225696\\
560	0.000699251876427312\\
561	0.0007395059311949\\
562	0.000780644872361251\\
563	0.000822720633982572\\
564	0.000870893647640451\\
565	0.000918570800930397\\
566	0.000959929760329168\\
567	0.00100162913418252\\
568	0.00104358144557724\\
569	0.00108631367623227\\
570	0.0011298652614946\\
571	0.00117427238843636\\
572	0.00171542356870662\\
573	0.00201939825841623\\
574	0.00212061739353802\\
575	0.0021898538250397\\
576	0.00225970681651282\\
577	0.00233067087660055\\
578	0.00240277681456536\\
579	0.00247604952405764\\
580	0.00255051455217948\\
581	0.00262619830282023\\
582	0.00270312813271299\\
583	0.00278133254043779\\
584	0.00286084163829612\\
585	0.00294168838147584\\
586	0.00302391178867891\\
587	0.00310756536465452\\
588	0.00319273906110311\\
589	0.00327961639013215\\
590	0.00336862265377152\\
591	0.00346039684658503\\
592	0.00355653402929193\\
593	0.00366119131364762\\
594	0.00378527046594782\\
595	0.00395742119285502\\
596	0.00425297862742962\\
597	0.0048700418989439\\
598	0.00632942537858856\\
599	0\\
600	0\\
};
\addplot [color=red,solid,forget plot]
  table[row sep=crcr]{%
1	0\\
2	0\\
3	0\\
4	0\\
5	0\\
6	0\\
7	0\\
8	0\\
9	0\\
10	0\\
11	0\\
12	0\\
13	0\\
14	0\\
15	0\\
16	0\\
17	0\\
18	0\\
19	0\\
20	0\\
21	0\\
22	0\\
23	0\\
24	0\\
25	0\\
26	0\\
27	0\\
28	0\\
29	0\\
30	0\\
31	0\\
32	0\\
33	0\\
34	0\\
35	0\\
36	0\\
37	0\\
38	0\\
39	0\\
40	0\\
41	0\\
42	0\\
43	0\\
44	0\\
45	0\\
46	0\\
47	0\\
48	0\\
49	0\\
50	0\\
51	0\\
52	0\\
53	0\\
54	0\\
55	0\\
56	0\\
57	0\\
58	0\\
59	0\\
60	0\\
61	0\\
62	0\\
63	0\\
64	0\\
65	0\\
66	0\\
67	0\\
68	0\\
69	0\\
70	0\\
71	0\\
72	0\\
73	0\\
74	0\\
75	0\\
76	0\\
77	0\\
78	0\\
79	0\\
80	0\\
81	0\\
82	0\\
83	0\\
84	0\\
85	0\\
86	0\\
87	0\\
88	0\\
89	0\\
90	0\\
91	0\\
92	0\\
93	0\\
94	0\\
95	0\\
96	0\\
97	0\\
98	0\\
99	0\\
100	0\\
101	0\\
102	0\\
103	0\\
104	0\\
105	0\\
106	0\\
107	0\\
108	0\\
109	0\\
110	0\\
111	0\\
112	0\\
113	0\\
114	0\\
115	0\\
116	0\\
117	0\\
118	0\\
119	0\\
120	0\\
121	0\\
122	0\\
123	0\\
124	0\\
125	0\\
126	0\\
127	0\\
128	0\\
129	0\\
130	0\\
131	0\\
132	0\\
133	0\\
134	0\\
135	0\\
136	0\\
137	0\\
138	0\\
139	0\\
140	0\\
141	0\\
142	0\\
143	0\\
144	0\\
145	0\\
146	0\\
147	0\\
148	0\\
149	0\\
150	0\\
151	0\\
152	0\\
153	0\\
154	0\\
155	0\\
156	0\\
157	0\\
158	0\\
159	0\\
160	0\\
161	0\\
162	0\\
163	0\\
164	0\\
165	0\\
166	0\\
167	0\\
168	0\\
169	0\\
170	0\\
171	0\\
172	0\\
173	0\\
174	0\\
175	0\\
176	0\\
177	0\\
178	0\\
179	0\\
180	0\\
181	0\\
182	0\\
183	0\\
184	0\\
185	0\\
186	0\\
187	0\\
188	0\\
189	0\\
190	0\\
191	0\\
192	0\\
193	0\\
194	0\\
195	0\\
196	0\\
197	0\\
198	0\\
199	0\\
200	0\\
201	0\\
202	0\\
203	0\\
204	0\\
205	0\\
206	0\\
207	0\\
208	0\\
209	0\\
210	0\\
211	0\\
212	0\\
213	0\\
214	0\\
215	0\\
216	0\\
217	0\\
218	0\\
219	0\\
220	0\\
221	0\\
222	0\\
223	0\\
224	0\\
225	0\\
226	0\\
227	0\\
228	0\\
229	0\\
230	0\\
231	0\\
232	0\\
233	0\\
234	0\\
235	0\\
236	0\\
237	0\\
238	0\\
239	0\\
240	0\\
241	0\\
242	0\\
243	0\\
244	0\\
245	0\\
246	0\\
247	0\\
248	0\\
249	0\\
250	0\\
251	0\\
252	0\\
253	0\\
254	0\\
255	0\\
256	0\\
257	0\\
258	0\\
259	0\\
260	0\\
261	0\\
262	0\\
263	0\\
264	0\\
265	0\\
266	0\\
267	0\\
268	0\\
269	0\\
270	0\\
271	0\\
272	0\\
273	0\\
274	0\\
275	0\\
276	0\\
277	0\\
278	0\\
279	0\\
280	0\\
281	0\\
282	0\\
283	0\\
284	0\\
285	0\\
286	0\\
287	0\\
288	0\\
289	0\\
290	0\\
291	0\\
292	0\\
293	0\\
294	0\\
295	0\\
296	0\\
297	0\\
298	0\\
299	0\\
300	0\\
301	0\\
302	0\\
303	0\\
304	0\\
305	0\\
306	0\\
307	0\\
308	0\\
309	0\\
310	0\\
311	0\\
312	0\\
313	0\\
314	0\\
315	0\\
316	0\\
317	0\\
318	0\\
319	0\\
320	0\\
321	0\\
322	0\\
323	0\\
324	0\\
325	0\\
326	0\\
327	0\\
328	0\\
329	0\\
330	0\\
331	0\\
332	0\\
333	0\\
334	0\\
335	0\\
336	0\\
337	0\\
338	0\\
339	0\\
340	0\\
341	0\\
342	0\\
343	0\\
344	0\\
345	0\\
346	0\\
347	0\\
348	0\\
349	0\\
350	0\\
351	0\\
352	0\\
353	0\\
354	0\\
355	0\\
356	0\\
357	0\\
358	0\\
359	0\\
360	0\\
361	0\\
362	0\\
363	0\\
364	0\\
365	0\\
366	0\\
367	0\\
368	0\\
369	0\\
370	0\\
371	0\\
372	0\\
373	0\\
374	0\\
375	0\\
376	0\\
377	0\\
378	0\\
379	0\\
380	0\\
381	0\\
382	0\\
383	0\\
384	0\\
385	0\\
386	0\\
387	0\\
388	0\\
389	0\\
390	0\\
391	0\\
392	0\\
393	0\\
394	0\\
395	0\\
396	0\\
397	0\\
398	0\\
399	0\\
400	0\\
401	0\\
402	0\\
403	0\\
404	0\\
405	0\\
406	0\\
407	0\\
408	0\\
409	0\\
410	0\\
411	0\\
412	0\\
413	0\\
414	0\\
415	0\\
416	0\\
417	0\\
418	0\\
419	0\\
420	0\\
421	0\\
422	0\\
423	0\\
424	0\\
425	0\\
426	0\\
427	0\\
428	0\\
429	0\\
430	0\\
431	0\\
432	0\\
433	0\\
434	0\\
435	0\\
436	0\\
437	0\\
438	0\\
439	0\\
440	0\\
441	0\\
442	0\\
443	0\\
444	0\\
445	0\\
446	0\\
447	0\\
448	0\\
449	0\\
450	0\\
451	0\\
452	0\\
453	0\\
454	0\\
455	0\\
456	0\\
457	0\\
458	0\\
459	0\\
460	0\\
461	0\\
462	0\\
463	0\\
464	0\\
465	0\\
466	0\\
467	0\\
468	0\\
469	0\\
470	0\\
471	0\\
472	0\\
473	0\\
474	0\\
475	0\\
476	0\\
477	0\\
478	0\\
479	0\\
480	0\\
481	0\\
482	0\\
483	0\\
484	0\\
485	0\\
486	0\\
487	0\\
488	0\\
489	0\\
490	0\\
491	0\\
492	0\\
493	0\\
494	0\\
495	0\\
496	0\\
497	0\\
498	0\\
499	0\\
500	0\\
501	0\\
502	0\\
503	0\\
504	0\\
505	0\\
506	0\\
507	0\\
508	0\\
509	0\\
510	0\\
511	0\\
512	0\\
513	0\\
514	0\\
515	0\\
516	0\\
517	0\\
518	0\\
519	0\\
520	0\\
521	0\\
522	0\\
523	0\\
524	0\\
525	0\\
526	0\\
527	0\\
528	0\\
529	0\\
530	0\\
531	0\\
532	0\\
533	0\\
534	0\\
535	0\\
536	0\\
537	0\\
538	0\\
539	0\\
540	2.72097330305856e-05\\
541	5.61210861900328e-05\\
542	8.5585431440554e-05\\
543	0.000115648715291517\\
544	0.000146340270906872\\
545	0.000177684940527103\\
546	0.000209720256944281\\
547	0.000242556230441567\\
548	0.000278557139405352\\
549	0.000314550976192727\\
550	0.00035020981117787\\
551	0.000383724276119911\\
552	0.000417754807534627\\
553	0.000452289907333273\\
554	0.000487256054489089\\
555	0.000522838353760378\\
556	0.000559179818522669\\
557	0.000596309200618129\\
558	0.000634247900431846\\
559	0.000673017586461841\\
560	0.00071263931167027\\
561	0.000753150047157175\\
562	0.000798519250497241\\
563	0.000845250121868294\\
564	0.000885803380539406\\
565	0.000926054546725999\\
566	0.00096653431220477\\
567	0.00100777034248639\\
568	0.00104980318090226\\
569	0.0010926538929629\\
570	0.00126482494133247\\
571	0.00184533543969108\\
572	0.0019701721867511\\
573	0.00205342365444128\\
574	0.00212111322836602\\
575	0.00218986312710426\\
576	0.0022597081208049\\
577	0.0023306715146185\\
578	0.00240277715940921\\
579	0.0024760497051662\\
580	0.00255051464193462\\
581	0.00262619834400932\\
582	0.00270312814987377\\
583	0.00278133254676196\\
584	0.0028608416402799\\
585	0.00294168838197388\\
586	0.00302391178876813\\
587	0.00310756536466312\\
588	0.00319273906110311\\
589	0.00327961639013215\\
590	0.00336862265377153\\
591	0.00346039684658503\\
592	0.00355653402929193\\
593	0.00366119131364763\\
594	0.00378527046594782\\
595	0.00395742119285503\\
596	0.00425297862742962\\
597	0.0048700418989439\\
598	0.00632942537858856\\
599	0\\
600	0\\
};
\addplot [color=mycolor20,solid,forget plot]
  table[row sep=crcr]{%
1	0\\
2	0\\
3	0\\
4	0\\
5	0\\
6	0\\
7	0\\
8	0\\
9	0\\
10	0\\
11	0\\
12	0\\
13	0\\
14	0\\
15	0\\
16	0\\
17	0\\
18	0\\
19	0\\
20	0\\
21	0\\
22	0\\
23	0\\
24	0\\
25	0\\
26	0\\
27	0\\
28	0\\
29	0\\
30	0\\
31	0\\
32	0\\
33	0\\
34	0\\
35	0\\
36	0\\
37	0\\
38	0\\
39	0\\
40	0\\
41	0\\
42	0\\
43	0\\
44	0\\
45	0\\
46	0\\
47	0\\
48	0\\
49	0\\
50	0\\
51	0\\
52	0\\
53	0\\
54	0\\
55	0\\
56	0\\
57	0\\
58	0\\
59	0\\
60	0\\
61	0\\
62	0\\
63	0\\
64	0\\
65	0\\
66	0\\
67	0\\
68	0\\
69	0\\
70	0\\
71	0\\
72	0\\
73	0\\
74	0\\
75	0\\
76	0\\
77	0\\
78	0\\
79	0\\
80	0\\
81	0\\
82	0\\
83	0\\
84	0\\
85	0\\
86	0\\
87	0\\
88	0\\
89	0\\
90	0\\
91	0\\
92	0\\
93	0\\
94	0\\
95	0\\
96	0\\
97	0\\
98	0\\
99	0\\
100	0\\
101	0\\
102	0\\
103	0\\
104	0\\
105	0\\
106	0\\
107	0\\
108	0\\
109	0\\
110	0\\
111	0\\
112	0\\
113	0\\
114	0\\
115	0\\
116	0\\
117	0\\
118	0\\
119	0\\
120	0\\
121	0\\
122	0\\
123	0\\
124	0\\
125	0\\
126	0\\
127	0\\
128	0\\
129	0\\
130	0\\
131	0\\
132	0\\
133	0\\
134	0\\
135	0\\
136	0\\
137	0\\
138	0\\
139	0\\
140	0\\
141	0\\
142	0\\
143	0\\
144	0\\
145	0\\
146	0\\
147	0\\
148	0\\
149	0\\
150	0\\
151	0\\
152	0\\
153	0\\
154	0\\
155	0\\
156	0\\
157	0\\
158	0\\
159	0\\
160	0\\
161	0\\
162	0\\
163	0\\
164	0\\
165	0\\
166	0\\
167	0\\
168	0\\
169	0\\
170	0\\
171	0\\
172	0\\
173	0\\
174	0\\
175	0\\
176	0\\
177	0\\
178	0\\
179	0\\
180	0\\
181	0\\
182	0\\
183	0\\
184	0\\
185	0\\
186	0\\
187	0\\
188	0\\
189	0\\
190	0\\
191	0\\
192	0\\
193	0\\
194	0\\
195	0\\
196	0\\
197	0\\
198	0\\
199	0\\
200	0\\
201	0\\
202	0\\
203	0\\
204	0\\
205	0\\
206	0\\
207	0\\
208	0\\
209	0\\
210	0\\
211	0\\
212	0\\
213	0\\
214	0\\
215	0\\
216	0\\
217	0\\
218	0\\
219	0\\
220	0\\
221	0\\
222	0\\
223	0\\
224	0\\
225	0\\
226	0\\
227	0\\
228	0\\
229	0\\
230	0\\
231	0\\
232	0\\
233	0\\
234	0\\
235	0\\
236	0\\
237	0\\
238	0\\
239	0\\
240	0\\
241	0\\
242	0\\
243	0\\
244	0\\
245	0\\
246	0\\
247	0\\
248	0\\
249	0\\
250	0\\
251	0\\
252	0\\
253	0\\
254	0\\
255	0\\
256	0\\
257	0\\
258	0\\
259	0\\
260	0\\
261	0\\
262	0\\
263	0\\
264	0\\
265	0\\
266	0\\
267	0\\
268	0\\
269	0\\
270	0\\
271	0\\
272	0\\
273	0\\
274	0\\
275	0\\
276	0\\
277	0\\
278	0\\
279	0\\
280	0\\
281	0\\
282	0\\
283	0\\
284	0\\
285	0\\
286	0\\
287	0\\
288	0\\
289	0\\
290	0\\
291	0\\
292	0\\
293	0\\
294	0\\
295	0\\
296	0\\
297	0\\
298	0\\
299	0\\
300	0\\
301	0\\
302	0\\
303	0\\
304	0\\
305	0\\
306	0\\
307	0\\
308	0\\
309	0\\
310	0\\
311	0\\
312	0\\
313	0\\
314	0\\
315	0\\
316	0\\
317	0\\
318	0\\
319	0\\
320	0\\
321	0\\
322	0\\
323	0\\
324	0\\
325	0\\
326	0\\
327	0\\
328	0\\
329	0\\
330	0\\
331	0\\
332	0\\
333	0\\
334	0\\
335	0\\
336	0\\
337	0\\
338	0\\
339	0\\
340	0\\
341	0\\
342	0\\
343	0\\
344	0\\
345	0\\
346	0\\
347	0\\
348	0\\
349	0\\
350	0\\
351	0\\
352	0\\
353	0\\
354	0\\
355	0\\
356	0\\
357	0\\
358	0\\
359	0\\
360	0\\
361	0\\
362	0\\
363	0\\
364	0\\
365	0\\
366	0\\
367	0\\
368	0\\
369	0\\
370	0\\
371	0\\
372	0\\
373	0\\
374	0\\
375	0\\
376	0\\
377	0\\
378	0\\
379	0\\
380	0\\
381	0\\
382	0\\
383	0\\
384	0\\
385	0\\
386	0\\
387	0\\
388	0\\
389	0\\
390	0\\
391	0\\
392	0\\
393	0\\
394	0\\
395	0\\
396	0\\
397	0\\
398	0\\
399	0\\
400	0\\
401	0\\
402	0\\
403	0\\
404	0\\
405	0\\
406	0\\
407	0\\
408	0\\
409	0\\
410	0\\
411	0\\
412	0\\
413	0\\
414	0\\
415	0\\
416	0\\
417	0\\
418	0\\
419	0\\
420	0\\
421	0\\
422	0\\
423	0\\
424	0\\
425	0\\
426	0\\
427	0\\
428	0\\
429	0\\
430	0\\
431	0\\
432	0\\
433	0\\
434	0\\
435	0\\
436	0\\
437	0\\
438	0\\
439	0\\
440	0\\
441	0\\
442	0\\
443	0\\
444	0\\
445	0\\
446	0\\
447	0\\
448	0\\
449	0\\
450	0\\
451	0\\
452	0\\
453	0\\
454	0\\
455	0\\
456	0\\
457	0\\
458	0\\
459	0\\
460	0\\
461	0\\
462	0\\
463	0\\
464	0\\
465	0\\
466	0\\
467	0\\
468	0\\
469	0\\
470	0\\
471	0\\
472	0\\
473	0\\
474	0\\
475	0\\
476	0\\
477	0\\
478	0\\
479	0\\
480	0\\
481	0\\
482	0\\
483	0\\
484	0\\
485	0\\
486	0\\
487	0\\
488	0\\
489	0\\
490	0\\
491	0\\
492	0\\
493	0\\
494	0\\
495	0\\
496	0\\
497	0\\
498	0\\
499	0\\
500	0\\
501	0\\
502	0\\
503	0\\
504	0\\
505	0\\
506	0\\
507	0\\
508	0\\
509	0\\
510	0\\
511	0\\
512	0\\
513	0\\
514	0\\
515	0\\
516	0\\
517	0\\
518	0\\
519	0\\
520	0\\
521	0\\
522	0\\
523	0\\
524	0\\
525	0\\
526	0\\
527	0\\
528	0\\
529	0\\
530	0\\
531	0\\
532	0\\
533	0\\
534	0\\
535	0\\
536	0\\
537	0\\
538	0\\
539	1.49492185966091e-05\\
540	4.3322083531691e-05\\
541	7.23064945701987e-05\\
542	0.000101919514130588\\
543	0.000132186799457533\\
544	0.000163148555289638\\
545	0.000196899412872687\\
546	0.000231163070872471\\
547	0.000265280313724257\\
548	0.000296939134999868\\
549	0.000329084648366537\\
550	0.000361719670533462\\
551	0.000394828145692352\\
552	0.000428606969412651\\
553	0.000463052122313129\\
554	0.000498111370312253\\
555	0.000533863357465193\\
556	0.00057039128795026\\
557	0.000607714878606064\\
558	0.000645854512820227\\
559	0.000684829623640465\\
560	0.000726776106591716\\
561	0.000772673465119796\\
562	0.000813534509467145\\
563	0.000852431652279271\\
564	0.000891516665074081\\
565	0.000931278020432426\\
566	0.000971798618077333\\
567	0.00101309731890282\\
568	0.00105519192697659\\
569	0.00134093035495452\\
570	0.00181655877875805\\
571	0.00191924729878529\\
572	0.00198681216416615\\
573	0.00205343742759807\\
574	0.00212111382330841\\
575	0.00218986332242957\\
576	0.00225970822322969\\
577	0.00233067156883697\\
578	0.00240277718689623\\
579	0.00247604971824475\\
580	0.00255051464767741\\
581	0.00262619834629172\\
582	0.0027031281506736\\
583	0.00278133254699977\\
584	0.00286084164033628\\
585	0.00294168838198338\\
586	0.00302391178876899\\
587	0.00310756536466312\\
588	0.00319273906110312\\
589	0.00327961639013216\\
590	0.00336862265377153\\
591	0.00346039684658503\\
592	0.00355653402929193\\
593	0.00366119131364763\\
594	0.00378527046594782\\
595	0.00395742119285503\\
596	0.00425297862742962\\
597	0.0048700418989439\\
598	0.00632942537858856\\
599	0\\
600	0\\
};
\addplot [color=mycolor21,solid,forget plot]
  table[row sep=crcr]{%
1	0\\
2	0\\
3	0\\
4	0\\
5	0\\
6	0\\
7	0\\
8	0\\
9	0\\
10	0\\
11	0\\
12	0\\
13	0\\
14	0\\
15	0\\
16	0\\
17	0\\
18	0\\
19	0\\
20	0\\
21	0\\
22	0\\
23	0\\
24	0\\
25	0\\
26	0\\
27	0\\
28	0\\
29	0\\
30	0\\
31	0\\
32	0\\
33	0\\
34	0\\
35	0\\
36	0\\
37	0\\
38	0\\
39	0\\
40	0\\
41	0\\
42	0\\
43	0\\
44	0\\
45	0\\
46	0\\
47	0\\
48	0\\
49	0\\
50	0\\
51	0\\
52	0\\
53	0\\
54	0\\
55	0\\
56	0\\
57	0\\
58	0\\
59	0\\
60	0\\
61	0\\
62	0\\
63	0\\
64	0\\
65	0\\
66	0\\
67	0\\
68	0\\
69	0\\
70	0\\
71	0\\
72	0\\
73	0\\
74	0\\
75	0\\
76	0\\
77	0\\
78	0\\
79	0\\
80	0\\
81	0\\
82	0\\
83	0\\
84	0\\
85	0\\
86	0\\
87	0\\
88	0\\
89	0\\
90	0\\
91	0\\
92	0\\
93	0\\
94	0\\
95	0\\
96	0\\
97	0\\
98	0\\
99	0\\
100	0\\
101	0\\
102	0\\
103	0\\
104	0\\
105	0\\
106	0\\
107	0\\
108	0\\
109	0\\
110	0\\
111	0\\
112	0\\
113	0\\
114	0\\
115	0\\
116	0\\
117	0\\
118	0\\
119	0\\
120	0\\
121	0\\
122	0\\
123	0\\
124	0\\
125	0\\
126	0\\
127	0\\
128	0\\
129	0\\
130	0\\
131	0\\
132	0\\
133	0\\
134	0\\
135	0\\
136	0\\
137	0\\
138	0\\
139	0\\
140	0\\
141	0\\
142	0\\
143	0\\
144	0\\
145	0\\
146	0\\
147	0\\
148	0\\
149	0\\
150	0\\
151	0\\
152	0\\
153	0\\
154	0\\
155	0\\
156	0\\
157	0\\
158	0\\
159	0\\
160	0\\
161	0\\
162	0\\
163	0\\
164	0\\
165	0\\
166	0\\
167	0\\
168	0\\
169	0\\
170	0\\
171	0\\
172	0\\
173	0\\
174	0\\
175	0\\
176	0\\
177	0\\
178	0\\
179	0\\
180	0\\
181	0\\
182	0\\
183	0\\
184	0\\
185	0\\
186	0\\
187	0\\
188	0\\
189	0\\
190	0\\
191	0\\
192	0\\
193	0\\
194	0\\
195	0\\
196	0\\
197	0\\
198	0\\
199	0\\
200	0\\
201	0\\
202	0\\
203	0\\
204	0\\
205	0\\
206	0\\
207	0\\
208	0\\
209	0\\
210	0\\
211	0\\
212	0\\
213	0\\
214	0\\
215	0\\
216	0\\
217	0\\
218	0\\
219	0\\
220	0\\
221	0\\
222	0\\
223	0\\
224	0\\
225	0\\
226	0\\
227	0\\
228	0\\
229	0\\
230	0\\
231	0\\
232	0\\
233	0\\
234	0\\
235	0\\
236	0\\
237	0\\
238	0\\
239	0\\
240	0\\
241	0\\
242	0\\
243	0\\
244	0\\
245	0\\
246	0\\
247	0\\
248	0\\
249	0\\
250	0\\
251	0\\
252	0\\
253	0\\
254	0\\
255	0\\
256	0\\
257	0\\
258	0\\
259	0\\
260	0\\
261	0\\
262	0\\
263	0\\
264	0\\
265	0\\
266	0\\
267	0\\
268	0\\
269	0\\
270	0\\
271	0\\
272	0\\
273	0\\
274	0\\
275	0\\
276	0\\
277	0\\
278	0\\
279	0\\
280	0\\
281	0\\
282	0\\
283	0\\
284	0\\
285	0\\
286	0\\
287	0\\
288	0\\
289	0\\
290	0\\
291	0\\
292	0\\
293	0\\
294	0\\
295	0\\
296	0\\
297	0\\
298	0\\
299	0\\
300	0\\
301	0\\
302	0\\
303	0\\
304	0\\
305	0\\
306	0\\
307	0\\
308	0\\
309	0\\
310	0\\
311	0\\
312	0\\
313	0\\
314	0\\
315	0\\
316	0\\
317	0\\
318	0\\
319	0\\
320	0\\
321	0\\
322	0\\
323	0\\
324	0\\
325	0\\
326	0\\
327	0\\
328	0\\
329	0\\
330	0\\
331	0\\
332	0\\
333	0\\
334	0\\
335	0\\
336	0\\
337	0\\
338	0\\
339	0\\
340	0\\
341	0\\
342	0\\
343	0\\
344	0\\
345	0\\
346	0\\
347	0\\
348	0\\
349	0\\
350	0\\
351	0\\
352	0\\
353	0\\
354	0\\
355	0\\
356	0\\
357	0\\
358	0\\
359	0\\
360	0\\
361	0\\
362	0\\
363	0\\
364	0\\
365	0\\
366	0\\
367	0\\
368	0\\
369	0\\
370	0\\
371	0\\
372	0\\
373	0\\
374	0\\
375	0\\
376	0\\
377	0\\
378	0\\
379	0\\
380	0\\
381	0\\
382	0\\
383	0\\
384	0\\
385	0\\
386	0\\
387	0\\
388	0\\
389	0\\
390	0\\
391	0\\
392	0\\
393	0\\
394	0\\
395	0\\
396	0\\
397	0\\
398	0\\
399	0\\
400	0\\
401	0\\
402	0\\
403	0\\
404	0\\
405	0\\
406	0\\
407	0\\
408	0\\
409	0\\
410	0\\
411	0\\
412	0\\
413	0\\
414	0\\
415	0\\
416	0\\
417	0\\
418	0\\
419	0\\
420	0\\
421	0\\
422	0\\
423	0\\
424	0\\
425	0\\
426	0\\
427	0\\
428	0\\
429	0\\
430	0\\
431	0\\
432	0\\
433	0\\
434	0\\
435	0\\
436	0\\
437	0\\
438	0\\
439	0\\
440	0\\
441	0\\
442	0\\
443	0\\
444	0\\
445	0\\
446	0\\
447	0\\
448	0\\
449	0\\
450	0\\
451	0\\
452	0\\
453	0\\
454	0\\
455	0\\
456	0\\
457	0\\
458	0\\
459	0\\
460	0\\
461	0\\
462	0\\
463	0\\
464	0\\
465	0\\
466	0\\
467	0\\
468	0\\
469	0\\
470	0\\
471	0\\
472	0\\
473	0\\
474	0\\
475	0\\
476	0\\
477	0\\
478	0\\
479	0\\
480	0\\
481	0\\
482	0\\
483	0\\
484	0\\
485	0\\
486	0\\
487	0\\
488	0\\
489	0\\
490	0\\
491	0\\
492	0\\
493	0\\
494	0\\
495	0\\
496	0\\
497	0\\
498	0\\
499	0\\
500	0\\
501	0\\
502	0\\
503	0\\
504	0\\
505	0\\
506	0\\
507	0\\
508	0\\
509	0\\
510	0\\
511	0\\
512	0\\
513	0\\
514	0\\
515	0\\
516	0\\
517	0\\
518	0\\
519	0\\
520	0\\
521	0\\
522	0\\
523	0\\
524	0\\
525	0\\
526	0\\
527	0\\
528	0\\
529	0\\
530	0\\
531	0\\
532	0\\
533	0\\
534	0\\
535	0\\
536	0\\
537	0\\
538	1.21694107886508e-06\\
539	2.92045651449222e-05\\
540	5.78049456237067e-05\\
541	8.70536593544758e-05\\
542	0.000118312744446052\\
543	0.000151016328109347\\
544	0.000183614512075941\\
545	0.000214055086743375\\
546	0.000244429244110815\\
547	0.000275261690455392\\
548	0.00030651277983146\\
549	0.000338404008295745\\
550	0.000370954886140724\\
551	0.00040418384758811\\
552	0.000438095852672766\\
553	0.000472675235959531\\
554	0.000507841527916458\\
555	0.000543753467495973\\
556	0.000580443856548278\\
557	0.000617931118089366\\
558	0.000656254294614929\\
559	0.000701069072514467\\
560	0.000743145599645497\\
561	0.000780787630662042\\
562	0.000818610761048887\\
563	0.000856957736854093\\
564	0.000896029558364049\\
565	0.000935845227354032\\
566	0.000976422049464571\\
567	0.00101777751913248\\
568	0.00138028712439642\\
569	0.00176338880029626\\
570	0.00185658807509324\\
571	0.00192121863126382\\
572	0.00198681271195394\\
573	0.00205343749576471\\
574	0.00212111385372883\\
575	0.00218986333840477\\
576	0.00225970823145245\\
577	0.00233067157285982\\
578	0.00240277718873636\\
579	0.00247604971901934\\
580	0.00255051464797174\\
581	0.00262619834639004\\
582	0.00270312815070138\\
583	0.00278133254700601\\
584	0.00286084164033727\\
585	0.00294168838198347\\
586	0.00302391178876899\\
587	0.00310756536466311\\
588	0.00319273906110311\\
589	0.00327961639013215\\
590	0.00336862265377153\\
591	0.00346039684658503\\
592	0.00355653402929193\\
593	0.00366119131364763\\
594	0.00378527046594783\\
595	0.00395742119285503\\
596	0.00425297862742962\\
597	0.0048700418989439\\
598	0.00632942537858856\\
599	0\\
600	0\\
};
\addplot [color=black!20!mycolor21,solid,forget plot]
  table[row sep=crcr]{%
1	0\\
2	0\\
3	0\\
4	0\\
5	0\\
6	0\\
7	0\\
8	0\\
9	0\\
10	0\\
11	0\\
12	0\\
13	0\\
14	0\\
15	0\\
16	0\\
17	0\\
18	0\\
19	0\\
20	0\\
21	0\\
22	0\\
23	0\\
24	0\\
25	0\\
26	0\\
27	0\\
28	0\\
29	0\\
30	0\\
31	0\\
32	0\\
33	0\\
34	0\\
35	0\\
36	0\\
37	0\\
38	0\\
39	0\\
40	0\\
41	0\\
42	0\\
43	0\\
44	0\\
45	0\\
46	0\\
47	0\\
48	0\\
49	0\\
50	0\\
51	0\\
52	0\\
53	0\\
54	0\\
55	0\\
56	0\\
57	0\\
58	0\\
59	0\\
60	0\\
61	0\\
62	0\\
63	0\\
64	0\\
65	0\\
66	0\\
67	0\\
68	0\\
69	0\\
70	0\\
71	0\\
72	0\\
73	0\\
74	0\\
75	0\\
76	0\\
77	0\\
78	0\\
79	0\\
80	0\\
81	0\\
82	0\\
83	0\\
84	0\\
85	0\\
86	0\\
87	0\\
88	0\\
89	0\\
90	0\\
91	0\\
92	0\\
93	0\\
94	0\\
95	0\\
96	0\\
97	0\\
98	0\\
99	0\\
100	0\\
101	0\\
102	0\\
103	0\\
104	0\\
105	0\\
106	0\\
107	0\\
108	0\\
109	0\\
110	0\\
111	0\\
112	0\\
113	0\\
114	0\\
115	0\\
116	0\\
117	0\\
118	0\\
119	0\\
120	0\\
121	0\\
122	0\\
123	0\\
124	0\\
125	0\\
126	0\\
127	0\\
128	0\\
129	0\\
130	0\\
131	0\\
132	0\\
133	0\\
134	0\\
135	0\\
136	0\\
137	0\\
138	0\\
139	0\\
140	0\\
141	0\\
142	0\\
143	0\\
144	0\\
145	0\\
146	0\\
147	0\\
148	0\\
149	0\\
150	0\\
151	0\\
152	0\\
153	0\\
154	0\\
155	0\\
156	0\\
157	0\\
158	0\\
159	0\\
160	0\\
161	0\\
162	0\\
163	0\\
164	0\\
165	0\\
166	0\\
167	0\\
168	0\\
169	0\\
170	0\\
171	0\\
172	0\\
173	0\\
174	0\\
175	0\\
176	0\\
177	0\\
178	0\\
179	0\\
180	0\\
181	0\\
182	0\\
183	0\\
184	0\\
185	0\\
186	0\\
187	0\\
188	0\\
189	0\\
190	0\\
191	0\\
192	0\\
193	0\\
194	0\\
195	0\\
196	0\\
197	0\\
198	0\\
199	0\\
200	0\\
201	0\\
202	0\\
203	0\\
204	0\\
205	0\\
206	0\\
207	0\\
208	0\\
209	0\\
210	0\\
211	0\\
212	0\\
213	0\\
214	0\\
215	0\\
216	0\\
217	0\\
218	0\\
219	0\\
220	0\\
221	0\\
222	0\\
223	0\\
224	0\\
225	0\\
226	0\\
227	0\\
228	0\\
229	0\\
230	0\\
231	0\\
232	0\\
233	0\\
234	0\\
235	0\\
236	0\\
237	0\\
238	0\\
239	0\\
240	0\\
241	0\\
242	0\\
243	0\\
244	0\\
245	0\\
246	0\\
247	0\\
248	0\\
249	0\\
250	0\\
251	0\\
252	0\\
253	0\\
254	0\\
255	0\\
256	0\\
257	0\\
258	0\\
259	0\\
260	0\\
261	0\\
262	0\\
263	0\\
264	0\\
265	0\\
266	0\\
267	0\\
268	0\\
269	0\\
270	0\\
271	0\\
272	0\\
273	0\\
274	0\\
275	0\\
276	0\\
277	0\\
278	0\\
279	0\\
280	0\\
281	0\\
282	0\\
283	0\\
284	0\\
285	0\\
286	0\\
287	0\\
288	0\\
289	0\\
290	0\\
291	0\\
292	0\\
293	0\\
294	0\\
295	0\\
296	0\\
297	0\\
298	0\\
299	0\\
300	0\\
301	0\\
302	0\\
303	0\\
304	0\\
305	0\\
306	0\\
307	0\\
308	0\\
309	0\\
310	0\\
311	0\\
312	0\\
313	0\\
314	0\\
315	0\\
316	0\\
317	0\\
318	0\\
319	0\\
320	0\\
321	0\\
322	0\\
323	0\\
324	0\\
325	0\\
326	0\\
327	0\\
328	0\\
329	0\\
330	0\\
331	0\\
332	0\\
333	0\\
334	0\\
335	0\\
336	0\\
337	0\\
338	0\\
339	0\\
340	0\\
341	0\\
342	0\\
343	0\\
344	0\\
345	0\\
346	0\\
347	0\\
348	0\\
349	0\\
350	0\\
351	0\\
352	0\\
353	0\\
354	0\\
355	0\\
356	0\\
357	0\\
358	0\\
359	0\\
360	0\\
361	0\\
362	0\\
363	0\\
364	0\\
365	0\\
366	0\\
367	0\\
368	0\\
369	0\\
370	0\\
371	0\\
372	0\\
373	0\\
374	0\\
375	0\\
376	0\\
377	0\\
378	0\\
379	0\\
380	0\\
381	0\\
382	0\\
383	0\\
384	0\\
385	0\\
386	0\\
387	0\\
388	0\\
389	0\\
390	0\\
391	0\\
392	0\\
393	0\\
394	0\\
395	0\\
396	0\\
397	0\\
398	0\\
399	0\\
400	0\\
401	0\\
402	0\\
403	0\\
404	0\\
405	0\\
406	0\\
407	0\\
408	0\\
409	0\\
410	0\\
411	0\\
412	0\\
413	0\\
414	0\\
415	0\\
416	0\\
417	0\\
418	0\\
419	0\\
420	0\\
421	0\\
422	0\\
423	0\\
424	0\\
425	0\\
426	0\\
427	0\\
428	0\\
429	0\\
430	0\\
431	0\\
432	0\\
433	0\\
434	0\\
435	0\\
436	0\\
437	0\\
438	0\\
439	0\\
440	0\\
441	0\\
442	0\\
443	0\\
444	0\\
445	0\\
446	0\\
447	0\\
448	0\\
449	0\\
450	0\\
451	0\\
452	0\\
453	0\\
454	0\\
455	0\\
456	0\\
457	0\\
458	0\\
459	0\\
460	0\\
461	0\\
462	0\\
463	0\\
464	0\\
465	0\\
466	0\\
467	0\\
468	0\\
469	0\\
470	0\\
471	0\\
472	0\\
473	0\\
474	0\\
475	0\\
476	0\\
477	0\\
478	0\\
479	0\\
480	0\\
481	0\\
482	0\\
483	0\\
484	0\\
485	0\\
486	0\\
487	0\\
488	0\\
489	0\\
490	0\\
491	0\\
492	0\\
493	0\\
494	0\\
495	0\\
496	0\\
497	0\\
498	0\\
499	0\\
500	0\\
501	0\\
502	0\\
503	0\\
504	0\\
505	0\\
506	0\\
507	0\\
508	0\\
509	0\\
510	0\\
511	0\\
512	0\\
513	0\\
514	0\\
515	0\\
516	0\\
517	0\\
518	0\\
519	0\\
520	0\\
521	0\\
522	0\\
523	0\\
524	0\\
525	0\\
526	0\\
527	0\\
528	0\\
529	0\\
530	0\\
531	0\\
532	0\\
533	0\\
534	0\\
535	0\\
536	0\\
537	0\\
538	1.426719923304e-05\\
539	4.29091327027637e-05\\
540	7.42159152757329e-05\\
541	0.00010540163623223\\
542	0.000135128139185704\\
543	0.000163860507202848\\
544	0.000193019670144231\\
545	0.000222567303902726\\
546	0.00025268187255965\\
547	0.000283418338955152\\
548	0.000314798578510525\\
549	0.000346837628639188\\
550	0.000379549362854509\\
551	0.00041294395250958\\
552	0.0004470207297917\\
553	0.000481745886966618\\
554	0.000517094416528536\\
555	0.000553212180424865\\
556	0.000590118819429489\\
557	0.000630458343480549\\
558	0.000674248409785912\\
559	0.000711217113060013\\
560	0.000747899309859749\\
561	0.0007848972137292\\
562	0.000822585793724534\\
563	0.00086098759293346\\
564	0.000900118540435697\\
565	0.000939995557047767\\
566	0.000980636019795443\\
567	0.00138345405867184\\
568	0.0017080408379249\\
569	0.00179304145117328\\
570	0.00185663516672567\\
571	0.00192121866981793\\
572	0.00198681272153112\\
573	0.00205343750043299\\
574	0.00212111385614151\\
575	0.00218986333960979\\
576	0.00225970823202167\\
577	0.00233067157311048\\
578	0.00240277718883768\\
579	0.00247604971905621\\
580	0.00255051464798352\\
581	0.00262619834639323\\
582	0.00270312815070207\\
583	0.00278133254700611\\
584	0.00286084164033729\\
585	0.00294168838198347\\
586	0.00302391178876899\\
587	0.00310756536466313\\
588	0.00319273906110312\\
589	0.00327961639013215\\
590	0.00336862265377153\\
591	0.00346039684658503\\
592	0.00355653402929193\\
593	0.00366119131364763\\
594	0.00378527046594782\\
595	0.00395742119285503\\
596	0.00425297862742962\\
597	0.0048700418989439\\
598	0.00632942537858856\\
599	0\\
600	0\\
};
\addplot [color=black!50!mycolor20,solid,forget plot]
  table[row sep=crcr]{%
1	0\\
2	0\\
3	0\\
4	0\\
5	0\\
6	0\\
7	0\\
8	0\\
9	0\\
10	0\\
11	0\\
12	0\\
13	0\\
14	0\\
15	0\\
16	0\\
17	0\\
18	0\\
19	0\\
20	0\\
21	0\\
22	0\\
23	0\\
24	0\\
25	0\\
26	0\\
27	0\\
28	0\\
29	0\\
30	0\\
31	0\\
32	0\\
33	0\\
34	0\\
35	0\\
36	0\\
37	0\\
38	0\\
39	0\\
40	0\\
41	0\\
42	0\\
43	0\\
44	0\\
45	0\\
46	0\\
47	0\\
48	0\\
49	0\\
50	0\\
51	0\\
52	0\\
53	0\\
54	0\\
55	0\\
56	0\\
57	0\\
58	0\\
59	0\\
60	0\\
61	0\\
62	0\\
63	0\\
64	0\\
65	0\\
66	0\\
67	0\\
68	0\\
69	0\\
70	0\\
71	0\\
72	0\\
73	0\\
74	0\\
75	0\\
76	0\\
77	0\\
78	0\\
79	0\\
80	0\\
81	0\\
82	0\\
83	0\\
84	0\\
85	0\\
86	0\\
87	0\\
88	0\\
89	0\\
90	0\\
91	0\\
92	0\\
93	0\\
94	0\\
95	0\\
96	0\\
97	0\\
98	0\\
99	0\\
100	0\\
101	0\\
102	0\\
103	0\\
104	0\\
105	0\\
106	0\\
107	0\\
108	0\\
109	0\\
110	0\\
111	0\\
112	0\\
113	0\\
114	0\\
115	0\\
116	0\\
117	0\\
118	0\\
119	0\\
120	0\\
121	0\\
122	0\\
123	0\\
124	0\\
125	0\\
126	0\\
127	0\\
128	0\\
129	0\\
130	0\\
131	0\\
132	0\\
133	0\\
134	0\\
135	0\\
136	0\\
137	0\\
138	0\\
139	0\\
140	0\\
141	0\\
142	0\\
143	0\\
144	0\\
145	0\\
146	0\\
147	0\\
148	0\\
149	0\\
150	0\\
151	0\\
152	0\\
153	0\\
154	0\\
155	0\\
156	0\\
157	0\\
158	0\\
159	0\\
160	0\\
161	0\\
162	0\\
163	0\\
164	0\\
165	0\\
166	0\\
167	0\\
168	0\\
169	0\\
170	0\\
171	0\\
172	0\\
173	0\\
174	0\\
175	0\\
176	0\\
177	0\\
178	0\\
179	0\\
180	0\\
181	0\\
182	0\\
183	0\\
184	0\\
185	0\\
186	0\\
187	0\\
188	0\\
189	0\\
190	0\\
191	0\\
192	0\\
193	0\\
194	0\\
195	0\\
196	0\\
197	0\\
198	0\\
199	0\\
200	0\\
201	0\\
202	0\\
203	0\\
204	0\\
205	0\\
206	0\\
207	0\\
208	0\\
209	0\\
210	0\\
211	0\\
212	0\\
213	0\\
214	0\\
215	0\\
216	0\\
217	0\\
218	0\\
219	0\\
220	0\\
221	0\\
222	0\\
223	0\\
224	0\\
225	0\\
226	0\\
227	0\\
228	0\\
229	0\\
230	0\\
231	0\\
232	0\\
233	0\\
234	0\\
235	0\\
236	0\\
237	0\\
238	0\\
239	0\\
240	0\\
241	0\\
242	0\\
243	0\\
244	0\\
245	0\\
246	0\\
247	0\\
248	0\\
249	0\\
250	0\\
251	0\\
252	0\\
253	0\\
254	0\\
255	0\\
256	0\\
257	0\\
258	0\\
259	0\\
260	0\\
261	0\\
262	0\\
263	0\\
264	0\\
265	0\\
266	0\\
267	0\\
268	0\\
269	0\\
270	0\\
271	0\\
272	0\\
273	0\\
274	0\\
275	0\\
276	0\\
277	0\\
278	0\\
279	0\\
280	0\\
281	0\\
282	0\\
283	0\\
284	0\\
285	0\\
286	0\\
287	0\\
288	0\\
289	0\\
290	0\\
291	0\\
292	0\\
293	0\\
294	0\\
295	0\\
296	0\\
297	0\\
298	0\\
299	0\\
300	0\\
301	0\\
302	0\\
303	0\\
304	0\\
305	0\\
306	0\\
307	0\\
308	0\\
309	0\\
310	0\\
311	0\\
312	0\\
313	0\\
314	0\\
315	0\\
316	0\\
317	0\\
318	0\\
319	0\\
320	0\\
321	0\\
322	0\\
323	0\\
324	0\\
325	0\\
326	0\\
327	0\\
328	0\\
329	0\\
330	0\\
331	0\\
332	0\\
333	0\\
334	0\\
335	0\\
336	0\\
337	0\\
338	0\\
339	0\\
340	0\\
341	0\\
342	0\\
343	0\\
344	0\\
345	0\\
346	0\\
347	0\\
348	0\\
349	0\\
350	0\\
351	0\\
352	0\\
353	0\\
354	0\\
355	0\\
356	0\\
357	0\\
358	0\\
359	0\\
360	0\\
361	0\\
362	0\\
363	0\\
364	0\\
365	0\\
366	0\\
367	0\\
368	0\\
369	0\\
370	0\\
371	0\\
372	0\\
373	0\\
374	0\\
375	0\\
376	0\\
377	0\\
378	0\\
379	0\\
380	0\\
381	0\\
382	0\\
383	0\\
384	0\\
385	0\\
386	0\\
387	0\\
388	0\\
389	0\\
390	0\\
391	0\\
392	0\\
393	0\\
394	0\\
395	0\\
396	0\\
397	0\\
398	0\\
399	0\\
400	0\\
401	0\\
402	0\\
403	0\\
404	0\\
405	0\\
406	0\\
407	0\\
408	0\\
409	0\\
410	0\\
411	0\\
412	0\\
413	0\\
414	0\\
415	0\\
416	0\\
417	0\\
418	0\\
419	0\\
420	0\\
421	0\\
422	0\\
423	0\\
424	0\\
425	0\\
426	0\\
427	0\\
428	0\\
429	0\\
430	0\\
431	0\\
432	0\\
433	0\\
434	0\\
435	0\\
436	0\\
437	0\\
438	0\\
439	0\\
440	0\\
441	0\\
442	0\\
443	0\\
444	0\\
445	0\\
446	0\\
447	0\\
448	0\\
449	0\\
450	0\\
451	0\\
452	0\\
453	0\\
454	0\\
455	0\\
456	0\\
457	0\\
458	0\\
459	0\\
460	0\\
461	0\\
462	0\\
463	0\\
464	0\\
465	0\\
466	0\\
467	0\\
468	0\\
469	0\\
470	0\\
471	0\\
472	0\\
473	0\\
474	0\\
475	0\\
476	0\\
477	0\\
478	0\\
479	0\\
480	0\\
481	0\\
482	0\\
483	0\\
484	0\\
485	0\\
486	0\\
487	0\\
488	0\\
489	0\\
490	0\\
491	0\\
492	0\\
493	0\\
494	0\\
495	0\\
496	0\\
497	0\\
498	0\\
499	0\\
500	0\\
501	0\\
502	0\\
503	0\\
504	0\\
505	0\\
506	0\\
507	0\\
508	0\\
509	0\\
510	0\\
511	0\\
512	0\\
513	0\\
514	0\\
515	0\\
516	0\\
517	0\\
518	0\\
519	0\\
520	0\\
521	0\\
522	0\\
523	0\\
524	0\\
525	0\\
526	0\\
527	0\\
528	0\\
529	0\\
530	0\\
531	0\\
532	0\\
533	0\\
534	0\\
535	0\\
536	0\\
537	6.49828894175203e-07\\
538	3.05652458911548e-05\\
539	5.99404394527822e-05\\
540	8.71312739401948e-05\\
541	0.000114716903641183\\
542	0.000142674508084298\\
543	0.000171100011748483\\
544	0.000200107410759274\\
545	0.000229717106180836\\
546	0.000259944556361032\\
547	0.000290804095303178\\
548	0.000322309893047164\\
549	0.000354475852884199\\
550	0.000387314440674418\\
551	0.000420833252736105\\
552	0.000455025048030861\\
553	0.00048983964048552\\
554	0.000525318728056389\\
555	0.000561574556760999\\
556	0.000603919415462835\\
557	0.00064369719785649\\
558	0.000679340140340759\\
559	0.000715080552741755\\
560	0.000751442297042198\\
561	0.000788488188292752\\
562	0.000826233343400132\\
563	0.000864693097341739\\
564	0.000903883877005878\\
565	0.000943822831240413\\
566	0.00135108213105103\\
567	0.00165100703747856\\
568	0.00173042226223131\\
569	0.0017930426874416\\
570	0.00185663517077488\\
571	0.00192121867121681\\
572	0.00198681272222949\\
573	0.00205343750078598\\
574	0.00212111385631247\\
575	0.00218986333968779\\
576	0.00225970823205476\\
577	0.00233067157312331\\
578	0.00240277718884216\\
579	0.00247604971905759\\
580	0.00255051464798387\\
581	0.0026261983463933\\
582	0.00270312815070208\\
583	0.0027813325470061\\
584	0.00286084164033727\\
585	0.00294168838198346\\
586	0.00302391178876899\\
587	0.00310756536466312\\
588	0.00319273906110312\\
589	0.00327961639013215\\
590	0.00336862265377153\\
591	0.00346039684658503\\
592	0.00355653402929193\\
593	0.00366119131364762\\
594	0.00378527046594782\\
595	0.00395742119285503\\
596	0.00425297862742961\\
597	0.0048700418989439\\
598	0.00632942537858856\\
599	0\\
600	0\\
};
\addplot [color=black!60!mycolor21,solid,forget plot]
  table[row sep=crcr]{%
1	0\\
2	0\\
3	0\\
4	0\\
5	0\\
6	0\\
7	0\\
8	0\\
9	0\\
10	0\\
11	0\\
12	0\\
13	0\\
14	0\\
15	0\\
16	0\\
17	0\\
18	0\\
19	0\\
20	0\\
21	0\\
22	0\\
23	0\\
24	0\\
25	0\\
26	0\\
27	0\\
28	0\\
29	0\\
30	0\\
31	0\\
32	0\\
33	0\\
34	0\\
35	0\\
36	0\\
37	0\\
38	0\\
39	0\\
40	0\\
41	0\\
42	0\\
43	0\\
44	0\\
45	0\\
46	0\\
47	0\\
48	0\\
49	0\\
50	0\\
51	0\\
52	0\\
53	0\\
54	0\\
55	0\\
56	0\\
57	0\\
58	0\\
59	0\\
60	0\\
61	0\\
62	0\\
63	0\\
64	0\\
65	0\\
66	0\\
67	0\\
68	0\\
69	0\\
70	0\\
71	0\\
72	0\\
73	0\\
74	0\\
75	0\\
76	0\\
77	0\\
78	0\\
79	0\\
80	0\\
81	0\\
82	0\\
83	0\\
84	0\\
85	0\\
86	0\\
87	0\\
88	0\\
89	0\\
90	0\\
91	0\\
92	0\\
93	0\\
94	0\\
95	0\\
96	0\\
97	0\\
98	0\\
99	0\\
100	0\\
101	0\\
102	0\\
103	0\\
104	0\\
105	0\\
106	0\\
107	0\\
108	0\\
109	0\\
110	0\\
111	0\\
112	0\\
113	0\\
114	0\\
115	0\\
116	0\\
117	0\\
118	0\\
119	0\\
120	0\\
121	0\\
122	0\\
123	0\\
124	0\\
125	0\\
126	0\\
127	0\\
128	0\\
129	0\\
130	0\\
131	0\\
132	0\\
133	0\\
134	0\\
135	0\\
136	0\\
137	0\\
138	0\\
139	0\\
140	0\\
141	0\\
142	0\\
143	0\\
144	0\\
145	0\\
146	0\\
147	0\\
148	0\\
149	0\\
150	0\\
151	0\\
152	0\\
153	0\\
154	0\\
155	0\\
156	0\\
157	0\\
158	0\\
159	0\\
160	0\\
161	0\\
162	0\\
163	0\\
164	0\\
165	0\\
166	0\\
167	0\\
168	0\\
169	0\\
170	0\\
171	0\\
172	0\\
173	0\\
174	0\\
175	0\\
176	0\\
177	0\\
178	0\\
179	0\\
180	0\\
181	0\\
182	0\\
183	0\\
184	0\\
185	0\\
186	0\\
187	0\\
188	0\\
189	0\\
190	0\\
191	0\\
192	0\\
193	0\\
194	0\\
195	0\\
196	0\\
197	0\\
198	0\\
199	0\\
200	0\\
201	0\\
202	0\\
203	0\\
204	0\\
205	0\\
206	0\\
207	0\\
208	0\\
209	0\\
210	0\\
211	0\\
212	0\\
213	0\\
214	0\\
215	0\\
216	0\\
217	0\\
218	0\\
219	0\\
220	0\\
221	0\\
222	0\\
223	0\\
224	0\\
225	0\\
226	0\\
227	0\\
228	0\\
229	0\\
230	0\\
231	0\\
232	0\\
233	0\\
234	0\\
235	0\\
236	0\\
237	0\\
238	0\\
239	0\\
240	0\\
241	0\\
242	0\\
243	0\\
244	0\\
245	0\\
246	0\\
247	0\\
248	0\\
249	0\\
250	0\\
251	0\\
252	0\\
253	0\\
254	0\\
255	0\\
256	0\\
257	0\\
258	0\\
259	0\\
260	0\\
261	0\\
262	0\\
263	0\\
264	0\\
265	0\\
266	0\\
267	0\\
268	0\\
269	0\\
270	0\\
271	0\\
272	0\\
273	0\\
274	0\\
275	0\\
276	0\\
277	0\\
278	0\\
279	0\\
280	0\\
281	0\\
282	0\\
283	0\\
284	0\\
285	0\\
286	0\\
287	0\\
288	0\\
289	0\\
290	0\\
291	0\\
292	0\\
293	0\\
294	0\\
295	0\\
296	0\\
297	0\\
298	0\\
299	0\\
300	0\\
301	0\\
302	0\\
303	0\\
304	0\\
305	0\\
306	0\\
307	0\\
308	0\\
309	0\\
310	0\\
311	0\\
312	0\\
313	0\\
314	0\\
315	0\\
316	0\\
317	0\\
318	0\\
319	0\\
320	0\\
321	0\\
322	0\\
323	0\\
324	0\\
325	0\\
326	0\\
327	0\\
328	0\\
329	0\\
330	0\\
331	0\\
332	0\\
333	0\\
334	0\\
335	0\\
336	0\\
337	0\\
338	0\\
339	0\\
340	0\\
341	0\\
342	0\\
343	0\\
344	0\\
345	0\\
346	0\\
347	0\\
348	0\\
349	0\\
350	0\\
351	0\\
352	0\\
353	0\\
354	0\\
355	0\\
356	0\\
357	0\\
358	0\\
359	0\\
360	0\\
361	0\\
362	0\\
363	0\\
364	0\\
365	0\\
366	0\\
367	0\\
368	0\\
369	0\\
370	0\\
371	0\\
372	0\\
373	0\\
374	0\\
375	0\\
376	0\\
377	0\\
378	0\\
379	0\\
380	0\\
381	0\\
382	0\\
383	0\\
384	0\\
385	0\\
386	0\\
387	0\\
388	0\\
389	0\\
390	0\\
391	0\\
392	0\\
393	0\\
394	0\\
395	0\\
396	0\\
397	0\\
398	0\\
399	0\\
400	0\\
401	0\\
402	0\\
403	0\\
404	0\\
405	0\\
406	0\\
407	0\\
408	0\\
409	0\\
410	0\\
411	0\\
412	0\\
413	0\\
414	0\\
415	0\\
416	0\\
417	0\\
418	0\\
419	0\\
420	0\\
421	0\\
422	0\\
423	0\\
424	0\\
425	0\\
426	0\\
427	0\\
428	0\\
429	0\\
430	0\\
431	0\\
432	0\\
433	0\\
434	0\\
435	0\\
436	0\\
437	0\\
438	0\\
439	0\\
440	0\\
441	0\\
442	0\\
443	0\\
444	0\\
445	0\\
446	0\\
447	0\\
448	0\\
449	0\\
450	0\\
451	0\\
452	0\\
453	0\\
454	0\\
455	0\\
456	0\\
457	0\\
458	0\\
459	0\\
460	0\\
461	0\\
462	0\\
463	0\\
464	0\\
465	0\\
466	0\\
467	0\\
468	0\\
469	0\\
470	0\\
471	0\\
472	0\\
473	0\\
474	0\\
475	0\\
476	0\\
477	0\\
478	0\\
479	0\\
480	0\\
481	0\\
482	0\\
483	0\\
484	0\\
485	0\\
486	0\\
487	0\\
488	0\\
489	0\\
490	0\\
491	0\\
492	0\\
493	0\\
494	0\\
495	0\\
496	0\\
497	0\\
498	0\\
499	0\\
500	0\\
501	0\\
502	0\\
503	0\\
504	0\\
505	0\\
506	0\\
507	0\\
508	0\\
509	0\\
510	0\\
511	0\\
512	0\\
513	0\\
514	0\\
515	0\\
516	0\\
517	0\\
518	0\\
519	0\\
520	0\\
521	0\\
522	0\\
523	0\\
524	0\\
525	0\\
526	0\\
527	0\\
528	0\\
529	0\\
530	0\\
531	0\\
532	0\\
533	0\\
534	0\\
535	0\\
536	0\\
537	1.41981037307091e-05\\
538	4.03194267452914e-05\\
539	6.68049003892867e-05\\
540	9.36455912896163e-05\\
541	0.000121031570767503\\
542	0.000148981487706946\\
543	0.000177511834873603\\
544	0.000206636133043701\\
545	0.00023636795016899\\
546	0.000266721132093178\\
547	0.000297709825439374\\
548	0.000329348337773177\\
549	0.000361650643511393\\
550	0.000394628923195159\\
551	0.000428289376922921\\
552	0.000462620483140471\\
553	0.000497556235318828\\
554	0.000534404821662524\\
555	0.00057682151016282\\
556	0.000612670727538693\\
557	0.000647448601386753\\
558	0.000682535085499909\\
559	0.000718276611240496\\
560	0.000754689045844615\\
561	0.000791786693129664\\
562	0.000829584407608802\\
563	0.000868098067449268\\
564	0.00090734463744055\\
565	0.00128398061588739\\
566	0.00159228887629074\\
567	0.00166875568695174\\
568	0.00173042230411937\\
569	0.00179304268794214\\
570	0.00185663517097841\\
571	0.00192121867131843\\
572	0.00198681272227957\\
573	0.0020534375008095\\
574	0.00212111385632287\\
575	0.00218986333969204\\
576	0.00225970823205636\\
577	0.00233067157312388\\
578	0.00240277718884235\\
579	0.00247604971905763\\
580	0.00255051464798387\\
581	0.00262619834639329\\
582	0.00270312815070207\\
583	0.00278133254700611\\
584	0.00286084164033729\\
585	0.00294168838198347\\
586	0.00302391178876898\\
587	0.00310756536466311\\
588	0.0031927390611031\\
589	0.00327961639013215\\
590	0.00336862265377153\\
591	0.00346039684658503\\
592	0.00355653402929193\\
593	0.00366119131364762\\
594	0.00378527046594782\\
595	0.00395742119285503\\
596	0.00425297862742963\\
597	0.0048700418989439\\
598	0.00632942537858856\\
599	0\\
600	0\\
};
\addplot [color=black!80!mycolor21,solid,forget plot]
  table[row sep=crcr]{%
1	0\\
2	0\\
3	0\\
4	0\\
5	0\\
6	0\\
7	0\\
8	0\\
9	0\\
10	0\\
11	0\\
12	0\\
13	0\\
14	0\\
15	0\\
16	0\\
17	0\\
18	0\\
19	0\\
20	0\\
21	0\\
22	0\\
23	0\\
24	0\\
25	0\\
26	0\\
27	0\\
28	0\\
29	0\\
30	0\\
31	0\\
32	0\\
33	0\\
34	0\\
35	0\\
36	0\\
37	0\\
38	0\\
39	0\\
40	0\\
41	0\\
42	0\\
43	0\\
44	0\\
45	0\\
46	0\\
47	0\\
48	0\\
49	0\\
50	0\\
51	0\\
52	0\\
53	0\\
54	0\\
55	0\\
56	0\\
57	0\\
58	0\\
59	0\\
60	0\\
61	0\\
62	0\\
63	0\\
64	0\\
65	0\\
66	0\\
67	0\\
68	0\\
69	0\\
70	0\\
71	0\\
72	0\\
73	0\\
74	0\\
75	0\\
76	0\\
77	0\\
78	0\\
79	0\\
80	0\\
81	0\\
82	0\\
83	0\\
84	0\\
85	0\\
86	0\\
87	0\\
88	0\\
89	0\\
90	0\\
91	0\\
92	0\\
93	0\\
94	0\\
95	0\\
96	0\\
97	0\\
98	0\\
99	0\\
100	0\\
101	0\\
102	0\\
103	0\\
104	0\\
105	0\\
106	0\\
107	0\\
108	0\\
109	0\\
110	0\\
111	0\\
112	0\\
113	0\\
114	0\\
115	0\\
116	0\\
117	0\\
118	0\\
119	0\\
120	0\\
121	0\\
122	0\\
123	0\\
124	0\\
125	0\\
126	0\\
127	0\\
128	0\\
129	0\\
130	0\\
131	0\\
132	0\\
133	0\\
134	0\\
135	0\\
136	0\\
137	0\\
138	0\\
139	0\\
140	0\\
141	0\\
142	0\\
143	0\\
144	0\\
145	0\\
146	0\\
147	0\\
148	0\\
149	0\\
150	0\\
151	0\\
152	0\\
153	0\\
154	0\\
155	0\\
156	0\\
157	0\\
158	0\\
159	0\\
160	0\\
161	0\\
162	0\\
163	0\\
164	0\\
165	0\\
166	0\\
167	0\\
168	0\\
169	0\\
170	0\\
171	0\\
172	0\\
173	0\\
174	0\\
175	0\\
176	0\\
177	0\\
178	0\\
179	0\\
180	0\\
181	0\\
182	0\\
183	0\\
184	0\\
185	0\\
186	0\\
187	0\\
188	0\\
189	0\\
190	0\\
191	0\\
192	0\\
193	0\\
194	0\\
195	0\\
196	0\\
197	0\\
198	0\\
199	0\\
200	0\\
201	0\\
202	0\\
203	0\\
204	0\\
205	0\\
206	0\\
207	0\\
208	0\\
209	0\\
210	0\\
211	0\\
212	0\\
213	0\\
214	0\\
215	0\\
216	0\\
217	0\\
218	0\\
219	0\\
220	0\\
221	0\\
222	0\\
223	0\\
224	0\\
225	0\\
226	0\\
227	0\\
228	0\\
229	0\\
230	0\\
231	0\\
232	0\\
233	0\\
234	0\\
235	0\\
236	0\\
237	0\\
238	0\\
239	0\\
240	0\\
241	0\\
242	0\\
243	0\\
244	0\\
245	0\\
246	0\\
247	0\\
248	0\\
249	0\\
250	0\\
251	0\\
252	0\\
253	0\\
254	0\\
255	0\\
256	0\\
257	0\\
258	0\\
259	0\\
260	0\\
261	0\\
262	0\\
263	0\\
264	0\\
265	0\\
266	0\\
267	0\\
268	0\\
269	0\\
270	0\\
271	0\\
272	0\\
273	0\\
274	0\\
275	0\\
276	0\\
277	0\\
278	0\\
279	0\\
280	0\\
281	0\\
282	0\\
283	0\\
284	0\\
285	0\\
286	0\\
287	0\\
288	0\\
289	0\\
290	0\\
291	0\\
292	0\\
293	0\\
294	0\\
295	0\\
296	0\\
297	0\\
298	0\\
299	0\\
300	0\\
301	0\\
302	0\\
303	0\\
304	0\\
305	0\\
306	0\\
307	0\\
308	0\\
309	0\\
310	0\\
311	0\\
312	0\\
313	0\\
314	0\\
315	0\\
316	0\\
317	0\\
318	0\\
319	0\\
320	0\\
321	0\\
322	0\\
323	0\\
324	0\\
325	0\\
326	0\\
327	0\\
328	0\\
329	0\\
330	0\\
331	0\\
332	0\\
333	0\\
334	0\\
335	0\\
336	0\\
337	0\\
338	0\\
339	0\\
340	0\\
341	0\\
342	0\\
343	0\\
344	0\\
345	0\\
346	0\\
347	0\\
348	0\\
349	0\\
350	0\\
351	0\\
352	0\\
353	0\\
354	0\\
355	0\\
356	0\\
357	0\\
358	0\\
359	0\\
360	0\\
361	0\\
362	0\\
363	0\\
364	0\\
365	0\\
366	0\\
367	0\\
368	0\\
369	0\\
370	0\\
371	0\\
372	0\\
373	0\\
374	0\\
375	0\\
376	0\\
377	0\\
378	0\\
379	0\\
380	0\\
381	0\\
382	0\\
383	0\\
384	0\\
385	0\\
386	0\\
387	0\\
388	0\\
389	0\\
390	0\\
391	0\\
392	0\\
393	0\\
394	0\\
395	0\\
396	0\\
397	0\\
398	0\\
399	0\\
400	0\\
401	0\\
402	0\\
403	0\\
404	0\\
405	0\\
406	0\\
407	0\\
408	0\\
409	0\\
410	0\\
411	0\\
412	0\\
413	0\\
414	0\\
415	0\\
416	0\\
417	0\\
418	0\\
419	0\\
420	0\\
421	0\\
422	0\\
423	0\\
424	0\\
425	0\\
426	0\\
427	0\\
428	0\\
429	0\\
430	0\\
431	0\\
432	0\\
433	0\\
434	0\\
435	0\\
436	0\\
437	0\\
438	0\\
439	0\\
440	0\\
441	0\\
442	0\\
443	0\\
444	0\\
445	0\\
446	0\\
447	0\\
448	0\\
449	0\\
450	0\\
451	0\\
452	0\\
453	0\\
454	0\\
455	0\\
456	0\\
457	0\\
458	0\\
459	0\\
460	0\\
461	0\\
462	0\\
463	0\\
464	0\\
465	0\\
466	0\\
467	0\\
468	0\\
469	0\\
470	0\\
471	0\\
472	0\\
473	0\\
474	0\\
475	0\\
476	0\\
477	0\\
478	0\\
479	0\\
480	0\\
481	0\\
482	0\\
483	0\\
484	0\\
485	0\\
486	0\\
487	0\\
488	0\\
489	0\\
490	0\\
491	0\\
492	0\\
493	0\\
494	0\\
495	0\\
496	0\\
497	0\\
498	0\\
499	0\\
500	0\\
501	0\\
502	0\\
503	0\\
504	0\\
505	0\\
506	0\\
507	0\\
508	0\\
509	0\\
510	0\\
511	0\\
512	0\\
513	0\\
514	0\\
515	0\\
516	0\\
517	0\\
518	0\\
519	0\\
520	0\\
521	0\\
522	0\\
523	0\\
524	0\\
525	0\\
526	0\\
527	0\\
528	0\\
529	0\\
530	0\\
531	0\\
532	0\\
533	0\\
534	0\\
535	0\\
536	0\\
537	2.00311818478477e-05\\
538	4.58947831388607e-05\\
539	7.22859642090788e-05\\
540	9.92227167136692e-05\\
541	0.000126717645174376\\
542	0.000154783396446169\\
543	0.000183432792668513\\
544	0.000212678988439243\\
545	0.000242535495908309\\
546	0.000273016183234799\\
547	0.000304135226905855\\
548	0.000335906938006349\\
549	0.000368345226021222\\
550	0.000401462007738327\\
551	0.00043526250918298\\
552	0.00046973121451101\\
553	0.000507806233402199\\
554	0.000547931958570056\\
555	0.000581827365374636\\
556	0.000615782626155859\\
557	0.000650269640106297\\
558	0.000685400368111508\\
559	0.000721188332977737\\
560	0.000757647340897565\\
561	0.00079479173610764\\
562	0.000832636728120108\\
563	0.000871198966466937\\
564	0.00118295200706542\\
565	0.00153193885094584\\
566	0.00160802507255323\\
567	0.00166875568906997\\
568	0.00173042230418516\\
569	0.0017930426879713\\
570	0.00185663517099282\\
571	0.00192121867132536\\
572	0.00198681272228273\\
573	0.00205343750081087\\
574	0.00212111385632341\\
575	0.00218986333969228\\
576	0.00225970823205644\\
577	0.0023306715731239\\
578	0.00240277718884234\\
579	0.00247604971905764\\
580	0.00255051464798387\\
581	0.00262619834639331\\
582	0.00270312815070208\\
583	0.00278133254700611\\
584	0.00286084164033728\\
585	0.00294168838198347\\
586	0.003023911788769\\
587	0.00310756536466313\\
588	0.00319273906110312\\
589	0.00327961639013215\\
590	0.00336862265377153\\
591	0.00346039684658503\\
592	0.00355653402929195\\
593	0.00366119131364764\\
594	0.00378527046594783\\
595	0.00395742119285504\\
596	0.00425297862742962\\
597	0.00487004189894391\\
598	0.00632942537858856\\
599	0\\
600	0\\
};
\addplot [color=black,solid,forget plot]
  table[row sep=crcr]{%
1	0\\
2	0\\
3	0\\
4	0\\
5	0\\
6	0\\
7	0\\
8	0\\
9	0\\
10	0\\
11	0\\
12	0\\
13	0\\
14	0\\
15	0\\
16	0\\
17	0\\
18	0\\
19	0\\
20	0\\
21	0\\
22	0\\
23	0\\
24	0\\
25	0\\
26	0\\
27	0\\
28	0\\
29	0\\
30	0\\
31	0\\
32	0\\
33	0\\
34	0\\
35	0\\
36	0\\
37	0\\
38	0\\
39	0\\
40	0\\
41	0\\
42	0\\
43	0\\
44	0\\
45	0\\
46	0\\
47	0\\
48	0\\
49	0\\
50	0\\
51	0\\
52	0\\
53	0\\
54	0\\
55	0\\
56	0\\
57	0\\
58	0\\
59	0\\
60	0\\
61	0\\
62	0\\
63	0\\
64	0\\
65	0\\
66	0\\
67	0\\
68	0\\
69	0\\
70	0\\
71	0\\
72	0\\
73	0\\
74	0\\
75	0\\
76	0\\
77	0\\
78	0\\
79	0\\
80	0\\
81	0\\
82	0\\
83	0\\
84	0\\
85	0\\
86	0\\
87	0\\
88	0\\
89	0\\
90	0\\
91	0\\
92	0\\
93	0\\
94	0\\
95	0\\
96	0\\
97	0\\
98	0\\
99	0\\
100	0\\
101	0\\
102	0\\
103	0\\
104	0\\
105	0\\
106	0\\
107	0\\
108	0\\
109	0\\
110	0\\
111	0\\
112	0\\
113	0\\
114	0\\
115	0\\
116	0\\
117	0\\
118	0\\
119	0\\
120	0\\
121	0\\
122	0\\
123	0\\
124	0\\
125	0\\
126	0\\
127	0\\
128	0\\
129	0\\
130	0\\
131	0\\
132	0\\
133	0\\
134	0\\
135	0\\
136	0\\
137	0\\
138	0\\
139	0\\
140	0\\
141	0\\
142	0\\
143	0\\
144	0\\
145	0\\
146	0\\
147	0\\
148	0\\
149	0\\
150	0\\
151	0\\
152	0\\
153	0\\
154	0\\
155	0\\
156	0\\
157	0\\
158	0\\
159	0\\
160	0\\
161	0\\
162	0\\
163	0\\
164	0\\
165	0\\
166	0\\
167	0\\
168	0\\
169	0\\
170	0\\
171	0\\
172	0\\
173	0\\
174	0\\
175	0\\
176	0\\
177	0\\
178	0\\
179	0\\
180	0\\
181	0\\
182	0\\
183	0\\
184	0\\
185	0\\
186	0\\
187	0\\
188	0\\
189	0\\
190	0\\
191	0\\
192	0\\
193	0\\
194	0\\
195	0\\
196	0\\
197	0\\
198	0\\
199	0\\
200	0\\
201	0\\
202	0\\
203	0\\
204	0\\
205	0\\
206	0\\
207	0\\
208	0\\
209	0\\
210	0\\
211	0\\
212	0\\
213	0\\
214	0\\
215	0\\
216	0\\
217	0\\
218	0\\
219	0\\
220	0\\
221	0\\
222	0\\
223	0\\
224	0\\
225	0\\
226	0\\
227	0\\
228	0\\
229	0\\
230	0\\
231	0\\
232	0\\
233	0\\
234	0\\
235	0\\
236	0\\
237	0\\
238	0\\
239	0\\
240	0\\
241	0\\
242	0\\
243	0\\
244	0\\
245	0\\
246	0\\
247	0\\
248	0\\
249	0\\
250	0\\
251	0\\
252	0\\
253	0\\
254	0\\
255	0\\
256	0\\
257	0\\
258	0\\
259	0\\
260	0\\
261	0\\
262	0\\
263	0\\
264	0\\
265	0\\
266	0\\
267	0\\
268	0\\
269	0\\
270	0\\
271	0\\
272	0\\
273	0\\
274	0\\
275	0\\
276	0\\
277	0\\
278	0\\
279	0\\
280	0\\
281	0\\
282	0\\
283	0\\
284	0\\
285	0\\
286	0\\
287	0\\
288	0\\
289	0\\
290	0\\
291	0\\
292	0\\
293	0\\
294	0\\
295	0\\
296	0\\
297	0\\
298	0\\
299	0\\
300	0\\
301	0\\
302	0\\
303	0\\
304	0\\
305	0\\
306	0\\
307	0\\
308	0\\
309	0\\
310	0\\
311	0\\
312	0\\
313	0\\
314	0\\
315	0\\
316	0\\
317	0\\
318	0\\
319	0\\
320	0\\
321	0\\
322	0\\
323	0\\
324	0\\
325	0\\
326	0\\
327	0\\
328	0\\
329	0\\
330	0\\
331	0\\
332	0\\
333	0\\
334	0\\
335	0\\
336	0\\
337	0\\
338	0\\
339	0\\
340	0\\
341	0\\
342	0\\
343	0\\
344	0\\
345	0\\
346	0\\
347	0\\
348	0\\
349	0\\
350	0\\
351	0\\
352	0\\
353	0\\
354	0\\
355	0\\
356	0\\
357	0\\
358	0\\
359	0\\
360	0\\
361	0\\
362	0\\
363	0\\
364	0\\
365	0\\
366	0\\
367	0\\
368	0\\
369	0\\
370	0\\
371	0\\
372	0\\
373	0\\
374	0\\
375	0\\
376	0\\
377	0\\
378	0\\
379	0\\
380	0\\
381	0\\
382	0\\
383	0\\
384	0\\
385	0\\
386	0\\
387	0\\
388	0\\
389	0\\
390	0\\
391	0\\
392	0\\
393	0\\
394	0\\
395	0\\
396	0\\
397	0\\
398	0\\
399	0\\
400	0\\
401	0\\
402	0\\
403	0\\
404	0\\
405	0\\
406	0\\
407	0\\
408	0\\
409	0\\
410	0\\
411	0\\
412	0\\
413	0\\
414	0\\
415	0\\
416	0\\
417	0\\
418	0\\
419	0\\
420	0\\
421	0\\
422	0\\
423	0\\
424	0\\
425	0\\
426	0\\
427	0\\
428	0\\
429	0\\
430	0\\
431	0\\
432	0\\
433	0\\
434	0\\
435	0\\
436	0\\
437	0\\
438	0\\
439	0\\
440	0\\
441	0\\
442	0\\
443	0\\
444	0\\
445	0\\
446	0\\
447	0\\
448	0\\
449	0\\
450	0\\
451	0\\
452	0\\
453	0\\
454	0\\
455	0\\
456	0\\
457	0\\
458	0\\
459	0\\
460	0\\
461	0\\
462	0\\
463	0\\
464	0\\
465	0\\
466	0\\
467	0\\
468	0\\
469	0\\
470	0\\
471	0\\
472	0\\
473	0\\
474	0\\
475	0\\
476	0\\
477	0\\
478	0\\
479	0\\
480	0\\
481	0\\
482	0\\
483	0\\
484	0\\
485	0\\
486	0\\
487	0\\
488	0\\
489	0\\
490	0\\
491	0\\
492	0\\
493	0\\
494	0\\
495	0\\
496	0\\
497	0\\
498	0\\
499	0\\
500	0\\
501	0\\
502	0\\
503	0\\
504	0\\
505	0\\
506	0\\
507	0\\
508	0\\
509	0\\
510	0\\
511	0\\
512	0\\
513	0\\
514	0\\
515	0\\
516	0\\
517	0\\
518	0\\
519	0\\
520	0\\
521	0\\
522	0\\
523	0\\
524	0\\
525	0\\
526	0\\
527	0\\
528	0\\
529	0\\
530	0\\
531	0\\
532	0\\
533	0\\
534	0\\
535	0\\
536	0\\
537	2.41449941958518e-05\\
538	5.01062475752778e-05\\
539	7.66056323086619e-05\\
540	0.000103655192469625\\
541	0.000131267283450877\\
542	0.000159454603022728\\
543	0.000188230216867344\\
544	0.0002176075760922\\
545	0.000247600530527473\\
546	0.000278223330210997\\
547	0.000309490592847296\\
548	0.000341417179393022\\
549	0.000374017835784068\\
550	0.000407306284469521\\
551	0.000441293207629602\\
552	0.000480387614922689\\
553	0.000517733822270981\\
554	0.000550900656801395\\
555	0.000584195800718177\\
556	0.000618106994392449\\
557	0.000652650864010514\\
558	0.000687840631956863\\
559	0.000723689894931791\\
560	0.000760212780169657\\
561	0.000797424370009976\\
562	0.000835341903111315\\
563	0.00104794067987499\\
564	0.0014700571270915\\
565	0.00154821321979401\\
566	0.00160802507270126\\
567	0.00166875568907878\\
568	0.00173042230418922\\
569	0.00179304268797327\\
570	0.00185663517099374\\
571	0.00192121867132575\\
572	0.00198681272228291\\
573	0.00205343750081092\\
574	0.00212111385632343\\
575	0.00218986333969225\\
576	0.00225970823205644\\
577	0.0023306715731239\\
578	0.00240277718884235\\
579	0.00247604971905765\\
580	0.00255051464798389\\
581	0.0026261983463933\\
582	0.00270312815070208\\
583	0.00278133254700612\\
584	0.00286084164033729\\
585	0.00294168838198346\\
586	0.00302391178876899\\
587	0.00310756536466312\\
588	0.00319273906110312\\
589	0.00327961639013216\\
590	0.00336862265377154\\
591	0.00346039684658502\\
592	0.00355653402929193\\
593	0.00366119131364762\\
594	0.00378527046594782\\
595	0.00395742119285502\\
596	0.00425297862742962\\
597	0.0048700418989439\\
598	0.00632942537858856\\
599	0\\
600	0\\
};
\end{axis}
\end{tikzpicture}%
 
%  \caption{Discrete Time}
%\end{subfigure}\\
%\vspace{1cm}
%\begin{subfigure}{.45\linewidth}
%  \centering
%  \setlength\figureheight{\linewidth} 
%  \setlength\figurewidth{\linewidth}
%  \tikzsetnextfilename{dp_cts_nFPC_z15}
%  % This file was created by matlab2tikz.
%
%The latest updates can be retrieved from
%  http://www.mathworks.com/matlabcentral/fileexchange/22022-matlab2tikz-matlab2tikz
%where you can also make suggestions and rate matlab2tikz.
%
\definecolor{mycolor1}{rgb}{0.00000,1.00000,0.14286}%
\definecolor{mycolor2}{rgb}{0.00000,1.00000,0.28571}%
\definecolor{mycolor3}{rgb}{0.00000,1.00000,0.42857}%
\definecolor{mycolor4}{rgb}{0.00000,1.00000,0.57143}%
\definecolor{mycolor5}{rgb}{0.00000,1.00000,0.71429}%
\definecolor{mycolor6}{rgb}{0.00000,1.00000,0.85714}%
\definecolor{mycolor7}{rgb}{0.00000,1.00000,1.00000}%
\definecolor{mycolor8}{rgb}{0.00000,0.87500,1.00000}%
\definecolor{mycolor9}{rgb}{0.00000,0.62500,1.00000}%
\definecolor{mycolor10}{rgb}{0.12500,0.00000,1.00000}%
\definecolor{mycolor11}{rgb}{0.25000,0.00000,1.00000}%
\definecolor{mycolor12}{rgb}{0.37500,0.00000,1.00000}%
\definecolor{mycolor13}{rgb}{0.50000,0.00000,1.00000}%
\definecolor{mycolor14}{rgb}{0.62500,0.00000,1.00000}%
\definecolor{mycolor15}{rgb}{0.75000,0.00000,1.00000}%
\definecolor{mycolor16}{rgb}{0.87500,0.00000,1.00000}%
\definecolor{mycolor17}{rgb}{1.00000,0.00000,1.00000}%
\definecolor{mycolor18}{rgb}{1.00000,0.00000,0.87500}%
\definecolor{mycolor19}{rgb}{1.00000,0.00000,0.62500}%
\definecolor{mycolor20}{rgb}{0.85714,0.00000,0.00000}%
\definecolor{mycolor21}{rgb}{0.71429,0.00000,0.00000}%
%
\begin{tikzpicture}[trim axis left, trim axis right]

\begin{axis}[%
width=\figurewidth,
height=\figureheight,
at={(0\figurewidth,0\figureheight)},
scale only axis,
every outer x axis line/.append style={black},
every x tick label/.append style={font=\color{black}},
xmin=0,
xmax=600,
every outer y axis line/.append style={black},
every y tick label/.append style={font=\color{black}},
ymin=0,
ymax=0.014,
axis background/.style={fill=white},
axis x line*=bottom,
axis y line*=left,
yticklabel style={
        /pgf/number format/fixed,
        /pgf/number format/precision=3
},
scaled y ticks=false
]
\addplot [color=green,solid,forget plot]
  table[row sep=crcr]{%
0.01	0.01\\
1.01	0.01\\
2.01	0.01\\
3.01	0.01\\
4.01	0.01\\
5.01	0.01\\
6.01	0.01\\
7.01	0.01\\
8.01	0.01\\
9.01	0.01\\
10.01	0.01\\
11.01	0.01\\
12.01	0.01\\
13.01	0.01\\
14.01	0.01\\
15.01	0.01\\
16.01	0.01\\
17.01	0.01\\
18.01	0.01\\
19.01	0.01\\
20.01	0.01\\
21.01	0.01\\
22.01	0.01\\
23.01	0.01\\
24.01	0.01\\
25.01	0.01\\
26.01	0.01\\
27.01	0.01\\
28.01	0.01\\
29.01	0.01\\
30.01	0.01\\
31.01	0.01\\
32.01	0.01\\
33.01	0.01\\
34.01	0.01\\
35.01	0.01\\
36.01	0.01\\
37.01	0.01\\
38.01	0.01\\
39.01	0.01\\
40.01	0.01\\
41.01	0.01\\
42.01	0.01\\
43.01	0.01\\
44.01	0.01\\
45.01	0.01\\
46.01	0.01\\
47.01	0.01\\
48.01	0.01\\
49.01	0.01\\
50.01	0.01\\
51.01	0.01\\
52.01	0.01\\
53.01	0.01\\
54.01	0.01\\
55.01	0.01\\
56.01	0.01\\
57.01	0.01\\
58.01	0.01\\
59.01	0.01\\
60.01	0.01\\
61.01	0.01\\
62.01	0.01\\
63.01	0.01\\
64.01	0.01\\
65.01	0.01\\
66.01	0.01\\
67.01	0.01\\
68.01	0.01\\
69.01	0.01\\
70.01	0.01\\
71.01	0.01\\
72.01	0.01\\
73.01	0.01\\
74.01	0.01\\
75.01	0.01\\
76.01	0.01\\
77.01	0.01\\
78.01	0.01\\
79.01	0.01\\
80.01	0.01\\
81.01	0.01\\
82.01	0.01\\
83.01	0.01\\
84.01	0.01\\
85.01	0.01\\
86.01	0.01\\
87.01	0.01\\
88.01	0.01\\
89.01	0.01\\
90.01	0.01\\
91.01	0.01\\
92.01	0.01\\
93.01	0.01\\
94.01	0.01\\
95.01	0.01\\
96.01	0.01\\
97.01	0.01\\
98.01	0.01\\
99.01	0.01\\
100.01	0.01\\
101.01	0.01\\
102.01	0.01\\
103.01	0.01\\
104.01	0.01\\
105.01	0.01\\
106.01	0.01\\
107.01	0.01\\
108.01	0.01\\
109.01	0.01\\
110.01	0.01\\
111.01	0.01\\
112.01	0.01\\
113.01	0.01\\
114.01	0.01\\
115.01	0.01\\
116.01	0.01\\
117.01	0.01\\
118.01	0.01\\
119.01	0.01\\
120.01	0.01\\
121.01	0.01\\
122.01	0.01\\
123.01	0.01\\
124.01	0.01\\
125.01	0.01\\
126.01	0.01\\
127.01	0.01\\
128.01	0.01\\
129.01	0.01\\
130.01	0.01\\
131.01	0.01\\
132.01	0.01\\
133.01	0.01\\
134.01	0.01\\
135.01	0.01\\
136.01	0.01\\
137.01	0.01\\
138.01	0.01\\
139.01	0.01\\
140.01	0.01\\
141.01	0.01\\
142.01	0.01\\
143.01	0.01\\
144.01	0.01\\
145.01	0.01\\
146.01	0.01\\
147.01	0.01\\
148.01	0.01\\
149.01	0.01\\
150.01	0.01\\
151.01	0.01\\
152.01	0.01\\
153.01	0.01\\
154.01	0.01\\
155.01	0.01\\
156.01	0.01\\
157.01	0.01\\
158.01	0.01\\
159.01	0.01\\
160.01	0.01\\
161.01	0.01\\
162.01	0.01\\
163.01	0.01\\
164.01	0.01\\
165.01	0.01\\
166.01	0.01\\
167.01	0.01\\
168.01	0.01\\
169.01	0.01\\
170.01	0.01\\
171.01	0.01\\
172.01	0.01\\
173.01	0.01\\
174.01	0.01\\
175.01	0.01\\
176.01	0.01\\
177.01	0.01\\
178.01	0.01\\
179.01	0.01\\
180.01	0.01\\
181.01	0.01\\
182.01	0.01\\
183.01	0.01\\
184.01	0.01\\
185.01	0.01\\
186.01	0.01\\
187.01	0.01\\
188.01	0.01\\
189.01	0.01\\
190.01	0.01\\
191.01	0.01\\
192.01	0.01\\
193.01	0.01\\
194.01	0.01\\
195.01	0.01\\
196.01	0.01\\
197.01	0.01\\
198.01	0.01\\
199.01	0.01\\
200.01	0.01\\
201.01	0.01\\
202.01	0.01\\
203.01	0.01\\
204.01	0.01\\
205.01	0.01\\
206.01	0.01\\
207.01	0.01\\
208.01	0.01\\
209.01	0.01\\
210.01	0.01\\
211.01	0.01\\
212.01	0.01\\
213.01	0.01\\
214.01	0.01\\
215.01	0.01\\
216.01	0.01\\
217.01	0.01\\
218.01	0.01\\
219.01	0.01\\
220.01	0.01\\
221.01	0.01\\
222.01	0.01\\
223.01	0.01\\
224.01	0.01\\
225.01	0.01\\
226.01	0.01\\
227.01	0.01\\
228.01	0.01\\
229.01	0.01\\
230.01	0.01\\
231.01	0.01\\
232.01	0.01\\
233.01	0.01\\
234.01	0.01\\
235.01	0.01\\
236.01	0.01\\
237.01	0.01\\
238.01	0.01\\
239.01	0.01\\
240.01	0.01\\
241.01	0.01\\
242.01	0.01\\
243.01	0.01\\
244.01	0.01\\
245.01	0.01\\
246.01	0.01\\
247.01	0.01\\
248.01	0.01\\
249.01	0.01\\
250.01	0.01\\
251.01	0.01\\
252.01	0.01\\
253.01	0.01\\
254.01	0.01\\
255.01	0.01\\
256.01	0.01\\
257.01	0.01\\
258.01	0.01\\
259.01	0.01\\
260.01	0.01\\
261.01	0.01\\
262.01	0.01\\
263.01	0.01\\
264.01	0.01\\
265.01	0.01\\
266.01	0.01\\
267.01	0.01\\
268.01	0.01\\
269.01	0.01\\
270.01	0.01\\
271.01	0.01\\
272.01	0.01\\
273.01	0.01\\
274.01	0.01\\
275.01	0.01\\
276.01	0.01\\
277.01	0.01\\
278.01	0.01\\
279.01	0.01\\
280.01	0.01\\
281.01	0.01\\
282.01	0.01\\
283.01	0.01\\
284.01	0.01\\
285.01	0.01\\
286.01	0.01\\
287.01	0.01\\
288.01	0.01\\
289.01	0.01\\
290.01	0.01\\
291.01	0.01\\
292.01	0.01\\
293.01	0.01\\
294.01	0.01\\
295.01	0.01\\
296.01	0.01\\
297.01	0.01\\
298.01	0.01\\
299.01	0.01\\
300.01	0.01\\
301.01	0.01\\
302.01	0.01\\
303.01	0.01\\
304.01	0.01\\
305.01	0.01\\
306.01	0.01\\
307.01	0.01\\
308.01	0.01\\
309.01	0.01\\
310.01	0.01\\
311.01	0.01\\
312.01	0.01\\
313.01	0.01\\
314.01	0.01\\
315.01	0.01\\
316.01	0.01\\
317.01	0.01\\
318.01	0.01\\
319.01	0.01\\
320.01	0.01\\
321.01	0.01\\
322.01	0.01\\
323.01	0.01\\
324.01	0.01\\
325.01	0.01\\
326.01	0.01\\
327.01	0.01\\
328.01	0.01\\
329.01	0.01\\
330.01	0.01\\
331.01	0.01\\
332.01	0.01\\
333.01	0.01\\
334.01	0.01\\
335.01	0.01\\
336.01	0.01\\
337.01	0.01\\
338.01	0.01\\
339.01	0.01\\
340.01	0.01\\
341.01	0.01\\
342.01	0.01\\
343.01	0.01\\
344.01	0.01\\
345.01	0.01\\
346.01	0.01\\
347.01	0.01\\
348.01	0.01\\
349.01	0.01\\
350.01	0.01\\
351.01	0.01\\
352.01	0.01\\
353.01	0.01\\
354.01	0.01\\
355.01	0.01\\
356.01	0.01\\
357.01	0.01\\
358.01	0.01\\
359.01	0.01\\
360.01	0.01\\
361.01	0.01\\
362.01	0.01\\
363.01	0.01\\
364.01	0.01\\
365.01	0.01\\
366.01	0.01\\
367.01	0.01\\
368.01	0.01\\
369.01	0.01\\
370.01	0.01\\
371.01	0.01\\
372.01	0.01\\
373.01	0.01\\
374.01	0.01\\
375.01	0.01\\
376.01	0.01\\
377.01	0.01\\
378.01	0.01\\
379.01	0.01\\
380.01	0.01\\
381.01	0.01\\
382.01	0.01\\
383.01	0.01\\
384.01	0.01\\
385.01	0.01\\
386.01	0.01\\
387.01	0.01\\
388.01	0.01\\
389.01	0.01\\
390.01	0.01\\
391.01	0.01\\
392.01	0.01\\
393.01	0.01\\
394.01	0.01\\
395.01	0.01\\
396.01	0.01\\
397.01	0.01\\
398.01	0.01\\
399.01	0.01\\
400.01	0.01\\
401.01	0.01\\
402.01	0.01\\
403.01	0.01\\
404.01	0.01\\
405.01	0.01\\
406.01	0.01\\
407.01	0.01\\
408.01	0.01\\
409.01	0.01\\
410.01	0.01\\
411.01	0.01\\
412.01	0.01\\
413.01	0.01\\
414.01	0.01\\
415.01	0.01\\
416.01	0.01\\
417.01	0.01\\
418.01	0.01\\
419.01	0.01\\
420.01	0.01\\
421.01	0.01\\
422.01	0.01\\
423.01	0.01\\
424.01	0.01\\
425.01	0.01\\
426.01	0.01\\
427.01	0.01\\
428.01	0.01\\
429.01	0.01\\
430.01	0.01\\
431.01	0.01\\
432.01	0.01\\
433.01	0.01\\
434.01	0.01\\
435.01	0.01\\
436.01	0.01\\
437.01	0.01\\
438.01	0.01\\
439.01	0.01\\
440.01	0.01\\
441.01	0.01\\
442.01	0.01\\
443.01	0.01\\
444.01	0.01\\
445.01	0.01\\
446.01	0.01\\
447.01	0.01\\
448.01	0.01\\
449.01	0.01\\
450.01	0.01\\
451.01	0.01\\
452.01	0.01\\
453.01	0.01\\
454.01	0.01\\
455.01	0.01\\
456.01	0.01\\
457.01	0.01\\
458.01	0.01\\
459.01	0.01\\
460.01	0.01\\
461.01	0.01\\
462.01	0.01\\
463.01	0.01\\
464.01	0.01\\
465.01	0.01\\
466.01	0.01\\
467.01	0.01\\
468.01	0.01\\
469.01	0.01\\
470.01	0.01\\
471.01	0.01\\
472.01	0.01\\
473.01	0.01\\
474.01	0.01\\
475.01	0.01\\
476.01	0.01\\
477.01	0.01\\
478.01	0.01\\
479.01	0.01\\
480.01	0.01\\
481.01	0.01\\
482.01	0.01\\
483.01	0.01\\
484.01	0.01\\
485.01	0.01\\
486.01	0.01\\
487.01	0.01\\
488.01	0.01\\
489.01	0.01\\
490.01	0.01\\
491.01	0.01\\
492.01	0.01\\
493.01	0.01\\
494.01	0.01\\
495.01	0.01\\
496.01	0.01\\
497.01	0.01\\
498.01	0.01\\
499.01	0.01\\
500.01	0.01\\
501.01	0.01\\
502.01	0.01\\
503.01	0.01\\
504.01	0.01\\
505.01	0.01\\
506.01	0.01\\
507.01	0.01\\
508.01	0.01\\
509.01	0.01\\
510.01	0.01\\
511.01	0.01\\
512.01	0.01\\
513.01	0.01\\
514.01	0.01\\
515.01	0.01\\
516.01	0.01\\
517.01	0.01\\
518.01	0.01\\
519.01	0.01\\
520.01	0.01\\
521.01	0.01\\
522.01	0.01\\
523.01	0.01\\
524.01	0.01\\
525.01	0.01\\
526.01	0.01\\
527.01	0.01\\
528.01	0.01\\
529.01	0.01\\
530.01	0.01\\
531.01	0.01\\
532.01	0.01\\
533.01	0.01\\
534.01	0.01\\
535.01	0.01\\
536.01	0.01\\
537.01	0.01\\
538.01	0.01\\
539.01	0.01\\
540.01	0.01\\
541.01	0.01\\
542.01	0.01\\
543.01	0.01\\
544.01	0.01\\
545.01	0.01\\
546.01	0.01\\
547.01	0.01\\
548.01	0.01\\
549.01	0.01\\
550.01	0.01\\
551.01	0.01\\
552.01	0.01\\
553.01	0.01\\
554.01	0.01\\
555.01	0.01\\
556.01	0.01\\
557.01	0.01\\
558.01	0.01\\
559.01	0.01\\
560.01	0.01\\
561.01	0.01\\
562.01	0.01\\
563.01	0.01\\
564.01	0.01\\
565.01	0.01\\
566.01	0.01\\
567.01	0.01\\
568.01	0.01\\
569.01	0.01\\
570.01	0.01\\
571.01	0.01\\
572.01	0.01\\
573.01	0.01\\
574.01	0.01\\
575.01	0.01\\
576.01	0.01\\
577.01	0.01\\
578.01	0.01\\
579.01	0.01\\
580.01	0.01\\
581.01	0.01\\
582.01	0.01\\
583.01	0.01\\
584.01	0.01\\
585.01	0.01\\
586.01	0.01\\
587.01	0.01\\
588.01	0.01\\
589.01	0.01\\
590.01	0.01\\
591.01	0.01\\
592.01	0.01\\
593.01	0.01\\
594.01	0.01\\
595.01	0.01\\
596.01	0.01\\
597.01	0.01\\
598.01	0.00863902290697456\\
599.01	0.00623513569400516\\
599.02	0.00619747359804885\\
599.03	0.00615944475593815\\
599.04	0.00612104556201828\\
599.05	0.00608227237517915\\
599.06	0.0060431215185067\\
599.07	0.00600358927893089\\
599.08	0.00596367190687008\\
599.09	0.00592336561587203\\
599.1	0.00588266658225147\\
599.11	0.00584157094472378\\
599.12	0.00580007480403545\\
599.13	0.00575817422259057\\
599.14	0.00571586522407394\\
599.15	0.00567314379307032\\
599.16	0.00563000587467982\\
599.17	0.0055864473741298\\
599.18	0.00554246415638276\\
599.19	0.00549805204574034\\
599.2	0.00545320682544357\\
599.21	0.00540792423726908\\
599.22	0.0053621999811214\\
599.23	0.0053160297146212\\
599.24	0.00526940905268953\\
599.25	0.00522233356712795\\
599.26	0.00517479878619452\\
599.27	0.00512680019417564\\
599.28	0.00507833323095367\\
599.29	0.0050293932915703\\
599.3	0.00497997572578562\\
599.31	0.00493007583763287\\
599.32	0.00487968888496881\\
599.33	0.00482881007901965\\
599.34	0.00477743458392257\\
599.35	0.00472555751626257\\
599.36	0.00467317394460506\\
599.37	0.00462027888902352\\
599.38	0.00456686732062279\\
599.39	0.00451293416105749\\
599.4	0.00445847428204579\\
599.41	0.00440348251095323\\
599.42	0.00434795363554174\\
599.43	0.00429188239249109\\
599.44	0.00423526346689831\\
599.45	0.00417809149177217\\
599.46	0.00412036104752264\\
599.47	0.00406206666144545\\
599.48	0.0040032028072016\\
599.49	0.00394376390429165\\
599.5	0.00388374431752493\\
599.51	0.00382313835648359\\
599.52	0.00376194027498136\\
599.53	0.00370014427051698\\
599.54	0.00363774448372229\\
599.55	0.00357473499780494\\
599.56	0.00351110983798562\\
599.57	0.00344686297092976\\
599.58	0.00338198830417367\\
599.59	0.00331647968554504\\
599.6	0.00325033090257784\\
599.61	0.00318353568192134\\
599.62	0.00311608768874351\\
599.63	0.00304798052612841\\
599.64	0.00297920773446781\\
599.65	0.00290976279084677\\
599.66	0.00283963910842317\\
599.67	0.00276883003580129\\
599.68	0.00269732885639913\\
599.69	0.00262512878780952\\
599.7	0.00255222298115511\\
599.71	0.00247860452043687\\
599.72	0.00240426642187631\\
599.73	0.00232920163325121\\
599.74	0.0022534030332249\\
599.75	0.00217686343066883\\
599.76	0.00209957556397869\\
599.77	0.00202153210038368\\
599.78	0.00194272563524906\\
599.79	0.00186314869137187\\
599.8	0.00178279371826977\\
599.81	0.00170165309146284\\
599.82	0.00161971911174842\\
599.83	0.00153698400446879\\
599.84	0.00145343991877172\\
599.85	0.00136907892686372\\
599.86	0.00128389302325598\\
599.87	0.001197874124003\\
599.88	0.00111101406593363\\
599.89	0.00102330460587469\\
599.9	0.000934737419866898\\
599.91	0.00084530410237319\\
599.92	0.000754996165479168\\
599.93	0.000663805038085871\\
599.94	0.000571722065094477\\
599.95	0.000478738506583132\\
599.96	0.000384845536975641\\
599.97	0.000290034244202046\\
599.98	0.00019429562885097\\
599.99	9.76206033136556e-05\\
600	0\\
};
\addplot [color=mycolor1,solid,forget plot]
  table[row sep=crcr]{%
0.01	0.01\\
1.01	0.01\\
2.01	0.01\\
3.01	0.01\\
4.01	0.01\\
5.01	0.01\\
6.01	0.01\\
7.01	0.01\\
8.01	0.01\\
9.01	0.01\\
10.01	0.01\\
11.01	0.01\\
12.01	0.01\\
13.01	0.01\\
14.01	0.01\\
15.01	0.01\\
16.01	0.01\\
17.01	0.01\\
18.01	0.01\\
19.01	0.01\\
20.01	0.01\\
21.01	0.01\\
22.01	0.01\\
23.01	0.01\\
24.01	0.01\\
25.01	0.01\\
26.01	0.01\\
27.01	0.01\\
28.01	0.01\\
29.01	0.01\\
30.01	0.01\\
31.01	0.01\\
32.01	0.01\\
33.01	0.01\\
34.01	0.01\\
35.01	0.01\\
36.01	0.01\\
37.01	0.01\\
38.01	0.01\\
39.01	0.01\\
40.01	0.01\\
41.01	0.01\\
42.01	0.01\\
43.01	0.01\\
44.01	0.01\\
45.01	0.01\\
46.01	0.01\\
47.01	0.01\\
48.01	0.01\\
49.01	0.01\\
50.01	0.01\\
51.01	0.01\\
52.01	0.01\\
53.01	0.01\\
54.01	0.01\\
55.01	0.01\\
56.01	0.01\\
57.01	0.01\\
58.01	0.01\\
59.01	0.01\\
60.01	0.01\\
61.01	0.01\\
62.01	0.01\\
63.01	0.01\\
64.01	0.01\\
65.01	0.01\\
66.01	0.01\\
67.01	0.01\\
68.01	0.01\\
69.01	0.01\\
70.01	0.01\\
71.01	0.01\\
72.01	0.01\\
73.01	0.01\\
74.01	0.01\\
75.01	0.01\\
76.01	0.01\\
77.01	0.01\\
78.01	0.01\\
79.01	0.01\\
80.01	0.01\\
81.01	0.01\\
82.01	0.01\\
83.01	0.01\\
84.01	0.01\\
85.01	0.01\\
86.01	0.01\\
87.01	0.01\\
88.01	0.01\\
89.01	0.01\\
90.01	0.01\\
91.01	0.01\\
92.01	0.01\\
93.01	0.01\\
94.01	0.01\\
95.01	0.01\\
96.01	0.01\\
97.01	0.01\\
98.01	0.01\\
99.01	0.01\\
100.01	0.01\\
101.01	0.01\\
102.01	0.01\\
103.01	0.01\\
104.01	0.01\\
105.01	0.01\\
106.01	0.01\\
107.01	0.01\\
108.01	0.01\\
109.01	0.01\\
110.01	0.01\\
111.01	0.01\\
112.01	0.01\\
113.01	0.01\\
114.01	0.01\\
115.01	0.01\\
116.01	0.01\\
117.01	0.01\\
118.01	0.01\\
119.01	0.01\\
120.01	0.01\\
121.01	0.01\\
122.01	0.01\\
123.01	0.01\\
124.01	0.01\\
125.01	0.01\\
126.01	0.01\\
127.01	0.01\\
128.01	0.01\\
129.01	0.01\\
130.01	0.01\\
131.01	0.01\\
132.01	0.01\\
133.01	0.01\\
134.01	0.01\\
135.01	0.01\\
136.01	0.01\\
137.01	0.01\\
138.01	0.01\\
139.01	0.01\\
140.01	0.01\\
141.01	0.01\\
142.01	0.01\\
143.01	0.01\\
144.01	0.01\\
145.01	0.01\\
146.01	0.01\\
147.01	0.01\\
148.01	0.01\\
149.01	0.01\\
150.01	0.01\\
151.01	0.01\\
152.01	0.01\\
153.01	0.01\\
154.01	0.01\\
155.01	0.01\\
156.01	0.01\\
157.01	0.01\\
158.01	0.01\\
159.01	0.01\\
160.01	0.01\\
161.01	0.01\\
162.01	0.01\\
163.01	0.01\\
164.01	0.01\\
165.01	0.01\\
166.01	0.01\\
167.01	0.01\\
168.01	0.01\\
169.01	0.01\\
170.01	0.01\\
171.01	0.01\\
172.01	0.01\\
173.01	0.01\\
174.01	0.01\\
175.01	0.01\\
176.01	0.01\\
177.01	0.01\\
178.01	0.01\\
179.01	0.01\\
180.01	0.01\\
181.01	0.01\\
182.01	0.01\\
183.01	0.01\\
184.01	0.01\\
185.01	0.01\\
186.01	0.01\\
187.01	0.01\\
188.01	0.01\\
189.01	0.01\\
190.01	0.01\\
191.01	0.01\\
192.01	0.01\\
193.01	0.01\\
194.01	0.01\\
195.01	0.01\\
196.01	0.01\\
197.01	0.01\\
198.01	0.01\\
199.01	0.01\\
200.01	0.01\\
201.01	0.01\\
202.01	0.01\\
203.01	0.01\\
204.01	0.01\\
205.01	0.01\\
206.01	0.01\\
207.01	0.01\\
208.01	0.01\\
209.01	0.01\\
210.01	0.01\\
211.01	0.01\\
212.01	0.01\\
213.01	0.01\\
214.01	0.01\\
215.01	0.01\\
216.01	0.01\\
217.01	0.01\\
218.01	0.01\\
219.01	0.01\\
220.01	0.01\\
221.01	0.01\\
222.01	0.01\\
223.01	0.01\\
224.01	0.01\\
225.01	0.01\\
226.01	0.01\\
227.01	0.01\\
228.01	0.01\\
229.01	0.01\\
230.01	0.01\\
231.01	0.01\\
232.01	0.01\\
233.01	0.01\\
234.01	0.01\\
235.01	0.01\\
236.01	0.01\\
237.01	0.01\\
238.01	0.01\\
239.01	0.01\\
240.01	0.01\\
241.01	0.01\\
242.01	0.01\\
243.01	0.01\\
244.01	0.01\\
245.01	0.01\\
246.01	0.01\\
247.01	0.01\\
248.01	0.01\\
249.01	0.01\\
250.01	0.01\\
251.01	0.01\\
252.01	0.01\\
253.01	0.01\\
254.01	0.01\\
255.01	0.01\\
256.01	0.01\\
257.01	0.01\\
258.01	0.01\\
259.01	0.01\\
260.01	0.01\\
261.01	0.01\\
262.01	0.01\\
263.01	0.01\\
264.01	0.01\\
265.01	0.01\\
266.01	0.01\\
267.01	0.01\\
268.01	0.01\\
269.01	0.01\\
270.01	0.01\\
271.01	0.01\\
272.01	0.01\\
273.01	0.01\\
274.01	0.01\\
275.01	0.01\\
276.01	0.01\\
277.01	0.01\\
278.01	0.01\\
279.01	0.01\\
280.01	0.01\\
281.01	0.01\\
282.01	0.01\\
283.01	0.01\\
284.01	0.01\\
285.01	0.01\\
286.01	0.01\\
287.01	0.01\\
288.01	0.01\\
289.01	0.01\\
290.01	0.01\\
291.01	0.01\\
292.01	0.01\\
293.01	0.01\\
294.01	0.01\\
295.01	0.01\\
296.01	0.01\\
297.01	0.01\\
298.01	0.01\\
299.01	0.01\\
300.01	0.01\\
301.01	0.01\\
302.01	0.01\\
303.01	0.01\\
304.01	0.01\\
305.01	0.01\\
306.01	0.01\\
307.01	0.01\\
308.01	0.01\\
309.01	0.01\\
310.01	0.01\\
311.01	0.01\\
312.01	0.01\\
313.01	0.01\\
314.01	0.01\\
315.01	0.01\\
316.01	0.01\\
317.01	0.01\\
318.01	0.01\\
319.01	0.01\\
320.01	0.01\\
321.01	0.01\\
322.01	0.01\\
323.01	0.01\\
324.01	0.01\\
325.01	0.01\\
326.01	0.01\\
327.01	0.01\\
328.01	0.01\\
329.01	0.01\\
330.01	0.01\\
331.01	0.01\\
332.01	0.01\\
333.01	0.01\\
334.01	0.01\\
335.01	0.01\\
336.01	0.01\\
337.01	0.01\\
338.01	0.01\\
339.01	0.01\\
340.01	0.01\\
341.01	0.01\\
342.01	0.01\\
343.01	0.01\\
344.01	0.01\\
345.01	0.01\\
346.01	0.01\\
347.01	0.01\\
348.01	0.01\\
349.01	0.01\\
350.01	0.01\\
351.01	0.01\\
352.01	0.01\\
353.01	0.01\\
354.01	0.01\\
355.01	0.01\\
356.01	0.01\\
357.01	0.01\\
358.01	0.01\\
359.01	0.01\\
360.01	0.01\\
361.01	0.01\\
362.01	0.01\\
363.01	0.01\\
364.01	0.01\\
365.01	0.01\\
366.01	0.01\\
367.01	0.01\\
368.01	0.01\\
369.01	0.01\\
370.01	0.01\\
371.01	0.01\\
372.01	0.01\\
373.01	0.01\\
374.01	0.01\\
375.01	0.01\\
376.01	0.01\\
377.01	0.01\\
378.01	0.01\\
379.01	0.01\\
380.01	0.01\\
381.01	0.01\\
382.01	0.01\\
383.01	0.01\\
384.01	0.01\\
385.01	0.01\\
386.01	0.01\\
387.01	0.01\\
388.01	0.01\\
389.01	0.01\\
390.01	0.01\\
391.01	0.01\\
392.01	0.01\\
393.01	0.01\\
394.01	0.01\\
395.01	0.01\\
396.01	0.01\\
397.01	0.01\\
398.01	0.01\\
399.01	0.01\\
400.01	0.01\\
401.01	0.01\\
402.01	0.01\\
403.01	0.01\\
404.01	0.01\\
405.01	0.01\\
406.01	0.01\\
407.01	0.01\\
408.01	0.01\\
409.01	0.01\\
410.01	0.01\\
411.01	0.01\\
412.01	0.01\\
413.01	0.01\\
414.01	0.01\\
415.01	0.01\\
416.01	0.01\\
417.01	0.01\\
418.01	0.01\\
419.01	0.01\\
420.01	0.01\\
421.01	0.01\\
422.01	0.01\\
423.01	0.01\\
424.01	0.01\\
425.01	0.01\\
426.01	0.01\\
427.01	0.01\\
428.01	0.01\\
429.01	0.01\\
430.01	0.01\\
431.01	0.01\\
432.01	0.01\\
433.01	0.01\\
434.01	0.01\\
435.01	0.01\\
436.01	0.01\\
437.01	0.01\\
438.01	0.01\\
439.01	0.01\\
440.01	0.01\\
441.01	0.01\\
442.01	0.01\\
443.01	0.01\\
444.01	0.01\\
445.01	0.01\\
446.01	0.01\\
447.01	0.01\\
448.01	0.01\\
449.01	0.01\\
450.01	0.01\\
451.01	0.01\\
452.01	0.01\\
453.01	0.01\\
454.01	0.01\\
455.01	0.01\\
456.01	0.01\\
457.01	0.01\\
458.01	0.01\\
459.01	0.01\\
460.01	0.01\\
461.01	0.01\\
462.01	0.01\\
463.01	0.01\\
464.01	0.01\\
465.01	0.01\\
466.01	0.01\\
467.01	0.01\\
468.01	0.01\\
469.01	0.01\\
470.01	0.01\\
471.01	0.01\\
472.01	0.01\\
473.01	0.01\\
474.01	0.01\\
475.01	0.01\\
476.01	0.01\\
477.01	0.01\\
478.01	0.01\\
479.01	0.01\\
480.01	0.01\\
481.01	0.01\\
482.01	0.01\\
483.01	0.01\\
484.01	0.01\\
485.01	0.01\\
486.01	0.01\\
487.01	0.01\\
488.01	0.01\\
489.01	0.01\\
490.01	0.01\\
491.01	0.01\\
492.01	0.01\\
493.01	0.01\\
494.01	0.01\\
495.01	0.01\\
496.01	0.01\\
497.01	0.01\\
498.01	0.01\\
499.01	0.01\\
500.01	0.01\\
501.01	0.01\\
502.01	0.01\\
503.01	0.01\\
504.01	0.01\\
505.01	0.01\\
506.01	0.01\\
507.01	0.01\\
508.01	0.01\\
509.01	0.01\\
510.01	0.01\\
511.01	0.01\\
512.01	0.01\\
513.01	0.01\\
514.01	0.01\\
515.01	0.01\\
516.01	0.01\\
517.01	0.01\\
518.01	0.01\\
519.01	0.01\\
520.01	0.01\\
521.01	0.01\\
522.01	0.01\\
523.01	0.01\\
524.01	0.01\\
525.01	0.01\\
526.01	0.01\\
527.01	0.01\\
528.01	0.01\\
529.01	0.01\\
530.01	0.01\\
531.01	0.01\\
532.01	0.01\\
533.01	0.01\\
534.01	0.01\\
535.01	0.01\\
536.01	0.01\\
537.01	0.01\\
538.01	0.01\\
539.01	0.01\\
540.01	0.01\\
541.01	0.01\\
542.01	0.01\\
543.01	0.01\\
544.01	0.01\\
545.01	0.01\\
546.01	0.01\\
547.01	0.01\\
548.01	0.01\\
549.01	0.01\\
550.01	0.01\\
551.01	0.01\\
552.01	0.01\\
553.01	0.01\\
554.01	0.01\\
555.01	0.01\\
556.01	0.01\\
557.01	0.01\\
558.01	0.01\\
559.01	0.01\\
560.01	0.01\\
561.01	0.01\\
562.01	0.01\\
563.01	0.01\\
564.01	0.01\\
565.01	0.01\\
566.01	0.01\\
567.01	0.01\\
568.01	0.01\\
569.01	0.01\\
570.01	0.01\\
571.01	0.01\\
572.01	0.01\\
573.01	0.01\\
574.01	0.01\\
575.01	0.01\\
576.01	0.01\\
577.01	0.01\\
578.01	0.01\\
579.01	0.01\\
580.01	0.01\\
581.01	0.01\\
582.01	0.01\\
583.01	0.01\\
584.01	0.01\\
585.01	0.01\\
586.01	0.01\\
587.01	0.01\\
588.01	0.01\\
589.01	0.01\\
590.01	0.01\\
591.01	0.01\\
592.01	0.01\\
593.01	0.01\\
594.01	0.01\\
595.01	0.01\\
596.01	0.01\\
597.01	0.01\\
598.01	0.0086390371714876\\
599.01	0.00623513569400516\\
599.02	0.00619747359804889\\
599.03	0.00615944475593817\\
599.04	0.00612104556201829\\
599.05	0.00608227237517917\\
599.06	0.00604312151850673\\
599.07	0.00600358927893092\\
599.08	0.00596367190687008\\
599.09	0.00592336561587206\\
599.1	0.0058826665822515\\
599.11	0.00584157094472381\\
599.12	0.00580007480403546\\
599.13	0.00575817422259058\\
599.14	0.00571586522407398\\
599.15	0.00567314379307033\\
599.16	0.00563000587467983\\
599.17	0.00558644737412983\\
599.18	0.00554246415638277\\
599.19	0.00549805204574035\\
599.2	0.00545320682544357\\
599.21	0.00540792423726907\\
599.22	0.00536219998112139\\
599.23	0.00531602971462118\\
599.24	0.0052694090526895\\
599.25	0.00522233356712793\\
599.26	0.00517479878619452\\
599.27	0.00512680019417564\\
599.28	0.00507833323095367\\
599.29	0.0050293932915703\\
599.3	0.0049799757257856\\
599.31	0.00493007583763285\\
599.32	0.00487968888496879\\
599.33	0.00482881007901965\\
599.34	0.00477743458392255\\
599.35	0.00472555751626256\\
599.36	0.00467317394460504\\
599.37	0.00462027888902352\\
599.38	0.0045668673206228\\
599.39	0.00451293416105749\\
599.4	0.00445847428204579\\
599.41	0.00440348251095323\\
599.42	0.00434795363554174\\
599.43	0.00429188239249109\\
599.44	0.00423526346689831\\
599.45	0.00417809149177215\\
599.46	0.00412036104752261\\
599.47	0.00406206666144544\\
599.48	0.0040032028072016\\
599.49	0.00394376390429164\\
599.5	0.00388374431752493\\
599.51	0.00382313835648359\\
599.52	0.00376194027498137\\
599.53	0.00370014427051698\\
599.54	0.00363774448372229\\
599.55	0.00357473499780495\\
599.56	0.00351110983798565\\
599.57	0.00344686297092978\\
599.58	0.00338198830417368\\
599.59	0.00331647968554506\\
599.6	0.00325033090257785\\
599.61	0.00318353568192136\\
599.62	0.00311608768874352\\
599.63	0.00304798052612842\\
599.64	0.00297920773446781\\
599.65	0.00290976279084675\\
599.66	0.00283963910842315\\
599.67	0.00276883003580128\\
599.68	0.00269732885639911\\
599.69	0.0026251287878095\\
599.7	0.0025522229811551\\
599.71	0.00247860452043685\\
599.72	0.00240426642187628\\
599.73	0.00232920163325119\\
599.74	0.00225340303322486\\
599.75	0.0021768634306688\\
599.76	0.00209957556397866\\
599.77	0.00202153210038365\\
599.78	0.00194272563524903\\
599.79	0.00186314869137185\\
599.8	0.00178279371826975\\
599.81	0.00170165309146282\\
599.82	0.00161971911174841\\
599.83	0.00153698400446878\\
599.84	0.00145343991877171\\
599.85	0.00136907892686371\\
599.86	0.00128389302325597\\
599.87	0.00119787412400299\\
599.88	0.00111101406593363\\
599.89	0.00102330460587469\\
599.9	0.000934737419866901\\
599.91	0.000845304102373181\\
599.92	0.000754996165479166\\
599.93	0.000663805038085864\\
599.94	0.000571722065094473\\
599.95	0.000478738506583126\\
599.96	0.000384845536975636\\
599.97	0.000290034244202042\\
599.98	0.00019429562885097\\
599.99	9.76206033136556e-05\\
600	0\\
};
\addplot [color=mycolor2,solid,forget plot]
  table[row sep=crcr]{%
0.01	0.01\\
1.01	0.01\\
2.01	0.01\\
3.01	0.01\\
4.01	0.01\\
5.01	0.01\\
6.01	0.01\\
7.01	0.01\\
8.01	0.01\\
9.01	0.01\\
10.01	0.01\\
11.01	0.01\\
12.01	0.01\\
13.01	0.01\\
14.01	0.01\\
15.01	0.01\\
16.01	0.01\\
17.01	0.01\\
18.01	0.01\\
19.01	0.01\\
20.01	0.01\\
21.01	0.01\\
22.01	0.01\\
23.01	0.01\\
24.01	0.01\\
25.01	0.01\\
26.01	0.01\\
27.01	0.01\\
28.01	0.01\\
29.01	0.01\\
30.01	0.01\\
31.01	0.01\\
32.01	0.01\\
33.01	0.01\\
34.01	0.01\\
35.01	0.01\\
36.01	0.01\\
37.01	0.01\\
38.01	0.01\\
39.01	0.01\\
40.01	0.01\\
41.01	0.01\\
42.01	0.01\\
43.01	0.01\\
44.01	0.01\\
45.01	0.01\\
46.01	0.01\\
47.01	0.01\\
48.01	0.01\\
49.01	0.01\\
50.01	0.01\\
51.01	0.01\\
52.01	0.01\\
53.01	0.01\\
54.01	0.01\\
55.01	0.01\\
56.01	0.01\\
57.01	0.01\\
58.01	0.01\\
59.01	0.01\\
60.01	0.01\\
61.01	0.01\\
62.01	0.01\\
63.01	0.01\\
64.01	0.01\\
65.01	0.01\\
66.01	0.01\\
67.01	0.01\\
68.01	0.01\\
69.01	0.01\\
70.01	0.01\\
71.01	0.01\\
72.01	0.01\\
73.01	0.01\\
74.01	0.01\\
75.01	0.01\\
76.01	0.01\\
77.01	0.01\\
78.01	0.01\\
79.01	0.01\\
80.01	0.01\\
81.01	0.01\\
82.01	0.01\\
83.01	0.01\\
84.01	0.01\\
85.01	0.01\\
86.01	0.01\\
87.01	0.01\\
88.01	0.01\\
89.01	0.01\\
90.01	0.01\\
91.01	0.01\\
92.01	0.01\\
93.01	0.01\\
94.01	0.01\\
95.01	0.01\\
96.01	0.01\\
97.01	0.01\\
98.01	0.01\\
99.01	0.01\\
100.01	0.01\\
101.01	0.01\\
102.01	0.01\\
103.01	0.01\\
104.01	0.01\\
105.01	0.01\\
106.01	0.01\\
107.01	0.01\\
108.01	0.01\\
109.01	0.01\\
110.01	0.01\\
111.01	0.01\\
112.01	0.01\\
113.01	0.01\\
114.01	0.01\\
115.01	0.01\\
116.01	0.01\\
117.01	0.01\\
118.01	0.01\\
119.01	0.01\\
120.01	0.01\\
121.01	0.01\\
122.01	0.01\\
123.01	0.01\\
124.01	0.01\\
125.01	0.01\\
126.01	0.01\\
127.01	0.01\\
128.01	0.01\\
129.01	0.01\\
130.01	0.01\\
131.01	0.01\\
132.01	0.01\\
133.01	0.01\\
134.01	0.01\\
135.01	0.01\\
136.01	0.01\\
137.01	0.01\\
138.01	0.01\\
139.01	0.01\\
140.01	0.01\\
141.01	0.01\\
142.01	0.01\\
143.01	0.01\\
144.01	0.01\\
145.01	0.01\\
146.01	0.01\\
147.01	0.01\\
148.01	0.01\\
149.01	0.01\\
150.01	0.01\\
151.01	0.01\\
152.01	0.01\\
153.01	0.01\\
154.01	0.01\\
155.01	0.01\\
156.01	0.01\\
157.01	0.01\\
158.01	0.01\\
159.01	0.01\\
160.01	0.01\\
161.01	0.01\\
162.01	0.01\\
163.01	0.01\\
164.01	0.01\\
165.01	0.01\\
166.01	0.01\\
167.01	0.01\\
168.01	0.01\\
169.01	0.01\\
170.01	0.01\\
171.01	0.01\\
172.01	0.01\\
173.01	0.01\\
174.01	0.01\\
175.01	0.01\\
176.01	0.01\\
177.01	0.01\\
178.01	0.01\\
179.01	0.01\\
180.01	0.01\\
181.01	0.01\\
182.01	0.01\\
183.01	0.01\\
184.01	0.01\\
185.01	0.01\\
186.01	0.01\\
187.01	0.01\\
188.01	0.01\\
189.01	0.01\\
190.01	0.01\\
191.01	0.01\\
192.01	0.01\\
193.01	0.01\\
194.01	0.01\\
195.01	0.01\\
196.01	0.01\\
197.01	0.01\\
198.01	0.01\\
199.01	0.01\\
200.01	0.01\\
201.01	0.01\\
202.01	0.01\\
203.01	0.01\\
204.01	0.01\\
205.01	0.01\\
206.01	0.01\\
207.01	0.01\\
208.01	0.01\\
209.01	0.01\\
210.01	0.01\\
211.01	0.01\\
212.01	0.01\\
213.01	0.01\\
214.01	0.01\\
215.01	0.01\\
216.01	0.01\\
217.01	0.01\\
218.01	0.01\\
219.01	0.01\\
220.01	0.01\\
221.01	0.01\\
222.01	0.01\\
223.01	0.01\\
224.01	0.01\\
225.01	0.01\\
226.01	0.01\\
227.01	0.01\\
228.01	0.01\\
229.01	0.01\\
230.01	0.01\\
231.01	0.01\\
232.01	0.01\\
233.01	0.01\\
234.01	0.01\\
235.01	0.01\\
236.01	0.01\\
237.01	0.01\\
238.01	0.01\\
239.01	0.01\\
240.01	0.01\\
241.01	0.01\\
242.01	0.01\\
243.01	0.01\\
244.01	0.01\\
245.01	0.01\\
246.01	0.01\\
247.01	0.01\\
248.01	0.01\\
249.01	0.01\\
250.01	0.01\\
251.01	0.01\\
252.01	0.01\\
253.01	0.01\\
254.01	0.01\\
255.01	0.01\\
256.01	0.01\\
257.01	0.01\\
258.01	0.01\\
259.01	0.01\\
260.01	0.01\\
261.01	0.01\\
262.01	0.01\\
263.01	0.01\\
264.01	0.01\\
265.01	0.01\\
266.01	0.01\\
267.01	0.01\\
268.01	0.01\\
269.01	0.01\\
270.01	0.01\\
271.01	0.01\\
272.01	0.01\\
273.01	0.01\\
274.01	0.01\\
275.01	0.01\\
276.01	0.01\\
277.01	0.01\\
278.01	0.01\\
279.01	0.01\\
280.01	0.01\\
281.01	0.01\\
282.01	0.01\\
283.01	0.01\\
284.01	0.01\\
285.01	0.01\\
286.01	0.01\\
287.01	0.01\\
288.01	0.01\\
289.01	0.01\\
290.01	0.01\\
291.01	0.01\\
292.01	0.01\\
293.01	0.01\\
294.01	0.01\\
295.01	0.01\\
296.01	0.01\\
297.01	0.01\\
298.01	0.01\\
299.01	0.01\\
300.01	0.01\\
301.01	0.01\\
302.01	0.01\\
303.01	0.01\\
304.01	0.01\\
305.01	0.01\\
306.01	0.01\\
307.01	0.01\\
308.01	0.01\\
309.01	0.01\\
310.01	0.01\\
311.01	0.01\\
312.01	0.01\\
313.01	0.01\\
314.01	0.01\\
315.01	0.01\\
316.01	0.01\\
317.01	0.01\\
318.01	0.01\\
319.01	0.01\\
320.01	0.01\\
321.01	0.01\\
322.01	0.01\\
323.01	0.01\\
324.01	0.01\\
325.01	0.01\\
326.01	0.01\\
327.01	0.01\\
328.01	0.01\\
329.01	0.01\\
330.01	0.01\\
331.01	0.01\\
332.01	0.01\\
333.01	0.01\\
334.01	0.01\\
335.01	0.01\\
336.01	0.01\\
337.01	0.01\\
338.01	0.01\\
339.01	0.01\\
340.01	0.01\\
341.01	0.01\\
342.01	0.01\\
343.01	0.01\\
344.01	0.01\\
345.01	0.01\\
346.01	0.01\\
347.01	0.01\\
348.01	0.01\\
349.01	0.01\\
350.01	0.01\\
351.01	0.01\\
352.01	0.01\\
353.01	0.01\\
354.01	0.01\\
355.01	0.01\\
356.01	0.01\\
357.01	0.01\\
358.01	0.01\\
359.01	0.01\\
360.01	0.01\\
361.01	0.01\\
362.01	0.01\\
363.01	0.01\\
364.01	0.01\\
365.01	0.01\\
366.01	0.01\\
367.01	0.01\\
368.01	0.01\\
369.01	0.01\\
370.01	0.01\\
371.01	0.01\\
372.01	0.01\\
373.01	0.01\\
374.01	0.01\\
375.01	0.01\\
376.01	0.01\\
377.01	0.01\\
378.01	0.01\\
379.01	0.01\\
380.01	0.01\\
381.01	0.01\\
382.01	0.01\\
383.01	0.01\\
384.01	0.01\\
385.01	0.01\\
386.01	0.01\\
387.01	0.01\\
388.01	0.01\\
389.01	0.01\\
390.01	0.01\\
391.01	0.01\\
392.01	0.01\\
393.01	0.01\\
394.01	0.01\\
395.01	0.01\\
396.01	0.01\\
397.01	0.01\\
398.01	0.01\\
399.01	0.01\\
400.01	0.01\\
401.01	0.01\\
402.01	0.01\\
403.01	0.01\\
404.01	0.01\\
405.01	0.01\\
406.01	0.01\\
407.01	0.01\\
408.01	0.01\\
409.01	0.01\\
410.01	0.01\\
411.01	0.01\\
412.01	0.01\\
413.01	0.01\\
414.01	0.01\\
415.01	0.01\\
416.01	0.01\\
417.01	0.01\\
418.01	0.01\\
419.01	0.01\\
420.01	0.01\\
421.01	0.01\\
422.01	0.01\\
423.01	0.01\\
424.01	0.01\\
425.01	0.01\\
426.01	0.01\\
427.01	0.01\\
428.01	0.01\\
429.01	0.01\\
430.01	0.01\\
431.01	0.01\\
432.01	0.01\\
433.01	0.01\\
434.01	0.01\\
435.01	0.01\\
436.01	0.01\\
437.01	0.01\\
438.01	0.01\\
439.01	0.01\\
440.01	0.01\\
441.01	0.01\\
442.01	0.01\\
443.01	0.01\\
444.01	0.01\\
445.01	0.01\\
446.01	0.01\\
447.01	0.01\\
448.01	0.01\\
449.01	0.01\\
450.01	0.01\\
451.01	0.01\\
452.01	0.01\\
453.01	0.01\\
454.01	0.01\\
455.01	0.01\\
456.01	0.01\\
457.01	0.01\\
458.01	0.01\\
459.01	0.01\\
460.01	0.01\\
461.01	0.01\\
462.01	0.01\\
463.01	0.01\\
464.01	0.01\\
465.01	0.01\\
466.01	0.01\\
467.01	0.01\\
468.01	0.01\\
469.01	0.01\\
470.01	0.01\\
471.01	0.01\\
472.01	0.01\\
473.01	0.01\\
474.01	0.01\\
475.01	0.01\\
476.01	0.01\\
477.01	0.01\\
478.01	0.01\\
479.01	0.01\\
480.01	0.01\\
481.01	0.01\\
482.01	0.01\\
483.01	0.01\\
484.01	0.01\\
485.01	0.01\\
486.01	0.01\\
487.01	0.01\\
488.01	0.01\\
489.01	0.01\\
490.01	0.01\\
491.01	0.01\\
492.01	0.01\\
493.01	0.01\\
494.01	0.01\\
495.01	0.01\\
496.01	0.01\\
497.01	0.01\\
498.01	0.01\\
499.01	0.01\\
500.01	0.01\\
501.01	0.01\\
502.01	0.01\\
503.01	0.01\\
504.01	0.01\\
505.01	0.01\\
506.01	0.01\\
507.01	0.01\\
508.01	0.01\\
509.01	0.01\\
510.01	0.01\\
511.01	0.01\\
512.01	0.01\\
513.01	0.01\\
514.01	0.01\\
515.01	0.01\\
516.01	0.01\\
517.01	0.01\\
518.01	0.01\\
519.01	0.01\\
520.01	0.01\\
521.01	0.01\\
522.01	0.01\\
523.01	0.01\\
524.01	0.01\\
525.01	0.01\\
526.01	0.01\\
527.01	0.01\\
528.01	0.01\\
529.01	0.01\\
530.01	0.01\\
531.01	0.01\\
532.01	0.01\\
533.01	0.01\\
534.01	0.01\\
535.01	0.01\\
536.01	0.01\\
537.01	0.01\\
538.01	0.01\\
539.01	0.01\\
540.01	0.01\\
541.01	0.01\\
542.01	0.01\\
543.01	0.01\\
544.01	0.01\\
545.01	0.01\\
546.01	0.01\\
547.01	0.01\\
548.01	0.01\\
549.01	0.01\\
550.01	0.01\\
551.01	0.01\\
552.01	0.01\\
553.01	0.01\\
554.01	0.01\\
555.01	0.01\\
556.01	0.01\\
557.01	0.01\\
558.01	0.01\\
559.01	0.01\\
560.01	0.01\\
561.01	0.01\\
562.01	0.01\\
563.01	0.01\\
564.01	0.01\\
565.01	0.01\\
566.01	0.01\\
567.01	0.01\\
568.01	0.01\\
569.01	0.01\\
570.01	0.01\\
571.01	0.01\\
572.01	0.01\\
573.01	0.01\\
574.01	0.01\\
575.01	0.01\\
576.01	0.01\\
577.01	0.01\\
578.01	0.01\\
579.01	0.01\\
580.01	0.01\\
581.01	0.01\\
582.01	0.01\\
583.01	0.01\\
584.01	0.01\\
585.01	0.01\\
586.01	0.01\\
587.01	0.01\\
588.01	0.01\\
589.01	0.01\\
590.01	0.01\\
591.01	0.01\\
592.01	0.01\\
593.01	0.01\\
594.01	0.01\\
595.01	0.01\\
596.01	0.01\\
597.01	0.01\\
598.01	0.00863908687856317\\
599.01	0.00623513569400518\\
599.02	0.00619747359804889\\
599.03	0.00615944475593821\\
599.04	0.00612104556201832\\
599.05	0.00608227237517919\\
599.06	0.00604312151850674\\
599.07	0.00600358927893092\\
599.08	0.00596367190687009\\
599.09	0.00592336561587206\\
599.1	0.00588266658225149\\
599.11	0.0058415709447238\\
599.12	0.00580007480403545\\
599.13	0.00575817422259057\\
599.14	0.00571586522407395\\
599.15	0.00567314379307032\\
599.16	0.00563000587467982\\
599.17	0.0055864473741298\\
599.18	0.00554246415638276\\
599.19	0.00549805204574034\\
599.2	0.00545320682544357\\
599.21	0.00540792423726908\\
599.22	0.00536219998112139\\
599.23	0.00531602971462118\\
599.24	0.00526940905268952\\
599.25	0.00522233356712793\\
599.26	0.0051747987861945\\
599.27	0.00512680019417563\\
599.28	0.00507833323095366\\
599.29	0.00502939329157028\\
599.3	0.0049799757257856\\
599.31	0.00493007583763287\\
599.32	0.00487968888496881\\
599.33	0.00482881007901966\\
599.34	0.00477743458392257\\
599.35	0.00472555751626259\\
599.36	0.00467317394460509\\
599.37	0.00462027888902355\\
599.38	0.00456686732062282\\
599.39	0.00451293416105752\\
599.4	0.00445847428204582\\
599.41	0.00440348251095326\\
599.42	0.00434795363554177\\
599.43	0.00429188239249111\\
599.44	0.00423526346689834\\
599.45	0.00417809149177218\\
599.46	0.00412036104752265\\
599.47	0.00406206666144548\\
599.48	0.00400320280720164\\
599.49	0.00394376390429169\\
599.5	0.00388374431752497\\
599.51	0.00382313835648363\\
599.52	0.0037619402749814\\
599.53	0.00370014427051701\\
599.54	0.00363774448372232\\
599.55	0.00357473499780497\\
599.56	0.00351110983798565\\
599.57	0.0034468629709298\\
599.58	0.0033819883041737\\
599.59	0.00331647968554508\\
599.6	0.00325033090257788\\
599.61	0.00318353568192138\\
599.62	0.00311608768874354\\
599.63	0.00304798052612844\\
599.64	0.00297920773446784\\
599.65	0.00290976279084679\\
599.66	0.00283963910842319\\
599.67	0.00276883003580131\\
599.68	0.00269732885639914\\
599.69	0.00262512878780953\\
599.7	0.00255222298115512\\
599.71	0.00247860452043688\\
599.72	0.00240426642187632\\
599.73	0.00232920163325123\\
599.74	0.0022534030332249\\
599.75	0.00217686343066884\\
599.76	0.0020995755639787\\
599.77	0.00202153210038368\\
599.78	0.00194272563524906\\
599.79	0.00186314869137187\\
599.8	0.00178279371826977\\
599.81	0.00170165309146284\\
599.82	0.00161971911174841\\
599.83	0.00153698400446879\\
599.84	0.00145343991877172\\
599.85	0.00136907892686371\\
599.86	0.00128389302325598\\
599.87	0.001197874124003\\
599.88	0.00111101406593362\\
599.89	0.00102330460587468\\
599.9	0.000934737419866898\\
599.91	0.000845304102373184\\
599.92	0.000754996165479166\\
599.93	0.000663805038085866\\
599.94	0.000571722065094472\\
599.95	0.000478738506583129\\
599.96	0.000384845536975638\\
599.97	0.000290034244202044\\
599.98	0.00019429562885097\\
599.99	9.76206033136556e-05\\
600	0\\
};
\addplot [color=mycolor3,solid,forget plot]
  table[row sep=crcr]{%
0.01	0.01\\
1.01	0.01\\
2.01	0.01\\
3.01	0.01\\
4.01	0.01\\
5.01	0.01\\
6.01	0.01\\
7.01	0.01\\
8.01	0.01\\
9.01	0.01\\
10.01	0.01\\
11.01	0.01\\
12.01	0.01\\
13.01	0.01\\
14.01	0.01\\
15.01	0.01\\
16.01	0.01\\
17.01	0.01\\
18.01	0.01\\
19.01	0.01\\
20.01	0.01\\
21.01	0.01\\
22.01	0.01\\
23.01	0.01\\
24.01	0.01\\
25.01	0.01\\
26.01	0.01\\
27.01	0.01\\
28.01	0.01\\
29.01	0.01\\
30.01	0.01\\
31.01	0.01\\
32.01	0.01\\
33.01	0.01\\
34.01	0.01\\
35.01	0.01\\
36.01	0.01\\
37.01	0.01\\
38.01	0.01\\
39.01	0.01\\
40.01	0.01\\
41.01	0.01\\
42.01	0.01\\
43.01	0.01\\
44.01	0.01\\
45.01	0.01\\
46.01	0.01\\
47.01	0.01\\
48.01	0.01\\
49.01	0.01\\
50.01	0.01\\
51.01	0.01\\
52.01	0.01\\
53.01	0.01\\
54.01	0.01\\
55.01	0.01\\
56.01	0.01\\
57.01	0.01\\
58.01	0.01\\
59.01	0.01\\
60.01	0.01\\
61.01	0.01\\
62.01	0.01\\
63.01	0.01\\
64.01	0.01\\
65.01	0.01\\
66.01	0.01\\
67.01	0.01\\
68.01	0.01\\
69.01	0.01\\
70.01	0.01\\
71.01	0.01\\
72.01	0.01\\
73.01	0.01\\
74.01	0.01\\
75.01	0.01\\
76.01	0.01\\
77.01	0.01\\
78.01	0.01\\
79.01	0.01\\
80.01	0.01\\
81.01	0.01\\
82.01	0.01\\
83.01	0.01\\
84.01	0.01\\
85.01	0.01\\
86.01	0.01\\
87.01	0.01\\
88.01	0.01\\
89.01	0.01\\
90.01	0.01\\
91.01	0.01\\
92.01	0.01\\
93.01	0.01\\
94.01	0.01\\
95.01	0.01\\
96.01	0.01\\
97.01	0.01\\
98.01	0.01\\
99.01	0.01\\
100.01	0.01\\
101.01	0.01\\
102.01	0.01\\
103.01	0.01\\
104.01	0.01\\
105.01	0.01\\
106.01	0.01\\
107.01	0.01\\
108.01	0.01\\
109.01	0.01\\
110.01	0.01\\
111.01	0.01\\
112.01	0.01\\
113.01	0.01\\
114.01	0.01\\
115.01	0.01\\
116.01	0.01\\
117.01	0.01\\
118.01	0.01\\
119.01	0.01\\
120.01	0.01\\
121.01	0.01\\
122.01	0.01\\
123.01	0.01\\
124.01	0.01\\
125.01	0.01\\
126.01	0.01\\
127.01	0.01\\
128.01	0.01\\
129.01	0.01\\
130.01	0.01\\
131.01	0.01\\
132.01	0.01\\
133.01	0.01\\
134.01	0.01\\
135.01	0.01\\
136.01	0.01\\
137.01	0.01\\
138.01	0.01\\
139.01	0.01\\
140.01	0.01\\
141.01	0.01\\
142.01	0.01\\
143.01	0.01\\
144.01	0.01\\
145.01	0.01\\
146.01	0.01\\
147.01	0.01\\
148.01	0.01\\
149.01	0.01\\
150.01	0.01\\
151.01	0.01\\
152.01	0.01\\
153.01	0.01\\
154.01	0.01\\
155.01	0.01\\
156.01	0.01\\
157.01	0.01\\
158.01	0.01\\
159.01	0.01\\
160.01	0.01\\
161.01	0.01\\
162.01	0.01\\
163.01	0.01\\
164.01	0.01\\
165.01	0.01\\
166.01	0.01\\
167.01	0.01\\
168.01	0.01\\
169.01	0.01\\
170.01	0.01\\
171.01	0.01\\
172.01	0.01\\
173.01	0.01\\
174.01	0.01\\
175.01	0.01\\
176.01	0.01\\
177.01	0.01\\
178.01	0.01\\
179.01	0.01\\
180.01	0.01\\
181.01	0.01\\
182.01	0.01\\
183.01	0.01\\
184.01	0.01\\
185.01	0.01\\
186.01	0.01\\
187.01	0.01\\
188.01	0.01\\
189.01	0.01\\
190.01	0.01\\
191.01	0.01\\
192.01	0.01\\
193.01	0.01\\
194.01	0.01\\
195.01	0.01\\
196.01	0.01\\
197.01	0.01\\
198.01	0.01\\
199.01	0.01\\
200.01	0.01\\
201.01	0.01\\
202.01	0.01\\
203.01	0.01\\
204.01	0.01\\
205.01	0.01\\
206.01	0.01\\
207.01	0.01\\
208.01	0.01\\
209.01	0.01\\
210.01	0.01\\
211.01	0.01\\
212.01	0.01\\
213.01	0.01\\
214.01	0.01\\
215.01	0.01\\
216.01	0.01\\
217.01	0.01\\
218.01	0.01\\
219.01	0.01\\
220.01	0.01\\
221.01	0.01\\
222.01	0.01\\
223.01	0.01\\
224.01	0.01\\
225.01	0.01\\
226.01	0.01\\
227.01	0.01\\
228.01	0.01\\
229.01	0.01\\
230.01	0.01\\
231.01	0.01\\
232.01	0.01\\
233.01	0.01\\
234.01	0.01\\
235.01	0.01\\
236.01	0.01\\
237.01	0.01\\
238.01	0.01\\
239.01	0.01\\
240.01	0.01\\
241.01	0.01\\
242.01	0.01\\
243.01	0.01\\
244.01	0.01\\
245.01	0.01\\
246.01	0.01\\
247.01	0.01\\
248.01	0.01\\
249.01	0.01\\
250.01	0.01\\
251.01	0.01\\
252.01	0.01\\
253.01	0.01\\
254.01	0.01\\
255.01	0.01\\
256.01	0.01\\
257.01	0.01\\
258.01	0.01\\
259.01	0.01\\
260.01	0.01\\
261.01	0.01\\
262.01	0.01\\
263.01	0.01\\
264.01	0.01\\
265.01	0.01\\
266.01	0.01\\
267.01	0.01\\
268.01	0.01\\
269.01	0.01\\
270.01	0.01\\
271.01	0.01\\
272.01	0.01\\
273.01	0.01\\
274.01	0.01\\
275.01	0.01\\
276.01	0.01\\
277.01	0.01\\
278.01	0.01\\
279.01	0.01\\
280.01	0.01\\
281.01	0.01\\
282.01	0.01\\
283.01	0.01\\
284.01	0.01\\
285.01	0.01\\
286.01	0.01\\
287.01	0.01\\
288.01	0.01\\
289.01	0.01\\
290.01	0.01\\
291.01	0.01\\
292.01	0.01\\
293.01	0.01\\
294.01	0.01\\
295.01	0.01\\
296.01	0.01\\
297.01	0.01\\
298.01	0.01\\
299.01	0.01\\
300.01	0.01\\
301.01	0.01\\
302.01	0.01\\
303.01	0.01\\
304.01	0.01\\
305.01	0.01\\
306.01	0.01\\
307.01	0.01\\
308.01	0.01\\
309.01	0.01\\
310.01	0.01\\
311.01	0.01\\
312.01	0.01\\
313.01	0.01\\
314.01	0.01\\
315.01	0.01\\
316.01	0.01\\
317.01	0.01\\
318.01	0.01\\
319.01	0.01\\
320.01	0.01\\
321.01	0.01\\
322.01	0.01\\
323.01	0.01\\
324.01	0.01\\
325.01	0.01\\
326.01	0.01\\
327.01	0.01\\
328.01	0.01\\
329.01	0.01\\
330.01	0.01\\
331.01	0.01\\
332.01	0.01\\
333.01	0.01\\
334.01	0.01\\
335.01	0.01\\
336.01	0.01\\
337.01	0.01\\
338.01	0.01\\
339.01	0.01\\
340.01	0.01\\
341.01	0.01\\
342.01	0.01\\
343.01	0.01\\
344.01	0.01\\
345.01	0.01\\
346.01	0.01\\
347.01	0.01\\
348.01	0.01\\
349.01	0.01\\
350.01	0.01\\
351.01	0.01\\
352.01	0.01\\
353.01	0.01\\
354.01	0.01\\
355.01	0.01\\
356.01	0.01\\
357.01	0.01\\
358.01	0.01\\
359.01	0.01\\
360.01	0.01\\
361.01	0.01\\
362.01	0.01\\
363.01	0.01\\
364.01	0.01\\
365.01	0.01\\
366.01	0.01\\
367.01	0.01\\
368.01	0.01\\
369.01	0.01\\
370.01	0.01\\
371.01	0.01\\
372.01	0.01\\
373.01	0.01\\
374.01	0.01\\
375.01	0.01\\
376.01	0.01\\
377.01	0.01\\
378.01	0.01\\
379.01	0.01\\
380.01	0.01\\
381.01	0.01\\
382.01	0.01\\
383.01	0.01\\
384.01	0.01\\
385.01	0.01\\
386.01	0.01\\
387.01	0.01\\
388.01	0.01\\
389.01	0.01\\
390.01	0.01\\
391.01	0.01\\
392.01	0.01\\
393.01	0.01\\
394.01	0.01\\
395.01	0.01\\
396.01	0.01\\
397.01	0.01\\
398.01	0.01\\
399.01	0.01\\
400.01	0.01\\
401.01	0.01\\
402.01	0.01\\
403.01	0.01\\
404.01	0.01\\
405.01	0.01\\
406.01	0.01\\
407.01	0.01\\
408.01	0.01\\
409.01	0.01\\
410.01	0.01\\
411.01	0.01\\
412.01	0.01\\
413.01	0.01\\
414.01	0.01\\
415.01	0.01\\
416.01	0.01\\
417.01	0.01\\
418.01	0.01\\
419.01	0.01\\
420.01	0.01\\
421.01	0.01\\
422.01	0.01\\
423.01	0.01\\
424.01	0.01\\
425.01	0.01\\
426.01	0.01\\
427.01	0.01\\
428.01	0.01\\
429.01	0.01\\
430.01	0.01\\
431.01	0.01\\
432.01	0.01\\
433.01	0.01\\
434.01	0.01\\
435.01	0.01\\
436.01	0.01\\
437.01	0.01\\
438.01	0.01\\
439.01	0.01\\
440.01	0.01\\
441.01	0.01\\
442.01	0.01\\
443.01	0.01\\
444.01	0.01\\
445.01	0.01\\
446.01	0.01\\
447.01	0.01\\
448.01	0.01\\
449.01	0.01\\
450.01	0.01\\
451.01	0.01\\
452.01	0.01\\
453.01	0.01\\
454.01	0.01\\
455.01	0.01\\
456.01	0.01\\
457.01	0.01\\
458.01	0.01\\
459.01	0.01\\
460.01	0.01\\
461.01	0.01\\
462.01	0.01\\
463.01	0.01\\
464.01	0.01\\
465.01	0.01\\
466.01	0.01\\
467.01	0.01\\
468.01	0.01\\
469.01	0.01\\
470.01	0.01\\
471.01	0.01\\
472.01	0.01\\
473.01	0.01\\
474.01	0.01\\
475.01	0.01\\
476.01	0.01\\
477.01	0.01\\
478.01	0.01\\
479.01	0.01\\
480.01	0.01\\
481.01	0.01\\
482.01	0.01\\
483.01	0.01\\
484.01	0.01\\
485.01	0.01\\
486.01	0.01\\
487.01	0.01\\
488.01	0.01\\
489.01	0.01\\
490.01	0.01\\
491.01	0.01\\
492.01	0.01\\
493.01	0.01\\
494.01	0.01\\
495.01	0.01\\
496.01	0.01\\
497.01	0.01\\
498.01	0.01\\
499.01	0.01\\
500.01	0.01\\
501.01	0.01\\
502.01	0.01\\
503.01	0.01\\
504.01	0.01\\
505.01	0.01\\
506.01	0.01\\
507.01	0.01\\
508.01	0.01\\
509.01	0.01\\
510.01	0.01\\
511.01	0.01\\
512.01	0.01\\
513.01	0.01\\
514.01	0.01\\
515.01	0.01\\
516.01	0.01\\
517.01	0.01\\
518.01	0.01\\
519.01	0.01\\
520.01	0.01\\
521.01	0.01\\
522.01	0.01\\
523.01	0.01\\
524.01	0.01\\
525.01	0.01\\
526.01	0.01\\
527.01	0.01\\
528.01	0.01\\
529.01	0.01\\
530.01	0.01\\
531.01	0.01\\
532.01	0.01\\
533.01	0.01\\
534.01	0.01\\
535.01	0.01\\
536.01	0.01\\
537.01	0.01\\
538.01	0.01\\
539.01	0.01\\
540.01	0.01\\
541.01	0.01\\
542.01	0.01\\
543.01	0.01\\
544.01	0.01\\
545.01	0.01\\
546.01	0.01\\
547.01	0.01\\
548.01	0.01\\
549.01	0.01\\
550.01	0.01\\
551.01	0.01\\
552.01	0.01\\
553.01	0.01\\
554.01	0.01\\
555.01	0.01\\
556.01	0.01\\
557.01	0.01\\
558.01	0.01\\
559.01	0.01\\
560.01	0.01\\
561.01	0.01\\
562.01	0.01\\
563.01	0.01\\
564.01	0.01\\
565.01	0.01\\
566.01	0.01\\
567.01	0.01\\
568.01	0.01\\
569.01	0.01\\
570.01	0.01\\
571.01	0.01\\
572.01	0.01\\
573.01	0.01\\
574.01	0.01\\
575.01	0.01\\
576.01	0.01\\
577.01	0.01\\
578.01	0.01\\
579.01	0.01\\
580.01	0.01\\
581.01	0.01\\
582.01	0.01\\
583.01	0.01\\
584.01	0.01\\
585.01	0.01\\
586.01	0.01\\
587.01	0.01\\
588.01	0.01\\
589.01	0.01\\
590.01	0.01\\
591.01	0.01\\
592.01	0.01\\
593.01	0.01\\
594.01	0.01\\
595.01	0.01\\
596.01	0.01\\
597.01	0.01\\
598.01	0.00863963718816675\\
599.01	0.00623513569400517\\
599.02	0.00619747359804888\\
599.03	0.00615944475593818\\
599.04	0.00612104556201831\\
599.05	0.00608227237517917\\
599.06	0.00604312151850673\\
599.07	0.00600358927893092\\
599.08	0.00596367190687011\\
599.09	0.00592336561587207\\
599.1	0.0058826665822515\\
599.11	0.00584157094472381\\
599.12	0.00580007480403546\\
599.13	0.00575817422259058\\
599.14	0.00571586522407398\\
599.15	0.00567314379307033\\
599.16	0.00563000587467983\\
599.17	0.00558644737412983\\
599.18	0.00554246415638279\\
599.19	0.00549805204574037\\
599.2	0.00545320682544359\\
599.21	0.00540792423726908\\
599.22	0.0053621999811214\\
599.23	0.0053160297146212\\
599.24	0.00526940905268953\\
599.25	0.00522233356712796\\
599.26	0.00517479878619454\\
599.27	0.00512680019417566\\
599.28	0.0050783332309537\\
599.29	0.00502939329157033\\
599.3	0.00497997572578563\\
599.31	0.00493007583763288\\
599.32	0.00487968888496882\\
599.33	0.00482881007901966\\
599.34	0.00477743458392257\\
599.35	0.00472555751626257\\
599.36	0.00467317394460506\\
599.37	0.00462027888902354\\
599.38	0.00456686732062282\\
599.39	0.00451293416105752\\
599.4	0.00445847428204582\\
599.41	0.00440348251095325\\
599.42	0.00434795363554174\\
599.43	0.0042918823924911\\
599.44	0.00423526346689832\\
599.45	0.00417809149177218\\
599.46	0.00412036104752265\\
599.47	0.00406206666144548\\
599.48	0.00400320280720163\\
599.49	0.00394376390429167\\
599.5	0.00388374431752494\\
599.51	0.00382313835648361\\
599.52	0.00376194027498137\\
599.53	0.00370014427051699\\
599.54	0.0036377444837223\\
599.55	0.00357473499780496\\
599.56	0.00351110983798566\\
599.57	0.0034468629709298\\
599.58	0.00338198830417371\\
599.59	0.00331647968554508\\
599.6	0.00325033090257786\\
599.61	0.00318353568192138\\
599.62	0.00311608768874353\\
599.63	0.00304798052612843\\
599.64	0.00297920773446783\\
599.65	0.00290976279084677\\
599.66	0.00283963910842317\\
599.67	0.00276883003580129\\
599.68	0.00269732885639912\\
599.69	0.00262512878780952\\
599.7	0.00255222298115511\\
599.71	0.00247860452043686\\
599.72	0.0024042664218763\\
599.73	0.00232920163325121\\
599.74	0.00225340303322488\\
599.75	0.00217686343066882\\
599.76	0.00209957556397868\\
599.77	0.00202153210038367\\
599.78	0.00194272563524905\\
599.79	0.00186314869137186\\
599.8	0.00178279371826976\\
599.81	0.00170165309146283\\
599.82	0.00161971911174842\\
599.83	0.0015369840044688\\
599.84	0.00145343991877173\\
599.85	0.00136907892686372\\
599.86	0.00128389302325599\\
599.87	0.001197874124003\\
599.88	0.00111101406593363\\
599.89	0.00102330460587469\\
599.9	0.000934737419866907\\
599.91	0.00084530410237319\\
599.92	0.000754996165479171\\
599.93	0.000663805038085873\\
599.94	0.000571722065094477\\
599.95	0.000478738506583131\\
599.96	0.000384845536975641\\
599.97	0.000290034244202044\\
599.98	0.00019429562885097\\
599.99	9.76206033136574e-05\\
600	0\\
};
\addplot [color=mycolor4,solid,forget plot]
  table[row sep=crcr]{%
0.01	0.01\\
1.01	0.01\\
2.01	0.01\\
3.01	0.01\\
4.01	0.01\\
5.01	0.01\\
6.01	0.01\\
7.01	0.01\\
8.01	0.01\\
9.01	0.01\\
10.01	0.01\\
11.01	0.01\\
12.01	0.01\\
13.01	0.01\\
14.01	0.01\\
15.01	0.01\\
16.01	0.01\\
17.01	0.01\\
18.01	0.01\\
19.01	0.01\\
20.01	0.01\\
21.01	0.01\\
22.01	0.01\\
23.01	0.01\\
24.01	0.01\\
25.01	0.01\\
26.01	0.01\\
27.01	0.01\\
28.01	0.01\\
29.01	0.01\\
30.01	0.01\\
31.01	0.01\\
32.01	0.01\\
33.01	0.01\\
34.01	0.01\\
35.01	0.01\\
36.01	0.01\\
37.01	0.01\\
38.01	0.01\\
39.01	0.01\\
40.01	0.01\\
41.01	0.01\\
42.01	0.01\\
43.01	0.01\\
44.01	0.01\\
45.01	0.01\\
46.01	0.01\\
47.01	0.01\\
48.01	0.01\\
49.01	0.01\\
50.01	0.01\\
51.01	0.01\\
52.01	0.01\\
53.01	0.01\\
54.01	0.01\\
55.01	0.01\\
56.01	0.01\\
57.01	0.01\\
58.01	0.01\\
59.01	0.01\\
60.01	0.01\\
61.01	0.01\\
62.01	0.01\\
63.01	0.01\\
64.01	0.01\\
65.01	0.01\\
66.01	0.01\\
67.01	0.01\\
68.01	0.01\\
69.01	0.01\\
70.01	0.01\\
71.01	0.01\\
72.01	0.01\\
73.01	0.01\\
74.01	0.01\\
75.01	0.01\\
76.01	0.01\\
77.01	0.01\\
78.01	0.01\\
79.01	0.01\\
80.01	0.01\\
81.01	0.01\\
82.01	0.01\\
83.01	0.01\\
84.01	0.01\\
85.01	0.01\\
86.01	0.01\\
87.01	0.01\\
88.01	0.01\\
89.01	0.01\\
90.01	0.01\\
91.01	0.01\\
92.01	0.01\\
93.01	0.01\\
94.01	0.01\\
95.01	0.01\\
96.01	0.01\\
97.01	0.01\\
98.01	0.01\\
99.01	0.01\\
100.01	0.01\\
101.01	0.01\\
102.01	0.01\\
103.01	0.01\\
104.01	0.01\\
105.01	0.01\\
106.01	0.01\\
107.01	0.01\\
108.01	0.01\\
109.01	0.01\\
110.01	0.01\\
111.01	0.01\\
112.01	0.01\\
113.01	0.01\\
114.01	0.01\\
115.01	0.01\\
116.01	0.01\\
117.01	0.01\\
118.01	0.01\\
119.01	0.01\\
120.01	0.01\\
121.01	0.01\\
122.01	0.01\\
123.01	0.01\\
124.01	0.01\\
125.01	0.01\\
126.01	0.01\\
127.01	0.01\\
128.01	0.01\\
129.01	0.01\\
130.01	0.01\\
131.01	0.01\\
132.01	0.01\\
133.01	0.01\\
134.01	0.01\\
135.01	0.01\\
136.01	0.01\\
137.01	0.01\\
138.01	0.01\\
139.01	0.01\\
140.01	0.01\\
141.01	0.01\\
142.01	0.01\\
143.01	0.01\\
144.01	0.01\\
145.01	0.01\\
146.01	0.01\\
147.01	0.01\\
148.01	0.01\\
149.01	0.01\\
150.01	0.01\\
151.01	0.01\\
152.01	0.01\\
153.01	0.01\\
154.01	0.01\\
155.01	0.01\\
156.01	0.01\\
157.01	0.01\\
158.01	0.01\\
159.01	0.01\\
160.01	0.01\\
161.01	0.01\\
162.01	0.01\\
163.01	0.01\\
164.01	0.01\\
165.01	0.01\\
166.01	0.01\\
167.01	0.01\\
168.01	0.01\\
169.01	0.01\\
170.01	0.01\\
171.01	0.01\\
172.01	0.01\\
173.01	0.01\\
174.01	0.01\\
175.01	0.01\\
176.01	0.01\\
177.01	0.01\\
178.01	0.01\\
179.01	0.01\\
180.01	0.01\\
181.01	0.01\\
182.01	0.01\\
183.01	0.01\\
184.01	0.01\\
185.01	0.01\\
186.01	0.01\\
187.01	0.01\\
188.01	0.01\\
189.01	0.01\\
190.01	0.01\\
191.01	0.01\\
192.01	0.01\\
193.01	0.01\\
194.01	0.01\\
195.01	0.01\\
196.01	0.01\\
197.01	0.01\\
198.01	0.01\\
199.01	0.01\\
200.01	0.01\\
201.01	0.01\\
202.01	0.01\\
203.01	0.01\\
204.01	0.01\\
205.01	0.01\\
206.01	0.01\\
207.01	0.01\\
208.01	0.01\\
209.01	0.01\\
210.01	0.01\\
211.01	0.01\\
212.01	0.01\\
213.01	0.01\\
214.01	0.01\\
215.01	0.01\\
216.01	0.01\\
217.01	0.01\\
218.01	0.01\\
219.01	0.01\\
220.01	0.01\\
221.01	0.01\\
222.01	0.01\\
223.01	0.01\\
224.01	0.01\\
225.01	0.01\\
226.01	0.01\\
227.01	0.01\\
228.01	0.01\\
229.01	0.01\\
230.01	0.01\\
231.01	0.01\\
232.01	0.01\\
233.01	0.01\\
234.01	0.01\\
235.01	0.01\\
236.01	0.01\\
237.01	0.01\\
238.01	0.01\\
239.01	0.01\\
240.01	0.01\\
241.01	0.01\\
242.01	0.01\\
243.01	0.01\\
244.01	0.01\\
245.01	0.01\\
246.01	0.01\\
247.01	0.01\\
248.01	0.01\\
249.01	0.01\\
250.01	0.01\\
251.01	0.01\\
252.01	0.01\\
253.01	0.01\\
254.01	0.01\\
255.01	0.01\\
256.01	0.01\\
257.01	0.01\\
258.01	0.01\\
259.01	0.01\\
260.01	0.01\\
261.01	0.01\\
262.01	0.01\\
263.01	0.01\\
264.01	0.01\\
265.01	0.01\\
266.01	0.01\\
267.01	0.01\\
268.01	0.01\\
269.01	0.01\\
270.01	0.01\\
271.01	0.01\\
272.01	0.01\\
273.01	0.01\\
274.01	0.01\\
275.01	0.01\\
276.01	0.01\\
277.01	0.01\\
278.01	0.01\\
279.01	0.01\\
280.01	0.01\\
281.01	0.01\\
282.01	0.01\\
283.01	0.01\\
284.01	0.01\\
285.01	0.01\\
286.01	0.01\\
287.01	0.01\\
288.01	0.01\\
289.01	0.01\\
290.01	0.01\\
291.01	0.01\\
292.01	0.01\\
293.01	0.01\\
294.01	0.01\\
295.01	0.01\\
296.01	0.01\\
297.01	0.01\\
298.01	0.01\\
299.01	0.01\\
300.01	0.01\\
301.01	0.01\\
302.01	0.01\\
303.01	0.01\\
304.01	0.01\\
305.01	0.01\\
306.01	0.01\\
307.01	0.01\\
308.01	0.01\\
309.01	0.01\\
310.01	0.01\\
311.01	0.01\\
312.01	0.01\\
313.01	0.01\\
314.01	0.01\\
315.01	0.01\\
316.01	0.01\\
317.01	0.01\\
318.01	0.01\\
319.01	0.01\\
320.01	0.01\\
321.01	0.01\\
322.01	0.01\\
323.01	0.01\\
324.01	0.01\\
325.01	0.01\\
326.01	0.01\\
327.01	0.01\\
328.01	0.01\\
329.01	0.01\\
330.01	0.01\\
331.01	0.01\\
332.01	0.01\\
333.01	0.01\\
334.01	0.01\\
335.01	0.01\\
336.01	0.01\\
337.01	0.01\\
338.01	0.01\\
339.01	0.01\\
340.01	0.01\\
341.01	0.01\\
342.01	0.01\\
343.01	0.01\\
344.01	0.01\\
345.01	0.01\\
346.01	0.01\\
347.01	0.01\\
348.01	0.01\\
349.01	0.01\\
350.01	0.01\\
351.01	0.01\\
352.01	0.01\\
353.01	0.01\\
354.01	0.01\\
355.01	0.01\\
356.01	0.01\\
357.01	0.01\\
358.01	0.01\\
359.01	0.01\\
360.01	0.01\\
361.01	0.01\\
362.01	0.01\\
363.01	0.01\\
364.01	0.01\\
365.01	0.01\\
366.01	0.01\\
367.01	0.01\\
368.01	0.01\\
369.01	0.01\\
370.01	0.01\\
371.01	0.01\\
372.01	0.01\\
373.01	0.01\\
374.01	0.01\\
375.01	0.01\\
376.01	0.01\\
377.01	0.01\\
378.01	0.01\\
379.01	0.01\\
380.01	0.01\\
381.01	0.01\\
382.01	0.01\\
383.01	0.01\\
384.01	0.01\\
385.01	0.01\\
386.01	0.01\\
387.01	0.01\\
388.01	0.01\\
389.01	0.01\\
390.01	0.01\\
391.01	0.01\\
392.01	0.01\\
393.01	0.01\\
394.01	0.01\\
395.01	0.01\\
396.01	0.01\\
397.01	0.01\\
398.01	0.01\\
399.01	0.01\\
400.01	0.01\\
401.01	0.01\\
402.01	0.01\\
403.01	0.01\\
404.01	0.01\\
405.01	0.01\\
406.01	0.01\\
407.01	0.01\\
408.01	0.01\\
409.01	0.01\\
410.01	0.01\\
411.01	0.01\\
412.01	0.01\\
413.01	0.01\\
414.01	0.01\\
415.01	0.01\\
416.01	0.01\\
417.01	0.01\\
418.01	0.01\\
419.01	0.01\\
420.01	0.01\\
421.01	0.01\\
422.01	0.01\\
423.01	0.01\\
424.01	0.01\\
425.01	0.01\\
426.01	0.01\\
427.01	0.01\\
428.01	0.01\\
429.01	0.01\\
430.01	0.01\\
431.01	0.01\\
432.01	0.01\\
433.01	0.01\\
434.01	0.01\\
435.01	0.01\\
436.01	0.01\\
437.01	0.01\\
438.01	0.01\\
439.01	0.01\\
440.01	0.01\\
441.01	0.01\\
442.01	0.01\\
443.01	0.01\\
444.01	0.01\\
445.01	0.01\\
446.01	0.01\\
447.01	0.01\\
448.01	0.01\\
449.01	0.01\\
450.01	0.01\\
451.01	0.01\\
452.01	0.01\\
453.01	0.01\\
454.01	0.01\\
455.01	0.01\\
456.01	0.01\\
457.01	0.01\\
458.01	0.01\\
459.01	0.01\\
460.01	0.01\\
461.01	0.01\\
462.01	0.01\\
463.01	0.01\\
464.01	0.01\\
465.01	0.01\\
466.01	0.01\\
467.01	0.01\\
468.01	0.01\\
469.01	0.01\\
470.01	0.01\\
471.01	0.01\\
472.01	0.01\\
473.01	0.01\\
474.01	0.01\\
475.01	0.01\\
476.01	0.01\\
477.01	0.01\\
478.01	0.01\\
479.01	0.01\\
480.01	0.01\\
481.01	0.01\\
482.01	0.01\\
483.01	0.01\\
484.01	0.01\\
485.01	0.01\\
486.01	0.01\\
487.01	0.01\\
488.01	0.01\\
489.01	0.01\\
490.01	0.01\\
491.01	0.01\\
492.01	0.01\\
493.01	0.01\\
494.01	0.01\\
495.01	0.01\\
496.01	0.01\\
497.01	0.01\\
498.01	0.01\\
499.01	0.01\\
500.01	0.01\\
501.01	0.01\\
502.01	0.01\\
503.01	0.01\\
504.01	0.01\\
505.01	0.01\\
506.01	0.01\\
507.01	0.01\\
508.01	0.01\\
509.01	0.01\\
510.01	0.01\\
511.01	0.01\\
512.01	0.01\\
513.01	0.01\\
514.01	0.01\\
515.01	0.01\\
516.01	0.01\\
517.01	0.01\\
518.01	0.01\\
519.01	0.01\\
520.01	0.01\\
521.01	0.01\\
522.01	0.01\\
523.01	0.01\\
524.01	0.01\\
525.01	0.01\\
526.01	0.01\\
527.01	0.01\\
528.01	0.01\\
529.01	0.01\\
530.01	0.01\\
531.01	0.01\\
532.01	0.01\\
533.01	0.01\\
534.01	0.01\\
535.01	0.01\\
536.01	0.01\\
537.01	0.01\\
538.01	0.01\\
539.01	0.01\\
540.01	0.01\\
541.01	0.01\\
542.01	0.01\\
543.01	0.01\\
544.01	0.01\\
545.01	0.01\\
546.01	0.01\\
547.01	0.01\\
548.01	0.01\\
549.01	0.01\\
550.01	0.01\\
551.01	0.01\\
552.01	0.01\\
553.01	0.01\\
554.01	0.01\\
555.01	0.01\\
556.01	0.01\\
557.01	0.01\\
558.01	0.01\\
559.01	0.01\\
560.01	0.01\\
561.01	0.01\\
562.01	0.01\\
563.01	0.01\\
564.01	0.01\\
565.01	0.01\\
566.01	0.01\\
567.01	0.01\\
568.01	0.01\\
569.01	0.01\\
570.01	0.01\\
571.01	0.01\\
572.01	0.01\\
573.01	0.01\\
574.01	0.01\\
575.01	0.01\\
576.01	0.01\\
577.01	0.01\\
578.01	0.01\\
579.01	0.01\\
580.01	0.01\\
581.01	0.01\\
582.01	0.01\\
583.01	0.01\\
584.01	0.01\\
585.01	0.01\\
586.01	0.01\\
587.01	0.01\\
588.01	0.01\\
589.01	0.01\\
590.01	0.01\\
591.01	0.01\\
592.01	0.01\\
593.01	0.01\\
594.01	0.01\\
595.01	0.01\\
596.01	0.01\\
597.01	0.01\\
598.01	0.00912621121604349\\
599.01	0.00623513569400514\\
599.02	0.00619747359804885\\
599.03	0.00615944475593814\\
599.04	0.00612104556201827\\
599.05	0.00608227237517913\\
599.06	0.0060431215185067\\
599.07	0.00600358927893089\\
599.08	0.00596367190687006\\
599.09	0.00592336561587204\\
599.1	0.00588266658225147\\
599.11	0.00584157094472378\\
599.12	0.00580007480403543\\
599.13	0.00575817422259054\\
599.14	0.00571586522407392\\
599.15	0.00567314379307028\\
599.16	0.00563000587467979\\
599.17	0.00558644737412979\\
599.18	0.00554246415638275\\
599.19	0.00549805204574031\\
599.2	0.00545320682544354\\
599.21	0.00540792423726905\\
599.22	0.00536219998112137\\
599.23	0.00531602971462116\\
599.24	0.0052694090526895\\
599.25	0.00522233356712793\\
599.26	0.0051747987861945\\
599.27	0.00512680019417563\\
599.28	0.00507833323095366\\
599.29	0.00502939329157028\\
599.3	0.0049799757257856\\
599.31	0.00493007583763285\\
599.32	0.00487968888496879\\
599.33	0.00482881007901965\\
599.34	0.00477743458392255\\
599.35	0.00472555751626257\\
599.36	0.00467317394460506\\
599.37	0.00462027888902352\\
599.38	0.00456686732062279\\
599.39	0.00451293416105749\\
599.4	0.00445847428204579\\
599.41	0.00440348251095323\\
599.42	0.00434795363554174\\
599.43	0.00429188239249109\\
599.44	0.00423526346689831\\
599.45	0.00417809149177215\\
599.46	0.00412036104752262\\
599.47	0.00406206666144545\\
599.48	0.00400320280720161\\
599.49	0.00394376390429166\\
599.5	0.00388374431752493\\
599.51	0.00382313835648359\\
599.52	0.00376194027498136\\
599.53	0.00370014427051698\\
599.54	0.00363774448372228\\
599.55	0.00357473499780494\\
599.56	0.00351110983798563\\
599.57	0.00344686297092977\\
599.58	0.00338198830417368\\
599.59	0.00331647968554506\\
599.6	0.00325033090257784\\
599.61	0.00318353568192134\\
599.62	0.00311608768874351\\
599.63	0.00304798052612841\\
599.64	0.00297920773446781\\
599.65	0.00290976279084676\\
599.66	0.00283963910842317\\
599.67	0.00276883003580129\\
599.68	0.00269732885639912\\
599.69	0.00262512878780951\\
599.7	0.0025522229811551\\
599.71	0.00247860452043686\\
599.72	0.00240426642187629\\
599.73	0.0023292016332512\\
599.74	0.00225340303322488\\
599.75	0.00217686343066882\\
599.76	0.00209957556397868\\
599.77	0.00202153210038366\\
599.78	0.00194272563524904\\
599.79	0.00186314869137186\\
599.8	0.00178279371826975\\
599.81	0.00170165309146282\\
599.82	0.00161971911174841\\
599.83	0.00153698400446878\\
599.84	0.00145343991877171\\
599.85	0.00136907892686371\\
599.86	0.00128389302325597\\
599.87	0.00119787412400299\\
599.88	0.00111101406593362\\
599.89	0.00102330460587469\\
599.9	0.0009347374198669\\
599.91	0.000845304102373186\\
599.92	0.000754996165479164\\
599.93	0.000663805038085868\\
599.94	0.000571722065094472\\
599.95	0.000478738506583129\\
599.96	0.000384845536975638\\
599.97	0.000290034244202044\\
599.98	0.00019429562885097\\
599.99	9.76206033136574e-05\\
600	0\\
};
\addplot [color=mycolor5,solid,forget plot]
  table[row sep=crcr]{%
0.01	0.01\\
1.01	0.01\\
2.01	0.01\\
3.01	0.01\\
4.01	0.01\\
5.01	0.01\\
6.01	0.01\\
7.01	0.01\\
8.01	0.01\\
9.01	0.01\\
10.01	0.01\\
11.01	0.01\\
12.01	0.01\\
13.01	0.01\\
14.01	0.01\\
15.01	0.01\\
16.01	0.01\\
17.01	0.01\\
18.01	0.01\\
19.01	0.01\\
20.01	0.01\\
21.01	0.01\\
22.01	0.01\\
23.01	0.01\\
24.01	0.01\\
25.01	0.01\\
26.01	0.01\\
27.01	0.01\\
28.01	0.01\\
29.01	0.01\\
30.01	0.01\\
31.01	0.01\\
32.01	0.01\\
33.01	0.01\\
34.01	0.01\\
35.01	0.01\\
36.01	0.01\\
37.01	0.01\\
38.01	0.01\\
39.01	0.01\\
40.01	0.01\\
41.01	0.01\\
42.01	0.01\\
43.01	0.01\\
44.01	0.01\\
45.01	0.01\\
46.01	0.01\\
47.01	0.01\\
48.01	0.01\\
49.01	0.01\\
50.01	0.01\\
51.01	0.01\\
52.01	0.01\\
53.01	0.01\\
54.01	0.01\\
55.01	0.01\\
56.01	0.01\\
57.01	0.01\\
58.01	0.01\\
59.01	0.01\\
60.01	0.01\\
61.01	0.01\\
62.01	0.01\\
63.01	0.01\\
64.01	0.01\\
65.01	0.01\\
66.01	0.01\\
67.01	0.01\\
68.01	0.01\\
69.01	0.01\\
70.01	0.01\\
71.01	0.01\\
72.01	0.01\\
73.01	0.01\\
74.01	0.01\\
75.01	0.01\\
76.01	0.01\\
77.01	0.01\\
78.01	0.01\\
79.01	0.01\\
80.01	0.01\\
81.01	0.01\\
82.01	0.01\\
83.01	0.01\\
84.01	0.01\\
85.01	0.01\\
86.01	0.01\\
87.01	0.01\\
88.01	0.01\\
89.01	0.01\\
90.01	0.01\\
91.01	0.01\\
92.01	0.01\\
93.01	0.01\\
94.01	0.01\\
95.01	0.01\\
96.01	0.01\\
97.01	0.01\\
98.01	0.01\\
99.01	0.01\\
100.01	0.01\\
101.01	0.01\\
102.01	0.01\\
103.01	0.01\\
104.01	0.01\\
105.01	0.01\\
106.01	0.01\\
107.01	0.01\\
108.01	0.01\\
109.01	0.01\\
110.01	0.01\\
111.01	0.01\\
112.01	0.01\\
113.01	0.01\\
114.01	0.01\\
115.01	0.01\\
116.01	0.01\\
117.01	0.01\\
118.01	0.01\\
119.01	0.01\\
120.01	0.01\\
121.01	0.01\\
122.01	0.01\\
123.01	0.01\\
124.01	0.01\\
125.01	0.01\\
126.01	0.01\\
127.01	0.01\\
128.01	0.01\\
129.01	0.01\\
130.01	0.01\\
131.01	0.01\\
132.01	0.01\\
133.01	0.01\\
134.01	0.01\\
135.01	0.01\\
136.01	0.01\\
137.01	0.01\\
138.01	0.01\\
139.01	0.01\\
140.01	0.01\\
141.01	0.01\\
142.01	0.01\\
143.01	0.01\\
144.01	0.01\\
145.01	0.01\\
146.01	0.01\\
147.01	0.01\\
148.01	0.01\\
149.01	0.01\\
150.01	0.01\\
151.01	0.01\\
152.01	0.01\\
153.01	0.01\\
154.01	0.01\\
155.01	0.01\\
156.01	0.01\\
157.01	0.01\\
158.01	0.01\\
159.01	0.01\\
160.01	0.01\\
161.01	0.01\\
162.01	0.01\\
163.01	0.01\\
164.01	0.01\\
165.01	0.01\\
166.01	0.01\\
167.01	0.01\\
168.01	0.01\\
169.01	0.01\\
170.01	0.01\\
171.01	0.01\\
172.01	0.01\\
173.01	0.01\\
174.01	0.01\\
175.01	0.01\\
176.01	0.01\\
177.01	0.01\\
178.01	0.01\\
179.01	0.01\\
180.01	0.01\\
181.01	0.01\\
182.01	0.01\\
183.01	0.01\\
184.01	0.01\\
185.01	0.01\\
186.01	0.01\\
187.01	0.01\\
188.01	0.01\\
189.01	0.01\\
190.01	0.01\\
191.01	0.01\\
192.01	0.01\\
193.01	0.01\\
194.01	0.01\\
195.01	0.01\\
196.01	0.01\\
197.01	0.01\\
198.01	0.01\\
199.01	0.01\\
200.01	0.01\\
201.01	0.01\\
202.01	0.01\\
203.01	0.01\\
204.01	0.01\\
205.01	0.01\\
206.01	0.01\\
207.01	0.01\\
208.01	0.01\\
209.01	0.01\\
210.01	0.01\\
211.01	0.01\\
212.01	0.01\\
213.01	0.01\\
214.01	0.01\\
215.01	0.01\\
216.01	0.01\\
217.01	0.01\\
218.01	0.01\\
219.01	0.01\\
220.01	0.01\\
221.01	0.01\\
222.01	0.01\\
223.01	0.01\\
224.01	0.01\\
225.01	0.01\\
226.01	0.01\\
227.01	0.01\\
228.01	0.01\\
229.01	0.01\\
230.01	0.01\\
231.01	0.01\\
232.01	0.01\\
233.01	0.01\\
234.01	0.01\\
235.01	0.01\\
236.01	0.01\\
237.01	0.01\\
238.01	0.01\\
239.01	0.01\\
240.01	0.01\\
241.01	0.01\\
242.01	0.01\\
243.01	0.01\\
244.01	0.01\\
245.01	0.01\\
246.01	0.01\\
247.01	0.01\\
248.01	0.01\\
249.01	0.01\\
250.01	0.01\\
251.01	0.01\\
252.01	0.01\\
253.01	0.01\\
254.01	0.01\\
255.01	0.01\\
256.01	0.01\\
257.01	0.01\\
258.01	0.01\\
259.01	0.01\\
260.01	0.01\\
261.01	0.01\\
262.01	0.01\\
263.01	0.01\\
264.01	0.01\\
265.01	0.01\\
266.01	0.01\\
267.01	0.01\\
268.01	0.01\\
269.01	0.01\\
270.01	0.01\\
271.01	0.01\\
272.01	0.01\\
273.01	0.01\\
274.01	0.01\\
275.01	0.01\\
276.01	0.01\\
277.01	0.01\\
278.01	0.01\\
279.01	0.01\\
280.01	0.01\\
281.01	0.01\\
282.01	0.01\\
283.01	0.01\\
284.01	0.01\\
285.01	0.01\\
286.01	0.01\\
287.01	0.01\\
288.01	0.01\\
289.01	0.01\\
290.01	0.01\\
291.01	0.01\\
292.01	0.01\\
293.01	0.01\\
294.01	0.01\\
295.01	0.01\\
296.01	0.01\\
297.01	0.01\\
298.01	0.01\\
299.01	0.01\\
300.01	0.01\\
301.01	0.01\\
302.01	0.01\\
303.01	0.01\\
304.01	0.01\\
305.01	0.01\\
306.01	0.01\\
307.01	0.01\\
308.01	0.01\\
309.01	0.01\\
310.01	0.01\\
311.01	0.01\\
312.01	0.01\\
313.01	0.01\\
314.01	0.01\\
315.01	0.01\\
316.01	0.01\\
317.01	0.01\\
318.01	0.01\\
319.01	0.01\\
320.01	0.01\\
321.01	0.01\\
322.01	0.01\\
323.01	0.01\\
324.01	0.01\\
325.01	0.01\\
326.01	0.01\\
327.01	0.01\\
328.01	0.01\\
329.01	0.01\\
330.01	0.01\\
331.01	0.01\\
332.01	0.01\\
333.01	0.01\\
334.01	0.01\\
335.01	0.01\\
336.01	0.01\\
337.01	0.01\\
338.01	0.01\\
339.01	0.01\\
340.01	0.01\\
341.01	0.01\\
342.01	0.01\\
343.01	0.01\\
344.01	0.01\\
345.01	0.01\\
346.01	0.01\\
347.01	0.01\\
348.01	0.01\\
349.01	0.01\\
350.01	0.01\\
351.01	0.01\\
352.01	0.01\\
353.01	0.01\\
354.01	0.01\\
355.01	0.01\\
356.01	0.01\\
357.01	0.01\\
358.01	0.01\\
359.01	0.01\\
360.01	0.01\\
361.01	0.01\\
362.01	0.01\\
363.01	0.01\\
364.01	0.01\\
365.01	0.01\\
366.01	0.01\\
367.01	0.01\\
368.01	0.01\\
369.01	0.01\\
370.01	0.01\\
371.01	0.01\\
372.01	0.01\\
373.01	0.01\\
374.01	0.01\\
375.01	0.01\\
376.01	0.01\\
377.01	0.01\\
378.01	0.01\\
379.01	0.01\\
380.01	0.01\\
381.01	0.01\\
382.01	0.01\\
383.01	0.01\\
384.01	0.01\\
385.01	0.01\\
386.01	0.01\\
387.01	0.01\\
388.01	0.01\\
389.01	0.01\\
390.01	0.01\\
391.01	0.01\\
392.01	0.01\\
393.01	0.01\\
394.01	0.01\\
395.01	0.01\\
396.01	0.01\\
397.01	0.01\\
398.01	0.01\\
399.01	0.01\\
400.01	0.01\\
401.01	0.01\\
402.01	0.01\\
403.01	0.01\\
404.01	0.01\\
405.01	0.01\\
406.01	0.01\\
407.01	0.01\\
408.01	0.01\\
409.01	0.01\\
410.01	0.01\\
411.01	0.01\\
412.01	0.01\\
413.01	0.01\\
414.01	0.01\\
415.01	0.01\\
416.01	0.01\\
417.01	0.01\\
418.01	0.01\\
419.01	0.01\\
420.01	0.01\\
421.01	0.01\\
422.01	0.01\\
423.01	0.01\\
424.01	0.01\\
425.01	0.01\\
426.01	0.01\\
427.01	0.01\\
428.01	0.01\\
429.01	0.01\\
430.01	0.01\\
431.01	0.01\\
432.01	0.01\\
433.01	0.01\\
434.01	0.01\\
435.01	0.01\\
436.01	0.01\\
437.01	0.01\\
438.01	0.01\\
439.01	0.01\\
440.01	0.01\\
441.01	0.01\\
442.01	0.01\\
443.01	0.01\\
444.01	0.01\\
445.01	0.01\\
446.01	0.01\\
447.01	0.01\\
448.01	0.01\\
449.01	0.01\\
450.01	0.01\\
451.01	0.01\\
452.01	0.01\\
453.01	0.01\\
454.01	0.01\\
455.01	0.01\\
456.01	0.01\\
457.01	0.01\\
458.01	0.01\\
459.01	0.01\\
460.01	0.01\\
461.01	0.01\\
462.01	0.01\\
463.01	0.01\\
464.01	0.01\\
465.01	0.01\\
466.01	0.01\\
467.01	0.01\\
468.01	0.01\\
469.01	0.01\\
470.01	0.01\\
471.01	0.01\\
472.01	0.01\\
473.01	0.01\\
474.01	0.01\\
475.01	0.01\\
476.01	0.01\\
477.01	0.01\\
478.01	0.01\\
479.01	0.01\\
480.01	0.01\\
481.01	0.01\\
482.01	0.01\\
483.01	0.01\\
484.01	0.01\\
485.01	0.01\\
486.01	0.01\\
487.01	0.01\\
488.01	0.01\\
489.01	0.01\\
490.01	0.01\\
491.01	0.01\\
492.01	0.01\\
493.01	0.01\\
494.01	0.01\\
495.01	0.01\\
496.01	0.01\\
497.01	0.01\\
498.01	0.01\\
499.01	0.01\\
500.01	0.01\\
501.01	0.01\\
502.01	0.01\\
503.01	0.01\\
504.01	0.01\\
505.01	0.01\\
506.01	0.01\\
507.01	0.01\\
508.01	0.01\\
509.01	0.01\\
510.01	0.01\\
511.01	0.01\\
512.01	0.01\\
513.01	0.01\\
514.01	0.01\\
515.01	0.01\\
516.01	0.01\\
517.01	0.01\\
518.01	0.01\\
519.01	0.01\\
520.01	0.01\\
521.01	0.01\\
522.01	0.01\\
523.01	0.01\\
524.01	0.01\\
525.01	0.01\\
526.01	0.01\\
527.01	0.01\\
528.01	0.01\\
529.01	0.01\\
530.01	0.01\\
531.01	0.01\\
532.01	0.01\\
533.01	0.01\\
534.01	0.01\\
535.01	0.01\\
536.01	0.01\\
537.01	0.01\\
538.01	0.01\\
539.01	0.01\\
540.01	0.01\\
541.01	0.01\\
542.01	0.01\\
543.01	0.01\\
544.01	0.01\\
545.01	0.01\\
546.01	0.01\\
547.01	0.01\\
548.01	0.01\\
549.01	0.01\\
550.01	0.01\\
551.01	0.01\\
552.01	0.01\\
553.01	0.01\\
554.01	0.01\\
555.01	0.01\\
556.01	0.01\\
557.01	0.01\\
558.01	0.01\\
559.01	0.01\\
560.01	0.01\\
561.01	0.01\\
562.01	0.01\\
563.01	0.01\\
564.01	0.01\\
565.01	0.01\\
566.01	0.01\\
567.01	0.01\\
568.01	0.01\\
569.01	0.01\\
570.01	0.01\\
571.01	0.01\\
572.01	0.01\\
573.01	0.01\\
574.01	0.01\\
575.01	0.01\\
576.01	0.01\\
577.01	0.01\\
578.01	0.01\\
579.01	0.01\\
580.01	0.01\\
581.01	0.01\\
582.01	0.01\\
583.01	0.01\\
584.01	0.01\\
585.01	0.01\\
586.01	0.01\\
587.01	0.01\\
588.01	0.01\\
589.01	0.01\\
590.01	0.01\\
591.01	0.01\\
592.01	0.01\\
593.01	0.01\\
594.01	0.01\\
595.01	0.01\\
596.01	0.01\\
597.01	0.01\\
598.01	0.01\\
599.01	0.00623513569400518\\
599.02	0.00619747359804889\\
599.03	0.0061594447559382\\
599.04	0.00612104556201832\\
599.05	0.0060822723751792\\
599.06	0.00604312151850676\\
599.07	0.00600358927893093\\
599.08	0.00596367190687011\\
599.09	0.00592336561587207\\
599.1	0.0058826665822515\\
599.11	0.00584157094472381\\
599.12	0.00580007480403546\\
599.13	0.00575817422259058\\
599.14	0.00571586522407396\\
599.15	0.00567314379307033\\
599.16	0.00563000587467983\\
599.17	0.00558644737412983\\
599.18	0.00554246415638277\\
599.19	0.00549805204574035\\
599.2	0.00545320682544357\\
599.21	0.00540792423726908\\
599.22	0.00536219998112139\\
599.23	0.00531602971462119\\
599.24	0.00526940905268952\\
599.25	0.00522233356712793\\
599.26	0.0051747987861945\\
599.27	0.00512680019417563\\
599.28	0.00507833323095366\\
599.29	0.00502939329157028\\
599.3	0.0049799757257856\\
599.31	0.00493007583763285\\
599.32	0.00487968888496879\\
599.33	0.00482881007901965\\
599.34	0.00477743458392255\\
599.35	0.00472555751626256\\
599.36	0.00467317394460504\\
599.37	0.00462027888902352\\
599.38	0.0045668673206228\\
599.39	0.00451293416105749\\
599.4	0.00445847428204579\\
599.41	0.00440348251095323\\
599.42	0.00434795363554174\\
599.43	0.00429188239249109\\
599.44	0.00423526346689832\\
599.45	0.00417809149177216\\
599.46	0.00412036104752262\\
599.47	0.00406206666144545\\
599.48	0.00400320280720162\\
599.49	0.00394376390429167\\
599.5	0.00388374431752495\\
599.51	0.00382313835648361\\
599.52	0.00376194027498138\\
599.53	0.00370014427051699\\
599.54	0.00363774448372231\\
599.55	0.00357473499780496\\
599.56	0.00351110983798565\\
599.57	0.00344686297092978\\
599.58	0.0033819883041737\\
599.59	0.00331647968554507\\
599.6	0.00325033090257786\\
599.61	0.00318353568192137\\
599.62	0.00311608768874353\\
599.63	0.00304798052612843\\
599.64	0.00297920773446783\\
599.65	0.00290976279084677\\
599.66	0.00283963910842317\\
599.67	0.00276883003580129\\
599.68	0.00269732885639912\\
599.69	0.00262512878780951\\
599.7	0.0025522229811551\\
599.71	0.00247860452043686\\
599.72	0.00240426642187629\\
599.73	0.0023292016332512\\
599.74	0.00225340303322488\\
599.75	0.00217686343066881\\
599.76	0.00209957556397868\\
599.77	0.00202153210038366\\
599.78	0.00194272563524905\\
599.79	0.00186314869137186\\
599.8	0.00178279371826976\\
599.81	0.00170165309146283\\
599.82	0.00161971911174841\\
599.83	0.00153698400446879\\
599.84	0.00145343991877172\\
599.85	0.00136907892686372\\
599.86	0.00128389302325598\\
599.87	0.001197874124003\\
599.88	0.00111101406593363\\
599.89	0.00102330460587469\\
599.9	0.000934737419866901\\
599.91	0.000845304102373188\\
599.92	0.000754996165479166\\
599.93	0.000663805038085868\\
599.94	0.000571722065094475\\
599.95	0.000478738506583129\\
599.96	0.000384845536975638\\
599.97	0.000290034244202044\\
599.98	0.00019429562885097\\
599.99	9.76206033136556e-05\\
600	0\\
};
\addplot [color=mycolor6,solid,forget plot]
  table[row sep=crcr]{%
0.01	0.01\\
1.01	0.01\\
2.01	0.01\\
3.01	0.01\\
4.01	0.01\\
5.01	0.01\\
6.01	0.01\\
7.01	0.01\\
8.01	0.01\\
9.01	0.01\\
10.01	0.01\\
11.01	0.01\\
12.01	0.01\\
13.01	0.01\\
14.01	0.01\\
15.01	0.01\\
16.01	0.01\\
17.01	0.01\\
18.01	0.01\\
19.01	0.01\\
20.01	0.01\\
21.01	0.01\\
22.01	0.01\\
23.01	0.01\\
24.01	0.01\\
25.01	0.01\\
26.01	0.01\\
27.01	0.01\\
28.01	0.01\\
29.01	0.01\\
30.01	0.01\\
31.01	0.01\\
32.01	0.01\\
33.01	0.01\\
34.01	0.01\\
35.01	0.01\\
36.01	0.01\\
37.01	0.01\\
38.01	0.01\\
39.01	0.01\\
40.01	0.01\\
41.01	0.01\\
42.01	0.01\\
43.01	0.01\\
44.01	0.01\\
45.01	0.01\\
46.01	0.01\\
47.01	0.01\\
48.01	0.01\\
49.01	0.01\\
50.01	0.01\\
51.01	0.01\\
52.01	0.01\\
53.01	0.01\\
54.01	0.01\\
55.01	0.01\\
56.01	0.01\\
57.01	0.01\\
58.01	0.01\\
59.01	0.01\\
60.01	0.01\\
61.01	0.01\\
62.01	0.01\\
63.01	0.01\\
64.01	0.01\\
65.01	0.01\\
66.01	0.01\\
67.01	0.01\\
68.01	0.01\\
69.01	0.01\\
70.01	0.01\\
71.01	0.01\\
72.01	0.01\\
73.01	0.01\\
74.01	0.01\\
75.01	0.01\\
76.01	0.01\\
77.01	0.01\\
78.01	0.01\\
79.01	0.01\\
80.01	0.01\\
81.01	0.01\\
82.01	0.01\\
83.01	0.01\\
84.01	0.01\\
85.01	0.01\\
86.01	0.01\\
87.01	0.01\\
88.01	0.01\\
89.01	0.01\\
90.01	0.01\\
91.01	0.01\\
92.01	0.01\\
93.01	0.01\\
94.01	0.01\\
95.01	0.01\\
96.01	0.01\\
97.01	0.01\\
98.01	0.01\\
99.01	0.01\\
100.01	0.01\\
101.01	0.01\\
102.01	0.01\\
103.01	0.01\\
104.01	0.01\\
105.01	0.01\\
106.01	0.01\\
107.01	0.01\\
108.01	0.01\\
109.01	0.01\\
110.01	0.01\\
111.01	0.01\\
112.01	0.01\\
113.01	0.01\\
114.01	0.01\\
115.01	0.01\\
116.01	0.01\\
117.01	0.01\\
118.01	0.01\\
119.01	0.01\\
120.01	0.01\\
121.01	0.01\\
122.01	0.01\\
123.01	0.01\\
124.01	0.01\\
125.01	0.01\\
126.01	0.01\\
127.01	0.01\\
128.01	0.01\\
129.01	0.01\\
130.01	0.01\\
131.01	0.01\\
132.01	0.01\\
133.01	0.01\\
134.01	0.01\\
135.01	0.01\\
136.01	0.01\\
137.01	0.01\\
138.01	0.01\\
139.01	0.01\\
140.01	0.01\\
141.01	0.01\\
142.01	0.01\\
143.01	0.01\\
144.01	0.01\\
145.01	0.01\\
146.01	0.01\\
147.01	0.01\\
148.01	0.01\\
149.01	0.01\\
150.01	0.01\\
151.01	0.01\\
152.01	0.01\\
153.01	0.01\\
154.01	0.01\\
155.01	0.01\\
156.01	0.01\\
157.01	0.01\\
158.01	0.01\\
159.01	0.01\\
160.01	0.01\\
161.01	0.01\\
162.01	0.01\\
163.01	0.01\\
164.01	0.01\\
165.01	0.01\\
166.01	0.01\\
167.01	0.01\\
168.01	0.01\\
169.01	0.01\\
170.01	0.01\\
171.01	0.01\\
172.01	0.01\\
173.01	0.01\\
174.01	0.01\\
175.01	0.01\\
176.01	0.01\\
177.01	0.01\\
178.01	0.01\\
179.01	0.01\\
180.01	0.01\\
181.01	0.01\\
182.01	0.01\\
183.01	0.01\\
184.01	0.01\\
185.01	0.01\\
186.01	0.01\\
187.01	0.01\\
188.01	0.01\\
189.01	0.01\\
190.01	0.01\\
191.01	0.01\\
192.01	0.01\\
193.01	0.01\\
194.01	0.01\\
195.01	0.01\\
196.01	0.01\\
197.01	0.01\\
198.01	0.01\\
199.01	0.01\\
200.01	0.01\\
201.01	0.01\\
202.01	0.01\\
203.01	0.01\\
204.01	0.01\\
205.01	0.01\\
206.01	0.01\\
207.01	0.01\\
208.01	0.01\\
209.01	0.01\\
210.01	0.01\\
211.01	0.01\\
212.01	0.01\\
213.01	0.01\\
214.01	0.01\\
215.01	0.01\\
216.01	0.01\\
217.01	0.01\\
218.01	0.01\\
219.01	0.01\\
220.01	0.01\\
221.01	0.01\\
222.01	0.01\\
223.01	0.01\\
224.01	0.01\\
225.01	0.01\\
226.01	0.01\\
227.01	0.01\\
228.01	0.01\\
229.01	0.01\\
230.01	0.01\\
231.01	0.01\\
232.01	0.01\\
233.01	0.01\\
234.01	0.01\\
235.01	0.01\\
236.01	0.01\\
237.01	0.01\\
238.01	0.01\\
239.01	0.01\\
240.01	0.01\\
241.01	0.01\\
242.01	0.01\\
243.01	0.01\\
244.01	0.01\\
245.01	0.01\\
246.01	0.01\\
247.01	0.01\\
248.01	0.01\\
249.01	0.01\\
250.01	0.01\\
251.01	0.01\\
252.01	0.01\\
253.01	0.01\\
254.01	0.01\\
255.01	0.01\\
256.01	0.01\\
257.01	0.01\\
258.01	0.01\\
259.01	0.01\\
260.01	0.01\\
261.01	0.01\\
262.01	0.01\\
263.01	0.01\\
264.01	0.01\\
265.01	0.01\\
266.01	0.01\\
267.01	0.01\\
268.01	0.01\\
269.01	0.01\\
270.01	0.01\\
271.01	0.01\\
272.01	0.01\\
273.01	0.01\\
274.01	0.01\\
275.01	0.01\\
276.01	0.01\\
277.01	0.01\\
278.01	0.01\\
279.01	0.01\\
280.01	0.01\\
281.01	0.01\\
282.01	0.01\\
283.01	0.01\\
284.01	0.01\\
285.01	0.01\\
286.01	0.01\\
287.01	0.01\\
288.01	0.01\\
289.01	0.01\\
290.01	0.01\\
291.01	0.01\\
292.01	0.01\\
293.01	0.01\\
294.01	0.01\\
295.01	0.01\\
296.01	0.01\\
297.01	0.01\\
298.01	0.01\\
299.01	0.01\\
300.01	0.01\\
301.01	0.01\\
302.01	0.01\\
303.01	0.01\\
304.01	0.01\\
305.01	0.01\\
306.01	0.01\\
307.01	0.01\\
308.01	0.01\\
309.01	0.01\\
310.01	0.01\\
311.01	0.01\\
312.01	0.01\\
313.01	0.01\\
314.01	0.01\\
315.01	0.01\\
316.01	0.01\\
317.01	0.01\\
318.01	0.01\\
319.01	0.01\\
320.01	0.01\\
321.01	0.01\\
322.01	0.01\\
323.01	0.01\\
324.01	0.01\\
325.01	0.01\\
326.01	0.01\\
327.01	0.01\\
328.01	0.01\\
329.01	0.01\\
330.01	0.01\\
331.01	0.01\\
332.01	0.01\\
333.01	0.01\\
334.01	0.01\\
335.01	0.01\\
336.01	0.01\\
337.01	0.01\\
338.01	0.01\\
339.01	0.01\\
340.01	0.01\\
341.01	0.01\\
342.01	0.01\\
343.01	0.01\\
344.01	0.01\\
345.01	0.01\\
346.01	0.01\\
347.01	0.01\\
348.01	0.01\\
349.01	0.01\\
350.01	0.01\\
351.01	0.01\\
352.01	0.01\\
353.01	0.01\\
354.01	0.01\\
355.01	0.01\\
356.01	0.01\\
357.01	0.01\\
358.01	0.01\\
359.01	0.01\\
360.01	0.01\\
361.01	0.01\\
362.01	0.01\\
363.01	0.01\\
364.01	0.01\\
365.01	0.01\\
366.01	0.01\\
367.01	0.01\\
368.01	0.01\\
369.01	0.01\\
370.01	0.01\\
371.01	0.01\\
372.01	0.01\\
373.01	0.01\\
374.01	0.01\\
375.01	0.01\\
376.01	0.01\\
377.01	0.01\\
378.01	0.01\\
379.01	0.01\\
380.01	0.01\\
381.01	0.01\\
382.01	0.01\\
383.01	0.01\\
384.01	0.01\\
385.01	0.01\\
386.01	0.01\\
387.01	0.01\\
388.01	0.01\\
389.01	0.01\\
390.01	0.01\\
391.01	0.01\\
392.01	0.01\\
393.01	0.01\\
394.01	0.01\\
395.01	0.01\\
396.01	0.01\\
397.01	0.01\\
398.01	0.01\\
399.01	0.01\\
400.01	0.01\\
401.01	0.01\\
402.01	0.01\\
403.01	0.01\\
404.01	0.01\\
405.01	0.01\\
406.01	0.01\\
407.01	0.01\\
408.01	0.01\\
409.01	0.01\\
410.01	0.01\\
411.01	0.01\\
412.01	0.01\\
413.01	0.01\\
414.01	0.01\\
415.01	0.01\\
416.01	0.01\\
417.01	0.01\\
418.01	0.01\\
419.01	0.01\\
420.01	0.01\\
421.01	0.01\\
422.01	0.01\\
423.01	0.01\\
424.01	0.01\\
425.01	0.01\\
426.01	0.01\\
427.01	0.01\\
428.01	0.01\\
429.01	0.01\\
430.01	0.01\\
431.01	0.01\\
432.01	0.01\\
433.01	0.01\\
434.01	0.01\\
435.01	0.01\\
436.01	0.01\\
437.01	0.01\\
438.01	0.01\\
439.01	0.01\\
440.01	0.01\\
441.01	0.01\\
442.01	0.01\\
443.01	0.01\\
444.01	0.01\\
445.01	0.01\\
446.01	0.01\\
447.01	0.01\\
448.01	0.01\\
449.01	0.01\\
450.01	0.01\\
451.01	0.01\\
452.01	0.01\\
453.01	0.01\\
454.01	0.01\\
455.01	0.01\\
456.01	0.01\\
457.01	0.01\\
458.01	0.01\\
459.01	0.01\\
460.01	0.01\\
461.01	0.01\\
462.01	0.01\\
463.01	0.01\\
464.01	0.01\\
465.01	0.01\\
466.01	0.01\\
467.01	0.01\\
468.01	0.01\\
469.01	0.01\\
470.01	0.01\\
471.01	0.01\\
472.01	0.01\\
473.01	0.01\\
474.01	0.01\\
475.01	0.01\\
476.01	0.01\\
477.01	0.01\\
478.01	0.01\\
479.01	0.01\\
480.01	0.01\\
481.01	0.01\\
482.01	0.01\\
483.01	0.01\\
484.01	0.01\\
485.01	0.01\\
486.01	0.01\\
487.01	0.01\\
488.01	0.01\\
489.01	0.01\\
490.01	0.01\\
491.01	0.01\\
492.01	0.01\\
493.01	0.01\\
494.01	0.01\\
495.01	0.01\\
496.01	0.01\\
497.01	0.01\\
498.01	0.01\\
499.01	0.01\\
500.01	0.01\\
501.01	0.01\\
502.01	0.01\\
503.01	0.01\\
504.01	0.01\\
505.01	0.01\\
506.01	0.01\\
507.01	0.01\\
508.01	0.01\\
509.01	0.01\\
510.01	0.01\\
511.01	0.01\\
512.01	0.01\\
513.01	0.01\\
514.01	0.01\\
515.01	0.01\\
516.01	0.01\\
517.01	0.01\\
518.01	0.01\\
519.01	0.01\\
520.01	0.01\\
521.01	0.01\\
522.01	0.01\\
523.01	0.01\\
524.01	0.01\\
525.01	0.01\\
526.01	0.01\\
527.01	0.01\\
528.01	0.01\\
529.01	0.01\\
530.01	0.01\\
531.01	0.01\\
532.01	0.01\\
533.01	0.01\\
534.01	0.01\\
535.01	0.01\\
536.01	0.01\\
537.01	0.01\\
538.01	0.01\\
539.01	0.01\\
540.01	0.01\\
541.01	0.01\\
542.01	0.01\\
543.01	0.01\\
544.01	0.01\\
545.01	0.01\\
546.01	0.01\\
547.01	0.01\\
548.01	0.01\\
549.01	0.01\\
550.01	0.01\\
551.01	0.01\\
552.01	0.01\\
553.01	0.01\\
554.01	0.01\\
555.01	0.01\\
556.01	0.01\\
557.01	0.01\\
558.01	0.01\\
559.01	0.01\\
560.01	0.01\\
561.01	0.01\\
562.01	0.01\\
563.01	0.01\\
564.01	0.01\\
565.01	0.01\\
566.01	0.01\\
567.01	0.01\\
568.01	0.01\\
569.01	0.01\\
570.01	0.01\\
571.01	0.01\\
572.01	0.01\\
573.01	0.01\\
574.01	0.01\\
575.01	0.01\\
576.01	0.01\\
577.01	0.01\\
578.01	0.01\\
579.01	0.01\\
580.01	0.01\\
581.01	0.01\\
582.01	0.01\\
583.01	0.01\\
584.01	0.01\\
585.01	0.01\\
586.01	0.01\\
587.01	0.01\\
588.01	0.01\\
589.01	0.01\\
590.01	0.01\\
591.01	0.01\\
592.01	0.01\\
593.01	0.01\\
594.01	0.01\\
595.01	0.01\\
596.01	0.01\\
597.01	0.01\\
598.01	0.01\\
599.01	0.00623513569400517\\
599.02	0.00619747359804888\\
599.03	0.00615944475593818\\
599.04	0.00612104556201831\\
599.05	0.00608227237517917\\
599.06	0.00604312151850673\\
599.07	0.00600358927893092\\
599.08	0.00596367190687009\\
599.09	0.00592336561587206\\
599.1	0.00588266658225149\\
599.11	0.0058415709447238\\
599.12	0.00580007480403545\\
599.13	0.00575817422259057\\
599.14	0.00571586522407396\\
599.15	0.00567314379307032\\
599.16	0.00563000587467982\\
599.17	0.0055864473741298\\
599.18	0.00554246415638276\\
599.19	0.00549805204574034\\
599.2	0.00545320682544357\\
599.21	0.00540792423726907\\
599.22	0.00536219998112139\\
599.23	0.00531602971462118\\
599.24	0.0052694090526895\\
599.25	0.00522233356712793\\
599.26	0.00517479878619452\\
599.27	0.00512680019417564\\
599.28	0.00507833323095367\\
599.29	0.00502939329157031\\
599.3	0.00497997572578562\\
599.31	0.00493007583763287\\
599.32	0.00487968888496882\\
599.33	0.00482881007901966\\
599.34	0.00477743458392257\\
599.35	0.00472555751626259\\
599.36	0.00467317394460509\\
599.37	0.00462027888902355\\
599.38	0.00456686732062282\\
599.39	0.00451293416105752\\
599.4	0.00445847428204581\\
599.41	0.00440348251095325\\
599.42	0.00434795363554175\\
599.43	0.00429188239249109\\
599.44	0.00423526346689832\\
599.45	0.00417809149177218\\
599.46	0.00412036104752264\\
599.47	0.00406206666144547\\
599.48	0.00400320280720163\\
599.49	0.00394376390429167\\
599.5	0.00388374431752495\\
599.51	0.0038231383564836\\
599.52	0.00376194027498136\\
599.53	0.00370014427051698\\
599.54	0.0036377444837223\\
599.55	0.00357473499780495\\
599.56	0.00351110983798565\\
599.57	0.00344686297092979\\
599.58	0.0033819883041737\\
599.59	0.00331647968554508\\
599.6	0.00325033090257787\\
599.61	0.00318353568192137\\
599.62	0.00311608768874353\\
599.63	0.00304798052612843\\
599.64	0.00297920773446783\\
599.65	0.00290976279084677\\
599.66	0.00283963910842317\\
599.67	0.00276883003580129\\
599.68	0.00269732885639913\\
599.69	0.00262512878780953\\
599.7	0.00255222298115511\\
599.71	0.00247860452043686\\
599.72	0.0024042664218763\\
599.73	0.00232920163325121\\
599.74	0.00225340303322488\\
599.75	0.00217686343066882\\
599.76	0.00209957556397868\\
599.77	0.00202153210038366\\
599.78	0.00194272563524904\\
599.79	0.00186314869137186\\
599.8	0.00178279371826976\\
599.81	0.00170165309146283\\
599.82	0.00161971911174841\\
599.83	0.00153698400446879\\
599.84	0.00145343991877172\\
599.85	0.00136907892686371\\
599.86	0.00128389302325598\\
599.87	0.00119787412400299\\
599.88	0.00111101406593363\\
599.89	0.00102330460587468\\
599.9	0.000934737419866896\\
599.91	0.000845304102373183\\
599.92	0.000754996165479164\\
599.93	0.000663805038085864\\
599.94	0.000571722065094472\\
599.95	0.000478738506583127\\
599.96	0.000384845536975636\\
599.97	0.000290034244202042\\
599.98	0.000194295628850972\\
599.99	9.76206033136556e-05\\
600	0\\
};
\addplot [color=mycolor7,solid,forget plot]
  table[row sep=crcr]{%
0.01	0.00999999999999999\\
1.01	0.00999999999999999\\
2.01	0.00999999999999999\\
3.01	0.00999999999999999\\
4.01	0.00999999999999999\\
5.01	0.00999999999999999\\
6.01	0.00999999999999999\\
7.01	0.00999999999999999\\
8.01	0.00999999999999999\\
9.01	0.00999999999999999\\
10.01	0.00999999999999999\\
11.01	0.00999999999999999\\
12.01	0.00999999999999999\\
13.01	0.00999999999999999\\
14.01	0.00999999999999999\\
15.01	0.00999999999999999\\
16.01	0.00999999999999999\\
17.01	0.00999999999999999\\
18.01	0.00999999999999999\\
19.01	0.00999999999999999\\
20.01	0.00999999999999999\\
21.01	0.00999999999999999\\
22.01	0.00999999999999999\\
23.01	0.00999999999999999\\
24.01	0.00999999999999999\\
25.01	0.00999999999999999\\
26.01	0.00999999999999999\\
27.01	0.00999999999999999\\
28.01	0.00999999999999999\\
29.01	0.00999999999999999\\
30.01	0.00999999999999999\\
31.01	0.00999999999999999\\
32.01	0.00999999999999999\\
33.01	0.00999999999999999\\
34.01	0.00999999999999999\\
35.01	0.00999999999999999\\
36.01	0.00999999999999999\\
37.01	0.00999999999999999\\
38.01	0.00999999999999999\\
39.01	0.00999999999999999\\
40.01	0.00999999999999999\\
41.01	0.00999999999999999\\
42.01	0.00999999999999999\\
43.01	0.00999999999999999\\
44.01	0.00999999999999999\\
45.01	0.00999999999999999\\
46.01	0.00999999999999999\\
47.01	0.00999999999999999\\
48.01	0.00999999999999999\\
49.01	0.00999999999999999\\
50.01	0.00999999999999999\\
51.01	0.00999999999999999\\
52.01	0.00999999999999999\\
53.01	0.00999999999999999\\
54.01	0.00999999999999999\\
55.01	0.00999999999999999\\
56.01	0.00999999999999999\\
57.01	0.00999999999999999\\
58.01	0.00999999999999999\\
59.01	0.00999999999999999\\
60.01	0.00999999999999999\\
61.01	0.00999999999999999\\
62.01	0.00999999999999999\\
63.01	0.00999999999999999\\
64.01	0.00999999999999999\\
65.01	0.00999999999999999\\
66.01	0.00999999999999999\\
67.01	0.00999999999999999\\
68.01	0.00999999999999999\\
69.01	0.00999999999999999\\
70.01	0.00999999999999999\\
71.01	0.00999999999999999\\
72.01	0.00999999999999999\\
73.01	0.00999999999999999\\
74.01	0.00999999999999999\\
75.01	0.00999999999999999\\
76.01	0.00999999999999999\\
77.01	0.00999999999999999\\
78.01	0.00999999999999999\\
79.01	0.00999999999999999\\
80.01	0.00999999999999999\\
81.01	0.00999999999999999\\
82.01	0.00999999999999999\\
83.01	0.00999999999999999\\
84.01	0.00999999999999999\\
85.01	0.00999999999999999\\
86.01	0.00999999999999999\\
87.01	0.00999999999999999\\
88.01	0.00999999999999999\\
89.01	0.00999999999999999\\
90.01	0.00999999999999999\\
91.01	0.00999999999999999\\
92.01	0.00999999999999999\\
93.01	0.00999999999999999\\
94.01	0.00999999999999999\\
95.01	0.00999999999999999\\
96.01	0.00999999999999999\\
97.01	0.00999999999999999\\
98.01	0.00999999999999999\\
99.01	0.00999999999999999\\
100.01	0.00999999999999999\\
101.01	0.00999999999999999\\
102.01	0.00999999999999999\\
103.01	0.00999999999999999\\
104.01	0.00999999999999999\\
105.01	0.00999999999999999\\
106.01	0.00999999999999999\\
107.01	0.00999999999999999\\
108.01	0.00999999999999999\\
109.01	0.00999999999999999\\
110.01	0.00999999999999999\\
111.01	0.00999999999999999\\
112.01	0.00999999999999999\\
113.01	0.00999999999999999\\
114.01	0.00999999999999999\\
115.01	0.00999999999999999\\
116.01	0.00999999999999999\\
117.01	0.00999999999999999\\
118.01	0.00999999999999999\\
119.01	0.00999999999999999\\
120.01	0.00999999999999999\\
121.01	0.00999999999999999\\
122.01	0.00999999999999999\\
123.01	0.00999999999999999\\
124.01	0.00999999999999999\\
125.01	0.00999999999999999\\
126.01	0.00999999999999999\\
127.01	0.00999999999999999\\
128.01	0.00999999999999999\\
129.01	0.00999999999999999\\
130.01	0.00999999999999999\\
131.01	0.00999999999999999\\
132.01	0.00999999999999999\\
133.01	0.00999999999999999\\
134.01	0.00999999999999999\\
135.01	0.00999999999999999\\
136.01	0.00999999999999999\\
137.01	0.00999999999999999\\
138.01	0.00999999999999999\\
139.01	0.00999999999999999\\
140.01	0.00999999999999999\\
141.01	0.00999999999999999\\
142.01	0.00999999999999999\\
143.01	0.00999999999999999\\
144.01	0.00999999999999999\\
145.01	0.00999999999999999\\
146.01	0.00999999999999999\\
147.01	0.00999999999999999\\
148.01	0.00999999999999999\\
149.01	0.00999999999999999\\
150.01	0.00999999999999999\\
151.01	0.00999999999999999\\
152.01	0.00999999999999999\\
153.01	0.00999999999999999\\
154.01	0.00999999999999999\\
155.01	0.00999999999999999\\
156.01	0.00999999999999999\\
157.01	0.00999999999999999\\
158.01	0.00999999999999999\\
159.01	0.00999999999999999\\
160.01	0.00999999999999999\\
161.01	0.00999999999999999\\
162.01	0.00999999999999999\\
163.01	0.00999999999999999\\
164.01	0.00999999999999999\\
165.01	0.00999999999999999\\
166.01	0.00999999999999999\\
167.01	0.00999999999999999\\
168.01	0.00999999999999999\\
169.01	0.00999999999999999\\
170.01	0.00999999999999999\\
171.01	0.00999999999999999\\
172.01	0.00999999999999999\\
173.01	0.00999999999999999\\
174.01	0.00999999999999999\\
175.01	0.00999999999999999\\
176.01	0.00999999999999999\\
177.01	0.00999999999999999\\
178.01	0.00999999999999999\\
179.01	0.00999999999999999\\
180.01	0.00999999999999999\\
181.01	0.00999999999999999\\
182.01	0.00999999999999999\\
183.01	0.00999999999999999\\
184.01	0.00999999999999999\\
185.01	0.00999999999999999\\
186.01	0.00999999999999999\\
187.01	0.00999999999999999\\
188.01	0.00999999999999999\\
189.01	0.00999999999999999\\
190.01	0.00999999999999999\\
191.01	0.00999999999999999\\
192.01	0.00999999999999999\\
193.01	0.00999999999999999\\
194.01	0.00999999999999999\\
195.01	0.00999999999999999\\
196.01	0.00999999999999999\\
197.01	0.00999999999999999\\
198.01	0.00999999999999999\\
199.01	0.00999999999999999\\
200.01	0.00999999999999999\\
201.01	0.00999999999999999\\
202.01	0.00999999999999999\\
203.01	0.00999999999999999\\
204.01	0.00999999999999999\\
205.01	0.00999999999999999\\
206.01	0.00999999999999999\\
207.01	0.00999999999999999\\
208.01	0.00999999999999999\\
209.01	0.00999999999999999\\
210.01	0.00999999999999999\\
211.01	0.00999999999999999\\
212.01	0.00999999999999999\\
213.01	0.00999999999999999\\
214.01	0.00999999999999999\\
215.01	0.00999999999999999\\
216.01	0.00999999999999999\\
217.01	0.00999999999999999\\
218.01	0.00999999999999999\\
219.01	0.00999999999999999\\
220.01	0.00999999999999999\\
221.01	0.00999999999999999\\
222.01	0.00999999999999999\\
223.01	0.00999999999999999\\
224.01	0.00999999999999999\\
225.01	0.00999999999999999\\
226.01	0.00999999999999999\\
227.01	0.00999999999999999\\
228.01	0.00999999999999999\\
229.01	0.00999999999999999\\
230.01	0.00999999999999999\\
231.01	0.00999999999999999\\
232.01	0.00999999999999999\\
233.01	0.00999999999999999\\
234.01	0.00999999999999999\\
235.01	0.00999999999999999\\
236.01	0.00999999999999999\\
237.01	0.00999999999999999\\
238.01	0.00999999999999999\\
239.01	0.00999999999999999\\
240.01	0.00999999999999999\\
241.01	0.00999999999999999\\
242.01	0.00999999999999999\\
243.01	0.00999999999999999\\
244.01	0.00999999999999999\\
245.01	0.00999999999999999\\
246.01	0.00999999999999999\\
247.01	0.00999999999999999\\
248.01	0.00999999999999999\\
249.01	0.00999999999999999\\
250.01	0.00999999999999999\\
251.01	0.00999999999999999\\
252.01	0.00999999999999999\\
253.01	0.00999999999999999\\
254.01	0.00999999999999999\\
255.01	0.00999999999999999\\
256.01	0.00999999999999999\\
257.01	0.00999999999999999\\
258.01	0.00999999999999999\\
259.01	0.00999999999999999\\
260.01	0.00999999999999999\\
261.01	0.00999999999999999\\
262.01	0.00999999999999999\\
263.01	0.00999999999999999\\
264.01	0.00999999999999999\\
265.01	0.00999999999999999\\
266.01	0.00999999999999999\\
267.01	0.00999999999999999\\
268.01	0.00999999999999999\\
269.01	0.00999999999999999\\
270.01	0.00999999999999999\\
271.01	0.00999999999999999\\
272.01	0.00999999999999999\\
273.01	0.00999999999999999\\
274.01	0.00999999999999999\\
275.01	0.00999999999999999\\
276.01	0.00999999999999999\\
277.01	0.00999999999999999\\
278.01	0.00999999999999999\\
279.01	0.00999999999999999\\
280.01	0.00999999999999999\\
281.01	0.00999999999999999\\
282.01	0.00999999999999999\\
283.01	0.00999999999999999\\
284.01	0.00999999999999999\\
285.01	0.00999999999999999\\
286.01	0.00999999999999999\\
287.01	0.00999999999999999\\
288.01	0.00999999999999999\\
289.01	0.00999999999999999\\
290.01	0.00999999999999999\\
291.01	0.00999999999999999\\
292.01	0.00999999999999999\\
293.01	0.00999999999999999\\
294.01	0.00999999999999999\\
295.01	0.00999999999999999\\
296.01	0.00999999999999999\\
297.01	0.00999999999999999\\
298.01	0.00999999999999999\\
299.01	0.00999999999999999\\
300.01	0.00999999999999999\\
301.01	0.00999999999999999\\
302.01	0.00999999999999999\\
303.01	0.00999999999999999\\
304.01	0.00999999999999999\\
305.01	0.00999999999999999\\
306.01	0.00999999999999999\\
307.01	0.00999999999999999\\
308.01	0.00999999999999999\\
309.01	0.00999999999999999\\
310.01	0.00999999999999999\\
311.01	0.00999999999999999\\
312.01	0.00999999999999999\\
313.01	0.00999999999999999\\
314.01	0.00999999999999999\\
315.01	0.00999999999999999\\
316.01	0.00999999999999999\\
317.01	0.00999999999999999\\
318.01	0.00999999999999999\\
319.01	0.00999999999999999\\
320.01	0.00999999999999999\\
321.01	0.00999999999999999\\
322.01	0.00999999999999999\\
323.01	0.00999999999999999\\
324.01	0.00999999999999999\\
325.01	0.00999999999999999\\
326.01	0.00999999999999999\\
327.01	0.00999999999999999\\
328.01	0.00999999999999999\\
329.01	0.00999999999999999\\
330.01	0.00999999999999999\\
331.01	0.00999999999999999\\
332.01	0.00999999999999999\\
333.01	0.00999999999999999\\
334.01	0.00999999999999999\\
335.01	0.00999999999999999\\
336.01	0.00999999999999999\\
337.01	0.00999999999999999\\
338.01	0.00999999999999999\\
339.01	0.00999999999999999\\
340.01	0.00999999999999999\\
341.01	0.00999999999999999\\
342.01	0.00999999999999999\\
343.01	0.00999999999999999\\
344.01	0.00999999999999999\\
345.01	0.00999999999999999\\
346.01	0.00999999999999999\\
347.01	0.00999999999999999\\
348.01	0.00999999999999999\\
349.01	0.00999999999999999\\
350.01	0.00999999999999999\\
351.01	0.00999999999999999\\
352.01	0.00999999999999999\\
353.01	0.00999999999999999\\
354.01	0.00999999999999999\\
355.01	0.00999999999999999\\
356.01	0.00999999999999999\\
357.01	0.00999999999999999\\
358.01	0.00999999999999999\\
359.01	0.00999999999999999\\
360.01	0.00999999999999999\\
361.01	0.00999999999999999\\
362.01	0.00999999999999999\\
363.01	0.00999999999999999\\
364.01	0.00999999999999999\\
365.01	0.00999999999999999\\
366.01	0.00999999999999999\\
367.01	0.00999999999999999\\
368.01	0.00999999999999999\\
369.01	0.00999999999999999\\
370.01	0.00999999999999999\\
371.01	0.00999999999999999\\
372.01	0.00999999999999999\\
373.01	0.00999999999999999\\
374.01	0.00999999999999999\\
375.01	0.00999999999999999\\
376.01	0.00999999999999999\\
377.01	0.00999999999999999\\
378.01	0.00999999999999999\\
379.01	0.00999999999999999\\
380.01	0.00999999999999999\\
381.01	0.00999999999999999\\
382.01	0.00999999999999999\\
383.01	0.00999999999999999\\
384.01	0.00999999999999999\\
385.01	0.00999999999999999\\
386.01	0.00999999999999999\\
387.01	0.00999999999999999\\
388.01	0.00999999999999999\\
389.01	0.00999999999999999\\
390.01	0.00999999999999999\\
391.01	0.00999999999999999\\
392.01	0.00999999999999999\\
393.01	0.00999999999999999\\
394.01	0.00999999999999999\\
395.01	0.00999999999999999\\
396.01	0.00999999999999999\\
397.01	0.00999999999999999\\
398.01	0.00999999999999999\\
399.01	0.00999999999999999\\
400.01	0.00999999999999999\\
401.01	0.00999999999999999\\
402.01	0.00999999999999999\\
403.01	0.00999999999999999\\
404.01	0.00999999999999999\\
405.01	0.00999999999999999\\
406.01	0.00999999999999999\\
407.01	0.00999999999999999\\
408.01	0.00999999999999999\\
409.01	0.00999999999999999\\
410.01	0.00999999999999999\\
411.01	0.00999999999999999\\
412.01	0.00999999999999999\\
413.01	0.00999999999999999\\
414.01	0.00999999999999999\\
415.01	0.00999999999999999\\
416.01	0.00999999999999999\\
417.01	0.00999999999999999\\
418.01	0.00999999999999999\\
419.01	0.00999999999999999\\
420.01	0.00999999999999999\\
421.01	0.00999999999999999\\
422.01	0.00999999999999999\\
423.01	0.00999999999999999\\
424.01	0.00999999999999999\\
425.01	0.00999999999999999\\
426.01	0.00999999999999999\\
427.01	0.00999999999999999\\
428.01	0.00999999999999999\\
429.01	0.00999999999999999\\
430.01	0.00999999999999999\\
431.01	0.00999999999999999\\
432.01	0.00999999999999999\\
433.01	0.00999999999999999\\
434.01	0.00999999999999999\\
435.01	0.00999999999999999\\
436.01	0.00999999999999999\\
437.01	0.00999999999999999\\
438.01	0.00999999999999999\\
439.01	0.00999999999999999\\
440.01	0.00999999999999999\\
441.01	0.00999999999999999\\
442.01	0.00999999999999999\\
443.01	0.00999999999999999\\
444.01	0.00999999999999999\\
445.01	0.00999999999999999\\
446.01	0.00999999999999999\\
447.01	0.00999999999999999\\
448.01	0.00999999999999999\\
449.01	0.00999999999999999\\
450.01	0.00999999999999999\\
451.01	0.00999999999999999\\
452.01	0.00999999999999999\\
453.01	0.00999999999999999\\
454.01	0.00999999999999999\\
455.01	0.00999999999999999\\
456.01	0.00999999999999999\\
457.01	0.00999999999999999\\
458.01	0.00999999999999999\\
459.01	0.00999999999999999\\
460.01	0.00999999999999999\\
461.01	0.00999999999999999\\
462.01	0.00999999999999999\\
463.01	0.00999999999999999\\
464.01	0.00999999999999999\\
465.01	0.00999999999999999\\
466.01	0.00999999999999999\\
467.01	0.00999999999999999\\
468.01	0.00999999999999999\\
469.01	0.00999999999999999\\
470.01	0.00999999999999999\\
471.01	0.00999999999999999\\
472.01	0.00999999999999999\\
473.01	0.00999999999999999\\
474.01	0.00999999999999999\\
475.01	0.00999999999999999\\
476.01	0.00999999999999999\\
477.01	0.00999999999999999\\
478.01	0.00999999999999999\\
479.01	0.00999999999999999\\
480.01	0.00999999999999999\\
481.01	0.00999999999999999\\
482.01	0.00999999999999999\\
483.01	0.00999999999999999\\
484.01	0.00999999999999999\\
485.01	0.00999999999999999\\
486.01	0.00999999999999999\\
487.01	0.00999999999999999\\
488.01	0.00999999999999999\\
489.01	0.00999999999999999\\
490.01	0.00999999999999999\\
491.01	0.00999999999999999\\
492.01	0.00999999999999999\\
493.01	0.00999999999999999\\
494.01	0.00999999999999999\\
495.01	0.00999999999999999\\
496.01	0.00999999999999999\\
497.01	0.00999999999999999\\
498.01	0.00999999999999999\\
499.01	0.00999999999999999\\
500.01	0.00999999999999999\\
501.01	0.00999999999999999\\
502.01	0.00999999999999999\\
503.01	0.00999999999999999\\
504.01	0.00999999999999999\\
505.01	0.00999999999999999\\
506.01	0.00999999999999999\\
507.01	0.00999999999999999\\
508.01	0.00999999999999999\\
509.01	0.00999999999999999\\
510.01	0.00999999999999999\\
511.01	0.00999999999999999\\
512.01	0.00999999999999999\\
513.01	0.00999999999999999\\
514.01	0.00999999999999999\\
515.01	0.00999999999999999\\
516.01	0.00999999999999999\\
517.01	0.00999999999999999\\
518.01	0.00999999999999999\\
519.01	0.00999999999999999\\
520.01	0.00999999999999999\\
521.01	0.00999999999999999\\
522.01	0.00999999999999999\\
523.01	0.00999999999999999\\
524.01	0.00999999999999999\\
525.01	0.00999999999999999\\
526.01	0.00999999999999999\\
527.01	0.00999999999999999\\
528.01	0.00999999999999999\\
529.01	0.00999999999999999\\
530.01	0.00999999999999999\\
531.01	0.00999999999999999\\
532.01	0.00999999999999999\\
533.01	0.00999999999999999\\
534.01	0.00999999999999999\\
535.01	0.00999999999999999\\
536.01	0.00999999999999999\\
537.01	0.00999999999999999\\
538.01	0.00999999999999999\\
539.01	0.00999999999999999\\
540.01	0.00999999999999999\\
541.01	0.00999999999999999\\
542.01	0.00999999999999999\\
543.01	0.00999999999999999\\
544.01	0.00999999999999999\\
545.01	0.00999999999999999\\
546.01	0.00999999999999999\\
547.01	0.00999999999999999\\
548.01	0.00999999999999999\\
549.01	0.00999999999999999\\
550.01	0.00999999999999999\\
551.01	0.00999999999999999\\
552.01	0.00999999999999999\\
553.01	0.00999999999999999\\
554.01	0.00999999999999999\\
555.01	0.00999999999999999\\
556.01	0.00999999999999999\\
557.01	0.00999999999999999\\
558.01	0.00999999999999999\\
559.01	0.00999999999999999\\
560.01	0.00999999999999999\\
561.01	0.00999999999999999\\
562.01	0.00999999999999999\\
563.01	0.00999999999999999\\
564.01	0.00999999999999999\\
565.01	0.00999999999999999\\
566.01	0.00999999999999999\\
567.01	0.00999999999999999\\
568.01	0.00999999999999999\\
569.01	0.00999999999999999\\
570.01	0.00999999999999999\\
571.01	0.00999999999999999\\
572.01	0.00999999999999999\\
573.01	0.00999999999999999\\
574.01	0.00999999999999999\\
575.01	0.00999999999999999\\
576.01	0.00999999999999999\\
577.01	0.00999999999999999\\
578.01	0.00999999999999999\\
579.01	0.00999999999999999\\
580.01	0.00999999999999999\\
581.01	0.00999999999999999\\
582.01	0.00999999999999999\\
583.01	0.00999999999999999\\
584.01	0.00999999999999999\\
585.01	0.00999999999999999\\
586.01	0.00999999999999999\\
587.01	0.00999999999999999\\
588.01	0.00999999999999999\\
589.01	0.00999999999999999\\
590.01	0.00999999999999999\\
591.01	0.00999999999999999\\
592.01	0.00999999999999999\\
593.01	0.00999999999999999\\
594.01	0.00999999999999999\\
595.01	0.00999999999999999\\
596.01	0.00999999999999999\\
597.01	0.00999999999999999\\
598.01	0.00999999999999999\\
599.01	0.00623513569400514\\
599.02	0.00619747359804885\\
599.03	0.00615944475593815\\
599.04	0.00612104556201828\\
599.05	0.00608227237517915\\
599.06	0.00604312151850672\\
599.07	0.00600358927893089\\
599.08	0.00596367190687008\\
599.09	0.00592336561587206\\
599.1	0.00588266658225149\\
599.11	0.0058415709447238\\
599.12	0.00580007480403545\\
599.13	0.00575817422259057\\
599.14	0.00571586522407395\\
599.15	0.00567314379307032\\
599.16	0.00563000587467982\\
599.17	0.00558644737412981\\
599.18	0.00554246415638277\\
599.19	0.00549805204574035\\
599.2	0.00545320682544357\\
599.21	0.00540792423726908\\
599.22	0.00536219998112139\\
599.23	0.00531602971462118\\
599.24	0.00526940905268952\\
599.25	0.00522233356712795\\
599.26	0.00517479878619452\\
599.27	0.00512680019417563\\
599.28	0.00507833323095367\\
599.29	0.0050293932915703\\
599.3	0.00497997572578562\\
599.31	0.00493007583763287\\
599.32	0.0048796888849688\\
599.33	0.00482881007901966\\
599.34	0.00477743458392256\\
599.35	0.00472555751626257\\
599.36	0.00467317394460506\\
599.37	0.00462027888902354\\
599.38	0.00456686732062281\\
599.39	0.00451293416105751\\
599.4	0.00445847428204581\\
599.41	0.00440348251095325\\
599.42	0.00434795363554175\\
599.43	0.0042918823924911\\
599.44	0.00423526346689832\\
599.45	0.00417809149177217\\
599.46	0.00412036104752262\\
599.47	0.00406206666144545\\
599.48	0.00400320280720161\\
599.49	0.00394376390429165\\
599.5	0.00388374431752494\\
599.51	0.00382313835648361\\
599.52	0.00376194027498138\\
599.53	0.00370014427051699\\
599.54	0.0036377444837223\\
599.55	0.00357473499780495\\
599.56	0.00351110983798564\\
599.57	0.00344686297092978\\
599.58	0.00338198830417369\\
599.59	0.00331647968554506\\
599.6	0.00325033090257786\\
599.61	0.00318353568192136\\
599.62	0.00311608768874353\\
599.63	0.00304798052612843\\
599.64	0.00297920773446783\\
599.65	0.00290976279084677\\
599.66	0.00283963910842317\\
599.67	0.00276883003580129\\
599.68	0.00269732885639912\\
599.69	0.00262512878780951\\
599.7	0.0025522229811551\\
599.71	0.00247860452043686\\
599.72	0.00240426642187629\\
599.73	0.0023292016332512\\
599.74	0.00225340303322488\\
599.75	0.00217686343066881\\
599.76	0.00209957556397867\\
599.77	0.00202153210038366\\
599.78	0.00194272563524904\\
599.79	0.00186314869137185\\
599.8	0.00178279371826975\\
599.81	0.00170165309146282\\
599.82	0.0016197191117484\\
599.83	0.00153698400446878\\
599.84	0.00145343991877171\\
599.85	0.00136907892686371\\
599.86	0.00128389302325598\\
599.87	0.001197874124003\\
599.88	0.00111101406593362\\
599.89	0.00102330460587468\\
599.9	0.0009347374198669\\
599.91	0.000845304102373186\\
599.92	0.000754996165479166\\
599.93	0.000663805038085868\\
599.94	0.000571722065094475\\
599.95	0.000478738506583127\\
599.96	0.000384845536975638\\
599.97	0.000290034244202044\\
599.98	0.00019429562885097\\
599.99	9.76206033136556e-05\\
600	0\\
};
\addplot [color=mycolor8,solid,forget plot]
  table[row sep=crcr]{%
0.01	0.00999999999999999\\
1.01	0.00999999999999999\\
2.01	0.00999999999999999\\
3.01	0.00999999999999999\\
4.01	0.00999999999999999\\
5.01	0.00999999999999999\\
6.01	0.00999999999999999\\
7.01	0.00999999999999999\\
8.01	0.00999999999999999\\
9.01	0.00999999999999999\\
10.01	0.00999999999999999\\
11.01	0.00999999999999999\\
12.01	0.00999999999999999\\
13.01	0.00999999999999999\\
14.01	0.00999999999999999\\
15.01	0.00999999999999999\\
16.01	0.00999999999999999\\
17.01	0.00999999999999999\\
18.01	0.00999999999999999\\
19.01	0.00999999999999999\\
20.01	0.00999999999999999\\
21.01	0.00999999999999999\\
22.01	0.00999999999999999\\
23.01	0.00999999999999999\\
24.01	0.00999999999999999\\
25.01	0.00999999999999999\\
26.01	0.00999999999999999\\
27.01	0.00999999999999999\\
28.01	0.00999999999999999\\
29.01	0.00999999999999999\\
30.01	0.00999999999999999\\
31.01	0.00999999999999999\\
32.01	0.00999999999999999\\
33.01	0.00999999999999999\\
34.01	0.00999999999999999\\
35.01	0.00999999999999999\\
36.01	0.00999999999999999\\
37.01	0.00999999999999999\\
38.01	0.00999999999999999\\
39.01	0.00999999999999999\\
40.01	0.00999999999999999\\
41.01	0.00999999999999999\\
42.01	0.00999999999999999\\
43.01	0.00999999999999999\\
44.01	0.00999999999999999\\
45.01	0.00999999999999999\\
46.01	0.00999999999999999\\
47.01	0.00999999999999999\\
48.01	0.00999999999999999\\
49.01	0.00999999999999999\\
50.01	0.00999999999999999\\
51.01	0.00999999999999999\\
52.01	0.00999999999999999\\
53.01	0.00999999999999999\\
54.01	0.00999999999999999\\
55.01	0.00999999999999999\\
56.01	0.00999999999999999\\
57.01	0.00999999999999999\\
58.01	0.00999999999999999\\
59.01	0.00999999999999999\\
60.01	0.00999999999999999\\
61.01	0.00999999999999999\\
62.01	0.00999999999999999\\
63.01	0.00999999999999999\\
64.01	0.00999999999999999\\
65.01	0.00999999999999999\\
66.01	0.00999999999999999\\
67.01	0.00999999999999999\\
68.01	0.00999999999999999\\
69.01	0.00999999999999999\\
70.01	0.00999999999999999\\
71.01	0.00999999999999999\\
72.01	0.00999999999999999\\
73.01	0.00999999999999999\\
74.01	0.00999999999999999\\
75.01	0.00999999999999999\\
76.01	0.00999999999999999\\
77.01	0.00999999999999999\\
78.01	0.00999999999999999\\
79.01	0.00999999999999999\\
80.01	0.00999999999999999\\
81.01	0.00999999999999999\\
82.01	0.00999999999999999\\
83.01	0.00999999999999999\\
84.01	0.00999999999999999\\
85.01	0.00999999999999999\\
86.01	0.00999999999999999\\
87.01	0.00999999999999999\\
88.01	0.00999999999999999\\
89.01	0.00999999999999999\\
90.01	0.00999999999999999\\
91.01	0.00999999999999999\\
92.01	0.00999999999999999\\
93.01	0.00999999999999999\\
94.01	0.00999999999999999\\
95.01	0.00999999999999999\\
96.01	0.00999999999999999\\
97.01	0.00999999999999999\\
98.01	0.00999999999999999\\
99.01	0.00999999999999999\\
100.01	0.00999999999999999\\
101.01	0.00999999999999999\\
102.01	0.00999999999999999\\
103.01	0.00999999999999999\\
104.01	0.00999999999999999\\
105.01	0.00999999999999999\\
106.01	0.00999999999999999\\
107.01	0.00999999999999999\\
108.01	0.00999999999999999\\
109.01	0.00999999999999999\\
110.01	0.00999999999999999\\
111.01	0.00999999999999999\\
112.01	0.00999999999999999\\
113.01	0.00999999999999999\\
114.01	0.00999999999999999\\
115.01	0.00999999999999999\\
116.01	0.00999999999999999\\
117.01	0.00999999999999999\\
118.01	0.00999999999999999\\
119.01	0.00999999999999999\\
120.01	0.00999999999999999\\
121.01	0.00999999999999999\\
122.01	0.00999999999999999\\
123.01	0.00999999999999999\\
124.01	0.00999999999999999\\
125.01	0.00999999999999999\\
126.01	0.00999999999999999\\
127.01	0.00999999999999999\\
128.01	0.00999999999999999\\
129.01	0.00999999999999999\\
130.01	0.00999999999999999\\
131.01	0.00999999999999999\\
132.01	0.00999999999999999\\
133.01	0.00999999999999999\\
134.01	0.00999999999999999\\
135.01	0.00999999999999999\\
136.01	0.00999999999999999\\
137.01	0.00999999999999999\\
138.01	0.00999999999999999\\
139.01	0.00999999999999999\\
140.01	0.00999999999999999\\
141.01	0.00999999999999999\\
142.01	0.00999999999999999\\
143.01	0.00999999999999999\\
144.01	0.00999999999999999\\
145.01	0.00999999999999999\\
146.01	0.00999999999999999\\
147.01	0.00999999999999999\\
148.01	0.00999999999999999\\
149.01	0.00999999999999999\\
150.01	0.00999999999999999\\
151.01	0.00999999999999999\\
152.01	0.00999999999999999\\
153.01	0.00999999999999999\\
154.01	0.00999999999999999\\
155.01	0.00999999999999999\\
156.01	0.00999999999999999\\
157.01	0.00999999999999999\\
158.01	0.00999999999999999\\
159.01	0.00999999999999999\\
160.01	0.00999999999999999\\
161.01	0.00999999999999999\\
162.01	0.00999999999999999\\
163.01	0.00999999999999999\\
164.01	0.00999999999999999\\
165.01	0.00999999999999999\\
166.01	0.00999999999999999\\
167.01	0.00999999999999999\\
168.01	0.00999999999999999\\
169.01	0.00999999999999999\\
170.01	0.00999999999999999\\
171.01	0.00999999999999999\\
172.01	0.00999999999999999\\
173.01	0.00999999999999999\\
174.01	0.00999999999999999\\
175.01	0.00999999999999999\\
176.01	0.00999999999999999\\
177.01	0.00999999999999999\\
178.01	0.00999999999999999\\
179.01	0.00999999999999999\\
180.01	0.00999999999999999\\
181.01	0.00999999999999999\\
182.01	0.00999999999999999\\
183.01	0.00999999999999999\\
184.01	0.00999999999999999\\
185.01	0.00999999999999999\\
186.01	0.00999999999999999\\
187.01	0.00999999999999999\\
188.01	0.00999999999999999\\
189.01	0.00999999999999999\\
190.01	0.00999999999999999\\
191.01	0.00999999999999999\\
192.01	0.00999999999999999\\
193.01	0.00999999999999999\\
194.01	0.00999999999999999\\
195.01	0.00999999999999999\\
196.01	0.00999999999999999\\
197.01	0.00999999999999999\\
198.01	0.00999999999999999\\
199.01	0.00999999999999999\\
200.01	0.00999999999999999\\
201.01	0.00999999999999999\\
202.01	0.00999999999999999\\
203.01	0.00999999999999999\\
204.01	0.00999999999999999\\
205.01	0.00999999999999999\\
206.01	0.00999999999999999\\
207.01	0.00999999999999999\\
208.01	0.00999999999999999\\
209.01	0.00999999999999999\\
210.01	0.00999999999999999\\
211.01	0.00999999999999999\\
212.01	0.00999999999999999\\
213.01	0.00999999999999999\\
214.01	0.00999999999999999\\
215.01	0.00999999999999999\\
216.01	0.00999999999999999\\
217.01	0.00999999999999999\\
218.01	0.00999999999999999\\
219.01	0.00999999999999999\\
220.01	0.00999999999999999\\
221.01	0.00999999999999999\\
222.01	0.00999999999999999\\
223.01	0.00999999999999999\\
224.01	0.00999999999999999\\
225.01	0.00999999999999999\\
226.01	0.00999999999999999\\
227.01	0.00999999999999999\\
228.01	0.00999999999999999\\
229.01	0.00999999999999999\\
230.01	0.00999999999999999\\
231.01	0.00999999999999999\\
232.01	0.00999999999999999\\
233.01	0.00999999999999999\\
234.01	0.00999999999999999\\
235.01	0.00999999999999999\\
236.01	0.00999999999999999\\
237.01	0.00999999999999999\\
238.01	0.00999999999999999\\
239.01	0.00999999999999999\\
240.01	0.00999999999999999\\
241.01	0.00999999999999999\\
242.01	0.00999999999999999\\
243.01	0.00999999999999999\\
244.01	0.00999999999999999\\
245.01	0.00999999999999999\\
246.01	0.00999999999999999\\
247.01	0.00999999999999999\\
248.01	0.00999999999999999\\
249.01	0.00999999999999999\\
250.01	0.00999999999999999\\
251.01	0.00999999999999999\\
252.01	0.00999999999999999\\
253.01	0.00999999999999999\\
254.01	0.00999999999999999\\
255.01	0.00999999999999999\\
256.01	0.00999999999999999\\
257.01	0.00999999999999999\\
258.01	0.00999999999999999\\
259.01	0.00999999999999999\\
260.01	0.00999999999999999\\
261.01	0.00999999999999999\\
262.01	0.00999999999999999\\
263.01	0.00999999999999999\\
264.01	0.00999999999999999\\
265.01	0.00999999999999999\\
266.01	0.00999999999999999\\
267.01	0.00999999999999999\\
268.01	0.00999999999999999\\
269.01	0.00999999999999999\\
270.01	0.00999999999999999\\
271.01	0.00999999999999999\\
272.01	0.00999999999999999\\
273.01	0.00999999999999999\\
274.01	0.00999999999999999\\
275.01	0.00999999999999999\\
276.01	0.00999999999999999\\
277.01	0.00999999999999999\\
278.01	0.00999999999999999\\
279.01	0.00999999999999999\\
280.01	0.00999999999999999\\
281.01	0.00999999999999999\\
282.01	0.00999999999999999\\
283.01	0.00999999999999999\\
284.01	0.00999999999999999\\
285.01	0.00999999999999999\\
286.01	0.00999999999999999\\
287.01	0.00999999999999999\\
288.01	0.00999999999999999\\
289.01	0.00999999999999999\\
290.01	0.00999999999999999\\
291.01	0.00999999999999999\\
292.01	0.00999999999999999\\
293.01	0.00999999999999999\\
294.01	0.00999999999999999\\
295.01	0.00999999999999999\\
296.01	0.00999999999999999\\
297.01	0.00999999999999999\\
298.01	0.00999999999999999\\
299.01	0.00999999999999999\\
300.01	0.00999999999999999\\
301.01	0.00999999999999999\\
302.01	0.00999999999999999\\
303.01	0.00999999999999999\\
304.01	0.00999999999999999\\
305.01	0.00999999999999999\\
306.01	0.00999999999999999\\
307.01	0.00999999999999999\\
308.01	0.00999999999999999\\
309.01	0.00999999999999999\\
310.01	0.00999999999999999\\
311.01	0.00999999999999999\\
312.01	0.00999999999999999\\
313.01	0.00999999999999999\\
314.01	0.00999999999999999\\
315.01	0.00999999999999999\\
316.01	0.00999999999999999\\
317.01	0.00999999999999999\\
318.01	0.00999999999999999\\
319.01	0.00999999999999999\\
320.01	0.00999999999999999\\
321.01	0.00999999999999999\\
322.01	0.00999999999999999\\
323.01	0.00999999999999999\\
324.01	0.00999999999999999\\
325.01	0.00999999999999999\\
326.01	0.00999999999999999\\
327.01	0.00999999999999999\\
328.01	0.00999999999999999\\
329.01	0.00999999999999999\\
330.01	0.00999999999999999\\
331.01	0.00999999999999999\\
332.01	0.00999999999999999\\
333.01	0.00999999999999999\\
334.01	0.00999999999999999\\
335.01	0.00999999999999999\\
336.01	0.00999999999999999\\
337.01	0.00999999999999999\\
338.01	0.00999999999999999\\
339.01	0.00999999999999999\\
340.01	0.00999999999999999\\
341.01	0.00999999999999999\\
342.01	0.00999999999999999\\
343.01	0.00999999999999999\\
344.01	0.00999999999999999\\
345.01	0.00999999999999999\\
346.01	0.00999999999999999\\
347.01	0.00999999999999999\\
348.01	0.00999999999999999\\
349.01	0.00999999999999999\\
350.01	0.00999999999999999\\
351.01	0.00999999999999999\\
352.01	0.00999999999999999\\
353.01	0.00999999999999999\\
354.01	0.00999999999999999\\
355.01	0.00999999999999999\\
356.01	0.00999999999999999\\
357.01	0.00999999999999999\\
358.01	0.00999999999999999\\
359.01	0.00999999999999999\\
360.01	0.00999999999999999\\
361.01	0.00999999999999999\\
362.01	0.00999999999999999\\
363.01	0.00999999999999999\\
364.01	0.00999999999999999\\
365.01	0.00999999999999999\\
366.01	0.00999999999999999\\
367.01	0.00999999999999999\\
368.01	0.00999999999999999\\
369.01	0.00999999999999999\\
370.01	0.00999999999999999\\
371.01	0.00999999999999999\\
372.01	0.00999999999999999\\
373.01	0.00999999999999999\\
374.01	0.00999999999999999\\
375.01	0.00999999999999999\\
376.01	0.00999999999999999\\
377.01	0.00999999999999999\\
378.01	0.00999999999999999\\
379.01	0.00999999999999999\\
380.01	0.00999999999999999\\
381.01	0.00999999999999999\\
382.01	0.00999999999999999\\
383.01	0.00999999999999999\\
384.01	0.00999999999999999\\
385.01	0.00999999999999999\\
386.01	0.00999999999999999\\
387.01	0.00999999999999999\\
388.01	0.00999999999999999\\
389.01	0.00999999999999999\\
390.01	0.00999999999999999\\
391.01	0.00999999999999999\\
392.01	0.00999999999999999\\
393.01	0.00999999999999999\\
394.01	0.00999999999999999\\
395.01	0.00999999999999999\\
396.01	0.00999999999999999\\
397.01	0.00999999999999999\\
398.01	0.00999999999999999\\
399.01	0.00999999999999999\\
400.01	0.00999999999999999\\
401.01	0.00999999999999999\\
402.01	0.00999999999999999\\
403.01	0.00999999999999999\\
404.01	0.00999999999999999\\
405.01	0.00999999999999999\\
406.01	0.00999999999999999\\
407.01	0.00999999999999999\\
408.01	0.00999999999999999\\
409.01	0.00999999999999999\\
410.01	0.00999999999999999\\
411.01	0.00999999999999999\\
412.01	0.00999999999999999\\
413.01	0.00999999999999999\\
414.01	0.00999999999999999\\
415.01	0.00999999999999999\\
416.01	0.00999999999999999\\
417.01	0.00999999999999999\\
418.01	0.00999999999999999\\
419.01	0.00999999999999999\\
420.01	0.00999999999999999\\
421.01	0.00999999999999999\\
422.01	0.00999999999999999\\
423.01	0.00999999999999999\\
424.01	0.00999999999999999\\
425.01	0.00999999999999999\\
426.01	0.00999999999999999\\
427.01	0.00999999999999999\\
428.01	0.00999999999999999\\
429.01	0.00999999999999999\\
430.01	0.00999999999999999\\
431.01	0.00999999999999999\\
432.01	0.00999999999999999\\
433.01	0.00999999999999999\\
434.01	0.00999999999999999\\
435.01	0.00999999999999999\\
436.01	0.00999999999999999\\
437.01	0.00999999999999999\\
438.01	0.00999999999999999\\
439.01	0.00999999999999999\\
440.01	0.00999999999999999\\
441.01	0.00999999999999999\\
442.01	0.00999999999999999\\
443.01	0.00999999999999999\\
444.01	0.00999999999999999\\
445.01	0.00999999999999999\\
446.01	0.00999999999999999\\
447.01	0.00999999999999999\\
448.01	0.00999999999999999\\
449.01	0.00999999999999999\\
450.01	0.00999999999999999\\
451.01	0.00999999999999999\\
452.01	0.00999999999999999\\
453.01	0.00999999999999999\\
454.01	0.00999999999999999\\
455.01	0.00999999999999999\\
456.01	0.00999999999999999\\
457.01	0.00999999999999999\\
458.01	0.00999999999999999\\
459.01	0.00999999999999999\\
460.01	0.00999999999999999\\
461.01	0.00999999999999999\\
462.01	0.00999999999999999\\
463.01	0.00999999999999999\\
464.01	0.00999999999999999\\
465.01	0.00999999999999999\\
466.01	0.00999999999999999\\
467.01	0.00999999999999999\\
468.01	0.00999999999999999\\
469.01	0.00999999999999999\\
470.01	0.00999999999999999\\
471.01	0.00999999999999999\\
472.01	0.00999999999999999\\
473.01	0.00999999999999999\\
474.01	0.00999999999999999\\
475.01	0.00999999999999999\\
476.01	0.00999999999999999\\
477.01	0.00999999999999999\\
478.01	0.00999999999999999\\
479.01	0.00999999999999999\\
480.01	0.00999999999999999\\
481.01	0.00999999999999999\\
482.01	0.00999999999999999\\
483.01	0.00999999999999999\\
484.01	0.00999999999999999\\
485.01	0.00999999999999999\\
486.01	0.00999999999999999\\
487.01	0.00999999999999999\\
488.01	0.00999999999999999\\
489.01	0.00999999999999999\\
490.01	0.00999999999999999\\
491.01	0.00999999999999999\\
492.01	0.00999999999999999\\
493.01	0.00999999999999999\\
494.01	0.00999999999999999\\
495.01	0.00999999999999999\\
496.01	0.00999999999999999\\
497.01	0.00999999999999999\\
498.01	0.00999999999999999\\
499.01	0.00999999999999999\\
500.01	0.00999999999999999\\
501.01	0.00999999999999999\\
502.01	0.00999999999999999\\
503.01	0.00999999999999999\\
504.01	0.00999999999999999\\
505.01	0.00999999999999999\\
506.01	0.00999999999999999\\
507.01	0.00999999999999999\\
508.01	0.00999999999999999\\
509.01	0.00999999999999999\\
510.01	0.00999999999999999\\
511.01	0.00999999999999999\\
512.01	0.00999999999999999\\
513.01	0.00999999999999999\\
514.01	0.00999999999999999\\
515.01	0.00999999999999999\\
516.01	0.00999999999999999\\
517.01	0.00999999999999999\\
518.01	0.00999999999999999\\
519.01	0.00999999999999999\\
520.01	0.00999999999999999\\
521.01	0.00999999999999999\\
522.01	0.00999999999999999\\
523.01	0.00999999999999999\\
524.01	0.00999999999999999\\
525.01	0.00999999999999999\\
526.01	0.00999999999999999\\
527.01	0.00999999999999999\\
528.01	0.00999999999999999\\
529.01	0.00999999999999999\\
530.01	0.00999999999999999\\
531.01	0.00999999999999999\\
532.01	0.00999999999999999\\
533.01	0.00999999999999999\\
534.01	0.00999999999999999\\
535.01	0.00999999999999999\\
536.01	0.00999999999999999\\
537.01	0.00999999999999999\\
538.01	0.00999999999999999\\
539.01	0.00999999999999999\\
540.01	0.00999999999999999\\
541.01	0.00999999999999999\\
542.01	0.00999999999999999\\
543.01	0.00999999999999999\\
544.01	0.00999999999999999\\
545.01	0.00999999999999999\\
546.01	0.00999999999999999\\
547.01	0.00999999999999999\\
548.01	0.00999999999999999\\
549.01	0.00999999999999999\\
550.01	0.00999999999999999\\
551.01	0.00999999999999999\\
552.01	0.00999999999999999\\
553.01	0.00999999999999999\\
554.01	0.00999999999999999\\
555.01	0.00999999999999999\\
556.01	0.00999999999999999\\
557.01	0.00999999999999999\\
558.01	0.00999999999999999\\
559.01	0.00999999999999999\\
560.01	0.00999999999999999\\
561.01	0.00999999999999999\\
562.01	0.00999999999999999\\
563.01	0.00999999999999999\\
564.01	0.00999999999999999\\
565.01	0.00999999999999999\\
566.01	0.00999999999999999\\
567.01	0.00999999999999999\\
568.01	0.00999999999999999\\
569.01	0.00999999999999999\\
570.01	0.00999999999999999\\
571.01	0.00999999999999999\\
572.01	0.00999999999999999\\
573.01	0.00999999999999999\\
574.01	0.00999999999999999\\
575.01	0.00999999999999999\\
576.01	0.00999999999999999\\
577.01	0.00999999999999999\\
578.01	0.00999999999999999\\
579.01	0.00999999999999999\\
580.01	0.00999999999999999\\
581.01	0.00999999999999999\\
582.01	0.00999999999999999\\
583.01	0.00999999999999999\\
584.01	0.00999999999999999\\
585.01	0.00999999999999999\\
586.01	0.00999999999999999\\
587.01	0.00999999999999999\\
588.01	0.00999999999999999\\
589.01	0.00999999999999999\\
590.01	0.00999999999999999\\
591.01	0.00999999999999999\\
592.01	0.00999999999999999\\
593.01	0.00999999999999999\\
594.01	0.00999999999999999\\
595.01	0.00999999999999999\\
596.01	0.00999999999999999\\
597.01	0.00999999999999999\\
598.01	0.00999999999999999\\
599.01	0.00623513569400518\\
599.02	0.00619747359804889\\
599.03	0.00615944475593818\\
599.04	0.00612104556201831\\
599.05	0.00608227237517916\\
599.06	0.00604312151850672\\
599.07	0.0060035892789309\\
599.08	0.00596367190687008\\
599.09	0.00592336561587204\\
599.1	0.00588266658225147\\
599.11	0.00584157094472378\\
599.12	0.00580007480403543\\
599.13	0.00575817422259056\\
599.14	0.00571586522407395\\
599.15	0.0056731437930703\\
599.16	0.0056300058746798\\
599.17	0.0055864473741298\\
599.18	0.00554246415638276\\
599.19	0.00549805204574034\\
599.2	0.00545320682544356\\
599.21	0.00540792423726907\\
599.22	0.00536219998112139\\
599.23	0.00531602971462118\\
599.24	0.00526940905268951\\
599.25	0.00522233356712793\\
599.26	0.00517479878619451\\
599.27	0.00512680019417564\\
599.28	0.00507833323095367\\
599.29	0.00502939329157029\\
599.3	0.0049799757257856\\
599.31	0.00493007583763286\\
599.32	0.00487968888496881\\
599.33	0.00482881007901966\\
599.34	0.00477743458392257\\
599.35	0.00472555751626257\\
599.36	0.00467317394460505\\
599.37	0.00462027888902352\\
599.38	0.0045668673206228\\
599.39	0.00451293416105749\\
599.4	0.0044584742820458\\
599.41	0.00440348251095323\\
599.42	0.00434795363554173\\
599.43	0.00429188239249109\\
599.44	0.00423526346689832\\
599.45	0.00417809149177216\\
599.46	0.00412036104752264\\
599.47	0.00406206666144547\\
599.48	0.00400320280720162\\
599.49	0.00394376390429167\\
599.5	0.00388374431752494\\
599.51	0.00382313835648361\\
599.52	0.00376194027498138\\
599.53	0.00370014427051699\\
599.54	0.0036377444837223\\
599.55	0.00357473499780495\\
599.56	0.00351110983798564\\
599.57	0.00344686297092978\\
599.58	0.00338198830417369\\
599.59	0.00331647968554506\\
599.6	0.00325033090257786\\
599.61	0.00318353568192136\\
599.62	0.00311608768874352\\
599.63	0.00304798052612842\\
599.64	0.00297920773446782\\
599.65	0.00290976279084676\\
599.66	0.00283963910842317\\
599.67	0.00276883003580129\\
599.68	0.00269732885639912\\
599.69	0.00262512878780951\\
599.7	0.0025522229811551\\
599.71	0.00247860452043686\\
599.72	0.00240426642187629\\
599.73	0.0023292016332512\\
599.74	0.00225340303322488\\
599.75	0.00217686343066881\\
599.76	0.00209957556397867\\
599.77	0.00202153210038366\\
599.78	0.00194272563524905\\
599.79	0.00186314869137186\\
599.8	0.00178279371826976\\
599.81	0.00170165309146283\\
599.82	0.00161971911174841\\
599.83	0.00153698400446879\\
599.84	0.00145343991877172\\
599.85	0.00136907892686372\\
599.86	0.00128389302325598\\
599.87	0.00119787412400299\\
599.88	0.00111101406593363\\
599.89	0.00102330460587468\\
599.9	0.000934737419866901\\
599.91	0.000845304102373188\\
599.92	0.000754996165479168\\
599.93	0.000663805038085868\\
599.94	0.000571722065094473\\
599.95	0.000478738506583127\\
599.96	0.000384845536975638\\
599.97	0.000290034244202044\\
599.98	0.00019429562885097\\
599.99	9.76206033136574e-05\\
600	0\\
};
\addplot [color=blue!25!mycolor7,solid,forget plot]
  table[row sep=crcr]{%
0.01	0.00999999999999999\\
1.01	0.00999999999999999\\
2.01	0.00999999999999999\\
3.01	0.00999999999999999\\
4.01	0.00999999999999999\\
5.01	0.00999999999999999\\
6.01	0.00999999999999999\\
7.01	0.00999999999999999\\
8.01	0.00999999999999999\\
9.01	0.00999999999999999\\
10.01	0.00999999999999999\\
11.01	0.00999999999999999\\
12.01	0.00999999999999999\\
13.01	0.00999999999999999\\
14.01	0.00999999999999999\\
15.01	0.00999999999999999\\
16.01	0.00999999999999999\\
17.01	0.00999999999999999\\
18.01	0.00999999999999999\\
19.01	0.00999999999999999\\
20.01	0.00999999999999999\\
21.01	0.00999999999999999\\
22.01	0.00999999999999999\\
23.01	0.00999999999999999\\
24.01	0.00999999999999999\\
25.01	0.00999999999999999\\
26.01	0.00999999999999999\\
27.01	0.00999999999999999\\
28.01	0.00999999999999999\\
29.01	0.00999999999999999\\
30.01	0.00999999999999999\\
31.01	0.00999999999999999\\
32.01	0.00999999999999999\\
33.01	0.00999999999999999\\
34.01	0.00999999999999999\\
35.01	0.00999999999999999\\
36.01	0.00999999999999999\\
37.01	0.00999999999999999\\
38.01	0.00999999999999999\\
39.01	0.00999999999999999\\
40.01	0.00999999999999999\\
41.01	0.00999999999999999\\
42.01	0.00999999999999999\\
43.01	0.00999999999999999\\
44.01	0.00999999999999999\\
45.01	0.00999999999999999\\
46.01	0.00999999999999999\\
47.01	0.00999999999999999\\
48.01	0.00999999999999999\\
49.01	0.00999999999999999\\
50.01	0.00999999999999999\\
51.01	0.00999999999999999\\
52.01	0.00999999999999999\\
53.01	0.00999999999999999\\
54.01	0.00999999999999999\\
55.01	0.00999999999999999\\
56.01	0.00999999999999999\\
57.01	0.00999999999999999\\
58.01	0.00999999999999999\\
59.01	0.00999999999999999\\
60.01	0.00999999999999999\\
61.01	0.00999999999999999\\
62.01	0.00999999999999999\\
63.01	0.00999999999999999\\
64.01	0.00999999999999999\\
65.01	0.00999999999999999\\
66.01	0.00999999999999999\\
67.01	0.00999999999999999\\
68.01	0.00999999999999999\\
69.01	0.00999999999999999\\
70.01	0.00999999999999999\\
71.01	0.00999999999999999\\
72.01	0.00999999999999999\\
73.01	0.00999999999999999\\
74.01	0.00999999999999999\\
75.01	0.00999999999999999\\
76.01	0.00999999999999999\\
77.01	0.00999999999999999\\
78.01	0.00999999999999999\\
79.01	0.00999999999999999\\
80.01	0.00999999999999999\\
81.01	0.00999999999999999\\
82.01	0.00999999999999999\\
83.01	0.00999999999999999\\
84.01	0.00999999999999999\\
85.01	0.00999999999999999\\
86.01	0.00999999999999999\\
87.01	0.00999999999999999\\
88.01	0.00999999999999999\\
89.01	0.00999999999999999\\
90.01	0.00999999999999999\\
91.01	0.00999999999999999\\
92.01	0.00999999999999999\\
93.01	0.00999999999999999\\
94.01	0.00999999999999999\\
95.01	0.00999999999999999\\
96.01	0.00999999999999999\\
97.01	0.00999999999999999\\
98.01	0.00999999999999999\\
99.01	0.00999999999999999\\
100.01	0.00999999999999999\\
101.01	0.00999999999999999\\
102.01	0.00999999999999999\\
103.01	0.00999999999999999\\
104.01	0.00999999999999999\\
105.01	0.00999999999999999\\
106.01	0.00999999999999999\\
107.01	0.00999999999999999\\
108.01	0.00999999999999999\\
109.01	0.00999999999999999\\
110.01	0.00999999999999999\\
111.01	0.00999999999999999\\
112.01	0.00999999999999999\\
113.01	0.00999999999999999\\
114.01	0.00999999999999999\\
115.01	0.00999999999999999\\
116.01	0.00999999999999999\\
117.01	0.00999999999999999\\
118.01	0.00999999999999999\\
119.01	0.00999999999999999\\
120.01	0.00999999999999999\\
121.01	0.00999999999999999\\
122.01	0.00999999999999999\\
123.01	0.00999999999999999\\
124.01	0.00999999999999999\\
125.01	0.00999999999999999\\
126.01	0.00999999999999999\\
127.01	0.00999999999999999\\
128.01	0.00999999999999999\\
129.01	0.00999999999999999\\
130.01	0.00999999999999999\\
131.01	0.00999999999999999\\
132.01	0.00999999999999999\\
133.01	0.00999999999999999\\
134.01	0.00999999999999999\\
135.01	0.00999999999999999\\
136.01	0.00999999999999999\\
137.01	0.00999999999999999\\
138.01	0.00999999999999999\\
139.01	0.00999999999999999\\
140.01	0.00999999999999999\\
141.01	0.00999999999999999\\
142.01	0.00999999999999999\\
143.01	0.00999999999999999\\
144.01	0.00999999999999999\\
145.01	0.00999999999999999\\
146.01	0.00999999999999999\\
147.01	0.00999999999999999\\
148.01	0.00999999999999999\\
149.01	0.00999999999999999\\
150.01	0.00999999999999999\\
151.01	0.00999999999999999\\
152.01	0.00999999999999999\\
153.01	0.00999999999999999\\
154.01	0.00999999999999999\\
155.01	0.00999999999999999\\
156.01	0.00999999999999999\\
157.01	0.00999999999999999\\
158.01	0.00999999999999999\\
159.01	0.00999999999999999\\
160.01	0.00999999999999999\\
161.01	0.00999999999999999\\
162.01	0.00999999999999999\\
163.01	0.00999999999999999\\
164.01	0.00999999999999999\\
165.01	0.00999999999999999\\
166.01	0.00999999999999999\\
167.01	0.00999999999999999\\
168.01	0.00999999999999999\\
169.01	0.00999999999999999\\
170.01	0.00999999999999999\\
171.01	0.00999999999999999\\
172.01	0.00999999999999999\\
173.01	0.00999999999999999\\
174.01	0.00999999999999999\\
175.01	0.00999999999999999\\
176.01	0.00999999999999999\\
177.01	0.00999999999999999\\
178.01	0.00999999999999999\\
179.01	0.00999999999999999\\
180.01	0.00999999999999999\\
181.01	0.00999999999999999\\
182.01	0.00999999999999999\\
183.01	0.00999999999999999\\
184.01	0.00999999999999999\\
185.01	0.00999999999999999\\
186.01	0.00999999999999999\\
187.01	0.00999999999999999\\
188.01	0.00999999999999999\\
189.01	0.00999999999999999\\
190.01	0.00999999999999999\\
191.01	0.00999999999999999\\
192.01	0.00999999999999999\\
193.01	0.00999999999999999\\
194.01	0.00999999999999999\\
195.01	0.00999999999999999\\
196.01	0.00999999999999999\\
197.01	0.00999999999999999\\
198.01	0.00999999999999999\\
199.01	0.00999999999999999\\
200.01	0.00999999999999999\\
201.01	0.00999999999999999\\
202.01	0.00999999999999999\\
203.01	0.00999999999999999\\
204.01	0.00999999999999999\\
205.01	0.00999999999999999\\
206.01	0.00999999999999999\\
207.01	0.00999999999999999\\
208.01	0.00999999999999999\\
209.01	0.00999999999999999\\
210.01	0.00999999999999999\\
211.01	0.00999999999999999\\
212.01	0.00999999999999999\\
213.01	0.00999999999999999\\
214.01	0.00999999999999999\\
215.01	0.00999999999999999\\
216.01	0.00999999999999999\\
217.01	0.00999999999999999\\
218.01	0.00999999999999999\\
219.01	0.00999999999999999\\
220.01	0.00999999999999999\\
221.01	0.00999999999999999\\
222.01	0.00999999999999999\\
223.01	0.00999999999999999\\
224.01	0.00999999999999999\\
225.01	0.00999999999999999\\
226.01	0.00999999999999999\\
227.01	0.00999999999999999\\
228.01	0.00999999999999999\\
229.01	0.00999999999999999\\
230.01	0.00999999999999999\\
231.01	0.00999999999999999\\
232.01	0.00999999999999999\\
233.01	0.00999999999999999\\
234.01	0.00999999999999999\\
235.01	0.00999999999999999\\
236.01	0.00999999999999999\\
237.01	0.00999999999999999\\
238.01	0.00999999999999999\\
239.01	0.00999999999999999\\
240.01	0.00999999999999999\\
241.01	0.00999999999999999\\
242.01	0.00999999999999999\\
243.01	0.00999999999999999\\
244.01	0.00999999999999999\\
245.01	0.00999999999999999\\
246.01	0.00999999999999999\\
247.01	0.00999999999999999\\
248.01	0.00999999999999999\\
249.01	0.00999999999999999\\
250.01	0.00999999999999999\\
251.01	0.00999999999999999\\
252.01	0.00999999999999999\\
253.01	0.00999999999999999\\
254.01	0.00999999999999999\\
255.01	0.00999999999999999\\
256.01	0.00999999999999999\\
257.01	0.00999999999999999\\
258.01	0.00999999999999999\\
259.01	0.00999999999999999\\
260.01	0.00999999999999999\\
261.01	0.00999999999999999\\
262.01	0.00999999999999999\\
263.01	0.00999999999999999\\
264.01	0.00999999999999999\\
265.01	0.00999999999999999\\
266.01	0.00999999999999999\\
267.01	0.00999999999999999\\
268.01	0.00999999999999999\\
269.01	0.00999999999999999\\
270.01	0.00999999999999999\\
271.01	0.00999999999999999\\
272.01	0.00999999999999999\\
273.01	0.00999999999999999\\
274.01	0.00999999999999999\\
275.01	0.00999999999999999\\
276.01	0.00999999999999999\\
277.01	0.00999999999999999\\
278.01	0.00999999999999999\\
279.01	0.00999999999999999\\
280.01	0.00999999999999999\\
281.01	0.00999999999999999\\
282.01	0.00999999999999999\\
283.01	0.00999999999999999\\
284.01	0.00999999999999999\\
285.01	0.00999999999999999\\
286.01	0.00999999999999999\\
287.01	0.00999999999999999\\
288.01	0.00999999999999999\\
289.01	0.00999999999999999\\
290.01	0.00999999999999999\\
291.01	0.00999999999999999\\
292.01	0.00999999999999999\\
293.01	0.00999999999999999\\
294.01	0.00999999999999999\\
295.01	0.00999999999999999\\
296.01	0.00999999999999999\\
297.01	0.00999999999999999\\
298.01	0.00999999999999999\\
299.01	0.00999999999999999\\
300.01	0.00999999999999999\\
301.01	0.00999999999999999\\
302.01	0.00999999999999999\\
303.01	0.00999999999999999\\
304.01	0.00999999999999999\\
305.01	0.00999999999999999\\
306.01	0.00999999999999999\\
307.01	0.00999999999999999\\
308.01	0.00999999999999999\\
309.01	0.00999999999999999\\
310.01	0.00999999999999999\\
311.01	0.00999999999999999\\
312.01	0.00999999999999999\\
313.01	0.00999999999999999\\
314.01	0.00999999999999999\\
315.01	0.00999999999999999\\
316.01	0.00999999999999999\\
317.01	0.00999999999999999\\
318.01	0.00999999999999999\\
319.01	0.00999999999999999\\
320.01	0.00999999999999999\\
321.01	0.00999999999999999\\
322.01	0.00999999999999999\\
323.01	0.00999999999999999\\
324.01	0.00999999999999999\\
325.01	0.00999999999999999\\
326.01	0.00999999999999999\\
327.01	0.00999999999999999\\
328.01	0.00999999999999999\\
329.01	0.00999999999999999\\
330.01	0.00999999999999999\\
331.01	0.00999999999999999\\
332.01	0.00999999999999999\\
333.01	0.00999999999999999\\
334.01	0.00999999999999999\\
335.01	0.00999999999999999\\
336.01	0.00999999999999999\\
337.01	0.00999999999999999\\
338.01	0.00999999999999999\\
339.01	0.00999999999999999\\
340.01	0.00999999999999999\\
341.01	0.00999999999999999\\
342.01	0.00999999999999999\\
343.01	0.00999999999999999\\
344.01	0.00999999999999999\\
345.01	0.00999999999999999\\
346.01	0.00999999999999999\\
347.01	0.00999999999999999\\
348.01	0.00999999999999999\\
349.01	0.00999999999999999\\
350.01	0.00999999999999999\\
351.01	0.00999999999999999\\
352.01	0.00999999999999999\\
353.01	0.00999999999999999\\
354.01	0.00999999999999999\\
355.01	0.00999999999999999\\
356.01	0.00999999999999999\\
357.01	0.00999999999999999\\
358.01	0.00999999999999999\\
359.01	0.00999999999999999\\
360.01	0.00999999999999999\\
361.01	0.00999999999999999\\
362.01	0.00999999999999999\\
363.01	0.00999999999999999\\
364.01	0.00999999999999999\\
365.01	0.00999999999999999\\
366.01	0.00999999999999999\\
367.01	0.00999999999999999\\
368.01	0.00999999999999999\\
369.01	0.00999999999999999\\
370.01	0.00999999999999999\\
371.01	0.00999999999999999\\
372.01	0.00999999999999999\\
373.01	0.00999999999999999\\
374.01	0.00999999999999999\\
375.01	0.00999999999999999\\
376.01	0.00999999999999999\\
377.01	0.00999999999999999\\
378.01	0.00999999999999999\\
379.01	0.00999999999999999\\
380.01	0.00999999999999999\\
381.01	0.00999999999999999\\
382.01	0.00999999999999999\\
383.01	0.00999999999999999\\
384.01	0.00999999999999999\\
385.01	0.00999999999999999\\
386.01	0.00999999999999999\\
387.01	0.00999999999999999\\
388.01	0.00999999999999999\\
389.01	0.00999999999999999\\
390.01	0.00999999999999999\\
391.01	0.00999999999999999\\
392.01	0.00999999999999999\\
393.01	0.00999999999999999\\
394.01	0.00999999999999999\\
395.01	0.00999999999999999\\
396.01	0.00999999999999999\\
397.01	0.00999999999999999\\
398.01	0.00999999999999999\\
399.01	0.00999999999999999\\
400.01	0.00999999999999999\\
401.01	0.00999999999999999\\
402.01	0.00999999999999999\\
403.01	0.00999999999999999\\
404.01	0.00999999999999999\\
405.01	0.00999999999999999\\
406.01	0.00999999999999999\\
407.01	0.00999999999999999\\
408.01	0.00999999999999999\\
409.01	0.00999999999999999\\
410.01	0.00999999999999999\\
411.01	0.00999999999999999\\
412.01	0.00999999999999999\\
413.01	0.00999999999999999\\
414.01	0.00999999999999999\\
415.01	0.00999999999999999\\
416.01	0.00999999999999999\\
417.01	0.00999999999999999\\
418.01	0.00999999999999999\\
419.01	0.00999999999999999\\
420.01	0.00999999999999999\\
421.01	0.00999999999999999\\
422.01	0.00999999999999999\\
423.01	0.00999999999999999\\
424.01	0.00999999999999999\\
425.01	0.00999999999999999\\
426.01	0.00999999999999999\\
427.01	0.00999999999999999\\
428.01	0.00999999999999999\\
429.01	0.00999999999999999\\
430.01	0.00999999999999999\\
431.01	0.00999999999999999\\
432.01	0.00999999999999999\\
433.01	0.00999999999999999\\
434.01	0.00999999999999999\\
435.01	0.00999999999999999\\
436.01	0.00999999999999999\\
437.01	0.00999999999999999\\
438.01	0.00999999999999999\\
439.01	0.00999999999999999\\
440.01	0.00999999999999999\\
441.01	0.00999999999999999\\
442.01	0.00999999999999999\\
443.01	0.00999999999999999\\
444.01	0.00999999999999999\\
445.01	0.00999999999999999\\
446.01	0.00999999999999999\\
447.01	0.00999999999999999\\
448.01	0.00999999999999999\\
449.01	0.00999999999999999\\
450.01	0.00999999999999999\\
451.01	0.00999999999999999\\
452.01	0.00999999999999999\\
453.01	0.00999999999999999\\
454.01	0.00999999999999999\\
455.01	0.00999999999999999\\
456.01	0.00999999999999999\\
457.01	0.00999999999999999\\
458.01	0.00999999999999999\\
459.01	0.00999999999999999\\
460.01	0.00999999999999999\\
461.01	0.00999999999999999\\
462.01	0.00999999999999999\\
463.01	0.00999999999999999\\
464.01	0.00999999999999999\\
465.01	0.00999999999999999\\
466.01	0.00999999999999999\\
467.01	0.00999999999999999\\
468.01	0.00999999999999999\\
469.01	0.00999999999999999\\
470.01	0.00999999999999999\\
471.01	0.00999999999999999\\
472.01	0.00999999999999999\\
473.01	0.00999999999999999\\
474.01	0.00999999999999999\\
475.01	0.00999999999999999\\
476.01	0.00999999999999999\\
477.01	0.00999999999999999\\
478.01	0.00999999999999999\\
479.01	0.00999999999999999\\
480.01	0.00999999999999999\\
481.01	0.00999999999999999\\
482.01	0.00999999999999999\\
483.01	0.00999999999999999\\
484.01	0.00999999999999999\\
485.01	0.00999999999999999\\
486.01	0.00999999999999999\\
487.01	0.00999999999999999\\
488.01	0.00999999999999999\\
489.01	0.00999999999999999\\
490.01	0.00999999999999999\\
491.01	0.00999999999999999\\
492.01	0.00999999999999999\\
493.01	0.00999999999999999\\
494.01	0.00999999999999999\\
495.01	0.00999999999999999\\
496.01	0.00999999999999999\\
497.01	0.00999999999999999\\
498.01	0.00999999999999999\\
499.01	0.00999999999999999\\
500.01	0.00999999999999999\\
501.01	0.00999999999999999\\
502.01	0.00999999999999999\\
503.01	0.00999999999999999\\
504.01	0.00999999999999999\\
505.01	0.00999999999999999\\
506.01	0.00999999999999999\\
507.01	0.00999999999999999\\
508.01	0.00999999999999999\\
509.01	0.00999999999999999\\
510.01	0.00999999999999999\\
511.01	0.00999999999999999\\
512.01	0.00999999999999999\\
513.01	0.00999999999999999\\
514.01	0.00999999999999999\\
515.01	0.00999999999999999\\
516.01	0.00999999999999999\\
517.01	0.00999999999999999\\
518.01	0.00999999999999999\\
519.01	0.00999999999999999\\
520.01	0.00999999999999999\\
521.01	0.00999999999999999\\
522.01	0.00999999999999999\\
523.01	0.00999999999999999\\
524.01	0.00999999999999999\\
525.01	0.00999999999999999\\
526.01	0.00999999999999999\\
527.01	0.00999999999999999\\
528.01	0.00999999999999999\\
529.01	0.00999999999999999\\
530.01	0.00999999999999999\\
531.01	0.00999999999999999\\
532.01	0.00999999999999999\\
533.01	0.00999999999999999\\
534.01	0.00999999999999999\\
535.01	0.00999999999999999\\
536.01	0.00999999999999999\\
537.01	0.00999999999999999\\
538.01	0.00999999999999999\\
539.01	0.00999999999999999\\
540.01	0.00999999999999999\\
541.01	0.00999999999999999\\
542.01	0.00999999999999999\\
543.01	0.00999999999999999\\
544.01	0.00999999999999999\\
545.01	0.00999999999999999\\
546.01	0.00999999999999999\\
547.01	0.00999999999999999\\
548.01	0.00999999999999999\\
549.01	0.00999999999999999\\
550.01	0.00999999999999999\\
551.01	0.00999999999999999\\
552.01	0.00999999999999999\\
553.01	0.00999999999999999\\
554.01	0.00999999999999999\\
555.01	0.00999999999999999\\
556.01	0.00999999999999999\\
557.01	0.00999999999999999\\
558.01	0.00999999999999999\\
559.01	0.00999999999999999\\
560.01	0.00999999999999999\\
561.01	0.00999999999999999\\
562.01	0.00999999999999999\\
563.01	0.00999999999999999\\
564.01	0.00999999999999999\\
565.01	0.00999999999999999\\
566.01	0.00999999999999999\\
567.01	0.00999999999999999\\
568.01	0.00999999999999999\\
569.01	0.00999999999999999\\
570.01	0.00999999999999999\\
571.01	0.00999999999999999\\
572.01	0.00999999999999999\\
573.01	0.00999999999999999\\
574.01	0.00999999999999999\\
575.01	0.00999999999999999\\
576.01	0.00999999999999999\\
577.01	0.00999999999999999\\
578.01	0.00999999999999999\\
579.01	0.00999999999999999\\
580.01	0.00999999999999999\\
581.01	0.00999999999999999\\
582.01	0.00999999999999999\\
583.01	0.00999999999999999\\
584.01	0.00999999999999999\\
585.01	0.00999999999999999\\
586.01	0.00999999999999999\\
587.01	0.00999999999999999\\
588.01	0.00999999999999999\\
589.01	0.00999999999999999\\
590.01	0.00999999999999999\\
591.01	0.00999999999999999\\
592.01	0.00999999999999999\\
593.01	0.00999999999999999\\
594.01	0.00999999999999999\\
595.01	0.00999999999999999\\
596.01	0.00999999999999999\\
597.01	0.00999999999999999\\
598.01	0.00999999999999999\\
599.01	0.00623513569400616\\
599.02	0.00619747359804981\\
599.03	0.00615944475593906\\
599.04	0.00612104556201914\\
599.05	0.00608227237517996\\
599.06	0.00604312151850748\\
599.07	0.00600358927893161\\
599.08	0.00596367190687074\\
599.09	0.00592336561587267\\
599.1	0.00588266658225206\\
599.11	0.00584157094472434\\
599.12	0.00580007480403596\\
599.13	0.00575817422259104\\
599.14	0.00571586522407439\\
599.15	0.00567314379307072\\
599.16	0.00563000587468019\\
599.17	0.00558644737413016\\
599.18	0.00554246415638309\\
599.19	0.00549805204574064\\
599.2	0.00545320682544385\\
599.21	0.00540792423726933\\
599.22	0.00536219998112162\\
599.23	0.00531602971462141\\
599.24	0.00526940905268973\\
599.25	0.00522233356712813\\
599.26	0.00517479878619469\\
599.27	0.00512680019417579\\
599.28	0.0050783332309538\\
599.29	0.00502939329157042\\
599.3	0.00497997572578572\\
599.31	0.00493007583763297\\
599.32	0.0048796888849689\\
599.33	0.00482881007901974\\
599.34	0.00477743458392264\\
599.35	0.00472555751626265\\
599.36	0.00467317394460513\\
599.37	0.00462027888902359\\
599.38	0.00456686732062286\\
599.39	0.00451293416105756\\
599.4	0.00445847428204585\\
599.41	0.0044034825109533\\
599.42	0.0043479536355418\\
599.43	0.00429188239249114\\
599.44	0.00423526346689836\\
599.45	0.0041780914917722\\
599.46	0.00412036104752266\\
599.47	0.00406206666144549\\
599.48	0.00400320280720165\\
599.49	0.00394376390429169\\
599.5	0.00388374431752497\\
599.51	0.00382313835648363\\
599.52	0.0037619402749814\\
599.53	0.00370014427051701\\
599.54	0.00363774448372232\\
599.55	0.00357473499780498\\
599.56	0.00351110983798567\\
599.57	0.00344686297092981\\
599.58	0.00338198830417371\\
599.59	0.00331647968554508\\
599.6	0.00325033090257787\\
599.61	0.00318353568192137\\
599.62	0.00311608768874353\\
599.63	0.00304798052612843\\
599.64	0.00297920773446783\\
599.65	0.00290976279084677\\
599.66	0.00283963910842317\\
599.67	0.0027688300358013\\
599.68	0.00269732885639913\\
599.69	0.00262512878780952\\
599.7	0.00255222298115511\\
599.71	0.00247860452043687\\
599.72	0.00240426642187631\\
599.73	0.00232920163325121\\
599.74	0.00225340303322489\\
599.75	0.00217686343066882\\
599.76	0.00209957556397868\\
599.77	0.00202153210038367\\
599.78	0.00194272563524905\\
599.79	0.00186314869137186\\
599.8	0.00178279371826976\\
599.81	0.00170165309146283\\
599.82	0.00161971911174841\\
599.83	0.00153698400446879\\
599.84	0.00145343991877172\\
599.85	0.00136907892686371\\
599.86	0.00128389302325598\\
599.87	0.00119787412400299\\
599.88	0.00111101406593363\\
599.89	0.00102330460587468\\
599.9	0.000934737419866903\\
599.91	0.000845304102373186\\
599.92	0.000754996165479169\\
599.93	0.000663805038085868\\
599.94	0.000571722065094473\\
599.95	0.000478738506583129\\
599.96	0.000384845536975638\\
599.97	0.000290034244202044\\
599.98	0.00019429562885097\\
599.99	9.76206033136556e-05\\
600	0\\
};
\addplot [color=mycolor9,solid,forget plot]
  table[row sep=crcr]{%
0.01	0.00999999999999999\\
1.01	0.00999999999999999\\
2.01	0.00999999999999999\\
3.01	0.00999999999999999\\
4.01	0.00999999999999999\\
5.01	0.00999999999999999\\
6.01	0.00999999999999999\\
7.01	0.00999999999999999\\
8.01	0.00999999999999999\\
9.01	0.00999999999999999\\
10.01	0.00999999999999999\\
11.01	0.00999999999999999\\
12.01	0.00999999999999999\\
13.01	0.00999999999999999\\
14.01	0.00999999999999999\\
15.01	0.00999999999999999\\
16.01	0.00999999999999999\\
17.01	0.00999999999999999\\
18.01	0.00999999999999999\\
19.01	0.00999999999999999\\
20.01	0.00999999999999999\\
21.01	0.00999999999999999\\
22.01	0.00999999999999999\\
23.01	0.00999999999999999\\
24.01	0.00999999999999999\\
25.01	0.00999999999999999\\
26.01	0.00999999999999999\\
27.01	0.00999999999999999\\
28.01	0.00999999999999999\\
29.01	0.00999999999999999\\
30.01	0.00999999999999999\\
31.01	0.00999999999999999\\
32.01	0.00999999999999999\\
33.01	0.00999999999999999\\
34.01	0.00999999999999999\\
35.01	0.00999999999999999\\
36.01	0.00999999999999999\\
37.01	0.00999999999999999\\
38.01	0.00999999999999999\\
39.01	0.00999999999999999\\
40.01	0.00999999999999999\\
41.01	0.00999999999999999\\
42.01	0.00999999999999999\\
43.01	0.00999999999999999\\
44.01	0.00999999999999999\\
45.01	0.00999999999999999\\
46.01	0.00999999999999999\\
47.01	0.00999999999999999\\
48.01	0.00999999999999999\\
49.01	0.00999999999999999\\
50.01	0.00999999999999999\\
51.01	0.00999999999999999\\
52.01	0.00999999999999999\\
53.01	0.00999999999999999\\
54.01	0.00999999999999999\\
55.01	0.00999999999999999\\
56.01	0.00999999999999999\\
57.01	0.00999999999999999\\
58.01	0.00999999999999999\\
59.01	0.00999999999999999\\
60.01	0.00999999999999999\\
61.01	0.00999999999999999\\
62.01	0.00999999999999999\\
63.01	0.00999999999999999\\
64.01	0.00999999999999999\\
65.01	0.00999999999999999\\
66.01	0.00999999999999999\\
67.01	0.00999999999999999\\
68.01	0.00999999999999999\\
69.01	0.00999999999999999\\
70.01	0.00999999999999999\\
71.01	0.00999999999999999\\
72.01	0.00999999999999999\\
73.01	0.00999999999999999\\
74.01	0.00999999999999999\\
75.01	0.00999999999999999\\
76.01	0.00999999999999999\\
77.01	0.00999999999999999\\
78.01	0.00999999999999999\\
79.01	0.00999999999999999\\
80.01	0.00999999999999999\\
81.01	0.00999999999999999\\
82.01	0.00999999999999999\\
83.01	0.00999999999999999\\
84.01	0.00999999999999999\\
85.01	0.00999999999999999\\
86.01	0.00999999999999999\\
87.01	0.00999999999999999\\
88.01	0.00999999999999999\\
89.01	0.00999999999999999\\
90.01	0.00999999999999999\\
91.01	0.00999999999999999\\
92.01	0.00999999999999999\\
93.01	0.00999999999999999\\
94.01	0.00999999999999999\\
95.01	0.00999999999999999\\
96.01	0.00999999999999999\\
97.01	0.00999999999999999\\
98.01	0.00999999999999999\\
99.01	0.00999999999999999\\
100.01	0.00999999999999999\\
101.01	0.00999999999999999\\
102.01	0.00999999999999999\\
103.01	0.00999999999999999\\
104.01	0.00999999999999999\\
105.01	0.00999999999999999\\
106.01	0.00999999999999999\\
107.01	0.00999999999999999\\
108.01	0.00999999999999999\\
109.01	0.00999999999999999\\
110.01	0.00999999999999999\\
111.01	0.00999999999999999\\
112.01	0.00999999999999999\\
113.01	0.00999999999999999\\
114.01	0.00999999999999999\\
115.01	0.00999999999999999\\
116.01	0.00999999999999999\\
117.01	0.00999999999999999\\
118.01	0.00999999999999999\\
119.01	0.00999999999999999\\
120.01	0.00999999999999999\\
121.01	0.00999999999999999\\
122.01	0.00999999999999999\\
123.01	0.00999999999999999\\
124.01	0.00999999999999999\\
125.01	0.00999999999999999\\
126.01	0.00999999999999999\\
127.01	0.00999999999999999\\
128.01	0.00999999999999999\\
129.01	0.00999999999999999\\
130.01	0.00999999999999999\\
131.01	0.00999999999999999\\
132.01	0.00999999999999999\\
133.01	0.00999999999999999\\
134.01	0.00999999999999999\\
135.01	0.00999999999999999\\
136.01	0.00999999999999999\\
137.01	0.00999999999999999\\
138.01	0.00999999999999999\\
139.01	0.00999999999999999\\
140.01	0.00999999999999999\\
141.01	0.00999999999999999\\
142.01	0.00999999999999999\\
143.01	0.00999999999999999\\
144.01	0.00999999999999999\\
145.01	0.00999999999999999\\
146.01	0.00999999999999999\\
147.01	0.00999999999999999\\
148.01	0.00999999999999999\\
149.01	0.00999999999999999\\
150.01	0.00999999999999999\\
151.01	0.00999999999999999\\
152.01	0.00999999999999999\\
153.01	0.00999999999999999\\
154.01	0.00999999999999999\\
155.01	0.00999999999999999\\
156.01	0.00999999999999999\\
157.01	0.00999999999999999\\
158.01	0.00999999999999999\\
159.01	0.00999999999999999\\
160.01	0.00999999999999999\\
161.01	0.00999999999999999\\
162.01	0.00999999999999999\\
163.01	0.00999999999999999\\
164.01	0.00999999999999999\\
165.01	0.00999999999999999\\
166.01	0.00999999999999999\\
167.01	0.00999999999999999\\
168.01	0.00999999999999999\\
169.01	0.00999999999999999\\
170.01	0.00999999999999999\\
171.01	0.00999999999999999\\
172.01	0.00999999999999999\\
173.01	0.00999999999999999\\
174.01	0.00999999999999999\\
175.01	0.00999999999999999\\
176.01	0.00999999999999999\\
177.01	0.00999999999999999\\
178.01	0.00999999999999999\\
179.01	0.00999999999999999\\
180.01	0.00999999999999999\\
181.01	0.00999999999999999\\
182.01	0.00999999999999999\\
183.01	0.00999999999999999\\
184.01	0.00999999999999999\\
185.01	0.00999999999999999\\
186.01	0.00999999999999999\\
187.01	0.00999999999999999\\
188.01	0.00999999999999999\\
189.01	0.00999999999999999\\
190.01	0.00999999999999999\\
191.01	0.00999999999999999\\
192.01	0.00999999999999999\\
193.01	0.00999999999999999\\
194.01	0.00999999999999999\\
195.01	0.00999999999999999\\
196.01	0.00999999999999999\\
197.01	0.00999999999999999\\
198.01	0.00999999999999999\\
199.01	0.00999999999999999\\
200.01	0.00999999999999999\\
201.01	0.00999999999999999\\
202.01	0.00999999999999999\\
203.01	0.00999999999999999\\
204.01	0.00999999999999999\\
205.01	0.00999999999999999\\
206.01	0.00999999999999999\\
207.01	0.00999999999999999\\
208.01	0.00999999999999999\\
209.01	0.00999999999999999\\
210.01	0.00999999999999999\\
211.01	0.00999999999999999\\
212.01	0.00999999999999999\\
213.01	0.00999999999999999\\
214.01	0.00999999999999999\\
215.01	0.00999999999999999\\
216.01	0.00999999999999999\\
217.01	0.00999999999999999\\
218.01	0.00999999999999999\\
219.01	0.00999999999999999\\
220.01	0.00999999999999999\\
221.01	0.00999999999999999\\
222.01	0.00999999999999999\\
223.01	0.00999999999999999\\
224.01	0.00999999999999999\\
225.01	0.00999999999999999\\
226.01	0.00999999999999999\\
227.01	0.00999999999999999\\
228.01	0.00999999999999999\\
229.01	0.00999999999999999\\
230.01	0.00999999999999999\\
231.01	0.00999999999999999\\
232.01	0.00999999999999999\\
233.01	0.00999999999999999\\
234.01	0.00999999999999999\\
235.01	0.00999999999999999\\
236.01	0.00999999999999999\\
237.01	0.00999999999999999\\
238.01	0.00999999999999999\\
239.01	0.00999999999999999\\
240.01	0.00999999999999999\\
241.01	0.00999999999999999\\
242.01	0.00999999999999999\\
243.01	0.00999999999999999\\
244.01	0.00999999999999999\\
245.01	0.00999999999999999\\
246.01	0.00999999999999999\\
247.01	0.00999999999999999\\
248.01	0.00999999999999999\\
249.01	0.00999999999999999\\
250.01	0.00999999999999999\\
251.01	0.00999999999999999\\
252.01	0.00999999999999999\\
253.01	0.00999999999999999\\
254.01	0.00999999999999999\\
255.01	0.00999999999999999\\
256.01	0.00999999999999999\\
257.01	0.00999999999999999\\
258.01	0.00999999999999999\\
259.01	0.00999999999999999\\
260.01	0.00999999999999999\\
261.01	0.00999999999999999\\
262.01	0.00999999999999999\\
263.01	0.00999999999999999\\
264.01	0.00999999999999999\\
265.01	0.00999999999999999\\
266.01	0.00999999999999999\\
267.01	0.00999999999999999\\
268.01	0.00999999999999999\\
269.01	0.00999999999999999\\
270.01	0.00999999999999999\\
271.01	0.00999999999999999\\
272.01	0.00999999999999999\\
273.01	0.00999999999999999\\
274.01	0.00999999999999999\\
275.01	0.00999999999999999\\
276.01	0.00999999999999999\\
277.01	0.00999999999999999\\
278.01	0.00999999999999999\\
279.01	0.00999999999999999\\
280.01	0.00999999999999999\\
281.01	0.00999999999999999\\
282.01	0.00999999999999999\\
283.01	0.00999999999999999\\
284.01	0.00999999999999999\\
285.01	0.00999999999999999\\
286.01	0.00999999999999999\\
287.01	0.00999999999999999\\
288.01	0.00999999999999999\\
289.01	0.00999999999999999\\
290.01	0.00999999999999999\\
291.01	0.00999999999999999\\
292.01	0.00999999999999999\\
293.01	0.00999999999999999\\
294.01	0.00999999999999999\\
295.01	0.00999999999999999\\
296.01	0.00999999999999999\\
297.01	0.00999999999999999\\
298.01	0.00999999999999999\\
299.01	0.00999999999999999\\
300.01	0.00999999999999999\\
301.01	0.00999999999999999\\
302.01	0.00999999999999999\\
303.01	0.00999999999999999\\
304.01	0.00999999999999999\\
305.01	0.00999999999999999\\
306.01	0.00999999999999999\\
307.01	0.00999999999999999\\
308.01	0.00999999999999999\\
309.01	0.00999999999999999\\
310.01	0.00999999999999999\\
311.01	0.00999999999999999\\
312.01	0.00999999999999999\\
313.01	0.00999999999999999\\
314.01	0.00999999999999999\\
315.01	0.00999999999999999\\
316.01	0.00999999999999999\\
317.01	0.00999999999999999\\
318.01	0.00999999999999999\\
319.01	0.00999999999999999\\
320.01	0.00999999999999999\\
321.01	0.00999999999999999\\
322.01	0.00999999999999999\\
323.01	0.00999999999999999\\
324.01	0.00999999999999999\\
325.01	0.00999999999999999\\
326.01	0.00999999999999999\\
327.01	0.00999999999999999\\
328.01	0.00999999999999999\\
329.01	0.00999999999999999\\
330.01	0.00999999999999999\\
331.01	0.00999999999999999\\
332.01	0.00999999999999999\\
333.01	0.00999999999999999\\
334.01	0.00999999999999999\\
335.01	0.00999999999999999\\
336.01	0.00999999999999999\\
337.01	0.00999999999999999\\
338.01	0.00999999999999999\\
339.01	0.00999999999999999\\
340.01	0.00999999999999999\\
341.01	0.00999999999999999\\
342.01	0.00999999999999999\\
343.01	0.00999999999999999\\
344.01	0.00999999999999999\\
345.01	0.00999999999999999\\
346.01	0.00999999999999999\\
347.01	0.00999999999999999\\
348.01	0.00999999999999999\\
349.01	0.00999999999999999\\
350.01	0.00999999999999999\\
351.01	0.00999999999999999\\
352.01	0.00999999999999999\\
353.01	0.00999999999999999\\
354.01	0.00999999999999999\\
355.01	0.00999999999999999\\
356.01	0.00999999999999999\\
357.01	0.00999999999999999\\
358.01	0.00999999999999999\\
359.01	0.00999999999999999\\
360.01	0.00999999999999999\\
361.01	0.00999999999999999\\
362.01	0.00999999999999999\\
363.01	0.00999999999999999\\
364.01	0.00999999999999999\\
365.01	0.00999999999999999\\
366.01	0.00999999999999999\\
367.01	0.00999999999999999\\
368.01	0.00999999999999999\\
369.01	0.00999999999999999\\
370.01	0.00999999999999999\\
371.01	0.00999999999999999\\
372.01	0.00999999999999999\\
373.01	0.00999999999999999\\
374.01	0.00999999999999999\\
375.01	0.00999999999999999\\
376.01	0.00999999999999999\\
377.01	0.00999999999999999\\
378.01	0.00999999999999999\\
379.01	0.00999999999999999\\
380.01	0.00999999999999999\\
381.01	0.00999999999999999\\
382.01	0.00999999999999999\\
383.01	0.00999999999999999\\
384.01	0.00999999999999999\\
385.01	0.00999999999999999\\
386.01	0.00999999999999999\\
387.01	0.00999999999999999\\
388.01	0.00999999999999999\\
389.01	0.00999999999999999\\
390.01	0.00999999999999999\\
391.01	0.00999999999999999\\
392.01	0.00999999999999999\\
393.01	0.00999999999999999\\
394.01	0.00999999999999999\\
395.01	0.00999999999999999\\
396.01	0.00999999999999999\\
397.01	0.00999999999999999\\
398.01	0.00999999999999999\\
399.01	0.00999999999999999\\
400.01	0.00999999999999999\\
401.01	0.00999999999999999\\
402.01	0.00999999999999999\\
403.01	0.00999999999999999\\
404.01	0.00999999999999999\\
405.01	0.00999999999999999\\
406.01	0.00999999999999999\\
407.01	0.00999999999999999\\
408.01	0.00999999999999999\\
409.01	0.00999999999999999\\
410.01	0.00999999999999999\\
411.01	0.00999999999999999\\
412.01	0.00999999999999999\\
413.01	0.00999999999999999\\
414.01	0.00999999999999999\\
415.01	0.00999999999999999\\
416.01	0.00999999999999999\\
417.01	0.00999999999999999\\
418.01	0.00999999999999999\\
419.01	0.00999999999999999\\
420.01	0.00999999999999999\\
421.01	0.00999999999999999\\
422.01	0.00999999999999999\\
423.01	0.00999999999999999\\
424.01	0.00999999999999999\\
425.01	0.00999999999999999\\
426.01	0.00999999999999999\\
427.01	0.00999999999999999\\
428.01	0.00999999999999999\\
429.01	0.00999999999999999\\
430.01	0.00999999999999999\\
431.01	0.00999999999999999\\
432.01	0.00999999999999999\\
433.01	0.00999999999999999\\
434.01	0.00999999999999999\\
435.01	0.00999999999999999\\
436.01	0.00999999999999999\\
437.01	0.00999999999999999\\
438.01	0.00999999999999999\\
439.01	0.00999999999999999\\
440.01	0.00999999999999999\\
441.01	0.00999999999999999\\
442.01	0.00999999999999999\\
443.01	0.00999999999999999\\
444.01	0.00999999999999999\\
445.01	0.00999999999999999\\
446.01	0.00999999999999999\\
447.01	0.00999999999999999\\
448.01	0.00999999999999999\\
449.01	0.00999999999999999\\
450.01	0.00999999999999999\\
451.01	0.00999999999999999\\
452.01	0.00999999999999999\\
453.01	0.00999999999999999\\
454.01	0.00999999999999999\\
455.01	0.00999999999999999\\
456.01	0.00999999999999999\\
457.01	0.00999999999999999\\
458.01	0.00999999999999999\\
459.01	0.00999999999999999\\
460.01	0.00999999999999999\\
461.01	0.00999999999999999\\
462.01	0.00999999999999999\\
463.01	0.00999999999999999\\
464.01	0.00999999999999999\\
465.01	0.00999999999999999\\
466.01	0.00999999999999999\\
467.01	0.00999999999999999\\
468.01	0.00999999999999999\\
469.01	0.00999999999999999\\
470.01	0.00999999999999999\\
471.01	0.00999999999999999\\
472.01	0.00999999999999999\\
473.01	0.00999999999999999\\
474.01	0.00999999999999999\\
475.01	0.00999999999999999\\
476.01	0.00999999999999999\\
477.01	0.00999999999999999\\
478.01	0.00999999999999999\\
479.01	0.00999999999999999\\
480.01	0.00999999999999999\\
481.01	0.00999999999999999\\
482.01	0.00999999999999999\\
483.01	0.00999999999999999\\
484.01	0.00999999999999999\\
485.01	0.00999999999999999\\
486.01	0.00999999999999999\\
487.01	0.00999999999999999\\
488.01	0.00999999999999999\\
489.01	0.00999999999999999\\
490.01	0.00999999999999999\\
491.01	0.00999999999999999\\
492.01	0.00999999999999999\\
493.01	0.00999999999999999\\
494.01	0.00999999999999999\\
495.01	0.00999999999999999\\
496.01	0.00999999999999999\\
497.01	0.00999999999999999\\
498.01	0.00999999999999999\\
499.01	0.00999999999999999\\
500.01	0.00999999999999999\\
501.01	0.00999999999999999\\
502.01	0.00999999999999999\\
503.01	0.00999999999999999\\
504.01	0.00999999999999999\\
505.01	0.00999999999999999\\
506.01	0.00999999999999999\\
507.01	0.00999999999999999\\
508.01	0.00999999999999999\\
509.01	0.00999999999999999\\
510.01	0.00999999999999999\\
511.01	0.00999999999999999\\
512.01	0.00999999999999999\\
513.01	0.00999999999999999\\
514.01	0.00999999999999999\\
515.01	0.00999999999999999\\
516.01	0.00999999999999999\\
517.01	0.00999999999999999\\
518.01	0.00999999999999999\\
519.01	0.00999999999999999\\
520.01	0.00999999999999999\\
521.01	0.00999999999999999\\
522.01	0.00999999999999999\\
523.01	0.00999999999999999\\
524.01	0.00999999999999999\\
525.01	0.00999999999999999\\
526.01	0.00999999999999999\\
527.01	0.00999999999999999\\
528.01	0.00999999999999999\\
529.01	0.00999999999999999\\
530.01	0.00999999999999999\\
531.01	0.00999999999999999\\
532.01	0.00999999999999999\\
533.01	0.00999999999999999\\
534.01	0.00999999999999999\\
535.01	0.00999999999999999\\
536.01	0.00999999999999999\\
537.01	0.00999999999999999\\
538.01	0.00999999999999999\\
539.01	0.00999999999999999\\
540.01	0.00999999999999999\\
541.01	0.00999999999999999\\
542.01	0.00999999999999999\\
543.01	0.00999999999999999\\
544.01	0.00999999999999999\\
545.01	0.00999999999999999\\
546.01	0.00999999999999999\\
547.01	0.00999999999999999\\
548.01	0.00999999999999999\\
549.01	0.00999999999999999\\
550.01	0.00999999999999999\\
551.01	0.00999999999999999\\
552.01	0.00999999999999999\\
553.01	0.00999999999999999\\
554.01	0.00999999999999999\\
555.01	0.00999999999999999\\
556.01	0.00999999999999999\\
557.01	0.00999999999999999\\
558.01	0.00999999999999999\\
559.01	0.00999999999999999\\
560.01	0.00999999999999999\\
561.01	0.00999999999999999\\
562.01	0.00999999999999999\\
563.01	0.00999999999999999\\
564.01	0.00999999999999999\\
565.01	0.00999999999999999\\
566.01	0.00999999999999999\\
567.01	0.00999999999999999\\
568.01	0.00999999999999999\\
569.01	0.00999999999999999\\
570.01	0.00999999999999999\\
571.01	0.00999999999999999\\
572.01	0.00999999999999999\\
573.01	0.00999999999999999\\
574.01	0.00999999999999999\\
575.01	0.00999999999999999\\
576.01	0.00999999999999999\\
577.01	0.00999999999999999\\
578.01	0.00999999999999999\\
579.01	0.00999999999999999\\
580.01	0.00999999999999999\\
581.01	0.00999999999999999\\
582.01	0.00999999999999999\\
583.01	0.00999999999999999\\
584.01	0.00999999999999999\\
585.01	0.00999999999999999\\
586.01	0.00999999999999999\\
587.01	0.00999999999999999\\
588.01	0.00999999999999999\\
589.01	0.00999999999999999\\
590.01	0.00999999999999999\\
591.01	0.00999999999999999\\
592.01	0.00999999999999999\\
593.01	0.00999999999999999\\
594.01	0.00999999999999999\\
595.01	0.00999999999999999\\
596.01	0.00999999999999999\\
597.01	0.00999999999999999\\
598.01	0.00999999999999999\\
599.01	0.00623513569405279\\
599.02	0.0061974735980932\\
599.03	0.00615944475597968\\
599.04	0.0061210455620573\\
599.05	0.00608227237521589\\
599.06	0.00604312151854133\\
599.07	0.00600358927896352\\
599.08	0.0059636719069008\\
599.09	0.00592336561590098\\
599.1	0.00588266658227871\\
599.11	0.00584157094474941\\
599.12	0.00580007480405952\\
599.13	0.00575817422261318\\
599.14	0.00571586522409518\\
599.15	0.00567314379309022\\
599.16	0.00563000587469847\\
599.17	0.00558644737414726\\
599.18	0.00554246415639909\\
599.19	0.00549805204575559\\
599.2	0.00545320682545778\\
599.21	0.00540792423728233\\
599.22	0.00536219998113371\\
599.23	0.00531602971463264\\
599.24	0.00526940905270014\\
599.25	0.00522233356713779\\
599.26	0.00517479878620363\\
599.27	0.00512680019418406\\
599.28	0.00507833323096144\\
599.29	0.00502939329157744\\
599.3	0.00497997572579218\\
599.31	0.00493007583763889\\
599.32	0.00487968888497432\\
599.33	0.00482881007902469\\
599.34	0.00477743458392716\\
599.35	0.00472555751626675\\
599.36	0.00467317394460885\\
599.37	0.00462027888902696\\
599.38	0.0045668673206259\\
599.39	0.0045129341610603\\
599.4	0.00445847428204831\\
599.41	0.00440348251095549\\
599.42	0.00434795363554375\\
599.43	0.00429188239249288\\
599.44	0.00423526346689991\\
599.45	0.00417809149177358\\
599.46	0.00412036104752388\\
599.47	0.00406206666144656\\
599.48	0.00400320280720258\\
599.49	0.00394376390429251\\
599.5	0.00388374431752568\\
599.51	0.00382313835648424\\
599.52	0.00376194027498192\\
599.53	0.00370014427051746\\
599.54	0.0036377444837227\\
599.55	0.00357473499780529\\
599.56	0.00351110983798593\\
599.57	0.00344686297093003\\
599.58	0.00338198830417389\\
599.59	0.00331647968554523\\
599.6	0.003250330902578\\
599.61	0.00318353568192148\\
599.62	0.00311608768874362\\
599.63	0.0030479805261285\\
599.64	0.00297920773446789\\
599.65	0.00290976279084682\\
599.66	0.0028396391084232\\
599.67	0.00276883003580131\\
599.68	0.00269732885639914\\
599.69	0.00262512878780953\\
599.7	0.00255222298115511\\
599.71	0.00247860452043686\\
599.72	0.0024042664218763\\
599.73	0.0023292016332512\\
599.74	0.00225340303322488\\
599.75	0.00217686343066881\\
599.76	0.00209957556397867\\
599.77	0.00202153210038367\\
599.78	0.00194272563524904\\
599.79	0.00186314869137186\\
599.8	0.00178279371826976\\
599.81	0.00170165309146283\\
599.82	0.00161971911174841\\
599.83	0.00153698400446879\\
599.84	0.00145343991877172\\
599.85	0.00136907892686372\\
599.86	0.00128389302325598\\
599.87	0.001197874124003\\
599.88	0.00111101406593363\\
599.89	0.00102330460587468\\
599.9	0.0009347374198669\\
599.91	0.000845304102373183\\
599.92	0.000754996165479164\\
599.93	0.000663805038085866\\
599.94	0.000571722065094472\\
599.95	0.000478738506583127\\
599.96	0.000384845536975636\\
599.97	0.000290034244202044\\
599.98	0.00019429562885097\\
599.99	9.76206033136556e-05\\
600	0\\
};
\addplot [color=blue!50!mycolor7,solid,forget plot]
  table[row sep=crcr]{%
0.01	0.00999999999999999\\
1.01	0.00999999999999999\\
2.01	0.00999999999999999\\
3.01	0.00999999999999999\\
4.01	0.00999999999999999\\
5.01	0.00999999999999999\\
6.01	0.00999999999999999\\
7.01	0.00999999999999999\\
8.01	0.00999999999999999\\
9.01	0.00999999999999999\\
10.01	0.00999999999999999\\
11.01	0.00999999999999999\\
12.01	0.00999999999999999\\
13.01	0.00999999999999999\\
14.01	0.00999999999999999\\
15.01	0.00999999999999999\\
16.01	0.00999999999999999\\
17.01	0.00999999999999999\\
18.01	0.00999999999999999\\
19.01	0.00999999999999999\\
20.01	0.00999999999999999\\
21.01	0.00999999999999999\\
22.01	0.00999999999999999\\
23.01	0.00999999999999999\\
24.01	0.00999999999999999\\
25.01	0.00999999999999999\\
26.01	0.00999999999999999\\
27.01	0.00999999999999999\\
28.01	0.00999999999999999\\
29.01	0.00999999999999999\\
30.01	0.00999999999999999\\
31.01	0.00999999999999999\\
32.01	0.00999999999999999\\
33.01	0.00999999999999999\\
34.01	0.00999999999999999\\
35.01	0.00999999999999999\\
36.01	0.00999999999999999\\
37.01	0.00999999999999999\\
38.01	0.00999999999999999\\
39.01	0.00999999999999999\\
40.01	0.00999999999999999\\
41.01	0.00999999999999999\\
42.01	0.00999999999999999\\
43.01	0.00999999999999999\\
44.01	0.00999999999999999\\
45.01	0.00999999999999999\\
46.01	0.00999999999999999\\
47.01	0.00999999999999999\\
48.01	0.00999999999999999\\
49.01	0.00999999999999999\\
50.01	0.00999999999999999\\
51.01	0.00999999999999999\\
52.01	0.00999999999999999\\
53.01	0.00999999999999999\\
54.01	0.00999999999999999\\
55.01	0.00999999999999999\\
56.01	0.00999999999999999\\
57.01	0.00999999999999999\\
58.01	0.00999999999999999\\
59.01	0.00999999999999999\\
60.01	0.00999999999999999\\
61.01	0.00999999999999999\\
62.01	0.00999999999999999\\
63.01	0.00999999999999999\\
64.01	0.00999999999999999\\
65.01	0.00999999999999999\\
66.01	0.00999999999999999\\
67.01	0.00999999999999999\\
68.01	0.00999999999999999\\
69.01	0.00999999999999999\\
70.01	0.00999999999999999\\
71.01	0.00999999999999999\\
72.01	0.00999999999999999\\
73.01	0.00999999999999999\\
74.01	0.00999999999999999\\
75.01	0.00999999999999999\\
76.01	0.00999999999999999\\
77.01	0.00999999999999999\\
78.01	0.00999999999999999\\
79.01	0.00999999999999999\\
80.01	0.00999999999999999\\
81.01	0.00999999999999999\\
82.01	0.00999999999999999\\
83.01	0.00999999999999999\\
84.01	0.00999999999999999\\
85.01	0.00999999999999999\\
86.01	0.00999999999999999\\
87.01	0.00999999999999999\\
88.01	0.00999999999999999\\
89.01	0.00999999999999999\\
90.01	0.00999999999999999\\
91.01	0.00999999999999999\\
92.01	0.00999999999999999\\
93.01	0.00999999999999999\\
94.01	0.00999999999999999\\
95.01	0.00999999999999999\\
96.01	0.00999999999999999\\
97.01	0.00999999999999999\\
98.01	0.00999999999999999\\
99.01	0.00999999999999999\\
100.01	0.00999999999999999\\
101.01	0.00999999999999999\\
102.01	0.00999999999999999\\
103.01	0.00999999999999999\\
104.01	0.00999999999999999\\
105.01	0.00999999999999999\\
106.01	0.00999999999999999\\
107.01	0.00999999999999999\\
108.01	0.00999999999999999\\
109.01	0.00999999999999999\\
110.01	0.00999999999999999\\
111.01	0.00999999999999999\\
112.01	0.00999999999999999\\
113.01	0.00999999999999999\\
114.01	0.00999999999999999\\
115.01	0.00999999999999999\\
116.01	0.00999999999999999\\
117.01	0.00999999999999999\\
118.01	0.00999999999999999\\
119.01	0.00999999999999999\\
120.01	0.00999999999999999\\
121.01	0.00999999999999999\\
122.01	0.00999999999999999\\
123.01	0.00999999999999999\\
124.01	0.00999999999999999\\
125.01	0.00999999999999999\\
126.01	0.00999999999999999\\
127.01	0.00999999999999999\\
128.01	0.00999999999999999\\
129.01	0.00999999999999999\\
130.01	0.00999999999999999\\
131.01	0.00999999999999999\\
132.01	0.00999999999999999\\
133.01	0.00999999999999999\\
134.01	0.00999999999999999\\
135.01	0.00999999999999999\\
136.01	0.00999999999999999\\
137.01	0.00999999999999999\\
138.01	0.00999999999999999\\
139.01	0.00999999999999999\\
140.01	0.00999999999999999\\
141.01	0.00999999999999999\\
142.01	0.00999999999999999\\
143.01	0.00999999999999999\\
144.01	0.00999999999999999\\
145.01	0.00999999999999999\\
146.01	0.00999999999999999\\
147.01	0.00999999999999999\\
148.01	0.00999999999999999\\
149.01	0.00999999999999999\\
150.01	0.00999999999999999\\
151.01	0.00999999999999999\\
152.01	0.00999999999999999\\
153.01	0.00999999999999999\\
154.01	0.00999999999999999\\
155.01	0.00999999999999999\\
156.01	0.00999999999999999\\
157.01	0.00999999999999999\\
158.01	0.00999999999999999\\
159.01	0.00999999999999999\\
160.01	0.00999999999999999\\
161.01	0.00999999999999999\\
162.01	0.00999999999999999\\
163.01	0.00999999999999999\\
164.01	0.00999999999999999\\
165.01	0.00999999999999999\\
166.01	0.00999999999999999\\
167.01	0.00999999999999999\\
168.01	0.00999999999999999\\
169.01	0.00999999999999999\\
170.01	0.00999999999999999\\
171.01	0.00999999999999999\\
172.01	0.00999999999999999\\
173.01	0.00999999999999999\\
174.01	0.00999999999999999\\
175.01	0.00999999999999999\\
176.01	0.00999999999999999\\
177.01	0.00999999999999999\\
178.01	0.00999999999999999\\
179.01	0.00999999999999999\\
180.01	0.00999999999999999\\
181.01	0.00999999999999999\\
182.01	0.00999999999999999\\
183.01	0.00999999999999999\\
184.01	0.00999999999999999\\
185.01	0.00999999999999999\\
186.01	0.00999999999999999\\
187.01	0.00999999999999999\\
188.01	0.00999999999999999\\
189.01	0.00999999999999999\\
190.01	0.00999999999999999\\
191.01	0.00999999999999999\\
192.01	0.00999999999999999\\
193.01	0.00999999999999999\\
194.01	0.00999999999999999\\
195.01	0.00999999999999999\\
196.01	0.00999999999999999\\
197.01	0.00999999999999999\\
198.01	0.00999999999999999\\
199.01	0.00999999999999999\\
200.01	0.00999999999999999\\
201.01	0.00999999999999999\\
202.01	0.00999999999999999\\
203.01	0.00999999999999999\\
204.01	0.00999999999999999\\
205.01	0.00999999999999999\\
206.01	0.00999999999999999\\
207.01	0.00999999999999999\\
208.01	0.00999999999999999\\
209.01	0.00999999999999999\\
210.01	0.00999999999999999\\
211.01	0.00999999999999999\\
212.01	0.00999999999999999\\
213.01	0.00999999999999999\\
214.01	0.00999999999999999\\
215.01	0.00999999999999999\\
216.01	0.00999999999999999\\
217.01	0.00999999999999999\\
218.01	0.00999999999999999\\
219.01	0.00999999999999999\\
220.01	0.00999999999999999\\
221.01	0.00999999999999999\\
222.01	0.00999999999999999\\
223.01	0.00999999999999999\\
224.01	0.00999999999999999\\
225.01	0.00999999999999999\\
226.01	0.00999999999999999\\
227.01	0.00999999999999999\\
228.01	0.00999999999999999\\
229.01	0.00999999999999999\\
230.01	0.00999999999999999\\
231.01	0.00999999999999999\\
232.01	0.00999999999999999\\
233.01	0.00999999999999999\\
234.01	0.00999999999999999\\
235.01	0.00999999999999999\\
236.01	0.00999999999999999\\
237.01	0.00999999999999999\\
238.01	0.00999999999999999\\
239.01	0.00999999999999999\\
240.01	0.00999999999999999\\
241.01	0.00999999999999999\\
242.01	0.00999999999999999\\
243.01	0.00999999999999999\\
244.01	0.00999999999999999\\
245.01	0.00999999999999999\\
246.01	0.00999999999999999\\
247.01	0.00999999999999999\\
248.01	0.00999999999999999\\
249.01	0.00999999999999999\\
250.01	0.00999999999999999\\
251.01	0.00999999999999999\\
252.01	0.00999999999999999\\
253.01	0.00999999999999999\\
254.01	0.00999999999999999\\
255.01	0.00999999999999999\\
256.01	0.00999999999999999\\
257.01	0.00999999999999999\\
258.01	0.00999999999999999\\
259.01	0.00999999999999999\\
260.01	0.00999999999999999\\
261.01	0.00999999999999999\\
262.01	0.00999999999999999\\
263.01	0.00999999999999999\\
264.01	0.00999999999999999\\
265.01	0.00999999999999999\\
266.01	0.00999999999999999\\
267.01	0.00999999999999999\\
268.01	0.00999999999999999\\
269.01	0.00999999999999999\\
270.01	0.00999999999999999\\
271.01	0.00999999999999999\\
272.01	0.00999999999999999\\
273.01	0.00999999999999999\\
274.01	0.00999999999999999\\
275.01	0.00999999999999999\\
276.01	0.00999999999999999\\
277.01	0.00999999999999999\\
278.01	0.00999999999999999\\
279.01	0.00999999999999999\\
280.01	0.00999999999999999\\
281.01	0.00999999999999999\\
282.01	0.00999999999999999\\
283.01	0.00999999999999999\\
284.01	0.00999999999999999\\
285.01	0.00999999999999999\\
286.01	0.00999999999999999\\
287.01	0.00999999999999999\\
288.01	0.00999999999999999\\
289.01	0.00999999999999999\\
290.01	0.00999999999999999\\
291.01	0.00999999999999999\\
292.01	0.00999999999999999\\
293.01	0.00999999999999999\\
294.01	0.00999999999999999\\
295.01	0.00999999999999999\\
296.01	0.00999999999999999\\
297.01	0.00999999999999999\\
298.01	0.00999999999999999\\
299.01	0.00999999999999999\\
300.01	0.00999999999999999\\
301.01	0.00999999999999999\\
302.01	0.00999999999999999\\
303.01	0.00999999999999999\\
304.01	0.00999999999999999\\
305.01	0.00999999999999999\\
306.01	0.00999999999999999\\
307.01	0.00999999999999999\\
308.01	0.00999999999999999\\
309.01	0.00999999999999999\\
310.01	0.00999999999999999\\
311.01	0.00999999999999999\\
312.01	0.00999999999999999\\
313.01	0.00999999999999999\\
314.01	0.00999999999999999\\
315.01	0.00999999999999999\\
316.01	0.00999999999999999\\
317.01	0.00999999999999999\\
318.01	0.00999999999999999\\
319.01	0.00999999999999999\\
320.01	0.00999999999999999\\
321.01	0.00999999999999999\\
322.01	0.00999999999999999\\
323.01	0.00999999999999999\\
324.01	0.00999999999999999\\
325.01	0.00999999999999999\\
326.01	0.00999999999999999\\
327.01	0.00999999999999999\\
328.01	0.00999999999999999\\
329.01	0.00999999999999999\\
330.01	0.00999999999999999\\
331.01	0.00999999999999999\\
332.01	0.00999999999999999\\
333.01	0.00999999999999999\\
334.01	0.00999999999999999\\
335.01	0.00999999999999999\\
336.01	0.00999999999999999\\
337.01	0.00999999999999999\\
338.01	0.00999999999999999\\
339.01	0.00999999999999999\\
340.01	0.00999999999999999\\
341.01	0.00999999999999999\\
342.01	0.00999999999999999\\
343.01	0.00999999999999999\\
344.01	0.00999999999999999\\
345.01	0.00999999999999999\\
346.01	0.00999999999999999\\
347.01	0.00999999999999999\\
348.01	0.00999999999999999\\
349.01	0.00999999999999999\\
350.01	0.00999999999999999\\
351.01	0.00999999999999999\\
352.01	0.00999999999999999\\
353.01	0.00999999999999999\\
354.01	0.00999999999999999\\
355.01	0.00999999999999999\\
356.01	0.00999999999999999\\
357.01	0.00999999999999999\\
358.01	0.00999999999999999\\
359.01	0.00999999999999999\\
360.01	0.00999999999999999\\
361.01	0.00999999999999999\\
362.01	0.00999999999999999\\
363.01	0.00999999999999999\\
364.01	0.00999999999999999\\
365.01	0.00999999999999999\\
366.01	0.00999999999999999\\
367.01	0.00999999999999999\\
368.01	0.00999999999999999\\
369.01	0.00999999999999999\\
370.01	0.00999999999999999\\
371.01	0.00999999999999999\\
372.01	0.00999999999999999\\
373.01	0.00999999999999999\\
374.01	0.00999999999999999\\
375.01	0.00999999999999999\\
376.01	0.00999999999999999\\
377.01	0.00999999999999999\\
378.01	0.00999999999999999\\
379.01	0.00999999999999999\\
380.01	0.00999999999999999\\
381.01	0.00999999999999999\\
382.01	0.00999999999999999\\
383.01	0.00999999999999999\\
384.01	0.00999999999999999\\
385.01	0.00999999999999999\\
386.01	0.00999999999999999\\
387.01	0.00999999999999999\\
388.01	0.00999999999999999\\
389.01	0.00999999999999999\\
390.01	0.00999999999999999\\
391.01	0.00999999999999999\\
392.01	0.00999999999999999\\
393.01	0.00999999999999999\\
394.01	0.00999999999999999\\
395.01	0.00999999999999999\\
396.01	0.00999999999999999\\
397.01	0.00999999999999999\\
398.01	0.00999999999999999\\
399.01	0.00999999999999999\\
400.01	0.00999999999999999\\
401.01	0.00999999999999999\\
402.01	0.00999999999999999\\
403.01	0.00999999999999999\\
404.01	0.00999999999999999\\
405.01	0.00999999999999999\\
406.01	0.00999999999999999\\
407.01	0.00999999999999999\\
408.01	0.00999999999999999\\
409.01	0.00999999999999999\\
410.01	0.00999999999999999\\
411.01	0.00999999999999999\\
412.01	0.00999999999999999\\
413.01	0.00999999999999999\\
414.01	0.00999999999999999\\
415.01	0.00999999999999999\\
416.01	0.00999999999999999\\
417.01	0.00999999999999999\\
418.01	0.00999999999999999\\
419.01	0.00999999999999999\\
420.01	0.00999999999999999\\
421.01	0.00999999999999999\\
422.01	0.00999999999999999\\
423.01	0.00999999999999999\\
424.01	0.00999999999999999\\
425.01	0.00999999999999999\\
426.01	0.00999999999999999\\
427.01	0.00999999999999999\\
428.01	0.00999999999999999\\
429.01	0.00999999999999999\\
430.01	0.00999999999999999\\
431.01	0.00999999999999999\\
432.01	0.00999999999999999\\
433.01	0.00999999999999999\\
434.01	0.00999999999999999\\
435.01	0.00999999999999999\\
436.01	0.00999999999999999\\
437.01	0.00999999999999999\\
438.01	0.00999999999999999\\
439.01	0.00999999999999999\\
440.01	0.00999999999999999\\
441.01	0.00999999999999999\\
442.01	0.00999999999999999\\
443.01	0.00999999999999999\\
444.01	0.00999999999999999\\
445.01	0.00999999999999999\\
446.01	0.00999999999999999\\
447.01	0.00999999999999999\\
448.01	0.00999999999999999\\
449.01	0.00999999999999999\\
450.01	0.00999999999999999\\
451.01	0.00999999999999999\\
452.01	0.00999999999999999\\
453.01	0.00999999999999999\\
454.01	0.00999999999999999\\
455.01	0.00999999999999999\\
456.01	0.00999999999999999\\
457.01	0.00999999999999999\\
458.01	0.00999999999999999\\
459.01	0.00999999999999999\\
460.01	0.00999999999999999\\
461.01	0.00999999999999999\\
462.01	0.00999999999999999\\
463.01	0.00999999999999999\\
464.01	0.00999999999999999\\
465.01	0.00999999999999999\\
466.01	0.00999999999999999\\
467.01	0.00999999999999999\\
468.01	0.00999999999999999\\
469.01	0.00999999999999999\\
470.01	0.00999999999999999\\
471.01	0.00999999999999999\\
472.01	0.00999999999999999\\
473.01	0.00999999999999999\\
474.01	0.00999999999999999\\
475.01	0.00999999999999999\\
476.01	0.00999999999999999\\
477.01	0.00999999999999999\\
478.01	0.00999999999999999\\
479.01	0.00999999999999999\\
480.01	0.00999999999999999\\
481.01	0.00999999999999999\\
482.01	0.00999999999999999\\
483.01	0.00999999999999999\\
484.01	0.00999999999999999\\
485.01	0.00999999999999999\\
486.01	0.00999999999999999\\
487.01	0.00999999999999999\\
488.01	0.00999999999999999\\
489.01	0.00999999999999999\\
490.01	0.00999999999999999\\
491.01	0.00999999999999999\\
492.01	0.00999999999999999\\
493.01	0.00999999999999999\\
494.01	0.00999999999999999\\
495.01	0.00999999999999999\\
496.01	0.00999999999999999\\
497.01	0.00999999999999999\\
498.01	0.00999999999999999\\
499.01	0.00999999999999999\\
500.01	0.00999999999999999\\
501.01	0.00999999999999999\\
502.01	0.00999999999999999\\
503.01	0.00999999999999999\\
504.01	0.00999999999999999\\
505.01	0.00999999999999999\\
506.01	0.00999999999999999\\
507.01	0.00999999999999999\\
508.01	0.00999999999999999\\
509.01	0.00999999999999999\\
510.01	0.00999999999999999\\
511.01	0.00999999999999999\\
512.01	0.00999999999999999\\
513.01	0.00999999999999999\\
514.01	0.00999999999999999\\
515.01	0.00999999999999999\\
516.01	0.00999999999999999\\
517.01	0.00999999999999999\\
518.01	0.00999999999999999\\
519.01	0.00999999999999999\\
520.01	0.00999999999999999\\
521.01	0.00999999999999999\\
522.01	0.00999999999999999\\
523.01	0.00999999999999999\\
524.01	0.00999999999999999\\
525.01	0.00999999999999999\\
526.01	0.00999999999999999\\
527.01	0.00999999999999999\\
528.01	0.00999999999999999\\
529.01	0.00999999999999999\\
530.01	0.00999999999999999\\
531.01	0.00999999999999999\\
532.01	0.00999999999999999\\
533.01	0.00999999999999999\\
534.01	0.00999999999999999\\
535.01	0.00999999999999999\\
536.01	0.00999999999999999\\
537.01	0.00999999999999999\\
538.01	0.00999999999999999\\
539.01	0.00999999999999999\\
540.01	0.00999999999999999\\
541.01	0.00999999999999999\\
542.01	0.00999999999999999\\
543.01	0.00999999999999999\\
544.01	0.00999999999999999\\
545.01	0.00999999999999999\\
546.01	0.00999999999999999\\
547.01	0.00999999999999999\\
548.01	0.00999999999999999\\
549.01	0.00999999999999999\\
550.01	0.00999999999999999\\
551.01	0.00999999999999999\\
552.01	0.00999999999999999\\
553.01	0.00999999999999999\\
554.01	0.00999999999999999\\
555.01	0.00999999999999999\\
556.01	0.00999999999999999\\
557.01	0.00999999999999999\\
558.01	0.00999999999999999\\
559.01	0.00999999999999999\\
560.01	0.00999999999999999\\
561.01	0.00999999999999999\\
562.01	0.00999999999999999\\
563.01	0.00999999999999999\\
564.01	0.00999999999999999\\
565.01	0.00999999999999999\\
566.01	0.00999999999999999\\
567.01	0.00999999999999999\\
568.01	0.00999999999999999\\
569.01	0.00999999999999999\\
570.01	0.00999999999999999\\
571.01	0.00999999999999999\\
572.01	0.00999999999999999\\
573.01	0.00999999999999999\\
574.01	0.00999999999999999\\
575.01	0.00999999999999999\\
576.01	0.00999999999999999\\
577.01	0.00999999999999999\\
578.01	0.00999999999999999\\
579.01	0.00999999999999999\\
580.01	0.00999999999999999\\
581.01	0.00999999999999999\\
582.01	0.00999999999999999\\
583.01	0.00999999999999999\\
584.01	0.00999999999999999\\
585.01	0.00999999999999999\\
586.01	0.00999999999999999\\
587.01	0.00999999999999999\\
588.01	0.00999999999999999\\
589.01	0.00999999999999999\\
590.01	0.00999999999999999\\
591.01	0.00999999999999999\\
592.01	0.00999999999999999\\
593.01	0.00999999999999999\\
594.01	0.00999999999999999\\
595.01	0.00999999999999999\\
596.01	0.00999999999999999\\
597.01	0.00999999999999999\\
598.01	0.00999999999999999\\
599.01	0.00623513569652336\\
599.02	0.00619747360025308\\
599.03	0.00615944475790929\\
599.04	0.00612104556381531\\
599.05	0.0060822723768421\\
599.06	0.00604312152006039\\
599.07	0.00600358928038925\\
599.08	0.00596367190824141\\
599.09	0.00592336561716329\\
599.1	0.00588266658346843\\
599.11	0.00584157094587136\\
599.12	0.00580007480511781\\
599.13	0.00575817422361138\\
599.14	0.00571586522503648\\
599.15	0.00567314379397755\\
599.16	0.00563000587553454\\
599.17	0.00558644737493464\\
599.18	0.00554246415714017\\
599.19	0.00549805204645267\\
599.2	0.00545320682611304\\
599.21	0.0054079242378978\\
599.22	0.00536219998171136\\
599.23	0.00531602971517431\\
599.24	0.0052694090532076\\
599.25	0.0052223335676127\\
599.26	0.00517479878664759\\
599.27	0.0051268001945986\\
599.28	0.00507833323134802\\
599.29	0.00502939329193748\\
599.3	0.004979975726127\\
599.31	0.00493007583794978\\
599.32	0.00487968888526254\\
599.33	0.00482881007929144\\
599.34	0.00477743458417358\\
599.35	0.00472555751649399\\
599.36	0.00467317394481798\\
599.37	0.00462027888921903\\
599.38	0.00456686732080193\\
599.39	0.00451293416122126\\
599.4	0.00445847428219516\\
599.41	0.00440348251108914\\
599.42	0.00434795363566509\\
599.43	0.00429188239260275\\
599.44	0.00423526346699912\\
599.45	0.00417809149186292\\
599.46	0.0041203610476041\\
599.47	0.00406206666151837\\
599.48	0.00400320280726666\\
599.49	0.0039437639043495\\
599.5	0.00388374431757618\\
599.51	0.00382313835652885\\
599.52	0.00376194027502117\\
599.53	0.00370014427055186\\
599.54	0.00363774448375272\\
599.55	0.00357473499783138\\
599.56	0.0035111098380085\\
599.57	0.00344686297094946\\
599.58	0.00338198830419054\\
599.59	0.00331647968555942\\
599.6	0.00325033090259001\\
599.61	0.00318353568193159\\
599.62	0.00311608768875208\\
599.63	0.00304798052613553\\
599.64	0.00297920773447369\\
599.65	0.00290976279085158\\
599.66	0.00283963910842707\\
599.67	0.00276883003580444\\
599.68	0.00269732885640163\\
599.69	0.0026251287878115\\
599.7	0.00255222298115666\\
599.71	0.00247860452043806\\
599.72	0.00240426642187721\\
599.73	0.00232920163325189\\
599.74	0.0022534030332254\\
599.75	0.00217686343066919\\
599.76	0.00209957556397894\\
599.77	0.00202153210038386\\
599.78	0.00194272563524917\\
599.79	0.00186314869137194\\
599.8	0.00178279371826981\\
599.81	0.00170165309146286\\
599.82	0.00161971911174843\\
599.83	0.0015369840044688\\
599.84	0.00145343991877172\\
599.85	0.00136907892686372\\
599.86	0.00128389302325598\\
599.87	0.00119787412400299\\
599.88	0.00111101406593363\\
599.89	0.00102330460587468\\
599.9	0.0009347374198669\\
599.91	0.000845304102373183\\
599.92	0.000754996165479166\\
599.93	0.000663805038085868\\
599.94	0.000571722065094473\\
599.95	0.000478738506583129\\
599.96	0.000384845536975638\\
599.97	0.000290034244202042\\
599.98	0.00019429562885097\\
599.99	9.76206033136556e-05\\
600	0\\
};
\addplot [color=blue!40!mycolor9,solid,forget plot]
  table[row sep=crcr]{%
0.01	0.01\\
1.01	0.01\\
2.01	0.01\\
3.01	0.01\\
4.01	0.01\\
5.01	0.01\\
6.01	0.01\\
7.01	0.01\\
8.01	0.01\\
9.01	0.01\\
10.01	0.01\\
11.01	0.01\\
12.01	0.01\\
13.01	0.01\\
14.01	0.01\\
15.01	0.01\\
16.01	0.01\\
17.01	0.01\\
18.01	0.01\\
19.01	0.01\\
20.01	0.01\\
21.01	0.01\\
22.01	0.01\\
23.01	0.01\\
24.01	0.01\\
25.01	0.01\\
26.01	0.01\\
27.01	0.01\\
28.01	0.01\\
29.01	0.01\\
30.01	0.01\\
31.01	0.01\\
32.01	0.01\\
33.01	0.01\\
34.01	0.01\\
35.01	0.01\\
36.01	0.01\\
37.01	0.01\\
38.01	0.01\\
39.01	0.01\\
40.01	0.01\\
41.01	0.01\\
42.01	0.01\\
43.01	0.01\\
44.01	0.01\\
45.01	0.01\\
46.01	0.01\\
47.01	0.01\\
48.01	0.01\\
49.01	0.01\\
50.01	0.01\\
51.01	0.01\\
52.01	0.01\\
53.01	0.01\\
54.01	0.01\\
55.01	0.01\\
56.01	0.01\\
57.01	0.01\\
58.01	0.01\\
59.01	0.01\\
60.01	0.01\\
61.01	0.01\\
62.01	0.01\\
63.01	0.01\\
64.01	0.01\\
65.01	0.01\\
66.01	0.01\\
67.01	0.01\\
68.01	0.01\\
69.01	0.01\\
70.01	0.01\\
71.01	0.01\\
72.01	0.01\\
73.01	0.01\\
74.01	0.01\\
75.01	0.01\\
76.01	0.01\\
77.01	0.01\\
78.01	0.01\\
79.01	0.01\\
80.01	0.01\\
81.01	0.01\\
82.01	0.01\\
83.01	0.01\\
84.01	0.01\\
85.01	0.01\\
86.01	0.01\\
87.01	0.01\\
88.01	0.01\\
89.01	0.01\\
90.01	0.01\\
91.01	0.01\\
92.01	0.01\\
93.01	0.01\\
94.01	0.01\\
95.01	0.01\\
96.01	0.01\\
97.01	0.01\\
98.01	0.01\\
99.01	0.01\\
100.01	0.01\\
101.01	0.01\\
102.01	0.01\\
103.01	0.01\\
104.01	0.01\\
105.01	0.01\\
106.01	0.01\\
107.01	0.01\\
108.01	0.01\\
109.01	0.01\\
110.01	0.01\\
111.01	0.01\\
112.01	0.01\\
113.01	0.01\\
114.01	0.01\\
115.01	0.01\\
116.01	0.01\\
117.01	0.01\\
118.01	0.01\\
119.01	0.01\\
120.01	0.01\\
121.01	0.01\\
122.01	0.01\\
123.01	0.01\\
124.01	0.01\\
125.01	0.01\\
126.01	0.01\\
127.01	0.01\\
128.01	0.01\\
129.01	0.01\\
130.01	0.01\\
131.01	0.01\\
132.01	0.01\\
133.01	0.01\\
134.01	0.01\\
135.01	0.01\\
136.01	0.01\\
137.01	0.01\\
138.01	0.01\\
139.01	0.01\\
140.01	0.01\\
141.01	0.01\\
142.01	0.01\\
143.01	0.01\\
144.01	0.01\\
145.01	0.01\\
146.01	0.01\\
147.01	0.01\\
148.01	0.01\\
149.01	0.01\\
150.01	0.01\\
151.01	0.01\\
152.01	0.01\\
153.01	0.01\\
154.01	0.01\\
155.01	0.01\\
156.01	0.01\\
157.01	0.01\\
158.01	0.01\\
159.01	0.01\\
160.01	0.01\\
161.01	0.01\\
162.01	0.01\\
163.01	0.01\\
164.01	0.01\\
165.01	0.01\\
166.01	0.01\\
167.01	0.01\\
168.01	0.01\\
169.01	0.01\\
170.01	0.01\\
171.01	0.01\\
172.01	0.01\\
173.01	0.01\\
174.01	0.01\\
175.01	0.01\\
176.01	0.01\\
177.01	0.01\\
178.01	0.01\\
179.01	0.01\\
180.01	0.01\\
181.01	0.01\\
182.01	0.01\\
183.01	0.01\\
184.01	0.01\\
185.01	0.01\\
186.01	0.01\\
187.01	0.01\\
188.01	0.01\\
189.01	0.01\\
190.01	0.01\\
191.01	0.01\\
192.01	0.01\\
193.01	0.01\\
194.01	0.01\\
195.01	0.01\\
196.01	0.01\\
197.01	0.01\\
198.01	0.01\\
199.01	0.01\\
200.01	0.01\\
201.01	0.01\\
202.01	0.01\\
203.01	0.01\\
204.01	0.01\\
205.01	0.01\\
206.01	0.01\\
207.01	0.01\\
208.01	0.01\\
209.01	0.01\\
210.01	0.01\\
211.01	0.01\\
212.01	0.01\\
213.01	0.01\\
214.01	0.01\\
215.01	0.01\\
216.01	0.01\\
217.01	0.01\\
218.01	0.01\\
219.01	0.01\\
220.01	0.01\\
221.01	0.01\\
222.01	0.01\\
223.01	0.01\\
224.01	0.01\\
225.01	0.01\\
226.01	0.01\\
227.01	0.01\\
228.01	0.01\\
229.01	0.01\\
230.01	0.01\\
231.01	0.01\\
232.01	0.01\\
233.01	0.01\\
234.01	0.01\\
235.01	0.01\\
236.01	0.01\\
237.01	0.01\\
238.01	0.01\\
239.01	0.01\\
240.01	0.01\\
241.01	0.01\\
242.01	0.01\\
243.01	0.01\\
244.01	0.01\\
245.01	0.01\\
246.01	0.01\\
247.01	0.01\\
248.01	0.01\\
249.01	0.01\\
250.01	0.01\\
251.01	0.01\\
252.01	0.01\\
253.01	0.01\\
254.01	0.01\\
255.01	0.01\\
256.01	0.01\\
257.01	0.01\\
258.01	0.01\\
259.01	0.01\\
260.01	0.01\\
261.01	0.01\\
262.01	0.01\\
263.01	0.01\\
264.01	0.01\\
265.01	0.01\\
266.01	0.01\\
267.01	0.01\\
268.01	0.01\\
269.01	0.01\\
270.01	0.01\\
271.01	0.01\\
272.01	0.01\\
273.01	0.01\\
274.01	0.01\\
275.01	0.01\\
276.01	0.01\\
277.01	0.01\\
278.01	0.01\\
279.01	0.01\\
280.01	0.01\\
281.01	0.01\\
282.01	0.01\\
283.01	0.01\\
284.01	0.01\\
285.01	0.01\\
286.01	0.01\\
287.01	0.01\\
288.01	0.01\\
289.01	0.01\\
290.01	0.01\\
291.01	0.01\\
292.01	0.01\\
293.01	0.01\\
294.01	0.01\\
295.01	0.01\\
296.01	0.01\\
297.01	0.01\\
298.01	0.01\\
299.01	0.01\\
300.01	0.01\\
301.01	0.01\\
302.01	0.01\\
303.01	0.01\\
304.01	0.01\\
305.01	0.01\\
306.01	0.01\\
307.01	0.01\\
308.01	0.01\\
309.01	0.01\\
310.01	0.01\\
311.01	0.01\\
312.01	0.01\\
313.01	0.01\\
314.01	0.01\\
315.01	0.01\\
316.01	0.01\\
317.01	0.01\\
318.01	0.01\\
319.01	0.01\\
320.01	0.01\\
321.01	0.01\\
322.01	0.01\\
323.01	0.01\\
324.01	0.01\\
325.01	0.01\\
326.01	0.01\\
327.01	0.01\\
328.01	0.01\\
329.01	0.01\\
330.01	0.01\\
331.01	0.01\\
332.01	0.01\\
333.01	0.01\\
334.01	0.01\\
335.01	0.01\\
336.01	0.01\\
337.01	0.01\\
338.01	0.01\\
339.01	0.01\\
340.01	0.01\\
341.01	0.01\\
342.01	0.01\\
343.01	0.01\\
344.01	0.01\\
345.01	0.01\\
346.01	0.01\\
347.01	0.01\\
348.01	0.01\\
349.01	0.01\\
350.01	0.01\\
351.01	0.01\\
352.01	0.01\\
353.01	0.01\\
354.01	0.01\\
355.01	0.01\\
356.01	0.01\\
357.01	0.01\\
358.01	0.01\\
359.01	0.01\\
360.01	0.01\\
361.01	0.01\\
362.01	0.01\\
363.01	0.01\\
364.01	0.01\\
365.01	0.01\\
366.01	0.01\\
367.01	0.01\\
368.01	0.01\\
369.01	0.01\\
370.01	0.01\\
371.01	0.01\\
372.01	0.01\\
373.01	0.01\\
374.01	0.01\\
375.01	0.01\\
376.01	0.01\\
377.01	0.01\\
378.01	0.01\\
379.01	0.01\\
380.01	0.01\\
381.01	0.01\\
382.01	0.01\\
383.01	0.01\\
384.01	0.01\\
385.01	0.01\\
386.01	0.01\\
387.01	0.01\\
388.01	0.01\\
389.01	0.01\\
390.01	0.01\\
391.01	0.01\\
392.01	0.01\\
393.01	0.01\\
394.01	0.01\\
395.01	0.01\\
396.01	0.01\\
397.01	0.01\\
398.01	0.01\\
399.01	0.01\\
400.01	0.01\\
401.01	0.01\\
402.01	0.01\\
403.01	0.01\\
404.01	0.01\\
405.01	0.01\\
406.01	0.01\\
407.01	0.01\\
408.01	0.01\\
409.01	0.01\\
410.01	0.01\\
411.01	0.01\\
412.01	0.01\\
413.01	0.01\\
414.01	0.01\\
415.01	0.01\\
416.01	0.01\\
417.01	0.01\\
418.01	0.01\\
419.01	0.01\\
420.01	0.01\\
421.01	0.01\\
422.01	0.01\\
423.01	0.01\\
424.01	0.01\\
425.01	0.01\\
426.01	0.01\\
427.01	0.01\\
428.01	0.01\\
429.01	0.01\\
430.01	0.01\\
431.01	0.01\\
432.01	0.01\\
433.01	0.01\\
434.01	0.01\\
435.01	0.01\\
436.01	0.01\\
437.01	0.01\\
438.01	0.01\\
439.01	0.01\\
440.01	0.01\\
441.01	0.01\\
442.01	0.01\\
443.01	0.01\\
444.01	0.01\\
445.01	0.01\\
446.01	0.01\\
447.01	0.01\\
448.01	0.01\\
449.01	0.01\\
450.01	0.01\\
451.01	0.01\\
452.01	0.01\\
453.01	0.01\\
454.01	0.01\\
455.01	0.01\\
456.01	0.01\\
457.01	0.01\\
458.01	0.01\\
459.01	0.01\\
460.01	0.01\\
461.01	0.01\\
462.01	0.01\\
463.01	0.01\\
464.01	0.01\\
465.01	0.01\\
466.01	0.01\\
467.01	0.01\\
468.01	0.01\\
469.01	0.01\\
470.01	0.01\\
471.01	0.01\\
472.01	0.01\\
473.01	0.01\\
474.01	0.01\\
475.01	0.01\\
476.01	0.01\\
477.01	0.01\\
478.01	0.01\\
479.01	0.01\\
480.01	0.01\\
481.01	0.01\\
482.01	0.01\\
483.01	0.01\\
484.01	0.01\\
485.01	0.01\\
486.01	0.01\\
487.01	0.01\\
488.01	0.01\\
489.01	0.01\\
490.01	0.01\\
491.01	0.01\\
492.01	0.01\\
493.01	0.01\\
494.01	0.01\\
495.01	0.01\\
496.01	0.01\\
497.01	0.01\\
498.01	0.01\\
499.01	0.01\\
500.01	0.01\\
501.01	0.01\\
502.01	0.01\\
503.01	0.01\\
504.01	0.01\\
505.01	0.01\\
506.01	0.01\\
507.01	0.01\\
508.01	0.01\\
509.01	0.01\\
510.01	0.01\\
511.01	0.01\\
512.01	0.01\\
513.01	0.01\\
514.01	0.01\\
515.01	0.01\\
516.01	0.01\\
517.01	0.01\\
518.01	0.01\\
519.01	0.01\\
520.01	0.01\\
521.01	0.01\\
522.01	0.01\\
523.01	0.01\\
524.01	0.01\\
525.01	0.01\\
526.01	0.01\\
527.01	0.01\\
528.01	0.01\\
529.01	0.01\\
530.01	0.01\\
531.01	0.01\\
532.01	0.01\\
533.01	0.01\\
534.01	0.01\\
535.01	0.01\\
536.01	0.01\\
537.01	0.01\\
538.01	0.01\\
539.01	0.01\\
540.01	0.01\\
541.01	0.01\\
542.01	0.01\\
543.01	0.01\\
544.01	0.01\\
545.01	0.01\\
546.01	0.01\\
547.01	0.01\\
548.01	0.01\\
549.01	0.01\\
550.01	0.01\\
551.01	0.01\\
552.01	0.01\\
553.01	0.01\\
554.01	0.01\\
555.01	0.01\\
556.01	0.01\\
557.01	0.01\\
558.01	0.01\\
559.01	0.01\\
560.01	0.01\\
561.01	0.01\\
562.01	0.01\\
563.01	0.01\\
564.01	0.01\\
565.01	0.01\\
566.01	0.01\\
567.01	0.01\\
568.01	0.01\\
569.01	0.01\\
570.01	0.01\\
571.01	0.01\\
572.01	0.01\\
573.01	0.01\\
574.01	0.01\\
575.01	0.01\\
576.01	0.01\\
577.01	0.01\\
578.01	0.01\\
579.01	0.01\\
580.01	0.01\\
581.01	0.01\\
582.01	0.01\\
583.01	0.01\\
584.01	0.01\\
585.01	0.01\\
586.01	0.01\\
587.01	0.01\\
588.01	0.01\\
589.01	0.01\\
590.01	0.01\\
591.01	0.01\\
592.01	0.01\\
593.01	0.01\\
594.01	0.01\\
595.01	0.01\\
596.01	0.01\\
597.01	0.01\\
598.01	0.01\\
599.01	0.00623513588550072\\
599.02	0.00619747375148484\\
599.03	0.00615944487982368\\
599.04	0.00612104566378498\\
599.05	0.00608227246107566\\
599.06	0.00604312159348489\\
599.07	0.00600358934652321\\
599.08	0.00596367196905715\\
599.09	0.00592336567339871\\
599.1	0.00588266663573343\\
599.11	0.00584157099466158\\
599.12	0.00580007485082921\\
599.13	0.00575817426655544\\
599.14	0.00571586526545588\\
599.15	0.00567314383206202\\
599.16	0.00563000591143662\\
599.17	0.00558644740878492\\
599.18	0.00554246418905517\\
599.19	0.00549805207654024\\
599.2	0.00545320685447463\\
599.21	0.0054079242646287\\
599.22	0.00536220000690092\\
599.23	0.00531602973890626\\
599.24	0.00526940907556028\\
599.25	0.00522233358865938\\
599.26	0.00517479880645675\\
599.27	0.00512680021323422\\
599.28	0.00507833324886986\\
599.29	0.0050293933084014\\
599.3	0.0049799757415853\\
599.31	0.00493007585245144\\
599.32	0.00487968889885358\\
599.33	0.00482881009201521\\
599.34	0.00477743459607107\\
599.35	0.00472555752760411\\
599.36	0.00467317395517781\\
599.37	0.0046202788988641\\
599.38	0.00456686732976642\\
599.39	0.00451293416953819\\
599.4	0.00445847428989657\\
599.41	0.00440348251820621\\
599.42	0.00434795364222823\\
599.43	0.00429188239864167\\
599.44	0.00423526347254286\\
599.45	0.00417809149693984\\
599.46	0.00412036105224189\\
599.47	0.00406206666574399\\
599.48	0.0040032028111063\\
599.49	0.00394376390782856\\
599.5	0.00388374432071923\\
599.51	0.0038231383593596\\
599.52	0.00376194027756244\\
599.53	0.00370014427282557\\
599.54	0.00363774448577988\\
599.55	0.00357473499963204\\
599.56	0.00351110983960176\\
599.57	0.00344686297235346\\
599.58	0.00338198830542247\\
599.59	0.00331647968663548\\
599.6	0.00325033090352545\\
599.61	0.0031835356827407\\
599.62	0.0031160876894482\\
599.63	0.00304798052673106\\
599.64	0.00297920773498011\\
599.65	0.00290976279127949\\
599.66	0.0028396391087862\\
599.67	0.00276883003610366\\
599.68	0.00269732885664901\\
599.69	0.00262512878801433\\
599.7	0.00255222298132147\\
599.71	0.00247860452057071\\
599.72	0.00240426642198287\\
599.73	0.0023292016333351\\
599.74	0.00225340303329012\\
599.75	0.00217686343071887\\
599.76	0.00209957556401652\\
599.77	0.00202153210041183\\
599.78	0.00194272563526962\\
599.79	0.0018631486913866\\
599.8	0.00178279371828008\\
599.81	0.00170165309146988\\
599.82	0.00161971911175309\\
599.83	0.00153698400447179\\
599.84	0.00145343991877358\\
599.85	0.00136907892686481\\
599.86	0.0012838930232566\\
599.87	0.00119787412400332\\
599.88	0.00111101406593378\\
599.89	0.00102330460587476\\
599.9	0.000934737419866927\\
599.91	0.000845304102373193\\
599.92	0.000754996165479169\\
599.93	0.000663805038085868\\
599.94	0.000571722065094473\\
599.95	0.000478738506583127\\
599.96	0.000384845536975638\\
599.97	0.000290034244202044\\
599.98	0.00019429562885097\\
599.99	9.76206033136574e-05\\
600	0\\
};
\addplot [color=blue!75!mycolor7,solid,forget plot]
  table[row sep=crcr]{%
0.01	0.01\\
1.01	0.01\\
2.01	0.01\\
3.01	0.01\\
4.01	0.01\\
5.01	0.01\\
6.01	0.01\\
7.01	0.01\\
8.01	0.01\\
9.01	0.01\\
10.01	0.01\\
11.01	0.01\\
12.01	0.01\\
13.01	0.01\\
14.01	0.01\\
15.01	0.01\\
16.01	0.01\\
17.01	0.01\\
18.01	0.01\\
19.01	0.01\\
20.01	0.01\\
21.01	0.01\\
22.01	0.01\\
23.01	0.01\\
24.01	0.01\\
25.01	0.01\\
26.01	0.01\\
27.01	0.01\\
28.01	0.01\\
29.01	0.01\\
30.01	0.01\\
31.01	0.01\\
32.01	0.01\\
33.01	0.01\\
34.01	0.01\\
35.01	0.01\\
36.01	0.01\\
37.01	0.01\\
38.01	0.01\\
39.01	0.01\\
40.01	0.01\\
41.01	0.01\\
42.01	0.01\\
43.01	0.01\\
44.01	0.01\\
45.01	0.01\\
46.01	0.01\\
47.01	0.01\\
48.01	0.01\\
49.01	0.01\\
50.01	0.01\\
51.01	0.01\\
52.01	0.01\\
53.01	0.01\\
54.01	0.01\\
55.01	0.01\\
56.01	0.01\\
57.01	0.01\\
58.01	0.01\\
59.01	0.01\\
60.01	0.01\\
61.01	0.01\\
62.01	0.01\\
63.01	0.01\\
64.01	0.01\\
65.01	0.01\\
66.01	0.01\\
67.01	0.01\\
68.01	0.01\\
69.01	0.01\\
70.01	0.01\\
71.01	0.01\\
72.01	0.01\\
73.01	0.01\\
74.01	0.01\\
75.01	0.01\\
76.01	0.01\\
77.01	0.01\\
78.01	0.01\\
79.01	0.01\\
80.01	0.01\\
81.01	0.01\\
82.01	0.01\\
83.01	0.01\\
84.01	0.01\\
85.01	0.01\\
86.01	0.01\\
87.01	0.01\\
88.01	0.01\\
89.01	0.01\\
90.01	0.01\\
91.01	0.01\\
92.01	0.01\\
93.01	0.01\\
94.01	0.01\\
95.01	0.01\\
96.01	0.01\\
97.01	0.01\\
98.01	0.01\\
99.01	0.01\\
100.01	0.01\\
101.01	0.01\\
102.01	0.01\\
103.01	0.01\\
104.01	0.01\\
105.01	0.01\\
106.01	0.01\\
107.01	0.01\\
108.01	0.01\\
109.01	0.01\\
110.01	0.01\\
111.01	0.01\\
112.01	0.01\\
113.01	0.01\\
114.01	0.01\\
115.01	0.01\\
116.01	0.01\\
117.01	0.01\\
118.01	0.01\\
119.01	0.01\\
120.01	0.01\\
121.01	0.01\\
122.01	0.01\\
123.01	0.01\\
124.01	0.01\\
125.01	0.01\\
126.01	0.01\\
127.01	0.01\\
128.01	0.01\\
129.01	0.01\\
130.01	0.01\\
131.01	0.01\\
132.01	0.01\\
133.01	0.01\\
134.01	0.01\\
135.01	0.01\\
136.01	0.01\\
137.01	0.01\\
138.01	0.01\\
139.01	0.01\\
140.01	0.01\\
141.01	0.01\\
142.01	0.01\\
143.01	0.01\\
144.01	0.01\\
145.01	0.01\\
146.01	0.01\\
147.01	0.01\\
148.01	0.01\\
149.01	0.01\\
150.01	0.01\\
151.01	0.01\\
152.01	0.01\\
153.01	0.01\\
154.01	0.01\\
155.01	0.01\\
156.01	0.01\\
157.01	0.01\\
158.01	0.01\\
159.01	0.01\\
160.01	0.01\\
161.01	0.01\\
162.01	0.01\\
163.01	0.01\\
164.01	0.01\\
165.01	0.01\\
166.01	0.01\\
167.01	0.01\\
168.01	0.01\\
169.01	0.01\\
170.01	0.01\\
171.01	0.01\\
172.01	0.01\\
173.01	0.01\\
174.01	0.01\\
175.01	0.01\\
176.01	0.01\\
177.01	0.01\\
178.01	0.01\\
179.01	0.01\\
180.01	0.01\\
181.01	0.01\\
182.01	0.01\\
183.01	0.01\\
184.01	0.01\\
185.01	0.01\\
186.01	0.01\\
187.01	0.01\\
188.01	0.01\\
189.01	0.01\\
190.01	0.01\\
191.01	0.01\\
192.01	0.01\\
193.01	0.01\\
194.01	0.01\\
195.01	0.01\\
196.01	0.01\\
197.01	0.01\\
198.01	0.01\\
199.01	0.01\\
200.01	0.01\\
201.01	0.01\\
202.01	0.01\\
203.01	0.01\\
204.01	0.01\\
205.01	0.01\\
206.01	0.01\\
207.01	0.01\\
208.01	0.01\\
209.01	0.01\\
210.01	0.01\\
211.01	0.01\\
212.01	0.01\\
213.01	0.01\\
214.01	0.01\\
215.01	0.01\\
216.01	0.01\\
217.01	0.01\\
218.01	0.01\\
219.01	0.01\\
220.01	0.01\\
221.01	0.01\\
222.01	0.01\\
223.01	0.01\\
224.01	0.01\\
225.01	0.01\\
226.01	0.01\\
227.01	0.01\\
228.01	0.01\\
229.01	0.01\\
230.01	0.01\\
231.01	0.01\\
232.01	0.01\\
233.01	0.01\\
234.01	0.01\\
235.01	0.01\\
236.01	0.01\\
237.01	0.01\\
238.01	0.01\\
239.01	0.01\\
240.01	0.01\\
241.01	0.01\\
242.01	0.01\\
243.01	0.01\\
244.01	0.01\\
245.01	0.01\\
246.01	0.01\\
247.01	0.01\\
248.01	0.01\\
249.01	0.01\\
250.01	0.01\\
251.01	0.01\\
252.01	0.01\\
253.01	0.01\\
254.01	0.01\\
255.01	0.01\\
256.01	0.01\\
257.01	0.01\\
258.01	0.01\\
259.01	0.01\\
260.01	0.01\\
261.01	0.01\\
262.01	0.01\\
263.01	0.01\\
264.01	0.01\\
265.01	0.01\\
266.01	0.01\\
267.01	0.01\\
268.01	0.01\\
269.01	0.01\\
270.01	0.01\\
271.01	0.01\\
272.01	0.01\\
273.01	0.01\\
274.01	0.01\\
275.01	0.01\\
276.01	0.01\\
277.01	0.01\\
278.01	0.01\\
279.01	0.01\\
280.01	0.01\\
281.01	0.01\\
282.01	0.01\\
283.01	0.01\\
284.01	0.01\\
285.01	0.01\\
286.01	0.01\\
287.01	0.01\\
288.01	0.01\\
289.01	0.01\\
290.01	0.01\\
291.01	0.01\\
292.01	0.01\\
293.01	0.01\\
294.01	0.01\\
295.01	0.01\\
296.01	0.01\\
297.01	0.01\\
298.01	0.01\\
299.01	0.01\\
300.01	0.01\\
301.01	0.01\\
302.01	0.01\\
303.01	0.01\\
304.01	0.01\\
305.01	0.01\\
306.01	0.01\\
307.01	0.01\\
308.01	0.01\\
309.01	0.01\\
310.01	0.01\\
311.01	0.01\\
312.01	0.01\\
313.01	0.01\\
314.01	0.01\\
315.01	0.01\\
316.01	0.01\\
317.01	0.01\\
318.01	0.01\\
319.01	0.01\\
320.01	0.01\\
321.01	0.01\\
322.01	0.01\\
323.01	0.01\\
324.01	0.01\\
325.01	0.01\\
326.01	0.01\\
327.01	0.01\\
328.01	0.01\\
329.01	0.01\\
330.01	0.01\\
331.01	0.01\\
332.01	0.01\\
333.01	0.01\\
334.01	0.01\\
335.01	0.01\\
336.01	0.01\\
337.01	0.01\\
338.01	0.01\\
339.01	0.01\\
340.01	0.01\\
341.01	0.01\\
342.01	0.01\\
343.01	0.01\\
344.01	0.01\\
345.01	0.01\\
346.01	0.01\\
347.01	0.01\\
348.01	0.01\\
349.01	0.01\\
350.01	0.01\\
351.01	0.01\\
352.01	0.01\\
353.01	0.01\\
354.01	0.01\\
355.01	0.01\\
356.01	0.01\\
357.01	0.01\\
358.01	0.01\\
359.01	0.01\\
360.01	0.01\\
361.01	0.01\\
362.01	0.01\\
363.01	0.01\\
364.01	0.01\\
365.01	0.01\\
366.01	0.01\\
367.01	0.01\\
368.01	0.01\\
369.01	0.01\\
370.01	0.01\\
371.01	0.01\\
372.01	0.01\\
373.01	0.01\\
374.01	0.01\\
375.01	0.01\\
376.01	0.01\\
377.01	0.01\\
378.01	0.01\\
379.01	0.01\\
380.01	0.01\\
381.01	0.01\\
382.01	0.01\\
383.01	0.01\\
384.01	0.01\\
385.01	0.01\\
386.01	0.01\\
387.01	0.01\\
388.01	0.01\\
389.01	0.01\\
390.01	0.01\\
391.01	0.01\\
392.01	0.01\\
393.01	0.01\\
394.01	0.01\\
395.01	0.01\\
396.01	0.01\\
397.01	0.01\\
398.01	0.01\\
399.01	0.01\\
400.01	0.01\\
401.01	0.01\\
402.01	0.01\\
403.01	0.01\\
404.01	0.01\\
405.01	0.01\\
406.01	0.01\\
407.01	0.01\\
408.01	0.01\\
409.01	0.01\\
410.01	0.01\\
411.01	0.01\\
412.01	0.01\\
413.01	0.01\\
414.01	0.01\\
415.01	0.01\\
416.01	0.01\\
417.01	0.01\\
418.01	0.01\\
419.01	0.01\\
420.01	0.01\\
421.01	0.01\\
422.01	0.01\\
423.01	0.01\\
424.01	0.01\\
425.01	0.01\\
426.01	0.01\\
427.01	0.01\\
428.01	0.01\\
429.01	0.01\\
430.01	0.01\\
431.01	0.01\\
432.01	0.01\\
433.01	0.01\\
434.01	0.01\\
435.01	0.01\\
436.01	0.01\\
437.01	0.01\\
438.01	0.01\\
439.01	0.01\\
440.01	0.01\\
441.01	0.01\\
442.01	0.01\\
443.01	0.01\\
444.01	0.01\\
445.01	0.01\\
446.01	0.01\\
447.01	0.01\\
448.01	0.01\\
449.01	0.01\\
450.01	0.01\\
451.01	0.01\\
452.01	0.01\\
453.01	0.01\\
454.01	0.01\\
455.01	0.01\\
456.01	0.01\\
457.01	0.01\\
458.01	0.01\\
459.01	0.01\\
460.01	0.01\\
461.01	0.01\\
462.01	0.01\\
463.01	0.01\\
464.01	0.01\\
465.01	0.01\\
466.01	0.01\\
467.01	0.01\\
468.01	0.01\\
469.01	0.01\\
470.01	0.01\\
471.01	0.01\\
472.01	0.01\\
473.01	0.01\\
474.01	0.01\\
475.01	0.01\\
476.01	0.01\\
477.01	0.01\\
478.01	0.01\\
479.01	0.01\\
480.01	0.01\\
481.01	0.01\\
482.01	0.01\\
483.01	0.01\\
484.01	0.01\\
485.01	0.01\\
486.01	0.01\\
487.01	0.01\\
488.01	0.01\\
489.01	0.01\\
490.01	0.01\\
491.01	0.01\\
492.01	0.01\\
493.01	0.01\\
494.01	0.01\\
495.01	0.01\\
496.01	0.01\\
497.01	0.01\\
498.01	0.01\\
499.01	0.01\\
500.01	0.01\\
501.01	0.01\\
502.01	0.01\\
503.01	0.01\\
504.01	0.01\\
505.01	0.01\\
506.01	0.01\\
507.01	0.01\\
508.01	0.01\\
509.01	0.01\\
510.01	0.01\\
511.01	0.01\\
512.01	0.01\\
513.01	0.01\\
514.01	0.01\\
515.01	0.01\\
516.01	0.01\\
517.01	0.01\\
518.01	0.01\\
519.01	0.01\\
520.01	0.01\\
521.01	0.01\\
522.01	0.01\\
523.01	0.01\\
524.01	0.01\\
525.01	0.01\\
526.01	0.01\\
527.01	0.01\\
528.01	0.01\\
529.01	0.01\\
530.01	0.01\\
531.01	0.01\\
532.01	0.01\\
533.01	0.01\\
534.01	0.01\\
535.01	0.01\\
536.01	0.01\\
537.01	0.01\\
538.01	0.01\\
539.01	0.01\\
540.01	0.01\\
541.01	0.01\\
542.01	0.01\\
543.01	0.01\\
544.01	0.01\\
545.01	0.01\\
546.01	0.01\\
547.01	0.01\\
548.01	0.01\\
549.01	0.01\\
550.01	0.01\\
551.01	0.01\\
552.01	0.01\\
553.01	0.01\\
554.01	0.01\\
555.01	0.01\\
556.01	0.01\\
557.01	0.01\\
558.01	0.01\\
559.01	0.01\\
560.01	0.01\\
561.01	0.01\\
562.01	0.01\\
563.01	0.01\\
564.01	0.01\\
565.01	0.01\\
566.01	0.01\\
567.01	0.01\\
568.01	0.01\\
569.01	0.01\\
570.01	0.01\\
571.01	0.01\\
572.01	0.01\\
573.01	0.01\\
574.01	0.01\\
575.01	0.01\\
576.01	0.01\\
577.01	0.01\\
578.01	0.01\\
579.01	0.01\\
580.01	0.01\\
581.01	0.01\\
582.01	0.01\\
583.01	0.01\\
584.01	0.01\\
585.01	0.01\\
586.01	0.01\\
587.01	0.01\\
588.01	0.01\\
589.01	0.01\\
590.01	0.01\\
591.01	0.01\\
592.01	0.01\\
593.01	0.01\\
594.01	0.01\\
595.01	0.01\\
596.01	0.01\\
597.01	0.01\\
598.01	0.01\\
599.01	0.00623515239359194\\
599.02	0.00619748725350673\\
599.03	0.00615945574907279\\
599.04	0.00612105427627316\\
599.05	0.00608227919580345\\
599.06	0.00604312683265317\\
599.07	0.00600359347579347\\
599.08	0.00596367537790493\\
599.09	0.00592336866516034\\
599.1	0.00588266927971629\\
599.11	0.00584157335142531\\
599.12	0.00580007697221952\\
599.13	0.00575817619584629\\
599.14	0.00571586703761439\\
599.15	0.00567314547415175\\
599.16	0.00563000744317626\\
599.17	0.00558644884328414\\
599.18	0.00554246553533496\\
599.19	0.00549805334084668\\
599.2	0.00545320804184588\\
599.21	0.0054079253798647\\
599.22	0.00536220105456069\\
599.23	0.00531603072330621\\
599.24	0.005269410000774\\
599.25	0.0052223344585187\\
599.26	0.00517479962455485\\
599.27	0.00512680098293079\\
599.28	0.00507833397329517\\
599.29	0.0050293939904593\\
599.3	0.00497997638395778\\
599.31	0.00493007645760491\\
599.32	0.00487968946904662\\
599.33	0.00482881062930801\\
599.34	0.00477743510233639\\
599.35	0.00472555800453965\\
599.36	0.00467317440432004\\
599.37	0.00462027932160314\\
599.38	0.00456686772736205\\
599.39	0.00451293454313655\\
599.4	0.00445847464054731\\
599.41	0.00440348284687955\\
599.42	0.00434795394983202\\
599.43	0.00429188268603743\\
599.44	0.00423526374056028\\
599.45	0.00417809174638914\\
599.46	0.00412036128392308\\
599.47	0.00406206688045222\\
599.48	0.00400320300963207\\
599.49	0.00394376409095614\\
599.5	0.00388374448922514\\
599.51	0.00382313851401101\\
599.52	0.00376194041911571\\
599.53	0.00370014440202472\\
599.54	0.00363774460335511\\
599.55	0.00357473510629834\\
599.56	0.00351110993605749\\
599.57	0.00344686305927907\\
599.58	0.00338198838347919\\
599.59	0.00331647975646418\\
599.6	0.00325033096574548\\
599.61	0.00318353573794888\\
599.62	0.00311608773821786\\
599.63	0.00304798056961123\\
599.64	0.00297920777249475\\
599.65	0.00290976282392687\\
599.66	0.0028396391370384\\
599.67	0.00276883006040616\\
599.68	0.00269732887742047\\
599.69	0.00262512880564641\\
599.7	0.00255222299617888\\
599.71	0.00247860453299131\\
599.72	0.00240426643227798\\
599.73	0.00232920164178993\\
599.74	0.00225340304016429\\
599.75	0.00217686343624712\\
599.76	0.00209957556840957\\
599.77	0.00202153210385725\\
599.78	0.00194272563793302\\
599.79	0.00186314869341269\\
599.8	0.00178279371979404\\
599.81	0.00170165309257868\\
599.82	0.00161971911254694\\
599.83	0.00153698400502566\\
599.84	0.0014534399191487\\
599.85	0.00136907892711024\\
599.86	0.00128389302341077\\
599.87	0.00119787412409557\\
599.88	0.00111101406598581\\
599.89	0.001023304605902\\
599.9	0.000934737419879901\\
599.91	0.000845304102378633\\
599.92	0.000754996165481069\\
599.93	0.000663805038086366\\
599.94	0.00057172206509455\\
599.95	0.000478738506583127\\
599.96	0.000384845536975638\\
599.97	0.000290034244202044\\
599.98	0.00019429562885097\\
599.99	9.76206033136556e-05\\
600	0\\
};
\addplot [color=blue!80!mycolor9,solid,forget plot]
  table[row sep=crcr]{%
0.01	0.01\\
1.01	0.01\\
2.01	0.01\\
3.01	0.01\\
4.01	0.01\\
5.01	0.01\\
6.01	0.01\\
7.01	0.01\\
8.01	0.01\\
9.01	0.01\\
10.01	0.01\\
11.01	0.01\\
12.01	0.01\\
13.01	0.01\\
14.01	0.01\\
15.01	0.01\\
16.01	0.01\\
17.01	0.01\\
18.01	0.01\\
19.01	0.01\\
20.01	0.01\\
21.01	0.01\\
22.01	0.01\\
23.01	0.01\\
24.01	0.01\\
25.01	0.01\\
26.01	0.01\\
27.01	0.01\\
28.01	0.01\\
29.01	0.01\\
30.01	0.01\\
31.01	0.01\\
32.01	0.01\\
33.01	0.01\\
34.01	0.01\\
35.01	0.01\\
36.01	0.01\\
37.01	0.01\\
38.01	0.01\\
39.01	0.01\\
40.01	0.01\\
41.01	0.01\\
42.01	0.01\\
43.01	0.01\\
44.01	0.01\\
45.01	0.01\\
46.01	0.01\\
47.01	0.01\\
48.01	0.01\\
49.01	0.01\\
50.01	0.01\\
51.01	0.01\\
52.01	0.01\\
53.01	0.01\\
54.01	0.01\\
55.01	0.01\\
56.01	0.01\\
57.01	0.01\\
58.01	0.01\\
59.01	0.01\\
60.01	0.01\\
61.01	0.01\\
62.01	0.01\\
63.01	0.01\\
64.01	0.01\\
65.01	0.01\\
66.01	0.01\\
67.01	0.01\\
68.01	0.01\\
69.01	0.01\\
70.01	0.01\\
71.01	0.01\\
72.01	0.01\\
73.01	0.01\\
74.01	0.01\\
75.01	0.01\\
76.01	0.01\\
77.01	0.01\\
78.01	0.01\\
79.01	0.01\\
80.01	0.01\\
81.01	0.01\\
82.01	0.01\\
83.01	0.01\\
84.01	0.01\\
85.01	0.01\\
86.01	0.01\\
87.01	0.01\\
88.01	0.01\\
89.01	0.01\\
90.01	0.01\\
91.01	0.01\\
92.01	0.01\\
93.01	0.01\\
94.01	0.01\\
95.01	0.01\\
96.01	0.01\\
97.01	0.01\\
98.01	0.01\\
99.01	0.01\\
100.01	0.01\\
101.01	0.01\\
102.01	0.01\\
103.01	0.01\\
104.01	0.01\\
105.01	0.01\\
106.01	0.01\\
107.01	0.01\\
108.01	0.01\\
109.01	0.01\\
110.01	0.01\\
111.01	0.01\\
112.01	0.01\\
113.01	0.01\\
114.01	0.01\\
115.01	0.01\\
116.01	0.01\\
117.01	0.01\\
118.01	0.01\\
119.01	0.01\\
120.01	0.01\\
121.01	0.01\\
122.01	0.01\\
123.01	0.01\\
124.01	0.01\\
125.01	0.01\\
126.01	0.01\\
127.01	0.01\\
128.01	0.01\\
129.01	0.01\\
130.01	0.01\\
131.01	0.01\\
132.01	0.01\\
133.01	0.01\\
134.01	0.01\\
135.01	0.01\\
136.01	0.01\\
137.01	0.01\\
138.01	0.01\\
139.01	0.01\\
140.01	0.01\\
141.01	0.01\\
142.01	0.01\\
143.01	0.01\\
144.01	0.01\\
145.01	0.01\\
146.01	0.01\\
147.01	0.01\\
148.01	0.01\\
149.01	0.01\\
150.01	0.01\\
151.01	0.01\\
152.01	0.01\\
153.01	0.01\\
154.01	0.01\\
155.01	0.01\\
156.01	0.01\\
157.01	0.01\\
158.01	0.01\\
159.01	0.01\\
160.01	0.01\\
161.01	0.01\\
162.01	0.01\\
163.01	0.01\\
164.01	0.01\\
165.01	0.01\\
166.01	0.01\\
167.01	0.01\\
168.01	0.01\\
169.01	0.01\\
170.01	0.01\\
171.01	0.01\\
172.01	0.01\\
173.01	0.01\\
174.01	0.01\\
175.01	0.01\\
176.01	0.01\\
177.01	0.01\\
178.01	0.01\\
179.01	0.01\\
180.01	0.01\\
181.01	0.01\\
182.01	0.01\\
183.01	0.01\\
184.01	0.01\\
185.01	0.01\\
186.01	0.01\\
187.01	0.01\\
188.01	0.01\\
189.01	0.01\\
190.01	0.01\\
191.01	0.01\\
192.01	0.01\\
193.01	0.01\\
194.01	0.01\\
195.01	0.01\\
196.01	0.01\\
197.01	0.01\\
198.01	0.01\\
199.01	0.01\\
200.01	0.01\\
201.01	0.01\\
202.01	0.01\\
203.01	0.01\\
204.01	0.01\\
205.01	0.01\\
206.01	0.01\\
207.01	0.01\\
208.01	0.01\\
209.01	0.01\\
210.01	0.01\\
211.01	0.01\\
212.01	0.01\\
213.01	0.01\\
214.01	0.01\\
215.01	0.01\\
216.01	0.01\\
217.01	0.01\\
218.01	0.01\\
219.01	0.01\\
220.01	0.01\\
221.01	0.01\\
222.01	0.01\\
223.01	0.01\\
224.01	0.01\\
225.01	0.01\\
226.01	0.01\\
227.01	0.01\\
228.01	0.01\\
229.01	0.01\\
230.01	0.01\\
231.01	0.01\\
232.01	0.01\\
233.01	0.01\\
234.01	0.01\\
235.01	0.01\\
236.01	0.01\\
237.01	0.01\\
238.01	0.01\\
239.01	0.01\\
240.01	0.01\\
241.01	0.01\\
242.01	0.01\\
243.01	0.01\\
244.01	0.01\\
245.01	0.01\\
246.01	0.01\\
247.01	0.01\\
248.01	0.01\\
249.01	0.01\\
250.01	0.01\\
251.01	0.01\\
252.01	0.01\\
253.01	0.01\\
254.01	0.01\\
255.01	0.01\\
256.01	0.01\\
257.01	0.01\\
258.01	0.01\\
259.01	0.01\\
260.01	0.01\\
261.01	0.01\\
262.01	0.01\\
263.01	0.01\\
264.01	0.01\\
265.01	0.01\\
266.01	0.01\\
267.01	0.01\\
268.01	0.01\\
269.01	0.01\\
270.01	0.01\\
271.01	0.01\\
272.01	0.01\\
273.01	0.01\\
274.01	0.01\\
275.01	0.01\\
276.01	0.01\\
277.01	0.01\\
278.01	0.01\\
279.01	0.01\\
280.01	0.01\\
281.01	0.01\\
282.01	0.01\\
283.01	0.01\\
284.01	0.01\\
285.01	0.01\\
286.01	0.01\\
287.01	0.01\\
288.01	0.01\\
289.01	0.01\\
290.01	0.01\\
291.01	0.01\\
292.01	0.01\\
293.01	0.01\\
294.01	0.01\\
295.01	0.01\\
296.01	0.01\\
297.01	0.01\\
298.01	0.01\\
299.01	0.01\\
300.01	0.01\\
301.01	0.01\\
302.01	0.01\\
303.01	0.01\\
304.01	0.01\\
305.01	0.01\\
306.01	0.01\\
307.01	0.01\\
308.01	0.01\\
309.01	0.01\\
310.01	0.01\\
311.01	0.01\\
312.01	0.01\\
313.01	0.01\\
314.01	0.01\\
315.01	0.01\\
316.01	0.01\\
317.01	0.01\\
318.01	0.01\\
319.01	0.01\\
320.01	0.01\\
321.01	0.01\\
322.01	0.01\\
323.01	0.01\\
324.01	0.01\\
325.01	0.01\\
326.01	0.01\\
327.01	0.01\\
328.01	0.01\\
329.01	0.01\\
330.01	0.01\\
331.01	0.01\\
332.01	0.01\\
333.01	0.01\\
334.01	0.01\\
335.01	0.01\\
336.01	0.01\\
337.01	0.01\\
338.01	0.01\\
339.01	0.01\\
340.01	0.01\\
341.01	0.01\\
342.01	0.01\\
343.01	0.01\\
344.01	0.01\\
345.01	0.01\\
346.01	0.01\\
347.01	0.01\\
348.01	0.01\\
349.01	0.01\\
350.01	0.01\\
351.01	0.01\\
352.01	0.01\\
353.01	0.01\\
354.01	0.01\\
355.01	0.01\\
356.01	0.01\\
357.01	0.01\\
358.01	0.01\\
359.01	0.01\\
360.01	0.01\\
361.01	0.01\\
362.01	0.01\\
363.01	0.01\\
364.01	0.01\\
365.01	0.01\\
366.01	0.01\\
367.01	0.01\\
368.01	0.01\\
369.01	0.01\\
370.01	0.01\\
371.01	0.01\\
372.01	0.01\\
373.01	0.01\\
374.01	0.01\\
375.01	0.01\\
376.01	0.01\\
377.01	0.01\\
378.01	0.01\\
379.01	0.01\\
380.01	0.01\\
381.01	0.01\\
382.01	0.01\\
383.01	0.01\\
384.01	0.01\\
385.01	0.01\\
386.01	0.01\\
387.01	0.01\\
388.01	0.01\\
389.01	0.01\\
390.01	0.01\\
391.01	0.01\\
392.01	0.01\\
393.01	0.01\\
394.01	0.01\\
395.01	0.01\\
396.01	0.01\\
397.01	0.01\\
398.01	0.01\\
399.01	0.01\\
400.01	0.01\\
401.01	0.01\\
402.01	0.01\\
403.01	0.01\\
404.01	0.01\\
405.01	0.01\\
406.01	0.01\\
407.01	0.01\\
408.01	0.01\\
409.01	0.01\\
410.01	0.01\\
411.01	0.01\\
412.01	0.01\\
413.01	0.01\\
414.01	0.01\\
415.01	0.01\\
416.01	0.01\\
417.01	0.01\\
418.01	0.01\\
419.01	0.01\\
420.01	0.01\\
421.01	0.01\\
422.01	0.01\\
423.01	0.01\\
424.01	0.01\\
425.01	0.01\\
426.01	0.01\\
427.01	0.01\\
428.01	0.01\\
429.01	0.01\\
430.01	0.01\\
431.01	0.01\\
432.01	0.01\\
433.01	0.01\\
434.01	0.01\\
435.01	0.01\\
436.01	0.01\\
437.01	0.01\\
438.01	0.01\\
439.01	0.01\\
440.01	0.01\\
441.01	0.01\\
442.01	0.01\\
443.01	0.01\\
444.01	0.01\\
445.01	0.01\\
446.01	0.01\\
447.01	0.01\\
448.01	0.01\\
449.01	0.01\\
450.01	0.01\\
451.01	0.01\\
452.01	0.01\\
453.01	0.01\\
454.01	0.01\\
455.01	0.01\\
456.01	0.01\\
457.01	0.01\\
458.01	0.01\\
459.01	0.01\\
460.01	0.01\\
461.01	0.01\\
462.01	0.01\\
463.01	0.01\\
464.01	0.01\\
465.01	0.01\\
466.01	0.01\\
467.01	0.01\\
468.01	0.01\\
469.01	0.01\\
470.01	0.01\\
471.01	0.01\\
472.01	0.01\\
473.01	0.01\\
474.01	0.01\\
475.01	0.01\\
476.01	0.01\\
477.01	0.01\\
478.01	0.01\\
479.01	0.01\\
480.01	0.01\\
481.01	0.01\\
482.01	0.01\\
483.01	0.01\\
484.01	0.01\\
485.01	0.01\\
486.01	0.01\\
487.01	0.01\\
488.01	0.01\\
489.01	0.01\\
490.01	0.01\\
491.01	0.01\\
492.01	0.01\\
493.01	0.01\\
494.01	0.01\\
495.01	0.01\\
496.01	0.01\\
497.01	0.01\\
498.01	0.01\\
499.01	0.01\\
500.01	0.01\\
501.01	0.01\\
502.01	0.01\\
503.01	0.01\\
504.01	0.01\\
505.01	0.01\\
506.01	0.01\\
507.01	0.01\\
508.01	0.01\\
509.01	0.01\\
510.01	0.01\\
511.01	0.01\\
512.01	0.01\\
513.01	0.01\\
514.01	0.01\\
515.01	0.01\\
516.01	0.01\\
517.01	0.01\\
518.01	0.01\\
519.01	0.01\\
520.01	0.01\\
521.01	0.01\\
522.01	0.01\\
523.01	0.01\\
524.01	0.01\\
525.01	0.01\\
526.01	0.01\\
527.01	0.01\\
528.01	0.01\\
529.01	0.01\\
530.01	0.01\\
531.01	0.01\\
532.01	0.01\\
533.01	0.01\\
534.01	0.01\\
535.01	0.01\\
536.01	0.01\\
537.01	0.01\\
538.01	0.01\\
539.01	0.01\\
540.01	0.01\\
541.01	0.01\\
542.01	0.01\\
543.01	0.01\\
544.01	0.01\\
545.01	0.01\\
546.01	0.01\\
547.01	0.01\\
548.01	0.01\\
549.01	0.01\\
550.01	0.01\\
551.01	0.01\\
552.01	0.01\\
553.01	0.01\\
554.01	0.01\\
555.01	0.01\\
556.01	0.01\\
557.01	0.01\\
558.01	0.01\\
559.01	0.01\\
560.01	0.01\\
561.01	0.01\\
562.01	0.01\\
563.01	0.01\\
564.01	0.01\\
565.01	0.01\\
566.01	0.01\\
567.01	0.01\\
568.01	0.01\\
569.01	0.01\\
570.01	0.01\\
571.01	0.01\\
572.01	0.01\\
573.01	0.01\\
574.01	0.01\\
575.01	0.01\\
576.01	0.01\\
577.01	0.01\\
578.01	0.01\\
579.01	0.01\\
580.01	0.01\\
581.01	0.01\\
582.01	0.01\\
583.01	0.01\\
584.01	0.01\\
585.01	0.01\\
586.01	0.01\\
587.01	0.01\\
588.01	0.01\\
589.01	0.01\\
590.01	0.01\\
591.01	0.01\\
592.01	0.01\\
593.01	0.01\\
594.01	0.01\\
595.01	0.01\\
596.01	0.01\\
597.01	0.01\\
598.01	0.01\\
599.01	0.00623588074663459\\
599.02	0.00619813582657612\\
599.03	0.00616002993582383\\
599.04	0.00612155943362391\\
599.05	0.00608272061297158\\
599.06	0.00604350974526074\\
599.07	0.00600392304473869\\
599.08	0.00596395665477922\\
599.09	0.00592360675084337\\
599.1	0.00588286958222237\\
599.11	0.0058417410941521\\
599.12	0.00580021715091981\\
599.13	0.00575829352900724\\
599.14	0.00571596591257865\\
599.15	0.0056732298887007\\
599.16	0.00563008094227915\\
599.17	0.0055865144500541\\
599.18	0.0055425253800716\\
599.19	0.00549810886649928\\
599.2	0.00545325991747207\\
599.21	0.00540797380783162\\
599.22	0.00536224623367933\\
599.23	0.005316072848296\\
599.24	0.00526944926174205\\
599.25	0.00522237104041281\\
599.26	0.00517483370647011\\
599.27	0.00512683273741956\\
599.28	0.00507836356645391\\
599.29	0.00502942158197927\\
599.3	0.00498000212655932\\
599.31	0.00493010049649361\\
599.32	0.00487971194140611\\
599.33	0.00482883166383815\\
599.34	0.00477745481884658\\
599.35	0.00472557651360802\\
599.36	0.00467319180703039\\
599.37	0.00462029570937272\\
599.38	0.00456688318187443\\
599.39	0.00451294913639552\\
599.4	0.00445848843506895\\
599.41	0.0044034958960355\\
599.42	0.00434796629837901\\
599.43	0.00429189437083759\\
599.44	0.00423527479152567\\
599.45	0.00417810218767722\\
599.46	0.00412037113541252\\
599.47	0.00406207615953092\\
599.48	0.00400321173333205\\
599.49	0.00394377227665567\\
599.5	0.00388375215461717\\
599.51	0.00382314567706917\\
599.52	0.00376194709805824\\
599.53	0.00370015061527655\\
599.54	0.00363775036950859\\
599.55	0.00357474044407246\\
599.56	0.00351111486425591\\
599.57	0.00344686759674706\\
599.58	0.00338199254905957\\
599.59	0.00331648356895239\\
599.6	0.00325033444384401\\
599.61	0.00318353890022104\\
599.62	0.0031160906030414\\
599.63	0.00304798315513165\\
599.64	0.00297921009657883\\
599.65	0.00290976490411649\\
599.66	0.00283964099050497\\
599.67	0.00276883170390586\\
599.68	0.00269733032725062\\
599.69	0.00262513007760331\\
599.7	0.00255222410551739\\
599.71	0.00247860549438648\\
599.72	0.0024042672597891\\
599.73	0.00232920234882737\\
599.74	0.00225340363945949\\
599.75	0.00217686393982603\\
599.76	0.00209957598757\\
599.77	0.00202153244915048\\
599.78	0.00194272591915\\
599.79	0.00186314891957526\\
599.8	0.00178279389915142\\
599.81	0.00170165323260962\\
599.82	0.00161971921996775\\
599.83	0.00153698408580431\\
599.84	0.00145343997852515\\
599.85	0.00136907896962321\\
599.86	0.00128389305293063\\
599.87	0.00119787414386362\\
599.88	0.00111101407865932\\
599.89	0.00102330461360488\\
599.9	0.000934737424258189\\
599.91	0.000845304104660138\\
599.92	0.000754996166537987\\
599.93	0.00066380503849954\\
599.94	0.000571722065217627\\
599.95	0.000478738506604516\\
599.96	0.000384845536975636\\
599.97	0.000290034244202044\\
599.98	0.00019429562885097\\
599.99	9.76206033136556e-05\\
600	0\\
};
\addplot [color=blue,solid,forget plot]
  table[row sep=crcr]{%
0.01	0.01\\
1.01	0.01\\
2.01	0.01\\
3.01	0.01\\
4.01	0.01\\
5.01	0.01\\
6.01	0.01\\
7.01	0.01\\
8.01	0.01\\
9.01	0.01\\
10.01	0.01\\
11.01	0.01\\
12.01	0.01\\
13.01	0.01\\
14.01	0.01\\
15.01	0.01\\
16.01	0.01\\
17.01	0.01\\
18.01	0.01\\
19.01	0.01\\
20.01	0.01\\
21.01	0.01\\
22.01	0.01\\
23.01	0.01\\
24.01	0.01\\
25.01	0.01\\
26.01	0.01\\
27.01	0.01\\
28.01	0.01\\
29.01	0.01\\
30.01	0.01\\
31.01	0.01\\
32.01	0.01\\
33.01	0.01\\
34.01	0.01\\
35.01	0.01\\
36.01	0.01\\
37.01	0.01\\
38.01	0.01\\
39.01	0.01\\
40.01	0.01\\
41.01	0.01\\
42.01	0.01\\
43.01	0.01\\
44.01	0.01\\
45.01	0.01\\
46.01	0.01\\
47.01	0.01\\
48.01	0.01\\
49.01	0.01\\
50.01	0.01\\
51.01	0.01\\
52.01	0.01\\
53.01	0.01\\
54.01	0.01\\
55.01	0.01\\
56.01	0.01\\
57.01	0.01\\
58.01	0.01\\
59.01	0.01\\
60.01	0.01\\
61.01	0.01\\
62.01	0.01\\
63.01	0.01\\
64.01	0.01\\
65.01	0.01\\
66.01	0.01\\
67.01	0.01\\
68.01	0.01\\
69.01	0.01\\
70.01	0.01\\
71.01	0.01\\
72.01	0.01\\
73.01	0.01\\
74.01	0.01\\
75.01	0.01\\
76.01	0.01\\
77.01	0.01\\
78.01	0.01\\
79.01	0.01\\
80.01	0.01\\
81.01	0.01\\
82.01	0.01\\
83.01	0.01\\
84.01	0.01\\
85.01	0.01\\
86.01	0.01\\
87.01	0.01\\
88.01	0.01\\
89.01	0.01\\
90.01	0.01\\
91.01	0.01\\
92.01	0.01\\
93.01	0.01\\
94.01	0.01\\
95.01	0.01\\
96.01	0.01\\
97.01	0.01\\
98.01	0.01\\
99.01	0.01\\
100.01	0.01\\
101.01	0.01\\
102.01	0.01\\
103.01	0.01\\
104.01	0.01\\
105.01	0.01\\
106.01	0.01\\
107.01	0.01\\
108.01	0.01\\
109.01	0.01\\
110.01	0.01\\
111.01	0.01\\
112.01	0.01\\
113.01	0.01\\
114.01	0.01\\
115.01	0.01\\
116.01	0.01\\
117.01	0.01\\
118.01	0.01\\
119.01	0.01\\
120.01	0.01\\
121.01	0.01\\
122.01	0.01\\
123.01	0.01\\
124.01	0.01\\
125.01	0.01\\
126.01	0.01\\
127.01	0.01\\
128.01	0.01\\
129.01	0.01\\
130.01	0.01\\
131.01	0.01\\
132.01	0.01\\
133.01	0.01\\
134.01	0.01\\
135.01	0.01\\
136.01	0.01\\
137.01	0.01\\
138.01	0.01\\
139.01	0.01\\
140.01	0.01\\
141.01	0.01\\
142.01	0.01\\
143.01	0.01\\
144.01	0.01\\
145.01	0.01\\
146.01	0.01\\
147.01	0.01\\
148.01	0.01\\
149.01	0.01\\
150.01	0.01\\
151.01	0.01\\
152.01	0.01\\
153.01	0.01\\
154.01	0.01\\
155.01	0.01\\
156.01	0.01\\
157.01	0.01\\
158.01	0.01\\
159.01	0.01\\
160.01	0.01\\
161.01	0.01\\
162.01	0.01\\
163.01	0.01\\
164.01	0.01\\
165.01	0.01\\
166.01	0.01\\
167.01	0.01\\
168.01	0.01\\
169.01	0.01\\
170.01	0.01\\
171.01	0.01\\
172.01	0.01\\
173.01	0.01\\
174.01	0.01\\
175.01	0.01\\
176.01	0.01\\
177.01	0.01\\
178.01	0.01\\
179.01	0.01\\
180.01	0.01\\
181.01	0.01\\
182.01	0.01\\
183.01	0.01\\
184.01	0.01\\
185.01	0.01\\
186.01	0.01\\
187.01	0.01\\
188.01	0.01\\
189.01	0.01\\
190.01	0.01\\
191.01	0.01\\
192.01	0.01\\
193.01	0.01\\
194.01	0.01\\
195.01	0.01\\
196.01	0.01\\
197.01	0.01\\
198.01	0.01\\
199.01	0.01\\
200.01	0.01\\
201.01	0.01\\
202.01	0.01\\
203.01	0.01\\
204.01	0.01\\
205.01	0.01\\
206.01	0.01\\
207.01	0.01\\
208.01	0.01\\
209.01	0.01\\
210.01	0.01\\
211.01	0.01\\
212.01	0.01\\
213.01	0.01\\
214.01	0.01\\
215.01	0.01\\
216.01	0.01\\
217.01	0.01\\
218.01	0.01\\
219.01	0.01\\
220.01	0.01\\
221.01	0.01\\
222.01	0.01\\
223.01	0.01\\
224.01	0.01\\
225.01	0.01\\
226.01	0.01\\
227.01	0.01\\
228.01	0.01\\
229.01	0.01\\
230.01	0.01\\
231.01	0.01\\
232.01	0.01\\
233.01	0.01\\
234.01	0.01\\
235.01	0.01\\
236.01	0.01\\
237.01	0.01\\
238.01	0.01\\
239.01	0.01\\
240.01	0.01\\
241.01	0.01\\
242.01	0.01\\
243.01	0.01\\
244.01	0.01\\
245.01	0.01\\
246.01	0.01\\
247.01	0.01\\
248.01	0.01\\
249.01	0.01\\
250.01	0.01\\
251.01	0.01\\
252.01	0.01\\
253.01	0.01\\
254.01	0.01\\
255.01	0.01\\
256.01	0.01\\
257.01	0.01\\
258.01	0.01\\
259.01	0.01\\
260.01	0.01\\
261.01	0.01\\
262.01	0.01\\
263.01	0.01\\
264.01	0.01\\
265.01	0.01\\
266.01	0.01\\
267.01	0.01\\
268.01	0.01\\
269.01	0.01\\
270.01	0.01\\
271.01	0.01\\
272.01	0.01\\
273.01	0.01\\
274.01	0.01\\
275.01	0.01\\
276.01	0.01\\
277.01	0.01\\
278.01	0.01\\
279.01	0.01\\
280.01	0.01\\
281.01	0.01\\
282.01	0.01\\
283.01	0.01\\
284.01	0.01\\
285.01	0.01\\
286.01	0.01\\
287.01	0.01\\
288.01	0.01\\
289.01	0.01\\
290.01	0.01\\
291.01	0.01\\
292.01	0.01\\
293.01	0.01\\
294.01	0.01\\
295.01	0.01\\
296.01	0.01\\
297.01	0.01\\
298.01	0.01\\
299.01	0.01\\
300.01	0.01\\
301.01	0.01\\
302.01	0.01\\
303.01	0.01\\
304.01	0.01\\
305.01	0.01\\
306.01	0.01\\
307.01	0.01\\
308.01	0.01\\
309.01	0.01\\
310.01	0.01\\
311.01	0.01\\
312.01	0.01\\
313.01	0.01\\
314.01	0.01\\
315.01	0.01\\
316.01	0.01\\
317.01	0.01\\
318.01	0.01\\
319.01	0.01\\
320.01	0.01\\
321.01	0.01\\
322.01	0.01\\
323.01	0.01\\
324.01	0.01\\
325.01	0.01\\
326.01	0.01\\
327.01	0.01\\
328.01	0.01\\
329.01	0.01\\
330.01	0.01\\
331.01	0.01\\
332.01	0.01\\
333.01	0.01\\
334.01	0.01\\
335.01	0.01\\
336.01	0.01\\
337.01	0.01\\
338.01	0.01\\
339.01	0.01\\
340.01	0.01\\
341.01	0.01\\
342.01	0.01\\
343.01	0.01\\
344.01	0.01\\
345.01	0.01\\
346.01	0.01\\
347.01	0.01\\
348.01	0.01\\
349.01	0.01\\
350.01	0.01\\
351.01	0.01\\
352.01	0.01\\
353.01	0.01\\
354.01	0.01\\
355.01	0.01\\
356.01	0.01\\
357.01	0.01\\
358.01	0.01\\
359.01	0.01\\
360.01	0.01\\
361.01	0.01\\
362.01	0.01\\
363.01	0.01\\
364.01	0.01\\
365.01	0.01\\
366.01	0.01\\
367.01	0.01\\
368.01	0.01\\
369.01	0.01\\
370.01	0.01\\
371.01	0.01\\
372.01	0.01\\
373.01	0.01\\
374.01	0.01\\
375.01	0.01\\
376.01	0.01\\
377.01	0.01\\
378.01	0.01\\
379.01	0.01\\
380.01	0.01\\
381.01	0.01\\
382.01	0.01\\
383.01	0.01\\
384.01	0.01\\
385.01	0.01\\
386.01	0.01\\
387.01	0.01\\
388.01	0.01\\
389.01	0.01\\
390.01	0.01\\
391.01	0.01\\
392.01	0.01\\
393.01	0.01\\
394.01	0.01\\
395.01	0.01\\
396.01	0.01\\
397.01	0.01\\
398.01	0.01\\
399.01	0.01\\
400.01	0.01\\
401.01	0.01\\
402.01	0.01\\
403.01	0.01\\
404.01	0.01\\
405.01	0.01\\
406.01	0.01\\
407.01	0.01\\
408.01	0.01\\
409.01	0.01\\
410.01	0.01\\
411.01	0.01\\
412.01	0.01\\
413.01	0.01\\
414.01	0.01\\
415.01	0.01\\
416.01	0.01\\
417.01	0.01\\
418.01	0.01\\
419.01	0.01\\
420.01	0.01\\
421.01	0.01\\
422.01	0.01\\
423.01	0.01\\
424.01	0.01\\
425.01	0.01\\
426.01	0.01\\
427.01	0.01\\
428.01	0.01\\
429.01	0.01\\
430.01	0.01\\
431.01	0.01\\
432.01	0.01\\
433.01	0.01\\
434.01	0.01\\
435.01	0.01\\
436.01	0.01\\
437.01	0.01\\
438.01	0.01\\
439.01	0.01\\
440.01	0.01\\
441.01	0.01\\
442.01	0.01\\
443.01	0.01\\
444.01	0.01\\
445.01	0.01\\
446.01	0.01\\
447.01	0.01\\
448.01	0.01\\
449.01	0.01\\
450.01	0.01\\
451.01	0.01\\
452.01	0.01\\
453.01	0.01\\
454.01	0.01\\
455.01	0.01\\
456.01	0.01\\
457.01	0.01\\
458.01	0.01\\
459.01	0.01\\
460.01	0.01\\
461.01	0.01\\
462.01	0.01\\
463.01	0.01\\
464.01	0.01\\
465.01	0.01\\
466.01	0.01\\
467.01	0.01\\
468.01	0.01\\
469.01	0.01\\
470.01	0.01\\
471.01	0.01\\
472.01	0.01\\
473.01	0.01\\
474.01	0.01\\
475.01	0.01\\
476.01	0.01\\
477.01	0.01\\
478.01	0.01\\
479.01	0.01\\
480.01	0.01\\
481.01	0.01\\
482.01	0.01\\
483.01	0.01\\
484.01	0.01\\
485.01	0.01\\
486.01	0.01\\
487.01	0.01\\
488.01	0.01\\
489.01	0.01\\
490.01	0.01\\
491.01	0.01\\
492.01	0.01\\
493.01	0.01\\
494.01	0.01\\
495.01	0.01\\
496.01	0.01\\
497.01	0.01\\
498.01	0.01\\
499.01	0.01\\
500.01	0.01\\
501.01	0.01\\
502.01	0.01\\
503.01	0.01\\
504.01	0.01\\
505.01	0.01\\
506.01	0.01\\
507.01	0.01\\
508.01	0.01\\
509.01	0.01\\
510.01	0.01\\
511.01	0.01\\
512.01	0.01\\
513.01	0.01\\
514.01	0.01\\
515.01	0.01\\
516.01	0.01\\
517.01	0.01\\
518.01	0.01\\
519.01	0.01\\
520.01	0.01\\
521.01	0.01\\
522.01	0.01\\
523.01	0.01\\
524.01	0.01\\
525.01	0.01\\
526.01	0.01\\
527.01	0.01\\
528.01	0.01\\
529.01	0.01\\
530.01	0.01\\
531.01	0.01\\
532.01	0.01\\
533.01	0.01\\
534.01	0.01\\
535.01	0.01\\
536.01	0.01\\
537.01	0.01\\
538.01	0.01\\
539.01	0.01\\
540.01	0.01\\
541.01	0.01\\
542.01	0.01\\
543.01	0.01\\
544.01	0.01\\
545.01	0.01\\
546.01	0.01\\
547.01	0.01\\
548.01	0.01\\
549.01	0.01\\
550.01	0.01\\
551.01	0.01\\
552.01	0.01\\
553.01	0.01\\
554.01	0.01\\
555.01	0.01\\
556.01	0.01\\
557.01	0.01\\
558.01	0.01\\
559.01	0.01\\
560.01	0.01\\
561.01	0.01\\
562.01	0.01\\
563.01	0.01\\
564.01	0.01\\
565.01	0.01\\
566.01	0.01\\
567.01	0.01\\
568.01	0.01\\
569.01	0.01\\
570.01	0.01\\
571.01	0.01\\
572.01	0.01\\
573.01	0.01\\
574.01	0.01\\
575.01	0.01\\
576.01	0.01\\
577.01	0.01\\
578.01	0.01\\
579.01	0.01\\
580.01	0.01\\
581.01	0.01\\
582.01	0.01\\
583.01	0.01\\
584.01	0.01\\
585.01	0.01\\
586.01	0.01\\
587.01	0.01\\
588.01	0.01\\
589.01	0.01\\
590.01	0.01\\
591.01	0.01\\
592.01	0.01\\
593.01	0.01\\
594.01	0.01\\
595.01	0.01\\
596.01	0.01\\
597.01	0.01\\
598.01	0.01\\
599.01	0.00629021083727977\\
599.02	0.006248202568395\\
599.03	0.00620593501768914\\
599.04	0.00616341592265327\\
599.05	0.00612065360930352\\
599.06	0.00607764614660782\\
599.07	0.00603440015398376\\
599.08	0.00599092538614726\\
599.09	0.00594722899179982\\
599.1	0.00590332194821352\\
599.11	0.00585921584344904\\
599.12	0.00581492271785932\\
599.13	0.00577045600772243\\
599.14	0.00572583003446701\\
599.15	0.00568106004996\\
599.16	0.00563616228426199\\
599.17	0.00559115399666164\\
599.18	0.00554605382720621\\
599.19	0.00550088126862895\\
599.2	0.00545565701151042\\
599.21	0.00541022061590217\\
599.22	0.00536434805706764\\
599.23	0.00531803513257949\\
599.24	0.00527127758210094\\
599.25	0.00522407109808666\\
599.26	0.00517641135752969\\
599.27	0.00512829397750254\\
599.28	0.00507971437988895\\
599.29	0.00503066794367191\\
599.3	0.004981150101723\\
599.31	0.00493115621964229\\
599.32	0.00488068159105543\\
599.33	0.00482972143447222\\
599.34	0.00477827088995449\\
599.35	0.00472632501558263\\
599.36	0.00467387878370851\\
599.37	0.00462092707698275\\
599.38	0.00456746468414297\\
599.39	0.00451348629554908\\
599.4	0.00445898649845102\\
599.41	0.00440395977782773\\
599.42	0.00434840051773466\\
599.43	0.00429230298406572\\
599.44	0.0042356613181626\\
599.45	0.00417846953050769\\
599.46	0.00412072149404243\\
599.47	0.00406241093708933\\
599.48	0.00400353143585462\\
599.49	0.00394407712271724\\
599.5	0.00388404237753246\\
599.51	0.00382342152498173\\
599.52	0.00376220883389879\\
599.53	0.00370039851658248\\
599.54	0.00363798472815847\\
599.55	0.00357496156595277\\
599.56	0.0035113230688626\\
599.57	0.00344706321673412\\
599.58	0.00338217592973297\\
599.59	0.00331665506770788\\
599.6	0.00325049442954719\\
599.61	0.00318368775252859\\
599.62	0.00311622871165683\\
599.63	0.00304811091899751\\
599.64	0.00297932792300593\\
599.65	0.00290987320784969\\
599.66	0.00283974019272552\\
599.67	0.00276892223117018\\
599.68	0.00269741261036607\\
599.69	0.00262520455044173\\
599.7	0.0025522912037677\\
599.71	0.00247866565424824\\
599.72	0.00240432091660945\\
599.73	0.00232924993568433\\
599.74	0.00225344558569628\\
599.75	0.00217690066954147\\
599.76	0.00209960791807056\\
599.77	0.00202155998937109\\
599.78	0.00194274946805127\\
599.79	0.0018631688645289\\
599.8	0.0017828106143252\\
599.81	0.00170166707736891\\
599.82	0.00161973053731143\\
599.83	0.00153699320085363\\
599.84	0.00145344719708646\\
599.85	0.00136908457684828\\
599.86	0.00128389731210149\\
599.87	0.001197877295332\\
599.88	0.00111101633897473\\
599.89	0.00102330617486933\\
599.9	0.000934738453750302\\
599.91	0.000845304744776263\\
599.92	0.000754996535103638\\
599.93	0.000663805229510536\\
599.94	0.000571722150077203\\
599.95	0.000478738535930024\\
599.96	0.000384845543056832\\
599.97	0.000290034244202044\\
599.98	0.00019429562885097\\
599.99	9.76206033136574e-05\\
600	0\\
};
\addplot [color=mycolor10,solid,forget plot]
  table[row sep=crcr]{%
0.01	0.01\\
1.01	0.01\\
2.01	0.01\\
3.01	0.01\\
4.01	0.01\\
5.01	0.01\\
6.01	0.01\\
7.01	0.01\\
8.01	0.01\\
9.01	0.01\\
10.01	0.01\\
11.01	0.01\\
12.01	0.01\\
13.01	0.01\\
14.01	0.01\\
15.01	0.01\\
16.01	0.01\\
17.01	0.01\\
18.01	0.01\\
19.01	0.01\\
20.01	0.01\\
21.01	0.01\\
22.01	0.01\\
23.01	0.01\\
24.01	0.01\\
25.01	0.01\\
26.01	0.01\\
27.01	0.01\\
28.01	0.01\\
29.01	0.01\\
30.01	0.01\\
31.01	0.01\\
32.01	0.01\\
33.01	0.01\\
34.01	0.01\\
35.01	0.01\\
36.01	0.01\\
37.01	0.01\\
38.01	0.01\\
39.01	0.01\\
40.01	0.01\\
41.01	0.01\\
42.01	0.01\\
43.01	0.01\\
44.01	0.01\\
45.01	0.01\\
46.01	0.01\\
47.01	0.01\\
48.01	0.01\\
49.01	0.01\\
50.01	0.01\\
51.01	0.01\\
52.01	0.01\\
53.01	0.01\\
54.01	0.01\\
55.01	0.01\\
56.01	0.01\\
57.01	0.01\\
58.01	0.01\\
59.01	0.01\\
60.01	0.01\\
61.01	0.01\\
62.01	0.01\\
63.01	0.01\\
64.01	0.01\\
65.01	0.01\\
66.01	0.01\\
67.01	0.01\\
68.01	0.01\\
69.01	0.01\\
70.01	0.01\\
71.01	0.01\\
72.01	0.01\\
73.01	0.01\\
74.01	0.01\\
75.01	0.01\\
76.01	0.01\\
77.01	0.01\\
78.01	0.01\\
79.01	0.01\\
80.01	0.01\\
81.01	0.01\\
82.01	0.01\\
83.01	0.01\\
84.01	0.01\\
85.01	0.01\\
86.01	0.01\\
87.01	0.01\\
88.01	0.01\\
89.01	0.01\\
90.01	0.01\\
91.01	0.01\\
92.01	0.01\\
93.01	0.01\\
94.01	0.01\\
95.01	0.01\\
96.01	0.01\\
97.01	0.01\\
98.01	0.01\\
99.01	0.01\\
100.01	0.01\\
101.01	0.01\\
102.01	0.01\\
103.01	0.01\\
104.01	0.01\\
105.01	0.01\\
106.01	0.01\\
107.01	0.01\\
108.01	0.01\\
109.01	0.01\\
110.01	0.01\\
111.01	0.01\\
112.01	0.01\\
113.01	0.01\\
114.01	0.01\\
115.01	0.01\\
116.01	0.01\\
117.01	0.01\\
118.01	0.01\\
119.01	0.01\\
120.01	0.01\\
121.01	0.01\\
122.01	0.01\\
123.01	0.01\\
124.01	0.01\\
125.01	0.01\\
126.01	0.01\\
127.01	0.01\\
128.01	0.01\\
129.01	0.01\\
130.01	0.01\\
131.01	0.01\\
132.01	0.01\\
133.01	0.01\\
134.01	0.01\\
135.01	0.01\\
136.01	0.01\\
137.01	0.01\\
138.01	0.01\\
139.01	0.01\\
140.01	0.01\\
141.01	0.01\\
142.01	0.01\\
143.01	0.01\\
144.01	0.01\\
145.01	0.01\\
146.01	0.01\\
147.01	0.01\\
148.01	0.01\\
149.01	0.01\\
150.01	0.01\\
151.01	0.01\\
152.01	0.01\\
153.01	0.01\\
154.01	0.01\\
155.01	0.01\\
156.01	0.01\\
157.01	0.01\\
158.01	0.01\\
159.01	0.01\\
160.01	0.01\\
161.01	0.01\\
162.01	0.01\\
163.01	0.01\\
164.01	0.01\\
165.01	0.01\\
166.01	0.01\\
167.01	0.01\\
168.01	0.01\\
169.01	0.01\\
170.01	0.01\\
171.01	0.01\\
172.01	0.01\\
173.01	0.01\\
174.01	0.01\\
175.01	0.01\\
176.01	0.01\\
177.01	0.01\\
178.01	0.01\\
179.01	0.01\\
180.01	0.01\\
181.01	0.01\\
182.01	0.01\\
183.01	0.01\\
184.01	0.01\\
185.01	0.01\\
186.01	0.01\\
187.01	0.01\\
188.01	0.01\\
189.01	0.01\\
190.01	0.01\\
191.01	0.01\\
192.01	0.01\\
193.01	0.01\\
194.01	0.01\\
195.01	0.01\\
196.01	0.01\\
197.01	0.01\\
198.01	0.01\\
199.01	0.01\\
200.01	0.01\\
201.01	0.01\\
202.01	0.01\\
203.01	0.01\\
204.01	0.01\\
205.01	0.01\\
206.01	0.01\\
207.01	0.01\\
208.01	0.01\\
209.01	0.01\\
210.01	0.01\\
211.01	0.01\\
212.01	0.01\\
213.01	0.01\\
214.01	0.01\\
215.01	0.01\\
216.01	0.01\\
217.01	0.01\\
218.01	0.01\\
219.01	0.01\\
220.01	0.01\\
221.01	0.01\\
222.01	0.01\\
223.01	0.01\\
224.01	0.01\\
225.01	0.01\\
226.01	0.01\\
227.01	0.01\\
228.01	0.01\\
229.01	0.01\\
230.01	0.01\\
231.01	0.01\\
232.01	0.01\\
233.01	0.01\\
234.01	0.01\\
235.01	0.01\\
236.01	0.01\\
237.01	0.01\\
238.01	0.01\\
239.01	0.01\\
240.01	0.01\\
241.01	0.01\\
242.01	0.01\\
243.01	0.01\\
244.01	0.01\\
245.01	0.01\\
246.01	0.01\\
247.01	0.01\\
248.01	0.01\\
249.01	0.01\\
250.01	0.01\\
251.01	0.01\\
252.01	0.01\\
253.01	0.01\\
254.01	0.01\\
255.01	0.01\\
256.01	0.01\\
257.01	0.01\\
258.01	0.01\\
259.01	0.01\\
260.01	0.01\\
261.01	0.01\\
262.01	0.01\\
263.01	0.01\\
264.01	0.01\\
265.01	0.01\\
266.01	0.01\\
267.01	0.01\\
268.01	0.01\\
269.01	0.01\\
270.01	0.01\\
271.01	0.01\\
272.01	0.01\\
273.01	0.01\\
274.01	0.01\\
275.01	0.01\\
276.01	0.01\\
277.01	0.01\\
278.01	0.01\\
279.01	0.01\\
280.01	0.01\\
281.01	0.01\\
282.01	0.01\\
283.01	0.01\\
284.01	0.01\\
285.01	0.01\\
286.01	0.01\\
287.01	0.01\\
288.01	0.01\\
289.01	0.01\\
290.01	0.01\\
291.01	0.01\\
292.01	0.01\\
293.01	0.01\\
294.01	0.01\\
295.01	0.01\\
296.01	0.01\\
297.01	0.01\\
298.01	0.01\\
299.01	0.01\\
300.01	0.01\\
301.01	0.01\\
302.01	0.01\\
303.01	0.01\\
304.01	0.01\\
305.01	0.01\\
306.01	0.01\\
307.01	0.01\\
308.01	0.01\\
309.01	0.01\\
310.01	0.01\\
311.01	0.01\\
312.01	0.01\\
313.01	0.01\\
314.01	0.01\\
315.01	0.01\\
316.01	0.01\\
317.01	0.01\\
318.01	0.01\\
319.01	0.01\\
320.01	0.01\\
321.01	0.01\\
322.01	0.01\\
323.01	0.01\\
324.01	0.01\\
325.01	0.01\\
326.01	0.01\\
327.01	0.01\\
328.01	0.01\\
329.01	0.01\\
330.01	0.01\\
331.01	0.01\\
332.01	0.01\\
333.01	0.01\\
334.01	0.01\\
335.01	0.01\\
336.01	0.01\\
337.01	0.01\\
338.01	0.01\\
339.01	0.01\\
340.01	0.01\\
341.01	0.01\\
342.01	0.01\\
343.01	0.01\\
344.01	0.01\\
345.01	0.01\\
346.01	0.01\\
347.01	0.01\\
348.01	0.01\\
349.01	0.01\\
350.01	0.01\\
351.01	0.01\\
352.01	0.01\\
353.01	0.01\\
354.01	0.01\\
355.01	0.01\\
356.01	0.01\\
357.01	0.01\\
358.01	0.01\\
359.01	0.01\\
360.01	0.01\\
361.01	0.01\\
362.01	0.01\\
363.01	0.01\\
364.01	0.01\\
365.01	0.01\\
366.01	0.01\\
367.01	0.01\\
368.01	0.01\\
369.01	0.01\\
370.01	0.01\\
371.01	0.01\\
372.01	0.01\\
373.01	0.01\\
374.01	0.01\\
375.01	0.01\\
376.01	0.01\\
377.01	0.01\\
378.01	0.01\\
379.01	0.01\\
380.01	0.01\\
381.01	0.01\\
382.01	0.01\\
383.01	0.01\\
384.01	0.01\\
385.01	0.01\\
386.01	0.01\\
387.01	0.01\\
388.01	0.01\\
389.01	0.01\\
390.01	0.01\\
391.01	0.01\\
392.01	0.01\\
393.01	0.01\\
394.01	0.01\\
395.01	0.01\\
396.01	0.01\\
397.01	0.01\\
398.01	0.01\\
399.01	0.01\\
400.01	0.01\\
401.01	0.01\\
402.01	0.01\\
403.01	0.01\\
404.01	0.01\\
405.01	0.01\\
406.01	0.01\\
407.01	0.01\\
408.01	0.01\\
409.01	0.01\\
410.01	0.01\\
411.01	0.01\\
412.01	0.01\\
413.01	0.01\\
414.01	0.01\\
415.01	0.01\\
416.01	0.01\\
417.01	0.01\\
418.01	0.01\\
419.01	0.01\\
420.01	0.01\\
421.01	0.01\\
422.01	0.01\\
423.01	0.01\\
424.01	0.01\\
425.01	0.01\\
426.01	0.01\\
427.01	0.01\\
428.01	0.01\\
429.01	0.01\\
430.01	0.01\\
431.01	0.01\\
432.01	0.01\\
433.01	0.01\\
434.01	0.01\\
435.01	0.01\\
436.01	0.01\\
437.01	0.01\\
438.01	0.01\\
439.01	0.01\\
440.01	0.01\\
441.01	0.01\\
442.01	0.01\\
443.01	0.01\\
444.01	0.01\\
445.01	0.01\\
446.01	0.01\\
447.01	0.01\\
448.01	0.01\\
449.01	0.01\\
450.01	0.01\\
451.01	0.01\\
452.01	0.01\\
453.01	0.01\\
454.01	0.01\\
455.01	0.01\\
456.01	0.01\\
457.01	0.01\\
458.01	0.01\\
459.01	0.01\\
460.01	0.01\\
461.01	0.01\\
462.01	0.01\\
463.01	0.01\\
464.01	0.01\\
465.01	0.01\\
466.01	0.01\\
467.01	0.01\\
468.01	0.01\\
469.01	0.01\\
470.01	0.01\\
471.01	0.01\\
472.01	0.01\\
473.01	0.01\\
474.01	0.01\\
475.01	0.01\\
476.01	0.01\\
477.01	0.01\\
478.01	0.01\\
479.01	0.01\\
480.01	0.01\\
481.01	0.01\\
482.01	0.01\\
483.01	0.01\\
484.01	0.01\\
485.01	0.01\\
486.01	0.01\\
487.01	0.01\\
488.01	0.01\\
489.01	0.01\\
490.01	0.01\\
491.01	0.01\\
492.01	0.01\\
493.01	0.01\\
494.01	0.01\\
495.01	0.01\\
496.01	0.01\\
497.01	0.01\\
498.01	0.01\\
499.01	0.01\\
500.01	0.01\\
501.01	0.01\\
502.01	0.01\\
503.01	0.01\\
504.01	0.01\\
505.01	0.01\\
506.01	0.01\\
507.01	0.01\\
508.01	0.01\\
509.01	0.01\\
510.01	0.01\\
511.01	0.01\\
512.01	0.01\\
513.01	0.01\\
514.01	0.01\\
515.01	0.01\\
516.01	0.01\\
517.01	0.01\\
518.01	0.01\\
519.01	0.01\\
520.01	0.01\\
521.01	0.01\\
522.01	0.01\\
523.01	0.01\\
524.01	0.01\\
525.01	0.01\\
526.01	0.01\\
527.01	0.01\\
528.01	0.01\\
529.01	0.01\\
530.01	0.01\\
531.01	0.01\\
532.01	0.01\\
533.01	0.01\\
534.01	0.01\\
535.01	0.01\\
536.01	0.01\\
537.01	0.01\\
538.01	0.01\\
539.01	0.01\\
540.01	0.01\\
541.01	0.01\\
542.01	0.01\\
543.01	0.01\\
544.01	0.01\\
545.01	0.01\\
546.01	0.01\\
547.01	0.01\\
548.01	0.01\\
549.01	0.01\\
550.01	0.01\\
551.01	0.01\\
552.01	0.01\\
553.01	0.01\\
554.01	0.01\\
555.01	0.01\\
556.01	0.01\\
557.01	0.01\\
558.01	0.01\\
559.01	0.01\\
560.01	0.01\\
561.01	0.01\\
562.01	0.01\\
563.01	0.01\\
564.01	0.01\\
565.01	0.01\\
566.01	0.01\\
567.01	0.01\\
568.01	0.01\\
569.01	0.01\\
570.01	0.01\\
571.01	0.01\\
572.01	0.01\\
573.01	0.01\\
574.01	0.01\\
575.01	0.01\\
576.01	0.01\\
577.01	0.01\\
578.01	0.01\\
579.01	0.01\\
580.01	0.01\\
581.01	0.01\\
582.01	0.01\\
583.01	0.01\\
584.01	0.01\\
585.01	0.01\\
586.01	0.01\\
587.01	0.01\\
588.01	0.01\\
589.01	0.01\\
590.01	0.01\\
591.01	0.01\\
592.01	0.01\\
593.01	0.01\\
594.01	0.01\\
595.01	0.01\\
596.01	0.01\\
597.01	0.01\\
598.01	0.01\\
599.01	0.00987101508599242\\
599.02	0.00966435710912938\\
599.03	0.00945619745033124\\
599.04	0.00924651414318886\\
599.05	0.00903528450781953\\
599.06	0.00882249594459827\\
599.07	0.0086081271628761\\
599.08	0.0083921536080565\\
599.09	0.00817455318581161\\
599.1	0.00795529996678749\\
599.11	0.00773436739351956\\
599.12	0.00751172833030985\\
599.13	0.00728735412338303\\
599.14	0.00706121511304294\\
599.15	0.00683328058997119\\
599.16	0.00660351874927539\\
599.17	0.00637189664212814\\
599.18	0.00613838011978858\\
599.19	0.00590293378481197\\
599.2	0.00566552093448356\\
599.21	0.00550985573121098\\
599.22	0.00545922463758062\\
599.23	0.00540818945802247\\
599.24	0.00535675333215924\\
599.25	0.00530491703217995\\
599.26	0.00525267475303119\\
599.27	0.00520003038725458\\
599.28	0.00514698602738943\\
599.29	0.00509354574416965\\
599.3	0.00503971428484819\\
599.31	0.00498549690289976\\
599.32	0.00493089970370655\\
599.33	0.00487592932800012\\
599.34	0.00482059297849655\\
599.35	0.0047648984478651\\
599.36	0.00470885414810968\\
599.37	0.00465246914144833\\
599.38	0.00459575317278141\\
599.39	0.00453871670384681\\
599.4	0.00448137094916667\\
599.41	0.00442372791380806\\
599.42	0.00436580029185107\\
599.43	0.00430760172657727\\
599.44	0.00424914685408985\\
599.45	0.00419045127414728\\
599.46	0.00413153159953789\\
599.47	0.00407240550824231\\
599.48	0.00401309179857429\\
599.49	0.00395332722398739\\
599.5	0.0038929821849682\\
599.51	0.0038320512203846\\
599.52	0.00377052884641714\\
599.53	0.00370840955592181\\
599.54	0.00364568780239088\\
599.55	0.00358235799429536\\
599.56	0.00351841449275467\\
599.57	0.00345385160745471\\
599.58	0.00338866359564696\\
599.59	0.00332284466108848\\
599.6	0.00325638895291892\\
599.61	0.0031892905644707\\
599.62	0.00312154353299735\\
599.63	0.00305314183776542\\
599.64	0.00298407939834351\\
599.65	0.00291435007302961\\
599.66	0.00284394765722354\\
599.67	0.0027728658817277\\
599.68	0.00270109841094376\\
599.69	0.00262863884095882\\
599.7	0.00255548069751408\\
599.71	0.00248161743384854\\
599.72	0.00240704242840991\\
599.73	0.00233174898242438\\
599.74	0.00225573031705331\\
599.75	0.00217897957054852\\
599.76	0.00210148979528493\\
599.77	0.00202325395460674\\
599.78	0.00194426491947531\\
599.79	0.00186451546417297\\
599.8	0.00178399826230015\\
599.81	0.00170270588132737\\
599.82	0.00162063077725271\\
599.83	0.00153776528948584\\
599.84	0.00145410163537975\\
599.85	0.001369631904385\\
599.86	0.00128434805179975\\
599.87	0.00119824189208689\\
599.88	0.00111130509172728\\
599.89	0.00102352916157608\\
599.9	0.00093490544868644\\
599.91	0.000845425127562385\\
599.92	0.000755079190799706\\
599.93	0.000663858439070786\\
599.94	0.000571753470405756\\
599.95	0.000478754668719049\\
599.96	0.000384852191526369\\
599.97	0.000290035956793153\\
599.98	0.00019429562885097\\
599.99	9.76206033136556e-05\\
600	0\\
};
\addplot [color=mycolor11,solid,forget plot]
  table[row sep=crcr]{%
0.01	0.01\\
1.01	0.01\\
2.01	0.01\\
3.01	0.01\\
4.01	0.01\\
5.01	0.01\\
6.01	0.01\\
7.01	0.01\\
8.01	0.01\\
9.01	0.01\\
10.01	0.01\\
11.01	0.01\\
12.01	0.01\\
13.01	0.01\\
14.01	0.01\\
15.01	0.01\\
16.01	0.01\\
17.01	0.01\\
18.01	0.01\\
19.01	0.01\\
20.01	0.01\\
21.01	0.01\\
22.01	0.01\\
23.01	0.01\\
24.01	0.01\\
25.01	0.01\\
26.01	0.01\\
27.01	0.01\\
28.01	0.01\\
29.01	0.01\\
30.01	0.01\\
31.01	0.01\\
32.01	0.01\\
33.01	0.01\\
34.01	0.01\\
35.01	0.01\\
36.01	0.01\\
37.01	0.01\\
38.01	0.01\\
39.01	0.01\\
40.01	0.01\\
41.01	0.01\\
42.01	0.01\\
43.01	0.01\\
44.01	0.01\\
45.01	0.01\\
46.01	0.01\\
47.01	0.01\\
48.01	0.01\\
49.01	0.01\\
50.01	0.01\\
51.01	0.01\\
52.01	0.01\\
53.01	0.01\\
54.01	0.01\\
55.01	0.01\\
56.01	0.01\\
57.01	0.01\\
58.01	0.01\\
59.01	0.01\\
60.01	0.01\\
61.01	0.01\\
62.01	0.01\\
63.01	0.01\\
64.01	0.01\\
65.01	0.01\\
66.01	0.01\\
67.01	0.01\\
68.01	0.01\\
69.01	0.01\\
70.01	0.01\\
71.01	0.01\\
72.01	0.01\\
73.01	0.01\\
74.01	0.01\\
75.01	0.01\\
76.01	0.01\\
77.01	0.01\\
78.01	0.01\\
79.01	0.01\\
80.01	0.01\\
81.01	0.01\\
82.01	0.01\\
83.01	0.01\\
84.01	0.01\\
85.01	0.01\\
86.01	0.01\\
87.01	0.01\\
88.01	0.01\\
89.01	0.01\\
90.01	0.01\\
91.01	0.01\\
92.01	0.01\\
93.01	0.01\\
94.01	0.01\\
95.01	0.01\\
96.01	0.01\\
97.01	0.01\\
98.01	0.01\\
99.01	0.01\\
100.01	0.01\\
101.01	0.01\\
102.01	0.01\\
103.01	0.01\\
104.01	0.01\\
105.01	0.01\\
106.01	0.01\\
107.01	0.01\\
108.01	0.01\\
109.01	0.01\\
110.01	0.01\\
111.01	0.01\\
112.01	0.01\\
113.01	0.01\\
114.01	0.01\\
115.01	0.01\\
116.01	0.01\\
117.01	0.01\\
118.01	0.01\\
119.01	0.01\\
120.01	0.01\\
121.01	0.01\\
122.01	0.01\\
123.01	0.01\\
124.01	0.01\\
125.01	0.01\\
126.01	0.01\\
127.01	0.01\\
128.01	0.01\\
129.01	0.01\\
130.01	0.01\\
131.01	0.01\\
132.01	0.01\\
133.01	0.01\\
134.01	0.01\\
135.01	0.01\\
136.01	0.01\\
137.01	0.01\\
138.01	0.01\\
139.01	0.01\\
140.01	0.01\\
141.01	0.01\\
142.01	0.01\\
143.01	0.01\\
144.01	0.01\\
145.01	0.01\\
146.01	0.01\\
147.01	0.01\\
148.01	0.01\\
149.01	0.01\\
150.01	0.01\\
151.01	0.01\\
152.01	0.01\\
153.01	0.01\\
154.01	0.01\\
155.01	0.01\\
156.01	0.01\\
157.01	0.01\\
158.01	0.01\\
159.01	0.01\\
160.01	0.01\\
161.01	0.01\\
162.01	0.01\\
163.01	0.01\\
164.01	0.01\\
165.01	0.01\\
166.01	0.01\\
167.01	0.01\\
168.01	0.01\\
169.01	0.01\\
170.01	0.01\\
171.01	0.01\\
172.01	0.01\\
173.01	0.01\\
174.01	0.01\\
175.01	0.01\\
176.01	0.01\\
177.01	0.01\\
178.01	0.01\\
179.01	0.01\\
180.01	0.01\\
181.01	0.01\\
182.01	0.01\\
183.01	0.01\\
184.01	0.01\\
185.01	0.01\\
186.01	0.01\\
187.01	0.01\\
188.01	0.01\\
189.01	0.01\\
190.01	0.01\\
191.01	0.01\\
192.01	0.01\\
193.01	0.01\\
194.01	0.01\\
195.01	0.01\\
196.01	0.01\\
197.01	0.01\\
198.01	0.01\\
199.01	0.01\\
200.01	0.01\\
201.01	0.01\\
202.01	0.01\\
203.01	0.01\\
204.01	0.01\\
205.01	0.01\\
206.01	0.01\\
207.01	0.01\\
208.01	0.01\\
209.01	0.01\\
210.01	0.01\\
211.01	0.01\\
212.01	0.01\\
213.01	0.01\\
214.01	0.01\\
215.01	0.01\\
216.01	0.01\\
217.01	0.01\\
218.01	0.01\\
219.01	0.01\\
220.01	0.01\\
221.01	0.01\\
222.01	0.01\\
223.01	0.01\\
224.01	0.01\\
225.01	0.01\\
226.01	0.01\\
227.01	0.01\\
228.01	0.01\\
229.01	0.01\\
230.01	0.01\\
231.01	0.01\\
232.01	0.01\\
233.01	0.01\\
234.01	0.01\\
235.01	0.01\\
236.01	0.01\\
237.01	0.01\\
238.01	0.01\\
239.01	0.01\\
240.01	0.01\\
241.01	0.01\\
242.01	0.01\\
243.01	0.01\\
244.01	0.01\\
245.01	0.01\\
246.01	0.01\\
247.01	0.01\\
248.01	0.01\\
249.01	0.01\\
250.01	0.01\\
251.01	0.01\\
252.01	0.01\\
253.01	0.01\\
254.01	0.01\\
255.01	0.01\\
256.01	0.01\\
257.01	0.01\\
258.01	0.01\\
259.01	0.01\\
260.01	0.01\\
261.01	0.01\\
262.01	0.01\\
263.01	0.01\\
264.01	0.01\\
265.01	0.01\\
266.01	0.01\\
267.01	0.01\\
268.01	0.01\\
269.01	0.01\\
270.01	0.01\\
271.01	0.01\\
272.01	0.01\\
273.01	0.01\\
274.01	0.01\\
275.01	0.01\\
276.01	0.01\\
277.01	0.01\\
278.01	0.01\\
279.01	0.01\\
280.01	0.01\\
281.01	0.01\\
282.01	0.01\\
283.01	0.01\\
284.01	0.01\\
285.01	0.01\\
286.01	0.01\\
287.01	0.01\\
288.01	0.01\\
289.01	0.01\\
290.01	0.01\\
291.01	0.01\\
292.01	0.01\\
293.01	0.01\\
294.01	0.01\\
295.01	0.01\\
296.01	0.01\\
297.01	0.01\\
298.01	0.01\\
299.01	0.01\\
300.01	0.01\\
301.01	0.01\\
302.01	0.01\\
303.01	0.01\\
304.01	0.01\\
305.01	0.01\\
306.01	0.01\\
307.01	0.01\\
308.01	0.01\\
309.01	0.01\\
310.01	0.01\\
311.01	0.01\\
312.01	0.01\\
313.01	0.01\\
314.01	0.01\\
315.01	0.01\\
316.01	0.01\\
317.01	0.01\\
318.01	0.01\\
319.01	0.01\\
320.01	0.01\\
321.01	0.01\\
322.01	0.01\\
323.01	0.01\\
324.01	0.01\\
325.01	0.01\\
326.01	0.01\\
327.01	0.01\\
328.01	0.01\\
329.01	0.01\\
330.01	0.01\\
331.01	0.01\\
332.01	0.01\\
333.01	0.01\\
334.01	0.01\\
335.01	0.01\\
336.01	0.01\\
337.01	0.01\\
338.01	0.01\\
339.01	0.01\\
340.01	0.01\\
341.01	0.01\\
342.01	0.01\\
343.01	0.01\\
344.01	0.01\\
345.01	0.01\\
346.01	0.01\\
347.01	0.01\\
348.01	0.01\\
349.01	0.01\\
350.01	0.01\\
351.01	0.01\\
352.01	0.01\\
353.01	0.01\\
354.01	0.01\\
355.01	0.01\\
356.01	0.01\\
357.01	0.01\\
358.01	0.01\\
359.01	0.01\\
360.01	0.01\\
361.01	0.01\\
362.01	0.01\\
363.01	0.01\\
364.01	0.01\\
365.01	0.01\\
366.01	0.01\\
367.01	0.01\\
368.01	0.01\\
369.01	0.01\\
370.01	0.01\\
371.01	0.01\\
372.01	0.01\\
373.01	0.01\\
374.01	0.01\\
375.01	0.01\\
376.01	0.01\\
377.01	0.01\\
378.01	0.01\\
379.01	0.01\\
380.01	0.01\\
381.01	0.01\\
382.01	0.01\\
383.01	0.01\\
384.01	0.01\\
385.01	0.01\\
386.01	0.01\\
387.01	0.01\\
388.01	0.01\\
389.01	0.01\\
390.01	0.01\\
391.01	0.01\\
392.01	0.01\\
393.01	0.01\\
394.01	0.01\\
395.01	0.01\\
396.01	0.01\\
397.01	0.01\\
398.01	0.01\\
399.01	0.01\\
400.01	0.01\\
401.01	0.01\\
402.01	0.01\\
403.01	0.01\\
404.01	0.01\\
405.01	0.01\\
406.01	0.01\\
407.01	0.01\\
408.01	0.01\\
409.01	0.01\\
410.01	0.01\\
411.01	0.01\\
412.01	0.01\\
413.01	0.01\\
414.01	0.01\\
415.01	0.01\\
416.01	0.01\\
417.01	0.01\\
418.01	0.01\\
419.01	0.01\\
420.01	0.01\\
421.01	0.01\\
422.01	0.01\\
423.01	0.01\\
424.01	0.01\\
425.01	0.01\\
426.01	0.01\\
427.01	0.01\\
428.01	0.01\\
429.01	0.01\\
430.01	0.01\\
431.01	0.01\\
432.01	0.01\\
433.01	0.01\\
434.01	0.01\\
435.01	0.01\\
436.01	0.01\\
437.01	0.01\\
438.01	0.01\\
439.01	0.01\\
440.01	0.01\\
441.01	0.01\\
442.01	0.01\\
443.01	0.01\\
444.01	0.01\\
445.01	0.01\\
446.01	0.01\\
447.01	0.01\\
448.01	0.01\\
449.01	0.01\\
450.01	0.01\\
451.01	0.01\\
452.01	0.01\\
453.01	0.01\\
454.01	0.01\\
455.01	0.01\\
456.01	0.01\\
457.01	0.01\\
458.01	0.01\\
459.01	0.01\\
460.01	0.01\\
461.01	0.01\\
462.01	0.01\\
463.01	0.01\\
464.01	0.01\\
465.01	0.01\\
466.01	0.01\\
467.01	0.01\\
468.01	0.01\\
469.01	0.01\\
470.01	0.01\\
471.01	0.01\\
472.01	0.01\\
473.01	0.01\\
474.01	0.01\\
475.01	0.01\\
476.01	0.01\\
477.01	0.01\\
478.01	0.01\\
479.01	0.01\\
480.01	0.01\\
481.01	0.01\\
482.01	0.01\\
483.01	0.01\\
484.01	0.01\\
485.01	0.01\\
486.01	0.01\\
487.01	0.01\\
488.01	0.01\\
489.01	0.01\\
490.01	0.01\\
491.01	0.01\\
492.01	0.01\\
493.01	0.01\\
494.01	0.01\\
495.01	0.01\\
496.01	0.01\\
497.01	0.01\\
498.01	0.01\\
499.01	0.01\\
500.01	0.01\\
501.01	0.01\\
502.01	0.01\\
503.01	0.01\\
504.01	0.01\\
505.01	0.01\\
506.01	0.01\\
507.01	0.01\\
508.01	0.01\\
509.01	0.01\\
510.01	0.01\\
511.01	0.01\\
512.01	0.01\\
513.01	0.01\\
514.01	0.01\\
515.01	0.01\\
516.01	0.01\\
517.01	0.01\\
518.01	0.01\\
519.01	0.01\\
520.01	0.01\\
521.01	0.01\\
522.01	0.01\\
523.01	0.01\\
524.01	0.01\\
525.01	0.01\\
526.01	0.01\\
527.01	0.01\\
528.01	0.01\\
529.01	0.01\\
530.01	0.01\\
531.01	0.01\\
532.01	0.01\\
533.01	0.01\\
534.01	0.01\\
535.01	0.01\\
536.01	0.01\\
537.01	0.01\\
538.01	0.01\\
539.01	0.01\\
540.01	0.01\\
541.01	0.01\\
542.01	0.01\\
543.01	0.01\\
544.01	0.01\\
545.01	0.01\\
546.01	0.01\\
547.01	0.01\\
548.01	0.01\\
549.01	0.01\\
550.01	0.01\\
551.01	0.01\\
552.01	0.01\\
553.01	0.01\\
554.01	0.01\\
555.01	0.01\\
556.01	0.01\\
557.01	0.01\\
558.01	0.01\\
559.01	0.01\\
560.01	0.01\\
561.01	0.01\\
562.01	0.01\\
563.01	0.01\\
564.01	0.01\\
565.01	0.01\\
566.01	0.01\\
567.01	0.01\\
568.01	0.01\\
569.01	0.01\\
570.01	0.01\\
571.01	0.01\\
572.01	0.01\\
573.01	0.01\\
574.01	0.01\\
575.01	0.01\\
576.01	0.01\\
577.01	0.01\\
578.01	0.01\\
579.01	0.01\\
580.01	0.01\\
581.01	0.01\\
582.01	0.01\\
583.01	0.01\\
584.01	0.01\\
585.01	0.01\\
586.01	0.01\\
587.01	0.01\\
588.01	0.01\\
589.01	0.01\\
590.01	0.01\\
591.01	0.01\\
592.01	0.01\\
593.01	0.01\\
594.01	0.01\\
595.01	0.01\\
596.01	0.01\\
597.01	0.01\\
598.01	0.01\\
599.01	0.01\\
599.02	0.01\\
599.03	0.01\\
599.04	0.01\\
599.05	0.01\\
599.06	0.01\\
599.07	0.01\\
599.08	0.01\\
599.09	0.01\\
599.1	0.01\\
599.11	0.01\\
599.12	0.01\\
599.13	0.01\\
599.14	0.01\\
599.15	0.01\\
599.16	0.01\\
599.17	0.01\\
599.18	0.01\\
599.19	0.01\\
599.2	0.01\\
599.21	0.00991642967089183\\
599.22	0.00972620919023257\\
599.23	0.00953476336815602\\
599.24	0.00934207553102435\\
599.25	0.0091481312066784\\
599.26	0.00895292229058141\\
599.27	0.00875643082365402\\
599.28	0.00855864061331242\\
599.29	0.00835953330433368\\
599.3	0.00815908957986898\\
599.31	0.00795728944928943\\
599.32	0.00775411195876306\\
599.33	0.00754953562889951\\
599.34	0.00734353825683295\\
599.35	0.00713609688778657\\
599.36	0.0069271877853486\\
599.37	0.00671678640038438\\
599.38	0.00650486733850286\\
599.39	0.00629140432599014\\
599.4	0.00607637017411562\\
599.41	0.00585973674171082\\
599.42	0.00564147504834083\\
599.43	0.00542155500704852\\
599.44	0.00519994537848073\\
599.45	0.0049766138012967\\
599.46	0.00475152674427759\\
599.47	0.00452464945590616\\
599.48	0.00429594591124254\\
599.49	0.00417928104780683\\
599.5	0.00411405932776891\\
599.51	0.00404820617347344\\
599.52	0.00398171253686825\\
599.53	0.00391456978350743\\
599.54	0.00384677251359477\\
599.55	0.00377831638301909\\
599.56	0.00370919748084717\\
599.57	0.00363941256110831\\
599.58	0.00356895850446607\\
599.59	0.0034978323251514\\
599.6	0.00342603117816287\\
599.61	0.0033535523667451\\
599.62	0.00328039320558156\\
599.63	0.00320655126747823\\
599.64	0.00313202434399071\\
599.65	0.00305681043605939\\
599.66	0.00298090774846901\\
599.67	0.0029043146896599\\
599.68	0.00282702988180877\\
599.69	0.00274905217132293\\
599.7	0.00267038063976806\\
599.71	0.00259101461525092\\
599.72	0.00251095368427965\\
599.73	0.00243019770412575\\
599.74	0.00234874690447312\\
599.75	0.00226660184212179\\
599.76	0.00218376339726204\\
599.77	0.00210023278902947\\
599.78	0.00201601159177037\\
599.79	0.00193110198940867\\
599.8	0.0018455066110789\\
599.81	0.00175922892935269\\
599.82	0.00167227313610708\\
599.83	0.00158464398574982\\
599.84	0.00149634682616875\\
599.85	0.00140738763146645\\
599.86	0.00131777303660006\\
599.87	0.00122751037405548\\
599.88	0.0011366077126951\\
599.89	0.00104507389892841\\
599.9	0.000952918600367008\\
599.91	0.000860152352137411\\
599.92	0.000766786606038976\\
599.93	0.000672833782748917\\
599.94	0.000578307327292129\\
599.95	0.000483221768011027\\
599.96	0.000387592779289264\\
599.97	0.000291437248303513\\
599.98	0.000194773346099669\\
599.99	9.76206033136556e-05\\
600	0\\
};
\addplot [color=mycolor12,solid,forget plot]
  table[row sep=crcr]{%
0.01	0.01\\
1.01	0.01\\
2.01	0.01\\
3.01	0.01\\
4.01	0.01\\
5.01	0.01\\
6.01	0.01\\
7.01	0.01\\
8.01	0.01\\
9.01	0.01\\
10.01	0.01\\
11.01	0.01\\
12.01	0.01\\
13.01	0.01\\
14.01	0.01\\
15.01	0.01\\
16.01	0.01\\
17.01	0.01\\
18.01	0.01\\
19.01	0.01\\
20.01	0.01\\
21.01	0.01\\
22.01	0.01\\
23.01	0.01\\
24.01	0.01\\
25.01	0.01\\
26.01	0.01\\
27.01	0.01\\
28.01	0.01\\
29.01	0.01\\
30.01	0.01\\
31.01	0.01\\
32.01	0.01\\
33.01	0.01\\
34.01	0.01\\
35.01	0.01\\
36.01	0.01\\
37.01	0.01\\
38.01	0.01\\
39.01	0.01\\
40.01	0.01\\
41.01	0.01\\
42.01	0.01\\
43.01	0.01\\
44.01	0.01\\
45.01	0.01\\
46.01	0.01\\
47.01	0.01\\
48.01	0.01\\
49.01	0.01\\
50.01	0.01\\
51.01	0.01\\
52.01	0.01\\
53.01	0.01\\
54.01	0.01\\
55.01	0.01\\
56.01	0.01\\
57.01	0.01\\
58.01	0.01\\
59.01	0.01\\
60.01	0.01\\
61.01	0.01\\
62.01	0.01\\
63.01	0.01\\
64.01	0.01\\
65.01	0.01\\
66.01	0.01\\
67.01	0.01\\
68.01	0.01\\
69.01	0.01\\
70.01	0.01\\
71.01	0.01\\
72.01	0.01\\
73.01	0.01\\
74.01	0.01\\
75.01	0.01\\
76.01	0.01\\
77.01	0.01\\
78.01	0.01\\
79.01	0.01\\
80.01	0.01\\
81.01	0.01\\
82.01	0.01\\
83.01	0.01\\
84.01	0.01\\
85.01	0.01\\
86.01	0.01\\
87.01	0.01\\
88.01	0.01\\
89.01	0.01\\
90.01	0.01\\
91.01	0.01\\
92.01	0.01\\
93.01	0.01\\
94.01	0.01\\
95.01	0.01\\
96.01	0.01\\
97.01	0.01\\
98.01	0.01\\
99.01	0.01\\
100.01	0.01\\
101.01	0.01\\
102.01	0.01\\
103.01	0.01\\
104.01	0.01\\
105.01	0.01\\
106.01	0.01\\
107.01	0.01\\
108.01	0.01\\
109.01	0.01\\
110.01	0.01\\
111.01	0.01\\
112.01	0.01\\
113.01	0.01\\
114.01	0.01\\
115.01	0.01\\
116.01	0.01\\
117.01	0.01\\
118.01	0.01\\
119.01	0.01\\
120.01	0.01\\
121.01	0.01\\
122.01	0.01\\
123.01	0.01\\
124.01	0.01\\
125.01	0.01\\
126.01	0.01\\
127.01	0.01\\
128.01	0.01\\
129.01	0.01\\
130.01	0.01\\
131.01	0.01\\
132.01	0.01\\
133.01	0.01\\
134.01	0.01\\
135.01	0.01\\
136.01	0.01\\
137.01	0.01\\
138.01	0.01\\
139.01	0.01\\
140.01	0.01\\
141.01	0.01\\
142.01	0.01\\
143.01	0.01\\
144.01	0.01\\
145.01	0.01\\
146.01	0.01\\
147.01	0.01\\
148.01	0.01\\
149.01	0.01\\
150.01	0.01\\
151.01	0.01\\
152.01	0.01\\
153.01	0.01\\
154.01	0.01\\
155.01	0.01\\
156.01	0.01\\
157.01	0.01\\
158.01	0.01\\
159.01	0.01\\
160.01	0.01\\
161.01	0.01\\
162.01	0.01\\
163.01	0.01\\
164.01	0.01\\
165.01	0.01\\
166.01	0.01\\
167.01	0.01\\
168.01	0.01\\
169.01	0.01\\
170.01	0.01\\
171.01	0.01\\
172.01	0.01\\
173.01	0.01\\
174.01	0.01\\
175.01	0.01\\
176.01	0.01\\
177.01	0.01\\
178.01	0.01\\
179.01	0.01\\
180.01	0.01\\
181.01	0.01\\
182.01	0.01\\
183.01	0.01\\
184.01	0.01\\
185.01	0.01\\
186.01	0.01\\
187.01	0.01\\
188.01	0.01\\
189.01	0.01\\
190.01	0.01\\
191.01	0.01\\
192.01	0.01\\
193.01	0.01\\
194.01	0.01\\
195.01	0.01\\
196.01	0.01\\
197.01	0.01\\
198.01	0.01\\
199.01	0.01\\
200.01	0.01\\
201.01	0.01\\
202.01	0.01\\
203.01	0.01\\
204.01	0.01\\
205.01	0.01\\
206.01	0.01\\
207.01	0.01\\
208.01	0.01\\
209.01	0.01\\
210.01	0.01\\
211.01	0.01\\
212.01	0.01\\
213.01	0.01\\
214.01	0.01\\
215.01	0.01\\
216.01	0.01\\
217.01	0.01\\
218.01	0.01\\
219.01	0.01\\
220.01	0.01\\
221.01	0.01\\
222.01	0.01\\
223.01	0.01\\
224.01	0.01\\
225.01	0.01\\
226.01	0.01\\
227.01	0.01\\
228.01	0.01\\
229.01	0.01\\
230.01	0.01\\
231.01	0.01\\
232.01	0.01\\
233.01	0.01\\
234.01	0.01\\
235.01	0.01\\
236.01	0.01\\
237.01	0.01\\
238.01	0.01\\
239.01	0.01\\
240.01	0.01\\
241.01	0.01\\
242.01	0.01\\
243.01	0.01\\
244.01	0.01\\
245.01	0.01\\
246.01	0.01\\
247.01	0.01\\
248.01	0.01\\
249.01	0.01\\
250.01	0.01\\
251.01	0.01\\
252.01	0.01\\
253.01	0.01\\
254.01	0.01\\
255.01	0.01\\
256.01	0.01\\
257.01	0.01\\
258.01	0.01\\
259.01	0.01\\
260.01	0.01\\
261.01	0.01\\
262.01	0.01\\
263.01	0.01\\
264.01	0.01\\
265.01	0.01\\
266.01	0.01\\
267.01	0.01\\
268.01	0.01\\
269.01	0.01\\
270.01	0.01\\
271.01	0.01\\
272.01	0.01\\
273.01	0.01\\
274.01	0.01\\
275.01	0.01\\
276.01	0.01\\
277.01	0.01\\
278.01	0.01\\
279.01	0.01\\
280.01	0.01\\
281.01	0.01\\
282.01	0.01\\
283.01	0.01\\
284.01	0.01\\
285.01	0.01\\
286.01	0.01\\
287.01	0.01\\
288.01	0.01\\
289.01	0.01\\
290.01	0.01\\
291.01	0.01\\
292.01	0.01\\
293.01	0.01\\
294.01	0.01\\
295.01	0.01\\
296.01	0.01\\
297.01	0.01\\
298.01	0.01\\
299.01	0.01\\
300.01	0.01\\
301.01	0.01\\
302.01	0.01\\
303.01	0.01\\
304.01	0.01\\
305.01	0.01\\
306.01	0.01\\
307.01	0.01\\
308.01	0.01\\
309.01	0.01\\
310.01	0.01\\
311.01	0.01\\
312.01	0.01\\
313.01	0.01\\
314.01	0.01\\
315.01	0.01\\
316.01	0.01\\
317.01	0.01\\
318.01	0.01\\
319.01	0.01\\
320.01	0.01\\
321.01	0.01\\
322.01	0.01\\
323.01	0.01\\
324.01	0.01\\
325.01	0.01\\
326.01	0.01\\
327.01	0.01\\
328.01	0.01\\
329.01	0.01\\
330.01	0.01\\
331.01	0.01\\
332.01	0.01\\
333.01	0.01\\
334.01	0.01\\
335.01	0.01\\
336.01	0.01\\
337.01	0.01\\
338.01	0.01\\
339.01	0.01\\
340.01	0.01\\
341.01	0.01\\
342.01	0.01\\
343.01	0.01\\
344.01	0.01\\
345.01	0.01\\
346.01	0.01\\
347.01	0.01\\
348.01	0.01\\
349.01	0.01\\
350.01	0.01\\
351.01	0.01\\
352.01	0.01\\
353.01	0.01\\
354.01	0.01\\
355.01	0.01\\
356.01	0.01\\
357.01	0.01\\
358.01	0.01\\
359.01	0.01\\
360.01	0.01\\
361.01	0.01\\
362.01	0.01\\
363.01	0.01\\
364.01	0.01\\
365.01	0.01\\
366.01	0.01\\
367.01	0.01\\
368.01	0.01\\
369.01	0.01\\
370.01	0.01\\
371.01	0.01\\
372.01	0.01\\
373.01	0.01\\
374.01	0.01\\
375.01	0.01\\
376.01	0.01\\
377.01	0.01\\
378.01	0.01\\
379.01	0.01\\
380.01	0.01\\
381.01	0.01\\
382.01	0.01\\
383.01	0.01\\
384.01	0.01\\
385.01	0.01\\
386.01	0.01\\
387.01	0.01\\
388.01	0.01\\
389.01	0.01\\
390.01	0.01\\
391.01	0.01\\
392.01	0.01\\
393.01	0.01\\
394.01	0.01\\
395.01	0.01\\
396.01	0.01\\
397.01	0.01\\
398.01	0.01\\
399.01	0.01\\
400.01	0.01\\
401.01	0.01\\
402.01	0.01\\
403.01	0.01\\
404.01	0.01\\
405.01	0.01\\
406.01	0.01\\
407.01	0.01\\
408.01	0.01\\
409.01	0.01\\
410.01	0.01\\
411.01	0.01\\
412.01	0.01\\
413.01	0.01\\
414.01	0.01\\
415.01	0.01\\
416.01	0.01\\
417.01	0.01\\
418.01	0.01\\
419.01	0.01\\
420.01	0.01\\
421.01	0.01\\
422.01	0.01\\
423.01	0.01\\
424.01	0.01\\
425.01	0.01\\
426.01	0.01\\
427.01	0.01\\
428.01	0.01\\
429.01	0.01\\
430.01	0.01\\
431.01	0.01\\
432.01	0.01\\
433.01	0.01\\
434.01	0.01\\
435.01	0.01\\
436.01	0.01\\
437.01	0.01\\
438.01	0.01\\
439.01	0.01\\
440.01	0.01\\
441.01	0.01\\
442.01	0.01\\
443.01	0.01\\
444.01	0.01\\
445.01	0.01\\
446.01	0.01\\
447.01	0.01\\
448.01	0.01\\
449.01	0.01\\
450.01	0.01\\
451.01	0.01\\
452.01	0.01\\
453.01	0.01\\
454.01	0.01\\
455.01	0.01\\
456.01	0.01\\
457.01	0.01\\
458.01	0.01\\
459.01	0.01\\
460.01	0.01\\
461.01	0.01\\
462.01	0.01\\
463.01	0.01\\
464.01	0.01\\
465.01	0.01\\
466.01	0.01\\
467.01	0.01\\
468.01	0.01\\
469.01	0.01\\
470.01	0.01\\
471.01	0.01\\
472.01	0.01\\
473.01	0.01\\
474.01	0.01\\
475.01	0.01\\
476.01	0.01\\
477.01	0.01\\
478.01	0.01\\
479.01	0.01\\
480.01	0.01\\
481.01	0.01\\
482.01	0.01\\
483.01	0.01\\
484.01	0.01\\
485.01	0.01\\
486.01	0.01\\
487.01	0.01\\
488.01	0.01\\
489.01	0.01\\
490.01	0.01\\
491.01	0.01\\
492.01	0.01\\
493.01	0.01\\
494.01	0.01\\
495.01	0.01\\
496.01	0.01\\
497.01	0.01\\
498.01	0.01\\
499.01	0.01\\
500.01	0.01\\
501.01	0.01\\
502.01	0.01\\
503.01	0.01\\
504.01	0.01\\
505.01	0.01\\
506.01	0.01\\
507.01	0.01\\
508.01	0.01\\
509.01	0.01\\
510.01	0.01\\
511.01	0.01\\
512.01	0.01\\
513.01	0.01\\
514.01	0.01\\
515.01	0.01\\
516.01	0.01\\
517.01	0.01\\
518.01	0.01\\
519.01	0.01\\
520.01	0.01\\
521.01	0.01\\
522.01	0.01\\
523.01	0.01\\
524.01	0.01\\
525.01	0.01\\
526.01	0.01\\
527.01	0.01\\
528.01	0.01\\
529.01	0.01\\
530.01	0.01\\
531.01	0.01\\
532.01	0.01\\
533.01	0.01\\
534.01	0.01\\
535.01	0.01\\
536.01	0.01\\
537.01	0.01\\
538.01	0.01\\
539.01	0.01\\
540.01	0.01\\
541.01	0.01\\
542.01	0.01\\
543.01	0.01\\
544.01	0.01\\
545.01	0.01\\
546.01	0.01\\
547.01	0.01\\
548.01	0.01\\
549.01	0.01\\
550.01	0.01\\
551.01	0.01\\
552.01	0.01\\
553.01	0.01\\
554.01	0.01\\
555.01	0.01\\
556.01	0.01\\
557.01	0.01\\
558.01	0.01\\
559.01	0.01\\
560.01	0.01\\
561.01	0.01\\
562.01	0.01\\
563.01	0.01\\
564.01	0.01\\
565.01	0.01\\
566.01	0.01\\
567.01	0.01\\
568.01	0.01\\
569.01	0.01\\
570.01	0.01\\
571.01	0.01\\
572.01	0.01\\
573.01	0.01\\
574.01	0.01\\
575.01	0.01\\
576.01	0.01\\
577.01	0.01\\
578.01	0.01\\
579.01	0.01\\
580.01	0.01\\
581.01	0.01\\
582.01	0.01\\
583.01	0.01\\
584.01	0.01\\
585.01	0.01\\
586.01	0.01\\
587.01	0.01\\
588.01	0.01\\
589.01	0.01\\
590.01	0.01\\
591.01	0.01\\
592.01	0.01\\
593.01	0.01\\
594.01	0.01\\
595.01	0.01\\
596.01	0.01\\
597.01	0.01\\
598.01	0.01\\
599.01	0.01\\
599.02	0.01\\
599.03	0.01\\
599.04	0.01\\
599.05	0.01\\
599.06	0.01\\
599.07	0.01\\
599.08	0.01\\
599.09	0.01\\
599.1	0.01\\
599.11	0.01\\
599.12	0.01\\
599.13	0.01\\
599.14	0.01\\
599.15	0.01\\
599.16	0.01\\
599.17	0.01\\
599.18	0.01\\
599.19	0.01\\
599.2	0.01\\
599.21	0.01\\
599.22	0.01\\
599.23	0.01\\
599.24	0.01\\
599.25	0.01\\
599.26	0.01\\
599.27	0.01\\
599.28	0.01\\
599.29	0.01\\
599.3	0.01\\
599.31	0.01\\
599.32	0.01\\
599.33	0.01\\
599.34	0.01\\
599.35	0.01\\
599.36	0.01\\
599.37	0.01\\
599.38	0.01\\
599.39	0.01\\
599.4	0.01\\
599.41	0.01\\
599.42	0.01\\
599.43	0.01\\
599.44	0.01\\
599.45	0.01\\
599.46	0.01\\
599.47	0.01\\
599.48	0.01\\
599.49	0.00988638009963208\\
599.5	0.00971984661000477\\
599.51	0.00955246047742145\\
599.52	0.00938421686791766\\
599.53	0.00921511031874022\\
599.54	0.00904513189583664\\
599.55	0.00887427136911025\\
599.56	0.00870251783210568\\
599.57	0.00852985946949492\\
599.58	0.00835628409069571\\
599.59	0.00818177911813606\\
599.6	0.00800633157515677\\
599.61	0.00782992807353735\\
599.62	0.00765255494364272\\
599.63	0.00747419798352703\\
599.64	0.00729484249349292\\
599.65	0.00711447330785863\\
599.66	0.00693307478015587\\
599.67	0.00675063076798984\\
599.68	0.00656712461739275\\
599.69	0.00638253914664802\\
599.7	0.00619685662956123\\
599.71	0.00601005877815255\\
599.72	0.00582212672474394\\
599.73	0.00563304100341284\\
599.74	0.00544278152954363\\
599.75	0.0052513275795273\\
599.76	0.00505865776998709\\
599.77	0.00486475003623244\\
599.78	0.00466958160990404\\
599.79	0.00447312899252071\\
599.8	0.00427536793068844\\
599.81	0.0040762733851378\\
599.82	0.00387581950088586\\
599.83	0.00367397957887031\\
599.84	0.00347072604640581\\
599.85	0.00326603042640224\\
599.86	0.00305986330528188\\
599.87	0.00285219429952965\\
599.88	0.00264299202080817\\
599.89	0.00243222403956698\\
599.9	0.00221985684707294\\
599.91	0.00200585581578775\\
599.92	0.00179018515801705\\
599.93	0.00157280788275587\\
599.94	0.00135368575065601\\
599.95	0.00113277922704381\\
599.96	0.000910047432921331\\
599.97	0.000685448093891365\\
599.98	0.000458937486957849\\
599.99	0.00023047038516868\\
600	0\\
};
\addplot [color=mycolor13,solid,forget plot]
  table[row sep=crcr]{%
0.01	0\\
1.01	0\\
2.01	0\\
3.01	0\\
4.01	0\\
5.01	0\\
6.01	0\\
7.01	0\\
8.01	0\\
9.01	0\\
10.01	0\\
11.01	0\\
12.01	0\\
13.01	0\\
14.01	0\\
15.01	0\\
16.01	0\\
17.01	0\\
18.01	0\\
19.01	0\\
20.01	0\\
21.01	0\\
22.01	0\\
23.01	0\\
24.01	0\\
25.01	0\\
26.01	0\\
27.01	0\\
28.01	0\\
29.01	0\\
30.01	0\\
31.01	0\\
32.01	0\\
33.01	0\\
34.01	0\\
35.01	0\\
36.01	0\\
37.01	0\\
38.01	0\\
39.01	0\\
40.01	0\\
41.01	0\\
42.01	0\\
43.01	0\\
44.01	0\\
45.01	0\\
46.01	0\\
47.01	0\\
48.01	0\\
49.01	0\\
50.01	0\\
51.01	0\\
52.01	0\\
53.01	0\\
54.01	0\\
55.01	0\\
56.01	0\\
57.01	0\\
58.01	0\\
59.01	0\\
60.01	0\\
61.01	0\\
62.01	0\\
63.01	0\\
64.01	0\\
65.01	0\\
66.01	0\\
67.01	0\\
68.01	0\\
69.01	0\\
70.01	0\\
71.01	0\\
72.01	0\\
73.01	0\\
74.01	0\\
75.01	0\\
76.01	0\\
77.01	0\\
78.01	0\\
79.01	0\\
80.01	0\\
81.01	0\\
82.01	0\\
83.01	0\\
84.01	0\\
85.01	0\\
86.01	0\\
87.01	0\\
88.01	0\\
89.01	0\\
90.01	0\\
91.01	0\\
92.01	0\\
93.01	0\\
94.01	0\\
95.01	0\\
96.01	0\\
97.01	0\\
98.01	0\\
99.01	0\\
100.01	0\\
101.01	0\\
102.01	0\\
103.01	0\\
104.01	0\\
105.01	0\\
106.01	0\\
107.01	0\\
108.01	0\\
109.01	0\\
110.01	0\\
111.01	0\\
112.01	0\\
113.01	0\\
114.01	0\\
115.01	0\\
116.01	0\\
117.01	0\\
118.01	0\\
119.01	0\\
120.01	0\\
121.01	0\\
122.01	0\\
123.01	0\\
124.01	0\\
125.01	0\\
126.01	0\\
127.01	0\\
128.01	0\\
129.01	0\\
130.01	0\\
131.01	0\\
132.01	0\\
133.01	0\\
134.01	0\\
135.01	0\\
136.01	0\\
137.01	0\\
138.01	0\\
139.01	0\\
140.01	0\\
141.01	0\\
142.01	0\\
143.01	0\\
144.01	0\\
145.01	0\\
146.01	0\\
147.01	0\\
148.01	0\\
149.01	0\\
150.01	0\\
151.01	0\\
152.01	0\\
153.01	0\\
154.01	0\\
155.01	0\\
156.01	0\\
157.01	0\\
158.01	0\\
159.01	0\\
160.01	0\\
161.01	0\\
162.01	0\\
163.01	0\\
164.01	0\\
165.01	0\\
166.01	0\\
167.01	0\\
168.01	0\\
169.01	0\\
170.01	0\\
171.01	0\\
172.01	0\\
173.01	0\\
174.01	0\\
175.01	0\\
176.01	0\\
177.01	0\\
178.01	0\\
179.01	0\\
180.01	0\\
181.01	0\\
182.01	0\\
183.01	0\\
184.01	0\\
185.01	0\\
186.01	0\\
187.01	0\\
188.01	0\\
189.01	0\\
190.01	0\\
191.01	0\\
192.01	0\\
193.01	0\\
194.01	0\\
195.01	0\\
196.01	0\\
197.01	0\\
198.01	0\\
199.01	0\\
200.01	0\\
201.01	0\\
202.01	0\\
203.01	0\\
204.01	0\\
205.01	0\\
206.01	0\\
207.01	0\\
208.01	0\\
209.01	0\\
210.01	0\\
211.01	0\\
212.01	0\\
213.01	0\\
214.01	0\\
215.01	0\\
216.01	0\\
217.01	0\\
218.01	0\\
219.01	0\\
220.01	0\\
221.01	0\\
222.01	0\\
223.01	0\\
224.01	0\\
225.01	0\\
226.01	0\\
227.01	0\\
228.01	0\\
229.01	0\\
230.01	0\\
231.01	0\\
232.01	0\\
233.01	0\\
234.01	0\\
235.01	0\\
236.01	0\\
237.01	0\\
238.01	0\\
239.01	0\\
240.01	0\\
241.01	0\\
242.01	0\\
243.01	0\\
244.01	0\\
245.01	0\\
246.01	0\\
247.01	0\\
248.01	0\\
249.01	0\\
250.01	0\\
251.01	0\\
252.01	0\\
253.01	0\\
254.01	0\\
255.01	0\\
256.01	0\\
257.01	0\\
258.01	0\\
259.01	0\\
260.01	0\\
261.01	0\\
262.01	0\\
263.01	0\\
264.01	0\\
265.01	0\\
266.01	0\\
267.01	0\\
268.01	0\\
269.01	0\\
270.01	0\\
271.01	0\\
272.01	0\\
273.01	0\\
274.01	0\\
275.01	0\\
276.01	0\\
277.01	0\\
278.01	0\\
279.01	0\\
280.01	0\\
281.01	0\\
282.01	0\\
283.01	0\\
284.01	0\\
285.01	0\\
286.01	0\\
287.01	0\\
288.01	0\\
289.01	0\\
290.01	0\\
291.01	0\\
292.01	0\\
293.01	0\\
294.01	0\\
295.01	0\\
296.01	0\\
297.01	0\\
298.01	0\\
299.01	0\\
300.01	0\\
301.01	0\\
302.01	0\\
303.01	0\\
304.01	0\\
305.01	0\\
306.01	0\\
307.01	0\\
308.01	0\\
309.01	0\\
310.01	0\\
311.01	0\\
312.01	0\\
313.01	0\\
314.01	0\\
315.01	0\\
316.01	0\\
317.01	0\\
318.01	0\\
319.01	0\\
320.01	0\\
321.01	0\\
322.01	0\\
323.01	0\\
324.01	0\\
325.01	0\\
326.01	0\\
327.01	0\\
328.01	0\\
329.01	0\\
330.01	0\\
331.01	0\\
332.01	0\\
333.01	0\\
334.01	0\\
335.01	0\\
336.01	0\\
337.01	0\\
338.01	0\\
339.01	0\\
340.01	0\\
341.01	0\\
342.01	0\\
343.01	0\\
344.01	0\\
345.01	0\\
346.01	0\\
347.01	0\\
348.01	0\\
349.01	0\\
350.01	0\\
351.01	0\\
352.01	0\\
353.01	0\\
354.01	0\\
355.01	0\\
356.01	0\\
357.01	0\\
358.01	0\\
359.01	0\\
360.01	0\\
361.01	0\\
362.01	0\\
363.01	0\\
364.01	0\\
365.01	0\\
366.01	0\\
367.01	0\\
368.01	0\\
369.01	0\\
370.01	0\\
371.01	0\\
372.01	0\\
373.01	0\\
374.01	0\\
375.01	0\\
376.01	0\\
377.01	0\\
378.01	0\\
379.01	0\\
380.01	0\\
381.01	0\\
382.01	0\\
383.01	0\\
384.01	0\\
385.01	0\\
386.01	0\\
387.01	0\\
388.01	0\\
389.01	0\\
390.01	0\\
391.01	0\\
392.01	0\\
393.01	0\\
394.01	0\\
395.01	0\\
396.01	0\\
397.01	0\\
398.01	0\\
399.01	0\\
400.01	0\\
401.01	0\\
402.01	0\\
403.01	0\\
404.01	0\\
405.01	0\\
406.01	0\\
407.01	0\\
408.01	0\\
409.01	0\\
410.01	0\\
411.01	0\\
412.01	0\\
413.01	0\\
414.01	0\\
415.01	0\\
416.01	0\\
417.01	0\\
418.01	0\\
419.01	0\\
420.01	0\\
421.01	0\\
422.01	0\\
423.01	0\\
424.01	0\\
425.01	0\\
426.01	0\\
427.01	0\\
428.01	0\\
429.01	0\\
430.01	0\\
431.01	0\\
432.01	0\\
433.01	0\\
434.01	0\\
435.01	0\\
436.01	0\\
437.01	0\\
438.01	0\\
439.01	0\\
440.01	0\\
441.01	0\\
442.01	0\\
443.01	0\\
444.01	0\\
445.01	0\\
446.01	0\\
447.01	0\\
448.01	0\\
449.01	0\\
450.01	0\\
451.01	0\\
452.01	0\\
453.01	0\\
454.01	0\\
455.01	0\\
456.01	0\\
457.01	0\\
458.01	0\\
459.01	0\\
460.01	0\\
461.01	0\\
462.01	0\\
463.01	0\\
464.01	0\\
465.01	0\\
466.01	0\\
467.01	0\\
468.01	0\\
469.01	0\\
470.01	0\\
471.01	0\\
472.01	0\\
473.01	0\\
474.01	0\\
475.01	0\\
476.01	0\\
477.01	0\\
478.01	0\\
479.01	0\\
480.01	0\\
481.01	0\\
482.01	0\\
483.01	0\\
484.01	0\\
485.01	0\\
486.01	0\\
487.01	0\\
488.01	0\\
489.01	0\\
490.01	0\\
491.01	0\\
492.01	0\\
493.01	0\\
494.01	0\\
495.01	0\\
496.01	0\\
497.01	0\\
498.01	0\\
499.01	0\\
500.01	0\\
501.01	0\\
502.01	0\\
503.01	0\\
504.01	0\\
505.01	0\\
506.01	0\\
507.01	0\\
508.01	0\\
509.01	0\\
510.01	0\\
511.01	0\\
512.01	0\\
513.01	0\\
514.01	0\\
515.01	0\\
516.01	0\\
517.01	0\\
518.01	0\\
519.01	0\\
520.01	0\\
521.01	0\\
522.01	0\\
523.01	0\\
524.01	0\\
525.01	0\\
526.01	0\\
527.01	0\\
528.01	0\\
529.01	0\\
530.01	0\\
531.01	0\\
532.01	0\\
533.01	0\\
534.01	0\\
535.01	0\\
536.01	0\\
537.01	0\\
538.01	0\\
539.01	0\\
540.01	0\\
541.01	0\\
542.01	0\\
543.01	0\\
544.01	0\\
545.01	0\\
546.01	0\\
547.01	0\\
548.01	0\\
549.01	0\\
550.01	0\\
551.01	0\\
552.01	0\\
553.01	0\\
554.01	0\\
555.01	0\\
556.01	0\\
557.01	0\\
558.01	0\\
559.01	0\\
560.01	0\\
561.01	0\\
562.01	0\\
563.01	0\\
564.01	0\\
565.01	0\\
566.01	0\\
567.01	0\\
568.01	0\\
569.01	0\\
570.01	0\\
571.01	0\\
572.01	0\\
573.01	0\\
574.01	0\\
575.01	0\\
576.01	0\\
577.01	0\\
578.01	0\\
579.01	0\\
580.01	0\\
581.01	0\\
582.01	0\\
583.01	0\\
584.01	0\\
585.01	0\\
586.01	0.000142170676920603\\
587.01	0.000355128567196075\\
588.01	0.000573748988783096\\
589.01	0.00079851512560985\\
590.01	0.00102998128508491\\
591.01	0.00126877990469422\\
592.01	0.00151563757222356\\
593.01	0.00177139624661609\\
594.01	0.00203704594793394\\
595.01	0.00231380797226837\\
596.01	0.00260357560208499\\
597.01	0.0029123341470069\\
598.01	0.00328029355590324\\
599.01	0.00407681843730328\\
599.02	0.00409226336446341\\
599.03	0.00410800018922777\\
599.04	0.00412403637215258\\
599.05	0.00414037958426491\\
599.06	0.00415703771382205\\
599.07	0.0041740188733256\\
599.08	0.00419133140680166\\
599.09	0.00420898389735902\\
599.1	0.00422698517503799\\
599.11	0.00424534432496329\\
599.12	0.00426407069581482\\
599.13	0.00428317390863123\\
599.14	0.0043026638659619\\
599.15	0.0043225507613836\\
599.16	0.00434284508939945\\
599.17	0.00436355765573831\\
599.18	0.00438469958807417\\
599.19	0.00440628234718589\\
599.2	0.00442831773857909\\
599.21	0.00445081792654077\\
599.22	0.00447379544795006\\
599.23	0.00449726322260612\\
599.24	0.00452123456695878\\
599.25	0.00454572320843149\\
599.26	0.00457074330036714\\
599.27	0.00459630944765488\\
599.28	0.00462243672849619\\
599.29	0.0046491406975709\\
599.3	0.00467643740391242\\
599.31	0.00470434342570141\\
599.32	0.00473287589009266\\
599.33	0.00476205247943\\
599.34	0.00479189145277651\\
599.35	0.00482241171590761\\
599.36	0.00485363281887117\\
599.37	0.00488557496399592\\
599.38	0.00491825903212873\\
599.39	0.0049517066101656\\
599.4	0.00498594001995394\\
599.41	0.00502098234864931\\
599.42	0.00505685748061574\\
599.43	0.00509359013096528\\
599.44	0.00513120588083937\\
599.45	0.00516973121454266\\
599.46	0.00520919355864768\\
599.47	0.00524962132319844\\
599.48	0.00529104394515042\\
599.49	0.00533348993758731\\
599.5	0.00537698981313269\\
599.51	0.00542157628050479\\
599.52	0.00546728327083941\\
599.53	0.00551414153003291\\
599.54	0.00556217430889\\
599.55	0.00561141758723835\\
599.56	0.00566190175434073\\
599.57	0.00571365261530301\\
599.58	0.00576670966414243\\
599.59	0.00582107484520414\\
599.6	0.00587678781064857\\
599.61	0.00593389135595841\\
599.62	0.00599242992818502\\
599.63	0.00605244970697346\\
599.64	0.00611399869066074\\
599.65	0.0061771267878398\\
599.66	0.0062418859148182\\
599.67	0.00630833009953979\\
599.68	0.00637651559233562\\
599.69	0.00644650098408254\\
599.7	0.00651834733242259\\
599.71	0.0065921182967225\\
599.72	0.00666788028252302\\
599.73	0.00674570259633623\\
599.74	0.00682565761164589\\
599.75	0.00690782094712556\\
599.76	0.00699227165820456\\
599.77	0.00707909244321486\\
599.78	0.00716836986547855\\
599.79	0.00726019459286861\\
599.8	0.00735466165654207\\
599.81	0.0074518707307418\\
599.82	0.00755192643578738\\
599.83	0.00765493866663039\\
599.84	0.00776102294964022\\
599.85	0.0078703008306186\\
599.86	0.0079829002974221\\
599.87	0.00809895624100861\\
599.88	0.00821861095922719\\
599.89	0.00834201470825166\\
599.9	0.00846932630723019\\
599.91	0.00860071380250349\\
599.92	0.00873635519865198\\
599.93	0.00887643926469208\\
599.94	0.00902116642498313\\
599.95	0.00917074974586409\\
599.96	0.00932541603075827\\
599.97	0.00948540703851739\\
599.98	0.00965098084219061\\
599.99	0.00982241334828153\\
600	0.01\\
};
\addplot [color=mycolor14,solid,forget plot]
  table[row sep=crcr]{%
0.01	0.00371232059378437\\
1.01	0.00371232059383302\\
2.01	0.00371232059388272\\
3.01	0.00371232059393349\\
4.01	0.00371232059398535\\
5.01	0.00371232059403833\\
6.01	0.00371232059409247\\
7.01	0.00371232059414777\\
8.01	0.00371232059420427\\
9.01	0.003712320594262\\
10.01	0.00371232059432096\\
11.01	0.0037123205943812\\
12.01	0.00371232059444275\\
13.01	0.00371232059450564\\
14.01	0.00371232059456988\\
15.01	0.00371232059463551\\
16.01	0.00371232059470256\\
17.01	0.00371232059477106\\
18.01	0.00371232059484106\\
19.01	0.00371232059491256\\
20.01	0.00371232059498561\\
21.01	0.00371232059506025\\
22.01	0.00371232059513651\\
23.01	0.00371232059521442\\
24.01	0.00371232059529402\\
25.01	0.00371232059537534\\
26.01	0.00371232059545843\\
27.01	0.00371232059554332\\
28.01	0.00371232059563005\\
29.01	0.00371232059571867\\
30.01	0.00371232059580921\\
31.01	0.00371232059590171\\
32.01	0.00371232059599623\\
33.01	0.0037123205960928\\
34.01	0.00371232059619147\\
35.01	0.00371232059629228\\
36.01	0.00371232059639528\\
37.01	0.00371232059650052\\
38.01	0.00371232059660805\\
39.01	0.00371232059671792\\
40.01	0.00371232059683017\\
41.01	0.00371232059694488\\
42.01	0.00371232059706207\\
43.01	0.00371232059718182\\
44.01	0.00371232059730417\\
45.01	0.00371232059742919\\
46.01	0.00371232059755694\\
47.01	0.00371232059768746\\
48.01	0.00371232059782083\\
49.01	0.00371232059795711\\
50.01	0.00371232059809636\\
51.01	0.00371232059823864\\
52.01	0.00371232059838403\\
53.01	0.00371232059853259\\
54.01	0.00371232059868438\\
55.01	0.0037123205988395\\
56.01	0.003712320598998\\
57.01	0.00371232059915996\\
58.01	0.00371232059932546\\
59.01	0.00371232059949457\\
60.01	0.00371232059966738\\
61.01	0.00371232059984397\\
62.01	0.00371232060002442\\
63.01	0.00371232060020881\\
64.01	0.00371232060039724\\
65.01	0.00371232060058978\\
66.01	0.00371232060078655\\
67.01	0.00371232060098762\\
68.01	0.00371232060119309\\
69.01	0.00371232060140306\\
70.01	0.00371232060161764\\
71.01	0.00371232060183691\\
72.01	0.00371232060206099\\
73.01	0.00371232060228998\\
74.01	0.00371232060252399\\
75.01	0.00371232060276314\\
76.01	0.00371232060300754\\
77.01	0.00371232060325729\\
78.01	0.00371232060351253\\
79.01	0.00371232060377339\\
80.01	0.00371232060403996\\
81.01	0.00371232060431239\\
82.01	0.00371232060459081\\
83.01	0.00371232060487535\\
84.01	0.00371232060516614\\
85.01	0.00371232060546334\\
86.01	0.00371232060576707\\
87.01	0.00371232060607748\\
88.01	0.00371232060639473\\
89.01	0.00371232060671898\\
90.01	0.00371232060705036\\
91.01	0.00371232060738904\\
92.01	0.00371232060773519\\
93.01	0.00371232060808896\\
94.01	0.00371232060845054\\
95.01	0.0037123206088201\\
96.01	0.00371232060919782\\
97.01	0.00371232060958387\\
98.01	0.00371232060997845\\
99.01	0.00371232061038174\\
100.01	0.00371232061079394\\
101.01	0.00371232061121525\\
102.01	0.00371232061164588\\
103.01	0.00371232061208604\\
104.01	0.00371232061253593\\
105.01	0.00371232061299578\\
106.01	0.00371232061346581\\
107.01	0.00371232061394625\\
108.01	0.00371232061443734\\
109.01	0.0037123206149393\\
110.01	0.00371232061545239\\
111.01	0.00371232061597687\\
112.01	0.00371232061651298\\
113.01	0.00371232061706098\\
114.01	0.00371232061762115\\
115.01	0.00371232061819376\\
116.01	0.00371232061877909\\
117.01	0.00371232061937743\\
118.01	0.00371232061998907\\
119.01	0.00371232062061432\\
120.01	0.00371232062125347\\
121.01	0.00371232062190684\\
122.01	0.00371232062257477\\
123.01	0.00371232062325757\\
124.01	0.00371232062395558\\
125.01	0.00371232062466915\\
126.01	0.00371232062539863\\
127.01	0.00371232062614438\\
128.01	0.00371232062690677\\
129.01	0.00371232062768619\\
130.01	0.003712320628483\\
131.01	0.00371232062929762\\
132.01	0.00371232063013044\\
133.01	0.00371232063098188\\
134.01	0.00371232063185236\\
135.01	0.00371232063274232\\
136.01	0.00371232063365221\\
137.01	0.00371232063458247\\
138.01	0.00371232063553356\\
139.01	0.00371232063650597\\
140.01	0.00371232063750019\\
141.01	0.0037123206385167\\
142.01	0.00371232063955604\\
143.01	0.00371232064061869\\
144.01	0.00371232064170521\\
145.01	0.00371232064281615\\
146.01	0.00371232064395205\\
147.01	0.00371232064511351\\
148.01	0.00371232064630108\\
149.01	0.00371232064751539\\
150.01	0.00371232064875705\\
151.01	0.00371232065002667\\
152.01	0.0037123206513249\\
153.01	0.00371232065265241\\
154.01	0.00371232065400985\\
155.01	0.00371232065539794\\
156.01	0.00371232065681736\\
157.01	0.00371232065826884\\
158.01	0.00371232065975312\\
159.01	0.00371232066127098\\
160.01	0.00371232066282315\\
161.01	0.00371232066441046\\
162.01	0.00371232066603369\\
163.01	0.00371232066769369\\
164.01	0.00371232066939131\\
165.01	0.00371232067112742\\
166.01	0.00371232067290291\\
167.01	0.0037123206747187\\
168.01	0.00371232067657572\\
169.01	0.00371232067847492\\
170.01	0.00371232068041728\\
171.01	0.00371232068240382\\
172.01	0.00371232068443554\\
173.01	0.00371232068651351\\
174.01	0.0037123206886388\\
175.01	0.00371232069081252\\
176.01	0.00371232069303579\\
177.01	0.00371232069530977\\
178.01	0.00371232069763563\\
179.01	0.00371232070001459\\
180.01	0.0037123207024479\\
181.01	0.00371232070493682\\
182.01	0.00371232070748265\\
183.01	0.00371232071008673\\
184.01	0.00371232071275042\\
185.01	0.0037123207154751\\
186.01	0.00371232071826223\\
187.01	0.00371232072111325\\
188.01	0.00371232072402968\\
189.01	0.00371232072701302\\
190.01	0.00371232073006491\\
191.01	0.0037123207331869\\
192.01	0.00371232073638066\\
193.01	0.00371232073964787\\
194.01	0.00371232074299028\\
195.01	0.00371232074640965\\
196.01	0.0037123207499078\\
197.01	0.00371232075348659\\
198.01	0.00371232075714793\\
199.01	0.00371232076089375\\
200.01	0.00371232076472607\\
201.01	0.00371232076864693\\
202.01	0.00371232077265842\\
203.01	0.0037123207767627\\
204.01	0.00371232078096196\\
205.01	0.00371232078525845\\
206.01	0.00371232078965449\\
207.01	0.00371232079415246\\
208.01	0.00371232079875472\\
209.01	0.00371232080346381\\
210.01	0.00371232080828226\\
211.01	0.00371232081321267\\
212.01	0.0037123208182577\\
213.01	0.00371232082342009\\
214.01	0.00371232082870263\\
215.01	0.00371232083410821\\
216.01	0.00371232083963975\\
217.01	0.00371232084530027\\
218.01	0.00371232085109285\\
219.01	0.00371232085702064\\
220.01	0.00371232086308692\\
221.01	0.00371232086929496\\
222.01	0.00371232087564818\\
223.01	0.00371232088215005\\
224.01	0.00371232088880417\\
225.01	0.00371232089561415\\
226.01	0.00371232090258377\\
227.01	0.00371232090971685\\
228.01	0.00371232091701734\\
229.01	0.00371232092448927\\
230.01	0.00371232093213675\\
231.01	0.00371232093996402\\
232.01	0.00371232094797543\\
233.01	0.00371232095617542\\
234.01	0.00371232096456853\\
235.01	0.00371232097315944\\
236.01	0.00371232098195293\\
237.01	0.00371232099095387\\
238.01	0.00371232100016732\\
239.01	0.00371232100959841\\
240.01	0.00371232101925242\\
241.01	0.00371232102913474\\
242.01	0.00371232103925093\\
243.01	0.00371232104960664\\
244.01	0.0037123210602077\\
245.01	0.00371232107106008\\
246.01	0.00371232108216986\\
247.01	0.00371232109354332\\
248.01	0.00371232110518687\\
249.01	0.00371232111710712\\
250.01	0.00371232112931075\\
251.01	0.00371232114180471\\
252.01	0.00371232115459606\\
253.01	0.00371232116769208\\
254.01	0.00371232118110014\\
255.01	0.00371232119482793\\
256.01	0.00371232120888322\\
257.01	0.00371232122327404\\
258.01	0.00371232123800859\\
259.01	0.00371232125309527\\
260.01	0.0037123212685427\\
261.01	0.00371232128435971\\
262.01	0.00371232130055537\\
263.01	0.00371232131713894\\
264.01	0.00371232133411994\\
265.01	0.00371232135150813\\
266.01	0.00371232136931347\\
267.01	0.00371232138754624\\
268.01	0.00371232140621689\\
269.01	0.00371232142533625\\
270.01	0.00371232144491527\\
271.01	0.0037123214649653\\
272.01	0.00371232148549791\\
273.01	0.00371232150652497\\
274.01	0.00371232152805866\\
275.01	0.00371232155011144\\
276.01	0.00371232157269612\\
277.01	0.0037123215958258\\
278.01	0.0037123216195139\\
279.01	0.00371232164377418\\
280.01	0.00371232166862078\\
281.01	0.00371232169406816\\
282.01	0.00371232172013114\\
283.01	0.00371232174682492\\
284.01	0.00371232177416507\\
285.01	0.00371232180216756\\
286.01	0.00371232183084877\\
287.01	0.00371232186022544\\
288.01	0.00371232189031481\\
289.01	0.00371232192113446\\
290.01	0.00371232195270249\\
291.01	0.00371232198503738\\
292.01	0.00371232201815815\\
293.01	0.00371232205208421\\
294.01	0.00371232208683553\\
295.01	0.00371232212243253\\
296.01	0.00371232215889617\\
297.01	0.00371232219624792\\
298.01	0.00371232223450978\\
299.01	0.00371232227370431\\
300.01	0.00371232231385464\\
301.01	0.00371232235498447\\
302.01	0.00371232239711811\\
303.01	0.00371232244028043\\
304.01	0.00371232248449698\\
305.01	0.0037123225297939\\
306.01	0.00371232257619805\\
307.01	0.0037123226237369\\
308.01	0.00371232267243861\\
309.01	0.00371232272233209\\
310.01	0.00371232277344695\\
311.01	0.00371232282581353\\
312.01	0.00371232287946294\\
313.01	0.0037123229344271\\
314.01	0.00371232299073868\\
315.01	0.00371232304843119\\
316.01	0.00371232310753902\\
317.01	0.00371232316809736\\
318.01	0.00371232323014234\\
319.01	0.00371232329371096\\
320.01	0.00371232335884119\\
321.01	0.0037123234255719\\
322.01	0.00371232349394301\\
323.01	0.00371232356399543\\
324.01	0.00371232363577105\\
325.01	0.00371232370931289\\
326.01	0.00371232378466501\\
327.01	0.00371232386187263\\
328.01	0.00371232394098209\\
329.01	0.00371232402204089\\
330.01	0.0037123241050978\\
331.01	0.00371232419020279\\
332.01	0.00371232427740711\\
333.01	0.00371232436676332\\
334.01	0.00371232445832535\\
335.01	0.00371232455214848\\
336.01	0.00371232464828945\\
337.01	0.00371232474680644\\
338.01	0.00371232484775911\\
339.01	0.00371232495120873\\
340.01	0.00371232505721808\\
341.01	0.00371232516585164\\
342.01	0.00371232527717554\\
343.01	0.00371232539125763\\
344.01	0.00371232550816756\\
345.01	0.00371232562797679\\
346.01	0.0037123257507587\\
347.01	0.00371232587658854\\
348.01	0.00371232600554365\\
349.01	0.00371232613770333\\
350.01	0.00371232627314907\\
351.01	0.00371232641196453\\
352.01	0.00371232655423558\\
353.01	0.00371232670005044\\
354.01	0.00371232684949972\\
355.01	0.00371232700267647\\
356.01	0.00371232715967632\\
357.01	0.00371232732059748\\
358.01	0.00371232748554088\\
359.01	0.00371232765461025\\
360.01	0.00371232782791217\\
361.01	0.00371232800555623\\
362.01	0.00371232818765508\\
363.01	0.00371232837432455\\
364.01	0.00371232856568371\\
365.01	0.00371232876185505\\
366.01	0.00371232896296457\\
367.01	0.00371232916914186\\
368.01	0.00371232938052026\\
369.01	0.00371232959723702\\
370.01	0.00371232981943338\\
371.01	0.00371233004725469\\
372.01	0.00371233028085067\\
373.01	0.00371233052037542\\
374.01	0.00371233076598773\\
375.01	0.00371233101785108\\
376.01	0.00371233127613399\\
377.01	0.00371233154101006\\
378.01	0.00371233181265826\\
379.01	0.00371233209126305\\
380.01	0.00371233237701473\\
381.01	0.00371233267010947\\
382.01	0.00371233297074968\\
383.01	0.00371233327914428\\
384.01	0.0037123335955088\\
385.01	0.00371233392006582\\
386.01	0.00371233425304517\\
387.01	0.00371233459468429\\
388.01	0.00371233494522846\\
389.01	0.00371233530493123\\
390.01	0.00371233567405473\\
391.01	0.00371233605287007\\
392.01	0.00371233644165774\\
393.01	0.00371233684070799\\
394.01	0.00371233725032134\\
395.01	0.00371233767080906\\
396.01	0.00371233810249358\\
397.01	0.00371233854570914\\
398.01	0.00371233900080227\\
399.01	0.00371233946813244\\
400.01	0.0037123399480727\\
401.01	0.00371234044101025\\
402.01	0.0037123409473473\\
403.01	0.00371234146750167\\
404.01	0.00371234200190762\\
405.01	0.00371234255101666\\
406.01	0.00371234311529841\\
407.01	0.0037123436952414\\
408.01	0.003712344291354\\
409.01	0.00371234490416529\\
410.01	0.00371234553422596\\
411.01	0.00371234618210913\\
412.01	0.00371234684841086\\
413.01	0.00371234753374937\\
414.01	0.00371234823876087\\
415.01	0.00371234896410251\\
416.01	0.00371234971051709\\
417.01	0.00371235047897498\\
418.01	0.00371235127189158\\
419.01	0.00371235210202903\\
420.01	0.00371235305360244\\
421.01	0.00371235463003312\\
422.01	0.00371235776114621\\
423.01	0.00371236120668805\\
424.01	0.00371236475388728\\
425.01	0.00371236840621751\\
426.01	0.00371237216728881\\
427.01	0.00371237604085278\\
428.01	0.00371238003080758\\
429.01	0.00371238414120308\\
430.01	0.0037123883762459\\
431.01	0.0037123927403043\\
432.01	0.00371239723791314\\
433.01	0.00371240187377847\\
434.01	0.00371240665278212\\
435.01	0.00371241157998598\\
436.01	0.00371241666063594\\
437.01	0.00371242190016565\\
438.01	0.00371242730419993\\
439.01	0.00371243287855778\\
440.01	0.00371243862925526\\
441.01	0.00371244456250806\\
442.01	0.00371245068473388\\
443.01	0.00371245700255509\\
444.01	0.00371246352280165\\
445.01	0.00371247025251468\\
446.01	0.00371247719895162\\
447.01	0.00371248436959319\\
448.01	0.00371249177215348\\
449.01	0.00371249941459339\\
450.01	0.00371250730513867\\
451.01	0.00371251545231206\\
452.01	0.00371252386500198\\
453.01	0.00371253255250386\\
454.01	0.00371254152459322\\
455.01	0.00371255079175736\\
456.01	0.00371256036607533\\
457.01	0.00371257026634798\\
458.01	0.00371258054933302\\
459.01	0.00371259147149347\\
460.01	0.003712603890558\\
461.01	0.0037126178219913\\
462.01	0.00371263219023903\\
463.01	0.0037126469724045\\
464.01	0.0037126621818956\\
465.01	0.00371267783273444\\
466.01	0.00371269393958189\\
467.01	0.00371271051771323\\
468.01	0.00371272758290139\\
469.01	0.00371274515163655\\
470.01	0.003712763242706\\
471.01	0.00371278187972719\\
472.01	0.00371280110797478\\
473.01	0.00371282105340721\\
474.01	0.00371284190039475\\
475.01	0.0037128634861283\\
476.01	0.00371288572413619\\
477.01	0.00371290864348864\\
478.01	0.00371293227573453\\
479.01	0.00371295665495277\\
480.01	0.00371298181803624\\
481.01	0.00371300780501231\\
482.01	0.00371303465940402\\
483.01	0.00371306242863533\\
484.01	0.00371309116447902\\
485.01	0.00371312092355296\\
486.01	0.00371315176798094\\
487.01	0.00371318376624932\\
488.01	0.00371321699369153\\
489.01	0.00371325153336621\\
490.01	0.00371328747728075\\
491.01	0.00371332492859875\\
492.01	0.00371336401075783\\
493.01	0.00371340492535683\\
494.01	0.00371344835289228\\
495.01	0.00371349807380068\\
496.01	0.00371357169769694\\
497.01	0.00371366827917016\\
498.01	0.00371376798644745\\
499.01	0.00371387090780026\\
500.01	0.00371397718638241\\
501.01	0.00371408697533663\\
502.01	0.00371420043877106\\
503.01	0.00371431775287138\\
504.01	0.00371443910796197\\
505.01	0.00371456471072978\\
506.01	0.00371469478465389\\
507.01	0.00371482957266362\\
508.01	0.00371496934003496\\
509.01	0.00371511437754475\\
510.01	0.00371526500522995\\
511.01	0.00371542157705445\\
512.01	0.00371558448791376\\
513.01	0.00371575419213165\\
514.01	0.00371593129393625\\
515.01	0.00371611707591379\\
516.01	0.00371631609988022\\
517.01	0.00371654061402795\\
518.01	0.0037167859127879\\
519.01	0.00371704263620946\\
520.01	0.00371731463846147\\
521.01	0.00371762255867969\\
522.01	0.00371805578781655\\
523.01	0.0037185876132974\\
524.01	0.00371913708452109\\
525.01	0.00371970512564631\\
526.01	0.00372029274048168\\
527.01	0.00372090101722667\\
528.01	0.00372153113909118\\
529.01	0.00372218439768938\\
530.01	0.00372286221314574\\
531.01	0.00372356620137852\\
532.01	0.00372429859800109\\
533.01	0.00372506559641567\\
534.01	0.00372590708245823\\
535.01	0.0037271657759248\\
536.01	0.00372880840188161\\
537.01	0.00373053250235488\\
538.01	0.00373234484324824\\
539.01	0.00373425920144489\\
540.01	0.00373632476310768\\
541.01	0.00373881151959034\\
542.01	0.00374264026881439\\
543.01	0.00374815218611865\\
544.01	0.0037538952752454\\
545.01	0.00375987008897414\\
546.01	0.00376610063725892\\
547.01	0.00377261378869254\\
548.01	0.00377945235751856\\
549.01	0.00378677440762343\\
550.01	0.00379555663285403\\
551.01	0.00380905636103867\\
552.01	0.00382361115705503\\
553.01	0.00383876210916985\\
554.01	0.00385455756218182\\
555.01	0.00387105201588933\\
556.01	0.00388830767127321\\
557.01	0.00390640016670082\\
558.01	0.00392545874195979\\
559.01	0.0039460192508848\\
560.01	0.00397268468044839\\
561.01	0.00401480471382403\\
562.01	0.00406056486634537\\
563.01	0.00411002754511294\\
564.01	0.0041612123977801\\
565.01	0.00421413501518444\\
566.01	0.00426897077941056\\
567.01	0.00432600352534416\\
568.01	0.00438542430483328\\
569.01	0.00444744552069263\\
570.01	0.00451229689357401\\
571.01	0.00458025456503107\\
572.01	0.00465204500330351\\
573.01	0.00473271924694896\\
574.01	0.00486279822256183\\
575.01	0.00501940891484108\\
576.01	0.00517995413927412\\
577.01	0.00534458503903426\\
578.01	0.00551327045537046\\
579.01	0.00568626589840416\\
580.01	0.00586390647491116\\
581.01	0.00604657367145632\\
582.01	0.00623470318704096\\
583.01	0.00642878031955761\\
584.01	0.00662933650587675\\
585.01	0.00684825160576811\\
586.01	0.00703475413857012\\
587.01	0.00717428404336578\\
588.01	0.00731761703393303\\
589.01	0.00746489083537659\\
590.01	0.00761623766918582\\
591.01	0.00777178314411316\\
592.01	0.00793164049198838\\
593.01	0.00809590288331832\\
594.01	0.00826463697850347\\
595.01	0.00843791101233693\\
596.01	0.00861615191248504\\
597.01	0.00880342261792005\\
598.01	0.00903592442409936\\
599.01	0.00957365461413272\\
599.02	0.00958277041975926\\
599.03	0.00959197264387863\\
599.04	0.00960125988934567\\
599.05	0.00961063057421246\\
599.06	0.00962008292013701\\
599.07	0.00962961494015077\\
599.08	0.00963922442574839\\
599.09	0.00964890893326474\\
599.1	0.00965866576950219\\
599.11	0.00966849197656794\\
599.12	0.00967838431587352\\
599.13	0.00968833925125193\\
599.14	0.0096983529311477\\
599.15	0.00970842116982741\\
599.16	0.00971853942755756\\
599.17	0.009728702789693\\
599.18	0.00973890594461645\\
599.19	0.00974914316046814\\
599.2	0.00975940826059864\\
599.21	0.00976969321646363\\
599.22	0.00977998868251903\\
599.23	0.00979028683966976\\
599.24	0.00980057930540565\\
599.25	0.00981085710148202\\
599.26	0.0098211106198357\\
599.27	0.00983132243721219\\
599.28	0.0098414706723303\\
599.29	0.00985154283965676\\
599.3	0.00986152563527719\\
599.31	0.00987139336310963\\
599.32	0.00988111946709059\\
599.33	0.00989068690280136\\
599.34	0.0099000775652852\\
599.35	0.0099092381529734\\
599.36	0.00991813369390734\\
599.37	0.00992674039223167\\
599.38	0.00993503304638513\\
599.39	0.00994298497352877\\
599.4	0.00995056793008674\\
599.41	0.00995775202821507\\
599.42	0.00996450564800631\\
599.43	0.00997079534523213\\
599.44	0.00997658575441854\\
599.45	0.00998183948704209\\
599.46	0.00998651702462867\\
599.47	0.0099905766065302\\
599.48	0.0099939741121483\\
599.49	0.0099966649971801\\
599.5	0.00999860126363529\\
599.51	0.00999973114745285\\
599.52	0.01\\
599.53	0.01\\
599.54	0.01\\
599.55	0.01\\
599.56	0.01\\
599.57	0.01\\
599.58	0.01\\
599.59	0.01\\
599.6	0.01\\
599.61	0.01\\
599.62	0.01\\
599.63	0.01\\
599.64	0.01\\
599.65	0.01\\
599.66	0.01\\
599.67	0.01\\
599.68	0.01\\
599.69	0.01\\
599.7	0.01\\
599.71	0.01\\
599.72	0.01\\
599.73	0.01\\
599.74	0.01\\
599.75	0.01\\
599.76	0.01\\
599.77	0.01\\
599.78	0.01\\
599.79	0.01\\
599.8	0.01\\
599.81	0.01\\
599.82	0.01\\
599.83	0.01\\
599.84	0.01\\
599.85	0.01\\
599.86	0.01\\
599.87	0.01\\
599.88	0.01\\
599.89	0.01\\
599.9	0.01\\
599.91	0.01\\
599.92	0.01\\
599.93	0.01\\
599.94	0.01\\
599.95	0.01\\
599.96	0.01\\
599.97	0.01\\
599.98	0.01\\
599.99	0.01\\
600	0.01\\
};
\addplot [color=mycolor15,solid,forget plot]
  table[row sep=crcr]{%
0.01	0.00622155120829262\\
1.01	0.00622155120887005\\
2.01	0.00622155120945996\\
3.01	0.00622155121006258\\
4.01	0.00622155121067824\\
5.01	0.00622155121130722\\
6.01	0.00622155121194978\\
7.01	0.00622155121260623\\
8.01	0.00622155121327687\\
9.01	0.00622155121396201\\
10.01	0.00622155121466198\\
11.01	0.0062215512153771\\
12.01	0.00622155121610768\\
13.01	0.00622155121685406\\
14.01	0.00622155121761659\\
15.01	0.00622155121839564\\
16.01	0.00622155121919155\\
17.01	0.0062215512200047\\
18.01	0.00622155122083545\\
19.01	0.00622155122168422\\
20.01	0.00622155122255136\\
21.01	0.0062215512234373\\
22.01	0.00622155122434243\\
23.01	0.00622155122526719\\
24.01	0.00622155122621201\\
25.01	0.0062215512271773\\
26.01	0.00622155122816355\\
27.01	0.00622155122917119\\
28.01	0.00622155123020068\\
29.01	0.00622155123125252\\
30.01	0.00622155123232719\\
31.01	0.00622155123342521\\
32.01	0.00622155123454706\\
33.01	0.00622155123569328\\
34.01	0.0062215512368644\\
35.01	0.00622155123806097\\
36.01	0.00622155123928354\\
37.01	0.0062215512405327\\
38.01	0.006221551241809\\
39.01	0.00622155124311307\\
40.01	0.0062215512444455\\
41.01	0.00622155124580691\\
42.01	0.00622155124719796\\
43.01	0.00622155124861928\\
44.01	0.00622155125007153\\
45.01	0.00622155125155546\\
46.01	0.00622155125307165\\
47.01	0.00622155125462087\\
48.01	0.00622155125620387\\
49.01	0.00622155125782135\\
50.01	0.00622155125947409\\
51.01	0.00622155126116289\\
52.01	0.00622155126288849\\
53.01	0.00622155126465175\\
54.01	0.00622155126645348\\
55.01	0.00622155126829452\\
56.01	0.00622155127017576\\
57.01	0.00622155127209806\\
58.01	0.00622155127406237\\
59.01	0.00622155127606956\\
60.01	0.00622155127812062\\
61.01	0.00622155128021649\\
62.01	0.00622155128235819\\
63.01	0.00622155128454673\\
64.01	0.00622155128678311\\
65.01	0.00622155128906842\\
66.01	0.00622155129140374\\
67.01	0.00622155129379017\\
68.01	0.00622155129622886\\
69.01	0.00622155129872094\\
70.01	0.00622155130126761\\
71.01	0.00622155130387008\\
72.01	0.00622155130652957\\
73.01	0.00622155130924737\\
74.01	0.00622155131202476\\
75.01	0.00622155131486306\\
76.01	0.00622155131776365\\
77.01	0.00622155132072787\\
78.01	0.00622155132375717\\
79.01	0.00622155132685297\\
80.01	0.00622155133001677\\
81.01	0.00622155133325006\\
82.01	0.00622155133655442\\
83.01	0.0062215513399314\\
84.01	0.00622155134338264\\
85.01	0.00622155134690977\\
86.01	0.00622155135051451\\
87.01	0.00622155135419857\\
88.01	0.00622155135796372\\
89.01	0.00622155136181178\\
90.01	0.0062215513657446\\
91.01	0.00622155136976405\\
92.01	0.0062215513738721\\
93.01	0.00622155137807071\\
94.01	0.00622155138236189\\
95.01	0.00622155138674776\\
96.01	0.00622155139123038\\
97.01	0.00622155139581195\\
98.01	0.00622155140049468\\
99.01	0.00622155140528083\\
100.01	0.00622155141017272\\
101.01	0.00622155141517272\\
102.01	0.00622155142028325\\
103.01	0.00622155142550681\\
104.01	0.00622155143084591\\
105.01	0.00622155143630316\\
106.01	0.00622155144188121\\
107.01	0.00622155144758276\\
108.01	0.00622155145341064\\
109.01	0.00622155145936764\\
110.01	0.0062215514654567\\
111.01	0.00622155147168074\\
112.01	0.00622155147804284\\
113.01	0.00622155148454611\\
114.01	0.00622155149119374\\
115.01	0.00622155149798896\\
116.01	0.00622155150493512\\
117.01	0.00622155151203561\\
118.01	0.00622155151929394\\
119.01	0.00622155152671365\\
120.01	0.00622155153429843\\
121.01	0.00622155154205196\\
122.01	0.00622155154997809\\
123.01	0.00622155155808072\\
124.01	0.00622155156636384\\
125.01	0.00622155157483155\\
126.01	0.00622155158348804\\
127.01	0.00622155159233758\\
128.01	0.00622155160138456\\
129.01	0.00622155161063344\\
130.01	0.00622155162008883\\
131.01	0.00622155162975541\\
132.01	0.00622155163963799\\
133.01	0.00622155164974149\\
134.01	0.00622155166007092\\
135.01	0.00622155167063143\\
136.01	0.0062215516814283\\
137.01	0.00622155169246688\\
138.01	0.00622155170375274\\
139.01	0.00622155171529148\\
140.01	0.00622155172708886\\
141.01	0.00622155173915083\\
142.01	0.00622155175148341\\
143.01	0.00622155176409279\\
144.01	0.0062215517769853\\
145.01	0.00622155179016744\\
146.01	0.00622155180364584\\
147.01	0.00622155181742724\\
148.01	0.00622155183151864\\
149.01	0.00622155184592715\\
150.01	0.00622155186066002\\
151.01	0.00622155187572472\\
152.01	0.00622155189112885\\
153.01	0.00622155190688023\\
154.01	0.00622155192298685\\
155.01	0.00622155193945687\\
156.01	0.00622155195629866\\
157.01	0.00622155197352082\\
158.01	0.00622155199113207\\
159.01	0.00622155200914142\\
160.01	0.00622155202755804\\
161.01	0.00622155204639131\\
162.01	0.00622155206565091\\
163.01	0.00622155208534666\\
164.01	0.00622155210548866\\
165.01	0.00622155212608725\\
166.01	0.00622155214715297\\
167.01	0.00622155216869668\\
168.01	0.00622155219072945\\
169.01	0.00622155221326262\\
170.01	0.00622155223630781\\
171.01	0.0062215522598769\\
172.01	0.00622155228398209\\
173.01	0.00622155230863583\\
174.01	0.00622155233385086\\
175.01	0.00622155235964032\\
176.01	0.00622155238601752\\
177.01	0.00622155241299619\\
178.01	0.00622155244059035\\
179.01	0.00622155246881438\\
180.01	0.00622155249768297\\
181.01	0.0062215525272112\\
182.01	0.00622155255741447\\
183.01	0.00622155258830858\\
184.01	0.00622155261990972\\
185.01	0.00622155265223444\\
186.01	0.00622155268529968\\
187.01	0.00622155271912284\\
188.01	0.00622155275372167\\
189.01	0.00622155278911444\\
190.01	0.00622155282531972\\
191.01	0.00622155286235669\\
192.01	0.00622155290024489\\
193.01	0.00622155293900436\\
194.01	0.00622155297865559\\
195.01	0.00622155301921964\\
196.01	0.00622155306071802\\
197.01	0.00622155310317279\\
198.01	0.00622155314660652\\
199.01	0.00622155319104236\\
200.01	0.00622155323650397\\
201.01	0.00622155328301566\\
202.01	0.00622155333060226\\
203.01	0.00622155337928923\\
204.01	0.00622155342910266\\
205.01	0.00622155348006928\\
206.01	0.00622155353221643\\
207.01	0.00622155358557215\\
208.01	0.0062215536401652\\
209.01	0.00622155369602495\\
210.01	0.00622155375318155\\
211.01	0.00622155381166588\\
212.01	0.00622155387150955\\
213.01	0.006221553932745\\
214.01	0.00622155399540538\\
215.01	0.00622155405952473\\
216.01	0.00622155412513788\\
217.01	0.00622155419228052\\
218.01	0.00622155426098922\\
219.01	0.00622155433130148\\
220.01	0.00622155440325561\\
221.01	0.006221554476891\\
222.01	0.00622155455224797\\
223.01	0.00622155462936774\\
224.01	0.00622155470829266\\
225.01	0.00622155478906609\\
226.01	0.00622155487173239\\
227.01	0.0062215549563371\\
228.01	0.00622155504292686\\
229.01	0.00622155513154942\\
230.01	0.00622155522225376\\
231.01	0.00622155531509003\\
232.01	0.00622155541010964\\
233.01	0.00622155550736529\\
234.01	0.00622155560691094\\
235.01	0.00622155570880189\\
236.01	0.00622155581309484\\
237.01	0.00622155591984788\\
238.01	0.00622155602912051\\
239.01	0.00622155614097376\\
240.01	0.00622155625547011\\
241.01	0.00622155637267364\\
242.01	0.00622155649264999\\
243.01	0.00622155661546641\\
244.01	0.00622155674119189\\
245.01	0.00622155686989707\\
246.01	0.00622155700165436\\
247.01	0.00622155713653796\\
248.01	0.00622155727462399\\
249.01	0.00622155741599027\\
250.01	0.00622155756071679\\
251.01	0.0062215577088854\\
252.01	0.00622155786058\\
253.01	0.00622155801588656\\
254.01	0.00622155817489329\\
255.01	0.00622155833769045\\
256.01	0.00622155850437065\\
257.01	0.00622155867502878\\
258.01	0.00622155884976208\\
259.01	0.00622155902867026\\
260.01	0.00622155921185544\\
261.01	0.00622155939942235\\
262.01	0.00622155959147829\\
263.01	0.00622155978813329\\
264.01	0.00622155998950005\\
265.01	0.00622156019569416\\
266.01	0.00622156040683405\\
267.01	0.00622156062304108\\
268.01	0.00622156084443972\\
269.01	0.00622156107115744\\
270.01	0.00622156130332501\\
271.01	0.00622156154107637\\
272.01	0.00622156178454885\\
273.01	0.00622156203388321\\
274.01	0.0062215622892237\\
275.01	0.00622156255071821\\
276.01	0.00622156281851827\\
277.01	0.00622156309277926\\
278.01	0.0062215633736604\\
279.01	0.0062215636613249\\
280.01	0.00622156395594006\\
281.01	0.00622156425767731\\
282.01	0.00622156456671241\\
283.01	0.00622156488322552\\
284.01	0.00622156520740129\\
285.01	0.00622156553942895\\
286.01	0.0062215658795025\\
287.01	0.00622156622782078\\
288.01	0.00622156658458758\\
289.01	0.00622156695001182\\
290.01	0.0062215673243076\\
291.01	0.0062215677076944\\
292.01	0.00622156810039717\\
293.01	0.0062215685026465\\
294.01	0.0062215689146787\\
295.01	0.00622156933673608\\
296.01	0.00622156976906689\\
297.01	0.00622157021192568\\
298.01	0.00622157066557331\\
299.01	0.0062215711302772\\
300.01	0.00622157160631143\\
301.01	0.00622157209395695\\
302.01	0.00622157259350171\\
303.01	0.00622157310524091\\
304.01	0.00622157362947712\\
305.01	0.00622157416652048\\
306.01	0.00622157471668884\\
307.01	0.00622157528030811\\
308.01	0.00622157585771229\\
309.01	0.00622157644924378\\
310.01	0.00622157705525353\\
311.01	0.0062215776761013\\
312.01	0.00622157831215591\\
313.01	0.00622157896379535\\
314.01	0.00622157963140718\\
315.01	0.00622158031538865\\
316.01	0.00622158101614698\\
317.01	0.00622158173409967\\
318.01	0.00622158246967468\\
319.01	0.00622158322331077\\
320.01	0.00622158399545772\\
321.01	0.00622158478657667\\
322.01	0.0062215855971404\\
323.01	0.0062215864276336\\
324.01	0.00622158727855324\\
325.01	0.00622158815040886\\
326.01	0.00622158904372287\\
327.01	0.00622158995903096\\
328.01	0.00622159089688241\\
329.01	0.00622159185784045\\
330.01	0.00622159284248266\\
331.01	0.00622159385140135\\
332.01	0.00622159488520392\\
333.01	0.00622159594451332\\
334.01	0.00622159702996845\\
335.01	0.0062215981422246\\
336.01	0.0062215992819539\\
337.01	0.00622160044984581\\
338.01	0.00622160164660756\\
339.01	0.00622160287296467\\
340.01	0.00622160412966148\\
341.01	0.00622160541746165\\
342.01	0.00622160673714875\\
343.01	0.0062216080895268\\
344.01	0.00622160947542085\\
345.01	0.00622161089567762\\
346.01	0.00622161235116614\\
347.01	0.00622161384277835\\
348.01	0.00622161537142986\\
349.01	0.00622161693806054\\
350.01	0.0062216185436354\\
351.01	0.00622162018914516\\
352.01	0.00622162187560722\\
353.01	0.00622162360406636\\
354.01	0.00622162537559557\\
355.01	0.00622162719129703\\
356.01	0.0062216290523029\\
357.01	0.00622163095977639\\
358.01	0.00622163291491256\\
359.01	0.00622163491893956\\
360.01	0.00622163697311954\\
361.01	0.00622163907874975\\
362.01	0.00622164123716379\\
363.01	0.00622164344973269\\
364.01	0.00622164571786625\\
365.01	0.00622164804301427\\
366.01	0.00622165042666785\\
367.01	0.00622165287036091\\
368.01	0.0062216553756716\\
369.01	0.00622165794422379\\
370.01	0.00622166057768865\\
371.01	0.00622166327778641\\
372.01	0.00622166604628798\\
373.01	0.00622166888501681\\
374.01	0.00622167179585077\\
375.01	0.00622167478072413\\
376.01	0.0062216778416296\\
377.01	0.00622168098062057\\
378.01	0.00622168419981326\\
379.01	0.00622168750138925\\
380.01	0.00622169088759773\\
381.01	0.00622169436075843\\
382.01	0.00622169792326414\\
383.01	0.00622170157758361\\
384.01	0.00622170532626473\\
385.01	0.0062217091719376\\
386.01	0.00622171311731796\\
387.01	0.00622171716521068\\
388.01	0.00622172131851364\\
389.01	0.00622172558022155\\
390.01	0.00622172995343021\\
391.01	0.00622173444134093\\
392.01	0.00622173904726521\\
393.01	0.00622174377462969\\
394.01	0.00622174862698143\\
395.01	0.0062217536079934\\
396.01	0.00622175872147051\\
397.01	0.00622176397135567\\
398.01	0.00622176936173661\\
399.01	0.00622177489685269\\
400.01	0.00622178058110238\\
401.01	0.00622178641905114\\
402.01	0.00622179241543954\\
403.01	0.00622179857519203\\
404.01	0.00622180490342598\\
405.01	0.00622181140546122\\
406.01	0.00622181808682995\\
407.01	0.00622182495328689\\
408.01	0.00622183201081992\\
409.01	0.00622183926566072\\
410.01	0.00622184672429543\\
411.01	0.00622185439347518\\
412.01	0.00622186228022559\\
413.01	0.00622187039185207\\
414.01	0.00622187873592565\\
415.01	0.00622188732023299\\
416.01	0.0062218961530539\\
417.01	0.00622190524358895\\
418.01	0.00622191460404523\\
419.01	0.0062219242689194\\
420.01	0.00622193444799481\\
421.01	0.00622194712215251\\
422.01	0.006221979253888\\
423.01	0.00622202009866039\\
424.01	0.00622206214640011\\
425.01	0.0062221054381339\\
426.01	0.00622215001650164\\
427.01	0.00622219592581707\\
428.01	0.00622224321212885\\
429.01	0.00622229192328177\\
430.01	0.00622234210897748\\
431.01	0.00622239382083473\\
432.01	0.00622244711244828\\
433.01	0.00622250203944632\\
434.01	0.00622255865954566\\
435.01	0.0062226170326043\\
436.01	0.00622267722067105\\
437.01	0.00622273928803165\\
438.01	0.00622280330125103\\
439.01	0.00622286932921189\\
440.01	0.00622293744314948\\
441.01	0.00622300771668329\\
442.01	0.00622308022584675\\
443.01	0.00622315504911658\\
444.01	0.00622323226744481\\
445.01	0.00622331196429756\\
446.01	0.00622339422570616\\
447.01	0.0062234791403391\\
448.01	0.00622356679960483\\
449.01	0.00622365729779789\\
450.01	0.00622375073229723\\
451.01	0.00622384720384408\\
452.01	0.0062239468171431\\
453.01	0.00622404968153397\\
454.01	0.00622415591147117\\
455.01	0.00622426562807719\\
456.01	0.0062243789623651\\
457.01	0.00622449606955524\\
458.01	0.00622461722507646\\
459.01	0.00622474356308274\\
460.01	0.00622488233570331\\
461.01	0.00622504597766834\\
462.01	0.00622521630290586\\
463.01	0.00622539153089623\\
464.01	0.00622557181976172\\
465.01	0.00622575733485051\\
466.01	0.00622594824915053\\
467.01	0.00622614474340794\\
468.01	0.00622634700494363\\
469.01	0.00622655522548624\\
470.01	0.00622676961479236\\
471.01	0.00622699041375242\\
472.01	0.00622721795204145\\
473.01	0.00622745313423117\\
474.01	0.00622769881616883\\
475.01	0.00622795462605029\\
476.01	0.00622821814989739\\
477.01	0.0062284897063457\\
478.01	0.00622876966278281\\
479.01	0.00622905841569682\\
480.01	0.00622935639392405\\
481.01	0.00622966406231198\\
482.01	0.00622998192585361\\
483.01	0.00623031053434989\\
484.01	0.00623065048763709\\
485.01	0.0062310024413455\\
486.01	0.00623136711357164\\
487.01	0.00623174529462344\\
488.01	0.00623213785435975\\
489.01	0.00623254575046085\\
490.01	0.00623297004129636\\
491.01	0.00623341190095683\\
492.01	0.00623387264973774\\
493.01	0.00623435388759447\\
494.01	0.00623485840305036\\
495.01	0.00623539795663178\\
496.01	0.00623609367330455\\
497.01	0.00623723115650128\\
498.01	0.00623841312880773\\
499.01	0.00623963317120134\\
500.01	0.00624089297069207\\
501.01	0.00624219433087263\\
502.01	0.00624353918395286\\
503.01	0.00624492960313452\\
504.01	0.00624636782012887\\
505.01	0.00624785625643472\\
506.01	0.0062493975316999\\
507.01	0.00625099448791722\\
508.01	0.0062526502240204\\
509.01	0.00625436813154553\\
510.01	0.00625615193677643\\
511.01	0.00625800575114142\\
512.01	0.00625993413429125\\
513.01	0.00626194219242883\\
514.01	0.00626403587083316\\
515.01	0.00626622368312657\\
516.01	0.00626853002351242\\
517.01	0.00627108363368029\\
518.01	0.00627398335748705\\
519.01	0.00627701725843666\\
520.01	0.00628019768657443\\
521.01	0.00628359192946967\\
522.01	0.00628791812806813\\
523.01	0.00629425286810341\\
524.01	0.00630080273615997\\
525.01	0.00630757700598057\\
526.01	0.00631458789434046\\
527.01	0.00632184865783699\\
528.01	0.0063293737043835\\
529.01	0.00633717875027882\\
530.01	0.00634528099472029\\
531.01	0.00635369939877915\\
532.01	0.00636245555927423\\
533.01	0.00637157828048203\\
534.01	0.00638112238410973\\
535.01	0.00639099780968543\\
536.01	0.00640113448444666\\
537.01	0.00641176212081205\\
538.01	0.00642288681870064\\
539.01	0.00643456213066824\\
540.01	0.00644692533589876\\
541.01	0.00646090346237331\\
542.01	0.00648361374822135\\
543.01	0.00650939644223189\\
544.01	0.00653605287866877\\
545.01	0.00656357799123173\\
546.01	0.00659203582323319\\
547.01	0.00662155071739224\\
548.01	0.00665221862704111\\
549.01	0.00668441019219837\\
550.01	0.00671956868728407\\
551.01	0.00675457158607755\\
552.01	0.00679036743480562\\
553.01	0.00682746902595869\\
554.01	0.00686597622574857\\
555.01	0.00690600127058039\\
556.01	0.00694767174634376\\
557.01	0.00699114032408532\\
558.01	0.0070366419278237\\
559.01	0.00708482349212101\\
560.01	0.00713626857861373\\
561.01	0.0071829610180714\\
562.01	0.00724162846701401\\
563.01	0.00734597812578352\\
564.01	0.00745479814104576\\
565.01	0.00756487131968776\\
566.01	0.00767486113237245\\
567.01	0.00778744862957332\\
568.01	0.00790292696768786\\
569.01	0.00802148922935688\\
570.01	0.00814336003385193\\
571.01	0.00826886548535915\\
572.01	0.00839882785219053\\
573.01	0.0085370466661109\\
574.01	0.00866729652998845\\
575.01	0.00878465492073466\\
576.01	0.00890431818522714\\
577.01	0.00902628977288373\\
578.01	0.00915073393145681\\
579.01	0.00927770547942364\\
580.01	0.00940721524314349\\
581.01	0.00953925685629601\\
582.01	0.00967379261945761\\
583.01	0.00981066165926008\\
584.01	0.0099465256436832\\
585.01	0.01\\
586.01	0.01\\
587.01	0.01\\
588.01	0.01\\
589.01	0.01\\
590.01	0.01\\
591.01	0.01\\
592.01	0.01\\
593.01	0.01\\
594.01	0.01\\
595.01	0.01\\
596.01	0.01\\
597.01	0.01\\
598.01	0.01\\
599.01	0.01\\
599.02	0.01\\
599.03	0.01\\
599.04	0.01\\
599.05	0.01\\
599.06	0.01\\
599.07	0.01\\
599.08	0.01\\
599.09	0.01\\
599.1	0.01\\
599.11	0.01\\
599.12	0.01\\
599.13	0.01\\
599.14	0.01\\
599.15	0.01\\
599.16	0.01\\
599.17	0.01\\
599.18	0.01\\
599.19	0.01\\
599.2	0.01\\
599.21	0.01\\
599.22	0.01\\
599.23	0.01\\
599.24	0.01\\
599.25	0.01\\
599.26	0.01\\
599.27	0.01\\
599.28	0.01\\
599.29	0.01\\
599.3	0.01\\
599.31	0.01\\
599.32	0.01\\
599.33	0.01\\
599.34	0.01\\
599.35	0.01\\
599.36	0.01\\
599.37	0.01\\
599.38	0.01\\
599.39	0.01\\
599.4	0.01\\
599.41	0.01\\
599.42	0.01\\
599.43	0.01\\
599.44	0.01\\
599.45	0.01\\
599.46	0.01\\
599.47	0.01\\
599.48	0.01\\
599.49	0.01\\
599.5	0.01\\
599.51	0.01\\
599.52	0.01\\
599.53	0.01\\
599.54	0.01\\
599.55	0.01\\
599.56	0.01\\
599.57	0.01\\
599.58	0.01\\
599.59	0.01\\
599.6	0.01\\
599.61	0.01\\
599.62	0.01\\
599.63	0.01\\
599.64	0.01\\
599.65	0.01\\
599.66	0.01\\
599.67	0.01\\
599.68	0.01\\
599.69	0.01\\
599.7	0.01\\
599.71	0.01\\
599.72	0.01\\
599.73	0.01\\
599.74	0.01\\
599.75	0.01\\
599.76	0.01\\
599.77	0.01\\
599.78	0.01\\
599.79	0.01\\
599.8	0.01\\
599.81	0.01\\
599.82	0.01\\
599.83	0.01\\
599.84	0.01\\
599.85	0.01\\
599.86	0.01\\
599.87	0.01\\
599.88	0.01\\
599.89	0.01\\
599.9	0.01\\
599.91	0.01\\
599.92	0.01\\
599.93	0.01\\
599.94	0.01\\
599.95	0.01\\
599.96	0.01\\
599.97	0.01\\
599.98	0.01\\
599.99	0.01\\
600	0.01\\
};
\addplot [color=mycolor16,solid,forget plot]
  table[row sep=crcr]{%
0.01	0.0075997228867832\\
1.01	0.00759972289627518\\
2.01	0.00759972290597215\\
3.01	0.00759972291587861\\
4.01	0.00759972292599908\\
5.01	0.00759972293633824\\
6.01	0.00759972294690085\\
7.01	0.00759972295769177\\
8.01	0.00759972296871599\\
9.01	0.00759972297997857\\
10.01	0.00759972299148474\\
11.01	0.00759972300323978\\
12.01	0.00759972301524913\\
13.01	0.00759972302751832\\
14.01	0.00759972304005305\\
15.01	0.00759972305285909\\
16.01	0.00759972306594238\\
17.01	0.00759972307930895\\
18.01	0.00759972309296501\\
19.01	0.00759972310691687\\
20.01	0.00759972312117099\\
21.01	0.00759972313573398\\
22.01	0.00759972315061258\\
23.01	0.00759972316581372\\
24.01	0.00759972318134442\\
25.01	0.00759972319721191\\
26.01	0.00759972321342354\\
27.01	0.00759972322998687\\
28.01	0.00759972324690956\\
29.01	0.00759972326419952\\
30.01	0.00759972328186475\\
31.01	0.00759972329991352\\
32.01	0.00759972331835418\\
33.01	0.00759972333719536\\
34.01	0.00759972335644583\\
35.01	0.00759972337611456\\
36.01	0.00759972339621072\\
37.01	0.0075997234167437\\
38.01	0.00759972343772309\\
39.01	0.00759972345915866\\
40.01	0.00759972348106046\\
41.01	0.00759972350343872\\
42.01	0.00759972352630391\\
43.01	0.00759972354966672\\
44.01	0.00759972357353811\\
45.01	0.00759972359792922\\
46.01	0.00759972362285154\\
47.01	0.00759972364831672\\
48.01	0.00759972367433672\\
49.01	0.00759972370092373\\
50.01	0.00759972372809028\\
51.01	0.00759972375584908\\
52.01	0.00759972378421323\\
53.01	0.00759972381319604\\
54.01	0.00759972384281115\\
55.01	0.00759972387307254\\
56.01	0.00759972390399441\\
57.01	0.00759972393559138\\
58.01	0.00759972396787833\\
59.01	0.00759972400087051\\
60.01	0.00759972403458348\\
61.01	0.00759972406903316\\
62.01	0.00759972410423585\\
63.01	0.0075997241402082\\
64.01	0.00759972417696721\\
65.01	0.00759972421453028\\
66.01	0.00759972425291524\\
67.01	0.00759972429214024\\
68.01	0.00759972433222391\\
69.01	0.00759972437318526\\
70.01	0.00759972441504373\\
71.01	0.00759972445781921\\
72.01	0.00759972450153205\\
73.01	0.00759972454620303\\
74.01	0.0075997245918534\\
75.01	0.0075997246385049\\
76.01	0.00759972468617976\\
77.01	0.00759972473490073\\
78.01	0.00759972478469102\\
79.01	0.00759972483557442\\
80.01	0.00759972488757523\\
81.01	0.0075997249407183\\
82.01	0.00759972499502906\\
83.01	0.0075997250505335\\
84.01	0.00759972510725817\\
85.01	0.00759972516523029\\
86.01	0.00759972522447765\\
87.01	0.00759972528502864\\
88.01	0.00759972534691238\\
89.01	0.00759972541015857\\
90.01	0.00759972547479763\\
91.01	0.00759972554086066\\
92.01	0.00759972560837946\\
93.01	0.00759972567738655\\
94.01	0.00759972574791521\\
95.01	0.00759972581999944\\
96.01	0.00759972589367407\\
97.01	0.00759972596897467\\
98.01	0.00759972604593764\\
99.01	0.00759972612460022\\
100.01	0.00759972620500049\\
101.01	0.00759972628717741\\
102.01	0.00759972637117079\\
103.01	0.0075997264570214\\
104.01	0.00759972654477091\\
105.01	0.00759972663446199\\
106.01	0.00759972672613821\\
107.01	0.00759972681984419\\
108.01	0.00759972691562557\\
109.01	0.00759972701352902\\
110.01	0.00759972711360229\\
111.01	0.00759972721589423\\
112.01	0.0075997273204548\\
113.01	0.0075997274273351\\
114.01	0.00759972753658743\\
115.01	0.00759972764826528\\
116.01	0.00759972776242337\\
117.01	0.00759972787911767\\
118.01	0.00759972799840547\\
119.01	0.00759972812034534\\
120.01	0.00759972824499724\\
121.01	0.00759972837242248\\
122.01	0.00759972850268381\\
123.01	0.00759972863584542\\
124.01	0.00759972877197297\\
125.01	0.00759972891113368\\
126.01	0.00759972905339626\\
127.01	0.00759972919883109\\
128.01	0.0075997293475101\\
129.01	0.00759972949950697\\
130.01	0.00759972965489703\\
131.01	0.00759972981375736\\
132.01	0.00759972997616687\\
133.01	0.00759973014220627\\
134.01	0.00759973031195815\\
135.01	0.00759973048550703\\
136.01	0.00759973066293941\\
137.01	0.00759973084434378\\
138.01	0.00759973102981073\\
139.01	0.00759973121943288\\
140.01	0.00759973141330514\\
141.01	0.00759973161152451\\
142.01	0.00759973181419033\\
143.01	0.00759973202140427\\
144.01	0.00759973223327033\\
145.01	0.00759973244989497\\
146.01	0.00759973267138716\\
147.01	0.00759973289785838\\
148.01	0.00759973312942279\\
149.01	0.00759973336619716\\
150.01	0.00759973360830106\\
151.01	0.00759973385585684\\
152.01	0.00759973410898973\\
153.01	0.0075997343678279\\
154.01	0.00759973463250259\\
155.01	0.00759973490314805\\
156.01	0.00759973517990174\\
157.01	0.00759973546290439\\
158.01	0.00759973575229999\\
159.01	0.00759973604823598\\
160.01	0.00759973635086327\\
161.01	0.00759973666033635\\
162.01	0.00759973697681332\\
163.01	0.00759973730045613\\
164.01	0.00759973763143042\\
165.01	0.0075997379699059\\
166.01	0.00759973831605624\\
167.01	0.00759973867005919\\
168.01	0.00759973903209684\\
169.01	0.00759973940235551\\
170.01	0.007599739781026\\
171.01	0.00759974016830364\\
172.01	0.00759974056438843\\
173.01	0.0075997409694851\\
174.01	0.00759974138380334\\
175.01	0.00759974180755772\\
176.01	0.00759974224096808\\
177.01	0.00759974268425942\\
178.01	0.00759974313766217\\
179.01	0.00759974360141224\\
180.01	0.00759974407575121\\
181.01	0.00759974456092645\\
182.01	0.00759974505719126\\
183.01	0.00759974556480504\\
184.01	0.00759974608403338\\
185.01	0.00759974661514828\\
186.01	0.00759974715842828\\
187.01	0.00759974771415862\\
188.01	0.0075997482826314\\
189.01	0.00759974886414575\\
190.01	0.00759974945900806\\
191.01	0.00759975006753205\\
192.01	0.00759975069003905\\
193.01	0.00759975132685816\\
194.01	0.0075997519783264\\
195.01	0.00759975264478898\\
196.01	0.00759975332659944\\
197.01	0.00759975402411987\\
198.01	0.00759975473772119\\
199.01	0.00759975546778324\\
200.01	0.00759975621469514\\
201.01	0.00759975697885542\\
202.01	0.00759975776067233\\
203.01	0.00759975856056398\\
204.01	0.00759975937895871\\
205.01	0.00759976021629528\\
206.01	0.00759976107302309\\
207.01	0.00759976194960251\\
208.01	0.0075997628465052\\
209.01	0.00759976376421423\\
210.01	0.0075997647032245\\
211.01	0.00759976566404304\\
212.01	0.0075997666471892\\
213.01	0.00759976765319507\\
214.01	0.00759976868260577\\
215.01	0.00759976973597974\\
216.01	0.0075997708138891\\
217.01	0.00759977191692\\
218.01	0.00759977304567301\\
219.01	0.00759977420076332\\
220.01	0.00759977538282132\\
221.01	0.00759977659249287\\
222.01	0.00759977783043965\\
223.01	0.00759977909733968\\
224.01	0.00759978039388761\\
225.01	0.00759978172079519\\
226.01	0.00759978307879174\\
227.01	0.00759978446862448\\
228.01	0.00759978589105911\\
229.01	0.00759978734688018\\
230.01	0.0075997888368916\\
231.01	0.00759979036191713\\
232.01	0.00759979192280089\\
233.01	0.00759979352040779\\
234.01	0.00759979515562419\\
235.01	0.00759979682935834\\
236.01	0.00759979854254096\\
237.01	0.00759980029612582\\
238.01	0.00759980209109026\\
239.01	0.00759980392843586\\
240.01	0.00759980580918903\\
241.01	0.0075998077344016\\
242.01	0.00759980970515147\\
243.01	0.00759981172254335\\
244.01	0.00759981378770926\\
245.01	0.00759981590180939\\
246.01	0.00759981806603273\\
247.01	0.0075998202815978\\
248.01	0.00759982254975339\\
249.01	0.00759982487177936\\
250.01	0.00759982724898734\\
251.01	0.00759982968272164\\
252.01	0.00759983217436\\
253.01	0.0075998347253144\\
254.01	0.00759983733703203\\
255.01	0.00759984001099609\\
256.01	0.0075998427487267\\
257.01	0.00759984555178185\\
258.01	0.00759984842175841\\
259.01	0.00759985136029292\\
260.01	0.00759985436906287\\
261.01	0.00759985744978746\\
262.01	0.00759986060422879\\
263.01	0.00759986383419293\\
264.01	0.00759986714153099\\
265.01	0.00759987052814024\\
266.01	0.00759987399596534\\
267.01	0.00759987754699946\\
268.01	0.0075998811832855\\
269.01	0.0075998849069174\\
270.01	0.00759988872004135\\
271.01	0.00759989262485712\\
272.01	0.00759989662361944\\
273.01	0.00759990071863936\\
274.01	0.00759990491228559\\
275.01	0.00759990920698607\\
276.01	0.00759991360522938\\
277.01	0.00759991810956623\\
278.01	0.00759992272261109\\
279.01	0.00759992744704378\\
280.01	0.00759993228561101\\
281.01	0.00759993724112818\\
282.01	0.00759994231648104\\
283.01	0.00759994751462741\\
284.01	0.00759995283859906\\
285.01	0.00759995829150353\\
286.01	0.00759996387652602\\
287.01	0.00759996959693133\\
288.01	0.00759997545606586\\
289.01	0.00759998145735966\\
290.01	0.00759998760432855\\
291.01	0.00759999390057616\\
292.01	0.00760000034979624\\
293.01	0.00760000695577486\\
294.01	0.0076000137223928\\
295.01	0.00760002065362776\\
296.01	0.00760002775355698\\
297.01	0.00760003502635958\\
298.01	0.00760004247631926\\
299.01	0.00760005010782679\\
300.01	0.00760005792538275\\
301.01	0.00760006593360032\\
302.01	0.00760007413720808\\
303.01	0.00760008254105289\\
304.01	0.00760009115010298\\
305.01	0.00760009996945083\\
306.01	0.00760010900431646\\
307.01	0.00760011826005058\\
308.01	0.0076001277421379\\
309.01	0.00760013745620057\\
310.01	0.0076001474080016\\
311.01	0.00760015760344849\\
312.01	0.00760016804859684\\
313.01	0.00760017874965424\\
314.01	0.00760018971298399\\
315.01	0.00760020094510929\\
316.01	0.00760021245271712\\
317.01	0.00760022424266258\\
318.01	0.00760023632197322\\
319.01	0.00760024869785342\\
320.01	0.00760026137768905\\
321.01	0.00760027436905213\\
322.01	0.00760028767970563\\
323.01	0.0076003013176086\\
324.01	0.00760031529092114\\
325.01	0.00760032960800982\\
326.01	0.00760034427745301\\
327.01	0.00760035930804658\\
328.01	0.0076003747088096\\
329.01	0.00760039048899038\\
330.01	0.00760040665807249\\
331.01	0.00760042322578122\\
332.01	0.00760044020208998\\
333.01	0.00760045759722715\\
334.01	0.00760047542168291\\
335.01	0.00760049368621652\\
336.01	0.00760051240186368\\
337.01	0.00760053157994412\\
338.01	0.00760055123206961\\
339.01	0.00760057137015203\\
340.01	0.00760059200641189\\
341.01	0.00760061315338696\\
342.01	0.00760063482394134\\
343.01	0.00760065703127484\\
344.01	0.00760067978893257\\
345.01	0.00760070311081498\\
346.01	0.00760072701118814\\
347.01	0.00760075150469457\\
348.01	0.00760077660636424\\
349.01	0.0076008023316262\\
350.01	0.00760082869632034\\
351.01	0.00760085571670991\\
352.01	0.00760088340949429\\
353.01	0.00760091179182226\\
354.01	0.00760094088130588\\
355.01	0.00760097069603476\\
356.01	0.00760100125459091\\
357.01	0.00760103257606429\\
358.01	0.00760106468006884\\
359.01	0.00760109758675905\\
360.01	0.00760113131684741\\
361.01	0.00760116589162238\\
362.01	0.0076012013329672\\
363.01	0.00760123766337925\\
364.01	0.00760127490599049\\
365.01	0.00760131308458853\\
366.01	0.00760135222363861\\
367.01	0.00760139234830653\\
368.01	0.00760143348448254\\
369.01	0.00760147565880621\\
370.01	0.00760151889869246\\
371.01	0.00760156323235858\\
372.01	0.00760160868885252\\
373.01	0.00760165529808248\\
374.01	0.0076017030908477\\
375.01	0.00760175209887078\\
376.01	0.00760180235483138\\
377.01	0.00760185389240156\\
378.01	0.00760190674628281\\
379.01	0.00760196095224471\\
380.01	0.00760201654716569\\
381.01	0.00760207356907561\\
382.01	0.00760213205720063\\
383.01	0.00760219205201022\\
384.01	0.00760225359526671\\
385.01	0.00760231673007743\\
386.01	0.00760238150094952\\
387.01	0.00760244795384794\\
388.01	0.00760251613625645\\
389.01	0.00760258609724204\\
390.01	0.00760265788752318\\
391.01	0.00760273155954169\\
392.01	0.007602807167539\\
393.01	0.00760288476763681\\
394.01	0.00760296441792241\\
395.01	0.00760304617853924\\
396.01	0.0076031301117826\\
397.01	0.00760321628220133\\
398.01	0.00760330475670521\\
399.01	0.00760339560467894\\
400.01	0.00760348889810256\\
401.01	0.00760358471167878\\
402.01	0.00760368312296734\\
403.01	0.00760378421252653\\
404.01	0.00760388806406186\\
405.01	0.0076039947645817\\
406.01	0.00760410440455963\\
407.01	0.00760421707810279\\
408.01	0.00760433288312531\\
409.01	0.00760445192152548\\
410.01	0.00760457429936466\\
411.01	0.00760470012704498\\
412.01	0.0076048295194813\\
413.01	0.00760496259625344\\
414.01	0.00760509948166441\\
415.01	0.00760524030430592\\
416.01	0.00760538519787622\\
417.01	0.00760553430562554\\
418.01	0.00760568777766635\\
419.01	0.00760584580684255\\
420.01	0.00760600885696256\\
421.01	0.00760617863052147\\
422.01	0.00760634640848477\\
423.01	0.00760651319648749\\
424.01	0.00760668478246389\\
425.01	0.00760686132953937\\
426.01	0.0076070430081998\\
427.01	0.00760722999677242\\
428.01	0.00760742248195161\\
429.01	0.00760762065937462\\
430.01	0.00760782473425401\\
431.01	0.00760803492207397\\
432.01	0.00760825144935882\\
433.01	0.00760847455452393\\
434.01	0.00760870448882025\\
435.01	0.00760894151738591\\
436.01	0.00760918592042077\\
437.01	0.00760943799450191\\
438.01	0.00760969805406225\\
439.01	0.00760996643305728\\
440.01	0.00761024348685034\\
441.01	0.00761052959435185\\
442.01	0.00761082516045514\\
443.01	0.00761113061881869\\
444.01	0.00761144643505497\\
445.01	0.00761177311039714\\
446.01	0.00761211118592914\\
447.01	0.00761246124748107\\
448.01	0.00761282393131137\\
449.01	0.00761319993071357\\
450.01	0.00761359000366071\\
451.01	0.0076139949814758\\
452.01	0.00761441578020722\\
453.01	0.0076148534162388\\
454.01	0.00761530901516673\\
455.01	0.00761578383195522\\
456.01	0.00761627927634888\\
457.01	0.0076167969664957\\
458.01	0.00761733901033187\\
459.01	0.00761791060538318\\
460.01	0.00761855555398422\\
461.01	0.00761960975108633\\
462.01	0.00762079844158388\\
463.01	0.00762202335305392\\
464.01	0.0076232856601801\\
465.01	0.00762458656691337\\
466.01	0.00762592730344935\\
467.01	0.00762730912025099\\
468.01	0.00762873326755406\\
469.01	0.00763020092171577\\
470.01	0.00763171325348148\\
471.01	0.00763327176100925\\
472.01	0.0076348781059127\\
473.01	0.00763653736701755\\
474.01	0.00763828430449948\\
475.01	0.00764015497130059\\
476.01	0.00764208033378298\\
477.01	0.00764405979089676\\
478.01	0.00764609515843145\\
479.01	0.0076481883331306\\
480.01	0.00765034129745081\\
481.01	0.00765255612462537\\
482.01	0.00765483498403869\\
483.01	0.00765718014689514\\
484.01	0.00765959399202028\\
485.01	0.00766207901078184\\
486.01	0.00766463780923741\\
487.01	0.00766727313250799\\
488.01	0.00766998787905202\\
489.01	0.0076727850659321\\
490.01	0.00767566785836698\\
491.01	0.00767863958285805\\
492.01	0.00768170376233876\\
493.01	0.00768486432687156\\
494.01	0.00768812708722715\\
495.01	0.00769150920317194\\
496.01	0.00769502995574532\\
497.01	0.00769841915877319\\
498.01	0.0077019095592544\\
499.01	0.00770551370753597\\
500.01	0.00770923696302975\\
501.01	0.00771308510211642\\
502.01	0.00771706436780387\\
503.01	0.00772118150695649\\
504.01	0.00772544380366489\\
505.01	0.00772985930962392\\
506.01	0.00773443685937805\\
507.01	0.00773918603597756\\
508.01	0.0077441173773363\\
509.01	0.00774924251481989\\
510.01	0.00775457433445633\\
511.01	0.00776012716816022\\
512.01	0.00776591702640784\\
513.01	0.0077719619228586\\
514.01	0.00777828267036745\\
515.01	0.00778490770485502\\
516.01	0.00779192274698233\\
517.01	0.0078002441517052\\
518.01	0.00781419974703577\\
519.01	0.00782910294009469\\
520.01	0.0078445061763311\\
521.01	0.00786050236092747\\
522.01	0.00787705367427117\\
523.01	0.00789294633197331\\
524.01	0.00790933261609763\\
525.01	0.00792623894042811\\
526.01	0.00794368808082389\\
527.01	0.00796170438911548\\
528.01	0.00798031352540036\\
529.01	0.00799954282651978\\
530.01	0.00801942148739208\\
531.01	0.00803998086201145\\
532.01	0.00806125546453226\\
533.01	0.00808328691513915\\
534.01	0.00810612538960956\\
535.01	0.00812974042920438\\
536.01	0.0081546802971531\\
537.01	0.00818147614643572\\
538.01	0.0082092359172629\\
539.01	0.00823802012322006\\
540.01	0.00826803462013862\\
541.01	0.00830004988378167\\
542.01	0.00833035837788714\\
543.01	0.00835914708340572\\
544.01	0.00838899167252555\\
545.01	0.00841909986528914\\
546.01	0.00844855488491541\\
547.01	0.00847937083965617\\
548.01	0.00851179792589284\\
549.01	0.0085468865663905\\
550.01	0.0086022412879655\\
551.01	0.00868415334816426\\
552.01	0.00876776352755252\\
553.01	0.00885322269770616\\
554.01	0.0089406222800461\\
555.01	0.00903006284670626\\
556.01	0.00912165662638719\\
557.01	0.00921553778926058\\
558.01	0.00931191955770558\\
559.01	0.00941127044642449\\
560.01	0.00951264538348107\\
561.01	0.00961458353049139\\
562.01	0.0097243347526615\\
563.01	0.00981553266355713\\
564.01	0.0099057566752182\\
565.01	0.00998914127892605\\
566.01	0.01\\
567.01	0.01\\
568.01	0.01\\
569.01	0.01\\
570.01	0.01\\
571.01	0.01\\
572.01	0.01\\
573.01	0.01\\
574.01	0.01\\
575.01	0.01\\
576.01	0.01\\
577.01	0.01\\
578.01	0.01\\
579.01	0.01\\
580.01	0.01\\
581.01	0.01\\
582.01	0.01\\
583.01	0.01\\
584.01	0.01\\
585.01	0.01\\
586.01	0.01\\
587.01	0.01\\
588.01	0.01\\
589.01	0.01\\
590.01	0.01\\
591.01	0.01\\
592.01	0.01\\
593.01	0.01\\
594.01	0.01\\
595.01	0.01\\
596.01	0.01\\
597.01	0.01\\
598.01	0.01\\
599.01	0.01\\
599.02	0.01\\
599.03	0.01\\
599.04	0.01\\
599.05	0.01\\
599.06	0.01\\
599.07	0.01\\
599.08	0.01\\
599.09	0.01\\
599.1	0.01\\
599.11	0.01\\
599.12	0.01\\
599.13	0.01\\
599.14	0.01\\
599.15	0.01\\
599.16	0.01\\
599.17	0.01\\
599.18	0.01\\
599.19	0.01\\
599.2	0.01\\
599.21	0.01\\
599.22	0.01\\
599.23	0.01\\
599.24	0.01\\
599.25	0.01\\
599.26	0.01\\
599.27	0.01\\
599.28	0.01\\
599.29	0.01\\
599.3	0.01\\
599.31	0.01\\
599.32	0.01\\
599.33	0.01\\
599.34	0.01\\
599.35	0.01\\
599.36	0.01\\
599.37	0.01\\
599.38	0.01\\
599.39	0.01\\
599.4	0.01\\
599.41	0.01\\
599.42	0.01\\
599.43	0.01\\
599.44	0.01\\
599.45	0.01\\
599.46	0.01\\
599.47	0.01\\
599.48	0.01\\
599.49	0.01\\
599.5	0.01\\
599.51	0.01\\
599.52	0.01\\
599.53	0.01\\
599.54	0.01\\
599.55	0.01\\
599.56	0.01\\
599.57	0.01\\
599.58	0.01\\
599.59	0.01\\
599.6	0.01\\
599.61	0.01\\
599.62	0.01\\
599.63	0.01\\
599.64	0.01\\
599.65	0.01\\
599.66	0.01\\
599.67	0.01\\
599.68	0.01\\
599.69	0.01\\
599.7	0.01\\
599.71	0.01\\
599.72	0.01\\
599.73	0.01\\
599.74	0.01\\
599.75	0.01\\
599.76	0.01\\
599.77	0.01\\
599.78	0.01\\
599.79	0.01\\
599.8	0.01\\
599.81	0.01\\
599.82	0.01\\
599.83	0.01\\
599.84	0.01\\
599.85	0.01\\
599.86	0.01\\
599.87	0.01\\
599.88	0.01\\
599.89	0.01\\
599.9	0.01\\
599.91	0.01\\
599.92	0.01\\
599.93	0.01\\
599.94	0.01\\
599.95	0.01\\
599.96	0.01\\
599.97	0.01\\
599.98	0.01\\
599.99	0.01\\
600	0.01\\
};
\addplot [color=mycolor17,solid,forget plot]
  table[row sep=crcr]{%
0.01	0.00851903386541942\\
1.01	0.00851903389681088\\
2.01	0.00851903392887927\\
3.01	0.00851903396163929\\
4.01	0.00851903399510599\\
5.01	0.0085190340292947\\
6.01	0.00851903406422111\\
7.01	0.00851903409990128\\
8.01	0.00851903413635157\\
9.01	0.00851903417358875\\
10.01	0.00851903421162992\\
11.01	0.00851903425049259\\
12.01	0.0085190342901946\\
13.01	0.00851903433075426\\
14.01	0.00851903437219022\\
15.01	0.00851903441452157\\
16.01	0.0085190344577678\\
17.01	0.00851903450194884\\
18.01	0.00851903454708506\\
19.01	0.00851903459319728\\
20.01	0.00851903464030676\\
21.01	0.00851903468843526\\
22.01	0.008519034737605\\
23.01	0.0085190347878387\\
24.01	0.00851903483915954\\
25.01	0.00851903489159127\\
26.01	0.00851903494515813\\
27.01	0.0085190349998849\\
28.01	0.00851903505579692\\
29.01	0.00851903511292007\\
30.01	0.00851903517128081\\
31.01	0.00851903523090618\\
32.01	0.00851903529182384\\
33.01	0.00851903535406203\\
34.01	0.00851903541764963\\
35.01	0.00851903548261616\\
36.01	0.00851903554899178\\
37.01	0.00851903561680733\\
38.01	0.00851903568609434\\
39.01	0.00851903575688501\\
40.01	0.0085190358292123\\
41.01	0.00851903590310986\\
42.01	0.00851903597861208\\
43.01	0.00851903605575415\\
44.01	0.00851903613457203\\
45.01	0.00851903621510241\\
46.01	0.00851903629738294\\
47.01	0.00851903638145197\\
48.01	0.00851903646734875\\
49.01	0.00851903655511342\\
50.01	0.008519036644787\\
51.01	0.0085190367364114\\
52.01	0.00851903683002949\\
53.01	0.00851903692568508\\
54.01	0.00851903702342297\\
55.01	0.00851903712328895\\
56.01	0.0085190372253298\\
57.01	0.00851903732959339\\
58.01	0.00851903743612863\\
59.01	0.00851903754498555\\
60.01	0.00851903765621525\\
61.01	0.00851903776987001\\
62.01	0.00851903788600325\\
63.01	0.00851903800466959\\
64.01	0.00851903812592489\\
65.01	0.00851903824982625\\
66.01	0.00851903837643202\\
67.01	0.0085190385058019\\
68.01	0.00851903863799688\\
69.01	0.00851903877307935\\
70.01	0.00851903891111307\\
71.01	0.00851903905216327\\
72.01	0.00851903919629661\\
73.01	0.00851903934358124\\
74.01	0.00851903949408684\\
75.01	0.0085190396478847\\
76.01	0.00851903980504764\\
77.01	0.00851903996565017\\
78.01	0.00851904012976845\\
79.01	0.00851904029748035\\
80.01	0.0085190404688655\\
81.01	0.00851904064400534\\
82.01	0.00851904082298309\\
83.01	0.0085190410058839\\
84.01	0.00851904119279483\\
85.01	0.00851904138380485\\
86.01	0.00851904157900499\\
87.01	0.00851904177848834\\
88.01	0.00851904198235004\\
89.01	0.00851904219068744\\
90.01	0.00851904240360003\\
91.01	0.00851904262118959\\
92.01	0.00851904284356017\\
93.01	0.00851904307081822\\
94.01	0.00851904330307255\\
95.01	0.00851904354043446\\
96.01	0.00851904378301777\\
97.01	0.00851904403093887\\
98.01	0.0085190442843168\\
99.01	0.00851904454327332\\
100.01	0.0085190448079329\\
101.01	0.00851904507842285\\
102.01	0.00851904535487344\\
103.01	0.00851904563741782\\
104.01	0.00851904592619221\\
105.01	0.00851904622133591\\
106.01	0.00851904652299143\\
107.01	0.00851904683130446\\
108.01	0.00851904714642407\\
109.01	0.00851904746850268\\
110.01	0.00851904779769626\\
111.01	0.00851904813416426\\
112.01	0.00851904847806982\\
113.01	0.0085190488295798\\
114.01	0.00851904918886484\\
115.01	0.00851904955609954\\
116.01	0.00851904993146245\\
117.01	0.00851905031513625\\
118.01	0.00851905070730774\\
119.01	0.00851905110816807\\
120.01	0.00851905151791277\\
121.01	0.00851905193674178\\
122.01	0.00851905236485974\\
123.01	0.00851905280247592\\
124.01	0.0085190532498044\\
125.01	0.00851905370706421\\
126.01	0.00851905417447942\\
127.01	0.00851905465227925\\
128.01	0.0085190551406982\\
129.01	0.00851905563997617\\
130.01	0.00851905615035861\\
131.01	0.00851905667209663\\
132.01	0.00851905720544713\\
133.01	0.00851905775067294\\
134.01	0.00851905830804299\\
135.01	0.00851905887783243\\
136.01	0.00851905946032274\\
137.01	0.00851906005580197\\
138.01	0.00851906066456478\\
139.01	0.00851906128691276\\
140.01	0.0085190619231544\\
141.01	0.0085190625736054\\
142.01	0.00851906323858882\\
143.01	0.00851906391843514\\
144.01	0.00851906461348262\\
145.01	0.00851906532407734\\
146.01	0.00851906605057345\\
147.01	0.00851906679333339\\
148.01	0.00851906755272798\\
149.01	0.00851906832913673\\
150.01	0.00851906912294802\\
151.01	0.00851906993455927\\
152.01	0.00851907076437724\\
153.01	0.00851907161281816\\
154.01	0.00851907248030799\\
155.01	0.00851907336728273\\
156.01	0.00851907427418853\\
157.01	0.00851907520148204\\
158.01	0.00851907614963062\\
159.01	0.00851907711911257\\
160.01	0.00851907811041749\\
161.01	0.00851907912404642\\
162.01	0.00851908016051225\\
163.01	0.00851908122033991\\
164.01	0.00851908230406669\\
165.01	0.00851908341224257\\
166.01	0.00851908454543047\\
167.01	0.00851908570420663\\
168.01	0.00851908688916085\\
169.01	0.00851908810089696\\
170.01	0.00851908934003295\\
171.01	0.00851909060720155\\
172.01	0.00851909190305039\\
173.01	0.00851909322824248\\
174.01	0.00851909458345653\\
175.01	0.00851909596938742\\
176.01	0.00851909738674645\\
177.01	0.00851909883626185\\
178.01	0.00851910031867916\\
179.01	0.00851910183476169\\
180.01	0.00851910338529087\\
181.01	0.00851910497106679\\
182.01	0.00851910659290859\\
183.01	0.00851910825165496\\
184.01	0.00851910994816461\\
185.01	0.00851911168331674\\
186.01	0.00851911345801163\\
187.01	0.00851911527317101\\
188.01	0.00851911712973874\\
189.01	0.00851911902868126\\
190.01	0.00851912097098817\\
191.01	0.00851912295767282\\
192.01	0.00851912498977289\\
193.01	0.00851912706835093\\
194.01	0.00851912919449509\\
195.01	0.00851913136931964\\
196.01	0.00851913359396569\\
197.01	0.00851913586960185\\
198.01	0.00851913819742483\\
199.01	0.00851914057866026\\
200.01	0.00851914301456329\\
201.01	0.00851914550641944\\
202.01	0.00851914805554521\\
203.01	0.008519150663289\\
204.01	0.00851915333103179\\
205.01	0.00851915606018802\\
206.01	0.00851915885220637\\
207.01	0.00851916170857064\\
208.01	0.00851916463080064\\
209.01	0.00851916762045307\\
210.01	0.00851917067912239\\
211.01	0.00851917380844189\\
212.01	0.00851917701008451\\
213.01	0.00851918028576393\\
214.01	0.00851918363723557\\
215.01	0.0085191870662976\\
216.01	0.00851919057479201\\
217.01	0.00851919416460578\\
218.01	0.00851919783767192\\
219.01	0.00851920159597066\\
220.01	0.00851920544153062\\
221.01	0.00851920937643004\\
222.01	0.00851921340279799\\
223.01	0.00851921752281573\\
224.01	0.00851922173871789\\
225.01	0.00851922605279391\\
226.01	0.0085192304673894\\
227.01	0.0085192349849075\\
228.01	0.00851923960781038\\
229.01	0.00851924433862071\\
230.01	0.00851924917992318\\
231.01	0.00851925413436603\\
232.01	0.00851925920466273\\
233.01	0.00851926439359349\\
234.01	0.00851926970400708\\
235.01	0.00851927513882249\\
236.01	0.00851928070103067\\
237.01	0.00851928639369638\\
238.01	0.00851929221996008\\
239.01	0.00851929818303978\\
240.01	0.00851930428623303\\
241.01	0.00851931053291889\\
242.01	0.00851931692656006\\
243.01	0.00851932347070488\\
244.01	0.0085193301689896\\
245.01	0.00851933702514052\\
246.01	0.00851934404297633\\
247.01	0.00851935122641035\\
248.01	0.00851935857945302\\
249.01	0.00851936610621427\\
250.01	0.00851937381090609\\
251.01	0.00851938169784507\\
252.01	0.00851938977145506\\
253.01	0.00851939803626987\\
254.01	0.00851940649693605\\
255.01	0.00851941515821571\\
256.01	0.0085194240249895\\
257.01	0.00851943310225955\\
258.01	0.00851944239515253\\
259.01	0.00851945190892285\\
260.01	0.00851946164895584\\
261.01	0.00851947162077104\\
262.01	0.00851948183002565\\
263.01	0.00851949228251794\\
264.01	0.00851950298419083\\
265.01	0.0085195139411356\\
266.01	0.00851952515959549\\
267.01	0.00851953664596972\\
268.01	0.00851954840681722\\
269.01	0.00851956044886084\\
270.01	0.00851957277899136\\
271.01	0.0085195854042718\\
272.01	0.0085195983319417\\
273.01	0.00851961156942161\\
274.01	0.00851962512431759\\
275.01	0.00851963900442599\\
276.01	0.00851965321773815\\
277.01	0.00851966777244529\\
278.01	0.0085196826769436\\
279.01	0.00851969793983942\\
280.01	0.00851971356995443\\
281.01	0.00851972957633114\\
282.01	0.00851974596823842\\
283.01	0.00851976275517718\\
284.01	0.00851977994688619\\
285.01	0.00851979755334807\\
286.01	0.00851981558479545\\
287.01	0.00851983405171714\\
288.01	0.0085198529648647\\
289.01	0.00851987233525888\\
290.01	0.00851989217419651\\
291.01	0.00851991249325733\\
292.01	0.00851993330431114\\
293.01	0.008519954619525\\
294.01	0.00851997645137079\\
295.01	0.00851999881263271\\
296.01	0.00852002171641526\\
297.01	0.00852004517615114\\
298.01	0.00852006920560947\\
299.01	0.00852009381890434\\
300.01	0.00852011903050333\\
301.01	0.00852014485523636\\
302.01	0.00852017130830491\\
303.01	0.00852019840529111\\
304.01	0.00852022616216744\\
305.01	0.00852025459530644\\
306.01	0.00852028372149079\\
307.01	0.00852031355792348\\
308.01	0.00852034412223848\\
309.01	0.00852037543251143\\
310.01	0.00852040750727082\\
311.01	0.00852044036550928\\
312.01	0.00852047402669528\\
313.01	0.00852050851078508\\
314.01	0.00852054383823503\\
315.01	0.00852058003001403\\
316.01	0.00852061710761662\\
317.01	0.00852065509307611\\
318.01	0.0085206940089782\\
319.01	0.00852073387847495\\
320.01	0.00852077472529914\\
321.01	0.00852081657377892\\
322.01	0.00852085944885293\\
323.01	0.00852090337608586\\
324.01	0.00852094838168435\\
325.01	0.00852099449251327\\
326.01	0.00852104173611275\\
327.01	0.00852109014071521\\
328.01	0.00852113973526331\\
329.01	0.00852119054942814\\
330.01	0.00852124261362796\\
331.01	0.00852129595904753\\
332.01	0.00852135061765802\\
333.01	0.00852140662223739\\
334.01	0.00852146400639137\\
335.01	0.00852152280457522\\
336.01	0.00852158305211584\\
337.01	0.00852164478523488\\
338.01	0.00852170804107222\\
339.01	0.00852177285771041\\
340.01	0.00852183927419964\\
341.01	0.00852190733058369\\
342.01	0.00852197706792645\\
343.01	0.0085220485283395\\
344.01	0.00852212175501035\\
345.01	0.0085221967922317\\
346.01	0.0085222736854316\\
347.01	0.00852235248120448\\
348.01	0.00852243322734335\\
349.01	0.00852251597287295\\
350.01	0.0085226007680839\\
351.01	0.00852268766456818\\
352.01	0.00852277671525556\\
353.01	0.00852286797445135\\
354.01	0.00852296149787546\\
355.01	0.00852305734270264\\
356.01	0.00852315556760422\\
357.01	0.00852325623279104\\
358.01	0.00852335940005817\\
359.01	0.00852346513283078\\
360.01	0.00852357349621188\\
361.01	0.00852368455703154\\
362.01	0.00852379838389776\\
363.01	0.00852391504724926\\
364.01	0.00852403461940987\\
365.01	0.00852415717464492\\
366.01	0.00852428278921963\\
367.01	0.0085244115414592\\
368.01	0.00852454351181135\\
369.01	0.00852467878291075\\
370.01	0.00852481743964565\\
371.01	0.00852495956922692\\
372.01	0.00852510526125928\\
373.01	0.00852525460781502\\
374.01	0.00852540770351018\\
375.01	0.00852556464558327\\
376.01	0.00852572553397657\\
377.01	0.00852589047142027\\
378.01	0.00852605956351916\\
379.01	0.00852623291884242\\
380.01	0.0085264106490161\\
381.01	0.00852659286881875\\
382.01	0.00852677969628006\\
383.01	0.00852697125278272\\
384.01	0.00852716766316754\\
385.01	0.00852736905584179\\
386.01	0.00852757556289121\\
387.01	0.00852778732019534\\
388.01	0.00852800446754665\\
389.01	0.00852822714877336\\
390.01	0.00852845551186626\\
391.01	0.00852868970910944\\
392.01	0.00852892989721528\\
393.01	0.00852917623746391\\
394.01	0.00852942889584704\\
395.01	0.00852968804321676\\
396.01	0.00852995385543926\\
397.01	0.00853022651355389\\
398.01	0.00853050620393787\\
399.01	0.0085307931184771\\
400.01	0.00853108745474314\\
401.01	0.00853138941617745\\
402.01	0.00853169921228288\\
403.01	0.00853201705882334\\
404.01	0.0085323431780325\\
405.01	0.00853267779883209\\
406.01	0.00853302115706107\\
407.01	0.00853337349571685\\
408.01	0.00853373506520972\\
409.01	0.00853410612363215\\
410.01	0.00853448693704495\\
411.01	0.0085348777797816\\
412.01	0.00853527893477052\\
413.01	0.00853569069385219\\
414.01	0.00853611335785201\\
415.01	0.00853654723387496\\
416.01	0.00853699262679034\\
417.01	0.00853744990332334\\
418.01	0.0085379194266223\\
419.01	0.00853840161436998\\
420.01	0.00853889716461353\\
421.01	0.00853940663745721\\
422.01	0.00853992386244635\\
423.01	0.00854045345562234\\
424.01	0.00854099745077567\\
425.01	0.00854155627968325\\
426.01	0.00854213038955616\\
427.01	0.00854272024378953\\
428.01	0.00854332632276554\\
429.01	0.00854394912471401\\
430.01	0.00854458916663657\\
431.01	0.00854524698530027\\
432.01	0.00854592313830782\\
433.01	0.008546618205252\\
434.01	0.00854733278896304\\
435.01	0.00854806751685886\\
436.01	0.00854882304240884\\
437.01	0.00854960004672395\\
438.01	0.00855039924028687\\
439.01	0.00855122136483846\\
440.01	0.00855206719543803\\
441.01	0.00855293754271855\\
442.01	0.00855383325535988\\
443.01	0.00855475522280683\\
444.01	0.00855570437826286\\
445.01	0.008556681701995\\
446.01	0.00855768822499032\\
447.01	0.00855872503301097\\
448.01	0.00855979327109931\\
449.01	0.00856089414856185\\
450.01	0.00856202894417724\\
451.01	0.0085631990090945\\
452.01	0.00856440576781709\\
453.01	0.00856565078392652\\
454.01	0.00856693569839355\\
455.01	0.0085682622418768\\
456.01	0.00856963227505992\\
457.01	0.00857104786178624\\
458.01	0.00857251180814016\\
459.01	0.00857403225298102\\
460.01	0.00857564600632979\\
461.01	0.00857716839994526\\
462.01	0.00857866471785247\\
463.01	0.00858021372895401\\
464.01	0.00858181839781352\\
465.01	0.00858348198743255\\
466.01	0.0085852081030211\\
467.01	0.00858700073482906\\
468.01	0.00858886420859308\\
469.01	0.00859080228702808\\
470.01	0.00859281669008256\\
471.01	0.00859491732257235\\
472.01	0.00859711429223264\\
473.01	0.00859943173771808\\
474.01	0.00860215295954599\\
475.01	0.00860684526039444\\
476.01	0.00861187566236694\\
477.01	0.00861704203429633\\
478.01	0.00862234839691171\\
479.01	0.00862779890889316\\
480.01	0.00863339787294322\\
481.01	0.0086391497422098\\
482.01	0.00864505912707618\\
483.01	0.00865113080227834\\
484.01	0.00865736971376215\\
485.01	0.00866378097929141\\
486.01	0.00867036984112\\
487.01	0.00867714171497274\\
488.01	0.00868410266418304\\
489.01	0.00869125864186352\\
490.01	0.00869861575874439\\
491.01	0.00870618036815419\\
492.01	0.00871395912455498\\
493.01	0.00872195930390418\\
494.01	0.00873019064284751\\
495.01	0.00873867081320213\\
496.01	0.00874736519847467\\
497.01	0.00875619829089197\\
498.01	0.0087652904495427\\
499.01	0.00877465138881615\\
500.01	0.00878429083743185\\
501.01	0.00879421906688389\\
502.01	0.00880444703401725\\
503.01	0.00881498630465298\\
504.01	0.00882584852487436\\
505.01	0.00883704667524924\\
506.01	0.008848596387323\\
507.01	0.00886051230198366\\
508.01	0.00887280983688171\\
509.01	0.00888550535659654\\
510.01	0.00889861626069576\\
511.01	0.00891216108505397\\
512.01	0.00892615962815026\\
513.01	0.0089406331862638\\
514.01	0.00895560564261965\\
515.01	0.00897111254928596\\
516.01	0.00898728333367069\\
517.01	0.00900462258209694\\
518.01	0.00902005533817399\\
519.01	0.00903556989142077\\
520.01	0.00905179543968464\\
521.01	0.00906929777513815\\
522.01	0.00908690158381172\\
523.01	0.00910325478368778\\
524.01	0.00911996382008945\\
525.01	0.00913726484255246\\
526.01	0.00915519484973122\\
527.01	0.00917379853846171\\
528.01	0.00919312249530783\\
529.01	0.00921321828829849\\
530.01	0.00923414366959399\\
531.01	0.00925596373087513\\
532.01	0.00927875287109759\\
533.01	0.00930259734269146\\
534.01	0.0093275873085074\\
535.01	0.00935394603339217\\
536.01	0.00938619944115896\\
537.01	0.00944802730378025\\
538.01	0.00951341341243095\\
539.01	0.00958015524452214\\
540.01	0.00964844390480317\\
541.01	0.00971857849326479\\
542.01	0.00978779473566883\\
543.01	0.00985784525650115\\
544.01	0.00992932943037659\\
545.01	0.00999424628944576\\
546.01	0.01\\
547.01	0.01\\
548.01	0.01\\
549.01	0.01\\
550.01	0.01\\
551.01	0.01\\
552.01	0.01\\
553.01	0.01\\
554.01	0.01\\
555.01	0.01\\
556.01	0.01\\
557.01	0.01\\
558.01	0.01\\
559.01	0.01\\
560.01	0.01\\
561.01	0.01\\
562.01	0.01\\
563.01	0.01\\
564.01	0.01\\
565.01	0.01\\
566.01	0.01\\
567.01	0.01\\
568.01	0.01\\
569.01	0.01\\
570.01	0.01\\
571.01	0.01\\
572.01	0.01\\
573.01	0.01\\
574.01	0.01\\
575.01	0.01\\
576.01	0.01\\
577.01	0.01\\
578.01	0.01\\
579.01	0.01\\
580.01	0.01\\
581.01	0.01\\
582.01	0.01\\
583.01	0.01\\
584.01	0.01\\
585.01	0.01\\
586.01	0.01\\
587.01	0.01\\
588.01	0.01\\
589.01	0.01\\
590.01	0.01\\
591.01	0.01\\
592.01	0.01\\
593.01	0.01\\
594.01	0.01\\
595.01	0.01\\
596.01	0.01\\
597.01	0.01\\
598.01	0.01\\
599.01	0.01\\
599.02	0.01\\
599.03	0.01\\
599.04	0.01\\
599.05	0.01\\
599.06	0.01\\
599.07	0.01\\
599.08	0.01\\
599.09	0.01\\
599.1	0.01\\
599.11	0.01\\
599.12	0.01\\
599.13	0.01\\
599.14	0.01\\
599.15	0.01\\
599.16	0.01\\
599.17	0.01\\
599.18	0.01\\
599.19	0.01\\
599.2	0.01\\
599.21	0.01\\
599.22	0.01\\
599.23	0.01\\
599.24	0.01\\
599.25	0.01\\
599.26	0.01\\
599.27	0.01\\
599.28	0.01\\
599.29	0.01\\
599.3	0.01\\
599.31	0.01\\
599.32	0.01\\
599.33	0.01\\
599.34	0.01\\
599.35	0.01\\
599.36	0.01\\
599.37	0.01\\
599.38	0.01\\
599.39	0.01\\
599.4	0.01\\
599.41	0.01\\
599.42	0.01\\
599.43	0.01\\
599.44	0.01\\
599.45	0.01\\
599.46	0.01\\
599.47	0.01\\
599.48	0.01\\
599.49	0.01\\
599.5	0.01\\
599.51	0.01\\
599.52	0.01\\
599.53	0.01\\
599.54	0.01\\
599.55	0.01\\
599.56	0.01\\
599.57	0.01\\
599.58	0.01\\
599.59	0.01\\
599.6	0.01\\
599.61	0.01\\
599.62	0.01\\
599.63	0.01\\
599.64	0.01\\
599.65	0.01\\
599.66	0.01\\
599.67	0.01\\
599.68	0.01\\
599.69	0.01\\
599.7	0.01\\
599.71	0.01\\
599.72	0.01\\
599.73	0.01\\
599.74	0.01\\
599.75	0.01\\
599.76	0.01\\
599.77	0.01\\
599.78	0.01\\
599.79	0.01\\
599.8	0.01\\
599.81	0.01\\
599.82	0.01\\
599.83	0.01\\
599.84	0.01\\
599.85	0.01\\
599.86	0.01\\
599.87	0.01\\
599.88	0.01\\
599.89	0.01\\
599.9	0.01\\
599.91	0.01\\
599.92	0.01\\
599.93	0.01\\
599.94	0.01\\
599.95	0.01\\
599.96	0.01\\
599.97	0.01\\
599.98	0.01\\
599.99	0.01\\
600	0.01\\
};
\addplot [color=mycolor18,solid,forget plot]
  table[row sep=crcr]{%
0.01	0.00921890721345464\\
1.01	0.00921890730374509\\
2.01	0.00921890739597887\\
3.01	0.00921890749019809\\
4.01	0.00921890758644576\\
5.01	0.00921890768476586\\
6.01	0.00921890778520332\\
7.01	0.00921890788780404\\
8.01	0.00921890799261493\\
9.01	0.00921890809968396\\
10.01	0.00921890820906009\\
11.01	0.00921890832079337\\
12.01	0.00921890843493498\\
13.01	0.0092189085515372\\
14.01	0.00921890867065344\\
15.01	0.00921890879233829\\
16.01	0.00921890891664755\\
17.01	0.00921890904363823\\
18.01	0.00921890917336859\\
19.01	0.00921890930589814\\
20.01	0.00921890944128779\\
21.01	0.00921890957959967\\
22.01	0.00921890972089736\\
23.01	0.00921890986524575\\
24.01	0.00921891001271128\\
25.01	0.00921891016336171\\
26.01	0.00921891031726639\\
27.01	0.00921891047449615\\
28.01	0.00921891063512338\\
29.01	0.00921891079922208\\
30.01	0.00921891096686785\\
31.01	0.00921891113813799\\
32.01	0.00921891131311146\\
33.01	0.009218911491869\\
34.01	0.00921891167449308\\
35.01	0.00921891186106804\\
36.01	0.00921891205168002\\
37.01	0.00921891224641713\\
38.01	0.00921891244536936\\
39.01	0.0092189126486287\\
40.01	0.0092189128562892\\
41.01	0.00921891306844694\\
42.01	0.00921891328520016\\
43.01	0.00921891350664924\\
44.01	0.00921891373289679\\
45.01	0.00921891396404769\\
46.01	0.00921891420020914\\
47.01	0.0092189144414907\\
48.01	0.00921891468800435\\
49.01	0.00921891493986457\\
50.01	0.00921891519718833\\
51.01	0.00921891546009524\\
52.01	0.00921891572870752\\
53.01	0.00921891600315015\\
54.01	0.00921891628355082\\
55.01	0.00921891657004004\\
56.01	0.00921891686275127\\
57.01	0.00921891716182092\\
58.01	0.00921891746738841\\
59.01	0.00921891777959622\\
60.01	0.00921891809859007\\
61.01	0.00921891842451887\\
62.01	0.00921891875753484\\
63.01	0.00921891909779355\\
64.01	0.00921891944545415\\
65.01	0.0092189198006792\\
66.01	0.0092189201636349\\
67.01	0.0092189205344912\\
68.01	0.00921892091342182\\
69.01	0.00921892130060432\\
70.01	0.00921892169622022\\
71.01	0.00921892210045515\\
72.01	0.00921892251349874\\
73.01	0.00921892293554501\\
74.01	0.00921892336679221\\
75.01	0.00921892380744307\\
76.01	0.00921892425770478\\
77.01	0.00921892471778922\\
78.01	0.00921892518791298\\
79.01	0.0092189256682975\\
80.01	0.00921892615916919\\
81.01	0.00921892666075947\\
82.01	0.009218927173305\\
83.01	0.00921892769704771\\
84.01	0.00921892823223495\\
85.01	0.00921892877911963\\
86.01	0.00921892933796032\\
87.01	0.00921892990902138\\
88.01	0.00921893049257313\\
89.01	0.00921893108889191\\
90.01	0.00921893169826033\\
91.01	0.0092189323209673\\
92.01	0.00921893295730824\\
93.01	0.00921893360758522\\
94.01	0.00921893427210712\\
95.01	0.0092189349511897\\
96.01	0.00921893564515591\\
97.01	0.00921893635433596\\
98.01	0.00921893707906747\\
99.01	0.00921893781969569\\
100.01	0.00921893857657367\\
101.01	0.00921893935006242\\
102.01	0.00921894014053106\\
103.01	0.00921894094835713\\
104.01	0.00921894177392664\\
105.01	0.00921894261763431\\
106.01	0.00921894347988386\\
107.01	0.00921894436108809\\
108.01	0.00921894526166918\\
109.01	0.00921894618205884\\
110.01	0.00921894712269858\\
111.01	0.00921894808403991\\
112.01	0.00921894906654464\\
113.01	0.00921895007068496\\
114.01	0.00921895109694385\\
115.01	0.00921895214581526\\
116.01	0.00921895321780434\\
117.01	0.00921895431342775\\
118.01	0.0092189554332139\\
119.01	0.0092189565777032\\
120.01	0.00921895774744843\\
121.01	0.0092189589430149\\
122.01	0.00921896016498084\\
123.01	0.00921896141393768\\
124.01	0.00921896269049034\\
125.01	0.00921896399525753\\
126.01	0.00921896532887211\\
127.01	0.0092189666919814\\
128.01	0.00921896808524758\\
129.01	0.00921896950934787\\
130.01	0.00921897096497513\\
131.01	0.00921897245283797\\
132.01	0.00921897397366134\\
133.01	0.00921897552818671\\
134.01	0.00921897711717267\\
135.01	0.00921897874139517\\
136.01	0.00921898040164799\\
137.01	0.00921898209874318\\
138.01	0.0092189838335114\\
139.01	0.00921898560680249\\
140.01	0.00921898741948585\\
141.01	0.00921898927245085\\
142.01	0.00921899116660746\\
143.01	0.00921899310288653\\
144.01	0.00921899508224044\\
145.01	0.00921899710564359\\
146.01	0.00921899917409285\\
147.01	0.00921900128860814\\
148.01	0.00921900345023298\\
149.01	0.00921900566003505\\
150.01	0.00921900791910675\\
151.01	0.0092190102285658\\
152.01	0.00921901258955581\\
153.01	0.00921901500324697\\
154.01	0.00921901747083657\\
155.01	0.0092190199935498\\
156.01	0.00921902257264027\\
157.01	0.00921902520939073\\
158.01	0.00921902790511385\\
159.01	0.0092190306611528\\
160.01	0.0092190334788821\\
161.01	0.00921903635970831\\
162.01	0.00921903930507079\\
163.01	0.00921904231644249\\
164.01	0.00921904539533084\\
165.01	0.00921904854327842\\
166.01	0.00921905176186396\\
167.01	0.00921905505270312\\
168.01	0.00921905841744938\\
169.01	0.00921906185779502\\
170.01	0.00921906537547199\\
171.01	0.00921906897225286\\
172.01	0.00921907264995184\\
173.01	0.0092190764104258\\
174.01	0.00921908025557525\\
175.01	0.00921908418734543\\
176.01	0.00921908820772733\\
177.01	0.00921909231875895\\
178.01	0.00921909652252625\\
179.01	0.00921910082116448\\
180.01	0.00921910521685924\\
181.01	0.00921910971184779\\
182.01	0.0092191143084203\\
183.01	0.00921911900892108\\
184.01	0.00921912381574998\\
185.01	0.0092191287313637\\
186.01	0.00921913375827713\\
187.01	0.00921913889906488\\
188.01	0.00921914415636263\\
189.01	0.00921914953286872\\
190.01	0.00921915503134557\\
191.01	0.00921916065462139\\
192.01	0.00921916640559164\\
193.01	0.00921917228722081\\
194.01	0.00921917830254402\\
195.01	0.00921918445466883\\
196.01	0.00921919074677693\\
197.01	0.00921919718212605\\
198.01	0.00921920376405177\\
199.01	0.00921921049596948\\
200.01	0.00921921738137632\\
201.01	0.0092192244238532\\
202.01	0.00921923162706686\\
203.01	0.00921923899477202\\
204.01	0.00921924653081352\\
205.01	0.00921925423912855\\
206.01	0.00921926212374895\\
207.01	0.00921927018880356\\
208.01	0.0092192784385206\\
209.01	0.00921928687723014\\
210.01	0.00921929550936663\\
211.01	0.0092193043394715\\
212.01	0.00921931337219581\\
213.01	0.00921932261230295\\
214.01	0.00921933206467149\\
215.01	0.00921934173429797\\
216.01	0.00921935162629992\\
217.01	0.00921936174591882\\
218.01	0.00921937209852319\\
219.01	0.0092193826896118\\
220.01	0.00921939352481688\\
221.01	0.00921940460990744\\
222.01	0.0092194159507928\\
223.01	0.00921942755352593\\
224.01	0.00921943942430716\\
225.01	0.00921945156948784\\
226.01	0.00921946399557413\\
227.01	0.00921947670923086\\
228.01	0.0092194897172855\\
229.01	0.00921950302673227\\
230.01	0.00921951664473636\\
231.01	0.00921953057863811\\
232.01	0.0092195448359575\\
233.01	0.00921955942439871\\
234.01	0.00921957435185456\\
235.01	0.00921958962641151\\
236.01	0.00921960525635432\\
237.01	0.00921962125017119\\
238.01	0.0092196376165588\\
239.01	0.00921965436442761\\
240.01	0.00921967150290721\\
241.01	0.00921968904135194\\
242.01	0.00921970698934639\\
243.01	0.00921972535671141\\
244.01	0.00921974415350995\\
245.01	0.00921976339005324\\
246.01	0.00921978307690698\\
247.01	0.0092198032248979\\
248.01	0.00921982384512024\\
249.01	0.00921984494894264\\
250.01	0.00921986654801494\\
251.01	0.00921988865427544\\
252.01	0.00921991127995808\\
253.01	0.00921993443760005\\
254.01	0.00921995814004934\\
255.01	0.00921998240047273\\
256.01	0.00922000723236382\\
257.01	0.00922003264955129\\
258.01	0.00922005866620747\\
259.01	0.00922008529685696\\
260.01	0.00922011255638563\\
261.01	0.00922014046004974\\
262.01	0.00922016902348541\\
263.01	0.00922019826271813\\
264.01	0.0092202281941728\\
265.01	0.00922025883468371\\
266.01	0.00922029020150506\\
267.01	0.00922032231232145\\
268.01	0.00922035518525899\\
269.01	0.0092203888388964\\
270.01	0.00922042329227649\\
271.01	0.00922045856491797\\
272.01	0.00922049467682749\\
273.01	0.00922053164851207\\
274.01	0.00922056950099175\\
275.01	0.00922060825581258\\
276.01	0.00922064793506005\\
277.01	0.00922068856137268\\
278.01	0.00922073015795612\\
279.01	0.0092207727485974\\
280.01	0.00922081635767986\\
281.01	0.00922086101019805\\
282.01	0.00922090673177337\\
283.01	0.00922095354866987\\
284.01	0.00922100148781057\\
285.01	0.0092210505767941\\
286.01	0.00922110084391179\\
287.01	0.00922115231816529\\
288.01	0.00922120502928439\\
289.01	0.00922125900774555\\
290.01	0.00922131428479074\\
291.01	0.00922137089244673\\
292.01	0.009221428863545\\
293.01	0.009221488231742\\
294.01	0.00922154903153996\\
295.01	0.00922161129830828\\
296.01	0.00922167506830539\\
297.01	0.00922174037870109\\
298.01	0.00922180726759964\\
299.01	0.00922187577406314\\
300.01	0.00922194593813585\\
301.01	0.00922201780086869\\
302.01	0.00922209140434469\\
303.01	0.00922216679170498\\
304.01	0.00922224400717528\\
305.01	0.00922232309609317\\
306.01	0.00922240410493612\\
307.01	0.00922248708134997\\
308.01	0.00922257207417836\\
309.01	0.00922265913349269\\
310.01	0.00922274831062301\\
311.01	0.00922283965818949\\
312.01	0.00922293323013474\\
313.01	0.00922302908175709\\
314.01	0.00922312726974437\\
315.01	0.0092232278522088\\
316.01	0.00922333088872266\\
317.01	0.00922343644035481\\
318.01	0.00922354456970816\\
319.01	0.0092236553409581\\
320.01	0.00922376881989177\\
321.01	0.00922388507394854\\
322.01	0.00922400417226123\\
323.01	0.0092241261856986\\
324.01	0.00922425118690877\\
325.01	0.00922437925036381\\
326.01	0.00922451045240545\\
327.01	0.00922464487129194\\
328.01	0.00922478258724609\\
329.01	0.00922492368250462\\
330.01	0.00922506824136872\\
331.01	0.00922521635025588\\
332.01	0.00922536809775322\\
333.01	0.00922552357467207\\
334.01	0.00922568287410402\\
335.01	0.00922584609147853\\
336.01	0.00922601332462195\\
337.01	0.00922618467381824\\
338.01	0.00922636024187123\\
339.01	0.00922654013416866\\
340.01	0.00922672445874789\\
341.01	0.00922691332636346\\
342.01	0.00922710685055648\\
343.01	0.00922730514772609\\
344.01	0.00922750833720266\\
345.01	0.00922771654132343\\
346.01	0.00922792988550996\\
347.01	0.00922814849834811\\
348.01	0.00922837251167\\
349.01	0.00922860206063883\\
350.01	0.00922883728383571\\
351.01	0.00922907832334947\\
352.01	0.00922932532486894\\
353.01	0.00922957843777813\\
354.01	0.00922983781525418\\
355.01	0.00923010361436858\\
356.01	0.00923037599619122\\
357.01	0.00923065512589792\\
358.01	0.00923094117288126\\
359.01	0.00923123431086496\\
360.01	0.00923153471802176\\
361.01	0.00923184257709542\\
362.01	0.00923215807552635\\
363.01	0.0092324814055815\\
364.01	0.0092328127644885\\
365.01	0.00923315235457411\\
366.01	0.00923350038340743\\
367.01	0.00923385706394779\\
368.01	0.00923422261469753\\
369.01	0.00923459725986001\\
370.01	0.00923498122950288\\
371.01	0.00923537475972682\\
372.01	0.00923577809284016\\
373.01	0.00923619147753918\\
374.01	0.00923661516909463\\
375.01	0.00923704942954462\\
376.01	0.00923749452789394\\
377.01	0.00923795074032027\\
378.01	0.00923841835038712\\
379.01	0.00923889764926422\\
380.01	0.00923938893595511\\
381.01	0.00923989251753244\\
382.01	0.00924040870938114\\
383.01	0.0092409378354496\\
384.01	0.00924148022850912\\
385.01	0.00924203623042203\\
386.01	0.00924260619241836\\
387.01	0.00924319047538165\\
388.01	0.00924378945014376\\
389.01	0.00924440349778941\\
390.01	0.00924503300997007\\
391.01	0.00924567838922794\\
392.01	0.00924634004933008\\
393.01	0.0092470184156128\\
394.01	0.00924771392533688\\
395.01	0.00924842702805363\\
396.01	0.00924915818598221\\
397.01	0.00924990787439864\\
398.01	0.00925067658203652\\
399.01	0.00925146481150018\\
400.01	0.00925227307969074\\
401.01	0.00925310191824505\\
402.01	0.00925395187398881\\
403.01	0.00925482350940373\\
404.01	0.00925571740311017\\
405.01	0.00925663415036531\\
406.01	0.00925757436357807\\
407.01	0.00925853867284173\\
408.01	0.00925952772648465\\
409.01	0.00926054219164027\\
410.01	0.00926158275483667\\
411.01	0.00926265012260504\\
412.01	0.00926374502210001\\
413.01	0.00926486820166126\\
414.01	0.00926602043076913\\
415.01	0.00926720249699666\\
416.01	0.00926841521199671\\
417.01	0.00926965943905911\\
418.01	0.0092709360566141\\
419.01	0.00927224602791092\\
420.01	0.00927359050517867\\
421.01	0.00927496955357544\\
422.01	0.00927638087076898\\
423.01	0.00927782874595184\\
424.01	0.00927931460337114\\
425.01	0.00928083949176558\\
426.01	0.00928240449039553\\
427.01	0.00928401071001225\\
428.01	0.00928565929386367\\
429.01	0.00928735141873935\\
430.01	0.00928908829605571\\
431.01	0.00929087117298365\\
432.01	0.00929270133362052\\
433.01	0.00929458010020767\\
434.01	0.0092965088343959\\
435.01	0.00929848893855982\\
436.01	0.00930052185716275\\
437.01	0.00930260907817301\\
438.01	0.00930475213453213\\
439.01	0.00930695260567474\\
440.01	0.00930921211909932\\
441.01	0.00931153235198747\\
442.01	0.00931391503286795\\
443.01	0.0093163619433199\\
444.01	0.00931887491970683\\
445.01	0.00932145585492944\\
446.01	0.00932410670018146\\
447.01	0.0093268294666857\\
448.01	0.00932962622737258\\
449.01	0.00933249911836562\\
450.01	0.00933545033908872\\
451.01	0.00933848213706191\\
452.01	0.00934159667112291\\
453.01	0.00934479659703667\\
454.01	0.00934808476999769\\
455.01	0.00935146371890439\\
456.01	0.0093549360494896\\
457.01	0.00935850454685896\\
458.01	0.00936217300499512\\
459.01	0.0093659511498217\\
460.01	0.00936984638181615\\
461.01	0.00937373246944184\\
462.01	0.00937770933007804\\
463.01	0.00938180059729186\\
464.01	0.00938601016855558\\
465.01	0.00939034213804762\\
466.01	0.00939480081308246\\
467.01	0.00939939069515051\\
468.01	0.00940411586563104\\
469.01	0.00940897021722652\\
470.01	0.00941386170534561\\
471.01	0.00941882801766466\\
472.01	0.00942395102325723\\
473.01	0.00942926726025694\\
474.01	0.00943497500895296\\
475.01	0.00944004627798229\\
476.01	0.00944508056108339\\
477.01	0.0094502528079978\\
478.01	0.00945556729011318\\
479.01	0.00946102843227558\\
480.01	0.00946664081958607\\
481.01	0.00947240920462607\\
482.01	0.00947833851514803\\
483.01	0.00948443386215957\\
484.01	0.00949070054692894\\
485.01	0.00949714404642335\\
486.01	0.00950376969226967\\
487.01	0.00951058130173159\\
488.01	0.00951758896062574\\
489.01	0.00952480075565114\\
490.01	0.00953222337148154\\
491.01	0.00953986370343647\\
492.01	0.00954772890825923\\
493.01	0.00955582676231791\\
494.01	0.00956416733201233\\
495.01	0.00957276200482003\\
496.01	0.00958156826414436\\
497.01	0.00959058878391121\\
498.01	0.00959987903732562\\
499.01	0.00960944673872136\\
500.01	0.00961929935139057\\
501.01	0.00962944415214239\\
502.01	0.00963989047896042\\
503.01	0.00965065096520565\\
504.01	0.00966173522705664\\
505.01	0.00967314736790401\\
506.01	0.00968493137212471\\
507.01	0.00969711067637659\\
508.01	0.00970970714788098\\
509.01	0.00972274467702539\\
510.01	0.00973624944209958\\
511.01	0.009750250218431\\
512.01	0.00976477875412207\\
513.01	0.00977987034961125\\
514.01	0.00979556583031401\\
515.01	0.0098119252359359\\
516.01	0.0098291106000916\\
517.01	0.0098470939149584\\
518.01	0.00986437059233572\\
519.01	0.00988253806599425\\
520.01	0.00990313228910655\\
521.01	0.00994326831934782\\
522.01	0.00998766257926984\\
523.01	0.01\\
524.01	0.01\\
525.01	0.01\\
526.01	0.01\\
527.01	0.01\\
528.01	0.01\\
529.01	0.01\\
530.01	0.01\\
531.01	0.01\\
532.01	0.01\\
533.01	0.01\\
534.01	0.01\\
535.01	0.01\\
536.01	0.01\\
537.01	0.01\\
538.01	0.01\\
539.01	0.01\\
540.01	0.01\\
541.01	0.01\\
542.01	0.01\\
543.01	0.01\\
544.01	0.01\\
545.01	0.01\\
546.01	0.01\\
547.01	0.01\\
548.01	0.01\\
549.01	0.01\\
550.01	0.01\\
551.01	0.01\\
552.01	0.01\\
553.01	0.01\\
554.01	0.01\\
555.01	0.01\\
556.01	0.01\\
557.01	0.01\\
558.01	0.01\\
559.01	0.01\\
560.01	0.01\\
561.01	0.01\\
562.01	0.01\\
563.01	0.01\\
564.01	0.01\\
565.01	0.01\\
566.01	0.01\\
567.01	0.01\\
568.01	0.01\\
569.01	0.01\\
570.01	0.01\\
571.01	0.01\\
572.01	0.01\\
573.01	0.01\\
574.01	0.01\\
575.01	0.01\\
576.01	0.01\\
577.01	0.01\\
578.01	0.01\\
579.01	0.01\\
580.01	0.01\\
581.01	0.01\\
582.01	0.01\\
583.01	0.01\\
584.01	0.01\\
585.01	0.01\\
586.01	0.01\\
587.01	0.01\\
588.01	0.01\\
589.01	0.01\\
590.01	0.01\\
591.01	0.01\\
592.01	0.01\\
593.01	0.01\\
594.01	0.01\\
595.01	0.01\\
596.01	0.01\\
597.01	0.01\\
598.01	0.01\\
599.01	0.01\\
599.02	0.01\\
599.03	0.01\\
599.04	0.01\\
599.05	0.01\\
599.06	0.01\\
599.07	0.01\\
599.08	0.01\\
599.09	0.01\\
599.1	0.01\\
599.11	0.01\\
599.12	0.01\\
599.13	0.01\\
599.14	0.01\\
599.15	0.01\\
599.16	0.01\\
599.17	0.01\\
599.18	0.01\\
599.19	0.01\\
599.2	0.01\\
599.21	0.01\\
599.22	0.01\\
599.23	0.01\\
599.24	0.01\\
599.25	0.01\\
599.26	0.01\\
599.27	0.01\\
599.28	0.01\\
599.29	0.01\\
599.3	0.01\\
599.31	0.01\\
599.32	0.01\\
599.33	0.01\\
599.34	0.01\\
599.35	0.01\\
599.36	0.01\\
599.37	0.01\\
599.38	0.01\\
599.39	0.01\\
599.4	0.01\\
599.41	0.01\\
599.42	0.01\\
599.43	0.01\\
599.44	0.01\\
599.45	0.01\\
599.46	0.01\\
599.47	0.01\\
599.48	0.01\\
599.49	0.01\\
599.5	0.01\\
599.51	0.01\\
599.52	0.01\\
599.53	0.01\\
599.54	0.01\\
599.55	0.01\\
599.56	0.01\\
599.57	0.01\\
599.58	0.01\\
599.59	0.01\\
599.6	0.01\\
599.61	0.01\\
599.62	0.01\\
599.63	0.01\\
599.64	0.01\\
599.65	0.01\\
599.66	0.01\\
599.67	0.01\\
599.68	0.01\\
599.69	0.01\\
599.7	0.01\\
599.71	0.01\\
599.72	0.01\\
599.73	0.01\\
599.74	0.01\\
599.75	0.01\\
599.76	0.01\\
599.77	0.01\\
599.78	0.01\\
599.79	0.01\\
599.8	0.01\\
599.81	0.01\\
599.82	0.01\\
599.83	0.01\\
599.84	0.01\\
599.85	0.01\\
599.86	0.01\\
599.87	0.01\\
599.88	0.01\\
599.89	0.01\\
599.9	0.01\\
599.91	0.01\\
599.92	0.01\\
599.93	0.01\\
599.94	0.01\\
599.95	0.01\\
599.96	0.01\\
599.97	0.01\\
599.98	0.01\\
599.99	0.01\\
600	0.01\\
};
\addplot [color=red!25!mycolor17,solid,forget plot]
  table[row sep=crcr]{%
0.01	0.00980203451190364\\
1.01	0.00980203461505372\\
2.01	0.00980203472041514\\
3.01	0.00980203482803558\\
4.01	0.00980203493796371\\
5.01	0.00980203505024933\\
6.01	0.00980203516494325\\
7.01	0.00980203528209744\\
8.01	0.00980203540176496\\
9.01	0.00980203552400002\\
10.01	0.00980203564885804\\
11.01	0.00980203577639563\\
12.01	0.00980203590667063\\
13.01	0.00980203603974211\\
14.01	0.00980203617567048\\
15.01	0.00980203631451742\\
16.01	0.00980203645634597\\
17.01	0.00980203660122053\\
18.01	0.0098020367492069\\
19.01	0.00980203690037236\\
20.01	0.00980203705478558\\
21.01	0.00980203721251679\\
22.01	0.00980203737363769\\
23.01	0.00980203753822158\\
24.01	0.00980203770634337\\
25.01	0.00980203787807956\\
26.01	0.00980203805350837\\
27.01	0.00980203823270968\\
28.01	0.00980203841576513\\
29.01	0.00980203860275812\\
30.01	0.00980203879377392\\
31.01	0.00980203898889962\\
32.01	0.00980203918822421\\
33.01	0.00980203939183867\\
34.01	0.00980203959983588\\
35.01	0.00980203981231083\\
36.01	0.00980204002936054\\
37.01	0.00980204025108421\\
38.01	0.0098020404775831\\
39.01	0.0098020407089608\\
40.01	0.00980204094532308\\
41.01	0.00980204118677806\\
42.01	0.00980204143343625\\
43.01	0.00980204168541053\\
44.01	0.00980204194281626\\
45.01	0.00980204220577137\\
46.01	0.00980204247439631\\
47.01	0.00980204274881421\\
48.01	0.00980204302915088\\
49.01	0.00980204331553492\\
50.01	0.00980204360809769\\
51.01	0.0098020439069735\\
52.01	0.00980204421229954\\
53.01	0.00980204452421604\\
54.01	0.00980204484286633\\
55.01	0.00980204516839683\\
56.01	0.00980204550095724\\
57.01	0.0098020458407005\\
58.01	0.00980204618778292\\
59.01	0.00980204654236427\\
60.01	0.00980204690460779\\
61.01	0.00980204727468032\\
62.01	0.00980204765275242\\
63.01	0.00980204803899836\\
64.01	0.00980204843359621\\
65.01	0.00980204883672801\\
66.01	0.0098020492485798\\
67.01	0.00980204966934171\\
68.01	0.00980205009920807\\
69.01	0.00980205053837746\\
70.01	0.00980205098705287\\
71.01	0.00980205144544175\\
72.01	0.00980205191375614\\
73.01	0.00980205239221273\\
74.01	0.00980205288103301\\
75.01	0.00980205338044336\\
76.01	0.00980205389067516\\
77.01	0.00980205441196491\\
78.01	0.00980205494455431\\
79.01	0.00980205548869043\\
80.01	0.00980205604462574\\
81.01	0.00980205661261838\\
82.01	0.00980205719293218\\
83.01	0.00980205778583676\\
84.01	0.00980205839160772\\
85.01	0.00980205901052681\\
86.01	0.009802059642882\\
87.01	0.0098020602889676\\
88.01	0.0098020609490845\\
89.01	0.00980206162354023\\
90.01	0.00980206231264915\\
91.01	0.00980206301673259\\
92.01	0.00980206373611903\\
93.01	0.0098020644711442\\
94.01	0.00980206522215133\\
95.01	0.00980206598949126\\
96.01	0.00980206677352265\\
97.01	0.00980206757461208\\
98.01	0.00980206839313434\\
99.01	0.00980206922947251\\
100.01	0.00980207008401826\\
101.01	0.00980207095717189\\
102.01	0.00980207184934274\\
103.01	0.00980207276094914\\
104.01	0.00980207369241881\\
105.01	0.00980207464418905\\
106.01	0.0098020756167068\\
107.01	0.00980207661042909\\
108.01	0.00980207762582308\\
109.01	0.00980207866336639\\
110.01	0.00980207972354731\\
111.01	0.00980208080686505\\
112.01	0.00980208191382994\\
113.01	0.00980208304496378\\
114.01	0.00980208420080002\\
115.01	0.00980208538188406\\
116.01	0.00980208658877349\\
117.01	0.00980208782203841\\
118.01	0.00980208908226172\\
119.01	0.00980209037003934\\
120.01	0.00980209168598059\\
121.01	0.00980209303070846\\
122.01	0.00980209440485993\\
123.01	0.00980209580908628\\
124.01	0.0098020972440534\\
125.01	0.00980209871044219\\
126.01	0.00980210020894885\\
127.01	0.00980210174028523\\
128.01	0.0098021033051792\\
129.01	0.00980210490437506\\
130.01	0.00980210653863381\\
131.01	0.00980210820873368\\
132.01	0.00980210991547038\\
133.01	0.00980211165965762\\
134.01	0.00980211344212744\\
135.01	0.00980211526373069\\
136.01	0.00980211712533743\\
137.01	0.00980211902783738\\
138.01	0.00980212097214039\\
139.01	0.0098021229591769\\
140.01	0.00980212498989838\\
141.01	0.00980212706527788\\
142.01	0.00980212918631045\\
143.01	0.00980213135401375\\
144.01	0.0098021335694285\\
145.01	0.00980213583361897\\
146.01	0.00980213814767369\\
147.01	0.00980214051270584\\
148.01	0.00980214292985395\\
149.01	0.00980214540028236\\
150.01	0.00980214792518196\\
151.01	0.00980215050577073\\
152.01	0.00980215314329436\\
153.01	0.00980215583902694\\
154.01	0.00980215859427159\\
155.01	0.00980216141036116\\
156.01	0.0098021642886589\\
157.01	0.00980216723055921\\
158.01	0.00980217023748829\\
159.01	0.00980217331090501\\
160.01	0.00980217645230155\\
161.01	0.00980217966320422\\
162.01	0.00980218294517434\\
163.01	0.00980218629980894\\
164.01	0.00980218972874166\\
165.01	0.00980219323364362\\
166.01	0.00980219681622425\\
167.01	0.00980220047823228\\
168.01	0.00980220422145658\\
169.01	0.00980220804772713\\
170.01	0.00980221195891603\\
171.01	0.00980221595693844\\
172.01	0.00980222004375361\\
173.01	0.00980222422136595\\
174.01	0.00980222849182608\\
175.01	0.0098022328572319\\
176.01	0.0098022373197298\\
177.01	0.00980224188151567\\
178.01	0.00980224654483618\\
179.01	0.00980225131198997\\
180.01	0.00980225618532883\\
181.01	0.00980226116725906\\
182.01	0.00980226626024265\\
183.01	0.00980227146679872\\
184.01	0.0098022767895048\\
185.01	0.00980228223099827\\
186.01	0.00980228779397779\\
187.01	0.00980229348120469\\
188.01	0.0098022992955046\\
189.01	0.00980230523976888\\
190.01	0.00980231131695622\\
191.01	0.00980231753009427\\
192.01	0.00980232388228131\\
193.01	0.00980233037668795\\
194.01	0.00980233701655877\\
195.01	0.00980234380521423\\
196.01	0.00980235074605243\\
197.01	0.00980235784255098\\
198.01	0.00980236509826898\\
199.01	0.00980237251684886\\
200.01	0.00980238010201851\\
201.01	0.0098023878575933\\
202.01	0.00980239578747825\\
203.01	0.00980240389567004\\
204.01	0.00980241218625947\\
205.01	0.00980242066343355\\
206.01	0.00980242933147791\\
207.01	0.00980243819477919\\
208.01	0.00980244725782755\\
209.01	0.00980245652521907\\
210.01	0.00980246600165843\\
211.01	0.00980247569196153\\
212.01	0.00980248560105821\\
213.01	0.00980249573399499\\
214.01	0.00980250609593801\\
215.01	0.00980251669217586\\
216.01	0.00980252752812266\\
217.01	0.00980253860932105\\
218.01	0.00980254994144547\\
219.01	0.00980256153030528\\
220.01	0.00980257338184813\\
221.01	0.00980258550216335\\
222.01	0.00980259789748542\\
223.01	0.00980261057419759\\
224.01	0.00980262353883554\\
225.01	0.00980263679809111\\
226.01	0.00980265035881616\\
227.01	0.00980266422802658\\
228.01	0.00980267841290636\\
229.01	0.0098026929208116\\
230.01	0.00980270775927498\\
231.01	0.00980272293601005\\
232.01	0.0098027384589157\\
233.01	0.0098027543360808\\
234.01	0.00980277057578893\\
235.01	0.00980278718652324\\
236.01	0.0098028041769714\\
237.01	0.00980282155603073\\
238.01	0.00980283933281337\\
239.01	0.00980285751665176\\
240.01	0.00980287611710407\\
241.01	0.00980289514395988\\
242.01	0.00980291460724596\\
243.01	0.00980293451723226\\
244.01	0.00980295488443798\\
245.01	0.00980297571963784\\
246.01	0.00980299703386851\\
247.01	0.00980301883843521\\
248.01	0.00980304114491846\\
249.01	0.00980306396518106\\
250.01	0.00980308731137515\\
251.01	0.00980311119594953\\
252.01	0.00980313563165728\\
253.01	0.00980316063156323\\
254.01	0.00980318620905208\\
255.01	0.00980321237783634\\
256.01	0.00980323915196476\\
257.01	0.00980326654583077\\
258.01	0.00980329457418129\\
259.01	0.00980332325212569\\
260.01	0.00980335259514508\\
261.01	0.00980338261910163\\
262.01	0.00980341334024839\\
263.01	0.00980344477523923\\
264.01	0.00980347694113912\\
265.01	0.00980350985543446\\
266.01	0.00980354353604404\\
267.01	0.00980357800132998\\
268.01	0.0098036132701091\\
269.01	0.00980364936166458\\
270.01	0.00980368629575788\\
271.01	0.00980372409264102\\
272.01	0.00980376277306918\\
273.01	0.00980380235831351\\
274.01	0.00980384287017456\\
275.01	0.00980388433099574\\
276.01	0.0098039267636773\\
277.01	0.00980397019169068\\
278.01	0.00980401463909328\\
279.01	0.00980406013054341\\
280.01	0.00980410669131584\\
281.01	0.00980415434731772\\
282.01	0.00980420312510484\\
283.01	0.00980425305189839\\
284.01	0.00980430415560211\\
285.01	0.00980435646481999\\
286.01	0.0098044100088742\\
287.01	0.00980446481782378\\
288.01	0.00980452092248366\\
289.01	0.00980457835444419\\
290.01	0.00980463714609115\\
291.01	0.00980469733062639\\
292.01	0.0098047589420889\\
293.01	0.00980482201537648\\
294.01	0.00980488658626797\\
295.01	0.00980495269144602\\
296.01	0.00980502036852058\\
297.01	0.00980508965605276\\
298.01	0.00980516059357952\\
299.01	0.00980523322163897\\
300.01	0.00980530758179619\\
301.01	0.00980538371666995\\
302.01	0.00980546166995975\\
303.01	0.00980554148647398\\
304.01	0.00980562321215851\\
305.01	0.00980570689412618\\
306.01	0.00980579258068682\\
307.01	0.00980588032137833\\
308.01	0.00980597016699834\\
309.01	0.00980606216963671\\
310.01	0.00980615638270898\\
311.01	0.00980625286099045\\
312.01	0.00980635166065133\\
313.01	0.00980645283929258\\
314.01	0.00980655645598292\\
315.01	0.00980666257129639\\
316.01	0.00980677124735119\\
317.01	0.00980688254784933\\
318.01	0.00980699653811726\\
319.01	0.00980711328514754\\
320.01	0.00980723285764154\\
321.01	0.00980735532605316\\
322.01	0.00980748076263367\\
323.01	0.00980760924147757\\
324.01	0.00980774083856965\\
325.01	0.00980787563183319\\
326.01	0.00980801370117916\\
327.01	0.00980815512855689\\
328.01	0.00980829999800574\\
329.01	0.00980844839570805\\
330.01	0.00980860041004337\\
331.01	0.00980875613164409\\
332.01	0.00980891565345213\\
333.01	0.00980907907077717\\
334.01	0.00980924648135629\\
335.01	0.00980941798541476\\
336.01	0.00980959368572856\\
337.01	0.00980977368768805\\
338.01	0.00980995809936338\\
339.01	0.00981014703157126\\
340.01	0.0098103405979433\\
341.01	0.00981053891499589\\
342.01	0.00981074210220188\\
343.01	0.00981095028206352\\
344.01	0.00981116358018744\\
345.01	0.00981138212536114\\
346.01	0.00981160604963131\\
347.01	0.00981183548838374\\
348.01	0.00981207058042535\\
349.01	0.00981231146806785\\
350.01	0.00981255829721333\\
351.01	0.00981281121744186\\
352.01	0.00981307038210104\\
353.01	0.00981333594839757\\
354.01	0.00981360807749095\\
355.01	0.00981388693458924\\
356.01	0.00981417268904712\\
357.01	0.0098144655144661\\
358.01	0.00981476558879709\\
359.01	0.00981507309444537\\
360.01	0.00981538821837796\\
361.01	0.00981571115223354\\
362.01	0.00981604209243492\\
363.01	0.00981638124030427\\
364.01	0.00981672880218104\\
365.01	0.00981708498954277\\
366.01	0.00981745001912881\\
367.01	0.00981782411306719\\
368.01	0.00981820749900443\\
369.01	0.00981860041023897\\
370.01	0.00981900308585763\\
371.01	0.00981941577087583\\
372.01	0.00981983871638133\\
373.01	0.00982027217968185\\
374.01	0.00982071642445649\\
375.01	0.00982117172091147\\
376.01	0.00982163834593977\\
377.01	0.00982211658328549\\
378.01	0.00982260672371261\\
379.01	0.00982310906517857\\
380.01	0.00982362391301274\\
381.01	0.0098241515801001\\
382.01	0.00982469238707007\\
383.01	0.00982524666249113\\
384.01	0.00982581474307093\\
385.01	0.00982639697386255\\
386.01	0.00982699370847676\\
387.01	0.00982760530930099\\
388.01	0.00982823214772481\\
389.01	0.00982887460437256\\
390.01	0.0098295330693433\\
391.01	0.00983020794245857\\
392.01	0.00983089963351855\\
393.01	0.0098316085625669\\
394.01	0.00983233516016546\\
395.01	0.00983307986767956\\
396.01	0.00983384313757525\\
397.01	0.00983462543373026\\
398.01	0.00983542723176098\\
399.01	0.00983624901936843\\
400.01	0.00983709129670747\\
401.01	0.0098379545767848\\
402.01	0.00983883938589321\\
403.01	0.00983974626409236\\
404.01	0.00984067576574953\\
405.01	0.00984162846015902\\
406.01	0.00984260493226483\\
407.01	0.00984360578352005\\
408.01	0.00984463163292778\\
409.01	0.00984568311832402\\
410.01	0.00984676089798293\\
411.01	0.00984786565265236\\
412.01	0.00984899808815074\\
413.01	0.00985015893850606\\
414.01	0.00985134896684873\\
415.01	0.00985256895867454\\
416.01	0.00985381972994977\\
417.01	0.0098551021324202\\
418.01	0.00985641704423509\\
419.01	0.00985776542681464\\
420.01	0.00985914827999014\\
421.01	0.00986056534351576\\
422.01	0.00986201655058395\\
423.01	0.00986350478886627\\
424.01	0.0098650311722287\\
425.01	0.00986659675550315\\
426.01	0.00986820262515298\\
427.01	0.00986984990029115\\
428.01	0.00987153973373128\\
429.01	0.00987327331307184\\
430.01	0.0098750518618138\\
431.01	0.00987687664051176\\
432.01	0.00987874894795836\\
433.01	0.00988067012240117\\
434.01	0.00988264154279109\\
435.01	0.00988466463006031\\
436.01	0.00988674084842737\\
437.01	0.0098888717067257\\
438.01	0.00989105875975091\\
439.01	0.00989330360962027\\
440.01	0.00989560790713621\\
441.01	0.0098979733531429\\
442.01	0.00990040169986228\\
443.01	0.00990289475219151\\
444.01	0.00990545436893974\\
445.01	0.00990808246397572\\
446.01	0.0099107810072504\\
447.01	0.00991355202564792\\
448.01	0.00991639760359253\\
449.01	0.00991931988315817\\
450.01	0.0099223210615491\\
451.01	0.00992540336719763\\
452.01	0.00992856901146673\\
453.01	0.00993182063564788\\
454.01	0.00993516080375945\\
455.01	0.00993859197332626\\
456.01	0.00994211666275934\\
457.01	0.00994573759471677\\
458.01	0.00994945870445162\\
459.01	0.00995328829824902\\
460.01	0.00995721174437341\\
461.01	0.00996117126676836\\
462.01	0.00996523561568764\\
463.01	0.00996941206370552\\
464.01	0.00997370348413337\\
465.01	0.00997811270845559\\
466.01	0.00998264248507081\\
467.01	0.00998729532035657\\
468.01	0.0099920711809774\\
469.01	0.00999691239801213\\
470.01	0.01\\
471.01	0.01\\
472.01	0.01\\
473.01	0.01\\
474.01	0.01\\
475.01	0.01\\
476.01	0.01\\
477.01	0.01\\
478.01	0.01\\
479.01	0.01\\
480.01	0.01\\
481.01	0.01\\
482.01	0.01\\
483.01	0.01\\
484.01	0.01\\
485.01	0.01\\
486.01	0.01\\
487.01	0.01\\
488.01	0.01\\
489.01	0.01\\
490.01	0.01\\
491.01	0.01\\
492.01	0.01\\
493.01	0.01\\
494.01	0.01\\
495.01	0.01\\
496.01	0.01\\
497.01	0.01\\
498.01	0.01\\
499.01	0.01\\
500.01	0.01\\
501.01	0.01\\
502.01	0.01\\
503.01	0.01\\
504.01	0.01\\
505.01	0.01\\
506.01	0.01\\
507.01	0.01\\
508.01	0.01\\
509.01	0.01\\
510.01	0.01\\
511.01	0.01\\
512.01	0.01\\
513.01	0.01\\
514.01	0.01\\
515.01	0.01\\
516.01	0.01\\
517.01	0.01\\
518.01	0.01\\
519.01	0.01\\
520.01	0.01\\
521.01	0.01\\
522.01	0.01\\
523.01	0.01\\
524.01	0.01\\
525.01	0.01\\
526.01	0.01\\
527.01	0.01\\
528.01	0.01\\
529.01	0.01\\
530.01	0.01\\
531.01	0.01\\
532.01	0.01\\
533.01	0.01\\
534.01	0.01\\
535.01	0.01\\
536.01	0.01\\
537.01	0.01\\
538.01	0.01\\
539.01	0.01\\
540.01	0.01\\
541.01	0.01\\
542.01	0.01\\
543.01	0.01\\
544.01	0.01\\
545.01	0.01\\
546.01	0.01\\
547.01	0.01\\
548.01	0.01\\
549.01	0.01\\
550.01	0.01\\
551.01	0.01\\
552.01	0.01\\
553.01	0.01\\
554.01	0.01\\
555.01	0.01\\
556.01	0.01\\
557.01	0.01\\
558.01	0.01\\
559.01	0.01\\
560.01	0.01\\
561.01	0.01\\
562.01	0.01\\
563.01	0.01\\
564.01	0.01\\
565.01	0.01\\
566.01	0.01\\
567.01	0.01\\
568.01	0.01\\
569.01	0.01\\
570.01	0.01\\
571.01	0.01\\
572.01	0.01\\
573.01	0.01\\
574.01	0.01\\
575.01	0.01\\
576.01	0.01\\
577.01	0.01\\
578.01	0.01\\
579.01	0.01\\
580.01	0.01\\
581.01	0.01\\
582.01	0.01\\
583.01	0.01\\
584.01	0.01\\
585.01	0.01\\
586.01	0.01\\
587.01	0.01\\
588.01	0.01\\
589.01	0.01\\
590.01	0.01\\
591.01	0.01\\
592.01	0.01\\
593.01	0.01\\
594.01	0.01\\
595.01	0.01\\
596.01	0.01\\
597.01	0.01\\
598.01	0.01\\
599.01	0.01\\
599.02	0.01\\
599.03	0.01\\
599.04	0.01\\
599.05	0.01\\
599.06	0.01\\
599.07	0.01\\
599.08	0.01\\
599.09	0.01\\
599.1	0.01\\
599.11	0.01\\
599.12	0.01\\
599.13	0.01\\
599.14	0.01\\
599.15	0.01\\
599.16	0.01\\
599.17	0.01\\
599.18	0.01\\
599.19	0.01\\
599.2	0.01\\
599.21	0.01\\
599.22	0.01\\
599.23	0.01\\
599.24	0.01\\
599.25	0.01\\
599.26	0.01\\
599.27	0.01\\
599.28	0.01\\
599.29	0.01\\
599.3	0.01\\
599.31	0.01\\
599.32	0.01\\
599.33	0.01\\
599.34	0.01\\
599.35	0.01\\
599.36	0.01\\
599.37	0.01\\
599.38	0.01\\
599.39	0.01\\
599.4	0.01\\
599.41	0.01\\
599.42	0.01\\
599.43	0.01\\
599.44	0.01\\
599.45	0.01\\
599.46	0.01\\
599.47	0.01\\
599.48	0.01\\
599.49	0.01\\
599.5	0.01\\
599.51	0.01\\
599.52	0.01\\
599.53	0.01\\
599.54	0.01\\
599.55	0.01\\
599.56	0.01\\
599.57	0.01\\
599.58	0.01\\
599.59	0.01\\
599.6	0.01\\
599.61	0.01\\
599.62	0.01\\
599.63	0.01\\
599.64	0.01\\
599.65	0.01\\
599.66	0.01\\
599.67	0.01\\
599.68	0.01\\
599.69	0.01\\
599.7	0.01\\
599.71	0.01\\
599.72	0.01\\
599.73	0.01\\
599.74	0.01\\
599.75	0.01\\
599.76	0.01\\
599.77	0.01\\
599.78	0.01\\
599.79	0.01\\
599.8	0.01\\
599.81	0.01\\
599.82	0.01\\
599.83	0.01\\
599.84	0.01\\
599.85	0.01\\
599.86	0.01\\
599.87	0.01\\
599.88	0.01\\
599.89	0.01\\
599.9	0.01\\
599.91	0.01\\
599.92	0.01\\
599.93	0.01\\
599.94	0.01\\
599.95	0.01\\
599.96	0.01\\
599.97	0.01\\
599.98	0.01\\
599.99	0.01\\
600	0.01\\
};
\addplot [color=mycolor19,solid,forget plot]
  table[row sep=crcr]{%
0.01	0.01\\
1.01	0.01\\
2.01	0.01\\
3.01	0.01\\
4.01	0.01\\
5.01	0.01\\
6.01	0.01\\
7.01	0.01\\
8.01	0.01\\
9.01	0.01\\
10.01	0.01\\
11.01	0.01\\
12.01	0.01\\
13.01	0.01\\
14.01	0.01\\
15.01	0.01\\
16.01	0.01\\
17.01	0.01\\
18.01	0.01\\
19.01	0.01\\
20.01	0.01\\
21.01	0.01\\
22.01	0.01\\
23.01	0.01\\
24.01	0.01\\
25.01	0.01\\
26.01	0.01\\
27.01	0.01\\
28.01	0.01\\
29.01	0.01\\
30.01	0.01\\
31.01	0.01\\
32.01	0.01\\
33.01	0.01\\
34.01	0.01\\
35.01	0.01\\
36.01	0.01\\
37.01	0.01\\
38.01	0.01\\
39.01	0.01\\
40.01	0.01\\
41.01	0.01\\
42.01	0.01\\
43.01	0.01\\
44.01	0.01\\
45.01	0.01\\
46.01	0.01\\
47.01	0.01\\
48.01	0.01\\
49.01	0.01\\
50.01	0.01\\
51.01	0.01\\
52.01	0.01\\
53.01	0.01\\
54.01	0.01\\
55.01	0.01\\
56.01	0.01\\
57.01	0.01\\
58.01	0.01\\
59.01	0.01\\
60.01	0.01\\
61.01	0.01\\
62.01	0.01\\
63.01	0.01\\
64.01	0.01\\
65.01	0.01\\
66.01	0.01\\
67.01	0.01\\
68.01	0.01\\
69.01	0.01\\
70.01	0.01\\
71.01	0.01\\
72.01	0.01\\
73.01	0.01\\
74.01	0.01\\
75.01	0.01\\
76.01	0.01\\
77.01	0.01\\
78.01	0.01\\
79.01	0.01\\
80.01	0.01\\
81.01	0.01\\
82.01	0.01\\
83.01	0.01\\
84.01	0.01\\
85.01	0.01\\
86.01	0.01\\
87.01	0.01\\
88.01	0.01\\
89.01	0.01\\
90.01	0.01\\
91.01	0.01\\
92.01	0.01\\
93.01	0.01\\
94.01	0.01\\
95.01	0.01\\
96.01	0.01\\
97.01	0.01\\
98.01	0.01\\
99.01	0.01\\
100.01	0.01\\
101.01	0.01\\
102.01	0.01\\
103.01	0.01\\
104.01	0.01\\
105.01	0.01\\
106.01	0.01\\
107.01	0.01\\
108.01	0.01\\
109.01	0.01\\
110.01	0.01\\
111.01	0.01\\
112.01	0.01\\
113.01	0.01\\
114.01	0.01\\
115.01	0.01\\
116.01	0.01\\
117.01	0.01\\
118.01	0.01\\
119.01	0.01\\
120.01	0.01\\
121.01	0.01\\
122.01	0.01\\
123.01	0.01\\
124.01	0.01\\
125.01	0.01\\
126.01	0.01\\
127.01	0.01\\
128.01	0.01\\
129.01	0.01\\
130.01	0.01\\
131.01	0.01\\
132.01	0.01\\
133.01	0.01\\
134.01	0.01\\
135.01	0.01\\
136.01	0.01\\
137.01	0.01\\
138.01	0.01\\
139.01	0.01\\
140.01	0.01\\
141.01	0.01\\
142.01	0.01\\
143.01	0.01\\
144.01	0.01\\
145.01	0.01\\
146.01	0.01\\
147.01	0.01\\
148.01	0.01\\
149.01	0.01\\
150.01	0.01\\
151.01	0.01\\
152.01	0.01\\
153.01	0.01\\
154.01	0.01\\
155.01	0.01\\
156.01	0.01\\
157.01	0.01\\
158.01	0.01\\
159.01	0.01\\
160.01	0.01\\
161.01	0.01\\
162.01	0.01\\
163.01	0.01\\
164.01	0.01\\
165.01	0.01\\
166.01	0.01\\
167.01	0.01\\
168.01	0.01\\
169.01	0.01\\
170.01	0.01\\
171.01	0.01\\
172.01	0.01\\
173.01	0.01\\
174.01	0.01\\
175.01	0.01\\
176.01	0.01\\
177.01	0.01\\
178.01	0.01\\
179.01	0.01\\
180.01	0.01\\
181.01	0.01\\
182.01	0.01\\
183.01	0.01\\
184.01	0.01\\
185.01	0.01\\
186.01	0.01\\
187.01	0.01\\
188.01	0.01\\
189.01	0.01\\
190.01	0.01\\
191.01	0.01\\
192.01	0.01\\
193.01	0.01\\
194.01	0.01\\
195.01	0.01\\
196.01	0.01\\
197.01	0.01\\
198.01	0.01\\
199.01	0.01\\
200.01	0.01\\
201.01	0.01\\
202.01	0.01\\
203.01	0.01\\
204.01	0.01\\
205.01	0.01\\
206.01	0.01\\
207.01	0.01\\
208.01	0.01\\
209.01	0.01\\
210.01	0.01\\
211.01	0.01\\
212.01	0.01\\
213.01	0.01\\
214.01	0.01\\
215.01	0.01\\
216.01	0.01\\
217.01	0.01\\
218.01	0.01\\
219.01	0.01\\
220.01	0.01\\
221.01	0.01\\
222.01	0.01\\
223.01	0.01\\
224.01	0.01\\
225.01	0.01\\
226.01	0.01\\
227.01	0.01\\
228.01	0.01\\
229.01	0.01\\
230.01	0.01\\
231.01	0.01\\
232.01	0.01\\
233.01	0.01\\
234.01	0.01\\
235.01	0.01\\
236.01	0.01\\
237.01	0.01\\
238.01	0.01\\
239.01	0.01\\
240.01	0.01\\
241.01	0.01\\
242.01	0.01\\
243.01	0.01\\
244.01	0.01\\
245.01	0.01\\
246.01	0.01\\
247.01	0.01\\
248.01	0.01\\
249.01	0.01\\
250.01	0.01\\
251.01	0.01\\
252.01	0.01\\
253.01	0.01\\
254.01	0.01\\
255.01	0.01\\
256.01	0.01\\
257.01	0.01\\
258.01	0.01\\
259.01	0.01\\
260.01	0.01\\
261.01	0.01\\
262.01	0.01\\
263.01	0.01\\
264.01	0.01\\
265.01	0.01\\
266.01	0.01\\
267.01	0.01\\
268.01	0.01\\
269.01	0.01\\
270.01	0.01\\
271.01	0.01\\
272.01	0.01\\
273.01	0.01\\
274.01	0.01\\
275.01	0.01\\
276.01	0.01\\
277.01	0.01\\
278.01	0.01\\
279.01	0.01\\
280.01	0.01\\
281.01	0.01\\
282.01	0.01\\
283.01	0.01\\
284.01	0.01\\
285.01	0.01\\
286.01	0.01\\
287.01	0.01\\
288.01	0.01\\
289.01	0.01\\
290.01	0.01\\
291.01	0.01\\
292.01	0.01\\
293.01	0.01\\
294.01	0.01\\
295.01	0.01\\
296.01	0.01\\
297.01	0.01\\
298.01	0.01\\
299.01	0.01\\
300.01	0.01\\
301.01	0.01\\
302.01	0.01\\
303.01	0.01\\
304.01	0.01\\
305.01	0.01\\
306.01	0.01\\
307.01	0.01\\
308.01	0.01\\
309.01	0.01\\
310.01	0.01\\
311.01	0.01\\
312.01	0.01\\
313.01	0.01\\
314.01	0.01\\
315.01	0.01\\
316.01	0.01\\
317.01	0.01\\
318.01	0.01\\
319.01	0.01\\
320.01	0.01\\
321.01	0.01\\
322.01	0.01\\
323.01	0.01\\
324.01	0.01\\
325.01	0.01\\
326.01	0.01\\
327.01	0.01\\
328.01	0.01\\
329.01	0.01\\
330.01	0.01\\
331.01	0.01\\
332.01	0.01\\
333.01	0.01\\
334.01	0.01\\
335.01	0.01\\
336.01	0.01\\
337.01	0.01\\
338.01	0.01\\
339.01	0.01\\
340.01	0.01\\
341.01	0.01\\
342.01	0.01\\
343.01	0.01\\
344.01	0.01\\
345.01	0.01\\
346.01	0.01\\
347.01	0.01\\
348.01	0.01\\
349.01	0.01\\
350.01	0.01\\
351.01	0.01\\
352.01	0.01\\
353.01	0.01\\
354.01	0.01\\
355.01	0.01\\
356.01	0.01\\
357.01	0.01\\
358.01	0.01\\
359.01	0.01\\
360.01	0.01\\
361.01	0.01\\
362.01	0.01\\
363.01	0.01\\
364.01	0.01\\
365.01	0.01\\
366.01	0.01\\
367.01	0.01\\
368.01	0.01\\
369.01	0.01\\
370.01	0.01\\
371.01	0.01\\
372.01	0.01\\
373.01	0.01\\
374.01	0.01\\
375.01	0.01\\
376.01	0.01\\
377.01	0.01\\
378.01	0.01\\
379.01	0.01\\
380.01	0.01\\
381.01	0.01\\
382.01	0.01\\
383.01	0.01\\
384.01	0.01\\
385.01	0.01\\
386.01	0.01\\
387.01	0.01\\
388.01	0.01\\
389.01	0.01\\
390.01	0.01\\
391.01	0.01\\
392.01	0.01\\
393.01	0.01\\
394.01	0.01\\
395.01	0.01\\
396.01	0.01\\
397.01	0.01\\
398.01	0.01\\
399.01	0.01\\
400.01	0.01\\
401.01	0.01\\
402.01	0.01\\
403.01	0.01\\
404.01	0.01\\
405.01	0.01\\
406.01	0.01\\
407.01	0.01\\
408.01	0.01\\
409.01	0.01\\
410.01	0.01\\
411.01	0.01\\
412.01	0.01\\
413.01	0.01\\
414.01	0.01\\
415.01	0.01\\
416.01	0.01\\
417.01	0.01\\
418.01	0.01\\
419.01	0.01\\
420.01	0.01\\
421.01	0.01\\
422.01	0.01\\
423.01	0.01\\
424.01	0.01\\
425.01	0.01\\
426.01	0.01\\
427.01	0.01\\
428.01	0.01\\
429.01	0.01\\
430.01	0.01\\
431.01	0.01\\
432.01	0.01\\
433.01	0.01\\
434.01	0.01\\
435.01	0.01\\
436.01	0.01\\
437.01	0.01\\
438.01	0.01\\
439.01	0.01\\
440.01	0.01\\
441.01	0.01\\
442.01	0.01\\
443.01	0.01\\
444.01	0.01\\
445.01	0.01\\
446.01	0.01\\
447.01	0.01\\
448.01	0.01\\
449.01	0.01\\
450.01	0.01\\
451.01	0.01\\
452.01	0.01\\
453.01	0.01\\
454.01	0.01\\
455.01	0.01\\
456.01	0.01\\
457.01	0.01\\
458.01	0.01\\
459.01	0.01\\
460.01	0.01\\
461.01	0.01\\
462.01	0.01\\
463.01	0.01\\
464.01	0.01\\
465.01	0.01\\
466.01	0.01\\
467.01	0.01\\
468.01	0.01\\
469.01	0.01\\
470.01	0.01\\
471.01	0.01\\
472.01	0.01\\
473.01	0.01\\
474.01	0.01\\
475.01	0.01\\
476.01	0.01\\
477.01	0.01\\
478.01	0.01\\
479.01	0.01\\
480.01	0.01\\
481.01	0.01\\
482.01	0.01\\
483.01	0.01\\
484.01	0.01\\
485.01	0.01\\
486.01	0.01\\
487.01	0.01\\
488.01	0.01\\
489.01	0.01\\
490.01	0.01\\
491.01	0.01\\
492.01	0.01\\
493.01	0.01\\
494.01	0.01\\
495.01	0.01\\
496.01	0.01\\
497.01	0.01\\
498.01	0.01\\
499.01	0.01\\
500.01	0.01\\
501.01	0.01\\
502.01	0.01\\
503.01	0.01\\
504.01	0.01\\
505.01	0.01\\
506.01	0.01\\
507.01	0.01\\
508.01	0.01\\
509.01	0.01\\
510.01	0.01\\
511.01	0.01\\
512.01	0.01\\
513.01	0.01\\
514.01	0.01\\
515.01	0.01\\
516.01	0.01\\
517.01	0.01\\
518.01	0.01\\
519.01	0.01\\
520.01	0.01\\
521.01	0.01\\
522.01	0.01\\
523.01	0.01\\
524.01	0.01\\
525.01	0.01\\
526.01	0.01\\
527.01	0.01\\
528.01	0.01\\
529.01	0.01\\
530.01	0.01\\
531.01	0.01\\
532.01	0.01\\
533.01	0.01\\
534.01	0.01\\
535.01	0.01\\
536.01	0.01\\
537.01	0.01\\
538.01	0.01\\
539.01	0.01\\
540.01	0.01\\
541.01	0.01\\
542.01	0.01\\
543.01	0.01\\
544.01	0.01\\
545.01	0.01\\
546.01	0.01\\
547.01	0.01\\
548.01	0.01\\
549.01	0.01\\
550.01	0.01\\
551.01	0.01\\
552.01	0.01\\
553.01	0.01\\
554.01	0.01\\
555.01	0.01\\
556.01	0.01\\
557.01	0.01\\
558.01	0.01\\
559.01	0.01\\
560.01	0.01\\
561.01	0.01\\
562.01	0.01\\
563.01	0.01\\
564.01	0.01\\
565.01	0.01\\
566.01	0.01\\
567.01	0.01\\
568.01	0.01\\
569.01	0.01\\
570.01	0.01\\
571.01	0.01\\
572.01	0.01\\
573.01	0.01\\
574.01	0.01\\
575.01	0.01\\
576.01	0.01\\
577.01	0.01\\
578.01	0.01\\
579.01	0.01\\
580.01	0.01\\
581.01	0.01\\
582.01	0.01\\
583.01	0.01\\
584.01	0.01\\
585.01	0.01\\
586.01	0.01\\
587.01	0.01\\
588.01	0.01\\
589.01	0.01\\
590.01	0.01\\
591.01	0.01\\
592.01	0.01\\
593.01	0.01\\
594.01	0.01\\
595.01	0.01\\
596.01	0.01\\
597.01	0.01\\
598.01	0.01\\
599.01	0.01\\
599.02	0.01\\
599.03	0.01\\
599.04	0.01\\
599.05	0.01\\
599.06	0.01\\
599.07	0.01\\
599.08	0.01\\
599.09	0.01\\
599.1	0.01\\
599.11	0.01\\
599.12	0.01\\
599.13	0.01\\
599.14	0.01\\
599.15	0.01\\
599.16	0.01\\
599.17	0.01\\
599.18	0.01\\
599.19	0.01\\
599.2	0.01\\
599.21	0.01\\
599.22	0.01\\
599.23	0.01\\
599.24	0.01\\
599.25	0.01\\
599.26	0.01\\
599.27	0.01\\
599.28	0.01\\
599.29	0.01\\
599.3	0.01\\
599.31	0.01\\
599.32	0.01\\
599.33	0.01\\
599.34	0.01\\
599.35	0.01\\
599.36	0.01\\
599.37	0.01\\
599.38	0.01\\
599.39	0.01\\
599.4	0.01\\
599.41	0.01\\
599.42	0.01\\
599.43	0.01\\
599.44	0.01\\
599.45	0.01\\
599.46	0.01\\
599.47	0.01\\
599.48	0.01\\
599.49	0.01\\
599.5	0.01\\
599.51	0.01\\
599.52	0.01\\
599.53	0.01\\
599.54	0.01\\
599.55	0.01\\
599.56	0.01\\
599.57	0.01\\
599.58	0.01\\
599.59	0.01\\
599.6	0.01\\
599.61	0.01\\
599.62	0.01\\
599.63	0.01\\
599.64	0.01\\
599.65	0.01\\
599.66	0.01\\
599.67	0.01\\
599.68	0.01\\
599.69	0.01\\
599.7	0.01\\
599.71	0.01\\
599.72	0.01\\
599.73	0.01\\
599.74	0.01\\
599.75	0.01\\
599.76	0.01\\
599.77	0.01\\
599.78	0.01\\
599.79	0.01\\
599.8	0.01\\
599.81	0.01\\
599.82	0.01\\
599.83	0.01\\
599.84	0.01\\
599.85	0.01\\
599.86	0.01\\
599.87	0.01\\
599.88	0.01\\
599.89	0.01\\
599.9	0.01\\
599.91	0.01\\
599.92	0.01\\
599.93	0.01\\
599.94	0.01\\
599.95	0.01\\
599.96	0.01\\
599.97	0.01\\
599.98	0.01\\
599.99	0.01\\
600	0.01\\
};
\addplot [color=red!50!mycolor17,solid,forget plot]
  table[row sep=crcr]{%
0.01	0.01\\
1.01	0.01\\
2.01	0.01\\
3.01	0.01\\
4.01	0.01\\
5.01	0.01\\
6.01	0.01\\
7.01	0.01\\
8.01	0.01\\
9.01	0.01\\
10.01	0.01\\
11.01	0.01\\
12.01	0.01\\
13.01	0.01\\
14.01	0.01\\
15.01	0.01\\
16.01	0.01\\
17.01	0.01\\
18.01	0.01\\
19.01	0.01\\
20.01	0.01\\
21.01	0.01\\
22.01	0.01\\
23.01	0.01\\
24.01	0.01\\
25.01	0.01\\
26.01	0.01\\
27.01	0.01\\
28.01	0.01\\
29.01	0.01\\
30.01	0.01\\
31.01	0.01\\
32.01	0.01\\
33.01	0.01\\
34.01	0.01\\
35.01	0.01\\
36.01	0.01\\
37.01	0.01\\
38.01	0.01\\
39.01	0.01\\
40.01	0.01\\
41.01	0.01\\
42.01	0.01\\
43.01	0.01\\
44.01	0.01\\
45.01	0.01\\
46.01	0.01\\
47.01	0.01\\
48.01	0.01\\
49.01	0.01\\
50.01	0.01\\
51.01	0.01\\
52.01	0.01\\
53.01	0.01\\
54.01	0.01\\
55.01	0.01\\
56.01	0.01\\
57.01	0.01\\
58.01	0.01\\
59.01	0.01\\
60.01	0.01\\
61.01	0.01\\
62.01	0.01\\
63.01	0.01\\
64.01	0.01\\
65.01	0.01\\
66.01	0.01\\
67.01	0.01\\
68.01	0.01\\
69.01	0.01\\
70.01	0.01\\
71.01	0.01\\
72.01	0.01\\
73.01	0.01\\
74.01	0.01\\
75.01	0.01\\
76.01	0.01\\
77.01	0.01\\
78.01	0.01\\
79.01	0.01\\
80.01	0.01\\
81.01	0.01\\
82.01	0.01\\
83.01	0.01\\
84.01	0.01\\
85.01	0.01\\
86.01	0.01\\
87.01	0.01\\
88.01	0.01\\
89.01	0.01\\
90.01	0.01\\
91.01	0.01\\
92.01	0.01\\
93.01	0.01\\
94.01	0.01\\
95.01	0.01\\
96.01	0.01\\
97.01	0.01\\
98.01	0.01\\
99.01	0.01\\
100.01	0.01\\
101.01	0.01\\
102.01	0.01\\
103.01	0.01\\
104.01	0.01\\
105.01	0.01\\
106.01	0.01\\
107.01	0.01\\
108.01	0.01\\
109.01	0.01\\
110.01	0.01\\
111.01	0.01\\
112.01	0.01\\
113.01	0.01\\
114.01	0.01\\
115.01	0.01\\
116.01	0.01\\
117.01	0.01\\
118.01	0.01\\
119.01	0.01\\
120.01	0.01\\
121.01	0.01\\
122.01	0.01\\
123.01	0.01\\
124.01	0.01\\
125.01	0.01\\
126.01	0.01\\
127.01	0.01\\
128.01	0.01\\
129.01	0.01\\
130.01	0.01\\
131.01	0.01\\
132.01	0.01\\
133.01	0.01\\
134.01	0.01\\
135.01	0.01\\
136.01	0.01\\
137.01	0.01\\
138.01	0.01\\
139.01	0.01\\
140.01	0.01\\
141.01	0.01\\
142.01	0.01\\
143.01	0.01\\
144.01	0.01\\
145.01	0.01\\
146.01	0.01\\
147.01	0.01\\
148.01	0.01\\
149.01	0.01\\
150.01	0.01\\
151.01	0.01\\
152.01	0.01\\
153.01	0.01\\
154.01	0.01\\
155.01	0.01\\
156.01	0.01\\
157.01	0.01\\
158.01	0.01\\
159.01	0.01\\
160.01	0.01\\
161.01	0.01\\
162.01	0.01\\
163.01	0.01\\
164.01	0.01\\
165.01	0.01\\
166.01	0.01\\
167.01	0.01\\
168.01	0.01\\
169.01	0.01\\
170.01	0.01\\
171.01	0.01\\
172.01	0.01\\
173.01	0.01\\
174.01	0.01\\
175.01	0.01\\
176.01	0.01\\
177.01	0.01\\
178.01	0.01\\
179.01	0.01\\
180.01	0.01\\
181.01	0.01\\
182.01	0.01\\
183.01	0.01\\
184.01	0.01\\
185.01	0.01\\
186.01	0.01\\
187.01	0.01\\
188.01	0.01\\
189.01	0.01\\
190.01	0.01\\
191.01	0.01\\
192.01	0.01\\
193.01	0.01\\
194.01	0.01\\
195.01	0.01\\
196.01	0.01\\
197.01	0.01\\
198.01	0.01\\
199.01	0.01\\
200.01	0.01\\
201.01	0.01\\
202.01	0.01\\
203.01	0.01\\
204.01	0.01\\
205.01	0.01\\
206.01	0.01\\
207.01	0.01\\
208.01	0.01\\
209.01	0.01\\
210.01	0.01\\
211.01	0.01\\
212.01	0.01\\
213.01	0.01\\
214.01	0.01\\
215.01	0.01\\
216.01	0.01\\
217.01	0.01\\
218.01	0.01\\
219.01	0.01\\
220.01	0.01\\
221.01	0.01\\
222.01	0.01\\
223.01	0.01\\
224.01	0.01\\
225.01	0.01\\
226.01	0.01\\
227.01	0.01\\
228.01	0.01\\
229.01	0.01\\
230.01	0.01\\
231.01	0.01\\
232.01	0.01\\
233.01	0.01\\
234.01	0.01\\
235.01	0.01\\
236.01	0.01\\
237.01	0.01\\
238.01	0.01\\
239.01	0.01\\
240.01	0.01\\
241.01	0.01\\
242.01	0.01\\
243.01	0.01\\
244.01	0.01\\
245.01	0.01\\
246.01	0.01\\
247.01	0.01\\
248.01	0.01\\
249.01	0.01\\
250.01	0.01\\
251.01	0.01\\
252.01	0.01\\
253.01	0.01\\
254.01	0.01\\
255.01	0.01\\
256.01	0.01\\
257.01	0.01\\
258.01	0.01\\
259.01	0.01\\
260.01	0.01\\
261.01	0.01\\
262.01	0.01\\
263.01	0.01\\
264.01	0.01\\
265.01	0.01\\
266.01	0.01\\
267.01	0.01\\
268.01	0.01\\
269.01	0.01\\
270.01	0.01\\
271.01	0.01\\
272.01	0.01\\
273.01	0.01\\
274.01	0.01\\
275.01	0.01\\
276.01	0.01\\
277.01	0.01\\
278.01	0.01\\
279.01	0.01\\
280.01	0.01\\
281.01	0.01\\
282.01	0.01\\
283.01	0.01\\
284.01	0.01\\
285.01	0.01\\
286.01	0.01\\
287.01	0.01\\
288.01	0.01\\
289.01	0.01\\
290.01	0.01\\
291.01	0.01\\
292.01	0.01\\
293.01	0.01\\
294.01	0.01\\
295.01	0.01\\
296.01	0.01\\
297.01	0.01\\
298.01	0.01\\
299.01	0.01\\
300.01	0.01\\
301.01	0.01\\
302.01	0.01\\
303.01	0.01\\
304.01	0.01\\
305.01	0.01\\
306.01	0.01\\
307.01	0.01\\
308.01	0.01\\
309.01	0.01\\
310.01	0.01\\
311.01	0.01\\
312.01	0.01\\
313.01	0.01\\
314.01	0.01\\
315.01	0.01\\
316.01	0.01\\
317.01	0.01\\
318.01	0.01\\
319.01	0.01\\
320.01	0.01\\
321.01	0.01\\
322.01	0.01\\
323.01	0.01\\
324.01	0.01\\
325.01	0.01\\
326.01	0.01\\
327.01	0.01\\
328.01	0.01\\
329.01	0.01\\
330.01	0.01\\
331.01	0.01\\
332.01	0.01\\
333.01	0.01\\
334.01	0.01\\
335.01	0.01\\
336.01	0.01\\
337.01	0.01\\
338.01	0.01\\
339.01	0.01\\
340.01	0.01\\
341.01	0.01\\
342.01	0.01\\
343.01	0.01\\
344.01	0.01\\
345.01	0.01\\
346.01	0.01\\
347.01	0.01\\
348.01	0.01\\
349.01	0.01\\
350.01	0.01\\
351.01	0.01\\
352.01	0.01\\
353.01	0.01\\
354.01	0.01\\
355.01	0.01\\
356.01	0.01\\
357.01	0.01\\
358.01	0.01\\
359.01	0.01\\
360.01	0.01\\
361.01	0.01\\
362.01	0.01\\
363.01	0.01\\
364.01	0.01\\
365.01	0.01\\
366.01	0.01\\
367.01	0.01\\
368.01	0.01\\
369.01	0.01\\
370.01	0.01\\
371.01	0.01\\
372.01	0.01\\
373.01	0.01\\
374.01	0.01\\
375.01	0.01\\
376.01	0.01\\
377.01	0.01\\
378.01	0.01\\
379.01	0.01\\
380.01	0.01\\
381.01	0.01\\
382.01	0.01\\
383.01	0.01\\
384.01	0.01\\
385.01	0.01\\
386.01	0.01\\
387.01	0.01\\
388.01	0.01\\
389.01	0.01\\
390.01	0.01\\
391.01	0.01\\
392.01	0.01\\
393.01	0.01\\
394.01	0.01\\
395.01	0.01\\
396.01	0.01\\
397.01	0.01\\
398.01	0.01\\
399.01	0.01\\
400.01	0.01\\
401.01	0.01\\
402.01	0.01\\
403.01	0.01\\
404.01	0.01\\
405.01	0.01\\
406.01	0.01\\
407.01	0.01\\
408.01	0.01\\
409.01	0.01\\
410.01	0.01\\
411.01	0.01\\
412.01	0.01\\
413.01	0.01\\
414.01	0.01\\
415.01	0.01\\
416.01	0.01\\
417.01	0.01\\
418.01	0.01\\
419.01	0.01\\
420.01	0.01\\
421.01	0.01\\
422.01	0.01\\
423.01	0.01\\
424.01	0.01\\
425.01	0.01\\
426.01	0.01\\
427.01	0.01\\
428.01	0.01\\
429.01	0.01\\
430.01	0.01\\
431.01	0.01\\
432.01	0.01\\
433.01	0.01\\
434.01	0.01\\
435.01	0.01\\
436.01	0.01\\
437.01	0.01\\
438.01	0.01\\
439.01	0.01\\
440.01	0.01\\
441.01	0.01\\
442.01	0.01\\
443.01	0.01\\
444.01	0.01\\
445.01	0.01\\
446.01	0.01\\
447.01	0.01\\
448.01	0.01\\
449.01	0.01\\
450.01	0.01\\
451.01	0.01\\
452.01	0.01\\
453.01	0.01\\
454.01	0.01\\
455.01	0.01\\
456.01	0.01\\
457.01	0.01\\
458.01	0.01\\
459.01	0.01\\
460.01	0.01\\
461.01	0.01\\
462.01	0.01\\
463.01	0.01\\
464.01	0.01\\
465.01	0.01\\
466.01	0.01\\
467.01	0.01\\
468.01	0.01\\
469.01	0.01\\
470.01	0.01\\
471.01	0.01\\
472.01	0.01\\
473.01	0.01\\
474.01	0.01\\
475.01	0.01\\
476.01	0.01\\
477.01	0.01\\
478.01	0.01\\
479.01	0.01\\
480.01	0.01\\
481.01	0.01\\
482.01	0.01\\
483.01	0.01\\
484.01	0.01\\
485.01	0.01\\
486.01	0.01\\
487.01	0.01\\
488.01	0.01\\
489.01	0.01\\
490.01	0.01\\
491.01	0.01\\
492.01	0.01\\
493.01	0.01\\
494.01	0.01\\
495.01	0.01\\
496.01	0.01\\
497.01	0.01\\
498.01	0.01\\
499.01	0.01\\
500.01	0.01\\
501.01	0.01\\
502.01	0.01\\
503.01	0.01\\
504.01	0.01\\
505.01	0.01\\
506.01	0.01\\
507.01	0.01\\
508.01	0.01\\
509.01	0.01\\
510.01	0.01\\
511.01	0.01\\
512.01	0.01\\
513.01	0.01\\
514.01	0.01\\
515.01	0.01\\
516.01	0.01\\
517.01	0.01\\
518.01	0.01\\
519.01	0.01\\
520.01	0.01\\
521.01	0.01\\
522.01	0.01\\
523.01	0.01\\
524.01	0.01\\
525.01	0.01\\
526.01	0.01\\
527.01	0.01\\
528.01	0.01\\
529.01	0.01\\
530.01	0.01\\
531.01	0.01\\
532.01	0.01\\
533.01	0.01\\
534.01	0.01\\
535.01	0.01\\
536.01	0.01\\
537.01	0.01\\
538.01	0.01\\
539.01	0.01\\
540.01	0.01\\
541.01	0.01\\
542.01	0.01\\
543.01	0.01\\
544.01	0.01\\
545.01	0.01\\
546.01	0.01\\
547.01	0.01\\
548.01	0.01\\
549.01	0.01\\
550.01	0.01\\
551.01	0.01\\
552.01	0.01\\
553.01	0.01\\
554.01	0.01\\
555.01	0.01\\
556.01	0.01\\
557.01	0.01\\
558.01	0.01\\
559.01	0.01\\
560.01	0.01\\
561.01	0.01\\
562.01	0.01\\
563.01	0.01\\
564.01	0.01\\
565.01	0.01\\
566.01	0.01\\
567.01	0.01\\
568.01	0.01\\
569.01	0.01\\
570.01	0.01\\
571.01	0.01\\
572.01	0.01\\
573.01	0.01\\
574.01	0.01\\
575.01	0.01\\
576.01	0.01\\
577.01	0.01\\
578.01	0.01\\
579.01	0.01\\
580.01	0.01\\
581.01	0.01\\
582.01	0.01\\
583.01	0.01\\
584.01	0.01\\
585.01	0.01\\
586.01	0.01\\
587.01	0.01\\
588.01	0.01\\
589.01	0.01\\
590.01	0.01\\
591.01	0.01\\
592.01	0.01\\
593.01	0.01\\
594.01	0.01\\
595.01	0.01\\
596.01	0.01\\
597.01	0.01\\
598.01	0.01\\
599.01	0.01\\
599.02	0.01\\
599.03	0.01\\
599.04	0.01\\
599.05	0.01\\
599.06	0.01\\
599.07	0.01\\
599.08	0.01\\
599.09	0.01\\
599.1	0.01\\
599.11	0.01\\
599.12	0.01\\
599.13	0.01\\
599.14	0.01\\
599.15	0.01\\
599.16	0.01\\
599.17	0.01\\
599.18	0.01\\
599.19	0.01\\
599.2	0.01\\
599.21	0.01\\
599.22	0.01\\
599.23	0.01\\
599.24	0.01\\
599.25	0.01\\
599.26	0.01\\
599.27	0.01\\
599.28	0.01\\
599.29	0.01\\
599.3	0.01\\
599.31	0.01\\
599.32	0.01\\
599.33	0.01\\
599.34	0.01\\
599.35	0.01\\
599.36	0.01\\
599.37	0.01\\
599.38	0.01\\
599.39	0.01\\
599.4	0.01\\
599.41	0.01\\
599.42	0.01\\
599.43	0.01\\
599.44	0.01\\
599.45	0.01\\
599.46	0.01\\
599.47	0.01\\
599.48	0.01\\
599.49	0.01\\
599.5	0.01\\
599.51	0.01\\
599.52	0.01\\
599.53	0.01\\
599.54	0.01\\
599.55	0.01\\
599.56	0.01\\
599.57	0.01\\
599.58	0.01\\
599.59	0.01\\
599.6	0.01\\
599.61	0.01\\
599.62	0.01\\
599.63	0.01\\
599.64	0.01\\
599.65	0.01\\
599.66	0.01\\
599.67	0.01\\
599.68	0.01\\
599.69	0.01\\
599.7	0.01\\
599.71	0.01\\
599.72	0.01\\
599.73	0.01\\
599.74	0.01\\
599.75	0.01\\
599.76	0.01\\
599.77	0.01\\
599.78	0.01\\
599.79	0.01\\
599.8	0.01\\
599.81	0.01\\
599.82	0.01\\
599.83	0.01\\
599.84	0.01\\
599.85	0.01\\
599.86	0.01\\
599.87	0.01\\
599.88	0.01\\
599.89	0.01\\
599.9	0.01\\
599.91	0.01\\
599.92	0.01\\
599.93	0.01\\
599.94	0.01\\
599.95	0.01\\
599.96	0.01\\
599.97	0.01\\
599.98	0.01\\
599.99	0.01\\
600	0.01\\
};
\addplot [color=red!40!mycolor19,solid,forget plot]
  table[row sep=crcr]{%
0.01	0.01\\
1.01	0.01\\
2.01	0.01\\
3.01	0.01\\
4.01	0.01\\
5.01	0.01\\
6.01	0.01\\
7.01	0.01\\
8.01	0.01\\
9.01	0.01\\
10.01	0.01\\
11.01	0.01\\
12.01	0.01\\
13.01	0.01\\
14.01	0.01\\
15.01	0.01\\
16.01	0.01\\
17.01	0.01\\
18.01	0.01\\
19.01	0.01\\
20.01	0.01\\
21.01	0.01\\
22.01	0.01\\
23.01	0.01\\
24.01	0.01\\
25.01	0.01\\
26.01	0.01\\
27.01	0.01\\
28.01	0.01\\
29.01	0.01\\
30.01	0.01\\
31.01	0.01\\
32.01	0.01\\
33.01	0.01\\
34.01	0.01\\
35.01	0.01\\
36.01	0.01\\
37.01	0.01\\
38.01	0.01\\
39.01	0.01\\
40.01	0.01\\
41.01	0.01\\
42.01	0.01\\
43.01	0.01\\
44.01	0.01\\
45.01	0.01\\
46.01	0.01\\
47.01	0.01\\
48.01	0.01\\
49.01	0.01\\
50.01	0.01\\
51.01	0.01\\
52.01	0.01\\
53.01	0.01\\
54.01	0.01\\
55.01	0.01\\
56.01	0.01\\
57.01	0.01\\
58.01	0.01\\
59.01	0.01\\
60.01	0.01\\
61.01	0.01\\
62.01	0.01\\
63.01	0.01\\
64.01	0.01\\
65.01	0.01\\
66.01	0.01\\
67.01	0.01\\
68.01	0.01\\
69.01	0.01\\
70.01	0.01\\
71.01	0.01\\
72.01	0.01\\
73.01	0.01\\
74.01	0.01\\
75.01	0.01\\
76.01	0.01\\
77.01	0.01\\
78.01	0.01\\
79.01	0.01\\
80.01	0.01\\
81.01	0.01\\
82.01	0.01\\
83.01	0.01\\
84.01	0.01\\
85.01	0.01\\
86.01	0.01\\
87.01	0.01\\
88.01	0.01\\
89.01	0.01\\
90.01	0.01\\
91.01	0.01\\
92.01	0.01\\
93.01	0.01\\
94.01	0.01\\
95.01	0.01\\
96.01	0.01\\
97.01	0.01\\
98.01	0.01\\
99.01	0.01\\
100.01	0.01\\
101.01	0.01\\
102.01	0.01\\
103.01	0.01\\
104.01	0.01\\
105.01	0.01\\
106.01	0.01\\
107.01	0.01\\
108.01	0.01\\
109.01	0.01\\
110.01	0.01\\
111.01	0.01\\
112.01	0.01\\
113.01	0.01\\
114.01	0.01\\
115.01	0.01\\
116.01	0.01\\
117.01	0.01\\
118.01	0.01\\
119.01	0.01\\
120.01	0.01\\
121.01	0.01\\
122.01	0.01\\
123.01	0.01\\
124.01	0.01\\
125.01	0.01\\
126.01	0.01\\
127.01	0.01\\
128.01	0.01\\
129.01	0.01\\
130.01	0.01\\
131.01	0.01\\
132.01	0.01\\
133.01	0.01\\
134.01	0.01\\
135.01	0.01\\
136.01	0.01\\
137.01	0.01\\
138.01	0.01\\
139.01	0.01\\
140.01	0.01\\
141.01	0.01\\
142.01	0.01\\
143.01	0.01\\
144.01	0.01\\
145.01	0.01\\
146.01	0.01\\
147.01	0.01\\
148.01	0.01\\
149.01	0.01\\
150.01	0.01\\
151.01	0.01\\
152.01	0.01\\
153.01	0.01\\
154.01	0.01\\
155.01	0.01\\
156.01	0.01\\
157.01	0.01\\
158.01	0.01\\
159.01	0.01\\
160.01	0.01\\
161.01	0.01\\
162.01	0.01\\
163.01	0.01\\
164.01	0.01\\
165.01	0.01\\
166.01	0.01\\
167.01	0.01\\
168.01	0.01\\
169.01	0.01\\
170.01	0.01\\
171.01	0.01\\
172.01	0.01\\
173.01	0.01\\
174.01	0.01\\
175.01	0.01\\
176.01	0.01\\
177.01	0.01\\
178.01	0.01\\
179.01	0.01\\
180.01	0.01\\
181.01	0.01\\
182.01	0.01\\
183.01	0.01\\
184.01	0.01\\
185.01	0.01\\
186.01	0.01\\
187.01	0.01\\
188.01	0.01\\
189.01	0.01\\
190.01	0.01\\
191.01	0.01\\
192.01	0.01\\
193.01	0.01\\
194.01	0.01\\
195.01	0.01\\
196.01	0.01\\
197.01	0.01\\
198.01	0.01\\
199.01	0.01\\
200.01	0.01\\
201.01	0.01\\
202.01	0.01\\
203.01	0.01\\
204.01	0.01\\
205.01	0.01\\
206.01	0.01\\
207.01	0.01\\
208.01	0.01\\
209.01	0.01\\
210.01	0.01\\
211.01	0.01\\
212.01	0.01\\
213.01	0.01\\
214.01	0.01\\
215.01	0.01\\
216.01	0.01\\
217.01	0.01\\
218.01	0.01\\
219.01	0.01\\
220.01	0.01\\
221.01	0.01\\
222.01	0.01\\
223.01	0.01\\
224.01	0.01\\
225.01	0.01\\
226.01	0.01\\
227.01	0.01\\
228.01	0.01\\
229.01	0.01\\
230.01	0.01\\
231.01	0.01\\
232.01	0.01\\
233.01	0.01\\
234.01	0.01\\
235.01	0.01\\
236.01	0.01\\
237.01	0.01\\
238.01	0.01\\
239.01	0.01\\
240.01	0.01\\
241.01	0.01\\
242.01	0.01\\
243.01	0.01\\
244.01	0.01\\
245.01	0.01\\
246.01	0.01\\
247.01	0.01\\
248.01	0.01\\
249.01	0.01\\
250.01	0.01\\
251.01	0.01\\
252.01	0.01\\
253.01	0.01\\
254.01	0.01\\
255.01	0.01\\
256.01	0.01\\
257.01	0.01\\
258.01	0.01\\
259.01	0.01\\
260.01	0.01\\
261.01	0.01\\
262.01	0.01\\
263.01	0.01\\
264.01	0.01\\
265.01	0.01\\
266.01	0.01\\
267.01	0.01\\
268.01	0.01\\
269.01	0.01\\
270.01	0.01\\
271.01	0.01\\
272.01	0.01\\
273.01	0.01\\
274.01	0.01\\
275.01	0.01\\
276.01	0.01\\
277.01	0.01\\
278.01	0.01\\
279.01	0.01\\
280.01	0.01\\
281.01	0.01\\
282.01	0.01\\
283.01	0.01\\
284.01	0.01\\
285.01	0.01\\
286.01	0.01\\
287.01	0.01\\
288.01	0.01\\
289.01	0.01\\
290.01	0.01\\
291.01	0.01\\
292.01	0.01\\
293.01	0.01\\
294.01	0.01\\
295.01	0.01\\
296.01	0.01\\
297.01	0.01\\
298.01	0.01\\
299.01	0.01\\
300.01	0.01\\
301.01	0.01\\
302.01	0.01\\
303.01	0.01\\
304.01	0.01\\
305.01	0.01\\
306.01	0.01\\
307.01	0.01\\
308.01	0.01\\
309.01	0.01\\
310.01	0.01\\
311.01	0.01\\
312.01	0.01\\
313.01	0.01\\
314.01	0.01\\
315.01	0.01\\
316.01	0.01\\
317.01	0.01\\
318.01	0.01\\
319.01	0.01\\
320.01	0.01\\
321.01	0.01\\
322.01	0.01\\
323.01	0.01\\
324.01	0.01\\
325.01	0.01\\
326.01	0.01\\
327.01	0.01\\
328.01	0.01\\
329.01	0.01\\
330.01	0.01\\
331.01	0.01\\
332.01	0.01\\
333.01	0.01\\
334.01	0.01\\
335.01	0.01\\
336.01	0.01\\
337.01	0.01\\
338.01	0.01\\
339.01	0.01\\
340.01	0.01\\
341.01	0.01\\
342.01	0.01\\
343.01	0.01\\
344.01	0.01\\
345.01	0.01\\
346.01	0.01\\
347.01	0.01\\
348.01	0.01\\
349.01	0.01\\
350.01	0.01\\
351.01	0.01\\
352.01	0.01\\
353.01	0.01\\
354.01	0.01\\
355.01	0.01\\
356.01	0.01\\
357.01	0.01\\
358.01	0.01\\
359.01	0.01\\
360.01	0.01\\
361.01	0.01\\
362.01	0.01\\
363.01	0.01\\
364.01	0.01\\
365.01	0.01\\
366.01	0.01\\
367.01	0.01\\
368.01	0.01\\
369.01	0.01\\
370.01	0.01\\
371.01	0.01\\
372.01	0.01\\
373.01	0.01\\
374.01	0.01\\
375.01	0.01\\
376.01	0.01\\
377.01	0.01\\
378.01	0.01\\
379.01	0.01\\
380.01	0.01\\
381.01	0.01\\
382.01	0.01\\
383.01	0.01\\
384.01	0.01\\
385.01	0.01\\
386.01	0.01\\
387.01	0.01\\
388.01	0.01\\
389.01	0.01\\
390.01	0.01\\
391.01	0.01\\
392.01	0.01\\
393.01	0.01\\
394.01	0.01\\
395.01	0.01\\
396.01	0.01\\
397.01	0.01\\
398.01	0.01\\
399.01	0.01\\
400.01	0.01\\
401.01	0.01\\
402.01	0.01\\
403.01	0.01\\
404.01	0.01\\
405.01	0.01\\
406.01	0.01\\
407.01	0.01\\
408.01	0.01\\
409.01	0.01\\
410.01	0.01\\
411.01	0.01\\
412.01	0.01\\
413.01	0.01\\
414.01	0.01\\
415.01	0.01\\
416.01	0.01\\
417.01	0.01\\
418.01	0.01\\
419.01	0.01\\
420.01	0.01\\
421.01	0.01\\
422.01	0.01\\
423.01	0.01\\
424.01	0.01\\
425.01	0.01\\
426.01	0.01\\
427.01	0.01\\
428.01	0.01\\
429.01	0.01\\
430.01	0.01\\
431.01	0.01\\
432.01	0.01\\
433.01	0.01\\
434.01	0.01\\
435.01	0.01\\
436.01	0.01\\
437.01	0.01\\
438.01	0.01\\
439.01	0.01\\
440.01	0.01\\
441.01	0.01\\
442.01	0.01\\
443.01	0.01\\
444.01	0.01\\
445.01	0.01\\
446.01	0.01\\
447.01	0.01\\
448.01	0.01\\
449.01	0.01\\
450.01	0.01\\
451.01	0.01\\
452.01	0.01\\
453.01	0.01\\
454.01	0.01\\
455.01	0.01\\
456.01	0.01\\
457.01	0.01\\
458.01	0.01\\
459.01	0.01\\
460.01	0.01\\
461.01	0.01\\
462.01	0.01\\
463.01	0.01\\
464.01	0.01\\
465.01	0.01\\
466.01	0.01\\
467.01	0.01\\
468.01	0.01\\
469.01	0.01\\
470.01	0.01\\
471.01	0.01\\
472.01	0.01\\
473.01	0.01\\
474.01	0.01\\
475.01	0.01\\
476.01	0.01\\
477.01	0.01\\
478.01	0.01\\
479.01	0.01\\
480.01	0.01\\
481.01	0.01\\
482.01	0.01\\
483.01	0.01\\
484.01	0.01\\
485.01	0.01\\
486.01	0.01\\
487.01	0.01\\
488.01	0.01\\
489.01	0.01\\
490.01	0.01\\
491.01	0.01\\
492.01	0.01\\
493.01	0.01\\
494.01	0.01\\
495.01	0.01\\
496.01	0.01\\
497.01	0.01\\
498.01	0.01\\
499.01	0.01\\
500.01	0.01\\
501.01	0.01\\
502.01	0.01\\
503.01	0.01\\
504.01	0.01\\
505.01	0.01\\
506.01	0.01\\
507.01	0.01\\
508.01	0.01\\
509.01	0.01\\
510.01	0.01\\
511.01	0.01\\
512.01	0.01\\
513.01	0.01\\
514.01	0.01\\
515.01	0.01\\
516.01	0.01\\
517.01	0.01\\
518.01	0.01\\
519.01	0.01\\
520.01	0.01\\
521.01	0.01\\
522.01	0.01\\
523.01	0.01\\
524.01	0.01\\
525.01	0.01\\
526.01	0.01\\
527.01	0.01\\
528.01	0.01\\
529.01	0.01\\
530.01	0.01\\
531.01	0.01\\
532.01	0.01\\
533.01	0.01\\
534.01	0.01\\
535.01	0.01\\
536.01	0.01\\
537.01	0.01\\
538.01	0.01\\
539.01	0.01\\
540.01	0.01\\
541.01	0.01\\
542.01	0.01\\
543.01	0.01\\
544.01	0.01\\
545.01	0.01\\
546.01	0.01\\
547.01	0.01\\
548.01	0.01\\
549.01	0.01\\
550.01	0.01\\
551.01	0.01\\
552.01	0.01\\
553.01	0.01\\
554.01	0.01\\
555.01	0.01\\
556.01	0.01\\
557.01	0.01\\
558.01	0.01\\
559.01	0.01\\
560.01	0.01\\
561.01	0.01\\
562.01	0.01\\
563.01	0.01\\
564.01	0.01\\
565.01	0.01\\
566.01	0.01\\
567.01	0.01\\
568.01	0.01\\
569.01	0.01\\
570.01	0.01\\
571.01	0.01\\
572.01	0.01\\
573.01	0.01\\
574.01	0.01\\
575.01	0.01\\
576.01	0.01\\
577.01	0.01\\
578.01	0.01\\
579.01	0.01\\
580.01	0.01\\
581.01	0.01\\
582.01	0.01\\
583.01	0.01\\
584.01	0.01\\
585.01	0.01\\
586.01	0.01\\
587.01	0.01\\
588.01	0.01\\
589.01	0.01\\
590.01	0.01\\
591.01	0.01\\
592.01	0.01\\
593.01	0.01\\
594.01	0.01\\
595.01	0.01\\
596.01	0.01\\
597.01	0.01\\
598.01	0.01\\
599.01	0.01\\
599.02	0.01\\
599.03	0.01\\
599.04	0.01\\
599.05	0.01\\
599.06	0.01\\
599.07	0.01\\
599.08	0.01\\
599.09	0.01\\
599.1	0.01\\
599.11	0.01\\
599.12	0.01\\
599.13	0.01\\
599.14	0.01\\
599.15	0.01\\
599.16	0.01\\
599.17	0.01\\
599.18	0.01\\
599.19	0.01\\
599.2	0.01\\
599.21	0.01\\
599.22	0.01\\
599.23	0.01\\
599.24	0.01\\
599.25	0.01\\
599.26	0.01\\
599.27	0.01\\
599.28	0.01\\
599.29	0.01\\
599.3	0.01\\
599.31	0.01\\
599.32	0.01\\
599.33	0.01\\
599.34	0.01\\
599.35	0.01\\
599.36	0.01\\
599.37	0.01\\
599.38	0.01\\
599.39	0.01\\
599.4	0.01\\
599.41	0.01\\
599.42	0.01\\
599.43	0.01\\
599.44	0.01\\
599.45	0.01\\
599.46	0.01\\
599.47	0.01\\
599.48	0.01\\
599.49	0.01\\
599.5	0.01\\
599.51	0.01\\
599.52	0.01\\
599.53	0.01\\
599.54	0.01\\
599.55	0.01\\
599.56	0.01\\
599.57	0.01\\
599.58	0.01\\
599.59	0.01\\
599.6	0.01\\
599.61	0.01\\
599.62	0.01\\
599.63	0.01\\
599.64	0.01\\
599.65	0.01\\
599.66	0.01\\
599.67	0.01\\
599.68	0.01\\
599.69	0.01\\
599.7	0.01\\
599.71	0.01\\
599.72	0.01\\
599.73	0.01\\
599.74	0.01\\
599.75	0.01\\
599.76	0.01\\
599.77	0.01\\
599.78	0.01\\
599.79	0.01\\
599.8	0.01\\
599.81	0.01\\
599.82	0.01\\
599.83	0.01\\
599.84	0.01\\
599.85	0.01\\
599.86	0.01\\
599.87	0.01\\
599.88	0.01\\
599.89	0.01\\
599.9	0.01\\
599.91	0.01\\
599.92	0.01\\
599.93	0.01\\
599.94	0.01\\
599.95	0.01\\
599.96	0.01\\
599.97	0.01\\
599.98	0.01\\
599.99	0.01\\
600	0.01\\
};
\addplot [color=red!75!mycolor17,solid,forget plot]
  table[row sep=crcr]{%
0.01	0.01\\
1.01	0.01\\
2.01	0.01\\
3.01	0.01\\
4.01	0.01\\
5.01	0.01\\
6.01	0.01\\
7.01	0.01\\
8.01	0.01\\
9.01	0.01\\
10.01	0.01\\
11.01	0.01\\
12.01	0.01\\
13.01	0.01\\
14.01	0.01\\
15.01	0.01\\
16.01	0.01\\
17.01	0.01\\
18.01	0.01\\
19.01	0.01\\
20.01	0.01\\
21.01	0.01\\
22.01	0.01\\
23.01	0.01\\
24.01	0.01\\
25.01	0.01\\
26.01	0.01\\
27.01	0.01\\
28.01	0.01\\
29.01	0.01\\
30.01	0.01\\
31.01	0.01\\
32.01	0.01\\
33.01	0.01\\
34.01	0.01\\
35.01	0.01\\
36.01	0.01\\
37.01	0.01\\
38.01	0.01\\
39.01	0.01\\
40.01	0.01\\
41.01	0.01\\
42.01	0.01\\
43.01	0.01\\
44.01	0.01\\
45.01	0.01\\
46.01	0.01\\
47.01	0.01\\
48.01	0.01\\
49.01	0.01\\
50.01	0.01\\
51.01	0.01\\
52.01	0.01\\
53.01	0.01\\
54.01	0.01\\
55.01	0.01\\
56.01	0.01\\
57.01	0.01\\
58.01	0.01\\
59.01	0.01\\
60.01	0.01\\
61.01	0.01\\
62.01	0.01\\
63.01	0.01\\
64.01	0.01\\
65.01	0.01\\
66.01	0.01\\
67.01	0.01\\
68.01	0.01\\
69.01	0.01\\
70.01	0.01\\
71.01	0.01\\
72.01	0.01\\
73.01	0.01\\
74.01	0.01\\
75.01	0.01\\
76.01	0.01\\
77.01	0.01\\
78.01	0.01\\
79.01	0.01\\
80.01	0.01\\
81.01	0.01\\
82.01	0.01\\
83.01	0.01\\
84.01	0.01\\
85.01	0.01\\
86.01	0.01\\
87.01	0.01\\
88.01	0.01\\
89.01	0.01\\
90.01	0.01\\
91.01	0.01\\
92.01	0.01\\
93.01	0.01\\
94.01	0.01\\
95.01	0.01\\
96.01	0.01\\
97.01	0.01\\
98.01	0.01\\
99.01	0.01\\
100.01	0.01\\
101.01	0.01\\
102.01	0.01\\
103.01	0.01\\
104.01	0.01\\
105.01	0.01\\
106.01	0.01\\
107.01	0.01\\
108.01	0.01\\
109.01	0.01\\
110.01	0.01\\
111.01	0.01\\
112.01	0.01\\
113.01	0.01\\
114.01	0.01\\
115.01	0.01\\
116.01	0.01\\
117.01	0.01\\
118.01	0.01\\
119.01	0.01\\
120.01	0.01\\
121.01	0.01\\
122.01	0.01\\
123.01	0.01\\
124.01	0.01\\
125.01	0.01\\
126.01	0.01\\
127.01	0.01\\
128.01	0.01\\
129.01	0.01\\
130.01	0.01\\
131.01	0.01\\
132.01	0.01\\
133.01	0.01\\
134.01	0.01\\
135.01	0.01\\
136.01	0.01\\
137.01	0.01\\
138.01	0.01\\
139.01	0.01\\
140.01	0.01\\
141.01	0.01\\
142.01	0.01\\
143.01	0.01\\
144.01	0.01\\
145.01	0.01\\
146.01	0.01\\
147.01	0.01\\
148.01	0.01\\
149.01	0.01\\
150.01	0.01\\
151.01	0.01\\
152.01	0.01\\
153.01	0.01\\
154.01	0.01\\
155.01	0.01\\
156.01	0.01\\
157.01	0.01\\
158.01	0.01\\
159.01	0.01\\
160.01	0.01\\
161.01	0.01\\
162.01	0.01\\
163.01	0.01\\
164.01	0.01\\
165.01	0.01\\
166.01	0.01\\
167.01	0.01\\
168.01	0.01\\
169.01	0.01\\
170.01	0.01\\
171.01	0.01\\
172.01	0.01\\
173.01	0.01\\
174.01	0.01\\
175.01	0.01\\
176.01	0.01\\
177.01	0.01\\
178.01	0.01\\
179.01	0.01\\
180.01	0.01\\
181.01	0.01\\
182.01	0.01\\
183.01	0.01\\
184.01	0.01\\
185.01	0.01\\
186.01	0.01\\
187.01	0.01\\
188.01	0.01\\
189.01	0.01\\
190.01	0.01\\
191.01	0.01\\
192.01	0.01\\
193.01	0.01\\
194.01	0.01\\
195.01	0.01\\
196.01	0.01\\
197.01	0.01\\
198.01	0.01\\
199.01	0.01\\
200.01	0.01\\
201.01	0.01\\
202.01	0.01\\
203.01	0.01\\
204.01	0.01\\
205.01	0.01\\
206.01	0.01\\
207.01	0.01\\
208.01	0.01\\
209.01	0.01\\
210.01	0.01\\
211.01	0.01\\
212.01	0.01\\
213.01	0.01\\
214.01	0.01\\
215.01	0.01\\
216.01	0.01\\
217.01	0.01\\
218.01	0.01\\
219.01	0.01\\
220.01	0.01\\
221.01	0.01\\
222.01	0.01\\
223.01	0.01\\
224.01	0.01\\
225.01	0.01\\
226.01	0.01\\
227.01	0.01\\
228.01	0.01\\
229.01	0.01\\
230.01	0.01\\
231.01	0.01\\
232.01	0.01\\
233.01	0.01\\
234.01	0.01\\
235.01	0.01\\
236.01	0.01\\
237.01	0.01\\
238.01	0.01\\
239.01	0.01\\
240.01	0.01\\
241.01	0.01\\
242.01	0.01\\
243.01	0.01\\
244.01	0.01\\
245.01	0.01\\
246.01	0.01\\
247.01	0.01\\
248.01	0.01\\
249.01	0.01\\
250.01	0.01\\
251.01	0.01\\
252.01	0.01\\
253.01	0.01\\
254.01	0.01\\
255.01	0.01\\
256.01	0.01\\
257.01	0.01\\
258.01	0.01\\
259.01	0.01\\
260.01	0.01\\
261.01	0.01\\
262.01	0.01\\
263.01	0.01\\
264.01	0.01\\
265.01	0.01\\
266.01	0.01\\
267.01	0.01\\
268.01	0.01\\
269.01	0.01\\
270.01	0.01\\
271.01	0.01\\
272.01	0.01\\
273.01	0.01\\
274.01	0.01\\
275.01	0.01\\
276.01	0.01\\
277.01	0.01\\
278.01	0.01\\
279.01	0.01\\
280.01	0.01\\
281.01	0.01\\
282.01	0.01\\
283.01	0.01\\
284.01	0.01\\
285.01	0.01\\
286.01	0.01\\
287.01	0.01\\
288.01	0.01\\
289.01	0.01\\
290.01	0.01\\
291.01	0.01\\
292.01	0.01\\
293.01	0.01\\
294.01	0.01\\
295.01	0.01\\
296.01	0.01\\
297.01	0.01\\
298.01	0.01\\
299.01	0.01\\
300.01	0.01\\
301.01	0.01\\
302.01	0.01\\
303.01	0.01\\
304.01	0.01\\
305.01	0.01\\
306.01	0.01\\
307.01	0.01\\
308.01	0.01\\
309.01	0.01\\
310.01	0.01\\
311.01	0.01\\
312.01	0.01\\
313.01	0.01\\
314.01	0.01\\
315.01	0.01\\
316.01	0.01\\
317.01	0.01\\
318.01	0.01\\
319.01	0.01\\
320.01	0.01\\
321.01	0.01\\
322.01	0.01\\
323.01	0.01\\
324.01	0.01\\
325.01	0.01\\
326.01	0.01\\
327.01	0.01\\
328.01	0.01\\
329.01	0.01\\
330.01	0.01\\
331.01	0.01\\
332.01	0.01\\
333.01	0.01\\
334.01	0.01\\
335.01	0.01\\
336.01	0.01\\
337.01	0.01\\
338.01	0.01\\
339.01	0.01\\
340.01	0.01\\
341.01	0.01\\
342.01	0.01\\
343.01	0.01\\
344.01	0.01\\
345.01	0.01\\
346.01	0.01\\
347.01	0.01\\
348.01	0.01\\
349.01	0.01\\
350.01	0.01\\
351.01	0.01\\
352.01	0.01\\
353.01	0.01\\
354.01	0.01\\
355.01	0.01\\
356.01	0.01\\
357.01	0.01\\
358.01	0.01\\
359.01	0.01\\
360.01	0.01\\
361.01	0.01\\
362.01	0.01\\
363.01	0.01\\
364.01	0.01\\
365.01	0.01\\
366.01	0.01\\
367.01	0.01\\
368.01	0.01\\
369.01	0.01\\
370.01	0.01\\
371.01	0.01\\
372.01	0.01\\
373.01	0.01\\
374.01	0.01\\
375.01	0.01\\
376.01	0.01\\
377.01	0.01\\
378.01	0.01\\
379.01	0.01\\
380.01	0.01\\
381.01	0.01\\
382.01	0.01\\
383.01	0.01\\
384.01	0.01\\
385.01	0.01\\
386.01	0.01\\
387.01	0.01\\
388.01	0.01\\
389.01	0.01\\
390.01	0.01\\
391.01	0.01\\
392.01	0.01\\
393.01	0.01\\
394.01	0.01\\
395.01	0.01\\
396.01	0.01\\
397.01	0.01\\
398.01	0.01\\
399.01	0.01\\
400.01	0.01\\
401.01	0.01\\
402.01	0.01\\
403.01	0.01\\
404.01	0.01\\
405.01	0.01\\
406.01	0.01\\
407.01	0.01\\
408.01	0.01\\
409.01	0.01\\
410.01	0.01\\
411.01	0.01\\
412.01	0.01\\
413.01	0.01\\
414.01	0.01\\
415.01	0.01\\
416.01	0.01\\
417.01	0.01\\
418.01	0.01\\
419.01	0.01\\
420.01	0.01\\
421.01	0.01\\
422.01	0.01\\
423.01	0.01\\
424.01	0.01\\
425.01	0.01\\
426.01	0.01\\
427.01	0.01\\
428.01	0.01\\
429.01	0.01\\
430.01	0.01\\
431.01	0.01\\
432.01	0.01\\
433.01	0.01\\
434.01	0.01\\
435.01	0.01\\
436.01	0.01\\
437.01	0.01\\
438.01	0.01\\
439.01	0.01\\
440.01	0.01\\
441.01	0.01\\
442.01	0.01\\
443.01	0.01\\
444.01	0.01\\
445.01	0.01\\
446.01	0.01\\
447.01	0.01\\
448.01	0.01\\
449.01	0.01\\
450.01	0.01\\
451.01	0.01\\
452.01	0.01\\
453.01	0.01\\
454.01	0.01\\
455.01	0.01\\
456.01	0.01\\
457.01	0.01\\
458.01	0.01\\
459.01	0.01\\
460.01	0.01\\
461.01	0.01\\
462.01	0.01\\
463.01	0.01\\
464.01	0.01\\
465.01	0.01\\
466.01	0.01\\
467.01	0.01\\
468.01	0.01\\
469.01	0.01\\
470.01	0.01\\
471.01	0.01\\
472.01	0.01\\
473.01	0.01\\
474.01	0.01\\
475.01	0.01\\
476.01	0.01\\
477.01	0.01\\
478.01	0.01\\
479.01	0.01\\
480.01	0.01\\
481.01	0.01\\
482.01	0.01\\
483.01	0.01\\
484.01	0.01\\
485.01	0.01\\
486.01	0.01\\
487.01	0.01\\
488.01	0.01\\
489.01	0.01\\
490.01	0.01\\
491.01	0.01\\
492.01	0.01\\
493.01	0.01\\
494.01	0.01\\
495.01	0.01\\
496.01	0.01\\
497.01	0.01\\
498.01	0.01\\
499.01	0.01\\
500.01	0.01\\
501.01	0.01\\
502.01	0.01\\
503.01	0.01\\
504.01	0.01\\
505.01	0.01\\
506.01	0.01\\
507.01	0.01\\
508.01	0.01\\
509.01	0.01\\
510.01	0.01\\
511.01	0.01\\
512.01	0.01\\
513.01	0.01\\
514.01	0.01\\
515.01	0.01\\
516.01	0.01\\
517.01	0.01\\
518.01	0.01\\
519.01	0.01\\
520.01	0.01\\
521.01	0.01\\
522.01	0.01\\
523.01	0.01\\
524.01	0.01\\
525.01	0.01\\
526.01	0.01\\
527.01	0.01\\
528.01	0.01\\
529.01	0.01\\
530.01	0.01\\
531.01	0.01\\
532.01	0.01\\
533.01	0.01\\
534.01	0.01\\
535.01	0.01\\
536.01	0.01\\
537.01	0.01\\
538.01	0.01\\
539.01	0.01\\
540.01	0.01\\
541.01	0.01\\
542.01	0.01\\
543.01	0.01\\
544.01	0.01\\
545.01	0.01\\
546.01	0.01\\
547.01	0.01\\
548.01	0.01\\
549.01	0.01\\
550.01	0.01\\
551.01	0.01\\
552.01	0.01\\
553.01	0.01\\
554.01	0.01\\
555.01	0.01\\
556.01	0.01\\
557.01	0.01\\
558.01	0.01\\
559.01	0.01\\
560.01	0.01\\
561.01	0.01\\
562.01	0.01\\
563.01	0.01\\
564.01	0.01\\
565.01	0.01\\
566.01	0.01\\
567.01	0.01\\
568.01	0.01\\
569.01	0.01\\
570.01	0.01\\
571.01	0.01\\
572.01	0.01\\
573.01	0.01\\
574.01	0.01\\
575.01	0.01\\
576.01	0.01\\
577.01	0.01\\
578.01	0.01\\
579.01	0.01\\
580.01	0.01\\
581.01	0.01\\
582.01	0.01\\
583.01	0.01\\
584.01	0.01\\
585.01	0.01\\
586.01	0.01\\
587.01	0.01\\
588.01	0.01\\
589.01	0.01\\
590.01	0.01\\
591.01	0.01\\
592.01	0.01\\
593.01	0.01\\
594.01	0.01\\
595.01	0.01\\
596.01	0.01\\
597.01	0.01\\
598.01	0.01\\
599.01	0.01\\
599.02	0.01\\
599.03	0.01\\
599.04	0.01\\
599.05	0.01\\
599.06	0.01\\
599.07	0.01\\
599.08	0.01\\
599.09	0.01\\
599.1	0.01\\
599.11	0.01\\
599.12	0.01\\
599.13	0.01\\
599.14	0.01\\
599.15	0.01\\
599.16	0.01\\
599.17	0.01\\
599.18	0.01\\
599.19	0.01\\
599.2	0.01\\
599.21	0.01\\
599.22	0.01\\
599.23	0.01\\
599.24	0.01\\
599.25	0.01\\
599.26	0.01\\
599.27	0.01\\
599.28	0.01\\
599.29	0.01\\
599.3	0.01\\
599.31	0.01\\
599.32	0.01\\
599.33	0.01\\
599.34	0.01\\
599.35	0.01\\
599.36	0.01\\
599.37	0.01\\
599.38	0.01\\
599.39	0.01\\
599.4	0.01\\
599.41	0.01\\
599.42	0.01\\
599.43	0.01\\
599.44	0.01\\
599.45	0.01\\
599.46	0.01\\
599.47	0.01\\
599.48	0.01\\
599.49	0.01\\
599.5	0.01\\
599.51	0.01\\
599.52	0.01\\
599.53	0.01\\
599.54	0.01\\
599.55	0.01\\
599.56	0.01\\
599.57	0.01\\
599.58	0.01\\
599.59	0.01\\
599.6	0.01\\
599.61	0.01\\
599.62	0.01\\
599.63	0.01\\
599.64	0.01\\
599.65	0.01\\
599.66	0.01\\
599.67	0.01\\
599.68	0.01\\
599.69	0.01\\
599.7	0.01\\
599.71	0.01\\
599.72	0.01\\
599.73	0.01\\
599.74	0.01\\
599.75	0.01\\
599.76	0.01\\
599.77	0.01\\
599.78	0.01\\
599.79	0.01\\
599.8	0.01\\
599.81	0.01\\
599.82	0.01\\
599.83	0.01\\
599.84	0.01\\
599.85	0.01\\
599.86	0.01\\
599.87	0.01\\
599.88	0.01\\
599.89	0.01\\
599.9	0.01\\
599.91	0.01\\
599.92	0.01\\
599.93	0.01\\
599.94	0.01\\
599.95	0.01\\
599.96	0.01\\
599.97	0.01\\
599.98	0.01\\
599.99	0.01\\
600	0.01\\
};
\addplot [color=red!80!mycolor19,solid,forget plot]
  table[row sep=crcr]{%
0.01	0.01\\
1.01	0.01\\
2.01	0.01\\
3.01	0.01\\
4.01	0.01\\
5.01	0.01\\
6.01	0.01\\
7.01	0.01\\
8.01	0.01\\
9.01	0.01\\
10.01	0.01\\
11.01	0.01\\
12.01	0.01\\
13.01	0.01\\
14.01	0.01\\
15.01	0.01\\
16.01	0.01\\
17.01	0.01\\
18.01	0.01\\
19.01	0.01\\
20.01	0.01\\
21.01	0.01\\
22.01	0.01\\
23.01	0.01\\
24.01	0.01\\
25.01	0.01\\
26.01	0.01\\
27.01	0.01\\
28.01	0.01\\
29.01	0.01\\
30.01	0.01\\
31.01	0.01\\
32.01	0.01\\
33.01	0.01\\
34.01	0.01\\
35.01	0.01\\
36.01	0.01\\
37.01	0.01\\
38.01	0.01\\
39.01	0.01\\
40.01	0.01\\
41.01	0.01\\
42.01	0.01\\
43.01	0.01\\
44.01	0.01\\
45.01	0.01\\
46.01	0.01\\
47.01	0.01\\
48.01	0.01\\
49.01	0.01\\
50.01	0.01\\
51.01	0.01\\
52.01	0.01\\
53.01	0.01\\
54.01	0.01\\
55.01	0.01\\
56.01	0.01\\
57.01	0.01\\
58.01	0.01\\
59.01	0.01\\
60.01	0.01\\
61.01	0.01\\
62.01	0.01\\
63.01	0.01\\
64.01	0.01\\
65.01	0.01\\
66.01	0.01\\
67.01	0.01\\
68.01	0.01\\
69.01	0.01\\
70.01	0.01\\
71.01	0.01\\
72.01	0.01\\
73.01	0.01\\
74.01	0.01\\
75.01	0.01\\
76.01	0.01\\
77.01	0.01\\
78.01	0.01\\
79.01	0.01\\
80.01	0.01\\
81.01	0.01\\
82.01	0.01\\
83.01	0.01\\
84.01	0.01\\
85.01	0.01\\
86.01	0.01\\
87.01	0.01\\
88.01	0.01\\
89.01	0.01\\
90.01	0.01\\
91.01	0.01\\
92.01	0.01\\
93.01	0.01\\
94.01	0.01\\
95.01	0.01\\
96.01	0.01\\
97.01	0.01\\
98.01	0.01\\
99.01	0.01\\
100.01	0.01\\
101.01	0.01\\
102.01	0.01\\
103.01	0.01\\
104.01	0.01\\
105.01	0.01\\
106.01	0.01\\
107.01	0.01\\
108.01	0.01\\
109.01	0.01\\
110.01	0.01\\
111.01	0.01\\
112.01	0.01\\
113.01	0.01\\
114.01	0.01\\
115.01	0.01\\
116.01	0.01\\
117.01	0.01\\
118.01	0.01\\
119.01	0.01\\
120.01	0.01\\
121.01	0.01\\
122.01	0.01\\
123.01	0.01\\
124.01	0.01\\
125.01	0.01\\
126.01	0.01\\
127.01	0.01\\
128.01	0.01\\
129.01	0.01\\
130.01	0.01\\
131.01	0.01\\
132.01	0.01\\
133.01	0.01\\
134.01	0.01\\
135.01	0.01\\
136.01	0.01\\
137.01	0.01\\
138.01	0.01\\
139.01	0.01\\
140.01	0.01\\
141.01	0.01\\
142.01	0.01\\
143.01	0.01\\
144.01	0.01\\
145.01	0.01\\
146.01	0.01\\
147.01	0.01\\
148.01	0.01\\
149.01	0.01\\
150.01	0.01\\
151.01	0.01\\
152.01	0.01\\
153.01	0.01\\
154.01	0.01\\
155.01	0.01\\
156.01	0.01\\
157.01	0.01\\
158.01	0.01\\
159.01	0.01\\
160.01	0.01\\
161.01	0.01\\
162.01	0.01\\
163.01	0.01\\
164.01	0.01\\
165.01	0.01\\
166.01	0.01\\
167.01	0.01\\
168.01	0.01\\
169.01	0.01\\
170.01	0.01\\
171.01	0.01\\
172.01	0.01\\
173.01	0.01\\
174.01	0.01\\
175.01	0.01\\
176.01	0.01\\
177.01	0.01\\
178.01	0.01\\
179.01	0.01\\
180.01	0.01\\
181.01	0.01\\
182.01	0.01\\
183.01	0.01\\
184.01	0.01\\
185.01	0.01\\
186.01	0.01\\
187.01	0.01\\
188.01	0.01\\
189.01	0.01\\
190.01	0.01\\
191.01	0.01\\
192.01	0.01\\
193.01	0.01\\
194.01	0.01\\
195.01	0.01\\
196.01	0.01\\
197.01	0.01\\
198.01	0.01\\
199.01	0.01\\
200.01	0.01\\
201.01	0.01\\
202.01	0.01\\
203.01	0.01\\
204.01	0.01\\
205.01	0.01\\
206.01	0.01\\
207.01	0.01\\
208.01	0.01\\
209.01	0.01\\
210.01	0.01\\
211.01	0.01\\
212.01	0.01\\
213.01	0.01\\
214.01	0.01\\
215.01	0.01\\
216.01	0.01\\
217.01	0.01\\
218.01	0.01\\
219.01	0.01\\
220.01	0.01\\
221.01	0.01\\
222.01	0.01\\
223.01	0.01\\
224.01	0.01\\
225.01	0.01\\
226.01	0.01\\
227.01	0.01\\
228.01	0.01\\
229.01	0.01\\
230.01	0.01\\
231.01	0.01\\
232.01	0.01\\
233.01	0.01\\
234.01	0.01\\
235.01	0.01\\
236.01	0.01\\
237.01	0.01\\
238.01	0.01\\
239.01	0.01\\
240.01	0.01\\
241.01	0.01\\
242.01	0.01\\
243.01	0.01\\
244.01	0.01\\
245.01	0.01\\
246.01	0.01\\
247.01	0.01\\
248.01	0.01\\
249.01	0.01\\
250.01	0.01\\
251.01	0.01\\
252.01	0.01\\
253.01	0.01\\
254.01	0.01\\
255.01	0.01\\
256.01	0.01\\
257.01	0.01\\
258.01	0.01\\
259.01	0.01\\
260.01	0.01\\
261.01	0.01\\
262.01	0.01\\
263.01	0.01\\
264.01	0.01\\
265.01	0.01\\
266.01	0.01\\
267.01	0.01\\
268.01	0.01\\
269.01	0.01\\
270.01	0.01\\
271.01	0.01\\
272.01	0.01\\
273.01	0.01\\
274.01	0.01\\
275.01	0.01\\
276.01	0.01\\
277.01	0.01\\
278.01	0.01\\
279.01	0.01\\
280.01	0.01\\
281.01	0.01\\
282.01	0.01\\
283.01	0.01\\
284.01	0.01\\
285.01	0.01\\
286.01	0.01\\
287.01	0.01\\
288.01	0.01\\
289.01	0.01\\
290.01	0.01\\
291.01	0.01\\
292.01	0.01\\
293.01	0.01\\
294.01	0.01\\
295.01	0.01\\
296.01	0.01\\
297.01	0.01\\
298.01	0.01\\
299.01	0.01\\
300.01	0.01\\
301.01	0.01\\
302.01	0.01\\
303.01	0.01\\
304.01	0.01\\
305.01	0.01\\
306.01	0.01\\
307.01	0.01\\
308.01	0.01\\
309.01	0.01\\
310.01	0.01\\
311.01	0.01\\
312.01	0.01\\
313.01	0.01\\
314.01	0.01\\
315.01	0.01\\
316.01	0.01\\
317.01	0.01\\
318.01	0.01\\
319.01	0.01\\
320.01	0.01\\
321.01	0.01\\
322.01	0.01\\
323.01	0.01\\
324.01	0.01\\
325.01	0.01\\
326.01	0.01\\
327.01	0.01\\
328.01	0.01\\
329.01	0.01\\
330.01	0.01\\
331.01	0.01\\
332.01	0.01\\
333.01	0.01\\
334.01	0.01\\
335.01	0.01\\
336.01	0.01\\
337.01	0.01\\
338.01	0.01\\
339.01	0.01\\
340.01	0.01\\
341.01	0.01\\
342.01	0.01\\
343.01	0.01\\
344.01	0.01\\
345.01	0.01\\
346.01	0.01\\
347.01	0.01\\
348.01	0.01\\
349.01	0.01\\
350.01	0.01\\
351.01	0.01\\
352.01	0.01\\
353.01	0.01\\
354.01	0.01\\
355.01	0.01\\
356.01	0.01\\
357.01	0.01\\
358.01	0.01\\
359.01	0.01\\
360.01	0.01\\
361.01	0.01\\
362.01	0.01\\
363.01	0.01\\
364.01	0.01\\
365.01	0.01\\
366.01	0.01\\
367.01	0.01\\
368.01	0.01\\
369.01	0.01\\
370.01	0.01\\
371.01	0.01\\
372.01	0.01\\
373.01	0.01\\
374.01	0.01\\
375.01	0.01\\
376.01	0.01\\
377.01	0.01\\
378.01	0.01\\
379.01	0.01\\
380.01	0.01\\
381.01	0.01\\
382.01	0.01\\
383.01	0.01\\
384.01	0.01\\
385.01	0.01\\
386.01	0.01\\
387.01	0.01\\
388.01	0.01\\
389.01	0.01\\
390.01	0.01\\
391.01	0.01\\
392.01	0.01\\
393.01	0.01\\
394.01	0.01\\
395.01	0.01\\
396.01	0.01\\
397.01	0.01\\
398.01	0.01\\
399.01	0.01\\
400.01	0.01\\
401.01	0.01\\
402.01	0.01\\
403.01	0.01\\
404.01	0.01\\
405.01	0.01\\
406.01	0.01\\
407.01	0.01\\
408.01	0.01\\
409.01	0.01\\
410.01	0.01\\
411.01	0.01\\
412.01	0.01\\
413.01	0.01\\
414.01	0.01\\
415.01	0.01\\
416.01	0.01\\
417.01	0.01\\
418.01	0.01\\
419.01	0.01\\
420.01	0.01\\
421.01	0.01\\
422.01	0.01\\
423.01	0.01\\
424.01	0.01\\
425.01	0.01\\
426.01	0.01\\
427.01	0.01\\
428.01	0.01\\
429.01	0.01\\
430.01	0.01\\
431.01	0.01\\
432.01	0.01\\
433.01	0.01\\
434.01	0.01\\
435.01	0.01\\
436.01	0.01\\
437.01	0.01\\
438.01	0.01\\
439.01	0.01\\
440.01	0.01\\
441.01	0.01\\
442.01	0.01\\
443.01	0.01\\
444.01	0.01\\
445.01	0.01\\
446.01	0.01\\
447.01	0.01\\
448.01	0.01\\
449.01	0.01\\
450.01	0.01\\
451.01	0.01\\
452.01	0.01\\
453.01	0.01\\
454.01	0.01\\
455.01	0.01\\
456.01	0.01\\
457.01	0.01\\
458.01	0.01\\
459.01	0.01\\
460.01	0.01\\
461.01	0.01\\
462.01	0.01\\
463.01	0.01\\
464.01	0.01\\
465.01	0.01\\
466.01	0.01\\
467.01	0.01\\
468.01	0.01\\
469.01	0.01\\
470.01	0.01\\
471.01	0.01\\
472.01	0.01\\
473.01	0.01\\
474.01	0.01\\
475.01	0.01\\
476.01	0.01\\
477.01	0.01\\
478.01	0.01\\
479.01	0.01\\
480.01	0.01\\
481.01	0.01\\
482.01	0.01\\
483.01	0.01\\
484.01	0.01\\
485.01	0.01\\
486.01	0.01\\
487.01	0.01\\
488.01	0.01\\
489.01	0.01\\
490.01	0.01\\
491.01	0.01\\
492.01	0.01\\
493.01	0.01\\
494.01	0.01\\
495.01	0.01\\
496.01	0.01\\
497.01	0.01\\
498.01	0.01\\
499.01	0.01\\
500.01	0.01\\
501.01	0.01\\
502.01	0.01\\
503.01	0.01\\
504.01	0.01\\
505.01	0.01\\
506.01	0.01\\
507.01	0.01\\
508.01	0.01\\
509.01	0.01\\
510.01	0.01\\
511.01	0.01\\
512.01	0.01\\
513.01	0.01\\
514.01	0.01\\
515.01	0.01\\
516.01	0.01\\
517.01	0.01\\
518.01	0.01\\
519.01	0.01\\
520.01	0.01\\
521.01	0.01\\
522.01	0.01\\
523.01	0.01\\
524.01	0.01\\
525.01	0.01\\
526.01	0.01\\
527.01	0.01\\
528.01	0.01\\
529.01	0.01\\
530.01	0.01\\
531.01	0.01\\
532.01	0.01\\
533.01	0.01\\
534.01	0.01\\
535.01	0.01\\
536.01	0.01\\
537.01	0.01\\
538.01	0.01\\
539.01	0.01\\
540.01	0.01\\
541.01	0.01\\
542.01	0.01\\
543.01	0.01\\
544.01	0.01\\
545.01	0.01\\
546.01	0.01\\
547.01	0.01\\
548.01	0.01\\
549.01	0.01\\
550.01	0.01\\
551.01	0.01\\
552.01	0.01\\
553.01	0.01\\
554.01	0.01\\
555.01	0.01\\
556.01	0.01\\
557.01	0.01\\
558.01	0.01\\
559.01	0.01\\
560.01	0.01\\
561.01	0.01\\
562.01	0.01\\
563.01	0.01\\
564.01	0.01\\
565.01	0.01\\
566.01	0.01\\
567.01	0.01\\
568.01	0.01\\
569.01	0.01\\
570.01	0.01\\
571.01	0.01\\
572.01	0.01\\
573.01	0.01\\
574.01	0.01\\
575.01	0.01\\
576.01	0.01\\
577.01	0.01\\
578.01	0.01\\
579.01	0.01\\
580.01	0.01\\
581.01	0.01\\
582.01	0.01\\
583.01	0.01\\
584.01	0.01\\
585.01	0.01\\
586.01	0.01\\
587.01	0.01\\
588.01	0.01\\
589.01	0.01\\
590.01	0.01\\
591.01	0.01\\
592.01	0.01\\
593.01	0.01\\
594.01	0.01\\
595.01	0.01\\
596.01	0.01\\
597.01	0.01\\
598.01	0.01\\
599.01	0.01\\
599.02	0.01\\
599.03	0.01\\
599.04	0.01\\
599.05	0.01\\
599.06	0.01\\
599.07	0.01\\
599.08	0.01\\
599.09	0.01\\
599.1	0.01\\
599.11	0.01\\
599.12	0.01\\
599.13	0.01\\
599.14	0.01\\
599.15	0.01\\
599.16	0.01\\
599.17	0.01\\
599.18	0.01\\
599.19	0.01\\
599.2	0.01\\
599.21	0.01\\
599.22	0.01\\
599.23	0.01\\
599.24	0.01\\
599.25	0.01\\
599.26	0.01\\
599.27	0.01\\
599.28	0.01\\
599.29	0.01\\
599.3	0.01\\
599.31	0.01\\
599.32	0.01\\
599.33	0.01\\
599.34	0.01\\
599.35	0.01\\
599.36	0.01\\
599.37	0.01\\
599.38	0.01\\
599.39	0.01\\
599.4	0.01\\
599.41	0.01\\
599.42	0.01\\
599.43	0.01\\
599.44	0.01\\
599.45	0.01\\
599.46	0.01\\
599.47	0.01\\
599.48	0.01\\
599.49	0.01\\
599.5	0.01\\
599.51	0.01\\
599.52	0.01\\
599.53	0.01\\
599.54	0.01\\
599.55	0.01\\
599.56	0.01\\
599.57	0.01\\
599.58	0.01\\
599.59	0.01\\
599.6	0.01\\
599.61	0.01\\
599.62	0.01\\
599.63	0.01\\
599.64	0.01\\
599.65	0.01\\
599.66	0.01\\
599.67	0.01\\
599.68	0.01\\
599.69	0.01\\
599.7	0.01\\
599.71	0.01\\
599.72	0.01\\
599.73	0.01\\
599.74	0.01\\
599.75	0.01\\
599.76	0.01\\
599.77	0.01\\
599.78	0.01\\
599.79	0.01\\
599.8	0.01\\
599.81	0.01\\
599.82	0.01\\
599.83	0.01\\
599.84	0.01\\
599.85	0.01\\
599.86	0.01\\
599.87	0.01\\
599.88	0.01\\
599.89	0.01\\
599.9	0.01\\
599.91	0.01\\
599.92	0.01\\
599.93	0.01\\
599.94	0.01\\
599.95	0.01\\
599.96	0.01\\
599.97	0.01\\
599.98	0.01\\
599.99	0.01\\
600	0.01\\
};
\addplot [color=red,solid,forget plot]
  table[row sep=crcr]{%
0.01	0.01\\
1.01	0.01\\
2.01	0.01\\
3.01	0.01\\
4.01	0.01\\
5.01	0.01\\
6.01	0.01\\
7.01	0.01\\
8.01	0.01\\
9.01	0.01\\
10.01	0.01\\
11.01	0.01\\
12.01	0.01\\
13.01	0.01\\
14.01	0.01\\
15.01	0.01\\
16.01	0.01\\
17.01	0.01\\
18.01	0.01\\
19.01	0.01\\
20.01	0.01\\
21.01	0.01\\
22.01	0.01\\
23.01	0.01\\
24.01	0.01\\
25.01	0.01\\
26.01	0.01\\
27.01	0.01\\
28.01	0.01\\
29.01	0.01\\
30.01	0.01\\
31.01	0.01\\
32.01	0.01\\
33.01	0.01\\
34.01	0.01\\
35.01	0.01\\
36.01	0.01\\
37.01	0.01\\
38.01	0.01\\
39.01	0.01\\
40.01	0.01\\
41.01	0.01\\
42.01	0.01\\
43.01	0.01\\
44.01	0.01\\
45.01	0.01\\
46.01	0.01\\
47.01	0.01\\
48.01	0.01\\
49.01	0.01\\
50.01	0.01\\
51.01	0.01\\
52.01	0.01\\
53.01	0.01\\
54.01	0.01\\
55.01	0.01\\
56.01	0.01\\
57.01	0.01\\
58.01	0.01\\
59.01	0.01\\
60.01	0.01\\
61.01	0.01\\
62.01	0.01\\
63.01	0.01\\
64.01	0.01\\
65.01	0.01\\
66.01	0.01\\
67.01	0.01\\
68.01	0.01\\
69.01	0.01\\
70.01	0.01\\
71.01	0.01\\
72.01	0.01\\
73.01	0.01\\
74.01	0.01\\
75.01	0.01\\
76.01	0.01\\
77.01	0.01\\
78.01	0.01\\
79.01	0.01\\
80.01	0.01\\
81.01	0.01\\
82.01	0.01\\
83.01	0.01\\
84.01	0.01\\
85.01	0.01\\
86.01	0.01\\
87.01	0.01\\
88.01	0.01\\
89.01	0.01\\
90.01	0.01\\
91.01	0.01\\
92.01	0.01\\
93.01	0.01\\
94.01	0.01\\
95.01	0.01\\
96.01	0.01\\
97.01	0.01\\
98.01	0.01\\
99.01	0.01\\
100.01	0.01\\
101.01	0.01\\
102.01	0.01\\
103.01	0.01\\
104.01	0.01\\
105.01	0.01\\
106.01	0.01\\
107.01	0.01\\
108.01	0.01\\
109.01	0.01\\
110.01	0.01\\
111.01	0.01\\
112.01	0.01\\
113.01	0.01\\
114.01	0.01\\
115.01	0.01\\
116.01	0.01\\
117.01	0.01\\
118.01	0.01\\
119.01	0.01\\
120.01	0.01\\
121.01	0.01\\
122.01	0.01\\
123.01	0.01\\
124.01	0.01\\
125.01	0.01\\
126.01	0.01\\
127.01	0.01\\
128.01	0.01\\
129.01	0.01\\
130.01	0.01\\
131.01	0.01\\
132.01	0.01\\
133.01	0.01\\
134.01	0.01\\
135.01	0.01\\
136.01	0.01\\
137.01	0.01\\
138.01	0.01\\
139.01	0.01\\
140.01	0.01\\
141.01	0.01\\
142.01	0.01\\
143.01	0.01\\
144.01	0.01\\
145.01	0.01\\
146.01	0.01\\
147.01	0.01\\
148.01	0.01\\
149.01	0.01\\
150.01	0.01\\
151.01	0.01\\
152.01	0.01\\
153.01	0.01\\
154.01	0.01\\
155.01	0.01\\
156.01	0.01\\
157.01	0.01\\
158.01	0.01\\
159.01	0.01\\
160.01	0.01\\
161.01	0.01\\
162.01	0.01\\
163.01	0.01\\
164.01	0.01\\
165.01	0.01\\
166.01	0.01\\
167.01	0.01\\
168.01	0.01\\
169.01	0.01\\
170.01	0.01\\
171.01	0.01\\
172.01	0.01\\
173.01	0.01\\
174.01	0.01\\
175.01	0.01\\
176.01	0.01\\
177.01	0.01\\
178.01	0.01\\
179.01	0.01\\
180.01	0.01\\
181.01	0.01\\
182.01	0.01\\
183.01	0.01\\
184.01	0.01\\
185.01	0.01\\
186.01	0.01\\
187.01	0.01\\
188.01	0.01\\
189.01	0.01\\
190.01	0.01\\
191.01	0.01\\
192.01	0.01\\
193.01	0.01\\
194.01	0.01\\
195.01	0.01\\
196.01	0.01\\
197.01	0.01\\
198.01	0.01\\
199.01	0.01\\
200.01	0.01\\
201.01	0.01\\
202.01	0.01\\
203.01	0.01\\
204.01	0.01\\
205.01	0.01\\
206.01	0.01\\
207.01	0.01\\
208.01	0.01\\
209.01	0.01\\
210.01	0.01\\
211.01	0.01\\
212.01	0.01\\
213.01	0.01\\
214.01	0.01\\
215.01	0.01\\
216.01	0.01\\
217.01	0.01\\
218.01	0.01\\
219.01	0.01\\
220.01	0.01\\
221.01	0.01\\
222.01	0.01\\
223.01	0.01\\
224.01	0.01\\
225.01	0.01\\
226.01	0.01\\
227.01	0.01\\
228.01	0.01\\
229.01	0.01\\
230.01	0.01\\
231.01	0.01\\
232.01	0.01\\
233.01	0.01\\
234.01	0.01\\
235.01	0.01\\
236.01	0.01\\
237.01	0.01\\
238.01	0.01\\
239.01	0.01\\
240.01	0.01\\
241.01	0.01\\
242.01	0.01\\
243.01	0.01\\
244.01	0.01\\
245.01	0.01\\
246.01	0.01\\
247.01	0.01\\
248.01	0.01\\
249.01	0.01\\
250.01	0.01\\
251.01	0.01\\
252.01	0.01\\
253.01	0.01\\
254.01	0.01\\
255.01	0.01\\
256.01	0.01\\
257.01	0.01\\
258.01	0.01\\
259.01	0.01\\
260.01	0.01\\
261.01	0.01\\
262.01	0.01\\
263.01	0.01\\
264.01	0.01\\
265.01	0.01\\
266.01	0.01\\
267.01	0.01\\
268.01	0.01\\
269.01	0.01\\
270.01	0.01\\
271.01	0.01\\
272.01	0.01\\
273.01	0.01\\
274.01	0.01\\
275.01	0.01\\
276.01	0.01\\
277.01	0.01\\
278.01	0.01\\
279.01	0.01\\
280.01	0.01\\
281.01	0.01\\
282.01	0.01\\
283.01	0.01\\
284.01	0.01\\
285.01	0.01\\
286.01	0.01\\
287.01	0.01\\
288.01	0.01\\
289.01	0.01\\
290.01	0.01\\
291.01	0.01\\
292.01	0.01\\
293.01	0.01\\
294.01	0.01\\
295.01	0.01\\
296.01	0.01\\
297.01	0.01\\
298.01	0.01\\
299.01	0.01\\
300.01	0.01\\
301.01	0.01\\
302.01	0.01\\
303.01	0.01\\
304.01	0.01\\
305.01	0.01\\
306.01	0.01\\
307.01	0.01\\
308.01	0.01\\
309.01	0.01\\
310.01	0.01\\
311.01	0.01\\
312.01	0.01\\
313.01	0.01\\
314.01	0.01\\
315.01	0.01\\
316.01	0.01\\
317.01	0.01\\
318.01	0.01\\
319.01	0.01\\
320.01	0.01\\
321.01	0.01\\
322.01	0.01\\
323.01	0.01\\
324.01	0.01\\
325.01	0.01\\
326.01	0.01\\
327.01	0.01\\
328.01	0.01\\
329.01	0.01\\
330.01	0.01\\
331.01	0.01\\
332.01	0.01\\
333.01	0.01\\
334.01	0.01\\
335.01	0.01\\
336.01	0.01\\
337.01	0.01\\
338.01	0.01\\
339.01	0.01\\
340.01	0.01\\
341.01	0.01\\
342.01	0.01\\
343.01	0.01\\
344.01	0.01\\
345.01	0.01\\
346.01	0.01\\
347.01	0.01\\
348.01	0.01\\
349.01	0.01\\
350.01	0.01\\
351.01	0.01\\
352.01	0.01\\
353.01	0.01\\
354.01	0.01\\
355.01	0.01\\
356.01	0.01\\
357.01	0.01\\
358.01	0.01\\
359.01	0.01\\
360.01	0.01\\
361.01	0.01\\
362.01	0.01\\
363.01	0.01\\
364.01	0.01\\
365.01	0.01\\
366.01	0.01\\
367.01	0.01\\
368.01	0.01\\
369.01	0.01\\
370.01	0.01\\
371.01	0.01\\
372.01	0.01\\
373.01	0.01\\
374.01	0.01\\
375.01	0.01\\
376.01	0.01\\
377.01	0.01\\
378.01	0.01\\
379.01	0.01\\
380.01	0.01\\
381.01	0.01\\
382.01	0.01\\
383.01	0.01\\
384.01	0.01\\
385.01	0.01\\
386.01	0.01\\
387.01	0.01\\
388.01	0.01\\
389.01	0.01\\
390.01	0.01\\
391.01	0.01\\
392.01	0.01\\
393.01	0.01\\
394.01	0.01\\
395.01	0.01\\
396.01	0.01\\
397.01	0.01\\
398.01	0.01\\
399.01	0.01\\
400.01	0.01\\
401.01	0.01\\
402.01	0.01\\
403.01	0.01\\
404.01	0.01\\
405.01	0.01\\
406.01	0.01\\
407.01	0.01\\
408.01	0.01\\
409.01	0.01\\
410.01	0.01\\
411.01	0.01\\
412.01	0.01\\
413.01	0.01\\
414.01	0.01\\
415.01	0.01\\
416.01	0.01\\
417.01	0.01\\
418.01	0.01\\
419.01	0.01\\
420.01	0.01\\
421.01	0.01\\
422.01	0.01\\
423.01	0.01\\
424.01	0.01\\
425.01	0.01\\
426.01	0.01\\
427.01	0.01\\
428.01	0.01\\
429.01	0.01\\
430.01	0.01\\
431.01	0.01\\
432.01	0.01\\
433.01	0.01\\
434.01	0.01\\
435.01	0.01\\
436.01	0.01\\
437.01	0.01\\
438.01	0.01\\
439.01	0.01\\
440.01	0.01\\
441.01	0.01\\
442.01	0.01\\
443.01	0.01\\
444.01	0.01\\
445.01	0.01\\
446.01	0.01\\
447.01	0.01\\
448.01	0.01\\
449.01	0.01\\
450.01	0.01\\
451.01	0.01\\
452.01	0.01\\
453.01	0.01\\
454.01	0.01\\
455.01	0.01\\
456.01	0.01\\
457.01	0.01\\
458.01	0.01\\
459.01	0.01\\
460.01	0.01\\
461.01	0.01\\
462.01	0.01\\
463.01	0.01\\
464.01	0.01\\
465.01	0.01\\
466.01	0.01\\
467.01	0.01\\
468.01	0.01\\
469.01	0.01\\
470.01	0.01\\
471.01	0.01\\
472.01	0.01\\
473.01	0.01\\
474.01	0.01\\
475.01	0.01\\
476.01	0.01\\
477.01	0.01\\
478.01	0.01\\
479.01	0.01\\
480.01	0.01\\
481.01	0.01\\
482.01	0.01\\
483.01	0.01\\
484.01	0.01\\
485.01	0.01\\
486.01	0.01\\
487.01	0.01\\
488.01	0.01\\
489.01	0.01\\
490.01	0.01\\
491.01	0.01\\
492.01	0.01\\
493.01	0.01\\
494.01	0.01\\
495.01	0.01\\
496.01	0.01\\
497.01	0.01\\
498.01	0.01\\
499.01	0.01\\
500.01	0.01\\
501.01	0.01\\
502.01	0.01\\
503.01	0.01\\
504.01	0.01\\
505.01	0.01\\
506.01	0.01\\
507.01	0.01\\
508.01	0.01\\
509.01	0.01\\
510.01	0.01\\
511.01	0.01\\
512.01	0.01\\
513.01	0.01\\
514.01	0.01\\
515.01	0.01\\
516.01	0.01\\
517.01	0.01\\
518.01	0.01\\
519.01	0.01\\
520.01	0.01\\
521.01	0.01\\
522.01	0.01\\
523.01	0.01\\
524.01	0.01\\
525.01	0.01\\
526.01	0.01\\
527.01	0.01\\
528.01	0.01\\
529.01	0.01\\
530.01	0.01\\
531.01	0.01\\
532.01	0.01\\
533.01	0.01\\
534.01	0.01\\
535.01	0.01\\
536.01	0.01\\
537.01	0.01\\
538.01	0.01\\
539.01	0.01\\
540.01	0.01\\
541.01	0.01\\
542.01	0.01\\
543.01	0.01\\
544.01	0.01\\
545.01	0.01\\
546.01	0.01\\
547.01	0.01\\
548.01	0.01\\
549.01	0.01\\
550.01	0.01\\
551.01	0.01\\
552.01	0.01\\
553.01	0.01\\
554.01	0.01\\
555.01	0.01\\
556.01	0.01\\
557.01	0.01\\
558.01	0.01\\
559.01	0.01\\
560.01	0.01\\
561.01	0.01\\
562.01	0.01\\
563.01	0.01\\
564.01	0.01\\
565.01	0.01\\
566.01	0.01\\
567.01	0.01\\
568.01	0.01\\
569.01	0.01\\
570.01	0.01\\
571.01	0.01\\
572.01	0.01\\
573.01	0.01\\
574.01	0.01\\
575.01	0.01\\
576.01	0.01\\
577.01	0.01\\
578.01	0.01\\
579.01	0.01\\
580.01	0.01\\
581.01	0.01\\
582.01	0.01\\
583.01	0.01\\
584.01	0.01\\
585.01	0.01\\
586.01	0.01\\
587.01	0.01\\
588.01	0.01\\
589.01	0.01\\
590.01	0.01\\
591.01	0.01\\
592.01	0.01\\
593.01	0.01\\
594.01	0.01\\
595.01	0.01\\
596.01	0.01\\
597.01	0.01\\
598.01	0.01\\
599.01	0.01\\
599.02	0.01\\
599.03	0.01\\
599.04	0.01\\
599.05	0.01\\
599.06	0.01\\
599.07	0.01\\
599.08	0.01\\
599.09	0.01\\
599.1	0.01\\
599.11	0.01\\
599.12	0.01\\
599.13	0.01\\
599.14	0.01\\
599.15	0.01\\
599.16	0.01\\
599.17	0.01\\
599.18	0.01\\
599.19	0.01\\
599.2	0.01\\
599.21	0.01\\
599.22	0.01\\
599.23	0.01\\
599.24	0.01\\
599.25	0.01\\
599.26	0.01\\
599.27	0.01\\
599.28	0.01\\
599.29	0.01\\
599.3	0.01\\
599.31	0.01\\
599.32	0.01\\
599.33	0.01\\
599.34	0.01\\
599.35	0.01\\
599.36	0.01\\
599.37	0.01\\
599.38	0.01\\
599.39	0.01\\
599.4	0.01\\
599.41	0.01\\
599.42	0.01\\
599.43	0.01\\
599.44	0.01\\
599.45	0.01\\
599.46	0.01\\
599.47	0.01\\
599.48	0.01\\
599.49	0.01\\
599.5	0.01\\
599.51	0.01\\
599.52	0.01\\
599.53	0.01\\
599.54	0.01\\
599.55	0.01\\
599.56	0.01\\
599.57	0.01\\
599.58	0.01\\
599.59	0.01\\
599.6	0.01\\
599.61	0.01\\
599.62	0.01\\
599.63	0.01\\
599.64	0.01\\
599.65	0.01\\
599.66	0.01\\
599.67	0.01\\
599.68	0.01\\
599.69	0.01\\
599.7	0.01\\
599.71	0.01\\
599.72	0.01\\
599.73	0.01\\
599.74	0.01\\
599.75	0.01\\
599.76	0.01\\
599.77	0.01\\
599.78	0.01\\
599.79	0.01\\
599.8	0.01\\
599.81	0.01\\
599.82	0.01\\
599.83	0.01\\
599.84	0.01\\
599.85	0.01\\
599.86	0.01\\
599.87	0.01\\
599.88	0.01\\
599.89	0.01\\
599.9	0.01\\
599.91	0.01\\
599.92	0.01\\
599.93	0.01\\
599.94	0.01\\
599.95	0.01\\
599.96	0.01\\
599.97	0.01\\
599.98	0.01\\
599.99	0.01\\
600	0.01\\
};
\addplot [color=mycolor20,solid,forget plot]
  table[row sep=crcr]{%
0.01	0.01\\
1.01	0.01\\
2.01	0.01\\
3.01	0.01\\
4.01	0.01\\
5.01	0.01\\
6.01	0.01\\
7.01	0.01\\
8.01	0.01\\
9.01	0.01\\
10.01	0.01\\
11.01	0.01\\
12.01	0.01\\
13.01	0.01\\
14.01	0.01\\
15.01	0.01\\
16.01	0.01\\
17.01	0.01\\
18.01	0.01\\
19.01	0.01\\
20.01	0.01\\
21.01	0.01\\
22.01	0.01\\
23.01	0.01\\
24.01	0.01\\
25.01	0.01\\
26.01	0.01\\
27.01	0.01\\
28.01	0.01\\
29.01	0.01\\
30.01	0.01\\
31.01	0.01\\
32.01	0.01\\
33.01	0.01\\
34.01	0.01\\
35.01	0.01\\
36.01	0.01\\
37.01	0.01\\
38.01	0.01\\
39.01	0.01\\
40.01	0.01\\
41.01	0.01\\
42.01	0.01\\
43.01	0.01\\
44.01	0.01\\
45.01	0.01\\
46.01	0.01\\
47.01	0.01\\
48.01	0.01\\
49.01	0.01\\
50.01	0.01\\
51.01	0.01\\
52.01	0.01\\
53.01	0.01\\
54.01	0.01\\
55.01	0.01\\
56.01	0.01\\
57.01	0.01\\
58.01	0.01\\
59.01	0.01\\
60.01	0.01\\
61.01	0.01\\
62.01	0.01\\
63.01	0.01\\
64.01	0.01\\
65.01	0.01\\
66.01	0.01\\
67.01	0.01\\
68.01	0.01\\
69.01	0.01\\
70.01	0.01\\
71.01	0.01\\
72.01	0.01\\
73.01	0.01\\
74.01	0.01\\
75.01	0.01\\
76.01	0.01\\
77.01	0.01\\
78.01	0.01\\
79.01	0.01\\
80.01	0.01\\
81.01	0.01\\
82.01	0.01\\
83.01	0.01\\
84.01	0.01\\
85.01	0.01\\
86.01	0.01\\
87.01	0.01\\
88.01	0.01\\
89.01	0.01\\
90.01	0.01\\
91.01	0.01\\
92.01	0.01\\
93.01	0.01\\
94.01	0.01\\
95.01	0.01\\
96.01	0.01\\
97.01	0.01\\
98.01	0.01\\
99.01	0.01\\
100.01	0.01\\
101.01	0.01\\
102.01	0.01\\
103.01	0.01\\
104.01	0.01\\
105.01	0.01\\
106.01	0.01\\
107.01	0.01\\
108.01	0.01\\
109.01	0.01\\
110.01	0.01\\
111.01	0.01\\
112.01	0.01\\
113.01	0.01\\
114.01	0.01\\
115.01	0.01\\
116.01	0.01\\
117.01	0.01\\
118.01	0.01\\
119.01	0.01\\
120.01	0.01\\
121.01	0.01\\
122.01	0.01\\
123.01	0.01\\
124.01	0.01\\
125.01	0.01\\
126.01	0.01\\
127.01	0.01\\
128.01	0.01\\
129.01	0.01\\
130.01	0.01\\
131.01	0.01\\
132.01	0.01\\
133.01	0.01\\
134.01	0.01\\
135.01	0.01\\
136.01	0.01\\
137.01	0.01\\
138.01	0.01\\
139.01	0.01\\
140.01	0.01\\
141.01	0.01\\
142.01	0.01\\
143.01	0.01\\
144.01	0.01\\
145.01	0.01\\
146.01	0.01\\
147.01	0.01\\
148.01	0.01\\
149.01	0.01\\
150.01	0.01\\
151.01	0.01\\
152.01	0.01\\
153.01	0.01\\
154.01	0.01\\
155.01	0.01\\
156.01	0.01\\
157.01	0.01\\
158.01	0.01\\
159.01	0.01\\
160.01	0.01\\
161.01	0.01\\
162.01	0.01\\
163.01	0.01\\
164.01	0.01\\
165.01	0.01\\
166.01	0.01\\
167.01	0.01\\
168.01	0.01\\
169.01	0.01\\
170.01	0.01\\
171.01	0.01\\
172.01	0.01\\
173.01	0.01\\
174.01	0.01\\
175.01	0.01\\
176.01	0.01\\
177.01	0.01\\
178.01	0.01\\
179.01	0.01\\
180.01	0.01\\
181.01	0.01\\
182.01	0.01\\
183.01	0.01\\
184.01	0.01\\
185.01	0.01\\
186.01	0.01\\
187.01	0.01\\
188.01	0.01\\
189.01	0.01\\
190.01	0.01\\
191.01	0.01\\
192.01	0.01\\
193.01	0.01\\
194.01	0.01\\
195.01	0.01\\
196.01	0.01\\
197.01	0.01\\
198.01	0.01\\
199.01	0.01\\
200.01	0.01\\
201.01	0.01\\
202.01	0.01\\
203.01	0.01\\
204.01	0.01\\
205.01	0.01\\
206.01	0.01\\
207.01	0.01\\
208.01	0.01\\
209.01	0.01\\
210.01	0.01\\
211.01	0.01\\
212.01	0.01\\
213.01	0.01\\
214.01	0.01\\
215.01	0.01\\
216.01	0.01\\
217.01	0.01\\
218.01	0.01\\
219.01	0.01\\
220.01	0.01\\
221.01	0.01\\
222.01	0.01\\
223.01	0.01\\
224.01	0.01\\
225.01	0.01\\
226.01	0.01\\
227.01	0.01\\
228.01	0.01\\
229.01	0.01\\
230.01	0.01\\
231.01	0.01\\
232.01	0.01\\
233.01	0.01\\
234.01	0.01\\
235.01	0.01\\
236.01	0.01\\
237.01	0.01\\
238.01	0.01\\
239.01	0.01\\
240.01	0.01\\
241.01	0.01\\
242.01	0.01\\
243.01	0.01\\
244.01	0.01\\
245.01	0.01\\
246.01	0.01\\
247.01	0.01\\
248.01	0.01\\
249.01	0.01\\
250.01	0.01\\
251.01	0.01\\
252.01	0.01\\
253.01	0.01\\
254.01	0.01\\
255.01	0.01\\
256.01	0.01\\
257.01	0.01\\
258.01	0.01\\
259.01	0.01\\
260.01	0.01\\
261.01	0.01\\
262.01	0.01\\
263.01	0.01\\
264.01	0.01\\
265.01	0.01\\
266.01	0.01\\
267.01	0.01\\
268.01	0.01\\
269.01	0.01\\
270.01	0.01\\
271.01	0.01\\
272.01	0.01\\
273.01	0.01\\
274.01	0.01\\
275.01	0.01\\
276.01	0.01\\
277.01	0.01\\
278.01	0.01\\
279.01	0.01\\
280.01	0.01\\
281.01	0.01\\
282.01	0.01\\
283.01	0.01\\
284.01	0.01\\
285.01	0.01\\
286.01	0.01\\
287.01	0.01\\
288.01	0.01\\
289.01	0.01\\
290.01	0.01\\
291.01	0.01\\
292.01	0.01\\
293.01	0.01\\
294.01	0.01\\
295.01	0.01\\
296.01	0.01\\
297.01	0.01\\
298.01	0.01\\
299.01	0.01\\
300.01	0.01\\
301.01	0.01\\
302.01	0.01\\
303.01	0.01\\
304.01	0.01\\
305.01	0.01\\
306.01	0.01\\
307.01	0.01\\
308.01	0.01\\
309.01	0.01\\
310.01	0.01\\
311.01	0.01\\
312.01	0.01\\
313.01	0.01\\
314.01	0.01\\
315.01	0.01\\
316.01	0.01\\
317.01	0.01\\
318.01	0.01\\
319.01	0.01\\
320.01	0.01\\
321.01	0.01\\
322.01	0.01\\
323.01	0.01\\
324.01	0.01\\
325.01	0.01\\
326.01	0.01\\
327.01	0.01\\
328.01	0.01\\
329.01	0.01\\
330.01	0.01\\
331.01	0.01\\
332.01	0.01\\
333.01	0.01\\
334.01	0.01\\
335.01	0.01\\
336.01	0.01\\
337.01	0.01\\
338.01	0.01\\
339.01	0.01\\
340.01	0.01\\
341.01	0.01\\
342.01	0.01\\
343.01	0.01\\
344.01	0.01\\
345.01	0.01\\
346.01	0.01\\
347.01	0.01\\
348.01	0.01\\
349.01	0.01\\
350.01	0.01\\
351.01	0.01\\
352.01	0.01\\
353.01	0.01\\
354.01	0.01\\
355.01	0.01\\
356.01	0.01\\
357.01	0.01\\
358.01	0.01\\
359.01	0.01\\
360.01	0.01\\
361.01	0.01\\
362.01	0.01\\
363.01	0.01\\
364.01	0.01\\
365.01	0.01\\
366.01	0.01\\
367.01	0.01\\
368.01	0.01\\
369.01	0.01\\
370.01	0.01\\
371.01	0.01\\
372.01	0.01\\
373.01	0.01\\
374.01	0.01\\
375.01	0.01\\
376.01	0.01\\
377.01	0.01\\
378.01	0.01\\
379.01	0.01\\
380.01	0.01\\
381.01	0.01\\
382.01	0.01\\
383.01	0.01\\
384.01	0.01\\
385.01	0.01\\
386.01	0.01\\
387.01	0.01\\
388.01	0.01\\
389.01	0.01\\
390.01	0.01\\
391.01	0.01\\
392.01	0.01\\
393.01	0.01\\
394.01	0.01\\
395.01	0.01\\
396.01	0.01\\
397.01	0.01\\
398.01	0.01\\
399.01	0.01\\
400.01	0.01\\
401.01	0.01\\
402.01	0.01\\
403.01	0.01\\
404.01	0.01\\
405.01	0.01\\
406.01	0.01\\
407.01	0.01\\
408.01	0.01\\
409.01	0.01\\
410.01	0.01\\
411.01	0.01\\
412.01	0.01\\
413.01	0.01\\
414.01	0.01\\
415.01	0.01\\
416.01	0.01\\
417.01	0.01\\
418.01	0.01\\
419.01	0.01\\
420.01	0.01\\
421.01	0.01\\
422.01	0.01\\
423.01	0.01\\
424.01	0.01\\
425.01	0.01\\
426.01	0.01\\
427.01	0.01\\
428.01	0.01\\
429.01	0.01\\
430.01	0.01\\
431.01	0.01\\
432.01	0.01\\
433.01	0.01\\
434.01	0.01\\
435.01	0.01\\
436.01	0.01\\
437.01	0.01\\
438.01	0.01\\
439.01	0.01\\
440.01	0.01\\
441.01	0.01\\
442.01	0.01\\
443.01	0.01\\
444.01	0.01\\
445.01	0.01\\
446.01	0.01\\
447.01	0.01\\
448.01	0.01\\
449.01	0.01\\
450.01	0.01\\
451.01	0.01\\
452.01	0.01\\
453.01	0.01\\
454.01	0.01\\
455.01	0.01\\
456.01	0.01\\
457.01	0.01\\
458.01	0.01\\
459.01	0.01\\
460.01	0.01\\
461.01	0.01\\
462.01	0.01\\
463.01	0.01\\
464.01	0.01\\
465.01	0.01\\
466.01	0.01\\
467.01	0.01\\
468.01	0.01\\
469.01	0.01\\
470.01	0.01\\
471.01	0.01\\
472.01	0.01\\
473.01	0.01\\
474.01	0.01\\
475.01	0.01\\
476.01	0.01\\
477.01	0.01\\
478.01	0.01\\
479.01	0.01\\
480.01	0.01\\
481.01	0.01\\
482.01	0.01\\
483.01	0.01\\
484.01	0.01\\
485.01	0.01\\
486.01	0.01\\
487.01	0.01\\
488.01	0.01\\
489.01	0.01\\
490.01	0.01\\
491.01	0.01\\
492.01	0.01\\
493.01	0.01\\
494.01	0.01\\
495.01	0.01\\
496.01	0.01\\
497.01	0.01\\
498.01	0.01\\
499.01	0.01\\
500.01	0.01\\
501.01	0.01\\
502.01	0.01\\
503.01	0.01\\
504.01	0.01\\
505.01	0.01\\
506.01	0.01\\
507.01	0.01\\
508.01	0.01\\
509.01	0.01\\
510.01	0.01\\
511.01	0.01\\
512.01	0.01\\
513.01	0.01\\
514.01	0.01\\
515.01	0.01\\
516.01	0.01\\
517.01	0.01\\
518.01	0.01\\
519.01	0.01\\
520.01	0.01\\
521.01	0.01\\
522.01	0.01\\
523.01	0.01\\
524.01	0.01\\
525.01	0.01\\
526.01	0.01\\
527.01	0.01\\
528.01	0.01\\
529.01	0.01\\
530.01	0.01\\
531.01	0.01\\
532.01	0.01\\
533.01	0.01\\
534.01	0.01\\
535.01	0.01\\
536.01	0.01\\
537.01	0.01\\
538.01	0.01\\
539.01	0.01\\
540.01	0.01\\
541.01	0.01\\
542.01	0.01\\
543.01	0.01\\
544.01	0.01\\
545.01	0.01\\
546.01	0.01\\
547.01	0.01\\
548.01	0.01\\
549.01	0.01\\
550.01	0.01\\
551.01	0.01\\
552.01	0.01\\
553.01	0.01\\
554.01	0.01\\
555.01	0.01\\
556.01	0.01\\
557.01	0.01\\
558.01	0.01\\
559.01	0.01\\
560.01	0.01\\
561.01	0.01\\
562.01	0.01\\
563.01	0.01\\
564.01	0.01\\
565.01	0.01\\
566.01	0.01\\
567.01	0.01\\
568.01	0.01\\
569.01	0.01\\
570.01	0.01\\
571.01	0.01\\
572.01	0.01\\
573.01	0.01\\
574.01	0.01\\
575.01	0.01\\
576.01	0.01\\
577.01	0.01\\
578.01	0.01\\
579.01	0.01\\
580.01	0.01\\
581.01	0.01\\
582.01	0.01\\
583.01	0.01\\
584.01	0.01\\
585.01	0.01\\
586.01	0.01\\
587.01	0.01\\
588.01	0.01\\
589.01	0.01\\
590.01	0.01\\
591.01	0.01\\
592.01	0.01\\
593.01	0.01\\
594.01	0.01\\
595.01	0.01\\
596.01	0.01\\
597.01	0.01\\
598.01	0.01\\
599.01	0.01\\
599.02	0.01\\
599.03	0.01\\
599.04	0.01\\
599.05	0.01\\
599.06	0.01\\
599.07	0.01\\
599.08	0.01\\
599.09	0.01\\
599.1	0.01\\
599.11	0.01\\
599.12	0.01\\
599.13	0.01\\
599.14	0.01\\
599.15	0.01\\
599.16	0.01\\
599.17	0.01\\
599.18	0.01\\
599.19	0.01\\
599.2	0.01\\
599.21	0.01\\
599.22	0.01\\
599.23	0.01\\
599.24	0.01\\
599.25	0.01\\
599.26	0.01\\
599.27	0.01\\
599.28	0.01\\
599.29	0.01\\
599.3	0.01\\
599.31	0.01\\
599.32	0.01\\
599.33	0.01\\
599.34	0.01\\
599.35	0.01\\
599.36	0.01\\
599.37	0.01\\
599.38	0.01\\
599.39	0.01\\
599.4	0.01\\
599.41	0.01\\
599.42	0.01\\
599.43	0.01\\
599.44	0.01\\
599.45	0.01\\
599.46	0.01\\
599.47	0.01\\
599.48	0.01\\
599.49	0.01\\
599.5	0.01\\
599.51	0.01\\
599.52	0.01\\
599.53	0.01\\
599.54	0.01\\
599.55	0.01\\
599.56	0.01\\
599.57	0.01\\
599.58	0.01\\
599.59	0.01\\
599.6	0.01\\
599.61	0.01\\
599.62	0.01\\
599.63	0.01\\
599.64	0.01\\
599.65	0.01\\
599.66	0.01\\
599.67	0.01\\
599.68	0.01\\
599.69	0.01\\
599.7	0.01\\
599.71	0.01\\
599.72	0.01\\
599.73	0.01\\
599.74	0.01\\
599.75	0.01\\
599.76	0.01\\
599.77	0.01\\
599.78	0.01\\
599.79	0.01\\
599.8	0.01\\
599.81	0.01\\
599.82	0.01\\
599.83	0.01\\
599.84	0.01\\
599.85	0.01\\
599.86	0.01\\
599.87	0.01\\
599.88	0.01\\
599.89	0.01\\
599.9	0.01\\
599.91	0.01\\
599.92	0.01\\
599.93	0.01\\
599.94	0.01\\
599.95	0.01\\
599.96	0.01\\
599.97	0.01\\
599.98	0.01\\
599.99	0.01\\
600	0.01\\
};
\addplot [color=mycolor21,solid,forget plot]
  table[row sep=crcr]{%
0.01	0.01\\
1.01	0.01\\
2.01	0.01\\
3.01	0.01\\
4.01	0.01\\
5.01	0.01\\
6.01	0.01\\
7.01	0.01\\
8.01	0.01\\
9.01	0.01\\
10.01	0.01\\
11.01	0.01\\
12.01	0.01\\
13.01	0.01\\
14.01	0.01\\
15.01	0.01\\
16.01	0.01\\
17.01	0.01\\
18.01	0.01\\
19.01	0.01\\
20.01	0.01\\
21.01	0.01\\
22.01	0.01\\
23.01	0.01\\
24.01	0.01\\
25.01	0.01\\
26.01	0.01\\
27.01	0.01\\
28.01	0.01\\
29.01	0.01\\
30.01	0.01\\
31.01	0.01\\
32.01	0.01\\
33.01	0.01\\
34.01	0.01\\
35.01	0.01\\
36.01	0.01\\
37.01	0.01\\
38.01	0.01\\
39.01	0.01\\
40.01	0.01\\
41.01	0.01\\
42.01	0.01\\
43.01	0.01\\
44.01	0.01\\
45.01	0.01\\
46.01	0.01\\
47.01	0.01\\
48.01	0.01\\
49.01	0.01\\
50.01	0.01\\
51.01	0.01\\
52.01	0.01\\
53.01	0.01\\
54.01	0.01\\
55.01	0.01\\
56.01	0.01\\
57.01	0.01\\
58.01	0.01\\
59.01	0.01\\
60.01	0.01\\
61.01	0.01\\
62.01	0.01\\
63.01	0.01\\
64.01	0.01\\
65.01	0.01\\
66.01	0.01\\
67.01	0.01\\
68.01	0.01\\
69.01	0.01\\
70.01	0.01\\
71.01	0.01\\
72.01	0.01\\
73.01	0.01\\
74.01	0.01\\
75.01	0.01\\
76.01	0.01\\
77.01	0.01\\
78.01	0.01\\
79.01	0.01\\
80.01	0.01\\
81.01	0.01\\
82.01	0.01\\
83.01	0.01\\
84.01	0.01\\
85.01	0.01\\
86.01	0.01\\
87.01	0.01\\
88.01	0.01\\
89.01	0.01\\
90.01	0.01\\
91.01	0.01\\
92.01	0.01\\
93.01	0.01\\
94.01	0.01\\
95.01	0.01\\
96.01	0.01\\
97.01	0.01\\
98.01	0.01\\
99.01	0.01\\
100.01	0.01\\
101.01	0.01\\
102.01	0.01\\
103.01	0.01\\
104.01	0.01\\
105.01	0.01\\
106.01	0.01\\
107.01	0.01\\
108.01	0.01\\
109.01	0.01\\
110.01	0.01\\
111.01	0.01\\
112.01	0.01\\
113.01	0.01\\
114.01	0.01\\
115.01	0.01\\
116.01	0.01\\
117.01	0.01\\
118.01	0.01\\
119.01	0.01\\
120.01	0.01\\
121.01	0.01\\
122.01	0.01\\
123.01	0.01\\
124.01	0.01\\
125.01	0.01\\
126.01	0.01\\
127.01	0.01\\
128.01	0.01\\
129.01	0.01\\
130.01	0.01\\
131.01	0.01\\
132.01	0.01\\
133.01	0.01\\
134.01	0.01\\
135.01	0.01\\
136.01	0.01\\
137.01	0.01\\
138.01	0.01\\
139.01	0.01\\
140.01	0.01\\
141.01	0.01\\
142.01	0.01\\
143.01	0.01\\
144.01	0.01\\
145.01	0.01\\
146.01	0.01\\
147.01	0.01\\
148.01	0.01\\
149.01	0.01\\
150.01	0.01\\
151.01	0.01\\
152.01	0.01\\
153.01	0.01\\
154.01	0.01\\
155.01	0.01\\
156.01	0.01\\
157.01	0.01\\
158.01	0.01\\
159.01	0.01\\
160.01	0.01\\
161.01	0.01\\
162.01	0.01\\
163.01	0.01\\
164.01	0.01\\
165.01	0.01\\
166.01	0.01\\
167.01	0.01\\
168.01	0.01\\
169.01	0.01\\
170.01	0.01\\
171.01	0.01\\
172.01	0.01\\
173.01	0.01\\
174.01	0.01\\
175.01	0.01\\
176.01	0.01\\
177.01	0.01\\
178.01	0.01\\
179.01	0.01\\
180.01	0.01\\
181.01	0.01\\
182.01	0.01\\
183.01	0.01\\
184.01	0.01\\
185.01	0.01\\
186.01	0.01\\
187.01	0.01\\
188.01	0.01\\
189.01	0.01\\
190.01	0.01\\
191.01	0.01\\
192.01	0.01\\
193.01	0.01\\
194.01	0.01\\
195.01	0.01\\
196.01	0.01\\
197.01	0.01\\
198.01	0.01\\
199.01	0.01\\
200.01	0.01\\
201.01	0.01\\
202.01	0.01\\
203.01	0.01\\
204.01	0.01\\
205.01	0.01\\
206.01	0.01\\
207.01	0.01\\
208.01	0.01\\
209.01	0.01\\
210.01	0.01\\
211.01	0.01\\
212.01	0.01\\
213.01	0.01\\
214.01	0.01\\
215.01	0.01\\
216.01	0.01\\
217.01	0.01\\
218.01	0.01\\
219.01	0.01\\
220.01	0.01\\
221.01	0.01\\
222.01	0.01\\
223.01	0.01\\
224.01	0.01\\
225.01	0.01\\
226.01	0.01\\
227.01	0.01\\
228.01	0.01\\
229.01	0.01\\
230.01	0.01\\
231.01	0.01\\
232.01	0.01\\
233.01	0.01\\
234.01	0.01\\
235.01	0.01\\
236.01	0.01\\
237.01	0.01\\
238.01	0.01\\
239.01	0.01\\
240.01	0.01\\
241.01	0.01\\
242.01	0.01\\
243.01	0.01\\
244.01	0.01\\
245.01	0.01\\
246.01	0.01\\
247.01	0.01\\
248.01	0.01\\
249.01	0.01\\
250.01	0.01\\
251.01	0.01\\
252.01	0.01\\
253.01	0.01\\
254.01	0.01\\
255.01	0.01\\
256.01	0.01\\
257.01	0.01\\
258.01	0.01\\
259.01	0.01\\
260.01	0.01\\
261.01	0.01\\
262.01	0.01\\
263.01	0.01\\
264.01	0.01\\
265.01	0.01\\
266.01	0.01\\
267.01	0.01\\
268.01	0.01\\
269.01	0.01\\
270.01	0.01\\
271.01	0.01\\
272.01	0.01\\
273.01	0.01\\
274.01	0.01\\
275.01	0.01\\
276.01	0.01\\
277.01	0.01\\
278.01	0.01\\
279.01	0.01\\
280.01	0.01\\
281.01	0.01\\
282.01	0.01\\
283.01	0.01\\
284.01	0.01\\
285.01	0.01\\
286.01	0.01\\
287.01	0.01\\
288.01	0.01\\
289.01	0.01\\
290.01	0.01\\
291.01	0.01\\
292.01	0.01\\
293.01	0.01\\
294.01	0.01\\
295.01	0.01\\
296.01	0.01\\
297.01	0.01\\
298.01	0.01\\
299.01	0.01\\
300.01	0.01\\
301.01	0.01\\
302.01	0.01\\
303.01	0.01\\
304.01	0.01\\
305.01	0.01\\
306.01	0.01\\
307.01	0.01\\
308.01	0.01\\
309.01	0.01\\
310.01	0.01\\
311.01	0.01\\
312.01	0.01\\
313.01	0.01\\
314.01	0.01\\
315.01	0.01\\
316.01	0.01\\
317.01	0.01\\
318.01	0.01\\
319.01	0.01\\
320.01	0.01\\
321.01	0.01\\
322.01	0.01\\
323.01	0.01\\
324.01	0.01\\
325.01	0.01\\
326.01	0.01\\
327.01	0.01\\
328.01	0.01\\
329.01	0.01\\
330.01	0.01\\
331.01	0.01\\
332.01	0.01\\
333.01	0.01\\
334.01	0.01\\
335.01	0.01\\
336.01	0.01\\
337.01	0.01\\
338.01	0.01\\
339.01	0.01\\
340.01	0.01\\
341.01	0.01\\
342.01	0.01\\
343.01	0.01\\
344.01	0.01\\
345.01	0.01\\
346.01	0.01\\
347.01	0.01\\
348.01	0.01\\
349.01	0.01\\
350.01	0.01\\
351.01	0.01\\
352.01	0.01\\
353.01	0.01\\
354.01	0.01\\
355.01	0.01\\
356.01	0.01\\
357.01	0.01\\
358.01	0.01\\
359.01	0.01\\
360.01	0.01\\
361.01	0.01\\
362.01	0.01\\
363.01	0.01\\
364.01	0.01\\
365.01	0.01\\
366.01	0.01\\
367.01	0.01\\
368.01	0.01\\
369.01	0.01\\
370.01	0.01\\
371.01	0.01\\
372.01	0.01\\
373.01	0.01\\
374.01	0.01\\
375.01	0.01\\
376.01	0.01\\
377.01	0.01\\
378.01	0.01\\
379.01	0.01\\
380.01	0.01\\
381.01	0.01\\
382.01	0.01\\
383.01	0.01\\
384.01	0.01\\
385.01	0.01\\
386.01	0.01\\
387.01	0.01\\
388.01	0.01\\
389.01	0.01\\
390.01	0.01\\
391.01	0.01\\
392.01	0.01\\
393.01	0.01\\
394.01	0.01\\
395.01	0.01\\
396.01	0.01\\
397.01	0.01\\
398.01	0.01\\
399.01	0.01\\
400.01	0.01\\
401.01	0.01\\
402.01	0.01\\
403.01	0.01\\
404.01	0.01\\
405.01	0.01\\
406.01	0.01\\
407.01	0.01\\
408.01	0.01\\
409.01	0.01\\
410.01	0.01\\
411.01	0.01\\
412.01	0.01\\
413.01	0.01\\
414.01	0.01\\
415.01	0.01\\
416.01	0.01\\
417.01	0.01\\
418.01	0.01\\
419.01	0.01\\
420.01	0.01\\
421.01	0.01\\
422.01	0.01\\
423.01	0.01\\
424.01	0.01\\
425.01	0.01\\
426.01	0.01\\
427.01	0.01\\
428.01	0.01\\
429.01	0.01\\
430.01	0.01\\
431.01	0.01\\
432.01	0.01\\
433.01	0.01\\
434.01	0.01\\
435.01	0.01\\
436.01	0.01\\
437.01	0.01\\
438.01	0.01\\
439.01	0.01\\
440.01	0.01\\
441.01	0.01\\
442.01	0.01\\
443.01	0.01\\
444.01	0.01\\
445.01	0.01\\
446.01	0.01\\
447.01	0.01\\
448.01	0.01\\
449.01	0.01\\
450.01	0.01\\
451.01	0.01\\
452.01	0.01\\
453.01	0.01\\
454.01	0.01\\
455.01	0.01\\
456.01	0.01\\
457.01	0.01\\
458.01	0.01\\
459.01	0.01\\
460.01	0.01\\
461.01	0.01\\
462.01	0.01\\
463.01	0.01\\
464.01	0.01\\
465.01	0.01\\
466.01	0.01\\
467.01	0.01\\
468.01	0.01\\
469.01	0.01\\
470.01	0.01\\
471.01	0.01\\
472.01	0.01\\
473.01	0.01\\
474.01	0.01\\
475.01	0.01\\
476.01	0.01\\
477.01	0.01\\
478.01	0.01\\
479.01	0.01\\
480.01	0.01\\
481.01	0.01\\
482.01	0.01\\
483.01	0.01\\
484.01	0.01\\
485.01	0.01\\
486.01	0.01\\
487.01	0.01\\
488.01	0.01\\
489.01	0.01\\
490.01	0.01\\
491.01	0.01\\
492.01	0.01\\
493.01	0.01\\
494.01	0.01\\
495.01	0.01\\
496.01	0.01\\
497.01	0.01\\
498.01	0.01\\
499.01	0.01\\
500.01	0.01\\
501.01	0.01\\
502.01	0.01\\
503.01	0.01\\
504.01	0.01\\
505.01	0.01\\
506.01	0.01\\
507.01	0.01\\
508.01	0.01\\
509.01	0.01\\
510.01	0.01\\
511.01	0.01\\
512.01	0.01\\
513.01	0.01\\
514.01	0.01\\
515.01	0.01\\
516.01	0.01\\
517.01	0.01\\
518.01	0.01\\
519.01	0.01\\
520.01	0.01\\
521.01	0.01\\
522.01	0.01\\
523.01	0.01\\
524.01	0.01\\
525.01	0.01\\
526.01	0.01\\
527.01	0.01\\
528.01	0.01\\
529.01	0.01\\
530.01	0.01\\
531.01	0.01\\
532.01	0.01\\
533.01	0.01\\
534.01	0.01\\
535.01	0.01\\
536.01	0.01\\
537.01	0.01\\
538.01	0.01\\
539.01	0.01\\
540.01	0.01\\
541.01	0.01\\
542.01	0.01\\
543.01	0.01\\
544.01	0.01\\
545.01	0.01\\
546.01	0.01\\
547.01	0.01\\
548.01	0.01\\
549.01	0.01\\
550.01	0.01\\
551.01	0.01\\
552.01	0.01\\
553.01	0.01\\
554.01	0.01\\
555.01	0.01\\
556.01	0.01\\
557.01	0.01\\
558.01	0.01\\
559.01	0.01\\
560.01	0.01\\
561.01	0.01\\
562.01	0.01\\
563.01	0.01\\
564.01	0.01\\
565.01	0.01\\
566.01	0.01\\
567.01	0.01\\
568.01	0.01\\
569.01	0.01\\
570.01	0.01\\
571.01	0.01\\
572.01	0.01\\
573.01	0.01\\
574.01	0.01\\
575.01	0.01\\
576.01	0.01\\
577.01	0.01\\
578.01	0.01\\
579.01	0.01\\
580.01	0.01\\
581.01	0.01\\
582.01	0.01\\
583.01	0.01\\
584.01	0.01\\
585.01	0.01\\
586.01	0.01\\
587.01	0.01\\
588.01	0.01\\
589.01	0.01\\
590.01	0.01\\
591.01	0.01\\
592.01	0.01\\
593.01	0.01\\
594.01	0.01\\
595.01	0.01\\
596.01	0.01\\
597.01	0.01\\
598.01	0.01\\
599.01	0.01\\
599.02	0.01\\
599.03	0.01\\
599.04	0.01\\
599.05	0.01\\
599.06	0.01\\
599.07	0.01\\
599.08	0.01\\
599.09	0.01\\
599.1	0.01\\
599.11	0.01\\
599.12	0.01\\
599.13	0.01\\
599.14	0.01\\
599.15	0.01\\
599.16	0.01\\
599.17	0.01\\
599.18	0.01\\
599.19	0.01\\
599.2	0.01\\
599.21	0.01\\
599.22	0.01\\
599.23	0.01\\
599.24	0.01\\
599.25	0.01\\
599.26	0.01\\
599.27	0.01\\
599.28	0.01\\
599.29	0.01\\
599.3	0.01\\
599.31	0.01\\
599.32	0.01\\
599.33	0.01\\
599.34	0.01\\
599.35	0.01\\
599.36	0.01\\
599.37	0.01\\
599.38	0.01\\
599.39	0.01\\
599.4	0.01\\
599.41	0.01\\
599.42	0.01\\
599.43	0.01\\
599.44	0.01\\
599.45	0.01\\
599.46	0.01\\
599.47	0.01\\
599.48	0.01\\
599.49	0.01\\
599.5	0.01\\
599.51	0.01\\
599.52	0.01\\
599.53	0.01\\
599.54	0.01\\
599.55	0.01\\
599.56	0.01\\
599.57	0.01\\
599.58	0.01\\
599.59	0.01\\
599.6	0.01\\
599.61	0.01\\
599.62	0.01\\
599.63	0.01\\
599.64	0.01\\
599.65	0.01\\
599.66	0.01\\
599.67	0.01\\
599.68	0.01\\
599.69	0.01\\
599.7	0.01\\
599.71	0.01\\
599.72	0.01\\
599.73	0.01\\
599.74	0.01\\
599.75	0.01\\
599.76	0.01\\
599.77	0.01\\
599.78	0.01\\
599.79	0.01\\
599.8	0.01\\
599.81	0.01\\
599.82	0.01\\
599.83	0.01\\
599.84	0.01\\
599.85	0.01\\
599.86	0.01\\
599.87	0.01\\
599.88	0.01\\
599.89	0.01\\
599.9	0.01\\
599.91	0.01\\
599.92	0.01\\
599.93	0.01\\
599.94	0.01\\
599.95	0.01\\
599.96	0.01\\
599.97	0.01\\
599.98	0.01\\
599.99	0.01\\
600	0.01\\
};
\addplot [color=black!20!mycolor21,solid,forget plot]
  table[row sep=crcr]{%
0.01	0.01\\
1.01	0.01\\
2.01	0.01\\
3.01	0.01\\
4.01	0.01\\
5.01	0.01\\
6.01	0.01\\
7.01	0.01\\
8.01	0.01\\
9.01	0.01\\
10.01	0.01\\
11.01	0.01\\
12.01	0.01\\
13.01	0.01\\
14.01	0.01\\
15.01	0.01\\
16.01	0.01\\
17.01	0.01\\
18.01	0.01\\
19.01	0.01\\
20.01	0.01\\
21.01	0.01\\
22.01	0.01\\
23.01	0.01\\
24.01	0.01\\
25.01	0.01\\
26.01	0.01\\
27.01	0.01\\
28.01	0.01\\
29.01	0.01\\
30.01	0.01\\
31.01	0.01\\
32.01	0.01\\
33.01	0.01\\
34.01	0.01\\
35.01	0.01\\
36.01	0.01\\
37.01	0.01\\
38.01	0.01\\
39.01	0.01\\
40.01	0.01\\
41.01	0.01\\
42.01	0.01\\
43.01	0.01\\
44.01	0.01\\
45.01	0.01\\
46.01	0.01\\
47.01	0.01\\
48.01	0.01\\
49.01	0.01\\
50.01	0.01\\
51.01	0.01\\
52.01	0.01\\
53.01	0.01\\
54.01	0.01\\
55.01	0.01\\
56.01	0.01\\
57.01	0.01\\
58.01	0.01\\
59.01	0.01\\
60.01	0.01\\
61.01	0.01\\
62.01	0.01\\
63.01	0.01\\
64.01	0.01\\
65.01	0.01\\
66.01	0.01\\
67.01	0.01\\
68.01	0.01\\
69.01	0.01\\
70.01	0.01\\
71.01	0.01\\
72.01	0.01\\
73.01	0.01\\
74.01	0.01\\
75.01	0.01\\
76.01	0.01\\
77.01	0.01\\
78.01	0.01\\
79.01	0.01\\
80.01	0.01\\
81.01	0.01\\
82.01	0.01\\
83.01	0.01\\
84.01	0.01\\
85.01	0.01\\
86.01	0.01\\
87.01	0.01\\
88.01	0.01\\
89.01	0.01\\
90.01	0.01\\
91.01	0.01\\
92.01	0.01\\
93.01	0.01\\
94.01	0.01\\
95.01	0.01\\
96.01	0.01\\
97.01	0.01\\
98.01	0.01\\
99.01	0.01\\
100.01	0.01\\
101.01	0.01\\
102.01	0.01\\
103.01	0.01\\
104.01	0.01\\
105.01	0.01\\
106.01	0.01\\
107.01	0.01\\
108.01	0.01\\
109.01	0.01\\
110.01	0.01\\
111.01	0.01\\
112.01	0.01\\
113.01	0.01\\
114.01	0.01\\
115.01	0.01\\
116.01	0.01\\
117.01	0.01\\
118.01	0.01\\
119.01	0.01\\
120.01	0.01\\
121.01	0.01\\
122.01	0.01\\
123.01	0.01\\
124.01	0.01\\
125.01	0.01\\
126.01	0.01\\
127.01	0.01\\
128.01	0.01\\
129.01	0.01\\
130.01	0.01\\
131.01	0.01\\
132.01	0.01\\
133.01	0.01\\
134.01	0.01\\
135.01	0.01\\
136.01	0.01\\
137.01	0.01\\
138.01	0.01\\
139.01	0.01\\
140.01	0.01\\
141.01	0.01\\
142.01	0.01\\
143.01	0.01\\
144.01	0.01\\
145.01	0.01\\
146.01	0.01\\
147.01	0.01\\
148.01	0.01\\
149.01	0.01\\
150.01	0.01\\
151.01	0.01\\
152.01	0.01\\
153.01	0.01\\
154.01	0.01\\
155.01	0.01\\
156.01	0.01\\
157.01	0.01\\
158.01	0.01\\
159.01	0.01\\
160.01	0.01\\
161.01	0.01\\
162.01	0.01\\
163.01	0.01\\
164.01	0.01\\
165.01	0.01\\
166.01	0.01\\
167.01	0.01\\
168.01	0.01\\
169.01	0.01\\
170.01	0.01\\
171.01	0.01\\
172.01	0.01\\
173.01	0.01\\
174.01	0.01\\
175.01	0.01\\
176.01	0.01\\
177.01	0.01\\
178.01	0.01\\
179.01	0.01\\
180.01	0.01\\
181.01	0.01\\
182.01	0.01\\
183.01	0.01\\
184.01	0.01\\
185.01	0.01\\
186.01	0.01\\
187.01	0.01\\
188.01	0.01\\
189.01	0.01\\
190.01	0.01\\
191.01	0.01\\
192.01	0.01\\
193.01	0.01\\
194.01	0.01\\
195.01	0.01\\
196.01	0.01\\
197.01	0.01\\
198.01	0.01\\
199.01	0.01\\
200.01	0.01\\
201.01	0.01\\
202.01	0.01\\
203.01	0.01\\
204.01	0.01\\
205.01	0.01\\
206.01	0.01\\
207.01	0.01\\
208.01	0.01\\
209.01	0.01\\
210.01	0.01\\
211.01	0.01\\
212.01	0.01\\
213.01	0.01\\
214.01	0.01\\
215.01	0.01\\
216.01	0.01\\
217.01	0.01\\
218.01	0.01\\
219.01	0.01\\
220.01	0.01\\
221.01	0.01\\
222.01	0.01\\
223.01	0.01\\
224.01	0.01\\
225.01	0.01\\
226.01	0.01\\
227.01	0.01\\
228.01	0.01\\
229.01	0.01\\
230.01	0.01\\
231.01	0.01\\
232.01	0.01\\
233.01	0.01\\
234.01	0.01\\
235.01	0.01\\
236.01	0.01\\
237.01	0.01\\
238.01	0.01\\
239.01	0.01\\
240.01	0.01\\
241.01	0.01\\
242.01	0.01\\
243.01	0.01\\
244.01	0.01\\
245.01	0.01\\
246.01	0.01\\
247.01	0.01\\
248.01	0.01\\
249.01	0.01\\
250.01	0.01\\
251.01	0.01\\
252.01	0.01\\
253.01	0.01\\
254.01	0.01\\
255.01	0.01\\
256.01	0.01\\
257.01	0.01\\
258.01	0.01\\
259.01	0.01\\
260.01	0.01\\
261.01	0.01\\
262.01	0.01\\
263.01	0.01\\
264.01	0.01\\
265.01	0.01\\
266.01	0.01\\
267.01	0.01\\
268.01	0.01\\
269.01	0.01\\
270.01	0.01\\
271.01	0.01\\
272.01	0.01\\
273.01	0.01\\
274.01	0.01\\
275.01	0.01\\
276.01	0.01\\
277.01	0.01\\
278.01	0.01\\
279.01	0.01\\
280.01	0.01\\
281.01	0.01\\
282.01	0.01\\
283.01	0.01\\
284.01	0.01\\
285.01	0.01\\
286.01	0.01\\
287.01	0.01\\
288.01	0.01\\
289.01	0.01\\
290.01	0.01\\
291.01	0.01\\
292.01	0.01\\
293.01	0.01\\
294.01	0.01\\
295.01	0.01\\
296.01	0.01\\
297.01	0.01\\
298.01	0.01\\
299.01	0.01\\
300.01	0.01\\
301.01	0.01\\
302.01	0.01\\
303.01	0.01\\
304.01	0.01\\
305.01	0.01\\
306.01	0.01\\
307.01	0.01\\
308.01	0.01\\
309.01	0.01\\
310.01	0.01\\
311.01	0.01\\
312.01	0.01\\
313.01	0.01\\
314.01	0.01\\
315.01	0.01\\
316.01	0.01\\
317.01	0.01\\
318.01	0.01\\
319.01	0.01\\
320.01	0.01\\
321.01	0.01\\
322.01	0.01\\
323.01	0.01\\
324.01	0.01\\
325.01	0.01\\
326.01	0.01\\
327.01	0.01\\
328.01	0.01\\
329.01	0.01\\
330.01	0.01\\
331.01	0.01\\
332.01	0.01\\
333.01	0.01\\
334.01	0.01\\
335.01	0.01\\
336.01	0.01\\
337.01	0.01\\
338.01	0.01\\
339.01	0.01\\
340.01	0.01\\
341.01	0.01\\
342.01	0.01\\
343.01	0.01\\
344.01	0.01\\
345.01	0.01\\
346.01	0.01\\
347.01	0.01\\
348.01	0.01\\
349.01	0.01\\
350.01	0.01\\
351.01	0.01\\
352.01	0.01\\
353.01	0.01\\
354.01	0.01\\
355.01	0.01\\
356.01	0.01\\
357.01	0.01\\
358.01	0.01\\
359.01	0.01\\
360.01	0.01\\
361.01	0.01\\
362.01	0.01\\
363.01	0.01\\
364.01	0.01\\
365.01	0.01\\
366.01	0.01\\
367.01	0.01\\
368.01	0.01\\
369.01	0.01\\
370.01	0.01\\
371.01	0.01\\
372.01	0.01\\
373.01	0.01\\
374.01	0.01\\
375.01	0.01\\
376.01	0.01\\
377.01	0.01\\
378.01	0.01\\
379.01	0.01\\
380.01	0.01\\
381.01	0.01\\
382.01	0.01\\
383.01	0.01\\
384.01	0.01\\
385.01	0.01\\
386.01	0.01\\
387.01	0.01\\
388.01	0.01\\
389.01	0.01\\
390.01	0.01\\
391.01	0.01\\
392.01	0.01\\
393.01	0.01\\
394.01	0.01\\
395.01	0.01\\
396.01	0.01\\
397.01	0.01\\
398.01	0.01\\
399.01	0.01\\
400.01	0.01\\
401.01	0.01\\
402.01	0.01\\
403.01	0.01\\
404.01	0.01\\
405.01	0.01\\
406.01	0.01\\
407.01	0.01\\
408.01	0.01\\
409.01	0.01\\
410.01	0.01\\
411.01	0.01\\
412.01	0.01\\
413.01	0.01\\
414.01	0.01\\
415.01	0.01\\
416.01	0.01\\
417.01	0.01\\
418.01	0.01\\
419.01	0.01\\
420.01	0.01\\
421.01	0.01\\
422.01	0.01\\
423.01	0.01\\
424.01	0.01\\
425.01	0.01\\
426.01	0.01\\
427.01	0.01\\
428.01	0.01\\
429.01	0.01\\
430.01	0.01\\
431.01	0.01\\
432.01	0.01\\
433.01	0.01\\
434.01	0.01\\
435.01	0.01\\
436.01	0.01\\
437.01	0.01\\
438.01	0.01\\
439.01	0.01\\
440.01	0.01\\
441.01	0.01\\
442.01	0.01\\
443.01	0.01\\
444.01	0.01\\
445.01	0.01\\
446.01	0.01\\
447.01	0.01\\
448.01	0.01\\
449.01	0.01\\
450.01	0.01\\
451.01	0.01\\
452.01	0.01\\
453.01	0.01\\
454.01	0.01\\
455.01	0.01\\
456.01	0.01\\
457.01	0.01\\
458.01	0.01\\
459.01	0.01\\
460.01	0.01\\
461.01	0.01\\
462.01	0.01\\
463.01	0.01\\
464.01	0.01\\
465.01	0.01\\
466.01	0.01\\
467.01	0.01\\
468.01	0.01\\
469.01	0.01\\
470.01	0.01\\
471.01	0.01\\
472.01	0.01\\
473.01	0.01\\
474.01	0.01\\
475.01	0.01\\
476.01	0.01\\
477.01	0.01\\
478.01	0.01\\
479.01	0.01\\
480.01	0.01\\
481.01	0.01\\
482.01	0.01\\
483.01	0.01\\
484.01	0.01\\
485.01	0.01\\
486.01	0.01\\
487.01	0.01\\
488.01	0.01\\
489.01	0.01\\
490.01	0.01\\
491.01	0.01\\
492.01	0.01\\
493.01	0.01\\
494.01	0.01\\
495.01	0.01\\
496.01	0.01\\
497.01	0.01\\
498.01	0.01\\
499.01	0.01\\
500.01	0.01\\
501.01	0.01\\
502.01	0.01\\
503.01	0.01\\
504.01	0.01\\
505.01	0.01\\
506.01	0.01\\
507.01	0.01\\
508.01	0.01\\
509.01	0.01\\
510.01	0.01\\
511.01	0.01\\
512.01	0.01\\
513.01	0.01\\
514.01	0.01\\
515.01	0.01\\
516.01	0.01\\
517.01	0.01\\
518.01	0.01\\
519.01	0.01\\
520.01	0.01\\
521.01	0.01\\
522.01	0.01\\
523.01	0.01\\
524.01	0.01\\
525.01	0.01\\
526.01	0.01\\
527.01	0.01\\
528.01	0.01\\
529.01	0.01\\
530.01	0.01\\
531.01	0.01\\
532.01	0.01\\
533.01	0.01\\
534.01	0.01\\
535.01	0.01\\
536.01	0.01\\
537.01	0.01\\
538.01	0.01\\
539.01	0.01\\
540.01	0.01\\
541.01	0.01\\
542.01	0.01\\
543.01	0.01\\
544.01	0.01\\
545.01	0.01\\
546.01	0.01\\
547.01	0.01\\
548.01	0.01\\
549.01	0.01\\
550.01	0.01\\
551.01	0.01\\
552.01	0.01\\
553.01	0.01\\
554.01	0.01\\
555.01	0.01\\
556.01	0.01\\
557.01	0.01\\
558.01	0.01\\
559.01	0.01\\
560.01	0.01\\
561.01	0.01\\
562.01	0.01\\
563.01	0.01\\
564.01	0.01\\
565.01	0.01\\
566.01	0.01\\
567.01	0.01\\
568.01	0.01\\
569.01	0.01\\
570.01	0.01\\
571.01	0.01\\
572.01	0.01\\
573.01	0.01\\
574.01	0.01\\
575.01	0.01\\
576.01	0.01\\
577.01	0.01\\
578.01	0.01\\
579.01	0.01\\
580.01	0.01\\
581.01	0.01\\
582.01	0.01\\
583.01	0.01\\
584.01	0.01\\
585.01	0.01\\
586.01	0.01\\
587.01	0.01\\
588.01	0.01\\
589.01	0.01\\
590.01	0.01\\
591.01	0.01\\
592.01	0.01\\
593.01	0.01\\
594.01	0.01\\
595.01	0.01\\
596.01	0.01\\
597.01	0.01\\
598.01	0.01\\
599.01	0.01\\
599.02	0.01\\
599.03	0.01\\
599.04	0.01\\
599.05	0.01\\
599.06	0.01\\
599.07	0.01\\
599.08	0.01\\
599.09	0.01\\
599.1	0.01\\
599.11	0.01\\
599.12	0.01\\
599.13	0.01\\
599.14	0.01\\
599.15	0.01\\
599.16	0.01\\
599.17	0.01\\
599.18	0.01\\
599.19	0.01\\
599.2	0.01\\
599.21	0.01\\
599.22	0.01\\
599.23	0.01\\
599.24	0.01\\
599.25	0.01\\
599.26	0.01\\
599.27	0.01\\
599.28	0.01\\
599.29	0.01\\
599.3	0.01\\
599.31	0.01\\
599.32	0.01\\
599.33	0.01\\
599.34	0.01\\
599.35	0.01\\
599.36	0.01\\
599.37	0.01\\
599.38	0.01\\
599.39	0.01\\
599.4	0.01\\
599.41	0.01\\
599.42	0.01\\
599.43	0.01\\
599.44	0.01\\
599.45	0.01\\
599.46	0.01\\
599.47	0.01\\
599.48	0.01\\
599.49	0.01\\
599.5	0.01\\
599.51	0.01\\
599.52	0.01\\
599.53	0.01\\
599.54	0.01\\
599.55	0.01\\
599.56	0.01\\
599.57	0.01\\
599.58	0.01\\
599.59	0.01\\
599.6	0.01\\
599.61	0.01\\
599.62	0.01\\
599.63	0.01\\
599.64	0.01\\
599.65	0.01\\
599.66	0.01\\
599.67	0.01\\
599.68	0.01\\
599.69	0.01\\
599.7	0.01\\
599.71	0.01\\
599.72	0.01\\
599.73	0.01\\
599.74	0.01\\
599.75	0.01\\
599.76	0.01\\
599.77	0.01\\
599.78	0.01\\
599.79	0.01\\
599.8	0.01\\
599.81	0.01\\
599.82	0.01\\
599.83	0.01\\
599.84	0.01\\
599.85	0.01\\
599.86	0.01\\
599.87	0.01\\
599.88	0.01\\
599.89	0.01\\
599.9	0.01\\
599.91	0.01\\
599.92	0.01\\
599.93	0.01\\
599.94	0.01\\
599.95	0.01\\
599.96	0.01\\
599.97	0.01\\
599.98	0.01\\
599.99	0.01\\
600	0.01\\
};
\addplot [color=black!50!mycolor20,solid,forget plot]
  table[row sep=crcr]{%
0.01	0.01\\
1.01	0.01\\
2.01	0.01\\
3.01	0.01\\
4.01	0.01\\
5.01	0.01\\
6.01	0.01\\
7.01	0.01\\
8.01	0.01\\
9.01	0.01\\
10.01	0.01\\
11.01	0.01\\
12.01	0.01\\
13.01	0.01\\
14.01	0.01\\
15.01	0.01\\
16.01	0.01\\
17.01	0.01\\
18.01	0.01\\
19.01	0.01\\
20.01	0.01\\
21.01	0.01\\
22.01	0.01\\
23.01	0.01\\
24.01	0.01\\
25.01	0.01\\
26.01	0.01\\
27.01	0.01\\
28.01	0.01\\
29.01	0.01\\
30.01	0.01\\
31.01	0.01\\
32.01	0.01\\
33.01	0.01\\
34.01	0.01\\
35.01	0.01\\
36.01	0.01\\
37.01	0.01\\
38.01	0.01\\
39.01	0.01\\
40.01	0.01\\
41.01	0.01\\
42.01	0.01\\
43.01	0.01\\
44.01	0.01\\
45.01	0.01\\
46.01	0.01\\
47.01	0.01\\
48.01	0.01\\
49.01	0.01\\
50.01	0.01\\
51.01	0.01\\
52.01	0.01\\
53.01	0.01\\
54.01	0.01\\
55.01	0.01\\
56.01	0.01\\
57.01	0.01\\
58.01	0.01\\
59.01	0.01\\
60.01	0.01\\
61.01	0.01\\
62.01	0.01\\
63.01	0.01\\
64.01	0.01\\
65.01	0.01\\
66.01	0.01\\
67.01	0.01\\
68.01	0.01\\
69.01	0.01\\
70.01	0.01\\
71.01	0.01\\
72.01	0.01\\
73.01	0.01\\
74.01	0.01\\
75.01	0.01\\
76.01	0.01\\
77.01	0.01\\
78.01	0.01\\
79.01	0.01\\
80.01	0.01\\
81.01	0.01\\
82.01	0.01\\
83.01	0.01\\
84.01	0.01\\
85.01	0.01\\
86.01	0.01\\
87.01	0.01\\
88.01	0.01\\
89.01	0.01\\
90.01	0.01\\
91.01	0.01\\
92.01	0.01\\
93.01	0.01\\
94.01	0.01\\
95.01	0.01\\
96.01	0.01\\
97.01	0.01\\
98.01	0.01\\
99.01	0.01\\
100.01	0.01\\
101.01	0.01\\
102.01	0.01\\
103.01	0.01\\
104.01	0.01\\
105.01	0.01\\
106.01	0.01\\
107.01	0.01\\
108.01	0.01\\
109.01	0.01\\
110.01	0.01\\
111.01	0.01\\
112.01	0.01\\
113.01	0.01\\
114.01	0.01\\
115.01	0.01\\
116.01	0.01\\
117.01	0.01\\
118.01	0.01\\
119.01	0.01\\
120.01	0.01\\
121.01	0.01\\
122.01	0.01\\
123.01	0.01\\
124.01	0.01\\
125.01	0.01\\
126.01	0.01\\
127.01	0.01\\
128.01	0.01\\
129.01	0.01\\
130.01	0.01\\
131.01	0.01\\
132.01	0.01\\
133.01	0.01\\
134.01	0.01\\
135.01	0.01\\
136.01	0.01\\
137.01	0.01\\
138.01	0.01\\
139.01	0.01\\
140.01	0.01\\
141.01	0.01\\
142.01	0.01\\
143.01	0.01\\
144.01	0.01\\
145.01	0.01\\
146.01	0.01\\
147.01	0.01\\
148.01	0.01\\
149.01	0.01\\
150.01	0.01\\
151.01	0.01\\
152.01	0.01\\
153.01	0.01\\
154.01	0.01\\
155.01	0.01\\
156.01	0.01\\
157.01	0.01\\
158.01	0.01\\
159.01	0.01\\
160.01	0.01\\
161.01	0.01\\
162.01	0.01\\
163.01	0.01\\
164.01	0.01\\
165.01	0.01\\
166.01	0.01\\
167.01	0.01\\
168.01	0.01\\
169.01	0.01\\
170.01	0.01\\
171.01	0.01\\
172.01	0.01\\
173.01	0.01\\
174.01	0.01\\
175.01	0.01\\
176.01	0.01\\
177.01	0.01\\
178.01	0.01\\
179.01	0.01\\
180.01	0.01\\
181.01	0.01\\
182.01	0.01\\
183.01	0.01\\
184.01	0.01\\
185.01	0.01\\
186.01	0.01\\
187.01	0.01\\
188.01	0.01\\
189.01	0.01\\
190.01	0.01\\
191.01	0.01\\
192.01	0.01\\
193.01	0.01\\
194.01	0.01\\
195.01	0.01\\
196.01	0.01\\
197.01	0.01\\
198.01	0.01\\
199.01	0.01\\
200.01	0.01\\
201.01	0.01\\
202.01	0.01\\
203.01	0.01\\
204.01	0.01\\
205.01	0.01\\
206.01	0.01\\
207.01	0.01\\
208.01	0.01\\
209.01	0.01\\
210.01	0.01\\
211.01	0.01\\
212.01	0.01\\
213.01	0.01\\
214.01	0.01\\
215.01	0.01\\
216.01	0.01\\
217.01	0.01\\
218.01	0.01\\
219.01	0.01\\
220.01	0.01\\
221.01	0.01\\
222.01	0.01\\
223.01	0.01\\
224.01	0.01\\
225.01	0.01\\
226.01	0.01\\
227.01	0.01\\
228.01	0.01\\
229.01	0.01\\
230.01	0.01\\
231.01	0.01\\
232.01	0.01\\
233.01	0.01\\
234.01	0.01\\
235.01	0.01\\
236.01	0.01\\
237.01	0.01\\
238.01	0.01\\
239.01	0.01\\
240.01	0.01\\
241.01	0.01\\
242.01	0.01\\
243.01	0.01\\
244.01	0.01\\
245.01	0.01\\
246.01	0.01\\
247.01	0.01\\
248.01	0.01\\
249.01	0.01\\
250.01	0.01\\
251.01	0.01\\
252.01	0.01\\
253.01	0.01\\
254.01	0.01\\
255.01	0.01\\
256.01	0.01\\
257.01	0.01\\
258.01	0.01\\
259.01	0.01\\
260.01	0.01\\
261.01	0.01\\
262.01	0.01\\
263.01	0.01\\
264.01	0.01\\
265.01	0.01\\
266.01	0.01\\
267.01	0.01\\
268.01	0.01\\
269.01	0.01\\
270.01	0.01\\
271.01	0.01\\
272.01	0.01\\
273.01	0.01\\
274.01	0.01\\
275.01	0.01\\
276.01	0.01\\
277.01	0.01\\
278.01	0.01\\
279.01	0.01\\
280.01	0.01\\
281.01	0.01\\
282.01	0.01\\
283.01	0.01\\
284.01	0.01\\
285.01	0.01\\
286.01	0.01\\
287.01	0.01\\
288.01	0.01\\
289.01	0.01\\
290.01	0.01\\
291.01	0.01\\
292.01	0.01\\
293.01	0.01\\
294.01	0.01\\
295.01	0.01\\
296.01	0.01\\
297.01	0.01\\
298.01	0.01\\
299.01	0.01\\
300.01	0.01\\
301.01	0.01\\
302.01	0.01\\
303.01	0.01\\
304.01	0.01\\
305.01	0.01\\
306.01	0.01\\
307.01	0.01\\
308.01	0.01\\
309.01	0.01\\
310.01	0.01\\
311.01	0.01\\
312.01	0.01\\
313.01	0.01\\
314.01	0.01\\
315.01	0.01\\
316.01	0.01\\
317.01	0.01\\
318.01	0.01\\
319.01	0.01\\
320.01	0.01\\
321.01	0.01\\
322.01	0.01\\
323.01	0.01\\
324.01	0.01\\
325.01	0.01\\
326.01	0.01\\
327.01	0.01\\
328.01	0.01\\
329.01	0.01\\
330.01	0.01\\
331.01	0.01\\
332.01	0.01\\
333.01	0.01\\
334.01	0.01\\
335.01	0.01\\
336.01	0.01\\
337.01	0.01\\
338.01	0.01\\
339.01	0.01\\
340.01	0.01\\
341.01	0.01\\
342.01	0.01\\
343.01	0.01\\
344.01	0.01\\
345.01	0.01\\
346.01	0.01\\
347.01	0.01\\
348.01	0.01\\
349.01	0.01\\
350.01	0.01\\
351.01	0.01\\
352.01	0.01\\
353.01	0.01\\
354.01	0.01\\
355.01	0.01\\
356.01	0.01\\
357.01	0.01\\
358.01	0.01\\
359.01	0.01\\
360.01	0.01\\
361.01	0.01\\
362.01	0.01\\
363.01	0.01\\
364.01	0.01\\
365.01	0.01\\
366.01	0.01\\
367.01	0.01\\
368.01	0.01\\
369.01	0.01\\
370.01	0.01\\
371.01	0.01\\
372.01	0.01\\
373.01	0.01\\
374.01	0.01\\
375.01	0.01\\
376.01	0.01\\
377.01	0.01\\
378.01	0.01\\
379.01	0.01\\
380.01	0.01\\
381.01	0.01\\
382.01	0.01\\
383.01	0.01\\
384.01	0.01\\
385.01	0.01\\
386.01	0.01\\
387.01	0.01\\
388.01	0.01\\
389.01	0.01\\
390.01	0.01\\
391.01	0.01\\
392.01	0.01\\
393.01	0.01\\
394.01	0.01\\
395.01	0.01\\
396.01	0.01\\
397.01	0.01\\
398.01	0.01\\
399.01	0.01\\
400.01	0.01\\
401.01	0.01\\
402.01	0.01\\
403.01	0.01\\
404.01	0.01\\
405.01	0.01\\
406.01	0.01\\
407.01	0.01\\
408.01	0.01\\
409.01	0.01\\
410.01	0.01\\
411.01	0.01\\
412.01	0.01\\
413.01	0.01\\
414.01	0.01\\
415.01	0.01\\
416.01	0.01\\
417.01	0.01\\
418.01	0.01\\
419.01	0.01\\
420.01	0.01\\
421.01	0.01\\
422.01	0.01\\
423.01	0.01\\
424.01	0.01\\
425.01	0.01\\
426.01	0.01\\
427.01	0.01\\
428.01	0.01\\
429.01	0.01\\
430.01	0.01\\
431.01	0.01\\
432.01	0.01\\
433.01	0.01\\
434.01	0.01\\
435.01	0.01\\
436.01	0.01\\
437.01	0.01\\
438.01	0.01\\
439.01	0.01\\
440.01	0.01\\
441.01	0.01\\
442.01	0.01\\
443.01	0.01\\
444.01	0.01\\
445.01	0.01\\
446.01	0.01\\
447.01	0.01\\
448.01	0.01\\
449.01	0.01\\
450.01	0.01\\
451.01	0.01\\
452.01	0.01\\
453.01	0.01\\
454.01	0.01\\
455.01	0.01\\
456.01	0.01\\
457.01	0.01\\
458.01	0.01\\
459.01	0.01\\
460.01	0.01\\
461.01	0.01\\
462.01	0.01\\
463.01	0.01\\
464.01	0.01\\
465.01	0.01\\
466.01	0.01\\
467.01	0.01\\
468.01	0.01\\
469.01	0.01\\
470.01	0.01\\
471.01	0.01\\
472.01	0.01\\
473.01	0.01\\
474.01	0.01\\
475.01	0.01\\
476.01	0.01\\
477.01	0.01\\
478.01	0.01\\
479.01	0.01\\
480.01	0.01\\
481.01	0.01\\
482.01	0.01\\
483.01	0.01\\
484.01	0.01\\
485.01	0.01\\
486.01	0.01\\
487.01	0.01\\
488.01	0.01\\
489.01	0.01\\
490.01	0.01\\
491.01	0.01\\
492.01	0.01\\
493.01	0.01\\
494.01	0.01\\
495.01	0.01\\
496.01	0.01\\
497.01	0.01\\
498.01	0.01\\
499.01	0.01\\
500.01	0.01\\
501.01	0.01\\
502.01	0.01\\
503.01	0.01\\
504.01	0.01\\
505.01	0.01\\
506.01	0.01\\
507.01	0.01\\
508.01	0.01\\
509.01	0.01\\
510.01	0.01\\
511.01	0.01\\
512.01	0.01\\
513.01	0.01\\
514.01	0.01\\
515.01	0.01\\
516.01	0.01\\
517.01	0.01\\
518.01	0.01\\
519.01	0.01\\
520.01	0.01\\
521.01	0.01\\
522.01	0.01\\
523.01	0.01\\
524.01	0.01\\
525.01	0.01\\
526.01	0.01\\
527.01	0.01\\
528.01	0.01\\
529.01	0.01\\
530.01	0.01\\
531.01	0.01\\
532.01	0.01\\
533.01	0.01\\
534.01	0.01\\
535.01	0.01\\
536.01	0.01\\
537.01	0.01\\
538.01	0.01\\
539.01	0.01\\
540.01	0.01\\
541.01	0.01\\
542.01	0.01\\
543.01	0.01\\
544.01	0.01\\
545.01	0.01\\
546.01	0.01\\
547.01	0.01\\
548.01	0.01\\
549.01	0.01\\
550.01	0.01\\
551.01	0.01\\
552.01	0.01\\
553.01	0.01\\
554.01	0.01\\
555.01	0.01\\
556.01	0.01\\
557.01	0.01\\
558.01	0.01\\
559.01	0.01\\
560.01	0.01\\
561.01	0.01\\
562.01	0.01\\
563.01	0.01\\
564.01	0.01\\
565.01	0.01\\
566.01	0.01\\
567.01	0.01\\
568.01	0.01\\
569.01	0.01\\
570.01	0.01\\
571.01	0.01\\
572.01	0.01\\
573.01	0.01\\
574.01	0.01\\
575.01	0.01\\
576.01	0.01\\
577.01	0.01\\
578.01	0.01\\
579.01	0.01\\
580.01	0.01\\
581.01	0.01\\
582.01	0.01\\
583.01	0.01\\
584.01	0.01\\
585.01	0.01\\
586.01	0.01\\
587.01	0.01\\
588.01	0.01\\
589.01	0.01\\
590.01	0.01\\
591.01	0.01\\
592.01	0.01\\
593.01	0.01\\
594.01	0.01\\
595.01	0.01\\
596.01	0.01\\
597.01	0.01\\
598.01	0.01\\
599.01	0.01\\
599.02	0.01\\
599.03	0.01\\
599.04	0.01\\
599.05	0.01\\
599.06	0.01\\
599.07	0.01\\
599.08	0.01\\
599.09	0.01\\
599.1	0.01\\
599.11	0.01\\
599.12	0.01\\
599.13	0.01\\
599.14	0.01\\
599.15	0.01\\
599.16	0.01\\
599.17	0.01\\
599.18	0.01\\
599.19	0.01\\
599.2	0.01\\
599.21	0.01\\
599.22	0.01\\
599.23	0.01\\
599.24	0.01\\
599.25	0.01\\
599.26	0.01\\
599.27	0.01\\
599.28	0.01\\
599.29	0.01\\
599.3	0.01\\
599.31	0.01\\
599.32	0.01\\
599.33	0.01\\
599.34	0.01\\
599.35	0.01\\
599.36	0.01\\
599.37	0.01\\
599.38	0.01\\
599.39	0.01\\
599.4	0.01\\
599.41	0.01\\
599.42	0.01\\
599.43	0.01\\
599.44	0.01\\
599.45	0.01\\
599.46	0.01\\
599.47	0.01\\
599.48	0.01\\
599.49	0.01\\
599.5	0.01\\
599.51	0.01\\
599.52	0.01\\
599.53	0.01\\
599.54	0.01\\
599.55	0.01\\
599.56	0.01\\
599.57	0.01\\
599.58	0.01\\
599.59	0.01\\
599.6	0.01\\
599.61	0.01\\
599.62	0.01\\
599.63	0.01\\
599.64	0.01\\
599.65	0.01\\
599.66	0.01\\
599.67	0.01\\
599.68	0.01\\
599.69	0.01\\
599.7	0.01\\
599.71	0.01\\
599.72	0.01\\
599.73	0.01\\
599.74	0.01\\
599.75	0.01\\
599.76	0.01\\
599.77	0.01\\
599.78	0.01\\
599.79	0.01\\
599.8	0.01\\
599.81	0.01\\
599.82	0.01\\
599.83	0.01\\
599.84	0.01\\
599.85	0.01\\
599.86	0.01\\
599.87	0.01\\
599.88	0.01\\
599.89	0.01\\
599.9	0.01\\
599.91	0.01\\
599.92	0.01\\
599.93	0.01\\
599.94	0.01\\
599.95	0.01\\
599.96	0.01\\
599.97	0.01\\
599.98	0.01\\
599.99	0.01\\
600	0.01\\
};
\addplot [color=black!60!mycolor21,solid,forget plot]
  table[row sep=crcr]{%
0.01	0.01\\
1.01	0.01\\
2.01	0.01\\
3.01	0.01\\
4.01	0.01\\
5.01	0.01\\
6.01	0.01\\
7.01	0.01\\
8.01	0.01\\
9.01	0.01\\
10.01	0.01\\
11.01	0.01\\
12.01	0.01\\
13.01	0.01\\
14.01	0.01\\
15.01	0.01\\
16.01	0.01\\
17.01	0.01\\
18.01	0.01\\
19.01	0.01\\
20.01	0.01\\
21.01	0.01\\
22.01	0.01\\
23.01	0.01\\
24.01	0.01\\
25.01	0.01\\
26.01	0.01\\
27.01	0.01\\
28.01	0.01\\
29.01	0.01\\
30.01	0.01\\
31.01	0.01\\
32.01	0.01\\
33.01	0.01\\
34.01	0.01\\
35.01	0.01\\
36.01	0.01\\
37.01	0.01\\
38.01	0.01\\
39.01	0.01\\
40.01	0.01\\
41.01	0.01\\
42.01	0.01\\
43.01	0.01\\
44.01	0.01\\
45.01	0.01\\
46.01	0.01\\
47.01	0.01\\
48.01	0.01\\
49.01	0.01\\
50.01	0.01\\
51.01	0.01\\
52.01	0.01\\
53.01	0.01\\
54.01	0.01\\
55.01	0.01\\
56.01	0.01\\
57.01	0.01\\
58.01	0.01\\
59.01	0.01\\
60.01	0.01\\
61.01	0.01\\
62.01	0.01\\
63.01	0.01\\
64.01	0.01\\
65.01	0.01\\
66.01	0.01\\
67.01	0.01\\
68.01	0.01\\
69.01	0.01\\
70.01	0.01\\
71.01	0.01\\
72.01	0.01\\
73.01	0.01\\
74.01	0.01\\
75.01	0.01\\
76.01	0.01\\
77.01	0.01\\
78.01	0.01\\
79.01	0.01\\
80.01	0.01\\
81.01	0.01\\
82.01	0.01\\
83.01	0.01\\
84.01	0.01\\
85.01	0.01\\
86.01	0.01\\
87.01	0.01\\
88.01	0.01\\
89.01	0.01\\
90.01	0.01\\
91.01	0.01\\
92.01	0.01\\
93.01	0.01\\
94.01	0.01\\
95.01	0.01\\
96.01	0.01\\
97.01	0.01\\
98.01	0.01\\
99.01	0.01\\
100.01	0.01\\
101.01	0.01\\
102.01	0.01\\
103.01	0.01\\
104.01	0.01\\
105.01	0.01\\
106.01	0.01\\
107.01	0.01\\
108.01	0.01\\
109.01	0.01\\
110.01	0.01\\
111.01	0.01\\
112.01	0.01\\
113.01	0.01\\
114.01	0.01\\
115.01	0.01\\
116.01	0.01\\
117.01	0.01\\
118.01	0.01\\
119.01	0.01\\
120.01	0.01\\
121.01	0.01\\
122.01	0.01\\
123.01	0.01\\
124.01	0.01\\
125.01	0.01\\
126.01	0.01\\
127.01	0.01\\
128.01	0.01\\
129.01	0.01\\
130.01	0.01\\
131.01	0.01\\
132.01	0.01\\
133.01	0.01\\
134.01	0.01\\
135.01	0.01\\
136.01	0.01\\
137.01	0.01\\
138.01	0.01\\
139.01	0.01\\
140.01	0.01\\
141.01	0.01\\
142.01	0.01\\
143.01	0.01\\
144.01	0.01\\
145.01	0.01\\
146.01	0.01\\
147.01	0.01\\
148.01	0.01\\
149.01	0.01\\
150.01	0.01\\
151.01	0.01\\
152.01	0.01\\
153.01	0.01\\
154.01	0.01\\
155.01	0.01\\
156.01	0.01\\
157.01	0.01\\
158.01	0.01\\
159.01	0.01\\
160.01	0.01\\
161.01	0.01\\
162.01	0.01\\
163.01	0.01\\
164.01	0.01\\
165.01	0.01\\
166.01	0.01\\
167.01	0.01\\
168.01	0.01\\
169.01	0.01\\
170.01	0.01\\
171.01	0.01\\
172.01	0.01\\
173.01	0.01\\
174.01	0.01\\
175.01	0.01\\
176.01	0.01\\
177.01	0.01\\
178.01	0.01\\
179.01	0.01\\
180.01	0.01\\
181.01	0.01\\
182.01	0.01\\
183.01	0.01\\
184.01	0.01\\
185.01	0.01\\
186.01	0.01\\
187.01	0.01\\
188.01	0.01\\
189.01	0.01\\
190.01	0.01\\
191.01	0.01\\
192.01	0.01\\
193.01	0.01\\
194.01	0.01\\
195.01	0.01\\
196.01	0.01\\
197.01	0.01\\
198.01	0.01\\
199.01	0.01\\
200.01	0.01\\
201.01	0.01\\
202.01	0.01\\
203.01	0.01\\
204.01	0.01\\
205.01	0.01\\
206.01	0.01\\
207.01	0.01\\
208.01	0.01\\
209.01	0.01\\
210.01	0.01\\
211.01	0.01\\
212.01	0.01\\
213.01	0.01\\
214.01	0.01\\
215.01	0.01\\
216.01	0.01\\
217.01	0.01\\
218.01	0.01\\
219.01	0.01\\
220.01	0.01\\
221.01	0.01\\
222.01	0.01\\
223.01	0.01\\
224.01	0.01\\
225.01	0.01\\
226.01	0.01\\
227.01	0.01\\
228.01	0.01\\
229.01	0.01\\
230.01	0.01\\
231.01	0.01\\
232.01	0.01\\
233.01	0.01\\
234.01	0.01\\
235.01	0.01\\
236.01	0.01\\
237.01	0.01\\
238.01	0.01\\
239.01	0.01\\
240.01	0.01\\
241.01	0.01\\
242.01	0.01\\
243.01	0.01\\
244.01	0.01\\
245.01	0.01\\
246.01	0.01\\
247.01	0.01\\
248.01	0.01\\
249.01	0.01\\
250.01	0.01\\
251.01	0.01\\
252.01	0.01\\
253.01	0.01\\
254.01	0.01\\
255.01	0.01\\
256.01	0.01\\
257.01	0.01\\
258.01	0.01\\
259.01	0.01\\
260.01	0.01\\
261.01	0.01\\
262.01	0.01\\
263.01	0.01\\
264.01	0.01\\
265.01	0.01\\
266.01	0.01\\
267.01	0.01\\
268.01	0.01\\
269.01	0.01\\
270.01	0.01\\
271.01	0.01\\
272.01	0.01\\
273.01	0.01\\
274.01	0.01\\
275.01	0.01\\
276.01	0.01\\
277.01	0.01\\
278.01	0.01\\
279.01	0.01\\
280.01	0.01\\
281.01	0.01\\
282.01	0.01\\
283.01	0.01\\
284.01	0.01\\
285.01	0.01\\
286.01	0.01\\
287.01	0.01\\
288.01	0.01\\
289.01	0.01\\
290.01	0.01\\
291.01	0.01\\
292.01	0.01\\
293.01	0.01\\
294.01	0.01\\
295.01	0.01\\
296.01	0.01\\
297.01	0.01\\
298.01	0.01\\
299.01	0.01\\
300.01	0.01\\
301.01	0.01\\
302.01	0.01\\
303.01	0.01\\
304.01	0.01\\
305.01	0.01\\
306.01	0.01\\
307.01	0.01\\
308.01	0.01\\
309.01	0.01\\
310.01	0.01\\
311.01	0.01\\
312.01	0.01\\
313.01	0.01\\
314.01	0.01\\
315.01	0.01\\
316.01	0.01\\
317.01	0.01\\
318.01	0.01\\
319.01	0.01\\
320.01	0.01\\
321.01	0.01\\
322.01	0.01\\
323.01	0.01\\
324.01	0.01\\
325.01	0.01\\
326.01	0.01\\
327.01	0.01\\
328.01	0.01\\
329.01	0.01\\
330.01	0.01\\
331.01	0.01\\
332.01	0.01\\
333.01	0.01\\
334.01	0.01\\
335.01	0.01\\
336.01	0.01\\
337.01	0.01\\
338.01	0.01\\
339.01	0.01\\
340.01	0.01\\
341.01	0.01\\
342.01	0.01\\
343.01	0.01\\
344.01	0.01\\
345.01	0.01\\
346.01	0.01\\
347.01	0.01\\
348.01	0.01\\
349.01	0.01\\
350.01	0.01\\
351.01	0.01\\
352.01	0.01\\
353.01	0.01\\
354.01	0.01\\
355.01	0.01\\
356.01	0.01\\
357.01	0.01\\
358.01	0.01\\
359.01	0.01\\
360.01	0.01\\
361.01	0.01\\
362.01	0.01\\
363.01	0.01\\
364.01	0.01\\
365.01	0.01\\
366.01	0.01\\
367.01	0.01\\
368.01	0.01\\
369.01	0.01\\
370.01	0.01\\
371.01	0.01\\
372.01	0.01\\
373.01	0.01\\
374.01	0.01\\
375.01	0.01\\
376.01	0.01\\
377.01	0.01\\
378.01	0.01\\
379.01	0.01\\
380.01	0.01\\
381.01	0.01\\
382.01	0.01\\
383.01	0.01\\
384.01	0.01\\
385.01	0.01\\
386.01	0.01\\
387.01	0.01\\
388.01	0.01\\
389.01	0.01\\
390.01	0.01\\
391.01	0.01\\
392.01	0.01\\
393.01	0.01\\
394.01	0.01\\
395.01	0.01\\
396.01	0.01\\
397.01	0.01\\
398.01	0.01\\
399.01	0.01\\
400.01	0.01\\
401.01	0.01\\
402.01	0.01\\
403.01	0.01\\
404.01	0.01\\
405.01	0.01\\
406.01	0.01\\
407.01	0.01\\
408.01	0.01\\
409.01	0.01\\
410.01	0.01\\
411.01	0.01\\
412.01	0.01\\
413.01	0.01\\
414.01	0.01\\
415.01	0.01\\
416.01	0.01\\
417.01	0.01\\
418.01	0.01\\
419.01	0.01\\
420.01	0.01\\
421.01	0.01\\
422.01	0.01\\
423.01	0.01\\
424.01	0.01\\
425.01	0.01\\
426.01	0.01\\
427.01	0.01\\
428.01	0.01\\
429.01	0.01\\
430.01	0.01\\
431.01	0.01\\
432.01	0.01\\
433.01	0.01\\
434.01	0.01\\
435.01	0.01\\
436.01	0.01\\
437.01	0.01\\
438.01	0.01\\
439.01	0.01\\
440.01	0.01\\
441.01	0.01\\
442.01	0.01\\
443.01	0.01\\
444.01	0.01\\
445.01	0.01\\
446.01	0.01\\
447.01	0.01\\
448.01	0.01\\
449.01	0.01\\
450.01	0.01\\
451.01	0.01\\
452.01	0.01\\
453.01	0.01\\
454.01	0.01\\
455.01	0.01\\
456.01	0.01\\
457.01	0.01\\
458.01	0.01\\
459.01	0.01\\
460.01	0.01\\
461.01	0.01\\
462.01	0.01\\
463.01	0.01\\
464.01	0.01\\
465.01	0.01\\
466.01	0.01\\
467.01	0.01\\
468.01	0.01\\
469.01	0.01\\
470.01	0.01\\
471.01	0.01\\
472.01	0.01\\
473.01	0.01\\
474.01	0.01\\
475.01	0.01\\
476.01	0.01\\
477.01	0.01\\
478.01	0.01\\
479.01	0.01\\
480.01	0.01\\
481.01	0.01\\
482.01	0.01\\
483.01	0.01\\
484.01	0.01\\
485.01	0.01\\
486.01	0.01\\
487.01	0.01\\
488.01	0.01\\
489.01	0.01\\
490.01	0.01\\
491.01	0.01\\
492.01	0.01\\
493.01	0.01\\
494.01	0.01\\
495.01	0.01\\
496.01	0.01\\
497.01	0.01\\
498.01	0.01\\
499.01	0.01\\
500.01	0.01\\
501.01	0.01\\
502.01	0.01\\
503.01	0.01\\
504.01	0.01\\
505.01	0.01\\
506.01	0.01\\
507.01	0.01\\
508.01	0.01\\
509.01	0.01\\
510.01	0.01\\
511.01	0.01\\
512.01	0.01\\
513.01	0.01\\
514.01	0.01\\
515.01	0.01\\
516.01	0.01\\
517.01	0.01\\
518.01	0.01\\
519.01	0.01\\
520.01	0.01\\
521.01	0.01\\
522.01	0.01\\
523.01	0.01\\
524.01	0.01\\
525.01	0.01\\
526.01	0.01\\
527.01	0.01\\
528.01	0.01\\
529.01	0.01\\
530.01	0.01\\
531.01	0.01\\
532.01	0.01\\
533.01	0.01\\
534.01	0.01\\
535.01	0.01\\
536.01	0.01\\
537.01	0.01\\
538.01	0.01\\
539.01	0.01\\
540.01	0.01\\
541.01	0.01\\
542.01	0.01\\
543.01	0.01\\
544.01	0.01\\
545.01	0.01\\
546.01	0.01\\
547.01	0.01\\
548.01	0.01\\
549.01	0.01\\
550.01	0.01\\
551.01	0.01\\
552.01	0.01\\
553.01	0.01\\
554.01	0.01\\
555.01	0.01\\
556.01	0.01\\
557.01	0.01\\
558.01	0.01\\
559.01	0.01\\
560.01	0.01\\
561.01	0.01\\
562.01	0.01\\
563.01	0.01\\
564.01	0.01\\
565.01	0.01\\
566.01	0.01\\
567.01	0.01\\
568.01	0.01\\
569.01	0.01\\
570.01	0.01\\
571.01	0.01\\
572.01	0.01\\
573.01	0.01\\
574.01	0.01\\
575.01	0.01\\
576.01	0.01\\
577.01	0.01\\
578.01	0.01\\
579.01	0.01\\
580.01	0.01\\
581.01	0.01\\
582.01	0.01\\
583.01	0.01\\
584.01	0.01\\
585.01	0.01\\
586.01	0.01\\
587.01	0.01\\
588.01	0.01\\
589.01	0.01\\
590.01	0.01\\
591.01	0.01\\
592.01	0.01\\
593.01	0.01\\
594.01	0.01\\
595.01	0.01\\
596.01	0.01\\
597.01	0.01\\
598.01	0.01\\
599.01	0.01\\
599.02	0.01\\
599.03	0.01\\
599.04	0.01\\
599.05	0.01\\
599.06	0.01\\
599.07	0.01\\
599.08	0.01\\
599.09	0.01\\
599.1	0.01\\
599.11	0.01\\
599.12	0.01\\
599.13	0.01\\
599.14	0.01\\
599.15	0.01\\
599.16	0.01\\
599.17	0.01\\
599.18	0.01\\
599.19	0.01\\
599.2	0.01\\
599.21	0.01\\
599.22	0.01\\
599.23	0.01\\
599.24	0.01\\
599.25	0.01\\
599.26	0.01\\
599.27	0.01\\
599.28	0.01\\
599.29	0.01\\
599.3	0.01\\
599.31	0.01\\
599.32	0.01\\
599.33	0.01\\
599.34	0.01\\
599.35	0.01\\
599.36	0.01\\
599.37	0.01\\
599.38	0.01\\
599.39	0.01\\
599.4	0.01\\
599.41	0.01\\
599.42	0.01\\
599.43	0.01\\
599.44	0.01\\
599.45	0.01\\
599.46	0.01\\
599.47	0.01\\
599.48	0.01\\
599.49	0.01\\
599.5	0.01\\
599.51	0.01\\
599.52	0.01\\
599.53	0.01\\
599.54	0.01\\
599.55	0.01\\
599.56	0.01\\
599.57	0.01\\
599.58	0.01\\
599.59	0.01\\
599.6	0.01\\
599.61	0.01\\
599.62	0.01\\
599.63	0.01\\
599.64	0.01\\
599.65	0.01\\
599.66	0.01\\
599.67	0.01\\
599.68	0.01\\
599.69	0.01\\
599.7	0.01\\
599.71	0.01\\
599.72	0.01\\
599.73	0.01\\
599.74	0.01\\
599.75	0.01\\
599.76	0.01\\
599.77	0.01\\
599.78	0.01\\
599.79	0.01\\
599.8	0.01\\
599.81	0.01\\
599.82	0.01\\
599.83	0.01\\
599.84	0.01\\
599.85	0.01\\
599.86	0.01\\
599.87	0.01\\
599.88	0.01\\
599.89	0.01\\
599.9	0.01\\
599.91	0.01\\
599.92	0.01\\
599.93	0.01\\
599.94	0.01\\
599.95	0.01\\
599.96	0.01\\
599.97	0.01\\
599.98	0.01\\
599.99	0.01\\
600	0.01\\
};
\addplot [color=black!80!mycolor21,solid,forget plot]
  table[row sep=crcr]{%
0.01	0.01\\
1.01	0.01\\
2.01	0.01\\
3.01	0.01\\
4.01	0.01\\
5.01	0.01\\
6.01	0.01\\
7.01	0.01\\
8.01	0.01\\
9.01	0.01\\
10.01	0.01\\
11.01	0.01\\
12.01	0.01\\
13.01	0.01\\
14.01	0.01\\
15.01	0.01\\
16.01	0.01\\
17.01	0.01\\
18.01	0.01\\
19.01	0.01\\
20.01	0.01\\
21.01	0.01\\
22.01	0.01\\
23.01	0.01\\
24.01	0.01\\
25.01	0.01\\
26.01	0.01\\
27.01	0.01\\
28.01	0.01\\
29.01	0.01\\
30.01	0.01\\
31.01	0.01\\
32.01	0.01\\
33.01	0.01\\
34.01	0.01\\
35.01	0.01\\
36.01	0.01\\
37.01	0.01\\
38.01	0.01\\
39.01	0.01\\
40.01	0.01\\
41.01	0.01\\
42.01	0.01\\
43.01	0.01\\
44.01	0.01\\
45.01	0.01\\
46.01	0.01\\
47.01	0.01\\
48.01	0.01\\
49.01	0.01\\
50.01	0.01\\
51.01	0.01\\
52.01	0.01\\
53.01	0.01\\
54.01	0.01\\
55.01	0.01\\
56.01	0.01\\
57.01	0.01\\
58.01	0.01\\
59.01	0.01\\
60.01	0.01\\
61.01	0.01\\
62.01	0.01\\
63.01	0.01\\
64.01	0.01\\
65.01	0.01\\
66.01	0.01\\
67.01	0.01\\
68.01	0.01\\
69.01	0.01\\
70.01	0.01\\
71.01	0.01\\
72.01	0.01\\
73.01	0.01\\
74.01	0.01\\
75.01	0.01\\
76.01	0.01\\
77.01	0.01\\
78.01	0.01\\
79.01	0.01\\
80.01	0.01\\
81.01	0.01\\
82.01	0.01\\
83.01	0.01\\
84.01	0.01\\
85.01	0.01\\
86.01	0.01\\
87.01	0.01\\
88.01	0.01\\
89.01	0.01\\
90.01	0.01\\
91.01	0.01\\
92.01	0.01\\
93.01	0.01\\
94.01	0.01\\
95.01	0.01\\
96.01	0.01\\
97.01	0.01\\
98.01	0.01\\
99.01	0.01\\
100.01	0.01\\
101.01	0.01\\
102.01	0.01\\
103.01	0.01\\
104.01	0.01\\
105.01	0.01\\
106.01	0.01\\
107.01	0.01\\
108.01	0.01\\
109.01	0.01\\
110.01	0.01\\
111.01	0.01\\
112.01	0.01\\
113.01	0.01\\
114.01	0.01\\
115.01	0.01\\
116.01	0.01\\
117.01	0.01\\
118.01	0.01\\
119.01	0.01\\
120.01	0.01\\
121.01	0.01\\
122.01	0.01\\
123.01	0.01\\
124.01	0.01\\
125.01	0.01\\
126.01	0.01\\
127.01	0.01\\
128.01	0.01\\
129.01	0.01\\
130.01	0.01\\
131.01	0.01\\
132.01	0.01\\
133.01	0.01\\
134.01	0.01\\
135.01	0.01\\
136.01	0.01\\
137.01	0.01\\
138.01	0.01\\
139.01	0.01\\
140.01	0.01\\
141.01	0.01\\
142.01	0.01\\
143.01	0.01\\
144.01	0.01\\
145.01	0.01\\
146.01	0.01\\
147.01	0.01\\
148.01	0.01\\
149.01	0.01\\
150.01	0.01\\
151.01	0.01\\
152.01	0.01\\
153.01	0.01\\
154.01	0.01\\
155.01	0.01\\
156.01	0.01\\
157.01	0.01\\
158.01	0.01\\
159.01	0.01\\
160.01	0.01\\
161.01	0.01\\
162.01	0.01\\
163.01	0.01\\
164.01	0.01\\
165.01	0.01\\
166.01	0.01\\
167.01	0.01\\
168.01	0.01\\
169.01	0.01\\
170.01	0.01\\
171.01	0.01\\
172.01	0.01\\
173.01	0.01\\
174.01	0.01\\
175.01	0.01\\
176.01	0.01\\
177.01	0.01\\
178.01	0.01\\
179.01	0.01\\
180.01	0.01\\
181.01	0.01\\
182.01	0.01\\
183.01	0.01\\
184.01	0.01\\
185.01	0.01\\
186.01	0.01\\
187.01	0.01\\
188.01	0.01\\
189.01	0.01\\
190.01	0.01\\
191.01	0.01\\
192.01	0.01\\
193.01	0.01\\
194.01	0.01\\
195.01	0.01\\
196.01	0.01\\
197.01	0.01\\
198.01	0.01\\
199.01	0.01\\
200.01	0.01\\
201.01	0.01\\
202.01	0.01\\
203.01	0.01\\
204.01	0.01\\
205.01	0.01\\
206.01	0.01\\
207.01	0.01\\
208.01	0.01\\
209.01	0.01\\
210.01	0.01\\
211.01	0.01\\
212.01	0.01\\
213.01	0.01\\
214.01	0.01\\
215.01	0.01\\
216.01	0.01\\
217.01	0.01\\
218.01	0.01\\
219.01	0.01\\
220.01	0.01\\
221.01	0.01\\
222.01	0.01\\
223.01	0.01\\
224.01	0.01\\
225.01	0.01\\
226.01	0.01\\
227.01	0.01\\
228.01	0.01\\
229.01	0.01\\
230.01	0.01\\
231.01	0.01\\
232.01	0.01\\
233.01	0.01\\
234.01	0.01\\
235.01	0.01\\
236.01	0.01\\
237.01	0.01\\
238.01	0.01\\
239.01	0.01\\
240.01	0.01\\
241.01	0.01\\
242.01	0.01\\
243.01	0.01\\
244.01	0.01\\
245.01	0.01\\
246.01	0.01\\
247.01	0.01\\
248.01	0.01\\
249.01	0.01\\
250.01	0.01\\
251.01	0.01\\
252.01	0.01\\
253.01	0.01\\
254.01	0.01\\
255.01	0.01\\
256.01	0.01\\
257.01	0.01\\
258.01	0.01\\
259.01	0.01\\
260.01	0.01\\
261.01	0.01\\
262.01	0.01\\
263.01	0.01\\
264.01	0.01\\
265.01	0.01\\
266.01	0.01\\
267.01	0.01\\
268.01	0.01\\
269.01	0.01\\
270.01	0.01\\
271.01	0.01\\
272.01	0.01\\
273.01	0.01\\
274.01	0.01\\
275.01	0.01\\
276.01	0.01\\
277.01	0.01\\
278.01	0.01\\
279.01	0.01\\
280.01	0.01\\
281.01	0.01\\
282.01	0.01\\
283.01	0.01\\
284.01	0.01\\
285.01	0.01\\
286.01	0.01\\
287.01	0.01\\
288.01	0.01\\
289.01	0.01\\
290.01	0.01\\
291.01	0.01\\
292.01	0.01\\
293.01	0.01\\
294.01	0.01\\
295.01	0.01\\
296.01	0.01\\
297.01	0.01\\
298.01	0.01\\
299.01	0.01\\
300.01	0.01\\
301.01	0.01\\
302.01	0.01\\
303.01	0.01\\
304.01	0.01\\
305.01	0.01\\
306.01	0.01\\
307.01	0.01\\
308.01	0.01\\
309.01	0.01\\
310.01	0.01\\
311.01	0.01\\
312.01	0.01\\
313.01	0.01\\
314.01	0.01\\
315.01	0.01\\
316.01	0.01\\
317.01	0.01\\
318.01	0.01\\
319.01	0.01\\
320.01	0.01\\
321.01	0.01\\
322.01	0.01\\
323.01	0.01\\
324.01	0.01\\
325.01	0.01\\
326.01	0.01\\
327.01	0.01\\
328.01	0.01\\
329.01	0.01\\
330.01	0.01\\
331.01	0.01\\
332.01	0.01\\
333.01	0.01\\
334.01	0.01\\
335.01	0.01\\
336.01	0.01\\
337.01	0.01\\
338.01	0.01\\
339.01	0.01\\
340.01	0.01\\
341.01	0.01\\
342.01	0.01\\
343.01	0.01\\
344.01	0.01\\
345.01	0.01\\
346.01	0.01\\
347.01	0.01\\
348.01	0.01\\
349.01	0.01\\
350.01	0.01\\
351.01	0.01\\
352.01	0.01\\
353.01	0.01\\
354.01	0.01\\
355.01	0.01\\
356.01	0.01\\
357.01	0.01\\
358.01	0.01\\
359.01	0.01\\
360.01	0.01\\
361.01	0.01\\
362.01	0.01\\
363.01	0.01\\
364.01	0.01\\
365.01	0.01\\
366.01	0.01\\
367.01	0.01\\
368.01	0.01\\
369.01	0.01\\
370.01	0.01\\
371.01	0.01\\
372.01	0.01\\
373.01	0.01\\
374.01	0.01\\
375.01	0.01\\
376.01	0.01\\
377.01	0.01\\
378.01	0.01\\
379.01	0.01\\
380.01	0.01\\
381.01	0.01\\
382.01	0.01\\
383.01	0.01\\
384.01	0.01\\
385.01	0.01\\
386.01	0.01\\
387.01	0.01\\
388.01	0.01\\
389.01	0.01\\
390.01	0.01\\
391.01	0.01\\
392.01	0.01\\
393.01	0.01\\
394.01	0.01\\
395.01	0.01\\
396.01	0.01\\
397.01	0.01\\
398.01	0.01\\
399.01	0.01\\
400.01	0.01\\
401.01	0.01\\
402.01	0.01\\
403.01	0.01\\
404.01	0.01\\
405.01	0.01\\
406.01	0.01\\
407.01	0.01\\
408.01	0.01\\
409.01	0.01\\
410.01	0.01\\
411.01	0.01\\
412.01	0.01\\
413.01	0.01\\
414.01	0.01\\
415.01	0.01\\
416.01	0.01\\
417.01	0.01\\
418.01	0.01\\
419.01	0.01\\
420.01	0.01\\
421.01	0.01\\
422.01	0.01\\
423.01	0.01\\
424.01	0.01\\
425.01	0.01\\
426.01	0.01\\
427.01	0.01\\
428.01	0.01\\
429.01	0.01\\
430.01	0.01\\
431.01	0.01\\
432.01	0.01\\
433.01	0.01\\
434.01	0.01\\
435.01	0.01\\
436.01	0.01\\
437.01	0.01\\
438.01	0.01\\
439.01	0.01\\
440.01	0.01\\
441.01	0.01\\
442.01	0.01\\
443.01	0.01\\
444.01	0.01\\
445.01	0.01\\
446.01	0.01\\
447.01	0.01\\
448.01	0.01\\
449.01	0.01\\
450.01	0.01\\
451.01	0.01\\
452.01	0.01\\
453.01	0.01\\
454.01	0.01\\
455.01	0.01\\
456.01	0.01\\
457.01	0.01\\
458.01	0.01\\
459.01	0.01\\
460.01	0.01\\
461.01	0.01\\
462.01	0.01\\
463.01	0.01\\
464.01	0.01\\
465.01	0.01\\
466.01	0.01\\
467.01	0.01\\
468.01	0.01\\
469.01	0.01\\
470.01	0.01\\
471.01	0.01\\
472.01	0.01\\
473.01	0.01\\
474.01	0.01\\
475.01	0.01\\
476.01	0.01\\
477.01	0.01\\
478.01	0.01\\
479.01	0.01\\
480.01	0.01\\
481.01	0.01\\
482.01	0.01\\
483.01	0.01\\
484.01	0.01\\
485.01	0.01\\
486.01	0.01\\
487.01	0.01\\
488.01	0.01\\
489.01	0.01\\
490.01	0.01\\
491.01	0.01\\
492.01	0.01\\
493.01	0.01\\
494.01	0.01\\
495.01	0.01\\
496.01	0.01\\
497.01	0.01\\
498.01	0.01\\
499.01	0.01\\
500.01	0.01\\
501.01	0.01\\
502.01	0.01\\
503.01	0.01\\
504.01	0.01\\
505.01	0.01\\
506.01	0.01\\
507.01	0.01\\
508.01	0.01\\
509.01	0.01\\
510.01	0.01\\
511.01	0.01\\
512.01	0.01\\
513.01	0.01\\
514.01	0.01\\
515.01	0.01\\
516.01	0.01\\
517.01	0.01\\
518.01	0.01\\
519.01	0.01\\
520.01	0.01\\
521.01	0.01\\
522.01	0.01\\
523.01	0.01\\
524.01	0.01\\
525.01	0.01\\
526.01	0.01\\
527.01	0.01\\
528.01	0.01\\
529.01	0.01\\
530.01	0.01\\
531.01	0.01\\
532.01	0.01\\
533.01	0.01\\
534.01	0.01\\
535.01	0.01\\
536.01	0.01\\
537.01	0.01\\
538.01	0.01\\
539.01	0.01\\
540.01	0.01\\
541.01	0.01\\
542.01	0.01\\
543.01	0.01\\
544.01	0.01\\
545.01	0.01\\
546.01	0.01\\
547.01	0.01\\
548.01	0.01\\
549.01	0.01\\
550.01	0.01\\
551.01	0.01\\
552.01	0.01\\
553.01	0.01\\
554.01	0.01\\
555.01	0.01\\
556.01	0.01\\
557.01	0.01\\
558.01	0.01\\
559.01	0.01\\
560.01	0.01\\
561.01	0.01\\
562.01	0.01\\
563.01	0.01\\
564.01	0.01\\
565.01	0.01\\
566.01	0.01\\
567.01	0.01\\
568.01	0.01\\
569.01	0.01\\
570.01	0.01\\
571.01	0.01\\
572.01	0.01\\
573.01	0.01\\
574.01	0.01\\
575.01	0.01\\
576.01	0.01\\
577.01	0.01\\
578.01	0.01\\
579.01	0.01\\
580.01	0.01\\
581.01	0.01\\
582.01	0.01\\
583.01	0.01\\
584.01	0.01\\
585.01	0.01\\
586.01	0.01\\
587.01	0.01\\
588.01	0.01\\
589.01	0.01\\
590.01	0.01\\
591.01	0.01\\
592.01	0.01\\
593.01	0.01\\
594.01	0.01\\
595.01	0.01\\
596.01	0.01\\
597.01	0.01\\
598.01	0.01\\
599.01	0.01\\
599.02	0.01\\
599.03	0.01\\
599.04	0.01\\
599.05	0.01\\
599.06	0.01\\
599.07	0.01\\
599.08	0.01\\
599.09	0.01\\
599.1	0.01\\
599.11	0.01\\
599.12	0.01\\
599.13	0.01\\
599.14	0.01\\
599.15	0.01\\
599.16	0.01\\
599.17	0.01\\
599.18	0.01\\
599.19	0.01\\
599.2	0.01\\
599.21	0.01\\
599.22	0.01\\
599.23	0.01\\
599.24	0.01\\
599.25	0.01\\
599.26	0.01\\
599.27	0.01\\
599.28	0.01\\
599.29	0.01\\
599.3	0.01\\
599.31	0.01\\
599.32	0.01\\
599.33	0.01\\
599.34	0.01\\
599.35	0.01\\
599.36	0.01\\
599.37	0.01\\
599.38	0.01\\
599.39	0.01\\
599.4	0.01\\
599.41	0.01\\
599.42	0.01\\
599.43	0.01\\
599.44	0.01\\
599.45	0.01\\
599.46	0.01\\
599.47	0.01\\
599.48	0.01\\
599.49	0.01\\
599.5	0.01\\
599.51	0.01\\
599.52	0.01\\
599.53	0.01\\
599.54	0.01\\
599.55	0.01\\
599.56	0.01\\
599.57	0.01\\
599.58	0.01\\
599.59	0.01\\
599.6	0.01\\
599.61	0.01\\
599.62	0.01\\
599.63	0.01\\
599.64	0.01\\
599.65	0.01\\
599.66	0.01\\
599.67	0.01\\
599.68	0.01\\
599.69	0.01\\
599.7	0.01\\
599.71	0.01\\
599.72	0.01\\
599.73	0.01\\
599.74	0.01\\
599.75	0.01\\
599.76	0.01\\
599.77	0.01\\
599.78	0.01\\
599.79	0.01\\
599.8	0.01\\
599.81	0.01\\
599.82	0.01\\
599.83	0.01\\
599.84	0.01\\
599.85	0.01\\
599.86	0.01\\
599.87	0.01\\
599.88	0.01\\
599.89	0.01\\
599.9	0.01\\
599.91	0.01\\
599.92	0.01\\
599.93	0.01\\
599.94	0.01\\
599.95	0.01\\
599.96	0.01\\
599.97	0.01\\
599.98	0.01\\
599.99	0.01\\
600	0.01\\
};
\addplot [color=black,solid,forget plot]
  table[row sep=crcr]{%
0.01	0.01\\
1.01	0.01\\
2.01	0.01\\
3.01	0.01\\
4.01	0.01\\
5.01	0.01\\
6.01	0.01\\
7.01	0.01\\
8.01	0.01\\
9.01	0.01\\
10.01	0.01\\
11.01	0.01\\
12.01	0.01\\
13.01	0.01\\
14.01	0.01\\
15.01	0.01\\
16.01	0.01\\
17.01	0.01\\
18.01	0.01\\
19.01	0.01\\
20.01	0.01\\
21.01	0.01\\
22.01	0.01\\
23.01	0.01\\
24.01	0.01\\
25.01	0.01\\
26.01	0.01\\
27.01	0.01\\
28.01	0.01\\
29.01	0.01\\
30.01	0.01\\
31.01	0.01\\
32.01	0.01\\
33.01	0.01\\
34.01	0.01\\
35.01	0.01\\
36.01	0.01\\
37.01	0.01\\
38.01	0.01\\
39.01	0.01\\
40.01	0.01\\
41.01	0.01\\
42.01	0.01\\
43.01	0.01\\
44.01	0.01\\
45.01	0.01\\
46.01	0.01\\
47.01	0.01\\
48.01	0.01\\
49.01	0.01\\
50.01	0.01\\
51.01	0.01\\
52.01	0.01\\
53.01	0.01\\
54.01	0.01\\
55.01	0.01\\
56.01	0.01\\
57.01	0.01\\
58.01	0.01\\
59.01	0.01\\
60.01	0.01\\
61.01	0.01\\
62.01	0.01\\
63.01	0.01\\
64.01	0.01\\
65.01	0.01\\
66.01	0.01\\
67.01	0.01\\
68.01	0.01\\
69.01	0.01\\
70.01	0.01\\
71.01	0.01\\
72.01	0.01\\
73.01	0.01\\
74.01	0.01\\
75.01	0.01\\
76.01	0.01\\
77.01	0.01\\
78.01	0.01\\
79.01	0.01\\
80.01	0.01\\
81.01	0.01\\
82.01	0.01\\
83.01	0.01\\
84.01	0.01\\
85.01	0.01\\
86.01	0.01\\
87.01	0.01\\
88.01	0.01\\
89.01	0.01\\
90.01	0.01\\
91.01	0.01\\
92.01	0.01\\
93.01	0.01\\
94.01	0.01\\
95.01	0.01\\
96.01	0.01\\
97.01	0.01\\
98.01	0.01\\
99.01	0.01\\
100.01	0.01\\
101.01	0.01\\
102.01	0.01\\
103.01	0.01\\
104.01	0.01\\
105.01	0.01\\
106.01	0.01\\
107.01	0.01\\
108.01	0.01\\
109.01	0.01\\
110.01	0.01\\
111.01	0.01\\
112.01	0.01\\
113.01	0.01\\
114.01	0.01\\
115.01	0.01\\
116.01	0.01\\
117.01	0.01\\
118.01	0.01\\
119.01	0.01\\
120.01	0.01\\
121.01	0.01\\
122.01	0.01\\
123.01	0.01\\
124.01	0.01\\
125.01	0.01\\
126.01	0.01\\
127.01	0.01\\
128.01	0.01\\
129.01	0.01\\
130.01	0.01\\
131.01	0.01\\
132.01	0.01\\
133.01	0.01\\
134.01	0.01\\
135.01	0.01\\
136.01	0.01\\
137.01	0.01\\
138.01	0.01\\
139.01	0.01\\
140.01	0.01\\
141.01	0.01\\
142.01	0.01\\
143.01	0.01\\
144.01	0.01\\
145.01	0.01\\
146.01	0.01\\
147.01	0.01\\
148.01	0.01\\
149.01	0.01\\
150.01	0.01\\
151.01	0.01\\
152.01	0.01\\
153.01	0.01\\
154.01	0.01\\
155.01	0.01\\
156.01	0.01\\
157.01	0.01\\
158.01	0.01\\
159.01	0.01\\
160.01	0.01\\
161.01	0.01\\
162.01	0.01\\
163.01	0.01\\
164.01	0.01\\
165.01	0.01\\
166.01	0.01\\
167.01	0.01\\
168.01	0.01\\
169.01	0.01\\
170.01	0.01\\
171.01	0.01\\
172.01	0.01\\
173.01	0.01\\
174.01	0.01\\
175.01	0.01\\
176.01	0.01\\
177.01	0.01\\
178.01	0.01\\
179.01	0.01\\
180.01	0.01\\
181.01	0.01\\
182.01	0.01\\
183.01	0.01\\
184.01	0.01\\
185.01	0.01\\
186.01	0.01\\
187.01	0.01\\
188.01	0.01\\
189.01	0.01\\
190.01	0.01\\
191.01	0.01\\
192.01	0.01\\
193.01	0.01\\
194.01	0.01\\
195.01	0.01\\
196.01	0.01\\
197.01	0.01\\
198.01	0.01\\
199.01	0.01\\
200.01	0.01\\
201.01	0.01\\
202.01	0.01\\
203.01	0.01\\
204.01	0.01\\
205.01	0.01\\
206.01	0.01\\
207.01	0.01\\
208.01	0.01\\
209.01	0.01\\
210.01	0.01\\
211.01	0.01\\
212.01	0.01\\
213.01	0.01\\
214.01	0.01\\
215.01	0.01\\
216.01	0.01\\
217.01	0.01\\
218.01	0.01\\
219.01	0.01\\
220.01	0.01\\
221.01	0.01\\
222.01	0.01\\
223.01	0.01\\
224.01	0.01\\
225.01	0.01\\
226.01	0.01\\
227.01	0.01\\
228.01	0.01\\
229.01	0.01\\
230.01	0.01\\
231.01	0.01\\
232.01	0.01\\
233.01	0.01\\
234.01	0.01\\
235.01	0.01\\
236.01	0.01\\
237.01	0.01\\
238.01	0.01\\
239.01	0.01\\
240.01	0.01\\
241.01	0.01\\
242.01	0.01\\
243.01	0.01\\
244.01	0.01\\
245.01	0.01\\
246.01	0.01\\
247.01	0.01\\
248.01	0.01\\
249.01	0.01\\
250.01	0.01\\
251.01	0.01\\
252.01	0.01\\
253.01	0.01\\
254.01	0.01\\
255.01	0.01\\
256.01	0.01\\
257.01	0.01\\
258.01	0.01\\
259.01	0.01\\
260.01	0.01\\
261.01	0.01\\
262.01	0.01\\
263.01	0.01\\
264.01	0.01\\
265.01	0.01\\
266.01	0.01\\
267.01	0.01\\
268.01	0.01\\
269.01	0.01\\
270.01	0.01\\
271.01	0.01\\
272.01	0.01\\
273.01	0.01\\
274.01	0.01\\
275.01	0.01\\
276.01	0.01\\
277.01	0.01\\
278.01	0.01\\
279.01	0.01\\
280.01	0.01\\
281.01	0.01\\
282.01	0.01\\
283.01	0.01\\
284.01	0.01\\
285.01	0.01\\
286.01	0.01\\
287.01	0.01\\
288.01	0.01\\
289.01	0.01\\
290.01	0.01\\
291.01	0.01\\
292.01	0.01\\
293.01	0.01\\
294.01	0.01\\
295.01	0.01\\
296.01	0.01\\
297.01	0.01\\
298.01	0.01\\
299.01	0.01\\
300.01	0.01\\
301.01	0.01\\
302.01	0.01\\
303.01	0.01\\
304.01	0.01\\
305.01	0.01\\
306.01	0.01\\
307.01	0.01\\
308.01	0.01\\
309.01	0.01\\
310.01	0.01\\
311.01	0.01\\
312.01	0.01\\
313.01	0.01\\
314.01	0.01\\
315.01	0.01\\
316.01	0.01\\
317.01	0.01\\
318.01	0.01\\
319.01	0.01\\
320.01	0.01\\
321.01	0.01\\
322.01	0.01\\
323.01	0.01\\
324.01	0.01\\
325.01	0.01\\
326.01	0.01\\
327.01	0.01\\
328.01	0.01\\
329.01	0.01\\
330.01	0.01\\
331.01	0.01\\
332.01	0.01\\
333.01	0.01\\
334.01	0.01\\
335.01	0.01\\
336.01	0.01\\
337.01	0.01\\
338.01	0.01\\
339.01	0.01\\
340.01	0.01\\
341.01	0.01\\
342.01	0.01\\
343.01	0.01\\
344.01	0.01\\
345.01	0.01\\
346.01	0.01\\
347.01	0.01\\
348.01	0.01\\
349.01	0.01\\
350.01	0.01\\
351.01	0.01\\
352.01	0.01\\
353.01	0.01\\
354.01	0.01\\
355.01	0.01\\
356.01	0.01\\
357.01	0.01\\
358.01	0.01\\
359.01	0.01\\
360.01	0.01\\
361.01	0.01\\
362.01	0.01\\
363.01	0.01\\
364.01	0.01\\
365.01	0.01\\
366.01	0.01\\
367.01	0.01\\
368.01	0.01\\
369.01	0.01\\
370.01	0.01\\
371.01	0.01\\
372.01	0.01\\
373.01	0.01\\
374.01	0.01\\
375.01	0.01\\
376.01	0.01\\
377.01	0.01\\
378.01	0.01\\
379.01	0.01\\
380.01	0.01\\
381.01	0.01\\
382.01	0.01\\
383.01	0.01\\
384.01	0.01\\
385.01	0.01\\
386.01	0.01\\
387.01	0.01\\
388.01	0.01\\
389.01	0.01\\
390.01	0.01\\
391.01	0.01\\
392.01	0.01\\
393.01	0.01\\
394.01	0.01\\
395.01	0.01\\
396.01	0.01\\
397.01	0.01\\
398.01	0.01\\
399.01	0.01\\
400.01	0.01\\
401.01	0.01\\
402.01	0.01\\
403.01	0.01\\
404.01	0.01\\
405.01	0.01\\
406.01	0.01\\
407.01	0.01\\
408.01	0.01\\
409.01	0.01\\
410.01	0.01\\
411.01	0.01\\
412.01	0.01\\
413.01	0.01\\
414.01	0.01\\
415.01	0.01\\
416.01	0.01\\
417.01	0.01\\
418.01	0.01\\
419.01	0.01\\
420.01	0.01\\
421.01	0.01\\
422.01	0.01\\
423.01	0.01\\
424.01	0.01\\
425.01	0.01\\
426.01	0.01\\
427.01	0.01\\
428.01	0.01\\
429.01	0.01\\
430.01	0.01\\
431.01	0.01\\
432.01	0.01\\
433.01	0.01\\
434.01	0.01\\
435.01	0.01\\
436.01	0.01\\
437.01	0.01\\
438.01	0.01\\
439.01	0.01\\
440.01	0.01\\
441.01	0.01\\
442.01	0.01\\
443.01	0.01\\
444.01	0.01\\
445.01	0.01\\
446.01	0.01\\
447.01	0.01\\
448.01	0.01\\
449.01	0.01\\
450.01	0.01\\
451.01	0.01\\
452.01	0.01\\
453.01	0.01\\
454.01	0.01\\
455.01	0.01\\
456.01	0.01\\
457.01	0.01\\
458.01	0.01\\
459.01	0.01\\
460.01	0.01\\
461.01	0.01\\
462.01	0.01\\
463.01	0.01\\
464.01	0.01\\
465.01	0.01\\
466.01	0.01\\
467.01	0.01\\
468.01	0.01\\
469.01	0.01\\
470.01	0.01\\
471.01	0.01\\
472.01	0.01\\
473.01	0.01\\
474.01	0.01\\
475.01	0.01\\
476.01	0.01\\
477.01	0.01\\
478.01	0.01\\
479.01	0.01\\
480.01	0.01\\
481.01	0.01\\
482.01	0.01\\
483.01	0.01\\
484.01	0.01\\
485.01	0.01\\
486.01	0.01\\
487.01	0.01\\
488.01	0.01\\
489.01	0.01\\
490.01	0.01\\
491.01	0.01\\
492.01	0.01\\
493.01	0.01\\
494.01	0.01\\
495.01	0.01\\
496.01	0.01\\
497.01	0.01\\
498.01	0.01\\
499.01	0.01\\
500.01	0.01\\
501.01	0.01\\
502.01	0.01\\
503.01	0.01\\
504.01	0.01\\
505.01	0.01\\
506.01	0.01\\
507.01	0.01\\
508.01	0.01\\
509.01	0.01\\
510.01	0.01\\
511.01	0.01\\
512.01	0.01\\
513.01	0.01\\
514.01	0.01\\
515.01	0.01\\
516.01	0.01\\
517.01	0.01\\
518.01	0.01\\
519.01	0.01\\
520.01	0.01\\
521.01	0.01\\
522.01	0.01\\
523.01	0.01\\
524.01	0.01\\
525.01	0.01\\
526.01	0.01\\
527.01	0.01\\
528.01	0.01\\
529.01	0.01\\
530.01	0.01\\
531.01	0.01\\
532.01	0.01\\
533.01	0.01\\
534.01	0.01\\
535.01	0.01\\
536.01	0.01\\
537.01	0.01\\
538.01	0.01\\
539.01	0.01\\
540.01	0.01\\
541.01	0.01\\
542.01	0.01\\
543.01	0.01\\
544.01	0.01\\
545.01	0.01\\
546.01	0.01\\
547.01	0.01\\
548.01	0.01\\
549.01	0.01\\
550.01	0.01\\
551.01	0.01\\
552.01	0.01\\
553.01	0.01\\
554.01	0.01\\
555.01	0.01\\
556.01	0.01\\
557.01	0.01\\
558.01	0.01\\
559.01	0.01\\
560.01	0.01\\
561.01	0.01\\
562.01	0.01\\
563.01	0.01\\
564.01	0.01\\
565.01	0.01\\
566.01	0.01\\
567.01	0.01\\
568.01	0.01\\
569.01	0.01\\
570.01	0.01\\
571.01	0.01\\
572.01	0.01\\
573.01	0.01\\
574.01	0.01\\
575.01	0.01\\
576.01	0.01\\
577.01	0.01\\
578.01	0.01\\
579.01	0.01\\
580.01	0.01\\
581.01	0.01\\
582.01	0.01\\
583.01	0.01\\
584.01	0.01\\
585.01	0.01\\
586.01	0.01\\
587.01	0.01\\
588.01	0.01\\
589.01	0.01\\
590.01	0.01\\
591.01	0.01\\
592.01	0.01\\
593.01	0.01\\
594.01	0.01\\
595.01	0.01\\
596.01	0.01\\
597.01	0.01\\
598.01	0.01\\
599.01	0.01\\
599.02	0.01\\
599.03	0.01\\
599.04	0.01\\
599.05	0.01\\
599.06	0.01\\
599.07	0.01\\
599.08	0.01\\
599.09	0.01\\
599.1	0.01\\
599.11	0.01\\
599.12	0.01\\
599.13	0.01\\
599.14	0.01\\
599.15	0.01\\
599.16	0.01\\
599.17	0.01\\
599.18	0.01\\
599.19	0.01\\
599.2	0.01\\
599.21	0.01\\
599.22	0.01\\
599.23	0.01\\
599.24	0.01\\
599.25	0.01\\
599.26	0.01\\
599.27	0.01\\
599.28	0.01\\
599.29	0.01\\
599.3	0.01\\
599.31	0.01\\
599.32	0.01\\
599.33	0.01\\
599.34	0.01\\
599.35	0.01\\
599.36	0.01\\
599.37	0.01\\
599.38	0.01\\
599.39	0.01\\
599.4	0.01\\
599.41	0.01\\
599.42	0.01\\
599.43	0.01\\
599.44	0.01\\
599.45	0.01\\
599.46	0.01\\
599.47	0.01\\
599.48	0.01\\
599.49	0.01\\
599.5	0.01\\
599.51	0.01\\
599.52	0.01\\
599.53	0.01\\
599.54	0.01\\
599.55	0.01\\
599.56	0.01\\
599.57	0.01\\
599.58	0.01\\
599.59	0.01\\
599.6	0.01\\
599.61	0.01\\
599.62	0.01\\
599.63	0.01\\
599.64	0.01\\
599.65	0.01\\
599.66	0.01\\
599.67	0.01\\
599.68	0.01\\
599.69	0.01\\
599.7	0.01\\
599.71	0.01\\
599.72	0.01\\
599.73	0.01\\
599.74	0.01\\
599.75	0.01\\
599.76	0.01\\
599.77	0.01\\
599.78	0.01\\
599.79	0.01\\
599.8	0.01\\
599.81	0.01\\
599.82	0.01\\
599.83	0.01\\
599.84	0.01\\
599.85	0.01\\
599.86	0.01\\
599.87	0.01\\
599.88	0.01\\
599.89	0.01\\
599.9	0.01\\
599.91	0.01\\
599.92	0.01\\
599.93	0.01\\
599.94	0.01\\
599.95	0.01\\
599.96	0.01\\
599.97	0.01\\
599.98	0.01\\
599.99	0.01\\
600	0.01\\
};
\end{axis}
\end{tikzpicture}%
 
%  \caption{Continuous Time w/ nFPC}
%\end{subfigure}%
%\hfill%
%\begin{subfigure}{.45\linewidth}
%  \centering
%  \setlength\figureheight{\linewidth} 
%  \setlength\figurewidth{\linewidth}
%  \tikzsetnextfilename{dp_dscr_nFPC_z15}
%  % This file was created by matlab2tikz.
%
%The latest updates can be retrieved from
%  http://www.mathworks.com/matlabcentral/fileexchange/22022-matlab2tikz-matlab2tikz
%where you can also make suggestions and rate matlab2tikz.
%
\definecolor{mycolor1}{rgb}{0.00000,1.00000,0.14286}%
\definecolor{mycolor2}{rgb}{0.00000,1.00000,0.28571}%
\definecolor{mycolor3}{rgb}{0.00000,1.00000,0.42857}%
\definecolor{mycolor4}{rgb}{0.00000,1.00000,0.57143}%
\definecolor{mycolor5}{rgb}{0.00000,1.00000,0.71429}%
\definecolor{mycolor6}{rgb}{0.00000,1.00000,0.85714}%
\definecolor{mycolor7}{rgb}{0.00000,1.00000,1.00000}%
\definecolor{mycolor8}{rgb}{0.00000,0.87500,1.00000}%
\definecolor{mycolor9}{rgb}{0.00000,0.62500,1.00000}%
\definecolor{mycolor10}{rgb}{0.12500,0.00000,1.00000}%
\definecolor{mycolor11}{rgb}{0.25000,0.00000,1.00000}%
\definecolor{mycolor12}{rgb}{0.37500,0.00000,1.00000}%
\definecolor{mycolor13}{rgb}{0.50000,0.00000,1.00000}%
\definecolor{mycolor14}{rgb}{0.62500,0.00000,1.00000}%
\definecolor{mycolor15}{rgb}{0.75000,0.00000,1.00000}%
\definecolor{mycolor16}{rgb}{0.87500,0.00000,1.00000}%
\definecolor{mycolor17}{rgb}{1.00000,0.00000,1.00000}%
\definecolor{mycolor18}{rgb}{1.00000,0.00000,0.87500}%
\definecolor{mycolor19}{rgb}{1.00000,0.00000,0.62500}%
\definecolor{mycolor20}{rgb}{0.85714,0.00000,0.00000}%
\definecolor{mycolor21}{rgb}{0.71429,0.00000,0.00000}%
%
\begin{tikzpicture}[trim axis left, trim axis right]

\begin{axis}[%
width=\figurewidth,
height=\figureheight,
at={(0\figurewidth,0\figureheight)},
scale only axis,
every outer x axis line/.append style={black},
every x tick label/.append style={font=\color{black}},
xmin=0,
xmax=600,
every outer y axis line/.append style={black},
every y tick label/.append style={font=\color{black}},
ymin=0,
ymax=0.014,
axis background/.style={fill=white},
axis x line*=bottom,
axis y line*=left,
yticklabel style={
        /pgf/number format/fixed,
        /pgf/number format/precision=3
},
scaled y ticks=false
]
\addplot [color=green,solid,forget plot]
  table[row sep=crcr]{%
1	0\\
2	0\\
3	0\\
4	0\\
5	0\\
6	0\\
7	0\\
8	0\\
9	0\\
10	0\\
11	0\\
12	0\\
13	0\\
14	0\\
15	0\\
16	0\\
17	0\\
18	0\\
19	0\\
20	0\\
21	0\\
22	0\\
23	0\\
24	0\\
25	0\\
26	0\\
27	0\\
28	0\\
29	0\\
30	0\\
31	0\\
32	0\\
33	0\\
34	0\\
35	0\\
36	0\\
37	0\\
38	0\\
39	0\\
40	0\\
41	0\\
42	0\\
43	0\\
44	0\\
45	0\\
46	0\\
47	0\\
48	0\\
49	0\\
50	0\\
51	0\\
52	0\\
53	0\\
54	0\\
55	0\\
56	0\\
57	0\\
58	0\\
59	0\\
60	0\\
61	0\\
62	0\\
63	0\\
64	0\\
65	0\\
66	0\\
67	0\\
68	0\\
69	0\\
70	0\\
71	0\\
72	0\\
73	0\\
74	0\\
75	0\\
76	0\\
77	0\\
78	0\\
79	0\\
80	0\\
81	0\\
82	0\\
83	0\\
84	0\\
85	0\\
86	0\\
87	0\\
88	0\\
89	0\\
90	0\\
91	0\\
92	0\\
93	0\\
94	0\\
95	0\\
96	0\\
97	0\\
98	0\\
99	0\\
100	0\\
101	0\\
102	0\\
103	0\\
104	0\\
105	0\\
106	0\\
107	0\\
108	0\\
109	0\\
110	0\\
111	0\\
112	0\\
113	0\\
114	0\\
115	0\\
116	0\\
117	0\\
118	0\\
119	0\\
120	0\\
121	0\\
122	0\\
123	0\\
124	0\\
125	0\\
126	0\\
127	0\\
128	0\\
129	0\\
130	0\\
131	0\\
132	0\\
133	0\\
134	0\\
135	0\\
136	0\\
137	0\\
138	0\\
139	0\\
140	0\\
141	0\\
142	0\\
143	0\\
144	0\\
145	0\\
146	0\\
147	0\\
148	0\\
149	0\\
150	0\\
151	0\\
152	0\\
153	0\\
154	0\\
155	0\\
156	0\\
157	0\\
158	0\\
159	0\\
160	0\\
161	0\\
162	0\\
163	0\\
164	0\\
165	0\\
166	0\\
167	0\\
168	0\\
169	0\\
170	0\\
171	0\\
172	0\\
173	0\\
174	0\\
175	0\\
176	0\\
177	0\\
178	0\\
179	0\\
180	0\\
181	0\\
182	0\\
183	0\\
184	0\\
185	0\\
186	0\\
187	0\\
188	0\\
189	0\\
190	0\\
191	0\\
192	0\\
193	0\\
194	0\\
195	0\\
196	0\\
197	0\\
198	0\\
199	0\\
200	0\\
201	0\\
202	0\\
203	0\\
204	0\\
205	0\\
206	0\\
207	0\\
208	0\\
209	0\\
210	0\\
211	0\\
212	0\\
213	0\\
214	0\\
215	0\\
216	0\\
217	0\\
218	0\\
219	0\\
220	0\\
221	0\\
222	0\\
223	0\\
224	0\\
225	0\\
226	0\\
227	0\\
228	0\\
229	0\\
230	0\\
231	0\\
232	0\\
233	0\\
234	0\\
235	0\\
236	0\\
237	0\\
238	0\\
239	0\\
240	0\\
241	0\\
242	0\\
243	0\\
244	0\\
245	0\\
246	0\\
247	0\\
248	0\\
249	0\\
250	0\\
251	0\\
252	0\\
253	0\\
254	0\\
255	0\\
256	0\\
257	0\\
258	0\\
259	0\\
260	0\\
261	0\\
262	0\\
263	0\\
264	0\\
265	0\\
266	0\\
267	0\\
268	0\\
269	0\\
270	0\\
271	0\\
272	0\\
273	0\\
274	0\\
275	0\\
276	0\\
277	0\\
278	0\\
279	0\\
280	0\\
281	0\\
282	0\\
283	0\\
284	0\\
285	0\\
286	0\\
287	0\\
288	0\\
289	0\\
290	0\\
291	0\\
292	0\\
293	0\\
294	0\\
295	0\\
296	0\\
297	0\\
298	0\\
299	0\\
300	0\\
301	0\\
302	0\\
303	0\\
304	0\\
305	0\\
306	0\\
307	0\\
308	0\\
309	0\\
310	0\\
311	0\\
312	0\\
313	0\\
314	0\\
315	0\\
316	0\\
317	0\\
318	0\\
319	0\\
320	0\\
321	0\\
322	0\\
323	0\\
324	0\\
325	0\\
326	0\\
327	0\\
328	0\\
329	0\\
330	0\\
331	0\\
332	0\\
333	0\\
334	0\\
335	0\\
336	0\\
337	0\\
338	0\\
339	0\\
340	0\\
341	0\\
342	0\\
343	0\\
344	0\\
345	0\\
346	0\\
347	0\\
348	0\\
349	0\\
350	0\\
351	0\\
352	0\\
353	0\\
354	0\\
355	0\\
356	0\\
357	0\\
358	0\\
359	0\\
360	0\\
361	0\\
362	0\\
363	0\\
364	0\\
365	0\\
366	0\\
367	0\\
368	0\\
369	0\\
370	0\\
371	0\\
372	0\\
373	0\\
374	0\\
375	0\\
376	0\\
377	0\\
378	0\\
379	0\\
380	0\\
381	0\\
382	0\\
383	0\\
384	0\\
385	0\\
386	0\\
387	0\\
388	0\\
389	0\\
390	0\\
391	0\\
392	0\\
393	0\\
394	0\\
395	0\\
396	0\\
397	0\\
398	0\\
399	0\\
400	0\\
401	0\\
402	0\\
403	0\\
404	0\\
405	0\\
406	0\\
407	0\\
408	0\\
409	0\\
410	0\\
411	0\\
412	0\\
413	0\\
414	0\\
415	0\\
416	0\\
417	0\\
418	0\\
419	0\\
420	0\\
421	0\\
422	0\\
423	0\\
424	0\\
425	0\\
426	0\\
427	0\\
428	0\\
429	0\\
430	0\\
431	0\\
432	0\\
433	0\\
434	0\\
435	0\\
436	0\\
437	0\\
438	0\\
439	0\\
440	0\\
441	0\\
442	0\\
443	0\\
444	0\\
445	0\\
446	0\\
447	0\\
448	0\\
449	0\\
450	0\\
451	0\\
452	0\\
453	0\\
454	0\\
455	0\\
456	0\\
457	0\\
458	0\\
459	0\\
460	0\\
461	0\\
462	0\\
463	0\\
464	0\\
465	0\\
466	0\\
467	0\\
468	0\\
469	0\\
470	0\\
471	0\\
472	0\\
473	0\\
474	0\\
475	0\\
476	0\\
477	0\\
478	0\\
479	0\\
480	0\\
481	0\\
482	0\\
483	0\\
484	0\\
485	0\\
486	0\\
487	0\\
488	0\\
489	0\\
490	0\\
491	0\\
492	0\\
493	0\\
494	0\\
495	0\\
496	0\\
497	0\\
498	0\\
499	0\\
500	0\\
501	0\\
502	0\\
503	0\\
504	0\\
505	0\\
506	0\\
507	0\\
508	0\\
509	0\\
510	0\\
511	0\\
512	0\\
513	0\\
514	0\\
515	0\\
516	0\\
517	0\\
518	0\\
519	0\\
520	0\\
521	0\\
522	0\\
523	0\\
524	0\\
525	0\\
526	0\\
527	0\\
528	0\\
529	0\\
530	0\\
531	0\\
532	0\\
533	0\\
534	0\\
535	0\\
536	0\\
537	0\\
538	0\\
539	0\\
540	0\\
541	0\\
542	0\\
543	0\\
544	0\\
545	0\\
546	0\\
547	0\\
548	0\\
549	0\\
550	0\\
551	0\\
552	0\\
553	0\\
554	0\\
555	0\\
556	0\\
557	0\\
558	0\\
559	0\\
560	0\\
561	0\\
562	0\\
563	0\\
564	0\\
565	0\\
566	0\\
567	0\\
568	0\\
569	0\\
570	0\\
571	0\\
572	0\\
573	0\\
574	0\\
575	0\\
576	0\\
577	0\\
578	0\\
579	0\\
580	0\\
581	0\\
582	0\\
583	0\\
584	0\\
585	0\\
586	0\\
587	0\\
588	0\\
589	0\\
590	0\\
591	0\\
592	0\\
593	0\\
594	0\\
595	0\\
596	0\\
597	0\\
598	0\\
599	0\\
600	0\\
};
\addplot [color=mycolor1,solid,forget plot]
  table[row sep=crcr]{%
1	0\\
2	0\\
3	0\\
4	0\\
5	0\\
6	0\\
7	0\\
8	0\\
9	0\\
10	0\\
11	0\\
12	0\\
13	0\\
14	0\\
15	0\\
16	0\\
17	0\\
18	0\\
19	0\\
20	0\\
21	0\\
22	0\\
23	0\\
24	0\\
25	0\\
26	0\\
27	0\\
28	0\\
29	0\\
30	0\\
31	0\\
32	0\\
33	0\\
34	0\\
35	0\\
36	0\\
37	0\\
38	0\\
39	0\\
40	0\\
41	0\\
42	0\\
43	0\\
44	0\\
45	0\\
46	0\\
47	0\\
48	0\\
49	0\\
50	0\\
51	0\\
52	0\\
53	0\\
54	0\\
55	0\\
56	0\\
57	0\\
58	0\\
59	0\\
60	0\\
61	0\\
62	0\\
63	0\\
64	0\\
65	0\\
66	0\\
67	0\\
68	0\\
69	0\\
70	0\\
71	0\\
72	0\\
73	0\\
74	0\\
75	0\\
76	0\\
77	0\\
78	0\\
79	0\\
80	0\\
81	0\\
82	0\\
83	0\\
84	0\\
85	0\\
86	0\\
87	0\\
88	0\\
89	0\\
90	0\\
91	0\\
92	0\\
93	0\\
94	0\\
95	0\\
96	0\\
97	0\\
98	0\\
99	0\\
100	0\\
101	0\\
102	0\\
103	0\\
104	0\\
105	0\\
106	0\\
107	0\\
108	0\\
109	0\\
110	0\\
111	0\\
112	0\\
113	0\\
114	0\\
115	0\\
116	0\\
117	0\\
118	0\\
119	0\\
120	0\\
121	0\\
122	0\\
123	0\\
124	0\\
125	0\\
126	0\\
127	0\\
128	0\\
129	0\\
130	0\\
131	0\\
132	0\\
133	0\\
134	0\\
135	0\\
136	0\\
137	0\\
138	0\\
139	0\\
140	0\\
141	0\\
142	0\\
143	0\\
144	0\\
145	0\\
146	0\\
147	0\\
148	0\\
149	0\\
150	0\\
151	0\\
152	0\\
153	0\\
154	0\\
155	0\\
156	0\\
157	0\\
158	0\\
159	0\\
160	0\\
161	0\\
162	0\\
163	0\\
164	0\\
165	0\\
166	0\\
167	0\\
168	0\\
169	0\\
170	0\\
171	0\\
172	0\\
173	0\\
174	0\\
175	0\\
176	0\\
177	0\\
178	0\\
179	0\\
180	0\\
181	0\\
182	0\\
183	0\\
184	0\\
185	0\\
186	0\\
187	0\\
188	0\\
189	0\\
190	0\\
191	0\\
192	0\\
193	0\\
194	0\\
195	0\\
196	0\\
197	0\\
198	0\\
199	0\\
200	0\\
201	0\\
202	0\\
203	0\\
204	0\\
205	0\\
206	0\\
207	0\\
208	0\\
209	0\\
210	0\\
211	0\\
212	0\\
213	0\\
214	0\\
215	0\\
216	0\\
217	0\\
218	0\\
219	0\\
220	0\\
221	0\\
222	0\\
223	0\\
224	0\\
225	0\\
226	0\\
227	0\\
228	0\\
229	0\\
230	0\\
231	0\\
232	0\\
233	0\\
234	0\\
235	0\\
236	0\\
237	0\\
238	0\\
239	0\\
240	0\\
241	0\\
242	0\\
243	0\\
244	0\\
245	0\\
246	0\\
247	0\\
248	0\\
249	0\\
250	0\\
251	0\\
252	0\\
253	0\\
254	0\\
255	0\\
256	0\\
257	0\\
258	0\\
259	0\\
260	0\\
261	0\\
262	0\\
263	0\\
264	0\\
265	0\\
266	0\\
267	0\\
268	0\\
269	0\\
270	0\\
271	0\\
272	0\\
273	0\\
274	0\\
275	0\\
276	0\\
277	0\\
278	0\\
279	0\\
280	0\\
281	0\\
282	0\\
283	0\\
284	0\\
285	0\\
286	0\\
287	0\\
288	0\\
289	0\\
290	0\\
291	0\\
292	0\\
293	0\\
294	0\\
295	0\\
296	0\\
297	0\\
298	0\\
299	0\\
300	0\\
301	0\\
302	0\\
303	0\\
304	0\\
305	0\\
306	0\\
307	0\\
308	0\\
309	0\\
310	0\\
311	0\\
312	0\\
313	0\\
314	0\\
315	0\\
316	0\\
317	0\\
318	0\\
319	0\\
320	0\\
321	0\\
322	0\\
323	0\\
324	0\\
325	0\\
326	0\\
327	0\\
328	0\\
329	0\\
330	0\\
331	0\\
332	0\\
333	0\\
334	0\\
335	0\\
336	0\\
337	0\\
338	0\\
339	0\\
340	0\\
341	0\\
342	0\\
343	0\\
344	0\\
345	0\\
346	0\\
347	0\\
348	0\\
349	0\\
350	0\\
351	0\\
352	0\\
353	0\\
354	0\\
355	0\\
356	0\\
357	0\\
358	0\\
359	0\\
360	0\\
361	0\\
362	0\\
363	0\\
364	0\\
365	0\\
366	0\\
367	0\\
368	0\\
369	0\\
370	0\\
371	0\\
372	0\\
373	0\\
374	0\\
375	0\\
376	0\\
377	0\\
378	0\\
379	0\\
380	0\\
381	0\\
382	0\\
383	0\\
384	0\\
385	0\\
386	0\\
387	0\\
388	0\\
389	0\\
390	0\\
391	0\\
392	0\\
393	0\\
394	0\\
395	0\\
396	0\\
397	0\\
398	0\\
399	0\\
400	0\\
401	0\\
402	0\\
403	0\\
404	0\\
405	0\\
406	0\\
407	0\\
408	0\\
409	0\\
410	0\\
411	0\\
412	0\\
413	0\\
414	0\\
415	0\\
416	0\\
417	0\\
418	0\\
419	0\\
420	0\\
421	0\\
422	0\\
423	0\\
424	0\\
425	0\\
426	0\\
427	0\\
428	0\\
429	0\\
430	0\\
431	0\\
432	0\\
433	0\\
434	0\\
435	0\\
436	0\\
437	0\\
438	0\\
439	0\\
440	0\\
441	0\\
442	0\\
443	0\\
444	0\\
445	0\\
446	0\\
447	0\\
448	0\\
449	0\\
450	0\\
451	0\\
452	0\\
453	0\\
454	0\\
455	0\\
456	0\\
457	0\\
458	0\\
459	0\\
460	0\\
461	0\\
462	0\\
463	0\\
464	0\\
465	0\\
466	0\\
467	0\\
468	0\\
469	0\\
470	0\\
471	0\\
472	0\\
473	0\\
474	0\\
475	0\\
476	0\\
477	0\\
478	0\\
479	0\\
480	0\\
481	0\\
482	0\\
483	0\\
484	0\\
485	0\\
486	0\\
487	0\\
488	0\\
489	0\\
490	0\\
491	0\\
492	0\\
493	0\\
494	0\\
495	0\\
496	0\\
497	0\\
498	0\\
499	0\\
500	0\\
501	0\\
502	0\\
503	0\\
504	0\\
505	0\\
506	0\\
507	0\\
508	0\\
509	0\\
510	0\\
511	0\\
512	0\\
513	0\\
514	0\\
515	0\\
516	0\\
517	0\\
518	0\\
519	0\\
520	0\\
521	0\\
522	0\\
523	0\\
524	0\\
525	0\\
526	0\\
527	0\\
528	0\\
529	0\\
530	0\\
531	0\\
532	0\\
533	0\\
534	0\\
535	0\\
536	0\\
537	0\\
538	0\\
539	0\\
540	0\\
541	0\\
542	0\\
543	0\\
544	0\\
545	0\\
546	0\\
547	0\\
548	0\\
549	0\\
550	0\\
551	0\\
552	0\\
553	0\\
554	0\\
555	0\\
556	0\\
557	0\\
558	0\\
559	0\\
560	0\\
561	0\\
562	0\\
563	0\\
564	0\\
565	0\\
566	0\\
567	0\\
568	0\\
569	0\\
570	0\\
571	0\\
572	0\\
573	0\\
574	0\\
575	0\\
576	0\\
577	0\\
578	0\\
579	0\\
580	0\\
581	0\\
582	0\\
583	0\\
584	0\\
585	0\\
586	0\\
587	0\\
588	0\\
589	0\\
590	0\\
591	0\\
592	0\\
593	0\\
594	0\\
595	0\\
596	0\\
597	0\\
598	0\\
599	0\\
600	0\\
};
\addplot [color=mycolor2,solid,forget plot]
  table[row sep=crcr]{%
1	0\\
2	0\\
3	0\\
4	0\\
5	0\\
6	0\\
7	0\\
8	0\\
9	0\\
10	0\\
11	0\\
12	0\\
13	0\\
14	0\\
15	0\\
16	0\\
17	0\\
18	0\\
19	0\\
20	0\\
21	0\\
22	0\\
23	0\\
24	0\\
25	0\\
26	0\\
27	0\\
28	0\\
29	0\\
30	0\\
31	0\\
32	0\\
33	0\\
34	0\\
35	0\\
36	0\\
37	0\\
38	0\\
39	0\\
40	0\\
41	0\\
42	0\\
43	0\\
44	0\\
45	0\\
46	0\\
47	0\\
48	0\\
49	0\\
50	0\\
51	0\\
52	0\\
53	0\\
54	0\\
55	0\\
56	0\\
57	0\\
58	0\\
59	0\\
60	0\\
61	0\\
62	0\\
63	0\\
64	0\\
65	0\\
66	0\\
67	0\\
68	0\\
69	0\\
70	0\\
71	0\\
72	0\\
73	0\\
74	0\\
75	0\\
76	0\\
77	0\\
78	0\\
79	0\\
80	0\\
81	0\\
82	0\\
83	0\\
84	0\\
85	0\\
86	0\\
87	0\\
88	0\\
89	0\\
90	0\\
91	0\\
92	0\\
93	0\\
94	0\\
95	0\\
96	0\\
97	0\\
98	0\\
99	0\\
100	0\\
101	0\\
102	0\\
103	0\\
104	0\\
105	0\\
106	0\\
107	0\\
108	0\\
109	0\\
110	0\\
111	0\\
112	0\\
113	0\\
114	0\\
115	0\\
116	0\\
117	0\\
118	0\\
119	0\\
120	0\\
121	0\\
122	0\\
123	0\\
124	0\\
125	0\\
126	0\\
127	0\\
128	0\\
129	0\\
130	0\\
131	0\\
132	0\\
133	0\\
134	0\\
135	0\\
136	0\\
137	0\\
138	0\\
139	0\\
140	0\\
141	0\\
142	0\\
143	0\\
144	0\\
145	0\\
146	0\\
147	0\\
148	0\\
149	0\\
150	0\\
151	0\\
152	0\\
153	0\\
154	0\\
155	0\\
156	0\\
157	0\\
158	0\\
159	0\\
160	0\\
161	0\\
162	0\\
163	0\\
164	0\\
165	0\\
166	0\\
167	0\\
168	0\\
169	0\\
170	0\\
171	0\\
172	0\\
173	0\\
174	0\\
175	0\\
176	0\\
177	0\\
178	0\\
179	0\\
180	0\\
181	0\\
182	0\\
183	0\\
184	0\\
185	0\\
186	0\\
187	0\\
188	0\\
189	0\\
190	0\\
191	0\\
192	0\\
193	0\\
194	0\\
195	0\\
196	0\\
197	0\\
198	0\\
199	0\\
200	0\\
201	0\\
202	0\\
203	0\\
204	0\\
205	0\\
206	0\\
207	0\\
208	0\\
209	0\\
210	0\\
211	0\\
212	0\\
213	0\\
214	0\\
215	0\\
216	0\\
217	0\\
218	0\\
219	0\\
220	0\\
221	0\\
222	0\\
223	0\\
224	0\\
225	0\\
226	0\\
227	0\\
228	0\\
229	0\\
230	0\\
231	0\\
232	0\\
233	0\\
234	0\\
235	0\\
236	0\\
237	0\\
238	0\\
239	0\\
240	0\\
241	0\\
242	0\\
243	0\\
244	0\\
245	0\\
246	0\\
247	0\\
248	0\\
249	0\\
250	0\\
251	0\\
252	0\\
253	0\\
254	0\\
255	0\\
256	0\\
257	0\\
258	0\\
259	0\\
260	0\\
261	0\\
262	0\\
263	0\\
264	0\\
265	0\\
266	0\\
267	0\\
268	0\\
269	0\\
270	0\\
271	0\\
272	0\\
273	0\\
274	0\\
275	0\\
276	0\\
277	0\\
278	0\\
279	0\\
280	0\\
281	0\\
282	0\\
283	0\\
284	0\\
285	0\\
286	0\\
287	0\\
288	0\\
289	0\\
290	0\\
291	0\\
292	0\\
293	0\\
294	0\\
295	0\\
296	0\\
297	0\\
298	0\\
299	0\\
300	0\\
301	0\\
302	0\\
303	0\\
304	0\\
305	0\\
306	0\\
307	0\\
308	0\\
309	0\\
310	0\\
311	0\\
312	0\\
313	0\\
314	0\\
315	0\\
316	0\\
317	0\\
318	0\\
319	0\\
320	0\\
321	0\\
322	0\\
323	0\\
324	0\\
325	0\\
326	0\\
327	0\\
328	0\\
329	0\\
330	0\\
331	0\\
332	0\\
333	0\\
334	0\\
335	0\\
336	0\\
337	0\\
338	0\\
339	0\\
340	0\\
341	0\\
342	0\\
343	0\\
344	0\\
345	0\\
346	0\\
347	0\\
348	0\\
349	0\\
350	0\\
351	0\\
352	0\\
353	0\\
354	0\\
355	0\\
356	0\\
357	0\\
358	0\\
359	0\\
360	0\\
361	0\\
362	0\\
363	0\\
364	0\\
365	0\\
366	0\\
367	0\\
368	0\\
369	0\\
370	0\\
371	0\\
372	0\\
373	0\\
374	0\\
375	0\\
376	0\\
377	0\\
378	0\\
379	0\\
380	0\\
381	0\\
382	0\\
383	0\\
384	0\\
385	0\\
386	0\\
387	0\\
388	0\\
389	0\\
390	0\\
391	0\\
392	0\\
393	0\\
394	0\\
395	0\\
396	0\\
397	0\\
398	0\\
399	0\\
400	0\\
401	0\\
402	0\\
403	0\\
404	0\\
405	0\\
406	0\\
407	0\\
408	0\\
409	0\\
410	0\\
411	0\\
412	0\\
413	0\\
414	0\\
415	0\\
416	0\\
417	0\\
418	0\\
419	0\\
420	0\\
421	0\\
422	0\\
423	0\\
424	0\\
425	0\\
426	0\\
427	0\\
428	0\\
429	0\\
430	0\\
431	0\\
432	0\\
433	0\\
434	0\\
435	0\\
436	0\\
437	0\\
438	0\\
439	0\\
440	0\\
441	0\\
442	0\\
443	0\\
444	0\\
445	0\\
446	0\\
447	0\\
448	0\\
449	0\\
450	0\\
451	0\\
452	0\\
453	0\\
454	0\\
455	0\\
456	0\\
457	0\\
458	0\\
459	0\\
460	0\\
461	0\\
462	0\\
463	0\\
464	0\\
465	0\\
466	0\\
467	0\\
468	0\\
469	0\\
470	0\\
471	0\\
472	0\\
473	0\\
474	0\\
475	0\\
476	0\\
477	0\\
478	0\\
479	0\\
480	0\\
481	0\\
482	0\\
483	0\\
484	0\\
485	0\\
486	0\\
487	0\\
488	0\\
489	0\\
490	0\\
491	0\\
492	0\\
493	0\\
494	0\\
495	0\\
496	0\\
497	0\\
498	0\\
499	0\\
500	0\\
501	0\\
502	0\\
503	0\\
504	0\\
505	0\\
506	0\\
507	0\\
508	0\\
509	0\\
510	0\\
511	0\\
512	0\\
513	0\\
514	0\\
515	0\\
516	0\\
517	0\\
518	0\\
519	0\\
520	0\\
521	0\\
522	0\\
523	0\\
524	0\\
525	0\\
526	0\\
527	0\\
528	0\\
529	0\\
530	0\\
531	0\\
532	0\\
533	0\\
534	0\\
535	0\\
536	0\\
537	0\\
538	0\\
539	0\\
540	0\\
541	0\\
542	0\\
543	0\\
544	0\\
545	0\\
546	0\\
547	0\\
548	0\\
549	0\\
550	0\\
551	0\\
552	0\\
553	0\\
554	0\\
555	0\\
556	0\\
557	0\\
558	0\\
559	0\\
560	0\\
561	0\\
562	0\\
563	0\\
564	0\\
565	0\\
566	0\\
567	0\\
568	0\\
569	0\\
570	0\\
571	0\\
572	0\\
573	0\\
574	0\\
575	0\\
576	0\\
577	0\\
578	0\\
579	0\\
580	0\\
581	0\\
582	0\\
583	0\\
584	0\\
585	0\\
586	0\\
587	0\\
588	0\\
589	0\\
590	0\\
591	0\\
592	0\\
593	0\\
594	0\\
595	0\\
596	0\\
597	0\\
598	0\\
599	0\\
600	0\\
};
\addplot [color=mycolor3,solid,forget plot]
  table[row sep=crcr]{%
1	0\\
2	0\\
3	0\\
4	0\\
5	0\\
6	0\\
7	0\\
8	0\\
9	0\\
10	0\\
11	0\\
12	0\\
13	0\\
14	0\\
15	0\\
16	0\\
17	0\\
18	0\\
19	0\\
20	0\\
21	0\\
22	0\\
23	0\\
24	0\\
25	0\\
26	0\\
27	0\\
28	0\\
29	0\\
30	0\\
31	0\\
32	0\\
33	0\\
34	0\\
35	0\\
36	0\\
37	0\\
38	0\\
39	0\\
40	0\\
41	0\\
42	0\\
43	0\\
44	0\\
45	0\\
46	0\\
47	0\\
48	0\\
49	0\\
50	0\\
51	0\\
52	0\\
53	0\\
54	0\\
55	0\\
56	0\\
57	0\\
58	0\\
59	0\\
60	0\\
61	0\\
62	0\\
63	0\\
64	0\\
65	0\\
66	0\\
67	0\\
68	0\\
69	0\\
70	0\\
71	0\\
72	0\\
73	0\\
74	0\\
75	0\\
76	0\\
77	0\\
78	0\\
79	0\\
80	0\\
81	0\\
82	0\\
83	0\\
84	0\\
85	0\\
86	0\\
87	0\\
88	0\\
89	0\\
90	0\\
91	0\\
92	0\\
93	0\\
94	0\\
95	0\\
96	0\\
97	0\\
98	0\\
99	0\\
100	0\\
101	0\\
102	0\\
103	0\\
104	0\\
105	0\\
106	0\\
107	0\\
108	0\\
109	0\\
110	0\\
111	0\\
112	0\\
113	0\\
114	0\\
115	0\\
116	0\\
117	0\\
118	0\\
119	0\\
120	0\\
121	0\\
122	0\\
123	0\\
124	0\\
125	0\\
126	0\\
127	0\\
128	0\\
129	0\\
130	0\\
131	0\\
132	0\\
133	0\\
134	0\\
135	0\\
136	0\\
137	0\\
138	0\\
139	0\\
140	0\\
141	0\\
142	0\\
143	0\\
144	0\\
145	0\\
146	0\\
147	0\\
148	0\\
149	0\\
150	0\\
151	0\\
152	0\\
153	0\\
154	0\\
155	0\\
156	0\\
157	0\\
158	0\\
159	0\\
160	0\\
161	0\\
162	0\\
163	0\\
164	0\\
165	0\\
166	0\\
167	0\\
168	0\\
169	0\\
170	0\\
171	0\\
172	0\\
173	0\\
174	0\\
175	0\\
176	0\\
177	0\\
178	0\\
179	0\\
180	0\\
181	0\\
182	0\\
183	0\\
184	0\\
185	0\\
186	0\\
187	0\\
188	0\\
189	0\\
190	0\\
191	0\\
192	0\\
193	0\\
194	0\\
195	0\\
196	0\\
197	0\\
198	0\\
199	0\\
200	0\\
201	0\\
202	0\\
203	0\\
204	0\\
205	0\\
206	0\\
207	0\\
208	0\\
209	0\\
210	0\\
211	0\\
212	0\\
213	0\\
214	0\\
215	0\\
216	0\\
217	0\\
218	0\\
219	0\\
220	0\\
221	0\\
222	0\\
223	0\\
224	0\\
225	0\\
226	0\\
227	0\\
228	0\\
229	0\\
230	0\\
231	0\\
232	0\\
233	0\\
234	0\\
235	0\\
236	0\\
237	0\\
238	0\\
239	0\\
240	0\\
241	0\\
242	0\\
243	0\\
244	0\\
245	0\\
246	0\\
247	0\\
248	0\\
249	0\\
250	0\\
251	0\\
252	0\\
253	0\\
254	0\\
255	0\\
256	0\\
257	0\\
258	0\\
259	0\\
260	0\\
261	0\\
262	0\\
263	0\\
264	0\\
265	0\\
266	0\\
267	0\\
268	0\\
269	0\\
270	0\\
271	0\\
272	0\\
273	0\\
274	0\\
275	0\\
276	0\\
277	0\\
278	0\\
279	0\\
280	0\\
281	0\\
282	0\\
283	0\\
284	0\\
285	0\\
286	0\\
287	0\\
288	0\\
289	0\\
290	0\\
291	0\\
292	0\\
293	0\\
294	0\\
295	0\\
296	0\\
297	0\\
298	0\\
299	0\\
300	0\\
301	0\\
302	0\\
303	0\\
304	0\\
305	0\\
306	0\\
307	0\\
308	0\\
309	0\\
310	0\\
311	0\\
312	0\\
313	0\\
314	0\\
315	0\\
316	0\\
317	0\\
318	0\\
319	0\\
320	0\\
321	0\\
322	0\\
323	0\\
324	0\\
325	0\\
326	0\\
327	0\\
328	0\\
329	0\\
330	0\\
331	0\\
332	0\\
333	0\\
334	0\\
335	0\\
336	0\\
337	0\\
338	0\\
339	0\\
340	0\\
341	0\\
342	0\\
343	0\\
344	0\\
345	0\\
346	0\\
347	0\\
348	0\\
349	0\\
350	0\\
351	0\\
352	0\\
353	0\\
354	0\\
355	0\\
356	0\\
357	0\\
358	0\\
359	0\\
360	0\\
361	0\\
362	0\\
363	0\\
364	0\\
365	0\\
366	0\\
367	0\\
368	0\\
369	0\\
370	0\\
371	0\\
372	0\\
373	0\\
374	0\\
375	0\\
376	0\\
377	0\\
378	0\\
379	0\\
380	0\\
381	0\\
382	0\\
383	0\\
384	0\\
385	0\\
386	0\\
387	0\\
388	0\\
389	0\\
390	0\\
391	0\\
392	0\\
393	0\\
394	0\\
395	0\\
396	0\\
397	0\\
398	0\\
399	0\\
400	0\\
401	0\\
402	0\\
403	0\\
404	0\\
405	0\\
406	0\\
407	0\\
408	0\\
409	0\\
410	0\\
411	0\\
412	0\\
413	0\\
414	0\\
415	0\\
416	0\\
417	0\\
418	0\\
419	0\\
420	0\\
421	0\\
422	0\\
423	0\\
424	0\\
425	0\\
426	0\\
427	0\\
428	0\\
429	0\\
430	0\\
431	0\\
432	0\\
433	0\\
434	0\\
435	0\\
436	0\\
437	0\\
438	0\\
439	0\\
440	0\\
441	0\\
442	0\\
443	0\\
444	0\\
445	0\\
446	0\\
447	0\\
448	0\\
449	0\\
450	0\\
451	0\\
452	0\\
453	0\\
454	0\\
455	0\\
456	0\\
457	0\\
458	0\\
459	0\\
460	0\\
461	0\\
462	0\\
463	0\\
464	0\\
465	0\\
466	0\\
467	0\\
468	0\\
469	0\\
470	0\\
471	0\\
472	0\\
473	0\\
474	0\\
475	0\\
476	0\\
477	0\\
478	0\\
479	0\\
480	0\\
481	0\\
482	0\\
483	0\\
484	0\\
485	0\\
486	0\\
487	0\\
488	0\\
489	0\\
490	0\\
491	0\\
492	0\\
493	0\\
494	0\\
495	0\\
496	0\\
497	0\\
498	0\\
499	0\\
500	0\\
501	0\\
502	0\\
503	0\\
504	0\\
505	0\\
506	0\\
507	0\\
508	0\\
509	0\\
510	0\\
511	0\\
512	0\\
513	0\\
514	0\\
515	0\\
516	0\\
517	0\\
518	0\\
519	0\\
520	0\\
521	0\\
522	0\\
523	0\\
524	0\\
525	0\\
526	0\\
527	0\\
528	0\\
529	0\\
530	0\\
531	0\\
532	0\\
533	0\\
534	0\\
535	0\\
536	0\\
537	0\\
538	0\\
539	0\\
540	0\\
541	0\\
542	0\\
543	0\\
544	0\\
545	0\\
546	0\\
547	0\\
548	0\\
549	0\\
550	0\\
551	0\\
552	0\\
553	0\\
554	0\\
555	0\\
556	0\\
557	0\\
558	0\\
559	0\\
560	0\\
561	0\\
562	0\\
563	0\\
564	0\\
565	0\\
566	0\\
567	0\\
568	0\\
569	0\\
570	0\\
571	0\\
572	0\\
573	0\\
574	0\\
575	0\\
576	0\\
577	0\\
578	0\\
579	0\\
580	0\\
581	0\\
582	0\\
583	0\\
584	0\\
585	0\\
586	0\\
587	0\\
588	0\\
589	0\\
590	0\\
591	0\\
592	0\\
593	0\\
594	0\\
595	0\\
596	0\\
597	0\\
598	0\\
599	0\\
600	0\\
};
\addplot [color=mycolor4,solid,forget plot]
  table[row sep=crcr]{%
1	0\\
2	0\\
3	0\\
4	0\\
5	0\\
6	0\\
7	0\\
8	0\\
9	0\\
10	0\\
11	0\\
12	0\\
13	0\\
14	0\\
15	0\\
16	0\\
17	0\\
18	0\\
19	0\\
20	0\\
21	0\\
22	0\\
23	0\\
24	0\\
25	0\\
26	0\\
27	0\\
28	0\\
29	0\\
30	0\\
31	0\\
32	0\\
33	0\\
34	0\\
35	0\\
36	0\\
37	0\\
38	0\\
39	0\\
40	0\\
41	0\\
42	0\\
43	0\\
44	0\\
45	0\\
46	0\\
47	0\\
48	0\\
49	0\\
50	0\\
51	0\\
52	0\\
53	0\\
54	0\\
55	0\\
56	0\\
57	0\\
58	0\\
59	0\\
60	0\\
61	0\\
62	0\\
63	0\\
64	0\\
65	0\\
66	0\\
67	0\\
68	0\\
69	0\\
70	0\\
71	0\\
72	0\\
73	0\\
74	0\\
75	0\\
76	0\\
77	0\\
78	0\\
79	0\\
80	0\\
81	0\\
82	0\\
83	0\\
84	0\\
85	0\\
86	0\\
87	0\\
88	0\\
89	0\\
90	0\\
91	0\\
92	0\\
93	0\\
94	0\\
95	0\\
96	0\\
97	0\\
98	0\\
99	0\\
100	0\\
101	0\\
102	0\\
103	0\\
104	0\\
105	0\\
106	0\\
107	0\\
108	0\\
109	0\\
110	0\\
111	0\\
112	0\\
113	0\\
114	0\\
115	0\\
116	0\\
117	0\\
118	0\\
119	0\\
120	0\\
121	0\\
122	0\\
123	0\\
124	0\\
125	0\\
126	0\\
127	0\\
128	0\\
129	0\\
130	0\\
131	0\\
132	0\\
133	0\\
134	0\\
135	0\\
136	0\\
137	0\\
138	0\\
139	0\\
140	0\\
141	0\\
142	0\\
143	0\\
144	0\\
145	0\\
146	0\\
147	0\\
148	0\\
149	0\\
150	0\\
151	0\\
152	0\\
153	0\\
154	0\\
155	0\\
156	0\\
157	0\\
158	0\\
159	0\\
160	0\\
161	0\\
162	0\\
163	0\\
164	0\\
165	0\\
166	0\\
167	0\\
168	0\\
169	0\\
170	0\\
171	0\\
172	0\\
173	0\\
174	0\\
175	0\\
176	0\\
177	0\\
178	0\\
179	0\\
180	0\\
181	0\\
182	0\\
183	0\\
184	0\\
185	0\\
186	0\\
187	0\\
188	0\\
189	0\\
190	0\\
191	0\\
192	0\\
193	0\\
194	0\\
195	0\\
196	0\\
197	0\\
198	0\\
199	0\\
200	0\\
201	0\\
202	0\\
203	0\\
204	0\\
205	0\\
206	0\\
207	0\\
208	0\\
209	0\\
210	0\\
211	0\\
212	0\\
213	0\\
214	0\\
215	0\\
216	0\\
217	0\\
218	0\\
219	0\\
220	0\\
221	0\\
222	0\\
223	0\\
224	0\\
225	0\\
226	0\\
227	0\\
228	0\\
229	0\\
230	0\\
231	0\\
232	0\\
233	0\\
234	0\\
235	0\\
236	0\\
237	0\\
238	0\\
239	0\\
240	0\\
241	0\\
242	0\\
243	0\\
244	0\\
245	0\\
246	0\\
247	0\\
248	0\\
249	0\\
250	0\\
251	0\\
252	0\\
253	0\\
254	0\\
255	0\\
256	0\\
257	0\\
258	0\\
259	0\\
260	0\\
261	0\\
262	0\\
263	0\\
264	0\\
265	0\\
266	0\\
267	0\\
268	0\\
269	0\\
270	0\\
271	0\\
272	0\\
273	0\\
274	0\\
275	0\\
276	0\\
277	0\\
278	0\\
279	0\\
280	0\\
281	0\\
282	0\\
283	0\\
284	0\\
285	0\\
286	0\\
287	0\\
288	0\\
289	0\\
290	0\\
291	0\\
292	0\\
293	0\\
294	0\\
295	0\\
296	0\\
297	0\\
298	0\\
299	0\\
300	0\\
301	0\\
302	0\\
303	0\\
304	0\\
305	0\\
306	0\\
307	0\\
308	0\\
309	0\\
310	0\\
311	0\\
312	0\\
313	0\\
314	0\\
315	0\\
316	0\\
317	0\\
318	0\\
319	0\\
320	0\\
321	0\\
322	0\\
323	0\\
324	0\\
325	0\\
326	0\\
327	0\\
328	0\\
329	0\\
330	0\\
331	0\\
332	0\\
333	0\\
334	0\\
335	0\\
336	0\\
337	0\\
338	0\\
339	0\\
340	0\\
341	0\\
342	0\\
343	0\\
344	0\\
345	0\\
346	0\\
347	0\\
348	0\\
349	0\\
350	0\\
351	0\\
352	0\\
353	0\\
354	0\\
355	0\\
356	0\\
357	0\\
358	0\\
359	0\\
360	0\\
361	0\\
362	0\\
363	0\\
364	0\\
365	0\\
366	0\\
367	0\\
368	0\\
369	0\\
370	0\\
371	0\\
372	0\\
373	0\\
374	0\\
375	0\\
376	0\\
377	0\\
378	0\\
379	0\\
380	0\\
381	0\\
382	0\\
383	0\\
384	0\\
385	0\\
386	0\\
387	0\\
388	0\\
389	0\\
390	0\\
391	0\\
392	0\\
393	0\\
394	0\\
395	0\\
396	0\\
397	0\\
398	0\\
399	0\\
400	0\\
401	0\\
402	0\\
403	0\\
404	0\\
405	0\\
406	0\\
407	0\\
408	0\\
409	0\\
410	0\\
411	0\\
412	0\\
413	0\\
414	0\\
415	0\\
416	0\\
417	0\\
418	0\\
419	0\\
420	0\\
421	0\\
422	0\\
423	0\\
424	0\\
425	0\\
426	0\\
427	0\\
428	0\\
429	0\\
430	0\\
431	0\\
432	0\\
433	0\\
434	0\\
435	0\\
436	0\\
437	0\\
438	0\\
439	0\\
440	0\\
441	0\\
442	0\\
443	0\\
444	0\\
445	0\\
446	0\\
447	0\\
448	0\\
449	0\\
450	0\\
451	0\\
452	0\\
453	0\\
454	0\\
455	0\\
456	0\\
457	0\\
458	0\\
459	0\\
460	0\\
461	0\\
462	0\\
463	0\\
464	0\\
465	0\\
466	0\\
467	0\\
468	0\\
469	0\\
470	0\\
471	0\\
472	0\\
473	0\\
474	0\\
475	0\\
476	0\\
477	0\\
478	0\\
479	0\\
480	0\\
481	0\\
482	0\\
483	0\\
484	0\\
485	0\\
486	0\\
487	0\\
488	0\\
489	0\\
490	0\\
491	0\\
492	0\\
493	0\\
494	0\\
495	0\\
496	0\\
497	0\\
498	0\\
499	0\\
500	0\\
501	0\\
502	0\\
503	0\\
504	0\\
505	0\\
506	0\\
507	0\\
508	0\\
509	0\\
510	0\\
511	0\\
512	0\\
513	0\\
514	0\\
515	0\\
516	0\\
517	0\\
518	0\\
519	0\\
520	0\\
521	0\\
522	0\\
523	0\\
524	0\\
525	0\\
526	0\\
527	0\\
528	0\\
529	0\\
530	0\\
531	0\\
532	0\\
533	0\\
534	0\\
535	0\\
536	0\\
537	0\\
538	0\\
539	0\\
540	0\\
541	0\\
542	0\\
543	0\\
544	0\\
545	0\\
546	0\\
547	0\\
548	0\\
549	0\\
550	0\\
551	0\\
552	0\\
553	0\\
554	0\\
555	0\\
556	0\\
557	0\\
558	0\\
559	0\\
560	0\\
561	0\\
562	0\\
563	0\\
564	0\\
565	0\\
566	0\\
567	0\\
568	0\\
569	0\\
570	0\\
571	0\\
572	0\\
573	0\\
574	0\\
575	0\\
576	0\\
577	0\\
578	0\\
579	0\\
580	0\\
581	0\\
582	0\\
583	0\\
584	0\\
585	0\\
586	0\\
587	0\\
588	0\\
589	0\\
590	0\\
591	0\\
592	0\\
593	0\\
594	0\\
595	0\\
596	0\\
597	0\\
598	0\\
599	0\\
600	0\\
};
\addplot [color=mycolor5,solid,forget plot]
  table[row sep=crcr]{%
1	0\\
2	0\\
3	0\\
4	0\\
5	0\\
6	0\\
7	0\\
8	0\\
9	0\\
10	0\\
11	0\\
12	0\\
13	0\\
14	0\\
15	0\\
16	0\\
17	0\\
18	0\\
19	0\\
20	0\\
21	0\\
22	0\\
23	0\\
24	0\\
25	0\\
26	0\\
27	0\\
28	0\\
29	0\\
30	0\\
31	0\\
32	0\\
33	0\\
34	0\\
35	0\\
36	0\\
37	0\\
38	0\\
39	0\\
40	0\\
41	0\\
42	0\\
43	0\\
44	0\\
45	0\\
46	0\\
47	0\\
48	0\\
49	0\\
50	0\\
51	0\\
52	0\\
53	0\\
54	0\\
55	0\\
56	0\\
57	0\\
58	0\\
59	0\\
60	0\\
61	0\\
62	0\\
63	0\\
64	0\\
65	0\\
66	0\\
67	0\\
68	0\\
69	0\\
70	0\\
71	0\\
72	0\\
73	0\\
74	0\\
75	0\\
76	0\\
77	0\\
78	0\\
79	0\\
80	0\\
81	0\\
82	0\\
83	0\\
84	0\\
85	0\\
86	0\\
87	0\\
88	0\\
89	0\\
90	0\\
91	0\\
92	0\\
93	0\\
94	0\\
95	0\\
96	0\\
97	0\\
98	0\\
99	0\\
100	0\\
101	0\\
102	0\\
103	0\\
104	0\\
105	0\\
106	0\\
107	0\\
108	0\\
109	0\\
110	0\\
111	0\\
112	0\\
113	0\\
114	0\\
115	0\\
116	0\\
117	0\\
118	0\\
119	0\\
120	0\\
121	0\\
122	0\\
123	0\\
124	0\\
125	0\\
126	0\\
127	0\\
128	0\\
129	0\\
130	0\\
131	0\\
132	0\\
133	0\\
134	0\\
135	0\\
136	0\\
137	0\\
138	0\\
139	0\\
140	0\\
141	0\\
142	0\\
143	0\\
144	0\\
145	0\\
146	0\\
147	0\\
148	0\\
149	0\\
150	0\\
151	0\\
152	0\\
153	0\\
154	0\\
155	0\\
156	0\\
157	0\\
158	0\\
159	0\\
160	0\\
161	0\\
162	0\\
163	0\\
164	0\\
165	0\\
166	0\\
167	0\\
168	0\\
169	0\\
170	0\\
171	0\\
172	0\\
173	0\\
174	0\\
175	0\\
176	0\\
177	0\\
178	0\\
179	0\\
180	0\\
181	0\\
182	0\\
183	0\\
184	0\\
185	0\\
186	0\\
187	0\\
188	0\\
189	0\\
190	0\\
191	0\\
192	0\\
193	0\\
194	0\\
195	0\\
196	0\\
197	0\\
198	0\\
199	0\\
200	0\\
201	0\\
202	0\\
203	0\\
204	0\\
205	0\\
206	0\\
207	0\\
208	0\\
209	0\\
210	0\\
211	0\\
212	0\\
213	0\\
214	0\\
215	0\\
216	0\\
217	0\\
218	0\\
219	0\\
220	0\\
221	0\\
222	0\\
223	0\\
224	0\\
225	0\\
226	0\\
227	0\\
228	0\\
229	0\\
230	0\\
231	0\\
232	0\\
233	0\\
234	0\\
235	0\\
236	0\\
237	0\\
238	0\\
239	0\\
240	0\\
241	0\\
242	0\\
243	0\\
244	0\\
245	0\\
246	0\\
247	0\\
248	0\\
249	0\\
250	0\\
251	0\\
252	0\\
253	0\\
254	0\\
255	0\\
256	0\\
257	0\\
258	0\\
259	0\\
260	0\\
261	0\\
262	0\\
263	0\\
264	0\\
265	0\\
266	0\\
267	0\\
268	0\\
269	0\\
270	0\\
271	0\\
272	0\\
273	0\\
274	0\\
275	0\\
276	0\\
277	0\\
278	0\\
279	0\\
280	0\\
281	0\\
282	0\\
283	0\\
284	0\\
285	0\\
286	0\\
287	0\\
288	0\\
289	0\\
290	0\\
291	0\\
292	0\\
293	0\\
294	0\\
295	0\\
296	0\\
297	0\\
298	0\\
299	0\\
300	0\\
301	0\\
302	0\\
303	0\\
304	0\\
305	0\\
306	0\\
307	0\\
308	0\\
309	0\\
310	0\\
311	0\\
312	0\\
313	0\\
314	0\\
315	0\\
316	0\\
317	0\\
318	0\\
319	0\\
320	0\\
321	0\\
322	0\\
323	0\\
324	0\\
325	0\\
326	0\\
327	0\\
328	0\\
329	0\\
330	0\\
331	0\\
332	0\\
333	0\\
334	0\\
335	0\\
336	0\\
337	0\\
338	0\\
339	0\\
340	0\\
341	0\\
342	0\\
343	0\\
344	0\\
345	0\\
346	0\\
347	0\\
348	0\\
349	0\\
350	0\\
351	0\\
352	0\\
353	0\\
354	0\\
355	0\\
356	0\\
357	0\\
358	0\\
359	0\\
360	0\\
361	0\\
362	0\\
363	0\\
364	0\\
365	0\\
366	0\\
367	0\\
368	0\\
369	0\\
370	0\\
371	0\\
372	0\\
373	0\\
374	0\\
375	0\\
376	0\\
377	0\\
378	0\\
379	0\\
380	0\\
381	0\\
382	0\\
383	0\\
384	0\\
385	0\\
386	0\\
387	0\\
388	0\\
389	0\\
390	0\\
391	0\\
392	0\\
393	0\\
394	0\\
395	0\\
396	0\\
397	0\\
398	0\\
399	0\\
400	0\\
401	0\\
402	0\\
403	0\\
404	0\\
405	0\\
406	0\\
407	0\\
408	0\\
409	0\\
410	0\\
411	0\\
412	0\\
413	0\\
414	0\\
415	0\\
416	0\\
417	0\\
418	0\\
419	0\\
420	0\\
421	0\\
422	0\\
423	0\\
424	0\\
425	0\\
426	0\\
427	0\\
428	0\\
429	0\\
430	0\\
431	0\\
432	0\\
433	0\\
434	0\\
435	0\\
436	0\\
437	0\\
438	0\\
439	0\\
440	0\\
441	0\\
442	0\\
443	0\\
444	0\\
445	0\\
446	0\\
447	0\\
448	0\\
449	0\\
450	0\\
451	0\\
452	0\\
453	0\\
454	0\\
455	0\\
456	0\\
457	0\\
458	0\\
459	0\\
460	0\\
461	0\\
462	0\\
463	0\\
464	0\\
465	0\\
466	0\\
467	0\\
468	0\\
469	0\\
470	0\\
471	0\\
472	0\\
473	0\\
474	0\\
475	0\\
476	0\\
477	0\\
478	0\\
479	0\\
480	0\\
481	0\\
482	0\\
483	0\\
484	0\\
485	0\\
486	0\\
487	0\\
488	0\\
489	0\\
490	0\\
491	0\\
492	0\\
493	0\\
494	0\\
495	0\\
496	0\\
497	0\\
498	0\\
499	0\\
500	0\\
501	0\\
502	0\\
503	0\\
504	0\\
505	0\\
506	0\\
507	0\\
508	0\\
509	0\\
510	0\\
511	0\\
512	0\\
513	0\\
514	0\\
515	0\\
516	0\\
517	0\\
518	0\\
519	0\\
520	0\\
521	0\\
522	0\\
523	0\\
524	0\\
525	0\\
526	0\\
527	0\\
528	0\\
529	0\\
530	0\\
531	0\\
532	0\\
533	0\\
534	0\\
535	0\\
536	0\\
537	0\\
538	0\\
539	0\\
540	0\\
541	0\\
542	0\\
543	0\\
544	0\\
545	0\\
546	0\\
547	0\\
548	0\\
549	0\\
550	0\\
551	0\\
552	0\\
553	0\\
554	0\\
555	0\\
556	0\\
557	0\\
558	0\\
559	0\\
560	0\\
561	0\\
562	0\\
563	0\\
564	0\\
565	0\\
566	0\\
567	0\\
568	0\\
569	0\\
570	0\\
571	0\\
572	0\\
573	0\\
574	0\\
575	0\\
576	0\\
577	0\\
578	0\\
579	0\\
580	0\\
581	0\\
582	0\\
583	0\\
584	0\\
585	0\\
586	0\\
587	0\\
588	0\\
589	0\\
590	0\\
591	0\\
592	0\\
593	0\\
594	0\\
595	0\\
596	0\\
597	0\\
598	0\\
599	0\\
600	0\\
};
\addplot [color=mycolor6,solid,forget plot]
  table[row sep=crcr]{%
1	0\\
2	0\\
3	0\\
4	0\\
5	0\\
6	0\\
7	0\\
8	0\\
9	0\\
10	0\\
11	0\\
12	0\\
13	0\\
14	0\\
15	0\\
16	0\\
17	0\\
18	0\\
19	0\\
20	0\\
21	0\\
22	0\\
23	0\\
24	0\\
25	0\\
26	0\\
27	0\\
28	0\\
29	0\\
30	0\\
31	0\\
32	0\\
33	0\\
34	0\\
35	0\\
36	0\\
37	0\\
38	0\\
39	0\\
40	0\\
41	0\\
42	0\\
43	0\\
44	0\\
45	0\\
46	0\\
47	0\\
48	0\\
49	0\\
50	0\\
51	0\\
52	0\\
53	0\\
54	0\\
55	0\\
56	0\\
57	0\\
58	0\\
59	0\\
60	0\\
61	0\\
62	0\\
63	0\\
64	0\\
65	0\\
66	0\\
67	0\\
68	0\\
69	0\\
70	0\\
71	0\\
72	0\\
73	0\\
74	0\\
75	0\\
76	0\\
77	0\\
78	0\\
79	0\\
80	0\\
81	0\\
82	0\\
83	0\\
84	0\\
85	0\\
86	0\\
87	0\\
88	0\\
89	0\\
90	0\\
91	0\\
92	0\\
93	0\\
94	0\\
95	0\\
96	0\\
97	0\\
98	0\\
99	0\\
100	0\\
101	0\\
102	0\\
103	0\\
104	0\\
105	0\\
106	0\\
107	0\\
108	0\\
109	0\\
110	0\\
111	0\\
112	0\\
113	0\\
114	0\\
115	0\\
116	0\\
117	0\\
118	0\\
119	0\\
120	0\\
121	0\\
122	0\\
123	0\\
124	0\\
125	0\\
126	0\\
127	0\\
128	0\\
129	0\\
130	0\\
131	0\\
132	0\\
133	0\\
134	0\\
135	0\\
136	0\\
137	0\\
138	0\\
139	0\\
140	0\\
141	0\\
142	0\\
143	0\\
144	0\\
145	0\\
146	0\\
147	0\\
148	0\\
149	0\\
150	0\\
151	0\\
152	0\\
153	0\\
154	0\\
155	0\\
156	0\\
157	0\\
158	0\\
159	0\\
160	0\\
161	0\\
162	0\\
163	0\\
164	0\\
165	0\\
166	0\\
167	0\\
168	0\\
169	0\\
170	0\\
171	0\\
172	0\\
173	0\\
174	0\\
175	0\\
176	0\\
177	0\\
178	0\\
179	0\\
180	0\\
181	0\\
182	0\\
183	0\\
184	0\\
185	0\\
186	0\\
187	0\\
188	0\\
189	0\\
190	0\\
191	0\\
192	0\\
193	0\\
194	0\\
195	0\\
196	0\\
197	0\\
198	0\\
199	0\\
200	0\\
201	0\\
202	0\\
203	0\\
204	0\\
205	0\\
206	0\\
207	0\\
208	0\\
209	0\\
210	0\\
211	0\\
212	0\\
213	0\\
214	0\\
215	0\\
216	0\\
217	0\\
218	0\\
219	0\\
220	0\\
221	0\\
222	0\\
223	0\\
224	0\\
225	0\\
226	0\\
227	0\\
228	0\\
229	0\\
230	0\\
231	0\\
232	0\\
233	0\\
234	0\\
235	0\\
236	0\\
237	0\\
238	0\\
239	0\\
240	0\\
241	0\\
242	0\\
243	0\\
244	0\\
245	0\\
246	0\\
247	0\\
248	0\\
249	0\\
250	0\\
251	0\\
252	0\\
253	0\\
254	0\\
255	0\\
256	0\\
257	0\\
258	0\\
259	0\\
260	0\\
261	0\\
262	0\\
263	0\\
264	0\\
265	0\\
266	0\\
267	0\\
268	0\\
269	0\\
270	0\\
271	0\\
272	0\\
273	0\\
274	0\\
275	0\\
276	0\\
277	0\\
278	0\\
279	0\\
280	0\\
281	0\\
282	0\\
283	0\\
284	0\\
285	0\\
286	0\\
287	0\\
288	0\\
289	0\\
290	0\\
291	0\\
292	0\\
293	0\\
294	0\\
295	0\\
296	0\\
297	0\\
298	0\\
299	0\\
300	0\\
301	0\\
302	0\\
303	0\\
304	0\\
305	0\\
306	0\\
307	0\\
308	0\\
309	0\\
310	0\\
311	0\\
312	0\\
313	0\\
314	0\\
315	0\\
316	0\\
317	0\\
318	0\\
319	0\\
320	0\\
321	0\\
322	0\\
323	0\\
324	0\\
325	0\\
326	0\\
327	0\\
328	0\\
329	0\\
330	0\\
331	0\\
332	0\\
333	0\\
334	0\\
335	0\\
336	0\\
337	0\\
338	0\\
339	0\\
340	0\\
341	0\\
342	0\\
343	0\\
344	0\\
345	0\\
346	0\\
347	0\\
348	0\\
349	0\\
350	0\\
351	0\\
352	0\\
353	0\\
354	0\\
355	0\\
356	0\\
357	0\\
358	0\\
359	0\\
360	0\\
361	0\\
362	0\\
363	0\\
364	0\\
365	0\\
366	0\\
367	0\\
368	0\\
369	0\\
370	0\\
371	0\\
372	0\\
373	0\\
374	0\\
375	0\\
376	0\\
377	0\\
378	0\\
379	0\\
380	0\\
381	0\\
382	0\\
383	0\\
384	0\\
385	0\\
386	0\\
387	0\\
388	0\\
389	0\\
390	0\\
391	0\\
392	0\\
393	0\\
394	0\\
395	0\\
396	0\\
397	0\\
398	0\\
399	0\\
400	0\\
401	0\\
402	0\\
403	0\\
404	0\\
405	0\\
406	0\\
407	0\\
408	0\\
409	0\\
410	0\\
411	0\\
412	0\\
413	0\\
414	0\\
415	0\\
416	0\\
417	0\\
418	0\\
419	0\\
420	0\\
421	0\\
422	0\\
423	0\\
424	0\\
425	0\\
426	0\\
427	0\\
428	0\\
429	0\\
430	0\\
431	0\\
432	0\\
433	0\\
434	0\\
435	0\\
436	0\\
437	0\\
438	0\\
439	0\\
440	0\\
441	0\\
442	0\\
443	0\\
444	0\\
445	0\\
446	0\\
447	0\\
448	0\\
449	0\\
450	0\\
451	0\\
452	0\\
453	0\\
454	0\\
455	0\\
456	0\\
457	0\\
458	0\\
459	0\\
460	0\\
461	0\\
462	0\\
463	0\\
464	0\\
465	0\\
466	0\\
467	0\\
468	0\\
469	0\\
470	0\\
471	0\\
472	0\\
473	0\\
474	0\\
475	0\\
476	0\\
477	0\\
478	0\\
479	0\\
480	0\\
481	0\\
482	0\\
483	0\\
484	0\\
485	0\\
486	0\\
487	0\\
488	0\\
489	0\\
490	0\\
491	0\\
492	0\\
493	0\\
494	0\\
495	0\\
496	0\\
497	0\\
498	0\\
499	0\\
500	0\\
501	0\\
502	0\\
503	0\\
504	0\\
505	0\\
506	0\\
507	0\\
508	0\\
509	0\\
510	0\\
511	0\\
512	0\\
513	0\\
514	0\\
515	0\\
516	0\\
517	0\\
518	0\\
519	0\\
520	0\\
521	0\\
522	0\\
523	0\\
524	0\\
525	0\\
526	0\\
527	0\\
528	0\\
529	0\\
530	0\\
531	0\\
532	0\\
533	0\\
534	0\\
535	0\\
536	0\\
537	0\\
538	0\\
539	0\\
540	0\\
541	0\\
542	0\\
543	0\\
544	0\\
545	0\\
546	0\\
547	0\\
548	0\\
549	0\\
550	0\\
551	0\\
552	0\\
553	0\\
554	0\\
555	0\\
556	0\\
557	0\\
558	0\\
559	0\\
560	0\\
561	0\\
562	0\\
563	0\\
564	0\\
565	0\\
566	0\\
567	0\\
568	0\\
569	0\\
570	0\\
571	0\\
572	0\\
573	0\\
574	0\\
575	0\\
576	0\\
577	0\\
578	0\\
579	0\\
580	0\\
581	0\\
582	0\\
583	0\\
584	0\\
585	0\\
586	0\\
587	0\\
588	0\\
589	0\\
590	0\\
591	0\\
592	0\\
593	0\\
594	0\\
595	0\\
596	0\\
597	0\\
598	0\\
599	0\\
600	0\\
};
\addplot [color=mycolor7,solid,forget plot]
  table[row sep=crcr]{%
1	0\\
2	0\\
3	0\\
4	0\\
5	0\\
6	0\\
7	0\\
8	0\\
9	0\\
10	0\\
11	0\\
12	0\\
13	0\\
14	0\\
15	0\\
16	0\\
17	0\\
18	0\\
19	0\\
20	0\\
21	0\\
22	0\\
23	0\\
24	0\\
25	0\\
26	0\\
27	0\\
28	0\\
29	0\\
30	0\\
31	0\\
32	0\\
33	0\\
34	0\\
35	0\\
36	0\\
37	0\\
38	0\\
39	0\\
40	0\\
41	0\\
42	0\\
43	0\\
44	0\\
45	0\\
46	0\\
47	0\\
48	0\\
49	0\\
50	0\\
51	0\\
52	0\\
53	0\\
54	0\\
55	0\\
56	0\\
57	0\\
58	0\\
59	0\\
60	0\\
61	0\\
62	0\\
63	0\\
64	0\\
65	0\\
66	0\\
67	0\\
68	0\\
69	0\\
70	0\\
71	0\\
72	0\\
73	0\\
74	0\\
75	0\\
76	0\\
77	0\\
78	0\\
79	0\\
80	0\\
81	0\\
82	0\\
83	0\\
84	0\\
85	0\\
86	0\\
87	0\\
88	0\\
89	0\\
90	0\\
91	0\\
92	0\\
93	0\\
94	0\\
95	0\\
96	0\\
97	0\\
98	0\\
99	0\\
100	0\\
101	0\\
102	0\\
103	0\\
104	0\\
105	0\\
106	0\\
107	0\\
108	0\\
109	0\\
110	0\\
111	0\\
112	0\\
113	0\\
114	0\\
115	0\\
116	0\\
117	0\\
118	0\\
119	0\\
120	0\\
121	0\\
122	0\\
123	0\\
124	0\\
125	0\\
126	0\\
127	0\\
128	0\\
129	0\\
130	0\\
131	0\\
132	0\\
133	0\\
134	0\\
135	0\\
136	0\\
137	0\\
138	0\\
139	0\\
140	0\\
141	0\\
142	0\\
143	0\\
144	0\\
145	0\\
146	0\\
147	0\\
148	0\\
149	0\\
150	0\\
151	0\\
152	0\\
153	0\\
154	0\\
155	0\\
156	0\\
157	0\\
158	0\\
159	0\\
160	0\\
161	0\\
162	0\\
163	0\\
164	0\\
165	0\\
166	0\\
167	0\\
168	0\\
169	0\\
170	0\\
171	0\\
172	0\\
173	0\\
174	0\\
175	0\\
176	0\\
177	0\\
178	0\\
179	0\\
180	0\\
181	0\\
182	0\\
183	0\\
184	0\\
185	0\\
186	0\\
187	0\\
188	0\\
189	0\\
190	0\\
191	0\\
192	0\\
193	0\\
194	0\\
195	0\\
196	0\\
197	0\\
198	0\\
199	0\\
200	0\\
201	0\\
202	0\\
203	0\\
204	0\\
205	0\\
206	0\\
207	0\\
208	0\\
209	0\\
210	0\\
211	0\\
212	0\\
213	0\\
214	0\\
215	0\\
216	0\\
217	0\\
218	0\\
219	0\\
220	0\\
221	0\\
222	0\\
223	0\\
224	0\\
225	0\\
226	0\\
227	0\\
228	0\\
229	0\\
230	0\\
231	0\\
232	0\\
233	0\\
234	0\\
235	0\\
236	0\\
237	0\\
238	0\\
239	0\\
240	0\\
241	0\\
242	0\\
243	0\\
244	0\\
245	0\\
246	0\\
247	0\\
248	0\\
249	0\\
250	0\\
251	0\\
252	0\\
253	0\\
254	0\\
255	0\\
256	0\\
257	0\\
258	0\\
259	0\\
260	0\\
261	0\\
262	0\\
263	0\\
264	0\\
265	0\\
266	0\\
267	0\\
268	0\\
269	0\\
270	0\\
271	0\\
272	0\\
273	0\\
274	0\\
275	0\\
276	0\\
277	0\\
278	0\\
279	0\\
280	0\\
281	0\\
282	0\\
283	0\\
284	0\\
285	0\\
286	0\\
287	0\\
288	0\\
289	0\\
290	0\\
291	0\\
292	0\\
293	0\\
294	0\\
295	0\\
296	0\\
297	0\\
298	0\\
299	0\\
300	0\\
301	0\\
302	0\\
303	0\\
304	0\\
305	0\\
306	0\\
307	0\\
308	0\\
309	0\\
310	0\\
311	0\\
312	0\\
313	0\\
314	0\\
315	0\\
316	0\\
317	0\\
318	0\\
319	0\\
320	0\\
321	0\\
322	0\\
323	0\\
324	0\\
325	0\\
326	0\\
327	0\\
328	0\\
329	0\\
330	0\\
331	0\\
332	0\\
333	0\\
334	0\\
335	0\\
336	0\\
337	0\\
338	0\\
339	0\\
340	0\\
341	0\\
342	0\\
343	0\\
344	0\\
345	0\\
346	0\\
347	0\\
348	0\\
349	0\\
350	0\\
351	0\\
352	0\\
353	0\\
354	0\\
355	0\\
356	0\\
357	0\\
358	0\\
359	0\\
360	0\\
361	0\\
362	0\\
363	0\\
364	0\\
365	0\\
366	0\\
367	0\\
368	0\\
369	0\\
370	0\\
371	0\\
372	0\\
373	0\\
374	0\\
375	0\\
376	0\\
377	0\\
378	0\\
379	0\\
380	0\\
381	0\\
382	0\\
383	0\\
384	0\\
385	0\\
386	0\\
387	0\\
388	0\\
389	0\\
390	0\\
391	0\\
392	0\\
393	0\\
394	0\\
395	0\\
396	0\\
397	0\\
398	0\\
399	0\\
400	0\\
401	0\\
402	0\\
403	0\\
404	0\\
405	0\\
406	0\\
407	0\\
408	0\\
409	0\\
410	0\\
411	0\\
412	0\\
413	0\\
414	0\\
415	0\\
416	0\\
417	0\\
418	0\\
419	0\\
420	0\\
421	0\\
422	0\\
423	0\\
424	0\\
425	0\\
426	0\\
427	0\\
428	0\\
429	0\\
430	0\\
431	0\\
432	0\\
433	0\\
434	0\\
435	0\\
436	0\\
437	0\\
438	0\\
439	0\\
440	0\\
441	0\\
442	0\\
443	0\\
444	0\\
445	0\\
446	0\\
447	0\\
448	0\\
449	0\\
450	0\\
451	0\\
452	0\\
453	0\\
454	0\\
455	0\\
456	0\\
457	0\\
458	0\\
459	0\\
460	0\\
461	0\\
462	0\\
463	0\\
464	0\\
465	0\\
466	0\\
467	0\\
468	0\\
469	0\\
470	0\\
471	0\\
472	0\\
473	0\\
474	0\\
475	0\\
476	0\\
477	0\\
478	0\\
479	0\\
480	0\\
481	0\\
482	0\\
483	0\\
484	0\\
485	0\\
486	0\\
487	0\\
488	0\\
489	0\\
490	0\\
491	0\\
492	0\\
493	0\\
494	0\\
495	0\\
496	0\\
497	0\\
498	0\\
499	0\\
500	0\\
501	0\\
502	0\\
503	0\\
504	0\\
505	0\\
506	0\\
507	0\\
508	0\\
509	0\\
510	0\\
511	0\\
512	0\\
513	0\\
514	0\\
515	0\\
516	0\\
517	0\\
518	0\\
519	0\\
520	0\\
521	0\\
522	0\\
523	0\\
524	0\\
525	0\\
526	0\\
527	0\\
528	0\\
529	0\\
530	0\\
531	0\\
532	0\\
533	0\\
534	0\\
535	0\\
536	0\\
537	0\\
538	0\\
539	0\\
540	0\\
541	0\\
542	0\\
543	0\\
544	0\\
545	0\\
546	0\\
547	0\\
548	0\\
549	0\\
550	0\\
551	0\\
552	0\\
553	0\\
554	0\\
555	0\\
556	0\\
557	0\\
558	0\\
559	0\\
560	0\\
561	0\\
562	0\\
563	0\\
564	0\\
565	0\\
566	0\\
567	0\\
568	0\\
569	0\\
570	0\\
571	0\\
572	0\\
573	0\\
574	0\\
575	0\\
576	0\\
577	0\\
578	0\\
579	0\\
580	0\\
581	0\\
582	0\\
583	0\\
584	0\\
585	0\\
586	0\\
587	0\\
588	0\\
589	0\\
590	0\\
591	0\\
592	0\\
593	0\\
594	0\\
595	0\\
596	0\\
597	0\\
598	0\\
599	0\\
600	0\\
};
\addplot [color=mycolor8,solid,forget plot]
  table[row sep=crcr]{%
1	0\\
2	0\\
3	0\\
4	0\\
5	0\\
6	0\\
7	0\\
8	0\\
9	0\\
10	0\\
11	0\\
12	0\\
13	0\\
14	0\\
15	0\\
16	0\\
17	0\\
18	0\\
19	0\\
20	0\\
21	0\\
22	0\\
23	0\\
24	0\\
25	0\\
26	0\\
27	0\\
28	0\\
29	0\\
30	0\\
31	0\\
32	0\\
33	0\\
34	0\\
35	0\\
36	0\\
37	0\\
38	0\\
39	0\\
40	0\\
41	0\\
42	0\\
43	0\\
44	0\\
45	0\\
46	0\\
47	0\\
48	0\\
49	0\\
50	0\\
51	0\\
52	0\\
53	0\\
54	0\\
55	0\\
56	0\\
57	0\\
58	0\\
59	0\\
60	0\\
61	0\\
62	0\\
63	0\\
64	0\\
65	0\\
66	0\\
67	0\\
68	0\\
69	0\\
70	0\\
71	0\\
72	0\\
73	0\\
74	0\\
75	0\\
76	0\\
77	0\\
78	0\\
79	0\\
80	0\\
81	0\\
82	0\\
83	0\\
84	0\\
85	0\\
86	0\\
87	0\\
88	0\\
89	0\\
90	0\\
91	0\\
92	0\\
93	0\\
94	0\\
95	0\\
96	0\\
97	0\\
98	0\\
99	0\\
100	0\\
101	0\\
102	0\\
103	0\\
104	0\\
105	0\\
106	0\\
107	0\\
108	0\\
109	0\\
110	0\\
111	0\\
112	0\\
113	0\\
114	0\\
115	0\\
116	0\\
117	0\\
118	0\\
119	0\\
120	0\\
121	0\\
122	0\\
123	0\\
124	0\\
125	0\\
126	0\\
127	0\\
128	0\\
129	0\\
130	0\\
131	0\\
132	0\\
133	0\\
134	0\\
135	0\\
136	0\\
137	0\\
138	0\\
139	0\\
140	0\\
141	0\\
142	0\\
143	0\\
144	0\\
145	0\\
146	0\\
147	0\\
148	0\\
149	0\\
150	0\\
151	0\\
152	0\\
153	0\\
154	0\\
155	0\\
156	0\\
157	0\\
158	0\\
159	0\\
160	0\\
161	0\\
162	0\\
163	0\\
164	0\\
165	0\\
166	0\\
167	0\\
168	0\\
169	0\\
170	0\\
171	0\\
172	0\\
173	0\\
174	0\\
175	0\\
176	0\\
177	0\\
178	0\\
179	0\\
180	0\\
181	0\\
182	0\\
183	0\\
184	0\\
185	0\\
186	0\\
187	0\\
188	0\\
189	0\\
190	0\\
191	0\\
192	0\\
193	0\\
194	0\\
195	0\\
196	0\\
197	0\\
198	0\\
199	0\\
200	0\\
201	0\\
202	0\\
203	0\\
204	0\\
205	0\\
206	0\\
207	0\\
208	0\\
209	0\\
210	0\\
211	0\\
212	0\\
213	0\\
214	0\\
215	0\\
216	0\\
217	0\\
218	0\\
219	0\\
220	0\\
221	0\\
222	0\\
223	0\\
224	0\\
225	0\\
226	0\\
227	0\\
228	0\\
229	0\\
230	0\\
231	0\\
232	0\\
233	0\\
234	0\\
235	0\\
236	0\\
237	0\\
238	0\\
239	0\\
240	0\\
241	0\\
242	0\\
243	0\\
244	0\\
245	0\\
246	0\\
247	0\\
248	0\\
249	0\\
250	0\\
251	0\\
252	0\\
253	0\\
254	0\\
255	0\\
256	0\\
257	0\\
258	0\\
259	0\\
260	0\\
261	0\\
262	0\\
263	0\\
264	0\\
265	0\\
266	0\\
267	0\\
268	0\\
269	0\\
270	0\\
271	0\\
272	0\\
273	0\\
274	0\\
275	0\\
276	0\\
277	0\\
278	0\\
279	0\\
280	0\\
281	0\\
282	0\\
283	0\\
284	0\\
285	0\\
286	0\\
287	0\\
288	0\\
289	0\\
290	0\\
291	0\\
292	0\\
293	0\\
294	0\\
295	0\\
296	0\\
297	0\\
298	0\\
299	0\\
300	0\\
301	0\\
302	0\\
303	0\\
304	0\\
305	0\\
306	0\\
307	0\\
308	0\\
309	0\\
310	0\\
311	0\\
312	0\\
313	0\\
314	0\\
315	0\\
316	0\\
317	0\\
318	0\\
319	0\\
320	0\\
321	0\\
322	0\\
323	0\\
324	0\\
325	0\\
326	0\\
327	0\\
328	0\\
329	0\\
330	0\\
331	0\\
332	0\\
333	0\\
334	0\\
335	0\\
336	0\\
337	0\\
338	0\\
339	0\\
340	0\\
341	0\\
342	0\\
343	0\\
344	0\\
345	0\\
346	0\\
347	0\\
348	0\\
349	0\\
350	0\\
351	0\\
352	0\\
353	0\\
354	0\\
355	0\\
356	0\\
357	0\\
358	0\\
359	0\\
360	0\\
361	0\\
362	0\\
363	0\\
364	0\\
365	0\\
366	0\\
367	0\\
368	0\\
369	0\\
370	0\\
371	0\\
372	0\\
373	0\\
374	0\\
375	0\\
376	0\\
377	0\\
378	0\\
379	0\\
380	0\\
381	0\\
382	0\\
383	0\\
384	0\\
385	0\\
386	0\\
387	0\\
388	0\\
389	0\\
390	0\\
391	0\\
392	0\\
393	0\\
394	0\\
395	0\\
396	0\\
397	0\\
398	0\\
399	0\\
400	0\\
401	0\\
402	0\\
403	0\\
404	0\\
405	0\\
406	0\\
407	0\\
408	0\\
409	0\\
410	0\\
411	0\\
412	0\\
413	0\\
414	0\\
415	0\\
416	0\\
417	0\\
418	0\\
419	0\\
420	0\\
421	0\\
422	0\\
423	0\\
424	0\\
425	0\\
426	0\\
427	0\\
428	0\\
429	0\\
430	0\\
431	0\\
432	0\\
433	0\\
434	0\\
435	0\\
436	0\\
437	0\\
438	0\\
439	0\\
440	0\\
441	0\\
442	0\\
443	0\\
444	0\\
445	0\\
446	0\\
447	0\\
448	0\\
449	0\\
450	0\\
451	0\\
452	0\\
453	0\\
454	0\\
455	0\\
456	0\\
457	0\\
458	0\\
459	0\\
460	0\\
461	0\\
462	0\\
463	0\\
464	0\\
465	0\\
466	0\\
467	0\\
468	0\\
469	0\\
470	0\\
471	0\\
472	0\\
473	0\\
474	0\\
475	0\\
476	0\\
477	0\\
478	0\\
479	0\\
480	0\\
481	0\\
482	0\\
483	0\\
484	0\\
485	0\\
486	0\\
487	0\\
488	0\\
489	0\\
490	0\\
491	0\\
492	0\\
493	0\\
494	0\\
495	0\\
496	0\\
497	0\\
498	0\\
499	0\\
500	0\\
501	0\\
502	0\\
503	0\\
504	0\\
505	0\\
506	0\\
507	0\\
508	0\\
509	0\\
510	0\\
511	0\\
512	0\\
513	0\\
514	0\\
515	0\\
516	0\\
517	0\\
518	0\\
519	0\\
520	0\\
521	0\\
522	0\\
523	0\\
524	0\\
525	0\\
526	0\\
527	0\\
528	0\\
529	0\\
530	0\\
531	0\\
532	0\\
533	0\\
534	0\\
535	0\\
536	0\\
537	0\\
538	0\\
539	0\\
540	0\\
541	0\\
542	0\\
543	0\\
544	0\\
545	0\\
546	0\\
547	0\\
548	0\\
549	0\\
550	0\\
551	0\\
552	0\\
553	0\\
554	0\\
555	0\\
556	0\\
557	0\\
558	0\\
559	0\\
560	0\\
561	0\\
562	0\\
563	0\\
564	0\\
565	0\\
566	0\\
567	0\\
568	0\\
569	0\\
570	0\\
571	0\\
572	0\\
573	0\\
574	0\\
575	0\\
576	0\\
577	0\\
578	0\\
579	0\\
580	0\\
581	0\\
582	0\\
583	0\\
584	0\\
585	0\\
586	0\\
587	0\\
588	0\\
589	0\\
590	0\\
591	0\\
592	0\\
593	0\\
594	0\\
595	0\\
596	0\\
597	0\\
598	0\\
599	0\\
600	0\\
};
\addplot [color=blue!25!mycolor7,solid,forget plot]
  table[row sep=crcr]{%
1	0\\
2	0\\
3	0\\
4	0\\
5	0\\
6	0\\
7	0\\
8	0\\
9	0\\
10	0\\
11	0\\
12	0\\
13	0\\
14	0\\
15	0\\
16	0\\
17	0\\
18	0\\
19	0\\
20	0\\
21	0\\
22	0\\
23	0\\
24	0\\
25	0\\
26	0\\
27	0\\
28	0\\
29	0\\
30	0\\
31	0\\
32	0\\
33	0\\
34	0\\
35	0\\
36	0\\
37	0\\
38	0\\
39	0\\
40	0\\
41	0\\
42	0\\
43	0\\
44	0\\
45	0\\
46	0\\
47	0\\
48	0\\
49	0\\
50	0\\
51	0\\
52	0\\
53	0\\
54	0\\
55	0\\
56	0\\
57	0\\
58	0\\
59	0\\
60	0\\
61	0\\
62	0\\
63	0\\
64	0\\
65	0\\
66	0\\
67	0\\
68	0\\
69	0\\
70	0\\
71	0\\
72	0\\
73	0\\
74	0\\
75	0\\
76	0\\
77	0\\
78	0\\
79	0\\
80	0\\
81	0\\
82	0\\
83	0\\
84	0\\
85	0\\
86	0\\
87	0\\
88	0\\
89	0\\
90	0\\
91	0\\
92	0\\
93	0\\
94	0\\
95	0\\
96	0\\
97	0\\
98	0\\
99	0\\
100	0\\
101	0\\
102	0\\
103	0\\
104	0\\
105	0\\
106	0\\
107	0\\
108	0\\
109	0\\
110	0\\
111	0\\
112	0\\
113	0\\
114	0\\
115	0\\
116	0\\
117	0\\
118	0\\
119	0\\
120	0\\
121	0\\
122	0\\
123	0\\
124	0\\
125	0\\
126	0\\
127	0\\
128	0\\
129	0\\
130	0\\
131	0\\
132	0\\
133	0\\
134	0\\
135	0\\
136	0\\
137	0\\
138	0\\
139	0\\
140	0\\
141	0\\
142	0\\
143	0\\
144	0\\
145	0\\
146	0\\
147	0\\
148	0\\
149	0\\
150	0\\
151	0\\
152	0\\
153	0\\
154	0\\
155	0\\
156	0\\
157	0\\
158	0\\
159	0\\
160	0\\
161	0\\
162	0\\
163	0\\
164	0\\
165	0\\
166	0\\
167	0\\
168	0\\
169	0\\
170	0\\
171	0\\
172	0\\
173	0\\
174	0\\
175	0\\
176	0\\
177	0\\
178	0\\
179	0\\
180	0\\
181	0\\
182	0\\
183	0\\
184	0\\
185	0\\
186	0\\
187	0\\
188	0\\
189	0\\
190	0\\
191	0\\
192	0\\
193	0\\
194	0\\
195	0\\
196	0\\
197	0\\
198	0\\
199	0\\
200	0\\
201	0\\
202	0\\
203	0\\
204	0\\
205	0\\
206	0\\
207	0\\
208	0\\
209	0\\
210	0\\
211	0\\
212	0\\
213	0\\
214	0\\
215	0\\
216	0\\
217	0\\
218	0\\
219	0\\
220	0\\
221	0\\
222	0\\
223	0\\
224	0\\
225	0\\
226	0\\
227	0\\
228	0\\
229	0\\
230	0\\
231	0\\
232	0\\
233	0\\
234	0\\
235	0\\
236	0\\
237	0\\
238	0\\
239	0\\
240	0\\
241	0\\
242	0\\
243	0\\
244	0\\
245	0\\
246	0\\
247	0\\
248	0\\
249	0\\
250	0\\
251	0\\
252	0\\
253	0\\
254	0\\
255	0\\
256	0\\
257	0\\
258	0\\
259	0\\
260	0\\
261	0\\
262	0\\
263	0\\
264	0\\
265	0\\
266	0\\
267	0\\
268	0\\
269	0\\
270	0\\
271	0\\
272	0\\
273	0\\
274	0\\
275	0\\
276	0\\
277	0\\
278	0\\
279	0\\
280	0\\
281	0\\
282	0\\
283	0\\
284	0\\
285	0\\
286	0\\
287	0\\
288	0\\
289	0\\
290	0\\
291	0\\
292	0\\
293	0\\
294	0\\
295	0\\
296	0\\
297	0\\
298	0\\
299	0\\
300	0\\
301	0\\
302	0\\
303	0\\
304	0\\
305	0\\
306	0\\
307	0\\
308	0\\
309	0\\
310	0\\
311	0\\
312	0\\
313	0\\
314	0\\
315	0\\
316	0\\
317	0\\
318	0\\
319	0\\
320	0\\
321	0\\
322	0\\
323	0\\
324	0\\
325	0\\
326	0\\
327	0\\
328	0\\
329	0\\
330	0\\
331	0\\
332	0\\
333	0\\
334	0\\
335	0\\
336	0\\
337	0\\
338	0\\
339	0\\
340	0\\
341	0\\
342	0\\
343	0\\
344	0\\
345	0\\
346	0\\
347	0\\
348	0\\
349	0\\
350	0\\
351	0\\
352	0\\
353	0\\
354	0\\
355	0\\
356	0\\
357	0\\
358	0\\
359	0\\
360	0\\
361	0\\
362	0\\
363	0\\
364	0\\
365	0\\
366	0\\
367	0\\
368	0\\
369	0\\
370	0\\
371	0\\
372	0\\
373	0\\
374	0\\
375	0\\
376	0\\
377	0\\
378	0\\
379	0\\
380	0\\
381	0\\
382	0\\
383	0\\
384	0\\
385	0\\
386	0\\
387	0\\
388	0\\
389	0\\
390	0\\
391	0\\
392	0\\
393	0\\
394	0\\
395	0\\
396	0\\
397	0\\
398	0\\
399	0\\
400	0\\
401	0\\
402	0\\
403	0\\
404	0\\
405	0\\
406	0\\
407	0\\
408	0\\
409	0\\
410	0\\
411	0\\
412	0\\
413	0\\
414	0\\
415	0\\
416	0\\
417	0\\
418	0\\
419	0\\
420	0\\
421	0\\
422	0\\
423	0\\
424	0\\
425	0\\
426	0\\
427	0\\
428	0\\
429	0\\
430	0\\
431	0\\
432	0\\
433	0\\
434	0\\
435	0\\
436	0\\
437	0\\
438	0\\
439	0\\
440	0\\
441	0\\
442	0\\
443	0\\
444	0\\
445	0\\
446	0\\
447	0\\
448	0\\
449	0\\
450	0\\
451	0\\
452	0\\
453	0\\
454	0\\
455	0\\
456	0\\
457	0\\
458	0\\
459	0\\
460	0\\
461	0\\
462	0\\
463	0\\
464	0\\
465	0\\
466	0\\
467	0\\
468	0\\
469	0\\
470	0\\
471	0\\
472	0\\
473	0\\
474	0\\
475	0\\
476	0\\
477	0\\
478	0\\
479	0\\
480	0\\
481	0\\
482	0\\
483	0\\
484	0\\
485	0\\
486	0\\
487	0\\
488	0\\
489	0\\
490	0\\
491	0\\
492	0\\
493	0\\
494	0\\
495	0\\
496	0\\
497	0\\
498	0\\
499	0\\
500	0\\
501	0\\
502	0\\
503	0\\
504	0\\
505	0\\
506	0\\
507	0\\
508	0\\
509	0\\
510	0\\
511	0\\
512	0\\
513	0\\
514	0\\
515	0\\
516	0\\
517	0\\
518	0\\
519	0\\
520	0\\
521	0\\
522	0\\
523	0\\
524	0\\
525	0\\
526	0\\
527	0\\
528	0\\
529	0\\
530	0\\
531	0\\
532	0\\
533	0\\
534	0\\
535	0\\
536	0\\
537	0\\
538	0\\
539	0\\
540	0\\
541	0\\
542	0\\
543	0\\
544	0\\
545	0\\
546	0\\
547	0\\
548	0\\
549	0\\
550	0\\
551	0\\
552	0\\
553	0\\
554	0\\
555	0\\
556	0\\
557	0\\
558	0\\
559	0\\
560	0\\
561	0\\
562	0\\
563	0\\
564	0\\
565	0\\
566	0\\
567	0\\
568	0\\
569	0\\
570	0\\
571	0\\
572	0\\
573	0\\
574	0\\
575	0\\
576	0\\
577	0\\
578	0\\
579	0\\
580	0\\
581	0\\
582	0\\
583	0\\
584	0\\
585	0\\
586	0\\
587	0\\
588	0\\
589	0\\
590	0\\
591	0\\
592	0\\
593	0\\
594	0\\
595	0\\
596	0\\
597	0\\
598	0\\
599	0\\
600	0\\
};
\addplot [color=mycolor9,solid,forget plot]
  table[row sep=crcr]{%
1	0\\
2	0\\
3	0\\
4	0\\
5	0\\
6	0\\
7	0\\
8	0\\
9	0\\
10	0\\
11	0\\
12	0\\
13	0\\
14	0\\
15	0\\
16	0\\
17	0\\
18	0\\
19	0\\
20	0\\
21	0\\
22	0\\
23	0\\
24	0\\
25	0\\
26	0\\
27	0\\
28	0\\
29	0\\
30	0\\
31	0\\
32	0\\
33	0\\
34	0\\
35	0\\
36	0\\
37	0\\
38	0\\
39	0\\
40	0\\
41	0\\
42	0\\
43	0\\
44	0\\
45	0\\
46	0\\
47	0\\
48	0\\
49	0\\
50	0\\
51	0\\
52	0\\
53	0\\
54	0\\
55	0\\
56	0\\
57	0\\
58	0\\
59	0\\
60	0\\
61	0\\
62	0\\
63	0\\
64	0\\
65	0\\
66	0\\
67	0\\
68	0\\
69	0\\
70	0\\
71	0\\
72	0\\
73	0\\
74	0\\
75	0\\
76	0\\
77	0\\
78	0\\
79	0\\
80	0\\
81	0\\
82	0\\
83	0\\
84	0\\
85	0\\
86	0\\
87	0\\
88	0\\
89	0\\
90	0\\
91	0\\
92	0\\
93	0\\
94	0\\
95	0\\
96	0\\
97	0\\
98	0\\
99	0\\
100	0\\
101	0\\
102	0\\
103	0\\
104	0\\
105	0\\
106	0\\
107	0\\
108	0\\
109	0\\
110	0\\
111	0\\
112	0\\
113	0\\
114	0\\
115	0\\
116	0\\
117	0\\
118	0\\
119	0\\
120	0\\
121	0\\
122	0\\
123	0\\
124	0\\
125	0\\
126	0\\
127	0\\
128	0\\
129	0\\
130	0\\
131	0\\
132	0\\
133	0\\
134	0\\
135	0\\
136	0\\
137	0\\
138	0\\
139	0\\
140	0\\
141	0\\
142	0\\
143	0\\
144	0\\
145	0\\
146	0\\
147	0\\
148	0\\
149	0\\
150	0\\
151	0\\
152	0\\
153	0\\
154	0\\
155	0\\
156	0\\
157	0\\
158	0\\
159	0\\
160	0\\
161	0\\
162	0\\
163	0\\
164	0\\
165	0\\
166	0\\
167	0\\
168	0\\
169	0\\
170	0\\
171	0\\
172	0\\
173	0\\
174	0\\
175	0\\
176	0\\
177	0\\
178	0\\
179	0\\
180	0\\
181	0\\
182	0\\
183	0\\
184	0\\
185	0\\
186	0\\
187	0\\
188	0\\
189	0\\
190	0\\
191	0\\
192	0\\
193	0\\
194	0\\
195	0\\
196	0\\
197	0\\
198	0\\
199	0\\
200	0\\
201	0\\
202	0\\
203	0\\
204	0\\
205	0\\
206	0\\
207	0\\
208	0\\
209	0\\
210	0\\
211	0\\
212	0\\
213	0\\
214	0\\
215	0\\
216	0\\
217	0\\
218	0\\
219	0\\
220	0\\
221	0\\
222	0\\
223	0\\
224	0\\
225	0\\
226	0\\
227	0\\
228	0\\
229	0\\
230	0\\
231	0\\
232	0\\
233	0\\
234	0\\
235	0\\
236	0\\
237	0\\
238	0\\
239	0\\
240	0\\
241	0\\
242	0\\
243	0\\
244	0\\
245	0\\
246	0\\
247	0\\
248	0\\
249	0\\
250	0\\
251	0\\
252	0\\
253	0\\
254	0\\
255	0\\
256	0\\
257	0\\
258	0\\
259	0\\
260	0\\
261	0\\
262	0\\
263	0\\
264	0\\
265	0\\
266	0\\
267	0\\
268	0\\
269	0\\
270	0\\
271	0\\
272	0\\
273	0\\
274	0\\
275	0\\
276	0\\
277	0\\
278	0\\
279	0\\
280	0\\
281	0\\
282	0\\
283	0\\
284	0\\
285	0\\
286	0\\
287	0\\
288	0\\
289	0\\
290	0\\
291	0\\
292	0\\
293	0\\
294	0\\
295	0\\
296	0\\
297	0\\
298	0\\
299	0\\
300	0\\
301	0\\
302	0\\
303	0\\
304	0\\
305	0\\
306	0\\
307	0\\
308	0\\
309	0\\
310	0\\
311	0\\
312	0\\
313	0\\
314	0\\
315	0\\
316	0\\
317	0\\
318	0\\
319	0\\
320	0\\
321	0\\
322	0\\
323	0\\
324	0\\
325	0\\
326	0\\
327	0\\
328	0\\
329	0\\
330	0\\
331	0\\
332	0\\
333	0\\
334	0\\
335	0\\
336	0\\
337	0\\
338	0\\
339	0\\
340	0\\
341	0\\
342	0\\
343	0\\
344	0\\
345	0\\
346	0\\
347	0\\
348	0\\
349	0\\
350	0\\
351	0\\
352	0\\
353	0\\
354	0\\
355	0\\
356	0\\
357	0\\
358	0\\
359	0\\
360	0\\
361	0\\
362	0\\
363	0\\
364	0\\
365	0\\
366	0\\
367	0\\
368	0\\
369	0\\
370	0\\
371	0\\
372	0\\
373	0\\
374	0\\
375	0\\
376	0\\
377	0\\
378	0\\
379	0\\
380	0\\
381	0\\
382	0\\
383	0\\
384	0\\
385	0\\
386	0\\
387	0\\
388	0\\
389	0\\
390	0\\
391	0\\
392	0\\
393	0\\
394	0\\
395	0\\
396	0\\
397	0\\
398	0\\
399	0\\
400	0\\
401	0\\
402	0\\
403	0\\
404	0\\
405	0\\
406	0\\
407	0\\
408	0\\
409	0\\
410	0\\
411	0\\
412	0\\
413	0\\
414	0\\
415	0\\
416	0\\
417	0\\
418	0\\
419	0\\
420	0\\
421	0\\
422	0\\
423	0\\
424	0\\
425	0\\
426	0\\
427	0\\
428	0\\
429	0\\
430	0\\
431	0\\
432	0\\
433	0\\
434	0\\
435	0\\
436	0\\
437	0\\
438	0\\
439	0\\
440	0\\
441	0\\
442	0\\
443	0\\
444	0\\
445	0\\
446	0\\
447	0\\
448	0\\
449	0\\
450	0\\
451	0\\
452	0\\
453	0\\
454	0\\
455	0\\
456	0\\
457	0\\
458	0\\
459	0\\
460	0\\
461	0\\
462	0\\
463	0\\
464	0\\
465	0\\
466	0\\
467	0\\
468	0\\
469	0\\
470	0\\
471	0\\
472	0\\
473	0\\
474	0\\
475	0\\
476	0\\
477	0\\
478	0\\
479	0\\
480	0\\
481	0\\
482	0\\
483	0\\
484	0\\
485	0\\
486	0\\
487	0\\
488	0\\
489	0\\
490	0\\
491	0\\
492	0\\
493	0\\
494	0\\
495	0\\
496	0\\
497	0\\
498	0\\
499	0\\
500	0\\
501	0\\
502	0\\
503	0\\
504	0\\
505	0\\
506	0\\
507	0\\
508	0\\
509	0\\
510	0\\
511	0\\
512	0\\
513	0\\
514	0\\
515	0\\
516	0\\
517	0\\
518	0\\
519	0\\
520	0\\
521	0\\
522	0\\
523	0\\
524	0\\
525	0\\
526	0\\
527	0\\
528	0\\
529	0\\
530	0\\
531	0\\
532	0\\
533	0\\
534	0\\
535	0\\
536	0\\
537	0\\
538	0\\
539	0\\
540	0\\
541	0\\
542	0\\
543	0\\
544	0\\
545	0\\
546	0\\
547	0\\
548	0\\
549	0\\
550	0\\
551	0\\
552	0\\
553	0\\
554	0\\
555	0\\
556	0\\
557	0\\
558	0\\
559	0\\
560	0\\
561	0\\
562	0\\
563	0\\
564	0\\
565	0\\
566	0\\
567	0\\
568	0\\
569	0\\
570	0\\
571	0\\
572	0\\
573	0\\
574	0\\
575	0\\
576	0\\
577	0\\
578	0\\
579	0\\
580	0\\
581	0\\
582	0\\
583	0\\
584	0\\
585	0\\
586	0\\
587	0\\
588	0\\
589	0\\
590	0\\
591	0\\
592	0\\
593	0\\
594	0\\
595	0\\
596	0\\
597	0\\
598	0\\
599	0\\
600	0\\
};
\addplot [color=blue!50!mycolor7,solid,forget plot]
  table[row sep=crcr]{%
1	0\\
2	0\\
3	0\\
4	0\\
5	0\\
6	0\\
7	0\\
8	0\\
9	0\\
10	0\\
11	0\\
12	0\\
13	0\\
14	0\\
15	0\\
16	0\\
17	0\\
18	0\\
19	0\\
20	0\\
21	0\\
22	0\\
23	0\\
24	0\\
25	0\\
26	0\\
27	0\\
28	0\\
29	0\\
30	0\\
31	0\\
32	0\\
33	0\\
34	0\\
35	0\\
36	0\\
37	0\\
38	0\\
39	0\\
40	0\\
41	0\\
42	0\\
43	0\\
44	0\\
45	0\\
46	0\\
47	0\\
48	0\\
49	0\\
50	0\\
51	0\\
52	0\\
53	0\\
54	0\\
55	0\\
56	0\\
57	0\\
58	0\\
59	0\\
60	0\\
61	0\\
62	0\\
63	0\\
64	0\\
65	0\\
66	0\\
67	0\\
68	0\\
69	0\\
70	0\\
71	0\\
72	0\\
73	0\\
74	0\\
75	0\\
76	0\\
77	0\\
78	0\\
79	0\\
80	0\\
81	0\\
82	0\\
83	0\\
84	0\\
85	0\\
86	0\\
87	0\\
88	0\\
89	0\\
90	0\\
91	0\\
92	0\\
93	0\\
94	0\\
95	0\\
96	0\\
97	0\\
98	0\\
99	0\\
100	0\\
101	0\\
102	0\\
103	0\\
104	0\\
105	0\\
106	0\\
107	0\\
108	0\\
109	0\\
110	0\\
111	0\\
112	0\\
113	0\\
114	0\\
115	0\\
116	0\\
117	0\\
118	0\\
119	0\\
120	0\\
121	0\\
122	0\\
123	0\\
124	0\\
125	0\\
126	0\\
127	0\\
128	0\\
129	0\\
130	0\\
131	0\\
132	0\\
133	0\\
134	0\\
135	0\\
136	0\\
137	0\\
138	0\\
139	0\\
140	0\\
141	0\\
142	0\\
143	0\\
144	0\\
145	0\\
146	0\\
147	0\\
148	0\\
149	0\\
150	0\\
151	0\\
152	0\\
153	0\\
154	0\\
155	0\\
156	0\\
157	0\\
158	0\\
159	0\\
160	0\\
161	0\\
162	0\\
163	0\\
164	0\\
165	0\\
166	0\\
167	0\\
168	0\\
169	0\\
170	0\\
171	0\\
172	0\\
173	0\\
174	0\\
175	0\\
176	0\\
177	0\\
178	0\\
179	0\\
180	0\\
181	0\\
182	0\\
183	0\\
184	0\\
185	0\\
186	0\\
187	0\\
188	0\\
189	0\\
190	0\\
191	0\\
192	0\\
193	0\\
194	0\\
195	0\\
196	0\\
197	0\\
198	0\\
199	0\\
200	0\\
201	0\\
202	0\\
203	0\\
204	0\\
205	0\\
206	0\\
207	0\\
208	0\\
209	0\\
210	0\\
211	0\\
212	0\\
213	0\\
214	0\\
215	0\\
216	0\\
217	0\\
218	0\\
219	0\\
220	0\\
221	0\\
222	0\\
223	0\\
224	0\\
225	0\\
226	0\\
227	0\\
228	0\\
229	0\\
230	0\\
231	0\\
232	0\\
233	0\\
234	0\\
235	0\\
236	0\\
237	0\\
238	0\\
239	0\\
240	0\\
241	0\\
242	0\\
243	0\\
244	0\\
245	0\\
246	0\\
247	0\\
248	0\\
249	0\\
250	0\\
251	0\\
252	0\\
253	0\\
254	0\\
255	0\\
256	0\\
257	0\\
258	0\\
259	0\\
260	0\\
261	0\\
262	0\\
263	0\\
264	0\\
265	0\\
266	0\\
267	0\\
268	0\\
269	0\\
270	0\\
271	0\\
272	0\\
273	0\\
274	0\\
275	0\\
276	0\\
277	0\\
278	0\\
279	0\\
280	0\\
281	0\\
282	0\\
283	0\\
284	0\\
285	0\\
286	0\\
287	0\\
288	0\\
289	0\\
290	0\\
291	0\\
292	0\\
293	0\\
294	0\\
295	0\\
296	0\\
297	0\\
298	0\\
299	0\\
300	0\\
301	0\\
302	0\\
303	0\\
304	0\\
305	0\\
306	0\\
307	0\\
308	0\\
309	0\\
310	0\\
311	0\\
312	0\\
313	0\\
314	0\\
315	0\\
316	0\\
317	0\\
318	0\\
319	0\\
320	0\\
321	0\\
322	0\\
323	0\\
324	0\\
325	0\\
326	0\\
327	0\\
328	0\\
329	0\\
330	0\\
331	0\\
332	0\\
333	0\\
334	0\\
335	0\\
336	0\\
337	0\\
338	0\\
339	0\\
340	0\\
341	0\\
342	0\\
343	0\\
344	0\\
345	0\\
346	0\\
347	0\\
348	0\\
349	0\\
350	0\\
351	0\\
352	0\\
353	0\\
354	0\\
355	0\\
356	0\\
357	0\\
358	0\\
359	0\\
360	0\\
361	0\\
362	0\\
363	0\\
364	0\\
365	0\\
366	0\\
367	0\\
368	0\\
369	0\\
370	0\\
371	0\\
372	0\\
373	0\\
374	0\\
375	0\\
376	0\\
377	0\\
378	0\\
379	0\\
380	0\\
381	0\\
382	0\\
383	0\\
384	0\\
385	0\\
386	0\\
387	0\\
388	0\\
389	0\\
390	0\\
391	0\\
392	0\\
393	0\\
394	0\\
395	0\\
396	0\\
397	0\\
398	0\\
399	0\\
400	0\\
401	0\\
402	0\\
403	0\\
404	0\\
405	0\\
406	0\\
407	0\\
408	0\\
409	0\\
410	0\\
411	0\\
412	0\\
413	0\\
414	0\\
415	0\\
416	0\\
417	0\\
418	0\\
419	0\\
420	0\\
421	0\\
422	0\\
423	0\\
424	0\\
425	0\\
426	0\\
427	0\\
428	0\\
429	0\\
430	0\\
431	0\\
432	0\\
433	0\\
434	0\\
435	0\\
436	0\\
437	0\\
438	0\\
439	0\\
440	0\\
441	0\\
442	0\\
443	0\\
444	0\\
445	0\\
446	0\\
447	0\\
448	0\\
449	0\\
450	0\\
451	0\\
452	0\\
453	0\\
454	0\\
455	0\\
456	0\\
457	0\\
458	0\\
459	0\\
460	0\\
461	0\\
462	0\\
463	0\\
464	0\\
465	0\\
466	0\\
467	0\\
468	0\\
469	0\\
470	0\\
471	0\\
472	0\\
473	0\\
474	0\\
475	0\\
476	0\\
477	0\\
478	0\\
479	0\\
480	0\\
481	0\\
482	0\\
483	0\\
484	0\\
485	0\\
486	0\\
487	0\\
488	0\\
489	0\\
490	0\\
491	0\\
492	0\\
493	0\\
494	0\\
495	0\\
496	0\\
497	0\\
498	0\\
499	0\\
500	0\\
501	0\\
502	0\\
503	0\\
504	0\\
505	0\\
506	0\\
507	0\\
508	0\\
509	0\\
510	0\\
511	0\\
512	0\\
513	0\\
514	0\\
515	0\\
516	0\\
517	0\\
518	0\\
519	0\\
520	0\\
521	0\\
522	0\\
523	0\\
524	0\\
525	0\\
526	0\\
527	0\\
528	0\\
529	0\\
530	0\\
531	0\\
532	0\\
533	0\\
534	0\\
535	0\\
536	0\\
537	0\\
538	0\\
539	0\\
540	0\\
541	0\\
542	0\\
543	0\\
544	0\\
545	0\\
546	0\\
547	0\\
548	0\\
549	0\\
550	0\\
551	0\\
552	0\\
553	0\\
554	0\\
555	0\\
556	0\\
557	0\\
558	0\\
559	0\\
560	0\\
561	0\\
562	0\\
563	0\\
564	0\\
565	0\\
566	0\\
567	0\\
568	0\\
569	0\\
570	0\\
571	0\\
572	0\\
573	0\\
574	0\\
575	0\\
576	0\\
577	0\\
578	0\\
579	0\\
580	0\\
581	0\\
582	0\\
583	0\\
584	0\\
585	0\\
586	0\\
587	0\\
588	0\\
589	0\\
590	0\\
591	0\\
592	0\\
593	0\\
594	0\\
595	0\\
596	0\\
597	0\\
598	0\\
599	0\\
600	0\\
};
\addplot [color=blue!40!mycolor9,solid,forget plot]
  table[row sep=crcr]{%
1	0\\
2	0\\
3	0\\
4	0\\
5	0\\
6	0\\
7	0\\
8	0\\
9	0\\
10	0\\
11	0\\
12	0\\
13	0\\
14	0\\
15	0\\
16	0\\
17	0\\
18	0\\
19	0\\
20	0\\
21	0\\
22	0\\
23	0\\
24	0\\
25	0\\
26	0\\
27	0\\
28	0\\
29	0\\
30	0\\
31	0\\
32	0\\
33	0\\
34	0\\
35	0\\
36	0\\
37	0\\
38	0\\
39	0\\
40	0\\
41	0\\
42	0\\
43	0\\
44	0\\
45	0\\
46	0\\
47	0\\
48	0\\
49	0\\
50	0\\
51	0\\
52	0\\
53	0\\
54	0\\
55	0\\
56	0\\
57	0\\
58	0\\
59	0\\
60	0\\
61	0\\
62	0\\
63	0\\
64	0\\
65	0\\
66	0\\
67	0\\
68	0\\
69	0\\
70	0\\
71	0\\
72	0\\
73	0\\
74	0\\
75	0\\
76	0\\
77	0\\
78	0\\
79	0\\
80	0\\
81	0\\
82	0\\
83	0\\
84	0\\
85	0\\
86	0\\
87	0\\
88	0\\
89	0\\
90	0\\
91	0\\
92	0\\
93	0\\
94	0\\
95	0\\
96	0\\
97	0\\
98	0\\
99	0\\
100	0\\
101	0\\
102	0\\
103	0\\
104	0\\
105	0\\
106	0\\
107	0\\
108	0\\
109	0\\
110	0\\
111	0\\
112	0\\
113	0\\
114	0\\
115	0\\
116	0\\
117	0\\
118	0\\
119	0\\
120	0\\
121	0\\
122	0\\
123	0\\
124	0\\
125	0\\
126	0\\
127	0\\
128	0\\
129	0\\
130	0\\
131	0\\
132	0\\
133	0\\
134	0\\
135	0\\
136	0\\
137	0\\
138	0\\
139	0\\
140	0\\
141	0\\
142	0\\
143	0\\
144	0\\
145	0\\
146	0\\
147	0\\
148	0\\
149	0\\
150	0\\
151	0\\
152	0\\
153	0\\
154	0\\
155	0\\
156	0\\
157	0\\
158	0\\
159	0\\
160	0\\
161	0\\
162	0\\
163	0\\
164	0\\
165	0\\
166	0\\
167	0\\
168	0\\
169	0\\
170	0\\
171	0\\
172	0\\
173	0\\
174	0\\
175	0\\
176	0\\
177	0\\
178	0\\
179	0\\
180	0\\
181	0\\
182	0\\
183	0\\
184	0\\
185	0\\
186	0\\
187	0\\
188	0\\
189	0\\
190	0\\
191	0\\
192	0\\
193	0\\
194	0\\
195	0\\
196	0\\
197	0\\
198	0\\
199	0\\
200	0\\
201	0\\
202	0\\
203	0\\
204	0\\
205	0\\
206	0\\
207	0\\
208	0\\
209	0\\
210	0\\
211	0\\
212	0\\
213	0\\
214	0\\
215	0\\
216	0\\
217	0\\
218	0\\
219	0\\
220	0\\
221	0\\
222	0\\
223	0\\
224	0\\
225	0\\
226	0\\
227	0\\
228	0\\
229	0\\
230	0\\
231	0\\
232	0\\
233	0\\
234	0\\
235	0\\
236	0\\
237	0\\
238	0\\
239	0\\
240	0\\
241	0\\
242	0\\
243	0\\
244	0\\
245	0\\
246	0\\
247	0\\
248	0\\
249	0\\
250	0\\
251	0\\
252	0\\
253	0\\
254	0\\
255	0\\
256	0\\
257	0\\
258	0\\
259	0\\
260	0\\
261	0\\
262	0\\
263	0\\
264	0\\
265	0\\
266	0\\
267	0\\
268	0\\
269	0\\
270	0\\
271	0\\
272	0\\
273	0\\
274	0\\
275	0\\
276	0\\
277	0\\
278	0\\
279	0\\
280	0\\
281	0\\
282	0\\
283	0\\
284	0\\
285	0\\
286	0\\
287	0\\
288	0\\
289	0\\
290	0\\
291	0\\
292	0\\
293	0\\
294	0\\
295	0\\
296	0\\
297	0\\
298	0\\
299	0\\
300	0\\
301	0\\
302	0\\
303	0\\
304	0\\
305	0\\
306	0\\
307	0\\
308	0\\
309	0\\
310	0\\
311	0\\
312	0\\
313	0\\
314	0\\
315	0\\
316	0\\
317	0\\
318	0\\
319	0\\
320	0\\
321	0\\
322	0\\
323	0\\
324	0\\
325	0\\
326	0\\
327	0\\
328	0\\
329	0\\
330	0\\
331	0\\
332	0\\
333	0\\
334	0\\
335	0\\
336	0\\
337	0\\
338	0\\
339	0\\
340	0\\
341	0\\
342	0\\
343	0\\
344	0\\
345	0\\
346	0\\
347	0\\
348	0\\
349	0\\
350	0\\
351	0\\
352	0\\
353	0\\
354	0\\
355	0\\
356	0\\
357	0\\
358	0\\
359	0\\
360	0\\
361	0\\
362	0\\
363	0\\
364	0\\
365	0\\
366	0\\
367	0\\
368	0\\
369	0\\
370	0\\
371	0\\
372	0\\
373	0\\
374	0\\
375	0\\
376	0\\
377	0\\
378	0\\
379	0\\
380	0\\
381	0\\
382	0\\
383	0\\
384	0\\
385	0\\
386	0\\
387	0\\
388	0\\
389	0\\
390	0\\
391	0\\
392	0\\
393	0\\
394	0\\
395	0\\
396	0\\
397	0\\
398	0\\
399	0\\
400	0\\
401	0\\
402	0\\
403	0\\
404	0\\
405	0\\
406	0\\
407	0\\
408	0\\
409	0\\
410	0\\
411	0\\
412	0\\
413	0\\
414	0\\
415	0\\
416	0\\
417	0\\
418	0\\
419	0\\
420	0\\
421	0\\
422	0\\
423	0\\
424	0\\
425	0\\
426	0\\
427	0\\
428	0\\
429	0\\
430	0\\
431	0\\
432	0\\
433	0\\
434	0\\
435	0\\
436	0\\
437	0\\
438	0\\
439	0\\
440	0\\
441	0\\
442	0\\
443	0\\
444	0\\
445	0\\
446	0\\
447	0\\
448	0\\
449	0\\
450	0\\
451	0\\
452	0\\
453	0\\
454	0\\
455	0\\
456	0\\
457	0\\
458	0\\
459	0\\
460	0\\
461	0\\
462	0\\
463	0\\
464	0\\
465	0\\
466	0\\
467	0\\
468	0\\
469	0\\
470	0\\
471	0\\
472	0\\
473	0\\
474	0\\
475	0\\
476	0\\
477	0\\
478	0\\
479	0\\
480	0\\
481	0\\
482	0\\
483	0\\
484	0\\
485	0\\
486	0\\
487	0\\
488	0\\
489	0\\
490	0\\
491	0\\
492	0\\
493	0\\
494	0\\
495	0\\
496	0\\
497	0\\
498	0\\
499	0\\
500	0\\
501	0\\
502	0\\
503	0\\
504	0\\
505	0\\
506	0\\
507	0\\
508	0\\
509	0\\
510	0\\
511	0\\
512	0\\
513	0\\
514	0\\
515	0\\
516	0\\
517	0\\
518	0\\
519	0\\
520	0\\
521	0\\
522	0\\
523	0\\
524	0\\
525	0\\
526	0\\
527	0\\
528	0\\
529	0\\
530	0\\
531	0\\
532	0\\
533	0\\
534	0\\
535	0\\
536	0\\
537	0\\
538	0\\
539	0\\
540	0\\
541	0\\
542	0\\
543	0\\
544	0\\
545	0\\
546	0\\
547	0\\
548	0\\
549	0\\
550	0\\
551	0\\
552	0\\
553	0\\
554	0\\
555	0\\
556	0\\
557	0\\
558	0\\
559	0\\
560	0\\
561	0\\
562	0\\
563	0\\
564	0\\
565	0\\
566	0\\
567	0\\
568	0\\
569	0\\
570	0\\
571	0\\
572	0\\
573	0\\
574	0\\
575	0\\
576	0\\
577	0\\
578	0\\
579	0\\
580	0\\
581	0\\
582	0\\
583	0\\
584	0\\
585	0\\
586	0\\
587	0\\
588	0\\
589	0\\
590	0\\
591	0\\
592	0\\
593	0\\
594	0\\
595	0\\
596	0\\
597	0\\
598	0\\
599	0\\
600	0\\
};
\addplot [color=blue!75!mycolor7,solid,forget plot]
  table[row sep=crcr]{%
1	0\\
2	0\\
3	0\\
4	0\\
5	0\\
6	0\\
7	0\\
8	0\\
9	0\\
10	0\\
11	0\\
12	0\\
13	0\\
14	0\\
15	0\\
16	0\\
17	0\\
18	0\\
19	0\\
20	0\\
21	0\\
22	0\\
23	0\\
24	0\\
25	0\\
26	0\\
27	0\\
28	0\\
29	0\\
30	0\\
31	0\\
32	0\\
33	0\\
34	0\\
35	0\\
36	0\\
37	0\\
38	0\\
39	0\\
40	0\\
41	0\\
42	0\\
43	0\\
44	0\\
45	0\\
46	0\\
47	0\\
48	0\\
49	0\\
50	0\\
51	0\\
52	0\\
53	0\\
54	0\\
55	0\\
56	0\\
57	0\\
58	0\\
59	0\\
60	0\\
61	0\\
62	0\\
63	0\\
64	0\\
65	0\\
66	0\\
67	0\\
68	0\\
69	0\\
70	0\\
71	0\\
72	0\\
73	0\\
74	0\\
75	0\\
76	0\\
77	0\\
78	0\\
79	0\\
80	0\\
81	0\\
82	0\\
83	0\\
84	0\\
85	0\\
86	0\\
87	0\\
88	0\\
89	0\\
90	0\\
91	0\\
92	0\\
93	0\\
94	0\\
95	0\\
96	0\\
97	0\\
98	0\\
99	0\\
100	0\\
101	0\\
102	0\\
103	0\\
104	0\\
105	0\\
106	0\\
107	0\\
108	0\\
109	0\\
110	0\\
111	0\\
112	0\\
113	0\\
114	0\\
115	0\\
116	0\\
117	0\\
118	0\\
119	0\\
120	0\\
121	0\\
122	0\\
123	0\\
124	0\\
125	0\\
126	0\\
127	0\\
128	0\\
129	0\\
130	0\\
131	0\\
132	0\\
133	0\\
134	0\\
135	0\\
136	0\\
137	0\\
138	0\\
139	0\\
140	0\\
141	0\\
142	0\\
143	0\\
144	0\\
145	0\\
146	0\\
147	0\\
148	0\\
149	0\\
150	0\\
151	0\\
152	0\\
153	0\\
154	0\\
155	0\\
156	0\\
157	0\\
158	0\\
159	0\\
160	0\\
161	0\\
162	0\\
163	0\\
164	0\\
165	0\\
166	0\\
167	0\\
168	0\\
169	0\\
170	0\\
171	0\\
172	0\\
173	0\\
174	0\\
175	0\\
176	0\\
177	0\\
178	0\\
179	0\\
180	0\\
181	0\\
182	0\\
183	0\\
184	0\\
185	0\\
186	0\\
187	0\\
188	0\\
189	0\\
190	0\\
191	0\\
192	0\\
193	0\\
194	0\\
195	0\\
196	0\\
197	0\\
198	0\\
199	0\\
200	0\\
201	0\\
202	0\\
203	0\\
204	0\\
205	0\\
206	0\\
207	0\\
208	0\\
209	0\\
210	0\\
211	0\\
212	0\\
213	0\\
214	0\\
215	0\\
216	0\\
217	0\\
218	0\\
219	0\\
220	0\\
221	0\\
222	0\\
223	0\\
224	0\\
225	0\\
226	0\\
227	0\\
228	0\\
229	0\\
230	0\\
231	0\\
232	0\\
233	0\\
234	0\\
235	0\\
236	0\\
237	0\\
238	0\\
239	0\\
240	0\\
241	0\\
242	0\\
243	0\\
244	0\\
245	0\\
246	0\\
247	0\\
248	0\\
249	0\\
250	0\\
251	0\\
252	0\\
253	0\\
254	0\\
255	0\\
256	0\\
257	0\\
258	0\\
259	0\\
260	0\\
261	0\\
262	0\\
263	0\\
264	0\\
265	0\\
266	0\\
267	0\\
268	0\\
269	0\\
270	0\\
271	0\\
272	0\\
273	0\\
274	0\\
275	0\\
276	0\\
277	0\\
278	0\\
279	0\\
280	0\\
281	0\\
282	0\\
283	0\\
284	0\\
285	0\\
286	0\\
287	0\\
288	0\\
289	0\\
290	0\\
291	0\\
292	0\\
293	0\\
294	0\\
295	0\\
296	0\\
297	0\\
298	0\\
299	0\\
300	0\\
301	0\\
302	0\\
303	0\\
304	0\\
305	0\\
306	0\\
307	0\\
308	0\\
309	0\\
310	0\\
311	0\\
312	0\\
313	0\\
314	0\\
315	0\\
316	0\\
317	0\\
318	0\\
319	0\\
320	0\\
321	0\\
322	0\\
323	0\\
324	0\\
325	0\\
326	0\\
327	0\\
328	0\\
329	0\\
330	0\\
331	0\\
332	0\\
333	0\\
334	0\\
335	0\\
336	0\\
337	0\\
338	0\\
339	0\\
340	0\\
341	0\\
342	0\\
343	0\\
344	0\\
345	0\\
346	0\\
347	0\\
348	0\\
349	0\\
350	0\\
351	0\\
352	0\\
353	0\\
354	0\\
355	0\\
356	0\\
357	0\\
358	0\\
359	0\\
360	0\\
361	0\\
362	0\\
363	0\\
364	0\\
365	0\\
366	0\\
367	0\\
368	0\\
369	0\\
370	0\\
371	0\\
372	0\\
373	0\\
374	0\\
375	0\\
376	0\\
377	0\\
378	0\\
379	0\\
380	0\\
381	0\\
382	0\\
383	0\\
384	0\\
385	0\\
386	0\\
387	0\\
388	0\\
389	0\\
390	0\\
391	0\\
392	0\\
393	0\\
394	0\\
395	0\\
396	0\\
397	0\\
398	0\\
399	0\\
400	0\\
401	0\\
402	0\\
403	0\\
404	0\\
405	0\\
406	0\\
407	0\\
408	0\\
409	0\\
410	0\\
411	0\\
412	0\\
413	0\\
414	0\\
415	0\\
416	0\\
417	0\\
418	0\\
419	0\\
420	0\\
421	0\\
422	0\\
423	0\\
424	0\\
425	0\\
426	0\\
427	0\\
428	0\\
429	0\\
430	0\\
431	0\\
432	0\\
433	0\\
434	0\\
435	0\\
436	0\\
437	0\\
438	0\\
439	0\\
440	0\\
441	0\\
442	0\\
443	0\\
444	0\\
445	0\\
446	0\\
447	0\\
448	0\\
449	0\\
450	0\\
451	0\\
452	0\\
453	0\\
454	0\\
455	0\\
456	0\\
457	0\\
458	0\\
459	0\\
460	0\\
461	0\\
462	0\\
463	0\\
464	0\\
465	0\\
466	0\\
467	0\\
468	0\\
469	0\\
470	0\\
471	0\\
472	0\\
473	0\\
474	0\\
475	0\\
476	0\\
477	0\\
478	0\\
479	0\\
480	0\\
481	0\\
482	0\\
483	0\\
484	0\\
485	0\\
486	0\\
487	0\\
488	0\\
489	0\\
490	0\\
491	0\\
492	0\\
493	0\\
494	0\\
495	0\\
496	0\\
497	0\\
498	0\\
499	0\\
500	0\\
501	0\\
502	0\\
503	0\\
504	0\\
505	0\\
506	0\\
507	0\\
508	0\\
509	0\\
510	0\\
511	0\\
512	0\\
513	0\\
514	0\\
515	0\\
516	0\\
517	0\\
518	0\\
519	0\\
520	0\\
521	0\\
522	0\\
523	0\\
524	0\\
525	0\\
526	0\\
527	0\\
528	0\\
529	0\\
530	0\\
531	0\\
532	0\\
533	0\\
534	0\\
535	0\\
536	0\\
537	0\\
538	0\\
539	0\\
540	0\\
541	0\\
542	0\\
543	0\\
544	0\\
545	0\\
546	0\\
547	0\\
548	0\\
549	0\\
550	0\\
551	0\\
552	0\\
553	0\\
554	0\\
555	0\\
556	0\\
557	0\\
558	0\\
559	0\\
560	0\\
561	0\\
562	0\\
563	0\\
564	0\\
565	0\\
566	0\\
567	0\\
568	0\\
569	0\\
570	0\\
571	0\\
572	0\\
573	0\\
574	0\\
575	0\\
576	0\\
577	0\\
578	0\\
579	0\\
580	0\\
581	0\\
582	0\\
583	0\\
584	0\\
585	0\\
586	0\\
587	0\\
588	0\\
589	0\\
590	0\\
591	0\\
592	0\\
593	0\\
594	0\\
595	0\\
596	0\\
597	0\\
598	0\\
599	0\\
600	0\\
};
\addplot [color=blue!80!mycolor9,solid,forget plot]
  table[row sep=crcr]{%
1	0.000420463240150569\\
2	0.00042044912030799\\
3	0.000420434759003375\\
4	0.000420420152089871\\
5	0.00042040529534938\\
6	0.000420390184491366\\
7	0.00042037481515161\\
8	0.00042035918289092\\
9	0.000420343283193896\\
10	0.000420327111467586\\
11	0.000420310663040172\\
12	0.000420293933159629\\
13	0.000420276916992343\\
14	0.000420259609621713\\
15	0.000420242006046732\\
16	0.00042022410118055\\
17	0.000420205889849007\\
18	0.000420187366789113\\
19	0.000420168526647563\\
20	0.000420149363979176\\
21	0.000420129873245317\\
22	0.000420110048812314\\
23	0.000420089884949808\\
24	0.000420069375829147\\
25	0.00042004851552164\\
26	0.00042002729799689\\
27	0.000420005717121061\\
28	0.000419983766655072\\
29	0.000419961440252826\\
30	0.000419938731459381\\
31	0.000419915633709058\\
32	0.0004198921403236\\
33	0.000419868244510194\\
34	0.000419843939359556\\
35	0.000419819217843911\\
36	0.000419794072814985\\
37	0.000419768497001938\\
38	0.000419742483009261\\
39	0.000419716023314677\\
40	0.000419689110266941\\
41	0.000419661736083651\\
42	0.000419633892849017\\
43	0.000419605572511561\\
44	0.00041957676688182\\
45	0.000419547467629971\\
46	0.000419517666283459\\
47	0.00041948735422453\\
48	0.000419456522687782\\
49	0.000419425162757618\\
50	0.000419393265365691\\
51	0.000419360821288323\\
52	0.000419327821143804\\
53	0.000419294255389739\\
54	0.000419260114320285\\
55	0.000419225388063368\\
56	0.000419190066577843\\
57	0.00041915413965062\\
58	0.000419117596893731\\
59	0.00041908042774133\\
60	0.000419042621446685\\
61	0.000419004167079077\\
62	0.000418965053520673\\
63	0.000418925269463327\\
64	0.000418884803405359\\
65	0.000418843643648231\\
66	0.000418801778293208\\
67	0.000418759195237954\\
68	0.000418715882173041\\
69	0.000418671826578452\\
70	0.000418627015719973\\
71	0.000418581436645575\\
72	0.000418535076181664\\
73	0.000418487920929352\\
74	0.000418439957260604\\
75	0.000418391171314359\\
76	0.000418341548992521\\
77	0.000418291075956001\\
78	0.000418239737620543\\
79	0.000418187519152627\\
80	0.000418134405465184\\
81	0.000418080381213302\\
82	0.000418025430789865\\
83	0.000417969538321074\\
84	0.000417912687661938\\
85	0.000417854862391662\\
86	0.000417796045808958\\
87	0.000417736220927311\\
88	0.000417675370470104\\
89	0.000417613476865731\\
90	0.00041755052224259\\
91	0.000417486488423966\\
92	0.000417421356922902\\
93	0.000417355108936909\\
94	0.000417287725342635\\
95	0.000417219186690436\\
96	0.000417149473198828\\
97	0.000417078564748892\\
98	0.000417006440878541\\
99	0.000416933080776717\\
100	0.000416858463277496\\
101	0.00041678256685407\\
102	0.000416705369612641\\
103	0.000416626849286235\\
104	0.000416546983228358\\
105	0.000416465748406587\\
106	0.000416383121396063\\
107	0.000416299078372831\\
108	0.000416213595107108\\
109	0.000416126646956417\\
110	0.000416038208858606\\
111	0.000415948255324737\\
112	0.000415856760431929\\
113	0.00041576369781595\\
114	0.000415669040663792\\
115	0.000415572761706077\\
116	0.000415474833209363\\
117	0.000415375226968243\\
118	0.000415273914297402\\
119	0.000415170866023496\\
120	0.000415066052476872\\
121	0.000414959443483184\\
122	0.000414851008354854\\
123	0.000414740715882372\\
124	0.000414628534325447\\
125	0.000414514431404022\\
126	0.000414398374289128\\
127	0.00041428032959356\\
128	0.000414160263362431\\
129	0.000414038141063515\\
130	0.000413913927577448\\
131	0.000413787587187775\\
132	0.000413659083570779\\
133	0.000413528379785166\\
134	0.000413395438261569\\
135	0.000413260220791854\\
136	0.000413122688518262\\
137	0.000412982801922331\\
138	0.000412840520813672\\
139	0.000412695804318507\\
140	0.000412548610868033\\
141	0.000412398898186595\\
142	0.000412246623279623\\
143	0.000412091742421269\\
144	0.000411934211141827\\
145	0.000411773984214908\\
146	0.00041161101564502\\
147	0.000411445258655372\\
148	0.000411276665673553\\
149	0.000411105188317988\\
150	0.000410930777384189\\
151	0.000410753382830757\\
152	0.000410572953765137\\
153	0.000410389438429166\\
154	0.00041020278418429\\
155	0.000410012937496649\\
156	0.00040981984392177\\
157	0.000409623448089111\\
158	0.000409423693686277\\
159	0.000409220523442977\\
160	0.000409013879114696\\
161	0.000408803701466113\\
162	0.000408589930254217\\
163	0.00040837250421111\\
164	0.000408151361026537\\
165	0.00040792643733012\\
166	0.00040769766867327\\
167	0.000407464989510785\\
168	0.000407228333182126\\
169	0.000406987631892374\\
170	0.000406742816692882\\
171	0.000406493817461535\\
172	0.0004062405628827\\
173	0.000405982980426843\\
174	0.000405720996329757\\
175	0.000405454535571414\\
176	0.000405183521854547\\
177	0.000404907877582701\\
178	0.000404627523838044\\
179	0.000404342380358698\\
180	0.000404052365515714\\
181	0.000403757396289626\\
182	0.000403457388246604\\
183	0.000403152255514178\\
184	0.000402841910756524\\
185	0.000402526265149358\\
186	0.000402205228354325\\
187	0.000401878708492977\\
188	0.000401546612120275\\
189	0.000401208844197626\\
190	0.00040086530806543\\
191	0.000400515905415176\\
192	0.000400160536260948\\
193	0.000399799098910555\\
194	0.000399431489936029\\
195	0.000399057604143654\\
196	0.000398677334543464\\
197	0.000398290572318125\\
198	0.000397897206791365\\
199	0.000397497125395721\\
200	0.000397090213639784\\
201	0.000396676355074823\\
202	0.000396255431260803\\
203	0.00039582732173178\\
204	0.000395391903960692\\
205	0.00039494905332346\\
206	0.000394498643062507\\
207	0.000394040544249511\\
208	0.000393574625747573\\
209	0.000393100754172621\\
210	0.000392618793854148\\
211	0.000392128606795174\\
212	0.000391630052631514\\
213	0.00039112298859028\\
214	0.000390607269447579\\
215	0.000390082747485471\\
216	0.000389549272448078\\
217	0.000389006691496911\\
218	0.000388454849165321\\
219	0.000387893587312119\\
220	0.000387322745074294\\
221	0.000386742158818885\\
222	0.000386151662093866\\
223	0.000385551085578178\\
224	0.000384940257030721\\
225	0.000384319001238465\\
226	0.000383687139963456\\
227	0.000383044491888938\\
228	0.00038239087256427\\
229	0.000381726094348952\\
230	0.000381049966355412\\
231	0.000380362294390771\\
232	0.00037966288089746\\
233	0.000378951524892641\\
234	0.000378228021906455\\
235	0.000377492163919073\\
236	0.000376743739296484\\
237	0.000375982532725003\\
238	0.000375208325144516\\
239	0.000374420893680352\\
240	0.000373620011573821\\
241	0.000372805448111359\\
242	0.000371976968552242\\
243	0.00037113433405487\\
244	0.000370277301601518\\
245	0.000369405623921625\\
246	0.00036851904941349\\
247	0.000367617322064392\\
248	0.000366700181369074\\
249	0.000365767362246593\\
250	0.000364818594955471\\
251	0.000363853605007071\\
252	0.000362872113077254\\
253	0.000361873834916212\\
254	0.000360858481256441\\
255	0.000359825757718859\\
256	0.000358775364716957\\
257	0.000357706997359041\\
258	0.000356620345348432\\
259	0.000355515092881627\\
260	0.000354390918544418\\
261	0.000353247495205839\\
262	0.000352084489909975\\
263	0.000350901563765589\\
264	0.000349698371833468\\
265	0.000348474563011527\\
266	0.000347229779917538\\
267	0.000345963658769561\\
268	0.000344675829263897\\
269	0.000343365914450595\\
270	0.000342033530606426\\
271	0.000340678287105166\\
272	0.000339299786285226\\
273	0.000337897623314623\\
274	0.000336471386054643\\
275	0.000335020654925319\\
276	0.000333545002775104\\
277	0.000332043994732079\\
278	0.000330517187994447\\
279	0.000328964131722927\\
280	0.000327384366888339\\
281	0.000325777426113161\\
282	0.000324142833509996\\
283	0.00032248010451688\\
284	0.000320788745729423\\
285	0.000319068254729721\\
286	0.000317318119912109\\
287	0.000315537820305653\\
288	0.000313726825393466\\
289	0.00031188459492877\\
290	0.000310010578747858\\
291	0.000308104216579848\\
292	0.00030616493785344\\
293	0.000304192161500579\\
294	0.000302185295757328\\
295	0.000300143737961838\\
296	0.000298066874349785\\
297	0.000295954079847361\\
298	0.000293804717861986\\
299	0.000291618140071164\\
300	0.00028939368620967\\
301	0.00028713068385552\\
302	0.000284828448215171\\
303	0.00028248628190848\\
304	0.000280103474754065\\
305	0.00027767930355581\\
306	0.000275213031891497\\
307	0.000272703909904661\\
308	0.000270151174100987\\
309	0.000267554047150566\\
310	0.000264911737695903\\
311	0.000262223440161519\\
312	0.000259488334553529\\
313	0.000256705586246752\\
314	0.00025387434585603\\
315	0.00025099374935309\\
316	0.000248062917857218\\
317	0.000245080957487294\\
318	0.00024204695931954\\
319	0.000238959999380692\\
320	0.00023581913868224\\
321	0.000232623423301923\\
322	0.00022937188451976\\
323	0.000226063539016568\\
324	0.000222697389144192\\
325	0.000219272423277816\\
326	0.000215787616262068\\
327	0.000212241929964235\\
328	0.000208634313949584\\
329	0.000204963706295905\\
330	0.000201229034566454\\
331	0.000197429216963232\\
332	0.000193563163685222\\
333	0.000189629778519623\\
334	0.000185627960697774\\
335	0.000181556607051661\\
336	0.000177414614511688\\
337	0.000173200882991794\\
338	0.000168914318714217\\
339	0.000164553838033065\\
340	0.000160118371823967\\
341	0.000155606870515935\\
342	0.000151018309852072\\
343	0.000146351697477284\\
344	0.000141606080464596\\
345	0.000136780553906929\\
346	0.000131874270718485\\
347	0.000126886452809903\\
348	0.000121816403823762\\
349	0.000116663523643309\\
350	0.000111427324916498\\
351	0.000106107451871358\\
352	0.000100703701736782\\
353	9.52160491260074e-05\\
354	8.96446737914473e-05\\
355	8.39899922321503e-05\\
356	7.82526937707919e-05\\
357	7.24337819585433e-05\\
358	6.65346219905049e-05\\
359	6.05569928351092e-05\\
360	5.45031451081293e-05\\
361	4.83758688722205e-05\\
362	4.21785594234741e-05\\
363	3.59152575781306e-05\\
364	2.95905654047414e-05\\
365	2.3209056409204e-05\\
366	1.67726731719349e-05\\
367	1.026963175326e-05\\
368	3.62063219975855e-06\\
369	0\\
370	0\\
371	0\\
372	0\\
373	0\\
374	0\\
375	0\\
376	0\\
377	0\\
378	0\\
379	0\\
380	0\\
381	0\\
382	0\\
383	0\\
384	0\\
385	0\\
386	0\\
387	0\\
388	0\\
389	0\\
390	0\\
391	0\\
392	0\\
393	0\\
394	0\\
395	0\\
396	0\\
397	0\\
398	0\\
399	0\\
400	0\\
401	0\\
402	0\\
403	0\\
404	0\\
405	0\\
406	0\\
407	0\\
408	0\\
409	0\\
410	0\\
411	0\\
412	0\\
413	0\\
414	0\\
415	0\\
416	0\\
417	0\\
418	0\\
419	0\\
420	0\\
421	0\\
422	0\\
423	0\\
424	0\\
425	0\\
426	0\\
427	0\\
428	0\\
429	0\\
430	0\\
431	0\\
432	0\\
433	0\\
434	0\\
435	0\\
436	0\\
437	0\\
438	0\\
439	0\\
440	0\\
441	0\\
442	0\\
443	0\\
444	0\\
445	0\\
446	0\\
447	0\\
448	0\\
449	0\\
450	0\\
451	0\\
452	0\\
453	0\\
454	0\\
455	0\\
456	0\\
457	0\\
458	0\\
459	0\\
460	0\\
461	0\\
462	0\\
463	0\\
464	0\\
465	0\\
466	0\\
467	0\\
468	0\\
469	0\\
470	0\\
471	0\\
472	0\\
473	0\\
474	0\\
475	0\\
476	0\\
477	0\\
478	0\\
479	0\\
480	0\\
481	0\\
482	0\\
483	0\\
484	0\\
485	0\\
486	0\\
487	0\\
488	0\\
489	0\\
490	0\\
491	0\\
492	0\\
493	0\\
494	0\\
495	0\\
496	0\\
497	0\\
498	0\\
499	0\\
500	0\\
501	0\\
502	0\\
503	0\\
504	0\\
505	0\\
506	0\\
507	0\\
508	0\\
509	0\\
510	0\\
511	0\\
512	0\\
513	0\\
514	0\\
515	0\\
516	0\\
517	0\\
518	0\\
519	0\\
520	0\\
521	0\\
522	0\\
523	0\\
524	0\\
525	0\\
526	0\\
527	0\\
528	0\\
529	0\\
530	0\\
531	0\\
532	0\\
533	0\\
534	0\\
535	0\\
536	0\\
537	0\\
538	0\\
539	0\\
540	0\\
541	0\\
542	0\\
543	0\\
544	0\\
545	0\\
546	0\\
547	0\\
548	0\\
549	0\\
550	0\\
551	0\\
552	0\\
553	0\\
554	0\\
555	0\\
556	0\\
557	0\\
558	0\\
559	0\\
560	0\\
561	0\\
562	0\\
563	0\\
564	0\\
565	0\\
566	0\\
567	0\\
568	0\\
569	0\\
570	0\\
571	0\\
572	0\\
573	0\\
574	0\\
575	0\\
576	0\\
577	0\\
578	0\\
579	0\\
580	0\\
581	0\\
582	0\\
583	0\\
584	0\\
585	0\\
586	0\\
587	0\\
588	0\\
589	0\\
590	0\\
591	0\\
592	0\\
593	0\\
594	0\\
595	0\\
596	0\\
597	0\\
598	0\\
599	0\\
600	0\\
};
\addplot [color=blue,solid,forget plot]
  table[row sep=crcr]{%
1	0.00225547679368458\\
2	0.00225546436267677\\
3	0.0022554517189349\\
4	0.00225543885880797\\
5	0.00225542577858241\\
6	0.00225541247448096\\
7	0.00225539894266163\\
8	0.00225538517921661\\
9	0.00225537118017108\\
10	0.00225535694148212\\
11	0.00225534245903754\\
12	0.00225532772865468\\
13	0.00225531274607924\\
14	0.00225529750698401\\
15	0.00225528200696767\\
16	0.00225526624155352\\
17	0.00225525020618817\\
18	0.00225523389624026\\
19	0.00225521730699912\\
20	0.00225520043367341\\
21	0.00225518327138976\\
22	0.00225516581519137\\
23	0.00225514806003657\\
24	0.0022551300007974\\
25	0.00225511163225812\\
26	0.00225509294911375\\
27	0.00225507394596848\\
28	0.00225505461733422\\
29	0.00225503495762892\\
30	0.00225501496117508\\
31	0.00225499462219804\\
32	0.00225497393482437\\
33	0.00225495289308018\\
34	0.00225493149088941\\
35	0.00225490972207209\\
36	0.00225488758034257\\
37	0.00225486505930775\\
38	0.00225484215246521\\
39	0.00225481885320136\\
40	0.00225479515478961\\
41	0.00225477105038835\\
42	0.00225474653303909\\
43	0.00225472159566441\\
44	0.00225469623106597\\
45	0.00225467043192244\\
46	0.00225464419078744\\
47	0.00225461750008738\\
48	0.00225459035211933\\
49	0.00225456273904881\\
50	0.00225453465290756\\
51	0.00225450608559128\\
52	0.0022544770288573\\
53	0.00225444747432227\\
54	0.00225441741345974\\
55	0.00225438683759773\\
56	0.00225435573791633\\
57	0.00225432410544509\\
58	0.00225429193106057\\
59	0.00225425920548367\\
60	0.00225422591927705\\
61	0.0022541920628424\\
62	0.00225415762641777\\
63	0.00225412260007477\\
64	0.00225408697371574\\
65	0.00225405073707093\\
66	0.00225401387969554\\
67	0.0022539763909668\\
68	0.00225393826008092\\
69	0.0022538994760501\\
70	0.00225386002769933\\
71	0.0022538199036633\\
72	0.00225377909238315\\
73	0.00225373758210322\\
74	0.0022536953608677\\
75	0.00225365241651727\\
76	0.00225360873668569\\
77	0.00225356430879626\\
78	0.0022535191200583\\
79	0.00225347315746352\\
80	0.0022534264077824\\
81	0.00225337885756039\\
82	0.00225333049311418\\
83	0.00225328130052779\\
84	0.0022532312656487\\
85	0.00225318037408382\\
86	0.00225312861119549\\
87	0.00225307596209731\\
88	0.00225302241164999\\
89	0.00225296794445707\\
90	0.0022529125448606\\
91	0.00225285619693674\\
92	0.0022527988844913\\
93	0.00225274059105517\\
94	0.0022526812998797\\
95	0.00225262099393201\\
96	0.00225255965589022\\
97	0.00225249726813854\\
98	0.00225243381276239\\
99	0.00225236927154336\\
100	0.00225230362595409\\
101	0.00225223685715307\\
102	0.00225216894597939\\
103	0.00225209987294737\\
104	0.00225202961824105\\
105	0.00225195816170873\\
106	0.00225188548285725\\
107	0.00225181156084631\\
108	0.00225173637448258\\
109	0.00225165990221383\\
110	0.00225158212212285\\
111	0.00225150301192136\\
112	0.00225142254894375\\
113	0.00225134071014072\\
114	0.00225125747207289\\
115	0.00225117281090421\\
116	0.00225108670239529\\
117	0.00225099912189661\\
118	0.0022509100443417\\
119	0.00225081944424\\
120	0.00225072729566989\\
121	0.00225063357227131\\
122	0.00225053824723845\\
123	0.00225044129331226\\
124	0.0022503426827728\\
125	0.0022502423874315\\
126	0.0022501403786233\\
127	0.00225003662719858\\
128	0.00224993110351509\\
129	0.00224982377742956\\
130	0.00224971461828938\\
131	0.00224960359492395\\
132	0.00224949067563599\\
133	0.00224937582819269\\
134	0.00224925901981669\\
135	0.00224914021717691\\
136	0.00224901938637924\\
137	0.00224889649295707\\
138	0.00224877150186167\\
139	0.00224864437745237\\
140	0.00224851508348662\\
141	0.00224838358310987\\
142	0.00224824983884521\\
143	0.00224811381258292\\
144	0.00224797546556979\\
145	0.00224783475839845\\
146	0.00224769165099652\\
147	0.00224754610261519\\
148	0.0022473980718179\\
149	0.00224724751646874\\
150	0.00224709439372072\\
151	0.00224693866000382\\
152	0.00224678027101282\\
153	0.00224661918169497\\
154	0.0022464553462374\\
155	0.00224628871805434\\
156	0.00224611924977416\\
157	0.00224594689322612\\
158	0.00224577159942695\\
159	0.00224559331856717\\
160	0.00224541199999721\\
161	0.00224522759221328\\
162	0.00224504004284297\\
163	0.00224484929863065\\
164	0.00224465530542263\\
165	0.002244458008152\\
166	0.00224425735082327\\
167	0.00224405327649673\\
168	0.00224384572727259\\
169	0.00224363464427471\\
170	0.00224341996763425\\
171	0.00224320163647287\\
172	0.00224297958888574\\
173	0.0022427537619242\\
174	0.00224252409157822\\
175	0.0022422905127584\\
176	0.00224205295927782\\
177	0.00224181136383346\\
178	0.0022415656579874\\
179	0.00224131577214762\\
180	0.00224106163554846\\
181	0.00224080317623084\\
182	0.00224054032102202\\
183	0.00224027299551509\\
184	0.00224000112404807\\
185	0.00223972462968265\\
186	0.00223944343418256\\
187	0.00223915745799157\\
188	0.00223886662021111\\
189	0.00223857083857746\\
190	0.0022382700294386\\
191	0.00223796410773064\\
192	0.00223765298695376\\
193	0.00223733657914782\\
194	0.00223701479486753\\
195	0.00223668754315712\\
196	0.00223635473152461\\
197	0.00223601626591562\\
198	0.0022356720506867\\
199	0.00223532198857818\\
200	0.00223496598068657\\
201	0.00223460392643642\\
202	0.00223423572355172\\
203	0.00223386126802676\\
204	0.00223348045409645\\
205	0.00223309317420614\\
206	0.00223269931898089\\
207	0.00223229877719413\\
208	0.00223189143573586\\
209	0.00223147717958013\\
210	0.00223105589175204\\
211	0.00223062745329407\\
212	0.00223019174323182\\
213	0.00222974863853913\\
214	0.00222929801410253\\
215	0.002228839742685\\
216	0.00222837369488917\\
217	0.00222789973911968\\
218	0.00222741774154498\\
219	0.00222692756605833\\
220	0.00222642907423807\\
221	0.00222592212530717\\
222	0.00222540657609198\\
223	0.00222488228098028\\
224	0.00222434909187839\\
225	0.00222380685816758\\
226	0.0022232554266596\\
227	0.00222269464155133\\
228	0.0022221243443786\\
229	0.00222154437396911\\
230	0.00222095456639435\\
231	0.00222035475492071\\
232	0.00221974476995955\\
233	0.00221912443901628\\
234	0.00221849358663846\\
235	0.00221785203436292\\
236	0.00221719960066168\\
237	0.00221653610088695\\
238	0.00221586134721493\\
239	0.00221517514858848\\
240	0.00221447731065868\\
241	0.00221376763572513\\
242	0.0022130459226751\\
243	0.00221231196692134\\
244	0.00221156556033872\\
245	0.00221080649119944\\
246	0.00221003454410697\\
247	0.00220924949992857\\
248	0.00220845113572638\\
249	0.00220763922468709\\
250	0.00220681353605007\\
251	0.00220597383503397\\
252	0.00220511988276175\\
253	0.00220425143618409\\
254	0.00220336824800114\\
255	0.00220247006658248\\
256	0.00220155663588542\\
257	0.00220062769537139\\
258	0.00219968297992052\\
259	0.00219872221974431\\
260	0.0021977451402962\\
261	0.00219675146218023\\
262	0.00219574090105754\\
263	0.00219471316755064\\
264	0.00219366796714554\\
265	0.00219260500009143\\
266	0.00219152396129811\\
267	0.00219042454023081\\
268	0.00218930642080246\\
269	0.00218816928126336\\
270	0.00218701279408796\\
271	0.00218583662585887\\
272	0.00218464043714798\\
273	0.00218342388239507\\
274	0.00218218660978429\\
275	0.00218092826111787\\
276	0.00217964847168101\\
277	0.00217834687009569\\
278	0.00217702307819071\\
279	0.00217567671085904\\
280	0.00217430737591045\\
281	0.00217291467392013\\
282	0.00217149819807274\\
283	0.00217005753400202\\
284	0.00216859225962569\\
285	0.00216710194497527\\
286	0.00216558615202082\\
287	0.00216404443449025\\
288	0.00216247633768293\\
289	0.00216088139827743\\
290	0.00215925914413306\\
291	0.00215760909408485\\
292	0.00215593075773177\\
293	0.00215422363521778\\
294	0.00215248721700542\\
295	0.00215072098364137\\
296	0.00214892440551389\\
297	0.00214709694260139\\
298	0.00214523804421187\\
299	0.00214334714871261\\
300	0.00214142368324968\\
301	0.00213946706345662\\
302	0.00213747669315173\\
303	0.00213545196402326\\
304	0.00213339225530193\\
305	0.00213129693341986\\
306	0.00212916535165536\\
307	0.00212699684976246\\
308	0.00212479075358427\\
309	0.00212254637464819\\
310	0.00212026300974022\\
311	0.0021179399404549\\
312	0.00211557643272357\\
313	0.0021131717363424\\
314	0.00211072508450726\\
315	0.00210823569325357\\
316	0.00210570276087912\\
317	0.00210312546735227\\
318	0.0021005029736881\\
319	0.00209783442128978\\
320	0.00209511893125242\\
321	0.00209235560362633\\
322	0.00208954351663622\\
323	0.00208668172585272\\
324	0.00208376926331188\\
325	0.00208080513657825\\
326	0.00207778832774644\\
327	0.00207471779237544\\
328	0.00207159245834967\\
329	0.0020684112246598\\
330	0.00206517296009569\\
331	0.00206187650184302\\
332	0.00205852065397409\\
333	0.00205510418582252\\
334	0.00205162583022995\\
335	0.00204808428165185\\
336	0.00204447819410802\\
337	0.00204080617896145\\
338	0.00203706680250769\\
339	0.00203325858335442\\
340	0.00202937998956875\\
341	0.00202542943556714\\
342	0.00202140527871949\\
343	0.00201730581563618\\
344	0.00201312927810233\\
345	0.00200887382861977\\
346	0.00200453755551226\\
347	0.00200011846754379\\
348	0.00199561448799404\\
349	0.00199102344812764\\
350	0.00198634307998626\\
351	0.00198157100842312\\
352	0.00197670474228982\\
353	0.00197174166467503\\
354	0.0019666790220888\\
355	0.00196151391248782\\
356	0.0019562432720273\\
357	0.00195086386027465\\
358	0.00194537224329756\\
359	0.00193976477445752\\
360	0.00193403757236674\\
361	0.00192818649089298\\
362	0.00192220706749014\\
363	0.00191609439936248\\
364	0.00190984276577908\\
365	0.001903444344481\\
366	0.00189688471607222\\
367	0.00189012753770206\\
368	0.00188307204594991\\
369	0.00187532607576948\\
370	0.00186741687321954\\
371	0.00185937371942756\\
372	0.00185119429012358\\
373	0.00184287622349694\\
374	0.00183441711805204\\
375	0.0018258145280678\\
376	0.00181706595713902\\
377	0.00180816886457825\\
378	0.0017991207509713\\
379	0.00178991932478652\\
380	0.00178056209059273\\
381	0.00177104636415697\\
382	0.0017613694237474\\
383	0.00175152851070232\\
384	0.00174152083010462\\
385	0.00173134355156662\\
386	0.00172099381012973\\
387	0.00171046870728252\\
388	0.00169976531209971\\
389	0.00168888066250364\\
390	0.00167781176664776\\
391	0.00166655560442005\\
392	0.00165510912906219\\
393	0.00164346926889762\\
394	0.00163163292916047\\
395	0.00161959699391801\\
396	0.00160735832808923\\
397	0.00159491377959592\\
398	0.00158226018178169\\
399	0.00156939435649901\\
400	0.00155631311892567\\
401	0.00154301328669679\\
402	0.00152949169902099\\
403	0.00151574525607928\\
404	0.00150177098980964\\
405	0.0014875661528874\\
406	0.00147312828649604\\
407	0.00145845627135828\\
408	0.00144355769227254\\
409	0.00142846262402851\\
410	0.00141326501158894\\
411	0.00139822731853432\\
412	0.00138388740253358\\
413	0.00137011704235746\\
414	0.00135607976325594\\
415	0.00134177068139713\\
416	0.0013271838316562\\
417	0.00131231249661213\\
418	0.0012971497300087\\
419	0.00128168834618063\\
420	0.00126592090865607\\
421	0.00124983971672581\\
422	0.00123343678767319\\
423	0.00121670383904977\\
424	0.00119963228876237\\
425	0.00118221324775915\\
426	0.00116443749003025\\
427	0.00114629543543994\\
428	0.0011277771315331\\
429	0.00110887223425105\\
430	0.00108956998748904\\
431	0.00106985920142357\\
432	0.00104972822953443\\
433	0.00102916494424355\\
434	0.00100815671108913\\
435	0.000986690361352274\\
436	0.000964752163052489\\
437	0.000942327790230635\\
438	0.000919402290441712\\
439	0.000895960050373461\\
440	0.000871984759420209\\
441	0.000847459370560392\\
442	0.000822366055769701\\
443	0.000796686145148563\\
444	0.000770400012820854\\
445	0.0007434868148554\\
446	0.000715924066459259\\
447	0.000687689175228605\\
448	0.000658765806217179\\
449	0.000629129746906497\\
450	0.000598755261626512\\
451	0.00056761530781351\\
452	0.00053568151467907\\
453	0.000502924087514885\\
454	0.000469311590360925\\
455	0.000434810187659242\\
456	0.000399380515904592\\
457	0.000362964325713837\\
458	0.000325427915679923\\
459	0.000286121859631025\\
460	0.00024577460918375\\
461	0.000204529627511481\\
462	0.000162346680408577\\
463	0.000119118848819356\\
464	7.45041232746955e-05\\
465	2.6781960106478e-05\\
466	0\\
467	0\\
468	0\\
469	0\\
470	0\\
471	0\\
472	0\\
473	0\\
474	0\\
475	0\\
476	0\\
477	0\\
478	0\\
479	0\\
480	0\\
481	0\\
482	0\\
483	0\\
484	0\\
485	0\\
486	0\\
487	0\\
488	0\\
489	0\\
490	0\\
491	0\\
492	0\\
493	0\\
494	0\\
495	0\\
496	0\\
497	0\\
498	0\\
499	0\\
500	0\\
501	0\\
502	0\\
503	0\\
504	0\\
505	0\\
506	0\\
507	0\\
508	0\\
509	0\\
510	0\\
511	0\\
512	0\\
513	0\\
514	0\\
515	0\\
516	0\\
517	0\\
518	0\\
519	0\\
520	0\\
521	0\\
522	0\\
523	0\\
524	0\\
525	0\\
526	0\\
527	0\\
528	0\\
529	0\\
530	0\\
531	0\\
532	0\\
533	0\\
534	0\\
535	0\\
536	0\\
537	0\\
538	0\\
539	0\\
540	0\\
541	0\\
542	0\\
543	0\\
544	0\\
545	0\\
546	0\\
547	0\\
548	0\\
549	0\\
550	0\\
551	0\\
552	0\\
553	0\\
554	0\\
555	0\\
556	0\\
557	0\\
558	0\\
559	0\\
560	0\\
561	0\\
562	0\\
563	0\\
564	0\\
565	0\\
566	0\\
567	0\\
568	0\\
569	0\\
570	0\\
571	0\\
572	0\\
573	0\\
574	0\\
575	0\\
576	0\\
577	0\\
578	0\\
579	0\\
580	0\\
581	0\\
582	0\\
583	0\\
584	0\\
585	0\\
586	0\\
587	0\\
588	0\\
589	0\\
590	0\\
591	0\\
592	0\\
593	0\\
594	0\\
595	0\\
596	0\\
597	0\\
598	0\\
599	0\\
600	0\\
};
\addplot [color=mycolor10,solid,forget plot]
  table[row sep=crcr]{%
1	0.00367136005952458\\
2	0.0036713542087014\\
3	0.00367134825773797\\
4	0.00367134220491635\\
5	0.00367133604848915\\
6	0.00367132978667907\\
7	0.00367132341767836\\
8	0.00367131693964833\\
9	0.00367131035071882\\
10	0.00367130364898763\\
11	0.00367129683252001\\
12	0.0036712898993481\\
13	0.00367128284747037\\
14	0.003671275674851\\
15	0.00367126837941938\\
16	0.00367126095906944\\
17	0.00367125341165909\\
18	0.00367124573500959\\
19	0.00367123792690494\\
20	0.00367122998509123\\
21	0.00367122190727602\\
22	0.00367121369112766\\
23	0.00367120533427462\\
24	0.00367119683430483\\
25	0.003671188188765\\
26	0.0036711793951599\\
27	0.00367117045095163\\
28	0.00367116135355894\\
29	0.00367115210035647\\
30	0.00367114268867399\\
31	0.00367113311579566\\
32	0.00367112337895925\\
33	0.00367111347535534\\
34	0.00367110340212652\\
35	0.00367109315636661\\
36	0.00367108273511978\\
37	0.00367107213537974\\
38	0.00367106135408891\\
39	0.00367105038813747\\
40	0.00367103923436256\\
41	0.00367102788954732\\
42	0.00367101635042003\\
43	0.00367100461365313\\
44	0.0036709926758623\\
45	0.00367098053360549\\
46	0.00367096818338196\\
47	0.00367095562163127\\
48	0.00367094284473225\\
49	0.00367092984900204\\
50	0.00367091663069497\\
51	0.00367090318600154\\
52	0.00367088951104733\\
53	0.00367087560189192\\
54	0.00367086145452774\\
55	0.00367084706487896\\
56	0.00367083242880035\\
57	0.00367081754207607\\
58	0.0036708024004185\\
59	0.00367078699946705\\
60	0.00367077133478688\\
61	0.00367075540186769\\
62	0.00367073919612242\\
63	0.00367072271288598\\
64	0.00367070594741391\\
65	0.00367068889488107\\
66	0.00367067155038026\\
67	0.00367065390892083\\
68	0.0036706359654273\\
69	0.00367061771473792\\
70	0.00367059915160324\\
71	0.00367058027068457\\
72	0.00367056106655255\\
73	0.0036705415336856\\
74	0.00367052166646835\\
75	0.0036705014591901\\
76	0.00367048090604318\\
77	0.00367046000112136\\
78	0.00367043873841816\\
79	0.00367041711182519\\
80	0.00367039511513042\\
81	0.00367037274201647\\
82	0.0036703499860588\\
83	0.00367032684072394\\
84	0.00367030329936769\\
85	0.00367027935523319\\
86	0.0036702550014491\\
87	0.00367023023102764\\
88	0.00367020503686266\\
89	0.00367017941172767\\
90	0.00367015334827377\\
91	0.00367012683902765\\
92	0.0036700998763895\\
93	0.00367007245263087\\
94	0.00367004455989254\\
95	0.00367001619018233\\
96	0.00366998733537283\\
97	0.00366995798719921\\
98	0.00366992813725689\\
99	0.00366989777699916\\
100	0.00366986689773488\\
101	0.003669835490626\\
102	0.00366980354668515\\
103	0.00366977105677311\\
104	0.00366973801159628\\
105	0.0036697044017041\\
106	0.00366967021748645\\
107	0.00366963544917095\\
108	0.00366960008682027\\
109	0.00366956412032934\\
110	0.00366952753942259\\
111	0.00366949033365109\\
112	0.00366945249238962\\
113	0.00366941400483377\\
114	0.00366937485999693\\
115	0.00366933504670721\\
116	0.00366929455360442\\
117	0.00366925336913683\\
118	0.00366921148155806\\
119	0.00366916887892377\\
120	0.00366912554908839\\
121	0.0036690814797017\\
122	0.00366903665820549\\
123	0.00366899107183002\\
124	0.00366894470759055\\
125	0.00366889755228367\\
126	0.00366884959248373\\
127	0.00366880081453906\\
128	0.00366875120456826\\
129	0.0036687007484563\\
130	0.00366864943185067\\
131	0.0036685972401574\\
132	0.00366854415853704\\
133	0.00366849017190053\\
134	0.00366843526490509\\
135	0.00366837942194993\\
136	0.00366832262717202\\
137	0.00366826486444162\\
138	0.00366820611735795\\
139	0.00366814636924458\\
140	0.00366808560314488\\
141	0.00366802380181734\\
142	0.00366796094773078\\
143	0.00366789702305959\\
144	0.0036678320096788\\
145	0.00366776588915917\\
146	0.00366769864276207\\
147	0.00366763025143433\\
148	0.00366756069580298\\
149	0.00366748995616998\\
150	0.00366741801250679\\
151	0.00366734484444885\\
152	0.00366727043129002\\
153	0.0036671947519769\\
154	0.00366711778510305\\
155	0.0036670395089031\\
156	0.00366695990124682\\
157	0.00366687893963302\\
158	0.0036667966011834\\
159	0.00366671286263627\\
160	0.0036666277003402\\
161	0.00366654109024748\\
162	0.00366645300790763\\
163	0.0036663634284606\\
164	0.00366627232663002\\
165	0.00366617967671628\\
166	0.00366608545258947\\
167	0.00366598962768222\\
168	0.00366589217498246\\
169	0.00366579306702598\\
170	0.00366569227588894\\
171	0.00366558977318025\\
172	0.0036654855300337\\
173	0.0036653795171002\\
174	0.00366527170453961\\
175	0.00366516206201264\\
176	0.00366505055867255\\
177	0.00366493716315665\\
178	0.00366482184357777\\
179	0.00366470456751549\\
180	0.00366458530200726\\
181	0.00366446401353939\\
182	0.00366434066803786\\
183	0.00366421523085895\\
184	0.0036640876667798\\
185	0.00366395793998869\\
186	0.00366382601407527\\
187	0.00366369185202057\\
188	0.00366355541618677\\
189	0.00366341666830697\\
190	0.00366327556947463\\
191	0.00366313208013285\\
192	0.00366298616006357\\
193	0.00366283776837647\\
194	0.00366268686349774\\
195	0.00366253340315862\\
196	0.00366237734438379\\
197	0.0036622186434795\\
198	0.00366205725602156\\
199	0.00366189313684307\\
200	0.00366172624002197\\
201	0.00366155651886834\\
202	0.00366138392591151\\
203	0.00366120841288693\\
204	0.00366102993072281\\
205	0.00366084842952655\\
206	0.00366066385857087\\
207	0.00366047616627979\\
208	0.00366028530021424\\
209	0.00366009120705756\\
210	0.00365989383260061\\
211	0.00365969312172667\\
212	0.00365948901839613\\
213	0.00365928146563077\\
214	0.0036590704054979\\
215	0.0036588557790941\\
216	0.00365863752652877\\
217	0.00365841558690728\\
218	0.00365818989831388\\
219	0.0036579603977943\\
220	0.00365772702133799\\
221	0.00365748970386007\\
222	0.00365724837918293\\
223	0.00365700298001751\\
224	0.00365675343794421\\
225	0.00365649968339345\\
226	0.0036562416456259\\
227	0.00365597925271227\\
228	0.00365571243151281\\
229	0.00365544110765633\\
230	0.00365516520551892\\
231	0.00365488464820216\\
232	0.00365459935751102\\
233	0.00365430925393122\\
234	0.0036540142566063\\
235	0.00365371428331407\\
236	0.00365340925044277\\
237	0.00365309907296666\\
238	0.00365278366442114\\
239	0.00365246293687742\\
240	0.00365213680091666\\
241	0.00365180516560358\\
242	0.00365146793845959\\
243	0.00365112502543529\\
244	0.00365077633088253\\
245	0.00365042175752576\\
246	0.00365006120643292\\
247	0.00364969457698562\\
248	0.00364932176684879\\
249	0.0036489426719396\\
250	0.00364855718639577\\
251	0.00364816520254322\\
252	0.00364776661086298\\
253	0.0036473612999574\\
254	0.0036469491565156\\
255	0.00364653006527817\\
256	0.0036461039090011\\
257	0.00364567056841883\\
258	0.0036452299222065\\
259	0.00364478184694136\\
260	0.00364432621706316\\
261	0.00364386290483372\\
262	0.00364339178029548\\
263	0.00364291271122899\\
264	0.00364242556310947\\
265	0.00364193019906216\\
266	0.00364142647981658\\
267	0.00364091426365965\\
268	0.00364039340638746\\
269	0.00363986376125587\\
270	0.00363932517892966\\
271	0.00363877750743043\\
272	0.00363822059208316\\
273	0.00363765427546155\\
274	0.00363707839733184\\
275	0.0036364927945936\\
276	0.00363589730121682\\
277	0.00363529174818207\\
278	0.00363467596341668\\
279	0.00363404977172866\\
280	0.00363341299473854\\
281	0.00363276545080885\\
282	0.00363210695497132\\
283	0.00363143731885151\\
284	0.0036307563505908\\
285	0.00363006385476552\\
286	0.00362935963230314\\
287	0.00362864348039527\\
288	0.00362791519240729\\
289	0.0036271745577844\\
290	0.00362642136195385\\
291	0.00362565538622313\\
292	0.00362487640767375\\
293	0.00362408419905044\\
294	0.00362327852864536\\
295	0.00362245916017698\\
296	0.0036216258526632\\
297	0.00362077836028839\\
298	0.00361991643226378\\
299	0.00361903981268069\\
300	0.00361814824035613\\
301	0.00361724144867005\\
302	0.00361631916539358\\
303	0.00361538111250761\\
304	0.00361442700601071\\
305	0.00361345655571573\\
306	0.00361246946503382\\
307	0.00361146543074481\\
308	0.00361044414275246\\
309	0.00360940528382247\\
310	0.0036083485293015\\
311	0.00360727354681671\\
312	0.00360617999595942\\
313	0.00360506752795359\\
314	0.00360393578528118\\
315	0.00360278440128144\\
316	0.00360161299972272\\
317	0.00360042119433957\\
318	0.00359920858833189\\
319	0.00359797477382177\\
320	0.0035967193312638\\
321	0.00359544182880375\\
322	0.00359414182157997\\
323	0.00359281885096147\\
324	0.00359147244371551\\
325	0.00359010211109698\\
326	0.00358870734785085\\
327	0.00358728763111789\\
328	0.00358584241923271\\
329	0.00358437115040184\\
330	0.00358287324124808\\
331	0.00358134808520573\\
332	0.00357979505074929\\
333	0.00357821347943625\\
334	0.00357660268374211\\
335	0.00357496194466306\\
336	0.00357329050905873\\
337	0.00357158758670406\\
338	0.00356985234701524\\
339	0.00356808391541051\\
340	0.00356628136926162\\
341	0.00356444373338597\\
342	0.0035625699750233\\
343	0.00356065899823356\\
344	0.00355870963764444\\
345	0.0035567206514678\\
346	0.00355469071369377\\
347	0.00355261840535946\\
348	0.00355050220477556\\
349	0.00354834047657885\\
350	0.0035461314594607\\
351	0.00354387325240232\\
352	0.00354156379922452\\
353	0.00353920087123593\\
354	0.00353678204773568\\
355	0.00353430469408663\\
356	0.00353176593699083\\
357	0.00352916263644676\\
358	0.00352649135382048\\
359	0.00352374831505654\\
360	0.00352092936611005\\
361	0.00351802991340023\\
362	0.00351504482719273\\
363	0.00351196823786181\\
364	0.00350879299930346\\
365	0.00350550909462353\\
366	0.00350209870266089\\
367	0.00349852084308253\\
368	0.00349465963001152\\
369	0.00349010686104009\\
370	0.00348544033389064\\
371	0.00348070343857037\\
372	0.00347589524788633\\
373	0.0034710148278648\\
374	0.00346606123691451\\
375	0.00346103352543611\\
376	0.00345593074002277\\
377	0.00345075194222006\\
378	0.00344549623478454\\
379	0.0034401626745596\\
380	0.00343475028031016\\
381	0.0034292580655807\\
382	0.00342368503853128\\
383	0.00341803020166984\\
384	0.00341229255146185\\
385	0.00340647107779616\\
386	0.00340056476328271\\
387	0.00339457258235557\\
388	0.00338849350015083\\
389	0.00338232647112589\\
390	0.00337607043738275\\
391	0.00336972432665376\\
392	0.0033632870499045\\
393	0.00335675749850444\\
394	0.00335013454091347\\
395	0.00334341701883367\\
396	0.00333660374278532\\
397	0.00332969348709849\\
398	0.00332268498439105\\
399	0.00331557691978132\\
400	0.00330836792542846\\
401	0.00330105657652901\\
402	0.00329364139035451\\
403	0.00328612082939462\\
404	0.00327849330934434\\
405	0.00327075724168134\\
406	0.00326291136982867\\
407	0.00325495636404665\\
408	0.00324689763874729\\
409	0.00323875231388637\\
410	0.00323056097037767\\
411	0.00322238312421398\\
412	0.00321417988858275\\
413	0.0032058389817077\\
414	0.00319735795331092\\
415	0.00318873408097052\\
416	0.00317996444183571\\
417	0.00317104601964941\\
418	0.0031619757002138\\
419	0.00315275026644325\\
420	0.00314336639261583\\
421	0.00313382063738896\\
422	0.00312410943678926\\
423	0.00311422910202385\\
424	0.00310417581523831\\
425	0.00309394561880077\\
426	0.00308353440793545\\
427	0.00307293792291095\\
428	0.00306215174075572\\
429	0.00305117126647352\\
430	0.00303999172372835\\
431	0.00302860814496947\\
432	0.0030170153609647\\
433	0.00300520798970998\\
434	0.00299318042468135\\
435	0.00298092682239276\\
436	0.00296844108921502\\
437	0.00295571686738979\\
438	0.00294274752011353\\
439	0.00292952611540876\\
440	0.00291604540808881\\
441	0.00290229781811762\\
442	0.00288827540156554\\
443	0.00287396980789372\\
444	0.00285937222539627\\
445	0.00284447339254048\\
446	0.00282926403681843\\
447	0.0028137358241746\\
448	0.00279787831489232\\
449	0.00278168049455409\\
450	0.00276513081731052\\
451	0.00274821718821128\\
452	0.00273092691441245\\
453	0.00271324658327365\\
454	0.00269516167107363\\
455	0.00267665513762852\\
456	0.0026577021184723\\
457	0.00263824930352147\\
458	0.00261813291909802\\
459	0.00259690828776238\\
460	0.00257525010215384\\
461	0.00255329429443533\\
462	0.00253101105647798\\
463	0.00250831014118649\\
464	0.00248489435582048\\
465	0.00246008673831945\\
466	0.00243435385994019\\
467	0.00240879072737593\\
468	0.00237947399244218\\
469	0.0023482221547078\\
470	0.00231642921410384\\
471	0.0022841442400962\\
472	0.00225135420759003\\
473	0.00221804954900144\\
474	0.00218423389328816\\
475	0.00214995437136689\\
476	0.00211539324347675\\
477	0.00208111881827806\\
478	0.00204850191368291\\
479	0.00201860096918738\\
480	0.0019879012972538\\
481	0.00195636036166827\\
482	0.00192393154007199\\
483	0.00189056366105782\\
484	0.00185619990513972\\
485	0.00182077534695337\\
486	0.00178420965799812\\
487	0.00174637801333911\\
488	0.00170696088632293\\
489	0.00166344164530458\\
490	0.00161757795728889\\
491	0.00157061342368385\\
492	0.00152251019137918\\
493	0.00147322862401854\\
494	0.00142272720797365\\
495	0.00137096245850899\\
496	0.00131788886726747\\
497	0.00126345902010643\\
498	0.00120762420237788\\
499	0.00115033542055204\\
500	0.00109153966695643\\
501	0.00103116247127905\\
502	0.000969113373693511\\
503	0.000905317299843979\\
504	0.000839604316548774\\
505	0.000771463659200474\\
506	0.000701477745376201\\
507	0.000629526791820348\\
508	0.000554952970295424\\
509	0.000476998662898304\\
510	0.000397384018534237\\
511	0.000316032786533761\\
512	0.00023278160760913\\
513	0.000147012106123333\\
514	5.53656290951778e-05\\
515	0\\
516	0\\
517	0\\
518	0\\
519	0\\
520	0\\
521	0\\
522	0\\
523	0\\
524	0\\
525	0\\
526	0\\
527	0\\
528	0\\
529	0\\
530	0\\
531	0\\
532	0\\
533	0\\
534	0\\
535	0\\
536	0\\
537	0\\
538	0\\
539	0\\
540	0\\
541	0\\
542	0\\
543	0\\
544	0\\
545	0\\
546	0\\
547	0\\
548	0\\
549	0\\
550	0\\
551	0\\
552	0\\
553	0\\
554	0\\
555	0\\
556	0\\
557	0\\
558	0\\
559	0\\
560	0\\
561	0\\
562	0\\
563	0\\
564	0\\
565	0\\
566	0\\
567	0\\
568	0\\
569	0\\
570	0\\
571	0\\
572	0\\
573	0\\
574	0\\
575	0\\
576	0\\
577	0\\
578	0\\
579	0\\
580	0\\
581	0\\
582	0\\
583	0\\
584	0\\
585	0\\
586	0\\
587	0\\
588	0\\
589	0\\
590	0\\
591	0\\
592	0\\
593	0\\
594	0\\
595	0\\
596	0\\
597	0\\
598	0\\
599	0\\
600	0\\
};
\addplot [color=mycolor11,solid,forget plot]
  table[row sep=crcr]{%
1	0.00516053505583464\\
2	0.00516053396220163\\
3	0.00516053284984937\\
4	0.00516053171845674\\
5	0.0051605305676971\\
6	0.00516052939723823\\
7	0.00516052820674222\\
8	0.00516052699586537\\
9	0.00516052576425811\\
10	0.00516052451156487\\
11	0.005160523237424\\
12	0.00516052194146766\\
13	0.00516052062332171\\
14	0.00516051928260563\\
15	0.00516051791893236\\
16	0.00516051653190824\\
17	0.00516051512113286\\
18	0.00516051368619896\\
19	0.00516051222669234\\
20	0.00516051074219169\\
21	0.00516050923226851\\
22	0.00516050769648695\\
23	0.00516050613440375\\
24	0.00516050454556803\\
25	0.00516050292952122\\
26	0.00516050128579693\\
27	0.00516049961392076\\
28	0.00516049791341025\\
29	0.00516049618377466\\
30	0.00516049442451487\\
31	0.00516049263512326\\
32	0.00516049081508352\\
33	0.0051604889638705\\
34	0.00516048708095013\\
35	0.00516048516577919\\
36	0.00516048321780519\\
37	0.00516048123646621\\
38	0.00516047922119073\\
39	0.0051604771713975\\
40	0.00516047508649533\\
41	0.00516047296588294\\
42	0.0051604708089488\\
43	0.00516046861507095\\
44	0.00516046638361682\\
45	0.00516046411394304\\
46	0.0051604618053953\\
47	0.00516045945730809\\
48	0.00516045706900458\\
49	0.00516045463979641\\
50	0.00516045216898348\\
51	0.00516044965585375\\
52	0.00516044709968305\\
53	0.00516044449973488\\
54	0.0051604418552602\\
55	0.0051604391654972\\
56	0.0051604364296711\\
57	0.00516043364699393\\
58	0.00516043081666431\\
59	0.00516042793786721\\
60	0.00516042500977374\\
61	0.00516042203154088\\
62	0.00516041900231132\\
63	0.0051604159212131\\
64	0.00516041278735947\\
65	0.0051604095998486\\
66	0.00516040635776331\\
67	0.00516040306017083\\
68	0.00516039970612255\\
69	0.00516039629465373\\
70	0.00516039282478322\\
71	0.00516038929551324\\
72	0.00516038570582903\\
73	0.00516038205469859\\
74	0.00516037834107243\\
75	0.00516037456388322\\
76	0.00516037072204551\\
77	0.00516036681445543\\
78	0.00516036283999039\\
79	0.00516035879750876\\
80	0.00516035468584951\\
81	0.00516035050383195\\
82	0.00516034625025537\\
83	0.00516034192389869\\
84	0.00516033752352013\\
85	0.00516033304785689\\
86	0.00516032849562474\\
87	0.00516032386551772\\
88	0.00516031915620774\\
89	0.00516031436634421\\
90	0.00516030949455368\\
91	0.00516030453943946\\
92	0.0051602994995812\\
93	0.00516029437353453\\
94	0.00516028915983063\\
95	0.00516028385697584\\
96	0.00516027846345123\\
97	0.00516027297771219\\
98	0.00516026739818799\\
99	0.00516026172328136\\
100	0.00516025595136801\\
101	0.00516025008079623\\
102	0.00516024410988637\\
103	0.00516023803693045\\
104	0.00516023186019159\\
105	0.00516022557790363\\
106	0.00516021918827055\\
107	0.00516021268946602\\
108	0.00516020607963291\\
109	0.0051601993568827\\
110	0.00516019251929506\\
111	0.00516018556491722\\
112	0.00516017849176349\\
113	0.00516017129781467\\
114	0.00516016398101754\\
115	0.00516015653928424\\
116	0.00516014897049172\\
117	0.00516014127248113\\
118	0.00516013344305727\\
119	0.00516012547998791\\
120	0.00516011738100323\\
121	0.00516010914379517\\
122	0.00516010076601679\\
123	0.00516009224528163\\
124	0.00516008357916303\\
125	0.00516007476519348\\
126	0.00516006580086392\\
127	0.00516005668362308\\
128	0.00516004741087672\\
129	0.00516003797998697\\
130	0.00516002838827154\\
131	0.00516001863300304\\
132	0.0051600087114082\\
133	0.0051599986206671\\
134	0.00515998835791238\\
135	0.00515997792022849\\
136	0.00515996730465084\\
137	0.00515995650816502\\
138	0.00515994552770596\\
139	0.00515993436015705\\
140	0.00515992300234932\\
141	0.00515991145106056\\
142	0.00515989970301442\\
143	0.00515988775487952\\
144	0.00515987560326854\\
145	0.00515986324473728\\
146	0.00515985067578372\\
147	0.00515983789284705\\
148	0.00515982489230669\\
149	0.00515981167048129\\
150	0.00515979822362773\\
151	0.00515978454794009\\
152	0.00515977063954862\\
153	0.00515975649451863\\
154	0.00515974210884946\\
155	0.00515972747847336\\
156	0.00515971259925437\\
157	0.00515969746698721\\
158	0.00515968207739608\\
159	0.00515966642613356\\
160	0.00515965050877932\\
161	0.00515963432083899\\
162	0.0051596178577429\\
163	0.00515960111484483\\
164	0.00515958408742071\\
165	0.00515956677066737\\
166	0.00515954915970118\\
167	0.00515953124955675\\
168	0.00515951303518555\\
169	0.0051594945114545\\
170	0.00515947567314463\\
171	0.00515945651494957\\
172	0.00515943703147415\\
173	0.00515941721723292\\
174	0.00515939706664861\\
175	0.00515937657405062\\
176	0.00515935573367348\\
177	0.00515933453965525\\
178	0.00515931298603592\\
179	0.00515929106675577\\
180	0.00515926877565369\\
181	0.00515924610646555\\
182	0.00515922305282241\\
183	0.00515919960824879\\
184	0.00515917576616094\\
185	0.00515915151986495\\
186	0.00515912686255499\\
187	0.00515910178731138\\
188	0.00515907628709871\\
189	0.00515905035476392\\
190	0.0051590239830343\\
191	0.0051589971645155\\
192	0.00515896989168953\\
193	0.0051589421569126\\
194	0.00515891395241311\\
195	0.00515888527028945\\
196	0.00515885610250783\\
197	0.00515882644090004\\
198	0.00515879627716127\\
199	0.00515876560284769\\
200	0.00515873440937424\\
201	0.00515870268801215\\
202	0.00515867042988658\\
203	0.00515863762597415\\
204	0.00515860426710041\\
205	0.00515857034393732\\
206	0.00515853584700062\\
207	0.00515850076664722\\
208	0.00515846509307253\\
209	0.00515842881630766\\
210	0.00515839192621668\\
211	0.0051583544124938\\
212	0.00515831626466043\\
213	0.00515827747206232\\
214	0.00515823802386648\\
215	0.0051581979090582\\
216	0.00515815711643796\\
217	0.00515811563461819\\
218	0.00515807345202016\\
219	0.00515803055687064\\
220	0.00515798693719857\\
221	0.0051579425808317\\
222	0.00515789747539312\\
223	0.00515785160829771\\
224	0.00515780496674857\\
225	0.00515775753773339\\
226	0.00515770930802069\\
227	0.00515766026415605\\
228	0.00515761039245821\\
229	0.00515755967901517\\
230	0.00515750810968016\\
231	0.00515745567006754\\
232	0.00515740234554864\\
233	0.0051573481212475\\
234	0.00515729298203654\\
235	0.00515723691253213\\
236	0.0051571798970901\\
237	0.00515712191980113\\
238	0.00515706296448607\\
239	0.00515700301469116\\
240	0.00515694205368316\\
241	0.00515688006444434\\
242	0.00515681702966746\\
243	0.00515675293175053\\
244	0.00515668775279156\\
245	0.00515662147458317\\
246	0.00515655407860704\\
247	0.00515648554602832\\
248	0.00515641585768987\\
249	0.00515634499410639\\
250	0.00515627293545845\\
251	0.00515619966158632\\
252	0.00515612515198376\\
253	0.00515604938579157\\
254	0.00515597234179111\\
255	0.00515589399839756\\
256	0.0051558143336531\\
257	0.00515573332521991\\
258	0.00515565095037301\\
259	0.00515556718599293\\
260	0.00515548200855819\\
261	0.00515539539413762\\
262	0.00515530731838246\\
263	0.00515521775651835\\
264	0.00515512668333697\\
265	0.00515503407318759\\
266	0.00515493989996835\\
267	0.0051548441371173\\
268	0.00515474675760322\\
269	0.0051546477339162\\
270	0.0051545470380579\\
271	0.00515444464153171\\
272	0.00515434051533257\\
273	0.00515423462993652\\
274	0.00515412695528961\\
275	0.00515401746079621\\
276	0.00515390611530788\\
277	0.00515379288711148\\
278	0.00515367774391692\\
279	0.00515356065284438\\
280	0.00515344158041124\\
281	0.00515332049251849\\
282	0.00515319735443667\\
283	0.00515307213079131\\
284	0.00515294478554787\\
285	0.00515281528199611\\
286	0.00515268358273379\\
287	0.00515254964964986\\
288	0.00515241344390688\\
289	0.00515227492592275\\
290	0.00515213405535171\\
291	0.00515199079106441\\
292	0.00515184509112727\\
293	0.00515169691278076\\
294	0.00515154621241674\\
295	0.00515139294555468\\
296	0.00515123706681681\\
297	0.0051510785299019\\
298	0.00515091728755775\\
299	0.00515075329155227\\
300	0.0051505864926429\\
301	0.0051504168405444\\
302	0.00515024428389477\\
303	0.00515006877021921\\
304	0.00514989024589186\\
305	0.00514970865609519\\
306	0.00514952394477684\\
307	0.00514933605460346\\
308	0.00514914492691124\\
309	0.00514895050165295\\
310	0.00514875271734128\\
311	0.00514855151098949\\
312	0.00514834681804893\\
313	0.00514813857233824\\
314	0.0051479267059676\\
315	0.00514771114925784\\
316	0.00514749183065296\\
317	0.00514726867662538\\
318	0.00514704161157321\\
319	0.00514681055770862\\
320	0.00514657543493639\\
321	0.00514633616072161\\
322	0.00514609264994523\\
323	0.00514584481474628\\
324	0.0051455925643491\\
325	0.00514533580487412\\
326	0.0051450744391301\\
327	0.00514480836638589\\
328	0.00514453748211933\\
329	0.0051442616777406\\
330	0.00514398084028718\\
331	0.005143694852087\\
332	0.00514340359038611\\
333	0.00514310692693669\\
334	0.00514280472754071\\
335	0.00514249685154394\\
336	0.00514218315127439\\
337	0.00514186347141843\\
338	0.00514153764832721\\
339	0.00514120550924475\\
340	0.00514086687144821\\
341	0.00514052154128956\\
342	0.00514016931312651\\
343	0.00513980996812899\\
344	0.00513944327294557\\
345	0.00513906897821246\\
346	0.00513868681688523\\
347	0.00513829650237064\\
348	0.00513789772643345\\
349	0.00513749015684897\\
350	0.0051370734347688\\
351	0.00513664717176195\\
352	0.00513621094648797\\
353	0.00513576430094946\\
354	0.00513530673625371\\
355	0.00513483770777226\\
356	0.00513435661949786\\
357	0.00513386281723007\\
358	0.00513335557972848\\
359	0.00513283410557918\\
360	0.00513229749024203\\
361	0.00513174467925832\\
362	0.00513117436263682\\
363	0.00513058472537829\\
364	0.00512997285665973\\
365	0.00512933339553511\\
366	0.0051286556461704\\
367	0.00512791809116632\\
368	0.00512707556718617\\
369	0.00512621365677636\\
370	0.00512533867006539\\
371	0.00512445043152696\\
372	0.00512354876402859\\
373	0.00512263348865814\\
374	0.00512170442482359\\
375	0.00512076139150791\\
376	0.00511980421108008\\
377	0.00511883271336735\\
378	0.00511784671898175\\
379	0.00511684604105501\\
380	0.00511583049149508\\
381	0.00511479988093934\\
382	0.00511375401868709\\
383	0.00511269271260762\\
384	0.00511161576901979\\
385	0.00511052299253852\\
386	0.00510941418588275\\
387	0.00510828914963941\\
388	0.00510714768197668\\
389	0.00510598957829947\\
390	0.00510481463083948\\
391	0.00510362262817123\\
392	0.00510241335464608\\
393	0.00510118658973624\\
394	0.00509994210728471\\
395	0.0050986796746647\\
396	0.00509739905187089\\
397	0.00509609999060445\\
398	0.0050947822334919\\
399	0.0050934455137228\\
400	0.00509208955564237\\
401	0.00509071407725901\\
402	0.00508931879649426\\
403	0.00508790344576597\\
404	0.00508646781104311\\
405	0.00508501185636\\
406	0.00508353610539178\\
407	0.00508204219128767\\
408	0.00508053380441878\\
409	0.00507901748525117\\
410	0.00507749997480398\\
411	0.00507597297051831\\
412	0.00507441994867542\\
413	0.00507284042967778\\
414	0.00507123388203909\\
415	0.00506959973481857\\
416	0.00506793739819613\\
417	0.00506624626253051\\
418	0.0050645256973056\\
419	0.00506277504987666\\
420	0.00506099364396145\\
421	0.00505918077819892\\
422	0.00505733572583228\\
423	0.00505545773387667\\
424	0.00505354602078519\\
425	0.00505159977490034\\
426	0.00504961815280787\\
427	0.00504760027758672\\
428	0.00504554523694768\\
429	0.00504345208125428\\
430	0.00504131982141767\\
431	0.00503914742665731\\
432	0.00503693382211835\\
433	0.00503467788633475\\
434	0.00503237844852357\\
435	0.00503003428568796\\
436	0.00502764411948796\\
437	0.00502520661279766\\
438	0.00502272036578203\\
439	0.00502018391115925\\
440	0.00501759570803928\\
441	0.00501495413345738\\
442	0.00501225747122418\\
443	0.00500950390204384\\
444	0.00500669151594826\\
445	0.00500381841065281\\
446	0.00500088292228883\\
447	0.0049978829043709\\
448	0.00499481608057586\\
449	0.00499168004997979\\
450	0.00498847227099568\\
451	0.00498519002024984\\
452	0.00498183029229815\\
453	0.00497838954526938\\
454	0.00497486305463673\\
455	0.00497124330908564\\
456	0.00496751623004757\\
457	0.00496365317341908\\
458	0.00495959803985892\\
459	0.00495545847701009\\
460	0.00495125238269577\\
461	0.0049469672450374\\
462	0.00494257469975514\\
463	0.00493801023687081\\
464	0.00493315076640688\\
465	0.00492798717113069\\
466	0.00492242062516231\\
467	0.00491552816158838\\
468	0.00490322237550712\\
469	0.00488875730748457\\
470	0.0048740630976783\\
471	0.0048591286169599\\
472	0.00484394361654972\\
473	0.00482850186063932\\
474	0.00481280854466351\\
475	0.00479689621963633\\
476	0.00478085290830506\\
477	0.00476484288526039\\
478	0.00474893442827621\\
479	0.0047325699405543\\
480	0.00471571948336329\\
481	0.00469834975067911\\
482	0.00468042346135682\\
483	0.00466189768238898\\
484	0.00464271982189616\\
485	0.00462281735643028\\
486	0.00460207060107256\\
487	0.00458023940772211\\
488	0.00455676418504533\\
489	0.00452948477304095\\
490	0.00450046664968492\\
491	0.00447100333840286\\
492	0.00444108528019643\\
493	0.0044107025642607\\
494	0.00437984492848087\\
495	0.00434850177656219\\
496	0.00431666220770141\\
497	0.00428431500858952\\
498	0.00425144835853428\\
499	0.00421804861420509\\
500	0.00418409693491697\\
501	0.00414957075012113\\
502	0.00411444927293261\\
503	0.00407869978160293\\
504	0.00404226856299079\\
505	0.00400519790882657\\
506	0.00396741722630471\\
507	0.0039287588917257\\
508	0.00388876341634874\\
509	0.00384719310549665\\
510	0.00380535624004568\\
511	0.00376318093942821\\
512	0.0037204691751195\\
513	0.00367670479506395\\
514	0.00363073524723451\\
515	0.00358294946602486\\
516	0.00353505653663763\\
517	0.00348709399632349\\
518	0.00343914842590566\\
519	0.00339142939890124\\
520	0.00334444598783759\\
521	0.00329937957587658\\
522	0.00325854851736096\\
523	0.00322419649826356\\
524	0.00318922244626424\\
525	0.00315326103247786\\
526	0.00311514836296493\\
527	0.00307123341315276\\
528	0.0030098509516261\\
529	0.00294020748052116\\
530	0.00286758526591559\\
531	0.00279104864096957\\
532	0.00270989433435107\\
533	0.00262004142303426\\
534	0.00252218784358832\\
535	0.00242178491691014\\
536	0.00231871896930725\\
537	0.00221285935169346\\
538	0.00210403327127977\\
539	0.00199194507799155\\
540	0.00187591619531658\\
541	0.00175411070440876\\
542	0.00162813783804569\\
543	0.00149958178120082\\
544	0.00136835829867665\\
545	0.00123439189756769\\
546	0.00109760179491459\\
547	0.000957899619460617\\
548	0.000815177798917114\\
549	0.000669268858551629\\
550	0.000519806197145862\\
551	0.000365747564450615\\
552	0.000203770911622493\\
553	2.3893647376284e-05\\
554	0\\
555	0\\
556	0\\
557	0\\
558	0\\
559	0\\
560	0\\
561	0\\
562	0\\
563	0\\
564	0\\
565	0\\
566	0\\
567	0\\
568	0\\
569	0\\
570	0\\
571	0\\
572	0\\
573	0\\
574	0\\
575	0\\
576	0\\
577	0\\
578	0\\
579	0\\
580	0\\
581	0\\
582	0\\
583	0\\
584	0\\
585	0\\
586	0\\
587	0\\
588	0\\
589	0\\
590	0\\
591	0\\
592	0\\
593	0\\
594	0\\
595	0\\
596	0\\
597	0\\
598	0\\
599	0\\
600	0\\
};
\addplot [color=mycolor12,solid,forget plot]
  table[row sep=crcr]{%
1	0.00638811260735049\\
2	0.00638811227827753\\
3	0.00638811194357173\\
4	0.00638811160313645\\
5	0.00638811125687342\\
6	0.00638811090468267\\
7	0.00638811054646253\\
8	0.00638811018210957\\
9	0.00638810981151861\\
10	0.00638810943458267\\
11	0.00638810905119292\\
12	0.00638810866123869\\
13	0.00638810826460739\\
14	0.00638810786118453\\
15	0.00638810745085365\\
16	0.0063881070334963\\
17	0.006388106608992\\
18	0.0063881061772182\\
19	0.00638810573805029\\
20	0.00638810529136148\\
21	0.00638810483702286\\
22	0.00638810437490328\\
23	0.00638810390486936\\
24	0.00638810342678545\\
25	0.00638810294051357\\
26	0.00638810244591339\\
27	0.00638810194284216\\
28	0.00638810143115473\\
29	0.00638810091070344\\
30	0.00638810038133811\\
31	0.00638809984290601\\
32	0.00638809929525179\\
33	0.00638809873821745\\
34	0.00638809817164229\\
35	0.00638809759536288\\
36	0.00638809700921297\\
37	0.00638809641302351\\
38	0.00638809580662254\\
39	0.00638809518983516\\
40	0.00638809456248351\\
41	0.00638809392438669\\
42	0.00638809327536069\\
43	0.00638809261521839\\
44	0.00638809194376946\\
45	0.00638809126082035\\
46	0.00638809056617417\\
47	0.00638808985963071\\
48	0.00638808914098634\\
49	0.00638808841003393\\
50	0.00638808766656286\\
51	0.00638808691035891\\
52	0.0063880861412042\\
53	0.00638808535887714\\
54	0.00638808456315237\\
55	0.00638808375380071\\
56	0.00638808293058904\\
57	0.00638808209328031\\
58	0.00638808124163341\\
59	0.00638808037540313\\
60	0.0063880794943401\\
61	0.00638807859819071\\
62	0.00638807768669701\\
63	0.00638807675959669\\
64	0.00638807581662296\\
65	0.00638807485750451\\
66	0.00638807388196541\\
67	0.00638807288972504\\
68	0.00638807188049802\\
69	0.0063880708539941\\
70	0.00638806980991812\\
71	0.0063880687479699\\
72	0.00638806766784416\\
73	0.00638806656923044\\
74	0.00638806545181301\\
75	0.00638806431527078\\
76	0.0063880631592772\\
77	0.00638806198350021\\
78	0.00638806078760207\\
79	0.00638805957123936\\
80	0.00638805833406281\\
81	0.00638805707571723\\
82	0.00638805579584141\\
83	0.00638805449406802\\
84	0.0063880531700235\\
85	0.00638805182332797\\
86	0.00638805045359511\\
87	0.00638804906043206\\
88	0.00638804764343931\\
89	0.00638804620221056\\
90	0.00638804473633268\\
91	0.0063880432453855\\
92	0.00638804172894177\\
93	0.00638804018656701\\
94	0.00638803861781938\\
95	0.00638803702224957\\
96	0.00638803539940067\\
97	0.00638803374880806\\
98	0.00638803206999925\\
99	0.00638803036249378\\
100	0.00638802862580305\\
101	0.00638802685943024\\
102	0.00638802506287009\\
103	0.00638802323560886\\
104	0.00638802137712409\\
105	0.00638801948688452\\
106	0.00638801756434994\\
107	0.00638801560897098\\
108	0.00638801362018904\\
109	0.00638801159743608\\
110	0.00638800954013446\\
111	0.00638800744769684\\
112	0.00638800531952593\\
113	0.00638800315501439\\
114	0.00638800095354464\\
115	0.00638799871448869\\
116	0.00638799643720797\\
117	0.00638799412105314\\
118	0.00638799176536393\\
119	0.00638798936946893\\
120	0.00638798693268543\\
121	0.00638798445431922\\
122	0.00638798193366442\\
123	0.00638797937000323\\
124	0.00638797676260578\\
125	0.00638797411072992\\
126	0.00638797141362101\\
127	0.0063879686705117\\
128	0.00638796588062174\\
129	0.00638796304315773\\
130	0.00638796015731296\\
131	0.00638795722226712\\
132	0.00638795423718612\\
133	0.00638795120122183\\
134	0.00638794811351189\\
135	0.00638794497317941\\
136	0.00638794177933278\\
137	0.0063879385310654\\
138	0.00638793522745544\\
139	0.00638793186756558\\
140	0.00638792845044275\\
141	0.00638792497511786\\
142	0.00638792144060558\\
143	0.00638791784590399\\
144	0.00638791418999439\\
145	0.00638791047184096\\
146	0.0063879066903905\\
147	0.00638790284457213\\
148	0.006387898933297\\
149	0.006387894955458\\
150	0.00638789090992943\\
151	0.00638788679556675\\
152	0.00638788261120617\\
153	0.00638787835566442\\
154	0.0063878740277384\\
155	0.00638786962620481\\
156	0.00638786514981987\\
157	0.00638786059731894\\
158	0.0063878559674162\\
159	0.00638785125880428\\
160	0.0063878464701539\\
161	0.00638784160011354\\
162	0.00638783664730901\\
163	0.00638783161034311\\
164	0.00638782648779528\\
165	0.00638782127822112\\
166	0.00638781598015209\\
167	0.00638781059209505\\
168	0.00638780511253186\\
169	0.00638779953991899\\
170	0.00638779387268706\\
171	0.00638778810924043\\
172	0.00638778224795679\\
173	0.00638777628718663\\
174	0.00638777022525291\\
175	0.00638776406045047\\
176	0.00638775779104569\\
177	0.00638775141527589\\
178	0.00638774493134897\\
179	0.00638773833744281\\
180	0.00638773163170484\\
181	0.00638772481225151\\
182	0.00638771787716775\\
183	0.00638771082450651\\
184	0.00638770365228814\\
185	0.00638769635849991\\
186	0.00638768894109543\\
187	0.00638768139799409\\
188	0.00638767372708052\\
189	0.00638766592620393\\
190	0.00638765799317761\\
191	0.00638764992577827\\
192	0.00638764172174544\\
193	0.00638763337878085\\
194	0.00638762489454782\\
195	0.00638761626667057\\
196	0.00638760749273358\\
197	0.00638759857028096\\
198	0.00638758949681571\\
199	0.00638758026979909\\
200	0.00638757088664987\\
201	0.00638756134474363\\
202	0.00638755164141204\\
203	0.00638754177394213\\
204	0.0063875317395755\\
205	0.00638752153550759\\
206	0.00638751115888687\\
207	0.00638750060681408\\
208	0.0063874898763414\\
209	0.00638747896447161\\
210	0.0063874678681573\\
211	0.00638745658429997\\
212	0.00638744510974919\\
213	0.00638743344130171\\
214	0.00638742157570056\\
215	0.00638740950963414\\
216	0.00638739723973526\\
217	0.00638738476258024\\
218	0.0063873720746879\\
219	0.00638735917251859\\
220	0.00638734605247319\\
221	0.00638733271089208\\
222	0.00638731914405413\\
223	0.00638730534817558\\
224	0.00638729131940902\\
225	0.00638727705384226\\
226	0.00638726254749722\\
227	0.00638724779632878\\
228	0.00638723279622361\\
229	0.00638721754299901\\
230	0.00638720203240168\\
231	0.00638718626010648\\
232	0.0063871702217152\\
233	0.00638715391275527\\
234	0.00638713732867844\\
235	0.00638712046485946\\
236	0.00638710331659474\\
237	0.00638708587910093\\
238	0.00638706814751352\\
239	0.00638705011688543\\
240	0.00638703178218551\\
241	0.00638701313829702\\
242	0.00638699418001616\\
243	0.00638697490205046\\
244	0.00638695529901721\\
245	0.0063869353654418\\
246	0.00638691509575612\\
247	0.00638689448429679\\
248	0.00638687352530348\\
249	0.00638685221291712\\
250	0.00638683054117809\\
251	0.00638680850402438\\
252	0.00638678609528968\\
253	0.00638676330870148\\
254	0.00638674013787907\\
255	0.00638671657633152\\
256	0.00638669261745562\\
257	0.00638666825453377\\
258	0.00638664348073181\\
259	0.00638661828909679\\
260	0.0063865926725547\\
261	0.00638656662390817\\
262	0.00638654013583404\\
263	0.00638651320088097\\
264	0.00638648581146687\\
265	0.00638645795987635\\
266	0.0063864296382581\\
267	0.00638640083862214\\
268	0.00638637155283704\\
269	0.00638634177262707\\
270	0.00638631148956926\\
271	0.00638628069509043\\
272	0.00638624938046401\\
273	0.00638621753680686\\
274	0.00638618515507574\\
275	0.00638615222606409\\
276	0.00638611874039846\\
277	0.00638608468853491\\
278	0.00638605006075521\\
279	0.00638601484716301\\
280	0.00638597903767974\\
281	0.00638594262204055\\
282	0.00638590558978992\\
283	0.00638586793027723\\
284	0.00638582963265214\\
285	0.00638579068585975\\
286	0.00638575107863561\\
287	0.00638571079950053\\
288	0.00638566983675517\\
289	0.00638562817847436\\
290	0.00638558581250126\\
291	0.00638554272644116\\
292	0.00638549890765511\\
293	0.00638545434325314\\
294	0.00638540902008723\\
295	0.00638536292474393\\
296	0.00638531604353653\\
297	0.00638526836249695\\
298	0.00638521986736709\\
299	0.00638517054358977\\
300	0.00638512037629914\\
301	0.00638506935031051\\
302	0.00638501745010968\\
303	0.00638496465984148\\
304	0.00638491096329772\\
305	0.00638485634390431\\
306	0.00638480078470742\\
307	0.00638474426835884\\
308	0.00638468677710006\\
309	0.00638462829274554\\
310	0.00638456879666494\\
311	0.00638450826976433\\
312	0.00638444669246502\\
313	0.00638438404468094\\
314	0.00638432030579448\\
315	0.00638425545463028\\
316	0.00638418946942683\\
317	0.00638412232780574\\
318	0.00638405400673823\\
319	0.00638398448250878\\
320	0.00638391373067538\\
321	0.00638384172602621\\
322	0.00638376844253228\\
323	0.00638369385329557\\
324	0.00638361793049211\\
325	0.00638354064530964\\
326	0.00638346196787895\\
327	0.00638338186719835\\
328	0.00638330031105052\\
329	0.00638321726591067\\
330	0.00638313269684517\\
331	0.00638304656739947\\
332	0.00638295883947404\\
333	0.00638286947318688\\
334	0.0063827784267211\\
335	0.0063826856561557\\
336	0.00638259111527748\\
337	0.00638249475537197\\
338	0.00638239652499067\\
339	0.00638229636969169\\
340	0.00638219423175074\\
341	0.00638209004983853\\
342	0.00638198375866058\\
343	0.00638187528855478\\
344	0.00638176456504129\\
345	0.00638165150831882\\
346	0.00638153603270058\\
347	0.00638141804598205\\
348	0.00638129744873196\\
349	0.00638117413349642\\
350	0.00638104798390462\\
351	0.00638091887366286\\
352	0.00638078666541956\\
353	0.00638065120947726\\
354	0.00638051234231107\\
355	0.00638036988481699\\
356	0.00638022364013296\\
357	0.0063800733906508\\
358	0.00637991889324608\\
359	0.00637975987032702\\
360	0.0063795959907522\\
361	0.00637942682626216\\
362	0.00637925175035374\\
363	0.00637906970942778\\
364	0.00637887874041027\\
365	0.00637867510162692\\
366	0.00637845230977798\\
367	0.00637820317999826\\
368	0.0063779486827292\\
369	0.00637769033540312\\
370	0.00637742808668752\\
371	0.00637716188478615\\
372	0.00637689167741668\\
373	0.00637661741192001\\
374	0.00637633903573714\\
375	0.00637605649738284\\
376	0.00637576974673577\\
377	0.00637547873112774\\
378	0.00637518339603238\\
379	0.00637488368657652\\
380	0.00637457954752541\\
381	0.00637427092326166\\
382	0.00637395775775675\\
383	0.00637363999453385\\
384	0.00637331757662061\\
385	0.00637299044649023\\
386	0.0063726585459893\\
387	0.00637232181625021\\
388	0.00637198019758615\\
389	0.00637163362936644\\
390	0.00637128204986959\\
391	0.00637092539611196\\
392	0.00637056360364985\\
393	0.00637019660635467\\
394	0.00636982433616361\\
395	0.00636944672281565\\
396	0.00636906369359698\\
397	0.00636867517314809\\
398	0.00636828108343642\\
399	0.00636788134409195\\
400	0.00636747587349228\\
401	0.00636706459147168\\
402	0.00636664742611355\\
403	0.00636622433229439\\
404	0.00636579534338305\\
405	0.00636536069391904\\
406	0.00636492099227602\\
407	0.00636447744632584\\
408	0.00636403184281951\\
409	0.00636358533742221\\
410	0.0063631352141959\\
411	0.00636267744808165\\
412	0.0063622118967298\\
413	0.00636173840340145\\
414	0.00636125680084869\\
415	0.00636076691629268\\
416	0.00636026857114308\\
417	0.00635976158068075\\
418	0.00635924575368311\\
419	0.00635872089199831\\
420	0.00635818679017809\\
421	0.00635764323535161\\
422	0.00635709000691684\\
423	0.00635652687590073\\
424	0.00635595360450518\\
425	0.00635536994562446\\
426	0.00635477564233225\\
427	0.00635417042733614\\
428	0.0063535540223977\\
429	0.00635292613771579\\
430	0.00635228647127053\\
431	0.00635163470812541\\
432	0.0063509705196839\\
433	0.0063502935628958\\
434	0.00634960347940539\\
435	0.00634889989462626\\
436	0.00634818241671315\\
437	0.00634745063537106\\
438	0.00634670412039079\\
439	0.00634594241973513\\
440	0.00634516505701116\\
441	0.00634437152861232\\
442	0.00634356130271134\\
443	0.00634273382722985\\
444	0.00634188855859463\\
445	0.0063410249902627\\
446	0.00634014249251388\\
447	0.00633924039810001\\
448	0.00633831800238986\\
449	0.00633737455658749\\
450	0.0063364092506127\\
451	0.00633542117093243\\
452	0.00633440919541388\\
453	0.00633337173729664\\
454	0.00633230615981661\\
455	0.00633120761774576\\
456	0.00633006763021697\\
457	0.00632887711133775\\
458	0.00632766282066212\\
459	0.00632642830815137\\
460	0.00632516878913098\\
461	0.00632387383674367\\
462	0.00632252232735194\\
463	0.00632108164693185\\
464	0.00631953543083207\\
465	0.00631779877024216\\
466	0.00631550642290275\\
467	0.00631181128253835\\
468	0.00630758251061971\\
469	0.0063032886687178\\
470	0.00629892670499869\\
471	0.00629449405557355\\
472	0.00628998981684965\\
473	0.00628541731536539\\
474	0.00628078894352176\\
475	0.00627613232540686\\
476	0.00627148627539327\\
477	0.00626684319508133\\
478	0.00626204858967383\\
479	0.00625711260959464\\
480	0.00625202665348447\\
481	0.00624677980986374\\
482	0.00624135926103719\\
483	0.00623574865960632\\
484	0.00622992393734656\\
485	0.00622384233034361\\
486	0.00621741401359264\\
487	0.00621043315997757\\
488	0.00620244420629681\\
489	0.00619400965776899\\
490	0.0061854524589268\\
491	0.00617676989141488\\
492	0.00616795912551411\\
493	0.0061590172263189\\
494	0.00614994117299156\\
495	0.00614072789386952\\
496	0.00613137429602339\\
497	0.00612187716581177\\
498	0.00611223259079458\\
499	0.00610243450128191\\
500	0.00609247530938651\\
501	0.00608234825131482\\
502	0.0060720425385402\\
503	0.00606153892911889\\
504	0.00605084367335183\\
505	0.00603994004356179\\
506	0.00602877154022212\\
507	0.00601721376170444\\
508	0.00600523421996853\\
509	0.00599317503690893\\
510	0.00598101128349881\\
511	0.00596867831489254\\
512	0.00595601896616613\\
513	0.00594273840786949\\
514	0.00592896458149047\\
515	0.00591514726359842\\
516	0.00590129512228386\\
517	0.00588743147298232\\
518	0.00587361595475236\\
519	0.00585999549864396\\
520	0.00584690599328163\\
521	0.00583500592475455\\
522	0.00582490792822357\\
523	0.0058146042098332\\
524	0.00580398987612161\\
525	0.00579273376978727\\
526	0.00577980378322902\\
527	0.00576202624003748\\
528	0.00572377841157942\\
529	0.00567699475830189\\
530	0.00562869022943325\\
531	0.0055786912893133\\
532	0.00552598034901514\\
533	0.00546614625759935\\
534	0.00539998269028576\\
535	0.00533380742098718\\
536	0.00526762319228078\\
537	0.00520142728789096\\
538	0.00513519630576685\\
539	0.00506883791671606\\
540	0.00500203947876236\\
541	0.00493320527137919\\
542	0.00486390184218159\\
543	0.00479565705149332\\
544	0.00472849590663815\\
545	0.00466244415217564\\
546	0.00459752828388643\\
547	0.0045337742286058\\
548	0.00447120240486943\\
549	0.00440981133989045\\
550	0.00434952421512032\\
551	0.00429001456373431\\
552	0.00423008609656117\\
553	0.0041662083985624\\
554	0.00410400778309653\\
555	0.00404418208447621\\
556	0.00398683441474163\\
557	0.00393197442332481\\
558	0.00387962678014643\\
559	0.00382986338433296\\
560	0.00378275613597533\\
561	0.00373847095787726\\
562	0.00369740504797728\\
563	0.00366052991955191\\
564	0.00363023942217948\\
565	0.00360500619487998\\
566	0.00352973838064425\\
567	0.00344272230220717\\
568	0.00334614191579287\\
569	0.00324691932347805\\
570	0.00314497980923681\\
571	0.0030401664970947\\
572	0.00293180965279257\\
573	0.00281238234143975\\
574	0.00264080565626875\\
575	0.00246652723666886\\
576	0.00228946683670179\\
577	0.00210943841006402\\
578	0.00192677268954282\\
579	0.00174132574584865\\
580	0.00155260207180459\\
581	0.00136033065535838\\
582	0.00116605697170137\\
583	0.000969668956010576\\
584	0.000770929493206139\\
585	0.000569036622337176\\
586	0.000361344392078551\\
587	0.000140168163487911\\
588	0\\
589	0\\
590	0\\
591	0\\
592	0\\
593	0\\
594	0\\
595	0\\
596	0\\
597	0\\
598	0\\
599	0\\
600	0\\
};
\addplot [color=mycolor13,solid,forget plot]
  table[row sep=crcr]{%
1	0.00142341042040566\\
2	0.00142341042040566\\
3	0.00142341042040566\\
4	0.00142341042040566\\
5	0.00142341042040566\\
6	0.00142341042040566\\
7	0.00142341042040566\\
8	0.00142341042040566\\
9	0.00142341042040566\\
10	0.00142341042040566\\
11	0.00142341042040566\\
12	0.00142341042040566\\
13	0.00142341042040566\\
14	0.00142341042040566\\
15	0.00142341042040566\\
16	0.00142341042040566\\
17	0.00142341042040566\\
18	0.00142341042040566\\
19	0.00142341042040566\\
20	0.00142341042040566\\
21	0.00142341042040566\\
22	0.00142341042040566\\
23	0.00142341042040566\\
24	0.00142341042040566\\
25	0.00142341042040566\\
26	0.00142341042040566\\
27	0.00142341042040566\\
28	0.00142341042040566\\
29	0.00142341042040566\\
30	0.00142341042040566\\
31	0.00142341042040566\\
32	0.00142341042040566\\
33	0.00142341042040566\\
34	0.00142341042040566\\
35	0.00142341042040566\\
36	0.00142341042040566\\
37	0.00142341042040566\\
38	0.00142341042040566\\
39	0.00142341042040566\\
40	0.00142341042040566\\
41	0.00142341042040566\\
42	0.00142341042040566\\
43	0.00142341042040566\\
44	0.00142341042040566\\
45	0.00142341042040566\\
46	0.00142341042040566\\
47	0.00142341042040566\\
48	0.00142341042040566\\
49	0.00142341042040566\\
50	0.00142341042040566\\
51	0.00142341042040566\\
52	0.00142341042040566\\
53	0.00142341042040566\\
54	0.00142341042040566\\
55	0.00142341042040566\\
56	0.00142341042040566\\
57	0.00142341042040566\\
58	0.00142341042040566\\
59	0.00142341042040566\\
60	0.00142341042040566\\
61	0.00142341042040566\\
62	0.00142341042040566\\
63	0.00142341042040566\\
64	0.00142341042040566\\
65	0.00142341042040566\\
66	0.00142341042040566\\
67	0.00142341042040566\\
68	0.00142341042040566\\
69	0.00142341042040566\\
70	0.00142341042040566\\
71	0.00142341042040566\\
72	0.00142341042040566\\
73	0.00142341042040566\\
74	0.00142341042040566\\
75	0.00142341042040566\\
76	0.00142341042040566\\
77	0.00142341042040566\\
78	0.00142341042040566\\
79	0.00142341042040566\\
80	0.00142341042040566\\
81	0.00142341042040566\\
82	0.00142341042040566\\
83	0.00142341042040566\\
84	0.00142341042040566\\
85	0.00142341042040566\\
86	0.00142341042040566\\
87	0.00142341042040566\\
88	0.00142341042040566\\
89	0.00142341042040566\\
90	0.00142341042040566\\
91	0.00142341042040566\\
92	0.00142341042040566\\
93	0.00142341042040566\\
94	0.00142341042040566\\
95	0.00142341042040566\\
96	0.00142341042040566\\
97	0.00142341042040566\\
98	0.00142341042040566\\
99	0.00142341042040566\\
100	0.00142341042040566\\
101	0.00142341042040566\\
102	0.00142341042040566\\
103	0.00142341042040566\\
104	0.00142341042040566\\
105	0.00142341042040566\\
106	0.00142341042040566\\
107	0.00142341042040566\\
108	0.00142341042040566\\
109	0.00142341042040566\\
110	0.00142341042040566\\
111	0.00142341042040566\\
112	0.00142341042040566\\
113	0.00142341042040566\\
114	0.00142341042040566\\
115	0.00142341042040566\\
116	0.00142341042040566\\
117	0.00142341042040566\\
118	0.00142341042040566\\
119	0.00142341042040566\\
120	0.00142341042040566\\
121	0.00142341042040566\\
122	0.00142341042040566\\
123	0.00142341042040566\\
124	0.00142341042040566\\
125	0.00142341042040566\\
126	0.00142341042040566\\
127	0.00142341042040566\\
128	0.00142341042040566\\
129	0.00142341042040566\\
130	0.00142341042040566\\
131	0.00142341042040566\\
132	0.00142341042040566\\
133	0.00142341042040566\\
134	0.00142341042040566\\
135	0.00142341042040566\\
136	0.00142341042040566\\
137	0.00142341042040566\\
138	0.00142341042040566\\
139	0.00142341042040566\\
140	0.00142341042040566\\
141	0.00142341042040566\\
142	0.00142341042040566\\
143	0.00142341042040566\\
144	0.00142341042040566\\
145	0.00142341042040566\\
146	0.00142341042040566\\
147	0.00142341042040566\\
148	0.00142341042040566\\
149	0.00142341042040566\\
150	0.00142341042040566\\
151	0.00142341042040566\\
152	0.00142341042040566\\
153	0.00142341042040566\\
154	0.00142341042040566\\
155	0.00142341042040566\\
156	0.00142341042040566\\
157	0.00142341042040566\\
158	0.00142341042040566\\
159	0.00142341042040566\\
160	0.00142341042040566\\
161	0.00142341042040566\\
162	0.00142341042040566\\
163	0.00142341042040566\\
164	0.00142341042040566\\
165	0.00142341042040566\\
166	0.00142341042040566\\
167	0.00142341042040566\\
168	0.00142341042040566\\
169	0.00142341042040566\\
170	0.00142341042040566\\
171	0.00142341042040566\\
172	0.00142341042040566\\
173	0.00142341042040566\\
174	0.00142341042040566\\
175	0.00142341042040566\\
176	0.00142341042040566\\
177	0.00142341042040566\\
178	0.00142341042040566\\
179	0.00142341042040566\\
180	0.00142341042040566\\
181	0.00142341042040566\\
182	0.00142341042040566\\
183	0.00142341042040566\\
184	0.00142341042040566\\
185	0.00142341042040566\\
186	0.00142341042040566\\
187	0.00142341042040566\\
188	0.00142341042040566\\
189	0.00142341042040566\\
190	0.00142341042040566\\
191	0.00142341042040566\\
192	0.00142341042040566\\
193	0.00142341042040566\\
194	0.00142341042040566\\
195	0.00142341042040566\\
196	0.00142341042040566\\
197	0.00142341042040566\\
198	0.00142341042040566\\
199	0.00142341042040566\\
200	0.00142341042040566\\
201	0.00142341042040566\\
202	0.00142341042040566\\
203	0.00142341042040566\\
204	0.00142341042040566\\
205	0.00142341042040566\\
206	0.00142341042040566\\
207	0.00142341042040566\\
208	0.00142341042040566\\
209	0.00142341042040566\\
210	0.00142341042040566\\
211	0.00142341042040566\\
212	0.00142341042040566\\
213	0.00142341042040566\\
214	0.00142341042040566\\
215	0.00142341042040566\\
216	0.00142341042040566\\
217	0.00142341042040566\\
218	0.00142341042040566\\
219	0.00142341042040566\\
220	0.00142341042040566\\
221	0.00142341042040566\\
222	0.00142341042040566\\
223	0.00142341042040566\\
224	0.00142341042040566\\
225	0.00142341042040566\\
226	0.00142341042040566\\
227	0.00142341042040566\\
228	0.00142341042040566\\
229	0.00142341042040566\\
230	0.00142341042040566\\
231	0.00142341042040566\\
232	0.00142341042040566\\
233	0.00142341042040566\\
234	0.00142341042040566\\
235	0.00142341042040566\\
236	0.00142341042040566\\
237	0.00142341042040566\\
238	0.00142341042040566\\
239	0.00142341042040566\\
240	0.00142341042040566\\
241	0.00142341042040566\\
242	0.00142341042040566\\
243	0.00142341042040566\\
244	0.00142341042040566\\
245	0.00142341042040566\\
246	0.00142341042040566\\
247	0.00142341042040566\\
248	0.00142341042040566\\
249	0.00142341042040566\\
250	0.00142341042040566\\
251	0.00142341042040566\\
252	0.00142341042040566\\
253	0.00142341042040566\\
254	0.00142341042040566\\
255	0.00142341042040566\\
256	0.00142341042040566\\
257	0.00142341042040566\\
258	0.00142341042040566\\
259	0.00142341042040566\\
260	0.00142341042040566\\
261	0.00142341042040566\\
262	0.00142341042040566\\
263	0.00142341042040566\\
264	0.00142341042040566\\
265	0.00142341042040566\\
266	0.00142341042040566\\
267	0.00142341042040566\\
268	0.00142341042040566\\
269	0.00142341042040566\\
270	0.00142341042040566\\
271	0.00142341042040566\\
272	0.00142341042040566\\
273	0.00142341042040566\\
274	0.00142341042040566\\
275	0.00142341042040566\\
276	0.00142341042040566\\
277	0.00142341042040566\\
278	0.00142341042040566\\
279	0.00142341042040566\\
280	0.00142341042040566\\
281	0.00142341042040566\\
282	0.00142341042040566\\
283	0.00142341042040566\\
284	0.00142341042040566\\
285	0.00142341042040566\\
286	0.00142341042040566\\
287	0.00142341042040566\\
288	0.00142341042040566\\
289	0.00142341042040566\\
290	0.00142341042040566\\
291	0.00142341042040566\\
292	0.00142341042040566\\
293	0.00142341042040566\\
294	0.00142341042040566\\
295	0.00142341042040566\\
296	0.00142341042040566\\
297	0.00142341042040566\\
298	0.00142341042040566\\
299	0.00142341042040566\\
300	0.00142341042040566\\
301	0.00142341042040566\\
302	0.00142341042040566\\
303	0.00142341042040566\\
304	0.00142341042040566\\
305	0.00142341042040566\\
306	0.00142341042040566\\
307	0.00142341042040566\\
308	0.00142341042040566\\
309	0.00142341042040566\\
310	0.00142341042040566\\
311	0.00142341042040566\\
312	0.00142341042040566\\
313	0.00142341042040566\\
314	0.00142341042040566\\
315	0.00142341042040566\\
316	0.00142341042040566\\
317	0.00142341042040566\\
318	0.00142341042040566\\
319	0.00142341042040566\\
320	0.00142341042040566\\
321	0.00142341042040566\\
322	0.00142341042040566\\
323	0.00142341042040566\\
324	0.00142341042040566\\
325	0.00142341042040566\\
326	0.00142341042040566\\
327	0.00142341042040566\\
328	0.00142341042040566\\
329	0.00142341042040566\\
330	0.00142341042040566\\
331	0.00142341042040566\\
332	0.00142341042040566\\
333	0.00142341042040566\\
334	0.00142341042040566\\
335	0.00142341042040566\\
336	0.00142341042040566\\
337	0.00142341042040566\\
338	0.00142341042040566\\
339	0.00142341042040566\\
340	0.00142341042040566\\
341	0.00142341042040566\\
342	0.00142341042040566\\
343	0.00142341042040566\\
344	0.00142341042040566\\
345	0.00142341042040566\\
346	0.00142341042040566\\
347	0.00142341042040566\\
348	0.00142341042040566\\
349	0.00142341042040566\\
350	0.00142341042040566\\
351	0.00142341042040566\\
352	0.00142341042040566\\
353	0.00142341042040566\\
354	0.00142341042040566\\
355	0.00142341042040566\\
356	0.00142341042040566\\
357	0.00142341042040566\\
358	0.00142341042040566\\
359	0.00142341042040566\\
360	0.00142341042040566\\
361	0.00142341042040566\\
362	0.00142341042040566\\
363	0.00142341042040566\\
364	0.00142341042040566\\
365	0.00142341042040566\\
366	0.00142341042040566\\
367	0.00142341042040566\\
368	0.00142341042040566\\
369	0.00142341042040566\\
370	0.00142341042040566\\
371	0.00142341042040566\\
372	0.00142341042040566\\
373	0.00142341042040566\\
374	0.00142341042040566\\
375	0.00142341042040566\\
376	0.00142341042040566\\
377	0.00142341042040566\\
378	0.00142341042040566\\
379	0.00142341042040566\\
380	0.00142341042040566\\
381	0.00142341042040566\\
382	0.00142341042040566\\
383	0.00142341042040566\\
384	0.00142341042040566\\
385	0.00142341042040566\\
386	0.00142341042040566\\
387	0.00142341042040566\\
388	0.00142341042040566\\
389	0.00142341042040566\\
390	0.00142341042040566\\
391	0.00142341042040566\\
392	0.00142341042040566\\
393	0.00142341042040566\\
394	0.00142341042040566\\
395	0.00142341042040566\\
396	0.00142341042040566\\
397	0.00142341042040566\\
398	0.00142341042040566\\
399	0.00142341042040566\\
400	0.00142341042040566\\
401	0.00142341042040566\\
402	0.00142341042040566\\
403	0.00142341042040566\\
404	0.00142341042040566\\
405	0.00142341042040566\\
406	0.00142341042040566\\
407	0.00142341042040566\\
408	0.00142341042040566\\
409	0.00142341042040566\\
410	0.00142341042040566\\
411	0.00142341042040566\\
412	0.00142341042040566\\
413	0.00142341042040566\\
414	0.00142341042040566\\
415	0.00142341042040566\\
416	0.00142341042040566\\
417	0.00142341042040566\\
418	0.00142341042040566\\
419	0.00142341042040566\\
420	0.00142341042040566\\
421	0.00142341042040566\\
422	0.00142341042040566\\
423	0.00142341042040566\\
424	0.00142341042040566\\
425	0.00142341042040566\\
426	0.00142341042040566\\
427	0.00142341042040566\\
428	0.00142341042040566\\
429	0.00142341042040566\\
430	0.00142341042040566\\
431	0.00142341042040566\\
432	0.00142341042040566\\
433	0.00142341042040566\\
434	0.00142341042040566\\
435	0.00142341042040566\\
436	0.00142341042040566\\
437	0.00142341042040566\\
438	0.00142341042040566\\
439	0.00142341042040566\\
440	0.00142341042040566\\
441	0.00142341042040566\\
442	0.00142341042040566\\
443	0.00142341042040566\\
444	0.00142341042040566\\
445	0.00142341042040566\\
446	0.00142341042040566\\
447	0.00142341042040566\\
448	0.00142341042040566\\
449	0.00142341042040566\\
450	0.00142341042040566\\
451	0.00142341042040566\\
452	0.00142341042040566\\
453	0.00142341042040566\\
454	0.00142341042040566\\
455	0.00142341042040566\\
456	0.00142341042040566\\
457	0.00142341042040566\\
458	0.00142341042040566\\
459	0.00142341042040566\\
460	0.00142341042040566\\
461	0.00142341042040566\\
462	0.00142341042040566\\
463	0.00142341042040566\\
464	0.00142341042040566\\
465	0.00142341042040566\\
466	0.00142341042040566\\
467	0.00142341042040566\\
468	0.00142341042040566\\
469	0.00142341042040566\\
470	0.00142341042040566\\
471	0.00142341042040566\\
472	0.00142341042040566\\
473	0.00142341042040566\\
474	0.00142341042040566\\
475	0.00142341042040566\\
476	0.00142341042040566\\
477	0.00142341042040566\\
478	0.00142341042040566\\
479	0.00142341042040566\\
480	0.00142341042040566\\
481	0.00142341042040566\\
482	0.00142341042040566\\
483	0.00142341042040566\\
484	0.00142341042040566\\
485	0.00142341042040566\\
486	0.00142341042040566\\
487	0.00142341042040566\\
488	0.00142341042040566\\
489	0.00142341042040566\\
490	0.00142341042040566\\
491	0.00142341042040566\\
492	0.00142341042040566\\
493	0.00142341042040566\\
494	0.00142341042040566\\
495	0.00142341042040566\\
496	0.00142341042040566\\
497	0.00142341042040566\\
498	0.00142341042040566\\
499	0.00142341042040566\\
500	0.00142341042040566\\
501	0.00142341042040566\\
502	0.00142341042040566\\
503	0.00142341042040566\\
504	0.00142341042040566\\
505	0.00142341042040566\\
506	0.00142341042040566\\
507	0.00142341042040566\\
508	0.00142341042040566\\
509	0.00142341042040566\\
510	0.00142341042040566\\
511	0.00142341042040566\\
512	0.00142341042040566\\
513	0.00142341042040566\\
514	0.00142341042040566\\
515	0.00142341042040566\\
516	0.00142341042040566\\
517	0.00142341042040566\\
518	0.00142341042040566\\
519	0.00142341042040566\\
520	0.00142341042040566\\
521	0.00142341042040566\\
522	0.00142341042040566\\
523	0.00142341042040566\\
524	0.00142341042040566\\
525	0.00142341042040566\\
526	0.00142341042040566\\
527	0.00142341042040566\\
528	0.00142341042040566\\
529	0.00142341042040566\\
530	0.00142341042040566\\
531	0.00142341042040566\\
532	0.00142341042040566\\
533	0.00142341042040566\\
534	0.00142341042040566\\
535	0.00142341042040566\\
536	0.00142341042040566\\
537	0.00142341042040566\\
538	0.00142341042040566\\
539	0.00142341042040566\\
540	0.00142341042040566\\
541	0.00142341042040566\\
542	0.00142341042040566\\
543	0.00142341042040566\\
544	0.00142341042040566\\
545	0.00142341042040566\\
546	0.00142341042040566\\
547	0.00142341042040566\\
548	0.00142341042040566\\
549	0.00142341042040566\\
550	0.00142341042040566\\
551	0.00142341042040566\\
552	0.00142341042040566\\
553	0.00142341042040566\\
554	0.00142341042040566\\
555	0.00142341042040566\\
556	0.00142341042040566\\
557	0.00142341042040566\\
558	0.00142341042040566\\
559	0.00142341042040566\\
560	0.00142341042040566\\
561	0.00142341042040566\\
562	0.00142341042040566\\
563	0.00142341042040566\\
564	0.00142341042040566\\
565	0.00141808680651778\\
566	0.00132573838914067\\
567	0.00121575937955735\\
568	0.00110234926539761\\
569	0.000991487125274636\\
570	0.000881523074320698\\
571	0.000772562041893994\\
572	0.000667166343829018\\
573	0.000574706660558508\\
574	0.000427188374771164\\
575	0.000287780647258003\\
576	0.000156402139129492\\
577	3.29294685952495e-05\\
578	0\\
579	0\\
580	0\\
581	0\\
582	0\\
583	0\\
584	0\\
585	0\\
586	0\\
587	0\\
588	0\\
589	0\\
590	0\\
591	0\\
592	5.56832960656924e-05\\
593	0.000164162311403634\\
594	0.00028999157753712\\
595	0.000432641173258269\\
596	0.000588326666482334\\
597	0.000749148955320047\\
598	0.00347642786857269\\
599	0\\
600	0\\
};
\addplot [color=mycolor14,solid,forget plot]
  table[row sep=crcr]{%
1	0\\
2	0\\
3	0\\
4	0\\
5	0\\
6	0\\
7	0\\
8	0\\
9	0\\
10	0\\
11	0\\
12	0\\
13	0\\
14	0\\
15	0\\
16	0\\
17	0\\
18	0\\
19	0\\
20	0\\
21	0\\
22	0\\
23	0\\
24	0\\
25	0\\
26	0\\
27	0\\
28	0\\
29	0\\
30	0\\
31	0\\
32	0\\
33	0\\
34	0\\
35	0\\
36	0\\
37	0\\
38	0\\
39	0\\
40	0\\
41	0\\
42	0\\
43	0\\
44	0\\
45	0\\
46	0\\
47	0\\
48	0\\
49	0\\
50	0\\
51	0\\
52	0\\
53	0\\
54	0\\
55	0\\
56	0\\
57	0\\
58	0\\
59	0\\
60	0\\
61	0\\
62	0\\
63	0\\
64	0\\
65	0\\
66	0\\
67	0\\
68	0\\
69	0\\
70	0\\
71	0\\
72	0\\
73	0\\
74	0\\
75	0\\
76	0\\
77	0\\
78	0\\
79	0\\
80	0\\
81	0\\
82	0\\
83	0\\
84	0\\
85	0\\
86	0\\
87	0\\
88	0\\
89	0\\
90	0\\
91	0\\
92	0\\
93	0\\
94	0\\
95	0\\
96	0\\
97	0\\
98	0\\
99	0\\
100	0\\
101	0\\
102	0\\
103	0\\
104	0\\
105	0\\
106	0\\
107	0\\
108	0\\
109	0\\
110	0\\
111	0\\
112	0\\
113	0\\
114	0\\
115	0\\
116	0\\
117	0\\
118	0\\
119	0\\
120	0\\
121	0\\
122	0\\
123	0\\
124	0\\
125	0\\
126	0\\
127	0\\
128	0\\
129	0\\
130	0\\
131	0\\
132	0\\
133	0\\
134	0\\
135	0\\
136	0\\
137	0\\
138	0\\
139	0\\
140	0\\
141	0\\
142	0\\
143	0\\
144	0\\
145	0\\
146	0\\
147	0\\
148	0\\
149	0\\
150	0\\
151	0\\
152	0\\
153	0\\
154	0\\
155	0\\
156	0\\
157	0\\
158	0\\
159	0\\
160	0\\
161	0\\
162	0\\
163	0\\
164	0\\
165	0\\
166	0\\
167	0\\
168	0\\
169	0\\
170	0\\
171	0\\
172	0\\
173	0\\
174	0\\
175	0\\
176	0\\
177	0\\
178	0\\
179	0\\
180	0\\
181	0\\
182	0\\
183	0\\
184	0\\
185	0\\
186	0\\
187	0\\
188	0\\
189	0\\
190	0\\
191	0\\
192	0\\
193	0\\
194	0\\
195	0\\
196	0\\
197	0\\
198	0\\
199	0\\
200	0\\
201	0\\
202	0\\
203	0\\
204	0\\
205	0\\
206	0\\
207	0\\
208	0\\
209	0\\
210	0\\
211	0\\
212	0\\
213	0\\
214	0\\
215	0\\
216	0\\
217	0\\
218	0\\
219	0\\
220	0\\
221	0\\
222	0\\
223	0\\
224	0\\
225	0\\
226	0\\
227	0\\
228	0\\
229	0\\
230	0\\
231	0\\
232	0\\
233	0\\
234	0\\
235	0\\
236	0\\
237	0\\
238	0\\
239	0\\
240	0\\
241	0\\
242	0\\
243	0\\
244	0\\
245	0\\
246	0\\
247	0\\
248	0\\
249	0\\
250	0\\
251	0\\
252	0\\
253	0\\
254	0\\
255	0\\
256	0\\
257	0\\
258	0\\
259	0\\
260	0\\
261	0\\
262	0\\
263	0\\
264	0\\
265	0\\
266	0\\
267	0\\
268	0\\
269	0\\
270	0\\
271	0\\
272	0\\
273	0\\
274	0\\
275	0\\
276	0\\
277	0\\
278	0\\
279	0\\
280	0\\
281	0\\
282	0\\
283	0\\
284	0\\
285	0\\
286	0\\
287	0\\
288	0\\
289	0\\
290	0\\
291	0\\
292	0\\
293	0\\
294	0\\
295	0\\
296	0\\
297	0\\
298	0\\
299	0\\
300	0\\
301	0\\
302	0\\
303	0\\
304	0\\
305	0\\
306	0\\
307	0\\
308	0\\
309	0\\
310	0\\
311	0\\
312	0\\
313	0\\
314	0\\
315	0\\
316	0\\
317	0\\
318	0\\
319	0\\
320	0\\
321	0\\
322	0\\
323	0\\
324	0\\
325	0\\
326	0\\
327	0\\
328	0\\
329	0\\
330	0\\
331	0\\
332	0\\
333	0\\
334	0\\
335	0\\
336	0\\
337	0\\
338	0\\
339	0\\
340	0\\
341	0\\
342	0\\
343	0\\
344	0\\
345	0\\
346	0\\
347	0\\
348	0\\
349	0\\
350	0\\
351	0\\
352	0\\
353	0\\
354	0\\
355	0\\
356	0\\
357	0\\
358	0\\
359	0\\
360	0\\
361	0\\
362	0\\
363	0\\
364	0\\
365	0\\
366	0\\
367	0\\
368	0\\
369	0\\
370	0\\
371	0\\
372	0\\
373	0\\
374	0\\
375	0\\
376	0\\
377	0\\
378	0\\
379	0\\
380	0\\
381	0\\
382	0\\
383	0\\
384	0\\
385	0\\
386	0\\
387	0\\
388	0\\
389	0\\
390	0\\
391	0\\
392	0\\
393	0\\
394	0\\
395	0\\
396	0\\
397	0\\
398	0\\
399	0\\
400	0\\
401	0\\
402	0\\
403	0\\
404	0\\
405	0\\
406	0\\
407	0\\
408	0\\
409	0\\
410	0\\
411	0\\
412	0\\
413	0\\
414	0\\
415	0\\
416	0\\
417	0\\
418	0\\
419	0\\
420	0\\
421	0\\
422	0\\
423	0\\
424	0\\
425	0\\
426	0\\
427	0\\
428	0\\
429	0\\
430	0\\
431	0\\
432	0\\
433	0\\
434	0\\
435	0\\
436	0\\
437	0\\
438	0\\
439	0\\
440	0\\
441	0\\
442	0\\
443	0\\
444	0\\
445	0\\
446	0\\
447	0\\
448	0\\
449	0\\
450	0\\
451	0\\
452	0\\
453	0\\
454	0\\
455	0\\
456	0\\
457	0\\
458	0\\
459	0\\
460	0\\
461	0\\
462	0\\
463	0\\
464	0\\
465	0\\
466	0\\
467	0\\
468	0\\
469	0\\
470	0\\
471	0\\
472	0\\
473	0\\
474	0\\
475	0\\
476	0\\
477	0\\
478	0\\
479	0\\
480	0\\
481	0\\
482	0\\
483	0\\
484	0\\
485	0\\
486	0\\
487	0\\
488	0\\
489	0\\
490	0\\
491	0\\
492	0\\
493	0\\
494	0\\
495	0\\
496	0\\
497	0\\
498	0\\
499	0\\
500	0\\
501	0\\
502	0\\
503	0\\
504	0\\
505	0\\
506	0\\
507	0\\
508	0\\
509	0\\
510	0\\
511	0\\
512	0\\
513	0\\
514	0\\
515	0\\
516	0\\
517	0\\
518	0\\
519	0\\
520	0\\
521	0\\
522	0\\
523	0\\
524	0\\
525	0\\
526	0\\
527	0\\
528	0\\
529	0\\
530	0\\
531	0\\
532	0\\
533	0\\
534	0\\
535	0\\
536	0\\
537	0\\
538	0\\
539	0\\
540	0\\
541	0\\
542	0\\
543	0\\
544	0\\
545	0\\
546	0\\
547	0\\
548	0\\
549	0\\
550	0\\
551	0\\
552	0\\
553	0\\
554	0\\
555	0\\
556	0\\
557	0\\
558	0\\
559	0\\
560	0\\
561	0\\
562	0\\
563	0\\
564	0\\
565	0\\
566	0\\
567	0\\
568	0\\
569	0\\
570	0\\
571	0\\
572	0\\
573	0\\
574	0\\
575	0\\
576	0\\
577	0\\
578	0\\
579	0\\
580	0\\
581	0\\
582	0\\
583	0\\
584	0\\
585	0.000134515837093855\\
586	0.000281838017575936\\
587	0.000435050286072654\\
588	0.000594542785312444\\
589	0.0007604561023499\\
590	0.000933270892818283\\
591	0.00111362236991578\\
592	0.00130217091110698\\
593	0.00149974464812233\\
594	0.00170715331224199\\
595	0.00192583625726673\\
596	0.00215816651678185\\
597	0.00340423165914315\\
598	0.00644286460810295\\
599	0\\
600	0\\
};
\addplot [color=mycolor15,solid,forget plot]
  table[row sep=crcr]{%
1	0\\
2	0\\
3	0\\
4	0\\
5	0\\
6	0\\
7	0\\
8	0\\
9	0\\
10	0\\
11	0\\
12	0\\
13	0\\
14	0\\
15	0\\
16	0\\
17	0\\
18	0\\
19	0\\
20	0\\
21	0\\
22	0\\
23	0\\
24	0\\
25	0\\
26	0\\
27	0\\
28	0\\
29	0\\
30	0\\
31	0\\
32	0\\
33	0\\
34	0\\
35	0\\
36	0\\
37	0\\
38	0\\
39	0\\
40	0\\
41	0\\
42	0\\
43	0\\
44	0\\
45	0\\
46	0\\
47	0\\
48	0\\
49	0\\
50	0\\
51	0\\
52	0\\
53	0\\
54	0\\
55	0\\
56	0\\
57	0\\
58	0\\
59	0\\
60	0\\
61	0\\
62	0\\
63	0\\
64	0\\
65	0\\
66	0\\
67	0\\
68	0\\
69	0\\
70	0\\
71	0\\
72	0\\
73	0\\
74	0\\
75	0\\
76	0\\
77	0\\
78	0\\
79	0\\
80	0\\
81	0\\
82	0\\
83	0\\
84	0\\
85	0\\
86	0\\
87	0\\
88	0\\
89	0\\
90	0\\
91	0\\
92	0\\
93	0\\
94	0\\
95	0\\
96	0\\
97	0\\
98	0\\
99	0\\
100	0\\
101	0\\
102	0\\
103	0\\
104	0\\
105	0\\
106	0\\
107	0\\
108	0\\
109	0\\
110	0\\
111	0\\
112	0\\
113	0\\
114	0\\
115	0\\
116	0\\
117	0\\
118	0\\
119	0\\
120	0\\
121	0\\
122	0\\
123	0\\
124	0\\
125	0\\
126	0\\
127	0\\
128	0\\
129	0\\
130	0\\
131	0\\
132	0\\
133	0\\
134	0\\
135	0\\
136	0\\
137	0\\
138	0\\
139	0\\
140	0\\
141	0\\
142	0\\
143	0\\
144	0\\
145	0\\
146	0\\
147	0\\
148	0\\
149	0\\
150	0\\
151	0\\
152	0\\
153	0\\
154	0\\
155	0\\
156	0\\
157	0\\
158	0\\
159	0\\
160	0\\
161	0\\
162	0\\
163	0\\
164	0\\
165	0\\
166	0\\
167	0\\
168	0\\
169	0\\
170	0\\
171	0\\
172	0\\
173	0\\
174	0\\
175	0\\
176	0\\
177	0\\
178	0\\
179	0\\
180	0\\
181	0\\
182	0\\
183	0\\
184	0\\
185	0\\
186	0\\
187	0\\
188	0\\
189	0\\
190	0\\
191	0\\
192	0\\
193	0\\
194	0\\
195	0\\
196	0\\
197	0\\
198	0\\
199	0\\
200	0\\
201	0\\
202	0\\
203	0\\
204	0\\
205	0\\
206	0\\
207	0\\
208	0\\
209	0\\
210	0\\
211	0\\
212	0\\
213	0\\
214	0\\
215	0\\
216	0\\
217	0\\
218	0\\
219	0\\
220	0\\
221	0\\
222	0\\
223	0\\
224	0\\
225	0\\
226	0\\
227	0\\
228	0\\
229	0\\
230	0\\
231	0\\
232	0\\
233	0\\
234	0\\
235	0\\
236	0\\
237	0\\
238	0\\
239	0\\
240	0\\
241	0\\
242	0\\
243	0\\
244	0\\
245	0\\
246	0\\
247	0\\
248	0\\
249	0\\
250	0\\
251	0\\
252	0\\
253	0\\
254	0\\
255	0\\
256	0\\
257	0\\
258	0\\
259	0\\
260	0\\
261	0\\
262	0\\
263	0\\
264	0\\
265	0\\
266	0\\
267	0\\
268	0\\
269	0\\
270	0\\
271	0\\
272	0\\
273	0\\
274	0\\
275	0\\
276	0\\
277	0\\
278	0\\
279	0\\
280	0\\
281	0\\
282	0\\
283	0\\
284	0\\
285	0\\
286	0\\
287	0\\
288	0\\
289	0\\
290	0\\
291	0\\
292	0\\
293	0\\
294	0\\
295	0\\
296	0\\
297	0\\
298	0\\
299	0\\
300	0\\
301	0\\
302	0\\
303	0\\
304	0\\
305	0\\
306	0\\
307	0\\
308	0\\
309	0\\
310	0\\
311	0\\
312	0\\
313	0\\
314	0\\
315	0\\
316	0\\
317	0\\
318	0\\
319	0\\
320	0\\
321	0\\
322	0\\
323	0\\
324	0\\
325	0\\
326	0\\
327	0\\
328	0\\
329	0\\
330	0\\
331	0\\
332	0\\
333	0\\
334	0\\
335	0\\
336	0\\
337	0\\
338	0\\
339	0\\
340	0\\
341	0\\
342	0\\
343	0\\
344	0\\
345	0\\
346	0\\
347	0\\
348	0\\
349	0\\
350	0\\
351	0\\
352	0\\
353	0\\
354	0\\
355	0\\
356	0\\
357	0\\
358	0\\
359	0\\
360	0\\
361	0\\
362	0\\
363	0\\
364	0\\
365	0\\
366	0\\
367	0\\
368	0\\
369	0\\
370	0\\
371	0\\
372	0\\
373	0\\
374	0\\
375	0\\
376	0\\
377	0\\
378	0\\
379	0\\
380	0\\
381	0\\
382	0\\
383	0\\
384	0\\
385	0\\
386	0\\
387	0\\
388	0\\
389	0\\
390	0\\
391	0\\
392	0\\
393	0\\
394	0\\
395	0\\
396	0\\
397	0\\
398	0\\
399	0\\
400	0\\
401	0\\
402	0\\
403	0\\
404	0\\
405	0\\
406	0\\
407	0\\
408	0\\
409	0\\
410	0\\
411	0\\
412	0\\
413	0\\
414	0\\
415	0\\
416	0\\
417	0\\
418	0\\
419	0\\
420	0\\
421	0\\
422	0\\
423	0\\
424	0\\
425	0\\
426	0\\
427	0\\
428	0\\
429	0\\
430	0\\
431	0\\
432	0\\
433	0\\
434	0\\
435	0\\
436	0\\
437	0\\
438	0\\
439	0\\
440	0\\
441	0\\
442	0\\
443	0\\
444	0\\
445	0\\
446	0\\
447	0\\
448	0\\
449	0\\
450	0\\
451	0\\
452	0\\
453	0\\
454	0\\
455	0\\
456	0\\
457	0\\
458	0\\
459	0\\
460	0\\
461	0\\
462	0\\
463	0\\
464	0\\
465	0\\
466	0\\
467	0\\
468	0\\
469	0\\
470	0\\
471	0\\
472	0\\
473	0\\
474	0\\
475	0\\
476	0\\
477	0\\
478	0\\
479	0\\
480	0\\
481	0\\
482	0\\
483	0\\
484	0\\
485	0\\
486	0\\
487	0\\
488	0\\
489	0\\
490	0\\
491	0\\
492	0\\
493	0\\
494	0\\
495	0\\
496	0\\
497	0\\
498	0\\
499	0\\
500	0\\
501	0\\
502	0\\
503	0\\
504	0\\
505	0\\
506	0\\
507	0\\
508	0\\
509	0\\
510	0\\
511	0\\
512	0\\
513	0\\
514	0\\
515	0\\
516	0\\
517	0\\
518	0\\
519	0\\
520	0\\
521	0\\
522	0\\
523	0\\
524	0\\
525	0\\
526	0\\
527	0\\
528	0\\
529	0\\
530	0\\
531	0\\
532	0\\
533	0\\
534	0\\
535	0\\
536	0\\
537	0\\
538	0\\
539	0\\
540	0\\
541	0\\
542	0\\
543	0\\
544	0\\
545	0\\
546	0\\
547	0\\
548	0\\
549	0\\
550	0\\
551	0\\
552	0\\
553	0\\
554	0\\
555	0\\
556	0\\
557	0\\
558	0\\
559	0\\
560	0\\
561	0\\
562	0\\
563	0\\
564	0\\
565	0\\
566	0\\
567	0\\
568	0\\
569	0\\
570	0\\
571	0\\
572	0\\
573	2.84550322669103e-05\\
574	0.000136105490885286\\
575	0.000246360904768022\\
576	0.00035920980371322\\
577	0.000474576836408107\\
578	0.000592251873280648\\
579	0.000712460866895543\\
580	0.000835098336304564\\
581	0.000931987463067577\\
582	0.00102435477630977\\
583	0.00111851701928916\\
584	0.00121497734033712\\
585	0.00131373271689993\\
586	0.00141477870147017\\
587	0.00151807891510166\\
588	0.00162355054108618\\
589	0.00173106951191172\\
590	0.00184045717402978\\
591	0.00195146461619112\\
592	0.00206375309347527\\
593	0.00217686668353044\\
594	0.00229019015194498\\
595	0.00264340599857734\\
596	0.00335667832726079\\
597	0.00469756192662931\\
598	0.00644286460810295\\
599	0\\
600	0\\
};
\addplot [color=mycolor16,solid,forget plot]
  table[row sep=crcr]{%
1	0\\
2	0\\
3	0\\
4	0\\
5	0\\
6	0\\
7	0\\
8	0\\
9	0\\
10	0\\
11	0\\
12	0\\
13	0\\
14	0\\
15	0\\
16	0\\
17	0\\
18	0\\
19	0\\
20	0\\
21	0\\
22	0\\
23	0\\
24	0\\
25	0\\
26	0\\
27	0\\
28	0\\
29	0\\
30	0\\
31	0\\
32	0\\
33	0\\
34	0\\
35	0\\
36	0\\
37	0\\
38	0\\
39	0\\
40	0\\
41	0\\
42	0\\
43	0\\
44	0\\
45	0\\
46	0\\
47	0\\
48	0\\
49	0\\
50	0\\
51	0\\
52	0\\
53	0\\
54	0\\
55	0\\
56	0\\
57	0\\
58	0\\
59	0\\
60	0\\
61	0\\
62	0\\
63	0\\
64	0\\
65	0\\
66	0\\
67	0\\
68	0\\
69	0\\
70	0\\
71	0\\
72	0\\
73	0\\
74	0\\
75	0\\
76	0\\
77	0\\
78	0\\
79	0\\
80	0\\
81	0\\
82	0\\
83	0\\
84	0\\
85	0\\
86	0\\
87	0\\
88	0\\
89	0\\
90	0\\
91	0\\
92	0\\
93	0\\
94	0\\
95	0\\
96	0\\
97	0\\
98	0\\
99	0\\
100	0\\
101	0\\
102	0\\
103	0\\
104	0\\
105	0\\
106	0\\
107	0\\
108	0\\
109	0\\
110	0\\
111	0\\
112	0\\
113	0\\
114	0\\
115	0\\
116	0\\
117	0\\
118	0\\
119	0\\
120	0\\
121	0\\
122	0\\
123	0\\
124	0\\
125	0\\
126	0\\
127	0\\
128	0\\
129	0\\
130	0\\
131	0\\
132	0\\
133	0\\
134	0\\
135	0\\
136	0\\
137	0\\
138	0\\
139	0\\
140	0\\
141	0\\
142	0\\
143	0\\
144	0\\
145	0\\
146	0\\
147	0\\
148	0\\
149	0\\
150	0\\
151	0\\
152	0\\
153	0\\
154	0\\
155	0\\
156	0\\
157	0\\
158	0\\
159	0\\
160	0\\
161	0\\
162	0\\
163	0\\
164	0\\
165	0\\
166	0\\
167	0\\
168	0\\
169	0\\
170	0\\
171	0\\
172	0\\
173	0\\
174	0\\
175	0\\
176	0\\
177	0\\
178	0\\
179	0\\
180	0\\
181	0\\
182	0\\
183	0\\
184	0\\
185	0\\
186	0\\
187	0\\
188	0\\
189	0\\
190	0\\
191	0\\
192	0\\
193	0\\
194	0\\
195	0\\
196	0\\
197	0\\
198	0\\
199	0\\
200	0\\
201	0\\
202	0\\
203	0\\
204	0\\
205	0\\
206	0\\
207	0\\
208	0\\
209	0\\
210	0\\
211	0\\
212	0\\
213	0\\
214	0\\
215	0\\
216	0\\
217	0\\
218	0\\
219	0\\
220	0\\
221	0\\
222	0\\
223	0\\
224	0\\
225	0\\
226	0\\
227	0\\
228	0\\
229	0\\
230	0\\
231	0\\
232	0\\
233	0\\
234	0\\
235	0\\
236	0\\
237	0\\
238	0\\
239	0\\
240	0\\
241	0\\
242	0\\
243	0\\
244	0\\
245	0\\
246	0\\
247	0\\
248	0\\
249	0\\
250	0\\
251	0\\
252	0\\
253	0\\
254	0\\
255	0\\
256	0\\
257	0\\
258	0\\
259	0\\
260	0\\
261	0\\
262	0\\
263	0\\
264	0\\
265	0\\
266	0\\
267	0\\
268	0\\
269	0\\
270	0\\
271	0\\
272	0\\
273	0\\
274	0\\
275	0\\
276	0\\
277	0\\
278	0\\
279	0\\
280	0\\
281	0\\
282	0\\
283	0\\
284	0\\
285	0\\
286	0\\
287	0\\
288	0\\
289	0\\
290	0\\
291	0\\
292	0\\
293	0\\
294	0\\
295	0\\
296	0\\
297	0\\
298	0\\
299	0\\
300	0\\
301	0\\
302	0\\
303	0\\
304	0\\
305	0\\
306	0\\
307	0\\
308	0\\
309	0\\
310	0\\
311	0\\
312	0\\
313	0\\
314	0\\
315	0\\
316	0\\
317	0\\
318	0\\
319	0\\
320	0\\
321	0\\
322	0\\
323	0\\
324	0\\
325	0\\
326	0\\
327	0\\
328	0\\
329	0\\
330	0\\
331	0\\
332	0\\
333	0\\
334	0\\
335	0\\
336	0\\
337	0\\
338	0\\
339	0\\
340	0\\
341	0\\
342	0\\
343	0\\
344	0\\
345	0\\
346	0\\
347	0\\
348	0\\
349	0\\
350	0\\
351	0\\
352	0\\
353	0\\
354	0\\
355	0\\
356	0\\
357	0\\
358	0\\
359	0\\
360	0\\
361	0\\
362	0\\
363	0\\
364	0\\
365	0\\
366	0\\
367	0\\
368	0\\
369	0\\
370	0\\
371	0\\
372	0\\
373	0\\
374	0\\
375	0\\
376	0\\
377	0\\
378	0\\
379	0\\
380	0\\
381	0\\
382	0\\
383	0\\
384	0\\
385	0\\
386	0\\
387	0\\
388	0\\
389	0\\
390	0\\
391	0\\
392	0\\
393	0\\
394	0\\
395	0\\
396	0\\
397	0\\
398	0\\
399	0\\
400	0\\
401	0\\
402	0\\
403	0\\
404	0\\
405	0\\
406	0\\
407	0\\
408	0\\
409	0\\
410	0\\
411	0\\
412	0\\
413	0\\
414	0\\
415	0\\
416	0\\
417	0\\
418	0\\
419	0\\
420	0\\
421	0\\
422	0\\
423	0\\
424	0\\
425	0\\
426	0\\
427	0\\
428	0\\
429	0\\
430	0\\
431	0\\
432	0\\
433	0\\
434	0\\
435	0\\
436	0\\
437	0\\
438	0\\
439	0\\
440	0\\
441	0\\
442	0\\
443	0\\
444	0\\
445	0\\
446	0\\
447	0\\
448	0\\
449	0\\
450	0\\
451	0\\
452	0\\
453	0\\
454	0\\
455	0\\
456	0\\
457	0\\
458	0\\
459	0\\
460	0\\
461	0\\
462	0\\
463	0\\
464	0\\
465	0\\
466	0\\
467	0\\
468	0\\
469	0\\
470	0\\
471	0\\
472	0\\
473	0\\
474	0\\
475	0\\
476	0\\
477	0\\
478	0\\
479	0\\
480	0\\
481	0\\
482	0\\
483	0\\
484	0\\
485	0\\
486	0\\
487	0\\
488	0\\
489	0\\
490	0\\
491	0\\
492	0\\
493	0\\
494	0\\
495	0\\
496	0\\
497	0\\
498	0\\
499	0\\
500	0\\
501	0\\
502	0\\
503	0\\
504	0\\
505	0\\
506	0\\
507	0\\
508	0\\
509	0\\
510	0\\
511	0\\
512	0\\
513	0\\
514	0\\
515	0\\
516	0\\
517	0\\
518	0\\
519	0\\
520	0\\
521	0\\
522	0\\
523	0\\
524	0\\
525	0\\
526	0\\
527	0\\
528	0\\
529	0\\
530	0\\
531	0\\
532	0\\
533	0\\
534	0\\
535	0\\
536	0\\
537	0\\
538	0\\
539	0\\
540	0\\
541	0\\
542	0\\
543	0\\
544	0\\
545	0\\
546	0\\
547	0\\
548	0\\
549	0\\
550	0\\
551	0\\
552	0\\
553	0\\
554	0\\
555	0\\
556	0\\
557	0\\
558	0\\
559	0\\
560	0\\
561	0\\
562	0\\
563	0\\
564	6.11241591688388e-05\\
565	0.000153833609333673\\
566	0.000247621371611974\\
567	0.000342275882838338\\
568	0.000426454386023444\\
569	0.000490146502349677\\
570	0.000554714051789001\\
571	0.000620092626673308\\
572	0.000686241977351121\\
573	0.000753338581932168\\
574	0.000821311602791528\\
575	0.000890073230408475\\
576	0.000959515814844145\\
577	0.0010295081722191\\
578	0.0010998907600857\\
579	0.00117046937600367\\
580	0.00124101035188772\\
581	0.00131279084750798\\
582	0.00138613243267675\\
583	0.00146105292021269\\
584	0.00153756685550637\\
585	0.00161568839009441\\
586	0.00169543074241313\\
587	0.00177681010814605\\
588	0.00185987394737956\\
589	0.00194472985818468\\
590	0.00203176867243585\\
591	0.00218539535278842\\
592	0.00246148836585481\\
593	0.00274934460856801\\
594	0.00309920221510837\\
595	0.0035236770965596\\
596	0.00425769487060698\\
597	0.00503983077166121\\
598	0.00644286460810295\\
599	0\\
600	0\\
};
\addplot [color=mycolor17,solid,forget plot]
  table[row sep=crcr]{%
1	0\\
2	0\\
3	0\\
4	0\\
5	0\\
6	0\\
7	0\\
8	0\\
9	0\\
10	0\\
11	0\\
12	0\\
13	0\\
14	0\\
15	0\\
16	0\\
17	0\\
18	0\\
19	0\\
20	0\\
21	0\\
22	0\\
23	0\\
24	0\\
25	0\\
26	0\\
27	0\\
28	0\\
29	0\\
30	0\\
31	0\\
32	0\\
33	0\\
34	0\\
35	0\\
36	0\\
37	0\\
38	0\\
39	0\\
40	0\\
41	0\\
42	0\\
43	0\\
44	0\\
45	0\\
46	0\\
47	0\\
48	0\\
49	0\\
50	0\\
51	0\\
52	0\\
53	0\\
54	0\\
55	0\\
56	0\\
57	0\\
58	0\\
59	0\\
60	0\\
61	0\\
62	0\\
63	0\\
64	0\\
65	0\\
66	0\\
67	0\\
68	0\\
69	0\\
70	0\\
71	0\\
72	0\\
73	0\\
74	0\\
75	0\\
76	0\\
77	0\\
78	0\\
79	0\\
80	0\\
81	0\\
82	0\\
83	0\\
84	0\\
85	0\\
86	0\\
87	0\\
88	0\\
89	0\\
90	0\\
91	0\\
92	0\\
93	0\\
94	0\\
95	0\\
96	0\\
97	0\\
98	0\\
99	0\\
100	0\\
101	0\\
102	0\\
103	0\\
104	0\\
105	0\\
106	0\\
107	0\\
108	0\\
109	0\\
110	0\\
111	0\\
112	0\\
113	0\\
114	0\\
115	0\\
116	0\\
117	0\\
118	0\\
119	0\\
120	0\\
121	0\\
122	0\\
123	0\\
124	0\\
125	0\\
126	0\\
127	0\\
128	0\\
129	0\\
130	0\\
131	0\\
132	0\\
133	0\\
134	0\\
135	0\\
136	0\\
137	0\\
138	0\\
139	0\\
140	0\\
141	0\\
142	0\\
143	0\\
144	0\\
145	0\\
146	0\\
147	0\\
148	0\\
149	0\\
150	0\\
151	0\\
152	0\\
153	0\\
154	0\\
155	0\\
156	0\\
157	0\\
158	0\\
159	0\\
160	0\\
161	0\\
162	0\\
163	0\\
164	0\\
165	0\\
166	0\\
167	0\\
168	0\\
169	0\\
170	0\\
171	0\\
172	0\\
173	0\\
174	0\\
175	0\\
176	0\\
177	0\\
178	0\\
179	0\\
180	0\\
181	0\\
182	0\\
183	0\\
184	0\\
185	0\\
186	0\\
187	0\\
188	0\\
189	0\\
190	0\\
191	0\\
192	0\\
193	0\\
194	0\\
195	0\\
196	0\\
197	0\\
198	0\\
199	0\\
200	0\\
201	0\\
202	0\\
203	0\\
204	0\\
205	0\\
206	0\\
207	0\\
208	0\\
209	0\\
210	0\\
211	0\\
212	0\\
213	0\\
214	0\\
215	0\\
216	0\\
217	0\\
218	0\\
219	0\\
220	0\\
221	0\\
222	0\\
223	0\\
224	0\\
225	0\\
226	0\\
227	0\\
228	0\\
229	0\\
230	0\\
231	0\\
232	0\\
233	0\\
234	0\\
235	0\\
236	0\\
237	0\\
238	0\\
239	0\\
240	0\\
241	0\\
242	0\\
243	0\\
244	0\\
245	0\\
246	0\\
247	0\\
248	0\\
249	0\\
250	0\\
251	0\\
252	0\\
253	0\\
254	0\\
255	0\\
256	0\\
257	0\\
258	0\\
259	0\\
260	0\\
261	0\\
262	0\\
263	0\\
264	0\\
265	0\\
266	0\\
267	0\\
268	0\\
269	0\\
270	0\\
271	0\\
272	0\\
273	0\\
274	0\\
275	0\\
276	0\\
277	0\\
278	0\\
279	0\\
280	0\\
281	0\\
282	0\\
283	0\\
284	0\\
285	0\\
286	0\\
287	0\\
288	0\\
289	0\\
290	0\\
291	0\\
292	0\\
293	0\\
294	0\\
295	0\\
296	0\\
297	0\\
298	0\\
299	0\\
300	0\\
301	0\\
302	0\\
303	0\\
304	0\\
305	0\\
306	0\\
307	0\\
308	0\\
309	0\\
310	0\\
311	0\\
312	0\\
313	0\\
314	0\\
315	0\\
316	0\\
317	0\\
318	0\\
319	0\\
320	0\\
321	0\\
322	0\\
323	0\\
324	0\\
325	0\\
326	0\\
327	0\\
328	0\\
329	0\\
330	0\\
331	0\\
332	0\\
333	0\\
334	0\\
335	0\\
336	0\\
337	0\\
338	0\\
339	0\\
340	0\\
341	0\\
342	0\\
343	0\\
344	0\\
345	0\\
346	0\\
347	0\\
348	0\\
349	0\\
350	0\\
351	0\\
352	0\\
353	0\\
354	0\\
355	0\\
356	0\\
357	0\\
358	0\\
359	0\\
360	0\\
361	0\\
362	0\\
363	0\\
364	0\\
365	0\\
366	0\\
367	0\\
368	0\\
369	0\\
370	0\\
371	0\\
372	0\\
373	0\\
374	0\\
375	0\\
376	0\\
377	0\\
378	0\\
379	0\\
380	0\\
381	0\\
382	0\\
383	0\\
384	0\\
385	0\\
386	0\\
387	0\\
388	0\\
389	0\\
390	0\\
391	0\\
392	0\\
393	0\\
394	0\\
395	0\\
396	0\\
397	0\\
398	0\\
399	0\\
400	0\\
401	0\\
402	0\\
403	0\\
404	0\\
405	0\\
406	0\\
407	0\\
408	0\\
409	0\\
410	0\\
411	0\\
412	0\\
413	0\\
414	0\\
415	0\\
416	0\\
417	0\\
418	0\\
419	0\\
420	0\\
421	0\\
422	0\\
423	0\\
424	0\\
425	0\\
426	0\\
427	0\\
428	0\\
429	0\\
430	0\\
431	0\\
432	0\\
433	0\\
434	0\\
435	0\\
436	0\\
437	0\\
438	0\\
439	0\\
440	0\\
441	0\\
442	0\\
443	0\\
444	0\\
445	0\\
446	0\\
447	0\\
448	0\\
449	0\\
450	0\\
451	0\\
452	0\\
453	0\\
454	0\\
455	0\\
456	0\\
457	0\\
458	0\\
459	0\\
460	0\\
461	0\\
462	0\\
463	0\\
464	0\\
465	0\\
466	0\\
467	0\\
468	0\\
469	0\\
470	0\\
471	0\\
472	0\\
473	0\\
474	0\\
475	0\\
476	0\\
477	0\\
478	0\\
479	0\\
480	0\\
481	0\\
482	0\\
483	0\\
484	0\\
485	0\\
486	0\\
487	0\\
488	0\\
489	0\\
490	0\\
491	0\\
492	0\\
493	0\\
494	0\\
495	0\\
496	0\\
497	0\\
498	0\\
499	0\\
500	0\\
501	0\\
502	0\\
503	0\\
504	0\\
505	0\\
506	0\\
507	0\\
508	0\\
509	0\\
510	0\\
511	0\\
512	0\\
513	0\\
514	0\\
515	0\\
516	0\\
517	0\\
518	0\\
519	0\\
520	0\\
521	0\\
522	0\\
523	0\\
524	0\\
525	0\\
526	0\\
527	0\\
528	0\\
529	0\\
530	0\\
531	0\\
532	0\\
533	0\\
534	0\\
535	0\\
536	0\\
537	0\\
538	0\\
539	0\\
540	0\\
541	0\\
542	0\\
543	0\\
544	0\\
545	0\\
546	0\\
547	0\\
548	0\\
549	0\\
550	0\\
551	0\\
552	0\\
553	0\\
554	0\\
555	2.26142481073285e-05\\
556	9.40356632703977e-05\\
557	0.000142834796507731\\
558	0.000192136179755935\\
559	0.000241890470607289\\
560	0.000292049373855764\\
561	0.000342545162924321\\
562	0.00039329687625306\\
563	0.000444211080034401\\
564	0.000495180373312776\\
565	0.000546081753293092\\
566	0.000596774292680829\\
567	0.000647098131380762\\
568	0.00069742347279142\\
569	0.000748696607560934\\
570	0.000800939380062976\\
571	0.000854177294682413\\
572	0.000908439254781951\\
573	0.000963750984062382\\
574	0.00102014366327751\\
575	0.00107765573610446\\
576	0.00113633523369278\\
577	0.00119624324802379\\
578	0.00125745746076917\\
579	0.00132007724518431\\
580	0.00138423112120714\\
581	0.0014499990846374\\
582	0.00151746183138254\\
583	0.001586633971824\\
584	0.00165775679116301\\
585	0.00173067211608291\\
586	0.00180616539338125\\
587	0.0018831539132818\\
588	0.00209862842761386\\
589	0.00237026081777734\\
590	0.00265207182564143\\
591	0.00293299955993958\\
592	0.0031940225228887\\
593	0.00348473277442626\\
594	0.00383471000680442\\
595	0.00412245907862188\\
596	0.00444436306303141\\
597	0.00503983077166121\\
598	0.00644286460810295\\
599	0\\
600	0\\
};
\addplot [color=mycolor18,solid,forget plot]
  table[row sep=crcr]{%
1	0\\
2	0\\
3	0\\
4	0\\
5	0\\
6	0\\
7	0\\
8	0\\
9	0\\
10	0\\
11	0\\
12	0\\
13	0\\
14	0\\
15	0\\
16	0\\
17	0\\
18	0\\
19	0\\
20	0\\
21	0\\
22	0\\
23	0\\
24	0\\
25	0\\
26	0\\
27	0\\
28	0\\
29	0\\
30	0\\
31	0\\
32	0\\
33	0\\
34	0\\
35	0\\
36	0\\
37	0\\
38	0\\
39	0\\
40	0\\
41	0\\
42	0\\
43	0\\
44	0\\
45	0\\
46	0\\
47	0\\
48	0\\
49	0\\
50	0\\
51	0\\
52	0\\
53	0\\
54	0\\
55	0\\
56	0\\
57	0\\
58	0\\
59	0\\
60	0\\
61	0\\
62	0\\
63	0\\
64	0\\
65	0\\
66	0\\
67	0\\
68	0\\
69	0\\
70	0\\
71	0\\
72	0\\
73	0\\
74	0\\
75	0\\
76	0\\
77	0\\
78	0\\
79	0\\
80	0\\
81	0\\
82	0\\
83	0\\
84	0\\
85	0\\
86	0\\
87	0\\
88	0\\
89	0\\
90	0\\
91	0\\
92	0\\
93	0\\
94	0\\
95	0\\
96	0\\
97	0\\
98	0\\
99	0\\
100	0\\
101	0\\
102	0\\
103	0\\
104	0\\
105	0\\
106	0\\
107	0\\
108	0\\
109	0\\
110	0\\
111	0\\
112	0\\
113	0\\
114	0\\
115	0\\
116	0\\
117	0\\
118	0\\
119	0\\
120	0\\
121	0\\
122	0\\
123	0\\
124	0\\
125	0\\
126	0\\
127	0\\
128	0\\
129	0\\
130	0\\
131	0\\
132	0\\
133	0\\
134	0\\
135	0\\
136	0\\
137	0\\
138	0\\
139	0\\
140	0\\
141	0\\
142	0\\
143	0\\
144	0\\
145	0\\
146	0\\
147	0\\
148	0\\
149	0\\
150	0\\
151	0\\
152	0\\
153	0\\
154	0\\
155	0\\
156	0\\
157	0\\
158	0\\
159	0\\
160	0\\
161	0\\
162	0\\
163	0\\
164	0\\
165	0\\
166	0\\
167	0\\
168	0\\
169	0\\
170	0\\
171	0\\
172	0\\
173	0\\
174	0\\
175	0\\
176	0\\
177	0\\
178	0\\
179	0\\
180	0\\
181	0\\
182	0\\
183	0\\
184	0\\
185	0\\
186	0\\
187	0\\
188	0\\
189	0\\
190	0\\
191	0\\
192	0\\
193	0\\
194	0\\
195	0\\
196	0\\
197	0\\
198	0\\
199	0\\
200	0\\
201	0\\
202	0\\
203	0\\
204	0\\
205	0\\
206	0\\
207	0\\
208	0\\
209	0\\
210	0\\
211	0\\
212	0\\
213	0\\
214	0\\
215	0\\
216	0\\
217	0\\
218	0\\
219	0\\
220	0\\
221	0\\
222	0\\
223	0\\
224	0\\
225	0\\
226	0\\
227	0\\
228	0\\
229	0\\
230	0\\
231	0\\
232	0\\
233	0\\
234	0\\
235	0\\
236	0\\
237	0\\
238	0\\
239	0\\
240	0\\
241	0\\
242	0\\
243	0\\
244	0\\
245	0\\
246	0\\
247	0\\
248	0\\
249	0\\
250	0\\
251	0\\
252	0\\
253	0\\
254	0\\
255	0\\
256	0\\
257	0\\
258	0\\
259	0\\
260	0\\
261	0\\
262	0\\
263	0\\
264	0\\
265	0\\
266	0\\
267	0\\
268	0\\
269	0\\
270	0\\
271	0\\
272	0\\
273	0\\
274	0\\
275	0\\
276	0\\
277	0\\
278	0\\
279	0\\
280	0\\
281	0\\
282	0\\
283	0\\
284	0\\
285	0\\
286	0\\
287	0\\
288	0\\
289	0\\
290	0\\
291	0\\
292	0\\
293	0\\
294	0\\
295	0\\
296	0\\
297	0\\
298	0\\
299	0\\
300	0\\
301	0\\
302	0\\
303	0\\
304	0\\
305	0\\
306	0\\
307	0\\
308	0\\
309	0\\
310	0\\
311	0\\
312	0\\
313	0\\
314	0\\
315	0\\
316	0\\
317	0\\
318	0\\
319	0\\
320	0\\
321	0\\
322	0\\
323	0\\
324	0\\
325	0\\
326	0\\
327	0\\
328	0\\
329	0\\
330	0\\
331	0\\
332	0\\
333	0\\
334	0\\
335	0\\
336	0\\
337	0\\
338	0\\
339	0\\
340	0\\
341	0\\
342	0\\
343	0\\
344	0\\
345	0\\
346	0\\
347	0\\
348	0\\
349	0\\
350	0\\
351	0\\
352	0\\
353	0\\
354	0\\
355	0\\
356	0\\
357	0\\
358	0\\
359	0\\
360	0\\
361	0\\
362	0\\
363	0\\
364	0\\
365	0\\
366	0\\
367	0\\
368	0\\
369	0\\
370	0\\
371	0\\
372	0\\
373	0\\
374	0\\
375	0\\
376	0\\
377	0\\
378	0\\
379	0\\
380	0\\
381	0\\
382	0\\
383	0\\
384	0\\
385	0\\
386	0\\
387	0\\
388	0\\
389	0\\
390	0\\
391	0\\
392	0\\
393	0\\
394	0\\
395	0\\
396	0\\
397	0\\
398	0\\
399	0\\
400	0\\
401	0\\
402	0\\
403	0\\
404	0\\
405	0\\
406	0\\
407	0\\
408	0\\
409	0\\
410	0\\
411	0\\
412	0\\
413	0\\
414	0\\
415	0\\
416	0\\
417	0\\
418	0\\
419	0\\
420	0\\
421	0\\
422	0\\
423	0\\
424	0\\
425	0\\
426	0\\
427	0\\
428	0\\
429	0\\
430	0\\
431	0\\
432	0\\
433	0\\
434	0\\
435	0\\
436	0\\
437	0\\
438	0\\
439	0\\
440	0\\
441	0\\
442	0\\
443	0\\
444	0\\
445	0\\
446	0\\
447	0\\
448	0\\
449	0\\
450	0\\
451	0\\
452	0\\
453	0\\
454	0\\
455	0\\
456	0\\
457	0\\
458	0\\
459	0\\
460	0\\
461	0\\
462	0\\
463	0\\
464	0\\
465	0\\
466	0\\
467	0\\
468	0\\
469	0\\
470	0\\
471	0\\
472	0\\
473	0\\
474	0\\
475	0\\
476	0\\
477	0\\
478	0\\
479	0\\
480	0\\
481	0\\
482	0\\
483	0\\
484	0\\
485	0\\
486	0\\
487	0\\
488	0\\
489	0\\
490	0\\
491	0\\
492	0\\
493	0\\
494	0\\
495	0\\
496	0\\
497	0\\
498	0\\
499	0\\
500	0\\
501	0\\
502	0\\
503	0\\
504	0\\
505	0\\
506	0\\
507	0\\
508	0\\
509	0\\
510	0\\
511	0\\
512	0\\
513	0\\
514	0\\
515	0\\
516	0\\
517	0\\
518	0\\
519	0\\
520	0\\
521	0\\
522	0\\
523	0\\
524	0\\
525	0\\
526	0\\
527	0\\
528	0\\
529	0\\
530	0\\
531	0\\
532	0\\
533	0\\
534	0\\
535	0\\
536	0\\
537	0\\
538	0\\
539	0\\
540	0\\
541	0\\
542	0\\
543	0\\
544	0\\
545	0\\
546	0\\
547	0\\
548	0\\
549	2.32520210821688e-05\\
550	6.29822389686342e-05\\
551	0.000102594284985584\\
552	0.000142000070962094\\
553	0.000181109221367284\\
554	0.000219804609413708\\
555	0.000257961564445722\\
556	0.000295780758493718\\
557	0.000334217678983119\\
558	0.000373283831162645\\
559	0.000412993211787373\\
560	0.000453363060135725\\
561	0.000494414867686697\\
562	0.000536175585096235\\
563	0.000578679087463031\\
564	0.000621967988206539\\
565	0.000666095878568849\\
566	0.000711130094990775\\
567	0.000757155082147057\\
568	0.000804247376185817\\
569	0.000852443417848277\\
570	0.00090178212884265\\
571	0.000952305207691893\\
572	0.00100405760152108\\
573	0.00105708760588472\\
574	0.00111144706147627\\
575	0.0011671914887648\\
576	0.00122442232407532\\
577	0.00128313096863521\\
578	0.0013433810070614\\
579	0.00140538131268943\\
580	0.00146902614413746\\
581	0.00153430958574611\\
582	0.00160209512119075\\
583	0.00167132716120464\\
584	0.0017419265568923\\
585	0.0019582562819987\\
586	0.00223293796086997\\
587	0.00252513250926167\\
588	0.00278502806553789\\
589	0.00303946052792341\\
590	0.0033058313870396\\
591	0.00353307097397887\\
592	0.00367241590802784\\
593	0.00382251134333667\\
594	0.00397190834964223\\
595	0.00415280575412962\\
596	0.00444436306303141\\
597	0.00503983077166121\\
598	0.00644286460810295\\
599	0\\
600	0\\
};
\addplot [color=red!25!mycolor17,solid,forget plot]
  table[row sep=crcr]{%
1	0\\
2	0\\
3	0\\
4	0\\
5	0\\
6	0\\
7	0\\
8	0\\
9	0\\
10	0\\
11	0\\
12	0\\
13	0\\
14	0\\
15	0\\
16	0\\
17	0\\
18	0\\
19	0\\
20	0\\
21	0\\
22	0\\
23	0\\
24	0\\
25	0\\
26	0\\
27	0\\
28	0\\
29	0\\
30	0\\
31	0\\
32	0\\
33	0\\
34	0\\
35	0\\
36	0\\
37	0\\
38	0\\
39	0\\
40	0\\
41	0\\
42	0\\
43	0\\
44	0\\
45	0\\
46	0\\
47	0\\
48	0\\
49	0\\
50	0\\
51	0\\
52	0\\
53	0\\
54	0\\
55	0\\
56	0\\
57	0\\
58	0\\
59	0\\
60	0\\
61	0\\
62	0\\
63	0\\
64	0\\
65	0\\
66	0\\
67	0\\
68	0\\
69	0\\
70	0\\
71	0\\
72	0\\
73	0\\
74	0\\
75	0\\
76	0\\
77	0\\
78	0\\
79	0\\
80	0\\
81	0\\
82	0\\
83	0\\
84	0\\
85	0\\
86	0\\
87	0\\
88	0\\
89	0\\
90	0\\
91	0\\
92	0\\
93	0\\
94	0\\
95	0\\
96	0\\
97	0\\
98	0\\
99	0\\
100	0\\
101	0\\
102	0\\
103	0\\
104	0\\
105	0\\
106	0\\
107	0\\
108	0\\
109	0\\
110	0\\
111	0\\
112	0\\
113	0\\
114	0\\
115	0\\
116	0\\
117	0\\
118	0\\
119	0\\
120	0\\
121	0\\
122	0\\
123	0\\
124	0\\
125	0\\
126	0\\
127	0\\
128	0\\
129	0\\
130	0\\
131	0\\
132	0\\
133	0\\
134	0\\
135	0\\
136	0\\
137	0\\
138	0\\
139	0\\
140	0\\
141	0\\
142	0\\
143	0\\
144	0\\
145	0\\
146	0\\
147	0\\
148	0\\
149	0\\
150	0\\
151	0\\
152	0\\
153	0\\
154	0\\
155	0\\
156	0\\
157	0\\
158	0\\
159	0\\
160	0\\
161	0\\
162	0\\
163	0\\
164	0\\
165	0\\
166	0\\
167	0\\
168	0\\
169	0\\
170	0\\
171	0\\
172	0\\
173	0\\
174	0\\
175	0\\
176	0\\
177	0\\
178	0\\
179	0\\
180	0\\
181	0\\
182	0\\
183	0\\
184	0\\
185	0\\
186	0\\
187	0\\
188	0\\
189	0\\
190	0\\
191	0\\
192	0\\
193	0\\
194	0\\
195	0\\
196	0\\
197	0\\
198	0\\
199	0\\
200	0\\
201	0\\
202	0\\
203	0\\
204	0\\
205	0\\
206	0\\
207	0\\
208	0\\
209	0\\
210	0\\
211	0\\
212	0\\
213	0\\
214	0\\
215	0\\
216	0\\
217	0\\
218	0\\
219	0\\
220	0\\
221	0\\
222	0\\
223	0\\
224	0\\
225	0\\
226	0\\
227	0\\
228	0\\
229	0\\
230	0\\
231	0\\
232	0\\
233	0\\
234	0\\
235	0\\
236	0\\
237	0\\
238	0\\
239	0\\
240	0\\
241	0\\
242	0\\
243	0\\
244	0\\
245	0\\
246	0\\
247	0\\
248	0\\
249	0\\
250	0\\
251	0\\
252	0\\
253	0\\
254	0\\
255	0\\
256	0\\
257	0\\
258	0\\
259	0\\
260	0\\
261	0\\
262	0\\
263	0\\
264	0\\
265	0\\
266	0\\
267	0\\
268	0\\
269	0\\
270	0\\
271	0\\
272	0\\
273	0\\
274	0\\
275	0\\
276	0\\
277	0\\
278	0\\
279	0\\
280	0\\
281	0\\
282	0\\
283	0\\
284	0\\
285	0\\
286	0\\
287	0\\
288	0\\
289	0\\
290	0\\
291	0\\
292	0\\
293	0\\
294	0\\
295	0\\
296	0\\
297	0\\
298	0\\
299	0\\
300	0\\
301	0\\
302	0\\
303	0\\
304	0\\
305	0\\
306	0\\
307	0\\
308	0\\
309	0\\
310	0\\
311	0\\
312	0\\
313	0\\
314	0\\
315	0\\
316	0\\
317	0\\
318	0\\
319	0\\
320	0\\
321	0\\
322	0\\
323	0\\
324	0\\
325	0\\
326	0\\
327	0\\
328	0\\
329	0\\
330	0\\
331	0\\
332	0\\
333	0\\
334	0\\
335	0\\
336	0\\
337	0\\
338	0\\
339	0\\
340	0\\
341	0\\
342	0\\
343	0\\
344	0\\
345	0\\
346	0\\
347	0\\
348	0\\
349	0\\
350	0\\
351	0\\
352	0\\
353	0\\
354	0\\
355	0\\
356	0\\
357	0\\
358	0\\
359	0\\
360	0\\
361	0\\
362	0\\
363	0\\
364	0\\
365	0\\
366	0\\
367	0\\
368	0\\
369	0\\
370	0\\
371	0\\
372	0\\
373	0\\
374	0\\
375	0\\
376	0\\
377	0\\
378	0\\
379	0\\
380	0\\
381	0\\
382	0\\
383	0\\
384	0\\
385	0\\
386	0\\
387	0\\
388	0\\
389	0\\
390	0\\
391	0\\
392	0\\
393	0\\
394	0\\
395	0\\
396	0\\
397	0\\
398	0\\
399	0\\
400	0\\
401	0\\
402	0\\
403	0\\
404	0\\
405	0\\
406	0\\
407	0\\
408	0\\
409	0\\
410	0\\
411	0\\
412	0\\
413	0\\
414	0\\
415	0\\
416	0\\
417	0\\
418	0\\
419	0\\
420	0\\
421	0\\
422	0\\
423	0\\
424	0\\
425	0\\
426	0\\
427	0\\
428	0\\
429	0\\
430	0\\
431	0\\
432	0\\
433	0\\
434	0\\
435	0\\
436	0\\
437	0\\
438	0\\
439	0\\
440	0\\
441	0\\
442	0\\
443	0\\
444	0\\
445	0\\
446	0\\
447	0\\
448	0\\
449	0\\
450	0\\
451	0\\
452	0\\
453	0\\
454	0\\
455	0\\
456	0\\
457	0\\
458	0\\
459	0\\
460	0\\
461	0\\
462	0\\
463	0\\
464	0\\
465	0\\
466	0\\
467	0\\
468	0\\
469	0\\
470	0\\
471	0\\
472	0\\
473	0\\
474	0\\
475	0\\
476	0\\
477	0\\
478	0\\
479	0\\
480	0\\
481	0\\
482	0\\
483	0\\
484	0\\
485	0\\
486	0\\
487	0\\
488	0\\
489	0\\
490	0\\
491	0\\
492	0\\
493	0\\
494	0\\
495	0\\
496	0\\
497	0\\
498	0\\
499	0\\
500	0\\
501	0\\
502	0\\
503	0\\
504	0\\
505	0\\
506	0\\
507	0\\
508	0\\
509	0\\
510	0\\
511	0\\
512	0\\
513	0\\
514	0\\
515	0\\
516	0\\
517	0\\
518	0\\
519	0\\
520	0\\
521	0\\
522	0\\
523	0\\
524	0\\
525	0\\
526	0\\
527	0\\
528	0\\
529	0\\
530	0\\
531	0\\
532	0\\
533	0\\
534	0\\
535	0\\
536	0\\
537	0\\
538	0\\
539	0\\
540	0\\
541	0\\
542	0\\
543	0\\
544	0\\
545	1.60914883874299e-05\\
546	4.61802426780253e-05\\
547	7.66629135509208e-05\\
548	0.000107555208654854\\
549	0.000138876541337362\\
550	0.000170650593952384\\
551	0.000202914517604868\\
552	0.000235708465118579\\
553	0.000269081896836238\\
554	0.000303087169673518\\
555	0.000337789238210994\\
556	0.000373251698431509\\
557	0.000409499257060565\\
558	0.000446558227820018\\
559	0.000484456617435164\\
560	0.000523224188690424\\
561	0.000562892484763894\\
562	0.000603494791845105\\
563	0.000645066023735202\\
564	0.000687642494400291\\
565	0.000731261531316776\\
566	0.000775960935897401\\
567	0.000821778458148387\\
568	0.000868751612779525\\
569	0.000916919738931042\\
570	0.000966334419799042\\
571	0.00101705468567766\\
572	0.00106906417782553\\
573	0.00112241341855564\\
574	0.00117715592173479\\
575	0.00123354169018176\\
576	0.00129128803526078\\
577	0.00135048220327041\\
578	0.00141156374788578\\
579	0.00147451749231703\\
580	0.00153861733116412\\
581	0.00160404882434682\\
582	0.0017589725145217\\
583	0.00203755571568777\\
584	0.00233345386073883\\
585	0.00259266855288805\\
586	0.00284513355551299\\
587	0.00310713312521459\\
588	0.0032721022788783\\
589	0.00339546551785458\\
590	0.00351237211383029\\
591	0.00361868673458143\\
592	0.0037291949721508\\
593	0.00384481075529708\\
594	0.00397672044113206\\
595	0.00415280575412961\\
596	0.00444436306303141\\
597	0.00503983077166121\\
598	0.00644286460810295\\
599	0\\
600	0\\
};
\addplot [color=mycolor19,solid,forget plot]
  table[row sep=crcr]{%
1	0\\
2	0\\
3	0\\
4	0\\
5	0\\
6	0\\
7	0\\
8	0\\
9	0\\
10	0\\
11	0\\
12	0\\
13	0\\
14	0\\
15	0\\
16	0\\
17	0\\
18	0\\
19	0\\
20	0\\
21	0\\
22	0\\
23	0\\
24	0\\
25	0\\
26	0\\
27	0\\
28	0\\
29	0\\
30	0\\
31	0\\
32	0\\
33	0\\
34	0\\
35	0\\
36	0\\
37	0\\
38	0\\
39	0\\
40	0\\
41	0\\
42	0\\
43	0\\
44	0\\
45	0\\
46	0\\
47	0\\
48	0\\
49	0\\
50	0\\
51	0\\
52	0\\
53	0\\
54	0\\
55	0\\
56	0\\
57	0\\
58	0\\
59	0\\
60	0\\
61	0\\
62	0\\
63	0\\
64	0\\
65	0\\
66	0\\
67	0\\
68	0\\
69	0\\
70	0\\
71	0\\
72	0\\
73	0\\
74	0\\
75	0\\
76	0\\
77	0\\
78	0\\
79	0\\
80	0\\
81	0\\
82	0\\
83	0\\
84	0\\
85	0\\
86	0\\
87	0\\
88	0\\
89	0\\
90	0\\
91	0\\
92	0\\
93	0\\
94	0\\
95	0\\
96	0\\
97	0\\
98	0\\
99	0\\
100	0\\
101	0\\
102	0\\
103	0\\
104	0\\
105	0\\
106	0\\
107	0\\
108	0\\
109	0\\
110	0\\
111	0\\
112	0\\
113	0\\
114	0\\
115	0\\
116	0\\
117	0\\
118	0\\
119	0\\
120	0\\
121	0\\
122	0\\
123	0\\
124	0\\
125	0\\
126	0\\
127	0\\
128	0\\
129	0\\
130	0\\
131	0\\
132	0\\
133	0\\
134	0\\
135	0\\
136	0\\
137	0\\
138	0\\
139	0\\
140	0\\
141	0\\
142	0\\
143	0\\
144	0\\
145	0\\
146	0\\
147	0\\
148	0\\
149	0\\
150	0\\
151	0\\
152	0\\
153	0\\
154	0\\
155	0\\
156	0\\
157	0\\
158	0\\
159	0\\
160	0\\
161	0\\
162	0\\
163	0\\
164	0\\
165	0\\
166	0\\
167	0\\
168	0\\
169	0\\
170	0\\
171	0\\
172	0\\
173	0\\
174	0\\
175	0\\
176	0\\
177	0\\
178	0\\
179	0\\
180	0\\
181	0\\
182	0\\
183	0\\
184	0\\
185	0\\
186	0\\
187	0\\
188	0\\
189	0\\
190	0\\
191	0\\
192	0\\
193	0\\
194	0\\
195	0\\
196	0\\
197	0\\
198	0\\
199	0\\
200	0\\
201	0\\
202	0\\
203	0\\
204	0\\
205	0\\
206	0\\
207	0\\
208	0\\
209	0\\
210	0\\
211	0\\
212	0\\
213	0\\
214	0\\
215	0\\
216	0\\
217	0\\
218	0\\
219	0\\
220	0\\
221	0\\
222	0\\
223	0\\
224	0\\
225	0\\
226	0\\
227	0\\
228	0\\
229	0\\
230	0\\
231	0\\
232	0\\
233	0\\
234	0\\
235	0\\
236	0\\
237	0\\
238	0\\
239	0\\
240	0\\
241	0\\
242	0\\
243	0\\
244	0\\
245	0\\
246	0\\
247	0\\
248	0\\
249	0\\
250	0\\
251	0\\
252	0\\
253	0\\
254	0\\
255	0\\
256	0\\
257	0\\
258	0\\
259	0\\
260	0\\
261	0\\
262	0\\
263	0\\
264	0\\
265	0\\
266	0\\
267	0\\
268	0\\
269	0\\
270	0\\
271	0\\
272	0\\
273	0\\
274	0\\
275	0\\
276	0\\
277	0\\
278	0\\
279	0\\
280	0\\
281	0\\
282	0\\
283	0\\
284	0\\
285	0\\
286	0\\
287	0\\
288	0\\
289	0\\
290	0\\
291	0\\
292	0\\
293	0\\
294	0\\
295	0\\
296	0\\
297	0\\
298	0\\
299	0\\
300	0\\
301	0\\
302	0\\
303	0\\
304	0\\
305	0\\
306	0\\
307	0\\
308	0\\
309	0\\
310	0\\
311	0\\
312	0\\
313	0\\
314	0\\
315	0\\
316	0\\
317	0\\
318	0\\
319	0\\
320	0\\
321	0\\
322	0\\
323	0\\
324	0\\
325	0\\
326	0\\
327	0\\
328	0\\
329	0\\
330	0\\
331	0\\
332	0\\
333	0\\
334	0\\
335	0\\
336	0\\
337	0\\
338	0\\
339	0\\
340	0\\
341	0\\
342	0\\
343	0\\
344	0\\
345	0\\
346	0\\
347	0\\
348	0\\
349	0\\
350	0\\
351	0\\
352	0\\
353	0\\
354	0\\
355	0\\
356	0\\
357	0\\
358	0\\
359	0\\
360	0\\
361	0\\
362	0\\
363	0\\
364	0\\
365	0\\
366	0\\
367	0\\
368	0\\
369	0\\
370	0\\
371	0\\
372	0\\
373	0\\
374	0\\
375	0\\
376	0\\
377	0\\
378	0\\
379	0\\
380	0\\
381	0\\
382	0\\
383	0\\
384	0\\
385	0\\
386	0\\
387	0\\
388	0\\
389	0\\
390	0\\
391	0\\
392	0\\
393	0\\
394	0\\
395	0\\
396	0\\
397	0\\
398	0\\
399	0\\
400	0\\
401	0\\
402	0\\
403	0\\
404	0\\
405	0\\
406	0\\
407	0\\
408	0\\
409	0\\
410	0\\
411	0\\
412	0\\
413	0\\
414	0\\
415	0\\
416	0\\
417	0\\
418	0\\
419	0\\
420	0\\
421	0\\
422	0\\
423	0\\
424	0\\
425	0\\
426	0\\
427	0\\
428	0\\
429	0\\
430	0\\
431	0\\
432	0\\
433	0\\
434	0\\
435	0\\
436	0\\
437	0\\
438	0\\
439	0\\
440	0\\
441	0\\
442	0\\
443	0\\
444	0\\
445	0\\
446	0\\
447	0\\
448	0\\
449	0\\
450	0\\
451	0\\
452	0\\
453	0\\
454	0\\
455	0\\
456	0\\
457	0\\
458	0\\
459	0\\
460	0\\
461	0\\
462	0\\
463	0\\
464	0\\
465	0\\
466	0\\
467	0\\
468	0\\
469	0\\
470	0\\
471	0\\
472	0\\
473	0\\
474	0\\
475	0\\
476	0\\
477	0\\
478	0\\
479	0\\
480	0\\
481	0\\
482	0\\
483	0\\
484	0\\
485	0\\
486	0\\
487	0\\
488	0\\
489	0\\
490	0\\
491	0\\
492	0\\
493	0\\
494	0\\
495	0\\
496	0\\
497	0\\
498	0\\
499	0\\
500	0\\
501	0\\
502	0\\
503	0\\
504	0\\
505	0\\
506	0\\
507	0\\
508	0\\
509	0\\
510	0\\
511	0\\
512	0\\
513	0\\
514	0\\
515	0\\
516	0\\
517	0\\
518	0\\
519	0\\
520	0\\
521	0\\
522	0\\
523	0\\
524	0\\
525	0\\
526	0\\
527	0\\
528	0\\
529	0\\
530	0\\
531	0\\
532	0\\
533	0\\
534	0\\
535	0\\
536	0\\
537	0\\
538	0\\
539	0\\
540	0\\
541	0\\
542	0\\
543	1.6281523779886e-05\\
544	4.35246637006791e-05\\
545	7.13087122634737e-05\\
546	9.96552821353406e-05\\
547	0.000128587210985315\\
548	0.000158128564717491\\
549	0.000188304258646555\\
550	0.000219151430388676\\
551	0.000250701151250487\\
552	0.000282981360877781\\
553	0.000316017445589515\\
554	0.000349834654432245\\
555	0.000384457427831735\\
556	0.000419909302190068\\
557	0.000456214671566793\\
558	0.000493398798486647\\
559	0.000531487829578367\\
560	0.000570508810018441\\
561	0.000610489697648098\\
562	0.000651459746879398\\
563	0.000693449329616425\\
564	0.000736489807219065\\
565	0.000780619589166497\\
566	0.000825901473907119\\
567	0.000872296831358665\\
568	0.000919843835008248\\
569	0.000968584807559033\\
570	0.00101855905585974\\
571	0.00106996928087748\\
572	0.0011226278711555\\
573	0.00117653489629999\\
574	0.00123176683108536\\
575	0.00128877167129672\\
576	0.00134736204101222\\
577	0.00140700004607742\\
578	0.00146757979920448\\
579	0.00152899875679746\\
580	0.00178174393908874\\
581	0.00207611800840707\\
582	0.00235517143267944\\
583	0.0026041334094422\\
584	0.00286156611296512\\
585	0.00301924574469048\\
586	0.00313397161576881\\
587	0.00323536006699526\\
588	0.00333221291604738\\
589	0.00342954842160734\\
590	0.0035273364337839\\
591	0.00362803120808679\\
592	0.00373275376052986\\
593	0.00384555904940006\\
594	0.00397672044113206\\
595	0.00415280575412961\\
596	0.00444436306303141\\
597	0.00503983077166121\\
598	0.00644286460810295\\
599	0\\
600	0\\
};
\addplot [color=red!50!mycolor17,solid,forget plot]
  table[row sep=crcr]{%
1	0\\
2	0\\
3	0\\
4	0\\
5	0\\
6	0\\
7	0\\
8	0\\
9	0\\
10	0\\
11	0\\
12	0\\
13	0\\
14	0\\
15	0\\
16	0\\
17	0\\
18	0\\
19	0\\
20	0\\
21	0\\
22	0\\
23	0\\
24	0\\
25	0\\
26	0\\
27	0\\
28	0\\
29	0\\
30	0\\
31	0\\
32	0\\
33	0\\
34	0\\
35	0\\
36	0\\
37	0\\
38	0\\
39	0\\
40	0\\
41	0\\
42	0\\
43	0\\
44	0\\
45	0\\
46	0\\
47	0\\
48	0\\
49	0\\
50	0\\
51	0\\
52	0\\
53	0\\
54	0\\
55	0\\
56	0\\
57	0\\
58	0\\
59	0\\
60	0\\
61	0\\
62	0\\
63	0\\
64	0\\
65	0\\
66	0\\
67	0\\
68	0\\
69	0\\
70	0\\
71	0\\
72	0\\
73	0\\
74	0\\
75	0\\
76	0\\
77	0\\
78	0\\
79	0\\
80	0\\
81	0\\
82	0\\
83	0\\
84	0\\
85	0\\
86	0\\
87	0\\
88	0\\
89	0\\
90	0\\
91	0\\
92	0\\
93	0\\
94	0\\
95	0\\
96	0\\
97	0\\
98	0\\
99	0\\
100	0\\
101	0\\
102	0\\
103	0\\
104	0\\
105	0\\
106	0\\
107	0\\
108	0\\
109	0\\
110	0\\
111	0\\
112	0\\
113	0\\
114	0\\
115	0\\
116	0\\
117	0\\
118	0\\
119	0\\
120	0\\
121	0\\
122	0\\
123	0\\
124	0\\
125	0\\
126	0\\
127	0\\
128	0\\
129	0\\
130	0\\
131	0\\
132	0\\
133	0\\
134	0\\
135	0\\
136	0\\
137	0\\
138	0\\
139	0\\
140	0\\
141	0\\
142	0\\
143	0\\
144	0\\
145	0\\
146	0\\
147	0\\
148	0\\
149	0\\
150	0\\
151	0\\
152	0\\
153	0\\
154	0\\
155	0\\
156	0\\
157	0\\
158	0\\
159	0\\
160	0\\
161	0\\
162	0\\
163	0\\
164	0\\
165	0\\
166	0\\
167	0\\
168	0\\
169	0\\
170	0\\
171	0\\
172	0\\
173	0\\
174	0\\
175	0\\
176	0\\
177	0\\
178	0\\
179	0\\
180	0\\
181	0\\
182	0\\
183	0\\
184	0\\
185	0\\
186	0\\
187	0\\
188	0\\
189	0\\
190	0\\
191	0\\
192	0\\
193	0\\
194	0\\
195	0\\
196	0\\
197	0\\
198	0\\
199	0\\
200	0\\
201	0\\
202	0\\
203	0\\
204	0\\
205	0\\
206	0\\
207	0\\
208	0\\
209	0\\
210	0\\
211	0\\
212	0\\
213	0\\
214	0\\
215	0\\
216	0\\
217	0\\
218	0\\
219	0\\
220	0\\
221	0\\
222	0\\
223	0\\
224	0\\
225	0\\
226	0\\
227	0\\
228	0\\
229	0\\
230	0\\
231	0\\
232	0\\
233	0\\
234	0\\
235	0\\
236	0\\
237	0\\
238	0\\
239	0\\
240	0\\
241	0\\
242	0\\
243	0\\
244	0\\
245	0\\
246	0\\
247	0\\
248	0\\
249	0\\
250	0\\
251	0\\
252	0\\
253	0\\
254	0\\
255	0\\
256	0\\
257	0\\
258	0\\
259	0\\
260	0\\
261	0\\
262	0\\
263	0\\
264	0\\
265	0\\
266	0\\
267	0\\
268	0\\
269	0\\
270	0\\
271	0\\
272	0\\
273	0\\
274	0\\
275	0\\
276	0\\
277	0\\
278	0\\
279	0\\
280	0\\
281	0\\
282	0\\
283	0\\
284	0\\
285	0\\
286	0\\
287	0\\
288	0\\
289	0\\
290	0\\
291	0\\
292	0\\
293	0\\
294	0\\
295	0\\
296	0\\
297	0\\
298	0\\
299	0\\
300	0\\
301	0\\
302	0\\
303	0\\
304	0\\
305	0\\
306	0\\
307	0\\
308	0\\
309	0\\
310	0\\
311	0\\
312	0\\
313	0\\
314	0\\
315	0\\
316	0\\
317	0\\
318	0\\
319	0\\
320	0\\
321	0\\
322	0\\
323	0\\
324	0\\
325	0\\
326	0\\
327	0\\
328	0\\
329	0\\
330	0\\
331	0\\
332	0\\
333	0\\
334	0\\
335	0\\
336	0\\
337	0\\
338	0\\
339	0\\
340	0\\
341	0\\
342	0\\
343	0\\
344	0\\
345	0\\
346	0\\
347	0\\
348	0\\
349	0\\
350	0\\
351	0\\
352	0\\
353	0\\
354	0\\
355	0\\
356	0\\
357	0\\
358	0\\
359	0\\
360	0\\
361	0\\
362	0\\
363	0\\
364	0\\
365	0\\
366	0\\
367	0\\
368	0\\
369	0\\
370	0\\
371	0\\
372	0\\
373	0\\
374	0\\
375	0\\
376	0\\
377	0\\
378	0\\
379	0\\
380	0\\
381	0\\
382	0\\
383	0\\
384	0\\
385	0\\
386	0\\
387	0\\
388	0\\
389	0\\
390	0\\
391	0\\
392	0\\
393	0\\
394	0\\
395	0\\
396	0\\
397	0\\
398	0\\
399	0\\
400	0\\
401	0\\
402	0\\
403	0\\
404	0\\
405	0\\
406	0\\
407	0\\
408	0\\
409	0\\
410	0\\
411	0\\
412	0\\
413	0\\
414	0\\
415	0\\
416	0\\
417	0\\
418	0\\
419	0\\
420	0\\
421	0\\
422	0\\
423	0\\
424	0\\
425	0\\
426	0\\
427	0\\
428	0\\
429	0\\
430	0\\
431	0\\
432	0\\
433	0\\
434	0\\
435	0\\
436	0\\
437	0\\
438	0\\
439	0\\
440	0\\
441	0\\
442	0\\
443	0\\
444	0\\
445	0\\
446	0\\
447	0\\
448	0\\
449	0\\
450	0\\
451	0\\
452	0\\
453	0\\
454	0\\
455	0\\
456	0\\
457	0\\
458	0\\
459	0\\
460	0\\
461	0\\
462	0\\
463	0\\
464	0\\
465	0\\
466	0\\
467	0\\
468	0\\
469	0\\
470	0\\
471	0\\
472	0\\
473	0\\
474	0\\
475	0\\
476	0\\
477	0\\
478	0\\
479	0\\
480	0\\
481	0\\
482	0\\
483	0\\
484	0\\
485	0\\
486	0\\
487	0\\
488	0\\
489	0\\
490	0\\
491	0\\
492	0\\
493	0\\
494	0\\
495	0\\
496	0\\
497	0\\
498	0\\
499	0\\
500	0\\
501	0\\
502	0\\
503	0\\
504	0\\
505	0\\
506	0\\
507	0\\
508	0\\
509	0\\
510	0\\
511	0\\
512	0\\
513	0\\
514	0\\
515	0\\
516	0\\
517	0\\
518	0\\
519	0\\
520	0\\
521	0\\
522	0\\
523	0\\
524	0\\
525	0\\
526	0\\
527	0\\
528	0\\
529	0\\
530	0\\
531	0\\
532	0\\
533	0\\
534	0\\
535	0\\
536	0\\
537	0\\
538	0\\
539	0\\
540	0\\
541	0\\
542	2.31819356943729e-05\\
543	4.97808753607325e-05\\
544	7.69628614176778e-05\\
545	0.000104747319125633\\
546	0.000133154249522804\\
547	0.000162204210497156\\
548	0.000191918039953681\\
549	0.000222329556741014\\
550	0.000253464773718292\\
551	0.000285344204403849\\
552	0.00031798791545124\\
553	0.000351416424199199\\
554	0.000385650737194353\\
555	0.00042071243627783\\
556	0.000456623783107471\\
557	0.000493407751197362\\
558	0.000531088593263046\\
559	0.00056969136713803\\
560	0.000609241967672466\\
561	0.000649799792050662\\
562	0.000691347682418427\\
563	0.000733888225033008\\
564	0.000777453378418064\\
565	0.00082207615087447\\
566	0.000867771153143799\\
567	0.000914688584196929\\
568	0.000962856697167533\\
569	0.00101209789593015\\
570	0.00106246397516016\\
571	0.00111390186467904\\
572	0.00116698002710393\\
573	0.00122180007812752\\
574	0.00127748242167784\\
575	0.00133371885625811\\
576	0.00139081764574271\\
577	0.00146406439474163\\
578	0.00175490587469454\\
579	0.00206663004967047\\
580	0.00232104643036537\\
581	0.00257412158610118\\
582	0.00277111122536755\\
583	0.00287961411679738\\
584	0.00297466536060553\\
585	0.00306490423400819\\
586	0.00315466992901367\\
587	0.00324537223361411\\
588	0.0033377976514749\\
589	0.00343211111647921\\
590	0.00352885108616147\\
591	0.00362859107064115\\
592	0.00373286830485375\\
593	0.00384555904940007\\
594	0.00397672044113206\\
595	0.00415280575412962\\
596	0.00444436306303141\\
597	0.00503983077166121\\
598	0.00644286460810295\\
599	0\\
600	0\\
};
\addplot [color=red!40!mycolor19,solid,forget plot]
  table[row sep=crcr]{%
1	0\\
2	0\\
3	0\\
4	0\\
5	0\\
6	0\\
7	0\\
8	0\\
9	0\\
10	0\\
11	0\\
12	0\\
13	0\\
14	0\\
15	0\\
16	0\\
17	0\\
18	0\\
19	0\\
20	0\\
21	0\\
22	0\\
23	0\\
24	0\\
25	0\\
26	0\\
27	0\\
28	0\\
29	0\\
30	0\\
31	0\\
32	0\\
33	0\\
34	0\\
35	0\\
36	0\\
37	0\\
38	0\\
39	0\\
40	0\\
41	0\\
42	0\\
43	0\\
44	0\\
45	0\\
46	0\\
47	0\\
48	0\\
49	0\\
50	0\\
51	0\\
52	0\\
53	0\\
54	0\\
55	0\\
56	0\\
57	0\\
58	0\\
59	0\\
60	0\\
61	0\\
62	0\\
63	0\\
64	0\\
65	0\\
66	0\\
67	0\\
68	0\\
69	0\\
70	0\\
71	0\\
72	0\\
73	0\\
74	0\\
75	0\\
76	0\\
77	0\\
78	0\\
79	0\\
80	0\\
81	0\\
82	0\\
83	0\\
84	0\\
85	0\\
86	0\\
87	0\\
88	0\\
89	0\\
90	0\\
91	0\\
92	0\\
93	0\\
94	0\\
95	0\\
96	0\\
97	0\\
98	0\\
99	0\\
100	0\\
101	0\\
102	0\\
103	0\\
104	0\\
105	0\\
106	0\\
107	0\\
108	0\\
109	0\\
110	0\\
111	0\\
112	0\\
113	0\\
114	0\\
115	0\\
116	0\\
117	0\\
118	0\\
119	0\\
120	0\\
121	0\\
122	0\\
123	0\\
124	0\\
125	0\\
126	0\\
127	0\\
128	0\\
129	0\\
130	0\\
131	0\\
132	0\\
133	0\\
134	0\\
135	0\\
136	0\\
137	0\\
138	0\\
139	0\\
140	0\\
141	0\\
142	0\\
143	0\\
144	0\\
145	0\\
146	0\\
147	0\\
148	0\\
149	0\\
150	0\\
151	0\\
152	0\\
153	0\\
154	0\\
155	0\\
156	0\\
157	0\\
158	0\\
159	0\\
160	0\\
161	0\\
162	0\\
163	0\\
164	0\\
165	0\\
166	0\\
167	0\\
168	0\\
169	0\\
170	0\\
171	0\\
172	0\\
173	0\\
174	0\\
175	0\\
176	0\\
177	0\\
178	0\\
179	0\\
180	0\\
181	0\\
182	0\\
183	0\\
184	0\\
185	0\\
186	0\\
187	0\\
188	0\\
189	0\\
190	0\\
191	0\\
192	0\\
193	0\\
194	0\\
195	0\\
196	0\\
197	0\\
198	0\\
199	0\\
200	0\\
201	0\\
202	0\\
203	0\\
204	0\\
205	0\\
206	0\\
207	0\\
208	0\\
209	0\\
210	0\\
211	0\\
212	0\\
213	0\\
214	0\\
215	0\\
216	0\\
217	0\\
218	0\\
219	0\\
220	0\\
221	0\\
222	0\\
223	0\\
224	0\\
225	0\\
226	0\\
227	0\\
228	0\\
229	0\\
230	0\\
231	0\\
232	0\\
233	0\\
234	0\\
235	0\\
236	0\\
237	0\\
238	0\\
239	0\\
240	0\\
241	0\\
242	0\\
243	0\\
244	0\\
245	0\\
246	0\\
247	0\\
248	0\\
249	0\\
250	0\\
251	0\\
252	0\\
253	0\\
254	0\\
255	0\\
256	0\\
257	0\\
258	0\\
259	0\\
260	0\\
261	0\\
262	0\\
263	0\\
264	0\\
265	0\\
266	0\\
267	0\\
268	0\\
269	0\\
270	0\\
271	0\\
272	0\\
273	0\\
274	0\\
275	0\\
276	0\\
277	0\\
278	0\\
279	0\\
280	0\\
281	0\\
282	0\\
283	0\\
284	0\\
285	0\\
286	0\\
287	0\\
288	0\\
289	0\\
290	0\\
291	0\\
292	0\\
293	0\\
294	0\\
295	0\\
296	0\\
297	0\\
298	0\\
299	0\\
300	0\\
301	0\\
302	0\\
303	0\\
304	0\\
305	0\\
306	0\\
307	0\\
308	0\\
309	0\\
310	0\\
311	0\\
312	0\\
313	0\\
314	0\\
315	0\\
316	0\\
317	0\\
318	0\\
319	0\\
320	0\\
321	0\\
322	0\\
323	0\\
324	0\\
325	0\\
326	0\\
327	0\\
328	0\\
329	0\\
330	0\\
331	0\\
332	0\\
333	0\\
334	0\\
335	0\\
336	0\\
337	0\\
338	0\\
339	0\\
340	0\\
341	0\\
342	0\\
343	0\\
344	0\\
345	0\\
346	0\\
347	0\\
348	0\\
349	0\\
350	0\\
351	0\\
352	0\\
353	0\\
354	0\\
355	0\\
356	0\\
357	0\\
358	0\\
359	0\\
360	0\\
361	0\\
362	0\\
363	0\\
364	0\\
365	0\\
366	0\\
367	0\\
368	0\\
369	0\\
370	0\\
371	0\\
372	0\\
373	0\\
374	0\\
375	0\\
376	0\\
377	0\\
378	0\\
379	0\\
380	0\\
381	0\\
382	0\\
383	0\\
384	0\\
385	0\\
386	0\\
387	0\\
388	0\\
389	0\\
390	0\\
391	0\\
392	0\\
393	0\\
394	0\\
395	0\\
396	0\\
397	0\\
398	0\\
399	0\\
400	0\\
401	0\\
402	0\\
403	0\\
404	0\\
405	0\\
406	0\\
407	0\\
408	0\\
409	0\\
410	0\\
411	0\\
412	0\\
413	0\\
414	0\\
415	0\\
416	0\\
417	0\\
418	0\\
419	0\\
420	0\\
421	0\\
422	0\\
423	0\\
424	0\\
425	0\\
426	0\\
427	0\\
428	0\\
429	0\\
430	0\\
431	0\\
432	0\\
433	0\\
434	0\\
435	0\\
436	0\\
437	0\\
438	0\\
439	0\\
440	0\\
441	0\\
442	0\\
443	0\\
444	0\\
445	0\\
446	0\\
447	0\\
448	0\\
449	0\\
450	0\\
451	0\\
452	0\\
453	0\\
454	0\\
455	0\\
456	0\\
457	0\\
458	0\\
459	0\\
460	0\\
461	0\\
462	0\\
463	0\\
464	0\\
465	0\\
466	0\\
467	0\\
468	0\\
469	0\\
470	0\\
471	0\\
472	0\\
473	0\\
474	0\\
475	0\\
476	0\\
477	0\\
478	0\\
479	0\\
480	0\\
481	0\\
482	0\\
483	0\\
484	0\\
485	0\\
486	0\\
487	0\\
488	0\\
489	0\\
490	0\\
491	0\\
492	0\\
493	0\\
494	0\\
495	0\\
496	0\\
497	0\\
498	0\\
499	0\\
500	0\\
501	0\\
502	0\\
503	0\\
504	0\\
505	0\\
506	0\\
507	0\\
508	0\\
509	0\\
510	0\\
511	0\\
512	0\\
513	0\\
514	0\\
515	0\\
516	0\\
517	0\\
518	0\\
519	0\\
520	0\\
521	0\\
522	0\\
523	0\\
524	0\\
525	0\\
526	0\\
527	0\\
528	0\\
529	0\\
530	0\\
531	0\\
532	0\\
533	0\\
534	0\\
535	0\\
536	0\\
537	0\\
538	0\\
539	0\\
540	0\\
541	2.25145755298818e-05\\
542	4.88271536578352e-05\\
543	7.57244088362082e-05\\
544	0.000103224216248849\\
545	0.000131344948204077\\
546	0.000160105482906932\\
547	0.000189525285251365\\
548	0.000219636184945385\\
549	0.00025046323928381\\
550	0.000282024413180058\\
551	0.000314338152631727\\
552	0.000347423417862681\\
553	0.000381299724883977\\
554	0.000415987720097952\\
555	0.000451508665813917\\
556	0.000487884480583933\\
557	0.000525184470262948\\
558	0.000563348240308322\\
559	0.000602398540839097\\
560	0.00064236432155272\\
561	0.00068325304476226\\
562	0.000725110760189988\\
563	0.000767980392286332\\
564	0.000812039238838757\\
565	0.000857143872510442\\
566	0.000903200365809114\\
567	0.000950252013351359\\
568	0.00099832220385339\\
569	0.0010475654496909\\
570	0.00109902998654938\\
571	0.00115110526262422\\
572	0.00120382222347614\\
573	0.00125713529321454\\
574	0.0013114218600033\\
575	0.00138416572299911\\
576	0.00168362914241537\\
577	0.00200121253368579\\
578	0.00225018105728763\\
579	0.00250573441948143\\
580	0.00263216434825482\\
581	0.00272775687042408\\
582	0.00281259168714533\\
583	0.00289684044792754\\
584	0.00298182678347091\\
585	0.00306822904379038\\
586	0.00315632033977421\\
587	0.00324628153905063\\
588	0.00333822811831753\\
589	0.00343235311897413\\
590	0.00352893796606579\\
591	0.00362860835901882\\
592	0.00373286830485375\\
593	0.00384555904940006\\
594	0.00397672044113206\\
595	0.00415280575412962\\
596	0.00444436306303141\\
597	0.00503983077166121\\
598	0.00644286460810295\\
599	0\\
600	0\\
};
\addplot [color=red!75!mycolor17,solid,forget plot]
  table[row sep=crcr]{%
1	0\\
2	0\\
3	0\\
4	0\\
5	0\\
6	0\\
7	0\\
8	0\\
9	0\\
10	0\\
11	0\\
12	0\\
13	0\\
14	0\\
15	0\\
16	0\\
17	0\\
18	0\\
19	0\\
20	0\\
21	0\\
22	0\\
23	0\\
24	0\\
25	0\\
26	0\\
27	0\\
28	0\\
29	0\\
30	0\\
31	0\\
32	0\\
33	0\\
34	0\\
35	0\\
36	0\\
37	0\\
38	0\\
39	0\\
40	0\\
41	0\\
42	0\\
43	0\\
44	0\\
45	0\\
46	0\\
47	0\\
48	0\\
49	0\\
50	0\\
51	0\\
52	0\\
53	0\\
54	0\\
55	0\\
56	0\\
57	0\\
58	0\\
59	0\\
60	0\\
61	0\\
62	0\\
63	0\\
64	0\\
65	0\\
66	0\\
67	0\\
68	0\\
69	0\\
70	0\\
71	0\\
72	0\\
73	0\\
74	0\\
75	0\\
76	0\\
77	0\\
78	0\\
79	0\\
80	0\\
81	0\\
82	0\\
83	0\\
84	0\\
85	0\\
86	0\\
87	0\\
88	0\\
89	0\\
90	0\\
91	0\\
92	0\\
93	0\\
94	0\\
95	0\\
96	0\\
97	0\\
98	0\\
99	0\\
100	0\\
101	0\\
102	0\\
103	0\\
104	0\\
105	0\\
106	0\\
107	0\\
108	0\\
109	0\\
110	0\\
111	0\\
112	0\\
113	0\\
114	0\\
115	0\\
116	0\\
117	0\\
118	0\\
119	0\\
120	0\\
121	0\\
122	0\\
123	0\\
124	0\\
125	0\\
126	0\\
127	0\\
128	0\\
129	0\\
130	0\\
131	0\\
132	0\\
133	0\\
134	0\\
135	0\\
136	0\\
137	0\\
138	0\\
139	0\\
140	0\\
141	0\\
142	0\\
143	0\\
144	0\\
145	0\\
146	0\\
147	0\\
148	0\\
149	0\\
150	0\\
151	0\\
152	0\\
153	0\\
154	0\\
155	0\\
156	0\\
157	0\\
158	0\\
159	0\\
160	0\\
161	0\\
162	0\\
163	0\\
164	0\\
165	0\\
166	0\\
167	0\\
168	0\\
169	0\\
170	0\\
171	0\\
172	0\\
173	0\\
174	0\\
175	0\\
176	0\\
177	0\\
178	0\\
179	0\\
180	0\\
181	0\\
182	0\\
183	0\\
184	0\\
185	0\\
186	0\\
187	0\\
188	0\\
189	0\\
190	0\\
191	0\\
192	0\\
193	0\\
194	0\\
195	0\\
196	0\\
197	0\\
198	0\\
199	0\\
200	0\\
201	0\\
202	0\\
203	0\\
204	0\\
205	0\\
206	0\\
207	0\\
208	0\\
209	0\\
210	0\\
211	0\\
212	0\\
213	0\\
214	0\\
215	0\\
216	0\\
217	0\\
218	0\\
219	0\\
220	0\\
221	0\\
222	0\\
223	0\\
224	0\\
225	0\\
226	0\\
227	0\\
228	0\\
229	0\\
230	0\\
231	0\\
232	0\\
233	0\\
234	0\\
235	0\\
236	0\\
237	0\\
238	0\\
239	0\\
240	0\\
241	0\\
242	0\\
243	0\\
244	0\\
245	0\\
246	0\\
247	0\\
248	0\\
249	0\\
250	0\\
251	0\\
252	0\\
253	0\\
254	0\\
255	0\\
256	0\\
257	0\\
258	0\\
259	0\\
260	0\\
261	0\\
262	0\\
263	0\\
264	0\\
265	0\\
266	0\\
267	0\\
268	0\\
269	0\\
270	0\\
271	0\\
272	0\\
273	0\\
274	0\\
275	0\\
276	0\\
277	0\\
278	0\\
279	0\\
280	0\\
281	0\\
282	0\\
283	0\\
284	0\\
285	0\\
286	0\\
287	0\\
288	0\\
289	0\\
290	0\\
291	0\\
292	0\\
293	0\\
294	0\\
295	0\\
296	0\\
297	0\\
298	0\\
299	0\\
300	0\\
301	0\\
302	0\\
303	0\\
304	0\\
305	0\\
306	0\\
307	0\\
308	0\\
309	0\\
310	0\\
311	0\\
312	0\\
313	0\\
314	0\\
315	0\\
316	0\\
317	0\\
318	0\\
319	0\\
320	0\\
321	0\\
322	0\\
323	0\\
324	0\\
325	0\\
326	0\\
327	0\\
328	0\\
329	0\\
330	0\\
331	0\\
332	0\\
333	0\\
334	0\\
335	0\\
336	0\\
337	0\\
338	0\\
339	0\\
340	0\\
341	0\\
342	0\\
343	0\\
344	0\\
345	0\\
346	0\\
347	0\\
348	0\\
349	0\\
350	0\\
351	0\\
352	0\\
353	0\\
354	0\\
355	0\\
356	0\\
357	0\\
358	0\\
359	0\\
360	0\\
361	0\\
362	0\\
363	0\\
364	0\\
365	0\\
366	0\\
367	0\\
368	0\\
369	0\\
370	0\\
371	0\\
372	0\\
373	0\\
374	0\\
375	0\\
376	0\\
377	0\\
378	0\\
379	0\\
380	0\\
381	0\\
382	0\\
383	0\\
384	0\\
385	0\\
386	0\\
387	0\\
388	0\\
389	0\\
390	0\\
391	0\\
392	0\\
393	0\\
394	0\\
395	0\\
396	0\\
397	0\\
398	0\\
399	0\\
400	0\\
401	0\\
402	0\\
403	0\\
404	0\\
405	0\\
406	0\\
407	0\\
408	0\\
409	0\\
410	0\\
411	0\\
412	0\\
413	0\\
414	0\\
415	0\\
416	0\\
417	0\\
418	0\\
419	0\\
420	0\\
421	0\\
422	0\\
423	0\\
424	0\\
425	0\\
426	0\\
427	0\\
428	0\\
429	0\\
430	0\\
431	0\\
432	0\\
433	0\\
434	0\\
435	0\\
436	0\\
437	0\\
438	0\\
439	0\\
440	0\\
441	0\\
442	0\\
443	0\\
444	0\\
445	0\\
446	0\\
447	0\\
448	0\\
449	0\\
450	0\\
451	0\\
452	0\\
453	0\\
454	0\\
455	0\\
456	0\\
457	0\\
458	0\\
459	0\\
460	0\\
461	0\\
462	0\\
463	0\\
464	0\\
465	0\\
466	0\\
467	0\\
468	0\\
469	0\\
470	0\\
471	0\\
472	0\\
473	0\\
474	0\\
475	0\\
476	0\\
477	0\\
478	0\\
479	0\\
480	0\\
481	0\\
482	0\\
483	0\\
484	0\\
485	0\\
486	0\\
487	0\\
488	0\\
489	0\\
490	0\\
491	0\\
492	0\\
493	0\\
494	0\\
495	0\\
496	0\\
497	0\\
498	0\\
499	0\\
500	0\\
501	0\\
502	0\\
503	0\\
504	0\\
505	0\\
506	0\\
507	0\\
508	0\\
509	0\\
510	0\\
511	0\\
512	0\\
513	0\\
514	0\\
515	0\\
516	0\\
517	0\\
518	0\\
519	0\\
520	0\\
521	0\\
522	0\\
523	0\\
524	0\\
525	0\\
526	0\\
527	0\\
528	0\\
529	0\\
530	0\\
531	0\\
532	0\\
533	0\\
534	0\\
535	0\\
536	0\\
537	0\\
538	0\\
539	0\\
540	1.80572611580464e-05\\
541	4.4118006200525e-05\\
542	7.07542032362198e-05\\
543	9.79828118109392e-05\\
544	0.000125821262619649\\
545	0.000154287471883814\\
546	0.000183399616108164\\
547	0.000213189224881146\\
548	0.000243680143284574\\
549	0.00027488924859386\\
550	0.000306834372255217\\
551	0.000339533902369266\\
552	0.00037300762948773\\
553	0.000407321246047958\\
554	0.000442407509995026\\
555	0.000478288527401575\\
556	0.000514990430784664\\
557	0.000552506247906185\\
558	0.000590897395376276\\
559	0.000630191158809039\\
560	0.000670413411890405\\
561	0.000711735510261396\\
562	0.00075401250105808\\
563	0.000797191808578931\\
564	0.000841224772035708\\
565	0.000886213838323151\\
566	0.000932258739848622\\
567	0.000979764676566103\\
568	0.00102876658617279\\
569	0.00107851644739995\\
570	0.00112845627858277\\
571	0.00117925389621404\\
572	0.00123071645413494\\
573	0.00128292972718838\\
574	0.00156765047686712\\
575	0.00189256888265377\\
576	0.00214409180946553\\
577	0.00239107643868339\\
578	0.00249057912042086\\
579	0.00257377822812136\\
580	0.00265393568636939\\
581	0.00273394370368479\\
582	0.00281520413723953\\
583	0.00289795905152283\\
584	0.00298235807749845\\
585	0.00306849821799598\\
586	0.00315646723333688\\
587	0.003246352526171\\
588	0.00333826624313612\\
589	0.00343236642273867\\
590	0.00352894054086423\\
591	0.00362860835901881\\
592	0.00373286830485375\\
593	0.00384555904940006\\
594	0.00397672044113205\\
595	0.00415280575412962\\
596	0.00444436306303141\\
597	0.00503983077166121\\
598	0.00644286460810295\\
599	0\\
600	0\\
};
\addplot [color=red!80!mycolor19,solid,forget plot]
  table[row sep=crcr]{%
1	0\\
2	0\\
3	0\\
4	0\\
5	0\\
6	0\\
7	0\\
8	0\\
9	0\\
10	0\\
11	0\\
12	0\\
13	0\\
14	0\\
15	0\\
16	0\\
17	0\\
18	0\\
19	0\\
20	0\\
21	0\\
22	0\\
23	0\\
24	0\\
25	0\\
26	0\\
27	0\\
28	0\\
29	0\\
30	0\\
31	0\\
32	0\\
33	0\\
34	0\\
35	0\\
36	0\\
37	0\\
38	0\\
39	0\\
40	0\\
41	0\\
42	0\\
43	0\\
44	0\\
45	0\\
46	0\\
47	0\\
48	0\\
49	0\\
50	0\\
51	0\\
52	0\\
53	0\\
54	0\\
55	0\\
56	0\\
57	0\\
58	0\\
59	0\\
60	0\\
61	0\\
62	0\\
63	0\\
64	0\\
65	0\\
66	0\\
67	0\\
68	0\\
69	0\\
70	0\\
71	0\\
72	0\\
73	0\\
74	0\\
75	0\\
76	0\\
77	0\\
78	0\\
79	0\\
80	0\\
81	0\\
82	0\\
83	0\\
84	0\\
85	0\\
86	0\\
87	0\\
88	0\\
89	0\\
90	0\\
91	0\\
92	0\\
93	0\\
94	0\\
95	0\\
96	0\\
97	0\\
98	0\\
99	0\\
100	0\\
101	0\\
102	0\\
103	0\\
104	0\\
105	0\\
106	0\\
107	0\\
108	0\\
109	0\\
110	0\\
111	0\\
112	0\\
113	0\\
114	0\\
115	0\\
116	0\\
117	0\\
118	0\\
119	0\\
120	0\\
121	0\\
122	0\\
123	0\\
124	0\\
125	0\\
126	0\\
127	0\\
128	0\\
129	0\\
130	0\\
131	0\\
132	0\\
133	0\\
134	0\\
135	0\\
136	0\\
137	0\\
138	0\\
139	0\\
140	0\\
141	0\\
142	0\\
143	0\\
144	0\\
145	0\\
146	0\\
147	0\\
148	0\\
149	0\\
150	0\\
151	0\\
152	0\\
153	0\\
154	0\\
155	0\\
156	0\\
157	0\\
158	0\\
159	0\\
160	0\\
161	0\\
162	0\\
163	0\\
164	0\\
165	0\\
166	0\\
167	0\\
168	0\\
169	0\\
170	0\\
171	0\\
172	0\\
173	0\\
174	0\\
175	0\\
176	0\\
177	0\\
178	0\\
179	0\\
180	0\\
181	0\\
182	0\\
183	0\\
184	0\\
185	0\\
186	0\\
187	0\\
188	0\\
189	0\\
190	0\\
191	0\\
192	0\\
193	0\\
194	0\\
195	0\\
196	0\\
197	0\\
198	0\\
199	0\\
200	0\\
201	0\\
202	0\\
203	0\\
204	0\\
205	0\\
206	0\\
207	0\\
208	0\\
209	0\\
210	0\\
211	0\\
212	0\\
213	0\\
214	0\\
215	0\\
216	0\\
217	0\\
218	0\\
219	0\\
220	0\\
221	0\\
222	0\\
223	0\\
224	0\\
225	0\\
226	0\\
227	0\\
228	0\\
229	0\\
230	0\\
231	0\\
232	0\\
233	0\\
234	0\\
235	0\\
236	0\\
237	0\\
238	0\\
239	0\\
240	0\\
241	0\\
242	0\\
243	0\\
244	0\\
245	0\\
246	0\\
247	0\\
248	0\\
249	0\\
250	0\\
251	0\\
252	0\\
253	0\\
254	0\\
255	0\\
256	0\\
257	0\\
258	0\\
259	0\\
260	0\\
261	0\\
262	0\\
263	0\\
264	0\\
265	0\\
266	0\\
267	0\\
268	0\\
269	0\\
270	0\\
271	0\\
272	0\\
273	0\\
274	0\\
275	0\\
276	0\\
277	0\\
278	0\\
279	0\\
280	0\\
281	0\\
282	0\\
283	0\\
284	0\\
285	0\\
286	0\\
287	0\\
288	0\\
289	0\\
290	0\\
291	0\\
292	0\\
293	0\\
294	0\\
295	0\\
296	0\\
297	0\\
298	0\\
299	0\\
300	0\\
301	0\\
302	0\\
303	0\\
304	0\\
305	0\\
306	0\\
307	0\\
308	0\\
309	0\\
310	0\\
311	0\\
312	0\\
313	0\\
314	0\\
315	0\\
316	0\\
317	0\\
318	0\\
319	0\\
320	0\\
321	0\\
322	0\\
323	0\\
324	0\\
325	0\\
326	0\\
327	0\\
328	0\\
329	0\\
330	0\\
331	0\\
332	0\\
333	0\\
334	0\\
335	0\\
336	0\\
337	0\\
338	0\\
339	0\\
340	0\\
341	0\\
342	0\\
343	0\\
344	0\\
345	0\\
346	0\\
347	0\\
348	0\\
349	0\\
350	0\\
351	0\\
352	0\\
353	0\\
354	0\\
355	0\\
356	0\\
357	0\\
358	0\\
359	0\\
360	0\\
361	0\\
362	0\\
363	0\\
364	0\\
365	0\\
366	0\\
367	0\\
368	0\\
369	0\\
370	0\\
371	0\\
372	0\\
373	0\\
374	0\\
375	0\\
376	0\\
377	0\\
378	0\\
379	0\\
380	0\\
381	0\\
382	0\\
383	0\\
384	0\\
385	0\\
386	0\\
387	0\\
388	0\\
389	0\\
390	0\\
391	0\\
392	0\\
393	0\\
394	0\\
395	0\\
396	0\\
397	0\\
398	0\\
399	0\\
400	0\\
401	0\\
402	0\\
403	0\\
404	0\\
405	0\\
406	0\\
407	0\\
408	0\\
409	0\\
410	0\\
411	0\\
412	0\\
413	0\\
414	0\\
415	0\\
416	0\\
417	0\\
418	0\\
419	0\\
420	0\\
421	0\\
422	0\\
423	0\\
424	0\\
425	0\\
426	0\\
427	0\\
428	0\\
429	0\\
430	0\\
431	0\\
432	0\\
433	0\\
434	0\\
435	0\\
436	0\\
437	0\\
438	0\\
439	0\\
440	0\\
441	0\\
442	0\\
443	0\\
444	0\\
445	0\\
446	0\\
447	0\\
448	0\\
449	0\\
450	0\\
451	0\\
452	0\\
453	0\\
454	0\\
455	0\\
456	0\\
457	0\\
458	0\\
459	0\\
460	0\\
461	0\\
462	0\\
463	0\\
464	0\\
465	0\\
466	0\\
467	0\\
468	0\\
469	0\\
470	0\\
471	0\\
472	0\\
473	0\\
474	0\\
475	0\\
476	0\\
477	0\\
478	0\\
479	0\\
480	0\\
481	0\\
482	0\\
483	0\\
484	0\\
485	0\\
486	0\\
487	0\\
488	0\\
489	0\\
490	0\\
491	0\\
492	0\\
493	0\\
494	0\\
495	0\\
496	0\\
497	0\\
498	0\\
499	0\\
500	0\\
501	0\\
502	0\\
503	0\\
504	0\\
505	0\\
506	0\\
507	0\\
508	0\\
509	0\\
510	0\\
511	0\\
512	0\\
513	0\\
514	0\\
515	0\\
516	0\\
517	0\\
518	0\\
519	0\\
520	0\\
521	0\\
522	0\\
523	0\\
524	0\\
525	0\\
526	0\\
527	0\\
528	0\\
529	0\\
530	0\\
531	0\\
532	0\\
533	0\\
534	0\\
535	0\\
536	0\\
537	0\\
538	0\\
539	1.09971123564833e-05\\
540	3.67777687958609e-05\\
541	6.31213625684237e-05\\
542	9.00441133591266e-05\\
543	0.00011756267562623\\
544	0.000145694153413752\\
545	0.000174455739705744\\
546	0.000203878155683555\\
547	0.00023398579118046\\
548	0.000264795179737243\\
549	0.000296366118124326\\
550	0.000328644365716624\\
551	0.000361637574244406\\
552	0.000395368372265394\\
553	0.000429826296342454\\
554	0.000465072355643675\\
555	0.00050112828782099\\
556	0.000538014176942687\\
557	0.00057573329173453\\
558	0.000614484468182088\\
559	0.000654187719273962\\
560	0.000694706083181759\\
561	0.00073598111257611\\
562	0.00077813831988304\\
563	0.000821236596718493\\
564	0.000865236156171182\\
565	0.000910969534659842\\
566	0.000957920024337713\\
567	0.00100529101218676\\
568	0.0010530258921727\\
569	0.00110148835858687\\
570	0.00115017727882363\\
571	0.00120010163547213\\
572	0.00140776483562387\\
573	0.00174519063704481\\
574	0.00200611363934863\\
575	0.00225233627854361\\
576	0.00234714808882544\\
577	0.00242410810580428\\
578	0.00250007651960148\\
579	0.00257684337440823\\
580	0.00265485410432229\\
581	0.00273433854836808\\
582	0.00281537804625545\\
583	0.0028980434054312\\
584	0.00298240150095583\\
585	0.00306852173201496\\
586	0.00315647873661191\\
587	0.00324635845014596\\
588	0.00333826825403972\\
589	0.00343236680134408\\
590	0.00352894054086423\\
591	0.00362860835901881\\
592	0.00373286830485375\\
593	0.00384555904940006\\
594	0.00397672044113206\\
595	0.00415280575412962\\
596	0.00444436306303141\\
597	0.00503983077166121\\
598	0.00644286460810295\\
599	0\\
600	0\\
};
\addplot [color=red,solid,forget plot]
  table[row sep=crcr]{%
1	0\\
2	0\\
3	0\\
4	0\\
5	0\\
6	0\\
7	0\\
8	0\\
9	0\\
10	0\\
11	0\\
12	0\\
13	0\\
14	0\\
15	0\\
16	0\\
17	0\\
18	0\\
19	0\\
20	0\\
21	0\\
22	0\\
23	0\\
24	0\\
25	0\\
26	0\\
27	0\\
28	0\\
29	0\\
30	0\\
31	0\\
32	0\\
33	0\\
34	0\\
35	0\\
36	0\\
37	0\\
38	0\\
39	0\\
40	0\\
41	0\\
42	0\\
43	0\\
44	0\\
45	0\\
46	0\\
47	0\\
48	0\\
49	0\\
50	0\\
51	0\\
52	0\\
53	0\\
54	0\\
55	0\\
56	0\\
57	0\\
58	0\\
59	0\\
60	0\\
61	0\\
62	0\\
63	0\\
64	0\\
65	0\\
66	0\\
67	0\\
68	0\\
69	0\\
70	0\\
71	0\\
72	0\\
73	0\\
74	0\\
75	0\\
76	0\\
77	0\\
78	0\\
79	0\\
80	0\\
81	0\\
82	0\\
83	0\\
84	0\\
85	0\\
86	0\\
87	0\\
88	0\\
89	0\\
90	0\\
91	0\\
92	0\\
93	0\\
94	0\\
95	0\\
96	0\\
97	0\\
98	0\\
99	0\\
100	0\\
101	0\\
102	0\\
103	0\\
104	0\\
105	0\\
106	0\\
107	0\\
108	0\\
109	0\\
110	0\\
111	0\\
112	0\\
113	0\\
114	0\\
115	0\\
116	0\\
117	0\\
118	0\\
119	0\\
120	0\\
121	0\\
122	0\\
123	0\\
124	0\\
125	0\\
126	0\\
127	0\\
128	0\\
129	0\\
130	0\\
131	0\\
132	0\\
133	0\\
134	0\\
135	0\\
136	0\\
137	0\\
138	0\\
139	0\\
140	0\\
141	0\\
142	0\\
143	0\\
144	0\\
145	0\\
146	0\\
147	0\\
148	0\\
149	0\\
150	0\\
151	0\\
152	0\\
153	0\\
154	0\\
155	0\\
156	0\\
157	0\\
158	0\\
159	0\\
160	0\\
161	0\\
162	0\\
163	0\\
164	0\\
165	0\\
166	0\\
167	0\\
168	0\\
169	0\\
170	0\\
171	0\\
172	0\\
173	0\\
174	0\\
175	0\\
176	0\\
177	0\\
178	0\\
179	0\\
180	0\\
181	0\\
182	0\\
183	0\\
184	0\\
185	0\\
186	0\\
187	0\\
188	0\\
189	0\\
190	0\\
191	0\\
192	0\\
193	0\\
194	0\\
195	0\\
196	0\\
197	0\\
198	0\\
199	0\\
200	0\\
201	0\\
202	0\\
203	0\\
204	0\\
205	0\\
206	0\\
207	0\\
208	0\\
209	0\\
210	0\\
211	0\\
212	0\\
213	0\\
214	0\\
215	0\\
216	0\\
217	0\\
218	0\\
219	0\\
220	0\\
221	0\\
222	0\\
223	0\\
224	0\\
225	0\\
226	0\\
227	0\\
228	0\\
229	0\\
230	0\\
231	0\\
232	0\\
233	0\\
234	0\\
235	0\\
236	0\\
237	0\\
238	0\\
239	0\\
240	0\\
241	0\\
242	0\\
243	0\\
244	0\\
245	0\\
246	0\\
247	0\\
248	0\\
249	0\\
250	0\\
251	0\\
252	0\\
253	0\\
254	0\\
255	0\\
256	0\\
257	0\\
258	0\\
259	0\\
260	0\\
261	0\\
262	0\\
263	0\\
264	0\\
265	0\\
266	0\\
267	0\\
268	0\\
269	0\\
270	0\\
271	0\\
272	0\\
273	0\\
274	0\\
275	0\\
276	0\\
277	0\\
278	0\\
279	0\\
280	0\\
281	0\\
282	0\\
283	0\\
284	0\\
285	0\\
286	0\\
287	0\\
288	0\\
289	0\\
290	0\\
291	0\\
292	0\\
293	0\\
294	0\\
295	0\\
296	0\\
297	0\\
298	0\\
299	0\\
300	0\\
301	0\\
302	0\\
303	0\\
304	0\\
305	0\\
306	0\\
307	0\\
308	0\\
309	0\\
310	0\\
311	0\\
312	0\\
313	0\\
314	0\\
315	0\\
316	0\\
317	0\\
318	0\\
319	0\\
320	0\\
321	0\\
322	0\\
323	0\\
324	0\\
325	0\\
326	0\\
327	0\\
328	0\\
329	0\\
330	0\\
331	0\\
332	0\\
333	0\\
334	0\\
335	0\\
336	0\\
337	0\\
338	0\\
339	0\\
340	0\\
341	0\\
342	0\\
343	0\\
344	0\\
345	0\\
346	0\\
347	0\\
348	0\\
349	0\\
350	0\\
351	0\\
352	0\\
353	0\\
354	0\\
355	0\\
356	0\\
357	0\\
358	0\\
359	0\\
360	0\\
361	0\\
362	0\\
363	0\\
364	0\\
365	0\\
366	0\\
367	0\\
368	0\\
369	0\\
370	0\\
371	0\\
372	0\\
373	0\\
374	0\\
375	0\\
376	0\\
377	0\\
378	0\\
379	0\\
380	0\\
381	0\\
382	0\\
383	0\\
384	0\\
385	0\\
386	0\\
387	0\\
388	0\\
389	0\\
390	0\\
391	0\\
392	0\\
393	0\\
394	0\\
395	0\\
396	0\\
397	0\\
398	0\\
399	0\\
400	0\\
401	0\\
402	0\\
403	0\\
404	0\\
405	0\\
406	0\\
407	0\\
408	0\\
409	0\\
410	0\\
411	0\\
412	0\\
413	0\\
414	0\\
415	0\\
416	0\\
417	0\\
418	0\\
419	0\\
420	0\\
421	0\\
422	0\\
423	0\\
424	0\\
425	0\\
426	0\\
427	0\\
428	0\\
429	0\\
430	0\\
431	0\\
432	0\\
433	0\\
434	0\\
435	0\\
436	0\\
437	0\\
438	0\\
439	0\\
440	0\\
441	0\\
442	0\\
443	0\\
444	0\\
445	0\\
446	0\\
447	0\\
448	0\\
449	0\\
450	0\\
451	0\\
452	0\\
453	0\\
454	0\\
455	0\\
456	0\\
457	0\\
458	0\\
459	0\\
460	0\\
461	0\\
462	0\\
463	0\\
464	0\\
465	0\\
466	0\\
467	0\\
468	0\\
469	0\\
470	0\\
471	0\\
472	0\\
473	0\\
474	0\\
475	0\\
476	0\\
477	0\\
478	0\\
479	0\\
480	0\\
481	0\\
482	0\\
483	0\\
484	0\\
485	0\\
486	0\\
487	0\\
488	0\\
489	0\\
490	0\\
491	0\\
492	0\\
493	0\\
494	0\\
495	0\\
496	0\\
497	0\\
498	0\\
499	0\\
500	0\\
501	0\\
502	0\\
503	0\\
504	0\\
505	0\\
506	0\\
507	0\\
508	0\\
509	0\\
510	0\\
511	0\\
512	0\\
513	0\\
514	0\\
515	0\\
516	0\\
517	0\\
518	0\\
519	0\\
520	0\\
521	0\\
522	0\\
523	0\\
524	0\\
525	0\\
526	0\\
527	0\\
528	0\\
529	0\\
530	0\\
531	0\\
532	0\\
533	0\\
534	0\\
535	0\\
536	0\\
537	0\\
538	1.87556343843139e-06\\
539	2.73381310324152e-05\\
540	5.33495288270524e-05\\
541	7.99252542932924e-05\\
542	0.000107081251424983\\
543	0.000134834453286783\\
544	0.000163201833838333\\
545	0.000192240295758037\\
546	0.000221947411190644\\
547	0.000252298954717094\\
548	0.000283315360389555\\
549	0.000314986488951994\\
550	0.000347362970500003\\
551	0.000380469167030157\\
552	0.000414321565112513\\
553	0.000448919563346855\\
554	0.00048433294388368\\
555	0.000520634170094015\\
556	0.000557948080283569\\
557	0.000596000691589419\\
558	0.000634749598647108\\
559	0.000674269636362017\\
560	0.000714637530630376\\
561	0.000755805415251623\\
562	0.000797923053763769\\
563	0.000841878133127252\\
564	0.000886913723009165\\
565	0.00093214151302975\\
566	0.000977757767295516\\
567	0.00102380313004\\
568	0.00107036709861644\\
569	0.00111805503001417\\
570	0.00120939113958143\\
571	0.00154663998556902\\
572	0.00183959615121693\\
573	0.00209756278158636\\
574	0.00220493769069792\\
575	0.00227885628634348\\
576	0.00235163153275207\\
577	0.00242542435592874\\
578	0.00250050669328025\\
579	0.00257697908889624\\
580	0.00265491355281054\\
581	0.00273436544093214\\
582	0.00281539134096267\\
583	0.00289805033361497\\
584	0.00298240522740585\\
585	0.00306852356540592\\
586	0.00315647964482259\\
587	0.0032463587502771\\
588	0.0033382683090296\\
589	0.00343236680134408\\
590	0.00352894054086422\\
591	0.00362860835901881\\
592	0.00373286830485375\\
593	0.00384555904940006\\
594	0.00397672044113206\\
595	0.00415280575412962\\
596	0.00444436306303141\\
597	0.00503983077166121\\
598	0.00644286460810295\\
599	0\\
600	0\\
};
\addplot [color=mycolor20,solid,forget plot]
  table[row sep=crcr]{%
1	0\\
2	0\\
3	0\\
4	0\\
5	0\\
6	0\\
7	0\\
8	0\\
9	0\\
10	0\\
11	0\\
12	0\\
13	0\\
14	0\\
15	0\\
16	0\\
17	0\\
18	0\\
19	0\\
20	0\\
21	0\\
22	0\\
23	0\\
24	0\\
25	0\\
26	0\\
27	0\\
28	0\\
29	0\\
30	0\\
31	0\\
32	0\\
33	0\\
34	0\\
35	0\\
36	0\\
37	0\\
38	0\\
39	0\\
40	0\\
41	0\\
42	0\\
43	0\\
44	0\\
45	0\\
46	0\\
47	0\\
48	0\\
49	0\\
50	0\\
51	0\\
52	0\\
53	0\\
54	0\\
55	0\\
56	0\\
57	0\\
58	0\\
59	0\\
60	0\\
61	0\\
62	0\\
63	0\\
64	0\\
65	0\\
66	0\\
67	0\\
68	0\\
69	0\\
70	0\\
71	0\\
72	0\\
73	0\\
74	0\\
75	0\\
76	0\\
77	0\\
78	0\\
79	0\\
80	0\\
81	0\\
82	0\\
83	0\\
84	0\\
85	0\\
86	0\\
87	0\\
88	0\\
89	0\\
90	0\\
91	0\\
92	0\\
93	0\\
94	0\\
95	0\\
96	0\\
97	0\\
98	0\\
99	0\\
100	0\\
101	0\\
102	0\\
103	0\\
104	0\\
105	0\\
106	0\\
107	0\\
108	0\\
109	0\\
110	0\\
111	0\\
112	0\\
113	0\\
114	0\\
115	0\\
116	0\\
117	0\\
118	0\\
119	0\\
120	0\\
121	0\\
122	0\\
123	0\\
124	0\\
125	0\\
126	0\\
127	0\\
128	0\\
129	0\\
130	0\\
131	0\\
132	0\\
133	0\\
134	0\\
135	0\\
136	0\\
137	0\\
138	0\\
139	0\\
140	0\\
141	0\\
142	0\\
143	0\\
144	0\\
145	0\\
146	0\\
147	0\\
148	0\\
149	0\\
150	0\\
151	0\\
152	0\\
153	0\\
154	0\\
155	0\\
156	0\\
157	0\\
158	0\\
159	0\\
160	0\\
161	0\\
162	0\\
163	0\\
164	0\\
165	0\\
166	0\\
167	0\\
168	0\\
169	0\\
170	0\\
171	0\\
172	0\\
173	0\\
174	0\\
175	0\\
176	0\\
177	0\\
178	0\\
179	0\\
180	0\\
181	0\\
182	0\\
183	0\\
184	0\\
185	0\\
186	0\\
187	0\\
188	0\\
189	0\\
190	0\\
191	0\\
192	0\\
193	0\\
194	0\\
195	0\\
196	0\\
197	0\\
198	0\\
199	0\\
200	0\\
201	0\\
202	0\\
203	0\\
204	0\\
205	0\\
206	0\\
207	0\\
208	0\\
209	0\\
210	0\\
211	0\\
212	0\\
213	0\\
214	0\\
215	0\\
216	0\\
217	0\\
218	0\\
219	0\\
220	0\\
221	0\\
222	0\\
223	0\\
224	0\\
225	0\\
226	0\\
227	0\\
228	0\\
229	0\\
230	0\\
231	0\\
232	0\\
233	0\\
234	0\\
235	0\\
236	0\\
237	0\\
238	0\\
239	0\\
240	0\\
241	0\\
242	0\\
243	0\\
244	0\\
245	0\\
246	0\\
247	0\\
248	0\\
249	0\\
250	0\\
251	0\\
252	0\\
253	0\\
254	0\\
255	0\\
256	0\\
257	0\\
258	0\\
259	0\\
260	0\\
261	0\\
262	0\\
263	0\\
264	0\\
265	0\\
266	0\\
267	0\\
268	0\\
269	0\\
270	0\\
271	0\\
272	0\\
273	0\\
274	0\\
275	0\\
276	0\\
277	0\\
278	0\\
279	0\\
280	0\\
281	0\\
282	0\\
283	0\\
284	0\\
285	0\\
286	0\\
287	0\\
288	0\\
289	0\\
290	0\\
291	0\\
292	0\\
293	0\\
294	0\\
295	0\\
296	0\\
297	0\\
298	0\\
299	0\\
300	0\\
301	0\\
302	0\\
303	0\\
304	0\\
305	0\\
306	0\\
307	0\\
308	0\\
309	0\\
310	0\\
311	0\\
312	0\\
313	0\\
314	0\\
315	0\\
316	0\\
317	0\\
318	0\\
319	0\\
320	0\\
321	0\\
322	0\\
323	0\\
324	0\\
325	0\\
326	0\\
327	0\\
328	0\\
329	0\\
330	0\\
331	0\\
332	0\\
333	0\\
334	0\\
335	0\\
336	0\\
337	0\\
338	0\\
339	0\\
340	0\\
341	0\\
342	0\\
343	0\\
344	0\\
345	0\\
346	0\\
347	0\\
348	0\\
349	0\\
350	0\\
351	0\\
352	0\\
353	0\\
354	0\\
355	0\\
356	0\\
357	0\\
358	0\\
359	0\\
360	0\\
361	0\\
362	0\\
363	0\\
364	0\\
365	0\\
366	0\\
367	0\\
368	0\\
369	0\\
370	0\\
371	0\\
372	0\\
373	0\\
374	0\\
375	0\\
376	0\\
377	0\\
378	0\\
379	0\\
380	0\\
381	0\\
382	0\\
383	0\\
384	0\\
385	0\\
386	0\\
387	0\\
388	0\\
389	0\\
390	0\\
391	0\\
392	0\\
393	0\\
394	0\\
395	0\\
396	0\\
397	0\\
398	0\\
399	0\\
400	0\\
401	0\\
402	0\\
403	0\\
404	0\\
405	0\\
406	0\\
407	0\\
408	0\\
409	0\\
410	0\\
411	0\\
412	0\\
413	0\\
414	0\\
415	0\\
416	0\\
417	0\\
418	0\\
419	0\\
420	0\\
421	0\\
422	0\\
423	0\\
424	0\\
425	0\\
426	0\\
427	0\\
428	0\\
429	0\\
430	0\\
431	0\\
432	0\\
433	0\\
434	0\\
435	0\\
436	0\\
437	0\\
438	0\\
439	0\\
440	0\\
441	0\\
442	0\\
443	0\\
444	0\\
445	0\\
446	0\\
447	0\\
448	0\\
449	0\\
450	0\\
451	0\\
452	0\\
453	0\\
454	0\\
455	0\\
456	0\\
457	0\\
458	0\\
459	0\\
460	0\\
461	0\\
462	0\\
463	0\\
464	0\\
465	0\\
466	0\\
467	0\\
468	0\\
469	0\\
470	0\\
471	0\\
472	0\\
473	0\\
474	0\\
475	0\\
476	0\\
477	0\\
478	0\\
479	0\\
480	0\\
481	0\\
482	0\\
483	0\\
484	0\\
485	0\\
486	0\\
487	0\\
488	0\\
489	0\\
490	0\\
491	0\\
492	0\\
493	0\\
494	0\\
495	0\\
496	0\\
497	0\\
498	0\\
499	0\\
500	0\\
501	0\\
502	0\\
503	0\\
504	0\\
505	0\\
506	0\\
507	0\\
508	0\\
509	0\\
510	0\\
511	0\\
512	0\\
513	0\\
514	0\\
515	0\\
516	0\\
517	0\\
518	0\\
519	0\\
520	0\\
521	0\\
522	0\\
523	0\\
524	0\\
525	0\\
526	0\\
527	0\\
528	0\\
529	0\\
530	0\\
531	0\\
532	0\\
533	0\\
534	0\\
535	0\\
536	0\\
537	0\\
538	1.6243643999888e-05\\
539	4.18918417134826e-05\\
540	6.80899130433921e-05\\
541	9.48564491413589e-05\\
542	0.000122248418487188\\
543	0.000150198053260629\\
544	0.000178739275558913\\
545	0.000207878949442831\\
546	0.000237642445840086\\
547	0.000268066066433872\\
548	0.000299164768053091\\
549	0.000330937123302737\\
550	0.000363440569240543\\
551	0.000396699448537722\\
552	0.000430733142917223\\
553	0.000465742743696337\\
554	0.000501529298393566\\
555	0.000537988092118647\\
556	0.000575079712924593\\
557	0.000612943047238721\\
558	0.000651541350800113\\
559	0.000690963850311138\\
560	0.000731349768291028\\
561	0.000773468762908814\\
562	0.000816756500267696\\
563	0.000860107264034101\\
564	0.000903803885929959\\
565	0.000947716400417519\\
566	0.000992275947024247\\
567	0.00103793439844082\\
568	0.00108481900650592\\
569	0.00130432755340724\\
570	0.00164623612034606\\
571	0.00190410168763289\\
572	0.00206499561851727\\
573	0.00213820104703062\\
574	0.00220835825523849\\
575	0.00227946097953584\\
576	0.00235181287951247\\
577	0.00242548455611651\\
578	0.0025005266540219\\
579	0.00257698800197174\\
580	0.00265491768689218\\
581	0.0027343675196084\\
582	0.00281539243421504\\
583	0.00289805091793385\\
584	0.00298240551503779\\
585	0.00306852370283625\\
586	0.00315647968906845\\
587	0.00324635875816935\\
588	0.00333826830902959\\
589	0.00343236680134408\\
590	0.00352894054086422\\
591	0.00362860835901881\\
592	0.00373286830485375\\
593	0.00384555904940006\\
594	0.00397672044113206\\
595	0.00415280575412961\\
596	0.00444436306303141\\
597	0.00503983077166121\\
598	0.00644286460810295\\
599	0\\
600	0\\
};
\addplot [color=mycolor21,solid,forget plot]
  table[row sep=crcr]{%
1	0\\
2	0\\
3	0\\
4	0\\
5	0\\
6	0\\
7	0\\
8	0\\
9	0\\
10	0\\
11	0\\
12	0\\
13	0\\
14	0\\
15	0\\
16	0\\
17	0\\
18	0\\
19	0\\
20	0\\
21	0\\
22	0\\
23	0\\
24	0\\
25	0\\
26	0\\
27	0\\
28	0\\
29	0\\
30	0\\
31	0\\
32	0\\
33	0\\
34	0\\
35	0\\
36	0\\
37	0\\
38	0\\
39	0\\
40	0\\
41	0\\
42	0\\
43	0\\
44	0\\
45	0\\
46	0\\
47	0\\
48	0\\
49	0\\
50	0\\
51	0\\
52	0\\
53	0\\
54	0\\
55	0\\
56	0\\
57	0\\
58	0\\
59	0\\
60	0\\
61	0\\
62	0\\
63	0\\
64	0\\
65	0\\
66	0\\
67	0\\
68	0\\
69	0\\
70	0\\
71	0\\
72	0\\
73	0\\
74	0\\
75	0\\
76	0\\
77	0\\
78	0\\
79	0\\
80	0\\
81	0\\
82	0\\
83	0\\
84	0\\
85	0\\
86	0\\
87	0\\
88	0\\
89	0\\
90	0\\
91	0\\
92	0\\
93	0\\
94	0\\
95	0\\
96	0\\
97	0\\
98	0\\
99	0\\
100	0\\
101	0\\
102	0\\
103	0\\
104	0\\
105	0\\
106	0\\
107	0\\
108	0\\
109	0\\
110	0\\
111	0\\
112	0\\
113	0\\
114	0\\
115	0\\
116	0\\
117	0\\
118	0\\
119	0\\
120	0\\
121	0\\
122	0\\
123	0\\
124	0\\
125	0\\
126	0\\
127	0\\
128	0\\
129	0\\
130	0\\
131	0\\
132	0\\
133	0\\
134	0\\
135	0\\
136	0\\
137	0\\
138	0\\
139	0\\
140	0\\
141	0\\
142	0\\
143	0\\
144	0\\
145	0\\
146	0\\
147	0\\
148	0\\
149	0\\
150	0\\
151	0\\
152	0\\
153	0\\
154	0\\
155	0\\
156	0\\
157	0\\
158	0\\
159	0\\
160	0\\
161	0\\
162	0\\
163	0\\
164	0\\
165	0\\
166	0\\
167	0\\
168	0\\
169	0\\
170	0\\
171	0\\
172	0\\
173	0\\
174	0\\
175	0\\
176	0\\
177	0\\
178	0\\
179	0\\
180	0\\
181	0\\
182	0\\
183	0\\
184	0\\
185	0\\
186	0\\
187	0\\
188	0\\
189	0\\
190	0\\
191	0\\
192	0\\
193	0\\
194	0\\
195	0\\
196	0\\
197	0\\
198	0\\
199	0\\
200	0\\
201	0\\
202	0\\
203	0\\
204	0\\
205	0\\
206	0\\
207	0\\
208	0\\
209	0\\
210	0\\
211	0\\
212	0\\
213	0\\
214	0\\
215	0\\
216	0\\
217	0\\
218	0\\
219	0\\
220	0\\
221	0\\
222	0\\
223	0\\
224	0\\
225	0\\
226	0\\
227	0\\
228	0\\
229	0\\
230	0\\
231	0\\
232	0\\
233	0\\
234	0\\
235	0\\
236	0\\
237	0\\
238	0\\
239	0\\
240	0\\
241	0\\
242	0\\
243	0\\
244	0\\
245	0\\
246	0\\
247	0\\
248	0\\
249	0\\
250	0\\
251	0\\
252	0\\
253	0\\
254	0\\
255	0\\
256	0\\
257	0\\
258	0\\
259	0\\
260	0\\
261	0\\
262	0\\
263	0\\
264	0\\
265	0\\
266	0\\
267	0\\
268	0\\
269	0\\
270	0\\
271	0\\
272	0\\
273	0\\
274	0\\
275	0\\
276	0\\
277	0\\
278	0\\
279	0\\
280	0\\
281	0\\
282	0\\
283	0\\
284	0\\
285	0\\
286	0\\
287	0\\
288	0\\
289	0\\
290	0\\
291	0\\
292	0\\
293	0\\
294	0\\
295	0\\
296	0\\
297	0\\
298	0\\
299	0\\
300	0\\
301	0\\
302	0\\
303	0\\
304	0\\
305	0\\
306	0\\
307	0\\
308	0\\
309	0\\
310	0\\
311	0\\
312	0\\
313	0\\
314	0\\
315	0\\
316	0\\
317	0\\
318	0\\
319	0\\
320	0\\
321	0\\
322	0\\
323	0\\
324	0\\
325	0\\
326	0\\
327	0\\
328	0\\
329	0\\
330	0\\
331	0\\
332	0\\
333	0\\
334	0\\
335	0\\
336	0\\
337	0\\
338	0\\
339	0\\
340	0\\
341	0\\
342	0\\
343	0\\
344	0\\
345	0\\
346	0\\
347	0\\
348	0\\
349	0\\
350	0\\
351	0\\
352	0\\
353	0\\
354	0\\
355	0\\
356	0\\
357	0\\
358	0\\
359	0\\
360	0\\
361	0\\
362	0\\
363	0\\
364	0\\
365	0\\
366	0\\
367	0\\
368	0\\
369	0\\
370	0\\
371	0\\
372	0\\
373	0\\
374	0\\
375	0\\
376	0\\
377	0\\
378	0\\
379	0\\
380	0\\
381	0\\
382	0\\
383	0\\
384	0\\
385	0\\
386	0\\
387	0\\
388	0\\
389	0\\
390	0\\
391	0\\
392	0\\
393	0\\
394	0\\
395	0\\
396	0\\
397	0\\
398	0\\
399	0\\
400	0\\
401	0\\
402	0\\
403	0\\
404	0\\
405	0\\
406	0\\
407	0\\
408	0\\
409	0\\
410	0\\
411	0\\
412	0\\
413	0\\
414	0\\
415	0\\
416	0\\
417	0\\
418	0\\
419	0\\
420	0\\
421	0\\
422	0\\
423	0\\
424	0\\
425	0\\
426	0\\
427	0\\
428	0\\
429	0\\
430	0\\
431	0\\
432	0\\
433	0\\
434	0\\
435	0\\
436	0\\
437	0\\
438	0\\
439	0\\
440	0\\
441	0\\
442	0\\
443	0\\
444	0\\
445	0\\
446	0\\
447	0\\
448	0\\
449	0\\
450	0\\
451	0\\
452	0\\
453	0\\
454	0\\
455	0\\
456	0\\
457	0\\
458	0\\
459	0\\
460	0\\
461	0\\
462	0\\
463	0\\
464	0\\
465	0\\
466	0\\
467	0\\
468	0\\
469	0\\
470	0\\
471	0\\
472	0\\
473	0\\
474	0\\
475	0\\
476	0\\
477	0\\
478	0\\
479	0\\
480	0\\
481	0\\
482	0\\
483	0\\
484	0\\
485	0\\
486	0\\
487	0\\
488	0\\
489	0\\
490	0\\
491	0\\
492	0\\
493	0\\
494	0\\
495	0\\
496	0\\
497	0\\
498	0\\
499	0\\
500	0\\
501	0\\
502	0\\
503	0\\
504	0\\
505	0\\
506	0\\
507	0\\
508	0\\
509	0\\
510	0\\
511	0\\
512	0\\
513	0\\
514	0\\
515	0\\
516	0\\
517	0\\
518	0\\
519	0\\
520	0\\
521	0\\
522	0\\
523	0\\
524	0\\
525	0\\
526	0\\
527	0\\
528	0\\
529	0\\
530	0\\
531	0\\
532	0\\
533	0\\
534	0\\
535	0\\
536	0\\
537	3.82550301366919e-06\\
538	2.9106131297474e-05\\
539	5.49338526053371e-05\\
540	8.12709214509474e-05\\
541	0.000108136282402065\\
542	0.000135519390228108\\
543	0.000163492059064573\\
544	0.000192077027657505\\
545	0.000221277232734906\\
546	0.000251125248797619\\
547	0.00028165907332523\\
548	0.000312894077851403\\
549	0.000344852457097526\\
550	0.000377575892313951\\
551	0.000411272352801193\\
552	0.000445600411375425\\
553	0.000480493797716272\\
554	0.000516041013830302\\
555	0.000552284502233579\\
556	0.000589200581950451\\
557	0.000626998320133113\\
558	0.000665716783444826\\
559	0.000705887875314932\\
560	0.000747561296246264\\
561	0.000789266191669505\\
562	0.000831174369496882\\
563	0.000873178095902131\\
564	0.000915807659414534\\
565	0.000959488231720514\\
566	0.00100432096429517\\
567	0.0010503705610188\\
568	0.00138555014159372\\
569	0.00168867149361451\\
570	0.00192455116100545\\
571	0.00200205571737336\\
572	0.00207001668793317\\
573	0.00213863941240985\\
574	0.00220843933480784\\
575	0.00227948582459503\\
576	0.00235182128121044\\
577	0.00242548747819655\\
578	0.00250052798438547\\
579	0.00257698863349794\\
580	0.0026549180091809\\
581	0.0027343676902359\\
582	0.0028153925248352\\
583	0.00289805096238546\\
584	0.00298240553557085\\
585	0.00306852370928107\\
586	0.00315647969018814\\
587	0.00324635875816936\\
588	0.0033382683090296\\
589	0.00343236680134408\\
590	0.00352894054086422\\
591	0.00362860835901882\\
592	0.00373286830485375\\
593	0.00384555904940006\\
594	0.00397672044113206\\
595	0.00415280575412961\\
596	0.00444436306303141\\
597	0.00503983077166121\\
598	0.00644286460810295\\
599	0\\
600	0\\
};
\addplot [color=black!20!mycolor21,solid,forget plot]
  table[row sep=crcr]{%
1	0\\
2	0\\
3	0\\
4	0\\
5	0\\
6	0\\
7	0\\
8	0\\
9	0\\
10	0\\
11	0\\
12	0\\
13	0\\
14	0\\
15	0\\
16	0\\
17	0\\
18	0\\
19	0\\
20	0\\
21	0\\
22	0\\
23	0\\
24	0\\
25	0\\
26	0\\
27	0\\
28	0\\
29	0\\
30	0\\
31	0\\
32	0\\
33	0\\
34	0\\
35	0\\
36	0\\
37	0\\
38	0\\
39	0\\
40	0\\
41	0\\
42	0\\
43	0\\
44	0\\
45	0\\
46	0\\
47	0\\
48	0\\
49	0\\
50	0\\
51	0\\
52	0\\
53	0\\
54	0\\
55	0\\
56	0\\
57	0\\
58	0\\
59	0\\
60	0\\
61	0\\
62	0\\
63	0\\
64	0\\
65	0\\
66	0\\
67	0\\
68	0\\
69	0\\
70	0\\
71	0\\
72	0\\
73	0\\
74	0\\
75	0\\
76	0\\
77	0\\
78	0\\
79	0\\
80	0\\
81	0\\
82	0\\
83	0\\
84	0\\
85	0\\
86	0\\
87	0\\
88	0\\
89	0\\
90	0\\
91	0\\
92	0\\
93	0\\
94	0\\
95	0\\
96	0\\
97	0\\
98	0\\
99	0\\
100	0\\
101	0\\
102	0\\
103	0\\
104	0\\
105	0\\
106	0\\
107	0\\
108	0\\
109	0\\
110	0\\
111	0\\
112	0\\
113	0\\
114	0\\
115	0\\
116	0\\
117	0\\
118	0\\
119	0\\
120	0\\
121	0\\
122	0\\
123	0\\
124	0\\
125	0\\
126	0\\
127	0\\
128	0\\
129	0\\
130	0\\
131	0\\
132	0\\
133	0\\
134	0\\
135	0\\
136	0\\
137	0\\
138	0\\
139	0\\
140	0\\
141	0\\
142	0\\
143	0\\
144	0\\
145	0\\
146	0\\
147	0\\
148	0\\
149	0\\
150	0\\
151	0\\
152	0\\
153	0\\
154	0\\
155	0\\
156	0\\
157	0\\
158	0\\
159	0\\
160	0\\
161	0\\
162	0\\
163	0\\
164	0\\
165	0\\
166	0\\
167	0\\
168	0\\
169	0\\
170	0\\
171	0\\
172	0\\
173	0\\
174	0\\
175	0\\
176	0\\
177	0\\
178	0\\
179	0\\
180	0\\
181	0\\
182	0\\
183	0\\
184	0\\
185	0\\
186	0\\
187	0\\
188	0\\
189	0\\
190	0\\
191	0\\
192	0\\
193	0\\
194	0\\
195	0\\
196	0\\
197	0\\
198	0\\
199	0\\
200	0\\
201	0\\
202	0\\
203	0\\
204	0\\
205	0\\
206	0\\
207	0\\
208	0\\
209	0\\
210	0\\
211	0\\
212	0\\
213	0\\
214	0\\
215	0\\
216	0\\
217	0\\
218	0\\
219	0\\
220	0\\
221	0\\
222	0\\
223	0\\
224	0\\
225	0\\
226	0\\
227	0\\
228	0\\
229	0\\
230	0\\
231	0\\
232	0\\
233	0\\
234	0\\
235	0\\
236	0\\
237	0\\
238	0\\
239	0\\
240	0\\
241	0\\
242	0\\
243	0\\
244	0\\
245	0\\
246	0\\
247	0\\
248	0\\
249	0\\
250	0\\
251	0\\
252	0\\
253	0\\
254	0\\
255	0\\
256	0\\
257	0\\
258	0\\
259	0\\
260	0\\
261	0\\
262	0\\
263	0\\
264	0\\
265	0\\
266	0\\
267	0\\
268	0\\
269	0\\
270	0\\
271	0\\
272	0\\
273	0\\
274	0\\
275	0\\
276	0\\
277	0\\
278	0\\
279	0\\
280	0\\
281	0\\
282	0\\
283	0\\
284	0\\
285	0\\
286	0\\
287	0\\
288	0\\
289	0\\
290	0\\
291	0\\
292	0\\
293	0\\
294	0\\
295	0\\
296	0\\
297	0\\
298	0\\
299	0\\
300	0\\
301	0\\
302	0\\
303	0\\
304	0\\
305	0\\
306	0\\
307	0\\
308	0\\
309	0\\
310	0\\
311	0\\
312	0\\
313	0\\
314	0\\
315	0\\
316	0\\
317	0\\
318	0\\
319	0\\
320	0\\
321	0\\
322	0\\
323	0\\
324	0\\
325	0\\
326	0\\
327	0\\
328	0\\
329	0\\
330	0\\
331	0\\
332	0\\
333	0\\
334	0\\
335	0\\
336	0\\
337	0\\
338	0\\
339	0\\
340	0\\
341	0\\
342	0\\
343	0\\
344	0\\
345	0\\
346	0\\
347	0\\
348	0\\
349	0\\
350	0\\
351	0\\
352	0\\
353	0\\
354	0\\
355	0\\
356	0\\
357	0\\
358	0\\
359	0\\
360	0\\
361	0\\
362	0\\
363	0\\
364	0\\
365	0\\
366	0\\
367	0\\
368	0\\
369	0\\
370	0\\
371	0\\
372	0\\
373	0\\
374	0\\
375	0\\
376	0\\
377	0\\
378	0\\
379	0\\
380	0\\
381	0\\
382	0\\
383	0\\
384	0\\
385	0\\
386	0\\
387	0\\
388	0\\
389	0\\
390	0\\
391	0\\
392	0\\
393	0\\
394	0\\
395	0\\
396	0\\
397	0\\
398	0\\
399	0\\
400	0\\
401	0\\
402	0\\
403	0\\
404	0\\
405	0\\
406	0\\
407	0\\
408	0\\
409	0\\
410	0\\
411	0\\
412	0\\
413	0\\
414	0\\
415	0\\
416	0\\
417	0\\
418	0\\
419	0\\
420	0\\
421	0\\
422	0\\
423	0\\
424	0\\
425	0\\
426	0\\
427	0\\
428	0\\
429	0\\
430	0\\
431	0\\
432	0\\
433	0\\
434	0\\
435	0\\
436	0\\
437	0\\
438	0\\
439	0\\
440	0\\
441	0\\
442	0\\
443	0\\
444	0\\
445	0\\
446	0\\
447	0\\
448	0\\
449	0\\
450	0\\
451	0\\
452	0\\
453	0\\
454	0\\
455	0\\
456	0\\
457	0\\
458	0\\
459	0\\
460	0\\
461	0\\
462	0\\
463	0\\
464	0\\
465	0\\
466	0\\
467	0\\
468	0\\
469	0\\
470	0\\
471	0\\
472	0\\
473	0\\
474	0\\
475	0\\
476	0\\
477	0\\
478	0\\
479	0\\
480	0\\
481	0\\
482	0\\
483	0\\
484	0\\
485	0\\
486	0\\
487	0\\
488	0\\
489	0\\
490	0\\
491	0\\
492	0\\
493	0\\
494	0\\
495	0\\
496	0\\
497	0\\
498	0\\
499	0\\
500	0\\
501	0\\
502	0\\
503	0\\
504	0\\
505	0\\
506	0\\
507	0\\
508	0\\
509	0\\
510	0\\
511	0\\
512	0\\
513	0\\
514	0\\
515	0\\
516	0\\
517	0\\
518	0\\
519	0\\
520	0\\
521	0\\
522	0\\
523	0\\
524	0\\
525	0\\
526	0\\
527	0\\
528	0\\
529	0\\
530	0\\
531	0\\
532	0\\
533	0\\
534	0\\
535	0\\
536	0\\
537	1.5257515615078e-05\\
538	4.05712495559513e-05\\
539	6.63714757039172e-05\\
540	9.27016742245395e-05\\
541	0.000119575710927847\\
542	0.000147000013995511\\
543	0.00017504869257464\\
544	0.000203735987885211\\
545	0.000233079425442273\\
546	0.000263096693183489\\
547	0.000293803947281356\\
548	0.000325302371137699\\
549	0.000357642385253218\\
550	0.000390582931632927\\
551	0.00042399701337393\\
552	0.000458068777525668\\
553	0.000492727463226109\\
554	0.000528135608302436\\
555	0.000564380253208309\\
556	0.000601502109540624\\
557	0.000639665961938304\\
558	0.000679881186959353\\
559	0.000720197887011426\\
560	0.000760489350549704\\
561	0.00080084928102869\\
562	0.000841683425308702\\
563	0.000883510817814942\\
564	0.000926424297840563\\
565	0.000970477398949334\\
566	0.00106843923578318\\
567	0.00144259009169947\\
568	0.00171200842408046\\
569	0.0018698844738892\\
570	0.00193635519907031\\
571	0.00200267110891418\\
572	0.00207007254415696\\
573	0.00213865022157796\\
574	0.00220844272080478\\
575	0.00227948699417363\\
576	0.0023518217070068\\
577	0.0024254876758363\\
578	0.00250052808022166\\
579	0.0025769886830341\\
580	0.00265491803552198\\
581	0.00273436770413291\\
582	0.00281539253160681\\
583	0.0028980509654155\\
584	0.00298240553649866\\
585	0.00306852370943816\\
586	0.00315647969018814\\
587	0.00324635875816936\\
588	0.0033382683090296\\
589	0.00343236680134408\\
590	0.00352894054086423\\
591	0.00362860835901881\\
592	0.00373286830485375\\
593	0.00384555904940006\\
594	0.00397672044113205\\
595	0.00415280575412962\\
596	0.00444436306303141\\
597	0.00503983077166121\\
598	0.00644286460810295\\
599	0\\
600	0\\
};
\addplot [color=black!50!mycolor20,solid,forget plot]
  table[row sep=crcr]{%
1	0\\
2	0\\
3	0\\
4	0\\
5	0\\
6	0\\
7	0\\
8	0\\
9	0\\
10	0\\
11	0\\
12	0\\
13	0\\
14	0\\
15	0\\
16	0\\
17	0\\
18	0\\
19	0\\
20	0\\
21	0\\
22	0\\
23	0\\
24	0\\
25	0\\
26	0\\
27	0\\
28	0\\
29	0\\
30	0\\
31	0\\
32	0\\
33	0\\
34	0\\
35	0\\
36	0\\
37	0\\
38	0\\
39	0\\
40	0\\
41	0\\
42	0\\
43	0\\
44	0\\
45	0\\
46	0\\
47	0\\
48	0\\
49	0\\
50	0\\
51	0\\
52	0\\
53	0\\
54	0\\
55	0\\
56	0\\
57	0\\
58	0\\
59	0\\
60	0\\
61	0\\
62	0\\
63	0\\
64	0\\
65	0\\
66	0\\
67	0\\
68	0\\
69	0\\
70	0\\
71	0\\
72	0\\
73	0\\
74	0\\
75	0\\
76	0\\
77	0\\
78	0\\
79	0\\
80	0\\
81	0\\
82	0\\
83	0\\
84	0\\
85	0\\
86	0\\
87	0\\
88	0\\
89	0\\
90	0\\
91	0\\
92	0\\
93	0\\
94	0\\
95	0\\
96	0\\
97	0\\
98	0\\
99	0\\
100	0\\
101	0\\
102	0\\
103	0\\
104	0\\
105	0\\
106	0\\
107	0\\
108	0\\
109	0\\
110	0\\
111	0\\
112	0\\
113	0\\
114	0\\
115	0\\
116	0\\
117	0\\
118	0\\
119	0\\
120	0\\
121	0\\
122	0\\
123	0\\
124	0\\
125	0\\
126	0\\
127	0\\
128	0\\
129	0\\
130	0\\
131	0\\
132	0\\
133	0\\
134	0\\
135	0\\
136	0\\
137	0\\
138	0\\
139	0\\
140	0\\
141	0\\
142	0\\
143	0\\
144	0\\
145	0\\
146	0\\
147	0\\
148	0\\
149	0\\
150	0\\
151	0\\
152	0\\
153	0\\
154	0\\
155	0\\
156	0\\
157	0\\
158	0\\
159	0\\
160	0\\
161	0\\
162	0\\
163	0\\
164	0\\
165	0\\
166	0\\
167	0\\
168	0\\
169	0\\
170	0\\
171	0\\
172	0\\
173	0\\
174	0\\
175	0\\
176	0\\
177	0\\
178	0\\
179	0\\
180	0\\
181	0\\
182	0\\
183	0\\
184	0\\
185	0\\
186	0\\
187	0\\
188	0\\
189	0\\
190	0\\
191	0\\
192	0\\
193	0\\
194	0\\
195	0\\
196	0\\
197	0\\
198	0\\
199	0\\
200	0\\
201	0\\
202	0\\
203	0\\
204	0\\
205	0\\
206	0\\
207	0\\
208	0\\
209	0\\
210	0\\
211	0\\
212	0\\
213	0\\
214	0\\
215	0\\
216	0\\
217	0\\
218	0\\
219	0\\
220	0\\
221	0\\
222	0\\
223	0\\
224	0\\
225	0\\
226	0\\
227	0\\
228	0\\
229	0\\
230	0\\
231	0\\
232	0\\
233	0\\
234	0\\
235	0\\
236	0\\
237	0\\
238	0\\
239	0\\
240	0\\
241	0\\
242	0\\
243	0\\
244	0\\
245	0\\
246	0\\
247	0\\
248	0\\
249	0\\
250	0\\
251	0\\
252	0\\
253	0\\
254	0\\
255	0\\
256	0\\
257	0\\
258	0\\
259	0\\
260	0\\
261	0\\
262	0\\
263	0\\
264	0\\
265	0\\
266	0\\
267	0\\
268	0\\
269	0\\
270	0\\
271	0\\
272	0\\
273	0\\
274	0\\
275	0\\
276	0\\
277	0\\
278	0\\
279	0\\
280	0\\
281	0\\
282	0\\
283	0\\
284	0\\
285	0\\
286	0\\
287	0\\
288	0\\
289	0\\
290	0\\
291	0\\
292	0\\
293	0\\
294	0\\
295	0\\
296	0\\
297	0\\
298	0\\
299	0\\
300	0\\
301	0\\
302	0\\
303	0\\
304	0\\
305	0\\
306	0\\
307	0\\
308	0\\
309	0\\
310	0\\
311	0\\
312	0\\
313	0\\
314	0\\
315	0\\
316	0\\
317	0\\
318	0\\
319	0\\
320	0\\
321	0\\
322	0\\
323	0\\
324	0\\
325	0\\
326	0\\
327	0\\
328	0\\
329	0\\
330	0\\
331	0\\
332	0\\
333	0\\
334	0\\
335	0\\
336	0\\
337	0\\
338	0\\
339	0\\
340	0\\
341	0\\
342	0\\
343	0\\
344	0\\
345	0\\
346	0\\
347	0\\
348	0\\
349	0\\
350	0\\
351	0\\
352	0\\
353	0\\
354	0\\
355	0\\
356	0\\
357	0\\
358	0\\
359	0\\
360	0\\
361	0\\
362	0\\
363	0\\
364	0\\
365	0\\
366	0\\
367	0\\
368	0\\
369	0\\
370	0\\
371	0\\
372	0\\
373	0\\
374	0\\
375	0\\
376	0\\
377	0\\
378	0\\
379	0\\
380	0\\
381	0\\
382	0\\
383	0\\
384	0\\
385	0\\
386	0\\
387	0\\
388	0\\
389	0\\
390	0\\
391	0\\
392	0\\
393	0\\
394	0\\
395	0\\
396	0\\
397	0\\
398	0\\
399	0\\
400	0\\
401	0\\
402	0\\
403	0\\
404	0\\
405	0\\
406	0\\
407	0\\
408	0\\
409	0\\
410	0\\
411	0\\
412	0\\
413	0\\
414	0\\
415	0\\
416	0\\
417	0\\
418	0\\
419	0\\
420	0\\
421	0\\
422	0\\
423	0\\
424	0\\
425	0\\
426	0\\
427	0\\
428	0\\
429	0\\
430	0\\
431	0\\
432	0\\
433	0\\
434	0\\
435	0\\
436	0\\
437	0\\
438	0\\
439	0\\
440	0\\
441	0\\
442	0\\
443	0\\
444	0\\
445	0\\
446	0\\
447	0\\
448	0\\
449	0\\
450	0\\
451	0\\
452	0\\
453	0\\
454	0\\
455	0\\
456	0\\
457	0\\
458	0\\
459	0\\
460	0\\
461	0\\
462	0\\
463	0\\
464	0\\
465	0\\
466	0\\
467	0\\
468	0\\
469	0\\
470	0\\
471	0\\
472	0\\
473	0\\
474	0\\
475	0\\
476	0\\
477	0\\
478	0\\
479	0\\
480	0\\
481	0\\
482	0\\
483	0\\
484	0\\
485	0\\
486	0\\
487	0\\
488	0\\
489	0\\
490	0\\
491	0\\
492	0\\
493	0\\
494	0\\
495	0\\
496	0\\
497	0\\
498	0\\
499	0\\
500	0\\
501	0\\
502	0\\
503	0\\
504	0\\
505	0\\
506	0\\
507	0\\
508	0\\
509	0\\
510	0\\
511	0\\
512	0\\
513	0\\
514	0\\
515	0\\
516	0\\
517	0\\
518	0\\
519	0\\
520	0\\
521	0\\
522	0\\
523	0\\
524	0\\
525	0\\
526	0\\
527	0\\
528	0\\
529	0\\
530	0\\
531	0\\
532	0\\
533	0\\
534	0\\
535	0\\
536	3.78359368673223e-07\\
537	2.51996931176172e-05\\
538	5.05127313110625e-05\\
539	7.6341234946441e-05\\
540	0.000102729583218711\\
541	0.000129693335033606\\
542	0.000157267192673928\\
543	0.000185470106681217\\
544	0.000214316398288709\\
545	0.000243821524482409\\
546	0.000274119909113173\\
547	0.000305191906140328\\
548	0.00033677287195165\\
549	0.000368831234562262\\
550	0.000401484922657279\\
551	0.000434681869006069\\
552	0.00046864575352553\\
553	0.000503409758869739\\
554	0.000539018952998663\\
555	0.000575501755643469\\
556	0.000613762678061962\\
557	0.000652925523529455\\
558	0.000691742840127115\\
559	0.000730684352716771\\
560	0.000769821722848036\\
561	0.000809880072409532\\
562	0.000850959223737115\\
563	0.000893107882836268\\
564	0.00093637240331507\\
565	0.00109010374729446\\
566	0.00145489706217203\\
567	0.0017205761280254\\
568	0.00180709087821166\\
569	0.00187129168921907\\
570	0.0019364300875453\\
571	0.00200267818666582\\
572	0.00207007397726512\\
573	0.00213865068083015\\
574	0.00220844288323674\\
575	0.00227948705593812\\
576	0.00235182173622467\\
577	0.00242548769027979\\
578	0.00250052808776776\\
579	0.00257698868705665\\
580	0.00265491803762909\\
581	0.00273436770515037\\
582	0.00281539253204859\\
583	0.00289805096554754\\
584	0.00298240553652045\\
585	0.00306852370943815\\
586	0.00315647969018813\\
587	0.00324635875816935\\
588	0.0033382683090296\\
589	0.00343236680134408\\
590	0.00352894054086423\\
591	0.00362860835901881\\
592	0.00373286830485375\\
593	0.00384555904940006\\
594	0.00397672044113205\\
595	0.00415280575412962\\
596	0.00444436306303141\\
597	0.00503983077166121\\
598	0.00644286460810295\\
599	0\\
600	0\\
};
\addplot [color=black!60!mycolor21,solid,forget plot]
  table[row sep=crcr]{%
1	0\\
2	0\\
3	0\\
4	0\\
5	0\\
6	0\\
7	0\\
8	0\\
9	0\\
10	0\\
11	0\\
12	0\\
13	0\\
14	0\\
15	0\\
16	0\\
17	0\\
18	0\\
19	0\\
20	0\\
21	0\\
22	0\\
23	0\\
24	0\\
25	0\\
26	0\\
27	0\\
28	0\\
29	0\\
30	0\\
31	0\\
32	0\\
33	0\\
34	0\\
35	0\\
36	0\\
37	0\\
38	0\\
39	0\\
40	0\\
41	0\\
42	0\\
43	0\\
44	0\\
45	0\\
46	0\\
47	0\\
48	0\\
49	0\\
50	0\\
51	0\\
52	0\\
53	0\\
54	0\\
55	0\\
56	0\\
57	0\\
58	0\\
59	0\\
60	0\\
61	0\\
62	0\\
63	0\\
64	0\\
65	0\\
66	0\\
67	0\\
68	0\\
69	0\\
70	0\\
71	0\\
72	0\\
73	0\\
74	0\\
75	0\\
76	0\\
77	0\\
78	0\\
79	0\\
80	0\\
81	0\\
82	0\\
83	0\\
84	0\\
85	0\\
86	0\\
87	0\\
88	0\\
89	0\\
90	0\\
91	0\\
92	0\\
93	0\\
94	0\\
95	0\\
96	0\\
97	0\\
98	0\\
99	0\\
100	0\\
101	0\\
102	0\\
103	0\\
104	0\\
105	0\\
106	0\\
107	0\\
108	0\\
109	0\\
110	0\\
111	0\\
112	0\\
113	0\\
114	0\\
115	0\\
116	0\\
117	0\\
118	0\\
119	0\\
120	0\\
121	0\\
122	0\\
123	0\\
124	0\\
125	0\\
126	0\\
127	0\\
128	0\\
129	0\\
130	0\\
131	0\\
132	0\\
133	0\\
134	0\\
135	0\\
136	0\\
137	0\\
138	0\\
139	0\\
140	0\\
141	0\\
142	0\\
143	0\\
144	0\\
145	0\\
146	0\\
147	0\\
148	0\\
149	0\\
150	0\\
151	0\\
152	0\\
153	0\\
154	0\\
155	0\\
156	0\\
157	0\\
158	0\\
159	0\\
160	0\\
161	0\\
162	0\\
163	0\\
164	0\\
165	0\\
166	0\\
167	0\\
168	0\\
169	0\\
170	0\\
171	0\\
172	0\\
173	0\\
174	0\\
175	0\\
176	0\\
177	0\\
178	0\\
179	0\\
180	0\\
181	0\\
182	0\\
183	0\\
184	0\\
185	0\\
186	0\\
187	0\\
188	0\\
189	0\\
190	0\\
191	0\\
192	0\\
193	0\\
194	0\\
195	0\\
196	0\\
197	0\\
198	0\\
199	0\\
200	0\\
201	0\\
202	0\\
203	0\\
204	0\\
205	0\\
206	0\\
207	0\\
208	0\\
209	0\\
210	0\\
211	0\\
212	0\\
213	0\\
214	0\\
215	0\\
216	0\\
217	0\\
218	0\\
219	0\\
220	0\\
221	0\\
222	0\\
223	0\\
224	0\\
225	0\\
226	0\\
227	0\\
228	0\\
229	0\\
230	0\\
231	0\\
232	0\\
233	0\\
234	0\\
235	0\\
236	0\\
237	0\\
238	0\\
239	0\\
240	0\\
241	0\\
242	0\\
243	0\\
244	0\\
245	0\\
246	0\\
247	0\\
248	0\\
249	0\\
250	0\\
251	0\\
252	0\\
253	0\\
254	0\\
255	0\\
256	0\\
257	0\\
258	0\\
259	0\\
260	0\\
261	0\\
262	0\\
263	0\\
264	0\\
265	0\\
266	0\\
267	0\\
268	0\\
269	0\\
270	0\\
271	0\\
272	0\\
273	0\\
274	0\\
275	0\\
276	0\\
277	0\\
278	0\\
279	0\\
280	0\\
281	0\\
282	0\\
283	0\\
284	0\\
285	0\\
286	0\\
287	0\\
288	0\\
289	0\\
290	0\\
291	0\\
292	0\\
293	0\\
294	0\\
295	0\\
296	0\\
297	0\\
298	0\\
299	0\\
300	0\\
301	0\\
302	0\\
303	0\\
304	0\\
305	0\\
306	0\\
307	0\\
308	0\\
309	0\\
310	0\\
311	0\\
312	0\\
313	0\\
314	0\\
315	0\\
316	0\\
317	0\\
318	0\\
319	0\\
320	0\\
321	0\\
322	0\\
323	0\\
324	0\\
325	0\\
326	0\\
327	0\\
328	0\\
329	0\\
330	0\\
331	0\\
332	0\\
333	0\\
334	0\\
335	0\\
336	0\\
337	0\\
338	0\\
339	0\\
340	0\\
341	0\\
342	0\\
343	0\\
344	0\\
345	0\\
346	0\\
347	0\\
348	0\\
349	0\\
350	0\\
351	0\\
352	0\\
353	0\\
354	0\\
355	0\\
356	0\\
357	0\\
358	0\\
359	0\\
360	0\\
361	0\\
362	0\\
363	0\\
364	0\\
365	0\\
366	0\\
367	0\\
368	0\\
369	0\\
370	0\\
371	0\\
372	0\\
373	0\\
374	0\\
375	0\\
376	0\\
377	0\\
378	0\\
379	0\\
380	0\\
381	0\\
382	0\\
383	0\\
384	0\\
385	0\\
386	0\\
387	0\\
388	0\\
389	0\\
390	0\\
391	0\\
392	0\\
393	0\\
394	0\\
395	0\\
396	0\\
397	0\\
398	0\\
399	0\\
400	0\\
401	0\\
402	0\\
403	0\\
404	0\\
405	0\\
406	0\\
407	0\\
408	0\\
409	0\\
410	0\\
411	0\\
412	0\\
413	0\\
414	0\\
415	0\\
416	0\\
417	0\\
418	0\\
419	0\\
420	0\\
421	0\\
422	0\\
423	0\\
424	0\\
425	0\\
426	0\\
427	0\\
428	0\\
429	0\\
430	0\\
431	0\\
432	0\\
433	0\\
434	0\\
435	0\\
436	0\\
437	0\\
438	0\\
439	0\\
440	0\\
441	0\\
442	0\\
443	0\\
444	0\\
445	0\\
446	0\\
447	0\\
448	0\\
449	0\\
450	0\\
451	0\\
452	0\\
453	0\\
454	0\\
455	0\\
456	0\\
457	0\\
458	0\\
459	0\\
460	0\\
461	0\\
462	0\\
463	0\\
464	0\\
465	0\\
466	0\\
467	0\\
468	0\\
469	0\\
470	0\\
471	0\\
472	0\\
473	0\\
474	0\\
475	0\\
476	0\\
477	0\\
478	0\\
479	0\\
480	0\\
481	0\\
482	0\\
483	0\\
484	0\\
485	0\\
486	0\\
487	0\\
488	0\\
489	0\\
490	0\\
491	0\\
492	0\\
493	0\\
494	0\\
495	0\\
496	0\\
497	0\\
498	0\\
499	0\\
500	0\\
501	0\\
502	0\\
503	0\\
504	0\\
505	0\\
506	0\\
507	0\\
508	0\\
509	0\\
510	0\\
511	0\\
512	0\\
513	0\\
514	0\\
515	0\\
516	0\\
517	0\\
518	0\\
519	0\\
520	0\\
521	0\\
522	0\\
523	0\\
524	0\\
525	0\\
526	0\\
527	0\\
528	0\\
529	0\\
530	0\\
531	0\\
532	0\\
533	0\\
534	0\\
535	0\\
536	9.02145954336361e-06\\
537	3.38870820114369e-05\\
538	5.92843281107733e-05\\
539	8.52316144653158e-05\\
540	0.000111744299307854\\
541	0.000138841481860631\\
542	0.000166552064353744\\
543	0.00019489091642814\\
544	0.0002240061624105\\
545	0.000253856225130063\\
546	0.000284149213149703\\
547	0.000314894675820076\\
548	0.000346177518049505\\
549	0.000378018067102527\\
550	0.000410577800625992\\
551	0.000443915441746857\\
552	0.00047805631403732\\
553	0.000513026094184202\\
554	0.00054907441425197\\
555	0.000587044911282405\\
556	0.000624783568171093\\
557	0.000662504517028127\\
558	0.000700022606374216\\
559	0.000738375853450751\\
560	0.000777685135972991\\
561	0.000817995895139835\\
562	0.000859350044913817\\
563	0.000901791512692584\\
564	0.00109093692626027\\
565	0.00144213576563695\\
566	0.00167960457611589\\
567	0.00174425603612933\\
568	0.00180725733542332\\
569	0.00187130073891484\\
570	0.00193643097966661\\
571	0.00200267837569042\\
572	0.00207007403928857\\
573	0.00213865070333988\\
574	0.00220844289215589\\
575	0.00227948706023545\\
576	0.00235182173838593\\
577	0.00242548769141891\\
578	0.00250052808837546\\
579	0.00257698868737254\\
580	0.00265491803777995\\
581	0.00273436770521399\\
582	0.00281539253206718\\
583	0.00289805096555054\\
584	0.00298240553652045\\
585	0.00306852370943814\\
586	0.00315647969018814\\
587	0.00324635875816936\\
588	0.0033382683090296\\
589	0.00343236680134407\\
590	0.00352894054086422\\
591	0.00362860835901882\\
592	0.00373286830485375\\
593	0.00384555904940006\\
594	0.00397672044113206\\
595	0.00415280575412962\\
596	0.00444436306303141\\
597	0.00503983077166121\\
598	0.00644286460810295\\
599	0\\
600	0\\
};
\addplot [color=black!80!mycolor21,solid,forget plot]
  table[row sep=crcr]{%
1	0\\
2	0\\
3	0\\
4	0\\
5	0\\
6	0\\
7	0\\
8	0\\
9	0\\
10	0\\
11	0\\
12	0\\
13	0\\
14	0\\
15	0\\
16	0\\
17	0\\
18	0\\
19	0\\
20	0\\
21	0\\
22	0\\
23	0\\
24	0\\
25	0\\
26	0\\
27	0\\
28	0\\
29	0\\
30	0\\
31	0\\
32	0\\
33	0\\
34	0\\
35	0\\
36	0\\
37	0\\
38	0\\
39	0\\
40	0\\
41	0\\
42	0\\
43	0\\
44	0\\
45	0\\
46	0\\
47	0\\
48	0\\
49	0\\
50	0\\
51	0\\
52	0\\
53	0\\
54	0\\
55	0\\
56	0\\
57	0\\
58	0\\
59	0\\
60	0\\
61	0\\
62	0\\
63	0\\
64	0\\
65	0\\
66	0\\
67	0\\
68	0\\
69	0\\
70	0\\
71	0\\
72	0\\
73	0\\
74	0\\
75	0\\
76	0\\
77	0\\
78	0\\
79	0\\
80	0\\
81	0\\
82	0\\
83	0\\
84	0\\
85	0\\
86	0\\
87	0\\
88	0\\
89	0\\
90	0\\
91	0\\
92	0\\
93	0\\
94	0\\
95	0\\
96	0\\
97	0\\
98	0\\
99	0\\
100	0\\
101	0\\
102	0\\
103	0\\
104	0\\
105	0\\
106	0\\
107	0\\
108	0\\
109	0\\
110	0\\
111	0\\
112	0\\
113	0\\
114	0\\
115	0\\
116	0\\
117	0\\
118	0\\
119	0\\
120	0\\
121	0\\
122	0\\
123	0\\
124	0\\
125	0\\
126	0\\
127	0\\
128	0\\
129	0\\
130	0\\
131	0\\
132	0\\
133	0\\
134	0\\
135	0\\
136	0\\
137	0\\
138	0\\
139	0\\
140	0\\
141	0\\
142	0\\
143	0\\
144	0\\
145	0\\
146	0\\
147	0\\
148	0\\
149	0\\
150	0\\
151	0\\
152	0\\
153	0\\
154	0\\
155	0\\
156	0\\
157	0\\
158	0\\
159	0\\
160	0\\
161	0\\
162	0\\
163	0\\
164	0\\
165	0\\
166	0\\
167	0\\
168	0\\
169	0\\
170	0\\
171	0\\
172	0\\
173	0\\
174	0\\
175	0\\
176	0\\
177	0\\
178	0\\
179	0\\
180	0\\
181	0\\
182	0\\
183	0\\
184	0\\
185	0\\
186	0\\
187	0\\
188	0\\
189	0\\
190	0\\
191	0\\
192	0\\
193	0\\
194	0\\
195	0\\
196	0\\
197	0\\
198	0\\
199	0\\
200	0\\
201	0\\
202	0\\
203	0\\
204	0\\
205	0\\
206	0\\
207	0\\
208	0\\
209	0\\
210	0\\
211	0\\
212	0\\
213	0\\
214	0\\
215	0\\
216	0\\
217	0\\
218	0\\
219	0\\
220	0\\
221	0\\
222	0\\
223	0\\
224	0\\
225	0\\
226	0\\
227	0\\
228	0\\
229	0\\
230	0\\
231	0\\
232	0\\
233	0\\
234	0\\
235	0\\
236	0\\
237	0\\
238	0\\
239	0\\
240	0\\
241	0\\
242	0\\
243	0\\
244	0\\
245	0\\
246	0\\
247	0\\
248	0\\
249	0\\
250	0\\
251	0\\
252	0\\
253	0\\
254	0\\
255	0\\
256	0\\
257	0\\
258	0\\
259	0\\
260	0\\
261	0\\
262	0\\
263	0\\
264	0\\
265	0\\
266	0\\
267	0\\
268	0\\
269	0\\
270	0\\
271	0\\
272	0\\
273	0\\
274	0\\
275	0\\
276	0\\
277	0\\
278	0\\
279	0\\
280	0\\
281	0\\
282	0\\
283	0\\
284	0\\
285	0\\
286	0\\
287	0\\
288	0\\
289	0\\
290	0\\
291	0\\
292	0\\
293	0\\
294	0\\
295	0\\
296	0\\
297	0\\
298	0\\
299	0\\
300	0\\
301	0\\
302	0\\
303	0\\
304	0\\
305	0\\
306	0\\
307	0\\
308	0\\
309	0\\
310	0\\
311	0\\
312	0\\
313	0\\
314	0\\
315	0\\
316	0\\
317	0\\
318	0\\
319	0\\
320	0\\
321	0\\
322	0\\
323	0\\
324	0\\
325	0\\
326	0\\
327	0\\
328	0\\
329	0\\
330	0\\
331	0\\
332	0\\
333	0\\
334	0\\
335	0\\
336	0\\
337	0\\
338	0\\
339	0\\
340	0\\
341	0\\
342	0\\
343	0\\
344	0\\
345	0\\
346	0\\
347	0\\
348	0\\
349	0\\
350	0\\
351	0\\
352	0\\
353	0\\
354	0\\
355	0\\
356	0\\
357	0\\
358	0\\
359	0\\
360	0\\
361	0\\
362	0\\
363	0\\
364	0\\
365	0\\
366	0\\
367	0\\
368	0\\
369	0\\
370	0\\
371	0\\
372	0\\
373	0\\
374	0\\
375	0\\
376	0\\
377	0\\
378	0\\
379	0\\
380	0\\
381	0\\
382	0\\
383	0\\
384	0\\
385	0\\
386	0\\
387	0\\
388	0\\
389	0\\
390	0\\
391	0\\
392	0\\
393	0\\
394	0\\
395	0\\
396	0\\
397	0\\
398	0\\
399	0\\
400	0\\
401	0\\
402	0\\
403	0\\
404	0\\
405	0\\
406	0\\
407	0\\
408	0\\
409	0\\
410	0\\
411	0\\
412	0\\
413	0\\
414	0\\
415	0\\
416	0\\
417	0\\
418	0\\
419	0\\
420	0\\
421	0\\
422	0\\
423	0\\
424	0\\
425	0\\
426	0\\
427	0\\
428	0\\
429	0\\
430	0\\
431	0\\
432	0\\
433	0\\
434	0\\
435	0\\
436	0\\
437	0\\
438	0\\
439	0\\
440	0\\
441	0\\
442	0\\
443	0\\
444	0\\
445	0\\
446	0\\
447	0\\
448	0\\
449	0\\
450	0\\
451	0\\
452	0\\
453	0\\
454	0\\
455	0\\
456	0\\
457	0\\
458	0\\
459	0\\
460	0\\
461	0\\
462	0\\
463	0\\
464	0\\
465	0\\
466	0\\
467	0\\
468	0\\
469	0\\
470	0\\
471	0\\
472	0\\
473	0\\
474	0\\
475	0\\
476	0\\
477	0\\
478	0\\
479	0\\
480	0\\
481	0\\
482	0\\
483	0\\
484	0\\
485	0\\
486	0\\
487	0\\
488	0\\
489	0\\
490	0\\
491	0\\
492	0\\
493	0\\
494	0\\
495	0\\
496	0\\
497	0\\
498	0\\
499	0\\
500	0\\
501	0\\
502	0\\
503	0\\
504	0\\
505	0\\
506	0\\
507	0\\
508	0\\
509	0\\
510	0\\
511	0\\
512	0\\
513	0\\
514	0\\
515	0\\
516	0\\
517	0\\
518	0\\
519	0\\
520	0\\
521	0\\
522	0\\
523	0\\
524	0\\
525	0\\
526	0\\
527	0\\
528	0\\
529	0\\
530	0\\
531	0\\
532	0\\
533	0\\
534	0\\
535	0\\
536	1.67358631738589e-05\\
537	4.17113787144746e-05\\
538	6.72225789001668e-05\\
539	9.32840951840519e-05\\
540	0.000119910708273509\\
541	0.000147126549934032\\
542	0.000175083363931519\\
543	0.000203765392134202\\
544	0.000232847505447229\\
545	0.000262347629977608\\
546	0.000292337815156544\\
547	0.000322869907415254\\
548	0.000354099010332315\\
549	0.000386067917324203\\
550	0.000418801906923749\\
551	0.000452323301093407\\
552	0.000486654809378515\\
553	0.000522613247243088\\
554	0.000559557726971212\\
555	0.000596060905993073\\
556	0.000632410263998094\\
557	0.000669141593433294\\
558	0.000706767938764675\\
559	0.000745333893433888\\
560	0.000784877380144605\\
561	0.000825438279498713\\
562	0.000867058997353549\\
563	0.00106897995898566\\
564	0.00141439733785219\\
565	0.00162107369184165\\
566	0.00168232368728045\\
567	0.00174427557656458\\
568	0.00180725842153865\\
569	0.00187130085080805\\
570	0.00193643100447991\\
571	0.00200267838403535\\
572	0.00207007404240156\\
573	0.00213865070462208\\
574	0.00220844289278456\\
575	0.00227948706055647\\
576	0.00235182173855629\\
577	0.00242548769150975\\
578	0.00250052808842228\\
579	0.00257698868739462\\
580	0.00265491803778901\\
581	0.00273436770521657\\
582	0.00281539253206758\\
583	0.00289805096555055\\
584	0.00298240553652045\\
585	0.00306852370943816\\
586	0.00315647969018813\\
587	0.00324635875816936\\
588	0.0033382683090296\\
589	0.00343236680134409\\
590	0.00352894054086423\\
591	0.00362860835901882\\
592	0.00373286830485376\\
593	0.00384555904940007\\
594	0.00397672044113206\\
595	0.00415280575412962\\
596	0.00444436306303141\\
597	0.00503983077166122\\
598	0.00644286460810295\\
599	0\\
600	0\\
};
\addplot [color=black,solid,forget plot]
  table[row sep=crcr]{%
1	0\\
2	0\\
3	0\\
4	0\\
5	0\\
6	0\\
7	0\\
8	0\\
9	0\\
10	0\\
11	0\\
12	0\\
13	0\\
14	0\\
15	0\\
16	0\\
17	0\\
18	0\\
19	0\\
20	0\\
21	0\\
22	0\\
23	0\\
24	0\\
25	0\\
26	0\\
27	0\\
28	0\\
29	0\\
30	0\\
31	0\\
32	0\\
33	0\\
34	0\\
35	0\\
36	0\\
37	0\\
38	0\\
39	0\\
40	0\\
41	0\\
42	0\\
43	0\\
44	0\\
45	0\\
46	0\\
47	0\\
48	0\\
49	0\\
50	0\\
51	0\\
52	0\\
53	0\\
54	0\\
55	0\\
56	0\\
57	0\\
58	0\\
59	0\\
60	0\\
61	0\\
62	0\\
63	0\\
64	0\\
65	0\\
66	0\\
67	0\\
68	0\\
69	0\\
70	0\\
71	0\\
72	0\\
73	0\\
74	0\\
75	0\\
76	0\\
77	0\\
78	0\\
79	0\\
80	0\\
81	0\\
82	0\\
83	0\\
84	0\\
85	0\\
86	0\\
87	0\\
88	0\\
89	0\\
90	0\\
91	0\\
92	0\\
93	0\\
94	0\\
95	0\\
96	0\\
97	0\\
98	0\\
99	0\\
100	0\\
101	0\\
102	0\\
103	0\\
104	0\\
105	0\\
106	0\\
107	0\\
108	0\\
109	0\\
110	0\\
111	0\\
112	0\\
113	0\\
114	0\\
115	0\\
116	0\\
117	0\\
118	0\\
119	0\\
120	0\\
121	0\\
122	0\\
123	0\\
124	0\\
125	0\\
126	0\\
127	0\\
128	0\\
129	0\\
130	0\\
131	0\\
132	0\\
133	0\\
134	0\\
135	0\\
136	0\\
137	0\\
138	0\\
139	0\\
140	0\\
141	0\\
142	0\\
143	0\\
144	0\\
145	0\\
146	0\\
147	0\\
148	0\\
149	0\\
150	0\\
151	0\\
152	0\\
153	0\\
154	0\\
155	0\\
156	0\\
157	0\\
158	0\\
159	0\\
160	0\\
161	0\\
162	0\\
163	0\\
164	0\\
165	0\\
166	0\\
167	0\\
168	0\\
169	0\\
170	0\\
171	0\\
172	0\\
173	0\\
174	0\\
175	0\\
176	0\\
177	0\\
178	0\\
179	0\\
180	0\\
181	0\\
182	0\\
183	0\\
184	0\\
185	0\\
186	0\\
187	0\\
188	0\\
189	0\\
190	0\\
191	0\\
192	0\\
193	0\\
194	0\\
195	0\\
196	0\\
197	0\\
198	0\\
199	0\\
200	0\\
201	0\\
202	0\\
203	0\\
204	0\\
205	0\\
206	0\\
207	0\\
208	0\\
209	0\\
210	0\\
211	0\\
212	0\\
213	0\\
214	0\\
215	0\\
216	0\\
217	0\\
218	0\\
219	0\\
220	0\\
221	0\\
222	0\\
223	0\\
224	0\\
225	0\\
226	0\\
227	0\\
228	0\\
229	0\\
230	0\\
231	0\\
232	0\\
233	0\\
234	0\\
235	0\\
236	0\\
237	0\\
238	0\\
239	0\\
240	0\\
241	0\\
242	0\\
243	0\\
244	0\\
245	0\\
246	0\\
247	0\\
248	0\\
249	0\\
250	0\\
251	0\\
252	0\\
253	0\\
254	0\\
255	0\\
256	0\\
257	0\\
258	0\\
259	0\\
260	0\\
261	0\\
262	0\\
263	0\\
264	0\\
265	0\\
266	0\\
267	0\\
268	0\\
269	0\\
270	0\\
271	0\\
272	0\\
273	0\\
274	0\\
275	0\\
276	0\\
277	0\\
278	0\\
279	0\\
280	0\\
281	0\\
282	0\\
283	0\\
284	0\\
285	0\\
286	0\\
287	0\\
288	0\\
289	0\\
290	0\\
291	0\\
292	0\\
293	0\\
294	0\\
295	0\\
296	0\\
297	0\\
298	0\\
299	0\\
300	0\\
301	0\\
302	0\\
303	0\\
304	0\\
305	0\\
306	0\\
307	0\\
308	0\\
309	0\\
310	0\\
311	0\\
312	0\\
313	0\\
314	0\\
315	0\\
316	0\\
317	0\\
318	0\\
319	0\\
320	0\\
321	0\\
322	0\\
323	0\\
324	0\\
325	0\\
326	0\\
327	0\\
328	0\\
329	0\\
330	0\\
331	0\\
332	0\\
333	0\\
334	0\\
335	0\\
336	0\\
337	0\\
338	0\\
339	0\\
340	0\\
341	0\\
342	0\\
343	0\\
344	0\\
345	0\\
346	0\\
347	0\\
348	0\\
349	0\\
350	0\\
351	0\\
352	0\\
353	0\\
354	0\\
355	0\\
356	0\\
357	0\\
358	0\\
359	0\\
360	0\\
361	0\\
362	0\\
363	0\\
364	0\\
365	0\\
366	0\\
367	0\\
368	0\\
369	0\\
370	0\\
371	0\\
372	0\\
373	0\\
374	0\\
375	0\\
376	0\\
377	0\\
378	0\\
379	0\\
380	0\\
381	0\\
382	0\\
383	0\\
384	0\\
385	0\\
386	0\\
387	0\\
388	0\\
389	0\\
390	0\\
391	0\\
392	0\\
393	0\\
394	0\\
395	0\\
396	0\\
397	0\\
398	0\\
399	0\\
400	0\\
401	0\\
402	0\\
403	0\\
404	0\\
405	0\\
406	0\\
407	0\\
408	0\\
409	0\\
410	0\\
411	0\\
412	0\\
413	0\\
414	0\\
415	0\\
416	0\\
417	0\\
418	0\\
419	0\\
420	0\\
421	0\\
422	0\\
423	0\\
424	0\\
425	0\\
426	0\\
427	0\\
428	0\\
429	0\\
430	0\\
431	0\\
432	0\\
433	0\\
434	0\\
435	0\\
436	0\\
437	0\\
438	0\\
439	0\\
440	0\\
441	0\\
442	0\\
443	0\\
444	0\\
445	0\\
446	0\\
447	0\\
448	0\\
449	0\\
450	0\\
451	0\\
452	0\\
453	0\\
454	0\\
455	0\\
456	0\\
457	0\\
458	0\\
459	0\\
460	0\\
461	0\\
462	0\\
463	0\\
464	0\\
465	0\\
466	0\\
467	0\\
468	0\\
469	0\\
470	0\\
471	0\\
472	0\\
473	0\\
474	0\\
475	0\\
476	0\\
477	0\\
478	0\\
479	0\\
480	0\\
481	0\\
482	0\\
483	0\\
484	0\\
485	0\\
486	0\\
487	0\\
488	0\\
489	0\\
490	0\\
491	0\\
492	0\\
493	0\\
494	0\\
495	0\\
496	0\\
497	0\\
498	0\\
499	0\\
500	0\\
501	0\\
502	0\\
503	0\\
504	0\\
505	0\\
506	0\\
507	0\\
508	0\\
509	0\\
510	0\\
511	0\\
512	0\\
513	0\\
514	0\\
515	0\\
516	0\\
517	0\\
518	0\\
519	0\\
520	0\\
521	0\\
522	0\\
523	0\\
524	0\\
525	0\\
526	0\\
527	0\\
528	0\\
529	0\\
530	0\\
531	0\\
532	0\\
533	0\\
534	0\\
535	0\\
536	2.3273899000436e-05\\
537	4.83574677493769e-05\\
538	7.39764419503986e-05\\
539	0.000100145056725748\\
540	0.000126927776089947\\
541	0.000154551208357287\\
542	0.000182537594263675\\
543	0.000210877875154721\\
544	0.000239654581585972\\
545	0.000268942646031741\\
546	0.000298901101391198\\
547	0.000329565590919966\\
548	0.000360958489926797\\
549	0.000393100761277305\\
550	0.0004260137847175\\
551	0.000459720061076944\\
552	0.000495493917640628\\
553	0.000531330574309899\\
554	0.000566818867744161\\
555	0.000602062179114877\\
556	0.000638100502800796\\
557	0.000675022523672534\\
558	0.000712862997966947\\
559	0.00075165794188291\\
560	0.000791445930933432\\
561	0.000832268907521252\\
562	0.00101617204181345\\
563	0.00137303696588922\\
564	0.0015613898810912\\
565	0.00162138355504735\\
566	0.00168232596368277\\
567	0.0017442757060301\\
568	0.00180725843549943\\
569	0.00187130085404748\\
570	0.00193643100559733\\
571	0.00200267838446417\\
572	0.00207007404258462\\
573	0.0021386507047133\\
574	0.00220844289283176\\
575	0.00227948706058165\\
576	0.00235182173856971\\
577	0.0024254876915166\\
578	0.00250052808842547\\
579	0.00257698868739587\\
580	0.00265491803778937\\
581	0.00273436770521664\\
582	0.00281539253206758\\
583	0.00289805096555053\\
584	0.00298240553652045\\
585	0.00306852370943815\\
586	0.00315647969018813\\
587	0.00324635875816936\\
588	0.00333826830902961\\
589	0.00343236680134409\\
590	0.00352894054086423\\
591	0.00362860835901881\\
592	0.00373286830485375\\
593	0.00384555904940006\\
594	0.00397672044113206\\
595	0.00415280575412962\\
596	0.00444436306303141\\
597	0.00503983077166122\\
598	0.00644286460810295\\
599	0\\
600	0\\
};
\end{axis}
\end{tikzpicture}%
 
%  \caption{Discrete Time w/ nFPC}
%\end{subfigure}\\
%
%\leavevmode\smash{\makebox[0pt]{\hspace{-7em}% HORIZONTAL POSITION           
%  \rotatebox[origin=l]{90}{\hspace{20em}% VERTICAL POSITION
%    Depth $\delta^+$}%
%}}\hspace{0pt plus 1filll}\null
%
%Time (s)
%
%\vspace{1cm}
%\begin{subfigure}{\linewidth}
%  \centering
%  \tikzsetnextfilename{deltalegend}
%  \definecolor{mycolor1}{rgb}{1.00000,0.00000,1.00000}%
\begin{tikzpicture}[framed]
    \begingroup
    % inits/clears the lists (which might be populated from previous
    % axes):
    \csname pgfplots@init@cleared@structures\endcsname
    \pgfplotsset{legend style={at={(0,1)},anchor=north west},legend columns=-1,legend style={draw=none,column sep=1ex},legend entries={$q=-4$,$q=-3$,$q=-2$,$q=-1$}}%
    
    \csname pgfplots@addlegendimage\endcsname{thick,green,dashed,sharp plot}
    \csname pgfplots@addlegendimage\endcsname{thick,mycolor1,dashed,sharp plot}
    \csname pgfplots@addlegendimage\endcsname{thick,red,dashed,sharp plot}
    \csname pgfplots@addlegendimage\endcsname{thick,blue,dashed,sharp plot}

    % draws the legend:
    \csname pgfplots@createlegend\endcsname
    \endgroup

    \begingroup
    % inits/clears the lists (which might be populated from previous
    % axes):
    \csname pgfplots@init@cleared@structures\endcsname
    \pgfplotsset{legend style={at={(3.75,0.5)},anchor=north west},legend columns=-1,legend style={draw=none,column sep=1ex},legend entries={$q=0$}}%

    \csname pgfplots@addlegendimage\endcsname{thick,black,sharp plot}

    % draws the legend:
    \csname pgfplots@createlegend\endcsname
    \endgroup

    \begingroup
    % inits/clears the lists (which might be populated from previous
    % axes):
    \csname pgfplots@init@cleared@structures\endcsname
    \pgfplotsset{legend style={at={(0,0)},anchor=north west},legend columns=-1,legend style={draw=none,column sep=1ex},legend entries={$q=+4$,$q=+3$,$q=+2$,$q=+1$}}%
    
    \csname pgfplots@addlegendimage\endcsname{thick,green,sharp plot}
    \csname pgfplots@addlegendimage\endcsname{thick,mycolor1,sharp plot}
    \csname pgfplots@addlegendimage\endcsname{thick,red,sharp plot}
    \csname pgfplots@addlegendimage\endcsname{thick,blue,sharp plot}

    % draws the legend:
    \csname pgfplots@createlegend\endcsname
    \endgroup
\end{tikzpicture} 
%\end{subfigure}%
%  \caption{Optimal buy depths $\delta^{+}$ for Markov state $Z=(\rho = +1, \Delta S = +1)$, implying heavy imbalance in favor of buy pressure, and having previously seen an upward price change. We expect the midprice to rise.}
%  \label{fig:comp_dp_z15}
%\end{figure}
%
%\fxnote{So which of these plots do we like better? the 9 choices of q, or the colormap version of q?}
\begin{figure}
\centering
\begin{subfigure}{.45\linewidth}
  \centering
  \setlength\figureheight{\linewidth} 
  \setlength\figurewidth{\linewidth}
  \tikzsetnextfilename{dp_colorbar/dp_cts_z1}
  % This file was created by matlab2tikz.
%
%The latest updates can be retrieved from
%  http://www.mathworks.com/matlabcentral/fileexchange/22022-matlab2tikz-matlab2tikz
%where you can also make suggestions and rate matlab2tikz.
%
\definecolor{mycolor1}{rgb}{0.00000,1.00000,0.14286}%
\definecolor{mycolor2}{rgb}{0.00000,1.00000,0.28571}%
\definecolor{mycolor3}{rgb}{0.00000,1.00000,0.42857}%
\definecolor{mycolor4}{rgb}{0.00000,1.00000,0.57143}%
\definecolor{mycolor5}{rgb}{0.00000,1.00000,0.71429}%
\definecolor{mycolor6}{rgb}{0.00000,1.00000,0.85714}%
\definecolor{mycolor7}{rgb}{0.00000,1.00000,1.00000}%
\definecolor{mycolor8}{rgb}{0.00000,0.87500,1.00000}%
\definecolor{mycolor9}{rgb}{0.00000,0.62500,1.00000}%
\definecolor{mycolor10}{rgb}{0.12500,0.00000,1.00000}%
\definecolor{mycolor11}{rgb}{0.25000,0.00000,1.00000}%
\definecolor{mycolor12}{rgb}{0.37500,0.00000,1.00000}%
\definecolor{mycolor13}{rgb}{0.50000,0.00000,1.00000}%
\definecolor{mycolor14}{rgb}{0.62500,0.00000,1.00000}%
\definecolor{mycolor15}{rgb}{0.75000,0.00000,1.00000}%
\definecolor{mycolor16}{rgb}{0.87500,0.00000,1.00000}%
\definecolor{mycolor17}{rgb}{1.00000,0.00000,1.00000}%
\definecolor{mycolor18}{rgb}{1.00000,0.00000,0.87500}%
\definecolor{mycolor19}{rgb}{1.00000,0.00000,0.62500}%
\definecolor{mycolor20}{rgb}{0.85714,0.00000,0.00000}%
\definecolor{mycolor21}{rgb}{0.71429,0.00000,0.00000}%
%
\begin{tikzpicture}[trim axis left, trim axis right]

\begin{axis}[%
width=\figurewidth,
height=\figureheight,
at={(0\figurewidth,0\figureheight)},
scale only axis,
point meta min=0,
point meta max=1,
every outer x axis line/.append style={black},
every x tick label/.append style={font=\color{black}},
xmin=0,
xmax=600,
every outer y axis line/.append style={black},
every y tick label/.append style={font=\color{black}},
ymin=0,
ymax=0.014,
axis background/.style={fill=white},
axis x line*=bottom,
axis y line*=left,
]
\addplot [color=green,solid,forget plot]
  table[row sep=crcr]{%
0.01	0\\
1.01	0\\
2.01	0\\
3.01	0\\
4.01	0\\
5.01	0\\
6.01	0\\
7.01	0\\
8.01	0\\
9.01	0\\
10.01	0\\
11.01	0\\
12.01	0\\
13.01	0\\
14.01	0\\
15.01	0\\
16.01	0\\
17.01	0\\
18.01	0\\
19.01	0\\
20.01	0\\
21.01	0\\
22.01	0\\
23.01	0\\
24.01	0\\
25.01	0\\
26.01	0\\
27.01	0\\
28.01	0\\
29.01	0\\
30.01	0\\
31.01	0\\
32.01	0\\
33.01	0\\
34.01	0\\
35.01	0\\
36.01	0\\
37.01	0\\
38.01	0\\
39.01	0\\
40.01	0\\
41.01	0\\
42.01	0\\
43.01	0\\
44.01	0\\
45.01	0\\
46.01	0\\
47.01	0\\
48.01	0\\
49.01	0\\
50.01	0\\
51.01	0\\
52.01	0\\
53.01	0\\
54.01	0\\
55.01	0\\
56.01	0\\
57.01	0\\
58.01	0\\
59.01	0\\
60.01	0\\
61.01	0\\
62.01	0\\
63.01	0\\
64.01	0\\
65.01	0\\
66.01	0\\
67.01	0\\
68.01	0\\
69.01	0\\
70.01	0\\
71.01	0\\
72.01	0\\
73.01	0\\
74.01	0\\
75.01	0\\
76.01	0\\
77.01	0\\
78.01	0\\
79.01	0\\
80.01	0\\
81.01	0\\
82.01	0\\
83.01	0\\
84.01	0\\
85.01	0\\
86.01	0\\
87.01	0\\
88.01	0\\
89.01	0\\
90.01	0\\
91.01	0\\
92.01	0\\
93.01	0\\
94.01	0\\
95.01	0\\
96.01	0\\
97.01	0\\
98.01	0\\
99.01	0\\
100.01	0\\
101.01	0\\
102.01	0\\
103.01	0\\
104.01	0\\
105.01	0\\
106.01	0\\
107.01	0\\
108.01	0\\
109.01	0\\
110.01	0\\
111.01	0\\
112.01	0\\
113.01	0\\
114.01	0\\
115.01	0\\
116.01	0\\
117.01	0\\
118.01	0\\
119.01	0\\
120.01	0\\
121.01	0\\
122.01	0\\
123.01	0\\
124.01	0\\
125.01	0\\
126.01	0\\
127.01	0\\
128.01	0\\
129.01	0\\
130.01	0\\
131.01	0\\
132.01	0\\
133.01	0\\
134.01	0\\
135.01	0\\
136.01	0\\
137.01	0\\
138.01	0\\
139.01	0\\
140.01	0\\
141.01	0\\
142.01	0\\
143.01	0\\
144.01	0\\
145.01	0\\
146.01	0\\
147.01	0\\
148.01	0\\
149.01	0\\
150.01	0\\
151.01	0\\
152.01	0\\
153.01	0\\
154.01	0\\
155.01	0\\
156.01	0\\
157.01	0\\
158.01	0\\
159.01	0\\
160.01	0\\
161.01	0\\
162.01	0\\
163.01	0\\
164.01	0\\
165.01	0\\
166.01	0\\
167.01	0\\
168.01	0\\
169.01	0\\
170.01	0\\
171.01	0\\
172.01	0\\
173.01	0\\
174.01	0\\
175.01	0\\
176.01	0\\
177.01	0\\
178.01	0\\
179.01	0\\
180.01	0\\
181.01	0\\
182.01	0\\
183.01	0\\
184.01	0\\
185.01	0\\
186.01	0\\
187.01	0\\
188.01	0\\
189.01	0\\
190.01	0\\
191.01	0\\
192.01	0\\
193.01	0\\
194.01	0\\
195.01	0\\
196.01	0\\
197.01	0\\
198.01	0\\
199.01	0\\
200.01	0\\
201.01	0\\
202.01	0\\
203.01	0\\
204.01	0\\
205.01	0\\
206.01	0\\
207.01	0\\
208.01	0\\
209.01	0\\
210.01	0\\
211.01	0\\
212.01	0\\
213.01	0\\
214.01	0\\
215.01	0\\
216.01	0\\
217.01	0\\
218.01	0\\
219.01	0\\
220.01	0\\
221.01	0\\
222.01	0\\
223.01	0\\
224.01	0\\
225.01	0\\
226.01	0\\
227.01	0\\
228.01	0\\
229.01	0\\
230.01	0\\
231.01	0\\
232.01	0\\
233.01	0\\
234.01	0\\
235.01	0\\
236.01	0\\
237.01	0\\
238.01	0\\
239.01	0\\
240.01	0\\
241.01	0\\
242.01	0\\
243.01	0\\
244.01	0\\
245.01	0\\
246.01	0\\
247.01	0\\
248.01	0\\
249.01	0\\
250.01	0\\
251.01	0\\
252.01	0\\
253.01	0\\
254.01	0\\
255.01	0\\
256.01	0\\
257.01	0\\
258.01	0\\
259.01	0\\
260.01	0\\
261.01	0\\
262.01	0\\
263.01	0\\
264.01	0\\
265.01	0\\
266.01	0\\
267.01	0\\
268.01	0\\
269.01	0\\
270.01	0\\
271.01	0\\
272.01	0\\
273.01	0\\
274.01	0\\
275.01	0\\
276.01	0\\
277.01	0\\
278.01	0\\
279.01	0\\
280.01	0\\
281.01	0\\
282.01	0\\
283.01	0\\
284.01	0\\
285.01	0\\
286.01	0\\
287.01	0\\
288.01	0\\
289.01	0\\
290.01	0\\
291.01	0\\
292.01	0\\
293.01	0\\
294.01	0\\
295.01	0\\
296.01	0\\
297.01	0\\
298.01	0\\
299.01	0\\
300.01	0\\
301.01	0\\
302.01	0\\
303.01	0\\
304.01	0\\
305.01	0\\
306.01	0\\
307.01	0\\
308.01	0\\
309.01	0\\
310.01	0\\
311.01	0\\
312.01	0\\
313.01	0\\
314.01	0\\
315.01	0\\
316.01	0\\
317.01	0\\
318.01	0\\
319.01	0\\
320.01	0\\
321.01	0\\
322.01	0\\
323.01	0\\
324.01	0\\
325.01	0\\
326.01	0\\
327.01	0\\
328.01	0\\
329.01	0\\
330.01	0\\
331.01	0\\
332.01	0\\
333.01	0\\
334.01	0\\
335.01	0\\
336.01	0\\
337.01	0\\
338.01	0\\
339.01	0\\
340.01	0\\
341.01	0\\
342.01	0\\
343.01	0\\
344.01	0\\
345.01	0\\
346.01	0\\
347.01	0\\
348.01	0\\
349.01	0\\
350.01	0\\
351.01	0\\
352.01	0\\
353.01	0\\
354.01	0\\
355.01	0\\
356.01	0\\
357.01	0\\
358.01	0\\
359.01	0\\
360.01	0\\
361.01	0\\
362.01	0\\
363.01	0\\
364.01	0\\
365.01	0\\
366.01	0\\
367.01	0\\
368.01	0\\
369.01	0\\
370.01	0\\
371.01	0\\
372.01	0\\
373.01	0\\
374.01	0\\
375.01	0\\
376.01	0\\
377.01	0\\
378.01	0\\
379.01	0\\
380.01	0\\
381.01	0\\
382.01	0\\
383.01	0\\
384.01	0\\
385.01	0\\
386.01	0\\
387.01	0\\
388.01	0\\
389.01	0\\
390.01	0\\
391.01	0\\
392.01	0\\
393.01	0\\
394.01	0\\
395.01	0\\
396.01	0\\
397.01	0\\
398.01	0\\
399.01	0\\
400.01	0\\
401.01	0\\
402.01	0\\
403.01	0\\
404.01	0\\
405.01	0\\
406.01	0\\
407.01	0\\
408.01	0\\
409.01	0\\
410.01	0\\
411.01	0\\
412.01	0\\
413.01	0\\
414.01	0\\
415.01	0\\
416.01	0\\
417.01	0\\
418.01	0\\
419.01	0\\
420.01	0\\
421.01	0\\
422.01	0\\
423.01	0\\
424.01	0\\
425.01	0\\
426.01	0\\
427.01	0\\
428.01	0\\
429.01	0\\
430.01	0\\
431.01	0\\
432.01	0\\
433.01	0\\
434.01	0\\
435.01	0\\
436.01	0\\
437.01	0\\
438.01	0\\
439.01	0\\
440.01	0\\
441.01	0\\
442.01	0\\
443.01	0\\
444.01	0\\
445.01	0\\
446.01	0\\
447.01	0\\
448.01	0\\
449.01	0\\
450.01	0\\
451.01	1.73472347597681e-18\\
452.01	0\\
453.01	0\\
454.01	0\\
455.01	0\\
456.01	0\\
457.01	0\\
458.01	0\\
459.01	0\\
460.01	0\\
461.01	0\\
462.01	0\\
463.01	0\\
464.01	0\\
465.01	0\\
466.01	0\\
467.01	0\\
468.01	0\\
469.01	1.73472347597681e-18\\
470.01	0\\
471.01	0\\
472.01	0\\
473.01	0\\
474.01	0\\
475.01	0\\
476.01	1.73472347597681e-18\\
477.01	0\\
478.01	0\\
479.01	0\\
480.01	0\\
481.01	0\\
482.01	0\\
483.01	0\\
484.01	0\\
485.01	0\\
486.01	0\\
487.01	0\\
488.01	0\\
489.01	0\\
490.01	0\\
491.01	0\\
492.01	0\\
493.01	0\\
494.01	0\\
495.01	0\\
496.01	0\\
497.01	0\\
498.01	0\\
499.01	0\\
500.01	0\\
501.01	0\\
502.01	0\\
503.01	0\\
504.01	0\\
505.01	0\\
506.01	0\\
507.01	0\\
508.01	0\\
509.01	0\\
510.01	0\\
511.01	0\\
512.01	1.73472347597681e-18\\
513.01	0\\
514.01	0\\
515.01	0\\
516.01	0\\
517.01	0\\
518.01	0\\
519.01	0\\
520.01	0\\
521.01	0\\
522.01	0\\
523.01	1.73472347597681e-18\\
524.01	0\\
525.01	0\\
526.01	0\\
527.01	0\\
528.01	0\\
529.01	0\\
530.01	0\\
531.01	0\\
532.01	0\\
533.01	0\\
534.01	0\\
535.01	0\\
536.01	1.73472347597681e-18\\
537.01	1.73472347597681e-18\\
538.01	1.73472347597681e-18\\
539.01	0\\
540.01	0\\
541.01	0\\
542.01	0\\
543.01	0\\
544.01	0\\
545.01	1.73472347597681e-18\\
546.01	1.73472347597681e-18\\
547.01	0\\
548.01	0\\
549.01	0\\
550.01	0\\
551.01	0\\
552.01	0\\
553.01	0\\
554.01	0\\
555.01	0\\
556.01	1.73472347597681e-18\\
557.01	0\\
558.01	0\\
559.01	0\\
560.01	1.73472347597681e-18\\
561.01	1.73472347597681e-18\\
562.01	0\\
563.01	0\\
564.01	0\\
565.01	0\\
566.01	0\\
567.01	0\\
568.01	0\\
569.01	0\\
570.01	0\\
571.01	0\\
572.01	0\\
573.01	0\\
574.01	0\\
575.01	0\\
576.01	0\\
577.01	0\\
578.01	1.73472347597681e-18\\
579.01	0\\
580.01	0\\
581.01	0\\
582.01	0\\
583.01	1.73472347597681e-18\\
584.01	0\\
585.01	0\\
586.01	0\\
587.01	0\\
588.01	0\\
589.01	0\\
590.01	0\\
591.01	0\\
592.01	0\\
593.01	0\\
594.01	0\\
595.01	0\\
596.01	0\\
597.01	0\\
598.01	0\\
599.01	0\\
599.02	0\\
599.03	0\\
599.04	0\\
599.05	0\\
599.06	0\\
599.07	0\\
599.08	0\\
599.09	0\\
599.1	0\\
599.11	0\\
599.12	0\\
599.13	0\\
599.14	0\\
599.15	0\\
599.16	0\\
599.17	0\\
599.18	0\\
599.19	0\\
599.2	0\\
599.21	0\\
599.22	0\\
599.23	0\\
599.24	0\\
599.25	0\\
599.26	0\\
599.27	0\\
599.28	0\\
599.29	0\\
599.3	0\\
599.31	0\\
599.32	0\\
599.33	0\\
599.34	0\\
599.35	0\\
599.36	0\\
599.37	0\\
599.38	0\\
599.39	0\\
599.4	0\\
599.41	0\\
599.42	0\\
599.43	0\\
599.44	0\\
599.45	0\\
599.46	0\\
599.47	0\\
599.48	0\\
599.49	0\\
599.5	0\\
599.51	0\\
599.52	0\\
599.53	0\\
599.54	0\\
599.55	0\\
599.56	0\\
599.57	0\\
599.58	0\\
599.59	0\\
599.6	0\\
599.61	0\\
599.62	0\\
599.63	0\\
599.64	0\\
599.65	0\\
599.66	0\\
599.67	0\\
599.68	0\\
599.69	0\\
599.7	0\\
599.71	0\\
599.72	0\\
599.73	0\\
599.74	0\\
599.75	0\\
599.76	0\\
599.77	0\\
599.78	0\\
599.79	0\\
599.8	0\\
599.81	0\\
599.82	0\\
599.83	0\\
599.84	0\\
599.85	0\\
599.86	0\\
599.87	0\\
599.88	0\\
599.89	0\\
599.9	0\\
599.91	0\\
599.92	0\\
599.93	0\\
599.94	0\\
599.95	0\\
599.96	0\\
599.97	0\\
599.98	0\\
599.99	0\\
600	0\\
};
\addplot [color=mycolor1,solid,forget plot]
  table[row sep=crcr]{%
0.01	0\\
1.01	0\\
2.01	0\\
3.01	0\\
4.01	0\\
5.01	0\\
6.01	0\\
7.01	0\\
8.01	0\\
9.01	0\\
10.01	0\\
11.01	0\\
12.01	0\\
13.01	0\\
14.01	0\\
15.01	0\\
16.01	0\\
17.01	0\\
18.01	0\\
19.01	0\\
20.01	0\\
21.01	0\\
22.01	0\\
23.01	0\\
24.01	0\\
25.01	0\\
26.01	0\\
27.01	0\\
28.01	0\\
29.01	0\\
30.01	0\\
31.01	0\\
32.01	0\\
33.01	0\\
34.01	0\\
35.01	0\\
36.01	0\\
37.01	0\\
38.01	0\\
39.01	0\\
40.01	0\\
41.01	0\\
42.01	0\\
43.01	0\\
44.01	0\\
45.01	0\\
46.01	0\\
47.01	0\\
48.01	0\\
49.01	0\\
50.01	0\\
51.01	0\\
52.01	0\\
53.01	0\\
54.01	0\\
55.01	0\\
56.01	0\\
57.01	0\\
58.01	0\\
59.01	0\\
60.01	0\\
61.01	0\\
62.01	0\\
63.01	0\\
64.01	0\\
65.01	0\\
66.01	0\\
67.01	0\\
68.01	0\\
69.01	0\\
70.01	0\\
71.01	0\\
72.01	0\\
73.01	0\\
74.01	0\\
75.01	0\\
76.01	0\\
77.01	0\\
78.01	0\\
79.01	0\\
80.01	0\\
81.01	0\\
82.01	0\\
83.01	0\\
84.01	0\\
85.01	0\\
86.01	0\\
87.01	0\\
88.01	0\\
89.01	0\\
90.01	0\\
91.01	0\\
92.01	0\\
93.01	0\\
94.01	0\\
95.01	0\\
96.01	0\\
97.01	0\\
98.01	0\\
99.01	0\\
100.01	0\\
101.01	0\\
102.01	0\\
103.01	0\\
104.01	0\\
105.01	0\\
106.01	0\\
107.01	0\\
108.01	0\\
109.01	0\\
110.01	0\\
111.01	0\\
112.01	0\\
113.01	0\\
114.01	0\\
115.01	0\\
116.01	0\\
117.01	0\\
118.01	0\\
119.01	0\\
120.01	0\\
121.01	0\\
122.01	0\\
123.01	0\\
124.01	0\\
125.01	0\\
126.01	0\\
127.01	0\\
128.01	0\\
129.01	0\\
130.01	0\\
131.01	0\\
132.01	0\\
133.01	0\\
134.01	0\\
135.01	0\\
136.01	0\\
137.01	0\\
138.01	0\\
139.01	0\\
140.01	0\\
141.01	0\\
142.01	0\\
143.01	0\\
144.01	0\\
145.01	0\\
146.01	0\\
147.01	0\\
148.01	0\\
149.01	0\\
150.01	0\\
151.01	0\\
152.01	0\\
153.01	0\\
154.01	0\\
155.01	0\\
156.01	0\\
157.01	0\\
158.01	0\\
159.01	0\\
160.01	0\\
161.01	0\\
162.01	0\\
163.01	0\\
164.01	0\\
165.01	0\\
166.01	0\\
167.01	0\\
168.01	0\\
169.01	0\\
170.01	0\\
171.01	0\\
172.01	0\\
173.01	0\\
174.01	0\\
175.01	0\\
176.01	0\\
177.01	0\\
178.01	0\\
179.01	0\\
180.01	0\\
181.01	0\\
182.01	0\\
183.01	0\\
184.01	0\\
185.01	0\\
186.01	0\\
187.01	0\\
188.01	0\\
189.01	0\\
190.01	0\\
191.01	0\\
192.01	0\\
193.01	0\\
194.01	0\\
195.01	0\\
196.01	0\\
197.01	0\\
198.01	0\\
199.01	0\\
200.01	0\\
201.01	0\\
202.01	0\\
203.01	0\\
204.01	0\\
205.01	0\\
206.01	0\\
207.01	0\\
208.01	0\\
209.01	0\\
210.01	0\\
211.01	0\\
212.01	0\\
213.01	0\\
214.01	0\\
215.01	0\\
216.01	0\\
217.01	0\\
218.01	0\\
219.01	0\\
220.01	0\\
221.01	0\\
222.01	0\\
223.01	0\\
224.01	0\\
225.01	0\\
226.01	0\\
227.01	0\\
228.01	0\\
229.01	0\\
230.01	0\\
231.01	0\\
232.01	0\\
233.01	0\\
234.01	0\\
235.01	0\\
236.01	0\\
237.01	0\\
238.01	0\\
239.01	0\\
240.01	0\\
241.01	0\\
242.01	0\\
243.01	0\\
244.01	0\\
245.01	0\\
246.01	0\\
247.01	0\\
248.01	0\\
249.01	0\\
250.01	0\\
251.01	0\\
252.01	0\\
253.01	0\\
254.01	0\\
255.01	0\\
256.01	0\\
257.01	0\\
258.01	0\\
259.01	0\\
260.01	0\\
261.01	0\\
262.01	0\\
263.01	0\\
264.01	0\\
265.01	0\\
266.01	0\\
267.01	0\\
268.01	0\\
269.01	0\\
270.01	0\\
271.01	0\\
272.01	0\\
273.01	0\\
274.01	0\\
275.01	0\\
276.01	0\\
277.01	0\\
278.01	0\\
279.01	0\\
280.01	0\\
281.01	0\\
282.01	0\\
283.01	0\\
284.01	0\\
285.01	0\\
286.01	0\\
287.01	0\\
288.01	0\\
289.01	0\\
290.01	0\\
291.01	0\\
292.01	0\\
293.01	0\\
294.01	0\\
295.01	0\\
296.01	0\\
297.01	0\\
298.01	0\\
299.01	0\\
300.01	0\\
301.01	0\\
302.01	0\\
303.01	0\\
304.01	0\\
305.01	0\\
306.01	0\\
307.01	0\\
308.01	0\\
309.01	0\\
310.01	0\\
311.01	0\\
312.01	0\\
313.01	0\\
314.01	0\\
315.01	0\\
316.01	0\\
317.01	0\\
318.01	0\\
319.01	0\\
320.01	0\\
321.01	0\\
322.01	0\\
323.01	0\\
324.01	0\\
325.01	0\\
326.01	0\\
327.01	0\\
328.01	0\\
329.01	0\\
330.01	0\\
331.01	0\\
332.01	0\\
333.01	0\\
334.01	0\\
335.01	0\\
336.01	0\\
337.01	0\\
338.01	0\\
339.01	0\\
340.01	0\\
341.01	0\\
342.01	0\\
343.01	0\\
344.01	0\\
345.01	0\\
346.01	0\\
347.01	0\\
348.01	0\\
349.01	0\\
350.01	0\\
351.01	0\\
352.01	0\\
353.01	0\\
354.01	0\\
355.01	0\\
356.01	0\\
357.01	0\\
358.01	0\\
359.01	0\\
360.01	0\\
361.01	0\\
362.01	0\\
363.01	0\\
364.01	0\\
365.01	0\\
366.01	0\\
367.01	0\\
368.01	0\\
369.01	0\\
370.01	0\\
371.01	0\\
372.01	0\\
373.01	0\\
374.01	0\\
375.01	0\\
376.01	0\\
377.01	0\\
378.01	0\\
379.01	0\\
380.01	0\\
381.01	0\\
382.01	0\\
383.01	0\\
384.01	0\\
385.01	0\\
386.01	0\\
387.01	0\\
388.01	0\\
389.01	0\\
390.01	0\\
391.01	0\\
392.01	0\\
393.01	0\\
394.01	0\\
395.01	0\\
396.01	0\\
397.01	0\\
398.01	0\\
399.01	0\\
400.01	0\\
401.01	0\\
402.01	0\\
403.01	0\\
404.01	0\\
405.01	0\\
406.01	0\\
407.01	0\\
408.01	0\\
409.01	0\\
410.01	0\\
411.01	0\\
412.01	0\\
413.01	0\\
414.01	0\\
415.01	0\\
416.01	0\\
417.01	0\\
418.01	0\\
419.01	0\\
420.01	0\\
421.01	0\\
422.01	0\\
423.01	0\\
424.01	0\\
425.01	0\\
426.01	0\\
427.01	0\\
428.01	0\\
429.01	0\\
430.01	0\\
431.01	0\\
432.01	0\\
433.01	0\\
434.01	0\\
435.01	0\\
436.01	0\\
437.01	0\\
438.01	0\\
439.01	0\\
440.01	0\\
441.01	0\\
442.01	0\\
443.01	0\\
444.01	0\\
445.01	0\\
446.01	0\\
447.01	0\\
448.01	0\\
449.01	0\\
450.01	0\\
451.01	1.73472347597681e-18\\
452.01	0\\
453.01	0\\
454.01	0\\
455.01	0\\
456.01	0\\
457.01	0\\
458.01	0\\
459.01	0\\
460.01	0\\
461.01	0\\
462.01	0\\
463.01	0\\
464.01	0\\
465.01	0\\
466.01	0\\
467.01	0\\
468.01	0\\
469.01	1.73472347597681e-18\\
470.01	0\\
471.01	0\\
472.01	0\\
473.01	0\\
474.01	0\\
475.01	0\\
476.01	1.73472347597681e-18\\
477.01	0\\
478.01	0\\
479.01	0\\
480.01	0\\
481.01	0\\
482.01	0\\
483.01	0\\
484.01	0\\
485.01	0\\
486.01	0\\
487.01	0\\
488.01	0\\
489.01	0\\
490.01	0\\
491.01	0\\
492.01	0\\
493.01	0\\
494.01	0\\
495.01	0\\
496.01	0\\
497.01	0\\
498.01	0\\
499.01	0\\
500.01	0\\
501.01	0\\
502.01	0\\
503.01	0\\
504.01	0\\
505.01	0\\
506.01	0\\
507.01	0\\
508.01	0\\
509.01	0\\
510.01	0\\
511.01	0\\
512.01	1.73472347597681e-18\\
513.01	0\\
514.01	0\\
515.01	0\\
516.01	0\\
517.01	0\\
518.01	0\\
519.01	0\\
520.01	0\\
521.01	0\\
522.01	0\\
523.01	1.73472347597681e-18\\
524.01	0\\
525.01	0\\
526.01	0\\
527.01	0\\
528.01	0\\
529.01	0\\
530.01	0\\
531.01	0\\
532.01	0\\
533.01	0\\
534.01	0\\
535.01	0\\
536.01	1.73472347597681e-18\\
537.01	1.73472347597681e-18\\
538.01	1.73472347597681e-18\\
539.01	0\\
540.01	0\\
541.01	0\\
542.01	0\\
543.01	0\\
544.01	0\\
545.01	1.73472347597681e-18\\
546.01	1.73472347597681e-18\\
547.01	0\\
548.01	0\\
549.01	0\\
550.01	0\\
551.01	0\\
552.01	0\\
553.01	0\\
554.01	0\\
555.01	0\\
556.01	1.73472347597681e-18\\
557.01	0\\
558.01	0\\
559.01	0\\
560.01	1.73472347597681e-18\\
561.01	1.73472347597681e-18\\
562.01	0\\
563.01	0\\
564.01	0\\
565.01	0\\
566.01	0\\
567.01	0\\
568.01	0\\
569.01	0\\
570.01	0\\
571.01	0\\
572.01	0\\
573.01	0\\
574.01	0\\
575.01	0\\
576.01	0\\
577.01	0\\
578.01	1.73472347597681e-18\\
579.01	0\\
580.01	0\\
581.01	0\\
582.01	0\\
583.01	1.73472347597681e-18\\
584.01	0\\
585.01	0\\
586.01	0\\
587.01	0\\
588.01	0\\
589.01	0\\
590.01	0\\
591.01	0\\
592.01	0\\
593.01	0\\
594.01	0\\
595.01	0\\
596.01	0\\
597.01	0\\
598.01	0\\
599.01	0\\
599.02	0\\
599.03	0\\
599.04	0\\
599.05	0\\
599.06	0\\
599.07	0\\
599.08	0\\
599.09	0\\
599.1	0\\
599.11	0\\
599.12	0\\
599.13	0\\
599.14	0\\
599.15	0\\
599.16	0\\
599.17	0\\
599.18	0\\
599.19	0\\
599.2	0\\
599.21	0\\
599.22	0\\
599.23	0\\
599.24	0\\
599.25	0\\
599.26	0\\
599.27	0\\
599.28	0\\
599.29	0\\
599.3	0\\
599.31	0\\
599.32	0\\
599.33	0\\
599.34	0\\
599.35	0\\
599.36	0\\
599.37	0\\
599.38	0\\
599.39	0\\
599.4	0\\
599.41	0\\
599.42	0\\
599.43	0\\
599.44	0\\
599.45	0\\
599.46	0\\
599.47	0\\
599.48	0\\
599.49	0\\
599.5	0\\
599.51	0\\
599.52	0\\
599.53	0\\
599.54	0\\
599.55	0\\
599.56	0\\
599.57	0\\
599.58	0\\
599.59	0\\
599.6	0\\
599.61	0\\
599.62	0\\
599.63	0\\
599.64	0\\
599.65	0\\
599.66	0\\
599.67	0\\
599.68	0\\
599.69	0\\
599.7	0\\
599.71	0\\
599.72	0\\
599.73	0\\
599.74	0\\
599.75	0\\
599.76	0\\
599.77	0\\
599.78	0\\
599.79	0\\
599.8	0\\
599.81	0\\
599.82	0\\
599.83	0\\
599.84	0\\
599.85	0\\
599.86	0\\
599.87	0\\
599.88	0\\
599.89	0\\
599.9	0\\
599.91	0\\
599.92	0\\
599.93	0\\
599.94	0\\
599.95	0\\
599.96	0\\
599.97	0\\
599.98	0\\
599.99	0\\
600	0\\
};
\addplot [color=mycolor2,solid,forget plot]
  table[row sep=crcr]{%
0.01	0\\
1.01	0\\
2.01	0\\
3.01	0\\
4.01	0\\
5.01	0\\
6.01	0\\
7.01	0\\
8.01	0\\
9.01	0\\
10.01	0\\
11.01	0\\
12.01	0\\
13.01	0\\
14.01	0\\
15.01	0\\
16.01	0\\
17.01	0\\
18.01	0\\
19.01	0\\
20.01	0\\
21.01	0\\
22.01	0\\
23.01	0\\
24.01	0\\
25.01	0\\
26.01	0\\
27.01	0\\
28.01	0\\
29.01	0\\
30.01	0\\
31.01	0\\
32.01	0\\
33.01	0\\
34.01	0\\
35.01	0\\
36.01	0\\
37.01	0\\
38.01	0\\
39.01	0\\
40.01	0\\
41.01	0\\
42.01	0\\
43.01	0\\
44.01	0\\
45.01	0\\
46.01	0\\
47.01	0\\
48.01	0\\
49.01	0\\
50.01	0\\
51.01	0\\
52.01	0\\
53.01	0\\
54.01	0\\
55.01	0\\
56.01	0\\
57.01	0\\
58.01	0\\
59.01	0\\
60.01	0\\
61.01	0\\
62.01	0\\
63.01	0\\
64.01	0\\
65.01	0\\
66.01	0\\
67.01	0\\
68.01	0\\
69.01	0\\
70.01	0\\
71.01	0\\
72.01	0\\
73.01	0\\
74.01	0\\
75.01	0\\
76.01	0\\
77.01	0\\
78.01	0\\
79.01	0\\
80.01	0\\
81.01	0\\
82.01	0\\
83.01	0\\
84.01	0\\
85.01	0\\
86.01	0\\
87.01	0\\
88.01	0\\
89.01	0\\
90.01	0\\
91.01	0\\
92.01	0\\
93.01	0\\
94.01	0\\
95.01	0\\
96.01	0\\
97.01	0\\
98.01	0\\
99.01	0\\
100.01	0\\
101.01	0\\
102.01	0\\
103.01	0\\
104.01	0\\
105.01	0\\
106.01	0\\
107.01	0\\
108.01	0\\
109.01	0\\
110.01	0\\
111.01	0\\
112.01	0\\
113.01	0\\
114.01	0\\
115.01	0\\
116.01	0\\
117.01	0\\
118.01	0\\
119.01	0\\
120.01	0\\
121.01	0\\
122.01	0\\
123.01	0\\
124.01	0\\
125.01	0\\
126.01	0\\
127.01	0\\
128.01	0\\
129.01	0\\
130.01	0\\
131.01	0\\
132.01	0\\
133.01	0\\
134.01	0\\
135.01	0\\
136.01	0\\
137.01	0\\
138.01	0\\
139.01	0\\
140.01	0\\
141.01	0\\
142.01	0\\
143.01	0\\
144.01	0\\
145.01	0\\
146.01	0\\
147.01	0\\
148.01	0\\
149.01	0\\
150.01	0\\
151.01	0\\
152.01	0\\
153.01	0\\
154.01	0\\
155.01	0\\
156.01	0\\
157.01	0\\
158.01	0\\
159.01	0\\
160.01	0\\
161.01	0\\
162.01	0\\
163.01	0\\
164.01	0\\
165.01	0\\
166.01	0\\
167.01	0\\
168.01	0\\
169.01	0\\
170.01	0\\
171.01	0\\
172.01	0\\
173.01	0\\
174.01	0\\
175.01	0\\
176.01	0\\
177.01	0\\
178.01	0\\
179.01	0\\
180.01	0\\
181.01	0\\
182.01	0\\
183.01	0\\
184.01	0\\
185.01	0\\
186.01	0\\
187.01	0\\
188.01	0\\
189.01	0\\
190.01	0\\
191.01	0\\
192.01	0\\
193.01	0\\
194.01	0\\
195.01	0\\
196.01	0\\
197.01	0\\
198.01	0\\
199.01	0\\
200.01	0\\
201.01	0\\
202.01	0\\
203.01	0\\
204.01	0\\
205.01	0\\
206.01	0\\
207.01	0\\
208.01	0\\
209.01	0\\
210.01	0\\
211.01	0\\
212.01	0\\
213.01	0\\
214.01	0\\
215.01	0\\
216.01	0\\
217.01	0\\
218.01	0\\
219.01	0\\
220.01	0\\
221.01	0\\
222.01	0\\
223.01	0\\
224.01	0\\
225.01	0\\
226.01	0\\
227.01	0\\
228.01	0\\
229.01	0\\
230.01	0\\
231.01	0\\
232.01	0\\
233.01	0\\
234.01	0\\
235.01	0\\
236.01	0\\
237.01	0\\
238.01	0\\
239.01	0\\
240.01	0\\
241.01	0\\
242.01	0\\
243.01	0\\
244.01	0\\
245.01	0\\
246.01	0\\
247.01	0\\
248.01	0\\
249.01	0\\
250.01	0\\
251.01	0\\
252.01	0\\
253.01	0\\
254.01	0\\
255.01	0\\
256.01	0\\
257.01	0\\
258.01	0\\
259.01	0\\
260.01	0\\
261.01	0\\
262.01	0\\
263.01	0\\
264.01	0\\
265.01	0\\
266.01	0\\
267.01	0\\
268.01	0\\
269.01	0\\
270.01	0\\
271.01	0\\
272.01	0\\
273.01	0\\
274.01	0\\
275.01	0\\
276.01	0\\
277.01	0\\
278.01	0\\
279.01	0\\
280.01	0\\
281.01	0\\
282.01	0\\
283.01	0\\
284.01	0\\
285.01	0\\
286.01	0\\
287.01	0\\
288.01	0\\
289.01	0\\
290.01	0\\
291.01	0\\
292.01	0\\
293.01	0\\
294.01	0\\
295.01	0\\
296.01	0\\
297.01	0\\
298.01	0\\
299.01	0\\
300.01	0\\
301.01	0\\
302.01	0\\
303.01	0\\
304.01	0\\
305.01	0\\
306.01	0\\
307.01	0\\
308.01	0\\
309.01	0\\
310.01	0\\
311.01	0\\
312.01	0\\
313.01	0\\
314.01	0\\
315.01	0\\
316.01	0\\
317.01	0\\
318.01	0\\
319.01	0\\
320.01	0\\
321.01	0\\
322.01	0\\
323.01	0\\
324.01	0\\
325.01	0\\
326.01	0\\
327.01	0\\
328.01	0\\
329.01	0\\
330.01	0\\
331.01	0\\
332.01	0\\
333.01	0\\
334.01	0\\
335.01	0\\
336.01	0\\
337.01	0\\
338.01	0\\
339.01	0\\
340.01	0\\
341.01	0\\
342.01	0\\
343.01	0\\
344.01	0\\
345.01	0\\
346.01	0\\
347.01	0\\
348.01	0\\
349.01	0\\
350.01	0\\
351.01	0\\
352.01	0\\
353.01	0\\
354.01	0\\
355.01	0\\
356.01	0\\
357.01	0\\
358.01	0\\
359.01	0\\
360.01	0\\
361.01	0\\
362.01	0\\
363.01	0\\
364.01	0\\
365.01	0\\
366.01	0\\
367.01	0\\
368.01	0\\
369.01	0\\
370.01	0\\
371.01	0\\
372.01	0\\
373.01	0\\
374.01	0\\
375.01	0\\
376.01	0\\
377.01	0\\
378.01	0\\
379.01	0\\
380.01	0\\
381.01	0\\
382.01	0\\
383.01	0\\
384.01	0\\
385.01	0\\
386.01	0\\
387.01	0\\
388.01	0\\
389.01	0\\
390.01	0\\
391.01	0\\
392.01	0\\
393.01	0\\
394.01	0\\
395.01	0\\
396.01	0\\
397.01	0\\
398.01	0\\
399.01	0\\
400.01	0\\
401.01	0\\
402.01	0\\
403.01	0\\
404.01	0\\
405.01	0\\
406.01	0\\
407.01	0\\
408.01	0\\
409.01	0\\
410.01	0\\
411.01	0\\
412.01	0\\
413.01	0\\
414.01	0\\
415.01	0\\
416.01	0\\
417.01	0\\
418.01	0\\
419.01	0\\
420.01	0\\
421.01	0\\
422.01	0\\
423.01	0\\
424.01	0\\
425.01	0\\
426.01	0\\
427.01	0\\
428.01	0\\
429.01	0\\
430.01	0\\
431.01	0\\
432.01	0\\
433.01	0\\
434.01	0\\
435.01	0\\
436.01	0\\
437.01	0\\
438.01	0\\
439.01	0\\
440.01	0\\
441.01	0\\
442.01	0\\
443.01	0\\
444.01	0\\
445.01	0\\
446.01	0\\
447.01	0\\
448.01	0\\
449.01	0\\
450.01	0\\
451.01	1.73472347597681e-18\\
452.01	0\\
453.01	0\\
454.01	0\\
455.01	0\\
456.01	0\\
457.01	0\\
458.01	0\\
459.01	0\\
460.01	0\\
461.01	0\\
462.01	0\\
463.01	0\\
464.01	0\\
465.01	0\\
466.01	0\\
467.01	0\\
468.01	0\\
469.01	1.73472347597681e-18\\
470.01	0\\
471.01	0\\
472.01	0\\
473.01	0\\
474.01	0\\
475.01	0\\
476.01	1.73472347597681e-18\\
477.01	0\\
478.01	0\\
479.01	0\\
480.01	0\\
481.01	0\\
482.01	0\\
483.01	0\\
484.01	0\\
485.01	0\\
486.01	0\\
487.01	0\\
488.01	0\\
489.01	0\\
490.01	0\\
491.01	0\\
492.01	0\\
493.01	0\\
494.01	0\\
495.01	0\\
496.01	0\\
497.01	0\\
498.01	0\\
499.01	0\\
500.01	0\\
501.01	0\\
502.01	0\\
503.01	0\\
504.01	0\\
505.01	0\\
506.01	0\\
507.01	0\\
508.01	0\\
509.01	0\\
510.01	0\\
511.01	0\\
512.01	1.73472347597681e-18\\
513.01	0\\
514.01	0\\
515.01	0\\
516.01	0\\
517.01	0\\
518.01	0\\
519.01	0\\
520.01	0\\
521.01	0\\
522.01	0\\
523.01	1.73472347597681e-18\\
524.01	0\\
525.01	0\\
526.01	0\\
527.01	0\\
528.01	0\\
529.01	0\\
530.01	0\\
531.01	0\\
532.01	0\\
533.01	0\\
534.01	0\\
535.01	0\\
536.01	1.73472347597681e-18\\
537.01	1.73472347597681e-18\\
538.01	1.73472347597681e-18\\
539.01	0\\
540.01	0\\
541.01	0\\
542.01	0\\
543.01	0\\
544.01	0\\
545.01	1.73472347597681e-18\\
546.01	1.73472347597681e-18\\
547.01	0\\
548.01	0\\
549.01	0\\
550.01	0\\
551.01	0\\
552.01	0\\
553.01	0\\
554.01	0\\
555.01	0\\
556.01	1.73472347597681e-18\\
557.01	0\\
558.01	0\\
559.01	0\\
560.01	1.73472347597681e-18\\
561.01	1.73472347597681e-18\\
562.01	0\\
563.01	0\\
564.01	0\\
565.01	0\\
566.01	0\\
567.01	0\\
568.01	0\\
569.01	0\\
570.01	0\\
571.01	0\\
572.01	0\\
573.01	0\\
574.01	0\\
575.01	0\\
576.01	0\\
577.01	0\\
578.01	1.73472347597681e-18\\
579.01	0\\
580.01	0\\
581.01	0\\
582.01	0\\
583.01	1.73472347597681e-18\\
584.01	0\\
585.01	0\\
586.01	0\\
587.01	0\\
588.01	0\\
589.01	0\\
590.01	0\\
591.01	0\\
592.01	0\\
593.01	0\\
594.01	0\\
595.01	0\\
596.01	0\\
597.01	0\\
598.01	0\\
599.01	0\\
599.02	0\\
599.03	0\\
599.04	0\\
599.05	0\\
599.06	0\\
599.07	0\\
599.08	0\\
599.09	0\\
599.1	0\\
599.11	0\\
599.12	0\\
599.13	0\\
599.14	0\\
599.15	0\\
599.16	0\\
599.17	0\\
599.18	0\\
599.19	0\\
599.2	0\\
599.21	0\\
599.22	0\\
599.23	0\\
599.24	0\\
599.25	0\\
599.26	0\\
599.27	0\\
599.28	0\\
599.29	0\\
599.3	0\\
599.31	0\\
599.32	0\\
599.33	0\\
599.34	0\\
599.35	0\\
599.36	0\\
599.37	0\\
599.38	0\\
599.39	0\\
599.4	0\\
599.41	0\\
599.42	0\\
599.43	0\\
599.44	0\\
599.45	0\\
599.46	0\\
599.47	0\\
599.48	0\\
599.49	0\\
599.5	0\\
599.51	0\\
599.52	0\\
599.53	0\\
599.54	0\\
599.55	0\\
599.56	0\\
599.57	0\\
599.58	0\\
599.59	0\\
599.6	0\\
599.61	0\\
599.62	0\\
599.63	0\\
599.64	0\\
599.65	0\\
599.66	0\\
599.67	0\\
599.68	0\\
599.69	0\\
599.7	0\\
599.71	0\\
599.72	0\\
599.73	0\\
599.74	0\\
599.75	0\\
599.76	0\\
599.77	0\\
599.78	0\\
599.79	0\\
599.8	0\\
599.81	0\\
599.82	0\\
599.83	0\\
599.84	0\\
599.85	0\\
599.86	0\\
599.87	0\\
599.88	0\\
599.89	0\\
599.9	0\\
599.91	0\\
599.92	0\\
599.93	0\\
599.94	0\\
599.95	0\\
599.96	0\\
599.97	0\\
599.98	0\\
599.99	0\\
600	0\\
};
\addplot [color=mycolor3,solid,forget plot]
  table[row sep=crcr]{%
0.01	0\\
1.01	0\\
2.01	0\\
3.01	0\\
4.01	0\\
5.01	0\\
6.01	0\\
7.01	0\\
8.01	0\\
9.01	0\\
10.01	0\\
11.01	0\\
12.01	0\\
13.01	0\\
14.01	0\\
15.01	0\\
16.01	0\\
17.01	0\\
18.01	0\\
19.01	0\\
20.01	0\\
21.01	0\\
22.01	0\\
23.01	0\\
24.01	0\\
25.01	0\\
26.01	0\\
27.01	0\\
28.01	0\\
29.01	0\\
30.01	0\\
31.01	0\\
32.01	0\\
33.01	0\\
34.01	0\\
35.01	0\\
36.01	0\\
37.01	0\\
38.01	0\\
39.01	0\\
40.01	0\\
41.01	0\\
42.01	0\\
43.01	0\\
44.01	0\\
45.01	0\\
46.01	0\\
47.01	0\\
48.01	0\\
49.01	0\\
50.01	0\\
51.01	0\\
52.01	0\\
53.01	0\\
54.01	0\\
55.01	0\\
56.01	0\\
57.01	0\\
58.01	0\\
59.01	0\\
60.01	0\\
61.01	0\\
62.01	0\\
63.01	0\\
64.01	0\\
65.01	0\\
66.01	0\\
67.01	0\\
68.01	0\\
69.01	0\\
70.01	0\\
71.01	0\\
72.01	0\\
73.01	0\\
74.01	0\\
75.01	0\\
76.01	0\\
77.01	0\\
78.01	0\\
79.01	0\\
80.01	0\\
81.01	0\\
82.01	0\\
83.01	0\\
84.01	0\\
85.01	0\\
86.01	0\\
87.01	0\\
88.01	0\\
89.01	0\\
90.01	0\\
91.01	0\\
92.01	0\\
93.01	0\\
94.01	0\\
95.01	0\\
96.01	0\\
97.01	0\\
98.01	0\\
99.01	0\\
100.01	0\\
101.01	0\\
102.01	0\\
103.01	0\\
104.01	0\\
105.01	0\\
106.01	0\\
107.01	0\\
108.01	0\\
109.01	0\\
110.01	0\\
111.01	0\\
112.01	0\\
113.01	0\\
114.01	0\\
115.01	0\\
116.01	0\\
117.01	0\\
118.01	0\\
119.01	0\\
120.01	0\\
121.01	0\\
122.01	0\\
123.01	0\\
124.01	0\\
125.01	0\\
126.01	0\\
127.01	0\\
128.01	0\\
129.01	0\\
130.01	0\\
131.01	0\\
132.01	0\\
133.01	0\\
134.01	0\\
135.01	0\\
136.01	0\\
137.01	0\\
138.01	0\\
139.01	0\\
140.01	0\\
141.01	0\\
142.01	0\\
143.01	0\\
144.01	0\\
145.01	0\\
146.01	0\\
147.01	0\\
148.01	0\\
149.01	0\\
150.01	0\\
151.01	0\\
152.01	0\\
153.01	0\\
154.01	0\\
155.01	0\\
156.01	0\\
157.01	0\\
158.01	0\\
159.01	0\\
160.01	0\\
161.01	0\\
162.01	0\\
163.01	0\\
164.01	0\\
165.01	0\\
166.01	0\\
167.01	0\\
168.01	0\\
169.01	0\\
170.01	0\\
171.01	0\\
172.01	0\\
173.01	0\\
174.01	0\\
175.01	0\\
176.01	0\\
177.01	0\\
178.01	0\\
179.01	0\\
180.01	0\\
181.01	0\\
182.01	0\\
183.01	0\\
184.01	0\\
185.01	0\\
186.01	0\\
187.01	0\\
188.01	0\\
189.01	0\\
190.01	0\\
191.01	0\\
192.01	0\\
193.01	0\\
194.01	0\\
195.01	0\\
196.01	0\\
197.01	0\\
198.01	0\\
199.01	0\\
200.01	0\\
201.01	0\\
202.01	0\\
203.01	0\\
204.01	0\\
205.01	0\\
206.01	0\\
207.01	0\\
208.01	0\\
209.01	0\\
210.01	0\\
211.01	0\\
212.01	0\\
213.01	0\\
214.01	0\\
215.01	0\\
216.01	0\\
217.01	0\\
218.01	0\\
219.01	0\\
220.01	0\\
221.01	0\\
222.01	0\\
223.01	0\\
224.01	0\\
225.01	0\\
226.01	0\\
227.01	0\\
228.01	0\\
229.01	0\\
230.01	0\\
231.01	0\\
232.01	0\\
233.01	0\\
234.01	0\\
235.01	0\\
236.01	0\\
237.01	0\\
238.01	0\\
239.01	0\\
240.01	0\\
241.01	0\\
242.01	0\\
243.01	0\\
244.01	0\\
245.01	0\\
246.01	0\\
247.01	0\\
248.01	0\\
249.01	0\\
250.01	0\\
251.01	0\\
252.01	0\\
253.01	0\\
254.01	0\\
255.01	0\\
256.01	0\\
257.01	0\\
258.01	0\\
259.01	0\\
260.01	0\\
261.01	0\\
262.01	0\\
263.01	0\\
264.01	0\\
265.01	0\\
266.01	0\\
267.01	0\\
268.01	0\\
269.01	0\\
270.01	0\\
271.01	0\\
272.01	0\\
273.01	0\\
274.01	0\\
275.01	0\\
276.01	0\\
277.01	0\\
278.01	0\\
279.01	0\\
280.01	0\\
281.01	0\\
282.01	0\\
283.01	0\\
284.01	0\\
285.01	0\\
286.01	0\\
287.01	0\\
288.01	0\\
289.01	0\\
290.01	0\\
291.01	0\\
292.01	0\\
293.01	0\\
294.01	0\\
295.01	0\\
296.01	0\\
297.01	0\\
298.01	0\\
299.01	0\\
300.01	0\\
301.01	0\\
302.01	0\\
303.01	0\\
304.01	0\\
305.01	0\\
306.01	0\\
307.01	0\\
308.01	0\\
309.01	0\\
310.01	0\\
311.01	0\\
312.01	0\\
313.01	0\\
314.01	0\\
315.01	0\\
316.01	0\\
317.01	0\\
318.01	0\\
319.01	0\\
320.01	0\\
321.01	0\\
322.01	0\\
323.01	0\\
324.01	0\\
325.01	0\\
326.01	0\\
327.01	0\\
328.01	0\\
329.01	0\\
330.01	0\\
331.01	0\\
332.01	0\\
333.01	0\\
334.01	0\\
335.01	0\\
336.01	0\\
337.01	0\\
338.01	0\\
339.01	0\\
340.01	0\\
341.01	0\\
342.01	0\\
343.01	0\\
344.01	0\\
345.01	0\\
346.01	0\\
347.01	0\\
348.01	0\\
349.01	0\\
350.01	0\\
351.01	0\\
352.01	0\\
353.01	0\\
354.01	0\\
355.01	0\\
356.01	0\\
357.01	0\\
358.01	0\\
359.01	0\\
360.01	0\\
361.01	0\\
362.01	0\\
363.01	0\\
364.01	0\\
365.01	0\\
366.01	0\\
367.01	0\\
368.01	0\\
369.01	0\\
370.01	0\\
371.01	0\\
372.01	0\\
373.01	0\\
374.01	0\\
375.01	0\\
376.01	0\\
377.01	0\\
378.01	0\\
379.01	0\\
380.01	0\\
381.01	0\\
382.01	0\\
383.01	0\\
384.01	0\\
385.01	0\\
386.01	0\\
387.01	0\\
388.01	0\\
389.01	0\\
390.01	0\\
391.01	0\\
392.01	0\\
393.01	0\\
394.01	0\\
395.01	0\\
396.01	0\\
397.01	0\\
398.01	0\\
399.01	0\\
400.01	0\\
401.01	0\\
402.01	0\\
403.01	0\\
404.01	0\\
405.01	0\\
406.01	0\\
407.01	0\\
408.01	0\\
409.01	0\\
410.01	0\\
411.01	0\\
412.01	0\\
413.01	0\\
414.01	0\\
415.01	0\\
416.01	0\\
417.01	0\\
418.01	0\\
419.01	0\\
420.01	0\\
421.01	0\\
422.01	0\\
423.01	0\\
424.01	0\\
425.01	0\\
426.01	0\\
427.01	0\\
428.01	0\\
429.01	0\\
430.01	0\\
431.01	0\\
432.01	0\\
433.01	0\\
434.01	0\\
435.01	0\\
436.01	0\\
437.01	0\\
438.01	0\\
439.01	0\\
440.01	0\\
441.01	0\\
442.01	0\\
443.01	0\\
444.01	0\\
445.01	0\\
446.01	0\\
447.01	0\\
448.01	0\\
449.01	0\\
450.01	0\\
451.01	1.73472347597681e-18\\
452.01	0\\
453.01	0\\
454.01	0\\
455.01	0\\
456.01	0\\
457.01	0\\
458.01	0\\
459.01	0\\
460.01	0\\
461.01	0\\
462.01	0\\
463.01	0\\
464.01	0\\
465.01	0\\
466.01	0\\
467.01	0\\
468.01	0\\
469.01	1.73472347597681e-18\\
470.01	0\\
471.01	0\\
472.01	0\\
473.01	0\\
474.01	0\\
475.01	0\\
476.01	1.73472347597681e-18\\
477.01	0\\
478.01	0\\
479.01	0\\
480.01	0\\
481.01	0\\
482.01	0\\
483.01	0\\
484.01	0\\
485.01	0\\
486.01	0\\
487.01	0\\
488.01	0\\
489.01	0\\
490.01	0\\
491.01	0\\
492.01	0\\
493.01	0\\
494.01	0\\
495.01	0\\
496.01	0\\
497.01	0\\
498.01	0\\
499.01	0\\
500.01	0\\
501.01	0\\
502.01	0\\
503.01	0\\
504.01	0\\
505.01	0\\
506.01	0\\
507.01	0\\
508.01	0\\
509.01	0\\
510.01	0\\
511.01	0\\
512.01	1.73472347597681e-18\\
513.01	0\\
514.01	0\\
515.01	0\\
516.01	0\\
517.01	0\\
518.01	0\\
519.01	0\\
520.01	0\\
521.01	0\\
522.01	0\\
523.01	1.73472347597681e-18\\
524.01	0\\
525.01	0\\
526.01	0\\
527.01	0\\
528.01	0\\
529.01	0\\
530.01	0\\
531.01	0\\
532.01	0\\
533.01	0\\
534.01	0\\
535.01	0\\
536.01	1.73472347597681e-18\\
537.01	1.73472347597681e-18\\
538.01	1.73472347597681e-18\\
539.01	0\\
540.01	0\\
541.01	0\\
542.01	0\\
543.01	0\\
544.01	0\\
545.01	1.73472347597681e-18\\
546.01	1.73472347597681e-18\\
547.01	0\\
548.01	0\\
549.01	0\\
550.01	0\\
551.01	0\\
552.01	0\\
553.01	0\\
554.01	0\\
555.01	0\\
556.01	1.73472347597681e-18\\
557.01	0\\
558.01	0\\
559.01	0\\
560.01	1.73472347597681e-18\\
561.01	1.73472347597681e-18\\
562.01	0\\
563.01	0\\
564.01	0\\
565.01	0\\
566.01	0\\
567.01	0\\
568.01	0\\
569.01	0\\
570.01	0\\
571.01	0\\
572.01	0\\
573.01	0\\
574.01	0\\
575.01	0\\
576.01	0\\
577.01	0\\
578.01	1.73472347597681e-18\\
579.01	0\\
580.01	0\\
581.01	0\\
582.01	0\\
583.01	1.73472347597681e-18\\
584.01	0\\
585.01	0\\
586.01	0\\
587.01	0\\
588.01	0\\
589.01	0\\
590.01	0\\
591.01	0\\
592.01	0\\
593.01	0\\
594.01	0\\
595.01	0\\
596.01	0\\
597.01	0\\
598.01	0\\
599.01	0\\
599.02	0\\
599.03	0\\
599.04	0\\
599.05	0\\
599.06	0\\
599.07	0\\
599.08	0\\
599.09	0\\
599.1	0\\
599.11	0\\
599.12	0\\
599.13	0\\
599.14	0\\
599.15	0\\
599.16	0\\
599.17	0\\
599.18	0\\
599.19	0\\
599.2	0\\
599.21	0\\
599.22	0\\
599.23	0\\
599.24	0\\
599.25	0\\
599.26	0\\
599.27	0\\
599.28	0\\
599.29	0\\
599.3	0\\
599.31	0\\
599.32	0\\
599.33	0\\
599.34	0\\
599.35	0\\
599.36	0\\
599.37	0\\
599.38	0\\
599.39	0\\
599.4	0\\
599.41	0\\
599.42	0\\
599.43	0\\
599.44	0\\
599.45	0\\
599.46	0\\
599.47	0\\
599.48	0\\
599.49	0\\
599.5	0\\
599.51	0\\
599.52	0\\
599.53	0\\
599.54	0\\
599.55	0\\
599.56	0\\
599.57	0\\
599.58	0\\
599.59	0\\
599.6	0\\
599.61	0\\
599.62	0\\
599.63	0\\
599.64	0\\
599.65	0\\
599.66	0\\
599.67	0\\
599.68	0\\
599.69	0\\
599.7	0\\
599.71	0\\
599.72	0\\
599.73	0\\
599.74	0\\
599.75	0\\
599.76	0\\
599.77	0\\
599.78	0\\
599.79	0\\
599.8	0\\
599.81	0\\
599.82	0\\
599.83	0\\
599.84	0\\
599.85	0\\
599.86	0\\
599.87	0\\
599.88	0\\
599.89	0\\
599.9	0\\
599.91	0\\
599.92	0\\
599.93	0\\
599.94	0\\
599.95	0\\
599.96	0\\
599.97	0\\
599.98	0\\
599.99	0\\
600	0\\
};
\addplot [color=mycolor4,solid,forget plot]
  table[row sep=crcr]{%
0.01	0\\
1.01	0\\
2.01	0\\
3.01	0\\
4.01	0\\
5.01	0\\
6.01	0\\
7.01	0\\
8.01	0\\
9.01	0\\
10.01	0\\
11.01	0\\
12.01	0\\
13.01	0\\
14.01	0\\
15.01	0\\
16.01	0\\
17.01	0\\
18.01	0\\
19.01	0\\
20.01	0\\
21.01	0\\
22.01	0\\
23.01	0\\
24.01	0\\
25.01	0\\
26.01	0\\
27.01	0\\
28.01	0\\
29.01	0\\
30.01	0\\
31.01	0\\
32.01	0\\
33.01	0\\
34.01	0\\
35.01	0\\
36.01	0\\
37.01	0\\
38.01	0\\
39.01	0\\
40.01	0\\
41.01	0\\
42.01	0\\
43.01	0\\
44.01	0\\
45.01	0\\
46.01	0\\
47.01	0\\
48.01	0\\
49.01	0\\
50.01	0\\
51.01	0\\
52.01	0\\
53.01	0\\
54.01	0\\
55.01	0\\
56.01	0\\
57.01	0\\
58.01	0\\
59.01	0\\
60.01	0\\
61.01	0\\
62.01	0\\
63.01	0\\
64.01	0\\
65.01	0\\
66.01	0\\
67.01	0\\
68.01	0\\
69.01	0\\
70.01	0\\
71.01	0\\
72.01	0\\
73.01	0\\
74.01	0\\
75.01	0\\
76.01	0\\
77.01	0\\
78.01	0\\
79.01	0\\
80.01	0\\
81.01	0\\
82.01	0\\
83.01	0\\
84.01	0\\
85.01	0\\
86.01	0\\
87.01	0\\
88.01	0\\
89.01	0\\
90.01	0\\
91.01	0\\
92.01	0\\
93.01	0\\
94.01	0\\
95.01	0\\
96.01	0\\
97.01	0\\
98.01	0\\
99.01	0\\
100.01	0\\
101.01	0\\
102.01	0\\
103.01	0\\
104.01	0\\
105.01	0\\
106.01	0\\
107.01	0\\
108.01	0\\
109.01	0\\
110.01	0\\
111.01	0\\
112.01	0\\
113.01	0\\
114.01	0\\
115.01	0\\
116.01	0\\
117.01	0\\
118.01	0\\
119.01	0\\
120.01	0\\
121.01	0\\
122.01	0\\
123.01	0\\
124.01	0\\
125.01	0\\
126.01	0\\
127.01	0\\
128.01	0\\
129.01	0\\
130.01	0\\
131.01	0\\
132.01	0\\
133.01	0\\
134.01	0\\
135.01	0\\
136.01	0\\
137.01	0\\
138.01	0\\
139.01	0\\
140.01	0\\
141.01	0\\
142.01	0\\
143.01	0\\
144.01	0\\
145.01	0\\
146.01	0\\
147.01	0\\
148.01	0\\
149.01	0\\
150.01	0\\
151.01	0\\
152.01	0\\
153.01	0\\
154.01	0\\
155.01	0\\
156.01	0\\
157.01	0\\
158.01	0\\
159.01	0\\
160.01	0\\
161.01	0\\
162.01	0\\
163.01	0\\
164.01	0\\
165.01	0\\
166.01	0\\
167.01	0\\
168.01	0\\
169.01	0\\
170.01	0\\
171.01	0\\
172.01	0\\
173.01	0\\
174.01	0\\
175.01	0\\
176.01	0\\
177.01	0\\
178.01	0\\
179.01	0\\
180.01	0\\
181.01	0\\
182.01	0\\
183.01	0\\
184.01	0\\
185.01	0\\
186.01	0\\
187.01	0\\
188.01	0\\
189.01	0\\
190.01	0\\
191.01	0\\
192.01	0\\
193.01	0\\
194.01	0\\
195.01	0\\
196.01	0\\
197.01	0\\
198.01	0\\
199.01	0\\
200.01	0\\
201.01	0\\
202.01	0\\
203.01	0\\
204.01	0\\
205.01	0\\
206.01	0\\
207.01	0\\
208.01	0\\
209.01	0\\
210.01	0\\
211.01	0\\
212.01	0\\
213.01	0\\
214.01	0\\
215.01	0\\
216.01	0\\
217.01	0\\
218.01	0\\
219.01	0\\
220.01	0\\
221.01	0\\
222.01	0\\
223.01	0\\
224.01	0\\
225.01	0\\
226.01	0\\
227.01	0\\
228.01	0\\
229.01	0\\
230.01	0\\
231.01	0\\
232.01	0\\
233.01	0\\
234.01	0\\
235.01	0\\
236.01	0\\
237.01	0\\
238.01	0\\
239.01	0\\
240.01	0\\
241.01	0\\
242.01	0\\
243.01	0\\
244.01	0\\
245.01	0\\
246.01	0\\
247.01	0\\
248.01	0\\
249.01	0\\
250.01	0\\
251.01	0\\
252.01	0\\
253.01	0\\
254.01	0\\
255.01	0\\
256.01	0\\
257.01	0\\
258.01	0\\
259.01	0\\
260.01	0\\
261.01	0\\
262.01	0\\
263.01	0\\
264.01	0\\
265.01	0\\
266.01	0\\
267.01	0\\
268.01	0\\
269.01	0\\
270.01	0\\
271.01	0\\
272.01	0\\
273.01	0\\
274.01	0\\
275.01	0\\
276.01	0\\
277.01	0\\
278.01	0\\
279.01	0\\
280.01	0\\
281.01	0\\
282.01	0\\
283.01	0\\
284.01	0\\
285.01	0\\
286.01	0\\
287.01	0\\
288.01	0\\
289.01	0\\
290.01	0\\
291.01	0\\
292.01	0\\
293.01	0\\
294.01	0\\
295.01	0\\
296.01	0\\
297.01	0\\
298.01	0\\
299.01	0\\
300.01	0\\
301.01	0\\
302.01	0\\
303.01	0\\
304.01	0\\
305.01	0\\
306.01	0\\
307.01	0\\
308.01	0\\
309.01	0\\
310.01	0\\
311.01	0\\
312.01	0\\
313.01	0\\
314.01	0\\
315.01	0\\
316.01	0\\
317.01	0\\
318.01	0\\
319.01	0\\
320.01	0\\
321.01	0\\
322.01	0\\
323.01	0\\
324.01	0\\
325.01	0\\
326.01	0\\
327.01	0\\
328.01	0\\
329.01	0\\
330.01	0\\
331.01	0\\
332.01	0\\
333.01	0\\
334.01	0\\
335.01	0\\
336.01	0\\
337.01	0\\
338.01	0\\
339.01	0\\
340.01	0\\
341.01	0\\
342.01	0\\
343.01	0\\
344.01	0\\
345.01	0\\
346.01	0\\
347.01	0\\
348.01	0\\
349.01	0\\
350.01	0\\
351.01	0\\
352.01	0\\
353.01	0\\
354.01	0\\
355.01	0\\
356.01	0\\
357.01	0\\
358.01	0\\
359.01	0\\
360.01	0\\
361.01	0\\
362.01	0\\
363.01	0\\
364.01	0\\
365.01	0\\
366.01	0\\
367.01	0\\
368.01	0\\
369.01	0\\
370.01	0\\
371.01	0\\
372.01	0\\
373.01	0\\
374.01	0\\
375.01	0\\
376.01	0\\
377.01	0\\
378.01	0\\
379.01	0\\
380.01	0\\
381.01	0\\
382.01	0\\
383.01	0\\
384.01	0\\
385.01	0\\
386.01	0\\
387.01	0\\
388.01	0\\
389.01	0\\
390.01	0\\
391.01	0\\
392.01	0\\
393.01	0\\
394.01	0\\
395.01	0\\
396.01	0\\
397.01	0\\
398.01	0\\
399.01	0\\
400.01	0\\
401.01	0\\
402.01	0\\
403.01	0\\
404.01	0\\
405.01	0\\
406.01	0\\
407.01	0\\
408.01	0\\
409.01	0\\
410.01	0\\
411.01	0\\
412.01	0\\
413.01	0\\
414.01	0\\
415.01	0\\
416.01	0\\
417.01	0\\
418.01	0\\
419.01	0\\
420.01	0\\
421.01	0\\
422.01	0\\
423.01	0\\
424.01	0\\
425.01	0\\
426.01	0\\
427.01	0\\
428.01	0\\
429.01	0\\
430.01	0\\
431.01	0\\
432.01	0\\
433.01	0\\
434.01	0\\
435.01	0\\
436.01	0\\
437.01	0\\
438.01	0\\
439.01	0\\
440.01	0\\
441.01	0\\
442.01	0\\
443.01	0\\
444.01	0\\
445.01	0\\
446.01	0\\
447.01	0\\
448.01	0\\
449.01	0\\
450.01	0\\
451.01	1.73472347597681e-18\\
452.01	0\\
453.01	0\\
454.01	0\\
455.01	0\\
456.01	0\\
457.01	0\\
458.01	0\\
459.01	0\\
460.01	0\\
461.01	0\\
462.01	0\\
463.01	0\\
464.01	0\\
465.01	0\\
466.01	0\\
467.01	0\\
468.01	0\\
469.01	1.73472347597681e-18\\
470.01	0\\
471.01	0\\
472.01	0\\
473.01	0\\
474.01	0\\
475.01	0\\
476.01	1.73472347597681e-18\\
477.01	0\\
478.01	0\\
479.01	0\\
480.01	0\\
481.01	0\\
482.01	0\\
483.01	0\\
484.01	0\\
485.01	0\\
486.01	0\\
487.01	0\\
488.01	0\\
489.01	0\\
490.01	0\\
491.01	0\\
492.01	0\\
493.01	0\\
494.01	0\\
495.01	0\\
496.01	0\\
497.01	0\\
498.01	0\\
499.01	0\\
500.01	0\\
501.01	0\\
502.01	0\\
503.01	0\\
504.01	0\\
505.01	0\\
506.01	0\\
507.01	0\\
508.01	0\\
509.01	0\\
510.01	0\\
511.01	0\\
512.01	1.73472347597681e-18\\
513.01	0\\
514.01	0\\
515.01	0\\
516.01	0\\
517.01	0\\
518.01	0\\
519.01	0\\
520.01	0\\
521.01	0\\
522.01	0\\
523.01	1.73472347597681e-18\\
524.01	0\\
525.01	0\\
526.01	0\\
527.01	0\\
528.01	0\\
529.01	0\\
530.01	0\\
531.01	0\\
532.01	0\\
533.01	0\\
534.01	0\\
535.01	0\\
536.01	1.73472347597681e-18\\
537.01	1.73472347597681e-18\\
538.01	1.73472347597681e-18\\
539.01	0\\
540.01	0\\
541.01	0\\
542.01	0\\
543.01	0\\
544.01	0\\
545.01	1.73472347597681e-18\\
546.01	1.73472347597681e-18\\
547.01	0\\
548.01	0\\
549.01	0\\
550.01	0\\
551.01	0\\
552.01	0\\
553.01	0\\
554.01	0\\
555.01	0\\
556.01	1.73472347597681e-18\\
557.01	0\\
558.01	0\\
559.01	0\\
560.01	1.73472347597681e-18\\
561.01	1.73472347597681e-18\\
562.01	0\\
563.01	0\\
564.01	0\\
565.01	0\\
566.01	0\\
567.01	0\\
568.01	0\\
569.01	0\\
570.01	0\\
571.01	0\\
572.01	0\\
573.01	0\\
574.01	0\\
575.01	0\\
576.01	0\\
577.01	0\\
578.01	1.73472347597681e-18\\
579.01	0\\
580.01	0\\
581.01	0\\
582.01	0\\
583.01	1.73472347597681e-18\\
584.01	0\\
585.01	0\\
586.01	0\\
587.01	0\\
588.01	0\\
589.01	0\\
590.01	0\\
591.01	0\\
592.01	0\\
593.01	0\\
594.01	0\\
595.01	0\\
596.01	0\\
597.01	0\\
598.01	0\\
599.01	0\\
599.02	0\\
599.03	0\\
599.04	0\\
599.05	0\\
599.06	0\\
599.07	0\\
599.08	0\\
599.09	0\\
599.1	0\\
599.11	0\\
599.12	0\\
599.13	0\\
599.14	0\\
599.15	0\\
599.16	0\\
599.17	0\\
599.18	0\\
599.19	0\\
599.2	0\\
599.21	0\\
599.22	0\\
599.23	0\\
599.24	0\\
599.25	0\\
599.26	0\\
599.27	0\\
599.28	0\\
599.29	0\\
599.3	0\\
599.31	0\\
599.32	0\\
599.33	0\\
599.34	0\\
599.35	0\\
599.36	0\\
599.37	0\\
599.38	0\\
599.39	0\\
599.4	0\\
599.41	0\\
599.42	0\\
599.43	0\\
599.44	0\\
599.45	0\\
599.46	0\\
599.47	0\\
599.48	0\\
599.49	0\\
599.5	0\\
599.51	0\\
599.52	0\\
599.53	0\\
599.54	0\\
599.55	0\\
599.56	0\\
599.57	0\\
599.58	0\\
599.59	0\\
599.6	0\\
599.61	0\\
599.62	0\\
599.63	0\\
599.64	0\\
599.65	0\\
599.66	0\\
599.67	0\\
599.68	0\\
599.69	0\\
599.7	0\\
599.71	0\\
599.72	0\\
599.73	0\\
599.74	0\\
599.75	0\\
599.76	0\\
599.77	0\\
599.78	0\\
599.79	0\\
599.8	0\\
599.81	0\\
599.82	0\\
599.83	0\\
599.84	0\\
599.85	0\\
599.86	0\\
599.87	0\\
599.88	0\\
599.89	0\\
599.9	0\\
599.91	0\\
599.92	0\\
599.93	0\\
599.94	0\\
599.95	0\\
599.96	0\\
599.97	0\\
599.98	0\\
599.99	0\\
600	0\\
};
\addplot [color=mycolor5,solid,forget plot]
  table[row sep=crcr]{%
0.01	0\\
1.01	0\\
2.01	0\\
3.01	0\\
4.01	0\\
5.01	0\\
6.01	0\\
7.01	0\\
8.01	0\\
9.01	0\\
10.01	0\\
11.01	0\\
12.01	0\\
13.01	0\\
14.01	0\\
15.01	0\\
16.01	0\\
17.01	0\\
18.01	0\\
19.01	0\\
20.01	0\\
21.01	0\\
22.01	0\\
23.01	0\\
24.01	0\\
25.01	0\\
26.01	0\\
27.01	0\\
28.01	0\\
29.01	0\\
30.01	0\\
31.01	0\\
32.01	0\\
33.01	0\\
34.01	0\\
35.01	0\\
36.01	0\\
37.01	0\\
38.01	0\\
39.01	0\\
40.01	0\\
41.01	0\\
42.01	0\\
43.01	0\\
44.01	0\\
45.01	0\\
46.01	0\\
47.01	0\\
48.01	0\\
49.01	0\\
50.01	0\\
51.01	0\\
52.01	0\\
53.01	0\\
54.01	0\\
55.01	0\\
56.01	0\\
57.01	0\\
58.01	0\\
59.01	0\\
60.01	0\\
61.01	0\\
62.01	0\\
63.01	0\\
64.01	0\\
65.01	0\\
66.01	0\\
67.01	0\\
68.01	0\\
69.01	0\\
70.01	0\\
71.01	0\\
72.01	0\\
73.01	0\\
74.01	0\\
75.01	0\\
76.01	0\\
77.01	0\\
78.01	0\\
79.01	0\\
80.01	0\\
81.01	0\\
82.01	0\\
83.01	0\\
84.01	0\\
85.01	0\\
86.01	0\\
87.01	0\\
88.01	0\\
89.01	0\\
90.01	0\\
91.01	0\\
92.01	0\\
93.01	0\\
94.01	0\\
95.01	0\\
96.01	0\\
97.01	0\\
98.01	0\\
99.01	0\\
100.01	0\\
101.01	0\\
102.01	0\\
103.01	0\\
104.01	0\\
105.01	0\\
106.01	0\\
107.01	0\\
108.01	0\\
109.01	0\\
110.01	0\\
111.01	0\\
112.01	0\\
113.01	0\\
114.01	0\\
115.01	0\\
116.01	0\\
117.01	0\\
118.01	0\\
119.01	0\\
120.01	0\\
121.01	0\\
122.01	0\\
123.01	0\\
124.01	0\\
125.01	0\\
126.01	0\\
127.01	0\\
128.01	0\\
129.01	0\\
130.01	0\\
131.01	0\\
132.01	0\\
133.01	0\\
134.01	0\\
135.01	0\\
136.01	0\\
137.01	0\\
138.01	0\\
139.01	0\\
140.01	0\\
141.01	0\\
142.01	0\\
143.01	0\\
144.01	0\\
145.01	0\\
146.01	0\\
147.01	0\\
148.01	0\\
149.01	0\\
150.01	0\\
151.01	0\\
152.01	0\\
153.01	0\\
154.01	0\\
155.01	0\\
156.01	0\\
157.01	0\\
158.01	0\\
159.01	0\\
160.01	0\\
161.01	0\\
162.01	0\\
163.01	0\\
164.01	0\\
165.01	0\\
166.01	0\\
167.01	0\\
168.01	0\\
169.01	0\\
170.01	0\\
171.01	0\\
172.01	0\\
173.01	0\\
174.01	0\\
175.01	0\\
176.01	0\\
177.01	0\\
178.01	0\\
179.01	0\\
180.01	0\\
181.01	0\\
182.01	0\\
183.01	0\\
184.01	0\\
185.01	0\\
186.01	0\\
187.01	0\\
188.01	0\\
189.01	0\\
190.01	0\\
191.01	0\\
192.01	0\\
193.01	0\\
194.01	0\\
195.01	0\\
196.01	0\\
197.01	0\\
198.01	0\\
199.01	0\\
200.01	0\\
201.01	0\\
202.01	0\\
203.01	0\\
204.01	0\\
205.01	0\\
206.01	0\\
207.01	0\\
208.01	0\\
209.01	0\\
210.01	0\\
211.01	0\\
212.01	0\\
213.01	0\\
214.01	0\\
215.01	0\\
216.01	0\\
217.01	0\\
218.01	0\\
219.01	0\\
220.01	0\\
221.01	0\\
222.01	0\\
223.01	0\\
224.01	0\\
225.01	0\\
226.01	0\\
227.01	0\\
228.01	0\\
229.01	0\\
230.01	0\\
231.01	0\\
232.01	0\\
233.01	0\\
234.01	0\\
235.01	0\\
236.01	0\\
237.01	0\\
238.01	0\\
239.01	0\\
240.01	0\\
241.01	0\\
242.01	0\\
243.01	0\\
244.01	0\\
245.01	0\\
246.01	0\\
247.01	0\\
248.01	0\\
249.01	0\\
250.01	0\\
251.01	0\\
252.01	0\\
253.01	0\\
254.01	0\\
255.01	0\\
256.01	0\\
257.01	0\\
258.01	0\\
259.01	0\\
260.01	0\\
261.01	0\\
262.01	0\\
263.01	0\\
264.01	0\\
265.01	0\\
266.01	0\\
267.01	0\\
268.01	0\\
269.01	0\\
270.01	0\\
271.01	0\\
272.01	0\\
273.01	0\\
274.01	0\\
275.01	0\\
276.01	0\\
277.01	0\\
278.01	0\\
279.01	0\\
280.01	0\\
281.01	0\\
282.01	0\\
283.01	0\\
284.01	0\\
285.01	0\\
286.01	0\\
287.01	0\\
288.01	0\\
289.01	0\\
290.01	0\\
291.01	0\\
292.01	0\\
293.01	0\\
294.01	0\\
295.01	0\\
296.01	0\\
297.01	0\\
298.01	0\\
299.01	0\\
300.01	0\\
301.01	0\\
302.01	0\\
303.01	0\\
304.01	0\\
305.01	0\\
306.01	0\\
307.01	0\\
308.01	0\\
309.01	0\\
310.01	0\\
311.01	0\\
312.01	0\\
313.01	0\\
314.01	0\\
315.01	0\\
316.01	0\\
317.01	0\\
318.01	0\\
319.01	0\\
320.01	0\\
321.01	0\\
322.01	0\\
323.01	0\\
324.01	0\\
325.01	0\\
326.01	0\\
327.01	0\\
328.01	0\\
329.01	0\\
330.01	0\\
331.01	0\\
332.01	0\\
333.01	0\\
334.01	0\\
335.01	0\\
336.01	0\\
337.01	0\\
338.01	0\\
339.01	0\\
340.01	0\\
341.01	0\\
342.01	0\\
343.01	0\\
344.01	0\\
345.01	0\\
346.01	0\\
347.01	0\\
348.01	0\\
349.01	0\\
350.01	0\\
351.01	0\\
352.01	0\\
353.01	0\\
354.01	0\\
355.01	0\\
356.01	0\\
357.01	0\\
358.01	0\\
359.01	0\\
360.01	0\\
361.01	0\\
362.01	0\\
363.01	0\\
364.01	0\\
365.01	0\\
366.01	0\\
367.01	0\\
368.01	0\\
369.01	0\\
370.01	0\\
371.01	0\\
372.01	0\\
373.01	0\\
374.01	0\\
375.01	0\\
376.01	0\\
377.01	0\\
378.01	0\\
379.01	0\\
380.01	0\\
381.01	0\\
382.01	0\\
383.01	0\\
384.01	0\\
385.01	0\\
386.01	0\\
387.01	0\\
388.01	0\\
389.01	0\\
390.01	0\\
391.01	0\\
392.01	0\\
393.01	0\\
394.01	0\\
395.01	0\\
396.01	0\\
397.01	0\\
398.01	0\\
399.01	0\\
400.01	0\\
401.01	0\\
402.01	0\\
403.01	0\\
404.01	0\\
405.01	0\\
406.01	0\\
407.01	0\\
408.01	0\\
409.01	0\\
410.01	0\\
411.01	0\\
412.01	0\\
413.01	0\\
414.01	0\\
415.01	0\\
416.01	0\\
417.01	0\\
418.01	0\\
419.01	0\\
420.01	0\\
421.01	0\\
422.01	0\\
423.01	0\\
424.01	0\\
425.01	0\\
426.01	0\\
427.01	0\\
428.01	0\\
429.01	0\\
430.01	0\\
431.01	0\\
432.01	0\\
433.01	0\\
434.01	0\\
435.01	0\\
436.01	0\\
437.01	0\\
438.01	0\\
439.01	0\\
440.01	0\\
441.01	0\\
442.01	0\\
443.01	0\\
444.01	0\\
445.01	0\\
446.01	0\\
447.01	0\\
448.01	0\\
449.01	0\\
450.01	0\\
451.01	1.73472347597681e-18\\
452.01	0\\
453.01	0\\
454.01	0\\
455.01	0\\
456.01	0\\
457.01	0\\
458.01	0\\
459.01	0\\
460.01	0\\
461.01	0\\
462.01	0\\
463.01	0\\
464.01	0\\
465.01	0\\
466.01	0\\
467.01	0\\
468.01	0\\
469.01	1.73472347597681e-18\\
470.01	0\\
471.01	0\\
472.01	0\\
473.01	0\\
474.01	0\\
475.01	0\\
476.01	1.73472347597681e-18\\
477.01	0\\
478.01	0\\
479.01	0\\
480.01	0\\
481.01	0\\
482.01	0\\
483.01	0\\
484.01	0\\
485.01	0\\
486.01	0\\
487.01	0\\
488.01	0\\
489.01	0\\
490.01	0\\
491.01	0\\
492.01	0\\
493.01	0\\
494.01	0\\
495.01	0\\
496.01	0\\
497.01	0\\
498.01	0\\
499.01	0\\
500.01	0\\
501.01	0\\
502.01	0\\
503.01	0\\
504.01	0\\
505.01	0\\
506.01	0\\
507.01	0\\
508.01	0\\
509.01	0\\
510.01	0\\
511.01	0\\
512.01	1.73472347597681e-18\\
513.01	0\\
514.01	0\\
515.01	0\\
516.01	0\\
517.01	0\\
518.01	0\\
519.01	0\\
520.01	0\\
521.01	0\\
522.01	0\\
523.01	1.73472347597681e-18\\
524.01	0\\
525.01	0\\
526.01	0\\
527.01	0\\
528.01	0\\
529.01	0\\
530.01	0\\
531.01	0\\
532.01	0\\
533.01	0\\
534.01	0\\
535.01	0\\
536.01	1.73472347597681e-18\\
537.01	1.73472347597681e-18\\
538.01	1.73472347597681e-18\\
539.01	0\\
540.01	0\\
541.01	0\\
542.01	0\\
543.01	0\\
544.01	0\\
545.01	1.73472347597681e-18\\
546.01	1.73472347597681e-18\\
547.01	0\\
548.01	0\\
549.01	0\\
550.01	0\\
551.01	0\\
552.01	0\\
553.01	0\\
554.01	0\\
555.01	0\\
556.01	1.73472347597681e-18\\
557.01	0\\
558.01	0\\
559.01	0\\
560.01	1.73472347597681e-18\\
561.01	1.73472347597681e-18\\
562.01	0\\
563.01	0\\
564.01	0\\
565.01	0\\
566.01	0\\
567.01	0\\
568.01	0\\
569.01	0\\
570.01	0\\
571.01	0\\
572.01	0\\
573.01	0\\
574.01	0\\
575.01	0\\
576.01	0\\
577.01	0\\
578.01	1.73472347597681e-18\\
579.01	0\\
580.01	0\\
581.01	0\\
582.01	0\\
583.01	1.73472347597681e-18\\
584.01	0\\
585.01	0\\
586.01	0\\
587.01	0\\
588.01	0\\
589.01	0\\
590.01	0\\
591.01	0\\
592.01	0\\
593.01	0\\
594.01	0\\
595.01	0\\
596.01	0\\
597.01	0\\
598.01	0\\
599.01	0\\
599.02	0\\
599.03	0\\
599.04	0\\
599.05	0\\
599.06	0\\
599.07	0\\
599.08	0\\
599.09	0\\
599.1	0\\
599.11	0\\
599.12	0\\
599.13	0\\
599.14	0\\
599.15	0\\
599.16	0\\
599.17	0\\
599.18	0\\
599.19	0\\
599.2	0\\
599.21	0\\
599.22	0\\
599.23	0\\
599.24	0\\
599.25	0\\
599.26	0\\
599.27	0\\
599.28	0\\
599.29	0\\
599.3	0\\
599.31	0\\
599.32	0\\
599.33	0\\
599.34	0\\
599.35	0\\
599.36	0\\
599.37	0\\
599.38	0\\
599.39	0\\
599.4	0\\
599.41	0\\
599.42	0\\
599.43	0\\
599.44	0\\
599.45	0\\
599.46	0\\
599.47	0\\
599.48	0\\
599.49	0\\
599.5	0\\
599.51	0\\
599.52	0\\
599.53	0\\
599.54	0\\
599.55	0\\
599.56	0\\
599.57	0\\
599.58	0\\
599.59	0\\
599.6	0\\
599.61	0\\
599.62	0\\
599.63	0\\
599.64	0\\
599.65	0\\
599.66	0\\
599.67	0\\
599.68	0\\
599.69	0\\
599.7	0\\
599.71	0\\
599.72	0\\
599.73	0\\
599.74	0\\
599.75	0\\
599.76	0\\
599.77	0\\
599.78	0\\
599.79	0\\
599.8	0\\
599.81	0\\
599.82	0\\
599.83	0\\
599.84	0\\
599.85	0\\
599.86	0\\
599.87	0\\
599.88	0\\
599.89	0\\
599.9	0\\
599.91	0\\
599.92	0\\
599.93	0\\
599.94	0\\
599.95	0\\
599.96	0\\
599.97	0\\
599.98	0\\
599.99	0\\
600	0\\
};
\addplot [color=mycolor6,solid,forget plot]
  table[row sep=crcr]{%
0.01	0\\
1.01	0\\
2.01	0\\
3.01	0\\
4.01	0\\
5.01	0\\
6.01	0\\
7.01	0\\
8.01	0\\
9.01	0\\
10.01	0\\
11.01	0\\
12.01	0\\
13.01	0\\
14.01	0\\
15.01	0\\
16.01	0\\
17.01	0\\
18.01	0\\
19.01	0\\
20.01	0\\
21.01	0\\
22.01	0\\
23.01	0\\
24.01	0\\
25.01	0\\
26.01	0\\
27.01	0\\
28.01	0\\
29.01	0\\
30.01	0\\
31.01	0\\
32.01	0\\
33.01	0\\
34.01	0\\
35.01	0\\
36.01	0\\
37.01	0\\
38.01	0\\
39.01	0\\
40.01	0\\
41.01	0\\
42.01	0\\
43.01	0\\
44.01	0\\
45.01	0\\
46.01	0\\
47.01	0\\
48.01	0\\
49.01	0\\
50.01	0\\
51.01	0\\
52.01	0\\
53.01	0\\
54.01	0\\
55.01	0\\
56.01	0\\
57.01	0\\
58.01	0\\
59.01	0\\
60.01	0\\
61.01	0\\
62.01	0\\
63.01	0\\
64.01	0\\
65.01	0\\
66.01	0\\
67.01	0\\
68.01	0\\
69.01	0\\
70.01	0\\
71.01	0\\
72.01	0\\
73.01	0\\
74.01	0\\
75.01	0\\
76.01	0\\
77.01	0\\
78.01	0\\
79.01	0\\
80.01	0\\
81.01	0\\
82.01	0\\
83.01	0\\
84.01	0\\
85.01	0\\
86.01	0\\
87.01	0\\
88.01	0\\
89.01	0\\
90.01	0\\
91.01	0\\
92.01	0\\
93.01	0\\
94.01	0\\
95.01	0\\
96.01	0\\
97.01	0\\
98.01	0\\
99.01	0\\
100.01	0\\
101.01	0\\
102.01	0\\
103.01	0\\
104.01	0\\
105.01	0\\
106.01	0\\
107.01	0\\
108.01	0\\
109.01	0\\
110.01	0\\
111.01	0\\
112.01	0\\
113.01	0\\
114.01	0\\
115.01	0\\
116.01	0\\
117.01	0\\
118.01	0\\
119.01	0\\
120.01	0\\
121.01	0\\
122.01	0\\
123.01	0\\
124.01	0\\
125.01	0\\
126.01	0\\
127.01	0\\
128.01	0\\
129.01	0\\
130.01	0\\
131.01	0\\
132.01	0\\
133.01	0\\
134.01	0\\
135.01	0\\
136.01	0\\
137.01	0\\
138.01	0\\
139.01	0\\
140.01	0\\
141.01	0\\
142.01	0\\
143.01	0\\
144.01	0\\
145.01	0\\
146.01	0\\
147.01	0\\
148.01	0\\
149.01	0\\
150.01	0\\
151.01	0\\
152.01	0\\
153.01	0\\
154.01	0\\
155.01	0\\
156.01	0\\
157.01	0\\
158.01	0\\
159.01	0\\
160.01	0\\
161.01	0\\
162.01	0\\
163.01	0\\
164.01	0\\
165.01	0\\
166.01	0\\
167.01	0\\
168.01	0\\
169.01	0\\
170.01	0\\
171.01	0\\
172.01	0\\
173.01	0\\
174.01	0\\
175.01	0\\
176.01	0\\
177.01	0\\
178.01	0\\
179.01	0\\
180.01	0\\
181.01	0\\
182.01	0\\
183.01	0\\
184.01	0\\
185.01	0\\
186.01	0\\
187.01	0\\
188.01	0\\
189.01	0\\
190.01	0\\
191.01	0\\
192.01	0\\
193.01	0\\
194.01	0\\
195.01	0\\
196.01	0\\
197.01	0\\
198.01	0\\
199.01	0\\
200.01	0\\
201.01	0\\
202.01	0\\
203.01	0\\
204.01	0\\
205.01	0\\
206.01	0\\
207.01	0\\
208.01	0\\
209.01	0\\
210.01	0\\
211.01	0\\
212.01	0\\
213.01	0\\
214.01	0\\
215.01	0\\
216.01	0\\
217.01	0\\
218.01	0\\
219.01	0\\
220.01	0\\
221.01	0\\
222.01	0\\
223.01	0\\
224.01	0\\
225.01	0\\
226.01	0\\
227.01	0\\
228.01	0\\
229.01	0\\
230.01	0\\
231.01	0\\
232.01	0\\
233.01	0\\
234.01	0\\
235.01	0\\
236.01	0\\
237.01	0\\
238.01	0\\
239.01	0\\
240.01	0\\
241.01	0\\
242.01	0\\
243.01	0\\
244.01	0\\
245.01	0\\
246.01	0\\
247.01	0\\
248.01	0\\
249.01	0\\
250.01	0\\
251.01	0\\
252.01	0\\
253.01	0\\
254.01	0\\
255.01	0\\
256.01	0\\
257.01	0\\
258.01	0\\
259.01	0\\
260.01	0\\
261.01	0\\
262.01	0\\
263.01	0\\
264.01	0\\
265.01	0\\
266.01	0\\
267.01	0\\
268.01	0\\
269.01	0\\
270.01	0\\
271.01	0\\
272.01	0\\
273.01	0\\
274.01	0\\
275.01	0\\
276.01	0\\
277.01	0\\
278.01	0\\
279.01	0\\
280.01	0\\
281.01	0\\
282.01	0\\
283.01	0\\
284.01	0\\
285.01	0\\
286.01	0\\
287.01	0\\
288.01	0\\
289.01	0\\
290.01	0\\
291.01	0\\
292.01	0\\
293.01	0\\
294.01	0\\
295.01	0\\
296.01	0\\
297.01	0\\
298.01	0\\
299.01	0\\
300.01	0\\
301.01	0\\
302.01	0\\
303.01	0\\
304.01	0\\
305.01	0\\
306.01	0\\
307.01	0\\
308.01	0\\
309.01	0\\
310.01	0\\
311.01	0\\
312.01	0\\
313.01	0\\
314.01	0\\
315.01	0\\
316.01	0\\
317.01	0\\
318.01	0\\
319.01	0\\
320.01	0\\
321.01	0\\
322.01	0\\
323.01	0\\
324.01	0\\
325.01	0\\
326.01	0\\
327.01	0\\
328.01	0\\
329.01	0\\
330.01	0\\
331.01	0\\
332.01	0\\
333.01	0\\
334.01	0\\
335.01	0\\
336.01	0\\
337.01	0\\
338.01	0\\
339.01	0\\
340.01	0\\
341.01	0\\
342.01	0\\
343.01	0\\
344.01	0\\
345.01	0\\
346.01	0\\
347.01	0\\
348.01	0\\
349.01	0\\
350.01	0\\
351.01	0\\
352.01	0\\
353.01	0\\
354.01	0\\
355.01	0\\
356.01	0\\
357.01	0\\
358.01	0\\
359.01	0\\
360.01	0\\
361.01	0\\
362.01	0\\
363.01	0\\
364.01	0\\
365.01	0\\
366.01	0\\
367.01	0\\
368.01	0\\
369.01	0\\
370.01	0\\
371.01	0\\
372.01	0\\
373.01	0\\
374.01	0\\
375.01	0\\
376.01	0\\
377.01	0\\
378.01	0\\
379.01	0\\
380.01	0\\
381.01	0\\
382.01	0\\
383.01	0\\
384.01	0\\
385.01	0\\
386.01	0\\
387.01	0\\
388.01	0\\
389.01	0\\
390.01	0\\
391.01	0\\
392.01	0\\
393.01	0\\
394.01	0\\
395.01	0\\
396.01	0\\
397.01	0\\
398.01	0\\
399.01	0\\
400.01	0\\
401.01	0\\
402.01	0\\
403.01	0\\
404.01	0\\
405.01	0\\
406.01	0\\
407.01	0\\
408.01	0\\
409.01	0\\
410.01	0\\
411.01	0\\
412.01	0\\
413.01	0\\
414.01	0\\
415.01	0\\
416.01	0\\
417.01	0\\
418.01	0\\
419.01	0\\
420.01	0\\
421.01	0\\
422.01	0\\
423.01	0\\
424.01	0\\
425.01	0\\
426.01	0\\
427.01	0\\
428.01	0\\
429.01	0\\
430.01	0\\
431.01	0\\
432.01	0\\
433.01	0\\
434.01	0\\
435.01	0\\
436.01	0\\
437.01	0\\
438.01	0\\
439.01	0\\
440.01	0\\
441.01	0\\
442.01	0\\
443.01	0\\
444.01	0\\
445.01	0\\
446.01	0\\
447.01	0\\
448.01	0\\
449.01	0\\
450.01	0\\
451.01	1.73472347597681e-18\\
452.01	0\\
453.01	0\\
454.01	0\\
455.01	0\\
456.01	0\\
457.01	0\\
458.01	0\\
459.01	0\\
460.01	0\\
461.01	0\\
462.01	0\\
463.01	0\\
464.01	0\\
465.01	0\\
466.01	0\\
467.01	0\\
468.01	0\\
469.01	1.73472347597681e-18\\
470.01	0\\
471.01	0\\
472.01	0\\
473.01	0\\
474.01	0\\
475.01	0\\
476.01	1.73472347597681e-18\\
477.01	0\\
478.01	0\\
479.01	0\\
480.01	0\\
481.01	0\\
482.01	0\\
483.01	0\\
484.01	0\\
485.01	0\\
486.01	0\\
487.01	0\\
488.01	0\\
489.01	0\\
490.01	0\\
491.01	0\\
492.01	0\\
493.01	0\\
494.01	0\\
495.01	0\\
496.01	0\\
497.01	0\\
498.01	0\\
499.01	0\\
500.01	0\\
501.01	0\\
502.01	0\\
503.01	0\\
504.01	0\\
505.01	0\\
506.01	0\\
507.01	0\\
508.01	0\\
509.01	0\\
510.01	0\\
511.01	0\\
512.01	1.73472347597681e-18\\
513.01	0\\
514.01	0\\
515.01	0\\
516.01	0\\
517.01	0\\
518.01	0\\
519.01	0\\
520.01	0\\
521.01	0\\
522.01	0\\
523.01	1.73472347597681e-18\\
524.01	0\\
525.01	0\\
526.01	0\\
527.01	0\\
528.01	0\\
529.01	0\\
530.01	0\\
531.01	0\\
532.01	0\\
533.01	0\\
534.01	0\\
535.01	0\\
536.01	1.73472347597681e-18\\
537.01	1.73472347597681e-18\\
538.01	1.73472347597681e-18\\
539.01	0\\
540.01	0\\
541.01	0\\
542.01	0\\
543.01	0\\
544.01	0\\
545.01	1.73472347597681e-18\\
546.01	1.73472347597681e-18\\
547.01	0\\
548.01	0\\
549.01	0\\
550.01	0\\
551.01	0\\
552.01	0\\
553.01	0\\
554.01	0\\
555.01	0\\
556.01	1.73472347597681e-18\\
557.01	0\\
558.01	0\\
559.01	0\\
560.01	1.73472347597681e-18\\
561.01	1.73472347597681e-18\\
562.01	0\\
563.01	0\\
564.01	0\\
565.01	0\\
566.01	0\\
567.01	0\\
568.01	0\\
569.01	0\\
570.01	0\\
571.01	0\\
572.01	0\\
573.01	0\\
574.01	0\\
575.01	0\\
576.01	0\\
577.01	0\\
578.01	1.73472347597681e-18\\
579.01	0\\
580.01	0\\
581.01	0\\
582.01	0\\
583.01	1.73472347597681e-18\\
584.01	0\\
585.01	0\\
586.01	0\\
587.01	0\\
588.01	0\\
589.01	0\\
590.01	0\\
591.01	0\\
592.01	0\\
593.01	0\\
594.01	0\\
595.01	0\\
596.01	0\\
597.01	0\\
598.01	0\\
599.01	0\\
599.02	0\\
599.03	0\\
599.04	0\\
599.05	0\\
599.06	0\\
599.07	0\\
599.08	0\\
599.09	0\\
599.1	0\\
599.11	0\\
599.12	0\\
599.13	0\\
599.14	0\\
599.15	0\\
599.16	0\\
599.17	0\\
599.18	0\\
599.19	0\\
599.2	0\\
599.21	0\\
599.22	0\\
599.23	0\\
599.24	0\\
599.25	0\\
599.26	0\\
599.27	0\\
599.28	0\\
599.29	0\\
599.3	0\\
599.31	0\\
599.32	0\\
599.33	0\\
599.34	0\\
599.35	0\\
599.36	0\\
599.37	0\\
599.38	0\\
599.39	0\\
599.4	0\\
599.41	0\\
599.42	0\\
599.43	0\\
599.44	0\\
599.45	0\\
599.46	0\\
599.47	0\\
599.48	0\\
599.49	0\\
599.5	0\\
599.51	0\\
599.52	0\\
599.53	0\\
599.54	0\\
599.55	0\\
599.56	0\\
599.57	0\\
599.58	0\\
599.59	0\\
599.6	0\\
599.61	0\\
599.62	0\\
599.63	0\\
599.64	0\\
599.65	0\\
599.66	0\\
599.67	0\\
599.68	0\\
599.69	0\\
599.7	0\\
599.71	0\\
599.72	0\\
599.73	0\\
599.74	0\\
599.75	0\\
599.76	0\\
599.77	0\\
599.78	0\\
599.79	0\\
599.8	0\\
599.81	0\\
599.82	0\\
599.83	0\\
599.84	0\\
599.85	0\\
599.86	0\\
599.87	0\\
599.88	0\\
599.89	0\\
599.9	0\\
599.91	0\\
599.92	0\\
599.93	0\\
599.94	0\\
599.95	0\\
599.96	0\\
599.97	0\\
599.98	0\\
599.99	0\\
600	0\\
};
\addplot [color=mycolor7,solid,forget plot]
  table[row sep=crcr]{%
0.01	0\\
1.01	0\\
2.01	0\\
3.01	0\\
4.01	0\\
5.01	0\\
6.01	0\\
7.01	0\\
8.01	0\\
9.01	0\\
10.01	0\\
11.01	0\\
12.01	0\\
13.01	0\\
14.01	0\\
15.01	0\\
16.01	0\\
17.01	0\\
18.01	0\\
19.01	0\\
20.01	0\\
21.01	0\\
22.01	0\\
23.01	0\\
24.01	0\\
25.01	0\\
26.01	0\\
27.01	0\\
28.01	0\\
29.01	0\\
30.01	0\\
31.01	0\\
32.01	0\\
33.01	0\\
34.01	0\\
35.01	0\\
36.01	0\\
37.01	0\\
38.01	0\\
39.01	0\\
40.01	0\\
41.01	0\\
42.01	0\\
43.01	0\\
44.01	0\\
45.01	0\\
46.01	0\\
47.01	0\\
48.01	0\\
49.01	0\\
50.01	0\\
51.01	0\\
52.01	0\\
53.01	0\\
54.01	0\\
55.01	0\\
56.01	0\\
57.01	0\\
58.01	0\\
59.01	0\\
60.01	0\\
61.01	0\\
62.01	0\\
63.01	0\\
64.01	0\\
65.01	0\\
66.01	0\\
67.01	0\\
68.01	0\\
69.01	0\\
70.01	0\\
71.01	0\\
72.01	0\\
73.01	0\\
74.01	0\\
75.01	0\\
76.01	0\\
77.01	0\\
78.01	0\\
79.01	0\\
80.01	0\\
81.01	0\\
82.01	0\\
83.01	0\\
84.01	0\\
85.01	0\\
86.01	0\\
87.01	0\\
88.01	0\\
89.01	0\\
90.01	0\\
91.01	0\\
92.01	0\\
93.01	0\\
94.01	0\\
95.01	0\\
96.01	0\\
97.01	0\\
98.01	0\\
99.01	0\\
100.01	0\\
101.01	0\\
102.01	0\\
103.01	0\\
104.01	0\\
105.01	0\\
106.01	0\\
107.01	0\\
108.01	0\\
109.01	0\\
110.01	0\\
111.01	0\\
112.01	0\\
113.01	0\\
114.01	0\\
115.01	0\\
116.01	0\\
117.01	0\\
118.01	0\\
119.01	0\\
120.01	0\\
121.01	0\\
122.01	0\\
123.01	0\\
124.01	0\\
125.01	0\\
126.01	0\\
127.01	0\\
128.01	0\\
129.01	0\\
130.01	0\\
131.01	0\\
132.01	0\\
133.01	0\\
134.01	0\\
135.01	0\\
136.01	0\\
137.01	0\\
138.01	0\\
139.01	0\\
140.01	0\\
141.01	0\\
142.01	0\\
143.01	0\\
144.01	0\\
145.01	0\\
146.01	0\\
147.01	0\\
148.01	0\\
149.01	0\\
150.01	0\\
151.01	0\\
152.01	0\\
153.01	0\\
154.01	0\\
155.01	0\\
156.01	0\\
157.01	0\\
158.01	0\\
159.01	0\\
160.01	0\\
161.01	0\\
162.01	0\\
163.01	0\\
164.01	0\\
165.01	0\\
166.01	0\\
167.01	0\\
168.01	0\\
169.01	0\\
170.01	0\\
171.01	0\\
172.01	0\\
173.01	0\\
174.01	0\\
175.01	0\\
176.01	0\\
177.01	0\\
178.01	0\\
179.01	0\\
180.01	0\\
181.01	0\\
182.01	0\\
183.01	0\\
184.01	0\\
185.01	0\\
186.01	0\\
187.01	0\\
188.01	0\\
189.01	0\\
190.01	0\\
191.01	0\\
192.01	0\\
193.01	0\\
194.01	0\\
195.01	0\\
196.01	0\\
197.01	0\\
198.01	0\\
199.01	0\\
200.01	0\\
201.01	0\\
202.01	0\\
203.01	0\\
204.01	0\\
205.01	0\\
206.01	0\\
207.01	0\\
208.01	0\\
209.01	0\\
210.01	0\\
211.01	0\\
212.01	0\\
213.01	0\\
214.01	0\\
215.01	0\\
216.01	0\\
217.01	0\\
218.01	0\\
219.01	0\\
220.01	0\\
221.01	0\\
222.01	0\\
223.01	0\\
224.01	0\\
225.01	0\\
226.01	0\\
227.01	0\\
228.01	0\\
229.01	0\\
230.01	0\\
231.01	0\\
232.01	0\\
233.01	0\\
234.01	0\\
235.01	0\\
236.01	0\\
237.01	0\\
238.01	0\\
239.01	0\\
240.01	0\\
241.01	0\\
242.01	0\\
243.01	0\\
244.01	0\\
245.01	0\\
246.01	0\\
247.01	0\\
248.01	0\\
249.01	0\\
250.01	0\\
251.01	0\\
252.01	0\\
253.01	0\\
254.01	0\\
255.01	0\\
256.01	0\\
257.01	0\\
258.01	0\\
259.01	0\\
260.01	0\\
261.01	0\\
262.01	0\\
263.01	0\\
264.01	0\\
265.01	0\\
266.01	0\\
267.01	0\\
268.01	0\\
269.01	0\\
270.01	0\\
271.01	0\\
272.01	0\\
273.01	0\\
274.01	0\\
275.01	0\\
276.01	0\\
277.01	0\\
278.01	0\\
279.01	0\\
280.01	0\\
281.01	0\\
282.01	0\\
283.01	0\\
284.01	0\\
285.01	0\\
286.01	0\\
287.01	0\\
288.01	0\\
289.01	0\\
290.01	0\\
291.01	0\\
292.01	0\\
293.01	0\\
294.01	0\\
295.01	0\\
296.01	0\\
297.01	0\\
298.01	0\\
299.01	0\\
300.01	0\\
301.01	0\\
302.01	0\\
303.01	0\\
304.01	0\\
305.01	0\\
306.01	0\\
307.01	0\\
308.01	0\\
309.01	0\\
310.01	0\\
311.01	0\\
312.01	0\\
313.01	0\\
314.01	0\\
315.01	0\\
316.01	0\\
317.01	0\\
318.01	0\\
319.01	0\\
320.01	0\\
321.01	0\\
322.01	0\\
323.01	0\\
324.01	0\\
325.01	0\\
326.01	0\\
327.01	0\\
328.01	0\\
329.01	0\\
330.01	0\\
331.01	0\\
332.01	0\\
333.01	0\\
334.01	0\\
335.01	0\\
336.01	0\\
337.01	0\\
338.01	0\\
339.01	0\\
340.01	0\\
341.01	0\\
342.01	0\\
343.01	0\\
344.01	0\\
345.01	0\\
346.01	0\\
347.01	0\\
348.01	0\\
349.01	0\\
350.01	0\\
351.01	0\\
352.01	0\\
353.01	0\\
354.01	0\\
355.01	0\\
356.01	0\\
357.01	0\\
358.01	0\\
359.01	0\\
360.01	0\\
361.01	0\\
362.01	0\\
363.01	0\\
364.01	0\\
365.01	0\\
366.01	0\\
367.01	0\\
368.01	0\\
369.01	0\\
370.01	0\\
371.01	0\\
372.01	0\\
373.01	0\\
374.01	0\\
375.01	0\\
376.01	0\\
377.01	0\\
378.01	0\\
379.01	0\\
380.01	0\\
381.01	0\\
382.01	0\\
383.01	0\\
384.01	0\\
385.01	0\\
386.01	0\\
387.01	0\\
388.01	0\\
389.01	0\\
390.01	0\\
391.01	0\\
392.01	0\\
393.01	0\\
394.01	0\\
395.01	0\\
396.01	0\\
397.01	0\\
398.01	0\\
399.01	0\\
400.01	0\\
401.01	0\\
402.01	0\\
403.01	0\\
404.01	0\\
405.01	0\\
406.01	0\\
407.01	0\\
408.01	0\\
409.01	0\\
410.01	0\\
411.01	0\\
412.01	0\\
413.01	0\\
414.01	0\\
415.01	0\\
416.01	0\\
417.01	0\\
418.01	0\\
419.01	0\\
420.01	0\\
421.01	0\\
422.01	0\\
423.01	0\\
424.01	0\\
425.01	0\\
426.01	0\\
427.01	0\\
428.01	0\\
429.01	0\\
430.01	0\\
431.01	0\\
432.01	0\\
433.01	0\\
434.01	0\\
435.01	0\\
436.01	0\\
437.01	0\\
438.01	0\\
439.01	0\\
440.01	0\\
441.01	0\\
442.01	0\\
443.01	0\\
444.01	0\\
445.01	0\\
446.01	0\\
447.01	0\\
448.01	0\\
449.01	0\\
450.01	0\\
451.01	1.73472347597681e-18\\
452.01	0\\
453.01	0\\
454.01	0\\
455.01	0\\
456.01	0\\
457.01	0\\
458.01	0\\
459.01	0\\
460.01	0\\
461.01	0\\
462.01	0\\
463.01	0\\
464.01	0\\
465.01	0\\
466.01	0\\
467.01	0\\
468.01	0\\
469.01	1.73472347597681e-18\\
470.01	0\\
471.01	0\\
472.01	0\\
473.01	0\\
474.01	0\\
475.01	0\\
476.01	1.73472347597681e-18\\
477.01	0\\
478.01	0\\
479.01	0\\
480.01	0\\
481.01	0\\
482.01	0\\
483.01	0\\
484.01	0\\
485.01	0\\
486.01	0\\
487.01	0\\
488.01	0\\
489.01	0\\
490.01	0\\
491.01	0\\
492.01	0\\
493.01	0\\
494.01	0\\
495.01	0\\
496.01	0\\
497.01	0\\
498.01	0\\
499.01	0\\
500.01	0\\
501.01	0\\
502.01	0\\
503.01	0\\
504.01	0\\
505.01	0\\
506.01	0\\
507.01	0\\
508.01	0\\
509.01	0\\
510.01	0\\
511.01	0\\
512.01	1.73472347597681e-18\\
513.01	0\\
514.01	0\\
515.01	0\\
516.01	0\\
517.01	0\\
518.01	0\\
519.01	0\\
520.01	0\\
521.01	0\\
522.01	0\\
523.01	1.73472347597681e-18\\
524.01	0\\
525.01	0\\
526.01	0\\
527.01	0\\
528.01	0\\
529.01	0\\
530.01	0\\
531.01	0\\
532.01	0\\
533.01	0\\
534.01	0\\
535.01	0\\
536.01	1.73472347597681e-18\\
537.01	1.73472347597681e-18\\
538.01	1.73472347597681e-18\\
539.01	0\\
540.01	0\\
541.01	0\\
542.01	0\\
543.01	0\\
544.01	0\\
545.01	1.73472347597681e-18\\
546.01	1.73472347597681e-18\\
547.01	0\\
548.01	0\\
549.01	0\\
550.01	0\\
551.01	0\\
552.01	0\\
553.01	0\\
554.01	0\\
555.01	0\\
556.01	1.73472347597681e-18\\
557.01	0\\
558.01	0\\
559.01	0\\
560.01	1.73472347597681e-18\\
561.01	1.73472347597681e-18\\
562.01	0\\
563.01	0\\
564.01	0\\
565.01	0\\
566.01	0\\
567.01	0\\
568.01	0\\
569.01	0\\
570.01	0\\
571.01	0\\
572.01	0\\
573.01	0\\
574.01	0\\
575.01	0\\
576.01	0\\
577.01	0\\
578.01	1.73472347597681e-18\\
579.01	0\\
580.01	0\\
581.01	0\\
582.01	0\\
583.01	1.73472347597681e-18\\
584.01	0\\
585.01	0\\
586.01	0\\
587.01	0\\
588.01	0\\
589.01	0\\
590.01	0\\
591.01	0\\
592.01	0\\
593.01	0\\
594.01	0\\
595.01	0\\
596.01	0\\
597.01	0\\
598.01	0\\
599.01	0\\
599.02	0\\
599.03	0\\
599.04	0\\
599.05	0\\
599.06	0\\
599.07	0\\
599.08	0\\
599.09	0\\
599.1	0\\
599.11	0\\
599.12	0\\
599.13	0\\
599.14	0\\
599.15	0\\
599.16	0\\
599.17	0\\
599.18	0\\
599.19	0\\
599.2	0\\
599.21	0\\
599.22	0\\
599.23	0\\
599.24	0\\
599.25	0\\
599.26	0\\
599.27	0\\
599.28	0\\
599.29	0\\
599.3	0\\
599.31	0\\
599.32	0\\
599.33	0\\
599.34	0\\
599.35	0\\
599.36	0\\
599.37	0\\
599.38	0\\
599.39	0\\
599.4	0\\
599.41	0\\
599.42	0\\
599.43	0\\
599.44	0\\
599.45	0\\
599.46	0\\
599.47	0\\
599.48	0\\
599.49	0\\
599.5	0\\
599.51	0\\
599.52	0\\
599.53	0\\
599.54	0\\
599.55	0\\
599.56	0\\
599.57	0\\
599.58	0\\
599.59	0\\
599.6	0\\
599.61	0\\
599.62	0\\
599.63	0\\
599.64	0\\
599.65	0\\
599.66	0\\
599.67	0\\
599.68	0\\
599.69	0\\
599.7	0\\
599.71	0\\
599.72	0\\
599.73	0\\
599.74	0\\
599.75	0\\
599.76	0\\
599.77	0\\
599.78	0\\
599.79	0\\
599.8	0\\
599.81	0\\
599.82	0\\
599.83	0\\
599.84	0\\
599.85	0\\
599.86	0\\
599.87	0\\
599.88	0\\
599.89	0\\
599.9	0\\
599.91	0\\
599.92	0\\
599.93	0\\
599.94	0\\
599.95	0\\
599.96	0\\
599.97	0\\
599.98	0\\
599.99	0\\
600	0\\
};
\addplot [color=mycolor8,solid,forget plot]
  table[row sep=crcr]{%
0.01	0\\
1.01	0\\
2.01	0\\
3.01	0\\
4.01	0\\
5.01	0\\
6.01	0\\
7.01	0\\
8.01	0\\
9.01	0\\
10.01	0\\
11.01	0\\
12.01	0\\
13.01	0\\
14.01	0\\
15.01	0\\
16.01	0\\
17.01	0\\
18.01	0\\
19.01	0\\
20.01	0\\
21.01	0\\
22.01	0\\
23.01	0\\
24.01	0\\
25.01	0\\
26.01	0\\
27.01	0\\
28.01	0\\
29.01	0\\
30.01	0\\
31.01	0\\
32.01	0\\
33.01	0\\
34.01	0\\
35.01	0\\
36.01	0\\
37.01	0\\
38.01	0\\
39.01	0\\
40.01	0\\
41.01	0\\
42.01	0\\
43.01	0\\
44.01	0\\
45.01	0\\
46.01	0\\
47.01	0\\
48.01	0\\
49.01	0\\
50.01	0\\
51.01	0\\
52.01	0\\
53.01	0\\
54.01	0\\
55.01	0\\
56.01	0\\
57.01	0\\
58.01	0\\
59.01	0\\
60.01	0\\
61.01	0\\
62.01	0\\
63.01	0\\
64.01	0\\
65.01	0\\
66.01	0\\
67.01	0\\
68.01	0\\
69.01	0\\
70.01	0\\
71.01	0\\
72.01	0\\
73.01	0\\
74.01	0\\
75.01	0\\
76.01	0\\
77.01	0\\
78.01	0\\
79.01	0\\
80.01	0\\
81.01	0\\
82.01	0\\
83.01	0\\
84.01	0\\
85.01	0\\
86.01	0\\
87.01	0\\
88.01	0\\
89.01	0\\
90.01	0\\
91.01	0\\
92.01	0\\
93.01	0\\
94.01	0\\
95.01	0\\
96.01	0\\
97.01	0\\
98.01	0\\
99.01	0\\
100.01	0\\
101.01	0\\
102.01	0\\
103.01	0\\
104.01	0\\
105.01	0\\
106.01	0\\
107.01	0\\
108.01	0\\
109.01	0\\
110.01	0\\
111.01	0\\
112.01	0\\
113.01	0\\
114.01	0\\
115.01	0\\
116.01	0\\
117.01	0\\
118.01	0\\
119.01	0\\
120.01	0\\
121.01	0\\
122.01	0\\
123.01	0\\
124.01	0\\
125.01	0\\
126.01	0\\
127.01	0\\
128.01	0\\
129.01	0\\
130.01	0\\
131.01	0\\
132.01	0\\
133.01	0\\
134.01	0\\
135.01	0\\
136.01	0\\
137.01	0\\
138.01	0\\
139.01	0\\
140.01	0\\
141.01	0\\
142.01	0\\
143.01	0\\
144.01	0\\
145.01	0\\
146.01	0\\
147.01	0\\
148.01	0\\
149.01	0\\
150.01	0\\
151.01	0\\
152.01	0\\
153.01	0\\
154.01	0\\
155.01	0\\
156.01	0\\
157.01	0\\
158.01	0\\
159.01	0\\
160.01	0\\
161.01	0\\
162.01	0\\
163.01	0\\
164.01	0\\
165.01	0\\
166.01	0\\
167.01	0\\
168.01	0\\
169.01	0\\
170.01	0\\
171.01	0\\
172.01	0\\
173.01	0\\
174.01	0\\
175.01	0\\
176.01	0\\
177.01	0\\
178.01	0\\
179.01	0\\
180.01	0\\
181.01	0\\
182.01	0\\
183.01	0\\
184.01	0\\
185.01	0\\
186.01	0\\
187.01	0\\
188.01	0\\
189.01	0\\
190.01	0\\
191.01	0\\
192.01	0\\
193.01	0\\
194.01	0\\
195.01	0\\
196.01	0\\
197.01	0\\
198.01	0\\
199.01	0\\
200.01	0\\
201.01	0\\
202.01	0\\
203.01	0\\
204.01	0\\
205.01	0\\
206.01	0\\
207.01	0\\
208.01	0\\
209.01	0\\
210.01	0\\
211.01	0\\
212.01	0\\
213.01	0\\
214.01	0\\
215.01	0\\
216.01	0\\
217.01	0\\
218.01	0\\
219.01	0\\
220.01	0\\
221.01	0\\
222.01	0\\
223.01	0\\
224.01	0\\
225.01	0\\
226.01	0\\
227.01	0\\
228.01	0\\
229.01	0\\
230.01	0\\
231.01	0\\
232.01	0\\
233.01	0\\
234.01	0\\
235.01	0\\
236.01	0\\
237.01	0\\
238.01	0\\
239.01	0\\
240.01	0\\
241.01	0\\
242.01	0\\
243.01	0\\
244.01	0\\
245.01	0\\
246.01	0\\
247.01	0\\
248.01	0\\
249.01	0\\
250.01	0\\
251.01	0\\
252.01	0\\
253.01	0\\
254.01	0\\
255.01	0\\
256.01	0\\
257.01	0\\
258.01	0\\
259.01	0\\
260.01	0\\
261.01	0\\
262.01	0\\
263.01	0\\
264.01	0\\
265.01	0\\
266.01	0\\
267.01	0\\
268.01	0\\
269.01	0\\
270.01	0\\
271.01	0\\
272.01	0\\
273.01	0\\
274.01	0\\
275.01	0\\
276.01	0\\
277.01	0\\
278.01	0\\
279.01	0\\
280.01	0\\
281.01	0\\
282.01	0\\
283.01	0\\
284.01	0\\
285.01	0\\
286.01	0\\
287.01	0\\
288.01	0\\
289.01	0\\
290.01	0\\
291.01	0\\
292.01	0\\
293.01	0\\
294.01	0\\
295.01	0\\
296.01	0\\
297.01	0\\
298.01	0\\
299.01	0\\
300.01	0\\
301.01	0\\
302.01	0\\
303.01	0\\
304.01	0\\
305.01	0\\
306.01	0\\
307.01	0\\
308.01	0\\
309.01	0\\
310.01	0\\
311.01	0\\
312.01	0\\
313.01	0\\
314.01	0\\
315.01	0\\
316.01	0\\
317.01	0\\
318.01	0\\
319.01	0\\
320.01	0\\
321.01	0\\
322.01	0\\
323.01	0\\
324.01	0\\
325.01	0\\
326.01	0\\
327.01	0\\
328.01	0\\
329.01	0\\
330.01	0\\
331.01	0\\
332.01	0\\
333.01	0\\
334.01	0\\
335.01	0\\
336.01	0\\
337.01	0\\
338.01	0\\
339.01	0\\
340.01	0\\
341.01	0\\
342.01	0\\
343.01	0\\
344.01	0\\
345.01	0\\
346.01	0\\
347.01	0\\
348.01	0\\
349.01	0\\
350.01	0\\
351.01	0\\
352.01	0\\
353.01	0\\
354.01	0\\
355.01	0\\
356.01	0\\
357.01	0\\
358.01	0\\
359.01	0\\
360.01	0\\
361.01	0\\
362.01	0\\
363.01	0\\
364.01	0\\
365.01	0\\
366.01	0\\
367.01	0\\
368.01	0\\
369.01	0\\
370.01	0\\
371.01	0\\
372.01	0\\
373.01	0\\
374.01	0\\
375.01	0\\
376.01	0\\
377.01	0\\
378.01	0\\
379.01	0\\
380.01	0\\
381.01	0\\
382.01	0\\
383.01	0\\
384.01	0\\
385.01	0\\
386.01	0\\
387.01	0\\
388.01	0\\
389.01	0\\
390.01	0\\
391.01	0\\
392.01	0\\
393.01	0\\
394.01	0\\
395.01	0\\
396.01	0\\
397.01	0\\
398.01	0\\
399.01	0\\
400.01	0\\
401.01	0\\
402.01	0\\
403.01	0\\
404.01	0\\
405.01	0\\
406.01	0\\
407.01	0\\
408.01	0\\
409.01	0\\
410.01	0\\
411.01	0\\
412.01	0\\
413.01	0\\
414.01	0\\
415.01	0\\
416.01	0\\
417.01	0\\
418.01	0\\
419.01	0\\
420.01	0\\
421.01	0\\
422.01	0\\
423.01	0\\
424.01	0\\
425.01	0\\
426.01	0\\
427.01	0\\
428.01	0\\
429.01	0\\
430.01	0\\
431.01	0\\
432.01	0\\
433.01	0\\
434.01	0\\
435.01	0\\
436.01	0\\
437.01	0\\
438.01	0\\
439.01	0\\
440.01	0\\
441.01	0\\
442.01	0\\
443.01	0\\
444.01	0\\
445.01	0\\
446.01	0\\
447.01	0\\
448.01	0\\
449.01	0\\
450.01	0\\
451.01	1.73472347597681e-18\\
452.01	0\\
453.01	0\\
454.01	0\\
455.01	0\\
456.01	0\\
457.01	0\\
458.01	0\\
459.01	0\\
460.01	0\\
461.01	0\\
462.01	0\\
463.01	0\\
464.01	0\\
465.01	0\\
466.01	0\\
467.01	0\\
468.01	0\\
469.01	1.73472347597681e-18\\
470.01	0\\
471.01	0\\
472.01	0\\
473.01	0\\
474.01	0\\
475.01	0\\
476.01	1.73472347597681e-18\\
477.01	0\\
478.01	0\\
479.01	0\\
480.01	0\\
481.01	0\\
482.01	0\\
483.01	0\\
484.01	0\\
485.01	0\\
486.01	0\\
487.01	0\\
488.01	0\\
489.01	0\\
490.01	0\\
491.01	0\\
492.01	0\\
493.01	0\\
494.01	0\\
495.01	0\\
496.01	0\\
497.01	0\\
498.01	0\\
499.01	0\\
500.01	0\\
501.01	0\\
502.01	0\\
503.01	0\\
504.01	0\\
505.01	0\\
506.01	0\\
507.01	0\\
508.01	0\\
509.01	0\\
510.01	0\\
511.01	0\\
512.01	1.73472347597681e-18\\
513.01	0\\
514.01	0\\
515.01	0\\
516.01	0\\
517.01	0\\
518.01	0\\
519.01	0\\
520.01	0\\
521.01	0\\
522.01	0\\
523.01	1.73472347597681e-18\\
524.01	0\\
525.01	0\\
526.01	0\\
527.01	0\\
528.01	0\\
529.01	0\\
530.01	0\\
531.01	0\\
532.01	0\\
533.01	0\\
534.01	0\\
535.01	0\\
536.01	1.73472347597681e-18\\
537.01	1.73472347597681e-18\\
538.01	1.73472347597681e-18\\
539.01	0\\
540.01	0\\
541.01	0\\
542.01	0\\
543.01	0\\
544.01	0\\
545.01	1.73472347597681e-18\\
546.01	1.73472347597681e-18\\
547.01	0\\
548.01	0\\
549.01	0\\
550.01	0\\
551.01	0\\
552.01	0\\
553.01	0\\
554.01	0\\
555.01	0\\
556.01	1.73472347597681e-18\\
557.01	0\\
558.01	0\\
559.01	0\\
560.01	1.73472347597681e-18\\
561.01	1.73472347597681e-18\\
562.01	0\\
563.01	0\\
564.01	0\\
565.01	0\\
566.01	0\\
567.01	0\\
568.01	0\\
569.01	0\\
570.01	0\\
571.01	0\\
572.01	0\\
573.01	0\\
574.01	0\\
575.01	0\\
576.01	0\\
577.01	0\\
578.01	1.73472347597681e-18\\
579.01	0\\
580.01	0\\
581.01	0\\
582.01	0\\
583.01	1.73472347597681e-18\\
584.01	0\\
585.01	0\\
586.01	0\\
587.01	0\\
588.01	0\\
589.01	0\\
590.01	0\\
591.01	0\\
592.01	0\\
593.01	0\\
594.01	0\\
595.01	0\\
596.01	0\\
597.01	0\\
598.01	0\\
599.01	0\\
599.02	0\\
599.03	0\\
599.04	0\\
599.05	0\\
599.06	0\\
599.07	0\\
599.08	0\\
599.09	0\\
599.1	0\\
599.11	0\\
599.12	0\\
599.13	0\\
599.14	0\\
599.15	0\\
599.16	0\\
599.17	0\\
599.18	0\\
599.19	0\\
599.2	0\\
599.21	0\\
599.22	0\\
599.23	0\\
599.24	0\\
599.25	0\\
599.26	0\\
599.27	0\\
599.28	0\\
599.29	0\\
599.3	0\\
599.31	0\\
599.32	0\\
599.33	0\\
599.34	0\\
599.35	0\\
599.36	0\\
599.37	0\\
599.38	0\\
599.39	0\\
599.4	0\\
599.41	0\\
599.42	0\\
599.43	0\\
599.44	0\\
599.45	0\\
599.46	0\\
599.47	0\\
599.48	0\\
599.49	0\\
599.5	0\\
599.51	0\\
599.52	0\\
599.53	0\\
599.54	0\\
599.55	0\\
599.56	0\\
599.57	0\\
599.58	0\\
599.59	0\\
599.6	0\\
599.61	0\\
599.62	0\\
599.63	0\\
599.64	0\\
599.65	0\\
599.66	0\\
599.67	0\\
599.68	0\\
599.69	0\\
599.7	0\\
599.71	0\\
599.72	0\\
599.73	0\\
599.74	0\\
599.75	0\\
599.76	0\\
599.77	0\\
599.78	0\\
599.79	0\\
599.8	0\\
599.81	0\\
599.82	0\\
599.83	0\\
599.84	0\\
599.85	0\\
599.86	0\\
599.87	0\\
599.88	0\\
599.89	0\\
599.9	0\\
599.91	0\\
599.92	0\\
599.93	0\\
599.94	0\\
599.95	0\\
599.96	0\\
599.97	0\\
599.98	0\\
599.99	0\\
600	0\\
};
\addplot [color=blue!25!mycolor7,solid,forget plot]
  table[row sep=crcr]{%
0.01	0\\
1.01	0\\
2.01	0\\
3.01	0\\
4.01	0\\
5.01	0\\
6.01	0\\
7.01	0\\
8.01	0\\
9.01	0\\
10.01	0\\
11.01	0\\
12.01	0\\
13.01	0\\
14.01	0\\
15.01	0\\
16.01	0\\
17.01	0\\
18.01	0\\
19.01	0\\
20.01	0\\
21.01	0\\
22.01	0\\
23.01	0\\
24.01	0\\
25.01	0\\
26.01	0\\
27.01	0\\
28.01	0\\
29.01	0\\
30.01	0\\
31.01	0\\
32.01	0\\
33.01	0\\
34.01	0\\
35.01	0\\
36.01	0\\
37.01	0\\
38.01	0\\
39.01	0\\
40.01	0\\
41.01	0\\
42.01	0\\
43.01	0\\
44.01	0\\
45.01	0\\
46.01	0\\
47.01	0\\
48.01	0\\
49.01	0\\
50.01	0\\
51.01	0\\
52.01	0\\
53.01	0\\
54.01	0\\
55.01	0\\
56.01	0\\
57.01	0\\
58.01	0\\
59.01	0\\
60.01	0\\
61.01	0\\
62.01	0\\
63.01	0\\
64.01	0\\
65.01	0\\
66.01	0\\
67.01	0\\
68.01	0\\
69.01	0\\
70.01	0\\
71.01	0\\
72.01	0\\
73.01	0\\
74.01	0\\
75.01	0\\
76.01	0\\
77.01	0\\
78.01	0\\
79.01	0\\
80.01	0\\
81.01	0\\
82.01	0\\
83.01	0\\
84.01	0\\
85.01	0\\
86.01	0\\
87.01	0\\
88.01	0\\
89.01	0\\
90.01	0\\
91.01	0\\
92.01	0\\
93.01	0\\
94.01	0\\
95.01	0\\
96.01	0\\
97.01	0\\
98.01	0\\
99.01	0\\
100.01	0\\
101.01	0\\
102.01	0\\
103.01	0\\
104.01	0\\
105.01	0\\
106.01	0\\
107.01	0\\
108.01	0\\
109.01	0\\
110.01	0\\
111.01	0\\
112.01	0\\
113.01	0\\
114.01	0\\
115.01	0\\
116.01	0\\
117.01	0\\
118.01	0\\
119.01	0\\
120.01	0\\
121.01	0\\
122.01	0\\
123.01	0\\
124.01	0\\
125.01	0\\
126.01	0\\
127.01	0\\
128.01	0\\
129.01	0\\
130.01	0\\
131.01	0\\
132.01	0\\
133.01	0\\
134.01	0\\
135.01	0\\
136.01	0\\
137.01	0\\
138.01	0\\
139.01	0\\
140.01	0\\
141.01	0\\
142.01	0\\
143.01	0\\
144.01	0\\
145.01	0\\
146.01	0\\
147.01	0\\
148.01	0\\
149.01	0\\
150.01	0\\
151.01	0\\
152.01	0\\
153.01	0\\
154.01	0\\
155.01	0\\
156.01	0\\
157.01	0\\
158.01	0\\
159.01	0\\
160.01	0\\
161.01	0\\
162.01	0\\
163.01	0\\
164.01	0\\
165.01	0\\
166.01	0\\
167.01	0\\
168.01	0\\
169.01	0\\
170.01	0\\
171.01	0\\
172.01	0\\
173.01	0\\
174.01	0\\
175.01	0\\
176.01	0\\
177.01	0\\
178.01	0\\
179.01	0\\
180.01	0\\
181.01	0\\
182.01	0\\
183.01	0\\
184.01	0\\
185.01	0\\
186.01	0\\
187.01	0\\
188.01	0\\
189.01	0\\
190.01	0\\
191.01	0\\
192.01	0\\
193.01	0\\
194.01	0\\
195.01	0\\
196.01	0\\
197.01	0\\
198.01	0\\
199.01	0\\
200.01	0\\
201.01	0\\
202.01	0\\
203.01	0\\
204.01	0\\
205.01	0\\
206.01	0\\
207.01	0\\
208.01	0\\
209.01	0\\
210.01	0\\
211.01	0\\
212.01	0\\
213.01	0\\
214.01	0\\
215.01	0\\
216.01	0\\
217.01	0\\
218.01	0\\
219.01	0\\
220.01	0\\
221.01	0\\
222.01	0\\
223.01	0\\
224.01	0\\
225.01	0\\
226.01	0\\
227.01	0\\
228.01	0\\
229.01	0\\
230.01	0\\
231.01	0\\
232.01	0\\
233.01	0\\
234.01	0\\
235.01	0\\
236.01	0\\
237.01	0\\
238.01	0\\
239.01	0\\
240.01	0\\
241.01	0\\
242.01	0\\
243.01	0\\
244.01	0\\
245.01	0\\
246.01	0\\
247.01	0\\
248.01	0\\
249.01	0\\
250.01	0\\
251.01	0\\
252.01	0\\
253.01	0\\
254.01	0\\
255.01	0\\
256.01	0\\
257.01	0\\
258.01	0\\
259.01	0\\
260.01	0\\
261.01	0\\
262.01	0\\
263.01	0\\
264.01	0\\
265.01	0\\
266.01	0\\
267.01	0\\
268.01	0\\
269.01	0\\
270.01	0\\
271.01	0\\
272.01	0\\
273.01	0\\
274.01	0\\
275.01	0\\
276.01	0\\
277.01	0\\
278.01	0\\
279.01	0\\
280.01	0\\
281.01	0\\
282.01	0\\
283.01	0\\
284.01	0\\
285.01	0\\
286.01	0\\
287.01	0\\
288.01	0\\
289.01	0\\
290.01	0\\
291.01	0\\
292.01	0\\
293.01	0\\
294.01	0\\
295.01	0\\
296.01	0\\
297.01	0\\
298.01	0\\
299.01	0\\
300.01	0\\
301.01	0\\
302.01	0\\
303.01	0\\
304.01	0\\
305.01	0\\
306.01	0\\
307.01	0\\
308.01	0\\
309.01	0\\
310.01	0\\
311.01	0\\
312.01	0\\
313.01	0\\
314.01	0\\
315.01	0\\
316.01	0\\
317.01	0\\
318.01	0\\
319.01	0\\
320.01	0\\
321.01	0\\
322.01	0\\
323.01	0\\
324.01	0\\
325.01	0\\
326.01	0\\
327.01	0\\
328.01	0\\
329.01	0\\
330.01	0\\
331.01	0\\
332.01	0\\
333.01	0\\
334.01	0\\
335.01	0\\
336.01	0\\
337.01	0\\
338.01	0\\
339.01	0\\
340.01	0\\
341.01	0\\
342.01	0\\
343.01	0\\
344.01	0\\
345.01	0\\
346.01	0\\
347.01	0\\
348.01	0\\
349.01	0\\
350.01	0\\
351.01	0\\
352.01	0\\
353.01	0\\
354.01	0\\
355.01	0\\
356.01	0\\
357.01	0\\
358.01	0\\
359.01	0\\
360.01	0\\
361.01	0\\
362.01	0\\
363.01	0\\
364.01	0\\
365.01	0\\
366.01	0\\
367.01	0\\
368.01	0\\
369.01	0\\
370.01	0\\
371.01	0\\
372.01	0\\
373.01	0\\
374.01	0\\
375.01	0\\
376.01	0\\
377.01	0\\
378.01	0\\
379.01	0\\
380.01	0\\
381.01	0\\
382.01	0\\
383.01	0\\
384.01	0\\
385.01	0\\
386.01	0\\
387.01	0\\
388.01	0\\
389.01	0\\
390.01	0\\
391.01	0\\
392.01	0\\
393.01	0\\
394.01	0\\
395.01	0\\
396.01	0\\
397.01	0\\
398.01	0\\
399.01	0\\
400.01	0\\
401.01	0\\
402.01	0\\
403.01	0\\
404.01	0\\
405.01	0\\
406.01	0\\
407.01	0\\
408.01	0\\
409.01	0\\
410.01	0\\
411.01	0\\
412.01	0\\
413.01	0\\
414.01	0\\
415.01	0\\
416.01	0\\
417.01	0\\
418.01	0\\
419.01	0\\
420.01	0\\
421.01	0\\
422.01	0\\
423.01	0\\
424.01	0\\
425.01	0\\
426.01	0\\
427.01	0\\
428.01	0\\
429.01	0\\
430.01	0\\
431.01	0\\
432.01	0\\
433.01	0\\
434.01	0\\
435.01	0\\
436.01	0\\
437.01	0\\
438.01	0\\
439.01	0\\
440.01	0\\
441.01	0\\
442.01	0\\
443.01	0\\
444.01	0\\
445.01	0\\
446.01	0\\
447.01	0\\
448.01	0\\
449.01	0\\
450.01	0\\
451.01	1.73472347597681e-18\\
452.01	0\\
453.01	0\\
454.01	0\\
455.01	0\\
456.01	0\\
457.01	0\\
458.01	0\\
459.01	0\\
460.01	0\\
461.01	0\\
462.01	0\\
463.01	0\\
464.01	0\\
465.01	0\\
466.01	0\\
467.01	0\\
468.01	0\\
469.01	1.73472347597681e-18\\
470.01	0\\
471.01	0\\
472.01	0\\
473.01	0\\
474.01	0\\
475.01	0\\
476.01	1.73472347597681e-18\\
477.01	0\\
478.01	0\\
479.01	0\\
480.01	0\\
481.01	0\\
482.01	0\\
483.01	0\\
484.01	0\\
485.01	0\\
486.01	0\\
487.01	0\\
488.01	0\\
489.01	0\\
490.01	0\\
491.01	0\\
492.01	0\\
493.01	0\\
494.01	0\\
495.01	0\\
496.01	0\\
497.01	0\\
498.01	0\\
499.01	0\\
500.01	0\\
501.01	0\\
502.01	0\\
503.01	0\\
504.01	0\\
505.01	0\\
506.01	0\\
507.01	0\\
508.01	0\\
509.01	0\\
510.01	0\\
511.01	0\\
512.01	1.73472347597681e-18\\
513.01	0\\
514.01	0\\
515.01	0\\
516.01	0\\
517.01	0\\
518.01	0\\
519.01	0\\
520.01	0\\
521.01	0\\
522.01	0\\
523.01	1.73472347597681e-18\\
524.01	0\\
525.01	0\\
526.01	0\\
527.01	0\\
528.01	0\\
529.01	0\\
530.01	0\\
531.01	0\\
532.01	0\\
533.01	0\\
534.01	0\\
535.01	0\\
536.01	1.73472347597681e-18\\
537.01	1.73472347597681e-18\\
538.01	1.73472347597681e-18\\
539.01	0\\
540.01	0\\
541.01	0\\
542.01	0\\
543.01	0\\
544.01	0\\
545.01	1.73472347597681e-18\\
546.01	1.73472347597681e-18\\
547.01	0\\
548.01	0\\
549.01	0\\
550.01	0\\
551.01	0\\
552.01	0\\
553.01	0\\
554.01	0\\
555.01	0\\
556.01	1.73472347597681e-18\\
557.01	0\\
558.01	0\\
559.01	0\\
560.01	1.73472347597681e-18\\
561.01	1.73472347597681e-18\\
562.01	0\\
563.01	0\\
564.01	0\\
565.01	0\\
566.01	0\\
567.01	0\\
568.01	0\\
569.01	0\\
570.01	0\\
571.01	0\\
572.01	0\\
573.01	0\\
574.01	0\\
575.01	0\\
576.01	0\\
577.01	0\\
578.01	1.73472347597681e-18\\
579.01	0\\
580.01	0\\
581.01	0\\
582.01	0\\
583.01	1.73472347597681e-18\\
584.01	0\\
585.01	0\\
586.01	0\\
587.01	0\\
588.01	0\\
589.01	0\\
590.01	0\\
591.01	0\\
592.01	0\\
593.01	0\\
594.01	0\\
595.01	0\\
596.01	0\\
597.01	0\\
598.01	0\\
599.01	0\\
599.02	0\\
599.03	0\\
599.04	0\\
599.05	0\\
599.06	0\\
599.07	0\\
599.08	0\\
599.09	0\\
599.1	0\\
599.11	0\\
599.12	0\\
599.13	0\\
599.14	0\\
599.15	0\\
599.16	0\\
599.17	0\\
599.18	0\\
599.19	0\\
599.2	0\\
599.21	0\\
599.22	0\\
599.23	0\\
599.24	0\\
599.25	0\\
599.26	0\\
599.27	0\\
599.28	0\\
599.29	0\\
599.3	0\\
599.31	0\\
599.32	0\\
599.33	0\\
599.34	0\\
599.35	0\\
599.36	0\\
599.37	0\\
599.38	0\\
599.39	0\\
599.4	0\\
599.41	0\\
599.42	0\\
599.43	0\\
599.44	0\\
599.45	0\\
599.46	0\\
599.47	0\\
599.48	0\\
599.49	0\\
599.5	0\\
599.51	0\\
599.52	0\\
599.53	0\\
599.54	0\\
599.55	0\\
599.56	0\\
599.57	0\\
599.58	0\\
599.59	0\\
599.6	0\\
599.61	0\\
599.62	0\\
599.63	0\\
599.64	0\\
599.65	0\\
599.66	0\\
599.67	0\\
599.68	0\\
599.69	0\\
599.7	0\\
599.71	0\\
599.72	0\\
599.73	0\\
599.74	0\\
599.75	0\\
599.76	0\\
599.77	0\\
599.78	0\\
599.79	0\\
599.8	0\\
599.81	0\\
599.82	0\\
599.83	0\\
599.84	0\\
599.85	0\\
599.86	0\\
599.87	0\\
599.88	0\\
599.89	0\\
599.9	0\\
599.91	0\\
599.92	0\\
599.93	0\\
599.94	0\\
599.95	0\\
599.96	0\\
599.97	0\\
599.98	0\\
599.99	0\\
600	0\\
};
\addplot [color=mycolor9,solid,forget plot]
  table[row sep=crcr]{%
0.01	0\\
1.01	0\\
2.01	0\\
3.01	0\\
4.01	0\\
5.01	0\\
6.01	0\\
7.01	0\\
8.01	0\\
9.01	0\\
10.01	0\\
11.01	0\\
12.01	0\\
13.01	0\\
14.01	0\\
15.01	0\\
16.01	0\\
17.01	0\\
18.01	0\\
19.01	0\\
20.01	0\\
21.01	0\\
22.01	0\\
23.01	0\\
24.01	0\\
25.01	0\\
26.01	0\\
27.01	0\\
28.01	0\\
29.01	0\\
30.01	0\\
31.01	0\\
32.01	0\\
33.01	0\\
34.01	0\\
35.01	0\\
36.01	0\\
37.01	0\\
38.01	0\\
39.01	0\\
40.01	0\\
41.01	0\\
42.01	0\\
43.01	0\\
44.01	0\\
45.01	0\\
46.01	0\\
47.01	0\\
48.01	0\\
49.01	0\\
50.01	0\\
51.01	0\\
52.01	0\\
53.01	0\\
54.01	0\\
55.01	0\\
56.01	0\\
57.01	0\\
58.01	0\\
59.01	0\\
60.01	0\\
61.01	0\\
62.01	0\\
63.01	0\\
64.01	0\\
65.01	0\\
66.01	0\\
67.01	0\\
68.01	0\\
69.01	0\\
70.01	0\\
71.01	0\\
72.01	0\\
73.01	0\\
74.01	0\\
75.01	0\\
76.01	0\\
77.01	0\\
78.01	0\\
79.01	0\\
80.01	0\\
81.01	0\\
82.01	0\\
83.01	0\\
84.01	0\\
85.01	0\\
86.01	0\\
87.01	0\\
88.01	0\\
89.01	0\\
90.01	0\\
91.01	0\\
92.01	0\\
93.01	0\\
94.01	0\\
95.01	0\\
96.01	0\\
97.01	0\\
98.01	0\\
99.01	0\\
100.01	0\\
101.01	0\\
102.01	0\\
103.01	0\\
104.01	0\\
105.01	0\\
106.01	0\\
107.01	0\\
108.01	0\\
109.01	0\\
110.01	0\\
111.01	0\\
112.01	0\\
113.01	0\\
114.01	0\\
115.01	0\\
116.01	0\\
117.01	0\\
118.01	0\\
119.01	0\\
120.01	0\\
121.01	0\\
122.01	0\\
123.01	0\\
124.01	0\\
125.01	0\\
126.01	0\\
127.01	0\\
128.01	0\\
129.01	0\\
130.01	0\\
131.01	0\\
132.01	0\\
133.01	0\\
134.01	0\\
135.01	0\\
136.01	0\\
137.01	0\\
138.01	0\\
139.01	0\\
140.01	0\\
141.01	0\\
142.01	0\\
143.01	0\\
144.01	0\\
145.01	0\\
146.01	0\\
147.01	0\\
148.01	0\\
149.01	0\\
150.01	0\\
151.01	0\\
152.01	0\\
153.01	0\\
154.01	0\\
155.01	0\\
156.01	0\\
157.01	0\\
158.01	0\\
159.01	0\\
160.01	0\\
161.01	0\\
162.01	0\\
163.01	0\\
164.01	0\\
165.01	0\\
166.01	0\\
167.01	0\\
168.01	0\\
169.01	0\\
170.01	0\\
171.01	0\\
172.01	0\\
173.01	0\\
174.01	0\\
175.01	0\\
176.01	0\\
177.01	0\\
178.01	0\\
179.01	0\\
180.01	0\\
181.01	0\\
182.01	0\\
183.01	0\\
184.01	0\\
185.01	0\\
186.01	0\\
187.01	0\\
188.01	0\\
189.01	0\\
190.01	0\\
191.01	0\\
192.01	0\\
193.01	0\\
194.01	0\\
195.01	0\\
196.01	0\\
197.01	0\\
198.01	0\\
199.01	0\\
200.01	0\\
201.01	0\\
202.01	0\\
203.01	0\\
204.01	0\\
205.01	0\\
206.01	0\\
207.01	0\\
208.01	0\\
209.01	0\\
210.01	0\\
211.01	0\\
212.01	0\\
213.01	0\\
214.01	0\\
215.01	0\\
216.01	0\\
217.01	0\\
218.01	0\\
219.01	0\\
220.01	0\\
221.01	0\\
222.01	0\\
223.01	0\\
224.01	0\\
225.01	0\\
226.01	0\\
227.01	0\\
228.01	0\\
229.01	0\\
230.01	0\\
231.01	0\\
232.01	0\\
233.01	0\\
234.01	0\\
235.01	0\\
236.01	0\\
237.01	0\\
238.01	0\\
239.01	0\\
240.01	0\\
241.01	0\\
242.01	0\\
243.01	0\\
244.01	0\\
245.01	0\\
246.01	0\\
247.01	0\\
248.01	0\\
249.01	0\\
250.01	0\\
251.01	0\\
252.01	0\\
253.01	0\\
254.01	0\\
255.01	0\\
256.01	0\\
257.01	0\\
258.01	0\\
259.01	0\\
260.01	0\\
261.01	0\\
262.01	0\\
263.01	0\\
264.01	0\\
265.01	0\\
266.01	0\\
267.01	0\\
268.01	0\\
269.01	0\\
270.01	0\\
271.01	0\\
272.01	0\\
273.01	0\\
274.01	0\\
275.01	0\\
276.01	0\\
277.01	0\\
278.01	0\\
279.01	0\\
280.01	0\\
281.01	0\\
282.01	0\\
283.01	0\\
284.01	0\\
285.01	0\\
286.01	0\\
287.01	0\\
288.01	0\\
289.01	0\\
290.01	0\\
291.01	0\\
292.01	0\\
293.01	0\\
294.01	0\\
295.01	0\\
296.01	0\\
297.01	0\\
298.01	0\\
299.01	0\\
300.01	0\\
301.01	0\\
302.01	0\\
303.01	0\\
304.01	0\\
305.01	0\\
306.01	0\\
307.01	0\\
308.01	0\\
309.01	0\\
310.01	0\\
311.01	0\\
312.01	0\\
313.01	0\\
314.01	0\\
315.01	0\\
316.01	0\\
317.01	0\\
318.01	0\\
319.01	0\\
320.01	0\\
321.01	0\\
322.01	0\\
323.01	0\\
324.01	0\\
325.01	0\\
326.01	0\\
327.01	0\\
328.01	0\\
329.01	0\\
330.01	0\\
331.01	0\\
332.01	0\\
333.01	0\\
334.01	0\\
335.01	0\\
336.01	0\\
337.01	0\\
338.01	0\\
339.01	0\\
340.01	0\\
341.01	0\\
342.01	0\\
343.01	0\\
344.01	0\\
345.01	0\\
346.01	0\\
347.01	0\\
348.01	0\\
349.01	0\\
350.01	0\\
351.01	0\\
352.01	0\\
353.01	0\\
354.01	0\\
355.01	0\\
356.01	0\\
357.01	0\\
358.01	0\\
359.01	0\\
360.01	0\\
361.01	0\\
362.01	0\\
363.01	0\\
364.01	0\\
365.01	0\\
366.01	0\\
367.01	0\\
368.01	0\\
369.01	0\\
370.01	0\\
371.01	0\\
372.01	0\\
373.01	0\\
374.01	0\\
375.01	0\\
376.01	0\\
377.01	0\\
378.01	0\\
379.01	0\\
380.01	0\\
381.01	0\\
382.01	0\\
383.01	0\\
384.01	0\\
385.01	0\\
386.01	0\\
387.01	0\\
388.01	0\\
389.01	0\\
390.01	0\\
391.01	0\\
392.01	0\\
393.01	0\\
394.01	0\\
395.01	0\\
396.01	0\\
397.01	0\\
398.01	0\\
399.01	0\\
400.01	0\\
401.01	0\\
402.01	0\\
403.01	0\\
404.01	0\\
405.01	0\\
406.01	0\\
407.01	0\\
408.01	0\\
409.01	0\\
410.01	0\\
411.01	0\\
412.01	0\\
413.01	0\\
414.01	0\\
415.01	0\\
416.01	0\\
417.01	0\\
418.01	0\\
419.01	0\\
420.01	0\\
421.01	0\\
422.01	0\\
423.01	0\\
424.01	0\\
425.01	0\\
426.01	0\\
427.01	0\\
428.01	0\\
429.01	0\\
430.01	0\\
431.01	0\\
432.01	0\\
433.01	0\\
434.01	0\\
435.01	0\\
436.01	0\\
437.01	0\\
438.01	0\\
439.01	0\\
440.01	0\\
441.01	0\\
442.01	0\\
443.01	0\\
444.01	0\\
445.01	0\\
446.01	0\\
447.01	0\\
448.01	0\\
449.01	0\\
450.01	0\\
451.01	1.73472347597681e-18\\
452.01	0\\
453.01	0\\
454.01	0\\
455.01	0\\
456.01	0\\
457.01	0\\
458.01	0\\
459.01	0\\
460.01	0\\
461.01	0\\
462.01	0\\
463.01	0\\
464.01	0\\
465.01	0\\
466.01	0\\
467.01	0\\
468.01	0\\
469.01	1.73472347597681e-18\\
470.01	0\\
471.01	0\\
472.01	0\\
473.01	0\\
474.01	0\\
475.01	0\\
476.01	1.73472347597681e-18\\
477.01	0\\
478.01	0\\
479.01	0\\
480.01	0\\
481.01	0\\
482.01	0\\
483.01	0\\
484.01	0\\
485.01	0\\
486.01	0\\
487.01	0\\
488.01	0\\
489.01	0\\
490.01	0\\
491.01	0\\
492.01	0\\
493.01	0\\
494.01	0\\
495.01	0\\
496.01	0\\
497.01	0\\
498.01	0\\
499.01	0\\
500.01	0\\
501.01	0\\
502.01	0\\
503.01	0\\
504.01	0\\
505.01	0\\
506.01	0\\
507.01	0\\
508.01	0\\
509.01	0\\
510.01	0\\
511.01	0\\
512.01	1.73472347597681e-18\\
513.01	0\\
514.01	0\\
515.01	0\\
516.01	0\\
517.01	0\\
518.01	0\\
519.01	0\\
520.01	0\\
521.01	0\\
522.01	0\\
523.01	1.73472347597681e-18\\
524.01	0\\
525.01	0\\
526.01	0\\
527.01	0\\
528.01	0\\
529.01	0\\
530.01	0\\
531.01	0\\
532.01	0\\
533.01	0\\
534.01	0\\
535.01	0\\
536.01	1.73472347597681e-18\\
537.01	1.73472347597681e-18\\
538.01	1.73472347597681e-18\\
539.01	0\\
540.01	0\\
541.01	0\\
542.01	0\\
543.01	0\\
544.01	0\\
545.01	1.73472347597681e-18\\
546.01	1.73472347597681e-18\\
547.01	0\\
548.01	0\\
549.01	0\\
550.01	0\\
551.01	0\\
552.01	0\\
553.01	0\\
554.01	0\\
555.01	0\\
556.01	1.73472347597681e-18\\
557.01	0\\
558.01	0\\
559.01	0\\
560.01	1.73472347597681e-18\\
561.01	1.73472347597681e-18\\
562.01	0\\
563.01	0\\
564.01	0\\
565.01	0\\
566.01	0\\
567.01	0\\
568.01	0\\
569.01	0\\
570.01	0\\
571.01	0\\
572.01	0\\
573.01	0\\
574.01	0\\
575.01	0\\
576.01	0\\
577.01	0\\
578.01	1.73472347597681e-18\\
579.01	0\\
580.01	0\\
581.01	0\\
582.01	0\\
583.01	1.73472347597681e-18\\
584.01	0\\
585.01	0\\
586.01	0\\
587.01	0\\
588.01	0\\
589.01	0\\
590.01	0\\
591.01	0\\
592.01	0\\
593.01	0\\
594.01	0\\
595.01	0\\
596.01	0\\
597.01	0\\
598.01	0\\
599.01	0\\
599.02	0\\
599.03	0\\
599.04	0\\
599.05	0\\
599.06	0\\
599.07	0\\
599.08	0\\
599.09	0\\
599.1	0\\
599.11	0\\
599.12	0\\
599.13	0\\
599.14	0\\
599.15	0\\
599.16	0\\
599.17	0\\
599.18	0\\
599.19	0\\
599.2	0\\
599.21	0\\
599.22	0\\
599.23	0\\
599.24	0\\
599.25	0\\
599.26	0\\
599.27	0\\
599.28	0\\
599.29	0\\
599.3	0\\
599.31	0\\
599.32	0\\
599.33	0\\
599.34	0\\
599.35	0\\
599.36	0\\
599.37	0\\
599.38	0\\
599.39	0\\
599.4	0\\
599.41	0\\
599.42	0\\
599.43	0\\
599.44	0\\
599.45	0\\
599.46	0\\
599.47	0\\
599.48	0\\
599.49	0\\
599.5	0\\
599.51	0\\
599.52	0\\
599.53	0\\
599.54	0\\
599.55	0\\
599.56	0\\
599.57	0\\
599.58	0\\
599.59	0\\
599.6	0\\
599.61	0\\
599.62	0\\
599.63	0\\
599.64	0\\
599.65	0\\
599.66	0\\
599.67	0\\
599.68	0\\
599.69	0\\
599.7	0\\
599.71	0\\
599.72	0\\
599.73	0\\
599.74	0\\
599.75	0\\
599.76	0\\
599.77	0\\
599.78	0\\
599.79	0\\
599.8	0\\
599.81	0\\
599.82	0\\
599.83	0\\
599.84	0\\
599.85	0\\
599.86	0\\
599.87	0\\
599.88	0\\
599.89	0\\
599.9	0\\
599.91	0\\
599.92	0\\
599.93	0\\
599.94	0\\
599.95	0\\
599.96	0\\
599.97	0\\
599.98	0\\
599.99	0\\
600	0\\
};
\addplot [color=blue!50!mycolor7,solid,forget plot]
  table[row sep=crcr]{%
0.01	0\\
1.01	0\\
2.01	0\\
3.01	0\\
4.01	0\\
5.01	0\\
6.01	0\\
7.01	0\\
8.01	0\\
9.01	0\\
10.01	0\\
11.01	0\\
12.01	0\\
13.01	0\\
14.01	0\\
15.01	0\\
16.01	0\\
17.01	0\\
18.01	0\\
19.01	0\\
20.01	0\\
21.01	0\\
22.01	0\\
23.01	0\\
24.01	0\\
25.01	0\\
26.01	0\\
27.01	0\\
28.01	0\\
29.01	0\\
30.01	0\\
31.01	0\\
32.01	0\\
33.01	0\\
34.01	0\\
35.01	0\\
36.01	0\\
37.01	0\\
38.01	0\\
39.01	0\\
40.01	0\\
41.01	0\\
42.01	0\\
43.01	0\\
44.01	0\\
45.01	0\\
46.01	0\\
47.01	0\\
48.01	0\\
49.01	0\\
50.01	0\\
51.01	0\\
52.01	0\\
53.01	0\\
54.01	0\\
55.01	0\\
56.01	0\\
57.01	0\\
58.01	0\\
59.01	0\\
60.01	0\\
61.01	0\\
62.01	0\\
63.01	0\\
64.01	0\\
65.01	0\\
66.01	0\\
67.01	0\\
68.01	0\\
69.01	0\\
70.01	0\\
71.01	0\\
72.01	0\\
73.01	0\\
74.01	0\\
75.01	0\\
76.01	0\\
77.01	0\\
78.01	0\\
79.01	0\\
80.01	0\\
81.01	0\\
82.01	0\\
83.01	0\\
84.01	0\\
85.01	0\\
86.01	0\\
87.01	0\\
88.01	0\\
89.01	0\\
90.01	0\\
91.01	0\\
92.01	0\\
93.01	0\\
94.01	0\\
95.01	0\\
96.01	0\\
97.01	0\\
98.01	0\\
99.01	0\\
100.01	0\\
101.01	0\\
102.01	0\\
103.01	0\\
104.01	0\\
105.01	0\\
106.01	0\\
107.01	0\\
108.01	0\\
109.01	0\\
110.01	0\\
111.01	0\\
112.01	0\\
113.01	0\\
114.01	0\\
115.01	0\\
116.01	0\\
117.01	0\\
118.01	0\\
119.01	0\\
120.01	0\\
121.01	0\\
122.01	0\\
123.01	0\\
124.01	0\\
125.01	0\\
126.01	0\\
127.01	0\\
128.01	0\\
129.01	0\\
130.01	0\\
131.01	0\\
132.01	0\\
133.01	0\\
134.01	0\\
135.01	0\\
136.01	0\\
137.01	0\\
138.01	0\\
139.01	0\\
140.01	0\\
141.01	0\\
142.01	0\\
143.01	0\\
144.01	0\\
145.01	0\\
146.01	0\\
147.01	0\\
148.01	0\\
149.01	0\\
150.01	0\\
151.01	0\\
152.01	0\\
153.01	0\\
154.01	0\\
155.01	0\\
156.01	0\\
157.01	0\\
158.01	0\\
159.01	0\\
160.01	0\\
161.01	0\\
162.01	0\\
163.01	0\\
164.01	0\\
165.01	0\\
166.01	0\\
167.01	0\\
168.01	0\\
169.01	0\\
170.01	0\\
171.01	0\\
172.01	0\\
173.01	0\\
174.01	0\\
175.01	0\\
176.01	0\\
177.01	0\\
178.01	0\\
179.01	0\\
180.01	0\\
181.01	0\\
182.01	0\\
183.01	0\\
184.01	0\\
185.01	0\\
186.01	0\\
187.01	0\\
188.01	0\\
189.01	0\\
190.01	0\\
191.01	0\\
192.01	0\\
193.01	0\\
194.01	0\\
195.01	0\\
196.01	0\\
197.01	0\\
198.01	0\\
199.01	0\\
200.01	0\\
201.01	0\\
202.01	0\\
203.01	0\\
204.01	0\\
205.01	0\\
206.01	0\\
207.01	0\\
208.01	0\\
209.01	0\\
210.01	0\\
211.01	0\\
212.01	0\\
213.01	0\\
214.01	0\\
215.01	0\\
216.01	0\\
217.01	0\\
218.01	0\\
219.01	0\\
220.01	0\\
221.01	0\\
222.01	0\\
223.01	0\\
224.01	0\\
225.01	0\\
226.01	0\\
227.01	0\\
228.01	0\\
229.01	0\\
230.01	0\\
231.01	0\\
232.01	0\\
233.01	0\\
234.01	0\\
235.01	0\\
236.01	0\\
237.01	0\\
238.01	0\\
239.01	0\\
240.01	0\\
241.01	0\\
242.01	0\\
243.01	0\\
244.01	0\\
245.01	0\\
246.01	0\\
247.01	0\\
248.01	0\\
249.01	0\\
250.01	0\\
251.01	0\\
252.01	0\\
253.01	0\\
254.01	0\\
255.01	0\\
256.01	0\\
257.01	0\\
258.01	0\\
259.01	0\\
260.01	0\\
261.01	0\\
262.01	0\\
263.01	0\\
264.01	0\\
265.01	0\\
266.01	0\\
267.01	0\\
268.01	0\\
269.01	0\\
270.01	0\\
271.01	0\\
272.01	0\\
273.01	0\\
274.01	0\\
275.01	0\\
276.01	0\\
277.01	0\\
278.01	0\\
279.01	0\\
280.01	0\\
281.01	0\\
282.01	0\\
283.01	0\\
284.01	0\\
285.01	0\\
286.01	0\\
287.01	0\\
288.01	0\\
289.01	0\\
290.01	0\\
291.01	0\\
292.01	0\\
293.01	0\\
294.01	0\\
295.01	0\\
296.01	0\\
297.01	0\\
298.01	0\\
299.01	0\\
300.01	0\\
301.01	0\\
302.01	0\\
303.01	0\\
304.01	0\\
305.01	0\\
306.01	0\\
307.01	0\\
308.01	0\\
309.01	0\\
310.01	0\\
311.01	0\\
312.01	0\\
313.01	0\\
314.01	0\\
315.01	0\\
316.01	0\\
317.01	0\\
318.01	0\\
319.01	0\\
320.01	0\\
321.01	0\\
322.01	0\\
323.01	0\\
324.01	0\\
325.01	0\\
326.01	0\\
327.01	0\\
328.01	0\\
329.01	0\\
330.01	0\\
331.01	0\\
332.01	0\\
333.01	0\\
334.01	0\\
335.01	0\\
336.01	0\\
337.01	0\\
338.01	0\\
339.01	0\\
340.01	0\\
341.01	0\\
342.01	0\\
343.01	0\\
344.01	0\\
345.01	0\\
346.01	0\\
347.01	0\\
348.01	0\\
349.01	0\\
350.01	0\\
351.01	0\\
352.01	0\\
353.01	0\\
354.01	0\\
355.01	0\\
356.01	0\\
357.01	0\\
358.01	0\\
359.01	0\\
360.01	0\\
361.01	0\\
362.01	0\\
363.01	0\\
364.01	0\\
365.01	0\\
366.01	0\\
367.01	0\\
368.01	0\\
369.01	0\\
370.01	0\\
371.01	0\\
372.01	0\\
373.01	0\\
374.01	0\\
375.01	0\\
376.01	0\\
377.01	0\\
378.01	0\\
379.01	0\\
380.01	0\\
381.01	0\\
382.01	0\\
383.01	0\\
384.01	0\\
385.01	0\\
386.01	0\\
387.01	0\\
388.01	0\\
389.01	0\\
390.01	0\\
391.01	0\\
392.01	0\\
393.01	0\\
394.01	0\\
395.01	0\\
396.01	0\\
397.01	0\\
398.01	0\\
399.01	0\\
400.01	0\\
401.01	0\\
402.01	0\\
403.01	0\\
404.01	0\\
405.01	0\\
406.01	0\\
407.01	0\\
408.01	0\\
409.01	0\\
410.01	0\\
411.01	0\\
412.01	0\\
413.01	0\\
414.01	0\\
415.01	0\\
416.01	0\\
417.01	0\\
418.01	0\\
419.01	0\\
420.01	0\\
421.01	0\\
422.01	0\\
423.01	0\\
424.01	0\\
425.01	0\\
426.01	0\\
427.01	0\\
428.01	0\\
429.01	0\\
430.01	0\\
431.01	0\\
432.01	0\\
433.01	0\\
434.01	0\\
435.01	0\\
436.01	0\\
437.01	0\\
438.01	0\\
439.01	0\\
440.01	0\\
441.01	0\\
442.01	0\\
443.01	0\\
444.01	0\\
445.01	0\\
446.01	0\\
447.01	0\\
448.01	0\\
449.01	0\\
450.01	0\\
451.01	1.73472347597681e-18\\
452.01	0\\
453.01	0\\
454.01	0\\
455.01	0\\
456.01	0\\
457.01	0\\
458.01	0\\
459.01	0\\
460.01	0\\
461.01	0\\
462.01	0\\
463.01	0\\
464.01	0\\
465.01	0\\
466.01	0\\
467.01	0\\
468.01	0\\
469.01	1.73472347597681e-18\\
470.01	0\\
471.01	0\\
472.01	0\\
473.01	0\\
474.01	0\\
475.01	0\\
476.01	1.73472347597681e-18\\
477.01	0\\
478.01	0\\
479.01	0\\
480.01	0\\
481.01	0\\
482.01	0\\
483.01	0\\
484.01	0\\
485.01	0\\
486.01	0\\
487.01	0\\
488.01	0\\
489.01	0\\
490.01	0\\
491.01	0\\
492.01	0\\
493.01	0\\
494.01	0\\
495.01	0\\
496.01	0\\
497.01	0\\
498.01	0\\
499.01	0\\
500.01	0\\
501.01	0\\
502.01	0\\
503.01	0\\
504.01	0\\
505.01	0\\
506.01	0\\
507.01	0\\
508.01	0\\
509.01	0\\
510.01	0\\
511.01	0\\
512.01	1.73472347597681e-18\\
513.01	0\\
514.01	0\\
515.01	0\\
516.01	0\\
517.01	0\\
518.01	0\\
519.01	0\\
520.01	0\\
521.01	0\\
522.01	0\\
523.01	1.73472347597681e-18\\
524.01	0\\
525.01	0\\
526.01	0\\
527.01	0\\
528.01	0\\
529.01	0\\
530.01	0\\
531.01	0\\
532.01	0\\
533.01	0\\
534.01	0\\
535.01	0\\
536.01	1.73472347597681e-18\\
537.01	1.73472347597681e-18\\
538.01	1.73472347597681e-18\\
539.01	0\\
540.01	0\\
541.01	0\\
542.01	0\\
543.01	0\\
544.01	0\\
545.01	1.73472347597681e-18\\
546.01	1.73472347597681e-18\\
547.01	0\\
548.01	0\\
549.01	0\\
550.01	0\\
551.01	0\\
552.01	0\\
553.01	0\\
554.01	0\\
555.01	0\\
556.01	1.73472347597681e-18\\
557.01	0\\
558.01	0\\
559.01	0\\
560.01	1.73472347597681e-18\\
561.01	1.73472347597681e-18\\
562.01	0\\
563.01	0\\
564.01	0\\
565.01	0\\
566.01	0\\
567.01	0\\
568.01	0\\
569.01	0\\
570.01	0\\
571.01	0\\
572.01	0\\
573.01	0\\
574.01	0\\
575.01	0\\
576.01	0\\
577.01	0\\
578.01	1.73472347597681e-18\\
579.01	0\\
580.01	0\\
581.01	0\\
582.01	0\\
583.01	1.73472347597681e-18\\
584.01	0\\
585.01	0\\
586.01	0\\
587.01	0\\
588.01	0\\
589.01	0\\
590.01	0\\
591.01	0\\
592.01	0\\
593.01	0\\
594.01	0\\
595.01	0\\
596.01	0\\
597.01	0\\
598.01	0\\
599.01	0\\
599.02	0\\
599.03	0\\
599.04	0\\
599.05	0\\
599.06	0\\
599.07	0\\
599.08	0\\
599.09	0\\
599.1	0\\
599.11	0\\
599.12	0\\
599.13	0\\
599.14	0\\
599.15	0\\
599.16	0\\
599.17	0\\
599.18	0\\
599.19	0\\
599.2	0\\
599.21	0\\
599.22	0\\
599.23	0\\
599.24	0\\
599.25	0\\
599.26	0\\
599.27	0\\
599.28	0\\
599.29	0\\
599.3	0\\
599.31	0\\
599.32	0\\
599.33	0\\
599.34	0\\
599.35	0\\
599.36	0\\
599.37	0\\
599.38	0\\
599.39	0\\
599.4	0\\
599.41	0\\
599.42	0\\
599.43	0\\
599.44	0\\
599.45	0\\
599.46	0\\
599.47	0\\
599.48	0\\
599.49	0\\
599.5	0\\
599.51	0\\
599.52	0\\
599.53	0\\
599.54	0\\
599.55	0\\
599.56	0\\
599.57	0\\
599.58	0\\
599.59	0\\
599.6	0\\
599.61	0\\
599.62	0\\
599.63	0\\
599.64	0\\
599.65	0\\
599.66	0\\
599.67	0\\
599.68	0\\
599.69	0\\
599.7	0\\
599.71	0\\
599.72	0\\
599.73	0\\
599.74	0\\
599.75	0\\
599.76	0\\
599.77	0\\
599.78	0\\
599.79	0\\
599.8	0\\
599.81	0\\
599.82	0\\
599.83	0\\
599.84	0\\
599.85	0\\
599.86	0\\
599.87	0\\
599.88	0\\
599.89	0\\
599.9	0\\
599.91	0\\
599.92	0\\
599.93	0\\
599.94	0\\
599.95	0\\
599.96	0\\
599.97	0\\
599.98	0\\
599.99	0\\
600	0\\
};
\addplot [color=blue!40!mycolor9,solid,forget plot]
  table[row sep=crcr]{%
0.01	0\\
1.01	0\\
2.01	0\\
3.01	0\\
4.01	0\\
5.01	0\\
6.01	0\\
7.01	0\\
8.01	0\\
9.01	0\\
10.01	0\\
11.01	0\\
12.01	0\\
13.01	0\\
14.01	0\\
15.01	0\\
16.01	0\\
17.01	0\\
18.01	0\\
19.01	0\\
20.01	0\\
21.01	0\\
22.01	0\\
23.01	0\\
24.01	0\\
25.01	0\\
26.01	0\\
27.01	0\\
28.01	0\\
29.01	0\\
30.01	0\\
31.01	0\\
32.01	0\\
33.01	0\\
34.01	0\\
35.01	0\\
36.01	0\\
37.01	0\\
38.01	0\\
39.01	0\\
40.01	0\\
41.01	0\\
42.01	0\\
43.01	0\\
44.01	0\\
45.01	0\\
46.01	0\\
47.01	0\\
48.01	0\\
49.01	0\\
50.01	0\\
51.01	0\\
52.01	0\\
53.01	0\\
54.01	0\\
55.01	0\\
56.01	0\\
57.01	0\\
58.01	0\\
59.01	0\\
60.01	0\\
61.01	0\\
62.01	0\\
63.01	0\\
64.01	0\\
65.01	0\\
66.01	0\\
67.01	0\\
68.01	0\\
69.01	0\\
70.01	0\\
71.01	0\\
72.01	0\\
73.01	0\\
74.01	0\\
75.01	0\\
76.01	0\\
77.01	0\\
78.01	0\\
79.01	0\\
80.01	0\\
81.01	0\\
82.01	0\\
83.01	0\\
84.01	0\\
85.01	0\\
86.01	0\\
87.01	0\\
88.01	0\\
89.01	0\\
90.01	0\\
91.01	0\\
92.01	0\\
93.01	0\\
94.01	0\\
95.01	0\\
96.01	0\\
97.01	0\\
98.01	0\\
99.01	0\\
100.01	0\\
101.01	0\\
102.01	0\\
103.01	0\\
104.01	0\\
105.01	0\\
106.01	0\\
107.01	0\\
108.01	0\\
109.01	0\\
110.01	0\\
111.01	0\\
112.01	0\\
113.01	0\\
114.01	0\\
115.01	0\\
116.01	0\\
117.01	0\\
118.01	0\\
119.01	0\\
120.01	0\\
121.01	0\\
122.01	0\\
123.01	0\\
124.01	0\\
125.01	0\\
126.01	0\\
127.01	0\\
128.01	0\\
129.01	0\\
130.01	0\\
131.01	0\\
132.01	0\\
133.01	0\\
134.01	0\\
135.01	0\\
136.01	0\\
137.01	0\\
138.01	0\\
139.01	0\\
140.01	0\\
141.01	0\\
142.01	0\\
143.01	0\\
144.01	0\\
145.01	0\\
146.01	0\\
147.01	0\\
148.01	0\\
149.01	0\\
150.01	0\\
151.01	0\\
152.01	0\\
153.01	0\\
154.01	0\\
155.01	0\\
156.01	0\\
157.01	0\\
158.01	0\\
159.01	0\\
160.01	0\\
161.01	0\\
162.01	0\\
163.01	0\\
164.01	0\\
165.01	0\\
166.01	0\\
167.01	0\\
168.01	0\\
169.01	0\\
170.01	0\\
171.01	0\\
172.01	0\\
173.01	0\\
174.01	0\\
175.01	0\\
176.01	0\\
177.01	0\\
178.01	0\\
179.01	0\\
180.01	0\\
181.01	0\\
182.01	0\\
183.01	0\\
184.01	0\\
185.01	0\\
186.01	0\\
187.01	0\\
188.01	0\\
189.01	0\\
190.01	0\\
191.01	0\\
192.01	0\\
193.01	0\\
194.01	0\\
195.01	0\\
196.01	0\\
197.01	0\\
198.01	0\\
199.01	0\\
200.01	0\\
201.01	0\\
202.01	0\\
203.01	0\\
204.01	0\\
205.01	0\\
206.01	0\\
207.01	0\\
208.01	0\\
209.01	0\\
210.01	0\\
211.01	0\\
212.01	0\\
213.01	0\\
214.01	0\\
215.01	0\\
216.01	0\\
217.01	0\\
218.01	0\\
219.01	0\\
220.01	0\\
221.01	0\\
222.01	0\\
223.01	0\\
224.01	0\\
225.01	0\\
226.01	0\\
227.01	0\\
228.01	0\\
229.01	0\\
230.01	0\\
231.01	0\\
232.01	0\\
233.01	0\\
234.01	0\\
235.01	0\\
236.01	0\\
237.01	0\\
238.01	0\\
239.01	0\\
240.01	0\\
241.01	0\\
242.01	0\\
243.01	0\\
244.01	0\\
245.01	0\\
246.01	0\\
247.01	0\\
248.01	0\\
249.01	0\\
250.01	0\\
251.01	0\\
252.01	0\\
253.01	0\\
254.01	0\\
255.01	0\\
256.01	0\\
257.01	0\\
258.01	0\\
259.01	0\\
260.01	0\\
261.01	0\\
262.01	0\\
263.01	0\\
264.01	0\\
265.01	0\\
266.01	0\\
267.01	0\\
268.01	0\\
269.01	0\\
270.01	0\\
271.01	0\\
272.01	0\\
273.01	0\\
274.01	0\\
275.01	0\\
276.01	0\\
277.01	0\\
278.01	0\\
279.01	0\\
280.01	0\\
281.01	0\\
282.01	0\\
283.01	0\\
284.01	0\\
285.01	0\\
286.01	0\\
287.01	0\\
288.01	0\\
289.01	0\\
290.01	0\\
291.01	0\\
292.01	0\\
293.01	0\\
294.01	0\\
295.01	0\\
296.01	0\\
297.01	0\\
298.01	0\\
299.01	0\\
300.01	0\\
301.01	0\\
302.01	0\\
303.01	0\\
304.01	0\\
305.01	0\\
306.01	0\\
307.01	0\\
308.01	0\\
309.01	0\\
310.01	0\\
311.01	0\\
312.01	0\\
313.01	0\\
314.01	0\\
315.01	0\\
316.01	0\\
317.01	0\\
318.01	0\\
319.01	0\\
320.01	0\\
321.01	0\\
322.01	0\\
323.01	0\\
324.01	0\\
325.01	0\\
326.01	0\\
327.01	0\\
328.01	0\\
329.01	0\\
330.01	0\\
331.01	0\\
332.01	0\\
333.01	0\\
334.01	0\\
335.01	0\\
336.01	0\\
337.01	0\\
338.01	0\\
339.01	0\\
340.01	0\\
341.01	0\\
342.01	0\\
343.01	0\\
344.01	0\\
345.01	0\\
346.01	0\\
347.01	0\\
348.01	0\\
349.01	0\\
350.01	0\\
351.01	0\\
352.01	0\\
353.01	0\\
354.01	0\\
355.01	0\\
356.01	0\\
357.01	0\\
358.01	0\\
359.01	0\\
360.01	0\\
361.01	0\\
362.01	0\\
363.01	0\\
364.01	0\\
365.01	0\\
366.01	0\\
367.01	0\\
368.01	0\\
369.01	0\\
370.01	0\\
371.01	0\\
372.01	0\\
373.01	0\\
374.01	0\\
375.01	0\\
376.01	0\\
377.01	0\\
378.01	0\\
379.01	0\\
380.01	0\\
381.01	0\\
382.01	0\\
383.01	0\\
384.01	0\\
385.01	0\\
386.01	0\\
387.01	0\\
388.01	0\\
389.01	0\\
390.01	0\\
391.01	0\\
392.01	0\\
393.01	0\\
394.01	0\\
395.01	0\\
396.01	0\\
397.01	0\\
398.01	0\\
399.01	0\\
400.01	0\\
401.01	0\\
402.01	0\\
403.01	0\\
404.01	0\\
405.01	0\\
406.01	0\\
407.01	0\\
408.01	0\\
409.01	0\\
410.01	0\\
411.01	0\\
412.01	0\\
413.01	0\\
414.01	0\\
415.01	0\\
416.01	0\\
417.01	0\\
418.01	0\\
419.01	0\\
420.01	0\\
421.01	0\\
422.01	0\\
423.01	0\\
424.01	0\\
425.01	0\\
426.01	0\\
427.01	0\\
428.01	0\\
429.01	0\\
430.01	0\\
431.01	0\\
432.01	0\\
433.01	0\\
434.01	0\\
435.01	0\\
436.01	0\\
437.01	0\\
438.01	0\\
439.01	0\\
440.01	0\\
441.01	0\\
442.01	0\\
443.01	0\\
444.01	0\\
445.01	0\\
446.01	0\\
447.01	0\\
448.01	0\\
449.01	0\\
450.01	0\\
451.01	1.73472347597681e-18\\
452.01	0\\
453.01	0\\
454.01	0\\
455.01	0\\
456.01	0\\
457.01	0\\
458.01	0\\
459.01	0\\
460.01	0\\
461.01	0\\
462.01	0\\
463.01	0\\
464.01	0\\
465.01	0\\
466.01	0\\
467.01	0\\
468.01	0\\
469.01	1.73472347597681e-18\\
470.01	0\\
471.01	0\\
472.01	0\\
473.01	0\\
474.01	0\\
475.01	0\\
476.01	1.73472347597681e-18\\
477.01	0\\
478.01	0\\
479.01	0\\
480.01	0\\
481.01	0\\
482.01	0\\
483.01	0\\
484.01	0\\
485.01	0\\
486.01	0\\
487.01	0\\
488.01	0\\
489.01	0\\
490.01	0\\
491.01	0\\
492.01	0\\
493.01	0\\
494.01	0\\
495.01	0\\
496.01	0\\
497.01	0\\
498.01	0\\
499.01	0\\
500.01	0\\
501.01	0\\
502.01	0\\
503.01	0\\
504.01	0\\
505.01	0\\
506.01	0\\
507.01	0\\
508.01	0\\
509.01	0\\
510.01	0\\
511.01	0\\
512.01	1.73472347597681e-18\\
513.01	0\\
514.01	0\\
515.01	0\\
516.01	0\\
517.01	0\\
518.01	0\\
519.01	0\\
520.01	0\\
521.01	0\\
522.01	0\\
523.01	1.73472347597681e-18\\
524.01	0\\
525.01	0\\
526.01	0\\
527.01	0\\
528.01	0\\
529.01	0\\
530.01	0\\
531.01	0\\
532.01	0\\
533.01	0\\
534.01	0\\
535.01	0\\
536.01	1.73472347597681e-18\\
537.01	1.73472347597681e-18\\
538.01	1.73472347597681e-18\\
539.01	0\\
540.01	0\\
541.01	0\\
542.01	0\\
543.01	0\\
544.01	0\\
545.01	1.73472347597681e-18\\
546.01	1.73472347597681e-18\\
547.01	0\\
548.01	0\\
549.01	0\\
550.01	0\\
551.01	0\\
552.01	0\\
553.01	0\\
554.01	0\\
555.01	0\\
556.01	1.73472347597681e-18\\
557.01	0\\
558.01	0\\
559.01	0\\
560.01	1.73472347597681e-18\\
561.01	1.73472347597681e-18\\
562.01	0\\
563.01	0\\
564.01	0\\
565.01	0\\
566.01	0\\
567.01	0\\
568.01	0\\
569.01	0\\
570.01	0\\
571.01	0\\
572.01	0\\
573.01	0\\
574.01	0\\
575.01	0\\
576.01	0\\
577.01	0\\
578.01	1.73472347597681e-18\\
579.01	0\\
580.01	0\\
581.01	0\\
582.01	0\\
583.01	1.73472347597681e-18\\
584.01	0\\
585.01	0\\
586.01	0\\
587.01	0\\
588.01	0\\
589.01	0\\
590.01	0\\
591.01	0\\
592.01	0\\
593.01	0\\
594.01	0\\
595.01	0\\
596.01	0\\
597.01	0\\
598.01	0\\
599.01	0\\
599.02	0\\
599.03	0\\
599.04	0\\
599.05	0\\
599.06	0\\
599.07	0\\
599.08	0\\
599.09	0\\
599.1	0\\
599.11	0\\
599.12	0\\
599.13	0\\
599.14	0\\
599.15	0\\
599.16	0\\
599.17	0\\
599.18	0\\
599.19	0\\
599.2	0\\
599.21	0\\
599.22	0\\
599.23	0\\
599.24	0\\
599.25	0\\
599.26	0\\
599.27	0\\
599.28	0\\
599.29	0\\
599.3	0\\
599.31	0\\
599.32	0\\
599.33	0\\
599.34	0\\
599.35	0\\
599.36	0\\
599.37	0\\
599.38	0\\
599.39	0\\
599.4	0\\
599.41	0\\
599.42	0\\
599.43	0\\
599.44	0\\
599.45	0\\
599.46	0\\
599.47	0\\
599.48	0\\
599.49	0\\
599.5	0\\
599.51	0\\
599.52	0\\
599.53	0\\
599.54	0\\
599.55	0\\
599.56	0\\
599.57	0\\
599.58	0\\
599.59	0\\
599.6	0\\
599.61	0\\
599.62	0\\
599.63	0\\
599.64	0\\
599.65	0\\
599.66	0\\
599.67	0\\
599.68	0\\
599.69	0\\
599.7	0\\
599.71	0\\
599.72	0\\
599.73	0\\
599.74	0\\
599.75	0\\
599.76	0\\
599.77	0\\
599.78	0\\
599.79	0\\
599.8	0\\
599.81	0\\
599.82	0\\
599.83	0\\
599.84	0\\
599.85	0\\
599.86	0\\
599.87	0\\
599.88	0\\
599.89	0\\
599.9	0\\
599.91	0\\
599.92	0\\
599.93	0\\
599.94	0\\
599.95	0\\
599.96	0\\
599.97	0\\
599.98	0\\
599.99	0\\
600	0\\
};
\addplot [color=blue!75!mycolor7,solid,forget plot]
  table[row sep=crcr]{%
0.01	0\\
1.01	0\\
2.01	0\\
3.01	0\\
4.01	0\\
5.01	0\\
6.01	0\\
7.01	0\\
8.01	0\\
9.01	0\\
10.01	0\\
11.01	0\\
12.01	0\\
13.01	0\\
14.01	0\\
15.01	0\\
16.01	0\\
17.01	0\\
18.01	0\\
19.01	0\\
20.01	0\\
21.01	0\\
22.01	0\\
23.01	0\\
24.01	0\\
25.01	0\\
26.01	0\\
27.01	0\\
28.01	0\\
29.01	0\\
30.01	0\\
31.01	0\\
32.01	0\\
33.01	0\\
34.01	0\\
35.01	0\\
36.01	0\\
37.01	0\\
38.01	0\\
39.01	0\\
40.01	0\\
41.01	0\\
42.01	0\\
43.01	0\\
44.01	0\\
45.01	0\\
46.01	0\\
47.01	0\\
48.01	0\\
49.01	0\\
50.01	0\\
51.01	0\\
52.01	0\\
53.01	0\\
54.01	0\\
55.01	0\\
56.01	0\\
57.01	0\\
58.01	0\\
59.01	0\\
60.01	0\\
61.01	0\\
62.01	0\\
63.01	0\\
64.01	0\\
65.01	0\\
66.01	0\\
67.01	0\\
68.01	0\\
69.01	0\\
70.01	0\\
71.01	0\\
72.01	0\\
73.01	0\\
74.01	0\\
75.01	0\\
76.01	0\\
77.01	0\\
78.01	0\\
79.01	0\\
80.01	0\\
81.01	0\\
82.01	0\\
83.01	0\\
84.01	0\\
85.01	0\\
86.01	0\\
87.01	0\\
88.01	0\\
89.01	0\\
90.01	0\\
91.01	0\\
92.01	0\\
93.01	0\\
94.01	0\\
95.01	0\\
96.01	0\\
97.01	0\\
98.01	0\\
99.01	0\\
100.01	0\\
101.01	0\\
102.01	0\\
103.01	0\\
104.01	0\\
105.01	0\\
106.01	0\\
107.01	0\\
108.01	0\\
109.01	0\\
110.01	0\\
111.01	0\\
112.01	0\\
113.01	0\\
114.01	0\\
115.01	0\\
116.01	0\\
117.01	0\\
118.01	0\\
119.01	0\\
120.01	0\\
121.01	0\\
122.01	0\\
123.01	0\\
124.01	0\\
125.01	0\\
126.01	0\\
127.01	0\\
128.01	0\\
129.01	0\\
130.01	0\\
131.01	0\\
132.01	0\\
133.01	0\\
134.01	0\\
135.01	0\\
136.01	0\\
137.01	0\\
138.01	0\\
139.01	0\\
140.01	0\\
141.01	0\\
142.01	0\\
143.01	0\\
144.01	0\\
145.01	0\\
146.01	0\\
147.01	0\\
148.01	0\\
149.01	0\\
150.01	0\\
151.01	0\\
152.01	0\\
153.01	0\\
154.01	0\\
155.01	0\\
156.01	0\\
157.01	0\\
158.01	0\\
159.01	0\\
160.01	0\\
161.01	0\\
162.01	0\\
163.01	0\\
164.01	0\\
165.01	0\\
166.01	0\\
167.01	0\\
168.01	0\\
169.01	0\\
170.01	0\\
171.01	0\\
172.01	0\\
173.01	0\\
174.01	0\\
175.01	0\\
176.01	0\\
177.01	0\\
178.01	0\\
179.01	0\\
180.01	0\\
181.01	0\\
182.01	0\\
183.01	0\\
184.01	0\\
185.01	0\\
186.01	0\\
187.01	0\\
188.01	0\\
189.01	0\\
190.01	0\\
191.01	0\\
192.01	0\\
193.01	0\\
194.01	0\\
195.01	0\\
196.01	0\\
197.01	0\\
198.01	0\\
199.01	0\\
200.01	0\\
201.01	0\\
202.01	0\\
203.01	0\\
204.01	0\\
205.01	0\\
206.01	0\\
207.01	0\\
208.01	0\\
209.01	0\\
210.01	0\\
211.01	0\\
212.01	0\\
213.01	0\\
214.01	0\\
215.01	0\\
216.01	0\\
217.01	0\\
218.01	0\\
219.01	0\\
220.01	0\\
221.01	0\\
222.01	0\\
223.01	0\\
224.01	0\\
225.01	0\\
226.01	0\\
227.01	0\\
228.01	0\\
229.01	0\\
230.01	0\\
231.01	0\\
232.01	0\\
233.01	0\\
234.01	0\\
235.01	0\\
236.01	0\\
237.01	0\\
238.01	0\\
239.01	0\\
240.01	0\\
241.01	0\\
242.01	0\\
243.01	0\\
244.01	0\\
245.01	0\\
246.01	0\\
247.01	0\\
248.01	0\\
249.01	0\\
250.01	0\\
251.01	0\\
252.01	0\\
253.01	0\\
254.01	0\\
255.01	0\\
256.01	0\\
257.01	0\\
258.01	0\\
259.01	0\\
260.01	0\\
261.01	0\\
262.01	0\\
263.01	0\\
264.01	0\\
265.01	0\\
266.01	0\\
267.01	0\\
268.01	0\\
269.01	0\\
270.01	0\\
271.01	0\\
272.01	0\\
273.01	0\\
274.01	0\\
275.01	0\\
276.01	0\\
277.01	0\\
278.01	0\\
279.01	0\\
280.01	0\\
281.01	0\\
282.01	0\\
283.01	0\\
284.01	0\\
285.01	0\\
286.01	0\\
287.01	0\\
288.01	0\\
289.01	0\\
290.01	0\\
291.01	0\\
292.01	0\\
293.01	0\\
294.01	0\\
295.01	0\\
296.01	0\\
297.01	0\\
298.01	0\\
299.01	0\\
300.01	0\\
301.01	0\\
302.01	0\\
303.01	0\\
304.01	0\\
305.01	0\\
306.01	0\\
307.01	0\\
308.01	0\\
309.01	0\\
310.01	0\\
311.01	0\\
312.01	0\\
313.01	0\\
314.01	0\\
315.01	0\\
316.01	0\\
317.01	0\\
318.01	0\\
319.01	0\\
320.01	0\\
321.01	0\\
322.01	0\\
323.01	0\\
324.01	0\\
325.01	0\\
326.01	0\\
327.01	0\\
328.01	0\\
329.01	0\\
330.01	0\\
331.01	0\\
332.01	0\\
333.01	0\\
334.01	0\\
335.01	0\\
336.01	0\\
337.01	0\\
338.01	0\\
339.01	0\\
340.01	0\\
341.01	0\\
342.01	0\\
343.01	0\\
344.01	0\\
345.01	0\\
346.01	0\\
347.01	0\\
348.01	0\\
349.01	0\\
350.01	0\\
351.01	0\\
352.01	0\\
353.01	0\\
354.01	0\\
355.01	0\\
356.01	0\\
357.01	0\\
358.01	0\\
359.01	0\\
360.01	0\\
361.01	0\\
362.01	0\\
363.01	0\\
364.01	0\\
365.01	0\\
366.01	0\\
367.01	0\\
368.01	0\\
369.01	0\\
370.01	0\\
371.01	0\\
372.01	0\\
373.01	0\\
374.01	0\\
375.01	0\\
376.01	0\\
377.01	0\\
378.01	0\\
379.01	0\\
380.01	0\\
381.01	0\\
382.01	0\\
383.01	0\\
384.01	0\\
385.01	0\\
386.01	0\\
387.01	0\\
388.01	0\\
389.01	0\\
390.01	0\\
391.01	0\\
392.01	0\\
393.01	0\\
394.01	0\\
395.01	0\\
396.01	0\\
397.01	0\\
398.01	0\\
399.01	0\\
400.01	0\\
401.01	0\\
402.01	0\\
403.01	0\\
404.01	0\\
405.01	0\\
406.01	0\\
407.01	0\\
408.01	0\\
409.01	0\\
410.01	0\\
411.01	0\\
412.01	0\\
413.01	0\\
414.01	0\\
415.01	0\\
416.01	0\\
417.01	0\\
418.01	0\\
419.01	0\\
420.01	0\\
421.01	0\\
422.01	0\\
423.01	0\\
424.01	0\\
425.01	0\\
426.01	0\\
427.01	0\\
428.01	0\\
429.01	0\\
430.01	0\\
431.01	0\\
432.01	0\\
433.01	0\\
434.01	0\\
435.01	0\\
436.01	0\\
437.01	0\\
438.01	0\\
439.01	0\\
440.01	0\\
441.01	0\\
442.01	0\\
443.01	0\\
444.01	0\\
445.01	0\\
446.01	0\\
447.01	0\\
448.01	0\\
449.01	0\\
450.01	0\\
451.01	1.73472347597681e-18\\
452.01	0\\
453.01	0\\
454.01	0\\
455.01	0\\
456.01	0\\
457.01	0\\
458.01	0\\
459.01	0\\
460.01	0\\
461.01	0\\
462.01	0\\
463.01	0\\
464.01	0\\
465.01	0\\
466.01	0\\
467.01	0\\
468.01	0\\
469.01	1.73472347597681e-18\\
470.01	0\\
471.01	0\\
472.01	0\\
473.01	0\\
474.01	0\\
475.01	0\\
476.01	1.73472347597681e-18\\
477.01	0\\
478.01	0\\
479.01	0\\
480.01	0\\
481.01	0\\
482.01	0\\
483.01	0\\
484.01	0\\
485.01	0\\
486.01	0\\
487.01	0\\
488.01	0\\
489.01	0\\
490.01	0\\
491.01	0\\
492.01	0\\
493.01	0\\
494.01	0\\
495.01	0\\
496.01	0\\
497.01	0\\
498.01	0\\
499.01	0\\
500.01	0\\
501.01	0\\
502.01	0\\
503.01	0\\
504.01	0\\
505.01	0\\
506.01	0\\
507.01	0\\
508.01	0\\
509.01	0\\
510.01	0\\
511.01	0\\
512.01	1.73472347597681e-18\\
513.01	0\\
514.01	0\\
515.01	0\\
516.01	0\\
517.01	0\\
518.01	0\\
519.01	0\\
520.01	0\\
521.01	0\\
522.01	0\\
523.01	1.73472347597681e-18\\
524.01	0\\
525.01	0\\
526.01	0\\
527.01	0\\
528.01	0\\
529.01	0\\
530.01	0\\
531.01	0\\
532.01	0\\
533.01	0\\
534.01	0\\
535.01	0\\
536.01	1.73472347597681e-18\\
537.01	1.73472347597681e-18\\
538.01	1.73472347597681e-18\\
539.01	0\\
540.01	0\\
541.01	0\\
542.01	0\\
543.01	0\\
544.01	0\\
545.01	1.73472347597681e-18\\
546.01	1.73472347597681e-18\\
547.01	0\\
548.01	0\\
549.01	0\\
550.01	0\\
551.01	0\\
552.01	0\\
553.01	0\\
554.01	0\\
555.01	0\\
556.01	1.73472347597681e-18\\
557.01	0\\
558.01	0\\
559.01	0\\
560.01	1.73472347597681e-18\\
561.01	1.73472347597681e-18\\
562.01	0\\
563.01	0\\
564.01	0\\
565.01	0\\
566.01	0\\
567.01	0\\
568.01	0\\
569.01	0\\
570.01	0\\
571.01	0\\
572.01	0\\
573.01	0\\
574.01	0\\
575.01	0\\
576.01	0\\
577.01	0\\
578.01	1.73472347597681e-18\\
579.01	0\\
580.01	0\\
581.01	0\\
582.01	0\\
583.01	1.73472347597681e-18\\
584.01	0\\
585.01	0\\
586.01	0\\
587.01	0\\
588.01	0\\
589.01	0\\
590.01	0\\
591.01	0\\
592.01	0\\
593.01	0\\
594.01	0\\
595.01	0\\
596.01	0\\
597.01	0\\
598.01	0\\
599.01	0\\
599.02	0\\
599.03	0\\
599.04	0\\
599.05	0\\
599.06	0\\
599.07	0\\
599.08	0\\
599.09	0\\
599.1	0\\
599.11	0\\
599.12	0\\
599.13	0\\
599.14	0\\
599.15	0\\
599.16	0\\
599.17	0\\
599.18	0\\
599.19	0\\
599.2	0\\
599.21	0\\
599.22	0\\
599.23	0\\
599.24	0\\
599.25	0\\
599.26	0\\
599.27	0\\
599.28	0\\
599.29	0\\
599.3	0\\
599.31	0\\
599.32	0\\
599.33	0\\
599.34	0\\
599.35	0\\
599.36	0\\
599.37	0\\
599.38	0\\
599.39	0\\
599.4	0\\
599.41	0\\
599.42	0\\
599.43	0\\
599.44	0\\
599.45	0\\
599.46	0\\
599.47	0\\
599.48	0\\
599.49	0\\
599.5	0\\
599.51	0\\
599.52	0\\
599.53	0\\
599.54	0\\
599.55	0\\
599.56	0\\
599.57	0\\
599.58	0\\
599.59	0\\
599.6	0\\
599.61	0\\
599.62	0\\
599.63	0\\
599.64	0\\
599.65	0\\
599.66	0\\
599.67	0\\
599.68	0\\
599.69	0\\
599.7	0\\
599.71	0\\
599.72	0\\
599.73	0\\
599.74	0\\
599.75	0\\
599.76	0\\
599.77	0\\
599.78	0\\
599.79	0\\
599.8	0\\
599.81	0\\
599.82	0\\
599.83	0\\
599.84	0\\
599.85	0\\
599.86	0\\
599.87	0\\
599.88	0\\
599.89	0\\
599.9	0\\
599.91	0\\
599.92	0\\
599.93	0\\
599.94	0\\
599.95	0\\
599.96	0\\
599.97	0\\
599.98	0\\
599.99	0\\
600	0\\
};
\addplot [color=blue!80!mycolor9,solid,forget plot]
  table[row sep=crcr]{%
0.01	0\\
1.01	0\\
2.01	0\\
3.01	0\\
4.01	0\\
5.01	0\\
6.01	0\\
7.01	0\\
8.01	0\\
9.01	0\\
10.01	0\\
11.01	0\\
12.01	0\\
13.01	0\\
14.01	0\\
15.01	0\\
16.01	0\\
17.01	0\\
18.01	0\\
19.01	0\\
20.01	0\\
21.01	0\\
22.01	0\\
23.01	0\\
24.01	0\\
25.01	0\\
26.01	0\\
27.01	0\\
28.01	0\\
29.01	0\\
30.01	0\\
31.01	0\\
32.01	0\\
33.01	0\\
34.01	0\\
35.01	0\\
36.01	0\\
37.01	0\\
38.01	0\\
39.01	0\\
40.01	0\\
41.01	0\\
42.01	0\\
43.01	0\\
44.01	0\\
45.01	0\\
46.01	0\\
47.01	0\\
48.01	0\\
49.01	0\\
50.01	0\\
51.01	0\\
52.01	0\\
53.01	0\\
54.01	0\\
55.01	0\\
56.01	0\\
57.01	0\\
58.01	0\\
59.01	0\\
60.01	0\\
61.01	0\\
62.01	0\\
63.01	0\\
64.01	0\\
65.01	0\\
66.01	0\\
67.01	0\\
68.01	0\\
69.01	0\\
70.01	0\\
71.01	0\\
72.01	0\\
73.01	0\\
74.01	0\\
75.01	0\\
76.01	0\\
77.01	0\\
78.01	0\\
79.01	0\\
80.01	0\\
81.01	0\\
82.01	0\\
83.01	0\\
84.01	0\\
85.01	0\\
86.01	0\\
87.01	0\\
88.01	0\\
89.01	0\\
90.01	0\\
91.01	0\\
92.01	0\\
93.01	0\\
94.01	0\\
95.01	0\\
96.01	0\\
97.01	0\\
98.01	0\\
99.01	0\\
100.01	0\\
101.01	0\\
102.01	0\\
103.01	0\\
104.01	0\\
105.01	0\\
106.01	0\\
107.01	0\\
108.01	0\\
109.01	0\\
110.01	0\\
111.01	0\\
112.01	0\\
113.01	0\\
114.01	0\\
115.01	0\\
116.01	0\\
117.01	0\\
118.01	0\\
119.01	0\\
120.01	0\\
121.01	0\\
122.01	0\\
123.01	0\\
124.01	0\\
125.01	0\\
126.01	0\\
127.01	0\\
128.01	0\\
129.01	0\\
130.01	0\\
131.01	0\\
132.01	0\\
133.01	0\\
134.01	0\\
135.01	0\\
136.01	0\\
137.01	0\\
138.01	0\\
139.01	0\\
140.01	0\\
141.01	0\\
142.01	0\\
143.01	0\\
144.01	0\\
145.01	0\\
146.01	0\\
147.01	0\\
148.01	0\\
149.01	0\\
150.01	0\\
151.01	0\\
152.01	0\\
153.01	0\\
154.01	0\\
155.01	0\\
156.01	0\\
157.01	0\\
158.01	0\\
159.01	0\\
160.01	0\\
161.01	0\\
162.01	0\\
163.01	0\\
164.01	0\\
165.01	0\\
166.01	0\\
167.01	0\\
168.01	0\\
169.01	0\\
170.01	0\\
171.01	0\\
172.01	0\\
173.01	0\\
174.01	0\\
175.01	0\\
176.01	0\\
177.01	0\\
178.01	0\\
179.01	0\\
180.01	0\\
181.01	0\\
182.01	0\\
183.01	0\\
184.01	0\\
185.01	0\\
186.01	0\\
187.01	0\\
188.01	0\\
189.01	0\\
190.01	0\\
191.01	0\\
192.01	0\\
193.01	0\\
194.01	0\\
195.01	0\\
196.01	0\\
197.01	0\\
198.01	0\\
199.01	0\\
200.01	0\\
201.01	0\\
202.01	0\\
203.01	0\\
204.01	0\\
205.01	0\\
206.01	0\\
207.01	0\\
208.01	0\\
209.01	0\\
210.01	0\\
211.01	0\\
212.01	0\\
213.01	0\\
214.01	0\\
215.01	0\\
216.01	0\\
217.01	0\\
218.01	0\\
219.01	0\\
220.01	0\\
221.01	0\\
222.01	0\\
223.01	0\\
224.01	0\\
225.01	0\\
226.01	0\\
227.01	0\\
228.01	0\\
229.01	0\\
230.01	0\\
231.01	0\\
232.01	0\\
233.01	0\\
234.01	0\\
235.01	0\\
236.01	0\\
237.01	0\\
238.01	0\\
239.01	0\\
240.01	0\\
241.01	0\\
242.01	0\\
243.01	0\\
244.01	0\\
245.01	0\\
246.01	0\\
247.01	0\\
248.01	0\\
249.01	0\\
250.01	0\\
251.01	0\\
252.01	0\\
253.01	0\\
254.01	0\\
255.01	0\\
256.01	0\\
257.01	0\\
258.01	0\\
259.01	0\\
260.01	0\\
261.01	0\\
262.01	0\\
263.01	0\\
264.01	0\\
265.01	0\\
266.01	0\\
267.01	0\\
268.01	0\\
269.01	0\\
270.01	0\\
271.01	0\\
272.01	0\\
273.01	0\\
274.01	0\\
275.01	0\\
276.01	0\\
277.01	0\\
278.01	0\\
279.01	0\\
280.01	0\\
281.01	0\\
282.01	0\\
283.01	0\\
284.01	0\\
285.01	0\\
286.01	0\\
287.01	0\\
288.01	0\\
289.01	0\\
290.01	0\\
291.01	0\\
292.01	0\\
293.01	0\\
294.01	0\\
295.01	0\\
296.01	0\\
297.01	0\\
298.01	0\\
299.01	0\\
300.01	0\\
301.01	0\\
302.01	0\\
303.01	0\\
304.01	0\\
305.01	0\\
306.01	0\\
307.01	0\\
308.01	0\\
309.01	0\\
310.01	0\\
311.01	0\\
312.01	0\\
313.01	0\\
314.01	0\\
315.01	0\\
316.01	0\\
317.01	0\\
318.01	0\\
319.01	0\\
320.01	0\\
321.01	0\\
322.01	0\\
323.01	0\\
324.01	0\\
325.01	0\\
326.01	0\\
327.01	0\\
328.01	0\\
329.01	0\\
330.01	0\\
331.01	0\\
332.01	0\\
333.01	0\\
334.01	0\\
335.01	0\\
336.01	0\\
337.01	0\\
338.01	0\\
339.01	0\\
340.01	0\\
341.01	0\\
342.01	0\\
343.01	0\\
344.01	0\\
345.01	0\\
346.01	0\\
347.01	0\\
348.01	0\\
349.01	0\\
350.01	0\\
351.01	0\\
352.01	0\\
353.01	0\\
354.01	0\\
355.01	0\\
356.01	0\\
357.01	0\\
358.01	0\\
359.01	0\\
360.01	0\\
361.01	0\\
362.01	0\\
363.01	0\\
364.01	0\\
365.01	0\\
366.01	0\\
367.01	0\\
368.01	0\\
369.01	0\\
370.01	0\\
371.01	0\\
372.01	0\\
373.01	0\\
374.01	0\\
375.01	0\\
376.01	0\\
377.01	0\\
378.01	0\\
379.01	0\\
380.01	0\\
381.01	0\\
382.01	0\\
383.01	0\\
384.01	0\\
385.01	0\\
386.01	0\\
387.01	0\\
388.01	0\\
389.01	0\\
390.01	0\\
391.01	0\\
392.01	0\\
393.01	0\\
394.01	0\\
395.01	0\\
396.01	0\\
397.01	0\\
398.01	0\\
399.01	0\\
400.01	0\\
401.01	0\\
402.01	0\\
403.01	0\\
404.01	0\\
405.01	0\\
406.01	0\\
407.01	0\\
408.01	0\\
409.01	0\\
410.01	0\\
411.01	0\\
412.01	0\\
413.01	0\\
414.01	0\\
415.01	0\\
416.01	0\\
417.01	0\\
418.01	0\\
419.01	0\\
420.01	0\\
421.01	0\\
422.01	0\\
423.01	0\\
424.01	0\\
425.01	0\\
426.01	0\\
427.01	0\\
428.01	0\\
429.01	0\\
430.01	0\\
431.01	0\\
432.01	0\\
433.01	0\\
434.01	0\\
435.01	0\\
436.01	0\\
437.01	0\\
438.01	0\\
439.01	0\\
440.01	0\\
441.01	0\\
442.01	0\\
443.01	0\\
444.01	0\\
445.01	0\\
446.01	0\\
447.01	0\\
448.01	0\\
449.01	0\\
450.01	0\\
451.01	1.73472347597681e-18\\
452.01	0\\
453.01	0\\
454.01	0\\
455.01	0\\
456.01	0\\
457.01	0\\
458.01	0\\
459.01	0\\
460.01	0\\
461.01	0\\
462.01	0\\
463.01	0\\
464.01	0\\
465.01	0\\
466.01	0\\
467.01	0\\
468.01	0\\
469.01	1.73472347597681e-18\\
470.01	0\\
471.01	0\\
472.01	0\\
473.01	0\\
474.01	0\\
475.01	0\\
476.01	1.73472347597681e-18\\
477.01	0\\
478.01	0\\
479.01	0\\
480.01	0\\
481.01	0\\
482.01	0\\
483.01	0\\
484.01	0\\
485.01	0\\
486.01	0\\
487.01	0\\
488.01	0\\
489.01	0\\
490.01	0\\
491.01	0\\
492.01	0\\
493.01	0\\
494.01	0\\
495.01	0\\
496.01	0\\
497.01	0\\
498.01	0\\
499.01	0\\
500.01	0\\
501.01	0\\
502.01	0\\
503.01	0\\
504.01	0\\
505.01	0\\
506.01	0\\
507.01	0\\
508.01	0\\
509.01	0\\
510.01	0\\
511.01	0\\
512.01	1.73472347597681e-18\\
513.01	0\\
514.01	0\\
515.01	0\\
516.01	0\\
517.01	0\\
518.01	0\\
519.01	0\\
520.01	0\\
521.01	0\\
522.01	0\\
523.01	1.73472347597681e-18\\
524.01	0\\
525.01	0\\
526.01	0\\
527.01	0\\
528.01	0\\
529.01	0\\
530.01	0\\
531.01	0\\
532.01	0\\
533.01	0\\
534.01	0\\
535.01	0\\
536.01	1.73472347597681e-18\\
537.01	1.73472347597681e-18\\
538.01	1.73472347597681e-18\\
539.01	0\\
540.01	0\\
541.01	0\\
542.01	0\\
543.01	0\\
544.01	0\\
545.01	1.73472347597681e-18\\
546.01	1.73472347597681e-18\\
547.01	0\\
548.01	0\\
549.01	0\\
550.01	0\\
551.01	0\\
552.01	0\\
553.01	0\\
554.01	0\\
555.01	0\\
556.01	1.73472347597681e-18\\
557.01	0\\
558.01	0\\
559.01	0\\
560.01	1.73472347597681e-18\\
561.01	1.73472347597681e-18\\
562.01	0\\
563.01	0\\
564.01	0\\
565.01	0\\
566.01	0\\
567.01	0\\
568.01	0\\
569.01	0\\
570.01	0\\
571.01	0\\
572.01	0\\
573.01	0\\
574.01	0\\
575.01	0\\
576.01	0\\
577.01	0\\
578.01	1.73472347597681e-18\\
579.01	0\\
580.01	0\\
581.01	0\\
582.01	0\\
583.01	1.73472347597681e-18\\
584.01	0\\
585.01	0\\
586.01	0\\
587.01	0\\
588.01	0\\
589.01	0\\
590.01	0\\
591.01	0\\
592.01	0\\
593.01	0\\
594.01	0\\
595.01	0\\
596.01	0\\
597.01	0\\
598.01	0\\
599.01	0\\
599.02	0\\
599.03	0\\
599.04	0\\
599.05	0\\
599.06	0\\
599.07	0\\
599.08	0\\
599.09	0\\
599.1	0\\
599.11	0\\
599.12	0\\
599.13	0\\
599.14	0\\
599.15	0\\
599.16	0\\
599.17	0\\
599.18	0\\
599.19	0\\
599.2	0\\
599.21	0\\
599.22	0\\
599.23	0\\
599.24	0\\
599.25	0\\
599.26	0\\
599.27	0\\
599.28	0\\
599.29	0\\
599.3	0\\
599.31	0\\
599.32	0\\
599.33	0\\
599.34	0\\
599.35	0\\
599.36	0\\
599.37	0\\
599.38	0\\
599.39	0\\
599.4	0\\
599.41	0\\
599.42	0\\
599.43	0\\
599.44	0\\
599.45	0\\
599.46	0\\
599.47	0\\
599.48	0\\
599.49	0\\
599.5	0\\
599.51	0\\
599.52	0\\
599.53	0\\
599.54	0\\
599.55	0\\
599.56	0\\
599.57	0\\
599.58	0\\
599.59	0\\
599.6	0\\
599.61	0\\
599.62	0\\
599.63	0\\
599.64	0\\
599.65	0\\
599.66	0\\
599.67	0\\
599.68	0\\
599.69	0\\
599.7	0\\
599.71	0\\
599.72	0\\
599.73	0\\
599.74	0\\
599.75	0\\
599.76	0\\
599.77	0\\
599.78	0\\
599.79	0\\
599.8	0\\
599.81	0\\
599.82	0\\
599.83	0\\
599.84	0\\
599.85	0\\
599.86	0\\
599.87	0\\
599.88	0\\
599.89	0\\
599.9	0\\
599.91	0\\
599.92	0\\
599.93	0\\
599.94	0\\
599.95	0\\
599.96	0\\
599.97	0\\
599.98	0\\
599.99	0\\
600	0\\
};
\addplot [color=blue,solid,forget plot]
  table[row sep=crcr]{%
0.01	0\\
1.01	0\\
2.01	0\\
3.01	0\\
4.01	0\\
5.01	0\\
6.01	0\\
7.01	0\\
8.01	0\\
9.01	0\\
10.01	0\\
11.01	0\\
12.01	0\\
13.01	0\\
14.01	0\\
15.01	0\\
16.01	0\\
17.01	0\\
18.01	0\\
19.01	0\\
20.01	0\\
21.01	0\\
22.01	0\\
23.01	0\\
24.01	0\\
25.01	0\\
26.01	0\\
27.01	0\\
28.01	0\\
29.01	0\\
30.01	0\\
31.01	0\\
32.01	0\\
33.01	0\\
34.01	0\\
35.01	0\\
36.01	0\\
37.01	0\\
38.01	0\\
39.01	0\\
40.01	0\\
41.01	0\\
42.01	0\\
43.01	0\\
44.01	0\\
45.01	0\\
46.01	0\\
47.01	0\\
48.01	0\\
49.01	0\\
50.01	0\\
51.01	0\\
52.01	0\\
53.01	0\\
54.01	0\\
55.01	0\\
56.01	0\\
57.01	0\\
58.01	0\\
59.01	0\\
60.01	0\\
61.01	0\\
62.01	0\\
63.01	0\\
64.01	0\\
65.01	0\\
66.01	0\\
67.01	0\\
68.01	0\\
69.01	0\\
70.01	0\\
71.01	0\\
72.01	0\\
73.01	0\\
74.01	0\\
75.01	0\\
76.01	0\\
77.01	0\\
78.01	0\\
79.01	0\\
80.01	0\\
81.01	0\\
82.01	0\\
83.01	0\\
84.01	0\\
85.01	0\\
86.01	0\\
87.01	0\\
88.01	0\\
89.01	0\\
90.01	0\\
91.01	0\\
92.01	0\\
93.01	0\\
94.01	0\\
95.01	0\\
96.01	0\\
97.01	0\\
98.01	0\\
99.01	0\\
100.01	0\\
101.01	0\\
102.01	0\\
103.01	0\\
104.01	0\\
105.01	0\\
106.01	0\\
107.01	0\\
108.01	0\\
109.01	0\\
110.01	0\\
111.01	0\\
112.01	0\\
113.01	0\\
114.01	0\\
115.01	0\\
116.01	0\\
117.01	0\\
118.01	0\\
119.01	0\\
120.01	0\\
121.01	0\\
122.01	0\\
123.01	0\\
124.01	0\\
125.01	0\\
126.01	0\\
127.01	0\\
128.01	0\\
129.01	0\\
130.01	0\\
131.01	0\\
132.01	0\\
133.01	0\\
134.01	0\\
135.01	0\\
136.01	0\\
137.01	0\\
138.01	0\\
139.01	0\\
140.01	0\\
141.01	0\\
142.01	0\\
143.01	0\\
144.01	0\\
145.01	0\\
146.01	0\\
147.01	0\\
148.01	0\\
149.01	0\\
150.01	0\\
151.01	0\\
152.01	0\\
153.01	0\\
154.01	0\\
155.01	0\\
156.01	0\\
157.01	0\\
158.01	0\\
159.01	0\\
160.01	0\\
161.01	0\\
162.01	0\\
163.01	0\\
164.01	0\\
165.01	0\\
166.01	0\\
167.01	0\\
168.01	0\\
169.01	0\\
170.01	0\\
171.01	0\\
172.01	0\\
173.01	0\\
174.01	0\\
175.01	0\\
176.01	0\\
177.01	0\\
178.01	0\\
179.01	0\\
180.01	0\\
181.01	0\\
182.01	0\\
183.01	0\\
184.01	0\\
185.01	0\\
186.01	0\\
187.01	0\\
188.01	0\\
189.01	0\\
190.01	0\\
191.01	0\\
192.01	0\\
193.01	0\\
194.01	0\\
195.01	0\\
196.01	0\\
197.01	0\\
198.01	0\\
199.01	0\\
200.01	0\\
201.01	0\\
202.01	0\\
203.01	0\\
204.01	0\\
205.01	0\\
206.01	0\\
207.01	0\\
208.01	0\\
209.01	0\\
210.01	0\\
211.01	0\\
212.01	0\\
213.01	0\\
214.01	0\\
215.01	0\\
216.01	0\\
217.01	0\\
218.01	0\\
219.01	0\\
220.01	0\\
221.01	0\\
222.01	0\\
223.01	0\\
224.01	0\\
225.01	0\\
226.01	0\\
227.01	0\\
228.01	0\\
229.01	0\\
230.01	0\\
231.01	0\\
232.01	0\\
233.01	0\\
234.01	0\\
235.01	0\\
236.01	0\\
237.01	0\\
238.01	0\\
239.01	0\\
240.01	0\\
241.01	0\\
242.01	0\\
243.01	0\\
244.01	0\\
245.01	0\\
246.01	0\\
247.01	0\\
248.01	0\\
249.01	0\\
250.01	0\\
251.01	0\\
252.01	0\\
253.01	0\\
254.01	0\\
255.01	0\\
256.01	0\\
257.01	0\\
258.01	0\\
259.01	0\\
260.01	0\\
261.01	0\\
262.01	0\\
263.01	0\\
264.01	0\\
265.01	0\\
266.01	0\\
267.01	0\\
268.01	0\\
269.01	0\\
270.01	0\\
271.01	0\\
272.01	0\\
273.01	0\\
274.01	0\\
275.01	0\\
276.01	0\\
277.01	0\\
278.01	0\\
279.01	0\\
280.01	0\\
281.01	0\\
282.01	0\\
283.01	0\\
284.01	0\\
285.01	0\\
286.01	0\\
287.01	0\\
288.01	0\\
289.01	0\\
290.01	0\\
291.01	0\\
292.01	0\\
293.01	0\\
294.01	0\\
295.01	0\\
296.01	0\\
297.01	0\\
298.01	0\\
299.01	0\\
300.01	0\\
301.01	0\\
302.01	0\\
303.01	0\\
304.01	0\\
305.01	0\\
306.01	0\\
307.01	0\\
308.01	0\\
309.01	0\\
310.01	0\\
311.01	0\\
312.01	0\\
313.01	0\\
314.01	0\\
315.01	0\\
316.01	0\\
317.01	0\\
318.01	0\\
319.01	0\\
320.01	0\\
321.01	0\\
322.01	0\\
323.01	0\\
324.01	0\\
325.01	0\\
326.01	0\\
327.01	0\\
328.01	0\\
329.01	0\\
330.01	0\\
331.01	0\\
332.01	0\\
333.01	0\\
334.01	0\\
335.01	0\\
336.01	0\\
337.01	0\\
338.01	0\\
339.01	0\\
340.01	0\\
341.01	0\\
342.01	0\\
343.01	0\\
344.01	0\\
345.01	0\\
346.01	0\\
347.01	0\\
348.01	0\\
349.01	0\\
350.01	0\\
351.01	0\\
352.01	0\\
353.01	0\\
354.01	0\\
355.01	0\\
356.01	0\\
357.01	0\\
358.01	0\\
359.01	0\\
360.01	0\\
361.01	0\\
362.01	0\\
363.01	0\\
364.01	0\\
365.01	0\\
366.01	0\\
367.01	0\\
368.01	0\\
369.01	0\\
370.01	0\\
371.01	0\\
372.01	0\\
373.01	0\\
374.01	0\\
375.01	0\\
376.01	0\\
377.01	0\\
378.01	0\\
379.01	0\\
380.01	0\\
381.01	0\\
382.01	0\\
383.01	0\\
384.01	0\\
385.01	0\\
386.01	0\\
387.01	0\\
388.01	0\\
389.01	0\\
390.01	0\\
391.01	0\\
392.01	0\\
393.01	0\\
394.01	0\\
395.01	0\\
396.01	0\\
397.01	0\\
398.01	0\\
399.01	0\\
400.01	0\\
401.01	0\\
402.01	0\\
403.01	0\\
404.01	0\\
405.01	0\\
406.01	0\\
407.01	0\\
408.01	0\\
409.01	0\\
410.01	0\\
411.01	0\\
412.01	0\\
413.01	0\\
414.01	0\\
415.01	0\\
416.01	0\\
417.01	0\\
418.01	0\\
419.01	0\\
420.01	0\\
421.01	0\\
422.01	0\\
423.01	0\\
424.01	0\\
425.01	0\\
426.01	0\\
427.01	0\\
428.01	0\\
429.01	0\\
430.01	0\\
431.01	0\\
432.01	0\\
433.01	0\\
434.01	0\\
435.01	0\\
436.01	0\\
437.01	0\\
438.01	0\\
439.01	0\\
440.01	0\\
441.01	0\\
442.01	0\\
443.01	0\\
444.01	0\\
445.01	0\\
446.01	0\\
447.01	0\\
448.01	0\\
449.01	0\\
450.01	0\\
451.01	1.73472347597681e-18\\
452.01	0\\
453.01	0\\
454.01	0\\
455.01	0\\
456.01	0\\
457.01	0\\
458.01	0\\
459.01	0\\
460.01	0\\
461.01	0\\
462.01	0\\
463.01	0\\
464.01	0\\
465.01	0\\
466.01	0\\
467.01	0\\
468.01	0\\
469.01	1.73472347597681e-18\\
470.01	0\\
471.01	0\\
472.01	0\\
473.01	0\\
474.01	0\\
475.01	0\\
476.01	1.73472347597681e-18\\
477.01	0\\
478.01	0\\
479.01	0\\
480.01	0\\
481.01	0\\
482.01	0\\
483.01	0\\
484.01	0\\
485.01	0\\
486.01	0\\
487.01	0\\
488.01	0\\
489.01	0\\
490.01	0\\
491.01	0\\
492.01	0\\
493.01	0\\
494.01	0\\
495.01	0\\
496.01	0\\
497.01	0\\
498.01	0\\
499.01	0\\
500.01	0\\
501.01	0\\
502.01	0\\
503.01	0\\
504.01	0\\
505.01	0\\
506.01	0\\
507.01	0\\
508.01	0\\
509.01	0\\
510.01	0\\
511.01	0\\
512.01	1.73472347597681e-18\\
513.01	0\\
514.01	0\\
515.01	0\\
516.01	0\\
517.01	0\\
518.01	0\\
519.01	0\\
520.01	0\\
521.01	0\\
522.01	0\\
523.01	1.73472347597681e-18\\
524.01	0\\
525.01	0\\
526.01	0\\
527.01	0\\
528.01	0\\
529.01	0\\
530.01	0\\
531.01	0\\
532.01	0\\
533.01	0\\
534.01	0\\
535.01	0\\
536.01	1.73472347597681e-18\\
537.01	1.73472347597681e-18\\
538.01	1.73472347597681e-18\\
539.01	0\\
540.01	0\\
541.01	0\\
542.01	0\\
543.01	0\\
544.01	0\\
545.01	1.73472347597681e-18\\
546.01	1.73472347597681e-18\\
547.01	0\\
548.01	0\\
549.01	0\\
550.01	0\\
551.01	0\\
552.01	0\\
553.01	0\\
554.01	0\\
555.01	0\\
556.01	1.73472347597681e-18\\
557.01	0\\
558.01	0\\
559.01	0\\
560.01	1.73472347597681e-18\\
561.01	1.73472347597681e-18\\
562.01	0\\
563.01	0\\
564.01	0\\
565.01	0\\
566.01	0\\
567.01	0\\
568.01	0\\
569.01	0\\
570.01	0\\
571.01	0\\
572.01	0\\
573.01	0\\
574.01	0\\
575.01	0\\
576.01	0\\
577.01	0\\
578.01	1.73472347597681e-18\\
579.01	0\\
580.01	0\\
581.01	0\\
582.01	0\\
583.01	1.73472347597681e-18\\
584.01	0\\
585.01	0\\
586.01	0\\
587.01	0\\
588.01	0\\
589.01	0\\
590.01	0\\
591.01	0\\
592.01	0\\
593.01	0\\
594.01	0\\
595.01	0\\
596.01	0\\
597.01	0\\
598.01	0\\
599.01	0\\
599.02	0\\
599.03	0\\
599.04	0\\
599.05	0\\
599.06	0\\
599.07	0\\
599.08	0\\
599.09	0\\
599.1	0\\
599.11	0\\
599.12	0\\
599.13	0\\
599.14	0\\
599.15	0\\
599.16	0\\
599.17	0\\
599.18	0\\
599.19	0\\
599.2	0\\
599.21	0\\
599.22	0\\
599.23	0\\
599.24	0\\
599.25	0\\
599.26	0\\
599.27	0\\
599.28	0\\
599.29	0\\
599.3	0\\
599.31	0\\
599.32	0\\
599.33	0\\
599.34	0\\
599.35	0\\
599.36	0\\
599.37	0\\
599.38	0\\
599.39	0\\
599.4	0\\
599.41	0\\
599.42	0\\
599.43	0\\
599.44	0\\
599.45	0\\
599.46	0\\
599.47	0\\
599.48	0\\
599.49	0\\
599.5	0\\
599.51	0\\
599.52	0\\
599.53	0\\
599.54	0\\
599.55	0\\
599.56	0\\
599.57	0\\
599.58	0\\
599.59	0\\
599.6	0\\
599.61	0\\
599.62	0\\
599.63	0\\
599.64	0\\
599.65	0\\
599.66	0\\
599.67	0\\
599.68	0\\
599.69	0\\
599.7	0\\
599.71	0\\
599.72	0\\
599.73	0\\
599.74	0\\
599.75	0\\
599.76	0\\
599.77	0\\
599.78	0\\
599.79	0\\
599.8	0\\
599.81	0\\
599.82	0\\
599.83	0\\
599.84	0\\
599.85	0\\
599.86	0\\
599.87	0\\
599.88	0\\
599.89	0\\
599.9	0\\
599.91	0\\
599.92	0\\
599.93	0\\
599.94	0\\
599.95	0\\
599.96	0\\
599.97	0\\
599.98	0\\
599.99	0\\
600	0\\
};
\addplot [color=mycolor10,solid,forget plot]
  table[row sep=crcr]{%
0.01	3.6332990175629e-05\\
1.01	3.6332990175629e-05\\
2.01	3.6332990175629e-05\\
3.01	3.6332990175629e-05\\
4.01	3.6332990175629e-05\\
5.01	3.6332990175629e-05\\
6.01	3.6332990175629e-05\\
7.01	3.6332990175629e-05\\
8.01	3.6332990175629e-05\\
9.01	3.6332990175629e-05\\
10.01	3.6332990175629e-05\\
11.01	3.6332990175629e-05\\
12.01	3.6332990175629e-05\\
13.01	3.6332990175629e-05\\
14.01	3.6332990175629e-05\\
15.01	3.6332990175629e-05\\
16.01	3.6332990175629e-05\\
17.01	3.6332990175629e-05\\
18.01	3.6332990175629e-05\\
19.01	3.6332990175629e-05\\
20.01	3.6332990175629e-05\\
21.01	3.6332990175629e-05\\
22.01	3.6332990175629e-05\\
23.01	3.6332990175629e-05\\
24.01	3.6332990175629e-05\\
25.01	3.6332990175629e-05\\
26.01	3.6332990175629e-05\\
27.01	3.6332990175629e-05\\
28.01	3.6332990175629e-05\\
29.01	3.6332990175629e-05\\
30.01	3.6332990175629e-05\\
31.01	3.6332990175629e-05\\
32.01	3.6332990175629e-05\\
33.01	3.6332990175629e-05\\
34.01	3.6332990175629e-05\\
35.01	3.6332990175629e-05\\
36.01	3.6332990175629e-05\\
37.01	3.6332990175629e-05\\
38.01	3.6332990175629e-05\\
39.01	3.6332990175629e-05\\
40.01	3.6332990175629e-05\\
41.01	3.6332990175629e-05\\
42.01	3.6332990175629e-05\\
43.01	3.6332990175629e-05\\
44.01	3.6332990175629e-05\\
45.01	3.6332990175629e-05\\
46.01	3.6332990175629e-05\\
47.01	3.6332990175629e-05\\
48.01	3.6332990175629e-05\\
49.01	3.6332990175629e-05\\
50.01	3.6332990175629e-05\\
51.01	3.6332990175629e-05\\
52.01	3.6332990175629e-05\\
53.01	3.6332990175629e-05\\
54.01	3.6332990175629e-05\\
55.01	3.6332990175629e-05\\
56.01	3.6332990175629e-05\\
57.01	3.6332990175629e-05\\
58.01	3.6332990175629e-05\\
59.01	3.6332990175629e-05\\
60.01	3.6332990175629e-05\\
61.01	3.6332990175629e-05\\
62.01	3.6332990175629e-05\\
63.01	3.6332990175629e-05\\
64.01	3.6332990175629e-05\\
65.01	3.6332990175629e-05\\
66.01	3.6332990175629e-05\\
67.01	3.6332990175629e-05\\
68.01	3.6332990175629e-05\\
69.01	3.6332990175629e-05\\
70.01	3.6332990175629e-05\\
71.01	3.6332990175629e-05\\
72.01	3.6332990175629e-05\\
73.01	3.6332990175629e-05\\
74.01	3.6332990175629e-05\\
75.01	3.6332990175629e-05\\
76.01	3.6332990175629e-05\\
77.01	3.6332990175629e-05\\
78.01	3.6332990175629e-05\\
79.01	3.6332990175629e-05\\
80.01	3.6332990175629e-05\\
81.01	3.6332990175629e-05\\
82.01	3.6332990175629e-05\\
83.01	3.6332990175629e-05\\
84.01	3.6332990175629e-05\\
85.01	3.6332990175629e-05\\
86.01	3.6332990175629e-05\\
87.01	3.6332990175629e-05\\
88.01	3.6332990175629e-05\\
89.01	3.6332990175629e-05\\
90.01	3.6332990175629e-05\\
91.01	3.6332990175629e-05\\
92.01	3.6332990175629e-05\\
93.01	3.6332990175629e-05\\
94.01	3.6332990175629e-05\\
95.01	3.6332990175629e-05\\
96.01	3.6332990175629e-05\\
97.01	3.6332990175629e-05\\
98.01	3.6332990175629e-05\\
99.01	3.6332990175629e-05\\
100.01	3.6332990175629e-05\\
101.01	3.6332990175629e-05\\
102.01	3.6332990175629e-05\\
103.01	3.6332990175629e-05\\
104.01	3.6332990175629e-05\\
105.01	3.6332990175629e-05\\
106.01	3.6332990175629e-05\\
107.01	3.6332990175629e-05\\
108.01	3.6332990175629e-05\\
109.01	3.6332990175629e-05\\
110.01	3.6332990175629e-05\\
111.01	3.6332990175629e-05\\
112.01	3.6332990175629e-05\\
113.01	3.6332990175629e-05\\
114.01	3.6332990175629e-05\\
115.01	3.6332990175629e-05\\
116.01	3.6332990175629e-05\\
117.01	3.6332990175629e-05\\
118.01	3.6332990175629e-05\\
119.01	3.6332990175629e-05\\
120.01	3.6332990175629e-05\\
121.01	3.6332990175629e-05\\
122.01	3.6332990175629e-05\\
123.01	3.6332990175629e-05\\
124.01	3.6332990175629e-05\\
125.01	3.6332990175629e-05\\
126.01	3.6332990175629e-05\\
127.01	3.6332990175629e-05\\
128.01	3.6332990175629e-05\\
129.01	3.6332990175629e-05\\
130.01	3.6332990175629e-05\\
131.01	3.6332990175629e-05\\
132.01	3.6332990175629e-05\\
133.01	3.6332990175629e-05\\
134.01	3.6332990175629e-05\\
135.01	3.6332990175629e-05\\
136.01	3.6332990175629e-05\\
137.01	3.6332990175629e-05\\
138.01	3.6332990175629e-05\\
139.01	3.6332990175629e-05\\
140.01	3.6332990175629e-05\\
141.01	3.6332990175629e-05\\
142.01	3.6332990175629e-05\\
143.01	3.6332990175629e-05\\
144.01	3.6332990175629e-05\\
145.01	3.6332990175629e-05\\
146.01	3.6332990175629e-05\\
147.01	3.6332990175629e-05\\
148.01	3.6332990175629e-05\\
149.01	3.6332990175629e-05\\
150.01	3.6332990175629e-05\\
151.01	3.6332990175629e-05\\
152.01	3.6332990175629e-05\\
153.01	3.6332990175629e-05\\
154.01	3.6332990175629e-05\\
155.01	3.6332990175629e-05\\
156.01	3.6332990175629e-05\\
157.01	3.6332990175629e-05\\
158.01	3.6332990175629e-05\\
159.01	3.6332990175629e-05\\
160.01	3.6332990175629e-05\\
161.01	3.6332990175629e-05\\
162.01	3.6332990175629e-05\\
163.01	3.6332990175629e-05\\
164.01	3.6332990175629e-05\\
165.01	3.6332990175629e-05\\
166.01	3.6332990175629e-05\\
167.01	3.6332990175629e-05\\
168.01	3.6332990175629e-05\\
169.01	3.6332990175629e-05\\
170.01	3.6332990175629e-05\\
171.01	3.6332990175629e-05\\
172.01	3.6332990175629e-05\\
173.01	3.6332990175629e-05\\
174.01	3.6332990175629e-05\\
175.01	3.6332990175629e-05\\
176.01	3.6332990175629e-05\\
177.01	3.6332990175629e-05\\
178.01	3.6332990175629e-05\\
179.01	3.6332990175629e-05\\
180.01	3.6332990175629e-05\\
181.01	3.6332990175629e-05\\
182.01	3.6332990175629e-05\\
183.01	3.6332990175629e-05\\
184.01	3.6332990175629e-05\\
185.01	3.6332990175629e-05\\
186.01	3.6332990175629e-05\\
187.01	3.6332990175629e-05\\
188.01	3.6332990175629e-05\\
189.01	3.6332990175629e-05\\
190.01	3.6332990175629e-05\\
191.01	3.6332990175629e-05\\
192.01	3.6332990175629e-05\\
193.01	3.6332990175629e-05\\
194.01	3.6332990175629e-05\\
195.01	3.6332990175629e-05\\
196.01	3.6332990175629e-05\\
197.01	3.6332990175629e-05\\
198.01	3.6332990175629e-05\\
199.01	3.6332990175629e-05\\
200.01	3.6332990175629e-05\\
201.01	3.6332990175629e-05\\
202.01	3.6332990175629e-05\\
203.01	3.6332990175629e-05\\
204.01	3.6332990175629e-05\\
205.01	3.6332990175629e-05\\
206.01	3.6332990175629e-05\\
207.01	3.6332990175629e-05\\
208.01	3.6332990175629e-05\\
209.01	3.6332990175629e-05\\
210.01	3.6332990175629e-05\\
211.01	3.6332990175629e-05\\
212.01	3.6332990175629e-05\\
213.01	3.6332990175629e-05\\
214.01	3.6332990175629e-05\\
215.01	3.6332990175629e-05\\
216.01	3.6332990175629e-05\\
217.01	3.6332990175629e-05\\
218.01	3.6332990175629e-05\\
219.01	3.6332990175629e-05\\
220.01	3.6332990175629e-05\\
221.01	3.6332990175629e-05\\
222.01	3.6332990175629e-05\\
223.01	3.6332990175629e-05\\
224.01	3.6332990175629e-05\\
225.01	3.6332990175629e-05\\
226.01	3.6332990175629e-05\\
227.01	3.6332990175629e-05\\
228.01	3.6332990175629e-05\\
229.01	3.6332990175629e-05\\
230.01	3.6332990175629e-05\\
231.01	3.6332990175629e-05\\
232.01	3.6332990175629e-05\\
233.01	3.6332990175629e-05\\
234.01	3.6332990175629e-05\\
235.01	3.6332990175629e-05\\
236.01	3.6332990175629e-05\\
237.01	3.6332990175629e-05\\
238.01	3.6332990175629e-05\\
239.01	3.6332990175629e-05\\
240.01	3.6332990175629e-05\\
241.01	3.6332990175629e-05\\
242.01	3.6332990175629e-05\\
243.01	3.6332990175629e-05\\
244.01	3.6332990175629e-05\\
245.01	3.6332990175629e-05\\
246.01	3.6332990175629e-05\\
247.01	3.6332990175629e-05\\
248.01	3.6332990175629e-05\\
249.01	3.6332990175629e-05\\
250.01	3.6332990175629e-05\\
251.01	3.6332990175629e-05\\
252.01	3.6332990175629e-05\\
253.01	3.6332990175629e-05\\
254.01	3.6332990175629e-05\\
255.01	3.6332990175629e-05\\
256.01	3.6332990175629e-05\\
257.01	3.6332990175629e-05\\
258.01	3.6332990175629e-05\\
259.01	3.6332990175629e-05\\
260.01	3.6332990175629e-05\\
261.01	3.6332990175629e-05\\
262.01	3.6332990175629e-05\\
263.01	3.6332990175629e-05\\
264.01	3.6332990175629e-05\\
265.01	3.6332990175629e-05\\
266.01	3.6332990175629e-05\\
267.01	3.6332990175629e-05\\
268.01	3.6332990175629e-05\\
269.01	3.6332990175629e-05\\
270.01	3.6332990175629e-05\\
271.01	3.6332990175629e-05\\
272.01	3.6332990175629e-05\\
273.01	3.6332990175629e-05\\
274.01	3.6332990175629e-05\\
275.01	3.6332990175629e-05\\
276.01	3.6332990175629e-05\\
277.01	3.6332990175629e-05\\
278.01	3.6332990175629e-05\\
279.01	3.6332990175629e-05\\
280.01	3.6332990175629e-05\\
281.01	3.6332990175629e-05\\
282.01	3.6332990175629e-05\\
283.01	3.6332990175629e-05\\
284.01	3.6332990175629e-05\\
285.01	3.6332990175629e-05\\
286.01	3.6332990175629e-05\\
287.01	3.6332990175629e-05\\
288.01	3.6332990175629e-05\\
289.01	3.6332990175629e-05\\
290.01	3.6332990175629e-05\\
291.01	3.6332990175629e-05\\
292.01	3.6332990175629e-05\\
293.01	3.6332990175629e-05\\
294.01	3.6332990175629e-05\\
295.01	3.6332990175629e-05\\
296.01	3.6332990175629e-05\\
297.01	3.6332990175629e-05\\
298.01	3.6332990175629e-05\\
299.01	3.6332990175629e-05\\
300.01	3.6332990175629e-05\\
301.01	3.6332990175629e-05\\
302.01	3.6332990175629e-05\\
303.01	3.6332990175629e-05\\
304.01	3.6332990175629e-05\\
305.01	3.6332990175629e-05\\
306.01	3.6332990175629e-05\\
307.01	3.6332990175629e-05\\
308.01	3.6332990175629e-05\\
309.01	3.6332990175629e-05\\
310.01	3.6332990175629e-05\\
311.01	3.6332990175629e-05\\
312.01	3.6332990175629e-05\\
313.01	3.6332990175629e-05\\
314.01	3.6332990175629e-05\\
315.01	3.6332990175629e-05\\
316.01	3.6332990175629e-05\\
317.01	3.6332990175629e-05\\
318.01	3.6332990175629e-05\\
319.01	3.6332990175629e-05\\
320.01	3.6332990175629e-05\\
321.01	3.6332990175629e-05\\
322.01	3.6332990175629e-05\\
323.01	3.6332990175629e-05\\
324.01	3.6332990175629e-05\\
325.01	3.6332990175629e-05\\
326.01	3.6332990175629e-05\\
327.01	3.6332990175629e-05\\
328.01	3.6332990175629e-05\\
329.01	3.6332990175629e-05\\
330.01	3.6332990175629e-05\\
331.01	3.6332990175629e-05\\
332.01	3.6332990175629e-05\\
333.01	3.6332990175629e-05\\
334.01	3.6332990175629e-05\\
335.01	3.6332990175629e-05\\
336.01	3.6332990175629e-05\\
337.01	3.6332990175629e-05\\
338.01	3.6332990175629e-05\\
339.01	3.6332990175629e-05\\
340.01	3.6332990175629e-05\\
341.01	3.6332990175629e-05\\
342.01	3.6332990175629e-05\\
343.01	3.6332990175629e-05\\
344.01	3.6332990175629e-05\\
345.01	3.6332990175629e-05\\
346.01	3.6332990175629e-05\\
347.01	3.6332990175629e-05\\
348.01	3.6332990175629e-05\\
349.01	3.6332990175629e-05\\
350.01	3.6332990175629e-05\\
351.01	3.6332990175629e-05\\
352.01	3.6332990175629e-05\\
353.01	3.6332990175629e-05\\
354.01	3.6332990175629e-05\\
355.01	3.6332990175629e-05\\
356.01	3.6332990175629e-05\\
357.01	3.6332990175629e-05\\
358.01	3.6332990175629e-05\\
359.01	3.6332990175629e-05\\
360.01	3.6332990175629e-05\\
361.01	3.6332990175629e-05\\
362.01	3.6332990175629e-05\\
363.01	3.6332990175629e-05\\
364.01	3.6332990175629e-05\\
365.01	3.6332990175629e-05\\
366.01	3.6332990175629e-05\\
367.01	3.6332990175629e-05\\
368.01	3.6332990175629e-05\\
369.01	3.6332990175629e-05\\
370.01	3.6332990175629e-05\\
371.01	3.6332990175629e-05\\
372.01	3.6332990175629e-05\\
373.01	3.6332990175629e-05\\
374.01	3.6332990175629e-05\\
375.01	3.6332990175629e-05\\
376.01	3.6332990175629e-05\\
377.01	3.6332990175629e-05\\
378.01	3.6332990175629e-05\\
379.01	3.6332990175629e-05\\
380.01	3.6332990175629e-05\\
381.01	3.6332990175629e-05\\
382.01	3.6332990175629e-05\\
383.01	3.6332990175629e-05\\
384.01	3.6332990175629e-05\\
385.01	3.6332990175629e-05\\
386.01	3.6332990175629e-05\\
387.01	3.6332990175629e-05\\
388.01	3.6332990175629e-05\\
389.01	3.6332990175629e-05\\
390.01	3.6332990175629e-05\\
391.01	3.6332990175629e-05\\
392.01	3.6332990175629e-05\\
393.01	3.6332990175629e-05\\
394.01	3.6332990175629e-05\\
395.01	3.6332990175629e-05\\
396.01	3.6332990175629e-05\\
397.01	3.6332990175629e-05\\
398.01	3.6332990175629e-05\\
399.01	3.6332990175629e-05\\
400.01	3.6332990175629e-05\\
401.01	3.6332990175629e-05\\
402.01	3.6332990175629e-05\\
403.01	3.6332990175629e-05\\
404.01	3.6332990175629e-05\\
405.01	3.6332990175629e-05\\
406.01	3.6332990175629e-05\\
407.01	3.6332990175629e-05\\
408.01	3.6332990175629e-05\\
409.01	3.6332990175629e-05\\
410.01	3.6332990175629e-05\\
411.01	3.6332990175629e-05\\
412.01	3.6332990175629e-05\\
413.01	3.6332990175629e-05\\
414.01	3.6332990175629e-05\\
415.01	3.6332990175629e-05\\
416.01	3.6332990175629e-05\\
417.01	3.6332990175629e-05\\
418.01	3.6332990175629e-05\\
419.01	3.6332990175629e-05\\
420.01	3.6332990175629e-05\\
421.01	3.6332990175629e-05\\
422.01	3.6332990175629e-05\\
423.01	3.6332990175629e-05\\
424.01	3.6332990175629e-05\\
425.01	3.6332990175629e-05\\
426.01	3.6332990175629e-05\\
427.01	3.6332990175629e-05\\
428.01	3.6332990175629e-05\\
429.01	3.6332990175629e-05\\
430.01	3.6332990175629e-05\\
431.01	3.6332990175629e-05\\
432.01	3.6332990175629e-05\\
433.01	3.6332990175629e-05\\
434.01	3.6332990175629e-05\\
435.01	3.6332990175629e-05\\
436.01	3.6332990175629e-05\\
437.01	3.6332990175629e-05\\
438.01	3.6332990175629e-05\\
439.01	3.6332990175629e-05\\
440.01	3.6332990175629e-05\\
441.01	3.6332990175629e-05\\
442.01	3.6332990175629e-05\\
443.01	3.6332990175629e-05\\
444.01	3.6332990175629e-05\\
445.01	3.63329901756273e-05\\
446.01	3.63329901756099e-05\\
447.01	3.633299017557e-05\\
448.01	3.63329901754642e-05\\
449.01	3.63329901751815e-05\\
450.01	3.63329901744112e-05\\
451.01	3.63329901723244e-05\\
452.01	3.63329901666726e-05\\
453.01	3.63329901513759e-05\\
454.01	3.63329901100322e-05\\
455.01	3.63329899984756e-05\\
456.01	3.63329896982331e-05\\
457.01	3.63329888930751e-05\\
458.01	3.63329867448508e-05\\
459.01	3.63329810541529e-05\\
460.01	3.63329661298185e-05\\
461.01	3.63329275367384e-05\\
462.01	3.63328297068363e-05\\
463.01	3.63325887119854e-05\\
464.01	3.63320195061381e-05\\
465.01	3.63307585765377e-05\\
466.01	3.63282378046548e-05\\
467.01	3.63240050212116e-05\\
468.01	3.63186526850575e-05\\
469.01	3.63131005853765e-05\\
470.01	3.63074235777052e-05\\
471.01	3.6301634700522e-05\\
472.01	3.62957697946677e-05\\
473.01	3.62899174158309e-05\\
474.01	3.62842786465652e-05\\
475.01	3.62792618784288e-05\\
476.01	3.6275545979637e-05\\
477.01	3.62738038318083e-05\\
478.01	3.62736023077313e-05\\
479.01	3.62736023077313e-05\\
480.01	3.62736023077313e-05\\
481.01	3.62736023077313e-05\\
482.01	3.62736023077313e-05\\
483.01	3.62736023077313e-05\\
484.01	3.62736023077313e-05\\
485.01	3.62736023077313e-05\\
486.01	3.62736023077313e-05\\
487.01	3.62736023077313e-05\\
488.01	3.62736023077313e-05\\
489.01	3.62736023077313e-05\\
490.01	3.62736023077313e-05\\
491.01	3.62736023077313e-05\\
492.01	3.62736023077313e-05\\
493.01	3.62736023077313e-05\\
494.01	3.62736023077313e-05\\
495.01	3.62736023077313e-05\\
496.01	3.62736023077313e-05\\
497.01	3.62736023077313e-05\\
498.01	3.62736023077313e-05\\
499.01	3.62736023077313e-05\\
500.01	3.62736023077313e-05\\
501.01	3.62736023077313e-05\\
502.01	3.62736023077313e-05\\
503.01	3.62736023077313e-05\\
504.01	3.62736023077313e-05\\
505.01	3.62736023077313e-05\\
506.01	3.62736023077313e-05\\
507.01	3.62736023077331e-05\\
508.01	3.62736023077331e-05\\
509.01	3.6273602307714e-05\\
510.01	3.6273602307688e-05\\
511.01	3.62736023076082e-05\\
512.01	3.62736023073896e-05\\
513.01	3.62736023067703e-05\\
514.01	3.62736023050304e-05\\
515.01	3.62736023001211e-05\\
516.01	3.62736022862329e-05\\
517.01	3.62736022467593e-05\\
518.01	3.62736021340768e-05\\
519.01	3.62736018107348e-05\\
520.01	3.62736008775299e-05\\
521.01	3.62735981668961e-05\\
522.01	3.62735902382626e-05\\
523.01	3.62735668706579e-05\\
524.01	3.62734974387602e-05\\
525.01	3.6273289345785e-05\\
526.01	3.62726599625867e-05\\
527.01	3.62707381364666e-05\\
528.01	3.62648115461908e-05\\
529.01	3.62463488691057e-05\\
530.01	3.61882420091067e-05\\
531.01	3.6003504478764e-05\\
532.01	3.54104452939367e-05\\
533.01	3.34895990750511e-05\\
534.01	2.86780701776249e-05\\
535.01	2.33349897528021e-05\\
536.01	1.7891461666756e-05\\
537.01	1.23300729334211e-05\\
538.01	6.77559521603671e-06\\
539.01	1.90599427558891e-06\\
540.01	0\\
541.01	0\\
542.01	0\\
543.01	0\\
544.01	0\\
545.01	1.73472347597681e-18\\
546.01	1.73472347597681e-18\\
547.01	0\\
548.01	0\\
549.01	0\\
550.01	0\\
551.01	0\\
552.01	0\\
553.01	0\\
554.01	0\\
555.01	0\\
556.01	1.73472347597681e-18\\
557.01	0\\
558.01	0\\
559.01	0\\
560.01	1.73472347597681e-18\\
561.01	1.73472347597681e-18\\
562.01	0\\
563.01	0\\
564.01	0\\
565.01	0\\
566.01	0\\
567.01	0\\
568.01	0\\
569.01	0\\
570.01	0\\
571.01	0\\
572.01	0\\
573.01	0\\
574.01	0\\
575.01	0\\
576.01	0\\
577.01	0\\
578.01	1.73472347597681e-18\\
579.01	0\\
580.01	0\\
581.01	0\\
582.01	0\\
583.01	1.73472347597681e-18\\
584.01	0\\
585.01	0\\
586.01	0\\
587.01	0\\
588.01	0\\
589.01	0\\
590.01	0\\
591.01	0\\
592.01	0\\
593.01	0\\
594.01	0\\
595.01	0\\
596.01	0\\
597.01	0\\
598.01	0\\
599.01	0\\
599.02	0\\
599.03	0\\
599.04	0\\
599.05	0\\
599.06	0\\
599.07	0\\
599.08	0\\
599.09	0\\
599.1	0\\
599.11	0\\
599.12	0\\
599.13	0\\
599.14	0\\
599.15	0\\
599.16	0\\
599.17	0\\
599.18	0\\
599.19	0\\
599.2	0\\
599.21	0\\
599.22	0\\
599.23	0\\
599.24	0\\
599.25	0\\
599.26	0\\
599.27	0\\
599.28	0\\
599.29	0\\
599.3	0\\
599.31	0\\
599.32	0\\
599.33	0\\
599.34	0\\
599.35	0\\
599.36	0\\
599.37	0\\
599.38	0\\
599.39	0\\
599.4	0\\
599.41	0\\
599.42	0\\
599.43	0\\
599.44	0\\
599.45	0\\
599.46	0\\
599.47	0\\
599.48	0\\
599.49	0\\
599.5	0\\
599.51	0\\
599.52	0\\
599.53	0\\
599.54	0\\
599.55	0\\
599.56	0\\
599.57	0\\
599.58	0\\
599.59	0\\
599.6	0\\
599.61	0\\
599.62	0\\
599.63	0\\
599.64	0\\
599.65	0\\
599.66	0\\
599.67	0\\
599.68	0\\
599.69	0\\
599.7	0\\
599.71	0\\
599.72	0\\
599.73	0\\
599.74	0\\
599.75	0\\
599.76	0\\
599.77	0\\
599.78	0\\
599.79	0\\
599.8	0\\
599.81	0\\
599.82	0\\
599.83	0\\
599.84	0\\
599.85	0\\
599.86	0\\
599.87	0\\
599.88	0\\
599.89	0\\
599.9	0\\
599.91	0\\
599.92	0\\
599.93	0\\
599.94	0\\
599.95	0\\
599.96	0\\
599.97	0\\
599.98	0\\
599.99	0\\
600	0\\
};
\addplot [color=mycolor11,solid,forget plot]
  table[row sep=crcr]{%
0.01	0.00032348326751387\\
1.01	0.00032348326751387\\
2.01	0.00032348326751387\\
3.01	0.00032348326751387\\
4.01	0.00032348326751387\\
5.01	0.00032348326751387\\
6.01	0.00032348326751387\\
7.01	0.00032348326751387\\
8.01	0.00032348326751387\\
9.01	0.00032348326751387\\
10.01	0.00032348326751387\\
11.01	0.00032348326751387\\
12.01	0.00032348326751387\\
13.01	0.00032348326751387\\
14.01	0.00032348326751387\\
15.01	0.00032348326751387\\
16.01	0.00032348326751387\\
17.01	0.00032348326751387\\
18.01	0.00032348326751387\\
19.01	0.00032348326751387\\
20.01	0.00032348326751387\\
21.01	0.00032348326751387\\
22.01	0.00032348326751387\\
23.01	0.00032348326751387\\
24.01	0.00032348326751387\\
25.01	0.00032348326751387\\
26.01	0.00032348326751387\\
27.01	0.00032348326751387\\
28.01	0.00032348326751387\\
29.01	0.00032348326751387\\
30.01	0.00032348326751387\\
31.01	0.00032348326751387\\
32.01	0.00032348326751387\\
33.01	0.00032348326751387\\
34.01	0.00032348326751387\\
35.01	0.00032348326751387\\
36.01	0.00032348326751387\\
37.01	0.00032348326751387\\
38.01	0.00032348326751387\\
39.01	0.00032348326751387\\
40.01	0.00032348326751387\\
41.01	0.00032348326751387\\
42.01	0.00032348326751387\\
43.01	0.00032348326751387\\
44.01	0.00032348326751387\\
45.01	0.00032348326751387\\
46.01	0.00032348326751387\\
47.01	0.00032348326751387\\
48.01	0.00032348326751387\\
49.01	0.00032348326751387\\
50.01	0.00032348326751387\\
51.01	0.00032348326751387\\
52.01	0.00032348326751387\\
53.01	0.00032348326751387\\
54.01	0.00032348326751387\\
55.01	0.00032348326751387\\
56.01	0.00032348326751387\\
57.01	0.00032348326751387\\
58.01	0.00032348326751387\\
59.01	0.00032348326751387\\
60.01	0.00032348326751387\\
61.01	0.00032348326751387\\
62.01	0.00032348326751387\\
63.01	0.00032348326751387\\
64.01	0.00032348326751387\\
65.01	0.00032348326751387\\
66.01	0.00032348326751387\\
67.01	0.00032348326751387\\
68.01	0.00032348326751387\\
69.01	0.00032348326751387\\
70.01	0.00032348326751387\\
71.01	0.00032348326751387\\
72.01	0.00032348326751387\\
73.01	0.00032348326751387\\
74.01	0.00032348326751387\\
75.01	0.00032348326751387\\
76.01	0.00032348326751387\\
77.01	0.00032348326751387\\
78.01	0.00032348326751387\\
79.01	0.00032348326751387\\
80.01	0.00032348326751387\\
81.01	0.00032348326751387\\
82.01	0.00032348326751387\\
83.01	0.00032348326751387\\
84.01	0.00032348326751387\\
85.01	0.00032348326751387\\
86.01	0.00032348326751387\\
87.01	0.00032348326751387\\
88.01	0.00032348326751387\\
89.01	0.00032348326751387\\
90.01	0.00032348326751387\\
91.01	0.00032348326751387\\
92.01	0.00032348326751387\\
93.01	0.00032348326751387\\
94.01	0.00032348326751387\\
95.01	0.00032348326751387\\
96.01	0.00032348326751387\\
97.01	0.00032348326751387\\
98.01	0.00032348326751387\\
99.01	0.00032348326751387\\
100.01	0.00032348326751387\\
101.01	0.00032348326751387\\
102.01	0.00032348326751387\\
103.01	0.00032348326751387\\
104.01	0.00032348326751387\\
105.01	0.00032348326751387\\
106.01	0.00032348326751387\\
107.01	0.00032348326751387\\
108.01	0.00032348326751387\\
109.01	0.00032348326751387\\
110.01	0.00032348326751387\\
111.01	0.00032348326751387\\
112.01	0.00032348326751387\\
113.01	0.00032348326751387\\
114.01	0.00032348326751387\\
115.01	0.00032348326751387\\
116.01	0.00032348326751387\\
117.01	0.00032348326751387\\
118.01	0.00032348326751387\\
119.01	0.00032348326751387\\
120.01	0.00032348326751387\\
121.01	0.00032348326751387\\
122.01	0.00032348326751387\\
123.01	0.00032348326751387\\
124.01	0.00032348326751387\\
125.01	0.00032348326751387\\
126.01	0.00032348326751387\\
127.01	0.00032348326751387\\
128.01	0.00032348326751387\\
129.01	0.00032348326751387\\
130.01	0.00032348326751387\\
131.01	0.00032348326751387\\
132.01	0.00032348326751387\\
133.01	0.00032348326751387\\
134.01	0.00032348326751387\\
135.01	0.00032348326751387\\
136.01	0.00032348326751387\\
137.01	0.00032348326751387\\
138.01	0.00032348326751387\\
139.01	0.00032348326751387\\
140.01	0.00032348326751387\\
141.01	0.00032348326751387\\
142.01	0.00032348326751387\\
143.01	0.00032348326751387\\
144.01	0.00032348326751387\\
145.01	0.00032348326751387\\
146.01	0.00032348326751387\\
147.01	0.00032348326751387\\
148.01	0.00032348326751387\\
149.01	0.00032348326751387\\
150.01	0.00032348326751387\\
151.01	0.00032348326751387\\
152.01	0.00032348326751387\\
153.01	0.00032348326751387\\
154.01	0.00032348326751387\\
155.01	0.00032348326751387\\
156.01	0.00032348326751387\\
157.01	0.00032348326751387\\
158.01	0.00032348326751387\\
159.01	0.00032348326751387\\
160.01	0.00032348326751387\\
161.01	0.00032348326751387\\
162.01	0.00032348326751387\\
163.01	0.00032348326751387\\
164.01	0.00032348326751387\\
165.01	0.00032348326751387\\
166.01	0.00032348326751387\\
167.01	0.00032348326751387\\
168.01	0.00032348326751387\\
169.01	0.00032348326751387\\
170.01	0.00032348326751387\\
171.01	0.00032348326751387\\
172.01	0.00032348326751387\\
173.01	0.00032348326751387\\
174.01	0.00032348326751387\\
175.01	0.00032348326751387\\
176.01	0.00032348326751387\\
177.01	0.00032348326751387\\
178.01	0.00032348326751387\\
179.01	0.00032348326751387\\
180.01	0.00032348326751387\\
181.01	0.00032348326751387\\
182.01	0.00032348326751387\\
183.01	0.00032348326751387\\
184.01	0.00032348326751387\\
185.01	0.00032348326751387\\
186.01	0.00032348326751387\\
187.01	0.00032348326751387\\
188.01	0.00032348326751387\\
189.01	0.00032348326751387\\
190.01	0.00032348326751387\\
191.01	0.00032348326751387\\
192.01	0.00032348326751387\\
193.01	0.00032348326751387\\
194.01	0.00032348326751387\\
195.01	0.00032348326751387\\
196.01	0.00032348326751387\\
197.01	0.00032348326751387\\
198.01	0.00032348326751387\\
199.01	0.00032348326751387\\
200.01	0.00032348326751387\\
201.01	0.00032348326751387\\
202.01	0.00032348326751387\\
203.01	0.00032348326751387\\
204.01	0.00032348326751387\\
205.01	0.00032348326751387\\
206.01	0.00032348326751387\\
207.01	0.00032348326751387\\
208.01	0.00032348326751387\\
209.01	0.00032348326751387\\
210.01	0.00032348326751387\\
211.01	0.00032348326751387\\
212.01	0.00032348326751387\\
213.01	0.00032348326751387\\
214.01	0.00032348326751387\\
215.01	0.00032348326751387\\
216.01	0.00032348326751387\\
217.01	0.00032348326751387\\
218.01	0.00032348326751387\\
219.01	0.00032348326751387\\
220.01	0.00032348326751387\\
221.01	0.00032348326751387\\
222.01	0.00032348326751387\\
223.01	0.00032348326751387\\
224.01	0.00032348326751387\\
225.01	0.00032348326751387\\
226.01	0.00032348326751387\\
227.01	0.00032348326751387\\
228.01	0.00032348326751387\\
229.01	0.00032348326751387\\
230.01	0.00032348326751387\\
231.01	0.00032348326751387\\
232.01	0.00032348326751387\\
233.01	0.00032348326751387\\
234.01	0.00032348326751387\\
235.01	0.00032348326751387\\
236.01	0.00032348326751387\\
237.01	0.00032348326751387\\
238.01	0.00032348326751387\\
239.01	0.00032348326751387\\
240.01	0.00032348326751387\\
241.01	0.00032348326751387\\
242.01	0.00032348326751387\\
243.01	0.00032348326751387\\
244.01	0.00032348326751387\\
245.01	0.00032348326751387\\
246.01	0.00032348326751387\\
247.01	0.00032348326751387\\
248.01	0.00032348326751387\\
249.01	0.00032348326751387\\
250.01	0.00032348326751387\\
251.01	0.00032348326751387\\
252.01	0.00032348326751387\\
253.01	0.00032348326751387\\
254.01	0.00032348326751387\\
255.01	0.00032348326751387\\
256.01	0.00032348326751387\\
257.01	0.00032348326751387\\
258.01	0.00032348326751387\\
259.01	0.00032348326751387\\
260.01	0.00032348326751387\\
261.01	0.00032348326751387\\
262.01	0.00032348326751387\\
263.01	0.00032348326751387\\
264.01	0.00032348326751387\\
265.01	0.00032348326751387\\
266.01	0.00032348326751387\\
267.01	0.00032348326751387\\
268.01	0.00032348326751387\\
269.01	0.00032348326751387\\
270.01	0.00032348326751387\\
271.01	0.00032348326751387\\
272.01	0.00032348326751387\\
273.01	0.00032348326751387\\
274.01	0.00032348326751387\\
275.01	0.00032348326751387\\
276.01	0.00032348326751387\\
277.01	0.00032348326751387\\
278.01	0.00032348326751387\\
279.01	0.00032348326751387\\
280.01	0.00032348326751387\\
281.01	0.00032348326751387\\
282.01	0.00032348326751387\\
283.01	0.00032348326751387\\
284.01	0.00032348326751387\\
285.01	0.00032348326751387\\
286.01	0.00032348326751387\\
287.01	0.00032348326751387\\
288.01	0.00032348326751387\\
289.01	0.00032348326751387\\
290.01	0.00032348326751387\\
291.01	0.00032348326751387\\
292.01	0.00032348326751387\\
293.01	0.00032348326751387\\
294.01	0.00032348326751387\\
295.01	0.00032348326751387\\
296.01	0.00032348326751387\\
297.01	0.00032348326751387\\
298.01	0.00032348326751387\\
299.01	0.00032348326751387\\
300.01	0.00032348326751387\\
301.01	0.00032348326751387\\
302.01	0.00032348326751387\\
303.01	0.00032348326751387\\
304.01	0.00032348326751387\\
305.01	0.00032348326751387\\
306.01	0.00032348326751387\\
307.01	0.00032348326751387\\
308.01	0.00032348326751387\\
309.01	0.00032348326751387\\
310.01	0.00032348326751387\\
311.01	0.00032348326751387\\
312.01	0.00032348326751387\\
313.01	0.00032348326751387\\
314.01	0.00032348326751387\\
315.01	0.00032348326751387\\
316.01	0.00032348326751387\\
317.01	0.00032348326751387\\
318.01	0.00032348326751387\\
319.01	0.00032348326751387\\
320.01	0.00032348326751387\\
321.01	0.00032348326751387\\
322.01	0.00032348326751387\\
323.01	0.00032348326751387\\
324.01	0.00032348326751387\\
325.01	0.00032348326751387\\
326.01	0.00032348326751387\\
327.01	0.00032348326751387\\
328.01	0.00032348326751387\\
329.01	0.00032348326751387\\
330.01	0.00032348326751387\\
331.01	0.00032348326751387\\
332.01	0.00032348326751387\\
333.01	0.00032348326751387\\
334.01	0.00032348326751387\\
335.01	0.00032348326751387\\
336.01	0.00032348326751387\\
337.01	0.00032348326751387\\
338.01	0.00032348326751387\\
339.01	0.00032348326751387\\
340.01	0.00032348326751387\\
341.01	0.00032348326751387\\
342.01	0.00032348326751387\\
343.01	0.00032348326751387\\
344.01	0.00032348326751387\\
345.01	0.00032348326751387\\
346.01	0.00032348326751387\\
347.01	0.00032348326751387\\
348.01	0.00032348326751387\\
349.01	0.00032348326751387\\
350.01	0.00032348326751387\\
351.01	0.00032348326751387\\
352.01	0.00032348326751387\\
353.01	0.00032348326751387\\
354.01	0.00032348326751387\\
355.01	0.00032348326751387\\
356.01	0.00032348326751387\\
357.01	0.00032348326751387\\
358.01	0.00032348326751387\\
359.01	0.00032348326751387\\
360.01	0.00032348326751387\\
361.01	0.00032348326751387\\
362.01	0.00032348326751387\\
363.01	0.00032348326751387\\
364.01	0.00032348326751387\\
365.01	0.00032348326751387\\
366.01	0.00032348326751387\\
367.01	0.00032348326751387\\
368.01	0.00032348326751387\\
369.01	0.00032348326751387\\
370.01	0.00032348326751387\\
371.01	0.00032348326751387\\
372.01	0.00032348326751387\\
373.01	0.00032348326751387\\
374.01	0.00032348326751387\\
375.01	0.00032348326751387\\
376.01	0.00032348326751387\\
377.01	0.00032348326751387\\
378.01	0.00032348326751387\\
379.01	0.00032348326751387\\
380.01	0.00032348326751387\\
381.01	0.00032348326751387\\
382.01	0.00032348326751387\\
383.01	0.00032348326751387\\
384.01	0.00032348326751387\\
385.01	0.00032348326751387\\
386.01	0.00032348326751387\\
387.01	0.00032348326751387\\
388.01	0.00032348326751387\\
389.01	0.00032348326751387\\
390.01	0.00032348326751387\\
391.01	0.00032348326751387\\
392.01	0.00032348326751387\\
393.01	0.00032348326751387\\
394.01	0.00032348326751387\\
395.01	0.00032348326751387\\
396.01	0.00032348326751387\\
397.01	0.00032348326751387\\
398.01	0.00032348326751387\\
399.01	0.00032348326751387\\
400.01	0.00032348326751387\\
401.01	0.00032348326751387\\
402.01	0.00032348326751387\\
403.01	0.00032348326751387\\
404.01	0.00032348326751387\\
405.01	0.00032348326751387\\
406.01	0.00032348326751387\\
407.01	0.00032348326751387\\
408.01	0.00032348326751387\\
409.01	0.00032348326751387\\
410.01	0.00032348326751387\\
411.01	0.00032348326751387\\
412.01	0.00032348326751387\\
413.01	0.00032348326751387\\
414.01	0.00032348326751387\\
415.01	0.00032348326751387\\
416.01	0.00032348326751387\\
417.01	0.00032348326751387\\
418.01	0.00032348326751387\\
419.01	0.00032348326751387\\
420.01	0.00032348326751387\\
421.01	0.00032348326751387\\
422.01	0.00032348326751387\\
423.01	0.00032348326751387\\
424.01	0.00032348326751387\\
425.01	0.00032348326751387\\
426.01	0.00032348326751387\\
427.01	0.00032348326751387\\
428.01	0.00032348326751387\\
429.01	0.00032348326751387\\
430.01	0.00032348326751387\\
431.01	0.00032348326751387\\
432.01	0.00032348326751387\\
433.01	0.00032348326751387\\
434.01	0.00032348326751387\\
435.01	0.00032348326751387\\
436.01	0.00032348326751387\\
437.01	0.00032348326751387\\
438.01	0.00032348326751387\\
439.01	0.00032348326751387\\
440.01	0.00032348326751387\\
441.01	0.00032348326751387\\
442.01	0.00032348326751387\\
443.01	0.00032348326751387\\
444.01	0.000323483267513872\\
445.01	0.000323483267513872\\
446.01	0.000323483267513854\\
447.01	0.000323483267513818\\
448.01	0.000323483267513719\\
449.01	0.00032348326751346\\
450.01	0.000323483267512777\\
451.01	0.00032348326751098\\
452.01	0.000323483267506265\\
453.01	0.000323483267493957\\
454.01	0.00032348326746201\\
455.01	0.000323483267379661\\
456.01	0.000323483267169111\\
457.01	0.000323483266636138\\
458.01	0.000323483265303676\\
459.01	0.000323483262024351\\
460.01	0.000323483254114245\\
461.01	0.000323483235527324\\
462.01	0.000323483193342078\\
463.01	0.000323483101985822\\
464.01	0.000323482916574127\\
465.01	0.000323482573337544\\
466.01	0.000323482016935115\\
467.01	0.000323481268148873\\
468.01	0.000323480441666813\\
469.01	0.000323479595467246\\
470.01	0.000323478733894969\\
471.01	0.000323477863201176\\
472.01	0.000323476997294479\\
473.01	0.000323476164510996\\
474.01	0.000323475416246714\\
475.01	0.000323474829585597\\
476.01	0.000323474481653553\\
477.01	0.000323474371361708\\
478.01	0.000323474365043552\\
479.01	0.000323474365043552\\
480.01	0.000323474365043552\\
481.01	0.000323474365043552\\
482.01	0.000323474365043552\\
483.01	0.000323474365043552\\
484.01	0.000323474365043552\\
485.01	0.000323474365043552\\
486.01	0.000323474365043552\\
487.01	0.000323474365043552\\
488.01	0.000323474365043552\\
489.01	0.000323474365043552\\
490.01	0.000323474365043552\\
491.01	0.000323474365043552\\
492.01	0.000323474365043552\\
493.01	0.000323474365043552\\
494.01	0.000323474365043552\\
495.01	0.000323474365043552\\
496.01	0.000323474365043552\\
497.01	0.000323474365043552\\
498.01	0.000323474365043552\\
499.01	0.000323474365043552\\
500.01	0.000323474365043552\\
501.01	0.000323474365043552\\
502.01	0.000323474365043552\\
503.01	0.000323474365043552\\
504.01	0.000323474365043552\\
505.01	0.000323474365043552\\
506.01	0.000323474365043552\\
507.01	0.000323474365043552\\
508.01	0.000323474365043552\\
509.01	0.000323474365043541\\
510.01	0.000323474365043508\\
511.01	0.000323474365043427\\
512.01	0.000323474365043201\\
513.01	0.000323474365042582\\
514.01	0.000323474365040866\\
515.01	0.000323474365036124\\
516.01	0.000323474365023004\\
517.01	0.000323474364986712\\
518.01	0.000323474364886271\\
519.01	0.000323474364608155\\
520.01	0.000323474363837714\\
521.01	0.000323474361702531\\
522.01	0.000323474355783794\\
523.01	0.00032347433937965\\
524.01	0.000323474293952497\\
525.01	0.000323474168395149\\
526.01	0.000323473822617017\\
527.01	0.000323472876248249\\
528.01	0.000323470312193687\\
529.01	0.000323463476687595\\
530.01	0.0003234457191502\\
531.01	0.000323401507088766\\
532.01	0.000323299349623227\\
533.01	0.000323096583750704\\
534.01	0.000322820494104223\\
535.01	0.000322544563133137\\
536.01	0.000322274405627094\\
537.01	0.000322012012008299\\
538.01	0.000321781721082232\\
539.01	0.000321635878047019\\
540.01	0.000321609567441481\\
541.01	0.000321609567440196\\
542.01	0.00032160956743655\\
543.01	0.000321609567426226\\
544.01	0.000321609567396994\\
545.01	0.000321609567314168\\
546.01	0.000321609567079316\\
547.01	0.00032160956641262\\
548.01	0.000321609564516894\\
549.01	0.000321609559114573\\
550.01	0.000321609543674653\\
551.01	0.00032160949938256\\
552.01	0.00032160937172819\\
553.01	0.000321609001700078\\
554.01	0.000321607921710718\\
555.01	0.000321604744132882\\
556.01	0.00032159530906575\\
557.01	0.000321567009926508\\
558.01	0.00032148121484275\\
559.01	0.000321218239453271\\
560.01	0.000320403516967051\\
561.01	0.000317854518528006\\
562.01	0.000309812765753915\\
563.01	0.000290521225180146\\
564.01	0.000258815636837589\\
565.01	0.00022319689057226\\
566.01	0.000180005113740115\\
567.01	0.000117402829674417\\
568.01	5.15310646992881e-05\\
569.01	4.39279151884069e-06\\
570.01	0\\
571.01	0\\
572.01	0\\
573.01	0\\
574.01	0\\
575.01	0\\
576.01	0\\
577.01	0\\
578.01	1.73472347597681e-18\\
579.01	0\\
580.01	0\\
581.01	0\\
582.01	0\\
583.01	1.73472347597681e-18\\
584.01	0\\
585.01	0\\
586.01	0\\
587.01	0\\
588.01	0\\
589.01	0\\
590.01	0\\
591.01	0\\
592.01	0\\
593.01	0\\
594.01	0\\
595.01	0\\
596.01	0\\
597.01	0\\
598.01	0\\
599.01	0\\
599.02	0\\
599.03	0\\
599.04	0\\
599.05	0\\
599.06	0\\
599.07	0\\
599.08	0\\
599.09	0\\
599.1	0\\
599.11	0\\
599.12	0\\
599.13	0\\
599.14	0\\
599.15	0\\
599.16	0\\
599.17	0\\
599.18	0\\
599.19	0\\
599.2	0\\
599.21	0\\
599.22	0\\
599.23	0\\
599.24	0\\
599.25	0\\
599.26	0\\
599.27	0\\
599.28	0\\
599.29	0\\
599.3	0\\
599.31	0\\
599.32	0\\
599.33	0\\
599.34	0\\
599.35	0\\
599.36	0\\
599.37	0\\
599.38	0\\
599.39	0\\
599.4	0\\
599.41	0\\
599.42	0\\
599.43	0\\
599.44	0\\
599.45	0\\
599.46	0\\
599.47	0\\
599.48	0\\
599.49	0\\
599.5	0\\
599.51	0\\
599.52	0\\
599.53	0\\
599.54	0\\
599.55	0\\
599.56	0\\
599.57	0\\
599.58	0\\
599.59	0\\
599.6	0\\
599.61	0\\
599.62	0\\
599.63	0\\
599.64	0\\
599.65	0\\
599.66	0\\
599.67	0\\
599.68	0\\
599.69	0\\
599.7	0\\
599.71	0\\
599.72	0\\
599.73	0\\
599.74	0\\
599.75	0\\
599.76	0\\
599.77	0\\
599.78	0\\
599.79	0\\
599.8	0\\
599.81	0\\
599.82	0\\
599.83	0\\
599.84	0\\
599.85	0\\
599.86	0\\
599.87	0\\
599.88	0\\
599.89	0\\
599.9	0\\
599.91	0\\
599.92	0\\
599.93	0\\
599.94	0\\
599.95	0\\
599.96	0\\
599.97	0\\
599.98	0\\
599.99	0\\
600	0\\
};
\addplot [color=mycolor12,solid,forget plot]
  table[row sep=crcr]{%
0.01	0.00230832677588388\\
1.01	0.00230832677588388\\
2.01	0.00230832677588388\\
3.01	0.00230832677588388\\
4.01	0.00230832677588388\\
5.01	0.00230832677588388\\
6.01	0.00230832677588388\\
7.01	0.00230832677588388\\
8.01	0.00230832677588388\\
9.01	0.00230832677588388\\
10.01	0.00230832677588388\\
11.01	0.00230832677588388\\
12.01	0.00230832677588388\\
13.01	0.00230832677588388\\
14.01	0.00230832677588388\\
15.01	0.00230832677588388\\
16.01	0.00230832677588388\\
17.01	0.00230832677588388\\
18.01	0.00230832677588388\\
19.01	0.00230832677588388\\
20.01	0.00230832677588388\\
21.01	0.00230832677588388\\
22.01	0.00230832677588388\\
23.01	0.00230832677588388\\
24.01	0.00230832677588388\\
25.01	0.00230832677588388\\
26.01	0.00230832677588388\\
27.01	0.00230832677588388\\
28.01	0.00230832677588388\\
29.01	0.00230832677588388\\
30.01	0.00230832677588388\\
31.01	0.00230832677588388\\
32.01	0.00230832677588388\\
33.01	0.00230832677588388\\
34.01	0.00230832677588388\\
35.01	0.00230832677588388\\
36.01	0.00230832677588388\\
37.01	0.00230832677588388\\
38.01	0.00230832677588388\\
39.01	0.00230832677588388\\
40.01	0.00230832677588388\\
41.01	0.00230832677588388\\
42.01	0.00230832677588388\\
43.01	0.00230832677588388\\
44.01	0.00230832677588388\\
45.01	0.00230832677588388\\
46.01	0.00230832677588388\\
47.01	0.00230832677588388\\
48.01	0.00230832677588388\\
49.01	0.00230832677588388\\
50.01	0.00230832677588388\\
51.01	0.00230832677588388\\
52.01	0.00230832677588388\\
53.01	0.00230832677588388\\
54.01	0.00230832677588388\\
55.01	0.00230832677588388\\
56.01	0.00230832677588388\\
57.01	0.00230832677588388\\
58.01	0.00230832677588388\\
59.01	0.00230832677588388\\
60.01	0.00230832677588388\\
61.01	0.00230832677588388\\
62.01	0.00230832677588388\\
63.01	0.00230832677588388\\
64.01	0.00230832677588388\\
65.01	0.00230832677588388\\
66.01	0.00230832677588388\\
67.01	0.00230832677588388\\
68.01	0.00230832677588388\\
69.01	0.00230832677588388\\
70.01	0.00230832677588388\\
71.01	0.00230832677588388\\
72.01	0.00230832677588388\\
73.01	0.00230832677588388\\
74.01	0.00230832677588388\\
75.01	0.00230832677588388\\
76.01	0.00230832677588388\\
77.01	0.00230832677588388\\
78.01	0.00230832677588388\\
79.01	0.00230832677588388\\
80.01	0.00230832677588388\\
81.01	0.00230832677588388\\
82.01	0.00230832677588388\\
83.01	0.00230832677588388\\
84.01	0.00230832677588388\\
85.01	0.00230832677588388\\
86.01	0.00230832677588388\\
87.01	0.00230832677588388\\
88.01	0.00230832677588388\\
89.01	0.00230832677588388\\
90.01	0.00230832677588388\\
91.01	0.00230832677588388\\
92.01	0.00230832677588388\\
93.01	0.00230832677588388\\
94.01	0.00230832677588388\\
95.01	0.00230832677588388\\
96.01	0.00230832677588388\\
97.01	0.00230832677588388\\
98.01	0.00230832677588388\\
99.01	0.00230832677588388\\
100.01	0.00230832677588388\\
101.01	0.00230832677588388\\
102.01	0.00230832677588388\\
103.01	0.00230832677588388\\
104.01	0.00230832677588388\\
105.01	0.00230832677588388\\
106.01	0.00230832677588388\\
107.01	0.00230832677588388\\
108.01	0.00230832677588388\\
109.01	0.00230832677588388\\
110.01	0.00230832677588388\\
111.01	0.00230832677588388\\
112.01	0.00230832677588388\\
113.01	0.00230832677588388\\
114.01	0.00230832677588388\\
115.01	0.00230832677588388\\
116.01	0.00230832677588388\\
117.01	0.00230832677588388\\
118.01	0.00230832677588388\\
119.01	0.00230832677588388\\
120.01	0.00230832677588388\\
121.01	0.00230832677588388\\
122.01	0.00230832677588388\\
123.01	0.00230832677588388\\
124.01	0.00230832677588388\\
125.01	0.00230832677588388\\
126.01	0.00230832677588388\\
127.01	0.00230832677588388\\
128.01	0.00230832677588388\\
129.01	0.00230832677588388\\
130.01	0.00230832677588388\\
131.01	0.00230832677588388\\
132.01	0.00230832677588388\\
133.01	0.00230832677588388\\
134.01	0.00230832677588388\\
135.01	0.00230832677588388\\
136.01	0.00230832677588388\\
137.01	0.00230832677588388\\
138.01	0.00230832677588388\\
139.01	0.00230832677588388\\
140.01	0.00230832677588388\\
141.01	0.00230832677588388\\
142.01	0.00230832677588388\\
143.01	0.00230832677588388\\
144.01	0.00230832677588388\\
145.01	0.00230832677588388\\
146.01	0.00230832677588388\\
147.01	0.00230832677588388\\
148.01	0.00230832677588388\\
149.01	0.00230832677588388\\
150.01	0.00230832677588388\\
151.01	0.00230832677588388\\
152.01	0.00230832677588388\\
153.01	0.00230832677588388\\
154.01	0.00230832677588388\\
155.01	0.00230832677588388\\
156.01	0.00230832677588388\\
157.01	0.00230832677588388\\
158.01	0.00230832677588388\\
159.01	0.00230832677588388\\
160.01	0.00230832677588388\\
161.01	0.00230832677588388\\
162.01	0.00230832677588388\\
163.01	0.00230832677588388\\
164.01	0.00230832677588388\\
165.01	0.00230832677588388\\
166.01	0.00230832677588388\\
167.01	0.00230832677588388\\
168.01	0.00230832677588388\\
169.01	0.00230832677588388\\
170.01	0.00230832677588388\\
171.01	0.00230832677588388\\
172.01	0.00230832677588388\\
173.01	0.00230832677588388\\
174.01	0.00230832677588388\\
175.01	0.00230832677588388\\
176.01	0.00230832677588388\\
177.01	0.00230832677588388\\
178.01	0.00230832677588388\\
179.01	0.00230832677588388\\
180.01	0.00230832677588388\\
181.01	0.00230832677588388\\
182.01	0.00230832677588388\\
183.01	0.00230832677588388\\
184.01	0.00230832677588388\\
185.01	0.00230832677588388\\
186.01	0.00230832677588388\\
187.01	0.00230832677588388\\
188.01	0.00230832677588388\\
189.01	0.00230832677588388\\
190.01	0.00230832677588388\\
191.01	0.00230832677588388\\
192.01	0.00230832677588388\\
193.01	0.00230832677588388\\
194.01	0.00230832677588388\\
195.01	0.00230832677588388\\
196.01	0.00230832677588388\\
197.01	0.00230832677588388\\
198.01	0.00230832677588388\\
199.01	0.00230832677588388\\
200.01	0.00230832677588388\\
201.01	0.00230832677588388\\
202.01	0.00230832677588388\\
203.01	0.00230832677588388\\
204.01	0.00230832677588388\\
205.01	0.00230832677588388\\
206.01	0.00230832677588388\\
207.01	0.00230832677588388\\
208.01	0.00230832677588388\\
209.01	0.00230832677588388\\
210.01	0.00230832677588388\\
211.01	0.00230832677588388\\
212.01	0.00230832677588388\\
213.01	0.00230832677588388\\
214.01	0.00230832677588388\\
215.01	0.00230832677588388\\
216.01	0.00230832677588388\\
217.01	0.00230832677588388\\
218.01	0.00230832677588388\\
219.01	0.00230832677588388\\
220.01	0.00230832677588388\\
221.01	0.00230832677588388\\
222.01	0.00230832677588388\\
223.01	0.00230832677588388\\
224.01	0.00230832677588388\\
225.01	0.00230832677588388\\
226.01	0.00230832677588388\\
227.01	0.00230832677588388\\
228.01	0.00230832677588388\\
229.01	0.00230832677588388\\
230.01	0.00230832677588388\\
231.01	0.00230832677588388\\
232.01	0.00230832677588388\\
233.01	0.00230832677588388\\
234.01	0.00230832677588388\\
235.01	0.00230832677588388\\
236.01	0.00230832677588388\\
237.01	0.00230832677588388\\
238.01	0.00230832677588388\\
239.01	0.00230832677588388\\
240.01	0.00230832677588388\\
241.01	0.00230832677588388\\
242.01	0.00230832677588388\\
243.01	0.00230832677588388\\
244.01	0.00230832677588388\\
245.01	0.00230832677588388\\
246.01	0.00230832677588388\\
247.01	0.00230832677588388\\
248.01	0.00230832677588388\\
249.01	0.00230832677588388\\
250.01	0.00230832677588388\\
251.01	0.00230832677588388\\
252.01	0.00230832677588388\\
253.01	0.00230832677588388\\
254.01	0.00230832677588388\\
255.01	0.00230832677588388\\
256.01	0.00230832677588388\\
257.01	0.00230832677588388\\
258.01	0.00230832677588388\\
259.01	0.00230832677588388\\
260.01	0.00230832677588388\\
261.01	0.00230832677588388\\
262.01	0.00230832677588388\\
263.01	0.00230832677588388\\
264.01	0.00230832677588388\\
265.01	0.00230832677588388\\
266.01	0.00230832677588388\\
267.01	0.00230832677588388\\
268.01	0.00230832677588388\\
269.01	0.00230832677588388\\
270.01	0.00230832677588388\\
271.01	0.00230832677588388\\
272.01	0.00230832677588388\\
273.01	0.00230832677588388\\
274.01	0.00230832677588388\\
275.01	0.00230832677588388\\
276.01	0.00230832677588388\\
277.01	0.00230832677588388\\
278.01	0.00230832677588388\\
279.01	0.00230832677588388\\
280.01	0.00230832677588388\\
281.01	0.00230832677588388\\
282.01	0.00230832677588388\\
283.01	0.00230832677588388\\
284.01	0.00230832677588388\\
285.01	0.00230832677588388\\
286.01	0.00230832677588388\\
287.01	0.00230832677588388\\
288.01	0.00230832677588388\\
289.01	0.00230832677588388\\
290.01	0.00230832677588388\\
291.01	0.00230832677588388\\
292.01	0.00230832677588388\\
293.01	0.00230832677588388\\
294.01	0.00230832677588388\\
295.01	0.00230832677588388\\
296.01	0.00230832677588388\\
297.01	0.00230832677588388\\
298.01	0.00230832677588388\\
299.01	0.00230832677588388\\
300.01	0.00230832677588388\\
301.01	0.00230832677588388\\
302.01	0.00230832677588388\\
303.01	0.00230832677588388\\
304.01	0.00230832677588388\\
305.01	0.00230832677588388\\
306.01	0.00230832677588388\\
307.01	0.00230832677588388\\
308.01	0.00230832677588388\\
309.01	0.00230832677588388\\
310.01	0.00230832677588388\\
311.01	0.00230832677588388\\
312.01	0.00230832677588388\\
313.01	0.00230832677588388\\
314.01	0.00230832677588388\\
315.01	0.00230832677588388\\
316.01	0.00230832677588388\\
317.01	0.00230832677588388\\
318.01	0.00230832677588388\\
319.01	0.00230832677588388\\
320.01	0.00230832677588388\\
321.01	0.00230832677588388\\
322.01	0.00230832677588388\\
323.01	0.00230832677588388\\
324.01	0.00230832677588388\\
325.01	0.00230832677588388\\
326.01	0.00230832677588388\\
327.01	0.00230832677588388\\
328.01	0.00230832677588388\\
329.01	0.00230832677588388\\
330.01	0.00230832677588388\\
331.01	0.00230832677588388\\
332.01	0.00230832677588388\\
333.01	0.00230832677588388\\
334.01	0.00230832677588388\\
335.01	0.00230832677588388\\
336.01	0.00230832677588388\\
337.01	0.00230832677588388\\
338.01	0.00230832677588388\\
339.01	0.00230832677588388\\
340.01	0.00230832677588388\\
341.01	0.00230832677588388\\
342.01	0.00230832677588388\\
343.01	0.00230832677588388\\
344.01	0.00230832677588388\\
345.01	0.00230832677588388\\
346.01	0.00230832677588388\\
347.01	0.00230832677588388\\
348.01	0.00230832677588388\\
349.01	0.00230832677588388\\
350.01	0.00230832677588388\\
351.01	0.00230832677588388\\
352.01	0.00230832677588388\\
353.01	0.00230832677588388\\
354.01	0.00230832677588388\\
355.01	0.00230832677588388\\
356.01	0.00230832677588388\\
357.01	0.00230832677588388\\
358.01	0.00230832677588388\\
359.01	0.00230832677588388\\
360.01	0.00230832677588388\\
361.01	0.00230832677588388\\
362.01	0.00230832677588388\\
363.01	0.00230832677588388\\
364.01	0.00230832677588388\\
365.01	0.00230832677588388\\
366.01	0.00230832677588388\\
367.01	0.00230832677588388\\
368.01	0.00230832677588388\\
369.01	0.00230832677588388\\
370.01	0.00230832677588388\\
371.01	0.00230832677588388\\
372.01	0.00230832677588388\\
373.01	0.00230832677588388\\
374.01	0.00230832677588388\\
375.01	0.00230832677588388\\
376.01	0.00230832677588388\\
377.01	0.00230832677588388\\
378.01	0.00230832677588388\\
379.01	0.00230832677588388\\
380.01	0.00230832677588388\\
381.01	0.00230832677588388\\
382.01	0.00230832677588388\\
383.01	0.00230832677588388\\
384.01	0.00230832677588388\\
385.01	0.00230832677588388\\
386.01	0.00230832677588388\\
387.01	0.00230832677588388\\
388.01	0.00230832677588388\\
389.01	0.00230832677588388\\
390.01	0.00230832677588388\\
391.01	0.00230832677588388\\
392.01	0.00230832677588388\\
393.01	0.00230832677588388\\
394.01	0.00230832677588388\\
395.01	0.00230832677588388\\
396.01	0.00230832677588388\\
397.01	0.00230832677588388\\
398.01	0.00230832677588388\\
399.01	0.00230832677588388\\
400.01	0.00230832677588388\\
401.01	0.00230832677588388\\
402.01	0.00230832677588388\\
403.01	0.00230832677588388\\
404.01	0.00230832677588388\\
405.01	0.00230832677588388\\
406.01	0.00230832677588388\\
407.01	0.00230832677588388\\
408.01	0.00230832677588388\\
409.01	0.00230832677588388\\
410.01	0.00230832677588388\\
411.01	0.00230832677588388\\
412.01	0.00230832677588388\\
413.01	0.00230832677588388\\
414.01	0.00230832677588388\\
415.01	0.00230832677588388\\
416.01	0.00230832677588388\\
417.01	0.00230832677588388\\
418.01	0.00230832677588388\\
419.01	0.00230832677588388\\
420.01	0.00230832677588388\\
421.01	0.00230832677588388\\
422.01	0.00230832677588388\\
423.01	0.00230832677588388\\
424.01	0.00230832677588388\\
425.01	0.00230832677588388\\
426.01	0.00230832677588388\\
427.01	0.00230832677588388\\
428.01	0.00230832677588388\\
429.01	0.00230832677588388\\
430.01	0.00230832677588388\\
431.01	0.00230832677588388\\
432.01	0.00230832677588388\\
433.01	0.00230832677588388\\
434.01	0.00230832677588388\\
435.01	0.00230832677588388\\
436.01	0.00230832677588388\\
437.01	0.00230832677588388\\
438.01	0.00230832677588388\\
439.01	0.00230832677588388\\
440.01	0.00230832677588388\\
441.01	0.00230832677588388\\
442.01	0.00230832677588388\\
443.01	0.00230832677588388\\
444.01	0.00230832677588388\\
445.01	0.00230832677588388\\
446.01	0.00230832677588388\\
447.01	0.00230832677588385\\
448.01	0.00230832677588378\\
449.01	0.00230832677588358\\
450.01	0.00230832677588308\\
451.01	0.00230832677588178\\
452.01	0.00230832677587848\\
453.01	0.00230832677587017\\
454.01	0.00230832677584946\\
455.01	0.00230832677579852\\
456.01	0.00230832677567521\\
457.01	0.00230832677538211\\
458.01	0.00230832677470067\\
459.01	0.00230832677315791\\
460.01	0.00230832676977686\\
461.01	0.00230832676265952\\
462.01	0.00230832674841642\\
463.01	0.00230832672169671\\
464.01	0.00230832667559282\\
465.01	0.00230832660424001\\
466.01	0.00230832650799597\\
467.01	0.00230832639644697\\
468.01	0.00230832628012502\\
469.01	0.00230832616191159\\
470.01	0.00230832604287346\\
471.01	0.00230832592509532\\
472.01	0.00230832581240644\\
473.01	0.00230832571105018\\
474.01	0.00230832562949901\\
475.01	0.00230832557582186\\
476.01	0.00230832555157799\\
477.01	0.00230832554657476\\
478.01	0.00230832554643708\\
479.01	0.00230832554643708\\
480.01	0.00230832554643708\\
481.01	0.00230832554643708\\
482.01	0.00230832554643708\\
483.01	0.00230832554643708\\
484.01	0.00230832554643708\\
485.01	0.00230832554643708\\
486.01	0.00230832554643708\\
487.01	0.00230832554643708\\
488.01	0.00230832554643708\\
489.01	0.00230832554643708\\
490.01	0.00230832554643708\\
491.01	0.00230832554643708\\
492.01	0.00230832554643708\\
493.01	0.00230832554643708\\
494.01	0.00230832554643708\\
495.01	0.00230832554643708\\
496.01	0.00230832554643708\\
497.01	0.00230832554643708\\
498.01	0.00230832554643708\\
499.01	0.00230832554643708\\
500.01	0.00230832554643708\\
501.01	0.00230832554643708\\
502.01	0.00230832554643708\\
503.01	0.00230832554643708\\
504.01	0.00230832554643708\\
505.01	0.00230832554643708\\
506.01	0.00230832554643708\\
507.01	0.00230832554643708\\
508.01	0.00230832554643708\\
509.01	0.00230832554643708\\
510.01	0.00230832554643706\\
511.01	0.00230832554643699\\
512.01	0.00230832554643681\\
513.01	0.00230832554643632\\
514.01	0.00230832554643497\\
515.01	0.0023083255464313\\
516.01	0.00230832554642139\\
517.01	0.00230832554639465\\
518.01	0.00230832554632288\\
519.01	0.00230832554613126\\
520.01	0.00230832554562292\\
521.01	0.00230832554428438\\
522.01	0.00230832554079122\\
523.01	0.0023083255317732\\
524.01	0.00230832550879643\\
525.01	0.00230832545119599\\
526.01	0.002308325309691\\
527.01	0.00230832497087839\\
528.01	0.00230832418619398\\
529.01	0.00230832244749176\\
530.01	0.00230831882192436\\
531.01	0.00230831189250172\\
532.01	0.00230830028466374\\
533.01	0.00230828450470487\\
534.01	0.00230826815361379\\
535.01	0.00230825300739821\\
536.01	0.00230823904377992\\
537.01	0.00230822681893579\\
538.01	0.00230821812485519\\
539.01	0.00230821447379591\\
540.01	0.00230821417331235\\
541.01	0.002308214173311\\
542.01	0.0023082141733072\\
543.01	0.00230821417329649\\
544.01	0.00230821417326634\\
545.01	0.00230821417318166\\
546.01	0.00230821417294417\\
547.01	0.00230821417227934\\
548.01	0.00230821417042241\\
549.01	0.00230821416524946\\
550.01	0.00230821415088388\\
551.01	0.00230821411113991\\
552.01	0.0023082140016871\\
553.01	0.00230821370195917\\
554.01	0.00230821288695113\\
555.01	0.00230821069054941\\
556.01	0.00230820483924602\\
557.01	0.00230818948614138\\
558.01	0.00230815002122632\\
559.01	0.00230805146354952\\
560.01	0.00230781562226196\\
561.01	0.00230728866371109\\
562.01	0.00230625128843328\\
563.01	0.00230463539557428\\
564.01	0.00230261463947325\\
565.01	0.00230027902002805\\
566.01	0.00229739045615409\\
567.01	0.00229414404146691\\
568.01	0.00229160501562282\\
569.01	0.00229055117153649\\
570.01	0.00229052498630496\\
571.01	0.00229052423870807\\
572.01	0.00229052200145302\\
573.01	0.00229051531235272\\
574.01	0.00229049533245125\\
575.01	0.00229043571706054\\
576.01	0.00229025804248841\\
577.01	0.00228972916792507\\
578.01	0.00228815701010785\\
579.01	0.00228349029843025\\
580.01	0.00226965902075653\\
581.01	0.00222878167195284\\
582.01	0.00213964864116229\\
583.01	0.00201726281566172\\
584.01	0.00185939417254293\\
585.01	0.00164117735174267\\
586.01	0.00140842561566351\\
587.01	0.00116447021169677\\
588.01	0.000899880963021769\\
589.01	0.00058797839348404\\
590.01	0.000224847652437047\\
591.01	3.75983539933726e-06\\
592.01	0\\
593.01	0\\
594.01	0\\
595.01	0\\
596.01	0\\
597.01	0\\
598.01	0\\
599.01	0\\
599.02	0\\
599.03	0\\
599.04	0\\
599.05	0\\
599.06	0\\
599.07	0\\
599.08	0\\
599.09	0\\
599.1	0\\
599.11	0\\
599.12	0\\
599.13	0\\
599.14	0\\
599.15	0\\
599.16	0\\
599.17	0\\
599.18	0\\
599.19	0\\
599.2	0\\
599.21	0\\
599.22	0\\
599.23	0\\
599.24	0\\
599.25	0\\
599.26	0\\
599.27	0\\
599.28	0\\
599.29	0\\
599.3	0\\
599.31	0\\
599.32	0\\
599.33	0\\
599.34	0\\
599.35	0\\
599.36	0\\
599.37	0\\
599.38	0\\
599.39	0\\
599.4	0\\
599.41	0\\
599.42	0\\
599.43	0\\
599.44	0\\
599.45	0\\
599.46	0\\
599.47	0\\
599.48	0\\
599.49	0\\
599.5	0\\
599.51	0\\
599.52	0\\
599.53	0\\
599.54	0\\
599.55	0\\
599.56	0\\
599.57	0\\
599.58	0\\
599.59	0\\
599.6	0\\
599.61	0\\
599.62	0\\
599.63	0\\
599.64	0\\
599.65	0\\
599.66	0\\
599.67	0\\
599.68	0\\
599.69	0\\
599.7	0\\
599.71	0\\
599.72	0\\
599.73	0\\
599.74	0\\
599.75	0\\
599.76	0\\
599.77	0\\
599.78	0\\
599.79	0\\
599.8	0\\
599.81	0\\
599.82	0\\
599.83	0\\
599.84	0\\
599.85	0\\
599.86	0\\
599.87	0\\
599.88	0\\
599.89	0\\
599.9	0\\
599.91	0\\
599.92	0\\
599.93	0\\
599.94	0\\
599.95	0\\
599.96	0\\
599.97	0\\
599.98	0\\
599.99	0\\
600	0\\
};
\addplot [color=mycolor13,solid,forget plot]
  table[row sep=crcr]{%
0.01	0\\
1.01	0\\
2.01	0\\
3.01	0\\
4.01	0\\
5.01	0\\
6.01	0\\
7.01	0\\
8.01	0\\
9.01	0\\
10.01	0\\
11.01	0\\
12.01	0\\
13.01	0\\
14.01	0\\
15.01	0\\
16.01	0\\
17.01	0\\
18.01	0\\
19.01	0\\
20.01	0\\
21.01	0\\
22.01	0\\
23.01	0\\
24.01	0\\
25.01	0\\
26.01	0\\
27.01	0\\
28.01	0\\
29.01	0\\
30.01	0\\
31.01	0\\
32.01	0\\
33.01	0\\
34.01	0\\
35.01	0\\
36.01	0\\
37.01	0\\
38.01	0\\
39.01	0\\
40.01	0\\
41.01	0\\
42.01	0\\
43.01	0\\
44.01	0\\
45.01	0\\
46.01	0\\
47.01	0\\
48.01	0\\
49.01	0\\
50.01	0\\
51.01	0\\
52.01	0\\
53.01	0\\
54.01	0\\
55.01	0\\
56.01	0\\
57.01	0\\
58.01	0\\
59.01	0\\
60.01	0\\
61.01	0\\
62.01	0\\
63.01	0\\
64.01	0\\
65.01	0\\
66.01	0\\
67.01	0\\
68.01	0\\
69.01	0\\
70.01	0\\
71.01	0\\
72.01	0\\
73.01	0\\
74.01	0\\
75.01	0\\
76.01	0\\
77.01	0\\
78.01	0\\
79.01	0\\
80.01	0\\
81.01	0\\
82.01	0\\
83.01	0\\
84.01	0\\
85.01	0\\
86.01	0\\
87.01	0\\
88.01	0\\
89.01	0\\
90.01	0\\
91.01	0\\
92.01	0\\
93.01	0\\
94.01	0\\
95.01	0\\
96.01	0\\
97.01	0\\
98.01	0\\
99.01	0\\
100.01	0\\
101.01	0\\
102.01	0\\
103.01	0\\
104.01	0\\
105.01	0\\
106.01	0\\
107.01	0\\
108.01	0\\
109.01	0\\
110.01	0\\
111.01	0\\
112.01	0\\
113.01	0\\
114.01	0\\
115.01	0\\
116.01	0\\
117.01	0\\
118.01	0\\
119.01	0\\
120.01	0\\
121.01	0\\
122.01	0\\
123.01	0\\
124.01	0\\
125.01	0\\
126.01	0\\
127.01	0\\
128.01	0\\
129.01	0\\
130.01	0\\
131.01	0\\
132.01	0\\
133.01	0\\
134.01	0\\
135.01	0\\
136.01	0\\
137.01	0\\
138.01	0\\
139.01	0\\
140.01	0\\
141.01	0\\
142.01	0\\
143.01	0\\
144.01	0\\
145.01	0\\
146.01	0\\
147.01	0\\
148.01	0\\
149.01	0\\
150.01	0\\
151.01	0\\
152.01	0\\
153.01	0\\
154.01	0\\
155.01	0\\
156.01	0\\
157.01	0\\
158.01	0\\
159.01	0\\
160.01	0\\
161.01	0\\
162.01	0\\
163.01	0\\
164.01	0\\
165.01	0\\
166.01	0\\
167.01	0\\
168.01	0\\
169.01	0\\
170.01	0\\
171.01	0\\
172.01	0\\
173.01	0\\
174.01	0\\
175.01	0\\
176.01	0\\
177.01	0\\
178.01	0\\
179.01	0\\
180.01	0\\
181.01	0\\
182.01	0\\
183.01	0\\
184.01	0\\
185.01	0\\
186.01	0\\
187.01	0\\
188.01	0\\
189.01	0\\
190.01	0\\
191.01	0\\
192.01	0\\
193.01	0\\
194.01	0\\
195.01	0\\
196.01	0\\
197.01	0\\
198.01	0\\
199.01	0\\
200.01	0\\
201.01	0\\
202.01	0\\
203.01	0\\
204.01	0\\
205.01	0\\
206.01	0\\
207.01	0\\
208.01	0\\
209.01	0\\
210.01	0\\
211.01	0\\
212.01	0\\
213.01	0\\
214.01	0\\
215.01	0\\
216.01	0\\
217.01	0\\
218.01	0\\
219.01	0\\
220.01	0\\
221.01	0\\
222.01	0\\
223.01	0\\
224.01	0\\
225.01	0\\
226.01	0\\
227.01	0\\
228.01	0\\
229.01	0\\
230.01	0\\
231.01	0\\
232.01	0\\
233.01	0\\
234.01	0\\
235.01	0\\
236.01	0\\
237.01	0\\
238.01	0\\
239.01	0\\
240.01	0\\
241.01	0\\
242.01	0\\
243.01	0\\
244.01	0\\
245.01	0\\
246.01	0\\
247.01	0\\
248.01	0\\
249.01	0\\
250.01	0\\
251.01	0\\
252.01	0\\
253.01	0\\
254.01	0\\
255.01	0\\
256.01	0\\
257.01	0\\
258.01	0\\
259.01	0\\
260.01	0\\
261.01	0\\
262.01	0\\
263.01	0\\
264.01	0\\
265.01	0\\
266.01	0\\
267.01	0\\
268.01	0\\
269.01	0\\
270.01	0\\
271.01	0\\
272.01	0\\
273.01	0\\
274.01	0\\
275.01	0\\
276.01	0\\
277.01	0\\
278.01	0\\
279.01	0\\
280.01	0\\
281.01	0\\
282.01	0\\
283.01	0\\
284.01	0\\
285.01	0\\
286.01	0\\
287.01	0\\
288.01	0\\
289.01	0\\
290.01	0\\
291.01	0\\
292.01	0\\
293.01	0\\
294.01	0\\
295.01	0\\
296.01	0\\
297.01	0\\
298.01	0\\
299.01	0\\
300.01	0\\
301.01	0\\
302.01	0\\
303.01	0\\
304.01	0\\
305.01	0\\
306.01	0\\
307.01	0\\
308.01	0\\
309.01	0\\
310.01	0\\
311.01	0\\
312.01	0\\
313.01	0\\
314.01	0\\
315.01	0\\
316.01	0\\
317.01	0\\
318.01	0\\
319.01	0\\
320.01	0\\
321.01	0\\
322.01	0\\
323.01	0\\
324.01	0\\
325.01	0\\
326.01	0\\
327.01	0\\
328.01	0\\
329.01	0\\
330.01	0\\
331.01	0\\
332.01	0\\
333.01	0\\
334.01	0\\
335.01	0\\
336.01	0\\
337.01	0\\
338.01	0\\
339.01	0\\
340.01	0\\
341.01	0\\
342.01	0\\
343.01	0\\
344.01	0\\
345.01	0\\
346.01	0\\
347.01	0\\
348.01	0\\
349.01	0\\
350.01	0\\
351.01	0\\
352.01	0\\
353.01	0\\
354.01	0\\
355.01	0\\
356.01	0\\
357.01	0\\
358.01	0\\
359.01	0\\
360.01	0\\
361.01	0\\
362.01	0\\
363.01	0\\
364.01	0\\
365.01	0\\
366.01	0\\
367.01	0\\
368.01	0\\
369.01	0\\
370.01	0\\
371.01	0\\
372.01	0\\
373.01	0\\
374.01	0\\
375.01	0\\
376.01	0\\
377.01	0\\
378.01	0\\
379.01	0\\
380.01	0\\
381.01	0\\
382.01	0\\
383.01	0\\
384.01	0\\
385.01	0\\
386.01	0\\
387.01	0\\
388.01	0\\
389.01	0\\
390.01	0\\
391.01	0\\
392.01	0\\
393.01	0\\
394.01	0\\
395.01	0\\
396.01	0\\
397.01	0\\
398.01	0\\
399.01	0\\
400.01	0\\
401.01	0\\
402.01	0\\
403.01	0\\
404.01	0\\
405.01	0\\
406.01	0\\
407.01	0\\
408.01	0\\
409.01	0\\
410.01	0\\
411.01	0\\
412.01	0\\
413.01	0\\
414.01	0\\
415.01	0\\
416.01	0\\
417.01	0\\
418.01	0\\
419.01	0\\
420.01	0\\
421.01	0\\
422.01	0\\
423.01	0\\
424.01	0\\
425.01	0\\
426.01	0\\
427.01	0\\
428.01	0\\
429.01	0\\
430.01	0\\
431.01	0\\
432.01	0\\
433.01	0\\
434.01	0\\
435.01	0\\
436.01	0\\
437.01	0\\
438.01	0\\
439.01	0\\
440.01	0\\
441.01	0\\
442.01	0\\
443.01	0\\
444.01	0\\
445.01	0\\
446.01	0\\
447.01	0\\
448.01	0\\
449.01	0\\
450.01	0\\
451.01	0\\
452.01	0\\
453.01	0\\
454.01	0\\
455.01	0\\
456.01	0\\
457.01	0\\
458.01	0\\
459.01	0\\
460.01	0\\
461.01	0\\
462.01	0\\
463.01	0\\
464.01	0\\
465.01	0\\
466.01	0\\
467.01	0\\
468.01	0\\
469.01	0\\
470.01	0\\
471.01	0\\
472.01	0\\
473.01	0\\
474.01	0\\
475.01	0\\
476.01	0\\
477.01	0\\
478.01	0\\
479.01	0\\
480.01	0\\
481.01	0\\
482.01	0\\
483.01	0\\
484.01	0\\
485.01	0\\
486.01	0\\
487.01	0\\
488.01	0\\
489.01	0\\
490.01	0\\
491.01	0\\
492.01	0\\
493.01	0\\
494.01	0\\
495.01	0\\
496.01	0\\
497.01	0\\
498.01	0\\
499.01	0\\
500.01	0\\
501.01	0\\
502.01	0\\
503.01	0\\
504.01	0\\
505.01	0\\
506.01	0\\
507.01	0\\
508.01	0\\
509.01	0\\
510.01	0\\
511.01	0\\
512.01	0\\
513.01	0\\
514.01	0\\
515.01	0\\
516.01	0\\
517.01	0\\
518.01	0\\
519.01	0\\
520.01	0\\
521.01	0\\
522.01	0\\
523.01	0\\
524.01	0\\
525.01	0\\
526.01	0\\
527.01	0\\
528.01	0\\
529.01	0\\
530.01	0\\
531.01	0\\
532.01	0\\
533.01	0\\
534.01	0\\
535.01	0\\
536.01	0\\
537.01	0\\
538.01	0\\
539.01	0\\
540.01	0\\
541.01	0\\
542.01	0\\
543.01	0\\
544.01	0\\
545.01	0\\
546.01	0\\
547.01	0\\
548.01	0\\
549.01	0\\
550.01	0\\
551.01	0\\
552.01	0\\
553.01	0\\
554.01	0\\
555.01	0\\
556.01	0\\
557.01	0\\
558.01	0\\
559.01	0\\
560.01	0\\
561.01	0\\
562.01	0\\
563.01	0\\
564.01	0\\
565.01	0\\
566.01	0\\
567.01	0\\
568.01	0\\
569.01	0\\
570.01	0\\
571.01	0\\
572.01	0\\
573.01	0\\
574.01	0\\
575.01	0\\
576.01	0\\
577.01	0\\
578.01	0\\
579.01	0\\
580.01	0\\
581.01	0\\
582.01	0\\
583.01	0\\
584.01	0\\
585.01	0\\
586.01	0\\
587.01	0\\
588.01	0\\
589.01	0\\
590.01	0\\
591.01	0\\
592.01	0\\
593.01	0\\
594.01	0\\
595.01	0\\
596.01	0\\
597.01	0\\
598.01	0\\
599.01	0.0022382818960328\\
599.02	0.00228829715400941\\
599.03	0.00233870844988766\\
599.04	0.00238951982934219\\
599.05	0.00244073538661882\\
599.06	0.00249235926524579\\
599.07	0.00254439565875786\\
599.08	0.00259684881143348\\
599.09	0.00264972301904528\\
599.1	0.0027030226296243\\
599.11	0.00275675204423814\\
599.12	0.00281091571778341\\
599.13	0.00286551815979286\\
599.14	0.00292056393525741\\
599.15	0.00297605766546351\\
599.16	0.00303200402884618\\
599.17	0.00308840776185803\\
599.18	0.00314527365985481\\
599.19	0.00320260657799763\\
599.2	0.00326041143217246\\
599.21	0.00331869319992726\\
599.22	0.00337745692142712\\
599.23	0.00343670770042793\\
599.24	0.00349645070526896\\
599.25	0.00355669116988489\\
599.26	0.0036174343948377\\
599.27	0.00367868574836898\\
599.28	0.00374045066747309\\
599.29	0.00380273465899182\\
599.3	0.00386554330073107\\
599.31	0.00392888224260002\\
599.32	0.00399275720777354\\
599.33	0.00405717399387833\\
599.34	0.00412213847420342\\
599.35	0.00418765659893578\\
599.36	0.00425373439642159\\
599.37	0.00432037797445393\\
599.38	0.00438759352158759\\
599.39	0.00445538730848178\\
599.4	0.00452376568927147\\
599.41	0.0045927351029681\\
599.42	0.00466230207489068\\
599.43	0.0047324732181279\\
599.44	0.00480325523503231\\
599.45	0.0048746549187474\\
599.46	0.00494667915476945\\
599.47	0.005019334922583\\
599.48	0.00509262929729296\\
599.49	0.00516656945129368\\
599.5	0.00524116265597606\\
599.51	0.00531641628347389\\
599.52	0.00539233780845056\\
599.53	0.00546893480992739\\
599.54	0.00554621497315491\\
599.55	0.00562418609152839\\
599.56	0.00570285606854896\\
599.57	0.00578223291983187\\
599.58	0.00586232477516331\\
599.59	0.00594313988060741\\
599.6	0.006024686600665\\
599.61	0.00610697342048581\\
599.62	0.00619000894813599\\
599.63	0.00627380191692264\\
599.64	0.00635836118777732\\
599.65	0.00644369575170061\\
599.66	0.00652981473226971\\
599.67	0.00661672738821123\\
599.68	0.00670444311604154\\
599.69	0.00679297145277696\\
599.7	0.00688232207871629\\
599.71	0.00697250482029821\\
599.72	0.0070635296530363\\
599.73	0.00715540670453446\\
599.74	0.00724814625758565\\
599.75	0.00734175875335703\\
599.76	0.00743625479466474\\
599.77	0.0075316451493416\\
599.78	0.00762794075370135\\
599.79	0.00772515271610301\\
599.8	0.00782329232061922\\
599.81	0.00792237103081269\\
599.82	0.00802240049362481\\
599.83	0.00812339254338108\\
599.84	0.00822535920591771\\
599.85	0.00832831270283452\\
599.86	0.00843226545587912\\
599.87	0.00853723009146771\\
599.88	0.00864321944534821\\
599.89	0.00875024656741154\\
599.9	0.0088583247266573\\
599.91	0.00896746741632041\\
599.92	0.00907768835916546\\
599.93	0.0091890015129561\\
599.94	0.00930142107610699\\
599.95	0.00941496149352625\\
599.96	0.009529637462657\\
599.97	0.00964546393972657\\
599.98	0.00976245614621301\\
599.99	0.00988062957553847\\
600	0.01\\
};
\addplot [color=mycolor14,solid,forget plot]
  table[row sep=crcr]{%
0.01	0\\
1.01	0\\
2.01	0\\
3.01	0\\
4.01	0\\
5.01	0\\
6.01	0\\
7.01	0\\
8.01	0\\
9.01	0\\
10.01	0\\
11.01	0\\
12.01	0\\
13.01	0\\
14.01	0\\
15.01	0\\
16.01	0\\
17.01	0\\
18.01	0\\
19.01	0\\
20.01	0\\
21.01	0\\
22.01	0\\
23.01	0\\
24.01	0\\
25.01	0\\
26.01	0\\
27.01	0\\
28.01	0\\
29.01	0\\
30.01	0\\
31.01	0\\
32.01	0\\
33.01	0\\
34.01	0\\
35.01	0\\
36.01	0\\
37.01	0\\
38.01	0\\
39.01	0\\
40.01	0\\
41.01	0\\
42.01	0\\
43.01	0\\
44.01	0\\
45.01	0\\
46.01	0\\
47.01	0\\
48.01	0\\
49.01	0\\
50.01	0\\
51.01	0\\
52.01	0\\
53.01	0\\
54.01	0\\
55.01	0\\
56.01	0\\
57.01	0\\
58.01	0\\
59.01	0\\
60.01	0\\
61.01	0\\
62.01	0\\
63.01	0\\
64.01	0\\
65.01	0\\
66.01	0\\
67.01	0\\
68.01	0\\
69.01	0\\
70.01	0\\
71.01	0\\
72.01	0\\
73.01	0\\
74.01	0\\
75.01	0\\
76.01	0\\
77.01	0\\
78.01	0\\
79.01	0\\
80.01	0\\
81.01	0\\
82.01	0\\
83.01	0\\
84.01	0\\
85.01	0\\
86.01	0\\
87.01	0\\
88.01	0\\
89.01	0\\
90.01	0\\
91.01	0\\
92.01	0\\
93.01	0\\
94.01	0\\
95.01	0\\
96.01	0\\
97.01	0\\
98.01	0\\
99.01	0\\
100.01	0\\
101.01	0\\
102.01	0\\
103.01	0\\
104.01	0\\
105.01	0\\
106.01	0\\
107.01	0\\
108.01	0\\
109.01	0\\
110.01	0\\
111.01	0\\
112.01	0\\
113.01	0\\
114.01	0\\
115.01	0\\
116.01	0\\
117.01	0\\
118.01	0\\
119.01	0\\
120.01	0\\
121.01	0\\
122.01	0\\
123.01	0\\
124.01	0\\
125.01	0\\
126.01	0\\
127.01	0\\
128.01	0\\
129.01	0\\
130.01	0\\
131.01	0\\
132.01	0\\
133.01	0\\
134.01	0\\
135.01	0\\
136.01	0\\
137.01	0\\
138.01	0\\
139.01	0\\
140.01	0\\
141.01	0\\
142.01	0\\
143.01	0\\
144.01	0\\
145.01	0\\
146.01	0\\
147.01	0\\
148.01	0\\
149.01	0\\
150.01	0\\
151.01	0\\
152.01	0\\
153.01	0\\
154.01	0\\
155.01	0\\
156.01	0\\
157.01	0\\
158.01	0\\
159.01	0\\
160.01	0\\
161.01	0\\
162.01	0\\
163.01	0\\
164.01	0\\
165.01	0\\
166.01	0\\
167.01	0\\
168.01	0\\
169.01	0\\
170.01	0\\
171.01	0\\
172.01	0\\
173.01	0\\
174.01	0\\
175.01	0\\
176.01	0\\
177.01	0\\
178.01	0\\
179.01	0\\
180.01	0\\
181.01	0\\
182.01	0\\
183.01	0\\
184.01	0\\
185.01	0\\
186.01	0\\
187.01	0\\
188.01	0\\
189.01	0\\
190.01	0\\
191.01	0\\
192.01	0\\
193.01	0\\
194.01	0\\
195.01	0\\
196.01	0\\
197.01	0\\
198.01	0\\
199.01	0\\
200.01	0\\
201.01	0\\
202.01	0\\
203.01	0\\
204.01	0\\
205.01	0\\
206.01	0\\
207.01	0\\
208.01	0\\
209.01	0\\
210.01	0\\
211.01	0\\
212.01	0\\
213.01	0\\
214.01	0\\
215.01	0\\
216.01	0\\
217.01	0\\
218.01	0\\
219.01	0\\
220.01	0\\
221.01	0\\
222.01	0\\
223.01	0\\
224.01	0\\
225.01	0\\
226.01	0\\
227.01	0\\
228.01	0\\
229.01	0\\
230.01	0\\
231.01	0\\
232.01	0\\
233.01	0\\
234.01	0\\
235.01	0\\
236.01	0\\
237.01	0\\
238.01	0\\
239.01	0\\
240.01	0\\
241.01	0\\
242.01	0\\
243.01	0\\
244.01	0\\
245.01	0\\
246.01	0\\
247.01	0\\
248.01	0\\
249.01	0\\
250.01	0\\
251.01	0\\
252.01	0\\
253.01	0\\
254.01	0\\
255.01	0\\
256.01	0\\
257.01	0\\
258.01	0\\
259.01	0\\
260.01	0\\
261.01	0\\
262.01	0\\
263.01	0\\
264.01	0\\
265.01	0\\
266.01	0\\
267.01	0\\
268.01	0\\
269.01	0\\
270.01	0\\
271.01	0\\
272.01	0\\
273.01	0\\
274.01	0\\
275.01	0\\
276.01	0\\
277.01	0\\
278.01	0\\
279.01	0\\
280.01	0\\
281.01	0\\
282.01	0\\
283.01	0\\
284.01	0\\
285.01	0\\
286.01	0\\
287.01	0\\
288.01	0\\
289.01	0\\
290.01	0\\
291.01	0\\
292.01	0\\
293.01	0\\
294.01	0\\
295.01	0\\
296.01	0\\
297.01	0\\
298.01	0\\
299.01	0\\
300.01	0\\
301.01	0\\
302.01	0\\
303.01	0\\
304.01	0\\
305.01	0\\
306.01	0\\
307.01	0\\
308.01	0\\
309.01	0\\
310.01	0\\
311.01	0\\
312.01	0\\
313.01	0\\
314.01	0\\
315.01	0\\
316.01	0\\
317.01	0\\
318.01	0\\
319.01	0\\
320.01	0\\
321.01	0\\
322.01	0\\
323.01	0\\
324.01	0\\
325.01	0\\
326.01	0\\
327.01	0\\
328.01	0\\
329.01	0\\
330.01	0\\
331.01	0\\
332.01	0\\
333.01	0\\
334.01	0\\
335.01	0\\
336.01	0\\
337.01	0\\
338.01	0\\
339.01	0\\
340.01	0\\
341.01	0\\
342.01	0\\
343.01	0\\
344.01	0\\
345.01	0\\
346.01	0\\
347.01	0\\
348.01	0\\
349.01	0\\
350.01	0\\
351.01	0\\
352.01	0\\
353.01	0\\
354.01	0\\
355.01	0\\
356.01	0\\
357.01	0\\
358.01	0\\
359.01	0\\
360.01	0\\
361.01	0\\
362.01	0\\
363.01	0\\
364.01	0\\
365.01	0\\
366.01	0\\
367.01	0\\
368.01	0\\
369.01	0\\
370.01	0\\
371.01	0\\
372.01	0\\
373.01	0\\
374.01	0\\
375.01	0\\
376.01	0\\
377.01	0\\
378.01	0\\
379.01	0\\
380.01	0\\
381.01	0\\
382.01	0\\
383.01	0\\
384.01	0\\
385.01	0\\
386.01	0\\
387.01	0\\
388.01	0\\
389.01	0\\
390.01	0\\
391.01	0\\
392.01	0\\
393.01	0\\
394.01	0\\
395.01	0\\
396.01	0\\
397.01	0\\
398.01	0\\
399.01	0\\
400.01	0\\
401.01	0\\
402.01	0\\
403.01	0\\
404.01	0\\
405.01	0\\
406.01	0\\
407.01	0\\
408.01	0\\
409.01	0\\
410.01	0\\
411.01	0\\
412.01	0\\
413.01	0\\
414.01	0\\
415.01	0\\
416.01	0\\
417.01	0\\
418.01	0\\
419.01	0\\
420.01	0\\
421.01	0\\
422.01	0\\
423.01	0\\
424.01	0\\
425.01	0\\
426.01	0\\
427.01	0\\
428.01	0\\
429.01	0\\
430.01	0\\
431.01	0\\
432.01	0\\
433.01	0\\
434.01	0\\
435.01	0\\
436.01	0\\
437.01	0\\
438.01	0\\
439.01	0\\
440.01	0\\
441.01	0\\
442.01	0\\
443.01	0\\
444.01	0\\
445.01	0\\
446.01	0\\
447.01	0\\
448.01	0\\
449.01	0\\
450.01	0\\
451.01	0\\
452.01	0\\
453.01	0\\
454.01	0\\
455.01	0\\
456.01	0\\
457.01	0\\
458.01	0\\
459.01	0\\
460.01	0\\
461.01	0\\
462.01	0\\
463.01	0\\
464.01	0\\
465.01	0\\
466.01	0\\
467.01	0\\
468.01	0\\
469.01	0\\
470.01	0\\
471.01	0\\
472.01	0\\
473.01	0\\
474.01	0\\
475.01	0\\
476.01	0\\
477.01	0\\
478.01	0\\
479.01	0\\
480.01	0\\
481.01	0\\
482.01	0\\
483.01	0\\
484.01	0\\
485.01	0\\
486.01	0\\
487.01	0\\
488.01	0\\
489.01	0\\
490.01	0\\
491.01	0\\
492.01	0\\
493.01	0\\
494.01	0\\
495.01	0\\
496.01	0\\
497.01	0\\
498.01	0\\
499.01	0\\
500.01	0\\
501.01	0\\
502.01	0\\
503.01	0\\
504.01	0\\
505.01	0\\
506.01	0\\
507.01	0\\
508.01	0\\
509.01	0\\
510.01	0\\
511.01	0\\
512.01	0\\
513.01	0\\
514.01	0\\
515.01	0\\
516.01	0\\
517.01	0\\
518.01	0\\
519.01	0\\
520.01	0\\
521.01	0\\
522.01	0\\
523.01	0\\
524.01	0\\
525.01	0\\
526.01	0\\
527.01	0\\
528.01	0\\
529.01	0\\
530.01	0\\
531.01	0\\
532.01	0\\
533.01	0\\
534.01	0\\
535.01	0\\
536.01	0\\
537.01	0\\
538.01	0\\
539.01	0\\
540.01	0\\
541.01	0\\
542.01	0\\
543.01	0\\
544.01	0\\
545.01	0\\
546.01	0\\
547.01	0\\
548.01	0\\
549.01	0\\
550.01	0\\
551.01	0\\
552.01	0\\
553.01	0\\
554.01	0\\
555.01	0\\
556.01	0\\
557.01	0\\
558.01	0\\
559.01	0\\
560.01	0\\
561.01	0\\
562.01	0\\
563.01	0\\
564.01	0\\
565.01	0\\
566.01	0\\
567.01	0\\
568.01	0\\
569.01	0\\
570.01	0\\
571.01	0\\
572.01	0\\
573.01	0\\
574.01	0\\
575.01	0\\
576.01	0\\
577.01	0\\
578.01	0\\
579.01	0\\
580.01	0\\
581.01	0\\
582.01	0\\
583.01	0\\
584.01	0\\
585.01	0\\
586.01	0\\
587.01	0\\
588.01	0\\
589.01	0\\
590.01	0\\
591.01	0\\
592.01	0\\
593.01	0\\
594.01	0\\
595.01	0\\
596.01	0\\
597.01	0\\
598.01	3.69463863061463e-05\\
599.01	0.00379444900894968\\
599.02	0.00383270374370267\\
599.03	0.00387131777133739\\
599.04	0.00391029453419965\\
599.05	0.0039496375063856\\
599.06	0.003989350193997\\
599.07	0.00402943613539713\\
599.08	0.00406989890146739\\
599.09	0.00411074209586436\\
599.1	0.00415196935527735\\
599.11	0.00419358434968641\\
599.12	0.00423559078262063\\
599.13	0.00427799239141662\\
599.14	0.00432079294747728\\
599.15	0.00436399625653049\\
599.16	0.00440760615888791\\
599.17	0.00445162652970357\\
599.18	0.00449606127923222\\
599.19	0.00454091435308739\\
599.2	0.00458618973249894\\
599.21	0.00463189143457006\\
599.22	0.00467802351253352\\
599.23	0.00472459005600711\\
599.24	0.00477159519124805\\
599.25	0.00481904308140627\\
599.26	0.00486693792678842\\
599.27	0.00491528396512382\\
599.28	0.00496408547181374\\
599.29	0.00501334676017881\\
599.3	0.00506307218170441\\
599.31	0.00511326612628387\\
599.32	0.00516393302245909\\
599.33	0.00521507733765866\\
599.34	0.00526670357843294\\
599.35	0.00531881629068601\\
599.36	0.00537142005990418\\
599.37	0.00542451951138086\\
599.38	0.00547811931043734\\
599.39	0.00553222416263933\\
599.4	0.00558683881400888\\
599.41	0.00564196805123132\\
599.42	0.00569761670185686\\
599.43	0.00575378963449661\\
599.44	0.00581049175901236\\
599.45	0.00586772802670002\\
599.46	0.00592550342982103\\
599.47	0.00598382297386091\\
599.48	0.00604269170566519\\
599.49	0.00610211471358306\\
599.5	0.00616209712760199\\
599.51	0.00622264411947269\\
599.52	0.00628376090282387\\
599.53	0.00634545273326615\\
599.54	0.00640772490848447\\
599.55	0.00647058276831837\\
599.56	0.00653403169482929\\
599.57	0.00659807711235419\\
599.58	0.00666272448754477\\
599.59	0.00672797932939122\\
599.6	0.00679384718922988\\
599.61	0.0068603336607336\\
599.62	0.00692744437988406\\
599.63	0.00699518502492475\\
599.64	0.00706356131629375\\
599.65	0.007132579016535\\
599.66	0.00720224393018678\\
599.67	0.00727256190364638\\
599.68	0.00734353882500926\\
599.69	0.00741518062388145\\
599.7	0.00748749327116373\\
599.71	0.00756048277880581\\
599.72	0.00763415519952897\\
599.73	0.00770851662651528\\
599.74	0.00778357319306161\\
599.75	0.00785933107219639\\
599.76	0.00793579647625698\\
599.77	0.00801297565642555\\
599.78	0.00809087490222101\\
599.79	0.00816950054094457\\
599.8	0.00824885893707632\\
599.81	0.00832895649161999\\
599.82	0.00840979964139308\\
599.83	0.00849139485825914\\
599.84	0.00857374864829899\\
599.85	0.0086568675509174\\
599.86	0.00874075813788151\\
599.87	0.00882542701228713\\
599.88	0.00891088080744883\\
599.89	0.00899712618570936\\
599.9	0.00908416983716382\\
599.91	0.00917201847829369\\
599.92	0.00926067885050539\\
599.93	0.00935015771856803\\
599.94	0.00944046186894425\\
599.95	0.00953159810800815\\
599.96	0.00962357326014348\\
599.97	0.00971639416571523\\
599.98	0.00981006767890704\\
599.99	0.00990460066541651\\
600	0.01\\
};
\addplot [color=mycolor15,solid,forget plot]
  table[row sep=crcr]{%
0.01	0\\
1.01	0\\
2.01	0\\
3.01	0\\
4.01	0\\
5.01	0\\
6.01	0\\
7.01	0\\
8.01	0\\
9.01	0\\
10.01	0\\
11.01	0\\
12.01	0\\
13.01	0\\
14.01	0\\
15.01	0\\
16.01	0\\
17.01	0\\
18.01	0\\
19.01	0\\
20.01	0\\
21.01	0\\
22.01	0\\
23.01	0\\
24.01	0\\
25.01	0\\
26.01	0\\
27.01	0\\
28.01	0\\
29.01	0\\
30.01	0\\
31.01	0\\
32.01	0\\
33.01	0\\
34.01	0\\
35.01	0\\
36.01	0\\
37.01	0\\
38.01	0\\
39.01	0\\
40.01	0\\
41.01	0\\
42.01	0\\
43.01	0\\
44.01	0\\
45.01	0\\
46.01	0\\
47.01	0\\
48.01	0\\
49.01	0\\
50.01	0\\
51.01	0\\
52.01	0\\
53.01	0\\
54.01	0\\
55.01	0\\
56.01	0\\
57.01	0\\
58.01	0\\
59.01	0\\
60.01	0\\
61.01	0\\
62.01	0\\
63.01	0\\
64.01	0\\
65.01	0\\
66.01	0\\
67.01	0\\
68.01	0\\
69.01	0\\
70.01	0\\
71.01	0\\
72.01	0\\
73.01	0\\
74.01	0\\
75.01	0\\
76.01	0\\
77.01	0\\
78.01	0\\
79.01	0\\
80.01	0\\
81.01	0\\
82.01	0\\
83.01	0\\
84.01	0\\
85.01	0\\
86.01	0\\
87.01	0\\
88.01	0\\
89.01	0\\
90.01	0\\
91.01	0\\
92.01	0\\
93.01	0\\
94.01	0\\
95.01	0\\
96.01	0\\
97.01	0\\
98.01	0\\
99.01	0\\
100.01	0\\
101.01	0\\
102.01	0\\
103.01	0\\
104.01	0\\
105.01	0\\
106.01	0\\
107.01	0\\
108.01	0\\
109.01	0\\
110.01	0\\
111.01	0\\
112.01	0\\
113.01	0\\
114.01	0\\
115.01	0\\
116.01	0\\
117.01	0\\
118.01	0\\
119.01	0\\
120.01	0\\
121.01	0\\
122.01	0\\
123.01	0\\
124.01	0\\
125.01	0\\
126.01	0\\
127.01	0\\
128.01	0\\
129.01	0\\
130.01	0\\
131.01	0\\
132.01	0\\
133.01	0\\
134.01	0\\
135.01	0\\
136.01	0\\
137.01	0\\
138.01	0\\
139.01	0\\
140.01	0\\
141.01	0\\
142.01	0\\
143.01	0\\
144.01	0\\
145.01	0\\
146.01	0\\
147.01	0\\
148.01	0\\
149.01	0\\
150.01	0\\
151.01	0\\
152.01	0\\
153.01	0\\
154.01	0\\
155.01	0\\
156.01	0\\
157.01	0\\
158.01	0\\
159.01	0\\
160.01	0\\
161.01	0\\
162.01	0\\
163.01	0\\
164.01	0\\
165.01	0\\
166.01	0\\
167.01	0\\
168.01	0\\
169.01	0\\
170.01	0\\
171.01	0\\
172.01	0\\
173.01	0\\
174.01	0\\
175.01	0\\
176.01	0\\
177.01	0\\
178.01	0\\
179.01	0\\
180.01	0\\
181.01	0\\
182.01	0\\
183.01	0\\
184.01	0\\
185.01	0\\
186.01	0\\
187.01	0\\
188.01	0\\
189.01	0\\
190.01	0\\
191.01	0\\
192.01	0\\
193.01	0\\
194.01	0\\
195.01	0\\
196.01	0\\
197.01	0\\
198.01	0\\
199.01	0\\
200.01	0\\
201.01	0\\
202.01	0\\
203.01	0\\
204.01	0\\
205.01	0\\
206.01	0\\
207.01	0\\
208.01	0\\
209.01	0\\
210.01	0\\
211.01	0\\
212.01	0\\
213.01	0\\
214.01	0\\
215.01	0\\
216.01	0\\
217.01	0\\
218.01	0\\
219.01	0\\
220.01	0\\
221.01	0\\
222.01	0\\
223.01	0\\
224.01	0\\
225.01	0\\
226.01	0\\
227.01	0\\
228.01	0\\
229.01	0\\
230.01	0\\
231.01	0\\
232.01	0\\
233.01	0\\
234.01	0\\
235.01	0\\
236.01	0\\
237.01	0\\
238.01	0\\
239.01	0\\
240.01	0\\
241.01	0\\
242.01	0\\
243.01	0\\
244.01	0\\
245.01	0\\
246.01	0\\
247.01	0\\
248.01	0\\
249.01	0\\
250.01	0\\
251.01	0\\
252.01	0\\
253.01	0\\
254.01	0\\
255.01	0\\
256.01	0\\
257.01	0\\
258.01	0\\
259.01	0\\
260.01	0\\
261.01	0\\
262.01	0\\
263.01	0\\
264.01	0\\
265.01	0\\
266.01	0\\
267.01	0\\
268.01	0\\
269.01	0\\
270.01	0\\
271.01	0\\
272.01	0\\
273.01	0\\
274.01	0\\
275.01	0\\
276.01	0\\
277.01	0\\
278.01	0\\
279.01	0\\
280.01	0\\
281.01	0\\
282.01	0\\
283.01	0\\
284.01	0\\
285.01	0\\
286.01	0\\
287.01	0\\
288.01	0\\
289.01	0\\
290.01	0\\
291.01	0\\
292.01	0\\
293.01	0\\
294.01	0\\
295.01	0\\
296.01	0\\
297.01	0\\
298.01	0\\
299.01	0\\
300.01	0\\
301.01	0\\
302.01	0\\
303.01	0\\
304.01	0\\
305.01	0\\
306.01	0\\
307.01	0\\
308.01	0\\
309.01	0\\
310.01	0\\
311.01	0\\
312.01	0\\
313.01	0\\
314.01	0\\
315.01	0\\
316.01	0\\
317.01	0\\
318.01	0\\
319.01	0\\
320.01	0\\
321.01	0\\
322.01	0\\
323.01	0\\
324.01	0\\
325.01	0\\
326.01	0\\
327.01	0\\
328.01	0\\
329.01	0\\
330.01	0\\
331.01	0\\
332.01	0\\
333.01	0\\
334.01	0\\
335.01	0\\
336.01	0\\
337.01	0\\
338.01	0\\
339.01	0\\
340.01	0\\
341.01	0\\
342.01	0\\
343.01	0\\
344.01	0\\
345.01	0\\
346.01	0\\
347.01	0\\
348.01	0\\
349.01	0\\
350.01	0\\
351.01	0\\
352.01	0\\
353.01	0\\
354.01	0\\
355.01	0\\
356.01	0\\
357.01	0\\
358.01	0\\
359.01	0\\
360.01	0\\
361.01	0\\
362.01	0\\
363.01	0\\
364.01	0\\
365.01	0\\
366.01	0\\
367.01	0\\
368.01	0\\
369.01	0\\
370.01	0\\
371.01	0\\
372.01	0\\
373.01	0\\
374.01	0\\
375.01	0\\
376.01	0\\
377.01	0\\
378.01	0\\
379.01	0\\
380.01	0\\
381.01	0\\
382.01	0\\
383.01	0\\
384.01	0\\
385.01	0\\
386.01	0\\
387.01	0\\
388.01	0\\
389.01	0\\
390.01	0\\
391.01	0\\
392.01	0\\
393.01	0\\
394.01	0\\
395.01	0\\
396.01	0\\
397.01	0\\
398.01	0\\
399.01	0\\
400.01	0\\
401.01	0\\
402.01	0\\
403.01	0\\
404.01	0\\
405.01	0\\
406.01	0\\
407.01	0\\
408.01	0\\
409.01	0\\
410.01	0\\
411.01	0\\
412.01	0\\
413.01	0\\
414.01	0\\
415.01	0\\
416.01	0\\
417.01	0\\
418.01	0\\
419.01	0\\
420.01	0\\
421.01	0\\
422.01	0\\
423.01	0\\
424.01	0\\
425.01	0\\
426.01	0\\
427.01	0\\
428.01	0\\
429.01	0\\
430.01	0\\
431.01	0\\
432.01	0\\
433.01	0\\
434.01	0\\
435.01	0\\
436.01	0\\
437.01	0\\
438.01	0\\
439.01	0\\
440.01	0\\
441.01	0\\
442.01	0\\
443.01	0\\
444.01	0\\
445.01	0\\
446.01	0\\
447.01	0\\
448.01	0\\
449.01	0\\
450.01	0\\
451.01	0\\
452.01	0\\
453.01	0\\
454.01	0\\
455.01	0\\
456.01	0\\
457.01	0\\
458.01	0\\
459.01	0\\
460.01	0\\
461.01	0\\
462.01	0\\
463.01	0\\
464.01	0\\
465.01	0\\
466.01	0\\
467.01	0\\
468.01	0\\
469.01	0\\
470.01	0\\
471.01	0\\
472.01	0\\
473.01	0\\
474.01	0\\
475.01	0\\
476.01	0\\
477.01	0\\
478.01	0\\
479.01	0\\
480.01	0\\
481.01	0\\
482.01	0\\
483.01	0\\
484.01	0\\
485.01	0\\
486.01	0\\
487.01	0\\
488.01	0\\
489.01	0\\
490.01	0\\
491.01	0\\
492.01	0\\
493.01	0\\
494.01	0\\
495.01	0\\
496.01	0\\
497.01	0\\
498.01	0\\
499.01	0\\
500.01	0\\
501.01	0\\
502.01	0\\
503.01	0\\
504.01	0\\
505.01	0\\
506.01	0\\
507.01	0\\
508.01	0\\
509.01	0\\
510.01	0\\
511.01	0\\
512.01	0\\
513.01	0\\
514.01	0\\
515.01	0\\
516.01	0\\
517.01	0\\
518.01	0\\
519.01	0\\
520.01	0\\
521.01	0\\
522.01	0\\
523.01	0\\
524.01	0\\
525.01	0\\
526.01	0\\
527.01	0\\
528.01	0\\
529.01	0\\
530.01	0\\
531.01	0\\
532.01	0\\
533.01	0\\
534.01	0\\
535.01	0\\
536.01	0\\
537.01	0\\
538.01	0\\
539.01	0\\
540.01	0\\
541.01	0\\
542.01	0\\
543.01	0\\
544.01	0\\
545.01	0\\
546.01	0\\
547.01	0\\
548.01	0\\
549.01	0\\
550.01	0\\
551.01	0\\
552.01	0\\
553.01	0\\
554.01	0\\
555.01	0\\
556.01	0\\
557.01	0\\
558.01	0\\
559.01	0\\
560.01	0\\
561.01	0\\
562.01	0\\
563.01	0\\
564.01	0\\
565.01	0\\
566.01	0\\
567.01	0\\
568.01	0\\
569.01	0\\
570.01	0\\
571.01	0\\
572.01	0\\
573.01	0\\
574.01	0\\
575.01	0\\
576.01	0\\
577.01	0\\
578.01	0\\
579.01	0\\
580.01	0\\
581.01	0\\
582.01	0\\
583.01	0\\
584.01	0\\
585.01	0\\
586.01	0\\
587.01	0\\
588.01	0\\
589.01	0\\
590.01	0\\
591.01	0\\
592.01	0\\
593.01	0\\
594.01	0\\
595.01	0\\
596.01	0\\
597.01	0\\
598.01	0.00141368560759025\\
599.01	0.00385460611783226\\
599.02	0.00389217668082419\\
599.03	0.00393010478508591\\
599.04	0.00396839388118488\\
599.05	0.00400704745291263\\
599.06	0.00404606901760485\\
599.07	0.00408546212646469\\
599.08	0.00412523036488917\\
599.09	0.00416537735279883\\
599.1	0.00420590674497062\\
599.11	0.00424682223137403\\
599.12	0.0042881275375106\\
599.13	0.00432982642475674\\
599.14	0.0043719226907099\\
599.15	0.0044144201695383\\
599.16	0.00445732273233393\\
599.17	0.00450063428746929\\
599.18	0.00454435878095746\\
599.19	0.00458850019681592\\
599.2	0.00463306255743389\\
599.21	0.00467804992394345\\
599.22	0.0047234663965943\\
599.23	0.00476931611513231\\
599.24	0.00481560325918193\\
599.25	0.00486233204863238\\
599.26	0.00490950673609608\\
599.27	0.00495713160496067\\
599.28	0.0050052109796917\\
599.29	0.00505374922622871\\
599.3	0.00510275075238528\\
599.31	0.00515222000825318\\
599.32	0.00520216148661058\\
599.33	0.00525257972333446\\
599.34	0.00530347929781725\\
599.35	0.00535486483338777\\
599.36	0.00540674099773647\\
599.37	0.00545911250334512\\
599.38	0.00551198410792095\\
599.39	0.00556536061483532\\
599.4	0.00561924687356703\\
599.41	0.00567364778015027\\
599.42	0.00572856827762734\\
599.43	0.00578401335650615\\
599.44	0.00583998805522266\\
599.45	0.00589649746060833\\
599.46	0.00595354670836325\\
599.47	0.00601114098356581\\
599.48	0.00606928552115735\\
599.49	0.00612798560643227\\
599.5	0.00618724657553358\\
599.51	0.00624707381595411\\
599.52	0.00630747276704345\\
599.53	0.00636844892052068\\
599.54	0.00643000782099308\\
599.55	0.00649215506648088\\
599.56	0.00655489630894826\\
599.57	0.00661823725484062\\
599.58	0.00668218366562836\\
599.59	0.00674674135835722\\
599.6	0.00681191620620549\\
599.61	0.006877714139048\\
599.62	0.00694414114402729\\
599.63	0.00701120326613206\\
599.64	0.00707890660878295\\
599.65	0.00714725733442608\\
599.66	0.00721626166513437\\
599.67	0.0072859258832169\\
599.68	0.00735625633183653\\
599.69	0.00742725941563611\\
599.7	0.00749894160137328\\
599.71	0.00757130941856446\\
599.72	0.00764436946013798\\
599.73	0.00771812838309687\\
599.74	0.00779259290919147\\
599.75	0.00786776982560227\\
599.76	0.0079436659856333\\
599.77	0.00802028830941633\\
599.78	0.00809764378462653\\
599.79	0.00817573946720956\\
599.8	0.00825458248212104\\
599.81	0.00833418002407833\\
599.82	0.00841453935832557\\
599.83	0.00849566782141209\\
599.84	0.00857757282198503\\
599.85	0.00866026184159656\\
599.86	0.00874374243552642\\
599.87	0.00882802223362036\\
599.88	0.00891310894114515\\
599.89	0.00899901033966106\\
599.9	0.0090857342879123\\
599.91	0.00917328872273657\\
599.92	0.00926168165999438\\
599.93	0.00935092119551921\\
599.94	0.0094410155060894\\
599.95	0.009531972850423\\
599.96	0.00962380157019665\\
599.97	0.00971651009108966\\
599.98	0.0098101069238547\\
599.99	0.00990460066541651\\
600	0.01\\
};
\addplot [color=mycolor16,solid,forget plot]
  table[row sep=crcr]{%
0.01	0\\
1.01	0\\
2.01	0\\
3.01	0\\
4.01	0\\
5.01	0\\
6.01	0\\
7.01	0\\
8.01	0\\
9.01	0\\
10.01	0\\
11.01	0\\
12.01	0\\
13.01	0\\
14.01	0\\
15.01	0\\
16.01	0\\
17.01	0\\
18.01	0\\
19.01	0\\
20.01	0\\
21.01	0\\
22.01	0\\
23.01	0\\
24.01	0\\
25.01	0\\
26.01	0\\
27.01	0\\
28.01	0\\
29.01	0\\
30.01	0\\
31.01	0\\
32.01	0\\
33.01	0\\
34.01	0\\
35.01	0\\
36.01	0\\
37.01	0\\
38.01	0\\
39.01	0\\
40.01	0\\
41.01	0\\
42.01	0\\
43.01	0\\
44.01	0\\
45.01	0\\
46.01	0\\
47.01	0\\
48.01	0\\
49.01	0\\
50.01	0\\
51.01	0\\
52.01	0\\
53.01	0\\
54.01	0\\
55.01	0\\
56.01	0\\
57.01	0\\
58.01	0\\
59.01	0\\
60.01	0\\
61.01	0\\
62.01	0\\
63.01	0\\
64.01	0\\
65.01	0\\
66.01	0\\
67.01	0\\
68.01	0\\
69.01	0\\
70.01	0\\
71.01	0\\
72.01	0\\
73.01	0\\
74.01	0\\
75.01	0\\
76.01	0\\
77.01	0\\
78.01	0\\
79.01	0\\
80.01	0\\
81.01	0\\
82.01	0\\
83.01	0\\
84.01	0\\
85.01	0\\
86.01	0\\
87.01	0\\
88.01	0\\
89.01	0\\
90.01	0\\
91.01	0\\
92.01	0\\
93.01	0\\
94.01	0\\
95.01	0\\
96.01	0\\
97.01	0\\
98.01	0\\
99.01	0\\
100.01	0\\
101.01	0\\
102.01	0\\
103.01	0\\
104.01	0\\
105.01	0\\
106.01	0\\
107.01	0\\
108.01	0\\
109.01	0\\
110.01	0\\
111.01	0\\
112.01	0\\
113.01	0\\
114.01	0\\
115.01	0\\
116.01	0\\
117.01	0\\
118.01	0\\
119.01	0\\
120.01	0\\
121.01	0\\
122.01	0\\
123.01	0\\
124.01	0\\
125.01	0\\
126.01	0\\
127.01	0\\
128.01	0\\
129.01	0\\
130.01	0\\
131.01	0\\
132.01	0\\
133.01	0\\
134.01	0\\
135.01	0\\
136.01	0\\
137.01	0\\
138.01	0\\
139.01	0\\
140.01	0\\
141.01	0\\
142.01	0\\
143.01	0\\
144.01	0\\
145.01	0\\
146.01	0\\
147.01	0\\
148.01	0\\
149.01	0\\
150.01	0\\
151.01	0\\
152.01	0\\
153.01	0\\
154.01	0\\
155.01	0\\
156.01	0\\
157.01	0\\
158.01	0\\
159.01	0\\
160.01	0\\
161.01	0\\
162.01	0\\
163.01	0\\
164.01	0\\
165.01	0\\
166.01	0\\
167.01	0\\
168.01	0\\
169.01	0\\
170.01	0\\
171.01	0\\
172.01	0\\
173.01	0\\
174.01	0\\
175.01	0\\
176.01	0\\
177.01	0\\
178.01	0\\
179.01	0\\
180.01	0\\
181.01	0\\
182.01	0\\
183.01	0\\
184.01	0\\
185.01	0\\
186.01	0\\
187.01	0\\
188.01	0\\
189.01	0\\
190.01	0\\
191.01	0\\
192.01	0\\
193.01	0\\
194.01	0\\
195.01	0\\
196.01	0\\
197.01	0\\
198.01	0\\
199.01	0\\
200.01	0\\
201.01	0\\
202.01	0\\
203.01	0\\
204.01	0\\
205.01	0\\
206.01	0\\
207.01	0\\
208.01	0\\
209.01	0\\
210.01	0\\
211.01	0\\
212.01	0\\
213.01	0\\
214.01	0\\
215.01	0\\
216.01	0\\
217.01	0\\
218.01	0\\
219.01	0\\
220.01	0\\
221.01	0\\
222.01	0\\
223.01	0\\
224.01	0\\
225.01	0\\
226.01	0\\
227.01	0\\
228.01	0\\
229.01	0\\
230.01	0\\
231.01	0\\
232.01	0\\
233.01	0\\
234.01	0\\
235.01	0\\
236.01	0\\
237.01	0\\
238.01	0\\
239.01	0\\
240.01	0\\
241.01	0\\
242.01	0\\
243.01	0\\
244.01	0\\
245.01	0\\
246.01	0\\
247.01	0\\
248.01	0\\
249.01	0\\
250.01	0\\
251.01	0\\
252.01	0\\
253.01	0\\
254.01	0\\
255.01	0\\
256.01	0\\
257.01	0\\
258.01	0\\
259.01	0\\
260.01	0\\
261.01	0\\
262.01	0\\
263.01	0\\
264.01	0\\
265.01	0\\
266.01	0\\
267.01	0\\
268.01	0\\
269.01	0\\
270.01	0\\
271.01	0\\
272.01	0\\
273.01	0\\
274.01	0\\
275.01	0\\
276.01	0\\
277.01	0\\
278.01	0\\
279.01	0\\
280.01	0\\
281.01	0\\
282.01	0\\
283.01	0\\
284.01	0\\
285.01	0\\
286.01	0\\
287.01	0\\
288.01	0\\
289.01	0\\
290.01	0\\
291.01	0\\
292.01	0\\
293.01	0\\
294.01	0\\
295.01	0\\
296.01	0\\
297.01	0\\
298.01	0\\
299.01	0\\
300.01	0\\
301.01	0\\
302.01	0\\
303.01	0\\
304.01	0\\
305.01	0\\
306.01	0\\
307.01	0\\
308.01	0\\
309.01	0\\
310.01	0\\
311.01	0\\
312.01	0\\
313.01	0\\
314.01	0\\
315.01	0\\
316.01	0\\
317.01	0\\
318.01	0\\
319.01	0\\
320.01	0\\
321.01	0\\
322.01	0\\
323.01	0\\
324.01	0\\
325.01	0\\
326.01	0\\
327.01	0\\
328.01	0\\
329.01	0\\
330.01	0\\
331.01	0\\
332.01	0\\
333.01	0\\
334.01	0\\
335.01	0\\
336.01	0\\
337.01	0\\
338.01	0\\
339.01	0\\
340.01	0\\
341.01	0\\
342.01	0\\
343.01	0\\
344.01	0\\
345.01	0\\
346.01	0\\
347.01	0\\
348.01	0\\
349.01	0\\
350.01	0\\
351.01	0\\
352.01	0\\
353.01	0\\
354.01	0\\
355.01	0\\
356.01	0\\
357.01	0\\
358.01	0\\
359.01	0\\
360.01	0\\
361.01	0\\
362.01	0\\
363.01	0\\
364.01	0\\
365.01	0\\
366.01	0\\
367.01	0\\
368.01	0\\
369.01	0\\
370.01	0\\
371.01	0\\
372.01	0\\
373.01	0\\
374.01	0\\
375.01	0\\
376.01	0\\
377.01	0\\
378.01	0\\
379.01	0\\
380.01	0\\
381.01	0\\
382.01	0\\
383.01	0\\
384.01	0\\
385.01	0\\
386.01	0\\
387.01	0\\
388.01	0\\
389.01	0\\
390.01	0\\
391.01	0\\
392.01	0\\
393.01	0\\
394.01	0\\
395.01	0\\
396.01	0\\
397.01	0\\
398.01	0\\
399.01	0\\
400.01	0\\
401.01	0\\
402.01	0\\
403.01	0\\
404.01	0\\
405.01	0\\
406.01	0\\
407.01	0\\
408.01	0\\
409.01	0\\
410.01	0\\
411.01	0\\
412.01	0\\
413.01	0\\
414.01	0\\
415.01	0\\
416.01	0\\
417.01	0\\
418.01	0\\
419.01	0\\
420.01	0\\
421.01	0\\
422.01	0\\
423.01	0\\
424.01	0\\
425.01	0\\
426.01	0\\
427.01	0\\
428.01	0\\
429.01	0\\
430.01	0\\
431.01	0\\
432.01	0\\
433.01	0\\
434.01	0\\
435.01	0\\
436.01	0\\
437.01	0\\
438.01	0\\
439.01	0\\
440.01	0\\
441.01	0\\
442.01	0\\
443.01	0\\
444.01	0\\
445.01	0\\
446.01	0\\
447.01	0\\
448.01	0\\
449.01	0\\
450.01	0\\
451.01	0\\
452.01	0\\
453.01	0\\
454.01	0\\
455.01	0\\
456.01	0\\
457.01	0\\
458.01	0\\
459.01	0\\
460.01	0\\
461.01	0\\
462.01	0\\
463.01	0\\
464.01	0\\
465.01	0\\
466.01	0\\
467.01	0\\
468.01	0\\
469.01	0\\
470.01	0\\
471.01	0\\
472.01	0\\
473.01	0\\
474.01	0\\
475.01	0\\
476.01	0\\
477.01	0\\
478.01	0\\
479.01	0\\
480.01	0\\
481.01	0\\
482.01	0\\
483.01	0\\
484.01	0\\
485.01	0\\
486.01	0\\
487.01	0\\
488.01	0\\
489.01	0\\
490.01	0\\
491.01	0\\
492.01	0\\
493.01	0\\
494.01	0\\
495.01	0\\
496.01	0\\
497.01	0\\
498.01	0\\
499.01	0\\
500.01	0\\
501.01	0\\
502.01	0\\
503.01	0\\
504.01	0\\
505.01	0\\
506.01	0\\
507.01	0\\
508.01	0\\
509.01	0\\
510.01	0\\
511.01	0\\
512.01	0\\
513.01	0\\
514.01	0\\
515.01	0\\
516.01	0\\
517.01	0\\
518.01	0\\
519.01	0\\
520.01	0\\
521.01	0\\
522.01	0\\
523.01	0\\
524.01	0\\
525.01	0\\
526.01	0\\
527.01	0\\
528.01	0\\
529.01	0\\
530.01	0\\
531.01	0\\
532.01	0\\
533.01	0\\
534.01	0\\
535.01	0\\
536.01	0\\
537.01	0\\
538.01	0\\
539.01	0\\
540.01	0\\
541.01	0\\
542.01	0\\
543.01	0\\
544.01	0\\
545.01	0\\
546.01	0\\
547.01	0\\
548.01	0\\
549.01	0\\
550.01	0\\
551.01	0\\
552.01	0\\
553.01	0\\
554.01	0\\
555.01	0\\
556.01	0\\
557.01	0\\
558.01	0\\
559.01	0\\
560.01	0\\
561.01	0\\
562.01	0\\
563.01	0\\
564.01	0\\
565.01	0\\
566.01	0\\
567.01	0\\
568.01	0\\
569.01	0\\
570.01	0\\
571.01	0\\
572.01	0\\
573.01	0\\
574.01	0\\
575.01	0\\
576.01	0\\
577.01	0\\
578.01	0\\
579.01	0\\
580.01	0\\
581.01	0\\
582.01	0\\
583.01	0\\
584.01	0\\
585.01	0\\
586.01	0\\
587.01	0\\
588.01	0\\
589.01	0\\
590.01	0\\
591.01	0\\
592.01	0\\
593.01	0\\
594.01	0\\
595.01	0\\
596.01	0\\
597.01	0\\
598.01	0.0014374037248278\\
599.01	0.00385656036521221\\
599.02	0.00389409724555795\\
599.03	0.00393199176826388\\
599.04	0.00397024738727775\\
599.05	0.00400886758988391\\
599.06	0.00404785589702493\\
599.07	0.0040872158636263\\
599.08	0.00412695107892428\\
599.09	0.00416706516679696\\
599.1	0.00420756178609841\\
599.11	0.00424844463099618\\
599.12	0.00428971743131196\\
599.13	0.00433138395286554\\
599.14	0.00437344799782218\\
599.15	0.00441591340504319\\
599.16	0.00445878405044\\
599.17	0.00450206384733162\\
599.18	0.0045457567468055\\
599.19	0.00458986673808198\\
599.2	0.00463439784888212\\
599.21	0.00467935414579919\\
599.22	0.00472473973467372\\
599.23	0.00477055876097212\\
599.24	0.00481681541016898\\
599.25	0.0048635139081331\\
599.26	0.00491065851083179\\
599.27	0.00495825350811072\\
599.28	0.00500630323104035\\
599.29	0.00505481205231203\\
599.3	0.00510378438663774\\
599.31	0.00515322469115379\\
599.32	0.0052031374658283\\
599.33	0.00525352725387272\\
599.34	0.00530439864215714\\
599.35	0.00535575626162978\\
599.36	0.00540760478774036\\
599.37	0.00545994894086756\\
599.38	0.00551279348675071\\
599.39	0.00556614323692543\\
599.4	0.00562000304916365\\
599.41	0.00567437782791768\\
599.42	0.00572927252476872\\
599.43	0.00578469213887949\\
599.44	0.00584064171745141\\
599.45	0.00589712635618597\\
599.46	0.00595415119975069\\
599.47	0.00601172144224942\\
599.48	0.00606984232769716\\
599.49	0.00612851915049951\\
599.5	0.00618775725593663\\
599.51	0.00624756204065183\\
599.52	0.00630793895314495\\
599.53	0.00636889349427038\\
599.54	0.00643043121773979\\
599.55	0.00649255773062987\\
599.56	0.00655527869389471\\
599.57	0.00661859982288322\\
599.58	0.00668252688786147\\
599.59	0.00674706571453995\\
599.6	0.00681222218460597\\
599.61	0.00687800223626108\\
599.62	0.00694441186476363\\
599.63	0.00701145712297651\\
599.64	0.00707914412192007\\
599.65	0.00714747903133038\\
599.66	0.00721646808022274\\
599.67	0.00728611755746052\\
599.68	0.00735643381232947\\
599.69	0.00742742325511746\\
599.7	0.00749909235769963\\
599.71	0.00757144765412921\\
599.72	0.00764449574123379\\
599.73	0.00771824327921728\\
599.74	0.00779269699226756\\
599.75	0.00786786366916977\\
599.76	0.00794375016392541\\
599.77	0.0080203633963772\\
599.78	0.00809771035283981\\
599.79	0.00817579808673647\\
599.8	0.00825463371924141\\
599.81	0.00833422443992839\\
599.82	0.00841457750742512\\
599.83	0.00849570025007373\\
599.84	0.00857760006659731\\
599.85	0.00866028442677256\\
599.86	0.0087437608721085\\
599.87	0.00882803701653141\\
599.88	0.00891312054707585\\
599.89	0.00899901922458194\\
599.9	0.00908574088439886\\
599.91	0.00917329343709451\\
599.92	0.00926168486917145\\
599.93	0.00935092324378912\\
599.94	0.00944101670149225\\
599.95	0.00953197346094551\\
599.96	0.00962380181967443\\
599.97	0.00971651015481255\\
599.98	0.0098101069238547\\
599.99	0.00990460066541651\\
600	0.01\\
};
\addplot [color=mycolor17,solid,forget plot]
  table[row sep=crcr]{%
0.01	0\\
1.01	0\\
2.01	0\\
3.01	0\\
4.01	0\\
5.01	0\\
6.01	0\\
7.01	0\\
8.01	0\\
9.01	0\\
10.01	0\\
11.01	0\\
12.01	0\\
13.01	0\\
14.01	0\\
15.01	0\\
16.01	0\\
17.01	0\\
18.01	0\\
19.01	0\\
20.01	0\\
21.01	0\\
22.01	0\\
23.01	0\\
24.01	0\\
25.01	0\\
26.01	0\\
27.01	0\\
28.01	0\\
29.01	0\\
30.01	0\\
31.01	0\\
32.01	0\\
33.01	0\\
34.01	0\\
35.01	0\\
36.01	0\\
37.01	0\\
38.01	0\\
39.01	0\\
40.01	0\\
41.01	0\\
42.01	0\\
43.01	0\\
44.01	0\\
45.01	0\\
46.01	0\\
47.01	0\\
48.01	0\\
49.01	0\\
50.01	0\\
51.01	0\\
52.01	0\\
53.01	0\\
54.01	0\\
55.01	0\\
56.01	0\\
57.01	0\\
58.01	0\\
59.01	0\\
60.01	0\\
61.01	0\\
62.01	0\\
63.01	0\\
64.01	0\\
65.01	0\\
66.01	0\\
67.01	0\\
68.01	0\\
69.01	0\\
70.01	0\\
71.01	0\\
72.01	0\\
73.01	0\\
74.01	0\\
75.01	0\\
76.01	0\\
77.01	0\\
78.01	0\\
79.01	0\\
80.01	0\\
81.01	0\\
82.01	0\\
83.01	0\\
84.01	0\\
85.01	0\\
86.01	0\\
87.01	0\\
88.01	0\\
89.01	0\\
90.01	0\\
91.01	0\\
92.01	0\\
93.01	0\\
94.01	0\\
95.01	0\\
96.01	0\\
97.01	0\\
98.01	0\\
99.01	0\\
100.01	0\\
101.01	0\\
102.01	0\\
103.01	0\\
104.01	0\\
105.01	0\\
106.01	0\\
107.01	0\\
108.01	0\\
109.01	0\\
110.01	0\\
111.01	0\\
112.01	0\\
113.01	0\\
114.01	0\\
115.01	0\\
116.01	0\\
117.01	0\\
118.01	0\\
119.01	0\\
120.01	0\\
121.01	0\\
122.01	0\\
123.01	0\\
124.01	0\\
125.01	0\\
126.01	0\\
127.01	0\\
128.01	0\\
129.01	0\\
130.01	0\\
131.01	0\\
132.01	0\\
133.01	0\\
134.01	0\\
135.01	0\\
136.01	0\\
137.01	0\\
138.01	0\\
139.01	0\\
140.01	0\\
141.01	0\\
142.01	0\\
143.01	0\\
144.01	0\\
145.01	0\\
146.01	0\\
147.01	0\\
148.01	0\\
149.01	0\\
150.01	0\\
151.01	0\\
152.01	0\\
153.01	0\\
154.01	0\\
155.01	0\\
156.01	0\\
157.01	0\\
158.01	0\\
159.01	0\\
160.01	0\\
161.01	0\\
162.01	0\\
163.01	0\\
164.01	0\\
165.01	0\\
166.01	0\\
167.01	0\\
168.01	0\\
169.01	0\\
170.01	0\\
171.01	0\\
172.01	0\\
173.01	0\\
174.01	0\\
175.01	0\\
176.01	0\\
177.01	0\\
178.01	0\\
179.01	0\\
180.01	0\\
181.01	0\\
182.01	0\\
183.01	0\\
184.01	0\\
185.01	0\\
186.01	0\\
187.01	0\\
188.01	0\\
189.01	0\\
190.01	0\\
191.01	0\\
192.01	0\\
193.01	0\\
194.01	0\\
195.01	0\\
196.01	0\\
197.01	0\\
198.01	0\\
199.01	0\\
200.01	0\\
201.01	0\\
202.01	0\\
203.01	0\\
204.01	0\\
205.01	0\\
206.01	0\\
207.01	0\\
208.01	0\\
209.01	0\\
210.01	0\\
211.01	0\\
212.01	0\\
213.01	0\\
214.01	0\\
215.01	0\\
216.01	0\\
217.01	0\\
218.01	0\\
219.01	0\\
220.01	0\\
221.01	0\\
222.01	0\\
223.01	0\\
224.01	0\\
225.01	0\\
226.01	0\\
227.01	0\\
228.01	0\\
229.01	0\\
230.01	0\\
231.01	0\\
232.01	0\\
233.01	0\\
234.01	0\\
235.01	0\\
236.01	0\\
237.01	0\\
238.01	0\\
239.01	0\\
240.01	0\\
241.01	0\\
242.01	0\\
243.01	0\\
244.01	0\\
245.01	0\\
246.01	0\\
247.01	0\\
248.01	0\\
249.01	0\\
250.01	0\\
251.01	0\\
252.01	0\\
253.01	0\\
254.01	0\\
255.01	0\\
256.01	0\\
257.01	0\\
258.01	0\\
259.01	0\\
260.01	0\\
261.01	0\\
262.01	0\\
263.01	0\\
264.01	0\\
265.01	0\\
266.01	0\\
267.01	0\\
268.01	0\\
269.01	0\\
270.01	0\\
271.01	0\\
272.01	0\\
273.01	0\\
274.01	0\\
275.01	0\\
276.01	0\\
277.01	0\\
278.01	0\\
279.01	0\\
280.01	0\\
281.01	0\\
282.01	0\\
283.01	0\\
284.01	0\\
285.01	0\\
286.01	0\\
287.01	0\\
288.01	0\\
289.01	0\\
290.01	0\\
291.01	0\\
292.01	0\\
293.01	0\\
294.01	0\\
295.01	0\\
296.01	0\\
297.01	0\\
298.01	0\\
299.01	0\\
300.01	0\\
301.01	0\\
302.01	0\\
303.01	0\\
304.01	0\\
305.01	0\\
306.01	0\\
307.01	0\\
308.01	0\\
309.01	0\\
310.01	0\\
311.01	0\\
312.01	0\\
313.01	0\\
314.01	0\\
315.01	0\\
316.01	0\\
317.01	0\\
318.01	0\\
319.01	0\\
320.01	0\\
321.01	0\\
322.01	0\\
323.01	0\\
324.01	0\\
325.01	0\\
326.01	0\\
327.01	0\\
328.01	0\\
329.01	0\\
330.01	0\\
331.01	0\\
332.01	0\\
333.01	0\\
334.01	0\\
335.01	0\\
336.01	0\\
337.01	0\\
338.01	0\\
339.01	0\\
340.01	0\\
341.01	0\\
342.01	0\\
343.01	0\\
344.01	0\\
345.01	0\\
346.01	0\\
347.01	0\\
348.01	0\\
349.01	0\\
350.01	0\\
351.01	0\\
352.01	0\\
353.01	0\\
354.01	0\\
355.01	0\\
356.01	0\\
357.01	0\\
358.01	0\\
359.01	0\\
360.01	0\\
361.01	0\\
362.01	0\\
363.01	0\\
364.01	0\\
365.01	0\\
366.01	0\\
367.01	0\\
368.01	0\\
369.01	0\\
370.01	0\\
371.01	0\\
372.01	0\\
373.01	0\\
374.01	0\\
375.01	0\\
376.01	0\\
377.01	0\\
378.01	0\\
379.01	0\\
380.01	0\\
381.01	0\\
382.01	0\\
383.01	0\\
384.01	0\\
385.01	0\\
386.01	0\\
387.01	0\\
388.01	0\\
389.01	0\\
390.01	0\\
391.01	0\\
392.01	0\\
393.01	0\\
394.01	0\\
395.01	0\\
396.01	0\\
397.01	0\\
398.01	0\\
399.01	0\\
400.01	0\\
401.01	0\\
402.01	0\\
403.01	0\\
404.01	0\\
405.01	0\\
406.01	0\\
407.01	0\\
408.01	0\\
409.01	0\\
410.01	0\\
411.01	0\\
412.01	0\\
413.01	0\\
414.01	0\\
415.01	0\\
416.01	0\\
417.01	0\\
418.01	0\\
419.01	0\\
420.01	0\\
421.01	0\\
422.01	0\\
423.01	0\\
424.01	0\\
425.01	0\\
426.01	0\\
427.01	0\\
428.01	0\\
429.01	0\\
430.01	0\\
431.01	0\\
432.01	0\\
433.01	0\\
434.01	0\\
435.01	0\\
436.01	0\\
437.01	0\\
438.01	0\\
439.01	0\\
440.01	0\\
441.01	0\\
442.01	0\\
443.01	0\\
444.01	0\\
445.01	0\\
446.01	0\\
447.01	0\\
448.01	0\\
449.01	0\\
450.01	0\\
451.01	0\\
452.01	0\\
453.01	0\\
454.01	0\\
455.01	0\\
456.01	0\\
457.01	0\\
458.01	0\\
459.01	0\\
460.01	0\\
461.01	0\\
462.01	0\\
463.01	0\\
464.01	0\\
465.01	0\\
466.01	0\\
467.01	0\\
468.01	0\\
469.01	0\\
470.01	0\\
471.01	0\\
472.01	0\\
473.01	0\\
474.01	0\\
475.01	0\\
476.01	0\\
477.01	0\\
478.01	0\\
479.01	0\\
480.01	0\\
481.01	0\\
482.01	0\\
483.01	0\\
484.01	0\\
485.01	0\\
486.01	0\\
487.01	0\\
488.01	0\\
489.01	0\\
490.01	0\\
491.01	0\\
492.01	0\\
493.01	0\\
494.01	0\\
495.01	0\\
496.01	0\\
497.01	0\\
498.01	0\\
499.01	0\\
500.01	0\\
501.01	0\\
502.01	0\\
503.01	0\\
504.01	0\\
505.01	0\\
506.01	0\\
507.01	0\\
508.01	0\\
509.01	0\\
510.01	0\\
511.01	0\\
512.01	0\\
513.01	0\\
514.01	0\\
515.01	0\\
516.01	0\\
517.01	0\\
518.01	0\\
519.01	0\\
520.01	0\\
521.01	0\\
522.01	0\\
523.01	0\\
524.01	0\\
525.01	0\\
526.01	0\\
527.01	0\\
528.01	0\\
529.01	0\\
530.01	0\\
531.01	0\\
532.01	0\\
533.01	0\\
534.01	0\\
535.01	0\\
536.01	0\\
537.01	0\\
538.01	0\\
539.01	0\\
540.01	0\\
541.01	0\\
542.01	0\\
543.01	0\\
544.01	0\\
545.01	0\\
546.01	0\\
547.01	0\\
548.01	0\\
549.01	0\\
550.01	0\\
551.01	0\\
552.01	0\\
553.01	0\\
554.01	0\\
555.01	0\\
556.01	0\\
557.01	0\\
558.01	0\\
559.01	0\\
560.01	0\\
561.01	0\\
562.01	0\\
563.01	0\\
564.01	0\\
565.01	0\\
566.01	0\\
567.01	0\\
568.01	0\\
569.01	0\\
570.01	0\\
571.01	0\\
572.01	0\\
573.01	0\\
574.01	0\\
575.01	0\\
576.01	0\\
577.01	0\\
578.01	0\\
579.01	0\\
580.01	0\\
581.01	0\\
582.01	0\\
583.01	0\\
584.01	0\\
585.01	0\\
586.01	0\\
587.01	0\\
588.01	0\\
589.01	0\\
590.01	0\\
591.01	0\\
592.01	0\\
593.01	0\\
594.01	0\\
595.01	0\\
596.01	0\\
597.01	0\\
598.01	0.00143822749983459\\
599.01	0.00385661144683571\\
599.02	0.00389414709863901\\
599.03	0.00393204040529412\\
599.04	0.00397029482087115\\
599.05	0.00400891383277854\\
599.06	0.00404790096208447\\
599.07	0.00408725976384149\\
599.08	0.00412699382741433\\
599.09	0.0041671067768107\\
599.1	0.00420760227101545\\
599.11	0.00424848400432787\\
599.12	0.00428975570670221\\
599.13	0.00433142114409155\\
599.14	0.00437348411879493\\
599.15	0.00441594846980785\\
599.16	0.00445881807317617\\
599.17	0.00450209684235336\\
599.18	0.00454578872856121\\
599.19	0.00458989772115408\\
599.2	0.00463442784798657\\
599.21	0.00467938317578476\\
599.22	0.00472476781052109\\
599.23	0.00477058589779272\\
599.24	0.00481684162320371\\
599.25	0.00486353921275068\\
599.26	0.0049106829225159\\
599.27	0.00495827704247297\\
599.28	0.00500632590381774\\
599.29	0.00505483387936404\\
599.3	0.00510380538394311\\
599.31	0.00515324487480703\\
599.32	0.00520315685203588\\
599.33	0.00525354585894884\\
599.34	0.00530441648251928\\
599.35	0.00535577335379373\\
599.36	0.005407621148315\\
599.37	0.00545996458654921\\
599.38	0.00551280843431708\\
599.39	0.00556615750322915\\
599.4	0.00562001665112538\\
599.41	0.00567439078251885\\
599.42	0.00572928484904367\\
599.43	0.00578470384990735\\
599.44	0.00584065283234732\\
599.45	0.00589713689209195\\
599.46	0.00595416117382597\\
599.47	0.00601173087166026\\
599.48	0.00606985122960629\\
599.49	0.006128527542055\\
599.5	0.0061877651542603\\
599.51	0.00624756946282722\\
599.52	0.00630794591620473\\
599.53	0.00636890001518324\\
599.54	0.00643043731339698\\
599.55	0.00649256341783105\\
599.56	0.00655528398933347\\
599.57	0.00661860474313204\\
599.58	0.00668253144935614\\
599.59	0.00674706993356369\\
599.6	0.00681222607727294\\
599.61	0.0068780058184995\\
599.62	0.0069444151522985\\
599.63	0.00701146013131191\\
599.64	0.0070791468663211\\
599.65	0.00714748152680477\\
599.66	0.00721647034150212\\
599.67	0.00728611959898148\\
599.68	0.00735643564821442\\
599.69	0.00742742489915527\\
599.7	0.00749909382332627\\
599.71	0.00757144895440832\\
599.72	0.00764449688883733\\
599.73	0.00771824428640635\\
599.74	0.00779269787087345\\
599.75	0.00786786443057545\\
599.76	0.00794375081904746\\
599.77	0.0080203639556484\\
599.78	0.00809771082619259\\
599.79	0.00817579848358721\\
599.8	0.00825463404847609\\
599.81	0.0083342247098895\\
599.82	0.0084145777259002\\
599.83	0.00849570042428592\\
599.84	0.00857760020319794\\
599.85	0.00866028453183631\\
599.86	0.00874376095113144\\
599.87	0.00882803707443219\\
599.88	0.00891312058820066\\
599.89	0.00899901925271357\\
599.9	0.00908574090277037\\
599.91	0.00917329344840818\\
599.92	0.00926168487562357\\
599.93	0.00935092324710127\\
599.94	0.00944101670294984\\
599.95	0.00953197346144448\\
599.96	0.00962380181977693\\
599.97	0.00971651015481255\\
599.98	0.0098101069238547\\
599.99	0.00990460066541651\\
600	0.01\\
};
\addplot [color=mycolor18,solid,forget plot]
  table[row sep=crcr]{%
0.01	0\\
1.01	0\\
2.01	0\\
3.01	0\\
4.01	0\\
5.01	0\\
6.01	0\\
7.01	0\\
8.01	0\\
9.01	0\\
10.01	0\\
11.01	0\\
12.01	0\\
13.01	0\\
14.01	0\\
15.01	0\\
16.01	0\\
17.01	0\\
18.01	0\\
19.01	0\\
20.01	0\\
21.01	0\\
22.01	0\\
23.01	0\\
24.01	0\\
25.01	0\\
26.01	0\\
27.01	0\\
28.01	0\\
29.01	0\\
30.01	0\\
31.01	0\\
32.01	0\\
33.01	0\\
34.01	0\\
35.01	0\\
36.01	0\\
37.01	0\\
38.01	0\\
39.01	0\\
40.01	0\\
41.01	0\\
42.01	0\\
43.01	0\\
44.01	0\\
45.01	0\\
46.01	0\\
47.01	0\\
48.01	0\\
49.01	0\\
50.01	0\\
51.01	0\\
52.01	0\\
53.01	0\\
54.01	0\\
55.01	0\\
56.01	0\\
57.01	0\\
58.01	0\\
59.01	0\\
60.01	0\\
61.01	0\\
62.01	0\\
63.01	0\\
64.01	0\\
65.01	0\\
66.01	0\\
67.01	0\\
68.01	0\\
69.01	0\\
70.01	0\\
71.01	0\\
72.01	0\\
73.01	0\\
74.01	0\\
75.01	0\\
76.01	0\\
77.01	0\\
78.01	0\\
79.01	0\\
80.01	0\\
81.01	0\\
82.01	0\\
83.01	0\\
84.01	0\\
85.01	0\\
86.01	0\\
87.01	0\\
88.01	0\\
89.01	0\\
90.01	0\\
91.01	0\\
92.01	0\\
93.01	0\\
94.01	0\\
95.01	0\\
96.01	0\\
97.01	0\\
98.01	0\\
99.01	0\\
100.01	0\\
101.01	0\\
102.01	0\\
103.01	0\\
104.01	0\\
105.01	0\\
106.01	0\\
107.01	0\\
108.01	0\\
109.01	0\\
110.01	0\\
111.01	0\\
112.01	0\\
113.01	0\\
114.01	0\\
115.01	0\\
116.01	0\\
117.01	0\\
118.01	0\\
119.01	0\\
120.01	0\\
121.01	0\\
122.01	0\\
123.01	0\\
124.01	0\\
125.01	0\\
126.01	0\\
127.01	0\\
128.01	0\\
129.01	0\\
130.01	0\\
131.01	0\\
132.01	0\\
133.01	0\\
134.01	0\\
135.01	0\\
136.01	0\\
137.01	0\\
138.01	0\\
139.01	0\\
140.01	0\\
141.01	0\\
142.01	0\\
143.01	0\\
144.01	0\\
145.01	0\\
146.01	0\\
147.01	0\\
148.01	0\\
149.01	0\\
150.01	0\\
151.01	0\\
152.01	0\\
153.01	0\\
154.01	0\\
155.01	0\\
156.01	0\\
157.01	0\\
158.01	0\\
159.01	0\\
160.01	0\\
161.01	0\\
162.01	0\\
163.01	0\\
164.01	0\\
165.01	0\\
166.01	0\\
167.01	0\\
168.01	0\\
169.01	0\\
170.01	0\\
171.01	0\\
172.01	0\\
173.01	0\\
174.01	0\\
175.01	0\\
176.01	0\\
177.01	0\\
178.01	0\\
179.01	0\\
180.01	0\\
181.01	0\\
182.01	0\\
183.01	0\\
184.01	0\\
185.01	0\\
186.01	0\\
187.01	0\\
188.01	0\\
189.01	0\\
190.01	0\\
191.01	0\\
192.01	0\\
193.01	0\\
194.01	0\\
195.01	0\\
196.01	0\\
197.01	0\\
198.01	0\\
199.01	0\\
200.01	0\\
201.01	0\\
202.01	0\\
203.01	0\\
204.01	0\\
205.01	0\\
206.01	0\\
207.01	0\\
208.01	0\\
209.01	0\\
210.01	0\\
211.01	0\\
212.01	0\\
213.01	0\\
214.01	0\\
215.01	0\\
216.01	0\\
217.01	0\\
218.01	0\\
219.01	0\\
220.01	0\\
221.01	0\\
222.01	0\\
223.01	0\\
224.01	0\\
225.01	0\\
226.01	0\\
227.01	0\\
228.01	0\\
229.01	0\\
230.01	0\\
231.01	0\\
232.01	0\\
233.01	0\\
234.01	0\\
235.01	0\\
236.01	0\\
237.01	0\\
238.01	0\\
239.01	0\\
240.01	0\\
241.01	0\\
242.01	0\\
243.01	0\\
244.01	0\\
245.01	0\\
246.01	0\\
247.01	0\\
248.01	0\\
249.01	0\\
250.01	0\\
251.01	0\\
252.01	0\\
253.01	0\\
254.01	0\\
255.01	0\\
256.01	0\\
257.01	0\\
258.01	0\\
259.01	0\\
260.01	0\\
261.01	0\\
262.01	0\\
263.01	0\\
264.01	0\\
265.01	0\\
266.01	0\\
267.01	0\\
268.01	0\\
269.01	0\\
270.01	0\\
271.01	0\\
272.01	0\\
273.01	0\\
274.01	0\\
275.01	0\\
276.01	0\\
277.01	0\\
278.01	0\\
279.01	0\\
280.01	0\\
281.01	0\\
282.01	0\\
283.01	0\\
284.01	0\\
285.01	0\\
286.01	0\\
287.01	0\\
288.01	0\\
289.01	0\\
290.01	0\\
291.01	0\\
292.01	0\\
293.01	0\\
294.01	0\\
295.01	0\\
296.01	0\\
297.01	0\\
298.01	0\\
299.01	0\\
300.01	0\\
301.01	0\\
302.01	0\\
303.01	0\\
304.01	0\\
305.01	0\\
306.01	0\\
307.01	0\\
308.01	0\\
309.01	0\\
310.01	0\\
311.01	0\\
312.01	0\\
313.01	0\\
314.01	0\\
315.01	0\\
316.01	0\\
317.01	0\\
318.01	0\\
319.01	0\\
320.01	0\\
321.01	0\\
322.01	0\\
323.01	0\\
324.01	0\\
325.01	0\\
326.01	0\\
327.01	0\\
328.01	0\\
329.01	0\\
330.01	0\\
331.01	0\\
332.01	0\\
333.01	0\\
334.01	0\\
335.01	0\\
336.01	0\\
337.01	0\\
338.01	0\\
339.01	0\\
340.01	0\\
341.01	0\\
342.01	0\\
343.01	0\\
344.01	0\\
345.01	0\\
346.01	0\\
347.01	0\\
348.01	0\\
349.01	0\\
350.01	0\\
351.01	0\\
352.01	0\\
353.01	0\\
354.01	0\\
355.01	0\\
356.01	0\\
357.01	0\\
358.01	0\\
359.01	0\\
360.01	0\\
361.01	0\\
362.01	0\\
363.01	0\\
364.01	0\\
365.01	0\\
366.01	0\\
367.01	0\\
368.01	0\\
369.01	0\\
370.01	0\\
371.01	0\\
372.01	0\\
373.01	0\\
374.01	0\\
375.01	0\\
376.01	0\\
377.01	0\\
378.01	0\\
379.01	0\\
380.01	0\\
381.01	0\\
382.01	0\\
383.01	0\\
384.01	0\\
385.01	0\\
386.01	0\\
387.01	0\\
388.01	0\\
389.01	0\\
390.01	0\\
391.01	0\\
392.01	0\\
393.01	0\\
394.01	0\\
395.01	0\\
396.01	0\\
397.01	0\\
398.01	0\\
399.01	0\\
400.01	0\\
401.01	0\\
402.01	0\\
403.01	0\\
404.01	0\\
405.01	0\\
406.01	0\\
407.01	0\\
408.01	0\\
409.01	0\\
410.01	0\\
411.01	0\\
412.01	0\\
413.01	0\\
414.01	0\\
415.01	0\\
416.01	0\\
417.01	0\\
418.01	0\\
419.01	0\\
420.01	0\\
421.01	0\\
422.01	0\\
423.01	0\\
424.01	0\\
425.01	0\\
426.01	0\\
427.01	0\\
428.01	0\\
429.01	0\\
430.01	0\\
431.01	0\\
432.01	0\\
433.01	0\\
434.01	0\\
435.01	0\\
436.01	0\\
437.01	0\\
438.01	0\\
439.01	0\\
440.01	0\\
441.01	0\\
442.01	0\\
443.01	0\\
444.01	0\\
445.01	0\\
446.01	0\\
447.01	0\\
448.01	0\\
449.01	0\\
450.01	0\\
451.01	0\\
452.01	0\\
453.01	0\\
454.01	0\\
455.01	0\\
456.01	0\\
457.01	0\\
458.01	0\\
459.01	0\\
460.01	0\\
461.01	0\\
462.01	0\\
463.01	0\\
464.01	0\\
465.01	0\\
466.01	0\\
467.01	0\\
468.01	0\\
469.01	0\\
470.01	0\\
471.01	0\\
472.01	0\\
473.01	0\\
474.01	0\\
475.01	0\\
476.01	0\\
477.01	0\\
478.01	0\\
479.01	0\\
480.01	0\\
481.01	0\\
482.01	0\\
483.01	0\\
484.01	0\\
485.01	0\\
486.01	0\\
487.01	0\\
488.01	0\\
489.01	0\\
490.01	0\\
491.01	0\\
492.01	0\\
493.01	0\\
494.01	0\\
495.01	0\\
496.01	0\\
497.01	0\\
498.01	0\\
499.01	0\\
500.01	0\\
501.01	0\\
502.01	0\\
503.01	0\\
504.01	0\\
505.01	0\\
506.01	0\\
507.01	0\\
508.01	0\\
509.01	0\\
510.01	0\\
511.01	0\\
512.01	0\\
513.01	0\\
514.01	0\\
515.01	0\\
516.01	0\\
517.01	0\\
518.01	0\\
519.01	0\\
520.01	0\\
521.01	0\\
522.01	0\\
523.01	0\\
524.01	0\\
525.01	0\\
526.01	0\\
527.01	0\\
528.01	0\\
529.01	0\\
530.01	0\\
531.01	0\\
532.01	0\\
533.01	0\\
534.01	0\\
535.01	0\\
536.01	0\\
537.01	0\\
538.01	0\\
539.01	0\\
540.01	0\\
541.01	0\\
542.01	0\\
543.01	0\\
544.01	0\\
545.01	0\\
546.01	0\\
547.01	0\\
548.01	0\\
549.01	0\\
550.01	0\\
551.01	0\\
552.01	0\\
553.01	0\\
554.01	0\\
555.01	0\\
556.01	0\\
557.01	0\\
558.01	0\\
559.01	0\\
560.01	0\\
561.01	0\\
562.01	0\\
563.01	0\\
564.01	0\\
565.01	0\\
566.01	0\\
567.01	0\\
568.01	0\\
569.01	0\\
570.01	0\\
571.01	0\\
572.01	0\\
573.01	0\\
574.01	0\\
575.01	0\\
576.01	0\\
577.01	0\\
578.01	0\\
579.01	0\\
580.01	0\\
581.01	0\\
582.01	0\\
583.01	0\\
584.01	0\\
585.01	0\\
586.01	0\\
587.01	0\\
588.01	0\\
589.01	0\\
590.01	0\\
591.01	0\\
592.01	0\\
593.01	0\\
594.01	0\\
595.01	0\\
596.01	0\\
597.01	0\\
598.01	0.00143832017857744\\
599.01	0.00385661255145558\\
599.02	0.00389414816850622\\
599.03	0.00393204144103766\\
599.04	0.00397029582311886\\
599.05	0.00400891480215704\\
599.06	0.00404790189921909\\
599.07	0.00408726066935627\\
599.08	0.00412699470193188\\
599.09	0.00416710762095217\\
599.1	0.00420760308540043\\
599.11	0.00424848478957431\\
599.12	0.00428975646342631\\
599.13	0.00433142187290768\\
599.14	0.00437348482031551\\
599.15	0.00441594914464327\\
599.16	0.00445881872193463\\
599.17	0.00450209746564076\\
599.18	0.00454578932698104\\
599.19	0.00458989829530726\\
599.2	0.0046344283984713\\
599.21	0.0046793837031964\\
599.22	0.00472476831545197\\
599.23	0.00477058638083201\\
599.24	0.00481684208493721\\
599.25	0.00486353965376069\\
599.26	0.00491068334338074\\
599.27	0.00495827744376709\\
599.28	0.00500632628611153\\
599.29	0.0050548342432236\\
599.3	0.00510380572993011\\
599.31	0.00515324520347841\\
599.32	0.00520315716394368\\
599.33	0.00525354615463998\\
599.34	0.00530441676253529\\
599.35	0.00535577361867056\\
599.36	0.00540762139858276\\
599.37	0.00545996482273196\\
599.38	0.00551280865693252\\
599.39	0.00556615771278843\\
599.4	0.00562001684813284\\
599.41	0.00567439096747174\\
599.42	0.00572928502243194\\
599.43	0.00578470401221333\\
599.44	0.00584065298404554\\
599.45	0.00589713703364883\\
599.46	0.00595416130569955\\
599.47	0.00601173099430001\\
599.48	0.00606985134345282\\
599.49	0.00612852764753978\\
599.5	0.00618776525180547\\
599.51	0.0062475695528453\\
599.52	0.0063079459990984\\
599.53	0.00636890009134512\\
599.54	0.00643043738320938\\
599.55	0.00649256348166578\\
599.56	0.0065552840475516\\
599.57	0.0066186047960837\\
599.58	0.00668253149738038\\
599.59	0.00674706997698825\\
599.6	0.00681222611641409\\
599.61	0.00687800585366193\\
599.62	0.00694441518377516\\
599.63	0.00701146015938389\\
599.64	0.00707914689125756\\
599.65	0.00714748154886281\\
599.66	0.00721647036092677\\
599.67	0.00728611961600567\\
599.68	0.00735643566305893\\
599.69	0.00742742491202878\\
599.7	0.00749909383442539\\
599.71	0.00757144896391765\\
599.72	0.00764449689692958\\
599.73	0.00771824429324247\\
599.74	0.0077926978766028\\
599.75	0.00786786443533595\\
599.76	0.00794375082296586\\
599.77	0.00802036395884057\\
599.78	0.00809771082876376\\
599.79	0.00817579848563237\\
599.8	0.00825463405008032\\
599.81	0.0083342247111284\\
599.82	0.00841457772684034\\
599.83	0.00849570042498529\\
599.84	0.00857760020370652\\
599.85	0.00866028453219658\\
599.86	0.00874376095137895\\
599.87	0.00882803707459619\\
599.88	0.00891312058830471\\
599.89	0.00899901925277615\\
599.9	0.00908574090280557\\
599.91	0.00917329344842634\\
599.92	0.0092616848756319\\
599.93	0.00935092324710449\\
599.94	0.00944101670295078\\
599.95	0.00953197346144465\\
599.96	0.00962380181977693\\
599.97	0.00971651015481255\\
599.98	0.0098101069238547\\
599.99	0.00990460066541651\\
600	0.01\\
};
\addplot [color=red!25!mycolor17,solid,forget plot]
  table[row sep=crcr]{%
0.01	0\\
1.01	0\\
2.01	0\\
3.01	0\\
4.01	0\\
5.01	0\\
6.01	0\\
7.01	0\\
8.01	0\\
9.01	0\\
10.01	0\\
11.01	0\\
12.01	0\\
13.01	0\\
14.01	0\\
15.01	0\\
16.01	0\\
17.01	0\\
18.01	0\\
19.01	0\\
20.01	0\\
21.01	0\\
22.01	0\\
23.01	0\\
24.01	0\\
25.01	0\\
26.01	0\\
27.01	0\\
28.01	0\\
29.01	0\\
30.01	0\\
31.01	0\\
32.01	0\\
33.01	0\\
34.01	0\\
35.01	0\\
36.01	0\\
37.01	0\\
38.01	0\\
39.01	0\\
40.01	0\\
41.01	0\\
42.01	0\\
43.01	0\\
44.01	0\\
45.01	0\\
46.01	0\\
47.01	0\\
48.01	0\\
49.01	0\\
50.01	0\\
51.01	0\\
52.01	0\\
53.01	0\\
54.01	0\\
55.01	0\\
56.01	0\\
57.01	0\\
58.01	0\\
59.01	0\\
60.01	0\\
61.01	0\\
62.01	0\\
63.01	0\\
64.01	0\\
65.01	0\\
66.01	0\\
67.01	0\\
68.01	0\\
69.01	0\\
70.01	0\\
71.01	0\\
72.01	0\\
73.01	0\\
74.01	0\\
75.01	0\\
76.01	0\\
77.01	0\\
78.01	0\\
79.01	0\\
80.01	0\\
81.01	0\\
82.01	0\\
83.01	0\\
84.01	0\\
85.01	0\\
86.01	0\\
87.01	0\\
88.01	0\\
89.01	0\\
90.01	0\\
91.01	0\\
92.01	0\\
93.01	0\\
94.01	0\\
95.01	0\\
96.01	0\\
97.01	0\\
98.01	0\\
99.01	0\\
100.01	0\\
101.01	0\\
102.01	0\\
103.01	0\\
104.01	0\\
105.01	0\\
106.01	0\\
107.01	0\\
108.01	0\\
109.01	0\\
110.01	0\\
111.01	0\\
112.01	0\\
113.01	0\\
114.01	0\\
115.01	0\\
116.01	0\\
117.01	0\\
118.01	0\\
119.01	0\\
120.01	0\\
121.01	0\\
122.01	0\\
123.01	0\\
124.01	0\\
125.01	0\\
126.01	0\\
127.01	0\\
128.01	0\\
129.01	0\\
130.01	0\\
131.01	0\\
132.01	0\\
133.01	0\\
134.01	0\\
135.01	0\\
136.01	0\\
137.01	0\\
138.01	0\\
139.01	0\\
140.01	0\\
141.01	0\\
142.01	0\\
143.01	0\\
144.01	0\\
145.01	0\\
146.01	0\\
147.01	0\\
148.01	0\\
149.01	0\\
150.01	0\\
151.01	0\\
152.01	0\\
153.01	0\\
154.01	0\\
155.01	0\\
156.01	0\\
157.01	0\\
158.01	0\\
159.01	0\\
160.01	0\\
161.01	0\\
162.01	0\\
163.01	0\\
164.01	0\\
165.01	0\\
166.01	0\\
167.01	0\\
168.01	0\\
169.01	0\\
170.01	0\\
171.01	0\\
172.01	0\\
173.01	0\\
174.01	0\\
175.01	0\\
176.01	0\\
177.01	0\\
178.01	0\\
179.01	0\\
180.01	0\\
181.01	0\\
182.01	0\\
183.01	0\\
184.01	0\\
185.01	0\\
186.01	0\\
187.01	0\\
188.01	0\\
189.01	0\\
190.01	0\\
191.01	0\\
192.01	0\\
193.01	0\\
194.01	0\\
195.01	0\\
196.01	0\\
197.01	0\\
198.01	0\\
199.01	0\\
200.01	0\\
201.01	0\\
202.01	0\\
203.01	0\\
204.01	0\\
205.01	0\\
206.01	0\\
207.01	0\\
208.01	0\\
209.01	0\\
210.01	0\\
211.01	0\\
212.01	0\\
213.01	0\\
214.01	0\\
215.01	0\\
216.01	0\\
217.01	0\\
218.01	0\\
219.01	0\\
220.01	0\\
221.01	0\\
222.01	0\\
223.01	0\\
224.01	0\\
225.01	0\\
226.01	0\\
227.01	0\\
228.01	0\\
229.01	0\\
230.01	0\\
231.01	0\\
232.01	0\\
233.01	0\\
234.01	0\\
235.01	0\\
236.01	0\\
237.01	0\\
238.01	0\\
239.01	0\\
240.01	0\\
241.01	0\\
242.01	0\\
243.01	0\\
244.01	0\\
245.01	0\\
246.01	0\\
247.01	0\\
248.01	0\\
249.01	0\\
250.01	0\\
251.01	0\\
252.01	0\\
253.01	0\\
254.01	0\\
255.01	0\\
256.01	0\\
257.01	0\\
258.01	0\\
259.01	0\\
260.01	0\\
261.01	0\\
262.01	0\\
263.01	0\\
264.01	0\\
265.01	0\\
266.01	0\\
267.01	0\\
268.01	0\\
269.01	0\\
270.01	0\\
271.01	0\\
272.01	0\\
273.01	0\\
274.01	0\\
275.01	0\\
276.01	0\\
277.01	0\\
278.01	0\\
279.01	0\\
280.01	0\\
281.01	0\\
282.01	0\\
283.01	0\\
284.01	0\\
285.01	0\\
286.01	0\\
287.01	0\\
288.01	0\\
289.01	0\\
290.01	0\\
291.01	0\\
292.01	0\\
293.01	0\\
294.01	0\\
295.01	0\\
296.01	0\\
297.01	0\\
298.01	0\\
299.01	0\\
300.01	0\\
301.01	0\\
302.01	0\\
303.01	0\\
304.01	0\\
305.01	0\\
306.01	0\\
307.01	0\\
308.01	0\\
309.01	0\\
310.01	0\\
311.01	0\\
312.01	0\\
313.01	0\\
314.01	0\\
315.01	0\\
316.01	0\\
317.01	0\\
318.01	0\\
319.01	0\\
320.01	0\\
321.01	0\\
322.01	0\\
323.01	0\\
324.01	0\\
325.01	0\\
326.01	0\\
327.01	0\\
328.01	0\\
329.01	0\\
330.01	0\\
331.01	0\\
332.01	0\\
333.01	0\\
334.01	0\\
335.01	0\\
336.01	0\\
337.01	0\\
338.01	0\\
339.01	0\\
340.01	0\\
341.01	0\\
342.01	0\\
343.01	0\\
344.01	0\\
345.01	0\\
346.01	0\\
347.01	0\\
348.01	0\\
349.01	0\\
350.01	0\\
351.01	0\\
352.01	0\\
353.01	0\\
354.01	0\\
355.01	0\\
356.01	0\\
357.01	0\\
358.01	0\\
359.01	0\\
360.01	0\\
361.01	0\\
362.01	0\\
363.01	0\\
364.01	0\\
365.01	0\\
366.01	0\\
367.01	0\\
368.01	0\\
369.01	0\\
370.01	0\\
371.01	0\\
372.01	0\\
373.01	0\\
374.01	0\\
375.01	0\\
376.01	0\\
377.01	0\\
378.01	0\\
379.01	0\\
380.01	0\\
381.01	0\\
382.01	0\\
383.01	0\\
384.01	0\\
385.01	0\\
386.01	0\\
387.01	0\\
388.01	0\\
389.01	0\\
390.01	0\\
391.01	0\\
392.01	0\\
393.01	0\\
394.01	0\\
395.01	0\\
396.01	0\\
397.01	0\\
398.01	0\\
399.01	0\\
400.01	0\\
401.01	0\\
402.01	0\\
403.01	0\\
404.01	0\\
405.01	0\\
406.01	0\\
407.01	0\\
408.01	0\\
409.01	0\\
410.01	0\\
411.01	0\\
412.01	0\\
413.01	0\\
414.01	0\\
415.01	0\\
416.01	0\\
417.01	0\\
418.01	0\\
419.01	0\\
420.01	0\\
421.01	0\\
422.01	0\\
423.01	0\\
424.01	0\\
425.01	0\\
426.01	0\\
427.01	0\\
428.01	0\\
429.01	0\\
430.01	0\\
431.01	0\\
432.01	0\\
433.01	0\\
434.01	0\\
435.01	0\\
436.01	0\\
437.01	0\\
438.01	0\\
439.01	0\\
440.01	0\\
441.01	0\\
442.01	0\\
443.01	0\\
444.01	0\\
445.01	0\\
446.01	0\\
447.01	0\\
448.01	0\\
449.01	0\\
450.01	0\\
451.01	0\\
452.01	0\\
453.01	0\\
454.01	0\\
455.01	0\\
456.01	0\\
457.01	0\\
458.01	0\\
459.01	0\\
460.01	0\\
461.01	0\\
462.01	0\\
463.01	0\\
464.01	0\\
465.01	0\\
466.01	0\\
467.01	0\\
468.01	0\\
469.01	0\\
470.01	0\\
471.01	0\\
472.01	0\\
473.01	0\\
474.01	0\\
475.01	0\\
476.01	0\\
477.01	0\\
478.01	0\\
479.01	0\\
480.01	0\\
481.01	0\\
482.01	0\\
483.01	0\\
484.01	0\\
485.01	0\\
486.01	0\\
487.01	0\\
488.01	0\\
489.01	0\\
490.01	0\\
491.01	0\\
492.01	0\\
493.01	0\\
494.01	0\\
495.01	0\\
496.01	0\\
497.01	0\\
498.01	0\\
499.01	0\\
500.01	0\\
501.01	0\\
502.01	0\\
503.01	0\\
504.01	0\\
505.01	0\\
506.01	0\\
507.01	0\\
508.01	0\\
509.01	0\\
510.01	0\\
511.01	0\\
512.01	0\\
513.01	0\\
514.01	0\\
515.01	0\\
516.01	0\\
517.01	0\\
518.01	0\\
519.01	0\\
520.01	0\\
521.01	0\\
522.01	0\\
523.01	0\\
524.01	0\\
525.01	0\\
526.01	0\\
527.01	0\\
528.01	0\\
529.01	0\\
530.01	0\\
531.01	0\\
532.01	0\\
533.01	0\\
534.01	0\\
535.01	0\\
536.01	0\\
537.01	0\\
538.01	0\\
539.01	0\\
540.01	0\\
541.01	0\\
542.01	0\\
543.01	0\\
544.01	0\\
545.01	0\\
546.01	0\\
547.01	0\\
548.01	0\\
549.01	0\\
550.01	0\\
551.01	0\\
552.01	0\\
553.01	0\\
554.01	0\\
555.01	0\\
556.01	0\\
557.01	0\\
558.01	0\\
559.01	0\\
560.01	0\\
561.01	0\\
562.01	0\\
563.01	0\\
564.01	0\\
565.01	0\\
566.01	0\\
567.01	0\\
568.01	0\\
569.01	0\\
570.01	0\\
571.01	0\\
572.01	0\\
573.01	0\\
574.01	0\\
575.01	0\\
576.01	0\\
577.01	0\\
578.01	0\\
579.01	0\\
580.01	0\\
581.01	0\\
582.01	0\\
583.01	0\\
584.01	0\\
585.01	0\\
586.01	0\\
587.01	0\\
588.01	0\\
589.01	0\\
590.01	0\\
591.01	0\\
592.01	0\\
593.01	0\\
594.01	0\\
595.01	0\\
596.01	0\\
597.01	0\\
598.01	0.00143832067364899\\
599.01	0.00385661257178434\\
599.02	0.0038941481880362\\
599.03	0.00393204145978999\\
599.04	0.00397029584111443\\
599.05	0.00400891481941645\\
599.06	0.00404790191576273\\
599.07	0.00408726068520424\\
599.08	0.00412699471710402\\
599.09	0.00416710763546808\\
599.1	0.00420760309927944\\
599.11	0.00424848480283546\\
599.12	0.00428975647608841\\
599.13	0.00433142188498926\\
599.14	0.00437348483183484\\
599.15	0.00441594915561834\\
599.16	0.00445881873238316\\
599.17	0.00450209747558018\\
599.18	0.00454578933642853\\
599.19	0.00458989830427969\\
599.2	0.00463442840698527\\
599.21	0.00467938371126824\\
599.22	0.00472476832309771\\
599.23	0.00477058638806738\\
599.24	0.00481684209177767\\
599.25	0.0048635396602214\\
599.26	0.00491068334947654\\
599.27	0.00495827744951256\\
599.28	0.00500632629152094\\
599.29	0.00505483424831089\\
599.3	0.00510380573470894\\
599.31	0.00515324520796213\\
599.32	0.00520315716814533\\
599.33	0.00525354615857229\\
599.34	0.00530441676621069\\
599.35	0.00535577362210116\\
599.36	0.00540762140178036\\
599.37	0.00545996482570803\\
599.38	0.00551280865969823\\
599.39	0.00556615771535462\\
599.4	0.00562001685051006\\
599.41	0.00567439096967018\\
599.42	0.00572928502446148\\
599.43	0.00578470401408357\\
599.44	0.00584065298576571\\
599.45	0.00589713703522788\\
599.46	0.0059541613071461\\
599.47	0.00601173099562236\\
599.48	0.00606985134465893\\
599.49	0.00612852764863735\\
599.5	0.00618776525280185\\
599.51	0.00624756955374753\\
599.52	0.00630794599991322\\
599.53	0.00636890009207898\\
599.54	0.00643043738386841\\
599.55	0.00649256348225581\\
599.56	0.00655528404807817\\
599.57	0.00661860479655207\\
599.58	0.00668253149779552\\
599.59	0.00674706997735484\\
599.6	0.00681222611673654\\
599.61	0.00687800585394439\\
599.62	0.00694441518402151\\
599.63	0.00701146015959775\\
599.64	0.0070791468914423\\
599.65	0.00714748154902157\\
599.66	0.00721647036106244\\
599.67	0.00728611961612093\\
599.68	0.00735643566315622\\
599.69	0.00742742491211035\\
599.7	0.00749909383449329\\
599.71	0.00757144896397373\\
599.72	0.0076444968969755\\
599.73	0.00771824429327973\\
599.74	0.00779269787663272\\
599.75	0.00786786443535973\\
599.76	0.00794375082298454\\
599.77	0.00802036395885504\\
599.78	0.00809771082877482\\
599.79	0.00817579848564069\\
599.8	0.00825463405008646\\
599.81	0.00833422471113284\\
599.82	0.00841457772684348\\
599.83	0.00849570042498746\\
599.84	0.00857760020370797\\
599.85	0.00866028453219752\\
599.86	0.00874376095137954\\
599.87	0.00882803707459654\\
599.88	0.0089131205883049\\
599.89	0.00899901925277625\\
599.9	0.00908574090280562\\
599.91	0.00917329344842636\\
599.92	0.0092616848756319\\
599.93	0.00935092324710449\\
599.94	0.00944101670295079\\
599.95	0.00953197346144465\\
599.96	0.00962380181977694\\
599.97	0.00971651015481255\\
599.98	0.0098101069238547\\
599.99	0.00990460066541651\\
600	0.01\\
};
\addplot [color=mycolor19,solid,forget plot]
  table[row sep=crcr]{%
0.01	0\\
1.01	0\\
2.01	0\\
3.01	0\\
4.01	0\\
5.01	0\\
6.01	0\\
7.01	0\\
8.01	0\\
9.01	0\\
10.01	0\\
11.01	0\\
12.01	0\\
13.01	0\\
14.01	0\\
15.01	0\\
16.01	0\\
17.01	0\\
18.01	0\\
19.01	0\\
20.01	0\\
21.01	0\\
22.01	0\\
23.01	0\\
24.01	0\\
25.01	0\\
26.01	0\\
27.01	0\\
28.01	0\\
29.01	0\\
30.01	0\\
31.01	0\\
32.01	0\\
33.01	0\\
34.01	0\\
35.01	0\\
36.01	0\\
37.01	0\\
38.01	0\\
39.01	0\\
40.01	0\\
41.01	0\\
42.01	0\\
43.01	0\\
44.01	0\\
45.01	0\\
46.01	0\\
47.01	0\\
48.01	0\\
49.01	0\\
50.01	0\\
51.01	0\\
52.01	0\\
53.01	0\\
54.01	0\\
55.01	0\\
56.01	0\\
57.01	0\\
58.01	0\\
59.01	0\\
60.01	0\\
61.01	0\\
62.01	0\\
63.01	0\\
64.01	0\\
65.01	0\\
66.01	0\\
67.01	0\\
68.01	0\\
69.01	0\\
70.01	0\\
71.01	0\\
72.01	0\\
73.01	0\\
74.01	0\\
75.01	0\\
76.01	0\\
77.01	0\\
78.01	0\\
79.01	0\\
80.01	0\\
81.01	0\\
82.01	0\\
83.01	0\\
84.01	0\\
85.01	0\\
86.01	0\\
87.01	0\\
88.01	0\\
89.01	0\\
90.01	0\\
91.01	0\\
92.01	0\\
93.01	0\\
94.01	0\\
95.01	0\\
96.01	0\\
97.01	0\\
98.01	0\\
99.01	0\\
100.01	0\\
101.01	0\\
102.01	0\\
103.01	0\\
104.01	0\\
105.01	0\\
106.01	0\\
107.01	0\\
108.01	0\\
109.01	0\\
110.01	0\\
111.01	0\\
112.01	0\\
113.01	0\\
114.01	0\\
115.01	0\\
116.01	0\\
117.01	0\\
118.01	0\\
119.01	0\\
120.01	0\\
121.01	0\\
122.01	0\\
123.01	0\\
124.01	0\\
125.01	0\\
126.01	0\\
127.01	0\\
128.01	0\\
129.01	0\\
130.01	0\\
131.01	0\\
132.01	0\\
133.01	0\\
134.01	0\\
135.01	0\\
136.01	0\\
137.01	0\\
138.01	0\\
139.01	0\\
140.01	0\\
141.01	0\\
142.01	0\\
143.01	0\\
144.01	0\\
145.01	0\\
146.01	0\\
147.01	0\\
148.01	0\\
149.01	0\\
150.01	0\\
151.01	0\\
152.01	0\\
153.01	0\\
154.01	0\\
155.01	0\\
156.01	0\\
157.01	0\\
158.01	0\\
159.01	0\\
160.01	0\\
161.01	0\\
162.01	0\\
163.01	0\\
164.01	0\\
165.01	0\\
166.01	0\\
167.01	0\\
168.01	0\\
169.01	0\\
170.01	0\\
171.01	0\\
172.01	0\\
173.01	0\\
174.01	0\\
175.01	0\\
176.01	0\\
177.01	0\\
178.01	0\\
179.01	0\\
180.01	0\\
181.01	0\\
182.01	0\\
183.01	0\\
184.01	0\\
185.01	0\\
186.01	0\\
187.01	0\\
188.01	0\\
189.01	0\\
190.01	0\\
191.01	0\\
192.01	0\\
193.01	0\\
194.01	0\\
195.01	0\\
196.01	0\\
197.01	0\\
198.01	0\\
199.01	0\\
200.01	0\\
201.01	0\\
202.01	0\\
203.01	0\\
204.01	0\\
205.01	0\\
206.01	0\\
207.01	0\\
208.01	0\\
209.01	0\\
210.01	0\\
211.01	0\\
212.01	0\\
213.01	0\\
214.01	0\\
215.01	0\\
216.01	0\\
217.01	0\\
218.01	0\\
219.01	0\\
220.01	0\\
221.01	0\\
222.01	0\\
223.01	0\\
224.01	0\\
225.01	0\\
226.01	0\\
227.01	0\\
228.01	0\\
229.01	0\\
230.01	0\\
231.01	0\\
232.01	0\\
233.01	0\\
234.01	0\\
235.01	0\\
236.01	0\\
237.01	0\\
238.01	0\\
239.01	0\\
240.01	0\\
241.01	0\\
242.01	0\\
243.01	0\\
244.01	0\\
245.01	0\\
246.01	0\\
247.01	0\\
248.01	0\\
249.01	0\\
250.01	0\\
251.01	0\\
252.01	0\\
253.01	0\\
254.01	0\\
255.01	0\\
256.01	0\\
257.01	0\\
258.01	0\\
259.01	0\\
260.01	0\\
261.01	0\\
262.01	0\\
263.01	0\\
264.01	0\\
265.01	0\\
266.01	0\\
267.01	0\\
268.01	0\\
269.01	0\\
270.01	0\\
271.01	0\\
272.01	0\\
273.01	0\\
274.01	0\\
275.01	0\\
276.01	0\\
277.01	0\\
278.01	0\\
279.01	0\\
280.01	0\\
281.01	0\\
282.01	0\\
283.01	0\\
284.01	0\\
285.01	0\\
286.01	0\\
287.01	0\\
288.01	0\\
289.01	0\\
290.01	0\\
291.01	0\\
292.01	0\\
293.01	0\\
294.01	0\\
295.01	0\\
296.01	0\\
297.01	0\\
298.01	0\\
299.01	0\\
300.01	0\\
301.01	0\\
302.01	0\\
303.01	0\\
304.01	0\\
305.01	0\\
306.01	0\\
307.01	0\\
308.01	0\\
309.01	0\\
310.01	0\\
311.01	0\\
312.01	0\\
313.01	0\\
314.01	0\\
315.01	0\\
316.01	0\\
317.01	0\\
318.01	0\\
319.01	0\\
320.01	0\\
321.01	0\\
322.01	0\\
323.01	0\\
324.01	0\\
325.01	0\\
326.01	0\\
327.01	0\\
328.01	0\\
329.01	0\\
330.01	0\\
331.01	0\\
332.01	0\\
333.01	0\\
334.01	0\\
335.01	0\\
336.01	0\\
337.01	0\\
338.01	0\\
339.01	0\\
340.01	0\\
341.01	0\\
342.01	0\\
343.01	0\\
344.01	0\\
345.01	0\\
346.01	0\\
347.01	0\\
348.01	0\\
349.01	0\\
350.01	0\\
351.01	0\\
352.01	0\\
353.01	0\\
354.01	0\\
355.01	0\\
356.01	0\\
357.01	0\\
358.01	0\\
359.01	0\\
360.01	0\\
361.01	0\\
362.01	0\\
363.01	0\\
364.01	0\\
365.01	0\\
366.01	0\\
367.01	0\\
368.01	0\\
369.01	0\\
370.01	0\\
371.01	0\\
372.01	0\\
373.01	0\\
374.01	0\\
375.01	0\\
376.01	0\\
377.01	0\\
378.01	0\\
379.01	0\\
380.01	0\\
381.01	0\\
382.01	0\\
383.01	0\\
384.01	0\\
385.01	0\\
386.01	0\\
387.01	0\\
388.01	0\\
389.01	0\\
390.01	0\\
391.01	0\\
392.01	0\\
393.01	0\\
394.01	0\\
395.01	0\\
396.01	0\\
397.01	0\\
398.01	0\\
399.01	0\\
400.01	0\\
401.01	0\\
402.01	0\\
403.01	0\\
404.01	0\\
405.01	0\\
406.01	0\\
407.01	0\\
408.01	0\\
409.01	0\\
410.01	0\\
411.01	0\\
412.01	0\\
413.01	0\\
414.01	0\\
415.01	0\\
416.01	0\\
417.01	0\\
418.01	0\\
419.01	0\\
420.01	0\\
421.01	0\\
422.01	0\\
423.01	0\\
424.01	0\\
425.01	0\\
426.01	0\\
427.01	0\\
428.01	0\\
429.01	0\\
430.01	0\\
431.01	0\\
432.01	0\\
433.01	0\\
434.01	0\\
435.01	0\\
436.01	0\\
437.01	0\\
438.01	0\\
439.01	0\\
440.01	0\\
441.01	0\\
442.01	0\\
443.01	0\\
444.01	0\\
445.01	0\\
446.01	0\\
447.01	0\\
448.01	0\\
449.01	0\\
450.01	0\\
451.01	0\\
452.01	0\\
453.01	0\\
454.01	0\\
455.01	0\\
456.01	0\\
457.01	0\\
458.01	0\\
459.01	0\\
460.01	0\\
461.01	0\\
462.01	0\\
463.01	0\\
464.01	0\\
465.01	0\\
466.01	0\\
467.01	0\\
468.01	0\\
469.01	0\\
470.01	0\\
471.01	0\\
472.01	0\\
473.01	0\\
474.01	0\\
475.01	0\\
476.01	0\\
477.01	0\\
478.01	0\\
479.01	0\\
480.01	0\\
481.01	0\\
482.01	0\\
483.01	0\\
484.01	0\\
485.01	0\\
486.01	0\\
487.01	0\\
488.01	0\\
489.01	0\\
490.01	0\\
491.01	0\\
492.01	0\\
493.01	0\\
494.01	0\\
495.01	0\\
496.01	0\\
497.01	0\\
498.01	0\\
499.01	0\\
500.01	0\\
501.01	0\\
502.01	0\\
503.01	0\\
504.01	0\\
505.01	0\\
506.01	0\\
507.01	0\\
508.01	0\\
509.01	0\\
510.01	0\\
511.01	0\\
512.01	0\\
513.01	0\\
514.01	0\\
515.01	0\\
516.01	0\\
517.01	0\\
518.01	0\\
519.01	0\\
520.01	0\\
521.01	0\\
522.01	0\\
523.01	0\\
524.01	0\\
525.01	0\\
526.01	0\\
527.01	0\\
528.01	0\\
529.01	0\\
530.01	0\\
531.01	0\\
532.01	0\\
533.01	0\\
534.01	0\\
535.01	0\\
536.01	0\\
537.01	0\\
538.01	0\\
539.01	0\\
540.01	0\\
541.01	0\\
542.01	0\\
543.01	0\\
544.01	0\\
545.01	0\\
546.01	0\\
547.01	0\\
548.01	0\\
549.01	0\\
550.01	0\\
551.01	0\\
552.01	0\\
553.01	0\\
554.01	0\\
555.01	0\\
556.01	0\\
557.01	0\\
558.01	0\\
559.01	0\\
560.01	0\\
561.01	0\\
562.01	0\\
563.01	0\\
564.01	0\\
565.01	0\\
566.01	0\\
567.01	0\\
568.01	0\\
569.01	0\\
570.01	0\\
571.01	0\\
572.01	0\\
573.01	0\\
574.01	0\\
575.01	0\\
576.01	0\\
577.01	0\\
578.01	0\\
579.01	0\\
580.01	0\\
581.01	0\\
582.01	0\\
583.01	0\\
584.01	0\\
585.01	0\\
586.01	0\\
587.01	0\\
588.01	0\\
589.01	0\\
590.01	0\\
591.01	0\\
592.01	0\\
593.01	0\\
594.01	0\\
595.01	0\\
596.01	0\\
597.01	0\\
598.01	0.00143832068163977\\
599.01	0.00385661257211143\\
599.02	0.00389414818834777\\
599.03	0.00393204146008657\\
599.04	0.00397029584139656\\
599.05	0.00400891481968466\\
599.06	0.00404790191601751\\
599.07	0.00408726068544609\\
599.08	0.00412699471733344\\
599.09	0.00416710763568554\\
599.1	0.0042076030994854\\
599.11	0.00424848480303038\\
599.12	0.00428975647627273\\
599.13	0.0043314218851634\\
599.14	0.00437348483199923\\
599.15	0.00441594915577338\\
599.16	0.00445881873252926\\
599.17	0.00450209747571773\\
599.18	0.00454578933655789\\
599.19	0.00458989830440124\\
599.2	0.00463442840709936\\
599.21	0.00467938371137522\\
599.22	0.0047247683231979\\
599.23	0.00477058638816113\\
599.24	0.00481684209186529\\
599.25	0.00486353966030319\\
599.26	0.0049106833495528\\
599.27	0.00495827744958357\\
599.28	0.00500632629158696\\
599.29	0.00505483424837222\\
599.3	0.00510380573476581\\
599.31	0.00515324520801481\\
599.32	0.00520315716819406\\
599.33	0.00525354615861728\\
599.34	0.00530441676625217\\
599.35	0.00535577362213935\\
599.36	0.00540762140181545\\
599.37	0.00545996482574022\\
599.38	0.0055128086597277\\
599.39	0.00556615771538157\\
599.4	0.00562001685053463\\
599.41	0.00567439096969254\\
599.42	0.00572928502448181\\
599.43	0.00578470401410199\\
599.44	0.00584065298578238\\
599.45	0.00589713703524292\\
599.46	0.00595416130715964\\
599.47	0.00601173099563451\\
599.48	0.00606985134466982\\
599.49	0.00612852764864707\\
599.5	0.0061877652528105\\
599.51	0.00624756955375522\\
599.52	0.00630794599992002\\
599.53	0.00636890009208498\\
599.54	0.00643043738387368\\
599.55	0.00649256348226043\\
599.56	0.0065552840480822\\
599.57	0.00661860479655558\\
599.58	0.00668253149779855\\
599.59	0.00674706997735745\\
599.6	0.00681222611673879\\
599.61	0.00687800585394631\\
599.62	0.00694441518402313\\
599.63	0.00701146015959912\\
599.64	0.00707914689144344\\
599.65	0.00714748154902252\\
599.66	0.00721647036106323\\
599.67	0.00728611961612157\\
599.68	0.00735643566315675\\
599.69	0.00742742491211077\\
599.7	0.00749909383449363\\
599.71	0.00757144896397399\\
599.72	0.0076444968969757\\
599.73	0.00771824429327989\\
599.74	0.00779269787663284\\
599.75	0.00786786443535982\\
599.76	0.0079437508229846\\
599.77	0.00802036395885509\\
599.78	0.00809771082877485\\
599.79	0.00817579848564071\\
599.8	0.00825463405008648\\
599.81	0.00833422471113285\\
599.82	0.00841457772684349\\
599.83	0.00849570042498746\\
599.84	0.00857760020370797\\
599.85	0.00866028453219752\\
599.86	0.00874376095137953\\
599.87	0.00882803707459654\\
599.88	0.0089131205883049\\
599.89	0.00899901925277625\\
599.9	0.00908574090280562\\
599.91	0.00917329344842636\\
599.92	0.0092616848756319\\
599.93	0.00935092324710449\\
599.94	0.00944101670295078\\
599.95	0.00953197346144465\\
599.96	0.00962380181977693\\
599.97	0.00971651015481255\\
599.98	0.0098101069238547\\
599.99	0.00990460066541651\\
600	0.01\\
};
\addplot [color=red!50!mycolor17,solid,forget plot]
  table[row sep=crcr]{%
0.01	0\\
1.01	0\\
2.01	0\\
3.01	0\\
4.01	0\\
5.01	0\\
6.01	0\\
7.01	0\\
8.01	0\\
9.01	0\\
10.01	0\\
11.01	0\\
12.01	0\\
13.01	0\\
14.01	0\\
15.01	0\\
16.01	0\\
17.01	0\\
18.01	0\\
19.01	0\\
20.01	0\\
21.01	0\\
22.01	0\\
23.01	0\\
24.01	0\\
25.01	0\\
26.01	0\\
27.01	0\\
28.01	0\\
29.01	0\\
30.01	0\\
31.01	0\\
32.01	0\\
33.01	0\\
34.01	0\\
35.01	0\\
36.01	0\\
37.01	0\\
38.01	0\\
39.01	0\\
40.01	0\\
41.01	0\\
42.01	0\\
43.01	0\\
44.01	0\\
45.01	0\\
46.01	0\\
47.01	0\\
48.01	0\\
49.01	0\\
50.01	0\\
51.01	0\\
52.01	0\\
53.01	0\\
54.01	0\\
55.01	0\\
56.01	0\\
57.01	0\\
58.01	0\\
59.01	0\\
60.01	0\\
61.01	0\\
62.01	0\\
63.01	0\\
64.01	0\\
65.01	0\\
66.01	0\\
67.01	0\\
68.01	0\\
69.01	0\\
70.01	0\\
71.01	0\\
72.01	0\\
73.01	0\\
74.01	0\\
75.01	0\\
76.01	0\\
77.01	0\\
78.01	0\\
79.01	0\\
80.01	0\\
81.01	0\\
82.01	0\\
83.01	0\\
84.01	0\\
85.01	0\\
86.01	0\\
87.01	0\\
88.01	0\\
89.01	0\\
90.01	0\\
91.01	0\\
92.01	0\\
93.01	0\\
94.01	0\\
95.01	0\\
96.01	0\\
97.01	0\\
98.01	0\\
99.01	0\\
100.01	0\\
101.01	0\\
102.01	0\\
103.01	0\\
104.01	0\\
105.01	0\\
106.01	0\\
107.01	0\\
108.01	0\\
109.01	0\\
110.01	0\\
111.01	0\\
112.01	0\\
113.01	0\\
114.01	0\\
115.01	0\\
116.01	0\\
117.01	0\\
118.01	0\\
119.01	0\\
120.01	0\\
121.01	0\\
122.01	0\\
123.01	0\\
124.01	0\\
125.01	0\\
126.01	0\\
127.01	0\\
128.01	0\\
129.01	0\\
130.01	0\\
131.01	0\\
132.01	0\\
133.01	0\\
134.01	0\\
135.01	0\\
136.01	0\\
137.01	0\\
138.01	0\\
139.01	0\\
140.01	0\\
141.01	0\\
142.01	0\\
143.01	0\\
144.01	0\\
145.01	0\\
146.01	0\\
147.01	0\\
148.01	0\\
149.01	0\\
150.01	0\\
151.01	0\\
152.01	0\\
153.01	0\\
154.01	0\\
155.01	0\\
156.01	0\\
157.01	0\\
158.01	0\\
159.01	0\\
160.01	0\\
161.01	0\\
162.01	0\\
163.01	0\\
164.01	0\\
165.01	0\\
166.01	0\\
167.01	0\\
168.01	0\\
169.01	0\\
170.01	0\\
171.01	0\\
172.01	0\\
173.01	0\\
174.01	0\\
175.01	0\\
176.01	0\\
177.01	0\\
178.01	0\\
179.01	0\\
180.01	0\\
181.01	0\\
182.01	0\\
183.01	0\\
184.01	0\\
185.01	0\\
186.01	0\\
187.01	0\\
188.01	0\\
189.01	0\\
190.01	0\\
191.01	0\\
192.01	0\\
193.01	0\\
194.01	0\\
195.01	0\\
196.01	0\\
197.01	0\\
198.01	0\\
199.01	0\\
200.01	0\\
201.01	0\\
202.01	0\\
203.01	0\\
204.01	0\\
205.01	0\\
206.01	0\\
207.01	0\\
208.01	0\\
209.01	0\\
210.01	0\\
211.01	0\\
212.01	0\\
213.01	0\\
214.01	0\\
215.01	0\\
216.01	0\\
217.01	0\\
218.01	0\\
219.01	0\\
220.01	0\\
221.01	0\\
222.01	0\\
223.01	0\\
224.01	0\\
225.01	0\\
226.01	0\\
227.01	0\\
228.01	0\\
229.01	0\\
230.01	0\\
231.01	0\\
232.01	0\\
233.01	0\\
234.01	0\\
235.01	0\\
236.01	0\\
237.01	0\\
238.01	0\\
239.01	0\\
240.01	0\\
241.01	0\\
242.01	0\\
243.01	0\\
244.01	0\\
245.01	0\\
246.01	0\\
247.01	0\\
248.01	0\\
249.01	0\\
250.01	0\\
251.01	0\\
252.01	0\\
253.01	0\\
254.01	0\\
255.01	0\\
256.01	0\\
257.01	0\\
258.01	0\\
259.01	0\\
260.01	0\\
261.01	0\\
262.01	0\\
263.01	0\\
264.01	0\\
265.01	0\\
266.01	0\\
267.01	0\\
268.01	0\\
269.01	0\\
270.01	0\\
271.01	0\\
272.01	0\\
273.01	0\\
274.01	0\\
275.01	0\\
276.01	0\\
277.01	0\\
278.01	0\\
279.01	0\\
280.01	0\\
281.01	0\\
282.01	0\\
283.01	0\\
284.01	0\\
285.01	0\\
286.01	0\\
287.01	0\\
288.01	0\\
289.01	0\\
290.01	0\\
291.01	0\\
292.01	0\\
293.01	0\\
294.01	0\\
295.01	0\\
296.01	0\\
297.01	0\\
298.01	0\\
299.01	0\\
300.01	0\\
301.01	0\\
302.01	0\\
303.01	0\\
304.01	0\\
305.01	0\\
306.01	0\\
307.01	0\\
308.01	0\\
309.01	0\\
310.01	0\\
311.01	0\\
312.01	0\\
313.01	0\\
314.01	0\\
315.01	0\\
316.01	0\\
317.01	0\\
318.01	0\\
319.01	0\\
320.01	0\\
321.01	0\\
322.01	0\\
323.01	0\\
324.01	0\\
325.01	0\\
326.01	0\\
327.01	0\\
328.01	0\\
329.01	0\\
330.01	0\\
331.01	0\\
332.01	0\\
333.01	0\\
334.01	0\\
335.01	0\\
336.01	0\\
337.01	0\\
338.01	0\\
339.01	0\\
340.01	0\\
341.01	0\\
342.01	0\\
343.01	0\\
344.01	0\\
345.01	0\\
346.01	0\\
347.01	0\\
348.01	0\\
349.01	0\\
350.01	0\\
351.01	0\\
352.01	0\\
353.01	0\\
354.01	0\\
355.01	0\\
356.01	0\\
357.01	0\\
358.01	0\\
359.01	0\\
360.01	0\\
361.01	0\\
362.01	0\\
363.01	0\\
364.01	0\\
365.01	0\\
366.01	0\\
367.01	0\\
368.01	0\\
369.01	0\\
370.01	0\\
371.01	0\\
372.01	0\\
373.01	0\\
374.01	0\\
375.01	0\\
376.01	0\\
377.01	0\\
378.01	0\\
379.01	0\\
380.01	0\\
381.01	0\\
382.01	0\\
383.01	0\\
384.01	0\\
385.01	0\\
386.01	0\\
387.01	0\\
388.01	0\\
389.01	0\\
390.01	0\\
391.01	0\\
392.01	0\\
393.01	0\\
394.01	0\\
395.01	0\\
396.01	0\\
397.01	0\\
398.01	0\\
399.01	0\\
400.01	0\\
401.01	0\\
402.01	0\\
403.01	0\\
404.01	0\\
405.01	0\\
406.01	0\\
407.01	0\\
408.01	0\\
409.01	0\\
410.01	0\\
411.01	0\\
412.01	0\\
413.01	0\\
414.01	0\\
415.01	0\\
416.01	0\\
417.01	0\\
418.01	0\\
419.01	0\\
420.01	0\\
421.01	0\\
422.01	0\\
423.01	0\\
424.01	0\\
425.01	0\\
426.01	0\\
427.01	0\\
428.01	0\\
429.01	0\\
430.01	0\\
431.01	0\\
432.01	0\\
433.01	0\\
434.01	0\\
435.01	0\\
436.01	0\\
437.01	0\\
438.01	0\\
439.01	0\\
440.01	0\\
441.01	0\\
442.01	0\\
443.01	0\\
444.01	0\\
445.01	0\\
446.01	0\\
447.01	0\\
448.01	0\\
449.01	0\\
450.01	0\\
451.01	0\\
452.01	0\\
453.01	0\\
454.01	0\\
455.01	0\\
456.01	0\\
457.01	0\\
458.01	0\\
459.01	0\\
460.01	0\\
461.01	0\\
462.01	0\\
463.01	0\\
464.01	0\\
465.01	0\\
466.01	0\\
467.01	0\\
468.01	0\\
469.01	0\\
470.01	0\\
471.01	0\\
472.01	0\\
473.01	0\\
474.01	0\\
475.01	0\\
476.01	0\\
477.01	0\\
478.01	0\\
479.01	0\\
480.01	0\\
481.01	0\\
482.01	0\\
483.01	0\\
484.01	0\\
485.01	0\\
486.01	0\\
487.01	0\\
488.01	0\\
489.01	0\\
490.01	0\\
491.01	0\\
492.01	0\\
493.01	0\\
494.01	0\\
495.01	0\\
496.01	0\\
497.01	0\\
498.01	0\\
499.01	0\\
500.01	0\\
501.01	0\\
502.01	0\\
503.01	0\\
504.01	0\\
505.01	0\\
506.01	0\\
507.01	0\\
508.01	0\\
509.01	0\\
510.01	0\\
511.01	0\\
512.01	0\\
513.01	0\\
514.01	0\\
515.01	0\\
516.01	0\\
517.01	0\\
518.01	0\\
519.01	0\\
520.01	0\\
521.01	0\\
522.01	0\\
523.01	0\\
524.01	0\\
525.01	0\\
526.01	0\\
527.01	0\\
528.01	0\\
529.01	0\\
530.01	0\\
531.01	0\\
532.01	0\\
533.01	0\\
534.01	0\\
535.01	0\\
536.01	0\\
537.01	0\\
538.01	0\\
539.01	0\\
540.01	0\\
541.01	0\\
542.01	0\\
543.01	0\\
544.01	0\\
545.01	0\\
546.01	0\\
547.01	0\\
548.01	0\\
549.01	0\\
550.01	0\\
551.01	0\\
552.01	0\\
553.01	0\\
554.01	0\\
555.01	0\\
556.01	0\\
557.01	0\\
558.01	0\\
559.01	0\\
560.01	0\\
561.01	0\\
562.01	0\\
563.01	0\\
564.01	0\\
565.01	0\\
566.01	0\\
567.01	0\\
568.01	0\\
569.01	0\\
570.01	0\\
571.01	0\\
572.01	0\\
573.01	0\\
574.01	0\\
575.01	0\\
576.01	0\\
577.01	0\\
578.01	0\\
579.01	0\\
580.01	0\\
581.01	0\\
582.01	0\\
583.01	0\\
584.01	0\\
585.01	0\\
586.01	0\\
587.01	0\\
588.01	0\\
589.01	0\\
590.01	0\\
591.01	0\\
592.01	0\\
593.01	0\\
594.01	0\\
595.01	0\\
596.01	0\\
597.01	0.000469956016972223\\
598.01	0.00143832068184645\\
599.01	0.00385661257211617\\
599.02	0.00389414818835224\\
599.03	0.00393204146009079\\
599.04	0.00397029584140053\\
599.05	0.0040089148196884\\
599.06	0.00404790191602104\\
599.07	0.00408726068544941\\
599.08	0.00412699471733656\\
599.09	0.00416710763568846\\
599.1	0.00420760309948814\\
599.11	0.00424848480303295\\
599.12	0.00428975647627514\\
599.13	0.00433142188516566\\
599.14	0.00437348483200135\\
599.15	0.00441594915577536\\
599.16	0.00445881873253109\\
599.17	0.00450209747571945\\
599.18	0.00454578933655949\\
599.19	0.00458989830440273\\
599.2	0.00463442840710074\\
599.21	0.00467938371137649\\
599.22	0.00472476832319909\\
599.23	0.00477058638816223\\
599.24	0.0048168420918663\\
599.25	0.00486353966030412\\
599.26	0.00491068334955366\\
599.27	0.00495827744958435\\
599.28	0.00500632629158769\\
599.29	0.00505483424837289\\
599.3	0.00510380573476642\\
599.31	0.00515324520801536\\
599.32	0.00520315716819456\\
599.33	0.00525354615861774\\
599.34	0.00530441676625258\\
599.35	0.00535577362213972\\
599.36	0.00540762140181579\\
599.37	0.00545996482574053\\
599.38	0.00551280865972798\\
599.39	0.00556615771538182\\
599.4	0.00562001685053486\\
599.41	0.00567439096969275\\
599.42	0.00572928502448199\\
599.43	0.00578470401410216\\
599.44	0.00584065298578252\\
599.45	0.00589713703524304\\
599.46	0.00595416130715974\\
599.47	0.0060117309956346\\
599.48	0.0060698513446699\\
599.49	0.00612852764864714\\
599.5	0.00618776525281056\\
599.51	0.00624756955375527\\
599.52	0.00630794599992007\\
599.53	0.00636890009208501\\
599.54	0.00643043738387371\\
599.55	0.00649256348226045\\
599.56	0.00655528404808222\\
599.57	0.00661860479655559\\
599.58	0.00668253149779856\\
599.59	0.00674706997735745\\
599.6	0.00681222611673878\\
599.61	0.0068780058539463\\
599.62	0.00694441518402313\\
599.63	0.00701146015959912\\
599.64	0.00707914689144345\\
599.65	0.00714748154902253\\
599.66	0.00721647036106323\\
599.67	0.00728611961612158\\
599.68	0.00735643566315675\\
599.69	0.00742742491211078\\
599.7	0.00749909383449363\\
599.71	0.007571448963974\\
599.72	0.00764449689697571\\
599.73	0.00771824429327989\\
599.74	0.00779269787663284\\
599.75	0.00786786443535982\\
599.76	0.0079437508229846\\
599.77	0.0080203639588551\\
599.78	0.00809771082877486\\
599.79	0.00817579848564071\\
599.8	0.00825463405008648\\
599.81	0.00833422471113285\\
599.82	0.00841457772684349\\
599.83	0.00849570042498747\\
599.84	0.00857760020370797\\
599.85	0.00866028453219752\\
599.86	0.00874376095137953\\
599.87	0.00882803707459654\\
599.88	0.0089131205883049\\
599.89	0.00899901925277625\\
599.9	0.00908574090280562\\
599.91	0.00917329344842635\\
599.92	0.0092616848756319\\
599.93	0.00935092324710449\\
599.94	0.00944101670295078\\
599.95	0.00953197346144465\\
599.96	0.00962380181977693\\
599.97	0.00971651015481255\\
599.98	0.0098101069238547\\
599.99	0.00990460066541651\\
600	0.01\\
};
\addplot [color=red!40!mycolor19,solid,forget plot]
  table[row sep=crcr]{%
0.01	0\\
1.01	0\\
2.01	0\\
3.01	0\\
4.01	0\\
5.01	0\\
6.01	0\\
7.01	0\\
8.01	0\\
9.01	0\\
10.01	0\\
11.01	0\\
12.01	0\\
13.01	0\\
14.01	0\\
15.01	0\\
16.01	0\\
17.01	0\\
18.01	0\\
19.01	0\\
20.01	0\\
21.01	0\\
22.01	0\\
23.01	0\\
24.01	0\\
25.01	0\\
26.01	0\\
27.01	0\\
28.01	0\\
29.01	0\\
30.01	0\\
31.01	0\\
32.01	0\\
33.01	0\\
34.01	0\\
35.01	0\\
36.01	0\\
37.01	0\\
38.01	0\\
39.01	0\\
40.01	0\\
41.01	0\\
42.01	0\\
43.01	0\\
44.01	0\\
45.01	0\\
46.01	0\\
47.01	0\\
48.01	0\\
49.01	0\\
50.01	0\\
51.01	0\\
52.01	0\\
53.01	0\\
54.01	0\\
55.01	0\\
56.01	0\\
57.01	0\\
58.01	0\\
59.01	0\\
60.01	0\\
61.01	0\\
62.01	0\\
63.01	0\\
64.01	0\\
65.01	0\\
66.01	0\\
67.01	0\\
68.01	0\\
69.01	0\\
70.01	0\\
71.01	0\\
72.01	0\\
73.01	0\\
74.01	0\\
75.01	0\\
76.01	0\\
77.01	0\\
78.01	0\\
79.01	0\\
80.01	0\\
81.01	0\\
82.01	0\\
83.01	0\\
84.01	0\\
85.01	0\\
86.01	0\\
87.01	0\\
88.01	0\\
89.01	0\\
90.01	0\\
91.01	0\\
92.01	0\\
93.01	0\\
94.01	0\\
95.01	0\\
96.01	0\\
97.01	0\\
98.01	0\\
99.01	0\\
100.01	0\\
101.01	0\\
102.01	0\\
103.01	0\\
104.01	0\\
105.01	0\\
106.01	0\\
107.01	0\\
108.01	0\\
109.01	0\\
110.01	0\\
111.01	0\\
112.01	0\\
113.01	0\\
114.01	0\\
115.01	0\\
116.01	0\\
117.01	0\\
118.01	0\\
119.01	0\\
120.01	0\\
121.01	0\\
122.01	0\\
123.01	0\\
124.01	0\\
125.01	0\\
126.01	0\\
127.01	0\\
128.01	0\\
129.01	0\\
130.01	0\\
131.01	0\\
132.01	0\\
133.01	0\\
134.01	0\\
135.01	0\\
136.01	0\\
137.01	0\\
138.01	0\\
139.01	0\\
140.01	0\\
141.01	0\\
142.01	0\\
143.01	0\\
144.01	0\\
145.01	0\\
146.01	0\\
147.01	0\\
148.01	0\\
149.01	0\\
150.01	0\\
151.01	0\\
152.01	0\\
153.01	0\\
154.01	0\\
155.01	0\\
156.01	0\\
157.01	0\\
158.01	0\\
159.01	0\\
160.01	0\\
161.01	0\\
162.01	0\\
163.01	0\\
164.01	0\\
165.01	0\\
166.01	0\\
167.01	0\\
168.01	0\\
169.01	0\\
170.01	0\\
171.01	0\\
172.01	0\\
173.01	0\\
174.01	0\\
175.01	0\\
176.01	0\\
177.01	0\\
178.01	0\\
179.01	0\\
180.01	0\\
181.01	0\\
182.01	0\\
183.01	0\\
184.01	0\\
185.01	0\\
186.01	0\\
187.01	0\\
188.01	0\\
189.01	0\\
190.01	0\\
191.01	0\\
192.01	0\\
193.01	0\\
194.01	0\\
195.01	0\\
196.01	0\\
197.01	0\\
198.01	0\\
199.01	0\\
200.01	0\\
201.01	0\\
202.01	0\\
203.01	0\\
204.01	0\\
205.01	0\\
206.01	0\\
207.01	0\\
208.01	0\\
209.01	0\\
210.01	0\\
211.01	0\\
212.01	0\\
213.01	0\\
214.01	0\\
215.01	0\\
216.01	0\\
217.01	0\\
218.01	0\\
219.01	0\\
220.01	0\\
221.01	0\\
222.01	0\\
223.01	0\\
224.01	0\\
225.01	0\\
226.01	0\\
227.01	0\\
228.01	0\\
229.01	0\\
230.01	0\\
231.01	0\\
232.01	0\\
233.01	0\\
234.01	0\\
235.01	0\\
236.01	0\\
237.01	0\\
238.01	0\\
239.01	0\\
240.01	0\\
241.01	0\\
242.01	0\\
243.01	0\\
244.01	0\\
245.01	0\\
246.01	0\\
247.01	0\\
248.01	0\\
249.01	0\\
250.01	0\\
251.01	0\\
252.01	0\\
253.01	0\\
254.01	0\\
255.01	0\\
256.01	0\\
257.01	0\\
258.01	0\\
259.01	0\\
260.01	0\\
261.01	0\\
262.01	0\\
263.01	0\\
264.01	0\\
265.01	0\\
266.01	0\\
267.01	0\\
268.01	0\\
269.01	0\\
270.01	0\\
271.01	0\\
272.01	0\\
273.01	0\\
274.01	0\\
275.01	0\\
276.01	0\\
277.01	0\\
278.01	0\\
279.01	0\\
280.01	0\\
281.01	0\\
282.01	0\\
283.01	0\\
284.01	0\\
285.01	0\\
286.01	0\\
287.01	0\\
288.01	0\\
289.01	0\\
290.01	0\\
291.01	0\\
292.01	0\\
293.01	0\\
294.01	0\\
295.01	0\\
296.01	0\\
297.01	0\\
298.01	0\\
299.01	0\\
300.01	0\\
301.01	0\\
302.01	0\\
303.01	0\\
304.01	0\\
305.01	0\\
306.01	0\\
307.01	0\\
308.01	0\\
309.01	0\\
310.01	0\\
311.01	0\\
312.01	0\\
313.01	0\\
314.01	0\\
315.01	0\\
316.01	0\\
317.01	0\\
318.01	0\\
319.01	0\\
320.01	0\\
321.01	0\\
322.01	0\\
323.01	0\\
324.01	0\\
325.01	0\\
326.01	0\\
327.01	0\\
328.01	0\\
329.01	0\\
330.01	0\\
331.01	0\\
332.01	0\\
333.01	0\\
334.01	0\\
335.01	0\\
336.01	0\\
337.01	0\\
338.01	0\\
339.01	0\\
340.01	0\\
341.01	0\\
342.01	0\\
343.01	0\\
344.01	0\\
345.01	0\\
346.01	0\\
347.01	0\\
348.01	0\\
349.01	0\\
350.01	0\\
351.01	0\\
352.01	0\\
353.01	0\\
354.01	0\\
355.01	0\\
356.01	0\\
357.01	0\\
358.01	0\\
359.01	0\\
360.01	0\\
361.01	0\\
362.01	0\\
363.01	0\\
364.01	0\\
365.01	0\\
366.01	0\\
367.01	0\\
368.01	0\\
369.01	0\\
370.01	0\\
371.01	0\\
372.01	0\\
373.01	0\\
374.01	0\\
375.01	0\\
376.01	0\\
377.01	0\\
378.01	0\\
379.01	0\\
380.01	0\\
381.01	0\\
382.01	0\\
383.01	0\\
384.01	0\\
385.01	0\\
386.01	0\\
387.01	0\\
388.01	0\\
389.01	0\\
390.01	0\\
391.01	0\\
392.01	0\\
393.01	0\\
394.01	0\\
395.01	0\\
396.01	0\\
397.01	0\\
398.01	0\\
399.01	0\\
400.01	0\\
401.01	0\\
402.01	0\\
403.01	0\\
404.01	0\\
405.01	0\\
406.01	0\\
407.01	0\\
408.01	0\\
409.01	0\\
410.01	0\\
411.01	0\\
412.01	0\\
413.01	0\\
414.01	0\\
415.01	0\\
416.01	0\\
417.01	0\\
418.01	0\\
419.01	0\\
420.01	0\\
421.01	0\\
422.01	0\\
423.01	0\\
424.01	0\\
425.01	0\\
426.01	0\\
427.01	0\\
428.01	0\\
429.01	0\\
430.01	0\\
431.01	0\\
432.01	0\\
433.01	0\\
434.01	0\\
435.01	0\\
436.01	0\\
437.01	0\\
438.01	0\\
439.01	0\\
440.01	0\\
441.01	0\\
442.01	0\\
443.01	0\\
444.01	0\\
445.01	0\\
446.01	0\\
447.01	0\\
448.01	0\\
449.01	0\\
450.01	0\\
451.01	0\\
452.01	0\\
453.01	0\\
454.01	0\\
455.01	0\\
456.01	0\\
457.01	0\\
458.01	0\\
459.01	0\\
460.01	0\\
461.01	0\\
462.01	0\\
463.01	0\\
464.01	0\\
465.01	0\\
466.01	0\\
467.01	0\\
468.01	0\\
469.01	0\\
470.01	0\\
471.01	0\\
472.01	0\\
473.01	0\\
474.01	0\\
475.01	0\\
476.01	0\\
477.01	0\\
478.01	0\\
479.01	0\\
480.01	0\\
481.01	0\\
482.01	0\\
483.01	0\\
484.01	0\\
485.01	0\\
486.01	0\\
487.01	0\\
488.01	0\\
489.01	0\\
490.01	0\\
491.01	0\\
492.01	0\\
493.01	0\\
494.01	0\\
495.01	0\\
496.01	0\\
497.01	0\\
498.01	0\\
499.01	0\\
500.01	0\\
501.01	0\\
502.01	0\\
503.01	0\\
504.01	0\\
505.01	0\\
506.01	0\\
507.01	0\\
508.01	0\\
509.01	0\\
510.01	0\\
511.01	0\\
512.01	0\\
513.01	0\\
514.01	0\\
515.01	0\\
516.01	0\\
517.01	0\\
518.01	0\\
519.01	0\\
520.01	0\\
521.01	0\\
522.01	0\\
523.01	0\\
524.01	0\\
525.01	0\\
526.01	0\\
527.01	0\\
528.01	0\\
529.01	0\\
530.01	0\\
531.01	0\\
532.01	0\\
533.01	0\\
534.01	0\\
535.01	0\\
536.01	0\\
537.01	0\\
538.01	0\\
539.01	0\\
540.01	0\\
541.01	0\\
542.01	0\\
543.01	0\\
544.01	0\\
545.01	0\\
546.01	0\\
547.01	0\\
548.01	0\\
549.01	0\\
550.01	0\\
551.01	0\\
552.01	0\\
553.01	0\\
554.01	0\\
555.01	0\\
556.01	0\\
557.01	0\\
558.01	0\\
559.01	0\\
560.01	0\\
561.01	0\\
562.01	0\\
563.01	0\\
564.01	0\\
565.01	0\\
566.01	0\\
567.01	0\\
568.01	0\\
569.01	0\\
570.01	0\\
571.01	0\\
572.01	0\\
573.01	0\\
574.01	0\\
575.01	0\\
576.01	0\\
577.01	0\\
578.01	0\\
579.01	0\\
580.01	0\\
581.01	0\\
582.01	0\\
583.01	0\\
584.01	0\\
585.01	0\\
586.01	0\\
587.01	0\\
588.01	0\\
589.01	0\\
590.01	0\\
591.01	0\\
592.01	0\\
593.01	0\\
594.01	0\\
595.01	0\\
596.01	0\\
597.01	0.000480262484284122\\
598.01	0.00143832068185175\\
599.01	0.00385661257211624\\
599.02	0.00389414818835231\\
599.03	0.00393204146009084\\
599.04	0.00397029584140058\\
599.05	0.00400891481968844\\
599.06	0.00404790191602107\\
599.07	0.00408726068544944\\
599.08	0.00412699471733659\\
599.09	0.00416710763568848\\
599.1	0.00420760309948815\\
599.11	0.00424848480303296\\
599.12	0.00428975647627514\\
599.13	0.00433142188516566\\
599.14	0.00437348483200135\\
599.15	0.00441594915577536\\
599.16	0.0044588187325311\\
599.17	0.00450209747571944\\
599.18	0.00454578933655948\\
599.19	0.00458989830440271\\
599.2	0.00463442840710074\\
599.21	0.00467938371137649\\
599.22	0.00472476832319909\\
599.23	0.00477058638816223\\
599.24	0.0048168420918663\\
599.25	0.00486353966030412\\
599.26	0.00491068334955367\\
599.27	0.00495827744958437\\
599.28	0.0050063262915877\\
599.29	0.00505483424837289\\
599.3	0.00510380573476642\\
599.31	0.00515324520801536\\
599.32	0.00520315716819456\\
599.33	0.00525354615861774\\
599.34	0.00530441676625259\\
599.35	0.00535577362213972\\
599.36	0.00540762140181578\\
599.37	0.00545996482574052\\
599.38	0.00551280865972797\\
599.39	0.0055661577153818\\
599.4	0.00562001685053484\\
599.41	0.00567439096969274\\
599.42	0.00572928502448198\\
599.43	0.00578470401410214\\
599.44	0.00584065298578251\\
599.45	0.00589713703524303\\
599.46	0.00595416130715973\\
599.47	0.0060117309956346\\
599.48	0.0060698513446699\\
599.49	0.00612852764864714\\
599.5	0.00618776525281056\\
599.51	0.00624756955375527\\
599.52	0.00630794599992007\\
599.53	0.00636890009208502\\
599.54	0.00643043738387371\\
599.55	0.00649256348226045\\
599.56	0.00655528404808223\\
599.57	0.0066186047965556\\
599.58	0.00668253149779857\\
599.59	0.00674706997735746\\
599.6	0.0068122261167388\\
599.61	0.00687800585394632\\
599.62	0.00694441518402314\\
599.63	0.00701146015959913\\
599.64	0.00707914689144346\\
599.65	0.00714748154902253\\
599.66	0.00721647036106323\\
599.67	0.00728611961612158\\
599.68	0.00735643566315675\\
599.69	0.00742742491211078\\
599.7	0.00749909383449363\\
599.71	0.007571448963974\\
599.72	0.00764449689697571\\
599.73	0.00771824429327989\\
599.74	0.00779269787663285\\
599.75	0.00786786443535982\\
599.76	0.00794375082298461\\
599.77	0.0080203639588551\\
599.78	0.00809771082877486\\
599.79	0.00817579848564071\\
599.8	0.00825463405008648\\
599.81	0.00833422471113285\\
599.82	0.00841457772684349\\
599.83	0.00849570042498747\\
599.84	0.00857760020370798\\
599.85	0.00866028453219752\\
599.86	0.00874376095137954\\
599.87	0.00882803707459654\\
599.88	0.0089131205883049\\
599.89	0.00899901925277625\\
599.9	0.00908574090280562\\
599.91	0.00917329344842636\\
599.92	0.00926168487563191\\
599.93	0.00935092324710449\\
599.94	0.00944101670295079\\
599.95	0.00953197346144465\\
599.96	0.00962380181977694\\
599.97	0.00971651015481255\\
599.98	0.0098101069238547\\
599.99	0.00990460066541651\\
600	0.01\\
};
\addplot [color=red!75!mycolor17,solid,forget plot]
  table[row sep=crcr]{%
0.01	0\\
1.01	0\\
2.01	0\\
3.01	0\\
4.01	0\\
5.01	0\\
6.01	0\\
7.01	0\\
8.01	0\\
9.01	0\\
10.01	0\\
11.01	0\\
12.01	0\\
13.01	0\\
14.01	0\\
15.01	0\\
16.01	0\\
17.01	0\\
18.01	0\\
19.01	0\\
20.01	0\\
21.01	0\\
22.01	0\\
23.01	0\\
24.01	0\\
25.01	0\\
26.01	0\\
27.01	0\\
28.01	0\\
29.01	0\\
30.01	0\\
31.01	0\\
32.01	0\\
33.01	0\\
34.01	0\\
35.01	0\\
36.01	0\\
37.01	0\\
38.01	0\\
39.01	0\\
40.01	0\\
41.01	0\\
42.01	0\\
43.01	0\\
44.01	0\\
45.01	0\\
46.01	0\\
47.01	0\\
48.01	0\\
49.01	0\\
50.01	0\\
51.01	0\\
52.01	0\\
53.01	0\\
54.01	0\\
55.01	0\\
56.01	0\\
57.01	0\\
58.01	0\\
59.01	0\\
60.01	0\\
61.01	0\\
62.01	0\\
63.01	0\\
64.01	0\\
65.01	0\\
66.01	0\\
67.01	0\\
68.01	0\\
69.01	0\\
70.01	0\\
71.01	0\\
72.01	0\\
73.01	0\\
74.01	0\\
75.01	0\\
76.01	0\\
77.01	0\\
78.01	0\\
79.01	0\\
80.01	0\\
81.01	0\\
82.01	0\\
83.01	0\\
84.01	0\\
85.01	0\\
86.01	0\\
87.01	0\\
88.01	0\\
89.01	0\\
90.01	0\\
91.01	0\\
92.01	0\\
93.01	0\\
94.01	0\\
95.01	0\\
96.01	0\\
97.01	0\\
98.01	0\\
99.01	0\\
100.01	0\\
101.01	0\\
102.01	0\\
103.01	0\\
104.01	0\\
105.01	0\\
106.01	0\\
107.01	0\\
108.01	0\\
109.01	0\\
110.01	0\\
111.01	0\\
112.01	0\\
113.01	0\\
114.01	0\\
115.01	0\\
116.01	0\\
117.01	0\\
118.01	0\\
119.01	0\\
120.01	0\\
121.01	0\\
122.01	0\\
123.01	0\\
124.01	0\\
125.01	0\\
126.01	0\\
127.01	0\\
128.01	0\\
129.01	0\\
130.01	0\\
131.01	0\\
132.01	0\\
133.01	0\\
134.01	0\\
135.01	0\\
136.01	0\\
137.01	0\\
138.01	0\\
139.01	0\\
140.01	0\\
141.01	0\\
142.01	0\\
143.01	0\\
144.01	0\\
145.01	0\\
146.01	0\\
147.01	0\\
148.01	0\\
149.01	0\\
150.01	0\\
151.01	0\\
152.01	0\\
153.01	0\\
154.01	0\\
155.01	0\\
156.01	0\\
157.01	0\\
158.01	0\\
159.01	0\\
160.01	0\\
161.01	0\\
162.01	0\\
163.01	0\\
164.01	0\\
165.01	0\\
166.01	0\\
167.01	0\\
168.01	0\\
169.01	0\\
170.01	0\\
171.01	0\\
172.01	0\\
173.01	0\\
174.01	0\\
175.01	0\\
176.01	0\\
177.01	0\\
178.01	0\\
179.01	0\\
180.01	0\\
181.01	0\\
182.01	0\\
183.01	0\\
184.01	0\\
185.01	0\\
186.01	0\\
187.01	0\\
188.01	0\\
189.01	0\\
190.01	0\\
191.01	0\\
192.01	0\\
193.01	0\\
194.01	0\\
195.01	0\\
196.01	0\\
197.01	0\\
198.01	0\\
199.01	0\\
200.01	0\\
201.01	0\\
202.01	0\\
203.01	0\\
204.01	0\\
205.01	0\\
206.01	0\\
207.01	0\\
208.01	0\\
209.01	0\\
210.01	0\\
211.01	0\\
212.01	0\\
213.01	0\\
214.01	0\\
215.01	0\\
216.01	0\\
217.01	0\\
218.01	0\\
219.01	0\\
220.01	0\\
221.01	0\\
222.01	0\\
223.01	0\\
224.01	0\\
225.01	0\\
226.01	0\\
227.01	0\\
228.01	0\\
229.01	0\\
230.01	0\\
231.01	0\\
232.01	0\\
233.01	0\\
234.01	0\\
235.01	0\\
236.01	0\\
237.01	0\\
238.01	0\\
239.01	0\\
240.01	0\\
241.01	0\\
242.01	0\\
243.01	0\\
244.01	0\\
245.01	0\\
246.01	0\\
247.01	0\\
248.01	0\\
249.01	0\\
250.01	0\\
251.01	0\\
252.01	0\\
253.01	0\\
254.01	0\\
255.01	0\\
256.01	0\\
257.01	0\\
258.01	0\\
259.01	0\\
260.01	0\\
261.01	0\\
262.01	0\\
263.01	0\\
264.01	0\\
265.01	0\\
266.01	0\\
267.01	0\\
268.01	0\\
269.01	0\\
270.01	0\\
271.01	0\\
272.01	0\\
273.01	0\\
274.01	0\\
275.01	0\\
276.01	0\\
277.01	0\\
278.01	0\\
279.01	0\\
280.01	0\\
281.01	0\\
282.01	0\\
283.01	0\\
284.01	0\\
285.01	0\\
286.01	0\\
287.01	0\\
288.01	0\\
289.01	0\\
290.01	0\\
291.01	0\\
292.01	0\\
293.01	0\\
294.01	0\\
295.01	0\\
296.01	0\\
297.01	0\\
298.01	0\\
299.01	0\\
300.01	0\\
301.01	0\\
302.01	0\\
303.01	0\\
304.01	0\\
305.01	0\\
306.01	0\\
307.01	0\\
308.01	0\\
309.01	0\\
310.01	0\\
311.01	0\\
312.01	0\\
313.01	0\\
314.01	0\\
315.01	0\\
316.01	0\\
317.01	0\\
318.01	0\\
319.01	0\\
320.01	0\\
321.01	0\\
322.01	0\\
323.01	0\\
324.01	0\\
325.01	0\\
326.01	0\\
327.01	0\\
328.01	0\\
329.01	0\\
330.01	0\\
331.01	0\\
332.01	0\\
333.01	0\\
334.01	0\\
335.01	0\\
336.01	0\\
337.01	0\\
338.01	0\\
339.01	0\\
340.01	0\\
341.01	0\\
342.01	0\\
343.01	0\\
344.01	0\\
345.01	0\\
346.01	0\\
347.01	0\\
348.01	0\\
349.01	0\\
350.01	0\\
351.01	0\\
352.01	0\\
353.01	0\\
354.01	0\\
355.01	0\\
356.01	0\\
357.01	0\\
358.01	0\\
359.01	0\\
360.01	0\\
361.01	0\\
362.01	0\\
363.01	0\\
364.01	0\\
365.01	0\\
366.01	0\\
367.01	0\\
368.01	0\\
369.01	0\\
370.01	0\\
371.01	0\\
372.01	0\\
373.01	0\\
374.01	0\\
375.01	0\\
376.01	0\\
377.01	0\\
378.01	0\\
379.01	0\\
380.01	0\\
381.01	0\\
382.01	0\\
383.01	0\\
384.01	0\\
385.01	0\\
386.01	0\\
387.01	0\\
388.01	0\\
389.01	0\\
390.01	0\\
391.01	0\\
392.01	0\\
393.01	0\\
394.01	0\\
395.01	0\\
396.01	0\\
397.01	0\\
398.01	0\\
399.01	0\\
400.01	0\\
401.01	0\\
402.01	0\\
403.01	0\\
404.01	0\\
405.01	0\\
406.01	0\\
407.01	0\\
408.01	0\\
409.01	0\\
410.01	0\\
411.01	0\\
412.01	0\\
413.01	0\\
414.01	0\\
415.01	0\\
416.01	0\\
417.01	0\\
418.01	0\\
419.01	0\\
420.01	0\\
421.01	0\\
422.01	0\\
423.01	0\\
424.01	0\\
425.01	0\\
426.01	0\\
427.01	0\\
428.01	0\\
429.01	0\\
430.01	0\\
431.01	0\\
432.01	0\\
433.01	0\\
434.01	0\\
435.01	0\\
436.01	0\\
437.01	0\\
438.01	0\\
439.01	0\\
440.01	0\\
441.01	0\\
442.01	0\\
443.01	0\\
444.01	0\\
445.01	0\\
446.01	0\\
447.01	0\\
448.01	0\\
449.01	0\\
450.01	0\\
451.01	0\\
452.01	0\\
453.01	0\\
454.01	0\\
455.01	0\\
456.01	0\\
457.01	0\\
458.01	0\\
459.01	0\\
460.01	0\\
461.01	0\\
462.01	0\\
463.01	0\\
464.01	0\\
465.01	0\\
466.01	0\\
467.01	0\\
468.01	0\\
469.01	0\\
470.01	0\\
471.01	0\\
472.01	0\\
473.01	0\\
474.01	0\\
475.01	0\\
476.01	0\\
477.01	0\\
478.01	0\\
479.01	0\\
480.01	0\\
481.01	0\\
482.01	0\\
483.01	0\\
484.01	0\\
485.01	0\\
486.01	0\\
487.01	0\\
488.01	0\\
489.01	0\\
490.01	0\\
491.01	0\\
492.01	0\\
493.01	0\\
494.01	0\\
495.01	0\\
496.01	0\\
497.01	0\\
498.01	0\\
499.01	0\\
500.01	0\\
501.01	0\\
502.01	0\\
503.01	0\\
504.01	0\\
505.01	0\\
506.01	0\\
507.01	0\\
508.01	0\\
509.01	0\\
510.01	0\\
511.01	0\\
512.01	0\\
513.01	0\\
514.01	0\\
515.01	0\\
516.01	0\\
517.01	0\\
518.01	0\\
519.01	0\\
520.01	0\\
521.01	0\\
522.01	0\\
523.01	0\\
524.01	0\\
525.01	0\\
526.01	0\\
527.01	0\\
528.01	0\\
529.01	0\\
530.01	0\\
531.01	0\\
532.01	0\\
533.01	0\\
534.01	0\\
535.01	0\\
536.01	0\\
537.01	0\\
538.01	0\\
539.01	0\\
540.01	0\\
541.01	0\\
542.01	0\\
543.01	0\\
544.01	0\\
545.01	0\\
546.01	0\\
547.01	0\\
548.01	0\\
549.01	0\\
550.01	0\\
551.01	0\\
552.01	0\\
553.01	0\\
554.01	0\\
555.01	0\\
556.01	0\\
557.01	0\\
558.01	0\\
559.01	0\\
560.01	0\\
561.01	0\\
562.01	0\\
563.01	0\\
564.01	0\\
565.01	0\\
566.01	0\\
567.01	0\\
568.01	0\\
569.01	0\\
570.01	0\\
571.01	0\\
572.01	0\\
573.01	0\\
574.01	0\\
575.01	0\\
576.01	0\\
577.01	0\\
578.01	0\\
579.01	0\\
580.01	0\\
581.01	0\\
582.01	0\\
583.01	0\\
584.01	0\\
585.01	0\\
586.01	0\\
587.01	0\\
588.01	0\\
589.01	0\\
590.01	0\\
591.01	0\\
592.01	0\\
593.01	0\\
594.01	0\\
595.01	0\\
596.01	0\\
597.01	0.000480601427308949\\
598.01	0.0014383206818518\\
599.01	0.00385661257211624\\
599.02	0.00389414818835231\\
599.03	0.00393204146009085\\
599.04	0.00397029584140059\\
599.05	0.00400891481968844\\
599.06	0.00404790191602107\\
599.07	0.00408726068544946\\
599.08	0.00412699471733659\\
599.09	0.00416710763568849\\
599.1	0.00420760309948817\\
599.11	0.00424848480303298\\
599.12	0.00428975647627516\\
599.13	0.00433142188516568\\
599.14	0.00437348483200135\\
599.15	0.00441594915577537\\
599.16	0.00445881873253111\\
599.17	0.00450209747571945\\
599.18	0.00454578933655949\\
599.19	0.00458989830440273\\
599.2	0.00463442840710075\\
599.21	0.00467938371137651\\
599.22	0.00472476832319909\\
599.23	0.00477058638816223\\
599.24	0.00481684209186631\\
599.25	0.00486353966030413\\
599.26	0.00491068334955367\\
599.27	0.00495827744958437\\
599.28	0.0050063262915877\\
599.29	0.00505483424837289\\
599.3	0.00510380573476643\\
599.31	0.00515324520801537\\
599.32	0.00520315716819457\\
599.33	0.00525354615861775\\
599.34	0.0053044167662526\\
599.35	0.00535577362213973\\
599.36	0.0054076214018158\\
599.37	0.00545996482574054\\
599.38	0.00551280865972798\\
599.39	0.00556615771538182\\
599.4	0.00562001685053486\\
599.41	0.00567439096969275\\
599.42	0.00572928502448199\\
599.43	0.00578470401410217\\
599.44	0.00584065298578253\\
599.45	0.00589713703524306\\
599.46	0.00595416130715976\\
599.47	0.00601173099563462\\
599.48	0.00606985134466991\\
599.49	0.00612852764864715\\
599.5	0.00618776525281057\\
599.51	0.00624756955375528\\
599.52	0.00630794599992008\\
599.53	0.00636890009208502\\
599.54	0.00643043738387372\\
599.55	0.00649256348226046\\
599.56	0.00655528404808223\\
599.57	0.0066186047965556\\
599.58	0.00668253149779857\\
599.59	0.00674706997735747\\
599.6	0.0068122261167388\\
599.61	0.00687800585394632\\
599.62	0.00694441518402314\\
599.63	0.00701146015959912\\
599.64	0.00707914689144345\\
599.65	0.00714748154902253\\
599.66	0.00721647036106324\\
599.67	0.00728611961612158\\
599.68	0.00735643566315675\\
599.69	0.00742742491211078\\
599.7	0.00749909383449363\\
599.71	0.00757144896397399\\
599.72	0.0076444968969757\\
599.73	0.00771824429327989\\
599.74	0.00779269787663285\\
599.75	0.00786786443535982\\
599.76	0.00794375082298461\\
599.77	0.0080203639588551\\
599.78	0.00809771082877485\\
599.79	0.00817579848564071\\
599.8	0.00825463405008648\\
599.81	0.00833422471113285\\
599.82	0.00841457772684349\\
599.83	0.00849570042498746\\
599.84	0.00857760020370797\\
599.85	0.00866028453219752\\
599.86	0.00874376095137953\\
599.87	0.00882803707459653\\
599.88	0.0089131205883049\\
599.89	0.00899901925277625\\
599.9	0.00908574090280562\\
599.91	0.00917329344842636\\
599.92	0.0092616848756319\\
599.93	0.00935092324710449\\
599.94	0.00944101670295079\\
599.95	0.00953197346144465\\
599.96	0.00962380181977694\\
599.97	0.00971651015481255\\
599.98	0.0098101069238547\\
599.99	0.00990460066541651\\
600	0.01\\
};
\addplot [color=red!80!mycolor19,solid,forget plot]
  table[row sep=crcr]{%
0.01	0\\
1.01	0\\
2.01	0\\
3.01	0\\
4.01	0\\
5.01	0\\
6.01	0\\
7.01	0\\
8.01	0\\
9.01	0\\
10.01	0\\
11.01	0\\
12.01	0\\
13.01	0\\
14.01	0\\
15.01	0\\
16.01	0\\
17.01	0\\
18.01	0\\
19.01	0\\
20.01	0\\
21.01	0\\
22.01	0\\
23.01	0\\
24.01	0\\
25.01	0\\
26.01	0\\
27.01	0\\
28.01	0\\
29.01	0\\
30.01	0\\
31.01	0\\
32.01	0\\
33.01	0\\
34.01	0\\
35.01	0\\
36.01	0\\
37.01	0\\
38.01	0\\
39.01	0\\
40.01	0\\
41.01	0\\
42.01	0\\
43.01	0\\
44.01	0\\
45.01	0\\
46.01	0\\
47.01	0\\
48.01	0\\
49.01	0\\
50.01	0\\
51.01	0\\
52.01	0\\
53.01	0\\
54.01	0\\
55.01	0\\
56.01	0\\
57.01	0\\
58.01	0\\
59.01	0\\
60.01	0\\
61.01	0\\
62.01	0\\
63.01	0\\
64.01	0\\
65.01	0\\
66.01	0\\
67.01	0\\
68.01	0\\
69.01	0\\
70.01	0\\
71.01	0\\
72.01	0\\
73.01	0\\
74.01	0\\
75.01	0\\
76.01	0\\
77.01	0\\
78.01	0\\
79.01	0\\
80.01	0\\
81.01	0\\
82.01	0\\
83.01	0\\
84.01	0\\
85.01	0\\
86.01	0\\
87.01	0\\
88.01	0\\
89.01	0\\
90.01	0\\
91.01	0\\
92.01	0\\
93.01	0\\
94.01	0\\
95.01	0\\
96.01	0\\
97.01	0\\
98.01	0\\
99.01	0\\
100.01	0\\
101.01	0\\
102.01	0\\
103.01	0\\
104.01	0\\
105.01	0\\
106.01	0\\
107.01	0\\
108.01	0\\
109.01	0\\
110.01	0\\
111.01	0\\
112.01	0\\
113.01	0\\
114.01	0\\
115.01	0\\
116.01	0\\
117.01	0\\
118.01	0\\
119.01	0\\
120.01	0\\
121.01	0\\
122.01	0\\
123.01	0\\
124.01	0\\
125.01	0\\
126.01	0\\
127.01	0\\
128.01	0\\
129.01	0\\
130.01	0\\
131.01	0\\
132.01	0\\
133.01	0\\
134.01	0\\
135.01	0\\
136.01	0\\
137.01	0\\
138.01	0\\
139.01	0\\
140.01	0\\
141.01	0\\
142.01	0\\
143.01	0\\
144.01	0\\
145.01	0\\
146.01	0\\
147.01	0\\
148.01	0\\
149.01	0\\
150.01	0\\
151.01	0\\
152.01	0\\
153.01	0\\
154.01	0\\
155.01	0\\
156.01	0\\
157.01	0\\
158.01	0\\
159.01	0\\
160.01	0\\
161.01	0\\
162.01	0\\
163.01	0\\
164.01	0\\
165.01	0\\
166.01	0\\
167.01	0\\
168.01	0\\
169.01	0\\
170.01	0\\
171.01	0\\
172.01	0\\
173.01	0\\
174.01	0\\
175.01	0\\
176.01	0\\
177.01	0\\
178.01	0\\
179.01	0\\
180.01	0\\
181.01	0\\
182.01	0\\
183.01	0\\
184.01	0\\
185.01	0\\
186.01	0\\
187.01	0\\
188.01	0\\
189.01	0\\
190.01	0\\
191.01	0\\
192.01	0\\
193.01	0\\
194.01	0\\
195.01	0\\
196.01	0\\
197.01	0\\
198.01	0\\
199.01	0\\
200.01	0\\
201.01	0\\
202.01	0\\
203.01	0\\
204.01	0\\
205.01	0\\
206.01	0\\
207.01	0\\
208.01	0\\
209.01	0\\
210.01	0\\
211.01	0\\
212.01	0\\
213.01	0\\
214.01	0\\
215.01	0\\
216.01	0\\
217.01	0\\
218.01	0\\
219.01	0\\
220.01	0\\
221.01	0\\
222.01	0\\
223.01	0\\
224.01	0\\
225.01	0\\
226.01	0\\
227.01	0\\
228.01	0\\
229.01	0\\
230.01	0\\
231.01	0\\
232.01	0\\
233.01	0\\
234.01	0\\
235.01	0\\
236.01	0\\
237.01	0\\
238.01	0\\
239.01	0\\
240.01	0\\
241.01	0\\
242.01	0\\
243.01	0\\
244.01	0\\
245.01	0\\
246.01	0\\
247.01	0\\
248.01	0\\
249.01	0\\
250.01	0\\
251.01	0\\
252.01	0\\
253.01	0\\
254.01	0\\
255.01	0\\
256.01	0\\
257.01	0\\
258.01	0\\
259.01	0\\
260.01	0\\
261.01	0\\
262.01	0\\
263.01	0\\
264.01	0\\
265.01	0\\
266.01	0\\
267.01	0\\
268.01	0\\
269.01	0\\
270.01	0\\
271.01	0\\
272.01	0\\
273.01	0\\
274.01	0\\
275.01	0\\
276.01	0\\
277.01	0\\
278.01	0\\
279.01	0\\
280.01	0\\
281.01	0\\
282.01	0\\
283.01	0\\
284.01	0\\
285.01	0\\
286.01	0\\
287.01	0\\
288.01	0\\
289.01	0\\
290.01	0\\
291.01	0\\
292.01	0\\
293.01	0\\
294.01	0\\
295.01	0\\
296.01	0\\
297.01	0\\
298.01	0\\
299.01	0\\
300.01	0\\
301.01	0\\
302.01	0\\
303.01	0\\
304.01	0\\
305.01	0\\
306.01	0\\
307.01	0\\
308.01	0\\
309.01	0\\
310.01	0\\
311.01	0\\
312.01	0\\
313.01	0\\
314.01	0\\
315.01	0\\
316.01	0\\
317.01	0\\
318.01	0\\
319.01	0\\
320.01	0\\
321.01	0\\
322.01	0\\
323.01	0\\
324.01	0\\
325.01	0\\
326.01	0\\
327.01	0\\
328.01	0\\
329.01	0\\
330.01	0\\
331.01	0\\
332.01	0\\
333.01	0\\
334.01	0\\
335.01	0\\
336.01	0\\
337.01	0\\
338.01	0\\
339.01	0\\
340.01	0\\
341.01	0\\
342.01	0\\
343.01	0\\
344.01	0\\
345.01	0\\
346.01	0\\
347.01	0\\
348.01	0\\
349.01	0\\
350.01	0\\
351.01	0\\
352.01	0\\
353.01	0\\
354.01	0\\
355.01	0\\
356.01	0\\
357.01	0\\
358.01	0\\
359.01	0\\
360.01	0\\
361.01	0\\
362.01	0\\
363.01	0\\
364.01	0\\
365.01	0\\
366.01	0\\
367.01	0\\
368.01	0\\
369.01	0\\
370.01	0\\
371.01	0\\
372.01	0\\
373.01	0\\
374.01	0\\
375.01	0\\
376.01	0\\
377.01	0\\
378.01	0\\
379.01	0\\
380.01	0\\
381.01	0\\
382.01	0\\
383.01	0\\
384.01	0\\
385.01	0\\
386.01	0\\
387.01	0\\
388.01	0\\
389.01	0\\
390.01	0\\
391.01	0\\
392.01	0\\
393.01	0\\
394.01	0\\
395.01	0\\
396.01	0\\
397.01	0\\
398.01	0\\
399.01	0\\
400.01	0\\
401.01	0\\
402.01	0\\
403.01	0\\
404.01	0\\
405.01	0\\
406.01	0\\
407.01	0\\
408.01	0\\
409.01	0\\
410.01	0\\
411.01	0\\
412.01	0\\
413.01	0\\
414.01	0\\
415.01	0\\
416.01	0\\
417.01	0\\
418.01	0\\
419.01	0\\
420.01	0\\
421.01	0\\
422.01	0\\
423.01	0\\
424.01	0\\
425.01	0\\
426.01	0\\
427.01	0\\
428.01	0\\
429.01	0\\
430.01	0\\
431.01	0\\
432.01	0\\
433.01	0\\
434.01	0\\
435.01	0\\
436.01	0\\
437.01	0\\
438.01	0\\
439.01	0\\
440.01	0\\
441.01	0\\
442.01	0\\
443.01	0\\
444.01	0\\
445.01	0\\
446.01	0\\
447.01	0\\
448.01	0\\
449.01	0\\
450.01	0\\
451.01	0\\
452.01	0\\
453.01	0\\
454.01	0\\
455.01	0\\
456.01	0\\
457.01	0\\
458.01	0\\
459.01	0\\
460.01	0\\
461.01	0\\
462.01	0\\
463.01	0\\
464.01	0\\
465.01	0\\
466.01	0\\
467.01	0\\
468.01	0\\
469.01	0\\
470.01	0\\
471.01	0\\
472.01	0\\
473.01	0\\
474.01	0\\
475.01	0\\
476.01	0\\
477.01	0\\
478.01	0\\
479.01	0\\
480.01	0\\
481.01	0\\
482.01	0\\
483.01	0\\
484.01	0\\
485.01	0\\
486.01	0\\
487.01	0\\
488.01	0\\
489.01	0\\
490.01	0\\
491.01	0\\
492.01	0\\
493.01	0\\
494.01	0\\
495.01	0\\
496.01	0\\
497.01	0\\
498.01	0\\
499.01	0\\
500.01	0\\
501.01	0\\
502.01	0\\
503.01	0\\
504.01	0\\
505.01	0\\
506.01	0\\
507.01	0\\
508.01	0\\
509.01	0\\
510.01	0\\
511.01	0\\
512.01	0\\
513.01	0\\
514.01	0\\
515.01	0\\
516.01	0\\
517.01	0\\
518.01	0\\
519.01	0\\
520.01	0\\
521.01	0\\
522.01	0\\
523.01	0\\
524.01	0\\
525.01	0\\
526.01	0\\
527.01	0\\
528.01	0\\
529.01	0\\
530.01	0\\
531.01	0\\
532.01	0\\
533.01	0\\
534.01	0\\
535.01	0\\
536.01	0\\
537.01	0\\
538.01	0\\
539.01	0\\
540.01	0\\
541.01	0\\
542.01	0\\
543.01	0\\
544.01	0\\
545.01	0\\
546.01	0\\
547.01	0\\
548.01	0\\
549.01	0\\
550.01	0\\
551.01	0\\
552.01	0\\
553.01	0\\
554.01	0\\
555.01	0\\
556.01	0\\
557.01	0\\
558.01	0\\
559.01	0\\
560.01	0\\
561.01	0\\
562.01	0\\
563.01	0\\
564.01	0\\
565.01	0\\
566.01	0\\
567.01	0\\
568.01	0\\
569.01	0\\
570.01	0\\
571.01	0\\
572.01	0\\
573.01	0\\
574.01	0\\
575.01	0\\
576.01	0\\
577.01	0\\
578.01	0\\
579.01	0\\
580.01	0\\
581.01	0\\
582.01	0\\
583.01	0\\
584.01	0\\
585.01	0\\
586.01	0\\
587.01	0\\
588.01	0\\
589.01	0\\
590.01	0\\
591.01	0\\
592.01	0\\
593.01	0\\
594.01	0\\
595.01	0\\
596.01	0\\
597.01	0.000480804875715751\\
598.01	0.00143832068185185\\
599.01	0.00385661257211618\\
599.02	0.00389414818835225\\
599.03	0.0039320414600908\\
599.04	0.00397029584140053\\
599.05	0.00400891481968839\\
599.06	0.00404790191602102\\
599.07	0.00408726068544939\\
599.08	0.00412699471733653\\
599.09	0.00416710763568844\\
599.1	0.00420760309948812\\
599.11	0.00424848480303291\\
599.12	0.00428975647627511\\
599.13	0.00433142188516562\\
599.14	0.00437348483200131\\
599.15	0.00441594915577533\\
599.16	0.00445881873253107\\
599.17	0.00450209747571942\\
599.18	0.00454578933655946\\
599.19	0.00458989830440269\\
599.2	0.00463442840710071\\
599.21	0.00467938371137647\\
599.22	0.00472476832319907\\
599.23	0.0047705863881622\\
599.24	0.00481684209186628\\
599.25	0.00486353966030409\\
599.26	0.00491068334955363\\
599.27	0.00495827744958433\\
599.28	0.00500632629158766\\
599.29	0.00505483424837287\\
599.3	0.0051038057347664\\
599.31	0.00515324520801534\\
599.32	0.00520315716819453\\
599.33	0.00525354615861772\\
599.34	0.00530441676625257\\
599.35	0.0053557736221397\\
599.36	0.00540762140181576\\
599.37	0.0054599648257405\\
599.38	0.00551280865972795\\
599.39	0.0055661577153818\\
599.4	0.00562001685053483\\
599.41	0.00567439096969272\\
599.42	0.00572928502448197\\
599.43	0.00578470401410213\\
599.44	0.0058406529857825\\
599.45	0.00589713703524302\\
599.46	0.00595416130715973\\
599.47	0.00601173099563459\\
599.48	0.00606985134466988\\
599.49	0.00612852764864712\\
599.5	0.00618776525281054\\
599.51	0.00624756955375525\\
599.52	0.00630794599992006\\
599.53	0.006368900092085\\
599.54	0.0064304373838737\\
599.55	0.00649256348226045\\
599.56	0.00655528404808222\\
599.57	0.00661860479655559\\
599.58	0.00668253149779856\\
599.59	0.00674706997735745\\
599.6	0.00681222611673878\\
599.61	0.0068780058539463\\
599.62	0.00694441518402313\\
599.63	0.00701146015959912\\
599.64	0.00707914689144345\\
599.65	0.00714748154902253\\
599.66	0.00721647036106324\\
599.67	0.00728611961612158\\
599.68	0.00735643566315675\\
599.69	0.00742742491211078\\
599.7	0.00749909383449363\\
599.71	0.00757144896397399\\
599.72	0.00764449689697571\\
599.73	0.00771824429327988\\
599.74	0.00779269787663284\\
599.75	0.00786786443535982\\
599.76	0.0079437508229846\\
599.77	0.00802036395885509\\
599.78	0.00809771082877485\\
599.79	0.00817579848564071\\
599.8	0.00825463405008648\\
599.81	0.00833422471113284\\
599.82	0.00841457772684349\\
599.83	0.00849570042498746\\
599.84	0.00857760020370797\\
599.85	0.00866028453219752\\
599.86	0.00874376095137954\\
599.87	0.00882803707459654\\
599.88	0.0089131205883049\\
599.89	0.00899901925277625\\
599.9	0.00908574090280562\\
599.91	0.00917329344842635\\
599.92	0.0092616848756319\\
599.93	0.00935092324710449\\
599.94	0.00944101670295078\\
599.95	0.00953197346144465\\
599.96	0.00962380181977693\\
599.97	0.00971651015481255\\
599.98	0.0098101069238547\\
599.99	0.00990460066541651\\
600	0.01\\
};
\addplot [color=red,solid,forget plot]
  table[row sep=crcr]{%
0.01	0\\
1.01	0\\
2.01	0\\
3.01	0\\
4.01	0\\
5.01	0\\
6.01	0\\
7.01	0\\
8.01	0\\
9.01	0\\
10.01	0\\
11.01	0\\
12.01	0\\
13.01	0\\
14.01	0\\
15.01	0\\
16.01	0\\
17.01	0\\
18.01	0\\
19.01	0\\
20.01	0\\
21.01	0\\
22.01	0\\
23.01	0\\
24.01	0\\
25.01	0\\
26.01	0\\
27.01	0\\
28.01	0\\
29.01	0\\
30.01	0\\
31.01	0\\
32.01	0\\
33.01	0\\
34.01	0\\
35.01	0\\
36.01	0\\
37.01	0\\
38.01	0\\
39.01	0\\
40.01	0\\
41.01	0\\
42.01	0\\
43.01	0\\
44.01	0\\
45.01	0\\
46.01	0\\
47.01	0\\
48.01	0\\
49.01	0\\
50.01	0\\
51.01	0\\
52.01	0\\
53.01	0\\
54.01	0\\
55.01	0\\
56.01	0\\
57.01	0\\
58.01	0\\
59.01	0\\
60.01	0\\
61.01	0\\
62.01	0\\
63.01	0\\
64.01	0\\
65.01	0\\
66.01	0\\
67.01	0\\
68.01	0\\
69.01	0\\
70.01	0\\
71.01	0\\
72.01	0\\
73.01	0\\
74.01	0\\
75.01	0\\
76.01	0\\
77.01	0\\
78.01	0\\
79.01	0\\
80.01	0\\
81.01	0\\
82.01	0\\
83.01	0\\
84.01	0\\
85.01	0\\
86.01	0\\
87.01	0\\
88.01	0\\
89.01	0\\
90.01	0\\
91.01	0\\
92.01	0\\
93.01	0\\
94.01	0\\
95.01	0\\
96.01	0\\
97.01	0\\
98.01	0\\
99.01	0\\
100.01	0\\
101.01	0\\
102.01	0\\
103.01	0\\
104.01	0\\
105.01	0\\
106.01	0\\
107.01	0\\
108.01	0\\
109.01	0\\
110.01	0\\
111.01	0\\
112.01	0\\
113.01	0\\
114.01	0\\
115.01	0\\
116.01	0\\
117.01	0\\
118.01	0\\
119.01	0\\
120.01	0\\
121.01	0\\
122.01	0\\
123.01	0\\
124.01	0\\
125.01	0\\
126.01	0\\
127.01	0\\
128.01	0\\
129.01	0\\
130.01	0\\
131.01	0\\
132.01	0\\
133.01	0\\
134.01	0\\
135.01	0\\
136.01	0\\
137.01	0\\
138.01	0\\
139.01	0\\
140.01	0\\
141.01	0\\
142.01	0\\
143.01	0\\
144.01	0\\
145.01	0\\
146.01	0\\
147.01	0\\
148.01	0\\
149.01	0\\
150.01	0\\
151.01	0\\
152.01	0\\
153.01	0\\
154.01	0\\
155.01	0\\
156.01	0\\
157.01	0\\
158.01	0\\
159.01	0\\
160.01	0\\
161.01	0\\
162.01	0\\
163.01	0\\
164.01	0\\
165.01	0\\
166.01	0\\
167.01	0\\
168.01	0\\
169.01	0\\
170.01	0\\
171.01	0\\
172.01	0\\
173.01	0\\
174.01	0\\
175.01	0\\
176.01	0\\
177.01	0\\
178.01	0\\
179.01	0\\
180.01	0\\
181.01	0\\
182.01	0\\
183.01	0\\
184.01	0\\
185.01	0\\
186.01	0\\
187.01	0\\
188.01	0\\
189.01	0\\
190.01	0\\
191.01	0\\
192.01	0\\
193.01	0\\
194.01	0\\
195.01	0\\
196.01	0\\
197.01	0\\
198.01	0\\
199.01	0\\
200.01	0\\
201.01	0\\
202.01	0\\
203.01	0\\
204.01	0\\
205.01	0\\
206.01	0\\
207.01	0\\
208.01	0\\
209.01	0\\
210.01	0\\
211.01	0\\
212.01	0\\
213.01	0\\
214.01	0\\
215.01	0\\
216.01	0\\
217.01	0\\
218.01	0\\
219.01	0\\
220.01	0\\
221.01	0\\
222.01	0\\
223.01	0\\
224.01	0\\
225.01	0\\
226.01	0\\
227.01	0\\
228.01	0\\
229.01	0\\
230.01	0\\
231.01	0\\
232.01	0\\
233.01	0\\
234.01	0\\
235.01	0\\
236.01	0\\
237.01	0\\
238.01	0\\
239.01	0\\
240.01	0\\
241.01	0\\
242.01	0\\
243.01	0\\
244.01	0\\
245.01	0\\
246.01	0\\
247.01	0\\
248.01	0\\
249.01	0\\
250.01	0\\
251.01	0\\
252.01	0\\
253.01	0\\
254.01	0\\
255.01	0\\
256.01	0\\
257.01	0\\
258.01	0\\
259.01	0\\
260.01	0\\
261.01	0\\
262.01	0\\
263.01	0\\
264.01	0\\
265.01	0\\
266.01	0\\
267.01	0\\
268.01	0\\
269.01	0\\
270.01	0\\
271.01	0\\
272.01	0\\
273.01	0\\
274.01	0\\
275.01	0\\
276.01	0\\
277.01	0\\
278.01	0\\
279.01	0\\
280.01	0\\
281.01	0\\
282.01	0\\
283.01	0\\
284.01	0\\
285.01	0\\
286.01	0\\
287.01	0\\
288.01	0\\
289.01	0\\
290.01	0\\
291.01	0\\
292.01	0\\
293.01	0\\
294.01	0\\
295.01	0\\
296.01	0\\
297.01	0\\
298.01	0\\
299.01	0\\
300.01	0\\
301.01	0\\
302.01	0\\
303.01	0\\
304.01	0\\
305.01	0\\
306.01	0\\
307.01	0\\
308.01	0\\
309.01	0\\
310.01	0\\
311.01	0\\
312.01	0\\
313.01	0\\
314.01	0\\
315.01	0\\
316.01	0\\
317.01	0\\
318.01	0\\
319.01	0\\
320.01	0\\
321.01	0\\
322.01	0\\
323.01	0\\
324.01	0\\
325.01	0\\
326.01	0\\
327.01	0\\
328.01	0\\
329.01	0\\
330.01	0\\
331.01	0\\
332.01	0\\
333.01	0\\
334.01	0\\
335.01	0\\
336.01	0\\
337.01	0\\
338.01	0\\
339.01	0\\
340.01	0\\
341.01	0\\
342.01	0\\
343.01	0\\
344.01	0\\
345.01	0\\
346.01	0\\
347.01	0\\
348.01	0\\
349.01	0\\
350.01	0\\
351.01	0\\
352.01	0\\
353.01	0\\
354.01	0\\
355.01	0\\
356.01	0\\
357.01	0\\
358.01	0\\
359.01	0\\
360.01	0\\
361.01	0\\
362.01	0\\
363.01	0\\
364.01	0\\
365.01	0\\
366.01	0\\
367.01	0\\
368.01	0\\
369.01	0\\
370.01	0\\
371.01	0\\
372.01	0\\
373.01	0\\
374.01	0\\
375.01	0\\
376.01	0\\
377.01	0\\
378.01	0\\
379.01	0\\
380.01	0\\
381.01	0\\
382.01	0\\
383.01	0\\
384.01	0\\
385.01	0\\
386.01	0\\
387.01	0\\
388.01	0\\
389.01	0\\
390.01	0\\
391.01	0\\
392.01	0\\
393.01	0\\
394.01	0\\
395.01	0\\
396.01	0\\
397.01	0\\
398.01	0\\
399.01	0\\
400.01	0\\
401.01	0\\
402.01	0\\
403.01	0\\
404.01	0\\
405.01	0\\
406.01	0\\
407.01	0\\
408.01	0\\
409.01	0\\
410.01	0\\
411.01	0\\
412.01	0\\
413.01	0\\
414.01	0\\
415.01	0\\
416.01	0\\
417.01	0\\
418.01	0\\
419.01	0\\
420.01	0\\
421.01	0\\
422.01	0\\
423.01	0\\
424.01	0\\
425.01	0\\
426.01	0\\
427.01	0\\
428.01	0\\
429.01	0\\
430.01	0\\
431.01	0\\
432.01	0\\
433.01	0\\
434.01	0\\
435.01	0\\
436.01	0\\
437.01	0\\
438.01	0\\
439.01	0\\
440.01	0\\
441.01	0\\
442.01	0\\
443.01	0\\
444.01	0\\
445.01	0\\
446.01	0\\
447.01	0\\
448.01	0\\
449.01	0\\
450.01	0\\
451.01	0\\
452.01	0\\
453.01	0\\
454.01	0\\
455.01	0\\
456.01	0\\
457.01	0\\
458.01	0\\
459.01	0\\
460.01	0\\
461.01	0\\
462.01	0\\
463.01	0\\
464.01	0\\
465.01	0\\
466.01	0\\
467.01	0\\
468.01	0\\
469.01	0\\
470.01	0\\
471.01	0\\
472.01	0\\
473.01	0\\
474.01	0\\
475.01	0\\
476.01	0\\
477.01	0\\
478.01	0\\
479.01	0\\
480.01	0\\
481.01	0\\
482.01	0\\
483.01	0\\
484.01	0\\
485.01	0\\
486.01	0\\
487.01	0\\
488.01	0\\
489.01	0\\
490.01	0\\
491.01	0\\
492.01	0\\
493.01	0\\
494.01	0\\
495.01	0\\
496.01	0\\
497.01	0\\
498.01	0\\
499.01	0\\
500.01	0\\
501.01	0\\
502.01	0\\
503.01	0\\
504.01	0\\
505.01	0\\
506.01	0\\
507.01	0\\
508.01	0\\
509.01	0\\
510.01	0\\
511.01	0\\
512.01	0\\
513.01	0\\
514.01	0\\
515.01	0\\
516.01	0\\
517.01	0\\
518.01	0\\
519.01	0\\
520.01	0\\
521.01	0\\
522.01	0\\
523.01	0\\
524.01	0\\
525.01	0\\
526.01	0\\
527.01	0\\
528.01	0\\
529.01	0\\
530.01	0\\
531.01	0\\
532.01	0\\
533.01	0\\
534.01	0\\
535.01	0\\
536.01	0\\
537.01	0\\
538.01	0\\
539.01	0\\
540.01	0\\
541.01	0\\
542.01	0\\
543.01	0\\
544.01	0\\
545.01	0\\
546.01	0\\
547.01	0\\
548.01	0\\
549.01	0\\
550.01	0\\
551.01	0\\
552.01	0\\
553.01	0\\
554.01	0\\
555.01	0\\
556.01	0\\
557.01	0\\
558.01	0\\
559.01	0\\
560.01	0\\
561.01	0\\
562.01	0\\
563.01	0\\
564.01	0\\
565.01	0\\
566.01	0\\
567.01	0\\
568.01	0\\
569.01	0\\
570.01	0\\
571.01	0\\
572.01	0\\
573.01	0\\
574.01	0\\
575.01	0\\
576.01	0\\
577.01	0\\
578.01	0\\
579.01	0\\
580.01	0\\
581.01	0\\
582.01	0\\
583.01	0\\
584.01	0\\
585.01	0\\
586.01	0\\
587.01	0\\
588.01	0\\
589.01	0\\
590.01	0\\
591.01	0\\
592.01	0\\
593.01	0\\
594.01	0\\
595.01	0\\
596.01	0\\
597.01	0.000480974774395043\\
598.01	0.00143832068185183\\
599.01	0.00385661257211624\\
599.02	0.00389414818835231\\
599.03	0.00393204146009085\\
599.04	0.0039702958414006\\
599.05	0.00400891481968847\\
599.06	0.00404790191602111\\
599.07	0.00408726068544948\\
599.08	0.00412699471733663\\
599.09	0.00416710763568853\\
599.1	0.0042076030994882\\
599.11	0.00424848480303301\\
599.12	0.00428975647627519\\
599.13	0.00433142188516572\\
599.14	0.0043734848320014\\
599.15	0.00441594915577541\\
599.16	0.00445881873253114\\
599.17	0.00450209747571947\\
599.18	0.00454578933655952\\
599.19	0.00458989830440275\\
599.2	0.00463442840710077\\
599.21	0.00467938371137651\\
599.22	0.0047247683231991\\
599.23	0.00477058638816223\\
599.24	0.00481684209186631\\
599.25	0.00486353966030413\\
599.26	0.00491068334955368\\
599.27	0.00495827744958438\\
599.28	0.0050063262915877\\
599.29	0.0050548342483729\\
599.3	0.00510380573476643\\
599.31	0.00515324520801537\\
599.32	0.00520315716819457\\
599.33	0.00525354615861776\\
599.34	0.00530441676625259\\
599.35	0.00535577362213973\\
599.36	0.00540762140181579\\
599.37	0.00545996482574052\\
599.38	0.00551280865972798\\
599.39	0.00556615771538182\\
599.4	0.00562001685053486\\
599.41	0.00567439096969275\\
599.42	0.00572928502448199\\
599.43	0.00578470401410216\\
599.44	0.00584065298578253\\
599.45	0.00589713703524305\\
599.46	0.00595416130715976\\
599.47	0.00601173099563462\\
599.48	0.00606985134466992\\
599.49	0.00612852764864716\\
599.5	0.00618776525281058\\
599.51	0.00624756955375529\\
599.52	0.00630794599992009\\
599.53	0.00636890009208504\\
599.54	0.00643043738387374\\
599.55	0.00649256348226048\\
599.56	0.00655528404808225\\
599.57	0.00661860479655562\\
599.58	0.00668253149779859\\
599.59	0.00674706997735749\\
599.6	0.00681222611673882\\
599.61	0.00687800585394634\\
599.62	0.00694441518402316\\
599.63	0.00701146015959914\\
599.64	0.00707914689144347\\
599.65	0.00714748154902254\\
599.66	0.00721647036106325\\
599.67	0.00728611961612159\\
599.68	0.00735643566315677\\
599.69	0.0074274249121108\\
599.7	0.00749909383449365\\
599.71	0.00757144896397401\\
599.72	0.00764449689697572\\
599.73	0.0077182442932799\\
599.74	0.00779269787663286\\
599.75	0.00786786443535983\\
599.76	0.00794375082298461\\
599.77	0.0080203639588551\\
599.78	0.00809771082877486\\
599.79	0.00817579848564072\\
599.8	0.00825463405008648\\
599.81	0.00833422471113285\\
599.82	0.0084145777268435\\
599.83	0.00849570042498747\\
599.84	0.00857760020370798\\
599.85	0.00866028453219753\\
599.86	0.00874376095137954\\
599.87	0.00882803707459654\\
599.88	0.0089131205883049\\
599.89	0.00899901925277625\\
599.9	0.00908574090280562\\
599.91	0.00917329344842636\\
599.92	0.00926168487563191\\
599.93	0.00935092324710449\\
599.94	0.00944101670295078\\
599.95	0.00953197346144465\\
599.96	0.00962380181977693\\
599.97	0.00971651015481255\\
599.98	0.0098101069238547\\
599.99	0.00990460066541651\\
600	0.01\\
};
\addplot [color=mycolor20,solid,forget plot]
  table[row sep=crcr]{%
0.01	0\\
1.01	0\\
2.01	0\\
3.01	0\\
4.01	0\\
5.01	0\\
6.01	0\\
7.01	0\\
8.01	0\\
9.01	0\\
10.01	0\\
11.01	0\\
12.01	0\\
13.01	0\\
14.01	0\\
15.01	0\\
16.01	0\\
17.01	0\\
18.01	0\\
19.01	0\\
20.01	0\\
21.01	0\\
22.01	0\\
23.01	0\\
24.01	0\\
25.01	0\\
26.01	0\\
27.01	0\\
28.01	0\\
29.01	0\\
30.01	0\\
31.01	0\\
32.01	0\\
33.01	0\\
34.01	0\\
35.01	0\\
36.01	0\\
37.01	0\\
38.01	0\\
39.01	0\\
40.01	0\\
41.01	0\\
42.01	0\\
43.01	0\\
44.01	0\\
45.01	0\\
46.01	0\\
47.01	0\\
48.01	0\\
49.01	0\\
50.01	0\\
51.01	0\\
52.01	0\\
53.01	0\\
54.01	0\\
55.01	0\\
56.01	0\\
57.01	0\\
58.01	0\\
59.01	0\\
60.01	0\\
61.01	0\\
62.01	0\\
63.01	0\\
64.01	0\\
65.01	0\\
66.01	0\\
67.01	0\\
68.01	0\\
69.01	0\\
70.01	0\\
71.01	0\\
72.01	0\\
73.01	0\\
74.01	0\\
75.01	0\\
76.01	0\\
77.01	0\\
78.01	0\\
79.01	0\\
80.01	0\\
81.01	0\\
82.01	0\\
83.01	0\\
84.01	0\\
85.01	0\\
86.01	0\\
87.01	0\\
88.01	0\\
89.01	0\\
90.01	0\\
91.01	0\\
92.01	0\\
93.01	0\\
94.01	0\\
95.01	0\\
96.01	0\\
97.01	0\\
98.01	0\\
99.01	0\\
100.01	0\\
101.01	0\\
102.01	0\\
103.01	0\\
104.01	0\\
105.01	0\\
106.01	0\\
107.01	0\\
108.01	0\\
109.01	0\\
110.01	0\\
111.01	0\\
112.01	0\\
113.01	0\\
114.01	0\\
115.01	0\\
116.01	0\\
117.01	0\\
118.01	0\\
119.01	0\\
120.01	0\\
121.01	0\\
122.01	0\\
123.01	0\\
124.01	0\\
125.01	0\\
126.01	0\\
127.01	0\\
128.01	0\\
129.01	0\\
130.01	0\\
131.01	0\\
132.01	0\\
133.01	0\\
134.01	0\\
135.01	0\\
136.01	0\\
137.01	0\\
138.01	0\\
139.01	0\\
140.01	0\\
141.01	0\\
142.01	0\\
143.01	0\\
144.01	0\\
145.01	0\\
146.01	0\\
147.01	0\\
148.01	0\\
149.01	0\\
150.01	0\\
151.01	0\\
152.01	0\\
153.01	0\\
154.01	0\\
155.01	0\\
156.01	0\\
157.01	0\\
158.01	0\\
159.01	0\\
160.01	0\\
161.01	0\\
162.01	0\\
163.01	0\\
164.01	0\\
165.01	0\\
166.01	0\\
167.01	0\\
168.01	0\\
169.01	0\\
170.01	0\\
171.01	0\\
172.01	0\\
173.01	0\\
174.01	0\\
175.01	0\\
176.01	0\\
177.01	0\\
178.01	0\\
179.01	0\\
180.01	0\\
181.01	0\\
182.01	0\\
183.01	0\\
184.01	0\\
185.01	0\\
186.01	0\\
187.01	0\\
188.01	0\\
189.01	0\\
190.01	0\\
191.01	0\\
192.01	0\\
193.01	0\\
194.01	0\\
195.01	0\\
196.01	0\\
197.01	0\\
198.01	0\\
199.01	0\\
200.01	0\\
201.01	0\\
202.01	0\\
203.01	0\\
204.01	0\\
205.01	0\\
206.01	0\\
207.01	0\\
208.01	0\\
209.01	0\\
210.01	0\\
211.01	0\\
212.01	0\\
213.01	0\\
214.01	0\\
215.01	0\\
216.01	0\\
217.01	0\\
218.01	0\\
219.01	0\\
220.01	0\\
221.01	0\\
222.01	0\\
223.01	0\\
224.01	0\\
225.01	0\\
226.01	0\\
227.01	0\\
228.01	0\\
229.01	0\\
230.01	0\\
231.01	0\\
232.01	0\\
233.01	0\\
234.01	0\\
235.01	0\\
236.01	0\\
237.01	0\\
238.01	0\\
239.01	0\\
240.01	0\\
241.01	0\\
242.01	0\\
243.01	0\\
244.01	0\\
245.01	0\\
246.01	0\\
247.01	0\\
248.01	0\\
249.01	0\\
250.01	0\\
251.01	0\\
252.01	0\\
253.01	0\\
254.01	0\\
255.01	0\\
256.01	0\\
257.01	0\\
258.01	0\\
259.01	0\\
260.01	0\\
261.01	0\\
262.01	0\\
263.01	0\\
264.01	0\\
265.01	0\\
266.01	0\\
267.01	0\\
268.01	0\\
269.01	0\\
270.01	0\\
271.01	0\\
272.01	0\\
273.01	0\\
274.01	0\\
275.01	0\\
276.01	0\\
277.01	0\\
278.01	0\\
279.01	0\\
280.01	0\\
281.01	0\\
282.01	0\\
283.01	0\\
284.01	0\\
285.01	0\\
286.01	0\\
287.01	0\\
288.01	0\\
289.01	0\\
290.01	0\\
291.01	0\\
292.01	0\\
293.01	0\\
294.01	0\\
295.01	0\\
296.01	0\\
297.01	0\\
298.01	0\\
299.01	0\\
300.01	0\\
301.01	0\\
302.01	0\\
303.01	0\\
304.01	0\\
305.01	0\\
306.01	0\\
307.01	0\\
308.01	0\\
309.01	0\\
310.01	0\\
311.01	0\\
312.01	0\\
313.01	0\\
314.01	0\\
315.01	0\\
316.01	0\\
317.01	0\\
318.01	0\\
319.01	0\\
320.01	0\\
321.01	0\\
322.01	0\\
323.01	0\\
324.01	0\\
325.01	0\\
326.01	0\\
327.01	0\\
328.01	0\\
329.01	0\\
330.01	0\\
331.01	0\\
332.01	0\\
333.01	0\\
334.01	0\\
335.01	0\\
336.01	0\\
337.01	0\\
338.01	0\\
339.01	0\\
340.01	0\\
341.01	0\\
342.01	0\\
343.01	0\\
344.01	0\\
345.01	0\\
346.01	0\\
347.01	0\\
348.01	0\\
349.01	0\\
350.01	0\\
351.01	0\\
352.01	0\\
353.01	0\\
354.01	0\\
355.01	0\\
356.01	0\\
357.01	0\\
358.01	0\\
359.01	0\\
360.01	0\\
361.01	0\\
362.01	0\\
363.01	0\\
364.01	0\\
365.01	0\\
366.01	0\\
367.01	0\\
368.01	0\\
369.01	0\\
370.01	0\\
371.01	0\\
372.01	0\\
373.01	0\\
374.01	0\\
375.01	0\\
376.01	0\\
377.01	0\\
378.01	0\\
379.01	0\\
380.01	0\\
381.01	0\\
382.01	0\\
383.01	0\\
384.01	0\\
385.01	0\\
386.01	0\\
387.01	0\\
388.01	0\\
389.01	0\\
390.01	0\\
391.01	0\\
392.01	0\\
393.01	0\\
394.01	0\\
395.01	0\\
396.01	0\\
397.01	0\\
398.01	0\\
399.01	0\\
400.01	0\\
401.01	0\\
402.01	0\\
403.01	0\\
404.01	0\\
405.01	0\\
406.01	0\\
407.01	0\\
408.01	0\\
409.01	0\\
410.01	0\\
411.01	0\\
412.01	0\\
413.01	0\\
414.01	0\\
415.01	0\\
416.01	0\\
417.01	0\\
418.01	0\\
419.01	0\\
420.01	0\\
421.01	0\\
422.01	0\\
423.01	0\\
424.01	0\\
425.01	0\\
426.01	0\\
427.01	0\\
428.01	0\\
429.01	0\\
430.01	0\\
431.01	0\\
432.01	0\\
433.01	0\\
434.01	0\\
435.01	0\\
436.01	0\\
437.01	0\\
438.01	0\\
439.01	0\\
440.01	0\\
441.01	0\\
442.01	0\\
443.01	0\\
444.01	0\\
445.01	0\\
446.01	0\\
447.01	0\\
448.01	0\\
449.01	0\\
450.01	0\\
451.01	0\\
452.01	0\\
453.01	0\\
454.01	0\\
455.01	0\\
456.01	0\\
457.01	0\\
458.01	0\\
459.01	0\\
460.01	0\\
461.01	0\\
462.01	0\\
463.01	0\\
464.01	0\\
465.01	0\\
466.01	0\\
467.01	0\\
468.01	0\\
469.01	0\\
470.01	0\\
471.01	0\\
472.01	0\\
473.01	0\\
474.01	0\\
475.01	0\\
476.01	0\\
477.01	0\\
478.01	0\\
479.01	0\\
480.01	0\\
481.01	0\\
482.01	0\\
483.01	0\\
484.01	0\\
485.01	0\\
486.01	0\\
487.01	0\\
488.01	0\\
489.01	0\\
490.01	0\\
491.01	0\\
492.01	0\\
493.01	0\\
494.01	0\\
495.01	0\\
496.01	0\\
497.01	0\\
498.01	0\\
499.01	0\\
500.01	0\\
501.01	0\\
502.01	0\\
503.01	0\\
504.01	0\\
505.01	0\\
506.01	0\\
507.01	0\\
508.01	0\\
509.01	0\\
510.01	0\\
511.01	0\\
512.01	0\\
513.01	0\\
514.01	0\\
515.01	0\\
516.01	0\\
517.01	0\\
518.01	0\\
519.01	0\\
520.01	0\\
521.01	0\\
522.01	0\\
523.01	0\\
524.01	0\\
525.01	0\\
526.01	0\\
527.01	0\\
528.01	0\\
529.01	0\\
530.01	0\\
531.01	0\\
532.01	0\\
533.01	0\\
534.01	0\\
535.01	0\\
536.01	0\\
537.01	0\\
538.01	0\\
539.01	0\\
540.01	0\\
541.01	0\\
542.01	0\\
543.01	0\\
544.01	0\\
545.01	0\\
546.01	0\\
547.01	0\\
548.01	0\\
549.01	0\\
550.01	0\\
551.01	0\\
552.01	0\\
553.01	0\\
554.01	0\\
555.01	0\\
556.01	0\\
557.01	0\\
558.01	0\\
559.01	0\\
560.01	0\\
561.01	0\\
562.01	0\\
563.01	0\\
564.01	0\\
565.01	0\\
566.01	0\\
567.01	0\\
568.01	0\\
569.01	0\\
570.01	0\\
571.01	0\\
572.01	0\\
573.01	0\\
574.01	0\\
575.01	0\\
576.01	0\\
577.01	0\\
578.01	0\\
579.01	0\\
580.01	0\\
581.01	0\\
582.01	0\\
583.01	0\\
584.01	0\\
585.01	0\\
586.01	0\\
587.01	0\\
588.01	0\\
589.01	0\\
590.01	0\\
591.01	0\\
592.01	0\\
593.01	0\\
594.01	0\\
595.01	0\\
596.01	0\\
597.01	0.000481110898971432\\
598.01	0.00143832068185183\\
599.01	0.00385661257211628\\
599.02	0.00389414818835235\\
599.03	0.00393204146009088\\
599.04	0.00397029584140061\\
599.05	0.00400891481968847\\
599.06	0.0040479019160211\\
599.07	0.00408726068544947\\
599.08	0.00412699471733662\\
599.09	0.00416710763568851\\
599.1	0.00420760309948819\\
599.11	0.004248484803033\\
599.12	0.00428975647627518\\
599.13	0.00433142188516569\\
599.14	0.00437348483200137\\
599.15	0.00441594915577538\\
599.16	0.00445881873253112\\
599.17	0.00450209747571947\\
599.18	0.00454578933655951\\
599.19	0.00458989830440275\\
599.2	0.00463442840710077\\
599.21	0.00467938371137652\\
599.22	0.00472476832319911\\
599.23	0.00477058638816225\\
599.24	0.00481684209186631\\
599.25	0.00486353966030412\\
599.26	0.00491068334955366\\
599.27	0.00495827744958435\\
599.28	0.00500632629158769\\
599.29	0.00505483424837289\\
599.3	0.00510380573476643\\
599.31	0.00515324520801537\\
599.32	0.00520315716819457\\
599.33	0.00525354615861776\\
599.34	0.00530441676625261\\
599.35	0.00535577362213975\\
599.36	0.0054076214018158\\
599.37	0.00545996482574054\\
599.38	0.00551280865972799\\
599.39	0.00556615771538183\\
599.4	0.00562001685053486\\
599.41	0.00567439096969276\\
599.42	0.00572928502448199\\
599.43	0.00578470401410217\\
599.44	0.00584065298578253\\
599.45	0.00589713703524306\\
599.46	0.00595416130715976\\
599.47	0.00601173099563461\\
599.48	0.00606985134466991\\
599.49	0.00612852764864715\\
599.5	0.00618776525281058\\
599.51	0.00624756955375528\\
599.52	0.00630794599992008\\
599.53	0.00636890009208502\\
599.54	0.00643043738387371\\
599.55	0.00649256348226045\\
599.56	0.00655528404808223\\
599.57	0.00661860479655559\\
599.58	0.00668253149779856\\
599.59	0.00674706997735745\\
599.6	0.00681222611673879\\
599.61	0.00687800585394631\\
599.62	0.00694441518402313\\
599.63	0.00701146015959912\\
599.64	0.00707914689144344\\
599.65	0.00714748154902252\\
599.66	0.00721647036106322\\
599.67	0.00728611961612156\\
599.68	0.00735643566315675\\
599.69	0.00742742491211077\\
599.7	0.00749909383449363\\
599.71	0.00757144896397399\\
599.72	0.00764449689697571\\
599.73	0.00771824429327989\\
599.74	0.00779269787663284\\
599.75	0.00786786443535982\\
599.76	0.0079437508229846\\
599.77	0.00802036395885509\\
599.78	0.00809771082877485\\
599.79	0.0081757984856407\\
599.8	0.00825463405008647\\
599.81	0.00833422471113284\\
599.82	0.00841457772684348\\
599.83	0.00849570042498746\\
599.84	0.00857760020370798\\
599.85	0.00866028453219752\\
599.86	0.00874376095137953\\
599.87	0.00882803707459654\\
599.88	0.00891312058830489\\
599.89	0.00899901925277625\\
599.9	0.00908574090280562\\
599.91	0.00917329344842636\\
599.92	0.00926168487563191\\
599.93	0.00935092324710449\\
599.94	0.00944101670295078\\
599.95	0.00953197346144465\\
599.96	0.00962380181977694\\
599.97	0.00971651015481255\\
599.98	0.0098101069238547\\
599.99	0.00990460066541651\\
600	0.01\\
};
\addplot [color=mycolor21,solid,forget plot]
  table[row sep=crcr]{%
0.01	0\\
1.01	0\\
2.01	0\\
3.01	0\\
4.01	0\\
5.01	0\\
6.01	0\\
7.01	0\\
8.01	0\\
9.01	0\\
10.01	0\\
11.01	0\\
12.01	0\\
13.01	0\\
14.01	0\\
15.01	0\\
16.01	0\\
17.01	0\\
18.01	0\\
19.01	0\\
20.01	0\\
21.01	0\\
22.01	0\\
23.01	0\\
24.01	0\\
25.01	0\\
26.01	0\\
27.01	0\\
28.01	0\\
29.01	0\\
30.01	0\\
31.01	0\\
32.01	0\\
33.01	0\\
34.01	0\\
35.01	0\\
36.01	0\\
37.01	0\\
38.01	0\\
39.01	0\\
40.01	0\\
41.01	0\\
42.01	0\\
43.01	0\\
44.01	0\\
45.01	0\\
46.01	0\\
47.01	0\\
48.01	0\\
49.01	0\\
50.01	0\\
51.01	0\\
52.01	0\\
53.01	0\\
54.01	0\\
55.01	0\\
56.01	0\\
57.01	0\\
58.01	0\\
59.01	0\\
60.01	0\\
61.01	0\\
62.01	0\\
63.01	0\\
64.01	0\\
65.01	0\\
66.01	0\\
67.01	0\\
68.01	0\\
69.01	0\\
70.01	0\\
71.01	0\\
72.01	0\\
73.01	0\\
74.01	0\\
75.01	0\\
76.01	0\\
77.01	0\\
78.01	0\\
79.01	0\\
80.01	0\\
81.01	0\\
82.01	0\\
83.01	0\\
84.01	0\\
85.01	0\\
86.01	0\\
87.01	0\\
88.01	0\\
89.01	0\\
90.01	0\\
91.01	0\\
92.01	0\\
93.01	0\\
94.01	0\\
95.01	0\\
96.01	0\\
97.01	0\\
98.01	0\\
99.01	0\\
100.01	0\\
101.01	0\\
102.01	0\\
103.01	0\\
104.01	0\\
105.01	0\\
106.01	0\\
107.01	0\\
108.01	0\\
109.01	0\\
110.01	0\\
111.01	0\\
112.01	0\\
113.01	0\\
114.01	0\\
115.01	0\\
116.01	0\\
117.01	0\\
118.01	0\\
119.01	0\\
120.01	0\\
121.01	0\\
122.01	0\\
123.01	0\\
124.01	0\\
125.01	0\\
126.01	0\\
127.01	0\\
128.01	0\\
129.01	0\\
130.01	0\\
131.01	0\\
132.01	0\\
133.01	0\\
134.01	0\\
135.01	0\\
136.01	0\\
137.01	0\\
138.01	0\\
139.01	0\\
140.01	0\\
141.01	0\\
142.01	0\\
143.01	0\\
144.01	0\\
145.01	0\\
146.01	0\\
147.01	0\\
148.01	0\\
149.01	0\\
150.01	0\\
151.01	0\\
152.01	0\\
153.01	0\\
154.01	0\\
155.01	0\\
156.01	0\\
157.01	0\\
158.01	0\\
159.01	0\\
160.01	0\\
161.01	0\\
162.01	0\\
163.01	0\\
164.01	0\\
165.01	0\\
166.01	0\\
167.01	0\\
168.01	0\\
169.01	0\\
170.01	0\\
171.01	0\\
172.01	0\\
173.01	0\\
174.01	0\\
175.01	0\\
176.01	0\\
177.01	0\\
178.01	0\\
179.01	0\\
180.01	0\\
181.01	0\\
182.01	0\\
183.01	0\\
184.01	0\\
185.01	0\\
186.01	0\\
187.01	0\\
188.01	0\\
189.01	0\\
190.01	0\\
191.01	0\\
192.01	0\\
193.01	0\\
194.01	0\\
195.01	0\\
196.01	0\\
197.01	0\\
198.01	0\\
199.01	0\\
200.01	0\\
201.01	0\\
202.01	0\\
203.01	0\\
204.01	0\\
205.01	0\\
206.01	0\\
207.01	0\\
208.01	0\\
209.01	0\\
210.01	0\\
211.01	0\\
212.01	0\\
213.01	0\\
214.01	0\\
215.01	0\\
216.01	0\\
217.01	0\\
218.01	0\\
219.01	0\\
220.01	0\\
221.01	0\\
222.01	0\\
223.01	0\\
224.01	0\\
225.01	0\\
226.01	0\\
227.01	0\\
228.01	0\\
229.01	0\\
230.01	0\\
231.01	0\\
232.01	0\\
233.01	0\\
234.01	0\\
235.01	0\\
236.01	0\\
237.01	0\\
238.01	0\\
239.01	0\\
240.01	0\\
241.01	0\\
242.01	0\\
243.01	0\\
244.01	0\\
245.01	0\\
246.01	0\\
247.01	0\\
248.01	0\\
249.01	0\\
250.01	0\\
251.01	0\\
252.01	0\\
253.01	0\\
254.01	0\\
255.01	0\\
256.01	0\\
257.01	0\\
258.01	0\\
259.01	0\\
260.01	0\\
261.01	0\\
262.01	0\\
263.01	0\\
264.01	0\\
265.01	0\\
266.01	0\\
267.01	0\\
268.01	0\\
269.01	0\\
270.01	0\\
271.01	0\\
272.01	0\\
273.01	0\\
274.01	0\\
275.01	0\\
276.01	0\\
277.01	0\\
278.01	0\\
279.01	0\\
280.01	0\\
281.01	0\\
282.01	0\\
283.01	0\\
284.01	0\\
285.01	0\\
286.01	0\\
287.01	0\\
288.01	0\\
289.01	0\\
290.01	0\\
291.01	0\\
292.01	0\\
293.01	0\\
294.01	0\\
295.01	0\\
296.01	0\\
297.01	0\\
298.01	0\\
299.01	0\\
300.01	0\\
301.01	0\\
302.01	0\\
303.01	0\\
304.01	0\\
305.01	0\\
306.01	0\\
307.01	0\\
308.01	0\\
309.01	0\\
310.01	0\\
311.01	0\\
312.01	0\\
313.01	0\\
314.01	0\\
315.01	0\\
316.01	0\\
317.01	0\\
318.01	0\\
319.01	0\\
320.01	0\\
321.01	0\\
322.01	0\\
323.01	0\\
324.01	0\\
325.01	0\\
326.01	0\\
327.01	0\\
328.01	0\\
329.01	0\\
330.01	0\\
331.01	0\\
332.01	0\\
333.01	0\\
334.01	0\\
335.01	0\\
336.01	0\\
337.01	0\\
338.01	0\\
339.01	0\\
340.01	0\\
341.01	0\\
342.01	0\\
343.01	0\\
344.01	0\\
345.01	0\\
346.01	0\\
347.01	0\\
348.01	0\\
349.01	0\\
350.01	0\\
351.01	0\\
352.01	0\\
353.01	0\\
354.01	0\\
355.01	0\\
356.01	0\\
357.01	0\\
358.01	0\\
359.01	0\\
360.01	0\\
361.01	0\\
362.01	0\\
363.01	0\\
364.01	0\\
365.01	0\\
366.01	0\\
367.01	0\\
368.01	0\\
369.01	0\\
370.01	0\\
371.01	0\\
372.01	0\\
373.01	0\\
374.01	0\\
375.01	0\\
376.01	0\\
377.01	0\\
378.01	0\\
379.01	0\\
380.01	0\\
381.01	0\\
382.01	0\\
383.01	0\\
384.01	0\\
385.01	0\\
386.01	0\\
387.01	0\\
388.01	0\\
389.01	0\\
390.01	0\\
391.01	0\\
392.01	0\\
393.01	0\\
394.01	0\\
395.01	0\\
396.01	0\\
397.01	0\\
398.01	0\\
399.01	0\\
400.01	0\\
401.01	0\\
402.01	0\\
403.01	0\\
404.01	0\\
405.01	0\\
406.01	0\\
407.01	0\\
408.01	0\\
409.01	0\\
410.01	0\\
411.01	0\\
412.01	0\\
413.01	0\\
414.01	0\\
415.01	0\\
416.01	0\\
417.01	0\\
418.01	0\\
419.01	0\\
420.01	0\\
421.01	0\\
422.01	0\\
423.01	0\\
424.01	0\\
425.01	0\\
426.01	0\\
427.01	0\\
428.01	0\\
429.01	0\\
430.01	0\\
431.01	0\\
432.01	0\\
433.01	0\\
434.01	0\\
435.01	0\\
436.01	0\\
437.01	0\\
438.01	0\\
439.01	0\\
440.01	0\\
441.01	0\\
442.01	0\\
443.01	0\\
444.01	0\\
445.01	0\\
446.01	0\\
447.01	0\\
448.01	0\\
449.01	0\\
450.01	0\\
451.01	0\\
452.01	0\\
453.01	0\\
454.01	0\\
455.01	0\\
456.01	0\\
457.01	0\\
458.01	0\\
459.01	0\\
460.01	0\\
461.01	0\\
462.01	0\\
463.01	0\\
464.01	0\\
465.01	0\\
466.01	0\\
467.01	0\\
468.01	0\\
469.01	0\\
470.01	0\\
471.01	0\\
472.01	0\\
473.01	0\\
474.01	0\\
475.01	0\\
476.01	0\\
477.01	0\\
478.01	0\\
479.01	0\\
480.01	0\\
481.01	0\\
482.01	0\\
483.01	0\\
484.01	0\\
485.01	0\\
486.01	0\\
487.01	0\\
488.01	0\\
489.01	0\\
490.01	0\\
491.01	0\\
492.01	0\\
493.01	0\\
494.01	0\\
495.01	0\\
496.01	0\\
497.01	0\\
498.01	0\\
499.01	0\\
500.01	0\\
501.01	0\\
502.01	0\\
503.01	0\\
504.01	0\\
505.01	0\\
506.01	0\\
507.01	0\\
508.01	0\\
509.01	0\\
510.01	0\\
511.01	0\\
512.01	0\\
513.01	0\\
514.01	0\\
515.01	0\\
516.01	0\\
517.01	0\\
518.01	0\\
519.01	0\\
520.01	0\\
521.01	0\\
522.01	0\\
523.01	0\\
524.01	0\\
525.01	0\\
526.01	0\\
527.01	0\\
528.01	0\\
529.01	0\\
530.01	0\\
531.01	0\\
532.01	0\\
533.01	0\\
534.01	0\\
535.01	0\\
536.01	0\\
537.01	0\\
538.01	0\\
539.01	0\\
540.01	0\\
541.01	0\\
542.01	0\\
543.01	0\\
544.01	0\\
545.01	0\\
546.01	0\\
547.01	0\\
548.01	0\\
549.01	0\\
550.01	0\\
551.01	0\\
552.01	0\\
553.01	0\\
554.01	0\\
555.01	0\\
556.01	0\\
557.01	0\\
558.01	0\\
559.01	0\\
560.01	0\\
561.01	0\\
562.01	0\\
563.01	0\\
564.01	0\\
565.01	0\\
566.01	0\\
567.01	0\\
568.01	0\\
569.01	0\\
570.01	0\\
571.01	0\\
572.01	0\\
573.01	0\\
574.01	0\\
575.01	0\\
576.01	0\\
577.01	0\\
578.01	0\\
579.01	0\\
580.01	0\\
581.01	0\\
582.01	0\\
583.01	0\\
584.01	0\\
585.01	0\\
586.01	0\\
587.01	0\\
588.01	0\\
589.01	0\\
590.01	0\\
591.01	0\\
592.01	0\\
593.01	0\\
594.01	0\\
595.01	0\\
596.01	0\\
597.01	0.000481187488961005\\
598.01	0.00143832068185176\\
599.01	0.00385661257211614\\
599.02	0.00389414818835221\\
599.03	0.00393204146009075\\
599.04	0.00397029584140049\\
599.05	0.00400891481968836\\
599.06	0.004047901916021\\
599.07	0.00408726068544937\\
599.08	0.00412699471733652\\
599.09	0.00416710763568842\\
599.1	0.00420760309948809\\
599.11	0.0042484848030329\\
599.12	0.00428975647627509\\
599.13	0.00433142188516562\\
599.14	0.00437348483200131\\
599.15	0.00441594915577533\\
599.16	0.00445881873253107\\
599.17	0.0045020974757194\\
599.18	0.00454578933655944\\
599.19	0.00458989830440268\\
599.2	0.0046344284071007\\
599.21	0.00467938371137645\\
599.22	0.00472476832319906\\
599.23	0.0047705863881622\\
599.24	0.00481684209186628\\
599.25	0.0048635396603041\\
599.26	0.00491068334955364\\
599.27	0.00495827744958434\\
599.28	0.00500632629158766\\
599.29	0.00505483424837286\\
599.3	0.00510380573476639\\
599.31	0.00515324520801533\\
599.32	0.00520315716819451\\
599.33	0.0052535461586177\\
599.34	0.00530441676625255\\
599.35	0.00535577362213968\\
599.36	0.00540762140181575\\
599.37	0.00545996482574049\\
599.38	0.00551280865972795\\
599.39	0.00556615771538178\\
599.4	0.00562001685053481\\
599.41	0.0056743909696927\\
599.42	0.00572928502448196\\
599.43	0.00578470401410212\\
599.44	0.0058406529857825\\
599.45	0.00589713703524301\\
599.46	0.00595416130715971\\
599.47	0.00601173099563457\\
599.48	0.00606985134466987\\
599.49	0.00612852764864711\\
599.5	0.00618776525281053\\
599.51	0.00624756955375524\\
599.52	0.00630794599992004\\
599.53	0.006368900092085\\
599.54	0.00643043738387369\\
599.55	0.00649256348226044\\
599.56	0.00655528404808221\\
599.57	0.00661860479655558\\
599.58	0.00668253149779856\\
599.59	0.00674706997735745\\
599.6	0.00681222611673878\\
599.61	0.0068780058539463\\
599.62	0.00694441518402313\\
599.63	0.00701146015959912\\
599.64	0.00707914689144345\\
599.65	0.00714748154902253\\
599.66	0.00721647036106324\\
599.67	0.00728611961612158\\
599.68	0.00735643566315675\\
599.69	0.00742742491211078\\
599.7	0.00749909383449363\\
599.71	0.00757144896397399\\
599.72	0.00764449689697571\\
599.73	0.00771824429327989\\
599.74	0.00779269787663285\\
599.75	0.00786786443535983\\
599.76	0.00794375082298461\\
599.77	0.0080203639588551\\
599.78	0.00809771082877486\\
599.79	0.00817579848564071\\
599.8	0.00825463405008649\\
599.81	0.00833422471113285\\
599.82	0.00841457772684349\\
599.83	0.00849570042498747\\
599.84	0.00857760020370797\\
599.85	0.00866028453219752\\
599.86	0.00874376095137953\\
599.87	0.00882803707459654\\
599.88	0.0089131205883049\\
599.89	0.00899901925277625\\
599.9	0.00908574090280562\\
599.91	0.00917329344842636\\
599.92	0.0092616848756319\\
599.93	0.00935092324710449\\
599.94	0.00944101670295078\\
599.95	0.00953197346144465\\
599.96	0.00962380181977693\\
599.97	0.00971651015481255\\
599.98	0.0098101069238547\\
599.99	0.00990460066541651\\
600	0.01\\
};
\addplot [color=black!20!mycolor21,solid,forget plot]
  table[row sep=crcr]{%
0.01	0\\
1.01	0\\
2.01	0\\
3.01	0\\
4.01	0\\
5.01	0\\
6.01	0\\
7.01	0\\
8.01	0\\
9.01	0\\
10.01	0\\
11.01	0\\
12.01	0\\
13.01	0\\
14.01	0\\
15.01	0\\
16.01	0\\
17.01	0\\
18.01	0\\
19.01	0\\
20.01	0\\
21.01	0\\
22.01	0\\
23.01	0\\
24.01	0\\
25.01	0\\
26.01	0\\
27.01	0\\
28.01	0\\
29.01	0\\
30.01	0\\
31.01	0\\
32.01	0\\
33.01	0\\
34.01	0\\
35.01	0\\
36.01	0\\
37.01	0\\
38.01	0\\
39.01	0\\
40.01	0\\
41.01	0\\
42.01	0\\
43.01	0\\
44.01	0\\
45.01	0\\
46.01	0\\
47.01	0\\
48.01	0\\
49.01	0\\
50.01	0\\
51.01	0\\
52.01	0\\
53.01	0\\
54.01	0\\
55.01	0\\
56.01	0\\
57.01	0\\
58.01	0\\
59.01	0\\
60.01	0\\
61.01	0\\
62.01	0\\
63.01	0\\
64.01	0\\
65.01	0\\
66.01	0\\
67.01	0\\
68.01	0\\
69.01	0\\
70.01	0\\
71.01	0\\
72.01	0\\
73.01	0\\
74.01	0\\
75.01	0\\
76.01	0\\
77.01	0\\
78.01	0\\
79.01	0\\
80.01	0\\
81.01	0\\
82.01	0\\
83.01	0\\
84.01	0\\
85.01	0\\
86.01	0\\
87.01	0\\
88.01	0\\
89.01	0\\
90.01	0\\
91.01	0\\
92.01	0\\
93.01	0\\
94.01	0\\
95.01	0\\
96.01	0\\
97.01	0\\
98.01	0\\
99.01	0\\
100.01	0\\
101.01	0\\
102.01	0\\
103.01	0\\
104.01	0\\
105.01	0\\
106.01	0\\
107.01	0\\
108.01	0\\
109.01	0\\
110.01	0\\
111.01	0\\
112.01	0\\
113.01	0\\
114.01	0\\
115.01	0\\
116.01	0\\
117.01	0\\
118.01	0\\
119.01	0\\
120.01	0\\
121.01	0\\
122.01	0\\
123.01	0\\
124.01	0\\
125.01	0\\
126.01	0\\
127.01	0\\
128.01	0\\
129.01	0\\
130.01	0\\
131.01	0\\
132.01	0\\
133.01	0\\
134.01	0\\
135.01	0\\
136.01	0\\
137.01	0\\
138.01	0\\
139.01	0\\
140.01	0\\
141.01	0\\
142.01	0\\
143.01	0\\
144.01	0\\
145.01	0\\
146.01	0\\
147.01	0\\
148.01	0\\
149.01	0\\
150.01	0\\
151.01	0\\
152.01	0\\
153.01	0\\
154.01	0\\
155.01	0\\
156.01	0\\
157.01	0\\
158.01	0\\
159.01	0\\
160.01	0\\
161.01	0\\
162.01	0\\
163.01	0\\
164.01	0\\
165.01	0\\
166.01	0\\
167.01	0\\
168.01	0\\
169.01	0\\
170.01	0\\
171.01	0\\
172.01	0\\
173.01	0\\
174.01	0\\
175.01	0\\
176.01	0\\
177.01	0\\
178.01	0\\
179.01	0\\
180.01	0\\
181.01	0\\
182.01	0\\
183.01	0\\
184.01	0\\
185.01	0\\
186.01	0\\
187.01	0\\
188.01	0\\
189.01	0\\
190.01	0\\
191.01	0\\
192.01	0\\
193.01	0\\
194.01	0\\
195.01	0\\
196.01	0\\
197.01	0\\
198.01	0\\
199.01	0\\
200.01	0\\
201.01	0\\
202.01	0\\
203.01	0\\
204.01	0\\
205.01	0\\
206.01	0\\
207.01	0\\
208.01	0\\
209.01	0\\
210.01	0\\
211.01	0\\
212.01	0\\
213.01	0\\
214.01	0\\
215.01	0\\
216.01	0\\
217.01	0\\
218.01	0\\
219.01	0\\
220.01	0\\
221.01	0\\
222.01	0\\
223.01	0\\
224.01	0\\
225.01	0\\
226.01	0\\
227.01	0\\
228.01	0\\
229.01	0\\
230.01	0\\
231.01	0\\
232.01	0\\
233.01	0\\
234.01	0\\
235.01	0\\
236.01	0\\
237.01	0\\
238.01	0\\
239.01	0\\
240.01	0\\
241.01	0\\
242.01	0\\
243.01	0\\
244.01	0\\
245.01	0\\
246.01	0\\
247.01	0\\
248.01	0\\
249.01	0\\
250.01	0\\
251.01	0\\
252.01	0\\
253.01	0\\
254.01	0\\
255.01	0\\
256.01	0\\
257.01	0\\
258.01	0\\
259.01	0\\
260.01	0\\
261.01	0\\
262.01	0\\
263.01	0\\
264.01	0\\
265.01	0\\
266.01	0\\
267.01	0\\
268.01	0\\
269.01	0\\
270.01	0\\
271.01	0\\
272.01	0\\
273.01	0\\
274.01	0\\
275.01	0\\
276.01	0\\
277.01	0\\
278.01	0\\
279.01	0\\
280.01	0\\
281.01	0\\
282.01	0\\
283.01	0\\
284.01	0\\
285.01	0\\
286.01	0\\
287.01	0\\
288.01	0\\
289.01	0\\
290.01	0\\
291.01	0\\
292.01	0\\
293.01	0\\
294.01	0\\
295.01	0\\
296.01	0\\
297.01	0\\
298.01	0\\
299.01	0\\
300.01	0\\
301.01	0\\
302.01	0\\
303.01	0\\
304.01	0\\
305.01	0\\
306.01	0\\
307.01	0\\
308.01	0\\
309.01	0\\
310.01	0\\
311.01	0\\
312.01	0\\
313.01	0\\
314.01	0\\
315.01	0\\
316.01	0\\
317.01	0\\
318.01	0\\
319.01	0\\
320.01	0\\
321.01	0\\
322.01	0\\
323.01	0\\
324.01	0\\
325.01	0\\
326.01	0\\
327.01	0\\
328.01	0\\
329.01	0\\
330.01	0\\
331.01	0\\
332.01	0\\
333.01	0\\
334.01	0\\
335.01	0\\
336.01	0\\
337.01	0\\
338.01	0\\
339.01	0\\
340.01	0\\
341.01	0\\
342.01	0\\
343.01	0\\
344.01	0\\
345.01	0\\
346.01	0\\
347.01	0\\
348.01	0\\
349.01	0\\
350.01	0\\
351.01	0\\
352.01	0\\
353.01	0\\
354.01	0\\
355.01	0\\
356.01	0\\
357.01	0\\
358.01	0\\
359.01	0\\
360.01	0\\
361.01	0\\
362.01	0\\
363.01	0\\
364.01	0\\
365.01	0\\
366.01	0\\
367.01	0\\
368.01	0\\
369.01	0\\
370.01	0\\
371.01	0\\
372.01	0\\
373.01	0\\
374.01	0\\
375.01	0\\
376.01	0\\
377.01	0\\
378.01	0\\
379.01	0\\
380.01	0\\
381.01	0\\
382.01	0\\
383.01	0\\
384.01	0\\
385.01	0\\
386.01	0\\
387.01	0\\
388.01	0\\
389.01	0\\
390.01	0\\
391.01	0\\
392.01	0\\
393.01	0\\
394.01	0\\
395.01	0\\
396.01	0\\
397.01	0\\
398.01	0\\
399.01	0\\
400.01	0\\
401.01	0\\
402.01	0\\
403.01	0\\
404.01	0\\
405.01	0\\
406.01	0\\
407.01	0\\
408.01	0\\
409.01	0\\
410.01	0\\
411.01	0\\
412.01	0\\
413.01	0\\
414.01	0\\
415.01	0\\
416.01	0\\
417.01	0\\
418.01	0\\
419.01	0\\
420.01	0\\
421.01	0\\
422.01	0\\
423.01	0\\
424.01	0\\
425.01	0\\
426.01	0\\
427.01	0\\
428.01	0\\
429.01	0\\
430.01	0\\
431.01	0\\
432.01	0\\
433.01	0\\
434.01	0\\
435.01	0\\
436.01	0\\
437.01	0\\
438.01	0\\
439.01	0\\
440.01	0\\
441.01	0\\
442.01	0\\
443.01	0\\
444.01	0\\
445.01	0\\
446.01	0\\
447.01	0\\
448.01	0\\
449.01	0\\
450.01	0\\
451.01	0\\
452.01	0\\
453.01	0\\
454.01	0\\
455.01	0\\
456.01	0\\
457.01	0\\
458.01	0\\
459.01	0\\
460.01	0\\
461.01	0\\
462.01	0\\
463.01	0\\
464.01	0\\
465.01	0\\
466.01	0\\
467.01	0\\
468.01	0\\
469.01	0\\
470.01	0\\
471.01	0\\
472.01	0\\
473.01	0\\
474.01	0\\
475.01	0\\
476.01	0\\
477.01	0\\
478.01	0\\
479.01	0\\
480.01	0\\
481.01	0\\
482.01	0\\
483.01	0\\
484.01	0\\
485.01	0\\
486.01	0\\
487.01	0\\
488.01	0\\
489.01	0\\
490.01	0\\
491.01	0\\
492.01	0\\
493.01	0\\
494.01	0\\
495.01	0\\
496.01	0\\
497.01	0\\
498.01	0\\
499.01	0\\
500.01	0\\
501.01	0\\
502.01	0\\
503.01	0\\
504.01	0\\
505.01	0\\
506.01	0\\
507.01	0\\
508.01	0\\
509.01	0\\
510.01	0\\
511.01	0\\
512.01	0\\
513.01	0\\
514.01	0\\
515.01	0\\
516.01	0\\
517.01	0\\
518.01	0\\
519.01	0\\
520.01	0\\
521.01	0\\
522.01	0\\
523.01	0\\
524.01	0\\
525.01	0\\
526.01	0\\
527.01	0\\
528.01	0\\
529.01	0\\
530.01	0\\
531.01	0\\
532.01	0\\
533.01	0\\
534.01	0\\
535.01	0\\
536.01	0\\
537.01	0\\
538.01	0\\
539.01	0\\
540.01	0\\
541.01	0\\
542.01	0\\
543.01	0\\
544.01	0\\
545.01	0\\
546.01	0\\
547.01	0\\
548.01	0\\
549.01	0\\
550.01	0\\
551.01	0\\
552.01	0\\
553.01	0\\
554.01	0\\
555.01	0\\
556.01	0\\
557.01	0\\
558.01	0\\
559.01	0\\
560.01	0\\
561.01	0\\
562.01	0\\
563.01	0\\
564.01	0\\
565.01	0\\
566.01	0\\
567.01	0\\
568.01	0\\
569.01	0\\
570.01	0\\
571.01	0\\
572.01	0\\
573.01	0\\
574.01	0\\
575.01	0\\
576.01	0\\
577.01	0\\
578.01	0\\
579.01	0\\
580.01	0\\
581.01	0\\
582.01	0\\
583.01	0\\
584.01	0\\
585.01	0\\
586.01	0\\
587.01	0\\
588.01	0\\
589.01	0\\
590.01	0\\
591.01	0\\
592.01	0\\
593.01	0\\
594.01	0\\
595.01	0\\
596.01	0\\
597.01	0.000481249826739166\\
598.01	0.00143832068185196\\
599.01	0.00385661257211625\\
599.02	0.00389414818835232\\
599.03	0.00393204146009087\\
599.04	0.0039702958414006\\
599.05	0.00400891481968846\\
599.06	0.00404790191602108\\
599.07	0.00408726068544946\\
599.08	0.0041269947173366\\
599.09	0.00416710763568851\\
599.1	0.00420760309948819\\
599.11	0.00424848480303298\\
599.12	0.00428975647627516\\
599.13	0.00433142188516568\\
599.14	0.00437348483200135\\
599.15	0.00441594915577537\\
599.16	0.00445881873253111\\
599.17	0.00450209747571946\\
599.18	0.00454578933655951\\
599.19	0.00458989830440275\\
599.2	0.00463442840710078\\
599.21	0.00467938371137652\\
599.22	0.00472476832319911\\
599.23	0.00477058638816225\\
599.24	0.00481684209186631\\
599.25	0.00486353966030413\\
599.26	0.00491068334955368\\
599.27	0.00495827744958437\\
599.28	0.0050063262915877\\
599.29	0.0050548342483729\\
599.3	0.00510380573476643\\
599.31	0.00515324520801537\\
599.32	0.00520315716819457\\
599.33	0.00525354615861774\\
599.34	0.00530441676625258\\
599.35	0.00535577362213972\\
599.36	0.00540762140181579\\
599.37	0.00545996482574053\\
599.38	0.00551280865972797\\
599.39	0.00556615771538181\\
599.4	0.00562001685053486\\
599.41	0.00567439096969276\\
599.42	0.00572928502448199\\
599.43	0.00578470401410215\\
599.44	0.00584065298578251\\
599.45	0.00589713703524303\\
599.46	0.00595416130715974\\
599.47	0.00601173099563461\\
599.48	0.0060698513446699\\
599.49	0.00612852764864714\\
599.5	0.00618776525281057\\
599.51	0.00624756955375527\\
599.52	0.00630794599992007\\
599.53	0.00636890009208502\\
599.54	0.00643043738387371\\
599.55	0.00649256348226046\\
599.56	0.00655528404808223\\
599.57	0.00661860479655561\\
599.58	0.00668253149779858\\
599.59	0.00674706997735747\\
599.6	0.0068122261167388\\
599.61	0.00687800585394632\\
599.62	0.00694441518402315\\
599.63	0.00701146015959913\\
599.64	0.00707914689144346\\
599.65	0.00714748154902253\\
599.66	0.00721647036106324\\
599.67	0.00728611961612158\\
599.68	0.00735643566315675\\
599.69	0.00742742491211078\\
599.7	0.00749909383449363\\
599.71	0.00757144896397399\\
599.72	0.0076444968969757\\
599.73	0.00771824429327989\\
599.74	0.00779269787663284\\
599.75	0.00786786443535982\\
599.76	0.0079437508229846\\
599.77	0.0080203639588551\\
599.78	0.00809771082877485\\
599.79	0.00817579848564071\\
599.8	0.00825463405008648\\
599.81	0.00833422471113284\\
599.82	0.00841457772684348\\
599.83	0.00849570042498746\\
599.84	0.00857760020370798\\
599.85	0.00866028453219752\\
599.86	0.00874376095137953\\
599.87	0.00882803707459653\\
599.88	0.00891312058830489\\
599.89	0.00899901925277624\\
599.9	0.00908574090280562\\
599.91	0.00917329344842635\\
599.92	0.0092616848756319\\
599.93	0.00935092324710449\\
599.94	0.00944101670295078\\
599.95	0.00953197346144465\\
599.96	0.00962380181977694\\
599.97	0.00971651015481255\\
599.98	0.0098101069238547\\
599.99	0.00990460066541651\\
600	0.01\\
};
\addplot [color=black!50!mycolor20,solid,forget plot]
  table[row sep=crcr]{%
0.01	0\\
1.01	0\\
2.01	0\\
3.01	0\\
4.01	0\\
5.01	0\\
6.01	0\\
7.01	0\\
8.01	0\\
9.01	0\\
10.01	0\\
11.01	0\\
12.01	0\\
13.01	0\\
14.01	0\\
15.01	0\\
16.01	0\\
17.01	0\\
18.01	0\\
19.01	0\\
20.01	0\\
21.01	0\\
22.01	0\\
23.01	0\\
24.01	0\\
25.01	0\\
26.01	0\\
27.01	0\\
28.01	0\\
29.01	0\\
30.01	0\\
31.01	0\\
32.01	0\\
33.01	0\\
34.01	0\\
35.01	0\\
36.01	0\\
37.01	0\\
38.01	0\\
39.01	0\\
40.01	0\\
41.01	0\\
42.01	0\\
43.01	0\\
44.01	0\\
45.01	0\\
46.01	0\\
47.01	0\\
48.01	0\\
49.01	0\\
50.01	0\\
51.01	0\\
52.01	0\\
53.01	0\\
54.01	0\\
55.01	0\\
56.01	0\\
57.01	0\\
58.01	0\\
59.01	0\\
60.01	0\\
61.01	0\\
62.01	0\\
63.01	0\\
64.01	0\\
65.01	0\\
66.01	0\\
67.01	0\\
68.01	0\\
69.01	0\\
70.01	0\\
71.01	0\\
72.01	0\\
73.01	0\\
74.01	0\\
75.01	0\\
76.01	0\\
77.01	0\\
78.01	0\\
79.01	0\\
80.01	0\\
81.01	0\\
82.01	0\\
83.01	0\\
84.01	0\\
85.01	0\\
86.01	0\\
87.01	0\\
88.01	0\\
89.01	0\\
90.01	0\\
91.01	0\\
92.01	0\\
93.01	0\\
94.01	0\\
95.01	0\\
96.01	0\\
97.01	0\\
98.01	0\\
99.01	0\\
100.01	0\\
101.01	0\\
102.01	0\\
103.01	0\\
104.01	0\\
105.01	0\\
106.01	0\\
107.01	0\\
108.01	0\\
109.01	0\\
110.01	0\\
111.01	0\\
112.01	0\\
113.01	0\\
114.01	0\\
115.01	0\\
116.01	0\\
117.01	0\\
118.01	0\\
119.01	0\\
120.01	0\\
121.01	0\\
122.01	0\\
123.01	0\\
124.01	0\\
125.01	0\\
126.01	0\\
127.01	0\\
128.01	0\\
129.01	0\\
130.01	0\\
131.01	0\\
132.01	0\\
133.01	0\\
134.01	0\\
135.01	0\\
136.01	0\\
137.01	0\\
138.01	0\\
139.01	0\\
140.01	0\\
141.01	0\\
142.01	0\\
143.01	0\\
144.01	0\\
145.01	0\\
146.01	0\\
147.01	0\\
148.01	0\\
149.01	0\\
150.01	0\\
151.01	0\\
152.01	0\\
153.01	0\\
154.01	0\\
155.01	0\\
156.01	0\\
157.01	0\\
158.01	0\\
159.01	0\\
160.01	0\\
161.01	0\\
162.01	0\\
163.01	0\\
164.01	0\\
165.01	0\\
166.01	0\\
167.01	0\\
168.01	0\\
169.01	0\\
170.01	0\\
171.01	0\\
172.01	0\\
173.01	0\\
174.01	0\\
175.01	0\\
176.01	0\\
177.01	0\\
178.01	0\\
179.01	0\\
180.01	0\\
181.01	0\\
182.01	0\\
183.01	0\\
184.01	0\\
185.01	0\\
186.01	0\\
187.01	0\\
188.01	0\\
189.01	0\\
190.01	0\\
191.01	0\\
192.01	0\\
193.01	0\\
194.01	0\\
195.01	0\\
196.01	0\\
197.01	0\\
198.01	0\\
199.01	0\\
200.01	0\\
201.01	0\\
202.01	0\\
203.01	0\\
204.01	0\\
205.01	0\\
206.01	0\\
207.01	0\\
208.01	0\\
209.01	0\\
210.01	0\\
211.01	0\\
212.01	0\\
213.01	0\\
214.01	0\\
215.01	0\\
216.01	0\\
217.01	0\\
218.01	0\\
219.01	0\\
220.01	0\\
221.01	0\\
222.01	0\\
223.01	0\\
224.01	0\\
225.01	0\\
226.01	0\\
227.01	0\\
228.01	0\\
229.01	0\\
230.01	0\\
231.01	0\\
232.01	0\\
233.01	0\\
234.01	0\\
235.01	0\\
236.01	0\\
237.01	0\\
238.01	0\\
239.01	0\\
240.01	0\\
241.01	0\\
242.01	0\\
243.01	0\\
244.01	0\\
245.01	0\\
246.01	0\\
247.01	0\\
248.01	0\\
249.01	0\\
250.01	0\\
251.01	0\\
252.01	0\\
253.01	0\\
254.01	0\\
255.01	0\\
256.01	0\\
257.01	0\\
258.01	0\\
259.01	0\\
260.01	0\\
261.01	0\\
262.01	0\\
263.01	0\\
264.01	0\\
265.01	0\\
266.01	0\\
267.01	0\\
268.01	0\\
269.01	0\\
270.01	0\\
271.01	0\\
272.01	0\\
273.01	0\\
274.01	0\\
275.01	0\\
276.01	0\\
277.01	0\\
278.01	0\\
279.01	0\\
280.01	0\\
281.01	0\\
282.01	0\\
283.01	0\\
284.01	0\\
285.01	0\\
286.01	0\\
287.01	0\\
288.01	0\\
289.01	0\\
290.01	0\\
291.01	0\\
292.01	0\\
293.01	0\\
294.01	0\\
295.01	0\\
296.01	0\\
297.01	0\\
298.01	0\\
299.01	0\\
300.01	0\\
301.01	0\\
302.01	0\\
303.01	0\\
304.01	0\\
305.01	0\\
306.01	0\\
307.01	0\\
308.01	0\\
309.01	0\\
310.01	0\\
311.01	0\\
312.01	0\\
313.01	0\\
314.01	0\\
315.01	0\\
316.01	0\\
317.01	0\\
318.01	0\\
319.01	0\\
320.01	0\\
321.01	0\\
322.01	0\\
323.01	0\\
324.01	0\\
325.01	0\\
326.01	0\\
327.01	0\\
328.01	0\\
329.01	0\\
330.01	0\\
331.01	0\\
332.01	0\\
333.01	0\\
334.01	0\\
335.01	0\\
336.01	0\\
337.01	0\\
338.01	0\\
339.01	0\\
340.01	0\\
341.01	0\\
342.01	0\\
343.01	0\\
344.01	0\\
345.01	0\\
346.01	0\\
347.01	0\\
348.01	0\\
349.01	0\\
350.01	0\\
351.01	0\\
352.01	0\\
353.01	0\\
354.01	0\\
355.01	0\\
356.01	0\\
357.01	0\\
358.01	0\\
359.01	0\\
360.01	0\\
361.01	0\\
362.01	0\\
363.01	0\\
364.01	0\\
365.01	0\\
366.01	0\\
367.01	0\\
368.01	0\\
369.01	0\\
370.01	0\\
371.01	0\\
372.01	0\\
373.01	0\\
374.01	0\\
375.01	0\\
376.01	0\\
377.01	0\\
378.01	0\\
379.01	0\\
380.01	0\\
381.01	0\\
382.01	0\\
383.01	0\\
384.01	0\\
385.01	0\\
386.01	0\\
387.01	0\\
388.01	0\\
389.01	0\\
390.01	0\\
391.01	0\\
392.01	0\\
393.01	0\\
394.01	0\\
395.01	0\\
396.01	0\\
397.01	0\\
398.01	0\\
399.01	0\\
400.01	0\\
401.01	0\\
402.01	0\\
403.01	0\\
404.01	0\\
405.01	0\\
406.01	0\\
407.01	0\\
408.01	0\\
409.01	0\\
410.01	0\\
411.01	0\\
412.01	0\\
413.01	0\\
414.01	0\\
415.01	0\\
416.01	0\\
417.01	0\\
418.01	0\\
419.01	0\\
420.01	0\\
421.01	0\\
422.01	0\\
423.01	0\\
424.01	0\\
425.01	0\\
426.01	0\\
427.01	0\\
428.01	0\\
429.01	0\\
430.01	0\\
431.01	0\\
432.01	0\\
433.01	0\\
434.01	0\\
435.01	0\\
436.01	0\\
437.01	0\\
438.01	0\\
439.01	0\\
440.01	0\\
441.01	0\\
442.01	0\\
443.01	0\\
444.01	0\\
445.01	0\\
446.01	0\\
447.01	0\\
448.01	0\\
449.01	0\\
450.01	0\\
451.01	0\\
452.01	0\\
453.01	0\\
454.01	0\\
455.01	0\\
456.01	0\\
457.01	0\\
458.01	0\\
459.01	0\\
460.01	0\\
461.01	0\\
462.01	0\\
463.01	0\\
464.01	0\\
465.01	0\\
466.01	0\\
467.01	0\\
468.01	0\\
469.01	0\\
470.01	0\\
471.01	0\\
472.01	0\\
473.01	0\\
474.01	0\\
475.01	0\\
476.01	0\\
477.01	0\\
478.01	0\\
479.01	0\\
480.01	0\\
481.01	0\\
482.01	0\\
483.01	0\\
484.01	0\\
485.01	0\\
486.01	0\\
487.01	0\\
488.01	0\\
489.01	0\\
490.01	0\\
491.01	0\\
492.01	0\\
493.01	0\\
494.01	0\\
495.01	0\\
496.01	0\\
497.01	0\\
498.01	0\\
499.01	0\\
500.01	0\\
501.01	0\\
502.01	0\\
503.01	0\\
504.01	0\\
505.01	0\\
506.01	0\\
507.01	0\\
508.01	0\\
509.01	0\\
510.01	0\\
511.01	0\\
512.01	0\\
513.01	0\\
514.01	0\\
515.01	0\\
516.01	0\\
517.01	0\\
518.01	0\\
519.01	0\\
520.01	0\\
521.01	0\\
522.01	0\\
523.01	0\\
524.01	0\\
525.01	0\\
526.01	0\\
527.01	0\\
528.01	0\\
529.01	0\\
530.01	0\\
531.01	0\\
532.01	0\\
533.01	0\\
534.01	0\\
535.01	0\\
536.01	0\\
537.01	0\\
538.01	0\\
539.01	0\\
540.01	0\\
541.01	0\\
542.01	0\\
543.01	0\\
544.01	0\\
545.01	0\\
546.01	0\\
547.01	0\\
548.01	0\\
549.01	0\\
550.01	0\\
551.01	0\\
552.01	0\\
553.01	0\\
554.01	0\\
555.01	0\\
556.01	0\\
557.01	0\\
558.01	0\\
559.01	0\\
560.01	0\\
561.01	0\\
562.01	0\\
563.01	0\\
564.01	0\\
565.01	0\\
566.01	0\\
567.01	0\\
568.01	0\\
569.01	0\\
570.01	0\\
571.01	0\\
572.01	0\\
573.01	0\\
574.01	0\\
575.01	0\\
576.01	0\\
577.01	0\\
578.01	0\\
579.01	0\\
580.01	0\\
581.01	0\\
582.01	0\\
583.01	0\\
584.01	0\\
585.01	0\\
586.01	0\\
587.01	0\\
588.01	0\\
589.01	0\\
590.01	0\\
591.01	0\\
592.01	0\\
593.01	0\\
594.01	0\\
595.01	0\\
596.01	0\\
597.01	0.000481304278336731\\
598.01	0.0014383206818519\\
599.01	0.00385661257211628\\
599.02	0.00389414818835235\\
599.03	0.00393204146009089\\
599.04	0.00397029584140063\\
599.05	0.00400891481968849\\
599.06	0.00404790191602111\\
599.07	0.00408726068544948\\
599.08	0.00412699471733663\\
599.09	0.00416710763568853\\
599.1	0.00420760309948821\\
599.11	0.00424848480303303\\
599.12	0.00428975647627521\\
599.13	0.00433142188516572\\
599.14	0.0043734848320014\\
599.15	0.00441594915577541\\
599.16	0.00445881873253114\\
599.17	0.00450209747571947\\
599.18	0.00454578933655951\\
599.19	0.00458989830440273\\
599.2	0.00463442840710075\\
599.21	0.00467938371137651\\
599.22	0.0047247683231991\\
599.23	0.00477058638816223\\
599.24	0.00481684209186631\\
599.25	0.00486353966030412\\
599.26	0.00491068334955366\\
599.27	0.00495827744958437\\
599.28	0.00500632629158769\\
599.29	0.00505483424837289\\
599.3	0.00510380573476643\\
599.31	0.00515324520801538\\
599.32	0.00520315716819458\\
599.33	0.00525354615861777\\
599.34	0.00530441676625261\\
599.35	0.00535577362213975\\
599.36	0.00540762140181582\\
599.37	0.00545996482574056\\
599.38	0.005512808659728\\
599.39	0.00556615771538184\\
599.4	0.00562001685053487\\
599.41	0.00567439096969276\\
599.42	0.005729285024482\\
599.43	0.00578470401410217\\
599.44	0.00584065298578254\\
599.45	0.00589713703524307\\
599.46	0.00595416130715977\\
599.47	0.00601173099563464\\
599.48	0.00606985134466993\\
599.49	0.00612852764864717\\
599.5	0.00618776525281059\\
599.51	0.00624756955375529\\
599.52	0.00630794599992009\\
599.53	0.00636890009208503\\
599.54	0.00643043738387374\\
599.55	0.00649256348226047\\
599.56	0.00655528404808225\\
599.57	0.00661860479655561\\
599.58	0.00668253149779858\\
599.59	0.00674706997735747\\
599.6	0.0068122261167388\\
599.61	0.00687800585394632\\
599.62	0.00694441518402315\\
599.63	0.00701146015959914\\
599.64	0.00707914689144346\\
599.65	0.00714748154902254\\
599.66	0.00721647036106324\\
599.67	0.00728611961612158\\
599.68	0.00735643566315676\\
599.69	0.00742742491211078\\
599.7	0.00749909383449363\\
599.71	0.007571448963974\\
599.72	0.00764449689697572\\
599.73	0.00771824429327989\\
599.74	0.00779269787663285\\
599.75	0.00786786443535983\\
599.76	0.00794375082298461\\
599.77	0.0080203639588551\\
599.78	0.00809771082877486\\
599.79	0.00817579848564072\\
599.8	0.00825463405008649\\
599.81	0.00833422471113286\\
599.82	0.00841457772684349\\
599.83	0.00849570042498747\\
599.84	0.00857760020370798\\
599.85	0.00866028453219753\\
599.86	0.00874376095137954\\
599.87	0.00882803707459654\\
599.88	0.0089131205883049\\
599.89	0.00899901925277626\\
599.9	0.00908574090280562\\
599.91	0.00917329344842636\\
599.92	0.00926168487563191\\
599.93	0.00935092324710449\\
599.94	0.00944101670295079\\
599.95	0.00953197346144465\\
599.96	0.00962380181977693\\
599.97	0.00971651015481255\\
599.98	0.0098101069238547\\
599.99	0.00990460066541651\\
600	0.01\\
};
\addplot [color=black!60!mycolor21,solid,forget plot]
  table[row sep=crcr]{%
0.01	0\\
1.01	0\\
2.01	0\\
3.01	0\\
4.01	0\\
5.01	0\\
6.01	0\\
7.01	0\\
8.01	0\\
9.01	0\\
10.01	0\\
11.01	0\\
12.01	0\\
13.01	0\\
14.01	0\\
15.01	0\\
16.01	0\\
17.01	0\\
18.01	0\\
19.01	0\\
20.01	0\\
21.01	0\\
22.01	0\\
23.01	0\\
24.01	0\\
25.01	0\\
26.01	0\\
27.01	0\\
28.01	0\\
29.01	0\\
30.01	0\\
31.01	0\\
32.01	0\\
33.01	0\\
34.01	0\\
35.01	0\\
36.01	0\\
37.01	0\\
38.01	0\\
39.01	0\\
40.01	0\\
41.01	0\\
42.01	0\\
43.01	0\\
44.01	0\\
45.01	0\\
46.01	0\\
47.01	0\\
48.01	0\\
49.01	0\\
50.01	0\\
51.01	0\\
52.01	0\\
53.01	0\\
54.01	0\\
55.01	0\\
56.01	0\\
57.01	0\\
58.01	0\\
59.01	0\\
60.01	0\\
61.01	0\\
62.01	0\\
63.01	0\\
64.01	0\\
65.01	0\\
66.01	0\\
67.01	0\\
68.01	0\\
69.01	0\\
70.01	0\\
71.01	0\\
72.01	0\\
73.01	0\\
74.01	0\\
75.01	0\\
76.01	0\\
77.01	0\\
78.01	0\\
79.01	0\\
80.01	0\\
81.01	0\\
82.01	0\\
83.01	0\\
84.01	0\\
85.01	0\\
86.01	0\\
87.01	0\\
88.01	0\\
89.01	0\\
90.01	0\\
91.01	0\\
92.01	0\\
93.01	0\\
94.01	0\\
95.01	0\\
96.01	0\\
97.01	0\\
98.01	0\\
99.01	0\\
100.01	0\\
101.01	0\\
102.01	0\\
103.01	0\\
104.01	0\\
105.01	0\\
106.01	0\\
107.01	0\\
108.01	0\\
109.01	0\\
110.01	0\\
111.01	0\\
112.01	0\\
113.01	0\\
114.01	0\\
115.01	0\\
116.01	0\\
117.01	0\\
118.01	0\\
119.01	0\\
120.01	0\\
121.01	0\\
122.01	0\\
123.01	0\\
124.01	0\\
125.01	0\\
126.01	0\\
127.01	0\\
128.01	0\\
129.01	0\\
130.01	0\\
131.01	0\\
132.01	0\\
133.01	0\\
134.01	0\\
135.01	0\\
136.01	0\\
137.01	0\\
138.01	0\\
139.01	0\\
140.01	0\\
141.01	0\\
142.01	0\\
143.01	0\\
144.01	0\\
145.01	0\\
146.01	0\\
147.01	0\\
148.01	0\\
149.01	0\\
150.01	0\\
151.01	0\\
152.01	0\\
153.01	0\\
154.01	0\\
155.01	0\\
156.01	0\\
157.01	0\\
158.01	0\\
159.01	0\\
160.01	0\\
161.01	0\\
162.01	0\\
163.01	0\\
164.01	0\\
165.01	0\\
166.01	0\\
167.01	0\\
168.01	0\\
169.01	0\\
170.01	0\\
171.01	0\\
172.01	0\\
173.01	0\\
174.01	0\\
175.01	0\\
176.01	0\\
177.01	0\\
178.01	0\\
179.01	0\\
180.01	0\\
181.01	0\\
182.01	0\\
183.01	0\\
184.01	0\\
185.01	0\\
186.01	0\\
187.01	0\\
188.01	0\\
189.01	0\\
190.01	0\\
191.01	0\\
192.01	0\\
193.01	0\\
194.01	0\\
195.01	0\\
196.01	0\\
197.01	0\\
198.01	0\\
199.01	0\\
200.01	0\\
201.01	0\\
202.01	0\\
203.01	0\\
204.01	0\\
205.01	0\\
206.01	0\\
207.01	0\\
208.01	0\\
209.01	0\\
210.01	0\\
211.01	0\\
212.01	0\\
213.01	0\\
214.01	0\\
215.01	0\\
216.01	0\\
217.01	0\\
218.01	0\\
219.01	0\\
220.01	0\\
221.01	0\\
222.01	0\\
223.01	0\\
224.01	0\\
225.01	0\\
226.01	0\\
227.01	0\\
228.01	0\\
229.01	0\\
230.01	0\\
231.01	0\\
232.01	0\\
233.01	0\\
234.01	0\\
235.01	0\\
236.01	0\\
237.01	0\\
238.01	0\\
239.01	0\\
240.01	0\\
241.01	0\\
242.01	0\\
243.01	0\\
244.01	0\\
245.01	0\\
246.01	0\\
247.01	0\\
248.01	0\\
249.01	0\\
250.01	0\\
251.01	0\\
252.01	0\\
253.01	0\\
254.01	0\\
255.01	0\\
256.01	0\\
257.01	0\\
258.01	0\\
259.01	0\\
260.01	0\\
261.01	0\\
262.01	0\\
263.01	0\\
264.01	0\\
265.01	0\\
266.01	0\\
267.01	0\\
268.01	0\\
269.01	0\\
270.01	0\\
271.01	0\\
272.01	0\\
273.01	0\\
274.01	0\\
275.01	0\\
276.01	0\\
277.01	0\\
278.01	0\\
279.01	0\\
280.01	0\\
281.01	0\\
282.01	0\\
283.01	0\\
284.01	0\\
285.01	0\\
286.01	0\\
287.01	0\\
288.01	0\\
289.01	0\\
290.01	0\\
291.01	0\\
292.01	0\\
293.01	0\\
294.01	0\\
295.01	0\\
296.01	0\\
297.01	0\\
298.01	0\\
299.01	0\\
300.01	0\\
301.01	0\\
302.01	0\\
303.01	0\\
304.01	0\\
305.01	0\\
306.01	0\\
307.01	0\\
308.01	0\\
309.01	0\\
310.01	0\\
311.01	0\\
312.01	0\\
313.01	0\\
314.01	0\\
315.01	0\\
316.01	0\\
317.01	0\\
318.01	0\\
319.01	0\\
320.01	0\\
321.01	0\\
322.01	0\\
323.01	0\\
324.01	0\\
325.01	0\\
326.01	0\\
327.01	0\\
328.01	0\\
329.01	0\\
330.01	0\\
331.01	0\\
332.01	0\\
333.01	0\\
334.01	0\\
335.01	0\\
336.01	0\\
337.01	0\\
338.01	0\\
339.01	0\\
340.01	0\\
341.01	0\\
342.01	0\\
343.01	0\\
344.01	0\\
345.01	0\\
346.01	0\\
347.01	0\\
348.01	0\\
349.01	0\\
350.01	0\\
351.01	0\\
352.01	0\\
353.01	0\\
354.01	0\\
355.01	0\\
356.01	0\\
357.01	0\\
358.01	0\\
359.01	0\\
360.01	0\\
361.01	0\\
362.01	0\\
363.01	0\\
364.01	0\\
365.01	0\\
366.01	0\\
367.01	0\\
368.01	0\\
369.01	0\\
370.01	0\\
371.01	0\\
372.01	0\\
373.01	0\\
374.01	0\\
375.01	0\\
376.01	0\\
377.01	0\\
378.01	0\\
379.01	0\\
380.01	0\\
381.01	0\\
382.01	0\\
383.01	0\\
384.01	0\\
385.01	0\\
386.01	0\\
387.01	0\\
388.01	0\\
389.01	0\\
390.01	0\\
391.01	0\\
392.01	0\\
393.01	0\\
394.01	0\\
395.01	0\\
396.01	0\\
397.01	0\\
398.01	0\\
399.01	0\\
400.01	0\\
401.01	0\\
402.01	0\\
403.01	0\\
404.01	0\\
405.01	0\\
406.01	0\\
407.01	0\\
408.01	0\\
409.01	0\\
410.01	0\\
411.01	0\\
412.01	0\\
413.01	0\\
414.01	0\\
415.01	0\\
416.01	0\\
417.01	0\\
418.01	0\\
419.01	0\\
420.01	0\\
421.01	0\\
422.01	0\\
423.01	0\\
424.01	0\\
425.01	0\\
426.01	0\\
427.01	0\\
428.01	0\\
429.01	0\\
430.01	0\\
431.01	0\\
432.01	0\\
433.01	0\\
434.01	0\\
435.01	0\\
436.01	0\\
437.01	0\\
438.01	0\\
439.01	0\\
440.01	0\\
441.01	0\\
442.01	0\\
443.01	0\\
444.01	0\\
445.01	0\\
446.01	0\\
447.01	0\\
448.01	0\\
449.01	0\\
450.01	0\\
451.01	0\\
452.01	0\\
453.01	0\\
454.01	0\\
455.01	0\\
456.01	0\\
457.01	0\\
458.01	0\\
459.01	0\\
460.01	0\\
461.01	0\\
462.01	0\\
463.01	0\\
464.01	0\\
465.01	0\\
466.01	0\\
467.01	0\\
468.01	0\\
469.01	0\\
470.01	0\\
471.01	0\\
472.01	0\\
473.01	0\\
474.01	0\\
475.01	0\\
476.01	0\\
477.01	0\\
478.01	0\\
479.01	0\\
480.01	0\\
481.01	0\\
482.01	0\\
483.01	0\\
484.01	0\\
485.01	0\\
486.01	0\\
487.01	0\\
488.01	0\\
489.01	0\\
490.01	0\\
491.01	0\\
492.01	0\\
493.01	0\\
494.01	0\\
495.01	0\\
496.01	0\\
497.01	0\\
498.01	0\\
499.01	0\\
500.01	0\\
501.01	0\\
502.01	0\\
503.01	0\\
504.01	0\\
505.01	0\\
506.01	0\\
507.01	0\\
508.01	0\\
509.01	0\\
510.01	0\\
511.01	0\\
512.01	0\\
513.01	0\\
514.01	0\\
515.01	0\\
516.01	0\\
517.01	0\\
518.01	0\\
519.01	0\\
520.01	0\\
521.01	0\\
522.01	0\\
523.01	0\\
524.01	0\\
525.01	0\\
526.01	0\\
527.01	0\\
528.01	0\\
529.01	0\\
530.01	0\\
531.01	0\\
532.01	0\\
533.01	0\\
534.01	0\\
535.01	0\\
536.01	0\\
537.01	0\\
538.01	0\\
539.01	0\\
540.01	0\\
541.01	0\\
542.01	0\\
543.01	0\\
544.01	0\\
545.01	0\\
546.01	0\\
547.01	0\\
548.01	0\\
549.01	0\\
550.01	0\\
551.01	0\\
552.01	0\\
553.01	0\\
554.01	0\\
555.01	0\\
556.01	0\\
557.01	0\\
558.01	0\\
559.01	0\\
560.01	0\\
561.01	0\\
562.01	0\\
563.01	0\\
564.01	0\\
565.01	0\\
566.01	0\\
567.01	0\\
568.01	0\\
569.01	0\\
570.01	0\\
571.01	0\\
572.01	0\\
573.01	0\\
574.01	0\\
575.01	0\\
576.01	0\\
577.01	0\\
578.01	0\\
579.01	0\\
580.01	0\\
581.01	0\\
582.01	0\\
583.01	0\\
584.01	0\\
585.01	0\\
586.01	0\\
587.01	0\\
588.01	0\\
589.01	0\\
590.01	0\\
591.01	0\\
592.01	0\\
593.01	0\\
594.01	0\\
595.01	0\\
596.01	0\\
597.01	0.000481322506726112\\
598.01	0.00143832068185176\\
599.01	0.00385661257211617\\
599.02	0.00389414818835224\\
599.03	0.00393204146009077\\
599.04	0.00397029584140052\\
599.05	0.00400891481968839\\
599.06	0.00404790191602103\\
599.07	0.0040872606854494\\
599.08	0.00412699471733655\\
599.09	0.00416710763568844\\
599.1	0.0042076030994881\\
599.11	0.00424848480303291\\
599.12	0.00428975647627511\\
599.13	0.00433142188516564\\
599.14	0.00437348483200133\\
599.15	0.00441594915577534\\
599.16	0.00445881873253108\\
599.17	0.00450209747571942\\
599.18	0.00454578933655947\\
599.19	0.00458989830440271\\
599.2	0.00463442840710072\\
599.21	0.00467938371137648\\
599.22	0.00472476832319908\\
599.23	0.00477058638816223\\
599.24	0.00481684209186631\\
599.25	0.00486353966030413\\
599.26	0.00491068334955366\\
599.27	0.00495827744958435\\
599.28	0.00500632629158769\\
599.29	0.00505483424837289\\
599.3	0.00510380573476642\\
599.31	0.00515324520801536\\
599.32	0.00520315716819454\\
599.33	0.00525354615861773\\
599.34	0.00530441676625257\\
599.35	0.0053557736221397\\
599.36	0.00540762140181576\\
599.37	0.0054599648257405\\
599.38	0.00551280865972796\\
599.39	0.00556615771538181\\
599.4	0.00562001685053484\\
599.41	0.00567439096969274\\
599.42	0.00572928502448199\\
599.43	0.00578470401410215\\
599.44	0.00584065298578251\\
599.45	0.00589713703524303\\
599.46	0.00595416130715973\\
599.47	0.00601173099563458\\
599.48	0.00606985134466989\\
599.49	0.00612852764864713\\
599.5	0.00618776525281055\\
599.51	0.00624756955375526\\
599.52	0.00630794599992006\\
599.53	0.006368900092085\\
599.54	0.00643043738387369\\
599.55	0.00649256348226043\\
599.56	0.00655528404808221\\
599.57	0.00661860479655559\\
599.58	0.00668253149779856\\
599.59	0.00674706997735745\\
599.6	0.00681222611673879\\
599.61	0.00687800585394631\\
599.62	0.00694441518402313\\
599.63	0.00701146015959912\\
599.64	0.00707914689144344\\
599.65	0.00714748154902252\\
599.66	0.00721647036106323\\
599.67	0.00728611961612157\\
599.68	0.00735643566315675\\
599.69	0.00742742491211077\\
599.7	0.00749909383449363\\
599.71	0.00757144896397399\\
599.72	0.0076444968969757\\
599.73	0.00771824429327989\\
599.74	0.00779269787663284\\
599.75	0.00786786443535981\\
599.76	0.0079437508229846\\
599.77	0.00802036395885509\\
599.78	0.00809771082877485\\
599.79	0.00817579848564071\\
599.8	0.00825463405008648\\
599.81	0.00833422471113284\\
599.82	0.00841457772684349\\
599.83	0.00849570042498746\\
599.84	0.00857760020370797\\
599.85	0.00866028453219752\\
599.86	0.00874376095137954\\
599.87	0.00882803707459654\\
599.88	0.0089131205883049\\
599.89	0.00899901925277625\\
599.9	0.00908574090280562\\
599.91	0.00917329344842635\\
599.92	0.0092616848756319\\
599.93	0.00935092324710449\\
599.94	0.00944101670295078\\
599.95	0.00953197346144465\\
599.96	0.00962380181977693\\
599.97	0.00971651015481255\\
599.98	0.0098101069238547\\
599.99	0.00990460066541651\\
600	0.01\\
};
\addplot [color=black!80!mycolor21,solid,forget plot]
  table[row sep=crcr]{%
0.01	0\\
1.01	0\\
2.01	0\\
3.01	0\\
4.01	0\\
5.01	0\\
6.01	0\\
7.01	0\\
8.01	0\\
9.01	0\\
10.01	0\\
11.01	0\\
12.01	0\\
13.01	0\\
14.01	0\\
15.01	0\\
16.01	0\\
17.01	0\\
18.01	0\\
19.01	0\\
20.01	0\\
21.01	0\\
22.01	0\\
23.01	0\\
24.01	0\\
25.01	0\\
26.01	0\\
27.01	0\\
28.01	0\\
29.01	0\\
30.01	0\\
31.01	0\\
32.01	0\\
33.01	0\\
34.01	0\\
35.01	0\\
36.01	0\\
37.01	0\\
38.01	0\\
39.01	0\\
40.01	0\\
41.01	0\\
42.01	0\\
43.01	0\\
44.01	0\\
45.01	0\\
46.01	0\\
47.01	0\\
48.01	0\\
49.01	0\\
50.01	0\\
51.01	0\\
52.01	0\\
53.01	0\\
54.01	0\\
55.01	0\\
56.01	0\\
57.01	0\\
58.01	0\\
59.01	0\\
60.01	0\\
61.01	0\\
62.01	0\\
63.01	0\\
64.01	0\\
65.01	0\\
66.01	0\\
67.01	0\\
68.01	0\\
69.01	0\\
70.01	0\\
71.01	0\\
72.01	0\\
73.01	0\\
74.01	0\\
75.01	0\\
76.01	0\\
77.01	0\\
78.01	0\\
79.01	0\\
80.01	0\\
81.01	0\\
82.01	0\\
83.01	0\\
84.01	0\\
85.01	0\\
86.01	0\\
87.01	0\\
88.01	0\\
89.01	0\\
90.01	0\\
91.01	0\\
92.01	0\\
93.01	0\\
94.01	0\\
95.01	0\\
96.01	0\\
97.01	0\\
98.01	0\\
99.01	0\\
100.01	0\\
101.01	0\\
102.01	0\\
103.01	0\\
104.01	0\\
105.01	0\\
106.01	0\\
107.01	0\\
108.01	0\\
109.01	0\\
110.01	0\\
111.01	0\\
112.01	0\\
113.01	0\\
114.01	0\\
115.01	0\\
116.01	0\\
117.01	0\\
118.01	0\\
119.01	0\\
120.01	0\\
121.01	0\\
122.01	0\\
123.01	0\\
124.01	0\\
125.01	0\\
126.01	0\\
127.01	0\\
128.01	0\\
129.01	0\\
130.01	0\\
131.01	0\\
132.01	0\\
133.01	0\\
134.01	0\\
135.01	0\\
136.01	0\\
137.01	0\\
138.01	0\\
139.01	0\\
140.01	0\\
141.01	0\\
142.01	0\\
143.01	0\\
144.01	0\\
145.01	0\\
146.01	0\\
147.01	0\\
148.01	0\\
149.01	0\\
150.01	0\\
151.01	0\\
152.01	0\\
153.01	0\\
154.01	0\\
155.01	0\\
156.01	0\\
157.01	0\\
158.01	0\\
159.01	0\\
160.01	0\\
161.01	0\\
162.01	0\\
163.01	0\\
164.01	0\\
165.01	0\\
166.01	0\\
167.01	0\\
168.01	0\\
169.01	0\\
170.01	0\\
171.01	0\\
172.01	0\\
173.01	0\\
174.01	0\\
175.01	0\\
176.01	0\\
177.01	0\\
178.01	0\\
179.01	0\\
180.01	0\\
181.01	0\\
182.01	0\\
183.01	0\\
184.01	0\\
185.01	0\\
186.01	0\\
187.01	0\\
188.01	0\\
189.01	0\\
190.01	0\\
191.01	0\\
192.01	0\\
193.01	0\\
194.01	0\\
195.01	0\\
196.01	0\\
197.01	0\\
198.01	0\\
199.01	0\\
200.01	0\\
201.01	0\\
202.01	0\\
203.01	0\\
204.01	0\\
205.01	0\\
206.01	0\\
207.01	0\\
208.01	0\\
209.01	0\\
210.01	0\\
211.01	0\\
212.01	0\\
213.01	0\\
214.01	0\\
215.01	0\\
216.01	0\\
217.01	0\\
218.01	0\\
219.01	0\\
220.01	0\\
221.01	0\\
222.01	0\\
223.01	0\\
224.01	0\\
225.01	0\\
226.01	0\\
227.01	0\\
228.01	0\\
229.01	0\\
230.01	0\\
231.01	0\\
232.01	0\\
233.01	0\\
234.01	0\\
235.01	0\\
236.01	0\\
237.01	0\\
238.01	0\\
239.01	0\\
240.01	0\\
241.01	0\\
242.01	0\\
243.01	0\\
244.01	0\\
245.01	0\\
246.01	0\\
247.01	0\\
248.01	0\\
249.01	0\\
250.01	0\\
251.01	0\\
252.01	0\\
253.01	0\\
254.01	0\\
255.01	0\\
256.01	0\\
257.01	0\\
258.01	0\\
259.01	0\\
260.01	0\\
261.01	0\\
262.01	0\\
263.01	0\\
264.01	0\\
265.01	0\\
266.01	0\\
267.01	0\\
268.01	0\\
269.01	0\\
270.01	0\\
271.01	0\\
272.01	0\\
273.01	0\\
274.01	0\\
275.01	0\\
276.01	0\\
277.01	0\\
278.01	0\\
279.01	0\\
280.01	0\\
281.01	0\\
282.01	0\\
283.01	0\\
284.01	0\\
285.01	0\\
286.01	0\\
287.01	0\\
288.01	0\\
289.01	0\\
290.01	0\\
291.01	0\\
292.01	0\\
293.01	0\\
294.01	0\\
295.01	0\\
296.01	0\\
297.01	0\\
298.01	0\\
299.01	0\\
300.01	0\\
301.01	0\\
302.01	0\\
303.01	0\\
304.01	0\\
305.01	0\\
306.01	0\\
307.01	0\\
308.01	0\\
309.01	0\\
310.01	0\\
311.01	0\\
312.01	0\\
313.01	0\\
314.01	0\\
315.01	0\\
316.01	0\\
317.01	0\\
318.01	0\\
319.01	0\\
320.01	0\\
321.01	0\\
322.01	0\\
323.01	0\\
324.01	0\\
325.01	0\\
326.01	0\\
327.01	0\\
328.01	0\\
329.01	0\\
330.01	0\\
331.01	0\\
332.01	0\\
333.01	0\\
334.01	0\\
335.01	0\\
336.01	0\\
337.01	0\\
338.01	0\\
339.01	0\\
340.01	0\\
341.01	0\\
342.01	0\\
343.01	0\\
344.01	0\\
345.01	0\\
346.01	0\\
347.01	0\\
348.01	0\\
349.01	0\\
350.01	0\\
351.01	0\\
352.01	0\\
353.01	0\\
354.01	0\\
355.01	0\\
356.01	0\\
357.01	0\\
358.01	0\\
359.01	0\\
360.01	0\\
361.01	0\\
362.01	0\\
363.01	0\\
364.01	0\\
365.01	0\\
366.01	0\\
367.01	0\\
368.01	0\\
369.01	0\\
370.01	0\\
371.01	0\\
372.01	0\\
373.01	0\\
374.01	0\\
375.01	0\\
376.01	0\\
377.01	0\\
378.01	0\\
379.01	0\\
380.01	0\\
381.01	0\\
382.01	0\\
383.01	0\\
384.01	0\\
385.01	0\\
386.01	0\\
387.01	0\\
388.01	0\\
389.01	0\\
390.01	0\\
391.01	0\\
392.01	0\\
393.01	0\\
394.01	0\\
395.01	0\\
396.01	0\\
397.01	0\\
398.01	0\\
399.01	0\\
400.01	0\\
401.01	0\\
402.01	0\\
403.01	0\\
404.01	0\\
405.01	0\\
406.01	0\\
407.01	0\\
408.01	0\\
409.01	0\\
410.01	0\\
411.01	0\\
412.01	0\\
413.01	0\\
414.01	0\\
415.01	0\\
416.01	0\\
417.01	0\\
418.01	0\\
419.01	0\\
420.01	0\\
421.01	0\\
422.01	0\\
423.01	0\\
424.01	0\\
425.01	0\\
426.01	0\\
427.01	0\\
428.01	0\\
429.01	0\\
430.01	0\\
431.01	0\\
432.01	0\\
433.01	0\\
434.01	0\\
435.01	0\\
436.01	0\\
437.01	0\\
438.01	0\\
439.01	0\\
440.01	0\\
441.01	0\\
442.01	0\\
443.01	0\\
444.01	0\\
445.01	0\\
446.01	0\\
447.01	0\\
448.01	0\\
449.01	0\\
450.01	0\\
451.01	0\\
452.01	0\\
453.01	0\\
454.01	0\\
455.01	0\\
456.01	0\\
457.01	0\\
458.01	0\\
459.01	0\\
460.01	0\\
461.01	0\\
462.01	0\\
463.01	0\\
464.01	0\\
465.01	0\\
466.01	0\\
467.01	0\\
468.01	0\\
469.01	0\\
470.01	0\\
471.01	0\\
472.01	0\\
473.01	0\\
474.01	0\\
475.01	0\\
476.01	0\\
477.01	0\\
478.01	0\\
479.01	0\\
480.01	0\\
481.01	0\\
482.01	0\\
483.01	0\\
484.01	0\\
485.01	0\\
486.01	0\\
487.01	0\\
488.01	0\\
489.01	0\\
490.01	0\\
491.01	0\\
492.01	0\\
493.01	0\\
494.01	0\\
495.01	0\\
496.01	0\\
497.01	0\\
498.01	0\\
499.01	0\\
500.01	0\\
501.01	0\\
502.01	0\\
503.01	0\\
504.01	0\\
505.01	0\\
506.01	0\\
507.01	0\\
508.01	0\\
509.01	0\\
510.01	0\\
511.01	0\\
512.01	0\\
513.01	0\\
514.01	0\\
515.01	0\\
516.01	0\\
517.01	0\\
518.01	0\\
519.01	0\\
520.01	0\\
521.01	0\\
522.01	0\\
523.01	0\\
524.01	0\\
525.01	0\\
526.01	0\\
527.01	0\\
528.01	0\\
529.01	0\\
530.01	0\\
531.01	0\\
532.01	0\\
533.01	0\\
534.01	0\\
535.01	0\\
536.01	0\\
537.01	0\\
538.01	0\\
539.01	0\\
540.01	0\\
541.01	0\\
542.01	0\\
543.01	0\\
544.01	0\\
545.01	0\\
546.01	0\\
547.01	0\\
548.01	0\\
549.01	0\\
550.01	0\\
551.01	0\\
552.01	0\\
553.01	0\\
554.01	0\\
555.01	0\\
556.01	0\\
557.01	0\\
558.01	0\\
559.01	0\\
560.01	0\\
561.01	0\\
562.01	0\\
563.01	0\\
564.01	0\\
565.01	0\\
566.01	0\\
567.01	0\\
568.01	0\\
569.01	0\\
570.01	0\\
571.01	0\\
572.01	0\\
573.01	0\\
574.01	0\\
575.01	0\\
576.01	0\\
577.01	0\\
578.01	0\\
579.01	0\\
580.01	0\\
581.01	0\\
582.01	0\\
583.01	0\\
584.01	0\\
585.01	0\\
586.01	0\\
587.01	0\\
588.01	0\\
589.01	0\\
590.01	0\\
591.01	0\\
592.01	0\\
593.01	0\\
594.01	0\\
595.01	0\\
596.01	0\\
597.01	0.000481328854172736\\
598.01	0.00143832068185187\\
599.01	0.00385661257211624\\
599.02	0.00389414818835231\\
599.03	0.00393204146009085\\
599.04	0.00397029584140059\\
599.05	0.00400891481968844\\
599.06	0.00404790191602107\\
599.07	0.00408726068544944\\
599.08	0.00412699471733657\\
599.09	0.00416710763568848\\
599.1	0.00420760309948816\\
599.11	0.00424848480303296\\
599.12	0.00428975647627514\\
599.13	0.00433142188516565\\
599.14	0.00437348483200133\\
599.15	0.00441594915577534\\
599.16	0.00445881873253108\\
599.17	0.00450209747571943\\
599.18	0.00454578933655947\\
599.19	0.00458989830440271\\
599.2	0.00463442840710074\\
599.21	0.00467938371137648\\
599.22	0.00472476832319907\\
599.23	0.0047705863881622\\
599.24	0.00481684209186627\\
599.25	0.00486353966030409\\
599.26	0.00491068334955364\\
599.27	0.00495827744958434\\
599.28	0.00500632629158768\\
599.29	0.00505483424837287\\
599.3	0.0051038057347664\\
599.31	0.00515324520801534\\
599.32	0.00520315716819454\\
599.33	0.00525354615861773\\
599.34	0.00530441676625258\\
599.35	0.00535577362213972\\
599.36	0.00540762140181578\\
599.37	0.0054599648257405\\
599.38	0.00551280865972795\\
599.39	0.00556615771538178\\
599.4	0.00562001685053483\\
599.41	0.00567439096969272\\
599.42	0.00572928502448196\\
599.43	0.00578470401410212\\
599.44	0.00584065298578249\\
599.45	0.005897137035243\\
599.46	0.00595416130715971\\
599.47	0.00601173099563457\\
599.48	0.00606985134466986\\
599.49	0.0061285276486471\\
599.5	0.00618776525281052\\
599.51	0.00624756955375523\\
599.52	0.00630794599992004\\
599.53	0.006368900092085\\
599.54	0.00643043738387369\\
599.55	0.00649256348226045\\
599.56	0.00655528404808221\\
599.57	0.00661860479655557\\
599.58	0.00668253149779854\\
599.59	0.00674706997735744\\
599.6	0.00681222611673878\\
599.61	0.0068780058539463\\
599.62	0.00694441518402312\\
599.63	0.00701146015959912\\
599.64	0.00707914689144344\\
599.65	0.00714748154902253\\
599.66	0.00721647036106324\\
599.67	0.00728611961612158\\
599.68	0.00735643566315676\\
599.69	0.00742742491211079\\
599.7	0.00749909383449364\\
599.71	0.007571448963974\\
599.72	0.00764449689697572\\
599.73	0.0077182442932799\\
599.74	0.00779269787663286\\
599.75	0.00786786443535983\\
599.76	0.00794375082298462\\
599.77	0.00802036395885511\\
599.78	0.00809771082877486\\
599.79	0.00817579848564072\\
599.8	0.00825463405008649\\
599.81	0.00833422471113286\\
599.82	0.0084145777268435\\
599.83	0.00849570042498747\\
599.84	0.00857760020370798\\
599.85	0.00866028453219752\\
599.86	0.00874376095137954\\
599.87	0.00882803707459654\\
599.88	0.0089131205883049\\
599.89	0.00899901925277625\\
599.9	0.00908574090280562\\
599.91	0.00917329344842635\\
599.92	0.0092616848756319\\
599.93	0.00935092324710449\\
599.94	0.00944101670295079\\
599.95	0.00953197346144465\\
599.96	0.00962380181977693\\
599.97	0.00971651015481255\\
599.98	0.0098101069238547\\
599.99	0.00990460066541651\\
600	0.01\\
};
\addplot [color=black,solid,forget plot]
  table[row sep=crcr]{%
0.01	0\\
1.01	0\\
2.01	0\\
3.01	0\\
4.01	0\\
5.01	0\\
6.01	0\\
7.01	0\\
8.01	0\\
9.01	0\\
10.01	0\\
11.01	0\\
12.01	0\\
13.01	0\\
14.01	0\\
15.01	0\\
16.01	0\\
17.01	0\\
18.01	0\\
19.01	0\\
20.01	0\\
21.01	0\\
22.01	0\\
23.01	0\\
24.01	0\\
25.01	0\\
26.01	0\\
27.01	0\\
28.01	0\\
29.01	0\\
30.01	0\\
31.01	0\\
32.01	0\\
33.01	0\\
34.01	0\\
35.01	0\\
36.01	0\\
37.01	0\\
38.01	0\\
39.01	0\\
40.01	0\\
41.01	0\\
42.01	0\\
43.01	0\\
44.01	0\\
45.01	0\\
46.01	0\\
47.01	0\\
48.01	0\\
49.01	0\\
50.01	0\\
51.01	0\\
52.01	0\\
53.01	0\\
54.01	0\\
55.01	0\\
56.01	0\\
57.01	0\\
58.01	0\\
59.01	0\\
60.01	0\\
61.01	0\\
62.01	0\\
63.01	0\\
64.01	0\\
65.01	0\\
66.01	0\\
67.01	0\\
68.01	0\\
69.01	0\\
70.01	0\\
71.01	0\\
72.01	0\\
73.01	0\\
74.01	0\\
75.01	0\\
76.01	0\\
77.01	0\\
78.01	0\\
79.01	0\\
80.01	0\\
81.01	0\\
82.01	0\\
83.01	0\\
84.01	0\\
85.01	0\\
86.01	0\\
87.01	0\\
88.01	0\\
89.01	0\\
90.01	0\\
91.01	0\\
92.01	0\\
93.01	0\\
94.01	0\\
95.01	0\\
96.01	0\\
97.01	0\\
98.01	0\\
99.01	0\\
100.01	0\\
101.01	0\\
102.01	0\\
103.01	0\\
104.01	0\\
105.01	0\\
106.01	0\\
107.01	0\\
108.01	0\\
109.01	0\\
110.01	0\\
111.01	0\\
112.01	0\\
113.01	0\\
114.01	0\\
115.01	0\\
116.01	0\\
117.01	0\\
118.01	0\\
119.01	0\\
120.01	0\\
121.01	0\\
122.01	0\\
123.01	0\\
124.01	0\\
125.01	0\\
126.01	0\\
127.01	0\\
128.01	0\\
129.01	0\\
130.01	0\\
131.01	0\\
132.01	0\\
133.01	0\\
134.01	0\\
135.01	0\\
136.01	0\\
137.01	0\\
138.01	0\\
139.01	0\\
140.01	0\\
141.01	0\\
142.01	0\\
143.01	0\\
144.01	0\\
145.01	0\\
146.01	0\\
147.01	0\\
148.01	0\\
149.01	0\\
150.01	0\\
151.01	0\\
152.01	0\\
153.01	0\\
154.01	0\\
155.01	0\\
156.01	0\\
157.01	0\\
158.01	0\\
159.01	0\\
160.01	0\\
161.01	0\\
162.01	0\\
163.01	0\\
164.01	0\\
165.01	0\\
166.01	0\\
167.01	0\\
168.01	0\\
169.01	0\\
170.01	0\\
171.01	0\\
172.01	0\\
173.01	0\\
174.01	0\\
175.01	0\\
176.01	0\\
177.01	0\\
178.01	0\\
179.01	0\\
180.01	0\\
181.01	0\\
182.01	0\\
183.01	0\\
184.01	0\\
185.01	0\\
186.01	0\\
187.01	0\\
188.01	0\\
189.01	0\\
190.01	0\\
191.01	0\\
192.01	0\\
193.01	0\\
194.01	0\\
195.01	0\\
196.01	0\\
197.01	0\\
198.01	0\\
199.01	0\\
200.01	0\\
201.01	0\\
202.01	0\\
203.01	0\\
204.01	0\\
205.01	0\\
206.01	0\\
207.01	0\\
208.01	0\\
209.01	0\\
210.01	0\\
211.01	0\\
212.01	0\\
213.01	0\\
214.01	0\\
215.01	0\\
216.01	0\\
217.01	0\\
218.01	0\\
219.01	0\\
220.01	0\\
221.01	0\\
222.01	0\\
223.01	0\\
224.01	0\\
225.01	0\\
226.01	0\\
227.01	0\\
228.01	0\\
229.01	0\\
230.01	0\\
231.01	0\\
232.01	0\\
233.01	0\\
234.01	0\\
235.01	0\\
236.01	0\\
237.01	0\\
238.01	0\\
239.01	0\\
240.01	0\\
241.01	0\\
242.01	0\\
243.01	0\\
244.01	0\\
245.01	0\\
246.01	0\\
247.01	0\\
248.01	0\\
249.01	0\\
250.01	0\\
251.01	0\\
252.01	0\\
253.01	0\\
254.01	0\\
255.01	0\\
256.01	0\\
257.01	0\\
258.01	0\\
259.01	0\\
260.01	0\\
261.01	0\\
262.01	0\\
263.01	0\\
264.01	0\\
265.01	0\\
266.01	0\\
267.01	0\\
268.01	0\\
269.01	0\\
270.01	0\\
271.01	0\\
272.01	0\\
273.01	0\\
274.01	0\\
275.01	0\\
276.01	0\\
277.01	0\\
278.01	0\\
279.01	0\\
280.01	0\\
281.01	0\\
282.01	0\\
283.01	0\\
284.01	0\\
285.01	0\\
286.01	0\\
287.01	0\\
288.01	0\\
289.01	0\\
290.01	0\\
291.01	0\\
292.01	0\\
293.01	0\\
294.01	0\\
295.01	0\\
296.01	0\\
297.01	0\\
298.01	0\\
299.01	0\\
300.01	0\\
301.01	0\\
302.01	0\\
303.01	0\\
304.01	0\\
305.01	0\\
306.01	0\\
307.01	0\\
308.01	0\\
309.01	0\\
310.01	0\\
311.01	0\\
312.01	0\\
313.01	0\\
314.01	0\\
315.01	0\\
316.01	0\\
317.01	0\\
318.01	0\\
319.01	0\\
320.01	0\\
321.01	0\\
322.01	0\\
323.01	0\\
324.01	0\\
325.01	0\\
326.01	0\\
327.01	0\\
328.01	0\\
329.01	0\\
330.01	0\\
331.01	0\\
332.01	0\\
333.01	0\\
334.01	0\\
335.01	0\\
336.01	0\\
337.01	0\\
338.01	0\\
339.01	0\\
340.01	0\\
341.01	0\\
342.01	0\\
343.01	0\\
344.01	0\\
345.01	0\\
346.01	0\\
347.01	0\\
348.01	0\\
349.01	0\\
350.01	0\\
351.01	0\\
352.01	0\\
353.01	0\\
354.01	0\\
355.01	0\\
356.01	0\\
357.01	0\\
358.01	0\\
359.01	0\\
360.01	0\\
361.01	0\\
362.01	0\\
363.01	0\\
364.01	0\\
365.01	0\\
366.01	0\\
367.01	0\\
368.01	0\\
369.01	0\\
370.01	0\\
371.01	0\\
372.01	0\\
373.01	0\\
374.01	0\\
375.01	0\\
376.01	0\\
377.01	0\\
378.01	0\\
379.01	0\\
380.01	0\\
381.01	0\\
382.01	0\\
383.01	0\\
384.01	0\\
385.01	0\\
386.01	0\\
387.01	0\\
388.01	0\\
389.01	0\\
390.01	0\\
391.01	0\\
392.01	0\\
393.01	0\\
394.01	0\\
395.01	0\\
396.01	0\\
397.01	0\\
398.01	0\\
399.01	0\\
400.01	0\\
401.01	0\\
402.01	0\\
403.01	0\\
404.01	0\\
405.01	0\\
406.01	0\\
407.01	0\\
408.01	0\\
409.01	0\\
410.01	0\\
411.01	0\\
412.01	0\\
413.01	0\\
414.01	0\\
415.01	0\\
416.01	0\\
417.01	0\\
418.01	0\\
419.01	0\\
420.01	0\\
421.01	0\\
422.01	0\\
423.01	0\\
424.01	0\\
425.01	0\\
426.01	0\\
427.01	0\\
428.01	0\\
429.01	0\\
430.01	0\\
431.01	0\\
432.01	0\\
433.01	0\\
434.01	0\\
435.01	0\\
436.01	0\\
437.01	0\\
438.01	0\\
439.01	0\\
440.01	0\\
441.01	0\\
442.01	0\\
443.01	0\\
444.01	0\\
445.01	0\\
446.01	0\\
447.01	0\\
448.01	0\\
449.01	0\\
450.01	0\\
451.01	0\\
452.01	0\\
453.01	0\\
454.01	0\\
455.01	0\\
456.01	0\\
457.01	0\\
458.01	0\\
459.01	0\\
460.01	0\\
461.01	0\\
462.01	0\\
463.01	0\\
464.01	0\\
465.01	0\\
466.01	0\\
467.01	0\\
468.01	0\\
469.01	0\\
470.01	0\\
471.01	0\\
472.01	0\\
473.01	0\\
474.01	0\\
475.01	0\\
476.01	0\\
477.01	0\\
478.01	0\\
479.01	0\\
480.01	0\\
481.01	0\\
482.01	0\\
483.01	0\\
484.01	0\\
485.01	0\\
486.01	0\\
487.01	0\\
488.01	0\\
489.01	0\\
490.01	0\\
491.01	0\\
492.01	0\\
493.01	0\\
494.01	0\\
495.01	0\\
496.01	0\\
497.01	0\\
498.01	0\\
499.01	0\\
500.01	0\\
501.01	0\\
502.01	0\\
503.01	0\\
504.01	0\\
505.01	0\\
506.01	0\\
507.01	0\\
508.01	0\\
509.01	0\\
510.01	0\\
511.01	0\\
512.01	0\\
513.01	0\\
514.01	0\\
515.01	0\\
516.01	0\\
517.01	0\\
518.01	0\\
519.01	0\\
520.01	0\\
521.01	0\\
522.01	0\\
523.01	0\\
524.01	0\\
525.01	0\\
526.01	0\\
527.01	0\\
528.01	0\\
529.01	0\\
530.01	0\\
531.01	0\\
532.01	0\\
533.01	0\\
534.01	0\\
535.01	0\\
536.01	0\\
537.01	0\\
538.01	0\\
539.01	0\\
540.01	0\\
541.01	0\\
542.01	0\\
543.01	0\\
544.01	0\\
545.01	0\\
546.01	0\\
547.01	0\\
548.01	0\\
549.01	0\\
550.01	0\\
551.01	0\\
552.01	0\\
553.01	0\\
554.01	0\\
555.01	0\\
556.01	0\\
557.01	0\\
558.01	0\\
559.01	0\\
560.01	0\\
561.01	0\\
562.01	0\\
563.01	0\\
564.01	0\\
565.01	0\\
566.01	0\\
567.01	0\\
568.01	0\\
569.01	0\\
570.01	0\\
571.01	0\\
572.01	0\\
573.01	0\\
574.01	0\\
575.01	0\\
576.01	0\\
577.01	0\\
578.01	0\\
579.01	0\\
580.01	0\\
581.01	0\\
582.01	0\\
583.01	0\\
584.01	0\\
585.01	0\\
586.01	0\\
587.01	0\\
588.01	0\\
589.01	0\\
590.01	0\\
591.01	0\\
592.01	0\\
593.01	0\\
594.01	0\\
595.01	0\\
596.01	0\\
597.01	0.00048133368794398\\
598.01	0.00143832068185187\\
599.01	0.00385661257211624\\
599.02	0.00389414818835231\\
599.03	0.00393204146009084\\
599.04	0.00397029584140057\\
599.05	0.00400891481968844\\
599.06	0.00404790191602107\\
599.07	0.00408726068544944\\
599.08	0.00412699471733659\\
599.09	0.00416710763568849\\
599.1	0.00420760309948817\\
599.11	0.00424848480303298\\
599.12	0.00428975647627516\\
599.13	0.00433142188516569\\
599.14	0.00437348483200138\\
599.15	0.00441594915577539\\
599.16	0.00445881873253112\\
599.17	0.00450209747571946\\
599.18	0.0045457893365595\\
599.19	0.00458989830440273\\
599.2	0.00463442840710075\\
599.21	0.00467938371137651\\
599.22	0.0047247683231991\\
599.23	0.00477058638816223\\
599.24	0.00481684209186631\\
599.25	0.00486353966030412\\
599.26	0.00491068334955366\\
599.27	0.00495827744958435\\
599.28	0.00500632629158768\\
599.29	0.00505483424837287\\
599.3	0.00510380573476642\\
599.31	0.00515324520801536\\
599.32	0.00520315716819456\\
599.33	0.00525354615861774\\
599.34	0.00530441676625258\\
599.35	0.00535577362213972\\
599.36	0.00540762140181579\\
599.37	0.00545996482574053\\
599.38	0.00551280865972797\\
599.39	0.00556615771538181\\
599.4	0.00562001685053484\\
599.41	0.00567439096969274\\
599.42	0.00572928502448199\\
599.43	0.00578470401410215\\
599.44	0.00584065298578253\\
599.45	0.00589713703524306\\
599.46	0.00595416130715976\\
599.47	0.00601173099563462\\
599.48	0.00606985134466992\\
599.49	0.00612852764864716\\
599.5	0.00618776525281058\\
599.51	0.00624756955375529\\
599.52	0.00630794599992009\\
599.53	0.00636890009208503\\
599.54	0.00643043738387374\\
599.55	0.00649256348226047\\
599.56	0.00655528404808225\\
599.57	0.00661860479655563\\
599.58	0.0066825314977986\\
599.59	0.00674706997735748\\
599.6	0.0068122261167388\\
599.61	0.00687800585394632\\
599.62	0.00694441518402315\\
599.63	0.00701146015959914\\
599.64	0.00707914689144346\\
599.65	0.00714748154902254\\
599.66	0.00721647036106325\\
599.67	0.00728611961612159\\
599.68	0.00735643566315677\\
599.69	0.00742742491211079\\
599.7	0.00749909383449364\\
599.71	0.00757144896397401\\
599.72	0.00764449689697572\\
599.73	0.00771824429327989\\
599.74	0.00779269787663285\\
599.75	0.00786786443535983\\
599.76	0.00794375082298461\\
599.77	0.0080203639588551\\
599.78	0.00809771082877486\\
599.79	0.00817579848564071\\
599.8	0.00825463405008648\\
599.81	0.00833422471113285\\
599.82	0.00841457772684349\\
599.83	0.00849570042498746\\
599.84	0.00857760020370797\\
599.85	0.00866028453219752\\
599.86	0.00874376095137953\\
599.87	0.00882803707459654\\
599.88	0.0089131205883049\\
599.89	0.00899901925277625\\
599.9	0.00908574090280562\\
599.91	0.00917329344842635\\
599.92	0.0092616848756319\\
599.93	0.00935092324710449\\
599.94	0.00944101670295078\\
599.95	0.00953197346144464\\
599.96	0.00962380181977693\\
599.97	0.00971651015481255\\
599.98	0.0098101069238547\\
599.99	0.00990460066541651\\
600	0.01\\
};
\end{axis}
\end{tikzpicture}%

  \caption{Continuous Time}
\end{subfigure}%
\hfill%
\begin{subfigure}{.45\linewidth}
  \centering
  \setlength\figureheight{\linewidth} 
  \setlength\figurewidth{\linewidth}
  \tikzsetnextfilename{dp_colorbar/dp_dscr_z1}
  % This file was created by matlab2tikz.
%
%The latest updates can be retrieved from
%  http://www.mathworks.com/matlabcentral/fileexchange/22022-matlab2tikz-matlab2tikz
%where you can also make suggestions and rate matlab2tikz.
%
\definecolor{mycolor1}{rgb}{0.00000,1.00000,0.14286}%
\definecolor{mycolor2}{rgb}{0.00000,1.00000,0.28571}%
\definecolor{mycolor3}{rgb}{0.00000,1.00000,0.42857}%
\definecolor{mycolor4}{rgb}{0.00000,1.00000,0.57143}%
\definecolor{mycolor5}{rgb}{0.00000,1.00000,0.71429}%
\definecolor{mycolor6}{rgb}{0.00000,1.00000,0.85714}%
\definecolor{mycolor7}{rgb}{0.00000,1.00000,1.00000}%
\definecolor{mycolor8}{rgb}{0.00000,0.87500,1.00000}%
\definecolor{mycolor9}{rgb}{0.00000,0.62500,1.00000}%
\definecolor{mycolor10}{rgb}{0.12500,0.00000,1.00000}%
\definecolor{mycolor11}{rgb}{0.25000,0.00000,1.00000}%
\definecolor{mycolor12}{rgb}{0.37500,0.00000,1.00000}%
\definecolor{mycolor13}{rgb}{0.50000,0.00000,1.00000}%
\definecolor{mycolor14}{rgb}{0.62500,0.00000,1.00000}%
\definecolor{mycolor15}{rgb}{0.75000,0.00000,1.00000}%
\definecolor{mycolor16}{rgb}{0.87500,0.00000,1.00000}%
\definecolor{mycolor17}{rgb}{1.00000,0.00000,1.00000}%
\definecolor{mycolor18}{rgb}{1.00000,0.00000,0.87500}%
\definecolor{mycolor19}{rgb}{1.00000,0.00000,0.62500}%
\definecolor{mycolor20}{rgb}{0.85714,0.00000,0.00000}%
\definecolor{mycolor21}{rgb}{0.71429,0.00000,0.00000}%
%
\begin{tikzpicture}[trim axis left, trim axis right]

\begin{axis}[%
width=\figurewidth,
height=\figureheight,
at={(0\figurewidth,0\figureheight)},
scale only axis,
point meta min=0,
point meta max=1,
every outer x axis line/.append style={black},
every x tick label/.append style={font=\color{black}},
xmin=0,
xmax=600,
every outer y axis line/.append style={black},
every y tick label/.append style={font=\color{black}},
ymin=0,
ymax=0.014,
axis background/.style={fill=white},
axis x line*=bottom,
axis y line*=left,
]
\addplot [color=green,solid,forget plot]
  table[row sep=crcr]{%
1	0.0120168416619381\\
2	0.012016841040951\\
3	0.0120168404084185\\
4	0.0120168397641242\\
5	0.0120168391078476\\
6	0.0120168384393642\\
7	0.0120168377584453\\
8	0.0120168370648576\\
9	0.0120168363583636\\
10	0.0120168356387214\\
11	0.0120168349056843\\
12	0.0120168341590011\\
13	0.0120168333984157\\
14	0.0120168326236673\\
15	0.0120168318344903\\
16	0.0120168310306136\\
17	0.0120168302117615\\
18	0.0120168293776528\\
19	0.012016828528001\\
20	0.0120168276625143\\
21	0.0120168267808953\\
22	0.0120168258828409\\
23	0.0120168249680424\\
24	0.0120168240361851\\
25	0.0120168230869486\\
26	0.0120168221200061\\
27	0.0120168211350248\\
28	0.0120168201316658\\
29	0.0120168191095832\\
30	0.0120168180684252\\
31	0.0120168170078329\\
32	0.0120168159274408\\
33	0.0120168148268764\\
34	0.0120168137057601\\
35	0.0120168125637052\\
36	0.0120168114003176\\
37	0.0120168102151957\\
38	0.0120168090079304\\
39	0.0120168077781048\\
40	0.012016806525294\\
41	0.0120168052490652\\
42	0.0120168039489771\\
43	0.0120168026245804\\
44	0.012016801275417\\
45	0.0120167999010202\\
46	0.0120167985009144\\
47	0.0120167970746151\\
48	0.0120167956216284\\
49	0.0120167941414512\\
50	0.0120167926335707\\
51	0.0120167910974644\\
52	0.0120167895325999\\
53	0.0120167879384347\\
54	0.012016786314416\\
55	0.0120167846599805\\
56	0.012016782974554\\
57	0.0120167812575515\\
58	0.0120167795083771\\
59	0.0120167777264233\\
60	0.0120167759110711\\
61	0.0120167740616897\\
62	0.0120167721776364\\
63	0.0120167702582561\\
64	0.0120167683028816\\
65	0.0120167663108325\\
66	0.0120167642814158\\
67	0.0120167622139252\\
68	0.0120167601076409\\
69	0.0120167579618295\\
70	0.0120167557757436\\
71	0.0120167535486216\\
72	0.0120167512796871\\
73	0.0120167489681494\\
74	0.0120167466132024\\
75	0.0120167442140246\\
76	0.0120167417697792\\
77	0.0120167392796129\\
78	0.0120167367426567\\
79	0.0120167341580247\\
80	0.0120167315248141\\
81	0.0120167288421051\\
82	0.01201672610896\\
83	0.0120167233244237\\
84	0.0120167204875224\\
85	0.012016717597264\\
86	0.0120167146526374\\
87	0.0120167116526122\\
88	0.0120167085961383\\
89	0.0120167054821456\\
90	0.0120167023095436\\
91	0.0120166990772208\\
92	0.0120166957840448\\
93	0.0120166924288611\\
94	0.0120166890104937\\
95	0.0120166855277437\\
96	0.0120166819793895\\
97	0.012016678364186\\
98	0.0120166746808645\\
99	0.0120166709281319\\
100	0.0120166671046705\\
101	0.0120166632091374\\
102	0.0120166592401638\\
103	0.012016655196355\\
104	0.0120166510762895\\
105	0.0120166468785187\\
106	0.0120166426015661\\
107	0.0120166382439272\\
108	0.0120166338040684\\
109	0.0120166292804271\\
110	0.0120166246714106\\
111	0.0120166199753957\\
112	0.0120166151907282\\
113	0.0120166103157222\\
114	0.0120166053486595\\
115	0.012016600287789\\
116	0.0120165951313261\\
117	0.012016589877452\\
118	0.0120165845243131\\
119	0.0120165790700202\\
120	0.0120165735126478\\
121	0.0120165678502339\\
122	0.0120165620807784\\
123	0.0120165562022433\\
124	0.0120165502125512\\
125	0.012016544109585\\
126	0.0120165378911871\\
127	0.0120165315551584\\
128	0.0120165250992576\\
129	0.0120165185212005\\
130	0.0120165118186591\\
131	0.0120165049892606\\
132	0.0120164980305868\\
133	0.0120164909401731\\
134	0.0120164837155075\\
135	0.0120164763540298\\
136	0.0120164688531309\\
137	0.0120164612101511\\
138	0.0120164534223802\\
139	0.0120164454870556\\
140	0.0120164374013618\\
141	0.0120164291624291\\
142	0.0120164207673328\\
143	0.0120164122130919\\
144	0.0120164034966684\\
145	0.0120163946149655\\
146	0.0120163855648274\\
147	0.0120163763430371\\
148	0.0120163669463162\\
149	0.0120163573713232\\
150	0.0120163476146523\\
151	0.0120163376728321\\
152	0.0120163275423248\\
153	0.0120163172195243\\
154	0.0120163067007552\\
155	0.0120162959822716\\
156	0.0120162850602554\\
157	0.012016273930815\\
158	0.0120162625899842\\
159	0.0120162510337203\\
160	0.012016239257903\\
161	0.0120162272583326\\
162	0.0120162150307287\\
163	0.0120162025707284\\
164	0.0120161898738851\\
165	0.0120161769356666\\
166	0.0120161637514533\\
167	0.0120161503165369\\
168	0.0120161366261185\\
169	0.012016122675307\\
170	0.0120161084591168\\
171	0.0120160939724668\\
172	0.0120160792101779\\
173	0.0120160641669715\\
174	0.0120160488374673\\
175	0.0120160332161817\\
176	0.0120160172975255\\
177	0.0120160010758018\\
178	0.0120159845452046\\
179	0.0120159676998159\\
180	0.0120159505336038\\
181	0.0120159330404208\\
182	0.0120159152140008\\
183	0.0120158970479575\\
184	0.0120158785357816\\
185	0.012015859670839\\
186	0.0120158404463678\\
187	0.0120158208554763\\
188	0.0120158008911404\\
189	0.0120157805462009\\
190	0.0120157598133615\\
191	0.0120157386851853\\
192	0.0120157171540931\\
193	0.0120156952123596\\
194	0.0120156728521112\\
195	0.0120156500653218\\
196	0.0120156268438103\\
197	0.0120156031792407\\
198	0.0120155790631165\\
199	0.0120155544867777\\
200	0.0120155294413975\\
201	0.0120155039179793\\
202	0.0120154779073529\\
203	0.0120154514001719\\
204	0.0120154243869094\\
205	0.0120153968578554\\
206	0.0120153688031126\\
207	0.0120153402125928\\
208	0.0120153110760136\\
209	0.0120152813828943\\
210	0.012015251122552\\
211	0.0120152202840978\\
212	0.0120151888564325\\
213	0.012015156828243\\
214	0.0120151241879974\\
215	0.0120150909239412\\
216	0.0120150570240928\\
217	0.0120150224762386\\
218	0.0120149872679291\\
219	0.0120149513864735\\
220	0.0120149148189352\\
221	0.0120148775521267\\
222	0.0120148395726048\\
223	0.0120148008666651\\
224	0.0120147614203367\\
225	0.0120147212193768\\
226	0.0120146802492652\\
227	0.0120146384951982\\
228	0.0120145959420828\\
229	0.0120145525745308\\
230	0.012014508376852\\
231	0.0120144633330484\\
232	0.0120144174268068\\
233	0.0120143706414924\\
234	0.0120143229601414\\
235	0.0120142743654538\\
236	0.0120142248397856\\
237	0.0120141743651412\\
238	0.0120141229231651\\
239	0.012014070495133\\
240	0.012014017061944\\
241	0.0120139626041104\\
242	0.0120139071017492\\
243	0.0120138505345714\\
244	0.0120137928818725\\
245	0.0120137341225212\\
246	0.012013674234949\\
247	0.0120136131971381\\
248	0.0120135509866095\\
249	0.0120134875804106\\
250	0.0120134229551023\\
251	0.012013357086745\\
252	0.0120132899508847\\
253	0.0120132215225385\\
254	0.012013151776179\\
255	0.0120130806857185\\
256	0.012013008224493\\
257	0.0120129343652448\\
258	0.0120128590801049\\
259	0.0120127823405759\\
260	0.0120127041175125\\
261	0.0120126243811036\\
262	0.012012543100853\\
263	0.0120124602455603\\
264	0.012012375783302\\
265	0.0120122896814129\\
266	0.0120122019064676\\
267	0.0120121124242639\\
268	0.0120120211998068\\
269	0.0120119281972951\\
270	0.0120118333801124\\
271	0.0120117367108245\\
272	0.0120116381511926\\
273	0.0120115376622204\\
274	0.0120114352042807\\
275	0.0120113307374105\\
276	0.0120112242218546\\
277	0.0120111156183025\\
278	0.0120110048827092\\
279	0.0120108919663135\\
280	0.012010776825855\\
281	0.0120106594172084\\
282	0.0120105396953649\\
283	0.0120104176144154\\
284	0.0120102931275316\\
285	0.0120101661869474\\
286	0.0120100367439404\\
287	0.012009904748812\\
288	0.0120097701508679\\
289	0.0120096328983978\\
290	0.012009492938655\\
291	0.0120093502178347\\
292	0.0120092046810536\\
293	0.012009056272327\\
294	0.0120089049345467\\
295	0.0120087506094586\\
296	0.0120085932376387\\
297	0.0120084327584698\\
298	0.0120082691101167\\
299	0.0120081022295019\\
300	0.01200793205228\\
301	0.0120077585128114\\
302	0.0120075815441369\\
303	0.0120074010779494\\
304	0.0120072170445675\\
305	0.0120070293729065\\
306	0.0120068379904501\\
307	0.0120066428232209\\
308	0.0120064437957503\\
309	0.012006240831048\\
310	0.0120060338505704\\
311	0.0120058227741891\\
312	0.0120056075201576\\
313	0.0120053880050783\\
314	0.0120051641438682\\
315	0.012004935849724\\
316	0.0120047030340867\\
317	0.0120044656066045\\
318	0.0120042234750963\\
319	0.012003976545513\\
320	0.0120037247218989\\
321	0.0120034679063515\\
322	0.0120032059989814\\
323	0.0120029388978699\\
324	0.0120026664990267\\
325	0.0120023886963465\\
326	0.0120021053815641\\
327	0.0120018164442088\\
328	0.0120015217715578\\
329	0.0120012212485885\\
330	0.0120009147579291\\
331	0.0120006021798094\\
332	0.0120002833920086\\
333	0.0119999582698036\\
334	0.0119996266859145\\
335	0.0119992885104505\\
336	0.0119989436108528\\
337	0.0119985918518372\\
338	0.0119982330953351\\
339	0.0119978672004328\\
340	0.0119974940233094\\
341	0.0119971134171738\\
342	0.011996725232199\\
343	0.0119963293154559\\
344	0.0119959255108448\\
345	0.0119955136590257\\
346	0.0119950935973465\\
347	0.0119946651597696\\
348	0.0119942281767972\\
349	0.0119937824753938\\
350	0.0119933278789079\\
351	0.0119928642069913\\
352	0.0119923912755166\\
353	0.0119919088964929\\
354	0.0119914168779795\\
355	0.0119909150239977\\
356	0.0119904031344405\\
357	0.0119898810049811\\
358	0.0119893484269779\\
359	0.0119888051873798\\
360	0.0119882510686273\\
361	0.0119876858485533\\
362	0.0119871093002814\\
363	0.0119865211921224\\
364	0.0119859212874694\\
365	0.0119853093446905\\
366	0.0119846851170202\\
367	0.0119840483524494\\
368	0.0119833987936132\\
369	0.0119827361776774\\
370	0.011982060236224\\
371	0.0119813706951342\\
372	0.0119806672744704\\
373	0.0119799496883571\\
374	0.0119792176448592\\
375	0.0119784708458597\\
376	0.0119777089869351\\
377	0.0119769317572299\\
378	0.0119761388393308\\
379	0.0119753299091428\\
380	0.0119745046357735\\
381	0.0119736626814303\\
382	0.0119728037013296\\
383	0.0119719273435596\\
384	0.0119710332486776\\
385	0.0119701210486854\\
386	0.0119691903672383\\
387	0.0119682408210636\\
388	0.011967272018416\\
389	0.0119662835588548\\
390	0.0119652750330143\\
391	0.0119642460223643\\
392	0.0119631960989623\\
393	0.011962124825196\\
394	0.0119610317535156\\
395	0.0119599164261553\\
396	0.0119587783748436\\
397	0.0119576171205018\\
398	0.0119564321729289\\
399	0.0119552230304736\\
400	0.0119539891796917\\
401	0.0119527300949877\\
402	0.0119514452382403\\
403	0.0119501340584101\\
404	0.0119487959911288\\
405	0.011947430458268\\
406	0.0119460368674869\\
407	0.0119446146117566\\
408	0.0119431630688604\\
409	0.0119416816008673\\
410	0.0119401695535767\\
411	0.0119386262559338\\
412	0.0119370510194114\\
413	0.0119354431373573\\
414	0.0119338018843034\\
415	0.0119321265152356\\
416	0.011930416264819\\
417	0.011928670346578\\
418	0.0119268879520266\\
419	0.0119250682497446\\
420	0.0119232103843938\\
421	0.0119213134756592\\
422	0.0119193766171024\\
423	0.0119173988749376\\
424	0.0119153792868497\\
425	0.0119133168607032\\
426	0.0119112105730325\\
427	0.0119090593675221\\
428	0.0119068621533832\\
429	0.0119046178036216\\
430	0.0119023251531855\\
431	0.0118999829969861\\
432	0.0118975900877788\\
433	0.0118951451338961\\
434	0.011892646796818\\
435	0.0118900936885688\\
436	0.0118874843689248\\
437	0.0118848173424181\\
438	0.0118820910551198\\
439	0.0118793038911842\\
440	0.0118764541691341\\
441	0.0118735401378653\\
442	0.0118705599723482\\
443	0.0118675117689975\\
444	0.0118643935406848\\
445	0.011861203211361\\
446	0.0118579386102535\\
447	0.0118545974656052\\
448	0.0118511773979167\\
449	0.0118476759126506\\
450	0.0118440903923217\\
451	0.0118404180879478\\
452	0.0118366561097923\\
453	0.0118328014173301\\
454	0.0118288508083663\\
455	0.0118248009072259\\
456	0.0118206481519238\\
457	0.0118163887802169\\
458	0.0118120188144275\\
459	0.0118075340449137\\
460	0.0118029300120497\\
461	0.0117982019865639\\
462	0.0117933449480621\\
463	0.0117883535615499\\
464	0.011783222151751\\
465	0.011777944675019\\
466	0.0117725146886901\\
467	0.0117669253179064\\
468	0.0117611692205262\\
469	0.0117552385524497\\
470	0.0117491249405385\\
471	0.0117428194837972\\
472	0.0117363128403338\\
473	0.0117295955560796\\
474	0.0117226590459481\\
475	0.0117154944841619\\
476	0.0117081007677849\\
477	0.0117004399492539\\
478	0.0116924911639723\\
479	0.0116842371389706\\
480	0.0116756592807096\\
481	0.011666737420585\\
482	0.011657449643096\\
483	0.0116477720973909\\
484	0.0116376787859447\\
485	0.011627141326173\\
486	0.0116161286806223\\
487	0.0116046068509186\\
488	0.0115925385297241\\
489	0.0115798827037196\\
490	0.0115665941987047\\
491	0.0115526231543386\\
492	0.011537914408131\\
493	0.0115224067484298\\
494	0.0115060319430138\\
495	0.0114887133045863\\
496	0.0114703631593111\\
497	0.0114508775402919\\
498	0.0114301238292634\\
499	0.011407911351236\\
500	0.0113799225560012\\
501	0.0113348917331087\\
502	0.0112878339916798\\
503	0.0112381899578923\\
504	0.0111211579001085\\
505	0.0109822138907321\\
506	0.0108406034240144\\
507	0.0107131600808254\\
508	0.0106794477019033\\
509	0.0106506709685816\\
510	0.0106281808630873\\
511	0.0106106296017312\\
512	0.0105936254972074\\
513	0.01057693397818\\
514	0.0105601789387908\\
515	0.0105431859749238\\
516	0.0105258935013059\\
517	0.0105082776463844\\
518	0.0104903245000416\\
519	0.010471970931807\\
520	0.010453204501249\\
521	0.0104340146738591\\
522	0.0104143910900067\\
523	0.0103943234840548\\
524	0.0103738016024305\\
525	0.0103528150266084\\
526	0.0103313531309081\\
527	0.0103094050049212\\
528	0.0102869592772049\\
529	0.0102640037208883\\
530	0.0102404079739401\\
531	0.0102159793504059\\
532	0.0101911034024066\\
533	0.0101638539240202\\
534	0.0101367367488312\\
535	0.0101113248467772\\
536	0.0100866450965714\\
537	0.0100615241813055\\
538	0.010035862866537\\
539	0.0100096435719068\\
540	0.00998285235584197\\
541	0.00995547442818611\\
542	0.00992749540687192\\
543	0.0098989013482874\\
544	0.00986967913586614\\
545	0.00983981882940722\\
546	0.00980932051751061\\
547	0.00977821424882196\\
548	0.00974648723673855\\
549	0.00971403640883855\\
550	0.00968084144507715\\
551	0.00964606093351074\\
552	0.00960798190194539\\
553	0.00957049440978274\\
554	0.00953772512363284\\
555	0.00950433882860447\\
556	0.00947028681113497\\
557	0.00943555341890868\\
558	0.00940012264022913\\
559	0.00936397796935504\\
560	0.009327102375457\\
561	0.00928947824470587\\
562	0.00925108724655004\\
563	0.00921191000844813\\
564	0.00917192533096021\\
565	0.00893409944319607\\
566	0.00851312471020558\\
567	0.00845082836229412\\
568	0.00838749699508529\\
569	0.00832310828726401\\
570	0.00825763909266039\\
571	0.00819106539479088\\
572	0.00812336225919132\\
573	0.00805450378356905\\
574	0.00798446304588267\\
575	0.00791321205051155\\
576	0.0078407216727229\\
577	0.00776696160163765\\
578	0.00769190028172314\\
579	0.00761550485219266\\
580	0.00753774108175192\\
581	0.00745857329076528\\
582	0.007377964238543\\
583	0.00729587491554804\\
584	0.00721226408110737\\
585	0.00712708712854453\\
586	0.0070402931868614\\
587	0.00695181762108359\\
588	0.00686156256306144\\
589	0.00676934637110727\\
590	0.00667477256501157\\
591	0.00657713355197375\\
592	0.00647481662499992\\
593	0.00636364764633836\\
594	0.00623271716130669\\
595	0.00605340994148281\\
596	0.00575058001197164\\
597	0.00512683753504546\\
598	0.00366374385960312\\
599	0\\
600	0\\
};
\addplot [color=mycolor1,solid,forget plot]
  table[row sep=crcr]{%
1	0.012017858178407\\
2	0.0120178574763449\\
3	0.0120178567613835\\
4	0.012017856033284\\
5	0.0120178552918031\\
6	0.012017854536693\\
7	0.0120178537677013\\
8	0.0120178529845709\\
9	0.0120178521870398\\
10	0.0120178513748412\\
11	0.0120178505477033\\
12	0.0120178497053491\\
13	0.0120178488474967\\
14	0.0120178479738587\\
15	0.0120178470841425\\
16	0.0120178461780497\\
17	0.0120178452552768\\
18	0.0120178443155144\\
19	0.0120178433584471\\
20	0.012017842383754\\
21	0.0120178413911079\\
22	0.0120178403801756\\
23	0.0120178393506177\\
24	0.0120178383020884\\
25	0.0120178372342354\\
26	0.0120178361467\\
27	0.0120178350391165\\
28	0.0120178339111125\\
29	0.0120178327623086\\
30	0.0120178315923185\\
31	0.0120178304007482\\
32	0.0120178291871969\\
33	0.0120178279512558\\
34	0.0120178266925088\\
35	0.0120178254105318\\
36	0.0120178241048927\\
37	0.0120178227751515\\
38	0.0120178214208599\\
39	0.0120178200415611\\
40	0.0120178186367898\\
41	0.0120178172060721\\
42	0.0120178157489249\\
43	0.0120178142648563\\
44	0.0120178127533651\\
45	0.0120178112139408\\
46	0.0120178096460631\\
47	0.0120178080492022\\
48	0.0120178064228181\\
49	0.0120178047663608\\
50	0.01201780307927\\
51	0.0120178013609748\\
52	0.0120177996108937\\
53	0.0120177978284342\\
54	0.0120177960129926\\
55	0.012017794163954\\
56	0.0120177922806919\\
57	0.012017790362568\\
58	0.0120177884089321\\
59	0.0120177864191216\\
60	0.0120177843924616\\
61	0.0120177823282645\\
62	0.0120177802258298\\
63	0.0120177780844436\\
64	0.012017775903379\\
65	0.012017773681895\\
66	0.0120177714192369\\
67	0.0120177691146358\\
68	0.0120177667673082\\
69	0.012017764376456\\
70	0.0120177619412661\\
71	0.01201775946091\\
72	0.0120177569345436\\
73	0.0120177543613069\\
74	0.0120177517403239\\
75	0.0120177490707019\\
76	0.0120177463515315\\
77	0.0120177435818861\\
78	0.0120177407608217\\
79	0.0120177378873766\\
80	0.0120177349605707\\
81	0.0120177319794059\\
82	0.0120177289428647\\
83	0.0120177258499111\\
84	0.012017722699489\\
85	0.0120177194905227\\
86	0.0120177162219162\\
87	0.0120177128925527\\
88	0.0120177095012946\\
89	0.0120177060469826\\
90	0.0120177025284356\\
91	0.0120176989444504\\
92	0.0120176952938008\\
93	0.0120176915752379\\
94	0.0120176877874887\\
95	0.0120176839292568\\
96	0.0120176799992207\\
97	0.0120176759960346\\
98	0.0120176719183269\\
99	0.0120176677647001\\
100	0.0120176635337307\\
101	0.0120176592239678\\
102	0.0120176548339337\\
103	0.0120176503621223\\
104	0.0120176458069994\\
105	0.0120176411670017\\
106	0.0120176364405365\\
107	0.0120176316259808\\
108	0.0120176267216811\\
109	0.0120176217259528\\
110	0.0120176166370794\\
111	0.0120176114533118\\
112	0.0120176061728681\\
113	0.0120176007939327\\
114	0.0120175953146557\\
115	0.0120175897331522\\
116	0.0120175840475018\\
117	0.0120175782557476\\
118	0.0120175723558961\\
119	0.0120175663459158\\
120	0.0120175602237371\\
121	0.012017553987251\\
122	0.0120175476343089\\
123	0.0120175411627216\\
124	0.0120175345702583\\
125	0.0120175278546464\\
126	0.01201752101357\\
127	0.0120175140446697\\
128	0.0120175069455413\\
129	0.0120174997137356\\
130	0.0120174923467565\\
131	0.0120174848420612\\
132	0.0120174771970588\\
133	0.0120174694091092\\
134	0.0120174614755226\\
135	0.0120174533935583\\
136	0.012017445160424\\
137	0.0120174367732743\\
138	0.0120174282292103\\
139	0.0120174195252782\\
140	0.0120174106584685\\
141	0.0120174016257147\\
142	0.0120173924238924\\
143	0.0120173830498183\\
144	0.0120173735002487\\
145	0.0120173637718789\\
146	0.0120173538613416\\
147	0.012017343765206\\
148	0.0120173334799762\\
149	0.0120173230020907\\
150	0.0120173123279205\\
151	0.012017301453768\\
152	0.0120172903758658\\
153	0.0120172790903755\\
154	0.0120172675933862\\
155	0.012017255880913\\
156	0.0120172439488959\\
157	0.0120172317931981\\
158	0.0120172194096051\\
159	0.0120172067938225\\
160	0.0120171939414749\\
161	0.0120171808481045\\
162	0.0120171675091693\\
163	0.0120171539200418\\
164	0.0120171400760071\\
165	0.0120171259722613\\
166	0.0120171116039102\\
167	0.0120170969659673\\
168	0.012017082053352\\
169	0.0120170668608882\\
170	0.0120170513833022\\
171	0.0120170356152211\\
172	0.0120170195511709\\
173	0.0120170031855745\\
174	0.0120169865127502\\
175	0.0120169695269092\\
176	0.0120169522221543\\
177	0.0120169345924772\\
178	0.0120169166317571\\
179	0.0120168983337581\\
180	0.0120168796921278\\
181	0.0120168607003946\\
182	0.0120168413519656\\
183	0.0120168216401248\\
184	0.0120168015580306\\
185	0.0120167810987138\\
186	0.012016760255075\\
187	0.0120167390198826\\
188	0.0120167173857706\\
189	0.0120166953452363\\
190	0.0120166728906383\\
191	0.0120166500141946\\
192	0.0120166267079812\\
193	0.0120166029639306\\
194	0.012016578773829\\
195	0.012016554129308\\
196	0.0120165290218211\\
197	0.0120165034426105\\
198	0.0120164773828393\\
199	0.0120164508335118\\
200	0.0120164237854672\\
201	0.0120163962293771\\
202	0.0120163681557422\\
203	0.0120163395548901\\
204	0.0120163104169717\\
205	0.0120162807319585\\
206	0.01201625048964\\
207	0.01201621967962\\
208	0.0120161882913144\\
209	0.0120161563139471\\
210	0.0120161237365481\\
211	0.0120160905479491\\
212	0.0120160567367816\\
213	0.0120160222914728\\
214	0.0120159872002426\\
215	0.0120159514511009\\
216	0.0120159150318435\\
217	0.0120158779300494\\
218	0.0120158401330775\\
219	0.0120158016280629\\
220	0.0120157624019139\\
221	0.0120157224413085\\
222	0.012015681732691\\
223	0.0120156402622687\\
224	0.0120155980160086\\
225	0.0120155549796336\\
226	0.0120155111386194\\
227	0.0120154664781908\\
228	0.0120154209833188\\
229	0.0120153746387161\\
230	0.0120153274288347\\
231	0.0120152793378615\\
232	0.0120152303497153\\
233	0.0120151804480431\\
234	0.0120151296162164\\
235	0.0120150778373275\\
236	0.0120150250941859\\
237	0.0120149713693149\\
238	0.012014916644947\\
239	0.0120148609030205\\
240	0.0120148041251754\\
241	0.0120147462927493\\
242	0.0120146873867728\\
243	0.0120146273879651\\
244	0.0120145662767293\\
245	0.0120145040331471\\
246	0.0120144406369735\\
247	0.0120143760676312\\
248	0.0120143103042038\\
249	0.0120142433254288\\
250	0.0120141751096906\\
251	0.0120141056350112\\
252	0.0120140348790412\\
253	0.0120139628190491\\
254	0.0120138894319089\\
255	0.0120138146940869\\
256	0.0120137385816259\\
257	0.0120136610701276\\
258	0.0120135821347321\\
259	0.0120135017500945\\
260	0.0120134198903585\\
261	0.0120133365291248\\
262	0.0120132516394155\\
263	0.012013165193633\\
264	0.0120130771635115\\
265	0.012012987520062\\
266	0.0120128962335073\\
267	0.0120128032732078\\
268	0.0120127086075736\\
269	0.0120126122039643\\
270	0.0120125140285728\\
271	0.0120124140462938\\
272	0.0120123122205904\\
273	0.0120122085133997\\
274	0.0120121028852558\\
275	0.0120119952962719\\
276	0.0120118857103786\\
277	0.0120117741118422\\
278	0.0120116605686941\\
279	0.0120115450443472\\
280	0.0120114272814002\\
281	0.0120113072363395\\
282	0.0120111848648041\\
283	0.0120110601215693\\
284	0.0120109329605297\\
285	0.0120108033346824\\
286	0.0120106711961095\\
287	0.0120105364959599\\
288	0.0120103991844317\\
289	0.0120102592107534\\
290	0.0120101165231651\\
291	0.0120099710688996\\
292	0.0120098227941623\\
293	0.0120096716441117\\
294	0.012009517562839\\
295	0.012009360493347\\
296	0.0120092003775294\\
297	0.0120090371561492\\
298	0.0120088707688166\\
299	0.0120087011539667\\
300	0.0120085282488365\\
301	0.012008351989442\\
302	0.0120081723105543\\
303	0.0120079891456753\\
304	0.0120078024270133\\
305	0.0120076120854576\\
306	0.0120074180505531\\
307	0.0120072202504741\\
308	0.0120070186119977\\
309	0.0120068130604765\\
310	0.0120066035198109\\
311	0.012006389912421\\
312	0.0120061721592176\\
313	0.0120059501795729\\
314	0.0120057238912902\\
315	0.0120054932105738\\
316	0.0120052580519972\\
317	0.0120050183284715\\
318	0.0120047739512128\\
319	0.0120045248297087\\
320	0.0120042708716848\\
321	0.0120040119830695\\
322	0.0120037480679588\\
323	0.0120034790285802\\
324	0.0120032047652555\\
325	0.0120029251763628\\
326	0.0120026401582981\\
327	0.0120023496054355\\
328	0.0120020534100868\\
329	0.0120017514624597\\
330	0.0120014436506155\\
331	0.0120011298604256\\
332	0.0120008099755267\\
333	0.012000483877275\\
334	0.0120001514446992\\
335	0.0119998125544523\\
336	0.0119994670807622\\
337	0.0119991148953802\\
338	0.0119987558675294\\
339	0.0119983898638503\\
340	0.0119980167483451\\
341	0.0119976363823211\\
342	0.0119972486243314\\
343	0.0119968533301138\\
344	0.0119964503525286\\
345	0.0119960395414932\\
346	0.0119956207439149\\
347	0.011995193803622\\
348	0.0119947585612914\\
349	0.0119943148543741\\
350	0.0119938625170185\\
351	0.0119934013799895\\
352	0.0119929312705857\\
353	0.0119924520125531\\
354	0.0119919634259951\\
355	0.0119914653272795\\
356	0.0119909575289408\\
357	0.0119904398395802\\
358	0.0119899120637601\\
359	0.011989374001895\\
360	0.0119888254501385\\
361	0.0119882662002649\\
362	0.0119876960395472\\
363	0.0119871147506296\\
364	0.0119865221113965\\
365	0.0119859178948358\\
366	0.0119853018688982\\
367	0.0119846737963526\\
368	0.0119840334346364\\
369	0.011983380535703\\
370	0.0119827148458664\\
371	0.0119820361056425\\
372	0.0119813440495901\\
373	0.0119806384061511\\
374	0.0119799188974922\\
375	0.0119791852393511\\
376	0.0119784371408883\\
377	0.0119776743045518\\
378	0.0119768964259612\\
379	0.0119761031938335\\
380	0.0119752942899961\\
381	0.0119744693896065\\
382	0.0119736281618532\\
383	0.0119727702716925\\
384	0.0119718953831306\\
385	0.0119710031607734\\
386	0.0119700932390592\\
387	0.0119691652183333\\
388	0.0119682187313951\\
389	0.0119672534030775\\
390	0.0119662688500361\\
391	0.0119652646805306\\
392	0.0119642404941992\\
393	0.0119631958818241\\
394	0.0119621304250888\\
395	0.0119610436963249\\
396	0.0119599352582509\\
397	0.0119588046637009\\
398	0.0119576514553435\\
399	0.0119564751653896\\
400	0.0119552753152894\\
401	0.011954051415417\\
402	0.0119528029647425\\
403	0.0119515294504905\\
404	0.0119502303477852\\
405	0.0119489051192814\\
406	0.0119475532147793\\
407	0.0119461740708232\\
408	0.011944767110283\\
409	0.0119433317419166\\
410	0.0119418673599135\\
411	0.0119403733434165\\
412	0.0119388490560209\\
413	0.0119372938452492\\
414	0.0119357070420005\\
415	0.0119340879599738\\
416	0.0119324358950677\\
417	0.0119307501247405\\
418	0.0119290299073486\\
419	0.0119272744814688\\
420	0.0119254830652124\\
421	0.0119236548555195\\
422	0.0119217890273048\\
423	0.0119198847319841\\
424	0.0119179410947082\\
425	0.0119159572148796\\
426	0.0119139321671055\\
427	0.0119118649977743\\
428	0.0119097547237217\\
429	0.0119076003308079\\
430	0.0119054007724006\\
431	0.011903154967755\\
432	0.0119008618002837\\
433	0.0118985201157074\\
434	0.0118961287200778\\
435	0.0118936863776612\\
436	0.0118911918086727\\
437	0.0118886436868493\\
438	0.0118860406368477\\
439	0.0118833812314542\\
440	0.0118806639885902\\
441	0.0118778873680984\\
442	0.0118750497682902\\
443	0.0118721495222375\\
444	0.0118691848937853\\
445	0.0118661540732619\\
446	0.0118630551728496\\
447	0.0118598862215725\\
448	0.0118566451598622\\
449	0.0118533298337747\\
450	0.0118499379890396\\
451	0.011846467264141\\
452	0.0118429151830773\\
453	0.0118392791475423\\
454	0.0118355564284757\\
455	0.0118317441569271\\
456	0.0118278393141703\\
457	0.0118238387209992\\
458	0.0118197390261304\\
459	0.0118155366936276\\
460	0.0118112279892551\\
461	0.0118068089656577\\
462	0.0118022754462527\\
463	0.0117976230077073\\
464	0.011792846960863\\
465	0.0117879423299594\\
466	0.0117829038300097\\
467	0.0117777258422164\\
468	0.0117724023874159\\
469	0.0117669270978056\\
470	0.0117612931880113\\
471	0.0117554934284298\\
472	0.0117495201282214\\
473	0.0117433651452857\\
474	0.0117370199610621\\
475	0.011730476079478\\
476	0.0117237265715313\\
477	0.0117167590747527\\
478	0.0117095701305726\\
479	0.0117021427379843\\
480	0.0116944459668972\\
481	0.0116864641480356\\
482	0.0116781805388822\\
483	0.0116695770703071\\
484	0.0116606341770892\\
485	0.0116513306364963\\
486	0.0116416433940128\\
487	0.0116315473667719\\
488	0.0116210152216209\\
489	0.0116100171235214\\
490	0.0115985204492672\\
491	0.0115864894600743\\
492	0.0115738849241734\\
493	0.0115606636758725\\
494	0.011546778088046\\
495	0.0115321754158476\\
496	0.0115167969349481\\
497	0.0115005767553568\\
498	0.0114834402288102\\
499	0.0114653023812198\\
500	0.01144606909077\\
501	0.0114256296401743\\
502	0.0114038072068435\\
503	0.0113803357780602\\
504	0.0113408015744358\\
505	0.0112948584770326\\
506	0.0112461743411741\\
507	0.0111816724789651\\
508	0.0110411234212246\\
509	0.0108981549051104\\
510	0.0107529165082137\\
511	0.0106566838649621\\
512	0.0106223392082801\\
513	0.0105932999974131\\
514	0.0105709305585536\\
515	0.0105516640313174\\
516	0.0105334593702064\\
517	0.010515535853531\\
518	0.0104974977882872\\
519	0.0104792075890283\\
520	0.0104605664268592\\
521	0.0104415263594337\\
522	0.0104220693344566\\
523	0.0104021790567106\\
524	0.0103818410645504\\
525	0.0103610436927962\\
526	0.0103397754415959\\
527	0.0103180248780044\\
528	0.0102957805435841\\
529	0.0102730307843135\\
530	0.0102497636816225\\
531	0.010225965968753\\
532	0.0102016152128647\\
533	0.0101762803903668\\
534	0.0101504779428705\\
535	0.010122974133396\\
536	0.010094618998852\\
537	0.0100668160198629\\
538	0.0100411100980949\\
539	0.0100149580061252\\
540	0.00998828200125511\\
541	0.00996102821975749\\
542	0.00993317705082673\\
543	0.00990471400124363\\
544	0.00987562528901442\\
545	0.00984589955554137\\
546	0.00981553271219468\\
547	0.00978454227666683\\
548	0.00975296444485807\\
549	0.00972066649117154\\
550	0.00968762890003407\\
551	0.00965383086217386\\
552	0.0096186803283981\\
553	0.00958132318008507\\
554	0.00954096285964986\\
555	0.00950678442230384\\
556	0.00947277922344007\\
557	0.00943809556783066\\
558	0.00940271546732934\\
559	0.00936662234550874\\
560	0.00932979912718505\\
561	0.00929222817532882\\
562	0.00925389122203143\\
563	0.00921476922165203\\
564	0.00917484201313446\\
565	0.00911444624945732\\
566	0.00879955097520018\\
567	0.00845082836277874\\
568	0.00838749699510775\\
569	0.00832310828727373\\
570	0.00825763909266504\\
571	0.00819106539479303\\
572	0.00812336225919221\\
573	0.00805450378356939\\
574	0.00798446304588278\\
575	0.00791321205051157\\
576	0.00784072167272291\\
577	0.00776696160163765\\
578	0.00769190028172313\\
579	0.00761550485219263\\
580	0.0075377410817519\\
581	0.00745857329076528\\
582	0.007377964238543\\
583	0.00729587491554803\\
584	0.00721226408110737\\
585	0.00712708712854452\\
586	0.0070402931868614\\
587	0.00695181762108361\\
588	0.00686156256306144\\
589	0.00676934637110728\\
590	0.00667477256501156\\
591	0.00657713355197374\\
592	0.00647481662499992\\
593	0.00636364764633836\\
594	0.0062327171613067\\
595	0.00605340994148282\\
596	0.00575058001197165\\
597	0.00512683753504546\\
598	0.00366374385960312\\
599	0\\
600	0\\
};
\addplot [color=mycolor2,solid,forget plot]
  table[row sep=crcr]{%
1	0.0120212943025742\\
2	0.0120212933501643\\
3	0.0120212923806444\\
4	0.0120212913937051\\
5	0.0120212903890313\\
6	0.0120212893663023\\
7	0.0120212883251915\\
8	0.0120212872653662\\
9	0.0120212861864879\\
10	0.0120212850882118\\
11	0.0120212839701868\\
12	0.0120212828320555\\
13	0.0120212816734538\\
14	0.0120212804940111\\
15	0.0120212792933501\\
16	0.0120212780710864\\
17	0.0120212768268288\\
18	0.012021275560179\\
19	0.0120212742707312\\
20	0.0120212729580723\\
21	0.0120212716217818\\
22	0.0120212702614313\\
23	0.0120212688765848\\
24	0.0120212674667982\\
25	0.0120212660316192\\
26	0.0120212645705876\\
27	0.0120212630832343\\
28	0.0120212615690821\\
29	0.0120212600276449\\
30	0.0120212584584275\\
31	0.0120212568609261\\
32	0.0120212552346272\\
33	0.0120212535790083\\
34	0.0120212518935373\\
35	0.0120212501776721\\
36	0.012021248430861\\
37	0.012021246652542\\
38	0.012021244842143\\
39	0.0120212429990813\\
40	0.0120212411227636\\
41	0.0120212392125858\\
42	0.0120212372679326\\
43	0.0120212352881775\\
44	0.0120212332726826\\
45	0.0120212312207983\\
46	0.0120212291318631\\
47	0.0120212270052033\\
48	0.0120212248401331\\
49	0.0120212226359538\\
50	0.0120212203919543\\
51	0.01202121810741\\
52	0.0120212157815835\\
53	0.0120212134137237\\
54	0.0120212110030655\\
55	0.0120212085488302\\
56	0.0120212060502246\\
57	0.0120212035064409\\
58	0.0120212009166567\\
59	0.0120211982800342\\
60	0.0120211955957206\\
61	0.0120211928628473\\
62	0.0120211900805297\\
63	0.012021187247867\\
64	0.012021184363942\\
65	0.0120211814278205\\
66	0.0120211784385513\\
67	0.0120211753951657\\
68	0.0120211722966772\\
69	0.0120211691420811\\
70	0.0120211659303544\\
71	0.0120211626604553\\
72	0.0120211593313229\\
73	0.0120211559418766\\
74	0.0120211524910163\\
75	0.0120211489776213\\
76	0.0120211454005507\\
77	0.0120211417586423\\
78	0.0120211380507128\\
79	0.012021134275557\\
80	0.0120211304319477\\
81	0.0120211265186349\\
82	0.012021122534346\\
83	0.0120211184777846\\
84	0.0120211143476307\\
85	0.01202111014254\\
86	0.0120211058611434\\
87	0.0120211015020468\\
88	0.0120210970638303\\
89	0.0120210925450479\\
90	0.0120210879442271\\
91	0.0120210832598681\\
92	0.0120210784904439\\
93	0.0120210736343989\\
94	0.0120210686901493\\
95	0.012021063656082\\
96	0.0120210585305542\\
97	0.0120210533118928\\
98	0.0120210479983941\\
99	0.0120210425883228\\
100	0.012021037079912\\
101	0.0120210314713618\\
102	0.0120210257608396\\
103	0.0120210199464789\\
104	0.0120210140263788\\
105	0.0120210079986035\\
106	0.0120210018611814\\
107	0.0120209956121047\\
108	0.0120209892493287\\
109	0.0120209827707709\\
110	0.0120209761743106\\
111	0.0120209694577879\\
112	0.0120209626190033\\
113	0.0120209556557167\\
114	0.0120209485656467\\
115	0.01202094134647\\
116	0.0120209339958204\\
117	0.0120209265112883\\
118	0.0120209188904195\\
119	0.0120209111307147\\
120	0.0120209032296287\\
121	0.0120208951845693\\
122	0.0120208869928967\\
123	0.0120208786519222\\
124	0.012020870158908\\
125	0.0120208615110656\\
126	0.0120208527055553\\
127	0.012020843739485\\
128	0.0120208346099095\\
129	0.0120208253138293\\
130	0.0120208158481898\\
131	0.0120208062098801\\
132	0.012020796395732\\
133	0.012020786402519\\
134	0.0120207762269554\\
135	0.0120207658656948\\
136	0.0120207553153295\\
137	0.012020744572389\\
138	0.0120207336333388\\
139	0.0120207224945797\\
140	0.0120207111524461\\
141	0.0120206996032051\\
142	0.0120206878430552\\
143	0.0120206758681248\\
144	0.0120206636744714\\
145	0.01202065125808\\
146	0.0120206386148617\\
147	0.0120206257406526\\
148	0.0120206126312123\\
149	0.0120205992822223\\
150	0.0120205856892852\\
151	0.0120205718479226\\
152	0.0120205577535739\\
153	0.0120205434015948\\
154	0.0120205287872558\\
155	0.0120205139057405\\
156	0.0120204987521444\\
157	0.0120204833214727\\
158	0.0120204676086393\\
159	0.0120204516084647\\
160	0.0120204353156743\\
161	0.0120204187248969\\
162	0.0120204018306631\\
163	0.0120203846274028\\
164	0.0120203671094442\\
165	0.0120203492710114\\
166	0.0120203311062229\\
167	0.0120203126090892\\
168	0.0120202937735112\\
169	0.0120202745932783\\
170	0.0120202550620659\\
171	0.0120202351734337\\
172	0.0120202149208237\\
173	0.0120201942975576\\
174	0.0120201732968349\\
175	0.0120201519117307\\
176	0.0120201301351935\\
177	0.0120201079600423\\
178	0.0120200853789651\\
179	0.0120200623845158\\
180	0.0120200389691121\\
181	0.0120200151250329\\
182	0.0120199908444157\\
183	0.0120199661192541\\
184	0.0120199409413948\\
185	0.0120199153025354\\
186	0.0120198891942214\\
187	0.0120198626078434\\
188	0.012019835534635\\
189	0.0120198079656703\\
190	0.0120197798918636\\
191	0.0120197513039714\\
192	0.0120197221926004\\
193	0.0120196925482286\\
194	0.0120196623612438\\
195	0.012019631621997\\
196	0.0120196003207862\\
197	0.0120195684473535\\
198	0.0120195359881855\\
199	0.0120195029324462\\
200	0.0120194692692369\\
201	0.0120194349874642\\
202	0.0120194000758374\\
203	0.0120193645228647\\
204	0.0120193283168502\\
205	0.0120192914458908\\
206	0.0120192538978724\\
207	0.0120192156604671\\
208	0.0120191767211296\\
209	0.0120191370670938\\
210	0.0120190966853692\\
211	0.0120190555627381\\
212	0.0120190136857514\\
213	0.012018971040726\\
214	0.0120189276137403\\
215	0.0120188833906318\\
216	0.0120188383569931\\
217	0.0120187924981685\\
218	0.0120187457992508\\
219	0.0120186982450778\\
220	0.0120186498202288\\
221	0.0120186005090218\\
222	0.0120185502955096\\
223	0.0120184991634773\\
224	0.0120184470964386\\
225	0.0120183940776333\\
226	0.0120183400900242\\
227	0.0120182851162943\\
228	0.0120182291388446\\
229	0.0120181721397913\\
230	0.0120181141009638\\
231	0.0120180550039028\\
232	0.0120179948298586\\
233	0.0120179335597897\\
234	0.0120178711743619\\
235	0.0120178076539481\\
236	0.0120177429786273\\
237	0.0120176771281859\\
238	0.0120176100821185\\
239	0.0120175418196294\\
240	0.0120174723196353\\
241	0.0120174015607688\\
242	0.0120173295213823\\
243	0.012017256179554\\
244	0.0120171815130946\\
245	0.012017105499555\\
246	0.0120170281162367\\
247	0.0120169493402033\\
248	0.0120168691482939\\
249	0.0120167875171396\\
250	0.0120167044231821\\
251	0.0120166198426952\\
252	0.0120165337518103\\
253	0.0120164461265451\\
254	0.0120163569428371\\
255	0.0120162661765821\\
256	0.0120161738036776\\
257	0.0120160798000737\\
258	0.0120159841418299\\
259	0.0120158868051817\\
260	0.0120157877666151\\
261	0.0120156870029525\\
262	0.0120155844914507\\
263	0.0120154802099127\\
264	0.0120153741368156\\
265	0.0120152662514562\\
266	0.0120151565341183\\
267	0.0120150449662634\\
268	0.0120149315307494\\
269	0.0120148162120812\\
270	0.0120146989966962\\
271	0.0120145798732875\\
272	0.0120144588331593\\
273	0.0120143358705786\\
274	0.0120142109829975\\
275	0.0120140841706996\\
276	0.0120139554343474\\
277	0.0120138247651544\\
278	0.0120136921090047\\
279	0.0120135618480969\\
280	0.012013436982326\\
281	0.0120133097613101\\
282	0.012013180140791\\
283	0.0120130480756863\\
284	0.0120129135200739\\
285	0.0120127764271771\\
286	0.0120126367493488\\
287	0.0120124944380558\\
288	0.0120123494438624\\
289	0.0120122017164143\\
290	0.0120120512044217\\
291	0.0120118978556424\\
292	0.0120117416168646\\
293	0.0120115824338894\\
294	0.0120114202515127\\
295	0.0120112550135073\\
296	0.0120110866626045\\
297	0.0120109151404753\\
298	0.0120107403877112\\
299	0.012010562343805\\
300	0.0120103809471313\\
301	0.0120101961349261\\
302	0.0120100078432669\\
303	0.0120098160070518\\
304	0.0120096205599789\\
305	0.0120094214345247\\
306	0.0120092185619225\\
307	0.0120090118721409\\
308	0.0120088012938612\\
309	0.0120085867544547\\
310	0.0120083681799603\\
311	0.0120081454950609\\
312	0.0120079186230598\\
313	0.012007687485857\\
314	0.0120074520039248\\
315	0.0120072120962835\\
316	0.0120069676804761\\
317	0.0120067186725437\\
318	0.0120064649869991\\
319	0.0120062065368019\\
320	0.0120059432333316\\
321	0.0120056749863614\\
322	0.0120054017040314\\
323	0.0120051232928213\\
324	0.0120048396575232\\
325	0.0120045507012139\\
326	0.0120042563252265\\
327	0.0120039564291225\\
328	0.0120036509106629\\
329	0.0120033396657796\\
330	0.0120030225885459\\
331	0.0120026995711472\\
332	0.0120023705038515\\
333	0.0120020352749788\\
334	0.0120016937708717\\
335	0.012001345875864\\
336	0.0120009914722503\\
337	0.0120006304402548\\
338	0.012000262658\\
339	0.0119998880014746\\
340	0.0119995063445021\\
341	0.0119991175587074\\
342	0.0119987215134849\\
343	0.0119983180759645\\
344	0.0119979071109782\\
345	0.0119974884810256\\
346	0.0119970620462388\\
347	0.0119966276643467\\
348	0.0119961851906381\\
349	0.0119957344779242\\
350	0.0119952753764991\\
351	0.0119948077341001\\
352	0.0119943313958647\\
353	0.0119938462042868\\
354	0.0119933519991702\\
355	0.0119928486175787\\
356	0.0119923358937839\\
357	0.011991813659208\\
358	0.0119912817423628\\
359	0.011990739968783\\
360	0.0119901881609529\\
361	0.011989626138226\\
362	0.0119890537167359\\
363	0.0119884707092964\\
364	0.0119878769252905\\
365	0.011987272170545\\
366	0.0119866562471886\\
367	0.0119860289534915\\
368	0.0119853900836829\\
369	0.0119847394277427\\
370	0.0119840767711626\\
371	0.0119834018946737\\
372	0.0119827145739322\\
373	0.0119820145791579\\
374	0.0119813016747184\\
375	0.0119805756186474\\
376	0.0119798361620893\\
377	0.0119790830486569\\
378	0.0119783160136945\\
379	0.011977534783451\\
380	0.0119767390742242\\
381	0.0119759285917464\\
382	0.0119751030318458\\
383	0.0119742620862168\\
384	0.011973405467553\\
385	0.0119725330076814\\
386	0.0119716450343202\\
387	0.0119707413139761\\
388	0.0119698201724342\\
389	0.0119688812692837\\
390	0.0119679242569677\\
391	0.0119669487805691\\
392	0.0119659544775831\\
393	0.0119649409776798\\
394	0.0119639079024865\\
395	0.0119628548653285\\
396	0.0119617814709377\\
397	0.0119606873151641\\
398	0.0119595719846748\\
399	0.011958435056641\\
400	0.0119572760984121\\
401	0.0119560946671755\\
402	0.0119548903095999\\
403	0.0119536625614684\\
404	0.0119524109472724\\
405	0.0119511349797859\\
406	0.01194983415966\\
407	0.0119485079749752\\
408	0.0119471559007755\\
409	0.011945777398583\\
410	0.0119443719158949\\
411	0.0119429388856611\\
412	0.0119414777257444\\
413	0.0119399878383559\\
414	0.0119384686094617\\
415	0.0119369194081628\\
416	0.0119353395861165\\
417	0.0119337284771905\\
418	0.0119320853968156\\
419	0.0119304096416062\\
420	0.0119287004893926\\
421	0.0119269572002021\\
422	0.0119251790192836\\
423	0.0119233651830204\\
424	0.0119215149199847\\
425	0.0119196273795223\\
426	0.0119177016602142\\
427	0.0119157369004203\\
428	0.0119137322123761\\
429	0.0119116866809773\\
430	0.011909599362491\\
431	0.0119074692831836\\
432	0.0119052954378612\\
433	0.0119030767883182\\
434	0.0119008122616852\\
435	0.0118985007486698\\
436	0.0118961411016814\\
437	0.0118937321328313\\
438	0.0118912726117987\\
439	0.0118887612635533\\
440	0.0118861967659203\\
441	0.0118835777469791\\
442	0.0118809027822839\\
443	0.0118781703918965\\
444	0.0118753790372304\\
445	0.011872527117709\\
446	0.0118696129672333\\
447	0.011866634850359\\
448	0.0118635909577247\\
449	0.0118604793995762\\
450	0.0118572981979823\\
451	0.0118540452906383\\
452	0.0118507185195902\\
453	0.0118473156249637\\
454	0.0118438342382083\\
455	0.0118402718748141\\
456	0.0118366259264537\\
457	0.0118328936525002\\
458	0.0118290721708644\\
459	0.0118251584480907\\
460	0.0118211492886442\\
461	0.0118170413233156\\
462	0.0118128309966606\\
463	0.0118085145533854\\
464	0.0118040880235743\\
465	0.0117995472066425\\
466	0.0117948876538708\\
467	0.0117901046493747\\
468	0.0117851931894531\\
469	0.0117801479605565\\
470	0.0117749633151626\\
471	0.0117696332460602\\
472	0.0117641513599051\\
473	0.0117585108519717\\
474	0.0117527044874203\\
475	0.0117467245944295\\
476	0.0117405630496169\\
477	0.0117342116691335\\
478	0.0117276628023863\\
479	0.0117209063970582\\
480	0.0117139320718934\\
481	0.0117067434107674\\
482	0.0116993046617714\\
483	0.0116915962327533\\
484	0.0116836024774085\\
485	0.0116753067087091\\
486	0.0116666908925494\\
487	0.011657735512381\\
488	0.0116484194114408\\
489	0.0116387196211643\\
490	0.0116286111676767\\
491	0.0116180668531775\\
492	0.0116070570078661\\
493	0.0115955492072172\\
494	0.0115835079480167\\
495	0.011570894274926\\
496	0.0115576653481582\\
497	0.0115437739440877\\
498	0.0115291678845862\\
499	0.0115137893699679\\
500	0.0114975739586083\\
501	0.0114804489455737\\
502	0.0114623322038971\\
503	0.0114431313060926\\
504	0.0114227500657037\\
505	0.0114010442641983\\
506	0.0113778066135288\\
507	0.0113500196181607\\
508	0.0113041321689353\\
509	0.0112558742749655\\
510	0.0112045326765312\\
511	0.0111097332188489\\
512	0.0109651003693197\\
513	0.0108179424333279\\
514	0.0106685963528404\\
515	0.0105982671538382\\
516	0.0105628667937857\\
517	0.0105330914377936\\
518	0.0105104258309053\\
519	0.0104892880612886\\
520	0.0104696410556042\\
521	0.0104503322380931\\
522	0.010430778893258\\
523	0.010410924496438\\
524	0.0103907109350068\\
525	0.010370065689303\\
526	0.0103489689272352\\
527	0.0103274027573535\\
528	0.0103053510070502\\
529	0.0102828000380246\\
530	0.0102597370891189\\
531	0.0102361494788378\\
532	0.0102120245456834\\
533	0.0101873495147147\\
534	0.0101621075080765\\
535	0.0101360149873251\\
536	0.0101092330629758\\
537	0.010081606928069\\
538	0.0100519479721389\\
539	0.0100223711098094\\
540	0.00999451555122586\\
541	0.00996726818527192\\
542	0.00993953621963588\\
543	0.00991120468542803\\
544	0.00988225526518685\\
545	0.00985267250871387\\
546	0.00982244948141088\\
547	0.00979159559488473\\
548	0.00976016293484706\\
549	0.00972803578235161\\
550	0.00969517692126071\\
551	0.00966156706231893\\
552	0.00962718609996824\\
553	0.00959174752024606\\
554	0.00955495298772372\\
555	0.00951380796051269\\
556	0.00947563820042506\\
557	0.00944096945327738\\
558	0.00940564223266134\\
559	0.00936960350752038\\
560	0.00933283573091086\\
561	0.00929532122306513\\
562	0.00925704172702615\\
563	0.009217978314658\\
564	0.00917811122032248\\
565	0.00913741944841397\\
566	0.00906187965960978\\
567	0.00869695649895891\\
568	0.00838749700177051\\
569	0.00832310828745102\\
570	0.00825763909273647\\
571	0.00819106539482789\\
572	0.00812336225920869\\
573	0.00805450378357674\\
574	0.00798446304588583\\
575	0.00791321205051272\\
576	0.00784072167272329\\
577	0.00776696160163777\\
578	0.00769190028172315\\
579	0.00761550485219264\\
580	0.00753774108175188\\
581	0.00745857329076526\\
582	0.00737796423854298\\
583	0.00729587491554803\\
584	0.00721226408110736\\
585	0.00712708712854453\\
586	0.00704029318686139\\
587	0.00695181762108358\\
588	0.00686156256306142\\
589	0.00676934637110724\\
590	0.00667477256501155\\
591	0.00657713355197373\\
592	0.00647481662499991\\
593	0.00636364764633834\\
594	0.00623271716130668\\
595	0.0060534099414828\\
596	0.00575058001197164\\
597	0.00512683753504545\\
598	0.00366374385960312\\
599	0\\
600	0\\
};
\addplot [color=mycolor3,solid,forget plot]
  table[row sep=crcr]{%
1	0.0120294184481938\\
2	0.012029417555819\\
3	0.0120294166474337\\
4	0.0120294157227489\\
5	0.0120294147814703\\
6	0.0120294138232981\\
7	0.0120294128479273\\
8	0.0120294118550471\\
9	0.0120294108443413\\
10	0.0120294098154878\\
11	0.0120294087681586\\
12	0.0120294077020199\\
13	0.0120294066167317\\
14	0.0120294055119477\\
15	0.0120294043873156\\
16	0.0120294032424764\\
17	0.0120294020770648\\
18	0.0120294008907087\\
19	0.0120293996830292\\
20	0.0120293984536406\\
21	0.0120293972021501\\
22	0.0120293959281578\\
23	0.0120293946312565\\
24	0.0120293933110315\\
25	0.0120293919670607\\
26	0.0120293905989142\\
27	0.0120293892061542\\
28	0.0120293877883351\\
29	0.012029386345003\\
30	0.0120293848756958\\
31	0.0120293833799429\\
32	0.0120293818572654\\
33	0.0120293803071752\\
34	0.0120293787291756\\
35	0.0120293771227608\\
36	0.0120293754874158\\
37	0.012029373822616\\
38	0.0120293721278274\\
39	0.0120293704025062\\
40	0.0120293686460988\\
41	0.0120293668580412\\
42	0.0120293650377593\\
43	0.0120293631846684\\
44	0.0120293612981734\\
45	0.0120293593776679\\
46	0.0120293574225348\\
47	0.0120293554321453\\
48	0.0120293534058596\\
49	0.0120293513430258\\
50	0.0120293492429802\\
51	0.0120293471050471\\
52	0.0120293449285382\\
53	0.0120293427127528\\
54	0.0120293404569773\\
55	0.0120293381604851\\
56	0.0120293358225361\\
57	0.0120293334423769\\
58	0.0120293310192402\\
59	0.0120293285523446\\
60	0.0120293260408945\\
61	0.0120293234840795\\
62	0.0120293208810748\\
63	0.0120293182310399\\
64	0.0120293155331194\\
65	0.012029312786442\\
66	0.0120293099901204\\
67	0.0120293071432511\\
68	0.0120293042449139\\
69	0.0120293012941721\\
70	0.0120292982900713\\
71	0.01202929523164\\
72	0.0120292921178886\\
73	0.0120292889478096\\
74	0.0120292857203768\\
75	0.0120292824345452\\
76	0.0120292790892507\\
77	0.0120292756834096\\
78	0.0120292722159184\\
79	0.0120292686856531\\
80	0.0120292650914693\\
81	0.0120292614322015\\
82	0.0120292577066627\\
83	0.0120292539136441\\
84	0.0120292500519149\\
85	0.0120292461202213\\
86	0.0120292421172868\\
87	0.0120292380418112\\
88	0.0120292338924706\\
89	0.0120292296679165\\
90	0.0120292253667758\\
91	0.0120292209876499\\
92	0.0120292165291149\\
93	0.0120292119897202\\
94	0.0120292073679888\\
95	0.0120292026624165\\
96	0.0120291978714712\\
97	0.0120291929935929\\
98	0.0120291880271927\\
99	0.0120291829706523\\
100	0.0120291778223239\\
101	0.0120291725805291\\
102	0.0120291672435587\\
103	0.012029161809672\\
104	0.0120291562770962\\
105	0.0120291506440258\\
106	0.0120291449086221\\
107	0.0120291390690125\\
108	0.0120291331232897\\
109	0.0120291270695115\\
110	0.0120291209056998\\
111	0.0120291146298399\\
112	0.01202910823988\\
113	0.0120291017337306\\
114	0.0120290951092636\\
115	0.0120290883643115\\
116	0.0120290814966669\\
117	0.0120290745040818\\
118	0.0120290673842665\\
119	0.0120290601348893\\
120	0.0120290527535754\\
121	0.0120290452379059\\
122	0.0120290375854178\\
123	0.0120290297936021\\
124	0.0120290218599039\\
125	0.012029013781721\\
126	0.0120290055564032\\
127	0.0120289971812513\\
128	0.0120289886535164\\
129	0.0120289799703989\\
130	0.0120289711290475\\
131	0.0120289621265582\\
132	0.0120289529599735\\
133	0.0120289436262814\\
134	0.0120289341224142\\
135	0.0120289244452476\\
136	0.0120289145915999\\
137	0.0120289045582304\\
138	0.0120288943418389\\
139	0.0120288839390641\\
140	0.0120288733464828\\
141	0.0120288625606089\\
142	0.0120288515778917\\
143	0.0120288403947153\\
144	0.012028829007397\\
145	0.0120288174121864\\
146	0.0120288056052639\\
147	0.0120287935827395\\
148	0.0120287813406517\\
149	0.012028768874966\\
150	0.0120287561815737\\
151	0.0120287432562902\\
152	0.0120287300948542\\
153	0.012028716692926\\
154	0.0120287030460859\\
155	0.0120286891498331\\
156	0.0120286749995839\\
157	0.0120286605906705\\
158	0.0120286459183392\\
159	0.0120286309777491\\
160	0.0120286157639703\\
161	0.0120286002719823\\
162	0.0120285844966728\\
163	0.0120285684328351\\
164	0.0120285520751674\\
165	0.0120285354182704\\
166	0.0120285184566459\\
167	0.0120285011846948\\
168	0.0120284835967152\\
169	0.0120284656869009\\
170	0.0120284474493389\\
171	0.0120284288780082\\
172	0.0120284099667771\\
173	0.0120283907094017\\
174	0.0120283710995235\\
175	0.0120283511306677\\
176	0.0120283307962404\\
177	0.0120283100895271\\
178	0.0120282890036903\\
179	0.0120282675317669\\
180	0.0120282456666662\\
181	0.0120282234011675\\
182	0.0120282007279176\\
183	0.0120281776394285\\
184	0.0120281541280747\\
185	0.0120281301860908\\
186	0.0120281058055692\\
187	0.0120280809784568\\
188	0.0120280556965533\\
189	0.0120280299515078\\
190	0.0120280037348172\\
191	0.0120279770378239\\
192	0.0120279498517154\\
193	0.0120279221675255\\
194	0.0120278939761393\\
195	0.0120278652683013\\
196	0.0120278360346111\\
197	0.0120278062654331\\
198	0.0120277759504069\\
199	0.0120277450795075\\
200	0.0120277136425499\\
201	0.0120276816291618\\
202	0.0120276490287804\\
203	0.0120276158306487\\
204	0.0120275820238121\\
205	0.012027547597114\\
206	0.0120275125391925\\
207	0.0120274768384759\\
208	0.0120274404831791\\
209	0.0120274034612994\\
210	0.0120273657606117\\
211	0.0120273273686646\\
212	0.0120272882727759\\
213	0.0120272484600275\\
214	0.0120272079172613\\
215	0.0120271666310738\\
216	0.012027124587811\\
217	0.0120270817735637\\
218	0.0120270381741617\\
219	0.0120269937751683\\
220	0.0120269485618749\\
221	0.0120269025192947\\
222	0.0120268556321566\\
223	0.0120268078848991\\
224	0.0120267592616635\\
225	0.0120267097462871\\
226	0.0120266593222959\\
227	0.0120266079728973\\
228	0.0120265556809724\\
229	0.0120265024290674\\
230	0.0120264481993857\\
231	0.0120263929737783\\
232	0.0120263367337346\\
233	0.0120262794603728\\
234	0.0120262211344286\\
235	0.0120261617362449\\
236	0.0120261012457592\\
237	0.0120260396424918\\
238	0.0120259769055316\\
239	0.0120259130135226\\
240	0.012025847944648\\
241	0.0120257816766139\\
242	0.0120257141866318\\
243	0.0120256454513992\\
244	0.0120255754470795\\
245	0.0120255041492789\\
246	0.0120254315330229\\
247	0.0120253575727294\\
248	0.0120252822421802\\
249	0.0120252055144893\\
250	0.0120251273620687\\
251	0.0120250477565904\\
252	0.0120249666689445\\
253	0.0120248840691939\\
254	0.0120247999265234\\
255	0.0120247142091835\\
256	0.0120246268844287\\
257	0.0120245379184485\\
258	0.0120244472762907\\
259	0.0120243549217768\\
260	0.0120242608174063\\
261	0.0120241649242511\\
262	0.0120240672018367\\
263	0.0120239676080095\\
264	0.0120238660987877\\
265	0.0120237626281936\\
266	0.0120236571480655\\
267	0.0120235496078454\\
268	0.0120234399543379\\
269	0.0120233281314393\\
270	0.0120232140798276\\
271	0.0120230977366076\\
272	0.0120229790348989\\
273	0.012022857903336\\
274	0.0120227342654058\\
275	0.0120226080383852\\
276	0.012022479131077\\
277	0.0120223474375137\\
278	0.0120222128162072\\
279	0.0120220715723733\\
280	0.0120219213295351\\
281	0.0120217683596326\\
282	0.012021612613815\\
283	0.0120214540423758\\
284	0.0120212925947387\\
285	0.0120211282194428\\
286	0.012020960864128\\
287	0.0120207904755191\\
288	0.012020616999411\\
289	0.0120204403806528\\
290	0.0120202605631322\\
291	0.0120200774897588\\
292	0.012019891102448\\
293	0.0120197013421045\\
294	0.012019508148605\\
295	0.0120193114607812\\
296	0.0120191112164021\\
297	0.0120189073521568\\
298	0.0120186998036358\\
299	0.0120184885053131\\
300	0.0120182733905276\\
301	0.0120180543914639\\
302	0.0120178314391339\\
303	0.0120176044633563\\
304	0.0120173733927379\\
305	0.0120171381546528\\
306	0.0120168986752226\\
307	0.0120166548792956\\
308	0.0120164066904259\\
309	0.0120161540308522\\
310	0.0120158968214763\\
311	0.0120156349818416\\
312	0.0120153684301107\\
313	0.0120150970830431\\
314	0.0120148208559728\\
315	0.0120145396627854\\
316	0.0120142534158947\\
317	0.0120139620262195\\
318	0.01201366540316\\
319	0.0120133634545736\\
320	0.0120130560867512\\
321	0.0120127432043928\\
322	0.0120124247105826\\
323	0.0120121005067651\\
324	0.0120117704927196\\
325	0.012011434566536\\
326	0.0120110926245892\\
327	0.0120107445615147\\
328	0.0120103902701834\\
329	0.0120100296416768\\
330	0.0120096625652621\\
331	0.0120092889283681\\
332	0.0120089086165606\\
333	0.0120085215135186\\
334	0.0120081275010108\\
335	0.0120077264588732\\
336	0.0120073182649862\\
337	0.0120069027952543\\
338	0.0120064799235848\\
339	0.0120060495218696\\
340	0.0120056114599669\\
341	0.0120051656056856\\
342	0.0120047118247706\\
343	0.0120042499808916\\
344	0.012003779935633\\
345	0.0120033015484878\\
346	0.0120028146768541\\
347	0.0120023191760361\\
348	0.012001814899249\\
349	0.0120013016976288\\
350	0.0120007794202484\\
351	0.0120002479141394\\
352	0.0119997070243222\\
353	0.0119991565938432\\
354	0.011998596463823\\
355	0.0119980264735143\\
356	0.0119974464603725\\
357	0.0119968562601403\\
358	0.0119962557069481\\
359	0.011995644633432\\
360	0.0119950228708724\\
361	0.0119943902493562\\
362	0.0119937465979637\\
363	0.0119930917449871\\
364	0.0119924255181811\\
365	0.0119917477450529\\
366	0.0119910582531953\\
367	0.011990356870669\\
368	0.011989643426442\\
369	0.0119889177508927\\
370	0.011988179676386\\
371	0.0119874290379337\\
372	0.0119866656739499\\
373	0.0119858894271167\\
374	0.0119851001453745\\
375	0.0119842976830583\\
376	0.011983481902199\\
377	0.0119826526740142\\
378	0.0119818098806119\\
379	0.0119809534169174\\
380	0.0119800831927854\\
381	0.0119791991350906\\
382	0.0119783011890194\\
383	0.01197738931588\\
384	0.0119764634782775\\
385	0.0119755235810857\\
386	0.0119745692569808\\
387	0.011973625409027\\
388	0.0119727124513343\\
389	0.0119717826992378\\
390	0.0119708358443789\\
391	0.0119698715731102\\
392	0.0119688895663477\\
393	0.0119678894992354\\
394	0.0119668710405976\\
395	0.011965833853782\\
396	0.0119647775967581\\
397	0.0119637019212927\\
398	0.0119626064726859\\
399	0.0119614908894618\\
400	0.0119603548030381\\
401	0.0119591978374037\\
402	0.0119580196087587\\
403	0.0119568197251915\\
404	0.0119555977866664\\
405	0.0119543533842356\\
406	0.0119530860987276\\
407	0.0119517955009753\\
408	0.0119504811512473\\
409	0.0119491425986175\\
410	0.0119477793802661\\
411	0.0119463910207103\\
412	0.0119449770309564\\
413	0.0119435369075439\\
414	0.0119420701313624\\
415	0.0119405761658887\\
416	0.0119390544541572\\
417	0.011937504414738\\
418	0.0119359254489342\\
419	0.0119343169344911\\
420	0.0119326782233582\\
421	0.0119310086428773\\
422	0.011929307507584\\
423	0.0119275741712286\\
424	0.0119258082305455\\
425	0.0119240103084827\\
426	0.0119221792712286\\
427	0.0119203122165478\\
428	0.0119184083403242\\
429	0.0119164668132416\\
430	0.0119144867795492\\
431	0.0119124673558056\\
432	0.0119104076294985\\
433	0.0119083066575484\\
434	0.0119061634647583\\
435	0.0119039770421726\\
436	0.0119017463453406\\
437	0.0118994702924773\\
438	0.0118971477625141\\
439	0.0118947775930574\\
440	0.0118923585782448\\
441	0.0118898894664903\\
442	0.0118873689581361\\
443	0.0118847957030411\\
444	0.0118821682981956\\
445	0.0118794852855955\\
446	0.0118767451509667\\
447	0.0118739463246458\\
448	0.011871087186619\\
449	0.0118681660736066\\
450	0.0118651812470562\\
451	0.0118621307121703\\
452	0.0118590125533093\\
453	0.0118558247686228\\
454	0.0118525652646353\\
455	0.0118492318504519\\
456	0.0118458222315362\\
457	0.0118423340030213\\
458	0.0118387646425096\\
459	0.0118351115023127\\
460	0.0118313718010989\\
461	0.0118275426148839\\
462	0.0118236208673154\\
463	0.0118196033191998\\
464	0.0118154865572239\\
465	0.0118112669818219\\
466	0.0118069407940596\\
467	0.0118025039810784\\
468	0.0117979522987619\\
469	0.0117932812501646\\
470	0.0117884860748163\\
471	0.011783561727839\\
472	0.0117785028554338\\
473	0.0117733037714346\\
474	0.0117679584324752\\
475	0.0117624604127193\\
476	0.0117568028825618\\
477	0.0117509785853709\\
478	0.0117449798086852\\
479	0.011738798513994\\
480	0.0117324267035955\\
481	0.0117258577258535\\
482	0.0117190787737757\\
483	0.0117120827645168\\
484	0.0117048720754383\\
485	0.0116974051805671\\
486	0.0116896676075278\\
487	0.0116816438179188\\
488	0.0116733172216274\\
489	0.0116646699106196\\
490	0.0116556825278185\\
491	0.01164633411655\\
492	0.0116366019565618\\
493	0.0116264613796394\\
494	0.0116158855619995\\
495	0.0116048452897468\\
496	0.0115933086932196\\
497	0.011581240944953\\
498	0.0115686039134893\\
499	0.0115553557601971\\
500	0.0115414504689702\\
501	0.0115268373147794\\
502	0.0115114602484086\\
503	0.0114952571479728\\
504	0.0114781582844479\\
505	0.0114600852174173\\
506	0.0114409493222894\\
507	0.0114206527983975\\
508	0.0113990867611616\\
509	0.0113760762083767\\
510	0.0113513752036678\\
511	0.0113159381479182\\
512	0.0112679636595168\\
513	0.0112173754871497\\
514	0.0111633610544983\\
515	0.0110428004028452\\
516	0.0108941234215818\\
517	0.0107427987078863\\
518	0.0105892321186162\\
519	0.010537604940941\\
520	0.010500770558179\\
521	0.0104697504356829\\
522	0.0104461025364528\\
523	0.0104232374023355\\
524	0.0104012318717967\\
525	0.0103802574261422\\
526	0.0103590187928536\\
527	0.0103374538335167\\
528	0.0103155080760614\\
529	0.0102931128473157\\
530	0.0102702293080313\\
531	0.0102468372365915\\
532	0.0102229185854888\\
533	0.0101984567735062\\
534	0.0101734373743994\\
535	0.010147846227661\\
536	0.0101216669001464\\
537	0.0100948073762948\\
538	0.0100669889691556\\
539	0.0100386403761204\\
540	0.0100084781527846\\
541	0.00997746649239294\\
542	0.00994664059615901\\
543	0.00991824089606246\\
544	0.00988933523322398\\
545	0.00985987564026084\\
546	0.00982978318555431\\
547	0.00979905747736052\\
548	0.00976773860790732\\
549	0.00973575981902065\\
550	0.0097030515171097\\
551	0.00966959432886941\\
552	0.00963536803785071\\
553	0.00960035150018078\\
554	0.00956451998046889\\
555	0.00952705413323322\\
556	0.00948710602387265\\
557	0.00944503338855385\\
558	0.00940887643030145\\
559	0.00937288992527579\\
560	0.00933618315537469\\
561	0.00929873143952121\\
562	0.00926051622101601\\
563	0.00922151860131325\\
564	0.00918171898438378\\
565	0.00914109688559529\\
566	0.00909963057080181\\
567	0.00901421719622151\\
568	0.00862731211635968\\
569	0.00832310839907635\\
570	0.00825763909419514\\
571	0.00819106539533925\\
572	0.0081233622594618\\
573	0.00805450378369977\\
574	0.00798446304594252\\
575	0.00791321205053708\\
576	0.0078407216727329\\
577	0.00776696160164116\\
578	0.00769190028172421\\
579	0.00761550485219292\\
580	0.00753774108175196\\
581	0.00745857329076527\\
582	0.00737796423854298\\
583	0.00729587491554802\\
584	0.00721226408110735\\
585	0.00712708712854451\\
586	0.00704029318686139\\
587	0.00695181762108358\\
588	0.00686156256306142\\
589	0.00676934637110726\\
590	0.00667477256501154\\
591	0.00657713355197374\\
592	0.0064748166249999\\
593	0.00636364764633834\\
594	0.00623271716130668\\
595	0.0060534099414828\\
596	0.00575058001197162\\
597	0.00512683753504545\\
598	0.00366374385960312\\
599	0\\
600	0\\
};
\addplot [color=mycolor4,solid,forget plot]
  table[row sep=crcr]{%
1	0.0120359606790683\\
2	0.0120359597145902\\
3	0.0120359587329864\\
4	0.0120359577339511\\
5	0.0120359567171732\\
6	0.0120359556823358\\
7	0.0120359546291166\\
8	0.0120359535571873\\
9	0.0120359524662137\\
10	0.0120359513558557\\
11	0.0120359502257671\\
12	0.0120359490755953\\
13	0.0120359479049816\\
14	0.0120359467135608\\
15	0.012035945500961\\
16	0.0120359442668036\\
17	0.0120359430107035\\
18	0.0120359417322684\\
19	0.0120359404310989\\
20	0.0120359391067887\\
21	0.0120359377589239\\
22	0.0120359363870833\\
23	0.0120359349908382\\
24	0.0120359335697521\\
25	0.0120359321233805\\
26	0.0120359306512711\\
27	0.0120359291529634\\
28	0.0120359276279886\\
29	0.0120359260758695\\
30	0.0120359244961203\\
31	0.0120359228882463\\
32	0.0120359212517442\\
33	0.0120359195861013\\
34	0.0120359178907959\\
35	0.0120359161652967\\
36	0.0120359144090631\\
37	0.0120359126215446\\
38	0.0120359108021806\\
39	0.0120359089504007\\
40	0.0120359070656242\\
41	0.0120359051472595\\
42	0.012035903194705\\
43	0.0120359012073476\\
44	0.0120358991845636\\
45	0.0120358971257179\\
46	0.0120358950301637\\
47	0.012035892897243\\
48	0.0120358907262856\\
49	0.0120358885166092\\
50	0.0120358862675193\\
51	0.0120358839783088\\
52	0.0120358816482579\\
53	0.0120358792766335\\
54	0.0120358768626898\\
55	0.012035874405667\\
56	0.0120358719047918\\
57	0.012035869359277\\
58	0.0120358667683209\\
59	0.0120358641311076\\
60	0.0120358614468063\\
61	0.0120358587145711\\
62	0.0120358559335409\\
63	0.012035853102839\\
64	0.0120358502215729\\
65	0.0120358472888338\\
66	0.0120358443036965\\
67	0.012035841265219\\
68	0.0120358381724425\\
69	0.0120358350243904\\
70	0.0120358318200688\\
71	0.0120358285584655\\
72	0.0120358252385501\\
73	0.0120358218592735\\
74	0.0120358184195676\\
75	0.0120358149183448\\
76	0.0120358113544979\\
77	0.0120358077268996\\
78	0.012035804034402\\
79	0.0120358002758366\\
80	0.0120357964500134\\
81	0.012035792555721\\
82	0.0120357885917259\\
83	0.0120357845567722\\
84	0.0120357804495813\\
85	0.0120357762688511\\
86	0.0120357720132559\\
87	0.0120357676814461\\
88	0.0120357632720474\\
89	0.0120357587836603\\
90	0.0120357542148601\\
91	0.0120357495641962\\
92	0.0120357448301912\\
93	0.0120357400113412\\
94	0.0120357351061147\\
95	0.0120357301129522\\
96	0.0120357250302659\\
97	0.0120357198564392\\
98	0.0120357145898256\\
99	0.0120357092287489\\
100	0.0120357037715022\\
101	0.0120356982163473\\
102	0.0120356925615144\\
103	0.0120356868052013\\
104	0.012035680945573\\
105	0.0120356749807607\\
106	0.0120356689088616\\
107	0.0120356627279381\\
108	0.0120356564360171\\
109	0.0120356500310893\\
110	0.0120356435111087\\
111	0.0120356368739918\\
112	0.0120356301176169\\
113	0.0120356232398234\\
114	0.0120356162384112\\
115	0.0120356091111397\\
116	0.0120356018557272\\
117	0.0120355944698502\\
118	0.0120355869511425\\
119	0.0120355792971943\\
120	0.0120355715055517\\
121	0.0120355635737156\\
122	0.0120355554991411\\
123	0.0120355472792362\\
124	0.0120355389113615\\
125	0.0120355303928288\\
126	0.0120355217209005\\
127	0.0120355128927886\\
128	0.0120355039056537\\
129	0.012035494756604\\
130	0.0120354854426944\\
131	0.0120354759609256\\
132	0.0120354663082429\\
133	0.0120354564815352\\
134	0.012035446477634\\
135	0.0120354362933123\\
136	0.0120354259252834\\
137	0.0120354153702001\\
138	0.0120354046246532\\
139	0.0120353936851705\\
140	0.0120353825482156\\
141	0.012035371210187\\
142	0.0120353596674161\\
143	0.0120353479161669\\
144	0.0120353359526342\\
145	0.0120353237729423\\
146	0.0120353113731439\\
147	0.0120352987492188\\
148	0.0120352858970722\\
149	0.0120352728125338\\
150	0.0120352594913562\\
151	0.0120352459292132\\
152	0.012035232121699\\
153	0.0120352180643262\\
154	0.0120352037525244\\
155	0.012035189181639\\
156	0.0120351743469293\\
157	0.0120351592435672\\
158	0.0120351438666356\\
159	0.0120351282111265\\
160	0.01203511227194\\
161	0.0120350960438821\\
162	0.0120350795216631\\
163	0.0120350626998962\\
164	0.0120350455730957\\
165	0.0120350281356749\\
166	0.0120350103819449\\
167	0.0120349923061124\\
168	0.012034973902278\\
169	0.0120349551644345\\
170	0.0120349360864647\\
171	0.0120349166621398\\
172	0.0120348968851174\\
173	0.0120348767489394\\
174	0.0120348562470302\\
175	0.0120348353726944\\
176	0.012034814119115\\
177	0.0120347924793509\\
178	0.0120347704463351\\
179	0.0120347480128722\\
180	0.0120347251716363\\
181	0.0120347019151684\\
182	0.0120346782358744\\
183	0.0120346541260219\\
184	0.0120346295777385\\
185	0.0120346045830083\\
186	0.0120345791336699\\
187	0.0120345532214131\\
188	0.0120345268377759\\
189	0.0120344999741422\\
190	0.0120344726217377\\
191	0.0120344447716273\\
192	0.0120344164147119\\
193	0.0120343875417244\\
194	0.0120343581432269\\
195	0.0120343282096068\\
196	0.0120342977310737\\
197	0.0120342666976557\\
198	0.012034235099196\\
199	0.012034202925348\\
200	0.0120341701655715\\
201	0.0120341368091284\\
202	0.0120341028450792\\
203	0.0120340682622785\\
204	0.0120340330493707\\
205	0.0120339971947859\\
206	0.0120339606867355\\
207	0.0120339235132071\\
208	0.0120338856619602\\
209	0.0120338471205213\\
210	0.0120338078761788\\
211	0.0120337679159779\\
212	0.0120337272267153\\
213	0.0120336857949339\\
214	0.0120336436069171\\
215	0.012033600648683\\
216	0.0120335569059784\\
217	0.0120335123642729\\
218	0.0120334670087527\\
219	0.0120334208243135\\
220	0.0120333737955544\\
221	0.0120333259067709\\
222	0.0120332771419474\\
223	0.0120332274847502\\
224	0.0120331769185197\\
225	0.0120331254262626\\
226	0.0120330729906438\\
227	0.0120330195939781\\
228	0.0120329652182214\\
229	0.0120329098449618\\
230	0.0120328534554103\\
231	0.0120327960303916\\
232	0.0120327375503336\\
233	0.012032677995258\\
234	0.0120326173447693\\
235	0.0120325555780439\\
236	0.0120324926738197\\
237	0.0120324286103835\\
238	0.0120323633655602\\
239	0.0120322969167001\\
240	0.0120322292406665\\
241	0.0120321603138232\\
242	0.0120320901120211\\
243	0.0120320186105852\\
244	0.0120319457843009\\
245	0.0120318716074002\\
246	0.0120317960535482\\
247	0.0120317190958286\\
248	0.0120316407067303\\
249	0.0120315608581331\\
250	0.0120314795212942\\
251	0.0120313966668353\\
252	0.0120313122647291\\
253	0.0120312262842875\\
254	0.0120311386941505\\
255	0.0120310494622759\\
256	0.0120309585559307\\
257	0.012030865941684\\
258	0.0120307715854032\\
259	0.0120306754522507\\
260	0.0120305775066864\\
261	0.0120304777124717\\
262	0.0120303760326797\\
263	0.0120302724297095\\
264	0.012030166865308\\
265	0.0120300593005975\\
266	0.0120299496961136\\
267	0.0120298380118518\\
268	0.0120297242073262\\
269	0.0120296082416417\\
270	0.0120294900735813\\
271	0.0120293696617117\\
272	0.0120292469645062\\
273	0.0120291219404863\\
274	0.0120289945483684\\
275	0.0120288647471818\\
276	0.0120287324962426\\
277	0.0120285977546729\\
278	0.0120284604796759\\
279	0.0120283198503998\\
280	0.0120281753226869\\
281	0.0120280282403208\\
282	0.0120278785593939\\
283	0.0120277262352805\\
284	0.0120275712226256\\
285	0.0120274134753339\\
286	0.0120272529465588\\
287	0.0120270895886905\\
288	0.0120269233533449\\
289	0.0120267541913517\\
290	0.0120265820527423\\
291	0.0120264068867378\\
292	0.0120262286417368\\
293	0.0120260472653024\\
294	0.0120258627041503\\
295	0.0120256749041348\\
296	0.0120254838102365\\
297	0.0120252893665483\\
298	0.0120250915162623\\
299	0.0120248902016553\\
300	0.012024685364075\\
301	0.0120244769439255\\
302	0.0120242648806526\\
303	0.0120240491127286\\
304	0.0120238295776373\\
305	0.012023606211858\\
306	0.0120233789508497\\
307	0.0120231477290347\\
308	0.012022912479782\\
309	0.0120226731353898\\
310	0.0120224296270684\\
311	0.0120221818849219\\
312	0.0120219298379298\\
313	0.0120216734139283\\
314	0.0120214125395901\\
315	0.0120211471404055\\
316	0.0120208771406605\\
317	0.0120206024634166\\
318	0.0120203230304882\\
319	0.0120200387624201\\
320	0.0120197495784643\\
321	0.0120194553965554\\
322	0.0120191561332859\\
323	0.0120188517038797\\
324	0.0120185420221655\\
325	0.0120182270005482\\
326	0.0120179065499801\\
327	0.0120175805799298\\
328	0.012017248998351\\
329	0.0120169117116484\\
330	0.0120165686246435\\
331	0.0120162196405377\\
332	0.0120158646608736\\
333	0.0120155035854954\\
334	0.0120151363125056\\
335	0.0120147627382206\\
336	0.0120143827571233\\
337	0.012013996261813\\
338	0.0120136031429519\\
339	0.0120132032892096\\
340	0.0120127965872026\\
341	0.0120123829214307\\
342	0.0120119621742093\\
343	0.0120115342255967\\
344	0.0120110989533162\\
345	0.0120106562326738\\
346	0.0120102059364678\\
347	0.0120097479348941\\
348	0.012009282095442\\
349	0.0120088082827839\\
350	0.012008326358654\\
351	0.0120078361817192\\
352	0.0120073376074382\\
353	0.0120068304879084\\
354	0.0120063146717\\
355	0.0120057900036757\\
356	0.0120052563247937\\
357	0.012004713471893\\
358	0.0120041612774585\\
359	0.0120035995693643\\
360	0.012003028170592\\
361	0.0120024468989218\\
362	0.0120018555665928\\
363	0.012001253979929\\
364	0.0120006419389284\\
365	0.0120000192368075\\
366	0.0119993856595003\\
367	0.0119987409851025\\
368	0.0119980849832564\\
369	0.0119974174144692\\
370	0.0119967380293545\\
371	0.0119960465677902\\
372	0.011995342757979\\
373	0.0119946263154014\\
374	0.0119938969416451\\
375	0.0119931543230939\\
376	0.0119923981294571\\
377	0.0119916280121149\\
378	0.0119908436022476\\
379	0.0119900445087076\\
380	0.0119892303155564\\
381	0.0119884005791028\\
382	0.0119875548239905\\
383	0.0119866925369336\\
384	0.0119858131533778\\
385	0.0119849160203859\\
386	0.0119840002743995\\
387	0.0119830447810233\\
388	0.0119820332865597\\
389	0.0119810043420775\\
390	0.0119799576332635\\
391	0.0119788928404135\\
392	0.0119778096411205\\
393	0.0119767077110179\\
394	0.0119755867194906\\
395	0.0119744463010784\\
396	0.0119732860919018\\
397	0.0119721057495234\\
398	0.0119709049305588\\
399	0.0119696832912273\\
400	0.0119684404875616\\
401	0.0119671761753424\\
402	0.0119658900101433\\
403	0.0119645816462572\\
404	0.0119632507360533\\
405	0.0119618969494199\\
406	0.0119605199699264\\
407	0.0119591194537006\\
408	0.0119576950576613\\
409	0.0119562464404785\\
410	0.0119547732637695\\
411	0.0119532751936683\\
412	0.0119517519030298\\
413	0.0119502030747196\\
414	0.0119486284065388\\
415	0.0119470276175077\\
416	0.0119454004498282\\
417	0.011943746637176\\
418	0.0119420657191672\\
419	0.0119403574311319\\
420	0.0119386215620075\\
421	0.0119368579280512\\
422	0.0119350663669119\\
423	0.0119332467052732\\
424	0.011931398634631\\
425	0.0119295212651841\\
426	0.0119276712979806\\
427	0.0119258618224502\\
428	0.0119240185054655\\
429	0.011922140633935\\
430	0.0119202274726783\\
431	0.0119182782625387\\
432	0.0119162922207086\\
433	0.0119142685397912\\
434	0.0119122063852172\\
435	0.0119101048935508\\
436	0.0119079631706583\\
437	0.0119057802897224\\
438	0.0119035552889629\\
439	0.0119012871686918\\
440	0.011898974888498\\
441	0.011896617364291\\
442	0.0118942134647693\\
443	0.0118917620075053\\
444	0.0118892617547244\\
445	0.0118867114092115\\
446	0.0118841096121044\\
447	0.011881454949172\\
448	0.0118787459900734\\
449	0.0118759814522358\\
450	0.0118731608374308\\
451	0.0118702852378632\\
452	0.0118673477930897\\
453	0.0118643467299626\\
454	0.0118612801968073\\
455	0.011858146258142\\
456	0.0118549428894319\\
457	0.0118516679715537\\
458	0.0118483192849539\\
459	0.0118448945034086\\
460	0.0118413911870938\\
461	0.0118378067754983\\
462	0.0118341385798008\\
463	0.0118303837748577\\
464	0.0118265393909793\\
465	0.0118226023060342\\
466	0.0118185692391883\\
467	0.0118144367487465\\
468	0.0118102012351429\\
469	0.01180585892488\\
470	0.0118014056451756\\
471	0.0117968370573102\\
472	0.011792148653276\\
473	0.0117873356601239\\
474	0.0117823930201877\\
475	0.0117773153697698\\
476	0.0117720970162177\\
477	0.0117667319140896\\
478	0.011761213642902\\
479	0.0117555353867836\\
480	0.0117496899160052\\
481	0.0117436695365881\\
482	0.0117374663740204\\
483	0.0117310727138272\\
484	0.0117244818640581\\
485	0.0117176807134393\\
486	0.0117106650736784\\
487	0.0117034291790748\\
488	0.0116959369600259\\
489	0.0116881740726698\\
490	0.0116801251570261\\
491	0.0116717738069703\\
492	0.0116631023328707\\
493	0.0116540916358305\\
494	0.0116447210641159\\
495	0.0116349682571253\\
496	0.0116248089702437\\
497	0.011614216878034\\
498	0.0116031633521892\\
499	0.0115916172107801\\
500	0.0115795444347776\\
501	0.0115669078443119\\
502	0.0115536667262967\\
503	0.0115397763968374\\
504	0.0115251877077862\\
505	0.0115098464716553\\
506	0.011493692798663\\
507	0.0114766601459512\\
508	0.011458673693352\\
509	0.0114396495179366\\
510	0.0114194931741318\\
511	0.0113981021064543\\
512	0.0113753435529281\\
513	0.0113510298642051\\
514	0.0113248481248743\\
515	0.0112826994026206\\
516	0.0112326650254009\\
517	0.0111798074843413\\
518	0.011123210112933\\
519	0.0109825812882413\\
520	0.0108299743020159\\
521	0.0106745698670279\\
522	0.0105172459195764\\
523	0.0104744672804384\\
524	0.0104361482762094\\
525	0.0104031045229496\\
526	0.0103774141853245\\
527	0.0103525128845029\\
528	0.0103282500875483\\
529	0.0103049500232545\\
530	0.010281860072798\\
531	0.0102584265610034\\
532	0.0102345853612172\\
533	0.0102102885372687\\
534	0.0101854675139607\\
535	0.0101600974623455\\
536	0.0101341575278408\\
537	0.010107629058335\\
538	0.0100804944109962\\
539	0.0100527338341062\\
540	0.0100240505803467\\
541	0.0099946343130577\\
542	0.00996455191135411\\
543	0.00993204582059295\\
544	0.00989954848574274\\
545	0.00986799151672372\\
546	0.00983787325018791\\
547	0.00980723263760193\\
548	0.00977603928561901\\
549	0.00974423691066886\\
550	0.00971171393718171\\
551	0.00967844964677866\\
552	0.00964442416272868\\
553	0.00960961707724233\\
554	0.00957400721930803\\
555	0.00953757253722866\\
556	0.00949991012500304\\
557	0.00946102814078937\\
558	0.00941785857696371\\
559	0.00937663850101491\\
560	0.00933985709546176\\
561	0.00930246673014417\\
562	0.00926431963294322\\
563	0.00922539226211695\\
564	0.00918566486579048\\
565	0.00914511719909901\\
566	0.00910372811490353\\
567	0.00906147522229277\\
568	0.00897133540686007\\
569	0.00859185485320159\\
570	0.0082576416188635\\
571	0.00819106540855579\\
572	0.00812336226305488\\
573	0.00805450378548458\\
574	0.0079844630468339\\
575	0.00791321205096173\\
576	0.0078407216729221\\
577	0.00776696160171876\\
578	0.00769190028175292\\
579	0.00761550485220227\\
580	0.00753774108175452\\
581	0.00745857329076583\\
582	0.00737796423854308\\
583	0.00729587491554804\\
584	0.00721226408110735\\
585	0.00712708712854453\\
586	0.00704029318686139\\
587	0.00695181762108359\\
588	0.00686156256306143\\
589	0.00676934637110726\\
590	0.00667477256501155\\
591	0.00657713355197374\\
592	0.00647481662499991\\
593	0.00636364764633835\\
594	0.00623271716130668\\
595	0.00605340994148281\\
596	0.00575058001197164\\
597	0.00512683753504545\\
598	0.00366374385960312\\
599	0\\
600	0\\
};
\addplot [color=mycolor5,solid,forget plot]
  table[row sep=crcr]{%
1	0.0120472198053511\\
2	0.012047218586242\\
3	0.0120472173458397\\
4	0.0120472160837709\\
5	0.012047214799656\\
6	0.0120472134931088\\
7	0.0120472121637361\\
8	0.0120472108111379\\
9	0.0120472094349072\\
10	0.0120472080346297\\
11	0.012047206609884\\
12	0.0120472051602411\\
13	0.0120472036852646\\
14	0.0120472021845103\\
15	0.0120472006575265\\
16	0.012047199103853\\
17	0.0120471975230221\\
18	0.0120471959145574\\
19	0.0120471942779744\\
20	0.01204719261278\\
21	0.0120471909184724\\
22	0.0120471891945409\\
23	0.012047187440466\\
24	0.0120471856557189\\
25	0.0120471838397615\\
26	0.0120471819920463\\
27	0.0120471801120162\\
28	0.0120471781991042\\
29	0.0120471762527332\\
30	0.0120471742723163\\
31	0.0120471722572559\\
32	0.012047170206944\\
33	0.012047168120762\\
34	0.0120471659980803\\
35	0.0120471638382582\\
36	0.0120471616406438\\
37	0.0120471594045736\\
38	0.0120471571293724\\
39	0.0120471548143533\\
40	0.0120471524588171\\
41	0.0120471500620524\\
42	0.0120471476233351\\
43	0.0120471451419285\\
44	0.0120471426170829\\
45	0.0120471400480352\\
46	0.0120471374340091\\
47	0.0120471347742145\\
48	0.0120471320678472\\
49	0.0120471293140892\\
50	0.0120471265121077\\
51	0.0120471236610554\\
52	0.0120471207600699\\
53	0.0120471178082739\\
54	0.0120471148047742\\
55	0.0120471117486622\\
56	0.0120471086390131\\
57	0.0120471054748856\\
58	0.0120471022553221\\
59	0.012047098979348\\
60	0.0120470956459714\\
61	0.0120470922541828\\
62	0.0120470888029552\\
63	0.0120470852912431\\
64	0.0120470817179827\\
65	0.0120470780820914\\
66	0.0120470743824674\\
67	0.0120470706179896\\
68	0.0120470667875167\\
69	0.0120470628898877\\
70	0.0120470589239207\\
71	0.012047054888413\\
72	0.0120470507821407\\
73	0.0120470466038582\\
74	0.0120470423522978\\
75	0.0120470380261695\\
76	0.0120470336241601\\
77	0.0120470291449337\\
78	0.0120470245871303\\
79	0.0120470199493659\\
80	0.0120470152302321\\
81	0.0120470104282955\\
82	0.0120470055420971\\
83	0.0120470005701523\\
84	0.0120469955109501\\
85	0.0120469903629524\\
86	0.0120469851245942\\
87	0.0120469797942825\\
88	0.0120469743703962\\
89	0.0120469688512851\\
90	0.01204696323527\\
91	0.0120469575206416\\
92	0.0120469517056604\\
93	0.0120469457885559\\
94	0.012046939767526\\
95	0.0120469336407368\\
96	0.0120469274063215\\
97	0.0120469210623801\\
98	0.0120469146069788\\
99	0.0120469080381494\\
100	0.0120469013538884\\
101	0.0120468945521568\\
102	0.012046887630879\\
103	0.0120468805879426\\
104	0.0120468734211972\\
105	0.0120468661284541\\
106	0.0120468587074856\\
107	0.0120468511560238\\
108	0.0120468434717607\\
109	0.0120468356523464\\
110	0.0120468276953893\\
111	0.012046819598455\\
112	0.012046811359065\\
113	0.0120468029746967\\
114	0.012046794442782\\
115	0.0120467857607068\\
116	0.0120467769258098\\
117	0.0120467679353819\\
118	0.0120467587866653\\
119	0.0120467494768524\\
120	0.0120467400030849\\
121	0.012046730362453\\
122	0.0120467205519943\\
123	0.012046710568693\\
124	0.0120467004094784\\
125	0.0120466900712244\\
126	0.0120466795507483\\
127	0.0120466688448096\\
128	0.0120466579501087\\
129	0.0120466468632865\\
130	0.0120466355809223\\
131	0.0120466240995334\\
132	0.0120466124155735\\
133	0.0120466005254315\\
134	0.0120465884254305\\
135	0.0120465761118261\\
136	0.0120465635808057\\
137	0.0120465508284863\\
138	0.012046537850914\\
139	0.0120465246440622\\
140	0.0120465112038301\\
141	0.0120464975260414\\
142	0.0120464836064429\\
143	0.0120464694407027\\
144	0.0120464550244089\\
145	0.0120464403530679\\
146	0.012046425422103\\
147	0.0120464102268524\\
148	0.0120463947625677\\
149	0.0120463790244124\\
150	0.0120463630074598\\
151	0.0120463467066914\\
152	0.012046330116995\\
153	0.0120463132331632\\
154	0.0120462960498911\\
155	0.0120462785617744\\
156	0.0120462607633077\\
157	0.0120462426488828\\
158	0.0120462242127857\\
159	0.0120462054491958\\
160	0.012046186352183\\
161	0.012046166915706\\
162	0.01204614713361\\
163	0.0120461269996247\\
164	0.0120461065073624\\
165	0.0120460856503152\\
166	0.0120460644218537\\
167	0.0120460428152241\\
168	0.0120460208235462\\
169	0.0120459984398118\\
170	0.0120459756568817\\
171	0.012045952467484\\
172	0.012045928864212\\
173	0.0120459048395219\\
174	0.0120458803857309\\
175	0.0120458554950147\\
176	0.0120458301594061\\
177	0.0120458043707924\\
178	0.0120457781209137\\
179	0.012045751401361\\
180	0.0120457242035742\\
181	0.01204569651884\\
182	0.0120456683382905\\
183	0.0120456396529009\\
184	0.0120456104534878\\
185	0.0120455807307074\\
186	0.0120455504750532\\
187	0.0120455196768545\\
188	0.0120454883262736\\
189	0.0120454564133039\\
190	0.0120454239277672\\
191	0.0120453908593108\\
192	0.012045357197404\\
193	0.0120453229313345\\
194	0.0120452880502034\\
195	0.01204525254292\\
196	0.0120452163981968\\
197	0.0120451796045461\\
198	0.0120451421502713\\
199	0.0120451040234629\\
200	0.0120450652119942\\
201	0.0120450257035177\\
202	0.0120449854854608\\
203	0.0120449445450216\\
204	0.0120449028691647\\
205	0.0120448604446167\\
206	0.0120448172578622\\
207	0.0120447732951389\\
208	0.0120447285424332\\
209	0.0120446829854758\\
210	0.0120446366097365\\
211	0.0120445894004198\\
212	0.01204454134246\\
213	0.0120444924205162\\
214	0.0120444426189671\\
215	0.0120443919219064\\
216	0.0120443403131373\\
217	0.0120442877761674\\
218	0.0120442342942033\\
219	0.0120441798501456\\
220	0.0120441244265833\\
221	0.0120440680057882\\
222	0.0120440105697099\\
223	0.0120439520999699\\
224	0.0120438925778561\\
225	0.0120438319843176\\
226	0.0120437702999586\\
227	0.0120437075050332\\
228	0.0120436435794399\\
229	0.0120435785027158\\
230	0.0120435122540312\\
231	0.0120434448121842\\
232	0.0120433761555956\\
233	0.0120433062623031\\
234	0.0120432351099565\\
235	0.0120431626758128\\
236	0.012043088936731\\
237	0.0120430138691678\\
238	0.012042937449173\\
239	0.0120428596523851\\
240	0.0120427804540282\\
241	0.0120426998289075\\
242	0.0120426177514068\\
243	0.0120425341954854\\
244	0.0120424491346762\\
245	0.0120423625420832\\
246	0.012042274390381\\
247	0.0120421846518138\\
248	0.0120420932981953\\
249	0.0120420003009097\\
250	0.012041905630913\\
251	0.0120418092587352\\
252	0.0120417111544835\\
253	0.0120416112878463\\
254	0.0120415096280978\\
255	0.0120414061441044\\
256	0.0120413008043309\\
257	0.0120411935768484\\
258	0.0120410844293431\\
259	0.0120409733291258\\
260	0.012040860243142\\
261	0.0120407451379833\\
262	0.0120406279798987\\
263	0.012040508734807\\
264	0.0120403873683088\\
265	0.0120402638456985\\
266	0.0120401381319763\\
267	0.0120400101918581\\
268	0.0120398799897854\\
269	0.0120397474899317\\
270	0.0120396126562066\\
271	0.0120394754522559\\
272	0.0120393358414557\\
273	0.0120391937868992\\
274	0.0120390492513723\\
275	0.0120389021973127\\
276	0.0120387525867447\\
277	0.0120386003811919\\
278	0.0120384455416318\\
279	0.01203828802883\\
280	0.0120381278016652\\
281	0.0120379648152236\\
282	0.0120377990239095\\
283	0.0120376303814367\\
284	0.0120374588408205\\
285	0.0120372843543689\\
286	0.0120371068736742\\
287	0.0120369263496043\\
288	0.0120367427322945\\
289	0.0120365559711384\\
290	0.0120363660147796\\
291	0.0120361728111025\\
292	0.0120359763072244\\
293	0.0120357764494858\\
294	0.0120355731834423\\
295	0.0120353664538554\\
296	0.0120351562046838\\
297	0.0120349423790742\\
298	0.0120347249193527\\
299	0.0120345037670158\\
300	0.0120342788627211\\
301	0.0120340501462785\\
302	0.0120338175566411\\
303	0.012033581031896\\
304	0.012033340509255\\
305	0.0120330959250457\\
306	0.012032847214702\\
307	0.0120325943127546\\
308	0.0120323371528221\\
309	0.012032075667601\\
310	0.0120318097888566\\
311	0.0120315394474131\\
312	0.0120312645731437\\
313	0.0120309850949613\\
314	0.0120307009408081\\
315	0.0120304120376458\\
316	0.0120301183114449\\
317	0.0120298196871748\\
318	0.0120295160887931\\
319	0.0120292074392344\\
320	0.0120288936603999\\
321	0.0120285746731454\\
322	0.0120282503972702\\
323	0.012027920751505\\
324	0.0120275856534996\\
325	0.0120272450198105\\
326	0.0120268987658871\\
327	0.0120265468060588\\
328	0.0120261890535205\\
329	0.0120258254203179\\
330	0.012025455817332\\
331	0.0120250801542631\\
332	0.0120246983396139\\
333	0.0120243102806713\\
334	0.0120239158834881\\
335	0.012023515052863\\
336	0.0120231076923196\\
337	0.0120226937040842\\
338	0.012022272989062\\
339	0.0120218454468127\\
340	0.0120214109755229\\
341	0.0120209694719786\\
342	0.0120205208315345\\
343	0.0120200649480815\\
344	0.0120196017140126\\
345	0.0120191310201861\\
346	0.0120186527558858\\
347	0.0120181668087795\\
348	0.012017673064874\\
349	0.0120171714084669\\
350	0.0120166617220958\\
351	0.0120161438864836\\
352	0.0120156177804806\\
353	0.0120150832810022\\
354	0.0120145402629637\\
355	0.0120139885992098\\
356	0.0120134281604413\\
357	0.0120128588151362\\
358	0.0120122804294677\\
359	0.0120116928672164\\
360	0.0120110959896799\\
361	0.0120104896555765\\
362	0.0120098737209457\\
363	0.0120092480390451\\
364	0.0120086124602429\\
365	0.0120079668319086\\
366	0.0120073109983002\\
367	0.0120066448004511\\
368	0.0120059680760553\\
369	0.0120052806593548\\
370	0.0120045823810283\\
371	0.0120038730680846\\
372	0.0120031525437631\\
373	0.0120024206274433\\
374	0.0120016771345673\\
375	0.0120009218765798\\
376	0.0120001546608895\\
377	0.0119993752908586\\
378	0.0119985835658262\\
379	0.011997779281172\\
380	0.0119969622284209\\
381	0.0119961321953737\\
382	0.0119952889661968\\
383	0.0119944323212464\\
384	0.0119935620359645\\
385	0.0119926778770069\\
386	0.0119917795909458\\
387	0.0119908624777247\\
388	0.0119899228214573\\
389	0.0119889688611929\\
390	0.0119880002896239\\
391	0.0119870167795216\\
392	0.0119860179826227\\
393	0.0119850035282624\\
394	0.0119839730213251\\
395	0.0119829260361939\\
396	0.0119818621230247\\
397	0.0119807808121115\\
398	0.0119796816122004\\
399	0.0119785640148432\\
400	0.0119774275006134\\
401	0.0119762715470793\\
402	0.0119750956427804\\
403	0.0119738993289339\\
404	0.0119726822097602\\
405	0.0119714438794811\\
406	0.011970183920554\\
407	0.0119689018950762\\
408	0.0119675973484922\\
409	0.0119662698082343\\
410	0.0119649187822249\\
411	0.0119635437572433\\
412	0.0119621441971765\\
413	0.0119607195412046\\
414	0.0119592692019899\\
415	0.0119577925637893\\
416	0.0119562889794543\\
417	0.011954757760998\\
418	0.0119531981416374\\
419	0.011951609340498\\
420	0.0119499905309098\\
421	0.0119483408280381\\
422	0.0119466592767789\\
423	0.0119449448252672\\
424	0.0119431962497694\\
425	0.0119414119050794\\
426	0.011939544330476\\
427	0.0119375833170973\\
428	0.0119355886718335\\
429	0.0119335597978319\\
430	0.0119314960902991\\
431	0.0119293969288426\\
432	0.0119272616306781\\
433	0.0119250895084772\\
434	0.0119228799029329\\
435	0.0119206321448455\\
436	0.011918345556092\\
437	0.0119160194517908\\
438	0.0119136531452393\\
439	0.0119112459549549\\
440	0.0119087971915859\\
441	0.0119063061586946\\
442	0.0119037721636091\\
443	0.0119011945212235\\
444	0.0118985725583592\\
445	0.0118959056183421\\
446	0.011893193064395\\
447	0.011890434276953\\
448	0.0118876286282989\\
449	0.011884775377955\\
450	0.01188187329274\\
451	0.0118789359570394\\
452	0.0118761187290379\\
453	0.0118732441583786\\
454	0.0118703106824494\\
455	0.0118673166788937\\
456	0.0118642604514183\\
457	0.0118611402226762\\
458	0.0118579541274538\\
459	0.0118547002077002\\
460	0.0118513764098386\\
461	0.0118479805695112\\
462	0.0118445104058764\\
463	0.0118409635122189\\
464	0.0118373373462516\\
465	0.0118336292212986\\
466	0.0118298363038166\\
467	0.0118259556373484\\
468	0.0118219842676887\\
469	0.0118179197503552\\
470	0.0118137621140052\\
471	0.01180950425274\\
472	0.011805139893548\\
473	0.0118006649908582\\
474	0.0117960752680153\\
475	0.0117913662006552\\
476	0.0117865329993363\\
477	0.0117815705898811\\
478	0.0117764735924932\\
479	0.0117712362997029\\
480	0.0117658526539322\\
481	0.0117603162295635\\
482	0.0117546202164943\\
483	0.0117487574127413\\
484	0.0117427202157511\\
485	0.0117365009825856\\
486	0.0117300922678666\\
487	0.0117234860181015\\
488	0.0117166705592115\\
489	0.0117096430946846\\
490	0.0117023924143927\\
491	0.0116948861812014\\
492	0.011687110203106\\
493	0.0116790493082437\\
494	0.0116706873039627\\
495	0.0116620067521509\\
496	0.0116529888513808\\
497	0.0116436132998294\\
498	0.0116338581469455\\
499	0.0116236996237757\\
500	0.0116131119545055\\
501	0.0116020671638132\\
502	0.0115905348320248\\
503	0.0115784818297717\\
504	0.0115658720170299\\
505	0.0115526658996702\\
506	0.0115388202319084\\
507	0.0115242875535829\\
508	0.0115090156703155\\
509	0.0114929470364955\\
510	0.0114760180503333\\
511	0.011458157917053\\
512	0.0114392873377584\\
513	0.0114193168259383\\
514	0.0113981416489737\\
515	0.0113756586727428\\
516	0.0113517080155585\\
517	0.011326059144954\\
518	0.0112984078901283\\
519	0.0112508252791484\\
520	0.01119879958361\\
521	0.0111437690299501\\
522	0.0110844158859491\\
523	0.0109311406455557\\
524	0.0107748825979114\\
525	0.0106155354733288\\
526	0.0104549496916452\\
527	0.0104095204059356\\
528	0.0103686165457181\\
529	0.0103328767538324\\
530	0.0103041717760264\\
531	0.010276904865161\\
532	0.0102503142032637\\
533	0.0102241546718795\\
534	0.010198960119134\\
535	0.0101734677595411\\
536	0.0101475515969584\\
537	0.0101211494948978\\
538	0.0100942094563074\\
539	0.0100666774092843\\
540	0.0100385289808489\\
541	0.0100097403033765\\
542	0.00998027217526259\\
543	0.00994971931279848\\
544	0.0099185772867922\\
545	0.00988611173148685\\
546	0.00985198060028316\\
547	0.0098178438786619\\
548	0.00978520062244087\\
549	0.00975339491219885\\
550	0.00972099753197099\\
551	0.0096879072496547\\
552	0.00965406861345961\\
553	0.00961945651714474\\
554	0.00958404815898549\\
555	0.00954782144334343\\
556	0.00951075368679787\\
557	0.00947281788527594\\
558	0.00943323273142403\\
559	0.0093918204066566\\
560	0.00934731903904993\\
561	0.00930667756038113\\
562	0.00926850736250741\\
563	0.00922964619803856\\
564	0.00918999094494944\\
565	0.00914951749255589\\
566	0.0091082046432926\\
567	0.00906603063041869\\
568	0.00902297257848852\\
569	0.00893326721015658\\
570	0.00859168342714514\\
571	0.00819114043826589\\
572	0.0081233624115439\\
573	0.00805450381049785\\
574	0.0079844630590578\\
575	0.00791321205721573\\
576	0.00784072167600322\\
577	0.00776696160314408\\
578	0.00769190028236148\\
579	0.00761550485243745\\
580	0.00753774108183466\\
581	0.007458573290789\\
582	0.00737796423854843\\
583	0.00729587491554892\\
584	0.00721226408110744\\
585	0.00712708712854452\\
586	0.00704029318686139\\
587	0.0069518176210836\\
588	0.00686156256306143\\
589	0.00676934637110726\\
590	0.00667477256501155\\
591	0.00657713355197374\\
592	0.00647481662499991\\
593	0.00636364764633835\\
594	0.00623271716130668\\
595	0.00605340994148281\\
596	0.00575058001197164\\
597	0.00512683753504545\\
598	0.00366374385960312\\
599	0\\
600	0\\
};
\addplot [color=mycolor6,solid,forget plot]
  table[row sep=crcr]{%
1	0.012076866751432\\
2	0.012076864946228\\
3	0.0120768631100084\\
4	0.0120768612422401\\
5	0.0120768593423807\\
6	0.0120768574098786\\
7	0.0120768554441726\\
8	0.012076853444692\\
9	0.0120768514108561\\
10	0.0120768493420744\\
11	0.012076847237746\\
12	0.0120768450972601\\
13	0.012076842919995\\
14	0.0120768407053184\\
15	0.0120768384525873\\
16	0.0120768361611475\\
17	0.0120768338303335\\
18	0.0120768314594685\\
19	0.0120768290478642\\
20	0.01207682659482\\
21	0.0120768240996237\\
22	0.0120768215615508\\
23	0.0120768189798641\\
24	0.0120768163538139\\
25	0.0120768136826377\\
26	0.0120768109655598\\
27	0.0120768082017911\\
28	0.0120768053905291\\
29	0.0120768025309573\\
30	0.0120767996222455\\
31	0.0120767966635488\\
32	0.0120767936540082\\
33	0.0120767905927497\\
34	0.0120767874788843\\
35	0.0120767843115078\\
36	0.0120767810897005\\
37	0.0120767778125267\\
38	0.0120767744790349\\
39	0.012076771088257\\
40	0.0120767676392084\\
41	0.0120767641308876\\
42	0.0120767605622759\\
43	0.012076756932337\\
44	0.0120767532400169\\
45	0.0120767494842434\\
46	0.0120767456639261\\
47	0.0120767417779556\\
48	0.0120767378252036\\
49	0.0120767338045224\\
50	0.0120767297147445\\
51	0.0120767255546825\\
52	0.0120767213231285\\
53	0.012076717018854\\
54	0.0120767126406092\\
55	0.0120767081871229\\
56	0.0120767036571022\\
57	0.0120766990492318\\
58	0.012076694362174\\
59	0.012076689594568\\
60	0.0120766847450298\\
61	0.0120766798121513\\
62	0.0120766747945007\\
63	0.0120766696906212\\
64	0.0120766644990311\\
65	0.0120766592182235\\
66	0.0120766538466653\\
67	0.0120766483827973\\
68	0.0120766428250334\\
69	0.0120766371717603\\
70	0.012076631421337\\
71	0.0120766255720944\\
72	0.0120766196223347\\
73	0.0120766135703309\\
74	0.0120766074143265\\
75	0.0120766011525345\\
76	0.0120765947831377\\
77	0.0120765883042874\\
78	0.0120765817141031\\
79	0.0120765750106723\\
80	0.0120765681920494\\
81	0.0120765612562555\\
82	0.0120765542012777\\
83	0.0120765470250685\\
84	0.0120765397255455\\
85	0.0120765323005901\\
86	0.0120765247480476\\
87	0.0120765170657263\\
88	0.0120765092513968\\
89	0.0120765013027913\\
90	0.0120764932176033\\
91	0.0120764849934865\\
92	0.0120764766280542\\
93	0.0120764681188789\\
94	0.0120764594634912\\
95	0.0120764506593796\\
96	0.012076441703989\\
97	0.0120764325947205\\
98	0.0120764233289308\\
99	0.0120764139039309\\
100	0.0120764043169855\\
101	0.0120763945653125\\
102	0.0120763846460816\\
103	0.0120763745564142\\
104	0.0120763642933818\\
105	0.0120763538540057\\
106	0.0120763432352558\\
107	0.01207633243405\\
108	0.0120763214472528\\
109	0.0120763102716749\\
110	0.012076298904072\\
111	0.0120762873411437\\
112	0.0120762755795329\\
113	0.0120762636158243\\
114	0.0120762514465438\\
115	0.0120762390681574\\
116	0.0120762264770697\\
117	0.0120762136696235\\
118	0.012076200642098\\
119	0.0120761873907084\\
120	0.012076173911604\\
121	0.0120761602008674\\
122	0.0120761462545135\\
123	0.0120761320684879\\
124	0.0120761176386657\\
125	0.0120761029608505\\
126	0.0120760880307727\\
127	0.0120760728440884\\
128	0.0120760573963782\\
129	0.0120760416831454\\
130	0.0120760256998146\\
131	0.0120760094417307\\
132	0.012075992904157\\
133	0.0120759760822737\\
134	0.0120759589711764\\
135	0.0120759415658745\\
136	0.0120759238612895\\
137	0.0120759058522534\\
138	0.0120758875335067\\
139	0.0120758688996972\\
140	0.0120758499453774\\
141	0.0120758306650031\\
142	0.0120758110529315\\
143	0.012075791103419\\
144	0.0120757708106194\\
145	0.0120757501685813\\
146	0.0120757291712467\\
147	0.0120757078124482\\
148	0.0120756860859068\\
149	0.0120756639852299\\
150	0.0120756415039082\\
151	0.0120756186353139\\
152	0.0120755953726978\\
153	0.0120755717091865\\
154	0.0120755476377799\\
155	0.0120755231513482\\
156	0.012075498242629\\
157	0.0120754729042245\\
158	0.012075447128598\\
159	0.0120754209080708\\
160	0.0120753942348191\\
161	0.0120753671008704\\
162	0.0120753394980998\\
163	0.0120753114182267\\
164	0.0120752828528108\\
165	0.0120752537932484\\
166	0.0120752242307682\\
167	0.0120751941564275\\
168	0.0120751635611079\\
169	0.012075132435511\\
170	0.0120751007701542\\
171	0.0120750685553661\\
172	0.0120750357812821\\
173	0.0120750024378402\\
174	0.0120749685147758\\
175	0.0120749340016178\\
176	0.0120748988876841\\
177	0.0120748631620766\\
178	0.0120748268136778\\
179	0.0120747898311461\\
180	0.0120747522029123\\
181	0.012074713917176\\
182	0.0120746749619026\\
183	0.0120746353248205\\
184	0.0120745949934195\\
185	0.0120745539549491\\
186	0.0120745121964184\\
187	0.0120744697045965\\
188	0.0120744264660147\\
189	0.0120743824669689\\
190	0.012074337693525\\
191	0.0120742921315251\\
192	0.0120742457665956\\
193	0.0120741985841573\\
194	0.0120741505694352\\
195	0.0120741017074662\\
196	0.0120740519831008\\
197	0.0120740013810086\\
198	0.0120739498857166\\
199	0.012073897481487\\
200	0.0120738441523073\\
201	0.0120737898818858\\
202	0.0120737346536473\\
203	0.0120736784507279\\
204	0.0120736212559707\\
205	0.0120735630519208\\
206	0.0120735038208204\\
207	0.0120734435446043\\
208	0.0120733822048946\\
209	0.0120733197829958\\
210	0.0120732562598897\\
211	0.0120731916162307\\
212	0.0120731258323401\\
213	0.0120730588882014\\
214	0.012072990763455\\
215	0.0120729214373928\\
216	0.0120728508889531\\
217	0.0120727790967152\\
218	0.0120727060388944\\
219	0.012072631693336\\
220	0.0120725560375106\\
221	0.0120724790485085\\
222	0.0120724007030342\\
223	0.012072320977401\\
224	0.012072239847526\\
225	0.012072157288924\\
226	0.0120720732767029\\
227	0.0120719877855578\\
228	0.0120719007897658\\
229	0.0120718122631808\\
230	0.0120717221792279\\
231	0.0120716305108983\\
232	0.0120715372307442\\
233	0.0120714423108734\\
234	0.012071345722944\\
235	0.0120712474381596\\
236	0.012071147427264\\
237	0.0120710456605363\\
238	0.0120709421077857\\
239	0.0120708367383467\\
240	0.0120707295210742\\
241	0.0120706204243385\\
242	0.0120705094160205\\
243	0.0120703964635068\\
244	0.0120702815336848\\
245	0.0120701645929382\\
246	0.0120700456071415\\
247	0.0120699245416553\\
248	0.0120698013613215\\
249	0.0120696760304578\\
250	0.0120695485128524\\
251	0.0120694187717586\\
252	0.0120692867698889\\
253	0.0120691524694088\\
254	0.0120690158319308\\
255	0.012068876818507\\
256	0.012068735389622\\
257	0.012068591505185\\
258	0.0120684451245209\\
259	0.0120682962063611\\
260	0.0120681447088335\\
261	0.012067990589451\\
262	0.0120678338050993\\
263	0.0120676743120241\\
264	0.0120675120658159\\
265	0.012067347021395\\
266	0.0120671791329933\\
267	0.0120670083541369\\
268	0.012066834637625\\
269	0.0120666579355091\\
270	0.0120664781990698\\
271	0.0120662953787928\\
272	0.0120661094243431\\
273	0.0120659202845386\\
274	0.0120657279073227\\
275	0.0120655322397366\\
276	0.0120653332278923\\
277	0.0120651308169491\\
278	0.0120649249510965\\
279	0.0120647155735271\\
280	0.0120645026264644\\
281	0.01206428605126\\
282	0.012064065788384\\
283	0.0120638417774153\\
284	0.0120636139570311\\
285	0.0120633822649979\\
286	0.0120631466381613\\
287	0.0120629070124367\\
288	0.0120626633227997\\
289	0.012062415503277\\
290	0.0120621634869367\\
291	0.0120619072058799\\
292	0.012061646591231\\
293	0.0120613815731299\\
294	0.0120611120807228\\
295	0.0120608380421543\\
296	0.0120605593845593\\
297	0.0120602760340553\\
298	0.0120599879157352\\
299	0.0120596949536598\\
300	0.0120593970708514\\
301	0.0120590941892876\\
302	0.0120587862298951\\
303	0.0120584731125445\\
304	0.0120581547560456\\
305	0.0120578310781425\\
306	0.0120575019955104\\
307	0.0120571674237524\\
308	0.0120568272773966\\
309	0.0120564814698954\\
310	0.0120561299136236\\
311	0.0120557725198792\\
312	0.0120554091988837\\
313	0.0120550398597848\\
314	0.0120546644106587\\
315	0.0120542827585149\\
316	0.0120538948093013\\
317	0.0120535004679109\\
318	0.0120530996381906\\
319	0.0120526922229503\\
320	0.0120522781239747\\
321	0.0120518572420367\\
322	0.0120514294769123\\
323	0.0120509947273977\\
324	0.0120505528913292\\
325	0.0120501038656046\\
326	0.0120496475462076\\
327	0.0120491838282352\\
328	0.0120487126059271\\
329	0.0120482337726995\\
330	0.0120477472211812\\
331	0.0120472528432538\\
332	0.0120467505300961\\
333	0.0120462401722314\\
334	0.0120457216595811\\
335	0.0120451948815214\\
336	0.0120446597269462\\
337	0.0120441160843349\\
338	0.0120435638418263\\
339	0.0120430028872988\\
340	0.0120424331084575\\
341	0.0120418543929279\\
342	0.0120412666283579\\
343	0.0120406697025282\\
344	0.0120400635034707\\
345	0.0120394479195972\\
346	0.0120388228398373\\
347	0.0120381881537882\\
348	0.0120375437518744\\
349	0.0120368895255208\\
350	0.0120362253673385\\
351	0.0120355511713236\\
352	0.0120348668330712\\
353	0.0120341722500051\\
354	0.0120334673216228\\
355	0.0120327519497585\\
356	0.012032026038864\\
357	0.0120312894963081\\
358	0.0120305422326963\\
359	0.0120297841622109\\
360	0.0120290152029721\\
361	0.0120282352774221\\
362	0.0120274443127307\\
363	0.0120266422412243\\
364	0.0120258290008383\\
365	0.0120250045355921\\
366	0.0120241687960872\\
367	0.0120233217400265\\
368	0.0120224633327547\\
369	0.0120215935478165\\
370	0.0120207123675309\\
371	0.0120198197835769\\
372	0.0120189157975863\\
373	0.0120180004217369\\
374	0.012017073679337\\
375	0.0120161356053925\\
376	0.0120151862471411\\
377	0.0120142256645391\\
378	0.0120132539306772\\
379	0.0120122711321011\\
380	0.0120112773690004\\
381	0.0120102727552214\\
382	0.0120092574180418\\
383	0.0120082314976216\\
384	0.0120071951460218\\
385	0.0120061485257101\\
386	0.0120050918078144\\
387	0.0120040251719482\\
388	0.0120029487940964\\
389	0.0120018628303055\\
390	0.0120007674303481\\
391	0.0119996627306729\\
392	0.0119985488451377\\
393	0.011997425852885\\
394	0.0119962937825089\\
395	0.0119951525912521\\
396	0.0119940021382087\\
397	0.0119928421501233\\
398	0.0119916721777353\\
399	0.0119904915401138\\
400	0.0119892992517129\\
401	0.0119880939164827\\
402	0.0119868735446339\\
403	0.0119856352281617\\
404	0.0119843764414293\\
405	0.0119830968517885\\
406	0.0119817961207085\\
407	0.0119804739037199\\
408	0.0119791298503657\\
409	0.0119777636041755\\
410	0.0119763748026687\\
411	0.0119749630773915\\
412	0.0119735280539953\\
413	0.0119720693523654\\
414	0.0119705865868082\\
415	0.0119690793663086\\
416	0.011967547294871\\
417	0.0119659899719639\\
418	0.0119644069930839\\
419	0.0119627979503317\\
420	0.0119611624331332\\
421	0.011959500028912\\
422	0.011957810323254\\
423	0.0119560928982393\\
424	0.0119543473252919\\
425	0.0119525731433739\\
426	0.0119507598264658\\
427	0.0119489048710412\\
428	0.0119470210362922\\
429	0.0119451078047242\\
430	0.0119431646339061\\
431	0.0119411909523468\\
432	0.0119391861479556\\
433	0.011937149574883\\
434	0.0119350805554316\\
435	0.0119329783688772\\
436	0.0119308422467828\\
437	0.0119286713679691\\
438	0.0119264648533749\\
439	0.011924221760646\\
440	0.0119219410743917\\
441	0.0119196216981448\\
442	0.0119172624472902\\
443	0.0119148620398431\\
444	0.011912419086394\\
445	0.0119099320790559\\
446	0.0119073993788302\\
447	0.0119048191992663\\
448	0.0119021895787571\\
449	0.011899508313721\\
450	0.0118967727505276\\
451	0.0118939666686319\\
452	0.0118909639850025\\
453	0.0118879074422213\\
454	0.0118847955020424\\
455	0.0118816267045609\\
456	0.0118783998005142\\
457	0.0118751135389869\\
458	0.0118717666282401\\
459	0.0118683576837656\\
460	0.0118648852376284\\
461	0.0118613479889332\\
462	0.0118577444982513\\
463	0.0118540732931081\\
464	0.0118503328656751\\
465	0.01184652167443\\
466	0.0118426381373628\\
467	0.011838680604638\\
468	0.0118346472641248\\
469	0.0118305358263164\\
470	0.0118263424675444\\
471	0.0118222282569492\\
472	0.0118181027549173\\
473	0.0118138810706521\\
474	0.0118095596161384\\
475	0.0118051345909209\\
476	0.0118006019615102\\
477	0.0117959574684708\\
478	0.0117911966009365\\
479	0.0117863145756489\\
480	0.011781306318024\\
481	0.0117761664405187\\
482	0.0117708892267801\\
483	0.0117654686386757\\
484	0.0117598984188618\\
485	0.0117541725345987\\
486	0.0117482869273429\\
487	0.0117422339262941\\
488	0.0117360003946928\\
489	0.0117295791986597\\
490	0.0117229616374423\\
491	0.0117161368740211\\
492	0.0117091026909894\\
493	0.0117018461542213\\
494	0.0116943361423598\\
495	0.0116865586014639\\
496	0.0116784985030909\\
497	0.011670139874852\\
498	0.0116614655694812\\
499	0.0116524571751639\\
500	0.0116430947865314\\
501	0.0116333566231288\\
502	0.011623219448945\\
503	0.0116126581133254\\
504	0.0116016453560485\\
505	0.0115901515885614\\
506	0.0115781446466673\\
507	0.0115655895123386\\
508	0.0115524479980813\\
509	0.0115386783866561\\
510	0.0115242350110054\\
511	0.0115090677645258\\
512	0.011493121532545\\
513	0.0114763355745899\\
514	0.0114586430323312\\
515	0.0114399690933532\\
516	0.0114202280920987\\
517	0.0113993226981617\\
518	0.0113771493036721\\
519	0.0113535717316111\\
520	0.0113284178717591\\
521	0.0113014744125244\\
522	0.0112723452388385\\
523	0.0112210253920977\\
524	0.011167116116309\\
525	0.0111100574647757\\
526	0.011047856114441\\
527	0.0108911870199347\\
528	0.0107314148744869\\
529	0.0105683625241118\\
530	0.010402582700154\\
531	0.0103433714657556\\
532	0.0102987208813198\\
533	0.0102594596397759\\
534	0.0102264173729237\\
535	0.0101964187124358\\
536	0.0101671495870776\\
537	0.0101384129584845\\
538	0.0101101347405005\\
539	0.0100823555888032\\
540	0.0100541461508269\\
541	0.0100254329653264\\
542	0.00999616110359175\\
543	0.00996627402406363\\
544	0.00993571988088713\\
545	0.00990431666479844\\
546	0.00987196571049124\\
547	0.00983898715188358\\
548	0.00980423349694136\\
549	0.00976841201801745\\
550	0.00973260294245857\\
551	0.00969848078071835\\
552	0.00966466365009313\\
553	0.00963019578557018\\
554	0.0095949827847948\\
555	0.0095589666279459\\
556	0.00952211973215971\\
557	0.00948441614080153\\
558	0.00944583115156459\\
559	0.00940611033237629\\
560	0.00936494180771247\\
561	0.00932091705477405\\
562	0.00927537531431877\\
563	0.00923449360961306\\
564	0.00919483593703645\\
565	0.00915443320240743\\
566	0.00911319714743514\\
567	0.00907110057975673\\
568	0.00902811986267383\\
569	0.00898423067887847\\
570	0.00890021985209273\\
571	0.00862745807166071\\
572	0.00812595060641661\\
573	0.00805450627922674\\
574	0.00798446323608824\\
575	0.00791321213868608\\
576	0.00784072171843002\\
577	0.00776696162475285\\
578	0.00769190029275072\\
579	0.00761550485706315\\
580	0.00753774108370465\\
581	0.00745857329145751\\
582	0.00737796423875186\\
583	0.00729587491559849\\
584	0.00721226408111606\\
585	0.00712708712854533\\
586	0.00704029318686139\\
587	0.00695181762108359\\
588	0.00686156256306143\\
589	0.00676934637110726\\
590	0.00667477256501156\\
591	0.00657713355197374\\
592	0.00647481662499991\\
593	0.00636364764633835\\
594	0.00623271716130669\\
595	0.00605340994148281\\
596	0.00575058001197164\\
597	0.00512683753504546\\
598	0.00366374385960312\\
599	0\\
600	0\\
};
\addplot [color=mycolor7,solid,forget plot]
  table[row sep=crcr]{%
1	0.0121672499298442\\
2	0.0121672467288244\\
3	0.0121672434734494\\
4	0.0121672401627974\\
5	0.0121672367959313\\
6	0.012167233371898\\
7	0.0121672298897283\\
8	0.0121672263484366\\
9	0.0121672227470206\\
10	0.0121672190844611\\
11	0.0121672153597217\\
12	0.0121672115717485\\
13	0.0121672077194695\\
14	0.0121672038017951\\
15	0.0121671998176168\\
16	0.0121671957658075\\
17	0.0121671916452213\\
18	0.0121671874546926\\
19	0.0121671831930363\\
20	0.0121671788590472\\
21	0.0121671744514998\\
22	0.0121671699691477\\
23	0.0121671654107239\\
24	0.0121671607749395\\
25	0.0121671560604841\\
26	0.012167151266025\\
27	0.0121671463902074\\
28	0.0121671414316531\\
29	0.0121671363889609\\
30	0.0121671312607061\\
31	0.0121671260454395\\
32	0.012167120741688\\
33	0.0121671153479532\\
34	0.0121671098627117\\
35	0.0121671042844141\\
36	0.0121670986114852\\
37	0.0121670928423231\\
38	0.0121670869752988\\
39	0.0121670810087559\\
40	0.01216707494101\\
41	0.0121670687703485\\
42	0.0121670624950297\\
43	0.0121670561132827\\
44	0.0121670496233066\\
45	0.0121670430232704\\
46	0.0121670363113119\\
47	0.0121670294855379\\
48	0.0121670225440231\\
49	0.0121670154848098\\
50	0.0121670083059073\\
51	0.0121670010052914\\
52	0.012166993580904\\
53	0.012166986030652\\
54	0.0121669783524075\\
55	0.0121669705440064\\
56	0.0121669626032483\\
57	0.012166954527896\\
58	0.0121669463156745\\
59	0.0121669379642704\\
60	0.0121669294713317\\
61	0.0121669208344665\\
62	0.0121669120512431\\
63	0.0121669031191886\\
64	0.0121668940357887\\
65	0.0121668847984868\\
66	0.0121668754046835\\
67	0.0121668658517354\\
68	0.012166856136955\\
69	0.0121668462576097\\
70	0.0121668362109208\\
71	0.0121668259940632\\
72	0.0121668156041641\\
73	0.0121668050383029\\
74	0.0121667942935097\\
75	0.0121667833667649\\
76	0.0121667722549983\\
77	0.012166760955088\\
78	0.0121667494638602\\
79	0.0121667377780875\\
80	0.0121667258944887\\
81	0.0121667138097274\\
82	0.0121667015204117\\
83	0.0121666890230924\\
84	0.0121666763142631\\
85	0.0121666633903583\\
86	0.012166650247753\\
87	0.0121666368827616\\
88	0.0121666232916368\\
89	0.0121666094705688\\
90	0.0121665954156838\\
91	0.0121665811230435\\
92	0.0121665665886437\\
93	0.0121665518084135\\
94	0.0121665367782137\\
95	0.0121665214938364\\
96	0.012166505951003\\
97	0.0121664901453638\\
98	0.0121664740724966\\
99	0.0121664577279053\\
100	0.0121664411070188\\
101	0.0121664242051899\\
102	0.0121664070176939\\
103	0.0121663895397274\\
104	0.0121663717664072\\
105	0.0121663536927684\\
106	0.0121663353137638\\
107	0.012166316624262\\
108	0.0121662976190465\\
109	0.0121662782928136\\
110	0.0121662586401719\\
111	0.0121662386556401\\
112	0.0121662183336458\\
113	0.0121661976685243\\
114	0.0121661766545165\\
115	0.012166155285768\\
116	0.0121661335563269\\
117	0.0121661114601427\\
118	0.0121660889910647\\
119	0.0121660661428399\\
120	0.0121660429091118\\
121	0.0121660192834186\\
122	0.0121659952591912\\
123	0.0121659708297518\\
124	0.0121659459883121\\
125	0.0121659207279713\\
126	0.0121658950417143\\
127	0.0121658689224099\\
128	0.0121658423628089\\
129	0.0121658153555424\\
130	0.0121657878931191\\
131	0.0121657599679242\\
132	0.0121657315722168\\
133	0.012165702698128\\
134	0.0121656733376586\\
135	0.0121656434826773\\
136	0.0121656131249184\\
137	0.0121655822559791\\
138	0.012165550867318\\
139	0.0121655189502521\\
140	0.0121654864959549\\
141	0.0121654534954536\\
142	0.0121654199396268\\
143	0.012165385819202\\
144	0.012165351124753\\
145	0.012165315846697\\
146	0.0121652799752923\\
147	0.0121652435006351\\
148	0.0121652064126569\\
149	0.0121651687011214\\
150	0.0121651303556214\\
151	0.0121650913655759\\
152	0.0121650517202267\\
153	0.0121650114086352\\
154	0.0121649704196785\\
155	0.0121649287420465\\
156	0.0121648863642376\\
157	0.0121648432745551\\
158	0.0121647994611032\\
159	0.0121647549117825\\
160	0.0121647096142858\\
161	0.0121646635560934\\
162	0.0121646167244682\\
163	0.0121645691064507\\
164	0.0121645206888533\\
165	0.0121644714582547\\
166	0.0121644214009938\\
167	0.0121643705031631\\
168	0.0121643187506019\\
169	0.0121642661288887\\
170	0.0121642126233333\\
171	0.0121641582189684\\
172	0.0121641029005403\\
173	0.0121640466524988\\
174	0.0121639894589868\\
175	0.0121639313038287\\
176	0.0121638721705174\\
177	0.0121638120422012\\
178	0.0121637509016689\\
179	0.0121636887313334\\
180	0.0121636255132146\\
181	0.0121635612289203\\
182	0.0121634958596253\\
183	0.0121634293860493\\
184	0.0121633617884322\\
185	0.0121632930465078\\
186	0.0121632231394751\\
187	0.0121631520459674\\
188	0.012163079744019\\
189	0.0121630062110298\\
190	0.0121629314237279\\
191	0.0121628553581313\\
192	0.0121627779895098\\
193	0.0121626992923517\\
194	0.0121626192403392\\
195	0.0121625378063422\\
196	0.0121624549624381\\
197	0.0121623706800006\\
198	0.0121622849301965\\
199	0.0121621976874229\\
200	0.0121621089257971\\
201	0.0121620186189936\\
202	0.0121619267402364\\
203	0.0121618332622918\\
204	0.012161738157461\\
205	0.0121616413975721\\
206	0.012161542953973\\
207	0.0121614427975231\\
208	0.0121613408985856\\
209	0.0121612372270193\\
210	0.0121611317521708\\
211	0.0121610244428658\\
212	0.012160915267401\\
213	0.0121608041935358\\
214	0.012160691188483\\
215	0.012160576218901\\
216	0.0121604592508843\\
217	0.0121603402499546\\
218	0.0121602191810522\\
219	0.0121600960085262\\
220	0.0121599706961255\\
221	0.0121598432069892\\
222	0.0121597135036372\\
223	0.0121595815479602\\
224	0.0121594473012101\\
225	0.0121593107239896\\
226	0.0121591717762425\\
227	0.0121590304172432\\
228	0.0121588866055861\\
229	0.0121587402991751\\
230	0.012158591455213\\
231	0.0121584400301901\\
232	0.0121582859798734\\
233	0.0121581292592954\\
234	0.0121579698227421\\
235	0.0121578076237419\\
236	0.0121576426150529\\
237	0.0121574747486517\\
238	0.0121573039757201\\
239	0.0121571302466333\\
240	0.0121569535109464\\
241	0.0121567737173816\\
242	0.0121565908138146\\
243	0.0121564047472612\\
244	0.0121562154638628\\
245	0.0121560229088727\\
246	0.0121558270266408\\
247	0.012155627760599\\
248	0.0121554250532457\\
249	0.01215521884613\\
250	0.0121550090798356\\
251	0.0121547956939642\\
252	0.0121545786271188\\
253	0.0121543578168857\\
254	0.0121541331998175\\
255	0.0121539047114142\\
256	0.0121536722861046\\
257	0.0121534358572276\\
258	0.0121531953570121\\
259	0.0121529507165576\\
260	0.0121527018658132\\
261	0.0121524487335575\\
262	0.0121521912473771\\
263	0.0121519293336456\\
264	0.0121516629175014\\
265	0.0121513919228266\\
266	0.0121511162722246\\
267	0.0121508358869984\\
268	0.0121505506871282\\
269	0.0121502605912499\\
270	0.0121499655166334\\
271	0.012149665379161\\
272	0.0121493600933067\\
273	0.0121490495721154\\
274	0.0121487337271832\\
275	0.0121484124686382\\
276	0.0121480857051216\\
277	0.0121477533437706\\
278	0.0121474152902003\\
279	0.0121470714484874\\
280	0.0121467217211516\\
281	0.0121463660091334\\
282	0.0121460042117708\\
283	0.0121456362267765\\
284	0.0121452619502152\\
285	0.0121448812764796\\
286	0.0121444940982674\\
287	0.0121441003065579\\
288	0.0121436997905878\\
289	0.0121432924378277\\
290	0.0121428781339582\\
291	0.0121424567628458\\
292	0.0121420282065189\\
293	0.012141592345144\\
294	0.0121411490570012\\
295	0.0121406982184603\\
296	0.0121402397039568\\
297	0.0121397733859678\\
298	0.0121392991349881\\
299	0.0121388168195068\\
300	0.012138326305983\\
301	0.0121378274588228\\
302	0.0121373201403563\\
303	0.0121368042108143\\
304	0.0121362795283057\\
305	0.012135745948796\\
306	0.0121352033260846\\
307	0.0121346515117843\\
308	0.0121340903553003\\
309	0.0121335197038105\\
310	0.0121329394022456\\
311	0.0121323492932712\\
312	0.0121317492172701\\
313	0.0121311390123254\\
314	0.0121305185142055\\
315	0.0121298875563495\\
316	0.0121292459698542\\
317	0.0121285935834632\\
318	0.0121279302235567\\
319	0.0121272557141434\\
320	0.0121265698768552\\
321	0.0121258725309424\\
322	0.0121251634932731\\
323	0.0121244425783338\\
324	0.0121237095982333\\
325	0.0121229643627103\\
326	0.0121222066791431\\
327	0.0121214363525644\\
328	0.0121206531856789\\
329	0.0121198569788863\\
330	0.0121190475303081\\
331	0.0121182246358208\\
332	0.0121173880890939\\
333	0.0121165376816341\\
334	0.0121156732028371\\
335	0.0121147944400461\\
336	0.0121139011786185\\
337	0.0121129932020016\\
338	0.0121120702918176\\
339	0.0121111322279596\\
340	0.0121101787886986\\
341	0.0121092097508031\\
342	0.0121082248896726\\
343	0.0121072239794858\\
344	0.0121062067933654\\
345	0.0121051731035605\\
346	0.012104122681649\\
347	0.0121030552987609\\
348	0.0121019707258257\\
349	0.0121008687338454\\
350	0.0120997490941957\\
351	0.0120986115789588\\
352	0.0120974559612898\\
353	0.0120962820158215\\
354	0.0120950895191102\\
355	0.0120938782501282\\
356	0.0120926479908055\\
357	0.0120913985266291\\
358	0.0120901296473035\\
359	0.0120888411474793\\
360	0.0120875328275592\\
361	0.012086204494588\\
362	0.0120848559632369\\
363	0.0120834870568932\\
364	0.0120820976088669\\
365	0.012080687463729\\
366	0.0120792564787962\\
367	0.0120778045257813\\
368	0.0120763314926286\\
369	0.0120748372855583\\
370	0.0120733218313474\\
371	0.0120717850798769\\
372	0.0120702270069822\\
373	0.0120686476176476\\
374	0.0120670469495931\\
375	0.0120654250773081\\
376	0.0120637821165994\\
377	0.012062118229727\\
378	0.012060433631219\\
379	0.0120587285944694\\
380	0.0120570034592443\\
381	0.0120552586402422\\
382	0.0120534946368853\\
383	0.0120517120445521\\
384	0.0120499115675062\\
385	0.0120480940338354\\
386	0.0120462604127716\\
387	0.012044411834686\\
388	0.0120425496144177\\
389	0.0120406752780346\\
390	0.0120387905969147\\
391	0.0120368976264479\\
392	0.0120349987516109\\
393	0.0120330967416673\\
394	0.0120311948169823\\
395	0.0120292967332153\\
396	0.0120274068732542\\
397	0.0120255303582682\\
398	0.0120236731943461\\
399	0.012021842462964\\
400	0.0120200465818283\\
401	0.0120182956815154\\
402	0.0120166021249311\\
403	0.0120149808186783\\
404	0.0120134051966291\\
405	0.0120118050173729\\
406	0.0120101800162975\\
407	0.012008529934249\\
408	0.0120068545182357\\
409	0.0120051535221704\\
410	0.0120034267076519\\
411	0.012001673844785\\
412	0.0119998947130367\\
413	0.0119980891021299\\
414	0.0119962568129681\\
415	0.0119943976585893\\
416	0.0119925114651317\\
417	0.0119905980727707\\
418	0.0119886573365104\\
419	0.0119866891271247\\
420	0.0119846933320899\\
421	0.0119826698564447\\
422	0.0119806186235152\\
423	0.0119785395754274\\
424	0.0119764326734344\\
425	0.0119742978988363\\
426	0.0119721352585366\\
427	0.0119699447634414\\
428	0.0119677264179037\\
429	0.0119654802499883\\
430	0.0119632063124931\\
431	0.0119609046837995\\
432	0.0119585754681763\\
433	0.0119562187958807\\
434	0.0119538348232256\\
435	0.0119514237322228\\
436	0.0119489857296504\\
437	0.0119465210453638\\
438	0.0119440299296241\\
439	0.0119415126491543\\
440	0.0119389694815311\\
441	0.0119364007074169\\
442	0.0119338066000023\\
443	0.0119311874108364\\
444	0.0119285433509625\\
445	0.0119258745659243\\
446	0.0119231811027\\
447	0.0119204628658186\\
448	0.0119177195584786\\
449	0.0119149506015577\\
450	0.0119121550169998\\
451	0.0119093284710531\\
452	0.0119064416637074\\
453	0.0119035232120867\\
454	0.0119005688437335\\
455	0.0118975724442471\\
456	0.0118945250215855\\
457	0.0118914243002373\\
458	0.0118882689223223\\
459	0.0118850574337212\\
460	0.0118817882765181\\
461	0.0118784598240128\\
462	0.0118750703146075\\
463	0.0118716178580883\\
464	0.0118681004207486\\
465	0.0118645158088869\\
466	0.0118608616469971\\
467	0.0118571353440785\\
468	0.011853334024718\\
469	0.0118494543430632\\
470	0.0118454918806472\\
471	0.0118413124685387\\
472	0.0118369871889903\\
473	0.0118325803753584\\
474	0.0118280893871499\\
475	0.0118235114289388\\
476	0.0118188434515252\\
477	0.0118140816567455\\
478	0.0118092224264765\\
479	0.0118042620391943\\
480	0.0117991965645841\\
481	0.01179402190825\\
482	0.0117887337676045\\
483	0.0117833276471263\\
484	0.0117777988265974\\
485	0.0117721421668744\\
486	0.0117663511629183\\
487	0.0117605354910614\\
488	0.0117547488071647\\
489	0.0117488018490309\\
490	0.011742687265597\\
491	0.011736397640294\\
492	0.011729926003789\\
493	0.0117232633521966\\
494	0.0117163987620694\\
495	0.0117093287142977\\
496	0.0117020467809634\\
497	0.0116945174299802\\
498	0.0116867268287663\\
499	0.0116786606184768\\
500	0.0116703055773252\\
501	0.0116616484001961\\
502	0.0116526620427354\\
503	0.0116433269590323\\
504	0.0116336221032504\\
505	0.0116235247802049\\
506	0.0116130104872893\\
507	0.0116020527128669\\
508	0.0115906227332482\\
509	0.0115786893916337\\
510	0.011566218853616\\
511	0.0115531743406828\\
512	0.0115395158224479\\
513	0.011525199490575\\
514	0.0115101771254553\\
515	0.0114943961763612\\
516	0.0114777989848389\\
517	0.0114603221763643\\
518	0.0114418955432169\\
519	0.0114224366743477\\
520	0.0114018578664005\\
521	0.0113800623340896\\
522	0.0113568983493816\\
523	0.0113322570736846\\
524	0.0113059704473728\\
525	0.0112778184763133\\
526	0.0112472523946274\\
527	0.0111943146120311\\
528	0.0111386976309552\\
529	0.0110798443997108\\
530	0.0110167833311675\\
531	0.010865529087143\\
532	0.0107024788368962\\
533	0.0105360981423921\\
534	0.0103667082895005\\
535	0.0102765910278294\\
536	0.010227222569959\\
537	0.0101827890109943\\
538	0.0101445658151101\\
539	0.010111371615044\\
540	0.0100789632735897\\
541	0.0100472067397383\\
542	0.010015832865925\\
543	0.00998498731553718\\
544	0.0099542235926231\\
545	0.00992295635263974\\
546	0.00989110830015612\\
547	0.00985862235006408\\
548	0.00982519568092307\\
549	0.00979095127016024\\
550	0.00975614293230419\\
551	0.00971929000415757\\
552	0.00968151833576426\\
553	0.00964363277383124\\
554	0.009607393326479\\
555	0.00957140391933581\\
556	0.00953471211834849\\
557	0.00949723437876546\\
558	0.00945889624815701\\
559	0.00941966748288923\\
560	0.00937951610888176\\
561	0.009337970523975\\
562	0.00929525080985312\\
563	0.00924896092881901\\
564	0.00920212158492495\\
565	0.00916014253809604\\
566	0.00911890457344965\\
567	0.00907690288812684\\
568	0.00903403143963363\\
569	0.00899025801317528\\
570	0.00894555664214493\\
571	0.00887271235043344\\
572	0.00869850024695905\\
573	0.00814933114564015\\
574	0.00798452580157154\\
575	0.00791321352136201\\
576	0.00784072225130142\\
577	0.00776696190277687\\
578	0.00769190043889123\\
579	0.00761550493014874\\
580	0.00753774111769362\\
581	0.00745857330586257\\
582	0.00737796424416958\\
583	0.00729587491733925\\
584	0.00721226408156575\\
585	0.00712708712862869\\
586	0.00704029318686975\\
587	0.00695181762108359\\
588	0.00686156256306142\\
589	0.00676934637110726\\
590	0.00667477256501155\\
591	0.00657713355197373\\
592	0.0064748166249999\\
593	0.00636364764633835\\
594	0.00623271716130668\\
595	0.0060534099414828\\
596	0.00575058001197163\\
597	0.00512683753504545\\
598	0.00366374385960312\\
599	0\\
600	0\\
};
\addplot [color=mycolor8,solid,forget plot]
  table[row sep=crcr]{%
1	0.012490636149034\\
2	0.0124906286463844\\
3	0.0124906210170971\\
4	0.0124906132590396\\
5	0.0124906053700436\\
6	0.0124905973479044\\
7	0.0124905891903803\\
8	0.0124905808951921\\
9	0.0124905724600222\\
10	0.0124905638825141\\
11	0.012490555160272\\
12	0.0124905462908596\\
13	0.0124905372718\\
14	0.0124905281005745\\
15	0.0124905187746225\\
16	0.0124905092913402\\
17	0.0124904996480803\\
18	0.0124904898421509\\
19	0.0124904798708151\\
20	0.0124904697312903\\
21	0.0124904594207471\\
22	0.0124904489363085\\
23	0.0124904382750496\\
24	0.0124904274339964\\
25	0.0124904164101251\\
26	0.012490405200361\\
27	0.0124903938015782\\
28	0.0124903822105984\\
29	0.0124903704241899\\
30	0.0124903584390671\\
31	0.0124903462518893\\
32	0.0124903338592598\\
33	0.0124903212577252\\
34	0.0124903084437742\\
35	0.0124902954138367\\
36	0.0124902821642829\\
37	0.0124902686914224\\
38	0.0124902549915029\\
39	0.0124902410607093\\
40	0.012490226895163\\
41	0.0124902124909201\\
42	0.0124901978439712\\
43	0.0124901829502394\\
44	0.0124901678055802\\
45	0.0124901524057793\\
46	0.0124901367465523\\
47	0.0124901208235431\\
48	0.0124901046323229\\
49	0.0124900881683888\\
50	0.0124900714271629\\
51	0.0124900544039905\\
52	0.0124900370941396\\
53	0.0124900194927989\\
54	0.0124900015950769\\
55	0.0124899833960004\\
56	0.0124899648905133\\
57	0.0124899460734752\\
58	0.0124899269396598\\
59	0.0124899074837537\\
60	0.0124898877003549\\
61	0.0124898675839714\\
62	0.0124898471290194\\
63	0.0124898263298224\\
64	0.0124898051806091\\
65	0.0124897836755119\\
66	0.0124897618085659\\
67	0.0124897395737063\\
68	0.0124897169647678\\
69	0.0124896939754822\\
70	0.0124896705994771\\
71	0.0124896468302738\\
72	0.0124896226612861\\
73	0.0124895980858181\\
74	0.0124895730970625\\
75	0.0124895476880991\\
76	0.0124895218518921\\
77	0.0124894955812893\\
78	0.0124894688690195\\
79	0.0124894417076904\\
80	0.0124894140897874\\
81	0.0124893860076709\\
82	0.0124893574535743\\
83	0.0124893284196023\\
84	0.0124892988977286\\
85	0.0124892688797938\\
86	0.0124892383575029\\
87	0.0124892073224237\\
88	0.012489175765984\\
89	0.0124891436794697\\
90	0.0124891110540223\\
91	0.0124890778806365\\
92	0.012489044150158\\
93	0.0124890098532808\\
94	0.0124889749805453\\
95	0.0124889395223349\\
96	0.0124889034688744\\
97	0.0124888668102267\\
98	0.0124888295362907\\
99	0.0124887916367982\\
100	0.0124887531013116\\
101	0.0124887139192208\\
102	0.0124886740797407\\
103	0.0124886335719083\\
104	0.0124885923845796\\
105	0.012488550506427\\
106	0.0124885079259363\\
107	0.0124884646314033\\
108	0.0124884206109314\\
109	0.012488375852428\\
110	0.0124883303436014\\
111	0.012488284071958\\
112	0.0124882370247985\\
113	0.0124881891892151\\
114	0.0124881405520878\\
115	0.0124880911000814\\
116	0.0124880408196416\\
117	0.0124879896969918\\
118	0.0124879377181297\\
119	0.0124878848688232\\
120	0.0124878311346073\\
121	0.0124877765007802\\
122	0.0124877209523993\\
123	0.0124876644742779\\
124	0.0124876070509808\\
125	0.0124875486668209\\
126	0.0124874893058548\\
127	0.012487428951879\\
128	0.0124873675884257\\
129	0.012487305198759\\
130	0.0124872417658701\\
131	0.0124871772724736\\
132	0.0124871117010029\\
133	0.012487045033606\\
134	0.0124869772521407\\
135	0.0124869083381709\\
136	0.012486838272961\\
137	0.0124867670374724\\
138	0.0124866946123579\\
139	0.0124866209779576\\
140	0.0124865461142939\\
141	0.0124864700010668\\
142	0.0124863926176488\\
143	0.0124863139430802\\
144	0.012486233956064\\
145	0.0124861526349609\\
146	0.0124860699577844\\
147	0.0124859859021952\\
148	0.0124859004454965\\
149	0.0124858135646288\\
150	0.0124857252361641\\
151	0.0124856354363014\\
152	0.0124855441408607\\
153	0.012485451325278\\
154	0.0124853569646003\\
155	0.0124852610334793\\
156	0.0124851635061672\\
157	0.0124850643565102\\
158	0.0124849635579443\\
159	0.012484861083489\\
160	0.0124847569057426\\
161	0.012484650996877\\
162	0.0124845433286323\\
163	0.012484433872312\\
164	0.012484322598778\\
165	0.0124842094784457\\
166	0.0124840944812796\\
167	0.0124839775767887\\
168	0.0124838587340227\\
169	0.0124837379215675\\
170	0.012483615107542\\
171	0.0124834902595951\\
172	0.0124833633449024\\
173	0.0124832343301642\\
174	0.0124831031816039\\
175	0.0124829698649667\\
176	0.0124828343455194\\
177	0.0124826965880507\\
178	0.012482556556873\\
179	0.0124824142158243\\
180	0.0124822695282726\\
181	0.0124821224571204\\
182	0.0124819729648115\\
183	0.0124818210133389\\
184	0.0124816665642556\\
185	0.0124815095786859\\
186	0.0124813500173406\\
187	0.0124811878405344\\
188	0.0124810230082078\\
189	0.0124808554799551\\
190	0.0124806852150662\\
191	0.0124805121726009\\
192	0.0124803363115452\\
193	0.012480157591189\\
194	0.0124799759721103\\
195	0.0124797914188318\\
196	0.0124796039071297\\
197	0.0124794134443417\\
198	0.0124792201261817\\
199	0.0124790234906164\\
200	0.0124788234627171\\
201	0.0124786199842485\\
202	0.0124784129959842\\
203	0.0124782024376896\\
204	0.0124779882481053\\
205	0.0124777703649293\\
206	0.0124775487248002\\
207	0.012477323263279\\
208	0.012477093914831\\
209	0.0124768606128075\\
210	0.0124766232894273\\
211	0.0124763818757571\\
212	0.0124761363016928\\
213	0.0124758864959394\\
214	0.0124756323859912\\
215	0.0124753738981114\\
216	0.0124751109573116\\
217	0.0124748434873302\\
218	0.0124745714106118\\
219	0.0124742946482848\\
220	0.0124740131201396\\
221	0.0124737267446061\\
222	0.0124734354387304\\
223	0.0124731391181521\\
224	0.0124728376970802\\
225	0.012472531088269\\
226	0.0124722192029938\\
227	0.0124719019510258\\
228	0.0124715792406066\\
229	0.0124712509784227\\
230	0.0124709170695786\\
231	0.0124705774175709\\
232	0.0124702319242603\\
233	0.0124698804898441\\
234	0.0124695230128285\\
235	0.0124691593899991\\
236	0.0124687895163922\\
237	0.0124684132852652\\
238	0.0124680305880657\\
239	0.0124676413144014\\
240	0.0124672453520085\\
241	0.0124668425867195\\
242	0.0124664329024314\\
243	0.0124660161810721\\
244	0.012465592302567\\
245	0.0124651611448049\\
246	0.0124647225836032\\
247	0.0124642764926722\\
248	0.0124638227435796\\
249	0.0124633612057136\\
250	0.0124628917462456\\
251	0.0124624142300926\\
252	0.0124619285198785\\
253	0.0124614344758952\\
254	0.0124609319560628\\
255	0.0124604208158893\\
256	0.0124599009084292\\
257	0.0124593720842425\\
258	0.0124588341913522\\
259	0.0124582870752011\\
260	0.0124577305786088\\
261	0.0124571645417272\\
262	0.0124565888019956\\
263	0.0124560031940958\\
264	0.0124554075499056\\
265	0.0124548016984523\\
266	0.0124541854658652\\
267	0.0124535586753277\\
268	0.0124529211470288\\
269	0.0124522726981133\\
270	0.0124516131426324\\
271	0.0124509422914922\\
272	0.0124502599524026\\
273	0.0124495659298245\\
274	0.0124488600249171\\
275	0.0124481420354833\\
276	0.0124474117559145\\
277	0.0124466689771347\\
278	0.0124459134865431\\
279	0.0124451450679556\\
280	0.012444363501545\\
281	0.0124435685637808\\
282	0.0124427600273675\\
283	0.0124419376611817\\
284	0.012441101230209\\
285	0.012440250495479\\
286	0.0124393852139998\\
287	0.0124385051386907\\
288	0.012437610018315\\
289	0.0124366995974106\\
290	0.0124357736162197\\
291	0.0124348318106178\\
292	0.0124338739120407\\
293	0.0124328996474114\\
294	0.0124319087390648\\
295	0.0124309009046711\\
296	0.0124298758571593\\
297	0.0124288333046373\\
298	0.0124277729503128\\
299	0.0124266944924109\\
300	0.0124255976240919\\
301	0.0124244820333668\\
302	0.0124233474030115\\
303	0.0124221934104797\\
304	0.0124210197278143\\
305	0.0124198260215568\\
306	0.0124186119526559\\
307	0.0124173771763737\\
308	0.0124161213421908\\
309	0.0124148440937093\\
310	0.0124135450685544\\
311	0.0124122238982738\\
312	0.0124108802082359\\
313	0.0124095136175251\\
314	0.0124081237388363\\
315	0.0124067101783664\\
316	0.0124052725357044\\
317	0.0124038104037189\\
318	0.0124023233684436\\
319	0.0124008110089607\\
320	0.0123992728972814\\
321	0.0123977085982243\\
322	0.0123961176692914\\
323	0.0123944996605409\\
324	0.0123928541144578\\
325	0.0123911805658216\\
326	0.0123894785415702\\
327	0.0123877475606619\\
328	0.0123859871339331\\
329	0.0123841967639533\\
330	0.0123823759448762\\
331	0.012380524162287\\
332	0.012378640893046\\
333	0.0123767256051278\\
334	0.0123747777574568\\
335	0.0123727967997372\\
336	0.0123707821722788\\
337	0.012368733305818\\
338	0.012366649621333\\
339	0.0123645305298533\\
340	0.0123623754322637\\
341	0.0123601837191021\\
342	0.01235795477035\\
343	0.0123556879552169\\
344	0.0123533826319166\\
345	0.0123510381474363\\
346	0.0123486538372969\\
347	0.0123462290253051\\
348	0.0123437630232958\\
349	0.0123412551308656\\
350	0.0123387046350953\\
351	0.0123361108102618\\
352	0.0123334729175395\\
353	0.0123307902046884\\
354	0.0123280619057314\\
355	0.0123252872406174\\
356	0.0123224654148719\\
357	0.0123195956192325\\
358	0.0123166770292713\\
359	0.0123137088050016\\
360	0.0123106900904701\\
361	0.0123076200133332\\
362	0.0123044976844193\\
363	0.0123013221972749\\
364	0.0122980926276977\\
365	0.0122948080332549\\
366	0.0122914674527901\\
367	0.01228806990592\\
368	0.0122846143925232\\
369	0.0122810998922244\\
370	0.0122775253638797\\
371	0.0122738897450669\\
372	0.0122701919515896\\
373	0.0122664308770044\\
374	0.0122626053921818\\
375	0.0122587143449171\\
376	0.012254756559608\\
377	0.0122507308370242\\
378	0.0122466359541946\\
379	0.0122424706644498\\
380	0.012238233697663\\
381	0.0122339237607445\\
382	0.0122295395384605\\
383	0.0122250796946701\\
384	0.0122205428741229\\
385	0.012215927705065\\
386	0.0122112328032047\\
387	0.0122064567784605\\
388	0.0122015982484379\\
389	0.0121966558695895\\
390	0.012191628246412\\
391	0.012186513987498\\
392	0.0121813117276193\\
393	0.0121760201323573\\
394	0.0121706378852504\\
395	0.01216516362561\\
396	0.0121595962902764\\
397	0.0121539352215202\\
398	0.012148180032555\\
399	0.0121423307017629\\
400	0.0121363877419276\\
401	0.0121303526073261\\
402	0.0121242289501595\\
403	0.0121180270678624\\
404	0.0121130536747471\\
405	0.0121104523023287\\
406	0.0121078083185364\\
407	0.01210512115808\\
408	0.0121023902599384\\
409	0.0120996150686293\\
410	0.0120967950356053\\
411	0.0120939296207917\\
412	0.0120910182942822\\
413	0.0120880605382211\\
414	0.0120850558489222\\
415	0.0120820037393318\\
416	0.0120789037420903\\
417	0.0120757554137948\\
418	0.0120725583418423\\
419	0.0120693121427133\\
420	0.0120660164652204\\
421	0.012062670995064\\
422	0.0120592754599662\\
423	0.0120558296354722\\
424	0.0120523333515216\\
425	0.0120487864998532\\
426	0.0120451890418797\\
427	0.0120415410171614\\
428	0.012037842556226\\
429	0.0120340938949395\\
430	0.0120302953893922\\
431	0.0120264475346821\\
432	0.0120225509918269\\
433	0.0120186066032817\\
434	0.0120146154068519\\
435	0.0120105786619206\\
436	0.0120064978795476\\
437	0.0120023748571068\\
438	0.0119982117182366\\
439	0.0119940109591714\\
440	0.0119897755030979\\
441	0.011985508763572\\
442	0.0119812147186527\\
443	0.0119768979982757\\
444	0.0119725639877198\\
445	0.0119682189507369\\
446	0.0119638701768505\\
447	0.0119595261585687\\
448	0.0119551968060185\\
449	0.0119508937093172\\
450	0.0119466304644099\\
451	0.0119424230905271\\
452	0.0119382905852469\\
453	0.0119342556219057\\
454	0.0119303457517199\\
455	0.0119265948528963\\
456	0.0119230429179789\\
457	0.0119194710046449\\
458	0.0119158476151377\\
459	0.011912172431185\\
460	0.0119084451603485\\
461	0.0119046655341048\\
462	0.011900833308525\\
463	0.0118969482645121\\
464	0.0118930102074505\\
465	0.0118890189659581\\
466	0.0118849743891262\\
467	0.0118808763408364\\
468	0.0118767246876381\\
469	0.011872519271243\\
470	0.0118682598441434\\
471	0.0118639175456341\\
472	0.0118595082453442\\
473	0.0118550461576761\\
474	0.0118505306681751\\
475	0.0118459609439259\\
476	0.0118413358455098\\
477	0.0118366537456646\\
478	0.0118319125774443\\
479	0.011827109630832\\
480	0.0118222413387052\\
481	0.0118173030358881\\
482	0.0118122886282725\\
483	0.0118071901462051\\
484	0.0118019970039349\\
485	0.011796694591656\\
486	0.0117912646905644\\
487	0.0117856076044617\\
488	0.0117796737059394\\
489	0.0117736075515142\\
490	0.0117674039004969\\
491	0.0117610571584606\\
492	0.0117545613395641\\
493	0.0117479101315451\\
494	0.0117410971372608\\
495	0.0117341162723479\\
496	0.01172696046367\\
497	0.0117196180438407\\
498	0.0117120827193301\\
499	0.0117043623189843\\
500	0.0116964204462794\\
501	0.0116882973672571\\
502	0.0116801970713572\\
503	0.0116718146623912\\
504	0.0116631326896446\\
505	0.0116541326997863\\
506	0.0116447950870197\\
507	0.0116350993701108\\
508	0.0116250235166712\\
509	0.0116145437033562\\
510	0.0116036342101411\\
511	0.0115922673594586\\
512	0.0115804138023002\\
513	0.0115680442561159\\
514	0.011555125983707\\
515	0.0115416112288776\\
516	0.0115274580251271\\
517	0.0115126205783209\\
518	0.0114970488302267\\
519	0.011480688169082\\
520	0.0114634785901327\\
521	0.0114453538456015\\
522	0.0114262353226874\\
523	0.0114060436987103\\
524	0.0113846838721412\\
525	0.0113620164291072\\
526	0.0113379306283637\\
527	0.0113123151655743\\
528	0.0112849995616158\\
529	0.0112557631806695\\
530	0.0112242777088344\\
531	0.0111717754723663\\
532	0.011114671874946\\
533	0.0110543070849606\\
534	0.0109898061926997\\
535	0.010857563607989\\
536	0.0106916036541881\\
537	0.0105221471154174\\
538	0.0103495172527279\\
539	0.0102113165682049\\
540	0.0101563443194975\\
541	0.0101056483225631\\
542	0.0100605936551069\\
543	0.0100224638458455\\
544	0.0099863417802264\\
545	0.0099509678935687\\
546	0.00991617660108073\\
547	0.0098816420316696\\
548	0.00984777035242589\\
549	0.00981361797051381\\
550	0.00977893593183029\\
551	0.00974344075680954\\
552	0.00970717472075751\\
553	0.00967020301537289\\
554	0.00963110590679491\\
555	0.00959105434587335\\
556	0.00955084132009492\\
557	0.00951190359616305\\
558	0.00947355548253387\\
559	0.00943445149762873\\
560	0.00939453485050878\\
561	0.00935369861231558\\
562	0.00931190536002771\\
563	0.009268550527102\\
564	0.00922417461261898\\
565	0.00917597825417376\\
566	0.00912753705130518\\
567	0.00908351322316861\\
568	0.00904056753481854\\
569	0.00899687644885152\\
570	0.00895228057669682\\
571	0.0089067404232288\\
572	0.00885045980593461\\
573	0.00873067527223201\\
574	0.00827491996561498\\
575	0.00791523178946964\\
576	0.00784073719470976\\
577	0.0077669654453666\\
578	0.00769190220158509\\
579	0.0076155058810726\\
580	0.00753774161250178\\
581	0.00745857354656985\\
582	0.007377964351367\\
583	0.00729587495987698\\
584	0.00721226409604604\\
585	0.00712708713261023\\
586	0.00704029318765944\\
587	0.00695181762116886\\
588	0.00686156256306143\\
589	0.00676934637110726\\
590	0.00667477256501155\\
591	0.00657713355197373\\
592	0.00647481662499992\\
593	0.00636364764633835\\
594	0.00623271716130669\\
595	0.00605340994148281\\
596	0.00575058001197164\\
597	0.00512683753504545\\
598	0.00366374385960312\\
599	0\\
600	0\\
};
\addplot [color=blue!25!mycolor7,solid,forget plot]
  table[row sep=crcr]{%
1	0.0126627991686105\\
2	0.0126627964209655\\
3	0.0126627936269292\\
4	0.0126627907857206\\
5	0.0126627878965456\\
6	0.0126627849585968\\
7	0.0126627819710535\\
8	0.0126627789330809\\
9	0.0126627758438304\\
10	0.0126627727024394\\
11	0.0126627695080304\\
12	0.0126627662597116\\
13	0.012662762956576\\
14	0.0126627595977018\\
15	0.0126627561821514\\
16	0.0126627527089716\\
17	0.0126627491771935\\
18	0.0126627455858316\\
19	0.0126627419338844\\
20	0.0126627382203332\\
21	0.0126627344441425\\
22	0.0126627306042596\\
23	0.0126627266996141\\
24	0.0126627227291175\\
25	0.0126627186916636\\
26	0.0126627145861273\\
27	0.012662710411365\\
28	0.0126627061662138\\
29	0.0126627018494916\\
30	0.0126626974599964\\
31	0.0126626929965063\\
32	0.0126626884577789\\
33	0.0126626838425512\\
34	0.0126626791495391\\
35	0.0126626743774371\\
36	0.0126626695249178\\
37	0.0126626645906321\\
38	0.012662659573208\\
39	0.0126626544712508\\
40	0.0126626492833428\\
41	0.0126626440080423\\
42	0.0126626386438841\\
43	0.0126626331893782\\
44	0.0126626276430101\\
45	0.01266262200324\\
46	0.0126626162685025\\
47	0.0126626104372064\\
48	0.0126626045077338\\
49	0.0126625984784401\\
50	0.0126625923476534\\
51	0.012662586113674\\
52	0.0126625797747739\\
53	0.0126625733291968\\
54	0.012662566775157\\
55	0.0126625601108392\\
56	0.0126625533343981\\
57	0.012662546443958\\
58	0.0126625394376119\\
59	0.0126625323134214\\
60	0.0126625250694159\\
61	0.0126625177035924\\
62	0.0126625102139147\\
63	0.0126625025983128\\
64	0.0126624948546829\\
65	0.012662486980886\\
66	0.0126624789747483\\
67	0.0126624708340597\\
68	0.0126624625565739\\
69	0.0126624541400075\\
70	0.0126624455820396\\
71	0.012662436880311\\
72	0.0126624280324236\\
73	0.01266241903594\\
74	0.0126624098883826\\
75	0.0126624005872332\\
76	0.0126623911299322\\
77	0.012662381513878\\
78	0.0126623717364262\\
79	0.0126623617948892\\
80	0.0126623516865353\\
81	0.012662341408588\\
82	0.0126623309582254\\
83	0.0126623203325796\\
84	0.0126623095287355\\
85	0.0126622985437306\\
86	0.0126622873745539\\
87	0.0126622760181454\\
88	0.0126622644713951\\
89	0.0126622527311424\\
90	0.0126622407941752\\
91	0.0126622286572292\\
92	0.0126622163169869\\
93	0.012662203770077\\
94	0.0126621910130735\\
95	0.0126621780424948\\
96	0.0126621648548028\\
97	0.0126621514464022\\
98	0.0126621378136396\\
99	0.0126621239528024\\
100	0.0126621098601182\\
101	0.0126620955317535\\
102	0.0126620809638133\\
103	0.0126620661523397\\
104	0.0126620510933112\\
105	0.0126620357826416\\
106	0.0126620202161793\\
107	0.012662004389706\\
108	0.0126619882989357\\
109	0.0126619719395141\\
110	0.0126619553070173\\
111	0.0126619383969505\\
112	0.0126619212047477\\
113	0.0126619037257699\\
114	0.0126618859553044\\
115	0.0126618678885638\\
116	0.0126618495206847\\
117	0.0126618308467269\\
118	0.0126618118616717\\
119	0.0126617925604217\\
120	0.0126617729377989\\
121	0.0126617529885438\\
122	0.0126617327073145\\
123	0.012661712088685\\
124	0.0126616911271448\\
125	0.0126616698170971\\
126	0.0126616481528576\\
127	0.012661626128654\\
128	0.0126616037386239\\
129	0.0126615809768143\\
130	0.0126615578371799\\
131	0.0126615343135822\\
132	0.0126615103997882\\
133	0.0126614860894689\\
134	0.0126614613761983\\
135	0.0126614362534524\\
136	0.0126614107146072\\
137	0.0126613847529382\\
138	0.0126613583616186\\
139	0.0126613315337183\\
140	0.0126613042622025\\
141	0.0126612765399306\\
142	0.0126612483596545\\
143	0.0126612197140177\\
144	0.0126611905955538\\
145	0.0126611609966852\\
146	0.0126611309097219\\
147	0.0126611003268598\\
148	0.0126610692401799\\
149	0.0126610376416467\\
150	0.0126610055231065\\
151	0.0126609728762867\\
152	0.0126609396927937\\
153	0.0126609059641122\\
154	0.012660871681603\\
155	0.0126608368365022\\
156	0.0126608014199192\\
157	0.0126607654228354\\
158	0.0126607288361026\\
159	0.0126606916504412\\
160	0.0126606538564384\\
161	0.0126606154445467\\
162	0.0126605764050815\\
163	0.0126605367282192\\
164	0.0126604964039951\\
165	0.0126604554223009\\
166	0.0126604137728819\\
167	0.0126603714453341\\
168	0.0126603284291013\\
169	0.0126602847134714\\
170	0.012660240287572\\
171	0.0126601951403666\\
172	0.0126601492606492\\
173	0.0126601026370384\\
174	0.0126600552579714\\
175	0.0126600071116959\\
176	0.0126599581862623\\
177	0.0126599084695136\\
178	0.0126598579490742\\
179	0.0126598066123372\\
180	0.0126597544464499\\
181	0.0126597014382966\\
182	0.0126596475744791\\
183	0.0126595928412948\\
184	0.0126595372247106\\
185	0.0126594807103342\\
186	0.0126594232833799\\
187	0.0126593649286309\\
188	0.0126593056303967\\
189	0.0126592453724657\\
190	0.0126591841380556\\
191	0.0126591219097689\\
192	0.0126590586695676\\
193	0.0126589943988123\\
194	0.0126589290784635\\
195	0.0126588626896697\\
196	0.0126587952151837\\
197	0.0126587266422602\\
198	0.012658656966977\\
199	0.0126585860957209\\
200	0.0126585140048563\\
201	0.0126584406735843\\
202	0.0126583660807569\\
203	0.0126582902048715\\
204	0.0126582130240652\\
205	0.0126581345161083\\
206	0.0126580546583992\\
207	0.0126579734279578\\
208	0.0126578908014192\\
209	0.0126578067550279\\
210	0.0126577212646309\\
211	0.0126576343056717\\
212	0.0126575458531834\\
213	0.0126574558817819\\
214	0.0126573643656596\\
215	0.0126572712785779\\
216	0.0126571765938606\\
217	0.0126570802843864\\
218	0.0126569823225821\\
219	0.0126568826804148\\
220	0.0126567813293844\\
221	0.0126566782405163\\
222	0.0126565733843536\\
223	0.0126564667309488\\
224	0.0126563582498562\\
225	0.0126562479101238\\
226	0.0126561356802845\\
227	0.0126560215283485\\
228	0.012655905421794\\
229	0.0126557873275588\\
230	0.0126556672120317\\
231	0.0126555450410431\\
232	0.0126554207798562\\
233	0.0126552943931575\\
234	0.0126551658450475\\
235	0.0126550350990311\\
236	0.0126549021180079\\
237	0.0126547668642622\\
238	0.012654629299453\\
239	0.0126544893846038\\
240	0.0126543470800925\\
241	0.0126542023456403\\
242	0.0126540551403015\\
243	0.0126539054224525\\
244	0.0126537531497808\\
245	0.0126535982792737\\
246	0.0126534407672071\\
247	0.0126532805691341\\
248	0.012653117639873\\
249	0.0126529519334958\\
250	0.0126527834033159\\
251	0.0126526120018764\\
252	0.0126524376809374\\
253	0.0126522603914633\\
254	0.012652080083611\\
255	0.0126518967067161\\
256	0.0126517102092807\\
257	0.0126515205389597\\
258	0.0126513276425479\\
259	0.0126511314659663\\
260	0.0126509319542486\\
261	0.0126507290515273\\
262	0.0126505227010196\\
263	0.0126503128450135\\
264	0.0126500994248532\\
265	0.0126498823809246\\
266	0.0126496616526407\\
267	0.0126494371784264\\
268	0.0126492088957035\\
269	0.0126489767408751\\
270	0.0126487406493103\\
271	0.0126485005553278\\
272	0.0126482563921802\\
273	0.0126480080920375\\
274	0.01264775558597\\
275	0.0126474988039321\\
276	0.0126472376747443\\
277	0.012646972126076\\
278	0.0126467020844278\\
279	0.0126464274751133\\
280	0.0126461482222406\\
281	0.0126458642486937\\
282	0.0126455754761138\\
283	0.0126452818248799\\
284	0.0126449832140889\\
285	0.0126446795615367\\
286	0.012644370783697\\
287	0.0126440567957016\\
288	0.012643737511319\\
289	0.0126434128429339\\
290	0.0126430827015251\\
291	0.012642746996644\\
292	0.0126424056363922\\
293	0.0126420585273991\\
294	0.0126417055747988\\
295	0.0126413466822065\\
296	0.012640981751695\\
297	0.0126406106837703\\
298	0.012640233377347\\
299	0.0126398497297228\\
300	0.0126394596365532\\
301	0.0126390629918254\\
302	0.0126386596878313\\
303	0.0126382496151403\\
304	0.0126378326625718\\
305	0.0126374087171665\\
306	0.0126369776641577\\
307	0.0126365393869412\\
308	0.0126360937670452\\
309	0.0126356406840993\\
310	0.0126351800158024\\
311	0.0126347116378901\\
312	0.0126342354241012\\
313	0.0126337512461437\\
314	0.0126332589736588\\
315	0.0126327584741849\\
316	0.0126322496131201\\
317	0.0126317322536837\\
318	0.0126312062568764\\
319	0.0126306714814394\\
320	0.0126301277838115\\
321	0.0126295750180859\\
322	0.0126290130359642\\
323	0.0126284416867095\\
324	0.0126278608170973\\
325	0.012627270271365\\
326	0.0126266698911585\\
327	0.0126260595154773\\
328	0.012625438980617\\
329	0.0126248081201088\\
330	0.012624166764657\\
331	0.0126235147420725\\
332	0.0126228518772038\\
333	0.0126221779918646\\
334	0.0126214929047562\\
335	0.0126207964313878\\
336	0.0126200883839903\\
337	0.0126193685714266\\
338	0.0126186367990958\\
339	0.0126178928688317\\
340	0.0126171365787953\\
341	0.0126163677233595\\
342	0.0126155860929871\\
343	0.0126147914741002\\
344	0.0126139836489396\\
345	0.0126131623954158\\
346	0.0126123274869473\\
347	0.0126114786922876\\
348	0.0126106157753389\\
349	0.0126097384949502\\
350	0.0126088466047003\\
351	0.0126079398526609\\
352	0.0126070179811412\\
353	0.0126060807264093\\
354	0.0126051278183887\\
355	0.0126041589803275\\
356	0.0126031739284356\\
357	0.0126021723714884\\
358	0.0126011540103906\\
359	0.0126001185376966\\
360	0.0125990656370813\\
361	0.0125979949827559\\
362	0.0125969062388204\\
363	0.0125957990585441\\
364	0.0125946730835669\\
365	0.0125935279430064\\
366	0.0125923632524616\\
367	0.0125911786128946\\
368	0.0125899736093762\\
369	0.0125887478096708\\
370	0.0125875007626403\\
371	0.0125862319964359\\
372	0.0125849410164452\\
373	0.0125836273029556\\
374	0.0125822903084857\\
375	0.0125809294547307\\
376	0.0125795441290541\\
377	0.0125781336804476\\
378	0.0125766974148622\\
379	0.0125752345897976\\
380	0.0125737444080068\\
381	0.0125722260101416\\
382	0.0125706784661053\\
383	0.0125691007647705\\
384	0.0125674918014544\\
385	0.0125658503618055\\
386	0.0125641750984852\\
387	0.0125624644897927\\
388	0.012560716745713\\
389	0.0125589295484406\\
390	0.012557101861502\\
391	0.0125552316388122\\
392	0.012553316370943\\
393	0.0125513533295656\\
394	0.0125493397699444\\
395	0.0125472736080522\\
396	0.0125451474116995\\
397	0.0125429522436535\\
398	0.0125406813320105\\
399	0.0125383266460232\\
400	0.0125358785399175\\
401	0.0125333251353853\\
402	0.0125306509543421\\
403	0.0125278331046089\\
404	0.0125238189354777\\
405	0.0125177247103424\\
406	0.0125115283167343\\
407	0.0125052279616382\\
408	0.0124988218172144\\
409	0.012492308020313\\
410	0.0124856846719898\\
411	0.0124789498369781\\
412	0.0124721015430077\\
413	0.0124651377797076\\
414	0.0124580564964506\\
415	0.0124508555975605\\
416	0.0124435329309816\\
417	0.0124360862607481\\
418	0.0124285131994143\\
419	0.0124208115287634\\
420	0.0124129790141578\\
421	0.0124050133710544\\
422	0.0123969122630009\\
423	0.0123886732997546\\
424	0.0123802940360336\\
425	0.0123717719724966\\
426	0.0123631045637948\\
427	0.0123542892476439\\
428	0.0123453233451612\\
429	0.0123362040273265\\
430	0.0123269283548499\\
431	0.0123174932190651\\
432	0.0123078951951936\\
433	0.0122981310331372\\
434	0.0122881977596803\\
435	0.0122780923491395\\
436	0.0122678117263272\\
437	0.0122573527712034\\
438	0.012246712325635\\
439	0.012235887199955\\
440	0.0122248741689738\\
441	0.0122136699878926\\
442	0.012202271417721\\
443	0.0121906752468003\\
444	0.0121788783183176\\
445	0.0121668775653097\\
446	0.012154670055046\\
447	0.0121422530451988\\
448	0.0121296240549782\\
449	0.0121167809557951\\
450	0.0121037220893232\\
451	0.0120904464307419\\
452	0.0120769538495803\\
453	0.0120632456526816\\
454	0.0120493260736496\\
455	0.0120352072503406\\
456	0.0120209269542652\\
457	0.0120141474630193\\
458	0.0120083362709143\\
459	0.0120024389967806\\
460	0.0119964557937994\\
461	0.0119903870880293\\
462	0.0119842335388187\\
463	0.0119779960722039\\
464	0.011971675919305\\
465	0.0119652746603985\\
466	0.0119587942756917\\
467	0.011952237203908\\
468	0.0119456064099843\\
469	0.0119389054636415\\
470	0.0119321386323924\\
471	0.0119253110049611\\
472	0.0119184285521223\\
473	0.0119114983090526\\
474	0.0119045285924051\\
475	0.0118975292233623\\
476	0.0118905117983631\\
477	0.0118834900224544\\
478	0.0118764800886545\\
479	0.0118695011579454\\
480	0.0118625759625213\\
481	0.0118557315769333\\
482	0.0118490004670204\\
483	0.011842421972757\\
484	0.0118360444205979\\
485	0.0118299272661661\\
486	0.0118240711065785\\
487	0.011818093823366\\
488	0.0118119836632568\\
489	0.0118057712209277\\
490	0.0117994541064894\\
491	0.0117930297142576\\
492	0.011786495183732\\
493	0.0117798473523641\\
494	0.0117730826932324\\
495	0.0117661972219266\\
496	0.011759186416015\\
497	0.0117520453057056\\
498	0.0117447684750535\\
499	0.0117373515372546\\
500	0.0117297811306234\\
501	0.0117219943323928\\
502	0.0117138105032538\\
503	0.0117054677285961\\
504	0.011696953652332\\
505	0.0116882207881493\\
506	0.0116792440778941\\
507	0.0116699905659572\\
508	0.0116604422902908\\
509	0.0116505823945872\\
510	0.0116403918843002\\
511	0.0116298506306759\\
512	0.0116189369472603\\
513	0.011607625654809\\
514	0.0115959861465712\\
515	0.0115841790707076\\
516	0.0115718780158105\\
517	0.0115590485282023\\
518	0.0115456531052509\\
519	0.0115316508974598\\
520	0.0115169974011455\\
521	0.0115016441650013\\
522	0.0114855387896299\\
523	0.0114686243726267\\
524	0.0114508390277682\\
525	0.0114321136290695\\
526	0.0114123788587613\\
527	0.0113915364004079\\
528	0.0113694545366788\\
529	0.0113460319457778\\
530	0.011321156324708\\
531	0.0112947145386821\\
532	0.0112665397799165\\
533	0.0112364169914739\\
534	0.0112040357446609\\
535	0.0111551667727741\\
536	0.011096919169057\\
537	0.0110354652627191\\
538	0.0109700878101411\\
539	0.0108708039195637\\
540	0.0107023961700645\\
541	0.0105303385885915\\
542	0.0103548327473627\\
543	0.010175555371547\\
544	0.0100891311152285\\
545	0.0100313394797322\\
546	0.00997814790714204\\
547	0.0099310988020503\\
548	0.00989068920853819\\
549	0.00985091345286371\\
550	0.0098118748301417\\
551	0.00977342506205893\\
552	0.0097353412845486\\
553	0.00969803635282579\\
554	0.00965987302914852\\
555	0.00962100115308273\\
556	0.00958148348237549\\
557	0.00954011250497676\\
558	0.00949743746264783\\
559	0.00945456395837911\\
560	0.00941223298613685\\
561	0.00937127836630832\\
562	0.00932951898304717\\
563	0.00928692394318492\\
564	0.0092433602842606\\
565	0.00919817203828855\\
566	0.00915199392923645\\
567	0.00910223641591428\\
568	0.00905183590070086\\
569	0.00900473865587824\\
570	0.00895991368126304\\
571	0.00891439948712266\\
572	0.00886794299834819\\
573	0.00882049631800914\\
574	0.00871402909880057\\
575	0.00843939249643411\\
576	0.00791197803111671\\
577	0.00776725103636003\\
578	0.00769192980440901\\
579	0.00761551684140999\\
580	0.00753774755897855\\
581	0.00745857676192871\\
582	0.00737796598891016\\
583	0.00729587572800404\\
584	0.00721226441867625\\
585	0.00712708724941016\\
586	0.00704029322199904\\
587	0.00695181762849465\\
588	0.00686156256391835\\
589	0.00676934637110726\\
590	0.00667477256501155\\
591	0.00657713355197375\\
592	0.00647481662499991\\
593	0.00636364764633835\\
594	0.00623271716130668\\
595	0.00605340994148281\\
596	0.00575058001197164\\
597	0.00512683753504545\\
598	0.00366374385960312\\
599	0\\
600	0\\
};
\addplot [color=mycolor9,solid,forget plot]
  table[row sep=crcr]{%
1	0.0127187035361118\\
2	0.012718702153434\\
3	0.0127187007474257\\
4	0.0127186993176947\\
5	0.0127186978638424\\
6	0.0127186963854635\\
7	0.012718694882146\\
8	0.0127186933534709\\
9	0.0127186917990122\\
10	0.0127186902183371\\
11	0.0127186886110052\\
12	0.012718686976569\\
13	0.0127186853145735\\
14	0.012718683624556\\
15	0.0127186819060464\\
16	0.0127186801585664\\
17	0.0127186783816301\\
18	0.0127186765747432\\
19	0.0127186747374033\\
20	0.0127186728690999\\
21	0.0127186709693135\\
22	0.0127186690375165\\
23	0.0127186670731722\\
24	0.0127186650757351\\
25	0.0127186630446506\\
26	0.0127186609793551\\
27	0.0127186588792753\\
28	0.0127186567438287\\
29	0.0127186545724229\\
30	0.012718652364456\\
31	0.0127186501193159\\
32	0.0127186478363803\\
33	0.0127186455150169\\
34	0.0127186431545826\\
35	0.0127186407544239\\
36	0.0127186383138764\\
37	0.0127186358322648\\
38	0.0127186333089026\\
39	0.0127186307430919\\
40	0.0127186281341233\\
41	0.0127186254812759\\
42	0.0127186227838166\\
43	0.0127186200410005\\
44	0.0127186172520702\\
45	0.0127186144162561\\
46	0.0127186115327755\\
47	0.0127186086008334\\
48	0.0127186056196213\\
49	0.0127186025883175\\
50	0.0127185995060869\\
51	0.0127185963720807\\
52	0.0127185931854361\\
53	0.0127185899452761\\
54	0.0127185866507094\\
55	0.0127185833008303\\
56	0.0127185798947179\\
57	0.0127185764314365\\
58	0.012718572910035\\
59	0.0127185693295469\\
60	0.0127185656889896\\
61	0.0127185619873649\\
62	0.012718558223658\\
63	0.0127185543968376\\
64	0.0127185505058558\\
65	0.0127185465496475\\
66	0.0127185425271304\\
67	0.0127185384372043\\
68	0.0127185342787515\\
69	0.0127185300506361\\
70	0.0127185257517036\\
71	0.0127185213807809\\
72	0.0127185169366761\\
73	0.0127185124181777\\
74	0.0127185078240548\\
75	0.0127185031530566\\
76	0.0127184984039122\\
77	0.0127184935753302\\
78	0.0127184886659982\\
79	0.012718483674583\\
80	0.0127184785997296\\
81	0.0127184734400617\\
82	0.0127184681941806\\
83	0.0127184628606653\\
84	0.0127184574380719\\
85	0.0127184519249336\\
86	0.0127184463197601\\
87	0.0127184406210372\\
88	0.0127184348272269\\
89	0.0127184289367663\\
90	0.0127184229480678\\
91	0.0127184168595186\\
92	0.0127184106694803\\
93	0.0127184043762886\\
94	0.0127183979782526\\
95	0.0127183914736548\\
96	0.0127183848607506\\
97	0.012718378137768\\
98	0.0127183713029067\\
99	0.0127183643543384\\
100	0.0127183572902059\\
101	0.0127183501086229\\
102	0.0127183428076735\\
103	0.0127183353854117\\
104	0.0127183278398614\\
105	0.0127183201690151\\
106	0.0127183123708346\\
107	0.0127183044432494\\
108	0.0127182963841571\\
109	0.0127182881914226\\
110	0.0127182798628776\\
111	0.0127182713963202\\
112	0.0127182627895144\\
113	0.0127182540401898\\
114	0.0127182451460409\\
115	0.0127182361047264\\
116	0.0127182269138694\\
117	0.012718217571056\\
118	0.0127182080738355\\
119	0.0127181984197195\\
120	0.0127181886061816\\
121	0.0127181786306566\\
122	0.0127181684905399\\
123	0.0127181581831876\\
124	0.0127181477059148\\
125	0.0127181370559963\\
126	0.0127181262306648\\
127	0.0127181152271113\\
128	0.0127181040424836\\
129	0.0127180926738865\\
130	0.0127180811183804\\
131	0.0127180693729812\\
132	0.0127180574346594\\
133	0.0127180453003393\\
134	0.0127180329668983\\
135	0.0127180204311662\\
136	0.0127180076899248\\
137	0.0127179947399063\\
138	0.0127179815777932\\
139	0.012717968200217\\
140	0.0127179546037576\\
141	0.0127179407849422\\
142	0.0127179267402445\\
143	0.0127179124660833\\
144	0.0127178979588221\\
145	0.0127178832147672\\
146	0.0127178682301672\\
147	0.0127178530012115\\
148	0.0127178375240289\\
149	0.0127178217946866\\
150	0.0127178058091882\\
151	0.0127177895634728\\
152	0.0127177730534127\\
153	0.0127177562748124\\
154	0.0127177392234061\\
155	0.0127177218948562\\
156	0.012717704284751\\
157	0.0127176863886023\\
158	0.0127176682018436\\
159	0.0127176497198269\\
160	0.0127176309378202\\
161	0.0127176118510048\\
162	0.0127175924544719\\
163	0.0127175727432191\\
164	0.0127175527121475\\
165	0.0127175323560569\\
166	0.0127175116696424\\
167	0.0127174906474899\\
168	0.0127174692840711\\
169	0.0127174475737387\\
170	0.0127174255107212\\
171	0.0127174030891168\\
172	0.0127173803028879\\
173	0.0127173571458545\\
174	0.0127173336116874\\
175	0.0127173096939014\\
176	0.0127172853858477\\
177	0.0127172606807064\\
178	0.0127172355714785\\
179	0.0127172100509778\\
180	0.0127171841118226\\
181	0.0127171577464273\\
182	0.0127171309469943\\
183	0.0127171037055057\\
184	0.0127170760137159\\
185	0.0127170478631442\\
186	0.0127170192450689\\
187	0.0127169901505219\\
188	0.0127169605702862\\
189	0.012716930494894\\
190	0.0127168999146298\\
191	0.012716868819537\\
192	0.0127168371994322\\
193	0.0127168050439309\\
194	0.0127167723424889\\
195	0.0127167390844593\\
196	0.0127167052591334\\
197	0.0127166708556113\\
198	0.012716635861977\\
199	0.0127166002681403\\
200	0.012716564063932\\
201	0.0127165272390142\\
202	0.0127164897828772\\
203	0.0127164516848373\\
204	0.0127164129340335\\
205	0.0127163735194248\\
206	0.0127163334297875\\
207	0.012716292653712\\
208	0.0127162511795998\\
209	0.0127162089956605\\
210	0.012716166089909\\
211	0.0127161224501618\\
212	0.0127160780640342\\
213	0.0127160329189369\\
214	0.0127159870020729\\
215	0.0127159403004337\\
216	0.0127158928007965\\
217	0.0127158444897201\\
218	0.012715795353542\\
219	0.0127157453783743\\
220	0.0127156945501004\\
221	0.0127156428543712\\
222	0.0127155902766013\\
223	0.0127155368019653\\
224	0.0127154824153938\\
225	0.0127154271015695\\
226	0.0127153708449233\\
227	0.0127153136296303\\
228	0.0127152554396052\\
229	0.0127151962584989\\
230	0.0127151360696935\\
231	0.0127150748562983\\
232	0.0127150126011455\\
233	0.0127149492867857\\
234	0.0127148848954829\\
235	0.0127148194092109\\
236	0.0127147528096475\\
237	0.0127146850781705\\
238	0.0127146161958528\\
239	0.012714546143457\\
240	0.0127144749014313\\
241	0.0127144024499037\\
242	0.012714328768677\\
243	0.0127142538372241\\
244	0.0127141776346823\\
245	0.0127141001398477\\
246	0.0127140213311706\\
247	0.0127139411867493\\
248	0.0127138596843244\\
249	0.0127137768012738\\
250	0.0127136925146064\\
251	0.0127136068009562\\
252	0.0127135196365767\\
253	0.0127134309973346\\
254	0.0127133408587036\\
255	0.0127132491957584\\
256	0.0127131559831682\\
257	0.0127130611951902\\
258	0.0127129648056632\\
259	0.0127128667880008\\
260	0.0127127671151846\\
261	0.0127126657597574\\
262	0.012712562693816\\
263	0.0127124578890042\\
264	0.0127123513165053\\
265	0.012712242947035\\
266	0.0127121327508333\\
267	0.0127120206976571\\
268	0.0127119067567724\\
269	0.0127117908969458\\
270	0.0127116730864366\\
271	0.0127115532929884\\
272	0.0127114314838201\\
273	0.0127113076256177\\
274	0.0127111816845249\\
275	0.0127110536261341\\
276	0.0127109234154768\\
277	0.0127107910170144\\
278	0.0127106563946279\\
279	0.0127105195116083\\
280	0.0127103803306458\\
281	0.0127102388138198\\
282	0.0127100949225878\\
283	0.0127099486177742\\
284	0.0127097998595593\\
285	0.0127096486074672\\
286	0.0127094948203544\\
287	0.0127093384563968\\
288	0.0127091794730776\\
289	0.0127090178271741\\
290	0.0127088534747444\\
291	0.0127086863711136\\
292	0.0127085164708595\\
293	0.0127083437277981\\
294	0.0127081680949687\\
295	0.012707989524618\\
296	0.0127078079681843\\
297	0.0127076233762807\\
298	0.0127074356986784\\
299	0.0127072448842884\\
300	0.0127070508811435\\
301	0.0127068536363791\\
302	0.0127066530962137\\
303	0.0127064492059284\\
304	0.0127062419098454\\
305	0.0127060311513064\\
306	0.0127058168726496\\
307	0.0127055990151857\\
308	0.0127053775191734\\
309	0.0127051523237934\\
310	0.0127049233671216\\
311	0.0127046905861013\\
312	0.0127044539165136\\
313	0.0127042132929472\\
314	0.0127039686487663\\
315	0.0127037199160774\\
316	0.0127034670256944\\
317	0.0127032099071021\\
318	0.0127029484884179\\
319	0.0127026826963519\\
320	0.0127024124561644\\
321	0.0127021376916225\\
322	0.0127018583249531\\
323	0.0127015742767945\\
324	0.0127012854661449\\
325	0.0127009918103091\\
326	0.0127006932248411\\
327	0.0127003896234845\\
328	0.0127000809181096\\
329	0.0126997670186471\\
330	0.0126994478330172\\
331	0.0126991232670562\\
332	0.0126987932244378\\
333	0.0126984576065904\\
334	0.0126981163126094\\
335	0.0126977692391644\\
336	0.0126974162804011\\
337	0.0126970573278366\\
338	0.0126966922702498\\
339	0.0126963209935632\\
340	0.0126959433807189\\
341	0.0126955593115466\\
342	0.0126951686626221\\
343	0.0126947713071185\\
344	0.0126943671146468\\
345	0.012693955951086\\
346	0.0126935376784026\\
347	0.0126931121544579\\
348	0.0126926792328023\\
349	0.0126922387624557\\
350	0.0126917905876727\\
351	0.0126913345476924\\
352	0.0126908704764694\\
353	0.0126903982023869\\
354	0.0126899175479491\\
355	0.0126894283294512\\
356	0.0126889303566258\\
357	0.0126884234322631\\
358	0.0126879073518029\\
359	0.0126873819028967\\
360	0.0126868468649355\\
361	0.0126863020085431\\
362	0.0126857470950287\\
363	0.0126851818757982\\
364	0.012684606091719\\
365	0.0126840194724339\\
366	0.0126834217356205\\
367	0.0126828125861917\\
368	0.0126821917154295\\
369	0.0126815588000498\\
370	0.0126809135011889\\
371	0.0126802554633061\\
372	0.0126795843129959\\
373	0.012678899657699\\
374	0.0126782010843071\\
375	0.0126774881576493\\
376	0.0126767604188521\\
377	0.0126760173835618\\
378	0.0126752585400196\\
379	0.0126744833469777\\
380	0.0126736912314439\\
381	0.0126728815862414\\
382	0.0126720537673614\\
383	0.0126712070910646\\
384	0.0126703408306282\\
385	0.0126694542124642\\
386	0.0126685464109121\\
387	0.0126676165399881\\
388	0.0126666636383049\\
389	0.0126656866406527\\
390	0.0126646846641529\\
391	0.0126636566193529\\
392	0.0126626013223191\\
393	0.0126615175508678\\
394	0.0126604040894902\\
395	0.0126592598112124\\
396	0.0126580827354889\\
397	0.0126568707871128\\
398	0.012655622418407\\
399	0.0126543360438814\\
400	0.0126530100315048\\
401	0.0126516426367741\\
402	0.0126502317655991\\
403	0.0126487742863174\\
404	0.0126470412231565\\
405	0.0126448394498039\\
406	0.0126426028906695\\
407	0.0126403310062329\\
408	0.012638023246909\\
409	0.0126356790527071\\
410	0.0126332978528633\\
411	0.0126308790654331\\
412	0.0126284220968131\\
413	0.0126259263411253\\
414	0.0126233911793037\\
415	0.0126208159775242\\
416	0.0126182000842078\\
417	0.0126155428240428\\
418	0.0126128434853768\\
419	0.0126101013742487\\
420	0.012607315778792\\
421	0.0126044859608663\\
422	0.0126016111524406\\
423	0.0125986905497337\\
424	0.0125957233009573\\
425	0.0125927084742787\\
426	0.0125896449622146\\
427	0.0125865311776374\\
428	0.0125833669648658\\
429	0.0125801524841931\\
430	0.012576887062375\\
431	0.0125735706441166\\
432	0.0125702050071502\\
433	0.0125667842615524\\
434	0.0125633011882153\\
435	0.0125597536815803\\
436	0.0125561394328797\\
437	0.0125524558990647\\
438	0.0125487002668484\\
439	0.0125448694163338\\
440	0.012540959859719\\
441	0.0125369676699011\\
442	0.0125328884048699\\
443	0.0125287170147654\\
444	0.01252444772953\\
445	0.0125200739225858\\
446	0.0125155879447591\\
447	0.0125109809210655\\
448	0.0125062425007236\\
449	0.0125013605473103\\
450	0.0124963207497012\\
451	0.0124911061204436\\
452	0.0124856963104355\\
453	0.0124800665523575\\
454	0.0124741856582942\\
455	0.0124680111442012\\
456	0.012461474571577\\
457	0.0124484630794266\\
458	0.0124343528801997\\
459	0.0124200022239117\\
460	0.0124054040698584\\
461	0.0123905512943999\\
462	0.0123754387233565\\
463	0.0123600610383275\\
464	0.0123444127726757\\
465	0.0123284883137113\\
466	0.0123122819039344\\
467	0.0122957876450154\\
468	0.0122789995066305\\
469	0.0122619113455549\\
470	0.012244516950215\\
471	0.0122268099785803\\
472	0.0122087839361696\\
473	0.0121904322792884\\
474	0.0121717484434963\\
475	0.0121527258665641\\
476	0.0121333579836693\\
477	0.0121136381365269\\
478	0.0120935604840553\\
479	0.0120731197552663\\
480	0.0120523113363066\\
481	0.0120311316413393\\
482	0.0120095781935255\\
483	0.0119876508158609\\
484	0.0119653550104346\\
485	0.011942713668386\\
486	0.0119217535111116\\
487	0.0119126553535508\\
488	0.0119034242335742\\
489	0.0118940621988752\\
490	0.0118845720085456\\
491	0.0118749572607399\\
492	0.0118652225447763\\
493	0.0118553736229872\\
494	0.0118454176490934\\
495	0.0118353634324121\\
496	0.0118252217593097\\
497	0.0118150057807003\\
498	0.0118047314801465\\
499	0.011794418152376\\
500	0.0117840892563953\\
501	0.0117737621951146\\
502	0.0117634343642958\\
503	0.0117532021922648\\
504	0.0117431177111872\\
505	0.0117332405380037\\
506	0.0117236518228485\\
507	0.0117144490297488\\
508	0.0117050984101696\\
509	0.011695540811201\\
510	0.0116857764281301\\
511	0.0116757512054881\\
512	0.0116654511142765\\
513	0.0116548598861948\\
514	0.0116438810727602\\
515	0.0116323495871931\\
516	0.0116204881962589\\
517	0.0116082752126049\\
518	0.0115956857098154\\
519	0.0115826912890469\\
520	0.0115692592239029\\
521	0.0115553509133539\\
522	0.0115409201179023\\
523	0.0115259101095428\\
524	0.0115102555767572\\
525	0.0114939090193393\\
526	0.0114769171695554\\
527	0.0114594426525852\\
528	0.011441086993902\\
529	0.0114217833923259\\
530	0.0114014574654767\\
531	0.0113799831495963\\
532	0.011357261802147\\
533	0.011333190732684\\
534	0.0113076561706216\\
535	0.0112805418077311\\
536	0.0112516938359571\\
537	0.011220917937706\\
538	0.0111879266671607\\
539	0.0111459660044578\\
540	0.0110869382246407\\
541	0.0110248217006492\\
542	0.0109590646109739\\
543	0.010888563122888\\
544	0.010738916237394\\
545	0.0105648875611786\\
546	0.0103871213951295\\
547	0.0102056152951943\\
548	0.0100299719568395\\
549	0.00996449037444903\\
550	0.00990245020666905\\
551	0.00984512024416929\\
552	0.00979412211129612\\
553	0.00974912543464925\\
554	0.00970501505029354\\
555	0.00966179010838528\\
556	0.00961907147515163\\
557	0.00957650089008366\\
558	0.00953431631936958\\
559	0.00949161053152337\\
560	0.00944769833697887\\
561	0.00940185771275797\\
562	0.00935583469263743\\
563	0.00930959204929004\\
564	0.00926525542069476\\
565	0.00922050872016028\\
566	0.00917487623137639\\
567	0.00912775069549462\\
568	0.00907955321167675\\
569	0.00902855805055227\\
570	0.00897578861335284\\
571	0.00892450720709081\\
572	0.00887761633395281\\
573	0.00883006113839492\\
574	0.00878163864994064\\
575	0.00870460980912668\\
576	0.00858817430045391\\
577	0.00813530455527196\\
578	0.00770022413322697\\
579	0.007615873705758\\
580	0.00753781942160739\\
581	0.00745861268072716\\
582	0.00737798602976268\\
583	0.00729588639332857\\
584	0.00721226969867924\\
585	0.0071270896051591\\
586	0.0070402941328682\\
587	0.00695181791635961\\
588	0.0068615626303918\\
589	0.00676934637959575\\
590	0.00667477256501156\\
591	0.00657713355197374\\
592	0.00647481662499991\\
593	0.00636364764633835\\
594	0.00623271716130669\\
595	0.00605340994148282\\
596	0.00575058001197164\\
597	0.00512683753504545\\
598	0.00366374385960312\\
599	0\\
600	0\\
};
\addplot [color=blue!50!mycolor7,solid,forget plot]
  table[row sep=crcr]{%
1	0.012792992079254\\
2	0.0127929907447039\\
3	0.0127929893876757\\
4	0.012792988007792\\
5	0.0127929866046692\\
6	0.0127929851779174\\
7	0.0127929837271399\\
8	0.0127929822519337\\
9	0.0127929807518888\\
10	0.0127929792265886\\
11	0.0127929776756093\\
12	0.0127929760985203\\
13	0.0127929744948836\\
14	0.012792972864254\\
15	0.0127929712061789\\
16	0.0127929695201981\\
17	0.0127929678058438\\
18	0.0127929660626403\\
19	0.0127929642901041\\
20	0.0127929624877435\\
21	0.0127929606550588\\
22	0.0127929587915418\\
23	0.012792956896676\\
24	0.0127929549699361\\
25	0.0127929530107883\\
26	0.0127929510186898\\
27	0.0127929489930888\\
28	0.0127929469334242\\
29	0.0127929448391258\\
30	0.0127929427096137\\
31	0.0127929405442986\\
32	0.0127929383425812\\
33	0.0127929361038523\\
34	0.0127929338274927\\
35	0.0127929315128728\\
36	0.0127929291593526\\
37	0.0127929267662814\\
38	0.0127929243329978\\
39	0.0127929218588295\\
40	0.0127929193430928\\
41	0.0127929167850929\\
42	0.0127929141841235\\
43	0.0127929115394664\\
44	0.0127929088503916\\
45	0.0127929061161572\\
46	0.0127929033360086\\
47	0.0127929005091791\\
48	0.0127928976348892\\
49	0.0127928947123465\\
50	0.0127928917407453\\
51	0.012792888719267\\
52	0.0127928856470791\\
53	0.0127928825233355\\
54	0.0127928793471761\\
55	0.0127928761177267\\
56	0.0127928728340985\\
57	0.0127928694953882\\
58	0.0127928661006774\\
59	0.0127928626490328\\
60	0.0127928591395054\\
61	0.012792855571131\\
62	0.0127928519429291\\
63	0.0127928482539032\\
64	0.0127928445030406\\
65	0.0127928406893116\\
66	0.0127928368116697\\
67	0.0127928328690513\\
68	0.0127928288603752\\
69	0.0127928247845423\\
70	0.0127928206404358\\
71	0.0127928164269201\\
72	0.0127928121428414\\
73	0.0127928077870268\\
74	0.0127928033582839\\
75	0.0127927988554011\\
76	0.0127927942771468\\
77	0.0127927896222691\\
78	0.0127927848894959\\
79	0.0127927800775338\\
80	0.0127927751850687\\
81	0.0127927702107647\\
82	0.012792765153264\\
83	0.0127927600111868\\
84	0.0127927547831306\\
85	0.01279274946767\\
86	0.0127927440633562\\
87	0.012792738568717\\
88	0.0127927329822558\\
89	0.0127927273024518\\
90	0.0127927215277594\\
91	0.0127927156566074\\
92	0.0127927096873995\\
93	0.0127927036185128\\
94	0.0127926974482984\\
95	0.0127926911750802\\
96	0.0127926847971548\\
97	0.0127926783127911\\
98	0.0127926717202297\\
99	0.0127926650176826\\
100	0.0127926582033325\\
101	0.0127926512753326\\
102	0.0127926442318059\\
103	0.0127926370708448\\
104	0.0127926297905106\\
105	0.012792622388833\\
106	0.0127926148638093\\
107	0.0127926072134045\\
108	0.0127925994355501\\
109	0.0127925915281439\\
110	0.0127925834890493\\
111	0.0127925753160948\\
112	0.0127925670070733\\
113	0.0127925585597417\\
114	0.0127925499718201\\
115	0.0127925412409911\\
116	0.0127925323648993\\
117	0.0127925233411507\\
118	0.012792514167312\\
119	0.0127925048409095\\
120	0.012792495359429\\
121	0.0127924857203148\\
122	0.0127924759209686\\
123	0.0127924659587495\\
124	0.0127924558309725\\
125	0.012792445534908\\
126	0.0127924350677809\\
127	0.0127924244267698\\
128	0.0127924136090061\\
129	0.0127924026115729\\
130	0.0127923914315044\\
131	0.0127923800657848\\
132	0.0127923685113468\\
133	0.0127923567650716\\
134	0.0127923448237867\\
135	0.0127923326842658\\
136	0.0127923203432267\\
137	0.0127923077973311\\
138	0.0127922950431825\\
139	0.0127922820773255\\
140	0.0127922688962442\\
141	0.0127922554963612\\
142	0.0127922418740356\\
143	0.0127922280255621\\
144	0.0127922139471691\\
145	0.0127921996350176\\
146	0.0127921850851988\\
147	0.0127921702937333\\
148	0.0127921552565688\\
149	0.0127921399695783\\
150	0.0127921244285586\\
151	0.0127921086292277\\
152	0.0127920925672236\\
153	0.0127920762381015\\
154	0.012792059637332\\
155	0.0127920427602989\\
156	0.0127920256022968\\
157	0.0127920081585285\\
158	0.012791990424103\\
159	0.0127919723940329\\
160	0.0127919540632313\\
161	0.0127919354265099\\
162	0.012791916478576\\
163	0.0127918972140295\\
164	0.0127918776273604\\
165	0.0127918577129463\\
166	0.0127918374650486\\
167	0.0127918168778108\\
168	0.0127917959452548\\
169	0.0127917746612785\\
170	0.0127917530196529\\
171	0.0127917310140195\\
172	0.0127917086378874\\
173	0.0127916858846313\\
174	0.0127916627474886\\
175	0.0127916392195576\\
176	0.0127916152937956\\
177	0.0127915909630169\\
178	0.0127915662198916\\
179	0.0127915410569447\\
180	0.0127915154665551\\
181	0.0127914894409557\\
182	0.0127914629722335\\
183	0.01279143605233\\
184	0.0127914086730433\\
185	0.0127913808260287\\
186	0.0127913525028021\\
187	0.0127913236947424\\
188	0.012791294393095\\
189	0.0127912645889755\\
190	0.0127912342733744\\
191	0.0127912034371615\\
192	0.0127911720710905\\
193	0.0127911401658038\\
194	0.0127911077118359\\
195	0.0127910746996145\\
196	0.0127910411194561\\
197	0.0127910069615579\\
198	0.0127909722160215\\
199	0.0127909368727825\\
200	0.0127909009216049\\
201	0.0127908643520776\\
202	0.0127908271536117\\
203	0.0127907893154376\\
204	0.0127907508266016\\
205	0.0127907116759629\\
206	0.0127906718521901\\
207	0.0127906313437582\\
208	0.0127905901389453\\
209	0.0127905482258289\\
210	0.0127905055922825\\
211	0.0127904622259724\\
212	0.0127904181143535\\
213	0.0127903732446662\\
214	0.0127903276039325\\
215	0.0127902811789519\\
216	0.012790233956298\\
217	0.0127901859223143\\
218	0.01279013706311\\
219	0.0127900873645565\\
220	0.0127900368122826\\
221	0.0127899853916706\\
222	0.0127899330878521\\
223	0.0127898798857032\\
224	0.0127898257698403\\
225	0.0127897707246156\\
226	0.0127897147341123\\
227	0.0127896577821399\\
228	0.0127895998522294\\
229	0.0127895409276283\\
230	0.0127894809912959\\
231	0.0127894200258978\\
232	0.0127893580138008\\
233	0.0127892949370681\\
234	0.012789230777453\\
235	0.0127891655163943\\
236	0.0127890991350101\\
237	0.0127890316140924\\
238	0.0127889629341011\\
239	0.0127888930751582\\
240	0.0127888220170417\\
241	0.0127887497391793\\
242	0.0127886762206425\\
243	0.0127886014401395\\
244	0.0127885253760096\\
245	0.0127884480062155\\
246	0.0127883693083373\\
247	0.0127882892595651\\
248	0.0127882078366921\\
249	0.0127881250161072\\
250	0.0127880407737878\\
251	0.0127879550852921\\
252	0.0127878679257514\\
253	0.0127877792698621\\
254	0.0127876890918782\\
255	0.0127875973656023\\
256	0.0127875040643776\\
257	0.0127874091610794\\
258	0.012787312628106\\
259	0.01278721443737\\
260	0.0127871145602889\\
261	0.0127870129677757\\
262	0.0127869096302293\\
263	0.0127868045175246\\
264	0.0127866975990026\\
265	0.0127865888434598\\
266	0.0127864782191376\\
267	0.0127863656937116\\
268	0.0127862512342805\\
269	0.0127861348073544\\
270	0.0127860163788436\\
271	0.0127858959140459\\
272	0.0127857733776352\\
273	0.0127856487336482\\
274	0.0127855219454717\\
275	0.0127853929758292\\
276	0.0127852617867676\\
277	0.0127851283396429\\
278	0.0127849925951058\\
279	0.0127848545130872\\
280	0.0127847140527828\\
281	0.0127845711726375\\
282	0.0127844258303296\\
283	0.0127842779827541\\
284	0.0127841275860058\\
285	0.0127839745953618\\
286	0.0127838189652637\\
287	0.0127836606492988\\
288	0.0127834996001816\\
289	0.0127833357697337\\
290	0.0127831691088637\\
291	0.0127829995675466\\
292	0.0127828270948023\\
293	0.0127826516386731\\
294	0.0127824731462014\\
295	0.012782291563406\\
296	0.0127821068352577\\
297	0.0127819189056544\\
298	0.0127817277173949\\
299	0.0127815332121529\\
300	0.0127813353304485\\
301	0.0127811340116206\\
302	0.0127809291937968\\
303	0.0127807208138632\\
304	0.0127805088074332\\
305	0.0127802931088144\\
306	0.0127800736509757\\
307	0.0127798503655119\\
308	0.0127796231826082\\
309	0.0127793920310026\\
310	0.0127791568379478\\
311	0.0127789175291708\\
312	0.0127786740288324\\
313	0.0127784262594837\\
314	0.0127781741420226\\
315	0.0127779175956476\\
316	0.0127776565378107\\
317	0.0127773908841681\\
318	0.0127771205485295\\
319	0.0127768454428058\\
320	0.0127765654769539\\
321	0.0127762805589213\\
322	0.0127759905945869\\
323	0.0127756954877008\\
324	0.0127753951398218\\
325	0.0127750894502525\\
326	0.0127747783159723\\
327	0.0127744616315679\\
328	0.0127741392891616\\
329	0.0127738111783369\\
330	0.0127734771860616\\
331	0.0127731371966085\\
332	0.0127727910914735\\
333	0.01277243874929\\
334	0.0127720800457421\\
335	0.0127717148534735\\
336	0.0127713430419948\\
337	0.0127709644775862\\
338	0.0127705790231993\\
339	0.012770186538354\\
340	0.0127697868790338\\
341	0.0127693798975773\\
342	0.0127689654425666\\
343	0.0127685433587135\\
344	0.0127681134867421\\
345	0.0127676756632684\\
346	0.0127672297206776\\
347	0.0127667754869984\\
348	0.0127663127857746\\
349	0.0127658414359342\\
350	0.0127653612516568\\
351	0.0127648720422383\\
352	0.0127643736119541\\
353	0.0127638657599209\\
354	0.0127633482799573\\
355	0.012762820960444\\
356	0.0127622835841832\\
357	0.012761735928259\\
358	0.0127611777638984\\
359	0.0127606088563344\\
360	0.0127600289646714\\
361	0.0127594378417537\\
362	0.0127588352340393\\
363	0.012758220881479\\
364	0.0127575945174025\\
365	0.0127569558684133\\
366	0.0127563046542935\\
367	0.0127556405879219\\
368	0.0127549633752047\\
369	0.0127542727150254\\
370	0.0127535682992122\\
371	0.0127528498125304\\
372	0.0127521169326994\\
373	0.0127513693304417\\
374	0.0127506066695671\\
375	0.0127498286070967\\
376	0.0127490347934348\\
377	0.0127482248725942\\
378	0.0127473984824831\\
379	0.0127465552552642\\
380	0.0127456948177941\\
381	0.0127448167921566\\
382	0.0127439207963021\\
383	0.0127430064448041\\
384	0.012742073349745\\
385	0.0127411211217393\\
386	0.0127401493711167\\
387	0.012739157709372\\
388	0.0127381457513711\\
389	0.0127371131201528\\
390	0.0127360594423528\\
391	0.0127349843552758\\
392	0.0127338875118904\\
393	0.0127327685852119\\
394	0.0127316272714285\\
395	0.0127304632856456\\
396	0.0127292763875721\\
397	0.0127280663904961\\
398	0.012726833150965\\
399	0.0127255765731411\\
400	0.0127242966105811\\
401	0.0127229932632295\\
402	0.0127216665707705\\
403	0.0127203166263298\\
404	0.0127189437435923\\
405	0.0127175481960241\\
406	0.0127161295954536\\
407	0.012714687532925\\
408	0.0127132215866604\\
409	0.0127117313210684\\
410	0.0127102162856905\\
411	0.0127086760141024\\
412	0.0127071100228092\\
413	0.0127055178101971\\
414	0.0127038988556273\\
415	0.0127022526187338\\
416	0.0127005785388669\\
417	0.0126988760349283\\
418	0.0126971445116799\\
419	0.0126953834233624\\
420	0.0126935922143988\\
421	0.0126917703081474\\
422	0.0126899171042727\\
423	0.0126880319751953\\
424	0.0126861142606266\\
425	0.0126841632578285\\
426	0.0126821782024126\\
427	0.0126801582305813\\
428	0.0126781026854419\\
429	0.0126760109028805\\
430	0.0126738820307764\\
431	0.012671715277872\\
432	0.0126695100338271\\
433	0.0126672643388395\\
434	0.0126649760781305\\
435	0.0126626440029699\\
436	0.0126602667674305\\
437	0.012657842907867\\
438	0.0126553707916276\\
439	0.0126528485190949\\
440	0.0126502744425797\\
441	0.0126476471696468\\
442	0.0126449652765735\\
443	0.0126422273153552\\
444	0.012639431823736\\
445	0.0126365773392313\\
446	0.0126336624183972\\
447	0.0126306856629217\\
448	0.0126276457543936\\
449	0.0126245414995428\\
450	0.0126213718863682\\
451	0.0126181361459287\\
452	0.012614833795841\\
453	0.0126114645838651\\
454	0.0126080280793682\\
455	0.012604522172994\\
456	0.0126009465470349\\
457	0.0125959707843802\\
458	0.0125907847269098\\
459	0.0125855907676074\\
460	0.0125803992631018\\
461	0.0125752072479814\\
462	0.0125699213757532\\
463	0.0125645386175389\\
464	0.0125590556880374\\
465	0.0125534689939647\\
466	0.0125477745882916\\
467	0.0125419681164783\\
468	0.0125360447533458\\
469	0.0125299991293525\\
470	0.012523825246106\\
471	0.0125175163548903\\
472	0.0125110648195536\\
473	0.0125044619735459\\
474	0.0124976979292119\\
475	0.0124907613433337\\
476	0.0124836391248707\\
477	0.012476316062334\\
478	0.0124687745419361\\
479	0.0124609939533591\\
480	0.0124529500114997\\
481	0.0124446139091603\\
482	0.0124359509911232\\
483	0.012426918794812\\
484	0.0124174629365302\\
485	0.012407506191671\\
486	0.0123953754340449\\
487	0.0123730025552894\\
488	0.0123502157826629\\
489	0.0123270054750613\\
490	0.0123033616289481\\
491	0.0122792738628729\\
492	0.0122547314024656\\
493	0.0122297230661976\\
494	0.0122042372522723\\
495	0.0121782619271292\\
496	0.0121517846162918\\
497	0.0121247923990123\\
498	0.0120972719094384\\
499	0.0120692093531988\\
500	0.0120405905413115\\
501	0.012011400956291\\
502	0.0119816258239681\\
503	0.0119512494044773\\
504	0.0119202583698927\\
505	0.0118886477107224\\
506	0.0118564281356895\\
507	0.0118236678927037\\
508	0.0118083044304957\\
509	0.0117949512162318\\
510	0.0117813795387892\\
511	0.0117675871516953\\
512	0.0117535813066232\\
513	0.0117393716166701\\
514	0.0117249531908821\\
515	0.0117103083315483\\
516	0.0116955111774054\\
517	0.0116806120037169\\
518	0.0116656720414887\\
519	0.0116507048545402\\
520	0.0116357214973517\\
521	0.0116207871970443\\
522	0.0116059855372909\\
523	0.0115914234563041\\
524	0.0115770818101273\\
525	0.011562231204689\\
526	0.0115467564795575\\
527	0.0115304506965384\\
528	0.0115135466893872\\
529	0.0114960040829992\\
530	0.0114777778829336\\
531	0.0114588120208159\\
532	0.0114390506818788\\
533	0.0114184308008632\\
534	0.011396877210855\\
535	0.0113742460102425\\
536	0.0113504353777355\\
537	0.0113253189725327\\
538	0.0112991495883925\\
539	0.0112714730354775\\
540	0.0112420912224632\\
541	0.0112108193672537\\
542	0.0111774211860306\\
543	0.0111415588186748\\
544	0.0110865857833767\\
545	0.0110241580600701\\
546	0.0109583457434496\\
547	0.010888416417644\\
548	0.0108052632907422\\
549	0.0106299429228413\\
550	0.0104510105857241\\
551	0.0102682287949327\\
552	0.0100804500499112\\
553	0.00990956786620671\\
554	0.00983838302665969\\
555	0.00977038980910216\\
556	0.00970680215494745\\
557	0.00964931687322562\\
558	0.009599162085229\\
559	0.00954967206260671\\
560	0.00950071602098221\\
561	0.00945204581332923\\
562	0.00940398971229489\\
563	0.00935663222758508\\
564	0.00930699225591728\\
565	0.00925693302201336\\
566	0.00920676347091189\\
567	0.00915704191852477\\
568	0.00910875713763971\\
569	0.00905921407971458\\
570	0.00900853798820848\\
571	0.00895649782086392\\
572	0.00890079172342068\\
573	0.00884574283971147\\
574	0.00879414734479222\\
575	0.00874417034575634\\
576	0.00869347303764065\\
577	0.00858507079136236\\
578	0.00839695950345147\\
579	0.00789852302704922\\
580	0.00754640048314225\\
581	0.00745926952565165\\
582	0.00737820372283263\\
583	0.00729600763418925\\
584	0.00721233608898569\\
585	0.00712712428178828\\
586	0.00704031063046766\\
587	0.00695182476096253\\
588	0.00686156497012343\\
589	0.00676934696886435\\
590	0.00667477264796359\\
591	0.00657713355197374\\
592	0.00647481662499991\\
593	0.00636364764633835\\
594	0.00623271716130668\\
595	0.0060534099414828\\
596	0.00575058001197163\\
597	0.00512683753504545\\
598	0.00366374385960312\\
599	0\\
600	0\\
};
\addplot [color=blue!40!mycolor9,solid,forget plot]
  table[row sep=crcr]{%
1	0.0131062123008924\\
2	0.0131062097189333\\
3	0.0131062070935401\\
4	0.0131062044239834\\
5	0.013106201709521\\
6	0.0131061989493986\\
7	0.0131061961428491\\
8	0.0131061932890925\\
9	0.0131061903873358\\
10	0.0131061874367725\\
11	0.0131061844365826\\
12	0.0131061813859325\\
13	0.0131061782839744\\
14	0.0131061751298463\\
15	0.0131061719226716\\
16	0.0131061686615592\\
17	0.0131061653456028\\
18	0.0131061619738811\\
19	0.0131061585454569\\
20	0.0131061550593777\\
21	0.0131061515146748\\
22	0.013106147910363\\
23	0.0131061442454409\\
24	0.01310614051889\\
25	0.0131061367296746\\
26	0.0131061328767419\\
27	0.0131061289590211\\
28	0.0131061249754234\\
29	0.0131061209248418\\
30	0.0131061168061506\\
31	0.0131061126182051\\
32	0.0131061083598414\\
33	0.0131061040298759\\
34	0.0131060996271052\\
35	0.0131060951503055\\
36	0.0131060905982325\\
37	0.0131060859696209\\
38	0.0131060812631838\\
39	0.0131060764776132\\
40	0.0131060716115784\\
41	0.0131060666637267\\
42	0.0131060616326823\\
43	0.0131060565170466\\
44	0.0131060513153969\\
45	0.0131060460262868\\
46	0.0131060406482455\\
47	0.0131060351797772\\
48	0.013106029619361\\
49	0.0131060239654501\\
50	0.0131060182164719\\
51	0.013106012370827\\
52	0.0131060064268889\\
53	0.0131060003830039\\
54	0.0131059942374901\\
55	0.0131059879886372\\
56	0.013105981634706\\
57	0.0131059751739278\\
58	0.0131059686045042\\
59	0.0131059619246061\\
60	0.0131059551323734\\
61	0.0131059482259147\\
62	0.0131059412033064\\
63	0.0131059340625925\\
64	0.0131059268017835\\
65	0.0131059194188564\\
66	0.0131059119117538\\
67	0.0131059042783835\\
68	0.0131058965166175\\
69	0.013105888624292\\
70	0.0131058805992063\\
71	0.0131058724391221\\
72	0.0131058641417634\\
73	0.0131058557048152\\
74	0.0131058471259232\\
75	0.0131058384026932\\
76	0.0131058295326898\\
77	0.0131058205134367\\
78	0.0131058113424148\\
79	0.0131058020170624\\
80	0.0131057925347741\\
81	0.0131057828928998\\
82	0.0131057730887443\\
83	0.0131057631195664\\
84	0.0131057529825779\\
85	0.013105742674943\\
86	0.0131057321937775\\
87	0.0131057215361474\\
88	0.013105710699069\\
89	0.013105699679507\\
90	0.0131056884743742\\
91	0.0131056770805305\\
92	0.0131056654947818\\
93	0.0131056537138792\\
94	0.0131056417345177\\
95	0.0131056295533356\\
96	0.0131056171669134\\
97	0.0131056045717727\\
98	0.0131055917643748\\
99	0.0131055787411203\\
100	0.0131055654983475\\
101	0.0131055520323315\\
102	0.013105538339283\\
103	0.0131055244153468\\
104	0.0131055102566015\\
105	0.0131054958590572\\
106	0.013105481218655\\
107	0.0131054663312656\\
108	0.0131054511926879\\
109	0.0131054357986475\\
110	0.0131054201447959\\
111	0.0131054042267086\\
112	0.013105388039884\\
113	0.013105371579742\\
114	0.0131053548416222\\
115	0.0131053378207831\\
116	0.0131053205123996\\
117	0.0131053029115626\\
118	0.0131052850132764\\
119	0.0131052668124578\\
120	0.0131052483039343\\
121	0.013105229482442\\
122	0.0131052103426246\\
123	0.0131051908790311\\
124	0.0131051710861142\\
125	0.0131051509582288\\
126	0.0131051304896297\\
127	0.0131051096744697\\
128	0.0131050885067981\\
129	0.0131050669805585\\
130	0.0131050450895869\\
131	0.0131050228276092\\
132	0.01310500018824\\
133	0.0131049771649795\\
134	0.0131049537512121\\
135	0.0131049299402039\\
136	0.0131049057251003\\
137	0.0131048810989239\\
138	0.0131048560545722\\
139	0.0131048305848152\\
140	0.0131048046822928\\
141	0.0131047783395126\\
142	0.0131047515488471\\
143	0.0131047243025316\\
144	0.013104696592661\\
145	0.0131046684111877\\
146	0.0131046397499186\\
147	0.0131046106005124\\
148	0.0131045809544773\\
149	0.0131045508031673\\
150	0.0131045201377802\\
151	0.0131044889493543\\
152	0.0131044572287656\\
153	0.0131044249667248\\
154	0.0131043921537743\\
155	0.0131043587802853\\
156	0.0131043248364546\\
157	0.0131042903123017\\
158	0.0131042551976655\\
159	0.0131042194822013\\
160	0.0131041831553779\\
161	0.0131041462064738\\
162	0.0131041086245749\\
163	0.0131040703985704\\
164	0.0131040315171507\\
165	0.0131039919688031\\
166	0.0131039517418096\\
167	0.013103910824243\\
168	0.0131038692039644\\
169	0.0131038268686193\\
170	0.0131037838056353\\
171	0.0131037400022184\\
172	0.01310369544535\\
173	0.013103650121784\\
174	0.0131036040180436\\
175	0.0131035571204182\\
176	0.0131035094149604\\
177	0.013103460887483\\
178	0.0131034115235556\\
179	0.0131033613085019\\
180	0.0131033102273963\\
181	0.0131032582650607\\
182	0.0131032054060611\\
183	0.0131031516347044\\
184	0.0131030969350345\\
185	0.013103041290829\\
186	0.0131029846855945\\
187	0.0131029271025632\\
188	0.0131028685246876\\
189	0.013102808934636\\
190	0.013102748314787\\
191	0.013102686647224\\
192	0.0131026239137285\\
193	0.0131025600957738\\
194	0.0131024951745173\\
195	0.013102429130793\\
196	0.013102361945103\\
197	0.0131022935976096\\
198	0.0131022240681262\\
199	0.0131021533361108\\
200	0.0131020813806595\\
201	0.0131020081805004\\
202	0.0131019337139861\\
203	0.0131018579590877\\
204	0.0131017808933873\\
205	0.0131017024940715\\
206	0.0131016227379234\\
207	0.013101541601316\\
208	0.0131014590602043\\
209	0.0131013750901179\\
210	0.0131012896661529\\
211	0.0131012027629645\\
212	0.0131011143547584\\
213	0.0131010244152828\\
214	0.0131009329178199\\
215	0.0131008398351778\\
216	0.0131007451396809\\
217	0.013100648803162\\
218	0.0131005507969523\\
219	0.013100451091873\\
220	0.0131003496582252\\
221	0.0131002464657809\\
222	0.0131001414837726\\
223	0.0131000346808838\\
224	0.0130999260252386\\
225	0.0130998154843912\\
226	0.0130997030253156\\
227	0.0130995886143948\\
228	0.0130994722174093\\
229	0.0130993537995267\\
230	0.0130992333252894\\
231	0.0130991107586034\\
232	0.0130989860627266\\
233	0.0130988592002558\\
234	0.0130987301331155\\
235	0.0130985988225441\\
236	0.0130984652290819\\
237	0.0130983293125573\\
238	0.013098191032074\\
239	0.0130980503459966\\
240	0.0130979072119374\\
241	0.0130977615867416\\
242	0.0130976134264728\\
243	0.0130974626863987\\
244	0.0130973093209751\\
245	0.0130971532838313\\
246	0.0130969945277537\\
247	0.0130968330046702\\
248	0.0130966686656334\\
249	0.0130965014608042\\
250	0.0130963313394342\\
251	0.0130961582498489\\
252	0.0130959821394293\\
253	0.0130958029545939\\
254	0.0130956206407805\\
255	0.0130954351424265\\
256	0.0130952464029504\\
257	0.0130950543647312\\
258	0.0130948589690888\\
259	0.0130946601562635\\
260	0.0130944578653943\\
261	0.013094252034498\\
262	0.0130940426004473\\
263	0.013093829498948\\
264	0.0130936126645164\\
265	0.0130933920304559\\
266	0.0130931675288334\\
267	0.0130929390904542\\
268	0.0130927066448383\\
269	0.0130924701201937\\
270	0.0130922294433916\\
271	0.0130919845399391\\
272	0.0130917353339526\\
273	0.0130914817481298\\
274	0.0130912237037218\\
275	0.0130909611205041\\
276	0.0130906939167472\\
277	0.0130904220091863\\
278	0.0130901453129908\\
279	0.0130898637417328\\
280	0.013089577207355\\
281	0.0130892856201379\\
282	0.0130889888886666\\
283	0.0130886869197962\\
284	0.0130883796186169\\
285	0.0130880668884187\\
286	0.0130877486306546\\
287	0.0130874247449034\\
288	0.0130870951288316\\
289	0.0130867596781547\\
290	0.0130864182865976\\
291	0.0130860708458535\\
292	0.0130857172455431\\
293	0.0130853573731718\\
294	0.0130849911140868\\
295	0.0130846183514328\\
296	0.0130842389661068\\
297	0.0130838528367124\\
298	0.0130834598395123\\
299	0.0130830598483807\\
300	0.0130826527347539\\
301	0.0130822383675805\\
302	0.01308181661327\\
303	0.0130813873356411\\
304	0.0130809503958676\\
305	0.0130805056524247\\
306	0.0130800529610334\\
307	0.0130795921746034\\
308	0.0130791231431758\\
309	0.013078645713864\\
310	0.0130781597307933\\
311	0.0130776650350397\\
312	0.0130771614645675\\
313	0.0130766488541653\\
314	0.0130761270353807\\
315	0.0130755958364548\\
316	0.0130750550822536\\
317	0.0130745045942002\\
318	0.0130739441902039\\
319	0.0130733736845894\\
320	0.0130727928880237\\
321	0.0130722016074424\\
322	0.0130715996459745\\
323	0.0130709868028659\\
324	0.0130703628734013\\
325	0.0130697276488255\\
326	0.0130690809162632\\
327	0.0130684224586368\\
328	0.0130677520545845\\
329	0.0130670694783756\\
330	0.0130663744998258\\
331	0.0130656668842106\\
332	0.0130649463921783\\
333	0.0130642127796611\\
334	0.0130634657977858\\
335	0.0130627051927833\\
336	0.0130619307058973\\
337	0.0130611420732919\\
338	0.0130603390259586\\
339	0.0130595212896225\\
340	0.0130586885846479\\
341	0.0130578406259437\\
342	0.0130569771228672\\
343	0.0130560977791293\\
344	0.0130552022926978\\
345	0.0130542903557015\\
346	0.0130533616543344\\
347	0.0130524158687594\\
348	0.0130514526730125\\
349	0.0130504717349081\\
350	0.0130494727159439\\
351	0.0130484552712069\\
352	0.0130474190492809\\
353	0.013046363692154\\
354	0.0130452888351284\\
355	0.0130441941067309\\
356	0.0130430791286257\\
357	0.0130419435155281\\
358	0.0130407868751212\\
359	0.013039608807974\\
360	0.0130384089074623\\
361	0.0130371867596918\\
362	0.013035941943424\\
363	0.0130346740300047\\
364	0.0130333825832956\\
365	0.013032067159609\\
366	0.0130307273076444\\
367	0.0130293625684301\\
368	0.0130279724752654\\
369	0.0130265565536678\\
370	0.0130251143213216\\
371	0.0130236452880295\\
372	0.0130221489556663\\
373	0.0130206248181345\\
374	0.0130190723613211\\
375	0.0130174910630551\\
376	0.0130158803930647\\
377	0.013014239812934\\
378	0.0130125687760568\\
379	0.0130108667275874\\
380	0.0130091331043857\\
381	0.0130073673349556\\
382	0.0130055688393732\\
383	0.0130037370292029\\
384	0.013001871307398\\
385	0.0129999710681821\\
386	0.0129980356969119\\
387	0.0129960645699217\\
388	0.0129940570543403\\
389	0.012992012507765\\
390	0.0129899302781134\\
391	0.0129878097032638\\
392	0.0129856501105441\\
393	0.0129834508160503\\
394	0.0129812111237702\\
395	0.012978930324836\\
396	0.0129766076959477\\
397	0.0129742424974519\\
398	0.0129718339716314\\
399	0.0129693813405619\\
400	0.0129668838033543\\
401	0.0129643405324378\\
402	0.012961750667796\\
403	0.0129591133040516\\
404	0.0129564274444927\\
405	0.0129536918679756\\
406	0.0129509058552706\\
407	0.0129480694158987\\
408	0.0129451816044291\\
409	0.0129422414651227\\
410	0.0129392480333532\\
411	0.0129362003376072\\
412	0.0129330974028044\\
413	0.0129299382569691\\
414	0.0129267219468194\\
415	0.0129234475775764\\
416	0.0129201144191165\\
417	0.0129167221947262\\
418	0.0129132718739799\\
419	0.0129097555658597\\
420	0.012906171085146\\
421	0.0129025172046721\\
422	0.0128987927050607\\
423	0.0128949963802688\\
424	0.0128911270441335\\
425	0.0128871835382241\\
426	0.0128831647418413\\
427	0.0128790695868142\\
428	0.012874897065317\\
429	0.0128706462456258\\
430	0.0128663162974818\\
431	0.0128619065136225\\
432	0.0128574163239453\\
433	0.0128528453643652\\
434	0.012848193529859\\
435	0.0128434610143885\\
436	0.0128386484115166\\
437	0.01283375687313\\
438	0.0128287883138137\\
439	0.0128237453121304\\
440	0.0128186124555014\\
441	0.012813376780429\\
442	0.012808036415159\\
443	0.012802589509648\\
444	0.0127970342466934\\
445	0.0127913688551719\\
446	0.0127855916257636\\
447	0.0127797009295912\\
448	0.012773695240214\\
449	0.012767573159328\\
450	0.0127613334461588\\
451	0.0127549750495214\\
452	0.0127484971391772\\
453	0.0127418991287986\\
454	0.0127351806789257\\
455	0.0127283416846906\\
456	0.0127213833968186\\
457	0.012714308182166\\
458	0.0127071175974745\\
459	0.0126998154614038\\
460	0.0126924110083193\\
461	0.0126849649695689\\
462	0.0126815145191367\\
463	0.0126779900021133\\
464	0.0126743884876139\\
465	0.012670707415893\\
466	0.0126669440878218\\
467	0.0126630956550081\\
468	0.0126591591094476\\
469	0.0126551312729997\\
470	0.0126510087853592\\
471	0.0126467880715858\\
472	0.0126424653298302\\
473	0.0126380365452358\\
474	0.012633497484306\\
475	0.0126288436920041\\
476	0.0126240704921423\\
477	0.0126191730039485\\
478	0.0126141462939614\\
479	0.0126089851664669\\
480	0.0126036838360823\\
481	0.0125982349089815\\
482	0.0125926372838476\\
483	0.0125868874724391\\
484	0.0125809801308669\\
485	0.0125749087287736\\
486	0.0125683235025073\\
487	0.0125594520239075\\
488	0.0125504280157903\\
489	0.0125412480341287\\
490	0.0125319083517077\\
491	0.0125224049054904\\
492	0.0125127332333894\\
493	0.012502888398347\\
494	0.0124928648972797\\
495	0.0124826565522084\\
496	0.0124722563810587\\
497	0.0124616564470727\\
498	0.0124508476907447\\
499	0.0124398197614155\\
500	0.0124285608946239\\
501	0.0124170579151011\\
502	0.012405296325065\\
503	0.0123932589960691\\
504	0.0123808206868512\\
505	0.0123677880674256\\
506	0.0123540612035565\\
507	0.0123394798057046\\
508	0.0123097390461796\\
509	0.0122774957760457\\
510	0.0122445595306122\\
511	0.0122109080176509\\
512	0.0121765185710484\\
513	0.0121413675045351\\
514	0.0121054300689422\\
515	0.0120686803431523\\
516	0.0120310919183313\\
517	0.011992641441605\\
518	0.0119533047159617\\
519	0.0119130471285004\\
520	0.0118718323983246\\
521	0.0118296329947836\\
522	0.0117864287093839\\
523	0.0117422289446107\\
524	0.0117012329633552\\
525	0.0116824680885595\\
526	0.0116633095787144\\
527	0.0116437397366907\\
528	0.0116238488887752\\
529	0.0116035371698284\\
530	0.0115828089704705\\
531	0.0115616813864128\\
532	0.0115401795632423\\
533	0.0115183388984585\\
534	0.0114962078795074\\
535	0.01147384555441\\
536	0.0114513452201563\\
537	0.0114288300780705\\
538	0.011405969759028\\
539	0.0113820128478913\\
540	0.0113568919014069\\
541	0.0113305207212212\\
542	0.011302802021007\\
543	0.0112736266560262\\
544	0.0112428814717731\\
545	0.011210408426141\\
546	0.0111760045349491\\
547	0.0111393997893615\\
548	0.0110985141478726\\
549	0.0110359958009047\\
550	0.0109703350304709\\
551	0.0109010455269748\\
552	0.0108272236335095\\
553	0.0107311067430109\\
554	0.0105521204355113\\
555	0.0103686484462171\\
556	0.0101803219317228\\
557	0.00998651054621628\\
558	0.00979370685284906\\
559	0.00971537036235166\\
560	0.00963954231455041\\
561	0.00956767902320676\\
562	0.00950150185643653\\
563	0.00944190396085486\\
564	0.00938514629032857\\
565	0.00932901308938975\\
566	0.0092735379510463\\
567	0.00921824308278224\\
568	0.00916273140610261\\
569	0.00910719383632812\\
570	0.00905161174129055\\
571	0.00899684129174705\\
572	0.00894254650472999\\
573	0.00888762933099036\\
574	0.00882982792747635\\
575	0.00877077156997362\\
576	0.00871241435043279\\
577	0.00865852168006084\\
578	0.00859208551626859\\
579	0.00847571802280514\\
580	0.00822405621502198\\
581	0.00773536955592913\\
582	0.00739020877460958\\
583	0.00729759652126533\\
584	0.00721310701082822\\
585	0.0071275233835895\\
586	0.00704052757817003\\
587	0.00695193522767474\\
588	0.00686161434916973\\
589	0.00676936536373383\\
590	0.00667477774794946\\
591	0.00657713435287329\\
592	0.00647481662499991\\
593	0.00636364764633836\\
594	0.00623271716130668\\
595	0.00605340994148281\\
596	0.00575058001197164\\
597	0.00512683753504545\\
598	0.00366374385960312\\
599	0\\
600	0\\
};
\addplot [color=blue!75!mycolor7,solid,forget plot]
  table[row sep=crcr]{%
1	0.013299675720752\\
2	0.0132996740620042\\
3	0.0132996723753582\\
4	0.0132996706603451\\
5	0.0132996689164881\\
6	0.0132996671433026\\
7	0.0132996653402954\\
8	0.0132996635069653\\
9	0.0132996616428026\\
10	0.013299659747289\\
11	0.0132996578198975\\
12	0.0132996558600922\\
13	0.0132996538673282\\
14	0.0132996518410513\\
15	0.0132996497806981\\
16	0.0132996476856956\\
17	0.0132996455554613\\
18	0.0132996433894027\\
19	0.0132996411869173\\
20	0.0132996389473927\\
21	0.0132996366702058\\
22	0.0132996343547233\\
23	0.0132996320003011\\
24	0.013299629606284\\
25	0.0132996271720062\\
26	0.0132996246967903\\
27	0.0132996221799476\\
28	0.0132996196207776\\
29	0.0132996170185683\\
30	0.0132996143725954\\
31	0.0132996116821224\\
32	0.0132996089464004\\
33	0.0132996061646678\\
34	0.0132996033361501\\
35	0.0132996004600598\\
36	0.0132995975355959\\
37	0.013299594561944\\
38	0.0132995915382758\\
39	0.0132995884637491\\
40	0.0132995853375073\\
41	0.0132995821586794\\
42	0.0132995789263796\\
43	0.0132995756397069\\
44	0.0132995722977453\\
45	0.0132995688995631\\
46	0.0132995654442129\\
47	0.013299561930731\\
48	0.0132995583581376\\
49	0.0132995547254362\\
50	0.0132995510316131\\
51	0.0132995472756377\\
52	0.0132995434564617\\
53	0.0132995395730191\\
54	0.0132995356242256\\
55	0.0132995316089785\\
56	0.0132995275261564\\
57	0.0132995233746187\\
58	0.0132995191532054\\
59	0.0132995148607368\\
60	0.0132995104960128\\
61	0.0132995060578131\\
62	0.0132995015448966\\
63	0.0132994969560007\\
64	0.0132994922898415\\
65	0.0132994875451131\\
66	0.0132994827204871\\
67	0.0132994778146127\\
68	0.0132994728261155\\
69	0.0132994677535981\\
70	0.0132994625956387\\
71	0.0132994573507914\\
72	0.0132994520175853\\
73	0.0132994465945244\\
74	0.0132994410800871\\
75	0.0132994354727253\\
76	0.0132994297708646\\
77	0.0132994239729035\\
78	0.0132994180772128\\
79	0.0132994120821353\\
80	0.0132994059859852\\
81	0.0132993997870477\\
82	0.0132993934835785\\
83	0.013299387073803\\
84	0.013299380555916\\
85	0.0132993739280812\\
86	0.0132993671884303\\
87	0.013299360335063\\
88	0.0132993533660459\\
89	0.013299346279412\\
90	0.0132993390731603\\
91	0.0132993317452552\\
92	0.0132993242936255\\
93	0.0132993167161641\\
94	0.0132993090107273\\
95	0.013299301175134\\
96	0.0132992932071653\\
97	0.0132992851045634\\
98	0.0132992768650312\\
99	0.0132992684862315\\
100	0.0132992599657864\\
101	0.0132992513012761\\
102	0.013299242490239\\
103	0.0132992335301697\\
104	0.0132992244185196\\
105	0.0132992151526949\\
106	0.0132992057300566\\
107	0.0132991961479191\\
108	0.0132991864035498\\
109	0.0132991764941679\\
110	0.0132991664169437\\
111	0.0132991561689977\\
112	0.0132991457473996\\
113	0.0132991351491673\\
114	0.0132991243712663\\
115	0.0132991134106081\\
116	0.0132991022640499\\
117	0.0132990909283932\\
118	0.0132990794003828\\
119	0.0132990676767059\\
120	0.0132990557539908\\
121	0.0132990436288063\\
122	0.0132990312976598\\
123	0.0132990187569971\\
124	0.0132990060032005\\
125	0.013298993032588\\
126	0.0132989798414122\\
127	0.0132989664258587\\
128	0.0132989527820455\\
129	0.013298938906021\\
130	0.0132989247937634\\
131	0.0132989104411789\\
132	0.013298895844101\\
133	0.0132988809982883\\
134	0.0132988658994241\\
135	0.0132988505431143\\
136	0.0132988349248863\\
137	0.0132988190401877\\
138	0.0132988028843846\\
139	0.0132987864527604\\
140	0.013298769740514\\
141	0.0132987527427587\\
142	0.0132987354545204\\
143	0.0132987178707359\\
144	0.013298699986252\\
145	0.0132986817958233\\
146	0.0132986632941105\\
147	0.0132986444756796\\
148	0.0132986253349993\\
149	0.0132986058664399\\
150	0.0132985860642716\\
151	0.0132985659226627\\
152	0.0132985454356775\\
153	0.0132985245972756\\
154	0.0132985034013089\\
155	0.0132984818415208\\
156	0.0132984599115439\\
157	0.0132984376048985\\
158	0.0132984149149905\\
159	0.0132983918351097\\
160	0.0132983683584281\\
161	0.0132983444779979\\
162	0.0132983201867494\\
163	0.0132982954774894\\
164	0.0132982703428994\\
165	0.0132982447755332\\
166	0.0132982187678151\\
167	0.0132981923120382\\
168	0.0132981654003621\\
169	0.0132981380248108\\
170	0.0132981101772708\\
171	0.013298081849489\\
172	0.0132980530330702\\
173	0.0132980237194754\\
174	0.0132979939000191\\
175	0.0132979635658671\\
176	0.0132979327080345\\
177	0.0132979013173826\\
178	0.0132978693846167\\
179	0.0132978369002837\\
180	0.0132978038547691\\
181	0.0132977702382942\\
182	0.0132977360409131\\
183	0.0132977012525102\\
184	0.0132976658627964\\
185	0.013297629861306\\
186	0.0132975932373936\\
187	0.0132975559802302\\
188	0.0132975180787999\\
189	0.0132974795218955\\
190	0.0132974402981155\\
191	0.0132974003958591\\
192	0.0132973598033231\\
193	0.0132973185084971\\
194	0.0132972764991594\\
195	0.0132972337628732\\
196	0.0132971902869821\\
197	0.013297146058606\\
198	0.0132971010646371\\
199	0.0132970552917359\\
200	0.0132970087263266\\
201	0.0132969613545932\\
202	0.0132969131624749\\
203	0.0132968641356615\\
204	0.0132968142595893\\
205	0.013296763519436\\
206	0.0132967119001162\\
207	0.0132966593862763\\
208	0.0132966059622901\\
209	0.0132965516122533\\
210	0.0132964963199788\\
211	0.0132964400689911\\
212	0.0132963828425212\\
213	0.0132963246235013\\
214	0.013296265394559\\
215	0.0132962051380119\\
216	0.0132961438358621\\
217	0.0132960814697897\\
218	0.0132960180211474\\
219	0.0132959534709544\\
220	0.0132958877998899\\
221	0.0132958209882872\\
222	0.0132957530161269\\
223	0.0132956838630307\\
224	0.0132956135082543\\
225	0.013295541930681\\
226	0.0132954691088147\\
227	0.0132953950207725\\
228	0.0132953196442779\\
229	0.013295242956653\\
230	0.0132951649348116\\
231	0.0132950855552508\\
232	0.013295004794044\\
233	0.0132949226268323\\
234	0.0132948390288167\\
235	0.0132947539747498\\
236	0.0132946674389274\\
237	0.0132945793951798\\
238	0.013294489816863\\
239	0.0132943986768499\\
240	0.0132943059475214\\
241	0.0132942116007563\\
242	0.0132941156079228\\
243	0.0132940179398683\\
244	0.0132939185669096\\
245	0.0132938174588227\\
246	0.013293714584833\\
247	0.0132936099136045\\
248	0.0132935034132292\\
249	0.0132933950512161\\
250	0.0132932847944803\\
251	0.013293172609332\\
252	0.013293058461464\\
253	0.0132929423159411\\
254	0.0132928241371873\\
255	0.0132927038889737\\
256	0.0132925815344065\\
257	0.0132924570359137\\
258	0.0132923303552325\\
259	0.0132922014533959\\
260	0.0132920702907192\\
261	0.0132919368267865\\
262	0.0132918010204365\\
263	0.013291662829748\\
264	0.0132915222120259\\
265	0.0132913791237858\\
266	0.0132912335207389\\
267	0.0132910853577772\\
268	0.0132909345889567\\
269	0.0132907811674824\\
270	0.0132906250456914\\
271	0.013290466175036\\
272	0.0132903045060676\\
273	0.0132901399884183\\
274	0.0132899725707841\\
275	0.0132898022009064\\
276	0.0132896288255542\\
277	0.0132894523905047\\
278	0.0132892728405251\\
279	0.0132890901193526\\
280	0.0132889041696748\\
281	0.0132887149331099\\
282	0.013288522350186\\
283	0.0132883263603201\\
284	0.0132881269017972\\
285	0.0132879239117483\\
286	0.0132877173261288\\
287	0.0132875070796957\\
288	0.013287293105985\\
289	0.0132870753372887\\
290	0.0132868537046304\\
291	0.0132866281377423\\
292	0.01328639856504\\
293	0.0132861649135978\\
294	0.0132859271091234\\
295	0.0132856850759324\\
296	0.0132854387369218\\
297	0.0132851880135434\\
298	0.0132849328257772\\
299	0.0132846730921032\\
300	0.0132844087294743\\
301	0.0132841396532871\\
302	0.0132838657773537\\
303	0.0132835870138722\\
304	0.0132833032733969\\
305	0.0132830144648082\\
306	0.0132827204952822\\
307	0.0132824212702593\\
308	0.0132821166934128\\
309	0.0132818066666168\\
310	0.0132814910899141\\
311	0.0132811698614832\\
312	0.0132808428776046\\
313	0.0132805100326276\\
314	0.0132801712189357\\
315	0.0132798263269123\\
316	0.013279475244905\\
317	0.0132791178591908\\
318	0.0132787540539397\\
319	0.0132783837111783\\
320	0.0132780067107532\\
321	0.0132776229302937\\
322	0.0132772322451739\\
323	0.0132768345284749\\
324	0.0132764296509462\\
325	0.0132760174809667\\
326	0.0132755978845057\\
327	0.0132751707250825\\
328	0.013274735863727\\
329	0.0132742931589386\\
330	0.0132738424666455\\
331	0.0132733836401631\\
332	0.0132729165301522\\
333	0.0132724409845768\\
334	0.0132719568486613\\
335	0.0132714639648473\\
336	0.0132709621727497\\
337	0.013270451309113\\
338	0.0132699312077659\\
339	0.0132694016995764\\
340	0.0132688626124059\\
341	0.0132683137710623\\
342	0.0132677549972528\\
343	0.013267186109536\\
344	0.0132666069232725\\
345	0.0132660172505756\\
346	0.0132654169002599\\
347	0.0132648056777896\\
348	0.013264183385225\\
349	0.0132635498211678\\
350	0.0132629047807052\\
351	0.0132622480553515\\
352	0.0132615794329885\\
353	0.0132608986978037\\
354	0.0132602056302259\\
355	0.013259500006858\\
356	0.0132587816004079\\
357	0.0132580501796152\\
358	0.0132573055091747\\
359	0.0132565473496562\\
360	0.01325577545742\\
361	0.0132549895845276\\
362	0.0132541894786472\\
363	0.0132533748829535\\
364	0.013252545536022\\
365	0.0132517011717154\\
366	0.0132508415190639\\
367	0.0132499663021366\\
368	0.0132490752399053\\
369	0.013248168046098\\
370	0.0132472444290431\\
371	0.0132463040915032\\
372	0.0132453467304968\\
373	0.0132443720371084\\
374	0.0132433796962854\\
375	0.0132423693866218\\
376	0.0132413407801275\\
377	0.013240293541982\\
378	0.0132392273302733\\
379	0.0132381417957202\\
380	0.0132370365813775\\
381	0.0132359113223241\\
382	0.0132347656453337\\
383	0.0132335991685267\\
384	0.0132324115010047\\
385	0.0132312022424661\\
386	0.0132299709828036\\
387	0.0132287173016828\\
388	0.0132274407681013\\
389	0.0132261409399316\\
390	0.0132248173634389\\
391	0.0132234695727793\\
392	0.0132220970894823\\
393	0.01322069942192\\
394	0.0132192760647692\\
395	0.013217826498462\\
396	0.0132163501886507\\
397	0.0132148465856894\\
398	0.0132133151241093\\
399	0.0132117552220694\\
400	0.0132101662807385\\
401	0.013208547683455\\
402	0.013206898794218\\
403	0.0132052189543544\\
404	0.0132035074752677\\
405	0.013201763623997\\
406	0.0131999867324026\\
407	0.0131981762026784\\
408	0.0131963312391025\\
409	0.0131944510159958\\
410	0.0131925346757647\\
411	0.0131905813269307\\
412	0.0131885900424134\\
413	0.0131865598587482\\
414	0.0131844897778328\\
415	0.0131823787746775\\
416	0.0131802258177056\\
417	0.0131780299097869\\
418	0.0131757901401662\\
419	0.0131735042328075\\
420	0.0131711709547028\\
421	0.013168789225222\\
422	0.013166357945463\\
423	0.0131638759998031\\
424	0.0131613422578505\\
425	0.0131587555768702\\
426	0.0131561148047549\\
427	0.0131534187834936\\
428	0.0131506663535254\\
429	0.0131478563588932\\
430	0.013144987653059\\
431	0.0131420591057864\\
432	0.0131390696121651\\
433	0.0131360181036858\\
434	0.0131329035668836\\
435	0.0131297250864497\\
436	0.0131264819667674\\
437	0.0131231741246124\\
438	0.013119803437367\\
439	0.0131163784993131\\
440	0.0131137359531825\\
441	0.0131116413008394\\
442	0.013109502564207\\
443	0.0131073184069381\\
444	0.0131050873694988\\
445	0.0131028078459677\\
446	0.0131004780558094\\
447	0.0130980960094271\\
448	0.0130956594659778\\
449	0.0130931658815169\\
450	0.0130906123449841\\
451	0.013087995498833\\
452	0.0130853114402517\\
453	0.0130825555980898\\
454	0.0130797225802734\\
455	0.0130768059868708\\
456	0.0130737981273337\\
457	0.0130706896151962\\
458	0.0130674687677329\\
459	0.0130641201752447\\
460	0.0130606208769816\\
461	0.0130569023893881\\
462	0.013049724837475\\
463	0.013042422609164\\
464	0.013034994567441\\
465	0.0130274155614741\\
466	0.013019682130087\\
467	0.0130117907274541\\
468	0.0130037376847426\\
469	0.012995519078651\\
470	0.0129871302867297\\
471	0.0129785689671273\\
472	0.0129698337951504\\
473	0.0129609211607287\\
474	0.0129518275786133\\
475	0.0129425499668079\\
476	0.0129330864017597\\
477	0.0129234381878213\\
478	0.012913584446632\\
479	0.0129035167419422\\
480	0.0128932304355082\\
481	0.0128827207061132\\
482	0.0128719837966622\\
483	0.0128610155427485\\
484	0.012849811625554\\
485	0.012838368064323\\
486	0.0128266819624904\\
487	0.0128147488374125\\
488	0.0128025584739185\\
489	0.0127901045718158\\
490	0.0127773808194113\\
491	0.0127643809417649\\
492	0.012751098762356\\
493	0.012737528281468\\
494	0.0127236637754498\\
495	0.0127094999222009\\
496	0.0126950319601255\\
497	0.0126802558916504\\
498	0.0126651687524712\\
499	0.0126497689991266\\
500	0.0126340571747312\\
501	0.0126180373935847\\
502	0.0126017215656472\\
503	0.0125851563441973\\
504	0.0125721011719379\\
505	0.0125639330051409\\
506	0.0125555005130236\\
507	0.0125467853759751\\
508	0.0125346611310508\\
509	0.0125218528537396\\
510	0.0125087480954591\\
511	0.012495330994115\\
512	0.012481583868136\\
513	0.0124674865405492\\
514	0.0124530158565289\\
515	0.0124381451063683\\
516	0.012422843315943\\
517	0.0124070742452977\\
518	0.0123907951739118\\
519	0.0123739555914239\\
520	0.0123564951170039\\
521	0.0123383397028977\\
522	0.0123193938071307\\
523	0.0122995191922814\\
524	0.0122752467093493\\
525	0.0122315354138963\\
526	0.0121869359452254\\
527	0.0121414467128114\\
528	0.0120950706271324\\
529	0.0120478054049151\\
530	0.0119995131456163\\
531	0.0119499286206226\\
532	0.0118989940718964\\
533	0.0118466471540871\\
534	0.0117928199997251\\
535	0.0117374369605637\\
536	0.0116804650467312\\
537	0.0116219596772739\\
538	0.0115666270542502\\
539	0.0115400648252121\\
540	0.0115127153882832\\
541	0.0114845647858405\\
542	0.0114556020442396\\
543	0.011425820358175\\
544	0.0113952192429664\\
545	0.0113637493192464\\
546	0.0113314022173966\\
547	0.0112982095221342\\
548	0.0112641817595797\\
549	0.0112291150552606\\
550	0.0111935039058502\\
551	0.0111561000943894\\
552	0.0111163453495756\\
553	0.0110701841428973\\
554	0.0110037507814572\\
555	0.0109341683335819\\
556	0.0108608936701748\\
557	0.0107831014065912\\
558	0.0106962357997787\\
559	0.0105129233178253\\
560	0.0103249861752007\\
561	0.0101317886664457\\
562	0.00993267883641647\\
563	0.00972720135460083\\
564	0.00960036565758178\\
565	0.0095149101347733\\
566	0.00943259613537835\\
567	0.00935440460098425\\
568	0.00928095805545381\\
569	0.00921567801913973\\
570	0.0091508731428269\\
571	0.00908578738650886\\
572	0.00902035159158171\\
573	0.00895696432559356\\
574	0.00889381827605941\\
575	0.00883133125598114\\
576	0.00877034863198628\\
577	0.00870594004185222\\
578	0.00864172161605577\\
579	0.00857843594035942\\
580	0.00849639660057001\\
581	0.00837926229131001\\
582	0.00812898741429046\\
583	0.00765189796216335\\
584	0.00723415767715551\\
585	0.00713458485247705\\
586	0.00704303138858332\\
587	0.00695323519973369\\
588	0.00686232363693157\\
589	0.00676970639866382\\
590	0.00667491749014196\\
591	0.00657717743770035\\
592	0.00647482429634717\\
593	0.00636364764633835\\
594	0.00623271716130668\\
595	0.00605340994148281\\
596	0.00575058001197163\\
597	0.00512683753504545\\
598	0.00366374385960312\\
599	0\\
600	0\\
};
\addplot [color=blue!80!mycolor9,solid,forget plot]
  table[row sep=crcr]{%
1	0.0134978178999\\
2	0.0134978172579578\\
3	0.0134978166052197\\
4	0.0134978159415042\\
5	0.0134978152666268\\
6	0.0134978145803998\\
7	0.0134978138826324\\
8	0.0134978131731304\\
9	0.0134978124516967\\
10	0.0134978117181304\\
11	0.0134978109722275\\
12	0.0134978102137807\\
13	0.0134978094425788\\
14	0.0134978086584073\\
15	0.0134978078610481\\
16	0.0134978070502793\\
17	0.0134978062258752\\
18	0.0134978053876066\\
19	0.0134978045352399\\
20	0.0134978036685382\\
21	0.01349780278726\\
22	0.0134978018911601\\
23	0.013497800979989\\
24	0.013497800053493\\
25	0.0134977991114141\\
26	0.01349779815349\\
27	0.0134977971794538\\
28	0.0134977961890343\\
29	0.0134977951819556\\
30	0.013497794157937\\
31	0.0134977931166933\\
32	0.0134977920579342\\
33	0.0134977909813647\\
34	0.0134977898866847\\
35	0.013497788773589\\
36	0.0134977876417672\\
37	0.0134977864909037\\
38	0.0134977853206775\\
39	0.0134977841307622\\
40	0.0134977829208257\\
41	0.0134977816905304\\
42	0.013497780439533\\
43	0.0134977791674841\\
44	0.0134977778740287\\
45	0.0134977765588055\\
46	0.0134977752214472\\
47	0.0134977738615801\\
48	0.0134977724788243\\
49	0.0134977710727933\\
50	0.013497769643094\\
51	0.0134977681893266\\
52	0.0134977667110846\\
53	0.0134977652079544\\
54	0.0134977636795155\\
55	0.01349776212534\\
56	0.0134977605449928\\
57	0.0134977589380314\\
58	0.0134977573040056\\
59	0.0134977556424576\\
60	0.0134977539529217\\
61	0.0134977522349243\\
62	0.0134977504879836\\
63	0.0134977487116094\\
64	0.0134977469053034\\
65	0.0134977450685585\\
66	0.0134977432008589\\
67	0.01349774130168\\
68	0.0134977393704882\\
69	0.0134977374067407\\
70	0.0134977354098852\\
71	0.0134977333793601\\
72	0.0134977313145939\\
73	0.0134977292150055\\
74	0.0134977270800034\\
75	0.0134977249089863\\
76	0.0134977227013422\\
77	0.0134977204564487\\
78	0.0134977181736724\\
79	0.0134977158523692\\
80	0.0134977134918836\\
81	0.0134977110915489\\
82	0.0134977086506867\\
83	0.013497706168607\\
84	0.0134977036446076\\
85	0.0134977010779743\\
86	0.0134976984679802\\
87	0.013497695813886\\
88	0.0134976931149394\\
89	0.013497690370375\\
90	0.013497687579414\\
91	0.0134976847412642\\
92	0.0134976818551193\\
93	0.013497678920159\\
94	0.0134976759355487\\
95	0.0134976729004391\\
96	0.0134976698139662\\
97	0.0134976666752506\\
98	0.0134976634833976\\
99	0.0134976602374969\\
100	0.0134976569366219\\
101	0.0134976535798301\\
102	0.0134976501661622\\
103	0.013497646694642\\
104	0.013497643164276\\
105	0.0134976395740536\\
106	0.013497635922946\\
107	0.0134976322099063\\
108	0.0134976284338693\\
109	0.0134976245937507\\
110	0.0134976206884474\\
111	0.0134976167168363\\
112	0.0134976126777749\\
113	0.0134976085701001\\
114	0.0134976043926285\\
115	0.0134976001441555\\
116	0.0134975958234553\\
117	0.0134975914292802\\
118	0.0134975869603604\\
119	0.0134975824154037\\
120	0.0134975777930947\\
121	0.0134975730920949\\
122	0.0134975683110417\\
123	0.0134975634485486\\
124	0.0134975585032043\\
125	0.0134975534735723\\
126	0.0134975483581907\\
127	0.0134975431555715\\
128	0.0134975378642004\\
129	0.0134975324825359\\
130	0.0134975270090093\\
131	0.013497521442024\\
132	0.0134975157799547\\
133	0.0134975100211477\\
134	0.0134975041639194\\
135	0.0134974982065567\\
136	0.0134974921473157\\
137	0.0134974859844219\\
138	0.0134974797160689\\
139	0.0134974733404187\\
140	0.0134974668556004\\
141	0.0134974602597101\\
142	0.01349745355081\\
143	0.0134974467269282\\
144	0.013497439786058\\
145	0.0134974327261571\\
146	0.0134974255451471\\
147	0.0134974182409133\\
148	0.0134974108113034\\
149	0.0134974032541275\\
150	0.0134973955671571\\
151	0.0134973877481245\\
152	0.0134973797947226\\
153	0.0134973717046035\\
154	0.0134973634753786\\
155	0.0134973551046174\\
156	0.0134973465898471\\
157	0.0134973379285518\\
158	0.013497329118172\\
159	0.0134973201561037\\
160	0.0134973110396975\\
161	0.0134973017662586\\
162	0.0134972923330452\\
163	0.0134972827372684\\
164	0.0134972729760911\\
165	0.0134972630466271\\
166	0.013497252945941\\
167	0.0134972426710464\\
168	0.0134972322189059\\
169	0.0134972215864299\\
170	0.0134972107704756\\
171	0.0134971997678466\\
172	0.0134971885752912\\
173	0.0134971771895024\\
174	0.0134971656071162\\
175	0.0134971538247108\\
176	0.0134971418388059\\
177	0.013497129645861\\
178	0.0134971172422751\\
179	0.013497104624385\\
180	0.0134970917884642\\
181	0.0134970787307219\\
182	0.0134970654473019\\
183	0.0134970519342809\\
184	0.0134970381876675\\
185	0.013497024203401\\
186	0.0134970099773498\\
187	0.0134969955053099\\
188	0.0134969807830041\\
189	0.01349696580608\\
190	0.0134969505701086\\
191	0.0134969350705833\\
192	0.0134969193029179\\
193	0.0134969032624452\\
194	0.013496886944416\\
195	0.013496870343997\\
196	0.0134968534562693\\
197	0.0134968362762275\\
198	0.0134968187987772\\
199	0.0134968010187341\\
200	0.0134967829308221\\
201	0.0134967645296715\\
202	0.0134967458098175\\
203	0.0134967267656986\\
204	0.0134967073916543\\
205	0.0134966876819239\\
206	0.0134966676306442\\
207	0.0134966472318479\\
208	0.0134966264794617\\
209	0.0134966053673042\\
210	0.013496583889084\\
211	0.0134965620383977\\
212	0.0134965398087279\\
213	0.0134965171934409\\
214	0.0134964941857848\\
215	0.0134964707788873\\
216	0.0134964469657533\\
217	0.0134964227392627\\
218	0.0134963980921684\\
219	0.0134963730170935\\
220	0.0134963475065292\\
221	0.0134963215528323\\
222	0.0134962951482228\\
223	0.013496268284781\\
224	0.0134962409544455\\
225	0.0134962131490101\\
226	0.0134961848601215\\
227	0.0134961560792762\\
228	0.013496126797818\\
229	0.0134960970069349\\
230	0.0134960666976567\\
231	0.0134960358608514\\
232	0.0134960044872227\\
233	0.0134959725673068\\
234	0.0134959400914694\\
235	0.013495907049902\\
236	0.0134958734326196\\
237	0.0134958392294564\\
238	0.0134958044300633\\
239	0.0134957690239038\\
240	0.013495733000251\\
241	0.013495696348184\\
242	0.0134956590565837\\
243	0.0134956211141301\\
244	0.0134955825092976\\
245	0.0134955432303518\\
246	0.0134955032653452\\
247	0.0134954626021137\\
248	0.013495421228272\\
249	0.0134953791312099\\
250	0.0134953362980878\\
251	0.0134952927158327\\
252	0.0134952483711337\\
253	0.0134952032504375\\
254	0.0134951573399441\\
255	0.0134951106256019\\
256	0.0134950630931033\\
257	0.0134950147278797\\
258	0.0134949655150968\\
259	0.0134949154396496\\
260	0.0134948644861573\\
261	0.0134948126389584\\
262	0.0134947598821051\\
263	0.0134947061993585\\
264	0.0134946515741826\\
265	0.0134945959897392\\
266	0.0134945394288824\\
267	0.0134944818741525\\
268	0.0134944233077706\\
269	0.0134943637116323\\
270	0.0134943030673021\\
271	0.0134942413560073\\
272	0.0134941785586313\\
273	0.0134941146557079\\
274	0.0134940496274146\\
275	0.0134939834535662\\
276	0.0134939161136079\\
277	0.0134938475866091\\
278	0.013493777851256\\
279	0.0134937068858451\\
280	0.0134936346682759\\
281	0.0134935611760438\\
282	0.0134934863862329\\
283	0.0134934102755085\\
284	0.0134933328201099\\
285	0.0134932539958423\\
286	0.0134931737780695\\
287	0.0134930921417059\\
288	0.0134930090612088\\
289	0.0134929245105699\\
290	0.0134928384633077\\
291	0.0134927508924586\\
292	0.0134926617705692\\
293	0.0134925710696874\\
294	0.0134924787613537\\
295	0.013492384816593\\
296	0.0134922892059054\\
297	0.0134921918992571\\
298	0.013492092866072\\
299	0.013491992075222\\
300	0.013491889495018\\
301	0.0134917850932007\\
302	0.0134916788369308\\
303	0.0134915706927797\\
304	0.01349146062672\\
305	0.0134913486041155\\
306	0.0134912345897113\\
307	0.0134911185476244\\
308	0.0134910004413329\\
309	0.0134908802336668\\
310	0.0134907578867968\\
311	0.0134906333622247\\
312	0.0134905066207731\\
313	0.0134903776225741\\
314	0.0134902463270597\\
315	0.0134901126929506\\
316	0.0134899766782455\\
317	0.0134898382402102\\
318	0.0134896973353671\\
319	0.0134895539194837\\
320	0.0134894079475616\\
321	0.0134892593738252\\
322	0.0134891081517108\\
323	0.0134889542338546\\
324	0.0134887975720812\\
325	0.0134886381173923\\
326	0.0134884758199544\\
327	0.0134883106290873\\
328	0.0134881424932514\\
329	0.0134879713600358\\
330	0.0134877971761456\\
331	0.0134876198873892\\
332	0.0134874394386652\\
333	0.0134872557739495\\
334	0.0134870688362815\\
335	0.01348687856775\\
336	0.0134866849094798\\
337	0.0134864878016161\\
338	0.0134862871833102\\
339	0.0134860829927034\\
340	0.0134858751669114\\
341	0.013485663642007\\
342	0.0134854483530034\\
343	0.0134852292338353\\
344	0.0134850062173405\\
345	0.0134847792352399\\
346	0.0134845482181164\\
347	0.0134843130953929\\
348	0.0134840737953097\\
349	0.0134838302448991\\
350	0.0134835823699599\\
351	0.0134833300950298\\
352	0.013483073343356\\
353	0.013482812036864\\
354	0.0134825460961245\\
355	0.0134822754403174\\
356	0.0134819999871948\\
357	0.0134817196530399\\
358	0.0134814343526238\\
359	0.0134811439991597\\
360	0.013480848504253\\
361	0.0134805477778486\\
362	0.0134802417281746\\
363	0.0134799302616814\\
364	0.0134796132829776\\
365	0.0134792906947609\\
366	0.0134789623977443\\
367	0.0134786282905782\\
368	0.013478288269766\\
369	0.013477942229576\\
370	0.0134775900619459\\
371	0.013477231656383\\
372	0.0134768668998567\\
373	0.0134764956766858\\
374	0.0134761178684185\\
375	0.0134757333537053\\
376	0.013475342008165\\
377	0.013474943704243\\
378	0.0134745383110612\\
379	0.0134741256942608\\
380	0.0134737057158351\\
381	0.0134732782339542\\
382	0.0134728431027795\\
383	0.0134724001722683\\
384	0.0134719492879671\\
385	0.0134714902907932\\
386	0.0134710230168033\\
387	0.0134705472969484\\
388	0.0134700629568116\\
389	0.0134695698163304\\
390	0.0134690676894971\\
391	0.0134685563840394\\
392	0.0134680357010739\\
393	0.0134675054347311\\
394	0.0134669653717453\\
395	0.0134664152910036\\
396	0.0134658549630444\\
397	0.0134652841494936\\
398	0.0134647026024256\\
399	0.0134641100636296\\
400	0.0134635062637539\\
401	0.0134628909212936\\
402	0.013462263741382\\
403	0.0134616244144063\\
404	0.0134609726147129\\
405	0.0134603080004289\\
406	0.0134596302091251\\
407	0.0134589388521121\\
408	0.0134582335196413\\
409	0.013457513779399\\
410	0.0134567791749599\\
411	0.0134560292242418\\
412	0.0134552634180291\\
413	0.0134544812186721\\
414	0.0134536820590924\\
415	0.013452865342125\\
416	0.0134520304396445\\
417	0.0134511766889951\\
418	0.0134503033799075\\
419	0.0134494098053107\\
420	0.0134484952161293\\
421	0.0134475588091908\\
422	0.0134465997209443\\
423	0.0134456170202447\\
424	0.0134446097000415\\
425	0.0134435766677741\\
426	0.0134425167342392\\
427	0.0134414286006498\\
428	0.0134403108435325\\
429	0.0134391618970388\\
430	0.0134379800321376\\
431	0.0134367633319616\\
432	0.0134355096621471\\
433	0.0134342166340143\\
434	0.0134328815553418\\
435	0.0134315013537595\\
436	0.0134300724257486\\
437	0.0134285902553484\\
438	0.0134270482693102\\
439	0.0134254340662547\\
440	0.0134230701205249\\
441	0.0134201546775523\\
442	0.0134171699905031\\
443	0.0134141137796265\\
444	0.0134109836592755\\
445	0.0134077771331828\\
446	0.0134044915899481\\
447	0.0134011242989055\\
448	0.0133976724065945\\
449	0.0133941329341324\\
450	0.013390502775878\\
451	0.0133867786998934\\
452	0.0133829573508258\\
453	0.0133790352558841\\
454	0.013375008834235\\
455	0.0133708744083719\\
456	0.0133666282131116\\
457	0.0133622663856774\\
458	0.0133577848874649\\
459	0.013353179229957\\
460	0.0133484436952482\\
461	0.0133435636803895\\
462	0.013337830751605\\
463	0.0133319767046786\\
464	0.0133260221190616\\
465	0.0133211591017198\\
466	0.0133162002847412\\
467	0.0133111432370257\\
468	0.0133059854032918\\
469	0.0133007240718036\\
470	0.0132953563194378\\
471	0.0132898794486607\\
472	0.0132842906941347\\
473	0.01327858670593\\
474	0.0132727639740842\\
475	0.013266818869552\\
476	0.0132607477374956\\
477	0.0132545470174358\\
478	0.0132482095608695\\
479	0.0132417303888866\\
480	0.0132351049762897\\
481	0.0132283285391178\\
482	0.0132213959758157\\
483	0.0132143018473306\\
484	0.0132070403490988\\
485	0.0131996052655958\\
486	0.0131919898787866\\
487	0.0131841869234585\\
488	0.0131761887111239\\
489	0.0131679869559688\\
490	0.0131595726981769\\
491	0.0131509362141319\\
492	0.0131420669108388\\
493	0.0131329532012644\\
494	0.0131235823564324\\
495	0.0131139403288144\\
496	0.0131040115393305\\
497	0.0130937786156154\\
498	0.0130832220576497\\
499	0.0130723197743846\\
500	0.0130610463365686\\
501	0.0130493714782708\\
502	0.0130372563518538\\
503	0.0130246440918271\\
504	0.0130085421346071\\
505	0.0129902929400844\\
506	0.0129743442802123\\
507	0.0129580375375457\\
508	0.012941368696631\\
509	0.0129243216642253\\
510	0.012906892506889\\
511	0.0128890507053126\\
512	0.0128707638287753\\
513	0.0128520174734674\\
514	0.0128327968478921\\
515	0.0128130867950133\\
516	0.0127928718190546\\
517	0.0127721361185569\\
518	0.0127508636208616\\
519	0.0127290379884259\\
520	0.0127066425217938\\
521	0.0126836597216777\\
522	0.0126600697331428\\
523	0.0126358452120687\\
524	0.0126102012522706\\
525	0.0125797637674508\\
526	0.0125486008261555\\
527	0.012516701834906\\
528	0.0124840859968582\\
529	0.0124506403646017\\
530	0.012421882458765\\
531	0.0124011496190387\\
532	0.012379760895246\\
533	0.0123576581795014\\
534	0.0123347715309565\\
535	0.0123110147450163\\
536	0.0122862828872557\\
537	0.0122603233418825\\
538	0.0122289258663907\\
539	0.0121717976209731\\
540	0.0121131111946628\\
541	0.0120528091740456\\
542	0.011990832808659\\
543	0.0119271218968977\\
544	0.0118616123748213\\
545	0.0117942284389449\\
546	0.0117248949720159\\
547	0.0116536580977404\\
548	0.0115806189640546\\
549	0.0115053699585196\\
550	0.0114274468585565\\
551	0.011381698839189\\
552	0.0113431571959004\\
553	0.0113030362611606\\
554	0.0112612795220593\\
555	0.0112177965128137\\
556	0.0111724851618522\\
557	0.0111252104114285\\
558	0.0110746474923157\\
559	0.0110014333785487\\
560	0.0109261242630267\\
561	0.0108485432095232\\
562	0.0107673722301904\\
563	0.0106831091580284\\
564	0.0105280498514088\\
565	0.0103348601344522\\
566	0.0101363722907275\\
567	0.0099316542325717\\
568	0.00972020831209711\\
569	0.00950501606326739\\
570	0.00940828015028893\\
571	0.00931333402576754\\
572	0.00922080861130992\\
573	0.0091312653549379\\
574	0.00904753506838823\\
575	0.00897092077645637\\
576	0.00889344865767624\\
577	0.00881637725871559\\
578	0.0087410276765376\\
579	0.00866753613908096\\
580	0.0085942859796107\\
581	0.00852160538424648\\
582	0.00842757392739441\\
583	0.00830398138734459\\
584	0.00812183334290264\\
585	0.0076557836760961\\
586	0.00717616754392714\\
587	0.00697576093390482\\
588	0.00687016646845473\\
589	0.00677423682775788\\
590	0.00667717773539374\\
591	0.00657820653135076\\
592	0.00647518044191995\\
593	0.00636372123806462\\
594	0.00623271716130668\\
595	0.00605340994148281\\
596	0.00575058001197164\\
597	0.00512683753504545\\
598	0.00366374385960312\\
599	0\\
600	0\\
};
\addplot [color=blue,solid,forget plot]
  table[row sep=crcr]{%
1	0.0135820465983962\\
2	0.0135820465214251\\
3	0.0135820464431596\\
4	0.0135820463635778\\
5	0.0135820462826577\\
6	0.0135820462003768\\
7	0.0135820461167122\\
8	0.0135820460316406\\
9	0.0135820459451383\\
10	0.0135820458571813\\
11	0.0135820457677451\\
12	0.0135820456768049\\
13	0.0135820455843352\\
14	0.0135820454903105\\
15	0.0135820453947045\\
16	0.0135820452974906\\
17	0.0135820451986418\\
18	0.0135820450981306\\
19	0.0135820449959289\\
20	0.0135820448920084\\
21	0.01358204478634\\
22	0.0135820446788944\\
23	0.0135820445696416\\
24	0.0135820444585513\\
25	0.0135820443455925\\
26	0.0135820442307336\\
27	0.0135820441139428\\
28	0.0135820439951874\\
29	0.0135820438744345\\
30	0.0135820437516502\\
31	0.0135820436268004\\
32	0.0135820434998503\\
33	0.0135820433707646\\
34	0.0135820432395071\\
35	0.0135820431060413\\
36	0.0135820429703299\\
37	0.0135820428323352\\
38	0.0135820426920185\\
39	0.0135820425493407\\
40	0.0135820424042621\\
41	0.013582042256742\\
42	0.0135820421067394\\
43	0.0135820419542122\\
44	0.0135820417991179\\
45	0.0135820416414132\\
46	0.0135820414810539\\
47	0.0135820413179953\\
48	0.0135820411521918\\
49	0.0135820409835969\\
50	0.0135820408121636\\
51	0.0135820406378438\\
52	0.0135820404605887\\
53	0.0135820402803489\\
54	0.0135820400970737\\
55	0.0135820399107118\\
56	0.013582039721211\\
57	0.0135820395285183\\
58	0.0135820393325795\\
59	0.0135820391333399\\
60	0.0135820389307433\\
61	0.0135820387247331\\
62	0.0135820385152513\\
63	0.0135820383022393\\
64	0.013582038085637\\
65	0.0135820378653837\\
66	0.0135820376414175\\
67	0.0135820374136754\\
68	0.0135820371820934\\
69	0.0135820369466062\\
70	0.0135820367071475\\
71	0.0135820364636501\\
72	0.0135820362160452\\
73	0.013582035964263\\
74	0.0135820357082327\\
75	0.0135820354478819\\
76	0.0135820351831372\\
77	0.0135820349139238\\
78	0.0135820346401658\\
79	0.0135820343617857\\
80	0.0135820340787049\\
81	0.0135820337908432\\
82	0.0135820334981193\\
83	0.0135820332004502\\
84	0.0135820328977516\\
85	0.0135820325899378\\
86	0.0135820322769214\\
87	0.0135820319586137\\
88	0.0135820316349243\\
89	0.0135820313057612\\
90	0.0135820309710309\\
91	0.0135820306306382\\
92	0.0135820302844863\\
93	0.0135820299324767\\
94	0.013582029574509\\
95	0.0135820292104813\\
96	0.0135820288402897\\
97	0.0135820284638286\\
98	0.0135820280809905\\
99	0.013582027691666\\
100	0.0135820272957438\\
101	0.0135820268931106\\
102	0.0135820264836511\\
103	0.0135820260672482\\
104	0.0135820256437823\\
105	0.013582025213132\\
106	0.0135820247751737\\
107	0.0135820243297816\\
108	0.0135820238768277\\
109	0.0135820234161817\\
110	0.0135820229477109\\
111	0.0135820224712804\\
112	0.0135820219867529\\
113	0.0135820214939887\\
114	0.0135820209928454\\
115	0.0135820204831783\\
116	0.01358201996484\\
117	0.0135820194376807\\
118	0.0135820189015476\\
119	0.0135820183562856\\
120	0.0135820178017363\\
121	0.0135820172377391\\
122	0.0135820166641301\\
123	0.0135820160807426\\
124	0.0135820154874071\\
125	0.0135820148839508\\
126	0.0135820142701982\\
127	0.0135820136459703\\
128	0.0135820130110851\\
129	0.0135820123653575\\
130	0.0135820117085988\\
131	0.0135820110406172\\
132	0.0135820103612174\\
133	0.0135820096702007\\
134	0.0135820089673648\\
135	0.0135820082525038\\
136	0.0135820075254081\\
137	0.0135820067858647\\
138	0.0135820060336564\\
139	0.0135820052685625\\
140	0.0135820044903582\\
141	0.0135820036988148\\
142	0.0135820028936996\\
143	0.0135820020747757\\
144	0.0135820012418021\\
145	0.0135820003945335\\
146	0.0135819995327205\\
147	0.013581998656109\\
148	0.0135819977644407\\
149	0.0135819968574525\\
150	0.0135819959348771\\
151	0.0135819949964423\\
152	0.0135819940418709\\
153	0.0135819930708814\\
154	0.0135819920831871\\
155	0.0135819910784962\\
156	0.0135819900565121\\
157	0.013581989016933\\
158	0.0135819879594516\\
159	0.0135819868837557\\
160	0.0135819857895275\\
161	0.0135819846764436\\
162	0.0135819835441754\\
163	0.0135819823923881\\
164	0.0135819812207417\\
165	0.01358198002889\\
166	0.0135819788164809\\
167	0.0135819775831565\\
168	0.0135819763285523\\
169	0.013581975052298\\
170	0.0135819737540167\\
171	0.0135819724333251\\
172	0.0135819710898334\\
173	0.0135819697231449\\
174	0.0135819683328563\\
175	0.0135819669185572\\
176	0.0135819654798302\\
177	0.0135819640162508\\
178	0.0135819625273871\\
179	0.0135819610127997\\
180	0.0135819594720415\\
181	0.0135819579046579\\
182	0.0135819563101863\\
183	0.0135819546881559\\
184	0.0135819530380879\\
185	0.0135819513594951\\
186	0.0135819496518816\\
187	0.0135819479147429\\
188	0.0135819461475659\\
189	0.013581944349828\\
190	0.0135819425209979\\
191	0.0135819406605345\\
192	0.0135819387678875\\
193	0.0135819368424968\\
194	0.0135819348837922\\
195	0.0135819328911938\\
196	0.013581930864111\\
197	0.0135819288019433\\
198	0.0135819267040791\\
199	0.0135819245698963\\
200	0.0135819223987615\\
201	0.0135819201900305\\
202	0.0135819179430472\\
203	0.0135819156571442\\
204	0.0135819133316421\\
205	0.0135819109658496\\
206	0.013581908559063\\
207	0.0135819061105662\\
208	0.0135819036196302\\
209	0.0135819010855133\\
210	0.0135818985074604\\
211	0.013581895884703\\
212	0.013581893216459\\
213	0.0135818905019323\\
214	0.0135818877403126\\
215	0.0135818849307752\\
216	0.0135818820724806\\
217	0.0135818791645746\\
218	0.0135818762061873\\
219	0.0135818731964336\\
220	0.0135818701344125\\
221	0.0135818670192067\\
222	0.0135818638498828\\
223	0.0135818606254904\\
224	0.0135818573450623\\
225	0.0135818540076139\\
226	0.0135818506121428\\
227	0.0135818471576288\\
228	0.0135818436430334\\
229	0.0135818400672992\\
230	0.0135818364293501\\
231	0.0135818327280905\\
232	0.0135818289624052\\
233	0.0135818251311588\\
234	0.0135818212331955\\
235	0.0135818172673387\\
236	0.0135818132323908\\
237	0.0135818091271322\\
238	0.0135818049503216\\
239	0.0135818007006953\\
240	0.0135817963769666\\
241	0.0135817919778257\\
242	0.0135817875019391\\
243	0.0135817829479493\\
244	0.013581778314474\\
245	0.0135817736001062\\
246	0.0135817688034131\\
247	0.0135817639229363\\
248	0.0135817589571907\\
249	0.0135817539046644\\
250	0.0135817487638182\\
251	0.0135817435330849\\
252	0.0135817382108687\\
253	0.013581732795545\\
254	0.0135817272854599\\
255	0.0135817216789292\\
256	0.0135817159742381\\
257	0.0135817101696408\\
258	0.0135817042633598\\
259	0.0135816982535851\\
260	0.013581692138474\\
261	0.0135816859161502\\
262	0.0135816795847033\\
263	0.0135816731421881\\
264	0.0135816665866242\\
265	0.013581659915995\\
266	0.0135816531282472\\
267	0.0135816462212903\\
268	0.0135816391929958\\
269	0.0135816320411962\\
270	0.013581624763685\\
271	0.0135816173582152\\
272	0.0135816098224992\\
273	0.0135816021542076\\
274	0.0135815943509689\\
275	0.0135815864103683\\
276	0.0135815783299473\\
277	0.0135815701072028\\
278	0.013581561739586\\
279	0.0135815532245021\\
280	0.0135815445593093\\
281	0.0135815357413177\\
282	0.0135815267677888\\
283	0.0135815176359345\\
284	0.0135815083429164\\
285	0.0135814988858446\\
286	0.0135814892617772\\
287	0.0135814794677191\\
288	0.0135814695006212\\
289	0.0135814593573797\\
290	0.0135814490348345\\
291	0.0135814385297693\\
292	0.0135814278389096\\
293	0.0135814169589225\\
294	0.0135814058864152\\
295	0.0135813946179344\\
296	0.0135813831499652\\
297	0.0135813714789299\\
298	0.0135813596011872\\
299	0.013581347513031\\
300	0.0135813352106898\\
301	0.013581322690325\\
302	0.0135813099480303\\
303	0.0135812969798305\\
304	0.0135812837816806\\
305	0.0135812703494642\\
306	0.0135812566789932\\
307	0.013581242766006\\
308	0.0135812286061669\\
309	0.0135812141950646\\
310	0.0135811995282114\\
311	0.0135811846010417\\
312	0.0135811694089113\\
313	0.013581153947096\\
314	0.0135811382107905\\
315	0.0135811221951072\\
316	0.0135811058950751\\
317	0.0135810893056386\\
318	0.0135810724216564\\
319	0.0135810552379\\
320	0.013581037749053\\
321	0.0135810199497093\\
322	0.0135810018343726\\
323	0.0135809833974543\\
324	0.0135809646332729\\
325	0.0135809455360527\\
326	0.0135809260999219\\
327	0.0135809063189123\\
328	0.0135808861869568\\
329	0.0135808656978893\\
330	0.0135808448454421\\
331	0.0135808236232456\\
332	0.013580802024826\\
333	0.0135807800436043\\
334	0.0135807576728946\\
335	0.0135807349059027\\
336	0.0135807117357242\\
337	0.0135806881553432\\
338	0.0135806641576301\\
339	0.0135806397353402\\
340	0.0135806148811115\\
341	0.0135805895874628\\
342	0.0135805638467918\\
343	0.0135805376513725\\
344	0.0135805109933533\\
345	0.0135804838647541\\
346	0.0135804562574641\\
347	0.0135804281632386\\
348	0.0135803995736966\\
349	0.013580370480317\\
350	0.0135803408744358\\
351	0.013580310747242\\
352	0.0135802800897743\\
353	0.0135802488929164\\
354	0.0135802171473932\\
355	0.0135801848437656\\
356	0.0135801519724254\\
357	0.0135801185235906\\
358	0.0135800844872985\\
359	0.0135800498534004\\
360	0.0135800146115544\\
361	0.0135799787512186\\
362	0.0135799422616432\\
363	0.0135799051318624\\
364	0.0135798673506859\\
365	0.0135798289066893\\
366	0.0135797897882046\\
367	0.0135797499833094\\
368	0.0135797094798155\\
369	0.0135796682652576\\
370	0.0135796263268804\\
371	0.013579583651625\\
372	0.013579540226115\\
373	0.0135794960366418\\
374	0.0135794510691482\\
375	0.0135794053092124\\
376	0.0135793587420297\\
377	0.0135793113523947\\
378	0.013579263124681\\
379	0.0135792140428211\\
380	0.0135791640902844\\
381	0.0135791132500541\\
382	0.0135790615046026\\
383	0.0135790088358664\\
384	0.0135789552252177\\
385	0.0135789006534355\\
386	0.0135788451006745\\
387	0.013578788546431\\
388	0.0135787309695064\\
389	0.0135786723479684\\
390	0.0135786126591077\\
391	0.0135785518793907\\
392	0.0135784899844079\\
393	0.0135784269488162\\
394	0.0135783627462741\\
395	0.0135782973493692\\
396	0.0135782307295354\\
397	0.0135781628569592\\
398	0.0135780937004717\\
399	0.013578023227424\\
400	0.0135779514035437\\
401	0.0135778781927702\\
402	0.0135778035570696\\
403	0.0135777274562328\\
404	0.0135776498476507\\
405	0.0135775706859901\\
406	0.0135774899230165\\
407	0.0135774075077132\\
408	0.013577323386081\\
409	0.0135772375009321\\
410	0.0135771497916844\\
411	0.0135770601941591\\
412	0.0135769686403918\\
413	0.013576875058459\\
414	0.0135767793723166\\
415	0.0135766815016254\\
416	0.0135765813615334\\
417	0.0135764788625137\\
418	0.0135763739108464\\
419	0.0135762664057732\\
420	0.0135761562383114\\
421	0.0135760432902424\\
422	0.0135759274329503\\
423	0.013575808526082\\
424	0.0135756864160007\\
425	0.0135755609339938\\
426	0.0135754318941911\\
427	0.0135752990911401\\
428	0.0135751622969668\\
429	0.0135750212580314\\
430	0.0135748756909365\\
431	0.0135747252776382\\
432	0.013574569659137\\
433	0.0135744084265066\\
434	0.0135742411061021\\
435	0.0135740671307341\\
436	0.013573885775618\\
437	0.0135736960065876\\
438	0.0135734961215769\\
439	0.0135732829690905\\
440	0.0135729081704589\\
441	0.0135724156562382\\
442	0.013571911372972\\
443	0.0135713949382745\\
444	0.0135708659545813\\
445	0.0135703240090924\\
446	0.0135697686739248\\
447	0.0135691995065429\\
448	0.0135686160505555\\
449	0.0135680178369965\\
450	0.0135674043862314\\
451	0.0135667752106715\\
452	0.0135661298185056\\
453	0.0135654677186674\\
454	0.0135647884271903\\
455	0.0135640914748855\\
456	0.0135633764156356\\
457	0.0135626428333041\\
458	0.0135618903432951\\
459	0.0135611185839216\\
460	0.0135603272023142\\
461	0.0135595158870753\\
462	0.0135586845841885\\
463	0.013557829555261\\
464	0.013556930497075\\
465	0.0135550240133501\\
466	0.0135530808659335\\
467	0.0135510999818061\\
468	0.0135490802236954\\
469	0.0135470203857685\\
470	0.0135449191936659\\
471	0.0135427752867835\\
472	0.0135405872104085\\
473	0.0135383534301489\\
474	0.0135360723311704\\
475	0.0135337422170446\\
476	0.013531361301743\\
477	0.0135289276778965\\
478	0.0135264394440204\\
479	0.0135238946049664\\
480	0.0135212910345299\\
481	0.0135186264629605\\
482	0.0135158984639776\\
483	0.0135131044396253\\
484	0.0135102416025157\\
485	0.0135073069549623\\
486	0.0135042972658295\\
487	0.013501209043831\\
488	0.0134980385020666\\
489	0.0134947815206354\\
490	0.0134914336030213\\
491	0.0134879898249848\\
492	0.0134844447743847\\
493	0.013480792479944\\
494	0.0134770263263802\\
495	0.0134731389523509\\
496	0.0134691221258066\\
497	0.0134649665871592\\
498	0.0134606618401376\\
499	0.0134561958417997\\
500	0.0134515544642281\\
501	0.0134467203786268\\
502	0.0134416703910603\\
503	0.0134363684584538\\
504	0.0134301109338322\\
505	0.0134206996946467\\
506	0.0134088559696673\\
507	0.013396743056697\\
508	0.0133843486283437\\
509	0.0133716599451185\\
510	0.013358663810878\\
511	0.0133453418125183\\
512	0.0133316747671679\\
513	0.013317645481152\\
514	0.0133032352912366\\
515	0.0132884239069768\\
516	0.0132731892362347\\
517	0.0132575071945262\\
518	0.013241351505\\
519	0.0132246935185095\\
520	0.0132075021638263\\
521	0.01318974442653\\
522	0.0131713877897399\\
523	0.0131524097829329\\
524	0.0131338902835204\\
525	0.0131174619770685\\
526	0.0131004259760457\\
527	0.0130827263013375\\
528	0.0130642915501048\\
529	0.013044982201245\\
530	0.0130202912901368\\
531	0.0129876063032289\\
532	0.0129539722259956\\
533	0.0129193559278304\\
534	0.0128837225475325\\
535	0.0128470504617317\\
536	0.0128092533794661\\
537	0.0127742465044304\\
538	0.0127424674144049\\
539	0.0127042975904532\\
540	0.0126649903423508\\
541	0.0126244739772393\\
542	0.0125826653456361\\
543	0.012539477527505\\
544	0.0124948524307789\\
545	0.0124487142253323\\
546	0.0124009538939307\\
547	0.0123513106965161\\
548	0.0122996579653759\\
549	0.0122470277426339\\
550	0.0122112561399698\\
551	0.0121456663671696\\
552	0.0120708318601181\\
553	0.0119937286458208\\
554	0.011914267974539\\
555	0.0118323524226605\\
556	0.0117477215100425\\
557	0.0116602661639905\\
558	0.0115700082060289\\
559	0.0114766789108114\\
560	0.0113796625377174\\
561	0.011278648796618\\
562	0.0111980143005405\\
563	0.0111421292706777\\
564	0.0110695374751004\\
565	0.0109874399878422\\
566	0.0109017749549262\\
567	0.0108119877133317\\
568	0.0107195897617478\\
569	0.0106209820844399\\
570	0.0104193965800069\\
571	0.0102138125244415\\
572	0.0100038158503817\\
573	0.00978711498739125\\
574	0.00956243326742183\\
575	0.00933812331163025\\
576	0.00922701819998303\\
577	0.00911696641490741\\
578	0.00900868084044329\\
579	0.00890286078583315\\
580	0.00879963372380029\\
581	0.00870465551720041\\
582	0.00861264400964136\\
583	0.00851949463509417\\
584	0.00841846506097205\\
585	0.0082781774636422\\
586	0.00814370437258449\\
587	0.00776443978473074\\
588	0.00729624552515174\\
589	0.00683356619974124\\
590	0.00671171289148691\\
591	0.00659293263946882\\
592	0.00648270502684632\\
593	0.00636662498220103\\
594	0.00623343894857411\\
595	0.00605340994148281\\
596	0.00575058001197164\\
597	0.00512683753504545\\
598	0.00366374385960312\\
599	0\\
600	0\\
};
\addplot [color=mycolor10,solid,forget plot]
  table[row sep=crcr]{%
1	0.01359190851983\\
2	0.013591908519541\\
3	0.0135919085192472\\
4	0.0135919085189484\\
5	0.0135919085186446\\
6	0.0135919085183357\\
7	0.0135919085180216\\
8	0.0135919085177023\\
9	0.0135919085173775\\
10	0.0135919085170473\\
11	0.0135919085167115\\
12	0.0135919085163701\\
13	0.013591908516023\\
14	0.01359190851567\\
15	0.0135919085153111\\
16	0.0135919085149461\\
17	0.013591908514575\\
18	0.0135919085141977\\
19	0.013591908513814\\
20	0.0135919085134238\\
21	0.0135919085130271\\
22	0.0135919085126238\\
23	0.0135919085122136\\
24	0.0135919085117965\\
25	0.0135919085113725\\
26	0.0135919085109413\\
27	0.0135919085105028\\
28	0.013591908510057\\
29	0.0135919085096036\\
30	0.0135919085091427\\
31	0.013591908508674\\
32	0.0135919085081974\\
33	0.0135919085077128\\
34	0.01359190850722\\
35	0.0135919085067189\\
36	0.0135919085062094\\
37	0.0135919085056914\\
38	0.0135919085051646\\
39	0.013591908504629\\
40	0.0135919085040843\\
41	0.0135919085035305\\
42	0.0135919085029673\\
43	0.0135919085023947\\
44	0.0135919085018124\\
45	0.0135919085012204\\
46	0.0135919085006183\\
47	0.0135919085000062\\
48	0.0135919084993837\\
49	0.0135919084987508\\
50	0.0135919084981072\\
51	0.0135919084974527\\
52	0.0135919084967873\\
53	0.0135919084961106\\
54	0.0135919084954225\\
55	0.0135919084947229\\
56	0.0135919084940114\\
57	0.013591908493288\\
58	0.0135919084925524\\
59	0.0135919084918044\\
60	0.0135919084910438\\
61	0.0135919084902704\\
62	0.0135919084894839\\
63	0.0135919084886842\\
64	0.013591908487871\\
65	0.0135919084870441\\
66	0.0135919084862032\\
67	0.0135919084853482\\
68	0.0135919084844788\\
69	0.0135919084835947\\
70	0.0135919084826957\\
71	0.0135919084817815\\
72	0.0135919084808519\\
73	0.0135919084799066\\
74	0.0135919084789454\\
75	0.0135919084779679\\
76	0.013591908476974\\
77	0.0135919084759632\\
78	0.0135919084749354\\
79	0.0135919084738903\\
80	0.0135919084728274\\
81	0.0135919084717467\\
82	0.0135919084706477\\
83	0.0135919084695301\\
84	0.0135919084683936\\
85	0.0135919084672379\\
86	0.0135919084660627\\
87	0.0135919084648676\\
88	0.0135919084636523\\
89	0.0135919084624165\\
90	0.0135919084611597\\
91	0.0135919084598817\\
92	0.013591908458582\\
93	0.0135919084572604\\
94	0.0135919084559164\\
95	0.0135919084545496\\
96	0.0135919084531597\\
97	0.0135919084517462\\
98	0.0135919084503088\\
99	0.013591908448847\\
100	0.0135919084473604\\
101	0.0135919084458487\\
102	0.0135919084443113\\
103	0.0135919084427478\\
104	0.0135919084411578\\
105	0.0135919084395408\\
106	0.0135919084378964\\
107	0.013591908436224\\
108	0.0135919084345233\\
109	0.0135919084327936\\
110	0.0135919084310346\\
111	0.0135919084292457\\
112	0.0135919084274263\\
113	0.0135919084255761\\
114	0.0135919084236943\\
115	0.0135919084217806\\
116	0.0135919084198342\\
117	0.0135919084178548\\
118	0.0135919084158416\\
119	0.0135919084137942\\
120	0.0135919084117118\\
121	0.013591908409594\\
122	0.0135919084074401\\
123	0.0135919084052494\\
124	0.0135919084030214\\
125	0.0135919084007553\\
126	0.0135919083984506\\
127	0.0135919083961066\\
128	0.0135919083937225\\
129	0.0135919083912976\\
130	0.0135919083888314\\
131	0.013591908386323\\
132	0.0135919083837717\\
133	0.0135919083811767\\
134	0.0135919083785374\\
135	0.0135919083758529\\
136	0.0135919083731224\\
137	0.0135919083703451\\
138	0.0135919083675203\\
139	0.0135919083646471\\
140	0.0135919083617247\\
141	0.0135919083587521\\
142	0.0135919083557286\\
143	0.0135919083526531\\
144	0.0135919083495249\\
145	0.013591908346343\\
146	0.0135919083431065\\
147	0.0135919083398144\\
148	0.0135919083364657\\
149	0.0135919083330595\\
150	0.0135919083295947\\
151	0.0135919083260704\\
152	0.0135919083224854\\
153	0.0135919083188388\\
154	0.0135919083151294\\
155	0.0135919083113562\\
156	0.013591908307518\\
157	0.0135919083036138\\
158	0.0135919082996423\\
159	0.0135919082956024\\
160	0.0135919082914929\\
161	0.0135919082873125\\
162	0.0135919082830601\\
163	0.0135919082787344\\
164	0.0135919082743341\\
165	0.0135919082698579\\
166	0.0135919082653046\\
167	0.0135919082606726\\
168	0.0135919082559608\\
169	0.0135919082511676\\
170	0.0135919082462918\\
171	0.0135919082413317\\
172	0.0135919082362861\\
173	0.0135919082311533\\
174	0.0135919082259319\\
175	0.0135919082206204\\
176	0.0135919082152171\\
177	0.0135919082097205\\
178	0.013591908204129\\
179	0.0135919081984408\\
180	0.0135919081926544\\
181	0.0135919081867681\\
182	0.01359190818078\\
183	0.0135919081746884\\
184	0.0135919081684916\\
185	0.0135919081621876\\
186	0.0135919081557747\\
187	0.013591908149251\\
188	0.0135919081426144\\
189	0.0135919081358632\\
190	0.0135919081289951\\
191	0.0135919081220084\\
192	0.0135919081149007\\
193	0.0135919081076702\\
194	0.0135919081003145\\
195	0.0135919080928317\\
196	0.0135919080852193\\
197	0.0135919080774752\\
198	0.0135919080695971\\
199	0.0135919080615827\\
200	0.0135919080534295\\
201	0.0135919080451352\\
202	0.0135919080366973\\
203	0.0135919080281132\\
204	0.0135919080193805\\
205	0.0135919080104966\\
206	0.0135919080014587\\
207	0.0135919079922643\\
208	0.0135919079829106\\
209	0.0135919079733947\\
210	0.0135919079637139\\
211	0.0135919079538653\\
212	0.013591907943846\\
213	0.0135919079336529\\
214	0.013591907923283\\
215	0.0135919079127333\\
216	0.0135919079020006\\
217	0.0135919078910816\\
218	0.0135919078799731\\
219	0.0135919078686719\\
220	0.0135919078571744\\
221	0.0135919078454774\\
222	0.0135919078335772\\
223	0.0135919078214703\\
224	0.013591907809153\\
225	0.0135919077966218\\
226	0.0135919077838728\\
227	0.0135919077709022\\
228	0.0135919077577061\\
229	0.0135919077442806\\
230	0.0135919077306216\\
231	0.0135919077167249\\
232	0.0135919077025865\\
233	0.0135919076882021\\
234	0.0135919076735673\\
235	0.0135919076586777\\
236	0.0135919076435288\\
237	0.0135919076281161\\
238	0.0135919076124349\\
239	0.0135919075964805\\
240	0.013591907580248\\
241	0.0135919075637325\\
242	0.0135919075469291\\
243	0.0135919075298326\\
244	0.0135919075124379\\
245	0.0135919074947397\\
246	0.0135919074767327\\
247	0.0135919074584113\\
248	0.01359190743977\\
249	0.0135919074208031\\
250	0.013591907401505\\
251	0.0135919073818696\\
252	0.013591907361891\\
253	0.0135919073415632\\
254	0.0135919073208799\\
255	0.0135919072998349\\
256	0.0135919072784217\\
257	0.0135919072566337\\
258	0.0135919072344644\\
259	0.013591907211907\\
260	0.0135919071889544\\
261	0.0135919071655998\\
262	0.013591907141836\\
263	0.0135919071176557\\
264	0.0135919070930515\\
265	0.0135919070680158\\
266	0.0135919070425409\\
267	0.0135919070166191\\
268	0.0135919069902424\\
269	0.0135919069634026\\
270	0.0135919069360915\\
271	0.0135919069083008\\
272	0.0135919068800217\\
273	0.0135919068512458\\
274	0.013591906821964\\
275	0.0135919067921673\\
276	0.0135919067618466\\
277	0.0135919067309925\\
278	0.0135919066995955\\
279	0.0135919066676459\\
280	0.0135919066351338\\
281	0.0135919066020491\\
282	0.0135919065683817\\
283	0.0135919065341211\\
284	0.0135919064992567\\
285	0.0135919064637778\\
286	0.0135919064276733\\
287	0.0135919063909321\\
288	0.0135919063535427\\
289	0.0135919063154937\\
290	0.0135919062767732\\
291	0.0135919062373693\\
292	0.0135919061972696\\
293	0.0135919061564619\\
294	0.0135919061149334\\
295	0.0135919060726712\\
296	0.0135919060296624\\
297	0.0135919059858934\\
298	0.0135919059413509\\
299	0.0135919058960209\\
300	0.0135919058498895\\
301	0.0135919058029423\\
302	0.0135919057551647\\
303	0.0135919057065421\\
304	0.0135919056570594\\
305	0.0135919056067012\\
306	0.013591905555452\\
307	0.0135919055032959\\
308	0.0135919054502168\\
309	0.0135919053961984\\
310	0.0135919053412239\\
311	0.0135919052852764\\
312	0.0135919052283387\\
313	0.0135919051703932\\
314	0.0135919051114221\\
315	0.0135919050514072\\
316	0.0135919049903302\\
317	0.0135919049281722\\
318	0.0135919048649143\\
319	0.013591904800537\\
320	0.0135919047350207\\
321	0.0135919046683454\\
322	0.0135919046004906\\
323	0.0135919045314357\\
324	0.0135919044611597\\
325	0.0135919043896413\\
326	0.0135919043168587\\
327	0.0135919042427899\\
328	0.0135919041674124\\
329	0.0135919040907035\\
330	0.0135919040126401\\
331	0.0135919039331986\\
332	0.0135919038523551\\
333	0.0135919037700855\\
334	0.013591903686365\\
335	0.0135919036011686\\
336	0.0135919035144709\\
337	0.013591903426246\\
338	0.0135919033364676\\
339	0.0135919032451092\\
340	0.0135919031521436\\
341	0.0135919030575433\\
342	0.0135919029612804\\
343	0.0135919028633264\\
344	0.0135919027636526\\
345	0.0135919026622295\\
346	0.0135919025590274\\
347	0.013591902454016\\
348	0.0135919023471646\\
349	0.0135919022384418\\
350	0.0135919021278159\\
351	0.0135919020152546\\
352	0.013591901900725\\
353	0.0135919017841936\\
354	0.0135919016656264\\
355	0.0135919015449888\\
356	0.0135919014222457\\
357	0.013591901297361\\
358	0.0135919011702984\\
359	0.0135919010410205\\
360	0.0135919009094895\\
361	0.0135919007756667\\
362	0.0135919006395127\\
363	0.0135919005009873\\
364	0.0135919003600493\\
365	0.013591900216657\\
366	0.0135919000707675\\
367	0.013591899922337\\
368	0.0135918997713208\\
369	0.0135918996176731\\
370	0.0135918994613471\\
371	0.0135918993022949\\
372	0.0135918991404671\\
373	0.0135918989758135\\
374	0.0135918988082823\\
375	0.0135918986378206\\
376	0.0135918984643738\\
377	0.0135918982878859\\
378	0.0135918981082995\\
379	0.0135918979255553\\
380	0.0135918977395924\\
381	0.0135918975503481\\
382	0.0135918973577577\\
383	0.0135918971617546\\
384	0.0135918969622701\\
385	0.0135918967592331\\
386	0.0135918965525703\\
387	0.013591896342206\\
388	0.0135918961280615\\
389	0.0135918959100558\\
390	0.0135918956881046\\
391	0.0135918954621207\\
392	0.0135918952320132\\
393	0.013591894997688\\
394	0.0135918947590467\\
395	0.013591894515987\\
396	0.0135918942684016\\
397	0.0135918940161786\\
398	0.0135918937591999\\
399	0.0135918934973415\\
400	0.0135918932304722\\
401	0.013591892958454\\
402	0.0135918926811415\\
403	0.0135918923983827\\
404	0.0135918921100159\\
405	0.0135918918158676\\
406	0.0135918915157543\\
407	0.0135918912094824\\
408	0.0135918908968471\\
409	0.0135918905776323\\
410	0.0135918902516104\\
411	0.0135918899185423\\
412	0.0135918895781768\\
413	0.0135918892302491\\
414	0.0135918888744765\\
415	0.0135918885105519\\
416	0.0135918881381426\\
417	0.0135918877569098\\
418	0.0135918873664904\\
419	0.0135918869664911\\
420	0.0135918865564846\\
421	0.0135918861360053\\
422	0.0135918857045439\\
423	0.0135918852615422\\
424	0.0135918848063852\\
425	0.0135918843383937\\
426	0.0135918838568133\\
427	0.0135918833608008\\
428	0.0135918828494039\\
429	0.0135918823215263\\
430	0.0135918817758588\\
431	0.0135918812107291\\
432	0.0135918806237562\\
433	0.0135918800110476\\
434	0.0135918793653702\\
435	0.0135918786721928\\
436	0.0135918779019084\\
437	0.0135918769973993\\
438	0.0135918758646862\\
439	0.0135918744110811\\
440	0.013591872788802\\
441	0.013591871129571\\
442	0.0135918694322569\\
443	0.013591867695688\\
444	0.013591865918653\\
445	0.0135918640999026\\
446	0.0135918622381519\\
447	0.0135918603320848\\
448	0.0135918583803587\\
449	0.0135918563816111\\
450	0.0135918543344665\\
451	0.013591852237538\\
452	0.0135918500894173\\
453	0.0135918478886263\\
454	0.0135918456334797\\
455	0.0135918433217444\\
456	0.0135918409498552\\
457	0.0135918385111969\\
458	0.0135918359923967\\
459	0.0135918333649876\\
460	0.0135918305642624\\
461	0.0135918274258205\\
462	0.0135918234734875\\
463	0.0135918174912898\\
464	0.0135918045174755\\
465	0.0135915740084851\\
466	0.0135913391041115\\
467	0.013591099669409\\
468	0.0135908555611643\\
469	0.0135906066273529\\
470	0.0135903527063131\\
471	0.0135900936268359\\
472	0.0135898292088706\\
473	0.013589559263522\\
474	0.0135892835930784\\
475	0.0135890019910042\\
476	0.0135887142422068\\
477	0.013588420125188\\
478	0.0135881194057676\\
479	0.013587811834604\\
480	0.0135874971458467\\
481	0.0135871750556311\\
482	0.0135868452603463\\
483	0.0135865074346414\\
484	0.0135861612291345\\
485	0.0135858062678069\\
486	0.0135854421450077\\
487	0.0135850684219721\\
488	0.0135846846229339\\
489	0.013584290230632\\
490	0.0135838846810895\\
491	0.0135834673575133\\
492	0.013583037583117\\
493	0.0135825946125993\\
494	0.013582137621883\\
495	0.0135816656954558\\
496	0.0135811778100659\\
497	0.013580672812118\\
498	0.0135801493827262\\
499	0.0135796059762644\\
500	0.0135790406994847\\
501	0.0135784510576631\\
502	0.0135778334172429\\
503	0.0135771819374208\\
504	0.0135764867838359\\
505	0.0135753188135231\\
506	0.0135736371624971\\
507	0.013571917310551\\
508	0.0135701573334925\\
509	0.0135683551256225\\
510	0.0135665083422997\\
511	0.0135646145612687\\
512	0.0135626712709945\\
513	0.0135606757178118\\
514	0.013558624876504\\
515	0.0135565154165732\\
516	0.0135543436626437\\
517	0.0135521055452458\\
518	0.0135497965315038\\
519	0.0135474115039713\\
520	0.0135449444864745\\
521	0.0135423878839583\\
522	0.0135397301108489\\
523	0.0135369477205464\\
524	0.0135331290459678\\
525	0.0135269367067296\\
526	0.0135205313336136\\
527	0.0135138937138777\\
528	0.0135069985011208\\
529	0.0134998096724892\\
530	0.0134913259654143\\
531	0.0134809596289797\\
532	0.013470240349157\\
533	0.0134591408298378\\
534	0.0134476274755564\\
535	0.0134356546878034\\
536	0.013423130212144\\
537	0.0134068337452876\\
538	0.0133857363659662\\
539	0.0133640067904496\\
540	0.0133415975166393\\
541	0.0133184549341791\\
542	0.0132945181338923\\
543	0.0132697195946383\\
544	0.0132439887419286\\
545	0.0132172412743684\\
546	0.0131893720312445\\
547	0.0131602259087115\\
548	0.0131296221170271\\
549	0.0130963173310625\\
550	0.0130457326939099\\
551	0.0129871618734068\\
552	0.012924988837248\\
553	0.0128606163998723\\
554	0.01279391552148\\
555	0.0127247750666354\\
556	0.0126590448696326\\
557	0.0125915651237542\\
558	0.0125212588405926\\
559	0.0124479598293873\\
560	0.0123818455907285\\
561	0.0123161653451569\\
562	0.0122272630202049\\
563	0.0121118495694013\\
564	0.0119919511824275\\
565	0.0118673938931047\\
566	0.011764611763121\\
567	0.0116578802124658\\
568	0.0115464435767891\\
569	0.0114294267343144\\
570	0.0112859427377856\\
571	0.0111362830903561\\
572	0.0109797626709145\\
573	0.0108586765742158\\
574	0.0107547619357607\\
575	0.0106388887178325\\
576	0.0104222260330238\\
577	0.0101998222515477\\
578	0.0099713298658148\\
579	0.00973555084991577\\
580	0.00949143485806943\\
581	0.00923867627785478\\
582	0.00907524279500573\\
583	0.0089436314279156\\
584	0.00881001385781117\\
585	0.00867625419638909\\
586	0.00854558854619178\\
587	0.00837875618173115\\
588	0.00821243744553536\\
589	0.00803422941175562\\
590	0.00755482615660152\\
591	0.00706957814576492\\
592	0.00658807525000149\\
593	0.00642698023275806\\
594	0.00625725512634445\\
595	0.0060609403391104\\
596	0.00575058001197164\\
597	0.00512683753504545\\
598	0.00366374385960312\\
599	0\\
600	0\\
};
\addplot [color=mycolor11,solid,forget plot]
  table[row sep=crcr]{%
1	0.0135940453118077\\
2	0.0135940453117958\\
3	0.0135940453117837\\
4	0.0135940453117713\\
5	0.0135940453117588\\
6	0.0135940453117461\\
7	0.0135940453117331\\
8	0.0135940453117199\\
9	0.0135940453117065\\
10	0.0135940453116929\\
11	0.0135940453116791\\
12	0.013594045311665\\
13	0.0135940453116507\\
14	0.0135940453116361\\
15	0.0135940453116213\\
16	0.0135940453116063\\
17	0.013594045311591\\
18	0.0135940453115754\\
19	0.0135940453115596\\
20	0.0135940453115435\\
21	0.0135940453115271\\
22	0.0135940453115105\\
23	0.0135940453114936\\
24	0.0135940453114764\\
25	0.0135940453114589\\
26	0.0135940453114411\\
27	0.013594045311423\\
28	0.0135940453114046\\
29	0.0135940453113859\\
30	0.0135940453113669\\
31	0.0135940453113476\\
32	0.0135940453113279\\
33	0.0135940453113079\\
34	0.0135940453112876\\
35	0.0135940453112669\\
36	0.0135940453112459\\
37	0.0135940453112246\\
38	0.0135940453112028\\
39	0.0135940453111807\\
40	0.0135940453111583\\
41	0.0135940453111354\\
42	0.0135940453111122\\
43	0.0135940453110886\\
44	0.0135940453110646\\
45	0.0135940453110402\\
46	0.0135940453110153\\
47	0.0135940453109901\\
48	0.0135940453109644\\
49	0.0135940453109383\\
50	0.0135940453109118\\
51	0.0135940453108848\\
52	0.0135940453108573\\
53	0.0135940453108294\\
54	0.013594045310801\\
55	0.0135940453107722\\
56	0.0135940453107428\\
57	0.013594045310713\\
58	0.0135940453106827\\
59	0.0135940453106518\\
60	0.0135940453106204\\
61	0.0135940453105885\\
62	0.0135940453105561\\
63	0.0135940453105231\\
64	0.0135940453104896\\
65	0.0135940453104555\\
66	0.0135940453104208\\
67	0.0135940453103855\\
68	0.0135940453103497\\
69	0.0135940453103132\\
70	0.0135940453102761\\
71	0.0135940453102384\\
72	0.0135940453102001\\
73	0.0135940453101611\\
74	0.0135940453101214\\
75	0.0135940453100811\\
76	0.0135940453100401\\
77	0.0135940453099984\\
78	0.013594045309956\\
79	0.0135940453099129\\
80	0.0135940453098691\\
81	0.0135940453098245\\
82	0.0135940453097792\\
83	0.0135940453097331\\
84	0.0135940453096862\\
85	0.0135940453096385\\
86	0.01359404530959\\
87	0.0135940453095408\\
88	0.0135940453094906\\
89	0.0135940453094396\\
90	0.0135940453093878\\
91	0.0135940453093351\\
92	0.0135940453092815\\
93	0.013594045309227\\
94	0.0135940453091715\\
95	0.0135940453091151\\
96	0.0135940453090578\\
97	0.0135940453089995\\
98	0.0135940453089402\\
99	0.0135940453088799\\
100	0.0135940453088186\\
101	0.0135940453087562\\
102	0.0135940453086928\\
103	0.0135940453086283\\
104	0.0135940453085627\\
105	0.013594045308496\\
106	0.0135940453084282\\
107	0.0135940453083592\\
108	0.013594045308289\\
109	0.0135940453082177\\
110	0.0135940453081451\\
111	0.0135940453080713\\
112	0.0135940453079962\\
113	0.0135940453079199\\
114	0.0135940453078423\\
115	0.0135940453077633\\
116	0.013594045307683\\
117	0.0135940453076014\\
118	0.0135940453075183\\
119	0.0135940453074338\\
120	0.0135940453073479\\
121	0.0135940453072606\\
122	0.0135940453071717\\
123	0.0135940453070813\\
124	0.0135940453069894\\
125	0.0135940453068959\\
126	0.0135940453068008\\
127	0.0135940453067041\\
128	0.0135940453066057\\
129	0.0135940453065057\\
130	0.0135940453064039\\
131	0.0135940453063004\\
132	0.0135940453061952\\
133	0.0135940453060881\\
134	0.0135940453059792\\
135	0.0135940453058684\\
136	0.0135940453057558\\
137	0.0135940453056412\\
138	0.0135940453055246\\
139	0.0135940453054061\\
140	0.0135940453052855\\
141	0.0135940453051628\\
142	0.0135940453050381\\
143	0.0135940453049112\\
144	0.0135940453047821\\
145	0.0135940453046508\\
146	0.0135940453045172\\
147	0.0135940453043814\\
148	0.0135940453042432\\
149	0.0135940453041026\\
150	0.0135940453039597\\
151	0.0135940453038142\\
152	0.0135940453036663\\
153	0.0135940453035158\\
154	0.0135940453033627\\
155	0.013594045303207\\
156	0.0135940453030486\\
157	0.0135940453028875\\
158	0.0135940453027236\\
159	0.0135940453025569\\
160	0.0135940453023873\\
161	0.0135940453022148\\
162	0.0135940453020393\\
163	0.0135940453018608\\
164	0.0135940453016792\\
165	0.0135940453014945\\
166	0.0135940453013066\\
167	0.0135940453011154\\
168	0.013594045300921\\
169	0.0135940453007232\\
170	0.013594045300522\\
171	0.0135940453003173\\
172	0.013594045300109\\
173	0.0135940452998972\\
174	0.0135940452996817\\
175	0.0135940452994625\\
176	0.0135940452992395\\
177	0.0135940452990127\\
178	0.0135940452987819\\
179	0.0135940452985472\\
180	0.0135940452983084\\
181	0.0135940452980655\\
182	0.0135940452978183\\
183	0.0135940452975669\\
184	0.0135940452973112\\
185	0.013594045297051\\
186	0.0135940452967864\\
187	0.0135940452965171\\
188	0.0135940452962432\\
189	0.0135940452959646\\
190	0.0135940452956812\\
191	0.0135940452953928\\
192	0.0135940452950995\\
193	0.0135940452948011\\
194	0.0135940452944975\\
195	0.0135940452941887\\
196	0.0135940452938745\\
197	0.0135940452935549\\
198	0.0135940452932298\\
199	0.013594045292899\\
200	0.0135940452925625\\
201	0.0135940452922202\\
202	0.013594045291872\\
203	0.0135940452915177\\
204	0.0135940452911573\\
205	0.0135940452907906\\
206	0.0135940452904176\\
207	0.0135940452900382\\
208	0.0135940452896521\\
209	0.0135940452892594\\
210	0.0135940452888598\\
211	0.0135940452884534\\
212	0.0135940452880398\\
213	0.0135940452876191\\
214	0.0135940452871912\\
215	0.0135940452867557\\
216	0.0135940452863128\\
217	0.0135940452858621\\
218	0.0135940452854036\\
219	0.0135940452849372\\
220	0.0135940452844627\\
221	0.0135940452839799\\
222	0.0135940452834887\\
223	0.013594045282989\\
224	0.0135940452824806\\
225	0.0135940452819634\\
226	0.0135940452814372\\
227	0.0135940452809019\\
228	0.0135940452803572\\
229	0.0135940452798031\\
230	0.0135940452792393\\
231	0.0135940452786657\\
232	0.0135940452780822\\
233	0.0135940452774884\\
234	0.0135940452768844\\
235	0.0135940452762698\\
236	0.0135940452756445\\
237	0.0135940452750084\\
238	0.0135940452743611\\
239	0.0135940452737026\\
240	0.0135940452730326\\
241	0.0135940452723509\\
242	0.0135940452716573\\
243	0.0135940452709516\\
244	0.0135940452702336\\
245	0.0135940452695031\\
246	0.0135940452687598\\
247	0.0135940452680035\\
248	0.0135940452672341\\
249	0.0135940452664511\\
250	0.0135940452656546\\
251	0.013594045264844\\
252	0.0135940452640194\\
253	0.0135940452631803\\
254	0.0135940452623265\\
255	0.0135940452614578\\
256	0.0135940452605739\\
257	0.0135940452596745\\
258	0.0135940452587593\\
259	0.0135940452578282\\
260	0.0135940452568807\\
261	0.0135940452559166\\
262	0.0135940452549357\\
263	0.0135940452539375\\
264	0.0135940452529218\\
265	0.0135940452518883\\
266	0.0135940452508367\\
267	0.0135940452497666\\
268	0.0135940452486778\\
269	0.0135940452475698\\
270	0.0135940452464424\\
271	0.0135940452452951\\
272	0.0135940452441277\\
273	0.0135940452429398\\
274	0.013594045241731\\
275	0.0135940452405009\\
276	0.0135940452392492\\
277	0.0135940452379755\\
278	0.0135940452366793\\
279	0.0135940452353604\\
280	0.0135940452340182\\
281	0.0135940452326524\\
282	0.0135940452312625\\
283	0.0135940452298481\\
284	0.0135940452284088\\
285	0.0135940452269441\\
286	0.0135940452254536\\
287	0.0135940452239368\\
288	0.0135940452223932\\
289	0.0135940452208224\\
290	0.0135940452192239\\
291	0.0135940452175972\\
292	0.0135940452159417\\
293	0.013594045214257\\
294	0.0135940452125426\\
295	0.0135940452107978\\
296	0.0135940452090223\\
297	0.0135940452072153\\
298	0.0135940452053764\\
299	0.0135940452035051\\
300	0.0135940452016006\\
301	0.0135940451996625\\
302	0.01359404519769\\
303	0.0135940451956827\\
304	0.0135940451936399\\
305	0.013594045191561\\
306	0.0135940451894453\\
307	0.0135940451872921\\
308	0.0135940451851009\\
309	0.0135940451828709\\
310	0.0135940451806014\\
311	0.0135940451782918\\
312	0.0135940451759414\\
313	0.0135940451735493\\
314	0.0135940451711149\\
315	0.0135940451686375\\
316	0.0135940451661162\\
317	0.0135940451635503\\
318	0.013594045160939\\
319	0.0135940451582816\\
320	0.0135940451555772\\
321	0.0135940451528249\\
322	0.013594045150024\\
323	0.0135940451471736\\
324	0.0135940451442729\\
325	0.0135940451413209\\
326	0.0135940451383167\\
327	0.0135940451352595\\
328	0.0135940451321484\\
329	0.0135940451289823\\
330	0.0135940451257604\\
331	0.0135940451224817\\
332	0.0135940451191452\\
333	0.0135940451157498\\
334	0.0135940451122947\\
335	0.0135940451087787\\
336	0.0135940451052008\\
337	0.01359404510156\\
338	0.0135940450978552\\
339	0.0135940450940852\\
340	0.013594045090249\\
341	0.0135940450863455\\
342	0.0135940450823734\\
343	0.0135940450783316\\
344	0.013594045074219\\
345	0.0135940450700343\\
346	0.0135940450657763\\
347	0.0135940450614438\\
348	0.0135940450570355\\
349	0.01359404505255\\
350	0.0135940450479862\\
351	0.0135940450433426\\
352	0.013594045038618\\
353	0.0135940450338109\\
354	0.01359404502892\\
355	0.0135940450239437\\
356	0.0135940450188808\\
357	0.0135940450137296\\
358	0.0135940450084887\\
359	0.0135940450031565\\
360	0.0135940449977316\\
361	0.0135940449922122\\
362	0.0135940449865968\\
363	0.0135940449808836\\
364	0.0135940449750711\\
365	0.0135940449691574\\
366	0.0135940449631408\\
367	0.0135940449570195\\
368	0.0135940449507916\\
369	0.0135940449444553\\
370	0.0135940449380084\\
371	0.0135940449314492\\
372	0.0135940449247755\\
373	0.0135940449179853\\
374	0.0135940449110763\\
375	0.0135940449040464\\
376	0.0135940448968933\\
377	0.0135940448896146\\
378	0.013594044882208\\
379	0.013594044874671\\
380	0.013594044867001\\
381	0.0135940448591953\\
382	0.0135940448512514\\
383	0.0135940448431664\\
384	0.0135940448349373\\
385	0.0135940448265613\\
386	0.0135940448180352\\
387	0.0135940448093558\\
388	0.0135940448005198\\
389	0.0135940447915238\\
390	0.0135940447823642\\
391	0.0135940447730373\\
392	0.0135940447635393\\
393	0.0135940447538661\\
394	0.0135940447440135\\
395	0.0135940447339772\\
396	0.0135940447237525\\
397	0.0135940447133346\\
398	0.0135940447027184\\
399	0.0135940446918985\\
400	0.0135940446808691\\
401	0.0135940446696245\\
402	0.0135940446581582\\
403	0.0135940446464636\\
404	0.0135940446345335\\
405	0.0135940446223604\\
406	0.0135940446099363\\
407	0.0135940445972529\\
408	0.0135940445843011\\
409	0.0135940445710718\\
410	0.013594044557555\\
411	0.0135940445437405\\
412	0.0135940445296173\\
413	0.0135940445151736\\
414	0.0135940445003966\\
415	0.0135940444852728\\
416	0.0135940444697881\\
417	0.0135940444539272\\
418	0.0135940444376739\\
419	0.01359404442101\\
420	0.0135940444039162\\
421	0.013594044386371\\
422	0.0135940443683509\\
423	0.0135940443498298\\
424	0.013594044330779\\
425	0.0135940443111661\\
426	0.0135940442909547\\
427	0.0135940442701025\\
428	0.0135940442485587\\
429	0.0135940442262581\\
430	0.0135940442031082\\
431	0.0135940441789624\\
432	0.0135940441535643\\
433	0.0135940441264396\\
434	0.0135940440967092\\
435	0.0135940440628293\\
436	0.013594044022417\\
437	0.0135940439727302\\
438	0.0135940439129532\\
439	0.0135940438482276\\
440	0.0135940437820306\\
441	0.0135940437143174\\
442	0.0135940436450416\\
443	0.0135940435741556\\
444	0.0135940435016101\\
445	0.0135940434273548\\
446	0.0135940433513381\\
447	0.0135940432735077\\
448	0.0135940431938102\\
449	0.0135940431121919\\
450	0.0135940430285971\\
451	0.0135940429429669\\
452	0.0135940428552326\\
453	0.0135940427653014\\
454	0.013594042673022\\
455	0.0135940425781079\\
456	0.0135940424799621\\
457	0.0135940423772787\\
458	0.013594042267121\\
459	0.0135940421427694\\
460	0.0135940419888291\\
461	0.0135940417713454\\
462	0.0135940414246864\\
463	0.0135940408468957\\
464	0.0135940399671601\\
465	0.0135940390712872\\
466	0.0135940381587915\\
467	0.0135940372291632\\
468	0.0135940362818697\\
469	0.0135940353163476\\
470	0.0135940343320011\\
471	0.01359403332821\\
472	0.0135940323043265\\
473	0.0135940312596649\\
474	0.0135940301934835\\
475	0.0135940291049847\\
476	0.0135940279933754\\
477	0.0135940268578239\\
478	0.0135940256974439\\
479	0.0135940245112887\\
480	0.0135940232983459\\
481	0.0135940220575301\\
482	0.013594020787675\\
483	0.0135940194875245\\
484	0.013594018155723\\
485	0.0135940167908042\\
486	0.0135940153911767\\
487	0.0135940139551084\\
488	0.0135940124807073\\
489	0.0135940109659\\
490	0.0135940094084036\\
491	0.0135940078056898\\
492	0.0135940061549331\\
493	0.0135940044529279\\
494	0.0135940026959357\\
495	0.0135940008793747\\
496	0.0135939989971485\\
497	0.0135939970401605\\
498	0.0135939949930508\\
499	0.0135939928272663\\
500	0.0135939904872408\\
501	0.0135939878657466\\
502	0.0135939847690407\\
503	0.01359398089622\\
504	0.013593975930866\\
505	0.0135939702662746\\
506	0.013593964498087\\
507	0.0135939586222699\\
508	0.0135939526333629\\
509	0.0135939465230392\\
510	0.013593940283437\\
511	0.0135939339079869\\
512	0.013593927389427\\
513	0.0135939207196255\\
514	0.0135939138892086\\
515	0.0135939068866672\\
516	0.0135938996960378\\
517	0.013593892290677\\
518	0.0135938846163582\\
519	0.0135938765454359\\
520	0.0135938677539973\\
521	0.0135938574003501\\
522	0.0135938433194432\\
523	0.0135938201658801\\
524	0.0135935886646046\\
525	0.0135928584855939\\
526	0.0135921057763526\\
527	0.0135913287160958\\
528	0.0135905250633436\\
529	0.01358969224065\\
530	0.0135888284581564\\
531	0.0135879321713535\\
532	0.0135869997092166\\
533	0.0135860263605028\\
534	0.0135850054939404\\
535	0.0135839264419661\\
536	0.013582770721826\\
537	0.0135808328121808\\
538	0.0135778902427088\\
539	0.013574859417277\\
540	0.0135717334431941\\
541	0.0135685045635268\\
542	0.0135651640488818\\
543	0.0135617020033984\\
544	0.0135581068285905\\
545	0.013554364611253\\
546	0.0135504579475602\\
547	0.0135463642743097\\
548	0.013542049651676\\
549	0.0135372492329162\\
550	0.0135287751582943\\
551	0.0135200154047505\\
552	0.0135109460369536\\
553	0.0135015375256678\\
554	0.0134917495954427\\
555	0.0134814949641083\\
556	0.0134657542586436\\
557	0.0134485344487215\\
558	0.0134304383644168\\
559	0.0134112877355424\\
560	0.0133826077339614\\
561	0.0133495839111871\\
562	0.0133107027875717\\
563	0.0132651074942661\\
564	0.0132174550510279\\
565	0.0131673829773579\\
566	0.0130935034775975\\
567	0.0130163853987856\\
568	0.0129361503549666\\
569	0.0128524923811328\\
570	0.0127649775367202\\
571	0.0126730522243384\\
572	0.0125759874486968\\
573	0.0124399283302884\\
574	0.0122820145229851\\
575	0.0121151381544788\\
576	0.0119201782367025\\
577	0.0117158816622068\\
578	0.0115013017629722\\
579	0.0113047644849782\\
580	0.0111157007364945\\
581	0.0109411421379581\\
582	0.0106781870066112\\
583	0.0103761124222753\\
584	0.0101250412749154\\
585	0.00986943912818245\\
586	0.00960283510563681\\
587	0.00932247274674759\\
588	0.00902904009979084\\
589	0.00884170618614843\\
590	0.00861050494941614\\
591	0.00837652661868457\\
592	0.00812230596562506\\
593	0.00756248410970657\\
594	0.00695994490712135\\
595	0.00626582126936309\\
596	0.00583905595420081\\
597	0.00512683753504545\\
598	0.00366374385960312\\
599	0\\
600	0\\
};
\addplot [color=mycolor12,solid,forget plot]
  table[row sep=crcr]{%
1	0.0136268878279312\\
2	0.0136268878279307\\
3	0.0136268878279302\\
4	0.0136268878279297\\
5	0.0136268878279292\\
6	0.0136268878279286\\
7	0.0136268878279281\\
8	0.0136268878279275\\
9	0.0136268878279269\\
10	0.0136268878279264\\
11	0.0136268878279258\\
12	0.0136268878279252\\
13	0.0136268878279246\\
14	0.013626887827924\\
15	0.0136268878279234\\
16	0.0136268878279227\\
17	0.0136268878279221\\
18	0.0136268878279214\\
19	0.0136268878279208\\
20	0.0136268878279201\\
21	0.0136268878279194\\
22	0.0136268878279187\\
23	0.013626887827918\\
24	0.0136268878279172\\
25	0.0136268878279165\\
26	0.0136268878279158\\
27	0.013626887827915\\
28	0.0136268878279142\\
29	0.0136268878279134\\
30	0.0136268878279126\\
31	0.0136268878279118\\
32	0.013626887827911\\
33	0.0136268878279101\\
34	0.0136268878279093\\
35	0.0136268878279084\\
36	0.0136268878279075\\
37	0.0136268878279066\\
38	0.0136268878279057\\
39	0.0136268878279048\\
40	0.0136268878279038\\
41	0.0136268878279029\\
42	0.0136268878279019\\
43	0.0136268878279009\\
44	0.0136268878278999\\
45	0.0136268878278989\\
46	0.0136268878278978\\
47	0.0136268878278967\\
48	0.0136268878278957\\
49	0.0136268878278946\\
50	0.0136268878278934\\
51	0.0136268878278923\\
52	0.0136268878278911\\
53	0.01362688782789\\
54	0.0136268878278888\\
55	0.0136268878278876\\
56	0.0136268878278863\\
57	0.0136268878278851\\
58	0.0136268878278838\\
59	0.0136268878278825\\
60	0.0136268878278812\\
61	0.0136268878278798\\
62	0.0136268878278784\\
63	0.013626887827877\\
64	0.0136268878278756\\
65	0.0136268878278742\\
66	0.0136268878278727\\
67	0.0136268878278712\\
68	0.0136268878278697\\
69	0.0136268878278682\\
70	0.0136268878278666\\
71	0.013626887827865\\
72	0.0136268878278634\\
73	0.0136268878278618\\
74	0.0136268878278601\\
75	0.0136268878278584\\
76	0.0136268878278567\\
77	0.0136268878278549\\
78	0.0136268878278531\\
79	0.0136268878278513\\
80	0.0136268878278495\\
81	0.0136268878278476\\
82	0.0136268878278457\\
83	0.0136268878278437\\
84	0.0136268878278418\\
85	0.0136268878278397\\
86	0.0136268878278377\\
87	0.0136268878278356\\
88	0.0136268878278335\\
89	0.0136268878278314\\
90	0.0136268878278292\\
91	0.013626887827827\\
92	0.0136268878278247\\
93	0.0136268878278224\\
94	0.0136268878278201\\
95	0.0136268878278177\\
96	0.0136268878278153\\
97	0.0136268878278128\\
98	0.0136268878278103\\
99	0.0136268878278078\\
100	0.0136268878278052\\
101	0.0136268878278025\\
102	0.0136268878277999\\
103	0.0136268878277971\\
104	0.0136268878277944\\
105	0.0136268878277916\\
106	0.0136268878277887\\
107	0.0136268878277858\\
108	0.0136268878277828\\
109	0.0136268878277798\\
110	0.0136268878277768\\
111	0.0136268878277737\\
112	0.0136268878277705\\
113	0.0136268878277673\\
114	0.013626887827764\\
115	0.0136268878277607\\
116	0.0136268878277573\\
117	0.0136268878277538\\
118	0.0136268878277503\\
119	0.0136268878277468\\
120	0.0136268878277431\\
121	0.0136268878277395\\
122	0.0136268878277357\\
123	0.0136268878277319\\
124	0.013626887827728\\
125	0.0136268878277241\\
126	0.0136268878277201\\
127	0.013626887827716\\
128	0.0136268878277118\\
129	0.0136268878277076\\
130	0.0136268878277033\\
131	0.013626887827699\\
132	0.0136268878276945\\
133	0.01362688782769\\
134	0.0136268878276854\\
135	0.0136268878276807\\
136	0.013626887827676\\
137	0.0136268878276712\\
138	0.0136268878276662\\
139	0.0136268878276612\\
140	0.0136268878276561\\
141	0.013626887827651\\
142	0.0136268878276457\\
143	0.0136268878276404\\
144	0.0136268878276349\\
145	0.0136268878276294\\
146	0.0136268878276237\\
147	0.013626887827618\\
148	0.0136268878276122\\
149	0.0136268878276062\\
150	0.0136268878276002\\
151	0.0136268878275941\\
152	0.0136268878275878\\
153	0.0136268878275815\\
154	0.013626887827575\\
155	0.0136268878275685\\
156	0.0136268878275618\\
157	0.013626887827555\\
158	0.0136268878275481\\
159	0.013626887827541\\
160	0.0136268878275339\\
161	0.0136268878275266\\
162	0.0136268878275192\\
163	0.0136268878275117\\
164	0.013626887827504\\
165	0.0136268878274962\\
166	0.0136268878274883\\
167	0.0136268878274802\\
168	0.013626887827472\\
169	0.0136268878274637\\
170	0.0136268878274552\\
171	0.0136268878274465\\
172	0.0136268878274378\\
173	0.0136268878274288\\
174	0.0136268878274197\\
175	0.0136268878274105\\
176	0.0136268878274011\\
177	0.0136268878273915\\
178	0.0136268878273818\\
179	0.0136268878273719\\
180	0.0136268878273618\\
181	0.0136268878273515\\
182	0.0136268878273411\\
183	0.0136268878273305\\
184	0.0136268878273197\\
185	0.0136268878273087\\
186	0.0136268878272976\\
187	0.0136268878272862\\
188	0.0136268878272747\\
189	0.0136268878272629\\
190	0.0136268878272509\\
191	0.0136268878272388\\
192	0.0136268878272264\\
193	0.0136268878272138\\
194	0.013626887827201\\
195	0.013626887827188\\
196	0.0136268878271747\\
197	0.0136268878271612\\
198	0.0136268878271475\\
199	0.0136268878271335\\
200	0.0136268878271194\\
201	0.0136268878271049\\
202	0.0136268878270902\\
203	0.0136268878270753\\
204	0.0136268878270601\\
205	0.0136268878270446\\
206	0.0136268878270288\\
207	0.0136268878270128\\
208	0.0136268878269965\\
209	0.01362688782698\\
210	0.0136268878269631\\
211	0.013626887826946\\
212	0.0136268878269285\\
213	0.0136268878269108\\
214	0.0136268878268927\\
215	0.0136268878268743\\
216	0.0136268878268556\\
217	0.0136268878268366\\
218	0.0136268878268173\\
219	0.0136268878267976\\
220	0.0136268878267776\\
221	0.0136268878267572\\
222	0.0136268878267365\\
223	0.0136268878267154\\
224	0.0136268878266939\\
225	0.0136268878266721\\
226	0.0136268878266499\\
227	0.0136268878266273\\
228	0.0136268878266043\\
229	0.013626887826581\\
230	0.0136268878265572\\
231	0.013626887826533\\
232	0.0136268878265083\\
233	0.0136268878264833\\
234	0.0136268878264578\\
235	0.0136268878264319\\
236	0.0136268878264055\\
237	0.0136268878263786\\
238	0.0136268878263513\\
239	0.0136268878263235\\
240	0.0136268878262952\\
241	0.0136268878262665\\
242	0.0136268878262372\\
243	0.0136268878262074\\
244	0.0136268878261771\\
245	0.0136268878261463\\
246	0.0136268878261149\\
247	0.013626887826083\\
248	0.0136268878260505\\
249	0.0136268878260175\\
250	0.0136268878259839\\
251	0.0136268878259497\\
252	0.0136268878259149\\
253	0.0136268878258795\\
254	0.0136268878258434\\
255	0.0136268878258068\\
256	0.0136268878257695\\
257	0.0136268878257315\\
258	0.0136268878256929\\
259	0.0136268878256536\\
260	0.0136268878256136\\
261	0.0136268878255729\\
262	0.0136268878255315\\
263	0.0136268878254894\\
264	0.0136268878254465\\
265	0.0136268878254029\\
266	0.0136268878253585\\
267	0.0136268878253133\\
268	0.0136268878252674\\
269	0.0136268878252206\\
270	0.013626887825173\\
271	0.0136268878251246\\
272	0.0136268878250754\\
273	0.0136268878250252\\
274	0.0136268878249742\\
275	0.0136268878249223\\
276	0.0136268878248694\\
277	0.0136268878248157\\
278	0.013626887824761\\
279	0.0136268878247053\\
280	0.0136268878246487\\
281	0.013626887824591\\
282	0.0136268878245323\\
283	0.0136268878244726\\
284	0.0136268878244119\\
285	0.0136268878243501\\
286	0.0136268878242872\\
287	0.0136268878242231\\
288	0.013626887824158\\
289	0.0136268878240917\\
290	0.0136268878240242\\
291	0.0136268878239555\\
292	0.0136268878238857\\
293	0.0136268878238146\\
294	0.0136268878237422\\
295	0.0136268878236685\\
296	0.0136268878235936\\
297	0.0136268878235173\\
298	0.0136268878234397\\
299	0.0136268878233607\\
300	0.0136268878232803\\
301	0.0136268878231985\\
302	0.0136268878231153\\
303	0.0136268878230305\\
304	0.0136268878229443\\
305	0.0136268878228566\\
306	0.0136268878227673\\
307	0.0136268878226764\\
308	0.0136268878225839\\
309	0.0136268878224898\\
310	0.013626887822394\\
311	0.0136268878222965\\
312	0.0136268878221973\\
313	0.0136268878220963\\
314	0.0136268878219936\\
315	0.013626887821889\\
316	0.0136268878217826\\
317	0.0136268878216743\\
318	0.0136268878215641\\
319	0.0136268878214519\\
320	0.0136268878213378\\
321	0.0136268878212216\\
322	0.0136268878211034\\
323	0.0136268878209831\\
324	0.0136268878208607\\
325	0.0136268878207361\\
326	0.0136268878206093\\
327	0.0136268878204803\\
328	0.013626887820349\\
329	0.0136268878202154\\
330	0.0136268878200794\\
331	0.0136268878199411\\
332	0.0136268878198003\\
333	0.013626887819657\\
334	0.0136268878195112\\
335	0.0136268878193628\\
336	0.0136268878192119\\
337	0.0136268878190583\\
338	0.0136268878189019\\
339	0.0136268878187429\\
340	0.013626887818581\\
341	0.0136268878184163\\
342	0.0136268878182487\\
343	0.0136268878180782\\
344	0.0136268878179047\\
345	0.0136268878177282\\
346	0.0136268878175485\\
347	0.0136268878173658\\
348	0.0136268878171798\\
349	0.0136268878169906\\
350	0.0136268878167981\\
351	0.0136268878166022\\
352	0.0136268878164029\\
353	0.0136268878162001\\
354	0.0136268878159938\\
355	0.013626887815784\\
356	0.0136268878155704\\
357	0.0136268878153532\\
358	0.0136268878151321\\
359	0.0136268878149072\\
360	0.0136268878146784\\
361	0.0136268878144457\\
362	0.0136268878142088\\
363	0.0136268878139679\\
364	0.0136268878137228\\
365	0.0136268878134734\\
366	0.0136268878132196\\
367	0.0136268878129615\\
368	0.0136268878126988\\
369	0.0136268878124316\\
370	0.0136268878121597\\
371	0.0136268878118831\\
372	0.0136268878116017\\
373	0.0136268878113153\\
374	0.0136268878110239\\
375	0.0136268878107274\\
376	0.0136268878104257\\
377	0.0136268878101187\\
378	0.0136268878098063\\
379	0.0136268878094884\\
380	0.0136268878091649\\
381	0.0136268878088356\\
382	0.0136268878085005\\
383	0.0136268878081594\\
384	0.0136268878078122\\
385	0.0136268878074587\\
386	0.0136268878070989\\
387	0.0136268878067327\\
388	0.0136268878063597\\
389	0.01362688780598\\
390	0.0136268878055934\\
391	0.0136268878051996\\
392	0.0136268878047985\\
393	0.01362688780439\\
394	0.0136268878039739\\
395	0.0136268878035499\\
396	0.0136268878031179\\
397	0.0136268878026777\\
398	0.0136268878022289\\
399	0.0136268878017715\\
400	0.013626887801305\\
401	0.0136268878008294\\
402	0.0136268878003442\\
403	0.0136268877998492\\
404	0.0136268877993441\\
405	0.0136268877988285\\
406	0.0136268877983021\\
407	0.0136268877977644\\
408	0.0136268877972152\\
409	0.013626887796654\\
410	0.0136268877960803\\
411	0.0136268877954937\\
412	0.0136268877948937\\
413	0.0136268877942797\\
414	0.0136268877936513\\
415	0.0136268877930076\\
416	0.0136268877923483\\
417	0.0136268877916724\\
418	0.0136268877909792\\
419	0.013626887790268\\
420	0.0136268877895377\\
421	0.0136268877887874\\
422	0.0136268877880159\\
423	0.013626887787222\\
424	0.0136268877864041\\
425	0.0136268877855607\\
426	0.0136268877846897\\
427	0.0136268877837885\\
428	0.0136268877828534\\
429	0.0136268877818787\\
430	0.0136268877808552\\
431	0.0136268877797667\\
432	0.0136268877785851\\
433	0.0136268877772648\\
434	0.0136268877757375\\
435	0.0136268877739199\\
436	0.0136268877717504\\
437	0.0136268877692613\\
438	0.0136268877666197\\
439	0.0136268877639182\\
440	0.0136268877611548\\
441	0.0136268877583276\\
442	0.0136268877554348\\
443	0.0136268877524744\\
444	0.0136268877494442\\
445	0.0136268877463424\\
446	0.0136268877431666\\
447	0.0136268877399149\\
448	0.0136268877365848\\
449	0.0136268877331741\\
450	0.0136268877296796\\
451	0.0136268877260972\\
452	0.0136268877224199\\
453	0.0136268877186344\\
454	0.0136268877147134\\
455	0.0136268877105991\\
456	0.0136268877061689\\
457	0.0136268877011652\\
458	0.0136268876950624\\
459	0.0136268876868414\\
460	0.0136268876747063\\
461	0.0136268876560219\\
462	0.0136268876281971\\
463	0.0136268875918093\\
464	0.0136268875547535\\
465	0.0136268875170094\\
466	0.0136268874785559\\
467	0.0136268874393708\\
468	0.0136268873994307\\
469	0.0136268873587108\\
470	0.0136268873171855\\
471	0.0136268872748276\\
472	0.0136268872316083\\
473	0.0136268871874966\\
474	0.0136268871424598\\
475	0.0136268870964648\\
476	0.0136268870494767\\
477	0.0136268870014582\\
478	0.0136268869523698\\
479	0.0136268869021687\\
480	0.0136268868508092\\
481	0.0136268867982423\\
482	0.0136268867444148\\
483	0.0136268866892695\\
484	0.0136268866327443\\
485	0.0136268865747716\\
486	0.0136268865152779\\
487	0.0136268864541824\\
488	0.0136268863913966\\
489	0.013626886326822\\
490	0.0136268862603486\\
491	0.0136268861918511\\
492	0.0136268861211822\\
493	0.0136268860481596\\
494	0.0136268859725398\\
495	0.0136268858939634\\
496	0.0136268858118456\\
497	0.0136268857251639\\
498	0.0136268856320766\\
499	0.0136268855293147\\
500	0.0136268854114232\\
501	0.0136268852703876\\
502	0.0136268850972404\\
503	0.0136268848886566\\
504	0.0136268846601735\\
505	0.0136268844275599\\
506	0.0136268841906322\\
507	0.0136268839491408\\
508	0.0136268837027589\\
509	0.0136268834511831\\
510	0.0136268831941374\\
511	0.0136268829313053\\
512	0.013626882662304\\
513	0.0136268823866249\\
514	0.013626882103487\\
515	0.0136268818114839\\
516	0.0136268815077355\\
517	0.0136268811858971\\
518	0.013626880831645\\
519	0.013626880412956\\
520	0.0136268798608944\\
521	0.0136268790373815\\
522	0.0136268777013764\\
523	0.0136268755541293\\
524	0.0136268726781752\\
525	0.0136268697003445\\
526	0.013626866600694\\
527	0.0136268633550434\\
528	0.0136268599452971\\
529	0.0136268563819244\\
530	0.0136268526770398\\
531	0.0136268487878813\\
532	0.013626844625871\\
533	0.0136268400092747\\
534	0.0136268345890394\\
535	0.013626827800565\\
536	0.0136268190273637\\
537	0.0136268088632286\\
538	0.0136267984599363\\
539	0.0136267877978721\\
540	0.0136267768548274\\
541	0.01362676560487\\
542	0.0136267540121859\\
543	0.0136267420127902\\
544	0.0136267294854962\\
545	0.0136267161994809\\
546	0.0136267017334953\\
547	0.0136266853329548\\
548	0.0136266658993872\\
549	0.0136266428322623\\
550	0.0136266190480606\\
551	0.0136265941294955\\
552	0.0136265670640316\\
553	0.0136265352048811\\
554	0.0136264915555887\\
555	0.0136264192183494\\
556	0.0136252301555777\\
557	0.0136238102218841\\
558	0.0136222972782903\\
559	0.0136206564703397\\
560	0.013617039838498\\
561	0.0136126015173747\\
562	0.0136079768228116\\
563	0.0136031411721728\\
564	0.0135980533199211\\
565	0.0135926446344186\\
566	0.0135822278964534\\
567	0.0135713356376805\\
568	0.0135600086889985\\
569	0.0135481945730692\\
570	0.0135358286051002\\
571	0.0135228252962376\\
572	0.0135090622348228\\
573	0.0134871046514208\\
574	0.0134607346999303\\
575	0.0134331506381204\\
576	0.0134042215466356\\
577	0.0133735941471642\\
578	0.013340749615476\\
579	0.0132830563963814\\
580	0.0132124138014288\\
581	0.0131172554217482\\
582	0.0130009559658341\\
583	0.0128731876267521\\
584	0.0126887187729786\\
585	0.0124916626897068\\
586	0.0122889547607575\\
587	0.0120852960853785\\
588	0.0118698675681612\\
589	0.0115422726391937\\
590	0.0111928033866319\\
591	0.010825014474107\\
592	0.010434224581246\\
593	0.00998273020622688\\
594	0.00950216794138242\\
595	0.00896778970282203\\
596	0.00790389929454537\\
597	0.00638278975403702\\
598	0.00366374385960312\\
599	0\\
600	0\\
};
\addplot [color=mycolor13,solid,forget plot]
  table[row sep=crcr]{%
1	0.00612333152159056\\
2	0.00612333152160971\\
3	0.0061233315216292\\
4	0.00612333152164904\\
5	0.00612333152166924\\
6	0.0061233315216898\\
7	0.00612333152171073\\
8	0.00612333152173202\\
9	0.00612333152175371\\
10	0.00612333152177577\\
11	0.00612333152179824\\
12	0.0061233315218211\\
13	0.00612333152184437\\
14	0.00612333152186806\\
15	0.00612333152189217\\
16	0.00612333152191671\\
17	0.0061233315219417\\
18	0.00612333152196712\\
19	0.00612333152199301\\
20	0.00612333152201935\\
21	0.00612333152204617\\
22	0.00612333152207346\\
23	0.00612333152210125\\
24	0.00612333152212952\\
25	0.00612333152215831\\
26	0.00612333152218761\\
27	0.00612333152221743\\
28	0.00612333152224779\\
29	0.00612333152227868\\
30	0.00612333152231013\\
31	0.00612333152234215\\
32	0.00612333152237473\\
33	0.00612333152240789\\
34	0.00612333152244166\\
35	0.00612333152247602\\
36	0.00612333152251099\\
37	0.00612333152254659\\
38	0.00612333152258283\\
39	0.00612333152261971\\
40	0.00612333152265725\\
41	0.00612333152269546\\
42	0.00612333152273436\\
43	0.00612333152277395\\
44	0.00612333152281424\\
45	0.00612333152285526\\
46	0.00612333152289701\\
47	0.00612333152293951\\
48	0.00612333152298276\\
49	0.00612333152302679\\
50	0.0061233315230716\\
51	0.00612333152311721\\
52	0.00612333152316364\\
53	0.00612333152321089\\
54	0.00612333152325899\\
55	0.00612333152330795\\
56	0.00612333152335778\\
57	0.00612333152340851\\
58	0.00612333152346013\\
59	0.00612333152351268\\
60	0.00612333152356617\\
61	0.00612333152362062\\
62	0.00612333152367603\\
63	0.00612333152373244\\
64	0.00612333152378985\\
65	0.00612333152384828\\
66	0.00612333152390776\\
67	0.0061233315239683\\
68	0.00612333152402993\\
69	0.00612333152409265\\
70	0.00612333152415649\\
71	0.00612333152422147\\
72	0.00612333152428761\\
73	0.00612333152435493\\
74	0.00612333152442345\\
75	0.0061233315244932\\
76	0.00612333152456419\\
77	0.00612333152463645\\
78	0.00612333152471\\
79	0.00612333152478486\\
80	0.00612333152486105\\
81	0.0061233315249386\\
82	0.00612333152501754\\
83	0.00612333152509789\\
84	0.00612333152517968\\
85	0.00612333152526292\\
86	0.00612333152534764\\
87	0.00612333152543388\\
88	0.00612333152552166\\
89	0.006123331525611\\
90	0.00612333152570194\\
91	0.0061233315257945\\
92	0.00612333152588871\\
93	0.00612333152598461\\
94	0.00612333152608221\\
95	0.00612333152618156\\
96	0.00612333152628268\\
97	0.0061233315263856\\
98	0.00612333152649035\\
99	0.00612333152659698\\
100	0.00612333152670551\\
101	0.00612333152681598\\
102	0.00612333152692842\\
103	0.00612333152704286\\
104	0.00612333152715934\\
105	0.00612333152727791\\
106	0.00612333152739859\\
107	0.00612333152752141\\
108	0.00612333152764644\\
109	0.00612333152777369\\
110	0.00612333152790321\\
111	0.00612333152803505\\
112	0.00612333152816923\\
113	0.00612333152830581\\
114	0.00612333152844483\\
115	0.00612333152858632\\
116	0.00612333152873034\\
117	0.00612333152887693\\
118	0.00612333152902614\\
119	0.006123331529178\\
120	0.00612333152933258\\
121	0.00612333152948991\\
122	0.00612333152965005\\
123	0.00612333152981305\\
124	0.00612333152997896\\
125	0.00612333153014782\\
126	0.0061233315303197\\
127	0.00612333153049464\\
128	0.00612333153067271\\
129	0.00612333153085395\\
130	0.00612333153103843\\
131	0.0061233315312262\\
132	0.00612333153141732\\
133	0.00612333153161184\\
134	0.00612333153180984\\
135	0.00612333153201138\\
136	0.0061233315322165\\
137	0.00612333153242529\\
138	0.00612333153263781\\
139	0.00612333153285411\\
140	0.00612333153307428\\
141	0.00612333153329838\\
142	0.00612333153352647\\
143	0.00612333153375864\\
144	0.00612333153399495\\
145	0.00612333153423547\\
146	0.00612333153448029\\
147	0.00612333153472948\\
148	0.00612333153498312\\
149	0.00612333153524129\\
150	0.00612333153550406\\
151	0.00612333153577153\\
152	0.00612333153604377\\
153	0.00612333153632087\\
154	0.00612333153660292\\
155	0.00612333153689\\
156	0.00612333153718221\\
157	0.00612333153747964\\
158	0.00612333153778238\\
159	0.00612333153809053\\
160	0.00612333153840418\\
161	0.00612333153872344\\
162	0.00612333153904839\\
163	0.00612333153937916\\
164	0.00612333153971583\\
165	0.00612333154005851\\
166	0.00612333154040732\\
167	0.00612333154076236\\
168	0.00612333154112374\\
169	0.00612333154149159\\
170	0.00612333154186601\\
171	0.00612333154224712\\
172	0.00612333154263504\\
173	0.0061233315430299\\
174	0.00612333154343182\\
175	0.00612333154384092\\
176	0.00612333154425734\\
177	0.00612333154468121\\
178	0.00612333154511266\\
179	0.00612333154555183\\
180	0.00612333154599885\\
181	0.00612333154645387\\
182	0.00612333154691703\\
183	0.00612333154738848\\
184	0.00612333154786837\\
185	0.00612333154835685\\
186	0.00612333154885407\\
187	0.00612333154936019\\
188	0.00612333154987538\\
189	0.00612333155039979\\
190	0.0061233315509336\\
191	0.00612333155147697\\
192	0.00612333155203007\\
193	0.00612333155259308\\
194	0.00612333155316619\\
195	0.00612333155374956\\
196	0.00612333155434339\\
197	0.00612333155494787\\
198	0.00612333155556319\\
199	0.00612333155618955\\
200	0.00612333155682714\\
201	0.00612333155747617\\
202	0.00612333155813684\\
203	0.00612333155880937\\
204	0.00612333155949397\\
205	0.00612333156019087\\
206	0.00612333156090028\\
207	0.00612333156162243\\
208	0.00612333156235755\\
209	0.00612333156310589\\
210	0.00612333156386767\\
211	0.00612333156464314\\
212	0.00612333156543255\\
213	0.00612333156623616\\
214	0.00612333156705423\\
215	0.00612333156788701\\
216	0.00612333156873478\\
217	0.0061233315695978\\
218	0.00612333157047637\\
219	0.00612333157137075\\
220	0.00612333157228125\\
221	0.00612333157320814\\
222	0.00612333157415174\\
223	0.00612333157511235\\
224	0.00612333157609028\\
225	0.00612333157708585\\
226	0.00612333157809938\\
227	0.00612333157913119\\
228	0.00612333158018164\\
229	0.00612333158125104\\
230	0.00612333158233976\\
231	0.00612333158344815\\
232	0.00612333158457657\\
233	0.00612333158572539\\
234	0.00612333158689498\\
235	0.00612333158808572\\
236	0.00612333158929801\\
237	0.00612333159053225\\
238	0.00612333159178882\\
239	0.00612333159306816\\
240	0.00612333159437068\\
241	0.00612333159569679\\
242	0.00612333159704696\\
243	0.00612333159842161\\
244	0.0061233315998212\\
245	0.00612333160124619\\
246	0.00612333160269706\\
247	0.00612333160417427\\
248	0.00612333160567832\\
249	0.00612333160720971\\
250	0.00612333160876895\\
251	0.00612333161035655\\
252	0.00612333161197304\\
253	0.00612333161361895\\
254	0.00612333161529484\\
255	0.00612333161700126\\
256	0.00612333161873879\\
257	0.00612333162050799\\
258	0.00612333162230947\\
259	0.00612333162414382\\
260	0.00612333162601166\\
261	0.00612333162791361\\
262	0.00612333162985031\\
263	0.00612333163182242\\
264	0.00612333163383059\\
265	0.0061233316358755\\
266	0.00612333163795784\\
267	0.0061233316400783\\
268	0.00612333164223762\\
269	0.00612333164443651\\
270	0.00612333164667572\\
271	0.006123331648956\\
272	0.00612333165127813\\
273	0.00612333165364289\\
274	0.00612333165605109\\
275	0.00612333165850355\\
276	0.00612333166100109\\
277	0.00612333166354458\\
278	0.00612333166613486\\
279	0.00612333166877283\\
280	0.00612333167145939\\
281	0.00612333167419545\\
282	0.00612333167698195\\
283	0.00612333167981984\\
284	0.00612333168271008\\
285	0.00612333168565368\\
286	0.00612333168865163\\
287	0.00612333169170497\\
288	0.00612333169481475\\
289	0.00612333169798202\\
290	0.00612333170120788\\
291	0.00612333170449343\\
292	0.00612333170783981\\
293	0.00612333171124817\\
294	0.00612333171471967\\
295	0.00612333171825551\\
296	0.00612333172185692\\
297	0.00612333172552512\\
298	0.00612333172926138\\
299	0.00612333173306699\\
300	0.00612333173694327\\
301	0.00612333174089153\\
302	0.00612333174491316\\
303	0.00612333174900952\\
304	0.00612333175318203\\
305	0.00612333175743213\\
306	0.00612333176176128\\
307	0.00612333176617098\\
308	0.00612333177066273\\
309	0.00612333177523808\\
310	0.00612333177989862\\
311	0.00612333178464593\\
312	0.00612333178948165\\
313	0.00612333179440743\\
314	0.00612333179942498\\
315	0.006123331804536\\
316	0.00612333180974224\\
317	0.0061233318150455\\
318	0.00612333182044757\\
319	0.00612333182595031\\
320	0.00612333183155559\\
321	0.00612333183726532\\
322	0.00612333184308144\\
323	0.00612333184900594\\
324	0.00612333185504081\\
325	0.00612333186118812\\
326	0.00612333186744993\\
327	0.00612333187382837\\
328	0.00612333188032559\\
329	0.00612333188694378\\
330	0.00612333189368517\\
331	0.00612333190055203\\
332	0.00612333190754666\\
333	0.0061233319146714\\
334	0.00612333192192864\\
335	0.00612333192932081\\
336	0.00612333193685038\\
337	0.00612333194451985\\
338	0.00612333195233178\\
339	0.00612333196028876\\
340	0.00612333196839344\\
341	0.0061233319766485\\
342	0.00612333198505669\\
343	0.00612333199362078\\
344	0.0061233320023436\\
345	0.00612333201122803\\
346	0.00612333202027701\\
347	0.00612333202949351\\
348	0.00612333203888058\\
349	0.0061233320484413\\
350	0.00612333205817882\\
351	0.00612333206809635\\
352	0.00612333207819715\\
353	0.00612333208848454\\
354	0.00612333209896192\\
355	0.00612333210963274\\
356	0.00612333212050051\\
357	0.00612333213156884\\
358	0.00612333214284137\\
359	0.00612333215432186\\
360	0.00612333216601411\\
361	0.00612333217792204\\
362	0.00612333219004961\\
363	0.0061233322024009\\
364	0.00612333221498009\\
365	0.00612333222779143\\
366	0.00612333224083928\\
367	0.00612333225412815\\
368	0.00612333226766259\\
369	0.00612333228144734\\
370	0.00612333229548723\\
371	0.00612333230978722\\
372	0.00612333232435245\\
373	0.00612333233918817\\
374	0.00612333235429981\\
375	0.00612333236969299\\
376	0.00612333238537349\\
377	0.00612333240134729\\
378	0.00612333241762059\\
379	0.00612333243419982\\
380	0.00612333245109164\\
381	0.006123332468303\\
382	0.00612333248584112\\
383	0.00612333250371355\\
384	0.00612333252192822\\
385	0.00612333254049343\\
386	0.00612333255941793\\
387	0.00612333257871095\\
388	0.00612333259838223\\
389	0.00612333261844197\\
390	0.00612333263890084\\
391	0.00612333265976994\\
392	0.00612333268106099\\
393	0.00612333270278657\\
394	0.0061233327249601\\
395	0.00612333274759572\\
396	0.0061233327707083\\
397	0.00612333279431344\\
398	0.00612333281842746\\
399	0.00612333284306739\\
400	0.00612333286825098\\
401	0.00612333289399696\\
402	0.00612333292032533\\
403	0.00612333294725796\\
404	0.00612333297481895\\
405	0.00612333300303422\\
406	0.00612333303193053\\
407	0.00612333306153556\\
408	0.00612333309187924\\
409	0.00612333312299409\\
410	0.0061233331549155\\
411	0.0061233331876822\\
412	0.00612333322133662\\
413	0.00612333325592553\\
414	0.00612333329150067\\
415	0.00612333332811961\\
416	0.00612333336584694\\
417	0.00612333340475602\\
418	0.00612333344493191\\
419	0.00612333348647652\\
420	0.00612333352951816\\
421	0.0061233335742297\\
422	0.00612333362086176\\
423	0.00612333366980066\\
424	0.00612333372165936\\
425	0.00612333377739507\\
426	0.00612333383839862\\
427	0.00612333390639754\\
428	0.00612333398288767\\
429	0.00612333406789588\\
430	0.00612333415878554\\
431	0.00612333425173984\\
432	0.00612333434682346\\
433	0.00612333444410418\\
434	0.00612333454365384\\
435	0.00612333464555074\\
436	0.00612333474988377\\
437	0.00612333485675714\\
438	0.0061233349662959\\
439	0.00612333507867288\\
440	0.00612333519417509\\
441	0.00612333531335248\\
442	0.00612333543735627\\
443	0.006123335568684\\
444	0.00612333571274892\\
445	0.00612333588098476\\
446	0.00612333609643078\\
447	0.0061233364022547\\
448	0.00612333687062499\\
449	0.00612333760035324\\
450	0.00612333867582629\\
451	0.00612334006033904\\
452	0.00612334151805552\\
453	0.00612334300283723\\
454	0.00612334451497056\\
455	0.00612334605477471\\
456	0.00612334762261034\\
457	0.00612334921888093\\
458	0.00612335084402158\\
459	0.00612335249847965\\
460	0.00612335418271947\\
461	0.00612335589731353\\
462	0.0061233576431221\\
463	0.00612335942126231\\
464	0.00612336123264819\\
465	0.00612336307821368\\
466	0.00612336495893177\\
467	0.00612336687583198\\
468	0.00612336882998634\\
469	0.00612337082254784\\
470	0.00612337285480842\\
471	0.00612337492822521\\
472	0.0061233770443228\\
473	0.00612337920451999\\
474	0.00612338141032804\\
475	0.00612338366335916\\
476	0.00612338596533573\\
477	0.00612338831810056\\
478	0.00612339072362854\\
479	0.00612339318403979\\
480	0.00612339570161451\\
481	0.00612339827881008\\
482	0.00612340091828052\\
483	0.00612340362289906\\
484	0.00612340639578422\\
485	0.00612340924033031\\
486	0.00612341216024336\\
487	0.00612341515958419\\
488	0.0061234182428213\\
489	0.00612342141489905\\
490	0.00612342468133214\\
491	0.00612342804834999\\
492	0.00612343152314132\\
493	0.00612343511430621\\
494	0.00612343883273651\\
495	0.0061234426933556\\
496	0.0061234467184931\\
497	0.00612345094410432\\
498	0.00612345543019226\\
499	0.00612346027535257\\
500	0.00612346562911269\\
501	0.00612347167931795\\
502	0.00612347856496256\\
503	0.00612348617018915\\
504	0.00612349399886649\\
505	0.00612350194271402\\
506	0.00612351001672883\\
507	0.00612351823541286\\
508	0.00612352660578226\\
509	0.00612353513630956\\
510	0.00612354383829377\\
511	0.00612355272923643\\
512	0.00612356184112266\\
513	0.0061235712405534\\
514	0.00612358107670167\\
515	0.00612359169192493\\
516	0.00612360386495179\\
517	0.00612361930707401\\
518	0.00612364155179061\\
519	0.00612367715832927\\
520	0.00612373606591696\\
521	0.00612382697681332\\
522	0.00612394161842495\\
523	0.00612405925830685\\
524	0.00612418014996012\\
525	0.0061243045842138\\
526	0.00612443291031743\\
527	0.0061245654992447\\
528	0.00612470263138651\\
529	0.00612484436620328\\
530	0.00612499084176697\\
531	0.00612514344412451\\
532	0.00612530515912619\\
533	0.00612548106565689\\
534	0.00612567850675068\\
535	0.00612590690897643\\
536	0.00612617082184989\\
537	0.0061264523687656\\
538	0.00612673858468065\\
539	0.00612702980938359\\
540	0.00612732643867049\\
541	0.00612762895759816\\
542	0.00612793801356077\\
543	0.00612825457131476\\
544	0.00612858020262912\\
545	0.00612891755369671\\
546	0.00612927095494212\\
547	0.00612964676683896\\
548	0.00613005186831006\\
549	0.00613048819487575\\
550	0.006130956352701\\
551	0.00613149812525851\\
552	0.00613223680990467\\
553	0.00613349561535179\\
554	0.00613590184192054\\
555	0.00614018159167847\\
556	0.00614488258111959\\
557	0.00615000623395128\\
558	0.00615584271042338\\
559	0.00616286636953639\\
560	0.00617134956848266\\
561	0.00618001076769368\\
562	0.00618890086222981\\
563	0.00619812775862577\\
564	0.00620787765005245\\
565	0.00621836770034789\\
566	0.00622925516894777\\
567	0.00624017357319983\\
568	0.00625114413080589\\
569	0.00626218623723885\\
570	0.0062732588449742\\
571	0.00628396345137723\\
572	0.00629359538457764\\
573	0.0063024565884034\\
574	0.00631193880828286\\
575	0.00632311293512667\\
576	0.00633912660832106\\
577	0.00636892477166563\\
578	0.00643825590142264\\
579	0.00662260426925778\\
580	0.00687560526233001\\
581	0.00716140658477241\\
582	0.00745664658486484\\
583	0.0077664865553155\\
584	0.00808893954834654\\
585	0.00841648692008067\\
586	0.00875120222867915\\
587	0.00909545523576583\\
588	0.0094551562817773\\
589	0.00982732353045407\\
590	0.0102219626909677\\
591	0.0106195309101401\\
592	0.0110105218793053\\
593	0.0113688158692212\\
594	0.0116257831346641\\
595	0.0118959542233756\\
596	0.0121471190838585\\
597	0.0124382624750838\\
598	0.0128122611038263\\
599	0\\
600	0\\
};
\addplot [color=mycolor14,solid,forget plot]
  table[row sep=crcr]{%
1	0.00582958538134794\\
2	0.0058295853817239\\
3	0.00582958538210659\\
4	0.00582958538249612\\
5	0.00582958538289261\\
6	0.00582958538329619\\
7	0.00582958538370699\\
8	0.00582958538412513\\
9	0.00582958538455074\\
10	0.00582958538498397\\
11	0.00582958538542493\\
12	0.00582958538587378\\
13	0.00582958538633065\\
14	0.00582958538679569\\
15	0.00582958538726904\\
16	0.00582958538775085\\
17	0.00582958538824128\\
18	0.00582958538874046\\
19	0.00582958538924857\\
20	0.00582958538976576\\
21	0.00582958539029219\\
22	0.00582958539082803\\
23	0.00582958539137344\\
24	0.0058295853919286\\
25	0.00582958539249368\\
26	0.00582958539306886\\
27	0.00582958539365431\\
28	0.00582958539425023\\
29	0.00582958539485679\\
30	0.00582958539547419\\
31	0.00582958539610262\\
32	0.00582958539674228\\
33	0.00582958539739337\\
34	0.00582958539805608\\
35	0.00582958539873064\\
36	0.00582958539941725\\
37	0.00582958540011612\\
38	0.00582958540082747\\
39	0.00582958540155153\\
40	0.00582958540228853\\
41	0.00582958540303869\\
42	0.00582958540380224\\
43	0.00582958540457944\\
44	0.00582958540537052\\
45	0.00582958540617572\\
46	0.0058295854069953\\
47	0.00582958540782953\\
48	0.00582958540867865\\
49	0.00582958540954293\\
50	0.00582958541042265\\
51	0.00582958541131808\\
52	0.00582958541222949\\
53	0.00582958541315718\\
54	0.00582958541410144\\
55	0.00582958541506256\\
56	0.00582958541604083\\
57	0.00582958541703658\\
58	0.0058295854180501\\
59	0.00582958541908171\\
60	0.00582958542013175\\
61	0.00582958542120053\\
62	0.00582958542228839\\
63	0.00582958542339567\\
64	0.00582958542452272\\
65	0.00582958542566989\\
66	0.00582958542683754\\
67	0.00582958542802603\\
68	0.00582958542923573\\
69	0.00582958543046703\\
70	0.00582958543172031\\
71	0.00582958543299595\\
72	0.00582958543429437\\
73	0.00582958543561595\\
74	0.00582958543696113\\
75	0.00582958543833032\\
76	0.00582958543972394\\
77	0.00582958544114243\\
78	0.00582958544258623\\
79	0.00582958544405581\\
80	0.00582958544555161\\
81	0.0058295854470741\\
82	0.00582958544862377\\
83	0.00582958545020108\\
84	0.00582958545180655\\
85	0.00582958545344066\\
86	0.00582958545510393\\
87	0.00582958545679688\\
88	0.00582958545852004\\
89	0.00582958546027394\\
90	0.00582958546205914\\
91	0.00582958546387619\\
92	0.00582958546572567\\
93	0.00582958546760814\\
94	0.0058295854695242\\
95	0.00582958547147444\\
96	0.00582958547345948\\
97	0.00582958547547994\\
98	0.00582958547753644\\
99	0.00582958547962963\\
100	0.00582958548176017\\
101	0.00582958548392871\\
102	0.00582958548613595\\
103	0.00582958548838256\\
104	0.00582958549066925\\
105	0.00582958549299674\\
106	0.00582958549536574\\
107	0.00582958549777701\\
108	0.0058295855002313\\
109	0.00582958550272937\\
110	0.005829585505272\\
111	0.00582958550785999\\
112	0.00582958551049416\\
113	0.00582958551317531\\
114	0.00582958551590429\\
115	0.00582958551868195\\
116	0.00582958552150917\\
117	0.00582958552438682\\
118	0.00582958552731581\\
119	0.00582958553029705\\
120	0.00582958553333147\\
121	0.00582958553642002\\
122	0.00582958553956367\\
123	0.00582958554276341\\
124	0.00582958554602022\\
125	0.00582958554933514\\
126	0.0058295855527092\\
127	0.00582958555614345\\
128	0.00582958555963896\\
129	0.00582958556319684\\
130	0.0058295855668182\\
131	0.00582958557050416\\
132	0.00582958557425589\\
133	0.00582958557807455\\
134	0.00582958558196135\\
135	0.00582958558591749\\
136	0.00582958558994423\\
137	0.00582958559404281\\
138	0.00582958559821452\\
139	0.00582958560246067\\
140	0.0058295856067826\\
141	0.00582958561118164\\
142	0.00582958561565919\\
143	0.00582958562021664\\
144	0.00582958562485542\\
145	0.00582958562957699\\
146	0.00582958563438283\\
147	0.00582958563927444\\
148	0.00582958564425336\\
149	0.00582958564932115\\
150	0.0058295856544794\\
151	0.00582958565972972\\
152	0.00582958566507377\\
153	0.00582958567051322\\
154	0.00582958567604978\\
155	0.00582958568168519\\
156	0.00582958568742122\\
157	0.00582958569325967\\
158	0.00582958569920236\\
159	0.00582958570525118\\
160	0.00582958571140802\\
161	0.00582958571767482\\
162	0.00582958572405355\\
163	0.00582958573054621\\
164	0.00582958573715484\\
165	0.00582958574388153\\
166	0.00582958575072839\\
167	0.00582958575769757\\
168	0.00582958576479128\\
169	0.00582958577201175\\
170	0.00582958577936124\\
171	0.00582958578684208\\
172	0.00582958579445663\\
173	0.00582958580220727\\
174	0.00582958581009648\\
175	0.00582958581812671\\
176	0.00582958582630053\\
177	0.00582958583462049\\
178	0.00582958584308924\\
179	0.00582958585170944\\
180	0.00582958586048383\\
181	0.00582958586941516\\
182	0.00582958587850628\\
183	0.00582958588776005\\
184	0.00582958589717941\\
185	0.00582958590676732\\
186	0.00582958591652684\\
187	0.00582958592646106\\
188	0.00582958593657311\\
189	0.00582958594686622\\
190	0.00582958595734363\\
191	0.00582958596800869\\
192	0.00582958597886477\\
193	0.00582958598991533\\
194	0.00582958600116386\\
195	0.00582958601261396\\
196	0.00582958602426926\\
197	0.00582958603613347\\
198	0.00582958604821037\\
199	0.00582958606050381\\
200	0.0058295860730177\\
201	0.00582958608575603\\
202	0.00582958609872287\\
203	0.00582958611192235\\
204	0.0058295861253587\\
205	0.0058295861390362\\
206	0.00582958615295923\\
207	0.00582958616713224\\
208	0.00582958618155977\\
209	0.00582958619624644\\
210	0.00582958621119695\\
211	0.0058295862264161\\
212	0.00582958624190878\\
213	0.00582958625767995\\
214	0.00582958627373469\\
215	0.00582958629007815\\
216	0.0058295863067156\\
217	0.00582958632365238\\
218	0.00582958634089395\\
219	0.00582958635844588\\
220	0.00582958637631382\\
221	0.00582958639450355\\
222	0.00582958641302093\\
223	0.00582958643187195\\
224	0.00582958645106271\\
225	0.00582958647059943\\
226	0.00582958649048843\\
227	0.00582958651073616\\
228	0.00582958653134919\\
229	0.00582958655233423\\
230	0.00582958657369808\\
231	0.00582958659544769\\
232	0.00582958661759016\\
233	0.00582958664013268\\
234	0.00582958666308262\\
235	0.00582958668644746\\
236	0.00582958671023483\\
237	0.00582958673445251\\
238	0.00582958675910843\\
239	0.00582958678421065\\
240	0.00582958680976741\\
241	0.0058295868357871\\
242	0.00582958686227825\\
243	0.00582958688924957\\
244	0.00582958691670994\\
245	0.0058295869446684\\
246	0.00582958697313417\\
247	0.00582958700211662\\
248	0.00582958703162535\\
249	0.00582958706167009\\
250	0.0058295870922608\\
251	0.00582958712340759\\
252	0.00582958715512081\\
253	0.00582958718741096\\
254	0.00582958722028878\\
255	0.00582958725376519\\
256	0.00582958728785134\\
257	0.00582958732255859\\
258	0.0058295873578985\\
259	0.00582958739388288\\
260	0.00582958743052374\\
261	0.00582958746783335\\
262	0.0058295875058242\\
263	0.00582958754450901\\
264	0.00582958758390076\\
265	0.00582958762401269\\
266	0.00582958766485827\\
267	0.00582958770645125\\
268	0.00582958774880565\\
269	0.00582958779193572\\
270	0.00582958783585604\\
271	0.00582958788058144\\
272	0.00582958792612704\\
273	0.00582958797250824\\
274	0.00582958801974077\\
275	0.00582958806784064\\
276	0.00582958811682418\\
277	0.00582958816670801\\
278	0.0058295882175091\\
279	0.00582958826924473\\
280	0.00582958832193252\\
281	0.00582958837559043\\
282	0.00582958843023677\\
283	0.00582958848589019\\
284	0.00582958854256969\\
285	0.00582958860029467\\
286	0.00582958865908486\\
287	0.0058295887189604\\
288	0.0058295887799418\\
289	0.00582958884204996\\
290	0.00582958890530619\\
291	0.00582958896973219\\
292	0.0058295890353501\\
293	0.00582958910218246\\
294	0.00582958917025223\\
295	0.00582958923958282\\
296	0.00582958931019809\\
297	0.00582958938212232\\
298	0.00582958945538028\\
299	0.00582958952999718\\
300	0.00582958960599873\\
301	0.00582958968341109\\
302	0.00582958976226092\\
303	0.00582958984257539\\
304	0.00582958992438216\\
305	0.0058295900077094\\
306	0.00582959009258581\\
307	0.00582959017904061\\
308	0.00582959026710356\\
309	0.00582959035680496\\
310	0.00582959044817567\\
311	0.0058295905412471\\
312	0.00582959063605123\\
313	0.00582959073262064\\
314	0.00582959083098847\\
315	0.00582959093118847\\
316	0.00582959103325497\\
317	0.00582959113722295\\
318	0.00582959124312798\\
319	0.00582959135100626\\
320	0.00582959146089465\\
321	0.00582959157283064\\
322	0.00582959168685238\\
323	0.0058295918029987\\
324	0.00582959192130909\\
325	0.00582959204182373\\
326	0.0058295921645835\\
327	0.00582959228962998\\
328	0.00582959241700549\\
329	0.00582959254675305\\
330	0.00582959267891644\\
331	0.00582959281354018\\
332	0.00582959295066956\\
333	0.00582959309035065\\
334	0.00582959323263029\\
335	0.00582959337755615\\
336	0.00582959352517671\\
337	0.00582959367554126\\
338	0.00582959382869997\\
339	0.00582959398470383\\
340	0.00582959414360476\\
341	0.00582959430545553\\
342	0.00582959447030984\\
343	0.00582959463822232\\
344	0.00582959480924854\\
345	0.00582959498344505\\
346	0.0058295951608694\\
347	0.00582959534158011\\
348	0.00582959552563678\\
349	0.00582959571310003\\
350	0.00582959590403161\\
351	0.00582959609849436\\
352	0.00582959629655224\\
353	0.00582959649827044\\
354	0.00582959670371531\\
355	0.0058295969129545\\
356	0.00582959712605689\\
357	0.00582959734309276\\
358	0.00582959756413371\\
359	0.0058295977892528\\
360	0.00582959801852459\\
361	0.00582959825202515\\
362	0.00582959848983218\\
363	0.00582959873202506\\
364	0.00582959897868491\\
365	0.0058295992298947\\
366	0.00582959948573932\\
367	0.00582959974630568\\
368	0.0058296000116828\\
369	0.00582960028196196\\
370	0.00582960055723677\\
371	0.00582960083760336\\
372	0.00582960112316047\\
373	0.00582960141400965\\
374	0.00582960171025541\\
375	0.00582960201200542\\
376	0.00582960231937068\\
377	0.0058296026324658\\
378	0.00582960295140919\\
379	0.00582960327632338\\
380	0.00582960360733529\\
381	0.00582960394457657\\
382	0.00582960428818403\\
383	0.00582960463830002\\
384	0.00582960499507305\\
385	0.00582960535865839\\
386	0.00582960572921891\\
387	0.005829606106926\\
388	0.00582960649196056\\
389	0.00582960688451369\\
390	0.00582960728478655\\
391	0.00582960769298912\\
392	0.00582960810933893\\
393	0.00582960853406322\\
394	0.00582960896740667\\
395	0.00582960940963215\\
396	0.0058296098610169\\
397	0.00582961032185298\\
398	0.00582961079244737\\
399	0.0058296112731215\\
400	0.00582961176421056\\
401	0.0058296122660629\\
402	0.00582961277904154\\
403	0.00582961330353036\\
404	0.00582961383994705\\
405	0.00582961438875788\\
406	0.00582961495047681\\
407	0.0058296155256341\\
408	0.00582961611476347\\
409	0.00582961671844148\\
410	0.00582961733729288\\
411	0.00582961797199679\\
412	0.00582961862329385\\
413	0.00582961929199461\\
414	0.0058296199789892\\
415	0.00582962068525896\\
416	0.00582962141189038\\
417	0.00582962216009264\\
418	0.00582962293122101\\
419	0.00582962372681105\\
420	0.0058296245486348\\
421	0.00582962539880278\\
422	0.00582962627996381\\
423	0.00582962719570759\\
424	0.00582962815136812\\
425	0.00582962915555119\\
426	0.00582963022276766\\
427	0.0058296313771673\\
428	0.00582963265555936\\
429	0.00582963410313205\\
430	0.00582963574838076\\
431	0.00582963755119696\\
432	0.00582963939479503\\
433	0.00582964128046762\\
434	0.00582964320957619\\
435	0.00582964518355714\\
436	0.00582964720393022\\
437	0.0058296492723123\\
438	0.0058296513904448\\
439	0.00582965356025649\\
440	0.00582965578401772\\
441	0.00582965806473182\\
442	0.00582966040713372\\
443	0.00582966282022291\\
444	0.0058296653235937\\
445	0.005829667962891\\
446	0.00582967084626833\\
447	0.00582967422602506\\
448	0.00582967866672493\\
449	0.00582968534224265\\
450	0.00582969639559791\\
451	0.00582971480129136\\
452	0.00582974191409959\\
453	0.005829770986764\\
454	0.00582980060131154\\
455	0.00582983076316588\\
456	0.00582986147834398\\
457	0.00582989275374301\\
458	0.00582992459738322\\
459	0.00582995701844631\\
460	0.00582999002693113\\
461	0.0058300236331148\\
462	0.00583005784812319\\
463	0.00583009268716015\\
464	0.00583012816791207\\
465	0.00583016430857424\\
466	0.0058302011275361\\
467	0.00583023864378673\\
468	0.00583027687760269\\
469	0.00583031584971009\\
470	0.00583035558186035\\
471	0.00583039609833999\\
472	0.00583043742780714\\
473	0.0058304796021032\\
474	0.00583052264882924\\
475	0.00583056659735822\\
476	0.00583061147899847\\
477	0.00583065732716743\\
478	0.00583070417758578\\
479	0.00583075206849526\\
480	0.00583080104090415\\
481	0.00583085113886539\\
482	0.00583090240979286\\
483	0.00583095490482302\\
484	0.00583100867923024\\
485	0.0058310637929063\\
486	0.00583112031091637\\
487	0.00583117830414725\\
488	0.00583123785006905\\
489	0.00583129903363768\\
490	0.0058313619483873\\
491	0.00583142669779906\\
492	0.00583149339713185\\
493	0.00583156217611772\\
494	0.00583163318344877\\
495	0.00583170659522542\\
496	0.00583178263227508\\
497	0.00583186159713117\\
498	0.00583194395304776\\
499	0.00583203048701399\\
500	0.00583212262081674\\
501	0.00583222291731705\\
502	0.0058323356133636\\
503	0.00583246613881396\\
504	0.00583261617545312\\
505	0.00583277179295458\\
506	0.00583292954933207\\
507	0.00583308975210002\\
508	0.00583325277222019\\
509	0.00583341873451183\\
510	0.00583358777851466\\
511	0.0058337600649959\\
512	0.00583393579219153\\
513	0.00583411523759399\\
514	0.00583429887008517\\
515	0.00583448765542392\\
516	0.00583468388718613\\
517	0.00583489341990013\\
518	0.00583513152082242\\
519	0.00583543748972403\\
520	0.00583590785938274\\
521	0.00583675591049733\\
522	0.00583833584480593\\
523	0.00584061277155873\\
524	0.00584294839343069\\
525	0.00584534686938407\\
526	0.00584781342617747\\
527	0.00585035481848645\\
528	0.00585297937938078\\
529	0.00585569497024925\\
530	0.00585850195453839\\
531	0.00586139016917696\\
532	0.00586437042898095\\
533	0.00586748028582171\\
534	0.00587079707657696\\
535	0.00587444466791646\\
536	0.00587866180477956\\
537	0.00588377530467723\\
538	0.00588937856467497\\
539	0.00589507424030023\\
540	0.00590086861589458\\
541	0.00590676874521496\\
542	0.00591278268876506\\
543	0.0059189200740091\\
544	0.00592519362455466\\
545	0.00593162277075402\\
546	0.00593824114150121\\
547	0.00594511143620707\\
548	0.0059523522096728\\
549	0.00596015137784947\\
550	0.00596856464983976\\
551	0.00597718259522469\\
552	0.0059859840032945\\
553	0.0059953471520073\\
554	0.00600863216314313\\
555	0.00603766671196547\\
556	0.00612112468559537\\
557	0.00621400588477447\\
558	0.00631333641990335\\
559	0.00642328807872805\\
560	0.00655649508322337\\
561	0.00673744096005576\\
562	0.00692495206043795\\
563	0.00711975569152903\\
564	0.00732361062886509\\
565	0.00754074109644888\\
566	0.00778401979610116\\
567	0.00804733790929998\\
568	0.00831811668216551\\
569	0.00859730677617569\\
570	0.00888668648769891\\
571	0.00919074662973911\\
572	0.00950307203274264\\
573	0.0097867009811926\\
574	0.0100321540985466\\
575	0.0102859780417936\\
576	0.0105482685783307\\
577	0.0108211031448169\\
578	0.0111064746220399\\
579	0.0113025369869154\\
580	0.0114465247212759\\
581	0.0115638816196064\\
582	0.0116972840244324\\
583	0.0118316131749101\\
584	0.0119494792138259\\
585	0.0120691663285315\\
586	0.0121908257470425\\
587	0.0123147668054481\\
588	0.012443283960156\\
589	0.0125623922825788\\
590	0.0126929540601676\\
591	0.0128425876192921\\
592	0.0129891468420991\\
593	0.0131272830991681\\
594	0.0132426187737169\\
595	0.0133471971667729\\
596	0.0134467378825905\\
597	0.0135473485564097\\
598	0.0136637438596031\\
599	0\\
600	0\\
};
\addplot [color=mycolor15,solid,forget plot]
  table[row sep=crcr]{%
1	0.00599032346902476\\
2	0.00599032347639302\\
3	0.00599032348389301\\
4	0.00599032349152711\\
5	0.00599032349929768\\
6	0.00599032350720719\\
7	0.0059903235152581\\
8	0.00599032352345294\\
9	0.00599032353179428\\
10	0.00599032354028474\\
11	0.00599032354892697\\
12	0.00599032355772368\\
13	0.00599032356667764\\
14	0.00599032357579165\\
15	0.00599032358506856\\
16	0.00599032359451129\\
17	0.00599032360412279\\
18	0.00599032361390607\\
19	0.00599032362386421\\
20	0.00599032363400032\\
21	0.00599032364431757\\
22	0.0059903236548192\\
23	0.00599032366550851\\
24	0.00599032367638883\\
25	0.00599032368746358\\
26	0.00599032369873622\\
27	0.00599032371021029\\
28	0.00599032372188939\\
29	0.00599032373377716\\
30	0.00599032374587734\\
31	0.0059903237581937\\
32	0.00599032377073012\\
33	0.00599032378349051\\
34	0.00599032379647887\\
35	0.00599032380969927\\
36	0.00599032382315584\\
37	0.00599032383685281\\
38	0.00599032385079445\\
39	0.00599032386498513\\
40	0.0059903238794293\\
41	0.00599032389413147\\
42	0.00599032390909625\\
43	0.00599032392432833\\
44	0.00599032393983246\\
45	0.0059903239556135\\
46	0.0059903239716764\\
47	0.00599032398802616\\
48	0.00599032400466793\\
49	0.00599032402160689\\
50	0.00599032403884835\\
51	0.00599032405639771\\
52	0.00599032407426046\\
53	0.00599032409244217\\
54	0.00599032411094856\\
55	0.00599032412978539\\
56	0.00599032414895856\\
57	0.00599032416847408\\
58	0.00599032418833805\\
59	0.00599032420855667\\
60	0.00599032422913628\\
61	0.00599032425008331\\
62	0.00599032427140431\\
63	0.00599032429310594\\
64	0.005990324315195\\
65	0.0059903243376784\\
66	0.00599032436056316\\
67	0.00599032438385643\\
68	0.00599032440756551\\
69	0.00599032443169781\\
70	0.00599032445626087\\
71	0.00599032448126237\\
72	0.00599032450671012\\
73	0.00599032453261209\\
74	0.00599032455897638\\
75	0.00599032458581121\\
76	0.00599032461312499\\
77	0.00599032464092625\\
78	0.00599032466922368\\
79	0.00599032469802612\\
80	0.00599032472734259\\
81	0.00599032475718224\\
82	0.00599032478755439\\
83	0.00599032481846856\\
84	0.00599032484993439\\
85	0.00599032488196172\\
86	0.00599032491456056\\
87	0.0059903249477411\\
88	0.00599032498151371\\
89	0.00599032501588895\\
90	0.00599032505087755\\
91	0.00599032508649046\\
92	0.0059903251227388\\
93	0.00599032515963391\\
94	0.00599032519718731\\
95	0.00599032523541073\\
96	0.00599032527431614\\
97	0.00599032531391568\\
98	0.00599032535422172\\
99	0.00599032539524688\\
100	0.00599032543700397\\
101	0.00599032547950603\\
102	0.00599032552276636\\
103	0.00599032556679847\\
104	0.00599032561161612\\
105	0.00599032565723333\\
106	0.00599032570366435\\
107	0.00599032575092369\\
108	0.00599032579902613\\
109	0.00599032584798669\\
110	0.00599032589782069\\
111	0.0059903259485437\\
112	0.00599032600017158\\
113	0.00599032605272047\\
114	0.00599032610620679\\
115	0.00599032616064727\\
116	0.00599032621605892\\
117	0.00599032627245907\\
118	0.00599032632986536\\
119	0.00599032638829573\\
120	0.00599032644776846\\
121	0.00599032650830214\\
122	0.0059903265699157\\
123	0.00599032663262842\\
124	0.00599032669645991\\
125	0.00599032676143013\\
126	0.00599032682755941\\
127	0.00599032689486842\\
128	0.00599032696337824\\
129	0.0059903270331103\\
130	0.00599032710408641\\
131	0.00599032717632878\\
132	0.00599032724986002\\
133	0.00599032732470315\\
134	0.00599032740088159\\
135	0.00599032747841918\\
136	0.0059903275573402\\
137	0.00599032763766937\\
138	0.00599032771943183\\
139	0.00599032780265318\\
140	0.0059903278873595\\
141	0.00599032797357731\\
142	0.00599032806133362\\
143	0.00599032815065594\\
144	0.00599032824157224\\
145	0.00599032833411101\\
146	0.00599032842830128\\
147	0.00599032852417254\\
148	0.00599032862175488\\
149	0.00599032872107888\\
150	0.00599032882217569\\
151	0.00599032892507703\\
152	0.00599032902981517\\
153	0.00599032913642299\\
154	0.00599032924493392\\
155	0.00599032935538204\\
156	0.00599032946780201\\
157	0.00599032958222914\\
158	0.00599032969869937\\
159	0.00599032981724927\\
160	0.00599032993791608\\
161	0.00599033006073774\\
162	0.00599033018575285\\
163	0.0059903303130007\\
164	0.00599033044252131\\
165	0.00599033057435542\\
166	0.00599033070854449\\
167	0.00599033084513077\\
168	0.00599033098415723\\
169	0.00599033112566765\\
170	0.00599033126970659\\
171	0.00599033141631942\\
172	0.00599033156555234\\
173	0.00599033171745239\\
174	0.00599033187206744\\
175	0.00599033202944626\\
176	0.0059903321896385\\
177	0.0059903323526947\\
178	0.00599033251866633\\
179	0.00599033268760579\\
180	0.00599033285956645\\
181	0.00599033303460263\\
182	0.00599033321276967\\
183	0.00599033339412388\\
184	0.00599033357872264\\
185	0.00599033376662434\\
186	0.00599033395788847\\
187	0.00599033415257559\\
188	0.00599033435074736\\
189	0.00599033455246659\\
190	0.00599033475779723\\
191	0.00599033496680439\\
192	0.00599033517955439\\
193	0.00599033539611476\\
194	0.00599033561655427\\
195	0.00599033584094296\\
196	0.00599033606935214\\
197	0.00599033630185446\\
198	0.00599033653852387\\
199	0.00599033677943572\\
200	0.00599033702466671\\
201	0.00599033727429499\\
202	0.00599033752840012\\
203	0.00599033778706316\\
204	0.00599033805036664\\
205	0.00599033831839463\\
206	0.00599033859123276\\
207	0.00599033886896823\\
208	0.00599033915168986\\
209	0.00599033943948813\\
210	0.00599033973245518\\
211	0.00599034003068488\\
212	0.00599034033427282\\
213	0.00599034064331638\\
214	0.00599034095791476\\
215	0.005990341278169\\
216	0.00599034160418201\\
217	0.00599034193605864\\
218	0.00599034227390568\\
219	0.00599034261783194\\
220	0.00599034296794822\\
221	0.00599034332436744\\
222	0.0059903436872046\\
223	0.00599034405657687\\
224	0.00599034443260361\\
225	0.00599034481540643\\
226	0.0059903452051092\\
227	0.00599034560183816\\
228	0.00599034600572187\\
229	0.00599034641689136\\
230	0.0059903468354801\\
231	0.00599034726162408\\
232	0.00599034769546186\\
233	0.00599034813713463\\
234	0.00599034858678621\\
235	0.00599034904456317\\
236	0.00599034951061486\\
237	0.00599034998509344\\
238	0.00599035046815397\\
239	0.00599035095995443\\
240	0.00599035146065582\\
241	0.00599035197042219\\
242	0.00599035248942072\\
243	0.00599035301782176\\
244	0.0059903535557989\\
245	0.00599035410352907\\
246	0.00599035466119253\\
247	0.00599035522897301\\
248	0.00599035580705774\\
249	0.00599035639563753\\
250	0.00599035699490685\\
251	0.00599035760506387\\
252	0.00599035822631057\\
253	0.0059903588588528\\
254	0.00599035950290036\\
255	0.00599036015866707\\
256	0.00599036082637085\\
257	0.00599036150623383\\
258	0.00599036219848237\\
259	0.00599036290334722\\
260	0.00599036362106355\\
261	0.00599036435187106\\
262	0.00599036509601407\\
263	0.00599036585374161\\
264	0.00599036662530749\\
265	0.00599036741097044\\
266	0.00599036821099415\\
267	0.00599036902564742\\
268	0.00599036985520423\\
269	0.00599037069994383\\
270	0.00599037156015089\\
271	0.00599037243611555\\
272	0.00599037332813355\\
273	0.00599037423650636\\
274	0.00599037516154125\\
275	0.00599037610355142\\
276	0.00599037706285613\\
277	0.00599037803978077\\
278	0.00599037903465704\\
279	0.005990380047823\\
280	0.00599038107962327\\
281	0.00599038213040906\\
282	0.00599038320053837\\
283	0.00599038429037609\\
284	0.00599038540029411\\
285	0.00599038653067149\\
286	0.00599038768189454\\
287	0.00599038885435699\\
288	0.00599039004846013\\
289	0.00599039126461291\\
290	0.00599039250323212\\
291	0.00599039376474249\\
292	0.00599039504957685\\
293	0.0059903963581763\\
294	0.00599039769099029\\
295	0.00599039904847684\\
296	0.00599040043110261\\
297	0.00599040183934313\\
298	0.00599040327368287\\
299	0.00599040473461546\\
300	0.0059904062226438\\
301	0.00599040773828022\\
302	0.00599040928204666\\
303	0.00599041085447477\\
304	0.00599041245610616\\
305	0.00599041408749244\\
306	0.00599041574919549\\
307	0.00599041744178756\\
308	0.00599041916585143\\
309	0.00599042092198059\\
310	0.00599042271077943\\
311	0.00599042453286332\\
312	0.00599042638885889\\
313	0.0059904282794041\\
314	0.00599043020514846\\
315	0.00599043216675317\\
316	0.00599043416489132\\
317	0.00599043620024807\\
318	0.00599043827352076\\
319	0.00599044038541915\\
320	0.0059904425366656\\
321	0.00599044472799519\\
322	0.00599044696015596\\
323	0.0059904492339091\\
324	0.00599045155002908\\
325	0.00599045390930391\\
326	0.00599045631253529\\
327	0.00599045876053885\\
328	0.00599046125414432\\
329	0.00599046379419578\\
330	0.00599046638155185\\
331	0.00599046901708592\\
332	0.0059904717016864\\
333	0.00599047443625695\\
334	0.0059904772217167\\
335	0.00599048005900056\\
336	0.00599048294905946\\
337	0.00599048589286061\\
338	0.00599048889138779\\
339	0.00599049194564168\\
340	0.00599049505664012\\
341	0.00599049822541846\\
342	0.00599050145302987\\
343	0.00599050474054569\\
344	0.00599050808905576\\
345	0.00599051149966887\\
346	0.00599051497351304\\
347	0.00599051851173602\\
348	0.00599052211550567\\
349	0.00599052578601042\\
350	0.00599052952445978\\
351	0.00599053333208481\\
352	0.00599053721013874\\
353	0.00599054115989747\\
354	0.00599054518266026\\
355	0.0059905492797504\\
356	0.00599055345251592\\
357	0.00599055770233039\\
358	0.00599056203059375\\
359	0.00599056643873321\\
360	0.00599057092820429\\
361	0.00599057550049181\\
362	0.00599058015711107\\
363	0.00599058489960911\\
364	0.00599058972956601\\
365	0.00599059464859639\\
366	0.00599059965835097\\
367	0.00599060476051826\\
368	0.00599060995682645\\
369	0.00599061524904537\\
370	0.00599062063898871\\
371	0.00599062612851638\\
372	0.00599063171953705\\
373	0.00599063741401096\\
374	0.00599064321395296\\
375	0.00599064912143581\\
376	0.00599065513859376\\
377	0.00599066126762651\\
378	0.0059906675108035\\
379	0.00599067387046862\\
380	0.00599068034904542\\
381	0.00599068694904281\\
382	0.00599069367306152\\
383	0.00599070052380127\\
384	0.00599070750406913\\
385	0.00599071461678928\\
386	0.00599072186501503\\
387	0.00599072925194376\\
388	0.00599073678093576\\
389	0.00599074445553644\\
390	0.00599075227949734\\
391	0.0059907602567838\\
392	0.00599076839154974\\
393	0.00599077668807921\\
394	0.00599078515078226\\
395	0.00599079378441388\\
396	0.00599080259412864\\
397	0.00599081158536669\\
398	0.00599082076386549\\
399	0.00599083013566539\\
400	0.00599083970710465\\
401	0.00599084948480091\\
402	0.00599085947562256\\
403	0.00599086968667511\\
404	0.00599088012536707\\
405	0.00599089079965561\\
406	0.00599090171848987\\
407	0.0059909128920625\\
408	0.00599092433100616\\
409	0.00599093604566406\\
410	0.00599094804719197\\
411	0.00599096034765849\\
412	0.00599097296016049\\
413	0.00599098589895647\\
414	0.00599099917962111\\
415	0.00599101281922494\\
416	0.00599102683654438\\
417	0.00599104125230952\\
418	0.00599105608950113\\
419	0.00599107137371775\\
420	0.00599108713365582\\
421	0.00599110340179911\\
422	0.00599112021554305\\
423	0.00599113761928811\\
424	0.00599115566874995\\
425	0.00599117444029749\\
426	0.0059911940512634\\
427	0.00599121470242646\\
428	0.00599123675903439\\
429	0.00599126087725931\\
430	0.00599128810623883\\
431	0.00599131962843508\\
432	0.00599135533124139\\
433	0.00599139183795031\\
434	0.00599142917377634\\
435	0.00599146736525415\\
436	0.00599150644033957\\
437	0.00599154642852507\\
438	0.00599158736098331\\
439	0.00599162927077509\\
440	0.00599167219322376\\
441	0.00599171616674131\\
442	0.0059917612349068\\
443	0.005991807452041\\
444	0.00599185489854579\\
445	0.00599190372341149\\
446	0.00599195426168655\\
447	0.00599200735582206\\
448	0.00599206521814722\\
449	0.00599213367019972\\
450	0.00599222761755948\\
451	0.00599238287805609\\
452	0.00599267419358242\\
453	0.00599320142795372\\
454	0.00599378085836966\\
455	0.00599437115849134\\
456	0.00599497243327278\\
457	0.00599558479693493\\
458	0.00599620837908925\\
459	0.00599684333147334\\
460	0.00599748983234564\\
461	0.00599814808180438\\
462	0.00599881828174953\\
463	0.00599950062359239\\
464	0.00600019539386675\\
465	0.00600090294588935\\
466	0.0060016236460515\\
467	0.00600235786094702\\
468	0.00600310596310707\\
469	0.00600386835861614\\
470	0.00600464545737945\\
471	0.0060054376764517\\
472	0.00600624546795066\\
473	0.00600706938373473\\
474	0.00600791010689268\\
475	0.00600876818034188\\
476	0.00600964418139982\\
477	0.00601053872500006\\
478	0.00601145246702651\\
479	0.00601238610802862\\
480	0.00601334039738017\\
481	0.00601431613795586\\
482	0.00601531419141411\\
483	0.00601633548419161\\
484	0.00601738101433689\\
485	0.00601845185933789\\
486	0.00601954918513108\\
487	0.00602067425651219\\
488	0.00602182844921712\\
489	0.00602301326403406\\
490	0.00602423034330445\\
491	0.00602548149043597\\
492	0.00602676869321937\\
493	0.00602809415256294\\
494	0.00602946031968202\\
495	0.00603086994855116\\
496	0.00603232618116426\\
497	0.00603383270871072\\
498	0.00603539411666303\\
499	0.00603701668047269\\
500	0.00603871024996858\\
501	0.00604049266486954\\
502	0.00604239962764516\\
503	0.00604450459405451\\
504	0.00604694951891126\\
505	0.0060499082676736\\
506	0.00605300983125695\\
507	0.00605615210312746\\
508	0.0060593407813001\\
509	0.00606258565355876\\
510	0.00606588912726207\\
511	0.00606925384372765\\
512	0.00607268271492154\\
513	0.00607617900036558\\
514	0.00607974646740256\\
515	0.00608338980084082\\
516	0.00608711579538251\\
517	0.00609093702913546\\
518	0.00609488359173495\\
519	0.00609904156391954\\
520	0.00610368229955621\\
521	0.00610970625651236\\
522	0.00612019689083974\\
523	0.00614595910776007\\
524	0.0061918955482278\\
525	0.00623920729408083\\
526	0.00628798809020923\\
527	0.0063383513221512\\
528	0.0063904436183794\\
529	0.00644446606450121\\
530	0.00650067913415457\\
531	0.00655921987394357\\
532	0.00661971651637051\\
533	0.0066821693836296\\
534	0.00674702570064966\\
535	0.00681553185147688\\
536	0.00688929070934847\\
537	0.00697272484397712\\
538	0.00708050838487127\\
539	0.00720430346932848\\
540	0.00733164834814307\\
541	0.00746279371554648\\
542	0.00759801929082208\\
543	0.00773763840445583\\
544	0.00788200385676951\\
545	0.00803152280858739\\
546	0.00818669749733148\\
547	0.00834822243528515\\
548	0.0085172314789705\\
549	0.00869615410810298\\
550	0.00889173048717438\\
551	0.00911191901563923\\
552	0.0093426467605882\\
553	0.00957820888500259\\
554	0.00978978294048548\\
555	0.00996250278044234\\
556	0.0100939384828728\\
557	0.0102245759089651\\
558	0.0103588032242156\\
559	0.0104969545349647\\
560	0.0106343065755494\\
561	0.010736880126034\\
562	0.0108496866690711\\
563	0.0109730606430093\\
564	0.0110985099408717\\
565	0.0112260664279247\\
566	0.0113424818351038\\
567	0.0114496888862977\\
568	0.0115590069721942\\
569	0.0116701319268866\\
570	0.0117835969687971\\
571	0.011899948581223\\
572	0.0120224663297457\\
573	0.012132495226828\\
574	0.0122315824166699\\
575	0.0123339138430404\\
576	0.01244517147675\\
577	0.0125563893121482\\
578	0.0126670103729859\\
579	0.0127598557290707\\
580	0.0128432753767443\\
581	0.0129212754827187\\
582	0.0129882112395968\\
583	0.0130515036111677\\
584	0.0131112486684305\\
585	0.0131701425753821\\
586	0.0132269000042697\\
587	0.013281671558656\\
588	0.0133309066443553\\
589	0.0133736658661722\\
590	0.0134129455168408\\
591	0.0134481178433173\\
592	0.0134797612143452\\
593	0.0135085207755725\\
594	0.0135350984511593\\
595	0.0135592314909786\\
596	0.0135833251697673\\
597	0.0136131889078992\\
598	0.0136637438596031\\
599	0\\
600	0\\
};
\addplot [color=mycolor16,solid,forget plot]
  table[row sep=crcr]{%
1	0.00604107245767237\\
2	0.00604107260381874\\
3	0.00604107275257828\\
4	0.00604107290399764\\
5	0.0060410730581243\\
6	0.00604107321500661\\
7	0.00604107337469376\\
8	0.00604107353723582\\
9	0.00604107370268375\\
10	0.00604107387108944\\
11	0.00604107404250568\\
12	0.00604107421698621\\
13	0.00604107439458573\\
14	0.0060410745753599\\
15	0.00604107475936541\\
16	0.00604107494665991\\
17	0.00604107513730212\\
18	0.00604107533135178\\
19	0.0060410755288697\\
20	0.00604107572991779\\
21	0.00604107593455903\\
22	0.00604107614285757\\
23	0.00604107635487865\\
24	0.00604107657068871\\
25	0.00604107679035536\\
26	0.00604107701394743\\
27	0.00604107724153494\\
28	0.0060410774731892\\
29	0.00604107770898275\\
30	0.00604107794898947\\
31	0.00604107819328452\\
32	0.0060410784419444\\
33	0.00604107869504699\\
34	0.00604107895267155\\
35	0.00604107921489876\\
36	0.00604107948181072\\
37	0.00604107975349099\\
38	0.00604108003002466\\
39	0.00604108031149828\\
40	0.00604108059799999\\
41	0.00604108088961946\\
42	0.006041081186448\\
43	0.00604108148857849\\
44	0.00604108179610553\\
45	0.00604108210912535\\
46	0.00604108242773594\\
47	0.00604108275203699\\
48	0.00604108308213\\
49	0.00604108341811827\\
50	0.00604108376010693\\
51	0.00604108410820299\\
52	0.00604108446251537\\
53	0.00604108482315494\\
54	0.00604108519023451\\
55	0.00604108556386894\\
56	0.0060410859441751\\
57	0.00604108633127197\\
58	0.00604108672528064\\
59	0.00604108712632436\\
60	0.00604108753452857\\
61	0.00604108795002095\\
62	0.00604108837293144\\
63	0.00604108880339232\\
64	0.00604108924153821\\
65	0.00604108968750613\\
66	0.00604109014143555\\
67	0.00604109060346842\\
68	0.00604109107374921\\
69	0.00604109155242498\\
70	0.00604109203964539\\
71	0.0060410925355628\\
72	0.00604109304033224\\
73	0.00604109355411155\\
74	0.00604109407706134\\
75	0.00604109460934511\\
76	0.00604109515112927\\
77	0.00604109570258319\\
78	0.00604109626387925\\
79	0.00604109683519292\\
80	0.0060410974167028\\
81	0.00604109800859065\\
82	0.00604109861104148\\
83	0.00604109922424361\\
84	0.00604109984838871\\
85	0.00604110048367186\\
86	0.00604110113029162\\
87	0.0060411017884501\\
88	0.00604110245835301\\
89	0.00604110314020972\\
90	0.00604110383423335\\
91	0.00604110454064079\\
92	0.00604110525965283\\
93	0.00604110599149419\\
94	0.00604110673639358\\
95	0.00604110749458382\\
96	0.00604110826630186\\
97	0.00604110905178889\\
98	0.00604110985129039\\
99	0.00604111066505623\\
100	0.00604111149334074\\
101	0.00604111233640278\\
102	0.00604111319450585\\
103	0.00604111406791813\\
104	0.0060411149569126\\
105	0.0060411158617671\\
106	0.00604111678276446\\
107	0.00604111772019253\\
108	0.00604111867434431\\
109	0.00604111964551804\\
110	0.00604112063401728\\
111	0.00604112164015101\\
112	0.00604112266423375\\
113	0.00604112370658561\\
114	0.00604112476753244\\
115	0.00604112584740592\\
116	0.00604112694654363\\
117	0.00604112806528921\\
118	0.00604112920399245\\
119	0.00604113036300936\\
120	0.00604113154270236\\
121	0.00604113274344032\\
122	0.00604113396559873\\
123	0.00604113520955978\\
124	0.00604113647571251\\
125	0.00604113776445293\\
126	0.00604113907618413\\
127	0.00604114041131642\\
128	0.00604114177026745\\
129	0.00604114315346236\\
130	0.0060411445613339\\
131	0.00604114599432259\\
132	0.00604114745287682\\
133	0.00604114893745302\\
134	0.00604115044851583\\
135	0.00604115198653819\\
136	0.00604115355200153\\
137	0.00604115514539594\\
138	0.00604115676722027\\
139	0.00604115841798234\\
140	0.00604116009819909\\
141	0.00604116180839674\\
142	0.00604116354911094\\
143	0.00604116532088699\\
144	0.00604116712427997\\
145	0.00604116895985495\\
146	0.00604117082818714\\
147	0.00604117272986213\\
148	0.006041174665476\\
149	0.00604117663563558\\
150	0.00604117864095861\\
151	0.00604118068207396\\
152	0.00604118275962181\\
153	0.0060411848742539\\
154	0.00604118702663369\\
155	0.00604118921743659\\
156	0.00604119144735022\\
157	0.00604119371707457\\
158	0.00604119602732228\\
159	0.00604119837881883\\
160	0.00604120077230283\\
161	0.0060412032085262\\
162	0.00604120568825446\\
163	0.00604120821226695\\
164	0.00604121078135712\\
165	0.00604121339633276\\
166	0.00604121605801628\\
167	0.00604121876724496\\
168	0.00604122152487127\\
169	0.00604122433176311\\
170	0.00604122718880408\\
171	0.00604123009689385\\
172	0.00604123305694836\\
173	0.00604123606990021\\
174	0.00604123913669891\\
175	0.0060412422583112\\
176	0.00604124543572141\\
177	0.00604124866993176\\
178	0.00604125196196269\\
179	0.0060412553128532\\
180	0.0060412587236612\\
181	0.0060412621954639\\
182	0.00604126572935811\\
183	0.00604126932646064\\
184	0.00604127298790867\\
185	0.00604127671486014\\
186	0.00604128050849412\\
187	0.00604128437001124\\
188	0.00604128830063406\\
189	0.00604129230160751\\
190	0.00604129637419929\\
191	0.00604130051970032\\
192	0.00604130473942516\\
193	0.00604130903471248\\
194	0.0060413134069255\\
195	0.00604131785745244\\
196	0.00604132238770704\\
197	0.006041326999129\\
198	0.0060413316931845\\
199	0.00604133647136667\\
200	0.00604134133519617\\
201	0.00604134628622164\\
202	0.00604135132602027\\
203	0.00604135645619836\\
204	0.00604136167839183\\
205	0.00604136699426683\\
206	0.00604137240552032\\
207	0.00604137791388061\\
208	0.00604138352110804\\
209	0.00604138922899551\\
210	0.00604139503936919\\
211	0.00604140095408909\\
212	0.00604140697504975\\
213	0.00604141310418091\\
214	0.00604141934344817\\
215	0.00604142569485369\\
216	0.00604143216043692\\
217	0.00604143874227528\\
218	0.00604144544248493\\
219	0.0060414522632215\\
220	0.0060414592066809\\
221	0.00604146627510004\\
222	0.00604147347075768\\
223	0.00604148079597522\\
224	0.00604148825311754\\
225	0.00604149584459385\\
226	0.00604150357285854\\
227	0.00604151144041212\\
228	0.00604151944980203\\
229	0.00604152760362364\\
230	0.00604153590452117\\
231	0.00604154435518861\\
232	0.00604155295837076\\
233	0.00604156171686419\\
234	0.00604157063351824\\
235	0.00604157971123613\\
236	0.00604158895297595\\
237	0.00604159836175177\\
238	0.00604160794063474\\
239	0.00604161769275425\\
240	0.00604162762129901\\
241	0.0060416377295183\\
242	0.00604164802072311\\
243	0.0060416584982874\\
244	0.00604166916564931\\
245	0.00604168002631246\\
246	0.00604169108384724\\
247	0.00604170234189212\\
248	0.00604171380415501\\
249	0.00604172547441465\\
250	0.00604173735652197\\
251	0.00604174945440157\\
252	0.00604176177205316\\
253	0.00604177431355303\\
254	0.0060417870830556\\
255	0.00604180008479495\\
256	0.00604181332308638\\
257	0.00604182680232807\\
258	0.00604184052700265\\
259	0.00604185450167894\\
260	0.0060418687310136\\
261	0.00604188321975289\\
262	0.00604189797273442\\
263	0.00604191299488896\\
264	0.0060419282912423\\
265	0.00604194386691706\\
266	0.00604195972713465\\
267	0.00604197587721716\\
268	0.00604199232258936\\
269	0.00604200906878071\\
270	0.00604202612142738\\
271	0.00604204348627433\\
272	0.00604206116917744\\
273	0.00604207917610567\\
274	0.00604209751314321\\
275	0.00604211618649171\\
276	0.00604213520247258\\
277	0.00604215456752925\\
278	0.0060421742882295\\
279	0.00604219437126785\\
280	0.00604221482346796\\
281	0.00604223565178506\\
282	0.00604225686330848\\
283	0.00604227846526408\\
284	0.0060423004650169\\
285	0.00604232287007367\\
286	0.00604234568808549\\
287	0.00604236892685046\\
288	0.00604239259431639\\
289	0.00604241669858353\\
290	0.00604244124790734\\
291	0.00604246625070126\\
292	0.00604249171553958\\
293	0.0060425176511603\\
294	0.00604254406646799\\
295	0.00604257097053676\\
296	0.00604259837261321\\
297	0.00604262628211939\\
298	0.00604265470865585\\
299	0.00604268366200469\\
300	0.00604271315213258\\
301	0.00604274318919392\\
302	0.00604277378353395\\
303	0.00604280494569188\\
304	0.00604283668640407\\
305	0.00604286901660724\\
306	0.0060429019474417\\
307	0.00604293549025455\\
308	0.00604296965660301\\
309	0.00604300445825766\\
310	0.00604303990720574\\
311	0.00604307601565452\\
312	0.00604311279603465\\
313	0.00604315026100349\\
314	0.00604318842344856\\
315	0.00604322729649096\\
316	0.00604326689348877\\
317	0.00604330722804057\\
318	0.00604334831398897\\
319	0.00604339016542409\\
320	0.00604343279668721\\
321	0.00604347622237433\\
322	0.0060435204573399\\
323	0.00604356551670048\\
324	0.00604361141583855\\
325	0.00604365817040637\\
326	0.00604370579632987\\
327	0.00604375430981263\\
328	0.00604380372734001\\
329	0.0060438540656833\\
330	0.00604390534190403\\
331	0.00604395757335837\\
332	0.00604401077770169\\
333	0.0060440649728932\\
334	0.00604412017720086\\
335	0.00604417640920629\\
336	0.00604423368780999\\
337	0.00604429203223669\\
338	0.00604435146204086\\
339	0.00604441199711243\\
340	0.00604447365768277\\
341	0.00604453646433083\\
342	0.00604460043798947\\
343	0.00604466559995213\\
344	0.00604473197187965\\
345	0.00604479957580744\\
346	0.00604486843415292\\
347	0.00604493856972325\\
348	0.00604501000572354\\
349	0.0060450827657654\\
350	0.00604515687387593\\
351	0.00604523235450739\\
352	0.00604530923254732\\
353	0.00604538753332946\\
354	0.0060454672826454\\
355	0.00604554850675698\\
356	0.00604563123240967\\
357	0.00604571548684694\\
358	0.0060458012978257\\
359	0.00604588869363288\\
360	0.00604597770310341\\
361	0.00604606835563952\\
362	0.00604616068123166\\
363	0.00604625471048106\\
364	0.00604635047462414\\
365	0.00604644800555896\\
366	0.00604654733587378\\
367	0.00604664849887804\\
368	0.00604675152863593\\
369	0.0060468564600027\\
370	0.00604696332866412\\
371	0.00604707217117922\\
372	0.00604718302502664\\
373	0.00604729592865502\\
374	0.00604741092153761\\
375	0.00604752804423182\\
376	0.00604764733844376\\
377	0.00604776884709875\\
378	0.0060478926144181\\
379	0.00604801868600297\\
380	0.00604814710892623\\
381	0.00604827793183324\\
382	0.00604841120505288\\
383	0.0060485469807205\\
384	0.00604868531291534\\
385	0.00604882625781609\\
386	0.00604896987388125\\
387	0.00604911622206545\\
388	0.00604926536609071\\
389	0.00604941737280087\\
390	0.00604957231262645\\
391	0.00604973026014073\\
392	0.00604989129451384\\
393	0.00605005549929356\\
394	0.00605022296069095\\
395	0.00605039376545726\\
396	0.00605056800689307\\
397	0.00605074578751558\\
398	0.00605092721555956\\
399	0.00605111240527569\\
400	0.00605130147713777\\
401	0.00605149455786511\\
402	0.00605169178012603\\
403	0.00605189328180619\\
404	0.00605209920498479\\
405	0.00605230969563584\\
406	0.00605252490688595\\
407	0.00605274500993518\\
408	0.0060529702088865\\
409	0.00605320072346447\\
410	0.00605343675722967\\
411	0.00605367852939864\\
412	0.00605392627674797\\
413	0.00605418025580464\\
414	0.00605444074537174\\
415	0.00605470804944785\\
416	0.00605498250061092\\
417	0.00605526446395273\\
418	0.00605555434167216\\
419	0.00605585257847013\\
420	0.00605615966795491\\
421	0.00605647616040805\\
422	0.00605680267260717\\
423	0.00605713990129653\\
424	0.00605748864427628\\
425	0.00605784983945151\\
426	0.00605822464906579\\
427	0.00605861466002468\\
428	0.00605902237900171\\
429	0.00605945244313382\\
430	0.00605991439894868\\
431	0.00606042809967847\\
432	0.00606102918372906\\
433	0.00606174518355961\\
434	0.00606247727923921\\
435	0.00606322597220445\\
436	0.00606399178991594\\
437	0.00606477528778837\\
438	0.00606557705128793\\
439	0.00606639769823559\\
440	0.00606723788140878\\
441	0.00606809829169523\\
442	0.00606897966253552\\
443	0.00606988277785511\\
444	0.00607080849019261\\
445	0.00607175776981034\\
446	0.0060727318503275\\
447	0.00607373268129346\\
448	0.00607476437618227\\
449	0.00607583795301545\\
450	0.00607698718544002\\
451	0.00607832269276952\\
452	0.00608022018707292\\
453	0.0060839880453354\\
454	0.00609427677510399\\
455	0.00610602598612627\\
456	0.00611801045428564\\
457	0.00613023305681882\\
458	0.00614269683042821\\
459	0.00615540509118221\\
460	0.00616836159537399\\
461	0.0061815707187161\\
462	0.00619503751751249\\
463	0.00620876727626111\\
464	0.00622276419136081\\
465	0.00623703469742191\\
466	0.00625158713579219\\
467	0.00626643030583339\\
468	0.00628157302414865\\
469	0.00629702404398672\\
470	0.00631279317771494\\
471	0.00632889036639022\\
472	0.00634532552123743\\
473	0.00636210877493495\\
474	0.00637925218463179\\
475	0.0063967732779601\\
476	0.00641468508647931\\
477	0.00643300146736872\\
478	0.00645173718233791\\
479	0.00647090797796981\\
480	0.0064905306751564\\
481	0.00651062326907722\\
482	0.00653120504144705\\
483	0.00655229668708556\\
484	0.00657392045725157\\
485	0.00659610032266658\\
486	0.00661886215977514\\
487	0.00664223396456578\\
488	0.00666624609900567\\
489	0.0066909315760426\\
490	0.00671632639188002\\
491	0.00674246991249381\\
492	0.00676940532775348\\
493	0.0067971801847125\\
494	0.00682584702209267\\
495	0.00685546413475298\\
496	0.0068860965102351\\
497	0.00691781705813337\\
498	0.00695070840770705\\
499	0.00698486600954099\\
500	0.00702040460012362\\
501	0.00705747392433718\\
502	0.0070963007897498\\
503	0.0071373072041076\\
504	0.0071814500473006\\
505	0.00723162610810303\\
506	0.00729745314543209\\
507	0.00736801731413665\\
508	0.00743995105733167\\
509	0.00751337158735593\\
510	0.00758861983498507\\
511	0.00766578718907356\\
512	0.00774497343798484\\
513	0.00782628766354034\\
514	0.00790984955612919\\
515	0.00799579078492592\\
516	0.00808425609444664\\
517	0.00817540328038095\\
518	0.00826939821203734\\
519	0.00836639020145198\\
520	0.00846641204396294\\
521	0.00856899519194826\\
522	0.00867171404819373\\
523	0.0087647045328796\\
524	0.00884301530583041\\
525	0.00892393530751383\\
526	0.00900765861368781\\
527	0.00909442832702381\\
528	0.00918460888670943\\
529	0.00927892176734096\\
530	0.00937920083063166\\
531	0.00949088647832879\\
532	0.00962035111008971\\
533	0.0097474577377326\\
534	0.0098590903546554\\
535	0.00994932493360521\\
536	0.0100416125047448\\
537	0.0101361613614959\\
538	0.010217173924014\\
539	0.010288560840392\\
540	0.0103617101615038\\
541	0.0104366645799178\\
542	0.0105134646396043\\
543	0.0105921497950417\\
544	0.0106727647579807\\
545	0.0107553072623498\\
546	0.0108396471253698\\
547	0.0109256324321949\\
548	0.0110136279709086\\
549	0.0111040797378401\\
550	0.0111938998042921\\
551	0.0112799328341239\\
552	0.0113737245489534\\
553	0.0114714566879937\\
554	0.0115689818513626\\
555	0.0116584763545022\\
556	0.0117397844527682\\
557	0.0118200973228264\\
558	0.0118996361358905\\
559	0.0119781018230238\\
560	0.0120544708172926\\
561	0.0121231089225033\\
562	0.0121844999004352\\
563	0.0122397314293899\\
564	0.0122953412749118\\
565	0.0123511564035504\\
566	0.0124050036232864\\
567	0.0124576550265305\\
568	0.0125113556956924\\
569	0.012570391603119\\
570	0.0126387511495328\\
571	0.0127067897504844\\
572	0.0127746116796269\\
573	0.012838925819619\\
574	0.0129017946708741\\
575	0.0129608236903396\\
576	0.0130103047378455\\
577	0.0130582618703364\\
578	0.0131044278302121\\
579	0.0131492299830451\\
580	0.0131922728506593\\
581	0.0132325781836974\\
582	0.0132679389451329\\
583	0.0133018403599666\\
584	0.0133343810661122\\
585	0.0133644452420943\\
586	0.0133926358436795\\
587	0.0134186816413615\\
588	0.0134418521091651\\
589	0.0134628106883845\\
590	0.0134816861851665\\
591	0.0134994363358936\\
592	0.0135162644577772\\
593	0.0135324897661906\\
594	0.0135486051791571\\
595	0.0135654641563757\\
596	0.0135851815686143\\
597	0.0136131889078992\\
598	0.0136637438596031\\
599	0\\
600	0\\
};
\addplot [color=mycolor17,solid,forget plot]
  table[row sep=crcr]{%
1	0.00693592557646952\\
2	0.0069359287400365\\
3	0.00693593196017262\\
4	0.00693593523788794\\
5	0.00693593857421057\\
6	0.00693594197018693\\
7	0.00693594542688211\\
8	0.00693594894538016\\
9	0.00693595252678446\\
10	0.0069359561722181\\
11	0.00693595988282414\\
12	0.00693596365976603\\
13	0.00693596750422798\\
14	0.00693597141741529\\
15	0.00693597540055474\\
16	0.00693597945489498\\
17	0.00693598358170693\\
18	0.00693598778228414\\
19	0.00693599205794324\\
20	0.00693599641002429\\
21	0.00693600083989127\\
22	0.00693600534893245\\
23	0.00693600993856085\\
24	0.00693601461021466\\
25	0.0069360193653577\\
26	0.0069360242054799\\
27	0.00693602913209772\\
28	0.00693603414675466\\
29	0.0069360392510217\\
30	0.00693604444649783\\
31	0.00693604973481054\\
32	0.00693605511761632\\
33	0.00693606059660116\\
34	0.00693606617348111\\
35	0.00693607185000279\\
36	0.00693607762794394\\
37	0.00693608350911399\\
38	0.00693608949535461\\
39	0.00693609558854026\\
40	0.00693610179057883\\
41	0.00693610810341218\\
42	0.00693611452901679\\
43	0.00693612106940435\\
44	0.00693612772662238\\
45	0.00693613450275491\\
46	0.00693614139992307\\
47	0.00693614842028581\\
48	0.00693615556604053\\
49	0.0069361628394238\\
50	0.00693617024271202\\
51	0.00693617777822215\\
52	0.00693618544831245\\
53	0.0069361932553832\\
54	0.00693620120187743\\
55	0.0069362092902817\\
56	0.0069362175231269\\
57	0.00693622590298898\\
58	0.00693623443248983\\
59	0.00693624311429801\\
60	0.00693625195112968\\
61	0.00693626094574937\\
62	0.00693627010097087\\
63	0.00693627941965812\\
64	0.0069362889047261\\
65	0.00693629855914172\\
66	0.00693630838592476\\
67	0.00693631838814885\\
68	0.00693632856894234\\
69	0.00693633893148938\\
70	0.00693634947903084\\
71	0.00693636021486536\\
72	0.00693637114235037\\
73	0.00693638226490314\\
74	0.00693639358600186\\
75	0.0069364051091867\\
76	0.00693641683806093\\
77	0.00693642877629208\\
78	0.00693644092761304\\
79	0.00693645329582324\\
80	0.00693646588478985\\
81	0.00693647869844899\\
82	0.00693649174080697\\
83	0.0069365050159415\\
84	0.00693651852800304\\
85	0.00693653228121604\\
86	0.00693654627988026\\
87	0.00693656052837221\\
88	0.00693657503114637\\
89	0.00693658979273672\\
90	0.0069366048177581\\
91	0.00693662011090768\\
92	0.00693663567696641\\
93	0.00693665152080054\\
94	0.00693666764736313\\
95	0.00693668406169562\\
96	0.00693670076892944\\
97	0.00693671777428754\\
98	0.00693673508308614\\
99	0.00693675270073629\\
100	0.00693677063274568\\
101	0.00693678888472027\\
102	0.00693680746236614\\
103	0.00693682637149123\\
104	0.0069368456180072\\
105	0.00693686520793127\\
106	0.00693688514738815\\
107	0.0069369054426119\\
108	0.006936926099948\\
109	0.00693694712585525\\
110	0.00693696852690788\\
111	0.00693699030979759\\
112	0.00693701248133567\\
113	0.00693703504845516\\
114	0.00693705801821303\\
115	0.00693708139779242\\
116	0.00693710519450491\\
117	0.00693712941579284\\
118	0.00693715406923166\\
119	0.00693717916253233\\
120	0.00693720470354375\\
121	0.00693723070025531\\
122	0.00693725716079934\\
123	0.00693728409345375\\
124	0.00693731150664463\\
125	0.00693733940894895\\
126	0.0069373678090973\\
127	0.0069373967159766\\
128	0.006937426138633\\
129	0.00693745608627473\\
130	0.00693748656827506\\
131	0.00693751759417525\\
132	0.00693754917368762\\
133	0.00693758131669867\\
134	0.00693761403327218\\
135	0.0069376473336525\\
136	0.00693768122826777\\
137	0.00693771572773331\\
138	0.00693775084285496\\
139	0.0069377865846326\\
140	0.00693782296426367\\
141	0.00693785999314673\\
142	0.00693789768288517\\
143	0.00693793604529091\\
144	0.00693797509238822\\
145	0.0069380148364176\\
146	0.00693805528983968\\
147	0.0069380964653393\\
148	0.00693813837582959\\
149	0.00693818103445611\\
150	0.00693822445460114\\
151	0.00693826864988801\\
152	0.00693831363418551\\
153	0.00693835942161236\\
154	0.00693840602654186\\
155	0.00693845346360649\\
156	0.00693850174770273\\
157	0.00693855089399587\\
158	0.00693860091792498\\
159	0.00693865183520793\\
160	0.00693870366184653\\
161	0.00693875641413177\\
162	0.00693881010864915\\
163	0.00693886476228411\\
164	0.00693892039222757\\
165	0.0069389770159816\\
166	0.00693903465136513\\
167	0.00693909331651986\\
168	0.00693915302991621\\
169	0.00693921381035942\\
170	0.0069392756769958\\
171	0.00693933864931899\\
172	0.0069394027471765\\
173	0.00693946799077621\\
174	0.00693953440069316\\
175	0.00693960199787633\\
176	0.00693967080365563\\
177	0.00693974083974905\\
178	0.00693981212826983\\
179	0.00693988469173393\\
180	0.00693995855306753\\
181	0.00694003373561471\\
182	0.00694011026314531\\
183	0.00694018815986288\\
184	0.00694026745041289\\
185	0.00694034815989098\\
186	0.00694043031385146\\
187	0.00694051393831594\\
188	0.00694059905978218\\
189	0.00694068570523302\\
190	0.00694077390214557\\
191	0.0069408636785006\\
192	0.006940955062792\\
193	0.00694104808403656\\
194	0.00694114277178389\\
195	0.00694123915612652\\
196	0.0069413372677102\\
197	0.00694143713774449\\
198	0.00694153879801346\\
199	0.00694164228088665\\
200	0.00694174761933026\\
201	0.00694185484691855\\
202	0.00694196399784546\\
203	0.00694207510693652\\
204	0.00694218820966091\\
205	0.00694230334214386\\
206	0.00694242054117922\\
207	0.00694253984424234\\
208	0.00694266128950323\\
209	0.00694278491583989\\
210	0.00694291076285202\\
211	0.00694303887087494\\
212	0.00694316928099387\\
213	0.00694330203505838\\
214	0.00694343717569727\\
215	0.00694357474633366\\
216	0.00694371479120044\\
217	0.00694385735535603\\
218	0.00694400248470044\\
219	0.00694415022599167\\
220	0.00694430062686247\\
221	0.00694445373583744\\
222	0.00694460960235046\\
223	0.00694476827676248\\
224	0.00694492981037971\\
225	0.00694509425547217\\
226	0.00694526166529259\\
227	0.00694543209409576\\
228	0.00694560559715825\\
229	0.00694578223079852\\
230	0.00694596205239747\\
231	0.00694614512041946\\
232	0.00694633149443361\\
233	0.00694652123513576\\
234	0.00694671440437071\\
235	0.00694691106515499\\
236	0.00694711128170008\\
237	0.00694731511943619\\
238	0.00694752264503637\\
239	0.00694773392644127\\
240	0.00694794903288435\\
241	0.00694816803491759\\
242	0.00694839100443779\\
243	0.00694861801471333\\
244	0.00694884914041155\\
245	0.00694908445762666\\
246	0.00694932404390824\\
247	0.00694956797829029\\
248	0.00694981634132092\\
249	0.00695006921509256\\
250	0.00695032668327294\\
251	0.00695058883113651\\
252	0.00695085574559664\\
253	0.00695112751523838\\
254	0.00695140423035197\\
255	0.00695168598296691\\
256	0.0069519728668868\\
257	0.00695226497772482\\
258	0.00695256241293997\\
259	0.00695286527187397\\
260	0.00695317365578888\\
261	0.00695348766790555\\
262	0.00695380741344268\\
263	0.00695413299965677\\
264	0.00695446453588274\\
265	0.00695480213357536\\
266	0.00695514590635153\\
267	0.00695549597003322\\
268	0.00695585244269139\\
269	0.00695621544469057\\
270	0.00695658509873436\\
271	0.00695696152991174\\
272	0.00695734486574422\\
273	0.00695773523623382\\
274	0.00695813277391196\\
275	0.00695853761388915\\
276	0.00695894989390558\\
277	0.00695936975438259\\
278	0.00695979733847502\\
279	0.00696023279212441\\
280	0.00696067626411313\\
281	0.00696112790611937\\
282	0.00696158787277305\\
283	0.00696205632171255\\
284	0.00696253341364248\\
285	0.00696301931239213\\
286	0.00696351418497501\\
287	0.00696401820164918\\
288	0.00696453153597842\\
289	0.00696505436489442\\
290	0.00696558686875968\\
291	0.00696612923143141\\
292	0.00696668164032621\\
293	0.00696724428648563\\
294	0.00696781736464258\\
295	0.00696840107328853\\
296	0.00696899561474163\\
297	0.00696960119521547\\
298	0.00697021802488886\\
299	0.00697084631797616\\
300	0.0069714862927986\\
301	0.00697213817185618\\
302	0.00697280218190045\\
303	0.00697347855400793\\
304	0.0069741675236543\\
305	0.00697486933078933\\
306	0.00697558421991238\\
307	0.00697631244014876\\
308	0.00697705424532668\\
309	0.00697780989405484\\
310	0.00697857964980084\\
311	0.00697936378097011\\
312	0.00698016256098566\\
313	0.00698097626836843\\
314	0.00698180518681845\\
315	0.00698264960529662\\
316	0.00698350981810742\\
317	0.0069843861249823\\
318	0.00698527883116396\\
319	0.00698618824749163\\
320	0.00698711469048721\\
321	0.00698805848244262\\
322	0.00698901995150818\\
323	0.00698999943178241\\
324	0.00699099726340311\\
325	0.00699201379264004\\
326	0.00699304937198935\\
327	0.00699410436026977\\
328	0.00699517912272096\\
329	0.0069962740311041\\
330	0.00699738946380493\\
331	0.0069985258059396\\
332	0.00699968344946339\\
333	0.0070008627932827\\
334	0.00700206424337056\\
335	0.00700328821288577\\
336	0.00700453512229612\\
337	0.00700580539950579\\
338	0.00700709947998727\\
339	0.00700841780691788\\
340	0.00700976083132122\\
341	0.00701112901221367\\
342	0.00701252281675612\\
343	0.00701394272041125\\
344	0.00701538920710653\\
345	0.00701686276940347\\
346	0.00701836390867339\\
347	0.00701989313528041\\
348	0.00702145096877222\\
349	0.00702303793807945\\
350	0.00702465458172487\\
351	0.00702630144804328\\
352	0.00702797909541394\\
353	0.00702968809250679\\
354	0.00703142901854438\\
355	0.00703320246358087\\
356	0.00703500902879987\\
357	0.00703684932683287\\
358	0.00703872398210015\\
359	0.00704063363117637\\
360	0.00704257892318313\\
361	0.00704456052021087\\
362	0.0070465790977729\\
363	0.0070486353452944\\
364	0.00705072996663958\\
365	0.00705286368068034\\
366	0.00705503722191009\\
367	0.00705725134110685\\
368	0.00705950680604973\\
369	0.00706180440229369\\
370	0.00706414493400754\\
371	0.00706652922488084\\
372	0.00706895811910579\\
373	0.00707143248244078\\
374	0.00707395320336307\\
375	0.0070765211943188\\
376	0.00707913739307948\\
377	0.00708180276421545\\
378	0.00708451830069771\\
379	0.00708728502564173\\
380	0.00709010399420837\\
381	0.00709297629567991\\
382	0.00709590305573201\\
383	0.00709888543892724\\
384	0.00710192465146155\\
385	0.00710502194420596\\
386	0.00710817861610449\\
387	0.00711139601802771\\
388	0.00711467555726204\\
389	0.00711801870297982\\
390	0.00712142699334772\\
391	0.00712490204539648\\
392	0.00712844556900577\\
393	0.00713205938442223\\
394	0.00713574543265157\\
395	0.00713950574021029\\
396	0.0071433422876819\\
397	0.00714725718194879\\
398	0.00715125278070888\\
399	0.00715533157393911\\
400	0.00715949619206064\\
401	0.0071637494130185\\
402	0.00716809416653157\\
403	0.00717253353219982\\
404	0.00717707072606006\\
405	0.00718170906979993\\
406	0.00718645194749512\\
407	0.00719130280262168\\
408	0.00719626535179438\\
409	0.00720134425810756\\
410	0.00720654488150459\\
411	0.00721187189326634\\
412	0.00721733031683269\\
413	0.00722292556988984\\
414	0.00722866351273491\\
415	0.0072345505039745\\
416	0.00724059346482704\\
417	0.00724679995354541\\
418	0.0072531782517778\\
419	0.00725973746505873\\
420	0.00726648764010154\\
421	0.00727343990221666\\
422	0.00728060661716942\\
423	0.00728800158358886\\
424	0.00729564026601822\\
425	0.00730354008922147\\
426	0.00731172084526395\\
427	0.00732020536153701\\
428	0.00732902089226944\\
429	0.00733820274777266\\
430	0.00734780527488531\\
431	0.00735793786200966\\
432	0.00736888822247437\\
433	0.0073815558343386\\
434	0.00739788986943627\\
435	0.00741461679079328\\
436	0.007431749828804\\
437	0.00744930292834301\\
438	0.00746729080307535\\
439	0.00748572899444429\\
440	0.00750463393570512\\
441	0.00752402302133271\\
442	0.00754391468193645\\
443	0.00756432846410616\\
444	0.00758528511215494\\
445	0.00760680664026718\\
446	0.00762891635426835\\
447	0.00765163867967215\\
448	0.00767499828909532\\
449	0.00769901671661117\\
450	0.0077236998978746\\
451	0.00774899258502784\\
452	0.00777461041323596\\
453	0.00779941497507422\\
454	0.00781906490806405\\
455	0.0078380894044167\\
456	0.00785760467298398\\
457	0.00787762175136461\\
458	0.00789815064129321\\
459	0.00791920016914555\\
460	0.00794077818616062\\
461	0.00796289261791559\\
462	0.00798555415032497\\
463	0.00800878031620292\\
464	0.00803259127499823\\
465	0.00805692439937071\\
466	0.00808175183128079\\
467	0.00810709293604959\\
468	0.00813297245462707\\
469	0.0081594042066264\\
470	0.00818637430444169\\
471	0.00821389267775709\\
472	0.00824196176545622\\
473	0.00827056760765769\\
474	0.00829966997130377\\
475	0.00832922668369863\\
476	0.00835944291153246\\
477	0.00839033815308526\\
478	0.00842193264304755\\
479	0.00845424765324777\\
480	0.00848730556827345\\
481	0.00852112996514715\\
482	0.00855574569694087\\
483	0.00859117898018266\\
484	0.00862745748583628\\
485	0.00866461043347\\
486	0.00870266868807612\\
487	0.00874166485959422\\
488	0.00878163340811694\\
489	0.00882261075596682\\
490	0.00886463539199106\\
491	0.00890774802683636\\
492	0.0089519917230595\\
493	0.00899741207554902\\
494	0.0090440573609577\\
495	0.00909197875295658\\
496	0.00914123076781363\\
497	0.0091918716796119\\
498	0.00924396430635652\\
499	0.00929757789943002\\
500	0.00935279335520063\\
501	0.00940971848270948\\
502	0.00946853384295804\\
503	0.00952963197949118\\
504	0.00959404354820472\\
505	0.00965937471217319\\
506	0.00971741718170556\\
507	0.00978100927947546\\
508	0.00984267120681476\\
509	0.00989645651890556\\
510	0.00994022900582074\\
511	0.0099850493671466\\
512	0.0100309357080922\\
513	0.0100779056943849\\
514	0.0101259711639922\\
515	0.0101751378927862\\
516	0.0102254066827219\\
517	0.0102767721325311\\
518	0.0103292205175522\\
519	0.0103827246591832\\
520	0.0104372277698731\\
521	0.0104925859218834\\
522	0.0105483541263133\\
523	0.0106031743169032\\
524	0.0106565553941962\\
525	0.0107108900777372\\
526	0.0107661194362737\\
527	0.0108222296441962\\
528	0.0108791887008099\\
529	0.0109369063091805\\
530	0.0109954246208243\\
531	0.0110554210458185\\
532	0.0111072387045219\\
533	0.0111592241307219\\
534	0.0112089994768058\\
535	0.0112555145534793\\
536	0.011302783334635\\
537	0.0113507882654209\\
538	0.0113970561054866\\
539	0.0114486499285055\\
540	0.0115050149328621\\
541	0.0115617821273209\\
542	0.0116188618623749\\
543	0.0116761532117478\\
544	0.0117335457566282\\
545	0.0117908775216927\\
546	0.0118479462871987\\
547	0.0119043952746839\\
548	0.0119598423665208\\
549	0.0120144037009114\\
550	0.0120672100335883\\
551	0.0121100067772152\\
552	0.0121481372446998\\
553	0.0121867398580826\\
554	0.0122257233689272\\
555	0.01226498370261\\
556	0.0123046282813976\\
557	0.0123445883022754\\
558	0.01238472481417\\
559	0.0124249698088742\\
560	0.0124653477302295\\
561	0.012505980730811\\
562	0.012545856282539\\
563	0.0125877993334322\\
564	0.0126437455304109\\
565	0.0126995253808788\\
566	0.0127551073637504\\
567	0.0128104401919974\\
568	0.0128653577953402\\
569	0.0129167710847116\\
570	0.0129594821584043\\
571	0.0130006793970259\\
572	0.0130396481804786\\
573	0.0130771861975367\\
574	0.0131132404057498\\
575	0.01314729680226\\
576	0.0131788280122701\\
577	0.0132093634648367\\
578	0.0132388265386339\\
579	0.0132668813072085\\
580	0.0132929590156252\\
581	0.0133185923536109\\
582	0.0133428334087323\\
583	0.0133656796464596\\
584	0.013386746951242\\
585	0.0134069339834939\\
586	0.0134258963353013\\
587	0.0134430288053913\\
588	0.0134592059786922\\
589	0.0134745052296271\\
590	0.013489420017042\\
591	0.013504239290221\\
592	0.0135190260558172\\
593	0.0135339097116538\\
594	0.0135491860906086\\
595	0.0135656125153459\\
596	0.0135851815686143\\
597	0.0136131889078992\\
598	0.0136637438596031\\
599	0\\
600	0\\
};
\addplot [color=mycolor18,solid,forget plot]
  table[row sep=crcr]{%
1	0.00840063842684203\\
2	0.00840064410390609\\
3	0.00840064988251434\\
4	0.00840065576448053\\
5	0.00840066175165082\\
6	0.00840066784590424\\
7	0.00840067404915337\\
8	0.00840068036334488\\
9	0.00840068679046016\\
10	0.00840069333251595\\
11	0.00840069999156492\\
12	0.00840070676969638\\
13	0.00840071366903687\\
14	0.00840072069175086\\
15	0.00840072784004142\\
16	0.00840073511615089\\
17	0.0084007425223616\\
18	0.00840075006099658\\
19	0.00840075773442027\\
20	0.00840076554503926\\
21	0.00840077349530306\\
22	0.00840078158770484\\
23	0.00840078982478224\\
24	0.00840079820911811\\
25	0.00840080674334138\\
26	0.00840081543012781\\
27	0.0084008242722009\\
28	0.00840083327233265\\
29	0.00840084243334454\\
30	0.00840085175810829\\
31	0.00840086124954684\\
32	0.00840087091063522\\
33	0.0084008807444015\\
34	0.00840089075392773\\
35	0.00840090094235089\\
36	0.00840091131286386\\
37	0.00840092186871645\\
38	0.0084009326132164\\
39	0.00840094354973039\\
40	0.00840095468168513\\
41	0.00840096601256838\\
42	0.00840097754593008\\
43	0.00840098928538343\\
44	0.00840100123460604\\
45	0.00840101339734108\\
46	0.00840102577739839\\
47	0.00840103837865576\\
48	0.00840105120506006\\
49	0.00840106426062852\\
50	0.00840107754944997\\
51	0.0084010910756861\\
52	0.00840110484357278\\
53	0.00840111885742137\\
54	0.00840113312162008\\
55	0.00840114764063532\\
56	0.00840116241901309\\
57	0.00840117746138044\\
58	0.00840119277244687\\
59	0.00840120835700582\\
60	0.00840122421993616\\
61	0.0084012403662037\\
62	0.0084012568008628\\
63	0.00840127352905786\\
64	0.008401290556025\\
65	0.00840130788709364\\
66	0.00840132552768823\\
67	0.00840134348332986\\
68	0.00840136175963806\\
69	0.00840138036233252\\
70	0.00840139929723488\\
71	0.00840141857027055\\
72	0.00840143818747056\\
73	0.00840145815497346\\
74	0.00840147847902722\\
75	0.0084014991659912\\
76	0.00840152022233811\\
77	0.00840154165465608\\
78	0.00840156346965066\\
79	0.00840158567414699\\
80	0.00840160827509186\\
81	0.00840163127955596\\
82	0.008401654694736\\
83	0.00840167852795706\\
84	0.00840170278667482\\
85	0.00840172747847792\\
86	0.00840175261109031\\
87	0.00840177819237371\\
88	0.00840180423033004\\
89	0.00840183073310394\\
90	0.00840185770898531\\
91	0.00840188516641195\\
92	0.00840191311397213\\
93	0.00840194156040735\\
94	0.00840197051461507\\
95	0.00840199998565147\\
96	0.00840202998273433\\
97	0.00840206051524588\\
98	0.0084020915927358\\
99	0.00840212322492417\\
100	0.00840215542170455\\
101	0.00840218819314707\\
102	0.00840222154950163\\
103	0.00840225550120109\\
104	0.00840229005886453\\
105	0.00840232523330067\\
106	0.00840236103551117\\
107	0.0084023974766942\\
108	0.00840243456824785\\
109	0.00840247232177382\\
110	0.00840251074908103\\
111	0.00840254986218933\\
112	0.00840258967333332\\
113	0.0084026301949662\\
114	0.00840267143976367\\
115	0.00840271342062798\\
116	0.00840275615069198\\
117	0.00840279964332326\\
118	0.00840284391212838\\
119	0.00840288897095719\\
120	0.00840293483390717\\
121	0.00840298151532795\\
122	0.0084030290298258\\
123	0.00840307739226826\\
124	0.00840312661778891\\
125	0.00840317672179209\\
126	0.00840322771995783\\
127	0.00840327962824682\\
128	0.00840333246290547\\
129	0.00840338624047108\\
130	0.0084034409777771\\
131	0.00840349669195845\\
132	0.00840355340045701\\
133	0.00840361112102718\\
134	0.00840366987174148\\
135	0.00840372967099638\\
136	0.00840379053751812\\
137	0.00840385249036871\\
138	0.00840391554895199\\
139	0.00840397973301986\\
140	0.00840404506267861\\
141	0.00840411155839529\\
142	0.00840417924100433\\
143	0.00840424813171417\\
144	0.0084043182521141\\
145	0.00840438962418116\\
146	0.0084044622702872\\
147	0.00840453621320608\\
148	0.00840461147612099\\
149	0.00840468808263192\\
150	0.00840476605676324\\
151	0.00840484542297146\\
152	0.00840492620615316\\
153	0.00840500843165293\\
154	0.00840509212527167\\
155	0.00840517731327487\\
156	0.00840526402240117\\
157	0.00840535227987093\\
158	0.00840544211339517\\
159	0.00840553355118452\\
160	0.00840562662195836\\
161	0.00840572135495422\\
162	0.00840581777993726\\
163	0.00840591592720998\\
164	0.00840601582762214\\
165	0.00840611751258079\\
166	0.00840622101406057\\
167	0.00840632636461419\\
168	0.00840643359738305\\
169	0.00840654274610817\\
170	0.00840665384514124\\
171	0.00840676692945591\\
172	0.00840688203465935\\
173	0.00840699919700393\\
174	0.00840711845339922\\
175	0.00840723984142419\\
176	0.00840736339933962\\
177	0.00840748916610079\\
178	0.00840761718137041\\
179	0.00840774748553177\\
180	0.00840788011970221\\
181	0.00840801512574676\\
182	0.00840815254629215\\
183	0.00840829242474103\\
184	0.00840843480528647\\
185	0.0084085797329268\\
186	0.00840872725348064\\
187	0.00840887741360238\\
188	0.00840903026079778\\
189	0.00840918584344005\\
190	0.00840934421078613\\
191	0.00840950541299334\\
192	0.0084096695011364\\
193	0.0084098365272247\\
194	0.00841000654421996\\
195	0.00841017960605429\\
196	0.00841035576764852\\
197	0.00841053508493098\\
198	0.0084107176148566\\
199	0.00841090341542639\\
200	0.00841109254570743\\
201	0.00841128506585303\\
202	0.00841148103712359\\
203	0.0084116805219076\\
204	0.00841188358374328\\
205	0.00841209028734056\\
206	0.00841230069860348\\
207	0.00841251488465316\\
208	0.00841273291385112\\
209	0.00841295485582314\\
210	0.00841318078148359\\
211	0.00841341076306026\\
212	0.00841364487411973\\
213	0.00841388318959322\\
214	0.00841412578580297\\
215	0.0084143727404892\\
216	0.00841462413283762\\
217	0.0084148800435075\\
218	0.00841514055466031\\
219	0.00841540574998895\\
220	0.00841567571474766\\
221	0.00841595053578246\\
222	0.00841623030156229\\
223	0.00841651510221073\\
224	0.00841680502953849\\
225	0.00841710017707649\\
226	0.00841740064010966\\
227	0.00841770651571147\\
228	0.00841801790277918\\
229	0.00841833490206982\\
230	0.00841865761623692\\
231	0.00841898614986805\\
232	0.00841932060952315\\
233	0.00841966110377364\\
234	0.00842000774324237\\
235	0.00842036064064445\\
236	0.00842071991082893\\
237	0.00842108567082134\\
238	0.00842145803986719\\
239	0.00842183713947632\\
240	0.00842222309346832\\
241	0.00842261602801875\\
242	0.00842301607170655\\
243	0.00842342335556227\\
244	0.00842383801311744\\
245	0.00842426018045505\\
246	0.00842468999626093\\
247	0.00842512760187642\\
248	0.00842557314135208\\
249	0.00842602676150256\\
250	0.00842648861196264\\
251	0.00842695884524453\\
252	0.0084274376167963\\
253	0.00842792508506169\\
254	0.00842842141154108\\
255	0.00842892676085386\\
256	0.00842944130080209\\
257	0.00842996520243556\\
258	0.0084304986401182\\
259	0.00843104179159596\\
260	0.00843159483806607\\
261	0.00843215796424791\\
262	0.00843273135845526\\
263	0.00843331521267014\\
264	0.00843390972261825\\
265	0.00843451508784597\\
266	0.00843513151179903\\
267	0.00843575920190283\\
268	0.00843639836964441\\
269	0.00843704923065627\\
270	0.00843771200480182\\
271	0.00843838691626268\\
272	0.00843907419362781\\
273	0.00843977406998446\\
274	0.00844048678301111\\
275	0.00844121257507213\\
276	0.00844195169331459\\
277	0.00844270438976698\\
278	0.00844347092143987\\
279	0.00844425155042873\\
280	0.00844504654401874\\
281	0.00844585617479173\\
282	0.00844668072073525\\
283	0.00844752046535381\\
284	0.00844837569778225\\
285	0.00844924671290141\\
286	0.00845013381145593\\
287	0.00845103730017439\\
288	0.00845195749189171\\
289	0.00845289470567383\\
290	0.00845384926694471\\
291	0.0084548215076157\\
292	0.00845581176621722\\
293	0.00845682038803285\\
294	0.00845784772523576\\
295	0.00845889413702753\\
296	0.00845995998977938\\
297	0.0084610456571758\\
298	0.00846215152036049\\
299	0.00846327796808484\\
300	0.00846442539685869\\
301	0.00846559421110346\\
302	0.00846678482330772\\
303	0.00846799765418506\\
304	0.0084692331328343\\
305	0.00847049169690202\\
306	0.00847177379274737\\
307	0.00847307987560912\\
308	0.00847441040977496\\
309	0.00847576586875299\\
310	0.0084771467354453\\
311	0.00847855350232376\\
312	0.0084799866716078\\
313	0.00848144675544426\\
314	0.00848293427608916\\
315	0.00848444976609151\\
316	0.00848599376847895\\
317	0.00848756683694522\\
318	0.00848916953603956\\
319	0.00849080244135781\\
320	0.00849246613973533\\
321	0.0084941612294417\\
322	0.00849588832037716\\
323	0.00849764803427094\\
324	0.00849944100488132\\
325	0.00850126787819771\\
326	0.00850312931264466\\
327	0.00850502597928813\\
328	0.00850695856204396\\
329	0.00850892775788904\\
330	0.00851093427707517\\
331	0.00851297884334615\\
332	0.00851506219415839\\
333	0.00851718508090553\\
334	0.00851934826914762\\
335	0.00852155253884539\\
336	0.00852379868460028\\
337	0.00852608751590099\\
338	0.00852841985737715\\
339	0.00853079654906097\\
340	0.00853321844665741\\
341	0.00853568642182322\\
342	0.00853820136245521\\
343	0.0085407641729874\\
344	0.00854337577469618\\
345	0.00854603710601338\\
346	0.00854874912284661\\
347	0.00855151279890608\\
348	0.00855432912603707\\
349	0.00855719911455682\\
350	0.00856012379359469\\
351	0.00856310421143466\\
352	0.00856614143585965\\
353	0.00856923655449866\\
354	0.00857239067517873\\
355	0.00857560492628687\\
356	0.00857888045714315\\
357	0.00858221843838626\\
358	0.00858562006237254\\
359	0.00858908654359013\\
360	0.00859261911908952\\
361	0.00859621904893227\\
362	0.0085998876166598\\
363	0.00860362612978406\\
364	0.00860743592030234\\
365	0.0086113183452385\\
366	0.00861527478721314\\
367	0.00861930665504517\\
368	0.00862341538438778\\
369	0.00862760243840159\\
370	0.0086318693084679\\
371	0.00863621751494532\\
372	0.0086406486079726\\
373	0.0086451641683208\\
374	0.00864976580829766\\
375	0.00865445517270669\\
376	0.00865923393986321\\
377	0.00866410382266882\\
378	0.00866906656974522\\
379	0.00867412396662695\\
380	0.00867927783701121\\
381	0.00868453004406164\\
382	0.00868988249176042\\
383	0.0086953371263025\\
384	0.00870089593752655\\
385	0.00870656096038763\\
386	0.0087123342765136\\
387	0.00871821801600059\\
388	0.00872421435993283\\
389	0.00873032554503007\\
390	0.00873655387430428\\
391	0.00874290174400483\\
392	0.00874937171252229\\
393	0.00875596666900957\\
394	0.00876269020304465\\
395	0.00876954720799981\\
396	0.00877654378363983\\
397	0.00878367967348816\\
398	0.00879095317877909\\
399	0.00879836745699547\\
400	0.00880592579819438\\
401	0.00881363162802031\\
402	0.00882148849594291\\
403	0.00882950002344964\\
404	0.00883766975101349\\
405	0.00884600074469163\\
406	0.00885449467954629\\
407	0.0088631499654199\\
408	0.00887195891188415\\
409	0.00888090810981988\\
410	0.00889000840579733\\
411	0.00889930190360543\\
412	0.00890879292868571\\
413	0.00891848583575452\\
414	0.00892838499058034\\
415	0.00893849474755592\\
416	0.00894881942191725\\
417	0.00895936325546501\\
418	0.00897013037439337\\
419	0.00898112473752426\\
420	0.00899235007286794\\
421	0.00900380979994188\\
422	0.00901550693478468\\
423	0.00902744397385639\\
424	0.00903962275202505\\
425	0.00905204426611349\\
426	0.0090647084425957\\
427	0.00907761377719326\\
428	0.00909075657317015\\
429	0.00910412872210131\\
430	0.00911770996208753\\
431	0.00913143910753164\\
432	0.00914510555251504\\
433	0.00915793959845932\\
434	0.00916812680014346\\
435	0.0091785293100627\\
436	0.00918915129984152\\
437	0.00919999693895049\\
438	0.00921107038370186\\
439	0.0092223757649605\\
440	0.0092339171744347\\
441	0.00924569864929151\\
442	0.00925772415444182\\
443	0.00926999756056873\\
444	0.00928252261203347\\
445	0.00929530286683722\\
446	0.00930834155523104\\
447	0.00932164120004579\\
448	0.00933520255013695\\
449	0.00934902159637897\\
450	0.0093630815196596\\
451	0.00937733253230136\\
452	0.00939164891864054\\
453	0.00940577529433277\\
454	0.00941948256359929\\
455	0.00943346910041966\\
456	0.00944791401931388\\
457	0.00946285406923779\\
458	0.00947833169538007\\
459	0.00949439676854473\\
460	0.00951111017649282\\
461	0.00952855289659365\\
462	0.00954685191139944\\
463	0.00956625932615199\\
464	0.00958740168856049\\
465	0.00961223868386329\\
466	0.00963907843714742\\
467	0.00966653251821401\\
468	0.00969476576736104\\
469	0.00972430675880319\\
470	0.00975555769263777\\
471	0.00978736612713335\\
472	0.00981965252879608\\
473	0.00985218440201735\\
474	0.00988430053592306\\
475	0.00991412733865065\\
476	0.00993633151347284\\
477	0.00995901323593493\\
478	0.00998219129664034\\
479	0.0100058744151527\\
480	0.0100300711817142\\
481	0.0100547900366138\\
482	0.010080039249981\\
483	0.010105826902284\\
484	0.0101321608658823\\
485	0.0101590487876677\\
486	0.0101864980703451\\
487	0.0102145158400948\\
488	0.0102431088657005\\
489	0.0102722834516399\\
490	0.0103020454023941\\
491	0.0103323989575677\\
492	0.010363347067156\\
493	0.0103948905245735\\
494	0.0104270282270168\\
495	0.0104597557159804\\
496	0.0104930617072032\\
497	0.0105269299339452\\
498	0.0105613382361768\\
499	0.0105962576073175\\
500	0.0106316514529724\\
501	0.0106674759130486\\
502	0.0107036839704322\\
503	0.0107402419634592\\
504	0.0107771858478174\\
505	0.010814432979152\\
506	0.0108497822256945\\
507	0.0108783361859115\\
508	0.0109068695375369\\
509	0.0109343329101461\\
510	0.0109602808304604\\
511	0.010986629916455\\
512	0.0110133632811433\\
513	0.0110404314592121\\
514	0.0110679381502852\\
515	0.0110959617571711\\
516	0.0111245164356795\\
517	0.01115361841351\\
518	0.0111832864721112\\
519	0.0112135424991692\\
520	0.0112444123127838\\
521	0.0112759268609692\\
522	0.0113081245732649\\
523	0.0113410870879542\\
524	0.0113749579651764\\
525	0.0114106852705433\\
526	0.0114541634769512\\
527	0.011497744098317\\
528	0.011541350761772\\
529	0.0115848799434719\\
530	0.0116282137951213\\
531	0.0116712427776204\\
532	0.0117120635829607\\
533	0.0117527503477585\\
534	0.0117934612746472\\
535	0.0118341183143062\\
536	0.0118746252819239\\
537	0.0119147515812094\\
538	0.0119543795365334\\
539	0.0119885467124485\\
540	0.0120191761477029\\
541	0.0120497131132624\\
542	0.0120800785188749\\
543	0.0121101868597433\\
544	0.0121399273081876\\
545	0.0121698618497962\\
546	0.0122000503549242\\
547	0.0122304275505205\\
548	0.0122609116642179\\
549	0.0122915131671246\\
550	0.0123221966405659\\
551	0.0123519173985611\\
552	0.0123814449602816\\
553	0.0124118819066886\\
554	0.012444036750918\\
555	0.0124777472735445\\
556	0.0125125640224238\\
557	0.0125486611876602\\
558	0.012593459146519\\
559	0.0126449216464747\\
560	0.0126961852118528\\
561	0.0127471637137742\\
562	0.0127978135291046\\
563	0.012847296247837\\
564	0.0128847976414583\\
565	0.012921448743291\\
566	0.0129570753468019\\
567	0.0129913280273469\\
568	0.013024002460064\\
569	0.0130534437148665\\
570	0.0130804277521042\\
571	0.0131081048327079\\
572	0.0131355149595239\\
573	0.0131621003091514\\
574	0.0131878670970299\\
575	0.0132118866214813\\
576	0.0132347030023971\\
577	0.0132576075025751\\
578	0.0132807780900296\\
579	0.0133029936974567\\
580	0.013323692266975\\
581	0.0133431253864752\\
582	0.0133625260871674\\
583	0.0133812244736883\\
584	0.0133986729773908\\
585	0.0134151985784402\\
586	0.0134311750708794\\
587	0.0134465365381261\\
588	0.0134614184599777\\
589	0.0134757971101556\\
590	0.0134901845159537\\
591	0.0135046456749847\\
592	0.0135192117584053\\
593	0.0135339752758402\\
594	0.0135491998980231\\
595	0.0135656125153459\\
596	0.0135851815686143\\
597	0.0136131889078992\\
598	0.0136637438596031\\
599	0\\
600	0\\
};
\addplot [color=red!25!mycolor17,solid,forget plot]
  table[row sep=crcr]{%
1	0.00940022538022224\\
2	0.00940023166219765\\
3	0.00940023805660001\\
4	0.00940024456543877\\
5	0.00940025119075924\\
6	0.00940025793464325\\
7	0.00940026479920974\\
8	0.00940027178661551\\
9	0.00940027889905582\\
10	0.00940028613876511\\
11	0.0094002935080177\\
12	0.00940030100912851\\
13	0.0094003086444537\\
14	0.00940031641639155\\
15	0.00940032432738308\\
16	0.00940033237991287\\
17	0.00940034057650985\\
18	0.00940034891974806\\
19	0.00940035741224744\\
20	0.00940036605667471\\
21	0.00940037485574413\\
22	0.0094003838122184\\
23	0.00940039292890949\\
24	0.00940040220867954\\
25	0.00940041165444174\\
26	0.00940042126916125\\
27	0.00940043105585614\\
28	0.00940044101759826\\
29	0.0094004511575143\\
30	0.00940046147878669\\
31	0.00940047198465464\\
32	0.00940048267841513\\
33	0.00940049356342391\\
34	0.00940050464309662\\
35	0.0094005159209098\\
36	0.00940052740040198\\
37	0.00940053908517481\\
38	0.00940055097889415\\
39	0.00940056308529126\\
40	0.00940057540816391\\
41	0.0094005879513776\\
42	0.00940060071886677\\
43	0.009400613714636\\
44	0.00940062694276129\\
45	0.00940064040739131\\
46	0.0094006541127487\\
47	0.00940066806313141\\
48	0.009400682262914\\
49	0.00940069671654903\\
50	0.00940071142856845\\
51	0.00940072640358503\\
52	0.00940074164629374\\
53	0.0094007571614733\\
54	0.00940077295398759\\
55	0.00940078902878723\\
56	0.00940080539091106\\
57	0.00940082204548781\\
58	0.0094008389977376\\
59	0.00940085625297363\\
60	0.0094008738166038\\
61	0.00940089169413245\\
62	0.00940090989116201\\
63	0.0094009284133948\\
64	0.0094009472666348\\
65	0.00940096645678943\\
66	0.00940098598987146\\
67	0.0094010058720008\\
68	0.0094010261094065\\
69	0.0094010467084286\\
70	0.0094010676755202\\
71	0.00940108901724943\\
72	0.00940111074030148\\
73	0.00940113285148074\\
74	0.00940115535771287\\
75	0.009401178266047\\
76	0.00940120158365793\\
77	0.00940122531784834\\
78	0.00940124947605111\\
79	0.0094012740658316\\
80	0.00940129909489005\\
81	0.00940132457106398\\
82	0.00940135050233062\\
83	0.00940137689680938\\
84	0.00940140376276447\\
85	0.0094014311086074\\
86	0.00940145894289964\\
87	0.00940148727435528\\
88	0.00940151611184381\\
89	0.00940154546439281\\
90	0.00940157534119082\\
91	0.00940160575159021\\
92	0.00940163670511011\\
93	0.00940166821143935\\
94	0.00940170028043953\\
95	0.00940173292214809\\
96	0.00940176614678146\\
97	0.00940179996473822\\
98	0.00940183438660241\\
99	0.00940186942314679\\
100	0.00940190508533625\\
101	0.00940194138433124\\
102	0.00940197833149121\\
103	0.00940201593837826\\
104	0.00940205421676068\\
105	0.00940209317861668\\
106	0.00940213283613811\\
107	0.00940217320173434\\
108	0.00940221428803609\\
109	0.0094022561078994\\
110	0.00940229867440969\\
111	0.00940234200088585\\
112	0.00940238610088441\\
113	0.00940243098820379\\
114	0.00940247667688865\\
115	0.00940252318123428\\
116	0.00940257051579113\\
117	0.00940261869536929\\
118	0.00940266773504327\\
119	0.0094027176501566\\
120	0.00940276845632678\\
121	0.00940282016945009\\
122	0.00940287280570666\\
123	0.0094029263815655\\
124	0.00940298091378976\\
125	0.00940303641944192\\
126	0.00940309291588922\\
127	0.00940315042080912\\
128	0.00940320895219488\\
129	0.00940326852836117\\
130	0.00940332916794996\\
131	0.00940339088993629\\
132	0.00940345371363431\\
133	0.00940351765870336\\
134	0.00940358274515424\\
135	0.0094036489933554\\
136	0.00940371642403953\\
137	0.00940378505831001\\
138	0.00940385491764761\\
139	0.00940392602391735\\
140	0.00940399839937532\\
141	0.00940407206667579\\
142	0.00940414704887842\\
143	0.00940422336945549\\
144	0.00940430105229942\\
145	0.0094043801217303\\
146	0.00940446060250366\\
147	0.00940454251981829\\
148	0.00940462589932429\\
149	0.00940471076713122\\
150	0.00940479714981639\\
151	0.00940488507443333\\
152	0.00940497456852043\\
153	0.00940506566010971\\
154	0.00940515837773575\\
155	0.00940525275044483\\
156	0.00940534880780415\\
157	0.00940544657991136\\
158	0.00940554609740415\\
159	0.00940564739147002\\
160	0.00940575049385635\\
161	0.0094058554368805\\
162	0.00940596225344023\\
163	0.00940607097702423\\
164	0.00940618164172293\\
165	0.00940629428223936\\
166	0.00940640893390045\\
167	0.00940652563266829\\
168	0.0094066444151518\\
169	0.00940676531861851\\
170	0.00940688838100659\\
171	0.00940701364093713\\
172	0.0094071411377266\\
173	0.00940727091139963\\
174	0.00940740300270194\\
175	0.00940753745311358\\
176	0.00940767430486237\\
177	0.00940781360093767\\
178	0.00940795538510432\\
179	0.00940809970191697\\
180	0.00940824659673453\\
181	0.00940839611573497\\
182	0.00940854830593051\\
183	0.0094087032151829\\
184	0.00940886089221913\\
185	0.00940902138664743\\
186	0.00940918474897353\\
187	0.00940935103061727\\
188	0.00940952028392951\\
189	0.00940969256220941\\
190	0.009409867919722\\
191	0.0094100464117161\\
192	0.0094102280944426\\
193	0.00941041302517312\\
194	0.00941060126221898\\
195	0.00941079286495059\\
196	0.00941098789381717\\
197	0.00941118641036694\\
198	0.0094113884772676\\
199	0.00941159415832733\\
200	0.0094118035185161\\
201	0.00941201662398747\\
202	0.00941223354210079\\
203	0.00941245434144385\\
204	0.009412679091856\\
205	0.00941290786445171\\
206	0.00941314073164458\\
207	0.0094133777671719\\
208	0.00941361904611963\\
209	0.0094138646449479\\
210	0.0094141146415171\\
211	0.00941436911511435\\
212	0.00941462814648064\\
213	0.00941489181783848\\
214	0.00941516021292007\\
215	0.0094154334169961\\
216	0.00941571151690512\\
217	0.00941599460108349\\
218	0.00941628275959597\\
219	0.00941657608416693\\
220	0.00941687466821222\\
221	0.00941717860687167\\
222	0.00941748799704229\\
223	0.00941780293741213\\
224	0.00941812352849486\\
225	0.00941844987266513\\
226	0.00941878207419454\\
227	0.00941912023928849\\
228	0.00941946447612374\\
229	0.00941981489488676\\
230	0.00942017160781296\\
231	0.00942053472922665\\
232	0.0094209043755819\\
233	0.00942128066550433\\
234	0.00942166371983367\\
235	0.00942205366166733\\
236	0.00942245061640486\\
237	0.00942285471179342\\
238	0.00942326607797412\\
239	0.00942368484752953\\
240	0.00942411115553208\\
241	0.00942454513959357\\
242	0.00942498693991578\\
243	0.00942543669934217\\
244	0.00942589456341071\\
245	0.00942636068040789\\
246	0.00942683520142393\\
247	0.00942731828040921\\
248	0.0094278100742319\\
249	0.00942831074273702\\
250	0.00942882044880666\\
251	0.00942933935842168\\
252	0.00942986764072468\\
253	0.00943040546808454\\
254	0.0094309530161623\\
255	0.00943151046397858\\
256	0.00943207799398257\\
257	0.00943265579212256\\
258	0.00943324404791808\\
259	0.00943384295453378\\
260	0.0094344527088549\\
261	0.00943507351156458\\
262	0.00943570556722294\\
263	0.009436349084348\\
264	0.00943700427549848\\
265	0.00943767135735856\\
266	0.00943835055082463\\
267	0.00943904208109411\\
268	0.00943974617775633\\
269	0.00944046307488556\\
270	0.00944119301113633\\
271	0.00944193622984091\\
272	0.00944269297910922\\
273	0.00944346351193113\\
274	0.00944424808628109\\
275	0.00944504696522546\\
276	0.00944586041703234\\
277	0.009446688715284\\
278	0.00944753213899218\\
279	0.00944839097271602\\
280	0.00944926550668304\\
281	0.00945015603691288\\
282	0.00945106286534422\\
283	0.00945198629996467\\
284	0.00945292665494392\\
285	0.00945388425077006\\
286	0.00945485941438938\\
287	0.00945585247934935\\
288	0.00945686378594535\\
289	0.0094578936813708\\
290	0.00945894251987102\\
291	0.00946001066290084\\
292	0.00946109847928603\\
293	0.00946220634538859\\
294	0.00946333464527616\\
295	0.00946448377089538\\
296	0.00946565412224947\\
297	0.00946684610758005\\
298	0.00946806014355325\\
299	0.00946929665545019\\
300	0.009470556077362\\
301	0.00947183885238926\\
302	0.0094731454328461\\
303	0.00947447628046895\\
304	0.00947583186662984\\
305	0.00947721267255471\\
306	0.0094786191895462\\
307	0.00948005191921142\\
308	0.00948151137369446\\
309	0.00948299807591373\\
310	0.00948451255980408\\
311	0.00948605537056373\\
312	0.00948762706490587\\
313	0.00948922821131497\\
314	0.00949085939030769\\
315	0.00949252119469822\\
316	0.00949421422986801\\
317	0.00949593911403967\\
318	0.00949769647855494\\
319	0.00949948696815639\\
320	0.00950131124127278\\
321	0.00950316997030759\\
322	0.00950506384193064\\
323	0.00950699355737229\\
324	0.00950895983271991\\
325	0.0095109633992162\\
326	0.00951300500355895\\
327	0.00951508540820172\\
328	0.00951720539165498\\
329	0.00951936574878728\\
330	0.00952156729112584\\
331	0.00952381084715627\\
332	0.00952609726262079\\
333	0.00952842740081486\\
334	0.00953080214288188\\
335	0.00953322238810603\\
336	0.00953568905420335\\
337	0.00953820307761199\\
338	0.00954076541378234\\
339	0.00954337703746915\\
340	0.00954603894302809\\
341	0.00954875214472098\\
342	0.00955151767703414\\
343	0.00955433659501601\\
344	0.00955720997463941\\
345	0.00956013891318104\\
346	0.00956312452962018\\
347	0.00956616796505788\\
348	0.00956927038315733\\
349	0.00957243297060424\\
350	0.0095756569375835\\
351	0.00957894351826476\\
352	0.00958229397128373\\
353	0.00958570958020108\\
354	0.00958919165392165\\
355	0.00959274152703758\\
356	0.00959636056015464\\
357	0.00960005014020471\\
358	0.00960381168074269\\
359	0.00960764662222614\\
360	0.00961155643227569\\
361	0.00961554260591438\\
362	0.0096196066657839\\
363	0.00962375016233579\\
364	0.00962797467399552\\
365	0.00963228180729747\\
366	0.00963667319698879\\
367	0.00964115050610019\\
368	0.00964571542598191\\
369	0.00965036967630297\\
370	0.00965511500501229\\
371	0.00965995318826026\\
372	0.00966488603027974\\
373	0.00966991536322567\\
374	0.00967504304697298\\
375	0.00968027096887291\\
376	0.00968560104346826\\
377	0.00969103521216887\\
378	0.00969657544288914\\
379	0.00970222372965012\\
380	0.00970798209214983\\
381	0.00971385257530607\\
382	0.00971983724877821\\
383	0.0097259382064777\\
384	0.00973215756608635\\
385	0.00973849746862825\\
386	0.00974496007822181\\
387	0.00975154758238821\\
388	0.00975826219407005\\
389	0.00976510615896127\\
390	0.00977208177951623\\
391	0.00977919149188917\\
392	0.00978643811257192\\
393	0.00979382563481065\\
394	0.00980136182558579\\
395	0.0098090667864548\\
396	0.00981700149961414\\
397	0.00982536414909979\\
398	0.00983408793016988\\
399	0.00984296899344703\\
400	0.0098520090471634\\
401	0.0098612096389363\\
402	0.00987057203340493\\
403	0.00988009690443604\\
404	0.00988978351906158\\
405	0.00989962751543673\\
406	0.00990961476531127\\
407	0.00991970426433803\\
408	0.00992978006952169\\
409	0.00993951524169936\\
410	0.00994798333454597\\
411	0.00995491838324685\\
412	0.00996198841372489\\
413	0.00996919548958295\\
414	0.00997654159946153\\
415	0.00998402863479882\\
416	0.00999165836872181\\
417	0.00999943243206585\\
418	0.0100073522861934\\
419	0.0100154191922237\\
420	0.0100236341762566\\
421	0.010031997990582\\
422	0.0100405110696584\\
423	0.0100491734805339\\
424	0.010057984862722\\
425	0.0100669443580834\\
426	0.0100760505258386\\
427	0.0100853012277024\\
428	0.010094693430481\\
429	0.0101042227332456\\
430	0.0101138819050338\\
431	0.0101236557866387\\
432	0.0101335027565515\\
433	0.01014329267964\\
434	0.010152780865034\\
435	0.0101624553039849\\
436	0.0101723188326044\\
437	0.0101823742111527\\
438	0.0101926241088854\\
439	0.0102030710875025\\
440	0.0102137175830946\\
441	0.0102245658864512\\
442	0.0102356181215214\\
443	0.0102468762216088\\
444	0.0102583419022861\\
445	0.0102700166282737\\
446	0.0102819015665088\\
447	0.010293997503506\\
448	0.0103063046670698\\
449	0.0103188222979053\\
450	0.010331547619344\\
451	0.0103444736083217\\
452	0.0103575857472083\\
453	0.0103708665543382\\
454	0.0103843659394169\\
455	0.0103980931536572\\
456	0.0104120452850369\\
457	0.010426218220351\\
458	0.0104406064484268\\
459	0.0104552028753707\\
460	0.010469998759181\\
461	0.010484984142475\\
462	0.0105001503655594\\
463	0.0105154974359771\\
464	0.0105310594028087\\
465	0.0105447224870884\\
466	0.0105574766869334\\
467	0.01057051836962\\
468	0.0105838793854706\\
469	0.0105976515894039\\
470	0.0106120051502292\\
471	0.0106269252441725\\
472	0.0106421295948054\\
473	0.0106575835354005\\
474	0.010673178062128\\
475	0.010688589946916\\
476	0.0107028779050528\\
477	0.0107174714743243\\
478	0.0107323758521718\\
479	0.0107475937145868\\
480	0.0107631268139347\\
481	0.0107789757985907\\
482	0.0107951400092252\\
483	0.0108116172528151\\
484	0.0108284035613643\\
485	0.0108454929568367\\
486	0.0108628772746906\\
487	0.0108805461402974\\
488	0.0108984871091588\\
489	0.0109166842737079\\
490	0.0109351163074438\\
491	0.0109537789935243\\
492	0.0109726465558139\\
493	0.0109917036341972\\
494	0.0110109171897054\\
495	0.0110302962362289\\
496	0.0110499300266453\\
497	0.0110698192119407\\
498	0.0110899651617066\\
499	0.0111103702428086\\
500	0.0111310381534936\\
501	0.0111519743023774\\
502	0.0111731861854438\\
503	0.0111946836310753\\
504	0.011216478645776\\
505	0.011238584689142\\
506	0.0112609591832853\\
507	0.0112826555779167\\
508	0.0113052072521908\\
509	0.0113291260037237\\
510	0.011359324669026\\
511	0.0113898550929772\\
512	0.0114207002532744\\
513	0.01145183489198\\
514	0.0114832570242686\\
515	0.0115149558859323\\
516	0.0115469050018112\\
517	0.0115790734695417\\
518	0.0116114261203464\\
519	0.011643923846543\\
520	0.0116765196739858\\
521	0.0117091586031359\\
522	0.0117417770214818\\
523	0.0117742984461677\\
524	0.0118066204743017\\
525	0.0118378964999197\\
526	0.0118629478512975\\
527	0.0118879336579292\\
528	0.0119127687895923\\
529	0.0119375126214545\\
530	0.0119621857565344\\
531	0.0119867611603447\\
532	0.0120112681719466\\
533	0.0120356795951325\\
534	0.0120599588300948\\
535	0.0120840468411578\\
536	0.0121078781420597\\
537	0.0121313596779242\\
538	0.0121543944822357\\
539	0.0121764699227433\\
540	0.0121983577120722\\
541	0.0122208428250783\\
542	0.0122439383126498\\
543	0.0122676584525512\\
544	0.0122920155260352\\
545	0.0123180942825718\\
546	0.0123452446050778\\
547	0.0123732279553314\\
548	0.0124020458624497\\
549	0.0124317513081825\\
550	0.0124624321916442\\
551	0.0124942611880166\\
552	0.0125275179193061\\
553	0.0125725789185033\\
554	0.0126202684013368\\
555	0.0126673374643105\\
556	0.0127141264605036\\
557	0.0127616720734353\\
558	0.0128020584216912\\
559	0.0128360640086293\\
560	0.0128691043257316\\
561	0.0129010157386838\\
562	0.0129315743184\\
563	0.0129589777156225\\
564	0.0129832273087086\\
565	0.0130071965598834\\
566	0.0130308997551807\\
567	0.0130552632004074\\
568	0.0130799921894672\\
569	0.0131039390985665\\
570	0.0131274182975934\\
571	0.0131494789434883\\
572	0.0131707902443108\\
573	0.0131919480224045\\
574	0.0132130137288525\\
575	0.013234922879128\\
576	0.0132560749400763\\
577	0.0132762429061295\\
578	0.0132951907148933\\
579	0.0133138262343339\\
580	0.0133325880297693\\
581	0.0133507372247235\\
582	0.013367860031664\\
583	0.0133844301868122\\
584	0.0134005909064923\\
585	0.0134163720167592\\
586	0.0134318302903064\\
587	0.01344693698094\\
588	0.0134616492495159\\
589	0.0134759212692747\\
590	0.0134902449069583\\
591	0.0135046704775126\\
592	0.0135192194811896\\
593	0.0135339766714342\\
594	0.0135491998980231\\
595	0.0135656125153459\\
596	0.0135851815686143\\
597	0.0136131889078992\\
598	0.0136637438596031\\
599	0\\
600	0\\
};
\addplot [color=mycolor19,solid,forget plot]
  table[row sep=crcr]{%
1	0.0102160108638796\\
2	0.010216014005395\\
3	0.0102160172032066\\
4	0.0102160204583219\\
5	0.0102160237717668\\
6	0.0102160271445851\\
7	0.0102160305778397\\
8	0.0102160340726122\\
9	0.0102160376300037\\
10	0.0102160412511347\\
11	0.0102160449371461\\
12	0.0102160486891989\\
13	0.010216052508475\\
14	0.0102160563961773\\
15	0.0102160603535304\\
16	0.0102160643817806\\
17	0.0102160684821964\\
18	0.0102160726560692\\
19	0.0102160769047134\\
20	0.0102160812294667\\
21	0.0102160856316908\\
22	0.0102160901127719\\
23	0.0102160946741208\\
24	0.0102160993171734\\
25	0.0102161040433915\\
26	0.0102161088542628\\
27	0.0102161137513016\\
28	0.0102161187360494\\
29	0.0102161238100751\\
30	0.0102161289749756\\
31	0.0102161342323764\\
32	0.010216139583932\\
33	0.0102161450313264\\
34	0.0102161505762738\\
35	0.0102161562205187\\
36	0.0102161619658371\\
37	0.0102161678140366\\
38	0.0102161737669569\\
39	0.0102161798264709\\
40	0.0102161859944845\\
41	0.0102161922729379\\
42	0.010216198663806\\
43	0.0102162051690987\\
44	0.0102162117908619\\
45	0.0102162185311781\\
46	0.0102162253921669\\
47	0.0102162323759856\\
48	0.0102162394848301\\
49	0.0102162467209357\\
50	0.0102162540865772\\
51	0.0102162615840702\\
52	0.0102162692157717\\
53	0.0102162769840806\\
54	0.0102162848914387\\
55	0.0102162929403313\\
56	0.0102163011332879\\
57	0.0102163094728836\\
58	0.0102163179617388\\
59	0.0102163266025212\\
60	0.0102163353979458\\
61	0.0102163443507759\\
62	0.0102163534638245\\
63	0.0102163627399544\\
64	0.0102163721820796\\
65	0.010216381793166\\
66	0.0102163915762325\\
67	0.0102164015343518\\
68	0.0102164116706513\\
69	0.0102164219883143\\
70	0.0102164324905807\\
71	0.0102164431807482\\
72	0.0102164540621734\\
73	0.0102164651382726\\
74	0.0102164764125231\\
75	0.0102164878884642\\
76	0.0102164995696981\\
77	0.0102165114598914\\
78	0.010216523562776\\
79	0.0102165358821503\\
80	0.0102165484218803\\
81	0.0102165611859009\\
82	0.0102165741782174\\
83	0.010216587402906\\
84	0.0102166008641159\\
85	0.0102166145660701\\
86	0.0102166285130668\\
87	0.0102166427094807\\
88	0.0102166571597647\\
89	0.0102166718684507\\
90	0.0102166868401515\\
91	0.0102167020795619\\
92	0.0102167175914606\\
93	0.0102167333807112\\
94	0.0102167494522638\\
95	0.0102167658111568\\
96	0.0102167824625184\\
97	0.010216799411568\\
98	0.0102168166636178\\
99	0.0102168342240747\\
100	0.010216852098442\\
101	0.0102168702923206\\
102	0.0102168888114115\\
103	0.0102169076615168\\
104	0.0102169268485421\\
105	0.0102169463784981\\
106	0.0102169662575024\\
107	0.0102169864917814\\
108	0.0102170070876723\\
109	0.0102170280516253\\
110	0.010217049390205\\
111	0.0102170711100931\\
112	0.01021709321809\\
113	0.010217115721117\\
114	0.0102171386262187\\
115	0.0102171619405649\\
116	0.010217185671453\\
117	0.01021720982631\\
118	0.0102172344126952\\
119	0.0102172594383022\\
120	0.0102172849109613\\
121	0.0102173108386423\\
122	0.0102173372294565\\
123	0.0102173640916595\\
124	0.0102173914336535\\
125	0.0102174192639903\\
126	0.0102174475913736\\
127	0.0102174764246616\\
128	0.0102175057728704\\
129	0.0102175356451758\\
130	0.0102175660509171\\
131	0.0102175969995993\\
132	0.0102176285008965\\
133	0.0102176605646544\\
134	0.0102176932008939\\
135	0.0102177264198139\\
136	0.0102177602317942\\
137	0.0102177946473994\\
138	0.0102178296773815\\
139	0.0102178653326834\\
140	0.0102179016244427\\
141	0.0102179385639943\\
142	0.010217976162875\\
143	0.0102180144328259\\
144	0.0102180533857969\\
145	0.01021809303395\\
146	0.0102181333896631\\
147	0.010218174465534\\
148	0.0102182162743839\\
149	0.0102182588292618\\
150	0.0102183021434483\\
151	0.0102183462304597\\
152	0.0102183911040521\\
153	0.010218436778226\\
154	0.01021848326723\\
155	0.010218530585566\\
156	0.0102185787479928\\
157	0.0102186277695315\\
158	0.0102186776654694\\
159	0.0102187284513652\\
160	0.0102187801430537\\
161	0.0102188327566504\\
162	0.010218886308557\\
163	0.0102189408154659\\
164	0.0102189962943657\\
165	0.0102190527625465\\
166	0.0102191102376047\\
167	0.0102191687374493\\
168	0.0102192282803066\\
169	0.0102192888847263\\
170	0.0102193505695869\\
171	0.0102194133541019\\
172	0.0102194772578255\\
173	0.0102195423006587\\
174	0.0102196085028552\\
175	0.010219675885028\\
176	0.0102197444681557\\
177	0.0102198142735888\\
178	0.0102198853230564\\
179	0.0102199576386727\\
180	0.0102200312429444\\
181	0.010220106158777\\
182	0.0102201824094822\\
183	0.010220260018785\\
184	0.0102203390108311\\
185	0.0102204194101943\\
186	0.0102205012418842\\
187	0.0102205845313539\\
188	0.0102206693045075\\
189	0.0102207555877089\\
190	0.010220843407789\\
191	0.0102209327920549\\
192	0.0102210237682977\\
193	0.0102211163648014\\
194	0.0102212106103516\\
195	0.0102213065342444\\
196	0.0102214041662955\\
197	0.0102215035368496\\
198	0.0102216046767893\\
199	0.0102217076175453\\
200	0.0102218123911059\\
201	0.0102219190300267\\
202	0.0102220275674412\\
203	0.0102221380370706\\
204	0.0102222504732348\\
205	0.0102223649108624\\
206	0.0102224813855024\\
207	0.0102225999333347\\
208	0.0102227205911814\\
209	0.0102228433965187\\
210	0.0102229683874881\\
211	0.0102230956029087\\
212	0.0102232250822891\\
213	0.01022335686584\\
214	0.0102234909944864\\
215	0.010223627509881\\
216	0.0102237664544167\\
217	0.0102239078712404\\
218	0.0102240518042662\\
219	0.0102241982981896\\
220	0.010224347398501\\
221	0.0102244991515008\\
222	0.0102246536043134\\
223	0.0102248108049025\\
224	0.0102249708020862\\
225	0.0102251336455524\\
226	0.0102252993858745\\
227	0.0102254680745278\\
228	0.0102256397639058\\
229	0.0102258145073367\\
230	0.0102259923591006\\
231	0.0102261733744472\\
232	0.0102263576096128\\
233	0.0102265451218391\\
234	0.0102267359693915\\
235	0.0102269302115774\\
236	0.0102271279087661\\
237	0.0102273291224079\\
238	0.0102275339150542\\
239	0.0102277423503781\\
240	0.0102279544931948\\
241	0.0102281704094831\\
242	0.0102283901664073\\
243	0.0102286138323386\\
244	0.0102288414768784\\
245	0.0102290731708812\\
246	0.0102293089864777\\
247	0.0102295489970996\\
248	0.0102297932775036\\
249	0.0102300419037967\\
250	0.0102302949534619\\
251	0.0102305525053844\\
252	0.0102308146398783\\
253	0.0102310814387143\\
254	0.0102313529851472\\
255	0.010231629363945\\
256	0.0102319106614183\\
257	0.0102321969654495\\
258	0.0102324883655245\\
259	0.0102327849527634\\
260	0.0102330868199528\\
261	0.0102333940615788\\
262	0.0102337067738608\\
263	0.0102340250547855\\
264	0.0102343490041429\\
265	0.0102346787235622\\
266	0.0102350143165489\\
267	0.0102353558885229\\
268	0.0102357035468579\\
269	0.0102360574009207\\
270	0.0102364175621132\\
271	0.0102367841439139\\
272	0.0102371572619215\\
273	0.0102375370338992\\
274	0.0102379235798206\\
275	0.0102383170219163\\
276	0.0102387174847226\\
277	0.0102391250951305\\
278	0.0102395399824371\\
279	0.010239962278398\\
280	0.0102403921172812\\
281	0.0102408296359224\\
282	0.0102412749737829\\
283	0.0102417282730078\\
284	0.0102421896784871\\
285	0.0102426593379181\\
286	0.0102431374018701\\
287	0.0102436240238508\\
288	0.010244119360375\\
289	0.0102446235710354\\
290	0.0102451368185759\\
291	0.0102456592689667\\
292	0.010246191091483\\
293	0.0102467324587851\\
294	0.0102472835470019\\
295	0.0102478445358171\\
296	0.0102484156085584\\
297	0.0102489969522896\\
298	0.0102495887579062\\
299	0.010250191220234\\
300	0.0102508045381314\\
301	0.0102514289145956\\
302	0.0102520645568722\\
303	0.0102527116765687\\
304	0.0102533704897726\\
305	0.0102540412171734\\
306	0.0102547240841893\\
307	0.0102554193210982\\
308	0.010256127163174\\
309	0.0102568478508274\\
310	0.0102575816297524\\
311	0.0102583287510776\\
312	0.0102590894715235\\
313	0.0102598640535654\\
314	0.010260652765602\\
315	0.0102614558821305\\
316	0.0102622736839272\\
317	0.0102631064582351\\
318	0.0102639544989577\\
319	0.0102648181068594\\
320	0.0102656975897724\\
321	0.0102665932628107\\
322	0.0102675054485899\\
323	0.0102684344774541\\
324	0.0102693806877097\\
325	0.0102703444258641\\
326	0.0102713260468714\\
327	0.0102723259143831\\
328	0.0102733444010033\\
329	0.0102743818885489\\
330	0.0102754387683117\\
331	0.0102765154413234\\
332	0.0102776123186206\\
333	0.0102787298215087\\
334	0.0102798683818217\\
335	0.0102810284421759\\
336	0.0102822104562145\\
337	0.0102834148888374\\
338	0.0102846422164129\\
339	0.0102858929269641\\
340	0.0102871675203215\\
341	0.0102884665082337\\
342	0.0102897904144325\\
343	0.0102911397746454\\
344	0.0102925151365476\\
345	0.0102939170598395\\
346	0.0102953461163202\\
347	0.0102968028899654\\
348	0.010298287977025\\
349	0.0102998019861563\\
350	0.010301345538624\\
351	0.0103029192686024\\
352	0.010304523823628\\
353	0.0103061598652284\\
354	0.0103078280696994\\
355	0.0103095291293405\\
356	0.0103112637528506\\
357	0.0103130326656513\\
358	0.0103148366102018\\
359	0.0103166763463019\\
360	0.0103185526513811\\
361	0.01032046632077\\
362	0.0103224181679513\\
363	0.0103244090247858\\
364	0.0103264397417111\\
365	0.0103285111879057\\
366	0.010330624251416\\
367	0.0103327798392395\\
368	0.0103349788773581\\
369	0.0103372223107155\\
370	0.0103395111031315\\
371	0.0103418462371451\\
372	0.0103442287137781\\
373	0.0103466595522112\\
374	0.0103491397893617\\
375	0.0103516704793535\\
376	0.0103542526928684\\
377	0.0103568875163673\\
378	0.0103595760511692\\
379	0.0103623194123768\\
380	0.0103651187276345\\
381	0.0103679751357077\\
382	0.01037088978487\\
383	0.0103738638310872\\
384	0.0103768984359875\\
385	0.010379994764614\\
386	0.0103831539829689\\
387	0.0103863772553981\\
388	0.0103896657419934\\
389	0.0103930205965776\\
390	0.0103964429670876\\
391	0.0103999340041957\\
392	0.0104034948971676\\
393	0.0104071269994475\\
394	0.0104108322519377\\
395	0.0104146146043903\\
396	0.0104184848226213\\
397	0.0104224769219781\\
398	0.0104266170680057\\
399	0.0104308695664695\\
400	0.0104352004506298\\
401	0.0104396104033915\\
402	0.0104441000292187\\
403	0.0104486698070544\\
404	0.0104533199583441\\
405	0.0104580500748369\\
406	0.0104628580682281\\
407	0.0104677371996859\\
408	0.0104726676393481\\
409	0.0104775923092647\\
410	0.0104823470834675\\
411	0.010486884196447\\
412	0.0104915002814511\\
413	0.0104961963054818\\
414	0.0105009734672681\\
415	0.0105058330705055\\
416	0.0105107764308712\\
417	0.0105158048741282\\
418	0.010520919733102\\
419	0.010526122344945\\
420	0.0105314140490443\\
421	0.0105367961736621\\
422	0.0105422700317515\\
423	0.0105478368998609\\
424	0.0105534980816071\\
425	0.010559254914826\\
426	0.0105651087800921\\
427	0.0105710611104227\\
428	0.0105771134015832\\
429	0.0105832672221906\\
430	0.0105895242247094\\
431	0.0105958861695747\\
432	0.0106023550128572\\
433	0.0106089341512195\\
434	0.0106156366260048\\
435	0.0106224646653094\\
436	0.0106294204940609\\
437	0.0106365063246536\\
438	0.0106437243458301\\
439	0.010651076709529\\
440	0.0106585655153817\\
441	0.0106661927924964\\
442	0.0106739604781271\\
443	0.010681870392781\\
444	0.010689924211298\\
445	0.0106981234294457\\
446	0.0107064693256434\\
447	0.0107149629173934\\
448	0.0107236049107167\\
449	0.0107323956323151\\
450	0.010741334890645\\
451	0.0107504219114158\\
452	0.0107596555063136\\
453	0.0107690339497155\\
454	0.010778554637225\\
455	0.0107882122525449\\
456	0.0107980000907883\\
457	0.0108079098889469\\
458	0.0108179316767824\\
459	0.0108280536863329\\
460	0.0108382623192952\\
461	0.0108485416956345\\
462	0.0108588669433471\\
463	0.0108692257906381\\
464	0.0108796040041826\\
465	0.0108896323702773\\
466	0.0108995751987845\\
467	0.0109096904092387\\
468	0.0109199828591609\\
469	0.0109304580797673\\
470	0.010941122976827\\
471	0.0109519880076902\\
472	0.0109630447368926\\
473	0.0109742926884899\\
474	0.0109857221422626\\
475	0.0109973602708092\\
476	0.0110092630320491\\
477	0.0110214448404314\\
478	0.0110339217960156\\
479	0.0110467119174964\\
480	0.0110598354063832\\
481	0.0110733149468219\\
482	0.0110871760472935\\
483	0.0111014474364632\\
484	0.011116161545282\\
485	0.0111313551747629\\
486	0.0111470706830823\\
487	0.0111633588603293\\
488	0.0111802877053137\\
489	0.0111985015705012\\
490	0.0112208323070155\\
491	0.0112434270011014\\
492	0.01126627719628\\
493	0.0112893756449881\\
494	0.0113127111015965\\
495	0.0113362783055923\\
496	0.0113600840848991\\
497	0.0113841177708688\\
498	0.0114083670366784\\
499	0.0114328176700319\\
500	0.0114574533068956\\
501	0.0114822551174988\\
502	0.0115072014309169\\
503	0.0115322672922947\\
504	0.0115574239746017\\
505	0.0115826385773421\\
506	0.0116078753247936\\
507	0.0116331185117995\\
508	0.0116582972416487\\
509	0.011682976487573\\
510	0.0117029182183745\\
511	0.0117229753407755\\
512	0.0117431304848434\\
513	0.0117633648421504\\
514	0.0117836580049199\\
515	0.0118039885361277\\
516	0.0118243355970798\\
517	0.0118446696352771\\
518	0.0118649301769579\\
519	0.0118850454386762\\
520	0.0119050625039699\\
521	0.0119249878098595\\
522	0.0119447913272117\\
523	0.0119644403906871\\
524	0.0119838967563132\\
525	0.0120029939191967\\
526	0.0120208686342032\\
527	0.0120388810354602\\
528	0.0120570239371964\\
529	0.012075317979301\\
530	0.012093771103102\\
531	0.0121123671487895\\
532	0.0121310863890163\\
533	0.0121499070509248\\
534	0.0121688761207841\\
535	0.0121889273014959\\
536	0.0122101674742018\\
537	0.0122319959865568\\
538	0.012254436864366\\
539	0.0122776213607122\\
540	0.0123016298278577\\
541	0.0123264885210862\\
542	0.0123522271906262\\
543	0.0123788508145031\\
544	0.0124064111370428\\
545	0.0124341899271275\\
546	0.012462879016296\\
547	0.0124929842955957\\
548	0.0125346753326465\\
549	0.0125795783189477\\
550	0.0126242494224955\\
551	0.0126695650210792\\
552	0.0127147548130075\\
553	0.0127495048228508\\
554	0.012780481350985\\
555	0.0128103328863578\\
556	0.0128389907278093\\
557	0.0128651902329027\\
558	0.0128887593915927\\
559	0.0129108394513219\\
560	0.0129325533708364\\
561	0.0129540156508846\\
562	0.0129752787946581\\
563	0.0129964965610991\\
564	0.0130192184412886\\
565	0.0130416663316402\\
566	0.0130638233245856\\
567	0.0130848801536914\\
568	0.0131050486548192\\
569	0.0131251862555491\\
570	0.0131452751480645\\
571	0.0131651374840012\\
572	0.013185614029845\\
573	0.0132064222300329\\
574	0.0132266477375825\\
575	0.0132453129505523\\
576	0.0132637709093837\\
577	0.0132819565315644\\
578	0.0133003146967692\\
579	0.0133184309476972\\
580	0.0133355606149827\\
581	0.0133522925858001\\
582	0.0133686912531438\\
583	0.0133848547771268\\
584	0.0134007994899512\\
585	0.0134164953173271\\
586	0.0134319025775093\\
587	0.013446976659015\\
588	0.0134616692113963\\
589	0.0134759300888777\\
590	0.0134902481999006\\
591	0.0135046713967894\\
592	0.0135192196273521\\
593	0.0135339766714342\\
594	0.0135491998980231\\
595	0.0135656125153459\\
596	0.0135851815686143\\
597	0.0136131889078992\\
598	0.0136637438596031\\
599	0\\
600	0\\
};
\addplot [color=red!50!mycolor17,solid,forget plot]
  table[row sep=crcr]{%
1	0.0104922631019662\\
2	0.0104922658281164\\
3	0.0104922686031995\\
4	0.010492271428093\\
5	0.0104922743036902\\
6	0.0104922772309002\\
7	0.0104922802106486\\
8	0.0104922832438776\\
9	0.0104922863315461\\
10	0.0104922894746302\\
11	0.0104922926741236\\
12	0.0104922959310376\\
13	0.0104922992464019\\
14	0.0104923026212644\\
15	0.0104923060566917\\
16	0.0104923095537698\\
17	0.0104923131136039\\
18	0.0104923167373189\\
19	0.0104923204260602\\
20	0.0104923241809933\\
21	0.0104923280033049\\
22	0.0104923318942026\\
23	0.010492335854916\\
24	0.0104923398866963\\
25	0.0104923439908175\\
26	0.010492348168576\\
27	0.0104923524212916\\
28	0.0104923567503079\\
29	0.0104923611569921\\
30	0.0104923656427363\\
31	0.0104923702089571\\
32	0.0104923748570969\\
33	0.0104923795886234\\
34	0.010492384405031\\
35	0.0104923893078406\\
36	0.0104923942986002\\
37	0.0104923993788858\\
38	0.0104924045503013\\
39	0.0104924098144795\\
40	0.0104924151730822\\
41	0.010492420627801\\
42	0.0104924261803578\\
43	0.0104924318325052\\
44	0.0104924375860271\\
45	0.0104924434427395\\
46	0.0104924494044907\\
47	0.0104924554731619\\
48	0.0104924616506682\\
49	0.0104924679389588\\
50	0.0104924743400177\\
51	0.0104924808558644\\
52	0.0104924874885545\\
53	0.0104924942401803\\
54	0.0104925011128714\\
55	0.0104925081087957\\
56	0.0104925152301594\\
57	0.0104925224792086\\
58	0.0104925298582292\\
59	0.0104925373695479\\
60	0.0104925450155332\\
61	0.0104925527985957\\
62	0.010492560721189\\
63	0.0104925687858107\\
64	0.0104925769950028\\
65	0.0104925853513528\\
66	0.0104925938574944\\
67	0.0104926025161081\\
68	0.0104926113299224\\
69	0.0104926203017146\\
70	0.0104926294343114\\
71	0.0104926387305898\\
72	0.0104926481934785\\
73	0.010492657825958\\
74	0.0104926676310624\\
75	0.0104926776118796\\
76	0.0104926877715526\\
77	0.0104926981132806\\
78	0.0104927086403197\\
79	0.0104927193559842\\
80	0.0104927302636472\\
81	0.0104927413667422\\
82	0.0104927526687637\\
83	0.0104927641732687\\
84	0.0104927758838773\\
85	0.0104927878042743\\
86	0.0104927999382101\\
87	0.0104928122895018\\
88	0.0104928248620345\\
89	0.0104928376597627\\
90	0.0104928506867109\\
91	0.0104928639469758\\
92	0.0104928774447264\\
93	0.0104928911842065\\
94	0.0104929051697349\\
95	0.0104929194057077\\
96	0.0104929338965988\\
97	0.0104929486469621\\
98	0.0104929636614321\\
99	0.0104929789447259\\
100	0.0104929945016447\\
101	0.0104930103370746\\
102	0.0104930264559889\\
103	0.0104930428634492\\
104	0.010493059564607\\
105	0.0104930765647054\\
106	0.0104930938690805\\
107	0.0104931114831635\\
108	0.0104931294124817\\
109	0.0104931476626607\\
110	0.010493166239426\\
111	0.0104931851486048\\
112	0.0104932043961274\\
113	0.0104932239880297\\
114	0.0104932439304545\\
115	0.0104932642296535\\
116	0.0104932848919895\\
117	0.0104933059239378\\
118	0.0104933273320889\\
119	0.0104933491231497\\
120	0.0104933713039462\\
121	0.0104933938814253\\
122	0.0104934168626569\\
123	0.0104934402548361\\
124	0.0104934640652854\\
125	0.0104934883014571\\
126	0.0104935129709352\\
127	0.0104935380814379\\
128	0.0104935636408201\\
129	0.0104935896570754\\
130	0.0104936161383389\\
131	0.0104936430928895\\
132	0.0104936705291525\\
133	0.0104936984557019\\
134	0.0104937268812632\\
135	0.0104937558147162\\
136	0.0104937852650972\\
137	0.0104938152416022\\
138	0.0104938457535896\\
139	0.0104938768105825\\
140	0.0104939084222723\\
141	0.0104939405985213\\
142	0.0104939733493656\\
143	0.0104940066850179\\
144	0.0104940406158713\\
145	0.0104940751525017\\
146	0.0104941103056712\\
147	0.0104941460863313\\
148	0.0104941825056264\\
149	0.0104942195748969\\
150	0.0104942573056825\\
151	0.0104942957097259\\
152	0.0104943347989761\\
153	0.0104943745855921\\
154	0.0104944150819465\\
155	0.0104944563006289\\
156	0.0104944982544502\\
157	0.0104945409564457\\
158	0.0104945844198797\\
159	0.0104946286582487\\
160	0.010494673685286\\
161	0.0104947195149653\\
162	0.0104947661615053\\
163	0.0104948136393733\\
164	0.01049486196329\\
165	0.0104949111482334\\
166	0.0104949612094436\\
167	0.0104950121624269\\
168	0.0104950640229607\\
169	0.0104951168070978\\
170	0.0104951705311712\\
171	0.010495225211799\\
172	0.0104952808658891\\
173	0.010495337510644\\
174	0.0104953951635663\\
175	0.0104954538424632\\
176	0.0104955135654519\\
177	0.0104955743509652\\
178	0.010495636217756\\
179	0.0104956991849036\\
180	0.0104957632718188\\
181	0.0104958284982492\\
182	0.0104958948842857\\
183	0.0104959624503672\\
184	0.0104960312172877\\
185	0.0104961012062009\\
186	0.0104961724386275\\
187	0.0104962449364606\\
188	0.010496318721972\\
189	0.0104963938178189\\
190	0.0104964702470499\\
191	0.0104965480331121\\
192	0.0104966271998571\\
193	0.0104967077715482\\
194	0.0104967897728671\\
195	0.0104968732289208\\
196	0.0104969581652489\\
197	0.0104970446078305\\
198	0.0104971325830917\\
199	0.0104972221179128\\
200	0.010497313239636\\
201	0.0104974059760731\\
202	0.0104975003555128\\
203	0.0104975964067293\\
204	0.0104976941589894\\
205	0.0104977936420615\\
206	0.0104978948862232\\
207	0.01049799792227\\
208	0.0104981027815234\\
209	0.0104982094958401\\
210	0.0104983180976203\\
211	0.0104984286198166\\
212	0.0104985410959434\\
213	0.0104986555600853\\
214	0.0104987720469073\\
215	0.0104988905916636\\
216	0.0104990112302072\\
217	0.010499133999\\
218	0.0104992589351221\\
219	0.0104993860762823\\
220	0.0104995154608281\\
221	0.0104996471277557\\
222	0.0104997811167208\\
223	0.0104999174680493\\
224	0.0105000562227476\\
225	0.0105001974225139\\
226	0.0105003411097492\\
227	0.0105004873275684\\
228	0.0105006361198119\\
229	0.0105007875310573\\
230	0.0105009416066306\\
231	0.0105010983926189\\
232	0.0105012579358818\\
233	0.0105014202840642\\
234	0.0105015854856083\\
235	0.0105017535897667\\
236	0.010501924646615\\
237	0.0105020987070646\\
238	0.0105022758228765\\
239	0.0105024560466742\\
240	0.0105026394319573\\
241	0.0105028260331158\\
242	0.0105030159054434\\
243	0.0105032091051526\\
244	0.0105034056893881\\
245	0.0105036057162423\\
246	0.01050380924477\\
247	0.0105040163350029\\
248	0.0105042270479657\\
249	0.0105044414456914\\
250	0.0105046595912368\\
251	0.0105048815486988\\
252	0.0105051073832308\\
253	0.0105053371610589\\
254	0.010505570949499\\
255	0.0105058088169737\\
256	0.0105060508330297\\
257	0.0105062970683554\\
258	0.0105065475947986\\
259	0.0105068024853851\\
260	0.010507061814337\\
261	0.0105073256570913\\
262	0.0105075940903192\\
263	0.0105078671919457\\
264	0.010508145041169\\
265	0.0105084277184811\\
266	0.010508715305688\\
267	0.0105090078859304\\
268	0.0105093055437056\\
269	0.0105096083648883\\
270	0.0105099164367536\\
271	0.0105102298479987\\
272	0.0105105486887663\\
273	0.010510873050668\\
274	0.0105112030268079\\
275	0.0105115387118076\\
276	0.0105118802018305\\
277	0.0105122275946078\\
278	0.0105125809894644\\
279	0.0105129404873456\\
280	0.0105133061908446\\
281	0.0105136782042302\\
282	0.0105140566334759\\
283	0.0105144415862893\\
284	0.0105148331721422\\
285	0.0105152315023017\\
286	0.0105156366898627\\
287	0.0105160488497803\\
288	0.0105164680989036\\
289	0.0105168945560115\\
290	0.0105173283418478\\
291	0.0105177695791592\\
292	0.0105182183927334\\
293	0.0105186749094392\\
294	0.010519139258268\\
295	0.010519611570376\\
296	0.0105200919791296\\
297	0.0105205806201507\\
298	0.0105210776313655\\
299	0.0105215831530543\\
300	0.0105220973279038\\
301	0.0105226203010616\\
302	0.0105231522201933\\
303	0.0105236932355421\\
304	0.0105242434999914\\
305	0.0105248031691303\\
306	0.0105253724013227\\
307	0.0105259513577794\\
308	0.0105265402026347\\
309	0.0105271391030267\\
310	0.0105277482291823\\
311	0.0105283677545068\\
312	0.0105289978556789\\
313	0.0105296387127513\\
314	0.0105302905092574\\
315	0.0105309534323245\\
316	0.0105316276727947\\
317	0.0105323134253526\\
318	0.0105330108886626\\
319	0.0105337202655145\\
320	0.0105344417629798\\
321	0.0105351755925783\\
322	0.0105359219704566\\
323	0.0105366811175805\\
324	0.01053745325994\\
325	0.0105382386287713\\
326	0.0105390374607944\\
327	0.0105398499984705\\
328	0.0105406764902786\\
329	0.0105415171910157\\
330	0.0105423723621204\\
331	0.0105432422720251\\
332	0.0105441271965372\\
333	0.0105450274192547\\
334	0.0105459432320183\\
335	0.0105468749354065\\
336	0.0105478228392777\\
337	0.0105487872633681\\
338	0.010549768537953\\
339	0.0105507670045796\\
340	0.0105517830168645\\
341	0.0105528169413062\\
342	0.0105538691580087\\
343	0.0105549400616397\\
344	0.0105560300627568\\
345	0.0105571395849841\\
346	0.0105582690650129\\
347	0.0105594189524104\\
348	0.0105605897091741\\
349	0.0105617818089554\\
350	0.0105629957358704\\
351	0.0105642319828643\\
352	0.0105654910497801\\
353	0.0105667734416641\\
354	0.0105680796674353\\
355	0.0105694102299327\\
356	0.010570765644879\\
357	0.0105721464428852\\
358	0.0105735531700711\\
359	0.0105749863887159\\
360	0.0105764466779409\\
361	0.0105779346344247\\
362	0.0105794508731527\\
363	0.0105809960282012\\
364	0.010582570753558\\
365	0.0105841757239788\\
366	0.0105858116358823\\
367	0.0105874792082816\\
368	0.0105891791837536\\
369	0.0105909123294459\\
370	0.0105926794381194\\
371	0.0105944813292259\\
372	0.0105963188500177\\
373	0.0105981928766863\\
374	0.010600104315526\\
375	0.0106020541041159\\
376	0.0106040432125143\\
377	0.0106060726444541\\
378	0.0106081434385303\\
379	0.0106102566693627\\
380	0.0106124134487185\\
381	0.0106146149265712\\
382	0.0106168622920723\\
383	0.0106191567744025\\
384	0.0106214996434668\\
385	0.0106238922103874\\
386	0.0106263358277431\\
387	0.0106288318894902\\
388	0.0106313818304932\\
389	0.0106339871255754\\
390	0.0106366492879889\\
391	0.0106393698671846\\
392	0.010642150445741\\
393	0.0106449926352949\\
394	0.0106478980712935\\
395	0.0106508684063931\\
396	0.0106539053024393\\
397	0.0106570104214851\\
398	0.0106601854181536\\
399	0.010663431923342\\
400	0.0106667515211934\\
401	0.0106701457270135\\
402	0.0106736159610154\\
403	0.0106771635172982\\
404	0.0106807895274217\\
405	0.0106844949179383\\
406	0.0106882803613391\\
407	0.0106921462202038\\
408	0.0106960924852302\\
409	0.0107001187097593\\
410	0.0107042239465306\\
411	0.010708406692336\\
412	0.0107126648710745\\
413	0.0107169956907362\\
414	0.0107213938194012\\
415	0.0107258577179076\\
416	0.0107303884949005\\
417	0.0107349873499031\\
418	0.0107396555825343\\
419	0.0107443945491256\\
420	0.0107492056447915\\
421	0.010754090654461\\
422	0.0107590516533602\\
423	0.0107640914643021\\
424	0.0107692114401413\\
425	0.0107744129987577\\
426	0.0107796976298033\\
427	0.010785066901977\\
428	0.01079052247092\\
429	0.0107960660881549\\
430	0.0108016996122699\\
431	0.0108074250244273\\
432	0.0108132444475376\\
433	0.01081916012765\\
434	0.0108251741797216\\
435	0.0108312888626474\\
436	0.0108375065981016\\
437	0.0108438299920576\\
438	0.0108502618593683\\
439	0.0108568052518437\\
440	0.0108634634903231\\
441	0.0108702402013144\\
442	0.0108771393588687\\
443	0.0108841653325115\\
444	0.0108913229423706\\
445	0.0108986175234693\\
446	0.0109060550036153\\
447	0.0109136420071115\\
448	0.0109213860218328\\
449	0.0109292957507719\\
450	0.0109373820478817\\
451	0.0109456536434308\\
452	0.0109541105728118\\
453	0.0109627553679178\\
454	0.0109715952766986\\
455	0.0109806977758795\\
456	0.0109900845475758\\
457	0.0109997801986754\\
458	0.0110098129116675\\
459	0.0110202158358084\\
460	0.0110310310121471\\
461	0.0110423224146991\\
462	0.0110564064852099\\
463	0.0110708948770853\\
464	0.0110855803674594\\
465	0.0111004767213673\\
466	0.0111155915247348\\
467	0.0111309255843516\\
468	0.0111464794308322\\
469	0.0111622533008335\\
470	0.0111782472053681\\
471	0.0111944612900714\\
472	0.0112108918541861\\
473	0.0112275359633692\\
474	0.0112443882060358\\
475	0.0112614484415217\\
476	0.0112787200694372\\
477	0.0112961973842977\\
478	0.0113138733222272\\
479	0.011331739249686\\
480	0.0113497847213654\\
481	0.011367997203887\\
482	0.0113863617612842\\
483	0.0114048606963048\\
484	0.0114234731324198\\
485	0.0114421744806877\\
486	0.0114609355615265\\
487	0.0114797204229935\\
488	0.0114984788764643\\
489	0.0115166660066114\\
490	0.0115318354217178\\
491	0.0115471363708895\\
492	0.0115625613392904\\
493	0.0115781020607263\\
494	0.0115937494686824\\
495	0.0116094935508971\\
496	0.0116253232256831\\
497	0.0116412265127655\\
498	0.0116571905147545\\
499	0.011673201409273\\
500	0.0116892444520922\\
501	0.0117053039621049\\
502	0.0117213633644552\\
503	0.0117374051852127\\
504	0.0117534110359217\\
505	0.0117693627276014\\
506	0.0117852395549783\\
507	0.0118010171733033\\
508	0.0118166667318458\\
509	0.0118320992866367\\
510	0.0118465763301248\\
511	0.0118611299157394\\
512	0.0118758315580416\\
513	0.0118906792584423\\
514	0.0119056711094668\\
515	0.0119208055951756\\
516	0.0119360845568445\\
517	0.0119515095264179\\
518	0.0119670778812097\\
519	0.011982786226415\\
520	0.0119986534641948\\
521	0.0120146925648888\\
522	0.0120309116528699\\
523	0.0120473209130739\\
524	0.0120639339183285\\
525	0.0120813714269544\\
526	0.0120994853871426\\
527	0.0121177974319214\\
528	0.012136285660672\\
529	0.0121549258671513\\
530	0.0121738599535043\\
531	0.0121934101250772\\
532	0.0122135910540104\\
533	0.0122344195004237\\
534	0.012255936213804\\
535	0.01227781202766\\
536	0.012299986826479\\
537	0.0123229992346631\\
538	0.0123468690334845\\
539	0.0123715943710749\\
540	0.0123972508259376\\
541	0.0124239552551283\\
542	0.012451953159002\\
543	0.0124888780177537\\
544	0.0125317851836834\\
545	0.0125749626463171\\
546	0.0126183997896953\\
547	0.012661041093234\\
548	0.0126932500319898\\
549	0.0127217852363899\\
550	0.0127492609051859\\
551	0.0127746658052106\\
552	0.012798208348949\\
553	0.0128191314755905\\
554	0.0128392130479626\\
555	0.0128590674270762\\
556	0.0128787040689377\\
557	0.0128979632954015\\
558	0.0129171275060748\\
559	0.0129364774215315\\
560	0.0129571735862411\\
561	0.0129784817787838\\
562	0.012999586534313\\
563	0.0130202237919064\\
564	0.013039402335484\\
565	0.0130586139383252\\
566	0.0130778453899885\\
567	0.0130969414096854\\
568	0.0131159704008785\\
569	0.0131351261912954\\
570	0.0131554993448754\\
571	0.0131756759297228\\
572	0.013194772201013\\
573	0.0132130537859863\\
574	0.0132311740231579\\
575	0.0132489771434035\\
576	0.0132670622357414\\
577	0.0132852418138994\\
578	0.0133025407158314\\
579	0.0133194135767091\\
580	0.0133360118213147\\
581	0.0133524572942046\\
582	0.0133687604594872\\
583	0.0133848934143346\\
584	0.0134008219604018\\
585	0.0134165078757943\\
586	0.0134319091005627\\
587	0.0134469797340211\\
588	0.0134616704768085\\
589	0.0134759305176615\\
590	0.0134902483085425\\
591	0.0135046714122469\\
592	0.0135192196273521\\
593	0.0135339766714342\\
594	0.0135491998980231\\
595	0.0135656125153459\\
596	0.0135851815686143\\
597	0.0136131889078992\\
598	0.0136637438596031\\
599	0\\
600	0\\
};
\addplot [color=red!40!mycolor19,solid,forget plot]
  table[row sep=crcr]{%
1	0.0106114853119478\\
2	0.0106114886185054\\
3	0.0106114919844647\\
4	0.0106114954108921\\
5	0.0106114988988731\\
6	0.0106115024495128\\
7	0.0106115060639359\\
8	0.0106115097432875\\
9	0.0106115134887331\\
10	0.0106115173014591\\
11	0.0106115211826732\\
12	0.0106115251336047\\
13	0.0106115291555051\\
14	0.0106115332496482\\
15	0.0106115374173307\\
16	0.0106115416598725\\
17	0.0106115459786171\\
18	0.0106115503749323\\
19	0.0106115548502102\\
20	0.010611559405868\\
21	0.0106115640433483\\
22	0.0106115687641193\\
23	0.0106115735696759\\
24	0.0106115784615396\\
25	0.0106115834412592\\
26	0.0106115885104112\\
27	0.0106115936706003\\
28	0.0106115989234603\\
29	0.0106116042706538\\
30	0.0106116097138735\\
31	0.0106116152548425\\
32	0.0106116208953144\\
33	0.0106116266370745\\
34	0.0106116324819402\\
35	0.0106116384317611\\
36	0.0106116444884203\\
37	0.0106116506538343\\
38	0.0106116569299542\\
39	0.0106116633187659\\
40	0.0106116698222909\\
41	0.0106116764425869\\
42	0.0106116831817486\\
43	0.0106116900419079\\
44	0.0106116970252352\\
45	0.0106117041339396\\
46	0.0106117113702697\\
47	0.0106117187365145\\
48	0.010611726235004\\
49	0.0106117338681097\\
50	0.0106117416382458\\
51	0.0106117495478696\\
52	0.0106117575994823\\
53	0.01061176579563\\
54	0.0106117741389045\\
55	0.0106117826319436\\
56	0.0106117912774326\\
57	0.0106118000781048\\
58	0.0106118090367423\\
59	0.0106118181561772\\
60	0.0106118274392921\\
61	0.0106118368890211\\
62	0.010611846508351\\
63	0.0106118563003219\\
64	0.0106118662680282\\
65	0.0106118764146198\\
66	0.0106118867433028\\
67	0.0106118972573408\\
68	0.0106119079600556\\
69	0.0106119188548285\\
70	0.0106119299451011\\
71	0.0106119412343767\\
72	0.010611952726221\\
73	0.0106119644242638\\
74	0.0106119763321994\\
75	0.0106119884537883\\
76	0.0106120007928581\\
77	0.010612013353305\\
78	0.0106120261390946\\
79	0.0106120391542633\\
80	0.0106120524029199\\
81	0.0106120658892463\\
82	0.0106120796174991\\
83	0.0106120935920112\\
84	0.0106121078171925\\
85	0.0106121222975319\\
86	0.0106121370375983\\
87	0.0106121520420422\\
88	0.010612167315597\\
89	0.0106121828630808\\
90	0.0106121986893976\\
91	0.0106122147995387\\
92	0.0106122311985847\\
93	0.0106122478917066\\
94	0.010612264884168\\
95	0.0106122821813259\\
96	0.0106122997886332\\
97	0.0106123177116399\\
98	0.0106123359559948\\
99	0.0106123545274477\\
100	0.0106123734318505\\
101	0.0106123926751598\\
102	0.0106124122634379\\
103	0.0106124322028555\\
104	0.010612452499693\\
105	0.0106124731603427\\
106	0.0106124941913108\\
107	0.0106125155992192\\
108	0.0106125373908078\\
109	0.0106125595729364\\
110	0.0106125821525871\\
111	0.0106126051368659\\
112	0.0106126285330055\\
113	0.0106126523483672\\
114	0.0106126765904431\\
115	0.0106127012668587\\
116	0.010612726385375\\
117	0.0106127519538908\\
118	0.0106127779804457\\
119	0.0106128044732218\\
120	0.0106128314405467\\
121	0.010612858890896\\
122	0.0106128868328956\\
123	0.0106129152753247\\
124	0.0106129442271185\\
125	0.0106129736973706\\
126	0.0106130036953361\\
127	0.0106130342304341\\
128	0.010613065312251\\
129	0.0106130969505431\\
130	0.0106131291552397\\
131	0.0106131619364462\\
132	0.0106131953044469\\
133	0.0106132292697085\\
134	0.0106132638428831\\
135	0.0106132990348112\\
136	0.0106133348565256\\
137	0.0106133713192541\\
138	0.0106134084344234\\
139	0.0106134462136622\\
140	0.0106134846688049\\
141	0.0106135238118953\\
142	0.01061356365519\\
143	0.0106136042111621\\
144	0.0106136454925053\\
145	0.0106136875121373\\
146	0.0106137302832041\\
147	0.0106137738190836\\
148	0.0106138181333898\\
149	0.0106138632399769\\
150	0.0106139091529434\\
151	0.0106139558866363\\
152	0.0106140034556557\\
153	0.0106140518748587\\
154	0.0106141011593641\\
155	0.010614151324557\\
156	0.0106142023860931\\
157	0.0106142543599038\\
158	0.0106143072622006\\
159	0.0106143611094798\\
160	0.0106144159185279\\
161	0.0106144717064263\\
162	0.0106145284905562\\
163	0.010614586288604\\
164	0.0106146451185665\\
165	0.0106147049987562\\
166	0.0106147659478065\\
167	0.0106148279846774\\
168	0.0106148911286613\\
169	0.0106149553993881\\
170	0.0106150208168314\\
171	0.0106150874013143\\
172	0.0106151551735151\\
173	0.0106152241544738\\
174	0.0106152943655977\\
175	0.010615365828668\\
176	0.0106154385658462\\
177	0.0106155125996802\\
178	0.0106155879531111\\
179	0.0106156646494798\\
180	0.0106157427125338\\
181	0.0106158221664337\\
182	0.0106159030357609\\
183	0.0106159853455239\\
184	0.0106160691211659\\
185	0.0106161543885722\\
186	0.0106162411740772\\
187	0.0106163295044722\\
188	0.0106164194070132\\
189	0.0106165109094282\\
190	0.0106166040399256\\
191	0.0106166988272018\\
192	0.0106167953004496\\
193	0.0106168934893662\\
194	0.0106169934241618\\
195	0.010617095135568\\
196	0.0106171986548464\\
197	0.0106173040137976\\
198	0.0106174112447697\\
199	0.0106175203806675\\
200	0.0106176314549618\\
201	0.0106177445016987\\
202	0.0106178595555088\\
203	0.0106179766516168\\
204	0.0106180958258514\\
205	0.0106182171146553\\
206	0.0106183405550946\\
207	0.0106184661848695\\
208	0.0106185940423246\\
209	0.010618724166459\\
210	0.0106188565969369\\
211	0.0106189913740988\\
212	0.0106191285389719\\
213	0.0106192681332814\\
214	0.0106194101994619\\
215	0.010619554780668\\
216	0.0106197019207869\\
217	0.0106198516644492\\
218	0.0106200040570411\\
219	0.0106201591447165\\
220	0.010620316974409\\
221	0.0106204775938443\\
222	0.0106206410515526\\
223	0.0106208073968814\\
224	0.0106209766800083\\
225	0.0106211489519541\\
226	0.0106213242645958\\
227	0.0106215026706799\\
228	0.0106216842238363\\
229	0.0106218689785917\\
230	0.0106220569903833\\
231	0.0106222483155734\\
232	0.0106224430114632\\
233	0.0106226411363072\\
234	0.010622842749328\\
235	0.010623047910731\\
236	0.0106232566817192\\
237	0.0106234691245085\\
238	0.0106236853023429\\
239	0.01062390527951\\
240	0.0106241291213568\\
241	0.0106243568943053\\
242	0.0106245886658686\\
243	0.0106248245046674\\
244	0.0106250644804459\\
245	0.0106253086640887\\
246	0.0106255571276377\\
247	0.0106258099443086\\
248	0.0106260671885082\\
249	0.0106263289358519\\
250	0.0106265952631809\\
251	0.01062686624858\\
252	0.0106271419713954\\
253	0.0106274225122526\\
254	0.0106277079530748\\
255	0.0106279983771009\\
256	0.0106282938689045\\
257	0.0106285945144121\\
258	0.0106289004009223\\
259	0.0106292116171247\\
260	0.0106295282531188\\
261	0.010629850400434\\
262	0.0106301781520484\\
263	0.0106305116024089\\
264	0.0106308508474508\\
265	0.0106311959846179\\
266	0.0106315471128821\\
267	0.0106319043327643\\
268	0.0106322677463545\\
269	0.0106326374573318\\
270	0.0106330135709858\\
271	0.0106333961942367\\
272	0.0106337854356564\\
273	0.0106341814054895\\
274	0.010634584215674\\
275	0.0106349939798628\\
276	0.0106354108134446\\
277	0.0106358348335654\\
278	0.0106362661591498\\
279	0.0106367049109225\\
280	0.0106371512114293\\
281	0.0106376051850594\\
282	0.0106380669580663\\
283	0.0106385366585895\\
284	0.0106390144166765\\
285	0.0106395003643038\\
286	0.0106399946353989\\
287	0.0106404973658617\\
288	0.0106410086935861\\
289	0.0106415287584815\\
290	0.0106420577024943\\
291	0.0106425956696292\\
292	0.0106431428059706\\
293	0.010643699259704\\
294	0.0106442651811368\\
295	0.0106448407227195\\
296	0.0106454260390668\\
297	0.0106460212869784\\
298	0.0106466266254592\\
299	0.0106472422157404\\
300	0.0106478682212995\\
301	0.0106485048078808\\
302	0.0106491521435151\\
303	0.0106498103985398\\
304	0.0106504797456186\\
305	0.0106511603597607\\
306	0.0106518524183406\\
307	0.0106525561011172\\
308	0.0106532715902523\\
309	0.0106539990703303\\
310	0.0106547387283762\\
311	0.0106554907538748\\
312	0.010656255338789\\
313	0.0106570326775785\\
314	0.0106578229672179\\
315	0.0106586264072161\\
316	0.0106594431996341\\
317	0.0106602735491048\\
318	0.0106611176628514\\
319	0.0106619757507073\\
320	0.0106628480251361\\
321	0.0106637347012522\\
322	0.010664635996842\\
323	0.0106655521323865\\
324	0.0106664833310841\\
325	0.0106674298188756\\
326	0.0106683918244704\\
327	0.0106693695793736\\
328	0.0106703633179169\\
329	0.0106713732772895\\
330	0.0106723996975737\\
331	0.010673442821782\\
332	0.0106745028958983\\
333	0.010675580168923\\
334	0.010676674892924\\
335	0.010677787323098\\
336	0.0106789177178506\\
337	0.0106800663389225\\
338	0.010681233451625\\
339	0.0106824193253267\\
340	0.0106836242344602\\
341	0.0106848484602662\\
342	0.0106860922918499\\
343	0.010687356014712\\
344	0.010688639902096\\
345	0.0106899443237689\\
346	0.0106912696713181\\
347	0.0106926163614273\\
348	0.0106939848395193\\
349	0.0106953755835486\\
350	0.0106967891073743\\
351	0.0106982259625612\\
352	0.0106996867372792\\
353	0.0107011720552247\\
354	0.0107026826054926\\
355	0.0107042193671596\\
356	0.0107057828480992\\
357	0.0107073735302338\\
358	0.0107089919070945\\
359	0.0107106384843793\\
360	0.0107123137805568\\
361	0.0107140183275212\\
362	0.0107157526713025\\
363	0.0107175173728395\\
364	0.0107193130088206\\
365	0.0107211401726009\\
366	0.010722999475204\\
367	0.0107248915464169\\
368	0.0107268170359897\\
369	0.0107287766149518\\
370	0.0107307709770569\\
371	0.0107328008403744\\
372	0.0107348669490421\\
373	0.0107369700752005\\
374	0.010739111021132\\
375	0.0107412906216275\\
376	0.0107435097466122\\
377	0.0107457693040592\\
378	0.0107480702432313\\
379	0.0107504135582904\\
380	0.010752800292323\\
381	0.0107552315418365\\
382	0.0107577084617878\\
383	0.0107602322712147\\
384	0.010762804259551\\
385	0.0107654257937172\\
386	0.0107680983260908\\
387	0.0107708234034773\\
388	0.0107736026772162\\
389	0.010776437914579\\
390	0.0107793310116346\\
391	0.0107822840077848\\
392	0.0107852991021959\\
393	0.0107883786723875\\
394	0.0107915252952693\\
395	0.010794741770959\\
396	0.0107980311497552\\
397	0.0108013967626701\\
398	0.0108048422558438\\
399	0.0108083716295327\\
400	0.0108119892823958\\
401	0.0108157000615815\\
402	0.010819509319323\\
403	0.0108234229768225\\
404	0.0108274475962958\\
405	0.010831590462225\\
406	0.0108358596733072\\
407	0.010840264247912\\
408	0.0108448142501159\\
409	0.0108495209574173\\
410	0.0108543971385928\\
411	0.0108594576730917\\
412	0.0108647213124221\\
413	0.0108702164268151\\
414	0.0108766774365729\\
415	0.0108836668553089\\
416	0.0108907591835057\\
417	0.0108979550636151\\
418	0.0109052553715621\\
419	0.0109126619799257\\
420	0.0109201748625144\\
421	0.0109277817741522\\
422	0.0109354738079255\\
423	0.0109432372541542\\
424	0.0109511253813273\\
425	0.0109591397210496\\
426	0.0109672817936892\\
427	0.0109755531043966\\
428	0.0109839551385272\\
429	0.0109924893564195\\
430	0.0110011571874733\\
431	0.0110099600232569\\
432	0.0110188992087364\\
433	0.0110279760321781\\
434	0.0110371917269777\\
435	0.0110465474564168\\
436	0.01105604430109\\
437	0.0110656832443159\\
438	0.011075465155203\\
439	0.0110853907689842\\
440	0.0110954606641771\\
441	0.0111056752360574\\
442	0.0111160346658579\\
443	0.0111265388850299\\
444	0.0111371875338444\\
445	0.0111479799136248\\
446	0.0111589149321948\\
447	0.0111699910433492\\
448	0.0111812061855367\\
449	0.0111925577401043\\
450	0.011204042581828\\
451	0.0112156562171155\\
452	0.0112273915059033\\
453	0.0112392403998852\\
454	0.0112511939583147\\
455	0.0112632515995506\\
456	0.0112754022603004\\
457	0.011287632793238\\
458	0.0112999275196998\\
459	0.0113122672387725\\
460	0.0113246260726944\\
461	0.0113369594759117\\
462	0.0113472939212519\\
463	0.0113575827214334\\
464	0.011367996979037\\
465	0.0113785348476215\\
466	0.0113891939746021\\
467	0.0113999716439867\\
468	0.0114108647499814\\
469	0.0114218697692738\\
470	0.011432982735025\\
471	0.0114441992153791\\
472	0.011455514308206\\
473	0.0114669225876084\\
474	0.0114784180701259\\
475	0.0114899941171424\\
476	0.0115016434001388\\
477	0.0115133579299799\\
478	0.0115251290859775\\
479	0.011536947611601\\
480	0.0115488036211499\\
481	0.0115606866186767\\
482	0.0115725855281146\\
483	0.0115844887303875\\
484	0.0115963840900195\\
485	0.0116082589167907\\
486	0.0116200997160952\\
487	0.0116318914087207\\
488	0.011643615696632\\
489	0.0116551704574625\\
490	0.0116661567153559\\
491	0.0116772663746034\\
492	0.0116884987836458\\
493	0.0116998533286945\\
494	0.0117113294407693\\
495	0.0117229266232878\\
496	0.0117346444885439\\
497	0.0117464827476359\\
498	0.0117584412535207\\
499	0.0117705200049101\\
500	0.0117827192596956\\
501	0.0117950402275496\\
502	0.0118074829898157\\
503	0.0118200470595363\\
504	0.011832729066091\\
505	0.0118454981484023\\
506	0.0118584090597587\\
507	0.0118714790623285\\
508	0.0118847170630902\\
509	0.0118981368719263\\
510	0.0119117776830312\\
511	0.0119256399778211\\
512	0.0119397437702992\\
513	0.0119540965950328\\
514	0.0119687880522303\\
515	0.0119846257642538\\
516	0.0120006792035105\\
517	0.012016948460691\\
518	0.012033433378248\\
519	0.0120501337038229\\
520	0.0120670487552183\\
521	0.0120841782478731\\
522	0.0121015230854084\\
523	0.0121190769550382\\
524	0.0121368289945457\\
525	0.0121542558384702\\
526	0.0121715689980968\\
527	0.0121894895629998\\
528	0.0122080315784222\\
529	0.0122272072725048\\
530	0.0122470601524105\\
531	0.0122676663885992\\
532	0.0122890490541949\\
533	0.0123112314739784\\
534	0.01233419358811\\
535	0.0123578916613893\\
536	0.0123824196727018\\
537	0.0124080735661055\\
538	0.0124376739012499\\
539	0.0124791957473463\\
540	0.012521320936842\\
541	0.0125628551667318\\
542	0.0126035863118111\\
543	0.0126363933286878\\
544	0.0126629956736317\\
545	0.0126879212814179\\
546	0.0127110005266763\\
547	0.0127327507761874\\
548	0.0127519089492287\\
549	0.0127702379348839\\
550	0.0127883494525872\\
551	0.0128061143307652\\
552	0.0128237030271585\\
553	0.0128414080727687\\
554	0.0128592914679222\\
555	0.0128773788132809\\
556	0.0128956830257798\\
557	0.0129159388381864\\
558	0.0129361789885641\\
559	0.0129562582617236\\
560	0.0129751695978427\\
561	0.0129934934782369\\
562	0.0130118829813184\\
563	0.0130302735780219\\
564	0.013048498827268\\
565	0.013066882287389\\
566	0.0130854146291867\\
567	0.0131041261948512\\
568	0.0131242851927021\\
569	0.0131439992612982\\
570	0.0131622809802794\\
571	0.0131802393603465\\
572	0.0131979881782582\\
573	0.013215611798017\\
574	0.0132333125336978\\
575	0.0132515920089052\\
576	0.0132691646651249\\
577	0.013286128642129\\
578	0.0133028709966482\\
579	0.0133194950173628\\
580	0.0133360358211135\\
581	0.0133524695865987\\
582	0.0133687674364789\\
583	0.0133848973350109\\
584	0.0134008240469672\\
585	0.0134165089013565\\
586	0.0134319095548253\\
587	0.01344697990865\\
588	0.0134616705317511\\
589	0.0134759305303028\\
590	0.0134902483101722\\
591	0.0135046714122469\\
592	0.0135192196273521\\
593	0.0135339766714342\\
594	0.0135491998980231\\
595	0.0135656125153459\\
596	0.0135851815686143\\
597	0.0136131889078992\\
598	0.0136637438596031\\
599	0\\
600	0\\
};
\addplot [color=red!75!mycolor17,solid,forget plot]
  table[row sep=crcr]{%
1	0.0107560004113626\\
2	0.0107560057046717\\
3	0.0107560110930745\\
4	0.0107560165782783\\
5	0.010756022162021\\
6	0.0107560278460718\\
7	0.0107560336322315\\
8	0.0107560395223334\\
9	0.0107560455182434\\
10	0.010756051621861\\
11	0.01075605783512\\
12	0.0107560641599884\\
13	0.01075607059847\\
14	0.0107560771526043\\
15	0.0107560838244673\\
16	0.0107560906161724\\
17	0.0107560975298708\\
18	0.0107561045677524\\
19	0.0107561117320462\\
20	0.0107561190250214\\
21	0.0107561264489876\\
22	0.0107561340062961\\
23	0.0107561416993402\\
24	0.0107561495305562\\
25	0.010756157502424\\
26	0.0107561656174681\\
27	0.0107561738782581\\
28	0.0107561822874099\\
29	0.0107561908475859\\
30	0.0107561995614968\\
31	0.0107562084319014\\
32	0.0107562174616083\\
33	0.0107562266534761\\
34	0.0107562360104151\\
35	0.0107562455353874\\
36	0.0107562552314083\\
37	0.0107562651015473\\
38	0.0107562751489288\\
39	0.0107562853767331\\
40	0.0107562957881976\\
41	0.0107563063866179\\
42	0.0107563171753484\\
43	0.0107563281578037\\
44	0.0107563393374597\\
45	0.0107563507178545\\
46	0.0107563623025897\\
47	0.0107563740953312\\
48	0.0107563860998108\\
49	0.0107563983198272\\
50	0.0107564107592471\\
51	0.0107564234220064\\
52	0.0107564363121116\\
53	0.0107564494336411\\
54	0.0107564627907461\\
55	0.0107564763876524\\
56	0.0107564902286614\\
57	0.0107565043181516\\
58	0.0107565186605798\\
59	0.0107565332604829\\
60	0.0107565481224789\\
61	0.0107565632512683\\
62	0.0107565786516363\\
63	0.0107565943284533\\
64	0.0107566102866771\\
65	0.0107566265313545\\
66	0.0107566430676224\\
67	0.0107566599007097\\
68	0.0107566770359391\\
69	0.0107566944787284\\
70	0.0107567122345926\\
71	0.0107567303091453\\
72	0.0107567487081006\\
73	0.0107567674372749\\
74	0.0107567865025889\\
75	0.0107568059100689\\
76	0.0107568256658493\\
77	0.0107568457761745\\
78	0.0107568662474001\\
79	0.0107568870859961\\
80	0.0107569082985477\\
81	0.0107569298917582\\
82	0.010756951872451\\
83	0.0107569742475712\\
84	0.0107569970241885\\
85	0.0107570202094988\\
86	0.010757043810827\\
87	0.0107570678356288\\
88	0.0107570922914934\\
89	0.0107571171861456\\
90	0.0107571425274485\\
91	0.0107571683234057\\
92	0.0107571945821641\\
93	0.010757221312016\\
94	0.0107572485214021\\
95	0.010757276218914\\
96	0.0107573044132971\\
97	0.0107573331134527\\
98	0.0107573623284415\\
99	0.0107573920674862\\
100	0.010757422339974\\
101	0.0107574531554603\\
102	0.0107574845236709\\
103	0.0107575164545054\\
104	0.0107575489580405\\
105	0.0107575820445327\\
106	0.0107576157244218\\
107	0.010757650008334\\
108	0.0107576849070855\\
109	0.0107577204316854\\
110	0.0107577565933395\\
111	0.0107577934034538\\
112	0.0107578308736376\\
113	0.010757869015708\\
114	0.0107579078416924\\
115	0.0107579473638334\\
116	0.0107579875945918\\
117	0.010758028546651\\
118	0.0107580702329203\\
119	0.0107581126665399\\
120	0.0107581558608839\\
121	0.0107581998295652\\
122	0.0107582445864395\\
123	0.0107582901456095\\
124	0.0107583365214296\\
125	0.0107583837285099\\
126	0.0107584317817211\\
127	0.0107584806961988\\
128	0.0107585304873485\\
129	0.0107585811708504\\
130	0.0107586327626638\\
131	0.0107586852790326\\
132	0.0107587387364898\\
133	0.0107587931518631\\
134	0.0107588485422799\\
135	0.0107589049251726\\
136	0.0107589623182837\\
137	0.0107590207396719\\
138	0.0107590802077172\\
139	0.0107591407411267\\
140	0.0107592023589404\\
141	0.010759265080537\\
142	0.01075932892564\\
143	0.0107593939143237\\
144	0.0107594600670194\\
145	0.0107595274045216\\
146	0.0107595959479945\\
147	0.0107596657189784\\
148	0.0107597367393967\\
149	0.010759809031562\\
150	0.0107598826181834\\
151	0.0107599575223734\\
152	0.010760033767655\\
153	0.0107601113779687\\
154	0.01076019037768\\
155	0.0107602707915871\\
156	0.0107603526449278\\
157	0.0107604359633882\\
158	0.0107605207731094\\
159	0.0107606071006967\\
160	0.0107606949732266\\
161	0.0107607844182559\\
162	0.0107608754638296\\
163	0.0107609681384897\\
164	0.0107610624712838\\
165	0.0107611584917739\\
166	0.0107612562300455\\
167	0.0107613557167166\\
168	0.010761456982947\\
169	0.0107615600604481\\
170	0.0107616649814919\\
171	0.0107617717789212\\
172	0.0107618804861596\\
173	0.0107619911372213\\
174	0.0107621037667216\\
175	0.0107622184098872\\
176	0.0107623351025669\\
177	0.0107624538812425\\
178	0.0107625747830398\\
179	0.0107626978457393\\
180	0.0107628231077884\\
181	0.010762950608312\\
182	0.0107630803871251\\
183	0.0107632124847441\\
184	0.0107633469423993\\
185	0.010763483802047\\
186	0.0107636231063823\\
187	0.0107637648988515\\
188	0.0107639092236656\\
189	0.0107640561258127\\
190	0.0107642056510721\\
191	0.0107643578460276\\
192	0.0107645127580812\\
193	0.0107646704354675\\
194	0.0107648309272677\\
195	0.0107649942834243\\
196	0.0107651605547557\\
197	0.0107653297929714\\
198	0.010765502050687\\
199	0.0107656773814401\\
200	0.0107658558397055\\
201	0.0107660374809116\\
202	0.0107662223614564\\
203	0.0107664105387244\\
204	0.0107666020711029\\
205	0.0107667970179993\\
206	0.0107669954398586\\
207	0.0107671973981806\\
208	0.0107674029555382\\
209	0.0107676121755951\\
210	0.0107678251231247\\
211	0.0107680418640286\\
212	0.0107682624653558\\
213	0.0107684869953218\\
214	0.0107687155233284\\
215	0.0107689481199838\\
216	0.0107691848571225\\
217	0.0107694258078264\\
218	0.010769671046445\\
219	0.0107699206486172\\
220	0.0107701746912928\\
221	0.0107704332527541\\
222	0.0107706964126381\\
223	0.0107709642519596\\
224	0.0107712368531335\\
225	0.0107715142999982\\
226	0.0107717966778393\\
227	0.0107720840734139\\
228	0.010772376574974\\
229	0.010772674272292\\
230	0.0107729772566856\\
231	0.0107732856210429\\
232	0.0107735994598484\\
233	0.0107739188692093\\
234	0.0107742439468819\\
235	0.0107745747922986\\
236	0.0107749115065951\\
237	0.0107752541926383\\
238	0.0107756029550546\\
239	0.010775957900258\\
240	0.0107763191364794\\
241	0.0107766867737958\\
242	0.0107770609241601\\
243	0.0107774417014312\\
244	0.0107778292214049\\
245	0.0107782236018444\\
246	0.0107786249625125\\
247	0.0107790334252026\\
248	0.0107794491137717\\
249	0.010779872154173\\
250	0.0107803026744887\\
251	0.0107807408049642\\
252	0.0107811866780418\\
253	0.0107816404283953\\
254	0.0107821021929649\\
255	0.0107825721109926\\
256	0.0107830503240583\\
257	0.0107835369761157\\
258	0.0107840322135292\\
259	0.0107845361851114\\
260	0.0107850490421601\\
261	0.0107855709384969\\
262	0.0107861020305056\\
263	0.0107866424771711\\
264	0.0107871924401189\\
265	0.0107877520836549\\
266	0.0107883215748057\\
267	0.0107889010833595\\
268	0.0107894907819068\\
269	0.0107900908458827\\
270	0.0107907014536082\\
271	0.0107913227863329\\
272	0.0107919550282781\\
273	0.0107925983666794\\
274	0.0107932529918309\\
275	0.0107939190971286\\
276	0.0107945968791154\\
277	0.0107952865375252\\
278	0.010795988275328\\
279	0.0107967022987757\\
280	0.0107974288174471\\
281	0.0107981680442942\\
282	0.0107989201956885\\
283	0.010799685491467\\
284	0.0108004641549791\\
285	0.0108012564131333\\
286	0.0108020624964442\\
287	0.0108028826390798\\
288	0.0108037170789083\\
289	0.0108045660575453\\
290	0.0108054298204016\\
291	0.0108063086167297\\
292	0.0108072026996711\\
293	0.0108081123263034\\
294	0.0108090377576868\\
295	0.0108099792589106\\
296	0.0108109370991395\\
297	0.010811911551659\\
298	0.0108129028939207\\
299	0.010813911407587\\
300	0.010814937378575\\
301	0.0108159810970992\\
302	0.0108170428577147\\
303	0.0108181229593573\\
304	0.0108192217053844\\
305	0.0108203394036132\\
306	0.0108214763663583\\
307	0.0108226329104675\\
308	0.0108238093573551\\
309	0.0108250060330344\\
310	0.010826223268147\\
311	0.0108274613979899\\
312	0.0108287207625402\\
313	0.0108300017064766\\
314	0.0108313045791973\\
315	0.0108326297348349\\
316	0.0108339775322666\\
317	0.010835348335121\\
318	0.010836742511779\\
319	0.0108381604353704\\
320	0.0108396024837644\\
321	0.0108410690395536\\
322	0.0108425604900315\\
323	0.0108440772271626\\
324	0.0108456196475437\\
325	0.010847188152357\\
326	0.0108487831473131\\
327	0.0108504050425831\\
328	0.0108520542527196\\
329	0.0108537311965649\\
330	0.0108554362971454\\
331	0.0108571699815516\\
332	0.0108589326808018\\
333	0.0108607248296908\\
334	0.0108625468666244\\
335	0.0108643992334509\\
336	0.0108662823753215\\
337	0.0108681967406926\\
338	0.0108701427818187\\
339	0.0108721209568684\\
340	0.0108741317373076\\
341	0.0108761756324051\\
342	0.0108782532697422\\
343	0.0108803656603173\\
344	0.0108825136733782\\
345	0.0108846950074879\\
346	0.0108869099389341\\
347	0.0108891587176373\\
348	0.0108914415609099\\
349	0.0108937586432021\\
350	0.0108961100758233\\
351	0.0108984958600843\\
352	0.0109009157678355\\
353	0.010903369020424\\
354	0.0109058534021411\\
355	0.0109083627740358\\
356	0.0109109142482262\\
357	0.010913509776825\\
358	0.010916150074059\\
359	0.010918835863868\\
360	0.0109215678799238\\
361	0.0109243468656362\\
362	0.010927173574143\\
363	0.0109300487682823\\
364	0.0109329732205445\\
365	0.0109359477129989\\
366	0.0109389730371938\\
367	0.0109420499940228\\
368	0.0109451793935537\\
369	0.0109483620548137\\
370	0.0109515988055234\\
371	0.0109548904817712\\
372	0.0109582379276189\\
373	0.0109616419946274\\
374	0.0109651035412889\\
375	0.0109686234323498\\
376	0.010972202538009\\
377	0.0109758417329675\\
378	0.0109795418953075\\
379	0.0109833039051726\\
380	0.0109871286432153\\
381	0.0109910169887747\\
382	0.0109949698177385\\
383	0.01099898800004\\
384	0.0110030723967259\\
385	0.0110072238565285\\
386	0.0110114432118573\\
387	0.0110157312741172\\
388	0.0110200888282406\\
389	0.0110245166263069\\
390	0.0110290153800974\\
391	0.0110335857524139\\
392	0.0110382283469545\\
393	0.0110429436965023\\
394	0.011047732249122\\
395	0.0110525943519436\\
396	0.0110575302318528\\
397	0.0110625399717327\\
398	0.0110676234923378\\
399	0.0110727805255804\\
400	0.0110780105755054\\
401	0.0110833128798142\\
402	0.0110886863655734\\
403	0.0110941295983592\\
404	0.0110996407240481\\
405	0.0111052174024484\\
406	0.011110856731972\\
407	0.0111165551644781\\
408	0.011122308408644\\
409	0.0111281113158106\\
410	0.0111339577211439\\
411	0.0111398400761058\\
412	0.0111457483486511\\
413	0.0111516660817744\\
414	0.011156964484749\\
415	0.0111619878578353\\
416	0.0111670862602773\\
417	0.0111722603121029\\
418	0.0111775106805761\\
419	0.0111828382298948\\
420	0.0111882435328969\\
421	0.0111937250392224\\
422	0.0111992817844259\\
423	0.0112049117878577\\
424	0.0112106247419682\\
425	0.0112164209651369\\
426	0.0112223007049886\\
427	0.0112282641317941\\
428	0.0112343113313877\\
429	0.0112404422975732\\
430	0.011246656923983\\
431	0.0112529549953617\\
432	0.011259336178302\\
433	0.0112658000116168\\
434	0.0112723458962629\\
435	0.0112789730832777\\
436	0.0112856806618511\\
437	0.0112924675467517\\
438	0.0112993324651341\\
439	0.0113062739427732\\
440	0.0113132902897978\\
441	0.0113203795860234\\
442	0.0113275396660291\\
443	0.0113347681041641\\
444	0.0113420621997308\\
445	0.0113494189626589\\
446	0.0113568351000644\\
447	0.0113643070041782\\
448	0.0113718307422455\\
449	0.0113794020490534\\
450	0.0113870163231759\\
451	0.0113946686344911\\
452	0.011402353738525\\
453	0.0114100660683377\\
454	0.0114177997226611\\
455	0.011425548407258\\
456	0.0114333055592898\\
457	0.0114410643381417\\
458	0.0114488175162195\\
459	0.0114565570982927\\
460	0.0114642733882142\\
461	0.0114719538126416\\
462	0.0114792554329784\\
463	0.0114866145247477\\
464	0.0114940688632244\\
465	0.0115016183534721\\
466	0.011509262862467\\
467	0.011517002226292\\
468	0.0115248362592818\\
469	0.0115327647654544\\
470	0.0115407875525308\\
471	0.0115489044487946\\
472	0.0115571153226117\\
473	0.0115654201064713\\
474	0.0115738188254873\\
475	0.0115823116332402\\
476	0.0115908988541599\\
477	0.0115995810310029\\
478	0.011608358977777\\
479	0.0116172338434812\\
480	0.0116262071176267\\
481	0.0116352806456348\\
482	0.0116444566723041\\
483	0.0116537378811409\\
484	0.0116631274226973\\
485	0.011672628922337\\
486	0.0116822464671054\\
487	0.0116919846370314\\
488	0.0117018488687023\\
489	0.0117118493223799\\
490	0.011722011854751\\
491	0.0117323382542566\\
492	0.0117428300020873\\
493	0.0117534904296702\\
494	0.0117643234498819\\
495	0.0117753332026987\\
496	0.0117865240435149\\
497	0.011797901522342\\
498	0.0118094698782025\\
499	0.0118212320871047\\
500	0.0118331862380586\\
501	0.0118453179781838\\
502	0.0118576821609804\\
503	0.0118703887331382\\
504	0.0118841323940433\\
505	0.0118980764872054\\
506	0.0119122308395106\\
507	0.0119265983264816\\
508	0.011941179670215\\
509	0.0119559759730999\\
510	0.011970987860511\\
511	0.0119862140672117\\
512	0.0120016532879103\\
513	0.0120173076330059\\
514	0.012033118226287\\
515	0.0120484035798358\\
516	0.012063927380855\\
517	0.0120796898961159\\
518	0.0120956921160588\\
519	0.0121119360931016\\
520	0.0121284139056513\\
521	0.0121451067230527\\
522	0.0121619933687936\\
523	0.0121792491352064\\
524	0.0121970856955373\\
525	0.0122154113898604\\
526	0.0122342977760097\\
527	0.0122539065579063\\
528	0.0122742599460032\\
529	0.0122953871171892\\
530	0.0123172649128693\\
531	0.0123399762252244\\
532	0.0123636472920366\\
533	0.0123887825764232\\
534	0.0124248948785538\\
535	0.0124653916607193\\
536	0.0125053032607519\\
537	0.0125444621519648\\
538	0.0125803186824415\\
539	0.0126049937783567\\
540	0.0126277827489329\\
541	0.0126494802269713\\
542	0.0126699082864006\\
543	0.0126880748582784\\
544	0.0127048644068053\\
545	0.0127213251630922\\
546	0.0127375132712684\\
547	0.0127536481786705\\
548	0.0127699093946887\\
549	0.0127863626598456\\
550	0.0128030284818492\\
551	0.0128199335315367\\
552	0.0128371151409249\\
553	0.01285476873447\\
554	0.0128743449254445\\
555	0.0128937619879908\\
556	0.0129129779678697\\
557	0.0129305492966507\\
558	0.0129481015008704\\
559	0.0129657315095173\\
560	0.012983253107484\\
561	0.0130008155217328\\
562	0.0130185598472839\\
563	0.0130364816534675\\
564	0.0130545809182304\\
565	0.0130728998578937\\
566	0.0130926915917238\\
567	0.013112029740397\\
568	0.0131297784569176\\
569	0.0131474445936999\\
570	0.0131648698721828\\
571	0.0131823037325431\\
572	0.0131997893033244\\
573	0.0132177150871859\\
574	0.0132356555798346\\
575	0.0132526603539149\\
576	0.0132694963051091\\
577	0.0132862068721776\\
578	0.0133028805932743\\
579	0.0133194990134759\\
580	0.0133360379891867\\
581	0.0133524707975475\\
582	0.013368768090595\\
583	0.0133848976667294\\
584	0.0134008242015108\\
585	0.0134165089658764\\
586	0.0134319095780783\\
587	0.0134469799154684\\
588	0.013461670533205\\
589	0.0134759305304731\\
590	0.0134902483101722\\
591	0.0135046714122469\\
592	0.0135192196273521\\
593	0.0135339766714342\\
594	0.0135491998980231\\
595	0.0135656125153459\\
596	0.0135851815686143\\
597	0.0136131889078992\\
598	0.0136637438596031\\
599	0\\
600	0\\
};
\addplot [color=red!80!mycolor19,solid,forget plot]
  table[row sep=crcr]{%
1	0.0110113268894945\\
2	0.0110113306646321\\
3	0.0110113345077589\\
4	0.0110113384200993\\
5	0.0110113424028998\\
6	0.0110113464574294\\
7	0.0110113505849797\\
8	0.011011354786866\\
9	0.0110113590644268\\
10	0.0110113634190252\\
11	0.0110113678520483\\
12	0.0110113723649087\\
13	0.0110113769590441\\
14	0.0110113816359182\\
15	0.0110113863970211\\
16	0.0110113912438696\\
17	0.0110113961780081\\
18	0.0110114012010084\\
19	0.011011406314471\\
20	0.0110114115200249\\
21	0.0110114168193288\\
22	0.0110114222140708\\
23	0.0110114277059698\\
24	0.0110114332967754\\
25	0.011011438988269\\
26	0.0110114447822638\\
27	0.0110114506806057\\
28	0.0110114566851739\\
29	0.0110114627978814\\
30	0.0110114690206758\\
31	0.0110114753555396\\
32	0.011011481804491\\
33	0.0110114883695847\\
34	0.0110114950529121\\
35	0.0110115018566026\\
36	0.0110115087828238\\
37	0.0110115158337822\\
38	0.0110115230117244\\
39	0.0110115303189369\\
40	0.011011537757748\\
41	0.0110115453305274\\
42	0.0110115530396877\\
43	0.0110115608876851\\
44	0.0110115688770197\\
45	0.0110115770102368\\
46	0.0110115852899275\\
47	0.0110115937187296\\
48	0.0110116022993283\\
49	0.0110116110344572\\
50	0.0110116199268992\\
51	0.0110116289794871\\
52	0.011011638195105\\
53	0.0110116475766886\\
54	0.0110116571272268\\
55	0.011011666849762\\
56	0.0110116767473915\\
57	0.0110116868232685\\
58	0.0110116970806027\\
59	0.0110117075226618\\
60	0.0110117181527724\\
61	0.0110117289743206\\
62	0.0110117399907539\\
63	0.0110117512055816\\
64	0.0110117626223763\\
65	0.0110117742447749\\
66	0.0110117860764796\\
67	0.0110117981212596\\
68	0.0110118103829517\\
69	0.0110118228654617\\
70	0.0110118355727659\\
71	0.0110118485089122\\
72	0.0110118616780211\\
73	0.0110118750842875\\
74	0.0110118887319818\\
75	0.0110119026254512\\
76	0.0110119167691211\\
77	0.0110119311674966\\
78	0.011011945825164\\
79	0.011011960746792\\
80	0.0110119759371333\\
81	0.0110119914010263\\
82	0.0110120071433964\\
83	0.0110120231692575\\
84	0.011012039483714\\
85	0.0110120560919619\\
86	0.0110120729992908\\
87	0.0110120902110857\\
88	0.0110121077328281\\
89	0.0110121255700984\\
90	0.0110121437285773\\
91	0.0110121622140479\\
92	0.0110121810323971\\
93	0.0110122001896179\\
94	0.0110122196918109\\
95	0.0110122395451868\\
96	0.0110122597560677\\
97	0.0110122803308895\\
98	0.0110123012762042\\
99	0.0110123225986811\\
100	0.0110123443051099\\
101	0.0110123664024023\\
102	0.0110123888975943\\
103	0.0110124117978484\\
104	0.0110124351104559\\
105	0.0110124588428394\\
106	0.0110124830025548\\
107	0.011012507597294\\
108	0.011012532634887\\
109	0.0110125581233049\\
110	0.0110125840706619\\
111	0.0110126104852181\\
112	0.0110126373753821\\
113	0.0110126647497138\\
114	0.0110126926169268\\
115	0.0110127209858914\\
116	0.0110127498656373\\
117	0.0110127792653567\\
118	0.0110128091944065\\
119	0.0110128396623124\\
120	0.0110128706787707\\
121	0.0110129022536522\\
122	0.0110129343970052\\
123	0.0110129671190581\\
124	0.0110130004302235\\
125	0.0110130343411007\\
126	0.0110130688624796\\
127	0.0110131040053439\\
128	0.0110131397808743\\
129	0.0110131762004526\\
130	0.0110132132756648\\
131	0.0110132510183051\\
132	0.0110132894403791\\
133	0.0110133285541083\\
134	0.0110133683719334\\
135	0.0110134089065183\\
136	0.0110134501707545\\
137	0.0110134921777645\\
138	0.0110135349409064\\
139	0.0110135784737781\\
140	0.0110136227902211\\
141	0.0110136679043256\\
142	0.0110137138304342\\
143	0.0110137605831468\\
144	0.0110138081773253\\
145	0.0110138566280977\\
146	0.0110139059508637\\
147	0.0110139561612985\\
148	0.0110140072753586\\
149	0.0110140593092863\\
150	0.0110141122796152\\
151	0.0110141662031747\\
152	0.011014221097096\\
153	0.011014276978817\\
154	0.0110143338660879\\
155	0.0110143917769767\\
156	0.0110144507298749\\
157	0.0110145107435033\\
158	0.0110145718369176\\
159	0.0110146340295145\\
160	0.0110146973410379\\
161	0.0110147617915846\\
162	0.0110148274016111\\
163	0.0110148941919395\\
164	0.0110149621837643\\
165	0.0110150313986588\\
166	0.0110151018585818\\
167	0.0110151735858845\\
168	0.0110152466033175\\
169	0.0110153209340375\\
170	0.0110153966016149\\
171	0.0110154736300409\\
172	0.0110155520437347\\
173	0.0110156318675515\\
174	0.0110157131267897\\
175	0.0110157958471993\\
176	0.011015880054989\\
177	0.011015965776835\\
178	0.0110160530398891\\
179	0.0110161418717864\\
180	0.0110162323006549\\
181	0.0110163243551229\\
182	0.0110164180643288\\
183	0.0110165134579295\\
184	0.0110166105661096\\
185	0.0110167094195904\\
186	0.0110168100496398\\
187	0.0110169124880813\\
188	0.0110170167673041\\
189	0.0110171229202726\\
190	0.0110172309805367\\
191	0.011017340982242\\
192	0.0110174529601399\\
193	0.0110175669495983\\
194	0.0110176829866123\\
195	0.0110178011078152\\
196	0.0110179213504891\\
197	0.0110180437525767\\
198	0.0110181683526926\\
199	0.0110182951901345\\
200	0.0110184243048954\\
201	0.0110185557376755\\
202	0.0110186895298944\\
203	0.0110188257237033\\
204	0.0110189643619979\\
205	0.011019105488431\\
206	0.0110192491474252\\
207	0.0110193953841868\\
208	0.0110195442447184\\
209	0.0110196957758331\\
210	0.0110198500251682\\
211	0.011020007041199\\
212	0.0110201668732534\\
213	0.0110203295715262\\
214	0.0110204951870939\\
215	0.0110206637719295\\
216	0.0110208353789179\\
217	0.0110210100618712\\
218	0.0110211878755444\\
219	0.0110213688756511\\
220	0.0110215531188802\\
221	0.0110217406629113\\
222	0.0110219315664325\\
223	0.0110221258891561\\
224	0.0110223236918367\\
225	0.0110225250362882\\
226	0.0110227299854012\\
227	0.0110229386031614\\
228	0.0110231509546673\\
229	0.0110233671061491\\
230	0.0110235871249868\\
231	0.0110238110797298\\
232	0.0110240390401159\\
233	0.0110242710770905\\
234	0.011024507262827\\
235	0.0110247476707463\\
236	0.0110249923755376\\
237	0.0110252414531787\\
238	0.0110254949809572\\
239	0.0110257530374914\\
240	0.0110260157027521\\
241	0.0110262830580844\\
242	0.0110265551862293\\
243	0.0110268321713468\\
244	0.0110271140990378\\
245	0.0110274010563679\\
246	0.0110276931318897\\
247	0.0110279904156673\\
248	0.0110282929992993\\
249	0.0110286009759437\\
250	0.0110289144403419\\
251	0.0110292334888433\\
252	0.0110295582194311\\
253	0.0110298887317465\\
254	0.0110302251271156\\
255	0.0110305675085742\\
256	0.0110309159808949\\
257	0.0110312706506131\\
258	0.0110316316260539\\
259	0.0110319990173595\\
260	0.0110323729365161\\
261	0.0110327534973822\\
262	0.0110331408157159\\
263	0.0110335350092037\\
264	0.0110339361974885\\
265	0.011034344502199\\
266	0.0110347600469782\\
267	0.0110351829575129\\
268	0.0110356133615633\\
269	0.0110360513889927\\
270	0.0110364971717979\\
271	0.011036950844139\\
272	0.01103741254237\\
273	0.0110378824050698\\
274	0.0110383605730728\\
275	0.0110388471895002\\
276	0.0110393423997912\\
277	0.0110398463517345\\
278	0.0110403591954997\\
279	0.0110408810836694\\
280	0.0110414121712706\\
281	0.011041952615807\\
282	0.0110425025772909\\
283	0.0110430622182754\\
284	0.011043631703886\\
285	0.0110442112018536\\
286	0.0110448008825458\\
287	0.0110454009189994\\
288	0.0110460114869523\\
289	0.0110466327648753\\
290	0.0110472649340037\\
291	0.0110479081783692\\
292	0.0110485626848308\\
293	0.011049228643106\\
294	0.0110499062458016\\
295	0.0110505956884438\\
296	0.0110512971695082\\
297	0.0110520108904489\\
298	0.0110527370557281\\
299	0.0110534758728432\\
300	0.0110542275523552\\
301	0.0110549923079149\\
302	0.0110557703562888\\
303	0.0110565619173842\\
304	0.0110573672142724\\
305	0.011058186473212\\
306	0.0110590199236697\\
307	0.0110598677983403\\
308	0.011060730333165\\
309	0.0110616077673483\\
310	0.0110625003433725\\
311	0.0110634083070104\\
312	0.0110643319073363\\
313	0.0110652713967339\\
314	0.0110662270309021\\
315	0.0110671990688578\\
316	0.0110681877729359\\
317	0.0110691934087859\\
318	0.011070216245365\\
319	0.0110712565549278\\
320	0.011072314613011\\
321	0.0110733906984149\\
322	0.0110744850931795\\
323	0.0110755980825559\\
324	0.0110767299549716\\
325	0.011077881001991\\
326	0.0110790515182692\\
327	0.0110802418014988\\
328	0.0110814521523498\\
329	0.0110826828744022\\
330	0.0110839342740697\\
331	0.0110852066605152\\
332	0.0110865003455558\\
333	0.0110878156435595\\
334	0.01108915287133\\
335	0.0110905123479843\\
336	0.0110918943948259\\
337	0.0110932993352358\\
338	0.0110947274946449\\
339	0.0110961792007985\\
340	0.0110976547849935\\
341	0.011099154586508\\
342	0.011100678967581\\
343	0.0111022283636207\\
344	0.0111038031975171\\
345	0.011105403345186\\
346	0.0111070291553165\\
347	0.0111086809821176\\
348	0.0111103591858499\\
349	0.0111120641329243\\
350	0.0111137961944744\\
351	0.0111155557403061\\
352	0.0111173431194767\\
353	0.0111191586026878\\
354	0.0111210022153724\\
355	0.0111228732544067\\
356	0.0111247750935166\\
357	0.0111267083682706\\
358	0.0111286734887712\\
359	0.0111306708636312\\
360	0.0111327008993839\\
361	0.0111347639998411\\
362	0.0111368605653959\\
363	0.0111389909922651\\
364	0.0111411556716668\\
365	0.0111433549889267\\
366	0.0111455893225098\\
367	0.0111478590429682\\
368	0.0111501645118016\\
369	0.0111525060802205\\
370	0.0111548840878072\\
371	0.0111572988610641\\
372	0.0111597507118427\\
373	0.0111622399356427\\
374	0.0111647668097724\\
375	0.0111673315913599\\
376	0.011169934515205\\
377	0.0111725757914596\\
378	0.0111752556031271\\
379	0.0111779741033681\\
380	0.0111807314126009\\
381	0.0111835276153863\\
382	0.0111863627570851\\
383	0.0111892368402792\\
384	0.011192149820946\\
385	0.0111951016043811\\
386	0.0111980920408634\\
387	0.0112011209210616\\
388	0.011204187971186\\
389	0.0112072928478934\\
390	0.0112104351329604\\
391	0.0112136143277499\\
392	0.0112168298475046\\
393	0.0112200810155127\\
394	0.0112233670572052\\
395	0.0112266870942536\\
396	0.0112300401387341\\
397	0.0112334250873728\\
398	0.0112368407176134\\
399	0.0112402856833753\\
400	0.0112437585104393\\
401	0.0112472575939932\\
402	0.0112507811976449\\
403	0.0112543274542802\\
404	0.0112578943691687\\
405	0.0112614798256961\\
406	0.0112650815939077\\
407	0.0112686973413479\\
408	0.011272324643515\\
409	0.0112759609850413\\
410	0.0112796037263457\\
411	0.0112832499676989\\
412	0.011286896203401\\
413	0.0112905378364156\\
414	0.0112940651081415\\
415	0.0112975687563616\\
416	0.0113011259956259\\
417	0.0113047373442405\\
418	0.0113084033125058\\
419	0.0113121244028058\\
420	0.0113159011126722\\
421	0.0113197339371375\\
422	0.0113236233494445\\
423	0.0113275697786033\\
424	0.0113315735650039\\
425	0.0113356350264169\\
426	0.0113397544571508\\
427	0.0113439321274055\\
428	0.011348168282849\\
429	0.0113524631444036\\
430	0.0113568169082831\\
431	0.0113612297463309\\
432	0.01136570180671\\
433	0.0113702332149925\\
434	0.0113748240757068\\
435	0.0113794744744569\\
436	0.0113841844806643\\
437	0.0113889541510194\\
438	0.0113937835337255\\
439	0.0113986726736303\\
440	0.0114036216183354\\
441	0.011408630425375\\
442	0.0114136991705488\\
443	0.0114188279574836\\
444	0.0114240169284743\\
445	0.0114292662766282\\
446	0.0114345762592995\\
447	0.0114399472127696\\
448	0.0114453795679461\\
449	0.0114508738693692\\
450	0.0114564307972314\\
451	0.011462051191343\\
452	0.0114677360767602\\
453	0.0114734866915618\\
454	0.0114793045164547\\
455	0.0114851913066866\\
456	0.0114911491170186\\
457	0.0114971803174441\\
458	0.0115032875961655\\
459	0.0115094739718772\\
460	0.0115157429445744\\
461	0.0115220991578076\\
462	0.0115285602767802\\
463	0.0115351293988298\\
464	0.0115418084414028\\
465	0.0115485994530532\\
466	0.0115555045519506\\
467	0.0115625259248113\\
468	0.011569665826814\\
469	0.011576926580244\\
470	0.0115843105715111\\
471	0.0115918202461318\\
472	0.0115994581011852\\
473	0.0116072266746456\\
474	0.0116151285310103\\
475	0.0116231662425808\\
476	0.0116313423658861\\
477	0.0116396594119471\\
478	0.0116481198015513\\
479	0.0116567257452577\\
480	0.0116654806395043\\
481	0.0116743888158184\\
482	0.0116834548673434\\
483	0.0116926836607282\\
484	0.0117020803479166\\
485	0.0117116503806736\\
486	0.0117213995366113\\
487	0.0117313339738432\\
488	0.0117414603097175\\
489	0.0117517855108233\\
490	0.0117623162204974\\
491	0.0117730600510751\\
492	0.0117846848635083\\
493	0.0117966360849784\\
494	0.0118087710848869\\
495	0.0118210886492517\\
496	0.0118335832452196\\
497	0.0118462250557034\\
498	0.0118590629481398\\
499	0.0118720983719465\\
500	0.0118853297638309\\
501	0.0118987538190686\\
502	0.0119123819322364\\
503	0.0119261386581738\\
504	0.0119394393424958\\
505	0.0119529683357346\\
506	0.0119667271448993\\
507	0.0119807171131835\\
508	0.0119949393507152\\
509	0.0120093948705986\\
510	0.0120240846035681\\
511	0.0120390087729354\\
512	0.012054166513775\\
513	0.0120695536440546\\
514	0.0120851450454524\\
515	0.0121008147449223\\
516	0.0121167319980734\\
517	0.0121328897458616\\
518	0.0121492694316621\\
519	0.0121658504724211\\
520	0.0121828274656008\\
521	0.012200391844306\\
522	0.0122185534629314\\
523	0.0122373639516898\\
524	0.0122568861919771\\
525	0.0122771432756959\\
526	0.0122981280436679\\
527	0.0123200037486382\\
528	0.0123441135063749\\
529	0.0123693950649721\\
530	0.0124083258512722\\
531	0.0124468460085786\\
532	0.0124846674899516\\
533	0.0125213967902214\\
534	0.0125483389305038\\
535	0.0125700394526752\\
536	0.0125907092933434\\
537	0.0126101954873346\\
538	0.0126280281792202\\
539	0.0126433950276803\\
540	0.0126584046595101\\
541	0.012673295349401\\
542	0.0126881234894621\\
543	0.0127030360854984\\
544	0.0127181359679043\\
545	0.0127334484122643\\
546	0.0127490011044552\\
547	0.0127648170780659\\
548	0.0127809056773097\\
549	0.0127972861353565\\
550	0.0128141964225559\\
551	0.0128329970503251\\
552	0.0128516714703716\\
553	0.0128700447187205\\
554	0.0128868360530066\\
555	0.0129036699683316\\
556	0.0129205718181761\\
557	0.0129372904245884\\
558	0.0129541842441257\\
559	0.012971274017061\\
560	0.0129885620297398\\
561	0.0130060470272279\\
562	0.0130237210647826\\
563	0.0130415890533139\\
564	0.0130609026410381\\
565	0.013079927001214\\
566	0.0130973634653552\\
567	0.0131147325999733\\
568	0.0131318544672218\\
569	0.0131490561835143\\
570	0.0131663382726025\\
571	0.0131836936800372\\
572	0.013201786544463\\
573	0.0132191763080944\\
574	0.0132360655357439\\
575	0.0132527843975138\\
576	0.0132695026506452\\
577	0.0132862082766414\\
578	0.0133028812714239\\
579	0.0133194993844997\\
580	0.0133360381907542\\
581	0.0133524709020724\\
582	0.0133687681411751\\
583	0.0133848976891148\\
584	0.0134008242103469\\
585	0.013416508968873\\
586	0.0134319095789007\\
587	0.0134469799156316\\
588	0.0134616705332227\\
589	0.0134759305304731\\
590	0.0134902483101722\\
591	0.0135046714122469\\
592	0.0135192196273521\\
593	0.0135339766714342\\
594	0.0135491998980231\\
595	0.0135656125153459\\
596	0.0135851815686143\\
597	0.0136131889078992\\
598	0.0136637438596031\\
599	0\\
600	0\\
};
\addplot [color=red,solid,forget plot]
  table[row sep=crcr]{%
1	0.0111473102704322\\
2	0.0111473127269773\\
3	0.0111473152279834\\
4	0.0111473177742563\\
5	0.011147320366616\\
6	0.0111473230058974\\
7	0.0111473256929507\\
8	0.0111473284286414\\
9	0.0111473312138509\\
10	0.0111473340494763\\
11	0.0111473369364313\\
12	0.0111473398756458\\
13	0.0111473428680668\\
14	0.0111473459146586\\
15	0.0111473490164028\\
16	0.0111473521742988\\
17	0.0111473553893642\\
18	0.011147358662635\\
19	0.0111473619951663\\
20	0.0111473653880319\\
21	0.0111473688423254\\
22	0.0111473723591601\\
23	0.0111473759396695\\
24	0.0111473795850078\\
25	0.0111473832963501\\
26	0.0111473870748927\\
27	0.0111473909218538\\
28	0.0111473948384736\\
29	0.0111473988260148\\
30	0.011147402885763\\
31	0.0111474070190274\\
32	0.0111474112271406\\
33	0.0111474155114596\\
34	0.0111474198733659\\
35	0.0111474243142664\\
36	0.0111474288355931\\
37	0.0111474334388043\\
38	0.0111474381253846\\
39	0.0111474428968458\\
40	0.011147447754727\\
41	0.0111474527005951\\
42	0.0111474577360457\\
43	0.0111474628627032\\
44	0.0111474680822217\\
45	0.0111474733962852\\
46	0.0111474788066081\\
47	0.0111474843149364\\
48	0.0111474899230474\\
49	0.0111474956327508\\
50	0.0111475014458893\\
51	0.0111475073643389\\
52	0.0111475133900098\\
53	0.0111475195248467\\
54	0.0111475257708299\\
55	0.0111475321299756\\
56	0.0111475386043364\\
57	0.0111475451960024\\
58	0.0111475519071017\\
59	0.0111475587398009\\
60	0.0111475656963062\\
61	0.0111475727788636\\
62	0.0111475799897601\\
63	0.0111475873313242\\
64	0.0111475948059267\\
65	0.0111476024159815\\
66	0.0111476101639464\\
67	0.0111476180523238\\
68	0.0111476260836616\\
69	0.0111476342605539\\
70	0.0111476425856422\\
71	0.0111476510616157\\
72	0.0111476596912128\\
73	0.0111476684772213\\
74	0.01114767742248\\
75	0.0111476865298791\\
76	0.0111476958023613\\
77	0.0111477052429231\\
78	0.0111477148546151\\
79	0.0111477246405436\\
80	0.0111477346038711\\
81	0.0111477447478179\\
82	0.0111477550756627\\
83	0.0111477655907437\\
84	0.01114777629646\\
85	0.0111477871962723\\
86	0.0111477982937044\\
87	0.0111478095923439\\
88	0.011147821095844\\
89	0.011147832807924\\
90	0.011147844732371\\
91	0.0111478568730408\\
92	0.0111478692338595\\
93	0.0111478818188245\\
94	0.0111478946320058\\
95	0.0111479076775476\\
96	0.0111479209596692\\
97	0.0111479344826669\\
98	0.0111479482509149\\
99	0.0111479622688672\\
100	0.0111479765410584\\
101	0.0111479910721059\\
102	0.0111480058667111\\
103	0.0111480209296605\\
104	0.0111480362658281\\
105	0.0111480518801763\\
106	0.0111480677777578\\
107	0.0111480839637171\\
108	0.0111481004432924\\
109	0.0111481172218171\\
110	0.0111481343047216\\
111	0.0111481516975351\\
112	0.0111481694058872\\
113	0.0111481874355101\\
114	0.0111482057922401\\
115	0.0111482244820196\\
116	0.0111482435108994\\
117	0.0111482628850398\\
118	0.0111482826107137\\
119	0.0111483026943076\\
120	0.0111483231423245\\
121	0.0111483439613854\\
122	0.011148365158232\\
123	0.0111483867397282\\
124	0.0111484087128631\\
125	0.0111484310847527\\
126	0.0111484538626424\\
127	0.0111484770539095\\
128	0.0111485006660653\\
129	0.0111485247067576\\
130	0.0111485491837736\\
131	0.0111485741050417\\
132	0.0111485994786347\\
133	0.0111486253127721\\
134	0.0111486516158229\\
135	0.011148678396308\\
136	0.0111487056629036\\
137	0.0111487334244433\\
138	0.0111487616899214\\
139	0.0111487904684957\\
140	0.0111488197694905\\
141	0.0111488496023995\\
142	0.0111488799768889\\
143	0.0111489109028007\\
144	0.0111489423901557\\
145	0.0111489744491566\\
146	0.0111490070901919\\
147	0.0111490403238385\\
148	0.0111490741608656\\
149	0.011149108612238\\
150	0.0111491436891196\\
151	0.0111491794028774\\
152	0.0111492157650844\\
153	0.0111492527875241\\
154	0.0111492904821938\\
155	0.0111493288613088\\
156	0.011149367937306\\
157	0.0111494077228482\\
158	0.0111494482308281\\
159	0.0111494894743721\\
160	0.0111495314668453\\
161	0.0111495742218549\\
162	0.0111496177532551\\
163	0.0111496620751515\\
164	0.0111497072019056\\
165	0.0111497531481392\\
166	0.0111497999287394\\
167	0.011149847558863\\
168	0.0111498960539419\\
169	0.0111499454296874\\
170	0.0111499957020957\\
171	0.0111500468874528\\
172	0.0111500990023396\\
173	0.0111501520636376\\
174	0.0111502060885339\\
175	0.0111502610945266\\
176	0.0111503170994307\\
177	0.0111503741213839\\
178	0.0111504321788515\\
179	0.0111504912906334\\
180	0.0111505514758692\\
181	0.0111506127540448\\
182	0.0111506751449981\\
183	0.0111507386689257\\
184	0.0111508033463892\\
185	0.0111508691983215\\
186	0.0111509362460335\\
187	0.0111510045112209\\
188	0.0111510740159709\\
189	0.0111511447827693\\
190	0.0111512168345074\\
191	0.0111512901944894\\
192	0.0111513648864393\\
193	0.0111514409345088\\
194	0.0111515183632845\\
195	0.0111515971977959\\
196	0.0111516774635228\\
197	0.0111517591864036\\
198	0.0111518423928429\\
199	0.0111519271097205\\
200	0.0111520133643988\\
201	0.0111521011847319\\
202	0.0111521905990739\\
203	0.0111522816362877\\
204	0.0111523743257542\\
205	0.0111524686973807\\
206	0.0111525647816107\\
207	0.0111526626094328\\
208	0.0111527622123903\\
209	0.0111528636225909\\
210	0.0111529668727165\\
211	0.0111530719960329\\
212	0.0111531790264\\
213	0.0111532879982823\\
214	0.0111533989467587\\
215	0.0111535119075338\\
216	0.0111536269169478\\
217	0.0111537440119883\\
218	0.0111538632303006\\
219	0.0111539846101992\\
220	0.0111541081906792\\
221	0.011154234011428\\
222	0.0111543621128366\\
223	0.0111544925360121\\
224	0.0111546253227891\\
225	0.0111547605157427\\
226	0.0111548981582002\\
227	0.0111550382942542\\
228	0.0111551809687753\\
229	0.0111553262274249\\
230	0.0111554741166686\\
231	0.0111556246837893\\
232	0.0111557779769008\\
233	0.0111559340449616\\
234	0.0111560929377888\\
235	0.011156254706072\\
236	0.0111564194013876\\
237	0.0111565870762135\\
238	0.0111567577839435\\
239	0.0111569315789022\\
240	0.01115710851636\\
241	0.0111572886525482\\
242	0.0111574720446747\\
243	0.0111576587509395\\
244	0.0111578488305501\\
245	0.0111580423437379\\
246	0.0111582393517742\\
247	0.0111584399169865\\
248	0.0111586441027751\\
249	0.0111588519736297\\
250	0.0111590635951465\\
251	0.0111592790340448\\
252	0.0111594983581851\\
253	0.0111597216365856\\
254	0.0111599489394404\\
255	0.0111601803381372\\
256	0.0111604159052752\\
257	0.0111606557146831\\
258	0.0111608998414377\\
259	0.0111611483618822\\
260	0.0111614013536447\\
261	0.0111616588956573\\
262	0.011161921068175\\
263	0.0111621879527944\\
264	0.0111624596324735\\
265	0.0111627361915508\\
266	0.0111630177157649\\
267	0.0111633042922742\\
268	0.0111635960096764\\
269	0.0111638929580291\\
270	0.0111641952288691\\
271	0.0111645029152331\\
272	0.0111648161116779\\
273	0.0111651349143002\\
274	0.0111654594207581\\
275	0.0111657897302906\\
276	0.0111661259437391\\
277	0.0111664681635675\\
278	0.0111668164938833\\
279	0.0111671710404585\\
280	0.0111675319107503\\
281	0.0111678992139222\\
282	0.0111682730608647\\
283	0.0111686535642168\\
284	0.0111690408383868\\
285	0.0111694349995731\\
286	0.0111698361657856\\
287	0.0111702444568667\\
288	0.011170659994512\\
289	0.0111710829022917\\
290	0.0111715133056715\\
291	0.0111719513320332\\
292	0.011172397110696\\
293	0.0111728507729371\\
294	0.0111733124520122\\
295	0.0111737822831767\\
296	0.0111742604037057\\
297	0.0111747469529147\\
298	0.0111752420721799\\
299	0.0111757459049585\\
300	0.0111762585968084\\
301	0.0111767802954084\\
302	0.0111773111505778\\
303	0.011177851314296\\
304	0.0111784009407219\\
305	0.0111789601862125\\
306	0.0111795292093429\\
307	0.0111801081709236\\
308	0.0111806972340199\\
309	0.0111812965639697\\
310	0.0111819063284009\\
311	0.0111825266972494\\
312	0.0111831578427759\\
313	0.0111837999395826\\
314	0.0111844531646293\\
315	0.0111851176972496\\
316	0.0111857937191659\\
317	0.0111864814145041\\
318	0.011187180969808\\
319	0.0111878925740528\\
320	0.0111886164186577\\
321	0.0111893526974983\\
322	0.011190101606918\\
323	0.0111908633457379\\
324	0.0111916381152674\\
325	0.011192426119312\\
326	0.0111932275641813\\
327	0.0111940426586961\\
328	0.0111948716141934\\
329	0.0111957146445319\\
330	0.0111965719660951\\
331	0.0111974437977952\\
332	0.0111983303610746\\
333	0.0111992318799085\\
334	0.0112001485808067\\
335	0.011201080692816\\
336	0.0112020284475241\\
337	0.0112029920790656\\
338	0.0112039718241326\\
339	0.0112049679219903\\
340	0.0112059806145044\\
341	0.0112070101461895\\
342	0.011208056764326\\
343	0.0112091207193404\\
344	0.0112102022661628\\
345	0.0112113016661054\\
346	0.0112124191835345\\
347	0.0112135550858418\\
348	0.0112147096433907\\
349	0.0112158831294242\\
350	0.0112170758199153\\
351	0.0112182879933285\\
352	0.011219519930215\\
353	0.0112207719124187\\
354	0.0112220442211199\\
355	0.0112233371311262\\
356	0.0112246508941022\\
357	0.0112259857587908\\
358	0.0112273419723362\\
359	0.0112287197799926\\
360	0.0112301194248212\\
361	0.0112315411473759\\
362	0.0112329851853761\\
363	0.01123445177337\\
364	0.0112359411423863\\
365	0.0112374535195782\\
366	0.0112389891278582\\
367	0.0112405481855274\\
368	0.0112421309058998\\
369	0.0112437374969252\\
370	0.0112453681608113\\
371	0.0112470230936513\\
372	0.0112487024850573\\
373	0.0112504065178066\\
374	0.0112521353675051\\
375	0.011253889202273\\
376	0.0112556681824605\\
377	0.0112574724604017\\
378	0.0112593021802138\\
379	0.0112611574776544\\
380	0.0112630384800454\\
381	0.0112649453062801\\
382	0.0112668780669247\\
383	0.0112688368644336\\
384	0.0112708217934956\\
385	0.0112728329415323\\
386	0.011274870389372\\
387	0.0112769342121245\\
388	0.0112790244802867\\
389	0.0112811412611087\\
390	0.0112832846202571\\
391	0.0112854546238134\\
392	0.0112876513406494\\
393	0.0112898748452273\\
394	0.0112921252208737\\
395	0.0112944025635879\\
396	0.0112967069864456\\
397	0.0112990386246729\\
398	0.0113013976414172\\
399	0.0113037842343011\\
400	0.0113061986428035\\
401	0.0113086411564646\\
402	0.0113111121239322\\
403	0.0113136119628371\\
404	0.0113161411704452\\
405	0.0113187003349482\\
406	0.0113212901470929\\
407	0.0113239114115065\\
408	0.0113265650564488\\
409	0.0113292521398623\\
410	0.0113319738499135\\
411	0.0113347315064679\\
412	0.0113375266058579\\
413	0.0113403610395717\\
414	0.0113432408659266\\
415	0.0113461693643335\\
416	0.0113491472737491\\
417	0.0113521753452866\\
418	0.0113552543429474\\
419	0.0113583850444081\\
420	0.0113615682417523\\
421	0.0113648047420469\\
422	0.0113680953682887\\
423	0.0113714409607659\\
424	0.0113748423798136\\
425	0.0113783005065397\\
426	0.0113818162431911\\
427	0.011385390512941\\
428	0.0113890242600531\\
429	0.0113927184512245\\
430	0.0113964740770638\\
431	0.0114002921537191\\
432	0.011404173724678\\
433	0.0114081198627595\\
434	0.0114121316723295\\
435	0.0114162102917705\\
436	0.0114203568962482\\
437	0.0114245727008251\\
438	0.0114288589639846\\
439	0.0114332169916449\\
440	0.0114376481417603\\
441	0.0114421538296371\\
442	0.0114467355341217\\
443	0.0114513948048625\\
444	0.0114561332708902\\
445	0.0114609526507881\\
446	0.0114658547646797\\
447	0.0114708415483316\\
448	0.0114759150724162\\
449	0.0114810775146525\\
450	0.0114863311410385\\
451	0.0114916783040837\\
452	0.0114971214393143\\
453	0.0115026630596356\\
454	0.0115083057470238\\
455	0.0115140521407895\\
456	0.0115199049220752\\
457	0.0115258667950607\\
458	0.0115319404682885\\
459	0.0115381286445818\\
460	0.011544434024117\\
461	0.0115508592630822\\
462	0.0115574063440456\\
463	0.0115640770860273\\
464	0.0115708731461204\\
465	0.0115777958834864\\
466	0.0115848482998898\\
467	0.0115920335970919\\
468	0.0115993551458568\\
469	0.0116068165037789\\
470	0.0116144214354527\\
471	0.0116221739351382\\
472	0.0116300782532637\\
473	0.0116381389273681\\
474	0.0116463608188084\\
475	0.0116547491591382\\
476	0.0116633096165548\\
477	0.011672048420183\\
478	0.0116809726580689\\
479	0.0116905204863379\\
480	0.0117005044269762\\
481	0.0117106532130661\\
482	0.0117209681075114\\
483	0.0117314501600944\\
484	0.0117421001737969\\
485	0.0117529186690691\\
486	0.0117639058492852\\
487	0.0117750615454548\\
488	0.0117863852481278\\
489	0.0117978764165742\\
490	0.0118095363454837\\
491	0.0118213717483199\\
492	0.0118328210307557\\
493	0.0118443243259626\\
494	0.0118560051910933\\
495	0.0118678939000001\\
496	0.0118799920426209\\
497	0.0118922968115098\\
498	0.0119048192289019\\
499	0.0119175617424726\\
500	0.0119305261740929\\
501	0.0119437134999593\\
502	0.011957121494349\\
503	0.0119707250869657\\
504	0.0119844175624323\\
505	0.0119983501189346\\
506	0.012012523925746\\
507	0.0120269399018215\\
508	0.0120415986912665\\
509	0.0120565005861726\\
510	0.0120716454611987\\
511	0.0120870327709694\\
512	0.012102662138615\\
513	0.0121185333071197\\
514	0.0121346391863825\\
515	0.0121509646394995\\
516	0.0121674846253816\\
517	0.0121843730191093\\
518	0.0122018418940742\\
519	0.0122199093285341\\
520	0.0122386373285286\\
521	0.0122580808793175\\
522	0.0122795220208374\\
523	0.0123016310832037\\
524	0.012324565846525\\
525	0.0123504689525452\\
526	0.0123881498461421\\
527	0.012424800444479\\
528	0.0124595768080078\\
529	0.0124933976748602\\
530	0.012514740800854\\
531	0.012534652282812\\
532	0.0125534699405362\\
533	0.0125709809393902\\
534	0.0125859580147234\\
535	0.012599834001914\\
536	0.0126135773576437\\
537	0.0126272341096508\\
538	0.0126408953663348\\
539	0.012654733078678\\
540	0.0126687744623121\\
541	0.0126830426454391\\
542	0.0126975607232427\\
543	0.012712343339761\\
544	0.0127273981388023\\
545	0.0127427298886337\\
546	0.0127583537406449\\
547	0.0127743389760305\\
548	0.0127924266957907\\
549	0.0128103931501383\\
550	0.0128280275808347\\
551	0.0128441396444115\\
552	0.0128603480007628\\
553	0.0128766037615932\\
554	0.0128926685840816\\
555	0.0129088943648097\\
556	0.0129253240933278\\
557	0.0129419662203101\\
558	0.0129588173154488\\
559	0.012975873011209\\
560	0.0129931272774625\\
561	0.0130105799293654\\
562	0.0130291935362831\\
563	0.013048015192612\\
564	0.0130652952278806\\
565	0.0130823786395838\\
566	0.0130992303318434\\
567	0.0131161823503507\\
568	0.0131332384982756\\
569	0.013150388897885\\
570	0.0131678002979293\\
571	0.0131857189375081\\
572	0.0132026379726609\\
573	0.0132193845885735\\
574	0.0132360735367549\\
575	0.0132527850905522\\
576	0.0132695028713411\\
577	0.0132862083899486\\
578	0.0133028813326694\\
579	0.0133194994167125\\
580	0.013336038206793\\
581	0.0133524709094905\\
582	0.013368768144301\\
583	0.0133848976902849\\
584	0.0134008242107216\\
585	0.0134165089689696\\
586	0.0134319095789186\\
587	0.0134469799156334\\
588	0.0134616705332227\\
589	0.0134759305304731\\
590	0.0134902483101722\\
591	0.0135046714122469\\
592	0.0135192196273521\\
593	0.0135339766714342\\
594	0.0135491998980231\\
595	0.0135656125153459\\
596	0.0135851815686143\\
597	0.0136131889078992\\
598	0.0136637438596031\\
599	0\\
600	0\\
};
\addplot [color=mycolor20,solid,forget plot]
  table[row sep=crcr]{%
1	0.0112011782251379\\
2	0.0112011799967281\\
3	0.0112011818005927\\
4	0.0112011836373212\\
5	0.0112011855075136\\
6	0.0112011874117807\\
7	0.0112011893507449\\
8	0.0112011913250396\\
9	0.0112011933353101\\
10	0.0112011953822134\\
11	0.0112011974664184\\
12	0.0112011995886065\\
13	0.0112012017494714\\
14	0.0112012039497195\\
15	0.0112012061900702\\
16	0.0112012084712562\\
17	0.0112012107940234\\
18	0.0112012131591314\\
19	0.0112012155673539\\
20	0.0112012180194785\\
21	0.0112012205163075\\
22	0.0112012230586577\\
23	0.011201225647361\\
24	0.0112012282832644\\
25	0.0112012309672307\\
26	0.0112012337001381\\
27	0.0112012364828813\\
28	0.0112012393163712\\
29	0.0112012422015354\\
30	0.0112012451393184\\
31	0.0112012481306824\\
32	0.0112012511766068\\
33	0.0112012542780892\\
34	0.0112012574361455\\
35	0.01120126065181\\
36	0.0112012639261364\\
37	0.0112012672601973\\
38	0.0112012706550852\\
39	0.0112012741119125\\
40	0.0112012776318122\\
41	0.0112012812159379\\
42	0.0112012848654645\\
43	0.0112012885815884\\
44	0.0112012923655279\\
45	0.0112012962185238\\
46	0.0112013001418395\\
47	0.0112013041367618\\
48	0.0112013082046008\\
49	0.011201312346691\\
50	0.0112013165643911\\
51	0.0112013208590851\\
52	0.011201325232182\\
53	0.011201329685117\\
54	0.0112013342193514\\
55	0.0112013388363736\\
56	0.0112013435376991\\
57	0.0112013483248713\\
58	0.0112013531994621\\
59	0.011201358163072\\
60	0.0112013632173312\\
61	0.0112013683638996\\
62	0.0112013736044677\\
63	0.0112013789407572\\
64	0.0112013843745212\\
65	0.0112013899075452\\
66	0.0112013955416473\\
67	0.0112014012786793\\
68	0.0112014071205268\\
69	0.0112014130691102\\
70	0.0112014191263851\\
71	0.0112014252943432\\
72	0.0112014315750128\\
73	0.0112014379704592\\
74	0.0112014444827862\\
75	0.0112014511141358\\
76	0.0112014578666897\\
77	0.0112014647426696\\
78	0.0112014717443381\\
79	0.0112014788739994\\
80	0.0112014861340002\\
81	0.0112014935267303\\
82	0.0112015010546234\\
83	0.0112015087201581\\
84	0.0112015165258585\\
85	0.0112015244742953\\
86	0.0112015325680864\\
87	0.011201540809898\\
88	0.0112015492024453\\
89	0.0112015577484934\\
90	0.0112015664508586\\
91	0.0112015753124087\\
92	0.0112015843360645\\
93	0.0112015935248007\\
94	0.0112016028816466\\
95	0.0112016124096873\\
96	0.0112016221120649\\
97	0.0112016319919791\\
98	0.0112016420526888\\
99	0.0112016522975129\\
100	0.0112016627298313\\
101	0.0112016733530864\\
102	0.0112016841707839\\
103	0.0112016951864941\\
104	0.0112017064038532\\
105	0.0112017178265644\\
106	0.011201729458399\\
107	0.0112017413031981\\
108	0.0112017533648734\\
109	0.0112017656474089\\
110	0.0112017781548618\\
111	0.0112017908913645\\
112	0.0112018038611252\\
113	0.0112018170684301\\
114	0.0112018305176441\\
115	0.0112018442132128\\
116	0.0112018581596639\\
117	0.0112018723616084\\
118	0.0112018868237427\\
119	0.0112019015508494\\
120	0.0112019165477998\\
121	0.0112019318195548\\
122	0.0112019473711671\\
123	0.0112019632077823\\
124	0.0112019793346412\\
125	0.0112019957570816\\
126	0.0112020124805392\\
127	0.0112020295105507\\
128	0.0112020468527546\\
129	0.0112020645128936\\
130	0.0112020824968166\\
131	0.0112021008104803\\
132	0.0112021194599515\\
133	0.0112021384514091\\
134	0.0112021577911459\\
135	0.0112021774855712\\
136	0.0112021975412126\\
137	0.0112022179647182\\
138	0.0112022387628589\\
139	0.0112022599425309\\
140	0.0112022815107575\\
141	0.011202303474692\\
142	0.0112023258416198\\
143	0.0112023486189609\\
144	0.0112023718142725\\
145	0.0112023954352513\\
146	0.0112024194897364\\
147	0.0112024439857117\\
148	0.0112024689313086\\
149	0.0112024943348089\\
150	0.0112025202046474\\
151	0.0112025465494149\\
152	0.0112025733778609\\
153	0.0112026006988967\\
154	0.0112026285215983\\
155	0.0112026568552095\\
156	0.0112026857091449\\
157	0.0112027150929934\\
158	0.0112027450165208\\
159	0.0112027754896737\\
160	0.0112028065225827\\
161	0.0112028381255654\\
162	0.0112028703091302\\
163	0.0112029030839801\\
164	0.0112029364610156\\
165	0.011202970451339\\
166	0.0112030050662577\\
167	0.0112030403172881\\
168	0.0112030762161599\\
169	0.0112031127748191\\
170	0.011203150005433\\
171	0.0112031879203936\\
172	0.0112032265323222\\
173	0.0112032658540731\\
174	0.0112033058987385\\
175	0.0112033466796524\\
176	0.0112033882103955\\
177	0.0112034305047993\\
178	0.0112034735769511\\
179	0.0112035174411983\\
180	0.0112035621121538\\
181	0.0112036076047002\\
182	0.0112036539339954\\
183	0.0112037011154769\\
184	0.0112037491648679\\
185	0.0112037980981818\\
186	0.0112038479317277\\
187	0.0112038986821162\\
188	0.0112039503662644\\
189	0.0112040030014018\\
190	0.0112040566050762\\
191	0.0112041111951591\\
192	0.011204166789852\\
193	0.0112042234076924\\
194	0.0112042810675595\\
195	0.0112043397886813\\
196	0.0112043995906402\\
197	0.0112044604933797\\
198	0.0112045225172113\\
199	0.0112045856828209\\
200	0.0112046500112754\\
201	0.0112047155240304\\
202	0.0112047822429364\\
203	0.0112048501902468\\
204	0.0112049193886243\\
205	0.0112049898611493\\
206	0.0112050616313269\\
207	0.0112051347230945\\
208	0.0112052091608301\\
209	0.01120528496936\\
210	0.0112053621739668\\
211	0.0112054408003978\\
212	0.0112055208748735\\
213	0.0112056024240956\\
214	0.0112056854752563\\
215	0.0112057700560469\\
216	0.0112058561946664\\
217	0.0112059439198311\\
218	0.0112060332607837\\
219	0.0112061242473024\\
220	0.0112062169097111\\
221	0.0112063112788883\\
222	0.0112064073862779\\
223	0.0112065052638984\\
224	0.0112066049443535\\
225	0.0112067064608427\\
226	0.0112068098471713\\
227	0.0112069151377614\\
228	0.0112070223676628\\
229	0.0112071315725639\\
230	0.011207242788803\\
231	0.0112073560533795\\
232	0.0112074714039659\\
233	0.0112075888789189\\
234	0.0112077085172918\\
235	0.0112078303588465\\
236	0.0112079544440655\\
237	0.0112080808141649\\
238	0.0112082095111062\\
239	0.01120834057761\\
240	0.0112084740571685\\
241	0.0112086099940586\\
242	0.0112087484333557\\
243	0.011208889420947\\
244	0.011209033003545\\
245	0.0112091792287022\\
246	0.0112093281448242\\
247	0.0112094798011848\\
248	0.0112096342479404\\
249	0.0112097915361442\\
250	0.0112099517177615\\
251	0.0112101148456848\\
252	0.0112102809737489\\
253	0.0112104501567461\\
254	0.0112106224504424\\
255	0.0112107979115929\\
256	0.0112109765979577\\
257	0.0112111585683184\\
258	0.0112113438824943\\
259	0.011211532601359\\
260	0.0112117247868568\\
261	0.0112119205020201\\
262	0.0112121198109859\\
263	0.0112123227790136\\
264	0.0112125294725019\\
265	0.0112127399590064\\
266	0.0112129543072578\\
267	0.0112131725871792\\
268	0.0112133948699043\\
269	0.0112136212277956\\
270	0.0112138517344629\\
271	0.0112140864647814\\
272	0.0112143254949105\\
273	0.0112145689023124\\
274	0.0112148167657712\\
275	0.0112150691654116\\
276	0.0112153261827181\\
277	0.0112155879005541\\
278	0.0112158544031813\\
279	0.0112161257762791\\
280	0.0112164021069643\\
281	0.0112166834838102\\
282	0.0112169699968666\\
283	0.0112172617376797\\
284	0.0112175587993117\\
285	0.0112178612763608\\
286	0.0112181692649812\\
287	0.0112184828629032\\
288	0.0112188021694534\\
289	0.0112191272855747\\
290	0.0112194583138468\\
291	0.0112197953585066\\
292	0.0112201385254683\\
293	0.0112204879223442\\
294	0.0112208436584652\\
295	0.0112212058449017\\
296	0.011221574594484\\
297	0.0112219500218237\\
298	0.0112223322433345\\
299	0.0112227213772532\\
300	0.0112231175436618\\
301	0.0112235208645085\\
302	0.0112239314636297\\
303	0.0112243494667723\\
304	0.0112247750016162\\
305	0.0112252081977967\\
306	0.0112256491869282\\
307	0.0112260981026274\\
308	0.011226555080538\\
309	0.0112270202583552\\
310	0.0112274937758511\\
311	0.0112279757749011\\
312	0.0112284663995106\\
313	0.0112289657958432\\
314	0.0112294741122495\\
315	0.011229991499297\\
316	0.0112305181098013\\
317	0.0112310540988585\\
318	0.011231599623879\\
319	0.0112321548446228\\
320	0.0112327199232361\\
321	0.0112332950242897\\
322	0.0112338803148189\\
323	0.0112344759643651\\
324	0.0112350821450188\\
325	0.0112356990314649\\
326	0.011236326801028\\
327	0.0112369656337208\\
328	0.0112376157122921\\
329	0.011238277222276\\
330	0.0112389503520416\\
331	0.0112396352928422\\
332	0.0112403322388634\\
333	0.0112410413872696\\
334	0.0112417629382481\\
335	0.0112424970950486\\
336	0.0112432440640184\\
337	0.0112440040546304\\
338	0.0112447772795042\\
339	0.0112455639544163\\
340	0.0112463642982997\\
341	0.0112471785332304\\
342	0.0112480068844006\\
343	0.0112488495800786\\
344	0.0112497068515277\\
345	0.0112505789328555\\
346	0.0112514660609589\\
347	0.0112523684754778\\
348	0.011253286418743\\
349	0.0112542201357159\\
350	0.0112551698739232\\
351	0.0112561358833897\\
352	0.0112571184165753\\
353	0.0112581177283186\\
354	0.0112591340757955\\
355	0.0112601677185754\\
356	0.0112612189192329\\
357	0.0112622879434293\\
358	0.0112633750599579\\
359	0.0112644805407952\\
360	0.0112656046611613\\
361	0.0112667476995882\\
362	0.0112679099379983\\
363	0.0112690916617956\\
364	0.0112702931599693\\
365	0.0112715147252129\\
366	0.0112727566540603\\
367	0.0112740192470409\\
368	0.0112753028088574\\
369	0.0112766076485859\\
370	0.0112779340799054\\
371	0.0112792824213557\\
372	0.0112806529966303\\
373	0.0112820461349056\\
374	0.0112834621712126\\
375	0.0112849014468532\\
376	0.0112863643098675\\
377	0.0112878511155551\\
378	0.0112893622270574\\
379	0.0112908980160046\\
380	0.0112924588632329\\
381	0.0112940451595784\\
382	0.0112956573067507\\
383	0.0112972957182934\\
384	0.0112989608206337\\
385	0.0113006530542256\\
386	0.0113023728747881\\
387	0.0113041207546382\\
388	0.0113058971841162\\
389	0.011307702673097\\
390	0.0113095377525747\\
391	0.0113114029763052\\
392	0.0113132989224773\\
393	0.0113152261953781\\
394	0.0113171854269967\\
395	0.0113191772784982\\
396	0.0113212024414711\\
397	0.0113232616388324\\
398	0.0113253556252256\\
399	0.0113274851871978\\
400	0.0113296511440915\\
401	0.0113318543487182\\
402	0.0113340956877552\\
403	0.0113363760818018\\
404	0.0113386964850165\\
405	0.0113410578842378\\
406	0.0113434612974713\\
407	0.0113459077716017\\
408	0.0113483983792447\\
409	0.0113509342149807\\
410	0.0113535163922435\\
411	0.011356146044051\\
412	0.0113588243302582\\
413	0.0113615524348826\\
414	0.0113643313799227\\
415	0.0113671621095503\\
416	0.0113700455847795\\
417	0.0113729827843173\\
418	0.0113759747057078\\
419	0.0113790223668585\\
420	0.0113821268080629\\
421	0.0113852890946732\\
422	0.0113885103206019\\
423	0.0113917916128862\\
424	0.0113951341375146\\
425	0.0113985391068362\\
426	0.0114020077888496\\
427	0.0114055415200318\\
428	0.0114091416976793\\
429	0.0114128097568815\\
430	0.0114165471707832\\
431	0.0114203554505699\\
432	0.0114242361451091\\
433	0.0114281908401637\\
434	0.0114322211570809\\
435	0.0114363287508342\\
436	0.0114405153072744\\
437	0.011444782539418\\
438	0.0114491321825672\\
439	0.0114535659880168\\
440	0.0114580857150578\\
441	0.0114626931209378\\
442	0.0114673899483952\\
443	0.011472177910368\\
444	0.0114770586715192\\
445	0.0114820338263178\\
446	0.0114871048729088\\
447	0.0114922731763596\\
448	0.011497539878388\\
449	0.011502906924413\\
450	0.0115083767907457\\
451	0.0115139520612988\\
452	0.0115196354368832\\
453	0.0115254297460682\\
454	0.0115313379580668\\
455	0.0115373631984863\\
456	0.0115435087673774\\
457	0.011549778161005\\
458	0.0115561750983072\\
459	0.0115627035521213\\
460	0.0115693677827651\\
461	0.0115761723752888\\
462	0.0115831223321389\\
463	0.0115902232038116\\
464	0.0115974814175953\\
465	0.0116054693123334\\
466	0.0116136130143116\\
467	0.0116218957646072\\
468	0.011630318948227\\
469	0.01163888373421\\
470	0.0116475909885039\\
471	0.0116564415174651\\
472	0.0116654359411272\\
473	0.0116745746217214\\
474	0.0116838578127286\\
475	0.0116932858082149\\
476	0.0117028602097702\\
477	0.0117125822800973\\
478	0.0117224579922088\\
479	0.0117321264017504\\
480	0.0117417390738246\\
481	0.0117515218576962\\
482	0.0117614769395935\\
483	0.0117716064679172\\
484	0.0117819125050376\\
485	0.0117923969951987\\
486	0.0118030617210243\\
487	0.0118139087702765\\
488	0.0118249385493135\\
489	0.0118361476498823\\
490	0.0118475167294026\\
491	0.0118590600272656\\
492	0.0118706883881438\\
493	0.0118825050948001\\
494	0.0118945395506284\\
495	0.011906799486869\\
496	0.0119192874847334\\
497	0.0119320059738172\\
498	0.0119449571486634\\
499	0.0119581429964553\\
500	0.0119715652296089\\
501	0.0119852251124589\\
502	0.0119991230552591\\
503	0.0120132587699048\\
504	0.0120276364674216\\
505	0.0120422563024704\\
506	0.0120571178099907\\
507	0.0120722197737664\\
508	0.012087560591793\\
509	0.0121031388393444\\
510	0.0121189534581114\\
511	0.0121350029211934\\
512	0.0121512724497392\\
513	0.0121677458213481\\
514	0.012184548696405\\
515	0.0122019601923641\\
516	0.012221190571426\\
517	0.0122409443092564\\
518	0.0122612294233469\\
519	0.0122821227877054\\
520	0.0123037993365533\\
521	0.0123292709005782\\
522	0.0123642235092747\\
523	0.012398281359588\\
524	0.0124314689725784\\
525	0.0124619690331083\\
526	0.012481401372793\\
527	0.0124997518719848\\
528	0.0125167592473171\\
529	0.0125325076729113\\
530	0.0125454955282533\\
531	0.0125582429988684\\
532	0.0125708758906019\\
533	0.012583449496072\\
534	0.0125961214399987\\
535	0.0126089786809454\\
536	0.0126220434902936\\
537	0.012635338574974\\
538	0.0126488834125194\\
539	0.0126626860399153\\
540	0.0126767537815583\\
541	0.012691092848305\\
542	0.0127057068913743\\
543	0.0127206055553765\\
544	0.0127358087999329\\
545	0.0127528558503585\\
546	0.0127701680106764\\
547	0.0127872922293651\\
548	0.0128027255822829\\
549	0.012818260263768\\
550	0.0128338478465422\\
551	0.0128492851458301\\
552	0.01286492608112\\
553	0.0128807601707231\\
554	0.0128967675656984\\
555	0.012912965406474\\
556	0.0129293765963599\\
557	0.012945997188559\\
558	0.0129628230671786\\
559	0.0129798480402323\\
560	0.0129975888140626\\
561	0.0130162293940542\\
562	0.0130335904484122\\
563	0.0130503947638634\\
564	0.0130669819616154\\
565	0.0130836539210182\\
566	0.0131004466716772\\
567	0.013117352028506\\
568	0.0131343604620953\\
569	0.0131517943041389\\
570	0.0131693600552189\\
571	0.0131861241189776\\
572	0.0132027353319319\\
573	0.0132193852557846\\
574	0.0132360736255491\\
575	0.0132527851257368\\
576	0.0132695028897663\\
577	0.013286208399693\\
578	0.0133028813376224\\
579	0.0133194994190823\\
580	0.0133360382078424\\
581	0.0133524709099125\\
582	0.0133687681444512\\
583	0.0133848976903304\\
584	0.0134008242107326\\
585	0.0134165089689715\\
586	0.0134319095789188\\
587	0.0134469799156334\\
588	0.0134616705332227\\
589	0.0134759305304731\\
590	0.0134902483101722\\
591	0.0135046714122469\\
592	0.0135192196273521\\
593	0.0135339766714342\\
594	0.0135491998980231\\
595	0.0135656125153459\\
596	0.0135851815686143\\
597	0.0136131889078992\\
598	0.0136637438596031\\
599	0\\
600	0\\
};
\addplot [color=mycolor21,solid,forget plot]
  table[row sep=crcr]{%
1	0.0112221845287511\\
2	0.0112221859756682\\
3	0.011222187449112\\
4	0.0112221889495701\\
5	0.0112221904775395\\
6	0.0112221920335261\\
7	0.0112221936180453\\
8	0.0112221952316218\\
9	0.0112221968747901\\
10	0.0112221985480948\\
11	0.0112222002520901\\
12	0.0112222019873409\\
13	0.0112222037544221\\
14	0.0112222055539196\\
15	0.0112222073864299\\
16	0.0112222092525606\\
17	0.0112222111529305\\
18	0.0112222130881697\\
19	0.0112222150589202\\
20	0.0112222170658357\\
21	0.0112222191095821\\
22	0.0112222211908374\\
23	0.0112222233102924\\
24	0.0112222254686505\\
25	0.011222227666628\\
26	0.0112222299049547\\
27	0.0112222321843738\\
28	0.0112222345056421\\
29	0.0112222368695307\\
30	0.0112222392768246\\
31	0.0112222417283235\\
32	0.0112222442248421\\
33	0.0112222467672098\\
34	0.0112222493562715\\
35	0.0112222519928878\\
36	0.0112222546779351\\
37	0.0112222574123062\\
38	0.0112222601969103\\
39	0.0112222630326735\\
40	0.0112222659205388\\
41	0.011222268861467\\
42	0.0112222718564366\\
43	0.011222274906444\\
44	0.0112222780125044\\
45	0.0112222811756515\\
46	0.0112222843969383\\
47	0.0112222876774373\\
48	0.0112222910182408\\
49	0.0112222944204615\\
50	0.0112222978852325\\
51	0.0112223014137081\\
52	0.0112223050070638\\
53	0.0112223086664971\\
54	0.0112223123932275\\
55	0.0112223161884973\\
56	0.0112223200535717\\
57	0.0112223239897394\\
58	0.011222327998313\\
59	0.0112223320806295\\
60	0.0112223362380505\\
61	0.0112223404719632\\
62	0.0112223447837803\\
63	0.0112223491749408\\
64	0.0112223536469102\\
65	0.0112223582011816\\
66	0.0112223628392754\\
67	0.0112223675627405\\
68	0.0112223723731545\\
69	0.0112223772721242\\
70	0.0112223822612863\\
71	0.011222387342308\\
72	0.0112223925168872\\
73	0.0112223977867536\\
74	0.011222403153669\\
75	0.0112224086194277\\
76	0.0112224141858577\\
77	0.0112224198548208\\
78	0.0112224256282132\\
79	0.0112224315079667\\
80	0.0112224374960489\\
81	0.0112224435944638\\
82	0.0112224498052529\\
83	0.0112224561304956\\
84	0.01122246257231\\
85	0.0112224691328534\\
86	0.0112224758143235\\
87	0.0112224826189588\\
88	0.0112224895490394\\
89	0.0112224966068878\\
90	0.01122250379487\\
91	0.0112225111153957\\
92	0.0112225185709197\\
93	0.0112225261639425\\
94	0.0112225338970109\\
95	0.0112225417727194\\
96	0.0112225497937108\\
97	0.0112225579626771\\
98	0.0112225662823604\\
99	0.011222574755554\\
100	0.0112225833851032\\
101	0.0112225921739063\\
102	0.0112226011249158\\
103	0.011222610241139\\
104	0.0112226195256394\\
105	0.0112226289815377\\
106	0.0112226386120126\\
107	0.0112226484203025\\
108	0.0112226584097057\\
109	0.0112226685835825\\
110	0.0112226789453555\\
111	0.0112226894985116\\
112	0.0112227002466025\\
113	0.0112227111932462\\
114	0.0112227223421285\\
115	0.0112227336970038\\
116	0.0112227452616968\\
117	0.0112227570401034\\
118	0.0112227690361926\\
119	0.0112227812540073\\
120	0.0112227936976662\\
121	0.0112228063713646\\
122	0.0112228192793767\\
123	0.0112228324260564\\
124	0.0112228458158388\\
125	0.0112228594532423\\
126	0.0112228733428697\\
127	0.0112228874894098\\
128	0.0112229018976392\\
129	0.0112229165724239\\
130	0.011222931518721\\
131	0.0112229467415804\\
132	0.0112229622461465\\
133	0.01122297803766\\
134	0.0112229941214598\\
135	0.0112230105029848\\
136	0.0112230271877759\\
137	0.0112230441814775\\
138	0.0112230614898401\\
139	0.0112230791187218\\
140	0.0112230970740908\\
141	0.0112231153620267\\
142	0.0112231339887236\\
143	0.0112231529604914\\
144	0.0112231722837587\\
145	0.0112231919650744\\
146	0.0112232120111105\\
147	0.0112232324286641\\
148	0.01122325322466\\
149	0.0112232744061528\\
150	0.0112232959803299\\
151	0.0112233179545134\\
152	0.0112233403361631\\
153	0.0112233631328788\\
154	0.0112233863524031\\
155	0.0112234100026244\\
156	0.0112234340915791\\
157	0.0112234586274547\\
158	0.0112234836185928\\
159	0.0112235090734917\\
160	0.0112235350008097\\
161	0.011223561409368\\
162	0.0112235883081536\\
163	0.0112236157063229\\
164	0.0112236436132045\\
165	0.0112236720383026\\
166	0.0112237009913005\\
167	0.0112237304820639\\
168	0.0112237605206441\\
169	0.011223791117282\\
170	0.0112238222824115\\
171	0.0112238540266629\\
172	0.0112238863608673\\
173	0.0112239192960596\\
174	0.0112239528434832\\
175	0.0112239870145933\\
176	0.0112240218210613\\
177	0.0112240572747788\\
178	0.0112240933878619\\
179	0.0112241301726553\\
180	0.0112241676417369\\
181	0.011224205807922\\
182	0.0112242446842681\\
183	0.0112242842840792\\
184	0.0112243246209107\\
185	0.0112243657085742\\
186	0.0112244075611425\\
187	0.0112244501929542\\
188	0.0112244936186192\\
189	0.0112245378530236\\
190	0.0112245829113352\\
191	0.0112246288090087\\
192	0.0112246755617913\\
193	0.0112247231857282\\
194	0.0112247716971683\\
195	0.0112248211127701\\
196	0.0112248714495076\\
197	0.0112249227246763\\
198	0.0112249749558993\\
199	0.0112250281611337\\
200	0.0112250823586769\\
201	0.0112251375671731\\
202	0.0112251938056199\\
203	0.0112252510933753\\
204	0.0112253094501642\\
205	0.011225368896086\\
206	0.011225429451621\\
207	0.0112254911376385\\
208	0.0112255539754037\\
209	0.0112256179865854\\
210	0.0112256831932639\\
211	0.0112257496179387\\
212	0.0112258172835366\\
213	0.01122588621342\\
214	0.0112259564313949\\
215	0.0112260279617198\\
216	0.0112261008291141\\
217	0.0112261750587671\\
218	0.0112262506763467\\
219	0.0112263277080092\\
220	0.0112264061804078\\
221	0.0112264861207027\\
222	0.0112265675565707\\
223	0.0112266505162151\\
224	0.0112267350283756\\
225	0.0112268211223386\\
226	0.011226908827948\\
227	0.0112269981756153\\
228	0.0112270891963311\\
229	0.0112271819216755\\
230	0.0112272763838299\\
231	0.0112273726155881\\
232	0.0112274706503684\\
233	0.0112275705222251\\
234	0.011227672265861\\
235	0.0112277759166393\\
236	0.0112278815105967\\
237	0.0112279890844561\\
238	0.0112280986756394\\
239	0.0112282103222809\\
240	0.0112283240632411\\
241	0.0112284399381203\\
242	0.0112285579872724\\
243	0.0112286782518199\\
244	0.0112288007736676\\
245	0.0112289255955179\\
246	0.011229052760886\\
247	0.011229182314115\\
248	0.0112293143003913\\
249	0.0112294487657612\\
250	0.0112295857571464\\
251	0.0112297253223607\\
252	0.011229867510127\\
253	0.0112300123700938\\
254	0.0112301599528529\\
255	0.0112303103099569\\
256	0.0112304634939374\\
257	0.0112306195583225\\
258	0.011230778557656\\
259	0.011230940547516\\
260	0.0112311055845337\\
261	0.0112312737264134\\
262	0.0112314450319516\\
263	0.0112316195610575\\
264	0.0112317973747732\\
265	0.0112319785352941\\
266	0.0112321631059903\\
267	0.0112323511514274\\
268	0.0112325427373884\\
269	0.0112327379308954\\
270	0.0112329368002318\\
271	0.0112331394149648\\
272	0.011233345845968\\
273	0.0112335561654447\\
274	0.0112337704469511\\
275	0.0112339887654198\\
276	0.011234211197184\\
277	0.0112344378200014\\
278	0.0112346687130787\\
279	0.0112349039570962\\
280	0.0112351436342329\\
281	0.0112353878281913\\
282	0.0112356366242233\\
283	0.0112358901091551\\
284	0.0112361483714135\\
285	0.0112364115010516\\
286	0.0112366795897748\\
287	0.0112369527309674\\
288	0.0112372310197184\\
289	0.0112375145528484\\
290	0.011237803428936\\
291	0.0112380977483446\\
292	0.0112383976132487\\
293	0.0112387031276607\\
294	0.0112390143974577\\
295	0.011239331530408\\
296	0.0112396546361978\\
297	0.0112399838264574\\
298	0.0112403192147879\\
299	0.0112406609167875\\
300	0.0112410090500772\\
301	0.0112413637343272\\
302	0.0112417250912825\\
303	0.0112420932447884\\
304	0.0112424683208162\\
305	0.0112428504474881\\
306	0.0112432397551022\\
307	0.0112436363761576\\
308	0.0112440404453791\\
309	0.0112444520997416\\
310	0.011244871478495\\
311	0.0112452987231886\\
312	0.0112457339776962\\
313	0.0112461773882414\\
314	0.0112466291034225\\
315	0.0112470892742399\\
316	0.0112475580541222\\
317	0.0112480355989557\\
318	0.0112485220671135\\
319	0.0112490176194881\\
320	0.0112495224195257\\
321	0.0112500366332634\\
322	0.0112505604293706\\
323	0.0112510939791941\\
324	0.0112516374568092\\
325	0.0112521910390758\\
326	0.0112527549057027\\
327	0.0112533292393193\\
328	0.0112539142255573\\
329	0.0112545100531425\\
330	0.0112551169139992\\
331	0.0112557350033674\\
332	0.011256364519935\\
333	0.0112570056659853\\
334	0.0112576586475614\\
335	0.0112583236746474\\
336	0.0112590009613661\\
337	0.0112596907261927\\
338	0.0112603931921805\\
339	0.0112611085871956\\
340	0.0112618371441518\\
341	0.0112625791012361\\
342	0.0112633347021083\\
343	0.0112641041960519\\
344	0.0112648878380441\\
345	0.011265685888711\\
346	0.0112664986141052\\
347	0.0112673262854511\\
348	0.0112681691792728\\
349	0.0112690275775281\\
350	0.0112699017677485\\
351	0.0112707920431862\\
352	0.0112716987029678\\
353	0.0112726220522521\\
354	0.0112735624023946\\
355	0.0112745200711203\\
356	0.0112754953826846\\
357	0.0112764886680353\\
358	0.0112775002649788\\
359	0.0112785305183488\\
360	0.0112795797801781\\
361	0.0112806484098733\\
362	0.0112817367743908\\
363	0.011282845248415\\
364	0.011283974214537\\
365	0.0112851240634348\\
366	0.0112862951940518\\
367	0.0112874880137752\\
368	0.0112887029386123\\
369	0.0112899403933638\\
370	0.0112912008117934\\
371	0.0112924846367924\\
372	0.0112937923205388\\
373	0.0112951243246491\\
374	0.0112964811203223\\
375	0.0112978631884755\\
376	0.0112992710198701\\
377	0.0113007051152286\\
378	0.0113021659853424\\
379	0.0113036541511709\\
380	0.0113051701439351\\
381	0.0113067145052077\\
382	0.0113082877870055\\
383	0.0113098905518903\\
384	0.0113115233730906\\
385	0.0113131868346545\\
386	0.0113148815316564\\
387	0.0113166080704795\\
388	0.0113183670692088\\
389	0.0113201591581762\\
390	0.0113219849807153\\
391	0.0113238451941965\\
392	0.0113257404714362\\
393	0.0113276715025997\\
394	0.0113296389977477\\
395	0.0113316436902041\\
396	0.0113336863409353\\
397	0.0113357677441374\\
398	0.0113378887348318\\
399	0.0113400501891257\\
400	0.0113422530048768\\
401	0.0113444981016797\\
402	0.0113467864207639\\
403	0.0113491189247945\\
404	0.0113514965975696\\
405	0.0113539204436045\\
406	0.0113563914876015\\
407	0.0113589107738157\\
408	0.0113614793653651\\
409	0.0113640983435972\\
410	0.0113667688076252\\
411	0.0113694918737642\\
412	0.0113722686734777\\
413	0.0113751003485068\\
414	0.0113779880549139\\
415	0.0113809329646181\\
416	0.0113839362623907\\
417	0.011386999142006\\
418	0.0113901228013718\\
419	0.0113933084364245\\
420	0.0113965572335405\\
421	0.0113998703601832\\
422	0.0114032489534711\\
423	0.011406694106363\\
424	0.0114102068511853\\
425	0.0114137881400628\\
426	0.0114174388188876\\
427	0.0114211595720168\\
428	0.0114249513806757\\
429	0.0114288158262218\\
430	0.011432754551876\\
431	0.0114367692681713\\
432	0.0114408617591386\\
433	0.0114450338893178\\
434	0.0114492876117109\\
435	0.0114536249769666\\
436	0.0114580481439158\\
437	0.0114625593916818\\
438	0.0114671611336305\\
439	0.011471855933476\\
440	0.0114766465239209\\
441	0.011481535828314\\
442	0.0114865269859937\\
443	0.0114916233824512\\
444	0.011496828686832\\
445	0.0115021469039296\\
446	0.0115075824644448\\
447	0.0115131404395331\\
448	0.0115191322062942\\
449	0.0115253991608814\\
450	0.0115317762313072\\
451	0.0115382647247336\\
452	0.0115448659241668\\
453	0.0115515811223587\\
454	0.0115584117023178\\
455	0.0115653587311498\\
456	0.0115724231632056\\
457	0.0115796058215558\\
458	0.0115869073784527\\
459	0.0115943283382833\\
460	0.0116018690360585\\
461	0.0116095296963436\\
462	0.0116173107300015\\
463	0.0116252138172487\\
464	0.011633245881548\\
465	0.0116409291186393\\
466	0.011648737192218\\
467	0.0116566872445538\\
468	0.0116647814716776\\
469	0.011673022064531\\
470	0.0116814112027402\\
471	0.0116899511018213\\
472	0.0116986439963342\\
473	0.0117074921346454\\
474	0.011716497799903\\
475	0.0117256632804195\\
476	0.0117349907755795\\
477	0.0117444810454503\\
478	0.0117541301058064\\
479	0.0117638676743984\\
480	0.011773735668206\\
481	0.011783785700611\\
482	0.0117940201873895\\
483	0.0118044413792205\\
484	0.0118150518355674\\
485	0.011825852454593\\
486	0.0118368405715724\\
487	0.0118479980445204\\
488	0.0118593537395021\\
489	0.0118709228414562\\
490	0.0118827043690247\\
491	0.0118947033303379\\
492	0.0119069278351825\\
493	0.0119193808282175\\
494	0.0119320644612938\\
495	0.011944980676853\\
496	0.0119581312197565\\
497	0.0119715176530992\\
498	0.0119851412676121\\
499	0.0119990028801546\\
500	0.0120131027660469\\
501	0.0120274406003132\\
502	0.0120420154112648\\
503	0.0120568252786763\\
504	0.0120718684325769\\
505	0.0120871484815884\\
506	0.0121026704547876\\
507	0.0121184413034275\\
508	0.0121344611355934\\
509	0.0121508368205792\\
510	0.01216842867978\\
511	0.0121862142017653\\
512	0.0122044455249945\\
513	0.0122231229061308\\
514	0.0122422283685575\\
515	0.0122618673473782\\
516	0.0122812016952955\\
517	0.0123039654967153\\
518	0.0123375914224916\\
519	0.0123703867767683\\
520	0.0124021639739272\\
521	0.0124304090221293\\
522	0.0124483093410831\\
523	0.012465186770289\\
524	0.0124809448608415\\
525	0.0124952318226891\\
526	0.0125071168720933\\
527	0.0125188577351294\\
528	0.0125305005114574\\
529	0.0125421071992136\\
530	0.0125538705882579\\
531	0.0125658181268851\\
532	0.0125779716808804\\
533	0.0125903529367803\\
534	0.0126029746354097\\
535	0.012615844187871\\
536	0.012628968201082\\
537	0.0126423524358235\\
538	0.0126560016126776\\
539	0.0126699196902501\\
540	0.0126841108125561\\
541	0.0126985931866322\\
542	0.0127143652210051\\
543	0.0127310739668326\\
544	0.0127476395016023\\
545	0.0127627334645779\\
546	0.0127776116276101\\
547	0.012792569092646\\
548	0.0128073408324845\\
549	0.012822322360404\\
550	0.0128375136959082\\
551	0.0128529220491045\\
552	0.0128685412994251\\
553	0.012884357167586\\
554	0.0129003310487705\\
555	0.012916510225626\\
556	0.012932898194355\\
557	0.0129494900945344\\
558	0.0129662896703516\\
559	0.0129846714157593\\
560	0.0130023054258907\\
561	0.0130188302252141\\
562	0.0130351869307859\\
563	0.0130515579693191\\
564	0.0130680628266491\\
565	0.0130846955837323\\
566	0.0131014483412305\\
567	0.0131183126116483\\
568	0.0131357162880282\\
569	0.0131529761041365\\
570	0.0131696048889977\\
571	0.0131861379800626\\
572	0.0132027353963806\\
573	0.0132193852681561\\
574	0.0132360736311067\\
575	0.0132527851286413\\
576	0.0132695028912621\\
577	0.0132862084004275\\
578	0.0133028813379604\\
579	0.0133194994192258\\
580	0.0133360382078977\\
581	0.0133524709099312\\
582	0.0133687681444566\\
583	0.0133848976903317\\
584	0.0134008242107329\\
585	0.0134165089689716\\
586	0.0134319095789188\\
587	0.0134469799156334\\
588	0.0134616705332227\\
589	0.0134759305304731\\
590	0.0134902483101722\\
591	0.0135046714122469\\
592	0.0135192196273521\\
593	0.0135339766714342\\
594	0.0135491998980231\\
595	0.0135656125153459\\
596	0.0135851815686143\\
597	0.0136131889078992\\
598	0.0136637438596031\\
599	0\\
600	0\\
};
\addplot [color=black!20!mycolor21,solid,forget plot]
  table[row sep=crcr]{%
1	0.0112307667062576\\
2	0.0112307680392395\\
3	0.0112307693967611\\
4	0.0112307707792757\\
5	0.011230772187245\\
6	0.0112307736211395\\
7	0.0112307750814381\\
8	0.0112307765686288\\
9	0.0112307780832086\\
10	0.0112307796256837\\
11	0.0112307811965698\\
12	0.0112307827963919\\
13	0.0112307844256851\\
14	0.0112307860849943\\
15	0.0112307877748743\\
16	0.0112307894958906\\
17	0.0112307912486188\\
18	0.0112307930336455\\
19	0.0112307948515681\\
20	0.0112307967029949\\
21	0.0112307985885459\\
22	0.0112308005088522\\
23	0.011230802464557\\
24	0.0112308044563151\\
25	0.0112308064847937\\
26	0.0112308085506724\\
27	0.0112308106546434\\
28	0.0112308127974115\\
29	0.0112308149796952\\
30	0.0112308172022257\\
31	0.0112308194657483\\
32	0.0112308217710218\\
33	0.0112308241188194\\
34	0.0112308265099286\\
35	0.0112308289451514\\
36	0.0112308314253048\\
37	0.0112308339512212\\
38	0.0112308365237481\\
39	0.0112308391437492\\
40	0.0112308418121039\\
41	0.0112308445297083\\
42	0.0112308472974749\\
43	0.0112308501163333\\
44	0.0112308529872305\\
45	0.0112308559111311\\
46	0.0112308588890175\\
47	0.0112308619218905\\
48	0.0112308650107697\\
49	0.0112308681566934\\
50	0.0112308713607196\\
51	0.0112308746239256\\
52	0.0112308779474092\\
53	0.0112308813322883\\
54	0.0112308847797018\\
55	0.01123088829081\\
56	0.0112308918667945\\
57	0.0112308955088591\\
58	0.0112308992182301\\
59	0.0112309029961566\\
60	0.0112309068439111\\
61	0.0112309107627896\\
62	0.0112309147541125\\
63	0.0112309188192248\\
64	0.0112309229594966\\
65	0.0112309271763235\\
66	0.0112309314711272\\
67	0.0112309358453559\\
68	0.011230940300485\\
69	0.0112309448380172\\
70	0.0112309494594835\\
71	0.0112309541664433\\
72	0.0112309589604853\\
73	0.0112309638432277\\
74	0.0112309688163191\\
75	0.0112309738814389\\
76	0.0112309790402978\\
77	0.0112309842946385\\
78	0.0112309896462362\\
79	0.0112309950968995\\
80	0.0112310006484706\\
81	0.0112310063028264\\
82	0.0112310120618786\\
83	0.0112310179275749\\
84	0.0112310239018994\\
85	0.0112310299868732\\
86	0.0112310361845553\\
87	0.0112310424970433\\
88	0.011231048926474\\
89	0.0112310554750243\\
90	0.0112310621449116\\
91	0.0112310689383952\\
92	0.0112310758577764\\
93	0.0112310829053999\\
94	0.0112310900836539\\
95	0.0112310973949718\\
96	0.0112311048418324\\
97	0.011231112426761\\
98	0.0112311201523302\\
99	0.0112311280211608\\
100	0.0112311360359229\\
101	0.0112311441993365\\
102	0.0112311525141728\\
103	0.0112311609832548\\
104	0.0112311696094586\\
105	0.0112311783957143\\
106	0.0112311873450069\\
107	0.0112311964603775\\
108	0.0112312057449242\\
109	0.0112312152018035\\
110	0.0112312248342309\\
111	0.0112312346454825\\
112	0.0112312446388959\\
113	0.0112312548178714\\
114	0.0112312651858733\\
115	0.0112312757464309\\
116	0.0112312865031398\\
117	0.0112312974596635\\
118	0.0112313086197342\\
119	0.0112313199871543\\
120	0.0112313315657976\\
121	0.0112313433596113\\
122	0.0112313553726163\\
123	0.0112313676089095\\
124	0.011231380072665\\
125	0.0112313927681353\\
126	0.0112314056996532\\
127	0.011231418871633\\
128	0.0112314322885721\\
129	0.0112314459550529\\
130	0.0112314598757441\\
131	0.0112314740554022\\
132	0.0112314884988737\\
133	0.0112315032110965\\
134	0.0112315181971013\\
135	0.0112315334620142\\
136	0.0112315490110578\\
137	0.0112315648495532\\
138	0.0112315809829222\\
139	0.011231597416689\\
140	0.0112316141564819\\
141	0.0112316312080358\\
142	0.0112316485771938\\
143	0.0112316662699096\\
144	0.0112316842922494\\
145	0.0112317026503942\\
146	0.0112317213506418\\
147	0.0112317403994092\\
148	0.0112317598032349\\
149	0.0112317795687812\\
150	0.0112317997028365\\
151	0.0112318202123177\\
152	0.0112318411042729\\
153	0.0112318623858836\\
154	0.0112318840644675\\
155	0.011231906147481\\
156	0.0112319286425221\\
157	0.0112319515573326\\
158	0.0112319748998015\\
159	0.0112319986779674\\
160	0.0112320229000214\\
161	0.0112320475743104\\
162	0.0112320727093397\\
163	0.0112320983137763\\
164	0.0112321243964519\\
165	0.011232150966366\\
166	0.0112321780326895\\
167	0.0112322056047677\\
168	0.0112322336921235\\
169	0.0112322623044614\\
170	0.0112322914516706\\
171	0.0112323211438286\\
172	0.011232351391205\\
173	0.0112323822042651\\
174	0.0112324135936739\\
175	0.0112324455702996\\
176	0.0112324781452179\\
177	0.0112325113297159\\
178	0.0112325451352962\\
179	0.0112325795736812\\
180	0.011232614656817\\
181	0.0112326503968785\\
182	0.0112326868062731\\
183	0.0112327238976455\\
184	0.0112327616838827\\
185	0.011232800178118\\
186	0.0112328393937365\\
187	0.0112328793443798\\
188	0.0112329200439507\\
189	0.0112329615066188\\
190	0.0112330037468255\\
191	0.0112330467792893\\
192	0.0112330906190113\\
193	0.0112331352812807\\
194	0.0112331807816806\\
195	0.0112332271360936\\
196	0.0112332743607078\\
197	0.0112333224720229\\
198	0.011233371486856\\
199	0.0112334214223485\\
200	0.0112334722959716\\
201	0.0112335241255339\\
202	0.0112335769291871\\
203	0.0112336307254334\\
204	0.0112336855331321\\
205	0.0112337413715071\\
206	0.0112337982601535\\
207	0.0112338562190457\\
208	0.0112339152685445\\
209	0.0112339754294047\\
210	0.0112340367227834\\
211	0.0112340991702475\\
212	0.0112341627937824\\
213	0.0112342276158001\\
214	0.0112342936591474\\
215	0.0112343609471154\\
216	0.0112344295034477\\
217	0.0112344993523498\\
218	0.0112345705184982\\
219	0.0112346430270503\\
220	0.0112347169036533\\
221	0.0112347921744551\\
222	0.0112348688661133\\
223	0.0112349470058063\\
224	0.0112350266212436\\
225	0.0112351077406765\\
226	0.0112351903929089\\
227	0.0112352746073088\\
228	0.0112353604138199\\
229	0.0112354478429727\\
230	0.011235536925897\\
231	0.011235627694334\\
232	0.0112357201806485\\
233	0.0112358144178421\\
234	0.0112359104395658\\
235	0.0112360082801335\\
236	0.0112361079745356\\
237	0.0112362095584531\\
238	0.0112363130682711\\
239	0.0112364185410943\\
240	0.0112365260147609\\
241	0.0112366355278583\\
242	0.0112367471197386\\
243	0.0112368608305339\\
244	0.0112369767011731\\
245	0.0112370947733981\\
246	0.0112372150897807\\
247	0.01123733769374\\
248	0.0112374626295598\\
249	0.0112375899424069\\
250	0.0112377196783496\\
251	0.0112378518843761\\
252	0.0112379866084142\\
253	0.0112381238993509\\
254	0.0112382638070524\\
255	0.0112384063823847\\
256	0.011238551677235\\
257	0.0112386997445326\\
258	0.0112388506382713\\
259	0.0112390044135318\\
260	0.0112391611265046\\
261	0.0112393208345137\\
262	0.0112394835960401\\
263	0.0112396494707471\\
264	0.0112398185195046\\
265	0.0112399908044153\\
266	0.0112401663888406\\
267	0.0112403453374274\\
268	0.0112405277161357\\
269	0.0112407135922663\\
270	0.0112409030344894\\
271	0.0112410961128739\\
272	0.011241292898917\\
273	0.011241493465575\\
274	0.0112416978872941\\
275	0.011241906240042\\
276	0.0112421186013408\\
277	0.0112423350502996\\
278	0.0112425556676486\\
279	0.0112427805357733\\
280	0.0112430097387497\\
281	0.0112432433623799\\
282	0.0112434814942289\\
283	0.0112437242236615\\
284	0.0112439716418803\\
285	0.0112442238419642\\
286	0.0112444809189072\\
287	0.0112447429696587\\
288	0.0112450100931639\\
289	0.0112452823904045\\
290	0.0112455599644406\\
291	0.0112458429204529\\
292	0.0112461313657854\\
293	0.0112464254099887\\
294	0.0112467251648633\\
295	0.011247030744504\\
296	0.0112473422653442\\
297	0.0112476598462\\
298	0.0112479836083154\\
299	0.0112483136754067\\
300	0.0112486501737073\\
301	0.0112489932320123\\
302	0.0112493429817223\\
303	0.0112496995568871\\
304	0.011250063094249\\
305	0.0112504337332846\\
306	0.0112508116162455\\
307	0.0112511968881987\\
308	0.011251589697064\\
309	0.0112519901936504\\
310	0.0112523985316899\\
311	0.0112528148678689\\
312	0.0112532393618569\\
313	0.0112536721763309\\
314	0.0112541134769972\\
315	0.0112545634326081\\
316	0.0112550222149743\\
317	0.0112554899989719\\
318	0.0112559669625436\\
319	0.0112564532866937\\
320	0.0112569491554766\\
321	0.0112574547559778\\
322	0.0112579702782874\\
323	0.011258495915466\\
324	0.0112590318635021\\
325	0.0112595783212621\\
326	0.0112601354904324\\
327	0.0112607035754546\\
328	0.011261282783455\\
329	0.0112618733241702\\
330	0.0112624754098717\\
331	0.0112630892552927\\
332	0.0112637150775627\\
333	0.0112643530961564\\
334	0.011265003532867\\
335	0.0112656666118139\\
336	0.0112663425595027\\
337	0.0112670316049559\\
338	0.0112677339799408\\
339	0.011268449919328\\
340	0.0112691796616234\\
341	0.0112699234497262\\
342	0.011270681531983\\
343	0.0112714541636173\\
344	0.0112722416086195\\
345	0.0112730441421911\\
346	0.011273862054093\\
347	0.0112746956487724\\
348	0.0112755452373909\\
349	0.0112764111380161\\
350	0.01127729367582\\
351	0.0112781931832859\\
352	0.0112791100004236\\
353	0.0112800444749947\\
354	0.011280996962749\\
355	0.0112819678276726\\
356	0.0112829574422343\\
357	0.0112839661876405\\
358	0.011284994454098\\
359	0.011286042641085\\
360	0.0112871111576312\\
361	0.0112882004226044\\
362	0.0112893108650061\\
363	0.0112904429242726\\
364	0.0112915970505834\\
365	0.0112927737051744\\
366	0.0112939733606545\\
367	0.0112951965013248\\
368	0.0112964436234978\\
369	0.0112977152358139\\
370	0.0112990118595516\\
371	0.0113003340289287\\
372	0.0113016822913876\\
373	0.0113030572078605\\
374	0.011304459353006\\
375	0.0113058893154091\\
376	0.0113073476977328\\
377	0.0113088351168096\\
378	0.0113103522036562\\
379	0.0113118996033932\\
380	0.0113134779750471\\
381	0.0113150879912066\\
382	0.0113167303375016\\
383	0.011318405711864\\
384	0.0113201148235244\\
385	0.011321858391686\\
386	0.0113236371438089\\
387	0.0113254518134204\\
388	0.0113273031373548\\
389	0.0113291918523007\\
390	0.0113311186905176\\
391	0.0113330843745502\\
392	0.0113350896107483\\
393	0.0113371350813761\\
394	0.0113392214350959\\
395	0.0113413492756428\\
396	0.011343519148421\\
397	0.011345731523411\\
398	0.0113479867633781\\
399	0.0113502852931765\\
400	0.0113526280027205\\
401	0.0113550158061234\\
402	0.0113574496431801\\
403	0.0113599304810429\\
404	0.0113624593161152\\
405	0.0113650371761958\\
406	0.0113676651229043\\
407	0.0113703442544188\\
408	0.0113730757085532\\
409	0.0113758606661902\\
410	0.0113787003551336\\
411	0.0113815960549853\\
412	0.0113845491032701\\
413	0.0113875609018832\\
414	0.0113906329241603\\
415	0.011393766722834\\
416	0.0113969639392339\\
417	0.0114002263138842\\
418	0.0114035556987356\\
419	0.0114069540713054\\
420	0.0114104235510158\\
421	0.0114139664181129\\
422	0.0114175851361284\\
423	0.0114212823795039\\
424	0.0114250610712117\\
425	0.0114289244426929\\
426	0.0114328761654531\\
427	0.0114370640982274\\
428	0.0114415379374501\\
429	0.0114460924029805\\
430	0.0114507286537309\\
431	0.0114554478303431\\
432	0.0114602510426257\\
433	0.0114651393613016\\
434	0.0114701138533264\\
435	0.011475175547086\\
436	0.0114803254241217\\
437	0.0114855644096148\\
438	0.0114908933614573\\
439	0.0114963130577435\\
440	0.0115018241826379\\
441	0.0115074273110264\\
442	0.0115131228940036\\
443	0.0115189112529816\\
444	0.0115247926103114\\
445	0.0115307672557026\\
446	0.0115368362056364\\
447	0.0115430036618903\\
448	0.0115490132673341\\
449	0.0115549947015359\\
450	0.0115610862595482\\
451	0.0115672897545382\\
452	0.0115736070318654\\
453	0.0115800399785694\\
454	0.0115865905355455\\
455	0.0115932606339144\\
456	0.0116000522373306\\
457	0.0116069673451786\\
458	0.0116140079956677\\
459	0.0116211762663818\\
460	0.0116284742624648\\
461	0.0116359040542482\\
462	0.0116434674208401\\
463	0.011651164826483\\
464	0.0116589913997526\\
465	0.0116668580464109\\
466	0.0116748712939301\\
467	0.0116830370779039\\
468	0.0116913580790573\\
469	0.011699837004147\\
470	0.0117084765842934\\
471	0.0117172795714657\\
472	0.01172624873421\\
473	0.0117353868512342\\
474	0.0117446966973311\\
475	0.0117541810099138\\
476	0.0117638423974898\\
477	0.0117736831347504\\
478	0.0117837046803641\\
479	0.0117939107459867\\
480	0.0118043046563185\\
481	0.0118148888470297\\
482	0.0118256641281432\\
483	0.0118366276489269\\
484	0.0118477610158207\\
485	0.011859093544736\\
486	0.011870639935368\\
487	0.011882399334264\\
488	0.0118943790589139\\
489	0.0119065838765882\\
490	0.0119190160358446\\
491	0.0119316777135256\\
492	0.0119445707170306\\
493	0.011957696419546\\
494	0.0119710557136896\\
495	0.0119846489551514\\
496	0.0119984758715109\\
497	0.0120125350473707\\
498	0.0120268270333737\\
499	0.0120413562861641\\
500	0.0120561278030908\\
501	0.0120711473781491\\
502	0.0120864222598817\\
503	0.0121022226856904\\
504	0.0121190727498333\\
505	0.0121360117600989\\
506	0.0121530092632708\\
507	0.0121700361156674\\
508	0.012187250361415\\
509	0.0122047689873851\\
510	0.0122217619396384\\
511	0.012239260157526\\
512	0.0122574551592304\\
513	0.0122775888917459\\
514	0.0123100310886368\\
515	0.0123416998353115\\
516	0.0123720190910494\\
517	0.0123991861831648\\
518	0.0124160639725055\\
519	0.0124319379072306\\
520	0.0124466770974654\\
521	0.0124598547014337\\
522	0.0124708480141983\\
523	0.012481708239206\\
524	0.0124924783118715\\
525	0.0125032377562866\\
526	0.012514155670743\\
527	0.0125252528002105\\
528	0.0125365503121378\\
529	0.0125480686772887\\
530	0.0125598158865116\\
531	0.0125717988484442\\
532	0.0125840236760444\\
533	0.0125964954365076\\
534	0.012609218675859\\
535	0.0126221979545071\\
536	0.0126354374974166\\
537	0.0126489398102175\\
538	0.0126627172712948\\
539	0.0126770675913136\\
540	0.0126932238069243\\
541	0.0127092401128757\\
542	0.0127242536202772\\
543	0.0127385024089967\\
544	0.0127528396853093\\
545	0.0127670273929129\\
546	0.012781355090042\\
547	0.0127958929273647\\
548	0.0128106515387539\\
549	0.0128256300351418\\
550	0.0128408265723288\\
551	0.0128562370638659\\
552	0.0128718528282949\\
553	0.0128876526856466\\
554	0.012903615942414\\
555	0.012919788613919\\
556	0.0129361672580484\\
557	0.0129533934080588\\
558	0.0129714059924364\\
559	0.0129877694482098\\
560	0.0130039237299284\\
561	0.0130199787653225\\
562	0.0130361763128579\\
563	0.0130525147202513\\
564	0.0130689870656023\\
565	0.0130855862276066\\
566	0.0131023047069769\\
567	0.0131196261688764\\
568	0.0131366518311452\\
569	0.0131531500754486\\
570	0.0131696057265081\\
571	0.0131861379869476\\
572	0.0132027353981748\\
573	0.0132193852690165\\
574	0.0132360736315502\\
575	0.0132527851288632\\
576	0.0132695028913673\\
577	0.0132862084004741\\
578	0.0133028813379795\\
579	0.0133194994192329\\
580	0.0133360382078999\\
581	0.0133524709099318\\
582	0.0133687681444567\\
583	0.0133848976903317\\
584	0.0134008242107329\\
585	0.0134165089689716\\
586	0.0134319095789188\\
587	0.0134469799156334\\
588	0.0134616705332227\\
589	0.0134759305304731\\
590	0.0134902483101722\\
591	0.0135046714122469\\
592	0.0135192196273521\\
593	0.0135339766714342\\
594	0.0135491998980231\\
595	0.0135656125153459\\
596	0.0135851815686143\\
597	0.0136131889078992\\
598	0.0136637438596031\\
599	0\\
600	0\\
};
\addplot [color=black!50!mycolor20,solid,forget plot]
  table[row sep=crcr]{%
1	0.0112351944782826\\
2	0.0112351958259356\\
3	0.0112351971984351\\
4	0.0112351985962409\\
5	0.0112352000198214\\
6	0.0112352014696533\\
7	0.0112352029462226\\
8	0.0112352044500242\\
9	0.011235205981562\\
10	0.0112352075413496\\
11	0.0112352091299098\\
12	0.0112352107477753\\
13	0.0112352123954888\\
14	0.0112352140736029\\
15	0.0112352157826806\\
16	0.0112352175232953\\
17	0.011235219296031\\
18	0.0112352211014828\\
19	0.0112352229402567\\
20	0.0112352248129698\\
21	0.0112352267202511\\
22	0.011235228662741\\
23	0.0112352306410919\\
24	0.0112352326559683\\
25	0.011235234708047\\
26	0.0112352367980177\\
27	0.0112352389265827\\
28	0.0112352410944572\\
29	0.0112352433023702\\
30	0.0112352455510638\\
31	0.0112352478412942\\
32	0.0112352501738315\\
33	0.0112352525494604\\
34	0.01123525496898\\
35	0.0112352574332043\\
36	0.0112352599429625\\
37	0.0112352624990993\\
38	0.0112352651024751\\
39	0.0112352677539663\\
40	0.0112352704544658\\
41	0.0112352732048828\\
42	0.0112352760061437\\
43	0.0112352788591922\\
44	0.0112352817649896\\
45	0.0112352847245149\\
46	0.0112352877387655\\
47	0.0112352908087575\\
48	0.0112352939355258\\
49	0.0112352971201247\\
50	0.0112353003636281\\
51	0.01123530366713\\
52	0.0112353070317447\\
53	0.0112353104586075\\
54	0.0112353139488747\\
55	0.0112353175037242\\
56	0.0112353211243561\\
57	0.0112353248119926\\
58	0.0112353285678789\\
59	0.0112353323932834\\
60	0.0112353362894982\\
61	0.0112353402578396\\
62	0.0112353442996484\\
63	0.0112353484162904\\
64	0.0112353526091571\\
65	0.0112353568796658\\
66	0.0112353612292604\\
67	0.0112353656594118\\
68	0.0112353701716183\\
69	0.0112353747674064\\
70	0.0112353794483308\\
71	0.0112353842159756\\
72	0.0112353890719545\\
73	0.011235394017911\\
74	0.0112353990555199\\
75	0.0112354041864869\\
76	0.0112354094125497\\
77	0.0112354147354788\\
78	0.0112354201570774\\
79	0.0112354256791828\\
80	0.0112354313036664\\
81	0.0112354370324349\\
82	0.0112354428674306\\
83	0.011235448810632\\
84	0.011235454864055\\
85	0.0112354610297528\\
86	0.0112354673098176\\
87	0.0112354737063802\\
88	0.0112354802216119\\
89	0.0112354868577243\\
90	0.0112354936169706\\
91	0.0112355005016462\\
92	0.0112355075140895\\
93	0.0112355146566828\\
94	0.011235521931853\\
95	0.0112355293420726\\
96	0.0112355368898604\\
97	0.0112355445777826\\
98	0.0112355524084532\\
99	0.0112355603845357\\
100	0.0112355685087431\\
101	0.0112355767838398\\
102	0.0112355852126418\\
103	0.011235593798018\\
104	0.0112356025428913\\
105	0.0112356114502392\\
106	0.0112356205230955\\
107	0.0112356297645507\\
108	0.0112356391777536\\
109	0.0112356487659119\\
110	0.0112356585322938\\
111	0.0112356684802288\\
112	0.011235678613109\\
113	0.0112356889343905\\
114	0.0112356994475941\\
115	0.0112357101563069\\
116	0.0112357210641835\\
117	0.0112357321749474\\
118	0.0112357434923919\\
119	0.0112357550203819\\
120	0.011235766762855\\
121	0.0112357787238227\\
122	0.0112357909073725\\
123	0.0112358033176684\\
124	0.0112358159589532\\
125	0.0112358288355493\\
126	0.0112358419518609\\
127	0.011235855312375\\
128	0.0112358689216631\\
129	0.0112358827843831\\
130	0.0112358969052805\\
131	0.0112359112891905\\
132	0.0112359259410393\\
133	0.0112359408658463\\
134	0.0112359560687254\\
135	0.0112359715548871\\
136	0.0112359873296402\\
137	0.011236003398394\\
138	0.0112360197666595\\
139	0.0112360364400522\\
140	0.0112360534242935\\
141	0.0112360707252129\\
142	0.0112360883487502\\
143	0.0112361063009574\\
144	0.011236124588001\\
145	0.0112361432161641\\
146	0.0112361621918486\\
147	0.0112361815215777\\
148	0.0112362012119977\\
149	0.0112362212698812\\
150	0.0112362417021287\\
151	0.0112362625157715\\
152	0.011236283717974\\
153	0.0112363053160365\\
154	0.0112363273173976\\
155	0.0112363497296369\\
156	0.0112363725604777\\
157	0.0112363958177899\\
158	0.0112364195095927\\
159	0.0112364436440572\\
160	0.01123646822951\\
161	0.0112364932744355\\
162	0.0112365187874795\\
163	0.0112365447774519\\
164	0.0112365712533299\\
165	0.0112365982242617\\
166	0.0112366256995691\\
167	0.0112366536887513\\
168	0.0112366822014883\\
169	0.0112367112476444\\
170	0.0112367408372715\\
171	0.011236770980613\\
172	0.0112368016881074\\
173	0.0112368329703923\\
174	0.0112368648383078\\
175	0.011236897302901\\
176	0.0112369303754294\\
177	0.0112369640673656\\
178	0.011236998390401\\
179	0.0112370333564505\\
180	0.0112370689776563\\
181	0.0112371052663928\\
182	0.0112371422352709\\
183	0.0112371798971428\\
184	0.0112372182651065\\
185	0.0112372573525106\\
186	0.0112372971729593\\
187	0.0112373377403177\\
188	0.0112373790687162\\
189	0.0112374211725562\\
190	0.0112374640665155\\
191	0.0112375077655533\\
192	0.0112375522849161\\
193	0.0112375976401432\\
194	0.0112376438470724\\
195	0.0112376909218461\\
196	0.0112377388809171\\
197	0.0112377877410546\\
198	0.011237837519351\\
199	0.0112378882332277\\
200	0.0112379399004419\\
201	0.0112379925390931\\
202	0.0112380461676301\\
203	0.0112381008048578\\
204	0.0112381564699443\\
205	0.011238213182428\\
206	0.0112382709622254\\
207	0.0112383298296379\\
208	0.0112383898053603\\
209	0.0112384509104882\\
210	0.0112385131665259\\
211	0.0112385765953951\\
212	0.0112386412194427\\
213	0.0112387070614498\\
214	0.0112387741446402\\
215	0.0112388424926893\\
216	0.0112389121297333\\
217	0.0112389830803786\\
218	0.011239055369711\\
219	0.0112391290233056\\
220	0.011239204067237\\
221	0.0112392805280889\\
222	0.0112393584329647\\
223	0.0112394378094984\\
224	0.011239518685865\\
225	0.0112396010907916\\
226	0.0112396850535691\\
227	0.0112397706040632\\
228	0.0112398577727267\\
229	0.0112399465906113\\
230	0.0112400370893801\\
231	0.0112401293013203\\
232	0.0112402232593558\\
233	0.0112403189970606\\
234	0.0112404165486726\\
235	0.0112405159491068\\
236	0.0112406172339701\\
237	0.0112407204395753\\
238	0.0112408256029562\\
239	0.0112409327618823\\
240	0.011241041954875\\
241	0.0112411532212228\\
242	0.0112412666009978\\
243	0.0112413821350724\\
244	0.0112414998651363\\
245	0.0112416198337136\\
246	0.0112417420841811\\
247	0.0112418666607862\\
248	0.0112419936086659\\
249	0.0112421229738654\\
250	0.0112422548033587\\
251	0.0112423891450676\\
252	0.0112425260478831\\
253	0.0112426655616863\\
254	0.0112428077373697\\
255	0.0112429526268598\\
256	0.0112431002831399\\
257	0.011243250760273\\
258	0.0112434041134258\\
259	0.0112435603988937\\
260	0.0112437196741254\\
261	0.0112438819977489\\
262	0.0112440474295982\\
263	0.0112442160307399\\
264	0.011244387863502\\
265	0.0112445629915017\\
266	0.0112447414796753\\
267	0.0112449233943081\\
268	0.0112451088030658\\
269	0.011245297775026\\
270	0.0112454903807114\\
271	0.011245686692123\\
272	0.011245886782775\\
273	0.0112460907277304\\
274	0.0112462986036377\\
275	0.011246510488768\\
276	0.0112467264630548\\
277	0.0112469466081329\\
278	0.01124717100738\\
279	0.0112473997459587\\
280	0.0112476329108604\\
281	0.01124787059095\\
282	0.0112481128770119\\
283	0.0112483598617982\\
284	0.0112486116400772\\
285	0.0112488683086845\\
286	0.0112491299665753\\
287	0.011249396714878\\
288	0.01124966865695\\
289	0.0112499458984355\\
290	0.0112502285473242\\
291	0.0112505167140129\\
292	0.0112508105113688\\
293	0.0112511100547948\\
294	0.0112514154622969\\
295	0.0112517268545542\\
296	0.0112520443549908\\
297	0.0112523680898506\\
298	0.0112526981882742\\
299	0.0112530347823781\\
300	0.0112533780073377\\
301	0.0112537280014717\\
302	0.01125408490633\\
303	0.0112544488667842\\
304	0.0112548200311212\\
305	0.0112551985511393\\
306	0.0112555845822474\\
307	0.0112559782835671\\
308	0.0112563798180377\\
309	0.0112567893525232\\
310	0.0112572070579229\\
311	0.011257633109283\\
312	0.0112580676859115\\
313	0.0112585109714935\\
314	0.011258963154208\\
315	0.0112594244268451\\
316	0.0112598949869225\\
317	0.0112603750368003\\
318	0.0112608647837929\\
319	0.0112613644402765\\
320	0.0112618742237889\\
321	0.0112623943571217\\
322	0.0112629250683987\\
323	0.0112634665911391\\
324	0.0112640191643\\
325	0.0112645830322926\\
326	0.0112651584449661\\
327	0.0112657456575503\\
328	0.0112663449305484\\
329	0.0112669565295663\\
330	0.0112675807250669\\
331	0.0112682177920301\\
332	0.011268868009498\\
333	0.0112695316599826\\
334	0.0112702090287028\\
335	0.0112709004026179\\
336	0.0112716060692117\\
337	0.0112723263149765\\
338	0.0112730614235338\\
339	0.0112738116733184\\
340	0.0112745773347391\\
341	0.0112753586667209\\
342	0.0112761559125331\\
343	0.0112769692948167\\
344	0.0112777990096898\\
345	0.0112786452192335\\
346	0.0112795080376671\\
347	0.0112803876049575\\
348	0.0112812842512423\\
349	0.0112821983134854\\
350	0.0112831301356409\\
351	0.0112840800688197\\
352	0.0112850484714556\\
353	0.0112860357094653\\
354	0.0112870421563975\\
355	0.0112880681936037\\
356	0.0112891142106824\\
357	0.0112901806057492\\
358	0.0112912677857093\\
359	0.011292376166548\\
360	0.0112935061736426\\
361	0.0112946582420975\\
362	0.0112958328171055\\
363	0.0112970303543392\\
364	0.0112982513203751\\
365	0.0112994961931557\\
366	0.0113007654624944\\
367	0.0113020596306281\\
368	0.0113033792128247\\
369	0.0113047247380533\\
370	0.0113060967497241\\
371	0.0113074958065102\\
372	0.0113089224832606\\
373	0.0113103773720188\\
374	0.0113118610831618\\
375	0.0113133742466777\\
376	0.0113149175136011\\
377	0.0113164915576308\\
378	0.0113180970769578\\
379	0.011319734796334\\
380	0.0113214054694192\\
381	0.0113231098814511\\
382	0.0113248488522854\\
383	0.0113266232398675\\
384	0.0113284339442032\\
385	0.0113302819119076\\
386	0.0113321681414276\\
387	0.0113340936890466\\
388	0.0113360596758014\\
389	0.0113380672954645\\
390	0.0113401178237735\\
391	0.0113422126291345\\
392	0.0113443531850954\\
393	0.0113465410850429\\
394	0.011348778060013\\
395	0.0113510660018995\\
396	0.0113534069992335\\
397	0.0113558034107334\\
398	0.0113583168250304\\
399	0.0113610460379201\\
400	0.0113638261856476\\
401	0.0113666581148067\\
402	0.0113695426763981\\
403	0.0113724807243164\\
404	0.01137547311312\\
405	0.0113785206945055\\
406	0.0113816243116338\\
407	0.0113847847909651\\
408	0.0113880029361386\\
409	0.0113912795517672\\
410	0.0113946156180478\\
411	0.0113980122560836\\
412	0.011401470277601\\
413	0.0114049904425198\\
414	0.0114085735015855\\
415	0.0114122201905185\\
416	0.0114159312069635\\
417	0.0114197072026157\\
418	0.0114235487710558\\
419	0.011427456419056\\
420	0.0114314305383849\\
421	0.0114354714701621\\
422	0.0114395795153607\\
423	0.0114437550997129\\
424	0.0114479984198532\\
425	0.011452310039942\\
426	0.0114566922528386\\
427	0.0114610260806607\\
428	0.0114652689912649\\
429	0.0114695916077998\\
430	0.0114739952737607\\
431	0.0114784813487846\\
432	0.0114830512078456\\
433	0.0114877062419988\\
434	0.0114924478660921\\
435	0.0114972775148057\\
436	0.0115021966439266\\
437	0.0115072067319429\\
438	0.0115123092820202\\
439	0.0115175058244099\\
440	0.0115227979192723\\
441	0.011528187159605\\
442	0.0115336751728246\\
443	0.0115392636152083\\
444	0.0115449541367792\\
445	0.0115507482301072\\
446	0.0115566466262736\\
447	0.0115626469125934\\
448	0.0115686995666977\\
449	0.0115748386193224\\
450	0.0115810961834864\\
451	0.0115874744552185\\
452	0.0115939756710901\\
453	0.0116006021089091\\
454	0.0116073560874389\\
455	0.0116142399679179\\
456	0.0116212561545987\\
457	0.0116284070949366\\
458	0.0116356952788893\\
459	0.0116431232357747\\
460	0.0116506935242219\\
461	0.0116584087027417\\
462	0.011666271247919\\
463	0.0116742833448325\\
464	0.0116824464369669\\
465	0.011690765514842\\
466	0.0116992432784483\\
467	0.0117078823670914\\
468	0.0117166854303381\\
469	0.011725655123599\\
470	0.0117347941028349\\
471	0.0117441050180588\\
472	0.0117535905049195\\
473	0.0117632531731367\\
474	0.0117730955903263\\
475	0.011783120243332\\
476	0.011793329460267\\
477	0.0118037253839663\\
478	0.0118143103311685\\
479	0.0118250852272747\\
480	0.0118360472282428\\
481	0.0118471787530638\\
482	0.0118585066197922\\
483	0.0118700470539891\\
484	0.0118817992347845\\
485	0.011893770724185\\
486	0.0119059661922436\\
487	0.011918387666641\\
488	0.0119310367172285\\
489	0.0119439144308247\\
490	0.0119570213167615\\
491	0.0119703565722088\\
492	0.0119839236319086\\
493	0.011997728078787\\
494	0.0120117761109523\\
495	0.0120260748158329\\
496	0.012040632949565\\
497	0.0120559168278183\\
498	0.0120719814775316\\
499	0.0120881446193133\\
500	0.0121043905968925\\
501	0.0121207034716817\\
502	0.0121370704421017\\
503	0.0121532550532704\\
504	0.0121687851616581\\
505	0.0121844771200109\\
506	0.0122005760501311\\
507	0.0122171054119467\\
508	0.012234205817976\\
509	0.0122521990039042\\
510	0.0122816849601013\\
511	0.0123121442619949\\
512	0.0123416091951965\\
513	0.0123691027454942\\
514	0.0123851353787613\\
515	0.012400191923301\\
516	0.0124141165882071\\
517	0.0124265510429925\\
518	0.0124367580222875\\
519	0.0124468328337305\\
520	0.0124568189466466\\
521	0.0124668029325324\\
522	0.0124769382084773\\
523	0.0124872447203615\\
524	0.0124977427497673\\
525	0.0125084502254296\\
526	0.0125193742630316\\
527	0.0125305213519228\\
528	0.0125418971964977\\
529	0.0125535064874172\\
530	0.012565353730664\\
531	0.0125774432769447\\
532	0.0125897794886255\\
533	0.0126023667317016\\
534	0.0126152088468381\\
535	0.0126283102321478\\
536	0.0126416900885925\\
537	0.0126566327469089\\
538	0.0126721436449114\\
539	0.0126872593255511\\
540	0.0127008999274142\\
541	0.0127146303254365\\
542	0.0127282920377257\\
543	0.0127419703158875\\
544	0.012755856263815\\
545	0.0127699596130699\\
546	0.0127842824736393\\
547	0.0127988246644857\\
548	0.0128135851567183\\
549	0.0128285622593643\\
550	0.0128437531543193\\
551	0.0128591526408781\\
552	0.0128747495286285\\
553	0.0128905167409036\\
554	0.0129064587151011\\
555	0.0129226098866358\\
556	0.012940262197184\\
557	0.0129572514711595\\
558	0.0129732190329192\\
559	0.0129889692473418\\
560	0.0130048428449475\\
561	0.0130208678413011\\
562	0.0130370385205704\\
563	0.0130533485311235\\
564	0.0130697914384829\\
565	0.0130863606440552\\
566	0.0131035466640366\\
567	0.0131204071165013\\
568	0.0131367816466169\\
569	0.0131531501361782\\
570	0.0131696057273078\\
571	0.0131861379872111\\
572	0.0132027353983047\\
573	0.0132193852690821\\
574	0.0132360736315821\\
575	0.0132527851288778\\
576	0.0132695028913736\\
577	0.0132862084004766\\
578	0.0133028813379803\\
579	0.0133194994192331\\
580	0.0133360382079\\
581	0.0133524709099319\\
582	0.0133687681444567\\
583	0.0133848976903317\\
584	0.0134008242107329\\
585	0.0134165089689716\\
586	0.0134319095789188\\
587	0.0134469799156334\\
588	0.0134616705332227\\
589	0.0134759305304731\\
590	0.0134902483101722\\
591	0.0135046714122469\\
592	0.0135192196273521\\
593	0.0135339766714342\\
594	0.0135491998980231\\
595	0.0135656125153459\\
596	0.0135851815686143\\
597	0.0136131889078992\\
598	0.0136637438596031\\
599	0\\
600	0\\
};
\addplot [color=black!60!mycolor21,solid,forget plot]
  table[row sep=crcr]{%
1	0.011239041352767\\
2	0.0112390428002761\\
3	0.0112390442744595\\
4	0.0112390457758103\\
5	0.0112390473048309\\
6	0.0112390488620331\\
7	0.011239050447938\\
8	0.0112390520630764\\
9	0.0112390537079891\\
10	0.0112390553832268\\
11	0.0112390570893506\\
12	0.0112390588269317\\
13	0.0112390605965523\\
14	0.0112390623988051\\
15	0.0112390642342939\\
16	0.0112390661036339\\
17	0.0112390680074515\\
18	0.0112390699463848\\
19	0.0112390719210837\\
20	0.0112390739322104\\
21	0.0112390759804391\\
22	0.0112390780664569\\
23	0.0112390801909632\\
24	0.0112390823546708\\
25	0.0112390845583056\\
26	0.0112390868026069\\
27	0.011239089088328\\
28	0.011239091416236\\
29	0.0112390937871122\\
30	0.0112390962017527\\
31	0.0112390986609682\\
32	0.0112391011655845\\
33	0.0112391037164428\\
34	0.0112391063144001\\
35	0.0112391089603289\\
36	0.0112391116551185\\
37	0.0112391143996743\\
38	0.0112391171949187\\
39	0.0112391200417915\\
40	0.0112391229412496\\
41	0.0112391258942679\\
42	0.0112391289018394\\
43	0.0112391319649758\\
44	0.0112391350847073\\
45	0.0112391382620836\\
46	0.0112391414981736\\
47	0.0112391447940665\\
48	0.0112391481508715\\
49	0.0112391515697185\\
50	0.0112391550517585\\
51	0.0112391585981641\\
52	0.0112391622101294\\
53	0.0112391658888711\\
54	0.0112391696356283\\
55	0.0112391734516633\\
56	0.011239177338262\\
57	0.0112391812967343\\
58	0.0112391853284142\\
59	0.0112391894346609\\
60	0.0112391936168589\\
61	0.0112391978764184\\
62	0.0112392022147759\\
63	0.0112392066333948\\
64	0.0112392111337658\\
65	0.0112392157174072\\
66	0.0112392203858659\\
67	0.0112392251407175\\
68	0.0112392299835671\\
69	0.0112392349160495\\
70	0.0112392399398303\\
71	0.0112392450566061\\
72	0.0112392502681051\\
73	0.0112392555760878\\
74	0.0112392609823477\\
75	0.0112392664887115\\
76	0.0112392720970403\\
77	0.0112392778092297\\
78	0.011239283627211\\
79	0.0112392895529513\\
80	0.0112392955884546\\
81	0.0112393017357622\\
82	0.0112393079969537\\
83	0.0112393143741475\\
84	0.0112393208695016\\
85	0.0112393274852143\\
86	0.0112393342235249\\
87	0.0112393410867147\\
88	0.0112393480771077\\
89	0.0112393551970711\\
90	0.0112393624490167\\
91	0.0112393698354011\\
92	0.011239377358727\\
93	0.0112393850215438\\
94	0.0112393928264488\\
95	0.0112394007760878\\
96	0.0112394088731559\\
97	0.0112394171203988\\
98	0.0112394255206137\\
99	0.01123943407665\\
100	0.0112394427914104\\
101	0.011239451667852\\
102	0.0112394607089874\\
103	0.0112394699178854\\
104	0.0112394792976723\\
105	0.011239488851533\\
106	0.0112394985827122\\
107	0.011239508494515\\
108	0.0112395185903088\\
109	0.0112395288735239\\
110	0.011239539347655\\
111	0.0112395500162621\\
112	0.0112395608829721\\
113	0.0112395719514798\\
114	0.0112395832255493\\
115	0.0112395947090153\\
116	0.0112396064057844\\
117	0.0112396183198363\\
118	0.0112396304552257\\
119	0.011239642816083\\
120	0.0112396554066162\\
121	0.0112396682311124\\
122	0.011239681293939\\
123	0.0112396945995453\\
124	0.0112397081524644\\
125	0.0112397219573142\\
126	0.0112397360187993\\
127	0.011239750341713\\
128	0.0112397649309381\\
129	0.0112397797914495\\
130	0.0112397949283154\\
131	0.0112398103466992\\
132	0.0112398260518614\\
133	0.0112398420491613\\
134	0.0112398583440588\\
135	0.0112398749421167\\
136	0.0112398918490021\\
137	0.0112399090704889\\
138	0.0112399266124593\\
139	0.0112399444809064\\
140	0.0112399626819358\\
141	0.0112399812217681\\
142	0.011240000106741\\
143	0.0112400193433113\\
144	0.0112400389380573\\
145	0.0112400588976814\\
146	0.0112400792290118\\
147	0.0112400999390056\\
148	0.0112401210347507\\
149	0.0112401425234686\\
150	0.0112401644125168\\
151	0.0112401867093915\\
152	0.0112402094217301\\
153	0.0112402325573139\\
154	0.0112402561240709\\
155	0.0112402801300785\\
156	0.0112403045835667\\
157	0.0112403294929203\\
158	0.0112403548666826\\
159	0.0112403807135579\\
160	0.0112404070424147\\
161	0.0112404338622891\\
162	0.0112404611823875\\
163	0.0112404890120904\\
164	0.0112405173609551\\
165	0.0112405462387196\\
166	0.0112405756553059\\
167	0.0112406056208233\\
168	0.0112406361455723\\
169	0.011240667240048\\
170	0.011240698914944\\
171	0.0112407311811559\\
172	0.0112407640497856\\
173	0.0112407975321448\\
174	0.0112408316397595\\
175	0.0112408663843736\\
176	0.0112409017779533\\
177	0.0112409378326915\\
178	0.0112409745610118\\
179	0.0112410119755735\\
180	0.0112410500892752\\
181	0.0112410889152605\\
182	0.0112411284669217\\
183	0.0112411687579053\\
184	0.0112412098021164\\
185	0.011241251613724\\
186	0.0112412942071659\\
187	0.0112413375971539\\
188	0.0112413817986791\\
189	0.0112414268270174\\
190	0.0112414726977347\\
191	0.0112415194266926\\
192	0.0112415670300544\\
193	0.0112416155242904\\
194	0.0112416649261842\\
195	0.0112417152528389\\
196	0.0112417665216826\\
197	0.0112418187504754\\
198	0.0112418719573156\\
199	0.011241926160646\\
200	0.0112419813792609\\
201	0.0112420376323127\\
202	0.0112420949393192\\
203	0.01124215332017\\
204	0.0112422127951346\\
205	0.0112422733848691\\
206	0.011242335110424\\
207	0.011242397993252\\
208	0.0112424620552155\\
209	0.0112425273185951\\
210	0.011242593806097\\
211	0.0112426615408622\\
212	0.0112427305464744\\
213	0.0112428008469688\\
214	0.011242872466841\\
215	0.0112429454310562\\
216	0.0112430197650579\\
217	0.011243095494778\\
218	0.0112431726466456\\
219	0.0112432512475977\\
220	0.0112433313250885\\
221	0.0112434129070996\\
222	0.0112434960221511\\
223	0.0112435806993115\\
224	0.0112436669682089\\
225	0.0112437548590421\\
226	0.0112438444025918\\
227	0.0112439356302323\\
228	0.011244028573943\\
229	0.011244123266321\\
230	0.0112442197405929\\
231	0.0112443180306275\\
232	0.0112444181709489\\
233	0.0112445201967493\\
234	0.0112446241439026\\
235	0.0112447300489779\\
236	0.0112448379492539\\
237	0.0112449478827324\\
238	0.0112450598881539\\
239	0.0112451740050118\\
240	0.0112452902735678\\
241	0.0112454087348678\\
242	0.0112455294307574\\
243	0.0112456524038985\\
244	0.0112457776977859\\
245	0.011245905356764\\
246	0.0112460354260447\\
247	0.011246167951725\\
248	0.0112463029808051\\
249	0.0112464405612071\\
250	0.0112465807417942\\
251	0.0112467235723898\\
252	0.011246869103798\\
253	0.0112470173878234\\
254	0.0112471684772925\\
255	0.0112473224260747\\
256	0.0112474792891046\\
257	0.0112476391224038\\
258	0.0112478019831046\\
259	0.0112479679294726\\
260	0.0112481370209317\\
261	0.0112483093180882\\
262	0.011248484882756\\
263	0.0112486637779829\\
264	0.0112488460680768\\
265	0.0112490318186331\\
266	0.0112492210965625\\
267	0.0112494139701195\\
268	0.0112496105089323\\
269	0.0112498107840322\\
270	0.0112500148678849\\
271	0.0112502228344225\\
272	0.0112504347590754\\
273	0.0112506507188065\\
274	0.0112508707921453\\
275	0.0112510950592233\\
276	0.0112513236018102\\
277	0.0112515565033518\\
278	0.0112517938490079\\
279	0.0112520357256926\\
280	0.0112522822221144\\
281	0.0112525334288191\\
282	0.0112527894382324\\
283	0.0112530503447052\\
284	0.0112533162445595\\
285	0.0112535872361362\\
286	0.0112538634198447\\
287	0.0112541448982132\\
288	0.0112544317759423\\
289	0.0112547241599591\\
290	0.0112550221594742\\
291	0.0112553258860407\\
292	0.0112556354536151\\
293	0.0112559509786208\\
294	0.0112562725800151\\
295	0.0112566003793575\\
296	0.0112569345008823\\
297	0.011257275071574\\
298	0.0112576222212465\\
299	0.0112579760826259\\
300	0.0112583367914379\\
301	0.0112587044864993\\
302	0.0112590793098156\\
303	0.0112594614066828\\
304	0.0112598509257967\\
305	0.011260248019368\\
306	0.0112606528432454\\
307	0.0112610655570471\\
308	0.011261486324301\\
309	0.0112619153125962\\
310	0.0112623526937454\\
311	0.0112627986439604\\
312	0.0112632533440423\\
313	0.0112637169795881\\
314	0.0112641897412159\\
315	0.0112646718248108\\
316	0.0112651634317953\\
317	0.0112656647694251\\
318	0.0112661760511175\\
319	0.011266697496813\\
320	0.0112672293333778\\
321	0.011267771795051\\
322	0.0112683251239445\\
323	0.0112688895706017\\
324	0.011269465394626\\
325	0.011270052865386\\
326	0.0112706522628133\\
327	0.011271263878303\\
328	0.0112718880157353\\
329	0.0112725249926361\\
330	0.0112731751414983\\
331	0.011273838811289\\
332	0.0112745163691734\\
333	0.0112752082024888\\
334	0.0112759147210093\\
335	0.011276636359551\\
336	0.0112773735809702\\
337	0.0112781268796244\\
338	0.011278896785372\\
339	0.0112796838682095\\
340	0.0112804887436729\\
341	0.0112813120791945\\
342	0.0112821546017914\\
343	0.0112830171080404\\
344	0.0112839004793356\\
345	0.0112848057129946\\
346	0.0112857590681972\\
347	0.0112868000115571\\
348	0.0112878610524782\\
349	0.0112889425743934\\
350	0.0112900449666733\\
351	0.0112911686238055\\
352	0.0112923139443557\\
353	0.0112934813317574\\
354	0.0112946712069793\\
355	0.0112958840735392\\
356	0.0112971203706034\\
357	0.0112983805387522\\
358	0.0112996650256684\\
359	0.011300974286136\\
360	0.0113023087820242\\
361	0.0113036689822559\\
362	0.0113050553627564\\
363	0.0113064684063821\\
364	0.0113079086028251\\
365	0.01130937644849\\
366	0.0113108724463407\\
367	0.0113123971057115\\
368	0.0113139509420787\\
369	0.0113155344767859\\
370	0.0113171482367189\\
371	0.0113187927539204\\
372	0.0113204685651377\\
373	0.0113221762112941\\
374	0.0113239162368713\\
375	0.0113256891891924\\
376	0.0113274956175895\\
377	0.0113293360724393\\
378	0.0113312111040474\\
379	0.0113331212613591\\
380	0.0113350670904709\\
381	0.011337049132911\\
382	0.0113390679236565\\
383	0.0113411239888445\\
384	0.0113432178431313\\
385	0.0113453499866442\\
386	0.0113475209014642\\
387	0.0113497310475676\\
388	0.0113519808581482\\
389	0.0113542707342486\\
390	0.0113566010386649\\
391	0.0113589720892446\\
392	0.0113613841522097\\
393	0.0113638374378285\\
394	0.011366332106384\\
395	0.01136886831157\\
396	0.0113714463758554\\
397	0.0113740674361707\\
398	0.0113766836405187\\
399	0.0113792123482623\\
400	0.0113817903305588\\
401	0.0113844184879349\\
402	0.0113870977339388\\
403	0.0113898289950864\\
404	0.0113926132106861\\
405	0.011395451332449\\
406	0.0113983443238048\\
407	0.0114012931591304\\
408	0.0114042988243542\\
409	0.0114073623244064\\
410	0.0114104847127292\\
411	0.0114136670790637\\
412	0.0114169104751911\\
413	0.0114202159635445\\
414	0.0114235846263744\\
415	0.0114270175662835\\
416	0.0114305159043431\\
417	0.0114340807812689\\
418	0.0114377133581531\\
419	0.0114414148151829\\
420	0.0114451863522635\\
421	0.0114490292045197\\
422	0.011452944643883\\
423	0.0114569339804359\\
424	0.0114609984057528\\
425	0.0114651387043717\\
426	0.0114693539460858\\
427	0.0114736198897073\\
428	0.011477933198766\\
429	0.0114823326681331\\
430	0.0114868199698509\\
431	0.0114913968077287\\
432	0.0114960649180925\\
433	0.0115008260706244\\
434	0.0115056820690209\\
435	0.0115106347517691\\
436	0.0115156859929351\\
437	0.0115208377029543\\
438	0.0115260918294043\\
439	0.0115314503577167\\
440	0.0115369153117144\\
441	0.0115424887536511\\
442	0.0115481727828192\\
443	0.011553969530018\\
444	0.0115598811402087\\
445	0.0115659097227945\\
446	0.0115720572205754\\
447	0.0115783251137235\\
448	0.0115847165823101\\
449	0.0115912345328617\\
450	0.0115978812282788\\
451	0.011604658955138\\
452	0.0116115700225534\\
453	0.0116186167608869\\
454	0.0116258015203381\\
455	0.0116331266693448\\
456	0.0116405945927863\\
457	0.0116482076899403\\
458	0.0116559683720931\\
459	0.0116638790595942\\
460	0.0116719421779993\\
461	0.0116801601529917\\
462	0.0116885354053477\\
463	0.0116970703555997\\
464	0.0117057674803261\\
465	0.0117146291956841\\
466	0.0117236579083083\\
467	0.0117328560135997\\
468	0.0117422258916711\\
469	0.0117517699022312\\
470	0.0117614903773437\\
471	0.0117713896097924\\
472	0.0117814698319882\\
473	0.011791733172256\\
474	0.0118021815560944\\
475	0.0118128167528688\\
476	0.0118236400101319\\
477	0.0118346485947756\\
478	0.0118458276192941\\
479	0.0118571954671726\\
480	0.0118687733282496\\
481	0.0118805600567881\\
482	0.0118925619641299\\
483	0.0119047830617299\\
484	0.0119172242011226\\
485	0.0119298854033905\\
486	0.0119427725223105\\
487	0.0119558920220597\\
488	0.0119692511356302\\
489	0.0119828581950146\\
490	0.0119967236138377\\
491	0.0120115765680266\\
492	0.0120268598345071\\
493	0.0120422451105933\\
494	0.0120577189585024\\
495	0.0120732676044461\\
496	0.0120888795717372\\
497	0.0121041561963896\\
498	0.0121190235880096\\
499	0.0121340269579018\\
500	0.0121491673421512\\
501	0.0121644366243783\\
502	0.0121798048427859\\
503	0.0121953751414982\\
504	0.0122112964868549\\
505	0.0122279984447559\\
506	0.0122530267148885\\
507	0.0122826169348248\\
508	0.0123113768672566\\
509	0.0123388641799021\\
510	0.0123555690371973\\
511	0.012369964685551\\
512	0.0123832779073909\\
513	0.0123952743523283\\
514	0.0124047811299181\\
515	0.0124141527954976\\
516	0.0124234312987462\\
517	0.0124326974282635\\
518	0.0124421053633064\\
519	0.0124516742957978\\
520	0.0124614236240149\\
521	0.0124713696781674\\
522	0.0124815191433029\\
523	0.0124918780997577\\
524	0.0125024518623407\\
525	0.0125132449070785\\
526	0.0125242615999327\\
527	0.0125355062107447\\
528	0.0125469829271314\\
529	0.0125586959141213\\
530	0.0125706495456539\\
531	0.0125828481043012\\
532	0.0125952945913644\\
533	0.0126079997885039\\
534	0.0126212665972596\\
535	0.0126363143644848\\
536	0.012651210100543\\
537	0.0126648475890322\\
538	0.0126779993348058\\
539	0.012691187853797\\
540	0.0127042330856999\\
541	0.0127174825159092\\
542	0.0127309431386996\\
543	0.0127446210527922\\
544	0.0127585169094392\\
545	0.012772630736796\\
546	0.0127869621423264\\
547	0.0128015103267207\\
548	0.0128162740111645\\
549	0.0128312512294928\\
550	0.0128464387796428\\
551	0.0128618307298279\\
552	0.0128774144513983\\
553	0.0128931568941647\\
554	0.0129093833575267\\
555	0.0129269612101602\\
556	0.0129429560467004\\
557	0.0129585621734848\\
558	0.0129740989886379\\
559	0.0129897947280717\\
560	0.0130056457798629\\
561	0.0130216467641238\\
562	0.0130377919156873\\
563	0.0130540753963924\\
564	0.013070491452824\\
565	0.0130874896882337\\
566	0.0131042520123742\\
567	0.0131205097152076\\
568	0.0131367816514045\\
569	0.0131531501362776\\
570	0.0131696057273463\\
571	0.0131861379872302\\
572	0.0132027353983141\\
573	0.0132193852690865\\
574	0.013236073631584\\
575	0.0132527851288787\\
576	0.0132695028913739\\
577	0.0132862084004767\\
578	0.0133028813379804\\
579	0.0133194994192331\\
580	0.0133360382079\\
581	0.0133524709099319\\
582	0.0133687681444567\\
583	0.0133848976903317\\
584	0.0134008242107329\\
585	0.0134165089689716\\
586	0.0134319095789188\\
587	0.0134469799156334\\
588	0.0134616705332227\\
589	0.0134759305304731\\
590	0.0134902483101722\\
591	0.0135046714122469\\
592	0.0135192196273521\\
593	0.0135339766714342\\
594	0.0135491998980231\\
595	0.0135656125153459\\
596	0.0135851815686143\\
597	0.0136131889078992\\
598	0.0136637438596031\\
599	0\\
600	0\\
};
\addplot [color=black!80!mycolor21,solid,forget plot]
  table[row sep=crcr]{%
1	0.0112465140737006\\
2	0.0112465158814963\\
3	0.0112465177224853\\
4	0.0112465195972787\\
5	0.0112465215064991\\
6	0.0112465234507803\\
7	0.0112465254307683\\
8	0.0112465274471205\\
9	0.0112465295005068\\
10	0.0112465315916093\\
11	0.0112465337211228\\
12	0.011246535889755\\
13	0.0112465380982265\\
14	0.0112465403472714\\
15	0.0112465426376371\\
16	0.0112465449700852\\
17	0.0112465473453911\\
18	0.0112465497643446\\
19	0.0112465522277501\\
20	0.0112465547364269\\
21	0.0112465572912095\\
22	0.0112465598929477\\
23	0.0112465625425072\\
24	0.0112465652407694\\
25	0.0112465679886323\\
26	0.0112465707870105\\
27	0.0112465736368354\\
28	0.0112465765390555\\
29	0.0112465794946372\\
30	0.0112465825045645\\
31	0.0112465855698397\\
32	0.0112465886914837\\
33	0.0112465918705363\\
34	0.0112465951080563\\
35	0.0112465984051224\\
36	0.0112466017628331\\
37	0.0112466051823073\\
38	0.0112466086646847\\
39	0.0112466122111258\\
40	0.0112466158228129\\
41	0.0112466195009501\\
42	0.0112466232467637\\
43	0.0112466270615028\\
44	0.0112466309464397\\
45	0.01124663490287\\
46	0.0112466389321137\\
47	0.0112466430355148\\
48	0.0112466472144426\\
49	0.0112466514702916\\
50	0.011246655804482\\
51	0.0112466602184605\\
52	0.0112466647137005\\
53	0.0112466692917029\\
54	0.0112466739539961\\
55	0.0112466787021372\\
56	0.0112466835377118\\
57	0.0112466884623351\\
58	0.0112466934776523\\
59	0.0112466985853388\\
60	0.0112467037871013\\
61	0.0112467090846782\\
62	0.0112467144798401\\
63	0.0112467199743901\\
64	0.0112467255701653\\
65	0.0112467312690364\\
66	0.0112467370729091\\
67	0.0112467429837243\\
68	0.0112467490034589\\
69	0.0112467551341265\\
70	0.0112467613777781\\
71	0.0112467677365027\\
72	0.0112467742124282\\
73	0.0112467808077217\\
74	0.0112467875245909\\
75	0.0112467943652841\\
76	0.0112468013320916\\
77	0.011246808427346\\
78	0.0112468156534235\\
79	0.0112468230127442\\
80	0.011246830507773\\
81	0.0112468381410208\\
82	0.0112468459150452\\
83	0.011246853832451\\
84	0.0112468618958916\\
85	0.0112468701080696\\
86	0.0112468784717378\\
87	0.0112468869897002\\
88	0.0112468956648127\\
89	0.0112469044999845\\
90	0.0112469134981788\\
91	0.0112469226624137\\
92	0.0112469319957635\\
93	0.0112469415013597\\
94	0.0112469511823919\\
95	0.0112469610421089\\
96	0.0112469710838201\\
97	0.0112469813108963\\
98	0.0112469917267709\\
99	0.0112470023349413\\
100	0.0112470131389698\\
101	0.011247024142485\\
102	0.0112470353491829\\
103	0.0112470467628283\\
104	0.0112470583872561\\
105	0.0112470702263723\\
106	0.0112470822841558\\
107	0.0112470945646594\\
108	0.0112471070720112\\
109	0.0112471198104163\\
110	0.011247132784158\\
111	0.0112471459975992\\
112	0.011247159455184\\
113	0.0112471731614394\\
114	0.0112471871209764\\
115	0.0112472013384919\\
116	0.0112472158187704\\
117	0.0112472305666853\\
118	0.0112472455872007\\
119	0.0112472608853734\\
120	0.0112472764663539\\
121	0.011247292335389\\
122	0.0112473084978231\\
123	0.0112473249591001\\
124	0.0112473417247653\\
125	0.0112473588004673\\
126	0.0112473761919599\\
127	0.0112473939051042\\
128	0.0112474119458703\\
129	0.0112474303203398\\
130	0.0112474490347075\\
131	0.0112474680952834\\
132	0.0112474875084956\\
133	0.0112475072808915\\
134	0.0112475274191409\\
135	0.0112475479300378\\
136	0.0112475688205028\\
137	0.0112475900975856\\
138	0.0112476117684673\\
139	0.011247633840463\\
140	0.0112476563210241\\
141	0.0112476792177409\\
142	0.0112477025383454\\
143	0.0112477262907138\\
144	0.0112477504828692\\
145	0.0112477751229843\\
146	0.0112478002193843\\
147	0.0112478257805498\\
148	0.0112478518151196\\
149	0.0112478783318937\\
150	0.0112479053398361\\
151	0.0112479328480786\\
152	0.0112479608659229\\
153	0.0112479894028446\\
154	0.0112480184684962\\
155	0.0112480480727103\\
156	0.011248078225503\\
157	0.0112481089370777\\
158	0.011248140217828\\
159	0.0112481720783417\\
160	0.0112482045294044\\
161	0.0112482375820031\\
162	0.01124827124733\\
163	0.0112483055367864\\
164	0.0112483404619864\\
165	0.0112483760347613\\
166	0.0112484122671634\\
167	0.0112484491714701\\
168	0.0112484867601884\\
169	0.0112485250460588\\
170	0.01124856404206\\
171	0.0112486037614133\\
172	0.0112486442175872\\
173	0.0112486854243018\\
174	0.0112487273955338\\
175	0.0112487701455211\\
176	0.0112488136887681\\
177	0.01124885804005\\
178	0.0112489032144185\\
179	0.0112489492272069\\
180	0.011248996094035\\
181	0.0112490438308151\\
182	0.0112490924537568\\
183	0.0112491419793732\\
184	0.0112491924244862\\
185	0.0112492438062323\\
186	0.011249296142069\\
187	0.0112493494497801\\
188	0.0112494037474825\\
189	0.011249459053632\\
190	0.0112495153870298\\
191	0.0112495727668291\\
192	0.0112496312125417\\
193	0.0112496907440445\\
194	0.0112497513815866\\
195	0.0112498131457964\\
196	0.0112498760576884\\
197	0.0112499401386705\\
198	0.0112500054105519\\
199	0.01125007189555\\
200	0.0112501396162984\\
201	0.0112502085958546\\
202	0.0112502788577082\\
203	0.0112503504257888\\
204	0.0112504233244743\\
205	0.0112504975785995\\
206	0.0112505732134646\\
207	0.0112506502548437\\
208	0.0112507287289943\\
209	0.0112508086626659\\
210	0.0112508900831093\\
211	0.0112509730180863\\
212	0.0112510574958793\\
213	0.0112511435453007\\
214	0.0112512311957033\\
215	0.0112513204769905\\
216	0.0112514114196262\\
217	0.0112515040546458\\
218	0.011251598413667\\
219	0.0112516945289003\\
220	0.0112517924331608\\
221	0.0112518921598789\\
222	0.0112519937431127\\
223	0.0112520972175589\\
224	0.0112522026185657\\
225	0.0112523099821446\\
226	0.0112524193449831\\
227	0.0112525307444572\\
228	0.0112526442186446\\
229	0.0112527598063381\\
230	0.0112528775470586\\
231	0.0112529974810693\\
232	0.0112531196493895\\
233	0.0112532440938087\\
234	0.0112533708569017\\
235	0.0112534999820427\\
236	0.011253631513421\\
237	0.0112537654960559\\
238	0.0112539019758129\\
239	0.0112540409994192\\
240	0.0112541826144801\\
241	0.011254326869496\\
242	0.0112544738138786\\
243	0.0112546234979686\\
244	0.0112547759730534\\
245	0.0112549312913845\\
246	0.011255089506196\\
247	0.0112552506717232\\
248	0.0112554148432215\\
249	0.0112555820769857\\
250	0.0112557524303696\\
251	0.0112559259618061\\
252	0.0112561027308276\\
253	0.0112562827980867\\
254	0.0112564662253776\\
255	0.0112566530756575\\
256	0.0112568434130687\\
257	0.0112570373029611\\
258	0.011257234811915\\
259	0.011257436007764\\
260	0.0112576409596196\\
261	0.0112578497378943\\
262	0.0112580624143271\\
263	0.0112582790620081\\
264	0.0112584997554041\\
265	0.0112587245703849\\
266	0.0112589535842494\\
267	0.0112591868757529\\
268	0.0112594245251342\\
269	0.0112596666141443\\
270	0.0112599132260741\\
271	0.0112601644457839\\
272	0.0112604203597327\\
273	0.0112606810560084\\
274	0.0112609466243581\\
275	0.0112612171562196\\
276	0.0112614927447525\\
277	0.0112617734848708\\
278	0.0112620594732754\\
279	0.0112623508084872\\
280	0.0112626475908807\\
281	0.0112629499227187\\
282	0.0112632579081863\\
283	0.0112635716534267\\
284	0.0112638912665764\\
285	0.0112642168578017\\
286	0.0112645485393349\\
287	0.0112648864255112\\
288	0.0112652306328062\\
289	0.0112655812798733\\
290	0.0112659384875815\\
291	0.0112663023790532\\
292	0.0112666730797028\\
293	0.0112670507172746\\
294	0.0112674354218808\\
295	0.0112678273260394\\
296	0.0112682265647121\\
297	0.0112686332753413\\
298	0.0112690475978865\\
299	0.0112694696748603\\
300	0.0112698996513627\\
301	0.0112703376751146\\
302	0.0112707838964896\\
303	0.0112712384685432\\
304	0.0112717015470409\\
305	0.0112721732904816\\
306	0.0112726538601196\\
307	0.0112731434199812\\
308	0.0112736421368771\\
309	0.0112741501804091\\
310	0.0112746677229712\\
311	0.0112751949397417\\
312	0.0112757320086684\\
313	0.0112762791104423\\
314	0.0112768364284604\\
315	0.011277404148775\\
316	0.0112779824600271\\
317	0.0112785715533625\\
318	0.0112791716223264\\
319	0.0112797828627351\\
320	0.0112804054725191\\
321	0.0112810396515359\\
322	0.0112816856013452\\
323	0.0112823435249429\\
324	0.0112830136264467\\
325	0.011283696110726\\
326	0.011284391182968\\
327	0.0112850990481702\\
328	0.0112858199105477\\
329	0.0112865539728434\\
330	0.0112873014355252\\
331	0.0112880624958532\\
332	0.0112888373467973\\
333	0.0112896261757813\\
334	0.0112904291632275\\
335	0.0112912464808683\\
336	0.011292078289789\\
337	0.0112929247381596\\
338	0.0112937859586129\\
339	0.0112946620652585\\
340	0.0112955531504619\\
341	0.0112964592820748\\
342	0.011297380503763\\
343	0.0112983168477426\\
344	0.0112992684081234\\
345	0.0113002356615827\\
346	0.0113011981866483\\
347	0.0113021229912858\\
348	0.011303066083019\\
349	0.011304027822147\\
350	0.0113050085755559\\
351	0.0113060087166638\\
352	0.0113070286254258\\
353	0.0113080686889163\\
354	0.0113091293041964\\
355	0.0113102108877347\\
356	0.0113113138503988\\
357	0.0113124386094914\\
358	0.0113135855898635\\
359	0.0113147552240034\\
360	0.011315947952124\\
361	0.011317164222248\\
362	0.0113184044902896\\
363	0.0113196692201357\\
364	0.0113209588837223\\
365	0.0113222739611102\\
366	0.0113236149405574\\
367	0.0113249823185897\\
368	0.0113263766000688\\
369	0.01132779829826\\
370	0.0113292479348981\\
371	0.0113307260402538\\
372	0.0113322331532014\\
373	0.0113337698212891\\
374	0.0113353366008134\\
375	0.011336934056901\\
376	0.0113385627636001\\
377	0.0113402233039846\\
378	0.0113419162702771\\
379	0.0113436422639938\\
380	0.0113454018961199\\
381	0.0113471957873225\\
382	0.0113490245682099\\
383	0.0113508888796513\\
384	0.0113527893731675\\
385	0.0113547267114124\\
386	0.0113567015687627\\
387	0.0113587146320412\\
388	0.0113607666013978\\
389	0.0113628581913758\\
390	0.0113649901321715\\
391	0.0113671631710247\\
392	0.0113693780733711\\
393	0.0113716356222197\\
394	0.0113739366096414\\
395	0.0113762817964131\\
396	0.0113786717458\\
397	0.0113811061612395\\
398	0.0113835744974096\\
399	0.0113860631236216\\
400	0.0113886028964451\\
401	0.0113911948733196\\
402	0.011393840134293\\
403	0.0113965397826018\\
404	0.0113992949452713\\
405	0.0114021067737427\\
406	0.0114049764445453\\
407	0.0114079051600715\\
408	0.0114108941495445\\
409	0.0114139446701564\\
410	0.0114170580072881\\
411	0.0114202354742386\\
412	0.0114234784141378\\
413	0.0114267882010151\\
414	0.0114301662407148\\
415	0.0114336139718114\\
416	0.0114371328666062\\
417	0.0114407244321219\\
418	0.0114443902110759\\
419	0.0114481317828855\\
420	0.0114519507646593\\
421	0.0114558488111916\\
422	0.0114598276126126\\
423	0.0114638888840721\\
424	0.0114680343388791\\
425	0.0114722656124757\\
426	0.0114765840818516\\
427	0.011480991769219\\
428	0.0114854913360267\\
429	0.0114900846649395\\
430	0.0114947736731113\\
431	0.011499560312358\\
432	0.0115044465692753\\
433	0.0115094344652891\\
434	0.0115145260566377\\
435	0.0115197234342728\\
436	0.0115250287236716\\
437	0.0115304440845463\\
438	0.0115359717104374\\
439	0.0115416138281704\\
440	0.0115473726971375\\
441	0.0115532506083381\\
442	0.011559249883036\\
443	0.0115653728707932\\
444	0.0115716219466123\\
445	0.0115779995077365\\
446	0.0115845079753972\\
447	0.0115911498259805\\
448	0.0115979275453705\\
449	0.0116048436257181\\
450	0.0116119005789811\\
451	0.0116191009345577\\
452	0.0116264472366378\\
453	0.0116339420412495\\
454	0.0116415879129732\\
455	0.0116493874212933\\
456	0.0116573431365553\\
457	0.0116654576254935\\
458	0.0116737334462865\\
459	0.0116821731431068\\
460	0.0116907792401442\\
461	0.0116995542351541\\
462	0.0117085005926779\\
463	0.0117176207368288\\
464	0.0117269170404246\\
465	0.0117363918167596\\
466	0.0117460473087593\\
467	0.0117558856761953\\
468	0.0117659089793798\\
469	0.0117761191552671\\
470	0.0117865179759158\\
471	0.0117971069657494\\
472	0.0118078872277539\\
473	0.0118188590973616\\
474	0.0118300214282654\\
475	0.0118413650894999\\
476	0.0118528811076133\\
477	0.0118646067501865\\
478	0.011876540298591\\
479	0.0118886848485102\\
480	0.0119010507400625\\
481	0.0119136454255147\\
482	0.0119264770762642\\
483	0.0119395550052135\\
484	0.0119529685670179\\
485	0.0119675075241263\\
486	0.0119821595371004\\
487	0.0119969122178181\\
488	0.0120117518763348\\
489	0.012026664669177\\
490	0.0120416408877232\\
491	0.0120560576879281\\
492	0.0120703513483297\\
493	0.0120847749973027\\
494	0.0120993227183574\\
495	0.0121139883995913\\
496	0.0121287638935181\\
497	0.0121435734228761\\
498	0.0121584259457046\\
499	0.0121733979081672\\
500	0.0121885337910087\\
501	0.0122042508721673\\
502	0.0122243248823405\\
503	0.012253126402231\\
504	0.0122811631091917\\
505	0.0123080968354763\\
506	0.0123271506439781\\
507	0.0123410130276573\\
508	0.0123538565201679\\
509	0.0123655060424077\\
510	0.0123746905951724\\
511	0.0123834342839682\\
512	0.0123920759355795\\
513	0.0124006804561605\\
514	0.0124094164154852\\
515	0.0124183022457615\\
516	0.0124273565678601\\
517	0.0124365953200095\\
518	0.0124460248223508\\
519	0.0124556508094527\\
520	0.0124654782759953\\
521	0.0124755114782222\\
522	0.0124857545724964\\
523	0.0124962116296147\\
524	0.012506886662141\\
525	0.0125177836824298\\
526	0.0125289067221078\\
527	0.0125402598750669\\
528	0.012551847360068\\
529	0.0125636734422102\\
530	0.0125757412289918\\
531	0.0125880677380157\\
532	0.0126016825351311\\
533	0.0126161648684101\\
534	0.0126302379574037\\
535	0.0126428367023998\\
536	0.0126555156544702\\
537	0.0126680745095018\\
538	0.0126807062746515\\
539	0.0126935408327277\\
540	0.0127065889381891\\
541	0.0127198518206078\\
542	0.0127333302740618\\
543	0.0127470246364701\\
544	0.0127609349332875\\
545	0.0127750608635254\\
546	0.0127894017721699\\
547	0.0128039566102232\\
548	0.0128187238632145\\
549	0.0128337013809305\\
550	0.0128488858805986\\
551	0.0128642713518595\\
552	0.0128798436938607\\
553	0.0128963032992081\\
554	0.0129132511743546\\
555	0.0129287107054386\\
556	0.0129439677980492\\
557	0.0129593224543217\\
558	0.0129748400360881\\
559	0.012990516155719\\
560	0.0130063459378701\\
561	0.013022324171331\\
562	0.0130384455273938\\
563	0.0130547049389776\\
564	0.0130714646410262\\
565	0.0130881947021808\\
566	0.0131043428556408\\
567	0.0131205097155993\\
568	0.0131367816514175\\
569	0.0131531501362831\\
570	0.0131696057273491\\
571	0.0131861379872315\\
572	0.0132027353983147\\
573	0.0132193852690868\\
574	0.0132360736315842\\
575	0.0132527851288787\\
576	0.0132695028913739\\
577	0.0132862084004767\\
578	0.0133028813379804\\
579	0.0133194994192331\\
580	0.0133360382079\\
581	0.0133524709099319\\
582	0.0133687681444567\\
583	0.0133848976903317\\
584	0.0134008242107329\\
585	0.0134165089689716\\
586	0.0134319095789188\\
587	0.0134469799156334\\
588	0.0134616705332227\\
589	0.0134759305304731\\
590	0.0134902483101722\\
591	0.0135046714122469\\
592	0.0135192196273521\\
593	0.0135339766714342\\
594	0.0135491998980231\\
595	0.0135656125153459\\
596	0.0135851815686143\\
597	0.0136131889078992\\
598	0.0136637438596031\\
599	0\\
600	0\\
};
\addplot [color=black,solid,forget plot]
  table[row sep=crcr]{%
1	0.0112576820807805\\
2	0.0112576836631175\\
3	0.0112576852745897\\
4	0.0112576869157352\\
5	0.0112576885871022\\
6	0.0112576902892489\\
7	0.011257692022744\\
8	0.0112576937881667\\
9	0.011257695586107\\
10	0.0112576974171655\\
11	0.0112576992819545\\
12	0.0112577011810972\\
13	0.0112577031152285\\
14	0.0112577050849952\\
15	0.0112577070910559\\
16	0.0112577091340816\\
17	0.0112577112147556\\
18	0.0112577133337738\\
19	0.0112577154918454\\
20	0.0112577176896923\\
21	0.01125771992805\\
22	0.0112577222076678\\
23	0.0112577245293086\\
24	0.0112577268937497\\
25	0.0112577293017827\\
26	0.0112577317542139\\
27	0.0112577342518647\\
28	0.0112577367955716\\
29	0.0112577393861866\\
30	0.0112577420245777\\
31	0.011257744711629\\
32	0.0112577474482407\\
33	0.0112577502353301\\
34	0.0112577530738314\\
35	0.0112577559646961\\
36	0.0112577589088935\\
37	0.0112577619074107\\
38	0.0112577649612534\\
39	0.0112577680714457\\
40	0.011257771239031\\
41	0.011257774465072\\
42	0.011257777750651\\
43	0.0112577810968705\\
44	0.0112577845048537\\
45	0.0112577879757444\\
46	0.0112577915107077\\
47	0.0112577951109305\\
48	0.0112577987776216\\
49	0.0112578025120125\\
50	0.0112578063153574\\
51	0.0112578101889339\\
52	0.0112578141340432\\
53	0.011257818152011\\
54	0.0112578222441875\\
55	0.0112578264119479\\
56	0.0112578306566932\\
57	0.0112578349798503\\
58	0.0112578393828726\\
59	0.0112578438672409\\
60	0.0112578484344631\\
61	0.0112578530860754\\
62	0.0112578578236426\\
63	0.0112578626487585\\
64	0.0112578675630467\\
65	0.0112578725681608\\
66	0.0112578776657855\\
67	0.0112578828576367\\
68	0.0112578881454621\\
69	0.0112578935310423\\
70	0.0112578990161906\\
71	0.0112579046027544\\
72	0.0112579102926155\\
73	0.0112579160876905\\
74	0.011257921989932\\
75	0.0112579280013287\\
76	0.0112579341239065\\
77	0.0112579403597291\\
78	0.0112579467108984\\
79	0.0112579531795557\\
80	0.0112579597678822\\
81	0.0112579664780995\\
82	0.0112579733124708\\
83	0.0112579802733016\\
84	0.01125798736294\\
85	0.0112579945837782\\
86	0.0112580019382528\\
87	0.0112580094288459\\
88	0.011258017058086\\
89	0.0112580248285484\\
90	0.0112580327428568\\
91	0.0112580408036835\\
92	0.0112580490137509\\
93	0.011258057375832\\
94	0.0112580658927517\\
95	0.0112580745673872\\
96	0.0112580834026699\\
97	0.0112580924015855\\
98	0.0112581015671754\\
99	0.0112581109025381\\
100	0.0112581204108295\\
101	0.0112581300952646\\
102	0.0112581399591183\\
103	0.0112581500057267\\
104	0.0112581602384882\\
105	0.0112581706608644\\
106	0.0112581812763818\\
107	0.0112581920886325\\
108	0.0112582031012757\\
109	0.0112582143180391\\
110	0.0112582257427198\\
111	0.0112582373791857\\
112	0.0112582492313773\\
113	0.0112582613033083\\
114	0.0112582735990675\\
115	0.0112582861228201\\
116	0.0112582988788091\\
117	0.0112583118713566\\
118	0.0112583251048657\\
119	0.0112583385838214\\
120	0.0112583523127927\\
121	0.011258366296434\\
122	0.0112583805394865\\
123	0.0112583950467801\\
124	0.011258409823235\\
125	0.0112584248738632\\
126	0.0112584402037703\\
127	0.0112584558181575\\
128	0.0112584717223232\\
129	0.0112584879216648\\
130	0.0112585044216804\\
131	0.011258521227971\\
132	0.0112585383462425\\
133	0.0112585557823071\\
134	0.0112585735420859\\
135	0.0112585916316107\\
136	0.011258610057026\\
137	0.0112586288245911\\
138	0.0112586479406826\\
139	0.011258667411796\\
140	0.0112586872445485\\
141	0.0112587074456811\\
142	0.0112587280220607\\
143	0.0112587489806826\\
144	0.0112587703286731\\
145	0.0112587920732918\\
146	0.0112588142219338\\
147	0.0112588367821329\\
148	0.0112588597615637\\
149	0.0112588831680442\\
150	0.0112589070095388\\
151	0.0112589312941609\\
152	0.0112589560301758\\
153	0.0112589812260031\\
154	0.0112590068902203\\
155	0.0112590330315652\\
156	0.0112590596589391\\
157	0.01125908678141\\
158	0.0112591144082153\\
159	0.0112591425487655\\
160	0.0112591712126471\\
161	0.0112592004096261\\
162	0.011259230149651\\
163	0.0112592604428568\\
164	0.011259291299568\\
165	0.0112593227303022\\
166	0.0112593547457741\\
167	0.0112593873568988\\
168	0.0112594205747956\\
169	0.011259454410792\\
170	0.0112594888764274\\
171	0.0112595239834574\\
172	0.0112595597438574\\
173	0.0112595961698268\\
174	0.0112596332737938\\
175	0.0112596710684187\\
176	0.0112597095665989\\
177	0.0112597487814733\\
178	0.0112597887264267\\
179	0.0112598294150941\\
180	0.0112598708613661\\
181	0.0112599130793928\\
182	0.0112599560835895\\
183	0.011259999888641\\
184	0.011260044509507\\
185	0.0112600899614271\\
186	0.0112601362599261\\
187	0.0112601834208193\\
188	0.0112602314602177\\
189	0.0112602803945341\\
190	0.011260330240488\\
191	0.0112603810151119\\
192	0.0112604327357569\\
193	0.0112604854200987\\
194	0.0112605390861434\\
195	0.0112605937522342\\
196	0.0112606494370571\\
197	0.0112607061596477\\
198	0.0112607639393976\\
199	0.0112608227960609\\
200	0.011260882749761\\
201	0.0112609438209977\\
202	0.0112610060306537\\
203	0.0112610694000025\\
204	0.0112611339507147\\
205	0.0112611997048663\\
206	0.0112612666849456\\
207	0.0112613349138612\\
208	0.0112614044149494\\
209	0.0112614752119827\\
210	0.0112615473291776\\
211	0.0112616207912027\\
212	0.0112616956231872\\
213	0.0112617718507296\\
214	0.011261849499906\\
215	0.0112619285972795\\
216	0.0112620091699085\\
217	0.0112620912453566\\
218	0.0112621748517012\\
219	0.0112622600175435\\
220	0.0112623467720183\\
221	0.0112624351448031\\
222	0.0112625251661289\\
223	0.0112626168667898\\
224	0.0112627102781538\\
225	0.0112628054321732\\
226	0.0112629023613951\\
227	0.0112630010989727\\
228	0.0112631016786763\\
229	0.0112632041349044\\
230	0.0112633085026956\\
231	0.0112634148177399\\
232	0.011263523116391\\
233	0.0112636334356782\\
234	0.0112637458133186\\
235	0.0112638602877301\\
236	0.0112639768980438\\
237	0.011264095684117\\
238	0.0112642166865466\\
239	0.0112643399466824\\
240	0.0112644655066406\\
241	0.0112645934093183\\
242	0.0112647236984067\\
243	0.0112648564184064\\
244	0.0112649916146413\\
245	0.0112651293332739\\
246	0.0112652696213201\\
247	0.0112654125266648\\
248	0.0112655580980772\\
249	0.0112657063852272\\
250	0.0112658574387008\\
251	0.0112660113100169\\
252	0.0112661680516442\\
253	0.0112663277170174\\
254	0.011266490360555\\
255	0.0112666560376767\\
256	0.0112668248048206\\
257	0.0112669967194621\\
258	0.0112671718401313\\
259	0.0112673502264324\\
260	0.0112675319390618\\
261	0.0112677170398279\\
262	0.0112679055916703\\
263	0.0112680976586796\\
264	0.0112682933061171\\
265	0.011268492600436\\
266	0.0112686956093012\\
267	0.0112689024016106\\
268	0.0112691130475165\\
269	0.011269327618447\\
270	0.011269546187128\\
271	0.0112697688276052\\
272	0.0112699956152668\\
273	0.0112702266268665\\
274	0.0112704619405461\\
275	0.0112707016358595\\
276	0.0112709457937964\\
277	0.0112711944968063\\
278	0.0112714478288227\\
279	0.0112717058752885\\
280	0.0112719687231806\\
281	0.0112722364610355\\
282	0.0112725091789749\\
283	0.0112727869687322\\
284	0.0112730699236784\\
285	0.0112733581388494\\
286	0.0112736517109728\\
287	0.0112739507384954\\
288	0.0112742553216112\\
289	0.0112745655622895\\
290	0.0112748815643033\\
291	0.0112752034332587\\
292	0.0112755312766235\\
293	0.0112758652037578\\
294	0.0112762053259433\\
295	0.0112765517564144\\
296	0.0112769046103891\\
297	0.0112772640051001\\
298	0.0112776300598274\\
299	0.0112780028959302\\
300	0.0112783826368803\\
301	0.0112787694082955\\
302	0.011279163337974\\
303	0.0112795645559294\\
304	0.0112799731944262\\
305	0.0112803893880167\\
306	0.011280813273578\\
307	0.0112812449903512\\
308	0.0112816846799803\\
309	0.0112821324865534\\
310	0.0112825885566452\\
311	0.0112830530393604\\
312	0.01128352608638\\
313	0.0112840078520083\\
314	0.0112844984932235\\
315	0.0112849981697299\\
316	0.0112855070440136\\
317	0.0112860252814014\\
318	0.0112865530501233\\
319	0.01128709052138\\
320	0.0112876378694145\\
321	0.0112881952715901\\
322	0.0112887629084743\\
323	0.0112893409639309\\
324	0.0112899296252196\\
325	0.0112905290831059\\
326	0.0112911395319821\\
327	0.0112917611700007\\
328	0.0112923941992221\\
329	0.0112930388257796\\
330	0.0112936952600627\\
331	0.0112943637169221\\
332	0.0112950444158994\\
333	0.0112957375814848\\
334	0.0112964434434048\\
335	0.0112971622369433\\
336	0.0112978942032914\\
337	0.0112986395899087\\
338	0.0112993986508361\\
339	0.0113001716467717\\
340	0.0113009588443532\\
341	0.0113017605129985\\
342	0.011302576914436\\
343	0.0113034082704757\\
344	0.0113042546679323\\
345	0.0113051157788013\\
346	0.0113059870847612\\
347	0.0113068634602973\\
348	0.0113077574729328\\
349	0.0113086694838385\\
350	0.0113095998618589\\
351	0.0113105489837004\\
352	0.0113115172341461\\
353	0.0113125050063101\\
354	0.0113135127018809\\
355	0.01131454073094\\
356	0.0113155895124201\\
357	0.0113166594743456\\
358	0.0113177510540604\\
359	0.011318864698463\\
360	0.01132000086425\\
361	0.0113211600181681\\
362	0.0113223426372747\\
363	0.0113235492092085\\
364	0.0113247802324689\\
365	0.0113260362167065\\
366	0.0113273176830242\\
367	0.0113286251642894\\
368	0.0113299592054583\\
369	0.0113313203639124\\
370	0.0113327092098079\\
371	0.0113341263264394\\
372	0.011335572310617\\
373	0.0113370477730589\\
374	0.011338553338799\\
375	0.0113400896476113\\
376	0.0113416573544503\\
377	0.0113432571299097\\
378	0.0113448896606983\\
379	0.011346555650135\\
380	0.011348255818663\\
381	0.0113499909043826\\
382	0.0113517616636048\\
383	0.0113535688714243\\
384	0.0113554133223126\\
385	0.0113572958307299\\
386	0.0113592172317557\\
387	0.0113611783817345\\
388	0.0113631801589288\\
389	0.0113652234641618\\
390	0.0113673092214005\\
391	0.0113694383781505\\
392	0.011371611905316\\
393	0.0113738307956219\\
394	0.0113760960582649\\
395	0.0113784087040299\\
396	0.0113807697074683\\
397	0.0113831799184833\\
398	0.0113856403676593\\
399	0.0113881527962993\\
400	0.0113907183294125\\
401	0.0113933381170359\\
402	0.0113960133347759\\
403	0.0113987451843574\\
404	0.0114015348941784\\
405	0.0114043837198715\\
406	0.0114072929448722\\
407	0.0114102638809921\\
408	0.0114132978689921\\
409	0.0114163962791378\\
410	0.011419560511758\\
411	0.0114227919978371\\
412	0.011426092199583\\
413	0.0114294626109854\\
414	0.0114329047583655\\
415	0.0114364202009167\\
416	0.0114400105312306\\
417	0.0114436773758051\\
418	0.0114474223955253\\
419	0.0114512472861004\\
420	0.011455153778408\\
421	0.0114591436386726\\
422	0.011463218668338\\
423	0.0114673807035545\\
424	0.0114716316144898\\
425	0.0114759733070446\\
426	0.0114804077392078\\
427	0.0114849369131814\\
428	0.0114895628525918\\
429	0.011494287621703\\
430	0.0114991133259348\\
431	0.0115040421123582\\
432	0.0115090761701658\\
433	0.0115142177311134\\
434	0.0115194690699294\\
435	0.0115248325046872\\
436	0.0115303103971372\\
437	0.0115359051529925\\
438	0.011541619222163\\
439	0.0115474550989317\\
440	0.0115534153220635\\
441	0.0115595024748401\\
442	0.0115657191850129\\
443	0.0115720681246817\\
444	0.0115785520101438\\
445	0.0115851736018212\\
446	0.0115919357042626\\
447	0.0115988411646546\\
448	0.0116058928723007\\
449	0.0116130937581414\\
450	0.0116204467936933\\
451	0.0116279549897897\\
452	0.0116356213951004\\
453	0.011643449094412\\
454	0.0116514412066435\\
455	0.0116596008825729\\
456	0.0116679313022476\\
457	0.0116764356720505\\
458	0.0116851172213899\\
459	0.0116939791989828\\
460	0.0117030248686985\\
461	0.0117122575049196\\
462	0.0117216803873619\\
463	0.0117312967952552\\
464	0.0117411100008999\\
465	0.0117511232622166\\
466	0.0117613398136458\\
467	0.0117717628534318\\
468	0.0117823955214753\\
469	0.0117932408502661\\
470	0.0118043016357616\\
471	0.0118155800651896\\
472	0.0118270765966559\\
473	0.0118387865091484\\
474	0.0118506924852321\\
475	0.0118628414200771\\
476	0.0118752354806045\\
477	0.0118878854083289\\
478	0.0119007971910868\\
479	0.0119149717250346\\
480	0.0119292492072798\\
481	0.0119436199748858\\
482	0.0119580713142637\\
483	0.0119725845726865\\
484	0.0119870708740365\\
485	0.0120009119529355\\
486	0.0120149260477244\\
487	0.0120291096423274\\
488	0.0120434579807057\\
489	0.0120579636188418\\
490	0.0120726122065619\\
491	0.0120872825221991\\
492	0.0121020742309012\\
493	0.0121170377351672\\
494	0.0121321673614265\\
495	0.0121474616644167\\
496	0.0121629343637475\\
497	0.0121786530264737\\
498	0.0121948547977051\\
499	0.012221457378298\\
500	0.0122497871374526\\
501	0.0122767903833679\\
502	0.0122990963844984\\
503	0.0123126577049702\\
504	0.0123251859336564\\
505	0.0123365518283011\\
506	0.0123458266959497\\
507	0.012354006909702\\
508	0.0123620707926574\\
509	0.0123700731811967\\
510	0.0123781756799936\\
511	0.0123864162622617\\
512	0.0123948132252244\\
513	0.0124033832834578\\
514	0.0124121326522197\\
515	0.0124210669970876\\
516	0.0124301912993531\\
517	0.0124395098495792\\
518	0.0124490268827363\\
519	0.0124587465968323\\
520	0.0124686731839293\\
521	0.0124788108770463\\
522	0.0124891639522178\\
523	0.0124997367307636\\
524	0.0125105335854768\\
525	0.012521558958305\\
526	0.0125328174269855\\
527	0.0125443139228337\\
528	0.0125560544133927\\
529	0.0125680480407597\\
530	0.0125818596302423\\
531	0.0125958734095194\\
532	0.0126087894960663\\
533	0.0126209508645745\\
534	0.0126331344642834\\
535	0.012645168775862\\
536	0.012657399769164\\
537	0.0126698368094777\\
538	0.0126824850373347\\
539	0.0126953459640422\\
540	0.0127084205674275\\
541	0.0127217095726791\\
542	0.0127352134366965\\
543	0.0127489323316446\\
544	0.0127628661204725\\
545	0.0127770143306754\\
546	0.0127913761272473\\
547	0.0128059502864518\\
548	0.0128207351763738\\
549	0.0128357287676952\\
550	0.0128509287688814\\
551	0.0128663332651597\\
552	0.0128829974101976\\
553	0.0128992255349403\\
554	0.0129144037198701\\
555	0.0129294166252927\\
556	0.0129445962025297\\
557	0.0129599414039861\\
558	0.0129754481860902\\
559	0.0129911120830214\\
560	0.0130069282939308\\
561	0.0130228919582639\\
562	0.0130389989624544\\
563	0.0130554776437141\\
564	0.0130722424729182\\
565	0.0130882887545376\\
566	0.0131043428556736\\
567	0.013120509715601\\
568	0.0131367816514183\\
569	0.0131531501362835\\
570	0.0131696057273492\\
571	0.0131861379872316\\
572	0.0132027353983147\\
573	0.0132193852690868\\
574	0.0132360736315842\\
575	0.0132527851288787\\
576	0.0132695028913739\\
577	0.0132862084004767\\
578	0.0133028813379804\\
579	0.0133194994192331\\
580	0.0133360382079\\
581	0.0133524709099319\\
582	0.0133687681444567\\
583	0.0133848976903317\\
584	0.0134008242107329\\
585	0.0134165089689716\\
586	0.0134319095789188\\
587	0.0134469799156334\\
588	0.0134616705332227\\
589	0.0134759305304731\\
590	0.0134902483101722\\
591	0.0135046714122469\\
592	0.0135192196273521\\
593	0.0135339766714342\\
594	0.0135491998980231\\
595	0.0135656125153459\\
596	0.0135851815686143\\
597	0.0136131889078992\\
598	0.0136637438596031\\
599	0\\
600	0\\
};
\end{axis}
\end{tikzpicture}%
 
  \caption{Discrete Time}
\end{subfigure}\\
\vspace{1cm}
\begin{subfigure}{.45\linewidth}
  \centering
  \setlength\figureheight{\linewidth} 
  \setlength\figurewidth{\linewidth}
  \tikzsetnextfilename{dp_colorbar/dp_cts_nFPC_z1}
  % This file was created by matlab2tikz.
%
%The latest updates can be retrieved from
%  http://www.mathworks.com/matlabcentral/fileexchange/22022-matlab2tikz-matlab2tikz
%where you can also make suggestions and rate matlab2tikz.
%
\definecolor{mycolor1}{rgb}{0.00000,1.00000,0.14286}%
\definecolor{mycolor2}{rgb}{0.00000,1.00000,0.28571}%
\definecolor{mycolor3}{rgb}{0.00000,1.00000,0.42857}%
\definecolor{mycolor4}{rgb}{0.00000,1.00000,0.57143}%
\definecolor{mycolor5}{rgb}{0.00000,1.00000,0.71429}%
\definecolor{mycolor6}{rgb}{0.00000,1.00000,0.85714}%
\definecolor{mycolor7}{rgb}{0.00000,1.00000,1.00000}%
\definecolor{mycolor8}{rgb}{0.00000,0.87500,1.00000}%
\definecolor{mycolor9}{rgb}{0.00000,0.62500,1.00000}%
\definecolor{mycolor10}{rgb}{0.12500,0.00000,1.00000}%
\definecolor{mycolor11}{rgb}{0.25000,0.00000,1.00000}%
\definecolor{mycolor12}{rgb}{0.37500,0.00000,1.00000}%
\definecolor{mycolor13}{rgb}{0.50000,0.00000,1.00000}%
\definecolor{mycolor14}{rgb}{0.62500,0.00000,1.00000}%
\definecolor{mycolor15}{rgb}{0.75000,0.00000,1.00000}%
\definecolor{mycolor16}{rgb}{0.87500,0.00000,1.00000}%
\definecolor{mycolor17}{rgb}{1.00000,0.00000,1.00000}%
\definecolor{mycolor18}{rgb}{1.00000,0.00000,0.87500}%
\definecolor{mycolor19}{rgb}{1.00000,0.00000,0.62500}%
\definecolor{mycolor20}{rgb}{0.85714,0.00000,0.00000}%
\definecolor{mycolor21}{rgb}{0.71429,0.00000,0.00000}%
%
\begin{tikzpicture}[trim axis left, trim axis right]

\begin{axis}[%
width=\figurewidth,
height=\figureheight,
at={(0\figurewidth,0\figureheight)},
scale only axis,
every outer x axis line/.append style={black},
every x tick label/.append style={font=\color{black}},
xmin=0,
xmax=600,
every outer y axis line/.append style={black},
every y tick label/.append style={font=\color{black}},
ymin=0,
ymax=0.014,
axis background/.style={fill=white},
axis x line*=bottom,
axis y line*=left,
yticklabel style={
        /pgf/number format/fixed,
        /pgf/number format/precision=3
},
scaled y ticks=false
]
\addplot [color=green,solid,forget plot]
  table[row sep=crcr]{%
0.01	1.73472347597681e-18\\
1.01	1.73472347597681e-18\\
2.01	0\\
3.01	1.73472347597681e-18\\
4.01	1.73472347597681e-18\\
5.01	0\\
6.01	0\\
7.01	1.73472347597681e-18\\
8.01	0\\
9.01	0\\
10.01	0\\
11.01	0\\
12.01	1.73472347597681e-18\\
13.01	1.73472347597681e-18\\
14.01	1.73472347597681e-18\\
15.01	1.73472347597681e-18\\
16.01	1.73472347597681e-18\\
17.01	1.73472347597681e-18\\
18.01	0\\
19.01	0\\
20.01	1.73472347597681e-18\\
21.01	0\\
22.01	1.73472347597681e-18\\
23.01	1.73472347597681e-18\\
24.01	1.73472347597681e-18\\
25.01	0\\
26.01	1.73472347597681e-18\\
27.01	0\\
28.01	1.73472347597681e-18\\
29.01	1.73472347597681e-18\\
30.01	0\\
31.01	0\\
32.01	1.73472347597681e-18\\
33.01	1.73472347597681e-18\\
34.01	1.73472347597681e-18\\
35.01	0\\
36.01	0\\
37.01	0\\
38.01	0\\
39.01	1.73472347597681e-18\\
40.01	1.73472347597681e-18\\
41.01	1.73472347597681e-18\\
42.01	1.73472347597681e-18\\
43.01	1.73472347597681e-18\\
44.01	0\\
45.01	1.73472347597681e-18\\
46.01	1.73472347597681e-18\\
47.01	1.73472347597681e-18\\
48.01	1.73472347597681e-18\\
49.01	0\\
50.01	1.73472347597681e-18\\
51.01	1.73472347597681e-18\\
52.01	0\\
53.01	0\\
54.01	0\\
55.01	1.73472347597681e-18\\
56.01	0\\
57.01	1.73472347597681e-18\\
58.01	0\\
59.01	0\\
60.01	1.73472347597681e-18\\
61.01	1.73472347597681e-18\\
62.01	1.73472347597681e-18\\
63.01	0\\
64.01	1.73472347597681e-18\\
65.01	0\\
66.01	1.73472347597681e-18\\
67.01	1.73472347597681e-18\\
68.01	0\\
69.01	1.73472347597681e-18\\
70.01	1.73472347597681e-18\\
71.01	0\\
72.01	0\\
73.01	1.73472347597681e-18\\
74.01	0\\
75.01	1.73472347597681e-18\\
76.01	0\\
77.01	0\\
78.01	0\\
79.01	0\\
80.01	1.73472347597681e-18\\
81.01	1.73472347597681e-18\\
82.01	0\\
83.01	0\\
84.01	0\\
85.01	1.73472347597681e-18\\
86.01	1.73472347597681e-18\\
87.01	0\\
88.01	0\\
89.01	0\\
90.01	1.73472347597681e-18\\
91.01	1.73472347597681e-18\\
92.01	0\\
93.01	1.73472347597681e-18\\
94.01	1.73472347597681e-18\\
95.01	0\\
96.01	1.73472347597681e-18\\
97.01	0\\
98.01	0\\
99.01	0\\
100.01	1.73472347597681e-18\\
101.01	0\\
102.01	1.73472347597681e-18\\
103.01	1.73472347597681e-18\\
104.01	0\\
105.01	0\\
106.01	0\\
107.01	0\\
108.01	1.73472347597681e-18\\
109.01	0\\
110.01	1.73472347597681e-18\\
111.01	0\\
112.01	0\\
113.01	0\\
114.01	1.73472347597681e-18\\
115.01	1.73472347597681e-18\\
116.01	0\\
117.01	0\\
118.01	0\\
119.01	0\\
120.01	0\\
121.01	0\\
122.01	0\\
123.01	1.73472347597681e-18\\
124.01	1.73472347597681e-18\\
125.01	1.73472347597681e-18\\
126.01	1.73472347597681e-18\\
127.01	0\\
128.01	1.73472347597681e-18\\
129.01	0\\
130.01	1.73472347597681e-18\\
131.01	1.73472347597681e-18\\
132.01	0\\
133.01	1.73472347597681e-18\\
134.01	1.73472347597681e-18\\
135.01	0\\
136.01	0\\
137.01	1.73472347597681e-18\\
138.01	0\\
139.01	1.73472347597681e-18\\
140.01	1.73472347597681e-18\\
141.01	1.73472347597681e-18\\
142.01	1.73472347597681e-18\\
143.01	0\\
144.01	1.73472347597681e-18\\
145.01	1.73472347597681e-18\\
146.01	1.73472347597681e-18\\
147.01	1.73472347597681e-18\\
148.01	1.73472347597681e-18\\
149.01	0\\
150.01	0\\
151.01	1.73472347597681e-18\\
152.01	1.73472347597681e-18\\
153.01	1.73472347597681e-18\\
154.01	1.73472347597681e-18\\
155.01	1.73472347597681e-18\\
156.01	1.73472347597681e-18\\
157.01	0\\
158.01	1.73472347597681e-18\\
159.01	1.73472347597681e-18\\
160.01	1.73472347597681e-18\\
161.01	1.73472347597681e-18\\
162.01	1.73472347597681e-18\\
163.01	1.73472347597681e-18\\
164.01	0\\
165.01	0\\
166.01	1.73472347597681e-18\\
167.01	0\\
168.01	1.73472347597681e-18\\
169.01	0\\
170.01	1.73472347597681e-18\\
171.01	0\\
172.01	0\\
173.01	0\\
174.01	1.73472347597681e-18\\
175.01	1.73472347597681e-18\\
176.01	1.73472347597681e-18\\
177.01	0\\
178.01	1.73472347597681e-18\\
179.01	0\\
180.01	0\\
181.01	1.73472347597681e-18\\
182.01	0\\
183.01	0\\
184.01	0\\
185.01	1.73472347597681e-18\\
186.01	0\\
187.01	0\\
188.01	0\\
189.01	0\\
190.01	1.73472347597681e-18\\
191.01	0\\
192.01	1.73472347597681e-18\\
193.01	0\\
194.01	1.73472347597681e-18\\
195.01	0\\
196.01	1.73472347597681e-18\\
197.01	0\\
198.01	0\\
199.01	0\\
200.01	0\\
201.01	0\\
202.01	1.73472347597681e-18\\
203.01	0\\
204.01	1.73472347597681e-18\\
205.01	1.73472347597681e-18\\
206.01	1.73472347597681e-18\\
207.01	1.73472347597681e-18\\
208.01	0\\
209.01	1.73472347597681e-18\\
210.01	0\\
211.01	1.73472347597681e-18\\
212.01	0\\
213.01	1.73472347597681e-18\\
214.01	0\\
215.01	0\\
216.01	0\\
217.01	0\\
218.01	1.73472347597681e-18\\
219.01	1.73472347597681e-18\\
220.01	0\\
221.01	0\\
222.01	0\\
223.01	1.73472347597681e-18\\
224.01	1.73472347597681e-18\\
225.01	0\\
226.01	1.73472347597681e-18\\
227.01	0\\
228.01	0\\
229.01	1.73472347597681e-18\\
230.01	0\\
231.01	0\\
232.01	0\\
233.01	0\\
234.01	1.73472347597681e-18\\
235.01	0\\
236.01	1.73472347597681e-18\\
237.01	1.73472347597681e-18\\
238.01	0\\
239.01	1.73472347597681e-18\\
240.01	0\\
241.01	1.73472347597681e-18\\
242.01	0\\
243.01	1.73472347597681e-18\\
244.01	1.73472347597681e-18\\
245.01	1.73472347597681e-18\\
246.01	1.73472347597681e-18\\
247.01	0\\
248.01	0\\
249.01	0\\
250.01	1.73472347597681e-18\\
251.01	0\\
252.01	1.73472347597681e-18\\
253.01	1.73472347597681e-18\\
254.01	1.73472347597681e-18\\
255.01	1.73472347597681e-18\\
256.01	1.73472347597681e-18\\
257.01	1.73472347597681e-18\\
258.01	0\\
259.01	0\\
260.01	0\\
261.01	0\\
262.01	1.73472347597681e-18\\
263.01	1.73472347597681e-18\\
264.01	0\\
265.01	1.73472347597681e-18\\
266.01	1.73472347597681e-18\\
267.01	0\\
268.01	0\\
269.01	1.73472347597681e-18\\
270.01	0\\
271.01	1.73472347597681e-18\\
272.01	0\\
273.01	1.73472347597681e-18\\
274.01	0\\
275.01	0\\
276.01	1.73472347597681e-18\\
277.01	1.73472347597681e-18\\
278.01	0\\
279.01	1.73472347597681e-18\\
280.01	0\\
281.01	1.73472347597681e-18\\
282.01	1.73472347597681e-18\\
283.01	1.73472347597681e-18\\
284.01	0\\
285.01	1.73472347597681e-18\\
286.01	1.73472347597681e-18\\
287.01	1.73472347597681e-18\\
288.01	1.73472347597681e-18\\
289.01	0\\
290.01	0\\
291.01	0\\
292.01	0\\
293.01	1.73472347597681e-18\\
294.01	1.73472347597681e-18\\
295.01	0\\
296.01	1.73472347597681e-18\\
297.01	0\\
298.01	1.73472347597681e-18\\
299.01	0\\
300.01	1.73472347597681e-18\\
301.01	0\\
302.01	0\\
303.01	0\\
304.01	1.73472347597681e-18\\
305.01	1.73472347597681e-18\\
306.01	0\\
307.01	1.73472347597681e-18\\
308.01	0\\
309.01	1.73472347597681e-18\\
310.01	1.73472347597681e-18\\
311.01	0\\
312.01	1.73472347597681e-18\\
313.01	1.73472347597681e-18\\
314.01	0\\
315.01	1.73472347597681e-18\\
316.01	0\\
317.01	1.73472347597681e-18\\
318.01	0\\
319.01	0\\
320.01	0\\
321.01	0\\
322.01	1.73472347597681e-18\\
323.01	1.73472347597681e-18\\
324.01	0\\
325.01	1.73472347597681e-18\\
326.01	1.73472347597681e-18\\
327.01	0\\
328.01	1.73472347597681e-18\\
329.01	0\\
330.01	0\\
331.01	1.73472347597681e-18\\
332.01	1.73472347597681e-18\\
333.01	1.73472347597681e-18\\
334.01	1.73472347597681e-18\\
335.01	1.73472347597681e-18\\
336.01	0\\
337.01	1.73472347597681e-18\\
338.01	0\\
339.01	0\\
340.01	1.73472347597681e-18\\
341.01	1.73472347597681e-18\\
342.01	1.73472347597681e-18\\
343.01	1.73472347597681e-18\\
344.01	0\\
345.01	0\\
346.01	0\\
347.01	1.73472347597681e-18\\
348.01	0\\
349.01	1.73472347597681e-18\\
350.01	0\\
351.01	0\\
352.01	0\\
353.01	1.73472347597681e-18\\
354.01	1.73472347597681e-18\\
355.01	0\\
356.01	0\\
357.01	1.73472347597681e-18\\
358.01	1.73472347597681e-18\\
359.01	1.73472347597681e-18\\
360.01	0\\
361.01	0\\
362.01	0\\
363.01	0\\
364.01	0\\
365.01	1.73472347597681e-18\\
366.01	1.73472347597681e-18\\
367.01	1.73472347597681e-18\\
368.01	1.73472347597681e-18\\
369.01	0\\
370.01	0\\
371.01	1.73472347597681e-18\\
372.01	1.73472347597681e-18\\
373.01	1.73472347597681e-18\\
374.01	0\\
375.01	1.73472347597681e-18\\
376.01	0\\
377.01	0\\
378.01	1.73472347597681e-18\\
379.01	0\\
380.01	1.73472347597681e-18\\
381.01	1.73472347597681e-18\\
382.01	1.73472347597681e-18\\
383.01	0\\
384.01	1.73472347597681e-18\\
385.01	0\\
386.01	1.73472347597681e-18\\
387.01	1.73472347597681e-18\\
388.01	1.73472347597681e-18\\
389.01	0\\
390.01	1.73472347597681e-18\\
391.01	1.73472347597681e-18\\
392.01	0\\
393.01	0\\
394.01	1.73472347597681e-18\\
395.01	1.73472347597681e-18\\
396.01	1.73472347597681e-18\\
397.01	1.73472347597681e-18\\
398.01	0\\
399.01	1.73472347597681e-18\\
400.01	0\\
401.01	0\\
402.01	0\\
403.01	1.73472347597681e-18\\
404.01	1.73472347597681e-18\\
405.01	1.73472347597681e-18\\
406.01	1.73472347597681e-18\\
407.01	1.73472347597681e-18\\
408.01	1.73472347597681e-18\\
409.01	0\\
410.01	1.73472347597681e-18\\
411.01	0\\
412.01	1.73472347597681e-18\\
413.01	1.73472347597681e-18\\
414.01	0\\
415.01	0\\
416.01	1.73472347597681e-18\\
417.01	1.73472347597681e-18\\
418.01	1.73472347597681e-18\\
419.01	1.73472347597681e-18\\
420.01	0\\
421.01	1.73472347597681e-18\\
422.01	1.73472347597681e-18\\
423.01	1.73472347597681e-18\\
424.01	1.73472347597681e-18\\
425.01	0\\
426.01	1.73472347597681e-18\\
427.01	1.73472347597681e-18\\
428.01	1.73472347597681e-18\\
429.01	0\\
430.01	1.73472347597681e-18\\
431.01	1.73472347597681e-18\\
432.01	1.73472347597681e-18\\
433.01	0\\
434.01	1.73472347597681e-18\\
435.01	1.73472347597681e-18\\
436.01	1.73472347597681e-18\\
437.01	1.73472347597681e-18\\
438.01	1.73472347597681e-18\\
439.01	0\\
440.01	1.73472347597681e-18\\
441.01	1.73472347597681e-18\\
442.01	0\\
443.01	0\\
444.01	0\\
445.01	1.73472347597681e-18\\
446.01	1.73472347597681e-18\\
447.01	0\\
448.01	1.73472347597681e-18\\
449.01	1.73472347597681e-18\\
450.01	0\\
451.01	0\\
452.01	0\\
453.01	1.73472347597681e-18\\
454.01	1.73472347597681e-18\\
455.01	0\\
456.01	1.73472347597681e-18\\
457.01	1.73472347597681e-18\\
458.01	0\\
459.01	1.73472347597681e-18\\
460.01	0\\
461.01	1.73472347597681e-18\\
462.01	0\\
463.01	0\\
464.01	0\\
465.01	1.73472347597681e-18\\
466.01	0\\
467.01	1.73472347597681e-18\\
468.01	0\\
469.01	1.73472347597681e-18\\
470.01	1.73472347597681e-18\\
471.01	1.73472347597681e-18\\
472.01	0\\
473.01	0\\
474.01	0\\
475.01	1.73472347597681e-18\\
476.01	0\\
477.01	0\\
478.01	1.73472347597681e-18\\
479.01	1.73472347597681e-18\\
480.01	0\\
481.01	0\\
482.01	1.73472347597681e-18\\
483.01	1.73472347597681e-18\\
484.01	0\\
485.01	0\\
486.01	0\\
487.01	1.73472347597681e-18\\
488.01	1.73472347597681e-18\\
489.01	1.73472347597681e-18\\
490.01	1.73472347597681e-18\\
491.01	0\\
492.01	1.73472347597681e-18\\
493.01	1.73472347597681e-18\\
494.01	0\\
495.01	1.73472347597681e-18\\
496.01	0\\
497.01	1.73472347597681e-18\\
498.01	1.73472347597681e-18\\
499.01	0\\
500.01	1.73472347597681e-18\\
501.01	1.73472347597681e-18\\
502.01	0\\
503.01	1.73472347597681e-18\\
504.01	1.73472347597681e-18\\
505.01	1.73472347597681e-18\\
506.01	0\\
507.01	0\\
508.01	0\\
509.01	0\\
510.01	1.73472347597681e-18\\
511.01	1.73472347597681e-18\\
512.01	0\\
513.01	1.73472347597681e-18\\
514.01	1.73472347597681e-18\\
515.01	0\\
516.01	0\\
517.01	1.73472347597681e-18\\
518.01	1.73472347597681e-18\\
519.01	1.73472347597681e-18\\
520.01	0\\
521.01	0\\
522.01	0\\
523.01	1.73472347597681e-18\\
524.01	0\\
525.01	1.73472347597681e-18\\
526.01	1.73472347597681e-18\\
527.01	0\\
528.01	0\\
529.01	0\\
530.01	1.73472347597681e-18\\
531.01	0\\
532.01	1.73472347597681e-18\\
533.01	0\\
534.01	1.73472347597681e-18\\
535.01	0\\
536.01	1.73472347597681e-18\\
537.01	1.73472347597681e-18\\
538.01	1.73472347597681e-18\\
539.01	0\\
540.01	0\\
541.01	1.73472347597681e-18\\
542.01	1.73472347597681e-18\\
543.01	0\\
544.01	1.73472347597681e-18\\
545.01	0\\
546.01	1.73472347597681e-18\\
547.01	0\\
548.01	0\\
549.01	0\\
550.01	0\\
551.01	0\\
552.01	0\\
553.01	0\\
554.01	0\\
555.01	1.73472347597681e-18\\
556.01	0\\
557.01	0\\
558.01	0\\
559.01	1.73472347597681e-18\\
560.01	1.73472347597681e-18\\
561.01	1.73472347597681e-18\\
562.01	1.73472347597681e-18\\
563.01	0\\
564.01	0\\
565.01	1.73472347597681e-18\\
566.01	0\\
567.01	0\\
568.01	0\\
569.01	0\\
570.01	0\\
571.01	0\\
572.01	1.73472347597681e-18\\
573.01	0\\
574.01	1.73472347597681e-18\\
575.01	0\\
576.01	1.73472347597681e-18\\
577.01	0\\
578.01	1.73472347597681e-18\\
579.01	0\\
580.01	0\\
581.01	1.73472347597681e-18\\
582.01	0\\
583.01	0\\
584.01	0\\
585.01	0\\
586.01	0\\
587.01	0\\
588.01	0\\
589.01	0\\
590.01	0\\
591.01	0\\
592.01	1.73472347597681e-18\\
593.01	0\\
594.01	0\\
595.01	0\\
596.01	0\\
597.01	0\\
598.01	0\\
599.01	0\\
599.02	0\\
599.03	1.73472347597681e-18\\
599.04	1.73472347597681e-18\\
599.05	0\\
599.06	0\\
599.07	1.73472347597681e-18\\
599.08	0\\
599.09	0\\
599.1	0\\
599.11	1.73472347597681e-18\\
599.12	1.73472347597681e-18\\
599.13	0\\
599.14	0\\
599.15	1.73472347597681e-18\\
599.16	1.73472347597681e-18\\
599.17	1.73472347597681e-18\\
599.18	0\\
599.19	0\\
599.2	0\\
599.21	0\\
599.22	0\\
599.23	1.73472347597681e-18\\
599.24	1.73472347597681e-18\\
599.25	1.73472347597681e-18\\
599.26	1.73472347597681e-18\\
599.27	0\\
599.28	0\\
599.29	0\\
599.3	0\\
599.31	0\\
599.32	1.73472347597681e-18\\
599.33	1.73472347597681e-18\\
599.34	0\\
599.35	0\\
599.36	0\\
599.37	1.73472347597681e-18\\
599.38	0\\
599.39	0\\
599.4	0\\
599.41	0\\
599.42	0\\
599.43	0\\
599.44	0\\
599.45	0\\
599.46	1.73472347597681e-18\\
599.47	1.73472347597681e-18\\
599.48	1.73472347597681e-18\\
599.49	1.73472347597681e-18\\
599.5	1.73472347597681e-18\\
599.51	0\\
599.52	0\\
599.53	1.73472347597681e-18\\
599.54	0\\
599.55	0\\
599.56	0\\
599.57	1.73472347597681e-18\\
599.58	0\\
599.59	0\\
599.6	0\\
599.61	0\\
599.62	1.73472347597681e-18\\
599.63	0\\
599.64	1.73472347597681e-18\\
599.65	0\\
599.66	0\\
599.67	0\\
599.68	0\\
599.69	1.73472347597681e-18\\
599.7	0\\
599.71	1.73472347597681e-18\\
599.72	1.73472347597681e-18\\
599.73	1.73472347597681e-18\\
599.74	0\\
599.75	0\\
599.76	0\\
599.77	0\\
599.78	0\\
599.79	0\\
599.8	0\\
599.81	0\\
599.82	0\\
599.83	1.73472347597681e-18\\
599.84	0\\
599.85	0\\
599.86	0\\
599.87	0\\
599.88	0\\
599.89	0\\
599.9	0\\
599.91	0\\
599.92	0\\
599.93	0\\
599.94	0\\
599.95	0\\
599.96	0\\
599.97	0\\
599.98	0\\
599.99	0\\
600	0\\
};
\addplot [color=mycolor1,solid,forget plot]
  table[row sep=crcr]{%
0.01	1.73472347597681e-18\\
1.01	1.73472347597681e-18\\
2.01	0\\
3.01	1.73472347597681e-18\\
4.01	1.73472347597681e-18\\
5.01	0\\
6.01	0\\
7.01	1.73472347597681e-18\\
8.01	0\\
9.01	0\\
10.01	0\\
11.01	0\\
12.01	1.73472347597681e-18\\
13.01	1.73472347597681e-18\\
14.01	1.73472347597681e-18\\
15.01	1.73472347597681e-18\\
16.01	1.73472347597681e-18\\
17.01	1.73472347597681e-18\\
18.01	0\\
19.01	0\\
20.01	1.73472347597681e-18\\
21.01	0\\
22.01	1.73472347597681e-18\\
23.01	1.73472347597681e-18\\
24.01	1.73472347597681e-18\\
25.01	0\\
26.01	1.73472347597681e-18\\
27.01	0\\
28.01	1.73472347597681e-18\\
29.01	1.73472347597681e-18\\
30.01	0\\
31.01	0\\
32.01	1.73472347597681e-18\\
33.01	1.73472347597681e-18\\
34.01	1.73472347597681e-18\\
35.01	0\\
36.01	0\\
37.01	0\\
38.01	0\\
39.01	1.73472347597681e-18\\
40.01	1.73472347597681e-18\\
41.01	1.73472347597681e-18\\
42.01	1.73472347597681e-18\\
43.01	1.73472347597681e-18\\
44.01	0\\
45.01	1.73472347597681e-18\\
46.01	1.73472347597681e-18\\
47.01	1.73472347597681e-18\\
48.01	1.73472347597681e-18\\
49.01	0\\
50.01	1.73472347597681e-18\\
51.01	1.73472347597681e-18\\
52.01	0\\
53.01	0\\
54.01	0\\
55.01	1.73472347597681e-18\\
56.01	0\\
57.01	1.73472347597681e-18\\
58.01	0\\
59.01	0\\
60.01	1.73472347597681e-18\\
61.01	1.73472347597681e-18\\
62.01	1.73472347597681e-18\\
63.01	0\\
64.01	1.73472347597681e-18\\
65.01	0\\
66.01	1.73472347597681e-18\\
67.01	1.73472347597681e-18\\
68.01	0\\
69.01	1.73472347597681e-18\\
70.01	1.73472347597681e-18\\
71.01	0\\
72.01	0\\
73.01	1.73472347597681e-18\\
74.01	0\\
75.01	1.73472347597681e-18\\
76.01	0\\
77.01	0\\
78.01	0\\
79.01	0\\
80.01	1.73472347597681e-18\\
81.01	1.73472347597681e-18\\
82.01	0\\
83.01	0\\
84.01	0\\
85.01	1.73472347597681e-18\\
86.01	1.73472347597681e-18\\
87.01	0\\
88.01	0\\
89.01	0\\
90.01	1.73472347597681e-18\\
91.01	1.73472347597681e-18\\
92.01	0\\
93.01	1.73472347597681e-18\\
94.01	1.73472347597681e-18\\
95.01	0\\
96.01	1.73472347597681e-18\\
97.01	0\\
98.01	0\\
99.01	0\\
100.01	1.73472347597681e-18\\
101.01	0\\
102.01	1.73472347597681e-18\\
103.01	1.73472347597681e-18\\
104.01	0\\
105.01	0\\
106.01	0\\
107.01	0\\
108.01	1.73472347597681e-18\\
109.01	0\\
110.01	1.73472347597681e-18\\
111.01	0\\
112.01	0\\
113.01	0\\
114.01	1.73472347597681e-18\\
115.01	1.73472347597681e-18\\
116.01	0\\
117.01	0\\
118.01	0\\
119.01	0\\
120.01	0\\
121.01	0\\
122.01	0\\
123.01	1.73472347597681e-18\\
124.01	1.73472347597681e-18\\
125.01	1.73472347597681e-18\\
126.01	1.73472347597681e-18\\
127.01	0\\
128.01	1.73472347597681e-18\\
129.01	0\\
130.01	1.73472347597681e-18\\
131.01	1.73472347597681e-18\\
132.01	0\\
133.01	1.73472347597681e-18\\
134.01	1.73472347597681e-18\\
135.01	0\\
136.01	0\\
137.01	1.73472347597681e-18\\
138.01	0\\
139.01	1.73472347597681e-18\\
140.01	1.73472347597681e-18\\
141.01	1.73472347597681e-18\\
142.01	1.73472347597681e-18\\
143.01	0\\
144.01	1.73472347597681e-18\\
145.01	1.73472347597681e-18\\
146.01	1.73472347597681e-18\\
147.01	1.73472347597681e-18\\
148.01	1.73472347597681e-18\\
149.01	0\\
150.01	0\\
151.01	1.73472347597681e-18\\
152.01	1.73472347597681e-18\\
153.01	1.73472347597681e-18\\
154.01	1.73472347597681e-18\\
155.01	1.73472347597681e-18\\
156.01	1.73472347597681e-18\\
157.01	0\\
158.01	1.73472347597681e-18\\
159.01	1.73472347597681e-18\\
160.01	1.73472347597681e-18\\
161.01	1.73472347597681e-18\\
162.01	1.73472347597681e-18\\
163.01	1.73472347597681e-18\\
164.01	0\\
165.01	0\\
166.01	1.73472347597681e-18\\
167.01	0\\
168.01	1.73472347597681e-18\\
169.01	0\\
170.01	1.73472347597681e-18\\
171.01	0\\
172.01	0\\
173.01	0\\
174.01	1.73472347597681e-18\\
175.01	1.73472347597681e-18\\
176.01	1.73472347597681e-18\\
177.01	0\\
178.01	1.73472347597681e-18\\
179.01	0\\
180.01	0\\
181.01	1.73472347597681e-18\\
182.01	0\\
183.01	0\\
184.01	0\\
185.01	1.73472347597681e-18\\
186.01	0\\
187.01	0\\
188.01	0\\
189.01	0\\
190.01	1.73472347597681e-18\\
191.01	0\\
192.01	1.73472347597681e-18\\
193.01	0\\
194.01	1.73472347597681e-18\\
195.01	0\\
196.01	1.73472347597681e-18\\
197.01	0\\
198.01	0\\
199.01	0\\
200.01	0\\
201.01	0\\
202.01	1.73472347597681e-18\\
203.01	0\\
204.01	1.73472347597681e-18\\
205.01	1.73472347597681e-18\\
206.01	1.73472347597681e-18\\
207.01	1.73472347597681e-18\\
208.01	0\\
209.01	1.73472347597681e-18\\
210.01	0\\
211.01	1.73472347597681e-18\\
212.01	0\\
213.01	1.73472347597681e-18\\
214.01	0\\
215.01	0\\
216.01	0\\
217.01	0\\
218.01	1.73472347597681e-18\\
219.01	1.73472347597681e-18\\
220.01	0\\
221.01	0\\
222.01	0\\
223.01	1.73472347597681e-18\\
224.01	1.73472347597681e-18\\
225.01	0\\
226.01	1.73472347597681e-18\\
227.01	0\\
228.01	0\\
229.01	1.73472347597681e-18\\
230.01	0\\
231.01	0\\
232.01	0\\
233.01	0\\
234.01	1.73472347597681e-18\\
235.01	0\\
236.01	1.73472347597681e-18\\
237.01	1.73472347597681e-18\\
238.01	0\\
239.01	1.73472347597681e-18\\
240.01	0\\
241.01	1.73472347597681e-18\\
242.01	0\\
243.01	1.73472347597681e-18\\
244.01	1.73472347597681e-18\\
245.01	1.73472347597681e-18\\
246.01	1.73472347597681e-18\\
247.01	0\\
248.01	0\\
249.01	0\\
250.01	1.73472347597681e-18\\
251.01	0\\
252.01	1.73472347597681e-18\\
253.01	1.73472347597681e-18\\
254.01	1.73472347597681e-18\\
255.01	1.73472347597681e-18\\
256.01	1.73472347597681e-18\\
257.01	1.73472347597681e-18\\
258.01	0\\
259.01	0\\
260.01	0\\
261.01	0\\
262.01	1.73472347597681e-18\\
263.01	1.73472347597681e-18\\
264.01	0\\
265.01	1.73472347597681e-18\\
266.01	1.73472347597681e-18\\
267.01	0\\
268.01	0\\
269.01	1.73472347597681e-18\\
270.01	0\\
271.01	1.73472347597681e-18\\
272.01	0\\
273.01	1.73472347597681e-18\\
274.01	0\\
275.01	0\\
276.01	1.73472347597681e-18\\
277.01	1.73472347597681e-18\\
278.01	0\\
279.01	1.73472347597681e-18\\
280.01	0\\
281.01	1.73472347597681e-18\\
282.01	1.73472347597681e-18\\
283.01	1.73472347597681e-18\\
284.01	0\\
285.01	1.73472347597681e-18\\
286.01	1.73472347597681e-18\\
287.01	1.73472347597681e-18\\
288.01	1.73472347597681e-18\\
289.01	0\\
290.01	0\\
291.01	0\\
292.01	0\\
293.01	1.73472347597681e-18\\
294.01	1.73472347597681e-18\\
295.01	0\\
296.01	1.73472347597681e-18\\
297.01	0\\
298.01	1.73472347597681e-18\\
299.01	0\\
300.01	1.73472347597681e-18\\
301.01	0\\
302.01	0\\
303.01	0\\
304.01	1.73472347597681e-18\\
305.01	1.73472347597681e-18\\
306.01	0\\
307.01	1.73472347597681e-18\\
308.01	0\\
309.01	1.73472347597681e-18\\
310.01	1.73472347597681e-18\\
311.01	0\\
312.01	1.73472347597681e-18\\
313.01	1.73472347597681e-18\\
314.01	0\\
315.01	1.73472347597681e-18\\
316.01	0\\
317.01	1.73472347597681e-18\\
318.01	0\\
319.01	0\\
320.01	0\\
321.01	0\\
322.01	1.73472347597681e-18\\
323.01	1.73472347597681e-18\\
324.01	0\\
325.01	1.73472347597681e-18\\
326.01	1.73472347597681e-18\\
327.01	0\\
328.01	1.73472347597681e-18\\
329.01	0\\
330.01	0\\
331.01	1.73472347597681e-18\\
332.01	1.73472347597681e-18\\
333.01	1.73472347597681e-18\\
334.01	1.73472347597681e-18\\
335.01	1.73472347597681e-18\\
336.01	0\\
337.01	1.73472347597681e-18\\
338.01	0\\
339.01	0\\
340.01	1.73472347597681e-18\\
341.01	1.73472347597681e-18\\
342.01	1.73472347597681e-18\\
343.01	1.73472347597681e-18\\
344.01	0\\
345.01	0\\
346.01	0\\
347.01	1.73472347597681e-18\\
348.01	0\\
349.01	1.73472347597681e-18\\
350.01	0\\
351.01	0\\
352.01	0\\
353.01	1.73472347597681e-18\\
354.01	1.73472347597681e-18\\
355.01	0\\
356.01	0\\
357.01	1.73472347597681e-18\\
358.01	1.73472347597681e-18\\
359.01	1.73472347597681e-18\\
360.01	0\\
361.01	0\\
362.01	0\\
363.01	0\\
364.01	0\\
365.01	1.73472347597681e-18\\
366.01	1.73472347597681e-18\\
367.01	1.73472347597681e-18\\
368.01	1.73472347597681e-18\\
369.01	0\\
370.01	0\\
371.01	1.73472347597681e-18\\
372.01	1.73472347597681e-18\\
373.01	1.73472347597681e-18\\
374.01	0\\
375.01	1.73472347597681e-18\\
376.01	0\\
377.01	0\\
378.01	1.73472347597681e-18\\
379.01	0\\
380.01	1.73472347597681e-18\\
381.01	1.73472347597681e-18\\
382.01	1.73472347597681e-18\\
383.01	0\\
384.01	1.73472347597681e-18\\
385.01	0\\
386.01	1.73472347597681e-18\\
387.01	1.73472347597681e-18\\
388.01	1.73472347597681e-18\\
389.01	0\\
390.01	1.73472347597681e-18\\
391.01	1.73472347597681e-18\\
392.01	0\\
393.01	0\\
394.01	1.73472347597681e-18\\
395.01	1.73472347597681e-18\\
396.01	1.73472347597681e-18\\
397.01	1.73472347597681e-18\\
398.01	0\\
399.01	1.73472347597681e-18\\
400.01	0\\
401.01	0\\
402.01	0\\
403.01	1.73472347597681e-18\\
404.01	1.73472347597681e-18\\
405.01	1.73472347597681e-18\\
406.01	1.73472347597681e-18\\
407.01	1.73472347597681e-18\\
408.01	1.73472347597681e-18\\
409.01	0\\
410.01	1.73472347597681e-18\\
411.01	0\\
412.01	1.73472347597681e-18\\
413.01	1.73472347597681e-18\\
414.01	0\\
415.01	0\\
416.01	1.73472347597681e-18\\
417.01	1.73472347597681e-18\\
418.01	1.73472347597681e-18\\
419.01	1.73472347597681e-18\\
420.01	0\\
421.01	1.73472347597681e-18\\
422.01	1.73472347597681e-18\\
423.01	1.73472347597681e-18\\
424.01	1.73472347597681e-18\\
425.01	0\\
426.01	1.73472347597681e-18\\
427.01	1.73472347597681e-18\\
428.01	1.73472347597681e-18\\
429.01	0\\
430.01	1.73472347597681e-18\\
431.01	1.73472347597681e-18\\
432.01	1.73472347597681e-18\\
433.01	0\\
434.01	1.73472347597681e-18\\
435.01	1.73472347597681e-18\\
436.01	1.73472347597681e-18\\
437.01	1.73472347597681e-18\\
438.01	1.73472347597681e-18\\
439.01	0\\
440.01	1.73472347597681e-18\\
441.01	1.73472347597681e-18\\
442.01	0\\
443.01	0\\
444.01	0\\
445.01	1.73472347597681e-18\\
446.01	1.73472347597681e-18\\
447.01	0\\
448.01	1.73472347597681e-18\\
449.01	1.73472347597681e-18\\
450.01	0\\
451.01	0\\
452.01	0\\
453.01	1.73472347597681e-18\\
454.01	1.73472347597681e-18\\
455.01	0\\
456.01	1.73472347597681e-18\\
457.01	1.73472347597681e-18\\
458.01	0\\
459.01	1.73472347597681e-18\\
460.01	0\\
461.01	1.73472347597681e-18\\
462.01	0\\
463.01	0\\
464.01	0\\
465.01	1.73472347597681e-18\\
466.01	0\\
467.01	1.73472347597681e-18\\
468.01	0\\
469.01	1.73472347597681e-18\\
470.01	1.73472347597681e-18\\
471.01	1.73472347597681e-18\\
472.01	0\\
473.01	0\\
474.01	0\\
475.01	1.73472347597681e-18\\
476.01	0\\
477.01	0\\
478.01	1.73472347597681e-18\\
479.01	1.73472347597681e-18\\
480.01	0\\
481.01	0\\
482.01	1.73472347597681e-18\\
483.01	1.73472347597681e-18\\
484.01	0\\
485.01	0\\
486.01	0\\
487.01	1.73472347597681e-18\\
488.01	1.73472347597681e-18\\
489.01	1.73472347597681e-18\\
490.01	1.73472347597681e-18\\
491.01	0\\
492.01	1.73472347597681e-18\\
493.01	1.73472347597681e-18\\
494.01	0\\
495.01	1.73472347597681e-18\\
496.01	0\\
497.01	1.73472347597681e-18\\
498.01	1.73472347597681e-18\\
499.01	0\\
500.01	1.73472347597681e-18\\
501.01	1.73472347597681e-18\\
502.01	0\\
503.01	1.73472347597681e-18\\
504.01	1.73472347597681e-18\\
505.01	1.73472347597681e-18\\
506.01	0\\
507.01	0\\
508.01	0\\
509.01	0\\
510.01	1.73472347597681e-18\\
511.01	1.73472347597681e-18\\
512.01	0\\
513.01	1.73472347597681e-18\\
514.01	1.73472347597681e-18\\
515.01	0\\
516.01	0\\
517.01	1.73472347597681e-18\\
518.01	1.73472347597681e-18\\
519.01	1.73472347597681e-18\\
520.01	0\\
521.01	0\\
522.01	0\\
523.01	1.73472347597681e-18\\
524.01	0\\
525.01	1.73472347597681e-18\\
526.01	1.73472347597681e-18\\
527.01	0\\
528.01	0\\
529.01	0\\
530.01	1.73472347597681e-18\\
531.01	0\\
532.01	1.73472347597681e-18\\
533.01	0\\
534.01	1.73472347597681e-18\\
535.01	0\\
536.01	1.73472347597681e-18\\
537.01	1.73472347597681e-18\\
538.01	1.73472347597681e-18\\
539.01	0\\
540.01	0\\
541.01	1.73472347597681e-18\\
542.01	1.73472347597681e-18\\
543.01	0\\
544.01	1.73472347597681e-18\\
545.01	0\\
546.01	1.73472347597681e-18\\
547.01	0\\
548.01	0\\
549.01	0\\
550.01	0\\
551.01	0\\
552.01	0\\
553.01	0\\
554.01	0\\
555.01	1.73472347597681e-18\\
556.01	0\\
557.01	0\\
558.01	0\\
559.01	1.73472347597681e-18\\
560.01	1.73472347597681e-18\\
561.01	1.73472347597681e-18\\
562.01	1.73472347597681e-18\\
563.01	0\\
564.01	0\\
565.01	1.73472347597681e-18\\
566.01	0\\
567.01	0\\
568.01	0\\
569.01	0\\
570.01	0\\
571.01	0\\
572.01	1.73472347597681e-18\\
573.01	0\\
574.01	1.73472347597681e-18\\
575.01	0\\
576.01	1.73472347597681e-18\\
577.01	0\\
578.01	1.73472347597681e-18\\
579.01	0\\
580.01	0\\
581.01	1.73472347597681e-18\\
582.01	0\\
583.01	0\\
584.01	0\\
585.01	0\\
586.01	0\\
587.01	0\\
588.01	0\\
589.01	0\\
590.01	0\\
591.01	0\\
592.01	1.73472347597681e-18\\
593.01	0\\
594.01	0\\
595.01	0\\
596.01	0\\
597.01	0\\
598.01	0\\
599.01	0\\
599.02	0\\
599.03	1.73472347597681e-18\\
599.04	1.73472347597681e-18\\
599.05	0\\
599.06	0\\
599.07	1.73472347597681e-18\\
599.08	0\\
599.09	0\\
599.1	0\\
599.11	1.73472347597681e-18\\
599.12	1.73472347597681e-18\\
599.13	0\\
599.14	0\\
599.15	1.73472347597681e-18\\
599.16	1.73472347597681e-18\\
599.17	1.73472347597681e-18\\
599.18	0\\
599.19	0\\
599.2	0\\
599.21	0\\
599.22	0\\
599.23	1.73472347597681e-18\\
599.24	1.73472347597681e-18\\
599.25	1.73472347597681e-18\\
599.26	1.73472347597681e-18\\
599.27	0\\
599.28	0\\
599.29	0\\
599.3	0\\
599.31	0\\
599.32	1.73472347597681e-18\\
599.33	1.73472347597681e-18\\
599.34	0\\
599.35	0\\
599.36	0\\
599.37	1.73472347597681e-18\\
599.38	0\\
599.39	0\\
599.4	0\\
599.41	0\\
599.42	0\\
599.43	0\\
599.44	0\\
599.45	0\\
599.46	1.73472347597681e-18\\
599.47	1.73472347597681e-18\\
599.48	1.73472347597681e-18\\
599.49	1.73472347597681e-18\\
599.5	1.73472347597681e-18\\
599.51	0\\
599.52	0\\
599.53	1.73472347597681e-18\\
599.54	0\\
599.55	0\\
599.56	0\\
599.57	1.73472347597681e-18\\
599.58	0\\
599.59	0\\
599.6	0\\
599.61	0\\
599.62	1.73472347597681e-18\\
599.63	0\\
599.64	1.73472347597681e-18\\
599.65	0\\
599.66	0\\
599.67	0\\
599.68	0\\
599.69	1.73472347597681e-18\\
599.7	0\\
599.71	1.73472347597681e-18\\
599.72	1.73472347597681e-18\\
599.73	1.73472347597681e-18\\
599.74	0\\
599.75	0\\
599.76	0\\
599.77	0\\
599.78	0\\
599.79	0\\
599.8	0\\
599.81	0\\
599.82	0\\
599.83	1.73472347597681e-18\\
599.84	0\\
599.85	0\\
599.86	0\\
599.87	0\\
599.88	0\\
599.89	0\\
599.9	0\\
599.91	0\\
599.92	0\\
599.93	0\\
599.94	0\\
599.95	0\\
599.96	0\\
599.97	0\\
599.98	0\\
599.99	0\\
600	0\\
};
\addplot [color=mycolor2,solid,forget plot]
  table[row sep=crcr]{%
0.01	1.73472347597681e-18\\
1.01	1.73472347597681e-18\\
2.01	0\\
3.01	1.73472347597681e-18\\
4.01	1.73472347597681e-18\\
5.01	0\\
6.01	0\\
7.01	1.73472347597681e-18\\
8.01	0\\
9.01	0\\
10.01	0\\
11.01	0\\
12.01	1.73472347597681e-18\\
13.01	1.73472347597681e-18\\
14.01	1.73472347597681e-18\\
15.01	1.73472347597681e-18\\
16.01	1.73472347597681e-18\\
17.01	1.73472347597681e-18\\
18.01	0\\
19.01	0\\
20.01	1.73472347597681e-18\\
21.01	0\\
22.01	1.73472347597681e-18\\
23.01	1.73472347597681e-18\\
24.01	1.73472347597681e-18\\
25.01	0\\
26.01	1.73472347597681e-18\\
27.01	0\\
28.01	1.73472347597681e-18\\
29.01	1.73472347597681e-18\\
30.01	0\\
31.01	0\\
32.01	1.73472347597681e-18\\
33.01	1.73472347597681e-18\\
34.01	1.73472347597681e-18\\
35.01	0\\
36.01	0\\
37.01	0\\
38.01	0\\
39.01	1.73472347597681e-18\\
40.01	1.73472347597681e-18\\
41.01	1.73472347597681e-18\\
42.01	1.73472347597681e-18\\
43.01	1.73472347597681e-18\\
44.01	0\\
45.01	1.73472347597681e-18\\
46.01	1.73472347597681e-18\\
47.01	1.73472347597681e-18\\
48.01	1.73472347597681e-18\\
49.01	0\\
50.01	1.73472347597681e-18\\
51.01	1.73472347597681e-18\\
52.01	0\\
53.01	0\\
54.01	0\\
55.01	1.73472347597681e-18\\
56.01	0\\
57.01	1.73472347597681e-18\\
58.01	0\\
59.01	0\\
60.01	1.73472347597681e-18\\
61.01	1.73472347597681e-18\\
62.01	1.73472347597681e-18\\
63.01	0\\
64.01	1.73472347597681e-18\\
65.01	0\\
66.01	1.73472347597681e-18\\
67.01	1.73472347597681e-18\\
68.01	0\\
69.01	1.73472347597681e-18\\
70.01	1.73472347597681e-18\\
71.01	0\\
72.01	0\\
73.01	1.73472347597681e-18\\
74.01	0\\
75.01	1.73472347597681e-18\\
76.01	0\\
77.01	0\\
78.01	0\\
79.01	0\\
80.01	1.73472347597681e-18\\
81.01	1.73472347597681e-18\\
82.01	0\\
83.01	0\\
84.01	0\\
85.01	1.73472347597681e-18\\
86.01	1.73472347597681e-18\\
87.01	0\\
88.01	0\\
89.01	0\\
90.01	1.73472347597681e-18\\
91.01	1.73472347597681e-18\\
92.01	0\\
93.01	1.73472347597681e-18\\
94.01	1.73472347597681e-18\\
95.01	0\\
96.01	1.73472347597681e-18\\
97.01	0\\
98.01	0\\
99.01	0\\
100.01	1.73472347597681e-18\\
101.01	0\\
102.01	1.73472347597681e-18\\
103.01	1.73472347597681e-18\\
104.01	0\\
105.01	0\\
106.01	0\\
107.01	0\\
108.01	1.73472347597681e-18\\
109.01	0\\
110.01	1.73472347597681e-18\\
111.01	0\\
112.01	0\\
113.01	0\\
114.01	1.73472347597681e-18\\
115.01	1.73472347597681e-18\\
116.01	0\\
117.01	0\\
118.01	0\\
119.01	0\\
120.01	0\\
121.01	0\\
122.01	0\\
123.01	1.73472347597681e-18\\
124.01	1.73472347597681e-18\\
125.01	1.73472347597681e-18\\
126.01	1.73472347597681e-18\\
127.01	0\\
128.01	1.73472347597681e-18\\
129.01	0\\
130.01	1.73472347597681e-18\\
131.01	1.73472347597681e-18\\
132.01	0\\
133.01	1.73472347597681e-18\\
134.01	1.73472347597681e-18\\
135.01	0\\
136.01	0\\
137.01	1.73472347597681e-18\\
138.01	0\\
139.01	1.73472347597681e-18\\
140.01	1.73472347597681e-18\\
141.01	1.73472347597681e-18\\
142.01	1.73472347597681e-18\\
143.01	0\\
144.01	1.73472347597681e-18\\
145.01	1.73472347597681e-18\\
146.01	1.73472347597681e-18\\
147.01	1.73472347597681e-18\\
148.01	1.73472347597681e-18\\
149.01	0\\
150.01	0\\
151.01	1.73472347597681e-18\\
152.01	1.73472347597681e-18\\
153.01	1.73472347597681e-18\\
154.01	1.73472347597681e-18\\
155.01	1.73472347597681e-18\\
156.01	1.73472347597681e-18\\
157.01	0\\
158.01	1.73472347597681e-18\\
159.01	1.73472347597681e-18\\
160.01	1.73472347597681e-18\\
161.01	1.73472347597681e-18\\
162.01	1.73472347597681e-18\\
163.01	1.73472347597681e-18\\
164.01	0\\
165.01	0\\
166.01	1.73472347597681e-18\\
167.01	0\\
168.01	1.73472347597681e-18\\
169.01	0\\
170.01	1.73472347597681e-18\\
171.01	0\\
172.01	0\\
173.01	0\\
174.01	1.73472347597681e-18\\
175.01	1.73472347597681e-18\\
176.01	1.73472347597681e-18\\
177.01	0\\
178.01	1.73472347597681e-18\\
179.01	0\\
180.01	0\\
181.01	1.73472347597681e-18\\
182.01	0\\
183.01	0\\
184.01	0\\
185.01	1.73472347597681e-18\\
186.01	0\\
187.01	0\\
188.01	0\\
189.01	0\\
190.01	1.73472347597681e-18\\
191.01	0\\
192.01	1.73472347597681e-18\\
193.01	0\\
194.01	1.73472347597681e-18\\
195.01	0\\
196.01	1.73472347597681e-18\\
197.01	0\\
198.01	0\\
199.01	0\\
200.01	0\\
201.01	0\\
202.01	1.73472347597681e-18\\
203.01	0\\
204.01	1.73472347597681e-18\\
205.01	1.73472347597681e-18\\
206.01	1.73472347597681e-18\\
207.01	1.73472347597681e-18\\
208.01	0\\
209.01	1.73472347597681e-18\\
210.01	0\\
211.01	1.73472347597681e-18\\
212.01	0\\
213.01	1.73472347597681e-18\\
214.01	0\\
215.01	0\\
216.01	0\\
217.01	0\\
218.01	1.73472347597681e-18\\
219.01	1.73472347597681e-18\\
220.01	0\\
221.01	0\\
222.01	0\\
223.01	1.73472347597681e-18\\
224.01	1.73472347597681e-18\\
225.01	0\\
226.01	1.73472347597681e-18\\
227.01	0\\
228.01	0\\
229.01	1.73472347597681e-18\\
230.01	0\\
231.01	0\\
232.01	0\\
233.01	0\\
234.01	1.73472347597681e-18\\
235.01	0\\
236.01	1.73472347597681e-18\\
237.01	1.73472347597681e-18\\
238.01	0\\
239.01	1.73472347597681e-18\\
240.01	0\\
241.01	1.73472347597681e-18\\
242.01	0\\
243.01	1.73472347597681e-18\\
244.01	1.73472347597681e-18\\
245.01	1.73472347597681e-18\\
246.01	1.73472347597681e-18\\
247.01	0\\
248.01	0\\
249.01	0\\
250.01	1.73472347597681e-18\\
251.01	0\\
252.01	1.73472347597681e-18\\
253.01	1.73472347597681e-18\\
254.01	1.73472347597681e-18\\
255.01	1.73472347597681e-18\\
256.01	1.73472347597681e-18\\
257.01	1.73472347597681e-18\\
258.01	0\\
259.01	0\\
260.01	0\\
261.01	0\\
262.01	1.73472347597681e-18\\
263.01	1.73472347597681e-18\\
264.01	0\\
265.01	1.73472347597681e-18\\
266.01	1.73472347597681e-18\\
267.01	0\\
268.01	0\\
269.01	1.73472347597681e-18\\
270.01	0\\
271.01	1.73472347597681e-18\\
272.01	0\\
273.01	1.73472347597681e-18\\
274.01	0\\
275.01	0\\
276.01	1.73472347597681e-18\\
277.01	1.73472347597681e-18\\
278.01	0\\
279.01	1.73472347597681e-18\\
280.01	0\\
281.01	1.73472347597681e-18\\
282.01	1.73472347597681e-18\\
283.01	1.73472347597681e-18\\
284.01	0\\
285.01	1.73472347597681e-18\\
286.01	1.73472347597681e-18\\
287.01	1.73472347597681e-18\\
288.01	1.73472347597681e-18\\
289.01	0\\
290.01	0\\
291.01	0\\
292.01	0\\
293.01	1.73472347597681e-18\\
294.01	1.73472347597681e-18\\
295.01	0\\
296.01	1.73472347597681e-18\\
297.01	0\\
298.01	1.73472347597681e-18\\
299.01	0\\
300.01	1.73472347597681e-18\\
301.01	0\\
302.01	0\\
303.01	0\\
304.01	1.73472347597681e-18\\
305.01	1.73472347597681e-18\\
306.01	0\\
307.01	1.73472347597681e-18\\
308.01	0\\
309.01	1.73472347597681e-18\\
310.01	1.73472347597681e-18\\
311.01	0\\
312.01	1.73472347597681e-18\\
313.01	1.73472347597681e-18\\
314.01	0\\
315.01	1.73472347597681e-18\\
316.01	0\\
317.01	1.73472347597681e-18\\
318.01	0\\
319.01	0\\
320.01	0\\
321.01	0\\
322.01	1.73472347597681e-18\\
323.01	1.73472347597681e-18\\
324.01	0\\
325.01	1.73472347597681e-18\\
326.01	1.73472347597681e-18\\
327.01	0\\
328.01	1.73472347597681e-18\\
329.01	0\\
330.01	0\\
331.01	1.73472347597681e-18\\
332.01	1.73472347597681e-18\\
333.01	1.73472347597681e-18\\
334.01	1.73472347597681e-18\\
335.01	1.73472347597681e-18\\
336.01	0\\
337.01	1.73472347597681e-18\\
338.01	0\\
339.01	0\\
340.01	1.73472347597681e-18\\
341.01	1.73472347597681e-18\\
342.01	1.73472347597681e-18\\
343.01	1.73472347597681e-18\\
344.01	0\\
345.01	0\\
346.01	0\\
347.01	1.73472347597681e-18\\
348.01	0\\
349.01	1.73472347597681e-18\\
350.01	0\\
351.01	0\\
352.01	0\\
353.01	1.73472347597681e-18\\
354.01	1.73472347597681e-18\\
355.01	0\\
356.01	0\\
357.01	1.73472347597681e-18\\
358.01	1.73472347597681e-18\\
359.01	1.73472347597681e-18\\
360.01	0\\
361.01	0\\
362.01	0\\
363.01	0\\
364.01	0\\
365.01	1.73472347597681e-18\\
366.01	1.73472347597681e-18\\
367.01	1.73472347597681e-18\\
368.01	1.73472347597681e-18\\
369.01	0\\
370.01	0\\
371.01	1.73472347597681e-18\\
372.01	1.73472347597681e-18\\
373.01	1.73472347597681e-18\\
374.01	0\\
375.01	1.73472347597681e-18\\
376.01	0\\
377.01	0\\
378.01	1.73472347597681e-18\\
379.01	0\\
380.01	1.73472347597681e-18\\
381.01	1.73472347597681e-18\\
382.01	1.73472347597681e-18\\
383.01	0\\
384.01	1.73472347597681e-18\\
385.01	0\\
386.01	1.73472347597681e-18\\
387.01	1.73472347597681e-18\\
388.01	1.73472347597681e-18\\
389.01	0\\
390.01	1.73472347597681e-18\\
391.01	1.73472347597681e-18\\
392.01	0\\
393.01	0\\
394.01	1.73472347597681e-18\\
395.01	1.73472347597681e-18\\
396.01	1.73472347597681e-18\\
397.01	1.73472347597681e-18\\
398.01	0\\
399.01	1.73472347597681e-18\\
400.01	0\\
401.01	0\\
402.01	0\\
403.01	1.73472347597681e-18\\
404.01	1.73472347597681e-18\\
405.01	1.73472347597681e-18\\
406.01	1.73472347597681e-18\\
407.01	1.73472347597681e-18\\
408.01	1.73472347597681e-18\\
409.01	0\\
410.01	1.73472347597681e-18\\
411.01	0\\
412.01	1.73472347597681e-18\\
413.01	1.73472347597681e-18\\
414.01	0\\
415.01	0\\
416.01	1.73472347597681e-18\\
417.01	1.73472347597681e-18\\
418.01	1.73472347597681e-18\\
419.01	1.73472347597681e-18\\
420.01	0\\
421.01	1.73472347597681e-18\\
422.01	1.73472347597681e-18\\
423.01	1.73472347597681e-18\\
424.01	1.73472347597681e-18\\
425.01	0\\
426.01	1.73472347597681e-18\\
427.01	1.73472347597681e-18\\
428.01	1.73472347597681e-18\\
429.01	0\\
430.01	1.73472347597681e-18\\
431.01	1.73472347597681e-18\\
432.01	1.73472347597681e-18\\
433.01	0\\
434.01	1.73472347597681e-18\\
435.01	1.73472347597681e-18\\
436.01	1.73472347597681e-18\\
437.01	1.73472347597681e-18\\
438.01	1.73472347597681e-18\\
439.01	0\\
440.01	1.73472347597681e-18\\
441.01	1.73472347597681e-18\\
442.01	0\\
443.01	0\\
444.01	0\\
445.01	1.73472347597681e-18\\
446.01	1.73472347597681e-18\\
447.01	0\\
448.01	1.73472347597681e-18\\
449.01	1.73472347597681e-18\\
450.01	0\\
451.01	0\\
452.01	0\\
453.01	1.73472347597681e-18\\
454.01	1.73472347597681e-18\\
455.01	0\\
456.01	1.73472347597681e-18\\
457.01	1.73472347597681e-18\\
458.01	0\\
459.01	1.73472347597681e-18\\
460.01	0\\
461.01	1.73472347597681e-18\\
462.01	0\\
463.01	0\\
464.01	0\\
465.01	1.73472347597681e-18\\
466.01	0\\
467.01	1.73472347597681e-18\\
468.01	0\\
469.01	1.73472347597681e-18\\
470.01	1.73472347597681e-18\\
471.01	1.73472347597681e-18\\
472.01	0\\
473.01	0\\
474.01	0\\
475.01	1.73472347597681e-18\\
476.01	0\\
477.01	0\\
478.01	1.73472347597681e-18\\
479.01	1.73472347597681e-18\\
480.01	0\\
481.01	0\\
482.01	1.73472347597681e-18\\
483.01	1.73472347597681e-18\\
484.01	0\\
485.01	0\\
486.01	0\\
487.01	1.73472347597681e-18\\
488.01	1.73472347597681e-18\\
489.01	1.73472347597681e-18\\
490.01	1.73472347597681e-18\\
491.01	0\\
492.01	1.73472347597681e-18\\
493.01	1.73472347597681e-18\\
494.01	0\\
495.01	1.73472347597681e-18\\
496.01	0\\
497.01	1.73472347597681e-18\\
498.01	1.73472347597681e-18\\
499.01	0\\
500.01	1.73472347597681e-18\\
501.01	1.73472347597681e-18\\
502.01	0\\
503.01	1.73472347597681e-18\\
504.01	1.73472347597681e-18\\
505.01	1.73472347597681e-18\\
506.01	0\\
507.01	0\\
508.01	0\\
509.01	0\\
510.01	1.73472347597681e-18\\
511.01	1.73472347597681e-18\\
512.01	0\\
513.01	1.73472347597681e-18\\
514.01	1.73472347597681e-18\\
515.01	0\\
516.01	0\\
517.01	1.73472347597681e-18\\
518.01	1.73472347597681e-18\\
519.01	1.73472347597681e-18\\
520.01	0\\
521.01	0\\
522.01	0\\
523.01	1.73472347597681e-18\\
524.01	0\\
525.01	1.73472347597681e-18\\
526.01	1.73472347597681e-18\\
527.01	0\\
528.01	0\\
529.01	0\\
530.01	1.73472347597681e-18\\
531.01	0\\
532.01	1.73472347597681e-18\\
533.01	0\\
534.01	1.73472347597681e-18\\
535.01	0\\
536.01	1.73472347597681e-18\\
537.01	1.73472347597681e-18\\
538.01	1.73472347597681e-18\\
539.01	0\\
540.01	0\\
541.01	1.73472347597681e-18\\
542.01	1.73472347597681e-18\\
543.01	0\\
544.01	1.73472347597681e-18\\
545.01	0\\
546.01	1.73472347597681e-18\\
547.01	0\\
548.01	0\\
549.01	0\\
550.01	0\\
551.01	0\\
552.01	0\\
553.01	0\\
554.01	0\\
555.01	1.73472347597681e-18\\
556.01	0\\
557.01	0\\
558.01	0\\
559.01	1.73472347597681e-18\\
560.01	1.73472347597681e-18\\
561.01	1.73472347597681e-18\\
562.01	1.73472347597681e-18\\
563.01	0\\
564.01	0\\
565.01	1.73472347597681e-18\\
566.01	0\\
567.01	0\\
568.01	0\\
569.01	0\\
570.01	0\\
571.01	0\\
572.01	1.73472347597681e-18\\
573.01	0\\
574.01	1.73472347597681e-18\\
575.01	0\\
576.01	1.73472347597681e-18\\
577.01	0\\
578.01	1.73472347597681e-18\\
579.01	0\\
580.01	0\\
581.01	1.73472347597681e-18\\
582.01	0\\
583.01	0\\
584.01	0\\
585.01	0\\
586.01	0\\
587.01	0\\
588.01	0\\
589.01	0\\
590.01	0\\
591.01	0\\
592.01	1.73472347597681e-18\\
593.01	0\\
594.01	0\\
595.01	0\\
596.01	0\\
597.01	0\\
598.01	0\\
599.01	0\\
599.02	0\\
599.03	1.73472347597681e-18\\
599.04	1.73472347597681e-18\\
599.05	0\\
599.06	0\\
599.07	1.73472347597681e-18\\
599.08	0\\
599.09	0\\
599.1	0\\
599.11	1.73472347597681e-18\\
599.12	1.73472347597681e-18\\
599.13	0\\
599.14	0\\
599.15	1.73472347597681e-18\\
599.16	1.73472347597681e-18\\
599.17	1.73472347597681e-18\\
599.18	0\\
599.19	0\\
599.2	0\\
599.21	0\\
599.22	0\\
599.23	1.73472347597681e-18\\
599.24	1.73472347597681e-18\\
599.25	1.73472347597681e-18\\
599.26	1.73472347597681e-18\\
599.27	0\\
599.28	0\\
599.29	0\\
599.3	0\\
599.31	0\\
599.32	1.73472347597681e-18\\
599.33	1.73472347597681e-18\\
599.34	0\\
599.35	0\\
599.36	0\\
599.37	1.73472347597681e-18\\
599.38	0\\
599.39	0\\
599.4	0\\
599.41	0\\
599.42	0\\
599.43	0\\
599.44	0\\
599.45	0\\
599.46	1.73472347597681e-18\\
599.47	1.73472347597681e-18\\
599.48	1.73472347597681e-18\\
599.49	1.73472347597681e-18\\
599.5	1.73472347597681e-18\\
599.51	0\\
599.52	0\\
599.53	1.73472347597681e-18\\
599.54	0\\
599.55	0\\
599.56	0\\
599.57	1.73472347597681e-18\\
599.58	0\\
599.59	0\\
599.6	0\\
599.61	0\\
599.62	1.73472347597681e-18\\
599.63	0\\
599.64	1.73472347597681e-18\\
599.65	0\\
599.66	0\\
599.67	0\\
599.68	0\\
599.69	1.73472347597681e-18\\
599.7	0\\
599.71	1.73472347597681e-18\\
599.72	1.73472347597681e-18\\
599.73	1.73472347597681e-18\\
599.74	0\\
599.75	0\\
599.76	0\\
599.77	0\\
599.78	0\\
599.79	0\\
599.8	0\\
599.81	0\\
599.82	0\\
599.83	1.73472347597681e-18\\
599.84	0\\
599.85	0\\
599.86	0\\
599.87	0\\
599.88	0\\
599.89	0\\
599.9	0\\
599.91	0\\
599.92	0\\
599.93	0\\
599.94	0\\
599.95	0\\
599.96	0\\
599.97	0\\
599.98	0\\
599.99	0\\
600	0\\
};
\addplot [color=mycolor3,solid,forget plot]
  table[row sep=crcr]{%
0.01	1.73472347597681e-18\\
1.01	1.73472347597681e-18\\
2.01	0\\
3.01	1.73472347597681e-18\\
4.01	1.73472347597681e-18\\
5.01	0\\
6.01	0\\
7.01	1.73472347597681e-18\\
8.01	0\\
9.01	0\\
10.01	0\\
11.01	0\\
12.01	1.73472347597681e-18\\
13.01	1.73472347597681e-18\\
14.01	1.73472347597681e-18\\
15.01	1.73472347597681e-18\\
16.01	1.73472347597681e-18\\
17.01	1.73472347597681e-18\\
18.01	0\\
19.01	0\\
20.01	1.73472347597681e-18\\
21.01	0\\
22.01	1.73472347597681e-18\\
23.01	1.73472347597681e-18\\
24.01	1.73472347597681e-18\\
25.01	0\\
26.01	1.73472347597681e-18\\
27.01	0\\
28.01	1.73472347597681e-18\\
29.01	1.73472347597681e-18\\
30.01	0\\
31.01	0\\
32.01	1.73472347597681e-18\\
33.01	1.73472347597681e-18\\
34.01	1.73472347597681e-18\\
35.01	0\\
36.01	0\\
37.01	0\\
38.01	0\\
39.01	1.73472347597681e-18\\
40.01	1.73472347597681e-18\\
41.01	1.73472347597681e-18\\
42.01	1.73472347597681e-18\\
43.01	1.73472347597681e-18\\
44.01	0\\
45.01	1.73472347597681e-18\\
46.01	1.73472347597681e-18\\
47.01	1.73472347597681e-18\\
48.01	1.73472347597681e-18\\
49.01	0\\
50.01	1.73472347597681e-18\\
51.01	1.73472347597681e-18\\
52.01	0\\
53.01	0\\
54.01	0\\
55.01	1.73472347597681e-18\\
56.01	0\\
57.01	1.73472347597681e-18\\
58.01	0\\
59.01	0\\
60.01	1.73472347597681e-18\\
61.01	1.73472347597681e-18\\
62.01	1.73472347597681e-18\\
63.01	0\\
64.01	1.73472347597681e-18\\
65.01	0\\
66.01	1.73472347597681e-18\\
67.01	1.73472347597681e-18\\
68.01	0\\
69.01	1.73472347597681e-18\\
70.01	1.73472347597681e-18\\
71.01	0\\
72.01	0\\
73.01	1.73472347597681e-18\\
74.01	0\\
75.01	1.73472347597681e-18\\
76.01	0\\
77.01	0\\
78.01	0\\
79.01	0\\
80.01	1.73472347597681e-18\\
81.01	1.73472347597681e-18\\
82.01	0\\
83.01	0\\
84.01	0\\
85.01	1.73472347597681e-18\\
86.01	1.73472347597681e-18\\
87.01	0\\
88.01	0\\
89.01	0\\
90.01	1.73472347597681e-18\\
91.01	1.73472347597681e-18\\
92.01	0\\
93.01	1.73472347597681e-18\\
94.01	1.73472347597681e-18\\
95.01	0\\
96.01	1.73472347597681e-18\\
97.01	0\\
98.01	0\\
99.01	0\\
100.01	1.73472347597681e-18\\
101.01	0\\
102.01	1.73472347597681e-18\\
103.01	1.73472347597681e-18\\
104.01	0\\
105.01	0\\
106.01	0\\
107.01	0\\
108.01	1.73472347597681e-18\\
109.01	0\\
110.01	1.73472347597681e-18\\
111.01	0\\
112.01	0\\
113.01	0\\
114.01	1.73472347597681e-18\\
115.01	1.73472347597681e-18\\
116.01	0\\
117.01	0\\
118.01	0\\
119.01	0\\
120.01	0\\
121.01	0\\
122.01	0\\
123.01	1.73472347597681e-18\\
124.01	1.73472347597681e-18\\
125.01	1.73472347597681e-18\\
126.01	1.73472347597681e-18\\
127.01	0\\
128.01	1.73472347597681e-18\\
129.01	0\\
130.01	1.73472347597681e-18\\
131.01	1.73472347597681e-18\\
132.01	0\\
133.01	1.73472347597681e-18\\
134.01	1.73472347597681e-18\\
135.01	0\\
136.01	0\\
137.01	1.73472347597681e-18\\
138.01	0\\
139.01	1.73472347597681e-18\\
140.01	1.73472347597681e-18\\
141.01	1.73472347597681e-18\\
142.01	1.73472347597681e-18\\
143.01	0\\
144.01	1.73472347597681e-18\\
145.01	1.73472347597681e-18\\
146.01	1.73472347597681e-18\\
147.01	1.73472347597681e-18\\
148.01	1.73472347597681e-18\\
149.01	0\\
150.01	0\\
151.01	1.73472347597681e-18\\
152.01	1.73472347597681e-18\\
153.01	1.73472347597681e-18\\
154.01	1.73472347597681e-18\\
155.01	1.73472347597681e-18\\
156.01	1.73472347597681e-18\\
157.01	0\\
158.01	1.73472347597681e-18\\
159.01	1.73472347597681e-18\\
160.01	1.73472347597681e-18\\
161.01	1.73472347597681e-18\\
162.01	1.73472347597681e-18\\
163.01	1.73472347597681e-18\\
164.01	0\\
165.01	0\\
166.01	1.73472347597681e-18\\
167.01	0\\
168.01	1.73472347597681e-18\\
169.01	0\\
170.01	1.73472347597681e-18\\
171.01	0\\
172.01	0\\
173.01	0\\
174.01	1.73472347597681e-18\\
175.01	1.73472347597681e-18\\
176.01	1.73472347597681e-18\\
177.01	0\\
178.01	1.73472347597681e-18\\
179.01	0\\
180.01	0\\
181.01	1.73472347597681e-18\\
182.01	0\\
183.01	0\\
184.01	0\\
185.01	1.73472347597681e-18\\
186.01	0\\
187.01	0\\
188.01	0\\
189.01	0\\
190.01	1.73472347597681e-18\\
191.01	0\\
192.01	1.73472347597681e-18\\
193.01	0\\
194.01	1.73472347597681e-18\\
195.01	0\\
196.01	1.73472347597681e-18\\
197.01	0\\
198.01	0\\
199.01	0\\
200.01	0\\
201.01	0\\
202.01	1.73472347597681e-18\\
203.01	0\\
204.01	1.73472347597681e-18\\
205.01	1.73472347597681e-18\\
206.01	1.73472347597681e-18\\
207.01	1.73472347597681e-18\\
208.01	0\\
209.01	1.73472347597681e-18\\
210.01	0\\
211.01	1.73472347597681e-18\\
212.01	0\\
213.01	1.73472347597681e-18\\
214.01	0\\
215.01	0\\
216.01	0\\
217.01	0\\
218.01	1.73472347597681e-18\\
219.01	1.73472347597681e-18\\
220.01	0\\
221.01	0\\
222.01	0\\
223.01	1.73472347597681e-18\\
224.01	1.73472347597681e-18\\
225.01	0\\
226.01	1.73472347597681e-18\\
227.01	0\\
228.01	0\\
229.01	1.73472347597681e-18\\
230.01	0\\
231.01	0\\
232.01	0\\
233.01	0\\
234.01	1.73472347597681e-18\\
235.01	0\\
236.01	1.73472347597681e-18\\
237.01	1.73472347597681e-18\\
238.01	0\\
239.01	1.73472347597681e-18\\
240.01	0\\
241.01	1.73472347597681e-18\\
242.01	0\\
243.01	1.73472347597681e-18\\
244.01	1.73472347597681e-18\\
245.01	1.73472347597681e-18\\
246.01	1.73472347597681e-18\\
247.01	0\\
248.01	0\\
249.01	0\\
250.01	1.73472347597681e-18\\
251.01	0\\
252.01	1.73472347597681e-18\\
253.01	1.73472347597681e-18\\
254.01	1.73472347597681e-18\\
255.01	1.73472347597681e-18\\
256.01	1.73472347597681e-18\\
257.01	1.73472347597681e-18\\
258.01	0\\
259.01	0\\
260.01	0\\
261.01	0\\
262.01	1.73472347597681e-18\\
263.01	1.73472347597681e-18\\
264.01	0\\
265.01	1.73472347597681e-18\\
266.01	1.73472347597681e-18\\
267.01	0\\
268.01	0\\
269.01	1.73472347597681e-18\\
270.01	0\\
271.01	1.73472347597681e-18\\
272.01	0\\
273.01	1.73472347597681e-18\\
274.01	0\\
275.01	0\\
276.01	1.73472347597681e-18\\
277.01	1.73472347597681e-18\\
278.01	0\\
279.01	1.73472347597681e-18\\
280.01	0\\
281.01	1.73472347597681e-18\\
282.01	1.73472347597681e-18\\
283.01	1.73472347597681e-18\\
284.01	0\\
285.01	1.73472347597681e-18\\
286.01	1.73472347597681e-18\\
287.01	1.73472347597681e-18\\
288.01	1.73472347597681e-18\\
289.01	0\\
290.01	0\\
291.01	0\\
292.01	0\\
293.01	1.73472347597681e-18\\
294.01	1.73472347597681e-18\\
295.01	0\\
296.01	1.73472347597681e-18\\
297.01	0\\
298.01	1.73472347597681e-18\\
299.01	0\\
300.01	1.73472347597681e-18\\
301.01	0\\
302.01	0\\
303.01	0\\
304.01	1.73472347597681e-18\\
305.01	1.73472347597681e-18\\
306.01	0\\
307.01	1.73472347597681e-18\\
308.01	0\\
309.01	1.73472347597681e-18\\
310.01	1.73472347597681e-18\\
311.01	0\\
312.01	1.73472347597681e-18\\
313.01	1.73472347597681e-18\\
314.01	0\\
315.01	1.73472347597681e-18\\
316.01	0\\
317.01	1.73472347597681e-18\\
318.01	0\\
319.01	0\\
320.01	0\\
321.01	0\\
322.01	1.73472347597681e-18\\
323.01	1.73472347597681e-18\\
324.01	0\\
325.01	1.73472347597681e-18\\
326.01	1.73472347597681e-18\\
327.01	0\\
328.01	1.73472347597681e-18\\
329.01	0\\
330.01	0\\
331.01	1.73472347597681e-18\\
332.01	1.73472347597681e-18\\
333.01	1.73472347597681e-18\\
334.01	1.73472347597681e-18\\
335.01	1.73472347597681e-18\\
336.01	0\\
337.01	1.73472347597681e-18\\
338.01	0\\
339.01	0\\
340.01	1.73472347597681e-18\\
341.01	1.73472347597681e-18\\
342.01	1.73472347597681e-18\\
343.01	1.73472347597681e-18\\
344.01	0\\
345.01	0\\
346.01	0\\
347.01	1.73472347597681e-18\\
348.01	0\\
349.01	1.73472347597681e-18\\
350.01	0\\
351.01	0\\
352.01	0\\
353.01	1.73472347597681e-18\\
354.01	1.73472347597681e-18\\
355.01	0\\
356.01	0\\
357.01	1.73472347597681e-18\\
358.01	1.73472347597681e-18\\
359.01	1.73472347597681e-18\\
360.01	0\\
361.01	0\\
362.01	0\\
363.01	0\\
364.01	0\\
365.01	1.73472347597681e-18\\
366.01	1.73472347597681e-18\\
367.01	1.73472347597681e-18\\
368.01	1.73472347597681e-18\\
369.01	0\\
370.01	0\\
371.01	1.73472347597681e-18\\
372.01	1.73472347597681e-18\\
373.01	1.73472347597681e-18\\
374.01	0\\
375.01	1.73472347597681e-18\\
376.01	0\\
377.01	0\\
378.01	1.73472347597681e-18\\
379.01	0\\
380.01	1.73472347597681e-18\\
381.01	1.73472347597681e-18\\
382.01	1.73472347597681e-18\\
383.01	0\\
384.01	1.73472347597681e-18\\
385.01	0\\
386.01	1.73472347597681e-18\\
387.01	1.73472347597681e-18\\
388.01	1.73472347597681e-18\\
389.01	0\\
390.01	1.73472347597681e-18\\
391.01	1.73472347597681e-18\\
392.01	0\\
393.01	0\\
394.01	1.73472347597681e-18\\
395.01	1.73472347597681e-18\\
396.01	1.73472347597681e-18\\
397.01	1.73472347597681e-18\\
398.01	0\\
399.01	1.73472347597681e-18\\
400.01	0\\
401.01	0\\
402.01	0\\
403.01	1.73472347597681e-18\\
404.01	1.73472347597681e-18\\
405.01	1.73472347597681e-18\\
406.01	1.73472347597681e-18\\
407.01	1.73472347597681e-18\\
408.01	1.73472347597681e-18\\
409.01	0\\
410.01	1.73472347597681e-18\\
411.01	0\\
412.01	1.73472347597681e-18\\
413.01	1.73472347597681e-18\\
414.01	0\\
415.01	0\\
416.01	1.73472347597681e-18\\
417.01	1.73472347597681e-18\\
418.01	1.73472347597681e-18\\
419.01	1.73472347597681e-18\\
420.01	0\\
421.01	1.73472347597681e-18\\
422.01	1.73472347597681e-18\\
423.01	1.73472347597681e-18\\
424.01	1.73472347597681e-18\\
425.01	0\\
426.01	1.73472347597681e-18\\
427.01	1.73472347597681e-18\\
428.01	1.73472347597681e-18\\
429.01	0\\
430.01	1.73472347597681e-18\\
431.01	1.73472347597681e-18\\
432.01	1.73472347597681e-18\\
433.01	0\\
434.01	1.73472347597681e-18\\
435.01	1.73472347597681e-18\\
436.01	1.73472347597681e-18\\
437.01	1.73472347597681e-18\\
438.01	1.73472347597681e-18\\
439.01	0\\
440.01	1.73472347597681e-18\\
441.01	1.73472347597681e-18\\
442.01	0\\
443.01	0\\
444.01	0\\
445.01	1.73472347597681e-18\\
446.01	1.73472347597681e-18\\
447.01	0\\
448.01	1.73472347597681e-18\\
449.01	1.73472347597681e-18\\
450.01	0\\
451.01	0\\
452.01	0\\
453.01	1.73472347597681e-18\\
454.01	1.73472347597681e-18\\
455.01	0\\
456.01	1.73472347597681e-18\\
457.01	1.73472347597681e-18\\
458.01	0\\
459.01	1.73472347597681e-18\\
460.01	0\\
461.01	1.73472347597681e-18\\
462.01	0\\
463.01	0\\
464.01	0\\
465.01	1.73472347597681e-18\\
466.01	0\\
467.01	1.73472347597681e-18\\
468.01	0\\
469.01	1.73472347597681e-18\\
470.01	1.73472347597681e-18\\
471.01	1.73472347597681e-18\\
472.01	0\\
473.01	0\\
474.01	0\\
475.01	1.73472347597681e-18\\
476.01	0\\
477.01	0\\
478.01	1.73472347597681e-18\\
479.01	1.73472347597681e-18\\
480.01	0\\
481.01	0\\
482.01	1.73472347597681e-18\\
483.01	1.73472347597681e-18\\
484.01	0\\
485.01	0\\
486.01	0\\
487.01	1.73472347597681e-18\\
488.01	1.73472347597681e-18\\
489.01	1.73472347597681e-18\\
490.01	1.73472347597681e-18\\
491.01	0\\
492.01	1.73472347597681e-18\\
493.01	1.73472347597681e-18\\
494.01	0\\
495.01	1.73472347597681e-18\\
496.01	0\\
497.01	1.73472347597681e-18\\
498.01	1.73472347597681e-18\\
499.01	0\\
500.01	1.73472347597681e-18\\
501.01	1.73472347597681e-18\\
502.01	0\\
503.01	1.73472347597681e-18\\
504.01	1.73472347597681e-18\\
505.01	1.73472347597681e-18\\
506.01	0\\
507.01	0\\
508.01	0\\
509.01	0\\
510.01	1.73472347597681e-18\\
511.01	1.73472347597681e-18\\
512.01	0\\
513.01	1.73472347597681e-18\\
514.01	1.73472347597681e-18\\
515.01	0\\
516.01	0\\
517.01	1.73472347597681e-18\\
518.01	1.73472347597681e-18\\
519.01	1.73472347597681e-18\\
520.01	0\\
521.01	0\\
522.01	0\\
523.01	1.73472347597681e-18\\
524.01	0\\
525.01	1.73472347597681e-18\\
526.01	1.73472347597681e-18\\
527.01	0\\
528.01	0\\
529.01	0\\
530.01	1.73472347597681e-18\\
531.01	0\\
532.01	1.73472347597681e-18\\
533.01	0\\
534.01	1.73472347597681e-18\\
535.01	0\\
536.01	1.73472347597681e-18\\
537.01	1.73472347597681e-18\\
538.01	1.73472347597681e-18\\
539.01	0\\
540.01	0\\
541.01	1.73472347597681e-18\\
542.01	1.73472347597681e-18\\
543.01	0\\
544.01	1.73472347597681e-18\\
545.01	0\\
546.01	1.73472347597681e-18\\
547.01	0\\
548.01	0\\
549.01	0\\
550.01	0\\
551.01	0\\
552.01	0\\
553.01	0\\
554.01	0\\
555.01	1.73472347597681e-18\\
556.01	0\\
557.01	0\\
558.01	0\\
559.01	1.73472347597681e-18\\
560.01	1.73472347597681e-18\\
561.01	1.73472347597681e-18\\
562.01	1.73472347597681e-18\\
563.01	0\\
564.01	0\\
565.01	1.73472347597681e-18\\
566.01	0\\
567.01	0\\
568.01	0\\
569.01	0\\
570.01	0\\
571.01	0\\
572.01	1.73472347597681e-18\\
573.01	0\\
574.01	1.73472347597681e-18\\
575.01	0\\
576.01	1.73472347597681e-18\\
577.01	0\\
578.01	1.73472347597681e-18\\
579.01	0\\
580.01	0\\
581.01	1.73472347597681e-18\\
582.01	0\\
583.01	0\\
584.01	0\\
585.01	0\\
586.01	0\\
587.01	0\\
588.01	0\\
589.01	0\\
590.01	0\\
591.01	0\\
592.01	1.73472347597681e-18\\
593.01	0\\
594.01	0\\
595.01	0\\
596.01	0\\
597.01	0\\
598.01	0\\
599.01	0\\
599.02	0\\
599.03	1.73472347597681e-18\\
599.04	1.73472347597681e-18\\
599.05	0\\
599.06	0\\
599.07	1.73472347597681e-18\\
599.08	0\\
599.09	0\\
599.1	0\\
599.11	1.73472347597681e-18\\
599.12	1.73472347597681e-18\\
599.13	0\\
599.14	0\\
599.15	1.73472347597681e-18\\
599.16	1.73472347597681e-18\\
599.17	1.73472347597681e-18\\
599.18	0\\
599.19	0\\
599.2	0\\
599.21	0\\
599.22	0\\
599.23	1.73472347597681e-18\\
599.24	1.73472347597681e-18\\
599.25	1.73472347597681e-18\\
599.26	1.73472347597681e-18\\
599.27	0\\
599.28	0\\
599.29	0\\
599.3	0\\
599.31	0\\
599.32	1.73472347597681e-18\\
599.33	1.73472347597681e-18\\
599.34	0\\
599.35	0\\
599.36	0\\
599.37	1.73472347597681e-18\\
599.38	0\\
599.39	0\\
599.4	0\\
599.41	0\\
599.42	0\\
599.43	0\\
599.44	0\\
599.45	0\\
599.46	1.73472347597681e-18\\
599.47	1.73472347597681e-18\\
599.48	1.73472347597681e-18\\
599.49	1.73472347597681e-18\\
599.5	1.73472347597681e-18\\
599.51	0\\
599.52	0\\
599.53	1.73472347597681e-18\\
599.54	0\\
599.55	0\\
599.56	0\\
599.57	1.73472347597681e-18\\
599.58	0\\
599.59	0\\
599.6	0\\
599.61	0\\
599.62	1.73472347597681e-18\\
599.63	0\\
599.64	1.73472347597681e-18\\
599.65	0\\
599.66	0\\
599.67	0\\
599.68	0\\
599.69	1.73472347597681e-18\\
599.7	0\\
599.71	1.73472347597681e-18\\
599.72	1.73472347597681e-18\\
599.73	1.73472347597681e-18\\
599.74	0\\
599.75	0\\
599.76	0\\
599.77	0\\
599.78	0\\
599.79	0\\
599.8	0\\
599.81	0\\
599.82	0\\
599.83	1.73472347597681e-18\\
599.84	0\\
599.85	0\\
599.86	0\\
599.87	0\\
599.88	0\\
599.89	0\\
599.9	0\\
599.91	0\\
599.92	0\\
599.93	0\\
599.94	0\\
599.95	0\\
599.96	0\\
599.97	0\\
599.98	0\\
599.99	0\\
600	0\\
};
\addplot [color=mycolor4,solid,forget plot]
  table[row sep=crcr]{%
0.01	1.73472347597681e-18\\
1.01	1.73472347597681e-18\\
2.01	0\\
3.01	1.73472347597681e-18\\
4.01	1.73472347597681e-18\\
5.01	0\\
6.01	0\\
7.01	1.73472347597681e-18\\
8.01	0\\
9.01	0\\
10.01	0\\
11.01	0\\
12.01	1.73472347597681e-18\\
13.01	1.73472347597681e-18\\
14.01	1.73472347597681e-18\\
15.01	1.73472347597681e-18\\
16.01	1.73472347597681e-18\\
17.01	1.73472347597681e-18\\
18.01	0\\
19.01	0\\
20.01	1.73472347597681e-18\\
21.01	0\\
22.01	1.73472347597681e-18\\
23.01	1.73472347597681e-18\\
24.01	1.73472347597681e-18\\
25.01	0\\
26.01	1.73472347597681e-18\\
27.01	0\\
28.01	1.73472347597681e-18\\
29.01	1.73472347597681e-18\\
30.01	0\\
31.01	0\\
32.01	1.73472347597681e-18\\
33.01	1.73472347597681e-18\\
34.01	1.73472347597681e-18\\
35.01	0\\
36.01	0\\
37.01	0\\
38.01	0\\
39.01	1.73472347597681e-18\\
40.01	1.73472347597681e-18\\
41.01	1.73472347597681e-18\\
42.01	1.73472347597681e-18\\
43.01	1.73472347597681e-18\\
44.01	0\\
45.01	1.73472347597681e-18\\
46.01	1.73472347597681e-18\\
47.01	1.73472347597681e-18\\
48.01	1.73472347597681e-18\\
49.01	0\\
50.01	1.73472347597681e-18\\
51.01	1.73472347597681e-18\\
52.01	0\\
53.01	0\\
54.01	0\\
55.01	1.73472347597681e-18\\
56.01	0\\
57.01	1.73472347597681e-18\\
58.01	0\\
59.01	0\\
60.01	1.73472347597681e-18\\
61.01	1.73472347597681e-18\\
62.01	1.73472347597681e-18\\
63.01	0\\
64.01	1.73472347597681e-18\\
65.01	0\\
66.01	1.73472347597681e-18\\
67.01	1.73472347597681e-18\\
68.01	0\\
69.01	1.73472347597681e-18\\
70.01	1.73472347597681e-18\\
71.01	0\\
72.01	0\\
73.01	1.73472347597681e-18\\
74.01	0\\
75.01	1.73472347597681e-18\\
76.01	0\\
77.01	0\\
78.01	0\\
79.01	0\\
80.01	1.73472347597681e-18\\
81.01	1.73472347597681e-18\\
82.01	0\\
83.01	0\\
84.01	0\\
85.01	1.73472347597681e-18\\
86.01	1.73472347597681e-18\\
87.01	0\\
88.01	0\\
89.01	0\\
90.01	1.73472347597681e-18\\
91.01	1.73472347597681e-18\\
92.01	0\\
93.01	1.73472347597681e-18\\
94.01	1.73472347597681e-18\\
95.01	0\\
96.01	1.73472347597681e-18\\
97.01	0\\
98.01	0\\
99.01	0\\
100.01	1.73472347597681e-18\\
101.01	0\\
102.01	1.73472347597681e-18\\
103.01	1.73472347597681e-18\\
104.01	0\\
105.01	0\\
106.01	0\\
107.01	0\\
108.01	1.73472347597681e-18\\
109.01	0\\
110.01	1.73472347597681e-18\\
111.01	0\\
112.01	0\\
113.01	0\\
114.01	1.73472347597681e-18\\
115.01	1.73472347597681e-18\\
116.01	0\\
117.01	0\\
118.01	0\\
119.01	0\\
120.01	0\\
121.01	0\\
122.01	0\\
123.01	1.73472347597681e-18\\
124.01	1.73472347597681e-18\\
125.01	1.73472347597681e-18\\
126.01	1.73472347597681e-18\\
127.01	0\\
128.01	1.73472347597681e-18\\
129.01	0\\
130.01	1.73472347597681e-18\\
131.01	1.73472347597681e-18\\
132.01	0\\
133.01	1.73472347597681e-18\\
134.01	1.73472347597681e-18\\
135.01	0\\
136.01	0\\
137.01	1.73472347597681e-18\\
138.01	0\\
139.01	1.73472347597681e-18\\
140.01	1.73472347597681e-18\\
141.01	1.73472347597681e-18\\
142.01	1.73472347597681e-18\\
143.01	0\\
144.01	1.73472347597681e-18\\
145.01	1.73472347597681e-18\\
146.01	1.73472347597681e-18\\
147.01	1.73472347597681e-18\\
148.01	1.73472347597681e-18\\
149.01	0\\
150.01	0\\
151.01	1.73472347597681e-18\\
152.01	1.73472347597681e-18\\
153.01	1.73472347597681e-18\\
154.01	1.73472347597681e-18\\
155.01	1.73472347597681e-18\\
156.01	1.73472347597681e-18\\
157.01	0\\
158.01	1.73472347597681e-18\\
159.01	1.73472347597681e-18\\
160.01	1.73472347597681e-18\\
161.01	1.73472347597681e-18\\
162.01	1.73472347597681e-18\\
163.01	1.73472347597681e-18\\
164.01	0\\
165.01	0\\
166.01	1.73472347597681e-18\\
167.01	0\\
168.01	1.73472347597681e-18\\
169.01	0\\
170.01	1.73472347597681e-18\\
171.01	0\\
172.01	0\\
173.01	0\\
174.01	1.73472347597681e-18\\
175.01	1.73472347597681e-18\\
176.01	1.73472347597681e-18\\
177.01	0\\
178.01	1.73472347597681e-18\\
179.01	0\\
180.01	0\\
181.01	1.73472347597681e-18\\
182.01	0\\
183.01	0\\
184.01	0\\
185.01	1.73472347597681e-18\\
186.01	0\\
187.01	0\\
188.01	0\\
189.01	0\\
190.01	1.73472347597681e-18\\
191.01	0\\
192.01	1.73472347597681e-18\\
193.01	0\\
194.01	1.73472347597681e-18\\
195.01	0\\
196.01	1.73472347597681e-18\\
197.01	0\\
198.01	0\\
199.01	0\\
200.01	0\\
201.01	0\\
202.01	1.73472347597681e-18\\
203.01	0\\
204.01	1.73472347597681e-18\\
205.01	1.73472347597681e-18\\
206.01	1.73472347597681e-18\\
207.01	1.73472347597681e-18\\
208.01	0\\
209.01	1.73472347597681e-18\\
210.01	0\\
211.01	1.73472347597681e-18\\
212.01	0\\
213.01	1.73472347597681e-18\\
214.01	0\\
215.01	0\\
216.01	0\\
217.01	0\\
218.01	1.73472347597681e-18\\
219.01	1.73472347597681e-18\\
220.01	0\\
221.01	0\\
222.01	0\\
223.01	1.73472347597681e-18\\
224.01	1.73472347597681e-18\\
225.01	0\\
226.01	1.73472347597681e-18\\
227.01	0\\
228.01	0\\
229.01	1.73472347597681e-18\\
230.01	0\\
231.01	0\\
232.01	0\\
233.01	0\\
234.01	1.73472347597681e-18\\
235.01	0\\
236.01	1.73472347597681e-18\\
237.01	1.73472347597681e-18\\
238.01	0\\
239.01	1.73472347597681e-18\\
240.01	0\\
241.01	1.73472347597681e-18\\
242.01	0\\
243.01	1.73472347597681e-18\\
244.01	1.73472347597681e-18\\
245.01	1.73472347597681e-18\\
246.01	1.73472347597681e-18\\
247.01	0\\
248.01	0\\
249.01	0\\
250.01	1.73472347597681e-18\\
251.01	0\\
252.01	1.73472347597681e-18\\
253.01	1.73472347597681e-18\\
254.01	1.73472347597681e-18\\
255.01	1.73472347597681e-18\\
256.01	1.73472347597681e-18\\
257.01	1.73472347597681e-18\\
258.01	0\\
259.01	0\\
260.01	0\\
261.01	0\\
262.01	1.73472347597681e-18\\
263.01	1.73472347597681e-18\\
264.01	0\\
265.01	1.73472347597681e-18\\
266.01	1.73472347597681e-18\\
267.01	0\\
268.01	0\\
269.01	1.73472347597681e-18\\
270.01	0\\
271.01	1.73472347597681e-18\\
272.01	0\\
273.01	1.73472347597681e-18\\
274.01	0\\
275.01	0\\
276.01	1.73472347597681e-18\\
277.01	1.73472347597681e-18\\
278.01	0\\
279.01	1.73472347597681e-18\\
280.01	0\\
281.01	1.73472347597681e-18\\
282.01	1.73472347597681e-18\\
283.01	1.73472347597681e-18\\
284.01	0\\
285.01	1.73472347597681e-18\\
286.01	1.73472347597681e-18\\
287.01	1.73472347597681e-18\\
288.01	1.73472347597681e-18\\
289.01	0\\
290.01	0\\
291.01	0\\
292.01	0\\
293.01	1.73472347597681e-18\\
294.01	1.73472347597681e-18\\
295.01	0\\
296.01	1.73472347597681e-18\\
297.01	0\\
298.01	1.73472347597681e-18\\
299.01	0\\
300.01	1.73472347597681e-18\\
301.01	0\\
302.01	0\\
303.01	0\\
304.01	1.73472347597681e-18\\
305.01	1.73472347597681e-18\\
306.01	0\\
307.01	1.73472347597681e-18\\
308.01	0\\
309.01	1.73472347597681e-18\\
310.01	1.73472347597681e-18\\
311.01	0\\
312.01	1.73472347597681e-18\\
313.01	1.73472347597681e-18\\
314.01	0\\
315.01	1.73472347597681e-18\\
316.01	0\\
317.01	1.73472347597681e-18\\
318.01	0\\
319.01	0\\
320.01	0\\
321.01	0\\
322.01	1.73472347597681e-18\\
323.01	1.73472347597681e-18\\
324.01	0\\
325.01	1.73472347597681e-18\\
326.01	1.73472347597681e-18\\
327.01	0\\
328.01	1.73472347597681e-18\\
329.01	0\\
330.01	0\\
331.01	1.73472347597681e-18\\
332.01	1.73472347597681e-18\\
333.01	1.73472347597681e-18\\
334.01	1.73472347597681e-18\\
335.01	1.73472347597681e-18\\
336.01	0\\
337.01	1.73472347597681e-18\\
338.01	0\\
339.01	0\\
340.01	1.73472347597681e-18\\
341.01	1.73472347597681e-18\\
342.01	1.73472347597681e-18\\
343.01	1.73472347597681e-18\\
344.01	0\\
345.01	0\\
346.01	0\\
347.01	1.73472347597681e-18\\
348.01	0\\
349.01	1.73472347597681e-18\\
350.01	0\\
351.01	0\\
352.01	0\\
353.01	1.73472347597681e-18\\
354.01	1.73472347597681e-18\\
355.01	0\\
356.01	0\\
357.01	1.73472347597681e-18\\
358.01	1.73472347597681e-18\\
359.01	1.73472347597681e-18\\
360.01	0\\
361.01	0\\
362.01	0\\
363.01	0\\
364.01	0\\
365.01	1.73472347597681e-18\\
366.01	1.73472347597681e-18\\
367.01	1.73472347597681e-18\\
368.01	1.73472347597681e-18\\
369.01	0\\
370.01	0\\
371.01	1.73472347597681e-18\\
372.01	1.73472347597681e-18\\
373.01	1.73472347597681e-18\\
374.01	0\\
375.01	1.73472347597681e-18\\
376.01	0\\
377.01	0\\
378.01	1.73472347597681e-18\\
379.01	0\\
380.01	1.73472347597681e-18\\
381.01	1.73472347597681e-18\\
382.01	1.73472347597681e-18\\
383.01	0\\
384.01	1.73472347597681e-18\\
385.01	0\\
386.01	1.73472347597681e-18\\
387.01	1.73472347597681e-18\\
388.01	1.73472347597681e-18\\
389.01	0\\
390.01	1.73472347597681e-18\\
391.01	1.73472347597681e-18\\
392.01	0\\
393.01	0\\
394.01	1.73472347597681e-18\\
395.01	1.73472347597681e-18\\
396.01	1.73472347597681e-18\\
397.01	1.73472347597681e-18\\
398.01	0\\
399.01	1.73472347597681e-18\\
400.01	0\\
401.01	0\\
402.01	0\\
403.01	1.73472347597681e-18\\
404.01	1.73472347597681e-18\\
405.01	1.73472347597681e-18\\
406.01	1.73472347597681e-18\\
407.01	1.73472347597681e-18\\
408.01	1.73472347597681e-18\\
409.01	0\\
410.01	1.73472347597681e-18\\
411.01	0\\
412.01	1.73472347597681e-18\\
413.01	1.73472347597681e-18\\
414.01	0\\
415.01	0\\
416.01	1.73472347597681e-18\\
417.01	1.73472347597681e-18\\
418.01	1.73472347597681e-18\\
419.01	1.73472347597681e-18\\
420.01	0\\
421.01	1.73472347597681e-18\\
422.01	1.73472347597681e-18\\
423.01	1.73472347597681e-18\\
424.01	1.73472347597681e-18\\
425.01	0\\
426.01	1.73472347597681e-18\\
427.01	1.73472347597681e-18\\
428.01	1.73472347597681e-18\\
429.01	0\\
430.01	1.73472347597681e-18\\
431.01	1.73472347597681e-18\\
432.01	1.73472347597681e-18\\
433.01	0\\
434.01	1.73472347597681e-18\\
435.01	1.73472347597681e-18\\
436.01	1.73472347597681e-18\\
437.01	1.73472347597681e-18\\
438.01	1.73472347597681e-18\\
439.01	0\\
440.01	1.73472347597681e-18\\
441.01	1.73472347597681e-18\\
442.01	0\\
443.01	0\\
444.01	0\\
445.01	1.73472347597681e-18\\
446.01	1.73472347597681e-18\\
447.01	0\\
448.01	1.73472347597681e-18\\
449.01	1.73472347597681e-18\\
450.01	0\\
451.01	0\\
452.01	0\\
453.01	1.73472347597681e-18\\
454.01	1.73472347597681e-18\\
455.01	0\\
456.01	1.73472347597681e-18\\
457.01	1.73472347597681e-18\\
458.01	0\\
459.01	1.73472347597681e-18\\
460.01	0\\
461.01	1.73472347597681e-18\\
462.01	0\\
463.01	0\\
464.01	0\\
465.01	1.73472347597681e-18\\
466.01	0\\
467.01	1.73472347597681e-18\\
468.01	0\\
469.01	1.73472347597681e-18\\
470.01	1.73472347597681e-18\\
471.01	1.73472347597681e-18\\
472.01	0\\
473.01	0\\
474.01	0\\
475.01	1.73472347597681e-18\\
476.01	0\\
477.01	0\\
478.01	1.73472347597681e-18\\
479.01	1.73472347597681e-18\\
480.01	0\\
481.01	0\\
482.01	1.73472347597681e-18\\
483.01	1.73472347597681e-18\\
484.01	0\\
485.01	0\\
486.01	0\\
487.01	1.73472347597681e-18\\
488.01	1.73472347597681e-18\\
489.01	1.73472347597681e-18\\
490.01	1.73472347597681e-18\\
491.01	0\\
492.01	1.73472347597681e-18\\
493.01	1.73472347597681e-18\\
494.01	0\\
495.01	1.73472347597681e-18\\
496.01	0\\
497.01	1.73472347597681e-18\\
498.01	1.73472347597681e-18\\
499.01	0\\
500.01	1.73472347597681e-18\\
501.01	1.73472347597681e-18\\
502.01	0\\
503.01	1.73472347597681e-18\\
504.01	1.73472347597681e-18\\
505.01	1.73472347597681e-18\\
506.01	0\\
507.01	0\\
508.01	0\\
509.01	0\\
510.01	1.73472347597681e-18\\
511.01	1.73472347597681e-18\\
512.01	0\\
513.01	1.73472347597681e-18\\
514.01	1.73472347597681e-18\\
515.01	0\\
516.01	0\\
517.01	1.73472347597681e-18\\
518.01	1.73472347597681e-18\\
519.01	1.73472347597681e-18\\
520.01	0\\
521.01	0\\
522.01	0\\
523.01	1.73472347597681e-18\\
524.01	0\\
525.01	1.73472347597681e-18\\
526.01	1.73472347597681e-18\\
527.01	0\\
528.01	0\\
529.01	0\\
530.01	1.73472347597681e-18\\
531.01	0\\
532.01	1.73472347597681e-18\\
533.01	0\\
534.01	1.73472347597681e-18\\
535.01	0\\
536.01	1.73472347597681e-18\\
537.01	1.73472347597681e-18\\
538.01	1.73472347597681e-18\\
539.01	0\\
540.01	0\\
541.01	1.73472347597681e-18\\
542.01	1.73472347597681e-18\\
543.01	0\\
544.01	1.73472347597681e-18\\
545.01	0\\
546.01	1.73472347597681e-18\\
547.01	0\\
548.01	0\\
549.01	0\\
550.01	0\\
551.01	0\\
552.01	0\\
553.01	0\\
554.01	0\\
555.01	1.73472347597681e-18\\
556.01	0\\
557.01	0\\
558.01	0\\
559.01	1.73472347597681e-18\\
560.01	1.73472347597681e-18\\
561.01	1.73472347597681e-18\\
562.01	1.73472347597681e-18\\
563.01	0\\
564.01	0\\
565.01	1.73472347597681e-18\\
566.01	0\\
567.01	0\\
568.01	0\\
569.01	0\\
570.01	0\\
571.01	0\\
572.01	1.73472347597681e-18\\
573.01	0\\
574.01	1.73472347597681e-18\\
575.01	0\\
576.01	1.73472347597681e-18\\
577.01	0\\
578.01	1.73472347597681e-18\\
579.01	0\\
580.01	0\\
581.01	1.73472347597681e-18\\
582.01	0\\
583.01	0\\
584.01	0\\
585.01	0\\
586.01	0\\
587.01	0\\
588.01	0\\
589.01	0\\
590.01	0\\
591.01	0\\
592.01	1.73472347597681e-18\\
593.01	0\\
594.01	0\\
595.01	0\\
596.01	0\\
597.01	0\\
598.01	0\\
599.01	0\\
599.02	0\\
599.03	1.73472347597681e-18\\
599.04	1.73472347597681e-18\\
599.05	0\\
599.06	0\\
599.07	1.73472347597681e-18\\
599.08	0\\
599.09	0\\
599.1	0\\
599.11	1.73472347597681e-18\\
599.12	1.73472347597681e-18\\
599.13	0\\
599.14	0\\
599.15	1.73472347597681e-18\\
599.16	1.73472347597681e-18\\
599.17	1.73472347597681e-18\\
599.18	0\\
599.19	0\\
599.2	0\\
599.21	0\\
599.22	0\\
599.23	1.73472347597681e-18\\
599.24	1.73472347597681e-18\\
599.25	1.73472347597681e-18\\
599.26	1.73472347597681e-18\\
599.27	0\\
599.28	0\\
599.29	0\\
599.3	0\\
599.31	0\\
599.32	1.73472347597681e-18\\
599.33	1.73472347597681e-18\\
599.34	0\\
599.35	0\\
599.36	0\\
599.37	1.73472347597681e-18\\
599.38	0\\
599.39	0\\
599.4	0\\
599.41	0\\
599.42	0\\
599.43	0\\
599.44	0\\
599.45	0\\
599.46	1.73472347597681e-18\\
599.47	1.73472347597681e-18\\
599.48	1.73472347597681e-18\\
599.49	1.73472347597681e-18\\
599.5	1.73472347597681e-18\\
599.51	0\\
599.52	0\\
599.53	1.73472347597681e-18\\
599.54	0\\
599.55	0\\
599.56	0\\
599.57	1.73472347597681e-18\\
599.58	0\\
599.59	0\\
599.6	0\\
599.61	0\\
599.62	1.73472347597681e-18\\
599.63	0\\
599.64	1.73472347597681e-18\\
599.65	0\\
599.66	0\\
599.67	0\\
599.68	0\\
599.69	1.73472347597681e-18\\
599.7	0\\
599.71	1.73472347597681e-18\\
599.72	1.73472347597681e-18\\
599.73	1.73472347597681e-18\\
599.74	0\\
599.75	0\\
599.76	0\\
599.77	0\\
599.78	0\\
599.79	0\\
599.8	0\\
599.81	0\\
599.82	0\\
599.83	1.73472347597681e-18\\
599.84	0\\
599.85	0\\
599.86	0\\
599.87	0\\
599.88	0\\
599.89	0\\
599.9	0\\
599.91	0\\
599.92	0\\
599.93	0\\
599.94	0\\
599.95	0\\
599.96	0\\
599.97	0\\
599.98	0\\
599.99	0\\
600	0\\
};
\addplot [color=mycolor5,solid,forget plot]
  table[row sep=crcr]{%
0.01	1.73472347597681e-18\\
1.01	1.73472347597681e-18\\
2.01	0\\
3.01	1.73472347597681e-18\\
4.01	1.73472347597681e-18\\
5.01	0\\
6.01	0\\
7.01	1.73472347597681e-18\\
8.01	0\\
9.01	0\\
10.01	0\\
11.01	0\\
12.01	1.73472347597681e-18\\
13.01	1.73472347597681e-18\\
14.01	1.73472347597681e-18\\
15.01	1.73472347597681e-18\\
16.01	1.73472347597681e-18\\
17.01	1.73472347597681e-18\\
18.01	0\\
19.01	0\\
20.01	1.73472347597681e-18\\
21.01	0\\
22.01	1.73472347597681e-18\\
23.01	1.73472347597681e-18\\
24.01	1.73472347597681e-18\\
25.01	0\\
26.01	1.73472347597681e-18\\
27.01	0\\
28.01	1.73472347597681e-18\\
29.01	1.73472347597681e-18\\
30.01	0\\
31.01	0\\
32.01	1.73472347597681e-18\\
33.01	1.73472347597681e-18\\
34.01	1.73472347597681e-18\\
35.01	0\\
36.01	0\\
37.01	0\\
38.01	0\\
39.01	1.73472347597681e-18\\
40.01	1.73472347597681e-18\\
41.01	1.73472347597681e-18\\
42.01	1.73472347597681e-18\\
43.01	1.73472347597681e-18\\
44.01	0\\
45.01	1.73472347597681e-18\\
46.01	1.73472347597681e-18\\
47.01	1.73472347597681e-18\\
48.01	1.73472347597681e-18\\
49.01	0\\
50.01	1.73472347597681e-18\\
51.01	1.73472347597681e-18\\
52.01	0\\
53.01	0\\
54.01	0\\
55.01	1.73472347597681e-18\\
56.01	0\\
57.01	1.73472347597681e-18\\
58.01	0\\
59.01	0\\
60.01	1.73472347597681e-18\\
61.01	1.73472347597681e-18\\
62.01	1.73472347597681e-18\\
63.01	0\\
64.01	1.73472347597681e-18\\
65.01	0\\
66.01	1.73472347597681e-18\\
67.01	1.73472347597681e-18\\
68.01	0\\
69.01	1.73472347597681e-18\\
70.01	1.73472347597681e-18\\
71.01	0\\
72.01	0\\
73.01	1.73472347597681e-18\\
74.01	0\\
75.01	1.73472347597681e-18\\
76.01	0\\
77.01	0\\
78.01	0\\
79.01	0\\
80.01	1.73472347597681e-18\\
81.01	1.73472347597681e-18\\
82.01	0\\
83.01	0\\
84.01	0\\
85.01	1.73472347597681e-18\\
86.01	1.73472347597681e-18\\
87.01	0\\
88.01	0\\
89.01	0\\
90.01	1.73472347597681e-18\\
91.01	1.73472347597681e-18\\
92.01	0\\
93.01	1.73472347597681e-18\\
94.01	1.73472347597681e-18\\
95.01	0\\
96.01	1.73472347597681e-18\\
97.01	0\\
98.01	0\\
99.01	0\\
100.01	1.73472347597681e-18\\
101.01	0\\
102.01	1.73472347597681e-18\\
103.01	1.73472347597681e-18\\
104.01	0\\
105.01	0\\
106.01	0\\
107.01	0\\
108.01	1.73472347597681e-18\\
109.01	0\\
110.01	1.73472347597681e-18\\
111.01	0\\
112.01	0\\
113.01	0\\
114.01	1.73472347597681e-18\\
115.01	1.73472347597681e-18\\
116.01	0\\
117.01	0\\
118.01	0\\
119.01	0\\
120.01	0\\
121.01	0\\
122.01	0\\
123.01	1.73472347597681e-18\\
124.01	1.73472347597681e-18\\
125.01	1.73472347597681e-18\\
126.01	1.73472347597681e-18\\
127.01	0\\
128.01	1.73472347597681e-18\\
129.01	0\\
130.01	1.73472347597681e-18\\
131.01	1.73472347597681e-18\\
132.01	0\\
133.01	1.73472347597681e-18\\
134.01	1.73472347597681e-18\\
135.01	0\\
136.01	0\\
137.01	1.73472347597681e-18\\
138.01	0\\
139.01	1.73472347597681e-18\\
140.01	1.73472347597681e-18\\
141.01	1.73472347597681e-18\\
142.01	1.73472347597681e-18\\
143.01	0\\
144.01	1.73472347597681e-18\\
145.01	1.73472347597681e-18\\
146.01	1.73472347597681e-18\\
147.01	1.73472347597681e-18\\
148.01	1.73472347597681e-18\\
149.01	0\\
150.01	0\\
151.01	1.73472347597681e-18\\
152.01	1.73472347597681e-18\\
153.01	1.73472347597681e-18\\
154.01	1.73472347597681e-18\\
155.01	1.73472347597681e-18\\
156.01	1.73472347597681e-18\\
157.01	0\\
158.01	1.73472347597681e-18\\
159.01	1.73472347597681e-18\\
160.01	1.73472347597681e-18\\
161.01	1.73472347597681e-18\\
162.01	1.73472347597681e-18\\
163.01	1.73472347597681e-18\\
164.01	0\\
165.01	0\\
166.01	1.73472347597681e-18\\
167.01	0\\
168.01	1.73472347597681e-18\\
169.01	0\\
170.01	1.73472347597681e-18\\
171.01	0\\
172.01	0\\
173.01	0\\
174.01	1.73472347597681e-18\\
175.01	1.73472347597681e-18\\
176.01	1.73472347597681e-18\\
177.01	0\\
178.01	1.73472347597681e-18\\
179.01	0\\
180.01	0\\
181.01	1.73472347597681e-18\\
182.01	0\\
183.01	0\\
184.01	0\\
185.01	1.73472347597681e-18\\
186.01	0\\
187.01	0\\
188.01	0\\
189.01	0\\
190.01	1.73472347597681e-18\\
191.01	0\\
192.01	1.73472347597681e-18\\
193.01	0\\
194.01	1.73472347597681e-18\\
195.01	0\\
196.01	1.73472347597681e-18\\
197.01	0\\
198.01	0\\
199.01	0\\
200.01	0\\
201.01	0\\
202.01	1.73472347597681e-18\\
203.01	0\\
204.01	1.73472347597681e-18\\
205.01	1.73472347597681e-18\\
206.01	1.73472347597681e-18\\
207.01	1.73472347597681e-18\\
208.01	0\\
209.01	1.73472347597681e-18\\
210.01	0\\
211.01	1.73472347597681e-18\\
212.01	0\\
213.01	1.73472347597681e-18\\
214.01	0\\
215.01	0\\
216.01	0\\
217.01	0\\
218.01	1.73472347597681e-18\\
219.01	1.73472347597681e-18\\
220.01	0\\
221.01	0\\
222.01	0\\
223.01	1.73472347597681e-18\\
224.01	1.73472347597681e-18\\
225.01	0\\
226.01	1.73472347597681e-18\\
227.01	0\\
228.01	0\\
229.01	1.73472347597681e-18\\
230.01	0\\
231.01	0\\
232.01	0\\
233.01	0\\
234.01	1.73472347597681e-18\\
235.01	0\\
236.01	1.73472347597681e-18\\
237.01	1.73472347597681e-18\\
238.01	0\\
239.01	1.73472347597681e-18\\
240.01	0\\
241.01	1.73472347597681e-18\\
242.01	0\\
243.01	1.73472347597681e-18\\
244.01	1.73472347597681e-18\\
245.01	1.73472347597681e-18\\
246.01	1.73472347597681e-18\\
247.01	0\\
248.01	0\\
249.01	0\\
250.01	1.73472347597681e-18\\
251.01	0\\
252.01	1.73472347597681e-18\\
253.01	1.73472347597681e-18\\
254.01	1.73472347597681e-18\\
255.01	1.73472347597681e-18\\
256.01	1.73472347597681e-18\\
257.01	1.73472347597681e-18\\
258.01	0\\
259.01	0\\
260.01	0\\
261.01	0\\
262.01	1.73472347597681e-18\\
263.01	1.73472347597681e-18\\
264.01	0\\
265.01	1.73472347597681e-18\\
266.01	1.73472347597681e-18\\
267.01	0\\
268.01	0\\
269.01	1.73472347597681e-18\\
270.01	0\\
271.01	1.73472347597681e-18\\
272.01	0\\
273.01	1.73472347597681e-18\\
274.01	0\\
275.01	0\\
276.01	1.73472347597681e-18\\
277.01	1.73472347597681e-18\\
278.01	0\\
279.01	1.73472347597681e-18\\
280.01	0\\
281.01	1.73472347597681e-18\\
282.01	1.73472347597681e-18\\
283.01	1.73472347597681e-18\\
284.01	0\\
285.01	1.73472347597681e-18\\
286.01	1.73472347597681e-18\\
287.01	1.73472347597681e-18\\
288.01	1.73472347597681e-18\\
289.01	0\\
290.01	0\\
291.01	0\\
292.01	0\\
293.01	1.73472347597681e-18\\
294.01	1.73472347597681e-18\\
295.01	0\\
296.01	1.73472347597681e-18\\
297.01	0\\
298.01	1.73472347597681e-18\\
299.01	0\\
300.01	1.73472347597681e-18\\
301.01	0\\
302.01	0\\
303.01	0\\
304.01	1.73472347597681e-18\\
305.01	1.73472347597681e-18\\
306.01	0\\
307.01	1.73472347597681e-18\\
308.01	0\\
309.01	1.73472347597681e-18\\
310.01	1.73472347597681e-18\\
311.01	0\\
312.01	1.73472347597681e-18\\
313.01	1.73472347597681e-18\\
314.01	0\\
315.01	1.73472347597681e-18\\
316.01	0\\
317.01	1.73472347597681e-18\\
318.01	0\\
319.01	0\\
320.01	0\\
321.01	0\\
322.01	1.73472347597681e-18\\
323.01	1.73472347597681e-18\\
324.01	0\\
325.01	1.73472347597681e-18\\
326.01	1.73472347597681e-18\\
327.01	0\\
328.01	1.73472347597681e-18\\
329.01	0\\
330.01	0\\
331.01	1.73472347597681e-18\\
332.01	1.73472347597681e-18\\
333.01	1.73472347597681e-18\\
334.01	1.73472347597681e-18\\
335.01	1.73472347597681e-18\\
336.01	0\\
337.01	1.73472347597681e-18\\
338.01	0\\
339.01	0\\
340.01	1.73472347597681e-18\\
341.01	1.73472347597681e-18\\
342.01	1.73472347597681e-18\\
343.01	1.73472347597681e-18\\
344.01	0\\
345.01	0\\
346.01	0\\
347.01	1.73472347597681e-18\\
348.01	0\\
349.01	1.73472347597681e-18\\
350.01	0\\
351.01	0\\
352.01	0\\
353.01	1.73472347597681e-18\\
354.01	1.73472347597681e-18\\
355.01	0\\
356.01	0\\
357.01	1.73472347597681e-18\\
358.01	1.73472347597681e-18\\
359.01	1.73472347597681e-18\\
360.01	0\\
361.01	0\\
362.01	0\\
363.01	0\\
364.01	0\\
365.01	1.73472347597681e-18\\
366.01	1.73472347597681e-18\\
367.01	1.73472347597681e-18\\
368.01	1.73472347597681e-18\\
369.01	0\\
370.01	0\\
371.01	1.73472347597681e-18\\
372.01	1.73472347597681e-18\\
373.01	1.73472347597681e-18\\
374.01	0\\
375.01	1.73472347597681e-18\\
376.01	0\\
377.01	0\\
378.01	1.73472347597681e-18\\
379.01	0\\
380.01	1.73472347597681e-18\\
381.01	1.73472347597681e-18\\
382.01	1.73472347597681e-18\\
383.01	0\\
384.01	1.73472347597681e-18\\
385.01	0\\
386.01	1.73472347597681e-18\\
387.01	1.73472347597681e-18\\
388.01	1.73472347597681e-18\\
389.01	0\\
390.01	1.73472347597681e-18\\
391.01	1.73472347597681e-18\\
392.01	0\\
393.01	0\\
394.01	1.73472347597681e-18\\
395.01	1.73472347597681e-18\\
396.01	1.73472347597681e-18\\
397.01	1.73472347597681e-18\\
398.01	0\\
399.01	1.73472347597681e-18\\
400.01	0\\
401.01	0\\
402.01	0\\
403.01	1.73472347597681e-18\\
404.01	1.73472347597681e-18\\
405.01	1.73472347597681e-18\\
406.01	1.73472347597681e-18\\
407.01	1.73472347597681e-18\\
408.01	1.73472347597681e-18\\
409.01	0\\
410.01	1.73472347597681e-18\\
411.01	0\\
412.01	1.73472347597681e-18\\
413.01	1.73472347597681e-18\\
414.01	0\\
415.01	0\\
416.01	1.73472347597681e-18\\
417.01	1.73472347597681e-18\\
418.01	1.73472347597681e-18\\
419.01	1.73472347597681e-18\\
420.01	0\\
421.01	1.73472347597681e-18\\
422.01	1.73472347597681e-18\\
423.01	1.73472347597681e-18\\
424.01	1.73472347597681e-18\\
425.01	0\\
426.01	1.73472347597681e-18\\
427.01	1.73472347597681e-18\\
428.01	1.73472347597681e-18\\
429.01	0\\
430.01	1.73472347597681e-18\\
431.01	1.73472347597681e-18\\
432.01	1.73472347597681e-18\\
433.01	0\\
434.01	1.73472347597681e-18\\
435.01	1.73472347597681e-18\\
436.01	1.73472347597681e-18\\
437.01	1.73472347597681e-18\\
438.01	1.73472347597681e-18\\
439.01	0\\
440.01	1.73472347597681e-18\\
441.01	1.73472347597681e-18\\
442.01	0\\
443.01	0\\
444.01	0\\
445.01	1.73472347597681e-18\\
446.01	1.73472347597681e-18\\
447.01	0\\
448.01	1.73472347597681e-18\\
449.01	1.73472347597681e-18\\
450.01	0\\
451.01	0\\
452.01	0\\
453.01	1.73472347597681e-18\\
454.01	1.73472347597681e-18\\
455.01	0\\
456.01	1.73472347597681e-18\\
457.01	1.73472347597681e-18\\
458.01	0\\
459.01	1.73472347597681e-18\\
460.01	0\\
461.01	1.73472347597681e-18\\
462.01	0\\
463.01	0\\
464.01	0\\
465.01	1.73472347597681e-18\\
466.01	0\\
467.01	1.73472347597681e-18\\
468.01	0\\
469.01	1.73472347597681e-18\\
470.01	1.73472347597681e-18\\
471.01	1.73472347597681e-18\\
472.01	0\\
473.01	0\\
474.01	0\\
475.01	1.73472347597681e-18\\
476.01	0\\
477.01	0\\
478.01	1.73472347597681e-18\\
479.01	1.73472347597681e-18\\
480.01	0\\
481.01	0\\
482.01	1.73472347597681e-18\\
483.01	1.73472347597681e-18\\
484.01	0\\
485.01	0\\
486.01	0\\
487.01	1.73472347597681e-18\\
488.01	1.73472347597681e-18\\
489.01	1.73472347597681e-18\\
490.01	1.73472347597681e-18\\
491.01	0\\
492.01	1.73472347597681e-18\\
493.01	1.73472347597681e-18\\
494.01	0\\
495.01	1.73472347597681e-18\\
496.01	0\\
497.01	1.73472347597681e-18\\
498.01	1.73472347597681e-18\\
499.01	0\\
500.01	1.73472347597681e-18\\
501.01	1.73472347597681e-18\\
502.01	0\\
503.01	1.73472347597681e-18\\
504.01	1.73472347597681e-18\\
505.01	1.73472347597681e-18\\
506.01	0\\
507.01	0\\
508.01	0\\
509.01	0\\
510.01	1.73472347597681e-18\\
511.01	1.73472347597681e-18\\
512.01	0\\
513.01	1.73472347597681e-18\\
514.01	1.73472347597681e-18\\
515.01	0\\
516.01	0\\
517.01	1.73472347597681e-18\\
518.01	1.73472347597681e-18\\
519.01	1.73472347597681e-18\\
520.01	0\\
521.01	0\\
522.01	0\\
523.01	1.73472347597681e-18\\
524.01	0\\
525.01	1.73472347597681e-18\\
526.01	1.73472347597681e-18\\
527.01	0\\
528.01	0\\
529.01	0\\
530.01	1.73472347597681e-18\\
531.01	0\\
532.01	1.73472347597681e-18\\
533.01	0\\
534.01	1.73472347597681e-18\\
535.01	0\\
536.01	1.73472347597681e-18\\
537.01	1.73472347597681e-18\\
538.01	1.73472347597681e-18\\
539.01	0\\
540.01	0\\
541.01	1.73472347597681e-18\\
542.01	1.73472347597681e-18\\
543.01	0\\
544.01	1.73472347597681e-18\\
545.01	0\\
546.01	1.73472347597681e-18\\
547.01	0\\
548.01	0\\
549.01	0\\
550.01	0\\
551.01	0\\
552.01	0\\
553.01	0\\
554.01	0\\
555.01	1.73472347597681e-18\\
556.01	0\\
557.01	0\\
558.01	0\\
559.01	1.73472347597681e-18\\
560.01	1.73472347597681e-18\\
561.01	1.73472347597681e-18\\
562.01	1.73472347597681e-18\\
563.01	0\\
564.01	0\\
565.01	1.73472347597681e-18\\
566.01	0\\
567.01	0\\
568.01	0\\
569.01	0\\
570.01	0\\
571.01	0\\
572.01	1.73472347597681e-18\\
573.01	0\\
574.01	1.73472347597681e-18\\
575.01	0\\
576.01	1.73472347597681e-18\\
577.01	0\\
578.01	1.73472347597681e-18\\
579.01	0\\
580.01	0\\
581.01	1.73472347597681e-18\\
582.01	0\\
583.01	0\\
584.01	0\\
585.01	0\\
586.01	0\\
587.01	0\\
588.01	0\\
589.01	0\\
590.01	0\\
591.01	0\\
592.01	1.73472347597681e-18\\
593.01	0\\
594.01	0\\
595.01	0\\
596.01	0\\
597.01	0\\
598.01	0\\
599.01	0\\
599.02	0\\
599.03	1.73472347597681e-18\\
599.04	1.73472347597681e-18\\
599.05	0\\
599.06	0\\
599.07	1.73472347597681e-18\\
599.08	0\\
599.09	0\\
599.1	0\\
599.11	1.73472347597681e-18\\
599.12	1.73472347597681e-18\\
599.13	0\\
599.14	0\\
599.15	1.73472347597681e-18\\
599.16	1.73472347597681e-18\\
599.17	1.73472347597681e-18\\
599.18	0\\
599.19	0\\
599.2	0\\
599.21	0\\
599.22	0\\
599.23	1.73472347597681e-18\\
599.24	1.73472347597681e-18\\
599.25	1.73472347597681e-18\\
599.26	1.73472347597681e-18\\
599.27	0\\
599.28	0\\
599.29	0\\
599.3	0\\
599.31	0\\
599.32	1.73472347597681e-18\\
599.33	1.73472347597681e-18\\
599.34	0\\
599.35	0\\
599.36	0\\
599.37	1.73472347597681e-18\\
599.38	0\\
599.39	0\\
599.4	0\\
599.41	0\\
599.42	0\\
599.43	0\\
599.44	0\\
599.45	0\\
599.46	1.73472347597681e-18\\
599.47	1.73472347597681e-18\\
599.48	1.73472347597681e-18\\
599.49	1.73472347597681e-18\\
599.5	1.73472347597681e-18\\
599.51	0\\
599.52	0\\
599.53	1.73472347597681e-18\\
599.54	0\\
599.55	0\\
599.56	0\\
599.57	1.73472347597681e-18\\
599.58	0\\
599.59	0\\
599.6	0\\
599.61	0\\
599.62	1.73472347597681e-18\\
599.63	0\\
599.64	1.73472347597681e-18\\
599.65	0\\
599.66	0\\
599.67	0\\
599.68	0\\
599.69	1.73472347597681e-18\\
599.7	0\\
599.71	1.73472347597681e-18\\
599.72	1.73472347597681e-18\\
599.73	1.73472347597681e-18\\
599.74	0\\
599.75	0\\
599.76	0\\
599.77	0\\
599.78	0\\
599.79	0\\
599.8	0\\
599.81	0\\
599.82	0\\
599.83	1.73472347597681e-18\\
599.84	0\\
599.85	0\\
599.86	0\\
599.87	0\\
599.88	0\\
599.89	0\\
599.9	0\\
599.91	0\\
599.92	0\\
599.93	0\\
599.94	0\\
599.95	0\\
599.96	0\\
599.97	0\\
599.98	0\\
599.99	0\\
600	0\\
};
\addplot [color=mycolor6,solid,forget plot]
  table[row sep=crcr]{%
0.01	1.73472347597681e-18\\
1.01	1.73472347597681e-18\\
2.01	0\\
3.01	1.73472347597681e-18\\
4.01	1.73472347597681e-18\\
5.01	0\\
6.01	0\\
7.01	1.73472347597681e-18\\
8.01	0\\
9.01	0\\
10.01	0\\
11.01	0\\
12.01	1.73472347597681e-18\\
13.01	1.73472347597681e-18\\
14.01	1.73472347597681e-18\\
15.01	1.73472347597681e-18\\
16.01	1.73472347597681e-18\\
17.01	1.73472347597681e-18\\
18.01	0\\
19.01	0\\
20.01	1.73472347597681e-18\\
21.01	0\\
22.01	1.73472347597681e-18\\
23.01	1.73472347597681e-18\\
24.01	1.73472347597681e-18\\
25.01	0\\
26.01	1.73472347597681e-18\\
27.01	0\\
28.01	1.73472347597681e-18\\
29.01	1.73472347597681e-18\\
30.01	0\\
31.01	0\\
32.01	1.73472347597681e-18\\
33.01	1.73472347597681e-18\\
34.01	1.73472347597681e-18\\
35.01	0\\
36.01	0\\
37.01	0\\
38.01	0\\
39.01	1.73472347597681e-18\\
40.01	1.73472347597681e-18\\
41.01	1.73472347597681e-18\\
42.01	1.73472347597681e-18\\
43.01	1.73472347597681e-18\\
44.01	0\\
45.01	1.73472347597681e-18\\
46.01	1.73472347597681e-18\\
47.01	1.73472347597681e-18\\
48.01	1.73472347597681e-18\\
49.01	0\\
50.01	1.73472347597681e-18\\
51.01	1.73472347597681e-18\\
52.01	0\\
53.01	0\\
54.01	0\\
55.01	1.73472347597681e-18\\
56.01	0\\
57.01	1.73472347597681e-18\\
58.01	0\\
59.01	0\\
60.01	1.73472347597681e-18\\
61.01	1.73472347597681e-18\\
62.01	1.73472347597681e-18\\
63.01	0\\
64.01	1.73472347597681e-18\\
65.01	0\\
66.01	1.73472347597681e-18\\
67.01	1.73472347597681e-18\\
68.01	0\\
69.01	1.73472347597681e-18\\
70.01	1.73472347597681e-18\\
71.01	0\\
72.01	0\\
73.01	1.73472347597681e-18\\
74.01	0\\
75.01	1.73472347597681e-18\\
76.01	0\\
77.01	0\\
78.01	0\\
79.01	0\\
80.01	1.73472347597681e-18\\
81.01	1.73472347597681e-18\\
82.01	0\\
83.01	0\\
84.01	0\\
85.01	1.73472347597681e-18\\
86.01	1.73472347597681e-18\\
87.01	0\\
88.01	0\\
89.01	0\\
90.01	1.73472347597681e-18\\
91.01	1.73472347597681e-18\\
92.01	0\\
93.01	1.73472347597681e-18\\
94.01	1.73472347597681e-18\\
95.01	0\\
96.01	1.73472347597681e-18\\
97.01	0\\
98.01	0\\
99.01	0\\
100.01	1.73472347597681e-18\\
101.01	0\\
102.01	1.73472347597681e-18\\
103.01	1.73472347597681e-18\\
104.01	0\\
105.01	0\\
106.01	0\\
107.01	0\\
108.01	1.73472347597681e-18\\
109.01	0\\
110.01	1.73472347597681e-18\\
111.01	0\\
112.01	0\\
113.01	0\\
114.01	1.73472347597681e-18\\
115.01	1.73472347597681e-18\\
116.01	0\\
117.01	0\\
118.01	0\\
119.01	0\\
120.01	0\\
121.01	0\\
122.01	0\\
123.01	1.73472347597681e-18\\
124.01	1.73472347597681e-18\\
125.01	1.73472347597681e-18\\
126.01	1.73472347597681e-18\\
127.01	0\\
128.01	1.73472347597681e-18\\
129.01	0\\
130.01	1.73472347597681e-18\\
131.01	1.73472347597681e-18\\
132.01	0\\
133.01	1.73472347597681e-18\\
134.01	1.73472347597681e-18\\
135.01	0\\
136.01	0\\
137.01	1.73472347597681e-18\\
138.01	0\\
139.01	1.73472347597681e-18\\
140.01	1.73472347597681e-18\\
141.01	1.73472347597681e-18\\
142.01	1.73472347597681e-18\\
143.01	0\\
144.01	1.73472347597681e-18\\
145.01	1.73472347597681e-18\\
146.01	1.73472347597681e-18\\
147.01	1.73472347597681e-18\\
148.01	1.73472347597681e-18\\
149.01	0\\
150.01	0\\
151.01	1.73472347597681e-18\\
152.01	1.73472347597681e-18\\
153.01	1.73472347597681e-18\\
154.01	1.73472347597681e-18\\
155.01	1.73472347597681e-18\\
156.01	1.73472347597681e-18\\
157.01	0\\
158.01	1.73472347597681e-18\\
159.01	1.73472347597681e-18\\
160.01	1.73472347597681e-18\\
161.01	1.73472347597681e-18\\
162.01	1.73472347597681e-18\\
163.01	1.73472347597681e-18\\
164.01	0\\
165.01	0\\
166.01	1.73472347597681e-18\\
167.01	0\\
168.01	1.73472347597681e-18\\
169.01	0\\
170.01	1.73472347597681e-18\\
171.01	0\\
172.01	0\\
173.01	0\\
174.01	1.73472347597681e-18\\
175.01	1.73472347597681e-18\\
176.01	1.73472347597681e-18\\
177.01	0\\
178.01	1.73472347597681e-18\\
179.01	0\\
180.01	0\\
181.01	1.73472347597681e-18\\
182.01	0\\
183.01	0\\
184.01	0\\
185.01	1.73472347597681e-18\\
186.01	0\\
187.01	0\\
188.01	0\\
189.01	0\\
190.01	1.73472347597681e-18\\
191.01	0\\
192.01	1.73472347597681e-18\\
193.01	0\\
194.01	1.73472347597681e-18\\
195.01	0\\
196.01	1.73472347597681e-18\\
197.01	0\\
198.01	0\\
199.01	0\\
200.01	0\\
201.01	0\\
202.01	1.73472347597681e-18\\
203.01	0\\
204.01	1.73472347597681e-18\\
205.01	1.73472347597681e-18\\
206.01	1.73472347597681e-18\\
207.01	1.73472347597681e-18\\
208.01	0\\
209.01	1.73472347597681e-18\\
210.01	0\\
211.01	1.73472347597681e-18\\
212.01	0\\
213.01	1.73472347597681e-18\\
214.01	0\\
215.01	0\\
216.01	0\\
217.01	0\\
218.01	1.73472347597681e-18\\
219.01	1.73472347597681e-18\\
220.01	0\\
221.01	0\\
222.01	0\\
223.01	1.73472347597681e-18\\
224.01	1.73472347597681e-18\\
225.01	0\\
226.01	1.73472347597681e-18\\
227.01	0\\
228.01	0\\
229.01	1.73472347597681e-18\\
230.01	0\\
231.01	0\\
232.01	0\\
233.01	0\\
234.01	1.73472347597681e-18\\
235.01	0\\
236.01	1.73472347597681e-18\\
237.01	1.73472347597681e-18\\
238.01	0\\
239.01	1.73472347597681e-18\\
240.01	0\\
241.01	1.73472347597681e-18\\
242.01	0\\
243.01	1.73472347597681e-18\\
244.01	1.73472347597681e-18\\
245.01	1.73472347597681e-18\\
246.01	1.73472347597681e-18\\
247.01	0\\
248.01	0\\
249.01	0\\
250.01	1.73472347597681e-18\\
251.01	0\\
252.01	1.73472347597681e-18\\
253.01	1.73472347597681e-18\\
254.01	1.73472347597681e-18\\
255.01	1.73472347597681e-18\\
256.01	1.73472347597681e-18\\
257.01	1.73472347597681e-18\\
258.01	0\\
259.01	0\\
260.01	0\\
261.01	0\\
262.01	1.73472347597681e-18\\
263.01	1.73472347597681e-18\\
264.01	0\\
265.01	1.73472347597681e-18\\
266.01	1.73472347597681e-18\\
267.01	0\\
268.01	0\\
269.01	1.73472347597681e-18\\
270.01	0\\
271.01	1.73472347597681e-18\\
272.01	0\\
273.01	1.73472347597681e-18\\
274.01	0\\
275.01	0\\
276.01	1.73472347597681e-18\\
277.01	1.73472347597681e-18\\
278.01	0\\
279.01	1.73472347597681e-18\\
280.01	0\\
281.01	1.73472347597681e-18\\
282.01	1.73472347597681e-18\\
283.01	1.73472347597681e-18\\
284.01	0\\
285.01	1.73472347597681e-18\\
286.01	1.73472347597681e-18\\
287.01	1.73472347597681e-18\\
288.01	1.73472347597681e-18\\
289.01	0\\
290.01	0\\
291.01	0\\
292.01	0\\
293.01	1.73472347597681e-18\\
294.01	1.73472347597681e-18\\
295.01	0\\
296.01	1.73472347597681e-18\\
297.01	0\\
298.01	1.73472347597681e-18\\
299.01	0\\
300.01	1.73472347597681e-18\\
301.01	0\\
302.01	0\\
303.01	0\\
304.01	1.73472347597681e-18\\
305.01	1.73472347597681e-18\\
306.01	0\\
307.01	1.73472347597681e-18\\
308.01	0\\
309.01	1.73472347597681e-18\\
310.01	1.73472347597681e-18\\
311.01	0\\
312.01	1.73472347597681e-18\\
313.01	1.73472347597681e-18\\
314.01	0\\
315.01	1.73472347597681e-18\\
316.01	0\\
317.01	1.73472347597681e-18\\
318.01	0\\
319.01	0\\
320.01	0\\
321.01	0\\
322.01	1.73472347597681e-18\\
323.01	1.73472347597681e-18\\
324.01	0\\
325.01	1.73472347597681e-18\\
326.01	1.73472347597681e-18\\
327.01	0\\
328.01	1.73472347597681e-18\\
329.01	0\\
330.01	0\\
331.01	1.73472347597681e-18\\
332.01	1.73472347597681e-18\\
333.01	1.73472347597681e-18\\
334.01	1.73472347597681e-18\\
335.01	1.73472347597681e-18\\
336.01	0\\
337.01	1.73472347597681e-18\\
338.01	0\\
339.01	0\\
340.01	1.73472347597681e-18\\
341.01	1.73472347597681e-18\\
342.01	1.73472347597681e-18\\
343.01	1.73472347597681e-18\\
344.01	0\\
345.01	0\\
346.01	0\\
347.01	1.73472347597681e-18\\
348.01	0\\
349.01	1.73472347597681e-18\\
350.01	0\\
351.01	0\\
352.01	0\\
353.01	1.73472347597681e-18\\
354.01	1.73472347597681e-18\\
355.01	0\\
356.01	0\\
357.01	1.73472347597681e-18\\
358.01	1.73472347597681e-18\\
359.01	1.73472347597681e-18\\
360.01	0\\
361.01	0\\
362.01	0\\
363.01	0\\
364.01	0\\
365.01	1.73472347597681e-18\\
366.01	1.73472347597681e-18\\
367.01	1.73472347597681e-18\\
368.01	1.73472347597681e-18\\
369.01	0\\
370.01	0\\
371.01	1.73472347597681e-18\\
372.01	1.73472347597681e-18\\
373.01	1.73472347597681e-18\\
374.01	0\\
375.01	1.73472347597681e-18\\
376.01	0\\
377.01	0\\
378.01	1.73472347597681e-18\\
379.01	0\\
380.01	1.73472347597681e-18\\
381.01	1.73472347597681e-18\\
382.01	1.73472347597681e-18\\
383.01	0\\
384.01	1.73472347597681e-18\\
385.01	0\\
386.01	1.73472347597681e-18\\
387.01	1.73472347597681e-18\\
388.01	1.73472347597681e-18\\
389.01	0\\
390.01	1.73472347597681e-18\\
391.01	1.73472347597681e-18\\
392.01	0\\
393.01	0\\
394.01	1.73472347597681e-18\\
395.01	1.73472347597681e-18\\
396.01	1.73472347597681e-18\\
397.01	1.73472347597681e-18\\
398.01	0\\
399.01	1.73472347597681e-18\\
400.01	0\\
401.01	0\\
402.01	0\\
403.01	1.73472347597681e-18\\
404.01	1.73472347597681e-18\\
405.01	1.73472347597681e-18\\
406.01	1.73472347597681e-18\\
407.01	1.73472347597681e-18\\
408.01	1.73472347597681e-18\\
409.01	0\\
410.01	1.73472347597681e-18\\
411.01	0\\
412.01	1.73472347597681e-18\\
413.01	1.73472347597681e-18\\
414.01	0\\
415.01	0\\
416.01	1.73472347597681e-18\\
417.01	1.73472347597681e-18\\
418.01	1.73472347597681e-18\\
419.01	1.73472347597681e-18\\
420.01	0\\
421.01	1.73472347597681e-18\\
422.01	1.73472347597681e-18\\
423.01	1.73472347597681e-18\\
424.01	1.73472347597681e-18\\
425.01	0\\
426.01	1.73472347597681e-18\\
427.01	1.73472347597681e-18\\
428.01	1.73472347597681e-18\\
429.01	0\\
430.01	1.73472347597681e-18\\
431.01	1.73472347597681e-18\\
432.01	1.73472347597681e-18\\
433.01	0\\
434.01	1.73472347597681e-18\\
435.01	1.73472347597681e-18\\
436.01	1.73472347597681e-18\\
437.01	1.73472347597681e-18\\
438.01	1.73472347597681e-18\\
439.01	0\\
440.01	1.73472347597681e-18\\
441.01	1.73472347597681e-18\\
442.01	0\\
443.01	0\\
444.01	0\\
445.01	1.73472347597681e-18\\
446.01	1.73472347597681e-18\\
447.01	0\\
448.01	1.73472347597681e-18\\
449.01	1.73472347597681e-18\\
450.01	0\\
451.01	0\\
452.01	0\\
453.01	1.73472347597681e-18\\
454.01	1.73472347597681e-18\\
455.01	0\\
456.01	1.73472347597681e-18\\
457.01	1.73472347597681e-18\\
458.01	0\\
459.01	1.73472347597681e-18\\
460.01	0\\
461.01	1.73472347597681e-18\\
462.01	0\\
463.01	0\\
464.01	0\\
465.01	1.73472347597681e-18\\
466.01	0\\
467.01	1.73472347597681e-18\\
468.01	0\\
469.01	1.73472347597681e-18\\
470.01	1.73472347597681e-18\\
471.01	1.73472347597681e-18\\
472.01	0\\
473.01	0\\
474.01	0\\
475.01	1.73472347597681e-18\\
476.01	0\\
477.01	0\\
478.01	1.73472347597681e-18\\
479.01	1.73472347597681e-18\\
480.01	0\\
481.01	0\\
482.01	1.73472347597681e-18\\
483.01	1.73472347597681e-18\\
484.01	0\\
485.01	0\\
486.01	0\\
487.01	1.73472347597681e-18\\
488.01	1.73472347597681e-18\\
489.01	1.73472347597681e-18\\
490.01	1.73472347597681e-18\\
491.01	0\\
492.01	1.73472347597681e-18\\
493.01	1.73472347597681e-18\\
494.01	0\\
495.01	1.73472347597681e-18\\
496.01	0\\
497.01	1.73472347597681e-18\\
498.01	1.73472347597681e-18\\
499.01	0\\
500.01	1.73472347597681e-18\\
501.01	1.73472347597681e-18\\
502.01	0\\
503.01	1.73472347597681e-18\\
504.01	1.73472347597681e-18\\
505.01	1.73472347597681e-18\\
506.01	0\\
507.01	0\\
508.01	0\\
509.01	0\\
510.01	1.73472347597681e-18\\
511.01	1.73472347597681e-18\\
512.01	0\\
513.01	1.73472347597681e-18\\
514.01	1.73472347597681e-18\\
515.01	0\\
516.01	0\\
517.01	1.73472347597681e-18\\
518.01	1.73472347597681e-18\\
519.01	1.73472347597681e-18\\
520.01	0\\
521.01	0\\
522.01	0\\
523.01	1.73472347597681e-18\\
524.01	0\\
525.01	1.73472347597681e-18\\
526.01	1.73472347597681e-18\\
527.01	0\\
528.01	0\\
529.01	0\\
530.01	1.73472347597681e-18\\
531.01	0\\
532.01	1.73472347597681e-18\\
533.01	0\\
534.01	1.73472347597681e-18\\
535.01	0\\
536.01	1.73472347597681e-18\\
537.01	1.73472347597681e-18\\
538.01	1.73472347597681e-18\\
539.01	0\\
540.01	0\\
541.01	1.73472347597681e-18\\
542.01	1.73472347597681e-18\\
543.01	0\\
544.01	1.73472347597681e-18\\
545.01	0\\
546.01	1.73472347597681e-18\\
547.01	0\\
548.01	0\\
549.01	0\\
550.01	0\\
551.01	0\\
552.01	0\\
553.01	0\\
554.01	0\\
555.01	1.73472347597681e-18\\
556.01	0\\
557.01	0\\
558.01	0\\
559.01	1.73472347597681e-18\\
560.01	1.73472347597681e-18\\
561.01	1.73472347597681e-18\\
562.01	1.73472347597681e-18\\
563.01	0\\
564.01	0\\
565.01	1.73472347597681e-18\\
566.01	0\\
567.01	0\\
568.01	0\\
569.01	0\\
570.01	0\\
571.01	0\\
572.01	1.73472347597681e-18\\
573.01	0\\
574.01	1.73472347597681e-18\\
575.01	0\\
576.01	1.73472347597681e-18\\
577.01	0\\
578.01	1.73472347597681e-18\\
579.01	0\\
580.01	0\\
581.01	1.73472347597681e-18\\
582.01	0\\
583.01	0\\
584.01	0\\
585.01	0\\
586.01	0\\
587.01	0\\
588.01	0\\
589.01	0\\
590.01	0\\
591.01	0\\
592.01	1.73472347597681e-18\\
593.01	0\\
594.01	0\\
595.01	0\\
596.01	0\\
597.01	0\\
598.01	0\\
599.01	0\\
599.02	0\\
599.03	1.73472347597681e-18\\
599.04	1.73472347597681e-18\\
599.05	0\\
599.06	0\\
599.07	1.73472347597681e-18\\
599.08	0\\
599.09	0\\
599.1	0\\
599.11	1.73472347597681e-18\\
599.12	1.73472347597681e-18\\
599.13	0\\
599.14	0\\
599.15	1.73472347597681e-18\\
599.16	1.73472347597681e-18\\
599.17	1.73472347597681e-18\\
599.18	0\\
599.19	0\\
599.2	0\\
599.21	0\\
599.22	0\\
599.23	1.73472347597681e-18\\
599.24	1.73472347597681e-18\\
599.25	1.73472347597681e-18\\
599.26	1.73472347597681e-18\\
599.27	0\\
599.28	0\\
599.29	0\\
599.3	0\\
599.31	0\\
599.32	1.73472347597681e-18\\
599.33	1.73472347597681e-18\\
599.34	0\\
599.35	0\\
599.36	0\\
599.37	1.73472347597681e-18\\
599.38	0\\
599.39	0\\
599.4	0\\
599.41	0\\
599.42	0\\
599.43	0\\
599.44	0\\
599.45	0\\
599.46	1.73472347597681e-18\\
599.47	1.73472347597681e-18\\
599.48	1.73472347597681e-18\\
599.49	1.73472347597681e-18\\
599.5	1.73472347597681e-18\\
599.51	0\\
599.52	0\\
599.53	1.73472347597681e-18\\
599.54	0\\
599.55	0\\
599.56	0\\
599.57	1.73472347597681e-18\\
599.58	0\\
599.59	0\\
599.6	0\\
599.61	0\\
599.62	1.73472347597681e-18\\
599.63	0\\
599.64	1.73472347597681e-18\\
599.65	0\\
599.66	0\\
599.67	0\\
599.68	0\\
599.69	1.73472347597681e-18\\
599.7	0\\
599.71	1.73472347597681e-18\\
599.72	1.73472347597681e-18\\
599.73	1.73472347597681e-18\\
599.74	0\\
599.75	0\\
599.76	0\\
599.77	0\\
599.78	0\\
599.79	0\\
599.8	0\\
599.81	0\\
599.82	0\\
599.83	1.73472347597681e-18\\
599.84	0\\
599.85	0\\
599.86	0\\
599.87	0\\
599.88	0\\
599.89	0\\
599.9	0\\
599.91	0\\
599.92	0\\
599.93	0\\
599.94	0\\
599.95	0\\
599.96	0\\
599.97	0\\
599.98	0\\
599.99	0\\
600	0\\
};
\addplot [color=mycolor7,solid,forget plot]
  table[row sep=crcr]{%
0.01	1.73472347597681e-18\\
1.01	1.73472347597681e-18\\
2.01	0\\
3.01	1.73472347597681e-18\\
4.01	1.73472347597681e-18\\
5.01	0\\
6.01	0\\
7.01	1.73472347597681e-18\\
8.01	0\\
9.01	0\\
10.01	0\\
11.01	0\\
12.01	1.73472347597681e-18\\
13.01	1.73472347597681e-18\\
14.01	1.73472347597681e-18\\
15.01	1.73472347597681e-18\\
16.01	1.73472347597681e-18\\
17.01	1.73472347597681e-18\\
18.01	0\\
19.01	0\\
20.01	1.73472347597681e-18\\
21.01	0\\
22.01	1.73472347597681e-18\\
23.01	1.73472347597681e-18\\
24.01	1.73472347597681e-18\\
25.01	0\\
26.01	1.73472347597681e-18\\
27.01	0\\
28.01	1.73472347597681e-18\\
29.01	1.73472347597681e-18\\
30.01	0\\
31.01	0\\
32.01	1.73472347597681e-18\\
33.01	1.73472347597681e-18\\
34.01	1.73472347597681e-18\\
35.01	0\\
36.01	0\\
37.01	0\\
38.01	0\\
39.01	1.73472347597681e-18\\
40.01	1.73472347597681e-18\\
41.01	1.73472347597681e-18\\
42.01	1.73472347597681e-18\\
43.01	1.73472347597681e-18\\
44.01	0\\
45.01	1.73472347597681e-18\\
46.01	1.73472347597681e-18\\
47.01	1.73472347597681e-18\\
48.01	1.73472347597681e-18\\
49.01	0\\
50.01	1.73472347597681e-18\\
51.01	1.73472347597681e-18\\
52.01	0\\
53.01	0\\
54.01	0\\
55.01	1.73472347597681e-18\\
56.01	0\\
57.01	1.73472347597681e-18\\
58.01	0\\
59.01	0\\
60.01	1.73472347597681e-18\\
61.01	1.73472347597681e-18\\
62.01	1.73472347597681e-18\\
63.01	0\\
64.01	1.73472347597681e-18\\
65.01	0\\
66.01	1.73472347597681e-18\\
67.01	1.73472347597681e-18\\
68.01	0\\
69.01	1.73472347597681e-18\\
70.01	1.73472347597681e-18\\
71.01	0\\
72.01	0\\
73.01	1.73472347597681e-18\\
74.01	0\\
75.01	1.73472347597681e-18\\
76.01	0\\
77.01	0\\
78.01	0\\
79.01	0\\
80.01	1.73472347597681e-18\\
81.01	1.73472347597681e-18\\
82.01	0\\
83.01	0\\
84.01	0\\
85.01	1.73472347597681e-18\\
86.01	1.73472347597681e-18\\
87.01	0\\
88.01	0\\
89.01	0\\
90.01	1.73472347597681e-18\\
91.01	1.73472347597681e-18\\
92.01	0\\
93.01	1.73472347597681e-18\\
94.01	1.73472347597681e-18\\
95.01	0\\
96.01	1.73472347597681e-18\\
97.01	0\\
98.01	0\\
99.01	0\\
100.01	1.73472347597681e-18\\
101.01	0\\
102.01	1.73472347597681e-18\\
103.01	1.73472347597681e-18\\
104.01	0\\
105.01	0\\
106.01	0\\
107.01	0\\
108.01	1.73472347597681e-18\\
109.01	0\\
110.01	1.73472347597681e-18\\
111.01	0\\
112.01	0\\
113.01	0\\
114.01	1.73472347597681e-18\\
115.01	1.73472347597681e-18\\
116.01	0\\
117.01	0\\
118.01	0\\
119.01	0\\
120.01	0\\
121.01	0\\
122.01	0\\
123.01	1.73472347597681e-18\\
124.01	1.73472347597681e-18\\
125.01	1.73472347597681e-18\\
126.01	1.73472347597681e-18\\
127.01	0\\
128.01	1.73472347597681e-18\\
129.01	0\\
130.01	1.73472347597681e-18\\
131.01	1.73472347597681e-18\\
132.01	0\\
133.01	1.73472347597681e-18\\
134.01	1.73472347597681e-18\\
135.01	0\\
136.01	0\\
137.01	1.73472347597681e-18\\
138.01	0\\
139.01	1.73472347597681e-18\\
140.01	1.73472347597681e-18\\
141.01	1.73472347597681e-18\\
142.01	1.73472347597681e-18\\
143.01	0\\
144.01	1.73472347597681e-18\\
145.01	1.73472347597681e-18\\
146.01	1.73472347597681e-18\\
147.01	1.73472347597681e-18\\
148.01	1.73472347597681e-18\\
149.01	0\\
150.01	0\\
151.01	1.73472347597681e-18\\
152.01	1.73472347597681e-18\\
153.01	1.73472347597681e-18\\
154.01	1.73472347597681e-18\\
155.01	1.73472347597681e-18\\
156.01	1.73472347597681e-18\\
157.01	0\\
158.01	1.73472347597681e-18\\
159.01	1.73472347597681e-18\\
160.01	1.73472347597681e-18\\
161.01	1.73472347597681e-18\\
162.01	1.73472347597681e-18\\
163.01	1.73472347597681e-18\\
164.01	0\\
165.01	0\\
166.01	1.73472347597681e-18\\
167.01	0\\
168.01	1.73472347597681e-18\\
169.01	0\\
170.01	1.73472347597681e-18\\
171.01	0\\
172.01	0\\
173.01	0\\
174.01	1.73472347597681e-18\\
175.01	1.73472347597681e-18\\
176.01	1.73472347597681e-18\\
177.01	0\\
178.01	1.73472347597681e-18\\
179.01	0\\
180.01	0\\
181.01	1.73472347597681e-18\\
182.01	0\\
183.01	0\\
184.01	0\\
185.01	1.73472347597681e-18\\
186.01	0\\
187.01	0\\
188.01	0\\
189.01	0\\
190.01	1.73472347597681e-18\\
191.01	0\\
192.01	1.73472347597681e-18\\
193.01	0\\
194.01	1.73472347597681e-18\\
195.01	0\\
196.01	1.73472347597681e-18\\
197.01	0\\
198.01	0\\
199.01	0\\
200.01	0\\
201.01	0\\
202.01	1.73472347597681e-18\\
203.01	0\\
204.01	1.73472347597681e-18\\
205.01	1.73472347597681e-18\\
206.01	1.73472347597681e-18\\
207.01	1.73472347597681e-18\\
208.01	0\\
209.01	1.73472347597681e-18\\
210.01	0\\
211.01	1.73472347597681e-18\\
212.01	0\\
213.01	1.73472347597681e-18\\
214.01	0\\
215.01	0\\
216.01	0\\
217.01	0\\
218.01	1.73472347597681e-18\\
219.01	1.73472347597681e-18\\
220.01	0\\
221.01	0\\
222.01	0\\
223.01	1.73472347597681e-18\\
224.01	1.73472347597681e-18\\
225.01	0\\
226.01	1.73472347597681e-18\\
227.01	0\\
228.01	0\\
229.01	1.73472347597681e-18\\
230.01	0\\
231.01	0\\
232.01	0\\
233.01	0\\
234.01	1.73472347597681e-18\\
235.01	0\\
236.01	1.73472347597681e-18\\
237.01	1.73472347597681e-18\\
238.01	0\\
239.01	1.73472347597681e-18\\
240.01	0\\
241.01	1.73472347597681e-18\\
242.01	0\\
243.01	1.73472347597681e-18\\
244.01	1.73472347597681e-18\\
245.01	1.73472347597681e-18\\
246.01	1.73472347597681e-18\\
247.01	0\\
248.01	0\\
249.01	0\\
250.01	1.73472347597681e-18\\
251.01	0\\
252.01	1.73472347597681e-18\\
253.01	1.73472347597681e-18\\
254.01	1.73472347597681e-18\\
255.01	1.73472347597681e-18\\
256.01	1.73472347597681e-18\\
257.01	1.73472347597681e-18\\
258.01	0\\
259.01	0\\
260.01	0\\
261.01	0\\
262.01	1.73472347597681e-18\\
263.01	1.73472347597681e-18\\
264.01	0\\
265.01	1.73472347597681e-18\\
266.01	1.73472347597681e-18\\
267.01	0\\
268.01	0\\
269.01	1.73472347597681e-18\\
270.01	0\\
271.01	1.73472347597681e-18\\
272.01	0\\
273.01	1.73472347597681e-18\\
274.01	0\\
275.01	0\\
276.01	1.73472347597681e-18\\
277.01	1.73472347597681e-18\\
278.01	0\\
279.01	1.73472347597681e-18\\
280.01	0\\
281.01	1.73472347597681e-18\\
282.01	1.73472347597681e-18\\
283.01	1.73472347597681e-18\\
284.01	0\\
285.01	1.73472347597681e-18\\
286.01	1.73472347597681e-18\\
287.01	1.73472347597681e-18\\
288.01	1.73472347597681e-18\\
289.01	0\\
290.01	0\\
291.01	0\\
292.01	0\\
293.01	1.73472347597681e-18\\
294.01	1.73472347597681e-18\\
295.01	0\\
296.01	1.73472347597681e-18\\
297.01	0\\
298.01	1.73472347597681e-18\\
299.01	0\\
300.01	1.73472347597681e-18\\
301.01	0\\
302.01	0\\
303.01	0\\
304.01	1.73472347597681e-18\\
305.01	1.73472347597681e-18\\
306.01	0\\
307.01	1.73472347597681e-18\\
308.01	0\\
309.01	1.73472347597681e-18\\
310.01	1.73472347597681e-18\\
311.01	0\\
312.01	1.73472347597681e-18\\
313.01	1.73472347597681e-18\\
314.01	0\\
315.01	1.73472347597681e-18\\
316.01	0\\
317.01	1.73472347597681e-18\\
318.01	0\\
319.01	0\\
320.01	0\\
321.01	0\\
322.01	1.73472347597681e-18\\
323.01	1.73472347597681e-18\\
324.01	0\\
325.01	1.73472347597681e-18\\
326.01	1.73472347597681e-18\\
327.01	0\\
328.01	1.73472347597681e-18\\
329.01	0\\
330.01	0\\
331.01	1.73472347597681e-18\\
332.01	1.73472347597681e-18\\
333.01	1.73472347597681e-18\\
334.01	1.73472347597681e-18\\
335.01	1.73472347597681e-18\\
336.01	0\\
337.01	1.73472347597681e-18\\
338.01	0\\
339.01	0\\
340.01	1.73472347597681e-18\\
341.01	1.73472347597681e-18\\
342.01	1.73472347597681e-18\\
343.01	1.73472347597681e-18\\
344.01	0\\
345.01	0\\
346.01	0\\
347.01	1.73472347597681e-18\\
348.01	0\\
349.01	1.73472347597681e-18\\
350.01	0\\
351.01	0\\
352.01	0\\
353.01	1.73472347597681e-18\\
354.01	1.73472347597681e-18\\
355.01	0\\
356.01	0\\
357.01	1.73472347597681e-18\\
358.01	1.73472347597681e-18\\
359.01	1.73472347597681e-18\\
360.01	0\\
361.01	0\\
362.01	0\\
363.01	0\\
364.01	0\\
365.01	1.73472347597681e-18\\
366.01	1.73472347597681e-18\\
367.01	1.73472347597681e-18\\
368.01	1.73472347597681e-18\\
369.01	0\\
370.01	0\\
371.01	1.73472347597681e-18\\
372.01	1.73472347597681e-18\\
373.01	1.73472347597681e-18\\
374.01	0\\
375.01	1.73472347597681e-18\\
376.01	0\\
377.01	0\\
378.01	1.73472347597681e-18\\
379.01	0\\
380.01	1.73472347597681e-18\\
381.01	1.73472347597681e-18\\
382.01	1.73472347597681e-18\\
383.01	0\\
384.01	1.73472347597681e-18\\
385.01	0\\
386.01	1.73472347597681e-18\\
387.01	1.73472347597681e-18\\
388.01	1.73472347597681e-18\\
389.01	0\\
390.01	1.73472347597681e-18\\
391.01	1.73472347597681e-18\\
392.01	0\\
393.01	0\\
394.01	1.73472347597681e-18\\
395.01	1.73472347597681e-18\\
396.01	1.73472347597681e-18\\
397.01	1.73472347597681e-18\\
398.01	0\\
399.01	1.73472347597681e-18\\
400.01	0\\
401.01	0\\
402.01	0\\
403.01	1.73472347597681e-18\\
404.01	1.73472347597681e-18\\
405.01	1.73472347597681e-18\\
406.01	1.73472347597681e-18\\
407.01	1.73472347597681e-18\\
408.01	1.73472347597681e-18\\
409.01	0\\
410.01	1.73472347597681e-18\\
411.01	0\\
412.01	1.73472347597681e-18\\
413.01	1.73472347597681e-18\\
414.01	0\\
415.01	0\\
416.01	1.73472347597681e-18\\
417.01	1.73472347597681e-18\\
418.01	1.73472347597681e-18\\
419.01	1.73472347597681e-18\\
420.01	0\\
421.01	1.73472347597681e-18\\
422.01	1.73472347597681e-18\\
423.01	1.73472347597681e-18\\
424.01	1.73472347597681e-18\\
425.01	0\\
426.01	1.73472347597681e-18\\
427.01	1.73472347597681e-18\\
428.01	1.73472347597681e-18\\
429.01	0\\
430.01	1.73472347597681e-18\\
431.01	1.73472347597681e-18\\
432.01	1.73472347597681e-18\\
433.01	0\\
434.01	1.73472347597681e-18\\
435.01	1.73472347597681e-18\\
436.01	1.73472347597681e-18\\
437.01	1.73472347597681e-18\\
438.01	1.73472347597681e-18\\
439.01	0\\
440.01	1.73472347597681e-18\\
441.01	1.73472347597681e-18\\
442.01	0\\
443.01	0\\
444.01	0\\
445.01	1.73472347597681e-18\\
446.01	1.73472347597681e-18\\
447.01	0\\
448.01	1.73472347597681e-18\\
449.01	1.73472347597681e-18\\
450.01	0\\
451.01	0\\
452.01	0\\
453.01	1.73472347597681e-18\\
454.01	1.73472347597681e-18\\
455.01	0\\
456.01	1.73472347597681e-18\\
457.01	1.73472347597681e-18\\
458.01	0\\
459.01	1.73472347597681e-18\\
460.01	0\\
461.01	1.73472347597681e-18\\
462.01	0\\
463.01	0\\
464.01	0\\
465.01	1.73472347597681e-18\\
466.01	0\\
467.01	1.73472347597681e-18\\
468.01	0\\
469.01	1.73472347597681e-18\\
470.01	1.73472347597681e-18\\
471.01	1.73472347597681e-18\\
472.01	0\\
473.01	0\\
474.01	0\\
475.01	1.73472347597681e-18\\
476.01	0\\
477.01	0\\
478.01	1.73472347597681e-18\\
479.01	1.73472347597681e-18\\
480.01	0\\
481.01	0\\
482.01	1.73472347597681e-18\\
483.01	1.73472347597681e-18\\
484.01	0\\
485.01	0\\
486.01	0\\
487.01	1.73472347597681e-18\\
488.01	1.73472347597681e-18\\
489.01	1.73472347597681e-18\\
490.01	1.73472347597681e-18\\
491.01	0\\
492.01	1.73472347597681e-18\\
493.01	1.73472347597681e-18\\
494.01	0\\
495.01	1.73472347597681e-18\\
496.01	0\\
497.01	1.73472347597681e-18\\
498.01	1.73472347597681e-18\\
499.01	0\\
500.01	1.73472347597681e-18\\
501.01	1.73472347597681e-18\\
502.01	0\\
503.01	1.73472347597681e-18\\
504.01	1.73472347597681e-18\\
505.01	1.73472347597681e-18\\
506.01	0\\
507.01	0\\
508.01	0\\
509.01	0\\
510.01	1.73472347597681e-18\\
511.01	1.73472347597681e-18\\
512.01	0\\
513.01	1.73472347597681e-18\\
514.01	1.73472347597681e-18\\
515.01	0\\
516.01	0\\
517.01	1.73472347597681e-18\\
518.01	1.73472347597681e-18\\
519.01	1.73472347597681e-18\\
520.01	0\\
521.01	0\\
522.01	0\\
523.01	1.73472347597681e-18\\
524.01	0\\
525.01	1.73472347597681e-18\\
526.01	1.73472347597681e-18\\
527.01	0\\
528.01	0\\
529.01	0\\
530.01	1.73472347597681e-18\\
531.01	0\\
532.01	1.73472347597681e-18\\
533.01	0\\
534.01	1.73472347597681e-18\\
535.01	0\\
536.01	1.73472347597681e-18\\
537.01	1.73472347597681e-18\\
538.01	1.73472347597681e-18\\
539.01	0\\
540.01	0\\
541.01	1.73472347597681e-18\\
542.01	1.73472347597681e-18\\
543.01	0\\
544.01	1.73472347597681e-18\\
545.01	0\\
546.01	1.73472347597681e-18\\
547.01	0\\
548.01	0\\
549.01	0\\
550.01	0\\
551.01	0\\
552.01	0\\
553.01	0\\
554.01	0\\
555.01	1.73472347597681e-18\\
556.01	0\\
557.01	0\\
558.01	0\\
559.01	1.73472347597681e-18\\
560.01	1.73472347597681e-18\\
561.01	1.73472347597681e-18\\
562.01	1.73472347597681e-18\\
563.01	0\\
564.01	0\\
565.01	1.73472347597681e-18\\
566.01	0\\
567.01	0\\
568.01	0\\
569.01	0\\
570.01	0\\
571.01	0\\
572.01	1.73472347597681e-18\\
573.01	0\\
574.01	1.73472347597681e-18\\
575.01	0\\
576.01	1.73472347597681e-18\\
577.01	0\\
578.01	1.73472347597681e-18\\
579.01	0\\
580.01	0\\
581.01	1.73472347597681e-18\\
582.01	0\\
583.01	0\\
584.01	0\\
585.01	0\\
586.01	0\\
587.01	0\\
588.01	0\\
589.01	0\\
590.01	0\\
591.01	0\\
592.01	1.73472347597681e-18\\
593.01	0\\
594.01	0\\
595.01	0\\
596.01	0\\
597.01	0\\
598.01	0\\
599.01	0\\
599.02	0\\
599.03	1.73472347597681e-18\\
599.04	1.73472347597681e-18\\
599.05	0\\
599.06	0\\
599.07	1.73472347597681e-18\\
599.08	0\\
599.09	0\\
599.1	0\\
599.11	1.73472347597681e-18\\
599.12	1.73472347597681e-18\\
599.13	0\\
599.14	0\\
599.15	1.73472347597681e-18\\
599.16	1.73472347597681e-18\\
599.17	1.73472347597681e-18\\
599.18	0\\
599.19	0\\
599.2	0\\
599.21	0\\
599.22	0\\
599.23	1.73472347597681e-18\\
599.24	1.73472347597681e-18\\
599.25	1.73472347597681e-18\\
599.26	1.73472347597681e-18\\
599.27	0\\
599.28	0\\
599.29	0\\
599.3	0\\
599.31	0\\
599.32	1.73472347597681e-18\\
599.33	1.73472347597681e-18\\
599.34	0\\
599.35	0\\
599.36	0\\
599.37	1.73472347597681e-18\\
599.38	0\\
599.39	0\\
599.4	0\\
599.41	0\\
599.42	0\\
599.43	0\\
599.44	0\\
599.45	0\\
599.46	1.73472347597681e-18\\
599.47	1.73472347597681e-18\\
599.48	1.73472347597681e-18\\
599.49	1.73472347597681e-18\\
599.5	1.73472347597681e-18\\
599.51	0\\
599.52	0\\
599.53	1.73472347597681e-18\\
599.54	0\\
599.55	0\\
599.56	0\\
599.57	1.73472347597681e-18\\
599.58	0\\
599.59	0\\
599.6	0\\
599.61	0\\
599.62	1.73472347597681e-18\\
599.63	0\\
599.64	1.73472347597681e-18\\
599.65	0\\
599.66	0\\
599.67	0\\
599.68	0\\
599.69	1.73472347597681e-18\\
599.7	0\\
599.71	1.73472347597681e-18\\
599.72	1.73472347597681e-18\\
599.73	1.73472347597681e-18\\
599.74	0\\
599.75	0\\
599.76	0\\
599.77	0\\
599.78	0\\
599.79	0\\
599.8	0\\
599.81	0\\
599.82	0\\
599.83	1.73472347597681e-18\\
599.84	0\\
599.85	0\\
599.86	0\\
599.87	0\\
599.88	0\\
599.89	0\\
599.9	0\\
599.91	0\\
599.92	0\\
599.93	0\\
599.94	0\\
599.95	0\\
599.96	0\\
599.97	0\\
599.98	0\\
599.99	0\\
600	0\\
};
\addplot [color=mycolor8,solid,forget plot]
  table[row sep=crcr]{%
0.01	1.73472347597681e-18\\
1.01	1.73472347597681e-18\\
2.01	0\\
3.01	1.73472347597681e-18\\
4.01	1.73472347597681e-18\\
5.01	0\\
6.01	0\\
7.01	1.73472347597681e-18\\
8.01	0\\
9.01	0\\
10.01	0\\
11.01	0\\
12.01	1.73472347597681e-18\\
13.01	1.73472347597681e-18\\
14.01	1.73472347597681e-18\\
15.01	1.73472347597681e-18\\
16.01	1.73472347597681e-18\\
17.01	1.73472347597681e-18\\
18.01	0\\
19.01	0\\
20.01	1.73472347597681e-18\\
21.01	0\\
22.01	1.73472347597681e-18\\
23.01	1.73472347597681e-18\\
24.01	1.73472347597681e-18\\
25.01	0\\
26.01	1.73472347597681e-18\\
27.01	0\\
28.01	1.73472347597681e-18\\
29.01	1.73472347597681e-18\\
30.01	0\\
31.01	0\\
32.01	1.73472347597681e-18\\
33.01	1.73472347597681e-18\\
34.01	1.73472347597681e-18\\
35.01	0\\
36.01	0\\
37.01	0\\
38.01	0\\
39.01	1.73472347597681e-18\\
40.01	1.73472347597681e-18\\
41.01	1.73472347597681e-18\\
42.01	1.73472347597681e-18\\
43.01	1.73472347597681e-18\\
44.01	0\\
45.01	1.73472347597681e-18\\
46.01	1.73472347597681e-18\\
47.01	1.73472347597681e-18\\
48.01	1.73472347597681e-18\\
49.01	0\\
50.01	1.73472347597681e-18\\
51.01	1.73472347597681e-18\\
52.01	0\\
53.01	0\\
54.01	0\\
55.01	1.73472347597681e-18\\
56.01	0\\
57.01	1.73472347597681e-18\\
58.01	0\\
59.01	0\\
60.01	1.73472347597681e-18\\
61.01	1.73472347597681e-18\\
62.01	1.73472347597681e-18\\
63.01	0\\
64.01	1.73472347597681e-18\\
65.01	0\\
66.01	1.73472347597681e-18\\
67.01	1.73472347597681e-18\\
68.01	0\\
69.01	1.73472347597681e-18\\
70.01	1.73472347597681e-18\\
71.01	0\\
72.01	0\\
73.01	1.73472347597681e-18\\
74.01	0\\
75.01	1.73472347597681e-18\\
76.01	0\\
77.01	0\\
78.01	0\\
79.01	0\\
80.01	1.73472347597681e-18\\
81.01	1.73472347597681e-18\\
82.01	0\\
83.01	0\\
84.01	0\\
85.01	1.73472347597681e-18\\
86.01	1.73472347597681e-18\\
87.01	0\\
88.01	0\\
89.01	0\\
90.01	1.73472347597681e-18\\
91.01	1.73472347597681e-18\\
92.01	0\\
93.01	1.73472347597681e-18\\
94.01	1.73472347597681e-18\\
95.01	0\\
96.01	1.73472347597681e-18\\
97.01	0\\
98.01	0\\
99.01	0\\
100.01	1.73472347597681e-18\\
101.01	0\\
102.01	1.73472347597681e-18\\
103.01	1.73472347597681e-18\\
104.01	0\\
105.01	0\\
106.01	0\\
107.01	0\\
108.01	1.73472347597681e-18\\
109.01	0\\
110.01	1.73472347597681e-18\\
111.01	0\\
112.01	0\\
113.01	0\\
114.01	1.73472347597681e-18\\
115.01	1.73472347597681e-18\\
116.01	0\\
117.01	0\\
118.01	0\\
119.01	0\\
120.01	0\\
121.01	0\\
122.01	0\\
123.01	1.73472347597681e-18\\
124.01	1.73472347597681e-18\\
125.01	1.73472347597681e-18\\
126.01	1.73472347597681e-18\\
127.01	0\\
128.01	1.73472347597681e-18\\
129.01	0\\
130.01	1.73472347597681e-18\\
131.01	1.73472347597681e-18\\
132.01	0\\
133.01	1.73472347597681e-18\\
134.01	1.73472347597681e-18\\
135.01	0\\
136.01	0\\
137.01	1.73472347597681e-18\\
138.01	0\\
139.01	1.73472347597681e-18\\
140.01	1.73472347597681e-18\\
141.01	1.73472347597681e-18\\
142.01	1.73472347597681e-18\\
143.01	0\\
144.01	1.73472347597681e-18\\
145.01	1.73472347597681e-18\\
146.01	1.73472347597681e-18\\
147.01	1.73472347597681e-18\\
148.01	1.73472347597681e-18\\
149.01	0\\
150.01	0\\
151.01	1.73472347597681e-18\\
152.01	1.73472347597681e-18\\
153.01	1.73472347597681e-18\\
154.01	1.73472347597681e-18\\
155.01	1.73472347597681e-18\\
156.01	1.73472347597681e-18\\
157.01	0\\
158.01	1.73472347597681e-18\\
159.01	1.73472347597681e-18\\
160.01	1.73472347597681e-18\\
161.01	1.73472347597681e-18\\
162.01	1.73472347597681e-18\\
163.01	1.73472347597681e-18\\
164.01	0\\
165.01	0\\
166.01	1.73472347597681e-18\\
167.01	0\\
168.01	1.73472347597681e-18\\
169.01	0\\
170.01	1.73472347597681e-18\\
171.01	0\\
172.01	0\\
173.01	0\\
174.01	1.73472347597681e-18\\
175.01	1.73472347597681e-18\\
176.01	1.73472347597681e-18\\
177.01	0\\
178.01	1.73472347597681e-18\\
179.01	0\\
180.01	0\\
181.01	1.73472347597681e-18\\
182.01	0\\
183.01	0\\
184.01	0\\
185.01	1.73472347597681e-18\\
186.01	0\\
187.01	0\\
188.01	0\\
189.01	0\\
190.01	1.73472347597681e-18\\
191.01	0\\
192.01	1.73472347597681e-18\\
193.01	0\\
194.01	1.73472347597681e-18\\
195.01	0\\
196.01	1.73472347597681e-18\\
197.01	0\\
198.01	0\\
199.01	0\\
200.01	0\\
201.01	0\\
202.01	1.73472347597681e-18\\
203.01	0\\
204.01	1.73472347597681e-18\\
205.01	1.73472347597681e-18\\
206.01	1.73472347597681e-18\\
207.01	1.73472347597681e-18\\
208.01	0\\
209.01	1.73472347597681e-18\\
210.01	0\\
211.01	1.73472347597681e-18\\
212.01	0\\
213.01	1.73472347597681e-18\\
214.01	0\\
215.01	0\\
216.01	0\\
217.01	0\\
218.01	1.73472347597681e-18\\
219.01	1.73472347597681e-18\\
220.01	0\\
221.01	0\\
222.01	0\\
223.01	1.73472347597681e-18\\
224.01	1.73472347597681e-18\\
225.01	0\\
226.01	1.73472347597681e-18\\
227.01	0\\
228.01	0\\
229.01	1.73472347597681e-18\\
230.01	0\\
231.01	0\\
232.01	0\\
233.01	0\\
234.01	1.73472347597681e-18\\
235.01	0\\
236.01	1.73472347597681e-18\\
237.01	1.73472347597681e-18\\
238.01	0\\
239.01	1.73472347597681e-18\\
240.01	0\\
241.01	1.73472347597681e-18\\
242.01	0\\
243.01	1.73472347597681e-18\\
244.01	1.73472347597681e-18\\
245.01	1.73472347597681e-18\\
246.01	1.73472347597681e-18\\
247.01	0\\
248.01	0\\
249.01	0\\
250.01	1.73472347597681e-18\\
251.01	0\\
252.01	1.73472347597681e-18\\
253.01	1.73472347597681e-18\\
254.01	1.73472347597681e-18\\
255.01	1.73472347597681e-18\\
256.01	1.73472347597681e-18\\
257.01	1.73472347597681e-18\\
258.01	0\\
259.01	0\\
260.01	0\\
261.01	0\\
262.01	1.73472347597681e-18\\
263.01	1.73472347597681e-18\\
264.01	0\\
265.01	1.73472347597681e-18\\
266.01	1.73472347597681e-18\\
267.01	0\\
268.01	0\\
269.01	1.73472347597681e-18\\
270.01	0\\
271.01	1.73472347597681e-18\\
272.01	0\\
273.01	1.73472347597681e-18\\
274.01	0\\
275.01	0\\
276.01	1.73472347597681e-18\\
277.01	1.73472347597681e-18\\
278.01	0\\
279.01	1.73472347597681e-18\\
280.01	0\\
281.01	1.73472347597681e-18\\
282.01	1.73472347597681e-18\\
283.01	1.73472347597681e-18\\
284.01	0\\
285.01	1.73472347597681e-18\\
286.01	1.73472347597681e-18\\
287.01	1.73472347597681e-18\\
288.01	1.73472347597681e-18\\
289.01	0\\
290.01	0\\
291.01	0\\
292.01	0\\
293.01	1.73472347597681e-18\\
294.01	1.73472347597681e-18\\
295.01	0\\
296.01	1.73472347597681e-18\\
297.01	0\\
298.01	1.73472347597681e-18\\
299.01	0\\
300.01	1.73472347597681e-18\\
301.01	0\\
302.01	0\\
303.01	0\\
304.01	1.73472347597681e-18\\
305.01	1.73472347597681e-18\\
306.01	0\\
307.01	1.73472347597681e-18\\
308.01	0\\
309.01	1.73472347597681e-18\\
310.01	1.73472347597681e-18\\
311.01	0\\
312.01	1.73472347597681e-18\\
313.01	1.73472347597681e-18\\
314.01	0\\
315.01	1.73472347597681e-18\\
316.01	0\\
317.01	1.73472347597681e-18\\
318.01	0\\
319.01	0\\
320.01	0\\
321.01	0\\
322.01	1.73472347597681e-18\\
323.01	1.73472347597681e-18\\
324.01	0\\
325.01	1.73472347597681e-18\\
326.01	1.73472347597681e-18\\
327.01	0\\
328.01	1.73472347597681e-18\\
329.01	0\\
330.01	0\\
331.01	1.73472347597681e-18\\
332.01	1.73472347597681e-18\\
333.01	1.73472347597681e-18\\
334.01	1.73472347597681e-18\\
335.01	1.73472347597681e-18\\
336.01	0\\
337.01	1.73472347597681e-18\\
338.01	0\\
339.01	0\\
340.01	1.73472347597681e-18\\
341.01	1.73472347597681e-18\\
342.01	1.73472347597681e-18\\
343.01	1.73472347597681e-18\\
344.01	0\\
345.01	0\\
346.01	0\\
347.01	1.73472347597681e-18\\
348.01	0\\
349.01	1.73472347597681e-18\\
350.01	0\\
351.01	0\\
352.01	0\\
353.01	1.73472347597681e-18\\
354.01	1.73472347597681e-18\\
355.01	0\\
356.01	0\\
357.01	1.73472347597681e-18\\
358.01	1.73472347597681e-18\\
359.01	1.73472347597681e-18\\
360.01	0\\
361.01	0\\
362.01	0\\
363.01	0\\
364.01	0\\
365.01	1.73472347597681e-18\\
366.01	1.73472347597681e-18\\
367.01	1.73472347597681e-18\\
368.01	1.73472347597681e-18\\
369.01	0\\
370.01	0\\
371.01	1.73472347597681e-18\\
372.01	1.73472347597681e-18\\
373.01	1.73472347597681e-18\\
374.01	0\\
375.01	1.73472347597681e-18\\
376.01	0\\
377.01	0\\
378.01	1.73472347597681e-18\\
379.01	0\\
380.01	1.73472347597681e-18\\
381.01	1.73472347597681e-18\\
382.01	1.73472347597681e-18\\
383.01	0\\
384.01	1.73472347597681e-18\\
385.01	0\\
386.01	1.73472347597681e-18\\
387.01	1.73472347597681e-18\\
388.01	1.73472347597681e-18\\
389.01	0\\
390.01	1.73472347597681e-18\\
391.01	1.73472347597681e-18\\
392.01	0\\
393.01	0\\
394.01	1.73472347597681e-18\\
395.01	1.73472347597681e-18\\
396.01	1.73472347597681e-18\\
397.01	1.73472347597681e-18\\
398.01	0\\
399.01	1.73472347597681e-18\\
400.01	0\\
401.01	0\\
402.01	0\\
403.01	1.73472347597681e-18\\
404.01	1.73472347597681e-18\\
405.01	1.73472347597681e-18\\
406.01	1.73472347597681e-18\\
407.01	1.73472347597681e-18\\
408.01	1.73472347597681e-18\\
409.01	0\\
410.01	1.73472347597681e-18\\
411.01	0\\
412.01	1.73472347597681e-18\\
413.01	1.73472347597681e-18\\
414.01	0\\
415.01	0\\
416.01	1.73472347597681e-18\\
417.01	1.73472347597681e-18\\
418.01	1.73472347597681e-18\\
419.01	1.73472347597681e-18\\
420.01	0\\
421.01	1.73472347597681e-18\\
422.01	1.73472347597681e-18\\
423.01	1.73472347597681e-18\\
424.01	1.73472347597681e-18\\
425.01	0\\
426.01	1.73472347597681e-18\\
427.01	1.73472347597681e-18\\
428.01	1.73472347597681e-18\\
429.01	0\\
430.01	1.73472347597681e-18\\
431.01	1.73472347597681e-18\\
432.01	1.73472347597681e-18\\
433.01	0\\
434.01	1.73472347597681e-18\\
435.01	1.73472347597681e-18\\
436.01	1.73472347597681e-18\\
437.01	1.73472347597681e-18\\
438.01	1.73472347597681e-18\\
439.01	0\\
440.01	1.73472347597681e-18\\
441.01	1.73472347597681e-18\\
442.01	0\\
443.01	0\\
444.01	0\\
445.01	1.73472347597681e-18\\
446.01	1.73472347597681e-18\\
447.01	0\\
448.01	1.73472347597681e-18\\
449.01	1.73472347597681e-18\\
450.01	0\\
451.01	0\\
452.01	0\\
453.01	1.73472347597681e-18\\
454.01	1.73472347597681e-18\\
455.01	0\\
456.01	1.73472347597681e-18\\
457.01	1.73472347597681e-18\\
458.01	0\\
459.01	1.73472347597681e-18\\
460.01	0\\
461.01	1.73472347597681e-18\\
462.01	0\\
463.01	0\\
464.01	0\\
465.01	1.73472347597681e-18\\
466.01	0\\
467.01	1.73472347597681e-18\\
468.01	0\\
469.01	1.73472347597681e-18\\
470.01	1.73472347597681e-18\\
471.01	1.73472347597681e-18\\
472.01	0\\
473.01	0\\
474.01	0\\
475.01	1.73472347597681e-18\\
476.01	0\\
477.01	0\\
478.01	1.73472347597681e-18\\
479.01	1.73472347597681e-18\\
480.01	0\\
481.01	0\\
482.01	1.73472347597681e-18\\
483.01	1.73472347597681e-18\\
484.01	0\\
485.01	0\\
486.01	0\\
487.01	1.73472347597681e-18\\
488.01	1.73472347597681e-18\\
489.01	1.73472347597681e-18\\
490.01	1.73472347597681e-18\\
491.01	0\\
492.01	1.73472347597681e-18\\
493.01	1.73472347597681e-18\\
494.01	0\\
495.01	1.73472347597681e-18\\
496.01	0\\
497.01	1.73472347597681e-18\\
498.01	1.73472347597681e-18\\
499.01	0\\
500.01	1.73472347597681e-18\\
501.01	1.73472347597681e-18\\
502.01	0\\
503.01	1.73472347597681e-18\\
504.01	1.73472347597681e-18\\
505.01	1.73472347597681e-18\\
506.01	0\\
507.01	0\\
508.01	0\\
509.01	0\\
510.01	1.73472347597681e-18\\
511.01	1.73472347597681e-18\\
512.01	0\\
513.01	1.73472347597681e-18\\
514.01	1.73472347597681e-18\\
515.01	0\\
516.01	0\\
517.01	1.73472347597681e-18\\
518.01	1.73472347597681e-18\\
519.01	1.73472347597681e-18\\
520.01	0\\
521.01	0\\
522.01	0\\
523.01	1.73472347597681e-18\\
524.01	0\\
525.01	1.73472347597681e-18\\
526.01	1.73472347597681e-18\\
527.01	0\\
528.01	0\\
529.01	0\\
530.01	1.73472347597681e-18\\
531.01	0\\
532.01	1.73472347597681e-18\\
533.01	0\\
534.01	1.73472347597681e-18\\
535.01	0\\
536.01	1.73472347597681e-18\\
537.01	1.73472347597681e-18\\
538.01	1.73472347597681e-18\\
539.01	0\\
540.01	0\\
541.01	1.73472347597681e-18\\
542.01	1.73472347597681e-18\\
543.01	0\\
544.01	1.73472347597681e-18\\
545.01	0\\
546.01	1.73472347597681e-18\\
547.01	0\\
548.01	0\\
549.01	0\\
550.01	0\\
551.01	0\\
552.01	0\\
553.01	0\\
554.01	0\\
555.01	1.73472347597681e-18\\
556.01	0\\
557.01	0\\
558.01	0\\
559.01	1.73472347597681e-18\\
560.01	1.73472347597681e-18\\
561.01	1.73472347597681e-18\\
562.01	1.73472347597681e-18\\
563.01	0\\
564.01	0\\
565.01	1.73472347597681e-18\\
566.01	0\\
567.01	0\\
568.01	0\\
569.01	0\\
570.01	0\\
571.01	0\\
572.01	1.73472347597681e-18\\
573.01	0\\
574.01	1.73472347597681e-18\\
575.01	0\\
576.01	1.73472347597681e-18\\
577.01	0\\
578.01	1.73472347597681e-18\\
579.01	0\\
580.01	0\\
581.01	1.73472347597681e-18\\
582.01	0\\
583.01	0\\
584.01	0\\
585.01	0\\
586.01	0\\
587.01	0\\
588.01	0\\
589.01	0\\
590.01	0\\
591.01	0\\
592.01	1.73472347597681e-18\\
593.01	0\\
594.01	0\\
595.01	0\\
596.01	0\\
597.01	0\\
598.01	0\\
599.01	0\\
599.02	0\\
599.03	1.73472347597681e-18\\
599.04	1.73472347597681e-18\\
599.05	0\\
599.06	0\\
599.07	1.73472347597681e-18\\
599.08	0\\
599.09	0\\
599.1	0\\
599.11	1.73472347597681e-18\\
599.12	1.73472347597681e-18\\
599.13	0\\
599.14	0\\
599.15	1.73472347597681e-18\\
599.16	1.73472347597681e-18\\
599.17	1.73472347597681e-18\\
599.18	0\\
599.19	0\\
599.2	0\\
599.21	0\\
599.22	0\\
599.23	1.73472347597681e-18\\
599.24	1.73472347597681e-18\\
599.25	1.73472347597681e-18\\
599.26	1.73472347597681e-18\\
599.27	0\\
599.28	0\\
599.29	0\\
599.3	0\\
599.31	0\\
599.32	1.73472347597681e-18\\
599.33	1.73472347597681e-18\\
599.34	0\\
599.35	0\\
599.36	0\\
599.37	1.73472347597681e-18\\
599.38	0\\
599.39	0\\
599.4	0\\
599.41	0\\
599.42	0\\
599.43	0\\
599.44	0\\
599.45	0\\
599.46	1.73472347597681e-18\\
599.47	1.73472347597681e-18\\
599.48	1.73472347597681e-18\\
599.49	1.73472347597681e-18\\
599.5	1.73472347597681e-18\\
599.51	0\\
599.52	0\\
599.53	1.73472347597681e-18\\
599.54	0\\
599.55	0\\
599.56	0\\
599.57	1.73472347597681e-18\\
599.58	0\\
599.59	0\\
599.6	0\\
599.61	0\\
599.62	1.73472347597681e-18\\
599.63	0\\
599.64	1.73472347597681e-18\\
599.65	0\\
599.66	0\\
599.67	0\\
599.68	0\\
599.69	1.73472347597681e-18\\
599.7	0\\
599.71	1.73472347597681e-18\\
599.72	1.73472347597681e-18\\
599.73	1.73472347597681e-18\\
599.74	0\\
599.75	0\\
599.76	0\\
599.77	0\\
599.78	0\\
599.79	0\\
599.8	0\\
599.81	0\\
599.82	0\\
599.83	1.73472347597681e-18\\
599.84	0\\
599.85	0\\
599.86	0\\
599.87	0\\
599.88	0\\
599.89	0\\
599.9	0\\
599.91	0\\
599.92	0\\
599.93	0\\
599.94	0\\
599.95	0\\
599.96	0\\
599.97	0\\
599.98	0\\
599.99	0\\
600	0\\
};
\addplot [color=blue!25!mycolor7,solid,forget plot]
  table[row sep=crcr]{%
0.01	1.73472347597681e-18\\
1.01	1.73472347597681e-18\\
2.01	0\\
3.01	1.73472347597681e-18\\
4.01	1.73472347597681e-18\\
5.01	0\\
6.01	0\\
7.01	1.73472347597681e-18\\
8.01	0\\
9.01	0\\
10.01	0\\
11.01	0\\
12.01	1.73472347597681e-18\\
13.01	1.73472347597681e-18\\
14.01	1.73472347597681e-18\\
15.01	1.73472347597681e-18\\
16.01	1.73472347597681e-18\\
17.01	1.73472347597681e-18\\
18.01	0\\
19.01	0\\
20.01	1.73472347597681e-18\\
21.01	0\\
22.01	1.73472347597681e-18\\
23.01	1.73472347597681e-18\\
24.01	1.73472347597681e-18\\
25.01	0\\
26.01	1.73472347597681e-18\\
27.01	0\\
28.01	1.73472347597681e-18\\
29.01	1.73472347597681e-18\\
30.01	0\\
31.01	0\\
32.01	1.73472347597681e-18\\
33.01	1.73472347597681e-18\\
34.01	1.73472347597681e-18\\
35.01	0\\
36.01	0\\
37.01	0\\
38.01	0\\
39.01	1.73472347597681e-18\\
40.01	1.73472347597681e-18\\
41.01	1.73472347597681e-18\\
42.01	1.73472347597681e-18\\
43.01	1.73472347597681e-18\\
44.01	0\\
45.01	1.73472347597681e-18\\
46.01	1.73472347597681e-18\\
47.01	1.73472347597681e-18\\
48.01	1.73472347597681e-18\\
49.01	0\\
50.01	1.73472347597681e-18\\
51.01	1.73472347597681e-18\\
52.01	0\\
53.01	0\\
54.01	0\\
55.01	1.73472347597681e-18\\
56.01	0\\
57.01	1.73472347597681e-18\\
58.01	0\\
59.01	0\\
60.01	1.73472347597681e-18\\
61.01	1.73472347597681e-18\\
62.01	1.73472347597681e-18\\
63.01	0\\
64.01	1.73472347597681e-18\\
65.01	0\\
66.01	1.73472347597681e-18\\
67.01	1.73472347597681e-18\\
68.01	0\\
69.01	1.73472347597681e-18\\
70.01	1.73472347597681e-18\\
71.01	0\\
72.01	0\\
73.01	1.73472347597681e-18\\
74.01	0\\
75.01	1.73472347597681e-18\\
76.01	0\\
77.01	0\\
78.01	0\\
79.01	0\\
80.01	1.73472347597681e-18\\
81.01	1.73472347597681e-18\\
82.01	0\\
83.01	0\\
84.01	0\\
85.01	1.73472347597681e-18\\
86.01	1.73472347597681e-18\\
87.01	0\\
88.01	0\\
89.01	0\\
90.01	1.73472347597681e-18\\
91.01	1.73472347597681e-18\\
92.01	0\\
93.01	1.73472347597681e-18\\
94.01	1.73472347597681e-18\\
95.01	0\\
96.01	1.73472347597681e-18\\
97.01	0\\
98.01	0\\
99.01	0\\
100.01	1.73472347597681e-18\\
101.01	0\\
102.01	1.73472347597681e-18\\
103.01	1.73472347597681e-18\\
104.01	0\\
105.01	0\\
106.01	0\\
107.01	0\\
108.01	1.73472347597681e-18\\
109.01	0\\
110.01	1.73472347597681e-18\\
111.01	0\\
112.01	0\\
113.01	0\\
114.01	1.73472347597681e-18\\
115.01	1.73472347597681e-18\\
116.01	0\\
117.01	0\\
118.01	0\\
119.01	0\\
120.01	0\\
121.01	0\\
122.01	0\\
123.01	1.73472347597681e-18\\
124.01	1.73472347597681e-18\\
125.01	1.73472347597681e-18\\
126.01	1.73472347597681e-18\\
127.01	0\\
128.01	1.73472347597681e-18\\
129.01	0\\
130.01	1.73472347597681e-18\\
131.01	1.73472347597681e-18\\
132.01	0\\
133.01	1.73472347597681e-18\\
134.01	1.73472347597681e-18\\
135.01	0\\
136.01	0\\
137.01	1.73472347597681e-18\\
138.01	0\\
139.01	1.73472347597681e-18\\
140.01	1.73472347597681e-18\\
141.01	1.73472347597681e-18\\
142.01	1.73472347597681e-18\\
143.01	0\\
144.01	1.73472347597681e-18\\
145.01	1.73472347597681e-18\\
146.01	1.73472347597681e-18\\
147.01	1.73472347597681e-18\\
148.01	1.73472347597681e-18\\
149.01	0\\
150.01	0\\
151.01	1.73472347597681e-18\\
152.01	1.73472347597681e-18\\
153.01	1.73472347597681e-18\\
154.01	1.73472347597681e-18\\
155.01	1.73472347597681e-18\\
156.01	1.73472347597681e-18\\
157.01	0\\
158.01	1.73472347597681e-18\\
159.01	1.73472347597681e-18\\
160.01	1.73472347597681e-18\\
161.01	1.73472347597681e-18\\
162.01	1.73472347597681e-18\\
163.01	1.73472347597681e-18\\
164.01	0\\
165.01	0\\
166.01	1.73472347597681e-18\\
167.01	0\\
168.01	1.73472347597681e-18\\
169.01	0\\
170.01	1.73472347597681e-18\\
171.01	0\\
172.01	0\\
173.01	0\\
174.01	1.73472347597681e-18\\
175.01	1.73472347597681e-18\\
176.01	1.73472347597681e-18\\
177.01	0\\
178.01	1.73472347597681e-18\\
179.01	0\\
180.01	0\\
181.01	1.73472347597681e-18\\
182.01	0\\
183.01	0\\
184.01	0\\
185.01	1.73472347597681e-18\\
186.01	0\\
187.01	0\\
188.01	0\\
189.01	0\\
190.01	1.73472347597681e-18\\
191.01	0\\
192.01	1.73472347597681e-18\\
193.01	0\\
194.01	1.73472347597681e-18\\
195.01	0\\
196.01	1.73472347597681e-18\\
197.01	0\\
198.01	0\\
199.01	0\\
200.01	0\\
201.01	0\\
202.01	1.73472347597681e-18\\
203.01	0\\
204.01	1.73472347597681e-18\\
205.01	1.73472347597681e-18\\
206.01	1.73472347597681e-18\\
207.01	1.73472347597681e-18\\
208.01	0\\
209.01	1.73472347597681e-18\\
210.01	0\\
211.01	1.73472347597681e-18\\
212.01	0\\
213.01	1.73472347597681e-18\\
214.01	0\\
215.01	0\\
216.01	0\\
217.01	0\\
218.01	1.73472347597681e-18\\
219.01	1.73472347597681e-18\\
220.01	0\\
221.01	0\\
222.01	0\\
223.01	1.73472347597681e-18\\
224.01	1.73472347597681e-18\\
225.01	0\\
226.01	1.73472347597681e-18\\
227.01	0\\
228.01	0\\
229.01	1.73472347597681e-18\\
230.01	0\\
231.01	0\\
232.01	0\\
233.01	0\\
234.01	1.73472347597681e-18\\
235.01	0\\
236.01	1.73472347597681e-18\\
237.01	1.73472347597681e-18\\
238.01	0\\
239.01	1.73472347597681e-18\\
240.01	0\\
241.01	1.73472347597681e-18\\
242.01	0\\
243.01	1.73472347597681e-18\\
244.01	1.73472347597681e-18\\
245.01	1.73472347597681e-18\\
246.01	1.73472347597681e-18\\
247.01	0\\
248.01	0\\
249.01	0\\
250.01	1.73472347597681e-18\\
251.01	0\\
252.01	1.73472347597681e-18\\
253.01	1.73472347597681e-18\\
254.01	1.73472347597681e-18\\
255.01	1.73472347597681e-18\\
256.01	1.73472347597681e-18\\
257.01	1.73472347597681e-18\\
258.01	0\\
259.01	0\\
260.01	0\\
261.01	0\\
262.01	1.73472347597681e-18\\
263.01	1.73472347597681e-18\\
264.01	0\\
265.01	1.73472347597681e-18\\
266.01	1.73472347597681e-18\\
267.01	0\\
268.01	0\\
269.01	1.73472347597681e-18\\
270.01	0\\
271.01	1.73472347597681e-18\\
272.01	0\\
273.01	1.73472347597681e-18\\
274.01	0\\
275.01	0\\
276.01	1.73472347597681e-18\\
277.01	1.73472347597681e-18\\
278.01	0\\
279.01	1.73472347597681e-18\\
280.01	0\\
281.01	1.73472347597681e-18\\
282.01	1.73472347597681e-18\\
283.01	1.73472347597681e-18\\
284.01	0\\
285.01	1.73472347597681e-18\\
286.01	1.73472347597681e-18\\
287.01	1.73472347597681e-18\\
288.01	1.73472347597681e-18\\
289.01	0\\
290.01	0\\
291.01	0\\
292.01	0\\
293.01	1.73472347597681e-18\\
294.01	1.73472347597681e-18\\
295.01	0\\
296.01	1.73472347597681e-18\\
297.01	0\\
298.01	1.73472347597681e-18\\
299.01	0\\
300.01	1.73472347597681e-18\\
301.01	0\\
302.01	0\\
303.01	0\\
304.01	1.73472347597681e-18\\
305.01	1.73472347597681e-18\\
306.01	0\\
307.01	1.73472347597681e-18\\
308.01	0\\
309.01	1.73472347597681e-18\\
310.01	1.73472347597681e-18\\
311.01	0\\
312.01	1.73472347597681e-18\\
313.01	1.73472347597681e-18\\
314.01	0\\
315.01	1.73472347597681e-18\\
316.01	0\\
317.01	1.73472347597681e-18\\
318.01	0\\
319.01	0\\
320.01	0\\
321.01	0\\
322.01	1.73472347597681e-18\\
323.01	1.73472347597681e-18\\
324.01	0\\
325.01	1.73472347597681e-18\\
326.01	1.73472347597681e-18\\
327.01	0\\
328.01	1.73472347597681e-18\\
329.01	0\\
330.01	0\\
331.01	1.73472347597681e-18\\
332.01	1.73472347597681e-18\\
333.01	1.73472347597681e-18\\
334.01	1.73472347597681e-18\\
335.01	1.73472347597681e-18\\
336.01	0\\
337.01	1.73472347597681e-18\\
338.01	0\\
339.01	0\\
340.01	1.73472347597681e-18\\
341.01	1.73472347597681e-18\\
342.01	1.73472347597681e-18\\
343.01	1.73472347597681e-18\\
344.01	0\\
345.01	0\\
346.01	0\\
347.01	1.73472347597681e-18\\
348.01	0\\
349.01	1.73472347597681e-18\\
350.01	0\\
351.01	0\\
352.01	0\\
353.01	1.73472347597681e-18\\
354.01	1.73472347597681e-18\\
355.01	0\\
356.01	0\\
357.01	1.73472347597681e-18\\
358.01	1.73472347597681e-18\\
359.01	1.73472347597681e-18\\
360.01	0\\
361.01	0\\
362.01	0\\
363.01	0\\
364.01	0\\
365.01	1.73472347597681e-18\\
366.01	1.73472347597681e-18\\
367.01	1.73472347597681e-18\\
368.01	1.73472347597681e-18\\
369.01	0\\
370.01	0\\
371.01	1.73472347597681e-18\\
372.01	1.73472347597681e-18\\
373.01	1.73472347597681e-18\\
374.01	0\\
375.01	1.73472347597681e-18\\
376.01	0\\
377.01	0\\
378.01	1.73472347597681e-18\\
379.01	0\\
380.01	1.73472347597681e-18\\
381.01	1.73472347597681e-18\\
382.01	1.73472347597681e-18\\
383.01	0\\
384.01	1.73472347597681e-18\\
385.01	0\\
386.01	1.73472347597681e-18\\
387.01	1.73472347597681e-18\\
388.01	1.73472347597681e-18\\
389.01	0\\
390.01	1.73472347597681e-18\\
391.01	1.73472347597681e-18\\
392.01	0\\
393.01	0\\
394.01	1.73472347597681e-18\\
395.01	1.73472347597681e-18\\
396.01	1.73472347597681e-18\\
397.01	1.73472347597681e-18\\
398.01	0\\
399.01	1.73472347597681e-18\\
400.01	0\\
401.01	0\\
402.01	0\\
403.01	1.73472347597681e-18\\
404.01	1.73472347597681e-18\\
405.01	1.73472347597681e-18\\
406.01	1.73472347597681e-18\\
407.01	1.73472347597681e-18\\
408.01	1.73472347597681e-18\\
409.01	0\\
410.01	1.73472347597681e-18\\
411.01	0\\
412.01	1.73472347597681e-18\\
413.01	1.73472347597681e-18\\
414.01	0\\
415.01	0\\
416.01	1.73472347597681e-18\\
417.01	1.73472347597681e-18\\
418.01	1.73472347597681e-18\\
419.01	1.73472347597681e-18\\
420.01	0\\
421.01	1.73472347597681e-18\\
422.01	1.73472347597681e-18\\
423.01	1.73472347597681e-18\\
424.01	1.73472347597681e-18\\
425.01	0\\
426.01	1.73472347597681e-18\\
427.01	1.73472347597681e-18\\
428.01	1.73472347597681e-18\\
429.01	0\\
430.01	1.73472347597681e-18\\
431.01	1.73472347597681e-18\\
432.01	1.73472347597681e-18\\
433.01	0\\
434.01	1.73472347597681e-18\\
435.01	1.73472347597681e-18\\
436.01	1.73472347597681e-18\\
437.01	1.73472347597681e-18\\
438.01	1.73472347597681e-18\\
439.01	0\\
440.01	1.73472347597681e-18\\
441.01	1.73472347597681e-18\\
442.01	0\\
443.01	0\\
444.01	0\\
445.01	1.73472347597681e-18\\
446.01	1.73472347597681e-18\\
447.01	0\\
448.01	1.73472347597681e-18\\
449.01	1.73472347597681e-18\\
450.01	0\\
451.01	0\\
452.01	0\\
453.01	1.73472347597681e-18\\
454.01	1.73472347597681e-18\\
455.01	0\\
456.01	1.73472347597681e-18\\
457.01	1.73472347597681e-18\\
458.01	0\\
459.01	1.73472347597681e-18\\
460.01	0\\
461.01	1.73472347597681e-18\\
462.01	0\\
463.01	0\\
464.01	0\\
465.01	1.73472347597681e-18\\
466.01	0\\
467.01	1.73472347597681e-18\\
468.01	0\\
469.01	1.73472347597681e-18\\
470.01	1.73472347597681e-18\\
471.01	1.73472347597681e-18\\
472.01	0\\
473.01	0\\
474.01	0\\
475.01	1.73472347597681e-18\\
476.01	0\\
477.01	0\\
478.01	1.73472347597681e-18\\
479.01	1.73472347597681e-18\\
480.01	0\\
481.01	0\\
482.01	1.73472347597681e-18\\
483.01	1.73472347597681e-18\\
484.01	0\\
485.01	0\\
486.01	0\\
487.01	1.73472347597681e-18\\
488.01	1.73472347597681e-18\\
489.01	1.73472347597681e-18\\
490.01	1.73472347597681e-18\\
491.01	0\\
492.01	1.73472347597681e-18\\
493.01	1.73472347597681e-18\\
494.01	0\\
495.01	1.73472347597681e-18\\
496.01	0\\
497.01	1.73472347597681e-18\\
498.01	1.73472347597681e-18\\
499.01	0\\
500.01	1.73472347597681e-18\\
501.01	1.73472347597681e-18\\
502.01	0\\
503.01	1.73472347597681e-18\\
504.01	1.73472347597681e-18\\
505.01	1.73472347597681e-18\\
506.01	0\\
507.01	0\\
508.01	0\\
509.01	0\\
510.01	1.73472347597681e-18\\
511.01	1.73472347597681e-18\\
512.01	0\\
513.01	1.73472347597681e-18\\
514.01	1.73472347597681e-18\\
515.01	0\\
516.01	0\\
517.01	1.73472347597681e-18\\
518.01	1.73472347597681e-18\\
519.01	1.73472347597681e-18\\
520.01	0\\
521.01	0\\
522.01	0\\
523.01	1.73472347597681e-18\\
524.01	0\\
525.01	1.73472347597681e-18\\
526.01	1.73472347597681e-18\\
527.01	0\\
528.01	0\\
529.01	0\\
530.01	1.73472347597681e-18\\
531.01	0\\
532.01	1.73472347597681e-18\\
533.01	0\\
534.01	1.73472347597681e-18\\
535.01	0\\
536.01	1.73472347597681e-18\\
537.01	1.73472347597681e-18\\
538.01	1.73472347597681e-18\\
539.01	0\\
540.01	0\\
541.01	1.73472347597681e-18\\
542.01	1.73472347597681e-18\\
543.01	0\\
544.01	1.73472347597681e-18\\
545.01	0\\
546.01	1.73472347597681e-18\\
547.01	0\\
548.01	0\\
549.01	0\\
550.01	0\\
551.01	0\\
552.01	0\\
553.01	0\\
554.01	0\\
555.01	1.73472347597681e-18\\
556.01	0\\
557.01	0\\
558.01	0\\
559.01	1.73472347597681e-18\\
560.01	1.73472347597681e-18\\
561.01	1.73472347597681e-18\\
562.01	1.73472347597681e-18\\
563.01	0\\
564.01	0\\
565.01	1.73472347597681e-18\\
566.01	0\\
567.01	0\\
568.01	0\\
569.01	0\\
570.01	0\\
571.01	0\\
572.01	1.73472347597681e-18\\
573.01	0\\
574.01	1.73472347597681e-18\\
575.01	0\\
576.01	1.73472347597681e-18\\
577.01	0\\
578.01	1.73472347597681e-18\\
579.01	0\\
580.01	0\\
581.01	1.73472347597681e-18\\
582.01	0\\
583.01	0\\
584.01	0\\
585.01	0\\
586.01	0\\
587.01	0\\
588.01	0\\
589.01	0\\
590.01	0\\
591.01	0\\
592.01	1.73472347597681e-18\\
593.01	0\\
594.01	0\\
595.01	0\\
596.01	0\\
597.01	0\\
598.01	0\\
599.01	0\\
599.02	0\\
599.03	1.73472347597681e-18\\
599.04	1.73472347597681e-18\\
599.05	0\\
599.06	0\\
599.07	1.73472347597681e-18\\
599.08	0\\
599.09	0\\
599.1	0\\
599.11	1.73472347597681e-18\\
599.12	1.73472347597681e-18\\
599.13	0\\
599.14	0\\
599.15	1.73472347597681e-18\\
599.16	1.73472347597681e-18\\
599.17	1.73472347597681e-18\\
599.18	0\\
599.19	0\\
599.2	0\\
599.21	0\\
599.22	0\\
599.23	1.73472347597681e-18\\
599.24	1.73472347597681e-18\\
599.25	1.73472347597681e-18\\
599.26	1.73472347597681e-18\\
599.27	0\\
599.28	0\\
599.29	0\\
599.3	0\\
599.31	0\\
599.32	1.73472347597681e-18\\
599.33	1.73472347597681e-18\\
599.34	0\\
599.35	0\\
599.36	0\\
599.37	1.73472347597681e-18\\
599.38	0\\
599.39	0\\
599.4	0\\
599.41	0\\
599.42	0\\
599.43	0\\
599.44	0\\
599.45	0\\
599.46	1.73472347597681e-18\\
599.47	1.73472347597681e-18\\
599.48	1.73472347597681e-18\\
599.49	1.73472347597681e-18\\
599.5	1.73472347597681e-18\\
599.51	0\\
599.52	0\\
599.53	1.73472347597681e-18\\
599.54	0\\
599.55	0\\
599.56	0\\
599.57	1.73472347597681e-18\\
599.58	0\\
599.59	0\\
599.6	0\\
599.61	0\\
599.62	1.73472347597681e-18\\
599.63	0\\
599.64	1.73472347597681e-18\\
599.65	0\\
599.66	0\\
599.67	0\\
599.68	0\\
599.69	1.73472347597681e-18\\
599.7	0\\
599.71	1.73472347597681e-18\\
599.72	1.73472347597681e-18\\
599.73	1.73472347597681e-18\\
599.74	0\\
599.75	0\\
599.76	0\\
599.77	0\\
599.78	0\\
599.79	0\\
599.8	0\\
599.81	0\\
599.82	0\\
599.83	1.73472347597681e-18\\
599.84	0\\
599.85	0\\
599.86	0\\
599.87	0\\
599.88	0\\
599.89	0\\
599.9	0\\
599.91	0\\
599.92	0\\
599.93	0\\
599.94	0\\
599.95	0\\
599.96	0\\
599.97	0\\
599.98	0\\
599.99	0\\
600	0\\
};
\addplot [color=mycolor9,solid,forget plot]
  table[row sep=crcr]{%
0.01	1.73472347597681e-18\\
1.01	1.73472347597681e-18\\
2.01	0\\
3.01	1.73472347597681e-18\\
4.01	1.73472347597681e-18\\
5.01	0\\
6.01	0\\
7.01	1.73472347597681e-18\\
8.01	0\\
9.01	0\\
10.01	0\\
11.01	0\\
12.01	1.73472347597681e-18\\
13.01	1.73472347597681e-18\\
14.01	1.73472347597681e-18\\
15.01	1.73472347597681e-18\\
16.01	1.73472347597681e-18\\
17.01	1.73472347597681e-18\\
18.01	0\\
19.01	0\\
20.01	1.73472347597681e-18\\
21.01	0\\
22.01	1.73472347597681e-18\\
23.01	1.73472347597681e-18\\
24.01	1.73472347597681e-18\\
25.01	0\\
26.01	1.73472347597681e-18\\
27.01	0\\
28.01	1.73472347597681e-18\\
29.01	1.73472347597681e-18\\
30.01	0\\
31.01	0\\
32.01	1.73472347597681e-18\\
33.01	1.73472347597681e-18\\
34.01	1.73472347597681e-18\\
35.01	0\\
36.01	0\\
37.01	0\\
38.01	0\\
39.01	1.73472347597681e-18\\
40.01	1.73472347597681e-18\\
41.01	1.73472347597681e-18\\
42.01	1.73472347597681e-18\\
43.01	1.73472347597681e-18\\
44.01	0\\
45.01	1.73472347597681e-18\\
46.01	1.73472347597681e-18\\
47.01	1.73472347597681e-18\\
48.01	1.73472347597681e-18\\
49.01	0\\
50.01	1.73472347597681e-18\\
51.01	1.73472347597681e-18\\
52.01	0\\
53.01	0\\
54.01	0\\
55.01	1.73472347597681e-18\\
56.01	0\\
57.01	1.73472347597681e-18\\
58.01	0\\
59.01	0\\
60.01	1.73472347597681e-18\\
61.01	1.73472347597681e-18\\
62.01	1.73472347597681e-18\\
63.01	0\\
64.01	1.73472347597681e-18\\
65.01	0\\
66.01	1.73472347597681e-18\\
67.01	1.73472347597681e-18\\
68.01	0\\
69.01	1.73472347597681e-18\\
70.01	1.73472347597681e-18\\
71.01	0\\
72.01	0\\
73.01	1.73472347597681e-18\\
74.01	0\\
75.01	1.73472347597681e-18\\
76.01	0\\
77.01	0\\
78.01	0\\
79.01	0\\
80.01	1.73472347597681e-18\\
81.01	1.73472347597681e-18\\
82.01	0\\
83.01	0\\
84.01	0\\
85.01	1.73472347597681e-18\\
86.01	1.73472347597681e-18\\
87.01	0\\
88.01	0\\
89.01	0\\
90.01	1.73472347597681e-18\\
91.01	1.73472347597681e-18\\
92.01	0\\
93.01	1.73472347597681e-18\\
94.01	1.73472347597681e-18\\
95.01	0\\
96.01	1.73472347597681e-18\\
97.01	0\\
98.01	0\\
99.01	0\\
100.01	1.73472347597681e-18\\
101.01	0\\
102.01	1.73472347597681e-18\\
103.01	1.73472347597681e-18\\
104.01	0\\
105.01	0\\
106.01	0\\
107.01	0\\
108.01	1.73472347597681e-18\\
109.01	0\\
110.01	1.73472347597681e-18\\
111.01	0\\
112.01	0\\
113.01	0\\
114.01	1.73472347597681e-18\\
115.01	1.73472347597681e-18\\
116.01	0\\
117.01	0\\
118.01	0\\
119.01	0\\
120.01	0\\
121.01	0\\
122.01	0\\
123.01	1.73472347597681e-18\\
124.01	1.73472347597681e-18\\
125.01	1.73472347597681e-18\\
126.01	1.73472347597681e-18\\
127.01	0\\
128.01	1.73472347597681e-18\\
129.01	0\\
130.01	1.73472347597681e-18\\
131.01	1.73472347597681e-18\\
132.01	0\\
133.01	1.73472347597681e-18\\
134.01	1.73472347597681e-18\\
135.01	0\\
136.01	0\\
137.01	1.73472347597681e-18\\
138.01	0\\
139.01	1.73472347597681e-18\\
140.01	1.73472347597681e-18\\
141.01	1.73472347597681e-18\\
142.01	1.73472347597681e-18\\
143.01	0\\
144.01	1.73472347597681e-18\\
145.01	1.73472347597681e-18\\
146.01	1.73472347597681e-18\\
147.01	1.73472347597681e-18\\
148.01	1.73472347597681e-18\\
149.01	0\\
150.01	0\\
151.01	1.73472347597681e-18\\
152.01	1.73472347597681e-18\\
153.01	1.73472347597681e-18\\
154.01	1.73472347597681e-18\\
155.01	1.73472347597681e-18\\
156.01	1.73472347597681e-18\\
157.01	0\\
158.01	1.73472347597681e-18\\
159.01	1.73472347597681e-18\\
160.01	1.73472347597681e-18\\
161.01	1.73472347597681e-18\\
162.01	1.73472347597681e-18\\
163.01	1.73472347597681e-18\\
164.01	0\\
165.01	0\\
166.01	1.73472347597681e-18\\
167.01	0\\
168.01	1.73472347597681e-18\\
169.01	0\\
170.01	1.73472347597681e-18\\
171.01	0\\
172.01	0\\
173.01	0\\
174.01	1.73472347597681e-18\\
175.01	1.73472347597681e-18\\
176.01	1.73472347597681e-18\\
177.01	0\\
178.01	1.73472347597681e-18\\
179.01	0\\
180.01	0\\
181.01	1.73472347597681e-18\\
182.01	0\\
183.01	0\\
184.01	0\\
185.01	1.73472347597681e-18\\
186.01	0\\
187.01	0\\
188.01	0\\
189.01	0\\
190.01	1.73472347597681e-18\\
191.01	0\\
192.01	1.73472347597681e-18\\
193.01	0\\
194.01	1.73472347597681e-18\\
195.01	0\\
196.01	1.73472347597681e-18\\
197.01	0\\
198.01	0\\
199.01	0\\
200.01	0\\
201.01	0\\
202.01	1.73472347597681e-18\\
203.01	0\\
204.01	1.73472347597681e-18\\
205.01	1.73472347597681e-18\\
206.01	1.73472347597681e-18\\
207.01	1.73472347597681e-18\\
208.01	0\\
209.01	1.73472347597681e-18\\
210.01	0\\
211.01	1.73472347597681e-18\\
212.01	0\\
213.01	1.73472347597681e-18\\
214.01	0\\
215.01	0\\
216.01	0\\
217.01	0\\
218.01	1.73472347597681e-18\\
219.01	1.73472347597681e-18\\
220.01	0\\
221.01	0\\
222.01	0\\
223.01	1.73472347597681e-18\\
224.01	1.73472347597681e-18\\
225.01	0\\
226.01	1.73472347597681e-18\\
227.01	0\\
228.01	0\\
229.01	1.73472347597681e-18\\
230.01	0\\
231.01	0\\
232.01	0\\
233.01	0\\
234.01	1.73472347597681e-18\\
235.01	0\\
236.01	1.73472347597681e-18\\
237.01	1.73472347597681e-18\\
238.01	0\\
239.01	1.73472347597681e-18\\
240.01	0\\
241.01	1.73472347597681e-18\\
242.01	0\\
243.01	1.73472347597681e-18\\
244.01	1.73472347597681e-18\\
245.01	1.73472347597681e-18\\
246.01	1.73472347597681e-18\\
247.01	0\\
248.01	0\\
249.01	0\\
250.01	1.73472347597681e-18\\
251.01	0\\
252.01	1.73472347597681e-18\\
253.01	1.73472347597681e-18\\
254.01	1.73472347597681e-18\\
255.01	1.73472347597681e-18\\
256.01	1.73472347597681e-18\\
257.01	1.73472347597681e-18\\
258.01	0\\
259.01	0\\
260.01	0\\
261.01	0\\
262.01	1.73472347597681e-18\\
263.01	1.73472347597681e-18\\
264.01	0\\
265.01	1.73472347597681e-18\\
266.01	1.73472347597681e-18\\
267.01	0\\
268.01	0\\
269.01	1.73472347597681e-18\\
270.01	0\\
271.01	1.73472347597681e-18\\
272.01	0\\
273.01	1.73472347597681e-18\\
274.01	0\\
275.01	0\\
276.01	1.73472347597681e-18\\
277.01	1.73472347597681e-18\\
278.01	0\\
279.01	1.73472347597681e-18\\
280.01	0\\
281.01	1.73472347597681e-18\\
282.01	1.73472347597681e-18\\
283.01	1.73472347597681e-18\\
284.01	0\\
285.01	1.73472347597681e-18\\
286.01	1.73472347597681e-18\\
287.01	1.73472347597681e-18\\
288.01	1.73472347597681e-18\\
289.01	0\\
290.01	0\\
291.01	0\\
292.01	0\\
293.01	1.73472347597681e-18\\
294.01	1.73472347597681e-18\\
295.01	0\\
296.01	1.73472347597681e-18\\
297.01	0\\
298.01	1.73472347597681e-18\\
299.01	0\\
300.01	1.73472347597681e-18\\
301.01	0\\
302.01	0\\
303.01	0\\
304.01	1.73472347597681e-18\\
305.01	1.73472347597681e-18\\
306.01	0\\
307.01	1.73472347597681e-18\\
308.01	0\\
309.01	1.73472347597681e-18\\
310.01	1.73472347597681e-18\\
311.01	0\\
312.01	1.73472347597681e-18\\
313.01	1.73472347597681e-18\\
314.01	0\\
315.01	1.73472347597681e-18\\
316.01	0\\
317.01	1.73472347597681e-18\\
318.01	0\\
319.01	0\\
320.01	0\\
321.01	0\\
322.01	1.73472347597681e-18\\
323.01	1.73472347597681e-18\\
324.01	0\\
325.01	1.73472347597681e-18\\
326.01	1.73472347597681e-18\\
327.01	0\\
328.01	1.73472347597681e-18\\
329.01	0\\
330.01	0\\
331.01	1.73472347597681e-18\\
332.01	1.73472347597681e-18\\
333.01	1.73472347597681e-18\\
334.01	1.73472347597681e-18\\
335.01	1.73472347597681e-18\\
336.01	0\\
337.01	1.73472347597681e-18\\
338.01	0\\
339.01	0\\
340.01	1.73472347597681e-18\\
341.01	1.73472347597681e-18\\
342.01	1.73472347597681e-18\\
343.01	1.73472347597681e-18\\
344.01	0\\
345.01	0\\
346.01	0\\
347.01	1.73472347597681e-18\\
348.01	0\\
349.01	1.73472347597681e-18\\
350.01	0\\
351.01	0\\
352.01	0\\
353.01	1.73472347597681e-18\\
354.01	1.73472347597681e-18\\
355.01	0\\
356.01	0\\
357.01	1.73472347597681e-18\\
358.01	1.73472347597681e-18\\
359.01	1.73472347597681e-18\\
360.01	0\\
361.01	0\\
362.01	0\\
363.01	0\\
364.01	0\\
365.01	1.73472347597681e-18\\
366.01	1.73472347597681e-18\\
367.01	1.73472347597681e-18\\
368.01	1.73472347597681e-18\\
369.01	0\\
370.01	0\\
371.01	1.73472347597681e-18\\
372.01	1.73472347597681e-18\\
373.01	1.73472347597681e-18\\
374.01	0\\
375.01	1.73472347597681e-18\\
376.01	0\\
377.01	0\\
378.01	1.73472347597681e-18\\
379.01	0\\
380.01	1.73472347597681e-18\\
381.01	1.73472347597681e-18\\
382.01	1.73472347597681e-18\\
383.01	0\\
384.01	1.73472347597681e-18\\
385.01	0\\
386.01	1.73472347597681e-18\\
387.01	1.73472347597681e-18\\
388.01	1.73472347597681e-18\\
389.01	0\\
390.01	1.73472347597681e-18\\
391.01	1.73472347597681e-18\\
392.01	0\\
393.01	0\\
394.01	1.73472347597681e-18\\
395.01	1.73472347597681e-18\\
396.01	1.73472347597681e-18\\
397.01	1.73472347597681e-18\\
398.01	0\\
399.01	1.73472347597681e-18\\
400.01	0\\
401.01	0\\
402.01	0\\
403.01	1.73472347597681e-18\\
404.01	1.73472347597681e-18\\
405.01	1.73472347597681e-18\\
406.01	1.73472347597681e-18\\
407.01	1.73472347597681e-18\\
408.01	1.73472347597681e-18\\
409.01	0\\
410.01	1.73472347597681e-18\\
411.01	0\\
412.01	1.73472347597681e-18\\
413.01	1.73472347597681e-18\\
414.01	0\\
415.01	0\\
416.01	1.73472347597681e-18\\
417.01	1.73472347597681e-18\\
418.01	1.73472347597681e-18\\
419.01	1.73472347597681e-18\\
420.01	0\\
421.01	1.73472347597681e-18\\
422.01	1.73472347597681e-18\\
423.01	1.73472347597681e-18\\
424.01	1.73472347597681e-18\\
425.01	0\\
426.01	1.73472347597681e-18\\
427.01	1.73472347597681e-18\\
428.01	1.73472347597681e-18\\
429.01	0\\
430.01	1.73472347597681e-18\\
431.01	1.73472347597681e-18\\
432.01	1.73472347597681e-18\\
433.01	0\\
434.01	1.73472347597681e-18\\
435.01	1.73472347597681e-18\\
436.01	1.73472347597681e-18\\
437.01	1.73472347597681e-18\\
438.01	1.73472347597681e-18\\
439.01	0\\
440.01	1.73472347597681e-18\\
441.01	1.73472347597681e-18\\
442.01	0\\
443.01	0\\
444.01	0\\
445.01	1.73472347597681e-18\\
446.01	1.73472347597681e-18\\
447.01	0\\
448.01	1.73472347597681e-18\\
449.01	1.73472347597681e-18\\
450.01	0\\
451.01	0\\
452.01	0\\
453.01	1.73472347597681e-18\\
454.01	1.73472347597681e-18\\
455.01	0\\
456.01	1.73472347597681e-18\\
457.01	1.73472347597681e-18\\
458.01	0\\
459.01	1.73472347597681e-18\\
460.01	0\\
461.01	1.73472347597681e-18\\
462.01	0\\
463.01	0\\
464.01	0\\
465.01	1.73472347597681e-18\\
466.01	0\\
467.01	1.73472347597681e-18\\
468.01	0\\
469.01	1.73472347597681e-18\\
470.01	1.73472347597681e-18\\
471.01	1.73472347597681e-18\\
472.01	0\\
473.01	0\\
474.01	0\\
475.01	1.73472347597681e-18\\
476.01	0\\
477.01	0\\
478.01	1.73472347597681e-18\\
479.01	1.73472347597681e-18\\
480.01	0\\
481.01	0\\
482.01	1.73472347597681e-18\\
483.01	1.73472347597681e-18\\
484.01	0\\
485.01	0\\
486.01	0\\
487.01	1.73472347597681e-18\\
488.01	1.73472347597681e-18\\
489.01	1.73472347597681e-18\\
490.01	1.73472347597681e-18\\
491.01	0\\
492.01	1.73472347597681e-18\\
493.01	1.73472347597681e-18\\
494.01	0\\
495.01	1.73472347597681e-18\\
496.01	0\\
497.01	1.73472347597681e-18\\
498.01	1.73472347597681e-18\\
499.01	0\\
500.01	1.73472347597681e-18\\
501.01	1.73472347597681e-18\\
502.01	0\\
503.01	1.73472347597681e-18\\
504.01	1.73472347597681e-18\\
505.01	1.73472347597681e-18\\
506.01	0\\
507.01	0\\
508.01	0\\
509.01	0\\
510.01	1.73472347597681e-18\\
511.01	1.73472347597681e-18\\
512.01	0\\
513.01	1.73472347597681e-18\\
514.01	1.73472347597681e-18\\
515.01	0\\
516.01	0\\
517.01	1.73472347597681e-18\\
518.01	1.73472347597681e-18\\
519.01	1.73472347597681e-18\\
520.01	0\\
521.01	0\\
522.01	0\\
523.01	1.73472347597681e-18\\
524.01	0\\
525.01	1.73472347597681e-18\\
526.01	1.73472347597681e-18\\
527.01	0\\
528.01	0\\
529.01	0\\
530.01	1.73472347597681e-18\\
531.01	0\\
532.01	1.73472347597681e-18\\
533.01	0\\
534.01	1.73472347597681e-18\\
535.01	0\\
536.01	1.73472347597681e-18\\
537.01	1.73472347597681e-18\\
538.01	1.73472347597681e-18\\
539.01	0\\
540.01	0\\
541.01	1.73472347597681e-18\\
542.01	1.73472347597681e-18\\
543.01	0\\
544.01	1.73472347597681e-18\\
545.01	0\\
546.01	1.73472347597681e-18\\
547.01	0\\
548.01	0\\
549.01	0\\
550.01	0\\
551.01	0\\
552.01	0\\
553.01	0\\
554.01	0\\
555.01	1.73472347597681e-18\\
556.01	0\\
557.01	0\\
558.01	0\\
559.01	1.73472347597681e-18\\
560.01	1.73472347597681e-18\\
561.01	1.73472347597681e-18\\
562.01	1.73472347597681e-18\\
563.01	0\\
564.01	0\\
565.01	1.73472347597681e-18\\
566.01	0\\
567.01	0\\
568.01	0\\
569.01	0\\
570.01	0\\
571.01	0\\
572.01	1.73472347597681e-18\\
573.01	0\\
574.01	1.73472347597681e-18\\
575.01	0\\
576.01	1.73472347597681e-18\\
577.01	0\\
578.01	1.73472347597681e-18\\
579.01	0\\
580.01	0\\
581.01	1.73472347597681e-18\\
582.01	0\\
583.01	0\\
584.01	0\\
585.01	0\\
586.01	0\\
587.01	0\\
588.01	0\\
589.01	0\\
590.01	0\\
591.01	0\\
592.01	1.73472347597681e-18\\
593.01	0\\
594.01	0\\
595.01	0\\
596.01	0\\
597.01	0\\
598.01	0\\
599.01	0\\
599.02	0\\
599.03	1.73472347597681e-18\\
599.04	1.73472347597681e-18\\
599.05	0\\
599.06	0\\
599.07	1.73472347597681e-18\\
599.08	0\\
599.09	0\\
599.1	0\\
599.11	1.73472347597681e-18\\
599.12	1.73472347597681e-18\\
599.13	0\\
599.14	0\\
599.15	1.73472347597681e-18\\
599.16	1.73472347597681e-18\\
599.17	1.73472347597681e-18\\
599.18	0\\
599.19	0\\
599.2	0\\
599.21	0\\
599.22	0\\
599.23	1.73472347597681e-18\\
599.24	1.73472347597681e-18\\
599.25	1.73472347597681e-18\\
599.26	1.73472347597681e-18\\
599.27	0\\
599.28	0\\
599.29	0\\
599.3	0\\
599.31	0\\
599.32	1.73472347597681e-18\\
599.33	1.73472347597681e-18\\
599.34	0\\
599.35	0\\
599.36	0\\
599.37	1.73472347597681e-18\\
599.38	0\\
599.39	0\\
599.4	0\\
599.41	0\\
599.42	0\\
599.43	0\\
599.44	0\\
599.45	0\\
599.46	1.73472347597681e-18\\
599.47	1.73472347597681e-18\\
599.48	1.73472347597681e-18\\
599.49	1.73472347597681e-18\\
599.5	1.73472347597681e-18\\
599.51	0\\
599.52	0\\
599.53	1.73472347597681e-18\\
599.54	0\\
599.55	0\\
599.56	0\\
599.57	1.73472347597681e-18\\
599.58	0\\
599.59	0\\
599.6	0\\
599.61	0\\
599.62	1.73472347597681e-18\\
599.63	0\\
599.64	1.73472347597681e-18\\
599.65	0\\
599.66	0\\
599.67	0\\
599.68	0\\
599.69	1.73472347597681e-18\\
599.7	0\\
599.71	1.73472347597681e-18\\
599.72	1.73472347597681e-18\\
599.73	1.73472347597681e-18\\
599.74	0\\
599.75	0\\
599.76	0\\
599.77	0\\
599.78	0\\
599.79	0\\
599.8	0\\
599.81	0\\
599.82	0\\
599.83	1.73472347597681e-18\\
599.84	0\\
599.85	0\\
599.86	0\\
599.87	0\\
599.88	0\\
599.89	0\\
599.9	0\\
599.91	0\\
599.92	0\\
599.93	0\\
599.94	0\\
599.95	0\\
599.96	0\\
599.97	0\\
599.98	0\\
599.99	0\\
600	0\\
};
\addplot [color=blue!50!mycolor7,solid,forget plot]
  table[row sep=crcr]{%
0.01	1.73472347597681e-18\\
1.01	1.73472347597681e-18\\
2.01	0\\
3.01	1.73472347597681e-18\\
4.01	1.73472347597681e-18\\
5.01	0\\
6.01	0\\
7.01	1.73472347597681e-18\\
8.01	0\\
9.01	0\\
10.01	0\\
11.01	0\\
12.01	1.73472347597681e-18\\
13.01	1.73472347597681e-18\\
14.01	1.73472347597681e-18\\
15.01	1.73472347597681e-18\\
16.01	1.73472347597681e-18\\
17.01	1.73472347597681e-18\\
18.01	0\\
19.01	0\\
20.01	1.73472347597681e-18\\
21.01	0\\
22.01	1.73472347597681e-18\\
23.01	1.73472347597681e-18\\
24.01	1.73472347597681e-18\\
25.01	0\\
26.01	1.73472347597681e-18\\
27.01	0\\
28.01	1.73472347597681e-18\\
29.01	1.73472347597681e-18\\
30.01	0\\
31.01	0\\
32.01	1.73472347597681e-18\\
33.01	1.73472347597681e-18\\
34.01	1.73472347597681e-18\\
35.01	0\\
36.01	0\\
37.01	0\\
38.01	0\\
39.01	1.73472347597681e-18\\
40.01	1.73472347597681e-18\\
41.01	1.73472347597681e-18\\
42.01	1.73472347597681e-18\\
43.01	1.73472347597681e-18\\
44.01	0\\
45.01	1.73472347597681e-18\\
46.01	1.73472347597681e-18\\
47.01	1.73472347597681e-18\\
48.01	1.73472347597681e-18\\
49.01	0\\
50.01	1.73472347597681e-18\\
51.01	1.73472347597681e-18\\
52.01	0\\
53.01	0\\
54.01	0\\
55.01	1.73472347597681e-18\\
56.01	0\\
57.01	1.73472347597681e-18\\
58.01	0\\
59.01	0\\
60.01	1.73472347597681e-18\\
61.01	1.73472347597681e-18\\
62.01	1.73472347597681e-18\\
63.01	0\\
64.01	1.73472347597681e-18\\
65.01	0\\
66.01	1.73472347597681e-18\\
67.01	1.73472347597681e-18\\
68.01	0\\
69.01	1.73472347597681e-18\\
70.01	1.73472347597681e-18\\
71.01	0\\
72.01	0\\
73.01	1.73472347597681e-18\\
74.01	0\\
75.01	1.73472347597681e-18\\
76.01	0\\
77.01	0\\
78.01	0\\
79.01	0\\
80.01	1.73472347597681e-18\\
81.01	1.73472347597681e-18\\
82.01	0\\
83.01	0\\
84.01	0\\
85.01	1.73472347597681e-18\\
86.01	1.73472347597681e-18\\
87.01	0\\
88.01	0\\
89.01	0\\
90.01	1.73472347597681e-18\\
91.01	1.73472347597681e-18\\
92.01	0\\
93.01	1.73472347597681e-18\\
94.01	1.73472347597681e-18\\
95.01	0\\
96.01	1.73472347597681e-18\\
97.01	0\\
98.01	0\\
99.01	0\\
100.01	1.73472347597681e-18\\
101.01	0\\
102.01	1.73472347597681e-18\\
103.01	1.73472347597681e-18\\
104.01	0\\
105.01	0\\
106.01	0\\
107.01	0\\
108.01	1.73472347597681e-18\\
109.01	0\\
110.01	1.73472347597681e-18\\
111.01	0\\
112.01	0\\
113.01	0\\
114.01	1.73472347597681e-18\\
115.01	1.73472347597681e-18\\
116.01	0\\
117.01	0\\
118.01	0\\
119.01	0\\
120.01	0\\
121.01	0\\
122.01	0\\
123.01	1.73472347597681e-18\\
124.01	1.73472347597681e-18\\
125.01	1.73472347597681e-18\\
126.01	1.73472347597681e-18\\
127.01	0\\
128.01	1.73472347597681e-18\\
129.01	0\\
130.01	1.73472347597681e-18\\
131.01	1.73472347597681e-18\\
132.01	0\\
133.01	1.73472347597681e-18\\
134.01	1.73472347597681e-18\\
135.01	0\\
136.01	0\\
137.01	1.73472347597681e-18\\
138.01	0\\
139.01	1.73472347597681e-18\\
140.01	1.73472347597681e-18\\
141.01	1.73472347597681e-18\\
142.01	1.73472347597681e-18\\
143.01	0\\
144.01	1.73472347597681e-18\\
145.01	1.73472347597681e-18\\
146.01	1.73472347597681e-18\\
147.01	1.73472347597681e-18\\
148.01	1.73472347597681e-18\\
149.01	0\\
150.01	0\\
151.01	1.73472347597681e-18\\
152.01	1.73472347597681e-18\\
153.01	1.73472347597681e-18\\
154.01	1.73472347597681e-18\\
155.01	1.73472347597681e-18\\
156.01	1.73472347597681e-18\\
157.01	0\\
158.01	1.73472347597681e-18\\
159.01	1.73472347597681e-18\\
160.01	1.73472347597681e-18\\
161.01	1.73472347597681e-18\\
162.01	1.73472347597681e-18\\
163.01	1.73472347597681e-18\\
164.01	0\\
165.01	0\\
166.01	1.73472347597681e-18\\
167.01	0\\
168.01	1.73472347597681e-18\\
169.01	0\\
170.01	1.73472347597681e-18\\
171.01	0\\
172.01	0\\
173.01	0\\
174.01	1.73472347597681e-18\\
175.01	1.73472347597681e-18\\
176.01	1.73472347597681e-18\\
177.01	0\\
178.01	1.73472347597681e-18\\
179.01	0\\
180.01	0\\
181.01	1.73472347597681e-18\\
182.01	0\\
183.01	0\\
184.01	0\\
185.01	1.73472347597681e-18\\
186.01	0\\
187.01	0\\
188.01	0\\
189.01	0\\
190.01	1.73472347597681e-18\\
191.01	0\\
192.01	1.73472347597681e-18\\
193.01	0\\
194.01	1.73472347597681e-18\\
195.01	0\\
196.01	1.73472347597681e-18\\
197.01	0\\
198.01	0\\
199.01	0\\
200.01	0\\
201.01	0\\
202.01	1.73472347597681e-18\\
203.01	0\\
204.01	1.73472347597681e-18\\
205.01	1.73472347597681e-18\\
206.01	1.73472347597681e-18\\
207.01	1.73472347597681e-18\\
208.01	0\\
209.01	1.73472347597681e-18\\
210.01	0\\
211.01	1.73472347597681e-18\\
212.01	0\\
213.01	1.73472347597681e-18\\
214.01	0\\
215.01	0\\
216.01	0\\
217.01	0\\
218.01	1.73472347597681e-18\\
219.01	1.73472347597681e-18\\
220.01	0\\
221.01	0\\
222.01	0\\
223.01	1.73472347597681e-18\\
224.01	1.73472347597681e-18\\
225.01	0\\
226.01	1.73472347597681e-18\\
227.01	0\\
228.01	0\\
229.01	1.73472347597681e-18\\
230.01	0\\
231.01	0\\
232.01	0\\
233.01	0\\
234.01	1.73472347597681e-18\\
235.01	0\\
236.01	1.73472347597681e-18\\
237.01	1.73472347597681e-18\\
238.01	0\\
239.01	1.73472347597681e-18\\
240.01	0\\
241.01	1.73472347597681e-18\\
242.01	0\\
243.01	1.73472347597681e-18\\
244.01	1.73472347597681e-18\\
245.01	1.73472347597681e-18\\
246.01	1.73472347597681e-18\\
247.01	0\\
248.01	0\\
249.01	0\\
250.01	1.73472347597681e-18\\
251.01	0\\
252.01	1.73472347597681e-18\\
253.01	1.73472347597681e-18\\
254.01	1.73472347597681e-18\\
255.01	1.73472347597681e-18\\
256.01	1.73472347597681e-18\\
257.01	1.73472347597681e-18\\
258.01	0\\
259.01	0\\
260.01	0\\
261.01	0\\
262.01	1.73472347597681e-18\\
263.01	1.73472347597681e-18\\
264.01	0\\
265.01	1.73472347597681e-18\\
266.01	1.73472347597681e-18\\
267.01	0\\
268.01	0\\
269.01	1.73472347597681e-18\\
270.01	0\\
271.01	1.73472347597681e-18\\
272.01	0\\
273.01	1.73472347597681e-18\\
274.01	0\\
275.01	0\\
276.01	1.73472347597681e-18\\
277.01	1.73472347597681e-18\\
278.01	0\\
279.01	1.73472347597681e-18\\
280.01	0\\
281.01	1.73472347597681e-18\\
282.01	1.73472347597681e-18\\
283.01	1.73472347597681e-18\\
284.01	0\\
285.01	1.73472347597681e-18\\
286.01	1.73472347597681e-18\\
287.01	1.73472347597681e-18\\
288.01	1.73472347597681e-18\\
289.01	0\\
290.01	0\\
291.01	0\\
292.01	0\\
293.01	1.73472347597681e-18\\
294.01	1.73472347597681e-18\\
295.01	0\\
296.01	1.73472347597681e-18\\
297.01	0\\
298.01	1.73472347597681e-18\\
299.01	0\\
300.01	1.73472347597681e-18\\
301.01	0\\
302.01	0\\
303.01	0\\
304.01	1.73472347597681e-18\\
305.01	1.73472347597681e-18\\
306.01	0\\
307.01	1.73472347597681e-18\\
308.01	0\\
309.01	1.73472347597681e-18\\
310.01	1.73472347597681e-18\\
311.01	0\\
312.01	1.73472347597681e-18\\
313.01	1.73472347597681e-18\\
314.01	0\\
315.01	1.73472347597681e-18\\
316.01	0\\
317.01	1.73472347597681e-18\\
318.01	0\\
319.01	0\\
320.01	0\\
321.01	0\\
322.01	1.73472347597681e-18\\
323.01	1.73472347597681e-18\\
324.01	0\\
325.01	1.73472347597681e-18\\
326.01	1.73472347597681e-18\\
327.01	0\\
328.01	1.73472347597681e-18\\
329.01	0\\
330.01	0\\
331.01	1.73472347597681e-18\\
332.01	1.73472347597681e-18\\
333.01	1.73472347597681e-18\\
334.01	1.73472347597681e-18\\
335.01	1.73472347597681e-18\\
336.01	0\\
337.01	1.73472347597681e-18\\
338.01	0\\
339.01	0\\
340.01	1.73472347597681e-18\\
341.01	1.73472347597681e-18\\
342.01	1.73472347597681e-18\\
343.01	1.73472347597681e-18\\
344.01	0\\
345.01	0\\
346.01	0\\
347.01	1.73472347597681e-18\\
348.01	0\\
349.01	1.73472347597681e-18\\
350.01	0\\
351.01	0\\
352.01	0\\
353.01	1.73472347597681e-18\\
354.01	1.73472347597681e-18\\
355.01	0\\
356.01	0\\
357.01	1.73472347597681e-18\\
358.01	1.73472347597681e-18\\
359.01	1.73472347597681e-18\\
360.01	0\\
361.01	0\\
362.01	0\\
363.01	0\\
364.01	0\\
365.01	1.73472347597681e-18\\
366.01	1.73472347597681e-18\\
367.01	1.73472347597681e-18\\
368.01	1.73472347597681e-18\\
369.01	0\\
370.01	0\\
371.01	1.73472347597681e-18\\
372.01	1.73472347597681e-18\\
373.01	1.73472347597681e-18\\
374.01	0\\
375.01	1.73472347597681e-18\\
376.01	0\\
377.01	0\\
378.01	1.73472347597681e-18\\
379.01	0\\
380.01	1.73472347597681e-18\\
381.01	1.73472347597681e-18\\
382.01	1.73472347597681e-18\\
383.01	0\\
384.01	1.73472347597681e-18\\
385.01	0\\
386.01	1.73472347597681e-18\\
387.01	1.73472347597681e-18\\
388.01	1.73472347597681e-18\\
389.01	0\\
390.01	1.73472347597681e-18\\
391.01	1.73472347597681e-18\\
392.01	0\\
393.01	0\\
394.01	1.73472347597681e-18\\
395.01	1.73472347597681e-18\\
396.01	1.73472347597681e-18\\
397.01	1.73472347597681e-18\\
398.01	0\\
399.01	1.73472347597681e-18\\
400.01	0\\
401.01	0\\
402.01	0\\
403.01	1.73472347597681e-18\\
404.01	1.73472347597681e-18\\
405.01	1.73472347597681e-18\\
406.01	1.73472347597681e-18\\
407.01	1.73472347597681e-18\\
408.01	1.73472347597681e-18\\
409.01	0\\
410.01	1.73472347597681e-18\\
411.01	0\\
412.01	1.73472347597681e-18\\
413.01	1.73472347597681e-18\\
414.01	0\\
415.01	0\\
416.01	1.73472347597681e-18\\
417.01	1.73472347597681e-18\\
418.01	1.73472347597681e-18\\
419.01	1.73472347597681e-18\\
420.01	0\\
421.01	1.73472347597681e-18\\
422.01	1.73472347597681e-18\\
423.01	1.73472347597681e-18\\
424.01	1.73472347597681e-18\\
425.01	0\\
426.01	1.73472347597681e-18\\
427.01	1.73472347597681e-18\\
428.01	1.73472347597681e-18\\
429.01	0\\
430.01	1.73472347597681e-18\\
431.01	1.73472347597681e-18\\
432.01	1.73472347597681e-18\\
433.01	0\\
434.01	1.73472347597681e-18\\
435.01	1.73472347597681e-18\\
436.01	1.73472347597681e-18\\
437.01	1.73472347597681e-18\\
438.01	1.73472347597681e-18\\
439.01	0\\
440.01	1.73472347597681e-18\\
441.01	1.73472347597681e-18\\
442.01	0\\
443.01	0\\
444.01	0\\
445.01	1.73472347597681e-18\\
446.01	1.73472347597681e-18\\
447.01	0\\
448.01	1.73472347597681e-18\\
449.01	1.73472347597681e-18\\
450.01	0\\
451.01	0\\
452.01	0\\
453.01	1.73472347597681e-18\\
454.01	1.73472347597681e-18\\
455.01	0\\
456.01	1.73472347597681e-18\\
457.01	1.73472347597681e-18\\
458.01	0\\
459.01	1.73472347597681e-18\\
460.01	0\\
461.01	1.73472347597681e-18\\
462.01	0\\
463.01	0\\
464.01	0\\
465.01	1.73472347597681e-18\\
466.01	0\\
467.01	1.73472347597681e-18\\
468.01	0\\
469.01	1.73472347597681e-18\\
470.01	1.73472347597681e-18\\
471.01	1.73472347597681e-18\\
472.01	0\\
473.01	0\\
474.01	0\\
475.01	1.73472347597681e-18\\
476.01	0\\
477.01	0\\
478.01	1.73472347597681e-18\\
479.01	1.73472347597681e-18\\
480.01	0\\
481.01	0\\
482.01	1.73472347597681e-18\\
483.01	1.73472347597681e-18\\
484.01	0\\
485.01	0\\
486.01	0\\
487.01	1.73472347597681e-18\\
488.01	1.73472347597681e-18\\
489.01	1.73472347597681e-18\\
490.01	1.73472347597681e-18\\
491.01	0\\
492.01	1.73472347597681e-18\\
493.01	1.73472347597681e-18\\
494.01	0\\
495.01	1.73472347597681e-18\\
496.01	0\\
497.01	1.73472347597681e-18\\
498.01	1.73472347597681e-18\\
499.01	0\\
500.01	1.73472347597681e-18\\
501.01	1.73472347597681e-18\\
502.01	0\\
503.01	1.73472347597681e-18\\
504.01	1.73472347597681e-18\\
505.01	1.73472347597681e-18\\
506.01	0\\
507.01	0\\
508.01	0\\
509.01	0\\
510.01	1.73472347597681e-18\\
511.01	1.73472347597681e-18\\
512.01	0\\
513.01	1.73472347597681e-18\\
514.01	1.73472347597681e-18\\
515.01	0\\
516.01	0\\
517.01	1.73472347597681e-18\\
518.01	1.73472347597681e-18\\
519.01	1.73472347597681e-18\\
520.01	0\\
521.01	0\\
522.01	0\\
523.01	1.73472347597681e-18\\
524.01	0\\
525.01	1.73472347597681e-18\\
526.01	1.73472347597681e-18\\
527.01	0\\
528.01	0\\
529.01	0\\
530.01	1.73472347597681e-18\\
531.01	0\\
532.01	1.73472347597681e-18\\
533.01	0\\
534.01	1.73472347597681e-18\\
535.01	0\\
536.01	1.73472347597681e-18\\
537.01	1.73472347597681e-18\\
538.01	1.73472347597681e-18\\
539.01	0\\
540.01	0\\
541.01	1.73472347597681e-18\\
542.01	1.73472347597681e-18\\
543.01	0\\
544.01	1.73472347597681e-18\\
545.01	0\\
546.01	1.73472347597681e-18\\
547.01	0\\
548.01	0\\
549.01	0\\
550.01	0\\
551.01	0\\
552.01	0\\
553.01	0\\
554.01	0\\
555.01	1.73472347597681e-18\\
556.01	0\\
557.01	0\\
558.01	0\\
559.01	1.73472347597681e-18\\
560.01	1.73472347597681e-18\\
561.01	1.73472347597681e-18\\
562.01	1.73472347597681e-18\\
563.01	0\\
564.01	0\\
565.01	1.73472347597681e-18\\
566.01	0\\
567.01	0\\
568.01	0\\
569.01	0\\
570.01	0\\
571.01	0\\
572.01	1.73472347597681e-18\\
573.01	0\\
574.01	1.73472347597681e-18\\
575.01	0\\
576.01	1.73472347597681e-18\\
577.01	0\\
578.01	1.73472347597681e-18\\
579.01	0\\
580.01	0\\
581.01	1.73472347597681e-18\\
582.01	0\\
583.01	0\\
584.01	0\\
585.01	0\\
586.01	0\\
587.01	0\\
588.01	0\\
589.01	0\\
590.01	0\\
591.01	0\\
592.01	1.73472347597681e-18\\
593.01	0\\
594.01	0\\
595.01	0\\
596.01	0\\
597.01	0\\
598.01	0\\
599.01	0\\
599.02	0\\
599.03	1.73472347597681e-18\\
599.04	1.73472347597681e-18\\
599.05	0\\
599.06	0\\
599.07	1.73472347597681e-18\\
599.08	0\\
599.09	0\\
599.1	0\\
599.11	1.73472347597681e-18\\
599.12	1.73472347597681e-18\\
599.13	0\\
599.14	0\\
599.15	1.73472347597681e-18\\
599.16	1.73472347597681e-18\\
599.17	1.73472347597681e-18\\
599.18	0\\
599.19	0\\
599.2	0\\
599.21	0\\
599.22	0\\
599.23	1.73472347597681e-18\\
599.24	1.73472347597681e-18\\
599.25	1.73472347597681e-18\\
599.26	1.73472347597681e-18\\
599.27	0\\
599.28	0\\
599.29	0\\
599.3	0\\
599.31	0\\
599.32	1.73472347597681e-18\\
599.33	1.73472347597681e-18\\
599.34	0\\
599.35	0\\
599.36	0\\
599.37	1.73472347597681e-18\\
599.38	0\\
599.39	0\\
599.4	0\\
599.41	0\\
599.42	0\\
599.43	0\\
599.44	0\\
599.45	0\\
599.46	1.73472347597681e-18\\
599.47	1.73472347597681e-18\\
599.48	1.73472347597681e-18\\
599.49	1.73472347597681e-18\\
599.5	1.73472347597681e-18\\
599.51	0\\
599.52	0\\
599.53	1.73472347597681e-18\\
599.54	0\\
599.55	0\\
599.56	0\\
599.57	1.73472347597681e-18\\
599.58	0\\
599.59	0\\
599.6	0\\
599.61	0\\
599.62	1.73472347597681e-18\\
599.63	0\\
599.64	1.73472347597681e-18\\
599.65	0\\
599.66	0\\
599.67	0\\
599.68	0\\
599.69	1.73472347597681e-18\\
599.7	0\\
599.71	1.73472347597681e-18\\
599.72	1.73472347597681e-18\\
599.73	1.73472347597681e-18\\
599.74	0\\
599.75	0\\
599.76	0\\
599.77	0\\
599.78	0\\
599.79	0\\
599.8	0\\
599.81	0\\
599.82	0\\
599.83	1.73472347597681e-18\\
599.84	0\\
599.85	0\\
599.86	0\\
599.87	0\\
599.88	0\\
599.89	0\\
599.9	0\\
599.91	0\\
599.92	0\\
599.93	0\\
599.94	0\\
599.95	0\\
599.96	0\\
599.97	0\\
599.98	0\\
599.99	0\\
600	0\\
};
\addplot [color=blue!40!mycolor9,solid,forget plot]
  table[row sep=crcr]{%
0.01	0.000191675496803481\\
1.01	0.000191675330892803\\
2.01	0.000191675161516676\\
3.01	0.000191674988602361\\
4.01	0.00019167481207555\\
5.01	0.000191674631860398\\
6.01	0.000191674447879429\\
7.01	0.000191674260053553\\
8.01	0.00019167406830198\\
9.01	0.000191673872542245\\
10.01	0.000191673672690089\\
11.01	0.000191673468659526\\
12.01	0.000191673260362736\\
13.01	0.000191673047709994\\
14.01	0.000191672830609725\\
15.01	0.000191672608968402\\
16.01	0.000191672382690531\\
17.01	0.00019167215167854\\
18.01	0.000191671915832857\\
19.01	0.000191671675051788\\
20.01	0.000191671429231424\\
21.01	0.000191671178265693\\
22.01	0.000191670922046345\\
23.01	0.000191670660462698\\
24.01	0.000191670393401821\\
25.01	0.000191670120748349\\
26.01	0.000191669842384471\\
27.01	0.00019166955818988\\
28.01	0.000191669268041728\\
29.01	0.000191668971814529\\
30.01	0.000191668669380132\\
31.01	0.000191668360607715\\
32.01	0.000191668045363609\\
33.01	0.000191667723511353\\
34.01	0.000191667394911549\\
35.01	0.000191667059421871\\
36.01	0.000191666716896939\\
37.01	0.000191666367188282\\
38.01	0.000191666010144287\\
39.01	0.000191665645610099\\
40.01	0.000191665273427564\\
41.01	0.000191664893435169\\
42.01	0.000191664505467959\\
43.01	0.000191664109357467\\
44.01	0.000191663704931634\\
45.01	0.000191663292014728\\
46.01	0.000191662870427297\\
47.01	0.00019166243998602\\
48.01	0.000191662000503702\\
49.01	0.000191661551789108\\
50.01	0.000191661093646997\\
51.01	0.00019166062587792\\
52.01	0.000191660148278124\\
53.01	0.000191659660639594\\
54.01	0.000191659162749795\\
55.01	0.000191658654391696\\
56.01	0.000191658135343617\\
57.01	0.000191657605379132\\
58.01	0.000191657064266977\\
59.01	0.000191656511770949\\
60.01	0.000191655947649768\\
61.01	0.000191655371657018\\
62.01	0.000191654783541004\\
63.01	0.000191654183044605\\
64.01	0.00019165356990526\\
65.01	0.000191652943854736\\
66.01	0.000191652304619051\\
67.01	0.000191651651918359\\
68.01	0.000191650985466823\\
69.01	0.000191650304972474\\
70.01	0.000191649610137045\\
71.01	0.000191648900655933\\
72.01	0.000191648176217947\\
73.01	0.000191647436505226\\
74.01	0.000191646681193073\\
75.01	0.000191645909949826\\
76.01	0.000191645122436699\\
77.01	0.000191644318307609\\
78.01	0.000191643497209059\\
79.01	0.000191642658779875\\
80.01	0.000191641802651221\\
81.01	0.000191640928446217\\
82.01	0.000191640035779957\\
83.01	0.000191639124259171\\
84.01	0.000191638193482143\\
85.01	0.000191637243038544\\
86.01	0.00019163627250912\\
87.01	0.000191635281465622\\
88.01	0.000191634269470541\\
89.01	0.000191633236076969\\
90.01	0.000191632180828296\\
91.01	0.000191631103258075\\
92.01	0.000191630002889799\\
93.01	0.000191628879236625\\
94.01	0.000191627731801232\\
95.01	0.000191626560075505\\
96.01	0.000191625363540377\\
97.01	0.000191624141665498\\
98.01	0.000191622893909069\\
99.01	0.000191621619717577\\
100.01	0.000191620318525479\\
101.01	0.000191618989754996\\
102.01	0.000191617632815813\\
103.01	0.000191616247104844\\
104.01	0.000191614832005879\\
105.01	0.000191613386889363\\
106.01	0.000191611911112053\\
107.01	0.000191610404016747\\
108.01	0.000191608864931999\\
109.01	0.000191607293171666\\
110.01	0.000191605688034826\\
111.01	0.000191604048805189\\
112.01	0.00019160237475095\\
113.01	0.000191600665124332\\
114.01	0.000191598919161316\\
115.01	0.000191597136081202\\
116.01	0.000191595315086281\\
117.01	0.000191593455361469\\
118.01	0.000191591556073898\\
119.01	0.000191589616372502\\
120.01	0.00019158763538764\\
121.01	0.000191585612230731\\
122.01	0.000191583545993712\\
123.01	0.000191581435748706\\
124.01	0.000191579280547553\\
125.01	0.00019157707942131\\
126.01	0.000191574831379859\\
127.01	0.000191572535411412\\
128.01	0.000191570190482023\\
129.01	0.000191567795535015\\
130.01	0.000191565349490651\\
131.01	0.000191562851245426\\
132.01	0.000191560299671658\\
133.01	0.000191557693616928\\
134.01	0.000191555031903436\\
135.01	0.000191552313327568\\
136.01	0.000191549536659218\\
137.01	0.000191546700641252\\
138.01	0.000191543803988821\\
139.01	0.000191540845388851\\
140.01	0.000191537823499318\\
141.01	0.000191534736948608\\
142.01	0.000191531584334914\\
143.01	0.000191528364225474\\
144.01	0.000191525075155902\\
145.01	0.000191521715629498\\
146.01	0.000191518284116505\\
147.01	0.000191514779053362\\
148.01	0.000191511198841912\\
149.01	0.000191507541848648\\
150.01	0.000191503806403944\\
151.01	0.000191499990801149\\
152.01	0.00019149609329582\\
153.01	0.000191492112104849\\
154.01	0.000191488045405591\\
155.01	0.000191483891334927\\
156.01	0.000191479647988385\\
157.01	0.000191475313419223\\
158.01	0.000191470885637363\\
159.01	0.00019146636260853\\
160.01	0.00019146174225315\\
161.01	0.000191457022445369\\
162.01	0.000191452201012006\\
163.01	0.000191447275731413\\
164.01	0.000191442244332407\\
165.01	0.000191437104493158\\
166.01	0.000191431853839992\\
167.01	0.000191426489946219\\
168.01	0.000191421010330909\\
169.01	0.000191415412457695\\
170.01	0.000191409693733423\\
171.01	0.00019140385150687\\
172.01	0.000191397883067509\\
173.01	0.000191391785643999\\
174.01	0.000191385556402869\\
175.01	0.000191379192447093\\
176.01	0.000191372690814566\\
177.01	0.000191366048476678\\
178.01	0.000191359262336715\\
179.01	0.000191352329228336\\
180.01	0.000191345245913957\\
181.01	0.000191338009083048\\
182.01	0.000191330615350512\\
183.01	0.000191323061254954\\
184.01	0.000191315343256863\\
185.01	0.000191307457736866\\
186.01	0.000191299400993794\\
187.01	0.000191291169242888\\
188.01	0.000191282758613754\\
189.01	0.000191274165148468\\
190.01	0.000191265384799423\\
191.01	0.000191256413427357\\
192.01	0.000191247246799158\\
193.01	0.000191237880585672\\
194.01	0.000191228310359505\\
195.01	0.00019121853159265\\
196.01	0.000191208539654252\\
197.01	0.000191198329808099\\
198.01	0.000191187897210164\\
199.01	0.000191177236906195\\
200.01	0.000191166343828963\\
201.01	0.000191155212795753\\
202.01	0.000191143838505591\\
203.01	0.000191132215536462\\
204.01	0.000191120338342484\\
205.01	0.000191108201251003\\
206.01	0.000191095798459611\\
207.01	0.000191083124033034\\
208.01	0.000191070171900077\\
209.01	0.00019105693585039\\
210.01	0.00019104340953117\\
211.01	0.000191029586443845\\
212.01	0.000191015459940531\\
213.01	0.000191001023220659\\
214.01	0.000190986269327157\\
215.01	0.000190971191142959\\
216.01	0.000190955781387095\\
217.01	0.000190940032610802\\
218.01	0.000190923937193574\\
219.01	0.000190907487339088\\
220.01	0.000190890675071028\\
221.01	0.00019087349222882\\
222.01	0.000190855930463219\\
223.01	0.000190837981231809\\
224.01	0.000190819635794494\\
225.01	0.000190800885208684\\
226.01	0.000190781720324528\\
227.01	0.000190762131779978\\
228.01	0.000190742109995661\\
229.01	0.000190721645169816\\
230.01	0.000190700727272822\\
231.01	0.000190679346041904\\
232.01	0.000190657490975478\\
233.01	0.000190635151327464\\
234.01	0.000190612316101414\\
235.01	0.000190588974044529\\
236.01	0.00019056511364157\\
237.01	0.000190540723108494\\
238.01	0.000190515790386035\\
239.01	0.000190490303133175\\
240.01	0.000190464248720256\\
241.01	0.000190437614222197\\
242.01	0.000190410386411289\\
243.01	0.000190382551750001\\
244.01	0.000190354096383507\\
245.01	0.000190325006132088\\
246.01	0.000190295266483299\\
247.01	0.000190264862584002\\
248.01	0.000190233779232185\\
249.01	0.000190202000868537\\
250.01	0.000190169511567884\\
251.01	0.000190136295030381\\
252.01	0.000190102334572563\\
253.01	0.000190067613118012\\
254.01	0.000190032113188016\\
255.01	0.000189995816891825\\
256.01	0.000189958705916796\\
257.01	0.000189920761518214\\
258.01	0.000189881964508922\\
259.01	0.000189842295248722\\
260.01	0.000189801733633413\\
261.01	0.000189760259083733\\
262.01	0.000189717850533866\\
263.01	0.000189674486419855\\
264.01	0.000189630144667585\\
265.01	0.000189584802680576\\
266.01	0.000189538437327408\\
267.01	0.000189491024929015\\
268.01	0.000189442541245458\\
269.01	0.000189392961462578\\
270.01	0.000189342260178209\\
271.01	0.00018929041138817\\
272.01	0.000189237388471852\\
273.01	0.000189183164177547\\
274.01	0.000189127710607388\\
275.01	0.000189070999201965\\
276.01	0.000189013000724573\\
277.01	0.000188953685245135\\
278.01	0.000188893022123765\\
279.01	0.000188830979993853\\
280.01	0.000188767526744975\\
281.01	0.000188702629505247\\
282.01	0.000188636254623302\\
283.01	0.000188568367650034\\
284.01	0.000188498933319672\\
285.01	0.000188427915530752\\
286.01	0.000188355277326345\\
287.01	0.000188280980874143\\
288.01	0.000188204987445961\\
289.01	0.000188127257396848\\
290.01	0.000188047750143716\\
291.01	0.000187966424143584\\
292.01	0.000187883236871319\\
293.01	0.000187798144796898\\
294.01	0.00018771110336223\\
295.01	0.000187622066957565\\
296.01	0.000187530988897249\\
297.01	0.000187437821395105\\
298.01	0.000187342515539398\\
299.01	0.00018724502126709\\
300.01	0.000187145287337759\\
301.01	0.000187043261306917\\
302.01	0.000186938889498877\\
303.01	0.000186832116978937\\
304.01	0.000186722887525284\\
305.01	0.000186611143600105\\
306.01	0.000186496826320265\\
307.01	0.000186379875427514\\
308.01	0.000186260229257921\\
309.01	0.000186137824710943\\
310.01	0.000186012597217818\\
311.01	0.000185884480709398\\
312.01	0.0001857534075834\\
313.01	0.000185619308671029\\
314.01	0.00018548211320304\\
315.01	0.000185341748775244\\
316.01	0.000185198141313155\\
317.01	0.000185051215036326\\
318.01	0.000184900892421789\\
319.01	0.000184747094167022\\
320.01	0.000184589739152115\\
321.01	0.000184428744401327\\
322.01	0.000184264025043996\\
323.01	0.000184095494274667\\
324.01	0.000183923063312526\\
325.01	0.000183746641360142\\
326.01	0.000183566135561407\\
327.01	0.000183381450958727\\
328.01	0.000183192490449445\\
329.01	0.000182999154741398\\
330.01	0.000182801342307663\\
331.01	0.000182598949340461\\
332.01	0.000182391869704155\\
333.01	0.000182179994887342\\
334.01	0.000181963213953909\\
335.01	0.000181741413493261\\
336.01	0.000181514477569386\\
337.01	0.000181282287668954\\
338.01	0.000181044722648302\\
339.01	0.00018080165867925\\
340.01	0.000180552969193797\\
341.01	0.000180298524827574\\
342.01	0.000180038193362023\\
343.01	0.000179771839665348\\
344.01	0.000179499325631981\\
345.01	0.000179220510120906\\
346.01	0.000178935248892264\\
347.01	0.000178643394542722\\
348.01	0.000178344796439215\\
349.01	0.000178039300651247\\
350.01	0.000177726749881571\\
351.01	0.000177406983395285\\
352.01	0.000177079836947357\\
353.01	0.000176745142708478\\
354.01	0.000176402729189255\\
355.01	0.0001760524211628\\
356.01	0.000175694039585566\\
357.01	0.000175327401516415\\
358.01	0.000174952320033924\\
359.01	0.00017456860415173\\
360.01	0.000174176058731738\\
361.01	0.000173774484395069\\
362.01	0.000173363677429981\\
363.01	0.00017294342969648\\
364.01	0.000172513528526505\\
365.01	0.000172073756618337\\
366.01	0.000171623891923186\\
367.01	0.000171163707521389\\
368.01	0.000170692971484062\\
369.01	0.000170211446714636\\
370.01	0.000169718890762957\\
371.01	0.000169215055600849\\
372.01	0.000168699687345003\\
373.01	0.000168172525906693\\
374.01	0.0001676333045418\\
375.01	0.000167081749269502\\
376.01	0.000166517578175698\\
377.01	0.000165940501199465\\
378.01	0.000165350222575527\\
379.01	0.000164746422497702\\
380.01	0.000164128763725698\\
381.01	0.000163496908888791\\
382.01	0.000162850511652208\\
383.01	0.000162189216550953\\
384.01	0.000161512658727167\\
385.01	0.000160820463659077\\
386.01	0.000160112246881736\\
387.01	0.000159387613698577\\
388.01	0.00015864615888426\\
389.01	0.000157887466377809\\
390.01	0.00015711110896618\\
391.01	0.0001563166479577\\
392.01	0.000155503632845158\\
393.01	0.000154671600958131\\
394.01	0.000153820077103977\\
395.01	0.00015294857319764\\
396.01	0.00015205658787922\\
397.01	0.000151143606119361\\
398.01	0.000150209098811858\\
399.01	0.000149252522352905\\
400.01	0.000148273318206845\\
401.01	0.000147270912457522\\
402.01	0.00014624471534524\\
403.01	0.000145194120788212\\
404.01	0.000144118505888446\\
405.01	0.000143017230421419\\
406.01	0.000141889636308597\\
407.01	0.00014073504707275\\
408.01	0.000139552767275\\
409.01	0.000138342081933239\\
410.01	0.000137102255921043\\
411.01	0.000135832533346527\\
412.01	0.000134532136910491\\
413.01	0.000133200267242934\\
414.01	0.000131836102217392\\
415.01	0.000130438796242293\\
416.01	0.000129007479528492\\
417.01	0.000127541257332353\\
418.01	0.000126039209173491\\
419.01	0.000124500388026453\\
420.01	0.000122923819485697\\
421.01	0.000121308500902943\\
422.01	0.000119653400496437\\
423.01	0.000117957456431445\\
424.01	0.000116219575871322\\
425.01	0.000114438633999054\\
426.01	0.000112613473008732\\
427.01	0.000110742901067063\\
428.01	0.000108825691245052\\
429.01	0.000106860580420495\\
430.01	0.00010484626815214\\
431.01	0.000102781415526892\\
432.01	0.000100664643982487\\
433.01	9.84945341082743e-05\\
434.01	9.62696244282846e-05\\
435.01	9.39884101717244e-05\\
436.01	9.16493420380481e-05\\
437.01	8.92508249650172e-05\\
438.01	8.67912169117014e-05\\
439.01	8.42688276705365e-05\\
440.01	8.16819177270559e-05\\
441.01	7.90286971904582e-05\\
442.01	7.63073248244862e-05\\
443.01	7.35159072154431e-05\\
444.01	7.06524981237393e-05\\
445.01	6.77150980773252e-05\\
446.01	6.4701654280153e-05\\
447.01	6.16100609273206e-05\\
448.01	5.84381600423397e-05\\
449.01	5.51837429808386e-05\\
450.01	5.18445527828617e-05\\
451.01	4.84182876016714e-05\\
452.01	4.49026054967338e-05\\
453.01	4.12951309523273e-05\\
454.01	3.75934635778053e-05\\
455.01	3.37951895642719e-05\\
456.01	2.98978966246349e-05\\
457.01	2.58991933472629e-05\\
458.01	2.17967342929452e-05\\
459.01	1.75882546981764e-05\\
460.01	1.32716547695307e-05\\
461.01	8.84584227558473e-06\\
462.01	4.32849815507237e-06\\
463.01	2.87754591011061e-07\\
464.01	0\\
465.01	1.73472347597681e-18\\
466.01	0\\
467.01	1.73472347597681e-18\\
468.01	0\\
469.01	1.73472347597681e-18\\
470.01	1.73472347597681e-18\\
471.01	1.73472347597681e-18\\
472.01	0\\
473.01	0\\
474.01	0\\
475.01	1.73472347597681e-18\\
476.01	0\\
477.01	0\\
478.01	1.73472347597681e-18\\
479.01	1.73472347597681e-18\\
480.01	0\\
481.01	0\\
482.01	1.73472347597681e-18\\
483.01	1.73472347597681e-18\\
484.01	0\\
485.01	0\\
486.01	0\\
487.01	1.73472347597681e-18\\
488.01	1.73472347597681e-18\\
489.01	1.73472347597681e-18\\
490.01	1.73472347597681e-18\\
491.01	0\\
492.01	1.73472347597681e-18\\
493.01	1.73472347597681e-18\\
494.01	0\\
495.01	1.73472347597681e-18\\
496.01	0\\
497.01	1.73472347597681e-18\\
498.01	1.73472347597681e-18\\
499.01	0\\
500.01	1.73472347597681e-18\\
501.01	1.73472347597681e-18\\
502.01	0\\
503.01	1.73472347597681e-18\\
504.01	1.73472347597681e-18\\
505.01	1.73472347597681e-18\\
506.01	0\\
507.01	0\\
508.01	0\\
509.01	0\\
510.01	1.73472347597681e-18\\
511.01	1.73472347597681e-18\\
512.01	0\\
513.01	1.73472347597681e-18\\
514.01	1.73472347597681e-18\\
515.01	0\\
516.01	0\\
517.01	1.73472347597681e-18\\
518.01	1.73472347597681e-18\\
519.01	1.73472347597681e-18\\
520.01	0\\
521.01	0\\
522.01	0\\
523.01	1.73472347597681e-18\\
524.01	0\\
525.01	1.73472347597681e-18\\
526.01	1.73472347597681e-18\\
527.01	0\\
528.01	0\\
529.01	0\\
530.01	1.73472347597681e-18\\
531.01	0\\
532.01	1.73472347597681e-18\\
533.01	0\\
534.01	1.73472347597681e-18\\
535.01	0\\
536.01	1.73472347597681e-18\\
537.01	1.73472347597681e-18\\
538.01	1.73472347597681e-18\\
539.01	0\\
540.01	0\\
541.01	1.73472347597681e-18\\
542.01	1.73472347597681e-18\\
543.01	0\\
544.01	1.73472347597681e-18\\
545.01	0\\
546.01	1.73472347597681e-18\\
547.01	0\\
548.01	0\\
549.01	0\\
550.01	0\\
551.01	0\\
552.01	0\\
553.01	0\\
554.01	0\\
555.01	1.73472347597681e-18\\
556.01	0\\
557.01	0\\
558.01	0\\
559.01	1.73472347597681e-18\\
560.01	1.73472347597681e-18\\
561.01	1.73472347597681e-18\\
562.01	1.73472347597681e-18\\
563.01	0\\
564.01	0\\
565.01	1.73472347597681e-18\\
566.01	0\\
567.01	0\\
568.01	0\\
569.01	0\\
570.01	0\\
571.01	0\\
572.01	1.73472347597681e-18\\
573.01	0\\
574.01	1.73472347597681e-18\\
575.01	0\\
576.01	1.73472347597681e-18\\
577.01	0\\
578.01	1.73472347597681e-18\\
579.01	0\\
580.01	0\\
581.01	1.73472347597681e-18\\
582.01	0\\
583.01	0\\
584.01	0\\
585.01	0\\
586.01	0\\
587.01	0\\
588.01	0\\
589.01	0\\
590.01	0\\
591.01	0\\
592.01	1.73472347597681e-18\\
593.01	0\\
594.01	0\\
595.01	0\\
596.01	0\\
597.01	0\\
598.01	0\\
599.01	0\\
599.02	0\\
599.03	1.73472347597681e-18\\
599.04	1.73472347597681e-18\\
599.05	0\\
599.06	0\\
599.07	1.73472347597681e-18\\
599.08	0\\
599.09	0\\
599.1	0\\
599.11	1.73472347597681e-18\\
599.12	1.73472347597681e-18\\
599.13	0\\
599.14	0\\
599.15	1.73472347597681e-18\\
599.16	1.73472347597681e-18\\
599.17	1.73472347597681e-18\\
599.18	0\\
599.19	0\\
599.2	0\\
599.21	0\\
599.22	0\\
599.23	1.73472347597681e-18\\
599.24	1.73472347597681e-18\\
599.25	1.73472347597681e-18\\
599.26	1.73472347597681e-18\\
599.27	0\\
599.28	0\\
599.29	0\\
599.3	0\\
599.31	0\\
599.32	1.73472347597681e-18\\
599.33	1.73472347597681e-18\\
599.34	0\\
599.35	0\\
599.36	0\\
599.37	1.73472347597681e-18\\
599.38	0\\
599.39	0\\
599.4	0\\
599.41	0\\
599.42	0\\
599.43	0\\
599.44	0\\
599.45	0\\
599.46	1.73472347597681e-18\\
599.47	1.73472347597681e-18\\
599.48	1.73472347597681e-18\\
599.49	1.73472347597681e-18\\
599.5	1.73472347597681e-18\\
599.51	0\\
599.52	0\\
599.53	1.73472347597681e-18\\
599.54	0\\
599.55	0\\
599.56	0\\
599.57	1.73472347597681e-18\\
599.58	0\\
599.59	0\\
599.6	0\\
599.61	0\\
599.62	1.73472347597681e-18\\
599.63	0\\
599.64	1.73472347597681e-18\\
599.65	0\\
599.66	0\\
599.67	0\\
599.68	0\\
599.69	1.73472347597681e-18\\
599.7	0\\
599.71	1.73472347597681e-18\\
599.72	1.73472347597681e-18\\
599.73	1.73472347597681e-18\\
599.74	0\\
599.75	0\\
599.76	0\\
599.77	0\\
599.78	0\\
599.79	0\\
599.8	0\\
599.81	0\\
599.82	0\\
599.83	1.73472347597681e-18\\
599.84	0\\
599.85	0\\
599.86	0\\
599.87	0\\
599.88	0\\
599.89	0\\
599.9	0\\
599.91	0\\
599.92	0\\
599.93	0\\
599.94	0\\
599.95	0\\
599.96	0\\
599.97	0\\
599.98	0\\
599.99	0\\
600	0\\
};
\addplot [color=blue!75!mycolor7,solid,forget plot]
  table[row sep=crcr]{%
0.01	0.00078891105654883\\
1.01	0.000788910902421567\\
2.01	0.000788910745062972\\
3.01	0.00078891058440487\\
4.01	0.000788910420377644\\
5.01	0.000788910252910195\\
6.01	0.000788910081929878\\
7.01	0.000788909907362559\\
8.01	0.000788909729132462\\
9.01	0.000788909547162256\\
10.01	0.000788909361372913\\
11.01	0.000788909171683764\\
12.01	0.000788908978012372\\
13.01	0.000788908780274612\\
14.01	0.000788908578384479\\
15.01	0.000788908372254217\\
16.01	0.000788908161794137\\
17.01	0.000788907946912666\\
18.01	0.000788907727516256\\
19.01	0.000788907503509398\\
20.01	0.00078890727479451\\
21.01	0.000788907041271941\\
22.01	0.000788906802839886\\
23.01	0.0007889065593944\\
24.01	0.000788906310829304\\
25.01	0.00078890605703613\\
26.01	0.000788905797904129\\
27.01	0.000788905533320125\\
28.01	0.000788905263168578\\
29.01	0.000788904987331447\\
30.01	0.000788904705688169\\
31.01	0.000788904418115579\\
32.01	0.000788904124487925\\
33.01	0.00078890382467669\\
34.01	0.00078890351855066\\
35.01	0.000788903205975777\\
36.01	0.000788902886815131\\
37.01	0.000788902560928855\\
38.01	0.000788902228174059\\
39.01	0.000788901888404853\\
40.01	0.000788901541472166\\
41.01	0.000788901187223726\\
42.01	0.00078890082550402\\
43.01	0.000788900456154132\\
44.01	0.000788900079011801\\
45.01	0.000788899693911235\\
46.01	0.000788899300683077\\
47.01	0.000788898899154339\\
48.01	0.000788898489148303\\
49.01	0.000788898070484484\\
50.01	0.000788897642978422\\
51.01	0.000788897206441761\\
52.01	0.000788896760682065\\
53.01	0.000788896305502723\\
54.01	0.000788895840702955\\
55.01	0.000788895366077536\\
56.01	0.000788894881416926\\
57.01	0.000788894386507\\
58.01	0.000788893881129028\\
59.01	0.000788893365059553\\
60.01	0.000788892838070274\\
61.01	0.000788892299928\\
62.01	0.000788891750394432\\
63.01	0.000788891189226168\\
64.01	0.000788890616174551\\
65.01	0.000788890030985487\\
66.01	0.00078888943339942\\
67.01	0.000788888823151162\\
68.01	0.00078888819996979\\
69.01	0.000788887563578497\\
70.01	0.000788886913694475\\
71.01	0.000788886250028766\\
72.01	0.000788885572286206\\
73.01	0.000788884880165141\\
74.01	0.000788884173357431\\
75.01	0.000788883451548204\\
76.01	0.000788882714415772\\
77.01	0.000788881961631431\\
78.01	0.000788881192859378\\
79.01	0.000788880407756412\\
80.01	0.000788879605971982\\
81.01	0.000788878787147817\\
82.01	0.000788877950917871\\
83.01	0.000788877096908122\\
84.01	0.00078887622473643\\
85.01	0.000788875334012255\\
86.01	0.00078887442433658\\
87.01	0.000788873495301717\\
88.01	0.000788872546491006\\
89.01	0.000788871577478716\\
90.01	0.000788870587829853\\
91.01	0.000788869577099864\\
92.01	0.0007888685448345\\
93.01	0.000788867490569582\\
94.01	0.000788866413830766\\
95.01	0.000788865314133308\\
96.01	0.000788864190981885\\
97.01	0.000788863043870299\\
98.01	0.000788861872281251\\
99.01	0.000788860675686101\\
100.01	0.000788859453544634\\
101.01	0.000788858205304757\\
102.01	0.000788856930402268\\
103.01	0.000788855628260558\\
104.01	0.000788854298290361\\
105.01	0.000788852939889459\\
106.01	0.000788851552442375\\
107.01	0.000788850135320105\\
108.01	0.00078884868787976\\
109.01	0.000788847209464356\\
110.01	0.000788845699402378\\
111.01	0.000788844157007544\\
112.01	0.000788842581578427\\
113.01	0.000788840972398121\\
114.01	0.000788839328733875\\
115.01	0.000788837649836794\\
116.01	0.000788835934941419\\
117.01	0.000788834183265393\\
118.01	0.000788832394009001\\
119.01	0.000788830566354913\\
120.01	0.000788828699467647\\
121.01	0.000788826792493228\\
122.01	0.000788824844558807\\
123.01	0.000788822854772131\\
124.01	0.00078882082222117\\
125.01	0.000788818745973708\\
126.01	0.000788816625076768\\
127.01	0.000788814458556259\\
128.01	0.000788812245416408\\
129.01	0.000788809984639332\\
130.01	0.00078880767518451\\
131.01	0.000788805315988251\\
132.01	0.000788802905963197\\
133.01	0.000788800443997753\\
134.01	0.000788797928955589\\
135.01	0.000788795359674981\\
136.01	0.000788792734968348\\
137.01	0.000788790053621561\\
138.01	0.000788787314393425\\
139.01	0.000788784516014942\\
140.01	0.000788781657188853\\
141.01	0.000788778736588799\\
142.01	0.000788775752858769\\
143.01	0.000788772704612408\\
144.01	0.000788769590432328\\
145.01	0.000788766408869329\\
146.01	0.000788763158441766\\
147.01	0.00078875983763472\\
148.01	0.000788756444899312\\
149.01	0.000788752978651832\\
150.01	0.000788749437273041\\
151.01	0.000788745819107253\\
152.01	0.000788742122461597\\
153.01	0.000788738345605078\\
154.01	0.000788734486767721\\
155.01	0.000788730544139702\\
156.01	0.000788726515870407\\
157.01	0.000788722400067483\\
158.01	0.000788718194795915\\
159.01	0.000788713898076951\\
160.01	0.000788709507887205\\
161.01	0.000788705022157556\\
162.01	0.00078870043877209\\
163.01	0.000788695755567068\\
164.01	0.000788690970329726\\
165.01	0.000788686080797253\\
166.01	0.000788681084655537\\
167.01	0.000788675979538028\\
168.01	0.000788670763024512\\
169.01	0.000788665432639836\\
170.01	0.000788659985852697\\
171.01	0.000788654420074244\\
172.01	0.000788648732656822\\
173.01	0.000788642920892575\\
174.01	0.000788636982012052\\
175.01	0.000788630913182762\\
176.01	0.000788624711507665\\
177.01	0.000788618374023794\\
178.01	0.000788611897700556\\
179.01	0.000788605279438333\\
180.01	0.000788598516066665\\
181.01	0.000788591604342812\\
182.01	0.000788584540949889\\
183.01	0.000788577322495232\\
184.01	0.000788569945508607\\
185.01	0.000788562406440407\\
186.01	0.00078855470165975\\
187.01	0.00078854682745262\\
188.01	0.000788538780019927\\
189.01	0.000788530555475493\\
190.01	0.00078852214984401\\
191.01	0.000788513559058962\\
192.01	0.000788504778960515\\
193.01	0.000788495805293268\\
194.01	0.000788486633704068\\
195.01	0.000788477259739682\\
196.01	0.000788467678844487\\
197.01	0.000788457886358023\\
198.01	0.000788447877512555\\
199.01	0.000788437647430565\\
200.01	0.000788427191122159\\
201.01	0.000788416503482438\\
202.01	0.00078840557928876\\
203.01	0.000788394413198038\\
204.01	0.00078838299974387\\
205.01	0.000788371333333598\\
206.01	0.000788359408245455\\
207.01	0.000788347218625406\\
208.01	0.00078833475848415\\
209.01	0.000788322021693823\\
210.01	0.000788309001984828\\
211.01	0.000788295692942457\\
212.01	0.000788282088003524\\
213.01	0.000788268180452809\\
214.01	0.000788253963419512\\
215.01	0.000788239429873624\\
216.01	0.000788224572622129\\
217.01	0.000788209384305181\\
218.01	0.000788193857392228\\
219.01	0.000788177984177919\\
220.01	0.000788161756778085\\
221.01	0.000788145167125427\\
222.01	0.00078812820696531\\
223.01	0.00078811086785128\\
224.01	0.000788093141140625\\
225.01	0.000788075017989711\\
226.01	0.000788056489349239\\
227.01	0.000788037545959538\\
228.01	0.000788018178345467\\
229.01	0.00078799837681145\\
230.01	0.000787978131436272\\
231.01	0.000787957432067809\\
232.01	0.000787936268317536\\
233.01	0.000787914629555147\\
234.01	0.000787892504902665\\
235.01	0.000787869883228829\\
236.01	0.000787846753143055\\
237.01	0.000787823102989403\\
238.01	0.000787798920840367\\
239.01	0.000787774194490517\\
240.01	0.000787748911449991\\
241.01	0.00078772305893786\\
242.01	0.00078769662387536\\
243.01	0.000787669592878909\\
244.01	0.000787641952252989\\
245.01	0.000787613687982894\\
246.01	0.000787584785727274\\
247.01	0.000787555230810635\\
248.01	0.000787525008215393\\
249.01	0.000787494102574034\\
250.01	0.000787462498161017\\
251.01	0.000787430178884366\\
252.01	0.000787397128277257\\
253.01	0.000787363329489305\\
254.01	0.000787328765277663\\
255.01	0.000787293417998018\\
256.01	0.000787257269595219\\
257.01	0.000787220301593854\\
258.01	0.000787182495088598\\
259.01	0.000787143830734178\\
260.01	0.00078710428873542\\
261.01	0.000787063848836733\\
262.01	0.000787022490311684\\
263.01	0.000786980191952091\\
264.01	0.00078693693205706\\
265.01	0.000786892688421579\\
266.01	0.000786847438325204\\
267.01	0.000786801158520057\\
268.01	0.000786753825218952\\
269.01	0.000786705414082977\\
270.01	0.00078665590020908\\
271.01	0.000786605258117102\\
272.01	0.00078655346173674\\
273.01	0.000786500484394157\\
274.01	0.000786446298798269\\
275.01	0.000786390877026812\\
276.01	0.000786334190512007\\
277.01	0.000786276210026056\\
278.01	0.000786216905666182\\
279.01	0.00078615624683947\\
280.01	0.000786094202247263\\
281.01	0.000786030739869352\\
282.01	0.000785965826947682\\
283.01	0.000785899429969874\\
284.01	0.000785831514652226\\
285.01	0.000785762045922447\\
286.01	0.000785690987902011\\
287.01	0.000785618303888092\\
288.01	0.000785543956335085\\
289.01	0.00078546790683586\\
290.01	0.000785390116102453\\
291.01	0.000785310543946314\\
292.01	0.000785229149258338\\
293.01	0.000785145889988228\\
294.01	0.00078506072312348\\
295.01	0.000784973604667913\\
296.01	0.00078488448961969\\
297.01	0.000784793331948931\\
298.01	0.000784700084574588\\
299.01	0.000784604699341039\\
300.01	0.000784507126994091\\
301.01	0.000784407317156196\\
302.01	0.000784305218301451\\
303.01	0.000784200777729714\\
304.01	0.000784093941540187\\
305.01	0.000783984654604412\\
306.01	0.000783872860538581\\
307.01	0.000783758501675053\\
308.01	0.000783641519033389\\
309.01	0.000783521852290371\\
310.01	0.000783399439749493\\
311.01	0.00078327421830956\\
312.01	0.000783146123432442\\
313.01	0.00078301508910999\\
314.01	0.000782881047830098\\
315.01	0.000782743930541766\\
316.01	0.000782603666619338\\
317.01	0.000782460183825507\\
318.01	0.000782313408273569\\
319.01	0.0007821632643883\\
320.01	0.000782009674865846\\
321.01	0.00078185256063249\\
322.01	0.000781691840802004\\
323.01	0.000781527432631936\\
324.01	0.000781359251478332\\
325.01	0.00078118721074919\\
326.01	0.000781011221856488\\
327.01	0.00078083119416658\\
328.01	0.000780647034949086\\
329.01	0.000780458649324122\\
330.01	0.000780265940207759\\
331.01	0.000780068808255792\\
332.01	0.000779867151805514\\
333.01	0.000779660866815643\\
334.01	0.000779449846804213\\
335.01	0.000779233982784356\\
336.01	0.000779013163197848\\
337.01	0.000778787273846523\\
338.01	0.000778556197821188\\
339.01	0.00077831981542823\\
340.01	0.00077807800411356\\
341.01	0.000777830638384129\\
342.01	0.000777577589726542\\
343.01	0.000777318726522981\\
344.01	0.000777053913964357\\
345.01	0.000776783013960237\\
346.01	0.000776505885045875\\
347.01	0.000776222382286187\\
348.01	0.000775932357176179\\
349.01	0.000775635657538421\\
350.01	0.000775332127416773\\
351.01	0.00077502160696691\\
352.01	0.000774703932343029\\
353.01	0.000774378935581061\\
354.01	0.000774046444478005\\
355.01	0.000773706282467592\\
356.01	0.000773358268491885\\
357.01	0.000773002216868957\\
358.01	0.000772637937156579\\
359.01	0.000772265234011556\\
360.01	0.00077188390704499\\
361.01	0.000771493750673131\\
362.01	0.000771094553963665\\
363.01	0.00077068610047761\\
364.01	0.000770268168106261\\
365.01	0.000769840528903462\\
366.01	0.00076940294891264\\
367.01	0.000768955187988649\\
368.01	0.000768496999613977\\
369.01	0.000768028130709331\\
370.01	0.000767548321437874\\
371.01	0.000767057305003109\\
372.01	0.000766554807439862\\
373.01	0.000766040547398124\\
374.01	0.000765514235919773\\
375.01	0.000764975576211037\\
376.01	0.000764424263447626\\
377.01	0.000763859985052047\\
378.01	0.000763282426039914\\
379.01	0.000762691257241056\\
380.01	0.000762086116124052\\
381.01	0.00076146665513232\\
382.01	0.000760832518163562\\
383.01	0.000760183339841269\\
384.01	0.00075951874526531\\
385.01	0.000758838349755616\\
386.01	0.000758141758588808\\
387.01	0.000757428566727588\\
388.01	0.000756698358542631\\
389.01	0.000755950707526735\\
390.01	0.000755185176001079\\
391.01	0.000754401314813273\\
392.01	0.000753598663026918\\
393.01	0.000752776747602541\\
394.01	0.000751935083069472\\
395.01	0.00075107317118832\\
396.01	0.000750190500604151\\
397.01	0.000749286546489444\\
398.01	0.000748360770176862\\
399.01	0.000747412618781562\\
400.01	0.000746441524812328\\
401.01	0.000745446905771377\\
402.01	0.000744428163742374\\
403.01	0.000743384684966221\\
404.01	0.000742315839404125\\
405.01	0.000741220980287475\\
406.01	0.000740099443653964\\
407.01	0.000738950547869563\\
408.01	0.000737773593135567\\
409.01	0.000736567860980265\\
410.01	0.000735332613734411\\
411.01	0.0007340670939901\\
412.01	0.000732770524041962\\
413.01	0.000731442105310129\\
414.01	0.000730081017744163\\
415.01	0.000728686419206944\\
416.01	0.00072725744483775\\
417.01	0.000725793206393379\\
418.01	0.000724292791566437\\
419.01	0.00072275526327968\\
420.01	0.000721179658955065\\
421.01	0.000719564989756555\\
422.01	0.000717910239805197\\
423.01	0.000716214365365056\\
424.01	0.000714476293998756\\
425.01	0.00071269492369091\\
426.01	0.000710869121937891\\
427.01	0.000708997724802247\\
428.01	0.000707079535929895\\
429.01	0.000705113325528412\\
430.01	0.000703097829304186\\
431.01	0.00070103174735661\\
432.01	0.00069891374302713\\
433.01	0.000696742441700702\\
434.01	0.000694516429557755\\
435.01	0.000692234252274012\\
436.01	0.000689894413665718\\
437.01	0.000687495374278013\\
438.01	0.000685035549913665\\
439.01	0.000682513310099797\\
440.01	0.000679926976489867\\
441.01	0.000677274821198443\\
442.01	0.00067455506506625\\
443.01	0.000671765875852692\\
444.01	0.00066890536635388\\
445.01	0.000665971592443181\\
446.01	0.000662962551032277\\
447.01	0.000659876177950198\\
448.01	0.000656710345737976\\
449.01	0.000653462861356197\\
450.01	0.000650131463802535\\
451.01	0.000646713821635564\\
452.01	0.000643207530400097\\
453.01	0.000639610109947413\\
454.01	0.000635919001640934\\
455.01	0.000632131565433734\\
456.01	0.000628245076798572\\
457.01	0.000624256723488157\\
458.01	0.00062016360216313\\
459.01	0.000615962715856013\\
460.01	0.000611650986581258\\
461.01	0.000607225500882805\\
462.01	0.000602687385822366\\
463.01	0.000598088340347467\\
464.01	0.000593484151361715\\
465.01	0.000588754218462854\\
466.01	0.000583874643218333\\
467.01	0.000578921336074694\\
468.01	0.000573954233032756\\
469.01	0.000568843481288115\\
470.01	0.000563592175141936\\
471.01	0.000558201097387594\\
472.01	0.000552647359145746\\
473.01	0.000546896934269207\\
474.01	0.000540740339582005\\
475.01	0.000535169185104023\\
476.01	0.000529759474592844\\
477.01	0.000524208045358894\\
478.01	0.000518510722889253\\
479.01	0.00051266320239155\\
480.01	0.000506661047067395\\
481.01	0.00050049968717973\\
482.01	0.000494174420170959\\
483.01	0.000487680412158522\\
484.01	0.000481012701221703\\
485.01	0.000474166203010365\\
486.01	0.000467135719422815\\
487.01	0.00045991595217851\\
488.01	0.000452501535153515\\
489.01	0.00044488726797943\\
490.01	0.000437070164910716\\
491.01	0.000429044529699786\\
492.01	0.000420809292186642\\
493.01	0.000412343419361179\\
494.01	0.000403634904553368\\
495.01	0.000394677209356745\\
496.01	0.000385464091523431\\
497.01	0.000375989603330807\\
498.01	0.000366245969979332\\
499.01	0.000356221230692162\\
500.01	0.000345899599529418\\
501.01	0.000335260647672983\\
502.01	0.000324344773929204\\
503.01	0.000313173748962269\\
504.01	0.000301649210040845\\
505.01	0.000289751300194941\\
506.01	0.000277458978449925\\
507.01	0.000264749128852363\\
508.01	0.000251596268624513\\
509.01	0.00023797220579236\\
510.01	0.000223845635329447\\
511.01	0.000209181653300443\\
512.01	0.00019394110106879\\
513.01	0.000178079053299222\\
514.01	0.000161536394333217\\
515.01	0.000144182427327151\\
516.01	0.000125783370083116\\
517.01	0.000107718309217698\\
518.01	8.36312000147641e-05\\
519.01	3.99456113915325e-05\\
520.01	1.85314525037901e-06\\
521.01	0\\
522.01	0\\
523.01	1.73472347597681e-18\\
524.01	0\\
525.01	1.73472347597681e-18\\
526.01	1.73472347597681e-18\\
527.01	0\\
528.01	0\\
529.01	0\\
530.01	1.73472347597681e-18\\
531.01	0\\
532.01	1.73472347597681e-18\\
533.01	0\\
534.01	1.73472347597681e-18\\
535.01	0\\
536.01	1.73472347597681e-18\\
537.01	1.73472347597681e-18\\
538.01	1.73472347597681e-18\\
539.01	0\\
540.01	0\\
541.01	1.73472347597681e-18\\
542.01	1.73472347597681e-18\\
543.01	0\\
544.01	1.73472347597681e-18\\
545.01	0\\
546.01	1.73472347597681e-18\\
547.01	0\\
548.01	0\\
549.01	0\\
550.01	0\\
551.01	0\\
552.01	0\\
553.01	0\\
554.01	0\\
555.01	1.73472347597681e-18\\
556.01	0\\
557.01	0\\
558.01	0\\
559.01	1.73472347597681e-18\\
560.01	1.73472347597681e-18\\
561.01	1.73472347597681e-18\\
562.01	1.73472347597681e-18\\
563.01	0\\
564.01	0\\
565.01	1.73472347597681e-18\\
566.01	0\\
567.01	0\\
568.01	0\\
569.01	0\\
570.01	0\\
571.01	0\\
572.01	1.73472347597681e-18\\
573.01	0\\
574.01	1.73472347597681e-18\\
575.01	0\\
576.01	1.73472347597681e-18\\
577.01	0\\
578.01	1.73472347597681e-18\\
579.01	0\\
580.01	0\\
581.01	1.73472347597681e-18\\
582.01	0\\
583.01	0\\
584.01	0\\
585.01	0\\
586.01	0\\
587.01	0\\
588.01	0\\
589.01	0\\
590.01	0\\
591.01	0\\
592.01	1.73472347597681e-18\\
593.01	0\\
594.01	0\\
595.01	0\\
596.01	0\\
597.01	0\\
598.01	0\\
599.01	0\\
599.02	0\\
599.03	1.73472347597681e-18\\
599.04	1.73472347597681e-18\\
599.05	0\\
599.06	0\\
599.07	1.73472347597681e-18\\
599.08	0\\
599.09	0\\
599.1	0\\
599.11	1.73472347597681e-18\\
599.12	1.73472347597681e-18\\
599.13	0\\
599.14	0\\
599.15	1.73472347597681e-18\\
599.16	1.73472347597681e-18\\
599.17	1.73472347597681e-18\\
599.18	0\\
599.19	0\\
599.2	0\\
599.21	0\\
599.22	0\\
599.23	1.73472347597681e-18\\
599.24	1.73472347597681e-18\\
599.25	1.73472347597681e-18\\
599.26	1.73472347597681e-18\\
599.27	0\\
599.28	0\\
599.29	0\\
599.3	0\\
599.31	0\\
599.32	1.73472347597681e-18\\
599.33	1.73472347597681e-18\\
599.34	0\\
599.35	0\\
599.36	0\\
599.37	1.73472347597681e-18\\
599.38	0\\
599.39	0\\
599.4	0\\
599.41	0\\
599.42	0\\
599.43	0\\
599.44	0\\
599.45	0\\
599.46	1.73472347597681e-18\\
599.47	1.73472347597681e-18\\
599.48	1.73472347597681e-18\\
599.49	1.73472347597681e-18\\
599.5	1.73472347597681e-18\\
599.51	0\\
599.52	0\\
599.53	1.73472347597681e-18\\
599.54	0\\
599.55	0\\
599.56	0\\
599.57	1.73472347597681e-18\\
599.58	0\\
599.59	0\\
599.6	0\\
599.61	0\\
599.62	1.73472347597681e-18\\
599.63	0\\
599.64	1.73472347597681e-18\\
599.65	0\\
599.66	0\\
599.67	0\\
599.68	0\\
599.69	1.73472347597681e-18\\
599.7	0\\
599.71	1.73472347597681e-18\\
599.72	1.73472347597681e-18\\
599.73	1.73472347597681e-18\\
599.74	0\\
599.75	0\\
599.76	0\\
599.77	0\\
599.78	0\\
599.79	0\\
599.8	0\\
599.81	0\\
599.82	0\\
599.83	1.73472347597681e-18\\
599.84	0\\
599.85	0\\
599.86	0\\
599.87	0\\
599.88	0\\
599.89	0\\
599.9	0\\
599.91	0\\
599.92	0\\
599.93	0\\
599.94	0\\
599.95	0\\
599.96	0\\
599.97	0\\
599.98	0\\
599.99	0\\
600	0\\
};
\addplot [color=blue!80!mycolor9,solid,forget plot]
  table[row sep=crcr]{%
0.01	0.00148667125888009\\
1.01	0.00148667120328983\\
2.01	0.00148667114653197\\
3.01	0.00148667108858188\\
4.01	0.00148667102941426\\
5.01	0.00148667096900342\\
6.01	0.00148667090732303\\
7.01	0.0014866708443462\\
8.01	0.00148667078004551\\
9.01	0.00148667071439291\\
10.01	0.0014866706473598\\
11.01	0.00148667057891692\\
12.01	0.00148667050903441\\
13.01	0.00148667043768172\\
14.01	0.00148667036482778\\
15.01	0.00148667029044068\\
16.01	0.00148667021448795\\
17.01	0.00148667013693641\\
18.01	0.0014866700577521\\
19.01	0.00148666997690039\\
20.01	0.00148666989434592\\
21.01	0.00148666981005255\\
22.01	0.00148666972398332\\
23.01	0.00148666963610057\\
24.01	0.00148666954636574\\
25.01	0.00148666945473954\\
26.01	0.00148666936118173\\
27.01	0.00148666926565128\\
28.01	0.00148666916810626\\
29.01	0.00148666906850383\\
30.01	0.00148666896680022\\
31.01	0.00148666886295076\\
32.01	0.00148666875690977\\
33.01	0.00148666864863063\\
34.01	0.00148666853806568\\
35.01	0.00148666842516626\\
36.01	0.00148666830988267\\
37.01	0.00148666819216408\\
38.01	0.00148666807195865\\
39.01	0.00148666794921335\\
40.01	0.00148666782387405\\
41.01	0.00148666769588546\\
42.01	0.00148666756519102\\
43.01	0.00148666743173307\\
44.01	0.0014866672954526\\
45.01	0.00148666715628939\\
46.01	0.00148666701418188\\
47.01	0.00148666686906721\\
48.01	0.00148666672088115\\
49.01	0.00148666656955804\\
50.01	0.00148666641503089\\
51.01	0.00148666625723117\\
52.01	0.00148666609608892\\
53.01	0.00148666593153265\\
54.01	0.00148666576348933\\
55.01	0.00148666559188432\\
56.01	0.00148666541664142\\
57.01	0.00148666523768268\\
58.01	0.00148666505492855\\
59.01	0.00148666486829774\\
60.01	0.00148666467770715\\
61.01	0.00148666448307192\\
62.01	0.00148666428430532\\
63.01	0.00148666408131874\\
64.01	0.00148666387402166\\
65.01	0.00148666366232158\\
66.01	0.00148666344612398\\
67.01	0.0014866632253323\\
68.01	0.00148666299984784\\
69.01	0.00148666276956983\\
70.01	0.00148666253439522\\
71.01	0.00148666229421875\\
72.01	0.00148666204893285\\
73.01	0.00148666179842765\\
74.01	0.00148666154259082\\
75.01	0.00148666128130761\\
76.01	0.00148666101446077\\
77.01	0.00148666074193048\\
78.01	0.00148666046359427\\
79.01	0.00148666017932705\\
80.01	0.00148665988900098\\
81.01	0.00148665959248535\\
82.01	0.00148665928964671\\
83.01	0.00148665898034858\\
84.01	0.00148665866445154\\
85.01	0.00148665834181312\\
86.01	0.00148665801228771\\
87.01	0.0014866576757265\\
88.01	0.00148665733197743\\
89.01	0.0014866569808851\\
90.01	0.00148665662229066\\
91.01	0.00148665625603183\\
92.01	0.00148665588194273\\
93.01	0.00148665549985383\\
94.01	0.00148665510959186\\
95.01	0.00148665471097977\\
96.01	0.00148665430383658\\
97.01	0.0014866538879773\\
98.01	0.00148665346321296\\
99.01	0.00148665302935031\\
100.01	0.00148665258619189\\
101.01	0.00148665213353589\\
102.01	0.00148665167117602\\
103.01	0.00148665119890144\\
104.01	0.00148665071649666\\
105.01	0.00148665022374141\\
106.01	0.00148664972041055\\
107.01	0.00148664920627398\\
108.01	0.00148664868109645\\
109.01	0.00148664814463758\\
110.01	0.00148664759665158\\
111.01	0.00148664703688726\\
112.01	0.00148664646508784\\
113.01	0.00148664588099085\\
114.01	0.001486645284328\\
115.01	0.00148664467482499\\
116.01	0.0014866440522015\\
117.01	0.0014866434161709\\
118.01	0.00148664276644025\\
119.01	0.00148664210271004\\
120.01	0.00148664142467412\\
121.01	0.00148664073201952\\
122.01	0.00148664002442625\\
123.01	0.00148663930156723\\
124.01	0.00148663856310809\\
125.01	0.00148663780870695\\
126.01	0.00148663703801433\\
127.01	0.00148663625067292\\
128.01	0.00148663544631744\\
129.01	0.00148663462457442\\
130.01	0.00148663378506205\\
131.01	0.00148663292738998\\
132.01	0.0014866320511591\\
133.01	0.00148663115596133\\
134.01	0.00148663024137949\\
135.01	0.00148662930698705\\
136.01	0.0014866283523478\\
137.01	0.00148662737701587\\
138.01	0.00148662638053528\\
139.01	0.00148662536243987\\
140.01	0.00148662432225292\\
141.01	0.00148662325948703\\
142.01	0.00148662217364386\\
143.01	0.00148662106421376\\
144.01	0.00148661993067566\\
145.01	0.00148661877249675\\
146.01	0.00148661758913215\\
147.01	0.0014866163800247\\
148.01	0.00148661514460468\\
149.01	0.00148661388228951\\
150.01	0.00148661259248339\\
151.01	0.0014866112745771\\
152.01	0.00148660992794762\\
153.01	0.00148660855195781\\
154.01	0.00148660714595611\\
155.01	0.00148660570927622\\
156.01	0.00148660424123671\\
157.01	0.00148660274114071\\
158.01	0.00148660120827551\\
159.01	0.00148659964191226\\
160.01	0.00148659804130553\\
161.01	0.00148659640569296\\
162.01	0.00148659473429484\\
163.01	0.00148659302631375\\
164.01	0.00148659128093412\\
165.01	0.00148658949732178\\
166.01	0.0014865876746236\\
167.01	0.00148658581196699\\
168.01	0.00148658390845943\\
169.01	0.0014865819631881\\
170.01	0.00148657997521934\\
171.01	0.00148657794359812\\
172.01	0.00148657586734763\\
173.01	0.00148657374546878\\
174.01	0.00148657157693952\\
175.01	0.00148656936071453\\
176.01	0.00148656709572455\\
177.01	0.00148656478087578\\
178.01	0.00148656241504947\\
179.01	0.0014865599971011\\
180.01	0.00148655752586004\\
181.01	0.00148655500012872\\
182.01	0.00148655241868216\\
183.01	0.00148654978026723\\
184.01	0.00148654708360201\\
185.01	0.00148654432737518\\
186.01	0.00148654151024523\\
187.01	0.00148653863083984\\
188.01	0.00148653568775512\\
189.01	0.00148653267955487\\
190.01	0.00148652960476984\\
191.01	0.00148652646189695\\
192.01	0.00148652324939849\\
193.01	0.00148651996570133\\
194.01	0.00148651660919605\\
195.01	0.00148651317823616\\
196.01	0.00148650967113715\\
197.01	0.00148650608617565\\
198.01	0.00148650242158854\\
199.01	0.00148649867557194\\
200.01	0.00148649484628034\\
201.01	0.00148649093182559\\
202.01	0.00148648693027587\\
203.01	0.00148648283965474\\
204.01	0.00148647865794004\\
205.01	0.00148647438306282\\
206.01	0.00148647001290629\\
207.01	0.00148646554530464\\
208.01	0.00148646097804195\\
209.01	0.001486456308851\\
210.01	0.00148645153541199\\
211.01	0.00148644665535152\\
212.01	0.00148644166624103\\
213.01	0.00148643656559584\\
214.01	0.00148643135087357\\
215.01	0.00148642601947294\\
216.01	0.00148642056873238\\
217.01	0.00148641499592855\\
218.01	0.00148640929827498\\
219.01	0.00148640347292058\\
220.01	0.00148639751694806\\
221.01	0.00148639142737249\\
222.01	0.00148638520113966\\
223.01	0.00148637883512451\\
224.01	0.00148637232612942\\
225.01	0.00148636567088258\\
226.01	0.00148635886603625\\
227.01	0.00148635190816494\\
228.01	0.00148634479376369\\
229.01	0.00148633751924618\\
230.01	0.00148633008094285\\
231.01	0.00148632247509892\\
232.01	0.00148631469787253\\
233.01	0.0014863067453326\\
234.01	0.00148629861345685\\
235.01	0.00148629029812964\\
236.01	0.00148628179513982\\
237.01	0.00148627310017856\\
238.01	0.00148626420883703\\
239.01	0.00148625511660417\\
240.01	0.00148624581886427\\
241.01	0.00148623631089461\\
242.01	0.00148622658786293\\
243.01	0.00148621664482497\\
244.01	0.0014862064767219\\
245.01	0.00148619607837764\\
246.01	0.0014861854444962\\
247.01	0.00148617456965896\\
248.01	0.00148616344832181\\
249.01	0.00148615207481229\\
250.01	0.00148614044332668\\
251.01	0.00148612854792698\\
252.01	0.00148611638253788\\
253.01	0.0014861039409436\\
254.01	0.00148609121678466\\
255.01	0.00148607820355467\\
256.01	0.00148606489459698\\
257.01	0.00148605128310123\\
258.01	0.00148603736209989\\
259.01	0.00148602312446468\\
260.01	0.00148600856290296\\
261.01	0.00148599366995398\\
262.01	0.00148597843798509\\
263.01	0.0014859628591878\\
264.01	0.00148594692557395\\
265.01	0.00148593062897149\\
266.01	0.00148591396102048\\
267.01	0.00148589691316872\\
268.01	0.00148587947666754\\
269.01	0.00148586164256732\\
270.01	0.00148584340171296\\
271.01	0.00148582474473932\\
272.01	0.00148580566206644\\
273.01	0.00148578614389478\\
274.01	0.00148576618020022\\
275.01	0.00148574576072908\\
276.01	0.00148572487499295\\
277.01	0.00148570351226347\\
278.01	0.00148568166156683\\
279.01	0.00148565931167841\\
280.01	0.00148563645111707\\
281.01	0.00148561306813944\\
282.01	0.00148558915073401\\
283.01	0.00148556468661512\\
284.01	0.00148553966321682\\
285.01	0.00148551406768654\\
286.01	0.00148548788687875\\
287.01	0.00148546110734816\\
288.01	0.0014854337153432\\
289.01	0.00148540569679897\\
290.01	0.00148537703733022\\
291.01	0.00148534772222409\\
292.01	0.0014853177364327\\
293.01	0.0014852870645655\\
294.01	0.00148525569088158\\
295.01	0.0014852235992816\\
296.01	0.00148519077329961\\
297.01	0.0014851571960947\\
298.01	0.00148512285044237\\
299.01	0.00148508771872568\\
300.01	0.00148505178292618\\
301.01	0.00148501502461468\\
302.01	0.00148497742494158\\
303.01	0.00148493896462712\\
304.01	0.00148489962395129\\
305.01	0.00148485938274348\\
306.01	0.00148481822037178\\
307.01	0.00148477611573207\\
308.01	0.00148473304723673\\
309.01	0.00148468899280306\\
310.01	0.00148464392984129\\
311.01	0.00148459783524234\\
312.01	0.00148455068536518\\
313.01	0.00148450245602375\\
314.01	0.00148445312247352\\
315.01	0.00148440265939768\\
316.01	0.00148435104089279\\
317.01	0.00148429824045417\\
318.01	0.00148424423096053\\
319.01	0.00148418898465847\\
320.01	0.00148413247314618\\
321.01	0.00148407466735677\\
322.01	0.00148401553754105\\
323.01	0.00148395505324966\\
324.01	0.00148389318331468\\
325.01	0.00148382989583073\\
326.01	0.00148376515813515\\
327.01	0.00148369893678785\\
328.01	0.00148363119755033\\
329.01	0.00148356190536394\\
330.01	0.00148349102432757\\
331.01	0.00148341851767448\\
332.01	0.00148334434774837\\
333.01	0.00148326847597874\\
334.01	0.00148319086285529\\
335.01	0.00148311146790156\\
336.01	0.00148303024964776\\
337.01	0.00148294716560248\\
338.01	0.00148286217222378\\
339.01	0.00148277522488909\\
340.01	0.00148268627786429\\
341.01	0.00148259528427165\\
342.01	0.00148250219605695\\
343.01	0.00148240696395542\\
344.01	0.00148230953745664\\
345.01	0.0014822098647684\\
346.01	0.00148210789277943\\
347.01	0.00148200356702099\\
348.01	0.00148189683162741\\
349.01	0.00148178762929527\\
350.01	0.00148167590124167\\
351.01	0.00148156158716107\\
352.01	0.00148144462518104\\
353.01	0.0014813249518167\\
354.01	0.00148120250192407\\
355.01	0.00148107720865193\\
356.01	0.00148094900339259\\
357.01	0.0014808178157313\\
358.01	0.00148068357339438\\
359.01	0.00148054620219602\\
360.01	0.00148040562598376\\
361.01	0.00148026176658271\\
362.01	0.00148011454373837\\
363.01	0.00147996387505812\\
364.01	0.00147980967595145\\
365.01	0.0014796518595688\\
366.01	0.00147949033673916\\
367.01	0.00147932501590623\\
368.01	0.00147915580306354\\
369.01	0.00147898260168817\\
370.01	0.0014788053126732\\
371.01	0.00147862383425929\\
372.01	0.00147843806196481\\
373.01	0.00147824788851521\\
374.01	0.00147805320377142\\
375.01	0.00147785389465882\\
376.01	0.00147764984510826\\
377.01	0.0014774409361123\\
378.01	0.00147722704622314\\
379.01	0.0014770080491819\\
380.01	0.00147678381519769\\
381.01	0.00147655421277934\\
382.01	0.00147631910696703\\
383.01	0.00147607835922564\\
384.01	0.00147583182734266\\
385.01	0.00147557936532291\\
386.01	0.00147532082327951\\
387.01	0.00147505604732124\\
388.01	0.00147478487943591\\
389.01	0.00147450715736971\\
390.01	0.00147422271450212\\
391.01	0.00147393137971652\\
392.01	0.00147363297726579\\
393.01	0.00147332732663323\\
394.01	0.00147301424238788\\
395.01	0.00147269353403461\\
396.01	0.00147236500585822\\
397.01	0.00147202845676139\\
398.01	0.0014716836800962\\
399.01	0.00147133046348868\\
400.01	0.00147096858865618\\
401.01	0.00147059783121699\\
402.01	0.00147021796049185\\
403.01	0.00146982873929673\\
404.01	0.00146942992372654\\
405.01	0.00146902126292897\\
406.01	0.0014686024988681\\
407.01	0.00146817336607682\\
408.01	0.00146773359139768\\
409.01	0.00146728289371101\\
410.01	0.00146682098364997\\
411.01	0.00146634756330103\\
412.01	0.00146586232588942\\
413.01	0.00146536495544833\\
414.01	0.00146485512647047\\
415.01	0.00146433250354117\\
416.01	0.00146379674095133\\
417.01	0.00146324748228891\\
418.01	0.00146268436000735\\
419.01	0.00146210699496918\\
420.01	0.00146151499596284\\
421.01	0.00146090795919073\\
422.01	0.00146028546772621\\
423.01	0.00145964709093706\\
424.01	0.00145899238387248\\
425.01	0.001458320886611\\
426.01	0.00145763212356564\\
427.01	0.00145692560274273\\
428.01	0.00145620081495063\\
429.01	0.00145545723295329\\
430.01	0.00145469431056425\\
431.01	0.0014539114816752\\
432.01	0.00145310815921283\\
433.01	0.00145228373401726\\
434.01	0.00145143757363426\\
435.01	0.00145056902101202\\
436.01	0.00144967739309335\\
437.01	0.00144876197929147\\
438.01	0.00144782203983707\\
439.01	0.00144685680398216\\
440.01	0.00144586546804417\\
441.01	0.00144484719327167\\
442.01	0.00144380110351006\\
443.01	0.00144272628264251\\
444.01	0.00144162177177731\\
445.01	0.00144048656614887\\
446.01	0.00143931961169353\\
447.01	0.00143811980125565\\
448.01	0.00143688597037131\\
449.01	0.00143561689256835\\
450.01	0.00143431127411032\\
451.01	0.00143296774809906\\
452.01	0.00143158486783438\\
453.01	0.00143016109931075\\
454.01	0.00142869481270724\\
455.01	0.00142718427269863\\
456.01	0.00142562762738081\\
457.01	0.00142402289556433\\
458.01	0.00142236795216188\\
459.01	0.0014206605116214\\
460.01	0.00141889811261707\\
461.01	0.00141707814489417\\
462.01	0.00141519830956914\\
463.01	0.00141325871671744\\
464.01	0.00141125511419306\\
465.01	0.00140917775241511\\
466.01	0.00140700083786376\\
467.01	0.00140473385456332\\
468.01	0.00140266671023066\\
469.01	0.00140051700651366\\
470.01	0.00139828407406295\\
471.01	0.00139598936318125\\
472.01	0.00139359556751587\\
473.01	0.00139107893837405\\
474.01	0.00138819797988816\\
475.01	0.00138318856130055\\
476.01	0.00137765691238938\\
477.01	0.00137198396486754\\
478.01	0.00136616572388151\\
479.01	0.00136019805855384\\
480.01	0.00135407669562829\\
481.01	0.00134779721267852\\
482.01	0.00134135503083478\\
483.01	0.00133474540697633\\
484.01	0.00132796342532756\\
485.01	0.00132100398838981\\
486.01	0.00131386180716428\\
487.01	0.00130653139098239\\
488.01	0.00129900704106502\\
489.01	0.00129128288397996\\
490.01	0.00128335298276893\\
491.01	0.00127521089758948\\
492.01	0.00126685010857471\\
493.01	0.00125826233474716\\
494.01	0.0012494401758256\\
495.01	0.00124037612333986\\
496.01	0.00123106232795712\\
497.01	0.0012214905381718\\
498.01	0.00121165191211895\\
499.01	0.00120153667441694\\
500.01	0.00119113210578245\\
501.01	0.00118041491583364\\
502.01	0.0011694069334897\\
503.01	0.00115825522553226\\
504.01	0.00114678010314293\\
505.01	0.00113496280434723\\
506.01	0.00112278951008377\\
507.01	0.00111024557025976\\
508.01	0.00109731542923216\\
509.01	0.00108398254178966\\
510.01	0.00107022927732357\\
511.01	0.00105603680460775\\
512.01	0.00104138490774409\\
513.01	0.00102625130966424\\
514.01	0.00101060650128075\\
515.01	0.00099436616793677\\
516.01	0.000977084734376887\\
517.01	0.000960691326306129\\
518.01	0.000944700367329772\\
519.01	0.000927791988661471\\
520.01	0.000911023289324185\\
521.01	0.000894775923625113\\
522.01	0.000878423206032023\\
523.01	0.000861850718137483\\
524.01	0.000844707102916892\\
525.01	0.000826955246655789\\
526.01	0.000808553839524584\\
527.01	0.000789456733757346\\
528.01	0.000769612375607442\\
529.01	0.000748963556818137\\
530.01	0.000727441266626895\\
531.01	0.000704969255035168\\
532.01	0.00068145902477487\\
533.01	0.000656768768742986\\
534.01	0.000629864955380999\\
535.01	0.000582592877329985\\
536.01	0.000519929998022089\\
537.01	0.000456021358906175\\
538.01	0.000390832352851327\\
539.01	0.000324198365391418\\
540.01	0.000255826929879356\\
541.01	0.000186537130846801\\
542.01	0.000118055601462464\\
543.01	4.8892884100234e-05\\
544.01	1.73472347597681e-18\\
545.01	0\\
546.01	1.73472347597681e-18\\
547.01	0\\
548.01	0\\
549.01	0\\
550.01	0\\
551.01	0\\
552.01	0\\
553.01	0\\
554.01	0\\
555.01	1.73472347597681e-18\\
556.01	0\\
557.01	0\\
558.01	0\\
559.01	1.73472347597681e-18\\
560.01	1.73472347597681e-18\\
561.01	1.73472347597681e-18\\
562.01	1.73472347597681e-18\\
563.01	0\\
564.01	0\\
565.01	1.73472347597681e-18\\
566.01	0\\
567.01	0\\
568.01	0\\
569.01	0\\
570.01	0\\
571.01	0\\
572.01	1.73472347597681e-18\\
573.01	0\\
574.01	1.73472347597681e-18\\
575.01	0\\
576.01	1.73472347597681e-18\\
577.01	0\\
578.01	1.73472347597681e-18\\
579.01	0\\
580.01	0\\
581.01	1.73472347597681e-18\\
582.01	0\\
583.01	0\\
584.01	0\\
585.01	0\\
586.01	0\\
587.01	0\\
588.01	0\\
589.01	0\\
590.01	0\\
591.01	0\\
592.01	1.73472347597681e-18\\
593.01	0\\
594.01	0\\
595.01	0\\
596.01	0\\
597.01	0\\
598.01	0\\
599.01	0\\
599.02	0\\
599.03	1.73472347597681e-18\\
599.04	1.73472347597681e-18\\
599.05	0\\
599.06	0\\
599.07	1.73472347597681e-18\\
599.08	0\\
599.09	0\\
599.1	0\\
599.11	1.73472347597681e-18\\
599.12	1.73472347597681e-18\\
599.13	0\\
599.14	0\\
599.15	1.73472347597681e-18\\
599.16	1.73472347597681e-18\\
599.17	1.73472347597681e-18\\
599.18	0\\
599.19	0\\
599.2	0\\
599.21	0\\
599.22	0\\
599.23	1.73472347597681e-18\\
599.24	1.73472347597681e-18\\
599.25	1.73472347597681e-18\\
599.26	1.73472347597681e-18\\
599.27	0\\
599.28	0\\
599.29	0\\
599.3	0\\
599.31	0\\
599.32	1.73472347597681e-18\\
599.33	1.73472347597681e-18\\
599.34	0\\
599.35	0\\
599.36	0\\
599.37	1.73472347597681e-18\\
599.38	0\\
599.39	0\\
599.4	0\\
599.41	0\\
599.42	0\\
599.43	0\\
599.44	0\\
599.45	0\\
599.46	1.73472347597681e-18\\
599.47	1.73472347597681e-18\\
599.48	1.73472347597681e-18\\
599.49	1.73472347597681e-18\\
599.5	1.73472347597681e-18\\
599.51	0\\
599.52	0\\
599.53	1.73472347597681e-18\\
599.54	0\\
599.55	0\\
599.56	0\\
599.57	1.73472347597681e-18\\
599.58	0\\
599.59	0\\
599.6	0\\
599.61	0\\
599.62	1.73472347597681e-18\\
599.63	0\\
599.64	1.73472347597681e-18\\
599.65	0\\
599.66	0\\
599.67	0\\
599.68	0\\
599.69	1.73472347597681e-18\\
599.7	0\\
599.71	1.73472347597681e-18\\
599.72	1.73472347597681e-18\\
599.73	1.73472347597681e-18\\
599.74	0\\
599.75	0\\
599.76	0\\
599.77	0\\
599.78	0\\
599.79	0\\
599.8	0\\
599.81	0\\
599.82	0\\
599.83	1.73472347597681e-18\\
599.84	0\\
599.85	0\\
599.86	0\\
599.87	0\\
599.88	0\\
599.89	0\\
599.9	0\\
599.91	0\\
599.92	0\\
599.93	0\\
599.94	0\\
599.95	0\\
599.96	0\\
599.97	0\\
599.98	0\\
599.99	0\\
600	0\\
};
\addplot [color=blue,solid,forget plot]
  table[row sep=crcr]{%
0.01	0.00240288571436453\\
1.01	0.00240288569658394\\
2.01	0.00240288567842936\\
3.01	0.00240288565989285\\
4.01	0.00240288564096636\\
5.01	0.00240288562164156\\
6.01	0.00240288560191004\\
7.01	0.00240288558176315\\
8.01	0.00240288556119211\\
9.01	0.00240288554018791\\
10.01	0.00240288551874136\\
11.01	0.00240288549684306\\
12.01	0.00240288547448342\\
13.01	0.00240288545165266\\
14.01	0.00240288542834077\\
15.01	0.0024028854045375\\
16.01	0.00240288538023244\\
17.01	0.00240288535541497\\
18.01	0.00240288533007409\\
19.01	0.00240288530419873\\
20.01	0.00240288527777753\\
21.01	0.00240288525079886\\
22.01	0.00240288522325087\\
23.01	0.00240288519512146\\
24.01	0.00240288516639823\\
25.01	0.00240288513706853\\
26.01	0.00240288510711945\\
27.01	0.00240288507653782\\
28.01	0.00240288504531012\\
29.01	0.00240288501342262\\
30.01	0.00240288498086125\\
31.01	0.00240288494761162\\
32.01	0.00240288491365906\\
33.01	0.00240288487898859\\
34.01	0.00240288484358486\\
35.01	0.00240288480743225\\
36.01	0.00240288477051476\\
37.01	0.00240288473281605\\
38.01	0.00240288469431945\\
39.01	0.0024028846550079\\
40.01	0.00240288461486397\\
41.01	0.00240288457386991\\
42.01	0.00240288453200749\\
43.01	0.00240288448925817\\
44.01	0.00240288444560298\\
45.01	0.00240288440102251\\
46.01	0.00240288435549698\\
47.01	0.00240288430900614\\
48.01	0.00240288426152931\\
49.01	0.00240288421304538\\
50.01	0.00240288416353279\\
51.01	0.00240288411296945\\
52.01	0.00240288406133287\\
53.01	0.00240288400859999\\
54.01	0.00240288395474732\\
55.01	0.00240288389975084\\
56.01	0.00240288384358595\\
57.01	0.00240288378622761\\
58.01	0.00240288372765017\\
59.01	0.00240288366782744\\
60.01	0.00240288360673266\\
61.01	0.00240288354433849\\
62.01	0.00240288348061699\\
63.01	0.00240288341553959\\
64.01	0.00240288334907712\\
65.01	0.00240288328119979\\
66.01	0.00240288321187712\\
67.01	0.00240288314107799\\
68.01	0.0024028830687706\\
69.01	0.00240288299492244\\
70.01	0.00240288291950029\\
71.01	0.00240288284247022\\
72.01	0.00240288276379753\\
73.01	0.00240288268344679\\
74.01	0.00240288260138179\\
75.01	0.0024028825175655\\
76.01	0.00240288243196009\\
77.01	0.00240288234452692\\
78.01	0.00240288225522649\\
79.01	0.00240288216401844\\
80.01	0.00240288207086146\\
81.01	0.00240288197571344\\
82.01	0.0024028818785313\\
83.01	0.00240288177927093\\
84.01	0.00240288167788737\\
85.01	0.00240288157433463\\
86.01	0.00240288146856569\\
87.01	0.00240288136053251\\
88.01	0.00240288125018597\\
89.01	0.0024028811374759\\
90.01	0.002402881022351\\
91.01	0.00240288090475882\\
92.01	0.0024028807846458\\
93.01	0.00240288066195717\\
94.01	0.00240288053663695\\
95.01	0.00240288040862791\\
96.01	0.0024028802778716\\
97.01	0.00240288014430822\\
98.01	0.00240288000787666\\
99.01	0.00240287986851449\\
100.01	0.00240287972615789\\
101.01	0.00240287958074156\\
102.01	0.00240287943219884\\
103.01	0.00240287928046153\\
104.01	0.00240287912545996\\
105.01	0.00240287896712286\\
106.01	0.00240287880537744\\
107.01	0.00240287864014922\\
108.01	0.00240287847136214\\
109.01	0.00240287829893839\\
110.01	0.00240287812279843\\
111.01	0.00240287794286099\\
112.01	0.00240287775904297\\
113.01	0.00240287757125939\\
114.01	0.00240287737942339\\
115.01	0.0024028771834462\\
116.01	0.00240287698323703\\
117.01	0.0024028767787031\\
118.01	0.00240287656974948\\
119.01	0.00240287635627922\\
120.01	0.00240287613819315\\
121.01	0.00240287591538986\\
122.01	0.00240287568776571\\
123.01	0.00240287545521472\\
124.01	0.00240287521762849\\
125.01	0.0024028749748963\\
126.01	0.00240287472690484\\
127.01	0.00240287447353829\\
128.01	0.00240287421467826\\
129.01	0.00240287395020368\\
130.01	0.00240287367999072\\
131.01	0.00240287340391281\\
132.01	0.00240287312184053\\
133.01	0.00240287283364155\\
134.01	0.00240287253918049\\
135.01	0.00240287223831905\\
136.01	0.00240287193091568\\
137.01	0.00240287161682571\\
138.01	0.0024028712959012\\
139.01	0.00240287096799085\\
140.01	0.00240287063293997\\
141.01	0.00240287029059032\\
142.01	0.00240286994078015\\
143.01	0.00240286958334401\\
144.01	0.00240286921811269\\
145.01	0.00240286884491319\\
146.01	0.0024028684635686\\
147.01	0.00240286807389793\\
148.01	0.00240286767571614\\
149.01	0.002402867268834\\
150.01	0.00240286685305795\\
151.01	0.00240286642819005\\
152.01	0.00240286599402787\\
153.01	0.00240286555036435\\
154.01	0.00240286509698779\\
155.01	0.00240286463368159\\
156.01	0.00240286416022426\\
157.01	0.00240286367638928\\
158.01	0.00240286318194495\\
159.01	0.00240286267665429\\
160.01	0.00240286216027489\\
161.01	0.0024028616325589\\
162.01	0.0024028610932527\\
163.01	0.00240286054209695\\
164.01	0.00240285997882638\\
165.01	0.00240285940316962\\
166.01	0.00240285881484913\\
167.01	0.00240285821358105\\
168.01	0.00240285759907495\\
169.01	0.00240285697103376\\
170.01	0.0024028563291537\\
171.01	0.00240285567312395\\
172.01	0.00240285500262658\\
173.01	0.00240285431733637\\
174.01	0.00240285361692066\\
175.01	0.00240285290103911\\
176.01	0.00240285216934366\\
177.01	0.00240285142147813\\
178.01	0.00240285065707829\\
179.01	0.00240284987577142\\
180.01	0.00240284907717633\\
181.01	0.00240284826090299\\
182.01	0.00240284742655246\\
183.01	0.00240284657371658\\
184.01	0.00240284570197779\\
185.01	0.00240284481090893\\
186.01	0.00240284390007302\\
187.01	0.00240284296902298\\
188.01	0.00240284201730143\\
189.01	0.00240284104444045\\
190.01	0.00240284004996135\\
191.01	0.00240283903337435\\
192.01	0.0024028379941784\\
193.01	0.00240283693186089\\
194.01	0.00240283584589736\\
195.01	0.00240283473575125\\
196.01	0.0024028336008736\\
197.01	0.00240283244070274\\
198.01	0.00240283125466409\\
199.01	0.00240283004216972\\
200.01	0.00240282880261815\\
201.01	0.00240282753539396\\
202.01	0.00240282623986752\\
203.01	0.00240282491539463\\
204.01	0.00240282356131619\\
205.01	0.00240282217695783\\
206.01	0.00240282076162963\\
207.01	0.00240281931462561\\
208.01	0.00240281783522351\\
209.01	0.00240281632268436\\
210.01	0.00240281477625204\\
211.01	0.00240281319515294\\
212.01	0.00240281157859556\\
213.01	0.00240280992577001\\
214.01	0.0024028082358477\\
215.01	0.00240280650798084\\
216.01	0.00240280474130197\\
217.01	0.00240280293492357\\
218.01	0.00240280108793759\\
219.01	0.00240279919941486\\
220.01	0.00240279726840477\\
221.01	0.00240279529393467\\
222.01	0.00240279327500939\\
223.01	0.00240279121061069\\
224.01	0.00240278909969677\\
225.01	0.0024027869412017\\
226.01	0.00240278473403488\\
227.01	0.00240278247708047\\
228.01	0.00240278016919677\\
229.01	0.00240277780921569\\
230.01	0.0024027753959421\\
231.01	0.00240277292815322\\
232.01	0.00240277040459795\\
233.01	0.00240276782399633\\
234.01	0.00240276518503872\\
235.01	0.00240276248638527\\
236.01	0.00240275972666509\\
237.01	0.00240275690447568\\
238.01	0.0024027540183821\\
239.01	0.00240275106691628\\
240.01	0.00240274804857626\\
241.01	0.00240274496182537\\
242.01	0.00240274180509153\\
243.01	0.00240273857676635\\
244.01	0.00240273527520439\\
245.01	0.00240273189872223\\
246.01	0.00240272844559766\\
247.01	0.00240272491406882\\
248.01	0.00240272130233324\\
249.01	0.002402717608547\\
250.01	0.00240271383082367\\
251.01	0.00240270996723346\\
252.01	0.00240270601580224\\
253.01	0.00240270197451042\\
254.01	0.00240269784129208\\
255.01	0.00240269361403378\\
256.01	0.00240268929057362\\
257.01	0.00240268486870005\\
258.01	0.00240268034615077\\
259.01	0.00240267572061163\\
260.01	0.00240267098971545\\
261.01	0.00240266615104076\\
262.01	0.00240266120211068\\
263.01	0.00240265614039162\\
264.01	0.00240265096329199\\
265.01	0.00240264566816094\\
266.01	0.00240264025228699\\
267.01	0.0024026347128967\\
268.01	0.00240262904715328\\
269.01	0.00240262325215511\\
270.01	0.00240261732493434\\
271.01	0.00240261126245542\\
272.01	0.00240260506161346\\
273.01	0.00240259871923284\\
274.01	0.00240259223206548\\
275.01	0.00240258559678927\\
276.01	0.00240257881000639\\
277.01	0.0024025718682416\\
278.01	0.00240256476794048\\
279.01	0.00240255750546768\\
280.01	0.00240255007710502\\
281.01	0.00240254247904967\\
282.01	0.00240253470741222\\
283.01	0.00240252675821469\\
284.01	0.00240251862738849\\
285.01	0.00240251031077243\\
286.01	0.00240250180411052\\
287.01	0.00240249310304982\\
288.01	0.00240248420313826\\
289.01	0.00240247509982224\\
290.01	0.00240246578844441\\
291.01	0.0024024562642412\\
292.01	0.00240244652234034\\
293.01	0.00240243655775832\\
294.01	0.00240242636539788\\
295.01	0.00240241594004516\\
296.01	0.00240240527636709\\
297.01	0.00240239436890853\\
298.01	0.00240238321208935\\
299.01	0.00240237180020147\\
300.01	0.0024023601274057\\
301.01	0.00240234818772876\\
302.01	0.00240233597505979\\
303.01	0.00240232348314723\\
304.01	0.00240231070559523\\
305.01	0.00240229763586016\\
306.01	0.00240228426724694\\
307.01	0.00240227059290527\\
308.01	0.00240225660582575\\
309.01	0.00240224229883593\\
310.01	0.00240222766459611\\
311.01	0.00240221269559514\\
312.01	0.00240219738414605\\
313.01	0.00240218172238142\\
314.01	0.00240216570224893\\
315.01	0.00240214931550629\\
316.01	0.00240213255371653\\
317.01	0.00240211540824264\\
318.01	0.00240209787024245\\
319.01	0.00240207993066313\\
320.01	0.0024020615802355\\
321.01	0.00240204280946832\\
322.01	0.00240202360864218\\
323.01	0.00240200396780334\\
324.01	0.0024019838767574\\
325.01	0.00240196332506262\\
326.01	0.00240194230202314\\
327.01	0.00240192079668196\\
328.01	0.00240189879781362\\
329.01	0.00240187629391681\\
330.01	0.00240185327320662\\
331.01	0.0024018297236065\\
332.01	0.00240180563274017\\
333.01	0.00240178098792303\\
334.01	0.00240175577615351\\
335.01	0.00240172998410406\\
336.01	0.00240170359811179\\
337.01	0.00240167660416911\\
338.01	0.00240164898791365\\
339.01	0.00240162073461831\\
340.01	0.00240159182918077\\
341.01	0.00240156225611275\\
342.01	0.00240153199952899\\
343.01	0.0024015010431359\\
344.01	0.0024014693702198\\
345.01	0.00240143696363506\\
346.01	0.00240140380579167\\
347.01	0.00240136987864255\\
348.01	0.00240133516367057\\
349.01	0.00240129964187519\\
350.01	0.00240126329375869\\
351.01	0.00240122609931209\\
352.01	0.0024011880380007\\
353.01	0.00240114908874926\\
354.01	0.00240110922992672\\
355.01	0.00240106843933059\\
356.01	0.00240102669417101\\
357.01	0.00240098397105425\\
358.01	0.00240094024596589\\
359.01	0.00240089549425364\\
360.01	0.00240084969060961\\
361.01	0.00240080280905227\\
362.01	0.00240075482290794\\
363.01	0.00240070570479182\\
364.01	0.00240065542658876\\
365.01	0.00240060395943334\\
366.01	0.00240055127368982\\
367.01	0.00240049733893152\\
368.01	0.00240044212391992\\
369.01	0.00240038559658326\\
370.01	0.00240032772399498\\
371.01	0.00240026847235161\\
372.01	0.00240020780695057\\
373.01	0.00240014569216741\\
374.01	0.00240008209143306\\
375.01	0.00240001696721089\\
376.01	0.0023999502809769\\
377.01	0.00239988199321554\\
378.01	0.00239981206340619\\
379.01	0.00239974044978371\\
380.01	0.00239966710956389\\
381.01	0.0023995919988901\\
382.01	0.00239951507270891\\
383.01	0.00239943628473266\\
384.01	0.00239935558740052\\
385.01	0.00239927293183783\\
386.01	0.00239918826781396\\
387.01	0.00239910154369817\\
388.01	0.00239901270641361\\
389.01	0.00239892170138943\\
390.01	0.00239882847251061\\
391.01	0.00239873296206552\\
392.01	0.00239863511069123\\
393.01	0.00239853485731606\\
394.01	0.00239843213909956\\
395.01	0.00239832689136959\\
396.01	0.00239821904755624\\
397.01	0.0023981085391226\\
398.01	0.00239799529549201\\
399.01	0.00239787924397152\\
400.01	0.00239776030967159\\
401.01	0.00239763841542143\\
402.01	0.00239751348167983\\
403.01	0.00239738542644139\\
404.01	0.00239725416513736\\
405.01	0.0023971196105312\\
406.01	0.00239698167260827\\
407.01	0.00239684025845907\\
408.01	0.00239669527215606\\
409.01	0.00239654661462303\\
410.01	0.0023963941834969\\
411.01	0.00239623787298124\\
412.01	0.0023960775736909\\
413.01	0.00239591317248699\\
414.01	0.00239574455230188\\
415.01	0.00239557159195282\\
416.01	0.00239539416594393\\
417.01	0.00239521214425515\\
418.01	0.00239502539211761\\
419.01	0.00239483376977369\\
420.01	0.00239463713222116\\
421.01	0.00239443532893965\\
422.01	0.00239422820359799\\
423.01	0.00239401559374111\\
424.01	0.00239379733045407\\
425.01	0.0023935732380018\\
426.01	0.00239334313344181\\
427.01	0.0023931068262076\\
428.01	0.00239286411765984\\
429.01	0.00239261480060215\\
430.01	0.00239235865875773\\
431.01	0.00239209546620304\\
432.01	0.00239182498675366\\
433.01	0.00239154697329714\\
434.01	0.00239126116706697\\
435.01	0.00239096729685058\\
436.01	0.00239066507812369\\
437.01	0.00239035421210179\\
438.01	0.00239003438469846\\
439.01	0.00238970526537812\\
440.01	0.00238936650588943\\
441.01	0.00238901773886284\\
442.01	0.00238865857625296\\
443.01	0.00238828860760366\\
444.01	0.00238790739810927\\
445.01	0.00238751448644124\\
446.01	0.00238710938230342\\
447.01	0.00238669156367283\\
448.01	0.00238626047367503\\
449.01	0.00238581551703253\\
450.01	0.00238535605601448\\
451.01	0.00238488140580067\\
452.01	0.00238439082915637\\
453.01	0.00238388353029419\\
454.01	0.00238335864777369\\
455.01	0.00238281524625954\\
456.01	0.00238225230692208\\
457.01	0.00238166871622011\\
458.01	0.00238106325275859\\
459.01	0.00238043457192086\\
460.01	0.00237978118854724\\
461.01	0.00237910146288947\\
462.01	0.0023783936176163\\
463.01	0.00237765573008159\\
464.01	0.00237688521573682\\
465.01	0.00237607804645507\\
466.01	0.00237521680126685\\
467.01	0.00237406363369648\\
468.01	0.0023723996421659\\
469.01	0.00237068404364005\\
470.01	0.00236890837206177\\
471.01	0.002367135533743\\
472.01	0.0023653201884989\\
473.01	0.00236344932990166\\
474.01	0.00236148983298677\\
475.01	0.00235938935000563\\
476.01	0.00235722803957903\\
477.01	0.0023550099223541\\
478.01	0.00235273318309994\\
479.01	0.00235039592526171\\
480.01	0.00234799616578743\\
481.01	0.00234553182953742\\
482.01	0.00234300074323746\\
483.01	0.0023404006289316\\
484.01	0.00233772909688802\\
485.01	0.00233498363790802\\
486.01	0.00233216161499807\\
487.01	0.00232926025447013\\
488.01	0.0023262766372841\\
489.01	0.00232320769423005\\
490.01	0.00232005019424852\\
491.01	0.00231680071685399\\
492.01	0.00231345562778019\\
493.01	0.00231001101689176\\
494.01	0.00230646282222201\\
495.01	0.00230280674238715\\
496.01	0.00229903821072988\\
497.01	0.00229515236771408\\
498.01	0.00229114400296414\\
499.01	0.00228700730893898\\
500.01	0.00228273423043378\\
501.01	0.00227830360804612\\
502.01	0.00227366869047343\\
503.01	0.00226919996914999\\
504.01	0.00226462357559349\\
505.01	0.00225988950061276\\
506.01	0.0022549887410712\\
507.01	0.00224991140125685\\
508.01	0.00224464656536145\\
509.01	0.00223918214737417\\
510.01	0.00223350471338938\\
511.01	0.00222759926749472\\
512.01	0.00222144896884423\\
513.01	0.00221503455724645\\
514.01	0.00220833150147027\\
515.01	0.00220128357537459\\
516.01	0.0021934419223718\\
517.01	0.00218051080749184\\
518.01	0.00216530521470991\\
519.01	0.00214963885173311\\
520.01	0.00213352591913959\\
521.01	0.00211682686193053\\
522.01	0.00209990162455342\\
523.01	0.00208334881810736\\
524.01	0.00206629842201661\\
525.01	0.0020487293273293\\
526.01	0.00203061912117578\\
527.01	0.0020119439851682\\
528.01	0.00199267861544261\\
529.01	0.00197279603566874\\
530.01	0.00195226717990642\\
531.01	0.0019310611918816\\
532.01	0.00190914400220951\\
533.01	0.00188646584897037\\
534.01	0.00186281675633515\\
535.01	0.00183733817879279\\
536.01	0.00181068563975129\\
537.01	0.00178316148253556\\
538.01	0.00175483070660634\\
539.01	0.00172543183520467\\
540.01	0.00169458528263656\\
541.01	0.0016619117138034\\
542.01	0.00163319129547519\\
543.01	0.00160397032611999\\
544.01	0.00157532742552503\\
545.01	0.00154672149291808\\
546.01	0.00151676918156295\\
547.01	0.00148528325632334\\
548.01	0.00145122918066596\\
549.01	0.00139645842151118\\
550.01	0.00131573204815777\\
551.01	0.00123401894266624\\
552.01	0.00115056170306921\\
553.01	0.00106527540330119\\
554.01	0.000978066999797552\\
555.01	0.000888833971152863\\
556.01	0.000797461203723393\\
557.01	0.00070380649333056\\
558.01	0.00060762032763984\\
559.01	0.000508366950882333\\
560.01	0.00040767854432403\\
561.01	0.000303109299583869\\
562.01	0.000206994944679002\\
563.01	0.000119489589999968\\
564.01	3.32849885533466e-05\\
565.01	1.73472347597681e-18\\
566.01	0\\
567.01	0\\
568.01	0\\
569.01	0\\
570.01	0\\
571.01	0\\
572.01	1.73472347597681e-18\\
573.01	0\\
574.01	1.73472347597681e-18\\
575.01	0\\
576.01	1.73472347597681e-18\\
577.01	0\\
578.01	1.73472347597681e-18\\
579.01	0\\
580.01	0\\
581.01	1.73472347597681e-18\\
582.01	0\\
583.01	0\\
584.01	0\\
585.01	0\\
586.01	0\\
587.01	0\\
588.01	0\\
589.01	0\\
590.01	0\\
591.01	0\\
592.01	1.73472347597681e-18\\
593.01	0\\
594.01	0\\
595.01	0\\
596.01	0\\
597.01	0\\
598.01	0\\
599.01	0\\
599.02	0\\
599.03	1.73472347597681e-18\\
599.04	1.73472347597681e-18\\
599.05	0\\
599.06	0\\
599.07	1.73472347597681e-18\\
599.08	0\\
599.09	0\\
599.1	0\\
599.11	1.73472347597681e-18\\
599.12	1.73472347597681e-18\\
599.13	0\\
599.14	0\\
599.15	1.73472347597681e-18\\
599.16	1.73472347597681e-18\\
599.17	1.73472347597681e-18\\
599.18	0\\
599.19	0\\
599.2	0\\
599.21	0\\
599.22	0\\
599.23	1.73472347597681e-18\\
599.24	1.73472347597681e-18\\
599.25	1.73472347597681e-18\\
599.26	1.73472347597681e-18\\
599.27	0\\
599.28	0\\
599.29	0\\
599.3	0\\
599.31	0\\
599.32	1.73472347597681e-18\\
599.33	1.73472347597681e-18\\
599.34	0\\
599.35	0\\
599.36	0\\
599.37	1.73472347597681e-18\\
599.38	0\\
599.39	0\\
599.4	0\\
599.41	0\\
599.42	0\\
599.43	0\\
599.44	0\\
599.45	0\\
599.46	1.73472347597681e-18\\
599.47	1.73472347597681e-18\\
599.48	1.73472347597681e-18\\
599.49	1.73472347597681e-18\\
599.5	1.73472347597681e-18\\
599.51	0\\
599.52	0\\
599.53	1.73472347597681e-18\\
599.54	0\\
599.55	0\\
599.56	0\\
599.57	1.73472347597681e-18\\
599.58	0\\
599.59	0\\
599.6	0\\
599.61	0\\
599.62	1.73472347597681e-18\\
599.63	0\\
599.64	1.73472347597681e-18\\
599.65	0\\
599.66	0\\
599.67	0\\
599.68	0\\
599.69	1.73472347597681e-18\\
599.7	0\\
599.71	1.73472347597681e-18\\
599.72	1.73472347597681e-18\\
599.73	1.73472347597681e-18\\
599.74	0\\
599.75	0\\
599.76	0\\
599.77	0\\
599.78	0\\
599.79	0\\
599.8	0\\
599.81	0\\
599.82	0\\
599.83	1.73472347597681e-18\\
599.84	0\\
599.85	0\\
599.86	0\\
599.87	0\\
599.88	0\\
599.89	0\\
599.9	0\\
599.91	0\\
599.92	0\\
599.93	0\\
599.94	0\\
599.95	0\\
599.96	0\\
599.97	0\\
599.98	0\\
599.99	0\\
600	0\\
};
\addplot [color=mycolor10,solid,forget plot]
  table[row sep=crcr]{%
0.01	0.00377125663032789\\
1.01	0.00377125662923481\\
2.01	0.00377125662811874\\
3.01	0.00377125662697919\\
4.01	0.00377125662581567\\
5.01	0.00377125662462766\\
6.01	0.00377125662341464\\
7.01	0.0037712566221761\\
8.01	0.00377125662091146\\
9.01	0.00377125661962021\\
10.01	0.00377125661830175\\
11.01	0.00377125661695552\\
12.01	0.00377125661558092\\
13.01	0.00377125661417738\\
14.01	0.00377125661274424\\
15.01	0.00377125661128089\\
16.01	0.0037712566097867\\
17.01	0.00377125660826095\\
18.01	0.00377125660670309\\
19.01	0.00377125660511236\\
20.01	0.00377125660348806\\
21.01	0.00377125660182949\\
22.01	0.00377125660013591\\
23.01	0.00377125659840658\\
24.01	0.00377125659664075\\
25.01	0.00377125659483764\\
26.01	0.00377125659299645\\
27.01	0.00377125659111635\\
28.01	0.00377125658919654\\
29.01	0.00377125658723616\\
30.01	0.00377125658523435\\
31.01	0.00377125658319023\\
32.01	0.00377125658110289\\
33.01	0.00377125657897141\\
34.01	0.00377125657679484\\
35.01	0.00377125657457225\\
36.01	0.0037712565723026\\
37.01	0.00377125656998494\\
38.01	0.00377125656761822\\
39.01	0.0037712565652014\\
40.01	0.00377125656273339\\
41.01	0.00377125656021311\\
42.01	0.00377125655763946\\
43.01	0.00377125655501125\\
44.01	0.00377125655232737\\
45.01	0.00377125654958658\\
46.01	0.0037712565467877\\
47.01	0.00377125654392947\\
48.01	0.00377125654101061\\
49.01	0.00377125653802985\\
50.01	0.00377125653498578\\
51.01	0.00377125653187715\\
52.01	0.00377125652870254\\
53.01	0.00377125652546051\\
54.01	0.00377125652214963\\
55.01	0.00377125651876842\\
56.01	0.00377125651531538\\
57.01	0.00377125651178895\\
58.01	0.00377125650818757\\
59.01	0.00377125650450962\\
60.01	0.00377125650075346\\
61.01	0.0037712564969174\\
62.01	0.00377125649299973\\
63.01	0.00377125648899868\\
64.01	0.00377125648491248\\
65.01	0.00377125648073927\\
66.01	0.0037712564764772\\
67.01	0.00377125647212436\\
68.01	0.00377125646767877\\
69.01	0.00377125646313843\\
70.01	0.00377125645850132\\
71.01	0.00377125645376534\\
72.01	0.00377125644892838\\
73.01	0.00377125644398821\\
74.01	0.00377125643894264\\
75.01	0.0037712564337894\\
76.01	0.00377125642852614\\
77.01	0.0037712564231505\\
78.01	0.00377125641766003\\
79.01	0.00377125641205227\\
80.01	0.00377125640632467\\
81.01	0.00377125640047466\\
82.01	0.00377125639449953\\
83.01	0.00377125638839665\\
84.01	0.00377125638216321\\
85.01	0.00377125637579638\\
86.01	0.00377125636929328\\
87.01	0.00377125636265095\\
88.01	0.00377125635586637\\
89.01	0.00377125634893645\\
90.01	0.00377125634185805\\
91.01	0.00377125633462793\\
92.01	0.0037712563272428\\
93.01	0.00377125631969928\\
94.01	0.00377125631199396\\
95.01	0.00377125630412329\\
96.01	0.00377125629608368\\
97.01	0.00377125628787146\\
98.01	0.00377125627948288\\
99.01	0.00377125627091409\\
100.01	0.00377125626216116\\
101.01	0.00377125625322007\\
102.01	0.00377125624408673\\
103.01	0.00377125623475696\\
104.01	0.00377125622522645\\
105.01	0.00377125621549082\\
106.01	0.0037712562055456\\
107.01	0.00377125619538622\\
108.01	0.00377125618500799\\
109.01	0.00377125617440612\\
110.01	0.00377125616357572\\
111.01	0.0037712561525118\\
112.01	0.00377125614120924\\
113.01	0.00377125612966283\\
114.01	0.0037712561178672\\
115.01	0.00377125610581691\\
116.01	0.00377125609350638\\
117.01	0.00377125608092989\\
118.01	0.00377125606808161\\
119.01	0.00377125605495556\\
120.01	0.00377125604154567\\
121.01	0.00377125602784567\\
122.01	0.00377125601384922\\
123.01	0.00377125599954977\\
124.01	0.00377125598494068\\
125.01	0.00377125597001511\\
126.01	0.00377125595476611\\
127.01	0.00377125593918656\\
128.01	0.00377125592326918\\
129.01	0.00377125590700649\\
130.01	0.00377125589039092\\
131.01	0.00377125587341465\\
132.01	0.00377125585606974\\
133.01	0.00377125583834803\\
134.01	0.00377125582024122\\
135.01	0.0037712558017408\\
136.01	0.00377125578283805\\
137.01	0.00377125576352404\\
138.01	0.00377125574378972\\
139.01	0.00377125572362578\\
140.01	0.00377125570302267\\
141.01	0.00377125568197071\\
142.01	0.00377125566045989\\
143.01	0.00377125563848008\\
144.01	0.00377125561602085\\
145.01	0.00377125559307157\\
146.01	0.00377125556962133\\
147.01	0.00377125554565905\\
148.01	0.00377125552117329\\
149.01	0.00377125549615244\\
150.01	0.00377125547058459\\
151.01	0.00377125544445756\\
152.01	0.00377125541775891\\
153.01	0.0037712553904759\\
154.01	0.00377125536259548\\
155.01	0.00377125533410436\\
156.01	0.0037712553049889\\
157.01	0.00377125527523517\\
158.01	0.00377125524482892\\
159.01	0.00377125521375555\\
160.01	0.00377125518200019\\
161.01	0.00377125514954756\\
162.01	0.00377125511638205\\
163.01	0.00377125508248775\\
164.01	0.0037712550478483\\
165.01	0.00377125501244702\\
166.01	0.00377125497626686\\
167.01	0.00377125493929032\\
168.01	0.00377125490149956\\
169.01	0.00377125486287629\\
170.01	0.00377125482340183\\
171.01	0.00377125478305704\\
172.01	0.00377125474182241\\
173.01	0.00377125469967789\\
174.01	0.00377125465660303\\
175.01	0.00377125461257688\\
176.01	0.00377125456757805\\
177.01	0.00377125452158462\\
178.01	0.00377125447457417\\
179.01	0.00377125442652378\\
180.01	0.00377125437741001\\
181.01	0.00377125432720886\\
182.01	0.00377125427589576\\
183.01	0.00377125422344564\\
184.01	0.0037712541698328\\
185.01	0.00377125411503094\\
186.01	0.00377125405901324\\
187.01	0.00377125400175211\\
188.01	0.00377125394321946\\
189.01	0.0037712538833865\\
190.01	0.00377125382222378\\
191.01	0.00377125375970115\\
192.01	0.00377125369578781\\
193.01	0.00377125363045223\\
194.01	0.0037712535636621\\
195.01	0.00377125349538445\\
196.01	0.0037712534255855\\
197.01	0.00377125335423071\\
198.01	0.00377125328128466\\
199.01	0.00377125320671129\\
200.01	0.00377125313047352\\
201.01	0.00377125305253351\\
202.01	0.00377125297285252\\
203.01	0.00377125289139092\\
204.01	0.00377125280810816\\
205.01	0.00377125272296276\\
206.01	0.00377125263591226\\
207.01	0.00377125254691324\\
208.01	0.00377125245592126\\
209.01	0.00377125236289087\\
210.01	0.00377125226777554\\
211.01	0.00377125217052766\\
212.01	0.00377125207109857\\
213.01	0.0037712519694384\\
214.01	0.00377125186549624\\
215.01	0.00377125175921985\\
216.01	0.00377125165055593\\
217.01	0.00377125153944987\\
218.01	0.00377125142584576\\
219.01	0.00377125130968648\\
220.01	0.00377125119091354\\
221.01	0.00377125106946709\\
222.01	0.0037712509452859\\
223.01	0.00377125081830735\\
224.01	0.00377125068846734\\
225.01	0.00377125055570028\\
226.01	0.00377125041993908\\
227.01	0.0037712502811151\\
228.01	0.0037712501391581\\
229.01	0.00377124999399622\\
230.01	0.00377124984555593\\
231.01	0.003771249693762\\
232.01	0.00377124953853749\\
233.01	0.00377124937980362\\
234.01	0.00377124921747985\\
235.01	0.00377124905148371\\
236.01	0.00377124888173092\\
237.01	0.00377124870813518\\
238.01	0.0037712485306082\\
239.01	0.00377124834905967\\
240.01	0.00377124816339722\\
241.01	0.00377124797352628\\
242.01	0.00377124777935015\\
243.01	0.00377124758076991\\
244.01	0.00377124737768429\\
245.01	0.00377124716998976\\
246.01	0.00377124695758038\\
247.01	0.00377124674034776\\
248.01	0.00377124651818101\\
249.01	0.00377124629096671\\
250.01	0.0037712460585888\\
251.01	0.00377124582092861\\
252.01	0.00377124557786464\\
253.01	0.0037712453292727\\
254.01	0.00377124507502568\\
255.01	0.00377124481499361\\
256.01	0.00377124454904345\\
257.01	0.00377124427703921\\
258.01	0.0037712439988417\\
259.01	0.00377124371430856\\
260.01	0.00377124342329422\\
261.01	0.00377124312564969\\
262.01	0.00377124282122264\\
263.01	0.00377124250985719\\
264.01	0.00377124219139397\\
265.01	0.00377124186566991\\
266.01	0.00377124153251821\\
267.01	0.00377124119176828\\
268.01	0.00377124084324565\\
269.01	0.00377124048677181\\
270.01	0.00377124012216424\\
271.01	0.0037712397492362\\
272.01	0.00377123936779675\\
273.01	0.00377123897765053\\
274.01	0.00377123857859778\\
275.01	0.00377123817043414\\
276.01	0.00377123775295064\\
277.01	0.0037712373259335\\
278.01	0.00377123688916414\\
279.01	0.00377123644241893\\
280.01	0.00377123598546919\\
281.01	0.00377123551808104\\
282.01	0.00377123504001524\\
283.01	0.0037712345510271\\
284.01	0.00377123405086641\\
285.01	0.00377123353927726\\
286.01	0.0037712330159978\\
287.01	0.00377123248076038\\
288.01	0.00377123193329115\\
289.01	0.00377123137331005\\
290.01	0.00377123080053061\\
291.01	0.00377123021465986\\
292.01	0.00377122961539813\\
293.01	0.00377122900243893\\
294.01	0.0037712283754687\\
295.01	0.00377122773416679\\
296.01	0.00377122707820523\\
297.01	0.00377122640724845\\
298.01	0.00377122572095324\\
299.01	0.00377122501896854\\
300.01	0.00377122430093521\\
301.01	0.00377122356648583\\
302.01	0.00377122281524456\\
303.01	0.00377122204682687\\
304.01	0.00377122126083937\\
305.01	0.00377122045687959\\
306.01	0.00377121963453571\\
307.01	0.00377121879338637\\
308.01	0.00377121793300043\\
309.01	0.0037712170529367\\
310.01	0.00377121615274374\\
311.01	0.00377121523195949\\
312.01	0.0037712142901111\\
313.01	0.00377121332671464\\
314.01	0.00377121234127477\\
315.01	0.00377121133328445\\
316.01	0.00377121030222468\\
317.01	0.00377120924756412\\
318.01	0.00377120816875886\\
319.01	0.00377120706525192\\
320.01	0.00377120593647308\\
321.01	0.00377120478183842\\
322.01	0.00377120360074998\\
323.01	0.00377120239259536\\
324.01	0.00377120115674734\\
325.01	0.00377119989256349\\
326.01	0.00377119859938572\\
327.01	0.00377119727653983\\
328.01	0.00377119592333516\\
329.01	0.00377119453906398\\
330.01	0.00377119312300113\\
331.01	0.00377119167440351\\
332.01	0.00377119019250953\\
333.01	0.00377118867653863\\
334.01	0.0037711871256907\\
335.01	0.00377118553914557\\
336.01	0.00377118391606246\\
337.01	0.00377118225557928\\
338.01	0.00377118055681215\\
339.01	0.0037711788188547\\
340.01	0.00377117704077747\\
341.01	0.0037711752216272\\
342.01	0.00377117336042621\\
343.01	0.00377117145617166\\
344.01	0.00377116950783488\\
345.01	0.00377116751436053\\
346.01	0.00377116547466594\\
347.01	0.00377116338764031\\
348.01	0.0037711612521439\\
349.01	0.00377115906700718\\
350.01	0.00377115683103001\\
351.01	0.0037711545429808\\
352.01	0.00377115220159558\\
353.01	0.0037711498055771\\
354.01	0.00377114735359391\\
355.01	0.00377114484427938\\
356.01	0.00377114227623072\\
357.01	0.00377113964800797\\
358.01	0.003771136958133\\
359.01	0.00377113420508843\\
360.01	0.00377113138731648\\
361.01	0.00377112850321798\\
362.01	0.00377112555115114\\
363.01	0.00377112252943045\\
364.01	0.00377111943632548\\
365.01	0.0037711162700596\\
366.01	0.00377111302880885\\
367.01	0.00377110971070064\\
368.01	0.00377110631381242\\
369.01	0.00377110283617048\\
370.01	0.00377109927574847\\
371.01	0.00377109563046624\\
372.01	0.00377109189818833\\
373.01	0.00377108807672269\\
374.01	0.00377108416381922\\
375.01	0.00377108015716865\\
376.01	0.0037710760544019\\
377.01	0.00377107185309136\\
378.01	0.00377106755074423\\
379.01	0.00377106314478657\\
380.01	0.0037710586325869\\
381.01	0.00377105401144644\\
382.01	0.00377104927859274\\
383.01	0.00377104443117732\\
384.01	0.00377103946627339\\
385.01	0.00377103438087322\\
386.01	0.0037710291718856\\
387.01	0.00377102383613311\\
388.01	0.00377101837034927\\
389.01	0.00377101277117559\\
390.01	0.0037710070351585\\
391.01	0.003771001158746\\
392.01	0.00377099513828446\\
393.01	0.0037709889700149\\
394.01	0.00377098265006942\\
395.01	0.00377097617446723\\
396.01	0.00377096953911052\\
397.01	0.00377096273978032\\
398.01	0.00377095577213189\\
399.01	0.00377094863169\\
400.01	0.00377094131384401\\
401.01	0.00377093381384256\\
402.01	0.00377092612678817\\
403.01	0.00377091824763121\\
404.01	0.00377091017116405\\
405.01	0.00377090189201435\\
406.01	0.00377089340463834\\
407.01	0.00377088470331359\\
408.01	0.00377087578213129\\
409.01	0.00377086663498825\\
410.01	0.00377085725557829\\
411.01	0.00377084763738311\\
412.01	0.00377083777366273\\
413.01	0.00377082765744519\\
414.01	0.0037708172815157\\
415.01	0.00377080663840508\\
416.01	0.00377079572037746\\
417.01	0.00377078451941714\\
418.01	0.00377077302721465\\
419.01	0.00377076123515187\\
420.01	0.00377074913428604\\
421.01	0.0037707367153328\\
422.01	0.003770723968648\\
423.01	0.00377071088420809\\
424.01	0.0037706974515894\\
425.01	0.00377068365994545\\
426.01	0.00377066949798302\\
427.01	0.00377065495393602\\
428.01	0.00377064001553759\\
429.01	0.00377062466998979\\
430.01	0.00377060890393099\\
431.01	0.00377059270340034\\
432.01	0.00377057605379947\\
433.01	0.00377055893985072\\
434.01	0.00377054134555156\\
435.01	0.00377052325412504\\
436.01	0.00377050464796542\\
437.01	0.00377048550857853\\
438.01	0.00377046581651641\\
439.01	0.00377044555130495\\
440.01	0.00377042469136421\\
441.01	0.00377040321391984\\
442.01	0.00377038109490479\\
443.01	0.00377035830884953\\
444.01	0.00377033482875936\\
445.01	0.00377031062597658\\
446.01	0.00377028567002527\\
447.01	0.0037702599284359\\
448.01	0.00377023336654646\\
449.01	0.00377020594727602\\
450.01	0.00377017763086607\\
451.01	0.00377014837458425\\
452.01	0.00377011813238344\\
453.01	0.00377008685450823\\
454.01	0.00377005448703902\\
455.01	0.00377002097136231\\
456.01	0.00376998624355294\\
457.01	0.0037699502336513\\
458.01	0.00376991286481851\\
459.01	0.0037698740523633\\
460.01	0.0037698337027361\\
461.01	0.00376979171303983\\
462.01	0.00376974797145467\\
463.01	0.00376970234428547\\
464.01	0.00376965461408954\\
465.01	0.00376960414450828\\
466.01	0.00376954711044049\\
467.01	0.00376946369118717\\
468.01	0.00376936110710564\\
469.01	0.00376925352522068\\
470.01	0.00376912119316176\\
471.01	0.00376888136306918\\
472.01	0.00376862514051226\\
473.01	0.00376836119045994\\
474.01	0.00376808652744747\\
475.01	0.00376780087669021\\
476.01	0.00376750749543886\\
477.01	0.00376720617289869\\
478.01	0.00376689660393025\\
479.01	0.00376657846271649\\
480.01	0.0037662514006514\\
481.01	0.0037659150439715\\
482.01	0.00376556899109603\\
483.01	0.00376521280963743\\
484.01	0.00376484603303844\\
485.01	0.00376446815678627\\
486.01	0.00376407863415389\\
487.01	0.00376367687143755\\
488.01	0.0037632622227638\\
489.01	0.00376283398453866\\
490.01	0.00376239138774652\\
491.01	0.00376193358974755\\
492.01	0.00376145966125149\\
493.01	0.00376096857969392\\
494.01	0.00376045922297468\\
495.01	0.00375993035225818\\
496.01	0.00375938059668709\\
497.01	0.00375880843395434\\
498.01	0.00375821215331006\\
499.01	0.00375758970411473\\
500.01	0.00375693769098714\\
501.01	0.00375624304836522\\
502.01	0.00375538283410615\\
503.01	0.00375388380446347\\
504.01	0.00375228680245878\\
505.01	0.00375063691587537\\
506.01	0.00374893144156919\\
507.01	0.00374716743881713\\
508.01	0.00374534169691954\\
509.01	0.00374345069696541\\
510.01	0.00374149056631271\\
511.01	0.00373945702259677\\
512.01	0.00373734529289359\\
513.01	0.00373514991110968\\
514.01	0.00373286365162165\\
515.01	0.00373046958084579\\
516.01	0.00372788180107596\\
517.01	0.00372490743254982\\
518.01	0.00372176839390527\\
519.01	0.00371849609881487\\
520.01	0.00371507009073484\\
521.01	0.00371137097610994\\
522.01	0.00370612686786882\\
523.01	0.00369939575242447\\
524.01	0.00369245046530768\\
525.01	0.00368527987059093\\
526.01	0.00367787189140608\\
527.01	0.00367021340166731\\
528.01	0.00366229010320468\\
529.01	0.00365408636436933\\
530.01	0.00364558505231754\\
531.01	0.00363676733036109\\
532.01	0.00362761208451302\\
533.01	0.00361809309921056\\
534.01	0.00360816454433315\\
535.01	0.00359776897808293\\
536.01	0.0035869023128831\\
537.01	0.00357551025994714\\
538.01	0.00356393476086114\\
539.01	0.00355182238397022\\
540.01	0.00353885783760149\\
541.01	0.00352207049789789\\
542.01	0.00349558486704233\\
543.01	0.00346850137793974\\
544.01	0.00344126642932281\\
545.01	0.00341313220386299\\
546.01	0.00338401104553813\\
547.01	0.0033538230294319\\
548.01	0.00332226370052044\\
549.01	0.00328792935660126\\
550.01	0.00325212899512041\\
551.01	0.00321710530789461\\
552.01	0.00318087389906738\\
553.01	0.00314334260828293\\
554.01	0.00310440778428242\\
555.01	0.00306395216153003\\
556.01	0.00302184109964779\\
557.01	0.00297790873028086\\
558.01	0.00293187370834995\\
559.01	0.00288288946532628\\
560.01	0.00283179093759702\\
561.01	0.00278149436602371\\
562.01	0.00269128150390921\\
563.01	0.00258475698378168\\
564.01	0.00247655079411002\\
565.01	0.0023686312360737\\
566.01	0.00225884532024653\\
567.01	0.00214636471298463\\
568.01	0.00203101614877091\\
569.01	0.00191260384401076\\
570.01	0.00179089309187064\\
571.01	0.00166546488008717\\
572.01	0.00153473123580838\\
573.01	0.00139480841276273\\
574.01	0.00127843382944641\\
575.01	0.00116237717509533\\
576.01	0.00104405713628715\\
577.01	0.000923474008441835\\
578.01	0.000800467496979472\\
579.01	0.000674981078808047\\
580.01	0.000547004326546188\\
581.01	0.000416546308886746\\
582.01	0.00028365363948627\\
583.01	0.000148672106345117\\
584.01	2.10202085324119e-05\\
585.01	0\\
586.01	0\\
587.01	0\\
588.01	0\\
589.01	0\\
590.01	0\\
591.01	0\\
592.01	1.73472347597681e-18\\
593.01	0\\
594.01	0\\
595.01	0\\
596.01	0\\
597.01	0\\
598.01	0\\
599.01	0\\
599.02	0\\
599.03	1.73472347597681e-18\\
599.04	1.73472347597681e-18\\
599.05	0\\
599.06	0\\
599.07	1.73472347597681e-18\\
599.08	0\\
599.09	0\\
599.1	0\\
599.11	1.73472347597681e-18\\
599.12	1.73472347597681e-18\\
599.13	0\\
599.14	0\\
599.15	1.73472347597681e-18\\
599.16	1.73472347597681e-18\\
599.17	1.73472347597681e-18\\
599.18	0\\
599.19	0\\
599.2	0\\
599.21	0\\
599.22	0\\
599.23	1.73472347597681e-18\\
599.24	1.73472347597681e-18\\
599.25	1.73472347597681e-18\\
599.26	1.73472347597681e-18\\
599.27	0\\
599.28	0\\
599.29	0\\
599.3	0\\
599.31	0\\
599.32	1.73472347597681e-18\\
599.33	1.73472347597681e-18\\
599.34	0\\
599.35	0\\
599.36	0\\
599.37	1.73472347597681e-18\\
599.38	0\\
599.39	0\\
599.4	0\\
599.41	0\\
599.42	0\\
599.43	0\\
599.44	0\\
599.45	0\\
599.46	1.73472347597681e-18\\
599.47	1.73472347597681e-18\\
599.48	1.73472347597681e-18\\
599.49	1.73472347597681e-18\\
599.5	1.73472347597681e-18\\
599.51	0\\
599.52	0\\
599.53	1.73472347597681e-18\\
599.54	0\\
599.55	0\\
599.56	0\\
599.57	1.73472347597681e-18\\
599.58	0\\
599.59	0\\
599.6	0\\
599.61	0\\
599.62	1.73472347597681e-18\\
599.63	0\\
599.64	1.73472347597681e-18\\
599.65	0\\
599.66	0\\
599.67	0\\
599.68	0\\
599.69	1.73472347597681e-18\\
599.7	0\\
599.71	1.73472347597681e-18\\
599.72	1.73472347597681e-18\\
599.73	1.73472347597681e-18\\
599.74	0\\
599.75	0\\
599.76	0\\
599.77	0\\
599.78	0\\
599.79	0\\
599.8	0\\
599.81	0\\
599.82	0\\
599.83	1.73472347597681e-18\\
599.84	0\\
599.85	0\\
599.86	0\\
599.87	0\\
599.88	0\\
599.89	0\\
599.9	0\\
599.91	0\\
599.92	0\\
599.93	0\\
599.94	0\\
599.95	0\\
599.96	0\\
599.97	0\\
599.98	0\\
599.99	0\\
600	0\\
};
\addplot [color=mycolor11,solid,forget plot]
  table[row sep=crcr]{%
0.01	0.00628526379179407\\
1.01	0.00628526379170105\\
2.01	0.00628526379160609\\
3.01	0.00628526379150912\\
4.01	0.00628526379141011\\
5.01	0.00628526379130903\\
6.01	0.0062852637912058\\
7.01	0.00628526379110041\\
8.01	0.0062852637909928\\
9.01	0.00628526379088292\\
10.01	0.00628526379077073\\
11.01	0.00628526379065617\\
12.01	0.0062852637905392\\
13.01	0.00628526379041977\\
14.01	0.00628526379029781\\
15.01	0.0062852637901733\\
16.01	0.00628526379004615\\
17.01	0.00628526378991631\\
18.01	0.00628526378978375\\
19.01	0.00628526378964839\\
20.01	0.00628526378951017\\
21.01	0.00628526378936903\\
22.01	0.00628526378922492\\
23.01	0.00628526378907776\\
24.01	0.00628526378892749\\
25.01	0.00628526378877405\\
26.01	0.00628526378861738\\
27.01	0.00628526378845739\\
28.01	0.00628526378829402\\
29.01	0.0062852637881272\\
30.01	0.00628526378795685\\
31.01	0.00628526378778291\\
32.01	0.00628526378760528\\
33.01	0.0062852637874239\\
34.01	0.00628526378723868\\
35.01	0.00628526378704954\\
36.01	0.00628526378685641\\
37.01	0.00628526378665918\\
38.01	0.00628526378645778\\
39.01	0.00628526378625211\\
40.01	0.00628526378604209\\
41.01	0.00628526378582762\\
42.01	0.00628526378560861\\
43.01	0.00628526378538495\\
44.01	0.00628526378515656\\
45.01	0.00628526378492332\\
46.01	0.00628526378468514\\
47.01	0.00628526378444191\\
48.01	0.00628526378419352\\
49.01	0.00628526378393985\\
50.01	0.00628526378368081\\
51.01	0.00628526378341627\\
52.01	0.0062852637831461\\
53.01	0.00628526378287021\\
54.01	0.00628526378258845\\
55.01	0.00628526378230071\\
56.01	0.00628526378200686\\
57.01	0.00628526378170676\\
58.01	0.00628526378140028\\
59.01	0.00628526378108728\\
60.01	0.00628526378076763\\
61.01	0.00628526378044117\\
62.01	0.00628526378010777\\
63.01	0.00628526377976728\\
64.01	0.00628526377941954\\
65.01	0.00628526377906439\\
66.01	0.00628526377870168\\
67.01	0.00628526377833124\\
68.01	0.00628526377795291\\
69.01	0.00628526377756652\\
70.01	0.00628526377717189\\
71.01	0.00628526377676884\\
72.01	0.00628526377635719\\
73.01	0.00628526377593677\\
74.01	0.00628526377550737\\
75.01	0.00628526377506882\\
76.01	0.00628526377462089\\
77.01	0.00628526377416341\\
78.01	0.00628526377369614\\
79.01	0.00628526377321889\\
80.01	0.00628526377273145\\
81.01	0.00628526377223358\\
82.01	0.00628526377172507\\
83.01	0.00628526377120568\\
84.01	0.00628526377067519\\
85.01	0.00628526377013333\\
86.01	0.00628526376957987\\
87.01	0.00628526376901458\\
88.01	0.00628526376843716\\
89.01	0.00628526376784738\\
90.01	0.00628526376724496\\
91.01	0.00628526376662963\\
92.01	0.0062852637660011\\
93.01	0.0062852637653591\\
94.01	0.00628526376470331\\
95.01	0.00628526376403345\\
96.01	0.00628526376334921\\
97.01	0.00628526376265028\\
98.01	0.00628526376193635\\
99.01	0.00628526376120706\\
100.01	0.00628526376046211\\
101.01	0.00628526375970114\\
102.01	0.00628526375892381\\
103.01	0.00628526375812975\\
104.01	0.0062852637573186\\
105.01	0.00628526375649001\\
106.01	0.00628526375564357\\
107.01	0.00628526375477889\\
108.01	0.00628526375389559\\
109.01	0.00628526375299326\\
110.01	0.00628526375207147\\
111.01	0.00628526375112981\\
112.01	0.00628526375016782\\
113.01	0.00628526374918508\\
114.01	0.00628526374818112\\
115.01	0.00628526374715549\\
116.01	0.0062852637461077\\
117.01	0.00628526374503727\\
118.01	0.00628526374394372\\
119.01	0.00628526374282652\\
120.01	0.00628526374168514\\
121.01	0.00628526374051908\\
122.01	0.00628526373932778\\
123.01	0.00628526373811068\\
124.01	0.00628526373686723\\
125.01	0.00628526373559685\\
126.01	0.00628526373429891\\
127.01	0.00628526373297285\\
128.01	0.00628526373161802\\
129.01	0.0062852637302338\\
130.01	0.00628526372881954\\
131.01	0.00628526372737458\\
132.01	0.00628526372589823\\
133.01	0.00628526372438981\\
134.01	0.0062852637228486\\
135.01	0.00628526372127384\\
136.01	0.00628526371966487\\
137.01	0.00628526371802091\\
138.01	0.00628526371634115\\
139.01	0.00628526371462481\\
140.01	0.00628526371287109\\
141.01	0.00628526371107915\\
142.01	0.00628526370924815\\
143.01	0.00628526370737723\\
144.01	0.00628526370546549\\
145.01	0.00628526370351202\\
146.01	0.00628526370151591\\
147.01	0.0062852636994762\\
148.01	0.00628526369739193\\
149.01	0.0062852636952621\\
150.01	0.00628526369308571\\
151.01	0.00628526369086171\\
152.01	0.00628526368858903\\
153.01	0.00628526368626661\\
154.01	0.00628526368389333\\
155.01	0.00628526368146806\\
156.01	0.00628526367898962\\
157.01	0.00628526367645685\\
158.01	0.00628526367386852\\
159.01	0.00628526367122339\\
160.01	0.0062852636685202\\
161.01	0.00628526366575764\\
162.01	0.0062852636629344\\
163.01	0.00628526366004909\\
164.01	0.00628526365710035\\
165.01	0.00628526365408674\\
166.01	0.00628526365100682\\
167.01	0.00628526364785907\\
168.01	0.00628526364464202\\
169.01	0.0062852636413541\\
170.01	0.00628526363799368\\
171.01	0.00628526363455917\\
172.01	0.00628526363104889\\
173.01	0.00628526362746113\\
174.01	0.00628526362379417\\
175.01	0.00628526362004621\\
176.01	0.00628526361621543\\
177.01	0.00628526361229996\\
178.01	0.0062852636082979\\
179.01	0.00628526360420729\\
180.01	0.00628526360002613\\
181.01	0.00628526359575239\\
182.01	0.00628526359138397\\
183.01	0.00628526358691873\\
184.01	0.00628526358235449\\
185.01	0.00628526357768902\\
186.01	0.00628526357291998\\
187.01	0.0062852635680451\\
188.01	0.00628526356306194\\
189.01	0.00628526355796806\\
190.01	0.00628526355276094\\
191.01	0.00628526354743804\\
192.01	0.00628526354199671\\
193.01	0.00628526353643427\\
194.01	0.00628526353074799\\
195.01	0.00628526352493503\\
196.01	0.00628526351899252\\
197.01	0.00628526351291754\\
198.01	0.00628526350670707\\
199.01	0.00628526350035799\\
200.01	0.00628526349386721\\
201.01	0.00628526348723147\\
202.01	0.00628526348044748\\
203.01	0.00628526347351186\\
204.01	0.00628526346642117\\
205.01	0.00628526345917185\\
206.01	0.00628526345176031\\
207.01	0.00628526344418285\\
208.01	0.00628526343643567\\
209.01	0.0062852634285149\\
210.01	0.00628526342041658\\
211.01	0.00628526341213667\\
212.01	0.00628526340367101\\
213.01	0.00628526339501537\\
214.01	0.00628526338616539\\
215.01	0.00628526337711664\\
216.01	0.00628526336786457\\
217.01	0.00628526335840451\\
218.01	0.00628526334873175\\
219.01	0.00628526333884138\\
220.01	0.00628526332872843\\
221.01	0.00628526331838782\\
222.01	0.00628526330781432\\
223.01	0.00628526329700258\\
224.01	0.00628526328594717\\
225.01	0.0062852632746425\\
226.01	0.00628526326308284\\
227.01	0.00628526325126236\\
228.01	0.00628526323917507\\
229.01	0.00628526322681484\\
230.01	0.00628526321417543\\
231.01	0.00628526320125042\\
232.01	0.00628526318803326\\
233.01	0.00628526317451723\\
234.01	0.00628526316069548\\
235.01	0.006285263146561\\
236.01	0.00628526313210659\\
237.01	0.00628526311732491\\
238.01	0.00628526310220844\\
239.01	0.00628526308674948\\
240.01	0.00628526307094018\\
241.01	0.00628526305477247\\
242.01	0.00628526303823814\\
243.01	0.00628526302132872\\
244.01	0.00628526300403563\\
245.01	0.00628526298635002\\
246.01	0.0062852629682629\\
247.01	0.00628526294976503\\
248.01	0.00628526293084694\\
249.01	0.006285262911499\\
250.01	0.00628526289171131\\
251.01	0.00628526287147376\\
252.01	0.00628526285077603\\
253.01	0.00628526282960753\\
254.01	0.00628526280795741\\
255.01	0.00628526278581463\\
256.01	0.00628526276316786\\
257.01	0.00628526274000552\\
258.01	0.00628526271631574\\
259.01	0.00628526269208641\\
260.01	0.00628526266730512\\
261.01	0.0062852626419592\\
262.01	0.00628526261603565\\
263.01	0.00628526258952125\\
264.01	0.00628526256240237\\
265.01	0.00628526253466515\\
266.01	0.00628526250629539\\
267.01	0.00628526247727856\\
268.01	0.00628526244759979\\
269.01	0.00628526241724389\\
270.01	0.00628526238619531\\
271.01	0.00628526235443817\\
272.01	0.00628526232195617\\
273.01	0.00628526228873271\\
274.01	0.00628526225475076\\
275.01	0.00628526221999294\\
276.01	0.00628526218444143\\
277.01	0.00628526214807805\\
278.01	0.00628526211088416\\
279.01	0.00628526207284073\\
280.01	0.00628526203392831\\
281.01	0.00628526199412694\\
282.01	0.00628526195341628\\
283.01	0.00628526191177548\\
284.01	0.00628526186918324\\
285.01	0.00628526182561777\\
286.01	0.00628526178105677\\
287.01	0.00628526173547744\\
288.01	0.00628526168885647\\
289.01	0.00628526164117002\\
290.01	0.00628526159239368\\
291.01	0.00628526154250251\\
292.01	0.00628526149147098\\
293.01	0.00628526143927302\\
294.01	0.00628526138588189\\
295.01	0.00628526133127032\\
296.01	0.00628526127541035\\
297.01	0.00628526121827342\\
298.01	0.00628526115983031\\
299.01	0.00628526110005111\\
300.01	0.00628526103890524\\
301.01	0.00628526097636141\\
302.01	0.00628526091238761\\
303.01	0.0062852608469511\\
304.01	0.00628526078001837\\
305.01	0.00628526071155514\\
306.01	0.00628526064152635\\
307.01	0.00628526056989611\\
308.01	0.00628526049662768\\
309.01	0.0062852604216835\\
310.01	0.00628526034502512\\
311.01	0.00628526026661318\\
312.01	0.00628526018640739\\
313.01	0.00628526010436657\\
314.01	0.00628526002044848\\
315.01	0.00628525993460997\\
316.01	0.00628525984680679\\
317.01	0.0062852597569937\\
318.01	0.00628525966512435\\
319.01	0.0062852595711513\\
320.01	0.00628525947502597\\
321.01	0.00628525937669862\\
322.01	0.00628525927611828\\
323.01	0.00628525917323279\\
324.01	0.00628525906798871\\
325.01	0.00628525896033128\\
326.01	0.00628525885020445\\
327.01	0.00628525873755075\\
328.01	0.00628525862231133\\
329.01	0.00628525850442589\\
330.01	0.00628525838383262\\
331.01	0.00628525826046818\\
332.01	0.00628525813426766\\
333.01	0.00628525800516456\\
334.01	0.00628525787309066\\
335.01	0.00628525773797607\\
336.01	0.00628525759974909\\
337.01	0.00628525745833627\\
338.01	0.00628525731366225\\
339.01	0.00628525716564976\\
340.01	0.00628525701421957\\
341.01	0.00628525685929041\\
342.01	0.00628525670077893\\
343.01	0.00628525653859963\\
344.01	0.00628525637266481\\
345.01	0.00628525620288447\\
346.01	0.00628525602916632\\
347.01	0.00628525585141562\\
348.01	0.00628525566953522\\
349.01	0.00628525548342537\\
350.01	0.00628525529298375\\
351.01	0.00628525509810531\\
352.01	0.00628525489868229\\
353.01	0.00628525469460406\\
354.01	0.00628525448575706\\
355.01	0.00628525427202476\\
356.01	0.00628525405328753\\
357.01	0.00628525382942255\\
358.01	0.00628525360030377\\
359.01	0.00628525336580178\\
360.01	0.00628525312578374\\
361.01	0.00628525288011325\\
362.01	0.00628525262865029\\
363.01	0.00628525237125112\\
364.01	0.00628525210776814\\
365.01	0.00628525183804985\\
366.01	0.00628525156194069\\
367.01	0.00628525127928093\\
368.01	0.00628525098990664\\
369.01	0.00628525069364946\\
370.01	0.00628525039033662\\
371.01	0.0062852500797907\\
372.01	0.00628524976182957\\
373.01	0.00628524943626634\\
374.01	0.00628524910290913\\
375.01	0.00628524876156111\\
376.01	0.00628524841202061\\
377.01	0.00628524805408113\\
378.01	0.00628524768752958\\
379.01	0.0062852473121461\\
380.01	0.00628524692770632\\
381.01	0.00628524653398002\\
382.01	0.00628524613073069\\
383.01	0.00628524571771536\\
384.01	0.00628524529468427\\
385.01	0.00628524486138085\\
386.01	0.00628524441754133\\
387.01	0.00628524396289458\\
388.01	0.00628524349716187\\
389.01	0.00628524302005659\\
390.01	0.00628524253128397\\
391.01	0.0062852420305409\\
392.01	0.00628524151751552\\
393.01	0.00628524099188693\\
394.01	0.00628524045332497\\
395.01	0.00628523990148974\\
396.01	0.00628523933603142\\
397.01	0.00628523875658973\\
398.01	0.00628523816279363\\
399.01	0.00628523755426096\\
400.01	0.00628523693059789\\
401.01	0.00628523629139856\\
402.01	0.00628523563624455\\
403.01	0.00628523496470447\\
404.01	0.0062852342763333\\
405.01	0.00628523357067193\\
406.01	0.00628523284724651\\
407.01	0.00628523210556788\\
408.01	0.00628523134513089\\
409.01	0.00628523056541366\\
410.01	0.0062852297658769\\
411.01	0.00628522894596309\\
412.01	0.00628522810509564\\
413.01	0.00628522724267802\\
414.01	0.00628522635809286\\
415.01	0.00628522545070084\\
416.01	0.0062852245198397\\
417.01	0.00628522356482316\\
418.01	0.00628522258493956\\
419.01	0.0062852215794507\\
420.01	0.0062852205475904\\
421.01	0.00628521948856302\\
422.01	0.0062852184015419\\
423.01	0.00628521728566767\\
424.01	0.00628521614004638\\
425.01	0.00628521496374766\\
426.01	0.00628521375580245\\
427.01	0.00628521251520094\\
428.01	0.00628521124088995\\
429.01	0.00628520993177042\\
430.01	0.00628520858669443\\
431.01	0.00628520720446228\\
432.01	0.00628520578381897\\
433.01	0.00628520432345067\\
434.01	0.00628520282198063\\
435.01	0.0062852012779649\\
436.01	0.00628519968988755\\
437.01	0.00628519805615545\\
438.01	0.00628519637509245\\
439.01	0.00628519464493309\\
440.01	0.00628519286381552\\
441.01	0.00628519102977374\\
442.01	0.00628518914072883\\
443.01	0.00628518719447931\\
444.01	0.0062851851886902\\
445.01	0.00628518312088083\\
446.01	0.00628518098841099\\
447.01	0.00628517878846539\\
448.01	0.00628517651803577\\
449.01	0.00628517417390082\\
450.01	0.00628517175260294\\
451.01	0.00628516925042169\\
452.01	0.00628516666334323\\
453.01	0.00628516398702483\\
454.01	0.00628516121675388\\
455.01	0.0062851583474001\\
456.01	0.00628515537335966\\
457.01	0.00628515228849003\\
458.01	0.00628514908603428\\
459.01	0.00628514575853754\\
460.01	0.00628514229777216\\
461.01	0.00628513869468703\\
462.01	0.0062851349389533\\
463.01	0.00628513101533705\\
464.01	0.0062851268862169\\
465.01	0.00628512239141112\\
466.01	0.00628511682057988\\
467.01	0.00628510891601267\\
468.01	0.00628510004372786\\
469.01	0.00628509011464676\\
470.01	0.00628507547351595\\
471.01	0.00628505436363106\\
472.01	0.00628503251003922\\
473.01	0.00628500992174695\\
474.01	0.00628498638191543\\
475.01	0.00628496205009947\\
476.01	0.00628493706343639\\
477.01	0.00628491139816609\\
478.01	0.00628488502794176\\
479.01	0.00628485792460767\\
480.01	0.00628483005801303\\
481.01	0.00628480139580312\\
482.01	0.00628477190318461\\
483.01	0.00628474154266182\\
484.01	0.00628471027373981\\
485.01	0.00628467805259078\\
486.01	0.00628464483168046\\
487.01	0.00628461055935842\\
488.01	0.00628457517942485\\
489.01	0.00628453863062311\\
490.01	0.00628450084588446\\
491.01	0.00628446175155596\\
492.01	0.00628442126618958\\
493.01	0.00628437930045945\\
494.01	0.00628433575613927\\
495.01	0.00628429052466838\\
496.01	0.00628424348566458\\
497.01	0.00628419450420187\\
498.01	0.00628414342030574\\
499.01	0.00628408998362221\\
500.01	0.00628403340176519\\
501.01	0.00628396934316834\\
502.01	0.00628387594674586\\
503.01	0.00628374474219451\\
504.01	0.00628360861067663\\
505.01	0.00628346796240913\\
506.01	0.00628332256427076\\
507.01	0.0062831721622902\\
508.01	0.00628301647875245\\
509.01	0.00628285520876357\\
510.01	0.00628268801605152\\
511.01	0.00628251452706827\\
512.01	0.00628233431759219\\
513.01	0.00628214685325238\\
514.01	0.00628195114215573\\
515.01	0.00628174387613744\\
516.01	0.00628151401165013\\
517.01	0.00628125786679415\\
518.01	0.00628099012826855\\
519.01	0.00628071031397971\\
520.01	0.00628041243823779\\
521.01	0.0062800628102503\\
522.01	0.00627955818186459\\
523.01	0.00627898773279302\\
524.01	0.00627839941801704\\
525.01	0.00627779231594755\\
526.01	0.00627716542706608\\
527.01	0.00627651766509509\\
528.01	0.00627584784634711\\
529.01	0.00627515467492565\\
530.01	0.00627443672923637\\
531.01	0.00627369243125106\\
532.01	0.00627291992734269\\
533.01	0.00627211658830004\\
534.01	0.00627127799423567\\
535.01	0.00627040002835317\\
536.01	0.0062694664043186\\
537.01	0.00626831754531038\\
538.01	0.00626649145606629\\
539.01	0.00626451741086056\\
540.01	0.0062623421422299\\
541.01	0.00625954994824356\\
542.01	0.00625607368878011\\
543.01	0.00625117487701888\\
544.01	0.00624511955552116\\
545.01	0.00623882134227631\\
546.01	0.00623225770111419\\
547.01	0.00622539543531563\\
548.01	0.00621814372336758\\
549.01	0.00621017509655851\\
550.01	0.00619919570469964\\
551.01	0.00618475779904904\\
552.01	0.00616977537202722\\
553.01	0.00615420605909365\\
554.01	0.00613800225184393\\
555.01	0.00612111014558708\\
556.01	0.00610346788073663\\
557.01	0.00608499731021465\\
558.01	0.00606554234287161\\
559.01	0.00604431311954412\\
560.01	0.00601401718810351\\
561.01	0.00597000049674904\\
562.01	0.00592183900587292\\
563.01	0.00587175812428921\\
564.01	0.00582006423633348\\
565.01	0.00576668023270784\\
566.01	0.0057113088810085\\
567.01	0.00565376273732039\\
568.01	0.00559384842142432\\
569.01	0.00553134735277533\\
570.01	0.00546601127051573\\
571.01	0.00539748147462386\\
572.01	0.00532425881929858\\
573.01	0.00523223672153153\\
574.01	0.00508205373471091\\
575.01	0.00492532175758901\\
576.01	0.00476471806081075\\
577.01	0.004600083778585\\
578.01	0.00443145190274027\\
579.01	0.00425857178856824\\
580.01	0.00408110912050806\\
581.01	0.00389868410416532\\
582.01	0.00371086719773854\\
583.01	0.00351722652953084\\
584.01	0.00331751222637499\\
585.01	0.00308986283227971\\
586.01	0.00292716497304702\\
587.01	0.00278898404456788\\
588.01	0.00264703419695185\\
589.01	0.00250118297809676\\
590.01	0.0023512946155854\\
591.01	0.00219724220680509\\
592.01	0.00203890843405107\\
593.01	0.0018761942862383\\
594.01	0.00170902362200746\\
595.01	0.00153731355151368\\
596.01	0.00136061732988777\\
597.01	0.00117486750438617\\
598.01	0.000944118809282362\\
599.01	0.000412707488818912\\
599.02	0.000403763707257631\\
599.03	0.000394738618952119\\
599.04	0.000385633799629855\\
599.05	0.000376451016479014\\
599.06	0.000367192239998626\\
599.07	0.000357859656498377\\
599.08	0.000348455681284034\\
599.09	0.000338982972567762\\
599.1	0.00032944444613737\\
599.11	0.000319843290829534\\
599.12	0.000310182984851588\\
599.13	0.000300467312994089\\
599.14	0.000290700384783934\\
599.15	0.000280886653628904\\
599.16	0.000271030937007258\\
599.17	0.000261138437758839\\
599.18	0.000251214766537227\\
599.19	0.000241265965485694\\
599.2	0.000231298533203026\\
599.21	0.00022132044869011\\
599.22	0.000211340509823443\\
599.23	0.000201366838978293\\
599.24	0.000191408133016347\\
599.25	0.000181474851944197\\
599.26	0.000171581039711705\\
599.27	0.000161737538486348\\
599.28	0.000151963437707387\\
599.29	0.000142273737523882\\
599.3	0.000132693943086811\\
599.31	0.000123249963771816\\
599.32	0.00011395817999805\\
599.33	0.000104835993820664\\
599.34	9.59018791959702e-05\\
599.35	8.7175440601309e-05\\
599.36	7.8677474740033e-05\\
599.37	7.04300354867297e-05\\
599.38	6.24565022345654e-05\\
599.39	5.47816518136814e-05\\
599.4	4.74317341564879e-05\\
599.41	4.04345518931083e-05\\
599.42	3.38517737233143e-05\\
599.43	2.77322506769592e-05\\
599.44	2.21111563562509e-05\\
599.45	1.70256734628362e-05\\
599.46	1.25150998088776e-05\\
599.47	8.62095959768341e-06\\
599.48	5.38712020206437e-06\\
599.49	2.8585274958904e-06\\
599.5	1.0833850832611e-06\\
599.51	1.13220873875983e-07\\
599.52	0\\
599.53	1.73472347597681e-18\\
599.54	0\\
599.55	0\\
599.56	0\\
599.57	1.73472347597681e-18\\
599.58	0\\
599.59	0\\
599.6	0\\
599.61	0\\
599.62	1.73472347597681e-18\\
599.63	0\\
599.64	1.73472347597681e-18\\
599.65	0\\
599.66	0\\
599.67	0\\
599.68	0\\
599.69	1.73472347597681e-18\\
599.7	0\\
599.71	1.73472347597681e-18\\
599.72	1.73472347597681e-18\\
599.73	1.73472347597681e-18\\
599.74	0\\
599.75	0\\
599.76	0\\
599.77	0\\
599.78	0\\
599.79	0\\
599.8	0\\
599.81	0\\
599.82	0\\
599.83	1.73472347597681e-18\\
599.84	0\\
599.85	0\\
599.86	0\\
599.87	0\\
599.88	0\\
599.89	0\\
599.9	0\\
599.91	0\\
599.92	0\\
599.93	0\\
599.94	0\\
599.95	0\\
599.96	0\\
599.97	0\\
599.98	0\\
599.99	0\\
600	0\\
};
\addplot [color=mycolor12,solid,forget plot]
  table[row sep=crcr]{%
0.01	0.01\\
1.01	0.01\\
2.01	0.01\\
3.01	0.01\\
4.01	0.01\\
5.01	0.01\\
6.01	0.01\\
7.01	0.01\\
8.01	0.01\\
9.01	0.01\\
10.01	0.01\\
11.01	0.01\\
12.01	0.01\\
13.01	0.01\\
14.01	0.01\\
15.01	0.01\\
16.01	0.01\\
17.01	0.01\\
18.01	0.01\\
19.01	0.01\\
20.01	0.01\\
21.01	0.01\\
22.01	0.01\\
23.01	0.01\\
24.01	0.01\\
25.01	0.01\\
26.01	0.01\\
27.01	0.01\\
28.01	0.01\\
29.01	0.01\\
30.01	0.01\\
31.01	0.01\\
32.01	0.01\\
33.01	0.01\\
34.01	0.01\\
35.01	0.01\\
36.01	0.01\\
37.01	0.01\\
38.01	0.01\\
39.01	0.01\\
40.01	0.01\\
41.01	0.01\\
42.01	0.01\\
43.01	0.01\\
44.01	0.01\\
45.01	0.01\\
46.01	0.01\\
47.01	0.01\\
48.01	0.01\\
49.01	0.01\\
50.01	0.01\\
51.01	0.01\\
52.01	0.01\\
53.01	0.01\\
54.01	0.01\\
55.01	0.01\\
56.01	0.01\\
57.01	0.01\\
58.01	0.01\\
59.01	0.01\\
60.01	0.01\\
61.01	0.01\\
62.01	0.01\\
63.01	0.01\\
64.01	0.01\\
65.01	0.01\\
66.01	0.01\\
67.01	0.01\\
68.01	0.01\\
69.01	0.01\\
70.01	0.01\\
71.01	0.01\\
72.01	0.01\\
73.01	0.01\\
74.01	0.01\\
75.01	0.01\\
76.01	0.01\\
77.01	0.01\\
78.01	0.01\\
79.01	0.01\\
80.01	0.01\\
81.01	0.01\\
82.01	0.01\\
83.01	0.01\\
84.01	0.01\\
85.01	0.01\\
86.01	0.01\\
87.01	0.01\\
88.01	0.01\\
89.01	0.01\\
90.01	0.01\\
91.01	0.01\\
92.01	0.01\\
93.01	0.01\\
94.01	0.01\\
95.01	0.01\\
96.01	0.01\\
97.01	0.01\\
98.01	0.01\\
99.01	0.01\\
100.01	0.01\\
101.01	0.01\\
102.01	0.01\\
103.01	0.01\\
104.01	0.01\\
105.01	0.01\\
106.01	0.01\\
107.01	0.01\\
108.01	0.01\\
109.01	0.01\\
110.01	0.01\\
111.01	0.01\\
112.01	0.01\\
113.01	0.01\\
114.01	0.01\\
115.01	0.01\\
116.01	0.01\\
117.01	0.01\\
118.01	0.01\\
119.01	0.01\\
120.01	0.01\\
121.01	0.01\\
122.01	0.01\\
123.01	0.01\\
124.01	0.01\\
125.01	0.01\\
126.01	0.01\\
127.01	0.01\\
128.01	0.01\\
129.01	0.01\\
130.01	0.01\\
131.01	0.01\\
132.01	0.01\\
133.01	0.01\\
134.01	0.01\\
135.01	0.01\\
136.01	0.01\\
137.01	0.01\\
138.01	0.01\\
139.01	0.01\\
140.01	0.01\\
141.01	0.01\\
142.01	0.01\\
143.01	0.01\\
144.01	0.01\\
145.01	0.01\\
146.01	0.01\\
147.01	0.01\\
148.01	0.01\\
149.01	0.01\\
150.01	0.01\\
151.01	0.01\\
152.01	0.01\\
153.01	0.01\\
154.01	0.01\\
155.01	0.01\\
156.01	0.01\\
157.01	0.01\\
158.01	0.01\\
159.01	0.01\\
160.01	0.01\\
161.01	0.01\\
162.01	0.01\\
163.01	0.01\\
164.01	0.01\\
165.01	0.01\\
166.01	0.01\\
167.01	0.01\\
168.01	0.01\\
169.01	0.01\\
170.01	0.01\\
171.01	0.01\\
172.01	0.01\\
173.01	0.01\\
174.01	0.01\\
175.01	0.01\\
176.01	0.01\\
177.01	0.01\\
178.01	0.01\\
179.01	0.01\\
180.01	0.01\\
181.01	0.01\\
182.01	0.01\\
183.01	0.01\\
184.01	0.01\\
185.01	0.01\\
186.01	0.01\\
187.01	0.01\\
188.01	0.01\\
189.01	0.01\\
190.01	0.01\\
191.01	0.01\\
192.01	0.01\\
193.01	0.01\\
194.01	0.01\\
195.01	0.01\\
196.01	0.01\\
197.01	0.01\\
198.01	0.01\\
199.01	0.01\\
200.01	0.01\\
201.01	0.01\\
202.01	0.01\\
203.01	0.01\\
204.01	0.01\\
205.01	0.01\\
206.01	0.01\\
207.01	0.01\\
208.01	0.01\\
209.01	0.01\\
210.01	0.01\\
211.01	0.01\\
212.01	0.01\\
213.01	0.01\\
214.01	0.01\\
215.01	0.01\\
216.01	0.01\\
217.01	0.01\\
218.01	0.01\\
219.01	0.01\\
220.01	0.01\\
221.01	0.01\\
222.01	0.01\\
223.01	0.01\\
224.01	0.01\\
225.01	0.01\\
226.01	0.01\\
227.01	0.01\\
228.01	0.01\\
229.01	0.01\\
230.01	0.01\\
231.01	0.01\\
232.01	0.01\\
233.01	0.01\\
234.01	0.01\\
235.01	0.01\\
236.01	0.01\\
237.01	0.01\\
238.01	0.01\\
239.01	0.01\\
240.01	0.01\\
241.01	0.01\\
242.01	0.01\\
243.01	0.01\\
244.01	0.01\\
245.01	0.01\\
246.01	0.01\\
247.01	0.01\\
248.01	0.01\\
249.01	0.01\\
250.01	0.01\\
251.01	0.01\\
252.01	0.01\\
253.01	0.01\\
254.01	0.01\\
255.01	0.01\\
256.01	0.01\\
257.01	0.01\\
258.01	0.01\\
259.01	0.01\\
260.01	0.01\\
261.01	0.01\\
262.01	0.01\\
263.01	0.01\\
264.01	0.01\\
265.01	0.01\\
266.01	0.01\\
267.01	0.01\\
268.01	0.01\\
269.01	0.01\\
270.01	0.01\\
271.01	0.01\\
272.01	0.01\\
273.01	0.01\\
274.01	0.01\\
275.01	0.01\\
276.01	0.01\\
277.01	0.01\\
278.01	0.01\\
279.01	0.01\\
280.01	0.01\\
281.01	0.01\\
282.01	0.01\\
283.01	0.01\\
284.01	0.01\\
285.01	0.01\\
286.01	0.01\\
287.01	0.01\\
288.01	0.01\\
289.01	0.01\\
290.01	0.01\\
291.01	0.01\\
292.01	0.01\\
293.01	0.01\\
294.01	0.01\\
295.01	0.01\\
296.01	0.01\\
297.01	0.01\\
298.01	0.01\\
299.01	0.01\\
300.01	0.01\\
301.01	0.01\\
302.01	0.01\\
303.01	0.01\\
304.01	0.01\\
305.01	0.01\\
306.01	0.01\\
307.01	0.01\\
308.01	0.01\\
309.01	0.01\\
310.01	0.01\\
311.01	0.01\\
312.01	0.01\\
313.01	0.01\\
314.01	0.01\\
315.01	0.01\\
316.01	0.01\\
317.01	0.01\\
318.01	0.01\\
319.01	0.01\\
320.01	0.01\\
321.01	0.01\\
322.01	0.01\\
323.01	0.01\\
324.01	0.01\\
325.01	0.01\\
326.01	0.01\\
327.01	0.01\\
328.01	0.01\\
329.01	0.01\\
330.01	0.01\\
331.01	0.01\\
332.01	0.01\\
333.01	0.01\\
334.01	0.01\\
335.01	0.01\\
336.01	0.01\\
337.01	0.01\\
338.01	0.01\\
339.01	0.01\\
340.01	0.01\\
341.01	0.01\\
342.01	0.01\\
343.01	0.01\\
344.01	0.01\\
345.01	0.01\\
346.01	0.01\\
347.01	0.01\\
348.01	0.01\\
349.01	0.01\\
350.01	0.01\\
351.01	0.01\\
352.01	0.01\\
353.01	0.01\\
354.01	0.01\\
355.01	0.01\\
356.01	0.01\\
357.01	0.01\\
358.01	0.01\\
359.01	0.01\\
360.01	0.01\\
361.01	0.01\\
362.01	0.01\\
363.01	0.01\\
364.01	0.01\\
365.01	0.01\\
366.01	0.01\\
367.01	0.01\\
368.01	0.01\\
369.01	0.01\\
370.01	0.01\\
371.01	0.01\\
372.01	0.01\\
373.01	0.01\\
374.01	0.01\\
375.01	0.01\\
376.01	0.01\\
377.01	0.01\\
378.01	0.01\\
379.01	0.01\\
380.01	0.01\\
381.01	0.01\\
382.01	0.01\\
383.01	0.01\\
384.01	0.01\\
385.01	0.01\\
386.01	0.01\\
387.01	0.01\\
388.01	0.01\\
389.01	0.01\\
390.01	0.01\\
391.01	0.01\\
392.01	0.01\\
393.01	0.01\\
394.01	0.01\\
395.01	0.01\\
396.01	0.01\\
397.01	0.01\\
398.01	0.01\\
399.01	0.01\\
400.01	0.01\\
401.01	0.01\\
402.01	0.01\\
403.01	0.01\\
404.01	0.01\\
405.01	0.01\\
406.01	0.01\\
407.01	0.01\\
408.01	0.01\\
409.01	0.01\\
410.01	0.01\\
411.01	0.01\\
412.01	0.01\\
413.01	0.01\\
414.01	0.01\\
415.01	0.01\\
416.01	0.01\\
417.01	0.01\\
418.01	0.01\\
419.01	0.01\\
420.01	0.01\\
421.01	0.01\\
422.01	0.01\\
423.01	0.01\\
424.01	0.01\\
425.01	0.01\\
426.01	0.01\\
427.01	0.01\\
428.01	0.01\\
429.01	0.01\\
430.01	0.01\\
431.01	0.01\\
432.01	0.01\\
433.01	0.01\\
434.01	0.01\\
435.01	0.01\\
436.01	0.01\\
437.01	0.01\\
438.01	0.01\\
439.01	0.01\\
440.01	0.01\\
441.01	0.01\\
442.01	0.01\\
443.01	0.01\\
444.01	0.01\\
445.01	0.01\\
446.01	0.01\\
447.01	0.01\\
448.01	0.01\\
449.01	0.01\\
450.01	0.01\\
451.01	0.01\\
452.01	0.01\\
453.01	0.01\\
454.01	0.01\\
455.01	0.01\\
456.01	0.01\\
457.01	0.01\\
458.01	0.01\\
459.01	0.01\\
460.01	0.01\\
461.01	0.01\\
462.01	0.01\\
463.01	0.01\\
464.01	0.01\\
465.01	0.01\\
466.01	0.01\\
467.01	0.01\\
468.01	0.01\\
469.01	0.01\\
470.01	0.01\\
471.01	0.01\\
472.01	0.01\\
473.01	0.01\\
474.01	0.01\\
475.01	0.01\\
476.01	0.01\\
477.01	0.01\\
478.01	0.01\\
479.01	0.01\\
480.01	0.01\\
481.01	0.01\\
482.01	0.01\\
483.01	0.01\\
484.01	0.01\\
485.01	0.01\\
486.01	0.01\\
487.01	0.01\\
488.01	0.01\\
489.01	0.01\\
490.01	0.01\\
491.01	0.01\\
492.01	0.01\\
493.01	0.01\\
494.01	0.01\\
495.01	0.01\\
496.01	0.01\\
497.01	0.01\\
498.01	0.01\\
499.01	0.01\\
500.01	0.01\\
501.01	0.01\\
502.01	0.01\\
503.01	0.01\\
504.01	0.01\\
505.01	0.01\\
506.01	0.01\\
507.01	0.01\\
508.01	0.01\\
509.01	0.01\\
510.01	0.01\\
511.01	0.01\\
512.01	0.01\\
513.01	0.01\\
514.01	0.01\\
515.01	0.01\\
516.01	0.01\\
517.01	0.01\\
518.01	0.01\\
519.01	0.01\\
520.01	0.01\\
521.01	0.01\\
522.01	0.01\\
523.01	0.01\\
524.01	0.01\\
525.01	0.01\\
526.01	0.01\\
527.01	0.01\\
528.01	0.01\\
529.01	0.01\\
530.01	0.01\\
531.01	0.01\\
532.01	0.01\\
533.01	0.01\\
534.01	0.01\\
535.01	0.01\\
536.01	0.01\\
537.01	0.01\\
538.01	0.01\\
539.01	0.01\\
540.01	0.01\\
541.01	0.01\\
542.01	0.01\\
543.01	0.01\\
544.01	0.01\\
545.01	0.01\\
546.01	0.01\\
547.01	0.01\\
548.01	0.01\\
549.01	0.01\\
550.01	0.01\\
551.01	0.01\\
552.01	0.01\\
553.01	0.01\\
554.01	0.01\\
555.01	0.01\\
556.01	0.01\\
557.01	0.01\\
558.01	0.01\\
559.01	0.01\\
560.01	0.01\\
561.01	0.01\\
562.01	0.01\\
563.01	0.01\\
564.01	0.01\\
565.01	0.01\\
566.01	0.01\\
567.01	0.01\\
568.01	0.01\\
569.01	0.01\\
570.01	0.01\\
571.01	0.01\\
572.01	0.01\\
573.01	0.01\\
574.01	0.01\\
575.01	0.01\\
576.01	0.01\\
577.01	0.01\\
578.01	0.01\\
579.01	0.01\\
580.01	0.01\\
581.01	0.01\\
582.01	0.01\\
583.01	0.01\\
584.01	0.01\\
585.01	0.01\\
586.01	0.0098241999093732\\
587.01	0.00961107658299926\\
588.01	0.00939241819816822\\
589.01	0.00916777261536746\\
590.01	0.00893660772687111\\
591.01	0.00869831500266637\\
592.01	0.00845219178373418\\
593.01	0.00819742465069798\\
594.01	0.00793306018611432\\
595.01	0.00765792511149237\\
596.01	0.00737019251449302\\
597.01	0.00706398766549632\\
598.01	0.00669942449313359\\
599.01	0.00590910258711776\\
599.02	0.00589375841662357\\
599.03	0.00587812370952707\\
599.04	0.0058621910335398\\
599.05	0.00584595274653486\\
599.06	0.00582940098980269\\
599.07	0.00581252768105277\\
599.08	0.00579532450714961\\
599.09	0.00577778291657132\\
599.1	0.00575989411157794\\
599.11	0.0057416490400764\\
599.12	0.00572303838716803\\
599.13	0.00570405256636397\\
599.14	0.0056846817104528\\
599.15	0.00566491566200403\\
599.16	0.00564474396349013\\
599.17	0.00562415584700877\\
599.18	0.00560314022358585\\
599.19	0.00558168567203894\\
599.2	0.00555978042737953\\
599.21	0.00553741236731075\\
599.22	0.00551456899916912\\
599.23	0.00549123744848741\\
599.24	0.00546740444529818\\
599.25	0.00544305630820636\\
599.26	0.00541817892527457\\
599.27	0.00539275774382929\\
599.28	0.00536677774354027\\
599.29	0.00534022342605056\\
599.3	0.00531307878341821\\
599.31	0.0052853272797771\\
599.32	0.00525695184621571\\
599.33	0.00522793486025739\\
599.34	0.00519825812438913\\
599.35	0.00516790284357519\\
599.36	0.00513684960169748\\
599.37	0.00510507833686053\\
599.38	0.00507256831549451\\
599.39	0.00503929810518537\\
599.4	0.00500524554615594\\
599.41	0.00497038772131652\\
599.42	0.00493470087968157\\
599.43	0.00489816042254417\\
599.44	0.00486074088969181\\
599.45	0.00482241592249718\\
599.46	0.00478315822505225\\
599.47	0.00474293952321862\\
599.48	0.00470173052145797\\
599.49	0.00465950220083545\\
599.5	0.00461622374084655\\
599.51	0.00457186255544324\\
599.52	0.00452638719662633\\
599.53	0.0044797659435815\\
599.54	0.00443196991276706\\
599.55	0.00438296832083659\\
599.56	0.00433274130624297\\
599.57	0.00428125092075455\\
599.58	0.00422845780611208\\
599.59	0.00417432112792187\\
599.6	0.00411879850560301\\
599.61	0.0040618459380975\\
599.62	0.00400344767773484\\
599.63	0.00394356701252722\\
599.64	0.0038821559017218\\
599.65	0.00381916439423846\\
599.66	0.00375454053122845\\
599.67	0.00368823024230304\\
599.68	0.00362017723488921\\
599.69	0.00355032287614973\\
599.7	0.00347860606684995\\
599.71	0.00340496310649239\\
599.72	0.00332932754897175\\
599.73	0.00325163004792685\\
599.74	0.00317179819083822\\
599.75	0.00308975632093623\\
599.76	0.00300542534578846\\
599.77	0.00291872253132427\\
599.78	0.00282956127992724\\
599.79	0.00273785089105788\\
599.8	0.00264349630273556\\
599.81	0.00254639781197214\\
599.82	0.00244645077204294\\
599.83	0.00234354526422315\\
599.84	0.00223756574132511\\
599.85	0.0021283906400423\\
599.86	0.00201589195872525\\
599.87	0.00189993479677869\\
599.88	0.00178037685136625\\
599.89	0.00165706786652964\\
599.9	0.00152984902915766\\
599.91	0.00139855230546207\\
599.92	0.00126299971071054\\
599.93	0.00112300250390927\\
599.94	0.000978360297888684\\
599.95	0.000828860073790389\\
599.96	0.000674275087237755\\
599.97	0.000514363651443209\\
599.98	0.00034886778009568\\
599.99	0.000177511669999688\\
600	0\\
};
\addplot [color=mycolor13,solid,forget plot]
  table[row sep=crcr]{%
0.01	0\\
1.01	0\\
2.01	0\\
3.01	0\\
4.01	0\\
5.01	0\\
6.01	0\\
7.01	0\\
8.01	0\\
9.01	0\\
10.01	0\\
11.01	0\\
12.01	0\\
13.01	0\\
14.01	0\\
15.01	0\\
16.01	0\\
17.01	0\\
18.01	0\\
19.01	0\\
20.01	0\\
21.01	0\\
22.01	0\\
23.01	0\\
24.01	0\\
25.01	0\\
26.01	0\\
27.01	0\\
28.01	0\\
29.01	0\\
30.01	0\\
31.01	0\\
32.01	0\\
33.01	0\\
34.01	0\\
35.01	0\\
36.01	0\\
37.01	0\\
38.01	0\\
39.01	0\\
40.01	0\\
41.01	0\\
42.01	0\\
43.01	0\\
44.01	0\\
45.01	0\\
46.01	0\\
47.01	0\\
48.01	0\\
49.01	0\\
50.01	0\\
51.01	0\\
52.01	0\\
53.01	0\\
54.01	0\\
55.01	0\\
56.01	0\\
57.01	0\\
58.01	0\\
59.01	0\\
60.01	0\\
61.01	0\\
62.01	0\\
63.01	0\\
64.01	0\\
65.01	0\\
66.01	0\\
67.01	0\\
68.01	0\\
69.01	0\\
70.01	0\\
71.01	0\\
72.01	0\\
73.01	0\\
74.01	0\\
75.01	0\\
76.01	0\\
77.01	0\\
78.01	0\\
79.01	0\\
80.01	0\\
81.01	0\\
82.01	0\\
83.01	0\\
84.01	0\\
85.01	0\\
86.01	0\\
87.01	0\\
88.01	0\\
89.01	0\\
90.01	0\\
91.01	0\\
92.01	0\\
93.01	0\\
94.01	0\\
95.01	0\\
96.01	0\\
97.01	0\\
98.01	0\\
99.01	0\\
100.01	0\\
101.01	0\\
102.01	0\\
103.01	0\\
104.01	0\\
105.01	0\\
106.01	0\\
107.01	0\\
108.01	0\\
109.01	0\\
110.01	0\\
111.01	0\\
112.01	0\\
113.01	0\\
114.01	0\\
115.01	0\\
116.01	0\\
117.01	0\\
118.01	0\\
119.01	0\\
120.01	0\\
121.01	0\\
122.01	0\\
123.01	0\\
124.01	0\\
125.01	0\\
126.01	0\\
127.01	0\\
128.01	0\\
129.01	0\\
130.01	0\\
131.01	0\\
132.01	0\\
133.01	0\\
134.01	0\\
135.01	0\\
136.01	0\\
137.01	0\\
138.01	0\\
139.01	0\\
140.01	0\\
141.01	0\\
142.01	0\\
143.01	0\\
144.01	0\\
145.01	0\\
146.01	0\\
147.01	0\\
148.01	0\\
149.01	0\\
150.01	0\\
151.01	0\\
152.01	0\\
153.01	0\\
154.01	0\\
155.01	0\\
156.01	0\\
157.01	0\\
158.01	0\\
159.01	0\\
160.01	0\\
161.01	0\\
162.01	0\\
163.01	0\\
164.01	0\\
165.01	0\\
166.01	0\\
167.01	0\\
168.01	0\\
169.01	0\\
170.01	0\\
171.01	0\\
172.01	0\\
173.01	0\\
174.01	0\\
175.01	0\\
176.01	0\\
177.01	0\\
178.01	0\\
179.01	0\\
180.01	0\\
181.01	0\\
182.01	0\\
183.01	0\\
184.01	0\\
185.01	0\\
186.01	0\\
187.01	0\\
188.01	0\\
189.01	0\\
190.01	0\\
191.01	0\\
192.01	0\\
193.01	0\\
194.01	0\\
195.01	0\\
196.01	0\\
197.01	0\\
198.01	0\\
199.01	0\\
200.01	0\\
201.01	0\\
202.01	0\\
203.01	0\\
204.01	0\\
205.01	0\\
206.01	0\\
207.01	0\\
208.01	0\\
209.01	0\\
210.01	0\\
211.01	0\\
212.01	0\\
213.01	0\\
214.01	0\\
215.01	0\\
216.01	0\\
217.01	0\\
218.01	0\\
219.01	0\\
220.01	0\\
221.01	0\\
222.01	0\\
223.01	0\\
224.01	0\\
225.01	0\\
226.01	0\\
227.01	0\\
228.01	0\\
229.01	0\\
230.01	0\\
231.01	0\\
232.01	0\\
233.01	0\\
234.01	0\\
235.01	0\\
236.01	0\\
237.01	0\\
238.01	0\\
239.01	0\\
240.01	0\\
241.01	0\\
242.01	0\\
243.01	0\\
244.01	0\\
245.01	0\\
246.01	0\\
247.01	0\\
248.01	0\\
249.01	0\\
250.01	0\\
251.01	0\\
252.01	0\\
253.01	0\\
254.01	0\\
255.01	0\\
256.01	0\\
257.01	0\\
258.01	0\\
259.01	0\\
260.01	0\\
261.01	0\\
262.01	0\\
263.01	0\\
264.01	0\\
265.01	0\\
266.01	0\\
267.01	0\\
268.01	0\\
269.01	0\\
270.01	0\\
271.01	0\\
272.01	0\\
273.01	0\\
274.01	0\\
275.01	0\\
276.01	0\\
277.01	0\\
278.01	0\\
279.01	0\\
280.01	0\\
281.01	0\\
282.01	0\\
283.01	0\\
284.01	0\\
285.01	0\\
286.01	0\\
287.01	0\\
288.01	0\\
289.01	0\\
290.01	0\\
291.01	0\\
292.01	0\\
293.01	0\\
294.01	0\\
295.01	0\\
296.01	0\\
297.01	0\\
298.01	0\\
299.01	0\\
300.01	0\\
301.01	0\\
302.01	0\\
303.01	0\\
304.01	0\\
305.01	0\\
306.01	0\\
307.01	0\\
308.01	0\\
309.01	0\\
310.01	0\\
311.01	0\\
312.01	0\\
313.01	0\\
314.01	0\\
315.01	0\\
316.01	0\\
317.01	0\\
318.01	0\\
319.01	0\\
320.01	0\\
321.01	0\\
322.01	0\\
323.01	0\\
324.01	0\\
325.01	0\\
326.01	0\\
327.01	0\\
328.01	0\\
329.01	0\\
330.01	0\\
331.01	0\\
332.01	0\\
333.01	0\\
334.01	0\\
335.01	0\\
336.01	0\\
337.01	0\\
338.01	0\\
339.01	0\\
340.01	0\\
341.01	0\\
342.01	0\\
343.01	0\\
344.01	0\\
345.01	0\\
346.01	0\\
347.01	0\\
348.01	0\\
349.01	0\\
350.01	0\\
351.01	0\\
352.01	0\\
353.01	0\\
354.01	0\\
355.01	0\\
356.01	0\\
357.01	0\\
358.01	0\\
359.01	0\\
360.01	0\\
361.01	0\\
362.01	0\\
363.01	0\\
364.01	0\\
365.01	0\\
366.01	0\\
367.01	0\\
368.01	0\\
369.01	0\\
370.01	0\\
371.01	0\\
372.01	0\\
373.01	0\\
374.01	0\\
375.01	0\\
376.01	0\\
377.01	0\\
378.01	0\\
379.01	0\\
380.01	0\\
381.01	0\\
382.01	0\\
383.01	0\\
384.01	0\\
385.01	0\\
386.01	0\\
387.01	0\\
388.01	0\\
389.01	0\\
390.01	0\\
391.01	0\\
392.01	0\\
393.01	0\\
394.01	0\\
395.01	0\\
396.01	0\\
397.01	0\\
398.01	0\\
399.01	0\\
400.01	0\\
401.01	0\\
402.01	0\\
403.01	0\\
404.01	0\\
405.01	0\\
406.01	0\\
407.01	0\\
408.01	0\\
409.01	0\\
410.01	0\\
411.01	0\\
412.01	0\\
413.01	0\\
414.01	0\\
415.01	0\\
416.01	0\\
417.01	0\\
418.01	0\\
419.01	0\\
420.01	0\\
421.01	0\\
422.01	0\\
423.01	0\\
424.01	0\\
425.01	0\\
426.01	0\\
427.01	0\\
428.01	0\\
429.01	0\\
430.01	0\\
431.01	0\\
432.01	0\\
433.01	0\\
434.01	0\\
435.01	0\\
436.01	0\\
437.01	0\\
438.01	0\\
439.01	0\\
440.01	0\\
441.01	0\\
442.01	0\\
443.01	0\\
444.01	0\\
445.01	0\\
446.01	0\\
447.01	0\\
448.01	0\\
449.01	0\\
450.01	0\\
451.01	0\\
452.01	0\\
453.01	0\\
454.01	0\\
455.01	0\\
456.01	0\\
457.01	0\\
458.01	0\\
459.01	0\\
460.01	0\\
461.01	0\\
462.01	0\\
463.01	0\\
464.01	0\\
465.01	0\\
466.01	0\\
467.01	0\\
468.01	0\\
469.01	0\\
470.01	0\\
471.01	0\\
472.01	0\\
473.01	0\\
474.01	0\\
475.01	0\\
476.01	0\\
477.01	0\\
478.01	0\\
479.01	0\\
480.01	0\\
481.01	0\\
482.01	0\\
483.01	0\\
484.01	0\\
485.01	0\\
486.01	0\\
487.01	0\\
488.01	0\\
489.01	0\\
490.01	0\\
491.01	0\\
492.01	0\\
493.01	0\\
494.01	0\\
495.01	0\\
496.01	0\\
497.01	0\\
498.01	0\\
499.01	0\\
500.01	0\\
501.01	0\\
502.01	0\\
503.01	0\\
504.01	0\\
505.01	0\\
506.01	0\\
507.01	0\\
508.01	0\\
509.01	0\\
510.01	0\\
511.01	0\\
512.01	0\\
513.01	0\\
514.01	0\\
515.01	0\\
516.01	0\\
517.01	0\\
518.01	0\\
519.01	0\\
520.01	0\\
521.01	0\\
522.01	0\\
523.01	0\\
524.01	0\\
525.01	0\\
526.01	0\\
527.01	0\\
528.01	0\\
529.01	0\\
530.01	0\\
531.01	0\\
532.01	0\\
533.01	0\\
534.01	0\\
535.01	0\\
536.01	0\\
537.01	0\\
538.01	0\\
539.01	0\\
540.01	0\\
541.01	0\\
542.01	0\\
543.01	0\\
544.01	0\\
545.01	0\\
546.01	0\\
547.01	0\\
548.01	0\\
549.01	0\\
550.01	0\\
551.01	0\\
552.01	0\\
553.01	0\\
554.01	0\\
555.01	0\\
556.01	0\\
557.01	0\\
558.01	0\\
559.01	0\\
560.01	0\\
561.01	0\\
562.01	0\\
563.01	0\\
564.01	0\\
565.01	0\\
566.01	0\\
567.01	0\\
568.01	0\\
569.01	0\\
570.01	0\\
571.01	0\\
572.01	0\\
573.01	0\\
574.01	0\\
575.01	0\\
576.01	0\\
577.01	0\\
578.01	0\\
579.01	0\\
580.01	0\\
581.01	0\\
582.01	0\\
583.01	0\\
584.01	0\\
585.01	0\\
586.01	0\\
587.01	0\\
588.01	0\\
589.01	0\\
590.01	0\\
591.01	0\\
592.01	0\\
593.01	0\\
594.01	0\\
595.01	0\\
596.01	0\\
597.01	0\\
598.01	0\\
599.01	0\\
599.02	0\\
599.03	0\\
599.04	0\\
599.05	0\\
599.06	0\\
599.07	0\\
599.08	0\\
599.09	0\\
599.1	0\\
599.11	0\\
599.12	0\\
599.13	0\\
599.14	0\\
599.15	0\\
599.16	0\\
599.17	0\\
599.18	0\\
599.19	0\\
599.2	0\\
599.21	0\\
599.22	0\\
599.23	0\\
599.24	0\\
599.25	0\\
599.26	0\\
599.27	0\\
599.28	0\\
599.29	0\\
599.3	0\\
599.31	0\\
599.32	0\\
599.33	0\\
599.34	0\\
599.35	0\\
599.36	0\\
599.37	0\\
599.38	0\\
599.39	0\\
599.4	0\\
599.41	0\\
599.42	0\\
599.43	0\\
599.44	0\\
599.45	0\\
599.46	0\\
599.47	0\\
599.48	0\\
599.49	0.000108462067765987\\
599.5	0.000274956428996023\\
599.51	0.000442304418925576\\
599.52	0.000610516577113952\\
599.53	0.000779597867790803\\
599.54	0.000949552333876835\\
599.55	0.00112039122267384\\
599.56	0.00129212590665678\\
599.57	0.00146476792988524\\
599.58	0.0016383296358674\\
599.59	0.00181282359304911\\
599.6	0.00198826293302643\\
599.61	0.00216466121189597\\
599.62	0.0023420324165617\\
599.63	0.00252039097831623\\
599.64	0.00269975178686579\\
599.65	0.00288013020481843\\
599.66	0.00306154208265583\\
599.67	0.00324400377459614\\
599.68	0.00342753215478128\\
599.69	0.00361214463388089\\
599.7	0.00379785917627751\\
599.71	0.0039846943178602\\
599.72	0.00417266918642947\\
599.73	0.00436180352525035\\
599.74	0.00455211771380802\\
599.75	0.00474363278903193\\
599.76	0.00493637046861567\\
599.77	0.00513035317607476\\
599.78	0.00532560406487874\\
599.79	0.0055221470435591\\
599.8	0.00572000680184516\\
599.81	0.00591920883788277\\
599.82	0.00611977948659422\\
599.83	0.00632174594924113\\
599.84	0.00652513632425557\\
599.85	0.00672997963940852\\
599.86	0.00693630588538855\\
599.87	0.00714414605086752\\
599.88	0.00735353215913409\\
599.89	0.00756449730637987\\
599.9	0.00777707570172686\\
599.91	0.00799130270908877\\
599.92	0.00820721489096223\\
599.93	0.00842485005424691\\
599.94	0.00864424729819635\\
599.95	0.00886544706460237\\
599.96	0.00908849119031689\\
599.97	0.00931342296221292\\
599.98	0.00954028717468348\\
599.99	0.00976913018976985\\
600	0.01\\
};
\addplot [color=mycolor14,solid,forget plot]
  table[row sep=crcr]{%
0.01	0\\
1.01	0\\
2.01	0\\
3.01	0\\
4.01	0\\
5.01	0\\
6.01	0\\
7.01	0\\
8.01	0\\
9.01	0\\
10.01	0\\
11.01	0\\
12.01	0\\
13.01	0\\
14.01	0\\
15.01	0\\
16.01	0\\
17.01	0\\
18.01	0\\
19.01	0\\
20.01	0\\
21.01	0\\
22.01	0\\
23.01	0\\
24.01	0\\
25.01	0\\
26.01	0\\
27.01	0\\
28.01	0\\
29.01	0\\
30.01	0\\
31.01	0\\
32.01	0\\
33.01	0\\
34.01	0\\
35.01	0\\
36.01	0\\
37.01	0\\
38.01	0\\
39.01	0\\
40.01	0\\
41.01	0\\
42.01	0\\
43.01	0\\
44.01	0\\
45.01	0\\
46.01	0\\
47.01	0\\
48.01	0\\
49.01	0\\
50.01	0\\
51.01	0\\
52.01	0\\
53.01	0\\
54.01	0\\
55.01	0\\
56.01	0\\
57.01	0\\
58.01	0\\
59.01	0\\
60.01	0\\
61.01	0\\
62.01	0\\
63.01	0\\
64.01	0\\
65.01	0\\
66.01	0\\
67.01	0\\
68.01	0\\
69.01	0\\
70.01	0\\
71.01	0\\
72.01	0\\
73.01	0\\
74.01	0\\
75.01	0\\
76.01	0\\
77.01	0\\
78.01	0\\
79.01	0\\
80.01	0\\
81.01	0\\
82.01	0\\
83.01	0\\
84.01	0\\
85.01	0\\
86.01	0\\
87.01	0\\
88.01	0\\
89.01	0\\
90.01	0\\
91.01	0\\
92.01	0\\
93.01	0\\
94.01	0\\
95.01	0\\
96.01	0\\
97.01	0\\
98.01	0\\
99.01	0\\
100.01	0\\
101.01	0\\
102.01	0\\
103.01	0\\
104.01	0\\
105.01	0\\
106.01	0\\
107.01	0\\
108.01	0\\
109.01	0\\
110.01	0\\
111.01	0\\
112.01	0\\
113.01	0\\
114.01	0\\
115.01	0\\
116.01	0\\
117.01	0\\
118.01	0\\
119.01	0\\
120.01	0\\
121.01	0\\
122.01	0\\
123.01	0\\
124.01	0\\
125.01	0\\
126.01	0\\
127.01	0\\
128.01	0\\
129.01	0\\
130.01	0\\
131.01	0\\
132.01	0\\
133.01	0\\
134.01	0\\
135.01	0\\
136.01	0\\
137.01	0\\
138.01	0\\
139.01	0\\
140.01	0\\
141.01	0\\
142.01	0\\
143.01	0\\
144.01	0\\
145.01	0\\
146.01	0\\
147.01	0\\
148.01	0\\
149.01	0\\
150.01	0\\
151.01	0\\
152.01	0\\
153.01	0\\
154.01	0\\
155.01	0\\
156.01	0\\
157.01	0\\
158.01	0\\
159.01	0\\
160.01	0\\
161.01	0\\
162.01	0\\
163.01	0\\
164.01	0\\
165.01	0\\
166.01	0\\
167.01	0\\
168.01	0\\
169.01	0\\
170.01	0\\
171.01	0\\
172.01	0\\
173.01	0\\
174.01	0\\
175.01	0\\
176.01	0\\
177.01	0\\
178.01	0\\
179.01	0\\
180.01	0\\
181.01	0\\
182.01	0\\
183.01	0\\
184.01	0\\
185.01	0\\
186.01	0\\
187.01	0\\
188.01	0\\
189.01	0\\
190.01	0\\
191.01	0\\
192.01	0\\
193.01	0\\
194.01	0\\
195.01	0\\
196.01	0\\
197.01	0\\
198.01	0\\
199.01	0\\
200.01	0\\
201.01	0\\
202.01	0\\
203.01	0\\
204.01	0\\
205.01	0\\
206.01	0\\
207.01	0\\
208.01	0\\
209.01	0\\
210.01	0\\
211.01	0\\
212.01	0\\
213.01	0\\
214.01	0\\
215.01	0\\
216.01	0\\
217.01	0\\
218.01	0\\
219.01	0\\
220.01	0\\
221.01	0\\
222.01	0\\
223.01	0\\
224.01	0\\
225.01	0\\
226.01	0\\
227.01	0\\
228.01	0\\
229.01	0\\
230.01	0\\
231.01	0\\
232.01	0\\
233.01	0\\
234.01	0\\
235.01	0\\
236.01	0\\
237.01	0\\
238.01	0\\
239.01	0\\
240.01	0\\
241.01	0\\
242.01	0\\
243.01	0\\
244.01	0\\
245.01	0\\
246.01	0\\
247.01	0\\
248.01	0\\
249.01	0\\
250.01	0\\
251.01	0\\
252.01	0\\
253.01	0\\
254.01	0\\
255.01	0\\
256.01	0\\
257.01	0\\
258.01	0\\
259.01	0\\
260.01	0\\
261.01	0\\
262.01	0\\
263.01	0\\
264.01	0\\
265.01	0\\
266.01	0\\
267.01	0\\
268.01	0\\
269.01	0\\
270.01	0\\
271.01	0\\
272.01	0\\
273.01	0\\
274.01	0\\
275.01	0\\
276.01	0\\
277.01	0\\
278.01	0\\
279.01	0\\
280.01	0\\
281.01	0\\
282.01	0\\
283.01	0\\
284.01	0\\
285.01	0\\
286.01	0\\
287.01	0\\
288.01	0\\
289.01	0\\
290.01	0\\
291.01	0\\
292.01	0\\
293.01	0\\
294.01	0\\
295.01	0\\
296.01	0\\
297.01	0\\
298.01	0\\
299.01	0\\
300.01	0\\
301.01	0\\
302.01	0\\
303.01	0\\
304.01	0\\
305.01	0\\
306.01	0\\
307.01	0\\
308.01	0\\
309.01	0\\
310.01	0\\
311.01	0\\
312.01	0\\
313.01	0\\
314.01	0\\
315.01	0\\
316.01	0\\
317.01	0\\
318.01	0\\
319.01	0\\
320.01	0\\
321.01	0\\
322.01	0\\
323.01	0\\
324.01	0\\
325.01	0\\
326.01	0\\
327.01	0\\
328.01	0\\
329.01	0\\
330.01	0\\
331.01	0\\
332.01	0\\
333.01	0\\
334.01	0\\
335.01	0\\
336.01	0\\
337.01	0\\
338.01	0\\
339.01	0\\
340.01	0\\
341.01	0\\
342.01	0\\
343.01	0\\
344.01	0\\
345.01	0\\
346.01	0\\
347.01	0\\
348.01	0\\
349.01	0\\
350.01	0\\
351.01	0\\
352.01	0\\
353.01	0\\
354.01	0\\
355.01	0\\
356.01	0\\
357.01	0\\
358.01	0\\
359.01	0\\
360.01	0\\
361.01	0\\
362.01	0\\
363.01	0\\
364.01	0\\
365.01	0\\
366.01	0\\
367.01	0\\
368.01	0\\
369.01	0\\
370.01	0\\
371.01	0\\
372.01	0\\
373.01	0\\
374.01	0\\
375.01	0\\
376.01	0\\
377.01	0\\
378.01	0\\
379.01	0\\
380.01	0\\
381.01	0\\
382.01	0\\
383.01	0\\
384.01	0\\
385.01	0\\
386.01	0\\
387.01	0\\
388.01	0\\
389.01	0\\
390.01	0\\
391.01	0\\
392.01	0\\
393.01	0\\
394.01	0\\
395.01	0\\
396.01	0\\
397.01	0\\
398.01	0\\
399.01	0\\
400.01	0\\
401.01	0\\
402.01	0\\
403.01	0\\
404.01	0\\
405.01	0\\
406.01	0\\
407.01	0\\
408.01	0\\
409.01	0\\
410.01	0\\
411.01	0\\
412.01	0\\
413.01	0\\
414.01	0\\
415.01	0\\
416.01	0\\
417.01	0\\
418.01	0\\
419.01	0\\
420.01	0\\
421.01	0\\
422.01	0\\
423.01	0\\
424.01	0\\
425.01	0\\
426.01	0\\
427.01	0\\
428.01	0\\
429.01	0\\
430.01	0\\
431.01	0\\
432.01	0\\
433.01	0\\
434.01	0\\
435.01	0\\
436.01	0\\
437.01	0\\
438.01	0\\
439.01	0\\
440.01	0\\
441.01	0\\
442.01	0\\
443.01	0\\
444.01	0\\
445.01	0\\
446.01	0\\
447.01	0\\
448.01	0\\
449.01	0\\
450.01	0\\
451.01	0\\
452.01	0\\
453.01	0\\
454.01	0\\
455.01	0\\
456.01	0\\
457.01	0\\
458.01	0\\
459.01	0\\
460.01	0\\
461.01	0\\
462.01	0\\
463.01	0\\
464.01	0\\
465.01	0\\
466.01	0\\
467.01	0\\
468.01	0\\
469.01	0\\
470.01	0\\
471.01	0\\
472.01	0\\
473.01	0\\
474.01	0\\
475.01	0\\
476.01	0\\
477.01	0\\
478.01	0\\
479.01	0\\
480.01	0\\
481.01	0\\
482.01	0\\
483.01	0\\
484.01	0\\
485.01	0\\
486.01	0\\
487.01	0\\
488.01	0\\
489.01	0\\
490.01	0\\
491.01	0\\
492.01	0\\
493.01	0\\
494.01	0\\
495.01	0\\
496.01	0\\
497.01	0\\
498.01	0\\
499.01	0\\
500.01	0\\
501.01	0\\
502.01	0\\
503.01	0\\
504.01	0\\
505.01	0\\
506.01	0\\
507.01	0\\
508.01	0\\
509.01	0\\
510.01	0\\
511.01	0\\
512.01	0\\
513.01	0\\
514.01	0\\
515.01	0\\
516.01	0\\
517.01	0\\
518.01	0\\
519.01	0\\
520.01	0\\
521.01	0\\
522.01	0\\
523.01	0\\
524.01	0\\
525.01	0\\
526.01	0\\
527.01	0\\
528.01	0\\
529.01	0\\
530.01	0\\
531.01	0\\
532.01	0\\
533.01	0\\
534.01	0\\
535.01	0\\
536.01	0\\
537.01	0\\
538.01	0\\
539.01	0\\
540.01	0\\
541.01	0\\
542.01	0\\
543.01	0\\
544.01	0\\
545.01	0\\
546.01	0\\
547.01	0\\
548.01	0\\
549.01	0\\
550.01	0\\
551.01	0\\
552.01	0\\
553.01	0\\
554.01	0\\
555.01	0\\
556.01	0\\
557.01	0\\
558.01	0\\
559.01	0\\
560.01	0\\
561.01	0\\
562.01	0\\
563.01	0\\
564.01	0\\
565.01	0\\
566.01	0\\
567.01	0\\
568.01	0\\
569.01	0\\
570.01	0\\
571.01	0\\
572.01	0\\
573.01	0\\
574.01	0\\
575.01	0\\
576.01	0\\
577.01	0\\
578.01	0\\
579.01	0\\
580.01	0\\
581.01	0\\
582.01	0\\
583.01	0\\
584.01	0\\
585.01	0\\
586.01	0\\
587.01	0\\
588.01	0\\
589.01	0\\
590.01	0\\
591.01	0\\
592.01	0\\
593.01	0\\
594.01	0\\
595.01	0\\
596.01	0\\
597.01	0\\
598.01	0\\
599.01	0\\
599.02	0\\
599.03	0\\
599.04	0\\
599.05	0\\
599.06	0\\
599.07	0\\
599.08	0\\
599.09	0\\
599.1	0\\
599.11	0\\
599.12	0\\
599.13	0\\
599.14	0\\
599.15	0\\
599.16	0\\
599.17	0\\
599.18	0\\
599.19	0\\
599.2	0\\
599.21	7.43240303883695e-05\\
599.22	0.000264470482307808\\
599.23	0.000455841366224598\\
599.24	0.000648453467216172\\
599.25	0.000842324105170485\\
599.26	0.00103747115493353\\
599.27	0.00123390811714826\\
599.28	0.001431643963674\\
599.29	0.00163069794137809\\
599.3	0.00183108993013174\\
599.31	0.00203284008212145\\
599.32	0.00223596924269511\\
599.33	0.00244049910317951\\
599.34	0.00264645207197108\\
599.35	0.00285385117329218\\
599.36	0.00306272030937642\\
599.37	0.00327308425972395\\
599.38	0.00348496865944219\\
599.39	0.00369840003378491\\
599.4	0.003913405834474\\
599.41	0.00413001447782843\\
599.42	0.00434825538458751\\
599.43	0.00456815902172324\\
599.44	0.00478975694637176\\
599.45	0.00501308185202392\\
599.46	0.00523816761712656\\
599.47	0.00546504935625849\\
599.48	0.00569376347405871\\
599.49	0.0058156133798876\\
599.5	0.00588090468334206\\
599.51	0.00594682986883297\\
599.52	0.00601339241646541\\
599.53	0.00608060161398367\\
599.54	0.00614846796314744\\
599.55	0.00621699500263725\\
599.56	0.0062861863936304\\
599.57	0.00635604587967757\\
599.58	0.00642657666087133\\
599.59	0.00649778197508687\\
599.6	0.00656966476333187\\
599.61	0.00664222781355835\\
599.62	0.00671547375969177\\
599.63	0.00678940507349391\\
599.64	0.00686402405609307\\
599.65	0.00693933282916512\\
599.66	0.00701533332574808\\
599.67	0.0070920272542144\\
599.68	0.00716941610179131\\
599.69	0.00724750113333021\\
599.7	0.0073262833801967\\
599.71	0.00740576362867076\\
599.72	0.00748594227390502\\
599.73	0.00756681913272001\\
599.74	0.00764839366574154\\
599.75	0.00773066498139155\\
599.76	0.0078136317257257\\
599.77	0.00789729199735544\\
599.78	0.00798164345615663\\
599.79	0.00806668329920511\\
599.8	0.00815240823537674\\
599.81	0.00823881445852518\\
599.82	0.00832589761914421\\
599.83	0.00841365279441413\\
599.84	0.00850207445652472\\
599.85	0.00859115643915884\\
599.86	0.00868089190201197\\
599.87	0.00877127329321369\\
599.88	0.00886229230950642\\
599.89	0.00895393985402594\\
599.9	0.00904620599151582\\
599.91	0.00913907990079482\\
599.92	0.00923254982428205\\
599.93	0.00932660301436908\\
599.94	0.00942122567641143\\
599.95	0.00951640290809338\\
599.96	0.00961211863490027\\
599.97	0.00970835554141063\\
599.98	0.00980509499809699\\
599.99	0.00990231698329844\\
600	0.01\\
};
\addplot [color=mycolor15,solid,forget plot]
  table[row sep=crcr]{%
0.01	0\\
1.01	0\\
2.01	0\\
3.01	0\\
4.01	0\\
5.01	0\\
6.01	0\\
7.01	0\\
8.01	0\\
9.01	0\\
10.01	0\\
11.01	0\\
12.01	0\\
13.01	0\\
14.01	0\\
15.01	0\\
16.01	0\\
17.01	0\\
18.01	0\\
19.01	0\\
20.01	0\\
21.01	0\\
22.01	0\\
23.01	0\\
24.01	0\\
25.01	0\\
26.01	0\\
27.01	0\\
28.01	0\\
29.01	0\\
30.01	0\\
31.01	0\\
32.01	0\\
33.01	0\\
34.01	0\\
35.01	0\\
36.01	0\\
37.01	0\\
38.01	0\\
39.01	0\\
40.01	0\\
41.01	0\\
42.01	0\\
43.01	0\\
44.01	0\\
45.01	0\\
46.01	0\\
47.01	0\\
48.01	0\\
49.01	0\\
50.01	0\\
51.01	0\\
52.01	0\\
53.01	0\\
54.01	0\\
55.01	0\\
56.01	0\\
57.01	0\\
58.01	0\\
59.01	0\\
60.01	0\\
61.01	0\\
62.01	0\\
63.01	0\\
64.01	0\\
65.01	0\\
66.01	0\\
67.01	0\\
68.01	0\\
69.01	0\\
70.01	0\\
71.01	0\\
72.01	0\\
73.01	0\\
74.01	0\\
75.01	0\\
76.01	0\\
77.01	0\\
78.01	0\\
79.01	0\\
80.01	0\\
81.01	0\\
82.01	0\\
83.01	0\\
84.01	0\\
85.01	0\\
86.01	0\\
87.01	0\\
88.01	0\\
89.01	0\\
90.01	0\\
91.01	0\\
92.01	0\\
93.01	0\\
94.01	0\\
95.01	0\\
96.01	0\\
97.01	0\\
98.01	0\\
99.01	0\\
100.01	0\\
101.01	0\\
102.01	0\\
103.01	0\\
104.01	0\\
105.01	0\\
106.01	0\\
107.01	0\\
108.01	0\\
109.01	0\\
110.01	0\\
111.01	0\\
112.01	0\\
113.01	0\\
114.01	0\\
115.01	0\\
116.01	0\\
117.01	0\\
118.01	0\\
119.01	0\\
120.01	0\\
121.01	0\\
122.01	0\\
123.01	0\\
124.01	0\\
125.01	0\\
126.01	0\\
127.01	0\\
128.01	0\\
129.01	0\\
130.01	0\\
131.01	0\\
132.01	0\\
133.01	0\\
134.01	0\\
135.01	0\\
136.01	0\\
137.01	0\\
138.01	0\\
139.01	0\\
140.01	0\\
141.01	0\\
142.01	0\\
143.01	0\\
144.01	0\\
145.01	0\\
146.01	0\\
147.01	0\\
148.01	0\\
149.01	0\\
150.01	0\\
151.01	0\\
152.01	0\\
153.01	0\\
154.01	0\\
155.01	0\\
156.01	0\\
157.01	0\\
158.01	0\\
159.01	0\\
160.01	0\\
161.01	0\\
162.01	0\\
163.01	0\\
164.01	0\\
165.01	0\\
166.01	0\\
167.01	0\\
168.01	0\\
169.01	0\\
170.01	0\\
171.01	0\\
172.01	0\\
173.01	0\\
174.01	0\\
175.01	0\\
176.01	0\\
177.01	0\\
178.01	0\\
179.01	0\\
180.01	0\\
181.01	0\\
182.01	0\\
183.01	0\\
184.01	0\\
185.01	0\\
186.01	0\\
187.01	0\\
188.01	0\\
189.01	0\\
190.01	0\\
191.01	0\\
192.01	0\\
193.01	0\\
194.01	0\\
195.01	0\\
196.01	0\\
197.01	0\\
198.01	0\\
199.01	0\\
200.01	0\\
201.01	0\\
202.01	0\\
203.01	0\\
204.01	0\\
205.01	0\\
206.01	0\\
207.01	0\\
208.01	0\\
209.01	0\\
210.01	0\\
211.01	0\\
212.01	0\\
213.01	0\\
214.01	0\\
215.01	0\\
216.01	0\\
217.01	0\\
218.01	0\\
219.01	0\\
220.01	0\\
221.01	0\\
222.01	0\\
223.01	0\\
224.01	0\\
225.01	0\\
226.01	0\\
227.01	0\\
228.01	0\\
229.01	0\\
230.01	0\\
231.01	0\\
232.01	0\\
233.01	0\\
234.01	0\\
235.01	0\\
236.01	0\\
237.01	0\\
238.01	0\\
239.01	0\\
240.01	0\\
241.01	0\\
242.01	0\\
243.01	0\\
244.01	0\\
245.01	0\\
246.01	0\\
247.01	0\\
248.01	0\\
249.01	0\\
250.01	0\\
251.01	0\\
252.01	0\\
253.01	0\\
254.01	0\\
255.01	0\\
256.01	0\\
257.01	0\\
258.01	0\\
259.01	0\\
260.01	0\\
261.01	0\\
262.01	0\\
263.01	0\\
264.01	0\\
265.01	0\\
266.01	0\\
267.01	0\\
268.01	0\\
269.01	0\\
270.01	0\\
271.01	0\\
272.01	0\\
273.01	0\\
274.01	0\\
275.01	0\\
276.01	0\\
277.01	0\\
278.01	0\\
279.01	0\\
280.01	0\\
281.01	0\\
282.01	0\\
283.01	0\\
284.01	0\\
285.01	0\\
286.01	0\\
287.01	0\\
288.01	0\\
289.01	0\\
290.01	0\\
291.01	0\\
292.01	0\\
293.01	0\\
294.01	0\\
295.01	0\\
296.01	0\\
297.01	0\\
298.01	0\\
299.01	0\\
300.01	0\\
301.01	0\\
302.01	0\\
303.01	0\\
304.01	0\\
305.01	0\\
306.01	0\\
307.01	0\\
308.01	0\\
309.01	0\\
310.01	0\\
311.01	0\\
312.01	0\\
313.01	0\\
314.01	0\\
315.01	0\\
316.01	0\\
317.01	0\\
318.01	0\\
319.01	0\\
320.01	0\\
321.01	0\\
322.01	0\\
323.01	0\\
324.01	0\\
325.01	0\\
326.01	0\\
327.01	0\\
328.01	0\\
329.01	0\\
330.01	0\\
331.01	0\\
332.01	0\\
333.01	0\\
334.01	0\\
335.01	0\\
336.01	0\\
337.01	0\\
338.01	0\\
339.01	0\\
340.01	0\\
341.01	0\\
342.01	0\\
343.01	0\\
344.01	0\\
345.01	0\\
346.01	0\\
347.01	0\\
348.01	0\\
349.01	0\\
350.01	0\\
351.01	0\\
352.01	0\\
353.01	0\\
354.01	0\\
355.01	0\\
356.01	0\\
357.01	0\\
358.01	0\\
359.01	0\\
360.01	0\\
361.01	0\\
362.01	0\\
363.01	0\\
364.01	0\\
365.01	0\\
366.01	0\\
367.01	0\\
368.01	0\\
369.01	0\\
370.01	0\\
371.01	0\\
372.01	0\\
373.01	0\\
374.01	0\\
375.01	0\\
376.01	0\\
377.01	0\\
378.01	0\\
379.01	0\\
380.01	0\\
381.01	0\\
382.01	0\\
383.01	0\\
384.01	0\\
385.01	0\\
386.01	0\\
387.01	0\\
388.01	0\\
389.01	0\\
390.01	0\\
391.01	0\\
392.01	0\\
393.01	0\\
394.01	0\\
395.01	0\\
396.01	0\\
397.01	0\\
398.01	0\\
399.01	0\\
400.01	0\\
401.01	0\\
402.01	0\\
403.01	0\\
404.01	0\\
405.01	0\\
406.01	0\\
407.01	0\\
408.01	0\\
409.01	0\\
410.01	0\\
411.01	0\\
412.01	0\\
413.01	0\\
414.01	0\\
415.01	0\\
416.01	0\\
417.01	0\\
418.01	0\\
419.01	0\\
420.01	0\\
421.01	0\\
422.01	0\\
423.01	0\\
424.01	0\\
425.01	0\\
426.01	0\\
427.01	0\\
428.01	0\\
429.01	0\\
430.01	0\\
431.01	0\\
432.01	0\\
433.01	0\\
434.01	0\\
435.01	0\\
436.01	0\\
437.01	0\\
438.01	0\\
439.01	0\\
440.01	0\\
441.01	0\\
442.01	0\\
443.01	0\\
444.01	0\\
445.01	0\\
446.01	0\\
447.01	0\\
448.01	0\\
449.01	0\\
450.01	0\\
451.01	0\\
452.01	0\\
453.01	0\\
454.01	0\\
455.01	0\\
456.01	0\\
457.01	0\\
458.01	0\\
459.01	0\\
460.01	0\\
461.01	0\\
462.01	0\\
463.01	0\\
464.01	0\\
465.01	0\\
466.01	0\\
467.01	0\\
468.01	0\\
469.01	0\\
470.01	0\\
471.01	0\\
472.01	0\\
473.01	0\\
474.01	0\\
475.01	0\\
476.01	0\\
477.01	0\\
478.01	0\\
479.01	0\\
480.01	0\\
481.01	0\\
482.01	0\\
483.01	0\\
484.01	0\\
485.01	0\\
486.01	0\\
487.01	0\\
488.01	0\\
489.01	0\\
490.01	0\\
491.01	0\\
492.01	0\\
493.01	0\\
494.01	0\\
495.01	0\\
496.01	0\\
497.01	0\\
498.01	0\\
499.01	0\\
500.01	0\\
501.01	0\\
502.01	0\\
503.01	0\\
504.01	0\\
505.01	0\\
506.01	0\\
507.01	0\\
508.01	0\\
509.01	0\\
510.01	0\\
511.01	0\\
512.01	0\\
513.01	0\\
514.01	0\\
515.01	0\\
516.01	0\\
517.01	0\\
518.01	0\\
519.01	0\\
520.01	0\\
521.01	0\\
522.01	0\\
523.01	0\\
524.01	0\\
525.01	0\\
526.01	0\\
527.01	0\\
528.01	0\\
529.01	0\\
530.01	0\\
531.01	0\\
532.01	0\\
533.01	0\\
534.01	0\\
535.01	0\\
536.01	0\\
537.01	0\\
538.01	0\\
539.01	0\\
540.01	0\\
541.01	0\\
542.01	0\\
543.01	0\\
544.01	0\\
545.01	0\\
546.01	0\\
547.01	0\\
548.01	0\\
549.01	0\\
550.01	0\\
551.01	0\\
552.01	0\\
553.01	0\\
554.01	0\\
555.01	0\\
556.01	0\\
557.01	0\\
558.01	0\\
559.01	0\\
560.01	0\\
561.01	0\\
562.01	0\\
563.01	0\\
564.01	0\\
565.01	0\\
566.01	0\\
567.01	0\\
568.01	0\\
569.01	0\\
570.01	0\\
571.01	0\\
572.01	0\\
573.01	0\\
574.01	0\\
575.01	0\\
576.01	0\\
577.01	0\\
578.01	0\\
579.01	0\\
580.01	0\\
581.01	0\\
582.01	0\\
583.01	0\\
584.01	0\\
585.01	0\\
586.01	0\\
587.01	0\\
588.01	0\\
589.01	0\\
590.01	0\\
591.01	0\\
592.01	0\\
593.01	0\\
594.01	0\\
595.01	0\\
596.01	0\\
597.01	0\\
598.01	0\\
599.01	0.000112606315978678\\
599.02	0.000319256592928665\\
599.03	0.000527406673816788\\
599.04	0.000737078609145313\\
599.05	0.000948295168053847\\
599.06	0.00116107986778447\\
599.07	0.00137545700451787\\
599.08	0.00159144811590323\\
599.09	0.00180906712192634\\
599.1	0.00202833953770087\\
599.11	0.00224929233225141\\
599.12	0.00247195326892454\\
599.13	0.00269635053012227\\
599.14	0.00292251398260875\\
599.15	0.00315047455348861\\
599.16	0.00338026427669825\\
599.17	0.00361191627693269\\
599.18	0.00384546479123293\\
599.19	0.00408094547119403\\
599.2	0.00431839528698449\\
599.21	0.00448336522131882\\
599.22	0.00453412889007315\\
599.23	0.00458529775335187\\
599.24	0.00463686864801558\\
599.25	0.00468883802852874\\
599.26	0.00474120194778565\\
599.27	0.00479396101985254\\
599.28	0.00484712061004249\\
599.29	0.00490067597179399\\
599.3	0.00495462188970531\\
599.31	0.00500895304553924\\
599.32	0.00506366359833286\\
599.33	0.0051187468066271\\
599.34	0.00517419536070024\\
599.35	0.00523000150170405\\
599.36	0.00528615675861414\\
599.37	0.00534265194840582\\
599.38	0.00539947719780977\\
599.39	0.00545662190877039\\
599.4	0.00551407471367819\\
599.41	0.00557182343363529\\
599.42	0.00562985504934023\\
599.43	0.00568815565847897\\
599.44	0.00574671043077997\\
599.45	0.00580550356057858\\
599.46	0.00586451821672388\\
599.47	0.00592373648964772\\
599.48	0.0059831393354014\\
599.49	0.00604297960426825\\
599.5	0.00610340045448543\\
599.51	0.00616440733117743\\
599.52	0.00622600573181369\\
599.53	0.00628820117461682\\
599.54	0.00635099918988433\\
599.55	0.00641440535722492\\
599.56	0.00647842530500787\\
599.57	0.00654306470995203\\
599.58	0.00660832930056135\\
599.59	0.00667422485699413\\
599.6	0.00674075721317562\\
599.61	0.00680793225800099\\
599.62	0.00687575593656359\\
599.63	0.00694423425145826\\
599.64	0.00701337326416413\\
599.65	0.00708317909651245\\
599.66	0.00715365793224462\\
599.67	0.00722481601874126\\
599.68	0.00729665966881985\\
599.69	0.00736919526261727\\
599.7	0.00744242924958997\\
599.71	0.00751636815063912\\
599.72	0.0075910185607637\\
599.73	0.00766638715239962\\
599.74	0.00774248067826639\\
599.75	0.00781930597433837\\
599.76	0.00789686996329332\\
599.77	0.0079751796583929\\
599.78	0.00805424216722275\\
599.79	0.00813406469569006\\
599.8	0.00821465455229681\\
599.81	0.00829601915270807\\
599.82	0.0083781660246363\\
599.83	0.008461102813064\\
599.84	0.00854483728582868\\
599.85	0.00862937733959608\\
599.86	0.00871473100624913\\
599.87	0.00880090645972257\\
599.88	0.00888791202331493\\
599.89	0.00897575617751244\\
599.9	0.00906444756836147\\
599.91	0.00915399501642923\\
599.92	0.00924440752639532\\
599.93	0.00933569429731976\\
599.94	0.00942786473363691\\
599.95	0.00952092845692796\\
599.96	0.00961489531852919\\
599.97	0.00970977541303696\\
599.98	0.00980557909277551\\
599.99	0.00990231698329844\\
600	0.01\\
};
\addplot [color=mycolor16,solid,forget plot]
  table[row sep=crcr]{%
0.01	0\\
1.01	0\\
2.01	0\\
3.01	0\\
4.01	0\\
5.01	0\\
6.01	0\\
7.01	0\\
8.01	0\\
9.01	0\\
10.01	0\\
11.01	0\\
12.01	0\\
13.01	0\\
14.01	0\\
15.01	0\\
16.01	0\\
17.01	0\\
18.01	0\\
19.01	0\\
20.01	0\\
21.01	0\\
22.01	0\\
23.01	0\\
24.01	0\\
25.01	0\\
26.01	0\\
27.01	0\\
28.01	0\\
29.01	0\\
30.01	0\\
31.01	0\\
32.01	0\\
33.01	0\\
34.01	0\\
35.01	0\\
36.01	0\\
37.01	0\\
38.01	0\\
39.01	0\\
40.01	0\\
41.01	0\\
42.01	0\\
43.01	0\\
44.01	0\\
45.01	0\\
46.01	0\\
47.01	0\\
48.01	0\\
49.01	0\\
50.01	0\\
51.01	0\\
52.01	0\\
53.01	0\\
54.01	0\\
55.01	0\\
56.01	0\\
57.01	0\\
58.01	0\\
59.01	0\\
60.01	0\\
61.01	0\\
62.01	0\\
63.01	0\\
64.01	0\\
65.01	0\\
66.01	0\\
67.01	0\\
68.01	0\\
69.01	0\\
70.01	0\\
71.01	0\\
72.01	0\\
73.01	0\\
74.01	0\\
75.01	0\\
76.01	0\\
77.01	0\\
78.01	0\\
79.01	0\\
80.01	0\\
81.01	0\\
82.01	0\\
83.01	0\\
84.01	0\\
85.01	0\\
86.01	0\\
87.01	0\\
88.01	0\\
89.01	0\\
90.01	0\\
91.01	0\\
92.01	0\\
93.01	0\\
94.01	0\\
95.01	0\\
96.01	0\\
97.01	0\\
98.01	0\\
99.01	0\\
100.01	0\\
101.01	0\\
102.01	0\\
103.01	0\\
104.01	0\\
105.01	0\\
106.01	0\\
107.01	0\\
108.01	0\\
109.01	0\\
110.01	0\\
111.01	0\\
112.01	0\\
113.01	0\\
114.01	0\\
115.01	0\\
116.01	0\\
117.01	0\\
118.01	0\\
119.01	0\\
120.01	0\\
121.01	0\\
122.01	0\\
123.01	0\\
124.01	0\\
125.01	0\\
126.01	0\\
127.01	0\\
128.01	0\\
129.01	0\\
130.01	0\\
131.01	0\\
132.01	0\\
133.01	0\\
134.01	0\\
135.01	0\\
136.01	0\\
137.01	0\\
138.01	0\\
139.01	0\\
140.01	0\\
141.01	0\\
142.01	0\\
143.01	0\\
144.01	0\\
145.01	0\\
146.01	0\\
147.01	0\\
148.01	0\\
149.01	0\\
150.01	0\\
151.01	0\\
152.01	0\\
153.01	0\\
154.01	0\\
155.01	0\\
156.01	0\\
157.01	0\\
158.01	0\\
159.01	0\\
160.01	0\\
161.01	0\\
162.01	0\\
163.01	0\\
164.01	0\\
165.01	0\\
166.01	0\\
167.01	0\\
168.01	0\\
169.01	0\\
170.01	0\\
171.01	0\\
172.01	0\\
173.01	0\\
174.01	0\\
175.01	0\\
176.01	0\\
177.01	0\\
178.01	0\\
179.01	0\\
180.01	0\\
181.01	0\\
182.01	0\\
183.01	0\\
184.01	0\\
185.01	0\\
186.01	0\\
187.01	0\\
188.01	0\\
189.01	0\\
190.01	0\\
191.01	0\\
192.01	0\\
193.01	0\\
194.01	0\\
195.01	0\\
196.01	0\\
197.01	0\\
198.01	0\\
199.01	0\\
200.01	0\\
201.01	0\\
202.01	0\\
203.01	0\\
204.01	0\\
205.01	0\\
206.01	0\\
207.01	0\\
208.01	0\\
209.01	0\\
210.01	0\\
211.01	0\\
212.01	0\\
213.01	0\\
214.01	0\\
215.01	0\\
216.01	0\\
217.01	0\\
218.01	0\\
219.01	0\\
220.01	0\\
221.01	0\\
222.01	0\\
223.01	0\\
224.01	0\\
225.01	0\\
226.01	0\\
227.01	0\\
228.01	0\\
229.01	0\\
230.01	0\\
231.01	0\\
232.01	0\\
233.01	0\\
234.01	0\\
235.01	0\\
236.01	0\\
237.01	0\\
238.01	0\\
239.01	0\\
240.01	0\\
241.01	0\\
242.01	0\\
243.01	0\\
244.01	0\\
245.01	0\\
246.01	0\\
247.01	0\\
248.01	0\\
249.01	0\\
250.01	0\\
251.01	0\\
252.01	0\\
253.01	0\\
254.01	0\\
255.01	0\\
256.01	0\\
257.01	0\\
258.01	0\\
259.01	0\\
260.01	0\\
261.01	0\\
262.01	0\\
263.01	0\\
264.01	0\\
265.01	0\\
266.01	0\\
267.01	0\\
268.01	0\\
269.01	0\\
270.01	0\\
271.01	0\\
272.01	0\\
273.01	0\\
274.01	0\\
275.01	0\\
276.01	0\\
277.01	0\\
278.01	0\\
279.01	0\\
280.01	0\\
281.01	0\\
282.01	0\\
283.01	0\\
284.01	0\\
285.01	0\\
286.01	0\\
287.01	0\\
288.01	0\\
289.01	0\\
290.01	0\\
291.01	0\\
292.01	0\\
293.01	0\\
294.01	0\\
295.01	0\\
296.01	0\\
297.01	0\\
298.01	0\\
299.01	0\\
300.01	0\\
301.01	0\\
302.01	0\\
303.01	0\\
304.01	0\\
305.01	0\\
306.01	0\\
307.01	0\\
308.01	0\\
309.01	0\\
310.01	0\\
311.01	0\\
312.01	0\\
313.01	0\\
314.01	0\\
315.01	0\\
316.01	0\\
317.01	0\\
318.01	0\\
319.01	0\\
320.01	0\\
321.01	0\\
322.01	0\\
323.01	0\\
324.01	0\\
325.01	0\\
326.01	0\\
327.01	0\\
328.01	0\\
329.01	0\\
330.01	0\\
331.01	0\\
332.01	0\\
333.01	0\\
334.01	0\\
335.01	0\\
336.01	0\\
337.01	0\\
338.01	0\\
339.01	0\\
340.01	0\\
341.01	0\\
342.01	0\\
343.01	0\\
344.01	0\\
345.01	0\\
346.01	0\\
347.01	0\\
348.01	0\\
349.01	0\\
350.01	0\\
351.01	0\\
352.01	0\\
353.01	0\\
354.01	0\\
355.01	0\\
356.01	0\\
357.01	0\\
358.01	0\\
359.01	0\\
360.01	0\\
361.01	0\\
362.01	0\\
363.01	0\\
364.01	0\\
365.01	0\\
366.01	0\\
367.01	0\\
368.01	0\\
369.01	0\\
370.01	0\\
371.01	0\\
372.01	0\\
373.01	0\\
374.01	0\\
375.01	0\\
376.01	0\\
377.01	0\\
378.01	0\\
379.01	0\\
380.01	0\\
381.01	0\\
382.01	0\\
383.01	0\\
384.01	0\\
385.01	0\\
386.01	0\\
387.01	0\\
388.01	0\\
389.01	0\\
390.01	0\\
391.01	0\\
392.01	0\\
393.01	0\\
394.01	0\\
395.01	0\\
396.01	0\\
397.01	0\\
398.01	0\\
399.01	0\\
400.01	0\\
401.01	0\\
402.01	0\\
403.01	0\\
404.01	0\\
405.01	0\\
406.01	0\\
407.01	0\\
408.01	0\\
409.01	0\\
410.01	0\\
411.01	0\\
412.01	0\\
413.01	0\\
414.01	0\\
415.01	0\\
416.01	0\\
417.01	0\\
418.01	0\\
419.01	0\\
420.01	0\\
421.01	0\\
422.01	0\\
423.01	0\\
424.01	0\\
425.01	0\\
426.01	0\\
427.01	0\\
428.01	0\\
429.01	0\\
430.01	0\\
431.01	0\\
432.01	0\\
433.01	0\\
434.01	0\\
435.01	0\\
436.01	0\\
437.01	0\\
438.01	0\\
439.01	0\\
440.01	0\\
441.01	0\\
442.01	0\\
443.01	0\\
444.01	0\\
445.01	0\\
446.01	0\\
447.01	0\\
448.01	0\\
449.01	0\\
450.01	0\\
451.01	0\\
452.01	0\\
453.01	0\\
454.01	0\\
455.01	0\\
456.01	0\\
457.01	0\\
458.01	0\\
459.01	0\\
460.01	0\\
461.01	0\\
462.01	0\\
463.01	0\\
464.01	0\\
465.01	0\\
466.01	0\\
467.01	0\\
468.01	0\\
469.01	0\\
470.01	0\\
471.01	0\\
472.01	0\\
473.01	0\\
474.01	0\\
475.01	0\\
476.01	0\\
477.01	0\\
478.01	0\\
479.01	0\\
480.01	0\\
481.01	0\\
482.01	0\\
483.01	0\\
484.01	0\\
485.01	0\\
486.01	0\\
487.01	0\\
488.01	0\\
489.01	0\\
490.01	0\\
491.01	0\\
492.01	0\\
493.01	0\\
494.01	0\\
495.01	0\\
496.01	0\\
497.01	0\\
498.01	0\\
499.01	0\\
500.01	0\\
501.01	0\\
502.01	0\\
503.01	0\\
504.01	0\\
505.01	0\\
506.01	0\\
507.01	0\\
508.01	0\\
509.01	0\\
510.01	0\\
511.01	0\\
512.01	0\\
513.01	0\\
514.01	0\\
515.01	0\\
516.01	0\\
517.01	0\\
518.01	0\\
519.01	0\\
520.01	0\\
521.01	0\\
522.01	0\\
523.01	0\\
524.01	0\\
525.01	0\\
526.01	0\\
527.01	0\\
528.01	0\\
529.01	0\\
530.01	0\\
531.01	0\\
532.01	0\\
533.01	0\\
534.01	0\\
535.01	0\\
536.01	0\\
537.01	0\\
538.01	0\\
539.01	0\\
540.01	0\\
541.01	0\\
542.01	0\\
543.01	0\\
544.01	0\\
545.01	0\\
546.01	0\\
547.01	0\\
548.01	0\\
549.01	0\\
550.01	0\\
551.01	0\\
552.01	0\\
553.01	0\\
554.01	0\\
555.01	0\\
556.01	0\\
557.01	0\\
558.01	0\\
559.01	0\\
560.01	0\\
561.01	0\\
562.01	0\\
563.01	0\\
564.01	0\\
565.01	0\\
566.01	0\\
567.01	0\\
568.01	0\\
569.01	0\\
570.01	0\\
571.01	0\\
572.01	0\\
573.01	0\\
574.01	0\\
575.01	0\\
576.01	0\\
577.01	0\\
578.01	0\\
579.01	0\\
580.01	0\\
581.01	0\\
582.01	0\\
583.01	0\\
584.01	0\\
585.01	0\\
586.01	0\\
587.01	0\\
588.01	0\\
589.01	0\\
590.01	0\\
591.01	0\\
592.01	0\\
593.01	0\\
594.01	0\\
595.01	0\\
596.01	0\\
597.01	0\\
598.01	0\\
599.01	0.00370208209411713\\
599.02	0.00374423082044583\\
599.03	0.00378664015556109\\
599.04	0.00382930214308572\\
599.05	0.00387220842322091\\
599.06	0.00391535001457705\\
599.07	0.00395871728279321\\
599.08	0.0040023034946127\\
599.09	0.00404610972041236\\
599.1	0.00409012556930278\\
599.11	0.00413433908054605\\
599.12	0.00417873761917283\\
599.13	0.00422330825718109\\
599.14	0.00426803650362734\\
599.15	0.00431290692702228\\
599.16	0.00435790310699198\\
599.17	0.00440300764863001\\
599.18	0.00444820215913953\\
599.19	0.00449346694222096\\
599.2	0.00453878109118118\\
599.21	0.00458428620748516\\
599.22	0.00463022774404547\\
599.23	0.00467660989602732\\
599.24	0.00472343691632505\\
599.25	0.00477071311787237\\
599.26	0.00481844287609376\\
599.27	0.00486663060728154\\
599.28	0.00491528074441994\\
599.29	0.00496439778463096\\
599.3	0.00501398629186756\\
599.31	0.00506405089766463\\
599.32	0.00511459630406084\\
599.33	0.00516562728865405\\
599.34	0.00521714870799032\\
599.35	0.00526916550026689\\
599.36	0.00532168268960642\\
599.37	0.00537470539040875\\
599.38	0.00542823881163046\\
599.39	0.00548228825765701\\
599.4	0.00553685912202147\\
599.41	0.00559195690690256\\
599.42	0.00564758722852464\\
599.43	0.00570375582288129\\
599.44	0.00576046855180099\\
599.45	0.00581773140937498\\
599.46	0.00587555052876837\\
599.47	0.00593393218943659\\
599.48	0.00599288282477091\\
599.49	0.00605240833238633\\
599.5	0.00611251432580253\\
599.51	0.00617320647360776\\
599.52	0.00623449050000827\\
599.53	0.00629637218551971\\
599.54	0.00635885736771363\\
599.55	0.00642195194182304\\
599.56	0.00648566186136196\\
599.57	0.00654999313875884\\
599.58	0.00661495184598612\\
599.59	0.00668054411520325\\
599.6	0.00674677613940176\\
599.61	0.00681365417305723\\
599.62	0.00688118453278823\\
599.63	0.00694937359802226\\
599.64	0.00701822781166827\\
599.65	0.00708775368079585\\
599.66	0.00715795777732064\\
599.67	0.00722884673869562\\
599.68	0.007300427268608\\
599.69	0.0073727061376814\\
599.7	0.00744569018418299\\
599.71	0.00751938631473504\\
599.72	0.00759380150502925\\
599.73	0.00766894280054146\\
599.74	0.00774481731724789\\
599.75	0.00782143224234223\\
599.76	0.00789879483495157\\
599.77	0.00797691242684947\\
599.78	0.00805579242316594\\
599.79	0.00813544230309276\\
599.8	0.00821586962058269\\
599.81	0.00829708200504056\\
599.82	0.0083790871620044\\
599.83	0.00846189287381431\\
599.84	0.00854550700026665\\
599.85	0.00862993747925072\\
599.86	0.00871519232736509\\
599.87	0.00880127964051009\\
599.88	0.00888820759445287\\
599.89	0.00897598444536097\\
599.9	0.00906461853029984\\
599.91	0.00915411826768946\\
599.92	0.00924449215771456\\
599.93	0.00933574878268246\\
599.94	0.00942789680732179\\
599.95	0.00952094497901499\\
599.96	0.00961490212795633\\
599.97	0.00970977716722672\\
599.98	0.00980557909277551\\
599.99	0.00990231698329844\\
600	0.01\\
};
\addplot [color=mycolor17,solid,forget plot]
  table[row sep=crcr]{%
0.01	0\\
1.01	0\\
2.01	0\\
3.01	0\\
4.01	0\\
5.01	0\\
6.01	0\\
7.01	0\\
8.01	0\\
9.01	0\\
10.01	0\\
11.01	0\\
12.01	0\\
13.01	0\\
14.01	0\\
15.01	0\\
16.01	0\\
17.01	0\\
18.01	0\\
19.01	0\\
20.01	0\\
21.01	0\\
22.01	0\\
23.01	0\\
24.01	0\\
25.01	0\\
26.01	0\\
27.01	0\\
28.01	0\\
29.01	0\\
30.01	0\\
31.01	0\\
32.01	0\\
33.01	0\\
34.01	0\\
35.01	0\\
36.01	0\\
37.01	0\\
38.01	0\\
39.01	0\\
40.01	0\\
41.01	0\\
42.01	0\\
43.01	0\\
44.01	0\\
45.01	0\\
46.01	0\\
47.01	0\\
48.01	0\\
49.01	0\\
50.01	0\\
51.01	0\\
52.01	0\\
53.01	0\\
54.01	0\\
55.01	0\\
56.01	0\\
57.01	0\\
58.01	0\\
59.01	0\\
60.01	0\\
61.01	0\\
62.01	0\\
63.01	0\\
64.01	0\\
65.01	0\\
66.01	0\\
67.01	0\\
68.01	0\\
69.01	0\\
70.01	0\\
71.01	0\\
72.01	0\\
73.01	0\\
74.01	0\\
75.01	0\\
76.01	0\\
77.01	0\\
78.01	0\\
79.01	0\\
80.01	0\\
81.01	0\\
82.01	0\\
83.01	0\\
84.01	0\\
85.01	0\\
86.01	0\\
87.01	0\\
88.01	0\\
89.01	0\\
90.01	0\\
91.01	0\\
92.01	0\\
93.01	0\\
94.01	0\\
95.01	0\\
96.01	0\\
97.01	0\\
98.01	0\\
99.01	0\\
100.01	0\\
101.01	0\\
102.01	0\\
103.01	0\\
104.01	0\\
105.01	0\\
106.01	0\\
107.01	0\\
108.01	0\\
109.01	0\\
110.01	0\\
111.01	0\\
112.01	0\\
113.01	0\\
114.01	0\\
115.01	0\\
116.01	0\\
117.01	0\\
118.01	0\\
119.01	0\\
120.01	0\\
121.01	0\\
122.01	0\\
123.01	0\\
124.01	0\\
125.01	0\\
126.01	0\\
127.01	0\\
128.01	0\\
129.01	0\\
130.01	0\\
131.01	0\\
132.01	0\\
133.01	0\\
134.01	0\\
135.01	0\\
136.01	0\\
137.01	0\\
138.01	0\\
139.01	0\\
140.01	0\\
141.01	0\\
142.01	0\\
143.01	0\\
144.01	0\\
145.01	0\\
146.01	0\\
147.01	0\\
148.01	0\\
149.01	0\\
150.01	0\\
151.01	0\\
152.01	0\\
153.01	0\\
154.01	0\\
155.01	0\\
156.01	0\\
157.01	0\\
158.01	0\\
159.01	0\\
160.01	0\\
161.01	0\\
162.01	0\\
163.01	0\\
164.01	0\\
165.01	0\\
166.01	0\\
167.01	0\\
168.01	0\\
169.01	0\\
170.01	0\\
171.01	0\\
172.01	0\\
173.01	0\\
174.01	0\\
175.01	0\\
176.01	0\\
177.01	0\\
178.01	0\\
179.01	0\\
180.01	0\\
181.01	0\\
182.01	0\\
183.01	0\\
184.01	0\\
185.01	0\\
186.01	0\\
187.01	0\\
188.01	0\\
189.01	0\\
190.01	0\\
191.01	0\\
192.01	0\\
193.01	0\\
194.01	0\\
195.01	0\\
196.01	0\\
197.01	0\\
198.01	0\\
199.01	0\\
200.01	0\\
201.01	0\\
202.01	0\\
203.01	0\\
204.01	0\\
205.01	0\\
206.01	0\\
207.01	0\\
208.01	0\\
209.01	0\\
210.01	0\\
211.01	0\\
212.01	0\\
213.01	0\\
214.01	0\\
215.01	0\\
216.01	0\\
217.01	0\\
218.01	0\\
219.01	0\\
220.01	0\\
221.01	0\\
222.01	0\\
223.01	0\\
224.01	0\\
225.01	0\\
226.01	0\\
227.01	0\\
228.01	0\\
229.01	0\\
230.01	0\\
231.01	0\\
232.01	0\\
233.01	0\\
234.01	0\\
235.01	0\\
236.01	0\\
237.01	0\\
238.01	0\\
239.01	0\\
240.01	0\\
241.01	0\\
242.01	0\\
243.01	0\\
244.01	0\\
245.01	0\\
246.01	0\\
247.01	0\\
248.01	0\\
249.01	0\\
250.01	0\\
251.01	0\\
252.01	0\\
253.01	0\\
254.01	0\\
255.01	0\\
256.01	0\\
257.01	0\\
258.01	0\\
259.01	0\\
260.01	0\\
261.01	0\\
262.01	0\\
263.01	0\\
264.01	0\\
265.01	0\\
266.01	0\\
267.01	0\\
268.01	0\\
269.01	0\\
270.01	0\\
271.01	0\\
272.01	0\\
273.01	0\\
274.01	0\\
275.01	0\\
276.01	0\\
277.01	0\\
278.01	0\\
279.01	0\\
280.01	0\\
281.01	0\\
282.01	0\\
283.01	0\\
284.01	0\\
285.01	0\\
286.01	0\\
287.01	0\\
288.01	0\\
289.01	0\\
290.01	0\\
291.01	0\\
292.01	0\\
293.01	0\\
294.01	0\\
295.01	0\\
296.01	0\\
297.01	0\\
298.01	0\\
299.01	0\\
300.01	0\\
301.01	0\\
302.01	0\\
303.01	0\\
304.01	0\\
305.01	0\\
306.01	0\\
307.01	0\\
308.01	0\\
309.01	0\\
310.01	0\\
311.01	0\\
312.01	0\\
313.01	0\\
314.01	0\\
315.01	0\\
316.01	0\\
317.01	0\\
318.01	0\\
319.01	0\\
320.01	0\\
321.01	0\\
322.01	0\\
323.01	0\\
324.01	0\\
325.01	0\\
326.01	0\\
327.01	0\\
328.01	0\\
329.01	0\\
330.01	0\\
331.01	0\\
332.01	0\\
333.01	0\\
334.01	0\\
335.01	0\\
336.01	0\\
337.01	0\\
338.01	0\\
339.01	0\\
340.01	0\\
341.01	0\\
342.01	0\\
343.01	0\\
344.01	0\\
345.01	0\\
346.01	0\\
347.01	0\\
348.01	0\\
349.01	0\\
350.01	0\\
351.01	0\\
352.01	0\\
353.01	0\\
354.01	0\\
355.01	0\\
356.01	0\\
357.01	0\\
358.01	0\\
359.01	0\\
360.01	0\\
361.01	0\\
362.01	0\\
363.01	0\\
364.01	0\\
365.01	0\\
366.01	0\\
367.01	0\\
368.01	0\\
369.01	0\\
370.01	0\\
371.01	0\\
372.01	0\\
373.01	0\\
374.01	0\\
375.01	0\\
376.01	0\\
377.01	0\\
378.01	0\\
379.01	0\\
380.01	0\\
381.01	0\\
382.01	0\\
383.01	0\\
384.01	0\\
385.01	0\\
386.01	0\\
387.01	0\\
388.01	0\\
389.01	0\\
390.01	0\\
391.01	0\\
392.01	0\\
393.01	0\\
394.01	0\\
395.01	0\\
396.01	0\\
397.01	0\\
398.01	0\\
399.01	0\\
400.01	0\\
401.01	0\\
402.01	0\\
403.01	0\\
404.01	0\\
405.01	0\\
406.01	0\\
407.01	0\\
408.01	0\\
409.01	0\\
410.01	0\\
411.01	0\\
412.01	0\\
413.01	0\\
414.01	0\\
415.01	0\\
416.01	0\\
417.01	0\\
418.01	0\\
419.01	0\\
420.01	0\\
421.01	0\\
422.01	0\\
423.01	0\\
424.01	0\\
425.01	0\\
426.01	0\\
427.01	0\\
428.01	0\\
429.01	0\\
430.01	0\\
431.01	0\\
432.01	0\\
433.01	0\\
434.01	0\\
435.01	0\\
436.01	0\\
437.01	0\\
438.01	0\\
439.01	0\\
440.01	0\\
441.01	0\\
442.01	0\\
443.01	0\\
444.01	0\\
445.01	0\\
446.01	0\\
447.01	0\\
448.01	0\\
449.01	0\\
450.01	0\\
451.01	0\\
452.01	0\\
453.01	0\\
454.01	0\\
455.01	0\\
456.01	0\\
457.01	0\\
458.01	0\\
459.01	0\\
460.01	0\\
461.01	0\\
462.01	0\\
463.01	0\\
464.01	0\\
465.01	0\\
466.01	0\\
467.01	0\\
468.01	0\\
469.01	0\\
470.01	0\\
471.01	0\\
472.01	0\\
473.01	0\\
474.01	0\\
475.01	0\\
476.01	0\\
477.01	0\\
478.01	0\\
479.01	0\\
480.01	0\\
481.01	0\\
482.01	0\\
483.01	0\\
484.01	0\\
485.01	0\\
486.01	0\\
487.01	0\\
488.01	0\\
489.01	0\\
490.01	0\\
491.01	0\\
492.01	0\\
493.01	0\\
494.01	0\\
495.01	0\\
496.01	0\\
497.01	0\\
498.01	0\\
499.01	0\\
500.01	0\\
501.01	0\\
502.01	0\\
503.01	0\\
504.01	0\\
505.01	0\\
506.01	0\\
507.01	0\\
508.01	0\\
509.01	0\\
510.01	0\\
511.01	0\\
512.01	0\\
513.01	0\\
514.01	0\\
515.01	0\\
516.01	0\\
517.01	0\\
518.01	0\\
519.01	0\\
520.01	0\\
521.01	0\\
522.01	0\\
523.01	0\\
524.01	0\\
525.01	0\\
526.01	0\\
527.01	0\\
528.01	0\\
529.01	0\\
530.01	0\\
531.01	0\\
532.01	0\\
533.01	0\\
534.01	0\\
535.01	0\\
536.01	0\\
537.01	0\\
538.01	0\\
539.01	0\\
540.01	0\\
541.01	0\\
542.01	0\\
543.01	0\\
544.01	0\\
545.01	0\\
546.01	0\\
547.01	0\\
548.01	0\\
549.01	0\\
550.01	0\\
551.01	0\\
552.01	0\\
553.01	0\\
554.01	0\\
555.01	0\\
556.01	0\\
557.01	0\\
558.01	0\\
559.01	0\\
560.01	0\\
561.01	0\\
562.01	0\\
563.01	0\\
564.01	0\\
565.01	0\\
566.01	0\\
567.01	0\\
568.01	0\\
569.01	0\\
570.01	0\\
571.01	0\\
572.01	0\\
573.01	0\\
574.01	0\\
575.01	0\\
576.01	0\\
577.01	0\\
578.01	0\\
579.01	0\\
580.01	0\\
581.01	0\\
582.01	0\\
583.01	0\\
584.01	0\\
585.01	0\\
586.01	0\\
587.01	0\\
588.01	0\\
589.01	0\\
590.01	0\\
591.01	0\\
592.01	0\\
593.01	0\\
594.01	0\\
595.01	0\\
596.01	0\\
597.01	0\\
598.01	0\\
599.01	0.0037573870629064\\
599.02	0.00379519556360755\\
599.03	0.00383336480333374\\
599.04	0.00387189858986088\\
599.05	0.00391080060823229\\
599.06	0.00395007461228412\\
599.07	0.00398972442806679\\
599.08	0.00402975394099463\\
599.09	0.00407016706079863\\
599.1	0.00411096777072455\\
599.11	0.00415216013486262\\
599.12	0.00419374830159151\\
599.13	0.00423573650508695\\
599.14	0.00427812907311622\\
599.15	0.00432093043192307\\
599.16	0.00436414511140291\\
599.17	0.00440777775022103\\
599.18	0.00445183310105154\\
599.19	0.00449631603752963\\
599.2	0.00454123156074996\\
599.21	0.00458658444372716\\
599.22	0.00463237898770875\\
599.23	0.00467861953666444\\
599.24	0.00472531047768157\\
599.25	0.00477245624135829\\
599.26	0.00482006130219408\\
599.27	0.00486813017906603\\
599.28	0.00491666743580523\\
599.29	0.00496567768161336\\
599.3	0.00501516557147688\\
599.31	0.00506513580658655\\
599.32	0.00511559313475406\\
599.33	0.00516654235081657\\
599.34	0.00521798829703587\\
599.35	0.00526993586349498\\
599.36	0.00532238998848544\\
599.37	0.00537535565888448\\
599.38	0.00542883791052227\\
599.39	0.00548284182457115\\
599.4	0.00553737251745407\\
599.41	0.00559243515521324\\
599.42	0.00564803495381788\\
599.43	0.00570417717945396\\
599.44	0.0057608671487941\\
599.45	0.00581811022924536\\
599.46	0.00587591183917258\\
599.47	0.00593427744809491\\
599.48	0.0059932125768525\\
599.49	0.00605272279952553\\
599.5	0.00611281374479376\\
599.51	0.00617349109647296\\
599.52	0.00623476059405683\\
599.53	0.00629662803326369\\
599.54	0.006359099266588\\
599.55	0.00642218020385709\\
599.56	0.00648587681279326\\
599.57	0.00655019511958122\\
599.58	0.00661514120944084\\
599.59	0.00668072122720545\\
599.6	0.00674694137790552\\
599.61	0.00681380792735788\\
599.62	0.00688132720276059\\
599.63	0.00694950559329325\\
599.64	0.00701834955072308\\
599.65	0.00708786559001664\\
599.66	0.00715806028995723\\
599.67	0.00722894029376811\\
599.68	0.00730051230974149\\
599.69	0.00737278311187335\\
599.7	0.00744575954050419\\
599.71	0.00751944850296568\\
599.72	0.00759385697423329\\
599.73	0.007668991997585\\
599.74	0.00774486068526607\\
599.75	0.00782147021915993\\
599.76	0.0078988278514654\\
599.77	0.0079769409053801\\
599.78	0.00805581677579022\\
599.79	0.00813546292996683\\
599.8	0.00821588690826864\\
599.81	0.00829709632485142\\
599.82	0.00837909886838419\\
599.83	0.00846190230277219\\
599.84	0.008545514467887\\
599.85	0.00862994328030364\\
599.86	0.00871519673404507\\
599.87	0.00880128290133424\\
599.88	0.00888820993335373\\
599.89	0.00897598606101348\\
599.9	0.00906461959572664\\
599.91	0.009154118930194\\
599.92	0.00924449253919727\\
599.93	0.00933574898040158\\
599.94	0.00942789689516777\\
599.95	0.00952094500937471\\
599.96	0.00961490213425244\\
599.97	0.00970977716722672\\
599.98	0.00980557909277551\\
599.99	0.00990231698329844\\
600	0.01\\
};
\addplot [color=mycolor18,solid,forget plot]
  table[row sep=crcr]{%
0.01	0\\
1.01	0\\
2.01	0\\
3.01	0\\
4.01	0\\
5.01	0\\
6.01	0\\
7.01	0\\
8.01	0\\
9.01	0\\
10.01	0\\
11.01	0\\
12.01	0\\
13.01	0\\
14.01	0\\
15.01	0\\
16.01	0\\
17.01	0\\
18.01	0\\
19.01	0\\
20.01	0\\
21.01	0\\
22.01	0\\
23.01	0\\
24.01	0\\
25.01	0\\
26.01	0\\
27.01	0\\
28.01	0\\
29.01	0\\
30.01	0\\
31.01	0\\
32.01	0\\
33.01	0\\
34.01	0\\
35.01	0\\
36.01	0\\
37.01	0\\
38.01	0\\
39.01	0\\
40.01	0\\
41.01	0\\
42.01	0\\
43.01	0\\
44.01	0\\
45.01	0\\
46.01	0\\
47.01	0\\
48.01	0\\
49.01	0\\
50.01	0\\
51.01	0\\
52.01	0\\
53.01	0\\
54.01	0\\
55.01	0\\
56.01	0\\
57.01	0\\
58.01	0\\
59.01	0\\
60.01	0\\
61.01	0\\
62.01	0\\
63.01	0\\
64.01	0\\
65.01	0\\
66.01	0\\
67.01	0\\
68.01	0\\
69.01	0\\
70.01	0\\
71.01	0\\
72.01	0\\
73.01	0\\
74.01	0\\
75.01	0\\
76.01	0\\
77.01	0\\
78.01	0\\
79.01	0\\
80.01	0\\
81.01	0\\
82.01	0\\
83.01	0\\
84.01	0\\
85.01	0\\
86.01	0\\
87.01	0\\
88.01	0\\
89.01	0\\
90.01	0\\
91.01	0\\
92.01	0\\
93.01	0\\
94.01	0\\
95.01	0\\
96.01	0\\
97.01	0\\
98.01	0\\
99.01	0\\
100.01	0\\
101.01	0\\
102.01	0\\
103.01	0\\
104.01	0\\
105.01	0\\
106.01	0\\
107.01	0\\
108.01	0\\
109.01	0\\
110.01	0\\
111.01	0\\
112.01	0\\
113.01	0\\
114.01	0\\
115.01	0\\
116.01	0\\
117.01	0\\
118.01	0\\
119.01	0\\
120.01	0\\
121.01	0\\
122.01	0\\
123.01	0\\
124.01	0\\
125.01	0\\
126.01	0\\
127.01	0\\
128.01	0\\
129.01	0\\
130.01	0\\
131.01	0\\
132.01	0\\
133.01	0\\
134.01	0\\
135.01	0\\
136.01	0\\
137.01	0\\
138.01	0\\
139.01	0\\
140.01	0\\
141.01	0\\
142.01	0\\
143.01	0\\
144.01	0\\
145.01	0\\
146.01	0\\
147.01	0\\
148.01	0\\
149.01	0\\
150.01	0\\
151.01	0\\
152.01	0\\
153.01	0\\
154.01	0\\
155.01	0\\
156.01	0\\
157.01	0\\
158.01	0\\
159.01	0\\
160.01	0\\
161.01	0\\
162.01	0\\
163.01	0\\
164.01	0\\
165.01	0\\
166.01	0\\
167.01	0\\
168.01	0\\
169.01	0\\
170.01	0\\
171.01	0\\
172.01	0\\
173.01	0\\
174.01	0\\
175.01	0\\
176.01	0\\
177.01	0\\
178.01	0\\
179.01	0\\
180.01	0\\
181.01	0\\
182.01	0\\
183.01	0\\
184.01	0\\
185.01	0\\
186.01	0\\
187.01	0\\
188.01	0\\
189.01	0\\
190.01	0\\
191.01	0\\
192.01	0\\
193.01	0\\
194.01	0\\
195.01	0\\
196.01	0\\
197.01	0\\
198.01	0\\
199.01	0\\
200.01	0\\
201.01	0\\
202.01	0\\
203.01	0\\
204.01	0\\
205.01	0\\
206.01	0\\
207.01	0\\
208.01	0\\
209.01	0\\
210.01	0\\
211.01	0\\
212.01	0\\
213.01	0\\
214.01	0\\
215.01	0\\
216.01	0\\
217.01	0\\
218.01	0\\
219.01	0\\
220.01	0\\
221.01	0\\
222.01	0\\
223.01	0\\
224.01	0\\
225.01	0\\
226.01	0\\
227.01	0\\
228.01	0\\
229.01	0\\
230.01	0\\
231.01	0\\
232.01	0\\
233.01	0\\
234.01	0\\
235.01	0\\
236.01	0\\
237.01	0\\
238.01	0\\
239.01	0\\
240.01	0\\
241.01	0\\
242.01	0\\
243.01	0\\
244.01	0\\
245.01	0\\
246.01	0\\
247.01	0\\
248.01	0\\
249.01	0\\
250.01	0\\
251.01	0\\
252.01	0\\
253.01	0\\
254.01	0\\
255.01	0\\
256.01	0\\
257.01	0\\
258.01	0\\
259.01	0\\
260.01	0\\
261.01	0\\
262.01	0\\
263.01	0\\
264.01	0\\
265.01	0\\
266.01	0\\
267.01	0\\
268.01	0\\
269.01	0\\
270.01	0\\
271.01	0\\
272.01	0\\
273.01	0\\
274.01	0\\
275.01	0\\
276.01	0\\
277.01	0\\
278.01	0\\
279.01	0\\
280.01	0\\
281.01	0\\
282.01	0\\
283.01	0\\
284.01	0\\
285.01	0\\
286.01	0\\
287.01	0\\
288.01	0\\
289.01	0\\
290.01	0\\
291.01	0\\
292.01	0\\
293.01	0\\
294.01	0\\
295.01	0\\
296.01	0\\
297.01	0\\
298.01	0\\
299.01	0\\
300.01	0\\
301.01	0\\
302.01	0\\
303.01	0\\
304.01	0\\
305.01	0\\
306.01	0\\
307.01	0\\
308.01	0\\
309.01	0\\
310.01	0\\
311.01	0\\
312.01	0\\
313.01	0\\
314.01	0\\
315.01	0\\
316.01	0\\
317.01	0\\
318.01	0\\
319.01	0\\
320.01	0\\
321.01	0\\
322.01	0\\
323.01	0\\
324.01	0\\
325.01	0\\
326.01	0\\
327.01	0\\
328.01	0\\
329.01	0\\
330.01	0\\
331.01	0\\
332.01	0\\
333.01	0\\
334.01	0\\
335.01	0\\
336.01	0\\
337.01	0\\
338.01	0\\
339.01	0\\
340.01	0\\
341.01	0\\
342.01	0\\
343.01	0\\
344.01	0\\
345.01	0\\
346.01	0\\
347.01	0\\
348.01	0\\
349.01	0\\
350.01	0\\
351.01	0\\
352.01	0\\
353.01	0\\
354.01	0\\
355.01	0\\
356.01	0\\
357.01	0\\
358.01	0\\
359.01	0\\
360.01	0\\
361.01	0\\
362.01	0\\
363.01	0\\
364.01	0\\
365.01	0\\
366.01	0\\
367.01	0\\
368.01	0\\
369.01	0\\
370.01	0\\
371.01	0\\
372.01	0\\
373.01	0\\
374.01	0\\
375.01	0\\
376.01	0\\
377.01	0\\
378.01	0\\
379.01	0\\
380.01	0\\
381.01	0\\
382.01	0\\
383.01	0\\
384.01	0\\
385.01	0\\
386.01	0\\
387.01	0\\
388.01	0\\
389.01	0\\
390.01	0\\
391.01	0\\
392.01	0\\
393.01	0\\
394.01	0\\
395.01	0\\
396.01	0\\
397.01	0\\
398.01	0\\
399.01	0\\
400.01	0\\
401.01	0\\
402.01	0\\
403.01	0\\
404.01	0\\
405.01	0\\
406.01	0\\
407.01	0\\
408.01	0\\
409.01	0\\
410.01	0\\
411.01	0\\
412.01	0\\
413.01	0\\
414.01	0\\
415.01	0\\
416.01	0\\
417.01	0\\
418.01	0\\
419.01	0\\
420.01	0\\
421.01	0\\
422.01	0\\
423.01	0\\
424.01	0\\
425.01	0\\
426.01	0\\
427.01	0\\
428.01	0\\
429.01	0\\
430.01	0\\
431.01	0\\
432.01	0\\
433.01	0\\
434.01	0\\
435.01	0\\
436.01	0\\
437.01	0\\
438.01	0\\
439.01	0\\
440.01	0\\
441.01	0\\
442.01	0\\
443.01	0\\
444.01	0\\
445.01	0\\
446.01	0\\
447.01	0\\
448.01	0\\
449.01	0\\
450.01	0\\
451.01	0\\
452.01	0\\
453.01	0\\
454.01	0\\
455.01	0\\
456.01	0\\
457.01	0\\
458.01	0\\
459.01	0\\
460.01	0\\
461.01	0\\
462.01	0\\
463.01	0\\
464.01	0\\
465.01	0\\
466.01	0\\
467.01	0\\
468.01	0\\
469.01	0\\
470.01	0\\
471.01	0\\
472.01	0\\
473.01	0\\
474.01	0\\
475.01	0\\
476.01	0\\
477.01	0\\
478.01	0\\
479.01	0\\
480.01	0\\
481.01	0\\
482.01	0\\
483.01	0\\
484.01	0\\
485.01	0\\
486.01	0\\
487.01	0\\
488.01	0\\
489.01	0\\
490.01	0\\
491.01	0\\
492.01	0\\
493.01	0\\
494.01	0\\
495.01	0\\
496.01	0\\
497.01	0\\
498.01	0\\
499.01	0\\
500.01	0\\
501.01	0\\
502.01	0\\
503.01	0\\
504.01	0\\
505.01	0\\
506.01	0\\
507.01	0\\
508.01	0\\
509.01	0\\
510.01	0\\
511.01	0\\
512.01	0\\
513.01	0\\
514.01	0\\
515.01	0\\
516.01	0\\
517.01	0\\
518.01	0\\
519.01	0\\
520.01	0\\
521.01	0\\
522.01	0\\
523.01	0\\
524.01	0\\
525.01	0\\
526.01	0\\
527.01	0\\
528.01	0\\
529.01	0\\
530.01	0\\
531.01	0\\
532.01	0\\
533.01	0\\
534.01	0\\
535.01	0\\
536.01	0\\
537.01	0\\
538.01	0\\
539.01	0\\
540.01	0\\
541.01	0\\
542.01	0\\
543.01	0\\
544.01	0\\
545.01	0\\
546.01	0\\
547.01	0\\
548.01	0\\
549.01	0\\
550.01	0\\
551.01	0\\
552.01	0\\
553.01	0\\
554.01	0\\
555.01	0\\
556.01	0\\
557.01	0\\
558.01	0\\
559.01	0\\
560.01	0\\
561.01	0\\
562.01	0\\
563.01	0\\
564.01	0\\
565.01	0\\
566.01	0\\
567.01	0\\
568.01	0\\
569.01	0\\
570.01	0\\
571.01	0\\
572.01	0\\
573.01	0\\
574.01	0\\
575.01	0\\
576.01	0\\
577.01	0\\
578.01	0\\
579.01	0\\
580.01	0\\
581.01	0\\
582.01	0\\
583.01	0\\
584.01	0\\
585.01	0\\
586.01	0\\
587.01	0\\
588.01	0\\
589.01	0\\
590.01	0\\
591.01	0\\
592.01	0\\
593.01	0\\
594.01	0\\
595.01	0\\
596.01	0\\
597.01	0\\
598.01	0\\
599.01	0.00375812205680727\\
599.02	0.00379584782681452\\
599.03	0.00383394044928508\\
599.04	0.00387240353649723\\
599.05	0.00391124073635877\\
599.06	0.00395045573268432\\
599.07	0.00399005224546545\\
599.08	0.00403003403118953\\
599.09	0.0040704048832826\\
599.1	0.00411116863239879\\
599.11	0.0041523291466905\\
599.12	0.00419389033207165\\
599.13	0.00423585613248164\\
599.14	0.00427823053012622\\
599.15	0.00432101754570572\\
599.16	0.00436422123862898\\
599.17	0.00440784570721336\\
599.18	0.0044518950888699\\
599.19	0.00449637356026637\\
599.2	0.00454128533746953\\
599.21	0.00458663467686692\\
599.22	0.00463242587666335\\
599.23	0.00467866327729075\\
599.24	0.00472535126182209\\
599.25	0.00477249425638942\\
599.26	0.00482009673060606\\
599.27	0.00486816319799273\\
599.28	0.00491669821640737\\
599.29	0.00496570638847912\\
599.3	0.00501519236204661\\
599.31	0.00506516083060044\\
599.32	0.00511561653373015\\
599.33	0.00516656425757546\\
599.34	0.00521800883528218\\
599.35	0.00526995514746255\\
599.36	0.0053224081226603\\
599.37	0.00537537273782044\\
599.38	0.00542885401876388\\
599.39	0.00548285703669257\\
599.4	0.00553738689825535\\
599.41	0.00559244876006856\\
599.42	0.00564804782920575\\
599.43	0.00570418936369266\\
599.44	0.00576087867300769\\
599.45	0.00581812111858793\\
599.46	0.00587592211434108\\
599.47	0.00593428712716332\\
599.48	0.00599322167746327\\
599.49	0.00605273133968783\\
599.5	0.00611282174285112\\
599.51	0.00617349857106859\\
599.52	0.00623476756409629\\
599.53	0.00629663451787551\\
599.54	0.00635910528508266\\
599.55	0.00642218577568459\\
599.56	0.00648588195749926\\
599.57	0.00655019985676195\\
599.58	0.00661514555869698\\
599.59	0.00668072520809509\\
599.6	0.00674694500989631\\
599.61	0.00681381122977868\\
599.62	0.00688133019475265\\
599.63	0.00694950829376133\\
599.64	0.00701835197828654\\
599.65	0.00708786776296088\\
599.66	0.00715806222618573\\
599.67	0.0072289420107553\\
599.68	0.00730051382448678\\
599.69	0.00737278444085667\\
599.7	0.00744576069964336\\
599.71	0.00751944950757594\\
599.72	0.00759385783898938\\
599.73	0.0076689927364862\\
599.74	0.00774486131160447\\
599.75	0.00782147074549251\\
599.76	0.00789882828959007\\
599.77	0.0079769412663163\\
599.78	0.00805581706976437\\
599.79	0.00813546316640295\\
599.8	0.00821588709578454\\
599.81	0.00829709647126081\\
599.82	0.00837909898070485\\
599.83	0.00846190238724062\\
599.84	0.00854551452997949\\
599.85	0.00862994332476403\\
599.86	0.00871519676491913\\
599.87	0.00880128292201045\\
599.88	0.00888820994661034\\
599.89	0.00897598606907129\\
599.9	0.00906461960030698\\
599.91	0.00915411893258095\\
599.92	0.00924449254030311\\
599.93	0.00933574898083391\\
599.94	0.00942789689529656\\
599.95	0.00952094500939709\\
599.96	0.00961490213425245\\
599.97	0.00970977716722672\\
599.98	0.00980557909277551\\
599.99	0.00990231698329844\\
600	0.01\\
};
\addplot [color=red!25!mycolor17,solid,forget plot]
  table[row sep=crcr]{%
0.01	0\\
1.01	0\\
2.01	0\\
3.01	0\\
4.01	0\\
5.01	0\\
6.01	0\\
7.01	0\\
8.01	0\\
9.01	0\\
10.01	0\\
11.01	0\\
12.01	0\\
13.01	0\\
14.01	0\\
15.01	0\\
16.01	0\\
17.01	0\\
18.01	0\\
19.01	0\\
20.01	0\\
21.01	0\\
22.01	0\\
23.01	0\\
24.01	0\\
25.01	0\\
26.01	0\\
27.01	0\\
28.01	0\\
29.01	0\\
30.01	0\\
31.01	0\\
32.01	0\\
33.01	0\\
34.01	0\\
35.01	0\\
36.01	0\\
37.01	0\\
38.01	0\\
39.01	0\\
40.01	0\\
41.01	0\\
42.01	0\\
43.01	0\\
44.01	0\\
45.01	0\\
46.01	0\\
47.01	0\\
48.01	0\\
49.01	0\\
50.01	0\\
51.01	0\\
52.01	0\\
53.01	0\\
54.01	0\\
55.01	0\\
56.01	0\\
57.01	0\\
58.01	0\\
59.01	0\\
60.01	0\\
61.01	0\\
62.01	0\\
63.01	0\\
64.01	0\\
65.01	0\\
66.01	0\\
67.01	0\\
68.01	0\\
69.01	0\\
70.01	0\\
71.01	0\\
72.01	0\\
73.01	0\\
74.01	0\\
75.01	0\\
76.01	0\\
77.01	0\\
78.01	0\\
79.01	0\\
80.01	0\\
81.01	0\\
82.01	0\\
83.01	0\\
84.01	0\\
85.01	0\\
86.01	0\\
87.01	0\\
88.01	0\\
89.01	0\\
90.01	0\\
91.01	0\\
92.01	0\\
93.01	0\\
94.01	0\\
95.01	0\\
96.01	0\\
97.01	0\\
98.01	0\\
99.01	0\\
100.01	0\\
101.01	0\\
102.01	0\\
103.01	0\\
104.01	0\\
105.01	0\\
106.01	0\\
107.01	0\\
108.01	0\\
109.01	0\\
110.01	0\\
111.01	0\\
112.01	0\\
113.01	0\\
114.01	0\\
115.01	0\\
116.01	0\\
117.01	0\\
118.01	0\\
119.01	0\\
120.01	0\\
121.01	0\\
122.01	0\\
123.01	0\\
124.01	0\\
125.01	0\\
126.01	0\\
127.01	0\\
128.01	0\\
129.01	0\\
130.01	0\\
131.01	0\\
132.01	0\\
133.01	0\\
134.01	0\\
135.01	0\\
136.01	0\\
137.01	0\\
138.01	0\\
139.01	0\\
140.01	0\\
141.01	0\\
142.01	0\\
143.01	0\\
144.01	0\\
145.01	0\\
146.01	0\\
147.01	0\\
148.01	0\\
149.01	0\\
150.01	0\\
151.01	0\\
152.01	0\\
153.01	0\\
154.01	0\\
155.01	0\\
156.01	0\\
157.01	0\\
158.01	0\\
159.01	0\\
160.01	0\\
161.01	0\\
162.01	0\\
163.01	0\\
164.01	0\\
165.01	0\\
166.01	0\\
167.01	0\\
168.01	0\\
169.01	0\\
170.01	0\\
171.01	0\\
172.01	0\\
173.01	0\\
174.01	0\\
175.01	0\\
176.01	0\\
177.01	0\\
178.01	0\\
179.01	0\\
180.01	0\\
181.01	0\\
182.01	0\\
183.01	0\\
184.01	0\\
185.01	0\\
186.01	0\\
187.01	0\\
188.01	0\\
189.01	0\\
190.01	0\\
191.01	0\\
192.01	0\\
193.01	0\\
194.01	0\\
195.01	0\\
196.01	0\\
197.01	0\\
198.01	0\\
199.01	0\\
200.01	0\\
201.01	0\\
202.01	0\\
203.01	0\\
204.01	0\\
205.01	0\\
206.01	0\\
207.01	0\\
208.01	0\\
209.01	0\\
210.01	0\\
211.01	0\\
212.01	0\\
213.01	0\\
214.01	0\\
215.01	0\\
216.01	0\\
217.01	0\\
218.01	0\\
219.01	0\\
220.01	0\\
221.01	0\\
222.01	0\\
223.01	0\\
224.01	0\\
225.01	0\\
226.01	0\\
227.01	0\\
228.01	0\\
229.01	0\\
230.01	0\\
231.01	0\\
232.01	0\\
233.01	0\\
234.01	0\\
235.01	0\\
236.01	0\\
237.01	0\\
238.01	0\\
239.01	0\\
240.01	0\\
241.01	0\\
242.01	0\\
243.01	0\\
244.01	0\\
245.01	0\\
246.01	0\\
247.01	0\\
248.01	0\\
249.01	0\\
250.01	0\\
251.01	0\\
252.01	0\\
253.01	0\\
254.01	0\\
255.01	0\\
256.01	0\\
257.01	0\\
258.01	0\\
259.01	0\\
260.01	0\\
261.01	0\\
262.01	0\\
263.01	0\\
264.01	0\\
265.01	0\\
266.01	0\\
267.01	0\\
268.01	0\\
269.01	0\\
270.01	0\\
271.01	0\\
272.01	0\\
273.01	0\\
274.01	0\\
275.01	0\\
276.01	0\\
277.01	0\\
278.01	0\\
279.01	0\\
280.01	0\\
281.01	0\\
282.01	0\\
283.01	0\\
284.01	0\\
285.01	0\\
286.01	0\\
287.01	0\\
288.01	0\\
289.01	0\\
290.01	0\\
291.01	0\\
292.01	0\\
293.01	0\\
294.01	0\\
295.01	0\\
296.01	0\\
297.01	0\\
298.01	0\\
299.01	0\\
300.01	0\\
301.01	0\\
302.01	0\\
303.01	0\\
304.01	0\\
305.01	0\\
306.01	0\\
307.01	0\\
308.01	0\\
309.01	0\\
310.01	0\\
311.01	0\\
312.01	0\\
313.01	0\\
314.01	0\\
315.01	0\\
316.01	0\\
317.01	0\\
318.01	0\\
319.01	0\\
320.01	0\\
321.01	0\\
322.01	0\\
323.01	0\\
324.01	0\\
325.01	0\\
326.01	0\\
327.01	0\\
328.01	0\\
329.01	0\\
330.01	0\\
331.01	0\\
332.01	0\\
333.01	0\\
334.01	0\\
335.01	0\\
336.01	0\\
337.01	0\\
338.01	0\\
339.01	0\\
340.01	0\\
341.01	0\\
342.01	0\\
343.01	0\\
344.01	0\\
345.01	0\\
346.01	0\\
347.01	0\\
348.01	0\\
349.01	0\\
350.01	0\\
351.01	0\\
352.01	0\\
353.01	0\\
354.01	0\\
355.01	0\\
356.01	0\\
357.01	0\\
358.01	0\\
359.01	0\\
360.01	0\\
361.01	0\\
362.01	0\\
363.01	0\\
364.01	0\\
365.01	0\\
366.01	0\\
367.01	0\\
368.01	0\\
369.01	0\\
370.01	0\\
371.01	0\\
372.01	0\\
373.01	0\\
374.01	0\\
375.01	0\\
376.01	0\\
377.01	0\\
378.01	0\\
379.01	0\\
380.01	0\\
381.01	0\\
382.01	0\\
383.01	0\\
384.01	0\\
385.01	0\\
386.01	0\\
387.01	0\\
388.01	0\\
389.01	0\\
390.01	0\\
391.01	0\\
392.01	0\\
393.01	0\\
394.01	0\\
395.01	0\\
396.01	0\\
397.01	0\\
398.01	0\\
399.01	0\\
400.01	0\\
401.01	0\\
402.01	0\\
403.01	0\\
404.01	0\\
405.01	0\\
406.01	0\\
407.01	0\\
408.01	0\\
409.01	0\\
410.01	0\\
411.01	0\\
412.01	0\\
413.01	0\\
414.01	0\\
415.01	0\\
416.01	0\\
417.01	0\\
418.01	0\\
419.01	0\\
420.01	0\\
421.01	0\\
422.01	0\\
423.01	0\\
424.01	0\\
425.01	0\\
426.01	0\\
427.01	0\\
428.01	0\\
429.01	0\\
430.01	0\\
431.01	0\\
432.01	0\\
433.01	0\\
434.01	0\\
435.01	0\\
436.01	0\\
437.01	0\\
438.01	0\\
439.01	0\\
440.01	0\\
441.01	0\\
442.01	0\\
443.01	0\\
444.01	0\\
445.01	0\\
446.01	0\\
447.01	0\\
448.01	0\\
449.01	0\\
450.01	0\\
451.01	0\\
452.01	0\\
453.01	0\\
454.01	0\\
455.01	0\\
456.01	0\\
457.01	0\\
458.01	0\\
459.01	0\\
460.01	0\\
461.01	0\\
462.01	0\\
463.01	0\\
464.01	0\\
465.01	0\\
466.01	0\\
467.01	0\\
468.01	0\\
469.01	0\\
470.01	0\\
471.01	0\\
472.01	0\\
473.01	0\\
474.01	0\\
475.01	0\\
476.01	0\\
477.01	0\\
478.01	0\\
479.01	0\\
480.01	0\\
481.01	0\\
482.01	0\\
483.01	0\\
484.01	0\\
485.01	0\\
486.01	0\\
487.01	0\\
488.01	0\\
489.01	0\\
490.01	0\\
491.01	0\\
492.01	0\\
493.01	0\\
494.01	0\\
495.01	0\\
496.01	0\\
497.01	0\\
498.01	0\\
499.01	0\\
500.01	0\\
501.01	0\\
502.01	0\\
503.01	0\\
504.01	0\\
505.01	0\\
506.01	0\\
507.01	0\\
508.01	0\\
509.01	0\\
510.01	0\\
511.01	0\\
512.01	0\\
513.01	0\\
514.01	0\\
515.01	0\\
516.01	0\\
517.01	0\\
518.01	0\\
519.01	0\\
520.01	0\\
521.01	0\\
522.01	0\\
523.01	0\\
524.01	0\\
525.01	0\\
526.01	0\\
527.01	0\\
528.01	0\\
529.01	0\\
530.01	0\\
531.01	0\\
532.01	0\\
533.01	0\\
534.01	0\\
535.01	0\\
536.01	0\\
537.01	0\\
538.01	0\\
539.01	0\\
540.01	0\\
541.01	0\\
542.01	0\\
543.01	0\\
544.01	0\\
545.01	0\\
546.01	0\\
547.01	0\\
548.01	0\\
549.01	0\\
550.01	0\\
551.01	0\\
552.01	0\\
553.01	0\\
554.01	0\\
555.01	0\\
556.01	0\\
557.01	0\\
558.01	0\\
559.01	0\\
560.01	0\\
561.01	0\\
562.01	0\\
563.01	0\\
564.01	0\\
565.01	0\\
566.01	0\\
567.01	0\\
568.01	0\\
569.01	0\\
570.01	0\\
571.01	0\\
572.01	0\\
573.01	0\\
574.01	0\\
575.01	0\\
576.01	0\\
577.01	0\\
578.01	0\\
579.01	0\\
580.01	0\\
581.01	0\\
582.01	0\\
583.01	0\\
584.01	0\\
585.01	0\\
586.01	0\\
587.01	0\\
588.01	0\\
589.01	0\\
590.01	0\\
591.01	0\\
592.01	0\\
593.01	0\\
594.01	0\\
595.01	0\\
596.01	0\\
597.01	0\\
598.01	0\\
599.01	0.00375813078223648\\
599.02	0.00379585547607227\\
599.03	0.00383394714291217\\
599.04	0.00387240938738669\\
599.05	0.00391124584952775\\
599.06	0.00395046020511587\\
599.07	0.00399005616603103\\
599.08	0.00403003748060689\\
599.09	0.00407040793398825\\
599.1	0.00411117134849209\\
599.11	0.00415233158397238\\
599.12	0.00419389253818866\\
599.13	0.00423585814717847\\
599.14	0.00427823238563368\\
599.15	0.00432101926728097\\
599.16	0.00436422284526639\\
599.17	0.00440784721254407\\
599.18	0.00445189650226941\\
599.19	0.00449637488819665\\
599.2	0.00454128658508102\\
599.21	0.00458663584908382\\
599.22	0.00463242697817922\\
599.23	0.00467866431256501\\
599.24	0.00472535223507736\\
599.25	0.00477249517160974\\
599.26	0.00482009759153586\\
599.27	0.0048681640081368\\
599.28	0.00491669897903243\\
599.29	0.00496570710661689\\
599.3	0.00501519303849851\\
599.31	0.005065161467944\\
599.32	0.00511561713432699\\
599.33	0.00516656482358099\\
599.34	0.00521800936865682\\
599.35	0.00526995564998455\\
599.36	0.0053224085959399\\
599.37	0.00537537318331537\\
599.38	0.00542885443779583\\
599.39	0.00548285743046435\\
599.4	0.00553738726786889\\
599.41	0.0055924491065429\\
599.42	0.0056480481534946\\
599.43	0.00570418966670113\\
599.44	0.00576087895560742\\
599.45	0.0058181213816301\\
599.46	0.00587592235866615\\
599.47	0.00593428735360669\\
599.48	0.00599322188685565\\
599.49	0.00605273153285365\\
599.5	0.00611282192060684\\
599.51	0.00617349873422108\\
599.52	0.00623476771344123\\
599.53	0.00629663465419579\\
599.54	0.00635910540914682\\
599.55	0.00642218588824529\\
599.56	0.00648588205929183\\
599.57	0.006550199948503\\
599.58	0.00661514564108305\\
599.59	0.00668072528180133\\
599.6	0.00674694507557537\\
599.61	0.00681381128805956\\
599.62	0.00688133024623971\\
599.63	0.00694950833903336\\
599.64	0.00701835201789602\\
599.65	0.00708786779743327\\
599.66	0.00715806225601893\\
599.67	0.00722894203641922\\
599.68	0.00730051384642307\\
599.69	0.00737278445947858\\
599.7	0.00744576071533571\\
599.71	0.00751944952069528\\
599.72	0.00759385784986428\\
599.73	0.00766899274541769\\
599.74	0.0077448613188666\\
599.75	0.0078214707513331\\
599.76	0.00789882829423158\\
599.77	0.00797694126995682\\
599.78	0.00805581707257874\\
599.79	0.00813546316854402\\
599.8	0.00821588709738451\\
599.81	0.00829709647243267\\
599.82	0.00837909898154391\\
599.83	0.00846190238782606\\
599.84	0.00854551453037601\\
599.85	0.00862994332502348\\
599.86	0.00871519676508213\\
599.87	0.00880128292210799\\
599.88	0.00888820994666535\\
599.89	0.0089759860691001\\
599.9	0.0090646196003207\\
599.91	0.00915411893258671\\
599.92	0.00924449254030512\\
599.93	0.00933574898083444\\
599.94	0.00942789689529664\\
599.95	0.00952094500939709\\
599.96	0.00961490213425244\\
599.97	0.00970977716722672\\
599.98	0.00980557909277551\\
599.99	0.00990231698329844\\
600	0.01\\
};
\addplot [color=mycolor19,solid,forget plot]
  table[row sep=crcr]{%
0.01	0\\
1.01	0\\
2.01	0\\
3.01	0\\
4.01	0\\
5.01	0\\
6.01	0\\
7.01	0\\
8.01	0\\
9.01	0\\
10.01	0\\
11.01	0\\
12.01	0\\
13.01	0\\
14.01	0\\
15.01	0\\
16.01	0\\
17.01	0\\
18.01	0\\
19.01	0\\
20.01	0\\
21.01	0\\
22.01	0\\
23.01	0\\
24.01	0\\
25.01	0\\
26.01	0\\
27.01	0\\
28.01	0\\
29.01	0\\
30.01	0\\
31.01	0\\
32.01	0\\
33.01	0\\
34.01	0\\
35.01	0\\
36.01	0\\
37.01	0\\
38.01	0\\
39.01	0\\
40.01	0\\
41.01	0\\
42.01	0\\
43.01	0\\
44.01	0\\
45.01	0\\
46.01	0\\
47.01	0\\
48.01	0\\
49.01	0\\
50.01	0\\
51.01	0\\
52.01	0\\
53.01	0\\
54.01	0\\
55.01	0\\
56.01	0\\
57.01	0\\
58.01	0\\
59.01	0\\
60.01	0\\
61.01	0\\
62.01	0\\
63.01	0\\
64.01	0\\
65.01	0\\
66.01	0\\
67.01	0\\
68.01	0\\
69.01	0\\
70.01	0\\
71.01	0\\
72.01	0\\
73.01	0\\
74.01	0\\
75.01	0\\
76.01	0\\
77.01	0\\
78.01	0\\
79.01	0\\
80.01	0\\
81.01	0\\
82.01	0\\
83.01	0\\
84.01	0\\
85.01	0\\
86.01	0\\
87.01	0\\
88.01	0\\
89.01	0\\
90.01	0\\
91.01	0\\
92.01	0\\
93.01	0\\
94.01	0\\
95.01	0\\
96.01	0\\
97.01	0\\
98.01	0\\
99.01	0\\
100.01	0\\
101.01	0\\
102.01	0\\
103.01	0\\
104.01	0\\
105.01	0\\
106.01	0\\
107.01	0\\
108.01	0\\
109.01	0\\
110.01	0\\
111.01	0\\
112.01	0\\
113.01	0\\
114.01	0\\
115.01	0\\
116.01	0\\
117.01	0\\
118.01	0\\
119.01	0\\
120.01	0\\
121.01	0\\
122.01	0\\
123.01	0\\
124.01	0\\
125.01	0\\
126.01	0\\
127.01	0\\
128.01	0\\
129.01	0\\
130.01	0\\
131.01	0\\
132.01	0\\
133.01	0\\
134.01	0\\
135.01	0\\
136.01	0\\
137.01	0\\
138.01	0\\
139.01	0\\
140.01	0\\
141.01	0\\
142.01	0\\
143.01	0\\
144.01	0\\
145.01	0\\
146.01	0\\
147.01	0\\
148.01	0\\
149.01	0\\
150.01	0\\
151.01	0\\
152.01	0\\
153.01	0\\
154.01	0\\
155.01	0\\
156.01	0\\
157.01	0\\
158.01	0\\
159.01	0\\
160.01	0\\
161.01	0\\
162.01	0\\
163.01	0\\
164.01	0\\
165.01	0\\
166.01	0\\
167.01	0\\
168.01	0\\
169.01	0\\
170.01	0\\
171.01	0\\
172.01	0\\
173.01	0\\
174.01	0\\
175.01	0\\
176.01	0\\
177.01	0\\
178.01	0\\
179.01	0\\
180.01	0\\
181.01	0\\
182.01	0\\
183.01	0\\
184.01	0\\
185.01	0\\
186.01	0\\
187.01	0\\
188.01	0\\
189.01	0\\
190.01	0\\
191.01	0\\
192.01	0\\
193.01	0\\
194.01	0\\
195.01	0\\
196.01	0\\
197.01	0\\
198.01	0\\
199.01	0\\
200.01	0\\
201.01	0\\
202.01	0\\
203.01	0\\
204.01	0\\
205.01	0\\
206.01	0\\
207.01	0\\
208.01	0\\
209.01	0\\
210.01	0\\
211.01	0\\
212.01	0\\
213.01	0\\
214.01	0\\
215.01	0\\
216.01	0\\
217.01	0\\
218.01	0\\
219.01	0\\
220.01	0\\
221.01	0\\
222.01	0\\
223.01	0\\
224.01	0\\
225.01	0\\
226.01	0\\
227.01	0\\
228.01	0\\
229.01	0\\
230.01	0\\
231.01	0\\
232.01	0\\
233.01	0\\
234.01	0\\
235.01	0\\
236.01	0\\
237.01	0\\
238.01	0\\
239.01	0\\
240.01	0\\
241.01	0\\
242.01	0\\
243.01	0\\
244.01	0\\
245.01	0\\
246.01	0\\
247.01	0\\
248.01	0\\
249.01	0\\
250.01	0\\
251.01	0\\
252.01	0\\
253.01	0\\
254.01	0\\
255.01	0\\
256.01	0\\
257.01	0\\
258.01	0\\
259.01	0\\
260.01	0\\
261.01	0\\
262.01	0\\
263.01	0\\
264.01	0\\
265.01	0\\
266.01	0\\
267.01	0\\
268.01	0\\
269.01	0\\
270.01	0\\
271.01	0\\
272.01	0\\
273.01	0\\
274.01	0\\
275.01	0\\
276.01	0\\
277.01	0\\
278.01	0\\
279.01	0\\
280.01	0\\
281.01	0\\
282.01	0\\
283.01	0\\
284.01	0\\
285.01	0\\
286.01	0\\
287.01	0\\
288.01	0\\
289.01	0\\
290.01	0\\
291.01	0\\
292.01	0\\
293.01	0\\
294.01	0\\
295.01	0\\
296.01	0\\
297.01	0\\
298.01	0\\
299.01	0\\
300.01	0\\
301.01	0\\
302.01	0\\
303.01	0\\
304.01	0\\
305.01	0\\
306.01	0\\
307.01	0\\
308.01	0\\
309.01	0\\
310.01	0\\
311.01	0\\
312.01	0\\
313.01	0\\
314.01	0\\
315.01	0\\
316.01	0\\
317.01	0\\
318.01	0\\
319.01	0\\
320.01	0\\
321.01	0\\
322.01	0\\
323.01	0\\
324.01	0\\
325.01	0\\
326.01	0\\
327.01	0\\
328.01	0\\
329.01	0\\
330.01	0\\
331.01	0\\
332.01	0\\
333.01	0\\
334.01	0\\
335.01	0\\
336.01	0\\
337.01	0\\
338.01	0\\
339.01	0\\
340.01	0\\
341.01	0\\
342.01	0\\
343.01	0\\
344.01	0\\
345.01	0\\
346.01	0\\
347.01	0\\
348.01	0\\
349.01	0\\
350.01	0\\
351.01	0\\
352.01	0\\
353.01	0\\
354.01	0\\
355.01	0\\
356.01	0\\
357.01	0\\
358.01	0\\
359.01	0\\
360.01	0\\
361.01	0\\
362.01	0\\
363.01	0\\
364.01	0\\
365.01	0\\
366.01	0\\
367.01	0\\
368.01	0\\
369.01	0\\
370.01	0\\
371.01	0\\
372.01	0\\
373.01	0\\
374.01	0\\
375.01	0\\
376.01	0\\
377.01	0\\
378.01	0\\
379.01	0\\
380.01	0\\
381.01	0\\
382.01	0\\
383.01	0\\
384.01	0\\
385.01	0\\
386.01	0\\
387.01	0\\
388.01	0\\
389.01	0\\
390.01	0\\
391.01	0\\
392.01	0\\
393.01	0\\
394.01	0\\
395.01	0\\
396.01	0\\
397.01	0\\
398.01	0\\
399.01	0\\
400.01	0\\
401.01	0\\
402.01	0\\
403.01	0\\
404.01	0\\
405.01	0\\
406.01	0\\
407.01	0\\
408.01	0\\
409.01	0\\
410.01	0\\
411.01	0\\
412.01	0\\
413.01	0\\
414.01	0\\
415.01	0\\
416.01	0\\
417.01	0\\
418.01	0\\
419.01	0\\
420.01	0\\
421.01	0\\
422.01	0\\
423.01	0\\
424.01	0\\
425.01	0\\
426.01	0\\
427.01	0\\
428.01	0\\
429.01	0\\
430.01	0\\
431.01	0\\
432.01	0\\
433.01	0\\
434.01	0\\
435.01	0\\
436.01	0\\
437.01	0\\
438.01	0\\
439.01	0\\
440.01	0\\
441.01	0\\
442.01	0\\
443.01	0\\
444.01	0\\
445.01	0\\
446.01	0\\
447.01	0\\
448.01	0\\
449.01	0\\
450.01	0\\
451.01	0\\
452.01	0\\
453.01	0\\
454.01	0\\
455.01	0\\
456.01	0\\
457.01	0\\
458.01	0\\
459.01	0\\
460.01	0\\
461.01	0\\
462.01	0\\
463.01	0\\
464.01	0\\
465.01	0\\
466.01	0\\
467.01	0\\
468.01	0\\
469.01	0\\
470.01	0\\
471.01	0\\
472.01	0\\
473.01	0\\
474.01	0\\
475.01	0\\
476.01	0\\
477.01	0\\
478.01	0\\
479.01	0\\
480.01	0\\
481.01	0\\
482.01	0\\
483.01	0\\
484.01	0\\
485.01	0\\
486.01	0\\
487.01	0\\
488.01	0\\
489.01	0\\
490.01	0\\
491.01	0\\
492.01	0\\
493.01	0\\
494.01	0\\
495.01	0\\
496.01	0\\
497.01	0\\
498.01	0\\
499.01	0\\
500.01	0\\
501.01	0\\
502.01	0\\
503.01	0\\
504.01	0\\
505.01	0\\
506.01	0\\
507.01	0\\
508.01	0\\
509.01	0\\
510.01	0\\
511.01	0\\
512.01	0\\
513.01	0\\
514.01	0\\
515.01	0\\
516.01	0\\
517.01	0\\
518.01	0\\
519.01	0\\
520.01	0\\
521.01	0\\
522.01	0\\
523.01	0\\
524.01	0\\
525.01	0\\
526.01	0\\
527.01	0\\
528.01	0\\
529.01	0\\
530.01	0\\
531.01	0\\
532.01	0\\
533.01	0\\
534.01	0\\
535.01	0\\
536.01	0\\
537.01	0\\
538.01	0\\
539.01	0\\
540.01	0\\
541.01	0\\
542.01	0\\
543.01	0\\
544.01	0\\
545.01	0\\
546.01	0\\
547.01	0\\
548.01	0\\
549.01	0\\
550.01	0\\
551.01	0\\
552.01	0\\
553.01	0\\
554.01	0\\
555.01	0\\
556.01	0\\
557.01	0\\
558.01	0\\
559.01	0\\
560.01	0\\
561.01	0\\
562.01	0\\
563.01	0\\
564.01	0\\
565.01	0\\
566.01	0\\
567.01	0\\
568.01	0\\
569.01	0\\
570.01	0\\
571.01	0\\
572.01	0\\
573.01	0\\
574.01	0\\
575.01	0\\
576.01	0\\
577.01	0\\
578.01	0\\
579.01	0\\
580.01	0\\
581.01	0\\
582.01	0\\
583.01	0\\
584.01	0\\
585.01	0\\
586.01	0\\
587.01	0\\
588.01	0\\
589.01	0\\
590.01	0\\
591.01	0\\
592.01	0\\
593.01	0\\
594.01	0\\
595.01	0\\
596.01	0\\
597.01	0\\
598.01	0\\
599.01	0.00375813090363212\\
599.02	0.0037958555854239\\
599.03	0.00383394724176133\\
599.04	0.00387240947710218\\
599.05	0.00391124593131475\\
599.06	0.00395046028002561\\
599.07	0.00399005623497103\\
599.08	0.00403003754435165\\
599.09	0.00407040799319081\\
599.1	0.00411117140369623\\
599.11	0.00415233163562539\\
599.12	0.00419389258665448\\
599.13	0.00423585819275083\\
599.14	0.00427823242854921\\
599.15	0.00432101930773168\\
599.16	0.0043642228834112\\
599.17	0.00440784724851909\\
599.18	0.00445189653619617\\
599.19	0.00449637492018781\\
599.2	0.00454128661524282\\
599.21	0.0045866358775163\\
599.22	0.00463242700497642\\
599.23	0.00467866433781515\\
599.24	0.00472535225886313\\
599.25	0.00477249519400854\\
599.26	0.00482009761262005\\
599.27	0.00486816402797405\\
599.28	0.00491669899768597\\
599.29	0.00496570712414586\\
599.3	0.00501519305495828\\
599.31	0.00506516148338648\\
599.32	0.00511561714880095\\
599.33	0.00516656483713241\\
599.34	0.00521800938132919\\
599.35	0.00526995566181915\\
599.36	0.00532240860697612\\
599.37	0.00537537319359091\\
599.38	0.005428854447347\\
599.39	0.00548285743932622\\
599.4	0.0055373872760755\\
599.41	0.00559244911412737\\
599.42	0.00564804816048927\\
599.43	0.00570418967313756\\
599.44	0.0057608789615165\\
599.45	0.00581812138704197\\
599.46	0.00587592236361025\\
599.47	0.00593428735811169\\
599.48	0.00599322189094943\\
599.49	0.00605273153656322\\
599.5	0.00611282192395836\\
599.51	0.00617349873723978\\
599.52	0.0062347677161514\\
599.53	0.00629663465662077\\
599.54	0.00635910541130898\\
599.55	0.00642218589016599\\
599.56	0.00648588206099141\\
599.57	0.00655019995000079\\
599.58	0.00661514564239734\\
599.59	0.00668072528294941\\
599.6	0.00674694507657347\\
599.61	0.00681381128892291\\
599.62	0.00688133024698253\\
599.63	0.00694950833966888\\
599.64	0.00701835201843649\\
599.65	0.00708786779788998\\
599.66	0.00715806225640225\\
599.67	0.00722894203673861\\
599.68	0.00730051384668715\\
599.69	0.00737278445969511\\
599.7	0.00744576071551167\\
599.71	0.00751944952083691\\
599.72	0.00759385784997709\\
599.73	0.00766899274550653\\
599.74	0.00774486131893572\\
599.75	0.00782147075138616\\
599.76	0.00789882829427172\\
599.77	0.00797694126998669\\
599.78	0.00805581707260058\\
599.79	0.00813546316855967\\
599.8	0.00821588709739548\\
599.81	0.00829709647244017\\
599.82	0.00837909898154888\\
599.83	0.00846190238782926\\
599.84	0.00854551453037799\\
599.85	0.00862994332502465\\
599.86	0.00871519676508279\\
599.87	0.00880128292210834\\
599.88	0.00888820994666552\\
599.89	0.00897598606910018\\
599.9	0.00906461960032072\\
599.91	0.00915411893258672\\
599.92	0.00924449254030512\\
599.93	0.00933574898083444\\
599.94	0.00942789689529664\\
599.95	0.00952094500939709\\
599.96	0.00961490213425244\\
599.97	0.00970977716722672\\
599.98	0.00980557909277551\\
599.99	0.00990231698329844\\
600	0.01\\
};
\addplot [color=red!50!mycolor17,solid,forget plot]
  table[row sep=crcr]{%
0.01	0\\
1.01	0\\
2.01	0\\
3.01	0\\
4.01	0\\
5.01	0\\
6.01	0\\
7.01	0\\
8.01	0\\
9.01	0\\
10.01	0\\
11.01	0\\
12.01	0\\
13.01	0\\
14.01	0\\
15.01	0\\
16.01	0\\
17.01	0\\
18.01	0\\
19.01	0\\
20.01	0\\
21.01	0\\
22.01	0\\
23.01	0\\
24.01	0\\
25.01	0\\
26.01	0\\
27.01	0\\
28.01	0\\
29.01	0\\
30.01	0\\
31.01	0\\
32.01	0\\
33.01	0\\
34.01	0\\
35.01	0\\
36.01	0\\
37.01	0\\
38.01	0\\
39.01	0\\
40.01	0\\
41.01	0\\
42.01	0\\
43.01	0\\
44.01	0\\
45.01	0\\
46.01	0\\
47.01	0\\
48.01	0\\
49.01	0\\
50.01	0\\
51.01	0\\
52.01	0\\
53.01	0\\
54.01	0\\
55.01	0\\
56.01	0\\
57.01	0\\
58.01	0\\
59.01	0\\
60.01	0\\
61.01	0\\
62.01	0\\
63.01	0\\
64.01	0\\
65.01	0\\
66.01	0\\
67.01	0\\
68.01	0\\
69.01	0\\
70.01	0\\
71.01	0\\
72.01	0\\
73.01	0\\
74.01	0\\
75.01	0\\
76.01	0\\
77.01	0\\
78.01	0\\
79.01	0\\
80.01	0\\
81.01	0\\
82.01	0\\
83.01	0\\
84.01	0\\
85.01	0\\
86.01	0\\
87.01	0\\
88.01	0\\
89.01	0\\
90.01	0\\
91.01	0\\
92.01	0\\
93.01	0\\
94.01	0\\
95.01	0\\
96.01	0\\
97.01	0\\
98.01	0\\
99.01	0\\
100.01	0\\
101.01	0\\
102.01	0\\
103.01	0\\
104.01	0\\
105.01	0\\
106.01	0\\
107.01	0\\
108.01	0\\
109.01	0\\
110.01	0\\
111.01	0\\
112.01	0\\
113.01	0\\
114.01	0\\
115.01	0\\
116.01	0\\
117.01	0\\
118.01	0\\
119.01	0\\
120.01	0\\
121.01	0\\
122.01	0\\
123.01	0\\
124.01	0\\
125.01	0\\
126.01	0\\
127.01	0\\
128.01	0\\
129.01	0\\
130.01	0\\
131.01	0\\
132.01	0\\
133.01	0\\
134.01	0\\
135.01	0\\
136.01	0\\
137.01	0\\
138.01	0\\
139.01	0\\
140.01	0\\
141.01	0\\
142.01	0\\
143.01	0\\
144.01	0\\
145.01	0\\
146.01	0\\
147.01	0\\
148.01	0\\
149.01	0\\
150.01	0\\
151.01	0\\
152.01	0\\
153.01	0\\
154.01	0\\
155.01	0\\
156.01	0\\
157.01	0\\
158.01	0\\
159.01	0\\
160.01	0\\
161.01	0\\
162.01	0\\
163.01	0\\
164.01	0\\
165.01	0\\
166.01	0\\
167.01	0\\
168.01	0\\
169.01	0\\
170.01	0\\
171.01	0\\
172.01	0\\
173.01	0\\
174.01	0\\
175.01	0\\
176.01	0\\
177.01	0\\
178.01	0\\
179.01	0\\
180.01	0\\
181.01	0\\
182.01	0\\
183.01	0\\
184.01	0\\
185.01	0\\
186.01	0\\
187.01	0\\
188.01	0\\
189.01	0\\
190.01	0\\
191.01	0\\
192.01	0\\
193.01	0\\
194.01	0\\
195.01	0\\
196.01	0\\
197.01	0\\
198.01	0\\
199.01	0\\
200.01	0\\
201.01	0\\
202.01	0\\
203.01	0\\
204.01	0\\
205.01	0\\
206.01	0\\
207.01	0\\
208.01	0\\
209.01	0\\
210.01	0\\
211.01	0\\
212.01	0\\
213.01	0\\
214.01	0\\
215.01	0\\
216.01	0\\
217.01	0\\
218.01	0\\
219.01	0\\
220.01	0\\
221.01	0\\
222.01	0\\
223.01	0\\
224.01	0\\
225.01	0\\
226.01	0\\
227.01	0\\
228.01	0\\
229.01	0\\
230.01	0\\
231.01	0\\
232.01	0\\
233.01	0\\
234.01	0\\
235.01	0\\
236.01	0\\
237.01	0\\
238.01	0\\
239.01	0\\
240.01	0\\
241.01	0\\
242.01	0\\
243.01	0\\
244.01	0\\
245.01	0\\
246.01	0\\
247.01	0\\
248.01	0\\
249.01	0\\
250.01	0\\
251.01	0\\
252.01	0\\
253.01	0\\
254.01	0\\
255.01	0\\
256.01	0\\
257.01	0\\
258.01	0\\
259.01	0\\
260.01	0\\
261.01	0\\
262.01	0\\
263.01	0\\
264.01	0\\
265.01	0\\
266.01	0\\
267.01	0\\
268.01	0\\
269.01	0\\
270.01	0\\
271.01	0\\
272.01	0\\
273.01	0\\
274.01	0\\
275.01	0\\
276.01	0\\
277.01	0\\
278.01	0\\
279.01	0\\
280.01	0\\
281.01	0\\
282.01	0\\
283.01	0\\
284.01	0\\
285.01	0\\
286.01	0\\
287.01	0\\
288.01	0\\
289.01	0\\
290.01	0\\
291.01	0\\
292.01	0\\
293.01	0\\
294.01	0\\
295.01	0\\
296.01	0\\
297.01	0\\
298.01	0\\
299.01	0\\
300.01	0\\
301.01	0\\
302.01	0\\
303.01	0\\
304.01	0\\
305.01	0\\
306.01	0\\
307.01	0\\
308.01	0\\
309.01	0\\
310.01	0\\
311.01	0\\
312.01	0\\
313.01	0\\
314.01	0\\
315.01	0\\
316.01	0\\
317.01	0\\
318.01	0\\
319.01	0\\
320.01	0\\
321.01	0\\
322.01	0\\
323.01	0\\
324.01	0\\
325.01	0\\
326.01	0\\
327.01	0\\
328.01	0\\
329.01	0\\
330.01	0\\
331.01	0\\
332.01	0\\
333.01	0\\
334.01	0\\
335.01	0\\
336.01	0\\
337.01	0\\
338.01	0\\
339.01	0\\
340.01	0\\
341.01	0\\
342.01	0\\
343.01	0\\
344.01	0\\
345.01	0\\
346.01	0\\
347.01	0\\
348.01	0\\
349.01	0\\
350.01	0\\
351.01	0\\
352.01	0\\
353.01	0\\
354.01	0\\
355.01	0\\
356.01	0\\
357.01	0\\
358.01	0\\
359.01	0\\
360.01	0\\
361.01	0\\
362.01	0\\
363.01	0\\
364.01	0\\
365.01	0\\
366.01	0\\
367.01	0\\
368.01	0\\
369.01	0\\
370.01	0\\
371.01	0\\
372.01	0\\
373.01	0\\
374.01	0\\
375.01	0\\
376.01	0\\
377.01	0\\
378.01	0\\
379.01	0\\
380.01	0\\
381.01	0\\
382.01	0\\
383.01	0\\
384.01	0\\
385.01	0\\
386.01	0\\
387.01	0\\
388.01	0\\
389.01	0\\
390.01	0\\
391.01	0\\
392.01	0\\
393.01	0\\
394.01	0\\
395.01	0\\
396.01	0\\
397.01	0\\
398.01	0\\
399.01	0\\
400.01	0\\
401.01	0\\
402.01	0\\
403.01	0\\
404.01	0\\
405.01	0\\
406.01	0\\
407.01	0\\
408.01	0\\
409.01	0\\
410.01	0\\
411.01	0\\
412.01	0\\
413.01	0\\
414.01	0\\
415.01	0\\
416.01	0\\
417.01	0\\
418.01	0\\
419.01	0\\
420.01	0\\
421.01	0\\
422.01	0\\
423.01	0\\
424.01	0\\
425.01	0\\
426.01	0\\
427.01	0\\
428.01	0\\
429.01	0\\
430.01	0\\
431.01	0\\
432.01	0\\
433.01	0\\
434.01	0\\
435.01	0\\
436.01	0\\
437.01	0\\
438.01	0\\
439.01	0\\
440.01	0\\
441.01	0\\
442.01	0\\
443.01	0\\
444.01	0\\
445.01	0\\
446.01	0\\
447.01	0\\
448.01	0\\
449.01	0\\
450.01	0\\
451.01	0\\
452.01	0\\
453.01	0\\
454.01	0\\
455.01	0\\
456.01	0\\
457.01	0\\
458.01	0\\
459.01	0\\
460.01	0\\
461.01	0\\
462.01	0\\
463.01	0\\
464.01	0\\
465.01	0\\
466.01	0\\
467.01	0\\
468.01	0\\
469.01	0\\
470.01	0\\
471.01	0\\
472.01	0\\
473.01	0\\
474.01	0\\
475.01	0\\
476.01	0\\
477.01	0\\
478.01	0\\
479.01	0\\
480.01	0\\
481.01	0\\
482.01	0\\
483.01	0\\
484.01	0\\
485.01	0\\
486.01	0\\
487.01	0\\
488.01	0\\
489.01	0\\
490.01	0\\
491.01	0\\
492.01	0\\
493.01	0\\
494.01	0\\
495.01	0\\
496.01	0\\
497.01	0\\
498.01	0\\
499.01	0\\
500.01	0\\
501.01	0\\
502.01	0\\
503.01	0\\
504.01	0\\
505.01	0\\
506.01	0\\
507.01	0\\
508.01	0\\
509.01	0\\
510.01	0\\
511.01	0\\
512.01	0\\
513.01	0\\
514.01	0\\
515.01	0\\
516.01	0\\
517.01	0\\
518.01	0\\
519.01	0\\
520.01	0\\
521.01	0\\
522.01	0\\
523.01	0\\
524.01	0\\
525.01	0\\
526.01	0\\
527.01	0\\
528.01	0\\
529.01	0\\
530.01	0\\
531.01	0\\
532.01	0\\
533.01	0\\
534.01	0\\
535.01	0\\
536.01	0\\
537.01	0\\
538.01	0\\
539.01	0\\
540.01	0\\
541.01	0\\
542.01	0\\
543.01	0\\
544.01	0\\
545.01	0\\
546.01	0\\
547.01	0\\
548.01	0\\
549.01	0\\
550.01	0\\
551.01	0\\
552.01	0\\
553.01	0\\
554.01	0\\
555.01	0\\
556.01	0\\
557.01	0\\
558.01	0\\
559.01	0\\
560.01	0\\
561.01	0\\
562.01	0\\
563.01	0\\
564.01	0\\
565.01	0\\
566.01	0\\
567.01	0\\
568.01	0\\
569.01	0\\
570.01	0\\
571.01	0\\
572.01	0\\
573.01	0\\
574.01	0\\
575.01	0\\
576.01	0\\
577.01	0\\
578.01	0\\
579.01	0\\
580.01	0\\
581.01	0\\
582.01	0\\
583.01	0\\
584.01	0\\
585.01	0\\
586.01	0\\
587.01	0\\
588.01	0\\
589.01	0\\
590.01	0\\
591.01	0\\
592.01	0\\
593.01	0\\
594.01	0\\
595.01	0\\
596.01	0\\
597.01	0\\
598.01	0\\
599.01	0.00375813090589555\\
599.02	0.00379585558753093\\
599.03	0.00383394724372792\\
599.04	0.00387240947894213\\
599.05	0.00391124593303989\\
599.06	0.00395046028164608\\
599.07	0.0039900562364955\\
599.08	0.0040300375457875\\
599.09	0.00407040799454436\\
599.1	0.00411117140497293\\
599.11	0.00415233163682999\\
599.12	0.00419389258779109\\
599.13	0.00423585819382318\\
599.14	0.00427823242956064\\
599.15	0.00432101930868529\\
599.16	0.00436422288430989\\
599.17	0.00440784724936557\\
599.18	0.004451896536993\\
599.19	0.00449637492093741\\
599.2	0.00454128661594752\\
599.21	0.00458663587817829\\
599.22	0.00463242700559778\\
599.23	0.00467866433839786\\
599.24	0.00472535225940909\\
599.25	0.00477249519451952\\
599.26	0.00482009761309778\\
599.27	0.00486816402842016\\
599.28	0.00491669899810202\\
599.29	0.00496570712453336\\
599.3	0.00501519305531867\\
599.31	0.00506516148372115\\
599.32	0.00511561714911123\\
599.33	0.0051665648374196\\
599.34	0.00521800938159454\\
599.35	0.00526995566206385\\
599.36	0.00532240860720133\\
599.37	0.00537537319379777\\
599.38	0.0054288544475366\\
599.39	0.00548285743949962\\
599.4	0.00553738727623371\\
599.41	0.00559244911427138\\
599.42	0.00564804816062001\\
599.43	0.00570418967325595\\
599.44	0.00576087896162341\\
599.45	0.00581812138713825\\
599.46	0.00587592236369672\\
599.47	0.0059342873581891\\
599.48	0.00599322189101851\\
599.49	0.00605273153662467\\
599.5	0.00611282192401282\\
599.51	0.00617349873728788\\
599.52	0.00623476771619373\\
599.53	0.00629663465665787\\
599.54	0.00635910541134135\\
599.55	0.00642218589019411\\
599.56	0.00648588206101574\\
599.57	0.00655019995002173\\
599.58	0.00661514564241528\\
599.59	0.00668072528296469\\
599.6	0.00674694507658642\\
599.61	0.00681381128893382\\
599.62	0.00688133024699166\\
599.63	0.00694950833967647\\
599.64	0.00701835201844275\\
599.65	0.00708786779789511\\
599.66	0.00715806225640642\\
599.67	0.00722894203674198\\
599.68	0.00730051384668983\\
599.69	0.00737278445969722\\
599.7	0.00744576071551332\\
599.71	0.00751944952083818\\
599.72	0.00759385784997806\\
599.73	0.00766899274550726\\
599.74	0.00774486131893626\\
599.75	0.00782147075138655\\
599.76	0.00789882829427199\\
599.77	0.00797694126998688\\
599.78	0.00805581707260071\\
599.79	0.00813546316855976\\
599.8	0.00821588709739554\\
599.81	0.0082970964724402\\
599.82	0.0083790989815489\\
599.83	0.00846190238782927\\
599.84	0.008545514530378\\
599.85	0.00862994332502466\\
599.86	0.00871519676508279\\
599.87	0.00880128292210833\\
599.88	0.00888820994666551\\
599.89	0.00897598606910017\\
599.9	0.00906461960032072\\
599.91	0.00915411893258671\\
599.92	0.00924449254030512\\
599.93	0.00933574898083444\\
599.94	0.00942789689529664\\
599.95	0.00952094500939709\\
599.96	0.00961490213425244\\
599.97	0.00970977716722672\\
599.98	0.00980557909277551\\
599.99	0.00990231698329844\\
600	0.01\\
};
\addplot [color=red!40!mycolor19,solid,forget plot]
  table[row sep=crcr]{%
0.01	0\\
1.01	0\\
2.01	0\\
3.01	0\\
4.01	0\\
5.01	0\\
6.01	0\\
7.01	0\\
8.01	0\\
9.01	0\\
10.01	0\\
11.01	0\\
12.01	0\\
13.01	0\\
14.01	0\\
15.01	0\\
16.01	0\\
17.01	0\\
18.01	0\\
19.01	0\\
20.01	0\\
21.01	0\\
22.01	0\\
23.01	0\\
24.01	0\\
25.01	0\\
26.01	0\\
27.01	0\\
28.01	0\\
29.01	0\\
30.01	0\\
31.01	0\\
32.01	0\\
33.01	0\\
34.01	0\\
35.01	0\\
36.01	0\\
37.01	0\\
38.01	0\\
39.01	0\\
40.01	0\\
41.01	0\\
42.01	0\\
43.01	0\\
44.01	0\\
45.01	0\\
46.01	0\\
47.01	0\\
48.01	0\\
49.01	0\\
50.01	0\\
51.01	0\\
52.01	0\\
53.01	0\\
54.01	0\\
55.01	0\\
56.01	0\\
57.01	0\\
58.01	0\\
59.01	0\\
60.01	0\\
61.01	0\\
62.01	0\\
63.01	0\\
64.01	0\\
65.01	0\\
66.01	0\\
67.01	0\\
68.01	0\\
69.01	0\\
70.01	0\\
71.01	0\\
72.01	0\\
73.01	0\\
74.01	0\\
75.01	0\\
76.01	0\\
77.01	0\\
78.01	0\\
79.01	0\\
80.01	0\\
81.01	0\\
82.01	0\\
83.01	0\\
84.01	0\\
85.01	0\\
86.01	0\\
87.01	0\\
88.01	0\\
89.01	0\\
90.01	0\\
91.01	0\\
92.01	0\\
93.01	0\\
94.01	0\\
95.01	0\\
96.01	0\\
97.01	0\\
98.01	0\\
99.01	0\\
100.01	0\\
101.01	0\\
102.01	0\\
103.01	0\\
104.01	0\\
105.01	0\\
106.01	0\\
107.01	0\\
108.01	0\\
109.01	0\\
110.01	0\\
111.01	0\\
112.01	0\\
113.01	0\\
114.01	0\\
115.01	0\\
116.01	0\\
117.01	0\\
118.01	0\\
119.01	0\\
120.01	0\\
121.01	0\\
122.01	0\\
123.01	0\\
124.01	0\\
125.01	0\\
126.01	0\\
127.01	0\\
128.01	0\\
129.01	0\\
130.01	0\\
131.01	0\\
132.01	0\\
133.01	0\\
134.01	0\\
135.01	0\\
136.01	0\\
137.01	0\\
138.01	0\\
139.01	0\\
140.01	0\\
141.01	0\\
142.01	0\\
143.01	0\\
144.01	0\\
145.01	0\\
146.01	0\\
147.01	0\\
148.01	0\\
149.01	0\\
150.01	0\\
151.01	0\\
152.01	0\\
153.01	0\\
154.01	0\\
155.01	0\\
156.01	0\\
157.01	0\\
158.01	0\\
159.01	0\\
160.01	0\\
161.01	0\\
162.01	0\\
163.01	0\\
164.01	0\\
165.01	0\\
166.01	0\\
167.01	0\\
168.01	0\\
169.01	0\\
170.01	0\\
171.01	0\\
172.01	0\\
173.01	0\\
174.01	0\\
175.01	0\\
176.01	0\\
177.01	0\\
178.01	0\\
179.01	0\\
180.01	0\\
181.01	0\\
182.01	0\\
183.01	0\\
184.01	0\\
185.01	0\\
186.01	0\\
187.01	0\\
188.01	0\\
189.01	0\\
190.01	0\\
191.01	0\\
192.01	0\\
193.01	0\\
194.01	0\\
195.01	0\\
196.01	0\\
197.01	0\\
198.01	0\\
199.01	0\\
200.01	0\\
201.01	0\\
202.01	0\\
203.01	0\\
204.01	0\\
205.01	0\\
206.01	0\\
207.01	0\\
208.01	0\\
209.01	0\\
210.01	0\\
211.01	0\\
212.01	0\\
213.01	0\\
214.01	0\\
215.01	0\\
216.01	0\\
217.01	0\\
218.01	0\\
219.01	0\\
220.01	0\\
221.01	0\\
222.01	0\\
223.01	0\\
224.01	0\\
225.01	0\\
226.01	0\\
227.01	0\\
228.01	0\\
229.01	0\\
230.01	0\\
231.01	0\\
232.01	0\\
233.01	0\\
234.01	0\\
235.01	0\\
236.01	0\\
237.01	0\\
238.01	0\\
239.01	0\\
240.01	0\\
241.01	0\\
242.01	0\\
243.01	0\\
244.01	0\\
245.01	0\\
246.01	0\\
247.01	0\\
248.01	0\\
249.01	0\\
250.01	0\\
251.01	0\\
252.01	0\\
253.01	0\\
254.01	0\\
255.01	0\\
256.01	0\\
257.01	0\\
258.01	0\\
259.01	0\\
260.01	0\\
261.01	0\\
262.01	0\\
263.01	0\\
264.01	0\\
265.01	0\\
266.01	0\\
267.01	0\\
268.01	0\\
269.01	0\\
270.01	0\\
271.01	0\\
272.01	0\\
273.01	0\\
274.01	0\\
275.01	0\\
276.01	0\\
277.01	0\\
278.01	0\\
279.01	0\\
280.01	0\\
281.01	0\\
282.01	0\\
283.01	0\\
284.01	0\\
285.01	0\\
286.01	0\\
287.01	0\\
288.01	0\\
289.01	0\\
290.01	0\\
291.01	0\\
292.01	0\\
293.01	0\\
294.01	0\\
295.01	0\\
296.01	0\\
297.01	0\\
298.01	0\\
299.01	0\\
300.01	0\\
301.01	0\\
302.01	0\\
303.01	0\\
304.01	0\\
305.01	0\\
306.01	0\\
307.01	0\\
308.01	0\\
309.01	0\\
310.01	0\\
311.01	0\\
312.01	0\\
313.01	0\\
314.01	0\\
315.01	0\\
316.01	0\\
317.01	0\\
318.01	0\\
319.01	0\\
320.01	0\\
321.01	0\\
322.01	0\\
323.01	0\\
324.01	0\\
325.01	0\\
326.01	0\\
327.01	0\\
328.01	0\\
329.01	0\\
330.01	0\\
331.01	0\\
332.01	0\\
333.01	0\\
334.01	0\\
335.01	0\\
336.01	0\\
337.01	0\\
338.01	0\\
339.01	0\\
340.01	0\\
341.01	0\\
342.01	0\\
343.01	0\\
344.01	0\\
345.01	0\\
346.01	0\\
347.01	0\\
348.01	0\\
349.01	0\\
350.01	0\\
351.01	0\\
352.01	0\\
353.01	0\\
354.01	0\\
355.01	0\\
356.01	0\\
357.01	0\\
358.01	0\\
359.01	0\\
360.01	0\\
361.01	0\\
362.01	0\\
363.01	0\\
364.01	0\\
365.01	0\\
366.01	0\\
367.01	0\\
368.01	0\\
369.01	0\\
370.01	0\\
371.01	0\\
372.01	0\\
373.01	0\\
374.01	0\\
375.01	0\\
376.01	0\\
377.01	0\\
378.01	0\\
379.01	0\\
380.01	0\\
381.01	0\\
382.01	0\\
383.01	0\\
384.01	0\\
385.01	0\\
386.01	0\\
387.01	0\\
388.01	0\\
389.01	0\\
390.01	0\\
391.01	0\\
392.01	0\\
393.01	0\\
394.01	0\\
395.01	0\\
396.01	0\\
397.01	0\\
398.01	0\\
399.01	0\\
400.01	0\\
401.01	0\\
402.01	0\\
403.01	0\\
404.01	0\\
405.01	0\\
406.01	0\\
407.01	0\\
408.01	0\\
409.01	0\\
410.01	0\\
411.01	0\\
412.01	0\\
413.01	0\\
414.01	0\\
415.01	0\\
416.01	0\\
417.01	0\\
418.01	0\\
419.01	0\\
420.01	0\\
421.01	0\\
422.01	0\\
423.01	0\\
424.01	0\\
425.01	0\\
426.01	0\\
427.01	0\\
428.01	0\\
429.01	0\\
430.01	0\\
431.01	0\\
432.01	0\\
433.01	0\\
434.01	0\\
435.01	0\\
436.01	0\\
437.01	0\\
438.01	0\\
439.01	0\\
440.01	0\\
441.01	0\\
442.01	0\\
443.01	0\\
444.01	0\\
445.01	0\\
446.01	0\\
447.01	0\\
448.01	0\\
449.01	0\\
450.01	0\\
451.01	0\\
452.01	0\\
453.01	0\\
454.01	0\\
455.01	0\\
456.01	0\\
457.01	0\\
458.01	0\\
459.01	0\\
460.01	0\\
461.01	0\\
462.01	0\\
463.01	0\\
464.01	0\\
465.01	0\\
466.01	0\\
467.01	0\\
468.01	0\\
469.01	0\\
470.01	0\\
471.01	0\\
472.01	0\\
473.01	0\\
474.01	0\\
475.01	0\\
476.01	0\\
477.01	0\\
478.01	0\\
479.01	0\\
480.01	0\\
481.01	0\\
482.01	0\\
483.01	0\\
484.01	0\\
485.01	0\\
486.01	0\\
487.01	0\\
488.01	0\\
489.01	0\\
490.01	0\\
491.01	0\\
492.01	0\\
493.01	0\\
494.01	0\\
495.01	0\\
496.01	0\\
497.01	0\\
498.01	0\\
499.01	0\\
500.01	0\\
501.01	0\\
502.01	0\\
503.01	0\\
504.01	0\\
505.01	0\\
506.01	0\\
507.01	0\\
508.01	0\\
509.01	0\\
510.01	0\\
511.01	0\\
512.01	0\\
513.01	0\\
514.01	0\\
515.01	0\\
516.01	0\\
517.01	0\\
518.01	0\\
519.01	0\\
520.01	0\\
521.01	0\\
522.01	0\\
523.01	0\\
524.01	0\\
525.01	0\\
526.01	0\\
527.01	0\\
528.01	0\\
529.01	0\\
530.01	0\\
531.01	0\\
532.01	0\\
533.01	0\\
534.01	0\\
535.01	0\\
536.01	0\\
537.01	0\\
538.01	0\\
539.01	0\\
540.01	0\\
541.01	0\\
542.01	0\\
543.01	0\\
544.01	0\\
545.01	0\\
546.01	0\\
547.01	0\\
548.01	0\\
549.01	0\\
550.01	0\\
551.01	0\\
552.01	0\\
553.01	0\\
554.01	0\\
555.01	0\\
556.01	0\\
557.01	0\\
558.01	0\\
559.01	0\\
560.01	0\\
561.01	0\\
562.01	0\\
563.01	0\\
564.01	0\\
565.01	0\\
566.01	0\\
567.01	0\\
568.01	0\\
569.01	0\\
570.01	0\\
571.01	0\\
572.01	0\\
573.01	0\\
574.01	0\\
575.01	0\\
576.01	0\\
577.01	0\\
578.01	0\\
579.01	0\\
580.01	0\\
581.01	0\\
582.01	0\\
583.01	0\\
584.01	0\\
585.01	0\\
586.01	0\\
587.01	0\\
588.01	0\\
589.01	0\\
590.01	0\\
591.01	0\\
592.01	0\\
593.01	0\\
594.01	0\\
595.01	0\\
596.01	0\\
597.01	0\\
598.01	0\\
599.01	0.00375813090594476\\
599.02	0.00379585558757729\\
599.03	0.00383394724377162\\
599.04	0.00387240947898333\\
599.05	0.00391124593307875\\
599.06	0.00395046028168274\\
599.07	0.00399005623653004\\
599.08	0.00403003754582007\\
599.09	0.00407040799457505\\
599.1	0.00411117140500182\\
599.11	0.00415233163685717\\
599.12	0.00419389258781667\\
599.13	0.00423585819384723\\
599.14	0.00427823242958322\\
599.15	0.00432101930870647\\
599.16	0.00436422288432972\\
599.17	0.00440784724938413\\
599.18	0.00445189653701036\\
599.19	0.00449637492095364\\
599.2	0.00454128661596266\\
599.21	0.00458663587819241\\
599.22	0.00463242700561092\\
599.23	0.00467866433841008\\
599.24	0.00472535225942043\\
599.25	0.00477249519453003\\
599.26	0.0048200976131075\\
599.27	0.00486816402842916\\
599.28	0.00491669899811033\\
599.29	0.004965707124541\\
599.3	0.0050151930553257\\
599.31	0.0050651614837276\\
599.32	0.00511561714911714\\
599.33	0.005166564837425\\
599.34	0.00521800938159945\\
599.35	0.00526995566206832\\
599.36	0.00532240860720539\\
599.37	0.00537537319380144\\
599.38	0.00542885444753992\\
599.39	0.0054828574395026\\
599.4	0.00553738727623639\\
599.41	0.00559244911427378\\
599.42	0.00564804816062216\\
599.43	0.00570418967325787\\
599.44	0.00576087896162512\\
599.45	0.00581812138713977\\
599.46	0.00587592236369806\\
599.47	0.00593428735819028\\
599.48	0.00599322189101955\\
599.49	0.00605273153662558\\
599.5	0.00611282192401361\\
599.51	0.00617349873728856\\
599.52	0.00623476771619432\\
599.53	0.00629663465665838\\
599.54	0.00635910541134179\\
599.55	0.00642218589019449\\
599.56	0.00648588206101606\\
599.57	0.006550199950022\\
599.58	0.00661514564241551\\
599.59	0.00668072528296489\\
599.6	0.00674694507658658\\
599.61	0.00681381128893395\\
599.62	0.00688133024699177\\
599.63	0.00694950833967656\\
599.64	0.00701835201844282\\
599.65	0.00708786779789516\\
599.66	0.00715806225640646\\
599.67	0.00722894203674201\\
599.68	0.00730051384668985\\
599.69	0.00737278445969725\\
599.7	0.00744576071551335\\
599.71	0.0075194495208382\\
599.72	0.00759385784997808\\
599.73	0.00766899274550727\\
599.74	0.00774486131893627\\
599.75	0.00782147075138656\\
599.76	0.00789882829427201\\
599.77	0.0079769412699869\\
599.78	0.00805581707260072\\
599.79	0.00813546316855976\\
599.8	0.00821588709739554\\
599.81	0.00829709647244021\\
599.82	0.00837909898154891\\
599.83	0.00846190238782927\\
599.84	0.008545514530378\\
599.85	0.00862994332502466\\
599.86	0.00871519676508279\\
599.87	0.00880128292210833\\
599.88	0.00888820994666551\\
599.89	0.00897598606910018\\
599.9	0.00906461960032072\\
599.91	0.00915411893258671\\
599.92	0.00924449254030512\\
599.93	0.00933574898083444\\
599.94	0.00942789689529664\\
599.95	0.00952094500939709\\
599.96	0.00961490213425244\\
599.97	0.00970977716722672\\
599.98	0.00980557909277551\\
599.99	0.00990231698329844\\
600	0.01\\
};
\addplot [color=red!75!mycolor17,solid,forget plot]
  table[row sep=crcr]{%
0.01	0\\
1.01	0\\
2.01	0\\
3.01	0\\
4.01	0\\
5.01	0\\
6.01	0\\
7.01	0\\
8.01	0\\
9.01	0\\
10.01	0\\
11.01	0\\
12.01	0\\
13.01	0\\
14.01	0\\
15.01	0\\
16.01	0\\
17.01	0\\
18.01	0\\
19.01	0\\
20.01	0\\
21.01	0\\
22.01	0\\
23.01	0\\
24.01	0\\
25.01	0\\
26.01	0\\
27.01	0\\
28.01	0\\
29.01	0\\
30.01	0\\
31.01	0\\
32.01	0\\
33.01	0\\
34.01	0\\
35.01	0\\
36.01	0\\
37.01	0\\
38.01	0\\
39.01	0\\
40.01	0\\
41.01	0\\
42.01	0\\
43.01	0\\
44.01	0\\
45.01	0\\
46.01	0\\
47.01	0\\
48.01	0\\
49.01	0\\
50.01	0\\
51.01	0\\
52.01	0\\
53.01	0\\
54.01	0\\
55.01	0\\
56.01	0\\
57.01	0\\
58.01	0\\
59.01	0\\
60.01	0\\
61.01	0\\
62.01	0\\
63.01	0\\
64.01	0\\
65.01	0\\
66.01	0\\
67.01	0\\
68.01	0\\
69.01	0\\
70.01	0\\
71.01	0\\
72.01	0\\
73.01	0\\
74.01	0\\
75.01	0\\
76.01	0\\
77.01	0\\
78.01	0\\
79.01	0\\
80.01	0\\
81.01	0\\
82.01	0\\
83.01	0\\
84.01	0\\
85.01	0\\
86.01	0\\
87.01	0\\
88.01	0\\
89.01	0\\
90.01	0\\
91.01	0\\
92.01	0\\
93.01	0\\
94.01	0\\
95.01	0\\
96.01	0\\
97.01	0\\
98.01	0\\
99.01	0\\
100.01	0\\
101.01	0\\
102.01	0\\
103.01	0\\
104.01	0\\
105.01	0\\
106.01	0\\
107.01	0\\
108.01	0\\
109.01	0\\
110.01	0\\
111.01	0\\
112.01	0\\
113.01	0\\
114.01	0\\
115.01	0\\
116.01	0\\
117.01	0\\
118.01	0\\
119.01	0\\
120.01	0\\
121.01	0\\
122.01	0\\
123.01	0\\
124.01	0\\
125.01	0\\
126.01	0\\
127.01	0\\
128.01	0\\
129.01	0\\
130.01	0\\
131.01	0\\
132.01	0\\
133.01	0\\
134.01	0\\
135.01	0\\
136.01	0\\
137.01	0\\
138.01	0\\
139.01	0\\
140.01	0\\
141.01	0\\
142.01	0\\
143.01	0\\
144.01	0\\
145.01	0\\
146.01	0\\
147.01	0\\
148.01	0\\
149.01	0\\
150.01	0\\
151.01	0\\
152.01	0\\
153.01	0\\
154.01	0\\
155.01	0\\
156.01	0\\
157.01	0\\
158.01	0\\
159.01	0\\
160.01	0\\
161.01	0\\
162.01	0\\
163.01	0\\
164.01	0\\
165.01	0\\
166.01	0\\
167.01	0\\
168.01	0\\
169.01	0\\
170.01	0\\
171.01	0\\
172.01	0\\
173.01	0\\
174.01	0\\
175.01	0\\
176.01	0\\
177.01	0\\
178.01	0\\
179.01	0\\
180.01	0\\
181.01	0\\
182.01	0\\
183.01	0\\
184.01	0\\
185.01	0\\
186.01	0\\
187.01	0\\
188.01	0\\
189.01	0\\
190.01	0\\
191.01	0\\
192.01	0\\
193.01	0\\
194.01	0\\
195.01	0\\
196.01	0\\
197.01	0\\
198.01	0\\
199.01	0\\
200.01	0\\
201.01	0\\
202.01	0\\
203.01	0\\
204.01	0\\
205.01	0\\
206.01	0\\
207.01	0\\
208.01	0\\
209.01	0\\
210.01	0\\
211.01	0\\
212.01	0\\
213.01	0\\
214.01	0\\
215.01	0\\
216.01	0\\
217.01	0\\
218.01	0\\
219.01	0\\
220.01	0\\
221.01	0\\
222.01	0\\
223.01	0\\
224.01	0\\
225.01	0\\
226.01	0\\
227.01	0\\
228.01	0\\
229.01	0\\
230.01	0\\
231.01	0\\
232.01	0\\
233.01	0\\
234.01	0\\
235.01	0\\
236.01	0\\
237.01	0\\
238.01	0\\
239.01	0\\
240.01	0\\
241.01	0\\
242.01	0\\
243.01	0\\
244.01	0\\
245.01	0\\
246.01	0\\
247.01	0\\
248.01	0\\
249.01	0\\
250.01	0\\
251.01	0\\
252.01	0\\
253.01	0\\
254.01	0\\
255.01	0\\
256.01	0\\
257.01	0\\
258.01	0\\
259.01	0\\
260.01	0\\
261.01	0\\
262.01	0\\
263.01	0\\
264.01	0\\
265.01	0\\
266.01	0\\
267.01	0\\
268.01	0\\
269.01	0\\
270.01	0\\
271.01	0\\
272.01	0\\
273.01	0\\
274.01	0\\
275.01	0\\
276.01	0\\
277.01	0\\
278.01	0\\
279.01	0\\
280.01	0\\
281.01	0\\
282.01	0\\
283.01	0\\
284.01	0\\
285.01	0\\
286.01	0\\
287.01	0\\
288.01	0\\
289.01	0\\
290.01	0\\
291.01	0\\
292.01	0\\
293.01	0\\
294.01	0\\
295.01	0\\
296.01	0\\
297.01	0\\
298.01	0\\
299.01	0\\
300.01	0\\
301.01	0\\
302.01	0\\
303.01	0\\
304.01	0\\
305.01	0\\
306.01	0\\
307.01	0\\
308.01	0\\
309.01	0\\
310.01	0\\
311.01	0\\
312.01	0\\
313.01	0\\
314.01	0\\
315.01	0\\
316.01	0\\
317.01	0\\
318.01	0\\
319.01	0\\
320.01	0\\
321.01	0\\
322.01	0\\
323.01	0\\
324.01	0\\
325.01	0\\
326.01	0\\
327.01	0\\
328.01	0\\
329.01	0\\
330.01	0\\
331.01	0\\
332.01	0\\
333.01	0\\
334.01	0\\
335.01	0\\
336.01	0\\
337.01	0\\
338.01	0\\
339.01	0\\
340.01	0\\
341.01	0\\
342.01	0\\
343.01	0\\
344.01	0\\
345.01	0\\
346.01	0\\
347.01	0\\
348.01	0\\
349.01	0\\
350.01	0\\
351.01	0\\
352.01	0\\
353.01	0\\
354.01	0\\
355.01	0\\
356.01	0\\
357.01	0\\
358.01	0\\
359.01	0\\
360.01	0\\
361.01	0\\
362.01	0\\
363.01	0\\
364.01	0\\
365.01	0\\
366.01	0\\
367.01	0\\
368.01	0\\
369.01	0\\
370.01	0\\
371.01	0\\
372.01	0\\
373.01	0\\
374.01	0\\
375.01	0\\
376.01	0\\
377.01	0\\
378.01	0\\
379.01	0\\
380.01	0\\
381.01	0\\
382.01	0\\
383.01	0\\
384.01	0\\
385.01	0\\
386.01	0\\
387.01	0\\
388.01	0\\
389.01	0\\
390.01	0\\
391.01	0\\
392.01	0\\
393.01	0\\
394.01	0\\
395.01	0\\
396.01	0\\
397.01	0\\
398.01	0\\
399.01	0\\
400.01	0\\
401.01	0\\
402.01	0\\
403.01	0\\
404.01	0\\
405.01	0\\
406.01	0\\
407.01	0\\
408.01	0\\
409.01	0\\
410.01	0\\
411.01	0\\
412.01	0\\
413.01	0\\
414.01	0\\
415.01	0\\
416.01	0\\
417.01	0\\
418.01	0\\
419.01	0\\
420.01	0\\
421.01	0\\
422.01	0\\
423.01	0\\
424.01	0\\
425.01	0\\
426.01	0\\
427.01	0\\
428.01	0\\
429.01	0\\
430.01	0\\
431.01	0\\
432.01	0\\
433.01	0\\
434.01	0\\
435.01	0\\
436.01	0\\
437.01	0\\
438.01	0\\
439.01	0\\
440.01	0\\
441.01	0\\
442.01	0\\
443.01	0\\
444.01	0\\
445.01	0\\
446.01	0\\
447.01	0\\
448.01	0\\
449.01	0\\
450.01	0\\
451.01	0\\
452.01	0\\
453.01	0\\
454.01	0\\
455.01	0\\
456.01	0\\
457.01	0\\
458.01	0\\
459.01	0\\
460.01	0\\
461.01	0\\
462.01	0\\
463.01	0\\
464.01	0\\
465.01	0\\
466.01	0\\
467.01	0\\
468.01	0\\
469.01	0\\
470.01	0\\
471.01	0\\
472.01	0\\
473.01	0\\
474.01	0\\
475.01	0\\
476.01	0\\
477.01	0\\
478.01	0\\
479.01	0\\
480.01	0\\
481.01	0\\
482.01	0\\
483.01	0\\
484.01	0\\
485.01	0\\
486.01	0\\
487.01	0\\
488.01	0\\
489.01	0\\
490.01	0\\
491.01	0\\
492.01	0\\
493.01	0\\
494.01	0\\
495.01	0\\
496.01	0\\
497.01	0\\
498.01	0\\
499.01	0\\
500.01	0\\
501.01	0\\
502.01	0\\
503.01	0\\
504.01	0\\
505.01	0\\
506.01	0\\
507.01	0\\
508.01	0\\
509.01	0\\
510.01	0\\
511.01	0\\
512.01	0\\
513.01	0\\
514.01	0\\
515.01	0\\
516.01	0\\
517.01	0\\
518.01	0\\
519.01	0\\
520.01	0\\
521.01	0\\
522.01	0\\
523.01	0\\
524.01	0\\
525.01	0\\
526.01	0\\
527.01	0\\
528.01	0\\
529.01	0\\
530.01	0\\
531.01	0\\
532.01	0\\
533.01	0\\
534.01	0\\
535.01	0\\
536.01	0\\
537.01	0\\
538.01	0\\
539.01	0\\
540.01	0\\
541.01	0\\
542.01	0\\
543.01	0\\
544.01	0\\
545.01	0\\
546.01	0\\
547.01	0\\
548.01	0\\
549.01	0\\
550.01	0\\
551.01	0\\
552.01	0\\
553.01	0\\
554.01	0\\
555.01	0\\
556.01	0\\
557.01	0\\
558.01	0\\
559.01	0\\
560.01	0\\
561.01	0\\
562.01	0\\
563.01	0\\
564.01	0\\
565.01	0\\
566.01	0\\
567.01	0\\
568.01	0\\
569.01	0\\
570.01	0\\
571.01	0\\
572.01	0\\
573.01	0\\
574.01	0\\
575.01	0\\
576.01	0\\
577.01	0\\
578.01	0\\
579.01	0\\
580.01	0\\
581.01	0\\
582.01	0\\
583.01	0\\
584.01	0\\
585.01	0\\
586.01	0\\
587.01	0\\
588.01	0\\
589.01	0\\
590.01	0\\
591.01	0\\
592.01	0\\
593.01	0\\
594.01	0\\
595.01	0\\
596.01	0\\
597.01	0\\
598.01	0\\
599.01	0.0037581309059459\\
599.02	0.00379585558757836\\
599.03	0.00383394724377263\\
599.04	0.00387240947898429\\
599.05	0.00391124593307965\\
599.06	0.00395046028168357\\
599.07	0.00399005623653083\\
599.08	0.00403003754582079\\
599.09	0.00407040799457573\\
599.1	0.00411117140500247\\
599.11	0.00415233163685778\\
599.12	0.00419389258781724\\
599.13	0.00423585819384774\\
599.14	0.00427823242958369\\
599.15	0.0043210193087069\\
599.16	0.00436422288433014\\
599.17	0.00440784724938452\\
599.18	0.00445189653701071\\
599.19	0.00449637492095395\\
599.2	0.00454128661596294\\
599.21	0.00458663587819265\\
599.22	0.00463242700561113\\
599.23	0.00467866433841027\\
599.24	0.0047253522594206\\
599.25	0.00477249519453019\\
599.26	0.00482009761310764\\
599.27	0.00486816402842927\\
599.28	0.00491669899811042\\
599.29	0.0049657071245411\\
599.3	0.00501519305532579\\
599.31	0.00506516148372766\\
599.32	0.00511561714911721\\
599.33	0.00516656483742505\\
599.34	0.00521800938159949\\
599.35	0.00526995566206836\\
599.36	0.00532240860720542\\
599.37	0.00537537319380147\\
599.38	0.00542885444753993\\
599.39	0.00548285743950262\\
599.4	0.00553738727623639\\
599.41	0.00559244911427378\\
599.42	0.00564804816062215\\
599.43	0.00570418967325786\\
599.44	0.00576087896162511\\
599.45	0.00581812138713976\\
599.46	0.00587592236369804\\
599.47	0.00593428735819027\\
599.48	0.00599322189101953\\
599.49	0.00605273153662556\\
599.5	0.00611282192401359\\
599.51	0.00617349873728854\\
599.52	0.0062347677161943\\
599.53	0.00629663465665836\\
599.54	0.00635910541134178\\
599.55	0.00642218589019448\\
599.56	0.00648588206101605\\
599.57	0.00655019995002199\\
599.58	0.0066151456424155\\
599.59	0.00668072528296487\\
599.6	0.00674694507658657\\
599.61	0.00681381128893395\\
599.62	0.00688133024699177\\
599.63	0.00694950833967656\\
599.64	0.00701835201844282\\
599.65	0.00708786779789517\\
599.66	0.00715806225640647\\
599.67	0.00722894203674201\\
599.68	0.00730051384668985\\
599.69	0.00737278445969725\\
599.7	0.00744576071551335\\
599.71	0.0075194495208382\\
599.72	0.00759385784997808\\
599.73	0.00766899274550728\\
599.74	0.00774486131893627\\
599.75	0.00782147075138656\\
599.76	0.00789882829427201\\
599.77	0.0079769412699869\\
599.78	0.00805581707260072\\
599.79	0.00813546316855976\\
599.8	0.00821588709739554\\
599.81	0.00829709647244021\\
599.82	0.00837909898154891\\
599.83	0.00846190238782927\\
599.84	0.008545514530378\\
599.85	0.00862994332502466\\
599.86	0.00871519676508279\\
599.87	0.00880128292210834\\
599.88	0.00888820994666552\\
599.89	0.00897598606910018\\
599.9	0.00906461960032073\\
599.91	0.00915411893258672\\
599.92	0.00924449254030512\\
599.93	0.00933574898083444\\
599.94	0.00942789689529664\\
599.95	0.00952094500939709\\
599.96	0.00961490213425244\\
599.97	0.00970977716722672\\
599.98	0.00980557909277551\\
599.99	0.00990231698329844\\
600	0.01\\
};
\addplot [color=red!80!mycolor19,solid,forget plot]
  table[row sep=crcr]{%
0.01	0\\
1.01	0\\
2.01	0\\
3.01	0\\
4.01	0\\
5.01	0\\
6.01	0\\
7.01	0\\
8.01	0\\
9.01	0\\
10.01	0\\
11.01	0\\
12.01	0\\
13.01	0\\
14.01	0\\
15.01	0\\
16.01	0\\
17.01	0\\
18.01	0\\
19.01	0\\
20.01	0\\
21.01	0\\
22.01	0\\
23.01	0\\
24.01	0\\
25.01	0\\
26.01	0\\
27.01	0\\
28.01	0\\
29.01	0\\
30.01	0\\
31.01	0\\
32.01	0\\
33.01	0\\
34.01	0\\
35.01	0\\
36.01	0\\
37.01	0\\
38.01	0\\
39.01	0\\
40.01	0\\
41.01	0\\
42.01	0\\
43.01	0\\
44.01	0\\
45.01	0\\
46.01	0\\
47.01	0\\
48.01	0\\
49.01	0\\
50.01	0\\
51.01	0\\
52.01	0\\
53.01	0\\
54.01	0\\
55.01	0\\
56.01	0\\
57.01	0\\
58.01	0\\
59.01	0\\
60.01	0\\
61.01	0\\
62.01	0\\
63.01	0\\
64.01	0\\
65.01	0\\
66.01	0\\
67.01	0\\
68.01	0\\
69.01	0\\
70.01	0\\
71.01	0\\
72.01	0\\
73.01	0\\
74.01	0\\
75.01	0\\
76.01	0\\
77.01	0\\
78.01	0\\
79.01	0\\
80.01	0\\
81.01	0\\
82.01	0\\
83.01	0\\
84.01	0\\
85.01	0\\
86.01	0\\
87.01	0\\
88.01	0\\
89.01	0\\
90.01	0\\
91.01	0\\
92.01	0\\
93.01	0\\
94.01	0\\
95.01	0\\
96.01	0\\
97.01	0\\
98.01	0\\
99.01	0\\
100.01	0\\
101.01	0\\
102.01	0\\
103.01	0\\
104.01	0\\
105.01	0\\
106.01	0\\
107.01	0\\
108.01	0\\
109.01	0\\
110.01	0\\
111.01	0\\
112.01	0\\
113.01	0\\
114.01	0\\
115.01	0\\
116.01	0\\
117.01	0\\
118.01	0\\
119.01	0\\
120.01	0\\
121.01	0\\
122.01	0\\
123.01	0\\
124.01	0\\
125.01	0\\
126.01	0\\
127.01	0\\
128.01	0\\
129.01	0\\
130.01	0\\
131.01	0\\
132.01	0\\
133.01	0\\
134.01	0\\
135.01	0\\
136.01	0\\
137.01	0\\
138.01	0\\
139.01	0\\
140.01	0\\
141.01	0\\
142.01	0\\
143.01	0\\
144.01	0\\
145.01	0\\
146.01	0\\
147.01	0\\
148.01	0\\
149.01	0\\
150.01	0\\
151.01	0\\
152.01	0\\
153.01	0\\
154.01	0\\
155.01	0\\
156.01	0\\
157.01	0\\
158.01	0\\
159.01	0\\
160.01	0\\
161.01	0\\
162.01	0\\
163.01	0\\
164.01	0\\
165.01	0\\
166.01	0\\
167.01	0\\
168.01	0\\
169.01	0\\
170.01	0\\
171.01	0\\
172.01	0\\
173.01	0\\
174.01	0\\
175.01	0\\
176.01	0\\
177.01	0\\
178.01	0\\
179.01	0\\
180.01	0\\
181.01	0\\
182.01	0\\
183.01	0\\
184.01	0\\
185.01	0\\
186.01	0\\
187.01	0\\
188.01	0\\
189.01	0\\
190.01	0\\
191.01	0\\
192.01	0\\
193.01	0\\
194.01	0\\
195.01	0\\
196.01	0\\
197.01	0\\
198.01	0\\
199.01	0\\
200.01	0\\
201.01	0\\
202.01	0\\
203.01	0\\
204.01	0\\
205.01	0\\
206.01	0\\
207.01	0\\
208.01	0\\
209.01	0\\
210.01	0\\
211.01	0\\
212.01	0\\
213.01	0\\
214.01	0\\
215.01	0\\
216.01	0\\
217.01	0\\
218.01	0\\
219.01	0\\
220.01	0\\
221.01	0\\
222.01	0\\
223.01	0\\
224.01	0\\
225.01	0\\
226.01	0\\
227.01	0\\
228.01	0\\
229.01	0\\
230.01	0\\
231.01	0\\
232.01	0\\
233.01	0\\
234.01	0\\
235.01	0\\
236.01	0\\
237.01	0\\
238.01	0\\
239.01	0\\
240.01	0\\
241.01	0\\
242.01	0\\
243.01	0\\
244.01	0\\
245.01	0\\
246.01	0\\
247.01	0\\
248.01	0\\
249.01	0\\
250.01	0\\
251.01	0\\
252.01	0\\
253.01	0\\
254.01	0\\
255.01	0\\
256.01	0\\
257.01	0\\
258.01	0\\
259.01	0\\
260.01	0\\
261.01	0\\
262.01	0\\
263.01	0\\
264.01	0\\
265.01	0\\
266.01	0\\
267.01	0\\
268.01	0\\
269.01	0\\
270.01	0\\
271.01	0\\
272.01	0\\
273.01	0\\
274.01	0\\
275.01	0\\
276.01	0\\
277.01	0\\
278.01	0\\
279.01	0\\
280.01	0\\
281.01	0\\
282.01	0\\
283.01	0\\
284.01	0\\
285.01	0\\
286.01	0\\
287.01	0\\
288.01	0\\
289.01	0\\
290.01	0\\
291.01	0\\
292.01	0\\
293.01	0\\
294.01	0\\
295.01	0\\
296.01	0\\
297.01	0\\
298.01	0\\
299.01	0\\
300.01	0\\
301.01	0\\
302.01	0\\
303.01	0\\
304.01	0\\
305.01	0\\
306.01	0\\
307.01	0\\
308.01	0\\
309.01	0\\
310.01	0\\
311.01	0\\
312.01	0\\
313.01	0\\
314.01	0\\
315.01	0\\
316.01	0\\
317.01	0\\
318.01	0\\
319.01	0\\
320.01	0\\
321.01	0\\
322.01	0\\
323.01	0\\
324.01	0\\
325.01	0\\
326.01	0\\
327.01	0\\
328.01	0\\
329.01	0\\
330.01	0\\
331.01	0\\
332.01	0\\
333.01	0\\
334.01	0\\
335.01	0\\
336.01	0\\
337.01	0\\
338.01	0\\
339.01	0\\
340.01	0\\
341.01	0\\
342.01	0\\
343.01	0\\
344.01	0\\
345.01	0\\
346.01	0\\
347.01	0\\
348.01	0\\
349.01	0\\
350.01	0\\
351.01	0\\
352.01	0\\
353.01	0\\
354.01	0\\
355.01	0\\
356.01	0\\
357.01	0\\
358.01	0\\
359.01	0\\
360.01	0\\
361.01	0\\
362.01	0\\
363.01	0\\
364.01	0\\
365.01	0\\
366.01	0\\
367.01	0\\
368.01	0\\
369.01	0\\
370.01	0\\
371.01	0\\
372.01	0\\
373.01	0\\
374.01	0\\
375.01	0\\
376.01	0\\
377.01	0\\
378.01	0\\
379.01	0\\
380.01	0\\
381.01	0\\
382.01	0\\
383.01	0\\
384.01	0\\
385.01	0\\
386.01	0\\
387.01	0\\
388.01	0\\
389.01	0\\
390.01	0\\
391.01	0\\
392.01	0\\
393.01	0\\
394.01	0\\
395.01	0\\
396.01	0\\
397.01	0\\
398.01	0\\
399.01	0\\
400.01	0\\
401.01	0\\
402.01	0\\
403.01	0\\
404.01	0\\
405.01	0\\
406.01	0\\
407.01	0\\
408.01	0\\
409.01	0\\
410.01	0\\
411.01	0\\
412.01	0\\
413.01	0\\
414.01	0\\
415.01	0\\
416.01	0\\
417.01	0\\
418.01	0\\
419.01	0\\
420.01	0\\
421.01	0\\
422.01	0\\
423.01	0\\
424.01	0\\
425.01	0\\
426.01	0\\
427.01	0\\
428.01	0\\
429.01	0\\
430.01	0\\
431.01	0\\
432.01	0\\
433.01	0\\
434.01	0\\
435.01	0\\
436.01	0\\
437.01	0\\
438.01	0\\
439.01	0\\
440.01	0\\
441.01	0\\
442.01	0\\
443.01	0\\
444.01	0\\
445.01	0\\
446.01	0\\
447.01	0\\
448.01	0\\
449.01	0\\
450.01	0\\
451.01	0\\
452.01	0\\
453.01	0\\
454.01	0\\
455.01	0\\
456.01	0\\
457.01	0\\
458.01	0\\
459.01	0\\
460.01	0\\
461.01	0\\
462.01	0\\
463.01	0\\
464.01	0\\
465.01	0\\
466.01	0\\
467.01	0\\
468.01	0\\
469.01	0\\
470.01	0\\
471.01	0\\
472.01	0\\
473.01	0\\
474.01	0\\
475.01	0\\
476.01	0\\
477.01	0\\
478.01	0\\
479.01	0\\
480.01	0\\
481.01	0\\
482.01	0\\
483.01	0\\
484.01	0\\
485.01	0\\
486.01	0\\
487.01	0\\
488.01	0\\
489.01	0\\
490.01	0\\
491.01	0\\
492.01	0\\
493.01	0\\
494.01	0\\
495.01	0\\
496.01	0\\
497.01	0\\
498.01	0\\
499.01	0\\
500.01	0\\
501.01	0\\
502.01	0\\
503.01	0\\
504.01	0\\
505.01	0\\
506.01	0\\
507.01	0\\
508.01	0\\
509.01	0\\
510.01	0\\
511.01	0\\
512.01	0\\
513.01	0\\
514.01	0\\
515.01	0\\
516.01	0\\
517.01	0\\
518.01	0\\
519.01	0\\
520.01	0\\
521.01	0\\
522.01	0\\
523.01	0\\
524.01	0\\
525.01	0\\
526.01	0\\
527.01	0\\
528.01	0\\
529.01	0\\
530.01	0\\
531.01	0\\
532.01	0\\
533.01	0\\
534.01	0\\
535.01	0\\
536.01	0\\
537.01	0\\
538.01	0\\
539.01	0\\
540.01	0\\
541.01	0\\
542.01	0\\
543.01	0\\
544.01	0\\
545.01	0\\
546.01	0\\
547.01	0\\
548.01	0\\
549.01	0\\
550.01	0\\
551.01	0\\
552.01	0\\
553.01	0\\
554.01	0\\
555.01	0\\
556.01	0\\
557.01	0\\
558.01	0\\
559.01	0\\
560.01	0\\
561.01	0\\
562.01	0\\
563.01	0\\
564.01	0\\
565.01	0\\
566.01	0\\
567.01	0\\
568.01	0\\
569.01	0\\
570.01	0\\
571.01	0\\
572.01	0\\
573.01	0\\
574.01	0\\
575.01	0\\
576.01	0\\
577.01	0\\
578.01	0\\
579.01	0\\
580.01	0\\
581.01	0\\
582.01	0\\
583.01	0\\
584.01	0\\
585.01	0\\
586.01	0\\
587.01	0\\
588.01	0\\
589.01	0\\
590.01	0\\
591.01	0\\
592.01	0\\
593.01	0\\
594.01	0\\
595.01	0\\
596.01	0\\
597.01	0\\
598.01	0\\
599.01	0.0037581309059459\\
599.02	0.00379585558757838\\
599.03	0.00383394724377265\\
599.04	0.0038724094789843\\
599.05	0.00391124593307965\\
599.06	0.00395046028168358\\
599.07	0.00399005623653084\\
599.08	0.00403003754582082\\
599.09	0.00407040799457575\\
599.1	0.00411117140500249\\
599.11	0.0041523316368578\\
599.12	0.00419389258781726\\
599.13	0.00423585819384777\\
599.14	0.00427823242958374\\
599.15	0.00432101930870694\\
599.16	0.00436422288433017\\
599.17	0.00440784724938455\\
599.18	0.00445189653701075\\
599.19	0.004496374920954\\
599.2	0.00454128661596299\\
599.21	0.00458663587819272\\
599.22	0.00463242700561121\\
599.23	0.00467866433841035\\
599.24	0.00472535225942068\\
599.25	0.00477249519453027\\
599.26	0.00482009761310773\\
599.27	0.00486816402842935\\
599.28	0.00491669899811049\\
599.29	0.00496570712454115\\
599.3	0.00501519305532583\\
599.31	0.00506516148372772\\
599.32	0.00511561714911725\\
599.33	0.0051665648374251\\
599.34	0.00521800938159954\\
599.35	0.00526995566206839\\
599.36	0.00532240860720545\\
599.37	0.00537537319380151\\
599.38	0.00542885444753997\\
599.39	0.00548285743950264\\
599.4	0.00553738727623643\\
599.41	0.00559244911427381\\
599.42	0.00564804816062218\\
599.43	0.00570418967325789\\
599.44	0.00576087896162513\\
599.45	0.00581812138713978\\
599.46	0.00587592236369807\\
599.47	0.00593428735819029\\
599.48	0.00599322189101956\\
599.49	0.00605273153662558\\
599.5	0.00611282192401362\\
599.51	0.00617349873728857\\
599.52	0.00623476771619433\\
599.53	0.00629663465665839\\
599.54	0.0063591054113418\\
599.55	0.0064221858901945\\
599.56	0.00648588206101607\\
599.57	0.00655019995002201\\
599.58	0.00661514564241552\\
599.59	0.0066807252829649\\
599.6	0.0067469450765866\\
599.61	0.00681381128893396\\
599.62	0.00688133024699178\\
599.63	0.00694950833967657\\
599.64	0.00701835201844284\\
599.65	0.00708786779789518\\
599.66	0.00715806225640647\\
599.67	0.00722894203674202\\
599.68	0.00730051384668987\\
599.69	0.00737278445969725\\
599.7	0.00744576071551335\\
599.71	0.0075194495208382\\
599.72	0.00759385784997808\\
599.73	0.00766899274550727\\
599.74	0.00774486131893627\\
599.75	0.00782147075138656\\
599.76	0.007898828294272\\
599.77	0.00797694126998689\\
599.78	0.00805581707260072\\
599.79	0.00813546316855976\\
599.8	0.00821588709739554\\
599.81	0.00829709647244021\\
599.82	0.00837909898154891\\
599.83	0.00846190238782927\\
599.84	0.008545514530378\\
599.85	0.00862994332502466\\
599.86	0.00871519676508279\\
599.87	0.00880128292210834\\
599.88	0.00888820994666552\\
599.89	0.00897598606910018\\
599.9	0.00906461960032073\\
599.91	0.00915411893258671\\
599.92	0.00924449254030512\\
599.93	0.00933574898083444\\
599.94	0.00942789689529664\\
599.95	0.00952094500939709\\
599.96	0.00961490213425245\\
599.97	0.00970977716722672\\
599.98	0.00980557909277551\\
599.99	0.00990231698329844\\
600	0.01\\
};
\addplot [color=red,solid,forget plot]
  table[row sep=crcr]{%
0.01	0\\
1.01	0\\
2.01	0\\
3.01	0\\
4.01	0\\
5.01	0\\
6.01	0\\
7.01	0\\
8.01	0\\
9.01	0\\
10.01	0\\
11.01	0\\
12.01	0\\
13.01	0\\
14.01	0\\
15.01	0\\
16.01	0\\
17.01	0\\
18.01	0\\
19.01	0\\
20.01	0\\
21.01	0\\
22.01	0\\
23.01	0\\
24.01	0\\
25.01	0\\
26.01	0\\
27.01	0\\
28.01	0\\
29.01	0\\
30.01	0\\
31.01	0\\
32.01	0\\
33.01	0\\
34.01	0\\
35.01	0\\
36.01	0\\
37.01	0\\
38.01	0\\
39.01	0\\
40.01	0\\
41.01	0\\
42.01	0\\
43.01	0\\
44.01	0\\
45.01	0\\
46.01	0\\
47.01	0\\
48.01	0\\
49.01	0\\
50.01	0\\
51.01	0\\
52.01	0\\
53.01	0\\
54.01	0\\
55.01	0\\
56.01	0\\
57.01	0\\
58.01	0\\
59.01	0\\
60.01	0\\
61.01	0\\
62.01	0\\
63.01	0\\
64.01	0\\
65.01	0\\
66.01	0\\
67.01	0\\
68.01	0\\
69.01	0\\
70.01	0\\
71.01	0\\
72.01	0\\
73.01	0\\
74.01	0\\
75.01	0\\
76.01	0\\
77.01	0\\
78.01	0\\
79.01	0\\
80.01	0\\
81.01	0\\
82.01	0\\
83.01	0\\
84.01	0\\
85.01	0\\
86.01	0\\
87.01	0\\
88.01	0\\
89.01	0\\
90.01	0\\
91.01	0\\
92.01	0\\
93.01	0\\
94.01	0\\
95.01	0\\
96.01	0\\
97.01	0\\
98.01	0\\
99.01	0\\
100.01	0\\
101.01	0\\
102.01	0\\
103.01	0\\
104.01	0\\
105.01	0\\
106.01	0\\
107.01	0\\
108.01	0\\
109.01	0\\
110.01	0\\
111.01	0\\
112.01	0\\
113.01	0\\
114.01	0\\
115.01	0\\
116.01	0\\
117.01	0\\
118.01	0\\
119.01	0\\
120.01	0\\
121.01	0\\
122.01	0\\
123.01	0\\
124.01	0\\
125.01	0\\
126.01	0\\
127.01	0\\
128.01	0\\
129.01	0\\
130.01	0\\
131.01	0\\
132.01	0\\
133.01	0\\
134.01	0\\
135.01	0\\
136.01	0\\
137.01	0\\
138.01	0\\
139.01	0\\
140.01	0\\
141.01	0\\
142.01	0\\
143.01	0\\
144.01	0\\
145.01	0\\
146.01	0\\
147.01	0\\
148.01	0\\
149.01	0\\
150.01	0\\
151.01	0\\
152.01	0\\
153.01	0\\
154.01	0\\
155.01	0\\
156.01	0\\
157.01	0\\
158.01	0\\
159.01	0\\
160.01	0\\
161.01	0\\
162.01	0\\
163.01	0\\
164.01	0\\
165.01	0\\
166.01	0\\
167.01	0\\
168.01	0\\
169.01	0\\
170.01	0\\
171.01	0\\
172.01	0\\
173.01	0\\
174.01	0\\
175.01	0\\
176.01	0\\
177.01	0\\
178.01	0\\
179.01	0\\
180.01	0\\
181.01	0\\
182.01	0\\
183.01	0\\
184.01	0\\
185.01	0\\
186.01	0\\
187.01	0\\
188.01	0\\
189.01	0\\
190.01	0\\
191.01	0\\
192.01	0\\
193.01	0\\
194.01	0\\
195.01	0\\
196.01	0\\
197.01	0\\
198.01	0\\
199.01	0\\
200.01	0\\
201.01	0\\
202.01	0\\
203.01	0\\
204.01	0\\
205.01	0\\
206.01	0\\
207.01	0\\
208.01	0\\
209.01	0\\
210.01	0\\
211.01	0\\
212.01	0\\
213.01	0\\
214.01	0\\
215.01	0\\
216.01	0\\
217.01	0\\
218.01	0\\
219.01	0\\
220.01	0\\
221.01	0\\
222.01	0\\
223.01	0\\
224.01	0\\
225.01	0\\
226.01	0\\
227.01	0\\
228.01	0\\
229.01	0\\
230.01	0\\
231.01	0\\
232.01	0\\
233.01	0\\
234.01	0\\
235.01	0\\
236.01	0\\
237.01	0\\
238.01	0\\
239.01	0\\
240.01	0\\
241.01	0\\
242.01	0\\
243.01	0\\
244.01	0\\
245.01	0\\
246.01	0\\
247.01	0\\
248.01	0\\
249.01	0\\
250.01	0\\
251.01	0\\
252.01	0\\
253.01	0\\
254.01	0\\
255.01	0\\
256.01	0\\
257.01	0\\
258.01	0\\
259.01	0\\
260.01	0\\
261.01	0\\
262.01	0\\
263.01	0\\
264.01	0\\
265.01	0\\
266.01	0\\
267.01	0\\
268.01	0\\
269.01	0\\
270.01	0\\
271.01	0\\
272.01	0\\
273.01	0\\
274.01	0\\
275.01	0\\
276.01	0\\
277.01	0\\
278.01	0\\
279.01	0\\
280.01	0\\
281.01	0\\
282.01	0\\
283.01	0\\
284.01	0\\
285.01	0\\
286.01	0\\
287.01	0\\
288.01	0\\
289.01	0\\
290.01	0\\
291.01	0\\
292.01	0\\
293.01	0\\
294.01	0\\
295.01	0\\
296.01	0\\
297.01	0\\
298.01	0\\
299.01	0\\
300.01	0\\
301.01	0\\
302.01	0\\
303.01	0\\
304.01	0\\
305.01	0\\
306.01	0\\
307.01	0\\
308.01	0\\
309.01	0\\
310.01	0\\
311.01	0\\
312.01	0\\
313.01	0\\
314.01	0\\
315.01	0\\
316.01	0\\
317.01	0\\
318.01	0\\
319.01	0\\
320.01	0\\
321.01	0\\
322.01	0\\
323.01	0\\
324.01	0\\
325.01	0\\
326.01	0\\
327.01	0\\
328.01	0\\
329.01	0\\
330.01	0\\
331.01	0\\
332.01	0\\
333.01	0\\
334.01	0\\
335.01	0\\
336.01	0\\
337.01	0\\
338.01	0\\
339.01	0\\
340.01	0\\
341.01	0\\
342.01	0\\
343.01	0\\
344.01	0\\
345.01	0\\
346.01	0\\
347.01	0\\
348.01	0\\
349.01	0\\
350.01	0\\
351.01	0\\
352.01	0\\
353.01	0\\
354.01	0\\
355.01	0\\
356.01	0\\
357.01	0\\
358.01	0\\
359.01	0\\
360.01	0\\
361.01	0\\
362.01	0\\
363.01	0\\
364.01	0\\
365.01	0\\
366.01	0\\
367.01	0\\
368.01	0\\
369.01	0\\
370.01	0\\
371.01	0\\
372.01	0\\
373.01	0\\
374.01	0\\
375.01	0\\
376.01	0\\
377.01	0\\
378.01	0\\
379.01	0\\
380.01	0\\
381.01	0\\
382.01	0\\
383.01	0\\
384.01	0\\
385.01	0\\
386.01	0\\
387.01	0\\
388.01	0\\
389.01	0\\
390.01	0\\
391.01	0\\
392.01	0\\
393.01	0\\
394.01	0\\
395.01	0\\
396.01	0\\
397.01	0\\
398.01	0\\
399.01	0\\
400.01	0\\
401.01	0\\
402.01	0\\
403.01	0\\
404.01	0\\
405.01	0\\
406.01	0\\
407.01	0\\
408.01	0\\
409.01	0\\
410.01	0\\
411.01	0\\
412.01	0\\
413.01	0\\
414.01	0\\
415.01	0\\
416.01	0\\
417.01	0\\
418.01	0\\
419.01	0\\
420.01	0\\
421.01	0\\
422.01	0\\
423.01	0\\
424.01	0\\
425.01	0\\
426.01	0\\
427.01	0\\
428.01	0\\
429.01	0\\
430.01	0\\
431.01	0\\
432.01	0\\
433.01	0\\
434.01	0\\
435.01	0\\
436.01	0\\
437.01	0\\
438.01	0\\
439.01	0\\
440.01	0\\
441.01	0\\
442.01	0\\
443.01	0\\
444.01	0\\
445.01	0\\
446.01	0\\
447.01	0\\
448.01	0\\
449.01	0\\
450.01	0\\
451.01	0\\
452.01	0\\
453.01	0\\
454.01	0\\
455.01	0\\
456.01	0\\
457.01	0\\
458.01	0\\
459.01	0\\
460.01	0\\
461.01	0\\
462.01	0\\
463.01	0\\
464.01	0\\
465.01	0\\
466.01	0\\
467.01	0\\
468.01	0\\
469.01	0\\
470.01	0\\
471.01	0\\
472.01	0\\
473.01	0\\
474.01	0\\
475.01	0\\
476.01	0\\
477.01	0\\
478.01	0\\
479.01	0\\
480.01	0\\
481.01	0\\
482.01	0\\
483.01	0\\
484.01	0\\
485.01	0\\
486.01	0\\
487.01	0\\
488.01	0\\
489.01	0\\
490.01	0\\
491.01	0\\
492.01	0\\
493.01	0\\
494.01	0\\
495.01	0\\
496.01	0\\
497.01	0\\
498.01	0\\
499.01	0\\
500.01	0\\
501.01	0\\
502.01	0\\
503.01	0\\
504.01	0\\
505.01	0\\
506.01	0\\
507.01	0\\
508.01	0\\
509.01	0\\
510.01	0\\
511.01	0\\
512.01	0\\
513.01	0\\
514.01	0\\
515.01	0\\
516.01	0\\
517.01	0\\
518.01	0\\
519.01	0\\
520.01	0\\
521.01	0\\
522.01	0\\
523.01	0\\
524.01	0\\
525.01	0\\
526.01	0\\
527.01	0\\
528.01	0\\
529.01	0\\
530.01	0\\
531.01	0\\
532.01	0\\
533.01	0\\
534.01	0\\
535.01	0\\
536.01	0\\
537.01	0\\
538.01	0\\
539.01	0\\
540.01	0\\
541.01	0\\
542.01	0\\
543.01	0\\
544.01	0\\
545.01	0\\
546.01	0\\
547.01	0\\
548.01	0\\
549.01	0\\
550.01	0\\
551.01	0\\
552.01	0\\
553.01	0\\
554.01	0\\
555.01	0\\
556.01	0\\
557.01	0\\
558.01	0\\
559.01	0\\
560.01	0\\
561.01	0\\
562.01	0\\
563.01	0\\
564.01	0\\
565.01	0\\
566.01	0\\
567.01	0\\
568.01	0\\
569.01	0\\
570.01	0\\
571.01	0\\
572.01	0\\
573.01	0\\
574.01	0\\
575.01	0\\
576.01	0\\
577.01	0\\
578.01	0\\
579.01	0\\
580.01	0\\
581.01	0\\
582.01	0\\
583.01	0\\
584.01	0\\
585.01	0\\
586.01	0\\
587.01	0\\
588.01	0\\
589.01	0\\
590.01	0\\
591.01	0\\
592.01	0\\
593.01	0\\
594.01	0\\
595.01	0\\
596.01	0\\
597.01	0\\
598.01	0\\
599.01	0.00375813090594593\\
599.02	0.0037958555875784\\
599.03	0.00383394724377267\\
599.04	0.00387240947898433\\
599.05	0.00391124593307969\\
599.06	0.00395046028168361\\
599.07	0.00399005623653087\\
599.08	0.00403003754582085\\
599.09	0.00407040799457577\\
599.1	0.0041111714050025\\
599.11	0.00415233163685781\\
599.12	0.00419389258781727\\
599.13	0.00423585819384778\\
599.14	0.00427823242958374\\
599.15	0.00432101930870694\\
599.16	0.00436422288433018\\
599.17	0.00440784724938456\\
599.18	0.00445189653701075\\
599.19	0.004496374920954\\
599.2	0.00454128661596299\\
599.21	0.0045866358781927\\
599.22	0.00463242700561119\\
599.23	0.00467866433841031\\
599.24	0.00472535225942064\\
599.25	0.00477249519453023\\
599.26	0.00482009761310769\\
599.27	0.00486816402842932\\
599.28	0.00491669899811048\\
599.29	0.00496570712454115\\
599.3	0.00501519305532583\\
599.31	0.00506516148372771\\
599.32	0.00511561714911725\\
599.33	0.0051665648374251\\
599.34	0.00521800938159955\\
599.35	0.00526995566206842\\
599.36	0.00532240860720548\\
599.37	0.00537537319380152\\
599.38	0.00542885444753999\\
599.39	0.00548285743950266\\
599.4	0.00553738727623644\\
599.41	0.00559244911427383\\
599.42	0.0056480481606222\\
599.43	0.00570418967325791\\
599.44	0.00576087896162515\\
599.45	0.00581812138713979\\
599.46	0.00587592236369808\\
599.47	0.00593428735819029\\
599.48	0.00599322189101956\\
599.49	0.00605273153662558\\
599.5	0.00611282192401361\\
599.51	0.00617349873728858\\
599.52	0.00623476771619433\\
599.53	0.00629663465665838\\
599.54	0.0063591054113418\\
599.55	0.00642218589019449\\
599.56	0.00648588206101606\\
599.57	0.006550199950022\\
599.58	0.00661514564241551\\
599.59	0.00668072528296488\\
599.6	0.00674694507658657\\
599.61	0.00681381128893395\\
599.62	0.00688133024699177\\
599.63	0.00694950833967656\\
599.64	0.00701835201844282\\
599.65	0.00708786779789516\\
599.66	0.00715806225640646\\
599.67	0.00722894203674201\\
599.68	0.00730051384668985\\
599.69	0.00737278445969725\\
599.7	0.00744576071551335\\
599.71	0.0075194495208382\\
599.72	0.00759385784997808\\
599.73	0.00766899274550727\\
599.74	0.00774486131893627\\
599.75	0.00782147075138656\\
599.76	0.007898828294272\\
599.77	0.00797694126998689\\
599.78	0.00805581707260072\\
599.79	0.00813546316855976\\
599.8	0.00821588709739554\\
599.81	0.00829709647244021\\
599.82	0.0083790989815489\\
599.83	0.00846190238782927\\
599.84	0.008545514530378\\
599.85	0.00862994332502466\\
599.86	0.00871519676508279\\
599.87	0.00880128292210834\\
599.88	0.00888820994666552\\
599.89	0.00897598606910018\\
599.9	0.00906461960032072\\
599.91	0.00915411893258671\\
599.92	0.00924449254030512\\
599.93	0.00933574898083444\\
599.94	0.00942789689529664\\
599.95	0.00952094500939709\\
599.96	0.00961490213425244\\
599.97	0.00970977716722672\\
599.98	0.00980557909277551\\
599.99	0.00990231698329844\\
600	0.01\\
};
\addplot [color=mycolor20,solid,forget plot]
  table[row sep=crcr]{%
0.01	0\\
1.01	0\\
2.01	0\\
3.01	0\\
4.01	0\\
5.01	0\\
6.01	0\\
7.01	0\\
8.01	0\\
9.01	0\\
10.01	0\\
11.01	0\\
12.01	0\\
13.01	0\\
14.01	0\\
15.01	0\\
16.01	0\\
17.01	0\\
18.01	0\\
19.01	0\\
20.01	0\\
21.01	0\\
22.01	0\\
23.01	0\\
24.01	0\\
25.01	0\\
26.01	0\\
27.01	0\\
28.01	0\\
29.01	0\\
30.01	0\\
31.01	0\\
32.01	0\\
33.01	0\\
34.01	0\\
35.01	0\\
36.01	0\\
37.01	0\\
38.01	0\\
39.01	0\\
40.01	0\\
41.01	0\\
42.01	0\\
43.01	0\\
44.01	0\\
45.01	0\\
46.01	0\\
47.01	0\\
48.01	0\\
49.01	0\\
50.01	0\\
51.01	0\\
52.01	0\\
53.01	0\\
54.01	0\\
55.01	0\\
56.01	0\\
57.01	0\\
58.01	0\\
59.01	0\\
60.01	0\\
61.01	0\\
62.01	0\\
63.01	0\\
64.01	0\\
65.01	0\\
66.01	0\\
67.01	0\\
68.01	0\\
69.01	0\\
70.01	0\\
71.01	0\\
72.01	0\\
73.01	0\\
74.01	0\\
75.01	0\\
76.01	0\\
77.01	0\\
78.01	0\\
79.01	0\\
80.01	0\\
81.01	0\\
82.01	0\\
83.01	0\\
84.01	0\\
85.01	0\\
86.01	0\\
87.01	0\\
88.01	0\\
89.01	0\\
90.01	0\\
91.01	0\\
92.01	0\\
93.01	0\\
94.01	0\\
95.01	0\\
96.01	0\\
97.01	0\\
98.01	0\\
99.01	0\\
100.01	0\\
101.01	0\\
102.01	0\\
103.01	0\\
104.01	0\\
105.01	0\\
106.01	0\\
107.01	0\\
108.01	0\\
109.01	0\\
110.01	0\\
111.01	0\\
112.01	0\\
113.01	0\\
114.01	0\\
115.01	0\\
116.01	0\\
117.01	0\\
118.01	0\\
119.01	0\\
120.01	0\\
121.01	0\\
122.01	0\\
123.01	0\\
124.01	0\\
125.01	0\\
126.01	0\\
127.01	0\\
128.01	0\\
129.01	0\\
130.01	0\\
131.01	0\\
132.01	0\\
133.01	0\\
134.01	0\\
135.01	0\\
136.01	0\\
137.01	0\\
138.01	0\\
139.01	0\\
140.01	0\\
141.01	0\\
142.01	0\\
143.01	0\\
144.01	0\\
145.01	0\\
146.01	0\\
147.01	0\\
148.01	0\\
149.01	0\\
150.01	0\\
151.01	0\\
152.01	0\\
153.01	0\\
154.01	0\\
155.01	0\\
156.01	0\\
157.01	0\\
158.01	0\\
159.01	0\\
160.01	0\\
161.01	0\\
162.01	0\\
163.01	0\\
164.01	0\\
165.01	0\\
166.01	0\\
167.01	0\\
168.01	0\\
169.01	0\\
170.01	0\\
171.01	0\\
172.01	0\\
173.01	0\\
174.01	0\\
175.01	0\\
176.01	0\\
177.01	0\\
178.01	0\\
179.01	0\\
180.01	0\\
181.01	0\\
182.01	0\\
183.01	0\\
184.01	0\\
185.01	0\\
186.01	0\\
187.01	0\\
188.01	0\\
189.01	0\\
190.01	0\\
191.01	0\\
192.01	0\\
193.01	0\\
194.01	0\\
195.01	0\\
196.01	0\\
197.01	0\\
198.01	0\\
199.01	0\\
200.01	0\\
201.01	0\\
202.01	0\\
203.01	0\\
204.01	0\\
205.01	0\\
206.01	0\\
207.01	0\\
208.01	0\\
209.01	0\\
210.01	0\\
211.01	0\\
212.01	0\\
213.01	0\\
214.01	0\\
215.01	0\\
216.01	0\\
217.01	0\\
218.01	0\\
219.01	0\\
220.01	0\\
221.01	0\\
222.01	0\\
223.01	0\\
224.01	0\\
225.01	0\\
226.01	0\\
227.01	0\\
228.01	0\\
229.01	0\\
230.01	0\\
231.01	0\\
232.01	0\\
233.01	0\\
234.01	0\\
235.01	0\\
236.01	0\\
237.01	0\\
238.01	0\\
239.01	0\\
240.01	0\\
241.01	0\\
242.01	0\\
243.01	0\\
244.01	0\\
245.01	0\\
246.01	0\\
247.01	0\\
248.01	0\\
249.01	0\\
250.01	0\\
251.01	0\\
252.01	0\\
253.01	0\\
254.01	0\\
255.01	0\\
256.01	0\\
257.01	0\\
258.01	0\\
259.01	0\\
260.01	0\\
261.01	0\\
262.01	0\\
263.01	0\\
264.01	0\\
265.01	0\\
266.01	0\\
267.01	0\\
268.01	0\\
269.01	0\\
270.01	0\\
271.01	0\\
272.01	0\\
273.01	0\\
274.01	0\\
275.01	0\\
276.01	0\\
277.01	0\\
278.01	0\\
279.01	0\\
280.01	0\\
281.01	0\\
282.01	0\\
283.01	0\\
284.01	0\\
285.01	0\\
286.01	0\\
287.01	0\\
288.01	0\\
289.01	0\\
290.01	0\\
291.01	0\\
292.01	0\\
293.01	0\\
294.01	0\\
295.01	0\\
296.01	0\\
297.01	0\\
298.01	0\\
299.01	0\\
300.01	0\\
301.01	0\\
302.01	0\\
303.01	0\\
304.01	0\\
305.01	0\\
306.01	0\\
307.01	0\\
308.01	0\\
309.01	0\\
310.01	0\\
311.01	0\\
312.01	0\\
313.01	0\\
314.01	0\\
315.01	0\\
316.01	0\\
317.01	0\\
318.01	0\\
319.01	0\\
320.01	0\\
321.01	0\\
322.01	0\\
323.01	0\\
324.01	0\\
325.01	0\\
326.01	0\\
327.01	0\\
328.01	0\\
329.01	0\\
330.01	0\\
331.01	0\\
332.01	0\\
333.01	0\\
334.01	0\\
335.01	0\\
336.01	0\\
337.01	0\\
338.01	0\\
339.01	0\\
340.01	0\\
341.01	0\\
342.01	0\\
343.01	0\\
344.01	0\\
345.01	0\\
346.01	0\\
347.01	0\\
348.01	0\\
349.01	0\\
350.01	0\\
351.01	0\\
352.01	0\\
353.01	0\\
354.01	0\\
355.01	0\\
356.01	0\\
357.01	0\\
358.01	0\\
359.01	0\\
360.01	0\\
361.01	0\\
362.01	0\\
363.01	0\\
364.01	0\\
365.01	0\\
366.01	0\\
367.01	0\\
368.01	0\\
369.01	0\\
370.01	0\\
371.01	0\\
372.01	0\\
373.01	0\\
374.01	0\\
375.01	0\\
376.01	0\\
377.01	0\\
378.01	0\\
379.01	0\\
380.01	0\\
381.01	0\\
382.01	0\\
383.01	0\\
384.01	0\\
385.01	0\\
386.01	0\\
387.01	0\\
388.01	0\\
389.01	0\\
390.01	0\\
391.01	0\\
392.01	0\\
393.01	0\\
394.01	0\\
395.01	0\\
396.01	0\\
397.01	0\\
398.01	0\\
399.01	0\\
400.01	0\\
401.01	0\\
402.01	0\\
403.01	0\\
404.01	0\\
405.01	0\\
406.01	0\\
407.01	0\\
408.01	0\\
409.01	0\\
410.01	0\\
411.01	0\\
412.01	0\\
413.01	0\\
414.01	0\\
415.01	0\\
416.01	0\\
417.01	0\\
418.01	0\\
419.01	0\\
420.01	0\\
421.01	0\\
422.01	0\\
423.01	0\\
424.01	0\\
425.01	0\\
426.01	0\\
427.01	0\\
428.01	0\\
429.01	0\\
430.01	0\\
431.01	0\\
432.01	0\\
433.01	0\\
434.01	0\\
435.01	0\\
436.01	0\\
437.01	0\\
438.01	0\\
439.01	0\\
440.01	0\\
441.01	0\\
442.01	0\\
443.01	0\\
444.01	0\\
445.01	0\\
446.01	0\\
447.01	0\\
448.01	0\\
449.01	0\\
450.01	0\\
451.01	0\\
452.01	0\\
453.01	0\\
454.01	0\\
455.01	0\\
456.01	0\\
457.01	0\\
458.01	0\\
459.01	0\\
460.01	0\\
461.01	0\\
462.01	0\\
463.01	0\\
464.01	0\\
465.01	0\\
466.01	0\\
467.01	0\\
468.01	0\\
469.01	0\\
470.01	0\\
471.01	0\\
472.01	0\\
473.01	0\\
474.01	0\\
475.01	0\\
476.01	0\\
477.01	0\\
478.01	0\\
479.01	0\\
480.01	0\\
481.01	0\\
482.01	0\\
483.01	0\\
484.01	0\\
485.01	0\\
486.01	0\\
487.01	0\\
488.01	0\\
489.01	0\\
490.01	0\\
491.01	0\\
492.01	0\\
493.01	0\\
494.01	0\\
495.01	0\\
496.01	0\\
497.01	0\\
498.01	0\\
499.01	0\\
500.01	0\\
501.01	0\\
502.01	0\\
503.01	0\\
504.01	0\\
505.01	0\\
506.01	0\\
507.01	0\\
508.01	0\\
509.01	0\\
510.01	0\\
511.01	0\\
512.01	0\\
513.01	0\\
514.01	0\\
515.01	0\\
516.01	0\\
517.01	0\\
518.01	0\\
519.01	0\\
520.01	0\\
521.01	0\\
522.01	0\\
523.01	0\\
524.01	0\\
525.01	0\\
526.01	0\\
527.01	0\\
528.01	0\\
529.01	0\\
530.01	0\\
531.01	0\\
532.01	0\\
533.01	0\\
534.01	0\\
535.01	0\\
536.01	0\\
537.01	0\\
538.01	0\\
539.01	0\\
540.01	0\\
541.01	0\\
542.01	0\\
543.01	0\\
544.01	0\\
545.01	0\\
546.01	0\\
547.01	0\\
548.01	0\\
549.01	0\\
550.01	0\\
551.01	0\\
552.01	0\\
553.01	0\\
554.01	0\\
555.01	0\\
556.01	0\\
557.01	0\\
558.01	0\\
559.01	0\\
560.01	0\\
561.01	0\\
562.01	0\\
563.01	0\\
564.01	0\\
565.01	0\\
566.01	0\\
567.01	0\\
568.01	0\\
569.01	0\\
570.01	0\\
571.01	0\\
572.01	0\\
573.01	0\\
574.01	0\\
575.01	0\\
576.01	0\\
577.01	0\\
578.01	0\\
579.01	0\\
580.01	0\\
581.01	0\\
582.01	0\\
583.01	0\\
584.01	0\\
585.01	0\\
586.01	0\\
587.01	0\\
588.01	0\\
589.01	0\\
590.01	0\\
591.01	0\\
592.01	0\\
593.01	0\\
594.01	0\\
595.01	0\\
596.01	0\\
597.01	0\\
598.01	0\\
599.01	0.0037581309059459\\
599.02	0.00379585558757836\\
599.03	0.00383394724377262\\
599.04	0.00387240947898428\\
599.05	0.00391124593307962\\
599.06	0.00395046028168355\\
599.07	0.00399005623653081\\
599.08	0.00403003754582079\\
599.09	0.00407040799457573\\
599.1	0.00411117140500246\\
599.11	0.00415233163685777\\
599.12	0.00419389258781722\\
599.13	0.00423585819384772\\
599.14	0.00427823242958369\\
599.15	0.0043210193087069\\
599.16	0.00436422288433012\\
599.17	0.00440784724938451\\
599.18	0.00445189653701071\\
599.19	0.00449637492095396\\
599.2	0.00454128661596297\\
599.21	0.00458663587819269\\
599.22	0.00463242700561117\\
599.23	0.00467866433841031\\
599.24	0.00472535225942064\\
599.25	0.00477249519453023\\
599.26	0.00482009761310769\\
599.27	0.00486816402842931\\
599.28	0.00491669899811047\\
599.29	0.00496570712454113\\
599.3	0.00501519305532581\\
599.31	0.0050651614837277\\
599.32	0.00511561714911722\\
599.33	0.00516656483742507\\
599.34	0.00521800938159951\\
599.35	0.00526995566206837\\
599.36	0.00532240860720543\\
599.37	0.00537537319380148\\
599.38	0.00542885444753995\\
599.39	0.00548285743950262\\
599.4	0.0055373872762364\\
599.41	0.00559244911427378\\
599.42	0.00564804816062216\\
599.43	0.00570418967325786\\
599.44	0.00576087896162511\\
599.45	0.00581812138713977\\
599.46	0.00587592236369805\\
599.47	0.00593428735819027\\
599.48	0.00599322189101954\\
599.49	0.00605273153662557\\
599.5	0.00611282192401361\\
599.51	0.00617349873728855\\
599.52	0.00623476771619431\\
599.53	0.00629663465665838\\
599.54	0.00635910541134178\\
599.55	0.00642218589019448\\
599.56	0.00648588206101606\\
599.57	0.006550199950022\\
599.58	0.00661514564241551\\
599.59	0.00668072528296489\\
599.6	0.00674694507658658\\
599.61	0.00681381128893395\\
599.62	0.00688133024699177\\
599.63	0.00694950833967656\\
599.64	0.00701835201844282\\
599.65	0.00708786779789517\\
599.66	0.00715806225640647\\
599.67	0.00722894203674201\\
599.68	0.00730051384668985\\
599.69	0.00737278445969725\\
599.7	0.00744576071551334\\
599.71	0.00751944952083819\\
599.72	0.00759385784997808\\
599.73	0.00766899274550727\\
599.74	0.00774486131893627\\
599.75	0.00782147075138655\\
599.76	0.007898828294272\\
599.77	0.00797694126998689\\
599.78	0.00805581707260072\\
599.79	0.00813546316855976\\
599.8	0.00821588709739554\\
599.81	0.0082970964724402\\
599.82	0.0083790989815489\\
599.83	0.00846190238782927\\
599.84	0.008545514530378\\
599.85	0.00862994332502466\\
599.86	0.00871519676508279\\
599.87	0.00880128292210833\\
599.88	0.00888820994666552\\
599.89	0.00897598606910018\\
599.9	0.00906461960032073\\
599.91	0.00915411893258672\\
599.92	0.00924449254030512\\
599.93	0.00933574898083444\\
599.94	0.00942789689529664\\
599.95	0.00952094500939709\\
599.96	0.00961490213425244\\
599.97	0.00970977716722672\\
599.98	0.00980557909277551\\
599.99	0.00990231698329844\\
600	0.01\\
};
\addplot [color=mycolor21,solid,forget plot]
  table[row sep=crcr]{%
0.01	0\\
1.01	0\\
2.01	0\\
3.01	0\\
4.01	0\\
5.01	0\\
6.01	0\\
7.01	0\\
8.01	0\\
9.01	0\\
10.01	0\\
11.01	0\\
12.01	0\\
13.01	0\\
14.01	0\\
15.01	0\\
16.01	0\\
17.01	0\\
18.01	0\\
19.01	0\\
20.01	0\\
21.01	0\\
22.01	0\\
23.01	0\\
24.01	0\\
25.01	0\\
26.01	0\\
27.01	0\\
28.01	0\\
29.01	0\\
30.01	0\\
31.01	0\\
32.01	0\\
33.01	0\\
34.01	0\\
35.01	0\\
36.01	0\\
37.01	0\\
38.01	0\\
39.01	0\\
40.01	0\\
41.01	0\\
42.01	0\\
43.01	0\\
44.01	0\\
45.01	0\\
46.01	0\\
47.01	0\\
48.01	0\\
49.01	0\\
50.01	0\\
51.01	0\\
52.01	0\\
53.01	0\\
54.01	0\\
55.01	0\\
56.01	0\\
57.01	0\\
58.01	0\\
59.01	0\\
60.01	0\\
61.01	0\\
62.01	0\\
63.01	0\\
64.01	0\\
65.01	0\\
66.01	0\\
67.01	0\\
68.01	0\\
69.01	0\\
70.01	0\\
71.01	0\\
72.01	0\\
73.01	0\\
74.01	0\\
75.01	0\\
76.01	0\\
77.01	0\\
78.01	0\\
79.01	0\\
80.01	0\\
81.01	0\\
82.01	0\\
83.01	0\\
84.01	0\\
85.01	0\\
86.01	0\\
87.01	0\\
88.01	0\\
89.01	0\\
90.01	0\\
91.01	0\\
92.01	0\\
93.01	0\\
94.01	0\\
95.01	0\\
96.01	0\\
97.01	0\\
98.01	0\\
99.01	0\\
100.01	0\\
101.01	0\\
102.01	0\\
103.01	0\\
104.01	0\\
105.01	0\\
106.01	0\\
107.01	0\\
108.01	0\\
109.01	0\\
110.01	0\\
111.01	0\\
112.01	0\\
113.01	0\\
114.01	0\\
115.01	0\\
116.01	0\\
117.01	0\\
118.01	0\\
119.01	0\\
120.01	0\\
121.01	0\\
122.01	0\\
123.01	0\\
124.01	0\\
125.01	0\\
126.01	0\\
127.01	0\\
128.01	0\\
129.01	0\\
130.01	0\\
131.01	0\\
132.01	0\\
133.01	0\\
134.01	0\\
135.01	0\\
136.01	0\\
137.01	0\\
138.01	0\\
139.01	0\\
140.01	0\\
141.01	0\\
142.01	0\\
143.01	0\\
144.01	0\\
145.01	0\\
146.01	0\\
147.01	0\\
148.01	0\\
149.01	0\\
150.01	0\\
151.01	0\\
152.01	0\\
153.01	0\\
154.01	0\\
155.01	0\\
156.01	0\\
157.01	0\\
158.01	0\\
159.01	0\\
160.01	0\\
161.01	0\\
162.01	0\\
163.01	0\\
164.01	0\\
165.01	0\\
166.01	0\\
167.01	0\\
168.01	0\\
169.01	0\\
170.01	0\\
171.01	0\\
172.01	0\\
173.01	0\\
174.01	0\\
175.01	0\\
176.01	0\\
177.01	0\\
178.01	0\\
179.01	0\\
180.01	0\\
181.01	0\\
182.01	0\\
183.01	0\\
184.01	0\\
185.01	0\\
186.01	0\\
187.01	0\\
188.01	0\\
189.01	0\\
190.01	0\\
191.01	0\\
192.01	0\\
193.01	0\\
194.01	0\\
195.01	0\\
196.01	0\\
197.01	0\\
198.01	0\\
199.01	0\\
200.01	0\\
201.01	0\\
202.01	0\\
203.01	0\\
204.01	0\\
205.01	0\\
206.01	0\\
207.01	0\\
208.01	0\\
209.01	0\\
210.01	0\\
211.01	0\\
212.01	0\\
213.01	0\\
214.01	0\\
215.01	0\\
216.01	0\\
217.01	0\\
218.01	0\\
219.01	0\\
220.01	0\\
221.01	0\\
222.01	0\\
223.01	0\\
224.01	0\\
225.01	0\\
226.01	0\\
227.01	0\\
228.01	0\\
229.01	0\\
230.01	0\\
231.01	0\\
232.01	0\\
233.01	0\\
234.01	0\\
235.01	0\\
236.01	0\\
237.01	0\\
238.01	0\\
239.01	0\\
240.01	0\\
241.01	0\\
242.01	0\\
243.01	0\\
244.01	0\\
245.01	0\\
246.01	0\\
247.01	0\\
248.01	0\\
249.01	0\\
250.01	0\\
251.01	0\\
252.01	0\\
253.01	0\\
254.01	0\\
255.01	0\\
256.01	0\\
257.01	0\\
258.01	0\\
259.01	0\\
260.01	0\\
261.01	0\\
262.01	0\\
263.01	0\\
264.01	0\\
265.01	0\\
266.01	0\\
267.01	0\\
268.01	0\\
269.01	0\\
270.01	0\\
271.01	0\\
272.01	0\\
273.01	0\\
274.01	0\\
275.01	0\\
276.01	0\\
277.01	0\\
278.01	0\\
279.01	0\\
280.01	0\\
281.01	0\\
282.01	0\\
283.01	0\\
284.01	0\\
285.01	0\\
286.01	0\\
287.01	0\\
288.01	0\\
289.01	0\\
290.01	0\\
291.01	0\\
292.01	0\\
293.01	0\\
294.01	0\\
295.01	0\\
296.01	0\\
297.01	0\\
298.01	0\\
299.01	0\\
300.01	0\\
301.01	0\\
302.01	0\\
303.01	0\\
304.01	0\\
305.01	0\\
306.01	0\\
307.01	0\\
308.01	0\\
309.01	0\\
310.01	0\\
311.01	0\\
312.01	0\\
313.01	0\\
314.01	0\\
315.01	0\\
316.01	0\\
317.01	0\\
318.01	0\\
319.01	0\\
320.01	0\\
321.01	0\\
322.01	0\\
323.01	0\\
324.01	0\\
325.01	0\\
326.01	0\\
327.01	0\\
328.01	0\\
329.01	0\\
330.01	0\\
331.01	0\\
332.01	0\\
333.01	0\\
334.01	0\\
335.01	0\\
336.01	0\\
337.01	0\\
338.01	0\\
339.01	0\\
340.01	0\\
341.01	0\\
342.01	0\\
343.01	0\\
344.01	0\\
345.01	0\\
346.01	0\\
347.01	0\\
348.01	0\\
349.01	0\\
350.01	0\\
351.01	0\\
352.01	0\\
353.01	0\\
354.01	0\\
355.01	0\\
356.01	0\\
357.01	0\\
358.01	0\\
359.01	0\\
360.01	0\\
361.01	0\\
362.01	0\\
363.01	0\\
364.01	0\\
365.01	0\\
366.01	0\\
367.01	0\\
368.01	0\\
369.01	0\\
370.01	0\\
371.01	0\\
372.01	0\\
373.01	0\\
374.01	0\\
375.01	0\\
376.01	0\\
377.01	0\\
378.01	0\\
379.01	0\\
380.01	0\\
381.01	0\\
382.01	0\\
383.01	0\\
384.01	0\\
385.01	0\\
386.01	0\\
387.01	0\\
388.01	0\\
389.01	0\\
390.01	0\\
391.01	0\\
392.01	0\\
393.01	0\\
394.01	0\\
395.01	0\\
396.01	0\\
397.01	0\\
398.01	0\\
399.01	0\\
400.01	0\\
401.01	0\\
402.01	0\\
403.01	0\\
404.01	0\\
405.01	0\\
406.01	0\\
407.01	0\\
408.01	0\\
409.01	0\\
410.01	0\\
411.01	0\\
412.01	0\\
413.01	0\\
414.01	0\\
415.01	0\\
416.01	0\\
417.01	0\\
418.01	0\\
419.01	0\\
420.01	0\\
421.01	0\\
422.01	0\\
423.01	0\\
424.01	0\\
425.01	0\\
426.01	0\\
427.01	0\\
428.01	0\\
429.01	0\\
430.01	0\\
431.01	0\\
432.01	0\\
433.01	0\\
434.01	0\\
435.01	0\\
436.01	0\\
437.01	0\\
438.01	0\\
439.01	0\\
440.01	0\\
441.01	0\\
442.01	0\\
443.01	0\\
444.01	0\\
445.01	0\\
446.01	0\\
447.01	0\\
448.01	0\\
449.01	0\\
450.01	0\\
451.01	0\\
452.01	0\\
453.01	0\\
454.01	0\\
455.01	0\\
456.01	0\\
457.01	0\\
458.01	0\\
459.01	0\\
460.01	0\\
461.01	0\\
462.01	0\\
463.01	0\\
464.01	0\\
465.01	0\\
466.01	0\\
467.01	0\\
468.01	0\\
469.01	0\\
470.01	0\\
471.01	0\\
472.01	0\\
473.01	0\\
474.01	0\\
475.01	0\\
476.01	0\\
477.01	0\\
478.01	0\\
479.01	0\\
480.01	0\\
481.01	0\\
482.01	0\\
483.01	0\\
484.01	0\\
485.01	0\\
486.01	0\\
487.01	0\\
488.01	0\\
489.01	0\\
490.01	0\\
491.01	0\\
492.01	0\\
493.01	0\\
494.01	0\\
495.01	0\\
496.01	0\\
497.01	0\\
498.01	0\\
499.01	0\\
500.01	0\\
501.01	0\\
502.01	0\\
503.01	0\\
504.01	0\\
505.01	0\\
506.01	0\\
507.01	0\\
508.01	0\\
509.01	0\\
510.01	0\\
511.01	0\\
512.01	0\\
513.01	0\\
514.01	0\\
515.01	0\\
516.01	0\\
517.01	0\\
518.01	0\\
519.01	0\\
520.01	0\\
521.01	0\\
522.01	0\\
523.01	0\\
524.01	0\\
525.01	0\\
526.01	0\\
527.01	0\\
528.01	0\\
529.01	0\\
530.01	0\\
531.01	0\\
532.01	0\\
533.01	0\\
534.01	0\\
535.01	0\\
536.01	0\\
537.01	0\\
538.01	0\\
539.01	0\\
540.01	0\\
541.01	0\\
542.01	0\\
543.01	0\\
544.01	0\\
545.01	0\\
546.01	0\\
547.01	0\\
548.01	0\\
549.01	0\\
550.01	0\\
551.01	0\\
552.01	0\\
553.01	0\\
554.01	0\\
555.01	0\\
556.01	0\\
557.01	0\\
558.01	0\\
559.01	0\\
560.01	0\\
561.01	0\\
562.01	0\\
563.01	0\\
564.01	0\\
565.01	0\\
566.01	0\\
567.01	0\\
568.01	0\\
569.01	0\\
570.01	0\\
571.01	0\\
572.01	0\\
573.01	0\\
574.01	0\\
575.01	0\\
576.01	0\\
577.01	0\\
578.01	0\\
579.01	0\\
580.01	0\\
581.01	0\\
582.01	0\\
583.01	0\\
584.01	0\\
585.01	0\\
586.01	0\\
587.01	0\\
588.01	0\\
589.01	0\\
590.01	0\\
591.01	0\\
592.01	0\\
593.01	0\\
594.01	0\\
595.01	0\\
596.01	0\\
597.01	0\\
598.01	0.000700248929397768\\
599.01	0.00375813090594587\\
599.02	0.00379585558757835\\
599.03	0.00383394724377262\\
599.04	0.00387240947898428\\
599.05	0.00391124593307964\\
599.06	0.00395046028168355\\
599.07	0.00399005623653081\\
599.08	0.00403003754582079\\
599.09	0.00407040799457573\\
599.1	0.00411117140500247\\
599.11	0.00415233163685777\\
599.12	0.00419389258781723\\
599.13	0.00423585819384774\\
599.14	0.00427823242958369\\
599.15	0.00432101930870692\\
599.16	0.00436422288433015\\
599.17	0.00440784724938452\\
599.18	0.00445189653701071\\
599.19	0.00449637492095395\\
599.2	0.00454128661596294\\
599.21	0.00458663587819266\\
599.22	0.00463242700561114\\
599.23	0.00467866433841028\\
599.24	0.00472535225942061\\
599.25	0.0047724951945302\\
599.26	0.00482009761310766\\
599.27	0.00486816402842928\\
599.28	0.00491669899811044\\
599.29	0.00496570712454111\\
599.3	0.00501519305532579\\
599.31	0.00506516148372767\\
599.32	0.00511561714911721\\
599.33	0.00516656483742504\\
599.34	0.0052180093815995\\
599.35	0.00526995566206837\\
599.36	0.00532240860720543\\
599.37	0.00537537319380148\\
599.38	0.00542885444753995\\
599.39	0.00548285743950262\\
599.4	0.0055373872762364\\
599.41	0.00559244911427378\\
599.42	0.00564804816062216\\
599.43	0.00570418967325786\\
599.44	0.00576087896162512\\
599.45	0.00581812138713976\\
599.46	0.00587592236369805\\
599.47	0.00593428735819028\\
599.48	0.00599322189101954\\
599.49	0.00605273153662557\\
599.5	0.00611282192401361\\
599.51	0.00617349873728855\\
599.52	0.0062347677161943\\
599.53	0.00629663465665836\\
599.54	0.00635910541134178\\
599.55	0.00642218589019448\\
599.56	0.00648588206101605\\
599.57	0.006550199950022\\
599.58	0.0066151456424155\\
599.59	0.00668072528296488\\
599.6	0.00674694507658658\\
599.61	0.00681381128893396\\
599.62	0.00688133024699177\\
599.63	0.00694950833967656\\
599.64	0.00701835201844283\\
599.65	0.00708786779789517\\
599.66	0.00715806225640647\\
599.67	0.00722894203674201\\
599.68	0.00730051384668986\\
599.69	0.00737278445969725\\
599.7	0.00744576071551335\\
599.71	0.0075194495208382\\
599.72	0.00759385784997809\\
599.73	0.00766899274550728\\
599.74	0.00774486131893628\\
599.75	0.00782147075138657\\
599.76	0.00789882829427201\\
599.77	0.0079769412699869\\
599.78	0.00805581707260072\\
599.79	0.00813546316855976\\
599.8	0.00821588709739554\\
599.81	0.00829709647244021\\
599.82	0.00837909898154891\\
599.83	0.00846190238782927\\
599.84	0.008545514530378\\
599.85	0.00862994332502466\\
599.86	0.00871519676508279\\
599.87	0.00880128292210834\\
599.88	0.00888820994666552\\
599.89	0.00897598606910018\\
599.9	0.00906461960032072\\
599.91	0.00915411893258671\\
599.92	0.00924449254030512\\
599.93	0.00933574898083444\\
599.94	0.00942789689529664\\
599.95	0.00952094500939709\\
599.96	0.00961490213425245\\
599.97	0.00970977716722672\\
599.98	0.00980557909277551\\
599.99	0.00990231698329844\\
600	0.01\\
};
\addplot [color=black!20!mycolor21,solid,forget plot]
  table[row sep=crcr]{%
0.01	0\\
1.01	0\\
2.01	0\\
3.01	0\\
4.01	0\\
5.01	0\\
6.01	0\\
7.01	0\\
8.01	0\\
9.01	0\\
10.01	0\\
11.01	0\\
12.01	0\\
13.01	0\\
14.01	0\\
15.01	0\\
16.01	0\\
17.01	0\\
18.01	0\\
19.01	0\\
20.01	0\\
21.01	0\\
22.01	0\\
23.01	0\\
24.01	0\\
25.01	0\\
26.01	0\\
27.01	0\\
28.01	0\\
29.01	0\\
30.01	0\\
31.01	0\\
32.01	0\\
33.01	0\\
34.01	0\\
35.01	0\\
36.01	0\\
37.01	0\\
38.01	0\\
39.01	0\\
40.01	0\\
41.01	0\\
42.01	0\\
43.01	0\\
44.01	0\\
45.01	0\\
46.01	0\\
47.01	0\\
48.01	0\\
49.01	0\\
50.01	0\\
51.01	0\\
52.01	0\\
53.01	0\\
54.01	0\\
55.01	0\\
56.01	0\\
57.01	0\\
58.01	0\\
59.01	0\\
60.01	0\\
61.01	0\\
62.01	0\\
63.01	0\\
64.01	0\\
65.01	0\\
66.01	0\\
67.01	0\\
68.01	0\\
69.01	0\\
70.01	0\\
71.01	0\\
72.01	0\\
73.01	0\\
74.01	0\\
75.01	0\\
76.01	0\\
77.01	0\\
78.01	0\\
79.01	0\\
80.01	0\\
81.01	0\\
82.01	0\\
83.01	0\\
84.01	0\\
85.01	0\\
86.01	0\\
87.01	0\\
88.01	0\\
89.01	0\\
90.01	0\\
91.01	0\\
92.01	0\\
93.01	0\\
94.01	0\\
95.01	0\\
96.01	0\\
97.01	0\\
98.01	0\\
99.01	0\\
100.01	0\\
101.01	0\\
102.01	0\\
103.01	0\\
104.01	0\\
105.01	0\\
106.01	0\\
107.01	0\\
108.01	0\\
109.01	0\\
110.01	0\\
111.01	0\\
112.01	0\\
113.01	0\\
114.01	0\\
115.01	0\\
116.01	0\\
117.01	0\\
118.01	0\\
119.01	0\\
120.01	0\\
121.01	0\\
122.01	0\\
123.01	0\\
124.01	0\\
125.01	0\\
126.01	0\\
127.01	0\\
128.01	0\\
129.01	0\\
130.01	0\\
131.01	0\\
132.01	0\\
133.01	0\\
134.01	0\\
135.01	0\\
136.01	0\\
137.01	0\\
138.01	0\\
139.01	0\\
140.01	0\\
141.01	0\\
142.01	0\\
143.01	0\\
144.01	0\\
145.01	0\\
146.01	0\\
147.01	0\\
148.01	0\\
149.01	0\\
150.01	0\\
151.01	0\\
152.01	0\\
153.01	0\\
154.01	0\\
155.01	0\\
156.01	0\\
157.01	0\\
158.01	0\\
159.01	0\\
160.01	0\\
161.01	0\\
162.01	0\\
163.01	0\\
164.01	0\\
165.01	0\\
166.01	0\\
167.01	0\\
168.01	0\\
169.01	0\\
170.01	0\\
171.01	0\\
172.01	0\\
173.01	0\\
174.01	0\\
175.01	0\\
176.01	0\\
177.01	0\\
178.01	0\\
179.01	0\\
180.01	0\\
181.01	0\\
182.01	0\\
183.01	0\\
184.01	0\\
185.01	0\\
186.01	0\\
187.01	0\\
188.01	0\\
189.01	0\\
190.01	0\\
191.01	0\\
192.01	0\\
193.01	0\\
194.01	0\\
195.01	0\\
196.01	0\\
197.01	0\\
198.01	0\\
199.01	0\\
200.01	0\\
201.01	0\\
202.01	0\\
203.01	0\\
204.01	0\\
205.01	0\\
206.01	0\\
207.01	0\\
208.01	0\\
209.01	0\\
210.01	0\\
211.01	0\\
212.01	0\\
213.01	0\\
214.01	0\\
215.01	0\\
216.01	0\\
217.01	0\\
218.01	0\\
219.01	0\\
220.01	0\\
221.01	0\\
222.01	0\\
223.01	0\\
224.01	0\\
225.01	0\\
226.01	0\\
227.01	0\\
228.01	0\\
229.01	0\\
230.01	0\\
231.01	0\\
232.01	0\\
233.01	0\\
234.01	0\\
235.01	0\\
236.01	0\\
237.01	0\\
238.01	0\\
239.01	0\\
240.01	0\\
241.01	0\\
242.01	0\\
243.01	0\\
244.01	0\\
245.01	0\\
246.01	0\\
247.01	0\\
248.01	0\\
249.01	0\\
250.01	0\\
251.01	0\\
252.01	0\\
253.01	0\\
254.01	0\\
255.01	0\\
256.01	0\\
257.01	0\\
258.01	0\\
259.01	0\\
260.01	0\\
261.01	0\\
262.01	0\\
263.01	0\\
264.01	0\\
265.01	0\\
266.01	0\\
267.01	0\\
268.01	0\\
269.01	0\\
270.01	0\\
271.01	0\\
272.01	0\\
273.01	0\\
274.01	0\\
275.01	0\\
276.01	0\\
277.01	0\\
278.01	0\\
279.01	0\\
280.01	0\\
281.01	0\\
282.01	0\\
283.01	0\\
284.01	0\\
285.01	0\\
286.01	0\\
287.01	0\\
288.01	0\\
289.01	0\\
290.01	0\\
291.01	0\\
292.01	0\\
293.01	0\\
294.01	0\\
295.01	0\\
296.01	0\\
297.01	0\\
298.01	0\\
299.01	0\\
300.01	0\\
301.01	0\\
302.01	0\\
303.01	0\\
304.01	0\\
305.01	0\\
306.01	0\\
307.01	0\\
308.01	0\\
309.01	0\\
310.01	0\\
311.01	0\\
312.01	0\\
313.01	0\\
314.01	0\\
315.01	0\\
316.01	0\\
317.01	0\\
318.01	0\\
319.01	0\\
320.01	0\\
321.01	0\\
322.01	0\\
323.01	0\\
324.01	0\\
325.01	0\\
326.01	0\\
327.01	0\\
328.01	0\\
329.01	0\\
330.01	0\\
331.01	0\\
332.01	0\\
333.01	0\\
334.01	0\\
335.01	0\\
336.01	0\\
337.01	0\\
338.01	0\\
339.01	0\\
340.01	0\\
341.01	0\\
342.01	0\\
343.01	0\\
344.01	0\\
345.01	0\\
346.01	0\\
347.01	0\\
348.01	0\\
349.01	0\\
350.01	0\\
351.01	0\\
352.01	0\\
353.01	0\\
354.01	0\\
355.01	0\\
356.01	0\\
357.01	0\\
358.01	0\\
359.01	0\\
360.01	0\\
361.01	0\\
362.01	0\\
363.01	0\\
364.01	0\\
365.01	0\\
366.01	0\\
367.01	0\\
368.01	0\\
369.01	0\\
370.01	0\\
371.01	0\\
372.01	0\\
373.01	0\\
374.01	0\\
375.01	0\\
376.01	0\\
377.01	0\\
378.01	0\\
379.01	0\\
380.01	0\\
381.01	0\\
382.01	0\\
383.01	0\\
384.01	0\\
385.01	0\\
386.01	0\\
387.01	0\\
388.01	0\\
389.01	0\\
390.01	0\\
391.01	0\\
392.01	0\\
393.01	0\\
394.01	0\\
395.01	0\\
396.01	0\\
397.01	0\\
398.01	0\\
399.01	0\\
400.01	0\\
401.01	0\\
402.01	0\\
403.01	0\\
404.01	0\\
405.01	0\\
406.01	0\\
407.01	0\\
408.01	0\\
409.01	0\\
410.01	0\\
411.01	0\\
412.01	0\\
413.01	0\\
414.01	0\\
415.01	0\\
416.01	0\\
417.01	0\\
418.01	0\\
419.01	0\\
420.01	0\\
421.01	0\\
422.01	0\\
423.01	0\\
424.01	0\\
425.01	0\\
426.01	0\\
427.01	0\\
428.01	0\\
429.01	0\\
430.01	0\\
431.01	0\\
432.01	0\\
433.01	0\\
434.01	0\\
435.01	0\\
436.01	0\\
437.01	0\\
438.01	0\\
439.01	0\\
440.01	0\\
441.01	0\\
442.01	0\\
443.01	0\\
444.01	0\\
445.01	0\\
446.01	0\\
447.01	0\\
448.01	0\\
449.01	0\\
450.01	0\\
451.01	0\\
452.01	0\\
453.01	0\\
454.01	0\\
455.01	0\\
456.01	0\\
457.01	0\\
458.01	0\\
459.01	0\\
460.01	0\\
461.01	0\\
462.01	0\\
463.01	0\\
464.01	0\\
465.01	0\\
466.01	0\\
467.01	0\\
468.01	0\\
469.01	0\\
470.01	0\\
471.01	0\\
472.01	0\\
473.01	0\\
474.01	0\\
475.01	0\\
476.01	0\\
477.01	0\\
478.01	0\\
479.01	0\\
480.01	0\\
481.01	0\\
482.01	0\\
483.01	0\\
484.01	0\\
485.01	0\\
486.01	0\\
487.01	0\\
488.01	0\\
489.01	0\\
490.01	0\\
491.01	0\\
492.01	0\\
493.01	0\\
494.01	0\\
495.01	0\\
496.01	0\\
497.01	0\\
498.01	0\\
499.01	0\\
500.01	0\\
501.01	0\\
502.01	0\\
503.01	0\\
504.01	0\\
505.01	0\\
506.01	0\\
507.01	0\\
508.01	0\\
509.01	0\\
510.01	0\\
511.01	0\\
512.01	0\\
513.01	0\\
514.01	0\\
515.01	0\\
516.01	0\\
517.01	0\\
518.01	0\\
519.01	0\\
520.01	0\\
521.01	0\\
522.01	0\\
523.01	0\\
524.01	0\\
525.01	0\\
526.01	0\\
527.01	0\\
528.01	0\\
529.01	0\\
530.01	0\\
531.01	0\\
532.01	0\\
533.01	0\\
534.01	0\\
535.01	0\\
536.01	0\\
537.01	0\\
538.01	0\\
539.01	0\\
540.01	0\\
541.01	0\\
542.01	0\\
543.01	0\\
544.01	0\\
545.01	0\\
546.01	0\\
547.01	0\\
548.01	0\\
549.01	0\\
550.01	0\\
551.01	0\\
552.01	0\\
553.01	0\\
554.01	0\\
555.01	0\\
556.01	0\\
557.01	0\\
558.01	0\\
559.01	0\\
560.01	0\\
561.01	0\\
562.01	0\\
563.01	0\\
564.01	0\\
565.01	0\\
566.01	0\\
567.01	0\\
568.01	0\\
569.01	0\\
570.01	0\\
571.01	0\\
572.01	0\\
573.01	0\\
574.01	0\\
575.01	0\\
576.01	0\\
577.01	0\\
578.01	0\\
579.01	0\\
580.01	0\\
581.01	0\\
582.01	0\\
583.01	0\\
584.01	0\\
585.01	0\\
586.01	0\\
587.01	0\\
588.01	0\\
589.01	0\\
590.01	0\\
591.01	0\\
592.01	0\\
593.01	0\\
594.01	0\\
595.01	0\\
596.01	0\\
597.01	0\\
598.01	0.00134745983720699\\
599.01	0.0037581309059459\\
599.02	0.00379585558757836\\
599.03	0.00383394724377263\\
599.04	0.0038724094789843\\
599.05	0.00391124593307965\\
599.06	0.00395046028168358\\
599.07	0.00399005623653084\\
599.08	0.00403003754582081\\
599.09	0.00407040799457574\\
599.1	0.00411117140500247\\
599.11	0.00415233163685778\\
599.12	0.00419389258781724\\
599.13	0.00423585819384774\\
599.14	0.00427823242958371\\
599.15	0.00432101930870692\\
599.16	0.00436422288433015\\
599.17	0.00440784724938453\\
599.18	0.00445189653701074\\
599.19	0.00449637492095399\\
599.2	0.00454128661596298\\
599.21	0.00458663587819269\\
599.22	0.00463242700561119\\
599.23	0.00467866433841033\\
599.24	0.00472535225942064\\
599.25	0.00477249519453023\\
599.26	0.00482009761310769\\
599.27	0.00486816402842932\\
599.28	0.00491669899811048\\
599.29	0.00496570712454114\\
599.3	0.00501519305532581\\
599.31	0.00506516148372771\\
599.32	0.00511561714911724\\
599.33	0.00516656483742509\\
599.34	0.00521800938159953\\
599.35	0.00526995566206838\\
599.36	0.00532240860720544\\
599.37	0.00537537319380149\\
599.38	0.00542885444753996\\
599.39	0.00548285743950264\\
599.4	0.00553738727623643\\
599.41	0.0055924491142738\\
599.42	0.00564804816062217\\
599.43	0.00570418967325788\\
599.44	0.00576087896162512\\
599.45	0.00581812138713977\\
599.46	0.00587592236369805\\
599.47	0.00593428735819027\\
599.48	0.00599322189101953\\
599.49	0.00605273153662556\\
599.5	0.00611282192401359\\
599.51	0.00617349873728855\\
599.52	0.00623476771619431\\
599.53	0.00629663465665838\\
599.54	0.00635910541134178\\
599.55	0.00642218589019448\\
599.56	0.00648588206101606\\
599.57	0.006550199950022\\
599.58	0.00661514564241551\\
599.59	0.00668072528296489\\
599.6	0.00674694507658658\\
599.61	0.00681381128893396\\
599.62	0.00688133024699177\\
599.63	0.00694950833967656\\
599.64	0.00701835201844282\\
599.65	0.00708786779789517\\
599.66	0.00715806225640647\\
599.67	0.00722894203674201\\
599.68	0.00730051384668986\\
599.69	0.00737278445969725\\
599.7	0.00744576071551335\\
599.71	0.00751944952083819\\
599.72	0.00759385784997808\\
599.73	0.00766899274550727\\
599.74	0.00774486131893627\\
599.75	0.00782147075138655\\
599.76	0.007898828294272\\
599.77	0.00797694126998689\\
599.78	0.00805581707260072\\
599.79	0.00813546316855976\\
599.8	0.00821588709739554\\
599.81	0.00829709647244021\\
599.82	0.00837909898154891\\
599.83	0.00846190238782927\\
599.84	0.008545514530378\\
599.85	0.00862994332502466\\
599.86	0.00871519676508279\\
599.87	0.00880128292210834\\
599.88	0.00888820994666551\\
599.89	0.00897598606910018\\
599.9	0.00906461960032073\\
599.91	0.00915411893258671\\
599.92	0.00924449254030512\\
599.93	0.00933574898083444\\
599.94	0.00942789689529665\\
599.95	0.00952094500939709\\
599.96	0.00961490213425244\\
599.97	0.00970977716722672\\
599.98	0.00980557909277551\\
599.99	0.00990231698329844\\
600	0.01\\
};
\addplot [color=black!50!mycolor20,solid,forget plot]
  table[row sep=crcr]{%
0.01	0\\
1.01	0\\
2.01	0\\
3.01	0\\
4.01	0\\
5.01	0\\
6.01	0\\
7.01	0\\
8.01	0\\
9.01	0\\
10.01	0\\
11.01	0\\
12.01	0\\
13.01	0\\
14.01	0\\
15.01	0\\
16.01	0\\
17.01	0\\
18.01	0\\
19.01	0\\
20.01	0\\
21.01	0\\
22.01	0\\
23.01	0\\
24.01	0\\
25.01	0\\
26.01	0\\
27.01	0\\
28.01	0\\
29.01	0\\
30.01	0\\
31.01	0\\
32.01	0\\
33.01	0\\
34.01	0\\
35.01	0\\
36.01	0\\
37.01	0\\
38.01	0\\
39.01	0\\
40.01	0\\
41.01	0\\
42.01	0\\
43.01	0\\
44.01	0\\
45.01	0\\
46.01	0\\
47.01	0\\
48.01	0\\
49.01	0\\
50.01	0\\
51.01	0\\
52.01	0\\
53.01	0\\
54.01	0\\
55.01	0\\
56.01	0\\
57.01	0\\
58.01	0\\
59.01	0\\
60.01	0\\
61.01	0\\
62.01	0\\
63.01	0\\
64.01	0\\
65.01	0\\
66.01	0\\
67.01	0\\
68.01	0\\
69.01	0\\
70.01	0\\
71.01	0\\
72.01	0\\
73.01	0\\
74.01	0\\
75.01	0\\
76.01	0\\
77.01	0\\
78.01	0\\
79.01	0\\
80.01	0\\
81.01	0\\
82.01	0\\
83.01	0\\
84.01	0\\
85.01	0\\
86.01	0\\
87.01	0\\
88.01	0\\
89.01	0\\
90.01	0\\
91.01	0\\
92.01	0\\
93.01	0\\
94.01	0\\
95.01	0\\
96.01	0\\
97.01	0\\
98.01	0\\
99.01	0\\
100.01	0\\
101.01	0\\
102.01	0\\
103.01	0\\
104.01	0\\
105.01	0\\
106.01	0\\
107.01	0\\
108.01	0\\
109.01	0\\
110.01	0\\
111.01	0\\
112.01	0\\
113.01	0\\
114.01	0\\
115.01	0\\
116.01	0\\
117.01	0\\
118.01	0\\
119.01	0\\
120.01	0\\
121.01	0\\
122.01	0\\
123.01	0\\
124.01	0\\
125.01	0\\
126.01	0\\
127.01	0\\
128.01	0\\
129.01	0\\
130.01	0\\
131.01	0\\
132.01	0\\
133.01	0\\
134.01	0\\
135.01	0\\
136.01	0\\
137.01	0\\
138.01	0\\
139.01	0\\
140.01	0\\
141.01	0\\
142.01	0\\
143.01	0\\
144.01	0\\
145.01	0\\
146.01	0\\
147.01	0\\
148.01	0\\
149.01	0\\
150.01	0\\
151.01	0\\
152.01	0\\
153.01	0\\
154.01	0\\
155.01	0\\
156.01	0\\
157.01	0\\
158.01	0\\
159.01	0\\
160.01	0\\
161.01	0\\
162.01	0\\
163.01	0\\
164.01	0\\
165.01	0\\
166.01	0\\
167.01	0\\
168.01	0\\
169.01	0\\
170.01	0\\
171.01	0\\
172.01	0\\
173.01	0\\
174.01	0\\
175.01	0\\
176.01	0\\
177.01	0\\
178.01	0\\
179.01	0\\
180.01	0\\
181.01	0\\
182.01	0\\
183.01	0\\
184.01	0\\
185.01	0\\
186.01	0\\
187.01	0\\
188.01	0\\
189.01	0\\
190.01	0\\
191.01	0\\
192.01	0\\
193.01	0\\
194.01	0\\
195.01	0\\
196.01	0\\
197.01	0\\
198.01	0\\
199.01	0\\
200.01	0\\
201.01	0\\
202.01	0\\
203.01	0\\
204.01	0\\
205.01	0\\
206.01	0\\
207.01	0\\
208.01	0\\
209.01	0\\
210.01	0\\
211.01	0\\
212.01	0\\
213.01	0\\
214.01	0\\
215.01	0\\
216.01	0\\
217.01	0\\
218.01	0\\
219.01	0\\
220.01	0\\
221.01	0\\
222.01	0\\
223.01	0\\
224.01	0\\
225.01	0\\
226.01	0\\
227.01	0\\
228.01	0\\
229.01	0\\
230.01	0\\
231.01	0\\
232.01	0\\
233.01	0\\
234.01	0\\
235.01	0\\
236.01	0\\
237.01	0\\
238.01	0\\
239.01	0\\
240.01	0\\
241.01	0\\
242.01	0\\
243.01	0\\
244.01	0\\
245.01	0\\
246.01	0\\
247.01	0\\
248.01	0\\
249.01	0\\
250.01	0\\
251.01	0\\
252.01	0\\
253.01	0\\
254.01	0\\
255.01	0\\
256.01	0\\
257.01	0\\
258.01	0\\
259.01	0\\
260.01	0\\
261.01	0\\
262.01	0\\
263.01	0\\
264.01	0\\
265.01	0\\
266.01	0\\
267.01	0\\
268.01	0\\
269.01	0\\
270.01	0\\
271.01	0\\
272.01	0\\
273.01	0\\
274.01	0\\
275.01	0\\
276.01	0\\
277.01	0\\
278.01	0\\
279.01	0\\
280.01	0\\
281.01	0\\
282.01	0\\
283.01	0\\
284.01	0\\
285.01	0\\
286.01	0\\
287.01	0\\
288.01	0\\
289.01	0\\
290.01	0\\
291.01	0\\
292.01	0\\
293.01	0\\
294.01	0\\
295.01	0\\
296.01	0\\
297.01	0\\
298.01	0\\
299.01	0\\
300.01	0\\
301.01	0\\
302.01	0\\
303.01	0\\
304.01	0\\
305.01	0\\
306.01	0\\
307.01	0\\
308.01	0\\
309.01	0\\
310.01	0\\
311.01	0\\
312.01	0\\
313.01	0\\
314.01	0\\
315.01	0\\
316.01	0\\
317.01	0\\
318.01	0\\
319.01	0\\
320.01	0\\
321.01	0\\
322.01	0\\
323.01	0\\
324.01	0\\
325.01	0\\
326.01	0\\
327.01	0\\
328.01	0\\
329.01	0\\
330.01	0\\
331.01	0\\
332.01	0\\
333.01	0\\
334.01	0\\
335.01	0\\
336.01	0\\
337.01	0\\
338.01	0\\
339.01	0\\
340.01	0\\
341.01	0\\
342.01	0\\
343.01	0\\
344.01	0\\
345.01	0\\
346.01	0\\
347.01	0\\
348.01	0\\
349.01	0\\
350.01	0\\
351.01	0\\
352.01	0\\
353.01	0\\
354.01	0\\
355.01	0\\
356.01	0\\
357.01	0\\
358.01	0\\
359.01	0\\
360.01	0\\
361.01	0\\
362.01	0\\
363.01	0\\
364.01	0\\
365.01	0\\
366.01	0\\
367.01	0\\
368.01	0\\
369.01	0\\
370.01	0\\
371.01	0\\
372.01	0\\
373.01	0\\
374.01	0\\
375.01	0\\
376.01	0\\
377.01	0\\
378.01	0\\
379.01	0\\
380.01	0\\
381.01	0\\
382.01	0\\
383.01	0\\
384.01	0\\
385.01	0\\
386.01	0\\
387.01	0\\
388.01	0\\
389.01	0\\
390.01	0\\
391.01	0\\
392.01	0\\
393.01	0\\
394.01	0\\
395.01	0\\
396.01	0\\
397.01	0\\
398.01	0\\
399.01	0\\
400.01	0\\
401.01	0\\
402.01	0\\
403.01	0\\
404.01	0\\
405.01	0\\
406.01	0\\
407.01	0\\
408.01	0\\
409.01	0\\
410.01	0\\
411.01	0\\
412.01	0\\
413.01	0\\
414.01	0\\
415.01	0\\
416.01	0\\
417.01	0\\
418.01	0\\
419.01	0\\
420.01	0\\
421.01	0\\
422.01	0\\
423.01	0\\
424.01	0\\
425.01	0\\
426.01	0\\
427.01	0\\
428.01	0\\
429.01	0\\
430.01	0\\
431.01	0\\
432.01	0\\
433.01	0\\
434.01	0\\
435.01	0\\
436.01	0\\
437.01	0\\
438.01	0\\
439.01	0\\
440.01	0\\
441.01	0\\
442.01	0\\
443.01	0\\
444.01	0\\
445.01	0\\
446.01	0\\
447.01	0\\
448.01	0\\
449.01	0\\
450.01	0\\
451.01	0\\
452.01	0\\
453.01	0\\
454.01	0\\
455.01	0\\
456.01	0\\
457.01	0\\
458.01	0\\
459.01	0\\
460.01	0\\
461.01	0\\
462.01	0\\
463.01	0\\
464.01	0\\
465.01	0\\
466.01	0\\
467.01	0\\
468.01	0\\
469.01	0\\
470.01	0\\
471.01	0\\
472.01	0\\
473.01	0\\
474.01	0\\
475.01	0\\
476.01	0\\
477.01	0\\
478.01	0\\
479.01	0\\
480.01	0\\
481.01	0\\
482.01	0\\
483.01	0\\
484.01	0\\
485.01	0\\
486.01	0\\
487.01	0\\
488.01	0\\
489.01	0\\
490.01	0\\
491.01	0\\
492.01	0\\
493.01	0\\
494.01	0\\
495.01	0\\
496.01	0\\
497.01	0\\
498.01	0\\
499.01	0\\
500.01	0\\
501.01	0\\
502.01	0\\
503.01	0\\
504.01	0\\
505.01	0\\
506.01	0\\
507.01	0\\
508.01	0\\
509.01	0\\
510.01	0\\
511.01	0\\
512.01	0\\
513.01	0\\
514.01	0\\
515.01	0\\
516.01	0\\
517.01	0\\
518.01	0\\
519.01	0\\
520.01	0\\
521.01	0\\
522.01	0\\
523.01	0\\
524.01	0\\
525.01	0\\
526.01	0\\
527.01	0\\
528.01	0\\
529.01	0\\
530.01	0\\
531.01	0\\
532.01	0\\
533.01	0\\
534.01	0\\
535.01	0\\
536.01	0\\
537.01	0\\
538.01	0\\
539.01	0\\
540.01	0\\
541.01	0\\
542.01	0\\
543.01	0\\
544.01	0\\
545.01	0\\
546.01	0\\
547.01	0\\
548.01	0\\
549.01	0\\
550.01	0\\
551.01	0\\
552.01	0\\
553.01	0\\
554.01	0\\
555.01	0\\
556.01	0\\
557.01	0\\
558.01	0\\
559.01	0\\
560.01	0\\
561.01	0\\
562.01	0\\
563.01	0\\
564.01	0\\
565.01	0\\
566.01	0\\
567.01	0\\
568.01	0\\
569.01	0\\
570.01	0\\
571.01	0\\
572.01	0\\
573.01	0\\
574.01	0\\
575.01	0\\
576.01	0\\
577.01	0\\
578.01	0\\
579.01	0\\
580.01	0\\
581.01	0\\
582.01	0\\
583.01	0\\
584.01	0\\
585.01	0\\
586.01	0\\
587.01	0\\
588.01	0\\
589.01	0\\
590.01	0\\
591.01	0\\
592.01	0\\
593.01	0\\
594.01	0\\
595.01	0\\
596.01	0\\
597.01	0\\
598.01	0.00134849457212285\\
599.01	0.0037581309059459\\
599.02	0.00379585558757838\\
599.03	0.00383394724377265\\
599.04	0.0038724094789843\\
599.05	0.00391124593307966\\
599.06	0.0039504602816836\\
599.07	0.00399005623653084\\
599.08	0.00403003754582082\\
599.09	0.00407040799457574\\
599.1	0.00411117140500247\\
599.11	0.00415233163685778\\
599.12	0.00419389258781723\\
599.13	0.00423585819384774\\
599.14	0.00427823242958369\\
599.15	0.0043210193087069\\
599.16	0.00436422288433012\\
599.17	0.00440784724938451\\
599.18	0.00445189653701071\\
599.19	0.00449637492095396\\
599.2	0.00454128661596297\\
599.21	0.00458663587819269\\
599.22	0.00463242700561117\\
599.23	0.00467866433841031\\
599.24	0.00472535225942064\\
599.25	0.00477249519453023\\
599.26	0.00482009761310769\\
599.27	0.00486816402842931\\
599.28	0.00491669899811045\\
599.29	0.00496570712454113\\
599.3	0.00501519305532581\\
599.31	0.0050651614837277\\
599.32	0.00511561714911724\\
599.33	0.00516656483742509\\
599.34	0.00521800938159954\\
599.35	0.00526995566206841\\
599.36	0.00532240860720548\\
599.37	0.00537537319380152\\
599.38	0.00542885444753999\\
599.39	0.00548285743950268\\
599.4	0.00553738727623646\\
599.41	0.00559244911427385\\
599.42	0.00564804816062223\\
599.43	0.00570418967325793\\
599.44	0.00576087896162518\\
599.45	0.00581812138713983\\
599.46	0.00587592236369812\\
599.47	0.00593428735819033\\
599.48	0.0059932218910196\\
599.49	0.00605273153662562\\
599.5	0.00611282192401366\\
599.51	0.00617349873728862\\
599.52	0.00623476771619437\\
599.53	0.00629663465665842\\
599.54	0.00635910541134183\\
599.55	0.00642218589019451\\
599.56	0.00648588206101608\\
599.57	0.00655019995002201\\
599.58	0.00661514564241553\\
599.59	0.00668072528296489\\
599.6	0.0067469450765866\\
599.61	0.00681381128893397\\
599.62	0.00688133024699178\\
599.63	0.00694950833967657\\
599.64	0.00701835201844284\\
599.65	0.00708786779789519\\
599.66	0.00715806225640649\\
599.67	0.00722894203674203\\
599.68	0.00730051384668987\\
599.69	0.00737278445969726\\
599.7	0.00744576071551335\\
599.71	0.0075194495208382\\
599.72	0.00759385784997809\\
599.73	0.00766899274550729\\
599.74	0.00774486131893628\\
599.75	0.00782147075138657\\
599.76	0.00789882829427201\\
599.77	0.0079769412699869\\
599.78	0.00805581707260073\\
599.79	0.00813546316855977\\
599.8	0.00821588709739555\\
599.81	0.00829709647244021\\
599.82	0.00837909898154891\\
599.83	0.00846190238782928\\
599.84	0.008545514530378\\
599.85	0.00862994332502466\\
599.86	0.0087151967650828\\
599.87	0.00880128292210834\\
599.88	0.00888820994666552\\
599.89	0.00897598606910018\\
599.9	0.00906461960032073\\
599.91	0.00915411893258672\\
599.92	0.00924449254030512\\
599.93	0.00933574898083444\\
599.94	0.00942789689529664\\
599.95	0.00952094500939709\\
599.96	0.00961490213425244\\
599.97	0.00970977716722672\\
599.98	0.00980557909277551\\
599.99	0.00990231698329844\\
600	0.01\\
};
\addplot [color=black!60!mycolor21,solid,forget plot]
  table[row sep=crcr]{%
0.01	0\\
1.01	0\\
2.01	0\\
3.01	0\\
4.01	0\\
5.01	0\\
6.01	0\\
7.01	0\\
8.01	0\\
9.01	0\\
10.01	0\\
11.01	0\\
12.01	0\\
13.01	0\\
14.01	0\\
15.01	0\\
16.01	0\\
17.01	0\\
18.01	0\\
19.01	0\\
20.01	0\\
21.01	0\\
22.01	0\\
23.01	0\\
24.01	0\\
25.01	0\\
26.01	0\\
27.01	0\\
28.01	0\\
29.01	0\\
30.01	0\\
31.01	0\\
32.01	0\\
33.01	0\\
34.01	0\\
35.01	0\\
36.01	0\\
37.01	0\\
38.01	0\\
39.01	0\\
40.01	0\\
41.01	0\\
42.01	0\\
43.01	0\\
44.01	0\\
45.01	0\\
46.01	0\\
47.01	0\\
48.01	0\\
49.01	0\\
50.01	0\\
51.01	0\\
52.01	0\\
53.01	0\\
54.01	0\\
55.01	0\\
56.01	0\\
57.01	0\\
58.01	0\\
59.01	0\\
60.01	0\\
61.01	0\\
62.01	0\\
63.01	0\\
64.01	0\\
65.01	0\\
66.01	0\\
67.01	0\\
68.01	0\\
69.01	0\\
70.01	0\\
71.01	0\\
72.01	0\\
73.01	0\\
74.01	0\\
75.01	0\\
76.01	0\\
77.01	0\\
78.01	0\\
79.01	0\\
80.01	0\\
81.01	0\\
82.01	0\\
83.01	0\\
84.01	0\\
85.01	0\\
86.01	0\\
87.01	0\\
88.01	0\\
89.01	0\\
90.01	0\\
91.01	0\\
92.01	0\\
93.01	0\\
94.01	0\\
95.01	0\\
96.01	0\\
97.01	0\\
98.01	0\\
99.01	0\\
100.01	0\\
101.01	0\\
102.01	0\\
103.01	0\\
104.01	0\\
105.01	0\\
106.01	0\\
107.01	0\\
108.01	0\\
109.01	0\\
110.01	0\\
111.01	0\\
112.01	0\\
113.01	0\\
114.01	0\\
115.01	0\\
116.01	0\\
117.01	0\\
118.01	0\\
119.01	0\\
120.01	0\\
121.01	0\\
122.01	0\\
123.01	0\\
124.01	0\\
125.01	0\\
126.01	0\\
127.01	0\\
128.01	0\\
129.01	0\\
130.01	0\\
131.01	0\\
132.01	0\\
133.01	0\\
134.01	0\\
135.01	0\\
136.01	0\\
137.01	0\\
138.01	0\\
139.01	0\\
140.01	0\\
141.01	0\\
142.01	0\\
143.01	0\\
144.01	0\\
145.01	0\\
146.01	0\\
147.01	0\\
148.01	0\\
149.01	0\\
150.01	0\\
151.01	0\\
152.01	0\\
153.01	0\\
154.01	0\\
155.01	0\\
156.01	0\\
157.01	0\\
158.01	0\\
159.01	0\\
160.01	0\\
161.01	0\\
162.01	0\\
163.01	0\\
164.01	0\\
165.01	0\\
166.01	0\\
167.01	0\\
168.01	0\\
169.01	0\\
170.01	0\\
171.01	0\\
172.01	0\\
173.01	0\\
174.01	0\\
175.01	0\\
176.01	0\\
177.01	0\\
178.01	0\\
179.01	0\\
180.01	0\\
181.01	0\\
182.01	0\\
183.01	0\\
184.01	0\\
185.01	0\\
186.01	0\\
187.01	0\\
188.01	0\\
189.01	0\\
190.01	0\\
191.01	0\\
192.01	0\\
193.01	0\\
194.01	0\\
195.01	0\\
196.01	0\\
197.01	0\\
198.01	0\\
199.01	0\\
200.01	0\\
201.01	0\\
202.01	0\\
203.01	0\\
204.01	0\\
205.01	0\\
206.01	0\\
207.01	0\\
208.01	0\\
209.01	0\\
210.01	0\\
211.01	0\\
212.01	0\\
213.01	0\\
214.01	0\\
215.01	0\\
216.01	0\\
217.01	0\\
218.01	0\\
219.01	0\\
220.01	0\\
221.01	0\\
222.01	0\\
223.01	0\\
224.01	0\\
225.01	0\\
226.01	0\\
227.01	0\\
228.01	0\\
229.01	0\\
230.01	0\\
231.01	0\\
232.01	0\\
233.01	0\\
234.01	0\\
235.01	0\\
236.01	0\\
237.01	0\\
238.01	0\\
239.01	0\\
240.01	0\\
241.01	0\\
242.01	0\\
243.01	0\\
244.01	0\\
245.01	0\\
246.01	0\\
247.01	0\\
248.01	0\\
249.01	0\\
250.01	0\\
251.01	0\\
252.01	0\\
253.01	0\\
254.01	0\\
255.01	0\\
256.01	0\\
257.01	0\\
258.01	0\\
259.01	0\\
260.01	0\\
261.01	0\\
262.01	0\\
263.01	0\\
264.01	0\\
265.01	0\\
266.01	0\\
267.01	0\\
268.01	0\\
269.01	0\\
270.01	0\\
271.01	0\\
272.01	0\\
273.01	0\\
274.01	0\\
275.01	0\\
276.01	0\\
277.01	0\\
278.01	0\\
279.01	0\\
280.01	0\\
281.01	0\\
282.01	0\\
283.01	0\\
284.01	0\\
285.01	0\\
286.01	0\\
287.01	0\\
288.01	0\\
289.01	0\\
290.01	0\\
291.01	0\\
292.01	0\\
293.01	0\\
294.01	0\\
295.01	0\\
296.01	0\\
297.01	0\\
298.01	0\\
299.01	0\\
300.01	0\\
301.01	0\\
302.01	0\\
303.01	0\\
304.01	0\\
305.01	0\\
306.01	0\\
307.01	0\\
308.01	0\\
309.01	0\\
310.01	0\\
311.01	0\\
312.01	0\\
313.01	0\\
314.01	0\\
315.01	0\\
316.01	0\\
317.01	0\\
318.01	0\\
319.01	0\\
320.01	0\\
321.01	0\\
322.01	0\\
323.01	0\\
324.01	0\\
325.01	0\\
326.01	0\\
327.01	0\\
328.01	0\\
329.01	0\\
330.01	0\\
331.01	0\\
332.01	0\\
333.01	0\\
334.01	0\\
335.01	0\\
336.01	0\\
337.01	0\\
338.01	0\\
339.01	0\\
340.01	0\\
341.01	0\\
342.01	0\\
343.01	0\\
344.01	0\\
345.01	0\\
346.01	0\\
347.01	0\\
348.01	0\\
349.01	0\\
350.01	0\\
351.01	0\\
352.01	0\\
353.01	0\\
354.01	0\\
355.01	0\\
356.01	0\\
357.01	0\\
358.01	0\\
359.01	0\\
360.01	0\\
361.01	0\\
362.01	0\\
363.01	0\\
364.01	0\\
365.01	0\\
366.01	0\\
367.01	0\\
368.01	0\\
369.01	0\\
370.01	0\\
371.01	0\\
372.01	0\\
373.01	0\\
374.01	0\\
375.01	0\\
376.01	0\\
377.01	0\\
378.01	0\\
379.01	0\\
380.01	0\\
381.01	0\\
382.01	0\\
383.01	0\\
384.01	0\\
385.01	0\\
386.01	0\\
387.01	0\\
388.01	0\\
389.01	0\\
390.01	0\\
391.01	0\\
392.01	0\\
393.01	0\\
394.01	0\\
395.01	0\\
396.01	0\\
397.01	0\\
398.01	0\\
399.01	0\\
400.01	0\\
401.01	0\\
402.01	0\\
403.01	0\\
404.01	0\\
405.01	0\\
406.01	0\\
407.01	0\\
408.01	0\\
409.01	0\\
410.01	0\\
411.01	0\\
412.01	0\\
413.01	0\\
414.01	0\\
415.01	0\\
416.01	0\\
417.01	0\\
418.01	0\\
419.01	0\\
420.01	0\\
421.01	0\\
422.01	0\\
423.01	0\\
424.01	0\\
425.01	0\\
426.01	0\\
427.01	0\\
428.01	0\\
429.01	0\\
430.01	0\\
431.01	0\\
432.01	0\\
433.01	0\\
434.01	0\\
435.01	0\\
436.01	0\\
437.01	0\\
438.01	0\\
439.01	0\\
440.01	0\\
441.01	0\\
442.01	0\\
443.01	0\\
444.01	0\\
445.01	0\\
446.01	0\\
447.01	0\\
448.01	0\\
449.01	0\\
450.01	0\\
451.01	0\\
452.01	0\\
453.01	0\\
454.01	0\\
455.01	0\\
456.01	0\\
457.01	0\\
458.01	0\\
459.01	0\\
460.01	0\\
461.01	0\\
462.01	0\\
463.01	0\\
464.01	0\\
465.01	0\\
466.01	0\\
467.01	0\\
468.01	0\\
469.01	0\\
470.01	0\\
471.01	0\\
472.01	0\\
473.01	0\\
474.01	0\\
475.01	0\\
476.01	0\\
477.01	0\\
478.01	0\\
479.01	0\\
480.01	0\\
481.01	0\\
482.01	0\\
483.01	0\\
484.01	0\\
485.01	0\\
486.01	0\\
487.01	0\\
488.01	0\\
489.01	0\\
490.01	0\\
491.01	0\\
492.01	0\\
493.01	0\\
494.01	0\\
495.01	0\\
496.01	0\\
497.01	0\\
498.01	0\\
499.01	0\\
500.01	0\\
501.01	0\\
502.01	0\\
503.01	0\\
504.01	0\\
505.01	0\\
506.01	0\\
507.01	0\\
508.01	0\\
509.01	0\\
510.01	0\\
511.01	0\\
512.01	0\\
513.01	0\\
514.01	0\\
515.01	0\\
516.01	0\\
517.01	0\\
518.01	0\\
519.01	0\\
520.01	0\\
521.01	0\\
522.01	0\\
523.01	0\\
524.01	0\\
525.01	0\\
526.01	0\\
527.01	0\\
528.01	0\\
529.01	0\\
530.01	0\\
531.01	0\\
532.01	0\\
533.01	0\\
534.01	0\\
535.01	0\\
536.01	0\\
537.01	0\\
538.01	0\\
539.01	0\\
540.01	0\\
541.01	0\\
542.01	0\\
543.01	0\\
544.01	0\\
545.01	0\\
546.01	0\\
547.01	0\\
548.01	0\\
549.01	0\\
550.01	0\\
551.01	0\\
552.01	0\\
553.01	0\\
554.01	0\\
555.01	0\\
556.01	0\\
557.01	0\\
558.01	0\\
559.01	0\\
560.01	0\\
561.01	0\\
562.01	0\\
563.01	0\\
564.01	0\\
565.01	0\\
566.01	0\\
567.01	0\\
568.01	0\\
569.01	0\\
570.01	0\\
571.01	0\\
572.01	0\\
573.01	0\\
574.01	0\\
575.01	0\\
576.01	0\\
577.01	0\\
578.01	0\\
579.01	0\\
580.01	0\\
581.01	0\\
582.01	0\\
583.01	0\\
584.01	0\\
585.01	0\\
586.01	0\\
587.01	0\\
588.01	0\\
589.01	0\\
590.01	0\\
591.01	0\\
592.01	0\\
593.01	0\\
594.01	0\\
595.01	0\\
596.01	0\\
597.01	0\\
598.01	0.00134872349121326\\
599.01	0.0037581309059459\\
599.02	0.00379585558757838\\
599.03	0.00383394724377262\\
599.04	0.00387240947898428\\
599.05	0.00391124593307964\\
599.06	0.00395046028168355\\
599.07	0.00399005623653081\\
599.08	0.00403003754582079\\
599.09	0.00407040799457573\\
599.1	0.00411117140500246\\
599.11	0.00415233163685777\\
599.12	0.00419389258781723\\
599.13	0.00423585819384774\\
599.14	0.00427823242958371\\
599.15	0.00432101930870692\\
599.16	0.00436422288433015\\
599.17	0.00440784724938453\\
599.18	0.00445189653701072\\
599.19	0.00449637492095398\\
599.2	0.00454128661596297\\
599.21	0.00458663587819268\\
599.22	0.00463242700561116\\
599.23	0.0046786643384103\\
599.24	0.00472535225942063\\
599.25	0.0047724951945302\\
599.26	0.00482009761310764\\
599.27	0.00486816402842928\\
599.28	0.00491669899811044\\
599.29	0.00496570712454111\\
599.3	0.00501519305532579\\
599.31	0.00506516148372767\\
599.32	0.00511561714911721\\
599.33	0.00516656483742506\\
599.34	0.0052180093815995\\
599.35	0.00526995566206835\\
599.36	0.00532240860720541\\
599.37	0.00537537319380146\\
599.38	0.00542885444753993\\
599.39	0.00548285743950261\\
599.4	0.00553738727623639\\
599.41	0.00559244911427376\\
599.42	0.00564804816062213\\
599.43	0.00570418967325784\\
599.44	0.00576087896162508\\
599.45	0.00581812138713973\\
599.46	0.00587592236369801\\
599.47	0.00593428735819024\\
599.48	0.0059932218910195\\
599.49	0.00605273153662553\\
599.5	0.00611282192401356\\
599.51	0.00617349873728851\\
599.52	0.00623476771619427\\
599.53	0.00629663465665833\\
599.54	0.00635910541134176\\
599.55	0.00642218589019446\\
599.56	0.00648588206101604\\
599.57	0.00655019995002198\\
599.58	0.00661514564241548\\
599.59	0.00668072528296487\\
599.6	0.00674694507658657\\
599.61	0.00681381128893394\\
599.62	0.00688133024699176\\
599.63	0.00694950833967655\\
599.64	0.00701835201844282\\
599.65	0.00708786779789516\\
599.66	0.00715806225640646\\
599.67	0.00722894203674201\\
599.68	0.00730051384668985\\
599.69	0.00737278445969725\\
599.7	0.00744576071551335\\
599.71	0.0075194495208382\\
599.72	0.00759385784997808\\
599.73	0.00766899274550727\\
599.74	0.00774486131893627\\
599.75	0.00782147075138656\\
599.76	0.00789882829427201\\
599.77	0.0079769412699869\\
599.78	0.00805581707260072\\
599.79	0.00813546316855976\\
599.8	0.00821588709739554\\
599.81	0.0082970964724402\\
599.82	0.00837909898154891\\
599.83	0.00846190238782927\\
599.84	0.008545514530378\\
599.85	0.00862994332502466\\
599.86	0.00871519676508279\\
599.87	0.00880128292210833\\
599.88	0.00888820994666551\\
599.89	0.00897598606910018\\
599.9	0.00906461960032073\\
599.91	0.00915411893258671\\
599.92	0.00924449254030512\\
599.93	0.00933574898083444\\
599.94	0.00942789689529664\\
599.95	0.00952094500939709\\
599.96	0.00961490213425245\\
599.97	0.00970977716722672\\
599.98	0.00980557909277551\\
599.99	0.00990231698329843\\
600	0.01\\
};
\addplot [color=black!80!mycolor21,solid,forget plot]
  table[row sep=crcr]{%
0.01	0\\
1.01	0\\
2.01	0\\
3.01	0\\
4.01	0\\
5.01	0\\
6.01	0\\
7.01	0\\
8.01	0\\
9.01	0\\
10.01	0\\
11.01	0\\
12.01	0\\
13.01	0\\
14.01	0\\
15.01	0\\
16.01	0\\
17.01	0\\
18.01	0\\
19.01	0\\
20.01	0\\
21.01	0\\
22.01	0\\
23.01	0\\
24.01	0\\
25.01	0\\
26.01	0\\
27.01	0\\
28.01	0\\
29.01	0\\
30.01	0\\
31.01	0\\
32.01	0\\
33.01	0\\
34.01	0\\
35.01	0\\
36.01	0\\
37.01	0\\
38.01	0\\
39.01	0\\
40.01	0\\
41.01	0\\
42.01	0\\
43.01	0\\
44.01	0\\
45.01	0\\
46.01	0\\
47.01	0\\
48.01	0\\
49.01	0\\
50.01	0\\
51.01	0\\
52.01	0\\
53.01	0\\
54.01	0\\
55.01	0\\
56.01	0\\
57.01	0\\
58.01	0\\
59.01	0\\
60.01	0\\
61.01	0\\
62.01	0\\
63.01	0\\
64.01	0\\
65.01	0\\
66.01	0\\
67.01	0\\
68.01	0\\
69.01	0\\
70.01	0\\
71.01	0\\
72.01	0\\
73.01	0\\
74.01	0\\
75.01	0\\
76.01	0\\
77.01	0\\
78.01	0\\
79.01	0\\
80.01	0\\
81.01	0\\
82.01	0\\
83.01	0\\
84.01	0\\
85.01	0\\
86.01	0\\
87.01	0\\
88.01	0\\
89.01	0\\
90.01	0\\
91.01	0\\
92.01	0\\
93.01	0\\
94.01	0\\
95.01	0\\
96.01	0\\
97.01	0\\
98.01	0\\
99.01	0\\
100.01	0\\
101.01	0\\
102.01	0\\
103.01	0\\
104.01	0\\
105.01	0\\
106.01	0\\
107.01	0\\
108.01	0\\
109.01	0\\
110.01	0\\
111.01	0\\
112.01	0\\
113.01	0\\
114.01	0\\
115.01	0\\
116.01	0\\
117.01	0\\
118.01	0\\
119.01	0\\
120.01	0\\
121.01	0\\
122.01	0\\
123.01	0\\
124.01	0\\
125.01	0\\
126.01	0\\
127.01	0\\
128.01	0\\
129.01	0\\
130.01	0\\
131.01	0\\
132.01	0\\
133.01	0\\
134.01	0\\
135.01	0\\
136.01	0\\
137.01	0\\
138.01	0\\
139.01	0\\
140.01	0\\
141.01	0\\
142.01	0\\
143.01	0\\
144.01	0\\
145.01	0\\
146.01	0\\
147.01	0\\
148.01	0\\
149.01	0\\
150.01	0\\
151.01	0\\
152.01	0\\
153.01	0\\
154.01	0\\
155.01	0\\
156.01	0\\
157.01	0\\
158.01	0\\
159.01	0\\
160.01	0\\
161.01	0\\
162.01	0\\
163.01	0\\
164.01	0\\
165.01	0\\
166.01	0\\
167.01	0\\
168.01	0\\
169.01	0\\
170.01	0\\
171.01	0\\
172.01	0\\
173.01	0\\
174.01	0\\
175.01	0\\
176.01	0\\
177.01	0\\
178.01	0\\
179.01	0\\
180.01	0\\
181.01	0\\
182.01	0\\
183.01	0\\
184.01	0\\
185.01	0\\
186.01	0\\
187.01	0\\
188.01	0\\
189.01	0\\
190.01	0\\
191.01	0\\
192.01	0\\
193.01	0\\
194.01	0\\
195.01	0\\
196.01	0\\
197.01	0\\
198.01	0\\
199.01	0\\
200.01	0\\
201.01	0\\
202.01	0\\
203.01	0\\
204.01	0\\
205.01	0\\
206.01	0\\
207.01	0\\
208.01	0\\
209.01	0\\
210.01	0\\
211.01	0\\
212.01	0\\
213.01	0\\
214.01	0\\
215.01	0\\
216.01	0\\
217.01	0\\
218.01	0\\
219.01	0\\
220.01	0\\
221.01	0\\
222.01	0\\
223.01	0\\
224.01	0\\
225.01	0\\
226.01	0\\
227.01	0\\
228.01	0\\
229.01	0\\
230.01	0\\
231.01	0\\
232.01	0\\
233.01	0\\
234.01	0\\
235.01	0\\
236.01	0\\
237.01	0\\
238.01	0\\
239.01	0\\
240.01	0\\
241.01	0\\
242.01	0\\
243.01	0\\
244.01	0\\
245.01	0\\
246.01	0\\
247.01	0\\
248.01	0\\
249.01	0\\
250.01	0\\
251.01	0\\
252.01	0\\
253.01	0\\
254.01	0\\
255.01	0\\
256.01	0\\
257.01	0\\
258.01	0\\
259.01	0\\
260.01	0\\
261.01	0\\
262.01	0\\
263.01	0\\
264.01	0\\
265.01	0\\
266.01	0\\
267.01	0\\
268.01	0\\
269.01	0\\
270.01	0\\
271.01	0\\
272.01	0\\
273.01	0\\
274.01	0\\
275.01	0\\
276.01	0\\
277.01	0\\
278.01	0\\
279.01	0\\
280.01	0\\
281.01	0\\
282.01	0\\
283.01	0\\
284.01	0\\
285.01	0\\
286.01	0\\
287.01	0\\
288.01	0\\
289.01	0\\
290.01	0\\
291.01	0\\
292.01	0\\
293.01	0\\
294.01	0\\
295.01	0\\
296.01	0\\
297.01	0\\
298.01	0\\
299.01	0\\
300.01	0\\
301.01	0\\
302.01	0\\
303.01	0\\
304.01	0\\
305.01	0\\
306.01	0\\
307.01	0\\
308.01	0\\
309.01	0\\
310.01	0\\
311.01	0\\
312.01	0\\
313.01	0\\
314.01	0\\
315.01	0\\
316.01	0\\
317.01	0\\
318.01	0\\
319.01	0\\
320.01	0\\
321.01	0\\
322.01	0\\
323.01	0\\
324.01	0\\
325.01	0\\
326.01	0\\
327.01	0\\
328.01	0\\
329.01	0\\
330.01	0\\
331.01	0\\
332.01	0\\
333.01	0\\
334.01	0\\
335.01	0\\
336.01	0\\
337.01	0\\
338.01	0\\
339.01	0\\
340.01	0\\
341.01	0\\
342.01	0\\
343.01	0\\
344.01	0\\
345.01	0\\
346.01	0\\
347.01	0\\
348.01	0\\
349.01	0\\
350.01	0\\
351.01	0\\
352.01	0\\
353.01	0\\
354.01	0\\
355.01	0\\
356.01	0\\
357.01	0\\
358.01	0\\
359.01	0\\
360.01	0\\
361.01	0\\
362.01	0\\
363.01	0\\
364.01	0\\
365.01	0\\
366.01	0\\
367.01	0\\
368.01	0\\
369.01	0\\
370.01	0\\
371.01	0\\
372.01	0\\
373.01	0\\
374.01	0\\
375.01	0\\
376.01	0\\
377.01	0\\
378.01	0\\
379.01	0\\
380.01	0\\
381.01	0\\
382.01	0\\
383.01	0\\
384.01	0\\
385.01	0\\
386.01	0\\
387.01	0\\
388.01	0\\
389.01	0\\
390.01	0\\
391.01	0\\
392.01	0\\
393.01	0\\
394.01	0\\
395.01	0\\
396.01	0\\
397.01	0\\
398.01	0\\
399.01	0\\
400.01	0\\
401.01	0\\
402.01	0\\
403.01	0\\
404.01	0\\
405.01	0\\
406.01	0\\
407.01	0\\
408.01	0\\
409.01	0\\
410.01	0\\
411.01	0\\
412.01	0\\
413.01	0\\
414.01	0\\
415.01	0\\
416.01	0\\
417.01	0\\
418.01	0\\
419.01	0\\
420.01	0\\
421.01	0\\
422.01	0\\
423.01	0\\
424.01	0\\
425.01	0\\
426.01	0\\
427.01	0\\
428.01	0\\
429.01	0\\
430.01	0\\
431.01	0\\
432.01	0\\
433.01	0\\
434.01	0\\
435.01	0\\
436.01	0\\
437.01	0\\
438.01	0\\
439.01	0\\
440.01	0\\
441.01	0\\
442.01	0\\
443.01	0\\
444.01	0\\
445.01	0\\
446.01	0\\
447.01	0\\
448.01	0\\
449.01	0\\
450.01	0\\
451.01	0\\
452.01	0\\
453.01	0\\
454.01	0\\
455.01	0\\
456.01	0\\
457.01	0\\
458.01	0\\
459.01	0\\
460.01	0\\
461.01	0\\
462.01	0\\
463.01	0\\
464.01	0\\
465.01	0\\
466.01	0\\
467.01	0\\
468.01	0\\
469.01	0\\
470.01	0\\
471.01	0\\
472.01	0\\
473.01	0\\
474.01	0\\
475.01	0\\
476.01	0\\
477.01	0\\
478.01	0\\
479.01	0\\
480.01	0\\
481.01	0\\
482.01	0\\
483.01	0\\
484.01	0\\
485.01	0\\
486.01	0\\
487.01	0\\
488.01	0\\
489.01	0\\
490.01	0\\
491.01	0\\
492.01	0\\
493.01	0\\
494.01	0\\
495.01	0\\
496.01	0\\
497.01	0\\
498.01	0\\
499.01	0\\
500.01	0\\
501.01	0\\
502.01	0\\
503.01	0\\
504.01	0\\
505.01	0\\
506.01	0\\
507.01	0\\
508.01	0\\
509.01	0\\
510.01	0\\
511.01	0\\
512.01	0\\
513.01	0\\
514.01	0\\
515.01	0\\
516.01	0\\
517.01	0\\
518.01	0\\
519.01	0\\
520.01	0\\
521.01	0\\
522.01	0\\
523.01	0\\
524.01	0\\
525.01	0\\
526.01	0\\
527.01	0\\
528.01	0\\
529.01	0\\
530.01	0\\
531.01	0\\
532.01	0\\
533.01	0\\
534.01	0\\
535.01	0\\
536.01	0\\
537.01	0\\
538.01	0\\
539.01	0\\
540.01	0\\
541.01	0\\
542.01	0\\
543.01	0\\
544.01	0\\
545.01	0\\
546.01	0\\
547.01	0\\
548.01	0\\
549.01	0\\
550.01	0\\
551.01	0\\
552.01	0\\
553.01	0\\
554.01	0\\
555.01	0\\
556.01	0\\
557.01	0\\
558.01	0\\
559.01	0\\
560.01	0\\
561.01	0\\
562.01	0\\
563.01	0\\
564.01	0\\
565.01	0\\
566.01	0\\
567.01	0\\
568.01	0\\
569.01	0\\
570.01	0\\
571.01	0\\
572.01	0\\
573.01	0\\
574.01	0\\
575.01	0\\
576.01	0\\
577.01	0\\
578.01	0\\
579.01	0\\
580.01	0\\
581.01	0\\
582.01	0\\
583.01	0\\
584.01	0\\
585.01	0\\
586.01	0\\
587.01	0\\
588.01	0\\
589.01	0\\
590.01	0\\
591.01	0\\
592.01	0\\
593.01	0\\
594.01	0\\
595.01	0\\
596.01	0\\
597.01	0\\
598.01	0.00134893218521348\\
599.01	0.0037581309059459\\
599.02	0.00379585558757836\\
599.03	0.00383394724377265\\
599.04	0.00387240947898429\\
599.05	0.00391124593307965\\
599.06	0.00395046028168355\\
599.07	0.00399005623653083\\
599.08	0.00403003754582081\\
599.09	0.00407040799457575\\
599.1	0.00411117140500249\\
599.11	0.0041523316368578\\
599.12	0.00419389258781726\\
599.13	0.00423585819384777\\
599.14	0.00427823242958372\\
599.15	0.00432101930870693\\
599.16	0.00436422288433015\\
599.17	0.00440784724938453\\
599.18	0.00445189653701074\\
599.19	0.00449637492095398\\
599.2	0.00454128661596297\\
599.21	0.00458663587819269\\
599.22	0.00463242700561119\\
599.23	0.00467866433841033\\
599.24	0.00472535225942065\\
599.25	0.00477249519453024\\
599.26	0.0048200976131077\\
599.27	0.00486816402842932\\
599.28	0.00491669899811048\\
599.29	0.00496570712454114\\
599.3	0.00501519305532583\\
599.31	0.00506516148372771\\
599.32	0.00511561714911724\\
599.33	0.00516656483742507\\
599.34	0.00521800938159953\\
599.35	0.00526995566206839\\
599.36	0.00532240860720545\\
599.37	0.00537537319380151\\
599.38	0.00542885444753997\\
599.39	0.00548285743950265\\
599.4	0.00553738727623644\\
599.41	0.00559244911427383\\
599.42	0.0056480481606222\\
599.43	0.00570418967325791\\
599.44	0.00576087896162515\\
599.45	0.0058181213871398\\
599.46	0.00587592236369808\\
599.47	0.00593428735819029\\
599.48	0.00599322189101957\\
599.49	0.00605273153662558\\
599.5	0.00611282192401361\\
599.51	0.00617349873728856\\
599.52	0.00623476771619431\\
599.53	0.00629663465665838\\
599.54	0.00635910541134178\\
599.55	0.00642218589019448\\
599.56	0.00648588206101605\\
599.57	0.00655019995002198\\
599.58	0.0066151456424155\\
599.59	0.00668072528296487\\
599.6	0.00674694507658657\\
599.61	0.00681381128893395\\
599.62	0.00688133024699177\\
599.63	0.00694950833967656\\
599.64	0.00701835201844282\\
599.65	0.00708786779789516\\
599.66	0.00715806225640645\\
599.67	0.00722894203674201\\
599.68	0.00730051384668984\\
599.69	0.00737278445969725\\
599.7	0.00744576071551334\\
599.71	0.00751944952083819\\
599.72	0.00759385784997808\\
599.73	0.00766899274550727\\
599.74	0.00774486131893627\\
599.75	0.00782147075138656\\
599.76	0.00789882829427201\\
599.77	0.0079769412699869\\
599.78	0.00805581707260072\\
599.79	0.00813546316855977\\
599.8	0.00821588709739555\\
599.81	0.00829709647244022\\
599.82	0.00837909898154891\\
599.83	0.00846190238782928\\
599.84	0.008545514530378\\
599.85	0.00862994332502466\\
599.86	0.00871519676508279\\
599.87	0.00880128292210834\\
599.88	0.00888820994666552\\
599.89	0.00897598606910018\\
599.9	0.00906461960032073\\
599.91	0.00915411893258672\\
599.92	0.00924449254030512\\
599.93	0.00933574898083444\\
599.94	0.00942789689529664\\
599.95	0.00952094500939709\\
599.96	0.00961490213425244\\
599.97	0.00970977716722672\\
599.98	0.00980557909277551\\
599.99	0.00990231698329844\\
600	0.01\\
};
\addplot [color=black,solid,forget plot]
  table[row sep=crcr]{%
0.01	0\\
1.01	0\\
2.01	0\\
3.01	0\\
4.01	0\\
5.01	0\\
6.01	0\\
7.01	0\\
8.01	0\\
9.01	0\\
10.01	0\\
11.01	0\\
12.01	0\\
13.01	0\\
14.01	0\\
15.01	0\\
16.01	0\\
17.01	0\\
18.01	0\\
19.01	0\\
20.01	0\\
21.01	0\\
22.01	0\\
23.01	0\\
24.01	0\\
25.01	0\\
26.01	0\\
27.01	0\\
28.01	0\\
29.01	0\\
30.01	0\\
31.01	0\\
32.01	0\\
33.01	0\\
34.01	0\\
35.01	0\\
36.01	0\\
37.01	0\\
38.01	0\\
39.01	0\\
40.01	0\\
41.01	0\\
42.01	0\\
43.01	0\\
44.01	0\\
45.01	0\\
46.01	0\\
47.01	0\\
48.01	0\\
49.01	0\\
50.01	0\\
51.01	0\\
52.01	0\\
53.01	0\\
54.01	0\\
55.01	0\\
56.01	0\\
57.01	0\\
58.01	0\\
59.01	0\\
60.01	0\\
61.01	0\\
62.01	0\\
63.01	0\\
64.01	0\\
65.01	0\\
66.01	0\\
67.01	0\\
68.01	0\\
69.01	0\\
70.01	0\\
71.01	0\\
72.01	0\\
73.01	0\\
74.01	0\\
75.01	0\\
76.01	0\\
77.01	0\\
78.01	0\\
79.01	0\\
80.01	0\\
81.01	0\\
82.01	0\\
83.01	0\\
84.01	0\\
85.01	0\\
86.01	0\\
87.01	0\\
88.01	0\\
89.01	0\\
90.01	0\\
91.01	0\\
92.01	0\\
93.01	0\\
94.01	0\\
95.01	0\\
96.01	0\\
97.01	0\\
98.01	0\\
99.01	0\\
100.01	0\\
101.01	0\\
102.01	0\\
103.01	0\\
104.01	0\\
105.01	0\\
106.01	0\\
107.01	0\\
108.01	0\\
109.01	0\\
110.01	0\\
111.01	0\\
112.01	0\\
113.01	0\\
114.01	0\\
115.01	0\\
116.01	0\\
117.01	0\\
118.01	0\\
119.01	0\\
120.01	0\\
121.01	0\\
122.01	0\\
123.01	0\\
124.01	0\\
125.01	0\\
126.01	0\\
127.01	0\\
128.01	0\\
129.01	0\\
130.01	0\\
131.01	0\\
132.01	0\\
133.01	0\\
134.01	0\\
135.01	0\\
136.01	0\\
137.01	0\\
138.01	0\\
139.01	0\\
140.01	0\\
141.01	0\\
142.01	0\\
143.01	0\\
144.01	0\\
145.01	0\\
146.01	0\\
147.01	0\\
148.01	0\\
149.01	0\\
150.01	0\\
151.01	0\\
152.01	0\\
153.01	0\\
154.01	0\\
155.01	0\\
156.01	0\\
157.01	0\\
158.01	0\\
159.01	0\\
160.01	0\\
161.01	0\\
162.01	0\\
163.01	0\\
164.01	0\\
165.01	0\\
166.01	0\\
167.01	0\\
168.01	0\\
169.01	0\\
170.01	0\\
171.01	0\\
172.01	0\\
173.01	0\\
174.01	0\\
175.01	0\\
176.01	0\\
177.01	0\\
178.01	0\\
179.01	0\\
180.01	0\\
181.01	0\\
182.01	0\\
183.01	0\\
184.01	0\\
185.01	0\\
186.01	0\\
187.01	0\\
188.01	0\\
189.01	0\\
190.01	0\\
191.01	0\\
192.01	0\\
193.01	0\\
194.01	0\\
195.01	0\\
196.01	0\\
197.01	0\\
198.01	0\\
199.01	0\\
200.01	0\\
201.01	0\\
202.01	0\\
203.01	0\\
204.01	0\\
205.01	0\\
206.01	0\\
207.01	0\\
208.01	0\\
209.01	0\\
210.01	0\\
211.01	0\\
212.01	0\\
213.01	0\\
214.01	0\\
215.01	0\\
216.01	0\\
217.01	0\\
218.01	0\\
219.01	0\\
220.01	0\\
221.01	0\\
222.01	0\\
223.01	0\\
224.01	0\\
225.01	0\\
226.01	0\\
227.01	0\\
228.01	0\\
229.01	0\\
230.01	0\\
231.01	0\\
232.01	0\\
233.01	0\\
234.01	0\\
235.01	0\\
236.01	0\\
237.01	0\\
238.01	0\\
239.01	0\\
240.01	0\\
241.01	0\\
242.01	0\\
243.01	0\\
244.01	0\\
245.01	0\\
246.01	0\\
247.01	0\\
248.01	0\\
249.01	0\\
250.01	0\\
251.01	0\\
252.01	0\\
253.01	0\\
254.01	0\\
255.01	0\\
256.01	0\\
257.01	0\\
258.01	0\\
259.01	0\\
260.01	0\\
261.01	0\\
262.01	0\\
263.01	0\\
264.01	0\\
265.01	0\\
266.01	0\\
267.01	0\\
268.01	0\\
269.01	0\\
270.01	0\\
271.01	0\\
272.01	0\\
273.01	0\\
274.01	0\\
275.01	0\\
276.01	0\\
277.01	0\\
278.01	0\\
279.01	0\\
280.01	0\\
281.01	0\\
282.01	0\\
283.01	0\\
284.01	0\\
285.01	0\\
286.01	0\\
287.01	0\\
288.01	0\\
289.01	0\\
290.01	0\\
291.01	0\\
292.01	0\\
293.01	0\\
294.01	0\\
295.01	0\\
296.01	0\\
297.01	0\\
298.01	0\\
299.01	0\\
300.01	0\\
301.01	0\\
302.01	0\\
303.01	0\\
304.01	0\\
305.01	0\\
306.01	0\\
307.01	0\\
308.01	0\\
309.01	0\\
310.01	0\\
311.01	0\\
312.01	0\\
313.01	0\\
314.01	0\\
315.01	0\\
316.01	0\\
317.01	0\\
318.01	0\\
319.01	0\\
320.01	0\\
321.01	0\\
322.01	0\\
323.01	0\\
324.01	0\\
325.01	0\\
326.01	0\\
327.01	0\\
328.01	0\\
329.01	0\\
330.01	0\\
331.01	0\\
332.01	0\\
333.01	0\\
334.01	0\\
335.01	0\\
336.01	0\\
337.01	0\\
338.01	0\\
339.01	0\\
340.01	0\\
341.01	0\\
342.01	0\\
343.01	0\\
344.01	0\\
345.01	0\\
346.01	0\\
347.01	0\\
348.01	0\\
349.01	0\\
350.01	0\\
351.01	0\\
352.01	0\\
353.01	0\\
354.01	0\\
355.01	0\\
356.01	0\\
357.01	0\\
358.01	0\\
359.01	0\\
360.01	0\\
361.01	0\\
362.01	0\\
363.01	0\\
364.01	0\\
365.01	0\\
366.01	0\\
367.01	0\\
368.01	0\\
369.01	0\\
370.01	0\\
371.01	0\\
372.01	0\\
373.01	0\\
374.01	0\\
375.01	0\\
376.01	0\\
377.01	0\\
378.01	0\\
379.01	0\\
380.01	0\\
381.01	0\\
382.01	0\\
383.01	0\\
384.01	0\\
385.01	0\\
386.01	0\\
387.01	0\\
388.01	0\\
389.01	0\\
390.01	0\\
391.01	0\\
392.01	0\\
393.01	0\\
394.01	0\\
395.01	0\\
396.01	0\\
397.01	0\\
398.01	0\\
399.01	0\\
400.01	0\\
401.01	0\\
402.01	0\\
403.01	0\\
404.01	0\\
405.01	0\\
406.01	0\\
407.01	0\\
408.01	0\\
409.01	0\\
410.01	0\\
411.01	0\\
412.01	0\\
413.01	0\\
414.01	0\\
415.01	0\\
416.01	0\\
417.01	0\\
418.01	0\\
419.01	0\\
420.01	0\\
421.01	0\\
422.01	0\\
423.01	0\\
424.01	0\\
425.01	0\\
426.01	0\\
427.01	0\\
428.01	0\\
429.01	0\\
430.01	0\\
431.01	0\\
432.01	0\\
433.01	0\\
434.01	0\\
435.01	0\\
436.01	0\\
437.01	0\\
438.01	0\\
439.01	0\\
440.01	0\\
441.01	0\\
442.01	0\\
443.01	0\\
444.01	0\\
445.01	0\\
446.01	0\\
447.01	0\\
448.01	0\\
449.01	0\\
450.01	0\\
451.01	0\\
452.01	0\\
453.01	0\\
454.01	0\\
455.01	0\\
456.01	0\\
457.01	0\\
458.01	0\\
459.01	0\\
460.01	0\\
461.01	0\\
462.01	0\\
463.01	0\\
464.01	0\\
465.01	0\\
466.01	0\\
467.01	0\\
468.01	0\\
469.01	0\\
470.01	0\\
471.01	0\\
472.01	0\\
473.01	0\\
474.01	0\\
475.01	0\\
476.01	0\\
477.01	0\\
478.01	0\\
479.01	0\\
480.01	0\\
481.01	0\\
482.01	0\\
483.01	0\\
484.01	0\\
485.01	0\\
486.01	0\\
487.01	0\\
488.01	0\\
489.01	0\\
490.01	0\\
491.01	0\\
492.01	0\\
493.01	0\\
494.01	0\\
495.01	0\\
496.01	0\\
497.01	0\\
498.01	0\\
499.01	0\\
500.01	0\\
501.01	0\\
502.01	0\\
503.01	0\\
504.01	0\\
505.01	0\\
506.01	0\\
507.01	0\\
508.01	0\\
509.01	0\\
510.01	0\\
511.01	0\\
512.01	0\\
513.01	0\\
514.01	0\\
515.01	0\\
516.01	0\\
517.01	0\\
518.01	0\\
519.01	0\\
520.01	0\\
521.01	0\\
522.01	0\\
523.01	0\\
524.01	0\\
525.01	0\\
526.01	0\\
527.01	0\\
528.01	0\\
529.01	0\\
530.01	0\\
531.01	0\\
532.01	0\\
533.01	0\\
534.01	0\\
535.01	0\\
536.01	0\\
537.01	0\\
538.01	0\\
539.01	0\\
540.01	0\\
541.01	0\\
542.01	0\\
543.01	0\\
544.01	0\\
545.01	0\\
546.01	0\\
547.01	0\\
548.01	0\\
549.01	0\\
550.01	0\\
551.01	0\\
552.01	0\\
553.01	0\\
554.01	0\\
555.01	0\\
556.01	0\\
557.01	0\\
558.01	0\\
559.01	0\\
560.01	0\\
561.01	0\\
562.01	0\\
563.01	0\\
564.01	0\\
565.01	0\\
566.01	0\\
567.01	0\\
568.01	0\\
569.01	0\\
570.01	0\\
571.01	0\\
572.01	0\\
573.01	0\\
574.01	0\\
575.01	0\\
576.01	0\\
577.01	0\\
578.01	0\\
579.01	0\\
580.01	0\\
581.01	0\\
582.01	0\\
583.01	0\\
584.01	0\\
585.01	0\\
586.01	0\\
587.01	0\\
588.01	0\\
589.01	0\\
590.01	0\\
591.01	0\\
592.01	0\\
593.01	0\\
594.01	0\\
595.01	0\\
596.01	0\\
597.01	0\\
598.01	0.00134910574383743\\
599.01	0.00375813090594593\\
599.02	0.00379585558757839\\
599.03	0.00383394724377265\\
599.04	0.00387240947898429\\
599.05	0.00391124593307965\\
599.06	0.00395046028168358\\
599.07	0.00399005623653081\\
599.08	0.00403003754582079\\
599.09	0.00407040799457573\\
599.1	0.00411117140500245\\
599.11	0.00415233163685777\\
599.12	0.00419389258781722\\
599.13	0.00423585819384772\\
599.14	0.00427823242958368\\
599.15	0.00432101930870689\\
599.16	0.00436422288433012\\
599.17	0.00440784724938452\\
599.18	0.0044518965370107\\
599.19	0.00449637492095395\\
599.2	0.00454128661596295\\
599.21	0.00458663587819266\\
599.22	0.00463242700561114\\
599.23	0.00467866433841028\\
599.24	0.00472535225942061\\
599.25	0.0047724951945302\\
599.26	0.00482009761310766\\
599.27	0.00486816402842929\\
599.28	0.00491669899811045\\
599.29	0.00496570712454113\\
599.3	0.0050151930553258\\
599.31	0.00506516148372768\\
599.32	0.00511561714911722\\
599.33	0.00516656483742507\\
599.34	0.00521800938159951\\
599.35	0.00526995566206837\\
599.36	0.00532240860720543\\
599.37	0.00537537319380148\\
599.38	0.00542885444753995\\
599.39	0.00548285743950262\\
599.4	0.0055373872762364\\
599.41	0.00559244911427378\\
599.42	0.00564804816062216\\
599.43	0.00570418967325786\\
599.44	0.00576087896162512\\
599.45	0.00581812138713977\\
599.46	0.00587592236369806\\
599.47	0.00593428735819028\\
599.48	0.00599322189101954\\
599.49	0.00605273153662557\\
599.5	0.00611282192401361\\
599.51	0.00617349873728855\\
599.52	0.0062347677161943\\
599.53	0.00629663465665836\\
599.54	0.00635910541134177\\
599.55	0.00642218589019447\\
599.56	0.00648588206101604\\
599.57	0.00655019995002198\\
599.58	0.0066151456424155\\
599.59	0.00668072528296487\\
599.6	0.00674694507658656\\
599.61	0.00681381128893394\\
599.62	0.00688133024699175\\
599.63	0.00694950833967654\\
599.64	0.00701835201844282\\
599.65	0.00708786779789516\\
599.66	0.00715806225640647\\
599.67	0.00722894203674201\\
599.68	0.00730051384668986\\
599.69	0.00737278445969724\\
599.7	0.00744576071551335\\
599.71	0.0075194495208382\\
599.72	0.00759385784997808\\
599.73	0.00766899274550727\\
599.74	0.00774486131893626\\
599.75	0.00782147075138655\\
599.76	0.00789882829427199\\
599.77	0.00797694126998689\\
599.78	0.00805581707260071\\
599.79	0.00813546316855975\\
599.8	0.00821588709739553\\
599.81	0.00829709647244019\\
599.82	0.0083790989815489\\
599.83	0.00846190238782927\\
599.84	0.008545514530378\\
599.85	0.00862994332502465\\
599.86	0.00871519676508278\\
599.87	0.00880128292210833\\
599.88	0.00888820994666551\\
599.89	0.00897598606910017\\
599.9	0.00906461960032073\\
599.91	0.00915411893258671\\
599.92	0.00924449254030512\\
599.93	0.00933574898083444\\
599.94	0.00942789689529664\\
599.95	0.00952094500939709\\
599.96	0.00961490213425244\\
599.97	0.00970977716722672\\
599.98	0.00980557909277551\\
599.99	0.00990231698329844\\
600	0.01\\
};
\end{axis}
\end{tikzpicture}%

  \caption{Continuous Time}
\end{subfigure}%
\hfill%
\begin{subfigure}{.45\linewidth}
  \centering
  \setlength\figureheight{\linewidth} 
  \setlength\figurewidth{\linewidth}
  \tikzsetnextfilename{dp_colorbar/dp_dscr_nFPC_z1}
  % This file was created by matlab2tikz.
%
%The latest updates can be retrieved from
%  http://www.mathworks.com/matlabcentral/fileexchange/22022-matlab2tikz-matlab2tikz
%where you can also make suggestions and rate matlab2tikz.
%
\definecolor{mycolor1}{rgb}{0.00000,1.00000,0.14286}%
\definecolor{mycolor2}{rgb}{0.00000,1.00000,0.28571}%
\definecolor{mycolor3}{rgb}{0.00000,1.00000,0.42857}%
\definecolor{mycolor4}{rgb}{0.00000,1.00000,0.57143}%
\definecolor{mycolor5}{rgb}{0.00000,1.00000,0.71429}%
\definecolor{mycolor6}{rgb}{0.00000,1.00000,0.85714}%
\definecolor{mycolor7}{rgb}{0.00000,1.00000,1.00000}%
\definecolor{mycolor8}{rgb}{0.00000,0.87500,1.00000}%
\definecolor{mycolor9}{rgb}{0.00000,0.62500,1.00000}%
\definecolor{mycolor10}{rgb}{0.12500,0.00000,1.00000}%
\definecolor{mycolor11}{rgb}{0.25000,0.00000,1.00000}%
\definecolor{mycolor12}{rgb}{0.37500,0.00000,1.00000}%
\definecolor{mycolor13}{rgb}{0.50000,0.00000,1.00000}%
\definecolor{mycolor14}{rgb}{0.62500,0.00000,1.00000}%
\definecolor{mycolor15}{rgb}{0.75000,0.00000,1.00000}%
\definecolor{mycolor16}{rgb}{0.87500,0.00000,1.00000}%
\definecolor{mycolor17}{rgb}{1.00000,0.00000,1.00000}%
\definecolor{mycolor18}{rgb}{1.00000,0.00000,0.87500}%
\definecolor{mycolor19}{rgb}{1.00000,0.00000,0.62500}%
\definecolor{mycolor20}{rgb}{0.85714,0.00000,0.00000}%
\definecolor{mycolor21}{rgb}{0.71429,0.00000,0.00000}%
%
\begin{tikzpicture}

\begin{axis}[%
width=4.1in,
height=3.803in,
at={(0.809in,0.513in)},
scale only axis,
point meta min=0,
point meta max=1,
every outer x axis line/.append style={black},
every x tick label/.append style={font=\color{black}},
xmin=0,
xmax=600,
every outer y axis line/.append style={black},
every y tick label/.append style={font=\color{black}},
ymin=0,
ymax=0.014,
axis background/.style={fill=white},
axis x line*=bottom,
axis y line*=left,
colormap={mymap}{[1pt] rgb(0pt)=(0,1,0); rgb(7pt)=(0,1,1); rgb(15pt)=(0,0,1); rgb(23pt)=(1,0,1); rgb(31pt)=(1,0,0); rgb(38pt)=(0,0,0)},
colorbar,
colorbar style={separate axis lines,every outer x axis line/.append style={black},every x tick label/.append style={font=\color{black}},every outer y axis line/.append style={black},every y tick label/.append style={font=\color{black}},yticklabels={{-19},{-17},{-15},{-13},{-11},{-9},{-7},{-5},{-3},{-1},{1},{3},{5},{7},{9},{11},{13},{15},{17},{19}}}
]
\addplot [color=green,solid,forget plot]
  table[row sep=crcr]{%
1	0.0124897681842806\\
2	0.0124897499947743\\
3	0.0124897313920624\\
4	0.0124897123667764\\
5	0.0124896929093365\\
6	0.0124896730099469\\
7	0.012489652658591\\
8	0.0124896318450267\\
9	0.0124896105587808\\
10	0.0124895887891448\\
11	0.0124895665251686\\
12	0.0124895437556562\\
13	0.0124895204691594\\
14	0.0124894966539725\\
15	0.012489472298127\\
16	0.0124894473893851\\
17	0.0124894219152343\\
18	0.012489395862881\\
19	0.0124893692192444\\
20	0.0124893419709503\\
21	0.0124893141043243\\
22	0.0124892856053857\\
23	0.0124892564598401\\
24	0.0124892266530732\\
25	0.0124891961701431\\
26	0.0124891649957736\\
27	0.0124891331143468\\
28	0.0124891005098951\\
29	0.0124890671660944\\
30	0.0124890330662554\\
31	0.0124889981933163\\
32	0.012488962529834\\
33	0.0124889260579766\\
34	0.012488888759514\\
35	0.0124888506158097\\
36	0.012488811607812\\
37	0.0124887717160446\\
38	0.0124887309205976\\
39	0.0124886892011177\\
40	0.012488646536799\\
41	0.0124886029063727\\
42	0.0124885582880974\\
43	0.0124885126597483\\
44	0.012488465998607\\
45	0.0124884182814507\\
46	0.012488369484541\\
47	0.0124883195836128\\
48	0.0124882685538628\\
49	0.0124882163699377\\
50	0.0124881630059218\\
51	0.0124881084353252\\
52	0.012488052631071\\
53	0.0124879955654823\\
54	0.012487937210269\\
55	0.0124878775365142\\
56	0.0124878165146607\\
57	0.0124877541144963\\
58	0.0124876903051394\\
59	0.0124876250550243\\
60	0.0124875583318856\\
61	0.0124874901027424\\
62	0.0124874203338824\\
63	0.0124873489908453\\
64	0.0124872760384054\\
65	0.0124872014405546\\
66	0.0124871251604837\\
67	0.0124870471605643\\
68	0.0124869674023295\\
69	0.0124868858464539\\
70	0.0124868024527337\\
71	0.0124867171800652\\
72	0.012486629986423\\
73	0.0124865408288378\\
74	0.012486449663373\\
75	0.0124863564451002\\
76	0.0124862611280743\\
77	0.0124861636653076\\
78	0.0124860640087424\\
79	0.0124859621092229\\
80	0.0124858579164657\\
81	0.0124857513790295\\
82	0.0124856424442823\\
83	0.0124855310583685\\
84	0.0124854171661736\\
85	0.0124853007112872\\
86	0.0124851816359643\\
87	0.0124850598810849\\
88	0.0124849353861111\\
89	0.0124848080890416\\
90	0.0124846779263645\\
91	0.0124845448330064\\
92	0.0124844087422797\\
93	0.0124842695858253\\
94	0.0124841272935532\\
95	0.012483981793578\\
96	0.0124838330121518\\
97	0.0124836808735911\\
98	0.0124835253001996\\
99	0.0124833662121859\\
100	0.0124832035275752\\
101	0.0124830371621138\\
102	0.0124828670291687\\
103	0.0124826930396173\\
104	0.0124825151017309\\
105	0.0124823331210479\\
106	0.0124821470002375\\
107	0.0124819566389527\\
108	0.0124817619336714\\
109	0.012481562777524\\
110	0.0124813590601073\\
111	0.0124811506672822\\
112	0.012480937480954\\
113	0.0124807193788339\\
114	0.0124804962341784\\
115	0.0124802679155066\\
116	0.0124800342862908\\
117	0.0124797952046179\\
118	0.0124795505228206\\
119	0.0124793000870714\\
120	0.0124790437369384\\
121	0.0124787813048972\\
122	0.0124785126157933\\
123	0.0124782374862508\\
124	0.0124779557240191\\
125	0.0124776671272519\\
126	0.0124773714837087\\
127	0.0124770685698707\\
128	0.0124767581499585\\
129	0.0124764399748413\\
130	0.0124761137808209\\
131	0.0124757792882774\\
132	0.0124754362001546\\
133	0.0124750842002665\\
134	0.0124747229513997\\
135	0.012474352093183\\
136	0.012473971239693\\
137	0.0124735799767584\\
138	0.0124731778589205\\
139	0.0124727644060008\\
140	0.0124723390992182\\
141	0.0124719013767853\\
142	0.0124714506288953\\
143	0.0124709861919651\\
144	0.0124705073418964\\
145	0.0124700132858296\\
146	0.0124695031510399\\
147	0.0124689759672094\\
148	0.0124684306310268\\
149	0.0124678658199048\\
150	0.0124672797549955\\
151	0.0124666695506285\\
152	0.0124655220176865\\
153	0.0124643209304721\\
154	0.0124631068994203\\
155	0.012461879779158\\
156	0.012460639422368\\
157	0.0124593856797469\\
158	0.0124581183999614\\
159	0.0124568374296027\\
160	0.0124555426131394\\
161	0.0124542337928678\\
162	0.012452910808861\\
163	0.0124515734989151\\
164	0.0124502216984937\\
165	0.0124488552406697\\
166	0.0124474739560649\\
167	0.0124460776727863\\
168	0.0124446662163604\\
169	0.0124432394096641\\
170	0.0124417970728524\\
171	0.0124403390232828\\
172	0.0124388650754367\\
173	0.0124373750408364\\
174	0.0124358687279584\\
175	0.0124343459421428\\
176	0.0124328064854981\\
177	0.0124312501568013\\
178	0.0124296767513929\\
179	0.0124280860610673\\
180	0.0124264778739565\\
181	0.0124248519744094\\
182	0.0124232081428632\\
183	0.0124215461557094\\
184	0.012419865785152\\
185	0.0124181667990585\\
186	0.0124164489608024\\
187	0.012414712029098\\
188	0.012412955757825\\
189	0.0124111798958441\\
190	0.0124093841868022\\
191	0.0124075683689261\\
192	0.0124057321748046\\
193	0.0124038753311582\\
194	0.0124019975585945\\
195	0.0124000985713498\\
196	0.0123981780770149\\
197	0.012396235776244\\
198	0.0123942713624457\\
199	0.0123922845214549\\
200	0.0123902749311838\\
201	0.0123882422612506\\
202	0.0123861861725835\\
203	0.0123841063169999\\
204	0.0123820023367559\\
205	0.0123798738640662\\
206	0.0123777205205909\\
207	0.0123755419168863\\
208	0.012373337651816\\
209	0.0123711073119209\\
210	0.0123688504707413\\
211	0.01236656668809\\
212	0.0123642555092687\\
213	0.0123619164642245\\
214	0.0123595490666397\\
215	0.0123571528129485\\
216	0.0123547271812729\\
217	0.0123522716302694\\
218	0.0123497855978778\\
219	0.0123472684999595\\
220	0.0123447197288151\\
221	0.0123421386515643\\
222	0.0123395246083695\\
223	0.0123368769104764\\
224	0.0123341948380201\\
225	0.012331477637488\\
226	0.0123287245185589\\
227	0.0123259346495284\\
228	0.0123231071489932\\
229	0.0123202410667435\\
230	0.0123173353319925\\
231	0.0123143885998255\\
232	0.012311398776063\\
233	0.0123083615872519\\
234	0.0123052684376352\\
235	0.0123021313433887\\
236	0.0122989492523277\\
237	0.0122957208417754\\
238	0.0122924447005439\\
239	0.0122891193196325\\
240	0.0122857430818107\\
241	0.0122823142501185\\
242	0.0122788309555883\\
243	0.0122752911851106\\
244	0.0122716927717539\\
245	0.0122680333929044\\
246	0.0122643105881803\\
247	0.0122605218229948\\
248	0.0122566646517401\\
249	0.0122527370840316\\
250	0.0122487382931443\\
251	0.0122446693060966\\
252	0.0122405279664819\\
253	0.0122362918059262\\
254	0.0122319544356621\\
255	0.012227508575915\\
256	0.0122229457722666\\
257	0.0122182556703026\\
258	0.0122131454626804\\
259	0.0122062744447709\\
260	0.0121993190353875\\
261	0.0121922780549319\\
262	0.012185150285039\\
263	0.0121779344268315\\
264	0.0121706289758907\\
265	0.0121632318755543\\
266	0.0121557401381313\\
267	0.0121481559197659\\
268	0.0121404785863628\\
269	0.0121327068436496\\
270	0.0121248393777778\\
271	0.012116874855248\\
272	0.0121088119228444\\
273	0.0121006492074875\\
274	0.0120923853156354\\
275	0.0120840188307729\\
276	0.0120755483030845\\
277	0.0120669722061883\\
278	0.0120582887441587\\
279	0.0120494948788807\\
280	0.0120405801522957\\
281	0.0120315452171778\\
282	0.0120223999677997\\
283	0.0120131434828155\\
284	0.0120037753301137\\
285	0.0119942959563901\\
286	0.0119847063235465\\
287	0.0119749990802778\\
288	0.0119651713396973\\
289	0.0119552250222492\\
290	0.0119451697753187\\
291	0.0119350259652305\\
292	0.0119247492573933\\
293	0.0119143380895385\\
294	0.0119037909564472\\
295	0.0118931064255736\\
296	0.0118822831554732\\
297	0.0118713199172857\\
298	0.0118602156188022\\
299	0.0118489693274923\\
300	0.0118375802751465\\
301	0.0118260477619261\\
302	0.0118143705196092\\
303	0.0118025425611362\\
304	0.0117905725623206\\
305	0.0117784636773141\\
306	0.0117662178431268\\
307	0.0117538377931612\\
308	0.011741327249895\\
309	0.0117286912184618\\
310	0.0117159365391645\\
311	0.0117030732499434\\
312	0.0116901190741196\\
313	0.0116770867132529\\
314	0.0116639745525467\\
315	0.0116569764559215\\
316	0.011651037953093\\
317	0.0116450685689349\\
318	0.0116390703016166\\
319	0.0116330453323779\\
320	0.0116269960005527\\
321	0.0116209248376083\\
322	0.0116148346067817\\
323	0.0116087283242837\\
324	0.0116026092856854\\
325	0.0115964811017978\\
326	0.011590347746216\\
327	0.0115842136682991\\
328	0.0115780837950509\\
329	0.0115719635163849\\
330	0.0115658587290047\\
331	0.0115597758843811\\
332	0.0115537220401241\\
333	0.011547704906536\\
334	0.0115417328920142\\
335	0.0115358152154615\\
336	0.0115299619918528\\
337	0.0115241843294441\\
338	0.0115184944422059\\
339	0.0115129057842033\\
340	0.01150743323468\\
341	0.0115020932665117\\
342	0.011496904043422\\
343	0.0114918856144818\\
344	0.0114870601341707\\
345	0.0114824521110562\\
346	0.0114780886866855\\
347	0.0114739999391547\\
348	0.0114702191719366\\
349	0.0114667568316486\\
350	0.0114633132849764\\
351	0.0114598906781047\\
352	0.0114564912071508\\
353	0.0114531171098194\\
354	0.0114497706511324\\
355	0.0114464541090349\\
356	0.0114431697585837\\
357	0.0114399198534554\\
358	0.0114367066029506\\
359	0.011433532143429\\
360	0.0114303985036616\\
361	0.0114273075627111\\
362	0.0114242609980642\\
363	0.0114212602230584\\
364	0.0114183063149119\\
365	0.0114153999288342\\
366	0.01141254119559\\
367	0.0114097295994013\\
368	0.0114069638325477\\
369	0.0114042416226381\\
370	0.0114015595285162\\
371	0.0113989126942821\\
372	0.0113962945554893\\
373	0.0113936964877304\\
374	0.0113911073853027\\
375	0.01138851473969\\
376	0.0113859181971972\\
377	0.0113833173119853\\
378	0.011380711538328\\
379	0.0113781002228792\\
380	0.0113754825971527\\
381	0.0113728577705123\\
382	0.0113702247239388\\
383	0.0113675823049288\\
384	0.0113649292240269\\
385	0.0113622640536292\\
386	0.0113595852298662\\
387	0.0113568910585767\\
388	0.011354179726601\\
389	0.0113514493200365\\
390	0.0113486978515352\\
391	0.0113459232992453\\
392	0.0113431236607245\\
393	0.0113402970259769\\
394	0.0113374416746488\\
395	0.0113345562038304\\
396	0.0113316396945492\\
397	0.0113286912786436\\
398	0.0113257100640905\\
399	0.0113226951375787\\
400	0.0113196455675965\\
401	0.0113165604080537\\
402	0.011313438702445\\
403	0.0113102794885385\\
404	0.0113070818035454\\
405	0.0113038446896826\\
406	0.0113005671999783\\
407	0.011297248404088\\
408	0.0112938873937712\\
409	0.0112904832875337\\
410	0.0112870352337436\\
411	0.0112835424112656\\
412	0.0112800040263188\\
413	0.0112764193038087\\
414	0.0112727874708002\\
415	0.0112691077489393\\
416	0.0112653793550945\\
417	0.0112616015019377\\
418	0.0112577733984404\\
419	0.0112538942502539\\
420	0.0112499632599395\\
421	0.011245979627018\\
422	0.0112419425478029\\
423	0.0112378512149911\\
424	0.0112337048169925\\
425	0.0112295025369944\\
426	0.0112252435517858\\
427	0.0112209270304042\\
428	0.0112165521327284\\
429	0.0112121180082258\\
430	0.0112076237951866\\
431	0.0112030686204205\\
432	0.0111984515989128\\
433	0.0111937718334402\\
434	0.0111890284141468\\
435	0.0111842204180812\\
436	0.0111793469087012\\
437	0.0111744069353487\\
438	0.0111693995327043\\
439	0.0111643237202288\\
440	0.0111591785016042\\
441	0.0111539628641816\\
442	0.0111486757784466\\
443	0.0111433161975055\\
444	0.0111378830565872\\
445	0.0111323752725393\\
446	0.0111267917433053\\
447	0.0111211313473823\\
448	0.0111153929432602\\
449	0.0111095753688422\\
450	0.0111036774408473\\
451	0.0110976979541931\\
452	0.0110916356813608\\
453	0.01108548937174\\
454	0.0110792577509524\\
455	0.0110729395201534\\
456	0.011066533355308\\
457	0.0110600379064397\\
458	0.011053451796849\\
459	0.0110467736222994\\
460	0.011040001950171\\
461	0.0110331353185774\\
462	0.011026172235447\\
463	0.0110191111775638\\
464	0.0110119505895676\\
465	0.0110046888829107\\
466	0.0109973244347679\\
467	0.0109898555868982\\
468	0.0109822806444541\\
469	0.0109745978747372\\
470	0.0109668055058956\\
471	0.0109589017255597\\
472	0.0109508846794146\\
473	0.0109427524697038\\
474	0.0109345031536602\\
475	0.0109261347418621\\
476	0.010917645196507\\
477	0.0109090324296014\\
478	0.0109002943010583\\
479	0.0108914286167004\\
480	0.0108824331261609\\
481	0.0108733055206763\\
482	0.0108640434307657\\
483	0.010854644423789\\
484	0.0108451060013766\\
485	0.0108354255967231\\
486	0.0108256005717376\\
487	0.0108156282140407\\
488	0.0108055057338\\
489	0.0107952302603946\\
490	0.0107847988388982\\
491	0.0107742084263705\\
492	0.0107634558879466\\
493	0.0107525379927126\\
494	0.0107414514093556\\
495	0.0107301927015777\\
496	0.0107187583232606\\
497	0.0107071446133703\\
498	0.0106953477905887\\
499	0.010683363947663\\
500	0.0106711890454607\\
501	0.0106588189067216\\
502	0.0106462492094991\\
503	0.010633475480284\\
504	0.01062049308681\\
505	0.010607297230538\\
506	0.010593882938827\\
507	0.0105802450567994\\
508	0.0105663782389195\\
509	0.0105522730167099\\
510	0.0105379282758502\\
511	0.0105233398131128\\
512	0.0105084982627157\\
513	0.0104933982717844\\
514	0.0104780387368352\\
515	0.0104624117159062\\
516	0.0104465168741575\\
517	0.010430338110904\\
518	0.0104138601073215\\
519	0.0103970752891333\\
520	0.0103799740894401\\
521	0.0103625472625547\\
522	0.0103447860033705\\
523	0.0103266810616174\\
524	0.0103082228900916\\
525	0.010289391268734\\
526	0.0102701819114095\\
527	0.0102506040189008\\
528	0.0102306249539969\\
529	0.0102102812552429\\
530	0.0101895704585159\\
531	0.0101684943826255\\
532	0.0101469696050343\\
533	0.0101249805933305\\
534	0.0101025062038\\
535	0.0100795317365609\\
536	0.0100560416729332\\
537	0.0100320209551519\\
538	0.0100074541320563\\
539	0.00998232879437465\\
540	0.00995661836385641\\
541	0.00993030072187439\\
542	0.00990331483917391\\
543	0.00987561760665483\\
544	0.00984723732256093\\
545	0.00981830109818854\\
546	0.009789145095759\\
547	0.00975939644714741\\
548	0.00972891729681173\\
549	0.00969767543412559\\
550	0.00966564626747568\\
551	0.00963280526397163\\
552	0.00959912703850655\\
553	0.00956458512770733\\
554	0.00952915185033005\\
555	0.00949279808571401\\
556	0.00945506717019782\\
557	0.00941596384115013\\
558	0.0093765693029433\\
559	0.00933810399915141\\
560	0.00929866190376292\\
561	0.00925816686011643\\
562	0.00921657377355508\\
563	0.00915714596620506\\
564	0.00888374711558651\\
565	0.00850308588273132\\
566	0.00844099822861057\\
567	0.0083778796614207\\
568	0.00831365319536097\\
569	0.00824828895484398\\
570	0.00818175575864172\\
571	0.00811402081575833\\
572	0.00804504960172116\\
573	0.00797480573046632\\
574	0.0079032508165112\\
575	0.00783034432612527\\
576	0.00775604341652007\\
577	0.00768030276197866\\
578	0.00760307436565794\\
579	0.00752430735533364\\
580	0.00744394776027091\\
581	0.00736193826362867\\
582	0.00727821791754122\\
583	0.00719272178875226\\
584	0.00710538045132748\\
585	0.00701611910584812\\
586	0.00692485573802265\\
587	0.00683149674994525\\
588	0.00673592587771182\\
589	0.00663797520465071\\
590	0.00653734834917362\\
591	0.00643341581933105\\
592	0.00632466858477865\\
593	0.00620725772627072\\
594	0.0060710901180166\\
595	0.00588956027306328\\
596	0.00559184182825265\\
597	0.00498881296807123\\
598	0.00357511483354343\\
599	0\\
600	0\\
};
\addplot [color=mycolor1,solid,forget plot]
  table[row sep=crcr]{%
1	0.01248982130969\\
2	0.0124898047810677\\
3	0.012489787892337\\
4	0.0124897706358697\\
5	0.0124897530038844\\
6	0.0124897349884446\\
7	0.0124897165814558\\
8	0.0124896977746626\\
9	0.012489678559646\\
10	0.0124896589278208\\
11	0.0124896388704325\\
12	0.0124896183785548\\
13	0.0124895974430862\\
14	0.0124895760547478\\
15	0.0124895542040798\\
16	0.0124895318814389\\
17	0.0124895090769953\\
18	0.0124894857807294\\
19	0.0124894619824294\\
20	0.0124894376716878\\
21	0.0124894128378984\\
22	0.0124893874702536\\
23	0.0124893615577411\\
24	0.0124893350891407\\
25	0.0124893080530215\\
26	0.0124892804377389\\
27	0.012489252231431\\
28	0.0124892234220162\\
29	0.0124891939971895\\
30	0.0124891639444198\\
31	0.0124891332509468\\
32	0.0124891019037778\\
33	0.0124890698896848\\
34	0.0124890371952012\\
35	0.012489003806619\\
36	0.012488969709986\\
37	0.0124889348911024\\
38	0.012488899335518\\
39	0.0124888630285294\\
40	0.0124888259551773\\
41	0.0124887881002431\\
42	0.0124887494482467\\
43	0.0124887099834434\\
44	0.0124886696898215\\
45	0.0124886285510993\\
46	0.012488586550723\\
47	0.0124885436718638\\
48	0.0124884998974159\\
49	0.0124884552099938\\
50	0.01248840959193\\
51	0.0124883630252734\\
52	0.0124883154917866\\
53	0.0124882669729444\\
54	0.0124882174499317\\
55	0.0124881669036418\\
56	0.0124881153146748\\
57	0.0124880626633362\\
58	0.0124880089296354\\
59	0.0124879540932847\\
60	0.012487898133698\\
61	0.0124878410299899\\
62	0.0124877827609752\\
63	0.0124877233051679\\
64	0.0124876626407812\\
65	0.0124876007457268\\
66	0.0124875375976154\\
67	0.0124874731737561\\
68	0.0124874074511573\\
69	0.012487340406527\\
70	0.0124872720162735\\
71	0.0124872022565061\\
72	0.0124871311030369\\
73	0.0124870585313812\\
74	0.0124869845167597\\
75	0.0124869090341002\\
76	0.0124868320580392\\
77	0.0124867535629247\\
78	0.0124866735228179\\
79	0.0124865919114967\\
80	0.0124865087024577\\
81	0.01248642386892\\
82	0.0124863373838279\\
83	0.0124862492198548\\
84	0.0124861593494067\\
85	0.012486067744626\\
86	0.0124859743773959\\
87	0.0124858792193441\\
88	0.0124857822418478\\
89	0.0124856834160377\\
90	0.012485582712803\\
91	0.012485480102796\\
92	0.0124853755564371\\
93	0.0124852690439197\\
94	0.0124851605352154\\
95	0.0124850500000788\\
96	0.0124849374080525\\
97	0.0124848227284729\\
98	0.0124847059304741\\
99	0.012484586982994\\
100	0.0124844658547785\\
101	0.0124843425143864\\
102	0.0124842169301945\\
103	0.0124840890704017\\
104	0.0124839589030336\\
105	0.0124838263959465\\
106	0.0124836915168317\\
107	0.0124835542332191\\
108	0.0124834145124811\\
109	0.0124832723218357\\
110	0.0124831276283501\\
111	0.0124829803989441\\
112	0.0124828306003928\\
113	0.0124826781993304\\
114	0.012482523162253\\
115	0.012482365455523\\
116	0.0124822050453726\\
117	0.0124820418979092\\
118	0.0124818759791209\\
119	0.0124817072548838\\
120	0.0124815356909711\\
121	0.0124813612530642\\
122	0.0124811839067668\\
123	0.012481003617623\\
124	0.0124808203511399\\
125	0.0124806340728157\\
126	0.0124804447481754\\
127	0.0124802523428154\\
128	0.0124800568224586\\
129	0.0124798581530236\\
130	0.0124796563007091\\
131	0.0124794512321\\
132	0.0124792429142964\\
133	0.0124790313150731\\
134	0.0124788164030755\\
135	0.0124785981480597\\
136	0.0124783765211862\\
137	0.0124781514953799\\
138	0.0124779230457703\\
139	0.0124776911502283\\
140	0.012477455790022\\
141	0.0124772169506114\\
142	0.0124769746226051\\
143	0.0124767288028848\\
144	0.0124764794958574\\
145	0.0124762267146486\\
146	0.0124759704816643\\
147	0.0124757108269577\\
148	0.0124754477806711\\
149	0.0124751813528551\\
150	0.0124749115024053\\
151	0.0124746382066063\\
152	0.012474362172165\\
153	0.012474083429475\\
154	0.0124738019515424\\
155	0.0124735177109414\\
156	0.0124732306797966\\
157	0.0124729408297643\\
158	0.0124726481320126\\
159	0.0124723525572006\\
160	0.0124720540754561\\
161	0.0124717526563525\\
162	0.0124714482688839\\
163	0.0124711408814397\\
164	0.0124708304617764\\
165	0.0124705169769896\\
166	0.0124702003934824\\
167	0.0124698806769344\\
168	0.0124695577922667\\
169	0.012469231703607\\
170	0.012468902374251\\
171	0.0124685697666232\\
172	0.0124682338422346\\
173	0.0124678945616387\\
174	0.0124675518843843\\
175	0.012467205768967\\
176	0.0124668561727765\\
177	0.0124665030520427\\
178	0.0124661463617774\\
179	0.012465786055714\\
180	0.012465422086243\\
181	0.0124650544043449\\
182	0.0124646829595185\\
183	0.0124643076997062\\
184	0.0124639285712149\\
185	0.0124635455186322\\
186	0.0124631584847388\\
187	0.0124627674104151\\
188	0.0124623722345443\\
189	0.0124619728939081\\
190	0.0124615693230784\\
191	0.0124611614543021\\
192	0.0124607492173796\\
193	0.0124603325395366\\
194	0.012459911345289\\
195	0.0124594855562993\\
196	0.0124590550912255\\
197	0.0124586198655612\\
198	0.012458179791466\\
199	0.0124577347775864\\
200	0.0124572847288657\\
201	0.0124568295463427\\
202	0.0124563691269379\\
203	0.0124559033632271\\
204	0.0124554321431998\\
205	0.0124549553500034\\
206	0.0124544728616703\\
207	0.0124539845508267\\
208	0.0124534902843832\\
209	0.012452989923203\\
210	0.012452483321748\\
211	0.0124519703276998\\
212	0.0124514507815527\\
213	0.0124509245161775\\
214	0.0124503913563522\\
215	0.0124498511182564\\
216	0.0124493036089256\\
217	0.0124487486256623\\
218	0.0124481859553975\\
219	0.012447615373998\\
220	0.0124470366455137\\
221	0.0124464495213551\\
222	0.0124458537393924\\
223	0.0124452490229572\\
224	0.0124446350797139\\
225	0.0124440116003234\\
226	0.0124433782567031\\
227	0.0124427346993644\\
228	0.0124420805524397\\
229	0.0124414154027706\\
230	0.01244073877421\\
231	0.0124400500696235\\
232	0.0124393484737726\\
233	0.012438633038978\\
234	0.012437904250913\\
235	0.012437161548183\\
236	0.0124364043130293\\
237	0.0124356318819047\\
238	0.0124348435408024\\
239	0.0124340385200036\\
240	0.0124332159881911\\
241	0.0124323750459106\\
242	0.0124315147184497\\
243	0.0124306339484011\\
244	0.0124297315886051\\
245	0.0124288063970172\\
246	0.0124278570365894\\
247	0.0124268820854545\\
248	0.0124258800631654\\
249	0.0124248494635782\\
250	0.0124237886938405\\
251	0.0124226955095855\\
252	0.0124215660834259\\
253	0.0124203975479223\\
254	0.0124191866615405\\
255	0.0124179296989049\\
256	0.0124166221971726\\
257	0.0124152581972048\\
258	0.0124135998771043\\
259	0.0124105464902129\\
260	0.0124074533112881\\
261	0.0124043196111587\\
262	0.0124011446274988\\
263	0.0123979275498318\\
264	0.01239466749984\\
265	0.012391363575555\\
266	0.0123880152332381\\
267	0.0123846216440767\\
268	0.0123811818915475\\
269	0.0123776950252452\\
270	0.0123741600587974\\
271	0.0123705759675712\\
272	0.0123669416860765\\
273	0.0123632561048083\\
274	0.0123595180657178\\
275	0.0123557263536714\\
276	0.0123518796749791\\
277	0.0123479765922832\\
278	0.0123440153100879\\
279	0.0123399929826119\\
280	0.012335904356084\\
281	0.012331749671693\\
282	0.0123275323610815\\
283	0.0123232508899895\\
284	0.0123189036977558\\
285	0.0123144890652403\\
286	0.0123100046228365\\
287	0.0123054485013464\\
288	0.0123008195614031\\
289	0.0122961181720901\\
290	0.0122913480976104\\
291	0.012286513195336\\
292	0.0122815914432025\\
293	0.0122765797181411\\
294	0.0122714746760928\\
295	0.0122662727285626\\
296	0.0122609700157525\\
297	0.0122555623751948\\
298	0.0122500453034464\\
299	0.0122444139040322\\
300	0.012238662800327\\
301	0.0122327859450087\\
302	0.0122267761397827\\
303	0.0122206244342492\\
304	0.0122143269975641\\
305	0.0122078763474986\\
306	0.0122012632738793\\
307	0.0121944775512155\\
308	0.0121875078110607\\
309	0.0121803414402418\\
310	0.0121729645878634\\
311	0.0121653624380007\\
312	0.0121575193403631\\
313	0.0121494114067925\\
314	0.0121410012302823\\
315	0.0121289991707545\\
316	0.0121162237446487\\
317	0.0121032760675475\\
318	0.0120901510109772\\
319	0.0120768300389825\\
320	0.0120633227376904\\
321	0.0120496380803449\\
322	0.0120357743204656\\
323	0.0120217303333258\\
324	0.012007506849554\\
325	0.0119931106168703\\
326	0.0119785707010117\\
327	0.0119638612863687\\
328	0.0119489529131119\\
329	0.0119338434367098\\
330	0.011918530725327\\
331	0.0119030122789395\\
332	0.0118872829361935\\
333	0.011871337033803\\
334	0.0118551852840994\\
335	0.0118388271871127\\
336	0.0118222628867497\\
337	0.0118054937113584\\
338	0.0117885238166105\\
339	0.011771366243537\\
340	0.01175402276002\\
341	0.0117364733917679\\
342	0.0117187228969905\\
343	0.0117007776331837\\
344	0.0116826458992591\\
345	0.0116643383754695\\
346	0.011645868740675\\
347	0.0116272547205502\\
348	0.0116085206182026\\
349	0.0115905098056133\\
350	0.0115827699634876\\
351	0.0115750160182305\\
352	0.0115672532899309\\
353	0.0115594876452058\\
354	0.0115517256661517\\
355	0.0115439746198808\\
356	0.0115362424670674\\
357	0.0115285378968242\\
358	0.0115208704188005\\
359	0.0115132504750047\\
360	0.0115056895429897\\
361	0.0114982002618669\\
362	0.0114907966116018\\
363	0.0114834940803675\\
364	0.0114763097184594\\
365	0.0114692623095194\\
366	0.0114623725629731\\
367	0.0114556633299413\\
368	0.0114491598433331\\
369	0.0114428899759606\\
370	0.0114368844969976\\
371	0.0114311774964501\\
372	0.011425806796031\\
373	0.0114208144276655\\
374	0.0114162472061001\\
375	0.0114121109389486\\
376	0.0114080082149506\\
377	0.0114039419716387\\
378	0.011399915135869\\
379	0.0113959305946254\\
380	0.0113919911587714\\
381	0.0113880995168144\\
382	0.0113842581813307\\
383	0.0113804694262423\\
384	0.0113767352115892\\
385	0.0113730570934725\\
386	0.0113694361164496\\
387	0.0113658726853779\\
388	0.0113623664141238\\
389	0.0113589159438883\\
390	0.0113555187249788\\
391	0.0113521707553167\\
392	0.0113488662655676\\
393	0.0113455973413804\\
394	0.0113423534754387\\
395	0.0113391210316181\\
396	0.0113358826040086\\
397	0.0113326349535495\\
398	0.0113293771781688\\
399	0.0113261082255134\\
400	0.0113228268844138\\
401	0.0113195317775774\\
402	0.0113162213560974\\
403	0.0113128938965208\\
404	0.0113095475013784\\
405	0.0113061801043885\\
406	0.0113027894819316\\
407	0.0112993732727906\\
408	0.0112959290088091\\
409	0.0112924541594801\\
410	0.0112889461943988\\
411	0.0112854026685634\\
412	0.0112818213366601\\
413	0.0112782003041048\\
414	0.0112745382245752\\
415	0.0112708339878973\\
416	0.0112670864592815\\
417	0.0112632944834878\\
418	0.0112594568896728\\
419	0.0112555724969145\\
420	0.0112516401203762\\
421	0.0112476585780188\\
422	0.0112436266976918\\
423	0.0112395433243453\\
424	0.0112354073269749\\
425	0.0112312176047389\\
426	0.0112269730914619\\
427	0.0112226727574288\\
428	0.0112183156069769\\
429	0.0112139006698625\\
430	0.0112094269836878\\
431	0.0112048935786911\\
432	0.0112002994783756\\
433	0.0111956437000278\\
434	0.0111909252550896\\
435	0.0111861431493417\\
436	0.0111812963828571\\
437	0.0111763839496849\\
438	0.011171404837229\\
439	0.0111663580253013\\
440	0.0111612424848455\\
441	0.0111560571763681\\
442	0.0111508010481581\\
443	0.0111454730344616\\
444	0.0111400720538851\\
445	0.0111345970084639\\
446	0.0111290467831078\\
447	0.0111234202449903\\
448	0.0111177162428817\\
449	0.0111119336064288\\
450	0.0111060711453838\\
451	0.0111001276487916\\
452	0.0110941018841419\\
453	0.011087992596497\\
454	0.0110817985076091\\
455	0.0110755183150369\\
456	0.0110691506912718\\
457	0.011062694282878\\
458	0.0110561477096353\\
459	0.0110495095636552\\
460	0.011042778408461\\
461	0.0110359527780285\\
462	0.0110290311757904\\
463	0.0110220120736005\\
464	0.0110148939106599\\
465	0.0110076750924007\\
466	0.0110003539893278\\
467	0.010992928935815\\
468	0.0109853982288538\\
469	0.0109777601267485\\
470	0.0109700128477561\\
471	0.0109621545686647\\
472	0.0109541834233072\\
473	0.0109460975010078\\
474	0.0109378948449566\\
475	0.0109295734505097\\
476	0.0109211312634098\\
477	0.0109125661779232\\
478	0.0109038760348892\\
479	0.0108950586196768\\
480	0.0108861116600428\\
481	0.0108770328238872\\
482	0.0108678197168994\\
483	0.0108584698800903\\
484	0.010848980787203\\
485	0.0108393498419968\\
486	0.0108295743753977\\
487	0.0108196516425083\\
488	0.010809578819469\\
489	0.010799353000165\\
490	0.0107889711927688\\
491	0.0107784303161116\\
492	0.010767727195875\\
493	0.0107568585605932\\
494	0.0107458210374583\\
495	0.0107346111479178\\
496	0.0107232253030569\\
497	0.0107116597987542\\
498	0.0106999108106023\\
499	0.0106879743885845\\
500	0.0106758464514961\\
501	0.0106635227811036\\
502	0.0106509990160317\\
503	0.0106382706453697\\
504	0.0106253330019917\\
505	0.0106121812555819\\
506	0.0105988104053619\\
507	0.0105852152725145\\
508	0.0105713904923051\\
509	0.0105573305058982\\
510	0.0105430294943534\\
511	0.0105284814171098\\
512	0.0105136777412571\\
513	0.0104986126775228\\
514	0.010483284269303\\
515	0.0104676848722985\\
516	0.0104518031448841\\
517	0.0104356359690962\\
518	0.0104191789036504\\
519	0.0104024228651959\\
520	0.010385367415071\\
521	0.010367993787575\\
522	0.010350286570482\\
523	0.0103322369326052\\
524	0.0103138343349133\\
525	0.0102950680180483\\
526	0.010275927693054\\
527	0.0102564017972843\\
528	0.0102364586923393\\
529	0.0102161261957729\\
530	0.0101953841960509\\
531	0.0101742141241981\\
532	0.0101526393269767\\
533	0.0101306616217065\\
534	0.0101082692106229\\
535	0.0100853868658679\\
536	0.0100619982790481\\
537	0.0100380807490736\\
538	0.0100136185087147\\
539	0.00998859739201236\\
540	0.00996299658999363\\
541	0.00993678699714986\\
542	0.00990995441930058\\
543	0.0098824840583619\\
544	0.00985426557139988\\
545	0.00982538631410531\\
546	0.00979576034912938\\
547	0.00976557228047118\\
548	0.00973513952339307\\
549	0.00970404442958007\\
550	0.0096721805993989\\
551	0.00963951287670261\\
552	0.00960601403348662\\
553	0.00957165740280535\\
554	0.00953641548189309\\
555	0.00950025959413018\\
556	0.00946315979362414\\
557	0.00942476702323156\\
558	0.00938505629975076\\
559	0.00934423285200159\\
560	0.00930470285384079\\
561	0.00926443506641366\\
562	0.00922308762457783\\
563	0.00918061162809973\\
564	0.00910086899317433\\
565	0.00882974345939778\\
566	0.00844143787727468\\
567	0.0083778817235865\\
568	0.00831365328902626\\
569	0.0082482889682748\\
570	0.00818175576141668\\
571	0.00811402081654658\\
572	0.0080450496020305\\
573	0.00797480573061697\\
574	0.00790325081658399\\
575	0.00783034432616349\\
576	0.00775604341653988\\
577	0.00768030276198882\\
578	0.00760307436566271\\
579	0.00752430735533561\\
580	0.00744394776027147\\
581	0.00736193826362876\\
582	0.00727821791754121\\
583	0.00719272178875226\\
584	0.00710538045132749\\
585	0.00701611910584812\\
586	0.00692485573802267\\
587	0.00683149674994525\\
588	0.00673592587771183\\
589	0.00663797520465071\\
590	0.00653734834917362\\
591	0.00643341581933106\\
592	0.00632466858477866\\
593	0.00620725772627072\\
594	0.00607109011801661\\
595	0.0058895602730633\\
596	0.00559184182825266\\
597	0.00498881296807123\\
598	0.00357511483354343\\
599	0\\
600	0\\
};
\addplot [color=mycolor2,solid,forget plot]
  table[row sep=crcr]{%
1	0.0124899058695097\\
2	0.0124898918345811\\
3	0.012489877513426\\
4	0.0124898629006011\\
5	0.0124898479905745\\
6	0.0124898327777244\\
7	0.0124898172563383\\
8	0.0124898014206125\\
9	0.0124897852646508\\
10	0.0124897687824639\\
11	0.0124897519679687\\
12	0.0124897348149873\\
13	0.0124897173172466\\
14	0.0124896994683773\\
15	0.0124896812619135\\
16	0.0124896626912916\\
17	0.0124896437498503\\
18	0.0124896244308294\\
19	0.0124896047273698\\
20	0.0124895846325127\\
21	0.0124895641391991\\
22	0.0124895432402696\\
23	0.0124895219284638\\
24	0.0124895001964202\\
25	0.0124894780366758\\
26	0.0124894554416659\\
27	0.0124894324037239\\
28	0.0124894089150813\\
29	0.0124893849678677\\
30	0.0124893605541104\\
31	0.0124893356657352\\
32	0.012489310294566\\
33	0.0124892844323252\\
34	0.012489258070634\\
35	0.012489231201013\\
36	0.0124892038148822\\
37	0.0124891759035619\\
38	0.0124891474582734\\
39	0.0124891184701396\\
40	0.0124890889301855\\
41	0.0124890588293399\\
42	0.0124890281584357\\
43	0.0124889969082112\\
44	0.0124889650693115\\
45	0.01248893263229\\
46	0.0124888995876092\\
47	0.0124888659256429\\
48	0.0124888316366775\\
49	0.0124887967109141\\
50	0.0124887611384702\\
51	0.0124887249093817\\
52	0.0124886880136055\\
53	0.0124886504410211\\
54	0.0124886121814337\\
55	0.0124885732245765\\
56	0.0124885335601134\\
57	0.0124884931776418\\
58	0.0124884520666956\\
59	0.0124884102167486\\
60	0.0124883676172173\\
61	0.0124883242574646\\
62	0.0124882801268035\\
63	0.0124882352145002\\
64	0.0124881895097784\\
65	0.0124881430018234\\
66	0.0124880956797856\\
67	0.0124880475327853\\
68	0.0124879985499165\\
69	0.0124879487202522\\
70	0.0124878980328481\\
71	0.0124878464767479\\
72	0.0124877940409881\\
73	0.0124877407146025\\
74	0.012487686486628\\
75	0.012487631346109\\
76	0.0124875752821032\\
77	0.0124875182836864\\
78	0.0124874603399584\\
79	0.012487401440048\\
80	0.0124873415731186\\
81	0.0124872807283737\\
82	0.0124872188950626\\
83	0.0124871560624855\\
84	0.0124870922199994\\
85	0.0124870273570234\\
86	0.0124869614630442\\
87	0.0124868945276211\\
88	0.012486826540392\\
89	0.0124867574910779\\
90	0.0124866873694881\\
91	0.0124866161655255\\
92	0.0124865438691906\\
93	0.0124864704705868\\
94	0.0124863959599242\\
95	0.0124863203275237\\
96	0.0124862435638211\\
97	0.0124861656593704\\
98	0.0124860866048471\\
99	0.0124860063910508\\
100	0.0124859250089081\\
101	0.0124858424494745\\
102	0.0124857587039359\\
103	0.0124856737636104\\
104	0.0124855876199487\\
105	0.0124855002645343\\
106	0.0124854116890838\\
107	0.0124853218854462\\
108	0.0124852308456014\\
109	0.0124851385616586\\
110	0.0124850450258544\\
111	0.0124849502305498\\
112	0.012484854168227\\
113	0.0124847568314853\\
114	0.0124846582130375\\
115	0.0124845583057042\\
116	0.0124844571024092\\
117	0.0124843545961735\\
118	0.012484250780109\\
119	0.0124841456474125\\
120	0.0124840391913581\\
121	0.0124839314052907\\
122	0.0124838222826185\\
123	0.0124837118168051\\
124	0.0124836000013626\\
125	0.0124834868298434\\
126	0.0124833722958328\\
127	0.0124832563929416\\
128	0.0124831391147989\\
129	0.0124830204550447\\
130	0.0124829004073231\\
131	0.0124827789652761\\
132	0.0124826561225369\\
133	0.0124825318727236\\
134	0.0124824062094329\\
135	0.0124822791262337\\
136	0.0124821506166592\\
137	0.0124820206741983\\
138	0.0124818892922846\\
139	0.0124817564642805\\
140	0.0124816221834558\\
141	0.0124814864429543\\
142	0.0124813492357403\\
143	0.0124812105545025\\
144	0.0124810703914728\\
145	0.0124809287380585\\
146	0.0124807855840914\\
147	0.0124806409163868\\
148	0.0124804947165609\\
149	0.0124803469604029\\
150	0.0124801976301356\\
151	0.0124800467548402\\
152	0.0124798943186481\\
153	0.0124797403015055\\
154	0.0124795846829034\\
155	0.0124794274418618\\
156	0.0124792685569136\\
157	0.0124791080060879\\
158	0.0124789457668929\\
159	0.0124787818162977\\
160	0.0124786161307143\\
161	0.0124784486859786\\
162	0.0124782794573303\\
163	0.0124781084193936\\
164	0.0124779355461554\\
165	0.0124777608109448\\
166	0.0124775841864102\\
167	0.012477405644497\\
168	0.0124772251564241\\
169	0.0124770426926598\\
170	0.0124768582228971\\
171	0.0124766717160279\\
172	0.0124764831401175\\
173	0.012476292462377\\
174	0.0124760996491364\\
175	0.0124759046658158\\
176	0.0124757074768966\\
177	0.012475508045892\\
178	0.0124753063353161\\
179	0.0124751023066531\\
180	0.012474895920325\\
181	0.0124746871356594\\
182	0.0124744759108559\\
183	0.0124742622029519\\
184	0.0124740459677879\\
185	0.012473827159972\\
186	0.0124736057328432\\
187	0.0124733816384351\\
188	0.0124731548274376\\
189	0.0124729252491587\\
190	0.0124726928514852\\
191	0.0124724575808432\\
192	0.012472219382157\\
193	0.0124719781988085\\
194	0.012471733972595\\
195	0.0124714866436863\\
196	0.0124712361505819\\
197	0.012470982430067\\
198	0.0124707254171673\\
199	0.0124704650451046\\
200	0.0124702012452499\\
201	0.0124699339470775\\
202	0.0124696630781169\\
203	0.0124693885639056\\
204	0.0124691103279396\\
205	0.0124688282916242\\
206	0.0124685423742239\\
207	0.0124682524928114\\
208	0.0124679585622155\\
209	0.012467660494969\\
210	0.0124673582012553\\
211	0.0124670515888537\\
212	0.0124667405630851\\
213	0.012466425026755\\
214	0.0124661048800976\\
215	0.0124657800207173\\
216	0.0124654503435301\\
217	0.0124651157407039\\
218	0.0124647761015979\\
219	0.0124644313127009\\
220	0.0124640812575685\\
221	0.012463725816759\\
222	0.0124633648677655\\
223	0.0124629982849404\\
224	0.0124626259394003\\
225	0.0124622476988772\\
226	0.0124618634274375\\
227	0.0124614729848716\\
228	0.0124610762253366\\
229	0.0124606729945205\\
230	0.0124602631248934\\
231	0.0124598464331194\\
232	0.0124594227435349\\
233	0.012458991978805\\
234	0.0124585539768254\\
235	0.0124581085708109\\
236	0.0124576555905474\\
237	0.0124571948625038\\
238	0.0124567262099996\\
239	0.0124562494534469\\
240	0.0124557644106965\\
241	0.0124552708975352\\
242	0.0124547687284186\\
243	0.0124542577175781\\
244	0.0124537376807467\\
245	0.0124532084378732\\
246	0.0124526698172451\\
247	0.0124521216608792\\
248	0.0124515638282067\\
249	0.0124509961864292\\
250	0.0124504185655711\\
251	0.0124498307268511\\
252	0.0124492325529148\\
253	0.0124486239497388\\
254	0.0124480048444645\\
255	0.0124473751665518\\
256	0.0124467347878819\\
257	0.0124460834245486\\
258	0.0124454210160211\\
259	0.0124447491737744\\
260	0.0124440676550846\\
261	0.0124433762057747\\
262	0.0124426745585101\\
263	0.0124419624323464\\
264	0.0124412395407332\\
265	0.0124405056153793\\
266	0.0124397603516958\\
267	0.0124390034265005\\
268	0.0124382345021435\\
269	0.0124374532256369\\
270	0.0124366592277048\\
271	0.0124358521217202\\
272	0.0124350315024562\\
273	0.012434196944436\\
274	0.0124333479992884\\
275	0.0124324841904358\\
276	0.0124316050006829\\
277	0.0124307098426495\\
278	0.0124297980009628\\
279	0.0124288686203389\\
280	0.0124279212851715\\
281	0.0124269557482776\\
282	0.012425971349151\\
283	0.0124249673909836\\
284	0.0124239431226206\\
285	0.0124228977130159\\
286	0.012421830355329\\
287	0.0124207402963908\\
288	0.0124196269044363\\
289	0.0124184897233157\\
290	0.0124173279624923\\
291	0.0124161389944833\\
292	0.0124149215629971\\
293	0.0124136743190888\\
294	0.0124123958115562\\
295	0.0124110844760014\\
296	0.0124097386222167\\
297	0.0124083564192622\\
298	0.0124069358768688\\
299	0.0124054748201287\\
300	0.0124039708519022\\
301	0.012402421302669\\
302	0.0124008232446478\\
303	0.0123991739232649\\
304	0.0123974700071286\\
305	0.0123957077359013\\
306	0.0123938829443561\\
307	0.0123919910088375\\
308	0.0123900267900115\\
309	0.0123879845731706\\
310	0.0123858579884471\\
311	0.0123836397747981\\
312	0.0123813208717979\\
313	0.0123788894398068\\
314	0.0123763285252103\\
315	0.012370921442076\\
316	0.0123649355844777\\
317	0.0123588561857783\\
318	0.0123526784241549\\
319	0.0123463935582653\\
320	0.0123400051360701\\
321	0.0123335154806087\\
322	0.0123269220701389\\
323	0.0123202227683259\\
324	0.0123134167450671\\
325	0.0123065066784082\\
326	0.0122995006730084\\
327	0.0122923819741245\\
328	0.0122851339179322\\
329	0.0122777516449987\\
330	0.0122702297905421\\
331	0.012262562134915\\
332	0.0122547408923101\\
333	0.0122467586426726\\
334	0.0122386136013522\\
335	0.0122302982138006\\
336	0.0122218044076812\\
337	0.0122131238511732\\
338	0.0122042486884234\\
339	0.012195172287013\\
340	0.0121858797646555\\
341	0.0121763465999835\\
342	0.012166557109027\\
343	0.01215649371462\\
344	0.0121461365985616\\
345	0.0121354631663459\\
346	0.012124446885626\\
347	0.012113053267976\\
348	0.0121012179216949\\
349	0.0120884965445485\\
350	0.0120698539880368\\
351	0.0120509823334755\\
352	0.0120318921016879\\
353	0.0120126267926436\\
354	0.0119931343073231\\
355	0.0119733861784105\\
356	0.0119533717973654\\
357	0.0119330957513884\\
358	0.011912564331459\\
359	0.0118917774008529\\
360	0.011870737833019\\
361	0.0118494585751671\\
362	0.0118279417678206\\
363	0.0118061564915626\\
364	0.0117841039848333\\
365	0.0117617864993389\\
366	0.0117392075201951\\
367	0.0117163720547322\\
368	0.0116932870613034\\
369	0.0116699622532959\\
370	0.0116464116899545\\
371	0.0116226501263682\\
372	0.0115986959567423\\
373	0.0115745717392626\\
374	0.0115503044350653\\
375	0.0115273377065095\\
376	0.0115179002283825\\
377	0.0115084824047774\\
378	0.0114990947983604\\
379	0.0114897490429226\\
380	0.0114804579830704\\
381	0.01147123589981\\
382	0.0114620985812708\\
383	0.0114530634187262\\
384	0.0114441495803473\\
385	0.011435378205776\\
386	0.0114267726226372\\
387	0.0114183585814081\\
388	0.0114101644821448\\
389	0.0114022217175612\\
390	0.0113945650690661\\
391	0.0113872331429737\\
392	0.0113802689147056\\
393	0.0113737203175515\\
394	0.0113676407667043\\
395	0.0113620898826448\\
396	0.0113571343228428\\
397	0.0113523059218674\\
398	0.0113475279039228\\
399	0.0113428035994453\\
400	0.0113381361493827\\
401	0.0113335284315835\\
402	0.0113289829722901\\
403	0.0113245018401987\\
404	0.0113200865209816\\
405	0.0113157377665044\\
406	0.0113114554121766\\
407	0.0113072381568359\\
408	0.0113030832941315\\
409	0.0112989863955218\\
410	0.0112949409309041\\
411	0.0112909378133669\\
412	0.011286964856682\\
413	0.0112830061260558\\
414	0.0112790411617502\\
415	0.0112750602871158\\
416	0.0112710620903373\\
417	0.0112670449604429\\
418	0.0112630070793571\\
419	0.0112589464166073\\
420	0.0112548607278733\\
421	0.0112507475588533\\
422	0.0112466042564819\\
423	0.0112424279896433\\
424	0.0112382157822969\\
425	0.011233964562621\\
426	0.0112296712327711\\
427	0.0112253327650429\\
428	0.0112209463317157\\
429	0.0112165094777162\\
430	0.0112120203475584\\
431	0.0112074775595681\\
432	0.0112028797061604\\
433	0.0111982253596221\\
434	0.011193513078722\\
435	0.0111887414160902\\
436	0.0111839089262452\\
437	0.0111790141740625\\
438	0.0111740557433569\\
439	0.0111690322450853\\
440	0.0111639423244585\\
441	0.0111587846659548\\
442	0.0111535579948385\\
443	0.0111482610732717\\
444	0.0111428926884298\\
445	0.0111374516291415\\
446	0.01113193667288\\
447	0.0111263465862123\\
448	0.0111206801250633\\
449	0.0111149360347422\\
450	0.0111091130496806\\
451	0.0111032098928331\\
452	0.0110972252746971\\
453	0.0110911578919272\\
454	0.0110850064255433\\
455	0.0110787695387785\\
456	0.0110724458746793\\
457	0.0110660340536702\\
458	0.0110595326714435\\
459	0.0110529402977432\\
460	0.0110462554753646\\
461	0.0110394767190718\\
462	0.0110326025144355\\
463	0.011025631316593\\
464	0.0110185615489396\\
465	0.011011391601758\\
466	0.0110041198307999\\
467	0.0109967445558312\\
468	0.0109892640591567\\
469	0.010981676584134\\
470	0.0109739803336804\\
471	0.010966173468764\\
472	0.0109582541068399\\
473	0.0109502203202189\\
474	0.0109420701343649\\
475	0.0109338015261205\\
476	0.0109254124218581\\
477	0.0109169006955545\\
478	0.0109082641667855\\
479	0.0108995005986375\\
480	0.0108906076955305\\
481	0.0108815831009482\\
482	0.0108724243950681\\
483	0.0108631290922849\\
484	0.0108536946386212\\
485	0.0108441184090199\\
486	0.0108343977045128\\
487	0.0108245297492601\\
488	0.0108145116874535\\
489	0.0108043405800779\\
490	0.0107940134015238\\
491	0.0107835270360439\\
492	0.0107728782740463\\
493	0.0107620638082176\\
494	0.0107510802294666\\
495	0.010739924022682\\
496	0.0107285915622965\\
497	0.0107170791076466\\
498	0.0107053827981228\\
499	0.0106934986480998\\
500	0.0106814225416374\\
501	0.0106691502269463\\
502	0.0106566773106061\\
503	0.0106439992515312\\
504	0.0106311113546724\\
505	0.010618008764449\\
506	0.0106046864579025\\
507	0.0105911392375658\\
508	0.0105773617240414\\
509	0.0105633483482841\\
510	0.0105490933451519\\
511	0.0105345907453926\\
512	0.0105198343666647\\
513	0.0105048177690006\\
514	0.0104895342397086\\
515	0.0104739768385048\\
516	0.010458135053507\\
517	0.010442004924358\\
518	0.0104255816028362\\
519	0.0104088564084067\\
520	0.0103918166774931\\
521	0.0103744590490279\\
522	0.0103567773162321\\
523	0.01033876119676\\
524	0.0103204077833572\\
525	0.0103017001063452\\
526	0.0102826195717509\\
527	0.0102631558089421\\
528	0.0102432975992738\\
529	0.0102230319184114\\
530	0.0102023414680112\\
531	0.0101812073683544\\
532	0.0101596464122478\\
533	0.0101376354415696\\
534	0.0101151618026545\\
535	0.0100922424281249\\
536	0.010068876465799\\
537	0.0100450542649095\\
538	0.0100206969305763\\
539	0.00999579045777402\\
540	0.00997031114870067\\
541	0.00994422540220995\\
542	0.00991751576362549\\
543	0.00989016716918039\\
544	0.0098621643912705\\
545	0.00983349114382052\\
546	0.00980401782346135\\
547	0.00977388035971209\\
548	0.00974295049822557\\
549	0.00971143590720675\\
550	0.00967961836267487\\
551	0.00964711305777354\\
552	0.00961380442103104\\
553	0.00957964880248824\\
554	0.00954461545166724\\
555	0.00950867526035693\\
556	0.00947179845028549\\
557	0.00943395392448495\\
558	0.00939495787475029\\
559	0.00935457285869148\\
560	0.00931275940407491\\
561	0.00927137351937616\\
562	0.00923024825685196\\
563	0.00918803432212118\\
564	0.00914466123074735\\
565	0.00906001065198841\\
566	0.00880282936135181\\
567	0.00838176684114641\\
568	0.00831367114185934\\
569	0.00824828975106419\\
570	0.00818175587048123\\
571	0.00811402083827017\\
572	0.00804504960790682\\
573	0.00797480573282636\\
574	0.00790325081764849\\
575	0.00783034432666621\\
576	0.00775604341680255\\
577	0.00768030276212431\\
578	0.00760307436573265\\
579	0.00752430735536871\\
580	0.00744394776028546\\
581	0.00736193826363283\\
582	0.00727821791754183\\
583	0.00719272178875225\\
584	0.0071053804513275\\
585	0.00701611910584812\\
586	0.00692485573802268\\
587	0.00683149674994525\\
588	0.00673592587771182\\
589	0.00663797520465072\\
590	0.00653734834917363\\
591	0.00643341581933105\\
592	0.00632466858477865\\
593	0.00620725772627071\\
594	0.0060710901180166\\
595	0.00588956027306328\\
596	0.00559184182825266\\
597	0.00498881296807124\\
598	0.00357511483354343\\
599	0\\
600	0\\
};
\addplot [color=mycolor3,solid,forget plot]
  table[row sep=crcr]{%
1	0.0124900117213761\\
2	0.0124900005757721\\
3	0.0124899892213819\\
4	0.0124899776547542\\
5	0.0124899658723958\\
6	0.0124899538707719\\
7	0.0124899416463056\\
8	0.0124899291953785\\
9	0.0124899165143306\\
10	0.0124899035994603\\
11	0.0124898904470247\\
12	0.0124898770532395\\
13	0.0124898634142798\\
14	0.0124898495262796\\
15	0.0124898353853323\\
16	0.0124898209874914\\
17	0.0124898063287702\\
18	0.0124897914051424\\
19	0.0124897762125426\\
20	0.0124897607468664\\
21	0.0124897450039712\\
22	0.0124897289796763\\
23	0.0124897126697639\\
24	0.0124896960699789\\
25	0.0124896791760304\\
26	0.0124896619835914\\
27	0.0124896444883002\\
28	0.0124896266857606\\
29	0.0124896085715428\\
30	0.0124895901411842\\
31	0.0124895713901902\\
32	0.0124895523140351\\
33	0.0124895329081626\\
34	0.0124895131679874\\
35	0.0124894930888953\\
36	0.0124894726662452\\
37	0.0124894518953693\\
38	0.0124894307715744\\
39	0.0124894092901435\\
40	0.0124893874463362\\
41	0.0124893652353904\\
42	0.0124893426525235\\
43	0.0124893196929333\\
44	0.0124892963517999\\
45	0.0124892726242864\\
46	0.0124892485055408\\
47	0.0124892239906969\\
48	0.0124891990748763\\
49	0.0124891737531894\\
50	0.0124891480207372\\
51	0.0124891218726124\\
52	0.0124890953039015\\
53	0.012489068309686\\
54	0.0124890408850438\\
55	0.0124890130250512\\
56	0.0124889847247843\\
57	0.0124889559793207\\
58	0.0124889267837408\\
59	0.0124888971331298\\
60	0.0124888670225789\\
61	0.0124888364471874\\
62	0.0124888054020637\\
63	0.0124887738823273\\
64	0.0124887418831101\\
65	0.012488709399558\\
66	0.0124886764268322\\
67	0.0124886429601109\\
68	0.0124886089945903\\
69	0.0124885745254862\\
70	0.012488539548035\\
71	0.0124885040574953\\
72	0.0124884680491484\\
73	0.0124884315182997\\
74	0.0124883944602793\\
75	0.0124883568704432\\
76	0.0124883187441736\\
77	0.0124882800768796\\
78	0.0124882408639976\\
79	0.0124882011009915\\
80	0.012488160783353\\
81	0.0124881199066013\\
82	0.012488078466283\\
83	0.0124880364579716\\
84	0.012487993877267\\
85	0.0124879507197948\\
86	0.0124879069812049\\
87	0.0124878626571703\\
88	0.0124878177433859\\
89	0.0124877722355667\\
90	0.0124877261294453\\
91	0.0124876794207702\\
92	0.0124876321053031\\
93	0.0124875841788157\\
94	0.0124875356370873\\
95	0.0124874864759004\\
96	0.012487436691038\\
97	0.0124873862782788\\
98	0.0124873352333933\\
99	0.0124872835521386\\
100	0.0124872312302543\\
101	0.0124871782634561\\
102	0.012487124647431\\
103	0.012487070377831\\
104	0.0124870154502668\\
105	0.0124869598603013\\
106	0.0124869036034428\\
107	0.0124868466751375\\
108	0.0124867890707623\\
109	0.012486730785617\\
110	0.0124866718149159\\
111	0.0124866121537802\\
112	0.0124865517972289\\
113	0.0124864907401702\\
114	0.0124864289773925\\
115	0.0124863665035556\\
116	0.0124863033131811\\
117	0.0124862394006428\\
118	0.0124861747601577\\
119	0.012486109385776\\
120	0.0124860432713715\\
121	0.0124859764106317\\
122	0.0124859087970485\\
123	0.0124858404239081\\
124	0.0124857712842813\\
125	0.0124857013710141\\
126	0.012485630676718\\
127	0.0124855591937606\\
128	0.012485486914256\\
129	0.0124854138300561\\
130	0.0124853399327408\\
131	0.0124852652136095\\
132	0.0124851896636722\\
133	0.0124851132736404\\
134	0.012485036033919\\
135	0.0124849579345971\\
136	0.01248487896544\\
137	0.0124847991158804\\
138	0.0124847183750098\\
139	0.0124846367315694\\
140	0.0124845541739403\\
141	0.0124844706901321\\
142	0.0124843862677663\\
143	0.0124843008940506\\
144	0.0124842145557326\\
145	0.0124841272390174\\
146	0.0124840389294419\\
147	0.0124839496117726\\
148	0.0124838592702673\\
149	0.0124837678900993\\
150	0.0124836754594427\\
151	0.0124835819624927\\
152	0.012483487382789\\
153	0.0124833917035093\\
154	0.012483294907461\\
155	0.0124831969770725\\
156	0.0124830978943846\\
157	0.012482997641042\\
158	0.0124828961982841\\
159	0.012482793546936\\
160	0.0124826896673999\\
161	0.0124825845396455\\
162	0.0124824781432011\\
163	0.0124823704571443\\
164	0.0124822614600923\\
165	0.0124821511301933\\
166	0.0124820394451161\\
167	0.0124819263820413\\
168	0.0124818119176518\\
169	0.0124816960281228\\
170	0.0124815786891126\\
171	0.0124814598757532\\
172	0.0124813395626401\\
173	0.0124812177238238\\
174	0.0124810943327992\\
175	0.012480969362497\\
176	0.0124808427852736\\
177	0.0124807145729023\\
178	0.0124805846965635\\
179	0.0124804531268356\\
180	0.0124803198336859\\
181	0.0124801847864616\\
182	0.0124800479538802\\
183	0.0124799093040214\\
184	0.0124797688043175\\
185	0.0124796264215453\\
186	0.0124794821218172\\
187	0.0124793358705726\\
188	0.0124791876325697\\
189	0.0124790373718772\\
190	0.0124788850518661\\
191	0.0124787306352014\\
192	0.0124785740838347\\
193	0.0124784153589955\\
194	0.0124782544211841\\
195	0.0124780912301634\\
196	0.0124779257449516\\
197	0.012477757923814\\
198	0.0124775877242555\\
199	0.0124774151030131\\
200	0.0124772400160479\\
201	0.0124770624185374\\
202	0.0124768822648671\\
203	0.0124766995086228\\
204	0.0124765141025823\\
205	0.0124763259987061\\
206	0.0124761351481293\\
207	0.0124759415011515\\
208	0.0124757450072277\\
209	0.0124755456149576\\
210	0.0124753432720753\\
211	0.0124751379254372\\
212	0.0124749295210105\\
213	0.0124747180038601\\
214	0.0124745033181348\\
215	0.0124742854070527\\
216	0.0124740642128855\\
217	0.0124738396769417\\
218	0.0124736117395489\\
219	0.0124733803400345\\
220	0.0124731454167046\\
221	0.0124729069068219\\
222	0.0124726647465798\\
223	0.0124724188710718\\
224	0.0124721692142512\\
225	0.0124719157088681\\
226	0.0124716582863629\\
227	0.0124713968766753\\
228	0.0124711314079379\\
229	0.0124708618061522\\
230	0.0124705879954977\\
231	0.0124703099010313\\
232	0.0124700274534102\\
233	0.0124697405752607\\
234	0.0124694491876557\\
235	0.0124691532101689\\
236	0.0124688525608331\\
237	0.0124685471560972\\
238	0.0124682369107845\\
239	0.0124679217380536\\
240	0.0124676015493646\\
241	0.0124672762544591\\
242	0.0124669457613653\\
243	0.0124666099764479\\
244	0.0124662688045272\\
245	0.0124659221490664\\
246	0.012465569912315\\
247	0.0124652119949602\\
248	0.0124648482942536\\
249	0.0124644786999544\\
250	0.0124641030937212\\
251	0.0124637213620397\\
252	0.0124633333870466\\
253	0.0124629390444868\\
254	0.0124625381996382\\
255	0.0124621307008008\\
256	0.0124617163777749\\
257	0.0124612950843974\\
258	0.0124608667782466\\
259	0.0124604312930264\\
260	0.0124599884577046\\
261	0.0124595380963293\\
262	0.012459080028112\\
263	0.0124586140683\\
264	0.0124581400293418\\
265	0.0124576577162257\\
266	0.012457166928084\\
267	0.0124566674584745\\
268	0.0124561590952074\\
269	0.012455641620162\\
270	0.0124551148090884\\
271	0.0124545784313791\\
272	0.0124540322497752\\
273	0.0124534760199154\\
274	0.0124529094895125\\
275	0.0124523323967338\\
276	0.0124517444672789\\
277	0.0124511454112616\\
278	0.0124505349298039\\
279	0.0124499127552903\\
280	0.0124492786218922\\
281	0.0124486322248941\\
282	0.0124479732504704\\
283	0.0124473013740687\\
284	0.0124466162600645\\
285	0.0124459175714466\\
286	0.0124452049726032\\
287	0.0124444781329385\\
288	0.0124437367206503\\
289	0.0124429803494812\\
290	0.0124422085043243\\
291	0.0124414207738386\\
292	0.0124406167344618\\
293	0.0124397959501881\\
294	0.01243895797239\\
295	0.0124381023396796\\
296	0.0124372285777689\\
297	0.0124363361992258\\
298	0.0124354247029345\\
299	0.0124344935732505\\
300	0.012433542280599\\
301	0.0124325702919805\\
302	0.012431577101701\\
303	0.0124305621744038\\
304	0.0124295249666692\\
305	0.0124284649379104\\
306	0.0124273815564308\\
307	0.0124262743074417\\
308	0.0124251427026119\\
309	0.0124239862867158\\
310	0.0124228046244221\\
311	0.0124215972348075\\
312	0.0124203635229112\\
313	0.0124191026464338\\
314	0.0124178134236839\\
315	0.0124164985515049\\
316	0.0124151576251906\\
317	0.0124137893669169\\
318	0.0124123922899686\\
319	0.01241096579974\\
320	0.0124095091010523\\
321	0.0124080209768983\\
322	0.0124065002097119\\
323	0.0124049456755524\\
324	0.012403356474694\\
325	0.0124017317356611\\
326	0.0124000685809139\\
327	0.0123983641545376\\
328	0.012396616370332\\
329	0.0123948229548816\\
330	0.0123929814020112\\
331	0.0123910889404741\\
332	0.012389142736172\\
333	0.0123871401553977\\
334	0.0123850778474462\\
335	0.0123829521799028\\
336	0.012380759239627\\
337	0.0123784948415397\\
338	0.0123761544033203\\
339	0.0123737321287599\\
340	0.0123712211093611\\
341	0.0123686146560381\\
342	0.0123659053050938\\
343	0.0123630846840279\\
344	0.0123601433096615\\
345	0.0123570702010703\\
346	0.0123538519145613\\
347	0.0123504697763404\\
348	0.012346894154993\\
349	0.0123427558617095\\
350	0.0123338818010271\\
351	0.0123248706417058\\
352	0.0123157265279167\\
353	0.0123064608165831\\
354	0.0122970436401246\\
355	0.0122874567199428\\
356	0.0122776905100024\\
357	0.0122677416625208\\
358	0.0122576063654486\\
359	0.0122472774433936\\
360	0.0122367486737082\\
361	0.0122260158450916\\
362	0.0122150663140686\\
363	0.0122038739015477\\
364	0.0121924251772472\\
365	0.012180705271155\\
366	0.0121686976221371\\
367	0.0121563835550873\\
368	0.0121437411931459\\
369	0.0121307408423981\\
370	0.0121173316477296\\
371	0.0121035191999451\\
372	0.0120892716440285\\
373	0.012074555217779\\
374	0.0120593399796327\\
375	0.0120428858819619\\
376	0.0120186426323913\\
377	0.0119940963925576\\
378	0.0119692490533195\\
379	0.0119441049902936\\
380	0.0119186860346378\\
381	0.0118929643586608\\
382	0.0118669198083094\\
383	0.0118405519421216\\
384	0.0118138611523436\\
385	0.0117868488637125\\
386	0.0117595178276716\\
387	0.0117318726798215\\
388	0.0117039214955012\\
389	0.0116756733381913\\
390	0.0116471384998108\\
391	0.0116183295054687\\
392	0.0115892603065783\\
393	0.0115599481408537\\
394	0.0115304191576603\\
395	0.01150070591498\\
396	0.0114708489372037\\
397	0.0114571246110284\\
398	0.0114461287698232\\
399	0.0114352074925657\\
400	0.0114243793106374\\
401	0.0114136646750546\\
402	0.0114030861647579\\
403	0.0113926687125702\\
404	0.0113824398248891\\
405	0.0113724298927981\\
406	0.0113626725928505\\
407	0.0113532053208341\\
408	0.0113440697912537\\
409	0.0113353124306812\\
410	0.0113269849679598\\
411	0.0113191451465055\\
412	0.0113118574369489\\
413	0.0113051939321546\\
414	0.0112992353672767\\
415	0.0112936057619535\\
416	0.0112880377296427\\
417	0.0112825347915983\\
418	0.0112771000917152\\
419	0.0112717362787034\\
420	0.0112664453606393\\
421	0.0112612285277664\\
422	0.0112560859330031\\
423	0.0112510164344459\\
424	0.0112460172862472\\
425	0.0112410837696491\\
426	0.0112362087507133\\
427	0.0112313821499731\\
428	0.0112265903065674\\
429	0.0112218152161048\\
430	0.0112170336179759\\
431	0.0112122278156015\\
432	0.0112073958616761\\
433	0.0112025355606028\\
434	0.0111976444619761\\
435	0.0111927198591086\\
436	0.0111877587942\\
437	0.0111827580723855\\
438	0.0111777142875351\\
439	0.0111726238634414\\
440	0.0111674831149901\\
441	0.0111622883351075\\
442	0.0111570359147843\\
443	0.0111517225053463\\
444	0.0111463452344593\\
445	0.0111409019903933\\
446	0.0111353910541228\\
447	0.0111298106795012\\
448	0.0111241591009583\\
449	0.0111184345421164\\
450	0.0111126352251774\\
451	0.0111067593808233\\
452	0.0111008052582152\\
453	0.0110947711344761\\
454	0.0110886553227582\\
455	0.011082456177628\\
456	0.0110761720959998\\
457	0.0110698015111858\\
458	0.0110633428767625\\
459	0.0110567946358115\\
460	0.0110501552119634\\
461	0.0110434230094554\\
462	0.0110365964129051\\
463	0.0110296737867357\\
464	0.0110226534741914\\
465	0.0110155337958875\\
466	0.0110083130478594\\
467	0.0110009894991038\\
468	0.0109935613886583\\
469	0.0109860269223436\\
470	0.0109783842694136\\
471	0.0109706315595301\\
472	0.0109627668807314\\
473	0.0109547882777024\\
474	0.0109466937499229\\
475	0.0109384812496959\\
476	0.0109301486800601\\
477	0.0109216938925958\\
478	0.0109131146851347\\
479	0.0109044087993883\\
480	0.0108955739185107\\
481	0.0108866076646103\\
482	0.0108775075962197\\
483	0.010868271205721\\
484	0.0108588959166984\\
485	0.0108493790811731\\
486	0.010839717976718\\
487	0.0108299098034481\\
488	0.0108199516808828\\
489	0.0108098406446772\\
490	0.0107995736432165\\
491	0.0107891475340695\\
492	0.0107785590802925\\
493	0.010767804946579\\
494	0.0107568816952437\\
495	0.0107457857820342\\
496	0.0107345135517603\\
497	0.0107230612337341\\
498	0.0107114249370147\\
499	0.0106996006454486\\
500	0.0106875842125\\
501	0.0106753713558623\\
502	0.0106629576518436\\
503	0.0106503385295184\\
504	0.0106375092646387\\
505	0.0106244649732957\\
506	0.0106112006053275\\
507	0.0105977109374638\\
508	0.0105839905662037\\
509	0.0105700339004211\\
510	0.0105558351536465\\
511	0.0105413883360576\\
512	0.0105266872461938\\
513	0.0105117254633539\\
514	0.0104964963398891\\
515	0.0104809929917471\\
516	0.0104652082888149\\
517	0.0104491347938747\\
518	0.0104327647825303\\
519	0.0104160902676691\\
520	0.0103990993857909\\
521	0.0103817877930226\\
522	0.0103641489966369\\
523	0.0103461731401486\\
524	0.0103278468408664\\
525	0.0103091639362961\\
526	0.0102901179732998\\
527	0.0102706974394484\\
528	0.0102508949463511\\
529	0.0102306996381357\\
530	0.0102100867601676\\
531	0.0101890440677434\\
532	0.0101675595467004\\
533	0.0101456104950239\\
534	0.0101231826669715\\
535	0.010100287680872\\
536	0.0100769016474549\\
537	0.0100530087982048\\
538	0.0100286259503887\\
539	0.0100037425761063\\
540	0.00997837563815114\\
541	0.00995241604058662\\
542	0.00992584375934148\\
543	0.00989863439054779\\
544	0.00987077093169437\\
545	0.00984223690468954\\
546	0.00981301526993071\\
547	0.00978308845229256\\
548	0.0097523199963511\\
549	0.00972086204257419\\
550	0.00968857864483313\\
551	0.00965565846421692\\
552	0.00962233801165364\\
553	0.00958835066292521\\
554	0.00955352863102991\\
555	0.00951781367241968\\
556	0.00948117010442207\\
557	0.00944356668974041\\
558	0.00940497068589146\\
559	0.00936534712600867\\
560	0.00932413031827952\\
561	0.00928171998247948\\
562	0.00923840442122668\\
563	0.00919625530291374\\
564	0.00915316897197163\\
565	0.00910890374113917\\
566	0.00902976035098331\\
567	0.00880259997933173\\
568	0.0083478056989328\\
569	0.00824844307218399\\
570	0.00818176236454212\\
571	0.00811402171848672\\
572	0.00804504977729814\\
573	0.00797480577650455\\
574	0.00790325083332592\\
575	0.0078303443341488\\
576	0.00775604342024438\\
577	0.00768030276391219\\
578	0.00760307436664889\\
579	0.00752430735584519\\
580	0.007443947760513\\
581	0.0073619382637318\\
582	0.00727821791757129\\
583	0.00719272178875675\\
584	0.00710538045132748\\
585	0.00701611910584813\\
586	0.00692485573802266\\
587	0.00683149674994524\\
588	0.00673592587771182\\
589	0.00663797520465071\\
590	0.00653734834917364\\
591	0.00643341581933105\\
592	0.00632466858477866\\
593	0.00620725772627071\\
594	0.00607109011801661\\
595	0.00588956027306329\\
596	0.00559184182825266\\
597	0.00498881296807123\\
598	0.00357511483354343\\
599	0\\
600	0\\
};
\addplot [color=mycolor4,solid,forget plot]
  table[row sep=crcr]{%
1	0.0124901248826776\\
2	0.0124901165401922\\
3	0.0124901080555749\\
4	0.0124900994268116\\
5	0.0124900906518713\\
6	0.012490081728706\\
7	0.0124900726552512\\
8	0.0124900634294259\\
9	0.0124900540491329\\
10	0.0124900445122589\\
11	0.012490034816675\\
12	0.0124900249602366\\
13	0.0124900149407839\\
14	0.0124900047561419\\
15	0.0124899944041212\\
16	0.0124899838825176\\
17	0.0124899731891129\\
18	0.0124899623216748\\
19	0.0124899512779577\\
20	0.0124899400557024\\
21	0.012489928652637\\
22	0.0124899170664768\\
23	0.012489905294925\\
24	0.0124898933356726\\
25	0.0124898811863992\\
26	0.0124898688447731\\
27	0.0124898563084516\\
28	0.0124898435750815\\
29	0.0124898306422995\\
30	0.0124898175077324\\
31	0.0124898041689977\\
32	0.0124897906237037\\
33	0.01248977686945\\
34	0.0124897629038278\\
35	0.0124897487244206\\
36	0.0124897343288039\\
37	0.0124897197145463\\
38	0.0124897048792093\\
39	0.0124896898203477\\
40	0.0124896745355104\\
41	0.0124896590222401\\
42	0.0124896432780741\\
43	0.0124896273005443\\
44	0.0124896110871775\\
45	0.0124895946354959\\
46	0.0124895779430171\\
47	0.0124895610072544\\
48	0.0124895438257171\\
49	0.0124895263959105\\
50	0.0124895087153361\\
51	0.0124894907814918\\
52	0.0124894725918722\\
53	0.0124894541439679\\
54	0.0124894354352663\\
55	0.0124894164632514\\
56	0.0124893972254034\\
57	0.0124893777191988\\
58	0.0124893579421104\\
59	0.0124893378916069\\
60	0.0124893175651526\\
61	0.0124892969602073\\
62	0.0124892760742256\\
63	0.0124892549046567\\
64	0.0124892334489439\\
65	0.012489211704524\\
66	0.0124891896688266\\
67	0.0124891673392731\\
68	0.0124891447132767\\
69	0.0124891217882407\\
70	0.012489098561558\\
71	0.0124890750306098\\
72	0.0124890511927648\\
73	0.0124890270453776\\
74	0.0124890025857876\\
75	0.0124889778113176\\
76	0.0124889527192721\\
77	0.012488927306936\\
78	0.0124889015715725\\
79	0.0124888755104216\\
80	0.0124888491206979\\
81	0.012488822399589\\
82	0.0124887953442526\\
83	0.0124887679518152\\
84	0.0124887402193688\\
85	0.0124887121439692\\
86	0.0124886837226327\\
87	0.0124886549523341\\
88	0.0124886258300033\\
89	0.0124885963525227\\
90	0.0124885665167242\\
91	0.0124885363193858\\
92	0.0124885057572285\\
93	0.012488474826913\\
94	0.0124884435250364\\
95	0.0124884118481283\\
96	0.0124883797926474\\
97	0.0124883473549778\\
98	0.0124883145314252\\
99	0.012488281318213\\
100	0.0124882477114785\\
101	0.0124882137072685\\
102	0.0124881793015358\\
103	0.0124881444901349\\
104	0.0124881092688176\\
105	0.0124880736332292\\
106	0.0124880375789042\\
107	0.0124880011012621\\
108	0.0124879641956029\\
109	0.0124879268571035\\
110	0.0124878890808128\\
111	0.0124878508616483\\
112	0.0124878121943913\\
113	0.012487773073683\\
114	0.0124877334940207\\
115	0.0124876934497535\\
116	0.0124876529350784\\
117	0.0124876119440367\\
118	0.0124875704705098\\
119	0.012487528508216\\
120	0.0124874860507064\\
121	0.0124874430913617\\
122	0.0124873996233887\\
123	0.012487355639817\\
124	0.0124873111334956\\
125	0.0124872660970903\\
126	0.0124872205230798\\
127	0.0124871744037537\\
128	0.0124871277312092\\
129	0.0124870804973482\\
130	0.012487032693875\\
131	0.0124869843122935\\
132	0.0124869353439047\\
133	0.0124868857798043\\
134	0.0124868356108801\\
135	0.0124867848278096\\
136	0.0124867334210577\\
137	0.0124866813808744\\
138	0.0124866286972919\\
139	0.0124865753601224\\
140	0.0124865213589551\\
141	0.0124864666831522\\
142	0.0124864113218445\\
143	0.0124863552639239\\
144	0.0124862984980327\\
145	0.0124862410125527\\
146	0.0124861827956046\\
147	0.0124861238350929\\
148	0.012486064118833\\
149	0.0124860036346186\\
150	0.0124859423696853\\
151	0.012485880310991\\
152	0.0124858174452364\\
153	0.0124857537588613\\
154	0.0124856892380398\\
155	0.0124856238686756\\
156	0.0124855576363979\\
157	0.0124854905265568\\
158	0.0124854225242184\\
159	0.0124853536141608\\
160	0.0124852837808688\\
161	0.01248521300853\\
162	0.0124851412810294\\
163	0.0124850685819453\\
164	0.0124849948945443\\
165	0.0124849202017767\\
166	0.0124848444862714\\
167	0.0124847677303316\\
168	0.0124846899159295\\
169	0.0124846110247015\\
170	0.0124845310379437\\
171	0.0124844499366062\\
172	0.0124843677012886\\
173	0.012484284312235\\
174	0.0124841997493283\\
175	0.0124841139920857\\
176	0.0124840270196528\\
177	0.0124839388107986\\
178	0.0124838493439101\\
179	0.0124837585969864\\
180	0.0124836665476336\\
181	0.0124835731730583\\
182	0.0124834784500624\\
183	0.0124833823550366\\
184	0.0124832848639545\\
185	0.0124831859523657\\
186	0.01248308559539\\
187	0.0124829837677099\\
188	0.0124828804435639\\
189	0.0124827755967394\\
190	0.0124826692005647\\
191	0.0124825612279017\\
192	0.0124824516511376\\
193	0.0124823404421762\\
194	0.0124822275724294\\
195	0.0124821130128081\\
196	0.012481996733712\\
197	0.0124818787050203\\
198	0.0124817588960806\\
199	0.0124816372756983\\
200	0.0124815138121247\\
201	0.0124813884730453\\
202	0.012481261225567\\
203	0.0124811320362044\\
204	0.0124810008708668\\
205	0.0124808676948427\\
206	0.0124807324727849\\
207	0.0124805951686948\\
208	0.0124804557459049\\
209	0.0124803141670621\\
210	0.0124801703941086\\
211	0.0124800243882632\\
212	0.0124798761100013\\
213	0.0124797255190338\\
214	0.0124795725742855\\
215	0.0124794172338726\\
216	0.0124792594550789\\
217	0.0124790991943316\\
218	0.0124789364071762\\
219	0.0124787710482499\\
220	0.0124786030712543\\
221	0.0124784324289277\\
222	0.0124782590730148\\
223	0.0124780829542349\\
224	0.012477904022247\\
225	0.0124777222256086\\
226	0.0124775375117266\\
227	0.0124773498268022\\
228	0.0124771591157911\\
229	0.0124769653224477\\
230	0.0124767683895453\\
231	0.012476568259038\\
232	0.0124763648710206\\
233	0.0124761581642267\\
234	0.0124759480760004\\
235	0.0124757345422605\\
236	0.0124755174974643\\
237	0.0124752968745714\\
238	0.0124750726050081\\
239	0.0124748446186323\\
240	0.0124746128437004\\
241	0.0124743772068374\\
242	0.0124741376330116\\
243	0.0124738940455159\\
244	0.0124736463659501\\
245	0.0124733945141879\\
246	0.0124731384082771\\
247	0.0124728779642054\\
248	0.0124726130955754\\
249	0.0124723437137052\\
250	0.0124720697285101\\
251	0.0124717910477025\\
252	0.012471507576558\\
253	0.0124712192175515\\
254	0.0124709258700459\\
255	0.012470627431071\\
256	0.012470323799531\\
257	0.0124700148796904\\
258	0.0124697005642099\\
259	0.0124693807436389\\
260	0.0124690553063781\\
261	0.0124687241386772\\
262	0.012468387124689\\
263	0.0124680441464679\\
264	0.0124676950835441\\
265	0.0124673398130423\\
266	0.0124669782096682\\
267	0.0124666101456524\\
268	0.0124662354906901\\
269	0.0124658541118766\\
270	0.0124654658736353\\
271	0.0124650706376342\\
272	0.0124646682626768\\
273	0.0124642586045469\\
274	0.0124638415157779\\
275	0.0124634168453613\\
276	0.0124629844386467\\
277	0.0124625441383024\\
278	0.0124620957869119\\
279	0.0124616392239826\\
280	0.0124611742830414\\
281	0.0124607007939315\\
282	0.0124602185826081\\
283	0.0124597274711809\\
284	0.0124592272786975\\
285	0.0124587178212171\\
286	0.0124581989116626\\
287	0.0124576703582858\\
288	0.0124571319598345\\
289	0.0124565835026562\\
290	0.0124560247774795\\
291	0.0124554555699073\\
292	0.0124548756602314\\
293	0.0124542848232372\\
294	0.0124536828279935\\
295	0.0124530694376166\\
296	0.0124524444089975\\
297	0.0124518074924801\\
298	0.0124511584315414\\
299	0.0124504969627406\\
300	0.0124498228165644\\
301	0.0124491357187345\\
302	0.0124484353848962\\
303	0.0124477215220361\\
304	0.0124469938286734\\
305	0.0124462519944237\\
306	0.0124454956994415\\
307	0.0124447246134749\\
308	0.0124439383937158\\
309	0.0124431366796135\\
310	0.012442319082969\\
311	0.0124414851786714\\
312	0.0124406344982282\\
313	0.0124397665622936\\
314	0.0124388811282895\\
315	0.0124379777391157\\
316	0.0124370558673665\\
317	0.0124361149707234\\
318	0.0124351545618017\\
319	0.0124341741192239\\
320	0.0124331730767785\\
321	0.0124321508580897\\
322	0.0124311068838852\\
323	0.0124300405738951\\
324	0.012428951307811\\
325	0.0124278382979973\\
326	0.0124267007582024\\
327	0.0124255379479097\\
328	0.0124243490966155\\
329	0.0124231334013247\\
330	0.012421890029075\\
331	0.0124206181379127\\
332	0.0124193168877339\\
333	0.0124179853703915\\
334	0.0124166226517812\\
335	0.0124152277755932\\
336	0.0124137997648297\\
337	0.0124123376054178\\
338	0.0124108401907023\\
339	0.0124093063615074\\
340	0.0124077350176528\\
341	0.0124061250444717\\
342	0.0124044753165431\\
343	0.0124027846952309\\
344	0.0124010520051072\\
345	0.0123992759537307\\
346	0.012397454948314\\
347	0.0123955870131983\\
348	0.0123936714362393\\
349	0.0123917066494807\\
350	0.0123896971275026\\
351	0.0123876415631515\\
352	0.0123855385826638\\
353	0.0123833836191224\\
354	0.0123811729224236\\
355	0.0123789033123675\\
356	0.0123765719355767\\
357	0.0123741756328625\\
358	0.0123717107336897\\
359	0.0123691733800568\\
360	0.0123665594116891\\
361	0.0123638634374432\\
362	0.0123610787492131\\
363	0.0123581992738843\\
364	0.0123552183015072\\
365	0.0123521283812704\\
366	0.0123489211592767\\
367	0.0123455870488059\\
368	0.0123421143719694\\
369	0.0123384870582913\\
370	0.0123346833061634\\
371	0.0123307044088364\\
372	0.0123265351574494\\
373	0.0123221600587547\\
374	0.0123175651069471\\
375	0.0123121181459393\\
376	0.0123004270121501\\
377	0.0122885332526515\\
378	0.0122764313610203\\
379	0.0122641167566825\\
380	0.0122515875703521\\
381	0.012238819515977\\
382	0.0122257920709943\\
383	0.0122124920527585\\
384	0.0121989049264763\\
385	0.0121850145044644\\
386	0.012170802221424\\
387	0.0121562443704533\\
388	0.0121412961328583\\
389	0.0121259490876007\\
390	0.0121101932839557\\
391	0.0120940059810074\\
392	0.0120773782769752\\
393	0.0120602799353118\\
394	0.0120426225520414\\
395	0.0120243592889549\\
396	0.0120054477854116\\
397	0.0119772530535398\\
398	0.0119472087565024\\
399	0.0119167897449176\\
400	0.0118859945880998\\
401	0.01185482266128\\
402	0.0118232743914859\\
403	0.0117913517432048\\
404	0.0117590596238437\\
405	0.0117264040495483\\
406	0.0116933909071998\\
407	0.0116600270972709\\
408	0.0116263176559654\\
409	0.0115922756940728\\
410	0.0115579186717889\\
411	0.0115232682246787\\
412	0.0114883546096828\\
413	0.0114532158136441\\
414	0.0114178997337225\\
415	0.0113962785871438\\
416	0.0113837072405572\\
417	0.0113712561017825\\
418	0.011358950878782\\
419	0.0113468199248681\\
420	0.0113348946539114\\
421	0.0113232099670389\\
422	0.0113118048583576\\
423	0.0113007227250728\\
424	0.0112900119126424\\
425	0.011279726281361\\
426	0.0112699259081022\\
427	0.011260677901755\\
428	0.0112520573444275\\
429	0.011244148378936\\
430	0.0112370454582933\\
431	0.0112305161061403\\
432	0.0112240566039267\\
433	0.0112176705630141\\
434	0.0112113610049963\\
435	0.0112051301763297\\
436	0.0111989793318932\\
437	0.0111929084752741\\
438	0.0111869160461271\\
439	0.0111809985440657\\
440	0.0111751500764268\\
441	0.0111693618147172\\
442	0.0111636213415874\\
443	0.011157911866852\\
444	0.0111522112879962\\
445	0.0111464910620393\\
446	0.0111407354884382\\
447	0.0111349419349087\\
448	0.0111291074670474\\
449	0.0111232288464062\\
450	0.0111173025366058\\
451	0.011111324720289\\
452	0.011105291330471\\
453	0.0110991981007967\\
454	0.0110930406403892\\
455	0.0110868145404838\\
456	0.0110805155221153\\
457	0.0110741396364773\\
458	0.0110676835325153\\
459	0.0110611448099239\\
460	0.0110545213142401\\
461	0.0110478108591665\\
462	0.0110410112361793\\
463	0.0110341202250999\\
464	0.0110271356053456\\
465	0.0110200551674\\
466	0.0110128767237927\\
467	0.0110055981185488\\
468	0.0109982172336279\\
469	0.0109907319902822\\
470	0.0109831403424805\\
471	0.0109754402585081\\
472	0.0109676296854838\\
473	0.0109597065361618\\
474	0.0109516686883546\\
475	0.0109435139839672\\
476	0.0109352402275613\\
477	0.0109268451843773\\
478	0.0109183265777507\\
479	0.0109096820858918\\
480	0.0109009093380414\\
481	0.010892005910093\\
482	0.0108829693198906\\
483	0.010873797022585\\
484	0.0108644864066869\\
485	0.0108550347912124\\
486	0.0108454394226476\\
487	0.0108356974717292\\
488	0.010825806030049\\
489	0.0108157621064867\\
490	0.0108055626234857\\
491	0.0107952044131859\\
492	0.0107846842134338\\
493	0.0107739986636859\\
494	0.0107631443008188\\
495	0.0107521175548441\\
496	0.0107409147444969\\
497	0.0107295320726408\\
498	0.010717965621484\\
499	0.0107062113476029\\
500	0.0106942650767707\\
501	0.0106821224985848\\
502	0.0106697791608902\\
503	0.0106572304639928\\
504	0.0106444716546572\\
505	0.0106314978198817\\
506	0.0106183038804425\\
507	0.0106048845842016\\
508	0.0105912344991705\\
509	0.0105773480063279\\
510	0.0105632192921886\\
511	0.010548842341124\\
512	0.0105342109274303\\
513	0.0105193186071187\\
514	0.0105041587094202\\
515	0.0104887243280487\\
516	0.0104730083122233\\
517	0.0104570032588406\\
518	0.0104407015036354\\
519	0.0104240951111839\\
520	0.0104071758647688\\
521	0.0103899352010984\\
522	0.0103723642390185\\
523	0.0103544537984531\\
524	0.0103361912858935\\
525	0.0103175697913127\\
526	0.0102985823644446\\
527	0.0102792179033952\\
528	0.0102594628736049\\
529	0.0102393058891938\\
530	0.0102187416612845\\
531	0.0101977572891147\\
532	0.0101763391650815\\
533	0.0101544858517146\\
534	0.0101321650698094\\
535	0.0101093618892514\\
536	0.0100860552896666\\
537	0.0100622279883189\\
538	0.0100378898760208\\
539	0.0100130190160264\\
540	0.00998759235516145\\
541	0.00996162910243391\\
542	0.00993509312227768\\
543	0.00990801253630866\\
544	0.00988029775326948\\
545	0.00985192229656894\\
546	0.00982286658872227\\
547	0.00979310716199265\\
548	0.00976262660047411\\
549	0.00973140497514659\\
550	0.00969931219528145\\
551	0.00966648344620747\\
552	0.00963281165389633\\
553	0.00959842102443266\\
554	0.00956349370009603\\
555	0.00952797230121755\\
556	0.00949159145210537\\
557	0.00945426789194161\\
558	0.00941596751559718\\
559	0.00937665349780352\\
560	0.00933629049997299\\
561	0.00929457458365241\\
562	0.00925160407166974\\
563	0.00920708885269026\\
564	0.00916271247427281\\
565	0.00911868685878312\\
566	0.00907356620335069\\
567	0.00900971250905663\\
568	0.00881541843488129\\
569	0.00839061335662357\\
570	0.00818306859050345\\
571	0.00811407518943064\\
572	0.0080450568354251\\
573	0.00797480709176036\\
574	0.00790325115715567\\
575	0.00783034444470094\\
576	0.00775604347260888\\
577	0.00768030278729031\\
578	0.00760307437871544\\
579	0.00752430736197342\\
580	0.00744394776372258\\
581	0.00736193826527503\\
582	0.00727821791826307\\
583	0.00719272178896829\\
584	0.00710538045136045\\
585	0.00701611910584813\\
586	0.00692485573802267\\
587	0.00683149674994525\\
588	0.00673592587771181\\
589	0.0066379752046507\\
590	0.00653734834917363\\
591	0.00643341581933105\\
592	0.00632466858477865\\
593	0.00620725772627071\\
594	0.0060710901180166\\
595	0.00588956027306329\\
596	0.00559184182825266\\
597	0.00498881296807123\\
598	0.00357511483354343\\
599	0\\
600	0\\
};
\addplot [color=mycolor5,solid,forget plot]
  table[row sep=crcr]{%
1	0.0124902353284585\\
2	0.0124902293664923\\
3	0.0124902233118918\\
4	0.0124902171635064\\
5	0.0124902109201776\\
6	0.012490204580739\\
7	0.0124901981440165\\
8	0.0124901916088281\\
9	0.0124901849739839\\
10	0.0124901782382866\\
11	0.0124901714005312\\
12	0.012490164459505\\
13	0.0124901574139876\\
14	0.0124901502627513\\
15	0.0124901430045607\\
16	0.0124901356381731\\
17	0.012490128162338\\
18	0.0124901205757979\\
19	0.0124901128772873\\
20	0.0124901050655338\\
21	0.012490097139257\\
22	0.0124900890971696\\
23	0.0124900809379764\\
24	0.0124900726603747\\
25	0.0124900642630546\\
26	0.0124900557446983\\
27	0.0124900471039804\\
28	0.012490038339568\\
29	0.0124900294501201\\
30	0.012490020434288\\
31	0.012490011290715\\
32	0.0124900020180364\\
33	0.0124899926148792\\
34	0.0124899830798619\\
35	0.0124899734115947\\
36	0.012489963608679\\
37	0.0124899536697074\\
38	0.0124899435932632\\
39	0.0124899333779204\\
40	0.0124899230222437\\
41	0.0124899125247875\\
42	0.0124899018840964\\
43	0.0124898910987043\\
44	0.0124898801671344\\
45	0.0124898690878988\\
46	0.0124898578594979\\
47	0.0124898464804201\\
48	0.0124898349491415\\
49	0.0124898232641253\\
50	0.0124898114238211\\
51	0.0124897994266649\\
52	0.0124897872710779\\
53	0.0124897749554665\\
54	0.0124897624782213\\
55	0.0124897498377165\\
56	0.0124897370323093\\
57	0.0124897240603392\\
58	0.012489710920127\\
59	0.0124896976099744\\
60	0.0124896841281627\\
61	0.0124896704729525\\
62	0.0124896566425823\\
63	0.0124896426352675\\
64	0.0124896284492\\
65	0.0124896140825465\\
66	0.0124895995334482\\
67	0.0124895848000186\\
68	0.0124895698803436\\
69	0.0124895547724793\\
70	0.0124895394744515\\
71	0.0124895239842538\\
72	0.0124895082998469\\
73	0.0124894924191568\\
74	0.0124894763400736\\
75	0.0124894600604502\\
76	0.0124894435781005\\
77	0.0124894268907983\\
78	0.0124894099962753\\
79	0.01248939289222\\
80	0.0124893755762759\\
81	0.0124893580460395\\
82	0.0124893402990593\\
83	0.0124893223328336\\
84	0.0124893041448087\\
85	0.0124892857323777\\
86	0.012489267092878\\
87	0.0124892482235898\\
88	0.0124892291217345\\
89	0.0124892097844723\\
90	0.0124891902089006\\
91	0.0124891703920523\\
92	0.0124891503308933\\
93	0.0124891300223211\\
94	0.0124891094631626\\
95	0.0124890886501721\\
96	0.0124890675800295\\
97	0.0124890462493382\\
98	0.0124890246546232\\
99	0.0124890027923289\\
100	0.0124889806588174\\
101	0.0124889582503665\\
102	0.0124889355631674\\
103	0.0124889125933234\\
104	0.0124888893368471\\
105	0.0124888657896591\\
106	0.012488841947586\\
107	0.0124888178063581\\
108	0.012488793361608\\
109	0.0124887686088683\\
110	0.01248874354357\\
111	0.0124887181610407\\
112	0.0124886924565025\\
113	0.0124886664250702\\
114	0.01248864006175\\
115	0.0124886133614371\\
116	0.0124885863189142\\
117	0.0124885589288498\\
118	0.0124885311857965\\
119	0.0124885030841889\\
120	0.0124884746183425\\
121	0.0124884457824513\\
122	0.0124884165705866\\
123	0.0124883869766949\\
124	0.0124883569945965\\
125	0.0124883266179836\\
126	0.0124882958404183\\
127	0.0124882646553313\\
128	0.0124882330560198\\
129	0.0124882010356458\\
130	0.0124881685872339\\
131	0.0124881357036699\\
132	0.0124881023776987\\
133	0.0124880686019221\\
134	0.0124880343687968\\
135	0.0124879996706327\\
136	0.0124879644995903\\
137	0.0124879288476787\\
138	0.0124878927067533\\
139	0.0124878560685132\\
140	0.0124878189244991\\
141	0.0124877812660901\\
142	0.0124877430845006\\
143	0.0124877043707775\\
144	0.0124876651157965\\
145	0.0124876253102614\\
146	0.0124875849447066\\
147	0.0124875440095055\\
148	0.0124875024948651\\
149	0.0124874603907828\\
150	0.0124874176870681\\
151	0.0124873743733414\\
152	0.0124873304390307\\
153	0.0124872858733691\\
154	0.0124872406653904\\
155	0.0124871948039269\\
156	0.012487148277605\\
157	0.0124871010748425\\
158	0.0124870531838447\\
159	0.0124870045926009\\
160	0.0124869552888807\\
161	0.0124869052602303\\
162	0.0124868544939689\\
163	0.0124868029771844\\
164	0.0124867506967297\\
165	0.0124866976392188\\
166	0.0124866437910222\\
167	0.0124865891382632\\
168	0.012486533666813\\
169	0.0124864773622869\\
170	0.0124864202100392\\
171	0.0124863621951586\\
172	0.0124863033024637\\
173	0.0124862435164981\\
174	0.0124861828215248\\
175	0.0124861212015215\\
176	0.0124860586401755\\
177	0.0124859951208774\\
178	0.0124859306267162\\
179	0.0124858651404732\\
180	0.012485798644616\\
181	0.0124857311212924\\
182	0.0124856625523242\\
183	0.0124855929192003\\
184	0.0124855222030705\\
185	0.012485450384738\\
186	0.0124853774446527\\
187	0.0124853033629034\\
188	0.0124852281192108\\
189	0.0124851516929191\\
190	0.012485074062988\\
191	0.0124849952079849\\
192	0.0124849151060755\\
193	0.0124848337350158\\
194	0.0124847510721424\\
195	0.0124846670943634\\
196	0.0124845817781487\\
197	0.01248449509952\\
198	0.012484407034041\\
199	0.0124843175568061\\
200	0.0124842266424307\\
201	0.0124841342650394\\
202	0.0124840403982546\\
203	0.0124839450151857\\
204	0.0124838480884163\\
205	0.0124837495899925\\
206	0.0124836494914103\\
207	0.012483547763603\\
208	0.0124834443769283\\
209	0.0124833393011548\\
210	0.0124832325054486\\
211	0.0124831239583597\\
212	0.012483013627808\\
213	0.0124829014810694\\
214	0.0124827874847608\\
215	0.0124826716048263\\
216	0.0124825538065221\\
217	0.0124824340544015\\
218	0.0124823123123004\\
219	0.0124821885433217\\
220	0.01248206270982\\
221	0.0124819347733864\\
222	0.0124818046948328\\
223	0.0124816724341753\\
224	0.0124815379506183\\
225	0.0124814012025367\\
226	0.0124812621474594\\
227	0.012481120742055\\
228	0.0124809769421269\\
229	0.0124808307026182\\
230	0.0124806819775986\\
231	0.0124805307201748\\
232	0.012480376882519\\
233	0.0124802204158544\\
234	0.0124800612704389\\
235	0.0124798993955506\\
236	0.0124797347394715\\
237	0.0124795672494726\\
238	0.0124793968717981\\
239	0.0124792235516502\\
240	0.0124790472331739\\
241	0.0124788678594421\\
242	0.0124786853724408\\
243	0.0124784997130536\\
244	0.012478310821043\\
245	0.0124781186350241\\
246	0.0124779230924268\\
247	0.0124777241294588\\
248	0.0124775216811095\\
249	0.0124773156811916\\
250	0.0124771060622543\\
251	0.012476892755541\\
252	0.0124766756909423\\
253	0.0124764547969787\\
254	0.0124762300009113\\
255	0.0124760012290689\\
256	0.0124757684068646\\
257	0.0124755314575005\\
258	0.0124752903027308\\
259	0.0124750448628285\\
260	0.0124747950565549\\
261	0.0124745408011309\\
262	0.0124742820121952\\
263	0.0124740186037331\\
264	0.0124737504880505\\
265	0.0124734775757329\\
266	0.0124731997756005\\
267	0.0124729169946593\\
268	0.0124726291380521\\
269	0.0124723361090053\\
270	0.0124720378087735\\
271	0.0124717341365791\\
272	0.0124714249895452\\
273	0.0124711102626228\\
274	0.0124707898485164\\
275	0.0124704636376436\\
276	0.0124701315181834\\
277	0.0124697933761739\\
278	0.0124694490951707\\
279	0.0124690985559843\\
280	0.0124687416368001\\
281	0.0124683782130985\\
282	0.0124680081576039\\
283	0.0124676313402836\\
284	0.0124672476282676\\
285	0.0124668568857217\\
286	0.0124664589735756\\
287	0.0124660537490618\\
288	0.0124656410656329\\
289	0.0124652207740919\\
290	0.0124647927217348\\
291	0.0124643567522475\\
292	0.012463912705599\\
293	0.012463460417931\\
294	0.012462999721442\\
295	0.0124625304442654\\
296	0.0124620524103428\\
297	0.0124615654393033\\
298	0.0124610693463769\\
299	0.0124605639423743\\
300	0.0124600490336132\\
301	0.0124595244213592\\
302	0.0124589899018524\\
303	0.0124584452662088\\
304	0.0124578903002627\\
305	0.0124573247843877\\
306	0.0124567484932576\\
307	0.0124561611954554\\
308	0.0124555626527871\\
309	0.0124549526193216\\
310	0.0124543308408411\\
311	0.0124536970554987\\
312	0.0124530509990368\\
313	0.012452392418624\\
314	0.012451721038384\\
315	0.0124510365714501\\
316	0.0124503387249437\\
317	0.0124496272039903\\
318	0.0124489017043215\\
319	0.0124481619117428\\
320	0.0124474075047502\\
321	0.0124466381546771\\
322	0.0124458535246763\\
323	0.0124450532649725\\
324	0.0124442370051025\\
325	0.0124434043675366\\
326	0.0124425549696913\\
327	0.0124416884174781\\
328	0.0124408043048335\\
329	0.0124399022138936\\
330	0.0124389817163567\\
331	0.0124380423728597\\
332	0.012437083726933\\
333	0.0124361053081795\\
334	0.0124351066317933\\
335	0.0124340871973913\\
336	0.0124330464858498\\
337	0.0124319839543395\\
338	0.0124308990402332\\
339	0.0124297911680455\\
340	0.0124286597409069\\
341	0.0124275041382953\\
342	0.0124263237122196\\
343	0.0124251177801756\\
344	0.0124238856125472\\
345	0.0124226264163516\\
346	0.0124213393436645\\
347	0.012420023597371\\
348	0.0124186783311737\\
349	0.0124173030564975\\
350	0.0124158968898539\\
351	0.0124144588741656\\
352	0.0124129877934208\\
353	0.0124114824825798\\
354	0.0124099418033697\\
355	0.0124083646192341\\
356	0.0124067497381982\\
357	0.0124050959010704\\
358	0.0124034018065269\\
359	0.0124016660890458\\
360	0.0123998872475998\\
361	0.0123980636728532\\
362	0.0123961937996587\\
363	0.0123942760012171\\
364	0.0123923085856317\\
365	0.0123902897861636\\
366	0.0123882177334646\\
367	0.0123860903869722\\
368	0.0123839054314689\\
369	0.012381660513089\\
370	0.0123793549436391\\
371	0.0123769869410401\\
372	0.0123745549049348\\
373	0.0123720573981417\\
374	0.0123694921657331\\
375	0.0123668525718493\\
376	0.0123641434723573\\
377	0.0123613612308389\\
378	0.0123585019905699\\
379	0.0123555614765225\\
380	0.012352533254238\\
381	0.0123494109560399\\
382	0.0123461883904216\\
383	0.0123428587357301\\
384	0.012339414416079\\
385	0.0123358468573877\\
386	0.0123321458613464\\
387	0.0123282977885207\\
388	0.012324282093484\\
389	0.0123200946013244\\
390	0.0123157300149129\\
391	0.01231117831966\\
392	0.0123064341129794\\
393	0.0123014792917886\\
394	0.0122962721209553\\
395	0.0122907884681141\\
396	0.0122849957863516\\
397	0.0122717496953838\\
398	0.0122570401769288\\
399	0.0122420484885861\\
400	0.0122267608862323\\
401	0.0122111621527093\\
402	0.0121952349369981\\
403	0.0121789572001542\\
404	0.012162287564809\\
405	0.012145210527299\\
406	0.0121277276505989\\
407	0.0121098184262074\\
408	0.0120915212598722\\
409	0.012072736941335\\
410	0.0120534270861025\\
411	0.0120335748564956\\
412	0.0120130715916027\\
413	0.0119918571744927\\
414	0.0119698626984003\\
415	0.0119397099674708\\
416	0.0119041788690955\\
417	0.0118682198352206\\
418	0.0118318339627868\\
419	0.0117950261756174\\
420	0.0117577990437274\\
421	0.0117201546414928\\
422	0.0116820909786053\\
423	0.0116436150224078\\
424	0.0116047369665029\\
425	0.0115654727440228\\
426	0.0115258431628656\\
427	0.0114858746588462\\
428	0.0114456005294182\\
429	0.0114050625288854\\
430	0.0113643131762869\\
431	0.011333352114047\\
432	0.0113191539921617\\
433	0.0113051183353929\\
434	0.011291278209906\\
435	0.011277670594299\\
436	0.0112643366313084\\
437	0.0112513221058282\\
438	0.0112386780243766\\
439	0.0112264612790195\\
440	0.0112147354101465\\
441	0.011203571485085\\
442	0.0111930491118217\\
443	0.0111832576062024\\
444	0.0111742973131731\\
445	0.0111662812296339\\
446	0.011158757195801\\
447	0.0111513097040762\\
448	0.0111439421227419\\
449	0.0111366569094769\\
450	0.0111294553533079\\
451	0.0111223372626344\\
452	0.0111153005885831\\
453	0.0111083409708036\\
454	0.0111014511906871\\
455	0.0110946205129858\\
456	0.0110878338881117\\
457	0.0110810709892686\\
458	0.0110743050536873\\
459	0.0110675014940417\\
460	0.0110606483815249\\
461	0.0110537422847705\\
462	0.0110467794114533\\
463	0.0110397556126491\\
464	0.0110326663993794\\
465	0.0110255069755062\\
466	0.011018272292372\\
467	0.0110109571319796\\
468	0.0110035562271484\\
469	0.0109960644293726\\
470	0.0109884769379745\\
471	0.0109807896077161\\
472	0.0109729993564934\\
473	0.0109651034414379\\
474	0.0109570990735483\\
475	0.0109489834288188\\
476	0.0109407536604105\\
477	0.0109324069114435\\
478	0.010923940327742\\
479	0.0109153510695292\\
480	0.010906636320608\\
481	0.0108977932929509\\
482	0.0108888192237977\\
483	0.0108797113612692\\
484	0.0108704669330506\\
485	0.0108610831069969\\
486	0.0108515569896736\\
487	0.0108418856244206\\
488	0.010832065988842\\
489	0.0108220949916262\\
490	0.0108119694686164\\
491	0.0108016861780791\\
492	0.0107912417951752\\
493	0.0107806329057225\\
494	0.0107698559994798\\
495	0.0107589074633869\\
496	0.0107477835754997\\
497	0.0107364805003073\\
498	0.0107249942837879\\
499	0.010713320848206\\
500	0.0107014559866558\\
501	0.010689395357361\\
502	0.0106771344777492\\
503	0.0106646687183215\\
504	0.0106519932963427\\
505	0.0106391032693764\\
506	0.0106259935286803\\
507	0.0106126587924579\\
508	0.0105990935989242\\
509	0.0105852922991342\\
510	0.0105712490495774\\
511	0.0105569578045387\\
512	0.0105424123082307\\
513	0.0105276060867004\\
514	0.0105125324395163\\
515	0.0104971844312409\\
516	0.010481554882691\\
517	0.0104656363619528\\
518	0.0104494211751765\\
519	0.0104329013571813\\
520	0.0104160686618839\\
521	0.010398914554042\\
522	0.0103814301998287\\
523	0.010363606456479\\
524	0.0103454338618867\\
525	0.0103269025770548\\
526	0.0103080023982409\\
527	0.0102887227844879\\
528	0.0102690509745637\\
529	0.0102489754108345\\
530	0.0102284897740996\\
531	0.0102075815648557\\
532	0.0101862375709781\\
533	0.0101644392206717\\
534	0.0101421829086899\\
535	0.010119454573835\\
536	0.0100962391115006\\
537	0.0100725256838767\\
538	0.0100482986053227\\
539	0.0100235261175771\\
540	0.00999818247812129\\
541	0.00997228941853672\\
542	0.00994581507060125\\
543	0.00991870999703562\\
544	0.00989099442621836\\
545	0.00986265677447937\\
546	0.00983370215581083\\
547	0.0098041036567173\\
548	0.00977379119656501\\
549	0.00974274914025589\\
550	0.00971094972091856\\
551	0.00967837072564764\\
552	0.00964490231605043\\
553	0.0096106266264305\\
554	0.00957550432320249\\
555	0.0095395458041511\\
556	0.00950287238659638\\
557	0.00946573856617259\\
558	0.0094276828600046\\
559	0.00938865888075084\\
560	0.00934860316479576\\
561	0.00930747516300949\\
562	0.00926523327873259\\
563	0.00922131617172663\\
564	0.00917625117725328\\
565	0.00913000690891885\\
566	0.0090841631426429\\
567	0.00903806347343111\\
568	0.00899074899630517\\
569	0.008809378738224\\
570	0.00846359421886279\\
571	0.00812512099772187\\
572	0.00804549373171191\\
573	0.00797486330201451\\
574	0.00790326132056656\\
575	0.0078303468390966\\
576	0.0077560442470378\\
577	0.0076803031522189\\
578	0.00760307453625226\\
579	0.00752430744274301\\
580	0.00744394780425085\\
581	0.00736193828664806\\
582	0.00727821792858083\\
583	0.00719272179374158\\
584	0.00710538045286112\\
585	0.00701611910608783\\
586	0.00692485573802267\\
587	0.00683149674994525\\
588	0.00673592587771182\\
589	0.00663797520465071\\
590	0.00653734834917363\\
591	0.00643341581933105\\
592	0.00632466858477866\\
593	0.00620725772627071\\
594	0.0060710901180166\\
595	0.00588956027306329\\
596	0.00559184182825266\\
597	0.00498881296807123\\
598	0.00357511483354343\\
599	0\\
600	0\\
};
\addplot [color=mycolor6,solid,forget plot]
  table[row sep=crcr]{%
1	0.0124903499082909\\
2	0.0124903457713311\\
3	0.0124903415746704\\
4	0.0124903373176105\\
5	0.0124903329994472\\
6	0.01249032861947\\
7	0.0124903241769623\\
8	0.0124903196712012\\
9	0.0124903151014572\\
10	0.0124903104669946\\
11	0.0124903057670707\\
12	0.0124903010009364\\
13	0.0124902961678356\\
14	0.0124902912670051\\
15	0.0124902862976747\\
16	0.0124902812590668\\
17	0.0124902761503966\\
18	0.0124902709708714\\
19	0.012490265719691\\
20	0.012490260396047\\
21	0.0124902549991232\\
22	0.0124902495280949\\
23	0.0124902439821287\\
24	0.0124902383603829\\
25	0.0124902326620065\\
26	0.0124902268861394\\
27	0.0124902210319122\\
28	0.0124902150984455\\
29	0.0124902090848503\\
30	0.0124902029902272\\
31	0.0124901968136663\\
32	0.0124901905542469\\
33	0.0124901842110371\\
34	0.0124901777830937\\
35	0.0124901712694616\\
36	0.0124901646691735\\
37	0.0124901579812498\\
38	0.0124901512046978\\
39	0.0124901443385116\\
40	0.0124901373816717\\
41	0.0124901303331444\\
42	0.0124901231918815\\
43	0.0124901159568197\\
44	0.0124901086268805\\
45	0.0124901012009692\\
46	0.0124900936779748\\
47	0.0124900860567693\\
48	0.0124900783362073\\
49	0.0124900705151254\\
50	0.0124900625923415\\
51	0.0124900545666544\\
52	0.0124900464368432\\
53	0.0124900382016665\\
54	0.0124900298598623\\
55	0.0124900214101464\\
56	0.0124900128512128\\
57	0.0124900041817323\\
58	0.0124899954003521\\
59	0.012489986505695\\
60	0.0124899774963588\\
61	0.0124899683709154\\
62	0.0124899591279101\\
63	0.012489949765861\\
64	0.0124899402832576\\
65	0.0124899306785611\\
66	0.0124899209502022\\
67	0.0124899110965816\\
68	0.0124899011160681\\
69	0.0124898910069984\\
70	0.0124898807676758\\
71	0.0124898703963696\\
72	0.0124898598913141\\
73	0.0124898492507074\\
74	0.012489838472711\\
75	0.0124898275554484\\
76	0.0124898164970044\\
77	0.0124898052954239\\
78	0.012489793948711\\
79	0.0124897824548284\\
80	0.0124897708116956\\
81	0.0124897590171886\\
82	0.0124897470691385\\
83	0.0124897349653306\\
84	0.0124897227035032\\
85	0.012489710281347\\
86	0.0124896976965033\\
87	0.0124896849465636\\
88	0.0124896720290683\\
89	0.0124896589415056\\
90	0.0124896456813102\\
91	0.0124896322458628\\
92	0.0124896186324884\\
93	0.0124896048384553\\
94	0.0124895908609746\\
95	0.0124895766971982\\
96	0.0124895623442183\\
97	0.0124895477990661\\
98	0.0124895330587106\\
99	0.0124895181200576\\
100	0.0124895029799485\\
101	0.0124894876351591\\
102	0.0124894720823987\\
103	0.0124894563183087\\
104	0.0124894403394615\\
105	0.0124894241423594\\
106	0.0124894077234333\\
107	0.0124893910790418\\
108	0.0124893742054697\\
109	0.0124893570989268\\
110	0.0124893397555472\\
111	0.0124893221713872\\
112	0.0124893043424251\\
113	0.0124892862645589\\
114	0.0124892679336058\\
115	0.0124892493453008\\
116	0.0124892304952948\\
117	0.0124892113791542\\
118	0.0124891919923588\\
119	0.0124891723303009\\
120	0.0124891523882836\\
121	0.0124891321615195\\
122	0.0124891116451293\\
123	0.0124890908341404\\
124	0.0124890697234852\\
125	0.0124890483079995\\
126	0.0124890265824214\\
127	0.0124890045413891\\
128	0.0124889821794397\\
129	0.0124889594910074\\
130	0.0124889364704216\\
131	0.0124889131119055\\
132	0.012488889409574\\
133	0.012488865357432\\
134	0.0124888409493727\\
135	0.0124888161791753\\
136	0.0124887910405034\\
137	0.0124887655269029\\
138	0.0124887396317999\\
139	0.0124887133484985\\
140	0.0124886866701788\\
141	0.0124886595898949\\
142	0.0124886321005722\\
143	0.0124886041950052\\
144	0.012488575865856\\
145	0.0124885471056511\\
146	0.0124885179067796\\
147	0.0124884882614893\\
148	0.0124884581618857\\
149	0.0124884275999282\\
150	0.0124883965674282\\
151	0.0124883650560457\\
152	0.0124883330572873\\
153	0.0124883005625026\\
154	0.0124882675628817\\
155	0.0124882340494525\\
156	0.0124882000130769\\
157	0.0124881654444488\\
158	0.0124881303340898\\
159	0.0124880946723469\\
160	0.0124880584493885\\
161	0.0124880216552018\\
162	0.0124879842795884\\
163	0.0124879463121616\\
164	0.0124879077423424\\
165	0.0124878685593557\\
166	0.0124878287522271\\
167	0.0124877883097781\\
168	0.0124877472206231\\
169	0.0124877054731645\\
170	0.0124876630555894\\
171	0.0124876199558645\\
172	0.0124875761617325\\
173	0.0124875316607072\\
174	0.012487486440069\\
175	0.0124874404868604\\
176	0.0124873937878812\\
177	0.0124873463296835\\
178	0.0124872980985669\\
179	0.0124872490805731\\
180	0.0124871992614812\\
181	0.0124871486268017\\
182	0.0124870971617717\\
183	0.0124870448513489\\
184	0.0124869916802059\\
185	0.0124869376327249\\
186	0.0124868826929909\\
187	0.0124868268447865\\
188	0.0124867700715851\\
189	0.0124867123565449\\
190	0.0124866536825024\\
191	0.0124865940319656\\
192	0.0124865333871077\\
193	0.01248647172976\\
194	0.012486409041405\\
195	0.0124863453031693\\
196	0.0124862804958165\\
197	0.0124862145997398\\
198	0.0124861475949546\\
199	0.0124860794610907\\
200	0.0124860101773851\\
201	0.0124859397226737\\
202	0.0124858680753834\\
203	0.0124857952135245\\
204	0.0124857211146819\\
205	0.0124856457560075\\
206	0.0124855691142111\\
207	0.0124854911655526\\
208	0.0124854118858327\\
209	0.0124853312503847\\
210	0.0124852492340655\\
211	0.0124851658112465\\
212	0.0124850809558048\\
213	0.0124849946411139\\
214	0.0124849068400345\\
215	0.012484817524905\\
216	0.0124847266675323\\
217	0.0124846342391818\\
218	0.0124845402105681\\
219	0.0124844445518448\\
220	0.0124843472325947\\
221	0.0124842482218196\\
222	0.0124841474879302\\
223	0.0124840449987352\\
224	0.0124839407214309\\
225	0.0124838346225904\\
226	0.0124837266681531\\
227	0.0124836168234147\\
228	0.0124835050530174\\
229	0.0124833913209377\\
230	0.012483275590469\\
231	0.0124831578242137\\
232	0.0124830379840704\\
233	0.0124829160312219\\
234	0.0124827919261219\\
235	0.012482665628482\\
236	0.0124825370972582\\
237	0.0124824062906366\\
238	0.0124822731660197\\
239	0.0124821376800112\\
240	0.012481999788401\\
241	0.0124818594461497\\
242	0.0124817166073725\\
243	0.0124815712253218\\
244	0.0124814232523689\\
245	0.0124812726399846\\
246	0.0124811193387193\\
247	0.0124809632981872\\
248	0.0124808044670495\\
249	0.0124806427929892\\
250	0.0124804782226875\\
251	0.0124803107018014\\
252	0.0124801401749449\\
253	0.0124799665856824\\
254	0.0124797898765304\\
255	0.0124796099889194\\
256	0.0124794268630742\\
257	0.0124792404380561\\
258	0.0124790506517356\\
259	0.0124788574407657\\
260	0.0124786607405545\\
261	0.0124784604852348\\
262	0.0124782566076328\\
263	0.0124780490392382\\
264	0.0124778377101742\\
265	0.0124776225491646\\
266	0.0124774034835018\\
267	0.0124771804390127\\
268	0.0124769533400247\\
269	0.0124767221093299\\
270	0.0124764866681489\\
271	0.0124762469360934\\
272	0.0124760028311273\\
273	0.0124757542695289\\
274	0.012475501165856\\
275	0.0124752434329172\\
276	0.0124749809817402\\
277	0.0124747137215045\\
278	0.0124744415594877\\
279	0.0124741644010382\\
280	0.0124738821495321\\
281	0.0124735947063317\\
282	0.0124733019707468\\
283	0.0124730038399861\\
284	0.0124727002091024\\
285	0.0124723909709238\\
286	0.012472076015978\\
287	0.0124717552324584\\
288	0.0124714285062603\\
289	0.0124710957208649\\
290	0.0124707567572845\\
291	0.0124704114940054\\
292	0.0124700598069298\\
293	0.0124697015693148\\
294	0.0124693366517105\\
295	0.0124689649218955\\
296	0.0124685862448123\\
297	0.0124682004825048\\
298	0.0124678074940573\\
299	0.0124674071355199\\
300	0.0124669992597981\\
301	0.0124665837165906\\
302	0.0124661603523118\\
303	0.0124657290100063\\
304	0.0124652895292559\\
305	0.0124648417460756\\
306	0.0124643854927882\\
307	0.0124639205978698\\
308	0.0124634468857859\\
309	0.012462964176897\\
310	0.0124624722875289\\
311	0.0124619710304049\\
312	0.0124614602151652\\
313	0.0124609396456024\\
314	0.0124604091206361\\
315	0.0124598684346352\\
316	0.012459317377499\\
317	0.0124587557339302\\
318	0.012458183283296\\
319	0.0124575997996936\\
320	0.0124570050517805\\
321	0.0124563988024518\\
322	0.0124557808082306\\
323	0.0124551508186964\\
324	0.0124545085774243\\
325	0.0124538538218701\\
326	0.0124531862826627\\
327	0.0124525056833877\\
328	0.0124518117404315\\
329	0.0124511041628745\\
330	0.0124503826521337\\
331	0.012449646901259\\
332	0.012448896594983\\
333	0.01244813140941\\
334	0.0124473510116013\\
335	0.0124465550589769\\
336	0.0124457431987077\\
337	0.0124449150678924\\
338	0.012444070293746\\
339	0.0124432084925006\\
340	0.0124423292688365\\
341	0.0124414322151302\\
342	0.0124405169103919\\
343	0.0124395829188694\\
344	0.0124386297890041\\
345	0.0124376570554372\\
346	0.0124366642467256\\
347	0.0124356508814378\\
348	0.0124346164902014\\
349	0.0124335605588321\\
350	0.0124324825511272\\
351	0.0124313818982378\\
352	0.0124302580222779\\
353	0.0124291103333074\\
354	0.0124279382259728\\
355	0.0124267410735406\\
356	0.0124255182263614\\
357	0.0124242690126032\\
358	0.0124229927343146\\
359	0.0124216886610799\\
360	0.0124203560335352\\
361	0.0124189940735666\\
362	0.0124176019732497\\
363	0.0124161788924943\\
364	0.0124147239555812\\
365	0.0124132362456239\\
366	0.0124117147965536\\
367	0.012410158589031\\
368	0.0124085665820927\\
369	0.0124069378036266\\
370	0.0124052711457885\\
371	0.0124035654530817\\
372	0.0124018194942684\\
373	0.0124000318633614\\
374	0.012398200882086\\
375	0.0123963256058482\\
376	0.0123944044999658\\
377	0.0123924359608527\\
378	0.0123904182783274\\
379	0.0123883495256649\\
380	0.0123862277531902\\
381	0.0123840509848023\\
382	0.0123818171483689\\
383	0.0123795240637716\\
384	0.012377169416278\\
385	0.0123747506945979\\
386	0.0123722650765578\\
387	0.0123697094640008\\
388	0.0123670820071721\\
389	0.0123643807447834\\
390	0.0123616034346911\\
391	0.0123587480375856\\
392	0.0123558114438562\\
393	0.0123527892847153\\
394	0.012349678499905\\
395	0.0123464751769515\\
396	0.0123431723992934\\
397	0.0123397719775921\\
398	0.0123362693978358\\
399	0.0123326580223497\\
400	0.0123289305131901\\
401	0.0123250786061563\\
402	0.0123210925229418\\
403	0.0123169592462861\\
404	0.0123126590341093\\
405	0.0123081849892889\\
406	0.0123035371390292\\
407	0.0122987089766026\\
408	0.0122937117987223\\
409	0.012288499239166\\
410	0.0122830527393027\\
411	0.0122773569247035\\
412	0.0122713590208821\\
413	0.01226502611952\\
414	0.0122583126776527\\
415	0.0122451202917145\\
416	0.0122275962977738\\
417	0.0122097245096778\\
418	0.0121914798875531\\
419	0.0121728035719554\\
420	0.0121537201252082\\
421	0.0121342283835598\\
422	0.0121143944015873\\
423	0.0120941035622057\\
424	0.0120733228868494\\
425	0.0120519704093852\\
426	0.0120300009381597\\
427	0.0120073630960932\\
428	0.0119839982400571\\
429	0.0119598391929415\\
430	0.0119348088642074\\
431	0.0119035718607269\\
432	0.0118626644383289\\
433	0.0118212840867717\\
434	0.0117794275159407\\
435	0.0117370890367737\\
436	0.0116942726696179\\
437	0.0116509860369291\\
438	0.0116072397585825\\
439	0.0115630480879793\\
440	0.0115184296888694\\
441	0.0114734085958313\\
442	0.0114280154387347\\
443	0.0113822891369773\\
444	0.0113362797848445\\
445	0.0112900485108053\\
446	0.0112604417470562\\
447	0.0112445871699175\\
448	0.0112289467408647\\
449	0.0112135628538908\\
450	0.0111984827755775\\
451	0.0111837592613211\\
452	0.0111694512583896\\
453	0.0111556247100162\\
454	0.0111423534654628\\
455	0.0111297203450589\\
456	0.0111178185251644\\
457	0.0111067530814234\\
458	0.0110966427290473\\
459	0.0110876217424687\\
460	0.0110789511509886\\
461	0.0110703603876142\\
462	0.0110618523880424\\
463	0.01105342880017\\
464	0.0110450896302439\\
465	0.0110368328130501\\
466	0.0110286536879707\\
467	0.0110205443636153\\
468	0.0110124929529589\\
469	0.0110044826511298\\
470	0.0109964906225358\\
471	0.0109884866589396\\
472	0.0109804315636604\\
473	0.010972309402298\\
474	0.0109641158735388\\
475	0.0109558462453186\\
476	0.0109474953634065\\
477	0.0109390576763981\\
478	0.0109305272825391\\
479	0.0109218980054371\\
480	0.0109131635076103\\
481	0.0109043174532553\\
482	0.0108953537346557\\
483	0.0108862667804902\\
484	0.0108770519691698\\
485	0.0108677057439626\\
486	0.0108582244698403\\
487	0.0108486044452207\\
488	0.0108388419150288\\
489	0.0108289330846297\\
490	0.0108188741338943\\
491	0.0108086612302689\\
492	0.0107982905391795\\
493	0.0107877582293795\\
494	0.0107770604698671\\
495	0.0107661934137049\\
496	0.010755153162328\\
497	0.0107439357145074\\
498	0.0107325369632783\\
499	0.0107209526922572\\
500	0.010709178571227\\
501	0.0106972101508757\\
502	0.0106850428565912\\
503	0.010672671981256\\
504	0.0106600926770576\\
505	0.01064729994645\\
506	0.0106342886325849\\
507	0.0106210534098167\\
508	0.0106075887752942\\
509	0.010593889041686\\
510	0.0105799483295842\\
511	0.0105657605595954\\
512	0.0105513194441342\\
513	0.010536618478944\\
514	0.010521650934377\\
515	0.0105064098464739\\
516	0.0104908880078865\\
517	0.0104750779586877\\
518	0.0104589719770975\\
519	0.0104425620701236\\
520	0.0104258399640507\\
521	0.0104087970947348\\
522	0.0103914245977489\\
523	0.0103737132984263\\
524	0.0103556537018252\\
525	0.010337235983902\\
526	0.0103184499820798\\
527	0.0102992851848024\\
528	0.0102797307211344\\
529	0.0102597753222085\\
530	0.0102394072935273\\
531	0.0102186145643185\\
532	0.0101973846774478\\
533	0.0101756997798491\\
534	0.0101535545461412\\
535	0.0101309351733925\\
536	0.0101078269266932\\
537	0.0100842124706025\\
538	0.0100600770696729\\
539	0.0100354112287115\\
540	0.0100101981158954\\
541	0.00998442768607106\\
542	0.00995808447059193\\
543	0.00993111835930172\\
544	0.00990352491215651\\
545	0.00987528721777292\\
546	0.0098463777269695\\
547	0.00981678641520482\\
548	0.00978652235414809\\
549	0.00975556467757381\\
550	0.00972393918831744\\
551	0.0096915577078595\\
552	0.00965838440114489\\
553	0.00962440005291455\\
554	0.00958951789549794\\
555	0.00955373339702645\\
556	0.00951711378512381\\
557	0.00947950003892096\\
558	0.0094411129256838\\
559	0.00940205772885252\\
560	0.00936225442642091\\
561	0.00932144327520121\\
562	0.0092795526745926\\
563	0.00923652889629026\\
564	0.00919214483043075\\
565	0.00914631274642346\\
566	0.0090990966384936\\
567	0.0090509252644886\\
568	0.00900320175753261\\
569	0.00895486878153357\\
570	0.0088151415759191\\
571	0.00856330285569335\\
572	0.00813856590265964\\
573	0.00797840592915509\\
574	0.00790370578700228\\
575	0.00783042495555452\\
576	0.00775606190032862\\
577	0.00768030853877576\\
578	0.00760307706941287\\
579	0.00752430849590189\\
580	0.0074439483406399\\
581	0.00736193855163164\\
582	0.00727821806930863\\
583	0.00719272186170981\\
584	0.00710538048536775\\
585	0.00701611911660432\\
586	0.00692485573974563\\
587	0.00683149674994525\\
588	0.00673592587771182\\
589	0.00663797520465072\\
590	0.00653734834917363\\
591	0.00643341581933105\\
592	0.00632466858477866\\
593	0.00620725772627072\\
594	0.00607109011801661\\
595	0.0058895602730633\\
596	0.00559184182825266\\
597	0.00498881296807123\\
598	0.00357511483354343\\
599	0\\
600	0\\
};
\addplot [color=mycolor7,solid,forget plot]
  table[row sep=crcr]{%
1	0.0124905348581479\\
2	0.0124905320101193\\
3	0.0124905291226254\\
4	0.0124905261951872\\
5	0.0124905232273194\\
6	0.0124905202185305\\
7	0.0124905171683227\\
8	0.0124905140761919\\
9	0.012490510941627\\
10	0.0124905077641104\\
11	0.0124905045431175\\
12	0.0124905012781165\\
13	0.0124904979685683\\
14	0.0124904946139264\\
15	0.0124904912136367\\
16	0.012490487767137\\
17	0.0124904842738573\\
18	0.0124904807332194\\
19	0.0124904771446363\\
20	0.0124904735075128\\
21	0.0124904698212445\\
22	0.012490466085218\\
23	0.0124904622988105\\
24	0.0124904584613897\\
25	0.0124904545723135\\
26	0.0124904506309296\\
27	0.0124904466365754\\
28	0.0124904425885778\\
29	0.0124904384862527\\
30	0.0124904343289049\\
31	0.0124904301158279\\
32	0.0124904258463031\\
33	0.0124904215196001\\
34	0.0124904171349763\\
35	0.012490412691676\\
36	0.012490408188931\\
37	0.0124904036259594\\
38	0.0124903990019657\\
39	0.0124903943161406\\
40	0.0124903895676602\\
41	0.012490384755686\\
42	0.0124903798793642\\
43	0.0124903749378258\\
44	0.0124903699301856\\
45	0.0124903648555425\\
46	0.0124903597129784\\
47	0.0124903545015582\\
48	0.0124903492203295\\
49	0.0124903438683217\\
50	0.0124903384445459\\
51	0.0124903329479946\\
52	0.0124903273776407\\
53	0.0124903217324375\\
54	0.0124903160113183\\
55	0.0124903102131955\\
56	0.0124903043369605\\
57	0.0124902983814829\\
58	0.0124902923456103\\
59	0.0124902862281677\\
60	0.0124902800279567\\
61	0.0124902737437554\\
62	0.0124902673743178\\
63	0.0124902609183729\\
64	0.0124902543746245\\
65	0.0124902477417506\\
66	0.0124902410184028\\
67	0.0124902342032057\\
68	0.0124902272947564\\
69	0.0124902202916237\\
70	0.0124902131923478\\
71	0.0124902059954398\\
72	0.0124901986993805\\
73	0.0124901913026205\\
74	0.012490183803579\\
75	0.0124901762006435\\
76	0.0124901684921693\\
77	0.0124901606764784\\
78	0.0124901527518593\\
79	0.012490144716566\\
80	0.0124901365688176\\
81	0.0124901283067975\\
82	0.0124901199286527\\
83	0.0124901114324933\\
84	0.0124901028163915\\
85	0.0124900940783813\\
86	0.0124900852164573\\
87	0.0124900762285744\\
88	0.0124900671126469\\
89	0.0124900578665479\\
90	0.0124900484881081\\
91	0.0124900389751158\\
92	0.0124900293253154\\
93	0.0124900195364071\\
94	0.0124900096060461\\
95	0.0124899995318414\\
96	0.0124899893113553\\
97	0.0124899789421028\\
98	0.0124899684215502\\
99	0.0124899577471148\\
100	0.0124899469161636\\
101	0.0124899359260128\\
102	0.0124899247739268\\
103	0.0124899134571169\\
104	0.012489901972741\\
105	0.0124898903179025\\
106	0.0124898784896489\\
107	0.0124898664849714\\
108	0.0124898543008035\\
109	0.0124898419340203\\
110	0.0124898293814372\\
111	0.0124898166398092\\
112	0.0124898037058292\\
113	0.0124897905761276\\
114	0.0124897772472707\\
115	0.0124897637157599\\
116	0.0124897499780301\\
117	0.0124897360304489\\
118	0.0124897218693152\\
119	0.0124897074908578\\
120	0.0124896928912346\\
121	0.0124896780665306\\
122	0.0124896630127571\\
123	0.0124896477258502\\
124	0.0124896322016691\\
125	0.0124896164359952\\
126	0.01248960042453\\
127	0.0124895841628944\\
128	0.0124895676466263\\
129	0.0124895508711796\\
130	0.0124895338319227\\
131	0.0124895165241365\\
132	0.012489498943013\\
133	0.0124894810836536\\
134	0.0124894629410675\\
135	0.01248944451017\\
136	0.0124894257857809\\
137	0.0124894067626225\\
138	0.0124893874353184\\
139	0.0124893677983916\\
140	0.0124893478462627\\
141	0.0124893275732491\\
142	0.0124893069735627\\
143	0.0124892860413098\\
144	0.0124892647704892\\
145	0.0124892431549916\\
146	0.0124892211885953\\
147	0.0124891988649551\\
148	0.0124891761775888\\
149	0.0124891531199295\\
150	0.0124891296852879\\
151	0.0124891058668498\\
152	0.0124890816576739\\
153	0.0124890570506893\\
154	0.012489032038693\\
155	0.0124890066143472\\
156	0.0124889807701772\\
157	0.0124889544985682\\
158	0.0124889277917629\\
159	0.0124889006418589\\
160	0.0124888730408054\\
161	0.0124888449804008\\
162	0.0124888164522896\\
163	0.0124887874479595\\
164	0.0124887579587381\\
165	0.0124887279757901\\
166	0.0124886974901141\\
167	0.0124886664925391\\
168	0.0124886349737216\\
169	0.0124886029241418\\
170	0.0124885703341007\\
171	0.0124885371937159\\
172	0.0124885034929189\\
173	0.0124884692214506\\
174	0.0124884343688584\\
175	0.0124883989244917\\
176	0.0124883628774985\\
177	0.0124883262168214\\
178	0.0124882889311937\\
179	0.0124882510091348\\
180	0.0124882124389469\\
181	0.0124881732087102\\
182	0.0124881333062785\\
183	0.0124880927192753\\
184	0.0124880514350891\\
185	0.0124880094408688\\
186	0.0124879667235192\\
187	0.0124879232696962\\
188	0.012487879065802\\
189	0.0124878340979806\\
190	0.0124877883521122\\
191	0.0124877418138088\\
192	0.0124876944684088\\
193	0.0124876463009715\\
194	0.0124875972962724\\
195	0.0124875474387975\\
196	0.0124874967127378\\
197	0.0124874451019837\\
198	0.0124873925901195\\
199	0.0124873391604176\\
200	0.0124872847958326\\
201	0.0124872294789953\\
202	0.0124871731922067\\
203	0.0124871159174319\\
204	0.0124870576362937\\
205	0.0124869983300665\\
206	0.0124869379796693\\
207	0.0124868765656599\\
208	0.0124868140682273\\
209	0.0124867504671856\\
210	0.0124866857419665\\
211	0.0124866198716126\\
212	0.0124865528347702\\
213	0.0124864846096814\\
214	0.0124864151741774\\
215	0.0124863445056701\\
216	0.0124862725811449\\
217	0.0124861993771526\\
218	0.0124861248698009\\
219	0.0124860490347467\\
220	0.0124859718471871\\
221	0.0124858932818512\\
222	0.012485813312991\\
223	0.0124857319143724\\
224	0.0124856490592661\\
225	0.0124855647204379\\
226	0.0124854788701395\\
227	0.0124853914800986\\
228	0.0124853025215084\\
229	0.0124852119650172\\
230	0.0124851197807177\\
231	0.0124850259381366\\
232	0.0124849304062229\\
233	0.0124848331533369\\
234	0.0124847341472384\\
235	0.0124846333550744\\
236	0.0124845307433674\\
237	0.0124844262780019\\
238	0.0124843199242127\\
239	0.0124842116465705\\
240	0.012484101408969\\
241	0.0124839891746109\\
242	0.0124838749059936\\
243	0.0124837585648943\\
244	0.0124836401123552\\
245	0.0124835195086681\\
246	0.0124833967133588\\
247	0.0124832716851716\\
248	0.0124831443820525\\
249	0.0124830147611318\\
250	0.0124828827787077\\
251	0.0124827483902288\\
252	0.0124826115502786\\
253	0.0124824722125582\\
254	0.0124823303298655\\
255	0.0124821858540693\\
256	0.0124820387360955\\
257	0.0124818889259074\\
258	0.012481736372485\\
259	0.0124815810238052\\
260	0.0124814228268198\\
261	0.0124812617274338\\
262	0.012481097670484\\
263	0.0124809305997161\\
264	0.012480760457762\\
265	0.0124805871861166\\
266	0.0124804107251132\\
267	0.0124802310139002\\
268	0.0124800479904154\\
269	0.0124798615913607\\
270	0.0124796717521767\\
271	0.0124794784070152\\
272	0.0124792814887133\\
273	0.0124790809287654\\
274	0.0124788766572966\\
275	0.0124786686030352\\
276	0.0124784566932851\\
277	0.0124782408538995\\
278	0.0124780210092316\\
279	0.0124777970821148\\
280	0.012477568993832\\
281	0.0124773366640827\\
282	0.0124771000109498\\
283	0.0124768589508636\\
284	0.0124766133985649\\
285	0.0124763632670672\\
286	0.012476108467622\\
287	0.0124758489096868\\
288	0.0124755845008812\\
289	0.012475315146946\\
290	0.0124750407517022\\
291	0.0124747612170079\\
292	0.012474476442715\\
293	0.0124741863266231\\
294	0.0124738907644337\\
295	0.0124735896497025\\
296	0.0124732828737903\\
297	0.0124729703258132\\
298	0.0124726518925897\\
299	0.0124723274585832\\
300	0.0124719969058491\\
301	0.0124716601139765\\
302	0.0124713169600303\\
303	0.0124709673184884\\
304	0.0124706110611779\\
305	0.0124702480572067\\
306	0.0124698781728918\\
307	0.0124695012716871\\
308	0.0124691172141201\\
309	0.0124687258577437\\
310	0.0124683270571089\\
311	0.0124679206637006\\
312	0.012467506525651\\
313	0.0124670844877404\\
314	0.0124666543913352\\
315	0.0124662160743862\\
316	0.0124657693712707\\
317	0.012465314112685\\
318	0.0124648501255666\\
319	0.0124643772329907\\
320	0.0124638952540501\\
321	0.012463404003713\\
322	0.0124629032927018\\
323	0.0124623929274717\\
324	0.0124618727100949\\
325	0.0124613424380998\\
326	0.0124608019043495\\
327	0.0124602508969206\\
328	0.0124596891989781\\
329	0.0124591165886197\\
330	0.0124585328386973\\
331	0.0124579377166939\\
332	0.012457330984562\\
333	0.0124567123985447\\
334	0.0124560817089774\\
335	0.0124554386600965\\
336	0.0124547829899094\\
337	0.0124541144300366\\
338	0.0124534327054414\\
339	0.0124527375341964\\
340	0.0124520286272237\\
341	0.0124513056880012\\
342	0.0124505684122523\\
343	0.0124498164877157\\
344	0.0124490495942073\\
345	0.0124482674039933\\
346	0.0124474695814075\\
347	0.0124466557835481\\
348	0.0124458256556707\\
349	0.0124449788327121\\
350	0.0124441149385019\\
351	0.0124432335872951\\
352	0.0124423343831529\\
353	0.0124414169192525\\
354	0.0124404807769839\\
355	0.0124395255254527\\
356	0.0124385507210872\\
357	0.0124375559068055\\
358	0.0124365406111347\\
359	0.0124355043482407\\
360	0.0124344466181821\\
361	0.0124333669053959\\
362	0.0124322646779233\\
363	0.0124311393864985\\
364	0.01242999046346\\
365	0.0124288173216336\\
366	0.0124276193540648\\
367	0.0124263959366591\\
368	0.012425146431859\\
369	0.0124238701729671\\
370	0.0124225664698571\\
371	0.0124212346050074\\
372	0.0124198738264399\\
373	0.012418483351476\\
374	0.0124170624278787\\
375	0.0124156102335757\\
376	0.0124141259156042\\
377	0.0124126085844373\\
378	0.0124110573064146\\
379	0.0124094711190097\\
380	0.012407849027586\\
381	0.0124061899971926\\
382	0.0124044929490002\\
383	0.0124027567549661\\
384	0.0124009802299257\\
385	0.0123991621244678\\
386	0.0123973011425606\\
387	0.0123953960353144\\
388	0.0123934454878276\\
389	0.0123914481037175\\
390	0.0123894024190804\\
391	0.0123873068038163\\
392	0.0123851594726926\\
393	0.012382958627836\\
394	0.0123807022797639\\
395	0.0123783881948406\\
396	0.0123760146205097\\
397	0.0123735793781997\\
398	0.0123710800760692\\
399	0.0123685141895256\\
400	0.0123658790321492\\
401	0.0123631716916794\\
402	0.0123603889115399\\
403	0.0123575270926111\\
404	0.0123545836608289\\
405	0.0123515564514014\\
406	0.0123484428830265\\
407	0.0123452410479942\\
408	0.012341945132935\\
409	0.0123385514087179\\
410	0.0123350559571012\\
411	0.0123314522659224\\
412	0.0123277355502006\\
413	0.0123238999417505\\
414	0.0123199372047438\\
415	0.0123158466433408\\
416	0.01231162464631\\
417	0.0123072605199656\\
418	0.0123027388411858\\
419	0.0122980347031105\\
420	0.0122931599545735\\
421	0.0122881184015738\\
422	0.0122829323480974\\
423	0.0122775474812349\\
424	0.0122719426269842\\
425	0.0122660775374262\\
426	0.0122599279472396\\
427	0.0122534663204804\\
428	0.0122466610753029\\
429	0.0122394748244065\\
430	0.0122318573162748\\
431	0.0122193934971236\\
432	0.01219907003095\\
433	0.0121783223572718\\
434	0.012157211188889\\
435	0.0121357821290657\\
436	0.0121138883848964\\
437	0.0120914713783048\\
438	0.0120684947526461\\
439	0.0120449175919429\\
440	0.0120206936737558\\
441	0.0119957705816139\\
442	0.0119700886624083\\
443	0.0119435798577895\\
444	0.0119161666529473\\
445	0.0118877591033575\\
446	0.0118493950643996\\
447	0.0118029847814933\\
448	0.0117560370758049\\
449	0.0117085550393122\\
450	0.0116605438013708\\
451	0.0116120111665148\\
452	0.0115629683312149\\
453	0.0115134307519362\\
454	0.0114634195782015\\
455	0.0114129623471881\\
456	0.0113620897836564\\
457	0.0113108397072014\\
458	0.0112592610890864\\
459	0.0112074202653949\\
460	0.0111807966783627\\
461	0.0111631539198066\\
462	0.0111457754608839\\
463	0.0111287137085472\\
464	0.0111120272758899\\
465	0.0110957818296251\\
466	0.0110800511382366\\
467	0.0110649182204461\\
468	0.0110504765446275\\
469	0.0110368314774263\\
470	0.0110241020251619\\
471	0.0110124228876155\\
472	0.0110019468752839\\
473	0.0109919135989446\\
474	0.0109819581271541\\
475	0.0109720831320626\\
476	0.0109622896634912\\
477	0.0109525766955367\\
478	0.0109429405795266\\
479	0.0109333743798174\\
480	0.0109238670689868\\
481	0.0109144025526902\\
482	0.0109049584893634\\
483	0.0108955048627365\\
484	0.010886002256269\\
485	0.0108764113287698\\
486	0.0108667269013731\\
487	0.0108569432753038\\
488	0.0108470542364019\\
489	0.0108370530777549\\
490	0.0108269326468195\\
491	0.010816685425205\\
492	0.0108063036515764\\
493	0.010795779500978\\
494	0.0107851053375398\\
495	0.0107742740619926\\
496	0.0107632795814143\\
497	0.0107521170989154\\
498	0.0107407816914889\\
499	0.0107292683223882\\
500	0.0107175718550653\\
501	0.0107056870681011\\
502	0.0106936086701972\\
503	0.0106813313137836\\
504	0.0106688496051062\\
505	0.0106561581077001\\
506	0.0106432513348758\\
507	0.0106301237251207\\
508	0.0106167695920173\\
509	0.0106031830772331\\
510	0.0105893581450665\\
511	0.0105752885762286\\
512	0.0105609679607158\\
513	0.0105463896896371\\
514	0.0105315469458894\\
515	0.0105164326936348\\
516	0.0105010396666399\\
517	0.0104853603557148\\
518	0.0104693869957639\\
519	0.0104531115533758\\
520	0.0104365257164948\\
521	0.0104196208844651\\
522	0.010402388157826\\
523	0.0103848183278934\\
524	0.0103669018661714\\
525	0.0103486289136151\\
526	0.0103299892698269\\
527	0.0103109723822915\\
528	0.0102915673357285\\
529	0.0102717628423874\\
530	0.0102515472330569\\
531	0.0102309084467258\\
532	0.0102098340203716\\
533	0.0101883110790446\\
534	0.0101663262511507\\
535	0.010143865743747\\
536	0.0101209153372074\\
537	0.0100974583940287\\
538	0.0100734798563586\\
539	0.0100489691822816\\
540	0.0100239098760763\\
541	0.00999828775940519\\
542	0.00997208773349542\\
543	0.00994528797321891\\
544	0.00991786997241802\\
545	0.00988981093756304\\
546	0.00986106789000255\\
547	0.00983163917964649\\
548	0.00980152321313676\\
549	0.00977069207681917\\
550	0.00973910583825725\\
551	0.00970678396253116\\
552	0.00967371302342107\\
553	0.00963987135698235\\
554	0.0096052821126277\\
555	0.00956983538884356\\
556	0.00953350247176351\\
557	0.00949615392586526\\
558	0.00945792993494292\\
559	0.00941872899557266\\
560	0.0093785649396831\\
561	0.00933753561474518\\
562	0.00929582166875998\\
563	0.00925314354820956\\
564	0.00920935743036538\\
565	0.00916438007450519\\
566	0.00911786024556143\\
567	0.00907002470464621\\
568	0.00902064875706754\\
569	0.00897037472747782\\
570	0.00892037973865424\\
571	0.00883136476694636\\
572	0.00865022597339645\\
573	0.00828928204857107\\
574	0.00793224760636447\\
575	0.00783391425016302\\
576	0.00775665882326094\\
577	0.00768043829538653\\
578	0.00760311425029811\\
579	0.00752432601413681\\
580	0.00744395532478318\\
581	0.00736194208712794\\
582	0.00727821978182829\\
583	0.00719272277813591\\
584	0.00710538092623947\\
585	0.00701611933500737\\
586	0.00692485581252974\\
587	0.00683149676218332\\
588	0.00673592587771182\\
589	0.00663797520465071\\
590	0.00653734834917363\\
591	0.00643341581933105\\
592	0.00632466858477865\\
593	0.00620725772627071\\
594	0.0060710901180166\\
595	0.00588956027306328\\
596	0.00559184182825266\\
597	0.00498881296807123\\
598	0.00357511483354343\\
599	0\\
600	0\\
};
\addplot [color=mycolor8,solid,forget plot]
  table[row sep=crcr]{%
1	0.0124911041531259\\
2	0.012491102138032\\
3	0.012491100094959\\
4	0.0124910980235295\\
5	0.0124910959233599\\
6	0.0124910937940607\\
7	0.0124910916352357\\
8	0.0124910894464824\\
9	0.0124910872273918\\
10	0.0124910849775477\\
11	0.0124910826965273\\
12	0.0124910803839006\\
13	0.0124910780392302\\
14	0.0124910756620713\\
15	0.0124910732519718\\
16	0.0124910708084714\\
17	0.0124910683311021\\
18	0.0124910658193876\\
19	0.0124910632728435\\
20	0.0124910606909768\\
21	0.0124910580732857\\
22	0.0124910554192597\\
23	0.012491052728379\\
24	0.0124910500001146\\
25	0.0124910472339282\\
26	0.0124910444292714\\
27	0.0124910415855863\\
28	0.0124910387023045\\
29	0.0124910357788474\\
30	0.0124910328146258\\
31	0.0124910298090396\\
32	0.0124910267614777\\
33	0.0124910236713177\\
34	0.0124910205379255\\
35	0.0124910173606555\\
36	0.0124910141388498\\
37	0.0124910108718382\\
38	0.012491007558938\\
39	0.0124910041994536\\
40	0.0124910007926764\\
41	0.0124909973378842\\
42	0.0124909938343414\\
43	0.0124909902812982\\
44	0.0124909866779906\\
45	0.0124909830236402\\
46	0.0124909793174537\\
47	0.0124909755586225\\
48	0.0124909717463228\\
49	0.0124909678797148\\
50	0.0124909639579429\\
51	0.0124909599801347\\
52	0.0124909559454014\\
53	0.0124909518528371\\
54	0.0124909477015184\\
55	0.0124909434905042\\
56	0.0124909392188353\\
57	0.0124909348855341\\
58	0.0124909304896041\\
59	0.0124909260300299\\
60	0.0124909215057763\\
61	0.0124909169157883\\
62	0.0124909122589908\\
63	0.0124909075342878\\
64	0.0124909027405624\\
65	0.0124908978766762\\
66	0.0124908929414691\\
67	0.0124908879337587\\
68	0.01249088285234\\
69	0.0124908776959849\\
70	0.0124908724634418\\
71	0.0124908671534353\\
72	0.0124908617646657\\
73	0.0124908562958084\\
74	0.0124908507455136\\
75	0.0124908451124061\\
76	0.0124908393950843\\
77	0.0124908335921201\\
78	0.0124908277020584\\
79	0.0124908217234166\\
80	0.012490815654684\\
81	0.0124908094943214\\
82	0.0124908032407609\\
83	0.0124907968924046\\
84	0.012490790447625\\
85	0.0124907839047639\\
86	0.0124907772621319\\
87	0.0124907705180083\\
88	0.01249076367064\\
89	0.0124907567182412\\
90	0.012490749658993\\
91	0.0124907424910425\\
92	0.0124907352125023\\
93	0.0124907278214501\\
94	0.0124907203159279\\
95	0.0124907126939414\\
96	0.0124907049534594\\
97	0.0124906970924132\\
98	0.0124906891086959\\
99	0.0124906810001617\\
100	0.0124906727646252\\
101	0.0124906643998609\\
102	0.0124906559036023\\
103	0.0124906472735411\\
104	0.0124906385073266\\
105	0.0124906296025651\\
106	0.0124906205568187\\
107	0.012490611367605\\
108	0.0124906020323956\\
109	0.0124905925486162\\
110	0.0124905829136448\\
111	0.0124905731248115\\
112	0.0124905631793972\\
113	0.0124905530746329\\
114	0.0124905428076984\\
115	0.0124905323757219\\
116	0.0124905217757781\\
117	0.0124905110048882\\
118	0.0124905000600176\\
119	0.0124904889380758\\
120	0.0124904776359146\\
121	0.012490466150327\\
122	0.0124904544780457\\
123	0.0124904426157421\\
124	0.0124904305600244\\
125	0.0124904183074365\\
126	0.012490405854456\\
127	0.0124903931974924\\
128	0.0124903803328859\\
129	0.0124903672569045\\
130	0.0124903539657427\\
131	0.0124903404555188\\
132	0.0124903267222724\\
133	0.0124903127619621\\
134	0.0124902985704625\\
135	0.0124902841435613\\
136	0.0124902694769555\\
137	0.0124902545662488\\
138	0.0124902394069469\\
139	0.0124902239944543\\
140	0.0124902083240696\\
141	0.0124901923909816\\
142	0.0124901761902656\\
143	0.0124901597168796\\
144	0.0124901429656638\\
145	0.0124901259313421\\
146	0.0124901086085269\\
147	0.0124900909917171\\
148	0.0124900730752285\\
149	0.0124900548530119\\
150	0.0124900363194713\\
151	0.0124900174689059\\
152	0.012489998295508\\
153	0.0124899787933609\\
154	0.0124899589564368\\
155	0.012489938778595\\
156	0.0124899182535792\\
157	0.0124898973750159\\
158	0.0124898761364117\\
159	0.0124898545311512\\
160	0.0124898325524947\\
161	0.0124898101935757\\
162	0.0124897874473986\\
163	0.0124897643068363\\
164	0.0124897407646275\\
165	0.0124897168133744\\
166	0.0124896924455396\\
167	0.0124896676534442\\
168	0.0124896424292644\\
169	0.0124896167650292\\
170	0.0124895906526174\\
171	0.0124895640837547\\
172	0.0124895370500111\\
173	0.0124895095427974\\
174	0.0124894815533627\\
175	0.0124894530727911\\
176	0.0124894240919986\\
177	0.0124893946017298\\
178	0.0124893645925551\\
179	0.0124893340548666\\
180	0.0124893029788755\\
181	0.0124892713546085\\
182	0.0124892391719037\\
183	0.0124892064204081\\
184	0.012489173089573\\
185	0.012489139168651\\
186	0.0124891046466917\\
187	0.0124890695125386\\
188	0.0124890337548242\\
189	0.0124889973619672\\
190	0.0124889603221673\\
191	0.0124889226234021\\
192	0.0124888842534222\\
193	0.0124888451997473\\
194	0.0124888054496618\\
195	0.0124887649902103\\
196	0.0124887238081929\\
197	0.0124886818901612\\
198	0.0124886392224129\\
199	0.0124885957909875\\
200	0.0124885515816612\\
201	0.0124885065799421\\
202	0.0124884607710651\\
203	0.0124884141399864\\
204	0.0124883666713791\\
205	0.0124883183496266\\
206	0.0124882691588184\\
207	0.0124882190827435\\
208	0.0124881681048854\\
209	0.0124881162084159\\
210	0.0124880633761894\\
211	0.0124880095907364\\
212	0.0124879548342582\\
213	0.0124878990886197\\
214	0.0124878423353433\\
215	0.0124877845556027\\
216	0.0124877257302155\\
217	0.012487665839637\\
218	0.0124876048639526\\
219	0.0124875427828714\\
220	0.0124874795757182\\
221	0.0124874152214263\\
222	0.0124873496985303\\
223	0.0124872829851575\\
224	0.0124872150590207\\
225	0.01248714589741\\
226	0.0124870754771841\\
227	0.0124870037747623\\
228	0.0124869307661155\\
229	0.0124868564267578\\
230	0.012486780731737\\
231	0.0124867036556259\\
232	0.0124866251725127\\
233	0.0124865452559911\\
234	0.0124864638791511\\
235	0.0124863810145688\\
236	0.0124862966342958\\
237	0.0124862107098496\\
238	0.0124861232122022\\
239	0.0124860341117697\\
240	0.0124859433784011\\
241	0.0124858509813671\\
242	0.0124857568893484\\
243	0.0124856610704243\\
244	0.0124855634920603\\
245	0.0124854641210963\\
246	0.0124853629237338\\
247	0.0124852598655234\\
248	0.0124851549113518\\
249	0.0124850480254286\\
250	0.0124849391712727\\
251	0.0124848283116992\\
252	0.0124847154088048\\
253	0.012484600423954\\
254	0.0124844833177631\\
255	0.0124843640500866\\
256	0.0124842425800017\\
257	0.0124841188657925\\
258	0.0124839928649349\\
259	0.0124838645340795\\
260	0.0124837338290358\\
261	0.0124836007047547\\
262	0.0124834651153118\\
263	0.0124833270138889\\
264	0.0124831863527567\\
265	0.0124830430832552\\
266	0.0124828971557754\\
267	0.0124827485197389\\
268	0.0124825971235783\\
269	0.0124824429147154\\
270	0.0124822858395403\\
271	0.0124821258433883\\
272	0.0124819628705165\\
273	0.0124817968640789\\
274	0.0124816277660993\\
275	0.0124814555174431\\
276	0.0124812800577948\\
277	0.0124811013256614\\
278	0.0124809192584328\\
279	0.0124807337921161\\
280	0.012480544861438\\
281	0.0124803523998258\\
282	0.0124801563393797\\
283	0.0124799566108442\\
284	0.0124797531435784\\
285	0.0124795458655267\\
286	0.0124793347031885\\
287	0.0124791195815857\\
288	0.0124789004242308\\
289	0.0124786771530937\\
290	0.0124784496885671\\
291	0.0124782179494323\\
292	0.0124779818528227\\
293	0.0124777413141876\\
294	0.0124774962472545\\
295	0.0124772465639906\\
296	0.0124769921745635\\
297	0.0124767329873011\\
298	0.0124764689086499\\
299	0.0124761998431337\\
300	0.0124759256933106\\
301	0.0124756463597299\\
302	0.0124753617408879\\
303	0.0124750717331831\\
304	0.012474776230871\\
305	0.0124744751260184\\
306	0.0124741683084588\\
307	0.0124738556657481\\
308	0.0124735370831249\\
309	0.0124732124434715\\
310	0.0124728816272755\\
311	0.0124725445125801\\
312	0.0124722009749567\\
313	0.0124718508874366\\
314	0.0124714941203138\\
315	0.0124711305407571\\
316	0.0124707600136177\\
317	0.0124703824011359\\
318	0.0124699975626429\\
319	0.0124696053544896\\
320	0.0124692056299724\\
321	0.0124687982392594\\
322	0.0124683830293228\\
323	0.0124679598438589\\
324	0.0124675285232051\\
325	0.0124670889042563\\
326	0.0124666408203799\\
327	0.0124661841013269\\
328	0.0124657185731383\\
329	0.0124652440580472\\
330	0.0124647603743824\\
331	0.0124642673364666\\
332	0.0124637647545081\\
333	0.0124632524344899\\
334	0.0124627301780566\\
335	0.0124621977824021\\
336	0.0124616550401511\\
337	0.012461101739226\\
338	0.0124605376627145\\
339	0.0124599625887298\\
340	0.0124593762902648\\
341	0.0124587785350435\\
342	0.012458169085379\\
343	0.0124575476980541\\
344	0.0124569141242101\\
345	0.0124562681091739\\
346	0.0124556093923005\\
347	0.0124549377064665\\
348	0.0124542527780373\\
349	0.0124535543266637\\
350	0.0124528420652271\\
351	0.0124521156996018\\
352	0.0124513749284031\\
353	0.0124506194427104\\
354	0.0124498489258216\\
355	0.0124490630530038\\
356	0.0124482614912072\\
357	0.0124474438987931\\
358	0.0124466099252914\\
359	0.0124457592111046\\
360	0.0124448913871545\\
361	0.0124440060745378\\
362	0.0124431028841549\\
363	0.012442181416315\\
364	0.0124412412603495\\
365	0.0124402819943223\\
366	0.012439303184906\\
367	0.0124383043870957\\
368	0.0124372851426542\\
369	0.0124362449801065\\
370	0.0124351834139574\\
371	0.0124340999439149\\
372	0.0124329940554009\\
373	0.0124318652225133\\
374	0.0124307128998605\\
375	0.0124295365251154\\
376	0.0124283355178597\\
377	0.0124271092785167\\
378	0.0124258571889347\\
379	0.012424578611565\\
380	0.0124232728879237\\
381	0.0124219393373652\\
382	0.012420577255762\\
383	0.0124191859140688\\
384	0.01241776455739\\
385	0.0124163124064769\\
386	0.0124148286623609\\
387	0.0124133124961171\\
388	0.0124117630465355\\
389	0.0124101794189236\\
390	0.0124085606771264\\
391	0.0124069058450466\\
392	0.0124052139132507\\
393	0.0124034838251913\\
394	0.0124017144798726\\
395	0.0123999047758579\\
396	0.0123980535423976\\
397	0.0123961595527441\\
398	0.0123942215274171\\
399	0.0123922381284047\\
400	0.0123902079507506\\
401	0.0123881295144388\\
402	0.0123860012789051\\
403	0.0123838217331691\\
404	0.0123815893319603\\
405	0.0123793024339158\\
406	0.0123769593353914\\
407	0.0123745579848499\\
408	0.012372096395045\\
409	0.0123695724308074\\
410	0.0123669836613538\\
411	0.0123643276713864\\
412	0.0123616018110315\\
413	0.0123588032073114\\
414	0.0123559293759664\\
415	0.012352977516097\\
416	0.0123499442104931\\
417	0.0123468255804046\\
418	0.0123436174626817\\
419	0.0123403182842309\\
420	0.0123369258799185\\
421	0.0123334384239431\\
422	0.0123298486515786\\
423	0.0123261514279178\\
424	0.0123223403454993\\
425	0.0123184103549643\\
426	0.0123143561538332\\
427	0.0123101721200259\\
428	0.0123058520719514\\
429	0.0123013884515147\\
430	0.0122967702770947\\
431	0.0122919872839269\\
432	0.0122870262947262\\
433	0.012281893121785\\
434	0.0122766172968557\\
435	0.0122712086989549\\
436	0.012265596747146\\
437	0.0122597519638662\\
438	0.0122536544500453\\
439	0.0122472818652323\\
440	0.012240609006385\\
441	0.0122336071818781\\
442	0.0122262429145973\\
443	0.0122184736049118\\
444	0.0122102262152079\\
445	0.0122014694010157\\
446	0.0121848916675161\\
447	0.0121619150759289\\
448	0.0121384537630224\\
449	0.0121144778828846\\
450	0.012089954056728\\
451	0.0120648449151303\\
452	0.0120391085326023\\
453	0.0120126977550609\\
454	0.0119855595646001\\
455	0.0119576338161843\\
456	0.0119288499825325\\
457	0.0118991267200434\\
458	0.0118683709474048\\
459	0.0118364772457061\\
460	0.0117899293887976\\
461	0.0117379330702364\\
462	0.0116853363928404\\
463	0.0116321444789565\\
464	0.0115783659745756\\
465	0.0115240119462968\\
466	0.0114690944420999\\
467	0.0114136296117527\\
468	0.0113576430991244\\
469	0.011301169870704\\
470	0.0112442545989068\\
471	0.011186953692074\\
472	0.0111293383178927\\
473	0.011097764439338\\
474	0.011078069400534\\
475	0.0110586702906042\\
476	0.0110396284626611\\
477	0.0110210129708733\\
478	0.0110029014976747\\
479	0.0109853815511372\\
480	0.0109685518350611\\
481	0.0109525238704125\\
482	0.0109374238939215\\
483	0.0109233951058549\\
484	0.0109106003704035\\
485	0.0108989145299648\\
486	0.0108872964848506\\
487	0.010875749312009\\
488	0.0108642743277868\\
489	0.0108528705813431\\
490	0.0108415342366413\\
491	0.0108302578195344\\
492	0.0108190293011009\\
493	0.0108078309834536\\
494	0.0107966381456029\\
495	0.0107854174027605\\
496	0.0107741247138657\\
497	0.0107627118024001\\
498	0.0107511720771215\\
499	0.0107394983103662\\
500	0.0107276826416192\\
501	0.0107157166032271\\
502	0.010703591176167\\
503	0.0106912968861572\\
504	0.0106788239532819\\
505	0.0106661625121368\\
506	0.0106533029241167\\
507	0.0106402362093446\\
508	0.0106269546333935\\
509	0.0106134514714393\\
510	0.0105997197988757\\
511	0.010585752504419\\
512	0.0105715423050824\\
513	0.0105570817622631\\
514	0.0105423632976747\\
515	0.010527379207154\\
516	0.0105121216693857\\
517	0.0104965827452528\\
518	0.0104807543617196\\
519	0.0104646282717054\\
520	0.0104481959781213\\
521	0.0104314487084238\\
522	0.0104143774066446\\
523	0.0103969727245863\\
524	0.0103792250120199\\
525	0.0103611243057383\\
526	0.0103426603173705\\
527	0.0103238224199568\\
528	0.0103045996334536\\
529	0.0102849806095972\\
530	0.0102649536169924\\
531	0.010244506528012\\
532	0.0102236268082363\\
533	0.01020230150594\\
534	0.0101805172436589\\
535	0.0101582602076974\\
536	0.0101355161377006\\
537	0.0101122703165035\\
538	0.0100885075307361\\
539	0.0100642120485176\\
540	0.01003936766622\\
541	0.0100139576998519\\
542	0.00998797089944421\\
543	0.00996138846326938\\
544	0.00993419115715787\\
545	0.00990635887018058\\
546	0.00987786587109372\\
547	0.00984868848857995\\
548	0.00981879802970254\\
549	0.00978818044588218\\
550	0.00975679906793436\\
551	0.00972466302807263\\
552	0.00969176439712891\\
553	0.00965808014673742\\
554	0.00962355671545677\\
555	0.00958821951216768\\
556	0.00955204355678784\\
557	0.00951500838590981\\
558	0.00947712588481273\\
559	0.00943823158357318\\
560	0.00939832134866453\\
561	0.00935745843282328\\
562	0.00931549302033459\\
563	0.00927253839052881\\
564	0.00922861881642663\\
565	0.00918392533577964\\
566	0.00913812936805613\\
567	0.00909110508593811\\
568	0.00904243698427758\\
569	0.00899247235927465\\
570	0.00894090723682039\\
571	0.00888831520764092\\
572	0.00883560063672551\\
573	0.00868299667447052\\
574	0.0084611546732484\\
575	0.00806442919934086\\
576	0.00778388453850264\\
577	0.00768497233769221\\
578	0.00760406498573608\\
579	0.00752458055450901\\
580	0.00744407604424876\\
581	0.00736198802737892\\
582	0.00727824292195357\\
583	0.00719273371534652\\
584	0.00710538683074545\\
585	0.00701612214882769\\
586	0.00692485725962403\\
587	0.00683149725950674\\
588	0.00673592596358478\\
589	0.00663797520465071\\
590	0.00653734834917363\\
591	0.00643341581933105\\
592	0.00632466858477866\\
593	0.00620725772627071\\
594	0.00607109011801661\\
595	0.00588956027306329\\
596	0.00559184182825266\\
597	0.00498881296807123\\
598	0.00357511483354343\\
599	0\\
600	0\\
};
\addplot [color=blue!25!mycolor7,solid,forget plot]
  table[row sep=crcr]{%
1	0.0124934896979223\\
2	0.0124934880957596\\
3	0.0124934864703468\\
4	0.012493484821332\\
5	0.012493483148357\\
6	0.0124934814510573\\
7	0.0124934797290618\\
8	0.0124934779819929\\
9	0.0124934762094662\\
10	0.0124934744110903\\
11	0.0124934725864671\\
12	0.012493470735191\\
13	0.0124934688568492\\
14	0.0124934669510214\\
15	0.0124934650172798\\
16	0.0124934630551889\\
17	0.0124934610643049\\
18	0.0124934590441764\\
19	0.0124934569943435\\
20	0.012493454914338\\
21	0.0124934528036832\\
22	0.0124934506618934\\
23	0.0124934484884744\\
24	0.0124934462829227\\
25	0.0124934440447255\\
26	0.0124934417733608\\
27	0.0124934394682967\\
28	0.0124934371289919\\
29	0.0124934347548947\\
30	0.0124934323454437\\
31	0.0124934299000666\\
32	0.012493427418181\\
33	0.0124934248991935\\
34	0.0124934223424998\\
35	0.0124934197474845\\
36	0.0124934171135208\\
37	0.0124934144399702\\
38	0.0124934117261825\\
39	0.0124934089714955\\
40	0.0124934061752348\\
41	0.0124934033367134\\
42	0.0124934004552317\\
43	0.0124933975300772\\
44	0.012493394560524\\
45	0.0124933915458333\\
46	0.0124933884852521\\
47	0.0124933853780139\\
48	0.012493382223338\\
49	0.0124933790204293\\
50	0.0124933757684779\\
51	0.0124933724666592\\
52	0.0124933691141334\\
53	0.0124933657100453\\
54	0.0124933622535239\\
55	0.0124933587436824\\
56	0.0124933551796176\\
57	0.0124933515604099\\
58	0.0124933478851228\\
59	0.0124933441528028\\
60	0.0124933403624789\\
61	0.0124933365131626\\
62	0.0124933326038472\\
63	0.0124933286335079\\
64	0.0124933246011012\\
65	0.0124933205055648\\
66	0.012493316345817\\
67	0.0124933121207569\\
68	0.0124933078292635\\
69	0.0124933034701956\\
70	0.0124932990423916\\
71	0.0124932945446691\\
72	0.0124932899758244\\
73	0.0124932853346323\\
74	0.012493280619846\\
75	0.0124932758301961\\
76	0.0124932709643908\\
77	0.0124932660211155\\
78	0.0124932609990322\\
79	0.0124932558967793\\
80	0.0124932507129712\\
81	0.012493245446198\\
82	0.012493240095025\\
83	0.0124932346579924\\
84	0.0124932291336149\\
85	0.0124932235203815\\
86	0.0124932178167547\\
87	0.0124932120211706\\
88	0.0124932061320383\\
89	0.0124932001477395\\
90	0.012493194066628\\
91	0.0124931878870296\\
92	0.0124931816072417\\
93	0.0124931752255326\\
94	0.0124931687401414\\
95	0.0124931621492777\\
96	0.0124931554511208\\
97	0.01249314864382\\
98	0.0124931417254936\\
99	0.012493134694229\\
100	0.012493127548082\\
101	0.0124931202850768\\
102	0.0124931129032055\\
103	0.0124931054004278\\
104	0.0124930977746709\\
105	0.0124930900238288\\
106	0.0124930821457626\\
107	0.0124930741382997\\
108	0.0124930659992342\\
109	0.0124930577263262\\
110	0.0124930493173018\\
111	0.0124930407698534\\
112	0.0124930320816387\\
113	0.0124930232502819\\
114	0.0124930142733726\\
115	0.0124930051484666\\
116	0.0124929958730855\\
117	0.0124929864447175\\
118	0.012492976860817\\
119	0.0124929671188054\\
120	0.0124929572160715\\
121	0.0124929471499716\\
122	0.0124929369178305\\
123	0.0124929265169421\\
124	0.01249291594457\\
125	0.0124929051979488\\
126	0.0124928942742848\\
127	0.012492883170757\\
128	0.0124928718845191\\
129	0.0124928604126997\\
130	0.012492848752405\\
131	0.0124928369007194\\
132	0.0124928248547077\\
133	0.0124928126114164\\
134	0.0124928001678754\\
135	0.0124927875210997\\
136	0.0124927746680904\\
137	0.0124927616058361\\
138	0.0124927483313133\\
139	0.0124927348414866\\
140	0.0124927211333086\\
141	0.0124927072037204\\
142	0.0124926930496555\\
143	0.0124926786680577\\
144	0.0124926640559419\\
145	0.0124926492105886\\
146	0.0124926341301557\\
147	0.0124926188156553\\
148	0.0124926032778825\\
149	0.0124925875454267\\
150	0.0124925715367068\\
151	0.0124925552466545\\
152	0.0124925386701069\\
153	0.0124925218018051\\
154	0.0124925046363921\\
155	0.0124924871684115\\
156	0.0124924693923051\\
157	0.012492451302411\\
158	0.0124924328929624\\
159	0.0124924141580844\\
160	0.0124923950917933\\
161	0.0124923756879937\\
162	0.0124923559404764\\
163	0.012492335842917\\
164	0.012492315388873\\
165	0.0124922945717819\\
166	0.0124922733849591\\
167	0.0124922518215951\\
168	0.012492229874754\\
169	0.0124922075373703\\
170	0.0124921848022471\\
171	0.0124921616620532\\
172	0.0124921381093209\\
173	0.0124921141364434\\
174	0.0124920897356721\\
175	0.0124920648991141\\
176	0.0124920396187294\\
177	0.012492013886328\\
178	0.0124919876935677\\
179	0.0124919610319504\\
180	0.0124919338928201\\
181	0.0124919062673592\\
182	0.0124918781465859\\
183	0.0124918495213511\\
184	0.0124918203823353\\
185	0.012491790720045\\
186	0.0124917605248103\\
187	0.0124917297867809\\
188	0.0124916984959227\\
189	0.0124916666420151\\
190	0.0124916342146468\\
191	0.0124916012032126\\
192	0.0124915675969094\\
193	0.0124915333847333\\
194	0.0124914985554749\\
195	0.012491463097716\\
196	0.0124914269998258\\
197	0.0124913902499564\\
198	0.0124913528360393\\
199	0.0124913147457809\\
200	0.0124912759666586\\
201	0.0124912364859161\\
202	0.0124911962905594\\
203	0.0124911553673521\\
204	0.012491113702811\\
205	0.0124910712832015\\
206	0.0124910280945325\\
207	0.0124909841225523\\
208	0.0124909393527428\\
209	0.0124908937703154\\
210	0.0124908473602051\\
211	0.012490800107066\\
212	0.0124907519952654\\
213	0.0124907030088788\\
214	0.0124906531316842\\
215	0.0124906023471566\\
216	0.0124905506384623\\
217	0.0124904979884528\\
218	0.0124904443796593\\
219	0.0124903897942864\\
220	0.0124903342142056\\
221	0.0124902776209498\\
222	0.0124902199957062\\
223	0.0124901613193101\\
224	0.0124901015722381\\
225	0.0124900407346015\\
226	0.0124899787861392\\
227	0.0124899157062108\\
228	0.0124898514737893\\
229	0.0124897860674542\\
230	0.0124897194653835\\
231	0.0124896516453464\\
232	0.0124895825846959\\
233	0.0124895122603604\\
234	0.0124894406488363\\
235	0.0124893677261793\\
236	0.0124892934679966\\
237	0.0124892178494386\\
238	0.0124891408451896\\
239	0.0124890624294603\\
240	0.0124889825759777\\
241	0.0124889012579772\\
242	0.012488818448193\\
243	0.0124887341188488\\
244	0.0124886482416487\\
245	0.0124885607877676\\
246	0.0124884717278416\\
247	0.012488381031958\\
248	0.0124882886696458\\
249	0.0124881946098656\\
250	0.0124880988209994\\
251	0.0124880012708406\\
252	0.0124879019265838\\
253	0.0124878007548143\\
254	0.0124876977214977\\
255	0.01248759279197\\
256	0.0124874859309268\\
257	0.0124873771024131\\
258	0.0124872662698131\\
259	0.01248715339584\\
260	0.0124870384425261\\
261	0.0124869213712125\\
262	0.01248680214254\\
263	0.0124866807164395\\
264	0.0124865570521227\\
265	0.012486431108074\\
266	0.0124863028420416\\
267	0.0124861722110299\\
268	0.0124860391712924\\
269	0.0124859036783241\\
270	0.0124857656868552\\
271	0.012485625150844\\
272	0.0124854820234694\\
273	0.0124853362571207\\
274	0.0124851878033816\\
275	0.0124850366130013\\
276	0.0124848826358461\\
277	0.0124847258208445\\
278	0.0124845661161088\\
279	0.0124844034702233\\
280	0.0124842378286065\\
281	0.0124840691355065\\
282	0.012483897334114\\
283	0.0124837223665395\\
284	0.0124835441737902\\
285	0.0124833626957451\\
286	0.0124831778711305\\
287	0.0124829896374928\\
288	0.0124827979311713\\
289	0.0124826026872696\\
290	0.0124824038396243\\
291	0.0124822013207743\\
292	0.0124819950619258\\
293	0.0124817849929178\\
294	0.0124815710421833\\
295	0.0124813531367093\\
296	0.0124811312019941\\
297	0.0124809051620007\\
298	0.0124806749391079\\
299	0.012480440454057\\
300	0.0124802016258953\\
301	0.012479958371914\\
302	0.0124797106075826\\
303	0.0124794582464776\\
304	0.0124792012002055\\
305	0.0124789393783213\\
306	0.0124786726882417\\
307	0.0124784010351532\\
308	0.0124781243219197\\
309	0.0124778424489904\\
310	0.0124775553143208\\
311	0.0124772628133248\\
312	0.0124769648388866\\
313	0.0124766612814485\\
314	0.0124763520290608\\
315	0.0124760369665911\\
316	0.012475715971063\\
317	0.0124753889246491\\
318	0.0124750557108248\\
319	0.0124747162107271\\
320	0.0124743703031029\\
321	0.0124740178642565\\
322	0.0124736587679943\\
323	0.0124732928855684\\
324	0.012472920085619\\
325	0.0124725402341145\\
326	0.0124721531942905\\
327	0.0124717588265863\\
328	0.0124713569885796\\
329	0.0124709475349203\\
330	0.0124705303172609\\
331	0.0124701051841856\\
332	0.0124696719811369\\
333	0.0124692305503398\\
334	0.012468780730725\\
335	0.0124683223578482\\
336	0.0124678552638066\\
337	0.0124673792771525\\
338	0.0124668942228049\\
339	0.0124663999219567\\
340	0.0124658961919803\\
341	0.0124653828463309\\
342	0.0124648596944482\\
343	0.0124643265416551\\
344	0.0124637831890467\\
345	0.0124632294333734\\
346	0.012462665066898\\
347	0.0124620898772859\\
348	0.0124615036474782\\
349	0.0124609061555697\\
350	0.0124602971746649\\
351	0.0124596764727285\\
352	0.0124590438124252\\
353	0.0124583989509545\\
354	0.0124577416398742\\
355	0.0124570716249094\\
356	0.0124563886457649\\
357	0.0124556924360045\\
358	0.0124549827231162\\
359	0.0124542592282132\\
360	0.0124535216651536\\
361	0.0124527697408658\\
362	0.0124520031551274\\
363	0.0124512216003319\\
364	0.0124504247612395\\
365	0.0124496123147366\\
366	0.0124487839296874\\
367	0.0124479392665319\\
368	0.0124470779770451\\
369	0.0124461997040359\\
370	0.0124453040810644\\
371	0.0124443907322648\\
372	0.0124434592722166\\
373	0.0124425093050651\\
374	0.0124415404244795\\
375	0.0124405522133178\\
376	0.0124395442433062\\
377	0.0124385160746847\\
378	0.0124374672552204\\
379	0.0124363973184357\\
380	0.012435305786179\\
381	0.0124341921676227\\
382	0.0124330559576152\\
383	0.012431896636168\\
384	0.0124307136681108\\
385	0.0124295065027671\\
386	0.0124282745725388\\
387	0.0124270172920972\\
388	0.0124257340575738\\
389	0.0124244242453342\\
390	0.0124230872114575\\
391	0.0124217222912526\\
392	0.0124203287975411\\
393	0.0124189060204611\\
394	0.0124174532291352\\
395	0.0124159696654504\\
396	0.0124144545440628\\
397	0.0124129070514929\\
398	0.0124113263444193\\
399	0.0124097115478212\\
400	0.0124080617535538\\
401	0.0124063760212524\\
402	0.0124046533828143\\
403	0.012402892834977\\
404	0.012401093333085\\
405	0.0123992537896467\\
406	0.0123973730553565\\
407	0.0123954499460333\\
408	0.0123934832239011\\
409	0.0123914715905959\\
410	0.0123894137133371\\
411	0.0123873082072819\\
412	0.012385153627286\\
413	0.0123829484324725\\
414	0.0123806910363104\\
415	0.0123783797876818\\
416	0.0123760129475657\\
417	0.0123735887310139\\
418	0.0123711054781399\\
419	0.0123685614077457\\
420	0.0123659546179662\\
421	0.0123632827294819\\
422	0.0123605434145629\\
423	0.0123577341676305\\
424	0.0123548524919875\\
425	0.0123518957476068\\
426	0.0123488611260978\\
427	0.0123457456021872\\
428	0.0123425458399115\\
429	0.0123392580886468\\
430	0.0123358785207434\\
431	0.0123324033774937\\
432	0.0123288306324937\\
433	0.0123251594699614\\
434	0.0123213863320206\\
435	0.0123175019945348\\
436	0.0123135004459836\\
437	0.0123093763866789\\
438	0.0123051242227054\\
439	0.0123007380391701\\
440	0.0122962115337123\\
441	0.0122915378104163\\
442	0.0122867086988441\\
443	0.0122817126384959\\
444	0.0122765313924144\\
445	0.0122711730094806\\
446	0.0122656783141936\\
447	0.0122600460436909\\
448	0.0122542044383031\\
449	0.0122481370496886\\
450	0.0122418255387772\\
451	0.0122352493046356\\
452	0.0122283847413856\\
453	0.0122212028493017\\
454	0.012213657087294\\
455	0.0122057179546538\\
456	0.0121974177852449\\
457	0.0121887350120657\\
458	0.0121796026867837\\
459	0.012169910760579\\
460	0.0121484525807216\\
461	0.0121225439671586\\
462	0.0120960687551813\\
463	0.0120689901445763\\
464	0.0120412671721926\\
465	0.0120128532454812\\
466	0.0119836943681672\\
467	0.0119537292121411\\
468	0.0119228897545696\\
469	0.0118910991619444\\
470	0.0118582694796218\\
471	0.0118242995178252\\
472	0.011789072511394\\
473	0.0117386084559066\\
474	0.0116810405201921\\
475	0.0116228033167768\\
476	0.0115638986607994\\
477	0.0115043304356759\\
478	0.0114441107324253\\
479	0.0113832567636374\\
480	0.0113217927406389\\
481	0.0112597506258005\\
482	0.0111971719801715\\
483	0.0111341098150239\\
484	0.0110706301898503\\
485	0.0110154864063124\\
486	0.0109933315999276\\
487	0.0109714636791583\\
488	0.0109499493266452\\
489	0.0109288637691311\\
490	0.0109082919550504\\
491	0.0108883299229187\\
492	0.0108690864107806\\
493	0.0108506847274538\\
494	0.0108332649923816\\
495	0.0108169867104067\\
496	0.0108020320341635\\
497	0.0107883763847002\\
498	0.0107747702255624\\
499	0.0107612163754469\\
500	0.0107477155937402\\
501	0.0107342659735556\\
502	0.0107208622008048\\
503	0.0107074946475425\\
504	0.0106941482646422\\
505	0.0106808012267657\\
506	0.0106674232797821\\
507	0.010653973729537\\
508	0.0106403989941711\\
509	0.0106266558014479\\
510	0.0106127353669084\\
511	0.0105986281112579\\
512	0.0105843236646759\\
513	0.0105698108994153\\
514	0.0105550780014204\\
515	0.0105401125946322\\
516	0.0105249019360662\\
517	0.0105094332044547\\
518	0.0104936939115285\\
519	0.0104776724734513\\
520	0.0104613589903642\\
521	0.0104447437527462\\
522	0.0104278167420056\\
523	0.0104105676451535\\
524	0.0103929858717878\\
525	0.0103750605723023\\
526	0.0103567806554969\\
527	0.0103381348027425\\
528	0.0103191114744017\\
529	0.0102996989022278\\
530	0.0102798850587414\\
531	0.0102596575908799\\
532	0.0102390037430913\\
533	0.0102179103492875\\
534	0.0101963638242602\\
535	0.0101743501544096\\
536	0.010151854887576\\
537	0.0101288631217804\\
538	0.0101053594935276\\
539	0.0100813281652646\\
540	0.0100567528104512\\
541	0.0100316165983594\\
542	0.0100059025314179\\
543	0.00997960737655468\\
544	0.00995269842410414\\
545	0.00992515671607556\\
546	0.00989696291884511\\
547	0.00986809291814794\\
548	0.00983852351201753\\
549	0.00980822536473573\\
550	0.00977717950896099\\
551	0.00974536590116786\\
552	0.00971276030523118\\
553	0.00967934025184406\\
554	0.00964507332415307\\
555	0.00960997266384452\\
556	0.0095740133838415\\
557	0.00953716611704428\\
558	0.0094994041575164\\
559	0.00946072850645546\\
560	0.00942110960231462\\
561	0.00938051491477584\\
562	0.00933888295206795\\
563	0.00929620306916978\\
564	0.00925246879248052\\
565	0.00920752969012616\\
566	0.00916151708845009\\
567	0.00911442729326383\\
568	0.00906639133305028\\
569	0.00901719711457537\\
570	0.00896631970186846\\
571	0.00891408668734256\\
572	0.00886028503627413\\
573	0.00880513086802375\\
574	0.00872464282348522\\
575	0.00855320411017449\\
576	0.00828883645919685\\
577	0.0078978638763854\\
578	0.00763833815339484\\
579	0.00753152624116889\\
580	0.00744580350857311\\
581	0.00736281717889\\
582	0.00727854262256624\\
583	0.00719288419631409\\
584	0.00710545584045065\\
585	0.00701615980828359\\
586	0.0069248749165042\\
587	0.00683150671091789\\
588	0.00673592931731008\\
589	0.00663797579971203\\
590	0.00653734834917363\\
591	0.00643341581933106\\
592	0.00632466858477866\\
593	0.00620725772627072\\
594	0.00607109011801661\\
595	0.00588956027306329\\
596	0.00559184182825266\\
597	0.00498881296807123\\
598	0.00357511483354343\\
599	0\\
600	0\\
};
\addplot [color=mycolor9,solid,forget plot]
  table[row sep=crcr]{%
1	0.0125042813964841\\
2	0.012504279863156\\
3	0.0125042783059943\\
4	0.0125042767246081\\
5	0.0125042751185993\\
6	0.0125042734875627\\
7	0.0125042718310859\\
8	0.0125042701487491\\
9	0.012504268440125\\
10	0.0125042667047787\\
11	0.0125042649422673\\
12	0.0125042631521402\\
13	0.0125042613339387\\
14	0.0125042594871957\\
15	0.0125042576114357\\
16	0.012504255706175\\
17	0.0125042537709208\\
18	0.0125042518051718\\
19	0.0125042498084175\\
20	0.0125042477801382\\
21	0.0125042457198049\\
22	0.0125042436268793\\
23	0.0125042415008131\\
24	0.0125042393410485\\
25	0.0125042371470174\\
26	0.0125042349181415\\
27	0.0125042326538324\\
28	0.0125042303534909\\
29	0.012504228016507\\
30	0.01250422564226\\
31	0.0125042232301178\\
32	0.0125042207794371\\
33	0.0125042182895631\\
34	0.0125042157598292\\
35	0.0125042131895569\\
36	0.0125042105780553\\
37	0.0125042079246216\\
38	0.01250420522854\\
39	0.0125042024890821\\
40	0.0125041997055065\\
41	0.0125041968770583\\
42	0.0125041940029694\\
43	0.0125041910824579\\
44	0.0125041881147279\\
45	0.0125041850989693\\
46	0.0125041820343575\\
47	0.0125041789200535\\
48	0.012504175755203\\
49	0.0125041725389366\\
50	0.0125041692703697\\
51	0.0125041659486015\\
52	0.0125041625727157\\
53	0.0125041591417793\\
54	0.0125041556548432\\
55	0.012504152110941\\
56	0.0125041485090895\\
57	0.012504144848288\\
58	0.0125041411275181\\
59	0.0125041373457435\\
60	0.0125041335019094\\
61	0.0125041295949425\\
62	0.0125041256237507\\
63	0.0125041215872223\\
64	0.0125041174842265\\
65	0.0125041133136123\\
66	0.0125041090742084\\
67	0.0125041047648231\\
68	0.0125041003842438\\
69	0.0125040959312366\\
70	0.0125040914045459\\
71	0.0125040868028942\\
72	0.0125040821249817\\
73	0.0125040773694858\\
74	0.0125040725350609\\
75	0.0125040676203378\\
76	0.0125040626239237\\
77	0.0125040575444012\\
78	0.0125040523803285\\
79	0.0125040471302388\\
80	0.0125040417926397\\
81	0.0125040363660129\\
82	0.0125040308488141\\
83	0.012504025239472\\
84	0.0125040195363882\\
85	0.0125040137379369\\
86	0.012504007842464\\
87	0.0125040018482873\\
88	0.0125039957536955\\
89	0.0125039895569478\\
90	0.0125039832562738\\
91	0.0125039768498727\\
92	0.012503970335913\\
93	0.012503963712532\\
94	0.0125039569778351\\
95	0.0125039501298957\\
96	0.0125039431667544\\
97	0.0125039360864189\\
98	0.0125039288868629\\
99	0.0125039215660261\\
100	0.0125039141218136\\
101	0.0125039065520955\\
102	0.0125038988547059\\
103	0.0125038910274432\\
104	0.0125038830680689\\
105	0.0125038749743075\\
106	0.0125038667438459\\
107	0.0125038583743329\\
108	0.0125038498633787\\
109	0.0125038412085545\\
110	0.0125038324073921\\
111	0.012503823457383\\
112	0.0125038143559786\\
113	0.0125038051005892\\
114	0.0125037956885839\\
115	0.0125037861172899\\
116	0.0125037763839924\\
117	0.0125037664859339\\
118	0.0125037564203142\\
119	0.0125037461842894\\
120	0.0125037357749724\\
121	0.0125037251894319\\
122	0.012503714424692\\
123	0.0125037034777327\\
124	0.0125036923454888\\
125	0.01250368102485\\
126	0.0125036695126603\\
127	0.0125036578057182\\
128	0.0125036459007761\\
129	0.0125036337945399\\
130	0.0125036214836688\\
131	0.012503608964775\\
132	0.0125035962344226\\
133	0.0125035832891278\\
134	0.0125035701253572\\
135	0.0125035567395276\\
136	0.0125035431280038\\
137	0.0125035292870977\\
138	0.0125035152130657\\
139	0.0125035009021064\\
140	0.0125034863503579\\
141	0.012503471553897\\
142	0.0125034565087424\\
143	0.0125034412108735\\
144	0.0125034256562931\\
145	0.0125034098412112\\
146	0.0125033937625495\\
147	0.0125033774192035\\
148	0.01250336081429\\
149	0.0125033439507468\\
150	0.0125033267890371\\
151	0.0125033093237698\\
152	0.0125032915494555\\
153	0.0125032734605051\\
154	0.0125032550512279\\
155	0.0125032363158293\\
156	0.0125032172484099\\
157	0.0125031978429624\\
158	0.0125031780933706\\
159	0.012503157993407\\
160	0.0125031375367307\\
161	0.0125031167168857\\
162	0.0125030955272985\\
163	0.012503073961276\\
164	0.0125030520120036\\
165	0.0125030296725425\\
166	0.012503006935828\\
167	0.012502983794667\\
168	0.0125029602417355\\
169	0.0125029362695764\\
170	0.0125029118705971\\
171	0.0125028870370672\\
172	0.0125028617611154\\
173	0.0125028360347278\\
174	0.0125028098497447\\
175	0.0125027831978582\\
176	0.0125027560706094\\
177	0.0125027284593859\\
178	0.0125027003554187\\
179	0.0125026717497797\\
180	0.0125026426333787\\
181	0.0125026129969601\\
182	0.0125025828311008\\
183	0.0125025521262061\\
184	0.0125025208725074\\
185	0.0125024890600587\\
186	0.0125024566787337\\
187	0.0125024237182219\\
188	0.012502390168026\\
189	0.0125023560174582\\
190	0.0125023212556367\\
191	0.0125022858714824\\
192	0.0125022498537149\\
193	0.0125022131908496\\
194	0.0125021758711933\\
195	0.012502137882841\\
196	0.0125020992136714\\
197	0.0125020598513437\\
198	0.0125020197832933\\
199	0.0125019789967277\\
200	0.0125019374786227\\
201	0.0125018952157176\\
202	0.0125018521945117\\
203	0.0125018084012593\\
204	0.0125017638219658\\
205	0.0125017184423829\\
206	0.0125016722480039\\
207	0.0125016252240594\\
208	0.0125015773555123\\
209	0.0125015286270531\\
210	0.0125014790230948\\
211	0.0125014285277679\\
212	0.0125013771249155\\
213	0.012501324798088\\
214	0.0125012715305376\\
215	0.0125012173052132\\
216	0.0125011621047545\\
217	0.012501105911487\\
218	0.0125010487074156\\
219	0.0125009904742192\\
220	0.0125009311932449\\
221	0.0125008708455015\\
222	0.0125008094116536\\
223	0.0125007468720152\\
224	0.0125006832065436\\
225	0.0125006183948325\\
226	0.0125005524161053\\
227	0.0125004852492088\\
228	0.0125004168726057\\
229	0.0125003472643677\\
230	0.0125002764021684\\
231	0.0125002042632757\\
232	0.0125001308245446\\
233	0.0125000560624088\\
234	0.0124999799528736\\
235	0.0124999024715071\\
236	0.0124998235934327\\
237	0.0124997432933199\\
238	0.012499661545376\\
239	0.0124995783233373\\
240	0.0124994936004597\\
241	0.0124994073495096\\
242	0.0124993195427543\\
243	0.0124992301519518\\
244	0.0124991391483412\\
245	0.0124990465026319\\
246	0.012498952184993\\
247	0.0124988561650423\\
248	0.0124987584118348\\
249	0.012498658893851\\
250	0.0124985575789849\\
251	0.0124984544345317\\
252	0.0124983494271742\\
253	0.0124982425229702\\
254	0.0124981336873387\\
255	0.0124980228850451\\
256	0.0124979100801874\\
257	0.0124977952361805\\
258	0.012497678315741\\
259	0.0124975592808711\\
260	0.0124974380928425\\
261	0.0124973147121796\\
262	0.0124971890986425\\
263	0.0124970612112107\\
264	0.0124969310080653\\
265	0.0124967984465734\\
266	0.0124966634832716\\
267	0.0124965260738513\\
268	0.012496386173146\\
269	0.0124962437351195\\
270	0.0124960987128586\\
271	0.0124959510585669\\
272	0.0124958007235603\\
273	0.0124956476582492\\
274	0.0124954918120677\\
275	0.0124953331331845\\
276	0.0124951715673754\\
277	0.0124950070533906\\
278	0.0124948395015193\\
279	0.0124946686712862\\
280	0.0124944948102393\\
281	0.0124943178863465\\
282	0.0124941378483797\\
283	0.0124939546444442\\
284	0.0124937682219852\\
285	0.0124935785277959\\
286	0.0124933855080263\\
287	0.0124931891081952\\
288	0.0124929892732033\\
289	0.012492785947348\\
290	0.0124925790743417\\
291	0.0124923685973309\\
292	0.0124921544589191\\
293	0.0124919366011912\\
294	0.0124917149657412\\
295	0.0124914894937022\\
296	0.0124912601257788\\
297	0.0124910268022822\\
298	0.0124907894631671\\
299	0.0124905480480709\\
300	0.0124903024963525\\
301	0.0124900527471327\\
302	0.0124897987393324\\
303	0.0124895404117075\\
304	0.012489277702879\\
305	0.0124890105513551\\
306	0.0124887388955438\\
307	0.0124884626737555\\
308	0.0124881818242108\\
309	0.0124878962850951\\
310	0.0124876059948107\\
311	0.0124873108928717\\
312	0.0124870109228012\\
313	0.0124867060412385\\
314	0.0124863962469504\\
315	0.0124860816786911\\
316	0.0124857629914215\\
317	0.0124854390499873\\
318	0.0124851092585999\\
319	0.0124847735129621\\
320	0.0124844317069656\\
321	0.0124840837326609\\
322	0.0124837294802262\\
323	0.0124833688379362\\
324	0.0124830016921301\\
325	0.0124826279271796\\
326	0.012482247425456\\
327	0.0124818600672971\\
328	0.0124814657309735\\
329	0.0124810642926547\\
330	0.0124806556263746\\
331	0.0124802396039965\\
332	0.0124798160951785\\
333	0.0124793849673377\\
334	0.0124789460856146\\
335	0.0124784993128374\\
336	0.012478044509486\\
337	0.0124775815336559\\
338	0.0124771102410219\\
339	0.012476630484803\\
340	0.0124761421157266\\
341	0.0124756449819938\\
342	0.0124751389292448\\
343	0.0124746238005248\\
344	0.0124740994362498\\
345	0.0124735656741724\\
346	0.0124730223493508\\
347	0.0124724692941191\\
348	0.0124719063380572\\
349	0.0124713333079614\\
350	0.0124707500278139\\
351	0.0124701563187508\\
352	0.0124695519990265\\
353	0.0124689368839671\\
354	0.0124683107859021\\
355	0.0124676735140504\\
356	0.0124670248743245\\
357	0.0124663646690444\\
358	0.0124656926969143\\
359	0.0124650087556106\\
360	0.0124643126403343\\
361	0.012463604133633\\
362	0.0124628830135796\\
363	0.0124621490536067\\
364	0.0124614020222679\\
365	0.0124606416827807\\
366	0.0124598677923641\\
367	0.0124590801032656\\
368	0.0124582783618723\\
369	0.0124574623083692\\
370	0.0124566316764213\\
371	0.0124557861928318\\
372	0.0124549255771381\\
373	0.0124540495412789\\
374	0.01245315778928\\
375	0.012452250017062\\
376	0.0124513259124923\\
377	0.0124503851557323\\
378	0.0124494274193314\\
379	0.012448452365398\\
380	0.0124474596283017\\
381	0.0124464488599826\\
382	0.012445419721733\\
383	0.0124443718672787\\
384	0.0124433049425044\\
385	0.0124422185850834\\
386	0.0124411124241317\\
387	0.0124399860798356\\
388	0.0124388391630253\\
389	0.0124376712747637\\
390	0.0124364820058789\\
391	0.0124352709363835\\
392	0.0124340376349621\\
393	0.0124327816584449\\
394	0.0124315025506512\\
395	0.0124301998416107\\
396	0.0124288730465825\\
397	0.0124275216648607\\
398	0.0124261451783975\\
399	0.0124247430502708\\
400	0.012423314723067\\
401	0.0124218596170114\\
402	0.0124203771267264\\
403	0.0124188666175301\\
404	0.0124173274211892\\
405	0.0124157588300651\\
406	0.0124141600946032\\
407	0.0124125304153304\\
408	0.0124108689318808\\
409	0.0124091747201955\\
410	0.0124074467894693\\
411	0.0124056840961264\\
412	0.0124038855800256\\
413	0.0124020501093927\\
414	0.0124001754678643\\
415	0.0123982599804733\\
416	0.012396302559117\\
417	0.0123943020900616\\
418	0.0123922574125513\\
419	0.0123901673130505\\
420	0.0123880305021444\\
421	0.0123858456507439\\
422	0.0123836113723348\\
423	0.0123813262319508\\
424	0.0123789887331281\\
425	0.0123765973159215\\
426	0.0123741503452912\\
427	0.0123716461012829\\
428	0.0123690827752048\\
429	0.0123664585001967\\
430	0.0123637713795259\\
431	0.0123610195900388\\
432	0.0123582012731023\\
433	0.0123553141942935\\
434	0.0123523556350233\\
435	0.0123493230463589\\
436	0.0123462138357786\\
437	0.0123430252714849\\
438	0.0123397544677696\\
439	0.0123363983619402\\
440	0.0123329536711212\\
441	0.0123294168069508\\
442	0.0123257837522329\\
443	0.0123220502643853\\
444	0.0123182142760293\\
445	0.0123142750316562\\
446	0.012310227560643\\
447	0.0123060624933546\\
448	0.0123017745491636\\
449	0.0122973581259335\\
450	0.012292807248134\\
451	0.0122881154333784\\
452	0.0122832752883144\\
453	0.0122782772912278\\
454	0.0122731077643023\\
455	0.0122677658691481\\
456	0.0122622753072779\\
457	0.0122566342913368\\
458	0.0122508214742564\\
459	0.0122448030757952\\
460	0.012238569051616\\
461	0.0122321038581154\\
462	0.0122253871779358\\
463	0.0122183877425042\\
464	0.0122110801628658\\
465	0.0122034644462425\\
466	0.012195571063476\\
467	0.012187361779633\\
468	0.0121787567379619\\
469	0.0121696938498687\\
470	0.012160125571854\\
471	0.0121499975585025\\
472	0.0121392474514685\\
473	0.012116281940348\\
474	0.0120875117292555\\
475	0.0120580980725064\\
476	0.0120279958636458\\
477	0.0119971541043067\\
478	0.0119655174753707\\
479	0.0119330235342684\\
480	0.0118996018511712\\
481	0.0118651722843274\\
482	0.0118296433086857\\
483	0.0117929098040878\\
484	0.0117548500491487\\
485	0.0117107522708203\\
486	0.0116477830587934\\
487	0.0115840543321413\\
488	0.0115195655239011\\
489	0.0114543215079448\\
490	0.0113883319428579\\
491	0.011321612301923\\
492	0.0112541847215448\\
493	0.0111860799831065\\
494	0.0111173382814737\\
495	0.0110480137380876\\
496	0.0109781715812912\\
497	0.0109142993365318\\
498	0.0108894084779344\\
499	0.0108648011256737\\
500	0.0108405517327763\\
501	0.0108167447507493\\
502	0.0107934759728688\\
503	0.0107708541884305\\
504	0.0107490031066106\\
505	0.0107280637373214\\
506	0.0107081971347483\\
507	0.0106895876388807\\
508	0.0106724468797263\\
509	0.0106563430009931\\
510	0.010640256212759\\
511	0.010624188281311\\
512	0.0106081384693105\\
513	0.0105921027885619\\
514	0.0105760730741135\\
515	0.0105600358460041\\
516	0.0105439709004266\\
517	0.0105278495835077\\
518	0.0105116326831166\\
519	0.010495267849565\\
520	0.0104786864472087\\
521	0.0104618629634467\\
522	0.010444785281913\\
523	0.0104274402750027\\
524	0.010409813813195\\
525	0.0103918908125962\\
526	0.0103736553352649\\
527	0.010355090761378\\
528	0.0103361800582437\\
529	0.0103169061783647\\
530	0.0102972526281948\\
531	0.0102772042613134\\
532	0.0102567473374211\\
533	0.010235867642011\\
534	0.0102145505011398\\
535	0.0101927808003158\\
536	0.0101705430067024\\
537	0.0101478211930789\\
538	0.0101245990608679\\
539	0.0101008599579727\\
540	0.0100765868850004\\
541	0.010051762480306\\
542	0.0100263689700029\\
543	0.0100003928707983\\
544	0.00997382390675952\\
545	0.00994663028234151\\
546	0.00991879207359581\\
547	0.00989028880719799\\
548	0.00986109951605072\\
549	0.00983119736487534\\
550	0.0098005491034348\\
551	0.00976913474017009\\
552	0.00973693074790581\\
553	0.00970391543383491\\
554	0.00967007172010916\\
555	0.00963537658421949\\
556	0.00959980612120528\\
557	0.00956333290141669\\
558	0.00952590825509406\\
559	0.00948755089445722\\
560	0.00944823206997917\\
561	0.00940792046637175\\
562	0.00936656211640174\\
563	0.00932417644149528\\
564	0.0092807369301494\\
565	0.00923610933042622\\
566	0.0091903927955922\\
567	0.00914356761387222\\
568	0.00909544730654022\\
569	0.00904609248987307\\
570	0.00899554686237181\\
571	0.00894379302847974\\
572	0.00889070804722582\\
573	0.0088360322748802\\
574	0.00877996467453759\\
575	0.00872201521456251\\
576	0.00861400657524299\\
577	0.00844456166084187\\
578	0.00817323631074009\\
579	0.00779191677837049\\
580	0.00749650363288286\\
581	0.00737443998447818\\
582	0.00728422197907741\\
583	0.00719482330235922\\
584	0.0071064288968317\\
585	0.00701658985358563\\
586	0.006925112872923\\
587	0.00683161554949806\\
588	0.00673599013947704\\
589	0.00663799811207612\\
590	0.00653735241989114\\
591	0.00643341581933105\\
592	0.00632466858477866\\
593	0.00620725772627072\\
594	0.00607109011801661\\
595	0.00588956027306329\\
596	0.00559184182825266\\
597	0.00498881296807123\\
598	0.00357511483354343\\
599	0\\
600	0\\
};
\addplot [color=blue!50!mycolor7,solid,forget plot]
  table[row sep=crcr]{%
1	0.0125539685419108\\
2	0.012553966399082\\
3	0.0125539642208014\\
4	0.0125539620064649\\
5	0.012553959755458\\
6	0.0125539574671551\\
7	0.0125539551409199\\
8	0.0125539527761048\\
9	0.012553950372051\\
10	0.012553947928088\\
11	0.0125539454435336\\
12	0.0125539429176937\\
13	0.012553940349862\\
14	0.0125539377393198\\
15	0.0125539350853357\\
16	0.0125539323871658\\
17	0.0125539296440529\\
18	0.0125539268552267\\
19	0.0125539240199031\\
20	0.0125539211372846\\
21	0.0125539182065597\\
22	0.0125539152269025\\
23	0.0125539121974728\\
24	0.0125539091174155\\
25	0.0125539059858609\\
26	0.0125539028019235\\
27	0.0125538995647029\\
28	0.0125538962732826\\
29	0.01255389292673\\
30	0.0125538895240965\\
31	0.0125538860644166\\
32	0.0125538825467081\\
33	0.0125538789699714\\
34	0.0125538753331897\\
35	0.0125538716353281\\
36	0.012553867875334\\
37	0.012553864052136\\
38	0.0125538601646442\\
39	0.0125538562117495\\
40	0.0125538521923237\\
41	0.0125538481052185\\
42	0.0125538439492658\\
43	0.0125538397232769\\
44	0.0125538354260426\\
45	0.0125538310563323\\
46	0.0125538266128941\\
47	0.0125538220944541\\
48	0.0125538174997161\\
49	0.0125538128273615\\
50	0.0125538080760486\\
51	0.012553803244412\\
52	0.0125537983310629\\
53	0.0125537933345879\\
54	0.0125537882535491\\
55	0.0125537830864834\\
56	0.0125537778319023\\
57	0.0125537724882912\\
58	0.0125537670541091\\
59	0.0125537615277882\\
60	0.012553755907733\\
61	0.0125537501923207\\
62	0.0125537443798997\\
63	0.0125537384687897\\
64	0.0125537324572813\\
65	0.012553726343635\\
66	0.012553720126081\\
67	0.0125537138028187\\
68	0.012553707372016\\
69	0.0125537008318089\\
70	0.0125536941803007\\
71	0.0125536874155618\\
72	0.0125536805356286\\
73	0.0125536735385036\\
74	0.0125536664221542\\
75	0.0125536591845124\\
76	0.0125536518234739\\
77	0.0125536443368978\\
78	0.0125536367226059\\
79	0.0125536289783816\\
80	0.0125536211019701\\
81	0.0125536130910766\\
82	0.0125536049433666\\
83	0.0125535966564646\\
84	0.0125535882279538\\
85	0.0125535796553749\\
86	0.0125535709362256\\
87	0.01255356206796\\
88	0.0125535530479875\\
89	0.0125535438736722\\
90	0.0125535345423322\\
91	0.0125535250512385\\
92	0.0125535153976142\\
93	0.0125535055786342\\
94	0.0125534955914235\\
95	0.0125534854330569\\
96	0.0125534751005578\\
97	0.0125534645908978\\
98	0.0125534539009949\\
99	0.0125534430277135\\
100	0.0125534319678626\\
101	0.0125534207181957\\
102	0.0125534092754088\\
103	0.0125533976361403\\
104	0.0125533857969695\\
105	0.0125533737544155\\
106	0.0125533615049363\\
107	0.0125533490449279\\
108	0.0125533363707226\\
109	0.0125533234785884\\
110	0.0125533103647278\\
111	0.0125532970252763\\
112	0.0125532834563015\\
113	0.0125532696538018\\
114	0.012553255613705\\
115	0.0125532413318674\\
116	0.0125532268040721\\
117	0.0125532120260279\\
118	0.0125531969933678\\
119	0.0125531817016478\\
120	0.0125531661463453\\
121	0.0125531503228579\\
122	0.0125531342265015\\
123	0.0125531178525091\\
124	0.0125531011960292\\
125	0.012553084252124\\
126	0.0125530670157678\\
127	0.0125530494818452\\
128	0.0125530316451497\\
129	0.0125530135003811\\
130	0.0125529950421442\\
131	0.0125529762649465\\
132	0.0125529571631961\\
133	0.0125529377311993\\
134	0.0125529179631586\\
135	0.01255289785317\\
136	0.0125528773952204\\
137	0.0125528565831849\\
138	0.0125528354108241\\
139	0.0125528138717809\\
140	0.0125527919595785\\
141	0.0125527696676186\\
142	0.012552746989183\\
143	0.0125527239174435\\
144	0.0125527004454877\\
145	0.0125526765663813\\
146	0.0125526522732679\\
147	0.0125526275593746\\
148	0.012552602417158\\
149	0.0125525768359298\\
150	0.0125525508079685\\
151	0.0125525243254168\\
152	0.0125524973802789\\
153	0.0125524699644186\\
154	0.0125524420695562\\
155	0.0125524136872667\\
156	0.0125523848089767\\
157	0.0125523554259619\\
158	0.0125523255293448\\
159	0.0125522951100915\\
160	0.0125522641590094\\
161	0.0125522326667443\\
162	0.0125522006237774\\
163	0.0125521680204227\\
164	0.0125521348468238\\
165	0.0125521010929512\\
166	0.012552066748599\\
167	0.012552031803382\\
168	0.0125519962467326\\
169	0.0125519600678976\\
170	0.0125519232559347\\
171	0.0125518857997096\\
172	0.0125518476878923\\
173	0.0125518089089542\\
174	0.0125517694511639\\
175	0.0125517293025843\\
176	0.0125516884510687\\
177	0.0125516468842574\\
178	0.0125516045895737\\
179	0.0125515615542202\\
180	0.0125515177651753\\
181	0.0125514732091887\\
182	0.0125514278727782\\
183	0.0125513817422249\\
184	0.0125513348035696\\
185	0.0125512870426085\\
186	0.0125512384448888\\
187	0.0125511889957047\\
188	0.0125511386800928\\
189	0.0125510874828277\\
190	0.0125510353884175\\
191	0.0125509823810991\\
192	0.0125509284448337\\
193	0.0125508735633018\\
194	0.0125508177198986\\
195	0.0125507608977289\\
196	0.0125507030796022\\
197	0.0125506442480276\\
198	0.0125505843852087\\
199	0.012550523473038\\
200	0.0125504614930922\\
201	0.0125503984266262\\
202	0.0125503342545677\\
203	0.012550268957512\\
204	0.0125502025157157\\
205	0.0125501349090913\\
206	0.0125500661172013\\
207	0.0125499961192519\\
208	0.0125499248940875\\
209	0.0125498524201839\\
210	0.0125497786756425\\
211	0.0125497036381835\\
212	0.0125496272851399\\
213	0.0125495495934505\\
214	0.0125494705396535\\
215	0.0125493900998794\\
216	0.0125493082498444\\
217	0.0125492249648432\\
218	0.0125491402197419\\
219	0.0125490539889709\\
220	0.0125489662465171\\
221	0.012548876965917\\
222	0.0125487861202487\\
223	0.0125486936821244\\
224	0.0125485996236822\\
225	0.0125485039165785\\
226	0.0125484065319798\\
227	0.0125483074405545\\
228	0.0125482066124642\\
229	0.0125481040173556\\
230	0.0125479996243518\\
231	0.0125478934020433\\
232	0.0125477853184791\\
233	0.012547675341158\\
234	0.0125475634370188\\
235	0.0125474495724314\\
236	0.012547333713187\\
237	0.0125472158244885\\
238	0.0125470958709407\\
239	0.0125469738165402\\
240	0.012546849624665\\
241	0.0125467232580648\\
242	0.0125465946788495\\
243	0.0125464638484795\\
244	0.0125463307277538\\
245	0.0125461952767997\\
246	0.0125460574550612\\
247	0.0125459172212873\\
248	0.0125457745335209\\
249	0.0125456293490864\\
250	0.0125454816245778\\
251	0.0125453313158465\\
252	0.0125451783779891\\
253	0.0125450227653341\\
254	0.01254486443143\\
255	0.0125447033290315\\
256	0.0125445394100871\\
257	0.0125443726257255\\
258	0.0125442029262424\\
259	0.0125440302610872\\
260	0.0125438545788501\\
261	0.0125436758272484\\
262	0.0125434939531138\\
263	0.0125433089023801\\
264	0.0125431206200705\\
265	0.0125429290502868\\
266	0.0125427341361986\\
267	0.0125425358200341\\
268	0.0125423340430724\\
269	0.0125421287456377\\
270	0.0125419198670949\\
271	0.012541707345846\\
272	0.0125414911193207\\
273	0.0125412711239449\\
274	0.0125410472950238\\
275	0.0125408195663476\\
276	0.0125405878688892\\
277	0.0125403521266324\\
278	0.0125401122446321\\
279	0.0125398680975298\\
280	0.0125396197518342\\
281	0.0125393671467273\\
282	0.012539110212118\\
283	0.0125388488769173\\
284	0.0125385830690327\\
285	0.012538312715364\\
286	0.012538037741799\\
287	0.0125377580732105\\
288	0.0125374736334527\\
289	0.0125371843453599\\
290	0.0125368901307449\\
291	0.0125365909103974\\
292	0.0125362866040847\\
293	0.012535977130551\\
294	0.0125356624075185\\
295	0.0125353423516877\\
296	0.0125350168787384\\
297	0.0125346859033298\\
298	0.0125343493391\\
299	0.0125340070986645\\
300	0.0125336590936123\\
301	0.0125333052344994\\
302	0.0125329454308381\\
303	0.0125325795910816\\
304	0.0125322076226013\\
305	0.0125318294316566\\
306	0.0125314449233571\\
307	0.0125310540016207\\
308	0.012530656569147\\
309	0.0125302525274589\\
310	0.0125298417771743\\
311	0.0125294242189521\\
312	0.0125289997563365\\
313	0.0125285683037279\\
314	0.012528129807243\\
315	0.0125276842903027\\
316	0.0125272318628926\\
317	0.012526771893439\\
318	0.0125263040346894\\
319	0.0125258281580859\\
320	0.0125253441332608\\
321	0.0125248518280226\\
322	0.0125243511083424\\
323	0.0125238418383405\\
324	0.0125233238802737\\
325	0.0125227970945226\\
326	0.0125222613395799\\
327	0.0125217164720388\\
328	0.012521162346582\\
329	0.0125205988159718\\
330	0.0125200257310399\\
331	0.0125194429406791\\
332	0.0125188502918345\\
333	0.0125182476294967\\
334	0.0125176347966954\\
335	0.0125170116344945\\
336	0.0125163779819879\\
337	0.0125157336762976\\
338	0.0125150785525734\\
339	0.0125144124439944\\
340	0.012513735181774\\
341	0.0125130465951674\\
342	0.0125123465114832\\
343	0.0125116347561004\\
344	0.0125109111524907\\
345	0.0125101755222495\\
346	0.0125094276851356\\
347	0.0125086674591237\\
348	0.0125078946604713\\
349	0.0125071091038047\\
350	0.0125063106022268\\
351	0.012505498967451\\
352	0.0125046740099563\\
353	0.0125038355391378\\
354	0.0125029833633412\\
355	0.0125021172893665\\
356	0.012501237119844\\
357	0.0125003426418489\\
358	0.0124994335750054\\
359	0.0124985092887477\\
360	0.0124975694338948\\
361	0.0124966150013712\\
362	0.0124956458335584\\
363	0.0124946617760865\\
364	0.012493662678593\\
365	0.0124926483951677\\
366	0.0124916187822288\\
367	0.0124905736896656\\
368	0.0124895129817267\\
369	0.0124884365284323\\
370	0.0124873442056209\\
371	0.0124862358953432\\
372	0.012485111486295\\
373	0.0124839708744501\\
374	0.0124828139644341\\
375	0.0124816406732979\\
376	0.0124804509417551\\
377	0.0124792447686313\\
378	0.0124780223194872\\
379	0.0124767842878654\\
380	0.0124755332908602\\
381	0.0124742656829281\\
382	0.0124729790186189\\
383	0.0124716731448712\\
384	0.0124703479191356\\
385	0.0124690032108148\\
386	0.0124676389028829\\
387	0.0124662548937047\\
388	0.0124648510990928\\
389	0.0124634274546289\\
390	0.0124619839182908\\
391	0.0124605204734457\\
392	0.012459037132254\\
393	0.012457533939517\\
394	0.0124560109771211\\
395	0.0124544683691344\\
396	0.0124529062876691\\
397	0.0124513249596597\\
398	0.0124497246747286\\
399	0.0124481057943539\\
400	0.0124464687625691\\
401	0.0124448141184359\\
402	0.0124431425107317\\
403	0.012441454715226\\
404	0.012439751654775\\
405	0.0124380344225396\\
406	0.0124363043092761\\
407	0.0124345628547054\\
408	0.0124328118954527\\
409	0.0124310535745122\\
410	0.0124292904567631\\
411	0.0124275256758275\\
412	0.0124257632536774\\
413	0.0124240089223275\\
414	0.0124222719390955\\
415	0.0124205205792832\\
416	0.0124187349564255\\
417	0.0124169142951179\\
418	0.0124150577916102\\
419	0.0124131646114102\\
420	0.012411233891286\\
421	0.0124092647380639\\
422	0.0124072562293581\\
423	0.0124052074077761\\
424	0.0124031172530648\\
425	0.0124009847230644\\
426	0.0123988088003222\\
427	0.0123965884314126\\
428	0.0123943225285245\\
429	0.0123920099710475\\
430	0.0123896496092874\\
431	0.0123872402488307\\
432	0.012384780621131\\
433	0.0123822693822572\\
434	0.0123797051597763\\
435	0.0123770865322508\\
436	0.0123744120181398\\
437	0.0123716800713922\\
438	0.0123688890758683\\
439	0.0123660373378166\\
440	0.0123631230766227\\
441	0.0123601444222086\\
442	0.0123570994589875\\
443	0.0123539863805868\\
444	0.0123508033216867\\
445	0.0123475479501921\\
446	0.0123442175795041\\
447	0.0123408097466111\\
448	0.0123373218590584\\
449	0.0123337511725555\\
450	0.0123300947644995\\
451	0.012326349482727\\
452	0.0123225118709252\\
453	0.0123185782771384\\
454	0.0123145462179362\\
455	0.0123104143066162\\
456	0.0123061785491335\\
457	0.0123018330595863\\
458	0.012297371306301\\
459	0.012292788197805\\
460	0.0122880781215863\\
461	0.0122832347049287\\
462	0.0122782499857507\\
463	0.0122731128283286\\
464	0.0122678176838001\\
465	0.0122623723303896\\
466	0.0122567932091308\\
467	0.0122510652800892\\
468	0.0122451583799879\\
469	0.0122390532822709\\
470	0.0122327351178461\\
471	0.0122261911498775\\
472	0.0122193999742811\\
473	0.0122123492021528\\
474	0.0122050282442201\\
475	0.0121974853517383\\
476	0.0121896739902676\\
477	0.012181584337354\\
478	0.0121731272355642\\
479	0.0121642686454854\\
480	0.012154970113043\\
481	0.0121451880492609\\
482	0.0121348728608001\\
483	0.0121239678834108\\
484	0.0121124081923805\\
485	0.0120963155019349\\
486	0.0120649176038134\\
487	0.0120328128768236\\
488	0.0119999514025083\\
489	0.0119662782897598\\
490	0.0119317320930044\\
491	0.0118962437302807\\
492	0.0118597350033185\\
493	0.0118221172405603\\
494	0.0117832889149886\\
495	0.0117431343588629\\
496	0.0117015184283364\\
497	0.0116549116336663\\
498	0.0115860121225372\\
499	0.0115162499669502\\
500	0.0114456167315407\\
501	0.0113741133849686\\
502	0.0113017488788742\\
503	0.0112285384910014\\
504	0.0111545057219939\\
505	0.0110796805654126\\
506	0.0110041039437151\\
507	0.010927831037003\\
508	0.0108509311393083\\
509	0.0107916350492174\\
510	0.0107635466787545\\
511	0.0107357347046062\\
512	0.0107082829141176\\
513	0.010681286929017\\
514	0.010654856098325\\
515	0.0106291155566186\\
516	0.0106042089535487\\
517	0.0105803013466249\\
518	0.0105575825649882\\
519	0.010536271481845\\
520	0.0105166211944132\\
521	0.0104973368161173\\
522	0.0104780134087807\\
523	0.0104586506232094\\
524	0.0104392449365551\\
525	0.0104197887016708\\
526	0.0104002689680724\\
527	0.0103806660200109\\
528	0.0103609515612236\\
529	0.0103410864712358\\
530	0.0103210180366013\\
531	0.0103006765426013\\
532	0.0102799958356522\\
533	0.0102589602236068\\
534	0.0102375527368014\\
535	0.0102157551054318\\
536	0.0101935477733669\\
537	0.010170909964095\\
538	0.0101478198195821\\
539	0.0101242546394033\\
540	0.0101001912560033\\
541	0.0100756065930546\\
542	0.0100504784681263\\
543	0.0100247867189688\\
544	0.00999851978886971\\
545	0.00997166200880703\\
546	0.00994417963842086\\
547	0.00991605082722985\\
548	0.00988725312771732\\
549	0.00985776361556839\\
550	0.00982755393573083\\
551	0.00979658948103769\\
552	0.00976484812125456\\
553	0.00973230776167854\\
554	0.00969894570077598\\
555	0.00966473846632073\\
556	0.00962966177862186\\
557	0.00959369033328697\\
558	0.00955679163728458\\
559	0.00951894667692255\\
560	0.00948012721252465\\
561	0.00944030388348263\\
562	0.00939942773091895\\
563	0.00935749447232193\\
564	0.00931447279380433\\
565	0.00927033353760196\\
566	0.0092250271380538\\
567	0.0091785227373342\\
568	0.00913074503562235\\
569	0.00908172329564748\\
570	0.00903146923488505\\
571	0.00897992032434836\\
572	0.00892692134903677\\
573	0.00887258532325634\\
574	0.00881679907641235\\
575	0.00875932341700503\\
576	0.00870064929330184\\
577	0.00864017654027085\\
578	0.00852481707220463\\
579	0.00835776302369814\\
580	0.00811579247804831\\
581	0.00774959098955616\\
582	0.00736199746753604\\
583	0.00723369909236408\\
584	0.00711888015058533\\
585	0.00702285427221594\\
586	0.00692775913869259\\
587	0.00683310670274828\\
588	0.00673664854901047\\
589	0.00663838358992553\\
590	0.00653749881589685\\
591	0.00643344329816635\\
592	0.00632466858477865\\
593	0.00620725772627071\\
594	0.0060710901180166\\
595	0.00588956027306329\\
596	0.00559184182825266\\
597	0.00498881296807123\\
598	0.00357511483354343\\
599	0\\
600	0\\
};
\addplot [color=blue!40!mycolor9,solid,forget plot]
  table[row sep=crcr]{%
1	0.0128221241513979\\
2	0.0128221186474875\\
3	0.0128221130499332\\
4	0.0128221073571268\\
5	0.0128221015674324\\
6	0.0128220956791858\\
7	0.0128220896906941\\
8	0.0128220836002352\\
9	0.0128220774060573\\
10	0.0128220711063784\\
11	0.0128220646993857\\
12	0.0128220581832353\\
13	0.0128220515560513\\
14	0.0128220448159256\\
15	0.012822037960917\\
16	0.0128220309890511\\
17	0.0128220238983193\\
18	0.0128220166866783\\
19	0.0128220093520498\\
20	0.0128220018923194\\
21	0.0128219943053364\\
22	0.0128219865889129\\
23	0.0128219787408236\\
24	0.0128219707588045\\
25	0.0128219626405527\\
26	0.0128219543837257\\
27	0.0128219459859404\\
28	0.0128219374447729\\
29	0.0128219287577573\\
30	0.0128219199223853\\
31	0.0128219109361054\\
32	0.0128219017963222\\
33	0.0128218925003954\\
34	0.0128218830456394\\
35	0.0128218734293221\\
36	0.0128218636486645\\
37	0.0128218537008397\\
38	0.0128218435829722\\
39	0.0128218332921366\\
40	0.0128218228253576\\
41	0.0128218121796083\\
42	0.0128218013518097\\
43	0.01282179033883\\
44	0.0128217791374831\\
45	0.0128217677445283\\
46	0.012821756156669\\
47	0.0128217443705517\\
48	0.0128217323827655\\
49	0.0128217201898403\\
50	0.0128217077882465\\
51	0.0128216951743938\\
52	0.0128216823446297\\
53	0.0128216692952393\\
54	0.0128216560224432\\
55	0.0128216425223974\\
56	0.0128216287911915\\
57	0.0128216148248475\\
58	0.0128216006193194\\
59	0.0128215861704912\\
60	0.0128215714741761\\
61	0.0128215565261152\\
62	0.0128215413219763\\
63	0.0128215258573526\\
64	0.0128215101277613\\
65	0.0128214941286427\\
66	0.0128214778553582\\
67	0.0128214613031896\\
68	0.0128214444673375\\
69	0.0128214273429196\\
70	0.0128214099249698\\
71	0.0128213922084363\\
72	0.0128213741881804\\
73	0.012821355858975\\
74	0.0128213372155029\\
75	0.0128213182523553\\
76	0.0128212989640304\\
77	0.0128212793449317\\
78	0.012821259389366\\
79	0.0128212390915426\\
80	0.0128212184455707\\
81	0.0128211974454583\\
82	0.0128211760851101\\
83	0.0128211543583258\\
84	0.0128211322587987\\
85	0.0128211097801131\\
86	0.0128210869157432\\
87	0.0128210636590505\\
88	0.0128210400032826\\
89	0.0128210159415705\\
90	0.0128209914669273\\
91	0.0128209665722455\\
92	0.0128209412502954\\
93	0.0128209154937231\\
94	0.0128208892950477\\
95	0.0128208626466598\\
96	0.0128208355408189\\
97	0.0128208079696515\\
98	0.0128207799251484\\
99	0.0128207513991627\\
100	0.0128207223834072\\
101	0.0128206928694522\\
102	0.012820662848723\\
103	0.0128206323124972\\
104	0.0128206012519027\\
105	0.0128205696579143\\
106	0.0128205375213521\\
107	0.0128205048328779\\
108	0.0128204715829931\\
109	0.0128204377620357\\
110	0.0128204033601777\\
111	0.0128203683674219\\
112	0.0128203327735994\\
113	0.0128202965683666\\
114	0.0128202597412018\\
115	0.0128202222814028\\
116	0.0128201841780833\\
117	0.0128201454201699\\
118	0.0128201059963993\\
119	0.0128200658953142\\
120	0.0128200251052609\\
121	0.0128199836143853\\
122	0.0128199414106298\\
123	0.0128198984817297\\
124	0.0128198548152098\\
125	0.0128198103983806\\
126	0.0128197652183347\\
127	0.0128197192619432\\
128	0.0128196725158518\\
129	0.012819624966477\\
130	0.012819576600002\\
131	0.0128195274023729\\
132	0.0128194773592948\\
133	0.0128194264562272\\
134	0.0128193746783801\\
135	0.0128193220107098\\
136	0.0128192684379143\\
137	0.0128192139444292\\
138	0.012819158514423\\
139	0.0128191021317929\\
140	0.0128190447801604\\
141	0.0128189864428677\\
142	0.0128189271029748\\
143	0.012818866743259\\
144	0.012818805346216\\
145	0.0128187428940601\\
146	0.0128186793686993\\
147	0.0128186147516327\\
148	0.0128185490238399\\
149	0.0128184821662585\\
150	0.0128184141595004\\
151	0.0128183449838462\\
152	0.0128182746192392\\
153	0.0128182030452802\\
154	0.0128181302412215\\
155	0.0128180561859607\\
156	0.0128179808580352\\
157	0.0128179042356155\\
158	0.0128178262964995\\
159	0.0128177470181059\\
160	0.0128176663774676\\
161	0.0128175843512257\\
162	0.0128175009156222\\
163	0.0128174160464938\\
164	0.0128173297192645\\
165	0.0128172419089391\\
166	0.0128171525900958\\
167	0.012817061736879\\
168	0.012816969322992\\
169	0.0128168753216895\\
170	0.0128167797057699\\
171	0.012816682447568\\
172	0.0128165835189464\\
173	0.0128164828912883\\
174	0.0128163805354888\\
175	0.012816276421947\\
176	0.0128161705205573\\
177	0.0128160628007011\\
178	0.0128159532312381\\
179	0.0128158417804973\\
180	0.0128157284162684\\
181	0.0128156131057922\\
182	0.0128154958157517\\
183	0.0128153765122623\\
184	0.0128152551608629\\
185	0.0128151317265053\\
186	0.012815006173545\\
187	0.0128148784657306\\
188	0.012814748566194\\
189	0.0128146164374397\\
190	0.0128144820413344\\
191	0.0128143453390961\\
192	0.0128142062912833\\
193	0.0128140648577838\\
194	0.0128139209978035\\
195	0.0128137746698547\\
196	0.0128136258317447\\
197	0.0128134744405639\\
198	0.0128133204526734\\
199	0.0128131638236932\\
200	0.0128130045084895\\
201	0.012812842461162\\
202	0.0128126776350311\\
203	0.012812509982625\\
204	0.0128123394556659\\
205	0.0128121660050569\\
206	0.0128119895808684\\
207	0.0128118101323236\\
208	0.0128116276077846\\
209	0.012811441954738\\
210	0.0128112531197804\\
211	0.0128110610486031\\
212	0.0128108656859772\\
213	0.0128106669757382\\
214	0.0128104648607704\\
215	0.0128102592829907\\
216	0.0128100501833332\\
217	0.0128098375017318\\
218	0.0128096211771042\\
219	0.0128094011473348\\
220	0.0128091773492575\\
221	0.012808949718638\\
222	0.0128087181901563\\
223	0.0128084826973885\\
224	0.0128082431727885\\
225	0.0128079995476693\\
226	0.0128077517521841\\
227	0.0128074997153071\\
228	0.0128072433648138\\
229	0.0128069826272612\\
230	0.0128067174279681\\
231	0.0128064476909936\\
232	0.0128061733391172\\
233	0.0128058942938175\\
234	0.0128056104752502\\
235	0.012805321802227\\
236	0.012805028192193\\
237	0.0128047295612044\\
238	0.0128044258239057\\
239	0.0128041168935065\\
240	0.0128038026817578\\
241	0.0128034830989285\\
242	0.0128031580537809\\
243	0.0128028274535461\\
244	0.0128024912038992\\
245	0.0128021492089338\\
246	0.0128018013711366\\
247	0.0128014475913609\\
248	0.0128010877688008\\
249	0.012800721800964\\
250	0.0128003495836446\\
251	0.012799971010896\\
252	0.0127995859750026\\
253	0.0127991943664522\\
254	0.0127987960739067\\
255	0.0127983909841739\\
256	0.0127979789821779\\
257	0.0127975599509297\\
258	0.0127971337714976\\
259	0.0127967003229768\\
260	0.0127962594824592\\
261	0.012795811125003\\
262	0.0127953551236015\\
263	0.0127948913491526\\
264	0.0127944196704277\\
265	0.0127939399540401\\
266	0.0127934520644142\\
267	0.012792955863754\\
268	0.0127924512120117\\
269	0.0127919379668566\\
270	0.0127914159836421\\
271	0.0127908851153716\\
272	0.0127903452126572\\
273	0.012789796123658\\
274	0.012789237693966\\
275	0.0127886697663585\\
276	0.012788092180285\\
277	0.012787504771209\\
278	0.0127869073725014\\
279	0.0127862998293162\\
280	0.0127856819722864\\
281	0.012785053628519\\
282	0.0127844146223136\\
283	0.0127837647751204\\
284	0.0127831039054983\\
285	0.0127824318290726\\
286	0.0127817483584917\\
287	0.0127810533033836\\
288	0.0127803464703117\\
289	0.0127796276627299\\
290	0.0127788966809378\\
291	0.0127781533220338\\
292	0.012777397379869\\
293	0.0127766286449993\\
294	0.0127758469046371\\
295	0.0127750519426019\\
296	0.0127742435392698\\
297	0.012773421471522\\
298	0.0127725855126917\\
299	0.01277173543251\\
300	0.0127708709970495\\
301	0.0127699919686672\\
302	0.0127690981059437\\
303	0.0127681891636222\\
304	0.012767264892544\\
305	0.0127663250395824\\
306	0.0127653693475776\\
307	0.0127643975552741\\
308	0.0127634093972768\\
309	0.0127624046040533\\
310	0.012761382902063\\
311	0.0127603440141851\\
312	0.0127592876607808\\
313	0.0127582135617308\\
314	0.0127571214380861\\
315	0.0127560110017227\\
316	0.0127548819023352\\
317	0.0127537338167616\\
318	0.0127525664375331\\
319	0.0127513794527453\\
320	0.0127501725460195\\
321	0.0127489453964651\\
322	0.0127476976786425\\
323	0.0127464290625272\\
324	0.0127451392134742\\
325	0.0127438277921845\\
326	0.0127424944546712\\
327	0.0127411388522275\\
328	0.0127397606313953\\
329	0.0127383594339348\\
330	0.0127369348967955\\
331	0.0127354866520871\\
332	0.0127340143270527\\
333	0.0127325175440416\\
334	0.0127309959204837\\
335	0.012729449068864\\
336	0.0127278765966982\\
337	0.0127262781065082\\
338	0.0127246531957985\\
339	0.0127230014570321\\
340	0.0127213224776067\\
341	0.0127196158398309\\
342	0.0127178811208987\\
343	0.012716117892865\\
344	0.0127143257226191\\
345	0.012712504171857\\
346	0.0127106527970532\\
347	0.0127087711494308\\
348	0.0127068587749304\\
349	0.012704915214177\\
350	0.0127029400024452\\
351	0.0127009326696169\\
352	0.0126988927401189\\
353	0.0126968197327897\\
354	0.0126947131605148\\
355	0.0126925725290967\\
356	0.0126903973335947\\
357	0.0126881870464451\\
358	0.0126859410815036\\
359	0.0126836587326256\\
360	0.0126813395091237\\
361	0.0126789834048322\\
362	0.0126765898920321\\
363	0.0126741584365697\\
364	0.0126716884997981\\
365	0.0126691795434975\\
366	0.0126666310385116\\
367	0.0126640424590216\\
368	0.0126614131256183\\
369	0.0126587424619835\\
370	0.0126560298704718\\
371	0.0126532747474923\\
372	0.0126504764835539\\
373	0.0126476344635425\\
374	0.0126447480678077\\
375	0.0126418166757314\\
376	0.0126388396765391\\
377	0.0126358165005232\\
378	0.0126327467048574\\
379	0.0126296301807253\\
380	0.0126264671049288\\
381	0.0126232549673068\\
382	0.0126199919893131\\
383	0.0126166774047729\\
384	0.0126133104398394\\
385	0.012609890313639\\
386	0.0126064162390709\\
387	0.0126028874237943\\
388	0.0125993030714365\\
389	0.0125956623830687\\
390	0.0125919645589999\\
391	0.0125882088009529\\
392	0.0125843943146948\\
393	0.0125805203132207\\
394	0.0125765860205964\\
395	0.0125725906765956\\
396	0.0125685335422928\\
397	0.012564413906809\\
398	0.0125602310954449\\
399	0.0125559844794782\\
400	0.0125516734879364\\
401	0.0125472976216594\\
402	0.0125428564697679\\
403	0.0125383497277677\\
404	0.0125337772129918\\
405	0.0125291388585198\\
406	0.0125244346009591\\
407	0.0125196637162193\\
408	0.0125148251632131\\
409	0.0125099245732156\\
410	0.0125049637052244\\
411	0.0124999449000599\\
412	0.0124948714160985\\
413	0.0124897487860816\\
414	0.0124845907621011\\
415	0.0124813395821847\\
416	0.0124791097948332\\
417	0.0124768435857581\\
418	0.0124745405489178\\
419	0.0124722002892683\\
420	0.012469822439971\\
421	0.0124674067170455\\
422	0.0124649530934125\\
423	0.0124624623825015\\
424	0.0124599384475586\\
425	0.0124573787205636\\
426	0.0124547743978464\\
427	0.0124521247917136\\
428	0.0124494292085696\\
429	0.0124466869492808\\
430	0.0124438973079154\\
431	0.0124410595699248\\
432	0.0124381730132808\\
433	0.0124352369125776\\
434	0.0124322505378799\\
435	0.0124292131544308\\
436	0.0124261240230152\\
437	0.0124229824003044\\
438	0.0124197875391983\\
439	0.0124165386894555\\
440	0.0124132350998524\\
441	0.0124098760247526\\
442	0.0124064607339108\\
443	0.0124029884936748\\
444	0.0123994585404522\\
445	0.0123958701013909\\
446	0.0123922224374629\\
447	0.0123885148041532\\
448	0.012384746465052\\
449	0.0123809167314593\\
450	0.0123770248782612\\
451	0.0123730701783668\\
452	0.0123690519602327\\
453	0.0123649697201982\\
454	0.0123608230758601\\
455	0.0123566115534947\\
456	0.0123523347708978\\
457	0.012347992523575\\
458	0.0123435836487435\\
459	0.0123391080984532\\
460	0.0123345663514086\\
461	0.0123299590957613\\
462	0.0123252873618982\\
463	0.0123205533906122\\
464	0.0123157610411223\\
465	0.012310914948296\\
466	0.0123060179926898\\
467	0.0123010737395066\\
468	0.0122960890425682\\
469	0.0122910761707491\\
470	0.0122860532579802\\
471	0.0122808911831758\\
472	0.0122755799847244\\
473	0.012270114742827\\
474	0.0122644967037555\\
475	0.0122587475957867\\
476	0.0122528499138829\\
477	0.0122467985890392\\
478	0.0122405583835642\\
479	0.0122341196501383\\
480	0.0122274713422001\\
481	0.012220597866337\\
482	0.0122134732558553\\
483	0.0122060936535532\\
484	0.012198460194408\\
485	0.0121906110028726\\
486	0.0121825203336693\\
487	0.0121741841374296\\
488	0.0121655485355651\\
489	0.0121565490614545\\
490	0.012147152934935\\
491	0.0121373231973741\\
492	0.0121270180414425\\
493	0.0121161900061922\\
494	0.0121047849719389\\
495	0.0120927410471809\\
496	0.0120799872030659\\
497	0.0120636331394824\\
498	0.012029218695471\\
499	0.0119940190150183\\
500	0.0119579738482337\\
501	0.0119210183688601\\
502	0.0118830809599045\\
503	0.0118440805517895\\
504	0.0118039251460552\\
505	0.011762511791863\\
506	0.0117197229902744\\
507	0.0116754242022384\\
508	0.0116294590585493\\
509	0.0115721076825285\\
510	0.0114966338628686\\
511	0.0114201756042756\\
512	0.0113427275167932\\
513	0.0112642867859971\\
514	0.0111848523399804\\
515	0.0111044314852391\\
516	0.0110230325129274\\
517	0.0109406771844838\\
518	0.0108574055503264\\
519	0.0107732722674924\\
520	0.0106883499578858\\
521	0.0106441427838155\\
522	0.010612107277377\\
523	0.0105803246125866\\
524	0.0105488907673634\\
525	0.0105179159298167\\
526	0.0104875267373642\\
527	0.010457868989134\\
528	0.0104291110278388\\
529	0.0104014476627862\\
530	0.0103751049387132\\
531	0.0103503458851833\\
532	0.0103268742489999\\
533	0.0103032680621459\\
534	0.0102795267589314\\
535	0.0102556468073683\\
536	0.0102316207352366\\
537	0.0102074359058474\\
538	0.0101830729830778\\
539	0.0101585040114805\\
540	0.0101336900207503\\
541	0.010108578040636\\
542	0.0100830973907246\\
543	0.0100571550882699\\
544	0.0100307088934362\\
545	0.0100037458230299\\
546	0.00997625334123156\\
547	0.00994818649442175\\
548	0.00991951861369859\\
549	0.00989022144399231\\
550	0.00986026544414329\\
551	0.00982961512878818\\
552	0.00979823029458212\\
553	0.00976607959939507\\
554	0.00973313584000431\\
555	0.00969937447888449\\
556	0.0096647699938608\\
557	0.00962929585758961\\
558	0.00959292452181416\\
559	0.00955562739215151\\
560	0.00951737483940085\\
561	0.00947813619386819\\
562	0.0094378797362552\\
563	0.00939656988361341\\
564	0.00935417160700795\\
565	0.00931065457927453\\
566	0.00926597358132882\\
567	0.00922009446115416\\
568	0.00917299651438998\\
569	0.00912463165109635\\
570	0.0090749525458088\\
571	0.00902388281798735\\
572	0.00897134716389313\\
573	0.00891744112791506\\
574	0.00886209983796009\\
575	0.00880519743323962\\
576	0.00874672803872491\\
577	0.00868656413408832\\
578	0.00862465950032357\\
579	0.00856132452076371\\
580	0.00845850078546796\\
581	0.00829534704719994\\
582	0.00811899472346424\\
583	0.00776287265425107\\
584	0.00738782812585071\\
585	0.00710250401533227\\
586	0.00696803363265066\\
587	0.00684919446956261\\
588	0.00674593216677205\\
589	0.0066422892676759\\
590	0.00653990407331532\\
591	0.00643439021582487\\
592	0.00632485151876375\\
593	0.00620725772627071\\
594	0.0060710901180166\\
595	0.00588956027306329\\
596	0.00559184182825266\\
597	0.00498881296807123\\
598	0.00357511483354343\\
599	0\\
600	0\\
};
\addplot [color=blue!75!mycolor7,solid,forget plot]
  table[row sep=crcr]{%
1	0.0130225150736602\\
2	0.0130225120193724\\
3	0.0130225089129799\\
4	0.0130225057535874\\
5	0.0130225025402839\\
6	0.013022499272143\\
7	0.0130224959482223\\
8	0.0130224925675629\\
9	0.0130224891291897\\
10	0.0130224856321107\\
11	0.0130224820753167\\
12	0.0130224784577813\\
13	0.0130224747784603\\
14	0.0130224710362915\\
15	0.0130224672301946\\
16	0.0130224633590705\\
17	0.0130224594218011\\
18	0.0130224554172494\\
19	0.0130224513442585\\
20	0.0130224472016516\\
21	0.0130224429882319\\
22	0.0130224387027817\\
23	0.0130224343440625\\
24	0.0130224299108144\\
25	0.0130224254017559\\
26	0.0130224208155833\\
27	0.0130224161509707\\
28	0.0130224114065691\\
29	0.0130224065810064\\
30	0.013022401672887\\
31	0.013022396680791\\
32	0.0130223916032744\\
33	0.0130223864388681\\
34	0.0130223811860778\\
35	0.0130223758433836\\
36	0.0130223704092391\\
37	0.0130223648820717\\
38	0.0130223592602815\\
39	0.0130223535422411\\
40	0.0130223477262952\\
41	0.0130223418107598\\
42	0.0130223357939222\\
43	0.0130223296740401\\
44	0.0130223234493411\\
45	0.0130223171180226\\
46	0.0130223106782507\\
47	0.01302230412816\\
48	0.0130222974658533\\
49	0.0130222906894003\\
50	0.0130222837968378\\
51	0.0130222767861688\\
52	0.0130222696553618\\
53	0.0130222624023506\\
54	0.0130222550250332\\
55	0.0130222475212715\\
56	0.0130222398888909\\
57	0.0130222321256789\\
58	0.0130222242293854\\
59	0.0130222161977214\\
60	0.0130222080283584\\
61	0.0130221997189281\\
62	0.0130221912670214\\
63	0.0130221826701876\\
64	0.0130221739259341\\
65	0.0130221650317253\\
66	0.013022155984982\\
67	0.0130221467830806\\
68	0.0130221374233525\\
69	0.0130221279030832\\
70	0.0130221182195113\\
71	0.0130221083698283\\
72	0.0130220983511771\\
73	0.0130220881606517\\
74	0.0130220777952959\\
75	0.0130220672521029\\
76	0.013022056528014\\
77	0.0130220456199182\\
78	0.0130220345246508\\
79	0.0130220232389928\\
80	0.0130220117596698\\
81	0.0130220000833512\\
82	0.0130219882066492\\
83	0.0130219761261178\\
84	0.0130219638382516\\
85	0.0130219513394851\\
86	0.0130219386261916\\
87	0.0130219256946821\\
88	0.0130219125412042\\
89	0.0130218991619409\\
90	0.01302188555301\\
91	0.0130218717104624\\
92	0.0130218576302811\\
93	0.0130218433083805\\
94	0.0130218287406045\\
95	0.0130218139227259\\
96	0.0130217988504449\\
97	0.0130217835193878\\
98	0.0130217679251062\\
99	0.0130217520630749\\
100	0.0130217359286916\\
101	0.0130217195172747\\
102	0.0130217028240625\\
103	0.0130216858442116\\
104	0.0130216685727958\\
105	0.0130216510048041\\
106	0.01302163313514\\
107	0.0130216149586195\\
108	0.0130215964699698\\
109	0.0130215776638279\\
110	0.0130215585347389\\
111	0.0130215390771545\\
112	0.0130215192854316\\
113	0.0130214991538303\\
114	0.0130214786765127\\
115	0.0130214578475409\\
116	0.0130214366608754\\
117	0.0130214151103737\\
118	0.013021393189788\\
119	0.0130213708927639\\
120	0.0130213482128382\\
121	0.0130213251434375\\
122	0.0130213016778759\\
123	0.0130212778093534\\
124	0.0130212535309538\\
125	0.013021228835643\\
126	0.0130212037162665\\
127	0.0130211781655479\\
128	0.0130211521760866\\
129	0.0130211257403557\\
130	0.0130210988506997\\
131	0.0130210714993327\\
132	0.0130210436783359\\
133	0.0130210153796555\\
134	0.0130209865951002\\
135	0.0130209573163393\\
136	0.0130209275348998\\
137	0.0130208972421644\\
138	0.0130208664293689\\
139	0.0130208350875998\\
140	0.0130208032077922\\
141	0.013020770780727\\
142	0.0130207377970292\\
143	0.0130207042471656\\
144	0.0130206701214413\\
145	0.0130206354099936\\
146	0.0130206001027788\\
147	0.0130205641895691\\
148	0.0130205276599894\\
149	0.0130204905034875\\
150	0.0130204527093318\\
151	0.0130204142666074\\
152	0.0130203751642137\\
153	0.0130203353908611\\
154	0.0130202949350675\\
155	0.0130202537851553\\
156	0.0130202119292483\\
157	0.0130201693552677\\
158	0.0130201260509294\\
159	0.0130200820037399\\
160	0.0130200372009933\\
161	0.0130199916297673\\
162	0.0130199452769198\\
163	0.0130198981290851\\
164	0.0130198501726701\\
165	0.0130198013938504\\
166	0.0130197517785665\\
167	0.0130197013125199\\
168	0.0130196499811689\\
169	0.0130195977697245\\
170	0.0130195446631464\\
171	0.0130194906461387\\
172	0.0130194357031453\\
173	0.0130193798183459\\
174	0.0130193229756516\\
175	0.0130192651586999\\
176	0.0130192063508504\\
177	0.0130191465351802\\
178	0.0130190856944789\\
179	0.0130190238112441\\
180	0.0130189608676761\\
181	0.013018896845673\\
182	0.0130188317268259\\
183	0.0130187654924134\\
184	0.0130186981233965\\
185	0.0130186296004131\\
186	0.013018559903773\\
187	0.0130184890134518\\
188	0.0130184169090858\\
189	0.0130183435699658\\
190	0.0130182689750319\\
191	0.0130181931028671\\
192	0.0130181159316915\\
193	0.0130180374393562\\
194	0.0130179576033374\\
195	0.0130178764007295\\
196	0.0130177938082393\\
197	0.0130177098021791\\
198	0.0130176243584603\\
199	0.0130175374525867\\
200	0.0130174490596477\\
201	0.0130173591543111\\
202	0.0130172677108163\\
203	0.0130171747029673\\
204	0.013017080104125\\
205	0.0130169838872\\
206	0.0130168860246453\\
207	0.0130167864884482\\
208	0.0130166852501232\\
209	0.0130165822807033\\
210	0.0130164775507325\\
211	0.0130163710302578\\
212	0.0130162626888204\\
213	0.0130161524954476\\
214	0.0130160404186443\\
215	0.0130159264263842\\
216	0.013015810486101\\
217	0.0130156925646794\\
218	0.0130155726284459\\
219	0.0130154506431599\\
220	0.013015326574004\\
221	0.0130152003855742\\
222	0.0130150720418708\\
223	0.0130149415062881\\
224	0.0130148087416045\\
225	0.0130146737099722\\
226	0.0130145363729071\\
227	0.013014396691278\\
228	0.0130142546252963\\
229	0.0130141101345046\\
230	0.0130139631777664\\
231	0.0130138137132542\\
232	0.0130136616984386\\
233	0.0130135070900766\\
234	0.0130133498442\\
235	0.0130131899161033\\
236	0.0130130272603318\\
237	0.0130128618306691\\
238	0.0130126935801252\\
239	0.0130125224609229\\
240	0.013012348424486\\
241	0.0130121714214256\\
242	0.0130119914015271\\
243	0.0130118083137367\\
244	0.0130116221061478\\
245	0.0130114327259875\\
246	0.013011240119602\\
247	0.0130110442324428\\
248	0.0130108450090522\\
249	0.0130106423930486\\
250	0.0130104363271119\\
251	0.013010226752968\\
252	0.0130100136113741\\
253	0.0130097968421031\\
254	0.0130095763839278\\
255	0.0130093521746051\\
256	0.0130091241508601\\
257	0.0130088922483701\\
258	0.0130086564017474\\
259	0.0130084165445237\\
260	0.0130081726091322\\
261	0.0130079245268914\\
262	0.0130076722279879\\
263	0.0130074156414583\\
264	0.0130071546951722\\
265	0.0130068893158146\\
266	0.0130066194288675\\
267	0.0130063449585917\\
268	0.0130060658280088\\
269	0.0130057819588823\\
270	0.0130054932716976\\
271	0.0130051996856409\\
272	0.013004901118574\\
273	0.0130045974870004\\
274	0.0130042887060179\\
275	0.0130039746892596\\
276	0.0130036553488967\\
277	0.013003330595985\\
278	0.0130030003412403\\
279	0.0130026644927914\\
280	0.013002322957221\\
281	0.0130019756396162\\
282	0.0130016224435491\\
283	0.0130012632710565\\
284	0.0130008980226195\\
285	0.0130005265971432\\
286	0.0130001488919369\\
287	0.0129997648026934\\
288	0.0129993742234691\\
289	0.0129989770466638\\
290	0.0129985731630008\\
291	0.0129981624615067\\
292	0.0129977448294915\\
293	0.0129973201525295\\
294	0.0129968883144391\\
295	0.0129964491972639\\
296	0.0129960026812535\\
297	0.0129955486448445\\
298	0.0129950869646423\\
299	0.0129946175154027\\
300	0.0129941401700143\\
301	0.0129936547994806\\
302	0.0129931612729041\\
303	0.0129926594574693\\
304	0.0129921492184279\\
305	0.012991630419085\\
306	0.0129911029207873\\
307	0.0129905665829175\\
308	0.0129900212628987\\
309	0.0129894668162209\\
310	0.01298890309651\\
311	0.0129883299556555\\
312	0.0129877472439461\\
313	0.0129871548098513\\
314	0.0129865524983448\\
315	0.0129859401478064\\
316	0.0129853175979958\\
317	0.012984684688532\\
318	0.0129840412570151\\
319	0.0129833871390295\\
320	0.0129827221681496\\
321	0.0129820461759471\\
322	0.0129813589920006\\
323	0.0129806604439079\\
324	0.0129799503573005\\
325	0.0129792285558602\\
326	0.0129784948613397\\
327	0.0129777490935848\\
328	0.0129769910705603\\
329	0.0129762206083789\\
330	0.0129754375213335\\
331	0.0129746416219327\\
332	0.0129738327209403\\
333	0.0129730106274183\\
334	0.0129721751487738\\
335	0.0129713260908104\\
336	0.012970463257783\\
337	0.0129695864524581\\
338	0.0129686954761775\\
339	0.0129677901289272\\
340	0.0129668702094111\\
341	0.0129659355151289\\
342	0.0129649858424594\\
343	0.012964020986748\\
344	0.0129630407423988\\
345	0.0129620449029715\\
346	0.0129610332612806\\
347	0.0129600056094995\\
348	0.0129589617392641\\
349	0.0129579014417778\\
350	0.0129568245079108\\
351	0.0129557307282876\\
352	0.0129546198933432\\
353	0.0129534917933025\\
354	0.0129523462179658\\
355	0.0129511829560209\\
356	0.012950001793339\\
357	0.0129488025100273\\
358	0.0129475848818788\\
359	0.0129463487117659\\
360	0.0129450938341597\\
361	0.0129438200333205\\
362	0.0129425270913271\\
363	0.0129412147885994\\
364	0.0129398829076805\\
365	0.0129385312479636\\
366	0.0129371596757962\\
367	0.0129357682822976\\
368	0.0129343577230402\\
369	0.012932924364349\\
370	0.0129314675434969\\
371	0.0129299868273339\\
372	0.0129284817700042\\
373	0.0129269519124138\\
374	0.0129253967818844\\
375	0.012923815892023\\
376	0.0129222087416671\\
377	0.0129205748041308\\
378	0.0129189134528469\\
379	0.0129172234793178\\
380	0.0129155048955577\\
381	0.0129137576337257\\
382	0.0129119810979793\\
383	0.0129101746681613\\
384	0.0129083376978115\\
385	0.0129064695119542\\
386	0.0129045694046296\\
387	0.0129026366361339\\
388	0.0129006704299246\\
389	0.0128986699691414\\
390	0.0128966343926832\\
391	0.0128945627907712\\
392	0.0128924541999154\\
393	0.0128903075971844\\
394	0.0128881218936615\\
395	0.0128858959269434\\
396	0.0128836284525145\\
397	0.0128813181337972\\
398	0.0128789635306518\\
399	0.0128765630860954\\
400	0.0128741151110797\\
401	0.0128716177674844\\
402	0.0128690690505257\\
403	0.0128664667750224\\
404	0.0128638085800816\\
405	0.0128610920005398\\
406	0.0128583147862191\\
407	0.0128554763748126\\
408	0.0128525667828647\\
409	0.0128495819374439\\
410	0.0128465168804551\\
411	0.012843365947154\\
412	0.0128401225690261\\
413	0.0128367789591462\\
414	0.0128333239203598\\
415	0.0128286990332797\\
416	0.0128234021262148\\
417	0.0128180166341239\\
418	0.0128125412308796\\
419	0.0128069745855871\\
420	0.0128013153800388\\
421	0.0127955623585586\\
422	0.0127897144625424\\
423	0.0127837711430261\\
424	0.0127777325604093\\
425	0.012771595603294\\
426	0.012765355119604\\
427	0.0127590095802008\\
428	0.0127525574495674\\
429	0.0127459971871596\\
430	0.0127393272489458\\
431	0.0127325460894349\\
432	0.0127256521640962\\
433	0.0127186439316089\\
434	0.0127115198564744\\
435	0.01270427841201\\
436	0.0126969180837037\\
437	0.0126894373729905\\
438	0.0126818348015502\\
439	0.0126741089163148\\
440	0.0126662582953799\\
441	0.0126582815543499\\
442	0.0126501773510782\\
443	0.0126419443916143\\
444	0.0126335814411897\\
445	0.0126250873377048\\
446	0.0126164610081213\\
447	0.0126077015314601\\
448	0.0125988069413607\\
449	0.0125897710377402\\
450	0.0125805998022605\\
451	0.0125712953924108\\
452	0.0125618575066632\\
453	0.012552286091783\\
454	0.0125425814447039\\
455	0.0125327445211897\\
456	0.0125227777856401\\
457	0.01251268777782\\
458	0.0125024928731723\\
459	0.0124921574130544\\
460	0.0124816607799896\\
461	0.0124710019331076\\
462	0.0124601798385086\\
463	0.0124491946459253\\
464	0.0124380475091773\\
465	0.0124267402040968\\
466	0.012415275367245\\
467	0.0124036567011664\\
468	0.0123918903601347\\
469	0.0123799886388254\\
470	0.0123681460639805\\
471	0.0123627380329631\\
472	0.0123572185042465\\
473	0.0123515856473842\\
474	0.0123458384320011\\
475	0.0123399726850037\\
476	0.0123339849132016\\
477	0.0123278696733113\\
478	0.0123216232513006\\
479	0.0123152415491759\\
480	0.0123087195710508\\
481	0.0123020499839554\\
482	0.0122952229451359\\
483	0.0122882391449178\\
484	0.0122811035814963\\
485	0.0122738329847773\\
486	0.0122664179415157\\
487	0.0122588585366582\\
488	0.0122511326954372\\
489	0.0122432175139735\\
490	0.0122351059750949\\
491	0.0122267903369046\\
492	0.0122182598862883\\
493	0.0122094885089349\\
494	0.0122004822387657\\
495	0.0121912061038296\\
496	0.0121817025186965\\
497	0.0121719836708441\\
498	0.0121621011380786\\
499	0.0121519932315128\\
500	0.0121417056217666\\
501	0.012131141935661\\
502	0.0121202500288472\\
503	0.0121090082357356\\
504	0.0120973930403815\\
505	0.0120851934508869\\
506	0.0120723389791654\\
507	0.0120587588538445\\
508	0.0120443720487467\\
509	0.0120211361412554\\
510	0.0119832050072543\\
511	0.0119443784207802\\
512	0.0119045898585676\\
513	0.0118637640360684\\
514	0.0118218148014813\\
515	0.0117786462247138\\
516	0.0117341464586208\\
517	0.0116881902035418\\
518	0.0116406370936603\\
519	0.0115913253925226\\
520	0.0115400677765833\\
521	0.0114649020041444\\
522	0.011381940911945\\
523	0.0112978096631671\\
524	0.0112125015917795\\
525	0.0111260164534872\\
526	0.0110383597403553\\
527	0.0109495423004872\\
528	0.0108595777399665\\
529	0.0107684884570977\\
530	0.0106763072113456\\
531	0.0105830834829332\\
532	0.0105043775785049\\
533	0.0104671142982166\\
534	0.0104299383117761\\
535	0.0103929426892914\\
536	0.0103562355725284\\
537	0.0103199426752268\\
538	0.0102842102569721\\
539	0.0102492086725187\\
540	0.0102151366268864\\
541	0.0101822263814329\\
542	0.0101507501044137\\
543	0.010121027558569\\
544	0.0100915954141609\\
545	0.0100618746939995\\
546	0.0100318607180066\\
547	0.0100015838587952\\
548	0.00997100666511039\\
549	0.00994009336497352\\
550	0.00990881845987775\\
551	0.00987714631783721\\
552	0.0098450239144289\\
553	0.00981237947121874\\
554	0.00977911752878245\\
555	0.00974512451415165\\
556	0.00971037583367489\\
557	0.00967484493140955\\
558	0.00963850313244329\\
559	0.0096013195092261\\
560	0.00956326079011629\\
561	0.00952429134141864\\
562	0.00948437325964169\\
563	0.00944346663225815\\
564	0.00940153003079942\\
565	0.009358521353756\\
566	0.0093143991753416\\
567	0.00926912467092151\\
568	0.00922265706987281\\
569	0.00917495068242558\\
570	0.00912596001895447\\
571	0.00907563019759986\\
572	0.00902392558975723\\
573	0.0089708046766837\\
574	0.00891620968789234\\
575	0.00886003700472624\\
576	0.00880221738574126\\
577	0.00874278903473106\\
578	0.00868170440886518\\
579	0.00861883520774838\\
580	0.00855387888569596\\
581	0.0084870822310664\\
582	0.00841538272261392\\
583	0.00825264072841415\\
584	0.0080871163608675\\
585	0.00784265262501878\\
586	0.00748707723152733\\
587	0.00711097509202256\\
588	0.006842981126967\\
589	0.00669995455437386\\
590	0.00656262541880796\\
591	0.00644916937420928\\
592	0.00633088788857397\\
593	0.00620845766590564\\
594	0.0060710901180166\\
595	0.00588956027306329\\
596	0.00559184182825266\\
597	0.00498881296807123\\
598	0.00357511483354343\\
599	0\\
600	0\\
};
\addplot [color=blue!80!mycolor9,solid,forget plot]
  table[row sep=crcr]{%
1	0.0132347781023203\\
2	0.0132347758295092\\
3	0.0132347735178182\\
4	0.0132347711665798\\
5	0.0132347687751148\\
6	0.0132347663427328\\
7	0.0132347638687312\\
8	0.0132347613523954\\
9	0.0132347587929987\\
10	0.0132347561898018\\
11	0.0132347535420527\\
12	0.0132347508489866\\
13	0.0132347481098255\\
14	0.0132347453237781\\
15	0.0132347424900394\\
16	0.0132347396077909\\
17	0.0132347366761998\\
18	0.013234733694419\\
19	0.013234730661587\\
20	0.0132347275768276\\
21	0.0132347244392494\\
22	0.0132347212479457\\
23	0.0132347180019945\\
24	0.0132347147004578\\
25	0.0132347113423815\\
26	0.0132347079267953\\
27	0.0132347044527122\\
28	0.0132347009191282\\
29	0.0132346973250222\\
30	0.0132346936693556\\
31	0.013234689951072\\
32	0.0132346861690969\\
33	0.0132346823223373\\
34	0.0132346784096817\\
35	0.0132346744299992\\
36	0.01323467038214\\
37	0.0132346662649343\\
38	0.0132346620771923\\
39	0.0132346578177039\\
40	0.0132346534852381\\
41	0.0132346490785432\\
42	0.0132346445963457\\
43	0.0132346400373505\\
44	0.0132346354002403\\
45	0.0132346306836753\\
46	0.0132346258862925\\
47	0.013234621006706\\
48	0.0132346160435058\\
49	0.0132346109952581\\
50	0.0132346058605044\\
51	0.0132346006377612\\
52	0.0132345953255198\\
53	0.0132345899222455\\
54	0.0132345844263776\\
55	0.0132345788363284\\
56	0.0132345731504833\\
57	0.0132345673672001\\
58	0.0132345614848083\\
59	0.0132345555016091\\
60	0.0132345494158745\\
61	0.0132345432258471\\
62	0.0132345369297393\\
63	0.0132345305257332\\
64	0.0132345240119795\\
65	0.0132345173865976\\
66	0.0132345106476747\\
67	0.0132345037932653\\
68	0.0132344968213907\\
69	0.0132344897300383\\
70	0.0132344825171614\\
71	0.0132344751806782\\
72	0.0132344677184712\\
73	0.013234460128387\\
74	0.0132344524082354\\
75	0.0132344445557888\\
76	0.0132344365687815\\
77	0.0132344284449094\\
78	0.0132344201818288\\
79	0.0132344117771562\\
80	0.0132344032284675\\
81	0.0132343945332971\\
82	0.0132343856891375\\
83	0.0132343766934383\\
84	0.0132343675436058\\
85	0.0132343582370021\\
86	0.0132343487709442\\
87	0.0132343391427035\\
88	0.0132343293495048\\
89	0.0132343193885258\\
90	0.0132343092568961\\
91	0.0132342989516962\\
92	0.0132342884699572\\
93	0.0132342778086595\\
94	0.0132342669647321\\
95	0.0132342559350517\\
96	0.0132342447164419\\
97	0.0132342333056724\\
98	0.0132342216994577\\
99	0.0132342098944566\\
100	0.0132341978872709\\
101	0.0132341856744447\\
102	0.0132341732524635\\
103	0.0132341606177527\\
104	0.0132341477666773\\
105	0.0132341346955404\\
106	0.013234121400582\\
107	0.0132341078779787\\
108	0.0132340941238417\\
109	0.0132340801342165\\
110	0.0132340659050812\\
111	0.0132340514323457\\
112	0.0132340367118504\\
113	0.013234021739365\\
114	0.0132340065105876\\
115	0.0132339910211431\\
116	0.0132339752665822\\
117	0.0132339592423801\\
118	0.0132339429439351\\
119	0.0132339263665676\\
120	0.0132339095055186\\
121	0.0132338923559482\\
122	0.0132338749129344\\
123	0.013233857171472\\
124	0.0132338391264705\\
125	0.0132338207727534\\
126	0.0132338021050563\\
127	0.0132337831180255\\
128	0.0132337638062165\\
129	0.0132337441640926\\
130	0.0132337241860232\\
131	0.0132337038662822\\
132	0.0132336831990464\\
133	0.0132336621783939\\
134	0.0132336407983026\\
135	0.013233619052648\\
136	0.0132335969352022\\
137	0.0132335744396313\\
138	0.0132335515594945\\
139	0.0132335282882416\\
140	0.0132335046192117\\
141	0.013233480545631\\
142	0.0132334560606112\\
143	0.0132334311571472\\
144	0.0132334058281146\\
145	0.0132333800662666\\
146	0.0132333538642328\\
147	0.0132333272145206\\
148	0.0132333001095103\\
149	0.0132332725414534\\
150	0.0132332445024699\\
151	0.0132332159845462\\
152	0.0132331869795333\\
153	0.0132331574791437\\
154	0.0132331274749498\\
155	0.0132330969583812\\
156	0.0132330659207223\\
157	0.0132330343531097\\
158	0.0132330022465301\\
159	0.0132329695918174\\
160	0.0132329363796503\\
161	0.0132329026005495\\
162	0.0132328682448754\\
163	0.0132328333028247\\
164	0.0132327977644286\\
165	0.0132327616195489\\
166	0.0132327248578761\\
167	0.0132326874689259\\
168	0.0132326494420364\\
169	0.0132326107663652\\
170	0.0132325714308862\\
171	0.0132325314243865\\
172	0.0132324907354632\\
173	0.0132324493525206\\
174	0.013232407263766\\
175	0.0132323644572074\\
176	0.0132323209206495\\
177	0.0132322766416902\\
178	0.0132322316077175\\
179	0.0132321858059057\\
180	0.0132321392232115\\
181	0.0132320918463709\\
182	0.0132320436618949\\
183	0.0132319946560659\\
184	0.013231944814934\\
185	0.0132318941243125\\
186	0.0132318425697743\\
187	0.0132317901366479\\
188	0.0132317368100128\\
189	0.0132316825746955\\
190	0.0132316274152651\\
191	0.0132315713160292\\
192	0.0132315142610289\\
193	0.0132314562340347\\
194	0.0132313972185415\\
195	0.0132313371977643\\
196	0.0132312761546329\\
197	0.0132312140717877\\
198	0.0132311509315738\\
199	0.0132310867160367\\
200	0.0132310214069171\\
201	0.0132309549856451\\
202	0.0132308874333354\\
203	0.0132308187307815\\
204	0.0132307488584507\\
205	0.0132306777964777\\
206	0.0132306055246597\\
207	0.0132305320224497\\
208	0.0132304572689512\\
209	0.0132303812429122\\
210	0.0132303039227183\\
211	0.0132302252863873\\
212	0.0132301453115623\\
213	0.0132300639755051\\
214	0.0132299812550902\\
215	0.0132298971267972\\
216	0.0132298115667046\\
217	0.0132297245504827\\
218	0.013229636053386\\
219	0.0132295460502464\\
220	0.0132294545154658\\
221	0.0132293614230082\\
222	0.0132292667463924\\
223	0.0132291704586841\\
224	0.0132290725324875\\
225	0.0132289729399379\\
226	0.0132288716526929\\
227	0.0132287686419241\\
228	0.0132286638783086\\
229	0.01322855733202\\
230	0.0132284489727197\\
231	0.013228338769548\\
232	0.0132282266911141\\
233	0.0132281127054875\\
234	0.0132279967801877\\
235	0.0132278788821747\\
236	0.013227758977839\\
237	0.0132276370329913\\
238	0.0132275130128521\\
239	0.013227386882041\\
240	0.013227258604566\\
241	0.0132271281438124\\
242	0.0132269954625315\\
243	0.0132268605228293\\
244	0.0132267232861545\\
245	0.0132265837132867\\
246	0.0132264417643243\\
247	0.0132262973986719\\
248	0.0132261505750277\\
249	0.0132260012513705\\
250	0.0132258493849466\\
251	0.0132256949322561\\
252	0.0132255378490395\\
253	0.0132253780902632\\
254	0.0132252156101057\\
255	0.0132250503619426\\
256	0.0132248822983317\\
257	0.0132247113709979\\
258	0.0132245375308178\\
259	0.0132243607278034\\
260	0.0132241809110861\\
261	0.0132239980289003\\
262	0.0132238120285662\\
263	0.0132236228564729\\
264	0.0132234304580604\\
265	0.0132232347778017\\
266	0.0132230357591846\\
267	0.0132228333446926\\
268	0.0132226274757858\\
269	0.013222418092881\\
270	0.013222205135331\\
271	0.0132219885414033\\
272	0.0132217682482567\\
273	0.0132215441919161\\
274	0.0132213163072492\\
275	0.0132210845279557\\
276	0.0132208487865939\\
277	0.0132206090146007\\
278	0.0132203651420567\\
279	0.0132201170977708\\
280	0.0132198648092627\\
281	0.0132196082027371\\
282	0.0132193472030574\\
283	0.0132190817337192\\
284	0.0132188117168224\\
285	0.0132185370730435\\
286	0.0132182577216067\\
287	0.0132179735802545\\
288	0.0132176845652177\\
289	0.0132173905911842\\
290	0.013217091571268\\
291	0.0132167874169769\\
292	0.0132164780381788\\
293	0.0132161633430689\\
294	0.0132158432381344\\
295	0.0132155176281194\\
296	0.0132151864159887\\
297	0.013214849502891\\
298	0.0132145067881207\\
299	0.0132141581690795\\
300	0.0132138035412366\\
301	0.0132134427980889\\
302	0.0132130758311191\\
303	0.0132127025297543\\
304	0.0132123227813227\\
305	0.0132119364710111\\
306	0.0132115434818211\\
307	0.0132111436945272\\
308	0.0132107369876367\\
309	0.0132103232373546\\
310	0.0132099023175512\\
311	0.0132094740997167\\
312	0.013209038452846\\
313	0.0132085952431701\\
314	0.0132081443339324\\
315	0.0132076855861243\\
316	0.0132072188582528\\
317	0.0132067440061001\\
318	0.0132062608826734\\
319	0.0132057693381547\\
320	0.0132052692198509\\
321	0.0132047603721447\\
322	0.0132042426364462\\
323	0.0132037158511462\\
324	0.0132031798515709\\
325	0.013202634469938\\
326	0.0132020795353164\\
327	0.0132015148735874\\
328	0.0132009403074109\\
329	0.0132003556561944\\
330	0.013199760736068\\
331	0.0131991553598638\\
332	0.0131985393371037\\
333	0.013197912473993\\
334	0.0131972745734243\\
335	0.013196625434991\\
336	0.0131959648550124\\
337	0.0131952926265729\\
338	0.0131946085395756\\
339	0.0131939123808146\\
340	0.0131932039340663\\
341	0.0131924829802043\\
342	0.0131917492973398\\
343	0.0131910026609915\\
344	0.01319024284429\\
345	0.0131894696182193\\
346	0.0131886827519027\\
347	0.0131878820129377\\
348	0.0131870671677877\\
349	0.0131862379822367\\
350	0.0131853942219163\\
351	0.0131845356529115\\
352	0.0131836620424491\\
353	0.0131827731596655\\
354	0.0131818687764362\\
355	0.0131809486682588\\
356	0.0131800126153635\\
357	0.0131790604049756\\
358	0.0131780918361405\\
359	0.013177106720808\\
360	0.0131761048789324\\
361	0.0131750861461626\\
362	0.0131740503778041\\
363	0.0131729974552272\\
364	0.01317192729905\\
365	0.0131708399031935\\
366	0.0131697354388771\\
367	0.0131686146244958\\
368	0.013167480455131\\
369	0.0131665168884495\\
370	0.0131655504787687\\
371	0.0131645672718775\\
372	0.0131635669259919\\
373	0.013162549089619\\
374	0.0131615134009261\\
375	0.0131604594860665\\
376	0.0131593869534426\\
377	0.0131582953737402\\
378	0.0131571842210047\\
379	0.0131560528341121\\
380	0.0131549011737056\\
381	0.0131537290664285\\
382	0.013152536054818\\
383	0.0131513216642898\\
384	0.0131500854019684\\
385	0.0131488267553962\\
386	0.0131475451911057\\
387	0.0131462401530352\\
388	0.0131449110607665\\
389	0.0131435573075603\\
390	0.013142178258158\\
391	0.0131407732463172\\
392	0.0131393415720401\\
393	0.0131378824984483\\
394	0.0131363952482487\\
395	0.0131348789997277\\
396	0.0131333328822022\\
397	0.0131317559708472\\
398	0.0131301472808314\\
399	0.0131285057607295\\
400	0.0131268302853374\\
401	0.0131251196484726\\
402	0.0131233725576429\\
403	0.0131215876360022\\
404	0.0131197634462196\\
405	0.0131178985721574\\
406	0.0131159918130475\\
407	0.0131140420985683\\
408	0.013112044706827\\
409	0.013109996486079\\
410	0.0131078948063959\\
411	0.013105736880694\\
412	0.0131035200203141\\
413	0.0131012374186524\\
414	0.0130988728798661\\
415	0.0130955680561331\\
416	0.0130917248225261\\
417	0.0130878146556782\\
418	0.0130838363593167\\
419	0.0130797887193233\\
420	0.013075670511126\\
421	0.0130714805151593\\
422	0.0130672175342066\\
423	0.0130628803382718\\
424	0.013058467354028\\
425	0.0130539768918502\\
426	0.0130494075817094\\
427	0.0130447580314616\\
428	0.0130400268268571\\
429	0.0130352125316226\\
430	0.0130303136876508\\
431	0.0130253288152923\\
432	0.0130202564137235\\
433	0.0130150949614558\\
434	0.0130098429170114\\
435	0.0130044987197909\\
436	0.0129990607911719\\
437	0.0129935275358878\\
438	0.0129878973437463\\
439	0.0129821685917298\\
440	0.0129763396464148\\
441	0.0129704088664562\\
442	0.012964374604684\\
443	0.0129582352067792\\
444	0.0129519889928034\\
445	0.0129456341636788\\
446	0.0129391683426349\\
447	0.0129325858670673\\
448	0.0129258909366011\\
449	0.0129190806169245\\
450	0.0129121567789233\\
451	0.0129051193176962\\
452	0.0128979670257757\\
453	0.0128906988273655\\
454	0.0128833138909198\\
455	0.012875811934488\\
456	0.0128681941311641\\
457	0.0128604659923326\\
458	0.0128526479478858\\
459	0.0128463432843658\\
460	0.0128402288433069\\
461	0.0128339852225642\\
462	0.0128276133713435\\
463	0.0128210948494779\\
464	0.01281441344157\\
465	0.0128075590747409\\
466	0.0128005205059016\\
467	0.0127932853917727\\
468	0.0127858339892735\\
469	0.0127781449890987\\
470	0.0127701016006216\\
471	0.0127582393463909\\
472	0.0127461733117576\\
473	0.0127338998980457\\
474	0.0127214152470802\\
475	0.0127087155480823\\
476	0.0126957968660497\\
477	0.0126826553162722\\
478	0.0126692866612546\\
479	0.0126556855465089\\
480	0.0126418427421058\\
481	0.0126277562256423\\
482	0.0126134338845041\\
483	0.0125988738737036\\
484	0.0125840749030257\\
485	0.01256903385509\\
486	0.0125537489084793\\
487	0.0125382194244606\\
488	0.0125224523292342\\
489	0.0125064490906939\\
490	0.0124901683099635\\
491	0.0124735867336545\\
492	0.0124566958023023\\
493	0.0124394826754598\\
494	0.0124219466772415\\
495	0.0124040745937468\\
496	0.0123858858916424\\
497	0.0123673886652956\\
498	0.0123486014693713\\
499	0.0123295029307065\\
500	0.0123101108101237\\
501	0.0122903853911431\\
502	0.0122703136627916\\
503	0.0122499026135327\\
504	0.0122291898553464\\
505	0.0122182341084881\\
506	0.0122075537815293\\
507	0.0121965778959416\\
508	0.0121853129971356\\
509	0.0121736477797267\\
510	0.012161602506892\\
511	0.0121491619363014\\
512	0.0121363035640099\\
513	0.0121230173672913\\
514	0.0121093475310175\\
515	0.0120951592706932\\
516	0.0120804790687188\\
517	0.0120652203880827\\
518	0.0120492328688599\\
519	0.0120324415501122\\
520	0.0120147603405112\\
521	0.011977952271915\\
522	0.0119348584053488\\
523	0.0118907851469197\\
524	0.0118456684488627\\
525	0.0117994378583295\\
526	0.0117520144632674\\
527	0.01170330891857\\
528	0.0116532180701453\\
529	0.0116016248096776\\
530	0.0115483953376152\\
531	0.011493093288488\\
532	0.0114273515266461\\
533	0.0113373290609614\\
534	0.0112459131750253\\
535	0.0111530808453629\\
536	0.0110588120087396\\
537	0.0109630906304736\\
538	0.0108659061000613\\
539	0.0107672551111521\\
540	0.0106671440848492\\
541	0.0105655906119601\\
542	0.0104626264988062\\
543	0.0103583043008088\\
544	0.0102980335130526\\
545	0.0102537894756843\\
546	0.01020948323368\\
547	0.0101652120206003\\
548	0.0101210898973623\\
549	0.0100772506471942\\
550	0.0100339538313825\\
551	0.00999133077126591\\
552	0.00994952019604942\\
553	0.00990879691879962\\
554	0.0098694885703536\\
555	0.00983193644346207\\
556	0.00979382750513552\\
557	0.00975515995707443\\
558	0.00971592989186324\\
559	0.0096761302407511\\
560	0.00963574938160194\\
561	0.00959476919138996\\
562	0.00955316263743115\\
563	0.00951089063585328\\
564	0.00946789796833104\\
565	0.00942410799241927\\
566	0.00937941580458399\\
567	0.0093336795217075\\
568	0.00928685833691943\\
569	0.00923891336036805\\
570	0.00918980277362079\\
571	0.00913948161419208\\
572	0.00908790133297858\\
573	0.00903500913785611\\
574	0.00898074783306652\\
575	0.00892505467335488\\
576	0.0088678514388995\\
577	0.00880907800309159\\
578	0.00874867907362212\\
579	0.00868655222200205\\
580	0.00862260334329572\\
581	0.00855684893553334\\
582	0.00848916944141833\\
583	0.00841938629888952\\
584	0.008347256035793\\
585	0.00823589926147293\\
586	0.0080735141024222\\
587	0.00790696828781916\\
588	0.00764918419075849\\
589	0.0072961566882175\\
590	0.0069260267358603\\
591	0.00657947678076211\\
592	0.00642061328358282\\
593	0.00624640749833555\\
594	0.00607883230598045\\
595	0.00588956027306329\\
596	0.00559184182825266\\
597	0.00498881296807123\\
598	0.00357511483354343\\
599	0\\
600	0\\
};
\addplot [color=blue,solid,forget plot]
  table[row sep=crcr]{%
1	0.013414928077931\\
2	0.0134149270175419\\
3	0.0134149259390028\\
4	0.0134149248420022\\
5	0.0134149237262232\\
6	0.0134149225913437\\
7	0.0134149214370357\\
8	0.013414920262966\\
9	0.0134149190687953\\
10	0.0134149178541788\\
11	0.0134149166187655\\
12	0.0134149153621987\\
13	0.0134149140841154\\
14	0.0134149127841463\\
15	0.0134149114619161\\
16	0.0134149101170427\\
17	0.0134149087491376\\
18	0.0134149073578057\\
19	0.0134149059426452\\
20	0.0134149045032472\\
21	0.013414903039196\\
22	0.0134149015500686\\
23	0.0134149000354349\\
24	0.0134148984948575\\
25	0.0134148969278912\\
26	0.0134148953340836\\
27	0.0134148937129743\\
28	0.013414892064095\\
29	0.0134148903869696\\
30	0.0134148886811136\\
31	0.0134148869460343\\
32	0.0134148851812307\\
33	0.0134148833861931\\
34	0.0134148815604031\\
35	0.0134148797033334\\
36	0.0134148778144479\\
37	0.013414875893201\\
38	0.013414873939038\\
39	0.0134148719513947\\
40	0.0134148699296971\\
41	0.0134148678733617\\
42	0.0134148657817947\\
43	0.0134148636543923\\
44	0.0134148614905404\\
45	0.0134148592896143\\
46	0.0134148570509788\\
47	0.0134148547739878\\
48	0.013414852457984\\
49	0.0134148501022991\\
50	0.0134148477062533\\
51	0.013414845269155\\
52	0.0134148427903012\\
53	0.0134148402689766\\
54	0.0134148377044538\\
55	0.0134148350959929\\
56	0.0134148324428414\\
57	0.0134148297442341\\
58	0.0134148269993926\\
59	0.0134148242075253\\
60	0.0134148213678269\\
61	0.0134148184794787\\
62	0.0134148155416478\\
63	0.0134148125534871\\
64	0.0134148095141353\\
65	0.0134148064227161\\
66	0.0134148032783385\\
67	0.0134148000800962\\
68	0.0134147968270675\\
69	0.0134147935183149\\
70	0.0134147901528851\\
71	0.0134147867298085\\
72	0.0134147832480989\\
73	0.0134147797067535\\
74	0.0134147761047521\\
75	0.0134147724410576\\
76	0.0134147687146148\\
77	0.0134147649243507\\
78	0.0134147610691743\\
79	0.0134147571479756\\
80	0.013414753159626\\
81	0.0134147491029777\\
82	0.0134147449768633\\
83	0.0134147407800955\\
84	0.0134147365114669\\
85	0.0134147321697497\\
86	0.0134147277536949\\
87	0.0134147232620327\\
88	0.0134147186934712\\
89	0.0134147140466971\\
90	0.0134147093203745\\
91	0.0134147045131448\\
92	0.0134146996236264\\
93	0.0134146946504142\\
94	0.0134146895920793\\
95	0.0134146844471685\\
96	0.013414679214204\\
97	0.0134146738916828\\
98	0.0134146684780766\\
99	0.013414662971831\\
100	0.0134146573713652\\
101	0.0134146516750717\\
102	0.0134146458813158\\
103	0.013414639988435\\
104	0.0134146339947384\\
105	0.0134146278985069\\
106	0.0134146216979918\\
107	0.0134146153914152\\
108	0.0134146089769685\\
109	0.0134146024528131\\
110	0.0134145958170788\\
111	0.0134145890678639\\
112	0.0134145822032344\\
113	0.0134145752212236\\
114	0.0134145681198316\\
115	0.0134145608970243\\
116	0.0134145535507335\\
117	0.0134145460788558\\
118	0.0134145384792523\\
119	0.0134145307497478\\
120	0.0134145228881303\\
121	0.0134145148921504\\
122	0.0134145067595206\\
123	0.0134144984879146\\
124	0.013414490074967\\
125	0.013414481518272\\
126	0.0134144728153834\\
127	0.0134144639638135\\
128	0.0134144549610325\\
129	0.0134144458044677\\
130	0.0134144364915032\\
131	0.0134144270194785\\
132	0.0134144173856882\\
133	0.0134144075873813\\
134	0.0134143976217601\\
135	0.0134143874859796\\
136	0.0134143771771468\\
137	0.0134143666923196\\
138	0.0134143560285064\\
139	0.0134143451826647\\
140	0.013414334151701\\
141	0.0134143229324692\\
142	0.0134143115217701\\
143	0.0134142999163502\\
144	0.0134142881129009\\
145	0.0134142761080575\\
146	0.013414263898399\\
147	0.0134142514804463\\
148	0.0134142388506616\\
149	0.0134142260054473\\
150	0.0134142129411449\\
151	0.0134141996540344\\
152	0.0134141861403325\\
153	0.0134141723961921\\
154	0.0134141584177013\\
155	0.0134141442008818\\
156	0.0134141297416882\\
157	0.0134141150360067\\
158	0.013414100079654\\
159	0.0134140848683761\\
160	0.0134140693978471\\
161	0.0134140536636681\\
162	0.0134140376613657\\
163	0.013414021386391\\
164	0.0134140048341181\\
165	0.013413987999843\\
166	0.0134139708787824\\
167	0.0134139534660718\\
168	0.0134139357567647\\
169	0.013413917745831\\
170	0.0134138994281554\\
171	0.0134138807985364\\
172	0.0134138618516842\\
173	0.01341384258222\\
174	0.0134138229846737\\
175	0.013413803053483\\
176	0.0134137827829913\\
177	0.0134137621674464\\
178	0.013413741200999\\
179	0.0134137198777005\\
180	0.0134136981915021\\
181	0.0134136761362524\\
182	0.0134136537056959\\
183	0.0134136308934714\\
184	0.01341360769311\\
185	0.0134135840980331\\
186	0.0134135601015509\\
187	0.0134135356968605\\
188	0.0134135108770435\\
189	0.0134134856350643\\
190	0.0134134599637685\\
191	0.0134134338558799\\
192	0.0134134073039996\\
193	0.0134133803006027\\
194	0.0134133528380371\\
195	0.0134133249085209\\
196	0.0134132965041399\\
197	0.013413267616846\\
198	0.0134132382384542\\
199	0.0134132083606407\\
200	0.0134131779749403\\
201	0.0134131470727441\\
202	0.0134131156452968\\
203	0.0134130836836943\\
204	0.0134130511788811\\
205	0.0134130181216477\\
206	0.0134129845026279\\
207	0.013412950312296\\
208	0.0134129155409642\\
209	0.0134128801787796\\
210	0.0134128442157214\\
211	0.013412807641598\\
212	0.0134127704460437\\
213	0.0134127326185163\\
214	0.0134126941482932\\
215	0.013412655024469\\
216	0.0134126152359517\\
217	0.0134125747714596\\
218	0.0134125336195182\\
219	0.0134124917684562\\
220	0.0134124492064028\\
221	0.0134124059212835\\
222	0.0134123619008166\\
223	0.01341231713251\\
224	0.0134122716036567\\
225	0.0134122253013316\\
226	0.0134121782123873\\
227	0.01341213032345\\
228	0.0134120816209157\\
229	0.0134120320909459\\
230	0.0134119817194636\\
231	0.0134119304921485\\
232	0.0134118783944332\\
233	0.0134118254114981\\
234	0.0134117715282674\\
235	0.0134117167294041\\
236	0.0134116609993051\\
237	0.0134116043220968\\
238	0.0134115466816296\\
239	0.0134114880614734\\
240	0.0134114284449117\\
241	0.013411367814937\\
242	0.0134113061542451\\
243	0.0134112434452295\\
244	0.013411179669976\\
245	0.0134111148102567\\
246	0.0134110488475242\\
247	0.013410981762906\\
248	0.0134109135371976\\
249	0.0134108441508569\\
250	0.0134107735839975\\
251	0.0134107018163823\\
252	0.0134106288274165\\
253	0.0134105545961411\\
254	0.0134104791012259\\
255	0.0134104023209621\\
256	0.0134103242332551\\
257	0.0134102448156171\\
258	0.0134101640451593\\
259	0.0134100818985841\\
260	0.0134099983521772\\
261	0.0134099133817993\\
262	0.0134098269628775\\
263	0.0134097390703971\\
264	0.0134096496788924\\
265	0.0134095587624378\\
266	0.0134094662946385\\
267	0.0134093722486211\\
268	0.0134092765970235\\
269	0.0134091793119852\\
270	0.0134090803651367\\
271	0.0134089797275882\\
272	0.0134088773699192\\
273	0.0134087732621672\\
274	0.0134086673738195\\
275	0.0134085596738074\\
276	0.013408450130494\\
277	0.0134083387116404\\
278	0.0134082253844058\\
279	0.0134081101153357\\
280	0.0134079928703478\\
281	0.0134078736147188\\
282	0.0134077523130701\\
283	0.0134076289293528\\
284	0.0134075034268335\\
285	0.013407375768078\\
286	0.0134072459149358\\
287	0.0134071138285235\\
288	0.0134069794692081\\
289	0.0134068427965894\\
290	0.0134067037694818\\
291	0.0134065623458963\\
292	0.0134064184830208\\
293	0.0134062721372005\\
294	0.0134061232639177\\
295	0.0134059718177704\\
296	0.0134058177524507\\
297	0.0134056610207223\\
298	0.0134055015743972\\
299	0.0134053393643115\\
300	0.0134051743403007\\
301	0.013405006451174\\
302	0.0134048356446874\\
303	0.0134046618675164\\
304	0.0134044850652274\\
305	0.0134043051822483\\
306	0.0134041221618388\\
307	0.0134039359460592\\
308	0.0134037464757396\\
309	0.0134035536904462\\
310	0.0134033575284439\\
311	0.0134031579266446\\
312	0.0134029548205422\\
313	0.0134027481441729\\
314	0.0134025378301587\\
315	0.0134023238096377\\
316	0.0134021060121973\\
317	0.0134018843658272\\
318	0.0134016587968699\\
319	0.0134014292299697\\
320	0.013401195588019\\
321	0.013400957792102\\
322	0.0134007157614362\\
323	0.0134004694133106\\
324	0.0134002186630216\\
325	0.0133999634238046\\
326	0.0133997036067635\\
327	0.0133994391207952\\
328	0.0133991698725107\\
329	0.0133988957661521\\
330	0.0133986167035043\\
331	0.0133983325838023\\
332	0.0133980433036318\\
333	0.0133977487568256\\
334	0.0133974488343512\\
335	0.0133971434241934\\
336	0.0133968324112274\\
337	0.0133965156770837\\
338	0.0133961931000041\\
339	0.0133958645546863\\
340	0.0133955299121172\\
341	0.0133951890393943\\
342	0.0133948417995313\\
343	0.0133944880512496\\
344	0.0133941276487516\\
345	0.0133937604414753\\
346	0.0133933862738272\\
347	0.0133930049848914\\
348	0.0133926164081135\\
349	0.0133922203709535\\
350	0.0133918166945069\\
351	0.0133914051930872\\
352	0.0133909856737642\\
353	0.0133905579358524\\
354	0.0133901217703497\\
355	0.0133896769593673\\
356	0.0133892232756554\\
357	0.0133887604822079\\
358	0.0133882883312323\\
359	0.0133878065626549\\
360	0.0133873149038293\\
361	0.0133868130687519\\
362	0.0133863007575523\\
363	0.0133857776569514\\
364	0.0133852434437845\\
365	0.013384697797695\\
366	0.0133841404379945\\
367	0.013383571187586\\
368	0.0133829895015238\\
369	0.0133822925247362\\
370	0.0133815762058525\\
371	0.013380847870976\\
372	0.0133801073094401\\
373	0.0133793543065388\\
374	0.0133785886430582\\
375	0.013377810093949\\
376	0.0133770184250384\\
377	0.0133762133884424\\
378	0.013375394734917\\
379	0.0133745622727953\\
380	0.0133737157824122\\
381	0.0133728550086828\\
382	0.0133719796906904\\
383	0.0133710895614253\\
384	0.0133701843475\\
385	0.0133692637688379\\
386	0.0133683275383337\\
387	0.0133673753614823\\
388	0.0133664069359745\\
389	0.0133654219512558\\
390	0.0133644200880473\\
391	0.0133634010178258\\
392	0.0133623644022626\\
393	0.0133613098926188\\
394	0.0133602371291007\\
395	0.0133591457401762\\
396	0.013358035341861\\
397	0.0133569055369912\\
398	0.0133557559145165\\
399	0.0133545860488946\\
400	0.0133533954997756\\
401	0.013352183812435\\
402	0.0133509505200123\\
403	0.0133496951496614\\
404	0.0133484172347461\\
405	0.0133471163227695\\
406	0.0133457918972152\\
407	0.0133444431307396\\
408	0.0133430694071201\\
409	0.0133416701955969\\
410	0.0133402449973673\\
411	0.0133387933723088\\
412	0.0133373146689459\\
413	0.0133358062911943\\
414	0.0133342644378274\\
415	0.01333269178558\\
416	0.0133310898335028\\
417	0.0133294579161511\\
418	0.0133277953474446\\
419	0.0133261014209448\\
420	0.0133243754105315\\
421	0.0133226165686372\\
422	0.013320824112104\\
423	0.0133189971931415\\
424	0.0133171349303551\\
425	0.0133152364461519\\
426	0.0133133008294374\\
427	0.0133113271336002\\
428	0.0133093143743227\\
429	0.013307261527198\\
430	0.0133051675251265\\
431	0.0133030312554618\\
432	0.0133008515568784\\
433	0.0132986272159265\\
434	0.013296356963232\\
435	0.0132940394692921\\
436	0.0132916733398147\\
437	0.0132892571105308\\
438	0.013286789241395\\
439	0.0132842681100409\\
440	0.0132816920042748\\
441	0.0132790591131363\\
442	0.0132763675150305\\
443	0.0132736151577758\\
444	0.0132707998127569\\
445	0.0132679189455976\\
446	0.0132649693828738\\
447	0.0132619473679989\\
448	0.0132588529433789\\
449	0.0132556835118719\\
450	0.0132524358996166\\
451	0.0132491065188316\\
452	0.0132456914464316\\
453	0.013242186386854\\
454	0.013238586649609\\
455	0.0132348871720032\\
456	0.0132310826168241\\
457	0.0132271672621339\\
458	0.013223130999815\\
459	0.0132180806965968\\
460	0.0132127388466801\\
461	0.0132072770582642\\
462	0.013201692785451\\
463	0.0131959752525703\\
464	0.0131901152721448\\
465	0.0131841051296938\\
466	0.0131779407292673\\
467	0.0131716244691192\\
468	0.0131654521055893\\
469	0.0131592267559905\\
470	0.0131527181562933\\
471	0.0131431202552702\\
472	0.0131333539943084\\
473	0.0131234151909817\\
474	0.0131132995360054\\
475	0.0131030026145868\\
476	0.0130925200630023\\
477	0.0130818481749884\\
478	0.013070983554991\\
479	0.0130599231643694\\
480	0.0130486448834303\\
481	0.0130371458135307\\
482	0.0130254259153766\\
483	0.0130134805420878\\
484	0.0130013048442184\\
485	0.0129888941927184\\
486	0.0129762447744911\\
487	0.0129633563459099\\
488	0.0129502438180365\\
489	0.0129379219495849\\
490	0.0129258140433674\\
491	0.0129134477691533\\
492	0.0129008132372538\\
493	0.012887900786866\\
494	0.0128746993445672\\
495	0.0128611992433007\\
496	0.0128473889037068\\
497	0.0128332557361546\\
498	0.0128187832680602\\
499	0.012803957731058\\
500	0.0127887667988419\\
501	0.0127731878736328\\
502	0.0127571691255732\\
503	0.0127406813699044\\
504	0.0127236798947353\\
505	0.012700707185963\\
506	0.0126769958202807\\
507	0.0126528336782187\\
508	0.0126293103073263\\
509	0.0126084183583063\\
510	0.0125870971357296\\
511	0.0125653406545835\\
512	0.0125431424245305\\
513	0.0125205059407741\\
514	0.0124974491568105\\
515	0.0124739309632564\\
516	0.0124500018553647\\
517	0.0124255678586512\\
518	0.0124005546324037\\
519	0.0123749475495255\\
520	0.0123487218301903\\
521	0.012321853105402\\
522	0.0122943109761287\\
523	0.0122660582332452\\
524	0.0122370489072429\\
525	0.0122071786946597\\
526	0.0121764208705988\\
527	0.0121447461470646\\
528	0.0121121607906459\\
529	0.0120786276241799\\
530	0.0120441829525235\\
531	0.012023255979046\\
532	0.0119949957240226\\
533	0.0119474905219952\\
534	0.0118988839602698\\
535	0.011849078594133\\
536	0.0117979877137699\\
537	0.011745513356831\\
538	0.0116915443412781\\
539	0.0116359539156808\\
540	0.0115785968561852\\
541	0.0115192449372247\\
542	0.0114576605551375\\
543	0.0113936126473879\\
544	0.0113031030936526\\
545	0.0112020883418527\\
546	0.0110994968145173\\
547	0.0109953160618321\\
548	0.0108895405722017\\
549	0.0107821735232877\\
550	0.0106732284943602\\
551	0.0105627303086983\\
552	0.0104506429843039\\
553	0.0103366487398546\\
554	0.0102207496829703\\
555	0.0101043963541868\\
556	0.0100509336587227\\
557	0.00999693072323994\\
558	0.00994246687712072\\
559	0.0098876392311834\\
560	0.00983254675135841\\
561	0.00977733583477016\\
562	0.00972218376679349\\
563	0.00966730481245616\\
564	0.00961295804394078\\
565	0.00955945714285719\\
566	0.00950718321543156\\
567	0.00945659810990165\\
568	0.0094052073164819\\
569	0.00935288772157468\\
570	0.00929962135438986\\
571	0.00924538513148849\\
572	0.00919015291893353\\
573	0.00913390631613418\\
574	0.0090766196163146\\
575	0.00901825650753089\\
576	0.00895876561842256\\
577	0.00889807436218399\\
578	0.00883608062306942\\
579	0.00877264139895938\\
580	0.00870759893646839\\
581	0.00864089830205915\\
582	0.00857246413507161\\
583	0.00850225076988874\\
584	0.00843013586447\\
585	0.00835611259214755\\
586	0.0082801197384134\\
587	0.00820181731099155\\
588	0.0080795783656128\\
589	0.00791554451502019\\
590	0.00774951452924046\\
591	0.0075392339828192\\
592	0.00718230859247705\\
593	0.00679628126808101\\
594	0.00631581315310309\\
595	0.00593854324538158\\
596	0.00559184182825266\\
597	0.00498881296807123\\
598	0.00357511483354343\\
599	0\\
600	0\\
};
\addplot [color=mycolor10,solid,forget plot]
  table[row sep=crcr]{%
1	0.0134827010444315\\
2	0.0134827009245038\\
3	0.0134827008025232\\
4	0.0134827006784545\\
5	0.013482700552262\\
6	0.0134827004239092\\
7	0.0134827002933591\\
8	0.0134827001605738\\
9	0.0134827000255151\\
10	0.0134826998881441\\
11	0.0134826997484209\\
12	0.0134826996063053\\
13	0.0134826994617562\\
14	0.013482699314732\\
15	0.01348269916519\\
16	0.0134826990130872\\
17	0.0134826988583797\\
18	0.0134826987010227\\
19	0.0134826985409708\\
20	0.0134826983781778\\
21	0.0134826982125967\\
22	0.0134826980441797\\
23	0.0134826978728782\\
24	0.0134826976986426\\
25	0.0134826975214228\\
26	0.0134826973411675\\
27	0.0134826971578247\\
28	0.0134826969713414\\
29	0.0134826967816639\\
30	0.0134826965887373\\
31	0.013482696392506\\
32	0.0134826961929134\\
33	0.0134826959899017\\
34	0.0134826957834125\\
35	0.0134826955733862\\
36	0.013482695359762\\
37	0.0134826951424784\\
38	0.0134826949214727\\
39	0.0134826946966811\\
40	0.0134826944680387\\
41	0.0134826942354795\\
42	0.0134826939989366\\
43	0.0134826937583415\\
44	0.013482693513625\\
45	0.0134826932647163\\
46	0.0134826930115438\\
47	0.0134826927540344\\
48	0.0134826924921138\\
49	0.0134826922257066\\
50	0.0134826919547358\\
51	0.0134826916791233\\
52	0.0134826913987897\\
53	0.0134826911136542\\
54	0.0134826908236346\\
55	0.0134826905286472\\
56	0.0134826902286071\\
57	0.0134826899234277\\
58	0.0134826896130211\\
59	0.0134826892972979\\
60	0.0134826889761672\\
61	0.0134826886495363\\
62	0.0134826883173112\\
63	0.0134826879793962\\
64	0.013482687635694\\
65	0.0134826872861056\\
66	0.0134826869305304\\
67	0.0134826865688659\\
68	0.013482686201008\\
69	0.0134826858268508\\
70	0.0134826854462867\\
71	0.0134826850592061\\
72	0.0134826846654977\\
73	0.0134826842650481\\
74	0.0134826838577421\\
75	0.0134826834434626\\
76	0.0134826830220905\\
77	0.0134826825935044\\
78	0.0134826821575813\\
79	0.0134826817141958\\
80	0.0134826812632203\\
81	0.0134826808045253\\
82	0.013482680337979\\
83	0.0134826798634472\\
84	0.0134826793807937\\
85	0.0134826788898798\\
86	0.0134826783905644\\
87	0.0134826778827042\\
88	0.0134826773661534\\
89	0.0134826768407636\\
90	0.013482676306384\\
91	0.0134826757628613\\
92	0.0134826752100394\\
93	0.0134826746477598\\
94	0.0134826740758611\\
95	0.0134826734941793\\
96	0.0134826729025476\\
97	0.0134826723007963\\
98	0.0134826716887529\\
99	0.0134826710662419\\
100	0.013482670433085\\
101	0.0134826697891007\\
102	0.0134826691341046\\
103	0.0134826684679089\\
104	0.0134826677903229\\
105	0.0134826671011526\\
106	0.0134826664002006\\
107	0.0134826656872663\\
108	0.0134826649621457\\
109	0.0134826642246313\\
110	0.013482663474512\\
111	0.0134826627115733\\
112	0.013482661935597\\
113	0.0134826611463613\\
114	0.0134826603436404\\
115	0.013482659527205\\
116	0.0134826586968217\\
117	0.0134826578522533\\
118	0.0134826569932585\\
119	0.013482656119592\\
120	0.0134826552310042\\
121	0.0134826543272414\\
122	0.0134826534080458\\
123	0.0134826524731549\\
124	0.0134826515223019\\
125	0.0134826505552157\\
126	0.0134826495716203\\
127	0.0134826485712352\\
128	0.0134826475537752\\
129	0.0134826465189503\\
130	0.0134826454664654\\
131	0.0134826443960207\\
132	0.0134826433073112\\
133	0.0134826422000268\\
134	0.0134826410738521\\
135	0.0134826399284663\\
136	0.0134826387635436\\
137	0.0134826375787521\\
138	0.0134826363737549\\
139	0.0134826351482091\\
140	0.0134826339017659\\
141	0.013482632634071\\
142	0.0134826313447638\\
143	0.0134826300334778\\
144	0.01348262869984\\
145	0.0134826273434716\\
146	0.013482625963987\\
147	0.0134826245609945\\
148	0.0134826231340954\\
149	0.0134826216828846\\
150	0.01348262020695\\
151	0.0134826187058726\\
152	0.0134826171792264\\
153	0.0134826156265782\\
154	0.0134826140474875\\
155	0.0134826124415063\\
156	0.0134826108081791\\
157	0.0134826091470429\\
158	0.0134826074576267\\
159	0.0134826057394516\\
160	0.0134826039920307\\
161	0.0134826022148689\\
162	0.0134826004074627\\
163	0.0134825985693002\\
164	0.0134825966998609\\
165	0.0134825947986154\\
166	0.0134825928650256\\
167	0.0134825908985442\\
168	0.0134825888986146\\
169	0.0134825868646711\\
170	0.0134825847961383\\
171	0.0134825826924311\\
172	0.0134825805529546\\
173	0.013482578377104\\
174	0.0134825761642641\\
175	0.0134825739138095\\
176	0.0134825716251043\\
177	0.0134825692975017\\
178	0.0134825669303443\\
179	0.0134825645229634\\
180	0.0134825620746792\\
181	0.0134825595848003\\
182	0.0134825570526237\\
183	0.0134825544774348\\
184	0.0134825518585066\\
185	0.0134825491951002\\
186	0.013482546486464\\
187	0.0134825437318339\\
188	0.0134825409304328\\
189	0.0134825380814707\\
190	0.013482535184144\\
191	0.013482532237636\\
192	0.0134825292411157\\
193	0.0134825261937386\\
194	0.0134825230946456\\
195	0.0134825199429633\\
196	0.0134825167378035\\
197	0.0134825134782631\\
198	0.0134825101634237\\
199	0.0134825067923515\\
200	0.0134825033640968\\
201	0.0134824998776939\\
202	0.0134824963321608\\
203	0.0134824927264992\\
204	0.0134824890596934\\
205	0.013482485330711\\
206	0.0134824815385019\\
207	0.0134824776819985\\
208	0.0134824737601147\\
209	0.0134824697717465\\
210	0.013482465715771\\
211	0.0134824615910463\\
212	0.0134824573964112\\
213	0.0134824531306847\\
214	0.013482448792666\\
215	0.0134824443811338\\
216	0.0134824398948461\\
217	0.0134824353325398\\
218	0.0134824306929305\\
219	0.0134824259747118\\
220	0.0134824211765551\\
221	0.0134824162971094\\
222	0.0134824113350005\\
223	0.013482406288831\\
224	0.0134824011571795\\
225	0.0134823959386006\\
226	0.0134823906316239\\
227	0.0134823852347544\\
228	0.0134823797464711\\
229	0.0134823741652273\\
230	0.0134823684894497\\
231	0.0134823627175382\\
232	0.0134823568478652\\
233	0.0134823508787753\\
234	0.0134823448085845\\
235	0.0134823386355801\\
236	0.0134823323580199\\
237	0.0134823259741318\\
238	0.013482319482113\\
239	0.0134823128801298\\
240	0.0134823061663169\\
241	0.0134822993387766\\
242	0.0134822923955787\\
243	0.0134822853347593\\
244	0.0134822781543207\\
245	0.0134822708522304\\
246	0.0134822634264207\\
247	0.0134822558747879\\
248	0.0134822481951918\\
249	0.0134822403854547\\
250	0.0134822324433612\\
251	0.0134822243666569\\
252	0.013482216153048\\
253	0.0134822078002005\\
254	0.0134821993057396\\
255	0.0134821906672486\\
256	0.0134821818822682\\
257	0.0134821729482957\\
258	0.0134821638627844\\
259	0.0134821546231423\\
260	0.0134821452267315\\
261	0.0134821356708671\\
262	0.0134821259528166\\
263	0.0134821160697987\\
264	0.0134821060189821\\
265	0.0134820957974851\\
266	0.013482085402374\\
267	0.0134820748306624\\
268	0.0134820640793096\\
269	0.0134820531452202\\
270	0.0134820420252421\\
271	0.0134820307161657\\
272	0.013482019214723\\
273	0.0134820075175864\\
274	0.0134819956213686\\
275	0.0134819835226207\\
276	0.0134819712178281\\
277	0.0134819587034108\\
278	0.0134819459757223\\
279	0.0134819330310478\\
280	0.0134819198656028\\
281	0.0134819064755315\\
282	0.0134818928569048\\
283	0.0134818790057194\\
284	0.013481864917895\\
285	0.0134818505892731\\
286	0.0134818360156151\\
287	0.0134818211925997\\
288	0.0134818061158217\\
289	0.0134817907807892\\
290	0.0134817751829214\\
291	0.0134817593175467\\
292	0.0134817431798998\\
293	0.0134817267651193\\
294	0.0134817100682453\\
295	0.0134816930842162\\
296	0.013481675807866\\
297	0.0134816582339214\\
298	0.0134816403569982\\
299	0.0134816221715984\\
300	0.0134816036721062\\
301	0.0134815848527846\\
302	0.0134815657077716\\
303	0.0134815462310755\\
304	0.0134815264165713\\
305	0.0134815062579958\\
306	0.013481485748943\\
307	0.0134814648828592\\
308	0.0134814436530372\\
309	0.0134814220526096\\
310	0.0134814000745399\\
311	0.0134813777116123\\
312	0.0134813549564259\\
313	0.0134813318013979\\
314	0.0134813082387514\\
315	0.0134812842605035\\
316	0.0134812598584552\\
317	0.0134812350241814\\
318	0.0134812097490192\\
319	0.0134811840240555\\
320	0.0134811578401136\\
321	0.0134811311877385\\
322	0.0134811040571813\\
323	0.0134810764383817\\
324	0.0134810483209489\\
325	0.0134810196941409\\
326	0.0134809905468423\\
327	0.0134809608675387\\
328	0.0134809306442897\\
329	0.013480899864699\\
330	0.0134808685158808\\
331	0.0134808365844235\\
332	0.0134808040563496\\
333	0.0134807709170706\\
334	0.0134807371513378\\
335	0.0134807027431879\\
336	0.0134806676758817\\
337	0.0134806319318374\\
338	0.0134805954925551\\
339	0.013480558338534\\
340	0.0134805204491793\\
341	0.0134804818026987\\
342	0.0134804423759873\\
343	0.0134804021444987\\
344	0.0134803610821009\\
345	0.0134803191609154\\
346	0.0134802763511367\\
347	0.0134802326208307\\
348	0.0134801879357078\\
349	0.0134801422588681\\
350	0.0134800955505154\\
351	0.0134800477676343\\
352	0.0134799988636274\\
353	0.0134799487879092\\
354	0.0134798974854586\\
355	0.0134798448963313\\
356	0.0134797909551102\\
357	0.0134797355901977\\
358	0.0134796787229736\\
359	0.013479620266971\\
360	0.013479560126764\\
361	0.0134794981964129\\
362	0.0134794343568317\\
363	0.0134793684701662\\
364	0.0134793003652451\\
365	0.0134792297950084\\
366	0.0134791563025685\\
367	0.0134790787814495\\
368	0.0134789940540261\\
369	0.0134788225905738\\
370	0.0134786421622733\\
371	0.0134784588125423\\
372	0.0134782724952062\\
373	0.0134780831631753\\
374	0.0134778907681663\\
375	0.0134776952603716\\
376	0.0134774965886828\\
377	0.0134772947032948\\
378	0.0134770895610133\\
379	0.01347688111418\\
380	0.0134766693104359\\
381	0.0134764540964931\\
382	0.0134762354180778\\
383	0.0134760132198659\\
384	0.0134757874454072\\
385	0.0134755580370403\\
386	0.0134753249357949\\
387	0.0134750880812801\\
388	0.0134748474115577\\
389	0.013474602862999\\
390	0.0134743543701214\\
391	0.0134741018654059\\
392	0.0134738452790901\\
393	0.0134735845389376\\
394	0.0134733195699808\\
395	0.0134730502942381\\
396	0.0134727766304081\\
397	0.0134724984935524\\
398	0.013472215794791\\
399	0.0134719284410651\\
400	0.0134716363350594\\
401	0.0134713393754015\\
402	0.0134710374571164\\
403	0.0134707304715531\\
404	0.0134704183026849\\
405	0.0134701008133592\\
406	0.0134697778267466\\
407	0.0134694491868978\\
408	0.0134691147405925\\
409	0.0134687743277694\\
410	0.0134684277697531\\
411	0.0134680747943538\\
412	0.0134677148855409\\
413	0.0134673475368608\\
414	0.0134669729739231\\
415	0.0134665912598975\\
416	0.0134662022158768\\
417	0.0134658056562991\\
418	0.0134654013887533\\
419	0.0134649892136385\\
420	0.0134645689231513\\
421	0.0134641402987486\\
422	0.0134637031079143\\
423	0.0134632571081398\\
424	0.0134628020505101\\
425	0.0134623376746722\\
426	0.0134618637080433\\
427	0.0134613798649436\\
428	0.0134608858456454\\
429	0.0134603813353257\\
430	0.013459866002912\\
431	0.013459339499804\\
432	0.0134588014584575\\
433	0.0134582514908082\\
434	0.0134576891865135\\
435	0.0134571141109814\\
436	0.0134565258031496\\
437	0.0134559237729594\\
438	0.0134553074984349\\
439	0.0134546764222034\\
440	0.0134540299471266\\
441	0.0134533674303381\\
442	0.0134526881742701\\
443	0.0134519914123622\\
444	0.0134512762883091\\
445	0.0134505418424898\\
446	0.0134497870852128\\
447	0.0134490113382093\\
448	0.0134482135369743\\
449	0.0134473924689366\\
450	0.013446546782067\\
451	0.0134456749784278\\
452	0.0134447753873113\\
453	0.0134438461213334\\
454	0.0134428849824512\\
455	0.0134418892079458\\
456	0.0134408546882444\\
457	0.0134397734222369\\
458	0.0134386253703978\\
459	0.0134367047619298\\
460	0.0134346007627936\\
461	0.0134324534544652\\
462	0.013430261091304\\
463	0.0134280221939364\\
464	0.0134257353518306\\
465	0.0134233989451662\\
466	0.0134210127654728\\
467	0.0134185746922587\\
468	0.0134159137836394\\
469	0.0134131253562001\\
470	0.0134102722380929\\
471	0.013407366022477\\
472	0.0134044046970774\\
473	0.0134013861008012\\
474	0.0133983079142918\\
475	0.0133951676661396\\
476	0.0133919627790866\\
477	0.0133886906994249\\
478	0.0133853584756237\\
479	0.0133819540096636\\
480	0.0133784670549945\\
481	0.0133748946406184\\
482	0.0133712330759159\\
483	0.0133674783849652\\
484	0.013363626298833\\
485	0.0133596721851578\\
486	0.0133556108381455\\
487	0.0133514354691886\\
488	0.0133471319214509\\
489	0.0133421365324358\\
490	0.0133367640115309\\
491	0.0133312607324878\\
492	0.0133256206557325\\
493	0.0133198371639317\\
494	0.0133139031896577\\
495	0.0133078108486197\\
496	0.0133015514080919\\
497	0.0132951150274935\\
498	0.0132884914106425\\
499	0.0132816700229335\\
500	0.0132746405374766\\
501	0.0132673828862216\\
502	0.0132598624666167\\
503	0.0132520638943832\\
504	0.0132439309280994\\
505	0.0132310060311702\\
506	0.013217506362061\\
507	0.0132036498742862\\
508	0.0131888657849896\\
509	0.0131720523245974\\
510	0.0131548950503627\\
511	0.013137382373354\\
512	0.01311950271671\\
513	0.0131012437962938\\
514	0.0130825865701937\\
515	0.0130635250231859\\
516	0.0130440558425903\\
517	0.0130241282541478\\
518	0.0130037110193216\\
519	0.0129827769137441\\
520	0.0129613858402691\\
521	0.0129394617355794\\
522	0.012916941040038\\
523	0.0128937960044298\\
524	0.0128699291158758\\
525	0.0128452834311488\\
526	0.0128198413153205\\
527	0.0127935851421469\\
528	0.0127674028256812\\
529	0.0127407523233205\\
530	0.0127132324135985\\
531	0.0126769985315014\\
532	0.0126397275406422\\
533	0.0126030266030857\\
534	0.0125670606395804\\
535	0.0125301138866154\\
536	0.0124921310664894\\
537	0.0124530517957935\\
538	0.0124128105316109\\
539	0.0123713403308501\\
540	0.0123286044295985\\
541	0.0122885437677416\\
542	0.012249640215702\\
543	0.0122092177372384\\
544	0.0121471371196391\\
545	0.0120759788892357\\
546	0.0120030094141084\\
547	0.0119281663124486\\
548	0.0118513316197983\\
549	0.0117723787037577\\
550	0.0116911819851764\\
551	0.0116076640950326\\
552	0.0115252412543895\\
553	0.0114568516315375\\
554	0.0113858548896472\\
555	0.0113112146929633\\
556	0.0111996008565847\\
557	0.0110859072514765\\
558	0.0109700816803649\\
559	0.0108520717474002\\
560	0.0107318195872003\\
561	0.0106092810625123\\
562	0.0104844183042965\\
563	0.0103571981014711\\
564	0.010229196474915\\
565	0.0100989357460154\\
566	0.00996556898396692\\
567	0.00982906327339909\\
568	0.00975677063061848\\
569	0.00968632648367908\\
570	0.0096150755783179\\
571	0.00954315619903026\\
572	0.00947061485625028\\
573	0.00939712619316549\\
574	0.00932283394429589\\
575	0.00924792300197075\\
576	0.00917262823381608\\
577	0.00909724756957369\\
578	0.00902216065203892\\
579	0.00894785644060203\\
580	0.00887419226306399\\
581	0.00879884330774817\\
582	0.00872180054699871\\
583	0.00864305944617903\\
584	0.0085626205974466\\
585	0.008480489806672\\
586	0.00839667853748955\\
587	0.00831120341405846\\
588	0.00822407719928457\\
589	0.00813527710953944\\
590	0.00804486435975251\\
591	0.00793561760306448\\
592	0.00776256831893187\\
593	0.00758639978571355\\
594	0.00738834244898588\\
595	0.00691750222005476\\
596	0.00589506097552246\\
597	0.00498881296807123\\
598	0.00357511483354343\\
599	0\\
600	0\\
};
\addplot [color=mycolor11,solid,forget plot]
  table[row sep=crcr]{%
1	0.0135446655882236\\
2	0.0135446655790726\\
3	0.013544665569765\\
4	0.0135446655602981\\
5	0.0135446655506691\\
6	0.0135446655408753\\
7	0.0135446655309139\\
8	0.0135446655207819\\
9	0.0135446655104765\\
10	0.0135446654999946\\
11	0.0135446654893333\\
12	0.0135446654784895\\
13	0.01354466546746\\
14	0.0135446654562416\\
15	0.0135446654448312\\
16	0.0135446654332254\\
17	0.0135446654214208\\
18	0.0135446654094141\\
19	0.0135446653972018\\
20	0.0135446653847804\\
21	0.0135446653721462\\
22	0.0135446653592956\\
23	0.013544665346225\\
24	0.0135446653329305\\
25	0.0135446653194084\\
26	0.0135446653056546\\
27	0.0135446652916653\\
28	0.0135446652774363\\
29	0.0135446652629637\\
30	0.0135446652482431\\
31	0.0135446652332704\\
32	0.0135446652180413\\
33	0.0135446652025513\\
34	0.0135446651867959\\
35	0.0135446651707707\\
36	0.013544665154471\\
37	0.0135446651378921\\
38	0.0135446651210292\\
39	0.0135446651038774\\
40	0.0135446650864319\\
41	0.0135446650686875\\
42	0.0135446650506391\\
43	0.0135446650322816\\
44	0.0135446650136096\\
45	0.0135446649946178\\
46	0.0135446649753006\\
47	0.0135446649556526\\
48	0.013544664935668\\
49	0.013544664915341\\
50	0.0135446648946659\\
51	0.0135446648736366\\
52	0.0135446648522471\\
53	0.0135446648304913\\
54	0.0135446648083628\\
55	0.0135446647858552\\
56	0.0135446647629622\\
57	0.013544664739677\\
58	0.013544664715993\\
59	0.0135446646919033\\
60	0.0135446646674011\\
61	0.0135446646424792\\
62	0.0135446646171305\\
63	0.0135446645913476\\
64	0.0135446645651232\\
65	0.0135446645384497\\
66	0.0135446645113194\\
67	0.0135446644837245\\
68	0.0135446644556571\\
69	0.013544664427109\\
70	0.0135446643980721\\
71	0.0135446643685379\\
72	0.0135446643384981\\
73	0.013544664307944\\
74	0.0135446642768667\\
75	0.0135446642452573\\
76	0.0135446642131068\\
77	0.0135446641804058\\
78	0.013544664147145\\
79	0.0135446641133149\\
80	0.0135446640789057\\
81	0.0135446640439074\\
82	0.0135446640083102\\
83	0.0135446639721036\\
84	0.0135446639352773\\
85	0.0135446638978208\\
86	0.0135446638597233\\
87	0.0135446638209738\\
88	0.0135446637815612\\
89	0.0135446637414742\\
90	0.0135446637007013\\
91	0.0135446636592308\\
92	0.0135446636170507\\
93	0.013544663574149\\
94	0.0135446635305135\\
95	0.0135446634861314\\
96	0.0135446634409902\\
97	0.0135446633950769\\
98	0.0135446633483783\\
99	0.013544663300881\\
100	0.0135446632525715\\
101	0.0135446632034358\\
102	0.0135446631534599\\
103	0.0135446631026296\\
104	0.0135446630509301\\
105	0.0135446629983468\\
106	0.0135446629448646\\
107	0.0135446628904681\\
108	0.0135446628351419\\
109	0.01354466277887\\
110	0.0135446627216363\\
111	0.0135446626634246\\
112	0.0135446626042182\\
113	0.013544662544\\
114	0.013544662482753\\
115	0.0135446624204596\\
116	0.013544662357102\\
117	0.0135446622926621\\
118	0.0135446622271215\\
119	0.0135446621604615\\
120	0.013544662092663\\
121	0.0135446620237068\\
122	0.0135446619535731\\
123	0.0135446618822419\\
124	0.0135446618096928\\
125	0.0135446617359052\\
126	0.0135446616608581\\
127	0.0135446615845299\\
128	0.013544661506899\\
129	0.0135446614279433\\
130	0.0135446613476402\\
131	0.0135446612659669\\
132	0.0135446611829001\\
133	0.0135446610984161\\
134	0.013544661012491\\
135	0.0135446609251001\\
136	0.0135446608362188\\
137	0.0135446607458216\\
138	0.0135446606538829\\
139	0.0135446605603765\\
140	0.0135446604652758\\
141	0.0135446603685539\\
142	0.0135446602701831\\
143	0.0135446601701355\\
144	0.0135446600683828\\
145	0.0135446599648959\\
146	0.0135446598596456\\
147	0.0135446597526019\\
148	0.0135446596437344\\
149	0.0135446595330122\\
150	0.013544659420404\\
151	0.0135446593058777\\
152	0.0135446591894009\\
153	0.0135446590709406\\
154	0.0135446589504631\\
155	0.0135446588279344\\
156	0.0135446587033196\\
157	0.0135446585765834\\
158	0.0135446584476901\\
159	0.013544658316603\\
160	0.0135446581832851\\
161	0.0135446580476985\\
162	0.013544657909805\\
163	0.0135446577695654\\
164	0.0135446576269402\\
165	0.0135446574818889\\
166	0.0135446573343706\\
167	0.0135446571843435\\
168	0.0135446570317652\\
169	0.0135446568765926\\
170	0.0135446567187818\\
171	0.0135446565582883\\
172	0.0135446563950666\\
173	0.0135446562290708\\
174	0.013544656060254\\
175	0.0135446558885684\\
176	0.0135446557139656\\
177	0.0135446555363964\\
178	0.0135446553558107\\
179	0.0135446551721574\\
180	0.0135446549853849\\
181	0.0135446547954404\\
182	0.0135446546022704\\
183	0.0135446544058205\\
184	0.0135446542060352\\
185	0.0135446540028583\\
186	0.0135446537962326\\
187	0.0135446535860998\\
188	0.0135446533724008\\
189	0.0135446531550754\\
190	0.0135446529340624\\
191	0.0135446527092996\\
192	0.0135446524807237\\
193	0.0135446522482705\\
194	0.0135446520118745\\
195	0.0135446517714693\\
196	0.0135446515269871\\
197	0.0135446512783594\\
198	0.0135446510255161\\
199	0.0135446507683862\\
200	0.0135446505068975\\
201	0.0135446502409763\\
202	0.0135446499705481\\
203	0.0135446496955367\\
204	0.013544649415865\\
205	0.0135446491314544\\
206	0.013544648842225\\
207	0.0135446485480956\\
208	0.0135446482489836\\
209	0.0135446479448049\\
210	0.0135446476354743\\
211	0.0135446473209048\\
212	0.0135446470010081\\
213	0.0135446466756945\\
214	0.0135446463448727\\
215	0.0135446460084497\\
216	0.0135446456663313\\
217	0.0135446453184213\\
218	0.0135446449646222\\
219	0.0135446446048347\\
220	0.0135446442389578\\
221	0.0135446438668889\\
222	0.0135446434885235\\
223	0.0135446431037556\\
224	0.0135446427124771\\
225	0.0135446423145782\\
226	0.0135446419099473\\
227	0.013544641498471\\
228	0.0135446410800336\\
229	0.0135446406545177\\
230	0.0135446402218041\\
231	0.0135446397817711\\
232	0.0135446393342953\\
233	0.0135446388792511\\
234	0.0135446384165106\\
235	0.0135446379459441\\
236	0.0135446374674193\\
237	0.0135446369808019\\
238	0.0135446364859551\\
239	0.0135446359827399\\
240	0.0135446354710149\\
241	0.0135446349506363\\
242	0.0135446344214578\\
243	0.0135446338833306\\
244	0.0135446333361034\\
245	0.0135446327796221\\
246	0.0135446322137303\\
247	0.0135446316382686\\
248	0.013544631053075\\
249	0.0135446304579846\\
250	0.0135446298528298\\
251	0.01354462923744\\
252	0.0135446286116418\\
253	0.0135446279752584\\
254	0.0135446273281105\\
255	0.0135446266700152\\
256	0.0135446260007867\\
257	0.0135446253202359\\
258	0.0135446246281703\\
259	0.0135446239243941\\
260	0.0135446232087082\\
261	0.0135446224809098\\
262	0.0135446217407927\\
263	0.0135446209881469\\
264	0.013544620222759\\
265	0.0135446194444116\\
266	0.0135446186528833\\
267	0.0135446178479492\\
268	0.0135446170293801\\
269	0.0135446161969426\\
270	0.0135446153503995\\
271	0.0135446144895089\\
272	0.013544613614025\\
273	0.0135446127236976\\
274	0.0135446118182721\\
275	0.0135446108974892\\
276	0.0135446099610847\\
277	0.0135446090087897\\
278	0.0135446080403303\\
279	0.0135446070554277\\
280	0.013544606053798\\
281	0.0135446050351519\\
282	0.013544603999195\\
283	0.0135446029456273\\
284	0.0135446018741432\\
285	0.0135446007844316\\
286	0.0135445996761752\\
287	0.0135445985490511\\
288	0.01354459740273\\
289	0.0135445962368765\\
290	0.0135445950511487\\
291	0.0135445938451983\\
292	0.0135445926186701\\
293	0.013544591371202\\
294	0.0135445901024249\\
295	0.0135445888119625\\
296	0.0135445874994309\\
297	0.0135445861644386\\
298	0.0135445848065865\\
299	0.0135445834254671\\
300	0.0135445820206647\\
301	0.0135445805917552\\
302	0.0135445791383057\\
303	0.0135445776598743\\
304	0.0135445761560098\\
305	0.0135445746262516\\
306	0.0135445730701293\\
307	0.0135445714871624\\
308	0.01354456987686\\
309	0.0135445682387198\\
310	0.0135445665722272\\
311	0.0135445648768556\\
312	0.0135445631520671\\
313	0.0135445613973118\\
314	0.0135445596120268\\
315	0.0135445577956357\\
316	0.0135445559475479\\
317	0.0135445540671584\\
318	0.0135445521538467\\
319	0.0135445502069765\\
320	0.0135445482258948\\
321	0.0135445462099312\\
322	0.013544544158397\\
323	0.0135445420705845\\
324	0.0135445399457655\\
325	0.0135445377831909\\
326	0.013544535582089\\
327	0.0135445333416647\\
328	0.0135445310610977\\
329	0.0135445287395411\\
330	0.0135445263761202\\
331	0.0135445239699301\\
332	0.0135445215200338\\
333	0.0135445190254605\\
334	0.0135445164852028\\
335	0.0135445138982141\\
336	0.0135445112634057\\
337	0.0135445085796439\\
338	0.0135445058457457\\
339	0.0135445030604759\\
340	0.0135445002225416\\
341	0.0135444973305881\\
342	0.0135444943831934\\
343	0.0135444913788617\\
344	0.0135444883160172\\
345	0.0135444851929964\\
346	0.0135444820080395\\
347	0.0135444787592812\\
348	0.0135444754447402\\
349	0.0135444720623068\\
350	0.0135444686097299\\
351	0.0135444650846012\\
352	0.0135444614843391\\
353	0.0135444578061706\\
354	0.0135444540471148\\
355	0.013544450203962\\
356	0.013544446273237\\
357	0.0135444422511474\\
358	0.0135444381335358\\
359	0.0135444339157749\\
360	0.013544429592543\\
361	0.0135444251572839\\
362	0.0135444206008107\\
363	0.0135444159075856\\
364	0.0135444110456751\\
365	0.0135444059397022\\
366	0.0135444003999544\\
367	0.0135443939560351\\
368	0.0135443856817013\\
369	0.0135443771544858\\
370	0.0135443684920978\\
371	0.0135443596924323\\
372	0.0135443507533073\\
373	0.0135443416724163\\
374	0.0135443324472715\\
375	0.0135443230752331\\
376	0.0135443135539124\\
377	0.0135443038821088\\
378	0.0135442940580924\\
379	0.0135442840795147\\
380	0.0135442739439858\\
381	0.0135442636490718\\
382	0.0135442531922924\\
383	0.013544242571117\\
384	0.0135442317829619\\
385	0.0135442208251853\\
386	0.0135442096950834\\
387	0.0135441983898843\\
388	0.0135441869067426\\
389	0.0135441752427324\\
390	0.0135441633948397\\
391	0.013544151359954\\
392	0.0135441391348589\\
393	0.0135441267162224\\
394	0.0135441141005864\\
395	0.0135441012843576\\
396	0.0135440882638017\\
397	0.0135440750350464\\
398	0.0135440615941042\\
399	0.0135440479369319\\
400	0.0135440340595492\\
401	0.0135440199582147\\
402	0.0135440056295404\\
403	0.0135439910700731\\
404	0.013543976274333\\
405	0.013543961231781\\
406	0.0135439459366184\\
407	0.0135439303856296\\
408	0.0135439145765875\\
409	0.0135438985047629\\
410	0.0135438821506532\\
411	0.0135438654612502\\
412	0.0135438483892285\\
413	0.0135438309839511\\
414	0.013543813270509\\
415	0.0135437952421574\\
416	0.0135437768919685\\
417	0.0135437582128495\\
418	0.0135437391975382\\
419	0.0135437198385026\\
420	0.0135437001276234\\
421	0.0135436800557971\\
422	0.0135436596136035\\
423	0.0135436387919014\\
424	0.0135436175811696\\
425	0.0135435959714813\\
426	0.0135435739524761\\
427	0.0135435515133297\\
428	0.0135435286427206\\
429	0.0135435053287926\\
430	0.0135434815591143\\
431	0.0135434573206333\\
432	0.013543432599626\\
433	0.013543407381641\\
434	0.0135433816514343\\
435	0.0135433553928954\\
436	0.0135433285889577\\
437	0.0135433012214827\\
438	0.0135432732710955\\
439	0.013543244716918\\
440	0.013543215536086\\
441	0.0135431857028279\\
442	0.0135431551867576\\
443	0.0135431239502636\\
444	0.0135430919471355\\
445	0.0135430591343879\\
446	0.0135430255265884\\
447	0.0135429910942649\\
448	0.0135429557988381\\
449	0.0135429195955067\\
450	0.0135428824339565\\
451	0.0135428442557001\\
452	0.0135428049872605\\
453	0.0135427645212475\\
454	0.0135427226626812\\
455	0.013542678979174\\
456	0.0135426324061411\\
457	0.0135425803572847\\
458	0.0135425177949178\\
459	0.0135424524910871\\
460	0.0135423861055671\\
461	0.0135423184888302\\
462	0.0135422494887769\\
463	0.0135421789156911\\
464	0.0135421064578075\\
465	0.0135420316906338\\
466	0.0135419531202762\\
467	0.0135418654838934\\
468	0.0135416341376981\\
469	0.0135413405976787\\
470	0.0135410411476683\\
471	0.0135407354421003\\
472	0.0135404231026969\\
473	0.0135401037194476\\
474	0.0135397768556062\\
475	0.0135394420526863\\
476	0.0135390988031708\\
477	0.0135387463233793\\
478	0.0135383789935058\\
479	0.0135379997824866\\
480	0.013537608211046\\
481	0.0135372033927514\\
482	0.0135367843330014\\
483	0.0135363498948024\\
484	0.0135358987101663\\
485	0.0135354289435399\\
486	0.0135349376054903\\
487	0.0135344184725542\\
488	0.013533855475381\\
489	0.0135328021465961\\
490	0.0135315156077701\\
491	0.0135301983322477\\
492	0.0135288489897223\\
493	0.01352746616148\\
494	0.013526048305324\\
495	0.0135245937549364\\
496	0.0135231006998862\\
497	0.0135215672633874\\
498	0.0135199914415026\\
499	0.0135183708444663\\
500	0.0135167015480944\\
501	0.013514977877994\\
502	0.0135131933366922\\
503	0.0135113290126056\\
504	0.0135093670876991\\
505	0.013507331459421\\
506	0.0135052119835207\\
507	0.0135029873558751\\
508	0.0135001802069156\\
509	0.0134959150840131\\
510	0.0134915597262709\\
511	0.0134871103269274\\
512	0.0134825626886634\\
513	0.0134779115567128\\
514	0.0134731517082004\\
515	0.0134682794976019\\
516	0.0134632887333625\\
517	0.0134581716044498\\
518	0.013452920103399\\
519	0.0134475235795158\\
520	0.0134419283221539\\
521	0.0134361529326394\\
522	0.0134302058308984\\
523	0.0134240708131129\\
524	0.0134177107485215\\
525	0.0134111164356789\\
526	0.0134042736876343\\
527	0.0133971492022143\\
528	0.0133892336723207\\
529	0.0133807705780515\\
530	0.0133718730784398\\
531	0.0133561714480059\\
532	0.0133398458429537\\
533	0.0133223131893943\\
534	0.0133033809070622\\
535	0.0132839320854405\\
536	0.0132639373003346\\
537	0.0132433638151846\\
538	0.0132221737264103\\
539	0.0132003183119116\\
540	0.013177717456573\\
541	0.0131521697522508\\
542	0.0131244708335213\\
543	0.0130959542172282\\
544	0.013066498295731\\
545	0.0130360408378794\\
546	0.0130045762794272\\
547	0.0129720461920879\\
548	0.0129383631174359\\
549	0.012903431437999\\
550	0.012867149193955\\
551	0.0128294058948418\\
552	0.0127883021395426\\
553	0.012736775160571\\
554	0.0126833866717845\\
555	0.012627347866077\\
556	0.0125409032726753\\
557	0.0124521488577097\\
558	0.0123609066199172\\
559	0.0122670518760954\\
560	0.0121703879470922\\
561	0.0120706871507593\\
562	0.011967664038773\\
563	0.0118611178333308\\
564	0.0117507628222118\\
565	0.0116369864425952\\
566	0.0115243134277903\\
567	0.011407630437138\\
568	0.0112598262504462\\
569	0.0111071282536938\\
570	0.0109509091848256\\
571	0.0107908752347999\\
572	0.010631835805226\\
573	0.0104904803742406\\
574	0.0103483159232565\\
575	0.0102019742201461\\
576	0.0100513986648676\\
577	0.00989680073265635\\
578	0.00973906102328593\\
579	0.00957582759199066\\
580	0.00942261438354543\\
581	0.00932261027378465\\
582	0.00921979343461604\\
583	0.00911412364342678\\
584	0.00900557245397146\\
585	0.00889412941968206\\
586	0.0087797599448596\\
587	0.00866252558838551\\
588	0.00854253765219076\\
589	0.00841997343584366\\
590	0.00829509753613871\\
591	0.00816828987892658\\
592	0.00804008534578783\\
593	0.00791122806731336\\
594	0.00778275427780483\\
595	0.00756131189408085\\
596	0.00719876254233496\\
597	0.00551136382163314\\
598	0.00357511483354343\\
599	0\\
600	0\\
};
\addplot [color=mycolor12,solid,forget plot]
  table[row sep=crcr]{%
1	0.0135729281156313\\
2	0.0135729281149894\\
3	0.0135729281143365\\
4	0.0135729281136724\\
5	0.013572928112997\\
6	0.01357292811231\\
7	0.0135729281116113\\
8	0.0135729281109005\\
9	0.0135729281101777\\
10	0.0135729281094424\\
11	0.0135729281086946\\
12	0.0135729281079339\\
13	0.0135729281071602\\
14	0.0135729281063733\\
15	0.0135729281055729\\
16	0.0135729281047588\\
17	0.0135729281039308\\
18	0.0135729281030885\\
19	0.0135729281022319\\
20	0.0135729281013606\\
21	0.0135729281004743\\
22	0.0135729280995729\\
23	0.0135729280986561\\
24	0.0135729280977235\\
25	0.013572928096775\\
26	0.0135729280958102\\
27	0.0135729280948289\\
28	0.0135729280938308\\
29	0.0135729280928156\\
30	0.013572928091783\\
31	0.0135729280907327\\
32	0.0135729280896644\\
33	0.0135729280885778\\
34	0.0135729280874727\\
35	0.0135729280863485\\
36	0.0135729280852052\\
37	0.0135729280840422\\
38	0.0135729280828593\\
39	0.0135729280816562\\
40	0.0135729280804325\\
41	0.0135729280791877\\
42	0.0135729280779217\\
43	0.013572928076634\\
44	0.0135729280753242\\
45	0.013572928073992\\
46	0.013572928072637\\
47	0.0135729280712587\\
48	0.0135729280698569\\
49	0.013572928068431\\
50	0.0135729280669808\\
51	0.0135729280655056\\
52	0.0135729280640052\\
53	0.0135729280624792\\
54	0.0135729280609269\\
55	0.0135729280593481\\
56	0.0135729280577423\\
57	0.0135729280561089\\
58	0.0135729280544476\\
59	0.0135729280527578\\
60	0.013572928051039\\
61	0.0135729280492909\\
62	0.0135729280475128\\
63	0.0135729280457042\\
64	0.0135729280438647\\
65	0.0135729280419937\\
66	0.0135729280400906\\
67	0.013572928038155\\
68	0.0135729280361862\\
69	0.0135729280341837\\
70	0.0135729280321469\\
71	0.0135729280300752\\
72	0.0135729280279681\\
73	0.0135729280258249\\
74	0.013572928023645\\
75	0.0135729280214278\\
76	0.0135729280191726\\
77	0.0135729280168788\\
78	0.0135729280145458\\
79	0.0135729280121728\\
80	0.0135729280097592\\
81	0.0135729280073044\\
82	0.0135729280048074\\
83	0.0135729280022678\\
84	0.0135729279996847\\
85	0.0135729279970574\\
86	0.0135729279943851\\
87	0.0135729279916671\\
88	0.0135729279889026\\
89	0.0135729279860908\\
90	0.0135729279832309\\
91	0.0135729279803221\\
92	0.0135729279773635\\
93	0.0135729279743544\\
94	0.0135729279712937\\
95	0.0135729279681807\\
96	0.0135729279650144\\
97	0.013572927961794\\
98	0.0135729279585186\\
99	0.0135729279551871\\
100	0.0135729279517986\\
101	0.0135729279483523\\
102	0.013572927944847\\
103	0.0135729279412817\\
104	0.0135729279376556\\
105	0.0135729279339674\\
106	0.0135729279302162\\
107	0.0135729279264009\\
108	0.0135729279225204\\
109	0.0135729279185736\\
110	0.0135729279145594\\
111	0.0135729279104765\\
112	0.0135729279063239\\
113	0.0135729279021004\\
114	0.0135729278978047\\
115	0.0135729278934356\\
116	0.0135729278889919\\
117	0.0135729278844723\\
118	0.0135729278798756\\
119	0.0135729278752003\\
120	0.0135729278704452\\
121	0.0135729278656089\\
122	0.0135729278606901\\
123	0.0135729278556873\\
124	0.0135729278505991\\
125	0.0135729278454241\\
126	0.0135729278401607\\
127	0.0135729278348075\\
128	0.013572927829363\\
129	0.0135729278238256\\
130	0.0135729278181937\\
131	0.0135729278124657\\
132	0.0135729278066401\\
133	0.013572927800715\\
134	0.0135729277946889\\
135	0.0135729277885601\\
136	0.0135729277823267\\
137	0.0135729277759871\\
138	0.0135729277695394\\
139	0.0135729277629817\\
140	0.0135729277563123\\
141	0.0135729277495293\\
142	0.0135729277426306\\
143	0.0135729277356143\\
144	0.0135729277284785\\
145	0.0135729277212212\\
146	0.0135729277138402\\
147	0.0135729277063334\\
148	0.0135729276986988\\
149	0.0135729276909342\\
150	0.0135729276830373\\
151	0.013572927675006\\
152	0.0135729276668379\\
153	0.0135729276585308\\
154	0.0135729276500823\\
155	0.0135729276414899\\
156	0.0135729276327513\\
157	0.013572927623864\\
158	0.0135729276148255\\
159	0.0135729276056332\\
160	0.0135729275962845\\
161	0.0135729275867767\\
162	0.0135729275771073\\
163	0.0135729275672733\\
164	0.0135729275572722\\
165	0.0135729275471009\\
166	0.0135729275367568\\
167	0.0135729275262367\\
168	0.0135729275155379\\
169	0.0135729275046571\\
170	0.0135729274935915\\
171	0.0135729274823377\\
172	0.0135729274708928\\
173	0.0135729274592534\\
174	0.0135729274474162\\
175	0.013572927435378\\
176	0.0135729274231353\\
177	0.0135729274106846\\
178	0.0135729273980225\\
179	0.0135729273851454\\
180	0.0135729273720496\\
181	0.0135729273587315\\
182	0.0135729273451874\\
183	0.0135729273314133\\
184	0.0135729273174055\\
185	0.0135729273031599\\
186	0.0135729272886726\\
187	0.0135729272739395\\
188	0.0135729272589564\\
189	0.0135729272437192\\
190	0.0135729272282235\\
191	0.0135729272124649\\
192	0.0135729271964392\\
193	0.0135729271801417\\
194	0.0135729271635678\\
195	0.0135729271467129\\
196	0.0135729271295723\\
197	0.0135729271121412\\
198	0.0135729270944146\\
199	0.0135729270763876\\
200	0.0135729270580551\\
201	0.0135729270394119\\
202	0.0135729270204529\\
203	0.0135729270011727\\
204	0.0135729269815659\\
205	0.0135729269616269\\
206	0.0135729269413503\\
207	0.0135729269207302\\
208	0.0135729268997609\\
209	0.0135729268784366\\
210	0.0135729268567512\\
211	0.0135729268346987\\
212	0.0135729268122728\\
213	0.0135729267894674\\
214	0.0135729267662759\\
215	0.0135729267426919\\
216	0.0135729267187087\\
217	0.0135729266943197\\
218	0.013572926669518\\
219	0.0135729266442966\\
220	0.0135729266186486\\
221	0.0135729265925665\\
222	0.0135729265660433\\
223	0.0135729265390714\\
224	0.0135729265116432\\
225	0.0135729264837511\\
226	0.0135729264553872\\
227	0.0135729264265437\\
228	0.0135729263972123\\
229	0.0135729263673849\\
230	0.0135729263370531\\
231	0.0135729263062084\\
232	0.0135729262748421\\
233	0.0135729262429455\\
234	0.0135729262105096\\
235	0.0135729261775252\\
236	0.0135729261439831\\
237	0.013572926109874\\
238	0.0135729260751882\\
239	0.0135729260399159\\
240	0.0135729260040474\\
241	0.0135729259675724\\
242	0.0135729259304807\\
243	0.013572925892762\\
244	0.0135729258544056\\
245	0.0135729258154006\\
246	0.0135729257757362\\
247	0.0135729257354012\\
248	0.0135729256943842\\
249	0.0135729256526736\\
250	0.0135729256102578\\
251	0.0135729255671247\\
252	0.0135729255232622\\
253	0.0135729254786578\\
254	0.0135729254332991\\
255	0.0135729253871732\\
256	0.0135729253402671\\
257	0.0135729252925674\\
258	0.0135729252440608\\
259	0.0135729251947335\\
260	0.0135729251445715\\
261	0.0135729250935605\\
262	0.0135729250416862\\
263	0.0135729249889338\\
264	0.0135729249352883\\
265	0.0135729248807345\\
266	0.0135729248252569\\
267	0.0135729247688396\\
268	0.0135729247114666\\
269	0.0135729246531215\\
270	0.0135729245937876\\
271	0.013572924533448\\
272	0.0135729244720855\\
273	0.0135729244096825\\
274	0.013572924346221\\
275	0.0135729242816829\\
276	0.0135729242160496\\
277	0.0135729241493022\\
278	0.0135729240814215\\
279	0.0135729240123878\\
280	0.0135729239421813\\
281	0.0135729238707816\\
282	0.013572923798168\\
283	0.0135729237243194\\
284	0.0135729236492143\\
285	0.0135729235728309\\
286	0.0135729234951468\\
287	0.0135729234161392\\
288	0.013572923335785\\
289	0.0135729232540607\\
290	0.013572923170942\\
291	0.0135729230864045\\
292	0.0135729230004232\\
293	0.0135729229129724\\
294	0.0135729228240262\\
295	0.0135729227335581\\
296	0.0135729226415408\\
297	0.0135729225479469\\
298	0.013572922452748\\
299	0.0135729223559154\\
300	0.0135729222574197\\
301	0.0135729221572309\\
302	0.0135729220553183\\
303	0.0135729219516506\\
304	0.0135729218461959\\
305	0.0135729217389213\\
306	0.0135729216297936\\
307	0.0135729215187785\\
308	0.0135729214058411\\
309	0.0135729212909455\\
310	0.0135729211740552\\
311	0.0135729210551326\\
312	0.0135729209341393\\
313	0.013572920811036\\
314	0.0135729206857825\\
315	0.0135729205583373\\
316	0.0135729204286579\\
317	0.013572920296701\\
318	0.0135729201624217\\
319	0.013572920025774\\
320	0.0135729198867107\\
321	0.0135729197451831\\
322	0.0135729196011413\\
323	0.0135729194545334\\
324	0.0135729193053064\\
325	0.0135729191534053\\
326	0.0135729189987732\\
327	0.0135729188413517\\
328	0.0135729186810798\\
329	0.0135729185178947\\
330	0.013572918351731\\
331	0.013572918182521\\
332	0.013572918010194\\
333	0.0135729178346768\\
334	0.0135729176558925\\
335	0.0135729174737614\\
336	0.0135729172881998\\
337	0.0135729170991199\\
338	0.0135729169064299\\
339	0.0135729167100331\\
340	0.0135729165098276\\
341	0.0135729163057062\\
342	0.0135729160975551\\
343	0.0135729158852543\\
344	0.013572915668676\\
345	0.0135729154476844\\
346	0.0135729152221349\\
347	0.0135729149918729\\
348	0.0135729147567329\\
349	0.0135729145165372\\
350	0.0135729142710951\\
351	0.0135729140202008\\
352	0.013572913763632\\
353	0.0135729135011484\\
354	0.0135729132324889\\
355	0.0135729129573694\\
356	0.0135729126754785\\
357	0.0135729123864727\\
358	0.013572912089966\\
359	0.0135729117855089\\
360	0.0135729114725381\\
361	0.0135729111502544\\
362	0.0135729108173162\\
363	0.0135729104710922\\
364	0.0135729101059222\\
365	0.0135729097095322\\
366	0.0135729092580673\\
367	0.0135729087233709\\
368	0.0135729081749621\\
369	0.0135729076179268\\
370	0.0135729070521349\\
371	0.0135729064774525\\
372	0.0135729058937422\\
373	0.0135729053008627\\
374	0.0135729046986732\\
375	0.0135729040870432\\
376	0.0135729034658593\\
377	0.0135729028349909\\
378	0.0135729021942936\\
379	0.0135729015436208\\
380	0.0135729008828234\\
381	0.0135729002117499\\
382	0.0135728995302461\\
383	0.0135728988381551\\
384	0.0135728981353166\\
385	0.013572897421567\\
386	0.0135728966967393\\
387	0.0135728959606623\\
388	0.0135728952131603\\
389	0.0135728944540529\\
390	0.0135728936831546\\
391	0.0135728929002735\\
392	0.0135728921052117\\
393	0.013572891297764\\
394	0.0135728904777176\\
395	0.0135728896448513\\
396	0.0135728887989359\\
397	0.0135728879397342\\
398	0.0135728870670026\\
399	0.0135728861804934\\
400	0.0135728852799571\\
401	0.013572884365139\\
402	0.0135728834357633\\
403	0.0135728824915085\\
404	0.0135728815320447\\
405	0.0135728805572851\\
406	0.0135728795674373\\
407	0.013572878563061\\
408	0.0135728775447914\\
409	0.013572876511889\\
410	0.0135728754594386\\
411	0.0135728743825948\\
412	0.013572873284697\\
413	0.013572872167358\\
414	0.0135728710301495\\
415	0.0135728698726291\\
416	0.0135728686943397\\
417	0.0135728674948077\\
418	0.0135728662735396\\
419	0.0135728650300171\\
420	0.0135728637636954\\
421	0.0135728624740132\\
422	0.0135728611603951\\
423	0.0135728598222425\\
424	0.0135728584589321\\
425	0.0135728570698137\\
426	0.013572855654209\\
427	0.0135728542114096\\
428	0.0135728527406747\\
429	0.0135728512412291\\
430	0.0135728497122602\\
431	0.0135728481529158\\
432	0.0135728465623006\\
433	0.0135728449394728\\
434	0.0135728432834402\\
435	0.0135728415931554\\
436	0.0135728398675097\\
437	0.0135728381053252\\
438	0.0135728363053431\\
439	0.0135728344662056\\
440	0.0135728325864286\\
441	0.0135728306643691\\
442	0.01357282869821\\
443	0.0135728266860471\\
444	0.0135728246262101\\
445	0.0135728225176014\\
446	0.0135728203581798\\
447	0.0135728181456281\\
448	0.0135728158773625\\
449	0.0135728135504532\\
450	0.0135728111613897\\
451	0.0135728087054866\\
452	0.0135728061753415\\
453	0.0135728035568695\\
454	0.013572800819553\\
455	0.0135727978949762\\
456	0.0135727946434485\\
457	0.0135727908854017\\
458	0.0135727869831818\\
459	0.0135727830092267\\
460	0.0135727789551242\\
461	0.0135727748071108\\
462	0.0135727705400786\\
463	0.0135727661129216\\
464	0.0135727614816987\\
465	0.0135727564885345\\
466	0.0135727505928484\\
467	0.0135727425811198\\
468	0.0135727336577712\\
469	0.0135727245843131\\
470	0.0135727153541195\\
471	0.0135727059607591\\
472	0.0135726963991296\\
473	0.0135726866678476\\
474	0.0135726767733334\\
475	0.0135726667338239\\
476	0.0135726565702078\\
477	0.0135726462369761\\
478	0.0135726316484591\\
479	0.013572616457069\\
480	0.0135726008475058\\
481	0.0135725847901606\\
482	0.0135725682465537\\
483	0.0135725511582732\\
484	0.0135725334176537\\
485	0.0135725147868961\\
486	0.0135724946782544\\
487	0.0135724715898321\\
488	0.0135724421276752\\
489	0.0135724098707835\\
490	0.0135723769809819\\
491	0.0135723434359845\\
492	0.0135723092149093\\
493	0.0135722742954621\\
494	0.0135722386584797\\
495	0.0135722022929575\\
496	0.0135721652038956\\
497	0.013572127416783\\
498	0.0135720889542622\\
499	0.0135720496660699\\
500	0.013572009200884\\
501	0.0135719672662555\\
502	0.0135719232805466\\
503	0.0135718691154561\\
504	0.0135718027415396\\
505	0.0135717334240164\\
506	0.0135716594194468\\
507	0.0135715758487889\\
508	0.0135714800971319\\
509	0.0135713830638696\\
510	0.0135712847109463\\
511	0.0135711849868987\\
512	0.0135710837974973\\
513	0.0135709809806871\\
514	0.0135708767134498\\
515	0.0135707709678568\\
516	0.0135706635658289\\
517	0.0135705544398054\\
518	0.0135704435231976\\
519	0.013570330716213\\
520	0.0135701783501662\\
521	0.0135700082626471\\
522	0.0135698353412184\\
523	0.0135696560324198\\
524	0.0135694707937673\\
525	0.0135692794948472\\
526	0.0135690791426417\\
527	0.0135688583052168\\
528	0.0135682154156232\\
529	0.0135673676922128\\
530	0.0135664530795002\\
531	0.0135654963789684\\
532	0.0135644778383099\\
533	0.0135627961861763\\
534	0.0135603011073321\\
535	0.0135577240095517\\
536	0.0135550577791303\\
537	0.0135522940269132\\
538	0.0135494221043857\\
539	0.0135464256209235\\
540	0.0135432666778108\\
541	0.0135381572659128\\
542	0.0135317392269093\\
543	0.0135251269767831\\
544	0.0135183081130177\\
545	0.0135112748453141\\
546	0.0135040125906924\\
547	0.0134965027045445\\
548	0.0134887245497606\\
549	0.0134806554663675\\
550	0.013472269619963\\
551	0.0134635274601709\\
552	0.0134528847211895\\
553	0.0134345591688223\\
554	0.0134156886229345\\
555	0.013396223717803\\
556	0.013376162949696\\
557	0.0133554547659071\\
558	0.0133340370095081\\
559	0.0133118730060486\\
560	0.0132888952198527\\
561	0.0132650238382838\\
562	0.0132401548743918\\
563	0.0132142247716468\\
564	0.0131871345066033\\
565	0.0131583158948313\\
566	0.0131248268231682\\
567	0.013088700249165\\
568	0.0130166689939226\\
569	0.0129412540129407\\
570	0.0128635173667425\\
571	0.0127832117855989\\
572	0.0126974946905088\\
573	0.012598252501507\\
574	0.0124957378383103\\
575	0.0123896817220364\\
576	0.0122798443130587\\
577	0.0121659373337167\\
578	0.0120482103024803\\
579	0.0119252550014735\\
580	0.0117882857577192\\
581	0.0116159413948298\\
582	0.0114382083633602\\
583	0.0112541254279651\\
584	0.0110633882730042\\
585	0.0108767719528068\\
586	0.0106827430451281\\
587	0.0104803507500256\\
588	0.0102688047348806\\
589	0.0100471280721329\\
590	0.00981391836833831\\
591	0.00956853936323224\\
592	0.00930998247240804\\
593	0.00903672163075643\\
594	0.00874682866488133\\
595	0.00843773941577757\\
596	0.00810565734955028\\
597	0.00735965377900303\\
598	0.00357511483354343\\
599	0\\
600	0\\
};
\addplot [color=mycolor13,solid,forget plot]
  table[row sep=crcr]{%
1	0.00626656850273689\\
2	0.00626656850273689\\
3	0.00626656850273689\\
4	0.00626656850273689\\
5	0.00626656850273689\\
6	0.00626656850273689\\
7	0.00626656850273689\\
8	0.00626656850273689\\
9	0.00626656850273689\\
10	0.00626656850273689\\
11	0.00626656850273689\\
12	0.00626656850273689\\
13	0.00626656850273689\\
14	0.00626656850273689\\
15	0.00626656850273689\\
16	0.00626656850273689\\
17	0.00626656850273689\\
18	0.00626656850273689\\
19	0.00626656850273689\\
20	0.00626656850273689\\
21	0.00626656850273689\\
22	0.00626656850273689\\
23	0.00626656850273689\\
24	0.00626656850273689\\
25	0.00626656850273689\\
26	0.00626656850273689\\
27	0.00626656850273689\\
28	0.00626656850273689\\
29	0.00626656850273689\\
30	0.00626656850273689\\
31	0.00626656850273689\\
32	0.00626656850273689\\
33	0.00626656850273689\\
34	0.00626656850273689\\
35	0.00626656850273689\\
36	0.00626656850273689\\
37	0.00626656850273689\\
38	0.00626656850273689\\
39	0.00626656850273689\\
40	0.00626656850273689\\
41	0.00626656850273689\\
42	0.00626656850273689\\
43	0.00626656850273689\\
44	0.00626656850273689\\
45	0.00626656850273689\\
46	0.00626656850273689\\
47	0.00626656850273689\\
48	0.00626656850273689\\
49	0.00626656850273689\\
50	0.00626656850273689\\
51	0.00626656850273689\\
52	0.00626656850273689\\
53	0.00626656850273689\\
54	0.00626656850273689\\
55	0.00626656850273689\\
56	0.00626656850273689\\
57	0.00626656850273689\\
58	0.00626656850273689\\
59	0.00626656850273689\\
60	0.00626656850273689\\
61	0.00626656850273689\\
62	0.00626656850273689\\
63	0.00626656850273689\\
64	0.00626656850273689\\
65	0.00626656850273689\\
66	0.00626656850273689\\
67	0.00626656850273689\\
68	0.00626656850273689\\
69	0.00626656850273689\\
70	0.00626656850273689\\
71	0.00626656850273689\\
72	0.00626656850273689\\
73	0.00626656850273689\\
74	0.00626656850273689\\
75	0.00626656850273689\\
76	0.00626656850273689\\
77	0.00626656850273689\\
78	0.00626656850273689\\
79	0.00626656850273689\\
80	0.00626656850273689\\
81	0.00626656850273689\\
82	0.00626656850273689\\
83	0.00626656850273689\\
84	0.00626656850273689\\
85	0.00626656850273689\\
86	0.00626656850273689\\
87	0.00626656850273689\\
88	0.00626656850273689\\
89	0.00626656850273689\\
90	0.00626656850273689\\
91	0.00626656850273689\\
92	0.00626656850273689\\
93	0.00626656850273689\\
94	0.00626656850273689\\
95	0.00626656850273689\\
96	0.00626656850273689\\
97	0.00626656850273689\\
98	0.00626656850273689\\
99	0.00626656850273689\\
100	0.00626656850273689\\
101	0.00626656850273689\\
102	0.00626656850273689\\
103	0.00626656850273689\\
104	0.00626656850273689\\
105	0.00626656850273689\\
106	0.00626656850273689\\
107	0.00626656850273689\\
108	0.00626656850273689\\
109	0.00626656850273689\\
110	0.00626656850273689\\
111	0.00626656850273689\\
112	0.00626656850273689\\
113	0.00626656850273689\\
114	0.00626656850273689\\
115	0.00626656850273689\\
116	0.00626656850273689\\
117	0.00626656850273689\\
118	0.00626656850273689\\
119	0.00626656850273689\\
120	0.00626656850273689\\
121	0.00626656850273689\\
122	0.00626656850273689\\
123	0.00626656850273689\\
124	0.00626656850273689\\
125	0.00626656850273689\\
126	0.00626656850273689\\
127	0.00626656850273689\\
128	0.00626656850273689\\
129	0.00626656850273689\\
130	0.00626656850273689\\
131	0.00626656850273689\\
132	0.00626656850273689\\
133	0.00626656850273689\\
134	0.00626656850273689\\
135	0.00626656850273689\\
136	0.00626656850273689\\
137	0.00626656850273689\\
138	0.00626656850273689\\
139	0.00626656850273689\\
140	0.00626656850273689\\
141	0.00626656850273689\\
142	0.00626656850273689\\
143	0.00626656850273689\\
144	0.00626656850273689\\
145	0.00626656850273689\\
146	0.00626656850273689\\
147	0.00626656850273689\\
148	0.00626656850273689\\
149	0.00626656850273689\\
150	0.00626656850273689\\
151	0.00626656850273689\\
152	0.00626656850273689\\
153	0.00626656850273689\\
154	0.00626656850273689\\
155	0.00626656850273689\\
156	0.00626656850273689\\
157	0.00626656850273689\\
158	0.00626656850273689\\
159	0.00626656850273689\\
160	0.00626656850273689\\
161	0.00626656850273689\\
162	0.00626656850273689\\
163	0.00626656850273689\\
164	0.00626656850273689\\
165	0.00626656850273689\\
166	0.00626656850273689\\
167	0.00626656850273689\\
168	0.00626656850273689\\
169	0.00626656850273689\\
170	0.00626656850273689\\
171	0.00626656850273689\\
172	0.00626656850273689\\
173	0.00626656850273689\\
174	0.00626656850273689\\
175	0.00626656850273689\\
176	0.00626656850273689\\
177	0.00626656850273689\\
178	0.00626656850273689\\
179	0.00626656850273689\\
180	0.00626656850273689\\
181	0.00626656850273689\\
182	0.00626656850273689\\
183	0.00626656850273689\\
184	0.00626656850273689\\
185	0.00626656850273689\\
186	0.00626656850273689\\
187	0.00626656850273689\\
188	0.00626656850273689\\
189	0.00626656850273689\\
190	0.00626656850273689\\
191	0.00626656850273689\\
192	0.00626656850273689\\
193	0.00626656850273689\\
194	0.00626656850273689\\
195	0.00626656850273689\\
196	0.00626656850273689\\
197	0.00626656850273689\\
198	0.00626656850273689\\
199	0.00626656850273689\\
200	0.00626656850273689\\
201	0.00626656850273689\\
202	0.00626656850273689\\
203	0.00626656850273689\\
204	0.00626656850273689\\
205	0.00626656850273689\\
206	0.00626656850273689\\
207	0.00626656850273689\\
208	0.00626656850273689\\
209	0.00626656850273689\\
210	0.00626656850273689\\
211	0.00626656850273689\\
212	0.00626656850273689\\
213	0.00626656850273689\\
214	0.00626656850273689\\
215	0.00626656850273689\\
216	0.00626656850273689\\
217	0.00626656850273689\\
218	0.00626656850273689\\
219	0.00626656850273689\\
220	0.00626656850273689\\
221	0.00626656850273689\\
222	0.00626656850273689\\
223	0.00626656850273689\\
224	0.00626656850273689\\
225	0.00626656850273689\\
226	0.00626656850273689\\
227	0.00626656850273689\\
228	0.00626656850273689\\
229	0.00626656850273689\\
230	0.00626656850273689\\
231	0.00626656850273689\\
232	0.00626656850273689\\
233	0.00626656850273689\\
234	0.00626656850273689\\
235	0.00626656850273689\\
236	0.00626656850273689\\
237	0.00626656850273689\\
238	0.00626656850273689\\
239	0.00626656850273689\\
240	0.00626656850273689\\
241	0.00626656850273689\\
242	0.00626656850273689\\
243	0.00626656850273689\\
244	0.00626656850273689\\
245	0.00626656850273689\\
246	0.00626656850273689\\
247	0.00626656850273689\\
248	0.00626656850273689\\
249	0.00626656850273689\\
250	0.00626656850273689\\
251	0.00626656850273689\\
252	0.00626656850273689\\
253	0.00626656850273689\\
254	0.00626656850273689\\
255	0.00626656850273689\\
256	0.00626656850273689\\
257	0.00626656850273689\\
258	0.00626656850273689\\
259	0.00626656850273689\\
260	0.00626656850273689\\
261	0.00626656850273689\\
262	0.00626656850273689\\
263	0.00626656850273689\\
264	0.00626656850273689\\
265	0.00626656850273689\\
266	0.00626656850273689\\
267	0.00626656850273689\\
268	0.00626656850273689\\
269	0.00626656850273689\\
270	0.00626656850273689\\
271	0.00626656850273689\\
272	0.00626656850273689\\
273	0.00626656850273689\\
274	0.00626656850273689\\
275	0.00626656850273689\\
276	0.00626656850273689\\
277	0.00626656850273689\\
278	0.00626656850273689\\
279	0.00626656850273689\\
280	0.00626656850273689\\
281	0.00626656850273689\\
282	0.00626656850273689\\
283	0.00626656850273689\\
284	0.00626656850273689\\
285	0.00626656850273689\\
286	0.00626656850273689\\
287	0.00626656850273689\\
288	0.00626656850273689\\
289	0.00626656850273689\\
290	0.00626656850273689\\
291	0.00626656850273689\\
292	0.00626656850273689\\
293	0.00626656850273689\\
294	0.00626656850273689\\
295	0.00626656850273689\\
296	0.00626656850273689\\
297	0.00626656850273689\\
298	0.00626656850273689\\
299	0.00626656850273689\\
300	0.00626656850273689\\
301	0.00626656850273689\\
302	0.00626656850273689\\
303	0.00626656850273689\\
304	0.00626656850273689\\
305	0.00626656850273689\\
306	0.00626656850273689\\
307	0.00626656850273689\\
308	0.00626656850273689\\
309	0.00626656850273689\\
310	0.00626656850273689\\
311	0.00626656850273689\\
312	0.00626656850273689\\
313	0.00626656850273689\\
314	0.00626656850273689\\
315	0.00626656850273689\\
316	0.00626656850273689\\
317	0.00626656850273689\\
318	0.00626656850273689\\
319	0.00626656850273689\\
320	0.00626656850273689\\
321	0.00626656850273689\\
322	0.00626656850273689\\
323	0.00626656850273689\\
324	0.00626656850273689\\
325	0.00626656850273689\\
326	0.00626656850273689\\
327	0.00626656850273689\\
328	0.00626656850273689\\
329	0.00626656850273689\\
330	0.00626656850273689\\
331	0.00626656850273689\\
332	0.00626656850273689\\
333	0.00626656850273689\\
334	0.00626656850273689\\
335	0.00626656850273689\\
336	0.00626656850273689\\
337	0.00626656850273689\\
338	0.00626656850273689\\
339	0.00626656850273689\\
340	0.00626656850273689\\
341	0.00626656850273689\\
342	0.00626656850273689\\
343	0.00626656850273689\\
344	0.00626656850273689\\
345	0.00626656850273689\\
346	0.00626656850273689\\
347	0.00626656850273689\\
348	0.00626656850273689\\
349	0.00626656850273689\\
350	0.00626656850273689\\
351	0.00626656850273689\\
352	0.00626656850273689\\
353	0.00626656850273689\\
354	0.00626656850273689\\
355	0.00626656850273689\\
356	0.00626656850273689\\
357	0.00626656850273689\\
358	0.00626656850273689\\
359	0.00626656850273689\\
360	0.00626656850273689\\
361	0.00626656850273689\\
362	0.00626656850273689\\
363	0.00626656850273689\\
364	0.00626656850273689\\
365	0.00626656850273689\\
366	0.00626656850273689\\
367	0.00626656850273689\\
368	0.00626656850273689\\
369	0.00626656850273689\\
370	0.00626656850273689\\
371	0.00626656850273689\\
372	0.00626656850273689\\
373	0.00626656850273689\\
374	0.00626656850273689\\
375	0.00626656850273689\\
376	0.00626656850273689\\
377	0.00626656850273689\\
378	0.00626656850273689\\
379	0.00626656850273689\\
380	0.00626656850273689\\
381	0.00626656850273689\\
382	0.00626656850273689\\
383	0.00626656850273689\\
384	0.00626656850273689\\
385	0.00626656850273689\\
386	0.00626656850273689\\
387	0.00626656850273689\\
388	0.00626656850273689\\
389	0.00626656850273689\\
390	0.00626656850273689\\
391	0.00626656850273689\\
392	0.00626656850273689\\
393	0.00626656850273689\\
394	0.00626656850273689\\
395	0.00626656850273689\\
396	0.00626656850273689\\
397	0.00626656850273689\\
398	0.00626656850273689\\
399	0.00626656850273689\\
400	0.00626656850273689\\
401	0.00626656850273689\\
402	0.00626656850273689\\
403	0.00626656850273689\\
404	0.00626656850273689\\
405	0.00626656850273689\\
406	0.00626656850273689\\
407	0.00626656850273689\\
408	0.00626656850273689\\
409	0.00626656850273689\\
410	0.00626656850273689\\
411	0.00626656850273689\\
412	0.00626656850273689\\
413	0.00626656850273689\\
414	0.00626656850273689\\
415	0.00626656850273689\\
416	0.00626656850273689\\
417	0.00626656850273689\\
418	0.00626656850273689\\
419	0.00626656850273689\\
420	0.00626656850273689\\
421	0.00626656850273689\\
422	0.00626656850273689\\
423	0.00626656850273689\\
424	0.00626656850273689\\
425	0.00626656850273689\\
426	0.00626656850273689\\
427	0.00626656850273689\\
428	0.00626656850273689\\
429	0.00626656850273689\\
430	0.00626656850273689\\
431	0.00626656850273689\\
432	0.00626656850273689\\
433	0.00626656850273689\\
434	0.00626656850273689\\
435	0.00626656850273689\\
436	0.00626656850273689\\
437	0.00626656850273689\\
438	0.00626656850273689\\
439	0.00626656850273689\\
440	0.00626656850273689\\
441	0.00626656850273689\\
442	0.00626656850273689\\
443	0.00626656850273689\\
444	0.00626656850273689\\
445	0.00626656850273689\\
446	0.00626656850273689\\
447	0.00626656850273689\\
448	0.00626656850273689\\
449	0.00626656850273689\\
450	0.00626656850273689\\
451	0.00626656850273689\\
452	0.00626656850273689\\
453	0.00626656850273689\\
454	0.00626656850273689\\
455	0.00626656850273689\\
456	0.00626656850273689\\
457	0.00626656850273689\\
458	0.00626656850273689\\
459	0.00626656850273689\\
460	0.00626656850273689\\
461	0.00626656850273689\\
462	0.00626656850273689\\
463	0.00626656850273689\\
464	0.00626656850273689\\
465	0.00626656850273689\\
466	0.00626656850273689\\
467	0.00626656850273689\\
468	0.00626656850273689\\
469	0.00626656850273689\\
470	0.00626656850273689\\
471	0.00626656850273689\\
472	0.00626656850273689\\
473	0.00626656850273689\\
474	0.00626656850273689\\
475	0.00626656850273689\\
476	0.00626656850273689\\
477	0.00626656850273689\\
478	0.00626656850273689\\
479	0.00626656850273689\\
480	0.00626656850273689\\
481	0.00626656850273689\\
482	0.00626656850273689\\
483	0.00626656850273689\\
484	0.00626656850273689\\
485	0.00626656850273689\\
486	0.00626656850273689\\
487	0.00626656850273689\\
488	0.00626656850273689\\
489	0.00626656850273689\\
490	0.00626656850273689\\
491	0.00626656850273689\\
492	0.00626656850273689\\
493	0.00626656850273689\\
494	0.00626656850273689\\
495	0.00626656850273689\\
496	0.00626656850273689\\
497	0.00626656850273689\\
498	0.00626656850273689\\
499	0.00626656850273689\\
500	0.00626656850273689\\
501	0.00626656850273689\\
502	0.00626656850273689\\
503	0.00626656850273689\\
504	0.00626656850273689\\
505	0.00626656850273689\\
506	0.00626656850273689\\
507	0.00626656850273689\\
508	0.00626656850273689\\
509	0.00626656850273689\\
510	0.00626656850273689\\
511	0.00626656850273689\\
512	0.00626656850273689\\
513	0.00626656850273689\\
514	0.00626656850273689\\
515	0.00626656850273689\\
516	0.00626656850273689\\
517	0.00626656850273689\\
518	0.00626656850273689\\
519	0.00626656850273689\\
520	0.00626656850273689\\
521	0.00626656850273689\\
522	0.00626656850273689\\
523	0.00626656850273689\\
524	0.00626656850273689\\
525	0.00626656850273689\\
526	0.00626656850273689\\
527	0.00626656850273689\\
528	0.00626656850273689\\
529	0.00626656850273689\\
530	0.00626656850273689\\
531	0.00626656850273689\\
532	0.00626656850273689\\
533	0.00626656850273689\\
534	0.00626656850273689\\
535	0.00626656850273689\\
536	0.00626656850273689\\
537	0.00626656850273689\\
538	0.00626656850273689\\
539	0.00626656850273689\\
540	0.00626656850273689\\
541	0.00626656850273689\\
542	0.00626656850273689\\
543	0.00626656850273689\\
544	0.00626656850273689\\
545	0.00626656850273689\\
546	0.00626656850273689\\
547	0.00626656850273689\\
548	0.00626656850273689\\
549	0.00626656850273689\\
550	0.00626656850273689\\
551	0.00626656850273689\\
552	0.00626656850273689\\
553	0.00626656850273689\\
554	0.00626656850273689\\
555	0.00626656850273689\\
556	0.00626656850273689\\
557	0.00626656850273689\\
558	0.00626656850273689\\
559	0.00626656850273689\\
560	0.00626656850273689\\
561	0.00626656850273689\\
562	0.00626656850273689\\
563	0.00626656850273689\\
564	0.00626656850273689\\
565	0.0062822687199181\\
566	0.00639112209030057\\
567	0.00650258018340242\\
568	0.00662081992140465\\
569	0.00674295056371734\\
570	0.00686670043208088\\
571	0.0069920636388678\\
572	0.00711629095688794\\
573	0.00724056915523556\\
574	0.00743208496156276\\
575	0.00762095467871718\\
576	0.00780725692531322\\
577	0.00799086507431176\\
578	0.00814787332658741\\
579	0.00830507905952759\\
580	0.0084565674425073\\
581	0.00858561074290109\\
582	0.00871475916838674\\
583	0.00884326036092976\\
584	0.00897128106394158\\
585	0.0090995281368202\\
586	0.00923067908272993\\
587	0.00937316043200138\\
588	0.00952438500706335\\
589	0.00967455307485254\\
590	0.0098231620292525\\
591	0.00997019543938102\\
592	0.0101168435362631\\
593	0.0102672147782407\\
594	0.0104258886577547\\
595	0.0106039016956345\\
596	0.0108298745780857\\
597	0.0111718112421279\\
598	0.0113889057314154\\
599	0\\
600	0\\
};
\addplot [color=mycolor14,solid,forget plot]
  table[row sep=crcr]{%
1	0.00367653017611331\\
2	0.00367653056165089\\
3	0.00367653095416569\\
4	0.00367653135378372\\
5	0.00367653176063332\\
6	0.00367653217484509\\
7	0.00367653259655201\\
8	0.00367653302588946\\
9	0.00367653346299522\\
10	0.0036765339080096\\
11	0.00367653436107539\\
12	0.00367653482233799\\
13	0.00367653529194539\\
14	0.00367653577004825\\
15	0.00367653625679995\\
16	0.00367653675235662\\
17	0.0036765372568772\\
18	0.0036765377705235\\
19	0.00367653829346022\\
20	0.00367653882585503\\
21	0.00367653936787864\\
22	0.0036765399197048\\
23	0.00367654048151039\\
24	0.00367654105347547\\
25	0.00367654163578336\\
26	0.00367654222862064\\
27	0.00367654283217727\\
28	0.00367654344664661\\
29	0.0036765440722255\\
30	0.00367654470911431\\
31	0.00367654535751704\\
32	0.00367654601764131\\
33	0.00367654668969849\\
34	0.00367654737390376\\
35	0.00367654807047615\\
36	0.00367654877963863\\
37	0.00367654950161815\\
38	0.00367655023664578\\
39	0.00367655098495669\\
40	0.00367655174679032\\
41	0.00367655252239035\\
42	0.0036765533120049\\
43	0.00367655411588647\\
44	0.00367655493429216\\
45	0.00367655576748364\\
46	0.00367655661572728\\
47	0.00367655747929424\\
48	0.00367655835846054\\
49	0.00367655925350714\\
50	0.00367656016472004\\
51	0.00367656109239038\\
52	0.00367656203681451\\
53	0.00367656299829409\\
54	0.0036765639771362\\
55	0.0036765649736534\\
56	0.00367656598816389\\
57	0.00367656702099152\\
58	0.00367656807246598\\
59	0.00367656914292285\\
60	0.00367657023270374\\
61	0.00367657134215635\\
62	0.00367657247163462\\
63	0.00367657362149883\\
64	0.00367657479211571\\
65	0.00367657598385855\\
66	0.00367657719710732\\
67	0.0036765784322488\\
68	0.0036765796896767\\
69	0.00367658096979173\\
70	0.00367658227300182\\
71	0.00367658359972218\\
72	0.00367658495037544\\
73	0.00367658632539179\\
74	0.00367658772520912\\
75	0.00367658915027315\\
76	0.00367659060103755\\
77	0.00367659207796413\\
78	0.00367659358152294\\
79	0.00367659511219242\\
80	0.00367659667045958\\
81	0.00367659825682013\\
82	0.00367659987177861\\
83	0.0036766015158486\\
84	0.00367660318955285\\
85	0.00367660489342344\\
86	0.00367660662800196\\
87	0.00367660839383966\\
88	0.00367661019149763\\
89	0.00367661202154699\\
90	0.00367661388456905\\
91	0.00367661578115548\\
92	0.00367661771190854\\
93	0.00367661967744119\\
94	0.00367662167837737\\
95	0.00367662371535214\\
96	0.00367662578901186\\
97	0.00367662790001445\\
98	0.00367663004902955\\
99	0.00367663223673873\\
100	0.00367663446383574\\
101	0.00367663673102667\\
102	0.0036766390390302\\
103	0.00367664138857783\\
104	0.00367664378041408\\
105	0.00367664621529675\\
106	0.00367664869399714\\
107	0.00367665121730028\\
108	0.00367665378600518\\
109	0.00367665640092511\\
110	0.00367665906288778\\
111	0.00367666177273567\\
112	0.00367666453132624\\
113	0.00367666733953224\\
114	0.00367667019824192\\
115	0.00367667310835936\\
116	0.00367667607080472\\
117	0.00367667908651455\\
118	0.00367668215644204\\
119	0.00367668528155735\\
120	0.00367668846284789\\
121	0.00367669170131865\\
122	0.00367669499799247\\
123	0.00367669835391038\\
124	0.00367670177013193\\
125	0.00367670524773551\\
126	0.00367670878781866\\
127	0.00367671239149845\\
128	0.00367671605991178\\
129	0.00367671979421577\\
130	0.00367672359558811\\
131	0.00367672746522738\\
132	0.00367673140435348\\
133	0.00367673541420797\\
134	0.00367673949605447\\
135	0.00367674365117902\\
136	0.00367674788089052\\
137	0.00367675218652109\\
138	0.00367675656942653\\
139	0.00367676103098667\\
140	0.00367676557260587\\
141	0.00367677019571339\\
142	0.00367677490176387\\
143	0.00367677969223775\\
144	0.00367678456864176\\
145	0.00367678953250935\\
146	0.00367679458540116\\
147	0.00367679972890555\\
148	0.00367680496463902\\
149	0.00367681029424676\\
150	0.00367681571940314\\
151	0.00367682124181221\\
152	0.00367682686320823\\
153	0.00367683258535624\\
154	0.00367683841005255\\
155	0.00367684433912532\\
156	0.00367685037443511\\
157	0.00367685651787547\\
158	0.0036768627713735\\
159	0.00367686913689048\\
160	0.00367687561642242\\
161	0.00367688221200071\\
162	0.00367688892569274\\
163	0.00367689575960252\\
164	0.00367690271587136\\
165	0.00367690979667849\\
166	0.00367691700424174\\
167	0.00367692434081824\\
168	0.00367693180870512\\
169	0.00367693941024017\\
170	0.00367694714780259\\
171	0.00367695502381376\\
172	0.00367696304073789\\
173	0.00367697120108288\\
174	0.00367697950740102\\
175	0.00367698796228979\\
176	0.0036769965683927\\
177	0.00367700532840006\\
178	0.00367701424504982\\
179	0.00367702332112841\\
180	0.00367703255947162\\
181	0.00367704196296542\\
182	0.00367705153454692\\
183	0.00367706127720524\\
184	0.0036770711939824\\
185	0.00367708128797431\\
186	0.00367709156233171\\
187	0.00367710202026111\\
188	0.00367711266502583\\
189	0.00367712349994695\\
190	0.0036771345284044\\
191	0.00367714575383793\\
192	0.00367715717974825\\
193	0.00367716880969803\\
194	0.00367718064731309\\
195	0.00367719269628342\\
196	0.00367720496036441\\
197	0.00367721744337796\\
198	0.00367723014921369\\
199	0.00367724308183012\\
200	0.0036772562452559\\
201	0.00367726964359109\\
202	0.00367728328100837\\
203	0.0036772971617544\\
204	0.00367731129015109\\
205	0.00367732567059696\\
206	0.00367734030756851\\
207	0.00367735520562158\\
208	0.0036773703693928\\
209	0.00367738580360103\\
210	0.0036774015130488\\
211	0.00367741750262384\\
212	0.00367743377730055\\
213	0.00367745034214161\\
214	0.00367746720229953\\
215	0.00367748436301824\\
216	0.00367750182963476\\
217	0.00367751960758083\\
218	0.00367753770238464\\
219	0.00367755611967255\\
220	0.00367757486517082\\
221	0.00367759394470745\\
222	0.00367761336421396\\
223	0.00367763312972729\\
224	0.00367765324739164\\
225	0.00367767372346047\\
226	0.00367769456429838\\
227	0.00367771577638319\\
228	0.0036777373663079\\
229	0.00367775934078283\\
230	0.00367778170663769\\
231	0.00367780447082373\\
232	0.00367782764041598\\
233	0.00367785122261541\\
234	0.00367787522475125\\
235	0.00367789965428329\\
236	0.00367792451880426\\
237	0.00367794982604216\\
238	0.00367797558386281\\
239	0.00367800180027224\\
240	0.0036780284834193\\
241	0.00367805564159819\\
242	0.00367808328325112\\
243	0.003678111416971\\
244	0.00367814005150414\\
245	0.00367816919575303\\
246	0.00367819885877921\\
247	0.00367822904980608\\
248	0.00367825977822194\\
249	0.00367829105358287\\
250	0.00367832288561585\\
251	0.00367835528422183\\
252	0.00367838825947891\\
253	0.00367842182164553\\
254	0.00367845598116377\\
255	0.00367849074866265\\
256	0.0036785261349616\\
257	0.00367856215107382\\
258	0.00367859880820991\\
259	0.00367863611778135\\
260	0.00367867409140425\\
261	0.00367871274090302\\
262	0.00367875207831416\\
263	0.00367879211589012\\
264	0.00367883286610326\\
265	0.00367887434164981\\
266	0.00367891655545394\\
267	0.00367895952067197\\
268	0.0036790032506965\\
269	0.00367904775916076\\
270	0.0036790930599429\\
271	0.00367913916717048\\
272	0.00367918609522497\\
273	0.00367923385874648\\
274	0.00367928247263846\\
275	0.00367933195207238\\
276	0.0036793823124926\\
277	0.00367943356962131\\
278	0.00367948573946352\\
279	0.00367953883831222\\
280	0.00367959288275351\\
281	0.00367964788967194\\
282	0.00367970387625579\\
283	0.00367976086000261\\
284	0.00367981885872467\\
285	0.00367987789055463\\
286	0.0036799379739512\\
287	0.00367999912770496\\
288	0.0036800613709442\\
289	0.0036801247231409\\
290	0.00368018920411674\\
291	0.0036802548340492\\
292	0.00368032163347778\\
293	0.00368038962331024\\
294	0.00368045882482892\\
295	0.00368052925969719\\
296	0.00368060094996587\\
297	0.00368067391807978\\
298	0.00368074818688432\\
299	0.00368082377963212\\
300	0.00368090071998967\\
301	0.0036809790320441\\
302	0.00368105874030985\\
303	0.00368113986973549\\
304	0.00368122244571041\\
305	0.00368130649407168\\
306	0.00368139204111075\\
307	0.00368147911358031\\
308	0.0036815677387011\\
309	0.00368165794416871\\
310	0.00368174975816006\\
311	0.00368184320933941\\
312	0.00368193832686427\\
313	0.00368203514039226\\
314	0.00368213368008714\\
315	0.00368223397662459\\
316	0.00368233606119779\\
317	0.00368243996552272\\
318	0.00368254572184307\\
319	0.00368265336293489\\
320	0.00368276292211071\\
321	0.00368287443322316\\
322	0.00368298793066811\\
323	0.00368310344938708\\
324	0.00368322102486893\\
325	0.00368334069315084\\
326	0.00368346249081827\\
327	0.00368358645500425\\
328	0.00368371262338791\\
329	0.00368384103419306\\
330	0.00368397172618796\\
331	0.00368410473868894\\
332	0.00368424011157142\\
333	0.00368437788529088\\
334	0.00368451810090468\\
335	0.00368466080004777\\
336	0.00368480602473909\\
337	0.00368495381687434\\
338	0.00368510421777288\\
339	0.00368525726973523\\
340	0.00368541301885574\\
341	0.00368557151543321\\
342	0.00368573281946763\\
343	0.00368589701275041\\
344	0.00368606422228157\\
345	0.00368623465440683\\
346	0.00368640860434799\\
347	0.00368658630102444\\
348	0.00368676741678235\\
349	0.00368695157816927\\
350	0.00368713883560509\\
351	0.00368732924026719\\
352	0.00368752284409484\\
353	0.00368771969979251\\
354	0.0036879198608321\\
355	0.00368812338145575\\
356	0.00368833031668297\\
357	0.00368854072231974\\
358	0.00368875465495341\\
359	0.00368897217195451\\
360	0.00368919333147824\\
361	0.00368941819246552\\
362	0.00368964681464368\\
363	0.00368987925852689\\
364	0.00369011558541633\\
365	0.00369035585740017\\
366	0.00369060013735352\\
367	0.0036908484889384\\
368	0.00369110097660387\\
369	0.00369135766558651\\
370	0.00369161862191167\\
371	0.0036918839123961\\
372	0.00369215360465332\\
373	0.00369242776710292\\
374	0.00369270646898323\\
375	0.00369298978035906\\
376	0.00369327777210676\\
377	0.00369357051587518\\
378	0.00369386808408436\\
379	0.00369417054985217\\
380	0.0036944779868713\\
381	0.00369479046928054\\
382	0.00369510807174475\\
383	0.00369543087032278\\
384	0.00369575894506103\\
385	0.00369609238414613\\
386	0.00369643128324265\\
387	0.00369677572714751\\
388	0.00369712579944964\\
389	0.00369748158607326\\
390	0.00369784317553442\\
391	0.00369821065913121\\
392	0.00369858413089656\\
393	0.00369896368688419\\
394	0.0036993494227463\\
395	0.00369974142716262\\
396	0.00370013976565961\\
397	0.00370054444340995\\
398	0.00370095532587845\\
399	0.00370137198739635\\
400	0.00370179347857104\\
401	0.00370221815351734\\
402	0.00370264418994138\\
403	0.00370307214811652\\
404	0.00370350814092187\\
405	0.00370395231912532\\
406	0.003704404840081\\
407	0.00370486586827905\\
408	0.00370533557584055\\
409	0.00370581414210759\\
410	0.00370630175176115\\
411	0.0037067985954751\\
412	0.00370730488679909\\
413	0.00370782084664511\\
414	0.00370834670366121\\
415	0.00370888269462724\\
416	0.00370942906487551\\
417	0.00370998606874464\\
418	0.0037105539700848\\
419	0.00371113304284162\\
420	0.00371172357172448\\
421	0.00371232585285225\\
422	0.00371294019413211\\
423	0.00371356691562249\\
424	0.00371420635060837\\
425	0.00371485884636767\\
426	0.00371552476499325\\
427	0.00371620448427976\\
428	0.00371689839869196\\
429	0.00371760692044713\\
430	0.00371833048077769\\
431	0.00371906953150134\\
432	0.00371982454712133\\
433	0.00372059602774614\\
434	0.00372138450285962\\
435	0.00372219053442274\\
436	0.00372301471297645\\
437	0.0037238576327883\\
438	0.00372471984803585\\
439	0.00372560194682476\\
440	0.00372650463129142\\
441	0.0037274286407001\\
442	0.00372837475378901\\
443	0.00372934379128035\\
444	0.00373033661855815\\
445	0.00373135414860738\\
446	0.00373239734602282\\
447	0.00373346723597689\\
448	0.00373456492957018\\
449	0.00373569167865208\\
450	0.00373684902585686\\
451	0.00373803921625152\\
452	0.00373926631493424\\
453	0.00374053920392484\\
454	0.00374187948124994\\
455	0.00374334189873179\\
456	0.00374506636686195\\
457	0.00374739713303209\\
458	0.00375123448627122\\
459	0.00375547858829391\\
460	0.00375981539069011\\
461	0.00376426624355029\\
462	0.00376884994748991\\
463	0.00377351286961032\\
464	0.00377825610449833\\
465	0.00378308290804518\\
466	0.00378799698411069\\
467	0.00379300258192583\\
468	0.0037981044831321\\
469	0.00380330734994562\\
470	0.00380861335918627\\
471	0.00381402561349215\\
472	0.00381954488625546\\
473	0.00382516478904953\\
474	0.00383086468800139\\
475	0.00383660418703059\\
476	0.00384234651524792\\
477	0.00384818144141559\\
478	0.00385421363574154\\
479	0.00386045807863611\\
480	0.0038669329136766\\
481	0.00387366255712396\\
482	0.00388068579283263\\
483	0.00388807688206091\\
484	0.00389599855758656\\
485	0.00390481807810079\\
486	0.00391475397534097\\
487	0.00392486785415363\\
488	0.00393516434445483\\
489	0.00394564821273966\\
490	0.00395632447012295\\
491	0.00396719847114864\\
492	0.003978276296484\\
493	0.0039895669600041\\
494	0.004001083731031\\
495	0.00401282028356549\\
496	0.00402478305185202\\
497	0.00403698064252631\\
498	0.00404942382408288\\
499	0.00406212004868167\\
500	0.00407507742575319\\
501	0.00408830519022187\\
502	0.00410181481165771\\
503	0.00411562224974021\\
504	0.00412974941077046\\
505	0.00414419721295197\\
506	0.00415900509793489\\
507	0.0041742699151639\\
508	0.00419019456615506\\
509	0.00420661104572812\\
510	0.00422326330758944\\
511	0.00424026539807649\\
512	0.004257888508676\\
513	0.00427660539481829\\
514	0.00429584792052824\\
515	0.00431525143541864\\
516	0.00433480567499767\\
517	0.00435447933183099\\
518	0.00437418886870854\\
519	0.00439372433973917\\
520	0.00441259873518383\\
521	0.00442986415347352\\
522	0.00444459825959609\\
523	0.00445986664630336\\
524	0.00447619869660579\\
525	0.00449523245035704\\
526	0.00452202013800674\\
527	0.00458256534737671\\
528	0.00464972517505194\\
529	0.00471971230707819\\
530	0.00479299280265683\\
531	0.00487037494914992\\
532	0.00495353421253993\\
533	0.0050506672768503\\
534	0.00515796174870662\\
535	0.00526773094624111\\
536	0.00538009872538758\\
537	0.00549520586632014\\
538	0.00561322008825465\\
539	0.005734377560753\\
540	0.00585911618037913\\
541	0.00598854183724758\\
542	0.00612722598213259\\
543	0.00626810047367561\\
544	0.0064112384690035\\
545	0.00655671821578339\\
546	0.00670462342312696\\
547	0.00685504635426182\\
548	0.00700810064875001\\
549	0.00716396460103585\\
550	0.00732302798753163\\
551	0.00748639253875404\\
552	0.00765758657047315\\
553	0.0078491920627388\\
554	0.0080466620690786\\
555	0.00824698686594431\\
556	0.00845028978904494\\
557	0.00865645048389266\\
558	0.00886543892503706\\
559	0.00907723732830055\\
560	0.00929166693914143\\
561	0.00950801999602768\\
562	0.00972414417303286\\
563	0.00993396175185058\\
564	0.010120992459127\\
565	0.0102532054663847\\
566	0.0103229888747475\\
567	0.0103922951339935\\
568	0.010455303889071\\
569	0.0105180462780884\\
570	0.0105805760211889\\
571	0.0106428416486077\\
572	0.0107042883450541\\
573	0.010763259951551\\
574	0.0108311877217053\\
575	0.0108998445256091\\
576	0.0109692295188531\\
577	0.0110393269431159\\
578	0.0111100636614941\\
579	0.0111814896114134\\
580	0.0112536358673312\\
581	0.011324914505972\\
582	0.0113967666038473\\
583	0.0114695405407929\\
584	0.0115432987726396\\
585	0.0116182821028575\\
586	0.0116962863978285\\
587	0.0117808462511651\\
588	0.0118879148142366\\
589	0.0120123879998692\\
590	0.0121374196899412\\
591	0.0122629658449694\\
592	0.0123891236352706\\
593	0.0125163895508632\\
594	0.0126463415783053\\
595	0.0127834870314329\\
596	0.0129402012848292\\
597	0.0131792323843585\\
598	0.0135751148335434\\
599	0\\
600	0\\
};
\addplot [color=mycolor15,solid,forget plot]
  table[row sep=crcr]{%
1	0.00554917154551039\\
2	0.00554917302704827\\
3	0.00554917453539954\\
4	0.00554917607104861\\
5	0.00554917763448863\\
6	0.00554917922622163\\
7	0.00554918084675867\\
8	0.00554918249662003\\
9	0.00554918417633534\\
10	0.0055491858864438\\
11	0.00554918762749428\\
12	0.00554918940004555\\
13	0.00554919120466645\\
14	0.00554919304193603\\
15	0.00554919491244378\\
16	0.0055491968167898\\
17	0.00554919875558499\\
18	0.00554920072945123\\
19	0.0055492027390216\\
20	0.00554920478494056\\
21	0.00554920686786417\\
22	0.00554920898846029\\
23	0.00554921114740879\\
24	0.00554921334540176\\
25	0.00554921558314375\\
26	0.00554921786135196\\
27	0.0055492201807565\\
28	0.00554922254210059\\
29	0.00554922494614082\\
30	0.00554922739364737\\
31	0.00554922988540426\\
32	0.00554923242220962\\
33	0.00554923500487589\\
34	0.00554923763423012\\
35	0.00554924031111422\\
36	0.00554924303638522\\
37	0.00554924581091554\\
38	0.00554924863559327\\
39	0.00554925151132244\\
40	0.00554925443902332\\
41	0.0055492574196327\\
42	0.00554926045410419\\
43	0.0055492635434085\\
44	0.00554926668853378\\
45	0.0055492698904859\\
46	0.0055492731502888\\
47	0.00554927646898475\\
48	0.00554927984763476\\
49	0.00554928328731886\\
50	0.00554928678913644\\
51	0.00554929035420661\\
52	0.00554929398366856\\
53	0.00554929767868188\\
54	0.00554930144042699\\
55	0.00554930527010542\\
56	0.00554930916894028\\
57	0.00554931313817658\\
58	0.00554931717908164\\
59	0.0055493212929455\\
60	0.0055493254810813\\
61	0.00554932974482571\\
62	0.00554933408553934\\
63	0.00554933850460718\\
64	0.005549343003439\\
65	0.00554934758346984\\
66	0.00554935224616042\\
67	0.00554935699299761\\
68	0.00554936182549489\\
69	0.00554936674519285\\
70	0.00554937175365962\\
71	0.00554937685249142\\
72	0.00554938204331301\\
73	0.00554938732777823\\
74	0.0055493927075705\\
75	0.00554939818440334\\
76	0.00554940376002093\\
77	0.00554940943619863\\
78	0.00554941521474355\\
79	0.0055494210974951\\
80	0.00554942708632558\\
81	0.00554943318314074\\
82	0.00554943938988042\\
83	0.00554944570851908\\
84	0.0055494521410665\\
85	0.00554945868956835\\
86	0.00554946535610684\\
87	0.00554947214280138\\
88	0.00554947905180921\\
89	0.00554948608532613\\
90	0.00554949324558711\\
91	0.00554950053486706\\
92	0.00554950795548146\\
93	0.00554951550978714\\
94	0.005549523200183\\
95	0.00554953102911074\\
96	0.00554953899905563\\
97	0.00554954711254728\\
98	0.00554955537216042\\
99	0.00554956378051571\\
100	0.00554957234028057\\
101	0.00554958105416994\\
102	0.00554958992494722\\
103	0.00554959895542506\\
104	0.00554960814846624\\
105	0.00554961750698458\\
106	0.00554962703394583\\
107	0.00554963673236861\\
108	0.0055496466053253\\
109	0.00554965665594304\\
110	0.00554966688740468\\
111	0.00554967730294975\\
112	0.00554968790587549\\
113	0.00554969869953787\\
114	0.00554970968735261\\
115	0.00554972087279625\\
116	0.00554973225940723\\
117	0.00554974385078697\\
118	0.00554975565060099\\
119	0.00554976766258006\\
120	0.00554977989052133\\
121	0.00554979233828953\\
122	0.00554980500981814\\
123	0.00554981790911062\\
124	0.00554983104024164\\
125	0.00554984440735834\\
126	0.00554985801468164\\
127	0.00554987186650748\\
128	0.00554988596720822\\
129	0.00554990032123394\\
130	0.00554991493311383\\
131	0.00554992980745759\\
132	0.00554994494895686\\
133	0.00554996036238665\\
134	0.00554997605260683\\
135	0.00554999202456361\\
136	0.00555000828329111\\
137	0.00555002483391284\\
138	0.00555004168164336\\
139	0.00555005883178983\\
140	0.00555007628975369\\
141	0.00555009406103226\\
142	0.00555011215122052\\
143	0.00555013056601277\\
144	0.00555014931120444\\
145	0.00555016839269379\\
146	0.00555018781648384\\
147	0.00555020758868412\\
148	0.00555022771551262\\
149	0.00555024820329768\\
150	0.00555026905847991\\
151	0.00555029028761426\\
152	0.00555031189737193\\
153	0.00555033389454252\\
154	0.00555035628603607\\
155	0.00555037907888518\\
156	0.00555040228024722\\
157	0.00555042589740652\\
158	0.00555044993777658\\
159	0.00555047440890239\\
160	0.00555049931846274\\
161	0.0055505246742726\\
162	0.00555055048428549\\
163	0.00555057675659599\\
164	0.00555060349944217\\
165	0.00555063072120818\\
166	0.00555065843042678\\
167	0.00555068663578204\\
168	0.00555071534611194\\
169	0.00555074457041114\\
170	0.00555077431783375\\
171	0.00555080459769613\\
172	0.00555083541947978\\
173	0.00555086679283425\\
174	0.00555089872758013\\
175	0.00555093123371205\\
176	0.00555096432140182\\
177	0.00555099800100148\\
178	0.00555103228304659\\
179	0.0055510671782594\\
180	0.00555110269755221\\
181	0.00555113885203072\\
182	0.00555117565299745\\
183	0.00555121311195524\\
184	0.00555125124061082\\
185	0.00555129005087839\\
186	0.00555132955488332\\
187	0.00555136976496592\\
188	0.00555141069368523\\
189	0.00555145235382288\\
190	0.00555149475838711\\
191	0.00555153792061673\\
192	0.00555158185398528\\
193	0.00555162657220513\\
194	0.0055516720892318\\
195	0.00555171841926823\\
196	0.00555176557676923\\
197	0.00555181357644591\\
198	0.00555186243327029\\
199	0.00555191216247993\\
200	0.00555196277958265\\
201	0.00555201430036137\\
202	0.00555206674087901\\
203	0.00555212011748346\\
204	0.00555217444681268\\
205	0.00555222974579992\\
206	0.00555228603167892\\
207	0.00555234332198933\\
208	0.00555240163458216\\
209	0.00555246098762535\\
210	0.00555252139960944\\
211	0.00555258288935336\\
212	0.00555264547601026\\
213	0.00555270917907356\\
214	0.00555277401838301\\
215	0.0055528400141309\\
216	0.0055529071868684\\
217	0.00555297555751199\\
218	0.00555304514735003\\
219	0.00555311597804942\\
220	0.00555318807166242\\
221	0.00555326145063358\\
222	0.00555333613780678\\
223	0.00555341215643244\\
224	0.00555348953017482\\
225	0.0055535682831195\\
226	0.00555364843978096\\
227	0.00555373002511031\\
228	0.00555381306450319\\
229	0.00555389758380775\\
230	0.0055539836093329\\
231	0.00555407116785656\\
232	0.00555416028663419\\
233	0.00555425099340741\\
234	0.0055543433164128\\
235	0.00555443728439086\\
236	0.00555453292659516\\
237	0.00555463027280163\\
238	0.00555472935331802\\
239	0.00555483019899356\\
240	0.0055549328412288\\
241	0.00555503731198561\\
242	0.0055551436437974\\
243	0.00555525186977948\\
244	0.00555536202363966\\
245	0.00555547413968905\\
246	0.00555558825285302\\
247	0.00555570439868238\\
248	0.00555582261336481\\
249	0.00555594293373644\\
250	0.00555606539729368\\
251	0.00555619004220528\\
252	0.00555631690732457\\
253	0.00555644603220201\\
254	0.00555657745709787\\
255	0.00555671122299522\\
256	0.00555684737161315\\
257	0.00555698594542021\\
258	0.0055571269876481\\
259	0.00555727054230569\\
260	0.00555741665419315\\
261	0.00555756536891654\\
262	0.00555771673290246\\
263	0.00555787079341314\\
264	0.00555802759856171\\
265	0.00555818719732783\\
266	0.00555834963957351\\
267	0.00555851497605933\\
268	0.00555868325846091\\
269	0.00555885453938563\\
270	0.00555902887238973\\
271	0.00555920631199553\\
272	0.00555938691370899\\
273	0.00555957073403767\\
274	0.00555975783050963\\
275	0.00555994826169223\\
276	0.00556014208721073\\
277	0.00556033936776763\\
278	0.00556054016516232\\
279	0.00556074454231111\\
280	0.00556095256326765\\
281	0.00556116429324352\\
282	0.00556137979862943\\
283	0.0055615991470165\\
284	0.00556182240721812\\
285	0.00556204964929201\\
286	0.00556228094456272\\
287	0.00556251636564436\\
288	0.0055627559864639\\
289	0.0055629998822846\\
290	0.00556324812972992\\
291	0.00556350080680767\\
292	0.00556375799293457\\
293	0.00556401976896115\\
294	0.0055642862171969\\
295	0.00556455742143578\\
296	0.00556483346698204\\
297	0.00556511444067629\\
298	0.00556540043092187\\
299	0.00556569152771148\\
300	0.00556598782265403\\
301	0.00556628940900166\\
302	0.00556659638167706\\
303	0.00556690883730083\\
304	0.00556722687421898\\
305	0.00556755059253054\\
306	0.00556788009411519\\
307	0.00556821548266091\\
308	0.00556855686369184\\
309	0.00556890434459633\\
310	0.00556925803465529\\
311	0.0055696180450699\\
312	0.0055699844889865\\
313	0.00557035748151957\\
314	0.00557073713978149\\
315	0.00557112358290785\\
316	0.00557151693208222\\
317	0.00557191731055991\\
318	0.00557232484369089\\
319	0.00557273965894149\\
320	0.0055731618859148\\
321	0.0055735916563695\\
322	0.00557402910423693\\
323	0.00557447436563612\\
324	0.00557492757888645\\
325	0.00557538888451763\\
326	0.00557585842527666\\
327	0.00557633634613128\\
328	0.00557682279426963\\
329	0.00557731791909589\\
330	0.00557782187222188\\
331	0.00557833480745608\\
332	0.00557885688079342\\
333	0.00557938825041474\\
334	0.00557992907671277\\
335	0.00558047952236603\\
336	0.00558103975243689\\
337	0.00558160993421012\\
338	0.00558219023571588\\
339	0.00558278082173439\\
340	0.00558338185697873\\
341	0.00558399351843381\\
342	0.00558461598996123\\
343	0.00558524947227382\\
344	0.00558589421064353\\
345	0.00558655056895177\\
346	0.00558721921073454\\
347	0.00558790143648038\\
348	0.00558859917276615\\
349	0.00558931119697032\\
350	0.00559003525001013\\
351	0.00559077153301133\\
352	0.00559152025019985\\
353	0.00559228160892518\\
354	0.00559305581967932\\
355	0.00559384309610832\\
356	0.00559464365501986\\
357	0.0055954577164114\\
358	0.0055962855035362\\
359	0.00559712724287495\\
360	0.00559798316414822\\
361	0.00559885350032749\\
362	0.00559973848764452\\
363	0.00560063836559943\\
364	0.00560155337696753\\
365	0.00560248376780521\\
366	0.005603429787455\\
367	0.00560439168855038\\
368	0.00560536972702051\\
369	0.00560636416209527\\
370	0.00560737525631138\\
371	0.00560840327551999\\
372	0.00560944848889809\\
373	0.00561051116896803\\
374	0.00561159159163486\\
375	0.00561269003625283\\
376	0.00561380678570863\\
377	0.00561494212642573\\
378	0.00561609634817915\\
379	0.00561726974429356\\
380	0.00561846261159394\\
381	0.0056196752502201\\
382	0.00562090796323909\\
383	0.00562216105619101\\
384	0.0056234348376587\\
385	0.00562472962480594\\
386	0.00562604576180873\\
387	0.00562738364234743\\
388	0.00562874361106573\\
389	0.00563012600508707\\
390	0.00563153117065767\\
391	0.00563295946429848\\
392	0.00563441125400398\\
393	0.00563588692027197\\
394	0.00563738685635825\\
395	0.00563891146612651\\
396	0.00564046115520807\\
397	0.00564203630438714\\
398	0.00564363719711188\\
399	0.00564526383219889\\
400	0.00564691546332293\\
401	0.00564858955428353\\
402	0.00565027981729866\\
403	0.00565197457579428\\
404	0.00565366593167166\\
405	0.00565538934452754\\
406	0.00565714542080634\\
407	0.00565893479283458\\
408	0.00566075812098736\\
409	0.00566261609718918\\
410	0.00566450944641961\\
411	0.00566643891717215\\
412	0.00566840525874305\\
413	0.00567040933327525\\
414	0.00567245203278873\\
415	0.00567453428069853\\
416	0.00567665703341976\\
417	0.00567882128206196\\
418	0.00568102805422755\\
419	0.00568327841596929\\
420	0.00568557347405532\\
421	0.00568791437877067\\
422	0.00569030232708791\\
423	0.00569273856433907\\
424	0.00569522438449874\\
425	0.00569776113514944\\
426	0.00570035022057722\\
427	0.00570299310508007\\
428	0.00570569131650847\\
429	0.00570844645006433\\
430	0.00571126017239986\\
431	0.00571413422609405\\
432	0.00571707043467132\\
433	0.00572007070853657\\
434	0.0057231370526813\\
435	0.00572627157793447\\
436	0.00572947651814869\\
437	0.00573275424871647\\
438	0.00573610725214837\\
439	0.00573953783983038\\
440	0.00574304836304245\\
441	0.0057466416638279\\
442	0.00575032073486893\\
443	0.00575408872900602\\
444	0.00575794896945599\\
445	0.005761904960634\\
446	0.00576596039876006\\
447	0.0057701191800909\\
448	0.00577438540903259\\
449	0.00577876345010988\\
450	0.00578325798230108\\
451	0.00578787413772847\\
452	0.00579261788333079\\
453	0.00579749707736224\\
454	0.00580252446374141\\
455	0.00580772623961451\\
456	0.00581316667848552\\
457	0.0058190368898063\\
458	0.00582571327832045\\
459	0.00584059042965996\\
460	0.00585773982093341\\
461	0.00587526023831121\\
462	0.00589324647962475\\
463	0.00591191430334263\\
464	0.00593094298113521\\
465	0.00595033503583898\\
466	0.00597010524595364\\
467	0.00599027022057246\\
468	0.00601084903345737\\
469	0.00603186440538486\\
470	0.00605334328932895\\
471	0.00607529614807655\\
472	0.00609774357483375\\
473	0.00612070573904595\\
474	0.00614418625135804\\
475	0.00616813886537289\\
476	0.00619238784280072\\
477	0.00621653461603059\\
478	0.00624093325313133\\
479	0.00626620758177765\\
480	0.00629242239479999\\
481	0.00631965084802119\\
482	0.00634797811094961\\
483	0.00637751059040254\\
484	0.00640840380342836\\
485	0.00644096571105693\\
486	0.0064770832840262\\
487	0.00652002518815927\\
488	0.00656393309169494\\
489	0.0066088396029996\\
490	0.00665477853556085\\
491	0.00670178546472533\\
492	0.00674989769604734\\
493	0.00679915242175098\\
494	0.00684958398774613\\
495	0.00690124491949539\\
496	0.00695418094922131\\
497	0.00700844006767366\\
498	0.00706407789785076\\
499	0.00712116504867729\\
500	0.00717975879219979\\
501	0.00723992021112285\\
502	0.00730171515664672\\
503	0.00736521714647497\\
504	0.00743051996773778\\
505	0.00749780697214923\\
506	0.00756703696234098\\
507	0.00763831259437869\\
508	0.00771190792031724\\
509	0.00778968158317259\\
510	0.00787110445015054\\
511	0.0079541675000024\\
512	0.00803913862598498\\
513	0.00812701441602769\\
514	0.00822278708979622\\
515	0.0083231838402092\\
516	0.00842553243644917\\
517	0.00852987839857534\\
518	0.00863622999382598\\
519	0.00874448590634122\\
520	0.00885419865111339\\
521	0.00896377401270416\\
522	0.00906759136888438\\
523	0.00914987739233347\\
524	0.00923459905345162\\
525	0.00932209233857995\\
526	0.0094132450287492\\
527	0.00948313344972559\\
528	0.0095511033115845\\
529	0.00962013191647017\\
530	0.00969020651136166\\
531	0.00976192860983588\\
532	0.00983654852695251\\
533	0.009905044886232\\
534	0.00996871227497252\\
535	0.010033213055165\\
536	0.0100985335652152\\
537	0.0101646565910092\\
538	0.010231566322025\\
539	0.0102992570504745\\
540	0.0103677613917222\\
541	0.0104371367358114\\
542	0.0105028361840887\\
543	0.0105698440644865\\
544	0.0106382189867722\\
545	0.0107079864248684\\
546	0.0107791824425344\\
547	0.0108518507671211\\
548	0.0109260426747462\\
549	0.0110018298799197\\
550	0.0110793484931092\\
551	0.0111589520417366\\
552	0.0112417508293042\\
553	0.0113293973419312\\
554	0.0114360310476757\\
555	0.0115467165314578\\
556	0.0116590405219288\\
557	0.0117737184096062\\
558	0.0118887522695475\\
559	0.0120040017738611\\
560	0.0121192339440008\\
561	0.0122339869402639\\
562	0.0123472020556088\\
563	0.012456211370228\\
564	0.0125541195156636\\
565	0.0126272129719429\\
566	0.0126698298096857\\
567	0.0127106595773791\\
568	0.0127463815351436\\
569	0.0127799532292079\\
570	0.0128131432611447\\
571	0.0128458920068395\\
572	0.0128779099188228\\
573	0.0129079583442177\\
574	0.0129377268471323\\
575	0.0129671630339223\\
576	0.0129961970959654\\
577	0.0130247542793291\\
578	0.0130527538829558\\
579	0.0130801305617351\\
580	0.0131067980943907\\
581	0.0131319344905414\\
582	0.013156257928101\\
583	0.0131798583419052\\
584	0.0132026276096981\\
585	0.0132244499636923\\
586	0.013245665840773\\
587	0.0132665132865002\\
588	0.0132869135184188\\
589	0.0133067844580865\\
590	0.0133260406985482\\
591	0.0133445930207368\\
592	0.0133623477620165\\
593	0.0133792066748557\\
594	0.0133950830302497\\
595	0.0134144577918238\\
596	0.0134418744311713\\
597	0.0135044530401156\\
598	0.0135751148335434\\
599	0\\
600	0\\
};
\addplot [color=mycolor16,solid,forget plot]
  table[row sep=crcr]{%
1	0.00696143322487153\\
2	0.00696144146032242\\
3	0.00696144984483376\\
4	0.00696145838109889\\
5	0.00696146707185975\\
6	0.00696147591990765\\
7	0.00696148492808418\\
8	0.00696149409928215\\
9	0.00696150343644645\\
10	0.00696151294257503\\
11	0.00696152262071983\\
12	0.00696153247398774\\
13	0.00696154250554162\\
14	0.00696155271860125\\
15	0.00696156311644441\\
16	0.00696157370240786\\
17	0.00696158447988844\\
18	0.00696159545234414\\
19	0.00696160662329517\\
20	0.00696161799632509\\
21	0.00696162957508196\\
22	0.00696164136327947\\
23	0.00696165336469813\\
24	0.00696166558318646\\
25	0.00696167802266219\\
26	0.00696169068711355\\
27	0.00696170358060049\\
28	0.00696171670725595\\
29	0.00696173007128721\\
30	0.00696174367697718\\
31	0.00696175752868578\\
32	0.00696177163085131\\
33	0.00696178598799185\\
34	0.00696180060470668\\
35	0.00696181548567772\\
36	0.00696183063567105\\
37	0.00696184605953839\\
38	0.00696186176221861\\
39	0.00696187774873931\\
40	0.00696189402421841\\
41	0.00696191059386575\\
42	0.00696192746298473\\
43	0.00696194463697401\\
44	0.00696196212132916\\
45	0.00696197992164443\\
46	0.00696199804361451\\
47	0.00696201649303632\\
48	0.0069620352758108\\
49	0.00696205439794482\\
50	0.00696207386555304\\
51	0.00696209368485984\\
52	0.00696211386220129\\
53	0.0069621344040271\\
54	0.00696215531690271\\
55	0.00696217660751129\\
56	0.00696219828265589\\
57	0.00696222034926157\\
58	0.00696224281437753\\
59	0.00696226568517939\\
60	0.00696228896897139\\
61	0.00696231267318872\\
62	0.00696233680539982\\
63	0.00696236137330879\\
64	0.00696238638475776\\
65	0.00696241184772942\\
66	0.00696243777034944\\
67	0.00696246416088909\\
68	0.00696249102776779\\
69	0.00696251837955576\\
70	0.00696254622497671\\
71	0.00696257457291057\\
72	0.00696260343239627\\
73	0.00696263281263455\\
74	0.00696266272299091\\
75	0.00696269317299844\\
76	0.00696272417236088\\
77	0.00696275573095565\\
78	0.00696278785883688\\
79	0.00696282056623865\\
80	0.00696285386357808\\
81	0.00696288776145869\\
82	0.00696292227067364\\
83	0.00696295740220915\\
84	0.00696299316724791\\
85	0.00696302957717258\\
86	0.00696306664356933\\
87	0.00696310437823148\\
88	0.00696314279316319\\
89	0.00696318190058316\\
90	0.0069632217129285\\
91	0.00696326224285858\\
92	0.00696330350325897\\
93	0.00696334550724549\\
94	0.0069633882681683\\
95	0.00696343179961601\\
96	0.00696347611542\\
97	0.00696352122965865\\
98	0.00696356715666178\\
99	0.00696361391101511\\
100	0.00696366150756478\\
101	0.006963709961422\\
102	0.00696375928796774\\
103	0.00696380950285753\\
104	0.00696386062202632\\
105	0.00696391266169348\\
106	0.00696396563836779\\
107	0.00696401956885262\\
108	0.00696407447025114\\
109	0.00696413035997168\\
110	0.00696418725573307\\
111	0.00696424517557023\\
112	0.00696430413783972\\
113	0.00696436416122549\\
114	0.00696442526474467\\
115	0.00696448746775348\\
116	0.00696455078995324\\
117	0.00696461525139654\\
118	0.0069646808724934\\
119	0.00696474767401765\\
120	0.00696481567711341\\
121	0.00696488490330162\\
122	0.00696495537448672\\
123	0.00696502711296349\\
124	0.00696510014142396\\
125	0.00696517448296441\\
126	0.00696525016109264\\
127	0.00696532719973518\\
128	0.00696540562324474\\
129	0.00696548545640784\\
130	0.00696556672445236\\
131	0.00696564945305554\\
132	0.00696573366835182\\
133	0.006965819396941\\
134	0.00696590666589651\\
135	0.00696599550277379\\
136	0.00696608593561885\\
137	0.00696617799297695\\
138	0.00696627170390153\\
139	0.00696636709796312\\
140	0.00696646420525861\\
141	0.00696656305642053\\
142	0.00696666368262657\\
143	0.00696676611560927\\
144	0.00696687038766588\\
145	0.00696697653166832\\
146	0.00696708458107346\\
147	0.00696719456993342\\
148	0.00696730653290622\\
149	0.00696742050526649\\
150	0.0069675365229164\\
151	0.00696765462239688\\
152	0.00696777484089888\\
153	0.00696789721627495\\
154	0.00696802178705102\\
155	0.0069681485924383\\
156	0.00696827767234553\\
157	0.00696840906739131\\
158	0.00696854281891675\\
159	0.00696867896899833\\
160	0.00696881756046091\\
161	0.00696895863689111\\
162	0.00696910224265082\\
163	0.00696924842289099\\
164	0.00696939722356568\\
165	0.00696954869144635\\
166	0.00696970287413637\\
167	0.00696985982008592\\
168	0.00697001957860697\\
169	0.00697018219988867\\
170	0.00697034773501297\\
171	0.00697051623597057\\
172	0.00697068775567703\\
173	0.00697086234798932\\
174	0.00697104006772262\\
175	0.00697122097066734\\
176	0.0069714051136066\\
177	0.00697159255433388\\
178	0.0069717833516711\\
179	0.00697197756548698\\
180	0.00697217525671572\\
181	0.00697237648737606\\
182	0.00697258132059065\\
183	0.00697278982060581\\
184	0.0069730020528116\\
185	0.00697321808376235\\
186	0.00697343798119741\\
187	0.00697366181406246\\
188	0.00697388965253106\\
189	0.00697412156802669\\
190	0.00697435763324516\\
191	0.00697459792217744\\
192	0.00697484251013287\\
193	0.00697509147376288\\
194	0.00697534489108506\\
195	0.00697560284150774\\
196	0.00697586540585497\\
197	0.00697613266639202\\
198	0.00697640470685131\\
199	0.0069766816124588\\
200	0.00697696346996095\\
201	0.00697725036765211\\
202	0.00697754239540244\\
203	0.00697783964468634\\
204	0.00697814220861145\\
205	0.00697845018194814\\
206	0.00697876366115959\\
207	0.00697908274443239\\
208	0.00697940753170776\\
209	0.00697973812471333\\
210	0.00698007462699552\\
211	0.00698041714395254\\
212	0.00698076578286798\\
213	0.00698112065294507\\
214	0.00698148186534155\\
215	0.00698184953320526\\
216	0.00698222377171029\\
217	0.00698260469809399\\
218	0.00698299243169447\\
219	0.00698338709398898\\
220	0.00698378880863296\\
221	0.0069841977014998\\
222	0.00698461390072141\\
223	0.00698503753672954\\
224	0.00698546874229788\\
225	0.00698590765258501\\
226	0.00698635440517809\\
227	0.00698680914013747\\
228	0.0069872720000421\\
229	0.00698774313003584\\
230	0.00698822267787467\\
231	0.00698871079397477\\
232	0.00698920763146153\\
233	0.00698971334621958\\
234	0.00699022809694367\\
235	0.00699075204519062\\
236	0.00699128535543224\\
237	0.00699182819510928\\
238	0.00699238073468644\\
239	0.00699294314770841\\
240	0.00699351561085704\\
241	0.00699409830400958\\
242	0.00699469141029811\\
243	0.00699529511617007\\
244	0.00699590961144998\\
245	0.00699653508940242\\
246	0.00699717174679613\\
247	0.00699781978396948\\
248	0.00699847940489714\\
249	0.00699915081725808\\
250	0.00699983423250492\\
251	0.00700052986593458\\
252	0.00700123793676039\\
253	0.00700195866818557\\
254	0.00700269228747813\\
255	0.0070034390260473\\
256	0.00700419911952142\\
257	0.00700497280782739\\
258	0.00700576033527162\\
259	0.00700656195062264\\
260	0.00700737790719535\\
261	0.00700820846293681\\
262	0.00700905388051388\\
263	0.00700991442740246\\
264	0.00701079037597854\\
265	0.00701168200361107\\
266	0.00701258959275662\\
267	0.0070135134310559\\
268	0.00701445381143224\\
269	0.00701541103219203\\
270	0.00701638539712703\\
271	0.00701737721561869\\
272	0.00701838680274421\\
273	0.00701941447938431\\
274	0.00702046057233346\\
275	0.0070215254144148\\
276	0.00702260934459632\\
277	0.00702371270810586\\
278	0.00702483585655055\\
279	0.00702597914803856\\
280	0.00702714294730321\\
281	0.00702832762582932\\
282	0.00702953356198208\\
283	0.0070307611411382\\
284	0.0070320107558196\\
285	0.00703328280582942\\
286	0.00703457769839071\\
287	0.0070358958482874\\
288	0.00703723767800791\\
289	0.00703860361789132\\
290	0.00703999410627592\\
291	0.00704140958965045\\
292	0.0070428505228078\\
293	0.00704431736900126\\
294	0.00704581060010326\\
295	0.00704733069676665\\
296	0.00704887814858842\\
297	0.00705045345427586\\
298	0.00705205712181509\\
299	0.00705368966864196\\
300	0.00705535162181517\\
301	0.00705704351819158\\
302	0.00705876590460362\\
303	0.00706051933803868\\
304	0.00706230438582024\\
305	0.00706412162579081\\
306	0.00706597164649623\\
307	0.00706785504737133\\
308	0.00706977243892669\\
309	0.00707172444293687\\
310	0.0070737116926303\\
311	0.00707573483288177\\
312	0.00707779452040469\\
313	0.00707989142393294\\
314	0.00708202622438877\\
315	0.00708419961508777\\
316	0.00708641230192291\\
317	0.00708866500354615\\
318	0.00709095845154684\\
319	0.00709329339062627\\
320	0.00709567057876735\\
321	0.00709809078739845\\
322	0.00710055480155039\\
323	0.00710306342000517\\
324	0.00710561745543499\\
325	0.00710821773453001\\
326	0.00711086509811281\\
327	0.0071135604012376\\
328	0.00711630451327133\\
329	0.0071190983179542\\
330	0.00712194271343594\\
331	0.00712483861228473\\
332	0.00712778694146582\\
333	0.00713078864229163\\
334	0.00713384467035772\\
335	0.00713695599551142\\
336	0.00714012360195022\\
337	0.00714334848850996\\
338	0.00714663166846703\\
339	0.00714997416463147\\
340	0.00715337698804149\\
341	0.00715684114724788\\
342	0.00716036771180513\\
343	0.00716395777212828\\
344	0.0071676124663467\\
345	0.00717133307328994\\
346	0.00717512132099987\\
347	0.00717898035561633\\
348	0.00718291742133764\\
349	0.00718694812492824\\
350	0.0071910664840161\\
351	0.00719525605491353\\
352	0.00719951808601074\\
353	0.00720385384708993\\
354	0.00720826462959008\\
355	0.007212751746859\\
356	0.00721731653437655\\
357	0.00722196034994399\\
358	0.00722668457394563\\
359	0.00723149060987166\\
360	0.00723637988433051\\
361	0.00724135384727935\\
362	0.00724641397224998\\
363	0.00725156175657058\\
364	0.00725679872158399\\
365	0.00726212641286367\\
366	0.00726754640042848\\
367	0.00727306027895821\\
368	0.00727866966801178\\
369	0.00728437621225086\\
370	0.00729018158167167\\
371	0.00729608747184848\\
372	0.00730209560419309\\
373	0.0073082077262375\\
374	0.00731442561195582\\
375	0.00732075106216334\\
376	0.00732718590506092\\
377	0.00733373199695315\\
378	0.00734039122278259\\
379	0.00734716549549618\\
380	0.00735405675769685\\
381	0.00736106698230057\\
382	0.00736819817295515\\
383	0.00737545236391039\\
384	0.00738283161938419\\
385	0.00739033803538435\\
386	0.00739797375866508\\
387	0.00740574106682857\\
388	0.00741364253887084\\
389	0.00742168049563502\\
390	0.00742985722816107\\
391	0.00743817509169642\\
392	0.00744663651316554\\
393	0.00745524399957295\\
394	0.00746400014702808\\
395	0.00747290764916912\\
396	0.00748196930100744\\
397	0.00749118798598605\\
398	0.00750056660970462\\
399	0.00751010787273188\\
400	0.00751981357245011\\
401	0.00752968257217448\\
402	0.00753970521337361\\
403	0.0075498494324153\\
404	0.00756003533420194\\
405	0.00757015939809591\\
406	0.0075804852622011\\
407	0.00759101712502076\\
408	0.00760175936052149\\
409	0.00761271653085893\\
410	0.00762389340746015\\
411	0.00763529499385015\\
412	0.00764692651424247\\
413	0.00765879325027115\\
414	0.00767090117412906\\
415	0.00768325647949881\\
416	0.00769586559325436\\
417	0.00770873518791299\\
418	0.00772187219487307\\
419	0.00773528381848334\\
420	0.00774897755107084\\
421	0.00776296118937236\\
422	0.00777724285353389\\
423	0.00779183100969855\\
424	0.00780673448794035\\
425	0.007821962490855\\
426	0.00783752462608151\\
427	0.00785343093061688\\
428	0.00786969189686395\\
429	0.00788631850054416\\
430	0.00790332223061324\\
431	0.00792071512132135\\
432	0.00793850978657212\\
433	0.00795671945680637\\
434	0.00797535801896725\\
435	0.00799444006130549\\
436	0.0080139809283583\\
437	0.00803399679912899\\
438	0.00805450480306563\\
439	0.00807552306308724\\
440	0.00809706933016405\\
441	0.00811916185548849\\
442	0.00814182179655631\\
443	0.00816507150804996\\
444	0.00818893462286462\\
445	0.00821343613903194\\
446	0.00823860251300918\\
447	0.00826446175758123\\
448	0.00829104353127232\\
449	0.00831837918308929\\
450	0.00834650198159369\\
451	0.00837544721419212\\
452	0.00840525228742067\\
453	0.00843595687049254\\
454	0.00846760314389626\\
455	0.00850023662132078\\
456	0.00853390941509973\\
457	0.00856868516294145\\
458	0.00860508663371908\\
459	0.00863660635565966\\
460	0.00866741315777336\\
461	0.00869886515522766\\
462	0.00873108124645628\\
463	0.00876454255342039\\
464	0.00880149541322494\\
465	0.00883919605348062\\
466	0.00887755606675658\\
467	0.00891657571202679\\
468	0.0089562547891163\\
469	0.00899659522467287\\
470	0.00903761013241903\\
471	0.0090793556040983\\
472	0.00912171397531188\\
473	0.00916463320757009\\
474	0.00920806236299831\\
475	0.00925185723714607\\
476	0.00929557787366925\\
477	0.00933781980842638\\
478	0.00937392481136324\\
479	0.00940555137214414\\
480	0.0094377079498843\\
481	0.00947038608381698\\
482	0.00950357471372327\\
483	0.00953726045095219\\
484	0.00957143027223439\\
485	0.00960608874650984\\
486	0.00964018797110519\\
487	0.00967109526646436\\
488	0.00970267192128993\\
489	0.00973493564128096\\
490	0.00976790584651184\\
491	0.00980160034254924\\
492	0.0098360369782581\\
493	0.00987123282150973\\
494	0.00990720440162515\\
495	0.00994397293568609\\
496	0.00998155569775987\\
497	0.0100199672566739\\
498	0.0100592141609306\\
499	0.0100993145340216\\
500	0.0101403632205247\\
501	0.0101823792078284\\
502	0.0102253804611579\\
503	0.010269384236512\\
504	0.010314410402072\\
505	0.0103605044547501\\
506	0.0104089455982218\\
507	0.0104584430138407\\
508	0.0105086808662988\\
509	0.0105585199560321\\
510	0.0106081368440963\\
511	0.0106587980734855\\
512	0.0107105831147677\\
513	0.0107638093492153\\
514	0.0108202617208306\\
515	0.0108832440905393\\
516	0.0109493411670594\\
517	0.0110164229674561\\
518	0.0110844758487037\\
519	0.011153436878803\\
520	0.0112231160368238\\
521	0.0112928785938636\\
522	0.0113604152267201\\
523	0.0114189106180415\\
524	0.0114781302422692\\
525	0.0115380184175737\\
526	0.0115984924560614\\
527	0.0116475134592567\\
528	0.0116950283817215\\
529	0.0117424929243432\\
530	0.0117898115791825\\
531	0.0118365633076319\\
532	0.0118819804741919\\
533	0.0119237491743803\\
534	0.0119627022017451\\
535	0.012001785933573\\
536	0.0120409602379146\\
537	0.0120801795329016\\
538	0.0121193937642504\\
539	0.0121585482894317\\
540	0.0121975824734986\\
541	0.0122363778195194\\
542	0.0122725438386972\\
543	0.0123088361792697\\
544	0.0123452218900791\\
545	0.0123830652189747\\
546	0.0124209737098417\\
547	0.0124583196058804\\
548	0.0124950331107372\\
549	0.0125310410219464\\
550	0.0125662667460497\\
551	0.0126006301311529\\
552	0.0126340462254209\\
553	0.0126653782393577\\
554	0.0126955728633947\\
555	0.012724847087084\\
556	0.012752807126695\\
557	0.0127788004957964\\
558	0.0128037596527478\\
559	0.0128275761396023\\
560	0.0128507411189602\\
561	0.0128732667339028\\
562	0.01289507328928\\
563	0.0129160749825545\\
564	0.0129361810864091\\
565	0.0129553557382554\\
566	0.012973893166586\\
567	0.0129917465047786\\
568	0.0130087234136089\\
569	0.0130249776205532\\
570	0.0130412538691682\\
571	0.0130575426055185\\
572	0.0130738402601048\\
573	0.0130901740700694\\
574	0.0131065312929425\\
575	0.0131228989683055\\
576	0.0131392640595054\\
577	0.0131556134451633\\
578	0.0131719338770583\\
579	0.0131882118427065\\
580	0.0132044337148521\\
581	0.0132206236488861\\
582	0.0132367682457335\\
583	0.0132528443750539\\
584	0.0132688278512317\\
585	0.0132846953190783\\
586	0.0133004226334248\\
587	0.0133159862811858\\
588	0.0133323317216059\\
589	0.0133490678702832\\
590	0.0133656835516412\\
591	0.0133817357881134\\
592	0.0133970600889429\\
593	0.0134115514637899\\
594	0.0134262135232344\\
595	0.0134435354835885\\
596	0.013477883826111\\
597	0.0135160557778789\\
598	0.0135751148335434\\
599	0\\
600	0\\
};
\addplot [color=mycolor17,solid,forget plot]
  table[row sep=crcr]{%
1	0.00892804838704115\\
2	0.00892805640983568\\
3	0.00892806457795805\\
4	0.0089280728940377\\
5	0.00892808136075152\\
6	0.00892808998082476\\
7	0.00892809875703188\\
8	0.00892810769219741\\
9	0.0089281167891969\\
10	0.00892812605095781\\
11	0.00892813548046043\\
12	0.00892814508073886\\
13	0.00892815485488195\\
14	0.0089281648060343\\
15	0.00892817493739729\\
16	0.00892818525223003\\
17	0.00892819575385047\\
18	0.0089282064456364\\
19	0.00892821733102659\\
20	0.0089282284135218\\
21	0.00892823969668598\\
22	0.00892825118414736\\
23	0.0089282628795996\\
24	0.00892827478680298\\
25	0.00892828690958562\\
26	0.00892829925184464\\
27	0.00892831181754744\\
28	0.00892832461073297\\
29	0.00892833763551297\\
30	0.00892835089607334\\
31	0.00892836439667542\\
32	0.00892837814165735\\
33	0.00892839213543547\\
34	0.00892840638250572\\
35	0.00892842088744504\\
36	0.00892843565491286\\
37	0.00892845068965257\\
38	0.008928465996493\\
39	0.00892848158034997\\
40	0.00892849744622789\\
41	0.00892851359922127\\
42	0.00892853004451639\\
43	0.00892854678739294\\
44	0.00892856383322567\\
45	0.00892858118748613\\
46	0.00892859885574438\\
47	0.00892861684367074\\
48	0.00892863515703764\\
49	0.00892865380172138\\
50	0.00892867278370407\\
51	0.00892869210907545\\
52	0.00892871178403485\\
53	0.00892873181489319\\
54	0.0089287522080749\\
55	0.00892877297012003\\
56	0.00892879410768626\\
57	0.00892881562755104\\
58	0.00892883753661372\\
59	0.00892885984189774\\
60	0.00892888255055286\\
61	0.0089289056698574\\
62	0.00892892920722054\\
63	0.00892895317018468\\
64	0.00892897756642782\\
65	0.00892900240376597\\
66	0.00892902769015562\\
67	0.00892905343369629\\
68	0.00892907964263303\\
69	0.00892910632535908\\
70	0.00892913349041845\\
71	0.00892916114650871\\
72	0.00892918930248363\\
73	0.00892921796735607\\
74	0.00892924715030075\\
75	0.00892927686065718\\
76	0.0089293071079326\\
77	0.00892933790180497\\
78	0.00892936925212603\\
79	0.00892940116892439\\
80	0.00892943366240869\\
81	0.00892946674297086\\
82	0.00892950042118933\\
83	0.00892953470783239\\
84	0.00892956961386161\\
85	0.00892960515043523\\
86	0.00892964132891171\\
87	0.00892967816085331\\
88	0.00892971565802973\\
89	0.00892975383242178\\
90	0.00892979269622517\\
91	0.00892983226185436\\
92	0.00892987254194646\\
93	0.00892991354936516\\
94	0.00892995529720484\\
95	0.00892999779879464\\
96	0.00893004106770267\\
97	0.00893008511774027\\
98	0.00893012996296635\\
99	0.00893017561769179\\
100	0.00893022209648395\\
101	0.00893026941417128\\
102	0.0089303175858479\\
103	0.00893036662687841\\
104	0.00893041655290266\\
105	0.00893046737984072\\
106	0.0089305191238978\\
107	0.00893057180156941\\
108	0.00893062542964648\\
109	0.00893068002522067\\
110	0.0089307356056897\\
111	0.00893079218876283\\
112	0.00893084979246641\\
113	0.00893090843514949\\
114	0.00893096813548967\\
115	0.00893102891249884\\
116	0.00893109078552924\\
117	0.00893115377427948\\
118	0.00893121789880069\\
119	0.00893128317950286\\
120	0.00893134963716121\\
121	0.00893141729292267\\
122	0.00893148616831257\\
123	0.00893155628524133\\
124	0.00893162766601135\\
125	0.00893170033332399\\
126	0.00893177431028667\\
127	0.00893184962042012\\
128	0.00893192628766576\\
129	0.00893200433639316\\
130	0.00893208379140772\\
131	0.00893216467795836\\
132	0.00893224702174554\\
133	0.0089323308489292\\
134	0.00893241618613701\\
135	0.00893250306047269\\
136	0.00893259149952448\\
137	0.00893268153137383\\
138	0.0089327731846041\\
139	0.00893286648830962\\
140	0.00893296147210469\\
141	0.00893305816613293\\
142	0.00893315660107668\\
143	0.00893325680816662\\
144	0.00893335881919151\\
145	0.00893346266650821\\
146	0.00893356838305174\\
147	0.00893367600234563\\
148	0.00893378555851242\\
149	0.00893389708628431\\
150	0.00893401062101406\\
151	0.00893412619868607\\
152	0.00893424385592761\\
153	0.00893436363002035\\
154	0.00893448555891196\\
155	0.00893460968122806\\
156	0.00893473603628429\\
157	0.00893486466409862\\
158	0.00893499560540393\\
159	0.00893512890166072\\
160	0.00893526459507013\\
161	0.00893540272858717\\
162	0.00893554334593417\\
163	0.00893568649161449\\
164	0.00893583221092651\\
165	0.00893598054997778\\
166	0.00893613155569951\\
167	0.00893628527586131\\
168	0.00893644175908615\\
169	0.00893660105486564\\
170	0.00893676321357556\\
171	0.00893692828649168\\
172	0.00893709632580585\\
173	0.00893726738464244\\
174	0.00893744151707497\\
175	0.00893761877814317\\
176	0.00893779922387023\\
177	0.00893798291128047\\
178	0.00893816989841723\\
179	0.00893836024436119\\
180	0.00893855400924893\\
181	0.00893875125429188\\
182	0.00893895204179562\\
183	0.0089391564351795\\
184	0.00893936449899665\\
185	0.00893957629895431\\
186	0.00893979190193462\\
187	0.00894001137601568\\
188	0.0089402347904931\\
189	0.00894046221590187\\
190	0.00894069372403866\\
191	0.00894092938798455\\
192	0.00894116928212814\\
193	0.00894141348218914\\
194	0.00894166206524232\\
195	0.008941915109742\\
196	0.00894217269554687\\
197	0.0089424349039454\\
198	0.00894270181768164\\
199	0.00894297352098154\\
200	0.0089432500995797\\
201	0.0089435316407467\\
202	0.00894381823331691\\
203	0.00894410996771675\\
204	0.00894440693599363\\
205	0.00894470923184526\\
206	0.00894501695064965\\
207	0.0089453301894956\\
208	0.00894564904721376\\
209	0.00894597362440836\\
210	0.0089463040234894\\
211	0.00894664034870562\\
212	0.00894698270617792\\
213	0.00894733120393355\\
214	0.00894768595194088\\
215	0.00894804706214487\\
216	0.00894841464850319\\
217	0.00894878882702303\\
218	0.00894916971579866\\
219	0.00894955743504967\\
220	0.00894995210715993\\
221	0.00895035385671736\\
222	0.00895076281055444\\
223	0.00895117909778948\\
224	0.00895160284986871\\
225	0.00895203420060922\\
226	0.0089524732862427\\
227	0.00895292024546002\\
228	0.00895337521945672\\
229	0.00895383835197938\\
230	0.00895430978937292\\
231	0.00895478968062875\\
232	0.008955278177434\\
233	0.00895577543422161\\
234	0.00895628160822148\\
235	0.0089567968595126\\
236	0.00895732135107625\\
237	0.00895785524885025\\
238	0.00895839872178423\\
239	0.00895895194189618\\
240	0.0089595150843299\\
241	0.00896008832741384\\
242	0.00896067185272095\\
243	0.00896126584512982\\
244	0.00896187049288706\\
245	0.00896248598767096\\
246	0.00896311252465636\\
247	0.00896375030258097\\
248	0.00896439952381293\\
249	0.00896506039441982\\
250	0.00896573312423908\\
251	0.0089664179269499\\
252	0.00896711502014647\\
253	0.00896782462541293\\
254	0.00896854696839975\\
255	0.00896928227890172\\
256	0.00897003079093755\\
257	0.00897079274283118\\
258	0.00897156837729477\\
259	0.00897235794151336\\
260	0.00897316168723145\\
261	0.00897397987084129\\
262	0.00897481275347308\\
263	0.008975660601087\\
264	0.0089765236845675\\
265	0.00897740227981922\\
266	0.00897829666786521\\
267	0.00897920713494724\\
268	0.00898013397262827\\
269	0.00898107747789723\\
270	0.00898203795327612\\
271	0.0089830157069295\\
272	0.00898401105277622\\
273	0.00898502431060269\\
274	0.00898605580617691\\
275	0.00898710587136502\\
276	0.00898817484426421\\
277	0.00898926306933571\\
278	0.00899037089752115\\
279	0.00899149868637397\\
280	0.00899264680019391\\
281	0.00899381561016477\\
282	0.00899500549449545\\
283	0.00899621683856444\\
284	0.00899745003506788\\
285	0.00899870548417121\\
286	0.00899998359366466\\
287	0.00900128477912276\\
288	0.00900260946406756\\
289	0.00900395808013647\\
290	0.00900533106725425\\
291	0.00900672887380945\\
292	0.00900815195683561\\
293	0.00900960078219738\\
294	0.00901107582478151\\
295	0.0090125775686932\\
296	0.00901410650745788\\
297	0.00901566314422855\\
298	0.00901724799199918\\
299	0.00901886157382402\\
300	0.00902050442304349\\
301	0.00902217708351661\\
302	0.00902388010986039\\
303	0.00902561406769645\\
304	0.00902737953390522\\
305	0.00902917709688795\\
306	0.00903100735683696\\
307	0.0090328709260144\\
308	0.00903476842903988\\
309	0.0090367005031876\\
310	0.00903866779869467\\
311	0.00904067097908386\\
312	0.00904271072150657\\
313	0.00904478771710377\\
314	0.00904690267133902\\
315	0.00904905630424975\\
316	0.00905124935093033\\
317	0.00905348256192723\\
318	0.00905575670365084\\
319	0.00905807255880501\\
320	0.00906043092683612\\
321	0.00906283262440331\\
322	0.00906527848587189\\
323	0.00906776936383229\\
324	0.00907030612964714\\
325	0.00907288967402974\\
326	0.00907552090765718\\
327	0.00907820076182264\\
328	0.0090809301891315\\
329	0.00908371016424696\\
330	0.00908654168469219\\
331	0.00908942577171742\\
332	0.00909236347124423\\
333	0.00909535585490737\\
334	0.00909840402123719\\
335	0.00910150909708704\\
336	0.00910467223956859\\
337	0.0091078946391192\\
338	0.00911117752489224\\
339	0.00911452217297208\\
340	0.009117929904183\\
341	0.00912140195114907\\
342	0.00912493944544214\\
343	0.00912854384149498\\
344	0.00913221667553634\\
345	0.0091359596499675\\
346	0.00913977493487116\\
347	0.00914366629969959\\
348	0.00914764358037652\\
349	0.00915174158444089\\
350	0.00915610935722876\\
351	0.00916071683785196\\
352	0.00916540651822822\\
353	0.00917017996260316\\
354	0.00917503876912129\\
355	0.00917998457066177\\
356	0.0091850190356612\\
357	0.00919014386884601\\
358	0.00919536081177479\\
359	0.00920067164361595\\
360	0.00920607818379611\\
361	0.00921158229156297\\
362	0.00921718586685545\\
363	0.00922289085116554\\
364	0.00922869922838611\\
365	0.00923461302563759\\
366	0.00924063431406558\\
367	0.00924676520959955\\
368	0.00925300787366149\\
369	0.00925936451381083\\
370	0.00926583738430968\\
371	0.00927242878658873\\
372	0.00927914106958991\\
373	0.009285976629957\\
374	0.0092929379120448\\
375	0.00930002740773486\\
376	0.00930724765611737\\
377	0.00931460124326209\\
378	0.00932209080235283\\
379	0.00932971901285813\\
380	0.0093374885902979\\
381	0.00934540228883939\\
382	0.00935346289644827\\
383	0.00936167322680546\\
384	0.00937003610528839\\
385	0.00937855434557265\\
386	0.00938723072045549\\
387	0.00939606797809044\\
388	0.00940506916934261\\
389	0.00941423935269848\\
390	0.00942358173118622\\
391	0.0094330988947954\\
392	0.00944279334612153\\
393	0.00945266747968104\\
394	0.00946272355793335\\
395	0.00947296368211432\\
396	0.00948338975256834\\
397	0.0094940034028413\\
398	0.00950480585988164\\
399	0.00951579758471138\\
400	0.00952697724586378\\
401	0.00953833864042698\\
402	0.00954986123602938\\
403	0.00956148061555462\\
404	0.0095729943293389\\
405	0.00958375414271156\\
406	0.00959222741999822\\
407	0.00960087756335211\\
408	0.00960970682560881\\
409	0.0096187173843982\\
410	0.00962791133042151\\
411	0.00963729069277058\\
412	0.00964685751165343\\
413	0.00965661394791322\\
414	0.00966656103481888\\
415	0.00967670279046636\\
416	0.0096870433218048\\
417	0.00969758682580962\\
418	0.00970833759051284\\
419	0.00971929999582829\\
420	0.00973047851406489\\
421	0.00974187771002526\\
422	0.00975350224103201\\
423	0.00976535685971458\\
424	0.00977744643006985\\
425	0.00978977592808905\\
426	0.00980235037981236\\
427	0.00981517490393675\\
428	0.00982825471004588\\
429	0.00984159509623094\\
430	0.00985520144601245\\
431	0.00986907922445911\\
432	0.00988323397338755\\
433	0.00989767130553936\\
434	0.00991239689776872\\
435	0.00992741648391694\\
436	0.00994273585059723\\
437	0.00995836084872592\\
438	0.00997429747010315\\
439	0.00999055218448742\\
440	0.0100071334877347\\
441	0.0100240438283082\\
442	0.0100412803436629\\
443	0.0100588483236861\\
444	0.0100767529587988\\
445	0.0100949993130888\\
446	0.0101135922958009\\
447	0.0101325366334713\\
448	0.0101518368468658\\
449	0.0101714972189087\\
450	0.0101915213798393\\
451	0.0102119127915055\\
452	0.0102326745812786\\
453	0.010253809258963\\
454	0.0102753183246448\\
455	0.0102972011577886\\
456	0.0103194511812423\\
457	0.0103420359173635\\
458	0.0103637599402654\\
459	0.010383191872\\
460	0.010402425628083\\
461	0.0104221546761598\\
462	0.01044242798672\\
463	0.0104634208839125\\
464	0.0104860432899365\\
465	0.010510615181872\\
466	0.0105357824310886\\
467	0.0105615127165632\\
468	0.0105878161380539\\
469	0.0106147031780618\\
470	0.0106421883015759\\
471	0.0106703058621089\\
472	0.0106997937195528\\
473	0.0107298931211558\\
474	0.0107603355654729\\
475	0.0107910604253917\\
476	0.0108218943414592\\
477	0.0108522775798314\\
478	0.0108801729146689\\
479	0.01090620775102\\
480	0.0109325095111674\\
481	0.0109590612965513\\
482	0.0109858369789297\\
483	0.0110128092622672\\
484	0.0110399491483798\\
485	0.0110672221140436\\
486	0.0110939716621178\\
487	0.0111187027085108\\
488	0.011143730164065\\
489	0.0111690445719119\\
490	0.0111946377319795\\
491	0.0112205190083465\\
492	0.011246678405485\\
493	0.0112731044986012\\
494	0.0112997846888265\\
495	0.0113267046482608\\
496	0.0113538481267316\\
497	0.0113811961224248\\
498	0.011408723906216\\
499	0.0114364086728828\\
500	0.0114642608042559\\
501	0.0114922556416281\\
502	0.0115203663822708\\
503	0.0115485640225937\\
504	0.0115768172754413\\
505	0.0116050923032829\\
506	0.0116326034873794\\
507	0.0116600285432745\\
508	0.0116875251793058\\
509	0.0117143157622327\\
510	0.0117405416631368\\
511	0.0117670484620547\\
512	0.0117938236606359\\
513	0.0118208529626045\\
514	0.0118481195309724\\
515	0.0118756083873784\\
516	0.0119033001665179\\
517	0.0119311845161186\\
518	0.0119592431864288\\
519	0.0119874515861353\\
520	0.0120157829496113\\
521	0.0120442086216939\\
522	0.0120726975450495\\
523	0.0121012173182994\\
524	0.0121297368341328\\
525	0.0121582224191919\\
526	0.0121866360774382\\
527	0.0122150593357926\\
528	0.0122434837731262\\
529	0.012271884608684\\
530	0.0123002348686295\\
531	0.0123282700170947\\
532	0.0123554997781726\\
533	0.0123829798826811\\
534	0.0124122116241502\\
535	0.0124410994665472\\
536	0.0124696062633754\\
537	0.0124976937165482\\
538	0.0125253225092751\\
539	0.0125524525053561\\
540	0.0125790427938751\\
541	0.0126050524613397\\
542	0.0126304764673356\\
543	0.0126552669333829\\
544	0.0126793745126733\\
545	0.0127019060328393\\
546	0.0127235404985523\\
547	0.0127445530443311\\
548	0.012764908444034\\
549	0.0127845717118121\\
550	0.0128035082420549\\
551	0.0128216841297711\\
552	0.0128390658052865\\
553	0.012855631894578\\
554	0.0128713391491722\\
555	0.0128864409532959\\
556	0.0129008913195492\\
557	0.012914325326947\\
558	0.0129275565336294\\
559	0.012940563094706\\
560	0.0129535966559269\\
561	0.0129666978623545\\
562	0.0129798688870274\\
563	0.0129931133670592\\
564	0.0130064365716933\\
565	0.0130198454289273\\
566	0.0130333460586304\\
567	0.0130469457085717\\
568	0.0130606659746072\\
569	0.0130745389185016\\
570	0.0130885627309631\\
571	0.0131027353386474\\
572	0.0131170541424563\\
573	0.0131315150127258\\
574	0.0131461132773175\\
575	0.0131608436180769\\
576	0.0131756999131583\\
577	0.013190677909612\\
578	0.0132057732483407\\
579	0.013220981360728\\
580	0.013236298980741\\
581	0.0132519674981059\\
582	0.0132687631281697\\
583	0.013285468939127\\
584	0.013302672513284\\
585	0.0133195446593288\\
586	0.0133364249559234\\
587	0.013352829393449\\
588	0.0133682594581791\\
589	0.0133825611506923\\
590	0.0133957579545669\\
591	0.0134085298188763\\
592	0.0134221484363206\\
593	0.0134355399849467\\
594	0.0134504085075599\\
595	0.0134677297271409\\
596	0.0134889777516421\\
597	0.0135160557778789\\
598	0.0135751148335434\\
599	0\\
600	0\\
};
\addplot [color=mycolor18,solid,forget plot]
  table[row sep=crcr]{%
1	0.0101435220610527\\
2	0.0101435269761815\\
3	0.0101435319804809\\
4	0.0101435370755683\\
5	0.0101435422630904\\
6	0.0101435475447237\\
7	0.0101435529221753\\
8	0.0101435583971832\\
9	0.0101435639715168\\
10	0.0101435696469776\\
11	0.0101435754253999\\
12	0.0101435813086511\\
13	0.0101435872986326\\
14	0.0101435933972802\\
15	0.0101435996065647\\
16	0.0101436059284929\\
17	0.0101436123651077\\
18	0.0101436189184893\\
19	0.0101436255907552\\
20	0.0101436323840618\\
21	0.0101436393006042\\
22	0.0101436463426174\\
23	0.010143653512377\\
24	0.0101436608121998\\
25	0.0101436682444445\\
26	0.0101436758115126\\
27	0.0101436835158492\\
28	0.0101436913599436\\
29	0.0101436993463302\\
30	0.0101437074775895\\
31	0.0101437157563485\\
32	0.010143724185282\\
33	0.0101437327671131\\
34	0.0101437415046144\\
35	0.0101437504006087\\
36	0.0101437594579697\\
37	0.0101437686796235\\
38	0.0101437780685489\\
39	0.0101437876277789\\
40	0.0101437973604013\\
41	0.0101438072695598\\
42	0.0101438173584551\\
43	0.0101438276303458\\
44	0.0101438380885495\\
45	0.0101438487364439\\
46	0.0101438595774679\\
47	0.0101438706151226\\
48	0.0101438818529725\\
49	0.0101438932946466\\
50	0.0101439049438395\\
51	0.010143916804313\\
52	0.0101439288798965\\
53	0.0101439411744891\\
54	0.0101439536920604\\
55	0.0101439664366515\\
56	0.010143979412377\\
57	0.0101439926234259\\
58	0.0101440060740627\\
59	0.0101440197686294\\
60	0.0101440337115463\\
61	0.0101440479073138\\
62	0.0101440623605135\\
63	0.0101440770758101\\
64	0.0101440920579525\\
65	0.0101441073117756\\
66	0.0101441228422015\\
67	0.0101441386542415\\
68	0.0101441547529975\\
69	0.0101441711436634\\
70	0.0101441878315273\\
71	0.0101442048219727\\
72	0.0101442221204803\\
73	0.0101442397326301\\
74	0.0101442576641027\\
75	0.0101442759206816\\
76	0.0101442945082545\\
77	0.0101443134328156\\
78	0.0101443327004675\\
79	0.0101443523174228\\
80	0.0101443722900065\\
81	0.0101443926246579\\
82	0.0101444133279322\\
83	0.0101444344065035\\
84	0.0101444558671659\\
85	0.0101444777168366\\
86	0.0101444999625574\\
87	0.0101445226114972\\
88	0.0101445456709546\\
89	0.0101445691483596\\
90	0.0101445930512764\\
91	0.0101446173874057\\
92	0.0101446421645873\\
93	0.0101446673908023\\
94	0.0101446930741756\\
95	0.0101447192229791\\
96	0.0101447458456336\\
97	0.0101447729507119\\
98	0.0101448005469413\\
99	0.0101448286432065\\
100	0.0101448572485525\\
101	0.0101448863721874\\
102	0.0101449160234851\\
103	0.0101449462119888\\
104	0.0101449769474133\\
105	0.0101450082396489\\
106	0.0101450400987639\\
107	0.0101450725350081\\
108	0.0101451055588159\\
109	0.01014513918081\\
110	0.010145173411804\\
111	0.0101452082628068\\
112	0.0101452437450251\\
113	0.0101452798698677\\
114	0.0101453166489489\\
115	0.0101453540940921\\
116	0.0101453922173334\\
117	0.0101454310309259\\
118	0.010145470547343\\
119	0.0101455107792828\\
120	0.0101455517396719\\
121	0.0101455934416695\\
122	0.0101456358986716\\
123	0.0101456791243153\\
124	0.010145723132483\\
125	0.010145767937307\\
126	0.0101458135531735\\
127	0.0101458599947277\\
128	0.0101459072768782\\
129	0.0101459554148017\\
130	0.0101460044239475\\
131	0.0101460543200432\\
132	0.0101461051190986\\
133	0.0101461568374118\\
134	0.0101462094915734\\
135	0.0101462630984726\\
136	0.0101463176753017\\
137	0.0101463732395623\\
138	0.0101464298090702\\
139	0.0101464874019615\\
140	0.0101465460366978\\
141	0.0101466057320725\\
142	0.0101466665072166\\
143	0.0101467283816045\\
144	0.0101467913750604\\
145	0.0101468555077645\\
146	0.0101469208002591\\
147	0.0101469872734555\\
148	0.0101470549486401\\
149	0.0101471238474816\\
150	0.0101471939920373\\
151	0.0101472654047606\\
152	0.0101473381085074\\
153	0.0101474121265441\\
154	0.0101474874825541\\
155	0.0101475642006457\\
156	0.0101476423053596\\
157	0.0101477218216767\\
158	0.0101478027750256\\
159	0.010147885191291\\
160	0.0101479690968217\\
161	0.0101480545184386\\
162	0.0101481414834436\\
163	0.0101482300196276\\
164	0.0101483201552797\\
165	0.0101484119191959\\
166	0.0101485053406881\\
167	0.0101486004495932\\
168	0.0101486972762827\\
169	0.0101487958516719\\
170	0.01014889620723\\
171	0.0101489983749895\\
172	0.0101491023875564\\
173	0.0101492082781206\\
174	0.010149316080466\\
175	0.0101494258289814\\
176	0.0101495375586709\\
177	0.0101496513051649\\
178	0.0101497671047315\\
179	0.0101498849942877\\
180	0.010150005011411\\
181	0.0101501271943507\\
182	0.0101502515820406\\
183	0.0101503782141105\\
184	0.0101505071308992\\
185	0.0101506383734664\\
186	0.0101507719836059\\
187	0.0101509080038589\\
188	0.0101510464775266\\
189	0.0101511874486843\\
190	0.010151330962195\\
191	0.0101514770637232\\
192	0.0101516257997493\\
193	0.0101517772175841\\
194	0.0101519313653835\\
195	0.0101520882921635\\
196	0.0101522480478154\\
197	0.0101524106831217\\
198	0.0101525762497715\\
199	0.0101527448003767\\
200	0.0101529163884886\\
201	0.0101530910686143\\
202	0.010153268896234\\
203	0.0101534499278179\\
204	0.0101536342208441\\
205	0.0101538218338163\\
206	0.010154012826282\\
207	0.0101542072588514\\
208	0.0101544051932159\\
209	0.0101546066921672\\
210	0.0101548118196174\\
211	0.0101550206406184\\
212	0.0101552332213824\\
213	0.0101554496293024\\
214	0.0101556699329734\\
215	0.0101558942022134\\
216	0.0101561225080857\\
217	0.0101563549229206\\
218	0.0101565915203381\\
219	0.010156832375271\\
220	0.0101570775639879\\
221	0.0101573271641175\\
222	0.0101575812546725\\
223	0.0101578399160745\\
224	0.0101581032301788\\
225	0.0101583712803005\\
226	0.0101586441512399\\
227	0.0101589219293099\\
228	0.0101592047023622\\
229	0.0101594925598153\\
230	0.0101597855926826\\
231	0.0101600838936007\\
232	0.0101603875568586\\
233	0.0101606966784276\\
234	0.0101610113559911\\
235	0.0101613316889758\\
236	0.0101616577785826\\
237	0.0101619897278191\\
238	0.0101623276415318\\
239	0.0101626716264394\\
240	0.0101630217911665\\
241	0.0101633782462786\\
242	0.0101637411043163\\
243	0.0101641104798322\\
244	0.0101644864894266\\
245	0.0101648692517853\\
246	0.0101652588877171\\
247	0.010165655520193\\
248	0.0101660592743853\\
249	0.0101664702777079\\
250	0.0101668886598575\\
251	0.0101673145528555\\
252	0.0101677480910905\\
253	0.0101681894113619\\
254	0.0101686386529243\\
255	0.0101690959575331\\
256	0.0101695614694902\\
257	0.0101700353356913\\
258	0.0101705177056742\\
259	0.0101710087316675\\
260	0.0101715085686408\\
261	0.0101720173743553\\
262	0.0101725353094163\\
263	0.0101730625373258\\
264	0.0101735992245366\\
265	0.0101741455405075\\
266	0.0101747016577595\\
267	0.0101752677519328\\
268	0.0101758440018454\\
269	0.0101764305895529\\
270	0.0101770277004085\\
271	0.0101776355231263\\
272	0.0101782542498442\\
273	0.0101788840761891\\
274	0.0101795252013414\\
275	0.0101801778280897\\
276	0.0101808421628714\\
277	0.0101815184158827\\
278	0.0101822068012022\\
279	0.0101829075368072\\
280	0.0101836208446486\\
281	0.0101843469507277\\
282	0.0101850860851745\\
283	0.0101858384823286\\
284	0.0101866043808211\\
285	0.0101873840236591\\
286	0.0101881776583126\\
287	0.0101889855368028\\
288	0.0101898079157935\\
289	0.0101906450566845\\
290	0.0101914972257075\\
291	0.0101923646940257\\
292	0.0101932477378342\\
293	0.0101941466384657\\
294	0.0101950616824979\\
295	0.0101959931618645\\
296	0.0101969413739702\\
297	0.010197906621809\\
298	0.0101988892140869\\
299	0.0101998894653484\\
300	0.0102009076961081\\
301	0.0102019442329865\\
302	0.0102029994088518\\
303	0.0102040735629662\\
304	0.0102051670411389\\
305	0.0102062801958849\\
306	0.0102074133865907\\
307	0.010208566979686\\
308	0.0102097413488239\\
309	0.0102109368750664\\
310	0.0102121539470784\\
311	0.0102133929613301\\
312	0.0102146543223186\\
313	0.0102159384428383\\
314	0.0102172457443439\\
315	0.0102185766572453\\
316	0.0102199316204754\\
317	0.0102213110822513\\
318	0.010222715500353\\
319	0.0102241453424153\\
320	0.0102256010862335\\
321	0.010227083220083\\
322	0.0102285922430552\\
323	0.0102301286654081\\
324	0.0102316930089353\\
325	0.0102332858073515\\
326	0.0102349076066967\\
327	0.0102365589657609\\
328	0.0102382404565271\\
329	0.0102399526646372\\
330	0.0102416961898784\\
331	0.0102434716466929\\
332	0.0102452796647128\\
333	0.0102471208893248\\
334	0.0102489959822803\\
335	0.0102509056224011\\
336	0.0102528505065336\\
337	0.0102548313512401\\
338	0.0102568488967948\\
339	0.010258903918802\\
340	0.0102609972675927\\
341	0.0102631300339618\\
342	0.0102653028529559\\
343	0.0102675151877864\\
344	0.0102697678584674\\
345	0.0102720617428466\\
346	0.0102743978672238\\
347	0.0102767777604592\\
348	0.0102792049484136\\
349	0.0102816920028306\\
350	0.0102843005072455\\
351	0.0102870664702293\\
352	0.0102899277453679\\
353	0.0102928389923943\\
354	0.0102958011327493\\
355	0.0102988151088651\\
356	0.0103018818849584\\
357	0.0103050024478722\\
358	0.0103081778077773\\
359	0.0103114089978145\\
360	0.0103146970723185\\
361	0.0103180431168744\\
362	0.010321448243885\\
363	0.0103249135936208\\
364	0.0103284403353321\\
365	0.0103320296684272\\
366	0.0103356828237209\\
367	0.0103394010647611\\
368	0.0103431856892365\\
369	0.0103470380304741\\
370	0.0103509594590322\\
371	0.0103549513843953\\
372	0.0103590152567756\\
373	0.0103631525690239\\
374	0.0103673648586404\\
375	0.0103716537098718\\
376	0.0103760207558824\\
377	0.0103804676811158\\
378	0.0103849962245748\\
379	0.0103896081868925\\
380	0.0103943054470678\\
381	0.0103990899238339\\
382	0.0104039636024742\\
383	0.0104089285361219\\
384	0.010413986842631\\
385	0.0104191406903532\\
386	0.0104243922558833\\
387	0.0104297436113626\\
388	0.0104351964228281\\
389	0.0104407509761606\\
390	0.0104464196538632\\
391	0.0104522080299345\\
392	0.0104581194279328\\
393	0.0104641572857469\\
394	0.0104703251558597\\
395	0.0104766267037832\\
396	0.01048306570269\\
397	0.0104896460186078\\
398	0.0104963715693859\\
399	0.0105032462061919\\
400	0.0105102733588599\\
401	0.0105174549464263\\
402	0.0105247879530466\\
403	0.0105322533669631\\
404	0.0105397788970637\\
405	0.0105471029250194\\
406	0.0105535268595702\\
407	0.0105601291534413\\
408	0.0105669162125581\\
409	0.0105738943422493\\
410	0.0105810700748047\\
411	0.0105884499409389\\
412	0.0105960404610984\\
413	0.0106038658691581\\
414	0.0106122895659858\\
415	0.0106208573604243\\
416	0.0106295710251152\\
417	0.0106384322996213\\
418	0.0106474428845753\\
419	0.0106566044350936\\
420	0.0106659185530677\\
421	0.010675386777421\\
422	0.0106850105703862\\
423	0.0106947912976457\\
424	0.010704730215568\\
425	0.0107148286257336\\
426	0.0107250879841503\\
427	0.0107355092024648\\
428	0.0107460930581462\\
429	0.0107568401812159\\
430	0.0107677510399862\\
431	0.0107788259257431\\
432	0.0107900649363112\\
433	0.0108014679584443\\
434	0.0108130346490223\\
435	0.0108247644151944\\
436	0.0108366563942658\\
437	0.0108487094369896\\
438	0.010860922111729\\
439	0.0108732928239393\\
440	0.0108858207015065\\
441	0.0108985019412572\\
442	0.0109113297678965\\
443	0.0109243005956562\\
444	0.0109374103444363\\
445	0.0109506544058948\\
446	0.0109640276099815\\
447	0.0109775241975619\\
448	0.0109911378224695\\
449	0.0110048616870874\\
450	0.0110186893065778\\
451	0.0110326093642179\\
452	0.0110466113642805\\
453	0.0110606851567003\\
454	0.0110748195903616\\
455	0.0110890020489438\\
456	0.0111032166937586\\
457	0.0111174374262609\\
458	0.0111310565936648\\
459	0.011144832545784\\
460	0.0111587758046904\\
461	0.0111728884152916\\
462	0.0111871676077102\\
463	0.0112016102047753\\
464	0.0112162126175109\\
465	0.0112309710075095\\
466	0.0112458811319446\\
467	0.0112609383465852\\
468	0.0112761376221032\\
469	0.0112914735669991\\
470	0.0113069404449298\\
471	0.0113225321037189\\
472	0.011337796997148\\
473	0.0113531357235195\\
474	0.0113686819050201\\
475	0.011384434526993\\
476	0.0114003923651173\\
477	0.0114165539977096\\
478	0.0114329177748791\\
479	0.0114494815644976\\
480	0.0114662429064611\\
481	0.0114832000198309\\
482	0.0115003480617858\\
483	0.0115176828887694\\
484	0.0115352013489387\\
485	0.0115529002564451\\
486	0.0115707835106369\\
487	0.0115888716723064\\
488	0.0116071607815357\\
489	0.0116256455427747\\
490	0.0116443210030474\\
491	0.0116631904884664\\
492	0.0116822489302158\\
493	0.0117014910401946\\
494	0.0117209115558965\\
495	0.0117405027695083\\
496	0.0117602584080895\\
497	0.011780171903838\\
498	0.011800236200024\\
499	0.011820444148541\\
500	0.0118407935260595\\
501	0.0118612772505659\\
502	0.0118818882735183\\
503	0.0119026197141651\\
504	0.0119234649558649\\
505	0.0119444175491668\\
506	0.0119648998097392\\
507	0.0119854864178446\\
508	0.0120063293698489\\
509	0.0120274371614115\\
510	0.012048816863781\\
511	0.012070463165724\\
512	0.0120923735388314\\
513	0.012114547599813\\
514	0.0121369835697775\\
515	0.0121596803118904\\
516	0.0121826356160977\\
517	0.0122058442662609\\
518	0.0122292996415255\\
519	0.0122529930214906\\
520	0.0122769126604267\\
521	0.0123010450348892\\
522	0.0123258353070334\\
523	0.012351865357181\\
524	0.0123776251241155\\
525	0.0124030882997499\\
526	0.0124282280941118\\
527	0.0124530159285305\\
528	0.012477422428172\\
529	0.0125014173827645\\
530	0.0125249700318251\\
531	0.0125480542233763\\
532	0.0125706527362421\\
533	0.0125926624148569\\
534	0.0126131517479606\\
535	0.0126332368725144\\
536	0.0126528951154892\\
537	0.0126721041700234\\
538	0.0126908422164513\\
539	0.0127090880194813\\
540	0.0127268210890813\\
541	0.0127440219513844\\
542	0.012760671536731\\
543	0.0127767519874682\\
544	0.012792246830458\\
545	0.0128065450720889\\
546	0.0128202368570078\\
547	0.0128335546706431\\
548	0.0128464796173316\\
549	0.0128589922721929\\
550	0.0128710728488695\\
551	0.0128826940736109\\
552	0.0128938462067012\\
553	0.0129049029111274\\
554	0.0129158596195024\\
555	0.0129268489577111\\
556	0.0129379717486856\\
557	0.0129492673407897\\
558	0.012960739688428\\
559	0.0129723928196637\\
560	0.0129842307690847\\
561	0.0129962576435906\\
562	0.0130084775898677\\
563	0.0130208947316017\\
564	0.0130335130929776\\
565	0.0130463365095152\\
566	0.013059368561547\\
567	0.0130726124286204\\
568	0.0130860700317245\\
569	0.0130997412816743\\
570	0.0131136255795875\\
571	0.0131277224368694\\
572	0.0131420332510873\\
573	0.0131565596205997\\
574	0.01317130349398\\
575	0.0131862674076862\\
576	0.0132027377827662\\
577	0.0132193260454859\\
578	0.0132359221615345\\
579	0.013253010425182\\
580	0.0132701523258826\\
581	0.013287008557868\\
582	0.0133033612106936\\
583	0.0133193574536422\\
584	0.0133344529366027\\
585	0.0133492502469239\\
586	0.0133643892379398\\
587	0.0133798012348973\\
588	0.0133938336010688\\
589	0.0134067477204202\\
590	0.0134188993505772\\
591	0.0134303077081716\\
592	0.0134403402753647\\
593	0.0134503526516272\\
594	0.0134614306417842\\
595	0.0134740407759367\\
596	0.0134889777516421\\
597	0.0135160557778789\\
598	0.0135751148335434\\
599	0\\
600	0\\
};
\addplot [color=red!25!mycolor17,solid,forget plot]
  table[row sep=crcr]{%
1	0.0106118810148568\\
2	0.0106118860083902\\
3	0.0106118910926437\\
4	0.0106118962692665\\
5	0.0106119015399378\\
6	0.0106119069063671\\
7	0.0106119123702954\\
8	0.0106119179334951\\
9	0.0106119235977711\\
10	0.0106119293649611\\
11	0.0106119352369363\\
12	0.0106119412156019\\
13	0.0106119473028979\\
14	0.0106119535007995\\
15	0.0106119598113181\\
16	0.0106119662365016\\
17	0.0106119727784353\\
18	0.0106119794392422\\
19	0.0106119862210844\\
20	0.0106119931261632\\
21	0.0106120001567199\\
22	0.0106120073150367\\
23	0.0106120146034375\\
24	0.0106120220242884\\
25	0.0106120295799987\\
26	0.0106120372730215\\
27	0.0106120451058547\\
28	0.0106120530810416\\
29	0.010612061201172\\
30	0.0106120694688827\\
31	0.0106120778868587\\
32	0.0106120864578339\\
33	0.0106120951845918\\
34	0.0106121040699669\\
35	0.0106121131168453\\
36	0.0106121223281655\\
37	0.0106121317069197\\
38	0.0106121412561546\\
39	0.0106121509789724\\
40	0.0106121608785317\\
41	0.0106121709580489\\
42	0.0106121812207989\\
43	0.010612191670116\\
44	0.0106122023093958\\
45	0.0106122131420951\\
46	0.0106122241717343\\
47	0.0106122354018975\\
48	0.0106122468362343\\
49	0.0106122584784608\\
50	0.0106122703323607\\
51	0.0106122824017865\\
52	0.0106122946906613\\
53	0.0106123072029793\\
54	0.0106123199428075\\
55	0.0106123329142872\\
56	0.0106123461216347\\
57	0.0106123595691437\\
58	0.0106123732611855\\
59	0.0106123872022113\\
60	0.0106124013967535\\
61	0.0106124158494268\\
62	0.01061243056493\\
63	0.0106124455480475\\
64	0.0106124608036508\\
65	0.0106124763367001\\
66	0.010612492152246\\
67	0.0106125082554308\\
68	0.0106125246514906\\
69	0.0106125413457569\\
70	0.0106125583436582\\
71	0.0106125756507216\\
72	0.0106125932725752\\
73	0.0106126112149491\\
74	0.010612629483678\\
75	0.0106126480847027\\
76	0.010612667024072\\
77	0.010612686307945\\
78	0.0106127059425927\\
79	0.0106127259344001\\
80	0.0106127462898686\\
81	0.0106127670156178\\
82	0.0106127881183876\\
83	0.0106128096050406\\
84	0.0106128314825642\\
85	0.0106128537580728\\
86	0.0106128764388105\\
87	0.0106128995321526\\
88	0.0106129230456089\\
89	0.0106129469868257\\
90	0.0106129713635882\\
91	0.0106129961838232\\
92	0.0106130214556015\\
93	0.0106130471871406\\
94	0.0106130733868076\\
95	0.0106131000631213\\
96	0.0106131272247555\\
97	0.0106131548805416\\
98	0.0106131830394713\\
99	0.0106132117106999\\
100	0.0106132409035488\\
101	0.0106132706275088\\
102	0.010613300892243\\
103	0.01061333170759\\
104	0.0106133630835672\\
105	0.0106133950303737\\
106	0.0106134275583938\\
107	0.0106134606782003\\
108	0.010613494400558\\
109	0.0106135287364271\\
110	0.0106135636969664\\
111	0.0106135992935376\\
112	0.0106136355377083\\
113	0.0106136724412559\\
114	0.0106137100161717\\
115	0.0106137482746642\\
116	0.0106137872291636\\
117	0.0106138268923253\\
118	0.0106138672770343\\
119	0.010613908396409\\
120	0.0106139502638059\\
121	0.0106139928928234\\
122	0.0106140362973064\\
123	0.0106140804913507\\
124	0.0106141254893075\\
125	0.010614171305788\\
126	0.0106142179556682\\
127	0.0106142654540933\\
128	0.010614313816483\\
129	0.0106143630585362\\
130	0.0106144131962359\\
131	0.0106144642458546\\
132	0.0106145162239593\\
133	0.0106145691474168\\
134	0.0106146230333993\\
135	0.0106146778993894\\
136	0.0106147337631863\\
137	0.0106147906429111\\
138	0.0106148485570125\\
139	0.010614907524273\\
140	0.0106149675638146\\
141	0.010615028695105\\
142	0.0106150909379639\\
143	0.0106151543125691\\
144	0.0106152188394632\\
145	0.0106152845395597\\
146	0.0106153514341502\\
147	0.0106154195449106\\
148	0.0106154888939084\\
149	0.0106155595036097\\
150	0.0106156313968858\\
151	0.0106157045970213\\
152	0.0106157791277208\\
153	0.0106158550131165\\
154	0.0106159322777764\\
155	0.0106160109467112\\
156	0.010616091045383\\
157	0.0106161725997129\\
158	0.0106162556360893\\
159	0.0106163401813764\\
160	0.0106164262629225\\
161	0.0106165139085685\\
162	0.010616603146657\\
163	0.0106166940060412\\
164	0.0106167865160937\\
165	0.0106168807067161\\
166	0.0106169766083481\\
167	0.0106170742519773\\
168	0.0106171736691488\\
169	0.0106172748919752\\
170	0.0106173779531469\\
171	0.0106174828859416\\
172	0.0106175897242357\\
173	0.0106176985025142\\
174	0.0106178092558819\\
175	0.0106179220200742\\
176	0.0106180368314683\\
177	0.0106181537270945\\
178	0.010618272744648\\
179	0.0106183939225004\\
180	0.0106185172997118\\
181	0.0106186429160428\\
182	0.0106187708119673\\
183	0.0106189010286845\\
184	0.010619033608132\\
185	0.0106191685929991\\
186	0.0106193060267394\\
187	0.0106194459535847\\
188	0.0106195884185588\\
189	0.0106197334674911\\
190	0.0106198811470308\\
191	0.0106200315046615\\
192	0.0106201845887159\\
193	0.0106203404483901\\
194	0.0106204991337597\\
195	0.0106206606957942\\
196	0.0106208251863734\\
197	0.010620992658303\\
198	0.0106211631653308\\
199	0.0106213367621631\\
200	0.0106215135044816\\
201	0.0106216934489602\\
202	0.0106218766532824\\
203	0.010622063176159\\
204	0.0106222530773456\\
205	0.010622446417661\\
206	0.0106226432590057\\
207	0.0106228436643807\\
208	0.0106230476979061\\
209	0.0106232554248411\\
210	0.0106234669116033\\
211	0.0106236822257888\\
212	0.0106239014361927\\
213	0.0106241246128294\\
214	0.0106243518269541\\
215	0.0106245831510835\\
216	0.0106248186590183\\
217	0.0106250584258644\\
218	0.0106253025280559\\
219	0.0106255510433777\\
220	0.0106258040509881\\
221	0.0106260616314433\\
222	0.0106263238667203\\
223	0.0106265908402416\\
224	0.0106268626368996\\
225	0.010627139343082\\
226	0.0106274210466967\\
227	0.0106277078371979\\
228	0.0106279998056124\\
229	0.0106282970445658\\
230	0.0106285996483102\\
231	0.0106289077127511\\
232	0.0106292213354756\\
233	0.0106295406157806\\
234	0.0106298656547018\\
235	0.0106301965550427\\
236	0.0106305334214048\\
237	0.010630876360217\\
238	0.0106312254797671\\
239	0.0106315808902325\\
240	0.0106319427037119\\
241	0.0106323110342575\\
242	0.010632685997908\\
243	0.0106330677127212\\
244	0.0106334562988083\\
245	0.010633851878368\\
246	0.0106342545757216\\
247	0.010634664517348\\
248	0.0106350818319202\\
249	0.010635506650342\\
250	0.0106359391057849\\
251	0.0106363793337266\\
252	0.010636827471989\\
253	0.0106372836607783\\
254	0.010637748042724\\
255	0.0106382207629208\\
256	0.0106387019689689\\
257	0.0106391918110173\\
258	0.010639690441806\\
259	0.0106401980167104\\
260	0.0106407146937859\\
261	0.0106412406338131\\
262	0.0106417760003446\\
263	0.0106423209597525\\
264	0.0106428756812758\\
265	0.0106434403370706\\
266	0.0106440151022595\\
267	0.0106446001549829\\
268	0.010645195676451\\
269	0.0106458018509966\\
270	0.0106464188661295\\
271	0.0106470469125917\\
272	0.0106476861844155\\
273	0.0106483368789849\\
274	0.0106489991971045\\
275	0.0106496733430687\\
276	0.0106503595246674\\
277	0.0106510579528655\\
278	0.0106517688421623\\
279	0.0106524924113025\\
280	0.0106532288827512\\
281	0.0106539784827518\\
282	0.0106547414413842\\
283	0.010655517992624\\
284	0.0106563083744015\\
285	0.0106571128286613\\
286	0.0106579316014228\\
287	0.0106587649428398\\
288	0.0106596131072611\\
289	0.0106604763532912\\
290	0.0106613549438508\\
291	0.0106622491462371\\
292	0.0106631592321841\\
293	0.0106640854779232\\
294	0.0106650281642423\\
295	0.0106659875765456\\
296	0.0106669640049118\\
297	0.0106679577441526\\
298	0.0106689690938695\\
299	0.0106699983585101\\
300	0.0106710458474235\\
301	0.0106721118749139\\
302	0.0106731967602937\\
303	0.0106743008279346\\
304	0.010675424407318\\
305	0.010676567833083\\
306	0.0106777314450737\\
307	0.0106789155883842\\
308	0.0106801206134012\\
309	0.0106813468758425\\
310	0.0106825947367876\\
311	0.0106838645626918\\
312	0.0106851567253783\\
313	0.0106864716020311\\
314	0.0106878095754062\\
315	0.0106891710350984\\
316	0.0106905563795825\\
317	0.0106919660081205\\
318	0.0106934003259084\\
319	0.0106948597441366\\
320	0.0106963446800506\\
321	0.0106978555570098\\
322	0.0106993928045455\\
323	0.0107009568584179\\
324	0.0107025481606708\\
325	0.0107041671596848\\
326	0.0107058143102274\\
327	0.0107074900735009\\
328	0.0107091949171853\\
329	0.010710929315478\\
330	0.0107126937491269\\
331	0.0107144887054575\\
332	0.0107163146783914\\
333	0.0107181721684569\\
334	0.0107200616827896\\
335	0.0107219837351306\\
336	0.010723938845848\\
337	0.0107259275420947\\
338	0.0107279503585661\\
339	0.0107300078409435\\
340	0.0107321005627026\\
341	0.0107342292239689\\
342	0.0107363943480571\\
343	0.0107385958861782\\
344	0.0107408343932305\\
345	0.0107431104299265\\
346	0.0107454245626258\\
347	0.010747777363131\\
348	0.0107501694084429\\
349	0.0107526012805055\\
350	0.0107550735662609\\
351	0.0107575868679606\\
352	0.0107601418041516\\
353	0.0107627389999947\\
354	0.0107653790872289\\
355	0.0107680627041626\\
356	0.0107707904957619\\
357	0.0107735631140218\\
358	0.0107763812190675\\
359	0.0107792454811975\\
360	0.0107821565739881\\
361	0.0107851151247211\\
362	0.0107881218220222\\
363	0.0107911773602134\\
364	0.0107942824390169\\
365	0.0107974377632159\\
366	0.0108006440422689\\
367	0.0108039019898689\\
368	0.0108072123234445\\
369	0.0108105757635928\\
370	0.0108139930334371\\
371	0.010817464857899\\
372	0.0108209919628695\\
373	0.0108245750742546\\
374	0.0108282149168441\\
375	0.0108319122128838\\
376	0.0108356676800647\\
377	0.0108394820282724\\
378	0.0108433559539194\\
379	0.0108472901320757\\
380	0.010851285228192\\
381	0.010855342141211\\
382	0.0108594615026474\\
383	0.0108636439247445\\
384	0.0108678899989933\\
385	0.0108722002891287\\
386	0.0108765753131043\\
387	0.010881015487897\\
388	0.0108855209364959\\
389	0.0108900906808207\\
390	0.0108947293294273\\
391	0.0108994384454068\\
392	0.0109042184936531\\
393	0.0109090698933269\\
394	0.0109139930137216\\
395	0.0109189881700072\\
396	0.010924055618897\\
397	0.0109291955542944\\
398	0.0109344081029983\\
399	0.010939693320565\\
400	0.0109450511874538\\
401	0.0109504816056073\\
402	0.0109559843956196\\
403	0.0109615592945567\\
404	0.0109672059540069\\
405	0.0109729239359125\\
406	0.0109787126903555\\
407	0.0109845714473305\\
408	0.0109904994688752\\
409	0.0109964963741616\\
410	0.0110025592449037\\
411	0.011008686953288\\
412	0.0110148782860777\\
413	0.0110211218543512\\
414	0.0110272124634939\\
415	0.011033403370867\\
416	0.0110396956533818\\
417	0.0110460903728176\\
418	0.0110525885748834\\
419	0.0110591912881104\\
420	0.0110658995219711\\
421	0.011072714262334\\
422	0.0110796364584252\\
423	0.0110866669831162\\
424	0.0110938065086708\\
425	0.0111010551110109\\
426	0.011108413941341\\
427	0.011115888277376\\
428	0.0111234789162698\\
429	0.0111311866042172\\
430	0.0111390120335635\\
431	0.0111469558399703\\
432	0.0111550185996715\\
433	0.0111632008268637\\
434	0.0111715029712725\\
435	0.011179925415924\\
436	0.011188468475186\\
437	0.0111971323931798\\
438	0.0112059173425158\\
439	0.0112148234231933\\
440	0.0112238506605836\\
441	0.0112329990288847\\
442	0.0112422684650907\\
443	0.011251658841322\\
444	0.011261169968479\\
445	0.0112708016013562\\
446	0.0112805534457133\\
447	0.0112904251693042\\
448	0.0113004164263859\\
449	0.0113105269413681\\
450	0.0113207569657761\\
451	0.011331104920786\\
452	0.0113415699663022\\
453	0.0113521518584216\\
454	0.011362850437223\\
455	0.0113736656422712\\
456	0.0113845975019663\\
457	0.0113956462149702\\
458	0.0114068201012005\\
459	0.0114181166633648\\
460	0.0114295375127436\\
461	0.0114410832029517\\
462	0.0114527522639768\\
463	0.011464543201394\\
464	0.0114764545193957\\
465	0.0114884847953951\\
466	0.0115006327062504\\
467	0.0115128970455312\\
468	0.0115252767847158\\
469	0.0115377711302869\\
470	0.0115503795495704\\
471	0.011563101670566\\
472	0.0115755976807566\\
473	0.0115882001439084\\
474	0.0116010202544778\\
475	0.0116140605745417\\
476	0.0116273238798091\\
477	0.0116408127119817\\
478	0.0116545295377051\\
479	0.0116684768044543\\
480	0.0116826569483087\\
481	0.0116970724148852\\
482	0.0117117257691259\\
483	0.0117266193775953\\
484	0.0117417544509667\\
485	0.0117571330750902\\
486	0.0117727567101795\\
487	0.0117886298437595\\
488	0.0118047550760018\\
489	0.0118211339007117\\
490	0.011837767558486\\
491	0.0118546570139821\\
492	0.0118718029897175\\
493	0.0118892060005306\\
494	0.0119068664993603\\
495	0.0119247837122073\\
496	0.011942957382418\\
497	0.0119613869949593\\
498	0.0119800716843142\\
499	0.0119990103335995\\
500	0.0120182035427005\\
501	0.0120376548269237\\
502	0.0120573684259205\\
503	0.0120773476901518\\
504	0.0120975951168858\\
505	0.012118112180513\\
506	0.0121389099196749\\
507	0.0121599872239522\\
508	0.0121813383573198\\
509	0.0122029558282275\\
510	0.012224828193531\\
511	0.0122477291121339\\
512	0.0122713864550201\\
513	0.0122948475956738\\
514	0.0123180885618494\\
515	0.012341084307241\\
516	0.012363808750135\\
517	0.0123862348531717\\
518	0.0124083347783927\\
519	0.0124300798698871\\
520	0.0124514407075311\\
521	0.0124723877471211\\
522	0.0124926180680551\\
523	0.0125117262421685\\
524	0.0125305169289231\\
525	0.0125489744278018\\
526	0.0125670837191241\\
527	0.012584830616434\\
528	0.012602201847976\\
529	0.0126191852959122\\
530	0.0126357702239889\\
531	0.0126519472870393\\
532	0.0126677084749431\\
533	0.0126829988156499\\
534	0.0126971869694241\\
535	0.0127111858830165\\
536	0.0127249879971069\\
537	0.0127385859261309\\
538	0.0127519724329252\\
539	0.0127651403910796\\
540	0.0127780827307337\\
541	0.0127907923608671\\
542	0.0128032620862291\\
543	0.0128154844882925\\
544	0.0128274517754999\\
545	0.0128391813037485\\
546	0.0128506648505122\\
547	0.0128618822363213\\
548	0.0128728117435316\\
549	0.012883430252791\\
550	0.0128937030963321\\
551	0.0129038504641312\\
552	0.0129140184754871\\
553	0.0129243824503119\\
554	0.0129349487963308\\
555	0.0129457237688582\\
556	0.0129567129855521\\
557	0.0129679204772345\\
558	0.0129793502337498\\
559	0.0129910061859714\\
560	0.0130028921879905\\
561	0.013015011997797\\
562	0.0130273692250912\\
563	0.0130399672839607\\
564	0.01305280935119\\
565	0.0130658983185893\\
566	0.0130792371516619\\
567	0.0130928313304846\\
568	0.0131066869229398\\
569	0.0131208108625806\\
570	0.013135528507557\\
571	0.0131515524531138\\
572	0.0131676617139305\\
573	0.0131838436215691\\
574	0.0132000978213791\\
575	0.0132172308088553\\
576	0.0132334596928721\\
577	0.0132494718756232\\
578	0.0132655223898101\\
579	0.0132812630730379\\
580	0.0132964667288486\\
581	0.0133119372502694\\
582	0.0133278177619972\\
583	0.0133433021010561\\
584	0.0133579650643655\\
585	0.0133720933187266\\
586	0.0133849338761277\\
587	0.0133966572273018\\
588	0.013407796560705\\
589	0.013418470452656\\
590	0.0134287051149722\\
591	0.0134382039281232\\
592	0.0134466094533236\\
593	0.0134548583791426\\
594	0.0134637112677116\\
595	0.0134740407759367\\
596	0.0134889777516421\\
597	0.0135160557778789\\
598	0.0135751148335434\\
599	0\\
600	0\\
};
\addplot [color=mycolor19,solid,forget plot]
  table[row sep=crcr]{%
1	0.0108784198979435\\
2	0.0108784238033297\\
3	0.0108784277798183\\
4	0.010878431828705\\
5	0.0108784359513092\\
6	0.0108784401489745\\
7	0.0108784444230692\\
8	0.0108784487749868\\
9	0.0108784532061459\\
10	0.0108784577179917\\
11	0.0108784623119954\\
12	0.0108784669896555\\
13	0.0108784717524978\\
14	0.0108784766020762\\
15	0.0108784815399732\\
16	0.0108784865678001\\
17	0.0108784916871979\\
18	0.0108784968998379\\
19	0.0108785022074219\\
20	0.0108785076116828\\
21	0.0108785131143857\\
22	0.0108785187173279\\
23	0.0108785244223397\\
24	0.0108785302312852\\
25	0.0108785361460625\\
26	0.0108785421686049\\
27	0.0108785483008811\\
28	0.0108785545448958\\
29	0.0108785609026908\\
30	0.0108785673763455\\
31	0.0108785739679774\\
32	0.010878580679743\\
33	0.0108785875138385\\
34	0.0108785944725004\\
35	0.0108786015580065\\
36	0.0108786087726765\\
37	0.0108786161188728\\
38	0.0108786235990012\\
39	0.0108786312155119\\
40	0.0108786389709002\\
41	0.0108786468677073\\
42	0.0108786549085214\\
43	0.010878663095978\\
44	0.0108786714327616\\
45	0.0108786799216059\\
46	0.010878688565295\\
47	0.0108786973666643\\
48	0.0108787063286016\\
49	0.0108787154540478\\
50	0.010878724745998\\
51	0.0108787342075026\\
52	0.0108787438416683\\
53	0.0108787536516589\\
54	0.0108787636406967\\
55	0.0108787738120634\\
56	0.0108787841691012\\
57	0.0108787947152141\\
58	0.0108788054538685\\
59	0.0108788163885953\\
60	0.0108788275229901\\
61	0.0108788388607152\\
62	0.0108788504055001\\
63	0.0108788621611436\\
64	0.0108788741315141\\
65	0.0108788863205518\\
66	0.0108788987322694\\
67	0.0108789113707538\\
68	0.0108789242401673\\
69	0.0108789373447491\\
70	0.0108789506888165\\
71	0.0108789642767667\\
72	0.010878978113078\\
73	0.0108789922023113\\
74	0.010879006549112\\
75	0.010879021158211\\
76	0.0108790360344267\\
77	0.0108790511826664\\
78	0.010879066607928\\
79	0.010879082315302\\
80	0.0108790983099725\\
81	0.0108791145972195\\
82	0.0108791311824207\\
83	0.0108791480710528\\
84	0.010879165268694\\
85	0.0108791827810252\\
86	0.0108792006138323\\
87	0.0108792187730083\\
88	0.0108792372645547\\
89	0.0108792560945841\\
90	0.0108792752693219\\
91	0.0108792947951083\\
92	0.010879314678401\\
93	0.0108793349257766\\
94	0.0108793555439332\\
95	0.0108793765396928\\
96	0.0108793979200032\\
97	0.0108794196919405\\
98	0.0108794418627117\\
99	0.0108794644396564\\
100	0.0108794874302503\\
101	0.0108795108421066\\
102	0.0108795346829795\\
103	0.010879558960766\\
104	0.0108795836835091\\
105	0.0108796088594003\\
106	0.0108796344967822\\
107	0.0108796606041513\\
108	0.0108796871901612\\
109	0.0108797142636249\\
110	0.0108797418335184\\
111	0.0108797699089829\\
112	0.0108797984993287\\
113	0.0108798276140376\\
114	0.0108798572627665\\
115	0.0108798874553504\\
116	0.0108799182018056\\
117	0.0108799495123335\\
118	0.0108799813973232\\
119	0.0108800138673558\\
120	0.0108800469332073\\
121	0.0108800806058524\\
122	0.0108801148964684\\
123	0.0108801498164383\\
124	0.0108801853773552\\
125	0.0108802215910258\\
126	0.0108802584694744\\
127	0.0108802960249469\\
128	0.0108803342699148\\
129	0.0108803732170796\\
130	0.0108804128793766\\
131	0.0108804532699793\\
132	0.0108804944023041\\
133	0.0108805362900143\\
134	0.0108805789470248\\
135	0.0108806223875069\\
136	0.0108806666258925\\
137	0.0108807116768795\\
138	0.010880757555436\\
139	0.0108808042768057\\
140	0.010880851856513\\
141	0.0108809003103675\\
142	0.01088094965447\\
143	0.0108809999052172\\
144	0.0108810510793076\\
145	0.0108811031937466\\
146	0.0108811562658522\\
147	0.0108812103132609\\
148	0.0108812653539333\\
149	0.01088132140616\\
150	0.0108813784885678\\
151	0.0108814366201258\\
152	0.0108814958201512\\
153	0.0108815561083163\\
154	0.0108816175046545\\
155	0.0108816800295671\\
156	0.0108817437038298\\
157	0.0108818085485997\\
158	0.010881874585422\\
159	0.0108819418362374\\
160	0.010882010323389\\
161	0.0108820800696295\\
162	0.0108821510981292\\
163	0.0108822234324828\\
164	0.0108822970967176\\
165	0.0108823721153015\\
166	0.0108824485131504\\
167	0.0108825263156366\\
168	0.0108826055485975\\
169	0.0108826862383431\\
170	0.0108827684116654\\
171	0.0108828520958466\\
172	0.0108829373186683\\
173	0.0108830241084202\\
174	0.0108831124939094\\
175	0.01088320250447\\
176	0.010883294169972\\
177	0.0108833875208317\\
178	0.0108834825880209\\
179	0.0108835794030773\\
180	0.0108836779981144\\
181	0.0108837784058321\\
182	0.0108838806595273\\
183	0.0108839847931041\\
184	0.0108840908410853\\
185	0.0108841988386233\\
186	0.0108843088215112\\
187	0.0108844208261943\\
188	0.0108845348897822\\
189	0.0108846510500601\\
190	0.0108847693455011\\
191	0.0108848898152787\\
192	0.0108850124992789\\
193	0.0108851374381133\\
194	0.0108852646731315\\
195	0.0108853942464348\\
196	0.0108855262008892\\
197	0.0108856605801389\\
198	0.0108857974286206\\
199	0.0108859367915769\\
200	0.0108860787150708\\
201	0.0108862232460005\\
202	0.0108863704321138\\
203	0.010886520322023\\
204	0.0108866729652205\\
205	0.0108868284120938\\
206	0.0108869867139417\\
207	0.0108871479229899\\
208	0.0108873120924075\\
209	0.0108874792763233\\
210	0.0108876495298424\\
211	0.0108878229090638\\
212	0.010887999471097\\
213	0.0108881792740799\\
214	0.0108883623771967\\
215	0.0108885488406958\\
216	0.0108887387259082\\
217	0.0108889320952662\\
218	0.0108891290123224\\
219	0.0108893295417687\\
220	0.0108895337494562\\
221	0.0108897417024143\\
222	0.0108899534688717\\
223	0.0108901691182759\\
224	0.0108903887213145\\
225	0.0108906123499357\\
226	0.0108908400773699\\
227	0.0108910719781508\\
228	0.0108913081281377\\
229	0.0108915486045374\\
230	0.0108917934859265\\
231	0.0108920428522746\\
232	0.0108922967849665\\
233	0.0108925553668262\\
234	0.0108928186821403\\
235	0.0108930868166816\\
236	0.0108933598577338\\
237	0.0108936378941154\\
238	0.0108939210162049\\
239	0.0108942093159653\\
240	0.0108945028869702\\
241	0.0108948018244284\\
242	0.0108951062252106\\
243	0.0108954161878751\\
244	0.0108957318126941\\
245	0.0108960532016807\\
246	0.0108963804586157\\
247	0.0108967136890745\\
248	0.010897053000455\\
249	0.0108973985020049\\
250	0.0108977503048499\\
251	0.0108981085220217\\
252	0.0108984732684866\\
253	0.0108988446611744\\
254	0.0108992228190069\\
255	0.0108996078629279\\
256	0.010899999915932\\
257	0.010900399103095\\
258	0.0109008055516042\\
259	0.0109012193907886\\
260	0.0109016407521501\\
261	0.0109020697693951\\
262	0.0109025065784662\\
263	0.0109029513175749\\
264	0.0109034041272348\\
265	0.0109038651502957\\
266	0.0109043345319781\\
267	0.0109048124199093\\
268	0.0109052989641607\\
269	0.010905794317286\\
270	0.0109062986343622\\
271	0.0109068120730327\\
272	0.0109073347935559\\
273	0.0109078669588656\\
274	0.0109084087346604\\
275	0.0109089602895707\\
276	0.0109095217954896\\
277	0.01091009342786\\
278	0.0109106753608225\\
279	0.0109112677692281\\
280	0.010911870837355\\
281	0.0109124847526095\\
282	0.0109131097055786\\
283	0.0109137458900847\\
284	0.01091439350324\\
285	0.0109150527455036\\
286	0.0109157238207387\\
287	0.0109164069362715\\
288	0.0109171023029514\\
289	0.0109178101352125\\
290	0.0109185306511361\\
291	0.0109192640725147\\
292	0.0109200106249174\\
293	0.0109207705377558\\
294	0.010921544044352\\
295	0.0109223313820069\\
296	0.0109231327920697\\
297	0.0109239485200079\\
298	0.0109247788154782\\
299	0.0109256239323974\\
300	0.010926484129013\\
301	0.0109273596679739\\
302	0.0109282508163996\\
303	0.0109291578459477\\
304	0.0109300810328794\\
305	0.0109310206581215\\
306	0.010931977007324\\
307	0.0109329503709109\\
308	0.0109339410441221\\
309	0.0109349493270397\\
310	0.0109359755245847\\
311	0.0109370199464508\\
312	0.0109380829068948\\
313	0.0109391647242095\\
314	0.0109402657196373\\
315	0.0109413862163965\\
316	0.0109425265493869\\
317	0.0109436871051978\\
318	0.0109448682190518\\
319	0.0109460702306741\\
320	0.0109472934843042\\
321	0.010948538328704\\
322	0.0109498051171631\\
323	0.0109510942075015\\
324	0.0109524059620685\\
325	0.0109537407477376\\
326	0.0109550989358991\\
327	0.0109564809024469\\
328	0.0109578870277628\\
329	0.0109593176966948\\
330	0.0109607732985324\\
331	0.0109622542269752\\
332	0.0109637608800978\\
333	0.0109652936603085\\
334	0.0109668529743026\\
335	0.0109684392330102\\
336	0.0109700528515367\\
337	0.0109716942490973\\
338	0.0109733638489393\\
339	0.0109750620782429\\
340	0.0109767893679639\\
341	0.0109785461525363\\
342	0.0109803328720018\\
343	0.0109821499752176\\
344	0.0109839979147761\\
345	0.0109858771468368\\
346	0.0109877881309379\\
347	0.0109897313297781\\
348	0.0109917072089557\\
349	0.0109937162366201\\
350	0.0109957588828856\\
351	0.0109978356204898\\
352	0.0109999469246025\\
353	0.0110020932713079\\
354	0.0110042751371526\\
355	0.0110064929986853\\
356	0.0110087473320744\\
357	0.0110110386131371\\
358	0.0110133673190802\\
359	0.0110157339373973\\
360	0.0110181390073736\\
361	0.0110205831461489\\
362	0.0110230662340049\\
363	0.0110255887305893\\
364	0.0110281510924594\\
365	0.0110307537725681\\
366	0.0110333972197378\\
367	0.0110360818781258\\
368	0.0110388081866868\\
369	0.0110415765786356\\
370	0.0110443874809169\\
371	0.0110472413136864\\
372	0.0110501384898063\\
373	0.0110530794143414\\
374	0.0110560644840054\\
375	0.0110590940863745\\
376	0.0110621685983069\\
377	0.0110652883818427\\
378	0.0110684537722806\\
379	0.0110716650418028\\
380	0.0110749222850876\\
381	0.0110782250516535\\
382	0.0110815763520183\\
383	0.0110849764849384\\
384	0.0110884257011822\\
385	0.0110919242237003\\
386	0.0110954722459129\\
387	0.0110990699300318\\
388	0.0111027174054295\\
389	0.0111064147665355\\
390	0.0111101620326892\\
391	0.0111139591744608\\
392	0.0111178061214048\\
393	0.0111217027612495\\
394	0.0111256489394152\\
395	0.0111296444589321\\
396	0.0111336890808424\\
397	0.0111377825251812\\
398	0.0111419244726446\\
399	0.0111461145670698\\
400	0.011150352418867\\
401	0.0111546376095694\\
402	0.0111589696977059\\
403	0.0111633482262796\\
404	0.0111677727323193\\
405	0.0111722427594448\\
406	0.0111767578739891\\
407	0.0111813176719716\\
408	0.0111859218596327\\
409	0.0111905704596121\\
410	0.0111952624676152\\
411	0.0111999978281606\\
412	0.0112047765992291\\
413	0.0112095912177072\\
414	0.0112142863462451\\
415	0.0112190672101078\\
416	0.0112239352313215\\
417	0.0112288918548266\\
418	0.0112339385493594\\
419	0.0112390768083189\\
420	0.0112443081503926\\
421	0.0112496341192204\\
422	0.0112550562798195\\
423	0.0112605762041164\\
424	0.0112661954178444\\
425	0.0112719151897949\\
426	0.0112777372397674\\
427	0.0112836652336045\\
428	0.0112897008790337\\
429	0.0112958459076668\\
430	0.0113021020757675\\
431	0.0113084711650986\\
432	0.0113149549838525\\
433	0.0113215553676806\\
434	0.0113282741809034\\
435	0.0113351133178814\\
436	0.0113420747043324\\
437	0.0113491602985643\\
438	0.0113563720933495\\
439	0.0113637121177307\\
440	0.0113711824390299\\
441	0.0113787851647043\\
442	0.0113865224441316\\
443	0.0113943964708904\\
444	0.0114024094850894\\
445	0.0114105637765687\\
446	0.0114188616887625\\
447	0.0114273056217474\\
448	0.0114358980320699\\
449	0.0114446414381959\\
450	0.0114535384248832\\
451	0.011462591653238\\
452	0.0114718038578371\\
453	0.0114811778457777\\
454	0.0114907164967417\\
455	0.0115004227586098\\
456	0.011510299669983\\
457	0.0115203503648837\\
458	0.0115305779215258\\
459	0.0115409855019593\\
460	0.0115515763502002\\
461	0.0115623538105289\\
462	0.011573321345073\\
463	0.0115844825408848\\
464	0.0115958410946219\\
465	0.011607400741314\\
466	0.0116191654543405\\
467	0.0116311395543891\\
468	0.0116433273324563\\
469	0.0116557331313184\\
470	0.0116683614040626\\
471	0.0116812167210318\\
472	0.0116943100675087\\
473	0.0117076466145337\\
474	0.0117212288587474\\
475	0.0117350586523905\\
476	0.0117491379682479\\
477	0.0117634720641831\\
478	0.0117780637472259\\
479	0.0117929149586821\\
480	0.0118080273423896\\
481	0.0118234022401821\\
482	0.0118390407265265\\
483	0.0118549434573832\\
484	0.0118711102209801\\
485	0.0118875408495325\\
486	0.0119042345399909\\
487	0.0119211916958068\\
488	0.0119384114516301\\
489	0.0119558939986645\\
490	0.0119736426188383\\
491	0.0119916598958243\\
492	0.0120099476874576\\
493	0.0120285067717896\\
494	0.0120473366367891\\
495	0.0120664353977148\\
496	0.0120857996136615\\
497	0.0121054241229139\\
498	0.0121253019467027\\
499	0.0121454245213979\\
500	0.0121669864780814\\
501	0.0121886598665547\\
502	0.0122101545398785\\
503	0.0122314520477386\\
504	0.012252530409118\\
505	0.0122733666930203\\
506	0.0122939370252466\\
507	0.0123142168386173\\
508	0.0123341810788034\\
509	0.0123538044294089\\
510	0.0123730608856005\\
511	0.0123914575771536\\
512	0.0124090854357912\\
513	0.0124264484037729\\
514	0.0124435345677051\\
515	0.0124603328222785\\
516	0.0124768330174607\\
517	0.0124930261129636\\
518	0.0125089043328037\\
519	0.0125244613817106\\
520	0.0125396927277238\\
521	0.0125545956971607\\
522	0.012568974804646\\
523	0.0125825910149726\\
524	0.0125961058527152\\
525	0.0126095177936548\\
526	0.0126228257858893\\
527	0.0126360292417792\\
528	0.012649128020776\\
529	0.0126621223960243\\
530	0.0126750130027323\\
531	0.012687800772377\\
532	0.0127004868511807\\
533	0.0127130744938489\\
534	0.0127255934823642\\
535	0.0127380358133138\\
536	0.0127503928385556\\
537	0.0127626551809634\\
538	0.0127748126411372\\
539	0.0127868540945184\\
540	0.0127987673784488\\
541	0.0128105391689269\\
542	0.0128221548461821\\
543	0.0128335983486842\\
544	0.0128448520149445\\
545	0.0128558952342837\\
546	0.0128667053002203\\
547	0.0128772577225203\\
548	0.0128875263142159\\
549	0.0128974718243012\\
550	0.01290741787158\\
551	0.0129175138458269\\
552	0.0129278290503619\\
553	0.0129383682213716\\
554	0.0129491360671807\\
555	0.0129601372657898\\
556	0.0129713764884326\\
557	0.0129828584761428\\
558	0.0129945880030828\\
559	0.0130065698691541\\
560	0.013018808890978\\
561	0.0130313098860317\\
562	0.0130440799239086\\
563	0.013057127769762\\
564	0.0130704633598196\\
565	0.0130842809431066\\
566	0.0130996126366805\\
567	0.013115059447439\\
568	0.0131306115337819\\
569	0.013146304511449\\
570	0.0131620042189724\\
571	0.0131777619914875\\
572	0.0131935571303295\\
573	0.0132092235907178\\
574	0.0132247963192703\\
575	0.0132399000096878\\
576	0.0132543756406824\\
577	0.0132704253583581\\
578	0.0132865863105213\\
579	0.0133020081118863\\
580	0.0133173908781042\\
581	0.0133320104924939\\
582	0.0133451046723532\\
583	0.0133576903926825\\
584	0.0133699320550395\\
585	0.0133816601464576\\
586	0.0133924615300803\\
587	0.0134024915452123\\
588	0.0134122145702396\\
589	0.0134215984264863\\
590	0.0134305710816451\\
591	0.0134390764668377\\
592	0.0134472000954262\\
593	0.0134551345125145\\
594	0.0134637112677116\\
595	0.0134740407759367\\
596	0.0134889777516421\\
597	0.0135160557778789\\
598	0.0135751148335434\\
599	0\\
600	0\\
};
\addplot [color=red!50!mycolor17,solid,forget plot]
  table[row sep=crcr]{%
1	0.0110134653792917\\
2	0.0110134679347945\\
3	0.0110134705370157\\
4	0.0110134731868078\\
5	0.0110134758850386\\
6	0.0110134786325922\\
7	0.0110134814303687\\
8	0.0110134842792851\\
9	0.0110134871802752\\
10	0.0110134901342898\\
11	0.0110134931422977\\
12	0.0110134962052852\\
13	0.0110134993242571\\
14	0.0110135025002365\\
15	0.0110135057342657\\
16	0.0110135090274062\\
17	0.0110135123807391\\
18	0.0110135157953654\\
19	0.0110135192724068\\
20	0.0110135228130056\\
21	0.0110135264183253\\
22	0.0110135300895511\\
23	0.0110135338278901\\
24	0.0110135376345717\\
25	0.0110135415108485\\
26	0.011013545457996\\
27	0.0110135494773136\\
28	0.011013553570125\\
29	0.0110135577377783\\
30	0.0110135619816468\\
31	0.0110135663031296\\
32	0.0110135707036514\\
33	0.011013575184664\\
34	0.011013579747646\\
35	0.0110135843941036\\
36	0.0110135891255713\\
37	0.011013593943612\\
38	0.011013598849818\\
39	0.0110136038458115\\
40	0.0110136089332448\\
41	0.0110136141138013\\
42	0.0110136193891958\\
43	0.0110136247611753\\
44	0.0110136302315195\\
45	0.0110136358020417\\
46	0.0110136414745888\\
47	0.0110136472510427\\
48	0.0110136531333206\\
49	0.0110136591233756\\
50	0.0110136652231975\\
51	0.0110136714348137\\
52	0.0110136777602894\\
53	0.0110136842017291\\
54	0.0110136907612765\\
55	0.011013697441116\\
56	0.0110137042434728\\
57	0.0110137111706145\\
58	0.011013718224851\\
59	0.0110137254085362\\
60	0.011013732724068\\
61	0.0110137401738898\\
62	0.0110137477604912\\
63	0.0110137554864086\\
64	0.0110137633542266\\
65	0.0110137713665785\\
66	0.0110137795261475\\
67	0.0110137878356673\\
68	0.0110137962979238\\
69	0.0110138049157552\\
70	0.0110138136920537\\
71	0.0110138226297663\\
72	0.0110138317318956\\
73	0.0110138410015015\\
74	0.0110138504417015\\
75	0.0110138600556726\\
76	0.0110138698466519\\
77	0.011013879817938\\
78	0.011013889972892\\
79	0.011013900314939\\
80	0.011013910847569\\
81	0.0110139215743385\\
82	0.0110139324988714\\
83	0.0110139436248606\\
84	0.0110139549560693\\
85	0.011013966496332\\
86	0.0110139782495564\\
87	0.0110139902197245\\
88	0.011014002410894\\
89	0.0110140148272\\
90	0.0110140274728561\\
91	0.0110140403521563\\
92	0.0110140534694764\\
93	0.0110140668292756\\
94	0.0110140804360978\\
95	0.0110140942945738\\
96	0.0110141084094226\\
97	0.011014122785453\\
98	0.0110141374275656\\
99	0.0110141523407544\\
100	0.0110141675301088\\
101	0.011014183000815\\
102	0.0110141987581583\\
103	0.0110142148075249\\
104	0.0110142311544036\\
105	0.011014247804388\\
106	0.0110142647631784\\
107	0.011014282036584\\
108	0.0110142996305247\\
109	0.0110143175510335\\
110	0.0110143358042587\\
111	0.0110143543964657\\
112	0.0110143733340396\\
113	0.0110143926234875\\
114	0.0110144122714405\\
115	0.0110144322846566\\
116	0.0110144526700226\\
117	0.0110144734345569\\
118	0.0110144945854119\\
119	0.0110145161298763\\
120	0.0110145380753784\\
121	0.0110145604294881\\
122	0.0110145831999196\\
123	0.0110146063945349\\
124	0.0110146300213455\\
125	0.0110146540885164\\
126	0.011014678604368\\
127	0.0110147035773798\\
128	0.0110147290161931\\
129	0.0110147549296141\\
130	0.011014781326617\\
131	0.0110148082163475\\
132	0.0110148356081256\\
133	0.0110148635114492\\
134	0.0110148919359975\\
135	0.0110149208916342\\
136	0.0110149503884112\\
137	0.0110149804365724\\
138	0.0110150110465567\\
139	0.0110150422290022\\
140	0.0110150739947501\\
141	0.011015106354848\\
142	0.0110151393205542\\
143	0.0110151729033416\\
144	0.0110152071149019\\
145	0.0110152419671495\\
146	0.0110152774722258\\
147	0.0110153136425035\\
148	0.0110153504905912\\
149	0.0110153880293374\\
150	0.0110154262718355\\
151	0.0110154652314281\\
152	0.0110155049217118\\
153	0.0110155453565422\\
154	0.0110155865500383\\
155	0.0110156285165882\\
156	0.0110156712708533\\
157	0.0110157148277744\\
158	0.011015759202576\\
159	0.0110158044107726\\
160	0.0110158504681733\\
161	0.011015897390888\\
162	0.0110159451953327\\
163	0.0110159938982354\\
164	0.0110160435166419\\
165	0.011016094067922\\
166	0.0110161455697752\\
167	0.0110161980402373\\
168	0.0110162514976864\\
169	0.0110163059608496\\
170	0.0110163614488093\\
171	0.0110164179810101\\
172	0.0110164755772656\\
173	0.0110165342577649\\
174	0.0110165940430804\\
175	0.0110166549541744\\
176	0.0110167170124063\\
177	0.0110167802395407\\
178	0.0110168446577545\\
179	0.0110169102896447\\
180	0.0110169771582362\\
181	0.0110170452869901\\
182	0.0110171146998116\\
183	0.0110171854210582\\
184	0.0110172574755486\\
185	0.0110173308885709\\
186	0.0110174056858914\\
187	0.0110174818937639\\
188	0.0110175595389384\\
189	0.0110176386486703\\
190	0.0110177192507304\\
191	0.0110178013734138\\
192	0.01101788504555\\
193	0.011017970296513\\
194	0.0110180571562311\\
195	0.0110181456551975\\
196	0.0110182358244805\\
197	0.0110183276957343\\
198	0.0110184213012102\\
199	0.011018516673767\\
200	0.011018613846883\\
201	0.0110187128546667\\
202	0.0110188137318694\\
203	0.0110189165138961\\
204	0.0110190212368184\\
205	0.0110191279373861\\
206	0.0110192366530405\\
207	0.0110193474219263\\
208	0.0110194602829054\\
209	0.0110195752755692\\
210	0.011019692440253\\
211	0.0110198118180488\\
212	0.0110199334508196\\
213	0.0110200573812137\\
214	0.0110201836526786\\
215	0.0110203123094759\\
216	0.0110204433966964\\
217	0.0110205769602746\\
218	0.0110207130470049\\
219	0.0110208517045564\\
220	0.0110209929814895\\
221	0.0110211369272717\\
222	0.011021283592294\\
223	0.0110214330278879\\
224	0.0110215852863422\\
225	0.0110217404209203\\
226	0.0110218984858778\\
227	0.0110220595364801\\
228	0.0110222236290209\\
229	0.0110223908208401\\
230	0.0110225611703432\\
231	0.0110227347370196\\
232	0.011022911581462\\
233	0.0110230917653863\\
234	0.0110232753516512\\
235	0.0110234624042784\\
236	0.0110236529884728\\
237	0.0110238471706438\\
238	0.0110240450184256\\
239	0.011024246600699\\
240	0.0110244519876127\\
241	0.0110246612506055\\
242	0.0110248744624279\\
243	0.0110250916971648\\
244	0.0110253130302581\\
245	0.0110255385385295\\
246	0.0110257683002036\\
247	0.0110260023949311\\
248	0.0110262409038126\\
249	0.0110264839094217\\
250	0.0110267314958295\\
251	0.0110269837486279\\
252	0.0110272407549538\\
253	0.0110275026035131\\
254	0.011027769384605\\
255	0.0110280411901456\\
256	0.0110283181136916\\
257	0.0110286002504644\\
258	0.0110288876973734\\
259	0.0110291805530386\\
260	0.0110294789178134\\
261	0.0110297828938067\\
262	0.0110300925849039\\
263	0.011030408096787\\
264	0.0110307295369543\\
265	0.0110310570147378\\
266	0.0110313906413199\\
267	0.0110317305297478\\
268	0.0110320767949457\\
269	0.0110324295537251\\
270	0.0110327889247916\\
271	0.0110331550287499\\
272	0.0110335279881075\\
273	0.0110339079272848\\
274	0.0110342949726584\\
275	0.0110346892527373\\
276	0.0110350908988745\\
277	0.0110355000482535\\
278	0.0110359168578037\\
279	0.011036341441527\\
280	0.0110367738693236\\
281	0.0110372142814829\\
282	0.0110376628206231\\
283	0.011038119631726\\
284	0.0110385848621722\\
285	0.011039058661778\\
286	0.0110395411828326\\
287	0.0110400325801372\\
288	0.0110405330110451\\
289	0.0110410426355035\\
290	0.011041561616097\\
291	0.0110420901180937\\
292	0.0110426283094929\\
293	0.0110431763610761\\
294	0.0110437344464607\\
295	0.0110443027421577\\
296	0.0110448814276331\\
297	0.0110454706853738\\
298	0.0110460707009598\\
299	0.0110466816631413\\
300	0.0110473037639238\\
301	0.0110479371986613\\
302	0.0110485821661579\\
303	0.0110492388687817\\
304	0.0110499075125894\\
305	0.0110505883074666\\
306	0.0110512814672827\\
307	0.0110519872100641\\
308	0.0110527057581858\\
309	0.0110534373385759\\
310	0.0110541821829188\\
311	0.0110549405277995\\
312	0.0110557126146155\\
313	0.0110564986887312\\
314	0.0110572989962471\\
315	0.0110581137732551\\
316	0.0110589432109055\\
317	0.0110597874847164\\
318	0.0110606474748093\\
319	0.0110615234635714\\
320	0.0110624157380557\\
321	0.0110633245900424\\
322	0.0110642503161012\\
323	0.0110651932176533\\
324	0.0110661536010337\\
325	0.0110671317775536\\
326	0.0110681280635622\\
327	0.0110691427805092\\
328	0.0110701762550067\\
329	0.0110712288188905\\
330	0.0110723008092819\\
331	0.0110733925686482\\
332	0.0110745044448631\\
333	0.0110756367912661\\
334	0.0110767899667217\\
335	0.0110779643356769\\
336	0.0110791602682179\\
337	0.0110803781401258\\
338	0.0110816183329304\\
339	0.0110828812339609\\
340	0.011084167236392\\
341	0.0110854767393046\\
342	0.011086810147729\\
343	0.0110881678725967\\
344	0.0110895503307751\\
345	0.0110909579450985\\
346	0.0110923911443969\\
347	0.0110938503635219\\
348	0.0110953360433703\\
349	0.0110968486309073\\
350	0.0110983885791906\\
351	0.0110999563473751\\
352	0.011101552400711\\
353	0.0111031772105596\\
354	0.0111048312544102\\
355	0.0111065150159073\\
356	0.0111082289849186\\
357	0.0111099736577607\\
358	0.0111117495380706\\
359	0.0111135571406408\\
360	0.0111153970115238\\
361	0.0111172697514782\\
362	0.0111191756327446\\
363	0.0111211151955518\\
364	0.0111230889882079\\
365	0.0111250975673466\\
366	0.0111271414982031\\
367	0.0111292213549222\\
368	0.011131337720903\\
369	0.0111334911891851\\
370	0.0111356823628799\\
371	0.0111379118556521\\
372	0.0111401802922549\\
373	0.0111424883091155\\
374	0.0111448365549561\\
375	0.0111472256913834\\
376	0.0111496563932419\\
377	0.0111521293480687\\
378	0.0111546452525182\\
379	0.0111572047984563\\
380	0.0111598086209993\\
381	0.0111624570801157\\
382	0.0111651521421757\\
383	0.0111678945878793\\
384	0.0111706852044207\\
385	0.0111735247963825\\
386	0.0111764141869729\\
387	0.0111793542193985\\
388	0.0111823457583774\\
389	0.0111853896917847\\
390	0.0111884869330546\\
391	0.0111916384233961\\
392	0.0111948451340808\\
393	0.0111981080689366\\
394	0.0112014282670564\\
395	0.0112048068057313\\
396	0.0112082448036124\\
397	0.0112117434241048\\
398	0.0112153038789915\\
399	0.0112189274322798\\
400	0.0112226154042574\\
401	0.0112263691757384\\
402	0.0112301901924618\\
403	0.0112340799695989\\
404	0.0112380400963005\\
405	0.0112420722401841\\
406	0.0112461781516689\\
407	0.0112503596683496\\
408	0.0112546187186486\\
409	0.0112589573251298\\
410	0.0112633776095775\\
411	0.0112678817873419\\
412	0.0112724721642422\\
413	0.0112771512785521\\
414	0.011281924554925\\
415	0.0112867938806169\\
416	0.0112917611800265\\
417	0.0112968284152576\\
418	0.0113019975866518\\
419	0.011307270733284\\
420	0.0113126499334187\\
421	0.0113181373049183\\
422	0.0113237350055213\\
423	0.0113294452330873\\
424	0.011335270225525\\
425	0.0113412122597856\\
426	0.011347273646975\\
427	0.0113534567280712\\
428	0.0113597638858074\\
429	0.01136619754502\\
430	0.0113727601729619\\
431	0.0113794542795683\\
432	0.0113862824176476\\
433	0.0113932471829123\\
434	0.0114003512137686\\
435	0.011407597191815\\
436	0.0114149878431512\\
437	0.0114225259380576\\
438	0.0114302142869262\\
439	0.0114380557423199\\
440	0.01144605319828\\
441	0.0114542095894351\\
442	0.0114625278898838\\
443	0.011471011111739\\
444	0.0114796623026787\\
445	0.0114884845401612\\
446	0.011497480932532\\
447	0.0115066546281675\\
448	0.0115160088204975\\
449	0.0115255467007211\\
450	0.0115352714778931\\
451	0.0115451863788122\\
452	0.0115552946425947\\
453	0.011565599518629\\
454	0.0115761042645763\\
455	0.0115868121337473\\
456	0.0115977262533013\\
457	0.0116088498542276\\
458	0.0116201863377686\\
459	0.0116317389752061\\
460	0.011643510893824\\
461	0.0116555051725634\\
462	0.0116677248295728\\
463	0.0116801728071662\\
464	0.0116928520231295\\
465	0.0117057652304036\\
466	0.011718913527052\\
467	0.0117322984327091\\
468	0.0117459242297021\\
469	0.0117597933232408\\
470	0.0117739069796817\\
471	0.011788266162539\\
472	0.0118028713846374\\
473	0.0118177228665846\\
474	0.0118328202434625\\
475	0.011848162602471\\
476	0.01186374874409\\
477	0.0118795793659911\\
478	0.0118956582779787\\
479	0.0119119883690854\\
480	0.0119285716844974\\
481	0.011945409426486\\
482	0.0119625018437398\\
483	0.011979848070994\\
484	0.0119974457623746\\
485	0.0120152910551991\\
486	0.0120333784233839\\
487	0.0120517005971582\\
488	0.0120707353628826\\
489	0.0120906807389837\\
490	0.0121104767839451\\
491	0.0121301036789161\\
492	0.0121495413620184\\
493	0.0121687709909655\\
494	0.0121877726395686\\
495	0.0122065248240623\\
496	0.0122250056082003\\
497	0.0122431927167854\\
498	0.0122610636721047\\
499	0.0122785959587757\\
500	0.0122950518261643\\
501	0.0123111336151311\\
502	0.0123269797860266\\
503	0.0123425823202101\\
504	0.0123579329236886\\
505	0.0123730243018242\\
506	0.0123878503906345\\
507	0.0124024064961853\\
508	0.0124166894563258\\
509	0.0124306977496756\\
510	0.0124444317962187\\
511	0.0124575606061362\\
512	0.0124702173215123\\
513	0.0124828196346015\\
514	0.0124953691812993\\
515	0.0125078681970754\\
516	0.0125203195175986\\
517	0.0125327265692689\\
518	0.0125450933481989\\
519	0.0125574243837514\\
520	0.0125697246818943\\
521	0.0125819996524807\\
522	0.0125942628780827\\
523	0.0126065385041128\\
524	0.0126188240929494\\
525	0.0126311169093886\\
526	0.0126434138652448\\
527	0.0126557114580474\\
528	0.0126680057035716\\
529	0.0126802920621906\\
530	0.0126925653590862\\
531	0.0127048196981507\\
532	0.0127170483693706\\
533	0.0127292436616183\\
534	0.0127413955781831\\
535	0.0127534932868318\\
536	0.0127655250522041\\
537	0.0127774781635861\\
538	0.0127893388578269\\
539	0.0128010922371522\\
540	0.0128127221816049\\
541	0.0128242112558164\\
542	0.0128355406097937\\
543	0.0128466898733903\\
544	0.0128576370440959\\
545	0.0128683584230701\\
546	0.0128788284969462\\
547	0.0128890197168221\\
548	0.0128988938146329\\
549	0.0129088204236672\\
550	0.0129189536911403\\
551	0.0129293183806309\\
552	0.0129399199659765\\
553	0.0129507641132976\\
554	0.0129618566619033\\
555	0.0129732036393451\\
556	0.0129848112750305\\
557	0.012996686004706\\
558	0.013008837736785\\
559	0.0130212777315602\\
560	0.0130340188542051\\
561	0.0130480619090636\\
562	0.0130627409734786\\
563	0.0130775458061291\\
564	0.0130924969899236\\
565	0.0131075709060964\\
566	0.013122007812545\\
567	0.0131368560289957\\
568	0.0131521961419429\\
569	0.0131674037980389\\
570	0.0131823061747648\\
571	0.0131962067427473\\
572	0.0132101656552701\\
573	0.0132264507710498\\
574	0.0132427750691417\\
575	0.0132583286052144\\
576	0.0132737542621561\\
577	0.0132880160445443\\
578	0.013301888510655\\
579	0.0133153070109448\\
580	0.0133282044538216\\
581	0.0133403078355908\\
582	0.013351430271074\\
583	0.013362268821379\\
584	0.0133728864246924\\
585	0.0133832261555067\\
586	0.0133932918913093\\
587	0.0134031094508248\\
588	0.0134126433937004\\
589	0.0134218515519248\\
590	0.0134306952401147\\
591	0.0134391556363722\\
592	0.0134472338645631\\
593	0.0134551345125145\\
594	0.0134637112677116\\
595	0.0134740407759367\\
596	0.0134889777516421\\
597	0.0135160557778789\\
598	0.0135751148335434\\
599	0\\
600	0\\
};
\addplot [color=red!40!mycolor19,solid,forget plot]
  table[row sep=crcr]{%
1	0.0110522556087146\\
2	0.0110522577840676\\
3	0.0110522599994303\\
4	0.0110522622555279\\
5	0.0110522645530989\\
6	0.0110522668928955\\
7	0.0110522692756832\\
8	0.0110522717022416\\
9	0.0110522741733645\\
10	0.0110522766898602\\
11	0.0110522792525516\\
12	0.0110522818622766\\
13	0.0110522845198887\\
14	0.0110522872262565\\
15	0.011052289982265\\
16	0.0110522927888151\\
17	0.0110522956468241\\
18	0.0110522985572262\\
19	0.0110523015209729\\
20	0.011052304539033\\
21	0.0110523076123929\\
22	0.0110523107420575\\
23	0.0110523139290498\\
24	0.011052317174412\\
25	0.0110523204792051\\
26	0.0110523238445099\\
27	0.0110523272714271\\
28	0.0110523307610776\\
29	0.0110523343146032\\
30	0.0110523379331666\\
31	0.0110523416179521\\
32	0.0110523453701659\\
33	0.0110523491910367\\
34	0.0110523530818157\\
35	0.0110523570437776\\
36	0.0110523610782207\\
37	0.0110523651864673\\
38	0.0110523693698644\\
39	0.0110523736297843\\
40	0.0110523779676245\\
41	0.0110523823848088\\
42	0.0110523868827876\\
43	0.0110523914630384\\
44	0.0110523961270662\\
45	0.0110524008764043\\
46	0.0110524057126148\\
47	0.0110524106372891\\
48	0.0110524156520482\\
49	0.0110524207585438\\
50	0.0110524259584588\\
51	0.0110524312535075\\
52	0.0110524366454367\\
53	0.011052442136026\\
54	0.0110524477270886\\
55	0.0110524534204722\\
56	0.0110524592180591\\
57	0.0110524651217675\\
58	0.0110524711335517\\
59	0.0110524772554034\\
60	0.0110524834893517\\
61	0.0110524898374646\\
62	0.0110524963018492\\
63	0.0110525028846528\\
64	0.0110525095880635\\
65	0.0110525164143112\\
66	0.0110525233656684\\
67	0.0110525304444509\\
68	0.0110525376530188\\
69	0.0110525449937773\\
70	0.0110525524691778\\
71	0.0110525600817185\\
72	0.0110525678339456\\
73	0.0110525757284543\\
74	0.0110525837678894\\
75	0.0110525919549468\\
76	0.011052600292374\\
77	0.0110526087829718\\
78	0.0110526174295946\\
79	0.0110526262351521\\
80	0.0110526352026101\\
81	0.0110526443349917\\
82	0.0110526536353785\\
83	0.0110526631069116\\
84	0.0110526727527931\\
85	0.011052682576287\\
86	0.0110526925807206\\
87	0.0110527027694859\\
88	0.0110527131460406\\
89	0.0110527237139099\\
90	0.0110527344766871\\
91	0.0110527454380357\\
92	0.0110527566016906\\
93	0.0110527679714593\\
94	0.0110527795512236\\
95	0.011052791344941\\
96	0.0110528033566464\\
97	0.0110528155904532\\
98	0.0110528280505555\\
99	0.0110528407412293\\
100	0.0110528536668343\\
101	0.0110528668318153\\
102	0.0110528802407044\\
103	0.0110528938981226\\
104	0.0110529078087813\\
105	0.0110529219774843\\
106	0.0110529364091297\\
107	0.0110529511087119\\
108	0.0110529660813232\\
109	0.0110529813321562\\
110	0.0110529968665052\\
111	0.0110530126897691\\
112	0.0110530288074526\\
113	0.0110530452251689\\
114	0.0110530619486417\\
115	0.0110530789837075\\
116	0.0110530963363175\\
117	0.0110531140125406\\
118	0.0110531320185649\\
119	0.0110531503607007\\
120	0.0110531690453828\\
121	0.011053188079173\\
122	0.0110532074687622\\
123	0.0110532272209739\\
124	0.0110532473427658\\
125	0.0110532678412332\\
126	0.0110532887236116\\
127	0.0110533099972791\\
128	0.0110533316697599\\
129	0.0110533537487267\\
130	0.0110533762420036\\
131	0.0110533991575695\\
132	0.0110534225035611\\
133	0.0110534462882758\\
134	0.0110534705201748\\
135	0.0110534952078868\\
136	0.011053520360211\\
137	0.0110535459861205\\
138	0.0110535720947659\\
139	0.0110535986954786\\
140	0.0110536257977745\\
141	0.0110536534113575\\
142	0.0110536815461236\\
143	0.0110537102121641\\
144	0.01105373941977\\
145	0.0110537691794357\\
146	0.0110537995018629\\
147	0.0110538303979648\\
148	0.0110538618788705\\
149	0.0110538939559287\\
150	0.0110539266407125\\
151	0.0110539599450236\\
152	0.011053993880897\\
153	0.011054028460605\\
154	0.0110540636966628\\
155	0.0110540996018324\\
156	0.0110541361891278\\
157	0.01105417347182\\
158	0.011054211463442\\
159	0.0110542501777937\\
160	0.0110542896289476\\
161	0.0110543298312535\\
162	0.0110543707993446\\
163	0.0110544125481424\\
164	0.0110544550928631\\
165	0.0110544984490224\\
166	0.0110545426324422\\
167	0.0110545876592561\\
168	0.0110546335459158\\
169	0.0110546803091969\\
170	0.0110547279662056\\
171	0.0110547765343852\\
172	0.0110548260315221\\
173	0.0110548764757532\\
174	0.0110549278855723\\
175	0.0110549802798372\\
176	0.0110550336777769\\
177	0.0110550880989984\\
178	0.0110551435634948\\
179	0.0110552000916522\\
180	0.0110552577042573\\
181	0.0110553164225061\\
182	0.0110553762680106\\
183	0.011055437262808\\
184	0.0110554994293684\\
185	0.0110555627906032\\
186	0.011055627369874\\
187	0.0110556931910014\\
188	0.0110557602782735\\
189	0.0110558286564554\\
190	0.0110558983507982\\
191	0.011055969387049\\
192	0.0110560417914597\\
193	0.0110561155907977\\
194	0.0110561908123555\\
195	0.0110562674839607\\
196	0.011056345633987\\
197	0.0110564252913644\\
198	0.0110565064855901\\
199	0.0110565892467398\\
200	0.0110566736054786\\
201	0.0110567595930726\\
202	0.0110568472414008\\
203	0.0110569365829667\\
204	0.0110570276509108\\
205	0.0110571204790225\\
206	0.0110572151017535\\
207	0.0110573115542297\\
208	0.0110574098722653\\
209	0.0110575100923753\\
210	0.01105761225179\\
211	0.011057716388468\\
212	0.0110578225411112\\
213	0.0110579307491785\\
214	0.0110580410529011\\
215	0.011058153493297\\
216	0.0110582681121869\\
217	0.011058384952209\\
218	0.0110585040568354\\
219	0.0110586254703882\\
220	0.0110587492380556\\
221	0.0110588754059093\\
222	0.011059004020921\\
223	0.0110591351309801\\
224	0.0110592687849114\\
225	0.0110594050324932\\
226	0.0110595439244757\\
227	0.0110596855125994\\
228	0.0110598298496148\\
229	0.0110599769893014\\
230	0.0110601269864876\\
231	0.0110602798970707\\
232	0.0110604357780377\\
233	0.0110605946874859\\
234	0.0110607566846445\\
235	0.0110609218298956\\
236	0.0110610901847968\\
237	0.0110612618121034\\
238	0.011061436775791\\
239	0.0110616151410787\\
240	0.011061796974453\\
241	0.011061982343691\\
242	0.0110621713178857\\
243	0.0110623639674699\\
244	0.0110625603642416\\
245	0.0110627605813898\\
246	0.0110629646935201\\
247	0.0110631727766809\\
248	0.0110633849083906\\
249	0.0110636011676641\\
250	0.0110638216350409\\
251	0.0110640463926123\\
252	0.0110642755240501\\
253	0.011064509114635\\
254	0.0110647472512854\\
255	0.0110649900225869\\
256	0.0110652375188217\\
257	0.0110654898319986\\
258	0.0110657470558829\\
259	0.0110660092860272\\
260	0.0110662766198016\\
261	0.0110665491564247\\
262	0.0110668269969946\\
263	0.0110671102445201\\
264	0.0110673990039514\\
265	0.011067693382212\\
266	0.011067993488229\\
267	0.0110682994329649\\
268	0.0110686113294479\\
269	0.011068929292803\\
270	0.0110692534402823\\
271	0.0110695838912953\\
272	0.0110699207674399\\
273	0.0110702641925361\\
274	0.0110706142926717\\
275	0.0110709711962971\\
276	0.0110713350345256\\
277	0.0110717059424392\\
278	0.0110720840663966\\
279	0.0110724695360078\\
280	0.0110728624619412\\
281	0.0110732629880194\\
282	0.0110736712608364\\
283	0.0110740874298149\\
284	0.0110745116472644\\
285	0.0110749440684417\\
286	0.0110753848516126\\
287	0.0110758341581161\\
288	0.0110762921524307\\
289	0.0110767590022423\\
290	0.0110772348785156\\
291	0.0110777199555667\\
292	0.0110782144111398\\
293	0.011078718426486\\
294	0.0110792321864452\\
295	0.0110797558795322\\
296	0.0110802896980258\\
297	0.0110808338380618\\
298	0.0110813884997312\\
299	0.0110819538871818\\
300	0.0110825302087258\\
301	0.0110831176769525\\
302	0.0110837165088469\\
303	0.0110843269259148\\
304	0.0110849491543147\\
305	0.0110855834249976\\
306	0.0110862299738547\\
307	0.0110868890418727\\
308	0.011087560875299\\
309	0.0110882457258121\\
310	0.0110889438506927\\
311	0.0110896555129733\\
312	0.0110903809815009\\
313	0.0110911205307067\\
314	0.0110918744394063\\
315	0.0110926429862487\\
316	0.0110934264321556\\
317	0.011094225023847\\
318	0.0110950393345729\\
319	0.011095869674358\\
320	0.0110967163593643\\
321	0.0110975797120146\\
322	0.01109846006112\\
323	0.0110993577420091\\
324	0.01110027309666\\
325	0.0111012064738365\\
326	0.0111021582292256\\
327	0.0111031287255795\\
328	0.0111041183328601\\
329	0.0111051274283868\\
330	0.0111061563969883\\
331	0.0111072056311575\\
332	0.0111082755312097\\
333	0.0111093665054456\\
334	0.0111104789703173\\
335	0.0111116133505989\\
336	0.0111127700795612\\
337	0.0111139495991509\\
338	0.0111151523601744\\
339	0.0111163788224865\\
340	0.011117629455184\\
341	0.0111189047368049\\
342	0.0111202051555328\\
343	0.0111215312094084\\
344	0.0111228834065475\\
345	0.0111242622653657\\
346	0.0111256683148103\\
347	0.0111271020945992\\
348	0.011128564155468\\
349	0.0111300550594255\\
350	0.0111315753800175\\
351	0.0111331257026007\\
352	0.0111347066246264\\
353	0.0111363187559361\\
354	0.011137962719067\\
355	0.0111396391495713\\
356	0.011141348696349\\
357	0.011143092021995\\
358	0.011144869803166\\
359	0.0111466827309701\\
360	0.0111485315114099\\
361	0.01115041686609\\
362	0.011152339533704\\
363	0.0111543002689499\\
364	0.0111562998429425\\
365	0.0111583390436369\\
366	0.011160418676265\\
367	0.0111625395637837\\
368	0.0111647025473359\\
369	0.0111669084867218\\
370	0.0111691582608829\\
371	0.0111714527683958\\
372	0.0111737929279753\\
373	0.0111761796789856\\
374	0.0111786139819576\\
375	0.011181096819109\\
376	0.0111836291948633\\
377	0.0111862121363583\\
378	0.0111888466939273\\
379	0.0111915339415102\\
380	0.0111942749768615\\
381	0.0111970709209596\\
382	0.0111999229133673\\
383	0.0112028321198713\\
384	0.0112057997332585\\
385	0.0112088269740515\\
386	0.0112119150912587\\
387	0.0112150653631339\\
388	0.0112182790979441\\
389	0.0112215576347483\\
390	0.0112249023441701\\
391	0.0112283146291635\\
392	0.0112317959257731\\
393	0.0112353477038784\\
394	0.0112389714679189\\
395	0.0112426687575904\\
396	0.0112464411485052\\
397	0.0112502902528098\\
398	0.0112542177197486\\
399	0.0112582252361496\\
400	0.0112623145268282\\
401	0.0112664873548811\\
402	0.0112707455218946\\
403	0.0112750908680181\\
404	0.0112795252719004\\
405	0.0112840506504983\\
406	0.0112886689586672\\
407	0.0112933821885801\\
408	0.0112981923689454\\
409	0.0113031015640329\\
410	0.0113081118724015\\
411	0.0113132254258913\\
412	0.0113184443900822\\
413	0.0113237709645955\\
414	0.0113292073180522\\
415	0.0113347556565547\\
416	0.0113404182239359\\
417	0.0113461973018183\\
418	0.0113520952096186\\
419	0.0113581143044745\\
420	0.0113642569810519\\
421	0.0113705256713308\\
422	0.01137692284451\\
423	0.0113834510061307\\
424	0.0113901126983908\\
425	0.0113969105001876\\
426	0.0114038470264269\\
427	0.0114109249243833\\
428	0.011418146873857\\
429	0.0114255155861988\\
430	0.0114330338026575\\
431	0.0114407042925282\\
432	0.0114485298510949\\
433	0.0114565132972639\\
434	0.011464657470103\\
435	0.0114729652212656\\
436	0.0114814394144858\\
437	0.0114900829389648\\
438	0.0114988987129087\\
439	0.0115078896310828\\
440	0.0115170585920306\\
441	0.0115264084952188\\
442	0.0115359422378573\\
443	0.011545662711382\\
444	0.011555572797905\\
445	0.0115656753635626\\
446	0.0115759731968166\\
447	0.0115864690116464\\
448	0.0115971655623689\\
449	0.0116080658308683\\
450	0.0116191724296148\\
451	0.0116304878545012\\
452	0.0116420145013206\\
453	0.0116537545652828\\
454	0.0116657100377861\\
455	0.0116778827411926\\
456	0.0116902744773645\\
457	0.0117028850114115\\
458	0.0117157147228857\\
459	0.0117287663310586\\
460	0.0117420415045572\\
461	0.0117555404334236\\
462	0.0117692630482441\\
463	0.0117832090462185\\
464	0.0117973777845297\\
465	0.0118117701810837\\
466	0.0118263891698841\\
467	0.0118412373969455\\
468	0.0118563182286269\\
469	0.0118716335044038\\
470	0.0118871837351555\\
471	0.0119029684067957\\
472	0.0119189858595252\\
473	0.0119352331897081\\
474	0.0119517058678664\\
475	0.0119683977776944\\
476	0.0119855238389395\\
477	0.0120037755866312\\
478	0.0120218955928688\\
479	0.0120398671701517\\
480	0.0120576727877737\\
481	0.0120752933899803\\
482	0.0120927099554993\\
483	0.01210990385284\\
484	0.0121268570530123\\
485	0.0121435496195092\\
486	0.012159961432781\\
487	0.012176072221324\\
488	0.0121915736449661\\
489	0.0122063006062571\\
490	0.0122208204769087\\
491	0.0122351247451908\\
492	0.0122492059064224\\
493	0.0122630582556204\\
494	0.0122766769927613\\
495	0.0122900581143079\\
496	0.012303199010541\\
497	0.0123160986087916\\
498	0.0123287575041728\\
499	0.0123411780259058\\
500	0.0123528531308694\\
501	0.0123643753846295\\
502	0.0123758678113748\\
503	0.0123873337574185\\
504	0.0123987772054143\\
505	0.0124102027756199\\
506	0.0124216157165108\\
507	0.0124330218850032\\
508	0.0124444277140491\\
509	0.0124558401678491\\
510	0.0124672666755119\\
511	0.0124787282871157\\
512	0.0124902417969524\\
513	0.0125018076272554\\
514	0.0125134260723856\\
515	0.0125250972578757\\
516	0.0125368210950568\\
517	0.0125485972312262\\
518	0.0125604249953626\\
519	0.0125723033395197\\
520	0.0125842307761826\\
521	0.012596205311608\\
522	0.0126082240407639\\
523	0.0126202826040755\\
524	0.012632376194594\\
525	0.0126444995217257\\
526	0.012656646773156\\
527	0.0126688115749867\\
528	0.0126809869501136\\
529	0.0126931652748699\\
530	0.0127053382339588\\
531	0.0127174967737022\\
532	0.0127296310536341\\
533	0.0127417304003928\\
534	0.0127537833132383\\
535	0.0127657774047621\\
536	0.0127776993371164\\
537	0.0127895347533509\\
538	0.0128012682033883\\
539	0.0128128830641245\\
540	0.0128243614530708\\
541	0.0128356841348946\\
542	0.0128468304201348\\
543	0.0128577780552621\\
544	0.0128685031034433\\
545	0.0128789798035197\\
546	0.0128891806169425\\
547	0.0128990787083977\\
548	0.0129090375888559\\
549	0.0129192334113919\\
550	0.0129296727272301\\
551	0.0129403623815975\\
552	0.0129513095471166\\
553	0.0129625218062059\\
554	0.0129740104864093\\
555	0.0129857884046474\\
556	0.0129978702446768\\
557	0.0130116747206433\\
558	0.0130256634777257\\
559	0.0130397906340676\\
560	0.0130541416912179\\
561	0.0130680591566844\\
562	0.0130818328647282\\
563	0.0130957308409921\\
564	0.0131102446031075\\
565	0.0131248163214706\\
566	0.0131387593933604\\
567	0.0131525032059518\\
568	0.0131658917407363\\
569	0.0131810263323357\\
570	0.0131972575083075\\
571	0.013213044026579\\
572	0.0132287581796109\\
573	0.0132429665120816\\
574	0.0132567840953481\\
575	0.013270365677866\\
576	0.0132838830721923\\
577	0.0132962593908739\\
578	0.0133079395111771\\
579	0.0133193907373955\\
580	0.0133304822426818\\
581	0.0133413468885363\\
582	0.0133520801152746\\
583	0.0133626748577664\\
584	0.0133731047252647\\
585	0.0133833476643832\\
586	0.0133933802868045\\
587	0.013403169344532\\
588	0.0134126782949752\\
589	0.013421869140597\\
590	0.0134307057822237\\
591	0.0134391597728698\\
592	0.0134472338645631\\
593	0.0134551345125145\\
594	0.0134637112677116\\
595	0.0134740407759367\\
596	0.0134889777516421\\
597	0.0135160557778789\\
598	0.0135751148335434\\
599	0\\
600	0\\
};
\addplot [color=red!75!mycolor17,solid,forget plot]
  table[row sep=crcr]{%
1	0.0110623654486931\\
2	0.0110623678189323\\
3	0.0110623702333114\\
4	0.011062372692627\\
5	0.0110623751976894\\
6	0.0110623777493233\\
7	0.0110623803483681\\
8	0.0110623829956779\\
9	0.011062385692122\\
10	0.011062388438585\\
11	0.0110623912359672\\
12	0.0110623940851848\\
13	0.0110623969871701\\
14	0.0110623999428721\\
15	0.0110624029532564\\
16	0.011062406019306\\
17	0.0110624091420211\\
18	0.0110624123224196\\
19	0.0110624155615376\\
20	0.0110624188604296\\
21	0.0110624222201687\\
22	0.0110624256418472\\
23	0.0110624291265768\\
24	0.0110624326754889\\
25	0.0110624362897351\\
26	0.0110624399704876\\
27	0.0110624437189393\\
28	0.0110624475363046\\
29	0.0110624514238194\\
30	0.0110624553827418\\
31	0.0110624594143523\\
32	0.0110624635199545\\
33	0.0110624677008751\\
34	0.0110624719584648\\
35	0.0110624762940983\\
36	0.0110624807091752\\
37	0.0110624852051201\\
38	0.0110624897833833\\
39	0.0110624944454412\\
40	0.0110624991927968\\
41	0.0110625040269803\\
42	0.0110625089495493\\
43	0.0110625139620897\\
44	0.0110625190662162\\
45	0.0110625242635726\\
46	0.0110625295558324\\
47	0.0110625349446998\\
48	0.0110625404319096\\
49	0.0110625460192285\\
50	0.0110625517084552\\
51	0.0110625575014211\\
52	0.0110625633999912\\
53	0.0110625694060646\\
54	0.011062575521575\\
55	0.0110625817484916\\
56	0.0110625880888198\\
57	0.0110625945446018\\
58	0.0110626011179172\\
59	0.0110626078108841\\
60	0.0110626146256596\\
61	0.0110626215644406\\
62	0.0110626286294645\\
63	0.0110626358230103\\
64	0.0110626431473992\\
65	0.0110626506049955\\
66	0.0110626581982072\\
67	0.0110626659294875\\
68	0.0110626738013351\\
69	0.0110626818162954\\
70	0.0110626899769612\\
71	0.0110626982859741\\
72	0.011062706746025\\
73	0.0110627153598554\\
74	0.0110627241302584\\
75	0.0110627330600794\\
76	0.0110627421522177\\
77	0.0110627514096273\\
78	0.0110627608353178\\
79	0.0110627704323561\\
80	0.0110627802038669\\
81	0.0110627901530346\\
82	0.0110628002831037\\
83	0.0110628105973809\\
84	0.0110628210992357\\
85	0.0110628317921018\\
86	0.0110628426794789\\
87	0.0110628537649334\\
88	0.0110628650521003\\
89	0.0110628765446843\\
90	0.0110628882464612\\
91	0.0110629001612798\\
92	0.0110629122930628\\
93	0.0110629246458088\\
94	0.0110629372235936\\
95	0.0110629500305721\\
96	0.0110629630709795\\
97	0.0110629763491335\\
98	0.0110629898694354\\
99	0.0110630036363723\\
100	0.0110630176545188\\
101	0.0110630319285388\\
102	0.0110630464631872\\
103	0.0110630612633119\\
104	0.0110630763338558\\
105	0.0110630916798588\\
106	0.0110631073064596\\
107	0.0110631232188981\\
108	0.0110631394225172\\
109	0.0110631559227651\\
110	0.0110631727251974\\
111	0.0110631898354795\\
112	0.0110632072593887\\
113	0.0110632250028167\\
114	0.011063243071772\\
115	0.011063261472382\\
116	0.0110632802108961\\
117	0.0110632992936875\\
118	0.0110633187272563\\
119	0.0110633385182322\\
120	0.0110633586733768\\
121	0.0110633791995865\\
122	0.0110634001038955\\
123	0.0110634213934785\\
124	0.0110634430756536\\
125	0.0110634651578855\\
126	0.0110634876477882\\
127	0.0110635105531284\\
128	0.0110635338818285\\
129	0.0110635576419699\\
130	0.0110635818417964\\
131	0.0110636064897174\\
132	0.0110636315943114\\
133	0.0110636571643294\\
134	0.0110636832086988\\
135	0.0110637097365266\\
136	0.0110637367571034\\
137	0.0110637642799072\\
138	0.0110637923146072\\
139	0.0110638208710674\\
140	0.0110638499593515\\
141	0.011063879589726\\
142	0.011063909772665\\
143	0.0110639405188544\\
144	0.0110639718391958\\
145	0.0110640037448116\\
146	0.0110640362470491\\
147	0.011064069357485\\
148	0.0110641030879307\\
149	0.0110641374504361\\
150	0.0110641724572956\\
151	0.0110642081210522\\
152	0.0110642444545031\\
153	0.0110642814707043\\
154	0.0110643191829768\\
155	0.0110643576049109\\
156	0.0110643967503724\\
157	0.0110644366335081\\
158	0.0110644772687511\\
159	0.011064518670827\\
160	0.0110645608547597\\
161	0.0110646038358773\\
162	0.0110646476298183\\
163	0.011064692252538\\
164	0.0110647377203148\\
165	0.0110647840497565\\
166	0.0110648312578072\\
167	0.0110648793617541\\
168	0.011064928379234\\
169	0.0110649783282408\\
170	0.0110650292271322\\
171	0.0110650810946374\\
172	0.0110651339498643\\
173	0.011065187812307\\
174	0.0110652427018539\\
175	0.0110652986387951\\
176	0.0110653556438308\\
177	0.0110654137380793\\
178	0.0110654729430855\\
179	0.011065533280829\\
180	0.0110655947737334\\
181	0.0110656574446747\\
182	0.0110657213169902\\
183	0.0110657864144884\\
184	0.0110658527614575\\
185	0.0110659203826754\\
186	0.0110659893034196\\
187	0.0110660595494767\\
188	0.011066131147153\\
189	0.0110662041232844\\
190	0.0110662785052474\\
191	0.0110663543209695\\
192	0.0110664315989401\\
193	0.0110665103682221\\
194	0.0110665906584631\\
195	0.0110666724999068\\
196	0.0110667559234054\\
197	0.0110668409604311\\
198	0.011066927643089\\
199	0.0110670160041294\\
200	0.0110671060769607\\
201	0.0110671978956626\\
202	0.0110672914949997\\
203	0.0110673869104345\\
204	0.0110674841781421\\
205	0.011067583335024\\
206	0.0110676844187228\\
207	0.0110677874676369\\
208	0.0110678925209357\\
209	0.0110679996185749\\
210	0.0110681088013126\\
211	0.0110682201107248\\
212	0.0110683335892225\\
213	0.0110684492800677\\
214	0.0110685672273913\\
215	0.0110686874762097\\
216	0.0110688100724432\\
217	0.0110689350629341\\
218	0.0110690624954649\\
219	0.0110691924187775\\
220	0.0110693248825925\\
221	0.0110694599376288\\
222	0.0110695976356238\\
223	0.011069738029354\\
224	0.0110698811726557\\
225	0.0110700271204467\\
226	0.011070175928748\\
227	0.0110703276547058\\
228	0.0110704823566147\\
229	0.0110706400939406\\
230	0.0110708009273445\\
231	0.0110709649187065\\
232	0.0110711321311506\\
233	0.0110713026290697\\
234	0.0110714764781517\\
235	0.0110716537454048\\
236	0.0110718344991853\\
237	0.0110720188092241\\
238	0.0110722067466547\\
239	0.011072398384042\\
240	0.0110725937954105\\
241	0.0110727930562748\\
242	0.0110729962436688\\
243	0.0110732034361774\\
244	0.0110734147139672\\
245	0.011073630158819\\
246	0.0110738498541602\\
247	0.0110740738850985\\
248	0.0110743023384559\\
249	0.0110745353028031\\
250	0.0110747728684958\\
251	0.0110750151277102\\
252	0.0110752621744804\\
253	0.011075514104736\\
254	0.011075771016341\\
255	0.0110760330091328\\
256	0.011076300184963\\
257	0.011076572647738\\
258	0.0110768505034616\\
259	0.0110771338602773\\
260	0.0110774228285132\\
261	0.0110777175207262\\
262	0.0110780180517482\\
263	0.0110783245387335\\
264	0.0110786371012066\\
265	0.011078955861112\\
266	0.0110792809428647\\
267	0.0110796124734022\\
268	0.0110799505822378\\
269	0.0110802954015155\\
270	0.0110806470660664\\
271	0.0110810057134666\\
272	0.0110813714840972\\
273	0.011081744521206\\
274	0.0110821249709714\\
275	0.0110825129825688\\
276	0.0110829087082404\\
277	0.0110833123033692\\
278	0.0110837239265696\\
279	0.0110841437398821\\
280	0.0110845719088712\\
281	0.0110850086025246\\
282	0.0110854539933274\\
283	0.0110859082573382\\
284	0.0110863715742675\\
285	0.0110868441275568\\
286	0.0110873261044608\\
287	0.0110878176961308\\
288	0.0110883190977001\\
289	0.0110888305083719\\
290	0.0110893521315085\\
291	0.0110898841747232\\
292	0.0110904268499742\\
293	0.0110909803736601\\
294	0.0110915449667182\\
295	0.0110921208547247\\
296	0.0110927082679972\\
297	0.0110933074416991\\
298	0.0110939186159467\\
299	0.0110945420359176\\
300	0.0110951779519625\\
301	0.0110958266197177\\
302	0.0110964883002206\\
303	0.0110971632600266\\
304	0.0110978517713283\\
305	0.0110985541120752\\
306	0.0110992705660963\\
307	0.0111000014232226\\
308	0.0111007469794111\\
309	0.0111015075368689\\
310	0.0111022834041772\\
311	0.0111030748964142\\
312	0.0111038823352762\\
313	0.0111047060491931\\
314	0.0111055463734355\\
315	0.0111064036501983\\
316	0.0111072782286153\\
317	0.0111081704644863\\
318	0.0111090807194461\\
319	0.0111100093626643\\
320	0.011110956771006\\
321	0.0111119233291967\\
322	0.0111129094299904\\
323	0.0111139154743415\\
324	0.0111149418715809\\
325	0.011115989039595\\
326	0.0111170574050098\\
327	0.011118147403378\\
328	0.011119259479371\\
329	0.0111203940869744\\
330	0.0111215516896885\\
331	0.0111227327607323\\
332	0.0111239377832529\\
333	0.0111251672505386\\
334	0.0111264216662376\\
335	0.0111277015445801\\
336	0.0111290074106068\\
337	0.0111303398004013\\
338	0.0111316992613282\\
339	0.011133086352276\\
340	0.0111345016439057\\
341	0.0111359457189044\\
342	0.0111374191722452\\
343	0.0111389226114518\\
344	0.0111404566568694\\
345	0.0111420219419419\\
346	0.0111436191134946\\
347	0.0111452488320226\\
348	0.0111469117719861\\
349	0.0111486086221115\\
350	0.0111503400856987\\
351	0.0111521068809346\\
352	0.0111539097412131\\
353	0.0111557494154613\\
354	0.0111576266684717\\
355	0.0111595422812408\\
356	0.0111614970513135\\
357	0.0111634917931331\\
358	0.0111655273383964\\
359	0.0111676045364133\\
360	0.0111697242544719\\
361	0.0111718873782144\\
362	0.0111740948119858\\
363	0.0111763474792146\\
364	0.0111786463227994\\
365	0.0111809923055023\\
366	0.0111833864103454\\
367	0.011185829641011\\
368	0.0111883230222484\\
369	0.0111908676002845\\
370	0.0111934644432376\\
371	0.0111961146415384\\
372	0.0111988193083541\\
373	0.011201579580017\\
374	0.011204396616457\\
375	0.0112072716016378\\
376	0.011210205743997\\
377	0.0112132002768906\\
378	0.0112162564590431\\
379	0.0112193755750066\\
380	0.0112225589356217\\
381	0.0112258078784459\\
382	0.0112291237683384\\
383	0.0112325079979172\\
384	0.0112359619880048\\
385	0.0112394871880598\\
386	0.0112430850766003\\
387	0.0112467571616288\\
388	0.0112505049810212\\
389	0.0112543301029138\\
390	0.0112582341260679\\
391	0.011262218680213\\
392	0.0112662854263687\\
393	0.0112704360571436\\
394	0.0112746722970092\\
395	0.0112789959025473\\
396	0.0112834086626643\\
397	0.0112879123987752\\
398	0.0112925089650012\\
399	0.0112972002483958\\
400	0.0113019881690343\\
401	0.0113068746801122\\
402	0.0113118617677982\\
403	0.0113169514513408\\
404	0.0113221457830343\\
405	0.0113274468481125\\
406	0.0113328567649902\\
407	0.0113383776848614\\
408	0.0113440117914816\\
409	0.011349761300827\\
410	0.0113556284606027\\
411	0.0113616155496045\\
412	0.0113677248776123\\
413	0.0113739587842945\\
414	0.0113803196399383\\
415	0.0113868098435883\\
416	0.011393431820022\\
417	0.0114001880198675\\
418	0.0114070809179138\\
419	0.0114141130112685\\
420	0.0114212868172353\\
421	0.0114286048703343\\
422	0.0114360697193411\\
423	0.0114436839278263\\
424	0.0114514500641058\\
425	0.0114593707074665\\
426	0.0114674484547167\\
427	0.0114756859260157\\
428	0.0114840857261699\\
429	0.0114926504566801\\
430	0.0115013827176414\\
431	0.0115102851035017\\
432	0.0115193601981362\\
433	0.0115286105693645\\
434	0.0115380387634268\\
435	0.0115476472960385\\
436	0.0115574385774572\\
437	0.0115674148914464\\
438	0.011577578564922\\
439	0.011587932102672\\
440	0.0115984774852266\\
441	0.0116092165040529\\
442	0.011620150751284\\
443	0.0116312816134764\\
444	0.0116426102709697\\
445	0.0116541377182533\\
446	0.0116658648843438\\
447	0.011677791931043\\
448	0.011689918040407\\
449	0.0117022424953743\\
450	0.0117147686385806\\
451	0.011727496325985\\
452	0.0117404256078264\\
453	0.0117535601034405\\
454	0.0117669030239899\\
455	0.0117804571029016\\
456	0.0117942245922939\\
457	0.0118082062188995\\
458	0.0118224022267346\\
459	0.0118368129474516\\
460	0.0118514372985312\\
461	0.0118662723938917\\
462	0.011881314050468\\
463	0.0118965566771358\\
464	0.011912434347859\\
465	0.0119290361045897\\
466	0.0119455173330837\\
467	0.0119618628127314\\
468	0.011978056691041\\
469	0.0119940827081719\\
470	0.0120099236613982\\
471	0.0120255614815943\\
472	0.012040978130715\\
473	0.0120561558017057\\
474	0.0120710778634681\\
475	0.0120857259650814\\
476	0.0120999502055335\\
477	0.0121133080250348\\
478	0.0121264779777807\\
479	0.0121394528642926\\
480	0.0121522261655328\\
481	0.0121647918166338\\
482	0.0121771449867835\\
483	0.0121892823291729\\
484	0.0122012022135653\\
485	0.0122129036872904\\
486	0.0122243873241518\\
487	0.0122356552980776\\
488	0.0122465048325158\\
489	0.0122568647365316\\
490	0.012267203758131\\
491	0.0122775255543116\\
492	0.0122878343997599\\
493	0.0122981351875699\\
494	0.0123084334337906\\
495	0.0123187352719486\\
496	0.0123290474325015\\
497	0.0123393772108173\\
498	0.0123497324226331\\
499	0.0123601213477198\\
500	0.0123705727010044\\
501	0.0123810930067297\\
502	0.0123916841290159\\
503	0.0124023478858576\\
504	0.0124130860160422\\
505	0.0124239001425876\\
506	0.0124347917327885\\
507	0.0124457620549489\\
508	0.0124568121319474\\
509	0.0124679426917336\\
510	0.0124791541151341\\
511	0.0124904458323131\\
512	0.0125018164140504\\
513	0.0125132641802315\\
514	0.0125247871784754\\
515	0.0125363831621353\\
516	0.0125480495677779\\
517	0.0125597834922592\\
518	0.0125715816695281\\
519	0.0125834404472952\\
520	0.012595355763706\\
521	0.0126073231241715\\
522	0.0126193375929205\\
523	0.0126313938069644\\
524	0.0126434859479025\\
525	0.0126556077119702\\
526	0.0126677522781776\\
527	0.012679912274367\\
528	0.0126920797409886\\
529	0.0127042460923695\\
530	0.0127164020752147\\
531	0.0127285377240438\\
532	0.0127406423132262\\
533	0.0127527043050506\\
534	0.0127647112909694\\
535	0.012776649928001\\
536	0.0127885058698199\\
537	0.0128002636920156\\
538	0.012811906810946\\
539	0.0128234173955354\\
540	0.0128347762712923\\
541	0.0128459628157843\\
542	0.0128569548448548\\
543	0.0128677284899802\\
544	0.0128782580554183\\
545	0.0128885161898169\\
546	0.0128984663566234\\
547	0.0129084925588052\\
548	0.0129187690136387\\
549	0.012929303906619\\
550	0.0129401089208385\\
551	0.0129511976775003\\
552	0.0129626122013664\\
553	0.0129757679149621\\
554	0.0129890552764016\\
555	0.0130024848838998\\
556	0.0130161738624674\\
557	0.0130291429473715\\
558	0.0130422222229939\\
559	0.0130554270019023\\
560	0.0130686757755564\\
561	0.0130821204103835\\
562	0.0130956145942559\\
563	0.0131091155854781\\
564	0.013122239068426\\
565	0.0131360495658837\\
566	0.0131514679660248\\
567	0.0131669530769827\\
568	0.0131826927920756\\
569	0.0131974144130573\\
570	0.0132112428947385\\
571	0.0132249692959164\\
572	0.0132383599524506\\
573	0.0132505511850947\\
574	0.013262803206264\\
575	0.0132748459865571\\
576	0.0132863897399275\\
577	0.0132977162412856\\
578	0.0133087815388813\\
579	0.0133197523049772\\
580	0.0133306397973883\\
581	0.0133414408591185\\
582	0.0133521371638476\\
583	0.0133627058867585\\
584	0.0133731226054228\\
585	0.0133833603746144\\
586	0.0133933887036049\\
587	0.0134031741941115\\
588	0.013412680773945\\
589	0.0134218705377325\\
590	0.0134307062886727\\
591	0.0134391597728698\\
592	0.0134472338645631\\
593	0.0134551345125145\\
594	0.0134637112677116\\
595	0.0134740407759367\\
596	0.0134889777516421\\
597	0.0135160557778789\\
598	0.0135751148335434\\
599	0\\
600	0\\
};
\addplot [color=red!80!mycolor19,solid,forget plot]
  table[row sep=crcr]{%
1	0.0110648549379197\\
2	0.0110648578589327\\
3	0.0110648608355688\\
4	0.0110648638688409\\
5	0.0110648669597798\\
6	0.0110648701094339\\
7	0.0110648733188699\\
8	0.0110648765891732\\
9	0.0110648799214475\\
10	0.0110648833168161\\
11	0.0110648867764213\\
12	0.0110648903014255\\
13	0.0110648938930109\\
14	0.0110648975523804\\
15	0.0110649012807573\\
16	0.0110649050793863\\
17	0.0110649089495335\\
18	0.0110649128924868\\
19	0.0110649169095564\\
20	0.0110649210020748\\
21	0.0110649251713979\\
22	0.0110649294189045\\
23	0.0110649337459975\\
24	0.0110649381541037\\
25	0.0110649426446746\\
26	0.0110649472191868\\
27	0.0110649518791419\\
28	0.0110649566260679\\
29	0.0110649614615186\\
30	0.0110649663870748\\
31	0.0110649714043446\\
32	0.0110649765149634\\
33	0.011064981720595\\
34	0.011064987022932\\
35	0.0110649924236957\\
36	0.0110649979246375\\
37	0.0110650035275386\\
38	0.011065009234211\\
39	0.0110650150464981\\
40	0.0110650209662748\\
41	0.0110650269954484\\
42	0.0110650331359593\\
43	0.011065039389781\\
44	0.0110650457589214\\
45	0.0110650522454229\\
46	0.0110650588513633\\
47	0.0110650655788562\\
48	0.0110650724300517\\
49	0.0110650794071374\\
50	0.0110650865123386\\
51	0.011065093747919\\
52	0.0110651011161818\\
53	0.0110651086194702\\
54	0.0110651162601678\\
55	0.0110651240407001\\
56	0.0110651319635342\\
57	0.0110651400311808\\
58	0.0110651482461938\\
59	0.0110651566111721\\
60	0.0110651651287597\\
61	0.0110651738016469\\
62	0.0110651826325711\\
63	0.0110651916243177\\
64	0.0110652007797209\\
65	0.0110652101016648\\
66	0.0110652195930839\\
67	0.0110652292569647\\
68	0.0110652390963462\\
69	0.011065249114321\\
70	0.0110652593140365\\
71	0.0110652696986957\\
72	0.0110652802715583\\
73	0.011065291035942\\
74	0.0110653019952236\\
75	0.0110653131528397\\
76	0.0110653245122883\\
77	0.0110653360771301\\
78	0.0110653478509891\\
79	0.0110653598375547\\
80	0.0110653720405821\\
81	0.0110653844638942\\
82	0.0110653971113829\\
83	0.01106540998701\\
84	0.0110654230948091\\
85	0.0110654364388868\\
86	0.0110654500234244\\
87	0.0110654638526787\\
88	0.0110654779309845\\
89	0.0110654922627555\\
90	0.0110655068524859\\
91	0.0110655217047525\\
92	0.0110655368242159\\
93	0.0110655522156225\\
94	0.011065567883806\\
95	0.0110655838336894\\
96	0.0110656000702868\\
97	0.0110656165987052\\
98	0.0110656334241465\\
99	0.0110656505519092\\
100	0.011065667987391\\
101	0.0110656857360901\\
102	0.0110657038036078\\
103	0.0110657221956507\\
104	0.0110657409180324\\
105	0.0110657599766763\\
106	0.0110657793776175\\
107	0.0110657991270055\\
108	0.0110658192311062\\
109	0.0110658396963048\\
110	0.0110658605291079\\
111	0.0110658817361465\\
112	0.0110659033241781\\
113	0.0110659253000901\\
114	0.0110659476709017\\
115	0.0110659704437677\\
116	0.0110659936259803\\
117	0.0110660172249731\\
118	0.0110660412483234\\
119	0.0110660657037554\\
120	0.0110660905991436\\
121	0.0110661159425158\\
122	0.0110661417420566\\
123	0.0110661680061106\\
124	0.0110661947431856\\
125	0.0110662219619568\\
126	0.0110662496712697\\
127	0.0110662778801442\\
128	0.0110663065977782\\
129	0.0110663358335515\\
130	0.0110663655970296\\
131	0.0110663958979679\\
132	0.0110664267463155\\
133	0.0110664581522199\\
134	0.0110664901260306\\
135	0.0110665226783041\\
136	0.0110665558198078\\
137	0.0110665895615252\\
138	0.0110666239146598\\
139	0.0110666588906404\\
140	0.0110666945011259\\
141	0.01106673075801\\
142	0.0110667676734265\\
143	0.0110668052597542\\
144	0.0110668435296224\\
145	0.0110668824959164\\
146	0.0110669221717826\\
147	0.0110669625706342\\
148	0.0110670037061574\\
149	0.0110670455923165\\
150	0.0110670882433603\\
151	0.0110671316738282\\
152	0.0110671758985563\\
153	0.0110672209326834\\
154	0.0110672667916581\\
155	0.0110673134912449\\
156	0.0110673610475308\\
157	0.0110674094769328\\
158	0.011067458796204\\
159	0.0110675090224413\\
160	0.0110675601730927\\
161	0.0110676122659641\\
162	0.0110676653192274\\
163	0.0110677193514281\\
164	0.0110677743814927\\
165	0.0110678304287372\\
166	0.011067887512875\\
167	0.011067945654025\\
168	0.0110680048727205\\
169	0.0110680651899173\\
170	0.0110681266270027\\
171	0.0110681892058045\\
172	0.0110682529486\\
173	0.0110683178781252\\
174	0.0110683840175844\\
175	0.0110684513906597\\
176	0.0110685200215207\\
177	0.0110685899348347\\
178	0.0110686611557769\\
179	0.0110687337100405\\
180	0.0110688076238472\\
181	0.0110688829239585\\
182	0.0110689596376861\\
183	0.011069037792903\\
184	0.0110691174180554\\
185	0.0110691985421738\\
186	0.011069281194885\\
187	0.0110693654064239\\
188	0.0110694512076462\\
189	0.0110695386300402\\
190	0.01106962770574\\
191	0.0110697184675384\\
192	0.0110698109488997\\
193	0.0110699051839737\\
194	0.011070001207609\\
195	0.011070099055367\\
196	0.0110701987635363\\
197	0.0110703003691469\\
198	0.0110704039099854\\
199	0.0110705094246097\\
200	0.0110706169523645\\
201	0.0110707265333969\\
202	0.0110708382086727\\
203	0.011070952019992\\
204	0.0110710680100065\\
205	0.0110711862222363\\
206	0.0110713067010868\\
207	0.0110714294918667\\
208	0.011071554640806\\
209	0.0110716821950741\\
210	0.0110718122027989\\
211	0.0110719447130856\\
212	0.0110720797760365\\
213	0.0110722174427707\\
214	0.0110723577654445\\
215	0.0110725007972725\\
216	0.0110726465925483\\
217	0.0110727952066664\\
218	0.0110729466961447\\
219	0.0110731011186463\\
220	0.0110732585330034\\
221	0.0110734189992403\\
222	0.0110735825785978\\
223	0.0110737493335579\\
224	0.0110739193278685\\
225	0.0110740926265696\\
226	0.0110742692960191\\
227	0.0110744494039203\\
228	0.0110746330193483\\
229	0.011074820212779\\
230	0.0110750110561171\\
231	0.0110752056227257\\
232	0.0110754039874557\\
233	0.011075606226677\\
234	0.0110758124183093\\
235	0.0110760226418537\\
236	0.011076236978426\\
237	0.0110764555107893\\
238	0.0110766783233885\\
239	0.0110769055023845\\
240	0.0110771371356903\\
241	0.0110773733130067\\
242	0.0110776141258597\\
243	0.0110778596676382\\
244	0.0110781100336324\\
245	0.0110783653210738\\
246	0.011078625629175\\
247	0.0110788910591711\\
248	0.0110791617143617\\
249	0.011079437700154\\
250	0.0110797191241063\\
251	0.011080006095973\\
252	0.0110802987277507\\
253	0.0110805971337242\\
254	0.0110809014305147\\
255	0.0110812117371288\\
256	0.0110815281750074\\
257	0.0110818508680776\\
258	0.0110821799428039\\
259	0.0110825155282415\\
260	0.0110828577560907\\
261	0.0110832067607521\\
262	0.0110835626793833\\
263	0.0110839256519565\\
264	0.0110842958213179\\
265	0.0110846733332478\\
266	0.0110850583365225\\
267	0.0110854509829773\\
268	0.0110858514275708\\
269	0.0110862598284509\\
270	0.011086676347022\\
271	0.0110871011480137\\
272	0.0110875343995513\\
273	0.0110879762732271\\
274	0.0110884269441741\\
275	0.0110888865911408\\
276	0.0110893553965671\\
277	0.0110898335466623\\
278	0.0110903212314849\\
279	0.0110908186450223\\
280	0.0110913259852709\\
281	0.0110918434543203\\
282	0.0110923712584401\\
283	0.0110929096081681\\
284	0.0110934587184004\\
285	0.0110940188084837\\
286	0.0110945901023092\\
287	0.0110951728284089\\
288	0.0110957672200534\\
289	0.0110963735153521\\
290	0.0110969919573555\\
291	0.011097622794159\\
292	0.0110982662790099\\
293	0.0110989226704149\\
294	0.0110995922322519\\
295	0.0111002752338817\\
296	0.0111009719502638\\
297	0.0111016826620732\\
298	0.0111024076558199\\
299	0.011103147223971\\
300	0.0111039016650744\\
301	0.0111046712838857\\
302	0.0111054563914966\\
303	0.0111062573054662\\
304	0.0111070743499554\\
305	0.0111079078558626\\
306	0.0111087581609628\\
307	0.0111096256100496\\
308	0.0111105105550793\\
309	0.0111114133553183\\
310	0.0111123343774934\\
311	0.0111132739959458\\
312	0.0111142325927873\\
313	0.0111152105580616\\
314	0.0111162082899093\\
315	0.0111172261947381\\
316	0.0111182646873969\\
317	0.0111193241913505\\
318	0.0111204051388896\\
319	0.0111215079713165\\
320	0.0111226331391339\\
321	0.0111237811022387\\
322	0.0111249523301192\\
323	0.0111261473020572\\
324	0.0111273665073336\\
325	0.0111286104454388\\
326	0.0111298796262875\\
327	0.0111311745704377\\
328	0.0111324958093141\\
329	0.0111338438854359\\
330	0.0111352193526494\\
331	0.0111366227763646\\
332	0.0111380547337965\\
333	0.0111395158142105\\
334	0.0111410066191729\\
335	0.0111425277628045\\
336	0.0111440798720399\\
337	0.0111456635868894\\
338	0.011147279560707\\
339	0.0111489284604614\\
340	0.0111506109670131\\
341	0.0111523277753931\\
342	0.0111540795950871\\
343	0.0111558671503246\\
344	0.0111576911803718\\
345	0.0111595524398298\\
346	0.0111614516989368\\
347	0.0111633897438769\\
348	0.0111653673770919\\
349	0.0111673854175996\\
350	0.0111694447013167\\
351	0.0111715460813874\\
352	0.0111736904285166\\
353	0.0111758786313101\\
354	0.0111781115966198\\
355	0.0111803902498956\\
356	0.0111827155355408\\
357	0.0111850884172744\\
358	0.0111875098784973\\
359	0.0111899809226641\\
360	0.0111925025736589\\
361	0.0111950758761741\\
362	0.0111977018960913\\
363	0.0112003817208641\\
364	0.0112031164598979\\
365	0.0112059072449313\\
366	0.0112087552304372\\
367	0.011211661593999\\
368	0.01121462753668\\
369	0.0112176542833938\\
370	0.0112207430832663\\
371	0.011223895209981\\
372	0.011227111962129\\
373	0.0112303946635582\\
374	0.0112337446637075\\
375	0.0112371633379344\\
376	0.0112406520878365\\
377	0.0112442123415659\\
378	0.0112478455541396\\
379	0.0112515532077437\\
380	0.0112553368120285\\
381	0.0112591979044047\\
382	0.0112631380503231\\
383	0.0112671588435501\\
384	0.011271261906499\\
385	0.0112754488905291\\
386	0.0112797214761567\\
387	0.0112840813732525\\
388	0.011288530321405\\
389	0.0112930700899896\\
390	0.0112977024783915\\
391	0.0113024293161043\\
392	0.0113072524627424\\
393	0.0113121738079602\\
394	0.0113171952712657\\
395	0.0113223188017189\\
396	0.0113275463774905\\
397	0.0113328800052219\\
398	0.0113383217191028\\
399	0.0113438735801538\\
400	0.0113495376762587\\
401	0.0113553161207687\\
402	0.0113612110518829\\
403	0.0113672246288339\\
404	0.0113733590315001\\
405	0.011379616458819\\
406	0.0113859991259946\\
407	0.0113925092672923\\
408	0.0113991491318927\\
409	0.0114059209837009\\
410	0.0114128271010054\\
411	0.0114198697752345\\
412	0.0114270513060838\\
413	0.0114343740037439\\
414	0.0114418401969867\\
415	0.0114494522318439\\
416	0.0114572124648189\\
417	0.0114651232362765\\
418	0.0114731868852768\\
419	0.0114814057444074\\
420	0.011489782133807\\
421	0.0114983183538967\\
422	0.0115070166679698\\
423	0.0115158792775034\\
424	0.0115249083464358\\
425	0.011534105825837\\
426	0.0115434735116269\\
427	0.0115530131156431\\
428	0.0115627264338472\\
429	0.0115726148699541\\
430	0.0115826795884032\\
431	0.0115929215744467\\
432	0.0116033416340704\\
433	0.0116139403989606\\
434	0.0116247183394427\\
435	0.0116356758047032\\
436	0.0116468131891485\\
437	0.011658130166222\\
438	0.0116696251085954\\
439	0.0116812995788674\\
440	0.0116931609430206\\
441	0.0117052123295878\\
442	0.0117174564439994\\
443	0.0117298954678474\\
444	0.0117425309673862\\
445	0.0117553637960252\\
446	0.0117683940461984\\
447	0.0117816206196432\\
448	0.0117950409081307\\
449	0.0118086511499242\\
450	0.0118224482200987\\
451	0.0118366017004372\\
452	0.0118516965814472\\
453	0.0118666902816137\\
454	0.0118815692773868\\
455	0.0118963194924524\\
456	0.0119109263223348\\
457	0.0119253747287171\\
458	0.0119396492821493\\
459	0.0119537343753797\\
460	0.0119676129564614\\
461	0.0119812683038507\\
462	0.0119946834547688\\
463	0.012007842314727\\
464	0.0120204687221322\\
465	0.0120324695224117\\
466	0.012044296598016\\
467	0.0120559434294043\\
468	0.0120674041586191\\
469	0.0120786737931792\\
470	0.0120897480629261\\
471	0.0121006235307648\\
472	0.0121112980683367\\
473	0.012121771009318\\
474	0.0121320436906436\\
475	0.0121421182091894\\
476	0.0121519036362758\\
477	0.0121611128328961\\
478	0.0121703036927578\\
479	0.0121794798290459\\
480	0.0121886454420582\\
481	0.0121978053289121\\
482	0.0122069648802642\\
483	0.0122161300711763\\
484	0.012225307447051\\
485	0.0122345041085784\\
486	0.0122437276773052\\
487	0.0122529862511417\\
488	0.0122622962442405\\
489	0.0122716777140519\\
490	0.0122811331814007\\
491	0.0122906651825224\\
492	0.0123002762436723\\
493	0.0123099688530845\\
494	0.0123197454299354\\
495	0.0123296082904271\\
496	0.0123395596112906\\
497	0.0123496013908835\\
498	0.0123597354080664\\
499	0.0123699631789268\\
500	0.0123802850986567\\
501	0.0123907012528107\\
502	0.0124012115798434\\
503	0.012411815857483\\
504	0.0124225136890124\\
505	0.0124333044895893\\
506	0.0124441874727437\\
507	0.012455161637205\\
508	0.0124662257542208\\
509	0.0124773783555404\\
510	0.0124886177222416\\
511	0.0124999418975655\\
512	0.0125113486972977\\
513	0.0125228356956447\\
514	0.0125344002104368\\
515	0.0125460392876003\\
516	0.0125577496848353\\
517	0.0125695278544169\\
518	0.0125813699250281\\
519	0.0125932716825087\\
520	0.0126052285493902\\
521	0.0126172355630605\\
522	0.0126292873517538\\
523	0.0126413781071186\\
524	0.0126535015547546\\
525	0.0126656509225293\\
526	0.0126778189064672\\
527	0.012689997633983\\
528	0.0127021786242022\\
529	0.0127143527450922\\
530	0.0127265101670925\\
531	0.0127386403129051\\
532	0.0127507318030685\\
533	0.012762772396914\\
534	0.0127747489285695\\
535	0.0127866472375425\\
536	0.0127984520933345\\
537	0.0128101471135007\\
538	0.0128217146745347\\
539	0.0128331358157645\\
540	0.0128443901359816\\
541	0.0128554556808294\\
542	0.0128663088147238\\
543	0.0128769240150361\\
544	0.0128872740476812\\
545	0.0128973178339668\\
546	0.012907432488429\\
547	0.012917820347576\\
548	0.0129284971828948\\
549	0.0129407509586339\\
550	0.0129533441211697\\
551	0.0129660773871302\\
552	0.0129790502129792\\
553	0.012991283400045\\
554	0.013003656613236\\
555	0.013016156892264\\
556	0.0130286737560499\\
557	0.0130406673739529\\
558	0.0130532662719135\\
559	0.0130662919523494\\
560	0.0130792815800368\\
561	0.0130918882960037\\
562	0.0131064744862137\\
563	0.0131215420187478\\
564	0.0131364758718319\\
565	0.0131513443502391\\
566	0.0131654824277249\\
567	0.0131793793020266\\
568	0.0131928702337398\\
569	0.0132053903351803\\
570	0.0132172787509962\\
571	0.0132291174803308\\
572	0.0132410089728038\\
573	0.0132529107767735\\
574	0.0132644260658691\\
575	0.0132757422995737\\
576	0.0132868208689815\\
577	0.0132978465512147\\
578	0.013308835298017\\
579	0.0133197760065791\\
580	0.0133306534585244\\
581	0.0133414489699862\\
582	0.0133521416318217\\
583	0.0133627085270204\\
584	0.0133731244400726\\
585	0.0133833615638285\\
586	0.0133933893817503\\
587	0.0134031745421507\\
588	0.0134126809585648\\
589	0.0134218705998152\\
590	0.0134307062886727\\
591	0.0134391597728698\\
592	0.0134472338645631\\
593	0.0134551345125145\\
594	0.0134637112677116\\
595	0.0134740407759367\\
596	0.0134889777516421\\
597	0.0135160557778789\\
598	0.0135751148335434\\
599	0\\
600	0\\
};
\addplot [color=red,solid,forget plot]
  table[row sep=crcr]{%
1	0.0110654788392184\\
2	0.0110654825560351\\
3	0.0110654863458908\\
4	0.0110654902101524\\
5	0.0110654941502113\\
6	0.0110654981674829\\
7	0.0110655022634074\\
8	0.0110655064394505\\
9	0.011065510697103\\
10	0.0110655150378819\\
11	0.0110655194633307\\
12	0.0110655239750196\\
13	0.0110655285745461\\
14	0.0110655332635352\\
15	0.0110655380436403\\
16	0.0110655429165433\\
17	0.011065547883955\\
18	0.0110655529476159\\
19	0.0110655581092963\\
20	0.011065563370797\\
21	0.0110655687339498\\
22	0.0110655742006179\\
23	0.0110655797726964\\
24	0.0110655854521127\\
25	0.0110655912408274\\
26	0.0110655971408345\\
27	0.0110656031541618\\
28	0.0110656092828718\\
29	0.0110656155290622\\
30	0.0110656218948661\\
31	0.0110656283824531\\
32	0.0110656349940292\\
33	0.0110656417318381\\
34	0.0110656485981614\\
35	0.0110656555953192\\
36	0.0110656627256707\\
37	0.011065669991615\\
38	0.0110656773955917\\
39	0.0110656849400813\\
40	0.0110656926276062\\
41	0.0110657004607311\\
42	0.0110657084420637\\
43	0.0110657165742556\\
44	0.0110657248600027\\
45	0.0110657333020462\\
46	0.011065741903173\\
47	0.0110657506662169\\
48	0.0110657595940587\\
49	0.0110657686896276\\
50	0.0110657779559015\\
51	0.0110657873959082\\
52	0.0110657970127258\\
53	0.0110658068094837\\
54	0.0110658167893635\\
55	0.0110658269555997\\
56	0.0110658373114808\\
57	0.0110658478603496\\
58	0.011065858605605\\
59	0.011065869550702\\
60	0.0110658806991533\\
61	0.0110658920545298\\
62	0.0110659036204617\\
63	0.0110659154006397\\
64	0.0110659273988156\\
65	0.0110659396188037\\
66	0.0110659520644815\\
67	0.011065964739791\\
68	0.0110659776487397\\
69	0.0110659907954016\\
70	0.0110660041839186\\
71	0.0110660178185012\\
72	0.0110660317034302\\
73	0.0110660458430574\\
74	0.0110660602418071\\
75	0.0110660749041774\\
76	0.0110660898347412\\
77	0.0110661050381476\\
78	0.0110661205191235\\
79	0.0110661362824744\\
80	0.0110661523330863\\
81	0.0110661686759268\\
82	0.0110661853160468\\
83	0.0110662022585817\\
84	0.0110662195087529\\
85	0.0110662370718696\\
86	0.0110662549533303\\
87	0.0110662731586241\\
88	0.0110662916933329\\
89	0.0110663105631324\\
90	0.0110663297737945\\
91	0.0110663493311886\\
92	0.0110663692412836\\
93	0.0110663895101496\\
94	0.01106641014396\\
95	0.0110664311489931\\
96	0.0110664525316346\\
97	0.011066474298379\\
98	0.0110664964558322\\
99	0.0110665190107133\\
100	0.0110665419698569\\
101	0.0110665653402152\\
102	0.0110665891288605\\
103	0.0110666133429874\\
104	0.0110666379899151\\
105	0.0110666630770901\\
106	0.0110666886120884\\
107	0.0110667146026183\\
108	0.011066741056523\\
109	0.0110667679817833\\
110	0.0110667953865202\\
111	0.0110668232789979\\
112	0.0110668516676267\\
113	0.011066880560966\\
114	0.0110669099677274\\
115	0.0110669398967773\\
116	0.0110669703571411\\
117	0.0110670013580056\\
118	0.0110670329087227\\
119	0.011067065018813\\
120	0.011067097697969\\
121	0.0110671309560593\\
122	0.0110671648031314\\
123	0.0110671992494167\\
124	0.0110672343053331\\
125	0.0110672699814903\\
126	0.0110673062886928\\
127	0.011067343237945\\
128	0.0110673808404548\\
129	0.0110674191076386\\
130	0.0110674580511255\\
131	0.0110674976827619\\
132	0.0110675380146165\\
133	0.0110675790589851\\
134	0.0110676208283956\\
135	0.011067663335613\\
136	0.0110677065936447\\
137	0.0110677506157462\\
138	0.0110677954154261\\
139	0.011067841006452\\
140	0.0110678874028564\\
141	0.0110679346189423\\
142	0.0110679826692898\\
143	0.0110680315687615\\
144	0.0110680813325096\\
145	0.0110681319759822\\
146	0.0110681835149296\\
147	0.0110682359654118\\
148	0.0110682893438047\\
149	0.0110683436668078\\
150	0.0110683989514515\\
151	0.0110684552151042\\
152	0.0110685124754801\\
153	0.0110685707506473\\
154	0.0110686300590356\\
155	0.0110686904194443\\
156	0.0110687518510513\\
157	0.0110688143734211\\
158	0.0110688780065137\\
159	0.0110689427706935\\
160	0.0110690086867385\\
161	0.0110690757758495\\
162	0.0110691440596597\\
163	0.0110692135602444\\
164	0.0110692843001309\\
165	0.0110693563023085\\
166	0.0110694295902392\\
167	0.0110695041878678\\
168	0.0110695801196331\\
169	0.0110696574104787\\
170	0.0110697360858638\\
171	0.0110698161717755\\
172	0.0110698976947397\\
173	0.0110699806818333\\
174	0.0110700651606963\\
175	0.0110701511595439\\
176	0.0110702387071796\\
177	0.0110703278330076\\
178	0.0110704185670457\\
179	0.0110705109399393\\
180	0.0110706049829742\\
181	0.0110707007280906\\
182	0.0110707982078977\\
183	0.011070897455687\\
184	0.0110709985054478\\
185	0.0110711013918812\\
186	0.0110712061504159\\
187	0.0110713128172229\\
188	0.0110714214292315\\
189	0.0110715320241451\\
190	0.0110716446404572\\
191	0.0110717593174676\\
192	0.0110718760952998\\
193	0.0110719950149173\\
194	0.0110721161181409\\
195	0.0110722394476667\\
196	0.0110723650470833\\
197	0.0110724929608906\\
198	0.0110726232345176\\
199	0.0110727559143416\\
200	0.0110728910477069\\
201	0.0110730286829444\\
202	0.0110731688693913\\
203	0.0110733116574108\\
204	0.0110734570984128\\
205	0.0110736052448746\\
206	0.0110737561503616\\
207	0.0110739098695489\\
208	0.0110740664582433\\
209	0.0110742259734052\\
210	0.011074388473171\\
211	0.0110745540168765\\
212	0.0110747226650797\\
213	0.0110748944795851\\
214	0.0110750695234676\\
215	0.0110752478610972\\
216	0.0110754295581641\\
217	0.0110756146817044\\
218	0.0110758033001263\\
219	0.0110759954832362\\
220	0.0110761913022664\\
221	0.0110763908299027\\
222	0.0110765941403121\\
223	0.0110768013091723\\
224	0.0110770124137009\\
225	0.0110772275326851\\
226	0.0110774467465129\\
227	0.0110776701372042\\
228	0.0110778977884429\\
229	0.0110781297856097\\
230	0.0110783662158154\\
231	0.0110786071679353\\
232	0.0110788527326445\\
233	0.011079103002453\\
234	0.0110793580717433\\
235	0.0110796180368069\\
236	0.0110798829958835\\
237	0.0110801530491999\\
238	0.0110804282990102\\
239	0.0110807088496368\\
240	0.0110809948075128\\
241	0.0110812862812248\\
242	0.0110815833815568\\
243	0.0110818862215358\\
244	0.0110821949164773\\
245	0.0110825095840328\\
246	0.0110828303442376\\
247	0.0110831573195607\\
248	0.0110834906349545\\
249	0.0110838304179067\\
250	0.0110841767984926\\
251	0.011084529909429\\
252	0.0110848898861291\\
253	0.0110852568667581\\
254	0.0110856309922908\\
255	0.0110860124065696\\
256	0.0110864012563643\\
257	0.0110867976914325\\
258	0.0110872018645819\\
259	0.0110876139317329\\
260	0.0110880340519836\\
261	0.0110884623876753\\
262	0.011088899104459\\
263	0.0110893443713644\\
264	0.0110897983608689\\
265	0.0110902612489689\\
266	0.0110907332152518\\
267	0.0110912144429701\\
268	0.0110917051191163\\
269	0.0110922054345\\
270	0.0110927155838255\\
271	0.0110932357657722\\
272	0.0110937661830754\\
273	0.01109430704261\\
274	0.0110948585554745\\
275	0.0110954209370778\\
276	0.0110959944072277\\
277	0.0110965791902206\\
278	0.0110971755149343\\
279	0.0110977836149213\\
280	0.0110984037285057\\
281	0.0110990360988814\\
282	0.0110996809742122\\
283	0.0111003386077354\\
284	0.0111010092578667\\
285	0.0111016931883079\\
286	0.0111023906681572\\
287	0.0111031019720217\\
288	0.0111038273801333\\
289	0.0111045671784663\\
290	0.0111053216588589\\
291	0.0111060911191363\\
292	0.011106875863238\\
293	0.0111076762013467\\
294	0.0111084924500211\\
295	0.0111093249323317\\
296	0.0111101739779984\\
297	0.0111110399235329\\
298	0.0111119231123825\\
299	0.0111128238950779\\
300	0.0111137426293833\\
301	0.0111146796804497\\
302	0.0111156354209713\\
303	0.0111166102313439\\
304	0.0111176044998271\\
305	0.0111186186227094\\
306	0.0111196530044754\\
307	0.0111207080579762\\
308	0.0111217842046012\\
309	0.0111228818744546\\
310	0.0111240015065333\\
311	0.0111251435489083\\
312	0.0111263084589084\\
313	0.0111274967033076\\
314	0.0111287087585151\\
315	0.011129945110768\\
316	0.0111312062563267\\
317	0.0111324927016753\\
318	0.0111338049637239\\
319	0.0111351435700148\\
320	0.0111365090589328\\
321	0.0111379019799188\\
322	0.011139322893688\\
323	0.0111407723724531\\
324	0.0111422510001514\\
325	0.0111437593726783\\
326	0.0111452980981248\\
327	0.0111468677970206\\
328	0.0111484691025837\\
329	0.011150102660976\\
330	0.011151769131564\\
331	0.0111534691871862\\
332	0.0111552035144266\\
333	0.0111569728138959\\
334	0.0111587778005172\\
335	0.011160619203818\\
336	0.0111624977682282\\
337	0.0111644142533813\\
338	0.0111663694344195\\
339	0.011168364102302\\
340	0.0111703990641225\\
341	0.0111724751434315\\
342	0.0111745931805426\\
343	0.0111767540328467\\
344	0.0111789585751259\\
345	0.0111812076998633\\
346	0.0111835023175514\\
347	0.0111858433569973\\
348	0.011188231765631\\
349	0.0111906685098013\\
350	0.0111931545750712\\
351	0.0111956909665116\\
352	0.01119827870899\\
353	0.0112009188474486\\
354	0.0112036124471892\\
355	0.0112063605941655\\
356	0.0112091643952692\\
357	0.0112120249786188\\
358	0.0112149434938525\\
359	0.011217921112428\\
360	0.0112209590279296\\
361	0.0112240584563857\\
362	0.0112272206365981\\
363	0.0112304468304815\\
364	0.0112337383234066\\
365	0.0112370964245236\\
366	0.0112405224670688\\
367	0.0112440178089514\\
368	0.0112475838331259\\
369	0.0112512219479109\\
370	0.0112549335873582\\
371	0.0112587202115855\\
372	0.0112625833069188\\
373	0.0112665243860911\\
374	0.011270544988467\\
375	0.0112746466801043\\
376	0.0112788310537305\\
377	0.0112830997286273\\
378	0.0112874543504153\\
379	0.0112918965907371\\
380	0.0112964281468345\\
381	0.0113010507410177\\
382	0.0113057661200114\\
383	0.0113105760541221\\
384	0.0113154823361842\\
385	0.0113204867811843\\
386	0.0113255912257362\\
387	0.0113307975266277\\
388	0.0113361075590537\\
389	0.0113415232173349\\
390	0.0113470464124603\\
391	0.0113526790729982\\
392	0.0113584231454704\\
393	0.0113642805947291\\
394	0.0113702534042987\\
395	0.0113763435766419\\
396	0.0113825531332989\\
397	0.0113888841147703\\
398	0.011395338579582\\
399	0.0114019186003659\\
400	0.0114086262613894\\
401	0.0114154636685429\\
402	0.0114224329417266\\
403	0.0114295362233984\\
404	0.0114367756423113\\
405	0.0114441533192023\\
406	0.0114516713606121\\
407	0.0114593318262876\\
408	0.0114671367678746\\
409	0.0114750881348862\\
410	0.0114831877665219\\
411	0.011491437385288\\
412	0.0114998385900102\\
413	0.0115083927934869\\
414	0.0115171012326338\\
415	0.0115259650778664\\
416	0.0115349854530479\\
417	0.0115441634406638\\
418	0.0115534997632943\\
419	0.0115629950134331\\
420	0.0115726496655751\\
421	0.0115824640968761\\
422	0.0115924386311388\\
423	0.0116025734950854\\
424	0.0116128692638888\\
425	0.011623329789713\\
426	0.011633957822805\\
427	0.0116447555055206\\
428	0.0116557249137582\\
429	0.0116668713611722\\
430	0.0116781974182577\\
431	0.0116897043609441\\
432	0.0117013927699411\\
433	0.0117132624133717\\
434	0.0117253121392036\\
435	0.0117375397732391\\
436	0.0117499420961473\\
437	0.0117625144310969\\
438	0.0117758560399216\\
439	0.0117894674134734\\
440	0.0118029894575224\\
441	0.0118164099705272\\
442	0.0118297162722751\\
443	0.0118428952446038\\
444	0.0118559332399957\\
445	0.0118688161988171\\
446	0.0118815297001557\\
447	0.0118940590655509\\
448	0.0119063895998902\\
449	0.0119185058611117\\
450	0.0119303922637502\\
451	0.0119419300008739\\
452	0.0119527470665781\\
453	0.0119634065695056\\
454	0.0119739030383462\\
455	0.0119842307349347\\
456	0.0119943845632567\\
457	0.0120043601928905\\
458	0.0120141541646721\\
459	0.0120237640974739\\
460	0.0120331881808683\\
461	0.0120424260409478\\
462	0.0120514785273426\\
463	0.012060348256128\\
464	0.0120688519203595\\
465	0.0120769633773652\\
466	0.0120850565833262\\
467	0.0120931350250382\\
468	0.0121012027318994\\
469	0.0121092642855019\\
470	0.0121173248245163\\
471	0.0121253900406385\\
472	0.012133466163393\\
473	0.0121415599377372\\
474	0.0121496785936689\\
475	0.0121578298162321\\
476	0.0121660252473006\\
477	0.0121742881427895\\
478	0.0121826212162018\\
479	0.0121910272252803\\
480	0.0121995089512901\\
481	0.0122080691758438\\
482	0.0122167106555299\\
483	0.0122254360943799\\
484	0.0122342481141834\\
485	0.0122431492225112\\
486	0.0122521417789232\\
487	0.0122612279594847\\
488	0.0122704094060967\\
489	0.0122796870356028\\
490	0.0122890616833396\\
491	0.0122985340938528\\
492	0.0123081049116668\\
493	0.012317774672219\\
494	0.0123275437930923\\
495	0.0123374125656944\\
496	0.0123473811475309\\
497	0.0123574495552308\\
498	0.0123676176584917\\
499	0.0123778851751265\\
500	0.0123882517006474\\
501	0.012398716707723\\
502	0.0124092795388883\\
503	0.0124199393990324\\
504	0.0124306953476433\\
505	0.0124415462907825\\
506	0.0124524909727523\\
507	0.0124635279674112\\
508	0.0124746556690771\\
509	0.0124858722829499\\
510	0.0124971758149644\\
511	0.0125085640600067\\
512	0.0125200345883983\\
513	0.0125315847314718\\
514	0.0125432115661564\\
515	0.0125549118984843\\
516	0.0125666822459204\\
517	0.0125785188184073\\
518	0.0125904174980062\\
519	0.0126023738170057\\
520	0.0126143829343539\\
521	0.0126264396102611\\
522	0.0126385381788273\\
523	0.0126506725185891\\
524	0.0126628360207987\\
525	0.0126750215552332\\
526	0.0126872214333117\\
527	0.0126994273682752\\
528	0.012711630432161\\
529	0.0127238210092769\\
530	0.0127359887458533\\
531	0.0127481224955192\\
532	0.0127602102602153\\
533	0.0127722391260862\\
534	0.0127841951938639\\
535	0.0127960635032276\\
536	0.0128078279519002\\
537	0.0128194712084866\\
538	0.012830974617619\\
539	0.012842318056499\\
540	0.0128534797603971\\
541	0.0128644361906405\\
542	0.0128751621616999\\
543	0.0128856343763971\\
544	0.0128958158226031\\
545	0.012906873956167\\
546	0.0129187928141379\\
547	0.0129308400010039\\
548	0.0129431205246169\\
549	0.0129547693726181\\
550	0.0129664376602281\\
551	0.0129782299193289\\
552	0.0129900292355768\\
553	0.0130012990174047\\
554	0.0130127502747141\\
555	0.0130245386352892\\
556	0.0130370554092916\\
557	0.0130495984537013\\
558	0.0130620343918673\\
559	0.0130767491608591\\
560	0.01309148410142\\
561	0.0131061948644112\\
562	0.0131197439673824\\
563	0.0131334739835205\\
564	0.0131473091419805\\
565	0.0131603272205647\\
566	0.0131723874138756\\
567	0.0131842407045699\\
568	0.0131958207769496\\
569	0.0132073548170946\\
570	0.0132189563508797\\
571	0.0132308155192537\\
572	0.0132423773181155\\
573	0.0132536887823157\\
574	0.0132647731328702\\
575	0.0132758105462532\\
576	0.0132868405055488\\
577	0.013297854502466\\
578	0.01330883885008\\
579	0.0133197780028337\\
580	0.0133306546245664\\
581	0.0133414496207317\\
582	0.0133521420230989\\
583	0.0133627087929893\\
584	0.0133731246090603\\
585	0.0133833616591944\\
586	0.0133933894304777\\
587	0.0134031745665083\\
588	0.0134126809661912\\
589	0.0134218705998152\\
590	0.0134307062886727\\
591	0.0134391597728698\\
592	0.0134472338645631\\
593	0.0134551345125145\\
594	0.0134637112677116\\
595	0.0134740407759367\\
596	0.0134889777516421\\
597	0.0135160557778789\\
598	0.0135751148335434\\
599	0\\
600	0\\
};
\addplot [color=mycolor20,solid,forget plot]
  table[row sep=crcr]{%
1	0.0110656588731661\\
2	0.0110656635661861\\
3	0.011065668354941\\
4	0.011065673241298\\
5	0.0110656782271588\\
6	0.0110656833144596\\
7	0.0110656885051719\\
8	0.0110656938013029\\
9	0.0110656992048966\\
10	0.0110657047180336\\
11	0.0110657103428322\\
12	0.0110657160814492\\
13	0.01106572193608\\
14	0.0110657279089596\\
15	0.011065734002363\\
16	0.011065740218606\\
17	0.0110657465600459\\
18	0.0110657530290821\\
19	0.0110657596281567\\
20	0.011065766359755\\
21	0.011065773226407\\
22	0.0110657802306869\\
23	0.0110657873752148\\
24	0.011065794662657\\
25	0.0110658020957267\\
26	0.0110658096771849\\
27	0.011065817409841\\
28	0.0110658252965537\\
29	0.0110658333402315\\
30	0.011065841543834\\
31	0.0110658499103719\\
32	0.0110658584429087\\
33	0.0110658671445607\\
34	0.0110658760184984\\
35	0.0110658850679469\\
36	0.0110658942961871\\
37	0.0110659037065563\\
38	0.0110659133024493\\
39	0.0110659230873188\\
40	0.0110659330646768\\
41	0.0110659432380954\\
42	0.0110659536112074\\
43	0.0110659641877075\\
44	0.0110659749713531\\
45	0.0110659859659653\\
46	0.0110659971754299\\
47	0.0110660086036983\\
48	0.0110660202547882\\
49	0.0110660321327854\\
50	0.0110660442418437\\
51	0.0110660565861869\\
52	0.0110660691701093\\
53	0.0110660819979767\\
54	0.0110660950742279\\
55	0.0110661084033752\\
56	0.011066121990006\\
57	0.0110661358387835\\
58	0.0110661499544482\\
59	0.0110661643418185\\
60	0.0110661790057925\\
61	0.0110661939513485\\
62	0.0110662091835468\\
63	0.0110662247075303\\
64	0.0110662405285262\\
65	0.011066256651847\\
66	0.0110662730828915\\
67	0.0110662898271466\\
68	0.0110663068901884\\
69	0.0110663242776829\\
70	0.0110663419953884\\
71	0.0110663600491558\\
72	0.0110663784449307\\
73	0.0110663971887542\\
74	0.0110664162867648\\
75	0.0110664357451994\\
76	0.0110664555703952\\
77	0.0110664757687905\\
78	0.0110664963469269\\
79	0.0110665173114504\\
80	0.0110665386691129\\
81	0.011066560426774\\
82	0.0110665825914024\\
83	0.0110666051700776\\
84	0.0110666281699915\\
85	0.0110666515984499\\
86	0.0110666754628747\\
87	0.0110666997708051\\
88	0.0110667245298995\\
89	0.0110667497479375\\
90	0.0110667754328214\\
91	0.0110668015925783\\
92	0.011066828235362\\
93	0.0110668553694546\\
94	0.011066883003269\\
95	0.0110669111453504\\
96	0.0110669398043786\\
97	0.0110669689891701\\
98	0.01106699870868\\
99	0.0110670289720046\\
100	0.0110670597883831\\
101	0.0110670911672003\\
102	0.0110671231179887\\
103	0.0110671556504308\\
104	0.0110671887743619\\
105	0.0110672224997719\\
106	0.0110672568368086\\
107	0.0110672917957797\\
108	0.0110673273871558\\
109	0.0110673636215728\\
110	0.011067400509835\\
111	0.0110674380629178\\
112	0.0110674762919705\\
113	0.0110675152083196\\
114	0.0110675548234713\\
115	0.0110675951491154\\
116	0.0110676361971277\\
117	0.0110676779795741\\
118	0.0110677205087132\\
119	0.0110677637970005\\
120	0.0110678078570913\\
121	0.0110678527018448\\
122	0.0110678983443277\\
123	0.0110679447978183\\
124	0.0110679920758098\\
125	0.0110680401920152\\
126	0.011068089160371\\
127	0.0110681389950414\\
128	0.011068189710423\\
129	0.0110682413211494\\
130	0.0110682938420954\\
131	0.0110683472883824\\
132	0.0110684016753828\\
133	0.0110684570187255\\
134	0.0110685133343006\\
135	0.0110685706382656\\
136	0.0110686289470499\\
137	0.0110686882773614\\
138	0.0110687486461916\\
139	0.0110688100708223\\
140	0.0110688725688311\\
141	0.0110689361580985\\
142	0.0110690008568137\\
143	0.011069066683482\\
144	0.0110691336569312\\
145	0.0110692017963191\\
146	0.0110692711211408\\
147	0.0110693416512361\\
148	0.0110694134067975\\
149	0.011069486408378\\
150	0.0110695606768993\\
151	0.0110696362336606\\
152	0.0110697131003469\\
153	0.0110697912990384\\
154	0.0110698708522192\\
155	0.0110699517827873\\
156	0.011070034114064\\
157	0.0110701178698039\\
158	0.0110702030742057\\
159	0.011070289751922\\
160	0.0110703779280709\\
161	0.0110704676282469\\
162	0.011070558878532\\
163	0.0110706517055084\\
164	0.0110707461362698\\
165	0.0110708421984341\\
166	0.0110709399201563\\
167	0.0110710393301417\\
168	0.0110711404576588\\
169	0.0110712433325538\\
170	0.0110713479852643\\
171	0.0110714544468339\\
172	0.0110715627489273\\
173	0.0110716729238451\\
174	0.01107178500454\\
175	0.0110718990246322\\
176	0.0110720150184262\\
177	0.0110721330209276\\
178	0.0110722530678597\\
179	0.0110723751956818\\
180	0.0110724994416068\\
181	0.0110726258436193\\
182	0.011072754440495\\
183	0.0110728852718191\\
184	0.0110730183780066\\
185	0.0110731538003218\\
186	0.0110732915808987\\
187	0.0110734317627623\\
188	0.0110735743898491\\
189	0.0110737195070293\\
190	0.0110738671601286\\
191	0.0110740173959506\\
192	0.0110741702622998\\
193	0.0110743258080048\\
194	0.0110744840829419\\
195	0.0110746451380594\\
196	0.0110748090254018\\
197	0.0110749757981348\\
198	0.0110751455105705\\
199	0.0110753182181933\\
200	0.0110754939776854\\
201	0.0110756728469539\\
202	0.011075854885157\\
203	0.0110760401527315\\
204	0.0110762287114203\\
205	0.0110764206243\\
206	0.0110766159558096\\
207	0.0110768147717787\\
208	0.0110770171394568\\
209	0.0110772231275421\\
210	0.0110774328062119\\
211	0.0110776462471517\\
212	0.0110778635235862\\
213	0.0110780847103094\\
214	0.0110783098837161\\
215	0.0110785391218329\\
216	0.01107877250435\\
217	0.011079010112653\\
218	0.0110792520298556\\
219	0.0110794983408319\\
220	0.01107974913225\\
221	0.0110800044926049\\
222	0.0110802645122528\\
223	0.0110805292834451\\
224	0.0110807989003635\\
225	0.0110810734591544\\
226	0.0110813530579656\\
227	0.0110816377969816\\
228	0.0110819277784606\\
229	0.0110822231067719\\
230	0.0110825238884336\\
231	0.0110828302321515\\
232	0.0110831422488579\\
233	0.011083460051752\\
234	0.0110837837563407\\
235	0.01108411348048\\
236	0.0110844493444183\\
237	0.0110847914708392\\
238	0.0110851399849069\\
239	0.0110854950143115\\
240	0.0110858566893161\\
241	0.0110862251428051\\
242	0.0110866005103335\\
243	0.0110869829301781\\
244	0.0110873725433894\\
245	0.0110877694938457\\
246	0.0110881739283083\\
247	0.0110885859964784\\
248	0.0110890058510558\\
249	0.0110894336477992\\
250	0.0110898695455883\\
251	0.0110903137064869\\
252	0.0110907662958094\\
253	0.0110912274821874\\
254	0.0110916974376391\\
255	0.0110921763376401\\
256	0.011092664361196\\
257	0.011093161690917\\
258	0.0110936685130938\\
259	0.0110941850177758\\
260	0.0110947113988498\\
261	0.0110952478541224\\
262	0.0110957945854023\\
263	0.0110963517985851\\
264	0.0110969197037383\\
265	0.0110974985151895\\
266	0.0110980884516149\\
267	0.01109868973613\\
268	0.0110993025963811\\
269	0.011099927264639\\
270	0.0111005639778934\\
271	0.0111012129779492\\
272	0.0111018745115237\\
273	0.0111025488303456\\
274	0.0111032361912549\\
275	0.0111039368563045\\
276	0.0111046510928631\\
277	0.0111053791737189\\
278	0.0111061213771856\\
279	0.0111068779872093\\
280	0.0111076492934765\\
281	0.0111084355915246\\
282	0.0111092371828535\\
283	0.0111100543750385\\
284	0.0111108874818465\\
285	0.0111117368233526\\
286	0.0111126027260602\\
287	0.0111134855230232\\
288	0.0111143855539703\\
289	0.0111153031654337\\
290	0.0111162387108792\\
291	0.0111171925508416\\
292	0.0111181650530624\\
293	0.011119156592633\\
294	0.0111201675521415\\
295	0.0111211983218244\\
296	0.0111222492997236\\
297	0.0111233208918489\\
298	0.0111244135123462\\
299	0.0111255275836715\\
300	0.0111266635367701\\
301	0.0111278218112625\\
302	0.0111290028556362\\
303	0.0111302071274427\\
304	0.0111314350934994\\
305	0.0111326872300972\\
306	0.0111339640232183\\
307	0.0111352659687583\\
308	0.0111365935727425\\
309	0.0111379473515509\\
310	0.0111393278321466\\
311	0.0111407355523043\\
312	0.0111421710608418\\
313	0.0111436349178529\\
314	0.0111451276949402\\
315	0.0111466499754475\\
316	0.011148202354693\\
317	0.0111497854401991\\
318	0.0111513998519233\\
319	0.0111530462224844\\
320	0.0111547251973883\\
321	0.0111564374352496\\
322	0.0111581836080109\\
323	0.011159964401158\\
324	0.0111617805139326\\
325	0.011163632659543\\
326	0.0111655215653784\\
327	0.0111674479732104\\
328	0.0111694126394004\\
329	0.0111714163351074\\
330	0.0111734598464994\\
331	0.0111755439749672\\
332	0.0111776695373386\\
333	0.0111798373661039\\
334	0.0111820483096634\\
335	0.0111843032325815\\
336	0.0111866030158573\\
337	0.0111889485572148\\
338	0.0111913407714107\\
339	0.0111937805905531\\
340	0.0111962689644245\\
341	0.0111988068609057\\
342	0.0112013952664816\\
343	0.0112040351865738\\
344	0.011206727645988\\
345	0.011209473689358\\
346	0.0112122743815796\\
347	0.0112151308082252\\
348	0.0112180440759428\\
349	0.0112210153129062\\
350	0.0112240456691409\\
351	0.0112271363168317\\
352	0.0112302884506114\\
353	0.0112335032877874\\
354	0.011236782068394\\
355	0.0112401260552502\\
356	0.0112435365340412\\
357	0.0112470148132364\\
358	0.0112505622239224\\
359	0.0112541801195469\\
360	0.0112578698755749\\
361	0.0112616328890624\\
362	0.0112654705781572\\
363	0.0112693843815436\\
364	0.0112733757578449\\
365	0.0112774461849569\\
366	0.0112815971590794\\
367	0.0112858301929141\\
368	0.0112901468173314\\
369	0.0112945485809067\\
370	0.0112990370495442\\
371	0.0113036138072524\\
372	0.0113082804573839\\
373	0.0113130386221138\\
374	0.011317889943075\\
375	0.0113228360832951\\
376	0.0113278787281646\\
377	0.0113330195863668\\
378	0.0113382603907182\\
379	0.0113436028988527\\
380	0.0113490488936747\\
381	0.0113546001835024\\
382	0.0113602586018019\\
383	0.0113660260063192\\
384	0.011371904276819\\
385	0.0113778953087333\\
386	0.0113840010146819\\
387	0.0113902233257655\\
388	0.0113965641818177\\
389	0.0114030255123961\\
390	0.011409609248763\\
391	0.0114163172479929\\
392	0.0114231512907846\\
393	0.0114301130741022\\
394	0.011437204204364\\
395	0.0114444261914239\\
396	0.0114517804436914\\
397	0.0114592682649465\\
398	0.0114668908537145\\
399	0.0114746493042378\\
400	0.0114825445740172\\
401	0.0114905774474638\\
402	0.0114987486950691\\
403	0.0115070590027617\\
404	0.0115155092029484\\
405	0.0115240998923759\\
406	0.0115328315832527\\
407	0.0115417047900948\\
408	0.0115507199473595\\
409	0.011559880551949\\
410	0.0115691900038285\\
411	0.0115786515690362\\
412	0.0115882683573538\\
413	0.0115980434026293\\
414	0.011607978875719\\
415	0.0116180758717898\\
416	0.0116283356287589\\
417	0.01163875981301\\
418	0.0116493513417758\\
419	0.0116601098803842\\
420	0.0116710342296406\\
421	0.0116821222266719\\
422	0.0116933706785615\\
423	0.0117049271349924\\
424	0.0117172129725119\\
425	0.0117294351192178\\
426	0.0117415830819485\\
427	0.0117536457520934\\
428	0.0117656116160091\\
429	0.0117774703065973\\
430	0.0117892100098877\\
431	0.0118008181955253\\
432	0.0118122819947597\\
433	0.011823588257149\\
434	0.0118347235171335\\
435	0.0118456741218799\\
436	0.0118564263042342\\
437	0.0118669662942504\\
438	0.0118769230014001\\
439	0.0118865480721362\\
440	0.0118960284246959\\
441	0.0119053582604444\\
442	0.0119145321531706\\
443	0.0119235454410943\\
444	0.011932394946901\\
445	0.0119410772972738\\
446	0.0119495899623667\\
447	0.0119579313732811\\
448	0.0119661011118754\\
449	0.0119740996408993\\
450	0.0119819286218101\\
451	0.0119895164676886\\
452	0.011996619753397\\
453	0.012003701392746\\
454	0.01201076468025\\
455	0.0120178130323459\\
456	0.0120248503630588\\
457	0.0120318810905243\\
458	0.0120389101387957\\
459	0.01204594293424\\
460	0.012052985397651\\
461	0.0120600439208837\\
462	0.0120671253375711\\
463	0.0120742368832063\\
464	0.0120813931208556\\
465	0.0120886103077395\\
466	0.012095891125198\\
467	0.0121032383081965\\
468	0.0121106546277406\\
469	0.0121181428711796\\
470	0.0121257058204047\\
471	0.0121333462280934\\
472	0.0121410667922466\\
473	0.0121488701291379\\
474	0.0121567587448235\\
475	0.0121647350050365\\
476	0.0121728009653169\\
477	0.0121809578887974\\
478	0.0121892069921755\\
479	0.0121975494389048\\
480	0.0122059863325002\\
481	0.0122145187100659\\
482	0.0122231475361539\\
483	0.0122318736970723\\
484	0.0122406979957727\\
485	0.0122496211474671\\
486	0.0122586437761191\\
487	0.0122677664119648\\
488	0.0122769895027841\\
489	0.012286313433967\\
490	0.0122957385247837\\
491	0.0123052650246323\\
492	0.0123148931092603\\
493	0.0123246228769539\\
494	0.0123344543446787\\
495	0.0123443874441547\\
496	0.0123544220178348\\
497	0.0123645578147505\\
498	0.0123747944861783\\
499	0.0123851315810655\\
500	0.0123955685397814\\
501	0.0124061046872306\\
502	0.0124167392255343\\
503	0.0124274712262411\\
504	0.012438299622023\\
505	0.0124492231978107\\
506	0.0124602405813163\\
507	0.0124713502328876\\
508	0.0124825504346326\\
509	0.0124938392787495\\
510	0.0125052146549891\\
511	0.0125166742372163\\
512	0.0125282154690323\\
513	0.0125398355483824\\
514	0.0125515314110666\\
515	0.0125632997130623\\
516	0.0125751368115598\\
517	0.0125870387446058\\
518	0.012599001209234\\
519	0.0126110195379583\\
520	0.0126230886734865\\
521	0.0126352031415027\\
522	0.0126473570213505\\
523	0.0126595439144294\\
524	0.0126717569100985\\
525	0.0126839885488629\\
526	0.0126962307825912\\
527	0.0127084749314948\\
528	0.0127207116375674\\
529	0.0127329308141568\\
530	0.0127451215912774\\
531	0.0127572722562432\\
532	0.0127693701891896\\
533	0.0127814017945067\\
534	0.0127933524269847\\
535	0.0128052063114802\\
536	0.0128169463939496\\
537	0.0128285542015397\\
538	0.0128400097268664\\
539	0.0128512933422269\\
540	0.0128623867952452\\
541	0.0128733502284654\\
542	0.0128854094050171\\
543	0.0128970012989598\\
544	0.012908595658433\\
545	0.0129198614533375\\
546	0.0129308452656219\\
547	0.0129419518643324\\
548	0.0129530879392266\\
549	0.0129637514845416\\
550	0.0129745017007499\\
551	0.0129854097915782\\
552	0.0129964303233532\\
553	0.0130082575189452\\
554	0.0130203859606286\\
555	0.0130328503132304\\
556	0.0130472039465158\\
557	0.0130616189824782\\
558	0.0130760062244382\\
559	0.0130889106820684\\
560	0.0131018923632267\\
561	0.0131152098797295\\
562	0.0131277924658908\\
563	0.0131398059732458\\
564	0.01315155121627\\
565	0.0131629995776241\\
566	0.0131743408391387\\
567	0.0131856935415066\\
568	0.0131970799257051\\
569	0.0132087003483928\\
570	0.0132203534899281\\
571	0.0132316468187079\\
572	0.0132427546344582\\
573	0.0132537627637236\\
574	0.0132647834058537\\
575	0.0132758134659479\\
576	0.013286841680444\\
577	0.013297855034141\\
578	0.0133088391434586\\
579	0.0133197781722865\\
580	0.0133306547203085\\
581	0.0133414496789316\\
582	0.0133521420618359\\
583	0.0133627088171417\\
584	0.0133731246225379\\
585	0.0133833616660035\\
586	0.0133933894336898\\
587	0.0134031745674478\\
588	0.0134126809661912\\
589	0.0134218705998152\\
590	0.0134307062886727\\
591	0.0134391597728698\\
592	0.0134472338645631\\
593	0.0134551345125145\\
594	0.0134637112677116\\
595	0.0134740407759367\\
596	0.0134889777516421\\
597	0.0135160557778789\\
598	0.0135751148335434\\
599	0\\
600	0\\
};
\addplot [color=mycolor21,solid,forget plot]
  table[row sep=crcr]{%
1	0.0110657345758834\\
2	0.0110657403531717\\
3	0.0110657462530208\\
4	0.011065752277939\\
5	0.0110657584304834\\
6	0.0110657647132605\\
7	0.0110657711289273\\
8	0.011065777680192\\
9	0.0110657843698148\\
10	0.0110657912006091\\
11	0.0110657981754424\\
12	0.0110658052972369\\
13	0.0110658125689709\\
14	0.0110658199936793\\
15	0.0110658275744552\\
16	0.0110658353144502\\
17	0.0110658432168759\\
18	0.0110658512850048\\
19	0.0110658595221714\\
20	0.0110658679317729\\
21	0.0110658765172707\\
22	0.0110658852821914\\
23	0.0110658942301277\\
24	0.0110659033647395\\
25	0.0110659126897556\\
26	0.0110659222089739\\
27	0.0110659319262633\\
28	0.0110659418455647\\
29	0.011065951970892\\
30	0.0110659623063334\\
31	0.0110659728560528\\
32	0.0110659836242906\\
33	0.0110659946153656\\
34	0.0110660058336755\\
35	0.0110660172836989\\
36	0.0110660289699962\\
37	0.0110660408972108\\
38	0.0110660530700708\\
39	0.0110660654933902\\
40	0.01106607817207\\
41	0.0110660911111001\\
42	0.0110661043155601\\
43	0.0110661177906212\\
44	0.0110661315415474\\
45	0.011066145573697\\
46	0.011066159892524\\
47	0.0110661745035796\\
48	0.0110661894125136\\
49	0.0110662046250763\\
50	0.0110662201471196\\
51	0.0110662359845985\\
52	0.011066252143573\\
53	0.0110662686302097\\
54	0.0110662854507829\\
55	0.0110663026116766\\
56	0.0110663201193861\\
57	0.0110663379805196\\
58	0.0110663562017997\\
59	0.0110663747900654\\
60	0.0110663937522734\\
61	0.0110664130955003\\
62	0.0110664328269436\\
63	0.0110664529539244\\
64	0.0110664734838882\\
65	0.0110664944244074\\
66	0.0110665157831826\\
67	0.0110665375680447\\
68	0.0110665597869566\\
69	0.0110665824480152\\
70	0.011066605559453\\
71	0.0110666291296402\\
72	0.0110666531670863\\
73	0.0110666776804425\\
74	0.0110667026785031\\
75	0.0110667281702079\\
76	0.0110667541646437\\
77	0.0110667806710466\\
78	0.0110668076988039\\
79	0.0110668352574563\\
80	0.0110668633566995\\
81	0.0110668920063865\\
82	0.0110669212165298\\
83	0.0110669509973032\\
84	0.0110669813590442\\
85	0.0110670123122557\\
86	0.0110670438676085\\
87	0.0110670760359435\\
88	0.0110671088282735\\
89	0.0110671422557856\\
90	0.0110671763298434\\
91	0.0110672110619895\\
92	0.011067246463947\\
93	0.0110672825476228\\
94	0.0110673193251088\\
95	0.0110673568086852\\
96	0.0110673950108223\\
97	0.0110674339441826\\
98	0.011067473621624\\
99	0.0110675140562014\\
100	0.0110675552611697\\
101	0.0110675972499858\\
102	0.0110676400363113\\
103	0.0110676836340151\\
104	0.0110677280571758\\
105	0.0110677733200842\\
106	0.0110678194372461\\
107	0.0110678664233847\\
108	0.0110679142934433\\
109	0.0110679630625882\\
110	0.0110680127462113\\
111	0.0110680633599326\\
112	0.0110681149196034\\
113	0.0110681674413089\\
114	0.0110682209413712\\
115	0.0110682754363523\\
116	0.0110683309430567\\
117	0.0110683874785348\\
118	0.011068445060086\\
119	0.0110685037052616\\
120	0.0110685634318679\\
121	0.0110686242579697\\
122	0.0110686862018938\\
123	0.0110687492822317\\
124	0.0110688135178437\\
125	0.011068878927862\\
126	0.0110689455316946\\
127	0.0110690133490285\\
128	0.0110690823998342\\
129	0.0110691527043687\\
130	0.0110692242831803\\
131	0.0110692971571117\\
132	0.0110693713473049\\
133	0.0110694468752052\\
134	0.0110695237625654\\
135	0.0110696020314506\\
136	0.0110696817042425\\
137	0.0110697628036443\\
138	0.0110698453526855\\
139	0.0110699293747272\\
140	0.0110700148934668\\
141	0.0110701019329439\\
142	0.0110701905175452\\
143	0.0110702806720109\\
144	0.0110703724214399\\
145	0.0110704657912963\\
146	0.0110705608074156\\
147	0.0110706574960111\\
148	0.0110707558836807\\
149	0.0110708559974137\\
150	0.0110709578645983\\
151	0.0110710615130289\\
152	0.0110711669709136\\
153	0.0110712742668825\\
154	0.011071383429996\\
155	0.0110714944897532\\
156	0.011071607476101\\
157	0.0110717224194434\\
158	0.0110718393506507\\
159	0.01107195830107\\
160	0.0110720793025352\\
161	0.011072202387378\\
162	0.0110723275884385\\
163	0.0110724549390776\\
164	0.011072584473188\\
165	0.0110727162252075\\
166	0.0110728502301316\\
167	0.0110729865235267\\
168	0.0110731251415443\\
169	0.0110732661209355\\
170	0.0110734094990657\\
171	0.0110735553139302\\
172	0.0110737036041706\\
173	0.0110738544090913\\
174	0.0110740077686766\\
175	0.0110741637236094\\
176	0.0110743223152889\\
177	0.0110744835858506\\
178	0.011074647578186\\
179	0.0110748143359633\\
180	0.011074983903649\\
181	0.0110751563265298\\
182	0.0110753316507355\\
183	0.0110755099232633\\
184	0.0110756911920015\\
185	0.0110758755057553\\
186	0.011076062914273\\
187	0.0110762534682728\\
188	0.0110764472194705\\
189	0.0110766442206088\\
190	0.0110768445254862\\
191	0.0110770481889879\\
192	0.0110772552671174\\
193	0.0110774658170281\\
194	0.0110776798970573\\
195	0.0110778975667598\\
196	0.0110781188869433\\
197	0.0110783439197041\\
198	0.0110785727284644\\
199	0.0110788053780096\\
200	0.0110790419345273\\
201	0.0110792824656472\\
202	0.0110795270404811\\
203	0.0110797757296646\\
204	0.0110800286053994\\
205	0.0110802857414962\\
206	0.0110805472134187\\
207	0.0110808130983281\\
208	0.0110810834751289\\
209	0.0110813584245149\\
210	0.0110816380290157\\
211	0.0110819223730447\\
212	0.011082211542947\\
213	0.0110825056270477\\
214	0.0110828047157015\\
215	0.0110831089013422\\
216	0.0110834182785323\\
217	0.0110837329440138\\
218	0.0110840529967585\\
219	0.0110843785380187\\
220	0.0110847096713784\\
221	0.0110850465028041\\
222	0.011085389140696\\
223	0.0110857376959389\\
224	0.0110860922819529\\
225	0.0110864530147445\\
226	0.0110868200129565\\
227	0.011087193397919\\
228	0.0110875732936984\\
229	0.0110879598271473\\
230	0.0110883531279539\\
231	0.0110887533286901\\
232	0.0110891605648598\\
233	0.0110895749749473\\
234	0.0110899967004638\\
235	0.0110904258859951\\
236	0.0110908626792475\\
237	0.0110913072310951\\
238	0.0110917596956248\\
239	0.0110922202301832\\
240	0.0110926889954224\\
241	0.0110931661553461\\
242	0.0110936518773561\\
243	0.0110941463323\\
244	0.0110946496945182\\
245	0.0110951621418935\\
246	0.0110956838559008\\
247	0.0110962150216589\\
248	0.0110967558279836\\
249	0.0110973064674436\\
250	0.011097867136418\\
251	0.0110984380351571\\
252	0.0110990193678457\\
253	0.0110996113426702\\
254	0.0111002141718887\\
255	0.0111008280719052\\
256	0.0111014532633479\\
257	0.0111020899711511\\
258	0.0111027384246422\\
259	0.011103398857632\\
260	0.0111040715085093\\
261	0.0111047566203393\\
262	0.0111054544409684\\
263	0.0111061652231393\\
264	0.0111068892245963\\
265	0.011107626708201\\
266	0.0111083779420513\\
267	0.0111091431996037\\
268	0.0111099227597985\\
269	0.0111107169071889\\
270	0.0111115259320722\\
271	0.0111123501306243\\
272	0.0111131898050368\\
273	0.0111140452636559\\
274	0.0111149168211245\\
275	0.0111158047985245\\
276	0.0111167095235218\\
277	0.0111176313305117\\
278	0.0111185705607648\\
279	0.0111195275625731\\
280	0.0111205026913958\\
281	0.0111214963100045\\
282	0.0111225087886257\\
283	0.0111235405050835\\
284	0.0111245918449379\\
285	0.0111256632016214\\
286	0.0111267549765721\\
287	0.0111278675793624\\
288	0.0111290014278251\\
289	0.0111301569481742\\
290	0.0111313345751215\\
291	0.0111325347519904\\
292	0.0111337579308241\\
293	0.0111350045724916\\
294	0.0111362751467908\\
295	0.0111375701325493\\
296	0.0111388900177249\\
297	0.0111402352995058\\
298	0.0111416064844155\\
299	0.0111430040884249\\
300	0.0111444286370709\\
301	0.0111458806655778\\
302	0.0111473607189953\\
303	0.0111488693523518\\
304	0.0111504071308217\\
305	0.0111519746299042\\
306	0.0111535724356114\\
307	0.0111552011447466\\
308	0.0111568613652193\\
309	0.0111585537162774\\
310	0.0111602788288254\\
311	0.0111620373457564\\
312	0.0111638299222964\\
313	0.0111656572263608\\
314	0.0111675199389377\\
315	0.0111694187544753\\
316	0.0111713543812801\\
317	0.0111733275419233\\
318	0.0111753389736531\\
319	0.0111773894288088\\
320	0.0111794796752347\\
321	0.0111816104966875\\
322	0.0111837826932345\\
323	0.0111859970816334\\
324	0.0111882544956891\\
325	0.0111905557865738\\
326	0.0111929018231188\\
327	0.0111952934921429\\
328	0.0111977316986042\\
329	0.011200217365753\\
330	0.0112027514352256\\
331	0.0112053348670792\\
332	0.0112079686397508\\
333	0.011210653749863\\
334	0.0112133912119566\\
335	0.0112161820582775\\
336	0.0112190273384082\\
337	0.011221928118835\\
338	0.0112248854824623\\
339	0.0112279005280751\\
340	0.0112309743696732\\
341	0.0112341081354315\\
342	0.0112373029673959\\
343	0.0112405600222574\\
344	0.0112438804701057\\
345	0.0112472654948526\\
346	0.0112507162947592\\
347	0.0112542340830538\\
348	0.0112578200885759\\
349	0.011261475556421\\
350	0.0112652017497522\\
351	0.0112689999506558\\
352	0.0112728714612763\\
353	0.0112768176053963\\
354	0.011280839730262\\
355	0.0112849392070173\\
356	0.011289117431807\\
357	0.0112933758281439\\
358	0.0112977158483011\\
359	0.011302138974555\\
360	0.0113066467201962\\
361	0.0113112406302194\\
362	0.0113159222815943\\
363	0.0113206932830229\\
364	0.0113255552741194\\
365	0.0113305099239933\\
366	0.0113355589291006\\
367	0.0113407040083539\\
368	0.0113459468804434\\
369	0.0113512892819949\\
370	0.0113567329308793\\
371	0.0113622795031801\\
372	0.0113679306271417\\
373	0.0113736878855441\\
374	0.011379552794633\\
375	0.0113855267930921\\
376	0.011391611248355\\
377	0.0113978074525361\\
378	0.0114041166193487\\
379	0.0114105398822545\\
380	0.0114170782941147\\
381	0.0114237328286583\\
382	0.0114305043841853\\
383	0.01143739379016\\
384	0.0114444018176207\\
385	0.0114515291910225\\
386	0.0114587765444463\\
387	0.0114661444825072\\
388	0.0114736336782116\\
389	0.0114812448732057\\
390	0.011488978732885\\
391	0.0114968385746228\\
392	0.0115048276910688\\
393	0.0115129493353054\\
394	0.0115212066891198\\
395	0.0115296028262494\\
396	0.0115381406703072\\
397	0.0115468229476236\\
398	0.0115556521373787\\
399	0.0115646304311188\\
400	0.0115737597662731\\
401	0.011583041426709\\
402	0.011592475085277\\
403	0.0116020606864586\\
404	0.0116117969989971\\
405	0.0116216851604882\\
406	0.0116317236324177\\
407	0.0116419714326053\\
408	0.0116529415657599\\
409	0.011663871589142\\
410	0.0116747530579822\\
411	0.011685577103292\\
412	0.0116963344439236\\
413	0.0117070154585782\\
414	0.0117176098923509\\
415	0.0117281068040252\\
416	0.0117384951411801\\
417	0.0117487638955964\\
418	0.0117589025452438\\
419	0.0117688991696902\\
420	0.011778741639107\\
421	0.0117884176539012\\
422	0.0117979148089755\\
423	0.0118071312004235\\
424	0.0118157966930601\\
425	0.0118243385415335\\
426	0.0118327510316807\\
427	0.0118410288040422\\
428	0.0118491669519784\\
429	0.0118571605663134\\
430	0.0118650055916352\\
431	0.0118726985615377\\
432	0.0118802366168283\\
433	0.0118876179274911\\
434	0.0118948421415168\\
435	0.0119019091209185\\
436	0.0119088197806752\\
437	0.0119155761766031\\
438	0.0119219236782942\\
439	0.0119280894434835\\
440	0.011934231756636\\
441	0.0119403532393258\\
442	0.0119464568938057\\
443	0.0119525462418354\\
444	0.0119586256347657\\
445	0.0119646994594469\\
446	0.0119707725565623\\
447	0.0119768502119905\\
448	0.0119829381412316\\
449	0.0119890424683802\\
450	0.0119951696953853\\
451	0.0120013293922907\\
452	0.0120075406688709\\
453	0.0120138060099553\\
454	0.0120201279596771\\
455	0.0120265091121708\\
456	0.0120329520957963\\
457	0.0120394595555845\\
458	0.0120460341339326\\
459	0.0120526784496094\\
460	0.0120593950751\\
461	0.0120661865127087\\
462	0.0120730551695187\\
463	0.0120800033314504\\
464	0.0120870328683587\\
465	0.0120941451355333\\
466	0.0121013414602092\\
467	0.0121086231363118\\
468	0.0121159914193454\\
469	0.0121234475215211\\
470	0.0121309926072284\\
471	0.0121386277889605\\
472	0.0121463541238032\\
473	0.0121541726106076\\
474	0.0121620841879688\\
475	0.0121700897331545\\
476	0.0121781900675709\\
477	0.0121863859817356\\
478	0.0121946782333429\\
479	0.0122030675453886\\
480	0.0122115546043566\\
481	0.0122201400584669\\
482	0.0122288245159795\\
483	0.0122376085435454\\
484	0.0122464926645886\\
485	0.012255477357698\\
486	0.0122645630549977\\
487	0.012273750140458\\
488	0.0122830389475949\\
489	0.0122924297561361\\
490	0.0123019227884933\\
491	0.0123115182060202\\
492	0.0123212161050358\\
493	0.0123310165125896\\
494	0.0123409193819435\\
495	0.0123509245877424\\
496	0.0123610319208435\\
497	0.0123712410827732\\
498	0.012381551679777\\
499	0.012391963216426\\
500	0.0124024750887987\\
501	0.0124130865772172\\
502	0.0124237968385022\\
503	0.0124346048977139\\
504	0.0124455096393396\\
505	0.0124565097978854\\
506	0.0124676039478301\\
507	0.0124787904928908\\
508	0.0124900676545509\\
509	0.0125014334597928\\
510	0.0125128857279764\\
511	0.0125244220567965\\
512	0.0125360398072444\\
513	0.0125477360874936\\
514	0.0125595077356205\\
515	0.0125713513010625\\
516	0.0125832630247094\\
517	0.0125952388175095\\
518	0.0126072742374641\\
519	0.0126193644648704\\
520	0.0126315042756588\\
521	0.0126436880126554\\
522	0.0126559095545852\\
523	0.0126681622826116\\
524	0.0126804390441894\\
525	0.0126927321139865\\
526	0.0127050331516083\\
527	0.0127173331558066\\
528	0.012729622414824\\
529	0.012741890452518\\
530	0.0127541259712878\\
531	0.0127663167909335\\
532	0.0127784497822529\\
533	0.0127905107251183\\
534	0.0128024841908161\\
535	0.0128143534429356\\
536	0.0128261034464043\\
537	0.0128377199325822\\
538	0.0128497704092146\\
539	0.0128624028130973\\
540	0.0128746190719394\\
541	0.0128864437642305\\
542	0.0128969698322725\\
543	0.0129074152513112\\
544	0.0129179189471904\\
545	0.0129281262494209\\
546	0.0129381782122842\\
547	0.0129483921258864\\
548	0.0129587224354308\\
549	0.0129692334624215\\
550	0.0129800122113186\\
551	0.012991750213386\\
552	0.0130038236646002\\
553	0.0130179787643114\\
554	0.0130320825458009\\
555	0.0130461072474813\\
556	0.0130586243535662\\
557	0.0130712246843271\\
558	0.0130837088663739\\
559	0.0130954305800795\\
560	0.0131075388420725\\
561	0.0131192453219686\\
562	0.0131306904565711\\
563	0.0131418981867739\\
564	0.0131530211332214\\
565	0.0131641860434995\\
566	0.0131754046757112\\
567	0.0131866740841253\\
568	0.0131983203629706\\
569	0.0132097122331134\\
570	0.0132208481972837\\
571	0.0132317968651077\\
572	0.0132427656992653\\
573	0.0132537642725711\\
574	0.0132647838367736\\
575	0.0132758136396235\\
576	0.0132868417600207\\
577	0.0132978550775063\\
578	0.0133088391683302\\
579	0.0133197781864943\\
580	0.0133306547289956\\
581	0.0133414496845994\\
582	0.0133521420653068\\
583	0.0133627088190544\\
584	0.0133731246234883\\
585	0.0133833616664273\\
586	0.013393389433806\\
587	0.0134031745674478\\
588	0.0134126809661912\\
589	0.0134218705998152\\
590	0.0134307062886727\\
591	0.0134391597728698\\
592	0.0134472338645631\\
593	0.0134551345125145\\
594	0.0134637112677116\\
595	0.0134740407759367\\
596	0.0134889777516421\\
597	0.0135160557778789\\
598	0.0135751148335434\\
599	0\\
600	0\\
};
\addplot [color=black!20!mycolor21,solid,forget plot]
  table[row sep=crcr]{%
1	0.0110657835302299\\
2	0.0110657904019744\\
3	0.0110657974249927\\
4	0.0110658046025324\\
5	0.0110658119379086\\
6	0.0110658194345046\\
7	0.0110658270957733\\
8	0.011065834925239\\
9	0.0110658429264982\\
10	0.0110658511032214\\
11	0.0110658594591541\\
12	0.0110658679981189\\
13	0.0110658767240162\\
14	0.0110658856408263\\
15	0.0110658947526105\\
16	0.011065904063513\\
17	0.011065913577762\\
18	0.0110659232996719\\
19	0.0110659332336443\\
20	0.01106594338417\\
21	0.0110659537558307\\
22	0.0110659643533006\\
23	0.0110659751813478\\
24	0.0110659862448368\\
25	0.0110659975487296\\
26	0.0110660090980877\\
27	0.0110660208980741\\
28	0.011066032953955\\
29	0.0110660452711019\\
30	0.0110660578549929\\
31	0.0110660707112157\\
32	0.0110660838454685\\
33	0.0110660972635628\\
34	0.011066110971425\\
35	0.0110661249750985\\
36	0.0110661392807461\\
37	0.0110661538946518\\
38	0.0110661688232231\\
39	0.0110661840729933\\
40	0.0110661996506234\\
41	0.0110662155629046\\
42	0.0110662318167607\\
43	0.01106624841925\\
44	0.0110662653775681\\
45	0.0110662826990501\\
46	0.0110663003911726\\
47	0.0110663184615572\\
48	0.0110663369179716\\
49	0.0110663557683333\\
50	0.0110663750207115\\
51	0.0110663946833298\\
52	0.0110664147645689\\
53	0.0110664352729691\\
54	0.0110664562172331\\
55	0.0110664776062285\\
56	0.0110664994489909\\
57	0.0110665217547262\\
58	0.0110665445328139\\
59	0.0110665677928095\\
60	0.0110665915444474\\
61	0.0110666157976443\\
62	0.0110666405625014\\
63	0.0110666658493079\\
64	0.0110666916685437\\
65	0.0110667180308827\\
66	0.0110667449471953\\
67	0.0110667724285524\\
68	0.0110668004862274\\
69	0.0110668291317003\\
70	0.0110668583766603\\
71	0.0110668882330092\\
72	0.0110669187128647\\
73	0.0110669498285635\\
74	0.0110669815926646\\
75	0.0110670140179529\\
76	0.0110670471174419\\
77	0.0110670809043779\\
78	0.0110671153922428\\
79	0.0110671505947576\\
80	0.0110671865258861\\
81	0.011067223199838\\
82	0.0110672606310728\\
83	0.0110672988343031\\
84	0.011067337824498\\
85	0.0110673776168868\\
86	0.0110674182269626\\
87	0.0110674596704861\\
88	0.0110675019634885\\
89	0.0110675451222761\\
90	0.0110675891634331\\
91	0.0110676341038258\\
92	0.011067679960606\\
93	0.0110677267512149\\
94	0.0110677744933866\\
95	0.0110678232051519\\
96	0.0110678729048421\\
97	0.0110679236110925\\
98	0.0110679753428464\\
99	0.0110680281193587\\
100	0.0110680819601998\\
101	0.011068136885259\\
102	0.0110681929147488\\
103	0.011068250069208\\
104	0.0110683083695062\\
105	0.0110683678368469\\
106	0.0110684284927715\\
107	0.0110684903591632\\
108	0.0110685534582507\\
109	0.0110686178126115\\
110	0.0110686834451763\\
111	0.0110687503792321\\
112	0.0110688186384266\\
113	0.011068888246771\\
114	0.0110689592286446\\
115	0.0110690316087977\\
116	0.0110691054123557\\
117	0.0110691806648227\\
118	0.0110692573920848\\
119	0.0110693356204142\\
120	0.0110694153764723\\
121	0.0110694966873134\\
122	0.0110695795803883\\
123	0.0110696640835479\\
124	0.0110697502250465\\
125	0.0110698380335452\\
126	0.0110699275381154\\
127	0.0110700187682425\\
128	0.0110701117538289\\
129	0.0110702065251973\\
130	0.0110703031130946\\
131	0.0110704015486946\\
132	0.0110705018636015\\
133	0.0110706040898532\\
134	0.0110707082599248\\
135	0.0110708144067311\\
136	0.0110709225636307\\
137	0.0110710327644284\\
138	0.011071145043379\\
139	0.01107125943519\\
140	0.0110713759750252\\
141	0.0110714946985073\\
142	0.0110716156417217\\
143	0.0110717388412192\\
144	0.0110718643340192\\
145	0.0110719921576132\\
146	0.0110721223499679\\
147	0.0110722549495281\\
148	0.0110723899952204\\
149	0.0110725275264565\\
150	0.0110726675831363\\
151	0.0110728102056517\\
152	0.0110729554348896\\
153	0.0110731033122362\\
154	0.0110732538795801\\
155	0.0110734071793167\\
156	0.0110735632543514\\
157	0.0110737221481043\\
158	0.0110738839045144\\
159	0.0110740485680435\\
160	0.0110742161836816\\
161	0.0110743867969508\\
162	0.0110745604539115\\
163	0.0110747372011664\\
164	0.0110749170858674\\
165	0.0110751001557205\\
166	0.0110752864589926\\
167	0.0110754760445181\\
168	0.0110756689617054\\
169	0.011075865260545\\
170	0.0110760649916171\\
171	0.0110762682060999\\
172	0.0110764749557784\\
173	0.0110766852930544\\
174	0.0110768992709562\\
175	0.0110771169431494\\
176	0.0110773383639485\\
177	0.0110775635883289\\
178	0.0110777926719402\\
179	0.01107802567112\\
180	0.0110782626429084\\
181	0.0110785036450641\\
182	0.0110787487360809\\
183	0.0110789979752056\\
184	0.0110792514224572\\
185	0.0110795091386467\\
186	0.0110797711853988\\
187	0.0110800376251747\\
188	0.0110803085212961\\
189	0.0110805839379712\\
190	0.0110808639403217\\
191	0.0110811485944119\\
192	0.0110814379672791\\
193	0.0110817321269661\\
194	0.0110820311425552\\
195	0.011082335084205\\
196	0.0110826440231877\\
197	0.0110829580319303\\
198	0.0110832771840566\\
199	0.0110836015544319\\
200	0.0110839312192105\\
201	0.0110842662558848\\
202	0.0110846067433374\\
203	0.0110849527618957\\
204	0.011085304393389\\
205	0.0110856617212082\\
206	0.0110860248303686\\
207	0.011086393807575\\
208	0.0110867687412897\\
209	0.011087149721804\\
210	0.0110875368413115\\
211	0.0110879301939853\\
212	0.0110883298760574\\
213	0.0110887359859012\\
214	0.0110891486241169\\
215	0.0110895678936199\\
216	0.0110899938997312\\
217	0.0110904267502718\\
218	0.0110908665556579\\
219	0.0110913134290005\\
220	0.0110917674862056\\
221	0.0110922288460778\\
222	0.0110926976304249\\
223	0.0110931739641652\\
224	0.0110936579754354\\
225	0.0110941497957005\\
226	0.0110946495598638\\
227	0.0110951574063784\\
228	0.0110956734773583\\
229	0.0110961979186895\\
230	0.0110967308801408\\
231	0.011097272515473\\
232	0.0110978229825476\\
233	0.0110983824434326\\
234	0.0110989510645061\\
235	0.0110995290165575\\
236	0.0111001164748847\\
237	0.0111007136193872\\
238	0.0111013206346553\\
239	0.0111019377100537\\
240	0.0111025650397997\\
241	0.0111032028230365\\
242	0.0111038512638985\\
243	0.0111045105715719\\
244	0.0111051809603468\\
245	0.0111058626496639\\
246	0.0111065558641533\\
247	0.0111072608336665\\
248	0.0111079777933021\\
249	0.0111087069834253\\
250	0.0111094486496812\\
251	0.0111102030430039\\
252	0.0111109704196209\\
253	0.0111117510410548\\
254	0.0111125451741238\\
255	0.0111133530909416\\
256	0.0111141750689186\\
257	0.0111150113907673\\
258	0.0111158623445123\\
259	0.0111167282235074\\
260	0.0111176093264602\\
261	0.0111185059574632\\
262	0.0111194184260277\\
263	0.0111203470471425\\
264	0.0111212921414419\\
265	0.0111222540352322\\
266	0.0111232330606059\\
267	0.0111242295555637\\
268	0.0111252438641468\\
269	0.0111262763365798\\
270	0.0111273273294237\\
271	0.0111283972057422\\
272	0.0111294863352793\\
273	0.0111305950946505\\
274	0.0111317238675473\\
275	0.0111328730449559\\
276	0.0111340430253903\\
277	0.0111352342151391\\
278	0.0111364470285273\\
279	0.011137681888192\\
280	0.0111389392253713\\
281	0.011140219480206\\
282	0.0111415231020534\\
283	0.0111428505498099\\
284	0.0111442022922433\\
285	0.01114557880833\\
286	0.0111469805875953\\
287	0.011148408130454\\
288	0.0111498619485467\\
289	0.011151342565068\\
290	0.0111528505150822\\
291	0.0111543863458202\\
292	0.0111559506169531\\
293	0.0111575439008362\\
294	0.0111591667827166\\
295	0.0111608198608988\\
296	0.0111625037468601\\
297	0.0111642190653089\\
298	0.0111659664541756\\
299	0.0111677465645368\\
300	0.0111695600605198\\
301	0.0111714076191318\\
302	0.0111732899299379\\
303	0.0111752076947216\\
304	0.0111771616270852\\
305	0.0111791524519845\\
306	0.0111811809051314\\
307	0.0111832477320961\\
308	0.0111853536880729\\
309	0.0111874995379867\\
310	0.0111896860554923\\
311	0.0111919140229662\\
312	0.0111941842315383\\
313	0.0111964974811282\\
314	0.0111988545804945\\
315	0.0112012563475014\\
316	0.0112037036093745\\
317	0.0112061972030382\\
318	0.0112087379755431\\
319	0.011211326784593\\
320	0.0112139644991812\\
321	0.0112166520003437\\
322	0.0112193901820381\\
323	0.0112221799521535\\
324	0.0112250222336518\\
325	0.0112279179658189\\
326	0.0112308681055581\\
327	0.0112338736287127\\
328	0.0112369355326493\\
329	0.0112400548374137\\
330	0.0112432325875771\\
331	0.0112464698541679\\
332	0.0112497677367301\\
333	0.0112531273654906\\
334	0.0112565499025665\\
335	0.0112600365427887\\
336	0.0112635885159453\\
337	0.011267207087792\\
338	0.0112708935606567\\
339	0.0112746492735851\\
340	0.0112784756019556\\
341	0.0112823739558984\\
342	0.0112863457721734\\
343	0.0112903925107825\\
344	0.0112945156569243\\
345	0.0112987166843379\\
346	0.0113029970509625\\
347	0.0113073581944529\\
348	0.0113118015274859\\
349	0.0113163284321257\\
350	0.0113209402504951\\
351	0.0113256382897553\\
352	0.0113304238146228\\
353	0.0113352980412534\\
354	0.0113402621351728\\
355	0.0113453172148611\\
356	0.0113504643380515\\
357	0.0113557044934944\\
358	0.0113610386106566\\
359	0.0113664675606969\\
360	0.0113719921590708\\
361	0.011377613170022\\
362	0.011383331313227\\
363	0.011389147272878\\
364	0.0113950617095649\\
365	0.0114010752756061\\
366	0.0114071886354735\\
367	0.0114134024959216\\
368	0.0114197176532242\\
369	0.0114261347676039\\
370	0.0114326562650147\\
371	0.0114392851397778\\
372	0.0114460244396681\\
373	0.0114528772342998\\
374	0.0114598467309583\\
375	0.0114669360862382\\
376	0.011474148285399\\
377	0.0114814862406787\\
378	0.0114889527566934\\
379	0.0114965504907627\\
380	0.0115042819078056\\
381	0.0115121492295691\\
382	0.0115201543782327\\
383	0.0115282989152114\\
384	0.0115365839791378\\
385	0.0115450102407521\\
386	0.0115535779683069\\
387	0.0115622861432367\\
388	0.011571132674777\\
389	0.0115801153282833\\
390	0.0115897596702056\\
391	0.0115994060521122\\
392	0.0116090309193411\\
393	0.0116186274222215\\
394	0.0116281883371054\\
395	0.011637706061919\\
396	0.0116471726146568\\
397	0.0116565796355751\\
398	0.0116659183946249\\
399	0.0116751798098652\\
400	0.0116843545119192\\
401	0.0116934327385675\\
402	0.0117024039761608\\
403	0.0117112578591969\\
404	0.0117199835681743\\
405	0.0117285714708566\\
406	0.0117370109778638\\
407	0.0117452544057623\\
408	0.0117529836034152\\
409	0.0117606101620528\\
410	0.0117681286223245\\
411	0.0117755337136377\\
412	0.0117828203989786\\
413	0.0117899839464346\\
414	0.011797019990035\\
415	0.0118039246351518\\
416	0.0118106944551571\\
417	0.0118173265812444\\
418	0.0118238187820556\\
419	0.0118301694881771\\
420	0.0118363777579459\\
421	0.0118424436889591\\
422	0.011848368232014\\
423	0.011854088946802\\
424	0.011859430961938\\
425	0.0118647458904544\\
426	0.011870035512057\\
427	0.0118753019859474\\
428	0.0118805478854587\\
429	0.0118857759896692\\
430	0.0118909896248554\\
431	0.0118961925086493\\
432	0.0119013887135043\\
433	0.0119065828043419\\
434	0.0119117800054663\\
435	0.0119169855769894\\
436	0.011922205129509\\
437	0.0119274445946663\\
438	0.0119327195633107\\
439	0.0119380378591766\\
440	0.0119434017471379\\
441	0.0119488135523095\\
442	0.0119542756474879\\
443	0.0119597904378146\\
444	0.0119653603445436\\
445	0.0119709877939126\\
446	0.0119766751991778\\
447	0.0119824249413178\\
448	0.0119882393485565\\
449	0.0119941206748065\\
450	0.0120000710772602\\
451	0.0120060924897895\\
452	0.0120121862414342\\
453	0.0120183536461105\\
454	0.0120245959985104\\
455	0.0120309145699489\\
456	0.0120373106043922\\
457	0.0120437853147501\\
458	0.0120503398795261\\
459	0.0120569754399226\\
460	0.0120636930975101\\
461	0.0120704939125621\\
462	0.0120773789031652\\
463	0.0120843490452153\\
464	0.0120914052837965\\
465	0.012098548549483\\
466	0.0121057797573558\\
467	0.0121130998061113\\
468	0.012120509577265\\
469	0.012128009934452\\
470	0.0121356017228233\\
471	0.0121432857685314\\
472	0.0121510628782945\\
473	0.0121589338390237\\
474	0.0121668994174921\\
475	0.0121749603600154\\
476	0.0121831173918932\\
477	0.0121913712158188\\
478	0.0121997225101997\\
479	0.0122081719273813\\
480	0.012216720091759\\
481	0.0122253675977668\\
482	0.0122341150077287\\
483	0.0122429628495551\\
484	0.0122519116142696\\
485	0.0122609617533462\\
486	0.0122701136758401\\
487	0.0122793677452894\\
488	0.0122887242763912\\
489	0.0122981835314651\\
490	0.012307745716693\\
491	0.0123174109781171\\
492	0.0123271793973813\\
493	0.0123370509871967\\
494	0.0123470256865151\\
495	0.0123571033553883\\
496	0.0123672837694935\\
497	0.0123775666143034\\
498	0.0123879514788759\\
499	0.0123984378492395\\
500	0.0124090251013459\\
501	0.0124197124935579\\
502	0.0124304991586391\\
503	0.0124413840952073\\
504	0.0124523661586138\\
505	0.0124634440512024\\
506	0.0124746163119035\\
507	0.012485881305109\\
508	0.0124972372087743\\
509	0.0125086820016835\\
510	0.0125202134498133\\
511	0.0125318290917197\\
512	0.0125435262228691\\
513	0.0125553018788246\\
514	0.0125671528171925\\
515	0.012579075498223\\
516	0.0125910660639501\\
517	0.0126031203157439\\
518	0.0126152336901377\\
519	0.0126274012327765\\
520	0.012639617570322\\
521	0.0126518768801308\\
522	0.0126641728575067\\
523	0.0126764986803076\\
524	0.01268884697067\\
525	0.0127012097535404\\
526	0.0127135784117239\\
527	0.0127259436381541\\
528	0.0127382953853961\\
529	0.0127506228108894\\
530	0.0127629141524267\\
531	0.0127751566183118\\
532	0.0127873363026626\\
533	0.0127994416148953\\
534	0.0128114616770597\\
535	0.0128242706075797\\
536	0.012837380131529\\
537	0.0128501705703827\\
538	0.0128622857200498\\
539	0.0128734910425667\\
540	0.0128842944965308\\
541	0.0128945473872983\\
542	0.0129039241484465\\
543	0.0129134592877277\\
544	0.0129231240488488\\
545	0.0129329412547109\\
546	0.0129429458595549\\
547	0.0129531340968814\\
548	0.0129638050212203\\
549	0.012975202483788\\
550	0.0129892054093718\\
551	0.013002956613446\\
552	0.0130167974146248\\
553	0.0130290441704488\\
554	0.0130412448851652\\
555	0.0130532261696414\\
556	0.0130642989858724\\
557	0.0130755113207519\\
558	0.013087286143706\\
559	0.0130988443324463\\
560	0.0131100721444647\\
561	0.0131210855541065\\
562	0.0131320001699415\\
563	0.013142969346439\\
564	0.0131540054757782\\
565	0.0131651052578191\\
566	0.0131763235313981\\
567	0.0131878715652356\\
568	0.0131990279449963\\
569	0.0132099881684336\\
570	0.0132208706968361\\
571	0.013231798444302\\
572	0.0132427659177417\\
573	0.0132537643358688\\
574	0.0132647838624601\\
575	0.0132758136515379\\
576	0.0132868417664646\\
577	0.0132978550811885\\
578	0.0133088391704532\\
579	0.0133197781877949\\
580	0.0133306547298284\\
581	0.0133414496851007\\
582	0.0133521420655792\\
583	0.013362708819187\\
584	0.0133731246235443\\
585	0.0133833616664418\\
586	0.013393389433806\\
587	0.0134031745674478\\
588	0.0134126809661912\\
589	0.0134218705998152\\
590	0.0134307062886727\\
591	0.0134391597728698\\
592	0.0134472338645631\\
593	0.0134551345125145\\
594	0.0134637112677116\\
595	0.0134740407759367\\
596	0.0134889777516421\\
597	0.0135160557778789\\
598	0.0135751148335434\\
599	0\\
600	0\\
};
\addplot [color=black!50!mycolor20,solid,forget plot]
  table[row sep=crcr]{%
1	0.011065821102699\\
2	0.0110658289648861\\
3	0.0110658370057177\\
4	0.0110658452291939\\
5	0.0110658536394024\\
6	0.0110658622405204\\
7	0.0110658710368164\\
8	0.011065880032652\\
9	0.0110658892324841\\
10	0.0110658986408668\\
11	0.0110659082624533\\
12	0.0110659181019983\\
13	0.0110659281643598\\
14	0.0110659384545014\\
15	0.0110659489774944\\
16	0.0110659597385203\\
17	0.0110659707428727\\
18	0.01106598199596\\
19	0.0110659935033077\\
20	0.0110660052705604\\
21	0.0110660173034849\\
22	0.0110660296079724\\
23	0.0110660421900408\\
24	0.011066055055838\\
25	0.0110660682116438\\
26	0.0110660816638731\\
27	0.0110660954190784\\
28	0.011066109483953\\
29	0.0110661238653332\\
30	0.0110661385702018\\
31	0.0110661536056909\\
32	0.0110661689790847\\
33	0.011066184697823\\
34	0.0110662007695039\\
35	0.0110662172018872\\
36	0.0110662340028976\\
37	0.0110662511806283\\
38	0.0110662687433437\\
39	0.0110662866994835\\
40	0.0110663050576658\\
41	0.0110663238266906\\
42	0.0110663430155439\\
43	0.0110663626334008\\
44	0.0110663826896293\\
45	0.0110664031937946\\
46	0.0110664241556623\\
47	0.0110664455852029\\
48	0.0110664674925953\\
49	0.0110664898882313\\
50	0.0110665127827193\\
51	0.011066536186889\\
52	0.011066560111795\\
53	0.011066584568722\\
54	0.0110666095691883\\
55	0.011066635124951\\
56	0.01106666124801\\
57	0.0110666879506131\\
58	0.0110667152452604\\
59	0.0110667431447091\\
60	0.0110667716619784\\
61	0.0110668008103544\\
62	0.0110668306033951\\
63	0.0110668610549354\\
64	0.0110668921790924\\
65	0.0110669239902705\\
66	0.0110669565031665\\
67	0.0110669897327755\\
68	0.0110670236943958\\
69	0.0110670584036347\\
70	0.0110670938764143\\
71	0.0110671301289765\\
72	0.0110671671778895\\
73	0.0110672050400534\\
74	0.0110672437327058\\
75	0.0110672832734283\\
76	0.0110673236801521\\
77	0.0110673649711645\\
78	0.0110674071651149\\
79	0.0110674502810212\\
80	0.0110674943382764\\
81	0.0110675393566545\\
82	0.0110675853563174\\
83	0.0110676323578216\\
84	0.0110676803821247\\
85	0.0110677294505921\\
86	0.011067779585004\\
87	0.0110678308075621\\
88	0.0110678831408967\\
89	0.0110679366080738\\
90	0.0110679912326019\\
91	0.0110680470384397\\
92	0.0110681040500027\\
93	0.0110681622921709\\
94	0.011068221790296\\
95	0.0110682825702092\\
96	0.0110683446582278\\
97	0.0110684080811638\\
98	0.0110684728663307\\
99	0.0110685390415514\\
100	0.011068606635166\\
101	0.0110686756760393\\
102	0.0110687461935687\\
103	0.011068818217692\\
104	0.0110688917788952\\
105	0.0110689669082204\\
106	0.0110690436372737\\
107	0.0110691219982331\\
108	0.0110692020238565\\
109	0.0110692837474896\\
110	0.0110693672030739\\
111	0.0110694524251547\\
112	0.011069539448889\\
113	0.0110696283100536\\
114	0.011069719045053\\
115	0.0110698116909273\\
116	0.0110699062853603\\
117	0.011070002866687\\
118	0.0110701014739023\\
119	0.0110702021466677\\
120	0.01107030492532\\
121	0.0110704098508787\\
122	0.0110705169650537\\
123	0.0110706263102526\\
124	0.0110707379295889\\
125	0.0110708518668887\\
126	0.0110709681666985\\
127	0.0110710868742924\\
128	0.0110712080356789\\
129	0.0110713316976082\\
130	0.0110714579075792\\
131	0.0110715867138459\\
132	0.0110717181654239\\
133	0.0110718523120973\\
134	0.0110719892044245\\
135	0.0110721288937444\\
136	0.0110722714321824\\
137	0.0110724168726557\\
138	0.0110725652688791\\
139	0.0110727166753699\\
140	0.0110728711474532\\
141	0.0110730287412661\\
142	0.0110731895137625\\
143	0.0110733535227174\\
144	0.01107352082673\\
145	0.0110736914852282\\
146	0.011073865558471\\
147	0.0110740431075521\\
148	0.0110742241944016\\
149	0.0110744088817891\\
150	0.0110745972333247\\
151	0.0110747893134609\\
152	0.0110749851874935\\
153	0.0110751849215619\\
154	0.0110753885826497\\
155	0.0110755962385844\\
156	0.0110758079580363\\
157	0.0110760238105175\\
158	0.0110762438663807\\
159	0.0110764681968163\\
160	0.0110766968738505\\
161	0.0110769299703415\\
162	0.0110771675599767\\
163	0.0110774097172679\\
164	0.0110776565175468\\
165	0.0110779080369602\\
166	0.011078164352464\\
167	0.0110784255418171\\
168	0.0110786916835746\\
169	0.011078962857081\\
170	0.0110792391424619\\
171	0.0110795206206163\\
172	0.0110798073732074\\
173	0.0110800994826542\\
174	0.0110803970321212\\
175	0.0110807001055086\\
176	0.0110810087874421\\
177	0.0110813231632619\\
178	0.0110816433190119\\
179	0.0110819693414277\\
180	0.0110823013179257\\
181	0.0110826393365908\\
182	0.0110829834861647\\
183	0.0110833338560337\\
184	0.0110836905362168\\
185	0.0110840536173539\\
186	0.0110844231906939\\
187	0.0110847993480832\\
188	0.0110851821819549\\
189	0.0110855717853179\\
190	0.0110859682517471\\
191	0.0110863716753741\\
192	0.011086782150879\\
193	0.0110871997734831\\
194	0.0110876246389431\\
195	0.0110880568435463\\
196	0.0110884964841078\\
197	0.0110889436579694\\
198	0.0110893984630008\\
199	0.0110898609976025\\
200	0.0110903313607123\\
201	0.0110908096518137\\
202	0.011091295970948\\
203	0.0110917904187298\\
204	0.0110922930963662\\
205	0.0110928041056802\\
206	0.0110933235491388\\
207	0.0110938515298859\\
208	0.0110943881517803\\
209	0.0110949335194399\\
210	0.0110954877382915\\
211	0.0110960509146281\\
212	0.0110966231556722\\
213	0.0110972045696477\\
214	0.0110977952658588\\
215	0.0110983953547788\\
216	0.0110990049481464\\
217	0.011099624159073\\
218	0.011100253102159\\
219	0.0111008918936211\\
220	0.0111015406514306\\
221	0.0111021994954631\\
222	0.0111028685476601\\
223	0.0111035479322029\\
224	0.0111042377756991\\
225	0.0111049382073823\\
226	0.0111056493593247\\
227	0.0111063713666632\\
228	0.0111071043678387\\
229	0.0111078485048495\\
230	0.011108603923517\\
231	0.0111093707737648\\
232	0.0111101492099112\\
233	0.0111109393909719\\
234	0.0111117414809754\\
235	0.0111125556492884\\
236	0.0111133820709493\\
237	0.0111142209270105\\
238	0.0111150724048858\\
239	0.0111159366987018\\
240	0.0111168140096519\\
241	0.0111177045463478\\
242	0.0111186085251691\\
243	0.011119526170605\\
244	0.011120457715586\\
245	0.0111214034018017\\
246	0.0111223634800016\\
247	0.0111233382102719\\
248	0.0111243278622881\\
249	0.0111253327155356\\
250	0.0111263530594949\\
251	0.0111273891937881\\
252	0.0111284414282803\\
253	0.0111295100831344\\
254	0.0111305954888137\\
255	0.0111316979860313\\
256	0.0111328179256436\\
257	0.0111339556684872\\
258	0.0111351115851612\\
259	0.0111362860557569\\
260	0.0111374794695377\\
261	0.0111386922245708\\
262	0.011139924727283\\
263	0.0111411773918651\\
264	0.0111424506395847\\
265	0.0111437448994041\\
266	0.0111450606068011\\
267	0.0111463982036485\\
268	0.0111477581380904\\
269	0.0111491408644129\\
270	0.0111505468429148\\
271	0.0111519765397796\\
272	0.0111534304269509\\
273	0.0111549089820174\\
274	0.0111564126881083\\
275	0.0111579420338064\\
276	0.0111594975130823\\
277	0.0111610796252556\\
278	0.0111626888749901\\
279	0.0111643257723283\\
280	0.0111659908327733\\
281	0.0111676845774264\\
282	0.0111694075331867\\
283	0.0111711602330231\\
284	0.0111729432163271\\
285	0.011174757029354\\
286	0.0111766022257638\\
287	0.0111784793672676\\
288	0.0111803890243907\\
289	0.0111823317773566\\
290	0.0111843082171011\\
291	0.0111863189464177\\
292	0.0111883645812396\\
293	0.0111904457520563\\
294	0.0111925631054612\\
295	0.011194717305824\\
296	0.0111969090370743\\
297	0.0111991390045748\\
298	0.0112014079370436\\
299	0.0112037165884391\\
300	0.0112060657397183\\
301	0.0112084562011035\\
302	0.0112108888143723\\
303	0.0112133644539645\\
304	0.0112158840284228\\
305	0.0112184484815796\\
306	0.0112210587934267\\
307	0.0112237159800599\\
308	0.0112264210887\\
309	0.0112291751987576\\
310	0.0112319794208268\\
311	0.011234834875442\\
312	0.011237742691343\\
313	0.0112407040036286\\
314	0.0112437199513603\\
315	0.0112467916741458\\
316	0.011249920310454\\
317	0.0112531069948132\\
318	0.0112563528548289\\
319	0.0112596590080321\\
320	0.0112630265585776\\
321	0.0112664565938174\\
322	0.0112699501807839\\
323	0.0112735083626306\\
324	0.0112771321550935\\
325	0.0112808225430402\\
326	0.0112845804770656\\
327	0.0112884068694165\\
328	0.0112923025866838\\
329	0.0112962684592011\\
330	0.0113003052724157\\
331	0.0113044137657668\\
332	0.0113085946326452\\
333	0.0113128485225369\\
334	0.0113171760480994\\
335	0.0113215777843965\\
336	0.0113260542637323\\
337	0.0113306059934804\\
338	0.0113352334659643\\
339	0.0113399371709906\\
340	0.0113447176118106\\
341	0.0113495753265301\\
342	0.0113545109191944\\
343	0.011359525017863\\
344	0.0113646187507557\\
345	0.0113697944840235\\
346	0.0113750546600878\\
347	0.011380401794997\\
348	0.0113858384751972\\
349	0.011391367354125\\
350	0.0113969911446045\\
351	0.0114027125351027\\
352	0.0114085342971328\\
353	0.0114144592426763\\
354	0.0114204901818168\\
355	0.0114266299085963\\
356	0.011432881283791\\
357	0.0114392470923177\\
358	0.0114457299089051\\
359	0.0114523321789638\\
360	0.0114590561790719\\
361	0.0114659039726035\\
362	0.0114728773604104\\
363	0.0114799778266626\\
364	0.0114872064802676\\
365	0.0114945639930024\\
366	0.0115020505375361\\
367	0.0115096657364463\\
368	0.011517408675879\\
369	0.0115256221280624\\
370	0.0115339508772211\\
371	0.0115422821613925\\
372	0.0115506110873121\\
373	0.0115589326462482\\
374	0.0115672412416417\\
375	0.011575532689852\\
376	0.011583802533005\\
377	0.0115920445535461\\
378	0.011600252206247\\
379	0.0116084186177617\\
380	0.0116165365895334\\
381	0.0116245986049176\\
382	0.0116325968304002\\
383	0.0116405231382912\\
384	0.011648369128459\\
385	0.0116561261640855\\
386	0.0116637854724604\\
387	0.0116713378327485\\
388	0.0116787737305949\\
389	0.011686083760333\\
390	0.0116929466123258\\
391	0.0116997204530993\\
392	0.0117064106890828\\
393	0.0117130123777576\\
394	0.0117195206850028\\
395	0.0117259309243712\\
396	0.0117322386001816\\
397	0.0117384394544537\\
398	0.0117445295176025\\
399	0.0117505051625792\\
400	0.0117563631614718\\
401	0.0117621007505314\\
402	0.0117677157086127\\
403	0.0117732064075792\\
404	0.0117785718810946\\
405	0.0117838118399652\\
406	0.0117889267467188\\
407	0.0117938913382587\\
408	0.0117984888910142\\
409	0.0118030574223082\\
410	0.011807598025411\\
411	0.0118121120935789\\
412	0.0118166013245636\\
413	0.0118210676024401\\
414	0.0118255132532196\\
415	0.0118299408703243\\
416	0.0118343539043087\\
417	0.0118387557791577\\
418	0.0118431502491407\\
419	0.011847541390846\\
420	0.0118519335374563\\
421	0.0118563314129836\\
422	0.0118607399947649\\
423	0.0118651669656371\\
424	0.0118696268407276\\
425	0.0118741215359956\\
426	0.0118786530357077\\
427	0.0118832233840815\\
428	0.011887834675857\\
429	0.0118924890462112\\
430	0.0118971886563122\\
431	0.0119019356793648\\
432	0.0119067322850935\\
433	0.0119115806219581\\
434	0.0119164827994091\\
435	0.0119214408741156\\
436	0.011926456831081\\
437	0.0119315325641222\\
438	0.011936669504429\\
439	0.0119418688790101\\
440	0.0119471319053326\\
441	0.0119524597878467\\
442	0.0119578537146328\\
443	0.0119633148542813\\
444	0.0119688443530573\\
445	0.0119744433322059\\
446	0.0119801128856566\\
447	0.0119858540782158\\
448	0.0119916679443386\\
449	0.0119975554875762\\
450	0.0120035176807954\\
451	0.0120095554712216\\
452	0.0120156698005237\\
453	0.0120218616042862\\
454	0.0120281318115697\\
455	0.0120344813445731\\
456	0.0120409111184003\\
457	0.0120474220409336\\
458	0.012054015012811\\
459	0.0120606909275037\\
460	0.012067450671483\\
461	0.0120742951244625\\
462	0.0120812251596977\\
463	0.0120882416443153\\
464	0.0120953454392403\\
465	0.0121025373984851\\
466	0.0121098183683999\\
467	0.0121171891868797\\
468	0.0121246506825181\\
469	0.0121322036737005\\
470	0.0121398489676272\\
471	0.0121475873592562\\
472	0.0121554196301551\\
473	0.0121633465472496\\
474	0.0121713688614584\\
475	0.0121794873061998\\
476	0.0121877025957688\\
477	0.0121960154236081\\
478	0.0122044264604697\\
479	0.0122129363524563\\
480	0.0122215457189367\\
481	0.0122302551503262\\
482	0.0122390652057242\\
483	0.0122479764103993\\
484	0.0122569892531134\\
485	0.0122661041832746\\
486	0.0122753216079108\\
487	0.0122846418884514\\
488	0.0122940653373078\\
489	0.0123035922142385\\
490	0.0123132227224855\\
491	0.012322957004665\\
492	0.0123327951383984\\
493	0.0123427371316635\\
494	0.0123527829178492\\
495	0.0123629323504907\\
496	0.0123731851976653\\
497	0.012383541136023\\
498	0.0123939997444267\\
499	0.0124045604971727\\
500	0.0124152227567621\\
501	0.0124259857661877\\
502	0.0124368486407018\\
503	0.0124478103590238\\
504	0.012458869753945\\
505	0.0124700255022841\\
506	0.0124812761141412\\
507	0.012492619921394\\
508	0.012504055065377\\
509	0.0125155794836737\\
510	0.0125271908959512\\
511	0.0125388867887563\\
512	0.0125506643991852\\
513	0.0125625206973315\\
514	0.0125744523674068\\
515	0.0125864557874208\\
516	0.0125985270072938\\
517	0.0126106617252646\\
518	0.0126228552624458\\
519	0.0126351025353607\\
520	0.0126473980262867\\
521	0.0126597357512105\\
522	0.012672109225188\\
523	0.0126845114248838\\
524	0.0126969347480496\\
525	0.0127093709717494\\
526	0.0127218112073998\\
527	0.012734245805138\\
528	0.0127466642371578\\
529	0.0127590550256304\\
530	0.0127714089380378\\
531	0.0127837179472196\\
532	0.0127969694239702\\
533	0.0128104550459459\\
534	0.0128236937604145\\
535	0.0128361030880106\\
536	0.0128478459615153\\
537	0.0128592172165552\\
538	0.0128699146974688\\
539	0.0128798914517335\\
540	0.0128895434255866\\
541	0.0128988013504321\\
542	0.0129081214668756\\
543	0.0129176240718101\\
544	0.0129273120545926\\
545	0.0129371900133201\\
546	0.0129476699430569\\
547	0.0129606047228869\\
548	0.0129743833009077\\
549	0.0129880655941136\\
550	0.0130002935798855\\
551	0.0130121056813867\\
552	0.0130237468212249\\
553	0.0130344063417142\\
554	0.0130451432075004\\
555	0.0130558999472609\\
556	0.0130672064350691\\
557	0.0130786964131599\\
558	0.0130897324850921\\
559	0.0131006000416406\\
560	0.013111303177285\\
561	0.0131220669732257\\
562	0.013132907115845\\
563	0.0131438212924558\\
564	0.0131548047831771\\
565	0.0131659765609794\\
566	0.0131773567450786\\
567	0.0131883678346686\\
568	0.0131991666585108\\
569	0.013209991322539\\
570	0.0132208709170359\\
571	0.0132317984756346\\
572	0.0132427659270061\\
573	0.0132537643396685\\
574	0.0132647838642445\\
575	0.0132758136524997\\
576	0.0132868417670135\\
577	0.0132978550815073\\
578	0.0133088391706484\\
579	0.0133197781879177\\
580	0.0133306547299011\\
581	0.0133414496851396\\
582	0.0133521420655977\\
583	0.0133627088191944\\
584	0.0133731246235461\\
585	0.0133833616664418\\
586	0.013393389433806\\
587	0.0134031745674478\\
588	0.0134126809661912\\
589	0.0134218705998152\\
590	0.0134307062886727\\
591	0.0134391597728698\\
592	0.0134472338645631\\
593	0.0134551345125145\\
594	0.0134637112677116\\
595	0.0134740407759367\\
596	0.0134889777516421\\
597	0.0135160557778789\\
598	0.0135751148335434\\
599	0\\
600	0\\
};
\addplot [color=black!60!mycolor21,solid,forget plot]
  table[row sep=crcr]{%
1	0.011065848836443\\
2	0.0110658574824421\\
3	0.0110658663297064\\
4	0.0110658753828921\\
5	0.0110658846467622\\
6	0.0110658941261889\\
7	0.0110659038261558\\
8	0.0110659137517605\\
9	0.0110659239082174\\
10	0.01106593430086\\
11	0.011065944935144\\
12	0.0110659558166494\\
13	0.011065966951084\\
14	0.0110659783442858\\
15	0.0110659900022261\\
16	0.0110660019310124\\
17	0.0110660141368915\\
18	0.0110660266262526\\
19	0.0110660394056307\\
20	0.0110660524817093\\
21	0.0110660658613245\\
22	0.0110660795514675\\
23	0.011066093559289\\
24	0.0110661078921019\\
25	0.0110661225573856\\
26	0.0110661375627892\\
27	0.0110661529161355\\
28	0.011066168625425\\
29	0.0110661846988395\\
30	0.0110662011447463\\
31	0.0110662179717024\\
32	0.0110662351884584\\
33	0.0110662528039632\\
34	0.0110662708273681\\
35	0.0110662892680311\\
36	0.011066308135522\\
37	0.0110663274396265\\
38	0.0110663471903513\\
39	0.0110663673979288\\
40	0.011066388072822\\
41	0.0110664092257299\\
42	0.0110664308675922\\
43	0.0110664530095949\\
44	0.0110664756631754\\
45	0.0110664988400283\\
46	0.0110665225521108\\
47	0.0110665468116486\\
48	0.0110665716311415\\
49	0.0110665970233694\\
50	0.0110666230013985\\
51	0.0110666495785877\\
52	0.0110666767685942\\
53	0.0110667045853807\\
54	0.0110667330432219\\
55	0.0110667621567105\\
56	0.0110667919407652\\
57	0.0110668224106365\\
58	0.0110668535819148\\
59	0.0110668854705371\\
60	0.0110669180927945\\
61	0.0110669514653401\\
62	0.0110669856051962\\
63	0.0110670205297625\\
64	0.011067056256824\\
65	0.0110670928045592\\
66	0.0110671301915484\\
67	0.0110671684367822\\
68	0.0110672075596701\\
69	0.0110672475800496\\
70	0.0110672885181946\\
71	0.0110673303948251\\
72	0.0110673732311163\\
73	0.0110674170487081\\
74	0.0110674618697148\\
75	0.0110675077167351\\
76	0.0110675546128617\\
77	0.011067602581692\\
78	0.0110676516473385\\
79	0.0110677018344388\\
80	0.0110677531681673\\
81	0.0110678056742454\\
82	0.011067859378953\\
83	0.0110679143091399\\
84	0.0110679704922375\\
85	0.0110680279562702\\
86	0.0110680867298677\\
87	0.011068146842277\\
88	0.011068208323375\\
89	0.0110682712036808\\
90	0.011068335514369\\
91	0.0110684012872821\\
92	0.0110684685549444\\
93	0.0110685373505748\\
94	0.011068607708101\\
95	0.0110686796621732\\
96	0.0110687532481782\\
97	0.0110688285022536\\
98	0.0110689054613028\\
99	0.0110689841630091\\
100	0.0110690646458513\\
101	0.0110691469491186\\
102	0.0110692311129261\\
103	0.0110693171782304\\
104	0.0110694051868458\\
105	0.0110694951814599\\
106	0.0110695872056503\\
107	0.0110696813039009\\
108	0.0110697775216188\\
109	0.0110698759051512\\
110	0.0110699765018028\\
111	0.0110700793598526\\
112	0.0110701845285724\\
113	0.0110702920582436\\
114	0.0110704020001761\\
115	0.0110705144067258\\
116	0.0110706293313135\\
117	0.0110707468284432\\
118	0.011070866953721\\
119	0.0110709897638737\\
120	0.0110711153167686\\
121	0.011071243671432\\
122	0.0110713748880691\\
123	0.0110715090280834\\
124	0.0110716461540966\\
125	0.011071786329968\\
126	0.0110719296208149\\
127	0.0110720760930324\\
128	0.0110722258143138\\
129	0.0110723788536705\\
130	0.0110725352814524\\
131	0.0110726951693684\\
132	0.0110728585905068\\
133	0.0110730256193553\\
134	0.0110731963318221\\
135	0.0110733708052554\\
136	0.0110735491184646\\
137	0.0110737313517402\\
138	0.0110739175868737\\
139	0.0110741079071785\\
140	0.0110743023975091\\
141	0.0110745011442813\\
142	0.0110747042354916\\
143	0.0110749117607369\\
144	0.0110751238112333\\
145	0.0110753404798355\\
146	0.0110755618610546\\
147	0.0110757880510769\\
148	0.0110760191477815\\
149	0.0110762552507575\\
150	0.0110764964613212\\
151	0.0110767428825322\\
152	0.0110769946192093\\
153	0.0110772517779456\\
154	0.0110775144671229\\
155	0.0110777827969256\\
156	0.0110780568793536\\
157	0.0110783368282344\\
158	0.0110786227592342\\
159	0.0110789147898681\\
160	0.0110792130395096\\
161	0.0110795176293982\\
162	0.0110798286826463\\
163	0.0110801463242451\\
164	0.0110804706810685\\
165	0.0110808018818759\\
166	0.0110811400573137\\
167	0.0110814853399147\\
168	0.0110818378640958\\
169	0.0110821977661549\\
170	0.0110825651842643\\
171	0.0110829402584631\\
172	0.0110833231306477\\
173	0.0110837139445592\\
174	0.011084112845769\\
175	0.011084519981662\\
176	0.0110849355014169\\
177	0.0110853595559841\\
178	0.0110857922980608\\
179	0.0110862338820627\\
180	0.0110866844640931\\
181	0.0110871442019089\\
182	0.0110876132548825\\
183	0.0110880917839612\\
184	0.0110885799516224\\
185	0.0110890779218253\\
186	0.0110895858599585\\
187	0.0110901039327842\\
188	0.0110906323083774\\
189	0.0110911711560619\\
190	0.0110917206463413\\
191	0.0110922809508256\\
192	0.0110928522421534\\
193	0.0110934346939098\\
194	0.0110940284805389\\
195	0.0110946337772519\\
196	0.0110952507599306\\
197	0.0110958796050258\\
198	0.0110965204894504\\
199	0.0110971735904689\\
200	0.0110978390855803\\
201	0.0110985171523983\\
202	0.0110992079685247\\
203	0.0110999117114205\\
204	0.0111006285582705\\
205	0.0111013586858455\\
206	0.0111021022703602\\
207	0.0111028594873266\\
208	0.0111036305114064\\
209	0.0111044155162583\\
210	0.0111052146743857\\
211	0.011106028156981\\
212	0.0111068561337703\\
213	0.0111076987728578\\
214	0.0111085562405715\\
215	0.0111094287013109\\
216	0.0111103163173976\\
217	0.0111112192489312\\
218	0.0111121376536508\\
219	0.0111130716868039\\
220	0.0111140215010256\\
221	0.0111149872462281\\
222	0.0111159690695046\\
223	0.0111169671150486\\
224	0.0111179815240917\\
225	0.0111190124348626\\
226	0.0111200599825698\\
227	0.0111211242994125\\
228	0.0111222055146206\\
229	0.0111233037545307\\
230	0.0111244191426985\\
231	0.0111255518000549\\
232	0.011126701845107\\
233	0.0111278693941904\\
234	0.0111290545617769\\
235	0.0111302574608417\\
236	0.011131478203295\\
237	0.011132716900483\\
238	0.0111339736637626\\
239	0.0111352486051542\\
240	0.011136541838077\\
241	0.0111378534781709\\
242	0.0111391836442071\\
243	0.0111405324590924\\
244	0.0111419000509658\\
245	0.011143286554392\\
246	0.0111446921116473\\
247	0.0111461168741002\\
248	0.0111475610036785\\
249	0.0111490246744212\\
250	0.0111505080741037\\
251	0.011152011405928\\
252	0.0111535348902619\\
253	0.0111550787664114\\
254	0.0111566432944048\\
255	0.0111582287567642\\
256	0.0111598354602357\\
257	0.0111614637374467\\
258	0.0111631139484544\\
259	0.0111647864821524\\
260	0.0111664817574989\\
261	0.011168200224542\\
262	0.0111699423651951\\
263	0.0111717086935359\\
264	0.0111734997544875\\
265	0.0111753161187042\\
266	0.0111771583941733\\
267	0.0111790272044009\\
268	0.0111809231881949\\
269	0.0111828469993635\\
270	0.0111847993063248\\
271	0.0111867807916173\\
272	0.0111887921513036\\
273	0.011190834094261\\
274	0.0111929073413485\\
275	0.0111950126244442\\
276	0.0111971506853451\\
277	0.0111993222745221\\
278	0.0112015281497237\\
279	0.0112037690744243\\
280	0.0112060458161125\\
281	0.0112083591444165\\
282	0.0112107098290685\\
283	0.0112130986377089\\
284	0.0112155263335368\\
285	0.0112179936728162\\
286	0.0112205014022522\\
287	0.0112230502562556\\
288	0.0112256409541215\\
289	0.0112282741971546\\
290	0.0112309506657801\\
291	0.0112336710166893\\
292	0.0112364358800798\\
293	0.0112392458570581\\
294	0.0112421015172894\\
295	0.0112450033969886\\
296	0.0112479519973661\\
297	0.0112509477836606\\
298	0.0112539911848951\\
299	0.0112570825944323\\
300	0.0112602223709475\\
301	0.0112634108379538\\
302	0.0112666482907384\\
303	0.0112699350085331\\
304	0.0112732712557301\\
305	0.0112766572913039\\
306	0.0112800933809172\\
307	0.0112835798133203\\
308	0.0112871169228193\\
309	0.011290705051912\\
310	0.0112943449622133\\
311	0.0112980381668326\\
312	0.0113017862403157\\
313	0.0113055908211205\\
314	0.0113094536156605\\
315	0.0113133763998072\\
316	0.0113173610064946\\
317	0.0113214093369247\\
318	0.011325523360919\\
319	0.0113297051167896\\
320	0.0113339567106441\\
321	0.0113382803150201\\
322	0.0113426781667316\\
323	0.0113471525638106\\
324	0.0113517058614398\\
325	0.011356340466859\\
326	0.0113610588334391\\
327	0.0113658634542428\\
328	0.0113707568504109\\
329	0.011375741480718\\
330	0.0113808198777477\\
331	0.0113859945529474\\
332	0.0113912679746864\\
333	0.0113966425423589\\
334	0.0114021205620233\\
335	0.0114077042822515\\
336	0.0114133958302027\\
337	0.0114191970233761\\
338	0.0114251094447915\\
339	0.0114311343985366\\
340	0.0114372728635781\\
341	0.0114435254502255\\
342	0.0114498923782943\\
343	0.0114564757110774\\
344	0.0114633583268099\\
345	0.0114702665365942\\
346	0.0114771978509434\\
347	0.0114841495969587\\
348	0.0114911189108963\\
349	0.0114981027360209\\
350	0.0115050978444322\\
351	0.0115121009864434\\
352	0.0115191077344752\\
353	0.0115261138952198\\
354	0.0115331152649056\\
355	0.0115401073191033\\
356	0.0115470850426171\\
357	0.0115540444728333\\
358	0.0115609816187135\\
359	0.0115678909243073\\
360	0.0115747665668397\\
361	0.0115816024619457\\
362	0.0115883922741627\\
363	0.0115951294273967\\
364	0.0116018071207027\\
365	0.011608418347648\\
366	0.011614955918549\\
367	0.0116214124827961\\
368	0.0116277805356299\\
369	0.0116338499163943\\
370	0.0116398074613504\\
371	0.0116457079907705\\
372	0.0116515471516567\\
373	0.0116573206668163\\
374	0.0116630241190706\\
375	0.0116686538939094\\
376	0.0116742064822266\\
377	0.0116796778379773\\
378	0.011685064054032\\
379	0.0116903613997309\\
380	0.0116955663617011\\
381	0.0117006756880017\\
382	0.0117056864309802\\
383	0.0117105959999162\\
384	0.0117154022123978\\
385	0.0117201033468445\\
386	0.011724698194018\\
387	0.0117291861178515\\
388	0.0117335671001144\\
389	0.0117378417289012\\
390	0.0117417857985826\\
391	0.0117456930296464\\
392	0.0117495718499579\\
393	0.0117534228427067\\
394	0.0117572468299464\\
395	0.011761044885906\\
396	0.0117648183493374\\
397	0.0117685688345873\\
398	0.011772298241053\\
399	0.0117760087606568\\
400	0.0117797028829875\\
401	0.0117833833975279\\
402	0.0117870533920869\\
403	0.0117907162478317\\
404	0.0117943756295744\\
405	0.0117980354722945\\
406	0.0118016999614246\\
407	0.0118053744796544\\
408	0.0118090725038309\\
409	0.0118127954711748\\
410	0.0118165450847441\\
411	0.0118203232641424\\
412	0.0118241317785936\\
413	0.0118279723825916\\
414	0.0118318469163682\\
415	0.0118357572065733\\
416	0.0118397053253837\\
417	0.011843693180799\\
418	0.011847722657131\\
419	0.0118517956180726\\
420	0.0118559138917806\\
421	0.0118600792538731\\
422	0.0118642934106802\\
423	0.0118685578954809\\
424	0.011872873806322\\
425	0.0118772422391113\\
426	0.0118816642846503\\
427	0.0118861410256874\\
428	0.0118906735340313\\
429	0.0118952628677567\\
430	0.0118999100686662\\
431	0.0119046161600133\\
432	0.0119093821445707\\
433	0.011914209003154\\
434	0.011919097693642\\
435	0.0119240491504071\\
436	0.0119290642843765\\
437	0.0119341439838081\\
438	0.0119392891291221\\
439	0.0119445005993858\\
440	0.0119497792720521\\
441	0.0119551260227887\\
442	0.0119605417254037\\
443	0.0119660272518677\\
444	0.0119715834724338\\
445	0.0119772112558594\\
446	0.011982911469726\\
447	0.0119886849808515\\
448	0.0119945326557821\\
449	0.0120004553613493\\
450	0.0120064539652724\\
451	0.0120125293366282\\
452	0.0120186823456013\\
453	0.0120249138632233\\
454	0.012031224761097\\
455	0.012037615911101\\
456	0.0120440881850682\\
457	0.012050642454434\\
458	0.0120572795898441\\
459	0.012064000460719\\
460	0.0120708059347626\\
461	0.0120776968774113\\
462	0.0120846741512115\\
463	0.0120917386151199\\
464	0.0120988911237318\\
465	0.0121061325264567\\
466	0.0121134636666355\\
467	0.0121208853805968\\
468	0.012128398496648\\
469	0.0121360038339963\\
470	0.0121437022015959\\
471	0.0121514943969178\\
472	0.0121593812046366\\
473	0.0121673633952307\\
474	0.012175441723492\\
475	0.0121836169269392\\
476	0.0121918897241327\\
477	0.0122002608128826\\
478	0.0122087308683447\\
479	0.0122173005409972\\
480	0.0122259704544909\\
481	0.0122347412033652\\
482	0.0122436133506212\\
483	0.0122525874251436\\
484	0.0122616639189613\\
485	0.0122708432843368\\
486	0.012280125930673\\
487	0.0122895122212268\\
488	0.0122990024696139\\
489	0.0123085969360949\\
490	0.0123182958236242\\
491	0.0123280992736481\\
492	0.012338007361634\\
493	0.0123480200923118\\
494	0.0123581373946084\\
495	0.0123683591162527\\
496	0.0123786850180273\\
497	0.0123891147676427\\
498	0.0123996479332052\\
499	0.0124102839762488\\
500	0.0124210222442989\\
501	0.0124318619629314\\
502	0.0124428022272907\\
503	0.0124538419930214\\
504	0.0124649800665717\\
505	0.0124762150948152\\
506	0.0124875455539396\\
507	0.0124989697375404\\
508	0.0125104857438556\\
509	0.0125220914620709\\
510	0.012533784557616\\
511	0.0125455624563677\\
512	0.0125574223276655\\
513	0.0125693610660378\\
514	0.0125813752715246\\
515	0.0125934612284751\\
516	0.0126056148826818\\
517	0.0126178318167036\\
518	0.0126301072232128\\
519	0.0126424358761864\\
520	0.012654812099742\\
521	0.0126672297344029\\
522	0.0126796821005508\\
523	0.0126921619588054\\
524	0.0127046614670423\\
525	0.0127171720379401\\
526	0.0127296842664932\\
527	0.0127421902127353\\
528	0.012754684209797\\
529	0.0127680816133568\\
530	0.0127818538770959\\
531	0.0127954264333314\\
532	0.0128081395105162\\
533	0.0128202969787341\\
534	0.0128321162869548\\
535	0.0128432026117559\\
536	0.0128537758574423\\
537	0.0128640759468053\\
538	0.0128740848734177\\
539	0.0128838231896525\\
540	0.0128932602927639\\
541	0.012902471419193\\
542	0.0129118513862837\\
543	0.0129214188625862\\
544	0.0129324016955739\\
545	0.0129460376420307\\
546	0.0129596516018453\\
547	0.012972043748237\\
548	0.0129837885407481\\
549	0.0129951835481094\\
550	0.0130054949438586\\
551	0.0130158189828727\\
552	0.0130261636024336\\
553	0.0130366296303294\\
554	0.0130473222522103\\
555	0.0130586764567069\\
556	0.0130696869228016\\
557	0.0130804505623059\\
558	0.0130909406703917\\
559	0.0131014900219591\\
560	0.0131121238283323\\
561	0.0131228397507894\\
562	0.0131336341112898\\
563	0.0131445027539313\\
564	0.0131556054830169\\
565	0.01316682675996\\
566	0.0131777050422799\\
567	0.0131884046862925\\
568	0.0131991670871561\\
569	0.0132099913528043\\
570	0.0132208709214972\\
571	0.0132317984769864\\
572	0.013242765927568\\
573	0.0132537643399357\\
574	0.0132647838643886\\
575	0.013275813652582\\
576	0.0132868417670616\\
577	0.0132978550815367\\
578	0.0133088391706665\\
579	0.0133197781879283\\
580	0.0133306547299067\\
581	0.0133414496851422\\
582	0.0133521420655987\\
583	0.0133627088191946\\
584	0.0133731246235461\\
585	0.0133833616664418\\
586	0.013393389433806\\
587	0.0134031745674478\\
588	0.0134126809661912\\
589	0.0134218705998152\\
590	0.0134307062886727\\
591	0.0134391597728698\\
592	0.0134472338645631\\
593	0.0134551345125145\\
594	0.0134637112677116\\
595	0.0134740407759367\\
596	0.0134889777516421\\
597	0.0135160557778789\\
598	0.0135751148335434\\
599	0\\
600	0\\
};
\addplot [color=black!80!mycolor21,solid,forget plot]
  table[row sep=crcr]{%
1	0.0110658665044562\\
2	0.0110658756687729\\
3	0.0110658850498207\\
4	0.0110658946527232\\
5	0.0110659044827253\\
6	0.0110659145451954\\
7	0.011065924845629\\
8	0.0110659353896511\\
9	0.0110659461830198\\
10	0.0110659572316289\\
11	0.0110659685415116\\
12	0.0110659801188434\\
13	0.0110659919699456\\
14	0.0110660041012888\\
15	0.0110660165194963\\
16	0.0110660292313478\\
17	0.0110660422437829\\
18	0.0110660555639052\\
19	0.0110660691989856\\
20	0.0110660831564669\\
21	0.0110660974439675\\
22	0.0110661120692853\\
23	0.0110661270404025\\
24	0.0110661423654895\\
25	0.0110661580529093\\
26	0.0110661741112223\\
27	0.0110661905491909\\
28	0.011066207375784\\
29	0.0110662246001821\\
30	0.0110662422317821\\
31	0.0110662602802026\\
32	0.0110662787552889\\
33	0.0110662976671183\\
34	0.0110663170260058\\
35	0.0110663368425095\\
36	0.0110663571274362\\
37	0.0110663778918476\\
38	0.0110663991470659\\
39	0.01106642090468\\
40	0.0110664431765519\\
41	0.011066465974823\\
42	0.0110664893119208\\
43	0.0110665132005652\\
44	0.0110665376537758\\
45	0.0110665626848787\\
46	0.0110665883075138\\
47	0.0110666145356419\\
48	0.0110666413835525\\
49	0.0110666688658714\\
50	0.0110666969975684\\
51	0.0110667257939656\\
52	0.0110667552707454\\
53	0.0110667854439592\\
54	0.0110668163300358\\
55	0.0110668479457902\\
56	0.0110668803084328\\
57	0.0110669134355784\\
58	0.0110669473452559\\
59	0.0110669820559177\\
60	0.0110670175864499\\
61	0.0110670539561819\\
62	0.0110670911848974\\
63	0.0110671292928443\\
64	0.011067168300746\\
65	0.0110672082298122\\
66	0.0110672491017505\\
67	0.0110672909387776\\
68	0.0110673337636314\\
69	0.0110673775995829\\
70	0.011067422470449\\
71	0.0110674684006047\\
72	0.0110675154149961\\
73	0.0110675635391541\\
74	0.0110676127992073\\
75	0.0110676632218965\\
76	0.0110677148345883\\
77	0.0110677676652899\\
78	0.011067821742664\\
79	0.0110678770960436\\
80	0.0110679337554477\\
81	0.0110679917515973\\
82	0.0110680511159312\\
83	0.0110681118806229\\
84	0.0110681740785973\\
85	0.0110682377435481\\
86	0.0110683029099556\\
87	0.0110683696131045\\
88	0.0110684378891026\\
89	0.0110685077748996\\
90	0.0110685793083065\\
91	0.0110686525280152\\
92	0.0110687274736188\\
93	0.0110688041856321\\
94	0.0110688827055128\\
95	0.0110689630756828\\
96	0.0110690453395505\\
97	0.0110691295415331\\
98	0.0110692157270794\\
99	0.0110693039426938\\
100	0.0110693942359597\\
101	0.011069486655564\\
102	0.0110695812513228\\
103	0.0110696780742061\\
104	0.0110697771763644\\
105	0.011069878611155\\
106	0.0110699824331699\\
107	0.0110700886982625\\
108	0.0110701974635771\\
109	0.011070308787577\\
110	0.0110704227300746\\
111	0.0110705393522615\\
112	0.011070658716739\\
113	0.01107078088755\\
114	0.0110709059302108\\
115	0.0110710339117438\\
116	0.0110711649007114\\
117	0.0110712989672493\\
118	0.0110714361831021\\
119	0.0110715766216581\\
120	0.0110717203579859\\
121	0.0110718674688713\\
122	0.0110720180328544\\
123	0.0110721721302687\\
124	0.0110723298432794\\
125	0.0110724912559238\\
126	0.0110726564541517\\
127	0.0110728255258667\\
128	0.0110729985609684\\
129	0.0110731756513951\\
130	0.0110733568911678\\
131	0.0110735423764343\\
132	0.0110737322055147\\
133	0.0110739264789475\\
134	0.0110741252995363\\
135	0.0110743287723974\\
136	0.0110745370050084\\
137	0.0110747501072575\\
138	0.0110749681914937\\
139	0.011075191372577\\
140	0.0110754197679312\\
141	0.0110756534975952\\
142	0.0110758926842773\\
143	0.0110761374534085\\
144	0.0110763879331978\\
145	0.0110766442546874\\
146	0.0110769065518095\\
147	0.011077174961443\\
148	0.0110774496234716\\
149	0.0110777306808421\\
150	0.0110780182796234\\
151	0.0110783125690671\\
152	0.0110786137016668\\
153	0.0110789218332203\\
154	0.0110792371228904\\
155	0.0110795597332672\\
156	0.0110798898304309\\
157	0.0110802275840147\\
158	0.0110805731672678\\
159	0.0110809267571198\\
160	0.011081288534244\\
161	0.0110816586831217\\
162	0.0110820373921066\\
163	0.0110824248534886\\
164	0.0110828212635584\\
165	0.0110832268226712\\
166	0.0110836417353102\\
167	0.0110840662101506\\
168	0.0110845004601215\\
169	0.0110849447024687\\
170	0.0110853991588157\\
171	0.0110858640552241\\
172	0.0110863396222528\\
173	0.0110868260950164\\
174	0.0110873237132408\\
175	0.0110878327213187\\
176	0.0110883533683621\\
177	0.0110888859082526\\
178	0.01108943059969\\
179	0.0110899877062376\\
180	0.0110905574963645\\
181	0.0110911402434848\\
182	0.0110917362259929\\
183	0.0110923457272949\\
184	0.0110929690358352\\
185	0.0110936064451192\\
186	0.0110942582537292\\
187	0.011094924765336\\
188	0.0110956062887032\\
189	0.0110963031376851\\
190	0.0110970156312171\\
191	0.0110977440932984\\
192	0.0110984888529654\\
193	0.011099250244256\\
194	0.0111000286061648\\
195	0.0111008242825855\\
196	0.0111016376222437\\
197	0.0111024689786157\\
198	0.0111033187098347\\
199	0.0111041871785824\\
200	0.0111050747519647\\
201	0.0111059818013723\\
202	0.011106908702322\\
203	0.0111078558342806\\
204	0.0111088235804684\\
205	0.0111098123276412\\
206	0.0111108224658502\\
207	0.0111118543881778\\
208	0.0111129084904479\\
209	0.0111139851709095\\
210	0.0111150848298919\\
211	0.0111162078694299\\
212	0.0111173546928574\\
213	0.0111185257043679\\
214	0.0111197213085404\\
215	0.0111209419098283\\
216	0.0111221879120103\\
217	0.0111234597176016\\
218	0.0111247577272237\\
219	0.0111260823389312\\
220	0.0111274339474936\\
221	0.0111288129436318\\
222	0.0111302197132077\\
223	0.0111316546363646\\
224	0.0111331180866195\\
225	0.0111346104299051\\
226	0.0111361320235623\\
227	0.0111376832152826\\
228	0.0111392643420009\\
229	0.0111408757287406\\
230	0.0111425176874111\\
231	0.0111441905155621\\
232	0.0111458944950957\\
233	0.0111476298909424\\
234	0.0111493969497045\\
235	0.0111511958982747\\
236	0.011153026942436\\
237	0.0111548902654543\\
238	0.0111567860266728\\
239	0.0111587143601218\\
240	0.0111606753731602\\
241	0.0111626691451633\\
242	0.0111646957262801\\
243	0.0111667551362811\\
244	0.0111688473635224\\
245	0.0111709723640567\\
246	0.0111731300609223\\
247	0.0111753203436479\\
248	0.0111775430680124\\
249	0.0111797980561049\\
250	0.0111820850967317\\
251	0.0111844039462241\\
252	0.0111867543297012\\
253	0.0111891359428473\\
254	0.0111915484542651\\
255	0.0111939915084685\\
256	0.0111964647295773\\
257	0.0111989677257794\\
258	0.0112015000946204\\
259	0.0112040614291846\\
260	0.0112066513252404\\
261	0.0112092693894599\\
262	0.0112119152489404\\
263	0.0112145885624849\\
264	0.011217289033909\\
265	0.0112200164186713\\
266	0.0112227704903705\\
267	0.0112255518088507\\
268	0.0112283609678844\\
269	0.0112311985972577\\
270	0.0112340653649282\\
271	0.0112369619792517\\
272	0.0112398891912686\\
273	0.0112428477970429\\
274	0.01124583864004\\
275	0.0112488626135312\\
276	0.0112519206630064\\
277	0.0112550137885755\\
278	0.011258143047333\\
279	0.0112613095556583\\
280	0.0112645144914174\\
281	0.0112677590960289\\
282	0.0112710446763472\\
283	0.0112743726063129\\
284	0.0112777443283106\\
285	0.0112811613541667\\
286	0.0112846252657135\\
287	0.0112881377148324\\
288	0.0112917004228831\\
289	0.0112953151794132\\
290	0.0112989838400317\\
291	0.0113027083233196\\
292	0.0113064906066402\\
293	0.0113103327207012\\
294	0.0113142367427128\\
295	0.0113182047879818\\
296	0.0113222389997891\\
297	0.011326341537422\\
298	0.0113305145623161\\
299	0.0113347602224364\\
300	0.0113390806351764\\
301	0.0113434778664861\\
302	0.0113479538634801\\
303	0.0113525104463515\\
304	0.0113571493695748\\
305	0.0113618722217628\\
306	0.0113666803938898\\
307	0.0113715750501255\\
308	0.0113765571193788\\
309	0.011381707475626\\
310	0.011387049894648\\
311	0.0113924239757057\\
312	0.0113978290558508\\
313	0.01140326439794\\
314	0.0114087291882998\\
315	0.0114142225594584\\
316	0.0114197436140022\\
317	0.0114252912382915\\
318	0.0114308642143504\\
319	0.0114364612144706\\
320	0.0114420807938518\\
321	0.0114477213830169\\
322	0.011453381280053\\
323	0.0114590586427585\\
324	0.0114647514808346\\
325	0.0114704576484583\\
326	0.0114761748382996\\
327	0.0114819005811336\\
328	0.0114876322701486\\
329	0.0114933673073714\\
330	0.0114991018404361\\
331	0.0115048326684356\\
332	0.0115105563892819\\
333	0.0115162693862624\\
334	0.0115219677922677\\
335	0.0115276473766927\\
336	0.0115333044276723\\
337	0.0115389357892891\\
338	0.0115445368102847\\
339	0.0115501026601679\\
340	0.0115556283419749\\
341	0.0115611087059232\\
342	0.0115665384587616\\
343	0.0115718520637225\\
344	0.0115769909083038\\
345	0.0115821020108902\\
346	0.0115871822424816\\
347	0.0115922284051057\\
348	0.011597237240375\\
349	0.0116022054415922\\
350	0.0116071296784015\\
351	0.0116120066966827\\
352	0.0116168328023537\\
353	0.011621604512462\\
354	0.0116263184700951\\
355	0.0116309713279657\\
356	0.0116355596732972\\
357	0.0116400807569049\\
358	0.01164453206391\\
359	0.0116489106468586\\
360	0.0116532137378992\\
361	0.0116574387840926\\
362	0.011661583485679\\
363	0.0116656458346156\\
364	0.011669624155156\\
365	0.0116735171449084\\
366	0.0116773239145632\\
367	0.0116810440206136\\
368	0.0116846774678253\\
369	0.0116880779536732\\
370	0.0116914043544729\\
371	0.0116947044842502\\
372	0.0116979782518993\\
373	0.0117012257199864\\
374	0.0117044471152909\\
375	0.0117076428349863\\
376	0.0117108134609995\\
377	0.0117139597771556\\
378	0.0117170827794904\\
379	0.0117201836855822\\
380	0.0117232639426276\\
381	0.0117263252339571\\
382	0.0117293694838072\\
383	0.0117323988597782\\
384	0.0117354157727272\\
385	0.0117384228737062\\
386	0.0117414230475045\\
387	0.0117444194016274\\
388	0.0117474152504524\\
389	0.0117504141070941\\
390	0.0117534276614737\\
391	0.0117564573856178\\
392	0.0117595045329531\\
393	0.0117625704201511\\
394	0.0117656564234616\\
395	0.0117687639741982\\
396	0.0117718945533349\\
397	0.011775049685182\\
398	0.0117782309301198\\
399	0.011781439876384\\
400	0.0117846781309048\\
401	0.0117879473092122\\
402	0.0117912490244349\\
403	0.0117945848753464\\
404	0.0117979564333971\\
405	0.0118013652289198\\
406	0.0118048127417615\\
407	0.0118083003045304\\
408	0.0118118289848222\\
409	0.0118153997592706\\
410	0.0118190136094087\\
411	0.0118226715799512\\
412	0.0118263746192451\\
413	0.0118301236686515\\
414	0.0118339196593265\\
415	0.0118377635096281\\
416	0.011841656121783\\
417	0.0118455983824789\\
418	0.0118495911616511\\
419	0.0118536353114133\\
420	0.0118577316654239\\
421	0.0118618810388004\\
422	0.0118660842286248\\
423	0.0118703420183335\\
424	0.0118746551922817\\
425	0.0118790245355134\\
426	0.0118834508336072\\
427	0.0118879348726027\\
428	0.0118924774390156\\
429	0.0118970793199447\\
430	0.011901741303273\\
431	0.0119064641779606\\
432	0.0119112487344297\\
433	0.0119160957650317\\
434	0.0119210060645893\\
435	0.0119259804310075\\
436	0.0119310196659368\\
437	0.0119361245754705\\
438	0.0119412959703513\\
439	0.0119465346659483\\
440	0.0119518414822347\\
441	0.0119572172437638\\
442	0.0119626627796386\\
443	0.0119681789234718\\
444	0.0119737665133319\\
445	0.0119794263916702\\
446	0.0119851594052236\\
447	0.0119909664048864\\
448	0.0119968482455476\\
449	0.0120028057858843\\
450	0.0120088398881079\\
451	0.012014951417662\\
452	0.0120211412428938\\
453	0.0120274102346974\\
454	0.0120337592661276\\
455	0.0120401892119805\\
456	0.0120467009483396\\
457	0.0120532953520852\\
458	0.0120599733003649\\
459	0.0120667356700221\\
460	0.0120735833369826\\
461	0.0120805171755945\\
462	0.0120875380579224\\
463	0.0120946468529916\\
464	0.0121018444259829\\
465	0.0121091316373719\\
466	0.0121165093420129\\
467	0.0121239783881612\\
468	0.012131539616432\\
469	0.0121391938586919\\
470	0.0121469419368785\\
471	0.012154784661744\\
472	0.0121627228315191\\
473	0.0121707572304911\\
474	0.0121788886274922\\
475	0.0121871177742917\\
476	0.0121954454038866\\
477	0.012203872228684\\
478	0.0122123989385686\\
479	0.0122210261988474\\
480	0.0122297546480653\\
481	0.012238584895681\\
482	0.0122475175195965\\
483	0.0122565530635282\\
484	0.0122656920342115\\
485	0.0122749348984256\\
486	0.012284282079828\\
487	0.0122937339555843\\
488	0.0123032908527809\\
489	0.0123129530446041\\
490	0.0123227207462692\\
491	0.012332594110683\\
492	0.0123425732238204\\
493	0.0123526580997929\\
494	0.0123628486755882\\
495	0.0123731448054568\\
496	0.0123835462549175\\
497	0.0123940526943552\\
498	0.0124046636921786\\
499	0.0124153787075053\\
500	0.0124261970823371\\
501	0.0124371180331857\\
502	0.0124481406421071\\
503	0.0124592638470957\\
504	0.01247048643179\\
505	0.012481807014433\\
506	0.0124932240360285\\
507	0.0125047357476292\\
508	0.0125163401966863\\
509	0.012528035212386\\
510	0.0125398183898921\\
511	0.012551687073408\\
512	0.0125636383379657\\
513	0.0125756689698452\\
514	0.0125877754455178\\
515	0.0125999539090094\\
516	0.0126122001475681\\
517	0.0126245095655259\\
518	0.0126368771562392\\
519	0.0126492974719996\\
520	0.0126617645918139\\
521	0.0126742720869686\\
522	0.0126868129843154\\
523	0.0126993797272539\\
524	0.0127119641344364\\
525	0.0127245621636429\\
526	0.0127378286130893\\
527	0.012751826021275\\
528	0.0127656488468785\\
529	0.012778699366717\\
530	0.0127911762106919\\
531	0.0128033516574677\\
532	0.0128147987411692\\
533	0.0128258321847314\\
534	0.0128366389593896\\
535	0.0128472305654094\\
536	0.0128576146530734\\
537	0.0128677694856517\\
538	0.0128776751184309\\
539	0.0128873046562076\\
540	0.0128966255102941\\
541	0.0129058876807344\\
542	0.0129179735341843\\
543	0.0129314604712211\\
544	0.0129444722025304\\
545	0.0129560621552084\\
546	0.0129673691036461\\
547	0.0129776396512952\\
548	0.012987558268592\\
549	0.0129974903528408\\
550	0.0130075431523381\\
551	0.013017718660953\\
552	0.0130280221223931\\
553	0.0130388412504882\\
554	0.0130499213058635\\
555	0.0130605523893644\\
556	0.0130709501988515\\
557	0.0130812783988437\\
558	0.0130916971982159\\
559	0.0131022058233943\\
560	0.013112800233129\\
561	0.0131234770442047\\
562	0.0131342326716463\\
563	0.0131452413591923\\
564	0.0131563341090672\\
565	0.0131670883192149\\
566	0.0131777102595777\\
567	0.0131884047434965\\
568	0.013199167091273\\
569	0.0132099913534359\\
570	0.0132208709216939\\
571	0.0132317984770694\\
572	0.013242765927608\\
573	0.0132537643399573\\
574	0.0132647838644009\\
575	0.0132758136525892\\
576	0.013286841767066\\
577	0.0132978550815394\\
578	0.0133088391706681\\
579	0.013319778187929\\
580	0.013330654729907\\
581	0.0133414496851423\\
582	0.0133521420655987\\
583	0.0133627088191946\\
584	0.0133731246235461\\
585	0.0133833616664418\\
586	0.013393389433806\\
587	0.0134031745674478\\
588	0.0134126809661912\\
589	0.0134218705998152\\
590	0.0134307062886727\\
591	0.0134391597728698\\
592	0.0134472338645631\\
593	0.0134551345125145\\
594	0.0134637112677116\\
595	0.0134740407759367\\
596	0.0134889777516421\\
597	0.0135160557778789\\
598	0.0135751148335434\\
599	0\\
600	0\\
};
\addplot [color=black,solid,forget plot]
  table[row sep=crcr]{%
1	0.0110658750514072\\
2	0.0110658844718838\\
3	0.0110658941169069\\
4	0.0110659039918427\\
5	0.0110659141021865\\
6	0.0110659244535651\\
7	0.0110659350517404\\
8	0.0110659459026126\\
9	0.0110659570122233\\
10	0.0110659683867593\\
11	0.0110659800325558\\
12	0.0110659919561\\
13	0.0110660041640349\\
14	0.011066016663163\\
15	0.0110660294604502\\
16	0.0110660425630297\\
17	0.0110660559782058\\
18	0.0110660697134585\\
19	0.0110660837764475\\
20	0.0110660981750166\\
21	0.0110661129171978\\
22	0.0110661280112164\\
23	0.0110661434654955\\
24	0.0110661592886605\\
25	0.0110661754895442\\
26	0.0110661920771921\\
27	0.0110662090608669\\
28	0.0110662264500542\\
29	0.0110662442544681\\
30	0.0110662624840561\\
31	0.0110662811490052\\
32	0.0110663002597477\\
33	0.0110663198269669\\
34	0.0110663398616034\\
35	0.0110663603748612\\
36	0.0110663813782143\\
37	0.0110664028834127\\
38	0.01106642490249\\
39	0.0110664474477694\\
40	0.0110664705318711\\
41	0.0110664941677197\\
42	0.0110665183685512\\
43	0.0110665431479209\\
44	0.0110665685197107\\
45	0.0110665944981377\\
46	0.0110666210977617\\
47	0.0110666483334941\\
48	0.0110666762206057\\
49	0.0110667047747363\\
50	0.0110667340119032\\
51	0.0110667639485102\\
52	0.0110667946013576\\
53	0.0110668259876512\\
54	0.0110668581250128\\
55	0.0110668910314896\\
56	0.0110669247255654\\
57	0.0110669592261706\\
58	0.0110669945526933\\
59	0.0110670307249905\\
60	0.0110670677633996\\
61	0.0110671056887498\\
62	0.0110671445223745\\
63	0.0110671842861234\\
64	0.0110672250023751\\
65	0.0110672666940501\\
66	0.011067309384624\\
67	0.0110673530981411\\
68	0.0110673978592282\\
69	0.0110674436931089\\
70	0.0110674906256183\\
71	0.011067538683218\\
72	0.0110675878930111\\
73	0.0110676382827582\\
74	0.0110676898808937\\
75	0.0110677427165416\\
76	0.0110677968195333\\
77	0.0110678522204243\\
78	0.0110679089505122\\
79	0.011067967041855\\
80	0.0110680265272896\\
81	0.011068087440451\\
82	0.0110681498157918\\
83	0.0110682136886027\\
84	0.0110682790950326\\
85	0.0110683460721099\\
86	0.0110684146577642\\
87	0.0110684848908485\\
88	0.011068556811162\\
89	0.0110686304594731\\
90	0.0110687058775436\\
91	0.011068783108153\\
92	0.011068862195124\\
93	0.0110689431833475\\
94	0.0110690261188098\\
95	0.0110691110486191\\
96	0.0110691980210335\\
97	0.0110692870854891\\
98	0.0110693782926295\\
99	0.0110694716943358\\
100	0.0110695673437565\\
101	0.0110696652953396\\
102	0.0110697656048647\\
103	0.0110698683294757\\
104	0.0110699735277146\\
105	0.0110700812595569\\
106	0.0110701915864463\\
107	0.0110703045713315\\
108	0.011070420278704\\
109	0.0110705387746356\\
110	0.0110706601268187\\
111	0.0110707844046057\\
112	0.0110709116790509\\
113	0.0110710420229528\\
114	0.0110711755108975\\
115	0.011071312219303\\
116	0.0110714522264654\\
117	0.0110715956126055\\
118	0.0110717424599168\\
119	0.011071892852615\\
120	0.0110720468769882\\
121	0.011072204621449\\
122	0.0110723661765877\\
123	0.0110725316352265\\
124	0.0110727010924757\\
125	0.0110728746457907\\
126	0.0110730523950312\\
127	0.0110732344425212\\
128	0.0110734208931107\\
129	0.0110736118542396\\
130	0.0110738074360022\\
131	0.0110740077512143\\
132	0.0110742129154814\\
133	0.0110744230472687\\
134	0.0110746382679735\\
135	0.0110748587019986\\
136	0.0110750844768279\\
137	0.0110753157231045\\
138	0.0110755525747098\\
139	0.0110757951688453\\
140	0.0110760436461163\\
141	0.0110762981506179\\
142	0.0110765588300229\\
143	0.0110768258356718\\
144	0.0110770993226659\\
145	0.011077379449962\\
146	0.0110776663804698\\
147	0.0110779602811514\\
148	0.0110782613231244\\
149	0.011078569681766\\
150	0.0110788855368212\\
151	0.0110792090725129\\
152	0.011079540477655\\
153	0.0110798799457685\\
154	0.0110802276752001\\
155	0.0110805838692443\\
156	0.011080948736268\\
157	0.0110813224898387\\
158	0.0110817053488555\\
159	0.0110820975376836\\
160	0.0110824992862918\\
161	0.0110829108303939\\
162	0.0110833324115926\\
163	0.0110837642775282\\
164	0.0110842066820295\\
165	0.0110846598852695\\
166	0.0110851241539239\\
167	0.0110855997613339\\
168	0.0110860869876726\\
169	0.0110865861201152\\
170	0.0110870974530134\\
171	0.0110876212880731\\
172	0.0110881579345372\\
173	0.0110887077093713\\
174	0.0110892709374541\\
175	0.0110898479517721\\
176	0.0110904390936175\\
177	0.0110910447127914\\
178	0.0110916651678104\\
179	0.0110923008261176\\
180	0.0110929520642972\\
181	0.0110936192682946\\
182	0.0110943028336386\\
183	0.0110950031656692\\
184	0.0110957206797684\\
185	0.0110964558015952\\
186	0.0110972089673243\\
187	0.0110979806238882\\
188	0.0110987712292227\\
189	0.0110995812525157\\
190	0.0111004111744591\\
191	0.0111012614875032\\
192	0.0111021326961134\\
193	0.0111030253170296\\
194	0.0111039398795263\\
195	0.0111048769256752\\
196	0.0111058370106075\\
197	0.0111068207027769\\
198	0.0111078285842225\\
199	0.0111088612508298\\
200	0.0111099193125908\\
201	0.0111110033938607\\
202	0.0111121141336113\\
203	0.01111325218568\\
204	0.0111144182190122\\
205	0.011115612917897\\
206	0.0111168369821947\\
207	0.011118091127554\\
208	0.0111193760856168\\
209	0.0111206926042102\\
210	0.0111220414475223\\
211	0.0111234233962604\\
212	0.011124839247788\\
213	0.0111262898162391\\
214	0.0111277759326066\\
215	0.0111292984448007\\
216	0.0111308582176747\\
217	0.011132456133014\\
218	0.011134093089484\\
219	0.0111357700025325\\
220	0.0111374878042418\\
221	0.0111392474431245\\
222	0.0111410498838575\\
223	0.011142896106949\\
224	0.0111447871083288\\
225	0.0111467238988578\\
226	0.0111487075037465\\
227	0.0111507389618748\\
228	0.0111528193250031\\
229	0.0111549496568659\\
230	0.0111571310321366\\
231	0.0111593645352508\\
232	0.0111616512590792\\
233	0.0111639923034324\\
234	0.0111663887733889\\
235	0.0111688417774267\\
236	0.0111713524253472\\
237	0.0111739218259714\\
238	0.0111765510845945\\
239	0.0111792413001783\\
240	0.0111819935622647\\
241	0.0111848089475906\\
242	0.0111876885163837\\
243	0.0111906333083215\\
244	0.011193644338132\\
245	0.0111967225908176\\
246	0.0111998690164836\\
247	0.0112030845247539\\
248	0.0112063699787597\\
249	0.0112097261886871\\
250	0.0112131539048771\\
251	0.0112166538104739\\
252	0.011220226513625\\
253	0.0112238725392461\\
254	0.0112275923203743\\
255	0.0112313861891475\\
256	0.0112352543674674\\
257	0.011239196957426\\
258	0.0112432139316089\\
259	0.0112473051234342\\
260	0.0112514702177651\\
261	0.0112557087421901\\
262	0.011260020059759\\
263	0.0112644033651342\\
264	0.0112688576902486\\
265	0.011273388597383\\
266	0.011278120117581\\
267	0.0112828810304852\\
268	0.0112876707737252\\
269	0.0112924887445411\\
270	0.0112973342983057\\
271	0.0113022067470334\\
272	0.0113071053578852\\
273	0.011312029351671\\
274	0.0113169779013589\\
275	0.0113219501305961\\
276	0.0113269451122499\\
277	0.01133196186697\\
278	0.0113369993617633\\
279	0.0113420565087861\\
280	0.0113471321641829\\
281	0.0113522251267056\\
282	0.0113573341366179\\
283	0.0113624578747126\\
284	0.0113675949614561\\
285	0.0113727439562827\\
286	0.0113779033570593\\
287	0.011383071599742\\
288	0.0113882470582507\\
289	0.0113934280445865\\
290	0.0113986128092188\\
291	0.0114037995417722\\
292	0.0114089863720399\\
293	0.0114141713713571\\
294	0.0114193525543642\\
295	0.0114245278811968\\
296	0.0114296952601534\\
297	0.0114348525509359\\
298	0.011439997568717\\
299	0.0114451280898342\\
300	0.0114502418619611\\
301	0.01145533663055\\
302	0.0114604102453435\\
303	0.0114654601104107\\
304	0.0114704831772255\\
305	0.0114754771528777\\
306	0.0114804397432685\\
307	0.0114853686550007\\
308	0.0114902615868697\\
309	0.0114950699852063\\
310	0.0114997756240239\\
311	0.0115044649884105\\
312	0.011509136143499\\
313	0.0115137871044368\\
314	0.0115184158262472\\
315	0.0115230201818247\\
316	0.0115275982669801\\
317	0.0115321492275243\\
318	0.0115366708191833\\
319	0.011541160768341\\
320	0.0115456167766223\\
321	0.0115500365260829\\
322	0.011554417685061\\
323	0.0115587579147462\\
324	0.011563054876542\\
325	0.0115673062403616\\
326	0.0115715096942642\\
327	0.0115756629571365\\
328	0.0115797638036078\\
329	0.0115838101614979\\
330	0.0115877995386961\\
331	0.011591729898201\\
332	0.0115955992938587\\
333	0.0115994058818124\\
334	0.0116031479213682\\
335	0.0116068237270561\\
336	0.0116104320785765\\
337	0.0116139722842614\\
338	0.0116174432712505\\
339	0.0116208441901195\\
340	0.0116241744350945\\
341	0.0116274336603087\\
342	0.0116306217831731\\
343	0.0116336953750199\\
344	0.0116366214021809\\
345	0.0116395206895466\\
346	0.0116423928362576\\
347	0.0116452375251151\\
348	0.0116480545296806\\
349	0.0116508437215552\\
350	0.0116536050778247\\
351	0.0116563386888731\\
352	0.0116590447684414\\
353	0.0116617236583792\\
354	0.0116643758346477\\
355	0.0116670019139605\\
356	0.0116696026593246\\
357	0.011672178981745\\
358	0.0116747319476121\\
359	0.0116772627890906\\
360	0.0116797729079845\\
361	0.011682263878433\\
362	0.0116847374481983\\
363	0.0116871955383681\\
364	0.0116896402411757\\
365	0.0116920738155775\\
366	0.0116944986801466\\
367	0.0116969174039349\\
368	0.0116993327066238\\
369	0.0117017525737516\\
370	0.0117041795572421\\
371	0.0117066145866293\\
372	0.0117090586440313\\
373	0.0117115127627278\\
374	0.0117139780252346\\
375	0.0117164555610023\\
376	0.0117189465435208\\
377	0.0117214521866742\\
378	0.011723973740466\\
379	0.0117265124860925\\
380	0.011729069730349\\
381	0.0117316467993629\\
382	0.0117342450316503\\
383	0.0117368657705124\\
384	0.0117395103557893\\
385	0.0117421801149998\\
386	0.0117448763539552\\
387	0.011747600347295\\
388	0.0117503533304433\\
389	0.011753136491348\\
390	0.0117559506574283\\
391	0.0117587966517825\\
392	0.011761675301312\\
393	0.0117645874348435\\
394	0.011767533881213\\
395	0.0117705154673326\\
396	0.0117735330162686\\
397	0.0117765873453587\\
398	0.0117796792644017\\
399	0.0117828095739561\\
400	0.0117859790637783\\
401	0.0117891885114308\\
402	0.0117924386810585\\
403	0.0117957303222432\\
404	0.0117990641685028\\
405	0.011802440933399\\
406	0.0118058612926688\\
407	0.0118093259224047\\
408	0.0118128356691035\\
409	0.011816391248266\\
410	0.0118199933401305\\
411	0.0118236426247192\\
412	0.0118273397829966\\
413	0.0118310854969216\\
414	0.0118348804496045\\
415	0.011838725325558\\
416	0.0118426208110733\\
417	0.0118465675946009\\
418	0.011850566367199\\
419	0.0118546178230534\\
420	0.011858722660061\\
421	0.0118628815804639\\
422	0.0118670952915211\\
423	0.0118713645060784\\
424	0.0118756899426467\\
425	0.0118800723254883\\
426	0.0118845123847101\\
427	0.0118890108563599\\
428	0.0118935684825252\\
429	0.0118981860114297\\
430	0.0119028641975259\\
431	0.0119076038015782\\
432	0.0119124055907354\\
433	0.0119172703385852\\
434	0.011922198825189\\
435	0.0119271918370916\\
436	0.0119322501673012\\
437	0.0119373746152351\\
438	0.011942565986647\\
439	0.0119478250935408\\
440	0.011953152754072\\
441	0.0119585497924328\\
442	0.0119640170387223\\
443	0.0119695553287987\\
444	0.0119751655041133\\
445	0.0119808484115245\\
446	0.0119866049030919\\
447	0.0119924358358483\\
448	0.0119983420715493\\
449	0.0120043244764002\\
450	0.0120103839207597\\
451	0.0120165212788188\\
452	0.012022737428255\\
453	0.0120290332498596\\
454	0.0120354096271364\\
455	0.0120418674458711\\
456	0.0120484075936679\\
457	0.0120550309594545\\
458	0.01206173843295\\
459	0.0120685309040974\\
460	0.0120754092624557\\
461	0.0120823743965509\\
462	0.012089427193184\\
463	0.0120965685366908\\
464	0.012103799308155\\
465	0.0121111203845674\\
466	0.0121185326379307\\
467	0.0121260369343065\\
468	0.0121336341327995\\
469	0.0121413250844764\\
470	0.012149110631216\\
471	0.0121569916044837\\
472	0.0121649688240296\\
473	0.0121730430965016\\
474	0.0121812152139712\\
475	0.0121894859523644\\
476	0.0121978560697927\\
477	0.0122063263047781\\
478	0.012214897374364\\
479	0.0122235699721055\\
480	0.0122323447659308\\
481	0.012241222395866\\
482	0.0122502034716117\\
483	0.0122592885699647\\
484	0.0122684782320714\\
485	0.0122777729605017\\
486	0.0122871732161331\\
487	0.0122966794148269\\
488	0.0123062919238869\\
489	0.0123160110582802\\
490	0.0123258370766045\\
491	0.0123357701767836\\
492	0.0123458104914685\\
493	0.0123559580831224\\
494	0.0123662129387656\\
495	0.0123765749643512\\
496	0.0123870439787446\\
497	0.0123976197072715\\
498	0.0124083017748015\\
499	0.0124190896983255\\
500	0.0124299828789861\\
501	0.0124409805935121\\
502	0.0124520819850047\\
503	0.0124632860530193\\
504	0.0124745916428773\\
505	0.0124859974341389\\
506	0.0124975019281584\\
507	0.0125091034346369\\
508	0.0125208000570761\\
509	0.0125325896770302\\
510	0.0125444699370366\\
511	0.0125564382221003\\
512	0.0125684916395874\\
513	0.0125806269973728\\
514	0.0125928407800687\\
515	0.0126051291231442\\
516	0.0126174877847281\\
517	0.0126299121148679\\
518	0.0126423970219956\\
519	0.0126549369363353\\
520	0.0126675257699616\\
521	0.0126801568732061\\
522	0.0126928229870882\\
523	0.0127055161914419\\
524	0.0127196267317596\\
525	0.0127337777619999\\
526	0.012747377540542\\
527	0.0127602292836677\\
528	0.0127727542762508\\
529	0.0127845821122983\\
530	0.0127959855671431\\
531	0.012807203733555\\
532	0.0128182626229069\\
533	0.0128291679587394\\
534	0.0128399046674119\\
535	0.0128504550386671\\
536	0.0128607967821975\\
537	0.0128709061162104\\
538	0.0128807570360296\\
539	0.0128903225124733\\
540	0.0129024979729618\\
541	0.0129162104251819\\
542	0.0129285405990742\\
543	0.0129400119605311\\
544	0.0129506631698186\\
545	0.0129602132373445\\
546	0.0129697686235692\\
547	0.0129794096850582\\
548	0.0129891740306582\\
549	0.0129990701938975\\
550	0.0130090977632239\\
551	0.0130192639639011\\
552	0.0130301758884645\\
553	0.013040920155061\\
554	0.0130513026381296\\
555	0.0130614236280406\\
556	0.0130716242256193\\
557	0.0130819211329755\\
558	0.0130923104237779\\
559	0.0131027877537724\\
560	0.013113349546112\\
561	0.0131239958493138\\
562	0.0131348838140754\\
563	0.0131458992837798\\
564	0.0131565457027411\\
565	0.0131670890149139\\
566	0.0131777102671135\\
567	0.0131884047440515\\
568	0.0131991670913618\\
569	0.0132099913534643\\
570	0.0132208709217061\\
571	0.0132317984770754\\
572	0.0132427659276113\\
573	0.0132537643399592\\
574	0.013264783864402\\
575	0.0132758136525899\\
576	0.0132868417670664\\
577	0.0132978550815396\\
578	0.0133088391706682\\
579	0.0133197781879291\\
580	0.013330654729907\\
581	0.0133414496851423\\
582	0.0133521420655987\\
583	0.0133627088191946\\
584	0.0133731246235461\\
585	0.0133833616664418\\
586	0.013393389433806\\
587	0.0134031745674478\\
588	0.0134126809661912\\
589	0.0134218705998152\\
590	0.0134307062886727\\
591	0.0134391597728698\\
592	0.0134472338645631\\
593	0.0134551345125145\\
594	0.0134637112677116\\
595	0.0134740407759367\\
596	0.0134889777516421\\
597	0.0135160557778789\\
598	0.0135751148335434\\
599	0\\
600	0\\
};
\end{axis}
\end{tikzpicture}% 
  \caption{Discrete Time}
\end{subfigure}\\

\leavevmode\smash{\makebox[0pt]{\hspace{-7em}% HORIZONTAL POSITION           
  \rotatebox[origin=l]{90}{\hspace{20em}% VERTICAL POSITION
    Depth $\delta^+$}%
}}\hspace{0pt plus 1filll}\null

Time (s)

\vspace{1cm}
\begin{subfigure}{\linewidth}
  \centering
  \tikzsetnextfilename{altdeltalegend}
  \definecolor{mycolor1}{rgb}{0.00000,1.00000,0.14286}%
\definecolor{mycolor2}{rgb}{0.00000,1.00000,0.28571}%
\definecolor{mycolor3}{rgb}{0.00000,1.00000,0.42857}%
\definecolor{mycolor4}{rgb}{0.00000,1.00000,0.57143}%
\definecolor{mycolor5}{rgb}{0.00000,1.00000,0.71429}%
\definecolor{mycolor6}{rgb}{0.00000,1.00000,0.85714}%
\definecolor{mycolor7}{rgb}{0.00000,1.00000,1.00000}%
\definecolor{mycolor8}{rgb}{0.00000,0.87500,1.00000}%
\definecolor{mycolor9}{rgb}{0.00000,0.62500,1.00000}%
\definecolor{mycolor10}{rgb}{0.12500,0.00000,1.00000}%
\definecolor{mycolor11}{rgb}{0.25000,0.00000,1.00000}%
\definecolor{mycolor12}{rgb}{0.37500,0.00000,1.00000}%
\definecolor{mycolor13}{rgb}{0.50000,0.00000,1.00000}%
\definecolor{mycolor14}{rgb}{0.62500,0.00000,1.00000}%
\definecolor{mycolor15}{rgb}{0.75000,0.00000,1.00000}%
\definecolor{mycolor16}{rgb}{0.87500,0.00000,1.00000}%
\definecolor{mycolor17}{rgb}{1.00000,0.00000,1.00000}%
\definecolor{mycolor18}{rgb}{1.00000,0.00000,0.87500}%
\definecolor{mycolor19}{rgb}{1.00000,0.00000,0.62500}%
\definecolor{mycolor20}{rgb}{0.85714,0.00000,0.00000}%
\definecolor{mycolor21}{rgb}{0.71429,0.00000,0.00000}%
%[trim axis left, trim axis right]
\begin{tikzpicture}
\begin{axis}[%
    hide axis,
    scale only axis,
    height=0pt,
    width=0pt,
    point meta min=-19,
    point meta max=19,
    colormap={mymap}{[1pt] rgb(0pt)=(0,1,0); rgb(7pt)=(0,1,1); rgb(15pt)=(0,0,1); rgb(23pt)=(1,0,1); rgb(31pt)=(1,0,0); rgb(38pt)=(0,0,0)},
    colorbar horizontal,
    colorbar style={width=15cm,xtick={{-15},{-10},{-5},{0},{5},{10},{15}}}
    %colorbar style={separate axis lines,every outer x axis line/.append style={black},every x tick label/.append style={font=\color{black}},every outer y axis line/.append style={black},every y tick label/.append style={font=\color{black}},yticklabels={{-19},{-17},{-15},{-13},{-11},{-9},{-7},{-5},{-3},{-1},{1},{3},{5},{7},{9},{11},{13},{15},{17},{19}}}
]%
    \addplot [draw=none] coordinates {(0,0)};
\end{axis}
\end{tikzpicture}
 
\end{subfigure}%
  \caption{Optimal buy depths $\delta^{+}$ for Markov state $Z=(\rho = -1, \Delta S = -1)$, implying heavy imbalance in favor of sell pressure, and having previously seen a downward price change. We expect the midprice to fall.}
  \label{fig:comp_dp_z1}
\end{figure}

\begin{figure}
\centering
\begin{subfigure}{.45\linewidth}
  \centering
  \setlength\figureheight{\linewidth} 
  \setlength\figurewidth{\linewidth}
  \tikzsetnextfilename{dp_colorbar/dp_cts_z8}
  % This file was created by matlab2tikz.
%
%The latest updates can be retrieved from
%  http://www.mathworks.com/matlabcentral/fileexchange/22022-matlab2tikz-matlab2tikz
%where you can also make suggestions and rate matlab2tikz.
%
\definecolor{mycolor1}{rgb}{0.00000,1.00000,0.14286}%
\definecolor{mycolor2}{rgb}{0.00000,1.00000,0.28571}%
\definecolor{mycolor3}{rgb}{0.00000,1.00000,0.42857}%
\definecolor{mycolor4}{rgb}{0.00000,1.00000,0.57143}%
\definecolor{mycolor5}{rgb}{0.00000,1.00000,0.71429}%
\definecolor{mycolor6}{rgb}{0.00000,1.00000,0.85714}%
\definecolor{mycolor7}{rgb}{0.00000,1.00000,1.00000}%
\definecolor{mycolor8}{rgb}{0.00000,0.87500,1.00000}%
\definecolor{mycolor9}{rgb}{0.00000,0.62500,1.00000}%
\definecolor{mycolor10}{rgb}{0.12500,0.00000,1.00000}%
\definecolor{mycolor11}{rgb}{0.25000,0.00000,1.00000}%
\definecolor{mycolor12}{rgb}{0.37500,0.00000,1.00000}%
\definecolor{mycolor13}{rgb}{0.50000,0.00000,1.00000}%
\definecolor{mycolor14}{rgb}{0.62500,0.00000,1.00000}%
\definecolor{mycolor15}{rgb}{0.75000,0.00000,1.00000}%
\definecolor{mycolor16}{rgb}{0.87500,0.00000,1.00000}%
\definecolor{mycolor17}{rgb}{1.00000,0.00000,1.00000}%
\definecolor{mycolor18}{rgb}{1.00000,0.00000,0.87500}%
\definecolor{mycolor19}{rgb}{1.00000,0.00000,0.62500}%
\definecolor{mycolor20}{rgb}{0.85714,0.00000,0.00000}%
\definecolor{mycolor21}{rgb}{0.71429,0.00000,0.00000}%
%
\begin{tikzpicture}

\begin{axis}[%
width=4.1in,
height=3.803in,
at={(0.809in,0.513in)},
scale only axis,
point meta min=0,
point meta max=1,
every outer x axis line/.append style={black},
every x tick label/.append style={font=\color{black}},
xmin=0,
xmax=600,
every outer y axis line/.append style={black},
every y tick label/.append style={font=\color{black}},
ymin=0,
ymax=0.012,
axis background/.style={fill=white},
axis x line*=bottom,
axis y line*=left,
colormap={mymap}{[1pt] rgb(0pt)=(0,1,0); rgb(7pt)=(0,1,1); rgb(15pt)=(0,0,1); rgb(23pt)=(1,0,1); rgb(31pt)=(1,0,0); rgb(38pt)=(0,0,0)},
colorbar,
colorbar style={separate axis lines,every outer x axis line/.append style={black},every x tick label/.append style={font=\color{black}},every outer y axis line/.append style={black},every y tick label/.append style={font=\color{black}},yticklabels={{-19},{-17},{-15},{-13},{-11},{-9},{-7},{-5},{-3},{-1},{1},{3},{5},{7},{9},{11},{13},{15},{17},{19}}}
]
\addplot [color=green,solid,forget plot]
  table[row sep=crcr]{%
0.01	0.00496798049573811\\
1.01	0.00496797956375651\\
2.01	0.00496797861304558\\
3.01	0.00496797764323127\\
4.01	0.00496797665393224\\
5.01	0.00496797564475905\\
6.01	0.00496797461531515\\
7.01	0.00496797356519586\\
8.01	0.00496797249398832\\
9.01	0.00496797140127171\\
10.01	0.00496797028661687\\
11.01	0.00496796914958604\\
12.01	0.00496796798973329\\
13.01	0.00496796680660276\\
14.01	0.00496796559973101\\
15.01	0.00496796436864398\\
16.01	0.00496796311285933\\
17.01	0.00496796183188498\\
18.01	0.0049679605252184\\
19.01	0.0049679591923482\\
20.01	0.00496795783275209\\
21.01	0.00496795644589804\\
22.01	0.00496795503124293\\
23.01	0.0049679535882333\\
24.01	0.00496795211630469\\
25.01	0.00496795061488138\\
26.01	0.00496794908337656\\
27.01	0.00496794752119143\\
28.01	0.00496794592771586\\
29.01	0.00496794430232737\\
30.01	0.00496794264439125\\
31.01	0.00496794095326062\\
32.01	0.00496793922827536\\
33.01	0.00496793746876279\\
34.01	0.00496793567403623\\
35.01	0.00496793384339637\\
36.01	0.00496793197612981\\
37.01	0.00496793007150893\\
38.01	0.00496792812879222\\
39.01	0.00496792614722328\\
40.01	0.00496792412603048\\
41.01	0.00496792206442768\\
42.01	0.00496791996161281\\
43.01	0.00496791781676843\\
44.01	0.00496791562906102\\
45.01	0.00496791339764023\\
46.01	0.0049679111216395\\
47.01	0.00496790880017509\\
48.01	0.00496790643234572\\
49.01	0.00496790401723289\\
50.01	0.00496790155389993\\
51.01	0.00496789904139095\\
52.01	0.00496789647873287\\
53.01	0.00496789386493225\\
54.01	0.00496789119897662\\
55.01	0.00496788847983418\\
56.01	0.00496788570645216\\
57.01	0.00496788287775749\\
58.01	0.00496787999265634\\
59.01	0.00496787705003291\\
60.01	0.0049678740487503\\
61.01	0.00496787098764873\\
62.01	0.00496786786554605\\
63.01	0.00496786468123687\\
64.01	0.00496786143349241\\
65.01	0.00496785812106\\
66.01	0.0049678547426623\\
67.01	0.00496785129699696\\
68.01	0.00496784778273635\\
69.01	0.00496784419852676\\
70.01	0.00496784054298853\\
71.01	0.00496783681471445\\
72.01	0.00496783301227056\\
73.01	0.00496782913419489\\
74.01	0.00496782517899622\\
75.01	0.00496782114515502\\
76.01	0.0049678170311221\\
77.01	0.00496781283531784\\
78.01	0.00496780855613233\\
79.01	0.00496780419192364\\
80.01	0.00496779974101881\\
81.01	0.00496779520171157\\
82.01	0.00496779057226297\\
83.01	0.0049677858509\\
84.01	0.00496778103581556\\
85.01	0.00496777612516723\\
86.01	0.00496777111707687\\
87.01	0.00496776600963035\\
88.01	0.00496776080087545\\
89.01	0.00496775548882287\\
90.01	0.00496775007144455\\
91.01	0.00496774454667329\\
92.01	0.00496773891240172\\
93.01	0.00496773316648137\\
94.01	0.00496772730672305\\
95.01	0.00496772133089444\\
96.01	0.00496771523672025\\
97.01	0.00496770902188127\\
98.01	0.00496770268401348\\
99.01	0.00496769622070757\\
100.01	0.00496768962950684\\
101.01	0.00496768290790804\\
102.01	0.00496767605335922\\
103.01	0.00496766906325914\\
104.01	0.00496766193495661\\
105.01	0.00496765466574904\\
106.01	0.00496764725288199\\
107.01	0.00496763969354833\\
108.01	0.00496763198488663\\
109.01	0.00496762412398001\\
110.01	0.00496761610785605\\
111.01	0.00496760793348489\\
112.01	0.00496759959777849\\
113.01	0.00496759109758947\\
114.01	0.00496758242970983\\
115.01	0.00496757359087045\\
116.01	0.00496756457773918\\
117.01	0.00496755538691985\\
118.01	0.00496754601495179\\
119.01	0.00496753645830684\\
120.01	0.00496752671339102\\
121.01	0.0049675167765396\\
122.01	0.00496750664401897\\
123.01	0.00496749631202413\\
124.01	0.00496748577667636\\
125.01	0.00496747503402365\\
126.01	0.00496746408003841\\
127.01	0.00496745291061582\\
128.01	0.00496744152157287\\
129.01	0.00496742990864672\\
130.01	0.00496741806749294\\
131.01	0.00496740599368548\\
132.01	0.00496739368271205\\
133.01	0.0049673811299762\\
134.01	0.00496736833079335\\
135.01	0.0049673552803895\\
136.01	0.0049673419738999\\
137.01	0.00496732840636764\\
138.01	0.00496731457274186\\
139.01	0.00496730046787498\\
140.01	0.00496728608652287\\
141.01	0.00496727142334064\\
142.01	0.00496725647288308\\
143.01	0.00496724122960136\\
144.01	0.00496722568784165\\
145.01	0.00496720984184302\\
146.01	0.00496719368573574\\
147.01	0.00496717721353888\\
148.01	0.00496716041915841\\
149.01	0.00496714329638557\\
150.01	0.0049671258388939\\
151.01	0.00496710804023749\\
152.01	0.00496708989384982\\
153.01	0.00496707139303954\\
154.01	0.00496705253098943\\
155.01	0.00496703330075438\\
156.01	0.00496701369525813\\
157.01	0.00496699370729182\\
158.01	0.00496697332951061\\
159.01	0.00496695255443192\\
160.01	0.0049669313744327\\
161.01	0.00496690978174658\\
162.01	0.00496688776846208\\
163.01	0.00496686532651897\\
164.01	0.00496684244770588\\
165.01	0.00496681912365849\\
166.01	0.00496679534585518\\
167.01	0.00496677110561573\\
168.01	0.00496674639409729\\
169.01	0.00496672120229214\\
170.01	0.00496669552102431\\
171.01	0.00496666934094662\\
172.01	0.00496664265253744\\
173.01	0.00496661544609829\\
174.01	0.00496658771175008\\
175.01	0.00496655943942979\\
176.01	0.00496653061888745\\
177.01	0.00496650123968218\\
178.01	0.00496647129118009\\
179.01	0.00496644076254919\\
180.01	0.00496640964275722\\
181.01	0.00496637792056726\\
182.01	0.00496634558453441\\
183.01	0.00496631262300174\\
184.01	0.00496627902409728\\
185.01	0.00496624477572911\\
186.01	0.00496620986558253\\
187.01	0.00496617428111516\\
188.01	0.0049661380095539\\
189.01	0.00496610103788987\\
190.01	0.00496606335287454\\
191.01	0.00496602494101573\\
192.01	0.00496598578857367\\
193.01	0.00496594588155548\\
194.01	0.0049659052057117\\
195.01	0.0049658637465313\\
196.01	0.00496582148923722\\
197.01	0.00496577841878199\\
198.01	0.00496573451984251\\
199.01	0.00496568977681478\\
200.01	0.00496564417381083\\
201.01	0.00496559769465164\\
202.01	0.00496555032286284\\
203.01	0.00496550204167005\\
204.01	0.00496545283399281\\
205.01	0.00496540268243993\\
206.01	0.00496535156930348\\
207.01	0.00496529947655403\\
208.01	0.00496524638583416\\
209.01	0.00496519227845361\\
210.01	0.00496513713538261\\
211.01	0.00496508093724666\\
212.01	0.00496502366432056\\
213.01	0.00496496529652169\\
214.01	0.00496490581340471\\
215.01	0.00496484519415411\\
216.01	0.00496478341757959\\
217.01	0.00496472046210784\\
218.01	0.00496465630577645\\
219.01	0.00496459092622768\\
220.01	0.00496452430070119\\
221.01	0.00496445640602756\\
222.01	0.00496438721862081\\
223.01	0.00496431671447074\\
224.01	0.00496424486913639\\
225.01	0.0049641716577395\\
226.01	0.00496409705495461\\
227.01	0.00496402103500305\\
228.01	0.00496394357164527\\
229.01	0.00496386463817209\\
230.01	0.00496378420739691\\
231.01	0.00496370225164814\\
232.01	0.00496361874275967\\
233.01	0.00496353365206307\\
234.01	0.00496344695037963\\
235.01	0.0049633586080099\\
236.01	0.00496326859472613\\
237.01	0.00496317687976287\\
238.01	0.00496308343180763\\
239.01	0.00496298821899086\\
240.01	0.0049628912088775\\
241.01	0.00496279236845641\\
242.01	0.00496269166413039\\
243.01	0.00496258906170711\\
244.01	0.00496248452638727\\
245.01	0.0049623780227551\\
246.01	0.00496226951476748\\
247.01	0.00496215896574246\\
248.01	0.00496204633834905\\
249.01	0.00496193159459572\\
250.01	0.00496181469581834\\
251.01	0.00496169560266875\\
252.01	0.00496157427510256\\
253.01	0.00496145067236737\\
254.01	0.00496132475298975\\
255.01	0.00496119647476285\\
256.01	0.0049610657947333\\
257.01	0.00496093266918805\\
258.01	0.00496079705364012\\
259.01	0.00496065890281581\\
260.01	0.00496051817063967\\
261.01	0.00496037481022158\\
262.01	0.00496022877383926\\
263.01	0.00496008001292559\\
264.01	0.00495992847805255\\
265.01	0.00495977411891441\\
266.01	0.00495961688431321\\
267.01	0.00495945672214072\\
268.01	0.0049592935793629\\
269.01	0.00495912740200139\\
270.01	0.00495895813511622\\
271.01	0.00495878572278778\\
272.01	0.00495861010809757\\
273.01	0.00495843123310991\\
274.01	0.00495824903885207\\
275.01	0.00495806346529388\\
276.01	0.00495787445132729\\
277.01	0.00495768193474498\\
278.01	0.00495748585221893\\
279.01	0.00495728613927838\\
280.01	0.00495708273028576\\
281.01	0.00495687555841527\\
282.01	0.00495666455562591\\
283.01	0.00495644965263901\\
284.01	0.00495623077891117\\
285.01	0.00495600786260931\\
286.01	0.00495578083058172\\
287.01	0.0049555496083324\\
288.01	0.00495531411999132\\
289.01	0.00495507428828458\\
290.01	0.00495483003450576\\
291.01	0.00495458127848298\\
292.01	0.00495432793854798\\
293.01	0.00495406993150228\\
294.01	0.00495380717258449\\
295.01	0.0049535395754343\\
296.01	0.00495326705205668\\
297.01	0.00495298951278534\\
298.01	0.00495270686624409\\
299.01	0.00495241901930786\\
300.01	0.00495212587706293\\
301.01	0.00495182734276457\\
302.01	0.00495152331779542\\
303.01	0.00495121370162072\\
304.01	0.00495089839174374\\
305.01	0.00495057728365992\\
306.01	0.00495025027080923\\
307.01	0.00494991724452621\\
308.01	0.00494957809399203\\
309.01	0.0049492327061817\\
310.01	0.0049488809658119\\
311.01	0.00494852275528771\\
312.01	0.00494815795464624\\
313.01	0.00494778644150185\\
314.01	0.00494740809098738\\
315.01	0.00494702277569646\\
316.01	0.0049466303656223\\
317.01	0.00494623072809781\\
318.01	0.00494582372773187\\
319.01	0.00494540922634768\\
320.01	0.00494498708291742\\
321.01	0.00494455715349724\\
322.01	0.00494411929116143\\
323.01	0.00494367334593533\\
324.01	0.00494321916472771\\
325.01	0.00494275659126265\\
326.01	0.00494228546601095\\
327.01	0.00494180562612141\\
328.01	0.00494131690535014\\
329.01	0.00494081913399276\\
330.01	0.00494031213881358\\
331.01	0.00493979574297719\\
332.01	0.00493926976597879\\
333.01	0.00493873402357512\\
334.01	0.00493818832771682\\
335.01	0.00493763248648003\\
336.01	0.00493706630400057\\
337.01	0.00493648958040672\\
338.01	0.00493590211175477\\
339.01	0.0049353036899649\\
340.01	0.00493469410275783\\
341.01	0.00493407313359368\\
342.01	0.00493344056161075\\
343.01	0.00493279616156606\\
344.01	0.00493213970377699\\
345.01	0.00493147095406264\\
346.01	0.00493078967368797\\
347.01	0.00493009561930683\\
348.01	0.00492938854290562\\
349.01	0.00492866819174676\\
350.01	0.00492793430831218\\
351.01	0.00492718663024442\\
352.01	0.00492642489028781\\
353.01	0.00492564881622632\\
354.01	0.00492485813081873\\
355.01	0.00492405255172952\\
356.01	0.00492323179145733\\
357.01	0.00492239555725589\\
358.01	0.00492154355104994\\
359.01	0.00492067546934425\\
360.01	0.00491979100312418\\
361.01	0.00491888983774789\\
362.01	0.00491797165283082\\
363.01	0.00491703612211642\\
364.01	0.00491608291334048\\
365.01	0.00491511168808271\\
366.01	0.00491412210160697\\
367.01	0.00491311380269122\\
368.01	0.00491208643344649\\
369.01	0.00491103962912593\\
370.01	0.00490997301792388\\
371.01	0.0049088862207663\\
372.01	0.00490777885109379\\
373.01	0.00490665051463788\\
374.01	0.0049055008091922\\
375.01	0.00490432932437882\\
376.01	0.00490313564141327\\
377.01	0.0049019193328663\\
378.01	0.00490067996242325\\
379.01	0.00489941708464335\\
380.01	0.00489813024471656\\
381.01	0.00489681897821812\\
382.01	0.00489548281086003\\
383.01	0.00489412125823797\\
384.01	0.00489273382557299\\
385.01	0.0048913200074467\\
386.01	0.00488987928753022\\
387.01	0.00488841113830419\\
388.01	0.00488691502077374\\
389.01	0.00488539038417344\\
390.01	0.00488383666566656\\
391.01	0.00488225329003651\\
392.01	0.00488063966936906\\
393.01	0.00487899520272906\\
394.01	0.00487731927582761\\
395.01	0.00487561126068354\\
396.01	0.00487387051527483\\
397.01	0.0048720963831845\\
398.01	0.00487028819323763\\
399.01	0.00486844525913183\\
400.01	0.00486656687906007\\
401.01	0.00486465233532584\\
402.01	0.0048627008939519\\
403.01	0.00486071180428182\\
404.01	0.00485868429857506\\
405.01	0.0048566175915949\\
406.01	0.00485451088019094\\
407.01	0.00485236334287564\\
408.01	0.00485017413939375\\
409.01	0.00484794241028806\\
410.01	0.00484566727645835\\
411.01	0.00484334783871655\\
412.01	0.00484098317733674\\
413.01	0.0048385723516004\\
414.01	0.00483611439933853\\
415.01	0.00483360833646877\\
416.01	0.00483105315652974\\
417.01	0.00482844783021164\\
418.01	0.0048257913048833\\
419.01	0.00482308250411575\\
420.01	0.00482032032720391\\
421.01	0.00481750364868409\\
422.01	0.00481463131784795\\
423.01	0.00481170215825402\\
424.01	0.00480871496723511\\
425.01	0.0048056685154016\\
426.01	0.0048025615461401\\
427.01	0.00479939277510788\\
428.01	0.00479616088972059\\
429.01	0.00479286454863443\\
430.01	0.00478950238122022\\
431.01	0.0047860729870299\\
432.01	0.0047825749352528\\
433.01	0.00477900676416195\\
434.01	0.00477536698054749\\
435.01	0.00477165405913904\\
436.01	0.00476786644201079\\
437.01	0.00476400253797278\\
438.01	0.0047600607219439\\
439.01	0.0047560393343079\\
440.01	0.00475193668024742\\
441.01	0.00474775102905966\\
442.01	0.0047434806134502\\
443.01	0.00473912362880386\\
444.01	0.00473467823243368\\
445.01	0.00473014254280768\\
446.01	0.00472551463875348\\
447.01	0.00472079255864059\\
448.01	0.0047159742995414\\
449.01	0.00471105781637388\\
450.01	0.00470604102102439\\
451.01	0.00470092178145423\\
452.01	0.00469569792079023\\
453.01	0.00469036721640279\\
454.01	0.00468492739897068\\
455.01	0.00467937615153474\\
456.01	0.00467371110854378\\
457.01	0.00466792985488893\\
458.01	0.00466202992492993\\
459.01	0.00465600880151068\\
460.01	0.00464986391496419\\
461.01	0.00464359264210391\\
462.01	0.0046371923051995\\
463.01	0.00463066017093431\\
464.01	0.0046239934493412\\
465.01	0.00461718929271441\\
466.01	0.00461024479449396\\
467.01	0.00460315698812097\\
468.01	0.00459592284586204\\
469.01	0.00458853927760013\\
470.01	0.00458100312959322\\
471.01	0.00457331118319749\\
472.01	0.00456546015355634\\
473.01	0.0045574466882538\\
474.01	0.00454926736593208\\
475.01	0.00454091869487459\\
476.01	0.00453239711155208\\
477.01	0.00452369897913415\\
478.01	0.0045148205859658\\
479.01	0.00450575814400844\\
480.01	0.00449650778724681\\
481.01	0.00448706557006175\\
482.01	0.00447742746556782\\
483.01	0.00446758936391779\\
484.01	0.00445754707057272\\
485.01	0.00444729630453656\\
486.01	0.00443683269655646\\
487.01	0.00442615178728698\\
488.01	0.00441524902541696\\
489.01	0.00440411976576088\\
490.01	0.00439275926730926\\
491.01	0.0043811626912425\\
492.01	0.00436932509890255\\
493.01	0.00435724144972625\\
494.01	0.00434490659913631\\
495.01	0.00433231529639334\\
496.01	0.00431946218240616\\
497.01	0.00430634178750338\\
498.01	0.00429294852916503\\
499.01	0.00427927670971572\\
500.01	0.00426532051398058\\
501.01	0.0042510740069029\\
502.01	0.0042365311311274\\
503.01	0.00422168570454761\\
504.01	0.00420653141781825\\
505.01	0.00419106183183649\\
506.01	0.00417527037519092\\
507.01	0.00415915034157993\\
508.01	0.0041426948872035\\
509.01	0.00412589702812783\\
510.01	0.00410874963762732\\
511.01	0.00409124544350514\\
512.01	0.00407337702539731\\
513.01	0.00405513681206413\\
514.01	0.00403651707867021\\
515.01	0.00401750994406269\\
516.01	0.00399810736804918\\
517.01	0.00397830114868427\\
518.01	0.00395808291956913\\
519.01	0.003937444147174\\
520.01	0.00391637612819088\\
521.01	0.00389486998692621\\
522.01	0.00387291667274421\\
523.01	0.0038505069575732\\
524.01	0.00382763143348763\\
525.01	0.00380428051038099\\
526.01	0.00378044441374601\\
527.01	0.0037561131825801\\
528.01	0.0037312766674369\\
529.01	0.00370592452864577\\
530.01	0.0036800462347252\\
531.01	0.0036536310610164\\
532.01	0.00362666808856926\\
533.01	0.00359914620331311\\
534.01	0.00357105409555154\\
535.01	0.00354238025982046\\
536.01	0.00351311299515762\\
537.01	0.00348324040583206\\
538.01	0.0034527504025914\\
539.01	0.003421630704487\\
540.01	0.0033898688413456\\
541.01	0.00335745215696268\\
542.01	0.00332436781309979\\
543.01	0.00329060279437702\\
544.01	0.00325614391416088\\
545.01	0.00322097782155683\\
546.01	0.00318509100962871\\
547.01	0.00314846982497814\\
548.01	0.00311110047882896\\
549.01	0.0030729690597782\\
550.01	0.00303406154838843\\
551.01	0.00299436383381546\\
552.01	0.00295386173268076\\
553.01	0.0029125410104198\\
554.01	0.00287038740535795\\
555.01	0.00282738665578867\\
556.01	0.00278352453035239\\
557.01	0.00273878686204152\\
558.01	0.00269315958618417\\
559.01	0.00264662878278779\\
560.01	0.00259918072365662\\
561.01	0.00255080192472444\\
562.01	0.00250147920408223\\
563.01	0.00245119974620713\\
564.01	0.0023999511729366\\
565.01	0.00234772162175995\\
566.01	0.0022944998320303\\
567.01	0.00224027523972262\\
568.01	0.00218503808138668\\
569.01	0.00212877950795229\\
570.01	0.00207149170904279\\
571.01	0.00201316804844078\\
572.01	0.00195380321131022\\
573.01	0.00189339336371602\\
574.01	0.00183193632488497\\
575.01	0.0017694317525057\\
576.01	0.00170588134116366\\
577.01	0.00164128903373116\\
578.01	0.00157566124516161\\
579.01	0.00150900709764865\\
580.01	0.00144133866547398\\
581.01	0.001372671227043\\
582.01	0.00130302352055003\\
583.01	0.00123241799836546\\
584.01	0.00116088107352443\\
585.01	0.00108844334953209\\
586.01	0.00101513982197411\\
587.01	0.000941010036997375\\
588.01	0.000866098187439544\\
589.01	0.000790453122029416\\
590.01	0.000714128236401343\\
591.01	0.000637181206352299\\
592.01	0.000559673513435248\\
593.01	0.000481669700155193\\
594.01	0.000403236276127809\\
595.01	0.00032444017686022\\
596.01	0.000245346652435533\\
597.01	0.000166140574224238\\
598.01	9.1337981519295e-05\\
599.01	2.91271958372443e-05\\
599.02	2.86192783627518e-05\\
599.03	2.81144237420042e-05\\
599.04	2.76126617978888e-05\\
599.05	2.71140226472798e-05\\
599.06	2.66185367039599e-05\\
599.07	2.61262346815533e-05\\
599.08	2.56371475965082e-05\\
599.09	2.51513067710835e-05\\
599.1	2.46687438363886e-05\\
599.11	2.41894907354479e-05\\
599.12	2.37135797262807e-05\\
599.13	2.3241043385025e-05\\
599.14	2.27719146091033e-05\\
599.15	2.23062266203888e-05\\
599.16	2.18440129684284e-05\\
599.17	2.13853075336917e-05\\
599.18	2.09301445308497e-05\\
599.19	2.04785585120812e-05\\
599.2	2.00305843704295e-05\\
599.21	1.95862573431627e-05\\
599.22	1.91456130152028e-05\\
599.23	1.87086873225627e-05\\
599.24	1.82755165558171e-05\\
599.25	1.7846137363638e-05\\
599.26	1.74205874164668e-05\\
599.27	1.69989072017502e-05\\
599.28	1.65811376101922e-05\\
599.29	1.61673199397579e-05\\
599.3	1.57574958996824e-05\\
599.31	1.53517076145714e-05\\
599.32	1.49499976284904e-05\\
599.33	1.45524089091333e-05\\
599.34	1.41589848520092e-05\\
599.35	1.37697692846814e-05\\
599.36	1.33848064710444e-05\\
599.37	1.30041411156422e-05\\
599.38	1.26278183680325e-05\\
599.39	1.22558838271912e-05\\
599.4	1.18883835459674e-05\\
599.41	1.15253640355657e-05\\
599.42	1.11668722700981e-05\\
599.43	1.08129556911519e-05\\
599.44	1.04636622124312e-05\\
599.45	1.01190402244222e-05\\
599.46	9.77913859911625e-06\\
599.47	9.44400669477576e-06\\
599.48	9.11369436074755e-06\\
599.49	8.7882519423238e-06\\
599.5	8.46773028565471e-06\\
599.51	8.15218074270464e-06\\
599.52	7.84165517625675e-06\\
599.53	7.5362059649732e-06\\
599.54	7.23588600849874e-06\\
599.55	6.94074873261973e-06\\
599.56	6.65084809447353e-06\\
599.57	6.36623858780126e-06\\
599.58	6.08697524826993e-06\\
599.59	5.81311365882228e-06\\
599.6	5.54470995511175e-06\\
599.61	5.28182083095637e-06\\
599.62	5.02450354387431e-06\\
599.63	4.7728159206558e-06\\
599.64	4.52681636300446e-06\\
599.65	4.28656385322197e-06\\
599.66	4.05211795995869e-06\\
599.67	3.82353884401457e-06\\
599.68	3.60088726420078e-06\\
599.69	3.38422458325341e-06\\
599.7	3.17361277382168e-06\\
599.71	2.96911442449442e-06\\
599.72	2.7707927458924e-06\\
599.73	2.57871157684567e-06\\
599.74	2.39293539058306e-06\\
599.75	2.21352930102926e-06\\
599.76	2.04055906913997e-06\\
599.77	1.87409110929959e-06\\
599.78	1.71419249580043e-06\\
599.79	1.56093096936177e-06\\
599.8	1.41437494372877e-06\\
599.81	1.27459351233379e-06\\
599.82	1.14165645502366e-06\\
599.83	1.01563424484766e-06\\
599.84	8.96598054920053e-07\\
599.85	7.84619765350353e-07\\
599.86	6.79771970232487e-07\\
599.87	5.82127984719016e-07\\
599.88	4.91761852147374e-07\\
599.89	4.08748351251112e-07\\
599.9	3.33163003438802e-07\\
599.91	2.6508208013365e-07\\
599.92	2.04582610201579e-07\\
599.93	1.51742387452178e-07\\
599.94	1.06639978191686e-07\\
599.95	6.93547288800611e-08\\
599.96	3.99667738487652e-08\\
599.97	1.85570430896731e-08\\
599.98	5.20727013418598e-09\\
599.99	0\\
600	0\\
};
\addplot [color=mycolor1,solid,forget plot]
  table[row sep=crcr]{%
0.01	0.00497879494448985\\
1.01	0.00497879398501761\\
2.01	0.00497879300616703\\
3.01	0.00497879200754837\\
4.01	0.00497879098876431\\
5.01	0.00497878994940958\\
6.01	0.00497878888907078\\
7.01	0.00497878780732623\\
8.01	0.00497878670374596\\
9.01	0.00497878557789113\\
10.01	0.0049787844293143\\
11.01	0.00497878325755954\\
12.01	0.00497878206216091\\
13.01	0.0049787808426441\\
14.01	0.00497877959852446\\
15.01	0.00497877832930862\\
16.01	0.00497877703449259\\
17.01	0.00497877571356219\\
18.01	0.00497877436599405\\
19.01	0.00497877299125341\\
20.01	0.00497877158879522\\
21.01	0.00497877015806326\\
22.01	0.00497876869849081\\
23.01	0.00497876720949955\\
24.01	0.00497876569049932\\
25.01	0.00497876414088867\\
26.01	0.00497876256005391\\
27.01	0.00497876094736925\\
28.01	0.00497875930219646\\
29.01	0.00497875762388458\\
30.01	0.0049787559117697\\
31.01	0.00497875416517459\\
32.01	0.00497875238340865\\
33.01	0.00497875056576746\\
34.01	0.00497874871153282\\
35.01	0.00497874681997224\\
36.01	0.00497874489033824\\
37.01	0.00497874292186881\\
38.01	0.00497874091378695\\
39.01	0.00497873886529973\\
40.01	0.00497873677559916\\
41.01	0.00497873464386062\\
42.01	0.00497873246924343\\
43.01	0.00497873025089011\\
44.01	0.00497872798792607\\
45.01	0.00497872567945961\\
46.01	0.00497872332458104\\
47.01	0.00497872092236274\\
48.01	0.00497871847185914\\
49.01	0.00497871597210512\\
50.01	0.00497871342211686\\
51.01	0.00497871082089129\\
52.01	0.00497870816740492\\
53.01	0.0049787054606144\\
54.01	0.00497870269945564\\
55.01	0.00497869988284319\\
56.01	0.00497869700967044\\
57.01	0.00497869407880886\\
58.01	0.00497869108910736\\
59.01	0.00497868803939233\\
60.01	0.00497868492846677\\
61.01	0.00497868175510988\\
62.01	0.00497867851807716\\
63.01	0.00497867521609914\\
64.01	0.00497867184788143\\
65.01	0.00497866841210389\\
66.01	0.00497866490742053\\
67.01	0.0049786613324587\\
68.01	0.00497865768581844\\
69.01	0.00497865396607265\\
70.01	0.00497865017176542\\
71.01	0.00497864630141282\\
72.01	0.00497864235350094\\
73.01	0.00497863832648619\\
74.01	0.004978634218795\\
75.01	0.00497863002882242\\
76.01	0.00497862575493193\\
77.01	0.00497862139545481\\
78.01	0.00497861694868921\\
79.01	0.00497861241290012\\
80.01	0.00497860778631811\\
81.01	0.00497860306713931\\
82.01	0.00497859825352378\\
83.01	0.00497859334359591\\
84.01	0.00497858833544299\\
85.01	0.00497858322711446\\
86.01	0.00497857801662163\\
87.01	0.00497857270193688\\
88.01	0.00497856728099273\\
89.01	0.00497856175168075\\
90.01	0.00497855611185149\\
91.01	0.00497855035931316\\
92.01	0.00497854449183084\\
93.01	0.00497853850712605\\
94.01	0.00497853240287508\\
95.01	0.00497852617670935\\
96.01	0.00497851982621329\\
97.01	0.00497851334892435\\
98.01	0.00497850674233157\\
99.01	0.0049785000038747\\
100.01	0.00497849313094353\\
101.01	0.00497848612087682\\
102.01	0.00497847897096102\\
103.01	0.00497847167842962\\
104.01	0.00497846424046198\\
105.01	0.00497845665418268\\
106.01	0.00497844891665952\\
107.01	0.00497844102490324\\
108.01	0.0049784329758661\\
109.01	0.00497842476644101\\
110.01	0.00497841639346018\\
111.01	0.00497840785369413\\
112.01	0.00497839914385016\\
113.01	0.00497839026057127\\
114.01	0.0049783812004354\\
115.01	0.00497837195995362\\
116.01	0.0049783625355689\\
117.01	0.00497835292365508\\
118.01	0.00497834312051534\\
119.01	0.00497833312238114\\
120.01	0.00497832292540997\\
121.01	0.00497831252568563\\
122.01	0.0049783019192149\\
123.01	0.00497829110192724\\
124.01	0.00497828006967328\\
125.01	0.00497826881822276\\
126.01	0.00497825734326337\\
127.01	0.00497824564039922\\
128.01	0.00497823370514904\\
129.01	0.00497822153294485\\
130.01	0.00497820911913003\\
131.01	0.00497819645895762\\
132.01	0.00497818354758924\\
133.01	0.00497817038009204\\
134.01	0.004978156951438\\
135.01	0.00497814325650218\\
136.01	0.00497812929006048\\
137.01	0.00497811504678777\\
138.01	0.00497810052125556\\
139.01	0.00497808570793124\\
140.01	0.00497807060117495\\
141.01	0.0049780551952385\\
142.01	0.00497803948426223\\
143.01	0.0049780234622741\\
144.01	0.00497800712318649\\
145.01	0.00497799046079498\\
146.01	0.0049779734687755\\
147.01	0.00497795614068234\\
148.01	0.00497793846994579\\
149.01	0.00497792044986987\\
150.01	0.00497790207363006\\
151.01	0.00497788333427072\\
152.01	0.00497786422470232\\
153.01	0.00497784473770007\\
154.01	0.00497782486590024\\
155.01	0.00497780460179779\\
156.01	0.00497778393774449\\
157.01	0.00497776286594497\\
158.01	0.0049777413784551\\
159.01	0.00497771946717878\\
160.01	0.0049776971238654\\
161.01	0.0049776743401063\\
162.01	0.00497765110733239\\
163.01	0.0049776274168112\\
164.01	0.00497760325964394\\
165.01	0.00497757862676163\\
166.01	0.00497755350892302\\
167.01	0.00497752789671095\\
168.01	0.00497750178052877\\
169.01	0.00497747515059727\\
170.01	0.00497744799695194\\
171.01	0.00497742030943834\\
172.01	0.00497739207771055\\
173.01	0.00497736329122509\\
174.01	0.00497733393923935\\
175.01	0.00497730401080691\\
176.01	0.00497727349477458\\
177.01	0.00497724237977778\\
178.01	0.00497721065423745\\
179.01	0.00497717830635548\\
180.01	0.0049771453241114\\
181.01	0.00497711169525796\\
182.01	0.00497707740731687\\
183.01	0.00497704244757503\\
184.01	0.0049770068030798\\
185.01	0.00497697046063547\\
186.01	0.00497693340679786\\
187.01	0.00497689562787066\\
188.01	0.00497685710990063\\
189.01	0.00497681783867275\\
190.01	0.00497677779970569\\
191.01	0.00497673697824758\\
192.01	0.00497669535927036\\
193.01	0.00497665292746503\\
194.01	0.00497660966723724\\
195.01	0.00497656556270144\\
196.01	0.00497652059767655\\
197.01	0.00497647475567963\\
198.01	0.00497642801992149\\
199.01	0.00497638037330122\\
200.01	0.00497633179839974\\
201.01	0.0049762822774752\\
202.01	0.00497623179245793\\
203.01	0.00497618032494245\\
204.01	0.00497612785618366\\
205.01	0.00497607436709\\
206.01	0.00497601983821763\\
207.01	0.00497596424976436\\
208.01	0.00497590758156299\\
209.01	0.00497584981307584\\
210.01	0.00497579092338769\\
211.01	0.00497573089119943\\
212.01	0.00497566969482138\\
213.01	0.00497560731216686\\
214.01	0.00497554372074528\\
215.01	0.00497547889765502\\
216.01	0.00497541281957636\\
217.01	0.00497534546276491\\
218.01	0.00497527680304393\\
219.01	0.00497520681579736\\
220.01	0.00497513547596213\\
221.01	0.00497506275802024\\
222.01	0.00497498863599211\\
223.01	0.00497491308342803\\
224.01	0.00497483607340115\\
225.01	0.00497475757849792\\
226.01	0.00497467757081176\\
227.01	0.00497459602193391\\
228.01	0.00497451290294537\\
229.01	0.00497442818440835\\
230.01	0.0049743418363581\\
231.01	0.00497425382829347\\
232.01	0.00497416412916902\\
233.01	0.0049740727073859\\
234.01	0.00497397953078169\\
235.01	0.0049738845666235\\
236.01	0.00497378778159678\\
237.01	0.00497368914179664\\
238.01	0.00497358861271796\\
239.01	0.00497348615924673\\
240.01	0.00497338174564912\\
241.01	0.00497327533556214\\
242.01	0.00497316689198343\\
243.01	0.00497305637726126\\
244.01	0.00497294375308392\\
245.01	0.00497282898046944\\
246.01	0.00497271201975529\\
247.01	0.00497259283058733\\
248.01	0.00497247137190871\\
249.01	0.00497234760194902\\
250.01	0.00497222147821346\\
251.01	0.00497209295747108\\
252.01	0.00497196199574329\\
253.01	0.00497182854829224\\
254.01	0.00497169256960917\\
255.01	0.00497155401340257\\
256.01	0.00497141283258575\\
257.01	0.00497126897926411\\
258.01	0.00497112240472345\\
259.01	0.00497097305941658\\
260.01	0.004970820892951\\
261.01	0.00497066585407497\\
262.01	0.00497050789066531\\
263.01	0.00497034694971283\\
264.01	0.00497018297730925\\
265.01	0.00497001591863312\\
266.01	0.00496984571793539\\
267.01	0.00496967231852515\\
268.01	0.00496949566275464\\
269.01	0.00496931569200448\\
270.01	0.00496913234666865\\
271.01	0.00496894556613838\\
272.01	0.00496875528878706\\
273.01	0.00496856145195312\\
274.01	0.00496836399192393\\
275.01	0.00496816284391918\\
276.01	0.00496795794207284\\
277.01	0.00496774921941665\\
278.01	0.00496753660786089\\
279.01	0.00496732003817638\\
280.01	0.00496709943997587\\
281.01	0.00496687474169378\\
282.01	0.00496664587056702\\
283.01	0.0049664127526138\\
284.01	0.00496617531261286\\
285.01	0.00496593347408155\\
286.01	0.00496568715925409\\
287.01	0.00496543628905751\\
288.01	0.00496518078308884\\
289.01	0.00496492055959027\\
290.01	0.00496465553542352\\
291.01	0.00496438562604442\\
292.01	0.00496411074547468\\
293.01	0.0049638308062757\\
294.01	0.00496354571951754\\
295.01	0.00496325539475028\\
296.01	0.00496295973997257\\
297.01	0.00496265866159888\\
298.01	0.0049623520644263\\
299.01	0.00496203985160015\\
300.01	0.00496172192457658\\
301.01	0.00496139818308617\\
302.01	0.0049610685250937\\
303.01	0.00496073284675865\\
304.01	0.00496039104239201\\
305.01	0.00496004300441223\\
306.01	0.00495968862329972\\
307.01	0.00495932778754915\\
308.01	0.00495896038361932\\
309.01	0.00495858629588227\\
310.01	0.00495820540656886\\
311.01	0.00495781759571324\\
312.01	0.0049574227410947\\
313.01	0.00495702071817684\\
314.01	0.00495661140004566\\
315.01	0.00495619465734356\\
316.01	0.00495577035820249\\
317.01	0.00495533836817307\\
318.01	0.004954898550153\\
319.01	0.00495445076431137\\
320.01	0.00495399486801113\\
321.01	0.0049535307157293\\
322.01	0.00495305815897399\\
323.01	0.00495257704619948\\
324.01	0.00495208722271928\\
325.01	0.00495158853061604\\
326.01	0.00495108080864983\\
327.01	0.00495056389216409\\
328.01	0.00495003761299109\\
329.01	0.00494950179935332\\
330.01	0.00494895627576577\\
331.01	0.00494840086293572\\
332.01	0.00494783537766226\\
333.01	0.00494725963273551\\
334.01	0.00494667343683519\\
335.01	0.00494607659443045\\
336.01	0.00494546890567955\\
337.01	0.00494485016633169\\
338.01	0.00494422016763028\\
339.01	0.0049435786962189\\
340.01	0.00494292553405081\\
341.01	0.00494226045830066\\
342.01	0.00494158324128281\\
343.01	0.00494089365037261\\
344.01	0.00494019144793501\\
345.01	0.00493947639125892\\
346.01	0.00493874823249811\\
347.01	0.00493800671862038\\
348.01	0.0049372515913643\\
349.01	0.00493648258720542\\
350.01	0.00493569943732986\\
351.01	0.00493490186761785\\
352.01	0.00493408959863593\\
353.01	0.00493326234563778\\
354.01	0.00493241981857407\\
355.01	0.00493156172210941\\
356.01	0.00493068775564597\\
357.01	0.00492979761335257\\
358.01	0.00492889098419829\\
359.01	0.00492796755198697\\
360.01	0.00492702699539246\\
361.01	0.00492606898799062\\
362.01	0.00492509319828515\\
363.01	0.0049240992897253\\
364.01	0.00492308692071073\\
365.01	0.00492205574458123\\
366.01	0.00492100540958731\\
367.01	0.00491993555883882\\
368.01	0.00491884583022844\\
369.01	0.00491773585632676\\
370.01	0.00491660526424896\\
371.01	0.00491545367549144\\
372.01	0.00491428070573694\\
373.01	0.00491308596463289\\
374.01	0.004911869055541\\
375.01	0.00491062957526776\\
376.01	0.00490936711377468\\
377.01	0.00490808125387745\\
378.01	0.00490677157094167\\
379.01	0.00490543763257748\\
380.01	0.0049040789983421\\
381.01	0.00490269521945232\\
382.01	0.0049012858385085\\
383.01	0.0048998503892306\\
384.01	0.00489838839620154\\
385.01	0.00489689937461228\\
386.01	0.00489538283000254\\
387.01	0.00489383825799477\\
388.01	0.00489226514401598\\
389.01	0.0048906629630096\\
390.01	0.00488903117913716\\
391.01	0.00488736924546662\\
392.01	0.00488567660365066\\
393.01	0.0048839526835923\\
394.01	0.00488219690309812\\
395.01	0.00488040866751922\\
396.01	0.00487858736937886\\
397.01	0.00487673238798788\\
398.01	0.00487484308904662\\
399.01	0.00487291882423342\\
400.01	0.00487095893078005\\
401.01	0.00486896273103315\\
402.01	0.00486692953200372\\
403.01	0.00486485862490164\\
404.01	0.00486274928465794\\
405.01	0.00486060076943418\\
406.01	0.00485841232011875\\
407.01	0.00485618315981099\\
408.01	0.00485391249329367\\
409.01	0.00485159950649396\\
410.01	0.00484924336593332\\
411.01	0.00484684321816763\\
412.01	0.0048443981892178\\
413.01	0.00484190738399237\\
414.01	0.00483936988570152\\
415.01	0.00483678475526597\\
416.01	0.00483415103071904\\
417.01	0.00483146772660592\\
418.01	0.00482873383337868\\
419.01	0.0048259483167901\\
420.01	0.004823110117287\\
421.01	0.00482021814940455\\
422.01	0.00481727130116219\\
423.01	0.00481426843346384\\
424.01	0.00481120837950176\\
425.01	0.00480808994416671\\
426.01	0.00480491190346474\\
427.01	0.00480167300394139\\
428.01	0.00479837196211399\\
429.01	0.00479500746391217\\
430.01	0.00479157816412741\\
431.01	0.00478808268587013\\
432.01	0.0047845196200353\\
433.01	0.00478088752477351\\
434.01	0.00477718492496825\\
435.01	0.00477341031171485\\
436.01	0.004769562141802\\
437.01	0.00476563883718955\\
438.01	0.00476163878448147\\
439.01	0.00475756033438891\\
440.01	0.00475340180118087\\
441.01	0.00474916146211554\\
442.01	0.00474483755684941\\
443.01	0.00474042828681942\\
444.01	0.00473593181459244\\
445.01	0.0047313462631784\\
446.01	0.00472666971530269\\
447.01	0.00472190021263489\\
448.01	0.00471703575497148\\
449.01	0.00471207429937067\\
450.01	0.00470701375923973\\
451.01	0.00470185200337697\\
452.01	0.00469658685497031\\
453.01	0.0046912160905575\\
454.01	0.00468573743895536\\
455.01	0.00468014858016443\\
456.01	0.00467444714425817\\
457.01	0.00466863071026671\\
458.01	0.00466269680506287\\
459.01	0.00465664290226129\\
460.01	0.00465046642113565\\
461.01	0.00464416472556011\\
462.01	0.00463773512297628\\
463.01	0.00463117486338517\\
464.01	0.00462448113835771\\
465.01	0.00461765108005793\\
466.01	0.0046106817602661\\
467.01	0.00460357018939419\\
468.01	0.00459631331547916\\
469.01	0.00458890802314913\\
470.01	0.00458135113255359\\
471.01	0.00457363939825514\\
472.01	0.00456576950807836\\
473.01	0.00455773808191323\\
474.01	0.00454954167047132\\
475.01	0.00454117675399141\\
476.01	0.00453263974089503\\
477.01	0.0045239269663906\\
478.01	0.00451503469102591\\
479.01	0.00450595909919154\\
480.01	0.00449669629757504\\
481.01	0.00448724231356893\\
482.01	0.00447759309363579\\
483.01	0.0044677445016323\\
484.01	0.00445769231709465\\
485.01	0.00444743223348967\\
486.01	0.0044369598564333\\
487.01	0.00442627070187613\\
488.01	0.00441536019425983\\
489.01	0.00440422366464025\\
490.01	0.0043928563487792\\
491.01	0.00438125338520022\\
492.01	0.00436940981320724\\
493.01	0.0043573205708602\\
494.01	0.00434498049290786\\
495.01	0.00433238430867183\\
496.01	0.00431952663988391\\
497.01	0.00430640199847244\\
498.01	0.0042930047843019\\
499.01	0.00427932928286493\\
500.01	0.00426536966292989\\
501.01	0.00425111997414721\\
502.01	0.00423657414461386\\
503.01	0.00422172597840191\\
504.01	0.00420656915305143\\
505.01	0.00419109721702921\\
506.01	0.00417530358715688\\
507.01	0.00415918154600919\\
508.01	0.00414272423928445\\
509.01	0.00412592467314963\\
510.01	0.00410877571156065\\
511.01	0.0040912700735638\\
512.01	0.0040734003305765\\
513.01	0.00405515890365519\\
514.01	0.00403653806075268\\
515.01	0.00401752991397058\\
516.01	0.00399812641681324\\
517.01	0.00397831936144701\\
518.01	0.00395810037597739\\
519.01	0.00393746092174652\\
520.01	0.00391639229066258\\
521.01	0.00389488560257104\\
522.01	0.00387293180267765\\
523.01	0.00385052165903453\\
524.01	0.00382764576010388\\
525.01	0.00380429451241306\\
526.01	0.00378045813831803\\
527.01	0.00375612667389289\\
528.01	0.00373128996696626\\
529.01	0.00370593767532707\\
530.01	0.00368005926512411\\
531.01	0.00365364400948821\\
532.01	0.00362668098740595\\
533.01	0.00359915908288108\\
534.01	0.00357106698441876\\
535.01	0.00354239318487553\\
536.01	0.00351312598172047\\
537.01	0.00348325347775771\\
538.01	0.00345276358236633\\
539.01	0.00342164401331963\\
540.01	0.0033898822992512\\
541.01	0.00335746578284247\\
542.01	0.00332438162481556\\
543.01	0.0032906168088207\\
544.01	0.00325615814731941\\
545.01	0.00322099228857354\\
546.01	0.00318510572486013\\
547.01	0.00314848480204696\\
548.01	0.00311111573067254\\
549.01	0.00307298459869323\\
550.01	0.00303407738607173\\
551.01	0.00299437998139996\\
552.01	0.00295387820076718\\
553.01	0.00291255780910315\\
554.01	0.0028704045442502\\
555.01	0.0028274041440351\\
556.01	0.00278354237664257\\
557.01	0.00273880507461447\\
558.01	0.0026931781728267\\
559.01	0.00264664775082694\\
560.01	0.00259920007994351\\
561.01	0.00255082167561239\\
562.01	0.00250149935539449\\
563.01	0.00245122030319754\\
564.01	0.00239997214024063\\
565.01	0.00234774300333622\\
566.01	0.00229452163109138\\
567.01	0.00224029745865615\\
568.01	0.00218506072166532\\
569.01	0.00212880257003134\\
570.01	0.00207151519224728\\
571.01	0.00201319195084011\\
572.01	0.00195382752958045\\
573.01	0.00189341809299105\\
574.01	0.00183196145859557\\
575.01	0.00176945728220805\\
576.01	0.00170590725635814\\
577.01	0.00164131532167358\\
578.01	0.00157568789066874\\
579.01	0.00150903408290212\\
580.01	0.00144136596982691\\
581.01	0.00137269882683558\\
582.01	0.00130305138894211\\
583.01	0.00123244610519661\\
584.01	0.00116090938521497\\
585.01	0.00108847182904192\\
586.01	0.00101516842884016\\
587.01	0.000941038727476666\\
588.01	0.000866126914790455\\
589.01	0.000790481836971451\\
590.01	0.00071415688780257\\
591.01	0.000637209742203689\\
592.01	0.000559701882183716\\
593.01	0.000481697852480112\\
594.01	0.000403264167260548\\
595.01	0.000324467769563085\\
596.01	0.000245373920771974\\
597.01	0.000166155908349813\\
598.01	9.13379815192916e-05\\
599.01	2.91271958372426e-05\\
599.02	2.86192783627535e-05\\
599.03	2.8114423742006e-05\\
599.04	2.76126617978888e-05\\
599.05	2.71140226472798e-05\\
599.06	2.66185367039581e-05\\
599.07	2.61262346815533e-05\\
599.08	2.56371475965064e-05\\
599.09	2.51513067710818e-05\\
599.1	2.46687438363886e-05\\
599.11	2.41894907354479e-05\\
599.12	2.3713579726279e-05\\
599.13	2.32410433850267e-05\\
599.14	2.27719146091033e-05\\
599.15	2.23062266203888e-05\\
599.16	2.18440129684267e-05\\
599.17	2.13853075336917e-05\\
599.18	2.0930144530848e-05\\
599.19	2.04785585120829e-05\\
599.2	2.00305843704295e-05\\
599.21	1.95862573431627e-05\\
599.22	1.91456130152045e-05\\
599.23	1.87086873225609e-05\\
599.24	1.82755165558171e-05\\
599.25	1.78461373636363e-05\\
599.26	1.74205874164651e-05\\
599.27	1.69989072017485e-05\\
599.28	1.6581137610194e-05\\
599.29	1.61673199397579e-05\\
599.3	1.57574958996841e-05\\
599.31	1.53517076145714e-05\\
599.32	1.49499976284904e-05\\
599.33	1.45524089091333e-05\\
599.34	1.41589848520092e-05\\
599.35	1.37697692846831e-05\\
599.36	1.33848064710462e-05\\
599.37	1.3004141115644e-05\\
599.38	1.26278183680325e-05\\
599.39	1.22558838271929e-05\\
599.4	1.18883835459674e-05\\
599.41	1.15253640355639e-05\\
599.42	1.11668722700964e-05\\
599.43	1.08129556911502e-05\\
599.44	1.04636622124312e-05\\
599.45	1.01190402244222e-05\\
599.46	9.77913859911798e-06\\
599.47	9.44400669477576e-06\\
599.48	9.11369436074581e-06\\
599.49	8.7882519423238e-06\\
599.5	8.46773028565471e-06\\
599.51	8.15218074270464e-06\\
599.52	7.84165517625675e-06\\
599.53	7.5362059649732e-06\\
599.54	7.23588600849874e-06\\
599.55	6.94074873262146e-06\\
599.56	6.65084809447353e-06\\
599.57	6.36623858780126e-06\\
599.58	6.08697524826819e-06\\
599.59	5.81311365882228e-06\\
599.6	5.54470995511175e-06\\
599.61	5.28182083095637e-06\\
599.62	5.02450354387431e-06\\
599.63	4.77281592065407e-06\\
599.64	4.52681636300273e-06\\
599.65	4.28656385322197e-06\\
599.66	4.05211795996042e-06\\
599.67	3.8235388440163e-06\\
599.68	3.60088726420078e-06\\
599.69	3.38422458325514e-06\\
599.7	3.17361277382341e-06\\
599.71	2.96911442449269e-06\\
599.72	2.77079274589413e-06\\
599.73	2.5787115768474e-06\\
599.74	2.3929353905848e-06\\
599.75	2.213529301031e-06\\
599.76	2.04055906913997e-06\\
599.77	1.87409110929959e-06\\
599.78	1.71419249580043e-06\\
599.79	1.56093096936004e-06\\
599.8	1.41437494372704e-06\\
599.81	1.27459351233379e-06\\
599.82	1.14165645502366e-06\\
599.83	1.01563424484766e-06\\
599.84	8.96598054920053e-07\\
599.85	7.84619765348618e-07\\
599.86	6.79771970232487e-07\\
599.87	5.82127984717282e-07\\
599.88	4.91761852145639e-07\\
599.89	4.08748351251112e-07\\
599.9	3.33163003437068e-07\\
599.91	2.65082080131915e-07\\
599.92	2.04582610201579e-07\\
599.93	1.51742387450443e-07\\
599.94	1.06639978189951e-07\\
599.95	6.93547288817958e-08\\
599.96	3.99667738487652e-08\\
599.97	1.85570430896731e-08\\
599.98	5.20727013418598e-09\\
599.99	0\\
600	0\\
};
\addplot [color=mycolor2,solid,forget plot]
  table[row sep=crcr]{%
0.01	0.00499837194810791\\
1.01	0.0049983709500581\\
2.01	0.00499836993172411\\
3.01	0.0049983688926949\\
4.01	0.00499836783255128\\
5.01	0.00499836675086541\\
6.01	0.004998365647201\\
7.01	0.00499836452111279\\
8.01	0.00499836337214647\\
9.01	0.0049983621998389\\
10.01	0.00499836100371725\\
11.01	0.00499835978329928\\
12.01	0.0049983585380931\\
13.01	0.00499835726759685\\
14.01	0.00499835597129857\\
15.01	0.00499835464867581\\
16.01	0.00499835329919573\\
17.01	0.00499835192231491\\
18.01	0.00499835051747875\\
19.01	0.00499834908412148\\
20.01	0.00499834762166594\\
21.01	0.00499834612952357\\
22.01	0.00499834460709351\\
23.01	0.00499834305376323\\
24.01	0.00499834146890773\\
25.01	0.0049983398518892\\
26.01	0.00499833820205734\\
27.01	0.00499833651874839\\
28.01	0.00499833480128571\\
29.01	0.00499833304897846\\
30.01	0.00499833126112235\\
31.01	0.00499832943699859\\
32.01	0.0049983275758743\\
33.01	0.00499832567700145\\
34.01	0.00499832373961732\\
35.01	0.00499832176294358\\
36.01	0.00499831974618637\\
37.01	0.00499831768853598\\
38.01	0.00499831558916598\\
39.01	0.0049983134472338\\
40.01	0.00499831126187974\\
41.01	0.00499830903222683\\
42.01	0.00499830675738047\\
43.01	0.00499830443642806\\
44.01	0.00499830206843873\\
45.01	0.00499829965246298\\
46.01	0.00499829718753196\\
47.01	0.00499829467265758\\
48.01	0.00499829210683183\\
49.01	0.00499828948902669\\
50.01	0.00499828681819303\\
51.01	0.00499828409326108\\
52.01	0.00499828131313945\\
53.01	0.00499827847671471\\
54.01	0.00499827558285147\\
55.01	0.00499827263039104\\
56.01	0.00499826961815198\\
57.01	0.00499826654492874\\
58.01	0.00499826340949194\\
59.01	0.00499826021058731\\
60.01	0.00499825694693557\\
61.01	0.00499825361723174\\
62.01	0.0049982502201447\\
63.01	0.0049982467543167\\
64.01	0.00499824321836263\\
65.01	0.00499823961086985\\
66.01	0.00499823593039719\\
67.01	0.00499823217547466\\
68.01	0.0049982283446032\\
69.01	0.00499822443625329\\
70.01	0.00499822044886503\\
71.01	0.00499821638084697\\
72.01	0.00499821223057625\\
73.01	0.00499820799639745\\
74.01	0.00499820367662184\\
75.01	0.00499819926952695\\
76.01	0.00499819477335602\\
77.01	0.00499819018631703\\
78.01	0.00499818550658229\\
79.01	0.00499818073228734\\
80.01	0.0049981758615306\\
81.01	0.00499817089237242\\
82.01	0.00499816582283464\\
83.01	0.00499816065089932\\
84.01	0.00499815537450828\\
85.01	0.00499814999156225\\
86.01	0.00499814449992018\\
87.01	0.00499813889739805\\
88.01	0.00499813318176858\\
89.01	0.00499812735075984\\
90.01	0.00499812140205455\\
91.01	0.00499811533328925\\
92.01	0.00499810914205335\\
93.01	0.00499810282588823\\
94.01	0.00499809638228636\\
95.01	0.00499808980868995\\
96.01	0.00499808310249072\\
97.01	0.00499807626102795\\
98.01	0.00499806928158813\\
99.01	0.00499806216140391\\
100.01	0.00499805489765244\\
101.01	0.00499804748745494\\
102.01	0.00499803992787549\\
103.01	0.00499803221591948\\
104.01	0.00499802434853285\\
105.01	0.0049980163226006\\
106.01	0.00499800813494621\\
107.01	0.00499799978232974\\
108.01	0.00499799126144694\\
109.01	0.0049979825689279\\
110.01	0.0049979737013357\\
111.01	0.00499796465516526\\
112.01	0.00499795542684211\\
113.01	0.00499794601272043\\
114.01	0.00499793640908229\\
115.01	0.00499792661213579\\
116.01	0.00499791661801425\\
117.01	0.004997906422774\\
118.01	0.00499789602239304\\
119.01	0.00499788541277001\\
120.01	0.0049978745897221\\
121.01	0.00499786354898375\\
122.01	0.00499785228620484\\
123.01	0.00499784079694906\\
124.01	0.00499782907669253\\
125.01	0.00499781712082165\\
126.01	0.00499780492463176\\
127.01	0.00499779248332488\\
128.01	0.00499777979200853\\
129.01	0.00499776684569321\\
130.01	0.00499775363929101\\
131.01	0.00499774016761349\\
132.01	0.00499772642536989\\
133.01	0.00499771240716477\\
134.01	0.00499769810749633\\
135.01	0.00499768352075453\\
136.01	0.00499766864121828\\
137.01	0.00499765346305373\\
138.01	0.00499763798031251\\
139.01	0.00499762218692859\\
140.01	0.00499760607671657\\
141.01	0.00499758964336939\\
142.01	0.00499757288045592\\
143.01	0.00499755578141847\\
144.01	0.00499753833957031\\
145.01	0.00499752054809328\\
146.01	0.00499750240003531\\
147.01	0.00499748388830772\\
148.01	0.00499746500568277\\
149.01	0.00499744574479062\\
150.01	0.00499742609811695\\
151.01	0.00499740605800016\\
152.01	0.0049973856166284\\
153.01	0.00499736476603675\\
154.01	0.00499734349810407\\
155.01	0.00499732180455063\\
156.01	0.00499729967693428\\
157.01	0.004997277106648\\
158.01	0.00499725408491626\\
159.01	0.00499723060279239\\
160.01	0.00499720665115418\\
161.01	0.0049971822207023\\
162.01	0.00499715730195526\\
163.01	0.00499713188524693\\
164.01	0.00499710596072261\\
165.01	0.00499707951833552\\
166.01	0.00499705254784331\\
167.01	0.00499702503880418\\
168.01	0.00499699698057329\\
169.01	0.00499696836229883\\
170.01	0.00499693917291786\\
171.01	0.00499690940115286\\
172.01	0.00499687903550713\\
173.01	0.00499684806426117\\
174.01	0.00499681647546797\\
175.01	0.00499678425694931\\
176.01	0.004996751396291\\
177.01	0.00499671788083855\\
178.01	0.00499668369769278\\
179.01	0.00499664883370536\\
180.01	0.00499661327547373\\
181.01	0.00499657700933676\\
182.01	0.00499654002136978\\
183.01	0.0049965022973798\\
184.01	0.00499646382290037\\
185.01	0.00499642458318667\\
186.01	0.00499638456321026\\
187.01	0.00499634374765393\\
188.01	0.00499630212090629\\
189.01	0.00499625966705638\\
190.01	0.00499621636988827\\
191.01	0.00499617221287514\\
192.01	0.00499612717917412\\
193.01	0.00499608125162022\\
194.01	0.00499603441272022\\
195.01	0.00499598664464723\\
196.01	0.00499593792923409\\
197.01	0.0049958882479677\\
198.01	0.00499583758198227\\
199.01	0.00499578591205326\\
200.01	0.00499573321859105\\
201.01	0.0049956794816339\\
202.01	0.00499562468084163\\
203.01	0.00499556879548895\\
204.01	0.00499551180445827\\
205.01	0.00499545368623272\\
206.01	0.00499539441888925\\
207.01	0.00499533398009169\\
208.01	0.00499527234708256\\
209.01	0.00499520949667663\\
210.01	0.00499514540525276\\
211.01	0.0049950800487466\\
212.01	0.00499501340264259\\
213.01	0.00499494544196635\\
214.01	0.0049948761412763\\
215.01	0.00499480547465623\\
216.01	0.00499473341570651\\
217.01	0.00499465993753581\\
218.01	0.00499458501275311\\
219.01	0.00499450861345861\\
220.01	0.00499443071123545\\
221.01	0.00499435127714077\\
222.01	0.00499427028169681\\
223.01	0.00499418769488206\\
224.01	0.00499410348612157\\
225.01	0.00499401762427875\\
226.01	0.00499393007764475\\
227.01	0.00499384081392993\\
228.01	0.00499374980025395\\
229.01	0.0049936570031358\\
230.01	0.00499356238848447\\
231.01	0.00499346592158848\\
232.01	0.00499336756710633\\
233.01	0.00499326728905578\\
234.01	0.00499316505080432\\
235.01	0.00499306081505766\\
236.01	0.00499295454385023\\
237.01	0.00499284619853421\\
238.01	0.00499273573976864\\
239.01	0.00499262312750883\\
240.01	0.00499250832099512\\
241.01	0.00499239127874211\\
242.01	0.00499227195852745\\
243.01	0.00499215031738026\\
244.01	0.00499202631157023\\
245.01	0.00499189989659601\\
246.01	0.00499177102717331\\
247.01	0.00499163965722368\\
248.01	0.00499150573986267\\
249.01	0.00499136922738777\\
250.01	0.00499123007126674\\
251.01	0.00499108822212551\\
252.01	0.00499094362973598\\
253.01	0.00499079624300416\\
254.01	0.00499064600995752\\
255.01	0.00499049287773259\\
256.01	0.00499033679256298\\
257.01	0.00499017769976685\\
258.01	0.004990015543734\\
259.01	0.00498985026791332\\
260.01	0.0049896818148004\\
261.01	0.00498951012592487\\
262.01	0.00498933514183677\\
263.01	0.00498915680209478\\
264.01	0.00498897504525236\\
265.01	0.00498878980884541\\
266.01	0.00498860102937881\\
267.01	0.00498840864231363\\
268.01	0.00498821258205379\\
269.01	0.00498801278193307\\
270.01	0.00498780917420153\\
271.01	0.00498760169001255\\
272.01	0.00498739025940909\\
273.01	0.00498717481131064\\
274.01	0.00498695527349969\\
275.01	0.0049867315726076\\
276.01	0.00498650363410184\\
277.01	0.00498627138227145\\
278.01	0.00498603474021383\\
279.01	0.00498579362982069\\
280.01	0.00498554797176367\\
281.01	0.00498529768548078\\
282.01	0.00498504268916163\\
283.01	0.00498478289973312\\
284.01	0.00498451823284513\\
285.01	0.00498424860285535\\
286.01	0.00498397392281443\\
287.01	0.00498369410445084\\
288.01	0.00498340905815486\\
289.01	0.00498311869296325\\
290.01	0.00498282291654267\\
291.01	0.00498252163517333\\
292.01	0.00498221475373195\\
293.01	0.00498190217567384\\
294.01	0.00498158380301523\\
295.01	0.00498125953631471\\
296.01	0.00498092927465335\\
297.01	0.00498059291561497\\
298.01	0.00498025035526523\\
299.01	0.00497990148812967\\
300.01	0.00497954620717112\\
301.01	0.00497918440376567\\
302.01	0.0049788159676769\\
303.01	0.00497844078703037\\
304.01	0.00497805874828512\\
305.01	0.00497766973620457\\
306.01	0.00497727363382445\\
307.01	0.00497687032242084\\
308.01	0.0049764596814739\\
309.01	0.00497604158863078\\
310.01	0.00497561591966619\\
311.01	0.00497518254843956\\
312.01	0.00497474134684996\\
313.01	0.0049742921847878\\
314.01	0.00497383493008324\\
315.01	0.00497336944845126\\
316.01	0.00497289560343311\\
317.01	0.00497241325633382\\
318.01	0.00497192226615476\\
319.01	0.00497142248952285\\
320.01	0.00497091378061419\\
321.01	0.00497039599107319\\
322.01	0.00496986896992603\\
323.01	0.0049693325634892\\
324.01	0.0049687866152714\\
325.01	0.00496823096587038\\
326.01	0.00496766545286285\\
327.01	0.0049670899106884\\
328.01	0.00496650417052665\\
329.01	0.00496590806016794\\
330.01	0.00496530140387685\\
331.01	0.004964684022249\\
332.01	0.00496405573206107\\
333.01	0.00496341634611457\\
334.01	0.00496276567307206\\
335.01	0.00496210351728766\\
336.01	0.00496142967863188\\
337.01	0.00496074395231044\\
338.01	0.00496004612867838\\
339.01	0.00495933599305066\\
340.01	0.00495861332550844\\
341.01	0.00495787790070489\\
342.01	0.00495712948766865\\
343.01	0.00495636784960901\\
344.01	0.00495559274372304\\
345.01	0.00495480392100705\\
346.01	0.00495400112607405\\
347.01	0.00495318409698125\\
348.01	0.00495235256506702\\
349.01	0.00495150625480289\\
350.01	0.00495064488366268\\
351.01	0.00494976816201098\\
352.01	0.00494887579301496\\
353.01	0.00494796747258315\\
354.01	0.00494704288933306\\
355.01	0.0049461017245915\\
356.01	0.00494514365243117\\
357.01	0.00494416833974473\\
358.01	0.00494317544635841\\
359.01	0.00494216462518827\\
360.01	0.00494113552243682\\
361.01	0.00494008777783191\\
362.01	0.00493902102490459\\
363.01	0.0049379348913036\\
364.01	0.00493682899914033\\
365.01	0.00493570296535907\\
366.01	0.00493455640212225\\
367.01	0.00493338891720158\\
368.01	0.00493220011436012\\
369.01	0.00493098959371433\\
370.01	0.00492975695205646\\
371.01	0.00492850178312323\\
372.01	0.00492722367779316\\
373.01	0.00492592222419604\\
374.01	0.00492459700772249\\
375.01	0.00492324761092191\\
376.01	0.0049218736132852\\
377.01	0.00492047459091522\\
378.01	0.00491905011609411\\
379.01	0.00491759975676965\\
380.01	0.00491612307598761\\
381.01	0.0049146196313097\\
382.01	0.00491308897425378\\
383.01	0.00491153064979922\\
384.01	0.00490994419598142\\
385.01	0.00490832914358414\\
386.01	0.00490668501590879\\
387.01	0.00490501132857611\\
388.01	0.00490330758932518\\
389.01	0.00490157329779979\\
390.01	0.00489980794532215\\
391.01	0.00489801101465234\\
392.01	0.00489618197973294\\
393.01	0.00489432030541775\\
394.01	0.0048924254471838\\
395.01	0.0048904968508254\\
396.01	0.00488853395213052\\
397.01	0.00488653617653644\\
398.01	0.00488450293876571\\
399.01	0.00488243364244059\\
400.01	0.00488032767967433\\
401.01	0.00487818443064068\\
402.01	0.00487600326311719\\
403.01	0.00487378353200537\\
404.01	0.00487152457882404\\
405.01	0.00486922573117583\\
406.01	0.00486688630218762\\
407.01	0.00486450558992194\\
408.01	0.00486208287676144\\
409.01	0.00485961742876332\\
410.01	0.00485710849498731\\
411.01	0.00485455530679276\\
412.01	0.00485195707710966\\
413.01	0.00484931299967979\\
414.01	0.00484662224827214\\
415.01	0.00484388397587165\\
416.01	0.00484109731384299\\
417.01	0.00483826137107097\\
418.01	0.00483537523307962\\
419.01	0.00483243796113237\\
420.01	0.00482944859131516\\
421.01	0.00482640613360631\\
422.01	0.00482330957093646\\
423.01	0.00482015785824115\\
424.01	0.00481694992151268\\
425.01	0.00481368465685277\\
426.01	0.00481036092953239\\
427.01	0.00480697757306484\\
428.01	0.00480353338829413\\
429.01	0.00480002714250805\\
430.01	0.00479645756857831\\
431.01	0.00479282336413425\\
432.01	0.00478912319077583\\
433.01	0.00478535567332973\\
434.01	0.00478151939915343\\
435.01	0.00477761291749091\\
436.01	0.00477363473888184\\
437.01	0.00476958333462676\\
438.01	0.00476545713630698\\
439.01	0.00476125453535923\\
440.01	0.00475697388269935\\
441.01	0.00475261348839167\\
442.01	0.00474817162135461\\
443.01	0.00474364650909256\\
444.01	0.00473903633744224\\
445.01	0.00473433925031657\\
446.01	0.00472955334943123\\
447.01	0.00472467669399176\\
448.01	0.00471970730032385\\
449.01	0.00471464314142204\\
450.01	0.00470948214639871\\
451.01	0.00470422219981142\\
452.01	0.00469886114085184\\
453.01	0.00469339676238324\\
454.01	0.00468782680981757\\
455.01	0.00468214897983199\\
456.01	0.00467636091893143\\
457.01	0.00467046022187449\\
458.01	0.00466444442998926\\
459.01	0.00465831102941401\\
460.01	0.00465205744930838\\
461.01	0.00464568106008406\\
462.01	0.0046391791717075\\
463.01	0.00463254903212104\\
464.01	0.00462578782582315\\
465.01	0.00461889267262927\\
466.01	0.00461186062661671\\
467.01	0.00460468867523036\\
468.01	0.00459737373851248\\
469.01	0.00458991266840214\\
470.01	0.00458230224805924\\
471.01	0.00457453919118591\\
472.01	0.00456662014132859\\
473.01	0.00455854167114941\\
474.01	0.0045503002816534\\
475.01	0.00454189240135999\\
476.01	0.00453331438540521\\
477.01	0.00452456251456477\\
478.01	0.00451563299418752\\
479.01	0.00450652195303194\\
480.01	0.00449722544200074\\
481.01	0.00448773943277216\\
482.01	0.00447805981632859\\
483.01	0.00446818240138797\\
484.01	0.00445810291274723\\
485.01	0.00444781698954911\\
486.01	0.00443732018348554\\
487.01	0.00442660795695588\\
488.01	0.0044156756811926\\
489.01	0.00440451863437273\\
490.01	0.00439313199972349\\
491.01	0.00438151086363007\\
492.01	0.00436965021374725\\
493.01	0.00435754493710989\\
494.01	0.00434518981822867\\
495.01	0.00433257953715834\\
496.01	0.00431970866751696\\
497.01	0.00430657167444331\\
498.01	0.0042931629124772\\
499.01	0.00427947662335989\\
500.01	0.00426550693375518\\
501.01	0.0042512478528956\\
502.01	0.00423669327016178\\
503.01	0.00422183695260067\\
504.01	0.00420667254239184\\
505.01	0.00419119355427002\\
506.01	0.00417539337290941\\
507.01	0.00415926525027823\\
508.01	0.00414280230296774\\
509.01	0.00412599750949901\\
510.01	0.00410884370761253\\
511.01	0.00409133359153772\\
512.01	0.00407345970924717\\
513.01	0.00405521445969382\\
514.01	0.00403659009003454\\
515.01	0.00401757869284169\\
516.01	0.00399817220330911\\
517.01	0.00397836239645968\\
518.01	0.00395814088436233\\
519.01	0.00393749911337053\\
520.01	0.00391642836139186\\
521.01	0.00389491973519967\\
522.01	0.00387296416779904\\
523.01	0.00385055241585978\\
524.01	0.00382767505722927\\
525.01	0.00380432248853902\\
526.01	0.0037804849229221\\
527.01	0.00375615238785891\\
528.01	0.00373131472317088\\
529.01	0.00370596157918466\\
530.01	0.00368008241509258\\
531.01	0.00365366649753617\\
532.01	0.00362670289944566\\
533.01	0.00359918049916647\\
534.01	0.00357108797991388\\
535.01	0.00354241382959441\\
536.01	0.00351314634104124\\
537.01	0.00348327361271283\\
538.01	0.00345278354991174\\
539.01	0.00342166386658361\\
540.01	0.00338990208776503\\
541.01	0.00335748555275526\\
542.01	0.00332440141909313\\
543.01	0.00329063666743112\\
544.01	0.00325617810740615\\
545.01	0.00322101238461686\\
546.01	0.00318512598882893\\
547.01	0.00314850526353988\\
548.01	0.00311113641705144\\
549.01	0.00307300553520775\\
550.01	0.00303409859597573\\
551.01	0.00299440148606015\\
552.01	0.00295390001976419\\
553.01	0.00291257996032587\\
554.01	0.00287042704398067\\
555.01	0.00282742700702653\\
556.01	0.00278356561618944\\
557.01	0.00273882870261338\\
558.01	0.00269320219982914\\
559.01	0.00264667218608162\\
560.01	0.00259922493143004\\
561.01	0.00255084695006314\\
562.01	0.00250152505830729\\
563.01	0.00245124643883594\\
564.01	0.00239999871162262\\
565.01	0.00234777001221102\\
566.01	0.00229454907790362\\
567.01	0.00224032534249755\\
568.01	0.0021850890402133\\
569.01	0.00212883131947533\\
570.01	0.00207154436720215\\
571.01	0.0020132215442478\\
572.01	0.00195385753260022\\
573.01	0.00189344849487884\\
574.01	0.00183199224657509\\
575.01	0.00176948844133388\\
576.01	0.00170593876937416\\
577.01	0.00164134716886916\\
578.01	0.00157572004973666\\
579.01	0.00150906652880276\\
580.01	0.00144139867466453\\
581.01	0.00137273175975388\\
582.01	0.00130308451604835\\
583.01	0.00123247938952499\\
584.01	0.00116094278674301\\
585.01	0.00108850530477681\\
586.01	0.00101520193299579\\
587.01	0.00094107221176625\\
588.01	0.000866160328865604\\
589.01	0.000790515129045323\\
590.01	0.00071419000550243\\
591.01	0.000637242633708195\\
592.01	0.000559734497711805\\
593.01	0.00048173014621177\\
594.01	0.000403296099787112\\
595.01	0.000324499310983717\\
596.01	0.000245405054576748\\
597.01	0.000166173366806249\\
598.01	9.13379815192916e-05\\
599.01	2.91271958372426e-05\\
599.02	2.86192783627518e-05\\
599.03	2.81144237420042e-05\\
599.04	2.76126617978888e-05\\
599.05	2.71140226472781e-05\\
599.06	2.66185367039599e-05\\
599.07	2.61262346815533e-05\\
599.08	2.56371475965082e-05\\
599.09	2.51513067710818e-05\\
599.1	2.46687438363886e-05\\
599.11	2.41894907354479e-05\\
599.12	2.3713579726279e-05\\
599.13	2.3241043385025e-05\\
599.14	2.2771914609105e-05\\
599.15	2.23062266203871e-05\\
599.16	2.18440129684284e-05\\
599.17	2.138530753369e-05\\
599.18	2.09301445308497e-05\\
599.19	2.04785585120812e-05\\
599.2	2.00305843704278e-05\\
599.21	1.95862573431644e-05\\
599.22	1.91456130152028e-05\\
599.23	1.87086873225609e-05\\
599.24	1.82755165558171e-05\\
599.25	1.7846137363638e-05\\
599.26	1.74205874164651e-05\\
599.27	1.69989072017502e-05\\
599.28	1.6581137610194e-05\\
599.29	1.61673199397579e-05\\
599.3	1.57574958996824e-05\\
599.31	1.53517076145714e-05\\
599.32	1.49499976284887e-05\\
599.33	1.45524089091333e-05\\
599.34	1.41589848520092e-05\\
599.35	1.37697692846831e-05\\
599.36	1.33848064710462e-05\\
599.37	1.30041411156422e-05\\
599.38	1.26278183680325e-05\\
599.39	1.22558838271912e-05\\
599.4	1.18883835459657e-05\\
599.41	1.15253640355657e-05\\
599.42	1.11668722700964e-05\\
599.43	1.08129556911519e-05\\
599.44	1.04636622124312e-05\\
599.45	1.01190402244222e-05\\
599.46	9.77913859911625e-06\\
599.47	9.44400669477576e-06\\
599.48	9.11369436074755e-06\\
599.49	8.78825194232206e-06\\
599.5	8.46773028565471e-06\\
599.51	8.15218074270291e-06\\
599.52	7.84165517625675e-06\\
599.53	7.53620596497147e-06\\
599.54	7.23588600849874e-06\\
599.55	6.94074873261973e-06\\
599.56	6.65084809447353e-06\\
599.57	6.366238587803e-06\\
599.58	6.08697524826819e-06\\
599.59	5.81311365882228e-06\\
599.6	5.54470995511175e-06\\
599.61	5.28182083095637e-06\\
599.62	5.02450354387257e-06\\
599.63	4.7728159206558e-06\\
599.64	4.52681636300446e-06\\
599.65	4.28656385322197e-06\\
599.66	4.05211795995869e-06\\
599.67	3.82353884401457e-06\\
599.68	3.60088726420078e-06\\
599.69	3.38422458325341e-06\\
599.7	3.17361277382168e-06\\
599.71	2.96911442449269e-06\\
599.72	2.77079274589413e-06\\
599.73	2.57871157684567e-06\\
599.74	2.39293539058306e-06\\
599.75	2.21352930102926e-06\\
599.76	2.04055906913823e-06\\
599.77	1.87409110930133e-06\\
599.78	1.71419249580043e-06\\
599.79	1.56093096936177e-06\\
599.8	1.41437494372877e-06\\
599.81	1.27459351233379e-06\\
599.82	1.14165645502366e-06\\
599.83	1.01563424484592e-06\\
599.84	8.96598054920053e-07\\
599.85	7.84619765350353e-07\\
599.86	6.79771970232487e-07\\
599.87	5.82127984719016e-07\\
599.88	4.91761852147374e-07\\
599.89	4.08748351251112e-07\\
599.9	3.33163003438802e-07\\
599.91	2.65082080131915e-07\\
599.92	2.04582610201579e-07\\
599.93	1.51742387452178e-07\\
599.94	1.06639978191686e-07\\
599.95	6.93547288800611e-08\\
599.96	3.99667738487652e-08\\
599.97	1.85570430896731e-08\\
599.98	5.20727013245126e-09\\
599.99	0\\
600	0\\
};
\addplot [color=mycolor3,solid,forget plot]
  table[row sep=crcr]{%
0.01	0.00503244980582076\\
1.01	0.0050324487594527\\
2.01	0.00503244769167412\\
3.01	0.00503244660204765\\
4.01	0.0050324454901266\\
5.01	0.00503244435545586\\
6.01	0.00503244319757019\\
7.01	0.00503244201599557\\
8.01	0.00503244081024797\\
9.01	0.00503243957983354\\
10.01	0.00503243832424839\\
11.01	0.00503243704297841\\
12.01	0.00503243573549884\\
13.01	0.00503243440127439\\
14.01	0.0050324330397588\\
15.01	0.00503243165039465\\
16.01	0.00503243023261314\\
17.01	0.00503242878583388\\
18.01	0.0050324273094648\\
19.01	0.00503242580290154\\
20.01	0.00503242426552755\\
21.01	0.00503242269671365\\
22.01	0.00503242109581802\\
23.01	0.00503241946218543\\
24.01	0.00503241779514747\\
25.01	0.0050324160940225\\
26.01	0.00503241435811418\\
27.01	0.0050324125867128\\
28.01	0.00503241077909349\\
29.01	0.00503240893451709\\
30.01	0.0050324070522293\\
31.01	0.00503240513146033\\
32.01	0.00503240317142475\\
33.01	0.00503240117132111\\
34.01	0.00503239913033189\\
35.01	0.00503239704762252\\
36.01	0.00503239492234174\\
37.01	0.00503239275362075\\
38.01	0.00503239054057329\\
39.01	0.00503238828229486\\
40.01	0.0050323859778628\\
41.01	0.00503238362633529\\
42.01	0.00503238122675157\\
43.01	0.00503237877813141\\
44.01	0.00503237627947449\\
45.01	0.00503237372976\\
46.01	0.00503237112794673\\
47.01	0.00503236847297204\\
48.01	0.00503236576375179\\
49.01	0.00503236299917937\\
50.01	0.00503236017812644\\
51.01	0.00503235729944098\\
52.01	0.0050323543619478\\
53.01	0.00503235136444798\\
54.01	0.00503234830571803\\
55.01	0.00503234518450956\\
56.01	0.0050323419995488\\
57.01	0.00503233874953634\\
58.01	0.00503233543314596\\
59.01	0.00503233204902479\\
60.01	0.0050323285957922\\
61.01	0.00503232507203969\\
62.01	0.00503232147633013\\
63.01	0.00503231780719697\\
64.01	0.0050323140631442\\
65.01	0.00503231024264517\\
66.01	0.00503230634414228\\
67.01	0.00503230236604648\\
68.01	0.00503229830673582\\
69.01	0.00503229416455597\\
70.01	0.00503228993781898\\
71.01	0.00503228562480235\\
72.01	0.00503228122374879\\
73.01	0.00503227673286538\\
74.01	0.00503227215032263\\
75.01	0.00503226747425396\\
76.01	0.00503226270275486\\
77.01	0.00503225783388235\\
78.01	0.00503225286565367\\
79.01	0.00503224779604628\\
80.01	0.00503224262299631\\
81.01	0.00503223734439799\\
82.01	0.00503223195810295\\
83.01	0.00503222646191911\\
84.01	0.00503222085361007\\
85.01	0.00503221513089406\\
86.01	0.00503220929144289\\
87.01	0.00503220333288127\\
88.01	0.00503219725278569\\
89.01	0.0050321910486836\\
90.01	0.00503218471805234\\
91.01	0.00503217825831803\\
92.01	0.00503217166685472\\
93.01	0.00503216494098317\\
94.01	0.00503215807796969\\
95.01	0.00503215107502565\\
96.01	0.00503214392930557\\
97.01	0.00503213663790648\\
98.01	0.00503212919786658\\
99.01	0.00503212160616396\\
100.01	0.00503211385971588\\
101.01	0.00503210595537693\\
102.01	0.00503209788993803\\
103.01	0.00503208966012522\\
104.01	0.00503208126259835\\
105.01	0.00503207269394952\\
106.01	0.00503206395070202\\
107.01	0.00503205502930895\\
108.01	0.0050320459261513\\
109.01	0.00503203663753715\\
110.01	0.00503202715969984\\
111.01	0.00503201748879668\\
112.01	0.00503200762090712\\
113.01	0.00503199755203162\\
114.01	0.00503198727808951\\
115.01	0.00503197679491799\\
116.01	0.00503196609826996\\
117.01	0.00503195518381265\\
118.01	0.00503194404712566\\
119.01	0.00503193268369961\\
120.01	0.00503192108893388\\
121.01	0.00503190925813501\\
122.01	0.00503189718651492\\
123.01	0.00503188486918881\\
124.01	0.00503187230117348\\
125.01	0.00503185947738504\\
126.01	0.00503184639263723\\
127.01	0.00503183304163921\\
128.01	0.00503181941899344\\
129.01	0.00503180551919353\\
130.01	0.00503179133662255\\
131.01	0.00503177686554999\\
132.01	0.00503176210012995\\
133.01	0.00503174703439897\\
134.01	0.00503173166227358\\
135.01	0.00503171597754753\\
136.01	0.0050316999738901\\
137.01	0.00503168364484295\\
138.01	0.00503166698381785\\
139.01	0.00503164998409419\\
140.01	0.00503163263881631\\
141.01	0.00503161494099051\\
142.01	0.00503159688348273\\
143.01	0.00503157845901536\\
144.01	0.00503155966016506\\
145.01	0.00503154047935915\\
146.01	0.00503152090887261\\
147.01	0.00503150094082613\\
148.01	0.00503148056718165\\
149.01	0.00503145977974037\\
150.01	0.00503143857013865\\
151.01	0.00503141692984551\\
152.01	0.00503139485015892\\
153.01	0.00503137232220257\\
154.01	0.00503134933692227\\
155.01	0.00503132588508246\\
156.01	0.005031301957263\\
157.01	0.00503127754385505\\
158.01	0.00503125263505771\\
159.01	0.00503122722087381\\
160.01	0.00503120129110709\\
161.01	0.00503117483535663\\
162.01	0.00503114784301425\\
163.01	0.00503112030326\\
164.01	0.00503109220505783\\
165.01	0.00503106353715156\\
166.01	0.00503103428806052\\
167.01	0.005031004446075\\
168.01	0.00503097399925205\\
169.01	0.00503094293541084\\
170.01	0.00503091124212806\\
171.01	0.00503087890673309\\
172.01	0.00503084591630309\\
173.01	0.00503081225765854\\
174.01	0.00503077791735762\\
175.01	0.00503074288169191\\
176.01	0.00503070713668027\\
177.01	0.00503067066806456\\
178.01	0.00503063346130337\\
179.01	0.00503059550156716\\
180.01	0.00503055677373244\\
181.01	0.00503051726237645\\
182.01	0.00503047695177069\\
183.01	0.00503043582587579\\
184.01	0.00503039386833516\\
185.01	0.00503035106246887\\
186.01	0.00503030739126768\\
187.01	0.00503026283738648\\
188.01	0.00503021738313796\\
189.01	0.00503017101048648\\
190.01	0.0050301237010407\\
191.01	0.00503007543604742\\
192.01	0.00503002619638435\\
193.01	0.00502997596255327\\
194.01	0.00502992471467295\\
195.01	0.00502987243247175\\
196.01	0.00502981909528031\\
197.01	0.00502976468202424\\
198.01	0.00502970917121621\\
199.01	0.00502965254094836\\
200.01	0.00502959476888455\\
201.01	0.00502953583225201\\
202.01	0.00502947570783336\\
203.01	0.00502941437195841\\
204.01	0.00502935180049576\\
205.01	0.00502928796884394\\
206.01	0.00502922285192303\\
207.01	0.00502915642416559\\
208.01	0.00502908865950821\\
209.01	0.00502901953138143\\
210.01	0.00502894901270135\\
211.01	0.00502887707585973\\
212.01	0.00502880369271473\\
213.01	0.00502872883458092\\
214.01	0.00502865247221946\\
215.01	0.00502857457582826\\
216.01	0.00502849511503165\\
217.01	0.0050284140588702\\
218.01	0.00502833137579014\\
219.01	0.00502824703363277\\
220.01	0.00502816099962352\\
221.01	0.00502807324036168\\
222.01	0.00502798372180807\\
223.01	0.00502789240927511\\
224.01	0.00502779926741474\\
225.01	0.00502770426020653\\
226.01	0.00502760735094691\\
227.01	0.00502750850223669\\
228.01	0.00502740767596873\\
229.01	0.00502730483331666\\
230.01	0.00502719993472168\\
231.01	0.00502709293988079\\
232.01	0.00502698380773348\\
233.01	0.00502687249644932\\
234.01	0.00502675896341495\\
235.01	0.00502664316522123\\
236.01	0.00502652505764949\\
237.01	0.00502640459565847\\
238.01	0.00502628173337099\\
239.01	0.00502615642405963\\
240.01	0.00502602862013372\\
241.01	0.00502589827312502\\
242.01	0.0050257653336735\\
243.01	0.00502562975151353\\
244.01	0.00502549147545935\\
245.01	0.00502535045339068\\
246.01	0.00502520663223826\\
247.01	0.00502505995796914\\
248.01	0.00502491037557206\\
249.01	0.00502475782904218\\
250.01	0.00502460226136691\\
251.01	0.00502444361450998\\
252.01	0.00502428182939741\\
253.01	0.00502411684590135\\
254.01	0.00502394860282549\\
255.01	0.00502377703788978\\
256.01	0.00502360208771492\\
257.01	0.00502342368780687\\
258.01	0.00502324177254212\\
259.01	0.00502305627515195\\
260.01	0.00502286712770687\\
261.01	0.00502267426110168\\
262.01	0.00502247760503997\\
263.01	0.00502227708801895\\
264.01	0.0050220726373142\\
265.01	0.00502186417896444\\
266.01	0.00502165163775659\\
267.01	0.00502143493721086\\
268.01	0.00502121399956596\\
269.01	0.00502098874576417\\
270.01	0.00502075909543693\\
271.01	0.00502052496689031\\
272.01	0.00502028627709141\\
273.01	0.00502004294165361\\
274.01	0.00501979487482338\\
275.01	0.00501954198946656\\
276.01	0.00501928419705517\\
277.01	0.00501902140765472\\
278.01	0.00501875352991081\\
279.01	0.00501848047103786\\
280.01	0.00501820213680684\\
281.01	0.00501791843153322\\
282.01	0.00501762925806636\\
283.01	0.00501733451777887\\
284.01	0.00501703411055576\\
285.01	0.00501672793478502\\
286.01	0.00501641588734835\\
287.01	0.00501609786361165\\
288.01	0.00501577375741771\\
289.01	0.0050154434610775\\
290.01	0.00501510686536327\\
291.01	0.0050147638595018\\
292.01	0.00501441433116787\\
293.01	0.00501405816647933\\
294.01	0.00501369524999145\\
295.01	0.0050133254646924\\
296.01	0.00501294869199959\\
297.01	0.00501256481175614\\
298.01	0.00501217370222772\\
299.01	0.00501177524010011\\
300.01	0.00501136930047735\\
301.01	0.00501095575687963\\
302.01	0.00501053448124273\\
303.01	0.00501010534391622\\
304.01	0.00500966821366288\\
305.01	0.0050092229576575\\
306.01	0.00500876944148657\\
307.01	0.00500830752914664\\
308.01	0.00500783708304336\\
309.01	0.00500735796398983\\
310.01	0.00500687003120323\\
311.01	0.00500637314230198\\
312.01	0.0050058671533011\\
313.01	0.0050053519186062\\
314.01	0.00500482729100569\\
315.01	0.00500429312166118\\
316.01	0.00500374926009473\\
317.01	0.00500319555417405\\
318.01	0.00500263185009457\\
319.01	0.00500205799235635\\
320.01	0.00500147382373797\\
321.01	0.0050008791852649\\
322.01	0.00500027391617245\\
323.01	0.00499965785386183\\
324.01	0.00499903083385003\\
325.01	0.00499839268971098\\
326.01	0.00499774325300887\\
327.01	0.0049970823532206\\
328.01	0.00499640981764838\\
329.01	0.00499572547132064\\
330.01	0.00499502913687951\\
331.01	0.00499432063445481\\
332.01	0.00499359978152308\\
333.01	0.0049928663927494\\
334.01	0.00499212027981295\\
335.01	0.00499136125121264\\
336.01	0.00499058911205303\\
337.01	0.00498980366380923\\
338.01	0.00498900470406936\\
339.01	0.00498819202625336\\
340.01	0.0049873654193082\\
341.01	0.00498652466737679\\
342.01	0.00498566954944258\\
343.01	0.00498479983894775\\
344.01	0.00498391530338564\\
345.01	0.00498301570386769\\
346.01	0.00498210079466709\\
347.01	0.00498117032273863\\
348.01	0.00498022402721888\\
349.01	0.00497926163890821\\
350.01	0.00497828287974025\\
351.01	0.00497728746224179\\
352.01	0.00497627508899112\\
353.01	0.00497524545208004\\
354.01	0.00497419823259\\
355.01	0.00497313310009095\\
356.01	0.00497204971217474\\
357.01	0.00497094771403626\\
358.01	0.004969826738117\\
359.01	0.0049686864038248\\
360.01	0.00496752631734916\\
361.01	0.00496634607158831\\
362.01	0.00496514524620618\\
363.01	0.00496392340783756\\
364.01	0.00496268011045868\\
365.01	0.00496141489593708\\
366.01	0.0049601272947751\\
367.01	0.00495881682705102\\
368.01	0.00495748300356199\\
369.01	0.00495612532715941\\
370.01	0.00495474329426123\\
371.01	0.00495333639651\\
372.01	0.00495190412253458\\
373.01	0.00495044595975613\\
374.01	0.00494896139616314\\
375.01	0.00494744992196539\\
376.01	0.0049459110310247\\
377.01	0.0049443442219518\\
378.01	0.00494274899876115\\
379.01	0.0049411248709896\\
380.01	0.0049394713532182\\
381.01	0.00493778796398858\\
382.01	0.00493607422418693\\
383.01	0.00493432965506405\\
384.01	0.00493255377617798\\
385.01	0.00493074610363249\\
386.01	0.00492890614899262\\
387.01	0.00492703341894679\\
388.01	0.0049251274152659\\
389.01	0.00492318763478241\\
390.01	0.00492121356935998\\
391.01	0.00491920470585568\\
392.01	0.00491716052607117\\
393.01	0.0049150805066926\\
394.01	0.00491296411921761\\
395.01	0.00491081082986713\\
396.01	0.00490862009948096\\
397.01	0.0049063913833954\\
398.01	0.0049041241312997\\
399.01	0.00490181778707027\\
400.01	0.00489947178858096\\
401.01	0.00489708556748454\\
402.01	0.004894658548966\\
403.01	0.00489219015146211\\
404.01	0.00488967978634674\\
405.01	0.00488712685757793\\
406.01	0.00488453076130294\\
407.01	0.00488189088542068\\
408.01	0.00487920660909488\\
409.01	0.00487647730221867\\
410.01	0.0048737023248231\\
411.01	0.00487088102643092\\
412.01	0.00486801274534842\\
413.01	0.00486509680789425\\
414.01	0.00486213252756273\\
415.01	0.00485911920411732\\
416.01	0.00485605612261326\\
417.01	0.00485294255234568\\
418.01	0.00484977774572384\\
419.01	0.00484656093706768\\
420.01	0.00484329134132815\\
421.01	0.00483996815273009\\
422.01	0.0048365905433399\\
423.01	0.00483315766155961\\
424.01	0.00482966863054951\\
425.01	0.0048261225465857\\
426.01	0.00482251847735665\\
427.01	0.00481885546020605\\
428.01	0.00481513250033278\\
429.01	0.00481134856895661\\
430.01	0.00480750260146232\\
431.01	0.0048035934955387\\
432.01	0.00479962010932603\\
433.01	0.00479558125959288\\
434.01	0.00479147571996046\\
435.01	0.00478730221919752\\
436.01	0.00478305943960904\\
437.01	0.00477874601554179\\
438.01	0.0047743605320341\\
439.01	0.00476990152363236\\
440.01	0.00476536747339961\\
441.01	0.00476075681213895\\
442.01	0.00475606791784981\\
443.01	0.00475129911543435\\
444.01	0.0047464486766616\\
445.01	0.00474151482039364\\
446.01	0.00473649571306537\\
447.01	0.00473138946940354\\
448.01	0.00472619415335193\\
449.01	0.00472090777916279\\
450.01	0.00471552831259486\\
451.01	0.00471005367214465\\
452.01	0.00470448173022477\\
453.01	0.00469881031418478\\
454.01	0.00469303720706679\\
455.01	0.00468716014797469\\
456.01	0.00468117683194328\\
457.01	0.0046750849091994\\
458.01	0.0046688819837303\\
459.01	0.00466256561110685\\
460.01	0.00465613329555711\\
461.01	0.00464958248634543\\
462.01	0.00464291057358258\\
463.01	0.00463611488366905\\
464.01	0.00462919267463919\\
465.01	0.00462214113172537\\
466.01	0.00461495736346909\\
467.01	0.00460763839865414\\
468.01	0.00460018118420783\\
469.01	0.00459258258401121\\
470.01	0.0045848393783379\\
471.01	0.00457694826359691\\
472.01	0.00456890585225387\\
473.01	0.00456070867290062\\
474.01	0.00455235317045032\\
475.01	0.00454383570642624\\
476.01	0.00453515255930865\\
477.01	0.00452629992489533\\
478.01	0.00451727391663021\\
479.01	0.00450807056584773\\
480.01	0.00449868582188116\\
481.01	0.00448911555198336\\
482.01	0.00447935554101247\\
483.01	0.00446940149084283\\
484.01	0.00445924901947154\\
485.01	0.00444889365980655\\
486.01	0.00443833085813825\\
487.01	0.00442755597231973\\
488.01	0.00441656426969869\\
489.01	0.00440535092486604\\
490.01	0.00439391101730345\\
491.01	0.0043822395290205\\
492.01	0.00437033134227339\\
493.01	0.00435818123744439\\
494.01	0.00434578389113603\\
495.01	0.004333133874493\\
496.01	0.004320225651719\\
497.01	0.00430705357870746\\
498.01	0.00429361190167717\\
499.01	0.00427989475570422\\
500.01	0.00426589616308109\\
501.01	0.00425161003148202\\
502.01	0.00423703015193442\\
503.01	0.00422215019660487\\
504.01	0.00420696371641547\\
505.01	0.0041914641385129\\
506.01	0.00417564476361897\\
507.01	0.00415949876329163\\
508.01	0.00414301917712979\\
509.01	0.00412619890995134\\
510.01	0.0041090307289686\\
511.01	0.00409150726097823\\
512.01	0.00407362098957318\\
513.01	0.00405536425237386\\
514.01	0.00403672923826686\\
515.01	0.00401770798463586\\
516.01	0.00399829237457026\\
517.01	0.00397847413404289\\
518.01	0.00395824482906013\\
519.01	0.00393759586279751\\
520.01	0.00391651847274203\\
521.01	0.00389500372786058\\
522.01	0.00387304252581928\\
523.01	0.00385062559026892\\
524.01	0.00382774346821949\\
525.01	0.00380438652751846\\
526.01	0.00378054495444964\\
527.01	0.00375620875146918\\
528.01	0.00373136773509426\\
529.01	0.00370601153396574\\
530.01	0.00368012958710549\\
531.01	0.00365371114239777\\
532.01	0.00362674525532476\\
533.01	0.00359922078799371\\
534.01	0.00357112640849502\\
535.01	0.00354245059063457\\
536.01	0.00351318161408584\\
537.01	0.00348330756501385\\
538.01	0.00345281633722467\\
539.01	0.00342169563390136\\
540.01	0.00338993296999411\\
541.01	0.00335751567533823\\
542.01	0.00332443089858232\\
543.01	0.00329066561201765\\
544.01	0.00325620661740762\\
545.01	0.00322104055292884\\
546.01	0.00318515390134406\\
547.01	0.00314853299953866\\
548.01	0.00311116404956736\\
549.01	0.00307303313137023\\
550.01	0.00303412621733353\\
551.01	0.00299442918888706\\
552.01	0.00295392785534823\\
553.01	0.00291260797524378\\
554.01	0.00287045528035889\\
555.01	0.00282745550278978\\
556.01	0.00278359440529631\\
557.01	0.00273885781528004\\
558.01	0.00269323166274068\\
559.01	0.00264670202259108\\
560.01	0.00259925516174355\\
561.01	0.0025508775914118\\
562.01	0.00250155612510365\\
563.01	0.00245127794281489\\
564.01	0.00240003066196555\\
565.01	0.00234780241565152\\
566.01	0.00229458193881511\\
567.01	0.00224035866296003\\
568.01	0.00218512282006012\\
569.01	0.00212886555631837\\
570.01	0.00207157905643533\\
571.01	0.00201325667902841\\
572.01	0.00195389310380886\\
573.01	0.00189348449105829\\
574.01	0.00183202865384843\\
575.01	0.00176952524330439\\
576.01	0.00170597594700881\\
577.01	0.00164138470036774\\
578.01	0.00157575791039109\\
579.01	0.00150910469085067\\
580.01	0.00144143710714304\\
581.01	0.00137277042836048\\
582.01	0.00130312338301832\\
583.01	0.00123251841353655\\
584.01	0.00116098192286425\\
585.01	0.00108854450447237\\
586.01	0.00101524114421543\\
587.01	0.000941111379143577\\
588.01	0.000866199394062414\\
589.01	0.000790554031285839\\
590.01	0.000714228682354051\\
591.01	0.000637281022179434\\
592.01	0.000559772535756788\\
593.01	0.000481767774754076\\
594.01	0.000403333265405191\\
595.01	0.00032453596943738\\
596.01	0.000245441175401659\\
597.01	0.000166193599565361\\
598.01	9.1337981519295e-05\\
599.01	2.91271958372443e-05\\
599.02	2.86192783627518e-05\\
599.03	2.81144237420042e-05\\
599.04	2.76126617978888e-05\\
599.05	2.71140226472798e-05\\
599.06	2.66185367039599e-05\\
599.07	2.61262346815533e-05\\
599.08	2.56371475965064e-05\\
599.09	2.51513067710818e-05\\
599.1	2.46687438363886e-05\\
599.11	2.41894907354479e-05\\
599.12	2.37135797262807e-05\\
599.13	2.32410433850267e-05\\
599.14	2.27719146091033e-05\\
599.15	2.23062266203888e-05\\
599.16	2.18440129684267e-05\\
599.17	2.13853075336917e-05\\
599.18	2.0930144530848e-05\\
599.19	2.04785585120812e-05\\
599.2	2.00305843704295e-05\\
599.21	1.95862573431627e-05\\
599.22	1.91456130152045e-05\\
599.23	1.87086873225609e-05\\
599.24	1.82755165558188e-05\\
599.25	1.7846137363638e-05\\
599.26	1.74205874164651e-05\\
599.27	1.69989072017485e-05\\
599.28	1.6581137610194e-05\\
599.29	1.61673199397562e-05\\
599.3	1.57574958996841e-05\\
599.31	1.53517076145696e-05\\
599.32	1.49499976284904e-05\\
599.33	1.45524089091333e-05\\
599.34	1.41589848520109e-05\\
599.35	1.37697692846831e-05\\
599.36	1.33848064710444e-05\\
599.37	1.30041411156422e-05\\
599.38	1.26278183680325e-05\\
599.39	1.22558838271912e-05\\
599.4	1.18883835459674e-05\\
599.41	1.15253640355657e-05\\
599.42	1.11668722700964e-05\\
599.43	1.08129556911519e-05\\
599.44	1.04636622124312e-05\\
599.45	1.01190402244222e-05\\
599.46	9.77913859911625e-06\\
599.47	9.44400669477576e-06\\
599.48	9.11369436074581e-06\\
599.49	8.7882519423238e-06\\
599.5	8.46773028565471e-06\\
599.51	8.15218074270464e-06\\
599.52	7.84165517625675e-06\\
599.53	7.5362059649732e-06\\
599.54	7.23588600849874e-06\\
599.55	6.94074873261973e-06\\
599.56	6.65084809447353e-06\\
599.57	6.36623858780126e-06\\
599.58	6.08697524826993e-06\\
599.59	5.81311365882228e-06\\
599.6	5.54470995511175e-06\\
599.61	5.28182083095637e-06\\
599.62	5.02450354387431e-06\\
599.63	4.7728159206558e-06\\
599.64	4.52681636300273e-06\\
599.65	4.28656385322197e-06\\
599.66	4.05211795995869e-06\\
599.67	3.8235388440163e-06\\
599.68	3.60088726420078e-06\\
599.69	3.38422458325341e-06\\
599.7	3.17361277382168e-06\\
599.71	2.96911442449269e-06\\
599.72	2.77079274589413e-06\\
599.73	2.57871157684567e-06\\
599.74	2.3929353905848e-06\\
599.75	2.213529301031e-06\\
599.76	2.04055906913997e-06\\
599.77	1.87409110929959e-06\\
599.78	1.71419249580043e-06\\
599.79	1.56093096936177e-06\\
599.8	1.41437494372877e-06\\
599.81	1.27459351233379e-06\\
599.82	1.14165645502366e-06\\
599.83	1.01563424484766e-06\\
599.84	8.96598054920053e-07\\
599.85	7.84619765350353e-07\\
599.86	6.79771970232487e-07\\
599.87	5.82127984717282e-07\\
599.88	4.91761852145639e-07\\
599.89	4.08748351251112e-07\\
599.9	3.33163003437068e-07\\
599.91	2.65082080131915e-07\\
599.92	2.04582610201579e-07\\
599.93	1.51742387450443e-07\\
599.94	1.06639978189951e-07\\
599.95	6.93547288800611e-08\\
599.96	3.99667738487652e-08\\
599.97	1.85570430896731e-08\\
599.98	5.20727013418598e-09\\
599.99	0\\
600	0\\
};
\addplot [color=mycolor4,solid,forget plot]
  table[row sep=crcr]{%
0.01	0.00509094250028793\\
1.01	0.0050909413980216\\
2.01	0.00509094027305127\\
3.01	0.00509093912490907\\
4.01	0.00509093795311797\\
5.01	0.00509093675719045\\
6.01	0.00509093553662952\\
7.01	0.00509093429092756\\
8.01	0.00509093301956685\\
9.01	0.00509093172201875\\
10.01	0.00509093039774377\\
11.01	0.00509092904619125\\
12.01	0.00509092766679937\\
13.01	0.00509092625899451\\
14.01	0.00509092482219151\\
15.01	0.00509092335579298\\
16.01	0.00509092185918925\\
17.01	0.00509092033175811\\
18.01	0.00509091877286421\\
19.01	0.00509091718185979\\
20.01	0.0050909155580832\\
21.01	0.00509091390085927\\
22.01	0.00509091220949911\\
23.01	0.00509091048329933\\
24.01	0.00509090872154248\\
25.01	0.00509090692349573\\
26.01	0.00509090508841182\\
27.01	0.00509090321552763\\
28.01	0.00509090130406455\\
29.01	0.00509089935322777\\
30.01	0.0050908973622061\\
31.01	0.0050908953301719\\
32.01	0.00509089325628013\\
33.01	0.00509089113966873\\
34.01	0.00509088897945757\\
35.01	0.00509088677474857\\
36.01	0.00509088452462498\\
37.01	0.00509088222815119\\
38.01	0.00509087988437259\\
39.01	0.00509087749231455\\
40.01	0.00509087505098255\\
41.01	0.00509087255936174\\
42.01	0.00509087001641603\\
43.01	0.00509086742108819\\
44.01	0.00509086477229912\\
45.01	0.00509086206894771\\
46.01	0.00509085930990987\\
47.01	0.00509085649403853\\
48.01	0.00509085362016291\\
49.01	0.00509085068708832\\
50.01	0.00509084769359505\\
51.01	0.00509084463843879\\
52.01	0.00509084152034938\\
53.01	0.00509083833803038\\
54.01	0.0050908350901587\\
55.01	0.00509083177538402\\
56.01	0.00509082839232848\\
57.01	0.00509082493958538\\
58.01	0.00509082141571937\\
59.01	0.00509081781926547\\
60.01	0.00509081414872855\\
61.01	0.00509081040258263\\
62.01	0.00509080657927035\\
63.01	0.00509080267720234\\
64.01	0.00509079869475621\\
65.01	0.00509079463027664\\
66.01	0.00509079048207394\\
67.01	0.00509078624842358\\
68.01	0.00509078192756587\\
69.01	0.00509077751770448\\
70.01	0.00509077301700613\\
71.01	0.00509076842360026\\
72.01	0.00509076373557727\\
73.01	0.00509075895098844\\
74.01	0.00509075406784498\\
75.01	0.00509074908411726\\
76.01	0.00509074399773357\\
77.01	0.00509073880657985\\
78.01	0.00509073350849835\\
79.01	0.00509072810128689\\
80.01	0.00509072258269822\\
81.01	0.00509071695043854\\
82.01	0.00509071120216712\\
83.01	0.00509070533549488\\
84.01	0.00509069934798343\\
85.01	0.0050906932371445\\
86.01	0.00509068700043856\\
87.01	0.00509068063527362\\
88.01	0.00509067413900453\\
89.01	0.00509066750893169\\
90.01	0.00509066074229996\\
91.01	0.00509065383629751\\
92.01	0.00509064678805472\\
93.01	0.00509063959464273\\
94.01	0.00509063225307272\\
95.01	0.0050906247602942\\
96.01	0.0050906171131939\\
97.01	0.00509060930859461\\
98.01	0.00509060134325381\\
99.01	0.00509059321386233\\
100.01	0.00509058491704274\\
101.01	0.00509057644934828\\
102.01	0.00509056780726128\\
103.01	0.0050905589871919\\
104.01	0.00509054998547617\\
105.01	0.00509054079837506\\
106.01	0.00509053142207263\\
107.01	0.00509052185267432\\
108.01	0.00509051208620588\\
109.01	0.00509050211861097\\
110.01	0.00509049194575009\\
111.01	0.00509048156339859\\
112.01	0.00509047096724497\\
113.01	0.0050904601528892\\
114.01	0.005090449115841\\
115.01	0.00509043785151748\\
116.01	0.00509042635524174\\
117.01	0.00509041462224082\\
118.01	0.00509040264764395\\
119.01	0.00509039042647985\\
120.01	0.00509037795367543\\
121.01	0.00509036522405326\\
122.01	0.0050903522323297\\
123.01	0.0050903389731127\\
124.01	0.00509032544089925\\
125.01	0.00509031163007364\\
126.01	0.00509029753490464\\
127.01	0.00509028314954362\\
128.01	0.00509026846802163\\
129.01	0.00509025348424733\\
130.01	0.00509023819200445\\
131.01	0.00509022258494903\\
132.01	0.00509020665660719\\
133.01	0.00509019040037217\\
134.01	0.00509017380950164\\
135.01	0.00509015687711486\\
136.01	0.00509013959619014\\
137.01	0.00509012195956197\\
138.01	0.00509010395991743\\
139.01	0.00509008558979416\\
140.01	0.00509006684157665\\
141.01	0.00509004770749325\\
142.01	0.00509002817961333\\
143.01	0.00509000824984351\\
144.01	0.00508998790992465\\
145.01	0.00508996715142834\\
146.01	0.00508994596575404\\
147.01	0.00508992434412425\\
148.01	0.00508990227758253\\
149.01	0.00508987975698869\\
150.01	0.00508985677301566\\
151.01	0.00508983331614541\\
152.01	0.00508980937666497\\
153.01	0.00508978494466291\\
154.01	0.00508976001002507\\
155.01	0.00508973456243036\\
156.01	0.00508970859134658\\
157.01	0.00508968208602633\\
158.01	0.00508965503550262\\
159.01	0.00508962742858412\\
160.01	0.00508959925385094\\
161.01	0.00508957049965001\\
162.01	0.0050895411540902\\
163.01	0.0050895112050376\\
164.01	0.00508948064011047\\
165.01	0.00508944944667449\\
166.01	0.0050894176118376\\
167.01	0.00508938512244456\\
168.01	0.00508935196507184\\
169.01	0.00508931812602267\\
170.01	0.00508928359132039\\
171.01	0.00508924834670393\\
172.01	0.00508921237762145\\
173.01	0.00508917566922473\\
174.01	0.0050891382063634\\
175.01	0.00508909997357835\\
176.01	0.00508906095509599\\
177.01	0.00508902113482193\\
178.01	0.00508898049633434\\
179.01	0.00508893902287745\\
180.01	0.0050888966973549\\
181.01	0.00508885350232277\\
182.01	0.00508880941998321\\
183.01	0.00508876443217666\\
184.01	0.00508871852037497\\
185.01	0.00508867166567397\\
186.01	0.00508862384878622\\
187.01	0.00508857505003305\\
188.01	0.00508852524933692\\
189.01	0.00508847442621349\\
190.01	0.00508842255976366\\
191.01	0.00508836962866526\\
192.01	0.00508831561116439\\
193.01	0.00508826048506745\\
194.01	0.00508820422773188\\
195.01	0.00508814681605782\\
196.01	0.00508808822647853\\
197.01	0.00508802843495155\\
198.01	0.00508796741694943\\
199.01	0.00508790514744991\\
200.01	0.0050878416009261\\
201.01	0.00508777675133724\\
202.01	0.00508771057211782\\
203.01	0.00508764303616791\\
204.01	0.00508757411584223\\
205.01	0.00508750378294013\\
206.01	0.0050874320086942\\
207.01	0.00508735876375952\\
208.01	0.00508728401820229\\
209.01	0.0050872077414885\\
210.01	0.00508712990247231\\
211.01	0.00508705046938434\\
212.01	0.00508696940981934\\
213.01	0.00508688669072412\\
214.01	0.00508680227838532\\
215.01	0.00508671613841619\\
216.01	0.00508662823574421\\
217.01	0.00508653853459765\\
218.01	0.00508644699849242\\
219.01	0.00508635359021819\\
220.01	0.00508625827182512\\
221.01	0.00508616100460911\\
222.01	0.00508606174909871\\
223.01	0.00508596046503921\\
224.01	0.00508585711137884\\
225.01	0.00508575164625374\\
226.01	0.00508564402697227\\
227.01	0.00508553421000003\\
228.01	0.00508542215094387\\
229.01	0.00508530780453626\\
230.01	0.00508519112461883\\
231.01	0.00508507206412603\\
232.01	0.00508495057506824\\
233.01	0.00508482660851554\\
234.01	0.00508470011457969\\
235.01	0.00508457104239697\\
236.01	0.005084439340111\\
237.01	0.00508430495485411\\
238.01	0.00508416783272951\\
239.01	0.00508402791879278\\
240.01	0.00508388515703334\\
241.01	0.00508373949035549\\
242.01	0.00508359086055902\\
243.01	0.0050834392083201\\
244.01	0.00508328447317157\\
245.01	0.005083126593483\\
246.01	0.00508296550644071\\
247.01	0.00508280114802701\\
248.01	0.00508263345300034\\
249.01	0.00508246235487425\\
250.01	0.00508228778589597\\
251.01	0.00508210967702601\\
252.01	0.00508192795791596\\
253.01	0.00508174255688761\\
254.01	0.00508155340091094\\
255.01	0.00508136041558203\\
256.01	0.00508116352510112\\
257.01	0.00508096265225067\\
258.01	0.00508075771837254\\
259.01	0.00508054864334582\\
260.01	0.0050803353455642\\
261.01	0.00508011774191309\\
262.01	0.00507989574774722\\
263.01	0.00507966927686769\\
264.01	0.00507943824149899\\
265.01	0.00507920255226643\\
266.01	0.00507896211817319\\
267.01	0.00507871684657763\\
268.01	0.00507846664317018\\
269.01	0.0050782114119512\\
270.01	0.00507795105520841\\
271.01	0.00507768547349428\\
272.01	0.00507741456560412\\
273.01	0.00507713822855394\\
274.01	0.00507685635755887\\
275.01	0.00507656884601195\\
276.01	0.00507627558546301\\
277.01	0.0050759764655979\\
278.01	0.00507567137421908\\
279.01	0.00507536019722508\\
280.01	0.00507504281859176\\
281.01	0.00507471912035403\\
282.01	0.00507438898258803\\
283.01	0.00507405228339385\\
284.01	0.00507370889887984\\
285.01	0.00507335870314669\\
286.01	0.00507300156827353\\
287.01	0.00507263736430464\\
288.01	0.00507226595923704\\
289.01	0.00507188721900948\\
290.01	0.00507150100749337\\
291.01	0.00507110718648397\\
292.01	0.00507070561569421\\
293.01	0.00507029615274887\\
294.01	0.00506987865318198\\
295.01	0.00506945297043459\\
296.01	0.00506901895585587\\
297.01	0.00506857645870491\\
298.01	0.00506812532615578\\
299.01	0.00506766540330411\\
300.01	0.00506719653317659\\
301.01	0.00506671855674299\\
302.01	0.00506623131293043\\
303.01	0.00506573463864115\\
304.01	0.00506522836877215\\
305.01	0.00506471233623913\\
306.01	0.0050641863720026\\
307.01	0.00506365030509771\\
308.01	0.00506310396266738\\
309.01	0.00506254716999854\\
310.01	0.00506197975056268\\
311.01	0.00506140152605938\\
312.01	0.00506081231646357\\
313.01	0.00506021194007659\\
314.01	0.00505960021358084\\
315.01	0.0050589769520983\\
316.01	0.00505834196925249\\
317.01	0.00505769507723388\\
318.01	0.0050570360868684\\
319.01	0.00505636480768943\\
320.01	0.00505568104801272\\
321.01	0.00505498461501344\\
322.01	0.0050542753148051\\
323.01	0.00505355295252096\\
324.01	0.00505281733239516\\
325.01	0.00505206825784484\\
326.01	0.00505130553155122\\
327.01	0.00505052895553891\\
328.01	0.00504973833125238\\
329.01	0.00504893345962804\\
330.01	0.00504811414116031\\
331.01	0.00504728017596019\\
332.01	0.00504643136380351\\
333.01	0.00504556750416772\\
334.01	0.00504468839625309\\
335.01	0.00504379383898717\\
336.01	0.00504288363100713\\
337.01	0.00504195757061827\\
338.01	0.00504101545572327\\
339.01	0.00504005708371843\\
340.01	0.00503908225135174\\
341.01	0.00503809075453745\\
342.01	0.00503708238812183\\
343.01	0.0050360569455929\\
344.01	0.00503501421872885\\
345.01	0.0050339539971773\\
346.01	0.00503287606795829\\
347.01	0.00503178021488342\\
348.01	0.00503066621788411\\
349.01	0.00502953385224091\\
350.01	0.00502838288770562\\
351.01	0.0050272130875118\\
352.01	0.00502602420726461\\
353.01	0.00502481599370699\\
354.01	0.00502358818335889\\
355.01	0.00502234050102702\\
356.01	0.00502107265818906\\
357.01	0.0050197843512555\\
358.01	0.00501847525972096\\
359.01	0.00501714504421948\\
360.01	0.00501579334450721\\
361.01	0.00501441977740002\\
362.01	0.00501302393470888\\
363.01	0.00501160538122043\\
364.01	0.00501016365278367\\
365.01	0.00500869825458022\\
366.01	0.00500720865966126\\
367.01	0.0050056943078557\\
368.01	0.00500415460516215\\
369.01	0.00500258892374896\\
370.01	0.00500099660269476\\
371.01	0.00499937694960236\\
372.01	0.00499772924321127\\
373.01	0.00499605273711389\\
374.01	0.00499434666464357\\
375.01	0.00499261024494476\\
376.01	0.00499084269014779\\
377.01	0.00498904321345464\\
378.01	0.00498721103779112\\
379.01	0.00498534540449191\\
380.01	0.00498344558127787\\
381.01	0.00498151086857018\\
382.01	0.00497954060301945\\
383.01	0.0049775341570865\\
384.01	0.00497549093374366\\
385.01	0.00497341035610876\\
386.01	0.0049712918540018\\
387.01	0.00496913485534641\\
388.01	0.00496693878503536\\
389.01	0.0049647030651768\\
390.01	0.0049624271153629\\
391.01	0.00496011035294964\\
392.01	0.00495775219334571\\
393.01	0.00495535205031138\\
394.01	0.00495290933626489\\
395.01	0.00495042346259684\\
396.01	0.00494789383998964\\
397.01	0.00494531987874166\\
398.01	0.00494270098909341\\
399.01	0.00494003658155518\\
400.01	0.00493732606723108\\
401.01	0.00493456885813996\\
402.01	0.00493176436752817\\
403.01	0.0049289120101708\\
404.01	0.00492601120265795\\
405.01	0.00492306136366175\\
406.01	0.0049200619141797\\
407.01	0.00491701227774755\\
408.01	0.00491391188061777\\
409.01	0.00491076015189521\\
410.01	0.00490755652362512\\
411.01	0.00490430043082325\\
412.01	0.00490099131144162\\
413.01	0.00489762860626115\\
414.01	0.00489421175869932\\
415.01	0.00489074021452438\\
416.01	0.00488721342146479\\
417.01	0.00488363082870177\\
418.01	0.00487999188623301\\
419.01	0.00487629604409436\\
420.01	0.00487254275142727\\
421.01	0.00486873145537775\\
422.01	0.00486486159981315\\
423.01	0.00486093262384335\\
424.01	0.00485694396013371\\
425.01	0.00485289503299578\\
426.01	0.00484878525624508\\
427.01	0.00484461403081522\\
428.01	0.00484038074211963\\
429.01	0.00483608475715432\\
430.01	0.00483172542133897\\
431.01	0.00482730205509661\\
432.01	0.00482281395017697\\
433.01	0.00481826036573343\\
434.01	0.00481364052417077\\
435.01	0.00480895360678758\\
436.01	0.0048041987492446\\
437.01	0.00479937503690344\\
438.01	0.00479448150008371\\
439.01	0.00478951710930519\\
440.01	0.00478448077058915\\
441.01	0.00477937132090532\\
442.01	0.0047741875238666\\
443.01	0.00476892806578173\\
444.01	0.00476359155219065\\
445.01	0.00475817650501215\\
446.01	0.00475268136044365\\
447.01	0.00474710446774809\\
448.01	0.00474144408906515\\
449.01	0.00473569840036508\\
450.01	0.00472986549364275\\
451.01	0.00472394338041665\\
452.01	0.00471792999654575\\
453.01	0.00471182320831667\\
454.01	0.00470562081966786\\
455.01	0.00469932058032547\\
456.01	0.00469292019450956\\
457.01	0.0046864173297475\\
458.01	0.00467980962520532\\
459.01	0.004673094698835\\
460.01	0.00466627015254277\\
461.01	0.00465933357455466\\
462.01	0.00465228253820061\\
463.01	0.00464511459651298\\
464.01	0.00463782727236597\\
465.01	0.0046304180444046\\
466.01	0.00462288432973456\\
467.01	0.00461522346520576\\
468.01	0.00460743268997008\\
469.01	0.00459950913243954\\
470.01	0.00459144980352513\\
471.01	0.00458325159387014\\
472.01	0.00457491127220498\\
473.01	0.00456642548437068\\
474.01	0.00455779075305001\\
475.01	0.00454900347823655\\
476.01	0.00454005993845263\\
477.01	0.00453095629270222\\
478.01	0.00452168858311627\\
479.01	0.00451225273821592\\
480.01	0.00450264457668138\\
481.01	0.00449285981147899\\
482.01	0.00448289405415866\\
483.01	0.00447274281910156\\
484.01	0.00446240152746782\\
485.01	0.00445186551057961\\
486.01	0.00444113001247249\\
487.01	0.00443019019136718\\
488.01	0.00441904111986471\\
489.01	0.00440767778374204\\
490.01	0.00439609507933523\\
491.01	0.00438428780963877\\
492.01	0.00437225067940555\\
493.01	0.00435997828969572\\
494.01	0.00434746513245579\\
495.01	0.00433470558577996\\
496.01	0.00432169391045496\\
497.01	0.00430842424817838\\
498.01	0.00429489062144068\\
499.01	0.00428108693452985\\
500.01	0.00426700697480779\\
501.01	0.00425264441371939\\
502.01	0.0042379928073723\\
503.01	0.0042230455966016\\
504.01	0.0042077961064585\\
505.01	0.00419223754508919\\
506.01	0.00417636300200287\\
507.01	0.00416016544577173\\
508.01	0.0041436377212398\\
509.01	0.00412677254635922\\
510.01	0.00410956250879883\\
511.01	0.00409200006248716\\
512.01	0.00407407752424347\\
513.01	0.00405578707062173\\
514.01	0.00403712073503673\\
515.01	0.00401807040516816\\
516.01	0.00399862782055691\\
517.01	0.00397878457025173\\
518.01	0.00395853209035544\\
519.01	0.00393786166138083\\
520.01	0.00391676440541885\\
521.01	0.00389523128316614\\
522.01	0.00387325309086969\\
523.01	0.00385082045725852\\
524.01	0.00382792384052074\\
525.01	0.00380455352538916\\
526.01	0.00378069962038027\\
527.01	0.00375635205522127\\
528.01	0.00373150057848589\\
529.01	0.0037061347554428\\
530.01	0.00368024396611962\\
531.01	0.00365381740358296\\
532.01	0.00362684407245007\\
533.01	0.00359931278766723\\
534.01	0.00357121217359918\\
535.01	0.00354253066349216\\
536.01	0.00351325649936497\\
537.01	0.0034833777323889\\
538.01	0.00345288222381592\\
539.01	0.0034217576465156\\
540.01	0.00338999148718528\\
541.01	0.00335757104930314\\
542.01	0.00332448345690271\\
543.01	0.00329071565925585\\
544.01	0.00325625443656603\\
545.01	0.00322108640678062\\
546.01	0.00318519803364702\\
547.01	0.00314857563614617\\
548.01	0.00311120539944831\\
549.01	0.0030730733875504\\
550.01	0.00303416555776898\\
551.01	0.00299446777727872\\
552.01	0.00295396584190528\\
553.01	0.00291264549740198\\
554.01	0.00287049246346125\\
555.01	0.00282749246073323\\
556.01	0.00278363124114932\\
557.01	0.00273889462187706\\
558.01	0.00269326852325377\\
559.01	0.00264673901108283\\
560.01	0.00259929234370175\\
561.01	0.00255091502426541\\
562.01	0.00250159385872054\\
563.01	0.0024513160199796\\
564.01	0.00240006911883537\\
565.01	0.00234784128218973\\
566.01	0.00229462123919728\\
567.01	0.00224039841595258\\
568.01	0.00218516303936678\\
569.01	0.00212890625089287\\
570.01	0.00207162023075652\\
571.01	0.00201329833333715\\
572.01	0.00195393523430371\\
573.01	0.00189352709004945\\
574.01	0.00183207170986988\\
575.01	0.00176956874118512\\
576.01	0.00170601986790419\\
577.01	0.00164142902175565\\
578.01	0.00157580260603646\\
579.01	0.00150914973074469\\
580.01	0.00144148245742554\\
581.01	0.0013728160512362\\
582.01	0.00130316923667782\\
583.01	0.00123256445209883\\
584.01	0.00116102809635882\\
585.01	0.00108859075888292\\
586.01	0.00101528742161295\\
587.01	0.000941157617941486\\
588.01	0.000866245529433705\\
589.01	0.000790599995791738\\
590.01	0.000714274406843124\\
591.01	0.000637326437030577\\
592.01	0.000559817572555011\\
593.01	0.000481812368507953\\
594.01	0.00040337735743835\\
595.01	0.00032457951111514\\
596.01	0.000245484132888262\\
597.01	0.000166217475640213\\
598.01	9.1337981519295e-05\\
599.01	2.91271958372426e-05\\
599.02	2.86192783627535e-05\\
599.03	2.8114423742006e-05\\
599.04	2.76126617978888e-05\\
599.05	2.71140226472798e-05\\
599.06	2.66185367039599e-05\\
599.07	2.61262346815533e-05\\
599.08	2.56371475965082e-05\\
599.09	2.51513067710835e-05\\
599.1	2.46687438363903e-05\\
599.11	2.41894907354479e-05\\
599.12	2.3713579726279e-05\\
599.13	2.32410433850267e-05\\
599.14	2.27719146091033e-05\\
599.15	2.23062266203871e-05\\
599.16	2.18440129684284e-05\\
599.17	2.13853075336917e-05\\
599.18	2.09301445308497e-05\\
599.19	2.04785585120829e-05\\
599.2	2.00305843704295e-05\\
599.21	1.95862573431627e-05\\
599.22	1.91456130152045e-05\\
599.23	1.87086873225609e-05\\
599.24	1.82755165558171e-05\\
599.25	1.78461373636363e-05\\
599.26	1.74205874164651e-05\\
599.27	1.69989072017502e-05\\
599.28	1.6581137610194e-05\\
599.29	1.61673199397579e-05\\
599.3	1.57574958996824e-05\\
599.31	1.53517076145714e-05\\
599.32	1.49499976284904e-05\\
599.33	1.45524089091333e-05\\
599.34	1.41589848520092e-05\\
599.35	1.37697692846814e-05\\
599.36	1.33848064710462e-05\\
599.37	1.3004141115644e-05\\
599.38	1.26278183680325e-05\\
599.39	1.22558838271912e-05\\
599.4	1.18883835459657e-05\\
599.41	1.15253640355657e-05\\
599.42	1.11668722700964e-05\\
599.43	1.08129556911519e-05\\
599.44	1.04636622124312e-05\\
599.45	1.01190402244239e-05\\
599.46	9.77913859911798e-06\\
599.47	9.44400669477576e-06\\
599.48	9.11369436074755e-06\\
599.49	8.7882519423238e-06\\
599.5	8.46773028565471e-06\\
599.51	8.15218074270464e-06\\
599.52	7.84165517625675e-06\\
599.53	7.5362059649732e-06\\
599.54	7.23588600849874e-06\\
599.55	6.94074873262146e-06\\
599.56	6.65084809447353e-06\\
599.57	6.36623858780126e-06\\
599.58	6.08697524826819e-06\\
599.59	5.81311365882402e-06\\
599.6	5.54470995511175e-06\\
599.61	5.28182083095637e-06\\
599.62	5.02450354387257e-06\\
599.63	4.77281592065407e-06\\
599.64	4.52681636300446e-06\\
599.65	4.28656385322197e-06\\
599.66	4.05211795996042e-06\\
599.67	3.8235388440163e-06\\
599.68	3.60088726420078e-06\\
599.69	3.38422458325514e-06\\
599.7	3.17361277382168e-06\\
599.71	2.96911442449269e-06\\
599.72	2.77079274589413e-06\\
599.73	2.5787115768474e-06\\
599.74	2.39293539058306e-06\\
599.75	2.213529301031e-06\\
599.76	2.04055906913997e-06\\
599.77	1.87409110929959e-06\\
599.78	1.71419249580043e-06\\
599.79	1.56093096936177e-06\\
599.8	1.41437494372877e-06\\
599.81	1.27459351233553e-06\\
599.82	1.14165645502366e-06\\
599.83	1.01563424484766e-06\\
599.84	8.96598054920053e-07\\
599.85	7.84619765348618e-07\\
599.86	6.79771970230753e-07\\
599.87	5.82127984719016e-07\\
599.88	4.91761852147374e-07\\
599.89	4.08748351252847e-07\\
599.9	3.33163003438802e-07\\
599.91	2.65082080131915e-07\\
599.92	2.04582610203313e-07\\
599.93	1.51742387450443e-07\\
599.94	1.06639978191686e-07\\
599.95	6.93547288800611e-08\\
599.96	3.99667738505e-08\\
599.97	1.85570430896731e-08\\
599.98	5.20727013418598e-09\\
599.99	0\\
600	0\\
};
\addplot [color=mycolor5,solid,forget plot]
  table[row sep=crcr]{%
0.01	0.00518985045865594\\
1.01	0.00518984929928745\\
2.01	0.00518984811589717\\
3.01	0.00518984690798687\\
4.01	0.00518984567504769\\
5.01	0.00518984441656035\\
6.01	0.00518984313199469\\
7.01	0.00518984182080953\\
8.01	0.00518984048245244\\
9.01	0.0051898391163596\\
10.01	0.00518983772195551\\
11.01	0.00518983629865253\\
12.01	0.00518983484585082\\
13.01	0.00518983336293832\\
14.01	0.00518983184928979\\
15.01	0.0051898303042675\\
16.01	0.00518982872722019\\
17.01	0.005189827117483\\
18.01	0.00518982547437757\\
19.01	0.00518982379721086\\
20.01	0.00518982208527604\\
21.01	0.00518982033785127\\
22.01	0.00518981855419968\\
23.01	0.00518981673356932\\
24.01	0.00518981487519226\\
25.01	0.00518981297828495\\
26.01	0.0051898110420471\\
27.01	0.00518980906566204\\
28.01	0.00518980704829631\\
29.01	0.00518980498909893\\
30.01	0.00518980288720111\\
31.01	0.00518980074171607\\
32.01	0.00518979855173877\\
33.01	0.00518979631634505\\
34.01	0.00518979403459159\\
35.01	0.00518979170551546\\
36.01	0.00518978932813395\\
37.01	0.00518978690144372\\
38.01	0.00518978442442033\\
39.01	0.00518978189601838\\
40.01	0.00518977931517049\\
41.01	0.00518977668078721\\
42.01	0.00518977399175633\\
43.01	0.00518977124694251\\
44.01	0.00518976844518681\\
45.01	0.00518976558530621\\
46.01	0.00518976266609294\\
47.01	0.00518975968631431\\
48.01	0.0051897566447117\\
49.01	0.00518975354000056\\
50.01	0.00518975037086952\\
51.01	0.0051897471359797\\
52.01	0.00518974383396468\\
53.01	0.00518974046342943\\
54.01	0.00518973702294982\\
55.01	0.00518973351107229\\
56.01	0.00518972992631276\\
57.01	0.00518972626715638\\
58.01	0.00518972253205689\\
59.01	0.00518971871943561\\
60.01	0.00518971482768111\\
61.01	0.00518971085514838\\
62.01	0.00518970680015819\\
63.01	0.00518970266099631\\
64.01	0.00518969843591297\\
65.01	0.00518969412312156\\
66.01	0.00518968972079873\\
67.01	0.00518968522708283\\
68.01	0.00518968064007366\\
69.01	0.00518967595783142\\
70.01	0.00518967117837573\\
71.01	0.00518966629968513\\
72.01	0.00518966131969614\\
73.01	0.00518965623630202\\
74.01	0.00518965104735233\\
75.01	0.00518964575065174\\
76.01	0.00518964034395943\\
77.01	0.00518963482498765\\
78.01	0.0051896291914012\\
79.01	0.00518962344081616\\
80.01	0.00518961757079861\\
81.01	0.00518961157886453\\
82.01	0.00518960546247765\\
83.01	0.00518959921904911\\
84.01	0.00518959284593613\\
85.01	0.00518958634044068\\
86.01	0.00518957969980875\\
87.01	0.00518957292122883\\
88.01	0.00518956600183094\\
89.01	0.00518955893868518\\
90.01	0.00518955172880054\\
91.01	0.00518954436912368\\
92.01	0.00518953685653779\\
93.01	0.00518952918786115\\
94.01	0.00518952135984562\\
95.01	0.00518951336917528\\
96.01	0.00518950521246538\\
97.01	0.00518949688626051\\
98.01	0.00518948838703316\\
99.01	0.00518947971118223\\
100.01	0.00518947085503186\\
101.01	0.00518946181482933\\
102.01	0.00518945258674392\\
103.01	0.00518944316686495\\
104.01	0.00518943355120032\\
105.01	0.00518942373567467\\
106.01	0.00518941371612753\\
107.01	0.00518940348831199\\
108.01	0.00518939304789231\\
109.01	0.0051893823904429\\
110.01	0.00518937151144555\\
111.01	0.00518936040628784\\
112.01	0.0051893490702614\\
113.01	0.00518933749855965\\
114.01	0.00518932568627586\\
115.01	0.00518931362840079\\
116.01	0.00518930131982125\\
117.01	0.00518928875531716\\
118.01	0.00518927592955957\\
119.01	0.00518926283710864\\
120.01	0.00518924947241103\\
121.01	0.00518923582979778\\
122.01	0.00518922190348179\\
123.01	0.00518920768755497\\
124.01	0.00518919317598663\\
125.01	0.00518917836261989\\
126.01	0.00518916324116988\\
127.01	0.00518914780522049\\
128.01	0.00518913204822209\\
129.01	0.00518911596348858\\
130.01	0.00518909954419419\\
131.01	0.00518908278337122\\
132.01	0.00518906567390646\\
133.01	0.00518904820853864\\
134.01	0.00518903037985504\\
135.01	0.00518901218028874\\
136.01	0.00518899360211469\\
137.01	0.00518897463744719\\
138.01	0.00518895527823605\\
139.01	0.00518893551626337\\
140.01	0.00518891534313996\\
141.01	0.00518889475030202\\
142.01	0.0051888737290068\\
143.01	0.0051888522703299\\
144.01	0.00518883036516069\\
145.01	0.00518880800419882\\
146.01	0.00518878517795015\\
147.01	0.00518876187672276\\
148.01	0.00518873809062268\\
149.01	0.00518871380954999\\
150.01	0.00518868902319427\\
151.01	0.00518866372103032\\
152.01	0.0051886378923138\\
153.01	0.00518861152607637\\
154.01	0.00518858461112139\\
155.01	0.00518855713601916\\
156.01	0.00518852908910152\\
157.01	0.00518850045845785\\
158.01	0.00518847123192905\\
159.01	0.00518844139710325\\
160.01	0.00518841094130972\\
161.01	0.00518837985161411\\
162.01	0.00518834811481294\\
163.01	0.00518831571742765\\
164.01	0.00518828264569917\\
165.01	0.00518824888558228\\
166.01	0.00518821442273881\\
167.01	0.00518817924253289\\
168.01	0.00518814333002394\\
169.01	0.00518810666996001\\
170.01	0.0051880692467723\\
171.01	0.00518803104456791\\
172.01	0.00518799204712354\\
173.01	0.00518795223787821\\
174.01	0.00518791159992654\\
175.01	0.00518787011601153\\
176.01	0.00518782776851751\\
177.01	0.00518778453946232\\
178.01	0.00518774041049012\\
179.01	0.00518769536286341\\
180.01	0.0051876493774553\\
181.01	0.00518760243474135\\
182.01	0.00518755451479137\\
183.01	0.00518750559726105\\
184.01	0.00518745566138339\\
185.01	0.00518740468596025\\
186.01	0.00518735264935298\\
187.01	0.00518729952947366\\
188.01	0.00518724530377576\\
189.01	0.00518718994924476\\
190.01	0.00518713344238819\\
191.01	0.00518707575922619\\
192.01	0.0051870168752815\\
193.01	0.00518695676556861\\
194.01	0.00518689540458407\\
195.01	0.00518683276629552\\
196.01	0.00518676882413061\\
197.01	0.00518670355096637\\
198.01	0.00518663691911726\\
199.01	0.00518656890032433\\
200.01	0.00518649946574306\\
201.01	0.00518642858593131\\
202.01	0.00518635623083735\\
203.01	0.00518628236978714\\
204.01	0.00518620697147179\\
205.01	0.00518613000393414\\
206.01	0.00518605143455629\\
207.01	0.0051859712300451\\
208.01	0.00518588935641946\\
209.01	0.00518580577899548\\
210.01	0.00518572046237251\\
211.01	0.00518563337041827\\
212.01	0.0051855444662542\\
213.01	0.00518545371224023\\
214.01	0.00518536106995914\\
215.01	0.00518526650020106\\
216.01	0.00518516996294714\\
217.01	0.00518507141735299\\
218.01	0.00518497082173251\\
219.01	0.0051848681335405\\
220.01	0.00518476330935507\\
221.01	0.00518465630486035\\
222.01	0.0051845470748279\\
223.01	0.00518443557309908\\
224.01	0.00518432175256559\\
225.01	0.00518420556515075\\
226.01	0.00518408696179002\\
227.01	0.00518396589241079\\
228.01	0.00518384230591273\\
229.01	0.00518371615014677\\
230.01	0.00518358737189447\\
231.01	0.00518345591684629\\
232.01	0.00518332172958027\\
233.01	0.00518318475353934\\
234.01	0.00518304493100925\\
235.01	0.0051829022030954\\
236.01	0.00518275650969934\\
237.01	0.00518260778949514\\
238.01	0.00518245597990508\\
239.01	0.00518230101707496\\
240.01	0.00518214283584911\\
241.01	0.00518198136974454\\
242.01	0.00518181655092525\\
243.01	0.00518164831017592\\
244.01	0.00518147657687442\\
245.01	0.00518130127896504\\
246.01	0.0051811223429304\\
247.01	0.00518093969376337\\
248.01	0.00518075325493808\\
249.01	0.00518056294838079\\
250.01	0.00518036869444017\\
251.01	0.00518017041185758\\
252.01	0.00517996801773575\\
253.01	0.00517976142750795\\
254.01	0.00517955055490675\\
255.01	0.00517933531193155\\
256.01	0.00517911560881625\\
257.01	0.00517889135399623\\
258.01	0.00517866245407491\\
259.01	0.00517842881378965\\
260.01	0.00517819033597789\\
261.01	0.00517794692154178\\
262.01	0.00517769846941321\\
263.01	0.00517744487651797\\
264.01	0.00517718603773993\\
265.01	0.00517692184588422\\
266.01	0.00517665219164034\\
267.01	0.00517637696354497\\
268.01	0.00517609604794434\\
269.01	0.00517580932895597\\
270.01	0.00517551668843058\\
271.01	0.00517521800591347\\
272.01	0.00517491315860531\\
273.01	0.0051746020213232\\
274.01	0.00517428446646141\\
275.01	0.00517396036395159\\
276.01	0.00517362958122315\\
277.01	0.0051732919831636\\
278.01	0.00517294743207849\\
279.01	0.00517259578765159\\
280.01	0.00517223690690545\\
281.01	0.00517187064416121\\
282.01	0.00517149685099972\\
283.01	0.00517111537622183\\
284.01	0.00517072606580955\\
285.01	0.00517032876288793\\
286.01	0.00516992330768668\\
287.01	0.00516950953750294\\
288.01	0.00516908728666476\\
289.01	0.00516865638649528\\
290.01	0.00516821666527762\\
291.01	0.00516776794822137\\
292.01	0.00516731005742999\\
293.01	0.00516684281186986\\
294.01	0.00516636602734045\\
295.01	0.00516587951644691\\
296.01	0.00516538308857364\\
297.01	0.00516487654986119\\
298.01	0.00516435970318472\\
299.01	0.00516383234813548\\
300.01	0.00516329428100495\\
301.01	0.00516274529477229\\
302.01	0.00516218517909523\\
303.01	0.00516161372030499\\
304.01	0.00516103070140488\\
305.01	0.00516043590207418\\
306.01	0.00515982909867575\\
307.01	0.00515921006427036\\
308.01	0.00515857856863553\\
309.01	0.00515793437829201\\
310.01	0.00515727725653543\\
311.01	0.00515660696347648\\
312.01	0.00515592325608833\\
313.01	0.00515522588826296\\
314.01	0.00515451461087604\\
315.01	0.00515378917186164\\
316.01	0.00515304931629828\\
317.01	0.00515229478650404\\
318.01	0.00515152532214645\\
319.01	0.00515074066036248\\
320.01	0.0051499405358934\\
321.01	0.0051491246812335\\
322.01	0.00514829282679385\\
323.01	0.00514744470108256\\
324.01	0.00514658003090139\\
325.01	0.00514569854156009\\
326.01	0.00514479995710914\\
327.01	0.00514388400059231\\
328.01	0.00514295039431843\\
329.01	0.00514199886015426\\
330.01	0.00514102911983835\\
331.01	0.00514004089531677\\
332.01	0.00513903390910029\\
333.01	0.00513800788464356\\
334.01	0.00513696254674596\\
335.01	0.00513589762197307\\
336.01	0.0051348128390989\\
337.01	0.00513370792956593\\
338.01	0.00513258262796248\\
339.01	0.00513143667251303\\
340.01	0.00513026980557944\\
341.01	0.00512908177416816\\
342.01	0.00512787233043716\\
343.01	0.00512664123219644\\
344.01	0.00512538824339381\\
345.01	0.00512411313457496\\
346.01	0.00512281568330687\\
347.01	0.0051214956745489\\
348.01	0.00512015290095575\\
349.01	0.00511878716309172\\
350.01	0.00511739826953352\\
351.01	0.00511598603683452\\
352.01	0.00511455028932025\\
353.01	0.00511309085868016\\
354.01	0.00511160758331573\\
355.01	0.00511010030740191\\
356.01	0.00510856887961103\\
357.01	0.00510701315144722\\
358.01	0.00510543297513202\\
359.01	0.00510382820097896\\
360.01	0.00510219867419295\\
361.01	0.0051005442310292\\
362.01	0.00509886469424607\\
363.01	0.00509715986779404\\
364.01	0.00509542953069209\\
365.01	0.00509367343005702\\
366.01	0.00509189127327863\\
367.01	0.00509008271936833\\
368.01	0.0050882473695548\\
369.01	0.00508638475726977\\
370.01	0.00508449433774629\\
371.01	0.00508257547756524\\
372.01	0.0050806274446134\\
373.01	0.0050786493990813\\
374.01	0.00507664038631425\\
375.01	0.00507459933253885\\
376.01	0.00507252504470578\\
377.01	0.00507041621589257\\
378.01	0.00506827143785643\\
379.01	0.00506608922234501\\
380.01	0.00506386803255252\\
381.01	0.0050616063254657\\
382.01	0.00505930260451195\\
383.01	0.00505695547948836\\
384.01	0.00505456372658877\\
385.01	0.00505212633438709\\
386.01	0.00504964248482725\\
387.01	0.00504711141476803\\
388.01	0.00504453235718774\\
389.01	0.00504190454022374\\
390.01	0.0050392271876562\\
391.01	0.00503649951942945\\
392.01	0.00503372075221465\\
393.01	0.00503089010001324\\
394.01	0.00502800677480556\\
395.01	0.00502506998724391\\
396.01	0.00502207894739259\\
397.01	0.00501903286551728\\
398.01	0.0050159309529246\\
399.01	0.00501277242285115\\
400.01	0.00500955649140807\\
401.01	0.00500628237857653\\
402.01	0.00500294930925779\\
403.01	0.00499955651437852\\
404.01	0.00499610323204939\\
405.01	0.00499258870877816\\
406.01	0.00498901220073488\\
407.01	0.00498537297506879\\
408.01	0.00498167031127348\\
409.01	0.00497790350259771\\
410.01	0.00497407185749765\\
411.01	0.00497017470112534\\
412.01	0.00496621137684656\\
413.01	0.00496218124778095\\
414.01	0.00495808369835382\\
415.01	0.00495391813585008\\
416.01	0.00494968399195533\\
417.01	0.0049453807242706\\
418.01	0.00494100781778096\\
419.01	0.00493656478625929\\
420.01	0.00493205117358051\\
421.01	0.00492746655492047\\
422.01	0.00492281053780932\\
423.01	0.00491808276300458\\
424.01	0.00491328290514819\\
425.01	0.00490841067316268\\
426.01	0.00490346581034312\\
427.01	0.00489844809409246\\
428.01	0.00489335733524447\\
429.01	0.00488819337691678\\
430.01	0.00488295609282807\\
431.01	0.00487764538501349\\
432.01	0.00487226118086618\\
433.01	0.00486680342943315\\
434.01	0.0048612720968921\\
435.01	0.00485566716113472\\
436.01	0.00484998860538954\\
437.01	0.00484423641081681\\
438.01	0.00483841054802516\\
439.01	0.00483251096746627\\
440.01	0.00482653758868795\\
441.01	0.0048204902884483\\
442.01	0.00481436888772832\\
443.01	0.00480817313771789\\
444.01	0.00480190270490477\\
445.01	0.00479555715545268\\
446.01	0.0047891359391279\\
447.01	0.00478263837311622\\
448.01	0.004776063626164\\
449.01	0.00476941070358255\\
450.01	0.00476267843376099\\
451.01	0.00475586545694359\\
452.01	0.00474897021713488\\
453.01	0.00474199095807932\\
454.01	0.00473492572432351\\
455.01	0.00472777236837013\\
456.01	0.00472052856486185\\
457.01	0.00471319183255276\\
458.01	0.00470575956449523\\
459.01	0.00469822906634802\\
460.01	0.00469059760196692\\
461.01	0.00468286244441021\\
462.01	0.00467502092919312\\
463.01	0.00466707050506619\\
464.01	0.00465900877590109\\
465.01	0.00465083352570604\\
466.01	0.0046425427178675\\
467.01	0.00463413446036445\\
468.01	0.00462560693245625\\
469.01	0.0046169582789292\\
470.01	0.00460818651563546\\
471.01	0.00459928949483304\\
472.01	0.00459026489236327\\
473.01	0.00458111019614933\\
474.01	0.00457182269565632\\
475.01	0.00456239947264625\\
476.01	0.00455283739358378\\
477.01	0.00454313310406019\\
478.01	0.00453328302559832\\
479.01	0.00452328335518012\\
480.01	0.00451313006779437\\
481.01	0.0045028189222251\\
482.01	0.0044923454701958\\
483.01	0.0044817050688328\\
484.01	0.00447089289622598\\
485.01	0.00445990396962706\\
486.01	0.0044487331655649\\
487.01	0.0044373752408577\\
488.01	0.00442582485320917\\
489.01	0.00441407657980662\\
490.01	0.00440212493215473\\
491.01	0.0043899643653333\\
492.01	0.00437758928005632\\
493.01	0.00436499401641587\\
494.01	0.00435217283911384\\
495.01	0.00433911991536675\\
496.01	0.00432582928848049\\
497.01	0.00431229485208758\\
498.01	0.00429851033159886\\
499.01	0.0042844692787446\\
500.01	0.00427016507750369\\
501.01	0.00425559095354556\\
502.01	0.00424073998420658\\
503.01	0.00422560510846292\\
504.01	0.00421017913641638\\
505.01	0.00419445475777532\\
506.01	0.00417842454882062\\
507.01	0.00416208097738592\\
508.01	0.00414541640548821\\
509.01	0.00412842308940153\\
510.01	0.00411109317718921\\
511.01	0.00409341870398167\\
512.01	0.00407539158558334\\
513.01	0.00405700361126819\\
514.01	0.00403824643681267\\
515.01	0.00401911157882534\\
516.01	0.00399959041117256\\
517.01	0.0039796741636992\\
518.01	0.00395935392255446\\
519.01	0.0039386206307375\\
520.01	0.00391746508787289\\
521.01	0.00389587794901097\\
522.01	0.00387384972249661\\
523.01	0.0038513707670231\\
524.01	0.00382843128805948\\
525.01	0.00380502133389307\\
526.01	0.00378113079156275\\
527.01	0.00375674938295798\\
528.01	0.00373186666131546\\
529.01	0.00370647200826305\\
530.01	0.0036805546314431\\
531.01	0.00365410356262667\\
532.01	0.0036271076561445\\
533.01	0.00359955558746853\\
534.01	0.00357143585189767\\
535.01	0.00354273676342345\\
536.01	0.00351344645390844\\
537.01	0.00348355287270617\\
538.01	0.00345304378684679\\
539.01	0.00342190678189653\\
540.01	0.00339012926358016\\
541.01	0.00335769846023385\\
542.01	0.00332460142614646\\
543.01	0.00329082504584472\\
544.01	0.00325635603939319\\
545.01	0.0032211809688097\\
546.01	0.00318528624572494\\
547.01	0.00314865814043742\\
548.01	0.0031112827925246\\
549.01	0.00307314622317565\\
550.01	0.00303423434942168\\
551.01	0.00299453300045077\\
552.01	0.00295402793620921\\
553.01	0.00291270486851015\\
554.01	0.00287054948489739\\
555.01	0.00282754747553415\\
556.01	0.0027836845634168\\
557.01	0.00273894653823668\\
558.01	0.00269331929424167\\
559.01	0.00264678887247523\\
560.01	0.00259934150780129\\
561.01	0.00255096368115431\\
562.01	0.00250164217748654\\
563.01	0.00245136414991943\\
564.01	0.00240011719063778\\
565.01	0.00234788940909758\\
566.01	0.00229466951814832\\
567.01	0.00224044692869491\\
568.01	0.00218521185354425\\
569.01	0.00212895542109336\\
570.01	0.00207166979951682\\
571.01	0.0020133483320942\\
572.01	0.00195398568428513\\
573.01	0.00189357800309545\\
574.01	0.00183212308917921\\
575.01	0.00176962058197929\\
576.01	0.00170607215800749\\
577.01	0.00164148174208821\\
578.01	0.00157585573102297\\
579.01	0.00150920322864397\\
580.01	0.00144153629058912\\
581.01	0.00137287017630887\\
582.01	0.0013032236047591\\
583.01	0.00123261900888581\\
584.01	0.00116108278229978\\
585.01	0.00108864550937499\\
586.01	0.00101534216728416\\
587.01	0.000941212285065976\\
588.01	0.000866300040538016\\
589.01	0.000790654270520734\\
590.01	0.000714328363169361\\
591.01	0.000637379992907505\\
592.01	0.000559870648136378\\
593.01	0.000481864889083335\\
594.01	0.000403429257268736\\
595.01	0.000324630738394923\\
596.01	0.000245534656112314\\
597.01	0.000166246033647394\\
598.01	9.13379815192916e-05\\
599.01	2.91271958372443e-05\\
599.02	2.86192783627518e-05\\
599.03	2.81144237420042e-05\\
599.04	2.76126617978871e-05\\
599.05	2.71140226472798e-05\\
599.06	2.66185367039581e-05\\
599.07	2.6126234681555e-05\\
599.08	2.56371475965064e-05\\
599.09	2.51513067710818e-05\\
599.1	2.46687438363886e-05\\
599.11	2.41894907354497e-05\\
599.12	2.37135797262807e-05\\
599.13	2.32410433850267e-05\\
599.14	2.2771914609105e-05\\
599.15	2.23062266203888e-05\\
599.16	2.18440129684267e-05\\
599.17	2.13853075336917e-05\\
599.18	2.0930144530848e-05\\
599.19	2.04785585120812e-05\\
599.2	2.00305843704278e-05\\
599.21	1.95862573431627e-05\\
599.22	1.91456130152028e-05\\
599.23	1.87086873225609e-05\\
599.24	1.82755165558171e-05\\
599.25	1.7846137363638e-05\\
599.26	1.74205874164668e-05\\
599.27	1.69989072017485e-05\\
599.28	1.6581137610194e-05\\
599.29	1.61673199397579e-05\\
599.3	1.57574958996841e-05\\
599.31	1.53517076145714e-05\\
599.32	1.49499976284904e-05\\
599.33	1.45524089091333e-05\\
599.34	1.41589848520092e-05\\
599.35	1.37697692846831e-05\\
599.36	1.33848064710462e-05\\
599.37	1.30041411156422e-05\\
599.38	1.26278183680325e-05\\
599.39	1.22558838271929e-05\\
599.4	1.18883835459674e-05\\
599.41	1.15253640355657e-05\\
599.42	1.11668722700981e-05\\
599.43	1.08129556911519e-05\\
599.44	1.04636622124312e-05\\
599.45	1.01190402244222e-05\\
599.46	9.77913859911798e-06\\
599.47	9.44400669477576e-06\\
599.48	9.11369436074581e-06\\
599.49	8.78825194232206e-06\\
599.5	8.46773028565471e-06\\
599.51	8.15218074270464e-06\\
599.52	7.84165517625848e-06\\
599.53	7.5362059649732e-06\\
599.54	7.23588600849874e-06\\
599.55	6.94074873261973e-06\\
599.56	6.65084809447353e-06\\
599.57	6.366238587803e-06\\
599.58	6.08697524826993e-06\\
599.59	5.81311365882228e-06\\
599.6	5.54470995511175e-06\\
599.61	5.28182083095637e-06\\
599.62	5.02450354387431e-06\\
599.63	4.7728159206558e-06\\
599.64	4.52681636300446e-06\\
599.65	4.28656385322197e-06\\
599.66	4.05211795995869e-06\\
599.67	3.82353884401457e-06\\
599.68	3.60088726420078e-06\\
599.69	3.38422458325341e-06\\
599.7	3.17361277382341e-06\\
599.71	2.96911442449269e-06\\
599.72	2.7707927458924e-06\\
599.73	2.57871157684567e-06\\
599.74	2.3929353905848e-06\\
599.75	2.21352930102926e-06\\
599.76	2.04055906913823e-06\\
599.77	1.87409110929959e-06\\
599.78	1.71419249580043e-06\\
599.79	1.56093096936004e-06\\
599.8	1.41437494372877e-06\\
599.81	1.27459351233379e-06\\
599.82	1.14165645502366e-06\\
599.83	1.01563424484766e-06\\
599.84	8.96598054920053e-07\\
599.85	7.84619765350353e-07\\
599.86	6.79771970232487e-07\\
599.87	5.82127984717282e-07\\
599.88	4.91761852145639e-07\\
599.89	4.08748351251112e-07\\
599.9	3.33163003437068e-07\\
599.91	2.65082080131915e-07\\
599.92	2.04582610201579e-07\\
599.93	1.51742387452178e-07\\
599.94	1.06639978189951e-07\\
599.95	6.93547288817958e-08\\
599.96	3.99667738487652e-08\\
599.97	1.85570430896731e-08\\
599.98	5.20727013418598e-09\\
599.99	0\\
600	0\\
};
\addplot [color=mycolor6,solid,forget plot]
  table[row sep=crcr]{%
0.01	0.00535165554740582\\
1.01	0.00535165435177768\\
2.01	0.00535165313126011\\
3.01	0.00535165188533369\\
4.01	0.00535165061346828\\
5.01	0.00535164931512245\\
6.01	0.00535164798974384\\
7.01	0.00535164663676811\\
8.01	0.00535164525561941\\
9.01	0.00535164384570965\\
10.01	0.00535164240643837\\
11.01	0.00535164093719274\\
12.01	0.00535163943734712\\
13.01	0.00535163790626262\\
14.01	0.00535163634328703\\
15.01	0.00535163474775443\\
16.01	0.00535163311898514\\
17.01	0.00535163145628507\\
18.01	0.0053516297589457\\
19.01	0.00535162802624365\\
20.01	0.00535162625744028\\
21.01	0.00535162445178166\\
22.01	0.00535162260849789\\
23.01	0.0053516207268031\\
24.01	0.00535161880589482\\
25.01	0.00535161684495391\\
26.01	0.005351614843144\\
27.01	0.00535161279961132\\
28.01	0.00535161071348379\\
29.01	0.00535160858387155\\
30.01	0.00535160640986589\\
31.01	0.00535160419053901\\
32.01	0.00535160192494384\\
33.01	0.00535159961211312\\
34.01	0.00535159725105974\\
35.01	0.00535159484077575\\
36.01	0.00535159238023177\\
37.01	0.00535158986837724\\
38.01	0.00535158730413941\\
39.01	0.00535158468642296\\
40.01	0.00535158201410972\\
41.01	0.00535157928605808\\
42.01	0.00535157650110247\\
43.01	0.00535157365805277\\
44.01	0.00535157075569409\\
45.01	0.00535156779278586\\
46.01	0.00535156476806174\\
47.01	0.00535156168022837\\
48.01	0.0053515585279657\\
49.01	0.00535155530992578\\
50.01	0.00535155202473251\\
51.01	0.00535154867098089\\
52.01	0.00535154524723616\\
53.01	0.00535154175203372\\
54.01	0.00535153818387834\\
55.01	0.00535153454124298\\
56.01	0.005351530822569\\
57.01	0.00535152702626476\\
58.01	0.00535152315070531\\
59.01	0.00535151919423157\\
60.01	0.00535151515514961\\
61.01	0.00535151103173001\\
62.01	0.00535150682220687\\
63.01	0.00535150252477731\\
64.01	0.00535149813760059\\
65.01	0.00535149365879743\\
66.01	0.00535148908644871\\
67.01	0.00535148441859534\\
68.01	0.00535147965323694\\
69.01	0.00535147478833106\\
70.01	0.00535146982179247\\
71.01	0.0053514647514921\\
72.01	0.00535145957525603\\
73.01	0.0053514542908648\\
74.01	0.00535144889605222\\
75.01	0.00535144338850447\\
76.01	0.00535143776585907\\
77.01	0.00535143202570394\\
78.01	0.00535142616557633\\
79.01	0.00535142018296148\\
80.01	0.00535141407529222\\
81.01	0.00535140783994703\\
82.01	0.00535140147424918\\
83.01	0.005351394975466\\
84.01	0.00535138834080721\\
85.01	0.00535138156742381\\
86.01	0.00535137465240689\\
87.01	0.00535136759278639\\
88.01	0.00535136038552964\\
89.01	0.0053513530275403\\
90.01	0.00535134551565688\\
91.01	0.00535133784665152\\
92.01	0.0053513300172282\\
93.01	0.00535132202402151\\
94.01	0.00535131386359549\\
95.01	0.00535130553244182\\
96.01	0.005351297026978\\
97.01	0.0053512883435465\\
98.01	0.00535127947841268\\
99.01	0.00535127042776328\\
100.01	0.00535126118770487\\
101.01	0.00535125175426191\\
102.01	0.00535124212337513\\
103.01	0.00535123229089991\\
104.01	0.00535122225260437\\
105.01	0.00535121200416742\\
106.01	0.00535120154117719\\
107.01	0.00535119085912848\\
108.01	0.0053511799534217\\
109.01	0.0053511688193598\\
110.01	0.00535115745214741\\
111.01	0.00535114584688781\\
112.01	0.00535113399858112\\
113.01	0.00535112190212204\\
114.01	0.00535110955229786\\
115.01	0.00535109694378628\\
116.01	0.00535108407115214\\
117.01	0.00535107092884647\\
118.01	0.00535105751120296\\
119.01	0.00535104381243604\\
120.01	0.00535102982663815\\
121.01	0.00535101554777718\\
122.01	0.00535100096969379\\
123.01	0.00535098608609906\\
124.01	0.00535097089057116\\
125.01	0.00535095537655314\\
126.01	0.00535093953734955\\
127.01	0.00535092336612398\\
128.01	0.00535090685589577\\
129.01	0.00535088999953703\\
130.01	0.00535087278976958\\
131.01	0.00535085521916177\\
132.01	0.00535083728012522\\
133.01	0.00535081896491143\\
134.01	0.00535080026560837\\
135.01	0.00535078117413728\\
136.01	0.00535076168224889\\
137.01	0.00535074178151985\\
138.01	0.00535072146334912\\
139.01	0.0053507007189542\\
140.01	0.00535067953936729\\
141.01	0.00535065791543123\\
142.01	0.00535063583779594\\
143.01	0.00535061329691356\\
144.01	0.00535059028303504\\
145.01	0.00535056678620547\\
146.01	0.00535054279625969\\
147.01	0.00535051830281814\\
148.01	0.0053504932952821\\
149.01	0.0053504677628291\\
150.01	0.00535044169440818\\
151.01	0.00535041507873501\\
152.01	0.00535038790428707\\
153.01	0.00535036015929866\\
154.01	0.00535033183175527\\
155.01	0.00535030290938899\\
156.01	0.00535027337967294\\
157.01	0.00535024322981521\\
158.01	0.00535021244675423\\
159.01	0.00535018101715209\\
160.01	0.00535014892738953\\
161.01	0.00535011616355948\\
162.01	0.00535008271146103\\
163.01	0.00535004855659329\\
164.01	0.00535001368414908\\
165.01	0.00534997807900831\\
166.01	0.00534994172573162\\
167.01	0.00534990460855314\\
168.01	0.00534986671137403\\
169.01	0.00534982801775518\\
170.01	0.00534978851091022\\
171.01	0.00534974817369797\\
172.01	0.00534970698861489\\
173.01	0.00534966493778769\\
174.01	0.00534962200296519\\
175.01	0.00534957816551045\\
176.01	0.00534953340639275\\
177.01	0.00534948770617875\\
178.01	0.00534944104502456\\
179.01	0.00534939340266683\\
180.01	0.00534934475841364\\
181.01	0.00534929509113565\\
182.01	0.00534924437925669\\
183.01	0.00534919260074481\\
184.01	0.00534913973310177\\
185.01	0.0053490857533536\\
186.01	0.00534903063804076\\
187.01	0.00534897436320744\\
188.01	0.00534891690439144\\
189.01	0.00534885823661272\\
190.01	0.00534879833436352\\
191.01	0.0053487371715961\\
192.01	0.005348674721712\\
193.01	0.00534861095755016\\
194.01	0.00534854585137504\\
195.01	0.00534847937486385\\
196.01	0.00534841149909541\\
197.01	0.00534834219453612\\
198.01	0.00534827143102793\\
199.01	0.00534819917777458\\
200.01	0.00534812540332819\\
201.01	0.00534805007557553\\
202.01	0.00534797316172387\\
203.01	0.00534789462828649\\
204.01	0.00534781444106817\\
205.01	0.00534773256515012\\
206.01	0.00534764896487432\\
207.01	0.00534756360382829\\
208.01	0.00534747644482889\\
209.01	0.00534738744990587\\
210.01	0.00534729658028554\\
211.01	0.00534720379637332\\
212.01	0.00534710905773658\\
213.01	0.00534701232308677\\
214.01	0.00534691355026138\\
215.01	0.0053468126962053\\
216.01	0.00534670971695187\\
217.01	0.00534660456760377\\
218.01	0.00534649720231294\\
219.01	0.0053463875742606\\
220.01	0.00534627563563692\\
221.01	0.0053461613376194\\
222.01	0.00534604463035201\\
223.01	0.00534592546292281\\
224.01	0.00534580378334183\\
225.01	0.00534567953851823\\
226.01	0.00534555267423655\\
227.01	0.00534542313513353\\
228.01	0.00534529086467326\\
229.01	0.00534515580512225\\
230.01	0.00534501789752436\\
231.01	0.00534487708167476\\
232.01	0.00534473329609347\\
233.01	0.00534458647799813\\
234.01	0.00534443656327662\\
235.01	0.00534428348645878\\
236.01	0.00534412718068752\\
237.01	0.00534396757768969\\
238.01	0.00534380460774562\\
239.01	0.005343638199659\\
240.01	0.00534346828072501\\
241.01	0.00534329477669881\\
242.01	0.0053431176117627\\
243.01	0.00534293670849275\\
244.01	0.00534275198782492\\
245.01	0.0053425633690203\\
246.01	0.00534237076962935\\
247.01	0.00534217410545625\\
248.01	0.0053419732905213\\
249.01	0.00534176823702359\\
250.01	0.0053415588553025\\
251.01	0.00534134505379794\\
252.01	0.00534112673901059\\
253.01	0.00534090381546105\\
254.01	0.00534067618564711\\
255.01	0.00534044375000232\\
256.01	0.00534020640685138\\
257.01	0.00533996405236632\\
258.01	0.00533971658052057\\
259.01	0.00533946388304308\\
260.01	0.00533920584937068\\
261.01	0.00533894236660026\\
262.01	0.00533867331943922\\
263.01	0.00533839859015534\\
264.01	0.00533811805852578\\
265.01	0.00533783160178449\\
266.01	0.00533753909456922\\
267.01	0.00533724040886725\\
268.01	0.00533693541395972\\
269.01	0.00533662397636562\\
270.01	0.00533630595978355\\
271.01	0.00533598122503373\\
272.01	0.00533564962999774\\
273.01	0.00533531102955811\\
274.01	0.00533496527553554\\
275.01	0.00533461221662648\\
276.01	0.00533425169833823\\
277.01	0.00533388356292384\\
278.01	0.00533350764931506\\
279.01	0.00533312379305492\\
280.01	0.0053327318262282\\
281.01	0.00533233157739187\\
282.01	0.00533192287150341\\
283.01	0.0053315055298484\\
284.01	0.00533107936996712\\
285.01	0.00533064420557965\\
286.01	0.00533019984651016\\
287.01	0.00532974609861\\
288.01	0.00532928276367988\\
289.01	0.0053288096393909\\
290.01	0.00532832651920482\\
291.01	0.00532783319229312\\
292.01	0.00532732944345536\\
293.01	0.0053268150530369\\
294.01	0.00532628979684592\\
295.01	0.00532575344606914\\
296.01	0.00532520576718839\\
297.01	0.00532464652189533\\
298.01	0.00532407546700686\\
299.01	0.00532349235438009\\
300.01	0.00532289693082711\\
301.01	0.00532228893803049\\
302.01	0.00532166811245883\\
303.01	0.00532103418528305\\
304.01	0.0053203868822939\\
305.01	0.00531972592381969\\
306.01	0.00531905102464723\\
307.01	0.00531836189394287\\
308.01	0.00531765823517729\\
309.01	0.0053169397460516\\
310.01	0.00531620611842825\\
311.01	0.00531545703826419\\
312.01	0.00531469218554937\\
313.01	0.00531391123424979\\
314.01	0.00531311385225667\\
315.01	0.00531229970134192\\
316.01	0.00531146843712099\\
317.01	0.00531061970902503\\
318.01	0.0053097531602816\\
319.01	0.00530886842790669\\
320.01	0.00530796514270932\\
321.01	0.00530704292930905\\
322.01	0.00530610140616988\\
323.01	0.00530514018564999\\
324.01	0.00530415887407184\\
325.01	0.00530315707181312\\
326.01	0.0053021343734214\\
327.01	0.00530109036775548\\
328.01	0.00530002463815613\\
329.01	0.00529893676264953\\
330.01	0.00529782631418644\\
331.01	0.00529669286092161\\
332.01	0.00529553596653746\\
333.01	0.00529435519061616\\
334.01	0.00529315008906588\\
335.01	0.00529192021460581\\
336.01	0.00529066511731612\\
337.01	0.00528938434525932\\
338.01	0.00528807744517931\\
339.01	0.00528674396328614\\
340.01	0.00528538344613292\\
341.01	0.00528399544159428\\
342.01	0.00528257949995408\\
343.01	0.00528113517511113\\
344.01	0.00527966202591209\\
345.01	0.00527815961762109\\
346.01	0.00527662752353379\\
347.01	0.00527506532674537\\
348.01	0.00527347262207958\\
349.01	0.00527184901818537\\
350.01	0.00527019413980575\\
351.01	0.0052685076302207\\
352.01	0.00526678915386272\\
353.01	0.00526503839909783\\
354.01	0.00526325508116023\\
355.01	0.00526143894521889\\
356.01	0.00525958976954443\\
357.01	0.00525770736873251\\
358.01	0.00525579159692007\\
359.01	0.0052538423509145\\
360.01	0.00525185957312387\\
361.01	0.00524984325415177\\
362.01	0.00524779343487412\\
363.01	0.0052457102077765\\
364.01	0.00524359371726887\\
365.01	0.005241444158633\\
366.01	0.00523926177518317\\
367.01	0.00523704685313197\\
368.01	0.00523479971356204\\
369.01	0.00523252070079974\\
370.01	0.00523021016638725\\
371.01	0.00522786844775066\\
372.01	0.00522549584059348\\
373.01	0.00522309256401391\\
374.01	0.00522065871740547\\
375.01	0.00521819422838948\\
376.01	0.00521569879144335\\
377.01	0.00521317179763731\\
378.01	0.00521061225715357\\
379.01	0.00520801871828652\\
380.01	0.00520538918976855\\
381.01	0.00520272107804836\\
382.01	0.00520001115829957\\
383.01	0.00519725560848613\\
384.01	0.00519445015123187\\
385.01	0.00519159039419118\\
386.01	0.00518867336361616\\
387.01	0.00518569786490309\\
388.01	0.00518266281706364\\
389.01	0.00517956712539902\\
390.01	0.00517640968182066\\
391.01	0.00517318936521948\\
392.01	0.00516990504188985\\
393.01	0.00516655556601164\\
394.01	0.00516313978019591\\
395.01	0.00515965651609796\\
396.01	0.00515610459510477\\
397.01	0.00515248282910091\\
398.01	0.00514879002132005\\
399.01	0.0051450249672888\\
400.01	0.00514118645586748\\
401.01	0.00513727327039889\\
402.01	0.00513328418996847\\
403.01	0.00512921799078697\\
404.01	0.00512507344770296\\
405.01	0.00512084933585352\\
406.01	0.00511654443246307\\
407.01	0.0051121575187992\\
408.01	0.00510768738229611\\
409.01	0.00510313281885506\\
410.01	0.00509849263533201\\
411.01	0.00509376565222385\\
412.01	0.00508895070656203\\
413.01	0.00508404665502501\\
414.01	0.00507905237727919\\
415.01	0.00507396677955732\\
416.01	0.00506878879848313\\
417.01	0.0050635174051499\\
418.01	0.00505815160945878\\
419.01	0.00505269046472162\\
420.01	0.0050471330725289\\
421.01	0.005041478587883\\
422.01	0.00503572622459047\\
423.01	0.00502987526090591\\
424.01	0.00502392504540896\\
425.01	0.00501787500309718\\
426.01	0.00501172464166102\\
427.01	0.00500547355790464\\
428.01	0.0049991214442613\\
429.01	0.00499266809533988\\
430.01	0.00498611341442513\\
431.01	0.00497945741983404\\
432.01	0.00497270025101373\\
433.01	0.00496584217424116\\
434.01	0.00495888358775831\\
435.01	0.00495182502614963\\
436.01	0.0049446671637332\\
437.01	0.00493741081670373\\
438.01	0.00493005694372453\\
439.01	0.00492260664462532\\
440.01	0.00491506115682083\\
441.01	0.00490742184902026\\
442.01	0.00489969021175484\\
443.01	0.00489186784421787\\
444.01	0.00488395643687475\\
445.01	0.00487595774929017\\
446.01	0.00486787358261556\\
447.01	0.00485970574621587\\
448.01	0.0048514560179773\\
449.01	0.00484312609796062\\
450.01	0.00483471755525123\\
451.01	0.00482623176813569\\
452.01	0.0048176698581155\\
453.01	0.00480903261880001\\
454.01	0.00480032044140021\\
455.01	0.00479153323943075\\
456.01	0.00478267037631216\\
457.01	0.0047737306008848\\
458.01	0.00476471199736208\\
459.01	0.00475561195792698\\
460.01	0.00474642718786406\\
461.01	0.00473715375459517\\
462.01	0.00472778719280914\\
463.01	0.0047183226773528\\
464.01	0.00470875527255164\\
465.01	0.0046990802593976\\
466.01	0.00468929352788079\\
467.01	0.00467939199655541\\
468.01	0.00466937397912064\\
469.01	0.00465923927801335\\
470.01	0.00464898837771446\\
471.01	0.00463862172466746\\
472.01	0.00462813964410839\\
473.01	0.00461754230514323\\
474.01	0.00460682968311623\\
475.01	0.00459600151964979\\
476.01	0.00458505728090664\\
477.01	0.00457399611483108\\
478.01	0.0045628168083555\\
479.01	0.00455151774581506\\
480.01	0.00454009687008133\\
481.01	0.00452855164820645\\
482.01	0.00451687904364322\\
483.01	0.00450507549734652\\
484.01	0.0044931369202116\\
485.01	0.00448105869933741\\
486.01	0.0044688357204274\\
487.01	0.0044564624081949\\
488.01	0.00444393278581005\\
489.01	0.004431240553115\\
490.01	0.00441837918142996\\
491.01	0.00440534202021139\\
492.01	0.00439212240759072\\
493.01	0.00437871377307201\\
494.01	0.00436510971680188\\
495.01	0.0043513040467084\\
496.01	0.00433729075408134\\
497.01	0.00432306391271775\\
498.01	0.00430861750177217\\
499.01	0.00429394520712976\\
500.01	0.00427904033761374\\
501.01	0.00426389583075258\\
502.01	0.00424850427227856\\
503.01	0.00423285792212514\\
504.01	0.00421694874645758\\
505.01	0.0042007684548149\\
506.01	0.00418430854090724\\
507.01	0.0041675603250459\\
508.01	0.00415051499562202\\
509.01	0.00413316364658496\\
510.01	0.00411549730760114\\
511.01	0.00409750696365053\\
512.01	0.00407918356141116\\
513.01	0.00406051800107318\\
514.01	0.00404150111435926\\
515.01	0.00402212363250523\\
516.01	0.00400237615150024\\
517.01	0.00398224910507929\\
518.01	0.00396173275631887\\
519.01	0.00394081720816598\\
520.01	0.00391949241931278\\
521.01	0.00389774821828174\\
522.01	0.00387557431448599\\
523.01	0.00385296030558535\\
524.01	0.00382989568070651\\
525.01	0.00380636981945661\\
526.01	0.0037823719871091\\
527.01	0.00375789132683927\\
528.01	0.0037329168503583\\
529.01	0.0037074374286157\\
530.01	0.00368144178426106\\
531.01	0.00365491848710693\\
532.01	0.00362785595281683\\
533.01	0.00360024244360208\\
534.01	0.00357206606891766\\
535.01	0.00354331478516688\\
536.01	0.00351397639449492\\
537.01	0.00348403854301973\\
538.01	0.00345348871893436\\
539.01	0.00342231425095655\\
540.01	0.00339050230758179\\
541.01	0.00335803989751079\\
542.01	0.00332491387146892\\
543.01	0.00329111092545075\\
544.01	0.00325661760525977\\
545.01	0.0032214203121659\\
546.01	0.00318550530962556\\
547.01	0.00314885873122135\\
548.01	0.00311146659008362\\
549.01	0.00307331479006465\\
550.01	0.00303438913891908\\
551.01	0.00299467536372011\\
552.01	0.00295415912871867\\
553.01	0.00291282605583985\\
554.01	0.00287066174801436\\
555.01	0.00282765181558129\\
556.01	0.00278378190604264\\
557.01	0.0027390377375069\\
558.01	0.00269340513618998\\
559.01	0.0026468700783629\\
560.01	0.00259941873715503\\
561.01	0.00255103753464357\\
562.01	0.00250171319969066\\
563.01	0.00245143283201888\\
564.01	0.00240018397305644\\
565.01	0.00234795468411658\\
566.01	0.00229473363250886\\
567.01	0.00224051018620399\\
568.01	0.00218527451769199\\
569.01	0.00212901771768357\\
570.01	0.00207173191930344\\
571.01	0.00201341043341097\\
572.01	0.00195404789564968\\
573.01	0.00189364042576395\\
574.01	0.00183218579962798\\
575.01	0.0017696836342872\\
576.01	0.00170613558611251\\
577.01	0.00164154556189476\\
578.01	0.00157591994233822\\
579.01	0.00150926781692674\\
580.01	0.00144160122850227\\
581.01	0.00137293542507396\\
582.01	0.00130328911532177\\
583.01	0.00123268472291186\\
584.01	0.00116114863303132\\
585.01	0.00108871142238989\\
586.01	0.00101540806121248\\
587.01	0.000941278072329516\\
588.01	0.000866365628191963\\
589.01	0.000790719561289255\\
590.01	0.000714393256781154\\
591.01	0.000637444387854076\\
592.01	0.000559934443995484\\
593.01	0.00048192798957512\\
594.01	0.000403491574243162\\
595.01	0.000324692196989477\\
596.01	0.000245595201369226\\
597.01	0.000166284632487157\\
598.01	9.1337981519295e-05\\
599.01	2.91271958372426e-05\\
599.02	2.86192783627518e-05\\
599.03	2.8114423742006e-05\\
599.04	2.76126617978888e-05\\
599.05	2.71140226472781e-05\\
599.06	2.66185367039581e-05\\
599.07	2.61262346815533e-05\\
599.08	2.56371475965082e-05\\
599.09	2.51513067710818e-05\\
599.1	2.46687438363886e-05\\
599.11	2.41894907354479e-05\\
599.12	2.3713579726279e-05\\
599.13	2.3241043385025e-05\\
599.14	2.27719146091033e-05\\
599.15	2.23062266203888e-05\\
599.16	2.18440129684284e-05\\
599.17	2.138530753369e-05\\
599.18	2.09301445308497e-05\\
599.19	2.04785585120829e-05\\
599.2	2.00305843704295e-05\\
599.21	1.95862573431644e-05\\
599.22	1.91456130152045e-05\\
599.23	1.87086873225609e-05\\
599.24	1.82755165558171e-05\\
599.25	1.7846137363638e-05\\
599.26	1.74205874164651e-05\\
599.27	1.69989072017502e-05\\
599.28	1.6581137610194e-05\\
599.29	1.61673199397579e-05\\
599.3	1.57574958996824e-05\\
599.31	1.53517076145714e-05\\
599.32	1.49499976284904e-05\\
599.33	1.45524089091333e-05\\
599.34	1.41589848520109e-05\\
599.35	1.37697692846831e-05\\
599.36	1.33848064710444e-05\\
599.37	1.30041411156422e-05\\
599.38	1.26278183680325e-05\\
599.39	1.22558838271912e-05\\
599.4	1.18883835459657e-05\\
599.41	1.15253640355657e-05\\
599.42	1.11668722700964e-05\\
599.43	1.08129556911519e-05\\
599.44	1.04636622124312e-05\\
599.45	1.01190402244222e-05\\
599.46	9.77913859911625e-06\\
599.47	9.44400669477576e-06\\
599.48	9.11369436074755e-06\\
599.49	8.7882519423238e-06\\
599.5	8.46773028565471e-06\\
599.51	8.15218074270464e-06\\
599.52	7.84165517625501e-06\\
599.53	7.5362059649732e-06\\
599.54	7.23588600849874e-06\\
599.55	6.94074873262146e-06\\
599.56	6.65084809447353e-06\\
599.57	6.36623858780126e-06\\
599.58	6.08697524826819e-06\\
599.59	5.81311365882228e-06\\
599.6	5.54470995511175e-06\\
599.61	5.28182083095637e-06\\
599.62	5.02450354387257e-06\\
599.63	4.77281592065407e-06\\
599.64	4.52681636300273e-06\\
599.65	4.28656385322197e-06\\
599.66	4.05211795996042e-06\\
599.67	3.8235388440163e-06\\
599.68	3.60088726420078e-06\\
599.69	3.38422458325514e-06\\
599.7	3.17361277382168e-06\\
599.71	2.96911442449442e-06\\
599.72	2.77079274589413e-06\\
599.73	2.5787115768474e-06\\
599.74	2.3929353905848e-06\\
599.75	2.213529301031e-06\\
599.76	2.04055906913823e-06\\
599.77	1.87409110929959e-06\\
599.78	1.71419249580043e-06\\
599.79	1.56093096936177e-06\\
599.8	1.41437494372704e-06\\
599.81	1.27459351233379e-06\\
599.82	1.14165645502366e-06\\
599.83	1.01563424484766e-06\\
599.84	8.96598054920053e-07\\
599.85	7.84619765348618e-07\\
599.86	6.79771970232487e-07\\
599.87	5.82127984719016e-07\\
599.88	4.91761852147374e-07\\
599.89	4.08748351251112e-07\\
599.9	3.33163003437068e-07\\
599.91	2.65082080131915e-07\\
599.92	2.04582610201579e-07\\
599.93	1.51742387450443e-07\\
599.94	1.06639978189951e-07\\
599.95	6.93547288800611e-08\\
599.96	3.99667738487652e-08\\
599.97	1.85570430914078e-08\\
599.98	5.20727013418598e-09\\
599.99	0\\
600	0\\
};
\addplot [color=mycolor7,solid,forget plot]
  table[row sep=crcr]{%
0.01	0.00559373691468814\\
1.01	0.00559373577872551\\
2.01	0.00559373461904626\\
3.01	0.00559373343515379\\
4.01	0.00559373222654108\\
5.01	0.0055937309926905\\
6.01	0.00559372973307341\\
7.01	0.0055937284471503\\
8.01	0.00559372713437013\\
9.01	0.00559372579417037\\
10.01	0.00559372442597669\\
11.01	0.00559372302920275\\
12.01	0.00559372160324958\\
13.01	0.00559372014750582\\
14.01	0.00559371866134741\\
15.01	0.00559371714413699\\
16.01	0.00559371559522367\\
17.01	0.00559371401394308\\
18.01	0.00559371239961689\\
19.01	0.00559371075155257\\
20.01	0.00559370906904269\\
21.01	0.00559370735136514\\
22.01	0.00559370559778273\\
23.01	0.00559370380754255\\
24.01	0.00559370197987606\\
25.01	0.00559370011399838\\
26.01	0.0055936982091083\\
27.01	0.00559369626438749\\
28.01	0.00559369427900063\\
29.01	0.00559369225209459\\
30.01	0.00559369018279852\\
31.01	0.00559368807022315\\
32.01	0.00559368591346039\\
33.01	0.0055936837115833\\
34.01	0.00559368146364506\\
35.01	0.00559367916867903\\
36.01	0.00559367682569837\\
37.01	0.00559367443369526\\
38.01	0.00559367199164091\\
39.01	0.00559366949848463\\
40.01	0.00559366695315376\\
41.01	0.00559366435455282\\
42.01	0.00559366170156327\\
43.01	0.00559365899304335\\
44.01	0.00559365622782688\\
45.01	0.00559365340472309\\
46.01	0.00559365052251629\\
47.01	0.00559364757996523\\
48.01	0.00559364457580231\\
49.01	0.00559364150873323\\
50.01	0.00559363837743648\\
51.01	0.00559363518056259\\
52.01	0.00559363191673375\\
53.01	0.00559362858454308\\
54.01	0.00559362518255385\\
55.01	0.00559362170929927\\
56.01	0.00559361816328135\\
57.01	0.00559361454297081\\
58.01	0.00559361084680584\\
59.01	0.00559360707319178\\
60.01	0.00559360322050039\\
61.01	0.00559359928706888\\
62.01	0.00559359527119959\\
63.01	0.0055935911711589\\
64.01	0.00559358698517652\\
65.01	0.00559358271144486\\
66.01	0.00559357834811817\\
67.01	0.00559357389331192\\
68.01	0.00559356934510149\\
69.01	0.00559356470152179\\
70.01	0.00559355996056631\\
71.01	0.00559355512018576\\
72.01	0.00559355017828785\\
73.01	0.00559354513273621\\
74.01	0.00559353998134907\\
75.01	0.00559353472189878\\
76.01	0.00559352935211054\\
77.01	0.00559352386966131\\
78.01	0.00559351827217927\\
79.01	0.00559351255724233\\
80.01	0.00559350672237696\\
81.01	0.00559350076505762\\
82.01	0.00559349468270554\\
83.01	0.00559348847268699\\
84.01	0.00559348213231282\\
85.01	0.00559347565883708\\
86.01	0.00559346904945529\\
87.01	0.00559346230130425\\
88.01	0.00559345541145987\\
89.01	0.00559344837693624\\
90.01	0.00559344119468431\\
91.01	0.00559343386159048\\
92.01	0.00559342637447532\\
93.01	0.00559341873009215\\
94.01	0.00559341092512545\\
95.01	0.00559340295618965\\
96.01	0.00559339481982745\\
97.01	0.00559338651250837\\
98.01	0.00559337803062708\\
99.01	0.00559336937050224\\
100.01	0.00559336052837412\\
101.01	0.00559335150040379\\
102.01	0.00559334228267094\\
103.01	0.00559333287117187\\
104.01	0.00559332326181835\\
105.01	0.00559331345043552\\
106.01	0.00559330343276005\\
107.01	0.00559329320443834\\
108.01	0.00559328276102447\\
109.01	0.00559327209797834\\
110.01	0.00559326121066336\\
111.01	0.00559325009434523\\
112.01	0.00559323874418876\\
113.01	0.00559322715525666\\
114.01	0.00559321532250665\\
115.01	0.00559320324078976\\
116.01	0.00559319090484779\\
117.01	0.0055931783093107\\
118.01	0.00559316544869513\\
119.01	0.00559315231740108\\
120.01	0.0055931389097097\\
121.01	0.00559312521978086\\
122.01	0.00559311124165062\\
123.01	0.00559309696922845\\
124.01	0.00559308239629466\\
125.01	0.0055930675164974\\
126.01	0.00559305232335015\\
127.01	0.00559303681022853\\
128.01	0.00559302097036774\\
129.01	0.00559300479685925\\
130.01	0.00559298828264788\\
131.01	0.00559297142052857\\
132.01	0.00559295420314327\\
133.01	0.00559293662297771\\
134.01	0.00559291867235801\\
135.01	0.0055929003434471\\
136.01	0.00559288162824178\\
137.01	0.00559286251856859\\
138.01	0.00559284300608037\\
139.01	0.0055928230822526\\
140.01	0.00559280273837951\\
141.01	0.0055927819655706\\
142.01	0.00559276075474589\\
143.01	0.005592739096633\\
144.01	0.0055927169817618\\
145.01	0.00559269440046124\\
146.01	0.00559267134285437\\
147.01	0.00559264779885437\\
148.01	0.00559262375815965\\
149.01	0.00559259921024959\\
150.01	0.00559257414437988\\
151.01	0.00559254854957762\\
152.01	0.0055925224146361\\
153.01	0.00559249572811038\\
154.01	0.00559246847831204\\
155.01	0.00559244065330381\\
156.01	0.00559241224089418\\
157.01	0.00559238322863223\\
158.01	0.00559235360380172\\
159.01	0.00559232335341602\\
160.01	0.00559229246421162\\
161.01	0.00559226092264278\\
162.01	0.00559222871487498\\
163.01	0.00559219582677904\\
164.01	0.00559216224392503\\
165.01	0.00559212795157504\\
166.01	0.0055920929346775\\
167.01	0.00559205717786007\\
168.01	0.00559202066542245\\
169.01	0.00559198338132997\\
170.01	0.00559194530920586\\
171.01	0.00559190643232435\\
172.01	0.00559186673360279\\
173.01	0.00559182619559434\\
174.01	0.00559178480048005\\
175.01	0.00559174253006067\\
176.01	0.00559169936574867\\
177.01	0.00559165528856021\\
178.01	0.00559161027910601\\
179.01	0.00559156431758309\\
180.01	0.00559151738376559\\
181.01	0.00559146945699598\\
182.01	0.00559142051617567\\
183.01	0.00559137053975526\\
184.01	0.00559131950572555\\
185.01	0.00559126739160681\\
186.01	0.00559121417443915\\
187.01	0.00559115983077203\\
188.01	0.00559110433665382\\
189.01	0.00559104766762112\\
190.01	0.00559098979868724\\
191.01	0.00559093070433161\\
192.01	0.0055908703584877\\
193.01	0.00559080873453166\\
194.01	0.00559074580526974\\
195.01	0.00559068154292718\\
196.01	0.00559061591913427\\
197.01	0.00559054890491436\\
198.01	0.00559048047067044\\
199.01	0.00559041058617194\\
200.01	0.00559033922054112\\
201.01	0.00559026634223887\\
202.01	0.00559019191905027\\
203.01	0.00559011591807045\\
204.01	0.00559003830568935\\
205.01	0.00558995904757671\\
206.01	0.00558987810866597\\
207.01	0.00558979545313911\\
208.01	0.00558971104440954\\
209.01	0.00558962484510629\\
210.01	0.00558953681705603\\
211.01	0.00558944692126669\\
212.01	0.00558935511790911\\
213.01	0.00558926136629889\\
214.01	0.00558916562487796\\
215.01	0.0055890678511956\\
216.01	0.00558896800188891\\
217.01	0.00558886603266324\\
218.01	0.00558876189827153\\
219.01	0.00558865555249388\\
220.01	0.00558854694811616\\
221.01	0.00558843603690878\\
222.01	0.00558832276960387\\
223.01	0.00558820709587309\\
224.01	0.00558808896430441\\
225.01	0.00558796832237821\\
226.01	0.00558784511644343\\
227.01	0.00558771929169254\\
228.01	0.00558759079213623\\
229.01	0.00558745956057783\\
230.01	0.00558732553858644\\
231.01	0.00558718866647005\\
232.01	0.00558704888324787\\
233.01	0.005586906126622\\
234.01	0.00558676033294844\\
235.01	0.00558661143720742\\
236.01	0.0055864593729732\\
237.01	0.00558630407238297\\
238.01	0.00558614546610514\\
239.01	0.005585983483307\\
240.01	0.00558581805162162\\
241.01	0.00558564909711353\\
242.01	0.00558547654424432\\
243.01	0.00558530031583694\\
244.01	0.00558512033303952\\
245.01	0.00558493651528768\\
246.01	0.00558474878026711\\
247.01	0.00558455704387397\\
248.01	0.00558436122017529\\
249.01	0.00558416122136823\\
250.01	0.00558395695773794\\
251.01	0.00558374833761504\\
252.01	0.00558353526733187\\
253.01	0.00558331765117725\\
254.01	0.00558309539135116\\
255.01	0.005582868387917\\
256.01	0.00558263653875407\\
257.01	0.00558239973950781\\
258.01	0.00558215788353952\\
259.01	0.00558191086187471\\
260.01	0.00558165856314998\\
261.01	0.00558140087355866\\
262.01	0.0055811376767955\\
263.01	0.00558086885399948\\
264.01	0.00558059428369524\\
265.01	0.00558031384173375\\
266.01	0.00558002740123074\\
267.01	0.00557973483250377\\
268.01	0.00557943600300798\\
269.01	0.00557913077726986\\
270.01	0.00557881901682009\\
271.01	0.00557850058012333\\
272.01	0.00557817532250776\\
273.01	0.0055778430960915\\
274.01	0.00557750374970861\\
275.01	0.00557715712883115\\
276.01	0.00557680307549146\\
277.01	0.00557644142820068\\
278.01	0.00557607202186592\\
279.01	0.00557569468770562\\
280.01	0.00557530925316167\\
281.01	0.00557491554180995\\
282.01	0.00557451337326819\\
283.01	0.00557410256310188\\
284.01	0.00557368292272669\\
285.01	0.00557325425930905\\
286.01	0.00557281637566413\\
287.01	0.00557236907015028\\
288.01	0.00557191213656099\\
289.01	0.00557144536401445\\
290.01	0.00557096853683888\\
291.01	0.00557048143445591\\
292.01	0.00556998383125993\\
293.01	0.0055694754964938\\
294.01	0.00556895619412241\\
295.01	0.00556842568270146\\
296.01	0.00556788371524252\\
297.01	0.00556733003907518\\
298.01	0.00556676439570424\\
299.01	0.00556618652066356\\
300.01	0.00556559614336517\\
301.01	0.00556499298694486\\
302.01	0.00556437676810192\\
303.01	0.00556374719693583\\
304.01	0.00556310397677668\\
305.01	0.00556244680401232\\
306.01	0.00556177536790905\\
307.01	0.00556108935042748\\
308.01	0.00556038842603379\\
309.01	0.00555967226150454\\
310.01	0.00555894051572644\\
311.01	0.00555819283949\\
312.01	0.00555742887527775\\
313.01	0.00555664825704568\\
314.01	0.00555585060999905\\
315.01	0.00555503555036238\\
316.01	0.00555420268514182\\
317.01	0.00555335161188214\\
318.01	0.00555248191841649\\
319.01	0.00555159318261087\\
320.01	0.00555068497210045\\
321.01	0.00554975684402067\\
322.01	0.00554880834473149\\
323.01	0.00554783900953537\\
324.01	0.00554684836238993\\
325.01	0.00554583591561429\\
326.01	0.00554480116959046\\
327.01	0.00554374361246058\\
328.01	0.00554266271981949\\
329.01	0.00554155795440458\\
330.01	0.00554042876578421\\
331.01	0.00553927459004481\\
332.01	0.00553809484947935\\
333.01	0.00553688895227882\\
334.01	0.00553565629222924\\
335.01	0.00553439624841643\\
336.01	0.00553310818494279\\
337.01	0.00553179145066014\\
338.01	0.00553044537892288\\
339.01	0.00552906928736697\\
340.01	0.00552766247772401\\
341.01	0.00552622423567475\\
342.01	0.00552475383075512\\
343.01	0.00552325051632485\\
344.01	0.00552171352961172\\
345.01	0.00552014209184854\\
346.01	0.0055185354085209\\
347.01	0.00551689266974859\\
348.01	0.00551521305082619\\
349.01	0.00551349571295421\\
350.01	0.00551173980419672\\
351.01	0.00550994446070776\\
352.01	0.00550810880827644\\
353.01	0.00550623196424911\\
354.01	0.00550431303989498\\
355.01	0.00550235114329564\\
356.01	0.00550034538284874\\
357.01	0.00549829487149175\\
358.01	0.00549619873176579\\
359.01	0.00549405610185998\\
360.01	0.00549186614279095\\
361.01	0.00548962804689502\\
362.01	0.00548734104782902\\
363.01	0.00548500443229107\\
364.01	0.00548261755369217\\
365.01	0.00548017984801175\\
366.01	0.00547769085207107\\
367.01	0.00547515022443356\\
368.01	0.00547255776909233\\
369.01	0.00546991346201305\\
370.01	0.00546721748044338\\
371.01	0.00546447023465396\\
372.01	0.00546167240139801\\
373.01	0.00545882495780955\\
374.01	0.00545592921362698\\
375.01	0.00545298683841993\\
376.01	0.00544999987875847\\
377.01	0.0054469707577876\\
378.01	0.00544390224615777\\
379.01	0.00544079738832047\\
380.01	0.00543765936125108\\
381.01	0.00543449123296023\\
382.01	0.00543129557461255\\
383.01	0.00542807386124254\\
384.01	0.0054248255699084\\
385.01	0.00542154473694886\\
386.01	0.00541820252034409\\
387.01	0.00541479053797377\\
388.01	0.00541130735626098\\
389.01	0.00540775151427604\\
390.01	0.00540412152342795\\
391.01	0.00540041586717635\\
392.01	0.00539663300076763\\
393.01	0.00539277135099786\\
394.01	0.00538882931600653\\
395.01	0.00538480526510645\\
396.01	0.00538069753865354\\
397.01	0.00537650444796261\\
398.01	0.00537222427527503\\
399.01	0.00536785527378419\\
400.01	0.00536339566772774\\
401.01	0.00535884365255251\\
402.01	0.00535419739516333\\
403.01	0.00534945503426495\\
404.01	0.00534461468080661\\
405.01	0.00533967441854422\\
406.01	0.00533463230473173\\
407.01	0.00532948637095737\\
408.01	0.00532423462414129\\
409.01	0.00531887504771252\\
410.01	0.00531340560298583\\
411.01	0.00530782423076023\\
412.01	0.00530212885316444\\
413.01	0.00529631737577533\\
414.01	0.00529038769003946\\
415.01	0.00528433767603014\\
416.01	0.00527816520557665\\
417.01	0.00527186814580291\\
418.01	0.00526544436311995\\
419.01	0.00525889172771916\\
420.01	0.00525220811861548\\
421.01	0.00524539142929793\\
422.01	0.00523843957404536\\
423.01	0.00523135049497339\\
424.01	0.0052241221698815\\
425.01	0.00521675262097326\\
426.01	0.00520923992453015\\
427.01	0.00520158222161964\\
428.01	0.00519377772992578\\
429.01	0.00518582475678883\\
430.01	0.00517772171354748\\
431.01	0.00516946713127028\\
432.01	0.00516105967796527\\
433.01	0.00515249817734653\\
434.01	0.00514378162922886\\
435.01	0.00513490923160295\\
436.01	0.00512588040441994\\
437.01	0.00511669481508393\\
438.01	0.00510735240560287\\
439.01	0.00509785342129101\\
440.01	0.00508819844083878\\
441.01	0.00507838840746606\\
442.01	0.00506842466074768\\
443.01	0.0050583089685407\\
444.01	0.00504804355824356\\
445.01	0.00503763114636934\\
446.01	0.00502707496511496\\
447.01	0.00501637878424186\\
448.01	0.00500554692614615\\
449.01	0.00499458427147987\\
450.01	0.00498349625208561\\
451.01	0.00497228882732307\\
452.01	0.00496096843911425\\
453.01	0.00494954194022756\\
454.01	0.00493801648952166\\
455.01	0.0049263994071451\\
456.01	0.00491469798218742\\
457.01	0.00490291922520205\\
458.01	0.00489106955871362\\
459.01	0.00487915444076691\\
460.01	0.00486717792052282\\
461.01	0.00485514213194154\\
462.01	0.00484304674330157\\
463.01	0.0048308883989877\\
464.01	0.00481866021894127\\
465.01	0.00480635146512563\\
466.01	0.00479394755005908\\
467.01	0.00478143065944836\\
468.01	0.00476878140324357\\
469.01	0.00475598556443265\\
470.01	0.00474304404666418\\
471.01	0.0047299610429181\\
472.01	0.00471674105847669\\
473.01	0.00470338888298013\\
474.01	0.00468990955240711\\
475.01	0.00467630829953881\\
476.01	0.00466259049144648\\
477.01	0.00464876155260209\\
478.01	0.0046348268723669\\
479.01	0.00462079169591118\\
480.01	0.00460666099807384\\
481.01	0.00459243934021112\\
482.01	0.00457813071094355\\
483.01	0.00456373835303028\\
484.01	0.0045492645803004\\
485.01	0.00453471059073541\\
486.01	0.00452007628449796\\
487.01	0.00450536009895463\\
488.01	0.00449055887648122\\
489.01	0.00447566778489476\\
490.01	0.0044606803143166\\
491.01	0.0044455883773876\\
492.01	0.00443038254075207\\
493.01	0.00441505241263782\\
494.01	0.00439958720112588\\
495.01	0.00438397643509666\\
496.01	0.00436821079696735\\
497.01	0.0043522829414424\\
498.01	0.00433618801951533\\
499.01	0.00431992275040816\\
500.01	0.00430348341949066\\
501.01	0.00428686543395026\\
502.01	0.00427006327273361\\
503.01	0.00425307045498408\\
504.01	0.00423587953069189\\
505.01	0.00421848209825503\\
506.01	0.00420086885319224\\
507.01	0.00418302967118661\\
508.01	0.00416495372677486\\
509.01	0.00414662964612302\\
510.01	0.00412804568826002\\
511.01	0.00410918994379239\\
512.01	0.00409005053361105\\
513.01	0.00407061578295647\\
514.01	0.00405087433967378\\
515.01	0.00403081520207299\\
516.01	0.00401042762599367\\
517.01	0.00398970090041908\\
518.01	0.00396862405249262\\
519.01	0.00394718570682074\\
520.01	0.00392537412349316\\
521.01	0.00390317726577278\\
522.01	0.00388058287172507\\
523.01	0.00385757852559785\\
524.01	0.00383415172398894\\
525.01	0.00381028993129461\\
526.01	0.00378598061897748\\
527.01	0.00376121128411935\\
528.01	0.00373596944483621\\
529.01	0.00371024261366338\\
530.01	0.00368401825497795\\
531.01	0.00365728373835029\\
532.01	0.00363002630467935\\
533.01	0.00360223306048554\\
534.01	0.00357389099454814\\
535.01	0.00354498699607012\\
536.01	0.00351550786737225\\
537.01	0.00348544033040776\\
538.01	0.00345477102722176\\
539.01	0.00342348651524043\\
540.01	0.00339157325907328\\
541.01	0.00335901762118412\\
542.01	0.00332580585411584\\
543.01	0.00329192409665191\\
544.01	0.00325735837513591\\
545.01	0.00322209460913947\\
546.01	0.00318611861885679\\
547.01	0.00314941613236037\\
548.01	0.00311197279279638\\
549.01	0.00307377416623498\\
550.01	0.00303480575098859\\
551.01	0.00299505298919984\\
552.01	0.00295450128137567\\
553.01	0.00291313600431742\\
554.01	0.00287094253261798\\
555.01	0.00282790626366628\\
556.01	0.00278401264606778\\
557.01	0.00273924721162163\\
558.01	0.00269359561129114\\
559.01	0.00264704365570551\\
560.01	0.0025995773607181\\
561.01	0.00255118299851699\\
562.01	0.00250184715474731\\
563.01	0.00245155679209126\\
564.01	0.00240029932076293\\
565.01	0.00234806267642719\\
566.01	0.00229483540612065\\
567.01	0.00224060676280948\\
568.01	0.0021853668092348\\
569.01	0.00212910653169041\\
570.01	0.00207181796436378\\
571.01	0.00201349432484503\\
572.01	0.00195413016137354\\
573.01	0.00189372151233586\\
574.01	0.00183226607843893\\
575.01	0.00176976340784625\\
576.01	0.00170621509436756\\
577.01	0.00164162498851609\\
578.01	0.00157599942088352\\
579.01	0.00150934743679982\\
580.01	0.00144168104061919\\
581.01	0.00137301544715818\\
582.01	0.00130336933676586\\
583.01	0.00123276510916492\\
584.01	0.00116122912949844\\
585.01	0.00108879195785878\\
586.01	0.00101548855085613\\
587.01	0.000941358420368682\\
588.01	0.000866445730334548\\
589.01	0.000790799307099266\\
590.01	0.000714472532160199\\
591.01	0.000637523077847722\\
592.01	0.000560012436164931\\
593.01	0.000482005178202047\\
594.01	0.000403567865669439\\
595.01	0.000324767516434078\\
596.01	0.00024566950161945\\
597.01	0.000166349346609286\\
598.01	9.1337981519295e-05\\
599.01	2.91271958372443e-05\\
599.02	2.86192783627535e-05\\
599.03	2.81144237420042e-05\\
599.04	2.76126617978888e-05\\
599.05	2.71140226472798e-05\\
599.06	2.66185367039599e-05\\
599.07	2.6126234681555e-05\\
599.08	2.56371475965064e-05\\
599.09	2.51513067710818e-05\\
599.1	2.46687438363903e-05\\
599.11	2.41894907354479e-05\\
599.12	2.3713579726279e-05\\
599.13	2.32410433850267e-05\\
599.14	2.27719146091033e-05\\
599.15	2.23062266203871e-05\\
599.16	2.18440129684267e-05\\
599.17	2.13853075336917e-05\\
599.18	2.0930144530848e-05\\
599.19	2.04785585120812e-05\\
599.2	2.00305843704295e-05\\
599.21	1.95862573431627e-05\\
599.22	1.91456130152028e-05\\
599.23	1.87086873225627e-05\\
599.24	1.82755165558171e-05\\
599.25	1.78461373636363e-05\\
599.26	1.74205874164668e-05\\
599.27	1.69989072017485e-05\\
599.28	1.6581137610194e-05\\
599.29	1.61673199397562e-05\\
599.3	1.57574958996824e-05\\
599.31	1.53517076145714e-05\\
599.32	1.49499976284904e-05\\
599.33	1.45524089091333e-05\\
599.34	1.41589848520109e-05\\
599.35	1.37697692846814e-05\\
599.36	1.33848064710462e-05\\
599.37	1.3004141115644e-05\\
599.38	1.26278183680325e-05\\
599.39	1.22558838271912e-05\\
599.4	1.18883835459674e-05\\
599.41	1.15253640355657e-05\\
599.42	1.11668722700964e-05\\
599.43	1.08129556911502e-05\\
599.44	1.04636622124312e-05\\
599.45	1.01190402244239e-05\\
599.46	9.77913859911798e-06\\
599.47	9.44400669477576e-06\\
599.48	9.11369436074581e-06\\
599.49	8.7882519423238e-06\\
599.5	8.46773028565471e-06\\
599.51	8.15218074270464e-06\\
599.52	7.84165517625675e-06\\
599.53	7.5362059649732e-06\\
599.54	7.23588600849874e-06\\
599.55	6.94074873261973e-06\\
599.56	6.65084809447353e-06\\
599.57	6.36623858780126e-06\\
599.58	6.08697524826819e-06\\
599.59	5.81311365882228e-06\\
599.6	5.54470995511175e-06\\
599.61	5.28182083095637e-06\\
599.62	5.02450354387431e-06\\
599.63	4.7728159206558e-06\\
599.64	4.52681636300446e-06\\
599.65	4.28656385322197e-06\\
599.66	4.05211795995869e-06\\
599.67	3.82353884401457e-06\\
599.68	3.60088726420078e-06\\
599.69	3.38422458325341e-06\\
599.7	3.17361277382168e-06\\
599.71	2.96911442449269e-06\\
599.72	2.77079274589413e-06\\
599.73	2.57871157684567e-06\\
599.74	2.39293539058306e-06\\
599.75	2.21352930102926e-06\\
599.76	2.04055906913997e-06\\
599.77	1.87409110929959e-06\\
599.78	1.71419249580043e-06\\
599.79	1.56093096936177e-06\\
599.8	1.41437494372877e-06\\
599.81	1.27459351233379e-06\\
599.82	1.14165645502366e-06\\
599.83	1.01563424484592e-06\\
599.84	8.96598054920053e-07\\
599.85	7.84619765350353e-07\\
599.86	6.79771970232487e-07\\
599.87	5.82127984717282e-07\\
599.88	4.91761852145639e-07\\
599.89	4.08748351251112e-07\\
599.9	3.33163003438802e-07\\
599.91	2.65082080131915e-07\\
599.92	2.04582610201579e-07\\
599.93	1.51742387450443e-07\\
599.94	1.06639978191686e-07\\
599.95	6.93547288800611e-08\\
599.96	3.99667738487652e-08\\
599.97	1.85570430896731e-08\\
599.98	5.20727013418598e-09\\
599.99	0\\
600	0\\
};
\addplot [color=mycolor8,solid,forget plot]
  table[row sep=crcr]{%
0.01	0.00586354776097545\\
1.01	0.00586354699753364\\
2.01	0.0058635462181614\\
3.01	0.00586354542252556\\
4.01	0.00586354461028569\\
5.01	0.00586354378109457\\
6.01	0.00586354293459732\\
7.01	0.00586354207043188\\
8.01	0.00586354118822833\\
9.01	0.00586354028760917\\
10.01	0.00586353936818904\\
11.01	0.00586353842957432\\
12.01	0.00586353747136338\\
13.01	0.00586353649314581\\
14.01	0.0058635354945027\\
15.01	0.00586353447500645\\
16.01	0.00586353343422046\\
17.01	0.00586353237169907\\
18.01	0.00586353128698678\\
19.01	0.00586353017961888\\
20.01	0.00586352904912117\\
21.01	0.00586352789500907\\
22.01	0.00586352671678787\\
23.01	0.00586352551395256\\
24.01	0.00586352428598749\\
25.01	0.00586352303236609\\
26.01	0.00586352175255092\\
27.01	0.00586352044599312\\
28.01	0.00586351911213232\\
29.01	0.00586351775039639\\
30.01	0.00586351636020111\\
31.01	0.00586351494094995\\
32.01	0.00586351349203392\\
33.01	0.00586351201283122\\
34.01	0.00586351050270696\\
35.01	0.00586350896101282\\
36.01	0.00586350738708712\\
37.01	0.0058635057802541\\
38.01	0.00586350413982343\\
39.01	0.0058635024650908\\
40.01	0.00586350075533689\\
41.01	0.00586349900982698\\
42.01	0.00586349722781158\\
43.01	0.00586349540852465\\
44.01	0.00586349355118437\\
45.01	0.00586349165499278\\
46.01	0.0058634897191345\\
47.01	0.00586348774277738\\
48.01	0.00586348572507183\\
49.01	0.00586348366515013\\
50.01	0.00586348156212629\\
51.01	0.00586347941509586\\
52.01	0.00586347722313541\\
53.01	0.00586347498530169\\
54.01	0.005863472700632\\
55.01	0.00586347036814312\\
56.01	0.00586346798683123\\
57.01	0.00586346555567125\\
58.01	0.00586346307361665\\
59.01	0.00586346053959878\\
60.01	0.00586345795252653\\
61.01	0.0058634553112857\\
62.01	0.00586345261473877\\
63.01	0.00586344986172414\\
64.01	0.00586344705105583\\
65.01	0.0058634441815228\\
66.01	0.00586344125188856\\
67.01	0.0058634382608904\\
68.01	0.00586343520723901\\
69.01	0.00586343208961823\\
70.01	0.00586342890668371\\
71.01	0.0058634256570632\\
72.01	0.00586342233935517\\
73.01	0.00586341895212866\\
74.01	0.00586341549392263\\
75.01	0.00586341196324509\\
76.01	0.00586340835857256\\
77.01	0.00586340467834979\\
78.01	0.00586340092098829\\
79.01	0.00586339708486616\\
80.01	0.00586339316832763\\
81.01	0.00586338916968147\\
82.01	0.00586338508720121\\
83.01	0.00586338091912381\\
84.01	0.00586337666364898\\
85.01	0.00586337231893833\\
86.01	0.00586336788311515\\
87.01	0.00586336335426255\\
88.01	0.00586335873042362\\
89.01	0.00586335400960003\\
90.01	0.00586334918975102\\
91.01	0.00586334426879306\\
92.01	0.00586333924459848\\
93.01	0.00586333411499474\\
94.01	0.00586332887776354\\
95.01	0.00586332353063956\\
96.01	0.00586331807130985\\
97.01	0.00586331249741234\\
98.01	0.00586330680653548\\
99.01	0.00586330099621634\\
100.01	0.00586329506394038\\
101.01	0.0058632890071395\\
102.01	0.0058632828231916\\
103.01	0.00586327650941947\\
104.01	0.00586327006308891\\
105.01	0.00586326348140817\\
106.01	0.00586325676152639\\
107.01	0.00586324990053264\\
108.01	0.00586324289545428\\
109.01	0.00586323574325608\\
110.01	0.00586322844083854\\
111.01	0.00586322098503655\\
112.01	0.00586321337261819\\
113.01	0.00586320560028331\\
114.01	0.00586319766466197\\
115.01	0.00586318956231267\\
116.01	0.00586318128972157\\
117.01	0.00586317284330049\\
118.01	0.00586316421938505\\
119.01	0.0058631554142335\\
120.01	0.00586314642402532\\
121.01	0.00586313724485888\\
122.01	0.00586312787274997\\
123.01	0.00586311830363034\\
124.01	0.0058631085333453\\
125.01	0.00586309855765268\\
126.01	0.00586308837222042\\
127.01	0.00586307797262473\\
128.01	0.00586306735434818\\
129.01	0.00586305651277781\\
130.01	0.00586304544320323\\
131.01	0.00586303414081391\\
132.01	0.00586302260069802\\
133.01	0.00586301081783939\\
134.01	0.00586299878711581\\
135.01	0.00586298650329675\\
136.01	0.00586297396104055\\
137.01	0.00586296115489279\\
138.01	0.00586294807928361\\
139.01	0.00586293472852487\\
140.01	0.00586292109680837\\
141.01	0.00586290717820243\\
142.01	0.00586289296665033\\
143.01	0.00586287845596634\\
144.01	0.00586286363983437\\
145.01	0.00586284851180404\\
146.01	0.00586283306528856\\
147.01	0.00586281729356154\\
148.01	0.00586280118975397\\
149.01	0.00586278474685129\\
150.01	0.00586276795769033\\
151.01	0.00586275081495593\\
152.01	0.00586273331117821\\
153.01	0.00586271543872859\\
154.01	0.00586269718981692\\
155.01	0.00586267855648781\\
156.01	0.0058626595306172\\
157.01	0.00586264010390873\\
158.01	0.00586262026789011\\
159.01	0.00586260001390948\\
160.01	0.00586257933313123\\
161.01	0.00586255821653237\\
162.01	0.00586253665489849\\
163.01	0.00586251463881976\\
164.01	0.00586249215868649\\
165.01	0.00586246920468516\\
166.01	0.00586244576679403\\
167.01	0.00586242183477841\\
168.01	0.0058623973981865\\
169.01	0.00586237244634459\\
170.01	0.00586234696835229\\
171.01	0.00586232095307766\\
172.01	0.00586229438915253\\
173.01	0.00586226726496695\\
174.01	0.00586223956866466\\
175.01	0.00586221128813726\\
176.01	0.00586218241101915\\
177.01	0.0058621529246819\\
178.01	0.00586212281622859\\
179.01	0.0058620920724884\\
180.01	0.00586206068001038\\
181.01	0.00586202862505758\\
182.01	0.00586199589360082\\
183.01	0.00586196247131266\\
184.01	0.00586192834356069\\
185.01	0.00586189349540151\\
186.01	0.00586185791157337\\
187.01	0.00586182157649009\\
188.01	0.00586178447423335\\
189.01	0.00586174658854625\\
190.01	0.00586170790282575\\
191.01	0.00586166840011501\\
192.01	0.00586162806309602\\
193.01	0.00586158687408193\\
194.01	0.00586154481500916\\
195.01	0.00586150186742886\\
196.01	0.00586145801249904\\
197.01	0.00586141323097598\\
198.01	0.00586136750320569\\
199.01	0.00586132080911484\\
200.01	0.00586127312820189\\
201.01	0.00586122443952771\\
202.01	0.00586117472170627\\
203.01	0.00586112395289524\\
204.01	0.00586107211078559\\
205.01	0.00586101917259156\\
206.01	0.005860965115041\\
207.01	0.00586090991436422\\
208.01	0.00586085354628339\\
209.01	0.0058607959860015\\
210.01	0.00586073720819117\\
211.01	0.00586067718698318\\
212.01	0.00586061589595468\\
213.01	0.00586055330811698\\
214.01	0.00586048939590358\\
215.01	0.00586042413115736\\
216.01	0.00586035748511773\\
217.01	0.00586028942840742\\
218.01	0.00586021993101945\\
219.01	0.00586014896230283\\
220.01	0.00586007649094895\\
221.01	0.00586000248497656\\
222.01	0.005859926911718\\
223.01	0.00585984973780342\\
224.01	0.00585977092914556\\
225.01	0.00585969045092385\\
226.01	0.00585960826756889\\
227.01	0.00585952434274532\\
228.01	0.00585943863933527\\
229.01	0.00585935111942087\\
230.01	0.00585926174426709\\
231.01	0.00585917047430313\\
232.01	0.00585907726910411\\
233.01	0.00585898208737228\\
234.01	0.00585888488691746\\
235.01	0.00585878562463736\\
236.01	0.00585868425649727\\
237.01	0.00585858073750922\\
238.01	0.00585847502171063\\
239.01	0.00585836706214264\\
240.01	0.00585825681082793\\
241.01	0.0058581442187477\\
242.01	0.0058580292358183\\
243.01	0.00585791181086724\\
244.01	0.00585779189160882\\
245.01	0.00585766942461882\\
246.01	0.00585754435530883\\
247.01	0.00585741662789988\\
248.01	0.00585728618539547\\
249.01	0.00585715296955359\\
250.01	0.00585701692085854\\
251.01	0.00585687797849164\\
252.01	0.00585673608030186\\
253.01	0.00585659116277448\\
254.01	0.00585644316100028\\
255.01	0.00585629200864336\\
256.01	0.00585613763790769\\
257.01	0.00585597997950364\\
258.01	0.00585581896261326\\
259.01	0.00585565451485434\\
260.01	0.00585548656224446\\
261.01	0.00585531502916256\\
262.01	0.0058551398383112\\
263.01	0.00585496091067647\\
264.01	0.00585477816548789\\
265.01	0.00585459152017635\\
266.01	0.00585440089033135\\
267.01	0.00585420618965691\\
268.01	0.0058540073299266\\
269.01	0.0058538042209368\\
270.01	0.00585359677045896\\
271.01	0.00585338488419084\\
272.01	0.0058531684657055\\
273.01	0.00585294741639975\\
274.01	0.00585272163544034\\
275.01	0.00585249101970952\\
276.01	0.00585225546374804\\
277.01	0.00585201485969681\\
278.01	0.00585176909723766\\
279.01	0.00585151806353051\\
280.01	0.00585126164315069\\
281.01	0.00585099971802295\\
282.01	0.00585073216735405\\
283.01	0.00585045886756262\\
284.01	0.00585017969220821\\
285.01	0.00584989451191661\\
286.01	0.00584960319430333\\
287.01	0.00584930560389508\\
288.01	0.00584900160204813\\
289.01	0.00584869104686419\\
290.01	0.00584837379310341\\
291.01	0.00584804969209476\\
292.01	0.0058477185916428\\
293.01	0.00584738033593237\\
294.01	0.0058470347654283\\
295.01	0.00584668171677343\\
296.01	0.00584632102268166\\
297.01	0.00584595251182788\\
298.01	0.00584557600873379\\
299.01	0.00584519133364926\\
300.01	0.00584479830242994\\
301.01	0.00584439672640918\\
302.01	0.0058439864122665\\
303.01	0.00584356716188987\\
304.01	0.00584313877223296\\
305.01	0.0058427010351669\\
306.01	0.00584225373732605\\
307.01	0.00584179665994759\\
308.01	0.00584132957870374\\
309.01	0.00584085226352845\\
310.01	0.00584036447843504\\
311.01	0.00583986598132764\\
312.01	0.00583935652380261\\
313.01	0.00583883585094326\\
314.01	0.00583830370110309\\
315.01	0.00583775980568039\\
316.01	0.00583720388888242\\
317.01	0.00583663566747789\\
318.01	0.00583605485053728\\
319.01	0.00583546113916102\\
320.01	0.00583485422619392\\
321.01	0.00583423379592433\\
322.01	0.00583359952376895\\
323.01	0.0058329510759397\\
324.01	0.00583228810909297\\
325.01	0.00583161026995991\\
326.01	0.00583091719495551\\
327.01	0.00583020850976483\\
328.01	0.00582948382890559\\
329.01	0.00582874275526401\\
330.01	0.00582798487960182\\
331.01	0.0058272097800327\\
332.01	0.00582641702146448\\
333.01	0.00582560615500482\\
334.01	0.00582477671732568\\
335.01	0.00582392822998462\\
336.01	0.00582306019869728\\
337.01	0.00582217211255575\\
338.01	0.00582126344318881\\
339.01	0.00582033364385754\\
340.01	0.00581938214847792\\
341.01	0.0058184083705646\\
342.01	0.00581741170208534\\
343.01	0.00581639151221692\\
344.01	0.0058153471459908\\
345.01	0.00581427792281459\\
346.01	0.00581318313485613\\
347.01	0.00581206204527136\\
348.01	0.00581091388625774\\
349.01	0.00580973785691031\\
350.01	0.00580853312085544\\
351.01	0.00580729880363309\\
352.01	0.00580603398979506\\
353.01	0.00580473771968051\\
354.01	0.0058034089858281\\
355.01	0.00580204672897402\\
356.01	0.00580064983358342\\
357.01	0.00579921712285168\\
358.01	0.00579774735311033\\
359.01	0.0057962392075587\\
360.01	0.005794691289242\\
361.01	0.00579310211318587\\
362.01	0.00579147009759589\\
363.01	0.00578979355402939\\
364.01	0.00578807067645009\\
365.01	0.00578629952909211\\
366.01	0.00578447803308131\\
367.01	0.00578260395180925\\
368.01	0.00578067487512613\\
369.01	0.00577868820252627\\
370.01	0.00577664112567159\\
371.01	0.00577453061084078\\
372.01	0.00577235338225243\\
373.01	0.00577010590772886\\
374.01	0.0057677843889034\\
375.01	0.00576538475922506\\
376.01	0.00576290269449254\\
377.01	0.00576033364272751\\
378.01	0.00575767288310775\\
379.01	0.0057549156277403\\
380.01	0.00575205718570993\\
381.01	0.00574909321669317\\
382.01	0.00574602011232909\\
383.01	0.00574283555863858\\
384.01	0.00573953935367677\\
385.01	0.00573613667950795\\
386.01	0.00573265465617449\\
387.01	0.00572909878097666\\
388.01	0.00572546746121069\\
389.01	0.00572175906798681\\
390.01	0.00571797193529673\\
391.01	0.00571410435905338\\
392.01	0.00571015459609959\\
393.01	0.00570612086318736\\
394.01	0.00570200133592409\\
395.01	0.00569779414768695\\
396.01	0.00569349738850444\\
397.01	0.00568910910390217\\
398.01	0.00568462729371486\\
399.01	0.00568004991086283\\
400.01	0.00567537486009077\\
401.01	0.00567059999667202\\
402.01	0.00566572312507429\\
403.01	0.0056607419975887\\
404.01	0.00565565431292296\\
405.01	0.0056504577147575\\
406.01	0.00564514979026664\\
407.01	0.00563972806860564\\
408.01	0.00563419001936582\\
409.01	0.00562853305100012\\
410.01	0.00562275450922289\\
411.01	0.00561685167538734\\
412.01	0.00561082176484714\\
413.01	0.00560466192530814\\
414.01	0.00559836923517888\\
415.01	0.0055919407019305\\
416.01	0.0055853732604774\\
417.01	0.00557866377159529\\
418.01	0.00557180902039386\\
419.01	0.00556480571486572\\
420.01	0.00555765048453973\\
421.01	0.0055503398792679\\
422.01	0.00554287036818504\\
423.01	0.00553523833888587\\
424.01	0.0055274400968719\\
425.01	0.00551947186533327\\
426.01	0.00551132978533816\\
427.01	0.00550300991652206\\
428.01	0.00549450823837915\\
429.01	0.00548582065228439\\
430.01	0.00547694298438933\\
431.01	0.00546787098956975\\
432.01	0.00545860035662863\\
433.01	0.00544912671499583\\
434.01	0.00543944564321008\\
435.01	0.00542955267951603\\
436.01	0.00541944333496989\\
437.01	0.0054091131095094\\
438.01	0.00539855751152491\\
439.01	0.00538777208155469\\
440.01	0.00537675242082517\\
441.01	0.00536549422547395\\
442.01	0.00535399332741376\\
443.01	0.00534224574293606\\
444.01	0.0053302477302982\\
445.01	0.00531799585769161\\
446.01	0.00530548708314142\\
447.01	0.00529271884802473\\
448.01	0.00527968918600638\\
449.01	0.0052663968492423\\
450.01	0.00525284145365738\\
451.01	0.00523902364490384\\
452.01	0.00522494528617217\\
453.01	0.00521060966822928\\
454.01	0.00519602174073492\\
455.01	0.00518118836179257\\
456.01	0.00516611855948602\\
457.01	0.00515082379435854\\
458.01	0.00513531820474412\\
459.01	0.00511961880662152\\
460.01	0.00510374560494541\\
461.01	0.0050877215523977\\
462.01	0.00507157226170662\\
463.01	0.00505532533562082\\
464.01	0.00503900911949455\\
465.01	0.00502265059858424\\
466.01	0.00500627204632678\\
467.01	0.00498988586825628\\
468.01	0.00497348660526829\\
469.01	0.00495693495608155\\
470.01	0.00494013045788559\\
471.01	0.00492307702817898\\
472.01	0.00490577951810552\\
473.01	0.00488824379014157\\
474.01	0.00487047679516144\\
475.01	0.00485248664654519\\
476.01	0.00483428268822604\\
477.01	0.00481587555260846\\
478.01	0.00479727720309214\\
479.01	0.00477850095447716\\
480.01	0.00475956146398742\\
481.01	0.0047404746854527\\
482.01	0.00472125777491141\\
483.01	0.00470192893380416\\
484.01	0.00468250717426736\\
485.01	0.00466301198928942\\
486.01	0.00464346290933059\\
487.01	0.00462387892700592\\
488.01	0.00460427777359631\\
489.01	0.00458467503696179\\
490.01	0.0045650831221016\\
491.01	0.00454551007643988\\
492.01	0.0045259583355222\\
493.01	0.00450642349236445\\
494.01	0.00448689326883421\\
495.01	0.00446734698849532\\
496.01	0.00444775602909078\\
497.01	0.00442808599578846\\
498.01	0.00440830387864604\\
499.01	0.0043884034691836\\
500.01	0.00436839295302929\\
501.01	0.00434827984393562\\
502.01	0.00432807049989737\\
503.01	0.00430776980078569\\
504.01	0.00428738081034435\\
505.01	0.00426690443532456\\
506.01	0.00424633909993131\\
507.01	0.00422568046015354\\
508.01	0.00420492118975192\\
509.01	0.00418405087717167\\
510.01	0.00416305607942218\\
511.01	0.00414192058318003\\
512.01	0.00412062592194331\\
513.01	0.00409915218592861\\
514.01	0.0040774791305508\\
515.01	0.00405558752729416\\
516.01	0.00403346058849445\\
517.01	0.00401108509045564\\
518.01	0.00398845063170133\\
519.01	0.00396554635680371\\
520.01	0.00394235996915327\\
521.01	0.00391887778858746\\
522.01	0.00389508487529542\\
523.01	0.00387096521214111\\
524.01	0.00384650194411\\
525.01	0.00382167766677865\\
526.01	0.00379647474643057\\
527.01	0.00377087564355615\\
528.01	0.00374486319990149\\
529.01	0.00371842083863281\\
530.01	0.0036915326220562\\
531.01	0.00366418311985197\\
532.01	0.00363635707841285\\
533.01	0.00360803903626365\\
534.01	0.00357921322779472\\
535.01	0.00354986368087305\\
536.01	0.00351997432567058\\
537.01	0.0034895290929901\\
538.01	0.00345851199351856\\
539.01	0.00342690716999389\\
540.01	0.00339469891645227\\
541.01	0.00336187166302356\\
542.01	0.0033284099314067\\
543.01	0.00329429827480038\\
544.01	0.00325952122485604\\
545.01	0.00322406327163148\\
546.01	0.00318790888131798\\
547.01	0.00315104252280078\\
548.01	0.00311344868702179\\
549.01	0.00307511189818222\\
550.01	0.00303601671811203\\
551.01	0.00299614774639155\\
552.01	0.00295548961988336\\
553.01	0.00291402701583492\\
554.01	0.00287174466218705\\
555.01	0.00282862735678643\\
556.01	0.00278465999400613\\
557.01	0.00273982759516147\\
558.01	0.00269411534113635\\
559.01	0.00264750860806679\\
560.01	0.00259999300743791\\
561.01	0.00255155443193755\\
562.01	0.00250217910821201\\
563.01	0.0024518536573069\\
564.01	0.00240056516314333\\
565.01	0.00234830124906577\\
566.01	0.00229505016254032\\
567.01	0.0022408008684914\\
568.01	0.0021855431520882\\
569.01	0.00212926773180824\\
570.01	0.00207196638350766\\
571.01	0.00201363207610263\\
572.01	0.00195425911933926\\
573.01	0.00189384332402592\\
574.01	0.00183238217502583\\
575.01	0.00176987501722761\\
576.01	0.00170632325455657\\
577.01	0.00164173056181327\\
578.01	0.00157610310873702\\
579.01	0.00150944979519541\\
580.01	0.00144178249576928\\
581.01	0.00137311631120603\\
582.01	0.00130346982319341\\
583.01	0.00123286534759725\\
584.01	0.00116132917962666\\
585.01	0.00108889182225053\\
586.01	0.00101558818648855\\
587.01	0.000941457748798396\\
588.01	0.000866544646519892\\
589.01	0.000790897686996644\\
590.01	0.000714570239330877\\
591.01	0.000637619969415817\\
592.01	0.000560108368556696\\
593.01	0.000482100013165267\\
594.01	0.000403661477119359\\
595.01	0.000324859798708405\\
596.01	0.000245760379760691\\
597.01	0.000166429767496855\\
598.01	9.13379815192933e-05\\
599.01	2.91271958372443e-05\\
599.02	2.86192783627518e-05\\
599.03	2.8114423742006e-05\\
599.04	2.76126617978871e-05\\
599.05	2.71140226472798e-05\\
599.06	2.66185367039581e-05\\
599.07	2.61262346815533e-05\\
599.08	2.56371475965064e-05\\
599.09	2.51513067710818e-05\\
599.1	2.46687438363886e-05\\
599.11	2.41894907354479e-05\\
599.12	2.37135797262807e-05\\
599.13	2.32410433850267e-05\\
599.14	2.2771914609105e-05\\
599.15	2.23062266203888e-05\\
599.16	2.18440129684284e-05\\
599.17	2.13853075336917e-05\\
599.18	2.09301445308497e-05\\
599.19	2.04785585120812e-05\\
599.2	2.00305843704278e-05\\
599.21	1.95862573431627e-05\\
599.22	1.91456130152045e-05\\
599.23	1.87086873225609e-05\\
599.24	1.82755165558188e-05\\
599.25	1.7846137363638e-05\\
599.26	1.74205874164651e-05\\
599.27	1.69989072017502e-05\\
599.28	1.6581137610194e-05\\
599.29	1.61673199397579e-05\\
599.3	1.57574958996841e-05\\
599.31	1.53517076145696e-05\\
599.32	1.49499976284887e-05\\
599.33	1.45524089091333e-05\\
599.34	1.41589848520092e-05\\
599.35	1.37697692846831e-05\\
599.36	1.33848064710444e-05\\
599.37	1.30041411156422e-05\\
599.38	1.26278183680325e-05\\
599.39	1.22558838271929e-05\\
599.4	1.18883835459657e-05\\
599.41	1.15253640355657e-05\\
599.42	1.11668722700964e-05\\
599.43	1.08129556911519e-05\\
599.44	1.04636622124312e-05\\
599.45	1.01190402244222e-05\\
599.46	9.77913859911625e-06\\
599.47	9.44400669477576e-06\\
599.48	9.11369436074755e-06\\
599.49	8.78825194232206e-06\\
599.5	8.46773028565471e-06\\
599.51	8.15218074270464e-06\\
599.52	7.84165517625675e-06\\
599.53	7.5362059649732e-06\\
599.54	7.23588600849874e-06\\
599.55	6.94074873262146e-06\\
599.56	6.65084809447353e-06\\
599.57	6.36623858780126e-06\\
599.58	6.08697524826993e-06\\
599.59	5.81311365882228e-06\\
599.6	5.54470995511175e-06\\
599.61	5.28182083095637e-06\\
599.62	5.02450354387257e-06\\
599.63	4.7728159206558e-06\\
599.64	4.52681636300273e-06\\
599.65	4.28656385322197e-06\\
599.66	4.05211795995869e-06\\
599.67	3.8235388440163e-06\\
599.68	3.60088726419905e-06\\
599.69	3.38422458325514e-06\\
599.7	3.17361277382168e-06\\
599.71	2.96911442449269e-06\\
599.72	2.77079274589413e-06\\
599.73	2.57871157684567e-06\\
599.74	2.39293539058306e-06\\
599.75	2.213529301031e-06\\
599.76	2.04055906913997e-06\\
599.77	1.87409110930133e-06\\
599.78	1.71419249580043e-06\\
599.79	1.56093096936177e-06\\
599.8	1.41437494372877e-06\\
599.81	1.27459351233379e-06\\
599.82	1.14165645502366e-06\\
599.83	1.01563424484766e-06\\
599.84	8.96598054920053e-07\\
599.85	7.84619765348618e-07\\
599.86	6.79771970232487e-07\\
599.87	5.82127984719016e-07\\
599.88	4.91761852147374e-07\\
599.89	4.08748351251112e-07\\
599.9	3.33163003437068e-07\\
599.91	2.65082080131915e-07\\
599.92	2.04582610201579e-07\\
599.93	1.51742387452178e-07\\
599.94	1.06639978189951e-07\\
599.95	6.93547288800611e-08\\
599.96	3.99667738487652e-08\\
599.97	1.85570430896731e-08\\
599.98	5.20727013418598e-09\\
599.99	0\\
600	0\\
};
\addplot [color=blue!25!mycolor7,solid,forget plot]
  table[row sep=crcr]{%
0.01	0.00599920774511898\\
1.01	0.00599920717618128\\
2.01	0.00599920659536472\\
3.01	0.00599920600242077\\
4.01	0.00599920539709537\\
5.01	0.00599920477912935\\
6.01	0.00599920414825806\\
7.01	0.00599920350421131\\
8.01	0.00599920284671312\\
9.01	0.00599920217548179\\
10.01	0.00599920149022968\\
11.01	0.0059992007906631\\
12.01	0.00599920007648221\\
13.01	0.00599919934738092\\
14.01	0.00599919860304673\\
15.01	0.00599919784316028\\
16.01	0.0059991970673959\\
17.01	0.00599919627542072\\
18.01	0.00599919546689532\\
19.01	0.0059991946414727\\
20.01	0.00599919379879872\\
21.01	0.00599919293851189\\
22.01	0.00599919206024303\\
23.01	0.00599919116361524\\
24.01	0.00599919024824362\\
25.01	0.00599918931373534\\
26.01	0.00599918835968907\\
27.01	0.00599918738569524\\
28.01	0.00599918639133552\\
29.01	0.00599918537618293\\
30.01	0.00599918433980144\\
31.01	0.00599918328174592\\
32.01	0.00599918220156164\\
33.01	0.00599918109878443\\
34.01	0.00599917997294056\\
35.01	0.00599917882354615\\
36.01	0.0059991776501071\\
37.01	0.00599917645211886\\
38.01	0.0059991752290666\\
39.01	0.00599917398042425\\
40.01	0.00599917270565479\\
41.01	0.00599917140421023\\
42.01	0.00599917007553037\\
43.01	0.00599916871904389\\
44.01	0.00599916733416706\\
45.01	0.00599916592030388\\
46.01	0.00599916447684598\\
47.01	0.00599916300317207\\
48.01	0.00599916149864776\\
49.01	0.00599915996262538\\
50.01	0.00599915839444355\\
51.01	0.00599915679342695\\
52.01	0.00599915515888597\\
53.01	0.00599915349011664\\
54.01	0.00599915178640011\\
55.01	0.00599915004700243\\
56.01	0.00599914827117402\\
57.01	0.00599914645814955\\
58.01	0.00599914460714769\\
59.01	0.0059991427173706\\
60.01	0.00599914078800345\\
61.01	0.00599913881821453\\
62.01	0.00599913680715434\\
63.01	0.00599913475395556\\
64.01	0.00599913265773274\\
65.01	0.00599913051758159\\
66.01	0.00599912833257874\\
67.01	0.00599912610178138\\
68.01	0.00599912382422704\\
69.01	0.00599912149893269\\
70.01	0.00599911912489449\\
71.01	0.00599911670108767\\
72.01	0.00599911422646573\\
73.01	0.00599911169996021\\
74.01	0.00599910912048001\\
75.01	0.00599910648691098\\
76.01	0.00599910379811559\\
77.01	0.00599910105293222\\
78.01	0.00599909825017465\\
79.01	0.00599909538863188\\
80.01	0.00599909246706711\\
81.01	0.00599908948421765\\
82.01	0.00599908643879377\\
83.01	0.00599908332947907\\
84.01	0.00599908015492903\\
85.01	0.00599907691377092\\
86.01	0.00599907360460263\\
87.01	0.00599907022599305\\
88.01	0.00599906677648032\\
89.01	0.00599906325457193\\
90.01	0.00599905965874401\\
91.01	0.0059990559874405\\
92.01	0.00599905223907211\\
93.01	0.00599904841201628\\
94.01	0.0059990445046161\\
95.01	0.00599904051517964\\
96.01	0.00599903644197924\\
97.01	0.00599903228325092\\
98.01	0.00599902803719312\\
99.01	0.00599902370196647\\
100.01	0.00599901927569264\\
101.01	0.00599901475645361\\
102.01	0.00599901014229074\\
103.01	0.00599900543120417\\
104.01	0.00599900062115153\\
105.01	0.00599899571004749\\
106.01	0.00599899069576272\\
107.01	0.00599898557612253\\
108.01	0.00599898034890668\\
109.01	0.00599897501184758\\
110.01	0.00599896956262994\\
111.01	0.0059989639988893\\
112.01	0.00599895831821158\\
113.01	0.00599895251813124\\
114.01	0.00599894659613085\\
115.01	0.0059989405496396\\
116.01	0.00599893437603241\\
117.01	0.00599892807262836\\
118.01	0.00599892163669021\\
119.01	0.0059989150654227\\
120.01	0.00599890835597124\\
121.01	0.00599890150542094\\
122.01	0.00599889451079527\\
123.01	0.00599888736905457\\
124.01	0.00599888007709506\\
125.01	0.005998872631747\\
126.01	0.0059988650297737\\
127.01	0.00599885726786993\\
128.01	0.00599884934266052\\
129.01	0.00599884125069864\\
130.01	0.00599883298846468\\
131.01	0.0059988245523644\\
132.01	0.00599881593872726\\
133.01	0.00599880714380532\\
134.01	0.00599879816377095\\
135.01	0.00599878899471553\\
136.01	0.00599877963264779\\
137.01	0.00599877007349139\\
138.01	0.00599876031308411\\
139.01	0.00599875034717532\\
140.01	0.00599874017142437\\
141.01	0.00599872978139846\\
142.01	0.00599871917257084\\
143.01	0.00599870834031886\\
144.01	0.00599869727992172\\
145.01	0.00599868598655861\\
146.01	0.00599867445530645\\
147.01	0.0059986626811376\\
148.01	0.00599865065891798\\
149.01	0.00599863838340458\\
150.01	0.00599862584924283\\
151.01	0.00599861305096492\\
152.01	0.00599859998298662\\
153.01	0.00599858663960552\\
154.01	0.00599857301499787\\
155.01	0.00599855910321631\\
156.01	0.00599854489818736\\
157.01	0.00599853039370847\\
158.01	0.00599851558344523\\
159.01	0.00599850046092872\\
160.01	0.00599848501955267\\
161.01	0.00599846925257051\\
162.01	0.00599845315309188\\
163.01	0.00599843671408033\\
164.01	0.00599841992834965\\
165.01	0.00599840278856103\\
166.01	0.00599838528721938\\
167.01	0.00599836741667027\\
168.01	0.00599834916909647\\
169.01	0.00599833053651428\\
170.01	0.00599831151077045\\
171.01	0.00599829208353769\\
172.01	0.00599827224631181\\
173.01	0.00599825199040755\\
174.01	0.00599823130695431\\
175.01	0.00599821018689306\\
176.01	0.00599818862097142\\
177.01	0.00599816659974013\\
178.01	0.00599814411354825\\
179.01	0.00599812115253914\\
180.01	0.00599809770664599\\
181.01	0.00599807376558748\\
182.01	0.0059980493188628\\
183.01	0.00599802435574706\\
184.01	0.00599799886528652\\
185.01	0.00599797283629355\\
186.01	0.00599794625734166\\
187.01	0.00599791911676021\\
188.01	0.00599789140262936\\
189.01	0.00599786310277441\\
190.01	0.00599783420476035\\
191.01	0.00599780469588631\\
192.01	0.00599777456317977\\
193.01	0.00599774379339065\\
194.01	0.00599771237298514\\
195.01	0.0059976802881398\\
196.01	0.00599764752473506\\
197.01	0.00599761406834871\\
198.01	0.00599757990424974\\
199.01	0.00599754501739094\\
200.01	0.00599750939240248\\
201.01	0.0059974730135848\\
202.01	0.00599743586490139\\
203.01	0.00599739792997112\\
204.01	0.00599735919206111\\
205.01	0.00599731963407892\\
206.01	0.00599727923856453\\
207.01	0.00599723798768228\\
208.01	0.00599719586321269\\
209.01	0.00599715284654393\\
210.01	0.00599710891866346\\
211.01	0.00599706406014889\\
212.01	0.00599701825115882\\
213.01	0.00599697147142395\\
214.01	0.00599692370023746\\
215.01	0.00599687491644507\\
216.01	0.0059968250984357\\
217.01	0.00599677422413087\\
218.01	0.00599672227097416\\
219.01	0.00599666921592113\\
220.01	0.00599661503542779\\
221.01	0.00599655970544041\\
222.01	0.00599650320138266\\
223.01	0.00599644549814529\\
224.01	0.00599638657007356\\
225.01	0.00599632639095526\\
226.01	0.00599626493400755\\
227.01	0.00599620217186472\\
228.01	0.00599613807656508\\
229.01	0.00599607261953705\\
230.01	0.00599600577158559\\
231.01	0.00599593750287803\\
232.01	0.00599586778292976\\
233.01	0.0059957965805889\\
234.01	0.00599572386402173\\
235.01	0.00599564960069651\\
236.01	0.0059955737573679\\
237.01	0.00599549630006043\\
238.01	0.00599541719405183\\
239.01	0.00599533640385553\\
240.01	0.00599525389320326\\
241.01	0.00599516962502701\\
242.01	0.00599508356144032\\
243.01	0.00599499566371927\\
244.01	0.00599490589228303\\
245.01	0.00599481420667374\\
246.01	0.00599472056553601\\
247.01	0.00599462492659598\\
248.01	0.00599452724663931\\
249.01	0.00599442748148934\\
250.01	0.0059943255859844\\
251.01	0.00599422151395413\\
252.01	0.00599411521819519\\
253.01	0.00599400665044746\\
254.01	0.00599389576136785\\
255.01	0.00599378250050464\\
256.01	0.0059936668162711\\
257.01	0.00599354865591747\\
258.01	0.00599342796550285\\
259.01	0.00599330468986659\\
260.01	0.00599317877259795\\
261.01	0.00599305015600591\\
262.01	0.00599291878108734\\
263.01	0.00599278458749503\\
264.01	0.00599264751350389\\
265.01	0.00599250749597694\\
266.01	0.00599236447032994\\
267.01	0.00599221837049516\\
268.01	0.00599206912888394\\
269.01	0.00599191667634849\\
270.01	0.00599176094214177\\
271.01	0.00599160185387725\\
272.01	0.00599143933748678\\
273.01	0.00599127331717706\\
274.01	0.00599110371538538\\
275.01	0.00599093045273321\\
276.01	0.0059907534479793\\
277.01	0.0059905726179706\\
278.01	0.0059903878775917\\
279.01	0.00599019913971328\\
280.01	0.00599000631513786\\
281.01	0.00598980931254505\\
282.01	0.00598960803843404\\
283.01	0.00598940239706462\\
284.01	0.00598919229039605\\
285.01	0.00598897761802438\\
286.01	0.00598875827711743\\
287.01	0.005988534162347\\
288.01	0.00598830516581982\\
289.01	0.0059880711770051\\
290.01	0.00598783208266051\\
291.01	0.00598758776675481\\
292.01	0.00598733811038847\\
293.01	0.00598708299171064\\
294.01	0.00598682228583366\\
295.01	0.00598655586474466\\
296.01	0.00598628359721345\\
297.01	0.0059860053486971\\
298.01	0.00598572098124139\\
299.01	0.00598543035337806\\
300.01	0.0059851333200183\\
301.01	0.00598482973234218\\
302.01	0.0059845194376842\\
303.01	0.00598420227941313\\
304.01	0.00598387809680867\\
305.01	0.00598354672493207\\
306.01	0.00598320799449195\\
307.01	0.00598286173170432\\
308.01	0.00598250775814763\\
309.01	0.00598214589061027\\
310.01	0.00598177594093358\\
311.01	0.00598139771584608\\
312.01	0.0059810110167925\\
313.01	0.00598061563975356\\
314.01	0.00598021137505921\\
315.01	0.00597979800719296\\
316.01	0.00597937531458689\\
317.01	0.00597894306940824\\
318.01	0.00597850103733571\\
319.01	0.00597804897732494\\
320.01	0.00597758664136327\\
321.01	0.00597711377421333\\
322.01	0.00597663011314294\\
323.01	0.00597613538764335\\
324.01	0.00597562931913234\\
325.01	0.005975111620643\\
326.01	0.00597458199649726\\
327.01	0.00597404014196246\\
328.01	0.00597348574288994\\
329.01	0.00597291847533586\\
330.01	0.00597233800516154\\
331.01	0.00597174398761329\\
332.01	0.0059711360668798\\
333.01	0.0059705138756261\\
334.01	0.00596987703450306\\
335.01	0.00596922515162998\\
336.01	0.00596855782205012\\
337.01	0.00596787462715672\\
338.01	0.00596717513408934\\
339.01	0.00596645889509628\\
340.01	0.00596572544686534\\
341.01	0.00596497430981876\\
342.01	0.00596420498737254\\
343.01	0.00596341696515803\\
344.01	0.00596260971020565\\
345.01	0.00596178267009125\\
346.01	0.00596093527204223\\
347.01	0.00596006692200784\\
348.01	0.00595917700369159\\
349.01	0.00595826487755111\\
350.01	0.00595732987976751\\
351.01	0.00595637132119052\\
352.01	0.00595538848626687\\
353.01	0.0059543806319626\\
354.01	0.00595334698669194\\
355.01	0.00595228674927193\\
356.01	0.00595119908792382\\
357.01	0.00595008313935151\\
358.01	0.00594893800793244\\
359.01	0.00594776276506671\\
360.01	0.00594655644874136\\
361.01	0.005945318063378\\
362.01	0.00594404658004975\\
363.01	0.00594274093717175\\
364.01	0.005941400041788\\
365.01	0.00594002277160655\\
366.01	0.00593860797795742\\
367.01	0.00593715448988461\\
368.01	0.00593566111960748\\
369.01	0.00593412666962958\\
370.01	0.00593254994179276\\
371.01	0.00593092974860451\\
372.01	0.00592926492716847\\
373.01	0.00592755435603001\\
374.01	0.00592579697518265\\
375.01	0.00592399180934374\\
376.01	0.00592213799436153\\
377.01	0.00592023480619791\\
378.01	0.00591828169126427\\
379.01	0.00591627829585061\\
380.01	0.00591422449080454\\
381.01	0.00591212038524867\\
382.01	0.00590996631961475\\
383.01	0.00590776282312728\\
384.01	0.00590551051338558\\
385.01	0.0059032098963159\\
386.01	0.00590086073621105\\
387.01	0.00589846203139617\\
388.01	0.00589601270542194\\
389.01	0.00589351165590037\\
390.01	0.00589095775369843\\
391.01	0.00588834984209453\\
392.01	0.00588568673589602\\
393.01	0.0058829672205153\\
394.01	0.0058801900510013\\
395.01	0.0058773539510239\\
396.01	0.00587445761180711\\
397.01	0.00587149969100976\\
398.01	0.00586847881154749\\
399.01	0.00586539356035274\\
400.01	0.00586224248707006\\
401.01	0.00585902410267831\\
402.01	0.00585573687803897\\
403.01	0.00585237924236085\\
404.01	0.00584894958157736\\
405.01	0.0058454462366283\\
406.01	0.00584186750163974\\
407.01	0.0058382116219927\\
408.01	0.00583447679227158\\
409.01	0.00583066115408409\\
410.01	0.00582676279373878\\
411.01	0.00582277973977041\\
412.01	0.00581870996029901\\
413.01	0.00581455136020655\\
414.01	0.00581030177811717\\
415.01	0.00580595898315922\\
416.01	0.00580152067149174\\
417.01	0.00579698446257012\\
418.01	0.0057923478951265\\
419.01	0.00578760842283554\\
420.01	0.00578276340963383\\
421.01	0.00577781012465565\\
422.01	0.00577274573674575\\
423.01	0.00576756730850078\\
424.01	0.00576227178978976\\
425.01	0.00575685601069225\\
426.01	0.00575131667378939\\
427.01	0.00574565034572987\\
428.01	0.00573985344798673\\
429.01	0.00573392224670478\\
430.01	0.00572785284152894\\
431.01	0.00572164115328557\\
432.01	0.00571528291037148\\
433.01	0.00570877363368667\\
434.01	0.00570210861992245\\
435.01	0.00569528292299155\\
436.01	0.00568829133335686\\
437.01	0.00568112835498331\\
438.01	0.00567378817960297\\
439.01	0.00566626465793952\\
440.01	0.00565855126750225\\
441.01	0.00565064107650711\\
442.01	0.00564252670344594\\
443.01	0.00563420027177326\\
444.01	0.00562565335914622\\
445.01	0.0056168769406252\\
446.01	0.00560786132523378\\
447.01	0.00559859608530133\\
448.01	0.00558906997808641\\
449.01	0.00557927085932592\\
450.01	0.00556918558861603\\
451.01	0.00555879992695325\\
452.01	0.00554809842741975\\
453.01	0.00553706432098528\\
454.01	0.00552567940086576\\
455.01	0.00551392391100613\\
456.01	0.00550177644733057\\
457.01	0.00548921388478204\\
458.01	0.00547621134939325\\
459.01	0.00546274226339325\\
460.01	0.005448778503656\\
461.01	0.00543429073100853\\
462.01	0.00541924897191974\\
463.01	0.00540362356749257\\
464.01	0.00538738665107438\\
465.01	0.00537051438013312\\
466.01	0.00535299023716055\\
467.01	0.00533480983766913\\
468.01	0.00531598810909099\\
469.01	0.00529667141390681\\
470.01	0.00527695873756643\\
471.01	0.00525684389729683\\
472.01	0.00523632107135242\\
473.01	0.00521538490662479\\
474.01	0.00519403064763003\\
475.01	0.00517225429072692\\
476.01	0.00515005276802421\\
477.01	0.00512742416620251\\
478.01	0.00510436798679032\\
479.01	0.00508088545394805\\
480.01	0.00505697984979161\\
481.01	0.00503265688938281\\
482.01	0.00500792516164674\\
483.01	0.00498279663299035\\
484.01	0.00495728721291673\\
485.01	0.00493141737059719\\
486.01	0.00490521278547553\\
487.01	0.00487870500103308\\
488.01	0.00485193202893283\\
489.01	0.00482493881792928\\
490.01	0.00479777745330554\\
491.01	0.00477050688384594\\
492.01	0.00474319194907003\\
493.01	0.00471590149384669\\
494.01	0.00468870505444633\\
495.01	0.00466166736378074\\
496.01	0.00463483967115599\\
497.01	0.00460824646084977\\
498.01	0.00458177047475227\\
499.01	0.00455510963545843\\
500.01	0.00452827058744963\\
501.01	0.00450128226744785\\
502.01	0.00447417550499826\\
503.01	0.00444698268078264\\
504.01	0.00441973723465563\\
505.01	0.0043924729924855\\
506.01	0.00436522327968771\\
507.01	0.00433801979095227\\
508.01	0.00431089119216303\\
509.01	0.00428386144489852\\
510.01	0.00425694787065462\\
511.01	0.00423015901772533\\
512.01	0.00420349246824837\\
513.01	0.00417693284040673\\
514.01	0.00415045042149732\\
515.01	0.00412400114070438\\
516.01	0.00409752899750191\\
517.01	0.0040709740866263\\
518.01	0.00404431015725351\\
519.01	0.00401754020522142\\
520.01	0.00399066522984844\\
521.01	0.00396368251504052\\
522.01	0.00393658516421696\\
523.01	0.00390936173362636\\
524.01	0.00388199602961315\\
525.01	0.00385446715333914\\
526.01	0.00382674988269075\\
527.01	0.00379881546461456\\
528.01	0.00377063287158874\\
529.01	0.00374217052785795\\
530.01	0.00371339840652742\\
531.01	0.00368429020704527\\
532.01	0.00365482488644382\\
533.01	0.00362498452441107\\
534.01	0.00359475008237013\\
535.01	0.0035641008951345\\
536.01	0.00353301497943292\\
537.01	0.00350146942467424\\
538.01	0.003469440847353\\
539.01	0.00343690587620745\\
540.01	0.00340384161757189\\
541.01	0.00337022603239339\\
542.01	0.0033360381432204\\
543.01	0.00330125799112801\\
544.01	0.00326586629706542\\
545.01	0.00322984392536571\\
546.01	0.00319317162090588\\
547.01	0.00315583011753067\\
548.01	0.00311780028889496\\
549.01	0.00307906327340484\\
550.01	0.00303960056088456\\
551.01	0.00299939403210085\\
552.01	0.00295842594890036\\
553.01	0.00291667890320695\\
554.01	0.00287413574665602\\
555.01	0.00283077953587981\\
556.01	0.00278659352953516\\
557.01	0.00274156123091943\\
558.01	0.00269566643484715\\
559.01	0.0026488932657813\\
560.01	0.00260122620891889\\
561.01	0.00255265013827661\\
562.01	0.00250315034739825\\
563.01	0.00245271258897792\\
564.01	0.00240132312857586\\
565.01	0.00234896881412779\\
566.01	0.00229563715800819\\
567.01	0.00224131642709743\\
568.01	0.002185995740384\\
569.01	0.00212966517598244\\
570.01	0.0020723158896105\\
571.01	0.0020139402462388\\
572.01	0.00195453196598035\\
573.01	0.00189408628446933\\
574.01	0.00183260012729144\\
575.01	0.00177007229786928\\
576.01	0.00170650367858114\\
577.01	0.00164189744502896\\
578.01	0.00157625929296629\\
579.01	0.00150959767673313\\
580.01	0.0014419240572397\\
581.01	0.00137325315661659\\
582.01	0.00130360321561083\\
583.01	0.00123299624859551\\
584.01	0.00116145828952293\\
585.01	0.0010890196201021\\
586.01	0.00101571496882195\\
587.01	0.000941583666090056\\
588.01	0.000866669736561883\\
589.01	0.000791021904488039\\
590.01	0.000714693481339309\\
591.01	0.00063774209673771\\
592.01	0.000560229223416786\\
593.01	0.000482219434073887\\
594.01	0.000403779311999168\\
595.01	0.000324975917580054\\
596.01	0.000245874688332452\\
597.01	0.000166536617556516\\
598.01	9.13379815193054e-05\\
599.01	2.91271958372426e-05\\
599.02	2.86192783627518e-05\\
599.03	2.81144237420042e-05\\
599.04	2.76126617978888e-05\\
599.05	2.71140226472798e-05\\
599.06	2.66185367039599e-05\\
599.07	2.61262346815533e-05\\
599.08	2.56371475965082e-05\\
599.09	2.51513067710835e-05\\
599.1	2.46687438363886e-05\\
599.11	2.41894907354479e-05\\
599.12	2.3713579726279e-05\\
599.13	2.32410433850267e-05\\
599.14	2.27719146091033e-05\\
599.15	2.23062266203888e-05\\
599.16	2.18440129684267e-05\\
599.17	2.13853075336917e-05\\
599.18	2.09301445308497e-05\\
599.19	2.04785585120829e-05\\
599.2	2.00305843704295e-05\\
599.21	1.95862573431627e-05\\
599.22	1.91456130152028e-05\\
599.23	1.87086873225609e-05\\
599.24	1.82755165558171e-05\\
599.25	1.7846137363638e-05\\
599.26	1.74205874164668e-05\\
599.27	1.69989072017485e-05\\
599.28	1.6581137610194e-05\\
599.29	1.61673199397579e-05\\
599.3	1.57574958996824e-05\\
599.31	1.53517076145714e-05\\
599.32	1.49499976284904e-05\\
599.33	1.45524089091333e-05\\
599.34	1.41589848520092e-05\\
599.35	1.37697692846831e-05\\
599.36	1.33848064710462e-05\\
599.37	1.30041411156422e-05\\
599.38	1.26278183680325e-05\\
599.39	1.22558838271929e-05\\
599.4	1.18883835459674e-05\\
599.41	1.15253640355657e-05\\
599.42	1.11668722700964e-05\\
599.43	1.08129556911519e-05\\
599.44	1.04636622124312e-05\\
599.45	1.01190402244222e-05\\
599.46	9.77913859911625e-06\\
599.47	9.44400669477576e-06\\
599.48	9.11369436074581e-06\\
599.49	8.7882519423238e-06\\
599.5	8.46773028565471e-06\\
599.51	8.15218074270464e-06\\
599.52	7.84165517625675e-06\\
599.53	7.5362059649732e-06\\
599.54	7.23588600849874e-06\\
599.55	6.94074873261973e-06\\
599.56	6.65084809447353e-06\\
599.57	6.366238587803e-06\\
599.58	6.08697524826819e-06\\
599.59	5.81311365882402e-06\\
599.6	5.54470995511175e-06\\
599.61	5.28182083095637e-06\\
599.62	5.02450354387431e-06\\
599.63	4.77281592065407e-06\\
599.64	4.52681636300446e-06\\
599.65	4.28656385322197e-06\\
599.66	4.05211795996042e-06\\
599.67	3.82353884401457e-06\\
599.68	3.60088726420078e-06\\
599.69	3.38422458325341e-06\\
599.7	3.17361277382168e-06\\
599.71	2.96911442449269e-06\\
599.72	2.77079274589413e-06\\
599.73	2.5787115768474e-06\\
599.74	2.39293539058306e-06\\
599.75	2.21352930102926e-06\\
599.76	2.04055906913823e-06\\
599.77	1.87409110929959e-06\\
599.78	1.71419249580043e-06\\
599.79	1.56093096936004e-06\\
599.8	1.41437494372877e-06\\
599.81	1.27459351233553e-06\\
599.82	1.14165645502366e-06\\
599.83	1.01563424484766e-06\\
599.84	8.96598054920053e-07\\
599.85	7.84619765350353e-07\\
599.86	6.79771970232487e-07\\
599.87	5.82127984717282e-07\\
599.88	4.91761852145639e-07\\
599.89	4.08748351251112e-07\\
599.9	3.33163003438802e-07\\
599.91	2.65082080131915e-07\\
599.92	2.04582610201579e-07\\
599.93	1.51742387450443e-07\\
599.94	1.06639978191686e-07\\
599.95	6.93547288817958e-08\\
599.96	3.99667738487652e-08\\
599.97	1.85570430896731e-08\\
599.98	5.20727013245126e-09\\
599.99	0\\
600	0\\
};
\addplot [color=mycolor9,solid,forget plot]
  table[row sep=crcr]{%
0.01	0.00613141925685511\\
1.01	0.00613141877852663\\
2.01	0.00613141829018322\\
3.01	0.00613141779161435\\
4.01	0.00613141728260539\\
5.01	0.00613141676293696\\
6.01	0.00613141623238522\\
7.01	0.0061314156907213\\
8.01	0.00613141513771185\\
9.01	0.00613141457311849\\
10.01	0.00613141399669758\\
11.01	0.00613141340820058\\
12.01	0.00613141280737362\\
13.01	0.00613141219395754\\
14.01	0.00613141156768764\\
15.01	0.00613141092829383\\
16.01	0.00613141027550002\\
17.01	0.00613140960902458\\
18.01	0.00613140892857955\\
19.01	0.00613140823387131\\
20.01	0.00613140752459977\\
21.01	0.00613140680045859\\
22.01	0.00613140606113511\\
23.01	0.0061314053063097\\
24.01	0.00613140453565651\\
25.01	0.00613140374884231\\
26.01	0.00613140294552697\\
27.01	0.00613140212536338\\
28.01	0.00613140128799692\\
29.01	0.00613140043306532\\
30.01	0.00613139956019907\\
31.01	0.00613139866902044\\
32.01	0.00613139775914414\\
33.01	0.00613139683017632\\
34.01	0.00613139588171504\\
35.01	0.00613139491334979\\
36.01	0.00613139392466131\\
37.01	0.00613139291522161\\
38.01	0.0061313918845936\\
39.01	0.00613139083233102\\
40.01	0.00613138975797804\\
41.01	0.00613138866106922\\
42.01	0.0061313875411294\\
43.01	0.00613138639767318\\
44.01	0.00613138523020505\\
45.01	0.00613138403821884\\
46.01	0.00613138282119798\\
47.01	0.00613138157861463\\
48.01	0.00613138030992992\\
49.01	0.00613137901459358\\
50.01	0.00613137769204372\\
51.01	0.00613137634170644\\
52.01	0.00613137496299578\\
53.01	0.00613137355531355\\
54.01	0.00613137211804844\\
55.01	0.00613137065057646\\
56.01	0.00613136915226047\\
57.01	0.00613136762244989\\
58.01	0.00613136606048014\\
59.01	0.00613136446567259\\
60.01	0.00613136283733439\\
61.01	0.00613136117475776\\
62.01	0.00613135947722029\\
63.01	0.00613135774398396\\
64.01	0.0061313559742952\\
65.01	0.00613135416738438\\
66.01	0.00613135232246574\\
67.01	0.00613135043873665\\
68.01	0.00613134851537762\\
69.01	0.00613134655155164\\
70.01	0.00613134454640412\\
71.01	0.00613134249906224\\
72.01	0.00613134040863477\\
73.01	0.00613133827421133\\
74.01	0.00613133609486241\\
75.01	0.00613133386963871\\
76.01	0.00613133159757098\\
77.01	0.00613132927766903\\
78.01	0.0061313269089219\\
79.01	0.00613132449029719\\
80.01	0.0061313220207405\\
81.01	0.00613131949917508\\
82.01	0.00613131692450155\\
83.01	0.00613131429559655\\
84.01	0.00613131161131343\\
85.01	0.00613130887048089\\
86.01	0.00613130607190306\\
87.01	0.00613130321435836\\
88.01	0.00613130029659934\\
89.01	0.00613129731735213\\
90.01	0.00613129427531566\\
91.01	0.00613129116916107\\
92.01	0.00613128799753172\\
93.01	0.00613128475904187\\
94.01	0.00613128145227612\\
95.01	0.00613127807578923\\
96.01	0.00613127462810521\\
97.01	0.00613127110771646\\
98.01	0.00613126751308354\\
99.01	0.00613126384263425\\
100.01	0.00613126009476282\\
101.01	0.00613125626782938\\
102.01	0.00613125236015923\\
103.01	0.0061312483700418\\
104.01	0.0061312442957304\\
105.01	0.00613124013544095\\
106.01	0.00613123588735144\\
107.01	0.00613123154960105\\
108.01	0.00613122712028939\\
109.01	0.00613122259747547\\
110.01	0.00613121797917708\\
111.01	0.00613121326336975\\
112.01	0.0061312084479858\\
113.01	0.00613120353091368\\
114.01	0.00613119850999654\\
115.01	0.00613119338303186\\
116.01	0.00613118814776975\\
117.01	0.00613118280191292\\
118.01	0.00613117734311466\\
119.01	0.00613117176897832\\
120.01	0.00613116607705617\\
121.01	0.00613116026484815\\
122.01	0.00613115432980107\\
123.01	0.00613114826930706\\
124.01	0.00613114208070265\\
125.01	0.00613113576126771\\
126.01	0.00613112930822355\\
127.01	0.0061311227187327\\
128.01	0.00613111598989675\\
129.01	0.00613110911875544\\
130.01	0.0061311021022852\\
131.01	0.00613109493739823\\
132.01	0.00613108762094029\\
133.01	0.00613108014968999\\
134.01	0.00613107252035707\\
135.01	0.00613106472958074\\
136.01	0.00613105677392838\\
137.01	0.00613104864989433\\
138.01	0.0061310403538974\\
139.01	0.00613103188228016\\
140.01	0.00613102323130672\\
141.01	0.00613101439716129\\
142.01	0.00613100537594639\\
143.01	0.00613099616368126\\
144.01	0.00613098675629949\\
145.01	0.00613097714964797\\
146.01	0.00613096733948448\\
147.01	0.00613095732147582\\
148.01	0.00613094709119606\\
149.01	0.00613093664412417\\
150.01	0.00613092597564249\\
151.01	0.00613091508103397\\
152.01	0.00613090395548086\\
153.01	0.0061308925940616\\
154.01	0.00613088099174931\\
155.01	0.00613086914340898\\
156.01	0.00613085704379548\\
157.01	0.00613084468755092\\
158.01	0.00613083206920266\\
159.01	0.00613081918315992\\
160.01	0.00613080602371232\\
161.01	0.00613079258502628\\
162.01	0.00613077886114318\\
163.01	0.0061307648459762\\
164.01	0.00613075053330715\\
165.01	0.00613073591678446\\
166.01	0.00613072098991972\\
167.01	0.0061307057460848\\
168.01	0.0061306901785089\\
169.01	0.00613067428027509\\
170.01	0.00613065804431758\\
171.01	0.00613064146341832\\
172.01	0.00613062453020359\\
173.01	0.00613060723714046\\
174.01	0.00613058957653357\\
175.01	0.00613057154052153\\
176.01	0.00613055312107313\\
177.01	0.00613053430998347\\
178.01	0.00613051509887085\\
179.01	0.00613049547917213\\
180.01	0.00613047544213893\\
181.01	0.00613045497883382\\
182.01	0.00613043408012581\\
183.01	0.00613041273668644\\
184.01	0.00613039093898515\\
185.01	0.00613036867728486\\
186.01	0.00613034594163763\\
187.01	0.00613032272187963\\
188.01	0.00613029900762659\\
189.01	0.00613027478826909\\
190.01	0.00613025005296706\\
191.01	0.00613022479064527\\
192.01	0.00613019898998776\\
193.01	0.00613017263943257\\
194.01	0.00613014572716603\\
195.01	0.00613011824111767\\
196.01	0.00613009016895396\\
197.01	0.00613006149807331\\
198.01	0.00613003221559867\\
199.01	0.0061300023083726\\
200.01	0.00612997176295064\\
201.01	0.00612994056559471\\
202.01	0.00612990870226678\\
203.01	0.00612987615862199\\
204.01	0.00612984292000166\\
205.01	0.00612980897142641\\
206.01	0.00612977429758902\\
207.01	0.00612973888284686\\
208.01	0.00612970271121422\\
209.01	0.00612966576635505\\
210.01	0.00612962803157442\\
211.01	0.00612958948981084\\
212.01	0.00612955012362771\\
213.01	0.00612950991520501\\
214.01	0.00612946884633029\\
215.01	0.0061294268983902\\
216.01	0.0061293840523607\\
217.01	0.0061293402887983\\
218.01	0.00612929558783022\\
219.01	0.00612924992914447\\
220.01	0.00612920329197995\\
221.01	0.00612915565511587\\
222.01	0.00612910699686189\\
223.01	0.0061290572950461\\
224.01	0.00612900652700514\\
225.01	0.00612895466957207\\
226.01	0.00612890169906489\\
227.01	0.00612884759127511\\
228.01	0.00612879232145452\\
229.01	0.00612873586430366\\
230.01	0.00612867819395805\\
231.01	0.00612861928397599\\
232.01	0.00612855910732409\\
233.01	0.00612849763636423\\
234.01	0.00612843484283882\\
235.01	0.00612837069785657\\
236.01	0.00612830517187709\\
237.01	0.00612823823469604\\
238.01	0.00612816985542942\\
239.01	0.00612810000249731\\
240.01	0.00612802864360717\\
241.01	0.00612795574573717\\
242.01	0.00612788127511858\\
243.01	0.00612780519721822\\
244.01	0.00612772747671982\\
245.01	0.00612764807750554\\
246.01	0.00612756696263651\\
247.01	0.00612748409433289\\
248.01	0.00612739943395385\\
249.01	0.00612731294197671\\
250.01	0.00612722457797518\\
251.01	0.00612713430059778\\
252.01	0.00612704206754512\\
253.01	0.00612694783554658\\
254.01	0.00612685156033707\\
255.01	0.00612675319663184\\
256.01	0.00612665269810169\\
257.01	0.00612655001734743\\
258.01	0.00612644510587337\\
259.01	0.00612633791405971\\
260.01	0.0061262283911347\\
261.01	0.0061261164851464\\
262.01	0.00612600214293276\\
263.01	0.00612588531009113\\
264.01	0.00612576593094754\\
265.01	0.00612564394852431\\
266.01	0.00612551930450735\\
267.01	0.00612539193921218\\
268.01	0.00612526179154911\\
269.01	0.00612512879898738\\
270.01	0.00612499289751862\\
271.01	0.00612485402161885\\
272.01	0.00612471210420932\\
273.01	0.00612456707661684\\
274.01	0.00612441886853247\\
275.01	0.00612426740796932\\
276.01	0.00612411262121866\\
277.01	0.00612395443280529\\
278.01	0.00612379276544136\\
279.01	0.00612362753997891\\
280.01	0.00612345867536119\\
281.01	0.00612328608857209\\
282.01	0.00612310969458471\\
283.01	0.006122929406308\\
284.01	0.00612274513453188\\
285.01	0.00612255678787113\\
286.01	0.00612236427270669\\
287.01	0.00612216749312673\\
288.01	0.00612196635086441\\
289.01	0.00612176074523501\\
290.01	0.00612155057307034\\
291.01	0.00612133572865188\\
292.01	0.00612111610364109\\
293.01	0.00612089158700883\\
294.01	0.00612066206496184\\
295.01	0.00612042742086683\\
296.01	0.00612018753517314\\
297.01	0.00611994228533281\\
298.01	0.00611969154571745\\
299.01	0.00611943518753377\\
300.01	0.00611917307873641\\
301.01	0.00611890508393728\\
302.01	0.00611863106431315\\
303.01	0.00611835087751008\\
304.01	0.00611806437754568\\
305.01	0.00611777141470708\\
306.01	0.00611747183544712\\
307.01	0.00611716548227741\\
308.01	0.00611685219365762\\
309.01	0.00611653180388203\\
310.01	0.00611620414296253\\
311.01	0.00611586903650876\\
312.01	0.00611552630560402\\
313.01	0.00611517576667837\\
314.01	0.00611481723137799\\
315.01	0.00611445050643047\\
316.01	0.00611407539350679\\
317.01	0.00611369168907982\\
318.01	0.00611329918427828\\
319.01	0.00611289766473754\\
320.01	0.0061124869104461\\
321.01	0.00611206669558833\\
322.01	0.00611163678838362\\
323.01	0.00611119695092103\\
324.01	0.00611074693899049\\
325.01	0.00611028650191061\\
326.01	0.00610981538235209\\
327.01	0.00610933331615735\\
328.01	0.00610884003215808\\
329.01	0.00610833525198795\\
330.01	0.00610781868989326\\
331.01	0.00610729005254041\\
332.01	0.00610674903882134\\
333.01	0.00610619533965678\\
334.01	0.00610562863779777\\
335.01	0.00610504860762697\\
336.01	0.00610445491495908\\
337.01	0.00610384721684288\\
338.01	0.0061032251613627\\
339.01	0.00610258838744573\\
340.01	0.00610193652467055\\
341.01	0.00610126919308258\\
342.01	0.00610058600301584\\
343.01	0.00609988655492459\\
344.01	0.00609917043922596\\
345.01	0.00609843723615625\\
346.01	0.00609768651564444\\
347.01	0.0060969178372051\\
348.01	0.00609613074985648\\
349.01	0.00609532479206568\\
350.01	0.0060944994917266\\
351.01	0.00609365436617652\\
352.01	0.00609278892225648\\
353.01	0.00609190265642249\\
354.01	0.00609099505491372\\
355.01	0.00609006559398646\\
356.01	0.00608911374022216\\
357.01	0.0060881389509176\\
358.01	0.00608714067456685\\
359.01	0.00608611835144441\\
360.01	0.00608507141429824\\
361.01	0.00608399928916177\\
362.01	0.00608290139629182\\
363.01	0.00608177715123693\\
364.01	0.00608062596603969\\
365.01	0.00607944725056773\\
366.01	0.0060782404139671\\
367.01	0.00607700486621656\\
368.01	0.00607574001975567\\
369.01	0.00607444529113813\\
370.01	0.00607312010264809\\
371.01	0.00607176388378742\\
372.01	0.00607037607251521\\
373.01	0.00606895611608468\\
374.01	0.00606750347128157\\
375.01	0.00606601760382692\\
376.01	0.00606449798666036\\
377.01	0.00606294409678792\\
378.01	0.00606135541036019\\
379.01	0.00605973139566939\\
380.01	0.00605807150384671\\
381.01	0.00605637515725608\\
382.01	0.00605464173598649\\
383.01	0.00605287056357264\\
384.01	0.00605106089427924\\
385.01	0.0060492119062795\\
386.01	0.00604732271166291\\
387.01	0.00604539238725572\\
388.01	0.00604341998557221\\
389.01	0.00604140453428834\\
390.01	0.00603934503542485\\
391.01	0.00603724046449459\\
392.01	0.00603508976961249\\
393.01	0.0060328918705657\\
394.01	0.00603064565784248\\
395.01	0.00602834999161672\\
396.01	0.00602600370068583\\
397.01	0.00602360558135929\\
398.01	0.00602115439629455\\
399.01	0.00601864887327858\\
400.01	0.00601608770394962\\
401.01	0.00601346954245762\\
402.01	0.0060107930040585\\
403.01	0.00600805666363816\\
404.01	0.0060052590541629\\
405.01	0.00600239866505064\\
406.01	0.0059994739404585\\
407.01	0.00599648327748143\\
408.01	0.00599342502425666\\
409.01	0.00599029747796719\\
410.01	0.00598709888273879\\
411.01	0.00598382742742423\\
412.01	0.00598048124326533\\
413.01	0.00597705840142861\\
414.01	0.00597355691040341\\
415.01	0.0059699747132562\\
416.01	0.00596630968473113\\
417.01	0.0059625596281877\\
418.01	0.005958722272366\\
419.01	0.00595479526796988\\
420.01	0.00595077618405627\\
421.01	0.00594666250422184\\
422.01	0.00594245162257568\\
423.01	0.00593814083948637\\
424.01	0.00593372735709521\\
425.01	0.00592920827458402\\
426.01	0.00592458058318966\\
427.01	0.00591984116095729\\
428.01	0.00591498676722701\\
429.01	0.00591001403685044\\
430.01	0.00590491947413866\\
431.01	0.00589969944654457\\
432.01	0.00589435017809278\\
433.01	0.00588886774257454\\
434.01	0.00588324805653506\\
435.01	0.00587748687209578\\
436.01	0.00587157976966431\\
437.01	0.00586552215061011\\
438.01	0.00585930923000071\\
439.01	0.00585293602952843\\
440.01	0.00584639737078868\\
441.01	0.00583968786911788\\
442.01	0.00583280192824764\\
443.01	0.00582573373609972\\
444.01	0.00581847726211837\\
445.01	0.00581102625662853\\
446.01	0.00580337425281525\\
447.01	0.00579551457204062\\
448.01	0.00578744033335727\\
449.01	0.00577914446823792\\
450.01	0.0057706197417122\\
451.01	0.00576185878129259\\
452.01	0.00575285411524921\\
453.01	0.00574359822197355\\
454.01	0.0057340835922873\\
455.01	0.00572430280659619\\
456.01	0.00571424862866944\\
457.01	0.00570391411745597\\
458.01	0.00569329275756899\\
459.01	0.00568237860767353\\
460.01	0.00567116646366767\\
461.01	0.00565965202979944\\
462.01	0.00564783208504811\\
463.01	0.00563570462326951\\
464.01	0.00562326893240791\\
465.01	0.00561052555859875\\
466.01	0.00559747607251173\\
467.01	0.00558412251397143\\
468.01	0.00557046633115286\\
469.01	0.00555650547784293\\
470.01	0.00554223133538896\\
471.01	0.00552763343169557\\
472.01	0.0055127004134383\\
473.01	0.00549741991536809\\
474.01	0.00548177841033745\\
475.01	0.00546576103898249\\
476.01	0.00544935141910451\\
477.01	0.00543253143675166\\
478.01	0.00541528102426584\\
479.01	0.00539757793587015\\
480.01	0.00537939754017152\\
481.01	0.00536071265478665\\
482.01	0.00534149324328653\\
483.01	0.00532170581780566\\
484.01	0.0053013130343658\\
485.01	0.00528027341191707\\
486.01	0.00525854117617961\\
487.01	0.00523606632551913\\
488.01	0.00521279506473645\\
489.01	0.00518867082220332\\
490.01	0.00516363616514728\\
491.01	0.00513763577635102\\
492.01	0.00511061761989139\\
493.01	0.005082533328188\\
494.01	0.00505334193289948\\
495.01	0.00502301575464103\\
496.01	0.00499154895893617\\
497.01	0.00495896985165954\\
498.01	0.00492545203691938\\
499.01	0.00489134146735274\\
500.01	0.00485667159027827\\
501.01	0.00482146026587745\\
502.01	0.00478573084122105\\
503.01	0.00474951299744011\\
504.01	0.00471284366497251\\
505.01	0.00467576798570704\\
506.01	0.0046383402842913\\
507.01	0.00460062498440062\\
508.01	0.0045626973659973\\
509.01	0.00452464400539763\\
510.01	0.004486562661826\\
511.01	0.00444856126176999\\
512.01	0.00441075547270626\\
513.01	0.00437326413161136\\
514.01	0.00433620147457205\\
515.01	0.00429966466474605\\
516.01	0.00426371448546018\\
517.01	0.00422826400791097\\
518.01	0.00419282246814783\\
519.01	0.00415735756061899\\
520.01	0.00412191637869819\\
521.01	0.00408654552542253\\
522.01	0.00405128958565977\\
523.01	0.00401618912812359\\
524.01	0.0039812781188851\\
525.01	0.00394658067462882\\
526.01	0.00391210752426048\\
527.01	0.00387785257411629\\
528.01	0.00384378984601807\\
529.01	0.00380987145583695\\
530.01	0.00377602777227343\\
531.01	0.00374217157602695\\
532.01	0.00370821547002062\\
533.01	0.00367412944135205\\
534.01	0.00363990671496194\\
535.01	0.00360553450406351\\
536.01	0.0035709934348749\\
537.01	0.00353625745259954\\
538.01	0.00350129414109099\\
539.01	0.00346606557610664\\
540.01	0.00343052981613095\\
541.01	0.00339464308259496\\
542.01	0.00335836256461435\\
543.01	0.00332164955867218\\
544.01	0.00328447223652297\\
545.01	0.00324680501038172\\
546.01	0.00320862242941572\\
547.01	0.00316989738464901\\
548.01	0.00313060163178219\\
549.01	0.00309070644051983\\
550.01	0.00305018329158036\\
551.01	0.0030090045480664\\
552.01	0.00296714399849765\\
553.01	0.0029245771474564\\
554.01	0.00288128113628113\\
555.01	0.0028372342396416\\
556.01	0.00279241518212518\\
557.01	0.00274680296146442\\
558.01	0.0027003770362638\\
559.01	0.00265311750804224\\
560.01	0.0026050052522004\\
561.01	0.00255602198689257\\
562.01	0.00250615027724643\\
563.01	0.0024553734886416\\
564.01	0.00240367572505146\\
565.01	0.00235104180713719\\
566.01	0.00229745733122292\\
567.01	0.0022429087741126\\
568.01	0.00218738359614695\\
569.01	0.00213087033732275\\
570.01	0.00207335871199557\\
571.01	0.00201483971048969\\
572.01	0.00195530571712336\\
573.01	0.0018947506524752\\
574.01	0.0018331701421967\\
575.01	0.00177056170682958\\
576.01	0.00170692496523384\\
577.01	0.00164226185009494\\
578.01	0.00157657683664837\\
579.01	0.00150987718522774\\
580.01	0.00144217319681236\\
581.01	0.00137347847869735\\
582.01	0.0013038102150756\\
583.01	0.0012331894353757\\
584.01	0.00116164127211568\\
585.01	0.00108919519887159\\
586.01	0.00101588523658629\\
587.01	0.000941750113060405\\
588.01	0.00086683335623894\\
589.01	0.000791183296731767\\
590.01	0.000714852948722578\\
591.01	0.000637899730722589\\
592.01	0.00056038497794602\\
593.01	0.000482373185672669\\
594.01	0.000403930907138838\\
595.01	0.000325125209595895\\
596.01	0.000246021567360271\\
597.01	0.000166681039954817\\
598.01	9.13379815197009e-05\\
599.01	2.91271958372443e-05\\
599.02	2.86192783627535e-05\\
599.03	2.8114423742006e-05\\
599.04	2.76126617978888e-05\\
599.05	2.71140226472798e-05\\
599.06	2.66185367039581e-05\\
599.07	2.61262346815533e-05\\
599.08	2.56371475965082e-05\\
599.09	2.51513067710818e-05\\
599.1	2.46687438363903e-05\\
599.11	2.41894907354479e-05\\
599.12	2.37135797262807e-05\\
599.13	2.3241043385025e-05\\
599.14	2.27719146091033e-05\\
599.15	2.23062266203871e-05\\
599.16	2.18440129684284e-05\\
599.17	2.138530753369e-05\\
599.18	2.09301445308497e-05\\
599.19	2.04785585120812e-05\\
599.2	2.00305843704295e-05\\
599.21	1.95862573431644e-05\\
599.22	1.91456130152045e-05\\
599.23	1.87086873225627e-05\\
599.24	1.82755165558171e-05\\
599.25	1.78461373636363e-05\\
599.26	1.74205874164651e-05\\
599.27	1.69989072017485e-05\\
599.28	1.65811376101922e-05\\
599.29	1.61673199397579e-05\\
599.3	1.57574958996841e-05\\
599.31	1.53517076145714e-05\\
599.32	1.49499976284904e-05\\
599.33	1.45524089091315e-05\\
599.34	1.41589848520109e-05\\
599.35	1.37697692846831e-05\\
599.36	1.33848064710462e-05\\
599.37	1.3004141115644e-05\\
599.38	1.26278183680325e-05\\
599.39	1.22558838271912e-05\\
599.4	1.18883835459657e-05\\
599.41	1.15253640355657e-05\\
599.42	1.11668722700964e-05\\
599.43	1.08129556911502e-05\\
599.44	1.04636622124312e-05\\
599.45	1.01190402244239e-05\\
599.46	9.77913859911798e-06\\
599.47	9.44400669477576e-06\\
599.48	9.11369436074755e-06\\
599.49	8.7882519423238e-06\\
599.5	8.46773028565471e-06\\
599.51	8.15218074270464e-06\\
599.52	7.84165517625675e-06\\
599.53	7.53620596497147e-06\\
599.54	7.23588600849874e-06\\
599.55	6.94074873261973e-06\\
599.56	6.65084809447353e-06\\
599.57	6.36623858780126e-06\\
599.58	6.08697524826993e-06\\
599.59	5.81311365882228e-06\\
599.6	5.54470995511175e-06\\
599.61	5.28182083095637e-06\\
599.62	5.02450354387257e-06\\
599.63	4.7728159206558e-06\\
599.64	4.52681636300273e-06\\
599.65	4.28656385322197e-06\\
599.66	4.05211795995869e-06\\
599.67	3.8235388440163e-06\\
599.68	3.60088726420078e-06\\
599.69	3.38422458325341e-06\\
599.7	3.17361277382341e-06\\
599.71	2.96911442449269e-06\\
599.72	2.7707927458924e-06\\
599.73	2.57871157684567e-06\\
599.74	2.3929353905848e-06\\
599.75	2.213529301031e-06\\
599.76	2.04055906913997e-06\\
599.77	1.87409110929959e-06\\
599.78	1.71419249580217e-06\\
599.79	1.56093096936177e-06\\
599.8	1.41437494372704e-06\\
599.81	1.27459351233379e-06\\
599.82	1.14165645502366e-06\\
599.83	1.01563424484766e-06\\
599.84	8.96598054920053e-07\\
599.85	7.84619765348618e-07\\
599.86	6.79771970230753e-07\\
599.87	5.82127984719016e-07\\
599.88	4.91761852147374e-07\\
599.89	4.08748351251112e-07\\
599.9	3.33163003437068e-07\\
599.91	2.65082080131915e-07\\
599.92	2.04582610201579e-07\\
599.93	1.51742387452178e-07\\
599.94	1.06639978189951e-07\\
599.95	6.93547288800611e-08\\
599.96	3.99667738505e-08\\
599.97	1.85570430896731e-08\\
599.98	5.20727013418598e-09\\
599.99	0\\
600	0\\
};
\addplot [color=blue!50!mycolor7,solid,forget plot]
  table[row sep=crcr]{%
0.01	0.00643064630349385\\
1.01	0.00643064582104349\\
2.01	0.00643064532843654\\
3.01	0.00643064482545811\\
4.01	0.00643064431188899\\
5.01	0.00643064378750511\\
6.01	0.00643064325207764\\
7.01	0.00643064270537298\\
8.01	0.00643064214715251\\
9.01	0.00643064157717261\\
10.01	0.0064306409951844\\
11.01	0.00643064040093377\\
12.01	0.00643063979416122\\
13.01	0.00643063917460158\\
14.01	0.00643063854198422\\
15.01	0.00643063789603248\\
16.01	0.00643063723646407\\
17.01	0.00643063656299057\\
18.01	0.00643063587531759\\
19.01	0.0064306351731443\\
20.01	0.0064306344561634\\
21.01	0.00643063372406137\\
22.01	0.00643063297651745\\
23.01	0.00643063221320462\\
24.01	0.00643063143378841\\
25.01	0.00643063063792766\\
26.01	0.00643062982527373\\
27.01	0.0064306289954705\\
28.01	0.00643062814815421\\
29.01	0.00643062728295375\\
30.01	0.00643062639948946\\
31.01	0.00643062549737406\\
32.01	0.00643062457621179\\
33.01	0.00643062363559861\\
34.01	0.00643062267512169\\
35.01	0.00643062169435945\\
36.01	0.00643062069288122\\
37.01	0.00643061967024741\\
38.01	0.00643061862600884\\
39.01	0.00643061755970665\\
40.01	0.00643061647087234\\
41.01	0.00643061535902733\\
42.01	0.00643061422368286\\
43.01	0.00643061306433972\\
44.01	0.00643061188048794\\
45.01	0.00643061067160677\\
46.01	0.00643060943716425\\
47.01	0.00643060817661714\\
48.01	0.00643060688941042\\
49.01	0.00643060557497739\\
50.01	0.00643060423273907\\
51.01	0.00643060286210411\\
52.01	0.00643060146246859\\
53.01	0.00643060003321559\\
54.01	0.00643059857371514\\
55.01	0.00643059708332355\\
56.01	0.0064305955613835\\
57.01	0.00643059400722356\\
58.01	0.0064305924201578\\
59.01	0.00643059079948582\\
60.01	0.00643058914449208\\
61.01	0.00643058745444574\\
62.01	0.00643058572860015\\
63.01	0.00643058396619301\\
64.01	0.00643058216644525\\
65.01	0.00643058032856141\\
66.01	0.00643057845172888\\
67.01	0.00643057653511765\\
68.01	0.00643057457787966\\
69.01	0.00643057257914896\\
70.01	0.00643057053804087\\
71.01	0.00643056845365179\\
72.01	0.00643056632505849\\
73.01	0.00643056415131834\\
74.01	0.0064305619314682\\
75.01	0.00643055966452425\\
76.01	0.00643055734948144\\
77.01	0.00643055498531337\\
78.01	0.00643055257097158\\
79.01	0.00643055010538465\\
80.01	0.00643054758745861\\
81.01	0.00643054501607576\\
82.01	0.0064305423900943\\
83.01	0.00643053970834811\\
84.01	0.00643053696964572\\
85.01	0.00643053417277017\\
86.01	0.00643053131647824\\
87.01	0.00643052839949989\\
88.01	0.00643052542053795\\
89.01	0.006430522378267\\
90.01	0.00643051927133346\\
91.01	0.00643051609835429\\
92.01	0.00643051285791676\\
93.01	0.00643050954857762\\
94.01	0.00643050616886266\\
95.01	0.00643050271726579\\
96.01	0.0064304991922484\\
97.01	0.0064304955922388\\
98.01	0.00643049191563138\\
99.01	0.00643048816078568\\
100.01	0.00643048432602608\\
101.01	0.00643048040964076\\
102.01	0.00643047640988088\\
103.01	0.00643047232495973\\
104.01	0.00643046815305227\\
105.01	0.00643046389229391\\
106.01	0.00643045954077953\\
107.01	0.00643045509656325\\
108.01	0.00643045055765674\\
109.01	0.00643044592202909\\
110.01	0.00643044118760507\\
111.01	0.00643043635226488\\
112.01	0.00643043141384262\\
113.01	0.00643042637012548\\
114.01	0.00643042121885298\\
115.01	0.00643041595771548\\
116.01	0.0064304105843536\\
117.01	0.00643040509635664\\
118.01	0.00643039949126147\\
119.01	0.00643039376655211\\
120.01	0.00643038791965752\\
121.01	0.00643038194795135\\
122.01	0.00643037584874987\\
123.01	0.00643036961931142\\
124.01	0.00643036325683483\\
125.01	0.00643035675845791\\
126.01	0.00643035012125671\\
127.01	0.00643034334224343\\
128.01	0.00643033641836551\\
129.01	0.00643032934650421\\
130.01	0.00643032212347274\\
131.01	0.00643031474601531\\
132.01	0.00643030721080514\\
133.01	0.00643029951444309\\
134.01	0.00643029165345632\\
135.01	0.00643028362429626\\
136.01	0.00643027542333703\\
137.01	0.00643026704687377\\
138.01	0.00643025849112109\\
139.01	0.00643024975221125\\
140.01	0.00643024082619188\\
141.01	0.00643023170902473\\
142.01	0.00643022239658341\\
143.01	0.00643021288465138\\
144.01	0.00643020316892044\\
145.01	0.00643019324498823\\
146.01	0.00643018310835617\\
147.01	0.00643017275442751\\
148.01	0.00643016217850524\\
149.01	0.00643015137578944\\
150.01	0.00643014034137553\\
151.01	0.00643012907025174\\
152.01	0.00643011755729637\\
153.01	0.00643010579727588\\
154.01	0.0064300937848422\\
155.01	0.00643008151453026\\
156.01	0.00643006898075504\\
157.01	0.00643005617780936\\
158.01	0.00643004309986107\\
159.01	0.00643002974095002\\
160.01	0.00643001609498536\\
161.01	0.00643000215574258\\
162.01	0.00642998791686077\\
163.01	0.00642997337183905\\
164.01	0.00642995851403413\\
165.01	0.00642994333665655\\
166.01	0.00642992783276766\\
167.01	0.00642991199527631\\
168.01	0.0064298958169353\\
169.01	0.00642987929033816\\
170.01	0.00642986240791533\\
171.01	0.00642984516193045\\
172.01	0.00642982754447661\\
173.01	0.00642980954747299\\
174.01	0.00642979116266056\\
175.01	0.006429772381598\\
176.01	0.0064297531956577\\
177.01	0.00642973359602176\\
178.01	0.00642971357367735\\
179.01	0.00642969311941259\\
180.01	0.00642967222381217\\
181.01	0.00642965087725234\\
182.01	0.00642962906989663\\
183.01	0.0064296067916905\\
184.01	0.00642958403235734\\
185.01	0.00642956078139252\\
186.01	0.00642953702805845\\
187.01	0.00642951276137988\\
188.01	0.00642948797013775\\
189.01	0.006429462642864\\
190.01	0.00642943676783618\\
191.01	0.00642941033307095\\
192.01	0.00642938332631894\\
193.01	0.0064293557350581\\
194.01	0.00642932754648815\\
195.01	0.00642929874752325\\
196.01	0.00642926932478665\\
197.01	0.00642923926460298\\
198.01	0.00642920855299219\\
199.01	0.00642917717566235\\
200.01	0.00642914511800234\\
201.01	0.00642911236507465\\
202.01	0.00642907890160809\\
203.01	0.00642904471198972\\
204.01	0.00642900978025752\\
205.01	0.00642897409009218\\
206.01	0.00642893762480835\\
207.01	0.006428900367347\\
208.01	0.00642886230026659\\
209.01	0.00642882340573394\\
210.01	0.00642878366551559\\
211.01	0.00642874306096793\\
212.01	0.0064287015730287\\
213.01	0.00642865918220635\\
214.01	0.00642861586857064\\
215.01	0.00642857161174234\\
216.01	0.00642852639088269\\
217.01	0.00642848018468272\\
218.01	0.00642843297135269\\
219.01	0.00642838472861051\\
220.01	0.00642833543367017\\
221.01	0.00642828506323037\\
222.01	0.00642823359346217\\
223.01	0.00642818099999707\\
224.01	0.00642812725791366\\
225.01	0.00642807234172527\\
226.01	0.00642801622536703\\
227.01	0.00642795888218092\\
228.01	0.00642790028490337\\
229.01	0.00642784040565012\\
230.01	0.00642777921590195\\
231.01	0.00642771668648957\\
232.01	0.00642765278757831\\
233.01	0.00642758748865235\\
234.01	0.00642752075849874\\
235.01	0.00642745256519069\\
236.01	0.00642738287607096\\
237.01	0.00642731165773421\\
238.01	0.00642723887600928\\
239.01	0.00642716449594127\\
240.01	0.00642708848177243\\
241.01	0.00642701079692326\\
242.01	0.00642693140397324\\
243.01	0.00642685026463986\\
244.01	0.00642676733975887\\
245.01	0.00642668258926261\\
246.01	0.00642659597215879\\
247.01	0.00642650744650802\\
248.01	0.0064264169694011\\
249.01	0.00642632449693564\\
250.01	0.00642622998419238\\
251.01	0.00642613338521051\\
252.01	0.00642603465296263\\
253.01	0.00642593373932892\\
254.01	0.00642583059507045\\
255.01	0.00642572516980294\\
256.01	0.00642561741196819\\
257.01	0.00642550726880583\\
258.01	0.00642539468632444\\
259.01	0.00642527960927127\\
260.01	0.0064251619811019\\
261.01	0.0064250417439487\\
262.01	0.00642491883858825\\
263.01	0.00642479320440894\\
264.01	0.00642466477937649\\
265.01	0.00642453349999969\\
266.01	0.00642439930129447\\
267.01	0.00642426211674758\\
268.01	0.00642412187827902\\
269.01	0.00642397851620397\\
270.01	0.00642383195919334\\
271.01	0.00642368213423309\\
272.01	0.00642352896658358\\
273.01	0.00642337237973691\\
274.01	0.00642321229537337\\
275.01	0.00642304863331709\\
276.01	0.00642288131149044\\
277.01	0.00642271024586754\\
278.01	0.00642253535042629\\
279.01	0.00642235653709898\\
280.01	0.00642217371572249\\
281.01	0.00642198679398665\\
282.01	0.00642179567738139\\
283.01	0.00642160026914291\\
284.01	0.00642140047019842\\
285.01	0.00642119617910903\\
286.01	0.00642098729201231\\
287.01	0.00642077370256248\\
288.01	0.00642055530186996\\
289.01	0.0064203319784387\\
290.01	0.00642010361810299\\
291.01	0.0064198701039615\\
292.01	0.00641963131631155\\
293.01	0.00641938713257997\\
294.01	0.00641913742725404\\
295.01	0.00641888207180969\\
296.01	0.00641862093463861\\
297.01	0.00641835388097361\\
298.01	0.00641808077281268\\
299.01	0.00641780146884092\\
300.01	0.00641751582435062\\
301.01	0.00641722369116083\\
302.01	0.0064169249175334\\
303.01	0.00641661934808929\\
304.01	0.00641630682372139\\
305.01	0.00641598718150675\\
306.01	0.00641566025461678\\
307.01	0.00641532587222526\\
308.01	0.00641498385941516\\
309.01	0.00641463403708344\\
310.01	0.00641427622184446\\
311.01	0.00641391022593068\\
312.01	0.00641353585709311\\
313.01	0.00641315291849869\\
314.01	0.00641276120862677\\
315.01	0.00641236052116347\\
316.01	0.00641195064489498\\
317.01	0.00641153136359834\\
318.01	0.00641110245593143\\
319.01	0.00641066369532103\\
320.01	0.00641021484984945\\
321.01	0.00640975568213994\\
322.01	0.00640928594923997\\
323.01	0.006408805402504\\
324.01	0.00640831378747466\\
325.01	0.0064078108437623\\
326.01	0.00640729630492403\\
327.01	0.00640676989834167\\
328.01	0.006406231345098\\
329.01	0.00640568035985288\\
330.01	0.00640511665071823\\
331.01	0.00640453991913279\\
332.01	0.00640394985973552\\
333.01	0.00640334616023881\\
334.01	0.00640272850130228\\
335.01	0.00640209655640494\\
336.01	0.00640144999171812\\
337.01	0.00640078846597794\\
338.01	0.00640011163035872\\
339.01	0.00639941912834587\\
340.01	0.00639871059560917\\
341.01	0.00639798565987746\\
342.01	0.00639724394081292\\
343.01	0.00639648504988657\\
344.01	0.00639570859025427\\
345.01	0.00639491415663374\\
346.01	0.0063941013351825\\
347.01	0.00639326970337583\\
348.01	0.0063924188298865\\
349.01	0.00639154827446339\\
350.01	0.00639065758781175\\
351.01	0.0063897463114725\\
352.01	0.00638881397770028\\
353.01	0.00638786010934019\\
354.01	0.00638688421970219\\
355.01	0.00638588581243139\\
356.01	0.00638486438137207\\
357.01	0.00638381941042528\\
358.01	0.00638275037339532\\
359.01	0.00638165673382387\\
360.01	0.00638053794480633\\
361.01	0.00637939344878729\\
362.01	0.00637822267732865\\
363.01	0.00637702505084507\\
364.01	0.00637579997829815\\
365.01	0.00637454685684178\\
366.01	0.00637326507140846\\
367.01	0.00637195399422571\\
368.01	0.00637061298425105\\
369.01	0.00636924138651295\\
370.01	0.00636783853134532\\
371.01	0.0063664037335029\\
372.01	0.00636493629114794\\
373.01	0.00636343548470029\\
374.01	0.00636190057554987\\
375.01	0.00636033080463596\\
376.01	0.00635872539091282\\
377.01	0.00635708352973136\\
378.01	0.00635540439119044\\
379.01	0.00635368711853009\\
380.01	0.00635193082666611\\
381.01	0.00635013460098661\\
382.01	0.0063482974965479\\
383.01	0.00634641853779242\\
384.01	0.00634449671886276\\
385.01	0.00634253100443558\\
386.01	0.00634052033068126\\
387.01	0.00633846360543564\\
388.01	0.0063363597075115\\
389.01	0.00633420748584856\\
390.01	0.0063320057586354\\
391.01	0.00632975331240574\\
392.01	0.00632744890110624\\
393.01	0.00632509124513656\\
394.01	0.00632267903036051\\
395.01	0.00632021090708838\\
396.01	0.00631768548902773\\
397.01	0.00631510135220493\\
398.01	0.00631245703385488\\
399.01	0.00630975103127821\\
400.01	0.00630698180066793\\
401.01	0.00630414775590178\\
402.01	0.00630124726730199\\
403.01	0.00629827866036199\\
404.01	0.00629524021443894\\
405.01	0.00629213016141334\\
406.01	0.00628894668431449\\
407.01	0.00628568791591252\\
408.01	0.0062823519372778\\
409.01	0.00627893677630708\\
410.01	0.00627544040621913\\
411.01	0.00627186074401835\\
412.01	0.00626819564893003\\
413.01	0.00626444292080767\\
414.01	0.00626060029851503\\
415.01	0.00625666545828518\\
416.01	0.00625263601205956\\
417.01	0.00624850950581186\\
418.01	0.00624428341785974\\
419.01	0.00623995515716993\\
420.01	0.00623552206166427\\
421.01	0.00623098139653246\\
422.01	0.00622633035255973\\
423.01	0.00622156604448145\\
424.01	0.00621668550937332\\
425.01	0.00621168570509246\\
426.01	0.00620656350878387\\
427.01	0.00620131571546979\\
428.01	0.00619593903674235\\
429.01	0.00619043009958347\\
430.01	0.00618478544533691\\
431.01	0.00617900152886527\\
432.01	0.0061730747179242\\
433.01	0.00616700129279502\\
434.01	0.0061607774462197\\
435.01	0.00615439928368823\\
436.01	0.00614786282413647\\
437.01	0.00614116400111684\\
438.01	0.00613429866451252\\
439.01	0.00612726258287525\\
440.01	0.00612005144647123\\
441.01	0.00611266087113022\\
442.01	0.00610508640299985\\
443.01	0.00609732352431131\\
444.01	0.00608936766027169\\
445.01	0.00608121418719717\\
446.01	0.00607285844200196\\
447.01	0.00606429573315153\\
448.01	0.00605552135317634\\
449.01	0.00604653059281799\\
450.01	0.00603731875684806\\
451.01	0.00602788118154597\\
452.01	0.00601821325375907\\
453.01	0.00600831043136694\\
454.01	0.00599816826485659\\
455.01	0.00598778241955891\\
456.01	0.00597714869790946\\
457.01	0.00596626306087276\\
458.01	0.00595512164741876\\
459.01	0.00594372079067494\\
460.01	0.0059320570291298\\
461.01	0.00592012711109118\\
462.01	0.00590792799059727\\
463.01	0.0058954568132993\\
464.01	0.00588271089173592\\
465.01	0.00586968767129047\\
466.01	0.00585638469160731\\
467.01	0.00584279955437026\\
468.01	0.00582892991892154\\
469.01	0.00581477357391869\\
470.01	0.00580032866011119\\
471.01	0.00578559382851982\\
472.01	0.00577056826338928\\
473.01	0.00575525166452245\\
474.01	0.00573964417868944\\
475.01	0.00572374625327787\\
476.01	0.00570755837379365\\
477.01	0.00569108063047653\\
478.01	0.00567431203588562\\
479.01	0.00565724948118748\\
480.01	0.00563988617705282\\
481.01	0.00562221074290286\\
482.01	0.00560421942995308\\
483.01	0.00558591482612433\\
484.01	0.0055672980890903\\
485.01	0.00554836727342983\\
486.01	0.00552911500826401\\
487.01	0.0055095251175396\\
488.01	0.00548956776392307\\
489.01	0.00546919253983309\\
490.01	0.00544831871406007\\
491.01	0.00542684577263245\\
492.01	0.00540473991092167\\
493.01	0.00538199139560689\\
494.01	0.00535859211425415\\
495.01	0.00533453520495696\\
496.01	0.00530981416923446\\
497.01	0.00528442116996276\\
498.01	0.00525834343539328\\
499.01	0.00523155481476314\\
500.01	0.00520402129849684\\
501.01	0.00517570600084901\\
502.01	0.00514656899797631\\
503.01	0.00511656711354349\\
504.01	0.00508565372376103\\
505.01	0.00505377859573597\\
506.01	0.00502088779737199\\
507.01	0.00498692380955012\\
508.01	0.0049518259575015\\
509.01	0.00491553127513892\\
510.01	0.00487797600476396\\
511.01	0.00483909802366951\\
512.01	0.00479884060839485\\
513.01	0.0047571581135219\\
514.01	0.00471402437318059\\
515.01	0.00466944495525777\\
516.01	0.0046234748460071\\
517.01	0.00457632493610646\\
518.01	0.00452859323452803\\
519.01	0.00448040581544159\\
520.01	0.00443181764632897\\
521.01	0.00438289402962435\\
522.01	0.00433371322031132\\
523.01	0.00428437016558703\\
524.01	0.00423498183390211\\
525.01	0.00418569014161186\\
526.01	0.00413665505471583\\
527.01	0.00408804816755842\\
528.01	0.00404004460393849\\
529.01	0.00399281066845933\\
530.01	0.00394648423884527\\
531.01	0.00390114445455031\\
532.01	0.00385652207826656\\
533.01	0.00381203123192864\\
534.01	0.00376770964252486\\
535.01	0.00372361446977168\\
536.01	0.00367979415281411\\
537.01	0.00363628455305909\\
538.01	0.00359310461580987\\
539.01	0.00355025179220907\\
540.01	0.00350769767107562\\
541.01	0.00346538460813856\\
542.01	0.00342322468084414\\
543.01	0.00338110309495784\\
544.01	0.00333889140574861\\
545.01	0.00329651627776666\\
546.01	0.00325395959978935\\
547.01	0.00321119741588083\\
548.01	0.00316819686683313\\
549.01	0.00312491664011164\\
550.01	0.00308130814178034\\
551.01	0.00303731755150926\\
552.01	0.00299288892686615\\
553.01	0.0029479682313795\\
554.01	0.0029025077891753\\
555.01	0.00285646988285535\\
556.01	0.00280982387424166\\
557.01	0.0027625388315893\\
558.01	0.00271458284913515\\
559.01	0.00266592390435199\\
560.01	0.00261653075024179\\
561.01	0.00256637376772603\\
562.01	0.00251542565903956\\
563.01	0.0024636617766947\\
564.01	0.00241105990989469\\
565.01	0.00235759950936004\\
566.01	0.00230326096645182\\
567.01	0.00224802567304017\\
568.01	0.00219187631544338\\
569.01	0.00213479711176823\\
570.01	0.00207677396396737\\
571.01	0.00201779451774308\\
572.01	0.00195784815156964\\
573.01	0.00189692594923144\\
574.01	0.00183502073790612\\
575.01	0.0017721272474868\\
576.01	0.00170824233381175\\
577.01	0.0016433652017279\\
578.01	0.00157749762286559\\
579.01	0.00151064415615248\\
580.01	0.00144281238208389\\
581.01	0.00137401316117766\\
582.01	0.0013042609204186\\
583.01	0.00123357395803039\\
584.01	0.00116197474577273\\
585.01	0.00108949021141372\\
586.01	0.00101615198804251\\
587.01	0.000941996614273563\\
588.01	0.000867065664706434\\
589.01	0.000791405783613175\\
590.01	0.000715068586964641\\
591.01	0.000638110389654499\\
592.01	0.000560591707059254\\
593.01	0.000482576471267222\\
594.01	0.000404130889408053\\
595.01	0.000325321853632796\\
596.01	0.000246214788620124\\
597.01	0.000166870791567188\\
598.01	9.1337981536533e-05\\
599.01	2.91271958373258e-05\\
599.02	2.86192783628299e-05\\
599.03	2.81144237420771e-05\\
599.04	2.76126617979548e-05\\
599.05	2.71140226473406e-05\\
599.06	2.66185367040154e-05\\
599.07	2.61262346816053e-05\\
599.08	2.5637147596555e-05\\
599.09	2.51513067711269e-05\\
599.1	2.46687438364285e-05\\
599.11	2.41894907354861e-05\\
599.12	2.37135797263137e-05\\
599.13	2.32410433850579e-05\\
599.14	2.27719146091345e-05\\
599.15	2.23062266204149e-05\\
599.16	2.1844012968451e-05\\
599.17	2.13853075337143e-05\\
599.18	2.09301445308688e-05\\
599.19	2.04785585121003e-05\\
599.2	2.00305843704451e-05\\
599.21	1.95862573431783e-05\\
599.22	1.91456130152166e-05\\
599.23	1.87086873225731e-05\\
599.24	1.82755165558292e-05\\
599.25	1.78461373636484e-05\\
599.26	1.74205874164755e-05\\
599.27	1.69989072017589e-05\\
599.28	1.65811376102026e-05\\
599.29	1.61673199397631e-05\\
599.3	1.57574958996893e-05\\
599.31	1.53517076145766e-05\\
599.32	1.49499976284957e-05\\
599.33	1.45524089091385e-05\\
599.34	1.41589848520127e-05\\
599.35	1.37697692846866e-05\\
599.36	1.33848064710479e-05\\
599.37	1.30041411156457e-05\\
599.38	1.2627818368036e-05\\
599.39	1.22558838271929e-05\\
599.4	1.18883835459691e-05\\
599.41	1.15253640355674e-05\\
599.42	1.11668722700981e-05\\
599.43	1.08129556911536e-05\\
599.44	1.04636622124312e-05\\
599.45	1.01190402244239e-05\\
599.46	9.77913859911798e-06\\
599.47	9.44400669477749e-06\\
599.48	9.11369436074755e-06\\
599.49	8.7882519423238e-06\\
599.5	8.46773028565471e-06\\
599.51	8.15218074270464e-06\\
599.52	7.84165517625848e-06\\
599.53	7.5362059649732e-06\\
599.54	7.23588600849874e-06\\
599.55	6.94074873262146e-06\\
599.56	6.65084809447526e-06\\
599.57	6.36623858780126e-06\\
599.58	6.08697524826819e-06\\
599.59	5.81311365882228e-06\\
599.6	5.54470995511175e-06\\
599.61	5.28182083095637e-06\\
599.62	5.02450354387431e-06\\
599.63	4.7728159206558e-06\\
599.64	4.52681636300446e-06\\
599.65	4.28656385322197e-06\\
599.66	4.05211795996042e-06\\
599.67	3.82353884401457e-06\\
599.68	3.60088726420078e-06\\
599.69	3.38422458325514e-06\\
599.7	3.17361277382168e-06\\
599.71	2.96911442449269e-06\\
599.72	2.77079274589413e-06\\
599.73	2.5787115768474e-06\\
599.74	2.39293539058306e-06\\
599.75	2.21352930102926e-06\\
599.76	2.04055906913997e-06\\
599.77	1.87409110929959e-06\\
599.78	1.71419249580043e-06\\
599.79	1.56093096936177e-06\\
599.8	1.41437494372877e-06\\
599.81	1.27459351233379e-06\\
599.82	1.14165645502366e-06\\
599.83	1.01563424484766e-06\\
599.84	8.96598054920053e-07\\
599.85	7.84619765350353e-07\\
599.86	6.79771970232487e-07\\
599.87	5.82127984717282e-07\\
599.88	4.91761852145639e-07\\
599.89	4.08748351251112e-07\\
599.9	3.33163003438802e-07\\
599.91	2.6508208013365e-07\\
599.92	2.04582610201579e-07\\
599.93	1.51742387450443e-07\\
599.94	1.06639978191686e-07\\
599.95	6.93547288800611e-08\\
599.96	3.99667738487652e-08\\
599.97	1.85570430896731e-08\\
599.98	5.20727013418598e-09\\
599.99	0\\
600	0\\
};
\addplot [color=blue!40!mycolor9,solid,forget plot]
  table[row sep=crcr]{%
0.01	0.00729752532496896\\
1.01	0.00729752466570372\\
2.01	0.00729752399247544\\
3.01	0.00729752330498688\\
4.01	0.00729752260293404\\
5.01	0.00729752188600661\\
6.01	0.00729752115388762\\
7.01	0.00729752040625335\\
8.01	0.00729751964277303\\
9.01	0.00729751886310865\\
10.01	0.00729751806691535\\
11.01	0.00729751725384054\\
12.01	0.00729751642352408\\
13.01	0.0072975155755983\\
14.01	0.00729751470968734\\
15.01	0.0072975138254076\\
16.01	0.00729751292236697\\
17.01	0.00729751200016488\\
18.01	0.00729751105839236\\
19.01	0.0072975100966315\\
20.01	0.00729750911445554\\
21.01	0.00729750811142823\\
22.01	0.00729750708710435\\
23.01	0.00729750604102884\\
24.01	0.00729750497273698\\
25.01	0.00729750388175379\\
26.01	0.0072975027675944\\
27.01	0.00729750162976317\\
28.01	0.00729750046775411\\
29.01	0.00729749928104998\\
30.01	0.00729749806912268\\
31.01	0.00729749683143267\\
32.01	0.00729749556742859\\
33.01	0.00729749427654743\\
34.01	0.00729749295821389\\
35.01	0.00729749161184042\\
36.01	0.00729749023682674\\
37.01	0.00729748883255935\\
38.01	0.00729748739841178\\
39.01	0.00729748593374409\\
40.01	0.00729748443790231\\
41.01	0.00729748291021846\\
42.01	0.00729748135001015\\
43.01	0.00729747975658009\\
44.01	0.00729747812921618\\
45.01	0.00729747646719074\\
46.01	0.00729747476976034\\
47.01	0.00729747303616558\\
48.01	0.00729747126563063\\
49.01	0.00729746945736278\\
50.01	0.00729746761055222\\
51.01	0.00729746572437177\\
52.01	0.00729746379797607\\
53.01	0.0072974618305016\\
54.01	0.00729745982106604\\
55.01	0.00729745776876817\\
56.01	0.00729745567268704\\
57.01	0.00729745353188178\\
58.01	0.00729745134539126\\
59.01	0.00729744911223331\\
60.01	0.00729744683140454\\
61.01	0.00729744450187991\\
62.01	0.00729744212261201\\
63.01	0.00729743969253067\\
64.01	0.00729743721054262\\
65.01	0.0072974346755307\\
66.01	0.00729743208635365\\
67.01	0.00729742944184526\\
68.01	0.00729742674081414\\
69.01	0.00729742398204288\\
70.01	0.00729742116428783\\
71.01	0.00729741828627794\\
72.01	0.00729741534671493\\
73.01	0.00729741234427186\\
74.01	0.0072974092775931\\
75.01	0.0072974061452937\\
76.01	0.00729740294595837\\
77.01	0.00729739967814097\\
78.01	0.00729739634036386\\
79.01	0.00729739293111747\\
80.01	0.00729738944885902\\
81.01	0.00729738589201225\\
82.01	0.0072973822589665\\
83.01	0.00729737854807601\\
84.01	0.00729737475765905\\
85.01	0.00729737088599732\\
86.01	0.00729736693133495\\
87.01	0.00729736289187786\\
88.01	0.00729735876579246\\
89.01	0.00729735455120554\\
90.01	0.00729735024620267\\
91.01	0.0072973458488277\\
92.01	0.00729734135708172\\
93.01	0.0072973367689221\\
94.01	0.00729733208226154\\
95.01	0.007297327294967\\
96.01	0.00729732240485881\\
97.01	0.00729731740970981\\
98.01	0.00729731230724368\\
99.01	0.00729730709513456\\
100.01	0.00729730177100555\\
101.01	0.00729729633242758\\
102.01	0.00729729077691843\\
103.01	0.00729728510194151\\
104.01	0.00729727930490448\\
105.01	0.00729727338315806\\
106.01	0.00729726733399516\\
107.01	0.007297261154649\\
108.01	0.00729725484229201\\
109.01	0.00729724839403466\\
110.01	0.00729724180692394\\
111.01	0.00729723507794183\\
112.01	0.00729722820400403\\
113.01	0.00729722118195852\\
114.01	0.00729721400858374\\
115.01	0.00729720668058757\\
116.01	0.00729719919460506\\
117.01	0.00729719154719732\\
118.01	0.00729718373485004\\
119.01	0.00729717575397083\\
120.01	0.00729716760088874\\
121.01	0.00729715927185167\\
122.01	0.00729715076302475\\
123.01	0.0072971420704886\\
124.01	0.00729713319023731\\
125.01	0.00729712411817658\\
126.01	0.00729711485012171\\
127.01	0.00729710538179554\\
128.01	0.00729709570882657\\
129.01	0.00729708582674661\\
130.01	0.00729707573098867\\
131.01	0.00729706541688462\\
132.01	0.00729705487966345\\
133.01	0.0072970441144483\\
134.01	0.00729703311625429\\
135.01	0.00729702187998626\\
136.01	0.00729701040043639\\
137.01	0.00729699867228122\\
138.01	0.00729698669007947\\
139.01	0.00729697444826889\\
140.01	0.00729696194116412\\
141.01	0.00729694916295372\\
142.01	0.00729693610769704\\
143.01	0.00729692276932173\\
144.01	0.00729690914162044\\
145.01	0.00729689521824786\\
146.01	0.00729688099271778\\
147.01	0.00729686645839981\\
148.01	0.0072968516085158\\
149.01	0.0072968364361372\\
150.01	0.00729682093418094\\
151.01	0.00729680509540626\\
152.01	0.0072967889124112\\
153.01	0.00729677237762894\\
154.01	0.00729675548332381\\
155.01	0.00729673822158769\\
156.01	0.0072967205843362\\
157.01	0.00729670256330462\\
158.01	0.00729668415004327\\
159.01	0.00729666533591438\\
160.01	0.00729664611208676\\
161.01	0.00729662646953205\\
162.01	0.00729660639902012\\
163.01	0.00729658589111428\\
164.01	0.00729656493616682\\
165.01	0.00729654352431402\\
166.01	0.00729652164547135\\
167.01	0.00729649928932837\\
168.01	0.00729647644534364\\
169.01	0.00729645310273948\\
170.01	0.00729642925049629\\
171.01	0.00729640487734771\\
172.01	0.00729637997177418\\
173.01	0.00729635452199763\\
174.01	0.00729632851597542\\
175.01	0.00729630194139441\\
176.01	0.00729627478566468\\
177.01	0.00729624703591328\\
178.01	0.00729621867897761\\
179.01	0.00729618970139889\\
180.01	0.00729616008941532\\
181.01	0.00729612982895523\\
182.01	0.00729609890562987\\
183.01	0.00729606730472636\\
184.01	0.00729603501119971\\
185.01	0.00729600200966582\\
186.01	0.00729596828439349\\
187.01	0.00729593381929612\\
188.01	0.00729589859792411\\
189.01	0.00729586260345605\\
190.01	0.00729582581869023\\
191.01	0.00729578822603618\\
192.01	0.00729574980750549\\
193.01	0.00729571054470255\\
194.01	0.00729567041881541\\
195.01	0.00729562941060609\\
196.01	0.00729558750040048\\
197.01	0.00729554466807877\\
198.01	0.00729550089306479\\
199.01	0.00729545615431571\\
200.01	0.00729541043031107\\
201.01	0.00729536369904184\\
202.01	0.00729531593799909\\
203.01	0.00729526712416246\\
204.01	0.00729521723398821\\
205.01	0.00729516624339727\\
206.01	0.00729511412776297\\
207.01	0.00729506086189784\\
208.01	0.00729500642004107\\
209.01	0.00729495077584507\\
210.01	0.00729489390236179\\
211.01	0.00729483577202908\\
212.01	0.00729477635665604\\
213.01	0.0072947156274087\\
214.01	0.007294653554795\\
215.01	0.00729459010864934\\
216.01	0.00729452525811728\\
217.01	0.00729445897163931\\
218.01	0.00729439121693432\\
219.01	0.007294321960983\\
220.01	0.00729425117001076\\
221.01	0.0072941788094698\\
222.01	0.00729410484402151\\
223.01	0.00729402923751758\\
224.01	0.00729395195298176\\
225.01	0.00729387295258993\\
226.01	0.00729379219765052\\
227.01	0.00729370964858466\\
228.01	0.00729362526490501\\
229.01	0.0072935390051947\\
230.01	0.00729345082708572\\
231.01	0.00729336068723664\\
232.01	0.00729326854130991\\
233.01	0.00729317434394859\\
234.01	0.00729307804875235\\
235.01	0.00729297960825348\\
236.01	0.00729287897389162\\
237.01	0.00729277609598836\\
238.01	0.00729267092372104\\
239.01	0.00729256340509588\\
240.01	0.00729245348692089\\
241.01	0.00729234111477758\\
242.01	0.00729222623299217\\
243.01	0.00729210878460673\\
244.01	0.00729198871134848\\
245.01	0.00729186595359962\\
246.01	0.00729174045036516\\
247.01	0.00729161213924123\\
248.01	0.00729148095638202\\
249.01	0.00729134683646588\\
250.01	0.00729120971266077\\
251.01	0.00729106951658887\\
252.01	0.00729092617829042\\
253.01	0.00729077962618685\\
254.01	0.00729062978704287\\
255.01	0.00729047658592718\\
256.01	0.00729031994617333\\
257.01	0.00729015978933894\\
258.01	0.00728999603516395\\
259.01	0.00728982860152826\\
260.01	0.00728965740440818\\
261.01	0.00728948235783168\\
262.01	0.00728930337383331\\
263.01	0.00728912036240686\\
264.01	0.00728893323145826\\
265.01	0.00728874188675637\\
266.01	0.00728854623188319\\
267.01	0.00728834616818269\\
268.01	0.00728814159470866\\
269.01	0.00728793240817102\\
270.01	0.00728771850288104\\
271.01	0.00728749977069578\\
272.01	0.00728727610096022\\
273.01	0.00728704738044919\\
274.01	0.00728681349330705\\
275.01	0.00728657432098675\\
276.01	0.0072863297421869\\
277.01	0.00728607963278766\\
278.01	0.00728582386578517\\
279.01	0.00728556231122452\\
280.01	0.00728529483613123\\
281.01	0.00728502130444085\\
282.01	0.00728474157692739\\
283.01	0.00728445551113003\\
284.01	0.00728416296127768\\
285.01	0.00728386377821308\\
286.01	0.00728355780931356\\
287.01	0.00728324489841133\\
288.01	0.00728292488571118\\
289.01	0.00728259760770695\\
290.01	0.00728226289709558\\
291.01	0.00728192058268982\\
292.01	0.00728157048932837\\
293.01	0.00728121243778467\\
294.01	0.00728084624467267\\
295.01	0.00728047172235189\\
296.01	0.00728008867882943\\
297.01	0.00727969691765975\\
298.01	0.00727929623784288\\
299.01	0.00727888643371967\\
300.01	0.00727846729486543\\
301.01	0.00727803860598041\\
302.01	0.00727760014677891\\
303.01	0.00727715169187461\\
304.01	0.00727669301066443\\
305.01	0.00727622386720973\\
306.01	0.007275744020114\\
307.01	0.00727525322239907\\
308.01	0.00727475122137744\\
309.01	0.00727423775852274\\
310.01	0.0072737125693362\\
311.01	0.00727317538321151\\
312.01	0.00727262592329506\\
313.01	0.00727206390634452\\
314.01	0.00727148904258305\\
315.01	0.00727090103555079\\
316.01	0.00727029958195293\\
317.01	0.00726968437150379\\
318.01	0.00726905508676791\\
319.01	0.00726841140299698\\
320.01	0.00726775298796305\\
321.01	0.00726707950178749\\
322.01	0.00726639059676636\\
323.01	0.00726568591719115\\
324.01	0.00726496509916505\\
325.01	0.00726422777041479\\
326.01	0.00726347355009767\\
327.01	0.00726270204860363\\
328.01	0.00726191286735221\\
329.01	0.00726110559858428\\
330.01	0.00726027982514809\\
331.01	0.0072594351202791\\
332.01	0.00725857104737481\\
333.01	0.00725768715976247\\
334.01	0.00725678300046007\\
335.01	0.00725585810193118\\
336.01	0.00725491198583203\\
337.01	0.00725394416275126\\
338.01	0.00725295413194201\\
339.01	0.00725194138104491\\
340.01	0.00725090538580397\\
341.01	0.00724984560977174\\
342.01	0.0072487615040063\\
343.01	0.00724765250675767\\
344.01	0.00724651804314418\\
345.01	0.00724535752481752\\
346.01	0.00724417034961681\\
347.01	0.00724295590121094\\
348.01	0.00724171354872714\\
349.01	0.00724044264636811\\
350.01	0.00723914253301424\\
351.01	0.00723781253181164\\
352.01	0.00723645194974541\\
353.01	0.0072350600771969\\
354.01	0.00723363618748432\\
355.01	0.00723217953638594\\
356.01	0.00723068936164578\\
357.01	0.00722916488245954\\
358.01	0.00722760529894151\\
359.01	0.00722600979157003\\
360.01	0.00722437752061259\\
361.01	0.00722270762552784\\
362.01	0.00722099922434488\\
363.01	0.00721925141301872\\
364.01	0.00721746326476159\\
365.01	0.00721563382934946\\
366.01	0.00721376213240344\\
367.01	0.00721184717464625\\
368.01	0.00720988793113445\\
369.01	0.00720788335046698\\
370.01	0.00720583235397099\\
371.01	0.00720373383486799\\
372.01	0.00720158665742248\\
373.01	0.00719938965607631\\
374.01	0.00719714163457338\\
375.01	0.00719484136507993\\
376.01	0.00719248758730472\\
377.01	0.00719007900762558\\
378.01	0.00718761429822667\\
379.01	0.00718509209624926\\
380.01	0.00718251100295767\\
381.01	0.00717986958291612\\
382.01	0.00717716636316688\\
383.01	0.0071743998323949\\
384.01	0.00717156844005421\\
385.01	0.00716867059543121\\
386.01	0.00716570466662249\\
387.01	0.00716266897943273\\
388.01	0.00715956181622413\\
389.01	0.00715638141473258\\
390.01	0.00715312596685178\\
391.01	0.0071497936173832\\
392.01	0.00714638246275295\\
393.01	0.00714289054969291\\
394.01	0.00713931587388703\\
395.01	0.00713565637857985\\
396.01	0.00713190995314948\\
397.01	0.00712807443164127\\
398.01	0.00712414759126293\\
399.01	0.00712012715084104\\
400.01	0.00711601076923545\\
401.01	0.00711179604371409\\
402.01	0.00710748050828523\\
403.01	0.00710306163198771\\
404.01	0.00709853681713742\\
405.01	0.00709390339753045\\
406.01	0.00708915863660106\\
407.01	0.00708429972553548\\
408.01	0.00707932378133915\\
409.01	0.00707422784485851\\
410.01	0.0070690088787558\\
411.01	0.00706366376543657\\
412.01	0.00705818930493069\\
413.01	0.00705258221272485\\
414.01	0.00704683911754739\\
415.01	0.00704095655910548\\
416.01	0.00703493098577409\\
417.01	0.0070287587522365\\
418.01	0.00702243611707767\\
419.01	0.0070159592403298\\
420.01	0.00700932418097047\\
421.01	0.00700252689437422\\
422.01	0.00699556322971849\\
423.01	0.00698842892734394\\
424.01	0.00698111961607107\\
425.01	0.00697363081047384\\
426.01	0.00696595790811191\\
427.01	0.00695809618672318\\
428.01	0.00695004080137968\\
429.01	0.00694178678160729\\
430.01	0.00693332902847468\\
431.01	0.0069246623116534\\
432.01	0.00691578126645409\\
433.01	0.00690668039084431\\
434.01	0.0068973540424542\\
435.01	0.00688779643557772\\
436.01	0.00687800163817969\\
437.01	0.00686796356892092\\
438.01	0.00685767599421733\\
439.01	0.00684713252535202\\
440.01	0.00683632661566616\\
441.01	0.00682525155786101\\
442.01	0.00681390048145097\\
443.01	0.00680226635042226\\
444.01	0.0067903419611648\\
445.01	0.00677811994076563\\
446.01	0.00676559274577735\\
447.01	0.00675275266160967\\
448.01	0.00673959180273363\\
449.01	0.0067261021139434\\
450.01	0.00671227537299304\\
451.01	0.00669810319501526\\
452.01	0.00668357703924631\\
453.01	0.00666868821873034\\
454.01	0.00665342791386457\\
455.01	0.00663778719088858\\
456.01	0.00662175702671926\\
457.01	0.00660532834191707\\
458.01	0.00658849204403995\\
459.01	0.00657123908423049\\
460.01	0.00655356053060433\\
461.01	0.00653544766288437\\
462.01	0.00651689209377258\\
463.01	0.00649788592375032\\
464.01	0.00647842193727471\\
465.01	0.00645849384938605\\
466.01	0.00643809661159538\\
467.01	0.00641722678149182\\
468.01	0.00639588294174248\\
469.01	0.00637406615110894\\
470.01	0.00635178065127839\\
471.01	0.00632903481355111\\
472.01	0.00630584226079985\\
473.01	0.00628222328826783\\
474.01	0.00625820667405755\\
475.01	0.00623383199391051\\
476.01	0.00620915259197159\\
477.01	0.00618423941211728\\
478.01	0.00615918597730179\\
479.01	0.00613411491311489\\
480.01	0.00610918623765697\\
481.01	0.00608439604413744\\
482.01	0.00605925887006073\\
483.01	0.00603379493791583\\
484.01	0.00600808471650222\\
485.01	0.00598223538922471\\
486.01	0.00595638857752813\\
487.01	0.00593073036687392\\
488.01	0.00590550435137352\\
489.01	0.0058810286444121\\
490.01	0.00585771811093029\\
491.01	0.00583492745131773\\
492.01	0.00581153101441279\\
493.01	0.00578751257433942\\
494.01	0.00576285618546294\\
495.01	0.00573754499475068\\
496.01	0.00571156110089086\\
497.01	0.00568488546783704\\
498.01	0.00565749796584674\\
499.01	0.00562937771148849\\
500.01	0.00560050357407001\\
501.01	0.00557085452658569\\
502.01	0.00554041010518881\\
503.01	0.00550915111070708\\
504.01	0.00547706066367205\\
505.01	0.00544412572887732\\
506.01	0.00541033797620297\\
507.01	0.0053756930186474\\
508.01	0.00534019071220625\\
509.01	0.00530383582615455\\
510.01	0.0052666386784107\\
511.01	0.00522861565502374\\
512.01	0.0051897894890629\\
513.01	0.00515018910921669\\
514.01	0.00510984877493633\\
515.01	0.00506880608133519\\
516.01	0.00502709822701115\\
517.01	0.00498475526112614\\
518.01	0.00494178415792848\\
519.01	0.00489816705544328\\
520.01	0.00485386328916163\\
521.01	0.00480879788007793\\
522.01	0.00476284469585961\\
523.01	0.00471580276867963\\
524.01	0.0046673739575744\\
525.01	0.00461732068128018\\
526.01	0.00456552036522299\\
527.01	0.00451190405559132\\
528.01	0.00445643718448044\\
529.01	0.00439912066840682\\
530.01	0.0043400101504628\\
531.01	0.00427924337089882\\
532.01	0.00421731809571334\\
533.01	0.00415501977916613\\
534.01	0.00409250425311458\\
535.01	0.00402992051392232\\
536.01	0.00396743628341949\\
537.01	0.00390523746308485\\
538.01	0.00384352588757257\\
539.01	0.0037825146019863\\
540.01	0.00372241963365504\\
541.01	0.00366344652277587\\
542.01	0.0036057689618982\\
543.01	0.00354949655435368\\
544.01	0.00349450365672057\\
545.01	0.00343991588066112\\
546.01	0.00338564234923884\\
547.01	0.00333174561849863\\
548.01	0.00327827103727041\\
549.01	0.0032252416502087\\
550.01	0.00317265336847283\\
551.01	0.00312046826653471\\
552.01	0.00306860796740712\\
553.01	0.00301695220746892\\
554.01	0.00296534440928589\\
555.01	0.00291361933827177\\
556.01	0.0028617036131168\\
557.01	0.00280956496612162\\
558.01	0.00275716146401953\\
559.01	0.00270444228312854\\
560.01	0.00265134944665799\\
561.01	0.002597819944586\\
562.01	0.00254378938571621\\
563.01	0.00248919751402807\\
564.01	0.00243399416831436\\
565.01	0.0023781426409719\\
566.01	0.00232161204526715\\
567.01	0.00226437095798489\\
568.01	0.00220638810077902\\
569.01	0.00214763356692056\\
570.01	0.00208808010357186\\
571.01	0.0020277041292811\\
572.01	0.00196648618517145\\
573.01	0.00190441059245309\\
574.01	0.00184146435238803\\
575.01	0.00177763626689189\\
576.01	0.00171291713759237\\
577.01	0.00164730022332831\\
578.01	0.00158078160588281\\
579.01	0.00151336043079601\\
580.01	0.00144503902502086\\
581.01	0.00137582293874799\\
582.01	0.00130572101420638\\
583.01	0.00123474560536223\\
584.01	0.00116291293976363\\
585.01	0.00109024350277839\\
586.01	0.00101676239875393\\
587.01	0.000942499671248528\\
588.01	0.00086749056337539\\
589.01	0.000791775698818989\\
590.01	0.000715401150971457\\
591.01	0.000638418341011752\\
592.01	0.000560883678765979\\
593.01	0.000482857857459612\\
594.01	0.000404404717382021\\
595.01	0.000325589587232982\\
596.01	0.000246477000692876\\
597.01	0.000167127667889054\\
598.01	9.13379822583427e-05\\
599.01	2.91271958431007e-05\\
599.02	2.86192783682145e-05\\
599.03	2.81144237470974e-05\\
599.04	2.76126618026333e-05\\
599.05	2.71140226516964e-05\\
599.06	2.6618536708066e-05\\
599.07	2.61262346853714e-05\\
599.08	2.56371476000505e-05\\
599.09	2.51513067743691e-05\\
599.1	2.46687438394382e-05\\
599.11	2.41894907382703e-05\\
599.12	2.37135797288915e-05\\
599.13	2.32410433874414e-05\\
599.14	2.27719146113341e-05\\
599.15	2.23062266224462e-05\\
599.16	2.18440129703245e-05\\
599.17	2.13853075354369e-05\\
599.18	2.09301445324543e-05\\
599.19	2.04785585135574e-05\\
599.2	2.00305843717826e-05\\
599.21	1.95862573444047e-05\\
599.22	1.9145613016339e-05\\
599.23	1.87086873236e-05\\
599.24	1.82755165567643e-05\\
599.25	1.78461373645019e-05\\
599.26	1.74205874172526e-05\\
599.27	1.69989072024649e-05\\
599.28	1.65811376108427e-05\\
599.29	1.61673199403477e-05\\
599.3	1.57574959002166e-05\\
599.31	1.53517076150536e-05\\
599.32	1.49499976289241e-05\\
599.33	1.45524089095253e-05\\
599.34	1.41589848523631e-05\\
599.35	1.37697692849988e-05\\
599.36	1.33848064713289e-05\\
599.37	1.30041411158955e-05\\
599.38	1.2627818368258e-05\\
599.39	1.22558838273942e-05\\
599.4	1.18883835461461e-05\\
599.41	1.15253640357253e-05\\
599.42	1.11668722702386e-05\\
599.43	1.08129556912751e-05\\
599.44	1.04636622125405e-05\\
599.45	1.01190402245176e-05\\
599.46	9.77913859919952e-06\\
599.47	9.44400669484861e-06\\
599.48	9.11369436081e-06\\
599.49	8.78825194237758e-06\\
599.5	8.46773028570155e-06\\
599.51	8.15218074274454e-06\\
599.52	7.84165517629144e-06\\
599.53	7.53620596500443e-06\\
599.54	7.23588600852476e-06\\
599.55	6.94074873264228e-06\\
599.56	6.65084809449087e-06\\
599.57	6.36623858781861e-06\\
599.58	6.08697524828207e-06\\
599.59	5.81311365883443e-06\\
599.6	5.54470995512042e-06\\
599.61	5.2818208309633e-06\\
599.62	5.02450354387951e-06\\
599.63	4.77281592066101e-06\\
599.64	4.52681636300793e-06\\
599.65	4.28656385322544e-06\\
599.66	4.05211795996216e-06\\
599.67	3.82353884401804e-06\\
599.68	3.60088726420252e-06\\
599.69	3.38422458325514e-06\\
599.7	3.17361277382341e-06\\
599.71	2.96911442449442e-06\\
599.72	2.77079274589413e-06\\
599.73	2.5787115768474e-06\\
599.74	2.3929353905848e-06\\
599.75	2.213529301031e-06\\
599.76	2.04055906913823e-06\\
599.77	1.87409110929959e-06\\
599.78	1.71419249580043e-06\\
599.79	1.56093096936177e-06\\
599.8	1.41437494372877e-06\\
599.81	1.27459351233379e-06\\
599.82	1.14165645502366e-06\\
599.83	1.01563424484592e-06\\
599.84	8.96598054920053e-07\\
599.85	7.84619765348618e-07\\
599.86	6.79771970232487e-07\\
599.87	5.82127984717282e-07\\
599.88	4.91761852147374e-07\\
599.89	4.08748351252847e-07\\
599.9	3.33163003437068e-07\\
599.91	2.65082080131915e-07\\
599.92	2.04582610201579e-07\\
599.93	1.51742387450443e-07\\
599.94	1.06639978189951e-07\\
599.95	6.93547288800611e-08\\
599.96	3.99667738487652e-08\\
599.97	1.85570430896731e-08\\
599.98	5.20727013418598e-09\\
599.99	0\\
600	0\\
};
\addplot [color=blue!75!mycolor7,solid,forget plot]
  table[row sep=crcr]{%
0.01	0.00902214729946002\\
1.01	0.00902214636557786\\
2.01	0.0090221454118588\\
3.01	0.00902214443787868\\
4.01	0.00902214344320444\\
5.01	0.00902214242739349\\
6.01	0.00902214138999379\\
7.01	0.00902214033054357\\
8.01	0.00902213924857109\\
9.01	0.00902213814359446\\
10.01	0.00902213701512133\\
11.01	0.00902213586264874\\
12.01	0.00902213468566294\\
13.01	0.00902213348363904\\
14.01	0.00902213225604088\\
15.01	0.00902213100232068\\
16.01	0.0090221297219188\\
17.01	0.00902212841426364\\
18.01	0.0090221270787711\\
19.01	0.00902212571484459\\
20.01	0.0090221243218745\\
21.01	0.00902212289923817\\
22.01	0.00902212144629944\\
23.01	0.00902211996240837\\
24.01	0.00902211844690088\\
25.01	0.00902211689909873\\
26.01	0.0090221153183088\\
27.01	0.00902211370382313\\
28.01	0.00902211205491836\\
29.01	0.00902211037085547\\
30.01	0.00902210865087943\\
31.01	0.00902210689421885\\
32.01	0.00902210510008571\\
33.01	0.00902210326767487\\
34.01	0.0090221013961636\\
35.01	0.00902209948471151\\
36.01	0.00902209753245992\\
37.01	0.00902209553853158\\
38.01	0.00902209350203016\\
39.01	0.00902209142203981\\
40.01	0.00902208929762494\\
41.01	0.00902208712782956\\
42.01	0.00902208491167694\\
43.01	0.00902208264816917\\
44.01	0.00902208033628665\\
45.01	0.00902207797498759\\
46.01	0.00902207556320759\\
47.01	0.00902207309985908\\
48.01	0.00902207058383085\\
49.01	0.00902206801398751\\
50.01	0.00902206538916903\\
51.01	0.00902206270819004\\
52.01	0.00902205996983952\\
53.01	0.00902205717287987\\
54.01	0.00902205431604674\\
55.01	0.00902205139804815\\
56.01	0.00902204841756396\\
57.01	0.00902204537324524\\
58.01	0.00902204226371373\\
59.01	0.00902203908756107\\
60.01	0.00902203584334826\\
61.01	0.00902203252960478\\
62.01	0.00902202914482812\\
63.01	0.0090220256874829\\
64.01	0.00902202215600024\\
65.01	0.00902201854877709\\
66.01	0.00902201486417526\\
67.01	0.00902201110052087\\
68.01	0.00902200725610342\\
69.01	0.00902200332917496\\
70.01	0.00902199931794933\\
71.01	0.00902199522060141\\
72.01	0.00902199103526603\\
73.01	0.00902198676003726\\
74.01	0.00902198239296747\\
75.01	0.00902197793206631\\
76.01	0.00902197337529987\\
77.01	0.00902196872058973\\
78.01	0.00902196396581186\\
79.01	0.00902195910879569\\
80.01	0.00902195414732309\\
81.01	0.00902194907912724\\
82.01	0.00902194390189153\\
83.01	0.00902193861324868\\
84.01	0.00902193321077929\\
85.01	0.00902192769201089\\
86.01	0.00902192205441676\\
87.01	0.00902191629541448\\
88.01	0.0090219104123652\\
89.01	0.00902190440257177\\
90.01	0.00902189826327783\\
91.01	0.00902189199166643\\
92.01	0.00902188558485854\\
93.01	0.00902187903991186\\
94.01	0.00902187235381927\\
95.01	0.00902186552350747\\
96.01	0.00902185854583548\\
97.01	0.00902185141759291\\
98.01	0.00902184413549887\\
99.01	0.00902183669620003\\
100.01	0.00902182909626898\\
101.01	0.0090218213322029\\
102.01	0.00902181340042147\\
103.01	0.00902180529726546\\
104.01	0.0090217970189948\\
105.01	0.00902178856178685\\
106.01	0.00902177992173439\\
107.01	0.00902177109484386\\
108.01	0.00902176207703357\\
109.01	0.00902175286413132\\
110.01	0.00902174345187275\\
111.01	0.00902173383589908\\
112.01	0.00902172401175507\\
113.01	0.00902171397488669\\
114.01	0.0090217037206392\\
115.01	0.00902169324425448\\
116.01	0.00902168254086924\\
117.01	0.00902167160551206\\
118.01	0.00902166043310138\\
119.01	0.00902164901844296\\
120.01	0.00902163735622711\\
121.01	0.00902162544102633\\
122.01	0.00902161326729258\\
123.01	0.00902160082935447\\
124.01	0.00902158812141463\\
125.01	0.00902157513754676\\
126.01	0.00902156187169281\\
127.01	0.00902154831765991\\
128.01	0.00902153446911727\\
129.01	0.0090215203195933\\
130.01	0.00902150586247225\\
131.01	0.00902149109099095\\
132.01	0.00902147599823548\\
133.01	0.00902146057713779\\
134.01	0.00902144482047227\\
135.01	0.00902142872085208\\
136.01	0.00902141227072544\\
137.01	0.00902139546237215\\
138.01	0.0090213782878994\\
139.01	0.00902136073923823\\
140.01	0.00902134280813927\\
141.01	0.00902132448616875\\
142.01	0.00902130576470425\\
143.01	0.00902128663493054\\
144.01	0.00902126708783508\\
145.01	0.00902124711420357\\
146.01	0.00902122670461531\\
147.01	0.00902120584943863\\
148.01	0.00902118453882594\\
149.01	0.0090211627627089\\
150.01	0.00902114051079341\\
151.01	0.00902111777255441\\
152.01	0.00902109453723066\\
153.01	0.0090210707938193\\
154.01	0.00902104653107041\\
155.01	0.0090210217374813\\
156.01	0.00902099640129076\\
157.01	0.00902097051047316\\
158.01	0.00902094405273254\\
159.01	0.0090209170154961\\
160.01	0.00902088938590829\\
161.01	0.0090208611508241\\
162.01	0.00902083229680238\\
163.01	0.00902080281009923\\
164.01	0.00902077267666104\\
165.01	0.00902074188211725\\
166.01	0.00902071041177333\\
167.01	0.00902067825060315\\
168.01	0.00902064538324151\\
169.01	0.00902061179397631\\
170.01	0.00902057746674059\\
171.01	0.00902054238510434\\
172.01	0.00902050653226642\\
173.01	0.00902046989104571\\
174.01	0.00902043244387255\\
175.01	0.00902039417277981\\
176.01	0.0090203550593938\\
177.01	0.00902031508492479\\
178.01	0.00902027423015762\\
179.01	0.00902023247544173\\
180.01	0.0090201898006814\\
181.01	0.00902014618532521\\
182.01	0.00902010160835585\\
183.01	0.00902005604827914\\
184.01	0.00902000948311318\\
185.01	0.00901996189037716\\
186.01	0.00901991324707981\\
187.01	0.00901986352970765\\
188.01	0.00901981271421294\\
189.01	0.00901976077600127\\
190.01	0.00901970768991928\\
191.01	0.00901965343024152\\
192.01	0.00901959797065711\\
193.01	0.00901954128425669\\
194.01	0.00901948334351816\\
195.01	0.00901942412029288\\
196.01	0.00901936358579104\\
197.01	0.0090193017105668\\
198.01	0.00901923846450342\\
199.01	0.00901917381679741\\
200.01	0.00901910773594297\\
201.01	0.00901904018971561\\
202.01	0.00901897114515559\\
203.01	0.00901890056855102\\
204.01	0.00901882842542045\\
205.01	0.00901875468049491\\
206.01	0.00901867929769993\\
207.01	0.00901860224013693\\
208.01	0.00901852347006404\\
209.01	0.00901844294887678\\
210.01	0.00901836063708794\\
211.01	0.00901827649430733\\
212.01	0.00901819047922082\\
213.01	0.00901810254956911\\
214.01	0.00901801266212574\\
215.01	0.00901792077267485\\
216.01	0.00901782683598827\\
217.01	0.00901773080580212\\
218.01	0.00901763263479289\\
219.01	0.00901753227455295\\
220.01	0.00901742967556563\\
221.01	0.00901732478717937\\
222.01	0.00901721755758161\\
223.01	0.00901710793377213\\
224.01	0.0090169958615353\\
225.01	0.00901688128541232\\
226.01	0.00901676414867232\\
227.01	0.0090166443932831\\
228.01	0.00901652195988103\\
229.01	0.00901639678774042\\
230.01	0.00901626881474201\\
231.01	0.00901613797734075\\
232.01	0.00901600421053311\\
233.01	0.00901586744782334\\
234.01	0.00901572762118908\\
235.01	0.00901558466104616\\
236.01	0.00901543849621265\\
237.01	0.00901528905387199\\
238.01	0.00901513625953546\\
239.01	0.0090149800370036\\
240.01	0.00901482030832685\\
241.01	0.00901465699376532\\
242.01	0.00901449001174758\\
243.01	0.00901431927882856\\
244.01	0.00901414470964644\\
245.01	0.00901396621687866\\
246.01	0.00901378371119698\\
247.01	0.00901359710122127\\
248.01	0.00901340629347245\\
249.01	0.00901321119232442\\
250.01	0.00901301169995473\\
251.01	0.00901280771629435\\
252.01	0.00901259913897617\\
253.01	0.00901238586328219\\
254.01	0.00901216778208983\\
255.01	0.00901194478581698\\
256.01	0.00901171676236574\\
257.01	0.00901148359706464\\
258.01	0.00901124517261017\\
259.01	0.00901100136900639\\
260.01	0.00901075206350364\\
261.01	0.00901049713053553\\
262.01	0.00901023644165471\\
263.01	0.00900996986546737\\
264.01	0.00900969726756579\\
265.01	0.00900941851045983\\
266.01	0.00900913345350662\\
267.01	0.00900884195283892\\
268.01	0.00900854386129167\\
269.01	0.00900823902832682\\
270.01	0.00900792729995694\\
271.01	0.00900760851866644\\
272.01	0.00900728252333164\\
273.01	0.00900694914913861\\
274.01	0.00900660822749957\\
275.01	0.0090062595859669\\
276.01	0.00900590304814578\\
277.01	0.00900553843360438\\
278.01	0.0090051655577822\\
279.01	0.00900478423189671\\
280.01	0.00900439426284703\\
281.01	0.00900399545311629\\
282.01	0.00900358760067128\\
283.01	0.00900317049885994\\
284.01	0.00900274393630672\\
285.01	0.00900230769680487\\
286.01	0.00900186155920724\\
287.01	0.00900140529731369\\
288.01	0.00900093867975641\\
289.01	0.00900046146988245\\
290.01	0.00899997342563339\\
291.01	0.00899947429942243\\
292.01	0.00899896383800842\\
293.01	0.00899844178236686\\
294.01	0.00899790786755827\\
295.01	0.00899736182259278\\
296.01	0.00899680337029201\\
297.01	0.00899623222714744\\
298.01	0.00899564810317535\\
299.01	0.00899505070176827\\
300.01	0.0089944397195428\\
301.01	0.00899381484618379\\
302.01	0.00899317576428434\\
303.01	0.00899252214918232\\
304.01	0.00899185366879247\\
305.01	0.00899116998343422\\
306.01	0.0089904707456555\\
307.01	0.00898975560005174\\
308.01	0.00898902418308034\\
309.01	0.00898827612287043\\
310.01	0.00898751103902754\\
311.01	0.00898672854243331\\
312.01	0.00898592823503992\\
313.01	0.0089851097096588\\
314.01	0.00898427254974418\\
315.01	0.00898341632917067\\
316.01	0.00898254061200452\\
317.01	0.0089816449522693\\
318.01	0.00898072889370453\\
319.01	0.00897979196951827\\
320.01	0.00897883370213262\\
321.01	0.00897785360292217\\
322.01	0.00897685117194539\\
323.01	0.00897582589766829\\
324.01	0.00897477725668059\\
325.01	0.00897370471340389\\
326.01	0.00897260771979137\\
327.01	0.0089714857150195\\
328.01	0.00897033812517077\\
329.01	0.00896916436290742\\
330.01	0.00896796382713621\\
331.01	0.00896673590266359\\
332.01	0.00896547995984101\\
333.01	0.00896419535420055\\
334.01	0.00896288142608007\\
335.01	0.00896153750023806\\
336.01	0.00896016288545745\\
337.01	0.00895875687413877\\
338.01	0.00895731874188163\\
339.01	0.00895584774705512\\
340.01	0.00895434313035604\\
341.01	0.00895280411435532\\
342.01	0.00895122990303212\\
343.01	0.00894961968129542\\
344.01	0.00894797261449294\\
345.01	0.00894628784790683\\
346.01	0.00894456450623659\\
347.01	0.00894280169306797\\
348.01	0.00894099849032874\\
349.01	0.00893915395773019\\
350.01	0.00893726713219465\\
351.01	0.00893533702726874\\
352.01	0.0089333626325217\\
353.01	0.00893134291292928\\
354.01	0.00892927680824225\\
355.01	0.00892716323233978\\
356.01	0.00892500107256717\\
357.01	0.00892278918905757\\
358.01	0.00892052641403791\\
359.01	0.00891821155111816\\
360.01	0.00891584337456376\\
361.01	0.00891342062855118\\
362.01	0.00891094202640624\\
363.01	0.00890840624982432\\
364.01	0.0089058119480727\\
365.01	0.00890315773717442\\
366.01	0.00890044219907318\\
367.01	0.00889766388077904\\
368.01	0.00889482129349435\\
369.01	0.00889191291171938\\
370.01	0.00888893717233722\\
371.01	0.00888589247367732\\
372.01	0.008882777174557\\
373.01	0.00887958959330008\\
374.01	0.00887632800673178\\
375.01	0.0088729906491486\\
376.01	0.00886957571126188\\
377.01	0.00886608133911376\\
378.01	0.00886250563296292\\
379.01	0.00885884664613867\\
380.01	0.00885510238386041\\
381.01	0.00885127080202006\\
382.01	0.00884734980592478\\
383.01	0.00884333724899739\\
384.01	0.00883923093143278\\
385.01	0.00883502859880898\\
386.01	0.00883072794065253\\
387.01	0.0088263265889584\\
388.01	0.00882182211666361\\
389.01	0.00881721203607317\\
390.01	0.00881249379723585\\
391.01	0.00880766478626766\\
392.01	0.00880272232362082\\
393.01	0.00879766366229554\\
394.01	0.00879248598599183\\
395.01	0.0087871864071987\\
396.01	0.00878176196521772\\
397.01	0.00877620962411707\\
398.01	0.00877052627061375\\
399.01	0.00876470871187895\\
400.01	0.00875875367326385\\
401.01	0.00875265779594081\\
402.01	0.00874641763445592\\
403.01	0.0087400296541879\\
404.01	0.0087334902287085\\
405.01	0.00872679563703865\\
406.01	0.00871994206079482\\
407.01	0.00871292558121942\\
408.01	0.00870574217608834\\
409.01	0.00869838771648888\\
410.01	0.00869085796346007\\
411.01	0.00868314856448815\\
412.01	0.00867525504984742\\
413.01	0.00866717282877817\\
414.01	0.00865889718549119\\
415.01	0.00865042327498873\\
416.01	0.00864174611869009\\
417.01	0.00863286059984992\\
418.01	0.00862376145875623\\
419.01	0.00861444328769367\\
420.01	0.00860490052565739\\
421.01	0.00859512745280099\\
422.01	0.00858511818460098\\
423.01	0.00857486666571896\\
424.01	0.00856436666354078\\
425.01	0.00855361176137078\\
426.01	0.00854259535125666\\
427.01	0.00853131062641902\\
428.01	0.0085197505732563\\
429.01	0.00850790796289468\\
430.01	0.00849577534224728\\
431.01	0.00848334502454542\\
432.01	0.00847060907929983\\
433.01	0.00845755932164421\\
434.01	0.00844418730100998\\
435.01	0.00843048428907305\\
436.01	0.00841644126690722\\
437.01	0.00840204891126913\\
438.01	0.00838729757993086\\
439.01	0.00837217729596297\\
440.01	0.00835667773085732\\
441.01	0.00834078818636127\\
442.01	0.00832449757487602\\
443.01	0.00830779439824565\\
444.01	0.00829066672473585\\
445.01	0.0082731021639658\\
446.01	0.00825508783951468\\
447.01	0.00823661035887356\\
448.01	0.00821765578035209\\
449.01	0.00819820957647621\\
450.01	0.00817825659332246\\
451.01	0.00815778100512831\\
452.01	0.0081367662633852\\
453.01	0.00811519503946528\\
454.01	0.00809304915964056\\
455.01	0.0080703095311234\\
456.01	0.00804695605748009\\
457.01	0.00802296754143559\\
458.01	0.0079983215726893\\
459.01	0.00797299439789086\\
460.01	0.00794696076938257\\
461.01	0.00792019376872971\\
462.01	0.00789266460054664\\
463.01	0.00786434235207658\\
464.01	0.00783519371563762\\
465.01	0.00780518267839087\\
466.01	0.00777427020982304\\
467.01	0.0077424140646549\\
468.01	0.0077095690915114\\
469.01	0.00767568632242102\\
470.01	0.00764070929011195\\
471.01	0.00760457477695843\\
472.01	0.00756721198221674\\
473.01	0.00752854110038599\\
474.01	0.00748847156388789\\
475.01	0.00744689985044509\\
476.01	0.00740370670259724\\
477.01	0.00735875348263947\\
478.01	0.00731187703838316\\
479.01	0.0072628837925216\\
480.01	0.00721154725105092\\
481.01	0.00718190514624087\\
482.01	0.00716230954357111\\
483.01	0.00714175761399098\\
484.01	0.00712013013151004\\
485.01	0.00709728208902563\\
486.01	0.0070730359896628\\
487.01	0.00704717323628954\\
488.01	0.00701942304826762\\
489.01	0.0069894481553389\\
490.01	0.00695682628098655\\
491.01	0.00692220285691665\\
492.01	0.00688667793783651\\
493.01	0.00685023293560006\\
494.01	0.00681284823645732\\
495.01	0.00677450416666226\\
496.01	0.00673518085100826\\
497.01	0.00669485795942825\\
498.01	0.00665351428663556\\
499.01	0.00661112708481027\\
500.01	0.0065676710391065\\
501.01	0.00652311675181007\\
502.01	0.00647742855102759\\
503.01	0.00643056136511647\\
504.01	0.00638245630966505\\
505.01	0.00633303927685509\\
506.01	0.00628226434297923\\
507.01	0.00623011202621469\\
508.01	0.00617656872138363\\
509.01	0.00612162812908867\\
510.01	0.00606529346721341\\
511.01	0.00600758029255231\\
512.01	0.00594852011353755\\
513.01	0.00588816503327066\\
514.01	0.0058265937402637\\
515.01	0.00576391926959179\\
516.01	0.00570029909933447\\
517.01	0.00563594834130146\\
518.01	0.00557115711633265\\
519.01	0.00550631321884673\\
520.01	0.00544193099400693\\
521.01	0.00537868871517858\\
522.01	0.0053174772958715\\
523.01	0.00525946392137587\\
524.01	0.00520531260447456\\
525.01	0.00515224219814917\\
526.01	0.00509900084438186\\
527.01	0.00504430539160467\\
528.01	0.004988157377622\\
529.01	0.00493056553609459\\
530.01	0.00487154406794097\\
531.01	0.00481110992258657\\
532.01	0.00474927598110803\\
533.01	0.00468603582532421\\
534.01	0.00462137786818744\\
535.01	0.00455529471347772\\
536.01	0.00448778347953893\\
537.01	0.00441884646204981\\
538.01	0.00434849204940306\\
539.01	0.00427673591926868\\
540.01	0.004203602664043\\
541.01	0.00412912804179831\\
542.01	0.00405336103587868\\
543.01	0.003976359182936\\
544.01	0.00389830370161603\\
545.01	0.0038200553540577\\
546.01	0.00374193698585412\\
547.01	0.0036641755998312\\
548.01	0.00358700119398903\\
549.01	0.00351062917203564\\
550.01	0.00343526607327019\\
551.01	0.00336120054192128\\
552.01	0.00328872981598254\\
553.01	0.00321810286178094\\
554.01	0.00314946895573235\\
555.01	0.00308243132699263\\
556.01	0.00301600563065513\\
557.01	0.00295019309929753\\
558.01	0.0028850146013036\\
559.01	0.00282046134724498\\
560.01	0.00275650087514034\\
561.01	0.00269307732904906\\
562.01	0.00263008779722273\\
563.01	0.00256737583515977\\
564.01	0.00250474150326022\\
565.01	0.00244200987170124\\
566.01	0.00237912602438769\\
567.01	0.00231604585404112\\
568.01	0.00225271645253628\\
569.01	0.00218907572873699\\
570.01	0.00212505471491989\\
571.01	0.00206058414873081\\
572.01	0.00199560260247596\\
573.01	0.00193006437600999\\
574.01	0.00186394204314159\\
575.01	0.00179721456938685\\
576.01	0.00172986002512933\\
577.01	0.00166185690585053\\
578.01	0.0015931860350652\\
579.01	0.00152383235677303\\
580.01	0.00145378623821854\\
581.01	0.00138304383207852\\
582.01	0.00131160612454736\\
583.01	0.00123947717929371\\
584.01	0.00116666351012589\\
585.01	0.0010931744009037\\
586.01	0.00101902232302626\\
587.01	0.000944223407204438\\
588.01	0.000868797861908435\\
589.01	0.000792770299867063\\
590.01	0.000716170051933755\\
591.01	0.000639031578739221\\
592.01	0.000561394895303546\\
593.01	0.000483305700085252\\
594.01	0.000404814972475517\\
595.01	0.000325977840673588\\
596.01	0.000246851522619258\\
597.01	0.000167492162777875\\
598.01	9.13380205773313e-05\\
599.01	2.91271962973242e-05\\
599.02	2.86192787954821e-05\\
599.03	2.81144241487431e-05\\
599.04	2.76126621799409e-05\\
599.05	2.7114023005901e-05\\
599.06	2.6618537040354e-05\\
599.07	2.61262349968722e-05\\
599.08	2.5637147891857e-05\\
599.09	2.51513070475221e-05\\
599.1	2.466874409493e-05\\
599.11	2.41894909770619e-05\\
599.12	2.37135799518919e-05\\
599.13	2.32410435955233e-05\\
599.14	2.27719148053352e-05\\
599.15	2.2306226803161e-05\\
599.16	2.18440131385146e-05\\
599.17	2.13853076918291e-05\\
599.18	2.09301446777409e-05\\
599.19	2.04785586483958e-05\\
599.2	2.00305844968024e-05\\
599.21	1.95862574602045e-05\\
599.22	1.91456131234894e-05\\
599.23	1.87086874226389e-05\\
599.24	1.82755166482085e-05\\
599.25	1.78461374488355e-05\\
599.26	1.74205874949405e-05\\
599.27	1.69989072739424e-05\\
599.28	1.65811376765281e-05\\
599.29	1.61673200006276e-05\\
599.3	1.57574959554676e-05\\
599.31	1.53517076656243e-05\\
599.32	1.49499976751493e-05\\
599.33	1.4552408951712e-05\\
599.34	1.4158984890808e-05\\
599.35	1.3769769319983e-05\\
599.36	1.33848065031108e-05\\
599.37	1.30041411447214e-05\\
599.38	1.26278183943569e-05\\
599.39	1.22558838509795e-05\\
599.4	1.18883835674242e-05\\
599.41	1.15253640548835e-05\\
599.42	1.1166872287454e-05\\
599.43	1.08129557067158e-05\\
599.44	1.04636622263576e-05\\
599.45	1.01190402368567e-05\\
599.46	9.77913861019246e-06\\
599.47	9.44400670461684e-06\\
599.48	9.11369436946974e-06\\
599.49	8.78825195003465e-06\\
599.5	8.4677302924531e-06\\
599.51	8.1521807486825e-06\\
599.52	7.84165518149561e-06\\
599.53	7.53620596955114e-06\\
599.54	7.23588601248687e-06\\
599.55	6.94074873608223e-06\\
599.56	6.65084809746766e-06\\
599.57	6.3662385903808e-06\\
599.58	6.0869752504817e-06\\
599.59	5.81311366071487e-06\\
599.6	5.54470995672157e-06\\
599.61	5.28182083231986e-06\\
599.62	5.02450354502269e-06\\
599.63	4.77281592161857e-06\\
599.64	4.52681636380764e-06\\
599.65	4.28656385388811e-06\\
599.66	4.0521179605086e-06\\
599.67	3.8235388444656e-06\\
599.68	3.60088726456508e-06\\
599.69	3.38422458354831e-06\\
599.7	3.1736127740576e-06\\
599.71	2.96911442468004e-06\\
599.72	2.77079274604158e-06\\
599.73	2.57871157696016e-06\\
599.74	2.39293539067154e-06\\
599.75	2.21352930109692e-06\\
599.76	2.04055906919028e-06\\
599.77	1.87409110933776e-06\\
599.78	1.71419249582645e-06\\
599.79	1.56093096938086e-06\\
599.8	1.41437494374265e-06\\
599.81	1.2745935123442e-06\\
599.82	1.1416564550306e-06\\
599.83	1.01563424485286e-06\\
599.84	8.96598054921788e-07\\
599.85	7.84619765352088e-07\\
599.86	6.79771970232487e-07\\
599.87	5.82127984719016e-07\\
599.88	4.91761852147374e-07\\
599.89	4.08748351251112e-07\\
599.9	3.33163003438802e-07\\
599.91	2.65082080131915e-07\\
599.92	2.04582610201579e-07\\
599.93	1.51742387452178e-07\\
599.94	1.06639978191686e-07\\
599.95	6.93547288817958e-08\\
599.96	3.99667738487652e-08\\
599.97	1.85570430914078e-08\\
599.98	5.20727013418598e-09\\
599.99	0\\
600	0\\
};
\addplot [color=blue!80!mycolor9,solid,forget plot]
  table[row sep=crcr]{%
0.01	0.01\\
1.01	0.01\\
2.01	0.01\\
3.01	0.01\\
4.01	0.01\\
5.01	0.01\\
6.01	0.01\\
7.01	0.01\\
8.01	0.01\\
9.01	0.01\\
10.01	0.01\\
11.01	0.01\\
12.01	0.01\\
13.01	0.01\\
14.01	0.01\\
15.01	0.01\\
16.01	0.01\\
17.01	0.01\\
18.01	0.01\\
19.01	0.01\\
20.01	0.01\\
21.01	0.01\\
22.01	0.01\\
23.01	0.01\\
24.01	0.01\\
25.01	0.01\\
26.01	0.01\\
27.01	0.01\\
28.01	0.01\\
29.01	0.01\\
30.01	0.01\\
31.01	0.01\\
32.01	0.01\\
33.01	0.01\\
34.01	0.01\\
35.01	0.01\\
36.01	0.01\\
37.01	0.01\\
38.01	0.01\\
39.01	0.01\\
40.01	0.01\\
41.01	0.01\\
42.01	0.01\\
43.01	0.01\\
44.01	0.01\\
45.01	0.01\\
46.01	0.01\\
47.01	0.01\\
48.01	0.01\\
49.01	0.01\\
50.01	0.01\\
51.01	0.01\\
52.01	0.01\\
53.01	0.01\\
54.01	0.01\\
55.01	0.01\\
56.01	0.01\\
57.01	0.01\\
58.01	0.01\\
59.01	0.01\\
60.01	0.01\\
61.01	0.01\\
62.01	0.01\\
63.01	0.01\\
64.01	0.01\\
65.01	0.01\\
66.01	0.01\\
67.01	0.01\\
68.01	0.01\\
69.01	0.01\\
70.01	0.01\\
71.01	0.01\\
72.01	0.01\\
73.01	0.01\\
74.01	0.01\\
75.01	0.01\\
76.01	0.01\\
77.01	0.01\\
78.01	0.01\\
79.01	0.01\\
80.01	0.01\\
81.01	0.01\\
82.01	0.01\\
83.01	0.01\\
84.01	0.01\\
85.01	0.01\\
86.01	0.01\\
87.01	0.01\\
88.01	0.01\\
89.01	0.01\\
90.01	0.01\\
91.01	0.01\\
92.01	0.01\\
93.01	0.01\\
94.01	0.01\\
95.01	0.01\\
96.01	0.01\\
97.01	0.01\\
98.01	0.01\\
99.01	0.01\\
100.01	0.01\\
101.01	0.01\\
102.01	0.01\\
103.01	0.01\\
104.01	0.01\\
105.01	0.01\\
106.01	0.01\\
107.01	0.01\\
108.01	0.01\\
109.01	0.01\\
110.01	0.01\\
111.01	0.01\\
112.01	0.01\\
113.01	0.01\\
114.01	0.01\\
115.01	0.01\\
116.01	0.01\\
117.01	0.01\\
118.01	0.01\\
119.01	0.01\\
120.01	0.01\\
121.01	0.01\\
122.01	0.01\\
123.01	0.01\\
124.01	0.01\\
125.01	0.01\\
126.01	0.01\\
127.01	0.01\\
128.01	0.01\\
129.01	0.01\\
130.01	0.01\\
131.01	0.01\\
132.01	0.01\\
133.01	0.01\\
134.01	0.01\\
135.01	0.01\\
136.01	0.01\\
137.01	0.01\\
138.01	0.01\\
139.01	0.01\\
140.01	0.01\\
141.01	0.01\\
142.01	0.01\\
143.01	0.01\\
144.01	0.01\\
145.01	0.01\\
146.01	0.01\\
147.01	0.01\\
148.01	0.01\\
149.01	0.01\\
150.01	0.01\\
151.01	0.01\\
152.01	0.01\\
153.01	0.01\\
154.01	0.01\\
155.01	0.01\\
156.01	0.01\\
157.01	0.01\\
158.01	0.01\\
159.01	0.01\\
160.01	0.01\\
161.01	0.01\\
162.01	0.01\\
163.01	0.01\\
164.01	0.01\\
165.01	0.01\\
166.01	0.01\\
167.01	0.01\\
168.01	0.01\\
169.01	0.01\\
170.01	0.01\\
171.01	0.01\\
172.01	0.01\\
173.01	0.01\\
174.01	0.01\\
175.01	0.01\\
176.01	0.01\\
177.01	0.01\\
178.01	0.01\\
179.01	0.01\\
180.01	0.01\\
181.01	0.01\\
182.01	0.01\\
183.01	0.01\\
184.01	0.01\\
185.01	0.01\\
186.01	0.01\\
187.01	0.01\\
188.01	0.01\\
189.01	0.01\\
190.01	0.01\\
191.01	0.01\\
192.01	0.01\\
193.01	0.01\\
194.01	0.01\\
195.01	0.01\\
196.01	0.01\\
197.01	0.01\\
198.01	0.01\\
199.01	0.01\\
200.01	0.01\\
201.01	0.01\\
202.01	0.01\\
203.01	0.01\\
204.01	0.01\\
205.01	0.01\\
206.01	0.01\\
207.01	0.01\\
208.01	0.01\\
209.01	0.01\\
210.01	0.01\\
211.01	0.01\\
212.01	0.01\\
213.01	0.01\\
214.01	0.01\\
215.01	0.01\\
216.01	0.01\\
217.01	0.01\\
218.01	0.01\\
219.01	0.01\\
220.01	0.01\\
221.01	0.01\\
222.01	0.01\\
223.01	0.01\\
224.01	0.01\\
225.01	0.01\\
226.01	0.01\\
227.01	0.01\\
228.01	0.01\\
229.01	0.01\\
230.01	0.01\\
231.01	0.01\\
232.01	0.01\\
233.01	0.01\\
234.01	0.01\\
235.01	0.01\\
236.01	0.01\\
237.01	0.01\\
238.01	0.01\\
239.01	0.01\\
240.01	0.01\\
241.01	0.01\\
242.01	0.01\\
243.01	0.01\\
244.01	0.01\\
245.01	0.01\\
246.01	0.01\\
247.01	0.01\\
248.01	0.01\\
249.01	0.01\\
250.01	0.01\\
251.01	0.01\\
252.01	0.01\\
253.01	0.01\\
254.01	0.01\\
255.01	0.01\\
256.01	0.01\\
257.01	0.01\\
258.01	0.01\\
259.01	0.01\\
260.01	0.01\\
261.01	0.01\\
262.01	0.01\\
263.01	0.01\\
264.01	0.01\\
265.01	0.01\\
266.01	0.01\\
267.01	0.01\\
268.01	0.01\\
269.01	0.01\\
270.01	0.01\\
271.01	0.01\\
272.01	0.01\\
273.01	0.01\\
274.01	0.01\\
275.01	0.01\\
276.01	0.01\\
277.01	0.01\\
278.01	0.01\\
279.01	0.01\\
280.01	0.01\\
281.01	0.01\\
282.01	0.01\\
283.01	0.01\\
284.01	0.01\\
285.01	0.01\\
286.01	0.01\\
287.01	0.01\\
288.01	0.01\\
289.01	0.01\\
290.01	0.01\\
291.01	0.01\\
292.01	0.01\\
293.01	0.01\\
294.01	0.01\\
295.01	0.01\\
296.01	0.01\\
297.01	0.01\\
298.01	0.01\\
299.01	0.01\\
300.01	0.01\\
301.01	0.01\\
302.01	0.01\\
303.01	0.01\\
304.01	0.01\\
305.01	0.01\\
306.01	0.01\\
307.01	0.01\\
308.01	0.01\\
309.01	0.01\\
310.01	0.01\\
311.01	0.01\\
312.01	0.01\\
313.01	0.01\\
314.01	0.01\\
315.01	0.01\\
316.01	0.01\\
317.01	0.01\\
318.01	0.01\\
319.01	0.01\\
320.01	0.01\\
321.01	0.01\\
322.01	0.01\\
323.01	0.01\\
324.01	0.01\\
325.01	0.01\\
326.01	0.01\\
327.01	0.01\\
328.01	0.01\\
329.01	0.01\\
330.01	0.01\\
331.01	0.01\\
332.01	0.01\\
333.01	0.01\\
334.01	0.01\\
335.01	0.01\\
336.01	0.01\\
337.01	0.01\\
338.01	0.01\\
339.01	0.01\\
340.01	0.01\\
341.01	0.01\\
342.01	0.01\\
343.01	0.01\\
344.01	0.01\\
345.01	0.01\\
346.01	0.01\\
347.01	0.01\\
348.01	0.01\\
349.01	0.01\\
350.01	0.01\\
351.01	0.01\\
352.01	0.01\\
353.01	0.01\\
354.01	0.01\\
355.01	0.01\\
356.01	0.01\\
357.01	0.01\\
358.01	0.01\\
359.01	0.01\\
360.01	0.01\\
361.01	0.01\\
362.01	0.01\\
363.01	0.01\\
364.01	0.01\\
365.01	0.01\\
366.01	0.01\\
367.01	0.01\\
368.01	0.01\\
369.01	0.01\\
370.01	0.01\\
371.01	0.01\\
372.01	0.01\\
373.01	0.01\\
374.01	0.01\\
375.01	0.01\\
376.01	0.01\\
377.01	0.01\\
378.01	0.01\\
379.01	0.01\\
380.01	0.01\\
381.01	0.01\\
382.01	0.01\\
383.01	0.01\\
384.01	0.01\\
385.01	0.01\\
386.01	0.01\\
387.01	0.01\\
388.01	0.01\\
389.01	0.01\\
390.01	0.01\\
391.01	0.01\\
392.01	0.01\\
393.01	0.01\\
394.01	0.01\\
395.01	0.01\\
396.01	0.01\\
397.01	0.01\\
398.01	0.01\\
399.01	0.01\\
400.01	0.01\\
401.01	0.01\\
402.01	0.01\\
403.01	0.01\\
404.01	0.01\\
405.01	0.01\\
406.01	0.01\\
407.01	0.01\\
408.01	0.01\\
409.01	0.01\\
410.01	0.01\\
411.01	0.01\\
412.01	0.01\\
413.01	0.01\\
414.01	0.01\\
415.01	0.01\\
416.01	0.01\\
417.01	0.01\\
418.01	0.01\\
419.01	0.01\\
420.01	0.01\\
421.01	0.01\\
422.01	0.01\\
423.01	0.01\\
424.01	0.01\\
425.01	0.01\\
426.01	0.01\\
427.01	0.01\\
428.01	0.01\\
429.01	0.01\\
430.01	0.01\\
431.01	0.01\\
432.01	0.01\\
433.01	0.01\\
434.01	0.01\\
435.01	0.01\\
436.01	0.01\\
437.01	0.01\\
438.01	0.01\\
439.01	0.01\\
440.01	0.01\\
441.01	0.01\\
442.01	0.01\\
443.01	0.01\\
444.01	0.01\\
445.01	0.01\\
446.01	0.01\\
447.01	0.01\\
448.01	0.01\\
449.01	0.01\\
450.01	0.01\\
451.01	0.01\\
452.01	0.01\\
453.01	0.01\\
454.01	0.01\\
455.01	0.01\\
456.01	0.01\\
457.01	0.01\\
458.01	0.01\\
459.01	0.01\\
460.01	0.01\\
461.01	0.01\\
462.01	0.01\\
463.01	0.01\\
464.01	0.01\\
465.01	0.01\\
466.01	0.01\\
467.01	0.01\\
468.01	0.01\\
469.01	0.01\\
470.01	0.01\\
471.01	0.01\\
472.01	0.01\\
473.01	0.01\\
474.01	0.01\\
475.01	0.01\\
476.01	0.01\\
477.01	0.01\\
478.01	0.01\\
479.01	0.01\\
480.01	0.01\\
481.01	0.00997590330187731\\
482.01	0.0099397710803562\\
483.01	0.009902506047415\\
484.01	0.00986406736092241\\
485.01	0.00982441532168716\\
486.01	0.00978351272117389\\
487.01	0.00974132671698551\\
488.01	0.00969783141771086\\
489.01	0.00965301141901538\\
490.01	0.0096068666127476\\
491.01	0.00955940405583456\\
492.01	0.00951058542054645\\
493.01	0.00946035531854987\\
494.01	0.00940865521854821\\
495.01	0.00935542338323276\\
496.01	0.00930059487426016\\
497.01	0.00924410165733517\\
498.01	0.00918587285084049\\
499.01	0.00912583517703891\\
500.01	0.00906391369582405\\
501.01	0.00900003292861265\\
502.01	0.00893411851633223\\
503.01	0.00886609960406335\\
504.01	0.00879591220965877\\
505.01	0.00872349919267541\\
506.01	0.00864876615224294\\
507.01	0.00857158411947091\\
508.01	0.00849181001384525\\
509.01	0.00840928475711463\\
510.01	0.00832383064150769\\
511.01	0.00823524816323673\\
512.01	0.00814331219581346\\
513.01	0.00804776734473679\\
514.01	0.00794832228261845\\
515.01	0.00784464280879699\\
516.01	0.00773634330593778\\
517.01	0.00762297617260505\\
518.01	0.00750401868706793\\
519.01	0.00737885659549818\\
520.01	0.00724676352707722\\
521.01	0.00710687507426357\\
522.01	0.00695815600761101\\
523.01	0.00679935860995764\\
524.01	0.00662982152981215\\
525.01	0.0064522538886912\\
526.01	0.00637048746293591\\
527.01	0.00628910728394081\\
528.01	0.00620543104607813\\
529.01	0.00611943051037367\\
530.01	0.00603108724159264\\
531.01	0.00594039608167631\\
532.01	0.00584736991317263\\
533.01	0.00575204627680054\\
534.01	0.00565449621512798\\
535.01	0.00555484123860636\\
536.01	0.0054532634251946\\
537.01	0.0053500142876607\\
538.01	0.00524543190603984\\
539.01	0.00513996137705213\\
540.01	0.00503418067720346\\
541.01	0.00492883311832434\\
542.01	0.00482493241400958\\
543.01	0.00472395111388705\\
544.01	0.00462778929292503\\
545.01	0.00453327746680012\\
546.01	0.00443779374764156\\
547.01	0.00434190759610771\\
548.01	0.0042464049440776\\
549.01	0.00415236225924511\\
550.01	0.00405829690783676\\
551.01	0.00396203517870896\\
552.01	0.00386376249144564\\
553.01	0.00376376412958557\\
554.01	0.00366246157832406\\
555.01	0.00356082309427307\\
556.01	0.00346030794138988\\
557.01	0.00336129707339248\\
558.01	0.0032640477506096\\
559.01	0.00316867271328949\\
560.01	0.00307507527510044\\
561.01	0.00298338436260538\\
562.01	0.00289399019343357\\
563.01	0.0028072164128482\\
564.01	0.00272322791111361\\
565.01	0.0026410043770131\\
566.01	0.00255988858995343\\
567.01	0.00247982934092541\\
568.01	0.00240074249336894\\
569.01	0.00232261423584147\\
570.01	0.00224537135654028\\
571.01	0.00216886546893075\\
572.01	0.00209288208425282\\
573.01	0.00201717448028996\\
574.01	0.00194157448049112\\
575.01	0.00186606419108644\\
576.01	0.00179062678453804\\
577.01	0.00171522496449762\\
578.01	0.00163980400119186\\
579.01	0.00156429794377999\\
580.01	0.00148863948022046\\
581.01	0.00141277325550749\\
582.01	0.00133667030299336\\
583.01	0.00126032737918952\\
584.01	0.00118374811240739\\
585.01	0.00110693731887654\\
586.01	0.00102989957727651\\
587.01	0.000952638358326865\\
588.01	0.00087515672606412\\
589.01	0.000797457530026434\\
590.01	0.000719542457697378\\
591.01	0.000641410561971532\\
592.01	0.000563058798570631\\
593.01	0.000484484304561451\\
594.01	0.000405687038562455\\
595.01	0.000326672291767633\\
596.01	0.000247452480598537\\
597.01	0.000168047438790727\\
598.01	9.13423856162294e-05\\
599.01	2.91272356096692e-05\\
599.02	2.86193161340032e-05\\
599.03	2.81144595931631e-05\\
599.04	2.76126958075537e-05\\
599.05	2.71140548916225e-05\\
599.06	2.66185672567684e-05\\
599.07	2.61262636142949e-05\\
599.08	2.56371749783636e-05\\
599.09	2.51513326690062e-05\\
599.1	2.46687683151548e-05\\
599.11	2.41895138576968e-05\\
599.12	2.37136015525663e-05\\
599.13	2.32410639738646e-05\\
599.14	2.27719340170195e-05\\
599.15	2.23062449019578e-05\\
599.16	2.18440301763267e-05\\
599.17	2.13853237187381e-05\\
599.18	2.09301597420501e-05\\
599.19	2.04785727966738e-05\\
599.2	2.00305977739214e-05\\
599.21	1.95862699093816e-05\\
599.22	1.91456247863337e-05\\
599.23	1.87086983391843e-05\\
599.24	1.8275526856959e-05\\
599.25	1.78461469867996e-05\\
599.26	1.74205963976177e-05\\
599.27	1.69989155754252e-05\\
599.28	1.65811454095281e-05\\
599.29	1.61673271965243e-05\\
599.3	1.57575026443323e-05\\
599.31	1.53517138762656e-05\\
599.32	1.49500034351425e-05\\
599.33	1.45524142874422e-05\\
599.34	1.41589898274928e-05\\
599.35	1.37697738817107e-05\\
599.36	1.33848107128787e-05\\
599.37	1.30041450244651e-05\\
599.38	1.26278219649834e-05\\
599.39	1.2255887132398e-05\\
599.4	1.18883865785728e-05\\
599.41	1.15253668137704e-05\\
599.42	1.11668748111818e-05\\
599.43	1.08129580115111e-05\\
599.44	1.04636643276056e-05\\
599.45	1.01190421491221e-05\\
599.46	9.77914034725434e-06\\
599.47	9.44400827949769e-06\\
599.48	9.1136957944566e-06\\
599.49	8.78825323671374e-06\\
599.5	8.46773145173063e-06\\
599.51	8.15218179081069e-06\\
599.52	7.84165611610738e-06\\
599.53	7.53620680567398e-06\\
599.54	7.23588675857409e-06\\
599.55	6.94074940003724e-06\\
599.56	6.65084868666475e-06\\
599.57	6.36623911169296e-06\\
599.58	6.08697571029791e-06\\
599.59	5.81311406496442e-06\\
599.6	5.54471031090353e-06\\
599.61	5.28182114151003e-06\\
599.62	5.0245038139083e-06\\
599.63	4.77281615451214e-06\\
599.64	4.52681656466433e-06\\
599.65	4.28656402633523e-06\\
599.66	4.05211810785081e-06\\
599.67	3.82353896971784e-06\\
599.68	3.60088737046127e-06\\
599.69	3.38422467256044e-06\\
599.7	3.17361284841132e-06\\
599.71	2.96911448637548e-06\\
599.72	2.77079279686725e-06\\
599.73	2.57871161851199e-06\\
599.74	2.39293542435466e-06\\
599.75	2.21352932815513e-06\\
599.76	2.04055909071126e-06\\
599.77	1.87409112627039e-06\\
599.78	1.71419250899474e-06\\
599.79	1.56093097948388e-06\\
599.8	1.41437495138237e-06\\
599.81	1.27459351803062e-06\\
599.82	1.14165645918526e-06\\
599.83	1.01563424782618e-06\\
599.84	8.96598057003456e-07\\
599.85	7.84619766767622e-07\\
599.86	6.79771971167503e-07\\
599.87	5.82127985312292e-07\\
599.88	4.91761852506462e-07\\
599.89	4.08748351459279e-07\\
599.9	3.33163003549825e-07\\
599.91	2.65082080187426e-07\\
599.92	2.04582610225865e-07\\
599.93	1.51742387459117e-07\\
599.94	1.06639978193421e-07\\
599.95	6.93547288800611e-08\\
599.96	3.99667738487652e-08\\
599.97	1.85570430896731e-08\\
599.98	5.20727013418598e-09\\
599.99	0\\
600	0\\
};
\addplot [color=blue,solid,forget plot]
  table[row sep=crcr]{%
0.01	0.01\\
1.01	0.01\\
2.01	0.01\\
3.01	0.01\\
4.01	0.01\\
5.01	0.01\\
6.01	0.01\\
7.01	0.01\\
8.01	0.01\\
9.01	0.01\\
10.01	0.01\\
11.01	0.01\\
12.01	0.01\\
13.01	0.01\\
14.01	0.01\\
15.01	0.01\\
16.01	0.01\\
17.01	0.01\\
18.01	0.01\\
19.01	0.01\\
20.01	0.01\\
21.01	0.01\\
22.01	0.01\\
23.01	0.01\\
24.01	0.01\\
25.01	0.01\\
26.01	0.01\\
27.01	0.01\\
28.01	0.01\\
29.01	0.01\\
30.01	0.01\\
31.01	0.01\\
32.01	0.01\\
33.01	0.01\\
34.01	0.01\\
35.01	0.01\\
36.01	0.01\\
37.01	0.01\\
38.01	0.01\\
39.01	0.01\\
40.01	0.01\\
41.01	0.01\\
42.01	0.01\\
43.01	0.01\\
44.01	0.01\\
45.01	0.01\\
46.01	0.01\\
47.01	0.01\\
48.01	0.01\\
49.01	0.01\\
50.01	0.01\\
51.01	0.01\\
52.01	0.01\\
53.01	0.01\\
54.01	0.01\\
55.01	0.01\\
56.01	0.01\\
57.01	0.01\\
58.01	0.01\\
59.01	0.01\\
60.01	0.01\\
61.01	0.01\\
62.01	0.01\\
63.01	0.01\\
64.01	0.01\\
65.01	0.01\\
66.01	0.01\\
67.01	0.01\\
68.01	0.01\\
69.01	0.01\\
70.01	0.01\\
71.01	0.01\\
72.01	0.01\\
73.01	0.01\\
74.01	0.01\\
75.01	0.01\\
76.01	0.01\\
77.01	0.01\\
78.01	0.01\\
79.01	0.01\\
80.01	0.01\\
81.01	0.01\\
82.01	0.01\\
83.01	0.01\\
84.01	0.01\\
85.01	0.01\\
86.01	0.01\\
87.01	0.01\\
88.01	0.01\\
89.01	0.01\\
90.01	0.01\\
91.01	0.01\\
92.01	0.01\\
93.01	0.01\\
94.01	0.01\\
95.01	0.01\\
96.01	0.01\\
97.01	0.01\\
98.01	0.01\\
99.01	0.01\\
100.01	0.01\\
101.01	0.01\\
102.01	0.01\\
103.01	0.01\\
104.01	0.01\\
105.01	0.01\\
106.01	0.01\\
107.01	0.01\\
108.01	0.01\\
109.01	0.01\\
110.01	0.01\\
111.01	0.01\\
112.01	0.01\\
113.01	0.01\\
114.01	0.01\\
115.01	0.01\\
116.01	0.01\\
117.01	0.01\\
118.01	0.01\\
119.01	0.01\\
120.01	0.01\\
121.01	0.01\\
122.01	0.01\\
123.01	0.01\\
124.01	0.01\\
125.01	0.01\\
126.01	0.01\\
127.01	0.01\\
128.01	0.01\\
129.01	0.01\\
130.01	0.01\\
131.01	0.01\\
132.01	0.01\\
133.01	0.01\\
134.01	0.01\\
135.01	0.01\\
136.01	0.01\\
137.01	0.01\\
138.01	0.01\\
139.01	0.01\\
140.01	0.01\\
141.01	0.01\\
142.01	0.01\\
143.01	0.01\\
144.01	0.01\\
145.01	0.01\\
146.01	0.01\\
147.01	0.01\\
148.01	0.01\\
149.01	0.01\\
150.01	0.01\\
151.01	0.01\\
152.01	0.01\\
153.01	0.01\\
154.01	0.01\\
155.01	0.01\\
156.01	0.01\\
157.01	0.01\\
158.01	0.01\\
159.01	0.01\\
160.01	0.01\\
161.01	0.01\\
162.01	0.01\\
163.01	0.01\\
164.01	0.01\\
165.01	0.01\\
166.01	0.01\\
167.01	0.01\\
168.01	0.01\\
169.01	0.01\\
170.01	0.01\\
171.01	0.01\\
172.01	0.01\\
173.01	0.01\\
174.01	0.01\\
175.01	0.01\\
176.01	0.01\\
177.01	0.01\\
178.01	0.01\\
179.01	0.01\\
180.01	0.01\\
181.01	0.01\\
182.01	0.01\\
183.01	0.01\\
184.01	0.01\\
185.01	0.01\\
186.01	0.01\\
187.01	0.01\\
188.01	0.01\\
189.01	0.01\\
190.01	0.01\\
191.01	0.01\\
192.01	0.01\\
193.01	0.01\\
194.01	0.01\\
195.01	0.01\\
196.01	0.01\\
197.01	0.01\\
198.01	0.01\\
199.01	0.01\\
200.01	0.01\\
201.01	0.01\\
202.01	0.01\\
203.01	0.01\\
204.01	0.01\\
205.01	0.01\\
206.01	0.01\\
207.01	0.01\\
208.01	0.01\\
209.01	0.01\\
210.01	0.01\\
211.01	0.01\\
212.01	0.01\\
213.01	0.01\\
214.01	0.01\\
215.01	0.01\\
216.01	0.01\\
217.01	0.01\\
218.01	0.01\\
219.01	0.01\\
220.01	0.01\\
221.01	0.01\\
222.01	0.01\\
223.01	0.01\\
224.01	0.01\\
225.01	0.01\\
226.01	0.01\\
227.01	0.01\\
228.01	0.01\\
229.01	0.01\\
230.01	0.01\\
231.01	0.01\\
232.01	0.01\\
233.01	0.01\\
234.01	0.01\\
235.01	0.01\\
236.01	0.01\\
237.01	0.01\\
238.01	0.01\\
239.01	0.01\\
240.01	0.01\\
241.01	0.01\\
242.01	0.01\\
243.01	0.01\\
244.01	0.01\\
245.01	0.01\\
246.01	0.01\\
247.01	0.01\\
248.01	0.01\\
249.01	0.01\\
250.01	0.01\\
251.01	0.01\\
252.01	0.01\\
253.01	0.01\\
254.01	0.01\\
255.01	0.01\\
256.01	0.01\\
257.01	0.01\\
258.01	0.01\\
259.01	0.01\\
260.01	0.01\\
261.01	0.01\\
262.01	0.01\\
263.01	0.01\\
264.01	0.01\\
265.01	0.01\\
266.01	0.01\\
267.01	0.01\\
268.01	0.01\\
269.01	0.01\\
270.01	0.01\\
271.01	0.01\\
272.01	0.01\\
273.01	0.01\\
274.01	0.01\\
275.01	0.01\\
276.01	0.01\\
277.01	0.01\\
278.01	0.01\\
279.01	0.01\\
280.01	0.01\\
281.01	0.01\\
282.01	0.01\\
283.01	0.01\\
284.01	0.01\\
285.01	0.01\\
286.01	0.01\\
287.01	0.01\\
288.01	0.01\\
289.01	0.01\\
290.01	0.01\\
291.01	0.01\\
292.01	0.01\\
293.01	0.01\\
294.01	0.01\\
295.01	0.01\\
296.01	0.01\\
297.01	0.01\\
298.01	0.01\\
299.01	0.01\\
300.01	0.01\\
301.01	0.01\\
302.01	0.01\\
303.01	0.01\\
304.01	0.01\\
305.01	0.01\\
306.01	0.01\\
307.01	0.01\\
308.01	0.01\\
309.01	0.01\\
310.01	0.01\\
311.01	0.01\\
312.01	0.01\\
313.01	0.01\\
314.01	0.01\\
315.01	0.01\\
316.01	0.01\\
317.01	0.01\\
318.01	0.01\\
319.01	0.01\\
320.01	0.01\\
321.01	0.01\\
322.01	0.01\\
323.01	0.01\\
324.01	0.01\\
325.01	0.01\\
326.01	0.01\\
327.01	0.01\\
328.01	0.01\\
329.01	0.01\\
330.01	0.01\\
331.01	0.01\\
332.01	0.01\\
333.01	0.01\\
334.01	0.01\\
335.01	0.01\\
336.01	0.01\\
337.01	0.01\\
338.01	0.01\\
339.01	0.01\\
340.01	0.01\\
341.01	0.01\\
342.01	0.01\\
343.01	0.01\\
344.01	0.01\\
345.01	0.01\\
346.01	0.01\\
347.01	0.01\\
348.01	0.01\\
349.01	0.01\\
350.01	0.01\\
351.01	0.01\\
352.01	0.01\\
353.01	0.01\\
354.01	0.01\\
355.01	0.01\\
356.01	0.01\\
357.01	0.01\\
358.01	0.01\\
359.01	0.01\\
360.01	0.01\\
361.01	0.01\\
362.01	0.01\\
363.01	0.01\\
364.01	0.01\\
365.01	0.01\\
366.01	0.01\\
367.01	0.01\\
368.01	0.01\\
369.01	0.01\\
370.01	0.01\\
371.01	0.01\\
372.01	0.01\\
373.01	0.01\\
374.01	0.01\\
375.01	0.01\\
376.01	0.01\\
377.01	0.01\\
378.01	0.01\\
379.01	0.01\\
380.01	0.01\\
381.01	0.01\\
382.01	0.01\\
383.01	0.01\\
384.01	0.01\\
385.01	0.01\\
386.01	0.01\\
387.01	0.01\\
388.01	0.01\\
389.01	0.01\\
390.01	0.01\\
391.01	0.01\\
392.01	0.01\\
393.01	0.01\\
394.01	0.01\\
395.01	0.01\\
396.01	0.01\\
397.01	0.01\\
398.01	0.01\\
399.01	0.01\\
400.01	0.01\\
401.01	0.01\\
402.01	0.01\\
403.01	0.01\\
404.01	0.01\\
405.01	0.01\\
406.01	0.01\\
407.01	0.01\\
408.01	0.01\\
409.01	0.01\\
410.01	0.01\\
411.01	0.01\\
412.01	0.01\\
413.01	0.01\\
414.01	0.01\\
415.01	0.01\\
416.01	0.01\\
417.01	0.01\\
418.01	0.01\\
419.01	0.01\\
420.01	0.01\\
421.01	0.01\\
422.01	0.01\\
423.01	0.01\\
424.01	0.01\\
425.01	0.01\\
426.01	0.01\\
427.01	0.01\\
428.01	0.01\\
429.01	0.01\\
430.01	0.01\\
431.01	0.01\\
432.01	0.01\\
433.01	0.01\\
434.01	0.01\\
435.01	0.01\\
436.01	0.01\\
437.01	0.01\\
438.01	0.01\\
439.01	0.01\\
440.01	0.01\\
441.01	0.01\\
442.01	0.01\\
443.01	0.01\\
444.01	0.01\\
445.01	0.01\\
446.01	0.01\\
447.01	0.01\\
448.01	0.01\\
449.01	0.01\\
450.01	0.01\\
451.01	0.01\\
452.01	0.01\\
453.01	0.01\\
454.01	0.01\\
455.01	0.01\\
456.01	0.01\\
457.01	0.01\\
458.01	0.01\\
459.01	0.01\\
460.01	0.01\\
461.01	0.01\\
462.01	0.01\\
463.01	0.01\\
464.01	0.01\\
465.01	0.01\\
466.01	0.01\\
467.01	0.01\\
468.01	0.01\\
469.01	0.01\\
470.01	0.01\\
471.01	0.01\\
472.01	0.01\\
473.01	0.01\\
474.01	0.01\\
475.01	0.01\\
476.01	0.01\\
477.01	0.01\\
478.01	0.01\\
479.01	0.01\\
480.01	0.01\\
481.01	0.01\\
482.01	0.01\\
483.01	0.01\\
484.01	0.01\\
485.01	0.01\\
486.01	0.01\\
487.01	0.01\\
488.01	0.01\\
489.01	0.01\\
490.01	0.01\\
491.01	0.01\\
492.01	0.01\\
493.01	0.01\\
494.01	0.01\\
495.01	0.01\\
496.01	0.01\\
497.01	0.01\\
498.01	0.01\\
499.01	0.01\\
500.01	0.01\\
501.01	0.01\\
502.01	0.01\\
503.01	0.01\\
504.01	0.01\\
505.01	0.01\\
506.01	0.01\\
507.01	0.01\\
508.01	0.01\\
509.01	0.01\\
510.01	0.01\\
511.01	0.01\\
512.01	0.01\\
513.01	0.01\\
514.01	0.01\\
515.01	0.01\\
516.01	0.01\\
517.01	0.01\\
518.01	0.01\\
519.01	0.01\\
520.01	0.01\\
521.01	0.01\\
522.01	0.01\\
523.01	0.01\\
524.01	0.01\\
525.01	0.01\\
526.01	0.00989722854349804\\
527.01	0.00978814273751665\\
528.01	0.00967516165466474\\
529.01	0.00955803374497832\\
530.01	0.00943648091405857\\
531.01	0.00931019467488656\\
532.01	0.00917883187684529\\
533.01	0.00904201077393832\\
534.01	0.00889930832443303\\
535.01	0.00875023292005354\\
536.01	0.00859420773776338\\
537.01	0.00843056963164173\\
538.01	0.00825857320407405\\
539.01	0.00807737494716381\\
540.01	0.00788601007179719\\
541.01	0.00768337231581051\\
542.01	0.00746812865996539\\
543.01	0.00723855993773447\\
544.01	0.00699263227818423\\
545.01	0.00673338432713554\\
546.01	0.00646300161813907\\
547.01	0.00618042366892763\\
548.01	0.00588439316956648\\
549.01	0.00557471857017588\\
550.01	0.0054365780548863\\
551.01	0.00529498022819429\\
552.01	0.00515020347795563\\
553.01	0.00500264559133276\\
554.01	0.00485285565777818\\
555.01	0.00470157180593755\\
556.01	0.00454976142488678\\
557.01	0.00439872524951088\\
558.01	0.00425022313750997\\
559.01	0.00410662240815378\\
560.01	0.00396855697861401\\
561.01	0.00382883467193446\\
562.01	0.00368720830181002\\
563.01	0.00354449059715707\\
564.01	0.00340185782150943\\
565.01	0.00326208932254077\\
566.01	0.0031280109045733\\
567.01	0.00300169631967474\\
568.01	0.00287904197142182\\
569.01	0.00275913443257296\\
570.01	0.00264254980737055\\
571.01	0.00252970762472733\\
572.01	0.00242071734339785\\
573.01	0.00231515549731677\\
574.01	0.00221140419527129\\
575.01	0.0021090280458826\\
576.01	0.00200828344457156\\
577.01	0.00190936925280396\\
578.01	0.00181239939701331\\
579.01	0.0017173717974502\\
580.01	0.00162413610956826\\
581.01	0.00153236543046933\\
582.01	0.00144157354335622\\
583.01	0.0013514547773948\\
584.01	0.0012619843991491\\
585.01	0.00117317775709303\\
586.01	0.00108508625059967\\
587.01	0.000997759365490835\\
588.01	0.00091123592620997\\
589.01	0.000825550488607604\\
590.01	0.000740739252561627\\
591.01	0.000656829007057548\\
592.01	0.000573812567071902\\
593.01	0.000491642606866438\\
594.01	0.000410231936064402\\
595.01	0.000329457487875126\\
596.01	0.000249168838092057\\
597.01	0.000169201630515214\\
598.01	9.14271566706763e-05\\
599.01	2.9130630291278e-05\\
599.02	2.86225745231034e-05\\
599.03	2.81175858083405e-05\\
599.04	2.76156938848176e-05\\
599.05	2.71169287851831e-05\\
599.06	2.66213208398331e-05\\
599.07	2.61289006798487e-05\\
599.08	2.56396992399759e-05\\
599.09	2.51537477616182e-05\\
599.1	2.46710777958743e-05\\
599.11	2.4191721206596e-05\\
599.12	2.37157101734831e-05\\
599.13	2.32430771952095e-05\\
599.14	2.27738550925647e-05\\
599.15	2.23080770116509e-05\\
599.16	2.18457764270957e-05\\
599.17	2.13869871452978e-05\\
599.18	2.09317433077106e-05\\
599.19	2.04800793941525e-05\\
599.2	2.00320302261563e-05\\
599.21	1.95876309703451e-05\\
599.22	1.91469171418445e-05\\
599.23	1.87099246077346e-05\\
599.24	1.82766895905177e-05\\
599.25	1.78472486716558e-05\\
599.26	1.74216394527651e-05\\
599.27	1.69999023562576e-05\\
599.28	1.65820782086351e-05\\
599.29	1.61682082444846e-05\\
599.3	1.57583341105112e-05\\
599.31	1.53524978696098e-05\\
599.32	1.49507420049849e-05\\
599.33	1.45531094242948e-05\\
599.34	1.41596434638531e-05\\
599.35	1.37703878928669e-05\\
599.36	1.33853869177126e-05\\
599.37	1.30046851862537e-05\\
599.38	1.26283277922107e-05\\
599.39	1.22563602795653e-05\\
599.4	1.18888286470102e-05\\
599.41	1.1525779352442e-05\\
599.42	1.11672593175044e-05\\
599.43	1.0813315932166e-05\\
599.44	1.04639970593626e-05\\
599.45	1.01193510396499e-05\\
599.46	9.7794266959491e-06\\
599.47	9.44427333829899e-06\\
599.48	9.11394076868154e-06\\
599.49	8.78847928588097e-06\\
599.5	8.46793969039999e-06\\
599.51	8.15237328941759e-06\\
599.52	7.8418319017972e-06\\
599.53	7.53636786315036e-06\\
599.54	7.23603403094721e-06\\
599.55	6.94088378966513e-06\\
599.56	6.65097105601725e-06\\
599.57	6.36635028419819e-06\\
599.58	6.08707647120968e-06\\
599.59	5.81320516222436e-06\\
599.6	5.54479245599981e-06\\
599.61	5.28189501036898e-06\\
599.62	5.02457004774616e-06\\
599.63	4.77287536072325e-06\\
599.64	4.52686931771279e-06\\
599.65	4.28661086862223e-06\\
599.66	4.05215955062363e-06\\
599.67	3.82357549394419e-06\\
599.68	3.60091942774518e-06\\
599.69	3.38425268603219e-06\\
599.7	3.17363721363643e-06\\
599.71	2.96913557226545e-06\\
599.72	2.77081094658509e-06\\
599.73	2.57872715039159e-06\\
599.74	2.39294863282891e-06\\
599.75	2.21354048466835e-06\\
599.76	2.04056844465075e-06\\
599.77	1.87409890589456e-06\\
599.78	1.71419892236284e-06\\
599.79	1.56093621539802e-06\\
599.8	1.41437918031902e-06\\
599.81	1.27459689307913e-06\\
599.82	1.14165911700367e-06\\
599.83	1.01563630956411e-06\\
599.84	8.9659962925967e-07\\
599.85	7.84620942530234e-07\\
599.86	6.79772830764619e-07\\
599.87	5.82128597343551e-07\\
599.88	4.91762274797136e-07\\
599.89	4.0874863199182e-07\\
599.9	3.33163181409288e-07\\
599.91	2.65082186487811e-07\\
599.92	2.04582669039929e-07\\
599.93	1.51742416741249e-07\\
599.94	1.06639990685164e-07\\
599.95	6.93547330260502e-08\\
599.96	3.99667746744936e-08\\
599.97	1.85570430896731e-08\\
599.98	5.20727013418598e-09\\
599.99	0\\
600	0\\
};
\addplot [color=mycolor10,solid,forget plot]
  table[row sep=crcr]{%
0.01	0.01\\
1.01	0.01\\
2.01	0.01\\
3.01	0.01\\
4.01	0.01\\
5.01	0.01\\
6.01	0.01\\
7.01	0.01\\
8.01	0.01\\
9.01	0.01\\
10.01	0.01\\
11.01	0.01\\
12.01	0.01\\
13.01	0.01\\
14.01	0.01\\
15.01	0.01\\
16.01	0.01\\
17.01	0.01\\
18.01	0.01\\
19.01	0.01\\
20.01	0.01\\
21.01	0.01\\
22.01	0.01\\
23.01	0.01\\
24.01	0.01\\
25.01	0.01\\
26.01	0.01\\
27.01	0.01\\
28.01	0.01\\
29.01	0.01\\
30.01	0.01\\
31.01	0.01\\
32.01	0.01\\
33.01	0.01\\
34.01	0.01\\
35.01	0.01\\
36.01	0.01\\
37.01	0.01\\
38.01	0.01\\
39.01	0.01\\
40.01	0.01\\
41.01	0.01\\
42.01	0.01\\
43.01	0.01\\
44.01	0.01\\
45.01	0.01\\
46.01	0.01\\
47.01	0.01\\
48.01	0.01\\
49.01	0.01\\
50.01	0.01\\
51.01	0.01\\
52.01	0.01\\
53.01	0.01\\
54.01	0.01\\
55.01	0.01\\
56.01	0.01\\
57.01	0.01\\
58.01	0.01\\
59.01	0.01\\
60.01	0.01\\
61.01	0.01\\
62.01	0.01\\
63.01	0.01\\
64.01	0.01\\
65.01	0.01\\
66.01	0.01\\
67.01	0.01\\
68.01	0.01\\
69.01	0.01\\
70.01	0.01\\
71.01	0.01\\
72.01	0.01\\
73.01	0.01\\
74.01	0.01\\
75.01	0.01\\
76.01	0.01\\
77.01	0.01\\
78.01	0.01\\
79.01	0.01\\
80.01	0.01\\
81.01	0.01\\
82.01	0.01\\
83.01	0.01\\
84.01	0.01\\
85.01	0.01\\
86.01	0.01\\
87.01	0.01\\
88.01	0.01\\
89.01	0.01\\
90.01	0.01\\
91.01	0.01\\
92.01	0.01\\
93.01	0.01\\
94.01	0.01\\
95.01	0.01\\
96.01	0.01\\
97.01	0.01\\
98.01	0.01\\
99.01	0.01\\
100.01	0.01\\
101.01	0.01\\
102.01	0.01\\
103.01	0.01\\
104.01	0.01\\
105.01	0.01\\
106.01	0.01\\
107.01	0.01\\
108.01	0.01\\
109.01	0.01\\
110.01	0.01\\
111.01	0.01\\
112.01	0.01\\
113.01	0.01\\
114.01	0.01\\
115.01	0.01\\
116.01	0.01\\
117.01	0.01\\
118.01	0.01\\
119.01	0.01\\
120.01	0.01\\
121.01	0.01\\
122.01	0.01\\
123.01	0.01\\
124.01	0.01\\
125.01	0.01\\
126.01	0.01\\
127.01	0.01\\
128.01	0.01\\
129.01	0.01\\
130.01	0.01\\
131.01	0.01\\
132.01	0.01\\
133.01	0.01\\
134.01	0.01\\
135.01	0.01\\
136.01	0.01\\
137.01	0.01\\
138.01	0.01\\
139.01	0.01\\
140.01	0.01\\
141.01	0.01\\
142.01	0.01\\
143.01	0.01\\
144.01	0.01\\
145.01	0.01\\
146.01	0.01\\
147.01	0.01\\
148.01	0.01\\
149.01	0.01\\
150.01	0.01\\
151.01	0.01\\
152.01	0.01\\
153.01	0.01\\
154.01	0.01\\
155.01	0.01\\
156.01	0.01\\
157.01	0.01\\
158.01	0.01\\
159.01	0.01\\
160.01	0.01\\
161.01	0.01\\
162.01	0.01\\
163.01	0.01\\
164.01	0.01\\
165.01	0.01\\
166.01	0.01\\
167.01	0.01\\
168.01	0.01\\
169.01	0.01\\
170.01	0.01\\
171.01	0.01\\
172.01	0.01\\
173.01	0.01\\
174.01	0.01\\
175.01	0.01\\
176.01	0.01\\
177.01	0.01\\
178.01	0.01\\
179.01	0.01\\
180.01	0.01\\
181.01	0.01\\
182.01	0.01\\
183.01	0.01\\
184.01	0.01\\
185.01	0.01\\
186.01	0.01\\
187.01	0.01\\
188.01	0.01\\
189.01	0.01\\
190.01	0.01\\
191.01	0.01\\
192.01	0.01\\
193.01	0.01\\
194.01	0.01\\
195.01	0.01\\
196.01	0.01\\
197.01	0.01\\
198.01	0.01\\
199.01	0.01\\
200.01	0.01\\
201.01	0.01\\
202.01	0.01\\
203.01	0.01\\
204.01	0.01\\
205.01	0.01\\
206.01	0.01\\
207.01	0.01\\
208.01	0.01\\
209.01	0.01\\
210.01	0.01\\
211.01	0.01\\
212.01	0.01\\
213.01	0.01\\
214.01	0.01\\
215.01	0.01\\
216.01	0.01\\
217.01	0.01\\
218.01	0.01\\
219.01	0.01\\
220.01	0.01\\
221.01	0.01\\
222.01	0.01\\
223.01	0.01\\
224.01	0.01\\
225.01	0.01\\
226.01	0.01\\
227.01	0.01\\
228.01	0.01\\
229.01	0.01\\
230.01	0.01\\
231.01	0.01\\
232.01	0.01\\
233.01	0.01\\
234.01	0.01\\
235.01	0.01\\
236.01	0.01\\
237.01	0.01\\
238.01	0.01\\
239.01	0.01\\
240.01	0.01\\
241.01	0.01\\
242.01	0.01\\
243.01	0.01\\
244.01	0.01\\
245.01	0.01\\
246.01	0.01\\
247.01	0.01\\
248.01	0.01\\
249.01	0.01\\
250.01	0.01\\
251.01	0.01\\
252.01	0.01\\
253.01	0.01\\
254.01	0.01\\
255.01	0.01\\
256.01	0.01\\
257.01	0.01\\
258.01	0.01\\
259.01	0.01\\
260.01	0.01\\
261.01	0.01\\
262.01	0.01\\
263.01	0.01\\
264.01	0.01\\
265.01	0.01\\
266.01	0.01\\
267.01	0.01\\
268.01	0.01\\
269.01	0.01\\
270.01	0.01\\
271.01	0.01\\
272.01	0.01\\
273.01	0.01\\
274.01	0.01\\
275.01	0.01\\
276.01	0.01\\
277.01	0.01\\
278.01	0.01\\
279.01	0.01\\
280.01	0.01\\
281.01	0.01\\
282.01	0.01\\
283.01	0.01\\
284.01	0.01\\
285.01	0.01\\
286.01	0.01\\
287.01	0.01\\
288.01	0.01\\
289.01	0.01\\
290.01	0.01\\
291.01	0.01\\
292.01	0.01\\
293.01	0.01\\
294.01	0.01\\
295.01	0.01\\
296.01	0.01\\
297.01	0.01\\
298.01	0.01\\
299.01	0.01\\
300.01	0.01\\
301.01	0.01\\
302.01	0.01\\
303.01	0.01\\
304.01	0.01\\
305.01	0.01\\
306.01	0.01\\
307.01	0.01\\
308.01	0.01\\
309.01	0.01\\
310.01	0.01\\
311.01	0.01\\
312.01	0.01\\
313.01	0.01\\
314.01	0.01\\
315.01	0.01\\
316.01	0.01\\
317.01	0.01\\
318.01	0.01\\
319.01	0.01\\
320.01	0.01\\
321.01	0.01\\
322.01	0.01\\
323.01	0.01\\
324.01	0.01\\
325.01	0.01\\
326.01	0.01\\
327.01	0.01\\
328.01	0.01\\
329.01	0.01\\
330.01	0.01\\
331.01	0.01\\
332.01	0.01\\
333.01	0.01\\
334.01	0.01\\
335.01	0.01\\
336.01	0.01\\
337.01	0.01\\
338.01	0.01\\
339.01	0.01\\
340.01	0.01\\
341.01	0.01\\
342.01	0.01\\
343.01	0.01\\
344.01	0.01\\
345.01	0.01\\
346.01	0.01\\
347.01	0.01\\
348.01	0.01\\
349.01	0.01\\
350.01	0.01\\
351.01	0.01\\
352.01	0.01\\
353.01	0.01\\
354.01	0.01\\
355.01	0.01\\
356.01	0.01\\
357.01	0.01\\
358.01	0.01\\
359.01	0.01\\
360.01	0.01\\
361.01	0.01\\
362.01	0.01\\
363.01	0.01\\
364.01	0.01\\
365.01	0.01\\
366.01	0.01\\
367.01	0.01\\
368.01	0.01\\
369.01	0.01\\
370.01	0.01\\
371.01	0.01\\
372.01	0.01\\
373.01	0.01\\
374.01	0.01\\
375.01	0.01\\
376.01	0.01\\
377.01	0.01\\
378.01	0.01\\
379.01	0.01\\
380.01	0.01\\
381.01	0.01\\
382.01	0.01\\
383.01	0.01\\
384.01	0.01\\
385.01	0.01\\
386.01	0.01\\
387.01	0.01\\
388.01	0.01\\
389.01	0.01\\
390.01	0.01\\
391.01	0.01\\
392.01	0.01\\
393.01	0.01\\
394.01	0.01\\
395.01	0.01\\
396.01	0.01\\
397.01	0.01\\
398.01	0.01\\
399.01	0.01\\
400.01	0.01\\
401.01	0.01\\
402.01	0.01\\
403.01	0.01\\
404.01	0.01\\
405.01	0.01\\
406.01	0.01\\
407.01	0.01\\
408.01	0.01\\
409.01	0.01\\
410.01	0.01\\
411.01	0.01\\
412.01	0.01\\
413.01	0.01\\
414.01	0.01\\
415.01	0.01\\
416.01	0.01\\
417.01	0.01\\
418.01	0.01\\
419.01	0.01\\
420.01	0.01\\
421.01	0.01\\
422.01	0.01\\
423.01	0.01\\
424.01	0.01\\
425.01	0.01\\
426.01	0.01\\
427.01	0.01\\
428.01	0.01\\
429.01	0.01\\
430.01	0.01\\
431.01	0.01\\
432.01	0.01\\
433.01	0.01\\
434.01	0.01\\
435.01	0.01\\
436.01	0.01\\
437.01	0.01\\
438.01	0.01\\
439.01	0.01\\
440.01	0.01\\
441.01	0.01\\
442.01	0.01\\
443.01	0.01\\
444.01	0.01\\
445.01	0.01\\
446.01	0.01\\
447.01	0.01\\
448.01	0.01\\
449.01	0.01\\
450.01	0.01\\
451.01	0.01\\
452.01	0.01\\
453.01	0.01\\
454.01	0.01\\
455.01	0.01\\
456.01	0.01\\
457.01	0.01\\
458.01	0.01\\
459.01	0.01\\
460.01	0.01\\
461.01	0.01\\
462.01	0.01\\
463.01	0.01\\
464.01	0.01\\
465.01	0.01\\
466.01	0.01\\
467.01	0.01\\
468.01	0.01\\
469.01	0.01\\
470.01	0.01\\
471.01	0.01\\
472.01	0.01\\
473.01	0.01\\
474.01	0.01\\
475.01	0.01\\
476.01	0.01\\
477.01	0.01\\
478.01	0.01\\
479.01	0.01\\
480.01	0.01\\
481.01	0.01\\
482.01	0.01\\
483.01	0.01\\
484.01	0.01\\
485.01	0.01\\
486.01	0.01\\
487.01	0.01\\
488.01	0.01\\
489.01	0.01\\
490.01	0.01\\
491.01	0.01\\
492.01	0.01\\
493.01	0.01\\
494.01	0.01\\
495.01	0.01\\
496.01	0.01\\
497.01	0.01\\
498.01	0.01\\
499.01	0.01\\
500.01	0.01\\
501.01	0.01\\
502.01	0.01\\
503.01	0.01\\
504.01	0.01\\
505.01	0.01\\
506.01	0.01\\
507.01	0.01\\
508.01	0.01\\
509.01	0.01\\
510.01	0.01\\
511.01	0.01\\
512.01	0.01\\
513.01	0.01\\
514.01	0.01\\
515.01	0.01\\
516.01	0.01\\
517.01	0.01\\
518.01	0.01\\
519.01	0.01\\
520.01	0.01\\
521.01	0.01\\
522.01	0.01\\
523.01	0.01\\
524.01	0.01\\
525.01	0.01\\
526.01	0.01\\
527.01	0.01\\
528.01	0.01\\
529.01	0.01\\
530.01	0.01\\
531.01	0.01\\
532.01	0.01\\
533.01	0.01\\
534.01	0.01\\
535.01	0.01\\
536.01	0.01\\
537.01	0.01\\
538.01	0.01\\
539.01	0.01\\
540.01	0.01\\
541.01	0.01\\
542.01	0.01\\
543.01	0.01\\
544.01	0.01\\
545.01	0.01\\
546.01	0.01\\
547.01	0.01\\
548.01	0.01\\
549.01	0.0099986849723122\\
550.01	0.00981196787112131\\
551.01	0.00961657413274832\\
552.01	0.00941162887606914\\
553.01	0.00919612090986087\\
554.01	0.0089688763992542\\
555.01	0.00872852663770928\\
556.01	0.00847346866094748\\
557.01	0.00820181618375634\\
558.01	0.00791133722938716\\
559.01	0.00759937552712021\\
560.01	0.0072652506873956\\
561.01	0.00691592078390264\\
562.01	0.00655121878740571\\
563.01	0.0061701041694901\\
564.01	0.00577130240260382\\
565.01	0.00535294336937349\\
566.01	0.00491261245564841\\
567.01	0.00455026635944565\\
568.01	0.00435240083757039\\
569.01	0.00415256776986028\\
570.01	0.00395251099631614\\
571.01	0.00375468014392657\\
572.01	0.00356247354745911\\
573.01	0.00337782507188291\\
574.01	0.00319314443084913\\
575.01	0.00300871330112231\\
576.01	0.00282572334554877\\
577.01	0.00264561290078165\\
578.01	0.00247010123583664\\
579.01	0.00230122116256501\\
580.01	0.0021413447694239\\
581.01	0.00199319231655826\\
582.01	0.00185835342257616\\
583.01	0.00172871168230353\\
584.01	0.0016014443999877\\
585.01	0.0014759666797958\\
586.01	0.00135209058275517\\
587.01	0.00123011324476261\\
588.01	0.00111029688784535\\
589.01	0.00099285380763359\\
590.01	0.000877986990115272\\
591.01	0.000766096817003731\\
592.01	0.000657656957495178\\
593.01	0.000553079177615124\\
594.01	0.000452675589256735\\
595.01	0.000356615005476036\\
596.01	0.000264873949180146\\
597.01	0.000177185395016351\\
598.01	9.36647801022521e-05\\
599.01	2.93891165479357e-05\\
599.02	2.88733427738237e-05\\
599.03	2.83607887843138e-05\\
599.04	2.78514829045555e-05\\
599.05	2.73454537503409e-05\\
599.06	2.68427302310189e-05\\
599.07	2.6343341552432e-05\\
599.08	2.58473172198743e-05\\
599.09	2.53546870411028e-05\\
599.1	2.4865481129354e-05\\
599.11	2.43797299064024e-05\\
599.12	2.38974641056534e-05\\
599.13	2.34187147752501e-05\\
599.14	2.29435132812292e-05\\
599.15	2.24718913107037e-05\\
599.16	2.20038808750617e-05\\
599.17	2.15395143132284e-05\\
599.18	2.10788242949178e-05\\
599.19	2.06218438239709e-05\\
599.2	2.01686062416655e-05\\
599.21	1.97191452301106e-05\\
599.22	1.92734948156557e-05\\
599.23	1.88316893723148e-05\\
599.24	1.83937636252657e-05\\
599.25	1.79597526543396e-05\\
599.26	1.75296921197578e-05\\
599.27	1.71036209698078e-05\\
599.28	1.66815785538049e-05\\
599.29	1.62636046261155e-05\\
599.3	1.58497393501957e-05\\
599.31	1.54400233026864e-05\\
599.32	1.50344974775372e-05\\
599.33	1.46332032901937e-05\\
599.34	1.42361825817925e-05\\
599.35	1.38434776234316e-05\\
599.36	1.3455131120459e-05\\
599.37	1.30711862168214e-05\\
599.38	1.26916864994445e-05\\
599.39	1.23166760026671e-05\\
599.4	1.19461992127128e-05\\
599.41	1.1580301072206e-05\\
599.42	1.12190269847406e-05\\
599.43	1.08624228194933e-05\\
599.44	1.05105349158703e-05\\
599.45	1.01634100882209e-05\\
599.46	9.82109562895975e-06\\
599.47	9.48363931251028e-06\\
599.48	9.15108940013724e-06\\
599.49	8.82349464480583e-06\\
599.5	8.50090429611351e-06\\
599.51	8.18336810525129e-06\\
599.52	7.8709363300327e-06\\
599.53	7.56365973996094e-06\\
599.54	7.26158962135841e-06\\
599.55	6.9647777825501e-06\\
599.56	6.67327655908337e-06\\
599.57	6.38713881902925e-06\\
599.58	6.10641796831846e-06\\
599.59	5.83116795613466e-06\\
599.6	5.56144328037883e-06\\
599.61	5.2972989931753e-06\\
599.62	5.03879070643844e-06\\
599.63	4.78597459751748e-06\\
599.64	4.53890741486211e-06\\
599.65	4.29764648379564e-06\\
599.66	4.06224971230724e-06\\
599.67	3.83277559694133e-06\\
599.68	3.60928322872336e-06\\
599.69	3.39183229917245e-06\\
599.7	3.18048310636067e-06\\
599.71	2.9752965610471e-06\\
599.72	2.77633419288811e-06\\
599.73	2.58365815670146e-06\\
599.74	2.39733123880842e-06\\
599.75	2.21741686344014e-06\\
599.76	2.04397909923763e-06\\
599.77	1.8770826657917e-06\\
599.78	1.71679294028865e-06\\
599.79	1.56317596421672e-06\\
599.8	1.41629845015583e-06\\
599.81	1.27622778864009e-06\\
599.82	1.14303205511236e-06\\
599.83	1.01678001696026e-06\\
599.84	8.97541140626804e-07\\
599.85	7.8538559881644e-07\\
599.86	6.80384277782636e-07\\
599.87	5.82608784729932e-07\\
599.88	4.92131455266318e-07\\
599.89	4.09025360990911e-07\\
599.9	3.33364317157969e-07\\
599.91	2.65222890446712e-07\\
599.92	2.04676406836968e-07\\
599.93	1.51800959578494e-07\\
599.94	1.06673417288664e-07\\
599.95	6.93714321420985e-08\\
599.96	3.99734481886654e-08\\
599.97	1.85587097807638e-08\\
599.98	5.20727013245126e-09\\
599.99	0\\
600	0\\
};
\addplot [color=mycolor11,solid,forget plot]
  table[row sep=crcr]{%
0.01	0.01\\
1.01	0.01\\
2.01	0.01\\
3.01	0.01\\
4.01	0.01\\
5.01	0.01\\
6.01	0.01\\
7.01	0.01\\
8.01	0.01\\
9.01	0.01\\
10.01	0.01\\
11.01	0.01\\
12.01	0.01\\
13.01	0.01\\
14.01	0.01\\
15.01	0.01\\
16.01	0.01\\
17.01	0.01\\
18.01	0.01\\
19.01	0.01\\
20.01	0.01\\
21.01	0.01\\
22.01	0.01\\
23.01	0.01\\
24.01	0.01\\
25.01	0.01\\
26.01	0.01\\
27.01	0.01\\
28.01	0.01\\
29.01	0.01\\
30.01	0.01\\
31.01	0.01\\
32.01	0.01\\
33.01	0.01\\
34.01	0.01\\
35.01	0.01\\
36.01	0.01\\
37.01	0.01\\
38.01	0.01\\
39.01	0.01\\
40.01	0.01\\
41.01	0.01\\
42.01	0.01\\
43.01	0.01\\
44.01	0.01\\
45.01	0.01\\
46.01	0.01\\
47.01	0.01\\
48.01	0.01\\
49.01	0.01\\
50.01	0.01\\
51.01	0.01\\
52.01	0.01\\
53.01	0.01\\
54.01	0.01\\
55.01	0.01\\
56.01	0.01\\
57.01	0.01\\
58.01	0.01\\
59.01	0.01\\
60.01	0.01\\
61.01	0.01\\
62.01	0.01\\
63.01	0.01\\
64.01	0.01\\
65.01	0.01\\
66.01	0.01\\
67.01	0.01\\
68.01	0.01\\
69.01	0.01\\
70.01	0.01\\
71.01	0.01\\
72.01	0.01\\
73.01	0.01\\
74.01	0.01\\
75.01	0.01\\
76.01	0.01\\
77.01	0.01\\
78.01	0.01\\
79.01	0.01\\
80.01	0.01\\
81.01	0.01\\
82.01	0.01\\
83.01	0.01\\
84.01	0.01\\
85.01	0.01\\
86.01	0.01\\
87.01	0.01\\
88.01	0.01\\
89.01	0.01\\
90.01	0.01\\
91.01	0.01\\
92.01	0.01\\
93.01	0.01\\
94.01	0.01\\
95.01	0.01\\
96.01	0.01\\
97.01	0.01\\
98.01	0.01\\
99.01	0.01\\
100.01	0.01\\
101.01	0.01\\
102.01	0.01\\
103.01	0.01\\
104.01	0.01\\
105.01	0.01\\
106.01	0.01\\
107.01	0.01\\
108.01	0.01\\
109.01	0.01\\
110.01	0.01\\
111.01	0.01\\
112.01	0.01\\
113.01	0.01\\
114.01	0.01\\
115.01	0.01\\
116.01	0.01\\
117.01	0.01\\
118.01	0.01\\
119.01	0.01\\
120.01	0.01\\
121.01	0.01\\
122.01	0.01\\
123.01	0.01\\
124.01	0.01\\
125.01	0.01\\
126.01	0.01\\
127.01	0.01\\
128.01	0.01\\
129.01	0.01\\
130.01	0.01\\
131.01	0.01\\
132.01	0.01\\
133.01	0.01\\
134.01	0.01\\
135.01	0.01\\
136.01	0.01\\
137.01	0.01\\
138.01	0.01\\
139.01	0.01\\
140.01	0.01\\
141.01	0.01\\
142.01	0.01\\
143.01	0.01\\
144.01	0.01\\
145.01	0.01\\
146.01	0.01\\
147.01	0.01\\
148.01	0.01\\
149.01	0.01\\
150.01	0.01\\
151.01	0.01\\
152.01	0.01\\
153.01	0.01\\
154.01	0.01\\
155.01	0.01\\
156.01	0.01\\
157.01	0.01\\
158.01	0.01\\
159.01	0.01\\
160.01	0.01\\
161.01	0.01\\
162.01	0.01\\
163.01	0.01\\
164.01	0.01\\
165.01	0.01\\
166.01	0.01\\
167.01	0.01\\
168.01	0.01\\
169.01	0.01\\
170.01	0.01\\
171.01	0.01\\
172.01	0.01\\
173.01	0.01\\
174.01	0.01\\
175.01	0.01\\
176.01	0.01\\
177.01	0.01\\
178.01	0.01\\
179.01	0.01\\
180.01	0.01\\
181.01	0.01\\
182.01	0.01\\
183.01	0.01\\
184.01	0.01\\
185.01	0.01\\
186.01	0.01\\
187.01	0.01\\
188.01	0.01\\
189.01	0.01\\
190.01	0.01\\
191.01	0.01\\
192.01	0.01\\
193.01	0.01\\
194.01	0.01\\
195.01	0.01\\
196.01	0.01\\
197.01	0.01\\
198.01	0.01\\
199.01	0.01\\
200.01	0.01\\
201.01	0.01\\
202.01	0.01\\
203.01	0.01\\
204.01	0.01\\
205.01	0.01\\
206.01	0.01\\
207.01	0.01\\
208.01	0.01\\
209.01	0.01\\
210.01	0.01\\
211.01	0.01\\
212.01	0.01\\
213.01	0.01\\
214.01	0.01\\
215.01	0.01\\
216.01	0.01\\
217.01	0.01\\
218.01	0.01\\
219.01	0.01\\
220.01	0.01\\
221.01	0.01\\
222.01	0.01\\
223.01	0.01\\
224.01	0.01\\
225.01	0.01\\
226.01	0.01\\
227.01	0.01\\
228.01	0.01\\
229.01	0.01\\
230.01	0.01\\
231.01	0.01\\
232.01	0.01\\
233.01	0.01\\
234.01	0.01\\
235.01	0.01\\
236.01	0.01\\
237.01	0.01\\
238.01	0.01\\
239.01	0.01\\
240.01	0.01\\
241.01	0.01\\
242.01	0.01\\
243.01	0.01\\
244.01	0.01\\
245.01	0.01\\
246.01	0.01\\
247.01	0.01\\
248.01	0.01\\
249.01	0.01\\
250.01	0.01\\
251.01	0.01\\
252.01	0.01\\
253.01	0.01\\
254.01	0.01\\
255.01	0.01\\
256.01	0.01\\
257.01	0.01\\
258.01	0.01\\
259.01	0.01\\
260.01	0.01\\
261.01	0.01\\
262.01	0.01\\
263.01	0.01\\
264.01	0.01\\
265.01	0.01\\
266.01	0.01\\
267.01	0.01\\
268.01	0.01\\
269.01	0.01\\
270.01	0.01\\
271.01	0.01\\
272.01	0.01\\
273.01	0.01\\
274.01	0.01\\
275.01	0.01\\
276.01	0.01\\
277.01	0.01\\
278.01	0.01\\
279.01	0.01\\
280.01	0.01\\
281.01	0.01\\
282.01	0.01\\
283.01	0.01\\
284.01	0.01\\
285.01	0.01\\
286.01	0.01\\
287.01	0.01\\
288.01	0.01\\
289.01	0.01\\
290.01	0.01\\
291.01	0.01\\
292.01	0.01\\
293.01	0.01\\
294.01	0.01\\
295.01	0.01\\
296.01	0.01\\
297.01	0.01\\
298.01	0.01\\
299.01	0.01\\
300.01	0.01\\
301.01	0.01\\
302.01	0.01\\
303.01	0.01\\
304.01	0.01\\
305.01	0.01\\
306.01	0.01\\
307.01	0.01\\
308.01	0.01\\
309.01	0.01\\
310.01	0.01\\
311.01	0.01\\
312.01	0.01\\
313.01	0.01\\
314.01	0.01\\
315.01	0.01\\
316.01	0.01\\
317.01	0.01\\
318.01	0.01\\
319.01	0.01\\
320.01	0.01\\
321.01	0.01\\
322.01	0.01\\
323.01	0.01\\
324.01	0.01\\
325.01	0.01\\
326.01	0.01\\
327.01	0.01\\
328.01	0.01\\
329.01	0.01\\
330.01	0.01\\
331.01	0.01\\
332.01	0.01\\
333.01	0.01\\
334.01	0.01\\
335.01	0.01\\
336.01	0.01\\
337.01	0.01\\
338.01	0.01\\
339.01	0.01\\
340.01	0.01\\
341.01	0.01\\
342.01	0.01\\
343.01	0.01\\
344.01	0.01\\
345.01	0.01\\
346.01	0.01\\
347.01	0.01\\
348.01	0.01\\
349.01	0.01\\
350.01	0.01\\
351.01	0.01\\
352.01	0.01\\
353.01	0.01\\
354.01	0.01\\
355.01	0.01\\
356.01	0.01\\
357.01	0.01\\
358.01	0.01\\
359.01	0.01\\
360.01	0.01\\
361.01	0.01\\
362.01	0.01\\
363.01	0.01\\
364.01	0.01\\
365.01	0.01\\
366.01	0.01\\
367.01	0.01\\
368.01	0.01\\
369.01	0.01\\
370.01	0.01\\
371.01	0.01\\
372.01	0.01\\
373.01	0.01\\
374.01	0.01\\
375.01	0.01\\
376.01	0.01\\
377.01	0.01\\
378.01	0.01\\
379.01	0.01\\
380.01	0.01\\
381.01	0.01\\
382.01	0.01\\
383.01	0.01\\
384.01	0.01\\
385.01	0.01\\
386.01	0.01\\
387.01	0.01\\
388.01	0.01\\
389.01	0.01\\
390.01	0.01\\
391.01	0.01\\
392.01	0.01\\
393.01	0.01\\
394.01	0.01\\
395.01	0.01\\
396.01	0.01\\
397.01	0.01\\
398.01	0.01\\
399.01	0.01\\
400.01	0.01\\
401.01	0.01\\
402.01	0.01\\
403.01	0.01\\
404.01	0.01\\
405.01	0.01\\
406.01	0.01\\
407.01	0.01\\
408.01	0.01\\
409.01	0.01\\
410.01	0.01\\
411.01	0.01\\
412.01	0.01\\
413.01	0.01\\
414.01	0.01\\
415.01	0.01\\
416.01	0.01\\
417.01	0.01\\
418.01	0.01\\
419.01	0.01\\
420.01	0.01\\
421.01	0.01\\
422.01	0.01\\
423.01	0.01\\
424.01	0.01\\
425.01	0.01\\
426.01	0.01\\
427.01	0.01\\
428.01	0.01\\
429.01	0.01\\
430.01	0.01\\
431.01	0.01\\
432.01	0.01\\
433.01	0.01\\
434.01	0.01\\
435.01	0.01\\
436.01	0.01\\
437.01	0.01\\
438.01	0.01\\
439.01	0.01\\
440.01	0.01\\
441.01	0.01\\
442.01	0.01\\
443.01	0.01\\
444.01	0.01\\
445.01	0.01\\
446.01	0.01\\
447.01	0.01\\
448.01	0.01\\
449.01	0.01\\
450.01	0.01\\
451.01	0.01\\
452.01	0.01\\
453.01	0.01\\
454.01	0.01\\
455.01	0.01\\
456.01	0.01\\
457.01	0.01\\
458.01	0.01\\
459.01	0.01\\
460.01	0.01\\
461.01	0.01\\
462.01	0.01\\
463.01	0.01\\
464.01	0.01\\
465.01	0.01\\
466.01	0.01\\
467.01	0.01\\
468.01	0.01\\
469.01	0.01\\
470.01	0.01\\
471.01	0.01\\
472.01	0.01\\
473.01	0.01\\
474.01	0.01\\
475.01	0.01\\
476.01	0.01\\
477.01	0.01\\
478.01	0.01\\
479.01	0.01\\
480.01	0.01\\
481.01	0.01\\
482.01	0.01\\
483.01	0.01\\
484.01	0.01\\
485.01	0.01\\
486.01	0.01\\
487.01	0.01\\
488.01	0.01\\
489.01	0.01\\
490.01	0.01\\
491.01	0.01\\
492.01	0.01\\
493.01	0.01\\
494.01	0.01\\
495.01	0.01\\
496.01	0.01\\
497.01	0.01\\
498.01	0.01\\
499.01	0.01\\
500.01	0.01\\
501.01	0.01\\
502.01	0.01\\
503.01	0.01\\
504.01	0.01\\
505.01	0.01\\
506.01	0.01\\
507.01	0.01\\
508.01	0.01\\
509.01	0.01\\
510.01	0.01\\
511.01	0.01\\
512.01	0.01\\
513.01	0.01\\
514.01	0.01\\
515.01	0.01\\
516.01	0.01\\
517.01	0.01\\
518.01	0.01\\
519.01	0.01\\
520.01	0.01\\
521.01	0.01\\
522.01	0.01\\
523.01	0.01\\
524.01	0.01\\
525.01	0.01\\
526.01	0.01\\
527.01	0.01\\
528.01	0.01\\
529.01	0.01\\
530.01	0.01\\
531.01	0.01\\
532.01	0.01\\
533.01	0.01\\
534.01	0.01\\
535.01	0.01\\
536.01	0.01\\
537.01	0.01\\
538.01	0.01\\
539.01	0.01\\
540.01	0.01\\
541.01	0.01\\
542.01	0.01\\
543.01	0.01\\
544.01	0.01\\
545.01	0.01\\
546.01	0.01\\
547.01	0.01\\
548.01	0.01\\
549.01	0.01\\
550.01	0.01\\
551.01	0.01\\
552.01	0.01\\
553.01	0.01\\
554.01	0.01\\
555.01	0.01\\
556.01	0.01\\
557.01	0.01\\
558.01	0.01\\
559.01	0.01\\
560.01	0.01\\
561.01	0.01\\
562.01	0.01\\
563.01	0.01\\
564.01	0.01\\
565.01	0.01\\
566.01	0.01\\
567.01	0.00989781364243563\\
568.01	0.0096109957314396\\
569.01	0.00930678972726892\\
570.01	0.00898281944686265\\
571.01	0.00863625634746471\\
572.01	0.0082637197398952\\
573.01	0.00786385895341286\\
574.01	0.00744596504207943\\
575.01	0.00701014204382437\\
576.01	0.00655491977496119\\
577.01	0.00607869128502517\\
578.01	0.00557970906787935\\
579.01	0.00505609315699457\\
580.01	0.00450587264837774\\
581.01	0.00392712360941451\\
582.01	0.00344097622170438\\
583.01	0.00320184843714926\\
584.01	0.00297373171794813\\
585.01	0.00275715000119383\\
586.01	0.00254044740799736\\
587.01	0.00232322729506633\\
588.01	0.00210635469391665\\
589.01	0.00189073040228258\\
590.01	0.00167646442262719\\
591.01	0.00146289296784564\\
592.01	0.00125144610456633\\
593.01	0.00104393128886474\\
594.01	0.000842494517544453\\
595.01	0.000649663713740501\\
596.01	0.000468392415796496\\
597.01	0.000302100291734208\\
598.01	0.000154705486090628\\
599.01	4.48417650300119e-05\\
599.02	4.40230021324511e-05\\
599.03	4.3210269553345e-05\\
599.04	4.24036022396417e-05\\
599.05	4.1603035432599e-05\\
599.06	4.08086046704923e-05\\
599.07	4.00203457913435e-05\\
599.08	3.92382949356684e-05\\
599.09	3.84624885492527e-05\\
599.1	3.76929633859428e-05\\
599.11	3.69297565104582e-05\\
599.12	3.61729053012273e-05\\
599.13	3.54224474532448e-05\\
599.14	3.46784209809495e-05\\
599.15	3.39408642211197e-05\\
599.16	3.32098158357941e-05\\
599.17	3.24853148152126e-05\\
599.18	3.1767400480779e-05\\
599.19	3.1056112488036e-05\\
599.2	3.03514908296752e-05\\
599.21	2.96535758385584e-05\\
599.22	2.89624081907525e-05\\
599.23	2.8278028908606e-05\\
599.24	2.76004793638204e-05\\
599.25	2.69298012805595e-05\\
599.26	2.62660367384721e-05\\
599.27	2.5609228174503e-05\\
599.28	2.49594183860512e-05\\
599.29	2.43166505341143e-05\\
599.3	2.36809681464778e-05\\
599.31	2.30524151209142e-05\\
599.32	2.24310357283923e-05\\
599.33	2.18168746163074e-05\\
599.34	2.12099768117386e-05\\
599.35	2.06103877247021e-05\\
599.36	2.00181531514448e-05\\
599.37	1.9433319277732e-05\\
599.38	1.88559326821588e-05\\
599.39	1.82860403394793e-05\\
599.4	1.77236896239403e-05\\
599.41	1.71689283126365e-05\\
599.42	1.66218045888709e-05\\
599.43	1.60823670455352e-05\\
599.44	1.55506646884879e-05\\
599.45	1.50267469399591e-05\\
599.46	1.45106669384547e-05\\
599.47	1.40024799546081e-05\\
599.48	1.35022417351478e-05\\
599.49	1.3010008506769e-05\\
599.5	1.25258369800228e-05\\
599.51	1.20497843531988e-05\\
599.52	1.15819083162261e-05\\
599.53	1.11222670545735e-05\\
599.54	1.06709192531571e-05\\
599.55	1.02279241002319e-05\\
599.56	9.79334129132085e-06\\
599.57	9.36723103309167e-06\\
599.58	8.94965404728103e-06\\
599.59	8.54067157457679e-06\\
599.6	8.14034537850901e-06\\
599.61	7.7487377493253e-06\\
599.62	7.36591150786621e-06\\
599.63	6.99193000940589e-06\\
599.64	6.62685714749797e-06\\
599.65	6.27075735779026e-06\\
599.66	5.92369562180985e-06\\
599.67	5.58573747075006e-06\\
599.68	5.25694898919141e-06\\
599.69	4.93739681882779e-06\\
599.7	4.62714816213197e-06\\
599.71	4.32627078599673e-06\\
599.72	4.03483302531885e-06\\
599.73	3.75290378655697e-06\\
599.74	3.480552551215e-06\\
599.75	3.21784937928893e-06\\
599.76	2.96486491263744e-06\\
599.77	2.72167037830041e-06\\
599.78	2.48833759172902e-06\\
599.79	2.26493895996022e-06\\
599.8	2.05154748467509e-06\\
599.81	1.84823676521034e-06\\
599.82	1.65508100143097e-06\\
599.83	1.47215499651972e-06\\
599.84	1.29953415965375e-06\\
599.85	1.13729450855297e-06\\
599.86	9.8551267190862e-07\\
599.87	8.44265891657495e-07\\
599.88	7.13632025122618e-07\\
599.89	5.93689546994278e-07\\
599.9	4.84517551132407e-07\\
599.91	3.86195752185084e-07\\
599.92	2.98804487031817e-07\\
599.93	2.22424715994388e-07\\
599.94	1.57138023849923e-07\\
599.95	1.0302662059071e-07\\
599.96	6.0173341939404e-08\\
599.97	2.86616496005671e-08\\
599.98	8.57563121035854e-09\\
599.99	0\\
600	0\\
};
\addplot [color=mycolor12,solid,forget plot]
  table[row sep=crcr]{%
0.01	0.01\\
1.01	0.01\\
2.01	0.01\\
3.01	0.01\\
4.01	0.01\\
5.01	0.01\\
6.01	0.01\\
7.01	0.01\\
8.01	0.01\\
9.01	0.01\\
10.01	0.01\\
11.01	0.01\\
12.01	0.01\\
13.01	0.01\\
14.01	0.01\\
15.01	0.01\\
16.01	0.01\\
17.01	0.01\\
18.01	0.01\\
19.01	0.01\\
20.01	0.01\\
21.01	0.01\\
22.01	0.01\\
23.01	0.01\\
24.01	0.01\\
25.01	0.01\\
26.01	0.01\\
27.01	0.01\\
28.01	0.01\\
29.01	0.01\\
30.01	0.01\\
31.01	0.01\\
32.01	0.01\\
33.01	0.01\\
34.01	0.01\\
35.01	0.01\\
36.01	0.01\\
37.01	0.01\\
38.01	0.01\\
39.01	0.01\\
40.01	0.01\\
41.01	0.01\\
42.01	0.01\\
43.01	0.01\\
44.01	0.01\\
45.01	0.01\\
46.01	0.01\\
47.01	0.01\\
48.01	0.01\\
49.01	0.01\\
50.01	0.01\\
51.01	0.01\\
52.01	0.01\\
53.01	0.01\\
54.01	0.01\\
55.01	0.01\\
56.01	0.01\\
57.01	0.01\\
58.01	0.01\\
59.01	0.01\\
60.01	0.01\\
61.01	0.01\\
62.01	0.01\\
63.01	0.01\\
64.01	0.01\\
65.01	0.01\\
66.01	0.01\\
67.01	0.01\\
68.01	0.01\\
69.01	0.01\\
70.01	0.01\\
71.01	0.01\\
72.01	0.01\\
73.01	0.01\\
74.01	0.01\\
75.01	0.01\\
76.01	0.01\\
77.01	0.01\\
78.01	0.01\\
79.01	0.01\\
80.01	0.01\\
81.01	0.01\\
82.01	0.01\\
83.01	0.01\\
84.01	0.01\\
85.01	0.01\\
86.01	0.01\\
87.01	0.01\\
88.01	0.01\\
89.01	0.01\\
90.01	0.01\\
91.01	0.01\\
92.01	0.01\\
93.01	0.01\\
94.01	0.01\\
95.01	0.01\\
96.01	0.01\\
97.01	0.01\\
98.01	0.01\\
99.01	0.01\\
100.01	0.01\\
101.01	0.01\\
102.01	0.01\\
103.01	0.01\\
104.01	0.01\\
105.01	0.01\\
106.01	0.01\\
107.01	0.01\\
108.01	0.01\\
109.01	0.01\\
110.01	0.01\\
111.01	0.01\\
112.01	0.01\\
113.01	0.01\\
114.01	0.01\\
115.01	0.01\\
116.01	0.01\\
117.01	0.01\\
118.01	0.01\\
119.01	0.01\\
120.01	0.01\\
121.01	0.01\\
122.01	0.01\\
123.01	0.01\\
124.01	0.01\\
125.01	0.01\\
126.01	0.01\\
127.01	0.01\\
128.01	0.01\\
129.01	0.01\\
130.01	0.01\\
131.01	0.01\\
132.01	0.01\\
133.01	0.01\\
134.01	0.01\\
135.01	0.01\\
136.01	0.01\\
137.01	0.01\\
138.01	0.01\\
139.01	0.01\\
140.01	0.01\\
141.01	0.01\\
142.01	0.01\\
143.01	0.01\\
144.01	0.01\\
145.01	0.01\\
146.01	0.01\\
147.01	0.01\\
148.01	0.01\\
149.01	0.01\\
150.01	0.01\\
151.01	0.01\\
152.01	0.01\\
153.01	0.01\\
154.01	0.01\\
155.01	0.01\\
156.01	0.01\\
157.01	0.01\\
158.01	0.01\\
159.01	0.01\\
160.01	0.01\\
161.01	0.01\\
162.01	0.01\\
163.01	0.01\\
164.01	0.01\\
165.01	0.01\\
166.01	0.01\\
167.01	0.01\\
168.01	0.01\\
169.01	0.01\\
170.01	0.01\\
171.01	0.01\\
172.01	0.01\\
173.01	0.01\\
174.01	0.01\\
175.01	0.01\\
176.01	0.01\\
177.01	0.01\\
178.01	0.01\\
179.01	0.01\\
180.01	0.01\\
181.01	0.01\\
182.01	0.01\\
183.01	0.01\\
184.01	0.01\\
185.01	0.01\\
186.01	0.01\\
187.01	0.01\\
188.01	0.01\\
189.01	0.01\\
190.01	0.01\\
191.01	0.01\\
192.01	0.01\\
193.01	0.01\\
194.01	0.01\\
195.01	0.01\\
196.01	0.01\\
197.01	0.01\\
198.01	0.01\\
199.01	0.01\\
200.01	0.01\\
201.01	0.01\\
202.01	0.01\\
203.01	0.01\\
204.01	0.01\\
205.01	0.01\\
206.01	0.01\\
207.01	0.01\\
208.01	0.01\\
209.01	0.01\\
210.01	0.01\\
211.01	0.01\\
212.01	0.01\\
213.01	0.01\\
214.01	0.01\\
215.01	0.01\\
216.01	0.01\\
217.01	0.01\\
218.01	0.01\\
219.01	0.01\\
220.01	0.01\\
221.01	0.01\\
222.01	0.01\\
223.01	0.01\\
224.01	0.01\\
225.01	0.01\\
226.01	0.01\\
227.01	0.01\\
228.01	0.01\\
229.01	0.01\\
230.01	0.01\\
231.01	0.01\\
232.01	0.01\\
233.01	0.01\\
234.01	0.01\\
235.01	0.01\\
236.01	0.01\\
237.01	0.01\\
238.01	0.01\\
239.01	0.01\\
240.01	0.01\\
241.01	0.01\\
242.01	0.01\\
243.01	0.01\\
244.01	0.01\\
245.01	0.01\\
246.01	0.01\\
247.01	0.01\\
248.01	0.01\\
249.01	0.01\\
250.01	0.01\\
251.01	0.01\\
252.01	0.01\\
253.01	0.01\\
254.01	0.01\\
255.01	0.01\\
256.01	0.01\\
257.01	0.01\\
258.01	0.01\\
259.01	0.01\\
260.01	0.01\\
261.01	0.01\\
262.01	0.01\\
263.01	0.01\\
264.01	0.01\\
265.01	0.01\\
266.01	0.01\\
267.01	0.01\\
268.01	0.01\\
269.01	0.01\\
270.01	0.01\\
271.01	0.01\\
272.01	0.01\\
273.01	0.01\\
274.01	0.01\\
275.01	0.01\\
276.01	0.01\\
277.01	0.01\\
278.01	0.01\\
279.01	0.01\\
280.01	0.01\\
281.01	0.01\\
282.01	0.01\\
283.01	0.01\\
284.01	0.01\\
285.01	0.01\\
286.01	0.01\\
287.01	0.01\\
288.01	0.01\\
289.01	0.01\\
290.01	0.01\\
291.01	0.01\\
292.01	0.01\\
293.01	0.01\\
294.01	0.01\\
295.01	0.01\\
296.01	0.01\\
297.01	0.01\\
298.01	0.01\\
299.01	0.01\\
300.01	0.01\\
301.01	0.01\\
302.01	0.01\\
303.01	0.01\\
304.01	0.01\\
305.01	0.01\\
306.01	0.01\\
307.01	0.01\\
308.01	0.01\\
309.01	0.01\\
310.01	0.01\\
311.01	0.01\\
312.01	0.01\\
313.01	0.01\\
314.01	0.01\\
315.01	0.01\\
316.01	0.01\\
317.01	0.01\\
318.01	0.01\\
319.01	0.01\\
320.01	0.01\\
321.01	0.01\\
322.01	0.01\\
323.01	0.01\\
324.01	0.01\\
325.01	0.01\\
326.01	0.01\\
327.01	0.01\\
328.01	0.01\\
329.01	0.01\\
330.01	0.01\\
331.01	0.01\\
332.01	0.01\\
333.01	0.01\\
334.01	0.01\\
335.01	0.01\\
336.01	0.01\\
337.01	0.01\\
338.01	0.01\\
339.01	0.01\\
340.01	0.01\\
341.01	0.01\\
342.01	0.01\\
343.01	0.01\\
344.01	0.01\\
345.01	0.01\\
346.01	0.01\\
347.01	0.01\\
348.01	0.01\\
349.01	0.01\\
350.01	0.01\\
351.01	0.01\\
352.01	0.01\\
353.01	0.01\\
354.01	0.01\\
355.01	0.01\\
356.01	0.01\\
357.01	0.01\\
358.01	0.01\\
359.01	0.01\\
360.01	0.01\\
361.01	0.01\\
362.01	0.01\\
363.01	0.01\\
364.01	0.01\\
365.01	0.01\\
366.01	0.01\\
367.01	0.01\\
368.01	0.01\\
369.01	0.01\\
370.01	0.01\\
371.01	0.01\\
372.01	0.01\\
373.01	0.01\\
374.01	0.01\\
375.01	0.01\\
376.01	0.01\\
377.01	0.01\\
378.01	0.01\\
379.01	0.01\\
380.01	0.01\\
381.01	0.01\\
382.01	0.01\\
383.01	0.01\\
384.01	0.01\\
385.01	0.01\\
386.01	0.01\\
387.01	0.01\\
388.01	0.01\\
389.01	0.01\\
390.01	0.01\\
391.01	0.01\\
392.01	0.01\\
393.01	0.01\\
394.01	0.01\\
395.01	0.01\\
396.01	0.01\\
397.01	0.01\\
398.01	0.01\\
399.01	0.01\\
400.01	0.01\\
401.01	0.01\\
402.01	0.01\\
403.01	0.01\\
404.01	0.01\\
405.01	0.01\\
406.01	0.01\\
407.01	0.01\\
408.01	0.01\\
409.01	0.01\\
410.01	0.01\\
411.01	0.01\\
412.01	0.01\\
413.01	0.01\\
414.01	0.01\\
415.01	0.01\\
416.01	0.01\\
417.01	0.01\\
418.01	0.01\\
419.01	0.01\\
420.01	0.01\\
421.01	0.01\\
422.01	0.01\\
423.01	0.01\\
424.01	0.01\\
425.01	0.01\\
426.01	0.01\\
427.01	0.01\\
428.01	0.01\\
429.01	0.01\\
430.01	0.01\\
431.01	0.01\\
432.01	0.01\\
433.01	0.01\\
434.01	0.01\\
435.01	0.01\\
436.01	0.01\\
437.01	0.01\\
438.01	0.01\\
439.01	0.01\\
440.01	0.01\\
441.01	0.01\\
442.01	0.01\\
443.01	0.01\\
444.01	0.01\\
445.01	0.01\\
446.01	0.01\\
447.01	0.01\\
448.01	0.01\\
449.01	0.01\\
450.01	0.01\\
451.01	0.01\\
452.01	0.01\\
453.01	0.01\\
454.01	0.01\\
455.01	0.01\\
456.01	0.01\\
457.01	0.01\\
458.01	0.01\\
459.01	0.01\\
460.01	0.01\\
461.01	0.01\\
462.01	0.01\\
463.01	0.01\\
464.01	0.01\\
465.01	0.01\\
466.01	0.01\\
467.01	0.01\\
468.01	0.01\\
469.01	0.01\\
470.01	0.01\\
471.01	0.01\\
472.01	0.01\\
473.01	0.01\\
474.01	0.01\\
475.01	0.01\\
476.01	0.01\\
477.01	0.01\\
478.01	0.01\\
479.01	0.01\\
480.01	0.01\\
481.01	0.01\\
482.01	0.01\\
483.01	0.01\\
484.01	0.01\\
485.01	0.01\\
486.01	0.01\\
487.01	0.01\\
488.01	0.01\\
489.01	0.01\\
490.01	0.01\\
491.01	0.01\\
492.01	0.01\\
493.01	0.01\\
494.01	0.01\\
495.01	0.01\\
496.01	0.01\\
497.01	0.01\\
498.01	0.01\\
499.01	0.01\\
500.01	0.01\\
501.01	0.01\\
502.01	0.01\\
503.01	0.01\\
504.01	0.01\\
505.01	0.01\\
506.01	0.01\\
507.01	0.01\\
508.01	0.01\\
509.01	0.01\\
510.01	0.01\\
511.01	0.01\\
512.01	0.01\\
513.01	0.01\\
514.01	0.01\\
515.01	0.01\\
516.01	0.01\\
517.01	0.01\\
518.01	0.01\\
519.01	0.01\\
520.01	0.01\\
521.01	0.01\\
522.01	0.01\\
523.01	0.01\\
524.01	0.01\\
525.01	0.01\\
526.01	0.01\\
527.01	0.01\\
528.01	0.01\\
529.01	0.01\\
530.01	0.01\\
531.01	0.01\\
532.01	0.01\\
533.01	0.01\\
534.01	0.01\\
535.01	0.01\\
536.01	0.01\\
537.01	0.01\\
538.01	0.01\\
539.01	0.01\\
540.01	0.01\\
541.01	0.01\\
542.01	0.01\\
543.01	0.01\\
544.01	0.01\\
545.01	0.01\\
546.01	0.01\\
547.01	0.01\\
548.01	0.01\\
549.01	0.01\\
550.01	0.01\\
551.01	0.01\\
552.01	0.01\\
553.01	0.01\\
554.01	0.01\\
555.01	0.01\\
556.01	0.01\\
557.01	0.01\\
558.01	0.01\\
559.01	0.01\\
560.01	0.01\\
561.01	0.01\\
562.01	0.01\\
563.01	0.01\\
564.01	0.01\\
565.01	0.01\\
566.01	0.01\\
567.01	0.01\\
568.01	0.01\\
569.01	0.01\\
570.01	0.01\\
571.01	0.01\\
572.01	0.01\\
573.01	0.01\\
574.01	0.01\\
575.01	0.01\\
576.01	0.01\\
577.01	0.01\\
578.01	0.01\\
579.01	0.01\\
580.01	0.01\\
581.01	0.01\\
582.01	0.00987863839394233\\
583.01	0.00949082414842157\\
584.01	0.00907610149477576\\
585.01	0.00863516110198265\\
586.01	0.00818029576360015\\
587.01	0.00771162108760121\\
588.01	0.00722758388451855\\
589.01	0.00672698722609763\\
590.01	0.00620999025475\\
591.01	0.0056775054428497\\
592.01	0.00512831674646623\\
593.01	0.0045611266017166\\
594.01	0.00397455252111775\\
595.01	0.00336704583775623\\
596.01	0.00273686338392287\\
597.01	0.00208203186321775\\
598.01	0.00140030197562107\\
599.01	0.000693147751672453\\
599.02	0.000686044700991121\\
599.03	0.000678942362288212\\
599.04	0.000671840766349614\\
599.05	0.000664739944319878\\
599.06	0.000657639927706705\\
599.07	0.000650540748385492\\
599.08	0.000643442438603979\\
599.09	0.000636345030986921\\
599.1	0.000629248558540869\\
599.11	0.000622153054659002\\
599.12	0.00061505855312604\\
599.13	0.000607965088123233\\
599.14	0.000600872694233425\\
599.15	0.000593781406446192\\
599.16	0.000586691260163077\\
599.17	0.000579602291202875\\
599.18	0.000572514535807034\\
599.19	0.00056542803064512\\
599.2	0.000558342812820379\\
599.21	0.000551258919875373\\
599.22	0.000544176389797724\\
599.23	0.000537095261025937\\
599.24	0.000530015572455318\\
599.25	0.00052293736344399\\
599.26	0.000515860673819007\\
599.27	0.000508785543882569\\
599.28	0.000501712014418322\\
599.29	0.000494640126697794\\
599.3	0.000487569922486903\\
599.31	0.000480501444052593\\
599.32	0.000473434734169572\\
599.33	0.000466369836127174\\
599.34	0.000459306793736311\\
599.35	0.000452245651336575\\
599.36	0.000445186453803434\\
599.37	0.000438129246555563\\
599.38	0.000431074075562301\\
599.39	0.000424020987351227\\
599.4	0.000416970029015879\\
599.41	0.000409921248223602\\
599.42	0.000402874693223528\\
599.43	0.000395830412854702\\
599.44	0.000388788456554356\\
599.45	0.000381748874366322\\
599.46	0.000374711716948105\\
599.47	0.000367677035578842\\
599.48	0.000360644882168144\\
599.49	0.000353615309265124\\
599.5	0.000346588370067558\\
599.51	0.000339564118431257\\
599.52	0.00033254260887957\\
599.53	0.000325523896613097\\
599.54	0.000318508037519561\\
599.55	0.000311495088183889\\
599.56	0.000304485105898459\\
599.57	0.00029747814867357\\
599.58	0.000290474275248089\\
599.59	0.000283473545100324\\
599.6	0.000276476018459097\\
599.61	0.00026948175631504\\
599.62	0.000262490820432117\\
599.63	0.000255503273359365\\
599.64	0.000248519178442886\\
599.65	0.000241538599838066\\
599.66	0.000234561602522058\\
599.67	0.000227588252306508\\
599.68	0.000220618615850559\\
599.69	0.0002136527606741\\
599.7	0.000206690755171332\\
599.71	0.00019973266862458\\
599.72	0.000192778571218435\\
599.73	0.000185828534054166\\
599.74	0.000178882629164484\\
599.75	0.000171940929528592\\
599.76	0.000165003509087581\\
599.77	0.000158070442760182\\
599.78	0.000151141806458848\\
599.79	0.000144217677106206\\
599.8	0.000137298132651904\\
599.81	0.00013038325208981\\
599.82	0.000123473115475644\\
599.83	0.000116567803945002\\
599.84	0.000109667399731816\\
599.85	0.000102771986187248\\
599.86	9.58816477990444e-05\\
599.87	8.89964702113586e-05\\
599.88	8.21165402450641e-05\\
599.89	7.52419459185572e-05\\
599.9	6.83727764690997e-05\\
599.91	6.15091223746932e-05\\
599.92	5.46510753764995e-05\\
599.93	4.77987285018664e-05\\
599.94	4.09521760879206e-05\\
599.95	3.4111513805812e-05\\
599.96	2.72768386855859e-05\\
599.97	2.04482491417222e-05\\
599.98	1.36258449993844e-05\\
599.99	6.80972752137125e-06\\
600	0\\
};
\addplot [color=mycolor13,solid,forget plot]
  table[row sep=crcr]{%
0.01	0\\
1.01	0\\
2.01	0\\
3.01	0\\
4.01	0\\
5.01	0\\
6.01	0\\
7.01	0\\
8.01	0\\
9.01	0\\
10.01	0\\
11.01	0\\
12.01	0\\
13.01	0\\
14.01	0\\
15.01	0\\
16.01	0\\
17.01	0\\
18.01	0\\
19.01	0\\
20.01	0\\
21.01	0\\
22.01	0\\
23.01	0\\
24.01	0\\
25.01	0\\
26.01	0\\
27.01	0\\
28.01	0\\
29.01	0\\
30.01	0\\
31.01	0\\
32.01	0\\
33.01	0\\
34.01	0\\
35.01	0\\
36.01	0\\
37.01	0\\
38.01	0\\
39.01	0\\
40.01	0\\
41.01	0\\
42.01	0\\
43.01	0\\
44.01	0\\
45.01	0\\
46.01	0\\
47.01	0\\
48.01	0\\
49.01	0\\
50.01	0\\
51.01	0\\
52.01	0\\
53.01	0\\
54.01	0\\
55.01	0\\
56.01	0\\
57.01	0\\
58.01	0\\
59.01	0\\
60.01	0\\
61.01	0\\
62.01	0\\
63.01	0\\
64.01	0\\
65.01	0\\
66.01	0\\
67.01	0\\
68.01	0\\
69.01	0\\
70.01	0\\
71.01	0\\
72.01	0\\
73.01	0\\
74.01	0\\
75.01	0\\
76.01	0\\
77.01	0\\
78.01	0\\
79.01	0\\
80.01	0\\
81.01	0\\
82.01	0\\
83.01	0\\
84.01	0\\
85.01	0\\
86.01	0\\
87.01	0\\
88.01	0\\
89.01	0\\
90.01	0\\
91.01	0\\
92.01	0\\
93.01	0\\
94.01	0\\
95.01	0\\
96.01	0\\
97.01	0\\
98.01	0\\
99.01	0\\
100.01	0\\
101.01	0\\
102.01	0\\
103.01	0\\
104.01	0\\
105.01	0\\
106.01	0\\
107.01	0\\
108.01	0\\
109.01	0\\
110.01	0\\
111.01	0\\
112.01	0\\
113.01	0\\
114.01	0\\
115.01	0\\
116.01	0\\
117.01	0\\
118.01	0\\
119.01	0\\
120.01	0\\
121.01	0\\
122.01	0\\
123.01	0\\
124.01	0\\
125.01	0\\
126.01	0\\
127.01	0\\
128.01	0\\
129.01	0\\
130.01	0\\
131.01	0\\
132.01	0\\
133.01	0\\
134.01	0\\
135.01	0\\
136.01	0\\
137.01	0\\
138.01	0\\
139.01	0\\
140.01	0\\
141.01	0\\
142.01	0\\
143.01	0\\
144.01	0\\
145.01	0\\
146.01	0\\
147.01	0\\
148.01	0\\
149.01	0\\
150.01	0\\
151.01	0\\
152.01	0\\
153.01	0\\
154.01	0\\
155.01	0\\
156.01	0\\
157.01	0\\
158.01	0\\
159.01	0\\
160.01	0\\
161.01	0\\
162.01	0\\
163.01	0\\
164.01	0\\
165.01	0\\
166.01	0\\
167.01	0\\
168.01	0\\
169.01	0\\
170.01	0\\
171.01	0\\
172.01	0\\
173.01	0\\
174.01	0\\
175.01	0\\
176.01	0\\
177.01	0\\
178.01	0\\
179.01	0\\
180.01	0\\
181.01	0\\
182.01	0\\
183.01	0\\
184.01	0\\
185.01	0\\
186.01	0\\
187.01	0\\
188.01	0\\
189.01	0\\
190.01	0\\
191.01	0\\
192.01	0\\
193.01	0\\
194.01	0\\
195.01	0\\
196.01	0\\
197.01	0\\
198.01	0\\
199.01	0\\
200.01	0\\
201.01	0\\
202.01	0\\
203.01	0\\
204.01	0\\
205.01	0\\
206.01	0\\
207.01	0\\
208.01	0\\
209.01	0\\
210.01	0\\
211.01	0\\
212.01	0\\
213.01	0\\
214.01	0\\
215.01	0\\
216.01	0\\
217.01	0\\
218.01	0\\
219.01	0\\
220.01	0\\
221.01	0\\
222.01	0\\
223.01	0\\
224.01	0\\
225.01	0\\
226.01	0\\
227.01	0\\
228.01	0\\
229.01	0\\
230.01	0\\
231.01	0\\
232.01	0\\
233.01	0\\
234.01	0\\
235.01	0\\
236.01	0\\
237.01	0\\
238.01	0\\
239.01	0\\
240.01	0\\
241.01	0\\
242.01	0\\
243.01	0\\
244.01	0\\
245.01	0\\
246.01	0\\
247.01	0\\
248.01	0\\
249.01	0\\
250.01	0\\
251.01	0\\
252.01	0\\
253.01	0\\
254.01	0\\
255.01	0\\
256.01	0\\
257.01	0\\
258.01	0\\
259.01	0\\
260.01	0\\
261.01	0\\
262.01	0\\
263.01	0\\
264.01	0\\
265.01	0\\
266.01	0\\
267.01	0\\
268.01	0\\
269.01	0\\
270.01	0\\
271.01	0\\
272.01	0\\
273.01	0\\
274.01	0\\
275.01	0\\
276.01	0\\
277.01	0\\
278.01	0\\
279.01	0\\
280.01	0\\
281.01	0\\
282.01	0\\
283.01	0\\
284.01	0\\
285.01	0\\
286.01	0\\
287.01	0\\
288.01	0\\
289.01	0\\
290.01	0\\
291.01	0\\
292.01	0\\
293.01	0\\
294.01	0\\
295.01	0\\
296.01	0\\
297.01	0\\
298.01	0\\
299.01	0\\
300.01	0\\
301.01	0\\
302.01	0\\
303.01	0\\
304.01	0\\
305.01	0\\
306.01	0\\
307.01	0\\
308.01	0\\
309.01	0\\
310.01	0\\
311.01	0\\
312.01	0\\
313.01	0\\
314.01	0\\
315.01	0\\
316.01	0\\
317.01	0\\
318.01	0\\
319.01	0\\
320.01	0\\
321.01	0\\
322.01	0\\
323.01	0\\
324.01	0\\
325.01	0\\
326.01	0\\
327.01	0\\
328.01	0\\
329.01	0\\
330.01	0\\
331.01	0\\
332.01	0\\
333.01	0\\
334.01	0\\
335.01	0\\
336.01	0\\
337.01	0\\
338.01	0\\
339.01	0\\
340.01	0\\
341.01	0\\
342.01	0\\
343.01	0\\
344.01	0\\
345.01	0\\
346.01	0\\
347.01	0\\
348.01	0\\
349.01	0\\
350.01	0\\
351.01	0\\
352.01	0\\
353.01	0\\
354.01	0\\
355.01	0\\
356.01	0\\
357.01	0\\
358.01	0\\
359.01	0\\
360.01	0\\
361.01	0\\
362.01	0\\
363.01	0\\
364.01	0\\
365.01	0\\
366.01	0\\
367.01	0\\
368.01	0\\
369.01	0\\
370.01	0\\
371.01	0\\
372.01	0\\
373.01	0\\
374.01	0\\
375.01	0\\
376.01	0\\
377.01	0\\
378.01	0\\
379.01	0\\
380.01	0\\
381.01	0\\
382.01	0\\
383.01	0\\
384.01	0\\
385.01	0\\
386.01	0\\
387.01	0\\
388.01	0\\
389.01	0\\
390.01	0\\
391.01	0\\
392.01	0\\
393.01	0\\
394.01	0\\
395.01	0\\
396.01	0\\
397.01	0\\
398.01	0\\
399.01	0\\
400.01	0\\
401.01	0\\
402.01	0\\
403.01	0\\
404.01	0\\
405.01	0\\
406.01	0\\
407.01	0\\
408.01	0\\
409.01	0\\
410.01	0\\
411.01	0\\
412.01	0\\
413.01	0\\
414.01	0\\
415.01	0\\
416.01	0\\
417.01	0\\
418.01	0\\
419.01	0\\
420.01	0\\
421.01	0\\
422.01	0\\
423.01	0\\
424.01	0\\
425.01	0\\
426.01	0\\
427.01	0\\
428.01	0\\
429.01	0\\
430.01	0\\
431.01	0\\
432.01	0\\
433.01	0\\
434.01	0\\
435.01	0\\
436.01	0\\
437.01	0\\
438.01	0\\
439.01	0\\
440.01	0\\
441.01	0\\
442.01	0\\
443.01	0\\
444.01	0\\
445.01	0\\
446.01	0\\
447.01	0\\
448.01	0\\
449.01	0\\
450.01	0\\
451.01	0\\
452.01	0\\
453.01	0\\
454.01	0\\
455.01	0\\
456.01	0\\
457.01	0\\
458.01	0\\
459.01	0\\
460.01	0\\
461.01	0\\
462.01	0\\
463.01	0\\
464.01	0\\
465.01	0\\
466.01	0\\
467.01	0\\
468.01	0\\
469.01	0\\
470.01	0\\
471.01	0\\
472.01	0\\
473.01	0\\
474.01	0\\
475.01	0\\
476.01	0\\
477.01	0\\
478.01	0\\
479.01	0\\
480.01	0\\
481.01	0\\
482.01	0\\
483.01	0\\
484.01	0\\
485.01	0\\
486.01	0\\
487.01	0\\
488.01	0\\
489.01	0\\
490.01	0\\
491.01	0\\
492.01	0\\
493.01	0\\
494.01	0\\
495.01	0\\
496.01	0\\
497.01	0\\
498.01	0\\
499.01	0\\
500.01	0\\
501.01	0\\
502.01	0\\
503.01	0\\
504.01	0\\
505.01	0\\
506.01	0\\
507.01	0\\
508.01	0\\
509.01	0\\
510.01	0\\
511.01	0\\
512.01	0\\
513.01	0\\
514.01	0\\
515.01	0\\
516.01	0\\
517.01	0\\
518.01	0\\
519.01	0\\
520.01	0\\
521.01	0\\
522.01	0\\
523.01	0\\
524.01	0\\
525.01	0\\
526.01	0\\
527.01	0\\
528.01	0\\
529.01	0\\
530.01	0\\
531.01	0\\
532.01	0\\
533.01	0\\
534.01	0\\
535.01	0\\
536.01	0\\
537.01	0\\
538.01	0\\
539.01	0\\
540.01	0\\
541.01	0\\
542.01	0\\
543.01	0\\
544.01	0\\
545.01	0\\
546.01	0\\
547.01	0\\
548.01	0\\
549.01	0\\
550.01	0\\
551.01	0\\
552.01	0\\
553.01	0\\
554.01	0\\
555.01	0\\
556.01	0\\
557.01	0\\
558.01	0\\
559.01	0\\
560.01	0\\
561.01	0\\
562.01	0\\
563.01	0\\
564.01	0\\
565.01	0\\
566.01	0\\
567.01	0\\
568.01	0\\
569.01	0\\
570.01	0\\
571.01	0\\
572.01	0\\
573.01	0\\
574.01	0\\
575.01	0\\
576.01	0\\
577.01	0\\
578.01	0\\
579.01	0\\
580.01	0\\
581.01	0\\
582.01	7.0103522963872e-05\\
583.01	0.000458226270178982\\
584.01	0.00087452273602337\\
585.01	0.00131560487810147\\
586.01	0.00177015318791887\\
587.01	0.00223901543772628\\
588.01	0.00272381225844904\\
589.01	0.00322575052278649\\
590.01	0.00374460846515789\\
591.01	0.00427944286429531\\
592.01	0.00483141274336544\\
593.01	0.00540178309593597\\
594.01	0.00599196228625841\\
595.01	0.00660353070603266\\
596.01	0.00723827207202806\\
597.01	0.00789821384338649\\
598.01	0.00858568062740413\\
599.01	0.00929935695921623\\
599.02	0.00930653081533247\\
599.03	0.00931370403649347\\
599.04	0.00932087659239074\\
599.05	0.00932804845236433\\
599.06	0.00933521958539841\\
599.07	0.00934238996011689\\
599.08	0.00934955954477884\\
599.09	0.00935672830727394\\
599.1	0.0093638962151178\\
599.11	0.00937106323544725\\
599.12	0.00937822933501552\\
599.13	0.00938539448018738\\
599.14	0.00939255863693421\\
599.15	0.00939972177082893\\
599.16	0.00940688384704095\\
599.17	0.00941404483033095\\
599.18	0.00942120468504563\\
599.19	0.00942836337511238\\
599.2	0.00943552086403382\\
599.21	0.00944267711488231\\
599.22	0.00944983209029436\\
599.23	0.0094569857524649\\
599.24	0.00946413806314154\\
599.25	0.00947128898361864\\
599.26	0.00947843847473141\\
599.27	0.00948558649684979\\
599.28	0.00949273300987232\\
599.29	0.00949987797321982\\
599.3	0.0095070213458291\\
599.31	0.00951416308614641\\
599.32	0.00952130315212091\\
599.33	0.00952844150119796\\
599.34	0.00953557809031229\\
599.35	0.00954271287588114\\
599.36	0.00954984581379719\\
599.37	0.0095569768594214\\
599.38	0.00956410596757575\\
599.39	0.00957123309253585\\
599.4	0.00957835818802338\\
599.41	0.00958548120719844\\
599.42	0.00959260210265181\\
599.43	0.00959972082639694\\
599.44	0.00960683732986194\\
599.45	0.00961395156388134\\
599.46	0.00962106347868781\\
599.47	0.00962817302390588\\
599.48	0.00963528014854335\\
599.49	0.00964238480098245\\
599.5	0.0096494869289709\\
599.51	0.00965658647961283\\
599.52	0.00966368339935945\\
599.53	0.00967077763399964\\
599.54	0.00967786912865027\\
599.55	0.00968495782774643\\
599.56	0.0096920436750314\\
599.57	0.00969912661354648\\
599.58	0.0097062065856206\\
599.59	0.00971328353285974\\
599.6	0.00972035739613614\\
599.61	0.00972742811557727\\
599.62	0.00973449563055469\\
599.63	0.00974155987967251\\
599.64	0.0097486208007558\\
599.65	0.00975567833083868\\
599.66	0.00976273240615213\\
599.67	0.00976978296211165\\
599.68	0.0097768299333046\\
599.69	0.00978387325347729\\
599.7	0.0097909128555218\\
599.71	0.00979794867146256\\
599.72	0.00980498063244257\\
599.73	0.00981200866870942\\
599.74	0.0098190327096009\\
599.75	0.00982605268353043\\
599.76	0.00983306851797204\\
599.77	0.0098400801394451\\
599.78	0.0098470874734987\\
599.79	0.00985409044469565\\
599.8	0.00986108897659616\\
599.81	0.0098680829917411\\
599.82	0.00987507241163493\\
599.83	0.00988205715672815\\
599.84	0.00988903714639948\\
599.85	0.00989601229893744\\
599.86	0.00990298253152164\\
599.87	0.00990994776020354\\
599.88	0.00991690789988676\\
599.89	0.00992386286430693\\
599.9	0.009930812566011\\
599.91	0.00993775691633607\\
599.92	0.00994469582538763\\
599.93	0.00995162920201733\\
599.94	0.00995855695380011\\
599.95	0.00996547898701074\\
599.96	0.00997239520659974\\
599.97	0.00997930551616874\\
599.98	0.00998620981794502\\
599.99	0.00999310801275551\\
600	0.01\\
};
\addplot [color=mycolor14,solid,forget plot]
  table[row sep=crcr]{%
0.01	0\\
1.01	0\\
2.01	0\\
3.01	0\\
4.01	0\\
5.01	0\\
6.01	0\\
7.01	0\\
8.01	0\\
9.01	0\\
10.01	0\\
11.01	0\\
12.01	0\\
13.01	0\\
14.01	0\\
15.01	0\\
16.01	0\\
17.01	0\\
18.01	0\\
19.01	0\\
20.01	0\\
21.01	0\\
22.01	0\\
23.01	0\\
24.01	0\\
25.01	0\\
26.01	0\\
27.01	0\\
28.01	0\\
29.01	0\\
30.01	0\\
31.01	0\\
32.01	0\\
33.01	0\\
34.01	0\\
35.01	0\\
36.01	0\\
37.01	0\\
38.01	0\\
39.01	0\\
40.01	0\\
41.01	0\\
42.01	0\\
43.01	0\\
44.01	0\\
45.01	0\\
46.01	0\\
47.01	0\\
48.01	0\\
49.01	0\\
50.01	0\\
51.01	0\\
52.01	0\\
53.01	0\\
54.01	0\\
55.01	0\\
56.01	0\\
57.01	0\\
58.01	0\\
59.01	0\\
60.01	0\\
61.01	0\\
62.01	0\\
63.01	0\\
64.01	0\\
65.01	0\\
66.01	0\\
67.01	0\\
68.01	0\\
69.01	0\\
70.01	0\\
71.01	0\\
72.01	0\\
73.01	0\\
74.01	0\\
75.01	0\\
76.01	0\\
77.01	0\\
78.01	0\\
79.01	0\\
80.01	0\\
81.01	0\\
82.01	0\\
83.01	0\\
84.01	0\\
85.01	0\\
86.01	0\\
87.01	0\\
88.01	0\\
89.01	0\\
90.01	0\\
91.01	0\\
92.01	0\\
93.01	0\\
94.01	0\\
95.01	0\\
96.01	0\\
97.01	0\\
98.01	0\\
99.01	0\\
100.01	0\\
101.01	0\\
102.01	0\\
103.01	0\\
104.01	0\\
105.01	0\\
106.01	0\\
107.01	0\\
108.01	0\\
109.01	0\\
110.01	0\\
111.01	0\\
112.01	0\\
113.01	0\\
114.01	0\\
115.01	0\\
116.01	0\\
117.01	0\\
118.01	0\\
119.01	0\\
120.01	0\\
121.01	0\\
122.01	0\\
123.01	0\\
124.01	0\\
125.01	0\\
126.01	0\\
127.01	0\\
128.01	0\\
129.01	0\\
130.01	0\\
131.01	0\\
132.01	0\\
133.01	0\\
134.01	0\\
135.01	0\\
136.01	0\\
137.01	0\\
138.01	0\\
139.01	0\\
140.01	0\\
141.01	0\\
142.01	0\\
143.01	0\\
144.01	0\\
145.01	0\\
146.01	0\\
147.01	0\\
148.01	0\\
149.01	0\\
150.01	0\\
151.01	0\\
152.01	0\\
153.01	0\\
154.01	0\\
155.01	0\\
156.01	0\\
157.01	0\\
158.01	0\\
159.01	0\\
160.01	0\\
161.01	0\\
162.01	0\\
163.01	0\\
164.01	0\\
165.01	0\\
166.01	0\\
167.01	0\\
168.01	0\\
169.01	0\\
170.01	0\\
171.01	0\\
172.01	0\\
173.01	0\\
174.01	0\\
175.01	0\\
176.01	0\\
177.01	0\\
178.01	0\\
179.01	0\\
180.01	0\\
181.01	0\\
182.01	0\\
183.01	0\\
184.01	0\\
185.01	0\\
186.01	0\\
187.01	0\\
188.01	0\\
189.01	0\\
190.01	0\\
191.01	0\\
192.01	0\\
193.01	0\\
194.01	0\\
195.01	0\\
196.01	0\\
197.01	0\\
198.01	0\\
199.01	0\\
200.01	0\\
201.01	0\\
202.01	0\\
203.01	0\\
204.01	0\\
205.01	0\\
206.01	0\\
207.01	0\\
208.01	0\\
209.01	0\\
210.01	0\\
211.01	0\\
212.01	0\\
213.01	0\\
214.01	0\\
215.01	0\\
216.01	0\\
217.01	0\\
218.01	0\\
219.01	0\\
220.01	0\\
221.01	0\\
222.01	0\\
223.01	0\\
224.01	0\\
225.01	0\\
226.01	0\\
227.01	0\\
228.01	0\\
229.01	0\\
230.01	0\\
231.01	0\\
232.01	0\\
233.01	0\\
234.01	0\\
235.01	0\\
236.01	0\\
237.01	0\\
238.01	0\\
239.01	0\\
240.01	0\\
241.01	0\\
242.01	0\\
243.01	0\\
244.01	0\\
245.01	0\\
246.01	0\\
247.01	0\\
248.01	0\\
249.01	0\\
250.01	0\\
251.01	0\\
252.01	0\\
253.01	0\\
254.01	0\\
255.01	0\\
256.01	0\\
257.01	0\\
258.01	0\\
259.01	0\\
260.01	0\\
261.01	0\\
262.01	0\\
263.01	0\\
264.01	0\\
265.01	0\\
266.01	0\\
267.01	0\\
268.01	0\\
269.01	0\\
270.01	0\\
271.01	0\\
272.01	0\\
273.01	0\\
274.01	0\\
275.01	0\\
276.01	0\\
277.01	0\\
278.01	0\\
279.01	0\\
280.01	0\\
281.01	0\\
282.01	0\\
283.01	0\\
284.01	0\\
285.01	0\\
286.01	0\\
287.01	0\\
288.01	0\\
289.01	0\\
290.01	0\\
291.01	0\\
292.01	0\\
293.01	0\\
294.01	0\\
295.01	0\\
296.01	0\\
297.01	0\\
298.01	0\\
299.01	0\\
300.01	0\\
301.01	0\\
302.01	0\\
303.01	0\\
304.01	0\\
305.01	0\\
306.01	0\\
307.01	0\\
308.01	0\\
309.01	0\\
310.01	0\\
311.01	0\\
312.01	0\\
313.01	0\\
314.01	0\\
315.01	0\\
316.01	0\\
317.01	0\\
318.01	0\\
319.01	0\\
320.01	0\\
321.01	0\\
322.01	0\\
323.01	0\\
324.01	0\\
325.01	0\\
326.01	0\\
327.01	0\\
328.01	0\\
329.01	0\\
330.01	0\\
331.01	0\\
332.01	0\\
333.01	0\\
334.01	0\\
335.01	0\\
336.01	0\\
337.01	0\\
338.01	0\\
339.01	0\\
340.01	0\\
341.01	0\\
342.01	0\\
343.01	0\\
344.01	0\\
345.01	0\\
346.01	0\\
347.01	0\\
348.01	0\\
349.01	0\\
350.01	0\\
351.01	0\\
352.01	0\\
353.01	0\\
354.01	0\\
355.01	0\\
356.01	0\\
357.01	0\\
358.01	0\\
359.01	0\\
360.01	0\\
361.01	0\\
362.01	0\\
363.01	0\\
364.01	0\\
365.01	0\\
366.01	0\\
367.01	0\\
368.01	0\\
369.01	0\\
370.01	0\\
371.01	0\\
372.01	0\\
373.01	0\\
374.01	0\\
375.01	0\\
376.01	0\\
377.01	0\\
378.01	0\\
379.01	0\\
380.01	0\\
381.01	0\\
382.01	0\\
383.01	0\\
384.01	0\\
385.01	0\\
386.01	0\\
387.01	0\\
388.01	0\\
389.01	0\\
390.01	0\\
391.01	0\\
392.01	0\\
393.01	0\\
394.01	0\\
395.01	0\\
396.01	0\\
397.01	0\\
398.01	0\\
399.01	0\\
400.01	0\\
401.01	0\\
402.01	0\\
403.01	0\\
404.01	0\\
405.01	0\\
406.01	0\\
407.01	0\\
408.01	0\\
409.01	0\\
410.01	0\\
411.01	0\\
412.01	0\\
413.01	0\\
414.01	0\\
415.01	0\\
416.01	0\\
417.01	0\\
418.01	0\\
419.01	0\\
420.01	0\\
421.01	0\\
422.01	0\\
423.01	0\\
424.01	0\\
425.01	0\\
426.01	0\\
427.01	0\\
428.01	0\\
429.01	0\\
430.01	0\\
431.01	0\\
432.01	0\\
433.01	0\\
434.01	0\\
435.01	0\\
436.01	0\\
437.01	0\\
438.01	0\\
439.01	0\\
440.01	0\\
441.01	0\\
442.01	0\\
443.01	0\\
444.01	0\\
445.01	0\\
446.01	0\\
447.01	0\\
448.01	0\\
449.01	0\\
450.01	0\\
451.01	0\\
452.01	0\\
453.01	0\\
454.01	0\\
455.01	0\\
456.01	0\\
457.01	0\\
458.01	0\\
459.01	0\\
460.01	0\\
461.01	0\\
462.01	0\\
463.01	0\\
464.01	0\\
465.01	0\\
466.01	0\\
467.01	0\\
468.01	0\\
469.01	0\\
470.01	0\\
471.01	0\\
472.01	0\\
473.01	0\\
474.01	0\\
475.01	0\\
476.01	0\\
477.01	0\\
478.01	0\\
479.01	0\\
480.01	0\\
481.01	0\\
482.01	0\\
483.01	0\\
484.01	0\\
485.01	0\\
486.01	0\\
487.01	0\\
488.01	0\\
489.01	0\\
490.01	0\\
491.01	0\\
492.01	0\\
493.01	0\\
494.01	0\\
495.01	0\\
496.01	0\\
497.01	0\\
498.01	0\\
499.01	0\\
500.01	0\\
501.01	0\\
502.01	0\\
503.01	0\\
504.01	0\\
505.01	0\\
506.01	0\\
507.01	0\\
508.01	0\\
509.01	0\\
510.01	0\\
511.01	0\\
512.01	0\\
513.01	0\\
514.01	0\\
515.01	0\\
516.01	0\\
517.01	0\\
518.01	0\\
519.01	0\\
520.01	0\\
521.01	0\\
522.01	0\\
523.01	0\\
524.01	0\\
525.01	0\\
526.01	0\\
527.01	0\\
528.01	0\\
529.01	0\\
530.01	0\\
531.01	0\\
532.01	0\\
533.01	0\\
534.01	0\\
535.01	0\\
536.01	0\\
537.01	0\\
538.01	0\\
539.01	0\\
540.01	0\\
541.01	0\\
542.01	0\\
543.01	0\\
544.01	0\\
545.01	0\\
546.01	0\\
547.01	0\\
548.01	0\\
549.01	0\\
550.01	0\\
551.01	0\\
552.01	0\\
553.01	0\\
554.01	0\\
555.01	0\\
556.01	0\\
557.01	0\\
558.01	0\\
559.01	0\\
560.01	0\\
561.01	0\\
562.01	0\\
563.01	0\\
564.01	0\\
565.01	0\\
566.01	0\\
567.01	0\\
568.01	0.000272550836893013\\
569.01	0.000573431998096324\\
570.01	0.000894661826268021\\
571.01	0.00123916709588926\\
572.01	0.00161046588219712\\
573.01	0.00200914876660706\\
574.01	0.00242577240038586\\
575.01	0.00286070485950432\\
576.01	0.00331543550699241\\
577.01	0.00379158961836269\\
578.01	0.00429093233054111\\
579.01	0.00481536241245629\\
580.01	0.00536687839479479\\
581.01	0.00594746706158842\\
582.01	0.00648755091140849\\
583.01	0.00673068723289618\\
584.01	0.00696177471066208\\
585.01	0.00718292114201277\\
586.01	0.00740501470221952\\
587.01	0.00762757491051881\\
588.01	0.00784970065629068\\
589.01	0.00807045080115363\\
590.01	0.00828974304449905\\
591.01	0.00850824026166992\\
592.01	0.00872445348147146\\
593.01	0.00893652110766042\\
594.01	0.00914223590543076\\
595.01	0.00933900026637371\\
596.01	0.00952378122840184\\
597.01	0.00969306778608631\\
598.01	0.00984283008036189\\
599.01	0.00995428645146456\\
599.02	0.00995511722778644\\
599.03	0.00995594190291016\\
599.04	0.00995676044198501\\
599.05	0.00995757280986887\\
599.06	0.0099583789711255\\
599.07	0.00995917889002186\\
599.08	0.00995997253052542\\
599.09	0.00996075985630137\\
599.1	0.0099615408307099\\
599.11	0.00996231541680342\\
599.12	0.00996308357732374\\
599.13	0.00996384527469924\\
599.14	0.00996460047104208\\
599.15	0.00996534912814527\\
599.16	0.00996609120747984\\
599.17	0.00996682667019189\\
599.18	0.0099675554770997\\
599.19	0.00996827758869077\\
599.2	0.00996899296511884\\
599.21	0.00996970156620092\\
599.22	0.00997040335141429\\
599.23	0.00997109827989345\\
599.24	0.00997178631042707\\
599.25	0.00997246740145495\\
599.26	0.00997314151106556\\
599.27	0.00997380859699371\\
599.28	0.00997446861661746\\
599.29	0.00997512152695501\\
599.3	0.00997576728466153\\
599.31	0.00997640584602604\\
599.32	0.00997703716696816\\
599.33	0.00997766120303499\\
599.34	0.00997827790939785\\
599.35	0.00997888724084906\\
599.36	0.00997948915179871\\
599.37	0.00998008359627136\\
599.38	0.00998067052790281\\
599.39	0.00998124989993678\\
599.4	0.00998182166522161\\
599.41	0.00998238577620696\\
599.42	0.00998294218494043\\
599.43	0.00998349084306428\\
599.44	0.00998403170181203\\
599.45	0.00998456471200512\\
599.46	0.00998508982393612\\
599.47	0.00998560698245901\\
599.48	0.00998611613195496\\
599.49	0.00998661721632843\\
599.5	0.0099871101790033\\
599.51	0.00998759496291904\\
599.52	0.00998807151052681\\
599.53	0.00998853976378555\\
599.54	0.00998899966415814\\
599.55	0.00998945115260745\\
599.56	0.00998989416959251\\
599.57	0.00999032865506456\\
599.58	0.00999075454846319\\
599.59	0.00999117178871244\\
599.6	0.00999158031421693\\
599.61	0.00999198006285796\\
599.62	0.00999237097198967\\
599.63	0.00999275297843516\\
599.64	0.00999312601848268\\
599.65	0.00999349002788177\\
599.66	0.00999384494183951\\
599.67	0.00999419069501666\\
599.68	0.00999452722152399\\
599.69	0.00999485445491849\\
599.7	0.00999517232819969\\
599.71	0.00999548077380601\\
599.72	0.00999577972361112\\
599.73	0.00999606910892039\\
599.74	0.00999634886046729\\
599.75	0.00999661890841001\\
599.76	0.00999687918232793\\
599.77	0.00999712961121834\\
599.78	0.00999737012349307\\
599.79	0.00999760064697533\\
599.8	0.0099978211088965\\
599.81	0.00999803143589311\\
599.82	0.00999823155400384\\
599.83	0.00999842138866665\\
599.84	0.009998600864716\\
599.85	0.0099987699063802\\
599.86	0.00999892843727885\\
599.87	0.00999907638042046\\
599.88	0.00999921365820013\\
599.89	0.00999934019239746\\
599.9	0.00999945590417459\\
599.91	0.00999956071407437\\
599.92	0.00999965454201877\\
599.93	0.00999973730730745\\
599.94	0.00999980892861654\\
599.95	0.00999986932399766\\
599.96	0.00999991841087715\\
599.97	0.00999995610605555\\
599.98	0.0099999823257074\\
599.99	0.00999999698538124\\
600	0.01\\
};
\addplot [color=mycolor15,solid,forget plot]
  table[row sep=crcr]{%
0.01	0\\
1.01	0\\
2.01	0\\
3.01	0\\
4.01	0\\
5.01	0\\
6.01	0\\
7.01	0\\
8.01	0\\
9.01	0\\
10.01	0\\
11.01	0\\
12.01	0\\
13.01	0\\
14.01	0\\
15.01	0\\
16.01	0\\
17.01	0\\
18.01	0\\
19.01	0\\
20.01	0\\
21.01	0\\
22.01	0\\
23.01	0\\
24.01	0\\
25.01	0\\
26.01	0\\
27.01	0\\
28.01	0\\
29.01	0\\
30.01	0\\
31.01	0\\
32.01	0\\
33.01	0\\
34.01	0\\
35.01	0\\
36.01	0\\
37.01	0\\
38.01	0\\
39.01	0\\
40.01	0\\
41.01	0\\
42.01	0\\
43.01	0\\
44.01	0\\
45.01	0\\
46.01	0\\
47.01	0\\
48.01	0\\
49.01	0\\
50.01	0\\
51.01	0\\
52.01	0\\
53.01	0\\
54.01	0\\
55.01	0\\
56.01	0\\
57.01	0\\
58.01	0\\
59.01	0\\
60.01	0\\
61.01	0\\
62.01	0\\
63.01	0\\
64.01	0\\
65.01	0\\
66.01	0\\
67.01	0\\
68.01	0\\
69.01	0\\
70.01	0\\
71.01	0\\
72.01	0\\
73.01	0\\
74.01	0\\
75.01	0\\
76.01	0\\
77.01	0\\
78.01	0\\
79.01	0\\
80.01	0\\
81.01	0\\
82.01	0\\
83.01	0\\
84.01	0\\
85.01	0\\
86.01	0\\
87.01	0\\
88.01	0\\
89.01	0\\
90.01	0\\
91.01	0\\
92.01	0\\
93.01	0\\
94.01	0\\
95.01	0\\
96.01	0\\
97.01	0\\
98.01	0\\
99.01	0\\
100.01	0\\
101.01	0\\
102.01	0\\
103.01	0\\
104.01	0\\
105.01	0\\
106.01	0\\
107.01	0\\
108.01	0\\
109.01	0\\
110.01	0\\
111.01	0\\
112.01	0\\
113.01	0\\
114.01	0\\
115.01	0\\
116.01	0\\
117.01	0\\
118.01	0\\
119.01	0\\
120.01	0\\
121.01	0\\
122.01	0\\
123.01	0\\
124.01	0\\
125.01	0\\
126.01	0\\
127.01	0\\
128.01	0\\
129.01	0\\
130.01	0\\
131.01	0\\
132.01	0\\
133.01	0\\
134.01	0\\
135.01	0\\
136.01	0\\
137.01	0\\
138.01	0\\
139.01	0\\
140.01	0\\
141.01	0\\
142.01	0\\
143.01	0\\
144.01	0\\
145.01	0\\
146.01	0\\
147.01	0\\
148.01	0\\
149.01	0\\
150.01	0\\
151.01	0\\
152.01	0\\
153.01	0\\
154.01	0\\
155.01	0\\
156.01	0\\
157.01	0\\
158.01	0\\
159.01	0\\
160.01	0\\
161.01	0\\
162.01	0\\
163.01	0\\
164.01	0\\
165.01	0\\
166.01	0\\
167.01	0\\
168.01	0\\
169.01	0\\
170.01	0\\
171.01	0\\
172.01	0\\
173.01	0\\
174.01	0\\
175.01	0\\
176.01	0\\
177.01	0\\
178.01	0\\
179.01	0\\
180.01	0\\
181.01	0\\
182.01	0\\
183.01	0\\
184.01	0\\
185.01	0\\
186.01	0\\
187.01	0\\
188.01	0\\
189.01	0\\
190.01	0\\
191.01	0\\
192.01	0\\
193.01	0\\
194.01	0\\
195.01	0\\
196.01	0\\
197.01	0\\
198.01	0\\
199.01	0\\
200.01	0\\
201.01	0\\
202.01	0\\
203.01	0\\
204.01	0\\
205.01	0\\
206.01	0\\
207.01	0\\
208.01	0\\
209.01	0\\
210.01	0\\
211.01	0\\
212.01	0\\
213.01	0\\
214.01	0\\
215.01	0\\
216.01	0\\
217.01	0\\
218.01	0\\
219.01	0\\
220.01	0\\
221.01	0\\
222.01	0\\
223.01	0\\
224.01	0\\
225.01	0\\
226.01	0\\
227.01	0\\
228.01	0\\
229.01	0\\
230.01	0\\
231.01	0\\
232.01	0\\
233.01	0\\
234.01	0\\
235.01	0\\
236.01	0\\
237.01	0\\
238.01	0\\
239.01	0\\
240.01	0\\
241.01	0\\
242.01	0\\
243.01	0\\
244.01	0\\
245.01	0\\
246.01	0\\
247.01	0\\
248.01	0\\
249.01	0\\
250.01	0\\
251.01	0\\
252.01	0\\
253.01	0\\
254.01	0\\
255.01	0\\
256.01	0\\
257.01	0\\
258.01	0\\
259.01	0\\
260.01	0\\
261.01	0\\
262.01	0\\
263.01	0\\
264.01	0\\
265.01	0\\
266.01	0\\
267.01	0\\
268.01	0\\
269.01	0\\
270.01	0\\
271.01	0\\
272.01	0\\
273.01	0\\
274.01	0\\
275.01	0\\
276.01	0\\
277.01	0\\
278.01	0\\
279.01	0\\
280.01	0\\
281.01	0\\
282.01	0\\
283.01	0\\
284.01	0\\
285.01	0\\
286.01	0\\
287.01	0\\
288.01	0\\
289.01	0\\
290.01	0\\
291.01	0\\
292.01	0\\
293.01	0\\
294.01	0\\
295.01	0\\
296.01	0\\
297.01	0\\
298.01	0\\
299.01	0\\
300.01	0\\
301.01	0\\
302.01	0\\
303.01	0\\
304.01	0\\
305.01	0\\
306.01	0\\
307.01	0\\
308.01	0\\
309.01	0\\
310.01	0\\
311.01	0\\
312.01	0\\
313.01	0\\
314.01	0\\
315.01	0\\
316.01	0\\
317.01	0\\
318.01	0\\
319.01	0\\
320.01	0\\
321.01	0\\
322.01	0\\
323.01	0\\
324.01	0\\
325.01	0\\
326.01	0\\
327.01	0\\
328.01	0\\
329.01	0\\
330.01	0\\
331.01	0\\
332.01	0\\
333.01	0\\
334.01	0\\
335.01	0\\
336.01	0\\
337.01	0\\
338.01	0\\
339.01	0\\
340.01	0\\
341.01	0\\
342.01	0\\
343.01	0\\
344.01	0\\
345.01	0\\
346.01	0\\
347.01	0\\
348.01	0\\
349.01	0\\
350.01	0\\
351.01	0\\
352.01	0\\
353.01	0\\
354.01	0\\
355.01	0\\
356.01	0\\
357.01	0\\
358.01	0\\
359.01	0\\
360.01	0\\
361.01	0\\
362.01	0\\
363.01	0\\
364.01	0\\
365.01	0\\
366.01	0\\
367.01	0\\
368.01	0\\
369.01	0\\
370.01	0\\
371.01	0\\
372.01	0\\
373.01	0\\
374.01	0\\
375.01	0\\
376.01	0\\
377.01	0\\
378.01	0\\
379.01	0\\
380.01	0\\
381.01	0\\
382.01	0\\
383.01	0\\
384.01	0\\
385.01	0\\
386.01	0\\
387.01	0\\
388.01	0\\
389.01	0\\
390.01	0\\
391.01	0\\
392.01	0\\
393.01	0\\
394.01	0\\
395.01	0\\
396.01	0\\
397.01	0\\
398.01	0\\
399.01	0\\
400.01	0\\
401.01	0\\
402.01	0\\
403.01	0\\
404.01	0\\
405.01	0\\
406.01	0\\
407.01	0\\
408.01	0\\
409.01	0\\
410.01	0\\
411.01	0\\
412.01	0\\
413.01	0\\
414.01	0\\
415.01	0\\
416.01	0\\
417.01	0\\
418.01	0\\
419.01	0\\
420.01	0\\
421.01	0\\
422.01	0\\
423.01	0\\
424.01	0\\
425.01	0\\
426.01	0\\
427.01	0\\
428.01	0\\
429.01	0\\
430.01	0\\
431.01	0\\
432.01	0\\
433.01	0\\
434.01	0\\
435.01	0\\
436.01	0\\
437.01	0\\
438.01	0\\
439.01	0\\
440.01	0\\
441.01	0\\
442.01	0\\
443.01	0\\
444.01	0\\
445.01	0\\
446.01	0\\
447.01	0\\
448.01	0\\
449.01	0\\
450.01	0\\
451.01	0\\
452.01	0\\
453.01	0\\
454.01	0\\
455.01	0\\
456.01	0\\
457.01	0\\
458.01	0\\
459.01	0\\
460.01	0\\
461.01	0\\
462.01	0\\
463.01	0\\
464.01	0\\
465.01	0\\
466.01	0\\
467.01	0\\
468.01	0\\
469.01	0\\
470.01	0\\
471.01	0\\
472.01	0\\
473.01	0\\
474.01	0\\
475.01	0\\
476.01	0\\
477.01	0\\
478.01	0\\
479.01	0\\
480.01	0\\
481.01	0\\
482.01	0\\
483.01	0\\
484.01	0\\
485.01	0\\
486.01	0\\
487.01	0\\
488.01	0\\
489.01	0\\
490.01	0\\
491.01	0\\
492.01	0\\
493.01	0\\
494.01	0\\
495.01	0\\
496.01	0\\
497.01	0\\
498.01	0\\
499.01	0\\
500.01	0\\
501.01	0\\
502.01	0\\
503.01	0\\
504.01	0\\
505.01	0\\
506.01	0\\
507.01	0\\
508.01	0\\
509.01	0\\
510.01	0\\
511.01	0\\
512.01	0\\
513.01	0\\
514.01	0\\
515.01	0\\
516.01	0\\
517.01	0\\
518.01	0\\
519.01	0\\
520.01	0\\
521.01	0\\
522.01	0\\
523.01	0\\
524.01	0\\
525.01	0\\
526.01	0\\
527.01	0\\
528.01	0\\
529.01	0\\
530.01	0\\
531.01	0\\
532.01	0\\
533.01	0\\
534.01	0\\
535.01	0\\
536.01	0\\
537.01	0\\
538.01	0\\
539.01	0\\
540.01	0\\
541.01	0\\
542.01	0\\
543.01	0\\
544.01	0\\
545.01	0\\
546.01	0\\
547.01	0\\
548.01	0\\
549.01	0\\
550.01	4.18278501244616e-05\\
551.01	0.000230792793490968\\
552.01	0.000429318385053151\\
553.01	0.000638423513595891\\
554.01	0.000859291226111458\\
555.01	0.00109330128463499\\
556.01	0.00134206999046035\\
557.01	0.00160749963610692\\
558.01	0.00189184148236909\\
559.01	0.00219777549602514\\
560.01	0.00252698266125781\\
561.01	0.00287236341131009\\
562.01	0.00323336009928089\\
563.01	0.00361112252905767\\
564.01	0.00400699194756734\\
565.01	0.00442260638993207\\
566.01	0.0048602106788841\\
567.01	0.00532249736082619\\
568.01	0.00553562477064916\\
569.01	0.00574040657268343\\
570.01	0.00594543331123816\\
571.01	0.00614813023233589\\
572.01	0.00634492779261982\\
573.01	0.00653458863917867\\
574.01	0.00672477895943008\\
575.01	0.00691480367421593\\
576.01	0.00710345125817615\\
577.01	0.00728925973806523\\
578.01	0.00747048244186695\\
579.01	0.00764505567993344\\
580.01	0.00781057365081174\\
581.01	0.00796428037686027\\
582.01	0.00810430631923353\\
583.01	0.00823778064650377\\
584.01	0.00836882604993697\\
585.01	0.00849805350251816\\
586.01	0.00862548745291904\\
587.01	0.00875080170555895\\
588.01	0.008873717407561\\
589.01	0.00899400983054572\\
590.01	0.00911146508396814\\
591.01	0.00922566189803017\\
592.01	0.00933610883364893\\
593.01	0.0094423826989696\\
594.01	0.00954416814589645\\
595.01	0.00964130354156367\\
596.01	0.00973383226636715\\
597.01	0.00982205623179395\\
598.01	0.00990570153168505\\
599.01	0.00997025669700849\\
599.02	0.00997077449256594\\
599.03	0.00997128908704524\\
599.04	0.00997180045248797\\
599.05	0.00997230856064844\\
599.06	0.00997281338299083\\
599.07	0.00997331489068629\\
599.08	0.00997381305460998\\
599.09	0.00997430784533816\\
599.1	0.00997479923314515\\
599.11	0.00997528718800035\\
599.12	0.00997577167956518\\
599.13	0.00997625267719001\\
599.14	0.00997673014991101\\
599.15	0.00997720406644708\\
599.16	0.00997767439519661\\
599.17	0.00997814110423434\\
599.18	0.00997860416130809\\
599.19	0.00997906353383548\\
599.2	0.00997951918890067\\
599.21	0.00997997109325102\\
599.22	0.00998041921329368\\
599.23	0.00998086351509228\\
599.24	0.0099813039643634\\
599.25	0.00998174052647318\\
599.26	0.00998217316503718\\
599.27	0.00998260184151771\\
599.28	0.00998302651697996\\
599.29	0.00998344715208793\\
599.3	0.00998386370710054\\
599.31	0.00998427614186751\\
599.32	0.00998468441582528\\
599.33	0.00998508848799292\\
599.34	0.00998548831696792\\
599.35	0.00998588386092202\\
599.36	0.00998627507759691\\
599.37	0.00998666192429999\\
599.38	0.00998704435789998\\
599.39	0.00998742233482257\\
599.4	0.00998779581104599\\
599.41	0.00998816474209656\\
599.42	0.0099885290830441\\
599.43	0.00998888878849749\\
599.44	0.00998924381259994\\
599.45	0.00998959410902443\\
599.46	0.00998993963096902\\
599.47	0.00999028033115466\\
599.48	0.00999061616182043\\
599.49	0.00999094707471871\\
599.5	0.00999127302111033\\
599.51	0.00999159395175964\\
599.52	0.00999190981692958\\
599.53	0.00999222056637663\\
599.54	0.00999252614934575\\
599.55	0.00999282651456528\\
599.56	0.00999312161024174\\
599.57	0.00999341138405464\\
599.58	0.00999369578315118\\
599.59	0.00999397475414089\\
599.6	0.00999424824309031\\
599.61	0.00999451619551744\\
599.62	0.00999477855638636\\
599.63	0.00999503527010156\\
599.64	0.00999528628050238\\
599.65	0.00999553153085729\\
599.66	0.0099957709638582\\
599.67	0.0099960045216146\\
599.68	0.00999623214564774\\
599.69	0.00999645377688465\\
599.7	0.00999666935565223\\
599.71	0.00999687882167107\\
599.72	0.00999708211404946\\
599.73	0.00999727917127708\\
599.74	0.00999746993121882\\
599.75	0.0099976543311084\\
599.76	0.00999783230754198\\
599.77	0.00999800379647171\\
599.78	0.00999816873319912\\
599.79	0.00999832705236858\\
599.8	0.00999847868796051\\
599.81	0.00999862357328467\\
599.82	0.00999876164097328\\
599.83	0.00999889282297404\\
599.84	0.00999901705054318\\
599.85	0.0099991342542383\\
599.86	0.00999924436391118\\
599.87	0.0099993473087005\\
599.88	0.0099994430170245\\
599.89	0.00999953141657346\\
599.9	0.00999961243430216\\
599.91	0.00999968599642223\\
599.92	0.00999975202839438\\
599.93	0.00999981045492053\\
599.94	0.00999986119993585\\
599.95	0.0099999041866007\\
599.96	0.00999993933729238\\
599.97	0.00999996657359692\\
599.98	0.00999998581630055\\
599.99	0.00999999698538124\\
600	0.01\\
};
\addplot [color=mycolor16,solid,forget plot]
  table[row sep=crcr]{%
0.01	0\\
1.01	0\\
2.01	0\\
3.01	0\\
4.01	0\\
5.01	0\\
6.01	0\\
7.01	0\\
8.01	0\\
9.01	0\\
10.01	0\\
11.01	0\\
12.01	0\\
13.01	0\\
14.01	0\\
15.01	0\\
16.01	0\\
17.01	0\\
18.01	0\\
19.01	0\\
20.01	0\\
21.01	0\\
22.01	0\\
23.01	0\\
24.01	0\\
25.01	0\\
26.01	0\\
27.01	0\\
28.01	0\\
29.01	0\\
30.01	0\\
31.01	0\\
32.01	0\\
33.01	0\\
34.01	0\\
35.01	0\\
36.01	0\\
37.01	0\\
38.01	0\\
39.01	0\\
40.01	0\\
41.01	0\\
42.01	0\\
43.01	0\\
44.01	0\\
45.01	0\\
46.01	0\\
47.01	0\\
48.01	0\\
49.01	0\\
50.01	0\\
51.01	0\\
52.01	0\\
53.01	0\\
54.01	0\\
55.01	0\\
56.01	0\\
57.01	0\\
58.01	0\\
59.01	0\\
60.01	0\\
61.01	0\\
62.01	0\\
63.01	0\\
64.01	0\\
65.01	0\\
66.01	0\\
67.01	0\\
68.01	0\\
69.01	0\\
70.01	0\\
71.01	0\\
72.01	0\\
73.01	0\\
74.01	0\\
75.01	0\\
76.01	0\\
77.01	0\\
78.01	0\\
79.01	0\\
80.01	0\\
81.01	0\\
82.01	0\\
83.01	0\\
84.01	0\\
85.01	0\\
86.01	0\\
87.01	0\\
88.01	0\\
89.01	0\\
90.01	0\\
91.01	0\\
92.01	0\\
93.01	0\\
94.01	0\\
95.01	0\\
96.01	0\\
97.01	0\\
98.01	0\\
99.01	0\\
100.01	0\\
101.01	0\\
102.01	0\\
103.01	0\\
104.01	0\\
105.01	0\\
106.01	0\\
107.01	0\\
108.01	0\\
109.01	0\\
110.01	0\\
111.01	0\\
112.01	0\\
113.01	0\\
114.01	0\\
115.01	0\\
116.01	0\\
117.01	0\\
118.01	0\\
119.01	0\\
120.01	0\\
121.01	0\\
122.01	0\\
123.01	0\\
124.01	0\\
125.01	0\\
126.01	0\\
127.01	0\\
128.01	0\\
129.01	0\\
130.01	0\\
131.01	0\\
132.01	0\\
133.01	0\\
134.01	0\\
135.01	0\\
136.01	0\\
137.01	0\\
138.01	0\\
139.01	0\\
140.01	0\\
141.01	0\\
142.01	0\\
143.01	0\\
144.01	0\\
145.01	0\\
146.01	0\\
147.01	0\\
148.01	0\\
149.01	0\\
150.01	0\\
151.01	0\\
152.01	0\\
153.01	0\\
154.01	0\\
155.01	0\\
156.01	0\\
157.01	0\\
158.01	0\\
159.01	0\\
160.01	0\\
161.01	0\\
162.01	0\\
163.01	0\\
164.01	0\\
165.01	0\\
166.01	0\\
167.01	0\\
168.01	0\\
169.01	0\\
170.01	0\\
171.01	0\\
172.01	0\\
173.01	0\\
174.01	0\\
175.01	0\\
176.01	0\\
177.01	0\\
178.01	0\\
179.01	0\\
180.01	0\\
181.01	0\\
182.01	0\\
183.01	0\\
184.01	0\\
185.01	0\\
186.01	0\\
187.01	0\\
188.01	0\\
189.01	0\\
190.01	0\\
191.01	0\\
192.01	0\\
193.01	0\\
194.01	0\\
195.01	0\\
196.01	0\\
197.01	0\\
198.01	0\\
199.01	0\\
200.01	0\\
201.01	0\\
202.01	0\\
203.01	0\\
204.01	0\\
205.01	0\\
206.01	0\\
207.01	0\\
208.01	0\\
209.01	0\\
210.01	0\\
211.01	0\\
212.01	0\\
213.01	0\\
214.01	0\\
215.01	0\\
216.01	0\\
217.01	0\\
218.01	0\\
219.01	0\\
220.01	0\\
221.01	0\\
222.01	0\\
223.01	0\\
224.01	0\\
225.01	0\\
226.01	0\\
227.01	0\\
228.01	0\\
229.01	0\\
230.01	0\\
231.01	0\\
232.01	0\\
233.01	0\\
234.01	0\\
235.01	0\\
236.01	0\\
237.01	0\\
238.01	0\\
239.01	0\\
240.01	0\\
241.01	0\\
242.01	0\\
243.01	0\\
244.01	0\\
245.01	0\\
246.01	0\\
247.01	0\\
248.01	0\\
249.01	0\\
250.01	0\\
251.01	0\\
252.01	0\\
253.01	0\\
254.01	0\\
255.01	0\\
256.01	0\\
257.01	0\\
258.01	0\\
259.01	0\\
260.01	0\\
261.01	0\\
262.01	0\\
263.01	0\\
264.01	0\\
265.01	0\\
266.01	0\\
267.01	0\\
268.01	0\\
269.01	0\\
270.01	0\\
271.01	0\\
272.01	0\\
273.01	0\\
274.01	0\\
275.01	0\\
276.01	0\\
277.01	0\\
278.01	0\\
279.01	0\\
280.01	0\\
281.01	0\\
282.01	0\\
283.01	0\\
284.01	0\\
285.01	0\\
286.01	0\\
287.01	0\\
288.01	0\\
289.01	0\\
290.01	0\\
291.01	0\\
292.01	0\\
293.01	0\\
294.01	0\\
295.01	0\\
296.01	0\\
297.01	0\\
298.01	0\\
299.01	0\\
300.01	0\\
301.01	0\\
302.01	0\\
303.01	0\\
304.01	0\\
305.01	0\\
306.01	0\\
307.01	0\\
308.01	0\\
309.01	0\\
310.01	0\\
311.01	0\\
312.01	0\\
313.01	0\\
314.01	0\\
315.01	0\\
316.01	0\\
317.01	0\\
318.01	0\\
319.01	0\\
320.01	0\\
321.01	0\\
322.01	0\\
323.01	0\\
324.01	0\\
325.01	0\\
326.01	0\\
327.01	0\\
328.01	0\\
329.01	0\\
330.01	0\\
331.01	0\\
332.01	0\\
333.01	0\\
334.01	0\\
335.01	0\\
336.01	0\\
337.01	0\\
338.01	0\\
339.01	0\\
340.01	0\\
341.01	0\\
342.01	0\\
343.01	0\\
344.01	0\\
345.01	0\\
346.01	0\\
347.01	0\\
348.01	0\\
349.01	0\\
350.01	0\\
351.01	0\\
352.01	0\\
353.01	0\\
354.01	0\\
355.01	0\\
356.01	0\\
357.01	0\\
358.01	0\\
359.01	0\\
360.01	0\\
361.01	0\\
362.01	0\\
363.01	0\\
364.01	0\\
365.01	0\\
366.01	0\\
367.01	0\\
368.01	0\\
369.01	0\\
370.01	0\\
371.01	0\\
372.01	0\\
373.01	0\\
374.01	0\\
375.01	0\\
376.01	0\\
377.01	0\\
378.01	0\\
379.01	0\\
380.01	0\\
381.01	0\\
382.01	0\\
383.01	0\\
384.01	0\\
385.01	0\\
386.01	0\\
387.01	0\\
388.01	0\\
389.01	0\\
390.01	0\\
391.01	0\\
392.01	0\\
393.01	0\\
394.01	0\\
395.01	0\\
396.01	0\\
397.01	0\\
398.01	0\\
399.01	0\\
400.01	0\\
401.01	0\\
402.01	0\\
403.01	0\\
404.01	0\\
405.01	0\\
406.01	0\\
407.01	0\\
408.01	0\\
409.01	0\\
410.01	0\\
411.01	0\\
412.01	0\\
413.01	0\\
414.01	0\\
415.01	0\\
416.01	0\\
417.01	0\\
418.01	0\\
419.01	0\\
420.01	0\\
421.01	0\\
422.01	0\\
423.01	0\\
424.01	0\\
425.01	0\\
426.01	0\\
427.01	0\\
428.01	0\\
429.01	0\\
430.01	0\\
431.01	0\\
432.01	0\\
433.01	0\\
434.01	0\\
435.01	0\\
436.01	0\\
437.01	0\\
438.01	0\\
439.01	0\\
440.01	0\\
441.01	0\\
442.01	0\\
443.01	0\\
444.01	0\\
445.01	0\\
446.01	0\\
447.01	0\\
448.01	0\\
449.01	0\\
450.01	0\\
451.01	0\\
452.01	0\\
453.01	0\\
454.01	0\\
455.01	0\\
456.01	0\\
457.01	0\\
458.01	0\\
459.01	0\\
460.01	0\\
461.01	0\\
462.01	0\\
463.01	0\\
464.01	0\\
465.01	0\\
466.01	0\\
467.01	0\\
468.01	0\\
469.01	0\\
470.01	0\\
471.01	0\\
472.01	0\\
473.01	0\\
474.01	0\\
475.01	0\\
476.01	0\\
477.01	0\\
478.01	0\\
479.01	0\\
480.01	0\\
481.01	0\\
482.01	0\\
483.01	0\\
484.01	0\\
485.01	0\\
486.01	0\\
487.01	0\\
488.01	0\\
489.01	0\\
490.01	0\\
491.01	0\\
492.01	0\\
493.01	0\\
494.01	0\\
495.01	0\\
496.01	0\\
497.01	0\\
498.01	0\\
499.01	0\\
500.01	0\\
501.01	0\\
502.01	0\\
503.01	0\\
504.01	0\\
505.01	0\\
506.01	0\\
507.01	0\\
508.01	0\\
509.01	0\\
510.01	0\\
511.01	0\\
512.01	0\\
513.01	0\\
514.01	0\\
515.01	0\\
516.01	0\\
517.01	0\\
518.01	0\\
519.01	0\\
520.01	0\\
521.01	0\\
522.01	0\\
523.01	0\\
524.01	0\\
525.01	0\\
526.01	0\\
527.01	7.58709029509003e-05\\
528.01	0.000182474483497404\\
529.01	0.000293106910623106\\
530.01	0.000408037296038929\\
531.01	0.000527564848188972\\
532.01	0.000652024036411215\\
533.01	0.000781791922589287\\
534.01	0.00091730227808122\\
535.01	0.00105904894748098\\
536.01	0.00120759049179552\\
537.01	0.00136356032051525\\
538.01	0.00152767859230953\\
539.01	0.00170076616388885\\
540.01	0.00188376095582964\\
541.01	0.00207773717383749\\
542.01	0.00228393819189821\\
543.01	0.00250393549168844\\
544.01	0.00273969443204838\\
545.01	0.0029920480527937\\
546.01	0.00325611513039836\\
547.01	0.00353212677023015\\
548.01	0.00382123777963479\\
549.01	0.00412481857923038\\
550.01	0.00440253967046936\\
551.01	0.00454701284676821\\
552.01	0.00469499207454682\\
553.01	0.00484609988606597\\
554.01	0.00499980750463021\\
555.01	0.00515539572510951\\
556.01	0.00531191602469303\\
557.01	0.00546809119478294\\
558.01	0.00562218531161904\\
559.01	0.00577185501385423\\
560.01	0.00591549938797714\\
561.01	0.00606040602741116\\
562.01	0.00620758878906431\\
563.01	0.00635622936558284\\
564.01	0.00650515733755577\\
565.01	0.00665171599421486\\
566.01	0.00679316347567889\\
567.01	0.00692694660728829\\
568.01	0.00705436387442827\\
569.01	0.00717888869534497\\
570.01	0.00729992529312984\\
571.01	0.00741704362151842\\
572.01	0.00753014340890209\\
573.01	0.00763968735015476\\
574.01	0.00774735512661497\\
575.01	0.00785350951050682\\
576.01	0.00795787532128202\\
577.01	0.00806023678096092\\
578.01	0.00816046589172386\\
579.01	0.00825855451912221\\
580.01	0.00835464755040728\\
581.01	0.00844907172503963\\
582.01	0.00854231050026635\\
583.01	0.00863470051440009\\
584.01	0.00872627653456776\\
585.01	0.0088170082496113\\
586.01	0.00890683476173956\\
587.01	0.00899570431881243\\
588.01	0.00908357815033255\\
589.01	0.0091704238016257\\
590.01	0.00925620895122939\\
591.01	0.00934091361666743\\
592.01	0.00942455622952417\\
593.01	0.00950719997922862\\
594.01	0.0095889524410984\\
595.01	0.00966996121364661\\
596.01	0.00975040472227273\\
597.01	0.00983047783702982\\
598.01	0.00990810332595271\\
599.01	0.00997052938090576\\
599.02	0.00997103903864904\\
599.03	0.00997154565629264\\
599.04	0.00997204920438894\\
599.05	0.00997254965319863\\
599.06	0.00997304697268781\\
599.07	0.00997354113252509\\
599.08	0.0099740321020787\\
599.09	0.00997451985041345\\
599.1	0.0099750043462878\\
599.11	0.00997548555815081\\
599.12	0.00997596345413907\\
599.13	0.00997643800207369\\
599.14	0.00997690916945707\\
599.15	0.00997737692346987\\
599.16	0.00997784123096778\\
599.17	0.00997830205847831\\
599.18	0.00997875937219755\\
599.19	0.00997921313798693\\
599.2	0.0099796633213699\\
599.21	0.0099801098875286\\
599.22	0.00998055280130045\\
599.23	0.00998099202717483\\
599.24	0.00998142752928957\\
599.25	0.00998185927142751\\
599.26	0.00998228721513154\\
599.27	0.00998271132031303\\
599.28	0.0099831315464829\\
599.29	0.00998354785274771\\
599.3	0.00998396019780564\\
599.31	0.00998436853994246\\
599.32	0.00998477283702748\\
599.33	0.00998517304650944\\
599.34	0.00998556912541235\\
599.35	0.0099859610303313\\
599.36	0.00998634871742824\\
599.37	0.00998673214242771\\
599.38	0.00998711126061251\\
599.39	0.00998748602681935\\
599.4	0.00998785639543445\\
599.41	0.0099882223203891\\
599.42	0.00998858375515517\\
599.43	0.00998894065274057\\
599.44	0.00998929296568469\\
599.45	0.00998964064605377\\
599.46	0.00998998364543622\\
599.47	0.00999032191493792\\
599.48	0.00999065540517746\\
599.49	0.00999098406628132\\
599.5	0.00999130784787903\\
599.51	0.00999162669909824\\
599.52	0.00999194056855979\\
599.53	0.00999224940437273\\
599.54	0.00999255315412921\\
599.55	0.00999285176489941\\
599.56	0.00999314518322642\\
599.57	0.009993433355121\\
599.58	0.00999371622605632\\
599.59	0.0099939937409627\\
599.6	0.0099942658442222\\
599.61	0.00999453247966325\\
599.62	0.00999479359055517\\
599.63	0.00999504911960265\\
599.64	0.00999529900894021\\
599.65	0.00999554320012651\\
599.66	0.00999578163413872\\
599.67	0.00999601425136677\\
599.68	0.00999624099160756\\
599.69	0.00999646179405908\\
599.7	0.00999667659731453\\
599.71	0.00999688533935636\\
599.72	0.0099970879575502\\
599.73	0.00999728438863882\\
599.74	0.00999747456873595\\
599.75	0.0099976584333201\\
599.76	0.00999783591722828\\
599.77	0.00999800695464968\\
599.78	0.00999817147911925\\
599.79	0.00999832942351129\\
599.8	0.00999848072003288\\
599.81	0.00999862530021737\\
599.82	0.00999876309491766\\
599.83	0.00999889403429953\\
599.84	0.00999901804783486\\
599.85	0.00999913506429477\\
599.86	0.00999924501174274\\
599.87	0.00999934781752758\\
599.88	0.00999944340827644\\
599.89	0.00999953170988768\\
599.9	0.00999961264752365\\
599.91	0.00999968614560349\\
599.92	0.00999975212779576\\
599.93	0.00999981051701107\\
599.94	0.0099998612353946\\
599.95	0.00999990420431857\\
599.96	0.00999993934437464\\
599.97	0.00999996657536618\\
599.98	0.00999998581630055\\
599.99	0.00999999698538124\\
600	0.01\\
};
\addplot [color=mycolor17,solid,forget plot]
  table[row sep=crcr]{%
0.01	0\\
1.01	0\\
2.01	0\\
3.01	0\\
4.01	0\\
5.01	0\\
6.01	0\\
7.01	0\\
8.01	0\\
9.01	0\\
10.01	0\\
11.01	0\\
12.01	0\\
13.01	0\\
14.01	0\\
15.01	0\\
16.01	0\\
17.01	0\\
18.01	0\\
19.01	0\\
20.01	0\\
21.01	0\\
22.01	0\\
23.01	0\\
24.01	0\\
25.01	0\\
26.01	0\\
27.01	0\\
28.01	0\\
29.01	0\\
30.01	0\\
31.01	0\\
32.01	0\\
33.01	0\\
34.01	0\\
35.01	0\\
36.01	0\\
37.01	0\\
38.01	0\\
39.01	0\\
40.01	0\\
41.01	0\\
42.01	0\\
43.01	0\\
44.01	0\\
45.01	0\\
46.01	0\\
47.01	0\\
48.01	0\\
49.01	0\\
50.01	0\\
51.01	0\\
52.01	0\\
53.01	0\\
54.01	0\\
55.01	0\\
56.01	0\\
57.01	0\\
58.01	0\\
59.01	0\\
60.01	0\\
61.01	0\\
62.01	0\\
63.01	0\\
64.01	0\\
65.01	0\\
66.01	0\\
67.01	0\\
68.01	0\\
69.01	0\\
70.01	0\\
71.01	0\\
72.01	0\\
73.01	0\\
74.01	0\\
75.01	0\\
76.01	0\\
77.01	0\\
78.01	0\\
79.01	0\\
80.01	0\\
81.01	0\\
82.01	0\\
83.01	0\\
84.01	0\\
85.01	0\\
86.01	0\\
87.01	0\\
88.01	0\\
89.01	0\\
90.01	0\\
91.01	0\\
92.01	0\\
93.01	0\\
94.01	0\\
95.01	0\\
96.01	0\\
97.01	0\\
98.01	0\\
99.01	0\\
100.01	0\\
101.01	0\\
102.01	0\\
103.01	0\\
104.01	0\\
105.01	0\\
106.01	0\\
107.01	0\\
108.01	0\\
109.01	0\\
110.01	0\\
111.01	0\\
112.01	0\\
113.01	0\\
114.01	0\\
115.01	0\\
116.01	0\\
117.01	0\\
118.01	0\\
119.01	0\\
120.01	0\\
121.01	0\\
122.01	0\\
123.01	0\\
124.01	0\\
125.01	0\\
126.01	0\\
127.01	0\\
128.01	0\\
129.01	0\\
130.01	0\\
131.01	0\\
132.01	0\\
133.01	0\\
134.01	0\\
135.01	0\\
136.01	0\\
137.01	0\\
138.01	0\\
139.01	0\\
140.01	0\\
141.01	0\\
142.01	0\\
143.01	0\\
144.01	0\\
145.01	0\\
146.01	0\\
147.01	0\\
148.01	0\\
149.01	0\\
150.01	0\\
151.01	0\\
152.01	0\\
153.01	0\\
154.01	0\\
155.01	0\\
156.01	0\\
157.01	0\\
158.01	0\\
159.01	0\\
160.01	0\\
161.01	0\\
162.01	0\\
163.01	0\\
164.01	0\\
165.01	0\\
166.01	0\\
167.01	0\\
168.01	0\\
169.01	0\\
170.01	0\\
171.01	0\\
172.01	0\\
173.01	0\\
174.01	0\\
175.01	0\\
176.01	0\\
177.01	0\\
178.01	0\\
179.01	0\\
180.01	0\\
181.01	0\\
182.01	0\\
183.01	0\\
184.01	0\\
185.01	0\\
186.01	0\\
187.01	0\\
188.01	0\\
189.01	0\\
190.01	0\\
191.01	0\\
192.01	0\\
193.01	0\\
194.01	0\\
195.01	0\\
196.01	0\\
197.01	0\\
198.01	0\\
199.01	0\\
200.01	0\\
201.01	0\\
202.01	0\\
203.01	0\\
204.01	0\\
205.01	0\\
206.01	0\\
207.01	0\\
208.01	0\\
209.01	0\\
210.01	0\\
211.01	0\\
212.01	0\\
213.01	0\\
214.01	0\\
215.01	0\\
216.01	0\\
217.01	0\\
218.01	0\\
219.01	0\\
220.01	0\\
221.01	0\\
222.01	0\\
223.01	0\\
224.01	0\\
225.01	0\\
226.01	0\\
227.01	0\\
228.01	0\\
229.01	0\\
230.01	0\\
231.01	0\\
232.01	0\\
233.01	0\\
234.01	0\\
235.01	0\\
236.01	0\\
237.01	0\\
238.01	0\\
239.01	0\\
240.01	0\\
241.01	0\\
242.01	0\\
243.01	0\\
244.01	0\\
245.01	0\\
246.01	0\\
247.01	0\\
248.01	0\\
249.01	0\\
250.01	0\\
251.01	0\\
252.01	0\\
253.01	0\\
254.01	0\\
255.01	0\\
256.01	0\\
257.01	0\\
258.01	0\\
259.01	0\\
260.01	0\\
261.01	0\\
262.01	0\\
263.01	0\\
264.01	0\\
265.01	0\\
266.01	0\\
267.01	0\\
268.01	0\\
269.01	0\\
270.01	0\\
271.01	0\\
272.01	0\\
273.01	0\\
274.01	0\\
275.01	0\\
276.01	0\\
277.01	0\\
278.01	0\\
279.01	0\\
280.01	0\\
281.01	0\\
282.01	0\\
283.01	0\\
284.01	0\\
285.01	0\\
286.01	0\\
287.01	0\\
288.01	0\\
289.01	0\\
290.01	0\\
291.01	0\\
292.01	0\\
293.01	0\\
294.01	0\\
295.01	0\\
296.01	0\\
297.01	0\\
298.01	0\\
299.01	0\\
300.01	0\\
301.01	0\\
302.01	0\\
303.01	0\\
304.01	0\\
305.01	0\\
306.01	0\\
307.01	0\\
308.01	0\\
309.01	0\\
310.01	0\\
311.01	0\\
312.01	0\\
313.01	0\\
314.01	0\\
315.01	0\\
316.01	0\\
317.01	0\\
318.01	0\\
319.01	0\\
320.01	0\\
321.01	0\\
322.01	0\\
323.01	0\\
324.01	0\\
325.01	0\\
326.01	0\\
327.01	0\\
328.01	0\\
329.01	0\\
330.01	0\\
331.01	0\\
332.01	0\\
333.01	0\\
334.01	0\\
335.01	0\\
336.01	0\\
337.01	0\\
338.01	0\\
339.01	0\\
340.01	0\\
341.01	0\\
342.01	0\\
343.01	0\\
344.01	0\\
345.01	0\\
346.01	0\\
347.01	0\\
348.01	0\\
349.01	0\\
350.01	0\\
351.01	0\\
352.01	0\\
353.01	0\\
354.01	0\\
355.01	0\\
356.01	0\\
357.01	0\\
358.01	0\\
359.01	0\\
360.01	0\\
361.01	0\\
362.01	0\\
363.01	0\\
364.01	0\\
365.01	0\\
366.01	0\\
367.01	0\\
368.01	0\\
369.01	0\\
370.01	0\\
371.01	0\\
372.01	0\\
373.01	0\\
374.01	0\\
375.01	0\\
376.01	0\\
377.01	0\\
378.01	0\\
379.01	0\\
380.01	0\\
381.01	0\\
382.01	0\\
383.01	0\\
384.01	0\\
385.01	0\\
386.01	0\\
387.01	0\\
388.01	0\\
389.01	0\\
390.01	0\\
391.01	0\\
392.01	0\\
393.01	0\\
394.01	0\\
395.01	0\\
396.01	0\\
397.01	0\\
398.01	0\\
399.01	0\\
400.01	0\\
401.01	0\\
402.01	0\\
403.01	0\\
404.01	0\\
405.01	0\\
406.01	0\\
407.01	0\\
408.01	0\\
409.01	0\\
410.01	0\\
411.01	0\\
412.01	0\\
413.01	0\\
414.01	0\\
415.01	0\\
416.01	0\\
417.01	0\\
418.01	0\\
419.01	0\\
420.01	0\\
421.01	0\\
422.01	0\\
423.01	0\\
424.01	0\\
425.01	0\\
426.01	0\\
427.01	0\\
428.01	0\\
429.01	0\\
430.01	0\\
431.01	0\\
432.01	0\\
433.01	0\\
434.01	0\\
435.01	0\\
436.01	0\\
437.01	0\\
438.01	0\\
439.01	0\\
440.01	0\\
441.01	0\\
442.01	0\\
443.01	0\\
444.01	0\\
445.01	0\\
446.01	0\\
447.01	0\\
448.01	0\\
449.01	0\\
450.01	0\\
451.01	0\\
452.01	0\\
453.01	0\\
454.01	0\\
455.01	0\\
456.01	0\\
457.01	0\\
458.01	0\\
459.01	0\\
460.01	0\\
461.01	0\\
462.01	0\\
463.01	0\\
464.01	0\\
465.01	0\\
466.01	0\\
467.01	0\\
468.01	0\\
469.01	0\\
470.01	0\\
471.01	0\\
472.01	0\\
473.01	0\\
474.01	0\\
475.01	0\\
476.01	0\\
477.01	0\\
478.01	0\\
479.01	0\\
480.01	0\\
481.01	0\\
482.01	2.41235054160918e-05\\
483.01	5.71856317362687e-05\\
484.01	9.13442313131146e-05\\
485.01	0.000126644284644054\\
486.01	0.000163131355588451\\
487.01	0.000200850837936584\\
488.01	0.000239846872422192\\
489.01	0.000280160817670599\\
490.01	0.0003218291193606\\
491.01	0.000364880369701101\\
492.01	0.000409331279972655\\
493.01	0.000455181451220144\\
494.01	0.000502437693257939\\
495.01	0.00055115312847636\\
496.01	0.000601389121154525\\
497.01	0.000653209377263549\\
498.01	0.000706679475544134\\
499.01	0.000761866126084601\\
500.01	0.00081883605400275\\
501.01	0.000877654370643646\\
502.01	0.000938382247661296\\
503.01	0.00100107364646979\\
504.01	0.00106577597635273\\
505.01	0.00113257078001797\\
506.01	0.00120156743382579\\
507.01	0.00127288609169367\\
508.01	0.00134665908291764\\
509.01	0.00142303276495134\\
510.01	0.00150216972989477\\
511.01	0.00158425144337775\\
512.01	0.00166948141423347\\
513.01	0.00175808901851725\\
514.01	0.00185033413367008\\
515.01	0.00194651278013633\\
516.01	0.00204696402140443\\
517.01	0.00215207844313003\\
518.01	0.00226230862293083\\
519.01	0.00237818212183268\\
520.01	0.00250031768790579\\
521.01	0.00262944556163042\\
522.01	0.00276643301954448\\
523.01	0.00291231665078933\\
524.01	0.00306834333907414\\
525.01	0.00323576254272315\\
526.01	0.00341209916038765\\
527.01	0.00352028318888959\\
528.01	0.00360406428927963\\
529.01	0.00369034534720328\\
530.01	0.00377916342956684\\
531.01	0.0038705441192735\\
532.01	0.00396449684460019\\
533.01	0.00406100742768074\\
534.01	0.00416002261598433\\
535.01	0.00426144479503196\\
536.01	0.00436512542922912\\
537.01	0.0044708521261581\\
538.01	0.00457833263506474\\
539.01	0.00468717510863339\\
540.01	0.00479686376382003\\
541.01	0.00490672887868641\\
542.01	0.00501589943169712\\
543.01	0.0051231210642415\\
544.01	0.00522665268863226\\
545.01	0.00532577361565408\\
546.01	0.0054256685235166\\
547.01	0.00552662808831839\\
548.01	0.00562801372395479\\
549.01	0.00572894003628174\\
550.01	0.00582833573384003\\
551.01	0.00592867321793919\\
552.01	0.00603119774121374\\
553.01	0.00613561888154726\\
554.01	0.00624150777552676\\
555.01	0.00634796511694036\\
556.01	0.00645334557089851\\
557.01	0.00655717610827899\\
558.01	0.00665915981026362\\
559.01	0.00675913613247077\\
560.01	0.00685716507348205\\
561.01	0.00695316601339255\\
562.01	0.00704673033753052\\
563.01	0.00713750430682598\\
564.01	0.00722529375586857\\
565.01	0.00731113838958158\\
566.01	0.00739571234032634\\
567.01	0.00747902290501914\\
568.01	0.00756119384135522\\
569.01	0.00764227563154391\\
570.01	0.00772234228954321\\
571.01	0.00780154653142491\\
572.01	0.00788010948001083\\
573.01	0.0079582842939276\\
574.01	0.00803623522961562\\
575.01	0.0081139736515744\\
576.01	0.00819151365356962\\
577.01	0.00826889123290443\\
578.01	0.00834616108441969\\
579.01	0.00842339004022776\\
580.01	0.00850064667423554\\
581.01	0.00857798729576122\\
582.01	0.00865544109675605\\
583.01	0.00873301147036204\\
584.01	0.00881069437075346\\
585.01	0.00888848570938689\\
586.01	0.00896638332143378\\
587.01	0.00904438745879397\\
588.01	0.0091224999717733\\
589.01	0.00920072415476087\\
590.01	0.00927906581763692\\
591.01	0.00935753490711693\\
592.01	0.00943614492328837\\
593.01	0.00951491048379248\\
594.01	0.00959384452110459\\
595.01	0.00967295564797865\\
596.01	0.00975224633779213\\
597.01	0.00983171277036777\\
598.01	0.00990824635186025\\
599.01	0.0099705330205288\\
599.02	0.00997104253216608\\
599.03	0.00997154900812005\\
599.04	0.0099720524188544\\
599.05	0.00997255273454198\\
599.06	0.00997304992506192\\
599.07	0.00997354395999672\\
599.08	0.0099740348086293\\
599.09	0.00997452243994009\\
599.1	0.00997500682260398\\
599.11	0.00997548792498733\\
599.12	0.00997596571514491\\
599.13	0.00997644016081682\\
599.14	0.00997691122942541\\
599.15	0.00997737888807207\\
599.16	0.00997784310353411\\
599.17	0.00997830384226156\\
599.18	0.00997876107037387\\
599.19	0.00997921475365675\\
599.2	0.00997966485755874\\
599.21	0.00998011134718799\\
599.22	0.00998055418730882\\
599.23	0.00998099334233836\\
599.24	0.0099814287763431\\
599.25	0.00998186045303541\\
599.26	0.00998228833388642\\
599.27	0.00998271237873929\\
599.28	0.00998313254703763\\
599.29	0.00998354879782151\\
599.3	0.00998396108972346\\
599.31	0.0099843693809645\\
599.32	0.00998477362935002\\
599.33	0.00998517379226572\\
599.34	0.0099855698266734\\
599.35	0.00998596168910684\\
599.36	0.00998634933566753\\
599.37	0.00998673272202041\\
599.38	0.00998711180338954\\
599.39	0.0099874865345538\\
599.4	0.00998785686984241\\
599.41	0.00998822276313055\\
599.42	0.00998858416783486\\
599.43	0.00998894103690888\\
599.44	0.00998929332283851\\
599.45	0.0099896409776374\\
599.46	0.00998998395284222\\
599.47	0.00999032219950803\\
599.48	0.00999065566820346\\
599.49	0.00999098430900594\\
599.5	0.00999130807149682\\
599.51	0.0099916269047565\\
599.52	0.00999194075735945\\
599.53	0.00999224957736924\\
599.54	0.00999255331233348\\
599.55	0.00999285190927873\\
599.56	0.00999314531470532\\
599.57	0.0099934334745822\\
599.58	0.00999371633434168\\
599.59	0.00999399383887409\\
599.6	0.00999426593252248\\
599.61	0.00999453255907718\\
599.62	0.00999479366177034\\
599.63	0.00999504918327044\\
599.64	0.00999529906567672\\
599.65	0.0099955432505135\\
599.66	0.00999578167872461\\
599.67	0.00999601429066754\\
599.68	0.00999624102610774\\
599.69	0.0099964618242127\\
599.7	0.00999667662354612\\
599.71	0.00999688536206187\\
599.72	0.00999708797709802\\
599.73	0.00999728440537074\\
599.74	0.00999747458296815\\
599.75	0.00999765844534413\\
599.76	0.00999783592731206\\
599.77	0.00999800696303847\\
599.78	0.00999817148603667\\
599.79	0.0099983294291603\\
599.8	0.00999848072459683\\
599.81	0.00999862530386092\\
599.82	0.00999876309778785\\
599.83	0.00999889403652677\\
599.84	0.00999901804953394\\
599.85	0.00999913506556586\\
599.86	0.00999924501267239\\
599.87	0.00999934781818976\\
599.88	0.00999944340873354\\
599.89	0.00999953171019148\\
599.9	0.00999961264771636\\
599.91	0.00999968614571872\\
599.92	0.00999975212785955\\
599.93	0.00999981051704284\\
599.94	0.00999986123540816\\
599.95	0.00999990420432308\\
599.96	0.00999993934437554\\
599.97	0.00999996657536618\\
599.98	0.00999998581630055\\
599.99	0.00999999698538124\\
600	0.01\\
};
\addplot [color=mycolor18,solid,forget plot]
  table[row sep=crcr]{%
0.01	0.00105848678225644\\
1.01	0.00105848738535112\\
2.01	0.00105848800148826\\
3.01	0.00105848863095175\\
4.01	0.00105848927403169\\
5.01	0.00105848993102456\\
6.01	0.00105849060223337\\
7.01	0.0010584912879677\\
8.01	0.00105849198854398\\
9.01	0.00105849270428564\\
10.01	0.00105849343552312\\
11.01	0.0010584941825942\\
12.01	0.00105849494584402\\
13.01	0.00105849572562536\\
14.01	0.00105849652229869\\
15.01	0.00105849733623244\\
16.01	0.00105849816780314\\
17.01	0.00105849901739562\\
18.01	0.00105849988540314\\
19.01	0.00105850077222761\\
20.01	0.00105850167827988\\
21.01	0.00105850260397972\\
22.01	0.00105850354975626\\
23.01	0.00105850451604794\\
24.01	0.00105850550330294\\
25.01	0.00105850651197934\\
26.01	0.00105850754254529\\
27.01	0.00105850859547925\\
28.01	0.00105850967127027\\
29.01	0.00105851077041814\\
30.01	0.00105851189343364\\
31.01	0.00105851304083886\\
32.01	0.00105851421316747\\
33.01	0.00105851541096488\\
34.01	0.00105851663478848\\
35.01	0.00105851788520805\\
36.01	0.00105851916280597\\
37.01	0.00105852046817738\\
38.01	0.00105852180193065\\
39.01	0.00105852316468762\\
40.01	0.00105852455708381\\
41.01	0.00105852597976887\\
42.01	0.00105852743340677\\
43.01	0.0010585289186762\\
44.01	0.00105853043627081\\
45.01	0.00105853198689978\\
46.01	0.00105853357128772\\
47.01	0.00105853519017553\\
48.01	0.00105853684432037\\
49.01	0.00105853853449629\\
50.01	0.00105854026149447\\
51.01	0.00105854202612354\\
52.01	0.00105854382921019\\
53.01	0.0010585456715994\\
54.01	0.00105854755415486\\
55.01	0.0010585494777595\\
56.01	0.00105855144331582\\
57.01	0.00105855345174632\\
58.01	0.00105855550399404\\
59.01	0.00105855760102301\\
60.01	0.00105855974381854\\
61.01	0.00105856193338804\\
62.01	0.00105856417076119\\
63.01	0.00105856645699056\\
64.01	0.00105856879315229\\
65.01	0.00105857118034624\\
66.01	0.00105857361969698\\
67.01	0.00105857611235399\\
68.01	0.00105857865949237\\
69.01	0.00105858126231338\\
70.01	0.00105858392204514\\
71.01	0.00105858663994311\\
72.01	0.0010585894172907\\
73.01	0.00105859225539998\\
74.01	0.00105859515561232\\
75.01	0.00105859811929897\\
76.01	0.00105860114786185\\
77.01	0.00105860424273414\\
78.01	0.00105860740538106\\
79.01	0.00105861063730053\\
80.01	0.00105861394002404\\
81.01	0.00105861731511729\\
82.01	0.00105862076418094\\
83.01	0.0010586242888515\\
84.01	0.00105862789080213\\
85.01	0.00105863157174338\\
86.01	0.00105863533342422\\
87.01	0.00105863917763265\\
88.01	0.00105864310619679\\
89.01	0.00105864712098581\\
90.01	0.00105865122391072\\
91.01	0.00105865541692534\\
92.01	0.0010586597020274\\
93.01	0.0010586640812595\\
94.01	0.00105866855671\\
95.01	0.00105867313051422\\
96.01	0.00105867780485542\\
97.01	0.00105868258196597\\
98.01	0.00105868746412843\\
99.01	0.00105869245367658\\
100.01	0.00105869755299677\\
101.01	0.00105870276452912\\
102.01	0.00105870809076852\\
103.01	0.00105871353426618\\
104.01	0.00105871909763068\\
105.01	0.00105872478352945\\
106.01	0.00105873059469002\\
107.01	0.00105873653390139\\
108.01	0.00105874260401557\\
109.01	0.0010587488079488\\
110.01	0.00105875514868323\\
111.01	0.00105876162926835\\
112.01	0.00105876825282249\\
113.01	0.00105877502253445\\
114.01	0.0010587819416652\\
115.01	0.0010587890135493\\
116.01	0.00105879624159688\\
117.01	0.00105880362929518\\
118.01	0.00105881118021041\\
119.01	0.00105881889798953\\
120.01	0.00105882678636213\\
121.01	0.00105883484914234\\
122.01	0.00105884309023076\\
123.01	0.00105885151361645\\
124.01	0.00105886012337895\\
125.01	0.0010588689236905\\
126.01	0.00105887791881799\\
127.01	0.00105888711312527\\
128.01	0.00105889651107538\\
129.01	0.0010589061172328\\
130.01	0.00105891593626581\\
131.01	0.00105892597294894\\
132.01	0.0010589362321654\\
133.01	0.00105894671890951\\
134.01	0.00105895743828946\\
135.01	0.00105896839552988\\
136.01	0.00105897959597436\\
137.01	0.00105899104508845\\
138.01	0.00105900274846239\\
139.01	0.00105901471181413\\
140.01	0.00105902694099205\\
141.01	0.00105903944197819\\
142.01	0.00105905222089137\\
143.01	0.00105906528399023\\
144.01	0.00105907863767659\\
145.01	0.00105909228849883\\
146.01	0.00105910624315519\\
147.01	0.00105912050849735\\
148.01	0.00105913509153394\\
149.01	0.00105914999943434\\
150.01	0.00105916523953226\\
151.01	0.00105918081932976\\
152.01	0.00105919674650106\\
153.01	0.00105921302889656\\
154.01	0.00105922967454707\\
155.01	0.00105924669166803\\
156.01	0.00105926408866368\\
157.01	0.0010592818741317\\
158.01	0.00105930005686757\\
159.01	0.00105931864586935\\
160.01	0.00105933765034238\\
161.01	0.00105935707970407\\
162.01	0.00105937694358912\\
163.01	0.00105939725185433\\
164.01	0.00105941801458413\\
165.01	0.00105943924209567\\
166.01	0.00105946094494445\\
167.01	0.00105948313393003\\
168.01	0.00105950582010154\\
169.01	0.00105952901476369\\
170.01	0.00105955272948303\\
171.01	0.00105957697609368\\
172.01	0.00105960176670405\\
173.01	0.00105962711370317\\
174.01	0.0010596530297673\\
175.01	0.00105967952786677\\
176.01	0.00105970662127308\\
177.01	0.00105973432356576\\
178.01	0.00105976264863988\\
179.01	0.00105979161071355\\
180.01	0.00105982122433548\\
181.01	0.00105985150439283\\
182.01	0.00105988246611936\\
183.01	0.00105991412510368\\
184.01	0.0010599464972975\\
185.01	0.00105997959902451\\
186.01	0.00106001344698909\\
187.01	0.00106004805828544\\
188.01	0.00106008345040687\\
189.01	0.00106011964125534\\
190.01	0.00106015664915116\\
191.01	0.00106019449284302\\
192.01	0.00106023319151823\\
193.01	0.00106027276481314\\
194.01	0.00106031323282407\\
195.01	0.0010603546161181\\
196.01	0.00106039693574442\\
197.01	0.00106044021324591\\
198.01	0.00106048447067094\\
199.01	0.00106052973058548\\
200.01	0.00106057601608568\\
201.01	0.00106062335081033\\
202.01	0.00106067175895409\\
203.01	0.00106072126528075\\
204.01	0.00106077189513696\\
205.01	0.00106082367446634\\
206.01	0.00106087662982359\\
207.01	0.00106093078838949\\
208.01	0.00106098617798584\\
209.01	0.00106104282709084\\
210.01	0.0010611007648551\\
211.01	0.00106116002111779\\
212.01	0.00106122062642309\\
213.01	0.00106128261203758\\
214.01	0.00106134600996736\\
215.01	0.00106141085297604\\
216.01	0.00106147717460314\\
217.01	0.00106154500918266\\
218.01	0.00106161439186259\\
219.01	0.00106168535862437\\
220.01	0.00106175794630319\\
221.01	0.00106183219260864\\
222.01	0.00106190813614611\\
223.01	0.00106198581643813\\
224.01	0.00106206527394699\\
225.01	0.0010621465500974\\
226.01	0.00106222968729979\\
227.01	0.00106231472897445\\
228.01	0.00106240171957593\\
229.01	0.00106249070461822\\
230.01	0.00106258173070062\\
231.01	0.00106267484553389\\
232.01	0.00106277009796765\\
233.01	0.00106286753801781\\
234.01	0.00106296721689526\\
235.01	0.00106306918703472\\
236.01	0.00106317350212472\\
237.01	0.00106328021713815\\
238.01	0.00106338938836358\\
239.01	0.00106350107343741\\
240.01	0.0010636153313767\\
241.01	0.00106373222261286\\
242.01	0.0010638518090263\\
243.01	0.00106397415398172\\
244.01	0.0010640993223644\\
245.01	0.00106422738061739\\
246.01	0.00106435839677956\\
247.01	0.00106449244052472\\
248.01	0.00106462958320148\\
249.01	0.00106476989787419\\
250.01	0.00106491345936512\\
251.01	0.00106506034429726\\
252.01	0.00106521063113856\\
253.01	0.00106536440024697\\
254.01	0.00106552173391688\\
255.01	0.00106568271642637\\
256.01	0.00106584743408583\\
257.01	0.0010660159752879\\
258.01	0.00106618843055824\\
259.01	0.00106636489260803\\
260.01	0.00106654545638736\\
261.01	0.00106673021914015\\
262.01	0.00106691928046022\\
263.01	0.00106711274234932\\
264.01	0.00106731070927565\\
265.01	0.00106751328823452\\
266.01	0.00106772058881043\\
267.01	0.00106793272324031\\
268.01	0.00106814980647867\\
269.01	0.00106837195626415\\
270.01	0.0010685992931878\\
271.01	0.00106883194076299\\
272.01	0.00106907002549712\\
273.01	0.00106931367696482\\
274.01	0.00106956302788325\\
275.01	0.00106981821418908\\
276.01	0.00107007937511739\\
277.01	0.00107034665328251\\
278.01	0.00107062019476071\\
279.01	0.00107090014917514\\
280.01	0.00107118666978256\\
281.01	0.00107147991356245\\
282.01	0.00107178004130811\\
283.01	0.00107208721772004\\
284.01	0.00107240161150159\\
285.01	0.00107272339545691\\
286.01	0.00107305274659131\\
287.01	0.00107338984621411\\
288.01	0.00107373488004374\\
289.01	0.0010740880383158\\
290.01	0.00107444951589341\\
291.01	0.00107481951238035\\
292.01	0.00107519823223692\\
293.01	0.00107558588489869\\
294.01	0.00107598268489788\\
295.01	0.00107638885198819\\
296.01	0.00107680461127202\\
297.01	0.00107723019333135\\
298.01	0.00107766583436113\\
299.01	0.00107811177630678\\
300.01	0.00107856826700423\\
301.01	0.00107903556032364\\
302.01	0.00107951391631688\\
303.01	0.00108000360136799\\
304.01	0.00108050488834801\\
305.01	0.00108101805677283\\
306.01	0.0010815433929655\\
307.01	0.00108208119022206\\
308.01	0.0010826317489817\\
309.01	0.00108319537700081\\
310.01	0.00108377238953147\\
311.01	0.00108436310950421\\
312.01	0.00108496786771517\\
313.01	0.00108558700301802\\
314.01	0.00108622086252048\\
315.01	0.0010868698017856\\
316.01	0.0010875341850381\\
317.01	0.00108821438537576\\
318.01	0.00108891078498597\\
319.01	0.00108962377536789\\
320.01	0.00109035375755977\\
321.01	0.00109110114237215\\
322.01	0.00109186635062686\\
323.01	0.00109264981340199\\
324.01	0.00109345197228301\\
325.01	0.00109427327962022\\
326.01	0.00109511419879276\\
327.01	0.0010959752044792\\
328.01	0.00109685678293532\\
329.01	0.00109775943227858\\
330.01	0.00109868366278041\\
331.01	0.00109962999716548\\
332.01	0.00110059897091945\\
333.01	0.00110159113260404\\
334.01	0.0011026070441808\\
335.01	0.00110364728134346\\
336.01	0.00110471243385864\\
337.01	0.00110580310591621\\
338.01	0.00110691991648883\\
339.01	0.00110806349970088\\
340.01	0.00110923450520787\\
341.01	0.00111043359858607\\
342.01	0.00111166146173262\\
343.01	0.00111291879327696\\
344.01	0.00111420630900358\\
345.01	0.00111552474228666\\
346.01	0.00111687484453691\\
347.01	0.00111825738566099\\
348.01	0.0011196731545343\\
349.01	0.00112112295948693\\
350.01	0.00112260762880405\\
351.01	0.00112412801124041\\
352.01	0.00112568497655043\\
353.01	0.0011272794160332\\
354.01	0.00112891224309403\\
355.01	0.0011305843938226\\
356.01	0.0011322968275882\\
357.01	0.00113405052765306\\
358.01	0.001135846501804\\
359.01	0.00113768578300327\\
360.01	0.00113956943005924\\
361.01	0.00114149852831735\\
362.01	0.00114347419037252\\
363.01	0.00114549755680328\\
364.01	0.00114756979692848\\
365.01	0.00114969210958765\\
366.01	0.00115186572394509\\
367.01	0.00115409190031938\\
368.01	0.00115637193103841\\
369.01	0.00115870714132105\\
370.01	0.00116109889018636\\
371.01	0.001163548571391\\
372.01	0.00116605761439588\\
373.01	0.00116862748536304\\
374.01	0.00117125968818357\\
375.01	0.00117395576553731\\
376.01	0.00117671729998578\\
377.01	0.00117954591509881\\
378.01	0.00118244327661627\\
379.01	0.00118541109364588\\
380.01	0.00118845111989776\\
381.01	0.00119156515495732\\
382.01	0.00119475504559751\\
383.01	0.00119802268713113\\
384.01	0.0012013700248051\\
385.01	0.00120479905523727\\
386.01	0.00120831182789741\\
387.01	0.00121191044663376\\
388.01	0.00121559707124597\\
389.01	0.0012193739191066\\
390.01	0.00122324326683203\\
391.01	0.00122720745200462\\
392.01	0.00123126887494768\\
393.01	0.00123543000055477\\
394.01	0.00123969336017508\\
395.01	0.00124406155355684\\
396.01	0.00124853725085029\\
397.01	0.00125312319467234\\
398.01	0.00125782220223461\\
399.01	0.0012626371675372\\
400.01	0.00126757106362953\\
401.01	0.00127262694494131\\
402.01	0.00127780794968466\\
403.01	0.00128311730233054\\
404.01	0.00128855831616062\\
405.01	0.0012941343958978\\
406.01	0.0012998490404172\\
407.01	0.00130570584553994\\
408.01	0.00131170850691277\\
409.01	0.00131786082297575\\
410.01	0.00132416669802152\\
411.01	0.00133063014534897\\
412.01	0.00133725529051567\\
413.01	0.00134404637469249\\
414.01	0.00135100775812517\\
415.01	0.00135814392370794\\
416.01	0.00136545948067388\\
417.01	0.00137295916840727\\
418.01	0.00138064786038253\\
419.01	0.0013885305682341\\
420.01	0.00139661244596099\\
421.01	0.00140489879426959\\
422.01	0.00141339506506206\\
423.01	0.00142210686608133\\
424.01	0.0014310399657305\\
425.01	0.00144020029808806\\
426.01	0.00144959396814871\\
427.01	0.00145922725732696\\
428.01	0.00146910662927189\\
429.01	0.00147923873605477\\
430.01	0.00148963042480637\\
431.01	0.00150028874489973\\
432.01	0.0015112209557965\\
433.01	0.00152243453569886\\
434.01	0.00153393719117675\\
435.01	0.00154573686796874\\
436.01	0.00155784176318209\\
437.01	0.00157026033914148\\
438.01	0.00158300133914879\\
439.01	0.00159607380541275\\
440.01	0.00160948709936898\\
441.01	0.00162325092452457\\
442.01	0.00163737535179515\\
443.01	0.00165187084701578\\
444.01	0.00166674829984293\\
445.01	0.00168201905254234\\
446.01	0.00169769492508298\\
447.01	0.00171378822820518\\
448.01	0.00173031178236495\\
449.01	0.00174727895075731\\
450.01	0.00176470367668922\\
451.01	0.00178260052490536\\
452.01	0.00180098472742406\\
453.01	0.00181987223453527\\
454.01	0.00183927977172944\\
455.01	0.00185922490346439\\
456.01	0.00187972610484628\\
457.01	0.00190080284250637\\
458.01	0.00192247566620689\\
459.01	0.00194476631301662\\
460.01	0.00196769782627855\\
461.01	0.00199129469206012\\
462.01	0.00201558299636367\\
463.01	0.00204059060710265\\
464.01	0.00206634738576364\\
465.01	0.00209288543482385\\
466.01	0.00212023938844311\\
467.01	0.00214844675578658\\
468.01	0.00217754832866662\\
469.01	0.00220758866816539\\
470.01	0.00223861668869814\\
471.01	0.0022706863628443\\
472.01	0.00230385757652669\\
473.01	0.00233819717216527\\
474.01	0.00237378022781636\\
475.01	0.00241069163372452\\
476.01	0.00244902804508352\\
477.01	0.00248890031230211\\
478.01	0.00253043651942755\\
479.01	0.00257378580007774\\
480.01	0.00261912315093559\\
481.01	0.00266665552933484\\
482.01	0.00269227161590402\\
483.01	0.00271078460624117\\
484.01	0.00273013395473023\\
485.01	0.00275041047004711\\
486.01	0.00277172327321201\\
487.01	0.00279420440781401\\
488.01	0.00281801472867966\\
489.01	0.00284335144662574\\
490.01	0.00287045782404049\\
491.01	0.00289963567094933\\
492.01	0.00293126149660711\\
493.01	0.00296573006204356\\
494.01	0.00300177950119776\\
495.01	0.00303880186890301\\
496.01	0.00307682062314735\\
497.01	0.00311586050323926\\
498.01	0.00315594821145552\\
499.01	0.00319711341922895\\
500.01	0.00323939023031811\\
501.01	0.00328281928464718\\
502.01	0.00332745075815011\\
503.01	0.00337334860782597\\
504.01	0.00342059126416831\\
505.01	0.00346923151883183\\
506.01	0.00351930199092616\\
507.01	0.00357083237919477\\
508.01	0.00362384856478\\
509.01	0.00367837126556257\\
510.01	0.00373441432713721\\
511.01	0.00379198254778269\\
512.01	0.00385106890334631\\
513.01	0.00391165099619443\\
514.01	0.00397368649671831\\
515.01	0.00403710727145986\\
516.01	0.00410181179213596\\
517.01	0.00416765528565727\\
518.01	0.00423443690432335\\
519.01	0.00430188292873504\\
520.01	0.00436962463512348\\
521.01	0.00443716985579608\\
522.01	0.00450386647884781\\
523.01	0.0045688549468685\\
524.01	0.00463100624357699\\
525.01	0.00468910256065676\\
526.01	0.00474566599357769\\
527.01	0.00480209616197203\\
528.01	0.00486003753646892\\
529.01	0.00491957866578477\\
530.01	0.00498071188703882\\
531.01	0.00504342381267531\\
532.01	0.00510769845015802\\
533.01	0.00517352687775602\\
534.01	0.00524091998050024\\
535.01	0.00530988917763227\\
536.01	0.00538044046120933\\
537.01	0.0054525732359726\\
538.01	0.00552627887232867\\
539.01	0.00560153885915037\\
540.01	0.0056783223867673\\
541.01	0.00575658310551631\\
542.01	0.00583625478120799\\
543.01	0.00591724929042586\\
544.01	0.00599945128942453\\
545.01	0.00608206031341499\\
546.01	0.00616454974127394\\
547.01	0.00624667176592535\\
548.01	0.00632815937212857\\
549.01	0.0064087471286963\\
550.01	0.00648819414034955\\
551.01	0.0065662572997745\\
552.01	0.00664262514675434\\
553.01	0.00671701960745354\\
554.01	0.00678925865178991\\
555.01	0.00685962124728082\\
556.01	0.0069292250010685\\
557.01	0.00699811736386689\\
558.01	0.00706626972803542\\
559.01	0.00713368620788777\\
560.01	0.0072003987304087\\
561.01	0.00726646369715596\\
562.01	0.0073319849738537\\
563.01	0.00739712337039811\\
564.01	0.00746208685838925\\
565.01	0.00752705934569325\\
566.01	0.00759209712118145\\
567.01	0.00765724568092983\\
568.01	0.00772256084138674\\
569.01	0.00778810616998565\\
570.01	0.00785395204076872\\
571.01	0.00792016901483149\\
572.01	0.00798681920062948\\
573.01	0.00805394793018242\\
574.01	0.0081215814501862\\
575.01	0.00818974004445938\\
576.01	0.00825844526606624\\
577.01	0.0083277184552464\\
578.01	0.00839757873349423\\
579.01	0.00846804113095645\\
580.01	0.00853911525761656\\
581.01	0.00861080500418132\\
582.01	0.0086831096670022\\
583.01	0.00875602586849257\\
584.01	0.00882954825853327\\
585.01	0.0089036692345867\\
586.01	0.00897837850677948\\
587.01	0.00905366261691402\\
588.01	0.00912950454200949\\
589.01	0.00920588340671642\\
590.01	0.00928277421719717\\
591.01	0.00936014749648086\\
592.01	0.00943796891956704\\
593.01	0.00951619928974091\\
594.01	0.00959479511364908\\
595.01	0.00967370999015509\\
596.01	0.00975289702425711\\
597.01	0.00983231245145954\\
598.01	0.00990826323208284\\
599.01	0.00997053306307254\\
599.02	0.00997104257257093\\
599.03	0.00997154904647258\\
599.04	0.00997205245523854\\
599.05	0.00997255276903906\\
599.06	0.00997304995775075\\
599.07	0.00997354399095361\\
599.08	0.00997403483792814\\
599.09	0.00997452246765239\\
599.1	0.00997500684879892\\
599.11	0.00997548794973182\\
599.12	0.00997596573850364\\
599.13	0.00997644018285231\\
599.14	0.00997691125019803\\
599.15	0.00997737890764013\\
599.16	0.0099778431219539\\
599.17	0.00997830385958736\\
599.18	0.00997876108665806\\
599.19	0.00997921476894979\\
599.2	0.00997966487190928\\
599.21	0.00998011136064285\\
599.22	0.00998055419991309\\
599.23	0.00998099335413542\\
599.24	0.00998142878737464\\
599.25	0.00998186046334153\\
599.26	0.00998228834350557\\
599.27	0.0099827123877084\\
599.28	0.00998313255539213\\
599.29	0.00998354880559535\\
599.3	0.00998396109694922\\
599.31	0.00998436938767336\\
599.32	0.00998477363557184\\
599.33	0.00998517379802903\\
599.34	0.0099855698320055\\
599.35	0.0099859616940338\\
599.36	0.00998634934021422\\
599.37	0.00998673272621056\\
599.38	0.00998711180724576\\
599.39	0.00998748653809762\\
599.4	0.00998785687309431\\
599.41	0.00998822276610999\\
599.42	0.00998858417056033\\
599.43	0.00998894103939792\\
599.44	0.00998929332510775\\
599.45	0.00998964097970257\\
599.46	0.00998998395471822\\
599.47	0.00999032220120892\\
599.48	0.00999065566974251\\
599.49	0.00999098431039566\\
599.5	0.00999130807274899\\
599.51	0.0099916269058822\\
599.52	0.00999194075836907\\
599.53	0.00999224957827254\\
599.54	0.00999255331313958\\
599.55	0.00999285190999615\\
599.56	0.00999314531534203\\
599.57	0.00999343347514561\\
599.58	0.00999371633483869\\
599.59	0.0099939938393111\\
599.6	0.00999426593290542\\
599.61	0.00999453255941153\\
599.62	0.00999479366206116\\
599.63	0.00999504918352239\\
599.64	0.00999529906589405\\
599.65	0.00999554325070014\\
599.66	0.00999578167888412\\
599.67	0.00999601429080317\\
599.68	0.00999624102622244\\
599.69	0.00999646182430915\\
599.7	0.00999667662362671\\
599.71	0.00999688536212877\\
599.72	0.00999708797715316\\
599.73	0.00999728440541583\\
599.74	0.00999747458300472\\
599.75	0.00999765844537353\\
599.76	0.00999783592733545\\
599.77	0.00999800696305688\\
599.78	0.00999817148605099\\
599.79	0.0099983294291713\\
599.8	0.00999848072460515\\
599.81	0.00999862530386712\\
599.82	0.00999876309779239\\
599.83	0.00999889403653002\\
599.84	0.00999901804953621\\
599.85	0.00999913506556741\\
599.86	0.00999924501267341\\
599.87	0.00999934781819042\\
599.88	0.00999944340873394\\
599.89	0.00999953171019171\\
599.9	0.00999961264771648\\
599.91	0.00999968614571878\\
599.92	0.00999975212785957\\
599.93	0.00999981051704285\\
599.94	0.00999986123540817\\
599.95	0.00999990420432308\\
599.96	0.00999993934437554\\
599.97	0.00999996657536618\\
599.98	0.00999998581630055\\
599.99	0.00999999698538124\\
600	0.01\\
};
\addplot [color=red!25!mycolor17,solid,forget plot]
  table[row sep=crcr]{%
0.01	0.00260539218361255\\
1.01	0.00260539264278432\\
2.01	0.00260539311185838\\
3.01	0.00260539359104946\\
4.01	0.00260539408057708\\
5.01	0.00260539458066539\\
6.01	0.00260539509154378\\
7.01	0.00260539561344637\\
8.01	0.00260539614661254\\
9.01	0.00260539669128666\\
10.01	0.00260539724771883\\
11.01	0.00260539781616441\\
12.01	0.00260539839688419\\
13.01	0.00260539899014492\\
14.01	0.0026053995962191\\
15.01	0.00260540021538518\\
16.01	0.00260540084792776\\
17.01	0.00260540149413763\\
18.01	0.00260540215431197\\
19.01	0.00260540282875445\\
20.01	0.00260540351777529\\
21.01	0.00260540422169167\\
22.01	0.00260540494082767\\
23.01	0.00260540567551457\\
24.01	0.00260540642609078\\
25.01	0.00260540719290203\\
26.01	0.00260540797630193\\
27.01	0.00260540877665164\\
28.01	0.00260540959432021\\
29.01	0.00260541042968503\\
30.01	0.00260541128313146\\
31.01	0.00260541215505362\\
32.01	0.00260541304585414\\
33.01	0.00260541395594439\\
34.01	0.00260541488574507\\
35.01	0.00260541583568576\\
36.01	0.00260541680620577\\
37.01	0.00260541779775401\\
38.01	0.00260541881078921\\
39.01	0.00260541984578019\\
40.01	0.00260542090320616\\
41.01	0.00260542198355669\\
42.01	0.00260542308733244\\
43.01	0.00260542421504464\\
44.01	0.00260542536721622\\
45.01	0.00260542654438137\\
46.01	0.00260542774708618\\
47.01	0.00260542897588868\\
48.01	0.00260543023135934\\
49.01	0.00260543151408094\\
50.01	0.00260543282464934\\
51.01	0.00260543416367354\\
52.01	0.00260543553177579\\
53.01	0.00260543692959229\\
54.01	0.0026054383577731\\
55.01	0.00260543981698275\\
56.01	0.00260544130790036\\
57.01	0.00260544283122015\\
58.01	0.00260544438765167\\
59.01	0.00260544597791984\\
60.01	0.00260544760276615\\
61.01	0.00260544926294792\\
62.01	0.0026054509592396\\
63.01	0.0026054526924327\\
64.01	0.00260545446333606\\
65.01	0.00260545627277671\\
66.01	0.00260545812159967\\
67.01	0.0026054600106689\\
68.01	0.00260546194086735\\
69.01	0.00260546391309763\\
70.01	0.00260546592828229\\
71.01	0.00260546798736432\\
72.01	0.00260547009130767\\
73.01	0.00260547224109756\\
74.01	0.00260547443774112\\
75.01	0.00260547668226792\\
76.01	0.00260547897573006\\
77.01	0.00260548131920335\\
78.01	0.00260548371378731\\
79.01	0.002605486160606\\
80.01	0.00260548866080827\\
81.01	0.00260549121556854\\
82.01	0.00260549382608743\\
83.01	0.00260549649359222\\
84.01	0.00260549921933741\\
85.01	0.00260550200460542\\
86.01	0.00260550485070714\\
87.01	0.00260550775898287\\
88.01	0.0026055107308026\\
89.01	0.00260551376756671\\
90.01	0.00260551687070691\\
91.01	0.00260552004168674\\
92.01	0.00260552328200254\\
93.01	0.00260552659318374\\
94.01	0.00260552997679393\\
95.01	0.00260553343443186\\
96.01	0.00260553696773154\\
97.01	0.00260554057836374\\
98.01	0.00260554426803632\\
99.01	0.00260554803849547\\
100.01	0.00260555189152625\\
101.01	0.00260555582895361\\
102.01	0.00260555985264348\\
103.01	0.00260556396450329\\
104.01	0.00260556816648316\\
105.01	0.00260557246057697\\
106.01	0.00260557684882302\\
107.01	0.00260558133330541\\
108.01	0.00260558591615481\\
109.01	0.00260559059954978\\
110.01	0.00260559538571746\\
111.01	0.00260560027693499\\
112.01	0.00260560527553061\\
113.01	0.00260561038388478\\
114.01	0.00260561560443133\\
115.01	0.00260562093965878\\
116.01	0.00260562639211136\\
117.01	0.00260563196439052\\
118.01	0.00260563765915636\\
119.01	0.00260564347912827\\
120.01	0.00260564942708721\\
121.01	0.00260565550587662\\
122.01	0.00260566171840369\\
123.01	0.00260566806764113\\
124.01	0.00260567455662879\\
125.01	0.00260568118847448\\
126.01	0.00260568796635635\\
127.01	0.00260569489352419\\
128.01	0.00260570197330107\\
129.01	0.00260570920908465\\
130.01	0.00260571660434966\\
131.01	0.00260572416264891\\
132.01	0.00260573188761569\\
133.01	0.00260573978296506\\
134.01	0.00260574785249615\\
135.01	0.002605756100094\\
136.01	0.00260576452973134\\
137.01	0.00260577314547086\\
138.01	0.00260578195146702\\
139.01	0.00260579095196832\\
140.01	0.00260580015131968\\
141.01	0.00260580955396407\\
142.01	0.00260581916444541\\
143.01	0.00260582898741055\\
144.01	0.00260583902761151\\
145.01	0.00260584928990837\\
146.01	0.0026058597792713\\
147.01	0.00260587050078337\\
148.01	0.00260588145964312\\
149.01	0.00260589266116713\\
150.01	0.00260590411079301\\
151.01	0.00260591581408147\\
152.01	0.00260592777672031\\
153.01	0.00260594000452657\\
154.01	0.00260595250344964\\
155.01	0.00260596527957435\\
156.01	0.00260597833912422\\
157.01	0.00260599168846465\\
158.01	0.00260600533410618\\
159.01	0.00260601928270789\\
160.01	0.00260603354108081\\
161.01	0.00260604811619151\\
162.01	0.0026060630151654\\
163.01	0.00260607824529115\\
164.01	0.00260609381402381\\
165.01	0.00260610972898892\\
166.01	0.00260612599798661\\
167.01	0.0026061426289954\\
168.01	0.00260615963017667\\
169.01	0.00260617700987879\\
170.01	0.00260619477664135\\
171.01	0.00260621293919987\\
172.01	0.00260623150649023\\
173.01	0.00260625048765341\\
174.01	0.00260626989204034\\
175.01	0.0026062897292168\\
176.01	0.00260631000896819\\
177.01	0.00260633074130517\\
178.01	0.00260635193646866\\
179.01	0.00260637360493511\\
180.01	0.00260639575742241\\
181.01	0.00260641840489532\\
182.01	0.00260644155857133\\
183.01	0.00260646522992639\\
184.01	0.00260648943070152\\
185.01	0.00260651417290868\\
186.01	0.00260653946883697\\
187.01	0.00260656533105974\\
188.01	0.00260659177244078\\
189.01	0.00260661880614148\\
190.01	0.00260664644562769\\
191.01	0.0026066747046771\\
192.01	0.00260670359738631\\
193.01	0.002606733138179\\
194.01	0.00260676334181278\\
195.01	0.00260679422338786\\
196.01	0.00260682579835496\\
197.01	0.00260685808252338\\
198.01	0.00260689109206985\\
199.01	0.00260692484354698\\
200.01	0.00260695935389256\\
201.01	0.0026069946404383\\
202.01	0.00260703072091945\\
203.01	0.00260706761348461\\
204.01	0.00260710533670494\\
205.01	0.00260714390958494\\
206.01	0.00260718335157251\\
207.01	0.0026072236825693\\
208.01	0.00260726492294198\\
209.01	0.00260730709353309\\
210.01	0.00260735021567236\\
211.01	0.00260739431118871\\
212.01	0.00260743940242193\\
213.01	0.0026074855122349\\
214.01	0.00260753266402636\\
215.01	0.0026075808817435\\
216.01	0.00260763018989545\\
217.01	0.00260768061356641\\
218.01	0.00260773217842991\\
219.01	0.00260778491076252\\
220.01	0.0026078388374586\\
221.01	0.00260789398604541\\
222.01	0.00260795038469781\\
223.01	0.00260800806225447\\
224.01	0.00260806704823355\\
225.01	0.00260812737284906\\
226.01	0.00260818906702792\\
227.01	0.00260825216242715\\
228.01	0.00260831669145122\\
229.01	0.00260838268727069\\
230.01	0.00260845018384015\\
231.01	0.00260851921591781\\
232.01	0.00260858981908457\\
233.01	0.00260866202976429\\
234.01	0.00260873588524395\\
235.01	0.00260881142369492\\
236.01	0.00260888868419415\\
237.01	0.00260896770674672\\
238.01	0.00260904853230784\\
239.01	0.00260913120280634\\
240.01	0.00260921576116828\\
241.01	0.0026093022513415\\
242.01	0.00260939071832027\\
243.01	0.00260948120817091\\
244.01	0.00260957376805804\\
245.01	0.00260966844627167\\
246.01	0.00260976529225425\\
247.01	0.00260986435662919\\
248.01	0.00260996569122962\\
249.01	0.00261006934912843\\
250.01	0.00261017538466818\\
251.01	0.00261028385349261\\
252.01	0.00261039481257819\\
253.01	0.00261050832026741\\
254.01	0.00261062443630211\\
255.01	0.00261074322185756\\
256.01	0.00261086473957832\\
257.01	0.00261098905361386\\
258.01	0.0026111162296563\\
259.01	0.00261124633497781\\
260.01	0.00261137943846972\\
261.01	0.00261151561068285\\
262.01	0.00261165492386814\\
263.01	0.00261179745201874\\
264.01	0.0026119432709131\\
265.01	0.00261209245815903\\
266.01	0.00261224509323914\\
267.01	0.00261240125755702\\
268.01	0.00261256103448493\\
269.01	0.00261272450941283\\
270.01	0.00261289176979809\\
271.01	0.00261306290521685\\
272.01	0.00261323800741661\\
273.01	0.00261341717037032\\
274.01	0.00261360049033131\\
275.01	0.00261378806589024\\
276.01	0.00261397999803324\\
277.01	0.00261417639020134\\
278.01	0.00261437734835157\\
279.01	0.00261458298101974\\
280.01	0.00261479339938464\\
281.01	0.00261500871733376\\
282.01	0.00261522905153085\\
283.01	0.00261545452148525\\
284.01	0.00261568524962291\\
285.01	0.00261592136135881\\
286.01	0.00261616298517209\\
287.01	0.00261641025268221\\
288.01	0.00261666329872784\\
289.01	0.00261692226144654\\
290.01	0.00261718728235789\\
291.01	0.00261745850644764\\
292.01	0.0026177360822548\\
293.01	0.00261802016196015\\
294.01	0.00261831090147771\\
295.01	0.002618608460548\\
296.01	0.00261891300283368\\
297.01	0.00261922469601805\\
298.01	0.00261954371190569\\
299.01	0.00261987022652566\\
300.01	0.0026202044202371\\
301.01	0.0026205464778384\\
302.01	0.0026208965886776\\
303.01	0.00262125494676742\\
304.01	0.00262162175090123\\
305.01	0.00262199720477367\\
306.01	0.00262238151710291\\
307.01	0.00262277490175722\\
308.01	0.00262317757788366\\
309.01	0.0026235897700408\\
310.01	0.00262401170833447\\
311.01	0.00262444362855645\\
312.01	0.00262488577232747\\
313.01	0.00262533838724324\\
314.01	0.00262580172702445\\
315.01	0.00262627605167021\\
316.01	0.00262676162761559\\
317.01	0.002627258727893\\
318.01	0.00262776763229758\\
319.01	0.00262828862755671\\
320.01	0.00262882200750381\\
321.01	0.00262936807325656\\
322.01	0.00262992713339941\\
323.01	0.00263049950417052\\
324.01	0.00263108550965396\\
325.01	0.00263168548197555\\
326.01	0.00263229976150509\\
327.01	0.00263292869706222\\
328.01	0.00263357264612828\\
329.01	0.00263423197506316\\
330.01	0.0026349070593274\\
331.01	0.00263559828371032\\
332.01	0.00263630604256286\\
333.01	0.00263703074003734\\
334.01	0.00263777279033259\\
335.01	0.00263853261794493\\
336.01	0.00263931065792652\\
337.01	0.00264010735614821\\
338.01	0.00264092316957112\\
339.01	0.00264175856652375\\
340.01	0.00264261402698605\\
341.01	0.00264349004288121\\
342.01	0.00264438711837414\\
343.01	0.00264530577017809\\
344.01	0.00264624652786823\\
345.01	0.00264720993420414\\
346.01	0.00264819654545972\\
347.01	0.00264920693176177\\
348.01	0.00265024167743746\\
349.01	0.00265130138137022\\
350.01	0.00265238665736549\\
351.01	0.00265349813452539\\
352.01	0.00265463645763316\\
353.01	0.00265580228754836\\
354.01	0.00265699630161169\\
355.01	0.00265821919406109\\
356.01	0.0026594716764589\\
357.01	0.00266075447813029\\
358.01	0.00266206834661378\\
359.01	0.00266341404812427\\
360.01	0.0026647923680284\\
361.01	0.00266620411133403\\
362.01	0.00266765010319234\\
363.01	0.00266913118941507\\
364.01	0.00267064823700659\\
365.01	0.00267220213471042\\
366.01	0.00267379379357369\\
367.01	0.0026754241475262\\
368.01	0.00267709415397809\\
369.01	0.00267880479443545\\
370.01	0.00268055707513433\\
371.01	0.00268235202769494\\
372.01	0.00268419070979604\\
373.01	0.00268607420587073\\
374.01	0.00268800362782412\\
375.01	0.0026899801157749\\
376.01	0.00269200483882065\\
377.01	0.00269407899582866\\
378.01	0.00269620381625282\\
379.01	0.00269838056097833\\
380.01	0.00270061052319469\\
381.01	0.00270289502929927\\
382.01	0.0027052354398312\\
383.01	0.00270763315043887\\
384.01	0.00271008959288041\\
385.01	0.00271260623606003\\
386.01	0.0027151845871016\\
387.01	0.00271782619245971\\
388.01	0.00272053263907178\\
389.01	0.00272330555555127\\
390.01	0.00272614661342452\\
391.01	0.0027290575284125\\
392.01	0.00273204006175974\\
393.01	0.00273509602161143\\
394.01	0.00273822726444185\\
395.01	0.00274143569653397\\
396.01	0.00274472327551371\\
397.01	0.00274809201193918\\
398.01	0.00275154397094765\\
399.01	0.00275508127395972\\
400.01	0.0027587061004439\\
401.01	0.00276242068974007\\
402.01	0.00276622734294342\\
403.01	0.00277012842484678\\
404.01	0.00277412636594257\\
405.01	0.00277822366447975\\
406.01	0.00278242288857405\\
407.01	0.00278672667836767\\
408.01	0.00279113774823274\\
409.01	0.00279565888901398\\
410.01	0.00280029297030408\\
411.01	0.00280504294274656\\
412.01	0.00280991184036238\\
413.01	0.00281490278289938\\
414.01	0.00282001897820912\\
415.01	0.00282526372466241\\
416.01	0.00283064041362675\\
417.01	0.00283615253204045\\
418.01	0.00284180366513195\\
419.01	0.0028475974993442\\
420.01	0.00285353782552022\\
421.01	0.00285962854237615\\
422.01	0.00286587366022172\\
423.01	0.00287227730489154\\
424.01	0.00287884372188733\\
425.01	0.00288557728073854\\
426.01	0.00289248247958418\\
427.01	0.00289956394997337\\
428.01	0.00290682646187324\\
429.01	0.00291427492886374\\
430.01	0.00292191441348548\\
431.01	0.00292975013268902\\
432.01	0.00293778746331484\\
433.01	0.00294603194750792\\
434.01	0.00295448929794276\\
435.01	0.00296316540270304\\
436.01	0.00297206632962868\\
437.01	0.00298119832991059\\
438.01	0.00299056784069227\\
439.01	0.00300018148642835\\
440.01	0.00301004607877679\\
441.01	0.0030201686148775\\
442.01	0.00303055627403\\
443.01	0.00304121641308117\\
444.01	0.00305215656133123\\
445.01	0.00306338441657139\\
446.01	0.00307490784610943\\
447.01	0.00308673490179431\\
448.01	0.00309887383173713\\
449.01	0.00311133308019725\\
450.01	0.00312412128530121\\
451.01	0.00313724727516976\\
452.01	0.00315072006209792\\
453.01	0.0031645488343562\\
454.01	0.00317874294508019\\
455.01	0.00319331189759522\\
456.01	0.00320826532636696\\
457.01	0.00322361297257633\\
458.01	0.00323936465306914\\
459.01	0.00325553022112308\\
460.01	0.00327211951707591\\
461.01	0.003289142306356\\
462.01	0.00330660820181151\\
463.01	0.00332452656640618\\
464.01	0.00334290639129109\\
465.01	0.00336175614289581\\
466.01	0.00338108357092422\\
467.01	0.0034008954668763\\
468.01	0.00342119735979278\\
469.01	0.00344199313214273\\
470.01	0.00346328453390141\\
471.01	0.00348507056656741\\
472.01	0.00350734670073966\\
473.01	0.00353010388038894\\
474.01	0.00355332725343989\\
475.01	0.00357699455088265\\
476.01	0.00360107401448095\\
477.01	0.00362552174478718\\
478.01	0.00365027829404276\\
479.01	0.00367526426636504\\
480.01	0.00370037461680556\\
481.01	0.00372547124262447\\
482.01	0.00375060948748071\\
483.01	0.00377623935259928\\
484.01	0.00380234984261363\\
485.01	0.0038288932514617\\
486.01	0.00385580378298964\\
487.01	0.00388299235541427\\
488.01	0.00391033988664548\\
489.01	0.0039376885964225\\
490.01	0.00396483071355128\\
491.01	0.00399149378017413\\
492.01	0.00401732148341034\\
493.01	0.00404192614285756\\
494.01	0.00406657881939221\\
495.01	0.0040919252318317\\
496.01	0.00411798501236087\\
497.01	0.00414477881907857\\
498.01	0.00417232844923256\\
499.01	0.00420065688942448\\
500.01	0.00422978823437099\\
501.01	0.00425974736012985\\
502.01	0.00429055910883907\\
503.01	0.00432224704114471\\
504.01	0.0043548320494991\\
505.01	0.00438833173951764\\
506.01	0.00442276120155938\\
507.01	0.00445813274145353\\
508.01	0.00449445526498889\\
509.01	0.00453173361332232\\
510.01	0.00456996787378437\\
511.01	0.00460915270816233\\
512.01	0.0046492767673382\\
513.01	0.00469032230037527\\
514.01	0.00473226512295134\\
515.01	0.00477507519167798\\
516.01	0.00481871814741117\\
517.01	0.00486315835623537\\
518.01	0.0049083642110119\\
519.01	0.00495431973710523\\
520.01	0.00500104011690757\\
521.01	0.00504856498876362\\
522.01	0.00509696564029012\\
523.01	0.00514636071447201\\
524.01	0.00519693833483995\\
525.01	0.00524898527419951\\
526.01	0.00530278659125299\\
527.01	0.00535847817748133\\
528.01	0.0054161337440972\\
529.01	0.00547577716078581\\
530.01	0.0055373886186283\\
531.01	0.00560088258551496\\
532.01	0.00566607636916837\\
533.01	0.0057322112561036\\
534.01	0.00579862118574934\\
535.01	0.00586514881663358\\
536.01	0.00593161618896077\\
537.01	0.00599782500740147\\
538.01	0.00606355839620054\\
539.01	0.0061285849129024\\
540.01	0.00619266594452595\\
541.01	0.00625556808300582\\
542.01	0.00631708273741376\\
543.01	0.00637705609183324\\
544.01	0.00643544340692038\\
545.01	0.00649303522913385\\
546.01	0.00655024829454778\\
547.01	0.00660701048841029\\
548.01	0.00666326784594137\\
549.01	0.00671899048338381\\
550.01	0.00677417826709985\\
551.01	0.00682886587750969\\
552.01	0.00688312947996752\\
553.01	0.00693709029534157\\
554.01	0.00699091008589604\\
555.01	0.00704476798888722\\
556.01	0.00709875436978568\\
557.01	0.00715290192790601\\
558.01	0.00720725297367087\\
559.01	0.00726185953489205\\
560.01	0.00731678141212418\\
561.01	0.00737208380128448\\
562.01	0.00742783350425905\\
563.01	0.00748409308545202\\
564.01	0.00754091443505246\\
565.01	0.00759833497221399\\
566.01	0.00765638572821262\\
567.01	0.00771509830781651\\
568.01	0.00777450411224081\\
569.01	0.00783463303892859\\
570.01	0.0078955121620811\\
571.01	0.00795716466522816\\
572.01	0.00801960939307402\\
573.01	0.00808286126938456\\
574.01	0.00814693252999731\\
575.01	0.00821183366184495\\
576.01	0.00827757317985894\\
577.01	0.00834415715868387\\
578.01	0.008411588870687\\
579.01	0.00847986856416004\\
580.01	0.00854899337845039\\
581.01	0.00861895734225392\\
582.01	0.00868975134008659\\
583.01	0.0087613629125818\\
584.01	0.00883377590927962\\
585.01	0.00890697012437737\\
586.01	0.008980920965974\\
587.01	0.00905559918081232\\
588.01	0.00913097065499155\\
589.01	0.00920699631211276\\
590.01	0.00928363214520442\\
591.01	0.00936082944810129\\
592.01	0.00943853534123744\\
593.01	0.0095166936882987\\
594.01	0.00959524649431259\\
595.01	0.00967413588099248\\
596.01	0.00975330674558366\\
597.01	0.00983271022700933\\
598.01	0.00990826330213188\\
599.01	0.00997053306356642\\
599.02	0.0099710425730354\\
599.03	0.0099715490469091\\
599.04	0.00997205245564851\\
599.05	0.00997255276942385\\
599.06	0.00997304995811164\\
599.07	0.00997354399129185\\
599.08	0.00997403483824492\\
599.09	0.00997452246794885\\
599.1	0.00997500684907616\\
599.11	0.00997548794999088\\
599.12	0.00997596573874551\\
599.13	0.00997644018307794\\
599.14	0.00997691125040834\\
599.15	0.009977378907836\\
599.16	0.00997784312213615\\
599.17	0.00997830385975678\\
599.18	0.00997876108681542\\
599.19	0.0099792147690958\\
599.2	0.00997966487204462\\
599.21	0.00998011136076818\\
599.22	0.00998055420002904\\
599.23	0.00998099335424256\\
599.24	0.00998142878747354\\
599.25	0.00998186046343272\\
599.26	0.00998228834358955\\
599.27	0.00998271238778565\\
599.28	0.00998313255546309\\
599.29	0.00998354880566047\\
599.3	0.00998396109700889\\
599.31	0.00998436938772796\\
599.32	0.00998477363562173\\
599.33	0.00998517379807456\\
599.34	0.00998556983204699\\
599.35	0.00998596169407154\\
599.36	0.0099863493402485\\
599.37	0.00998673272624164\\
599.38	0.0099871118072739\\
599.39	0.00998748653812304\\
599.4	0.00998785687311724\\
599.41	0.00998822276613064\\
599.42	0.00998858417057887\\
599.43	0.00998894103941455\\
599.44	0.00998929332512263\\
599.45	0.00998964097971585\\
599.46	0.00998998395473005\\
599.47	0.00999032220121943\\
599.48	0.00999065566975183\\
599.49	0.0099909843104039\\
599.5	0.00999130807275626\\
599.51	0.00999162690588858\\
599.52	0.00999194075837467\\
599.53	0.00999224957827743\\
599.54	0.00999255331314384\\
599.55	0.00999285190999985\\
599.56	0.00999314531534523\\
599.57	0.00999343347514837\\
599.58	0.00999371633484106\\
599.59	0.00999399383931313\\
599.6	0.00999426593290714\\
599.61	0.00999453255941299\\
599.62	0.00999479366206239\\
599.63	0.00999504918352342\\
599.64	0.00999529906589491\\
599.65	0.00999554325070085\\
599.66	0.00999578167888471\\
599.67	0.00999601429080365\\
599.68	0.00999624102622283\\
599.69	0.00999646182430947\\
599.7	0.00999667662362697\\
599.71	0.00999688536212897\\
599.72	0.00999708797715331\\
599.73	0.00999728440541595\\
599.74	0.00999747458300482\\
599.75	0.0099976584453736\\
599.76	0.00999783592733551\\
599.77	0.00999800696305692\\
599.78	0.00999817148605102\\
599.79	0.00999832942917133\\
599.8	0.00999848072460517\\
599.81	0.00999862530386713\\
599.82	0.00999876309779239\\
599.83	0.00999889403653003\\
599.84	0.00999901804953622\\
599.85	0.00999913506556741\\
599.86	0.00999924501267341\\
599.87	0.00999934781819042\\
599.88	0.00999944340873394\\
599.89	0.0099995317101917\\
599.9	0.00999961264771648\\
599.91	0.00999968614571878\\
599.92	0.00999975212785958\\
599.93	0.00999981051704285\\
599.94	0.00999986123540817\\
599.95	0.00999990420432308\\
599.96	0.00999993934437554\\
599.97	0.00999996657536618\\
599.98	0.00999998581630055\\
599.99	0.00999999698538124\\
600	0.01\\
};
\addplot [color=mycolor19,solid,forget plot]
  table[row sep=crcr]{%
0.01	0.00343559417631374\\
1.01	0.00343559453459977\\
2.01	0.00343559490056855\\
3.01	0.00343559527438557\\
4.01	0.00343559565621998\\
5.01	0.00343559604624476\\
6.01	0.00343559644463601\\
7.01	0.00343559685157423\\
8.01	0.00343559726724352\\
9.01	0.00343559769183228\\
10.01	0.0034355981255328\\
11.01	0.00343559856854134\\
12.01	0.0034355990210589\\
13.01	0.00343559948329042\\
14.01	0.00343559995544546\\
15.01	0.00343560043773815\\
16.01	0.00343560093038708\\
17.01	0.00343560143361579\\
18.01	0.00343560194765262\\
19.01	0.00343560247273081\\
20.01	0.00343560300908884\\
21.01	0.00343560355697012\\
22.01	0.00343560411662333\\
23.01	0.00343560468830276\\
24.01	0.00343560527226813\\
25.01	0.0034356058687849\\
26.01	0.00343560647812413\\
27.01	0.00343560710056288\\
28.01	0.00343560773638417\\
29.01	0.00343560838587707\\
30.01	0.00343560904933735\\
31.01	0.00343560972706672\\
32.01	0.00343561041937369\\
33.01	0.00343561112657348\\
34.01	0.00343561184898813\\
35.01	0.00343561258694682\\
36.01	0.00343561334078553\\
37.01	0.00343561411084795\\
38.01	0.00343561489748496\\
39.01	0.00343561570105522\\
40.01	0.00343561652192528\\
41.01	0.00343561736046957\\
42.01	0.00343561821707056\\
43.01	0.00343561909211944\\
44.01	0.0034356199860154\\
45.01	0.00343562089916669\\
46.01	0.00343562183199032\\
47.01	0.00343562278491255\\
48.01	0.0034356237583687\\
49.01	0.00343562475280382\\
50.01	0.0034356257686723\\
51.01	0.00343562680643879\\
52.01	0.00343562786657808\\
53.01	0.00343562894957501\\
54.01	0.00343563005592538\\
55.01	0.00343563118613566\\
56.01	0.00343563234072308\\
57.01	0.0034356335202166\\
58.01	0.00343563472515658\\
59.01	0.00343563595609524\\
60.01	0.00343563721359651\\
61.01	0.00343563849823729\\
62.01	0.00343563981060649\\
63.01	0.00343564115130619\\
64.01	0.00343564252095148\\
65.01	0.00343564392017094\\
66.01	0.00343564534960719\\
67.01	0.00343564680991645\\
68.01	0.00343564830176954\\
69.01	0.00343564982585194\\
70.01	0.00343565138286406\\
71.01	0.00343565297352166\\
72.01	0.0034356545985562\\
73.01	0.00343565625871528\\
74.01	0.0034356579547626\\
75.01	0.0034356596874788\\
76.01	0.00343566145766162\\
77.01	0.0034356632661261\\
78.01	0.00343566511370523\\
79.01	0.00343566700125036\\
80.01	0.0034356689296314\\
81.01	0.00343567089973741\\
82.01	0.00343567291247676\\
83.01	0.00343567496877807\\
84.01	0.00343567706959001\\
85.01	0.0034356792158823\\
86.01	0.00343568140864589\\
87.01	0.00343568364889331\\
88.01	0.00343568593765953\\
89.01	0.0034356882760023\\
90.01	0.00343569066500228\\
91.01	0.00343569310576429\\
92.01	0.00343569559941705\\
93.01	0.00343569814711428\\
94.01	0.00343570075003523\\
95.01	0.00343570340938464\\
96.01	0.00343570612639409\\
97.01	0.00343570890232213\\
98.01	0.00343571173845511\\
99.01	0.00343571463610742\\
100.01	0.0034357175966228\\
101.01	0.00343572062137415\\
102.01	0.00343572371176479\\
103.01	0.00343572686922904\\
104.01	0.00343573009523267\\
105.01	0.00343573339127373\\
106.01	0.00343573675888344\\
107.01	0.00343574019962674\\
108.01	0.00343574371510288\\
109.01	0.00343574730694651\\
110.01	0.00343575097682841\\
111.01	0.0034357547264561\\
112.01	0.00343575855757483\\
113.01	0.00343576247196846\\
114.01	0.00343576647145993\\
115.01	0.00343577055791282\\
116.01	0.00343577473323166\\
117.01	0.00343577899936305\\
118.01	0.00343578335829654\\
119.01	0.00343578781206608\\
120.01	0.00343579236274995\\
121.01	0.00343579701247274\\
122.01	0.00343580176340601\\
123.01	0.00343580661776942\\
124.01	0.00343581157783149\\
125.01	0.00343581664591136\\
126.01	0.00343582182437925\\
127.01	0.00343582711565779\\
128.01	0.00343583252222322\\
129.01	0.0034358380466071\\
130.01	0.00343584369139657\\
131.01	0.00343584945923642\\
132.01	0.00343585535282986\\
133.01	0.00343586137494031\\
134.01	0.00343586752839228\\
135.01	0.00343587381607297\\
136.01	0.00343588024093384\\
137.01	0.00343588680599172\\
138.01	0.00343589351433037\\
139.01	0.00343590036910227\\
140.01	0.00343590737352961\\
141.01	0.00343591453090657\\
142.01	0.00343592184460027\\
143.01	0.00343592931805282\\
144.01	0.00343593695478288\\
145.01	0.00343594475838728\\
146.01	0.00343595273254306\\
147.01	0.00343596088100898\\
148.01	0.00343596920762754\\
149.01	0.00343597771632702\\
150.01	0.00343598641112262\\
151.01	0.00343599529611985\\
152.01	0.0034360043755151\\
153.01	0.00343601365359857\\
154.01	0.00343602313475617\\
155.01	0.00343603282347156\\
156.01	0.00343604272432865\\
157.01	0.00343605284201348\\
158.01	0.00343606318131694\\
159.01	0.00343607374713666\\
160.01	0.00343608454447993\\
161.01	0.00343609557846589\\
162.01	0.00343610685432839\\
163.01	0.00343611837741792\\
164.01	0.00343613015320478\\
165.01	0.00343614218728207\\
166.01	0.00343615448536774\\
167.01	0.00343616705330792\\
168.01	0.00343617989707969\\
169.01	0.00343619302279408\\
170.01	0.00343620643669909\\
171.01	0.00343622014518301\\
172.01	0.00343623415477737\\
173.01	0.00343624847215995\\
174.01	0.0034362631041589\\
175.01	0.00343627805775558\\
176.01	0.0034362933400881\\
177.01	0.00343630895845508\\
178.01	0.00343632492031914\\
179.01	0.00343634123331095\\
180.01	0.00343635790523251\\
181.01	0.00343637494406185\\
182.01	0.00343639235795604\\
183.01	0.0034364101552566\\
184.01	0.0034364283444925\\
185.01	0.00343644693438507\\
186.01	0.00343646593385247\\
187.01	0.00343648535201395\\
188.01	0.00343650519819466\\
189.01	0.00343652548193003\\
190.01	0.00343654621297124\\
191.01	0.00343656740128961\\
192.01	0.00343658905708197\\
193.01	0.00343661119077565\\
194.01	0.003436633813034\\
195.01	0.00343665693476168\\
196.01	0.0034366805671102\\
197.01	0.00343670472148404\\
198.01	0.00343672940954601\\
199.01	0.00343675464322341\\
200.01	0.00343678043471383\\
201.01	0.00343680679649225\\
202.01	0.00343683374131652\\
203.01	0.00343686128223443\\
204.01	0.00343688943259063\\
205.01	0.00343691820603274\\
206.01	0.00343694761651915\\
207.01	0.0034369776783261\\
208.01	0.00343700840605471\\
209.01	0.00343703981463908\\
210.01	0.00343707191935368\\
211.01	0.00343710473582111\\
212.01	0.00343713828002092\\
213.01	0.00343717256829719\\
214.01	0.00343720761736765\\
215.01	0.00343724344433219\\
216.01	0.00343728006668149\\
217.01	0.00343731750230686\\
218.01	0.00343735576950878\\
219.01	0.00343739488700738\\
220.01	0.00343743487395204\\
221.01	0.00343747574993084\\
222.01	0.0034375175349818\\
223.01	0.00343756024960291\\
224.01	0.00343760391476328\\
225.01	0.003437648551914\\
226.01	0.00343769418299976\\
227.01	0.00343774083047033\\
228.01	0.00343778851729246\\
229.01	0.00343783726696252\\
230.01	0.00343788710351885\\
231.01	0.00343793805155436\\
232.01	0.0034379901362299\\
233.01	0.00343804338328796\\
234.01	0.00343809781906636\\
235.01	0.00343815347051218\\
236.01	0.00343821036519684\\
237.01	0.00343826853133047\\
238.01	0.00343832799777757\\
239.01	0.00343838879407235\\
240.01	0.00343845095043513\\
241.01	0.00343851449778838\\
242.01	0.00343857946777387\\
243.01	0.00343864589276993\\
244.01	0.00343871380590909\\
245.01	0.00343878324109616\\
246.01	0.00343885423302703\\
247.01	0.00343892681720771\\
248.01	0.00343900102997403\\
249.01	0.00343907690851143\\
250.01	0.00343915449087556\\
251.01	0.00343923381601378\\
252.01	0.00343931492378666\\
253.01	0.00343939785498987\\
254.01	0.00343948265137684\\
255.01	0.00343956935568286\\
256.01	0.00343965801164872\\
257.01	0.00343974866404484\\
258.01	0.00343984135869671\\
259.01	0.00343993614251079\\
260.01	0.00344003306350138\\
261.01	0.00344013217081733\\
262.01	0.00344023351476999\\
263.01	0.0034403371468618\\
264.01	0.00344044311981602\\
265.01	0.00344055148760665\\
266.01	0.00344066230548902\\
267.01	0.00344077563003226\\
268.01	0.00344089151915118\\
269.01	0.0034410100321399\\
270.01	0.00344113122970625\\
271.01	0.00344125517400698\\
272.01	0.00344138192868342\\
273.01	0.0034415115588991\\
274.01	0.00344164413137751\\
275.01	0.00344177971444133\\
276.01	0.00344191837805223\\
277.01	0.00344206019385215\\
278.01	0.00344220523520564\\
279.01	0.00344235357724308\\
280.01	0.00344250529690504\\
281.01	0.00344266047298834\\
282.01	0.00344281918619285\\
283.01	0.00344298151916927\\
284.01	0.00344314755656906\\
285.01	0.00344331738509535\\
286.01	0.0034434910935543\\
287.01	0.00344366877290943\\
288.01	0.00344385051633594\\
289.01	0.00344403641927789\\
290.01	0.00344422657950545\\
291.01	0.0034444210971753\\
292.01	0.00344462007489048\\
293.01	0.00344482361776452\\
294.01	0.00344503183348516\\
295.01	0.00344524483238066\\
296.01	0.00344546272748816\\
297.01	0.00344568563462302\\
298.01	0.00344591367245109\\
299.01	0.00344614696256205\\
300.01	0.00344638562954548\\
301.01	0.00344662980106772\\
302.01	0.00344687960795297\\
303.01	0.00344713518426423\\
304.01	0.00344739666738807\\
305.01	0.00344766419812098\\
306.01	0.00344793792075825\\
307.01	0.0034482179831853\\
308.01	0.00344850453697115\\
309.01	0.00344879773746484\\
310.01	0.00344909774389428\\
311.01	0.00344940471946746\\
312.01	0.00344971883147739\\
313.01	0.00345004025140842\\
314.01	0.00345036915504633\\
315.01	0.00345070572259158\\
316.01	0.00345105013877459\\
317.01	0.00345140259297511\\
318.01	0.00345176327934427\\
319.01	0.00345213239692985\\
320.01	0.00345251014980524\\
321.01	0.00345289674720113\\
322.01	0.00345329240364159\\
323.01	0.00345369733908277\\
324.01	0.00345411177905587\\
325.01	0.00345453595481371\\
326.01	0.00345497010348066\\
327.01	0.00345541446820728\\
328.01	0.00345586929832845\\
329.01	0.00345633484952479\\
330.01	0.00345681138398987\\
331.01	0.00345729917060003\\
332.01	0.00345779848508969\\
333.01	0.00345830961023015\\
334.01	0.00345883283601314\\
335.01	0.00345936845983914\\
336.01	0.00345991678670948\\
337.01	0.00346047812942478\\
338.01	0.00346105280878561\\
339.01	0.00346164115380063\\
340.01	0.00346224350189756\\
341.01	0.00346286019913965\\
342.01	0.00346349160044822\\
343.01	0.00346413806982857\\
344.01	0.00346479998060209\\
345.01	0.00346547771564318\\
346.01	0.00346617166762116\\
347.01	0.00346688223924807\\
348.01	0.00346760984353063\\
349.01	0.00346835490402887\\
350.01	0.00346911785511868\\
351.01	0.00346989914226076\\
352.01	0.00347069922227436\\
353.01	0.00347151856361652\\
354.01	0.00347235764666654\\
355.01	0.00347321696401559\\
356.01	0.00347409702076235\\
357.01	0.00347499833481291\\
358.01	0.00347592143718652\\
359.01	0.00347686687232664\\
360.01	0.00347783519841638\\
361.01	0.0034788269876997\\
362.01	0.00347984282680747\\
363.01	0.0034808833170881\\
364.01	0.00348194907494316\\
365.01	0.00348304073216839\\
366.01	0.0034841589362978\\
367.01	0.00348530435095379\\
368.01	0.00348647765620154\\
369.01	0.00348767954890673\\
370.01	0.00348891074309854\\
371.01	0.00349017197033631\\
372.01	0.0034914639800805\\
373.01	0.0034927875400674\\
374.01	0.00349414343668847\\
375.01	0.00349553247537289\\
376.01	0.00349695548097463\\
377.01	0.00349841329816351\\
378.01	0.00349990679182058\\
379.01	0.00350143684743776\\
380.01	0.00350300437152321\\
381.01	0.00350461029201103\\
382.01	0.00350625555867759\\
383.01	0.00350794114356474\\
384.01	0.00350966804141054\\
385.01	0.00351143727008986\\
386.01	0.00351324987106556\\
387.01	0.00351510690985322\\
388.01	0.00351700947650166\\
389.01	0.00351895868609191\\
390.01	0.00352095567926007\\
391.01	0.00352300162274778\\
392.01	0.00352509770998558\\
393.01	0.0035272451617181\\
394.01	0.00352944522667605\\
395.01	0.00353169918230826\\
396.01	0.00353400833558194\\
397.01	0.00353637402386713\\
398.01	0.00353879761591694\\
399.01	0.00354128051296511\\
400.01	0.00354382414995573\\
401.01	0.00354642999692976\\
402.01	0.00354909956058921\\
403.01	0.00355183438606478\\
404.01	0.00355463605890932\\
405.01	0.00355750620734187\\
406.01	0.0035604465047627\\
407.01	0.00356345867255292\\
408.01	0.00356654448316425\\
409.01	0.00356970576349025\\
410.01	0.00357294439849051\\
411.01	0.00357626233501153\\
412.01	0.00357966158571654\\
413.01	0.00358314423299301\\
414.01	0.00358671243266216\\
415.01	0.0035903684172703\\
416.01	0.00359411449870714\\
417.01	0.00359795306989642\\
418.01	0.0036018866053745\\
419.01	0.00360591766076593\\
420.01	0.00361004887161692\\
421.01	0.00361428295307557\\
422.01	0.00361862270160335\\
423.01	0.003623070997386\\
424.01	0.00362763080677886\\
425.01	0.00363230518474876\\
426.01	0.00363709727729497\\
427.01	0.00364201032382824\\
428.01	0.00364704765948635\\
429.01	0.00365221271735601\\
430.01	0.00365750903057391\\
431.01	0.00366294023427299\\
432.01	0.00366851006733962\\
433.01	0.00367422237394438\\
434.01	0.00368008110481122\\
435.01	0.00368609031818995\\
436.01	0.00369225418050002\\
437.01	0.00369857696662226\\
438.01	0.00370506305982138\\
439.01	0.0037117169512967\\
440.01	0.00371854323936845\\
441.01	0.00372554662832292\\
442.01	0.00373273192694601\\
443.01	0.00374010404676978\\
444.01	0.00374766800002781\\
445.01	0.00375542889723789\\
446.01	0.0037633919441596\\
447.01	0.00377156243746551\\
448.01	0.00377994575851832\\
449.01	0.00378854736558161\\
450.01	0.00379737278454688\\
451.01	0.00380642759803404\\
452.01	0.0038157174327139\\
453.01	0.00382524794470275\\
454.01	0.00383502480289286\\
455.01	0.00384505367009871\\
456.01	0.00385534018193391\\
457.01	0.00386588992337877\\
458.01	0.00387670840307017\\
459.01	0.00388780102543614\\
460.01	0.00389917306092506\\
461.01	0.0039108296147447\\
462.01	0.00392277559473228\\
463.01	0.00393501567924152\\
464.01	0.00394755428624646\\
465.01	0.00396039554523857\\
466.01	0.0039735432739177\\
467.01	0.00398700096213324\\
468.01	0.00400077176597986\\
469.01	0.00401485851532114\\
470.01	0.00402926373817213\\
471.01	0.00404398970512304\\
472.01	0.00405903849599316\\
473.01	0.00407441208866829\\
474.01	0.00409011246579707\\
475.01	0.00410614172744046\\
476.01	0.00412250216221274\\
477.01	0.0041391962609163\\
478.01	0.00415622700906697\\
479.01	0.00417359849123737\\
480.01	0.00419131681237889\\
481.01	0.0042093914679963\\
482.01	0.00422783567027035\\
483.01	0.00424665417779234\\
484.01	0.0042658457506006\\
485.01	0.00428540920979736\\
486.01	0.00430534427978185\\
487.01	0.00432565281531641\\
488.01	0.00434634063993837\\
489.01	0.0043674202444339\\
490.01	0.00438891468839978\\
491.01	0.00441086317693595\\
492.01	0.00443332896078841\\
493.01	0.00445641002823711\\
494.01	0.00448020119439861\\
495.01	0.00450472674675936\\
496.01	0.00453000020260878\\
497.01	0.00455603353798351\\
498.01	0.00458283761352188\\
499.01	0.00461042315757104\\
500.01	0.00463880262288641\\
501.01	0.0046679941228801\\
502.01	0.00469803096466226\\
503.01	0.00472895550970411\\
504.01	0.00476081357674989\\
505.01	0.0047936546479168\\
506.01	0.00482753206190936\\
507.01	0.00486250307105099\\
508.01	0.00489862872544714\\
509.01	0.00493597349382256\\
510.01	0.00497460448799456\\
511.01	0.00501459010252792\\
512.01	0.00505599780398509\\
513.01	0.00509889069695039\\
514.01	0.0051433223451022\\
515.01	0.00518932911893609\\
516.01	0.00523691905500377\\
517.01	0.00528605581371308\\
518.01	0.00533663577069891\\
519.01	0.00538813206587271\\
520.01	0.0054400953075425\\
521.01	0.00549245788702683\\
522.01	0.00554514035176473\\
523.01	0.00559804894147087\\
524.01	0.00565107219355912\\
525.01	0.00570407617318674\\
526.01	0.00575690025895429\\
527.01	0.00580936340913263\\
528.01	0.00586127294259233\\
529.01	0.0059124345303605\\
530.01	0.00596266719498285\\
531.01	0.00601182666056869\\
532.01	0.00605984110978693\\
533.01	0.00610721093348789\\
534.01	0.00615438034487321\\
535.01	0.00620128224286311\\
536.01	0.00624785773615075\\
537.01	0.00629406015322152\\
538.01	0.00633985961947704\\
539.01	0.00638524801153224\\
540.01	0.0064302438979217\\
541.01	0.00647489676076576\\
542.01	0.00651928931073945\\
543.01	0.00656353598216361\\
544.01	0.00660777457638503\\
545.01	0.00665211780239825\\
546.01	0.00669659427891765\\
547.01	0.00674122708944817\\
548.01	0.00678604903754087\\
549.01	0.00683110222486148\\
550.01	0.00687643684249011\\
551.01	0.00692210903677819\\
552.01	0.00696817772137494\\
553.01	0.00701470032773292\\
554.01	0.00706172796307548\\
555.01	0.00710930116119387\\
556.01	0.00715745173033026\\
557.01	0.00720621117745391\\
558.01	0.00725561207670931\\
559.01	0.00730568713866519\\
560.01	0.00735646823249806\\
561.01	0.0074079854563134\\
562.01	0.00746026637928039\\
563.01	0.00751333567174559\\
564.01	0.00756721532443995\\
565.01	0.00762192549083276\\
566.01	0.00767748529274446\\
567.01	0.00773391277025212\\
568.01	0.00779122457082556\\
569.01	0.00784943570452103\\
570.01	0.0079085593937317\\
571.01	0.0079686070252471\\
572.01	0.00802958818111597\\
573.01	0.00809151068745984\\
574.01	0.00815438059004539\\
575.01	0.00821820199788514\\
576.01	0.00828297686270888\\
577.01	0.00834870476305525\\
578.01	0.00841538269761691\\
579.01	0.00848300487906297\\
580.01	0.00855156251639029\\
581.01	0.00862104357475559\\
582.01	0.00869143250950521\\
583.01	0.00876270998652133\\
584.01	0.00883485261300233\\
585.01	0.0089078326983209\\
586.01	0.00898161806023668\\
587.01	0.00905617189448366\\
588.01	0.00913145273077657\\
589.01	0.00920741450519178\\
590.01	0.00928400678726883\\
591.01	0.00936117520881615\\
592.01	0.0094388621492115\\
593.01	0.00951700774080956\\
594.01	0.00959555127130736\\
595.01	0.00967443307848172\\
596.01	0.00975359705744108\\
597.01	0.00983299394040848\\
598.01	0.00990826330297363\\
599.01	0.00997053306357279\\
599.02	0.00997104257304134\\
599.03	0.00997154904691464\\
599.04	0.00997205245565367\\
599.05	0.00997255276942865\\
599.06	0.00997304995811611\\
599.07	0.009973543991296\\
599.08	0.00997403483824878\\
599.09	0.00997452246795243\\
599.1	0.00997500684907947\\
599.11	0.00997548794999394\\
599.12	0.00997596573874834\\
599.13	0.00997644018308057\\
599.14	0.00997691125041076\\
599.15	0.00997737890783823\\
599.16	0.0099778431221382\\
599.17	0.00997830385975868\\
599.18	0.00997876108681716\\
599.19	0.0099792147690974\\
599.2	0.00997966487204609\\
599.21	0.00998011136076953\\
599.22	0.00998055420003026\\
599.23	0.00998099335424368\\
599.24	0.00998142878747457\\
599.25	0.00998186046343366\\
599.26	0.0099822883435904\\
599.27	0.00998271238778643\\
599.28	0.0099831325554638\\
599.29	0.00998354880566111\\
599.3	0.00998396109700947\\
599.31	0.00998436938772849\\
599.32	0.0099847736356222\\
599.33	0.00998517379807499\\
599.34	0.00998556983204737\\
599.35	0.00998596169407188\\
599.36	0.00998634934024881\\
599.37	0.00998673272624191\\
599.38	0.00998711180727415\\
599.39	0.00998748653812326\\
599.4	0.00998785687311743\\
599.41	0.00998822276613081\\
599.42	0.00998858417057903\\
599.43	0.00998894103941468\\
599.44	0.00998929332512275\\
599.45	0.00998964097971596\\
599.46	0.00998998395473014\\
599.47	0.00999032220121951\\
599.48	0.0099906556697519\\
599.49	0.00999098431040396\\
599.5	0.00999130807275631\\
599.51	0.00999162690588863\\
599.52	0.00999194075837471\\
599.53	0.00999224957827746\\
599.54	0.00999255331314386\\
599.55	0.00999285190999987\\
599.56	0.00999314531534524\\
599.57	0.00999343347514839\\
599.58	0.00999371633484107\\
599.59	0.00999399383931314\\
599.6	0.00999426593290715\\
599.61	0.009994532559413\\
599.62	0.0099947936620624\\
599.63	0.00999504918352342\\
599.64	0.00999529906589491\\
599.65	0.00999554325070086\\
599.66	0.00999578167888471\\
599.67	0.00999601429080366\\
599.68	0.00999624102622283\\
599.69	0.00999646182430947\\
599.7	0.00999667662362697\\
599.71	0.00999688536212897\\
599.72	0.00999708797715332\\
599.73	0.00999728440541596\\
599.74	0.00999747458300482\\
599.75	0.0099976584453736\\
599.76	0.0099978359273355\\
599.77	0.00999800696305692\\
599.78	0.00999817148605102\\
599.79	0.00999832942917133\\
599.8	0.00999848072460517\\
599.81	0.00999862530386713\\
599.82	0.00999876309779239\\
599.83	0.00999889403653003\\
599.84	0.00999901804953622\\
599.85	0.00999913506556741\\
599.86	0.00999924501267341\\
599.87	0.00999934781819042\\
599.88	0.00999944340873394\\
599.89	0.0099995317101917\\
599.9	0.00999961264771648\\
599.91	0.00999968614571878\\
599.92	0.00999975212785957\\
599.93	0.00999981051704285\\
599.94	0.00999986123540817\\
599.95	0.00999990420432308\\
599.96	0.00999993934437554\\
599.97	0.00999996657536618\\
599.98	0.00999998581630055\\
599.99	0.00999999698538124\\
600	0.01\\
};
\addplot [color=red!50!mycolor17,solid,forget plot]
  table[row sep=crcr]{%
0.01	0.00372498017920368\\
1.01	0.00372498055307331\\
2.01	0.00372498093491974\\
3.01	0.00372498132491364\\
4.01	0.00372498172322953\\
5.01	0.00372498213004546\\
6.01	0.00372498254554356\\
7.01	0.00372498296990988\\
8.01	0.00372498340333439\\
9.01	0.00372498384601103\\
10.01	0.0037249842981379\\
11.01	0.00372498475991775\\
12.01	0.0037249852315572\\
13.01	0.00372498571326757\\
14.01	0.00372498620526469\\
15.01	0.00372498670776908\\
16.01	0.00372498722100591\\
17.01	0.0037249877452052\\
18.01	0.00372498828060196\\
19.01	0.00372498882743637\\
20.01	0.0037249893859535\\
21.01	0.00372498995640398\\
22.01	0.00372499053904392\\
23.01	0.00372499113413442\\
24.01	0.00372499174194264\\
25.01	0.00372499236274153\\
26.01	0.00372499299680955\\
27.01	0.00372499364443153\\
28.01	0.00372499430589837\\
29.01	0.00372499498150711\\
30.01	0.00372499567156116\\
31.01	0.00372499637637076\\
32.01	0.00372499709625244\\
33.01	0.00372499783152987\\
34.01	0.00372499858253345\\
35.01	0.00372499934960088\\
36.01	0.00372500013307728\\
37.01	0.00372500093331488\\
38.01	0.00372500175067365\\
39.01	0.00372500258552136\\
40.01	0.00372500343823362\\
41.01	0.00372500430919441\\
42.01	0.00372500519879559\\
43.01	0.00372500610743773\\
44.01	0.00372500703553016\\
45.01	0.00372500798349075\\
46.01	0.00372500895174664\\
47.01	0.00372500994073388\\
48.01	0.00372501095089831\\
49.01	0.00372501198269508\\
50.01	0.00372501303658947\\
51.01	0.00372501411305672\\
52.01	0.00372501521258209\\
53.01	0.00372501633566162\\
54.01	0.00372501748280195\\
55.01	0.00372501865452052\\
56.01	0.00372501985134646\\
57.01	0.00372502107381967\\
58.01	0.00372502232249201\\
59.01	0.00372502359792756\\
60.01	0.00372502490070223\\
61.01	0.0037250262314043\\
62.01	0.00372502759063517\\
63.01	0.00372502897900885\\
64.01	0.00372503039715306\\
65.01	0.0037250318457088\\
66.01	0.00372503332533074\\
67.01	0.00372503483668811\\
68.01	0.00372503638046465\\
69.01	0.00372503795735857\\
70.01	0.00372503956808343\\
71.01	0.00372504121336796\\
72.01	0.0037250428939572\\
73.01	0.00372504461061168\\
74.01	0.00372504636410882\\
75.01	0.00372504815524272\\
76.01	0.00372504998482476\\
77.01	0.00372505185368396\\
78.01	0.00372505376266724\\
79.01	0.00372505571263956\\
80.01	0.00372505770448507\\
81.01	0.00372505973910685\\
82.01	0.00372506181742771\\
83.01	0.00372506394039017\\
84.01	0.00372506610895751\\
85.01	0.00372506832411379\\
86.01	0.00372507058686426\\
87.01	0.00372507289823623\\
88.01	0.00372507525927906\\
89.01	0.00372507767106492\\
90.01	0.00372508013468935\\
91.01	0.00372508265127157\\
92.01	0.00372508522195517\\
93.01	0.0037250878479084\\
94.01	0.00372509053032499\\
95.01	0.0037250932704246\\
96.01	0.0037250960694533\\
97.01	0.00372509892868417\\
98.01	0.00372510184941803\\
99.01	0.00372510483298405\\
100.01	0.00372510788073996\\
101.01	0.0037251109940733\\
102.01	0.00372511417440156\\
103.01	0.00372511742317305\\
104.01	0.0037251207418677\\
105.01	0.00372512413199746\\
106.01	0.00372512759510714\\
107.01	0.00372513113277498\\
108.01	0.00372513474661397\\
109.01	0.0037251384382718\\
110.01	0.00372514220943214\\
111.01	0.0037251460618153\\
112.01	0.00372514999717892\\
113.01	0.00372515401731874\\
114.01	0.0037251581240699\\
115.01	0.00372516231930727\\
116.01	0.00372516660494658\\
117.01	0.00372517098294538\\
118.01	0.00372517545530357\\
119.01	0.00372518002406449\\
120.01	0.00372518469131648\\
121.01	0.00372518945919284\\
122.01	0.00372519432987377\\
123.01	0.00372519930558657\\
124.01	0.00372520438860723\\
125.01	0.00372520958126132\\
126.01	0.00372521488592517\\
127.01	0.00372522030502681\\
128.01	0.00372522584104701\\
129.01	0.00372523149652088\\
130.01	0.00372523727403887\\
131.01	0.00372524317624758\\
132.01	0.0037252492058516\\
133.01	0.00372525536561445\\
134.01	0.00372526165835989\\
135.01	0.0037252680869732\\
136.01	0.00372527465440264\\
137.01	0.00372528136366093\\
138.01	0.00372528821782658\\
139.01	0.00372529522004507\\
140.01	0.00372530237353076\\
141.01	0.00372530968156805\\
142.01	0.00372531714751304\\
143.01	0.00372532477479516\\
144.01	0.00372533256691898\\
145.01	0.00372534052746537\\
146.01	0.00372534866009359\\
147.01	0.00372535696854284\\
148.01	0.00372536545663396\\
149.01	0.00372537412827131\\
150.01	0.0037253829874446\\
151.01	0.00372539203823097\\
152.01	0.00372540128479661\\
153.01	0.00372541073139868\\
154.01	0.00372542038238768\\
155.01	0.00372543024220925\\
156.01	0.00372544031540625\\
157.01	0.00372545060662118\\
158.01	0.00372546112059782\\
159.01	0.00372547186218419\\
160.01	0.00372548283633418\\
161.01	0.00372549404811039\\
162.01	0.00372550550268611\\
163.01	0.00372551720534807\\
164.01	0.00372552916149904\\
165.01	0.00372554137665974\\
166.01	0.00372555385647234\\
167.01	0.00372556660670259\\
168.01	0.0037255796332426\\
169.01	0.0037255929421138\\
170.01	0.00372560653946968\\
171.01	0.00372562043159876\\
172.01	0.00372563462492725\\
173.01	0.00372564912602319\\
174.01	0.00372566394159801\\
175.01	0.00372567907851083\\
176.01	0.00372569454377142\\
177.01	0.00372571034454358\\
178.01	0.00372572648814846\\
179.01	0.00372574298206808\\
180.01	0.00372575983394896\\
181.01	0.00372577705160555\\
182.01	0.00372579464302471\\
183.01	0.00372581261636813\\
184.01	0.00372583097997777\\
185.01	0.00372584974237876\\
186.01	0.00372586891228406\\
187.01	0.00372588849859841\\
188.01	0.00372590851042257\\
189.01	0.00372592895705792\\
190.01	0.00372594984801061\\
191.01	0.00372597119299634\\
192.01	0.00372599300194502\\
193.01	0.00372601528500528\\
194.01	0.00372603805254988\\
195.01	0.00372606131518032\\
196.01	0.00372608508373203\\
197.01	0.00372610936927953\\
198.01	0.00372613418314181\\
199.01	0.00372615953688818\\
200.01	0.00372618544234314\\
201.01	0.00372621191159263\\
202.01	0.00372623895698999\\
203.01	0.00372626659116139\\
204.01	0.00372629482701235\\
205.01	0.00372632367773415\\
206.01	0.00372635315680996\\
207.01	0.00372638327802134\\
208.01	0.00372641405545511\\
209.01	0.0037264455035102\\
210.01	0.00372647763690498\\
211.01	0.00372651047068392\\
212.01	0.00372654402022504\\
213.01	0.00372657830124797\\
214.01	0.00372661332982088\\
215.01	0.00372664912236892\\
216.01	0.00372668569568202\\
217.01	0.00372672306692338\\
218.01	0.00372676125363779\\
219.01	0.00372680027376023\\
220.01	0.0037268401456247\\
221.01	0.00372688088797352\\
222.01	0.00372692251996626\\
223.01	0.00372696506118946\\
224.01	0.00372700853166619\\
225.01	0.00372705295186623\\
226.01	0.00372709834271619\\
227.01	0.00372714472560948\\
228.01	0.00372719212241801\\
229.01	0.00372724055550186\\
230.01	0.00372729004772124\\
231.01	0.00372734062244794\\
232.01	0.0037273923035768\\
233.01	0.00372744511553777\\
234.01	0.00372749908330823\\
235.01	0.00372755423242579\\
236.01	0.00372761058900085\\
237.01	0.00372766817973001\\
238.01	0.00372772703190941\\
239.01	0.00372778717344917\\
240.01	0.00372784863288678\\
241.01	0.00372791143940218\\
242.01	0.00372797562283254\\
243.01	0.00372804121368737\\
244.01	0.00372810824316441\\
245.01	0.0037281767431655\\
246.01	0.0037282467463131\\
247.01	0.00372831828596691\\
248.01	0.00372839139624145\\
249.01	0.00372846611202339\\
250.01	0.00372854246899006\\
251.01	0.00372862050362743\\
252.01	0.00372870025324983\\
253.01	0.00372878175601873\\
254.01	0.00372886505096376\\
255.01	0.00372895017800218\\
256.01	0.00372903717796023\\
257.01	0.00372912609259508\\
258.01	0.00372921696461667\\
259.01	0.00372930983771033\\
260.01	0.00372940475655993\\
261.01	0.00372950176687185\\
262.01	0.0037296009153998\\
263.01	0.00372970224996946\\
264.01	0.00372980581950426\\
265.01	0.00372991167405207\\
266.01	0.00373001986481211\\
267.01	0.00373013044416289\\
268.01	0.00373024346569066\\
269.01	0.00373035898421877\\
270.01	0.00373047705583764\\
271.01	0.00373059773793565\\
272.01	0.00373072108923136\\
273.01	0.00373084716980503\\
274.01	0.00373097604113301\\
275.01	0.0037311077661214\\
276.01	0.00373124240914142\\
277.01	0.00373138003606595\\
278.01	0.0037315207143062\\
279.01	0.00373166451285015\\
280.01	0.00373181150230198\\
281.01	0.0037319617549217\\
282.01	0.00373211534466722\\
283.01	0.00373227234723672\\
284.01	0.00373243284011204\\
285.01	0.00373259690260373\\
286.01	0.00373276461589765\\
287.01	0.00373293606310184\\
288.01	0.00373311132929612\\
289.01	0.00373329050158126\\
290.01	0.00373347366913161\\
291.01	0.00373366092324715\\
292.01	0.00373385235740929\\
293.01	0.00373404806733561\\
294.01	0.00373424815103884\\
295.01	0.0037344527088856\\
296.01	0.00373466184365764\\
297.01	0.00373487566061501\\
298.01	0.00373509426756033\\
299.01	0.00373531777490574\\
300.01	0.00373554629574148\\
301.01	0.00373577994590657\\
302.01	0.00373601884406109\\
303.01	0.00373626311176116\\
304.01	0.00373651287353641\\
305.01	0.00373676825696875\\
306.01	0.00373702939277445\\
307.01	0.00373729641488817\\
308.01	0.00373756946054972\\
309.01	0.00373784867039348\\
310.01	0.00373813418853994\\
311.01	0.00373842616269159\\
312.01	0.00373872474422919\\
313.01	0.00373903008831394\\
314.01	0.00373934235399027\\
315.01	0.00373966170429316\\
316.01	0.00373998830635869\\
317.01	0.00374032233153752\\
318.01	0.00374066395551208\\
319.01	0.0037410133584176\\
320.01	0.00374137072496661\\
321.01	0.0037417362445779\\
322.01	0.00374211011150859\\
323.01	0.0037424925249914\\
324.01	0.00374288368937518\\
325.01	0.00374328381427115\\
326.01	0.00374369311470253\\
327.01	0.0037441118112595\\
328.01	0.00374454013025884\\
329.01	0.00374497830390957\\
330.01	0.00374542657048217\\
331.01	0.00374588517448465\\
332.01	0.00374635436684358\\
333.01	0.00374683440509112\\
334.01	0.0037473255535577\\
335.01	0.00374782808357142\\
336.01	0.00374834227366341\\
337.01	0.00374886840977949\\
338.01	0.00374940678549973\\
339.01	0.00374995770226307\\
340.01	0.00375052146960118\\
341.01	0.00375109840537878\\
342.01	0.00375168883604102\\
343.01	0.00375229309686972\\
344.01	0.00375291153224748\\
345.01	0.00375354449592946\\
346.01	0.00375419235132436\\
347.01	0.00375485547178386\\
348.01	0.00375553424090107\\
349.01	0.00375622905281801\\
350.01	0.00375694031254284\\
351.01	0.00375766843627626\\
352.01	0.00375841385174804\\
353.01	0.00375917699856304\\
354.01	0.00375995832855833\\
355.01	0.0037607583061693\\
356.01	0.00376157740880652\\
357.01	0.00376241612724397\\
358.01	0.00376327496601676\\
359.01	0.00376415444382948\\
360.01	0.00376505509397547\\
361.01	0.00376597746476599\\
362.01	0.00376692211996997\\
363.01	0.00376788963926346\\
364.01	0.00376888061868925\\
365.01	0.00376989567112455\\
366.01	0.00377093542675871\\
367.01	0.00377200053357817\\
368.01	0.00377309165785811\\
369.01	0.00377420948466127\\
370.01	0.00377535471834049\\
371.01	0.00377652808304577\\
372.01	0.00377773032323167\\
373.01	0.00377896220416556\\
374.01	0.00378022451243198\\
375.01	0.0037815180564322\\
376.01	0.00378284366687458\\
377.01	0.00378420219725253\\
378.01	0.00378559452430612\\
379.01	0.00378702154846132\\
380.01	0.0037884841942422\\
381.01	0.00378998341065021\\
382.01	0.00379152017150261\\
383.01	0.00379309547572136\\
384.01	0.00379471034756595\\
385.01	0.00379636583679713\\
386.01	0.00379806301876236\\
387.01	0.00379980299439042\\
388.01	0.00380158689008128\\
389.01	0.00380341585747766\\
390.01	0.00380529107310197\\
391.01	0.00380721373784297\\
392.01	0.00380918507627581\\
393.01	0.00381120633579773\\
394.01	0.00381327878556506\\
395.01	0.0038154037152149\\
396.01	0.00381758243336124\\
397.01	0.00381981626585658\\
398.01	0.00382210655382005\\
399.01	0.00382445465143698\\
400.01	0.00382686192355398\\
401.01	0.00382932974310531\\
402.01	0.00383185948843176\\
403.01	0.00383445254058133\\
404.01	0.00383711028071845\\
405.01	0.00383983408781334\\
406.01	0.00384262533683892\\
407.01	0.0038454853977703\\
408.01	0.00384841563575423\\
409.01	0.00385141741289995\\
410.01	0.00385449209222417\\
411.01	0.00385764104435753\\
412.01	0.003860865657658\\
413.01	0.00386416735236141\\
414.01	0.00386754759926033\\
415.01	0.00387100794307384\\
416.01	0.00387455003001864\\
417.01	0.00387817563792722\\
418.01	0.00388188670528924\\
419.01	0.00388568535240304\\
420.01	0.00388957388001282\\
421.01	0.00389355471654147\\
422.01	0.00389763037732452\\
423.01	0.00390180346490945\\
424.01	0.00390607667301784\\
425.01	0.00391045279072186\\
426.01	0.00391493470684755\\
427.01	0.00391952541461714\\
428.01	0.00392422801654253\\
429.01	0.00392904572958684\\
430.01	0.0039339818906044\\
431.01	0.00393903996207619\\
432.01	0.00394422353815439\\
433.01	0.00394953635103058\\
434.01	0.0039549822776398\\
435.01	0.00396056534671387\\
436.01	0.0039662897461922\\
437.01	0.00397215983099526\\
438.01	0.00397818013116025\\
439.01	0.00398435536032973\\
440.01	0.00399069042457371\\
441.01	0.00399719043150867\\
442.01	0.00400386069965954\\
443.01	0.004010706767985\\
444.01	0.00401773440545669\\
445.01	0.00402494962054829\\
446.01	0.00403235867045577\\
447.01	0.00403996806983509\\
448.01	0.00404778459881801\\
449.01	0.00405581531000204\\
450.01	0.00406406753401853\\
451.01	0.00407254888318878\\
452.01	0.00408126725267054\\
453.01	0.00409023081837311\\
454.01	0.00409944803077297\\
455.01	0.0041089276036044\\
456.01	0.00411867849621471\\
457.01	0.00412870988818528\\
458.01	0.00413903114462237\\
459.01	0.00414965177033383\\
460.01	0.00416058135095958\\
461.01	0.00417182947904102\\
462.01	0.00418340566307071\\
463.01	0.00419531921783382\\
464.01	0.0042075791349757\\
465.01	0.00422019393389371\\
466.01	0.00423317149503469\\
467.01	0.00424651888088312\\
468.01	0.00426024215492449\\
469.01	0.00427434621647897\\
470.01	0.00428883468069133\\
471.01	0.0043037098497742\\
472.01	0.00431897284616421\\
473.01	0.00433462401382316\\
474.01	0.00435066374511892\\
475.01	0.00436709400149751\\
476.01	0.00438392296299855\\
477.01	0.00440116621040738\\
478.01	0.00441884136915584\\
479.01	0.00443696778367654\\
480.01	0.00445556670597234\\
481.01	0.00447466146957008\\
482.01	0.0044942776317114\\
483.01	0.004514443281654\\
484.01	0.00453518962365206\\
485.01	0.00455655140357669\\
486.01	0.00457856728399696\\
487.01	0.00460128018570108\\
488.01	0.00462473754685566\\
489.01	0.00464899142450229\\
490.01	0.00467409832483183\\
491.01	0.00470011859366921\\
492.01	0.00472711511991687\\
493.01	0.00475515099407547\\
494.01	0.00478428645556545\\
495.01	0.0048145785588599\\
496.01	0.00484607977893972\\
497.01	0.00487883345753433\\
498.01	0.00491286704141813\\
499.01	0.00494818233849838\\
500.01	0.00498474161227278\\
501.01	0.00502233835144018\\
502.01	0.0050605647173563\\
503.01	0.0050993730901166\\
504.01	0.0051387323839106\\
505.01	0.0051786037412119\\
506.01	0.00521893946779375\\
507.01	0.00525968193544965\\
508.01	0.00530076250996607\\
509.01	0.00534210059584102\\
510.01	0.00538360293910421\\
511.01	0.00542516340129653\\
512.01	0.00546666352000906\\
513.01	0.00550797431673349\\
514.01	0.00554896001829935\\
515.01	0.00558948464782177\\
516.01	0.00562942284808851\\
517.01	0.00566867687090637\\
518.01	0.00570720246416827\\
519.01	0.00574537905056015\\
520.01	0.00578353611736662\\
521.01	0.00582161977278742\\
522.01	0.00585957714925499\\
523.01	0.00589735821902541\\
524.01	0.00593491811364376\\
525.01	0.00597222004792183\\
526.01	0.00600923893205159\\
527.01	0.00604596537409513\\
528.01	0.00608240949177337\\
529.01	0.00611860415938913\\
530.01	0.00615460689620581\\
531.01	0.00619049903407586\\
532.01	0.00622638000393134\\
533.01	0.00626233778218718\\
534.01	0.00629839323853544\\
535.01	0.00633455660955361\\
536.01	0.00637084532387994\\
537.01	0.00640728423694983\\
538.01	0.00644390545520038\\
539.01	0.00648074761806494\\
540.01	0.00651785451344932\\
541.01	0.00655527294240604\\
542.01	0.00659304984595892\\
543.01	0.0066312288998787\\
544.01	0.0066698471308591\\
545.01	0.0067089334512704\\
546.01	0.00674851419094998\\
547.01	0.00678861730368158\\
548.01	0.00682927204376869\\
549.01	0.00687050825608435\\
550.01	0.00691235560656272\\
551.01	0.00695484283077342\\
552.01	0.00699799710791779\\
553.01	0.00704184369372662\\
554.01	0.00708640594671313\\
555.01	0.00713170582569912\\
556.01	0.00717776466084599\\
557.01	0.0072246034337525\\
558.01	0.00727224258703751\\
559.01	0.00732070182243218\\
560.01	0.00736999995604683\\
561.01	0.00742015484321033\\
562.01	0.00747118337583044\\
563.01	0.00752310153781753\\
564.01	0.00757592447950262\\
565.01	0.00762966655047324\\
566.01	0.00768434124365575\\
567.01	0.00773996108862188\\
568.01	0.0077965375481144\\
569.01	0.00785408092385981\\
570.01	0.00791260026552221\\
571.01	0.00797210327360572\\
572.01	0.00803259618583545\\
573.01	0.00809408363856894\\
574.01	0.00815656850118068\\
575.01	0.00822005169027631\\
576.01	0.00828453197232036\\
577.01	0.00835000575654852\\
578.01	0.00841646687724602\\
579.01	0.00848390636516268\\
580.01	0.00855231220950171\\
581.01	0.0086216691142773\\
582.01	0.00869195825546762\\
583.01	0.00876315704751246\\
584.01	0.00883523892879311\\
585.01	0.00890817317693382\\
586.01	0.00898192476731992\\
587.01	0.00905645429185747\\
588.01	0.00913171795951237\\
589.01	0.00920766770565841\\
590.01	0.00928425144388741\\
591.01	0.00936141350201614\\
592.01	0.00943909529417743\\
593.01	0.00951723629392387\\
594.01	0.00959577538991279\\
595.01	0.00967465272665585\\
596.01	0.00975381215885204\\
597.01	0.00983319200642706\\
598.01	0.00990826330299313\\
599.01	0.00997053306357288\\
599.02	0.00997104257304143\\
599.03	0.00997154904691472\\
599.04	0.00997205245565375\\
599.05	0.00997255276942872\\
599.06	0.00997304995811617\\
599.07	0.00997354399129606\\
599.08	0.00997403483824883\\
599.09	0.00997452246795248\\
599.1	0.00997500684907952\\
599.11	0.00997548794999398\\
599.12	0.00997596573874838\\
599.13	0.0099764401830806\\
599.14	0.0099769112504108\\
599.15	0.00997737890783826\\
599.16	0.00997784312213823\\
599.17	0.0099783038597587\\
599.18	0.00997876108681718\\
599.19	0.00997921476909742\\
599.2	0.00997966487204611\\
599.21	0.00998011136076955\\
599.22	0.00998055420003028\\
599.23	0.0099809933542437\\
599.24	0.00998142878747458\\
599.25	0.00998186046343367\\
599.26	0.00998228834359041\\
599.27	0.00998271238778643\\
599.28	0.00998313255546381\\
599.29	0.00998354880566111\\
599.3	0.00998396109700947\\
599.31	0.00998436938772849\\
599.32	0.00998477363562221\\
599.33	0.00998517379807499\\
599.34	0.00998556983204737\\
599.35	0.00998596169407188\\
599.36	0.00998634934024881\\
599.37	0.00998673272624192\\
599.38	0.00998711180727415\\
599.39	0.00998748653812326\\
599.4	0.00998785687311743\\
599.41	0.00998822276613081\\
599.42	0.00998858417057903\\
599.43	0.00998894103941468\\
599.44	0.00998929332512275\\
599.45	0.00998964097971596\\
599.46	0.00998998395473014\\
599.47	0.00999032220121951\\
599.48	0.0099906556697519\\
599.49	0.00999098431040396\\
599.5	0.00999130807275631\\
599.51	0.00999162690588863\\
599.52	0.00999194075837471\\
599.53	0.00999224957827746\\
599.54	0.00999255331314387\\
599.55	0.00999285190999987\\
599.56	0.00999314531534524\\
599.57	0.00999343347514839\\
599.58	0.00999371633484107\\
599.59	0.00999399383931314\\
599.6	0.00999426593290715\\
599.61	0.009994532559413\\
599.62	0.0099947936620624\\
599.63	0.00999504918352342\\
599.64	0.00999529906589491\\
599.65	0.00999554325070086\\
599.66	0.00999578167888471\\
599.67	0.00999601429080366\\
599.68	0.00999624102622283\\
599.69	0.00999646182430947\\
599.7	0.00999667662362697\\
599.71	0.00999688536212897\\
599.72	0.00999708797715332\\
599.73	0.00999728440541595\\
599.74	0.00999747458300482\\
599.75	0.0099976584453736\\
599.76	0.00999783592733551\\
599.77	0.00999800696305692\\
599.78	0.00999817148605102\\
599.79	0.00999832942917133\\
599.8	0.00999848072460517\\
599.81	0.00999862530386713\\
599.82	0.00999876309779239\\
599.83	0.00999889403653003\\
599.84	0.00999901804953622\\
599.85	0.00999913506556741\\
599.86	0.00999924501267341\\
599.87	0.00999934781819042\\
599.88	0.00999944340873394\\
599.89	0.0099995317101917\\
599.9	0.00999961264771648\\
599.91	0.00999968614571878\\
599.92	0.00999975212785958\\
599.93	0.00999981051704285\\
599.94	0.00999986123540817\\
599.95	0.00999990420432308\\
599.96	0.00999993934437554\\
599.97	0.00999996657536618\\
599.98	0.00999998581630055\\
599.99	0.00999999698538124\\
600	0.01\\
};
\addplot [color=red!40!mycolor19,solid,forget plot]
  table[row sep=crcr]{%
0.01	0.00384875486878031\\
1.01	0.00384875533045602\\
2.01	0.00384875580195604\\
3.01	0.00384875628349013\\
4.01	0.00384875677527215\\
5.01	0.00384875727752121\\
6.01	0.00384875779046065\\
7.01	0.00384875831431862\\
8.01	0.00384875884932827\\
9.01	0.00384875939572767\\
10.01	0.00384875995376011\\
11.01	0.00384876052367375\\
12.01	0.00384876110572258\\
13.01	0.00384876170016588\\
14.01	0.00384876230726797\\
15.01	0.00384876292729951\\
16.01	0.0038487635605368\\
17.01	0.00384876420726182\\
18.01	0.0038487648677628\\
19.01	0.00384876554233396\\
20.01	0.00384876623127613\\
21.01	0.00384876693489627\\
22.01	0.00384876765350804\\
23.01	0.0038487683874321\\
24.01	0.00384876913699551\\
25.01	0.0038487699025327\\
26.01	0.00384877068438512\\
27.01	0.00384877148290162\\
28.01	0.00384877229843843\\
29.01	0.00384877313135953\\
30.01	0.00384877398203673\\
31.01	0.00384877485084956\\
32.01	0.00384877573818605\\
33.01	0.00384877664444244\\
34.01	0.00384877757002339\\
35.01	0.00384877851534236\\
36.01	0.00384877948082141\\
37.01	0.00384878046689194\\
38.01	0.00384878147399474\\
39.01	0.0038487825025799\\
40.01	0.00384878355310701\\
41.01	0.00384878462604556\\
42.01	0.00384878572187557\\
43.01	0.00384878684108672\\
44.01	0.00384878798417967\\
45.01	0.0038487891516657\\
46.01	0.0038487903440671\\
47.01	0.00384879156191735\\
48.01	0.00384879280576131\\
49.01	0.00384879407615581\\
50.01	0.00384879537366947\\
51.01	0.00384879669888306\\
52.01	0.0038487980523899\\
53.01	0.00384879943479619\\
54.01	0.00384880084672094\\
55.01	0.00384880228879672\\
56.01	0.0038488037616695\\
57.01	0.00384880526599934\\
58.01	0.00384880680246034\\
59.01	0.00384880837174098\\
60.01	0.00384880997454495\\
61.01	0.00384881161159073\\
62.01	0.00384881328361245\\
63.01	0.00384881499135992\\
64.01	0.00384881673559906\\
65.01	0.00384881851711236\\
66.01	0.00384882033669913\\
67.01	0.00384882219517596\\
68.01	0.00384882409337686\\
69.01	0.00384882603215397\\
70.01	0.00384882801237758\\
71.01	0.00384883003493711\\
72.01	0.00384883210074047\\
73.01	0.0038488342107157\\
74.01	0.00384883636581072\\
75.01	0.00384883856699361\\
76.01	0.00384884081525362\\
77.01	0.00384884311160097\\
78.01	0.00384884545706804\\
79.01	0.00384884785270923\\
80.01	0.00384885029960157\\
81.01	0.0038488527988456\\
82.01	0.00384885535156525\\
83.01	0.00384885795890889\\
84.01	0.0038488606220496\\
85.01	0.00384886334218554\\
86.01	0.00384886612054111\\
87.01	0.00384886895836679\\
88.01	0.00384887185694011\\
89.01	0.00384887481756617\\
90.01	0.00384887784157831\\
91.01	0.00384888093033844\\
92.01	0.00384888408523801\\
93.01	0.00384888730769828\\
94.01	0.00384889059917125\\
95.01	0.00384889396114044\\
96.01	0.00384889739512118\\
97.01	0.00384890090266159\\
98.01	0.00384890448534314\\
99.01	0.00384890814478146\\
100.01	0.00384891188262695\\
101.01	0.00384891570056578\\
102.01	0.00384891960032059\\
103.01	0.00384892358365092\\
104.01	0.00384892765235449\\
105.01	0.00384893180826783\\
106.01	0.00384893605326712\\
107.01	0.00384894038926914\\
108.01	0.00384894481823191\\
109.01	0.00384894934215577\\
110.01	0.00384895396308424\\
111.01	0.00384895868310507\\
112.01	0.00384896350435124\\
113.01	0.00384896842900159\\
114.01	0.00384897345928227\\
115.01	0.00384897859746712\\
116.01	0.00384898384587952\\
117.01	0.00384898920689283\\
118.01	0.0038489946829318\\
119.01	0.00384900027647395\\
120.01	0.00384900599004959\\
121.01	0.00384901182624441\\
122.01	0.00384901778769957\\
123.01	0.00384902387711361\\
124.01	0.00384903009724338\\
125.01	0.0038490364509052\\
126.01	0.00384904294097643\\
127.01	0.00384904957039653\\
128.01	0.00384905634216895\\
129.01	0.00384906325936164\\
130.01	0.00384907032510892\\
131.01	0.00384907754261352\\
132.01	0.0038490849151466\\
133.01	0.00384909244605075\\
134.01	0.00384910013874072\\
135.01	0.00384910799670515\\
136.01	0.00384911602350821\\
137.01	0.00384912422279104\\
138.01	0.00384913259827395\\
139.01	0.00384914115375749\\
140.01	0.00384914989312453\\
141.01	0.00384915882034226\\
142.01	0.00384916793946358\\
143.01	0.00384917725462916\\
144.01	0.00384918677006937\\
145.01	0.00384919649010618\\
146.01	0.00384920641915507\\
147.01	0.00384921656172746\\
148.01	0.00384922692243228\\
149.01	0.0038492375059784\\
150.01	0.00384924831717694\\
151.01	0.00384925936094281\\
152.01	0.00384927064229761\\
153.01	0.003849282166372\\
154.01	0.00384929393840754\\
155.01	0.00384930596375969\\
156.01	0.00384931824789976\\
157.01	0.0038493307964178\\
158.01	0.0038493436150249\\
159.01	0.00384935670955626\\
160.01	0.00384937008597344\\
161.01	0.0038493837503672\\
162.01	0.00384939770896062\\
163.01	0.0038494119681118\\
164.01	0.00384942653431637\\
165.01	0.00384944141421127\\
166.01	0.0038494566145771\\
167.01	0.00384947214234196\\
168.01	0.00384948800458407\\
169.01	0.00384950420853521\\
170.01	0.00384952076158425\\
171.01	0.00384953767128018\\
172.01	0.0038495549453364\\
173.01	0.00384957259163313\\
174.01	0.00384959061822199\\
175.01	0.00384960903332954\\
176.01	0.00384962784536078\\
177.01	0.00384964706290317\\
178.01	0.00384966669473094\\
179.01	0.00384968674980868\\
180.01	0.00384970723729584\\
181.01	0.00384972816655102\\
182.01	0.00384974954713578\\
183.01	0.00384977138881974\\
184.01	0.00384979370158474\\
185.01	0.00384981649562967\\
186.01	0.00384983978137504\\
187.01	0.00384986356946779\\
188.01	0.00384988787078654\\
189.01	0.00384991269644636\\
190.01	0.00384993805780408\\
191.01	0.0038499639664635\\
192.01	0.00384999043428073\\
193.01	0.0038500174733701\\
194.01	0.00385004509610926\\
195.01	0.00385007331514527\\
196.01	0.00385010214340037\\
197.01	0.00385013159407825\\
198.01	0.00385016168066972\\
199.01	0.00385019241695959\\
200.01	0.00385022381703296\\
201.01	0.00385025589528125\\
202.01	0.00385028866640959\\
203.01	0.0038503221454434\\
204.01	0.00385035634773567\\
205.01	0.00385039128897369\\
206.01	0.0038504269851868\\
207.01	0.00385046345275357\\
208.01	0.00385050070841\\
209.01	0.00385053876925687\\
210.01	0.00385057765276775\\
211.01	0.00385061737679758\\
212.01	0.00385065795959111\\
213.01	0.00385069941979058\\
214.01	0.00385074177644565\\
215.01	0.00385078504902181\\
216.01	0.00385082925740939\\
217.01	0.00385087442193333\\
218.01	0.00385092056336243\\
219.01	0.0038509677029194\\
220.01	0.00385101586229071\\
221.01	0.00385106506363686\\
222.01	0.00385111532960304\\
223.01	0.00385116668332969\\
224.01	0.00385121914846331\\
225.01	0.00385127274916822\\
226.01	0.00385132751013714\\
227.01	0.00385138345660398\\
228.01	0.00385144061435491\\
229.01	0.00385149900974111\\
230.01	0.00385155866969135\\
231.01	0.00385161962172451\\
232.01	0.00385168189396309\\
233.01	0.00385174551514626\\
234.01	0.00385181051464401\\
235.01	0.00385187692247087\\
236.01	0.00385194476930062\\
237.01	0.00385201408648062\\
238.01	0.0038520849060474\\
239.01	0.00385215726074126\\
240.01	0.00385223118402295\\
241.01	0.00385230671008926\\
242.01	0.00385238387388961\\
243.01	0.00385246271114327\\
244.01	0.0038525432583562\\
245.01	0.00385262555283911\\
246.01	0.00385270963272513\\
247.01	0.00385279553698928\\
248.01	0.00385288330546632\\
249.01	0.00385297297887111\\
250.01	0.00385306459881798\\
251.01	0.00385315820784145\\
252.01	0.00385325384941676\\
253.01	0.00385335156798175\\
254.01	0.00385345140895808\\
255.01	0.00385355341877441\\
256.01	0.00385365764488861\\
257.01	0.00385376413581204\\
258.01	0.00385387294113282\\
259.01	0.00385398411154134\\
260.01	0.00385409769885486\\
261.01	0.00385421375604393\\
262.01	0.00385433233725853\\
263.01	0.00385445349785557\\
264.01	0.00385457729442669\\
265.01	0.00385470378482684\\
266.01	0.00385483302820358\\
267.01	0.00385496508502703\\
268.01	0.00385510001712074\\
269.01	0.00385523788769319\\
270.01	0.00385537876137027\\
271.01	0.00385552270422845\\
272.01	0.00385566978382858\\
273.01	0.0038558200692513\\
274.01	0.00385597363113238\\
275.01	0.00385613054169973\\
276.01	0.00385629087481112\\
277.01	0.00385645470599272\\
278.01	0.00385662211247893\\
279.01	0.00385679317325323\\
280.01	0.00385696796908966\\
281.01	0.00385714658259634\\
282.01	0.00385732909825905\\
283.01	0.00385751560248686\\
284.01	0.00385770618365856\\
285.01	0.00385790093217053\\
286.01	0.0038580999404859\\
287.01	0.00385830330318483\\
288.01	0.003858511117016\\
289.01	0.003858723480951\\
290.01	0.00385894049623794\\
291.01	0.00385916226645865\\
292.01	0.00385938889758582\\
293.01	0.00385962049804375\\
294.01	0.00385985717876804\\
295.01	0.00386009905327003\\
296.01	0.00386034623770097\\
297.01	0.00386059885091891\\
298.01	0.00386085701455754\\
299.01	0.00386112085309692\\
300.01	0.00386139049393619\\
301.01	0.00386166606746884\\
302.01	0.00386194770716014\\
303.01	0.00386223554962652\\
304.01	0.00386252973471762\\
305.01	0.00386283040560144\\
306.01	0.00386313770885108\\
307.01	0.00386345179453493\\
308.01	0.00386377281630966\\
309.01	0.00386410093151575\\
310.01	0.00386443630127654\\
311.01	0.00386477909060014\\
312.01	0.00386512946848498\\
313.01	0.00386548760802831\\
314.01	0.00386585368653898\\
315.01	0.00386622788565294\\
316.01	0.00386661039145403\\
317.01	0.00386700139459747\\
318.01	0.00386740109043877\\
319.01	0.00386780967916585\\
320.01	0.00386822736593687\\
321.01	0.00386865436102218\\
322.01	0.00386909087995135\\
323.01	0.003869537143666\\
324.01	0.00386999337867754\\
325.01	0.00387045981723068\\
326.01	0.00387093669747325\\
327.01	0.00387142426363196\\
328.01	0.00387192276619501\\
329.01	0.00387243246210124\\
330.01	0.00387295361493713\\
331.01	0.00387348649514033\\
332.01	0.00387403138021206\\
333.01	0.00387458855493709\\
334.01	0.00387515831161349\\
335.01	0.00387574095028989\\
336.01	0.003876336779014\\
337.01	0.00387694611409026\\
338.01	0.0038775692803483\\
339.01	0.0038782066114227\\
340.01	0.00387885845004482\\
341.01	0.00387952514834604\\
342.01	0.00388020706817502\\
343.01	0.00388090458142858\\
344.01	0.00388161807039665\\
345.01	0.00388234792812257\\
346.01	0.00388309455878016\\
347.01	0.00388385837806733\\
348.01	0.00388463981361746\\
349.01	0.00388543930543053\\
350.01	0.00388625730632405\\
351.01	0.00388709428240512\\
352.01	0.00388795071356546\\
353.01	0.00388882709399971\\
354.01	0.00388972393274964\\
355.01	0.00389064175427477\\
356.01	0.00389158109905192\\
357.01	0.00389254252420378\\
358.01	0.00389352660416029\\
359.01	0.0038945339313536\\
360.01	0.0038955651169489\\
361.01	0.0038966207916132\\
362.01	0.0038977016063247\\
363.01	0.00389880823322475\\
364.01	0.00389994136651454\\
365.01	0.00390110172340126\\
366.01	0.00390229004509375\\
367.01	0.00390350709785288\\
368.01	0.00390475367409801\\
369.01	0.00390603059357446\\
370.01	0.00390733870458411\\
371.01	0.00390867888528189\\
372.01	0.00391005204504401\\
373.01	0.00391145912590796\\
374.01	0.00391290110409093\\
375.01	0.00391437899158689\\
376.01	0.00391589383784735\\
377.01	0.0039174467315468\\
378.01	0.0039190388024356\\
379.01	0.00392067122328169\\
380.01	0.00392234521190106\\
381.01	0.00392406203327494\\
382.01	0.00392582300175241\\
383.01	0.0039276294833331\\
384.01	0.00392948289802104\\
385.01	0.00393138472224056\\
386.01	0.00393333649129618\\
387.01	0.00393533980185767\\
388.01	0.00393739631444119\\
389.01	0.00393950775585072\\
390.01	0.0039416759215347\\
391.01	0.0039439026777978\\
392.01	0.00394618996379574\\
393.01	0.00394853979322038\\
394.01	0.00395095425556306\\
395.01	0.00395343551681587\\
396.01	0.00395598581944177\\
397.01	0.00395860748140703\\
398.01	0.00396130289402878\\
399.01	0.00396407451834235\\
400.01	0.00396692487963876\\
401.01	0.00396985655976391\\
402.01	0.00397287218670914\\
403.01	0.00397597442095758\\
404.01	0.00397916593799423\\
405.01	0.00398244940634153\\
406.01	0.003985827460465\\
407.01	0.00398930266792481\\
408.01	0.0039928774902608\\
409.01	0.00399655423733657\\
410.01	0.00400033501530679\\
411.01	0.00400422166910225\\
412.01	0.00400821572150827\\
413.01	0.00401231831273878\\
414.01	0.00401653014719362\\
415.01	0.00402085145824729\\
416.01	0.00402528200806303\\
417.01	0.00402982114841728\\
418.01	0.00403446798156182\\
419.01	0.00403922167896034\\
420.01	0.00404408237476168\\
421.01	0.00404905231270609\\
422.01	0.00405413425232351\\
423.01	0.00405933104602202\\
424.01	0.00406464564393058\\
425.01	0.00407008109911691\\
426.01	0.00407564057321982\\
427.01	0.00408132734253863\\
428.01	0.00408714480462842\\
429.01	0.00409309648545557\\
430.01	0.0040991860471756\\
431.01	0.00410541729660299\\
432.01	0.00411179419445057\\
433.01	0.00411832086542861\\
434.01	0.0041250016093043\\
435.01	0.00413184091303385\\
436.01	0.00413884346409908\\
437.01	0.00414601416519249\\
438.01	0.00415335815041885\\
439.01	0.00416088080320058\\
440.01	0.00416858777610178\\
441.01	0.00417648501281281\\
442.01	0.00418457877256984\\
443.01	0.00419287565732265\\
444.01	0.00420138264199804\\
445.01	0.00421010710825831\\
446.01	0.00421905688219473\\
447.01	0.00422824027645134\\
448.01	0.00423766613732213\\
449.01	0.00424734389741394\\
450.01	0.00425728363451118\\
451.01	0.00426749613731328\\
452.01	0.00427799297872741\\
453.01	0.00428878659738625\\
454.01	0.00429989038799372\\
455.01	0.00431131880096315\\
456.01	0.00432308745156734\\
457.01	0.00433521323841857\\
458.01	0.00434771447046317\\
459.01	0.00436061100073\\
460.01	0.00437392436366178\\
461.01	0.0043876779108203\\
462.01	0.00440189693681349\\
463.01	0.00441660878311331\\
464.01	0.00443184290152039\\
465.01	0.00444763085071449\\
466.01	0.00446400618770507\\
467.01	0.00448100419977486\\
468.01	0.00449866139996727\\
469.01	0.00451701467790344\\
470.01	0.00453609995443392\\
471.01	0.0045559501288019\\
472.01	0.00457659202439141\\
473.01	0.00459804192517578\\
474.01	0.00462029913787318\\
475.01	0.00464332508813105\\
476.01	0.00466689683504538\\
477.01	0.00469094932399357\\
478.01	0.00471548675428907\\
479.01	0.00474051214457602\\
480.01	0.00476602707176335\\
481.01	0.00479203136697413\\
482.01	0.00481852276470723\\
483.01	0.00484549650011952\\
484.01	0.00487294484167969\\
485.01	0.0049008565537592\\
486.01	0.00492921629070482\\
487.01	0.00495800392499192\\
488.01	0.0049871938186043\\
489.01	0.00501675405767654\\
490.01	0.00504664568720932\\
491.01	0.0050768220080112\\
492.01	0.00510722803578761\\
493.01	0.00513780027796394\\
494.01	0.0051684670553882\\
495.01	0.0051991495838186\\
496.01	0.00522976409131098\\
497.01	0.00526022558976409\\
498.01	0.00529045412589123\\
499.01	0.00532038464015382\\
500.01	0.00534998202242072\\
501.01	0.00537937490757551\\
502.01	0.00540890750955372\\
503.01	0.00543856765434311\\
504.01	0.00546831797970659\\
505.01	0.00549811956810894\\
506.01	0.00552793260853647\\
507.01	0.0055577172825164\\
508.01	0.00558743491008898\\
509.01	0.0056170493872629\\
510.01	0.00564652893542456\\
511.01	0.00567584816120002\\
512.01	0.00570499038638812\\
513.01	0.00573395014301281\\
514.01	0.00576273562554443\\
515.01	0.00579137073254395\\
516.01	0.00581989608708264\\
517.01	0.00584836806176718\\
518.01	0.00587685429683296\\
519.01	0.00590541214361032\\
520.01	0.00593404905157574\\
521.01	0.00596276379842327\\
522.01	0.00599155949860458\\
523.01	0.00602044410378867\\
524.01	0.00604943077707996\\
525.01	0.00607853806625773\\
526.01	0.00610778978314085\\
527.01	0.00613721448690835\\
528.01	0.00616684448893449\\
529.01	0.00619671432899642\\
530.01	0.00622685873489252\\
531.01	0.00625731020827809\\
532.01	0.00628809661523301\\
533.01	0.0063192399069386\\
534.01	0.00635075976253935\\
535.01	0.00638267741778237\\
536.01	0.00641501570232442\\
537.01	0.00644779862787694\\
538.01	0.00648105089573116\\
539.01	0.00651479735401082\\
540.01	0.00654906245383268\\
541.01	0.00658386977391781\\
542.01	0.0066192417007872\\
543.01	0.00665519935786468\\
544.01	0.00669176285587489\\
545.01	0.00672895184647044\\
546.01	0.00676678598891822\\
547.01	0.00680528492509266\\
548.01	0.00684446810936269\\
549.01	0.00688435466246006\\
550.01	0.00692496326702412\\
551.01	0.00696631211540255\\
552.01	0.0070084189134092\\
553.01	0.00705130093263077\\
554.01	0.00709497508902708\\
555.01	0.00713945801015897\\
556.01	0.00718476604845364\\
557.01	0.00723091523913168\\
558.01	0.0072779212495796\\
559.01	0.00732579933887666\\
560.01	0.00737456432605899\\
561.01	0.00742423056265521\\
562.01	0.00747481190325003\\
563.01	0.00752632166704584\\
564.01	0.00757877258459237\\
565.01	0.00763217672773581\\
566.01	0.007686545426493\\
567.01	0.00774188917822005\\
568.01	0.00779821754982424\\
569.01	0.00785553907106167\\
570.01	0.00791386111679867\\
571.01	0.00797318977653028\\
572.01	0.00803352971022384\\
573.01	0.00809488399052311\\
574.01	0.0081572539321533\\
575.01	0.00822063890954425\\
576.01	0.00828503616326107\\
577.01	0.00835044059570835\\
578.01	0.00841684455706406\\
579.01	0.00848423762317128\\
580.01	0.00855260636803562\\
581.01	0.00862193413461367\\
582.01	0.0086922008087244\\
583.01	0.00876338260223467\\
584.01	0.0088354518533399\\
585.01	0.00890837685398345\\
586.01	0.00898212171733074\\
587.01	0.00905664630181658\\
588.01	0.00913190621277292\\
589.01	0.009207852908237\\
590.01	0.00928443394251917\\
591.01	0.00936159338983446\\
592.01	0.00943927250119964\\
593.01	0.00951741066136708\\
594.01	0.00959594672938032\\
595.01	0.00967482086709944\\
596.01	0.0097539769856271\\
597.01	0.00983333321553398\\
598.01	0.00990826330299361\\
599.01	0.00997053306357289\\
599.02	0.00997104257304143\\
599.03	0.00997154904691472\\
599.04	0.00997205245565375\\
599.05	0.00997255276942872\\
599.06	0.00997304995811617\\
599.07	0.00997354399129606\\
599.08	0.00997403483824883\\
599.09	0.00997452246795248\\
599.1	0.00997500684907951\\
599.11	0.00997548794999398\\
599.12	0.00997596573874838\\
599.13	0.0099764401830806\\
599.14	0.0099769112504108\\
599.15	0.00997737890783826\\
599.16	0.00997784312213823\\
599.17	0.0099783038597587\\
599.18	0.00997876108681718\\
599.19	0.00997921476909742\\
599.2	0.00997966487204611\\
599.21	0.00998011136076955\\
599.22	0.00998055420003028\\
599.23	0.00998099335424369\\
599.24	0.00998142878747458\\
599.25	0.00998186046343367\\
599.26	0.00998228834359041\\
599.27	0.00998271238778643\\
599.28	0.00998313255546381\\
599.29	0.00998354880566111\\
599.3	0.00998396109700947\\
599.31	0.00998436938772849\\
599.32	0.00998477363562221\\
599.33	0.00998517379807499\\
599.34	0.00998556983204737\\
599.35	0.00998596169407188\\
599.36	0.00998634934024881\\
599.37	0.00998673272624192\\
599.38	0.00998711180727415\\
599.39	0.00998748653812326\\
599.4	0.00998785687311743\\
599.41	0.00998822276613081\\
599.42	0.00998858417057903\\
599.43	0.00998894103941468\\
599.44	0.00998929332512275\\
599.45	0.00998964097971596\\
599.46	0.00998998395473014\\
599.47	0.00999032220121951\\
599.48	0.0099906556697519\\
599.49	0.00999098431040396\\
599.5	0.00999130807275631\\
599.51	0.00999162690588862\\
599.52	0.00999194075837471\\
599.53	0.00999224957827746\\
599.54	0.00999255331314386\\
599.55	0.00999285190999987\\
599.56	0.00999314531534524\\
599.57	0.00999343347514839\\
599.58	0.00999371633484107\\
599.59	0.00999399383931314\\
599.6	0.00999426593290715\\
599.61	0.009994532559413\\
599.62	0.0099947936620624\\
599.63	0.00999504918352342\\
599.64	0.00999529906589491\\
599.65	0.00999554325070086\\
599.66	0.00999578167888471\\
599.67	0.00999601429080366\\
599.68	0.00999624102622284\\
599.69	0.00999646182430947\\
599.7	0.00999667662362697\\
599.71	0.00999688536212897\\
599.72	0.00999708797715331\\
599.73	0.00999728440541595\\
599.74	0.00999747458300482\\
599.75	0.0099976584453736\\
599.76	0.00999783592733551\\
599.77	0.00999800696305692\\
599.78	0.00999817148605102\\
599.79	0.00999832942917133\\
599.8	0.00999848072460517\\
599.81	0.00999862530386713\\
599.82	0.00999876309779239\\
599.83	0.00999889403653003\\
599.84	0.00999901804953622\\
599.85	0.00999913506556741\\
599.86	0.00999924501267341\\
599.87	0.00999934781819042\\
599.88	0.00999944340873394\\
599.89	0.0099995317101917\\
599.9	0.00999961264771648\\
599.91	0.00999968614571878\\
599.92	0.00999975212785957\\
599.93	0.00999981051704285\\
599.94	0.00999986123540817\\
599.95	0.00999990420432308\\
599.96	0.00999993934437554\\
599.97	0.00999996657536618\\
599.98	0.00999998581630055\\
599.99	0.00999999698538124\\
600	0.01\\
};
\addplot [color=red!75!mycolor17,solid,forget plot]
  table[row sep=crcr]{%
0.01	0.00396769113260109\\
1.01	0.00396769177020938\\
2.01	0.00396769242137629\\
3.01	0.00396769308639122\\
4.01	0.00396769376554953\\
5.01	0.00396769445915275\\
6.01	0.00396769516750901\\
7.01	0.00396769589093288\\
8.01	0.0039676966297459\\
9.01	0.00396769738427642\\
10.01	0.0039676981548598\\
11.01	0.0039676989418386\\
12.01	0.00396769974556255\\
13.01	0.00396770056638894\\
14.01	0.00396770140468322\\
15.01	0.00396770226081775\\
16.01	0.0039677031351734\\
17.01	0.00396770402813927\\
18.01	0.00396770494011262\\
19.01	0.00396770587149925\\
20.01	0.00396770682271361\\
21.01	0.00396770779417913\\
22.01	0.00396770878632824\\
23.01	0.00396770979960269\\
24.01	0.00396771083445369\\
25.01	0.00396771189134193\\
26.01	0.00396771297073839\\
27.01	0.0039677140731237\\
28.01	0.00396771519898899\\
29.01	0.00396771634883605\\
30.01	0.00396771752317712\\
31.01	0.00396771872253589\\
32.01	0.0039677199474468\\
33.01	0.0039677211984559\\
34.01	0.00396772247612104\\
35.01	0.00396772378101196\\
36.01	0.00396772511371082\\
37.01	0.00396772647481196\\
38.01	0.00396772786492264\\
39.01	0.00396772928466314\\
40.01	0.00396773073466704\\
41.01	0.0039677322155816\\
42.01	0.00396773372806764\\
43.01	0.0039677352728007\\
44.01	0.00396773685047039\\
45.01	0.00396773846178129\\
46.01	0.00396774010745304\\
47.01	0.0039677417882208\\
48.01	0.0039677435048356\\
49.01	0.00396774525806447\\
50.01	0.00396774704869077\\
51.01	0.00396774887751485\\
52.01	0.00396775074535444\\
53.01	0.00396775265304471\\
54.01	0.00396775460143854\\
55.01	0.00396775659140745\\
56.01	0.00396775862384134\\
57.01	0.0039677606996496\\
58.01	0.00396776281976091\\
59.01	0.00396776498512429\\
60.01	0.00396776719670855\\
61.01	0.00396776945550387\\
62.01	0.0039677717625213\\
63.01	0.00396777411879395\\
64.01	0.00396777652537718\\
65.01	0.00396777898334862\\
66.01	0.00396778149380962\\
67.01	0.00396778405788478\\
68.01	0.00396778667672308\\
69.01	0.0039677893514983\\
70.01	0.00396779208340937\\
71.01	0.00396779487368093\\
72.01	0.00396779772356412\\
73.01	0.00396780063433684\\
74.01	0.00396780360730464\\
75.01	0.00396780664380129\\
76.01	0.00396780974518895\\
77.01	0.00396781291285943\\
78.01	0.00396781614823408\\
79.01	0.00396781945276524\\
80.01	0.00396782282793649\\
81.01	0.00396782627526315\\
82.01	0.00396782979629334\\
83.01	0.00396783339260852\\
84.01	0.00396783706582401\\
85.01	0.00396784081759033\\
86.01	0.00396784464959313\\
87.01	0.00396784856355457\\
88.01	0.00396785256123405\\
89.01	0.00396785664442856\\
90.01	0.00396786081497409\\
91.01	0.00396786507474611\\
92.01	0.00396786942566049\\
93.01	0.00396787386967452\\
94.01	0.00396787840878758\\
95.01	0.0039678830450421\\
96.01	0.00396788778052475\\
97.01	0.00396789261736694\\
98.01	0.00396789755774617\\
99.01	0.00396790260388696\\
100.01	0.00396790775806193\\
101.01	0.00396791302259243\\
102.01	0.00396791839984992\\
103.01	0.00396792389225731\\
104.01	0.00396792950228941\\
105.01	0.00396793523247471\\
106.01	0.00396794108539625\\
107.01	0.00396794706369274\\
108.01	0.00396795317005978\\
109.01	0.00396795940725127\\
110.01	0.00396796577808052\\
111.01	0.00396797228542163\\
112.01	0.00396797893221054\\
113.01	0.00396798572144694\\
114.01	0.00396799265619489\\
115.01	0.00396799973958503\\
116.01	0.00396800697481547\\
117.01	0.00396801436515318\\
118.01	0.00396802191393611\\
119.01	0.00396802962457379\\
120.01	0.00396803750055007\\
121.01	0.00396804554542347\\
122.01	0.00396805376282972\\
123.01	0.0039680621564831\\
124.01	0.00396807073017803\\
125.01	0.00396807948779108\\
126.01	0.00396808843328268\\
127.01	0.00396809757069876\\
128.01	0.00396810690417269\\
129.01	0.00396811643792719\\
130.01	0.00396812617627647\\
131.01	0.00396813612362767\\
132.01	0.00396814628448382\\
133.01	0.00396815666344473\\
134.01	0.0039681672652101\\
135.01	0.00396817809458115\\
136.01	0.00396818915646294\\
137.01	0.0039682004558667\\
138.01	0.00396821199791209\\
139.01	0.00396822378782956\\
140.01	0.00396823583096281\\
141.01	0.00396824813277083\\
142.01	0.00396826069883113\\
143.01	0.0039682735348419\\
144.01	0.0039682866466246\\
145.01	0.00396830004012673\\
146.01	0.00396831372142474\\
147.01	0.00396832769672614\\
148.01	0.00396834197237329\\
149.01	0.00396835655484569\\
150.01	0.00396837145076337\\
151.01	0.00396838666688958\\
152.01	0.00396840221013409\\
153.01	0.00396841808755638\\
154.01	0.00396843430636869\\
155.01	0.00396845087393959\\
156.01	0.00396846779779748\\
157.01	0.00396848508563376\\
158.01	0.00396850274530658\\
159.01	0.00396852078484405\\
160.01	0.00396853921244861\\
161.01	0.00396855803650049\\
162.01	0.00396857726556138\\
163.01	0.00396859690837859\\
164.01	0.00396861697388921\\
165.01	0.00396863747122393\\
166.01	0.0039686584097117\\
167.01	0.00396867979888334\\
168.01	0.00396870164847651\\
169.01	0.00396872396844029\\
170.01	0.00396874676893914\\
171.01	0.00396877006035841\\
172.01	0.00396879385330839\\
173.01	0.00396881815862969\\
174.01	0.0039688429873983\\
175.01	0.00396886835092998\\
176.01	0.0039688942607866\\
177.01	0.00396892072878072\\
178.01	0.00396894776698106\\
179.01	0.00396897538771862\\
180.01	0.00396900360359177\\
181.01	0.00396903242747235\\
182.01	0.00396906187251186\\
183.01	0.00396909195214723\\
184.01	0.00396912268010698\\
185.01	0.0039691540704178\\
186.01	0.00396918613741108\\
187.01	0.00396921889572945\\
188.01	0.00396925236033333\\
189.01	0.00396928654650831\\
190.01	0.00396932146987196\\
191.01	0.00396935714638096\\
192.01	0.00396939359233903\\
193.01	0.00396943082440386\\
194.01	0.00396946885959507\\
195.01	0.00396950771530226\\
196.01	0.00396954740929303\\
197.01	0.00396958795972087\\
198.01	0.00396962938513432\\
199.01	0.00396967170448466\\
200.01	0.00396971493713552\\
201.01	0.00396975910287152\\
202.01	0.00396980422190737\\
203.01	0.00396985031489741\\
204.01	0.00396989740294529\\
205.01	0.00396994550761362\\
206.01	0.00396999465093401\\
207.01	0.00397004485541752\\
208.01	0.00397009614406499\\
209.01	0.00397014854037767\\
210.01	0.00397020206836825\\
211.01	0.0039702567525722\\
212.01	0.00397031261805877\\
213.01	0.0039703696904434\\
214.01	0.00397042799589886\\
215.01	0.00397048756116784\\
216.01	0.00397054841357571\\
217.01	0.00397061058104263\\
218.01	0.00397067409209725\\
219.01	0.00397073897588941\\
220.01	0.00397080526220421\\
221.01	0.003970872981476\\
222.01	0.00397094216480219\\
223.01	0.00397101284395801\\
224.01	0.00397108505141175\\
225.01	0.00397115882033909\\
226.01	0.00397123418463971\\
227.01	0.00397131117895225\\
228.01	0.00397138983867107\\
229.01	0.0039714701999629\\
230.01	0.00397155229978335\\
231.01	0.00397163617589495\\
232.01	0.00397172186688433\\
233.01	0.00397180941218067\\
234.01	0.00397189885207395\\
235.01	0.00397199022773438\\
236.01	0.00397208358123138\\
237.01	0.00397217895555359\\
238.01	0.00397227639462915\\
239.01	0.00397237594334665\\
240.01	0.0039724776475758\\
241.01	0.00397258155418947\\
242.01	0.00397268771108618\\
243.01	0.00397279616721204\\
244.01	0.00397290697258462\\
245.01	0.00397302017831611\\
246.01	0.00397313583663846\\
247.01	0.00397325400092682\\
248.01	0.00397337472572634\\
249.01	0.00397349806677728\\
250.01	0.00397362408104181\\
251.01	0.00397375282673136\\
252.01	0.00397388436333408\\
253.01	0.00397401875164358\\
254.01	0.00397415605378797\\
255.01	0.00397429633325934\\
256.01	0.00397443965494465\\
257.01	0.00397458608515648\\
258.01	0.00397473569166543\\
259.01	0.00397488854373228\\
260.01	0.00397504471214175\\
261.01	0.00397520426923631\\
262.01	0.00397536728895163\\
263.01	0.00397553384685206\\
264.01	0.00397570402016739\\
265.01	0.00397587788783033\\
266.01	0.00397605553051503\\
267.01	0.00397623703067639\\
268.01	0.00397642247259025\\
269.01	0.00397661194239492\\
270.01	0.00397680552813294\\
271.01	0.00397700331979466\\
272.01	0.00397720540936261\\
273.01	0.00397741189085668\\
274.01	0.00397762286038064\\
275.01	0.00397783841616962\\
276.01	0.0039780586586393\\
277.01	0.0039782836904354\\
278.01	0.00397851361648541\\
279.01	0.0039787485440506\\
280.01	0.0039789885827801\\
281.01	0.00397923384476608\\
282.01	0.00397948444459989\\
283.01	0.00397974049943065\\
284.01	0.00398000212902407\\
285.01	0.00398026945582379\\
286.01	0.00398054260501357\\
287.01	0.00398082170458154\\
288.01	0.00398110688538615\\
289.01	0.00398139828122336\\
290.01	0.00398169602889592\\
291.01	0.00398200026828435\\
292.01	0.00398231114242037\\
293.01	0.00398262879756076\\
294.01	0.00398295338326518\\
295.01	0.00398328505247446\\
296.01	0.00398362396159208\\
297.01	0.00398397027056686\\
298.01	0.00398432414297918\\
299.01	0.00398468574612829\\
300.01	0.00398505525112286\\
301.01	0.00398543283297384\\
302.01	0.00398581867068952\\
303.01	0.0039862129473742\\
304.01	0.00398661585032878\\
305.01	0.00398702757115467\\
306.01	0.00398744830586066\\
307.01	0.00398787825497264\\
308.01	0.00398831762364706\\
309.01	0.00398876662178717\\
310.01	0.00398922546416321\\
311.01	0.00398969437053592\\
312.01	0.00399017356578395\\
313.01	0.00399066328003468\\
314.01	0.00399116374880031\\
315.01	0.00399167521311687\\
316.01	0.00399219791968798\\
317.01	0.00399273212103379\\
318.01	0.00399327807564361\\
319.01	0.00399383604813475\\
320.01	0.00399440630941532\\
321.01	0.00399498913685312\\
322.01	0.00399558481445038\\
323.01	0.0039961936330233\\
324.01	0.00399681589038901\\
325.01	0.00399745189155805\\
326.01	0.00399810194893385\\
327.01	0.00399876638251954\\
328.01	0.00399944552013129\\
329.01	0.00400013969762028\\
330.01	0.00400084925910212\\
331.01	0.00400157455719517\\
332.01	0.00400231595326683\\
333.01	0.00400307381769054\\
334.01	0.00400384853011121\\
335.01	0.00400464047972214\\
336.01	0.00400545006555145\\
337.01	0.00400627769676144\\
338.01	0.00400712379295835\\
339.01	0.00400798878451636\\
340.01	0.00400887311291334\\
341.01	0.00400977723108244\\
342.01	0.00401070160377703\\
343.01	0.00401164670795204\\
344.01	0.00401261303316232\\
345.01	0.00401360108197817\\
346.01	0.0040146113704197\\
347.01	0.00401564442841108\\
348.01	0.00401670080025608\\
349.01	0.00401778104513627\\
350.01	0.00401888573763207\\
351.01	0.00402001546827048\\
352.01	0.00402117084409927\\
353.01	0.00402235248929085\\
354.01	0.00402356104577592\\
355.01	0.0040247971739118\\
356.01	0.00402606155318498\\
357.01	0.00402735488295299\\
358.01	0.00402867788322644\\
359.01	0.00403003129549559\\
360.01	0.00403141588360482\\
361.01	0.00403283243467717\\
362.01	0.00403428176009619\\
363.01	0.00403576469654615\\
364.01	0.00403728210711875\\
365.01	0.00403883488248996\\
366.01	0.00404042394217372\\
367.01	0.00404205023585984\\
368.01	0.00404371474484419\\
369.01	0.00404541848355822\\
370.01	0.0040471625012093\\
371.01	0.00404894788354172\\
372.01	0.00405077575473087\\
373.01	0.00405264727942405\\
374.01	0.00405456366494364\\
375.01	0.00405652616367\\
376.01	0.00405853607562413\\
377.01	0.00406059475127134\\
378.01	0.00406270359457264\\
379.01	0.00406486406631171\\
380.01	0.00406707768772983\\
381.01	0.00406934604450744\\
382.01	0.00407167079113238\\
383.01	0.00407405365570443\\
384.01	0.00407649644523024\\
385.01	0.00407900105147033\\
386.01	0.0040815694574101\\
387.01	0.00408420374443365\\
388.01	0.00408690610029082\\
389.01	0.00408967882796181\\
390.01	0.00409252435553178\\
391.01	0.0040954452472062\\
392.01	0.0040984442156084\\
393.01	0.00410152413551731\\
394.01	0.00410468805921486\\
395.01	0.00410793923362744\\
396.01	0.00411128111944832\\
397.01	0.00411471741243497\\
398.01	0.00411825206706065\\
399.01	0.00412188932267647\\
400.01	0.00412563373228683\\
401.01	0.00412949019395649\\
402.01	0.00413346398472439\\
403.01	0.00413756079668556\\
404.01	0.0041417867745759\\
405.01	0.00414614855372214\\
406.01	0.00415065329652599\\
407.01	0.00415530872466337\\
408.01	0.00416012314277424\\
409.01	0.00416510544743282\\
410.01	0.00417026511239615\\
411.01	0.00417561213723204\\
412.01	0.00418115694098542\\
413.01	0.00418691017498596\\
414.01	0.00419288241841614\\
415.01	0.00419908370574985\\
416.01	0.00420552281511527\\
417.01	0.0042122062189507\\
418.01	0.00421913656014379\\
419.01	0.00422631046422273\\
420.01	0.00423368312743107\\
421.01	0.00424121925948537\\
422.01	0.00424892264628497\\
423.01	0.00425679716728809\\
424.01	0.00426484679749374\\
425.01	0.0042730756093373\\
426.01	0.00428148777446545\\
427.01	0.00429008756535393\\
428.01	0.00429887935672305\\
429.01	0.00430786762669753\\
430.01	0.00431705695764628\\
431.01	0.00432645203662765\\
432.01	0.0043360576553494\\
433.01	0.00434587870953865\\
434.01	0.00435592019759466\\
435.01	0.00436618721837802\\
436.01	0.00437668496796002\\
437.01	0.00438741873512619\\
438.01	0.00439839389539268\\
439.01	0.00440961590324765\\
440.01	0.00442109028228488\\
441.01	0.00443282261283391\\
442.01	0.0044448185166292\\
443.01	0.00445708363797592\\
444.01	0.00446962362078902\\
445.01	0.0044824440807707\\
446.01	0.00449555057188222\\
447.01	0.00450894854612764\\
448.01	0.00452264330552268\\
449.01	0.00453663994495723\\
450.01	0.00455094328448361\\
451.01	0.00456555778937208\\
452.01	0.004580487476087\\
453.01	0.00459573580214639\\
454.01	0.00461130553766607\\
455.01	0.00462719861627069\\
456.01	0.00464341596302448\\
457.01	0.00465995729713684\\
458.01	0.00467682090752762\\
459.01	0.00469400339999125\\
460.01	0.0047114994158568\\
461.01	0.00472930132390904\\
462.01	0.00474739889026811\\
463.01	0.00476577893533903\\
464.01	0.00478442499349607\\
465.01	0.00480331700073351\\
466.01	0.00482243104930566\\
467.01	0.00484173926811322\\
468.01	0.00486120991560975\\
469.01	0.00488080781156211\\
470.01	0.004900495289608\\
471.01	0.00492023393042901\\
472.01	0.0049399874440854\\
473.01	0.00495972622148666\\
474.01	0.00497943428541837\\
475.01	0.00499913145132538\\
476.01	0.00501902720808554\\
477.01	0.00503918180239209\\
478.01	0.00505958481623904\\
479.01	0.00508022431726177\\
480.01	0.00510108678874226\\
481.01	0.00512215707757294\\
482.01	0.00514341836753322\\
483.01	0.00516485218691057\\
484.01	0.00518643846156579\\
485.01	0.00520815562694403\\
486.01	0.00522998081483872\\
487.01	0.00525189013303434\\
488.01	0.00527385905804288\\
489.01	0.00529586296252547\\
490.01	0.00531787779894417\\
491.01	0.00533988095841141\\
492.01	0.00536185231690581\\
493.01	0.0053837754674602\\
494.01	0.00540563911290601\\
495.01	0.00542743855628496\\
496.01	0.00544917717266305\\
497.01	0.00547086765803982\\
498.01	0.00549253271345604\\
499.01	0.00551420461998016\\
500.01	0.00553592286251936\\
501.01	0.00555772596865946\\
502.01	0.00557962160141781\\
503.01	0.00560160101697943\\
504.01	0.00562365677908444\\
505.01	0.00564578336202322\\
506.01	0.0056679775498117\\
507.01	0.00569023882959235\\
508.01	0.00571256975497018\\
509.01	0.00573497624721442\\
510.01	0.005757467793804\\
511.01	0.00578005749547837\\
512.01	0.00580276190630355\\
513.01	0.00582560060892912\\
514.01	0.00584859547358037\\
515.01	0.00587176957142045\\
516.01	0.00589514576173578\\
517.01	0.00591874506485783\\
518.01	0.00594258509454382\\
519.01	0.00596667933551578\\
520.01	0.00599103986781382\\
521.01	0.00601568016865678\\
522.01	0.00604061526373739\\
523.01	0.00606586156842957\\
524.01	0.00609143666198887\\
525.01	0.00611735899771229\\
526.01	0.00614364756017159\\
527.01	0.00617032149155369\\
528.01	0.0061973997219789\\
529.01	0.00622490065177947\\
530.01	0.00625284194488335\\
531.01	0.00628124049607781\\
532.01	0.006310112620824\\
533.01	0.00633947446115379\\
534.01	0.00636934236912462\\
535.01	0.00639973293020422\\
536.01	0.00643066283958986\\
537.01	0.00646214878170193\\
538.01	0.00649420732798515\\
539.01	0.00652685486259719\\
540.01	0.00656010754382047\\
541.01	0.00659398130508659\\
542.01	0.006628491892921\\
543.01	0.00666365492999969\\
544.01	0.00669948598115028\\
545.01	0.00673600059244936\\
546.01	0.00677321428426587\\
547.01	0.00681114252189872\\
548.01	0.00684980069166331\\
549.01	0.00688920408657994\\
550.01	0.00692936790038687\\
551.01	0.00697030722712198\\
552.01	0.00701203706224635\\
553.01	0.0070545723005976\\
554.01	0.00709792772687743\\
555.01	0.00714211799637007\\
556.01	0.0071871576071451\\
557.01	0.00723306086787254\\
558.01	0.00727984186354769\\
559.01	0.00732751441838491\\
560.01	0.00737609205440055\\
561.01	0.00742558794421976\\
562.01	0.00747601485687485\\
563.01	0.00752738509579055\\
564.01	0.00757971042866334\\
565.01	0.00763300200932895\\
566.01	0.00768727029169022\\
567.01	0.00774252493532158\\
568.01	0.0077987747020062\\
569.01	0.00785602734245903\\
570.01	0.0079142894726261\\
571.01	0.0079735664391176\\
572.01	0.00803386217349965\\
573.01	0.00809517903531001\\
574.01	0.00815751764376637\\
575.01	0.00822087669824639\\
576.01	0.00828525278781392\\
577.01	0.00835064019039778\\
578.01	0.00841703066268948\\
579.01	0.008484413222397\\
580.01	0.00855277392521094\\
581.01	0.00862209563973181\\
582.01	0.00869235782474246\\
583.01	0.00876353631465262\\
584.01	0.00883560312078353\\
585.01	0.00890852625849137\\
586.01	0.00898226961305655\\
587.01	0.00905679286092972\\
588.01	0.00913205146750303\\
589.01	0.00920799678828045\\
590.01	0.00928457630742256\\
591.01	0.00936173405646149\\
592.01	0.00943941126691484\\
593.01	0.00951754732405423\\
594.01	0.00959608110579566\\
595.01	0.00967495281130476\\
596.01	0.00975410640935049\\
597.01	0.0098334186165282\\
598.01	0.00990826330299362\\
599.01	0.00997053306357288\\
599.02	0.00997104257304143\\
599.03	0.00997154904691472\\
599.04	0.00997205245565375\\
599.05	0.00997255276942872\\
599.06	0.00997304995811617\\
599.07	0.00997354399129606\\
599.08	0.00997403483824883\\
599.09	0.00997452246795248\\
599.1	0.00997500684907951\\
599.11	0.00997548794999398\\
599.12	0.00997596573874838\\
599.13	0.0099764401830806\\
599.14	0.0099769112504108\\
599.15	0.00997737890783826\\
599.16	0.00997784312213823\\
599.17	0.0099783038597587\\
599.18	0.00997876108681718\\
599.19	0.00997921476909742\\
599.2	0.00997966487204611\\
599.21	0.00998011136076955\\
599.22	0.00998055420003028\\
599.23	0.00998099335424369\\
599.24	0.00998142878747458\\
599.25	0.00998186046343367\\
599.26	0.00998228834359041\\
599.27	0.00998271238778643\\
599.28	0.00998313255546381\\
599.29	0.00998354880566111\\
599.3	0.00998396109700947\\
599.31	0.00998436938772849\\
599.32	0.00998477363562221\\
599.33	0.00998517379807499\\
599.34	0.00998556983204737\\
599.35	0.00998596169407188\\
599.36	0.00998634934024881\\
599.37	0.00998673272624191\\
599.38	0.00998711180727415\\
599.39	0.00998748653812326\\
599.4	0.00998785687311743\\
599.41	0.00998822276613081\\
599.42	0.00998858417057903\\
599.43	0.00998894103941468\\
599.44	0.00998929332512275\\
599.45	0.00998964097971596\\
599.46	0.00998998395473014\\
599.47	0.00999032220121951\\
599.48	0.0099906556697519\\
599.49	0.00999098431040396\\
599.5	0.00999130807275631\\
599.51	0.00999162690588863\\
599.52	0.00999194075837471\\
599.53	0.00999224957827746\\
599.54	0.00999255331314387\\
599.55	0.00999285190999987\\
599.56	0.00999314531534524\\
599.57	0.00999343347514839\\
599.58	0.00999371633484107\\
599.59	0.00999399383931314\\
599.6	0.00999426593290715\\
599.61	0.009994532559413\\
599.62	0.0099947936620624\\
599.63	0.00999504918352342\\
599.64	0.00999529906589491\\
599.65	0.00999554325070086\\
599.66	0.00999578167888471\\
599.67	0.00999601429080366\\
599.68	0.00999624102622283\\
599.69	0.00999646182430947\\
599.7	0.00999667662362697\\
599.71	0.00999688536212897\\
599.72	0.00999708797715332\\
599.73	0.00999728440541596\\
599.74	0.00999747458300482\\
599.75	0.0099976584453736\\
599.76	0.0099978359273355\\
599.77	0.00999800696305692\\
599.78	0.00999817148605102\\
599.79	0.00999832942917133\\
599.8	0.00999848072460517\\
599.81	0.00999862530386713\\
599.82	0.00999876309779239\\
599.83	0.00999889403653003\\
599.84	0.00999901804953622\\
599.85	0.00999913506556741\\
599.86	0.00999924501267341\\
599.87	0.00999934781819042\\
599.88	0.00999944340873394\\
599.89	0.0099995317101917\\
599.9	0.00999961264771648\\
599.91	0.00999968614571878\\
599.92	0.00999975212785957\\
599.93	0.00999981051704285\\
599.94	0.00999986123540817\\
599.95	0.00999990420432308\\
599.96	0.00999993934437554\\
599.97	0.00999996657536618\\
599.98	0.00999998581630055\\
599.99	0.00999999698538124\\
600	0.01\\
};
\addplot [color=red!80!mycolor19,solid,forget plot]
  table[row sep=crcr]{%
0.01	0.00420172588690648\\
1.01	0.00420172685462246\\
2.01	0.00420172784292355\\
3.01	0.00420172885224883\\
4.01	0.0042017298830468\\
5.01	0.00420173093577565\\
6.01	0.00420173201090362\\
7.01	0.0042017331089086\\
8.01	0.00420173423027881\\
9.01	0.00420173537551309\\
10.01	0.00420173654512063\\
11.01	0.00420173773962192\\
12.01	0.00420173895954846\\
13.01	0.00420174020544292\\
14.01	0.00420174147785983\\
15.01	0.00420174277736569\\
16.01	0.0042017441045389\\
17.01	0.00420174545997036\\
18.01	0.00420174684426371\\
19.01	0.00420174825803533\\
20.01	0.00420174970191506\\
21.01	0.00420175117654616\\
22.01	0.00420175268258559\\
23.01	0.00420175422070443\\
24.01	0.00420175579158825\\
25.01	0.00420175739593738\\
26.01	0.00420175903446679\\
27.01	0.00420176070790722\\
28.01	0.00420176241700494\\
29.01	0.00420176416252213\\
30.01	0.0042017659452375\\
31.01	0.00420176776594622\\
32.01	0.00420176962546091\\
33.01	0.00420177152461128\\
34.01	0.00420177346424515\\
35.01	0.00420177544522823\\
36.01	0.00420177746844499\\
37.01	0.00420177953479893\\
38.01	0.00420178164521279\\
39.01	0.00420178380062924\\
40.01	0.00420178600201125\\
41.01	0.00420178825034244\\
42.01	0.00420179054662748\\
43.01	0.00420179289189276\\
44.01	0.00420179528718653\\
45.01	0.0042017977335799\\
46.01	0.00420180023216656\\
47.01	0.00420180278406401\\
48.01	0.00420180539041372\\
49.01	0.00420180805238159\\
50.01	0.00420181077115887\\
51.01	0.00420181354796213\\
52.01	0.00420181638403413\\
53.01	0.00420181928064438\\
54.01	0.00420182223908996\\
55.01	0.00420182526069544\\
56.01	0.00420182834681423\\
57.01	0.00420183149882858\\
58.01	0.00420183471815069\\
59.01	0.00420183800622304\\
60.01	0.00420184136451911\\
61.01	0.00420184479454436\\
62.01	0.00420184829783626\\
63.01	0.00420185187596585\\
64.01	0.0042018555305374\\
65.01	0.0042018592631904\\
66.01	0.00420186307559901\\
67.01	0.00420186696947389\\
68.01	0.00420187094656237\\
69.01	0.0042018750086492\\
70.01	0.00420187915755808\\
71.01	0.00420188339515166\\
72.01	0.00420188772333287\\
73.01	0.00420189214404561\\
74.01	0.00420189665927558\\
75.01	0.00420190127105129\\
76.01	0.00420190598144504\\
77.01	0.00420191079257388\\
78.01	0.00420191570660039\\
79.01	0.00420192072573393\\
80.01	0.00420192585223129\\
81.01	0.00420193108839795\\
82.01	0.00420193643658928\\
83.01	0.00420194189921129\\
84.01	0.00420194747872223\\
85.01	0.00420195317763293\\
86.01	0.00420195899850874\\
87.01	0.00420196494397025\\
88.01	0.00420197101669458\\
89.01	0.00420197721941677\\
90.01	0.00420198355493073\\
91.01	0.00420199002609065\\
92.01	0.00420199663581257\\
93.01	0.00420200338707546\\
94.01	0.00420201028292251\\
95.01	0.00420201732646266\\
96.01	0.00420202452087198\\
97.01	0.0042020318693954\\
98.01	0.00420203937534767\\
99.01	0.00420204704211528\\
100.01	0.00420205487315785\\
101.01	0.00420206287200989\\
102.01	0.00420207104228214\\
103.01	0.00420207938766354\\
104.01	0.00420208791192274\\
105.01	0.00420209661890997\\
106.01	0.00420210551255851\\
107.01	0.00420211459688684\\
108.01	0.00420212387600019\\
109.01	0.00420213335409295\\
110.01	0.00420214303544998\\
111.01	0.00420215292444885\\
112.01	0.00420216302556202\\
113.01	0.00420217334335846\\
114.01	0.00420218388250633\\
115.01	0.00420219464777458\\
116.01	0.00420220564403539\\
117.01	0.00420221687626694\\
118.01	0.00420222834955447\\
119.01	0.00420224006909378\\
120.01	0.00420225204019301\\
121.01	0.00420226426827544\\
122.01	0.00420227675888177\\
123.01	0.0042022895176726\\
124.01	0.00420230255043144\\
125.01	0.00420231586306694\\
126.01	0.00420232946161561\\
127.01	0.00420234335224519\\
128.01	0.00420235754125675\\
129.01	0.00420237203508836\\
130.01	0.00420238684031733\\
131.01	0.00420240196366387\\
132.01	0.00420241741199351\\
133.01	0.00420243319232119\\
134.01	0.0042024493118135\\
135.01	0.00420246577779278\\
136.01	0.00420248259774024\\
137.01	0.0042024997792995\\
138.01	0.00420251733027965\\
139.01	0.00420253525865973\\
140.01	0.00420255357259154\\
141.01	0.00420257228040419\\
142.01	0.00420259139060747\\
143.01	0.00420261091189592\\
144.01	0.00420263085315276\\
145.01	0.0042026512234544\\
146.01	0.00420267203207395\\
147.01	0.00420269328848648\\
148.01	0.00420271500237257\\
149.01	0.00420273718362328\\
150.01	0.00420275984234433\\
151.01	0.00420278298886134\\
152.01	0.00420280663372415\\
153.01	0.00420283078771193\\
154.01	0.00420285546183846\\
155.01	0.00420288066735673\\
156.01	0.00420290641576412\\
157.01	0.00420293271880852\\
158.01	0.00420295958849301\\
159.01	0.00420298703708202\\
160.01	0.00420301507710659\\
161.01	0.00420304372137022\\
162.01	0.00420307298295526\\
163.01	0.00420310287522871\\
164.01	0.00420313341184826\\
165.01	0.0042031646067693\\
166.01	0.00420319647425037\\
167.01	0.00420322902886092\\
168.01	0.00420326228548756\\
169.01	0.00420329625934077\\
170.01	0.00420333096596262\\
171.01	0.00420336642123348\\
172.01	0.00420340264137964\\
173.01	0.00420343964298109\\
174.01	0.00420347744297875\\
175.01	0.00420351605868287\\
176.01	0.00420355550778106\\
177.01	0.00420359580834613\\
178.01	0.004203636978845\\
179.01	0.00420367903814729\\
180.01	0.00420372200553398\\
181.01	0.0042037659007062\\
182.01	0.00420381074379494\\
183.01	0.00420385655537051\\
184.01	0.00420390335645147\\
185.01	0.00420395116851523\\
186.01	0.00420400001350772\\
187.01	0.00420404991385355\\
188.01	0.00420410089246709\\
189.01	0.00420415297276246\\
190.01	0.00420420617866502\\
191.01	0.00420426053462265\\
192.01	0.00420431606561683\\
193.01	0.00420437279717452\\
194.01	0.00420443075538043\\
195.01	0.00420448996688871\\
196.01	0.00420455045893599\\
197.01	0.00420461225935385\\
198.01	0.00420467539658205\\
199.01	0.00420473989968179\\
200.01	0.00420480579834943\\
201.01	0.00420487312293046\\
202.01	0.00420494190443378\\
203.01	0.00420501217454659\\
204.01	0.00420508396564852\\
205.01	0.00420515731082764\\
206.01	0.00420523224389555\\
207.01	0.00420530879940367\\
208.01	0.00420538701265945\\
209.01	0.00420546691974286\\
210.01	0.00420554855752335\\
211.01	0.0042056319636775\\
212.01	0.00420571717670665\\
213.01	0.00420580423595522\\
214.01	0.00420589318162891\\
215.01	0.00420598405481427\\
216.01	0.00420607689749763\\
217.01	0.00420617175258493\\
218.01	0.0042062686639225\\
219.01	0.00420636767631748\\
220.01	0.0042064688355589\\
221.01	0.00420657218843963\\
222.01	0.00420667778277824\\
223.01	0.00420678566744215\\
224.01	0.00420689589237011\\
225.01	0.00420700850859657\\
226.01	0.00420712356827535\\
227.01	0.00420724112470472\\
228.01	0.00420736123235277\\
229.01	0.00420748394688271\\
230.01	0.0042076093251803\\
231.01	0.00420773742537993\\
232.01	0.00420786830689271\\
233.01	0.004208002030435\\
234.01	0.00420813865805676\\
235.01	0.00420827825317152\\
236.01	0.00420842088058652\\
237.01	0.00420856660653361\\
238.01	0.0042087154987008\\
239.01	0.00420886762626471\\
240.01	0.00420902305992332\\
241.01	0.00420918187193034\\
242.01	0.00420934413612904\\
243.01	0.00420950992798827\\
244.01	0.00420967932463812\\
245.01	0.00420985240490712\\
246.01	0.00421002924936002\\
247.01	0.00421020994033658\\
248.01	0.00421039456199077\\
249.01	0.00421058320033137\\
250.01	0.00421077594326359\\
251.01	0.00421097288063096\\
252.01	0.00421117410425897\\
253.01	0.0042113797079991\\
254.01	0.00421158978777403\\
255.01	0.00421180444162433\\
256.01	0.00421202376975553\\
257.01	0.00421224787458679\\
258.01	0.00421247686080005\\
259.01	0.00421271083539126\\
260.01	0.00421294990772209\\
261.01	0.00421319418957315\\
262.01	0.00421344379519807\\
263.01	0.00421369884137901\\
264.01	0.00421395944748389\\
265.01	0.00421422573552468\\
266.01	0.00421449783021652\\
267.01	0.00421477585903855\\
268.01	0.00421505995229708\\
269.01	0.00421535024318848\\
270.01	0.00421564686786532\\
271.01	0.00421594996550283\\
272.01	0.00421625967836696\\
273.01	0.00421657615188512\\
274.01	0.00421689953471753\\
275.01	0.00421722997883092\\
276.01	0.00421756763957329\\
277.01	0.00421791267575096\\
278.01	0.00421826524970746\\
279.01	0.00421862552740411\\
280.01	0.0042189936785028\\
281.01	0.00421936987645027\\
282.01	0.00421975429856489\\
283.01	0.00422014712612516\\
284.01	0.00422054854446124\\
285.01	0.00422095874304701\\
286.01	0.00422137791559618\\
287.01	0.00422180626015968\\
288.01	0.00422224397922555\\
289.01	0.00422269127982176\\
290.01	0.00422314837362106\\
291.01	0.00422361547704868\\
292.01	0.00422409281139248\\
293.01	0.00422458060291618\\
294.01	0.00422507908297536\\
295.01	0.0042255884881358\\
296.01	0.00422610906029544\\
297.01	0.00422664104680948\\
298.01	0.00422718470061817\\
299.01	0.00422774028037795\\
300.01	0.0042283080505961\\
301.01	0.00422888828176908\\
302.01	0.00422948125052353\\
303.01	0.00423008723976222\\
304.01	0.00423070653881283\\
305.01	0.0042313394435804\\
306.01	0.00423198625670501\\
307.01	0.00423264728772248\\
308.01	0.00423332285322914\\
309.01	0.00423401327705263\\
310.01	0.00423471889042508\\
311.01	0.00423544003216229\\
312.01	0.00423617704884756\\
313.01	0.00423693029502038\\
314.01	0.00423770013336931\\
315.01	0.00423848693493228\\
316.01	0.00423929107930002\\
317.01	0.00424011295482681\\
318.01	0.00424095295884599\\
319.01	0.00424181149789189\\
320.01	0.00424268898792847\\
321.01	0.00424358585458334\\
322.01	0.00424450253338882\\
323.01	0.00424543947003042\\
324.01	0.00424639712060161\\
325.01	0.00424737595186599\\
326.01	0.00424837644152752\\
327.01	0.00424939907850835\\
328.01	0.00425044436323427\\
329.01	0.00425151280792907\\
330.01	0.0042526049369172\\
331.01	0.00425372128693537\\
332.01	0.00425486240745378\\
333.01	0.00425602886100651\\
334.01	0.00425722122353138\\
335.01	0.00425844008472079\\
336.01	0.00425968604838278\\
337.01	0.00426095973281229\\
338.01	0.00426226177117511\\
339.01	0.00426359281190181\\
340.01	0.00426495351909451\\
341.01	0.00426634457294582\\
342.01	0.00426776667017044\\
343.01	0.00426922052444982\\
344.01	0.00427070686688991\\
345.01	0.00427222644649376\\
346.01	0.00427378003064713\\
347.01	0.00427536840561905\\
348.01	0.00427699237707777\\
349.01	0.00427865277062074\\
350.01	0.00428035043232144\\
351.01	0.00428208622929097\\
352.01	0.00428386105025584\\
353.01	0.00428567580615186\\
354.01	0.00428753143073425\\
355.01	0.0042894288812035\\
356.01	0.00429136913884787\\
357.01	0.0042933532097009\\
358.01	0.00429538212521485\\
359.01	0.00429745694294845\\
360.01	0.00429957874726847\\
361.01	0.00430174865006501\\
362.01	0.00430396779147629\\
363.01	0.00430623734062516\\
364.01	0.00430855849636115\\
365.01	0.00431093248800789\\
366.01	0.00431336057611158\\
367.01	0.00431584405318577\\
368.01	0.00431838424444871\\
369.01	0.0043209825085456\\
370.01	0.00432364023824944\\
371.01	0.00432635886113185\\
372.01	0.00432913984019159\\
373.01	0.00433198467443087\\
374.01	0.00433489489936206\\
375.01	0.00433787208742854\\
376.01	0.00434091784831726\\
377.01	0.00434403382914045\\
378.01	0.00434722171445423\\
379.01	0.00435048322608143\\
380.01	0.0043538201226959\\
381.01	0.00435723419912171\\
382.01	0.00436072728528873\\
383.01	0.00436430124477905\\
384.01	0.00436795797288421\\
385.01	0.00437169939408341\\
386.01	0.00437552745883306\\
387.01	0.00437944413954312\\
388.01	0.00438345142559465\\
389.01	0.0043875513172268\\
390.01	0.00439174581809861\\
391.01	0.00439603692629674\\
392.01	0.0044004266235303\\
393.01	0.00440491686221406\\
394.01	0.0044095095501035\\
395.01	0.00441420653210201\\
396.01	0.00441900956882048\\
397.01	0.00442392031142733\\
398.01	0.0044289402722959\\
399.01	0.00443407079093332\\
400.01	0.00443931299467878\\
401.01	0.00444466775369047\\
402.01	0.00445013562983471\\
403.01	0.00445571681925999\\
404.01	0.00446141108873049\\
405.01	0.00446721770626145\\
406.01	0.00447313536731395\\
407.01	0.00447916211888656\\
408.01	0.00448529528542208\\
409.01	0.00449153140274489\\
410.01	0.00449786616954412\\
411.01	0.00450429443060907\\
412.01	0.00451081021265944\\
413.01	0.00451740684295129\\
414.01	0.00452407719390642\\
415.01	0.00453081411525441\\
416.01	0.00453761114055384\\
417.01	0.00454446359018167\\
418.01	0.00455137024166802\\
419.01	0.00455833580572441\\
420.01	0.00456540750156501\\
421.01	0.00457262625384917\\
422.01	0.00457999472743888\\
423.01	0.00458751559386393\\
424.01	0.00459519152627431\\
425.01	0.00460302519375548\\
426.01	0.0046110192549417\\
427.01	0.00461917635085789\\
428.01	0.00462749909691135\\
429.01	0.00463599007395152\\
430.01	0.00464465181830881\\
431.01	0.00465348681071508\\
432.01	0.00466249746400581\\
433.01	0.00467168610949367\\
434.01	0.0046810549819018\\
435.01	0.0046906062027377\\
436.01	0.00470034176198623\\
437.01	0.00471026349799995\\
438.01	0.00472037307546404\\
439.01	0.00473067196132161\\
440.01	0.00474116139854958\\
441.01	0.00475184237769755\\
442.01	0.00476271560611873\\
443.01	0.00477378147486197\\
444.01	0.00478504002323578\\
445.01	0.00479649090112061\\
446.01	0.00480813332918634\\
447.01	0.00481996605728037\\
448.01	0.00483198732138755\\
449.01	0.00484419479973806\\
450.01	0.00485658556885794\\
451.01	0.00486915606062784\\
452.01	0.00488190202174907\\
453.01	0.00489481847742305\\
454.01	0.00490789970154039\\
455.01	0.00492113919625864\\
456.01	0.0049345296845308\\
457.01	0.00494806311993755\\
458.01	0.00496173071906386\\
459.01	0.00497552302263643\\
460.01	0.00498942999265438\\
461.01	0.00500344115374819\\
462.01	0.00501754578785374\\
463.01	0.00503173319181553\\
464.01	0.00504599300743304\\
465.01	0.00506031563227545\\
466.01	0.00507469271665208\\
467.01	0.00508911774641395\\
468.01	0.0051035867013233\\
469.01	0.00511809876242053\\
470.01	0.00513265701607844\\
471.01	0.00514726906286329\\
472.01	0.00516194737965422\\
473.01	0.00517670919482519\\
474.01	0.00519157550609556\\
475.01	0.00520656856541381\\
476.01	0.00522169966977432\\
477.01	0.00523696353715798\\
478.01	0.00525235315274364\\
479.01	0.00526786132161748\\
480.01	0.00528348076292864\\
481.01	0.00529920422313938\\
482.01	0.00531502460960456\\
483.01	0.00533093514519564\\
484.01	0.00534692954392317\\
485.01	0.0053630022063977\\
486.01	0.00537914843246287\\
487.01	0.00539536464632598\\
488.01	0.00541164862694314\\
489.01	0.00542799973321532\\
490.01	0.00544441910970511\\
491.01	0.00546090985416549\\
492.01	0.00547747712339409\\
493.01	0.00549412814925879\\
494.01	0.00551087213306035\\
495.01	0.00552771998516163\\
496.01	0.00554468388021921\\
497.01	0.00556177661002802\\
498.01	0.00557901074193831\\
499.01	0.00559639763997383\\
500.01	0.00561394649157437\\
501.01	0.00563166366258844\\
502.01	0.00564955376614987\\
503.01	0.00566762202633554\\
504.01	0.00568587478306898\\
505.01	0.00570431951910345\\
506.01	0.00572296485685063\\
507.01	0.00574182052090387\\
508.01	0.00576089726175706\\
509.01	0.00578020673786796\\
510.01	0.00579976135590433\\
511.01	0.0058195740729757\\
512.01	0.00583965817005484\\
513.01	0.00586002701264503\\
514.01	0.00588069382273103\\
515.01	0.00590167149422482\\
516.01	0.00592297249050802\\
517.01	0.00594460886351864\\
518.01	0.00596659242253623\\
519.01	0.00598893504221198\\
520.01	0.00601164894380868\\
521.01	0.00603474672338097\\
522.01	0.00605824127291234\\
523.01	0.00608214569417413\\
524.01	0.00610647321396675\\
525.01	0.00613123710668263\\
526.01	0.00615645063063178\\
527.01	0.00618212698434462\\
528.01	0.00620827928773983\\
529.01	0.00623492059029043\\
530.01	0.00626206390390152\\
531.01	0.00628972225221723\\
532.01	0.00631790872134252\\
533.01	0.00634663649186952\\
534.01	0.00637591883753616\\
535.01	0.00640576910431087\\
536.01	0.00643620069168729\\
537.01	0.00646722704104985\\
538.01	0.00649886163132984\\
539.01	0.0065311179812656\\
540.01	0.00656400965665378\\
541.01	0.0065975502801195\\
542.01	0.00663175354036419\\
543.01	0.0066666331978926\\
544.01	0.00670220308517147\\
545.01	0.00673847710119523\\
546.01	0.00677546920293627\\
547.01	0.00681319339652738\\
548.01	0.00685166372872077\\
549.01	0.00689089427789721\\
550.01	0.0069308991437064\\
551.01	0.00697169243442191\\
552.01	0.00701328825122415\\
553.01	0.00705570066888292\\
554.01	0.00709894371263774\\
555.01	0.0071430313313644\\
556.01	0.007187977367202\\
557.01	0.00723379552157973\\
558.01	0.00728049931723415\\
559.01	0.00732810205567089\\
560.01	0.0073766167695427\\
561.01	0.0074260561694634\\
562.01	0.00747643258482896\\
563.01	0.00752775789824955\\
564.01	0.0075800434731939\\
565.01	0.00763330007440498\\
566.01	0.00768753778057827\\
567.01	0.00774276588874375\\
568.01	0.00779899280979131\\
569.01	0.00785622595460367\\
570.01	0.00791447161029822\\
571.01	0.00797373480612668\\
572.01	0.00803401916864335\\
573.01	0.00809532676583651\\
574.01	0.00815765794003961\\
575.01	0.00822101112961452\\
576.01	0.00828538267964387\\
577.01	0.00835076664220159\\
578.01	0.00841715456719787\\
579.01	0.00848453528535402\\
580.01	0.00855289468557565\\
581.01	0.00862221548991046\\
582.01	0.00869247703044523\\
583.01	0.00876365503397906\\
584.01	0.00883572142218282\\
585.01	0.00890864413730972\\
586.01	0.00898238700647075\\
587.01	0.00905690966117702\\
588.01	0.00913216753344726\\
589.01	0.00920811195549929\\
590.01	0.00928469039714949\\
591.01	0.00936184688385573\\
592.01	0.00943952264925284\\
593.01	0.00951765708953015\\
594.01	0.00959618910368957\\
595.01	0.00967505892433464\\
596.01	0.00975421056908414\\
597.01	0.00983347437625554\\
598.01	0.00990826330299362\\
599.01	0.00997053306357289\\
599.02	0.00997104257304143\\
599.03	0.00997154904691472\\
599.04	0.00997205245565375\\
599.05	0.00997255276942872\\
599.06	0.00997304995811617\\
599.07	0.00997354399129606\\
599.08	0.00997403483824883\\
599.09	0.00997452246795248\\
599.1	0.00997500684907952\\
599.11	0.00997548794999398\\
599.12	0.00997596573874838\\
599.13	0.0099764401830806\\
599.14	0.0099769112504108\\
599.15	0.00997737890783826\\
599.16	0.00997784312213823\\
599.17	0.0099783038597587\\
599.18	0.00997876108681718\\
599.19	0.00997921476909742\\
599.2	0.00997966487204611\\
599.21	0.00998011136076955\\
599.22	0.00998055420003028\\
599.23	0.00998099335424369\\
599.24	0.00998142878747458\\
599.25	0.00998186046343367\\
599.26	0.00998228834359041\\
599.27	0.00998271238778643\\
599.28	0.00998313255546381\\
599.29	0.00998354880566111\\
599.3	0.00998396109700947\\
599.31	0.00998436938772849\\
599.32	0.00998477363562221\\
599.33	0.00998517379807499\\
599.34	0.00998556983204737\\
599.35	0.00998596169407188\\
599.36	0.00998634934024881\\
599.37	0.00998673272624192\\
599.38	0.00998711180727415\\
599.39	0.00998748653812326\\
599.4	0.00998785687311743\\
599.41	0.00998822276613081\\
599.42	0.00998858417057903\\
599.43	0.00998894103941468\\
599.44	0.00998929332512275\\
599.45	0.00998964097971596\\
599.46	0.00998998395473014\\
599.47	0.00999032220121951\\
599.48	0.0099906556697519\\
599.49	0.00999098431040396\\
599.5	0.00999130807275631\\
599.51	0.00999162690588863\\
599.52	0.00999194075837471\\
599.53	0.00999224957827746\\
599.54	0.00999255331314386\\
599.55	0.00999285190999987\\
599.56	0.00999314531534524\\
599.57	0.00999343347514839\\
599.58	0.00999371633484107\\
599.59	0.00999399383931314\\
599.6	0.00999426593290715\\
599.61	0.009994532559413\\
599.62	0.0099947936620624\\
599.63	0.00999504918352342\\
599.64	0.00999529906589492\\
599.65	0.00999554325070086\\
599.66	0.00999578167888471\\
599.67	0.00999601429080366\\
599.68	0.00999624102622283\\
599.69	0.00999646182430947\\
599.7	0.00999667662362697\\
599.71	0.00999688536212897\\
599.72	0.00999708797715332\\
599.73	0.00999728440541595\\
599.74	0.00999747458300482\\
599.75	0.0099976584453736\\
599.76	0.0099978359273355\\
599.77	0.00999800696305692\\
599.78	0.00999817148605102\\
599.79	0.00999832942917133\\
599.8	0.00999848072460517\\
599.81	0.00999862530386713\\
599.82	0.00999876309779239\\
599.83	0.00999889403653003\\
599.84	0.00999901804953622\\
599.85	0.00999913506556741\\
599.86	0.00999924501267341\\
599.87	0.00999934781819042\\
599.88	0.00999944340873394\\
599.89	0.0099995317101917\\
599.9	0.00999961264771648\\
599.91	0.00999968614571878\\
599.92	0.00999975212785958\\
599.93	0.00999981051704285\\
599.94	0.00999986123540817\\
599.95	0.00999990420432308\\
599.96	0.00999993934437554\\
599.97	0.00999996657536618\\
599.98	0.00999998581630055\\
599.99	0.00999999698538124\\
600	0.01\\
};
\addplot [color=red,solid,forget plot]
  table[row sep=crcr]{%
0.01	0.00446279362943701\\
1.01	0.00446279467437111\\
2.01	0.00446279574146974\\
3.01	0.00446279683120417\\
4.01	0.00446279794405589\\
5.01	0.00446279908051643\\
6.01	0.00446280024108764\\
7.01	0.00446280142628253\\
8.01	0.00446280263662477\\
9.01	0.00446280387264912\\
10.01	0.00446280513490203\\
11.01	0.00446280642394131\\
12.01	0.00446280774033689\\
13.01	0.00446280908467081\\
14.01	0.00446281045753742\\
15.01	0.00446281185954392\\
16.01	0.00446281329131025\\
17.01	0.00446281475346984\\
18.01	0.0044628162466694\\
19.01	0.00446281777156965\\
20.01	0.00446281932884526\\
21.01	0.00446282091918541\\
22.01	0.00446282254329389\\
23.01	0.00446282420188951\\
24.01	0.00446282589570661\\
25.01	0.00446282762549499\\
26.01	0.00446282939202075\\
27.01	0.00446283119606608\\
28.01	0.0044628330384299\\
29.01	0.00446283491992829\\
30.01	0.0044628368413948\\
31.01	0.00446283880368069\\
32.01	0.00446284080765535\\
33.01	0.00446284285420691\\
34.01	0.00446284494424239\\
35.01	0.0044628470786883\\
36.01	0.0044628492584908\\
37.01	0.00446285148461635\\
38.01	0.00446285375805211\\
39.01	0.00446285607980614\\
40.01	0.00446285845090834\\
41.01	0.00446286087241034\\
42.01	0.00446286334538661\\
43.01	0.00446286587093418\\
44.01	0.00446286845017398\\
45.01	0.00446287108425042\\
46.01	0.00446287377433295\\
47.01	0.00446287652161562\\
48.01	0.00446287932731806\\
49.01	0.00446288219268617\\
50.01	0.00446288511899234\\
51.01	0.00446288810753637\\
52.01	0.00446289115964578\\
53.01	0.00446289427667639\\
54.01	0.00446289746001305\\
55.01	0.0044629007110704\\
56.01	0.00446290403129351\\
57.01	0.00446290742215796\\
58.01	0.00446291088517126\\
59.01	0.00446291442187279\\
60.01	0.00446291803383553\\
61.01	0.00446292172266559\\
62.01	0.00446292549000384\\
63.01	0.00446292933752594\\
64.01	0.00446293326694371\\
65.01	0.00446293728000552\\
66.01	0.00446294137849734\\
67.01	0.00446294556424327\\
68.01	0.0044629498391065\\
69.01	0.00446295420499028\\
70.01	0.00446295866383831\\
71.01	0.00446296321763646\\
72.01	0.00446296786841259\\
73.01	0.00446297261823851\\
74.01	0.00446297746923029\\
75.01	0.00446298242354926\\
76.01	0.00446298748340318\\
77.01	0.00446299265104712\\
78.01	0.0044629979287845\\
79.01	0.00446300331896818\\
80.01	0.00446300882400139\\
81.01	0.00446301444633918\\
82.01	0.00446302018848894\\
83.01	0.00446302605301209\\
84.01	0.00446303204252478\\
85.01	0.00446303815969951\\
86.01	0.0044630444072659\\
87.01	0.00446305078801243\\
88.01	0.00446305730478727\\
89.01	0.0044630639604996\\
90.01	0.00446307075812129\\
91.01	0.00446307770068795\\
92.01	0.00446308479130011\\
93.01	0.00446309203312531\\
94.01	0.00446309942939881\\
95.01	0.00446310698342536\\
96.01	0.00446311469858085\\
97.01	0.00446312257831347\\
98.01	0.00446313062614562\\
99.01	0.0044631388456753\\
100.01	0.00446314724057793\\
101.01	0.0044631558146077\\
102.01	0.00446316457159972\\
103.01	0.00446317351547124\\
104.01	0.00446318265022398\\
105.01	0.00446319197994535\\
106.01	0.00446320150881091\\
107.01	0.00446321124108578\\
108.01	0.0044632211811271\\
109.01	0.00446323133338516\\
110.01	0.00446324170240645\\
111.01	0.00446325229283491\\
112.01	0.0044632631094144\\
113.01	0.00446327415699101\\
114.01	0.00446328544051491\\
115.01	0.00446329696504284\\
116.01	0.00446330873574025\\
117.01	0.00446332075788357\\
118.01	0.00446333303686298\\
119.01	0.00446334557818465\\
120.01	0.00446335838747333\\
121.01	0.00446337147047461\\
122.01	0.00446338483305785\\
123.01	0.00446339848121874\\
124.01	0.0044634124210821\\
125.01	0.00446342665890456\\
126.01	0.00446344120107753\\
127.01	0.00446345605412986\\
128.01	0.00446347122473133\\
129.01	0.004463486719695\\
130.01	0.00446350254598102\\
131.01	0.00446351871069903\\
132.01	0.00446353522111248\\
133.01	0.00446355208464071\\
134.01	0.00446356930886331\\
135.01	0.00446358690152292\\
136.01	0.00446360487052914\\
137.01	0.00446362322396197\\
138.01	0.0044636419700758\\
139.01	0.00446366111730254\\
140.01	0.00446368067425619\\
141.01	0.00446370064973658\\
142.01	0.00446372105273289\\
143.01	0.00446374189242815\\
144.01	0.00446376317820379\\
145.01	0.00446378491964319\\
146.01	0.00446380712653657\\
147.01	0.00446382980888542\\
148.01	0.00446385297690685\\
149.01	0.00446387664103824\\
150.01	0.00446390081194241\\
151.01	0.00446392550051194\\
152.01	0.00446395071787495\\
153.01	0.00446397647539928\\
154.01	0.00446400278469811\\
155.01	0.00446402965763548\\
156.01	0.00446405710633162\\
157.01	0.00446408514316785\\
158.01	0.00446411378079342\\
159.01	0.00446414303213014\\
160.01	0.00446417291037918\\
161.01	0.00446420342902669\\
162.01	0.00446423460185\\
163.01	0.00446426644292403\\
164.01	0.00446429896662766\\
165.01	0.00446433218765029\\
166.01	0.0044643661209989\\
167.01	0.00446440078200434\\
168.01	0.0044644361863289\\
169.01	0.00446447234997324\\
170.01	0.00446450928928349\\
171.01	0.00446454702095951\\
172.01	0.00446458556206192\\
173.01	0.0044646249300198\\
174.01	0.00446466514263921\\
175.01	0.0044647062181111\\
176.01	0.00446474817501948\\
177.01	0.00446479103235033\\
178.01	0.00446483480950005\\
179.01	0.00446487952628447\\
180.01	0.00446492520294781\\
181.01	0.00446497186017238\\
182.01	0.00446501951908744\\
183.01	0.00446506820127945\\
184.01	0.00446511792880173\\
185.01	0.00446516872418462\\
186.01	0.00446522061044594\\
187.01	0.00446527361110154\\
188.01	0.00446532775017572\\
189.01	0.00446538305221306\\
190.01	0.00446543954228924\\
191.01	0.00446549724602228\\
192.01	0.00446555618958482\\
193.01	0.00446561639971589\\
194.01	0.00446567790373335\\
195.01	0.00446574072954646\\
196.01	0.00446580490566843\\
197.01	0.00446587046123002\\
198.01	0.00446593742599249\\
199.01	0.00446600583036165\\
200.01	0.00446607570540163\\
201.01	0.00446614708284929\\
202.01	0.00446621999512875\\
203.01	0.00446629447536617\\
204.01	0.00446637055740541\\
205.01	0.00446644827582336\\
206.01	0.004466527665946\\
207.01	0.00446660876386459\\
208.01	0.00446669160645191\\
209.01	0.00446677623137995\\
210.01	0.00446686267713675\\
211.01	0.00446695098304456\\
212.01	0.00446704118927747\\
213.01	0.00446713333688023\\
214.01	0.00446722746778715\\
215.01	0.00446732362484133\\
216.01	0.00446742185181453\\
217.01	0.00446752219342743\\
218.01	0.00446762469536992\\
219.01	0.00446772940432243\\
220.01	0.00446783636797721\\
221.01	0.00446794563506084\\
222.01	0.00446805725535614\\
223.01	0.00446817127972544\\
224.01	0.00446828776013407\\
225.01	0.00446840674967409\\
226.01	0.00446852830258902\\
227.01	0.00446865247429892\\
228.01	0.0044687793214257\\
229.01	0.00446890890181946\\
230.01	0.00446904127458507\\
231.01	0.00446917650010961\\
232.01	0.00446931464009022\\
233.01	0.00446945575756227\\
234.01	0.00446959991692916\\
235.01	0.00446974718399143\\
236.01	0.00446989762597724\\
237.01	0.0044700513115738\\
238.01	0.00447020831095871\\
239.01	0.00447036869583235\\
240.01	0.00447053253945165\\
241.01	0.0044706999166631\\
242.01	0.00447087090393777\\
243.01	0.0044710455794067\\
244.01	0.00447122402289703\\
245.01	0.00447140631596839\\
246.01	0.00447159254195124\\
247.01	0.00447178278598523\\
248.01	0.0044719771350583\\
249.01	0.00447217567804721\\
250.01	0.00447237850575861\\
251.01	0.00447258571097083\\
252.01	0.00447279738847675\\
253.01	0.004473013635128\\
254.01	0.00447323454987954\\
255.01	0.00447346023383552\\
256.01	0.00447369079029594\\
257.01	0.00447392632480438\\
258.01	0.00447416694519714\\
259.01	0.00447441276165293\\
260.01	0.00447466388674389\\
261.01	0.00447492043548736\\
262.01	0.0044751825253998\\
263.01	0.00447545027655073\\
264.01	0.00447572381161841\\
265.01	0.00447600325594621\\
266.01	0.00447628873760142\\
267.01	0.00447658038743401\\
268.01	0.00447687833913714\\
269.01	0.00447718272930933\\
270.01	0.0044774936975174\\
271.01	0.00447781138636115\\
272.01	0.00447813594153966\\
273.01	0.00447846751191823\\
274.01	0.00447880624959775\\
275.01	0.00447915230998462\\
276.01	0.00447950585186276\\
277.01	0.00447986703746754\\
278.01	0.00448023603256008\\
279.01	0.00448061300650454\\
280.01	0.00448099813234611\\
281.01	0.00448139158689079\\
282.01	0.00448179355078797\\
283.01	0.00448220420861294\\
284.01	0.00448262374895256\\
285.01	0.00448305236449262\\
286.01	0.00448349025210623\\
287.01	0.00448393761294513\\
288.01	0.00448439465253256\\
289.01	0.00448486158085724\\
290.01	0.00448533861247102\\
291.01	0.00448582596658741\\
292.01	0.00448632386718223\\
293.01	0.00448683254309704\\
294.01	0.00448735222814395\\
295.01	0.0044878831612131\\
296.01	0.00448842558638252\\
297.01	0.00448897975302982\\
298.01	0.00448954591594614\\
299.01	0.00449012433545301\\
300.01	0.00449071527752133\\
301.01	0.00449131901389231\\
302.01	0.00449193582220168\\
303.01	0.00449256598610571\\
304.01	0.00449320979540965\\
305.01	0.00449386754619996\\
306.01	0.00449453954097724\\
307.01	0.00449522608879312\\
308.01	0.00449592750538984\\
309.01	0.00449664411334082\\
310.01	0.00449737624219611\\
311.01	0.00449812422862897\\
312.01	0.00449888841658568\\
313.01	0.00449966915743803\\
314.01	0.00450046681013858\\
315.01	0.00450128174137823\\
316.01	0.00450211432574686\\
317.01	0.00450296494589604\\
318.01	0.00450383399270537\\
319.01	0.00450472186545071\\
320.01	0.00450562897197439\\
321.01	0.00450655572885922\\
322.01	0.00450750256160418\\
323.01	0.00450846990480219\\
324.01	0.0045094582023204\\
325.01	0.00451046790748309\\
326.01	0.00451149948325541\\
327.01	0.00451255340242973\\
328.01	0.00451363014781372\\
329.01	0.00451473021241956\\
330.01	0.0045158540996538\\
331.01	0.00451700232350923\\
332.01	0.00451817540875693\\
333.01	0.00451937389113802\\
334.01	0.0045205983175562\\
335.01	0.00452184924626919\\
336.01	0.00452312724707914\\
337.01	0.00452443290152228\\
338.01	0.00452576680305498\\
339.01	0.00452712955723814\\
340.01	0.00452852178191794\\
341.01	0.00452994410740114\\
342.01	0.0045313971766261\\
343.01	0.00453288164532724\\
344.01	0.00453439818219144\\
345.01	0.00453594746900562\\
346.01	0.00453753020079468\\
347.01	0.00453914708594719\\
348.01	0.00454079884632719\\
349.01	0.00454248621737152\\
350.01	0.00454420994816808\\
351.01	0.00454597080151523\\
352.01	0.0045477695539579\\
353.01	0.00454960699579852\\
354.01	0.00455148393107861\\
355.01	0.0045534011775293\\
356.01	0.00455535956648434\\
357.01	0.00455735994275371\\
358.01	0.00455940316445122\\
359.01	0.00456149010277289\\
360.01	0.00456362164171761\\
361.01	0.00456579867774727\\
362.01	0.00456802211937603\\
363.01	0.00457029288668391\\
364.01	0.00457261191074437\\
365.01	0.00457498013295868\\
366.01	0.00457739850428496\\
367.01	0.00457986798435359\\
368.01	0.00458238954045488\\
369.01	0.0045849641463882\\
370.01	0.00458759278115758\\
371.01	0.0045902764274988\\
372.01	0.00459301607022335\\
373.01	0.00459581269436037\\
374.01	0.00459866728307932\\
375.01	0.00460158081537414\\
376.01	0.00460455426348784\\
377.01	0.00460758859005705\\
378.01	0.00461068474495481\\
379.01	0.00461384366180809\\
380.01	0.00461706625417126\\
381.01	0.00462035341132955\\
382.01	0.00462370599371752\\
383.01	0.00462712482793086\\
384.01	0.00463061070132095\\
385.01	0.00463416435616079\\
386.01	0.00463778648338296\\
387.01	0.00464147771589749\\
388.01	0.00464523862151071\\
389.01	0.00464906969548462\\
390.01	0.00465297135279585\\
391.01	0.0046569439201833\\
392.01	0.00466098762810693\\
393.01	0.00466510260278355\\
394.01	0.0046692888585225\\
395.01	0.0046735462906439\\
396.01	0.00467787466935003\\
397.01	0.00468227363501243\\
398.01	0.00468674269545274\\
399.01	0.00469128122593594\\
400.01	0.00469588847274542\\
401.01	0.00470056356139702\\
402.01	0.0047053055107453\\
403.01	0.00471011325445241\\
404.01	0.00471498567151782\\
405.01	0.00471992162777217\\
406.01	0.00472492003042431\\
407.01	0.004729979897832\\
408.01	0.00473510044661705\\
409.01	0.00474028119792783\\
410.01	0.00474552210393546\\
411.01	0.00475082369429569\\
412.01	0.00475618724000343\\
413.01	0.00476161492831731\\
414.01	0.00476711003659286\\
415.01	0.0047726770839408\\
416.01	0.00477832192627343\\
417.01	0.0047840517405757\\
418.01	0.00478987481540984\\
419.01	0.00479580002291086\\
420.01	0.00480183477499227\\
421.01	0.00480798152282351\\
422.01	0.00481424155114779\\
423.01	0.00482061609714657\\
424.01	0.00482710634457248\\
425.01	0.004833713417513\\
426.01	0.00484043837377887\\
427.01	0.00484728219791494\\
428.01	0.00485424579383584\\
429.01	0.00486132997709308\\
430.01	0.00486853546678707\\
431.01	0.00487586287714664\\
432.01	0.00488331270880467\\
433.01	0.00489088533981411\\
434.01	0.00489858101645825\\
435.01	0.00490639984392569\\
436.01	0.00491434177693975\\
437.01	0.00492240661045185\\
438.01	0.00493059397053315\\
439.01	0.00493890330562723\\
440.01	0.00494733387835797\\
441.01	0.00495588475812132\\
442.01	0.00496455481473447\\
443.01	0.00497334271345355\\
444.01	0.00498224691172661\\
445.01	0.00499126565809582\\
446.01	0.00500039699372201\\
447.01	0.00500963875705959\\
448.01	0.00501898859226985\\
449.01	0.00502844396201267\\
450.01	0.00503800216530673\\
451.01	0.00504766036118512\\
452.01	0.0050574155988901\\
453.01	0.00506726485534463\\
454.01	0.00507720508058666\\
455.01	0.00508723325175496\\
456.01	0.00509734643604093\\
457.01	0.00510754186275515\\
458.01	0.00511781700427356\\
459.01	0.00512816966509334\\
460.01	0.00513859807751232\\
461.01	0.00514910100151217\\
462.01	0.00515967782524113\\
463.01	0.00517032866103863\\
464.01	0.0051810544302055\\
465.01	0.00519185692774328\\
466.01	0.00520273885614411\\
467.01	0.00521370381520816\\
468.01	0.0052247562331483\\
469.01	0.00523590122351676\\
470.01	0.00524714435373932\\
471.01	0.00525849131582871\\
472.01	0.0052699475006005\\
473.01	0.00528151749711198\\
474.01	0.00529320457465202\\
475.01	0.005305010264441\\
476.01	0.0053169344596347\\
477.01	0.00532897666318638\\
478.01	0.00534113664366383\\
479.01	0.0053534145068606\\
480.01	0.00536581074972134\\
481.01	0.00537832631442095\\
482.01	0.00539096264107522\\
483.01	0.00540372171722552\\
484.01	0.0054166061218886\\
485.01	0.00542961906163633\\
486.01	0.00544276439587896\\
487.01	0.00545604664833761\\
488.01	0.00546947100165257\\
489.01	0.00548304327227303\\
490.01	0.0054967698633107\\
491.01	0.00551065769401794\\
492.01	0.00552471410611784\\
493.01	0.00553894674947311\\
494.01	0.00555336345262983\\
495.01	0.00556797208761162\\
496.01	0.00558278044280461\\
497.01	0.00559779612238419\\
498.01	0.00561302649450048\\
499.01	0.00562847871143375\\
500.01	0.00564415981983526\\
501.01	0.00566007696195645\\
502.01	0.00567623760351374\\
503.01	0.00569264962297497\\
504.01	0.00570932129199433\\
505.01	0.00572626123942883\\
506.01	0.00574347840841211\\
507.01	0.00576098200787976\\
508.01	0.00577878146059106\\
509.01	0.00579688635037232\\
510.01	0.00581530637193427\\
511.01	0.00583405128707744\\
512.01	0.00585313089125472\\
513.01	0.00587255499412151\\
514.01	0.00589233341665268\\
515.01	0.00591247600545589\\
516.01	0.00593299266193996\\
517.01	0.005953893380126\\
518.01	0.00597518828272067\\
519.01	0.00599688764218327\\
520.01	0.00601900187762823\\
521.01	0.00604154153722314\\
522.01	0.00606451728071976\\
523.01	0.00608793986572709\\
524.01	0.00611182013853306\\
525.01	0.00613616902991156\\
526.01	0.0061609975559043\\
527.01	0.00618631682301966\\
528.01	0.00621213803670315\\
529.01	0.00623847251140029\\
530.01	0.00626533168021377\\
531.01	0.00629272710222778\\
532.01	0.00632067046622661\\
533.01	0.00634917359086493\\
534.01	0.00637824842303027\\
535.01	0.00640790703661753\\
536.01	0.00643816163244836\\
537.01	0.0064690245390617\\
538.01	0.00650050821389488\\
539.01	0.00653262524430583\\
540.01	0.00656538834788008\\
541.01	0.00659881037154011\\
542.01	0.0066329042891283\\
543.01	0.00666768319734616\\
544.01	0.00670316031014888\\
545.01	0.00673934895181159\\
546.01	0.00677626254881891\\
547.01	0.00681391462049223\\
548.01	0.00685231876808774\\
549.01	0.0068914886620728\\
550.01	0.00693143802730413\\
551.01	0.0069721806258649\\
552.01	0.00701373023735284\\
553.01	0.00705610063643022\\
554.01	0.0070993055674506\\
555.01	0.0071433587159477\\
556.01	0.00718827367671961\\
557.01	0.0072340639181856\\
558.01	0.00728074274265835\\
559.01	0.00732832324215779\\
560.01	0.00737681824938\\
561.01	0.00742624028341765\\
562.01	0.0074766014898075\\
563.01	0.00752791357445203\\
564.01	0.00758018773093511\\
565.01	0.0076334345607198\\
566.01	0.00768766398569602\\
567.01	0.00774288515253303\\
568.01	0.0077991063282865\\
569.01	0.00785633478671762\\
570.01	0.00791457668480008\\
571.01	0.00797383692892611\\
572.01	0.0080341190303829\\
573.01	0.00809542494975993\\
574.01	0.00815775493008039\\
575.01	0.00822110731863136\\
576.01	0.00828547837771707\\
577.01	0.00835086208488767\\
578.01	0.00841724992363038\\
579.01	0.00848463066607123\\
580.01	0.00855299014995957\\
581.01	0.0086223110531343\\
582.01	0.0086925726698485\\
583.01	0.00876375069482038\\
584.01	0.00883581702275711\\
585.01	0.00890873957345649\\
586.01	0.00898248215554676\\
587.01	0.00905700438561268\\
588.01	0.0091322616840545\\
589.01	0.00920820537474489\\
590.01	0.00928478292265227\\
591.01	0.00936193835240739\\
592.01	0.00943961290170265\\
593.01	0.00951774597691674\\
594.01	0.00959627649505263\\
595.01	0.00967514471670249\\
596.01	0.00975429470021483\\
597.01	0.00983351919724493\\
598.01	0.00990826330299362\\
599.01	0.00997053306357289\\
599.02	0.00997104257304143\\
599.03	0.00997154904691472\\
599.04	0.00997205245565375\\
599.05	0.00997255276942872\\
599.06	0.00997304995811617\\
599.07	0.00997354399129606\\
599.08	0.00997403483824883\\
599.09	0.00997452246795248\\
599.1	0.00997500684907952\\
599.11	0.00997548794999398\\
599.12	0.00997596573874838\\
599.13	0.0099764401830806\\
599.14	0.0099769112504108\\
599.15	0.00997737890783826\\
599.16	0.00997784312213823\\
599.17	0.0099783038597587\\
599.18	0.00997876108681718\\
599.19	0.00997921476909742\\
599.2	0.00997966487204611\\
599.21	0.00998011136076955\\
599.22	0.00998055420003028\\
599.23	0.0099809933542437\\
599.24	0.00998142878747458\\
599.25	0.00998186046343367\\
599.26	0.00998228834359041\\
599.27	0.00998271238778643\\
599.28	0.00998313255546381\\
599.29	0.00998354880566111\\
599.3	0.00998396109700947\\
599.31	0.00998436938772849\\
599.32	0.00998477363562221\\
599.33	0.00998517379807499\\
599.34	0.00998556983204737\\
599.35	0.00998596169407188\\
599.36	0.00998634934024881\\
599.37	0.00998673272624192\\
599.38	0.00998711180727415\\
599.39	0.00998748653812326\\
599.4	0.00998785687311743\\
599.41	0.00998822276613081\\
599.42	0.00998858417057903\\
599.43	0.00998894103941468\\
599.44	0.00998929332512275\\
599.45	0.00998964097971596\\
599.46	0.00998998395473014\\
599.47	0.00999032220121951\\
599.48	0.0099906556697519\\
599.49	0.00999098431040396\\
599.5	0.00999130807275631\\
599.51	0.00999162690588863\\
599.52	0.00999194075837471\\
599.53	0.00999224957827746\\
599.54	0.00999255331314387\\
599.55	0.00999285190999987\\
599.56	0.00999314531534524\\
599.57	0.00999343347514839\\
599.58	0.00999371633484107\\
599.59	0.00999399383931314\\
599.6	0.00999426593290715\\
599.61	0.009994532559413\\
599.62	0.0099947936620624\\
599.63	0.00999504918352342\\
599.64	0.00999529906589491\\
599.65	0.00999554325070086\\
599.66	0.00999578167888471\\
599.67	0.00999601429080366\\
599.68	0.00999624102622283\\
599.69	0.00999646182430947\\
599.7	0.00999667662362697\\
599.71	0.00999688536212897\\
599.72	0.00999708797715331\\
599.73	0.00999728440541595\\
599.74	0.00999747458300482\\
599.75	0.0099976584453736\\
599.76	0.00999783592733551\\
599.77	0.00999800696305692\\
599.78	0.00999817148605102\\
599.79	0.00999832942917133\\
599.8	0.00999848072460517\\
599.81	0.00999862530386713\\
599.82	0.00999876309779239\\
599.83	0.00999889403653003\\
599.84	0.00999901804953622\\
599.85	0.00999913506556741\\
599.86	0.00999924501267341\\
599.87	0.00999934781819042\\
599.88	0.00999944340873394\\
599.89	0.0099995317101917\\
599.9	0.00999961264771648\\
599.91	0.00999968614571878\\
599.92	0.00999975212785957\\
599.93	0.00999981051704285\\
599.94	0.00999986123540817\\
599.95	0.00999990420432308\\
599.96	0.00999993934437554\\
599.97	0.00999996657536618\\
599.98	0.00999998581630055\\
599.99	0.00999999698538124\\
600	0.01\\
};
\addplot [color=mycolor20,solid,forget plot]
  table[row sep=crcr]{%
0.01	0.00464973790653378\\
1.01	0.00464973895170358\\
2.01	0.00464974001894404\\
3.01	0.00464974110872185\\
4.01	0.00464974222151349\\
5.01	0.00464974335780581\\
6.01	0.00464974451809574\\
7.01	0.00464974570289085\\
8.01	0.00464974691270935\\
9.01	0.00464974814808068\\
10.01	0.00464974940954528\\
11.01	0.00464975069765509\\
12.01	0.00464975201297374\\
13.01	0.00464975335607683\\
14.01	0.00464975472755216\\
15.01	0.00464975612799989\\
16.01	0.00464975755803317\\
17.01	0.00464975901827772\\
18.01	0.00464976050937299\\
19.01	0.00464976203197151\\
20.01	0.00464976358674003\\
21.01	0.00464976517435919\\
22.01	0.00464976679552427\\
23.01	0.0046497684509452\\
24.01	0.00464977014134682\\
25.01	0.0046497718674695\\
26.01	0.00464977363006917\\
27.01	0.00464977542991794\\
28.01	0.00464977726780437\\
29.01	0.00464977914453342\\
30.01	0.00464978106092738\\
31.01	0.00464978301782595\\
32.01	0.00464978501608659\\
33.01	0.00464978705658488\\
34.01	0.00464978914021499\\
35.01	0.00464979126789013\\
36.01	0.0046497934405428\\
37.01	0.00464979565912546\\
38.01	0.00464979792461059\\
39.01	0.00464980023799141\\
40.01	0.00464980260028218\\
41.01	0.00464980501251853\\
42.01	0.00464980747575822\\
43.01	0.0046498099910815\\
44.01	0.00464981255959152\\
45.01	0.00464981518241461\\
46.01	0.00464981786070119\\
47.01	0.00464982059562617\\
48.01	0.00464982338838932\\
49.01	0.00464982624021579\\
50.01	0.00464982915235676\\
51.01	0.00464983212609014\\
52.01	0.00464983516272083\\
53.01	0.00464983826358134\\
54.01	0.00464984143003266\\
55.01	0.00464984466346453\\
56.01	0.00464984796529611\\
57.01	0.004649851336977\\
58.01	0.00464985477998742\\
59.01	0.00464985829583896\\
60.01	0.00464986188607529\\
61.01	0.00464986555227288\\
62.01	0.00464986929604176\\
63.01	0.00464987311902611\\
64.01	0.00464987702290505\\
65.01	0.00464988100939338\\
66.01	0.00464988508024218\\
67.01	0.00464988923723986\\
68.01	0.00464989348221277\\
69.01	0.00464989781702611\\
70.01	0.00464990224358472\\
71.01	0.00464990676383367\\
72.01	0.00464991137975977\\
73.01	0.00464991609339165\\
74.01	0.00464992090680129\\
75.01	0.00464992582210448\\
76.01	0.00464993084146224\\
77.01	0.0046499359670813\\
78.01	0.00464994120121542\\
79.01	0.00464994654616613\\
80.01	0.00464995200428411\\
81.01	0.00464995757796963\\
82.01	0.00464996326967425\\
83.01	0.00464996908190156\\
84.01	0.0046499750172083\\
85.01	0.00464998107820581\\
86.01	0.00464998726756059\\
87.01	0.00464999358799618\\
88.01	0.0046500000422937\\
89.01	0.00465000663329372\\
90.01	0.00465001336389692\\
91.01	0.00465002023706589\\
92.01	0.00465002725582619\\
93.01	0.0046500344232674\\
94.01	0.00465004174254526\\
95.01	0.00465004921688221\\
96.01	0.00465005684956957\\
97.01	0.00465006464396846\\
98.01	0.00465007260351157\\
99.01	0.00465008073170467\\
100.01	0.00465008903212784\\
101.01	0.0046500975084376\\
102.01	0.00465010616436813\\
103.01	0.00465011500373314\\
104.01	0.00465012403042728\\
105.01	0.00465013324842835\\
106.01	0.00465014266179846\\
107.01	0.0046501522746864\\
108.01	0.00465016209132914\\
109.01	0.00465017211605392\\
110.01	0.00465018235327977\\
111.01	0.00465019280752009\\
112.01	0.00465020348338416\\
113.01	0.00465021438557953\\
114.01	0.00465022551891367\\
115.01	0.00465023688829648\\
116.01	0.00465024849874256\\
117.01	0.004650260355373\\
118.01	0.00465027246341811\\
119.01	0.00465028482821934\\
120.01	0.00465029745523195\\
121.01	0.00465031035002725\\
122.01	0.00465032351829534\\
123.01	0.00465033696584731\\
124.01	0.00465035069861795\\
125.01	0.00465036472266848\\
126.01	0.00465037904418932\\
127.01	0.00465039366950243\\
128.01	0.00465040860506462\\
129.01	0.00465042385747007\\
130.01	0.00465043943345331\\
131.01	0.00465045533989245\\
132.01	0.00465047158381193\\
133.01	0.004650488172386\\
134.01	0.00465050511294131\\
135.01	0.0046505224129611\\
136.01	0.00465054008008767\\
137.01	0.00465055812212613\\
138.01	0.00465057654704773\\
139.01	0.00465059536299378\\
140.01	0.00465061457827878\\
141.01	0.00465063420139443\\
142.01	0.00465065424101323\\
143.01	0.00465067470599261\\
144.01	0.00465069560537837\\
145.01	0.00465071694840936\\
146.01	0.00465073874452133\\
147.01	0.00465076100335039\\
148.01	0.00465078373473866\\
149.01	0.00465080694873762\\
150.01	0.00465083065561295\\
151.01	0.00465085486584914\\
152.01	0.00465087959015376\\
153.01	0.00465090483946284\\
154.01	0.00465093062494543\\
155.01	0.00465095695800827\\
156.01	0.00465098385030127\\
157.01	0.00465101131372267\\
158.01	0.00465103936042409\\
159.01	0.00465106800281628\\
160.01	0.00465109725357415\\
161.01	0.00465112712564295\\
162.01	0.00465115763224354\\
163.01	0.00465118878687854\\
164.01	0.00465122060333849\\
165.01	0.00465125309570741\\
166.01	0.00465128627836979\\
167.01	0.00465132016601638\\
168.01	0.00465135477365089\\
169.01	0.00465139011659707\\
170.01	0.00465142621050516\\
171.01	0.00465146307135836\\
172.01	0.0046515007154808\\
173.01	0.00465153915954444\\
174.01	0.00465157842057629\\
175.01	0.00465161851596646\\
176.01	0.00465165946347512\\
177.01	0.0046517012812413\\
178.01	0.00465174398779027\\
179.01	0.00465178760204174\\
180.01	0.00465183214331881\\
181.01	0.00465187763135624\\
182.01	0.00465192408630904\\
183.01	0.00465197152876162\\
184.01	0.00465201997973698\\
185.01	0.00465206946070575\\
186.01	0.00465211999359595\\
187.01	0.00465217160080285\\
188.01	0.00465222430519863\\
189.01	0.00465227813014253\\
190.01	0.00465233309949122\\
191.01	0.00465238923760947\\
192.01	0.00465244656938106\\
193.01	0.00465250512021954\\
194.01	0.00465256491607949\\
195.01	0.00465262598346819\\
196.01	0.00465268834945734\\
197.01	0.00465275204169491\\
198.01	0.00465281708841761\\
199.01	0.00465288351846294\\
200.01	0.00465295136128226\\
201.01	0.00465302064695397\\
202.01	0.00465309140619645\\
203.01	0.00465316367038184\\
204.01	0.00465323747154976\\
205.01	0.00465331284242161\\
206.01	0.00465338981641497\\
207.01	0.00465346842765835\\
208.01	0.00465354871100597\\
209.01	0.00465363070205359\\
210.01	0.00465371443715411\\
211.01	0.0046537999534331\\
212.01	0.00465388728880586\\
213.01	0.0046539764819936\\
214.01	0.00465406757254096\\
215.01	0.0046541606008326\\
216.01	0.00465425560811185\\
217.01	0.00465435263649835\\
218.01	0.00465445172900655\\
219.01	0.00465455292956495\\
220.01	0.00465465628303503\\
221.01	0.00465476183523107\\
222.01	0.00465486963294019\\
223.01	0.00465497972394283\\
224.01	0.00465509215703389\\
225.01	0.00465520698204382\\
226.01	0.00465532424986058\\
227.01	0.00465544401245184\\
228.01	0.00465556632288749\\
229.01	0.00465569123536354\\
230.01	0.0046558188052247\\
231.01	0.00465594908898936\\
232.01	0.00465608214437382\\
233.01	0.00465621803031775\\
234.01	0.00465635680700931\\
235.01	0.00465649853591181\\
236.01	0.00465664327979021\\
237.01	0.00465679110273851\\
238.01	0.00465694207020716\\
239.01	0.00465709624903191\\
240.01	0.00465725370746248\\
241.01	0.00465741451519225\\
242.01	0.00465757874338817\\
243.01	0.00465774646472161\\
244.01	0.00465791775339952\\
245.01	0.00465809268519705\\
246.01	0.00465827133748926\\
247.01	0.00465845378928482\\
248.01	0.00465864012126004\\
249.01	0.0046588304157934\\
250.01	0.0046590247570006\\
251.01	0.00465922323077068\\
252.01	0.00465942592480326\\
253.01	0.00465963292864495\\
254.01	0.00465984433372846\\
255.01	0.00466006023341049\\
256.01	0.00466028072301221\\
257.01	0.0046605058998596\\
258.01	0.00466073586332434\\
259.01	0.0046609707148659\\
260.01	0.0046612105580745\\
261.01	0.00466145549871461\\
262.01	0.00466170564476932\\
263.01	0.00466196110648606\\
264.01	0.00466222199642249\\
265.01	0.00466248842949307\\
266.01	0.00466276052301791\\
267.01	0.00466303839677097\\
268.01	0.00466332217302982\\
269.01	0.00466361197662639\\
270.01	0.00466390793499857\\
271.01	0.00466421017824244\\
272.01	0.00466451883916598\\
273.01	0.00466483405334356\\
274.01	0.00466515595917055\\
275.01	0.00466548469792066\\
276.01	0.00466582041380288\\
277.01	0.00466616325401943\\
278.01	0.00466651336882565\\
279.01	0.00466687091158999\\
280.01	0.00466723603885535\\
281.01	0.00466760891040184\\
282.01	0.00466798968930963\\
283.01	0.00466837854202384\\
284.01	0.00466877563841961\\
285.01	0.00466918115186886\\
286.01	0.0046695952593077\\
287.01	0.00467001814130505\\
288.01	0.00467044998213202\\
289.01	0.00467089096983278\\
290.01	0.00467134129629609\\
291.01	0.00467180115732775\\
292.01	0.00467227075272444\\
293.01	0.00467275028634806\\
294.01	0.00467323996620149\\
295.01	0.00467374000450517\\
296.01	0.00467425061777418\\
297.01	0.00467477202689687\\
298.01	0.00467530445721384\\
299.01	0.00467584813859828\\
300.01	0.00467640330553634\\
301.01	0.00467697019720896\\
302.01	0.00467754905757434\\
303.01	0.00467814013545036\\
304.01	0.00467874368459884\\
305.01	0.00467935996380875\\
306.01	0.00467998923698161\\
307.01	0.00468063177321582\\
308.01	0.00468128784689199\\
309.01	0.00468195773775858\\
310.01	0.00468264173101725\\
311.01	0.00468334011740852\\
312.01	0.00468405319329711\\
313.01	0.00468478126075703\\
314.01	0.0046855246276564\\
315.01	0.00468628360774119\\
316.01	0.0046870585207195\\
317.01	0.00468784969234311\\
318.01	0.00468865745448946\\
319.01	0.00468948214524142\\
320.01	0.00469032410896557\\
321.01	0.00469118369638891\\
322.01	0.00469206126467328\\
323.01	0.00469295717748722\\
324.01	0.00469387180507526\\
325.01	0.00469480552432427\\
326.01	0.00469575871882572\\
327.01	0.00469673177893462\\
328.01	0.00469772510182394\\
329.01	0.00469873909153376\\
330.01	0.00469977415901608\\
331.01	0.00470083072217305\\
332.01	0.00470190920588842\\
333.01	0.0047030100420528\\
334.01	0.0047041336695806\\
335.01	0.00470528053441804\\
336.01	0.00470645108954323\\
337.01	0.00470764579495417\\
338.01	0.00470886511764762\\
339.01	0.00471010953158497\\
340.01	0.00471137951764473\\
341.01	0.0047126755635621\\
342.01	0.00471399816385247\\
343.01	0.00471534781971848\\
344.01	0.00471672503893963\\
345.01	0.00471813033574209\\
346.01	0.00471956423064854\\
347.01	0.00472102725030538\\
348.01	0.00472251992728619\\
349.01	0.00472404279986911\\
350.01	0.00472559641178789\\
351.01	0.004727181311952\\
352.01	0.00472879805413681\\
353.01	0.00473044719663934\\
354.01	0.0047321293018996\\
355.01	0.00473384493608242\\
356.01	0.0047355946686205\\
357.01	0.00473737907171459\\
358.01	0.00473919871978871\\
359.01	0.00474105418889803\\
360.01	0.00474294605608805\\
361.01	0.00474487489870062\\
362.01	0.00474684129362674\\
363.01	0.00474884581650259\\
364.01	0.00475088904084812\\
365.01	0.00475297153714498\\
366.01	0.00475509387185392\\
367.01	0.00475725660637023\\
368.01	0.0047594602959166\\
369.01	0.00476170548837417\\
370.01	0.00476399272305296\\
371.01	0.0047663225294034\\
372.01	0.00476869542567324\\
373.01	0.00477111191751402\\
374.01	0.00477357249654356\\
375.01	0.00477607763887379\\
376.01	0.0047786278036154\\
377.01	0.00478122343137151\\
378.01	0.00478386494273864\\
379.01	0.00478655273683741\\
380.01	0.00478928718989483\\
381.01	0.00479206865391218\\
382.01	0.00479489745545153\\
383.01	0.00479777389458493\\
384.01	0.00480069824405516\\
385.01	0.00480367074870561\\
386.01	0.00480669162524516\\
387.01	0.00480976106242338\\
388.01	0.00481287922170244\\
389.01	0.00481604623852108\\
390.01	0.00481926222425845\\
391.01	0.00482252726901594\\
392.01	0.00482584144534431\\
393.01	0.00482920481305503\\
394.01	0.00483261742525727\\
395.01	0.0048360793357707\\
396.01	0.00483959060805844\\
397.01	0.004843151325815\\
398.01	0.00484676160532981\\
399.01	0.00485042160970917\\
400.01	0.00485413156499516\\
401.01	0.00485789177814569\\
402.01	0.00486170265674173\\
403.01	0.00486556473015579\\
404.01	0.00486947867174012\\
405.01	0.00487344532137304\\
406.01	0.0048774657074195\\
407.01	0.00488154106683368\\
408.01	0.00488567286172653\\
409.01	0.00488986279028111\\
410.01	0.00489411278941566\\
411.01	0.00489842502613223\\
412.01	0.0049028018741073\\
413.01	0.00490724587192772\\
414.01	0.00491175965964245\\
415.01	0.00491634589133311\\
416.01	0.00492100712370258\\
417.01	0.00492574568501461\\
418.01	0.00493056353624796\\
419.01	0.00493546214874704\\
420.01	0.00494044246212998\\
421.01	0.00494550512435542\\
422.01	0.00495065072569029\\
423.01	0.00495587981638243\\
424.01	0.00496119290438493\\
425.01	0.00496659045314297\\
426.01	0.0049720728794735\\
427.01	0.00497764055157077\\
428.01	0.00498329378717578\\
429.01	0.00498903285195244\\
430.01	0.0049948579581198\\
431.01	0.00500076926339145\\
432.01	0.00500676687028612\\
433.01	0.00501285082587105\\
434.01	0.00501902112201255\\
435.01	0.00502527769621157\\
436.01	0.00503162043310858\\
437.01	0.00503804916674847\\
438.01	0.00504456368370031\\
439.01	0.00505116372713159\\
440.01	0.00505784900193995\\
441.01	0.00506461918104596\\
442.01	0.00507147391294728\\
443.01	0.00507841283063389\\
444.01	0.00508543556195244\\
445.01	0.00509254174149429\\
446.01	0.00509973102406414\\
447.01	0.00510700309975812\\
448.01	0.00511435771064417\\
449.01	0.00512179466899504\\
450.01	0.00512931387696365\\
451.01	0.00513691534752551\\
452.01	0.00514459922642688\\
453.01	0.00515236581477999\\
454.01	0.00516021559183744\\
455.01	0.00516814923734875\\
456.01	0.00517616765276563\\
457.01	0.00518427198042107\\
458.01	0.00519246361966003\\
459.01	0.00520074423876754\\
460.01	0.00520911578143264\\
461.01	0.00521758046642593\\
462.01	0.00522614077917966\\
463.01	0.00523479945408369\\
464.01	0.00524355944658267\\
465.01	0.0052524238946363\\
466.01	0.00526139606983764\\
467.01	0.00527047931952553\\
468.01	0.0052796770026217\\
469.01	0.00528899242367249\\
470.01	0.00529842877161917\\
471.01	0.00530798907198014\\
472.01	0.00531767616299825\\
473.01	0.00532749270713549\\
474.01	0.00533744124778705\\
475.01	0.00534752431500228\\
476.01	0.0053577445655274\\
477.01	0.00536810488489441\\
478.01	0.00537860841506019\\
479.01	0.00538925856432541\\
480.01	0.00540005901450961\\
481.01	0.00541101372510494\\
482.01	0.00542212693396527\\
483.01	0.00543340315415176\\
484.01	0.00544484716664075\\
485.01	0.00545646400873522\\
486.01	0.00546825895820363\\
487.01	0.00548023751340282\\
488.01	0.00549240536992349\\
489.01	0.00550476839463081\\
490.01	0.00551733259832464\\
491.01	0.00553010410861765\\
492.01	0.00554308914494975\\
493.01	0.00555629399789594\\
494.01	0.00556972501498051\\
495.01	0.00558338859499675\\
496.01	0.00559729119222796\\
497.01	0.00561143933087332\\
498.01	0.00562583962832382\\
499.01	0.00564049882376123\\
500.01	0.00565542380614559\\
501.01	0.00567062163372628\\
502.01	0.00568609953794767\\
503.01	0.00570186491384957\\
504.01	0.00571792530677094\\
505.01	0.0057342883991483\\
506.01	0.00575096199816087\\
507.01	0.00576795402485959\\
508.01	0.00578527250539405\\
509.01	0.0058029255648646\\
510.01	0.00582092142417603\\
511.01	0.00583926840003969\\
512.01	0.00585797490797739\\
513.01	0.0058770494678337\\
514.01	0.00589650071095073\\
515.01	0.00591633738786673\\
516.01	0.00593656837525181\\
517.01	0.0059572026809165\\
518.01	0.00597824944621915\\
519.01	0.0059997179460891\\
520.01	0.00602161758796987\\
521.01	0.0060439579112614\\
522.01	0.00606674858788415\\
523.01	0.00608999942394906\\
524.01	0.00611372036237272\\
525.01	0.00613792148619637\\
526.01	0.00616261302230617\\
527.01	0.0061878053452192\\
528.01	0.0062135089806068\\
529.01	0.00623973460829085\\
530.01	0.0062664930645544\\
531.01	0.00629379534375352\\
532.01	0.00632165259935641\\
533.01	0.00635007614461695\\
534.01	0.00637907745305501\\
535.01	0.00640866815875527\\
536.01	0.00643886005636567\\
537.01	0.00646966510064252\\
538.01	0.00650109540539154\\
539.01	0.00653316324167511\\
540.01	0.00656588103517573\\
541.01	0.00659926136263406\\
542.01	0.0066333169472997\\
543.01	0.00666806065334119\\
544.01	0.00670350547915122\\
545.01	0.0067396645494603\\
546.01	0.00677655110613201\\
547.01	0.00681417849748861\\
548.01	0.00685256016600407\\
549.01	0.0068917096341935\\
550.01	0.00693164048852518\\
551.01	0.00697236636117179\\
552.01	0.00701390090940104\\
553.01	0.00705625779239237\\
554.01	0.00709945064524056\\
555.01	0.0071434930498821\\
556.01	0.00718839850265694\\
557.01	0.0072341803781922\\
558.01	0.0072808518892716\\
559.01	0.00732842604233042\\
560.01	0.00737691558819096\\
561.01	0.00742633296762457\\
562.01	0.00747669025130068\\
563.01	0.00752799907365417\\
564.01	0.00758027056017663\\
565.01	0.00763351524761381\\
566.01	0.00768774299653275\\
567.01	0.00774296289570682\\
568.01	0.00779918315776296\\
569.01	0.0078564110055382\\
570.01	0.00791465254861082\\
571.01	0.00797391264950997\\
572.01	0.00803419477916598\\
573.01	0.00809550086125977\\
574.01	0.00815783110525981\\
575.01	0.00822118382812191\\
576.01	0.00828555526487572\\
577.01	0.00835093936865398\\
578.01	0.00841732760115599\\
579.01	0.00848470871510037\\
580.01	0.00855306853095043\\
581.01	0.00862238971112278\\
582.01	0.00869265153606965\\
583.01	0.00876382968811866\\
584.01	0.00883589605083215\\
585.01	0.00890881853400895\\
586.01	0.00898256093740584\\
587.01	0.00905708286994538\\
588.01	0.00913233974577567\\
589.01	0.00920828288426781\\
590.01	0.00928485974814205\\
591.01	0.00936201436272632\\
592.01	0.00943968797026876\\
593.01	0.0095178199867371\\
594.01	0.0095963493452442\\
595.01	0.00967521633087718\\
596.01	0.00975436503718722\\
597.01	0.00983355571844508\\
598.01	0.00990826330299362\\
599.01	0.00997053306357289\\
599.02	0.00997104257304143\\
599.03	0.00997154904691472\\
599.04	0.00997205245565375\\
599.05	0.00997255276942872\\
599.06	0.00997304995811617\\
599.07	0.00997354399129606\\
599.08	0.00997403483824883\\
599.09	0.00997452246795248\\
599.1	0.00997500684907951\\
599.11	0.00997548794999398\\
599.12	0.00997596573874838\\
599.13	0.0099764401830806\\
599.14	0.0099769112504108\\
599.15	0.00997737890783826\\
599.16	0.00997784312213823\\
599.17	0.0099783038597587\\
599.18	0.00997876108681718\\
599.19	0.00997921476909742\\
599.2	0.00997966487204611\\
599.21	0.00998011136076955\\
599.22	0.00998055420003028\\
599.23	0.00998099335424369\\
599.24	0.00998142878747458\\
599.25	0.00998186046343367\\
599.26	0.00998228834359041\\
599.27	0.00998271238778643\\
599.28	0.00998313255546381\\
599.29	0.00998354880566111\\
599.3	0.00998396109700947\\
599.31	0.00998436938772849\\
599.32	0.00998477363562221\\
599.33	0.00998517379807499\\
599.34	0.00998556983204737\\
599.35	0.00998596169407188\\
599.36	0.00998634934024881\\
599.37	0.00998673272624192\\
599.38	0.00998711180727415\\
599.39	0.00998748653812326\\
599.4	0.00998785687311743\\
599.41	0.00998822276613081\\
599.42	0.00998858417057903\\
599.43	0.00998894103941468\\
599.44	0.00998929332512275\\
599.45	0.00998964097971596\\
599.46	0.00998998395473014\\
599.47	0.00999032220121951\\
599.48	0.0099906556697519\\
599.49	0.00999098431040396\\
599.5	0.00999130807275631\\
599.51	0.00999162690588863\\
599.52	0.00999194075837471\\
599.53	0.00999224957827746\\
599.54	0.00999255331314386\\
599.55	0.00999285190999987\\
599.56	0.00999314531534524\\
599.57	0.00999343347514839\\
599.58	0.00999371633484107\\
599.59	0.00999399383931314\\
599.6	0.00999426593290715\\
599.61	0.009994532559413\\
599.62	0.0099947936620624\\
599.63	0.00999504918352342\\
599.64	0.00999529906589491\\
599.65	0.00999554325070086\\
599.66	0.00999578167888471\\
599.67	0.00999601429080366\\
599.68	0.00999624102622284\\
599.69	0.00999646182430947\\
599.7	0.00999667662362697\\
599.71	0.00999688536212897\\
599.72	0.00999708797715332\\
599.73	0.00999728440541596\\
599.74	0.00999747458300482\\
599.75	0.0099976584453736\\
599.76	0.00999783592733551\\
599.77	0.00999800696305692\\
599.78	0.00999817148605102\\
599.79	0.00999832942917133\\
599.8	0.00999848072460517\\
599.81	0.00999862530386713\\
599.82	0.00999876309779239\\
599.83	0.00999889403653003\\
599.84	0.00999901804953622\\
599.85	0.00999913506556741\\
599.86	0.00999924501267341\\
599.87	0.00999934781819042\\
599.88	0.00999944340873394\\
599.89	0.0099995317101917\\
599.9	0.00999961264771648\\
599.91	0.00999968614571878\\
599.92	0.00999975212785958\\
599.93	0.00999981051704285\\
599.94	0.00999986123540817\\
599.95	0.00999990420432308\\
599.96	0.00999993934437554\\
599.97	0.00999996657536618\\
599.98	0.00999998581630055\\
599.99	0.00999999698538124\\
600	0.01\\
};
\addplot [color=mycolor21,solid,forget plot]
  table[row sep=crcr]{%
0.01	0.00476861616730102\\
1.01	0.00476861719411399\\
2.01	0.00476861824248907\\
3.01	0.00476861931287929\\
4.01	0.00476862040574691\\
5.01	0.00476862152156397\\
6.01	0.00476862266081259\\
7.01	0.00476862382398489\\
8.01	0.0047686250115834\\
9.01	0.00476862622412094\\
10.01	0.00476862746212145\\
11.01	0.00476862872611968\\
12.01	0.00476863001666161\\
13.01	0.00476863133430465\\
14.01	0.00476863267961813\\
15.01	0.00476863405318324\\
16.01	0.00476863545559318\\
17.01	0.00476863688745406\\
18.01	0.00476863834938408\\
19.01	0.00476863984201516\\
20.01	0.00476864136599198\\
21.01	0.00476864292197279\\
22.01	0.00476864451062983\\
23.01	0.00476864613264942\\
24.01	0.00476864778873234\\
25.01	0.00476864947959391\\
26.01	0.0047686512059648\\
27.01	0.00476865296859073\\
28.01	0.00476865476823304\\
29.01	0.00476865660566943\\
30.01	0.00476865848169362\\
31.01	0.00476866039711596\\
32.01	0.00476866235276407\\
33.01	0.00476866434948292\\
34.01	0.00476866638813504\\
35.01	0.00476866846960131\\
36.01	0.00476867059478095\\
37.01	0.00476867276459203\\
38.01	0.00476867497997197\\
39.01	0.00476867724187799\\
40.01	0.00476867955128718\\
41.01	0.00476868190919749\\
42.01	0.00476868431662754\\
43.01	0.00476868677461749\\
44.01	0.00476868928422916\\
45.01	0.00476869184654694\\
46.01	0.00476869446267791\\
47.01	0.00476869713375229\\
48.01	0.00476869986092417\\
49.01	0.0047687026453717\\
50.01	0.00476870548829811\\
51.01	0.00476870839093149\\
52.01	0.00476871135452607\\
53.01	0.00476871438036226\\
54.01	0.00476871746974734\\
55.01	0.00476872062401618\\
56.01	0.00476872384453153\\
57.01	0.00476872713268495\\
58.01	0.00476873048989689\\
59.01	0.00476873391761808\\
60.01	0.00476873741732951\\
61.01	0.00476874099054314\\
62.01	0.00476874463880295\\
63.01	0.004768748363685\\
64.01	0.00476875216679896\\
65.01	0.00476875604978781\\
66.01	0.00476876001432934\\
67.01	0.00476876406213643\\
68.01	0.00476876819495806\\
69.01	0.00476877241457976\\
70.01	0.00476877672282479\\
71.01	0.00476878112155443\\
72.01	0.00476878561266933\\
73.01	0.00476879019810998\\
74.01	0.00476879487985733\\
75.01	0.0047687996599344\\
76.01	0.00476880454040637\\
77.01	0.00476880952338192\\
78.01	0.004768814611014\\
79.01	0.00476881980550074\\
80.01	0.00476882510908637\\
81.01	0.00476883052406239\\
82.01	0.00476883605276847\\
83.01	0.00476884169759309\\
84.01	0.0047688474609754\\
85.01	0.0047688533454051\\
86.01	0.00476885935342482\\
87.01	0.00476886548763007\\
88.01	0.0047688717506712\\
89.01	0.00476887814525404\\
90.01	0.0047688846741414\\
91.01	0.0047688913401539\\
92.01	0.00476889814617147\\
93.01	0.00476890509513456\\
94.01	0.00476891219004512\\
95.01	0.00476891943396873\\
96.01	0.00476892683003471\\
97.01	0.00476893438143813\\
98.01	0.00476894209144139\\
99.01	0.00476894996337524\\
100.01	0.00476895800064057\\
101.01	0.00476896620670937\\
102.01	0.00476897458512662\\
103.01	0.00476898313951174\\
104.01	0.00476899187356015\\
105.01	0.00476900079104474\\
106.01	0.00476900989581783\\
107.01	0.00476901919181233\\
108.01	0.0047690286830437\\
109.01	0.00476903837361184\\
110.01	0.00476904826770286\\
111.01	0.00476905836959025\\
112.01	0.00476906868363763\\
113.01	0.00476907921429993\\
114.01	0.00476908996612559\\
115.01	0.00476910094375872\\
116.01	0.00476911215194051\\
117.01	0.00476912359551201\\
118.01	0.0047691352794156\\
119.01	0.00476914720869763\\
120.01	0.0047691593885099\\
121.01	0.00476917182411275\\
122.01	0.00476918452087656\\
123.01	0.00476919748428457\\
124.01	0.00476921071993493\\
125.01	0.00476922423354344\\
126.01	0.00476923803094538\\
127.01	0.00476925211809902\\
128.01	0.00476926650108682\\
129.01	0.00476928118611947\\
130.01	0.0047692961795377\\
131.01	0.00476931148781534\\
132.01	0.00476932711756173\\
133.01	0.00476934307552487\\
134.01	0.00476935936859459\\
135.01	0.00476937600380462\\
136.01	0.00476939298833644\\
137.01	0.00476941032952231\\
138.01	0.00476942803484792\\
139.01	0.00476944611195597\\
140.01	0.00476946456864935\\
141.01	0.00476948341289439\\
142.01	0.00476950265282454\\
143.01	0.00476952229674375\\
144.01	0.00476954235312999\\
145.01	0.00476956283063863\\
146.01	0.00476958373810657\\
147.01	0.00476960508455605\\
148.01	0.00476962687919795\\
149.01	0.00476964913143642\\
150.01	0.00476967185087242\\
151.01	0.00476969504730804\\
152.01	0.00476971873075087\\
153.01	0.00476974291141755\\
154.01	0.00476976759973927\\
155.01	0.00476979280636534\\
156.01	0.00476981854216808\\
157.01	0.0047698448182474\\
158.01	0.00476987164593562\\
159.01	0.00476989903680212\\
160.01	0.00476992700265862\\
161.01	0.00476995555556388\\
162.01	0.0047699847078289\\
163.01	0.00477001447202256\\
164.01	0.00477004486097647\\
165.01	0.00477007588779071\\
166.01	0.0047701075658392\\
167.01	0.00477013990877577\\
168.01	0.00477017293053987\\
169.01	0.00477020664536186\\
170.01	0.00477024106777015\\
171.01	0.00477027621259675\\
172.01	0.00477031209498327\\
173.01	0.00477034873038823\\
174.01	0.00477038613459291\\
175.01	0.00477042432370826\\
176.01	0.00477046331418203\\
177.01	0.00477050312280514\\
178.01	0.00477054376671938\\
179.01	0.00477058526342429\\
180.01	0.00477062763078501\\
181.01	0.0047706708870392\\
182.01	0.00477071505080554\\
183.01	0.00477076014109076\\
184.01	0.00477080617729847\\
185.01	0.00477085317923694\\
186.01	0.00477090116712756\\
187.01	0.00477095016161314\\
188.01	0.00477100018376705\\
189.01	0.00477105125510177\\
190.01	0.00477110339757805\\
191.01	0.00477115663361444\\
192.01	0.00477121098609609\\
193.01	0.00477126647838494\\
194.01	0.00477132313432947\\
195.01	0.00477138097827436\\
196.01	0.00477144003507117\\
197.01	0.00477150033008836\\
198.01	0.00477156188922195\\
199.01	0.00477162473890695\\
200.01	0.00477168890612775\\
201.01	0.00477175441842938\\
202.01	0.00477182130392964\\
203.01	0.00477188959133029\\
204.01	0.00477195930992912\\
205.01	0.00477203048963237\\
206.01	0.00477210316096681\\
207.01	0.00477217735509267\\
208.01	0.00477225310381692\\
209.01	0.00477233043960577\\
210.01	0.00477240939559841\\
211.01	0.00477249000562105\\
212.01	0.00477257230420047\\
213.01	0.00477265632657851\\
214.01	0.00477274210872641\\
215.01	0.00477282968736015\\
216.01	0.00477291909995488\\
217.01	0.0047730103847609\\
218.01	0.00477310358081895\\
219.01	0.00477319872797646\\
220.01	0.00477329586690389\\
221.01	0.00477339503911099\\
222.01	0.00477349628696452\\
223.01	0.0047735996537048\\
224.01	0.00477370518346353\\
225.01	0.0047738129212819\\
226.01	0.00477392291312882\\
227.01	0.00477403520591931\\
228.01	0.0047741498475338\\
229.01	0.00477426688683723\\
230.01	0.0047743863736991\\
231.01	0.00477450835901302\\
232.01	0.00477463289471761\\
233.01	0.00477476003381704\\
234.01	0.00477488983040249\\
235.01	0.00477502233967354\\
236.01	0.00477515761796042\\
237.01	0.00477529572274589\\
238.01	0.0047754367126888\\
239.01	0.00477558064764674\\
240.01	0.00477572758869982\\
241.01	0.0047758775981749\\
242.01	0.00477603073966978\\
243.01	0.00477618707807849\\
244.01	0.00477634667961647\\
245.01	0.00477650961184622\\
246.01	0.00477667594370403\\
247.01	0.00477684574552628\\
248.01	0.00477701908907684\\
249.01	0.00477719604757469\\
250.01	0.00477737669572197\\
251.01	0.00477756110973278\\
252.01	0.00477774936736183\\
253.01	0.00477794154793462\\
254.01	0.00477813773237686\\
255.01	0.00477833800324561\\
256.01	0.00477854244475999\\
257.01	0.00477875114283273\\
258.01	0.004778964185102\\
259.01	0.00477918166096468\\
260.01	0.00477940366160852\\
261.01	0.00477963028004651\\
262.01	0.00477986161115046\\
263.01	0.00478009775168553\\
264.01	0.00478033880034535\\
265.01	0.00478058485778853\\
266.01	0.00478083602667312\\
267.01	0.00478109241169455\\
268.01	0.00478135411962249\\
269.01	0.00478162125933836\\
270.01	0.00478189394187305\\
271.01	0.0047821722804466\\
272.01	0.00478245639050634\\
273.01	0.00478274638976706\\
274.01	0.00478304239825111\\
275.01	0.00478334453832885\\
276.01	0.00478365293475938\\
277.01	0.00478396771473295\\
278.01	0.0047842890079122\\
279.01	0.00478461694647471\\
280.01	0.00478495166515576\\
281.01	0.00478529330129172\\
282.01	0.00478564199486355\\
283.01	0.00478599788854078\\
284.01	0.00478636112772578\\
285.01	0.00478673186059865\\
286.01	0.00478711023816181\\
287.01	0.00478749641428546\\
288.01	0.00478789054575286\\
289.01	0.00478829279230639\\
290.01	0.00478870331669305\\
291.01	0.00478912228471053\\
292.01	0.00478954986525319\\
293.01	0.00478998623035888\\
294.01	0.0047904315552543\\
295.01	0.00479088601840179\\
296.01	0.00479134980154507\\
297.01	0.00479182308975522\\
298.01	0.00479230607147642\\
299.01	0.00479279893857149\\
300.01	0.0047933018863669\\
301.01	0.00479381511369782\\
302.01	0.00479433882295176\\
303.01	0.00479487322011277\\
304.01	0.00479541851480386\\
305.01	0.00479597492032989\\
306.01	0.00479654265371819\\
307.01	0.00479712193575974\\
308.01	0.00479771299104766\\
309.01	0.00479831604801585\\
310.01	0.00479893133897547\\
311.01	0.0047995591001504\\
312.01	0.00480019957171043\\
313.01	0.00480085299780354\\
314.01	0.00480151962658572\\
315.01	0.00480219971024873\\
316.01	0.00480289350504531\\
317.01	0.00480360127131285\\
318.01	0.00480432327349309\\
319.01	0.00480505978014947\\
320.01	0.00480581106398194\\
321.01	0.00480657740183742\\
322.01	0.0048073590747173\\
323.01	0.00480815636778088\\
324.01	0.00480896957034463\\
325.01	0.00480979897587665\\
326.01	0.00481064488198756\\
327.01	0.00481150759041511\\
328.01	0.00481238740700418\\
329.01	0.00481328464168064\\
330.01	0.00481419960841982\\
331.01	0.0048151326252079\\
332.01	0.00481608401399722\\
333.01	0.00481705410065363\\
334.01	0.00481804321489739\\
335.01	0.00481905169023573\\
336.01	0.00482007986388665\\
337.01	0.00482112807669473\\
338.01	0.00482219667303738\\
339.01	0.00482328600072193\\
340.01	0.00482439641087255\\
341.01	0.00482552825780748\\
342.01	0.00482668189890442\\
343.01	0.00482785769445584\\
344.01	0.00482905600751238\\
345.01	0.00483027720371537\\
346.01	0.00483152165111599\\
347.01	0.00483278971998301\\
348.01	0.00483408178259816\\
349.01	0.00483539821303822\\
350.01	0.00483673938694509\\
351.01	0.00483810568128304\\
352.01	0.00483949747408331\\
353.01	0.00484091514417623\\
354.01	0.00484235907091192\\
355.01	0.00484382963386931\\
356.01	0.00484532721255457\\
357.01	0.00484685218608939\\
358.01	0.00484840493289183\\
359.01	0.00484998583034842\\
360.01	0.00485159525448194\\
361.01	0.0048532335796154\\
362.01	0.00485490117803507\\
363.01	0.00485659841965547\\
364.01	0.00485832567168921\\
365.01	0.00486008329832617\\
366.01	0.00486187166042616\\
367.01	0.00486369111522957\\
368.01	0.00486554201609203\\
369.01	0.0048674247122505\\
370.01	0.0048693395486261\\
371.01	0.00487128686567313\\
372.01	0.00487326699928278\\
373.01	0.0048752802807516\\
374.01	0.00487732703682547\\
375.01	0.0048794075898316\\
376.01	0.00488152225791\\
377.01	0.00488367135536014\\
378.01	0.00488585519311742\\
379.01	0.00488807407937422\\
380.01	0.00489032832036324\\
381.01	0.00489261822132035\\
382.01	0.00489494408764431\\
383.01	0.00489730622627159\\
384.01	0.00489970494728327\\
385.01	0.00490214056576063\\
386.01	0.00490461340390473\\
387.01	0.00490712379343222\\
388.01	0.00490967207825653\\
389.01	0.00491225861745935\\
390.01	0.0049148837885504\\
391.01	0.00491754799100624\\
392.01	0.00492025165006933\\
393.01	0.00492299522077592\\
394.01	0.00492577919216772\\
395.01	0.00492860409162466\\
396.01	0.0049314704892353\\
397.01	0.00493437900210082\\
398.01	0.00493733029844079\\
399.01	0.0049403251013409\\
400.01	0.00494336419195468\\
401.01	0.00494644841193792\\
402.01	0.00494957866486805\\
403.01	0.00495275591637341\\
404.01	0.00495598119267769\\
405.01	0.00495925557725979\\
406.01	0.00496258020533975\\
407.01	0.00496595625593554\\
408.01	0.00496938494131029\\
409.01	0.00497286749374186\\
410.01	0.00497640514972274\\
411.01	0.00497999913193614\\
412.01	0.00498365062967331\\
413.01	0.00498736077875032\\
414.01	0.0049911306424406\\
415.01	0.00499496119542432\\
416.01	0.00499885331319024\\
417.01	0.0050028077695501\\
418.01	0.00500682524468592\\
419.01	0.00501090634500699\\
420.01	0.00501505163303954\\
421.01	0.00501926165440415\\
422.01	0.00502353694588678\\
423.01	0.00502787803618943\\
424.01	0.00503228544658343\\
425.01	0.0050367596917496\\
426.01	0.00504130128082537\\
427.01	0.00504591071867695\\
428.01	0.00505058850741656\\
429.01	0.00505533514818503\\
430.01	0.00506015114321863\\
431.01	0.00506503699822133\\
432.01	0.00506999322505837\\
433.01	0.00507502034479057\\
434.01	0.0050801188910639\\
435.01	0.00508528941386699\\
436.01	0.00509053248366578\\
437.01	0.00509584869592081\\
438.01	0.00510123867598589\\
439.01	0.00510670308438306\\
440.01	0.00511224262243777\\
441.01	0.00511785803825243\\
442.01	0.00512355013298382\\
443.01	0.0051293197673799\\
444.01	0.00513516786851617\\
445.01	0.00514109543666084\\
446.01	0.00514710355217816\\
447.01	0.00515319338236579\\
448.01	0.00515936618810334\\
449.01	0.00516562333017308\\
450.01	0.00517196627509768\\
451.01	0.00517839660032253\\
452.01	0.00518491599856333\\
453.01	0.00519152628112898\\
454.01	0.00519822938003053\\
455.01	0.00520502734869473\\
456.01	0.0052119223611215\\
457.01	0.0052189167093529\\
458.01	0.00522601279917589\\
459.01	0.00523321314404475\\
460.01	0.00524052035729951\\
461.01	0.00524793714286628\\
462.01	0.00525546628475967\\
463.01	0.0052631106358542\\
464.01	0.00527087310655482\\
465.01	0.00527875665415938\\
466.01	0.00528676427384443\\
467.01	0.00529489899230969\\
468.01	0.00530316386513296\\
469.01	0.00531156197878918\\
470.01	0.00532009645801362\\
471.01	0.00532877047870171\\
472.01	0.00533758728579034\\
473.01	0.00534655021458\\
474.01	0.00535566271280512\\
475.01	0.00536492835972273\\
476.01	0.00537435087817047\\
477.01	0.00538393413822291\\
478.01	0.00539368215652637\\
479.01	0.0054035990943118\\
480.01	0.00541368925441719\\
481.01	0.00542395707736736\\
482.01	0.00543440713660659\\
483.01	0.00544504413301856\\
484.01	0.00545587288891948\\
485.01	0.00546689834175822\\
486.01	0.00547812553780169\\
487.01	0.00548955962611743\\
488.01	0.00550120585319264\\
489.01	0.00551306955852603\\
490.01	0.00552515617151154\\
491.01	0.00553747120987087\\
492.01	0.00555002027980198\\
493.01	0.00556280907787551\\
494.01	0.00557584339454135\\
495.01	0.00558912911891428\\
496.01	0.00560267224430848\\
497.01	0.00561647887382405\\
498.01	0.00563055522520521\\
499.01	0.00564490763424581\\
500.01	0.00565954255627336\\
501.01	0.00567446656573335\\
502.01	0.00568968635454427\\
503.01	0.00570520873028521\\
504.01	0.00572104061487515\\
505.01	0.00573718904392297\\
506.01	0.00575366116681389\\
507.01	0.00577046424755846\\
508.01	0.00578760566638339\\
509.01	0.00580509292199594\\
510.01	0.00582293363440174\\
511.01	0.00584113554811737\\
512.01	0.00585970653558543\\
513.01	0.00587865460059492\\
514.01	0.00589798788152887\\
515.01	0.00591771465430749\\
516.01	0.00593784333497871\\
517.01	0.00595838248199877\\
518.01	0.00597934079833327\\
519.01	0.00600072713355781\\
520.01	0.00602255048610337\\
521.01	0.00604482000568882\\
522.01	0.00606754499589372\\
523.01	0.00609073491679551\\
524.01	0.00611439938759015\\
525.01	0.00613854818911833\\
526.01	0.00616319126622557\\
527.01	0.00618833872990029\\
528.01	0.00621400085915008\\
529.01	0.00624018810259756\\
530.01	0.00626691107978722\\
531.01	0.00629418058220452\\
532.01	0.00632200757400153\\
533.01	0.00635040319240993\\
534.01	0.0063793787477955\\
535.01	0.00640894572329868\\
536.01	0.00643911577399439\\
537.01	0.00646990072550448\\
538.01	0.00650131257200101\\
539.01	0.0065333634735321\\
540.01	0.00656606575260754\\
541.01	0.00659943188997036\\
542.01	0.00663347451947655\\
543.01	0.00666820642199731\\
544.01	0.00670364051824117\\
545.01	0.00673978986038594\\
546.01	0.0067766676223955\\
547.01	0.0068142870888892\\
548.01	0.00685266164241449\\
549.01	0.00689180474896772\\
550.01	0.0069317299415903\\
551.01	0.00697245080185081\\
552.01	0.00701398093901015\\
553.01	0.00705633396664328\\
554.01	0.00709952347647309\\
555.01	0.00714356300915191\\
556.01	0.00718846602169959\\
557.01	0.00723424585128786\\
558.01	0.00728091567503281\\
559.01	0.00732848846543331\\
560.01	0.00737697694106471\\
561.01	0.00742639351211209\\
562.01	0.00747675022029617\\
563.01	0.00752805867272219\\
564.01	0.00758032996915353\\
565.01	0.00763357462219004\\
566.01	0.00768780246981217\\
567.01	0.0077430225797369\\
568.01	0.00779924314502788\\
569.01	0.00785647137040364\\
570.01	0.00791471334870932\\
571.01	0.00797397392705373\\
572.01	0.00803425656217577\\
573.01	0.0080955631646969\\
574.01	0.00815789393205248\\
575.01	0.00822124717007715\\
576.01	0.00828561910347195\\
577.01	0.00835100367571266\\
578.01	0.00841739233939433\\
579.01	0.00848477383857343\\
580.01	0.00855313398539493\\
581.01	0.00862245543422146\\
582.01	0.00869271745766205\\
583.01	0.00876389573039058\\
584.01	0.00883596212852419\\
585.01	0.00890888455469373\\
586.01	0.00898262680189275\\
587.01	0.00905714847288408\\
588.01	0.00913240497654327\\
589.01	0.00920834762824205\\
590.01	0.00928492388848376\\
591.01	0.00936207778282116\\
592.01	0.00943975055701092\\
593.01	0.00951788163487973\\
594.01	0.00959640996309208\\
595.01	0.00967527584766158\\
596.01	0.00975442341254193\\
597.01	0.00983358307576999\\
598.01	0.00990826330299362\\
599.01	0.00997053306357289\\
599.02	0.00997104257304143\\
599.03	0.00997154904691472\\
599.04	0.00997205245565375\\
599.05	0.00997255276942872\\
599.06	0.00997304995811617\\
599.07	0.00997354399129606\\
599.08	0.00997403483824883\\
599.09	0.00997452246795248\\
599.1	0.00997500684907952\\
599.11	0.00997548794999398\\
599.12	0.00997596573874838\\
599.13	0.0099764401830806\\
599.14	0.0099769112504108\\
599.15	0.00997737890783826\\
599.16	0.00997784312213823\\
599.17	0.0099783038597587\\
599.18	0.00997876108681718\\
599.19	0.00997921476909742\\
599.2	0.00997966487204611\\
599.21	0.00998011136076955\\
599.22	0.00998055420003028\\
599.23	0.0099809933542437\\
599.24	0.00998142878747458\\
599.25	0.00998186046343367\\
599.26	0.00998228834359041\\
599.27	0.00998271238778643\\
599.28	0.00998313255546381\\
599.29	0.00998354880566111\\
599.3	0.00998396109700947\\
599.31	0.00998436938772849\\
599.32	0.00998477363562221\\
599.33	0.00998517379807499\\
599.34	0.00998556983204737\\
599.35	0.00998596169407188\\
599.36	0.00998634934024881\\
599.37	0.00998673272624192\\
599.38	0.00998711180727415\\
599.39	0.00998748653812326\\
599.4	0.00998785687311743\\
599.41	0.00998822276613081\\
599.42	0.00998858417057903\\
599.43	0.00998894103941468\\
599.44	0.00998929332512275\\
599.45	0.00998964097971596\\
599.46	0.00998998395473014\\
599.47	0.00999032220121951\\
599.48	0.0099906556697519\\
599.49	0.00999098431040396\\
599.5	0.00999130807275631\\
599.51	0.00999162690588863\\
599.52	0.00999194075837471\\
599.53	0.00999224957827746\\
599.54	0.00999255331314387\\
599.55	0.00999285190999987\\
599.56	0.00999314531534524\\
599.57	0.00999343347514839\\
599.58	0.00999371633484107\\
599.59	0.00999399383931314\\
599.6	0.00999426593290715\\
599.61	0.009994532559413\\
599.62	0.0099947936620624\\
599.63	0.00999504918352342\\
599.64	0.00999529906589491\\
599.65	0.00999554325070086\\
599.66	0.00999578167888471\\
599.67	0.00999601429080366\\
599.68	0.00999624102622283\\
599.69	0.00999646182430947\\
599.7	0.00999667662362697\\
599.71	0.00999688536212897\\
599.72	0.00999708797715332\\
599.73	0.00999728440541595\\
599.74	0.00999747458300482\\
599.75	0.0099976584453736\\
599.76	0.0099978359273355\\
599.77	0.00999800696305692\\
599.78	0.00999817148605102\\
599.79	0.00999832942917133\\
599.8	0.00999848072460517\\
599.81	0.00999862530386713\\
599.82	0.00999876309779239\\
599.83	0.00999889403653003\\
599.84	0.00999901804953622\\
599.85	0.00999913506556741\\
599.86	0.00999924501267341\\
599.87	0.00999934781819042\\
599.88	0.00999944340873394\\
599.89	0.00999953171019171\\
599.9	0.00999961264771648\\
599.91	0.00999968614571878\\
599.92	0.00999975212785957\\
599.93	0.00999981051704285\\
599.94	0.00999986123540817\\
599.95	0.00999990420432308\\
599.96	0.00999993934437554\\
599.97	0.00999996657536618\\
599.98	0.00999998581630055\\
599.99	0.00999999698538124\\
600	0.01\\
};
\addplot [color=black!20!mycolor21,solid,forget plot]
  table[row sep=crcr]{%
0.01	0.0048411429700902\\
1.01	0.00484114397590001\\
2.01	0.00484114500270012\\
3.01	0.00484114605092785\\
4.01	0.00484114712102995\\
5.01	0.00484114821346211\\
6.01	0.00484114932868977\\
7.01	0.00484115046718776\\
8.01	0.00484115162944103\\
9.01	0.00484115281594452\\
10.01	0.00484115402720339\\
11.01	0.00484115526373359\\
12.01	0.00484115652606144\\
13.01	0.00484115781472463\\
14.01	0.00484115913027153\\
15.01	0.00484116047326229\\
16.01	0.00484116184426858\\
17.01	0.00484116324387371\\
18.01	0.00484116467267361\\
19.01	0.00484116613127625\\
20.01	0.00484116762030226\\
21.01	0.0048411691403854\\
22.01	0.00484117069217238\\
23.01	0.00484117227632343\\
24.01	0.00484117389351243\\
25.01	0.00484117554442746\\
26.01	0.00484117722977057\\
27.01	0.00484117895025858\\
28.01	0.00484118070662343\\
29.01	0.00484118249961185\\
30.01	0.00484118432998638\\
31.01	0.0048411861985254\\
32.01	0.00484118810602319\\
33.01	0.00484119005329087\\
34.01	0.0048411920411561\\
35.01	0.00484119407046386\\
36.01	0.00484119614207673\\
37.01	0.00484119825687521\\
38.01	0.00484120041575797\\
39.01	0.00484120261964256\\
40.01	0.00484120486946539\\
41.01	0.00484120716618226\\
42.01	0.00484120951076895\\
43.01	0.00484121190422147\\
44.01	0.00484121434755651\\
45.01	0.00484121684181192\\
46.01	0.00484121938804682\\
47.01	0.00484122198734267\\
48.01	0.00484122464080317\\
49.01	0.00484122734955504\\
50.01	0.00484123011474827\\
51.01	0.00484123293755674\\
52.01	0.00484123581917857\\
53.01	0.00484123876083713\\
54.01	0.00484124176378073\\
55.01	0.00484124482928384\\
56.01	0.00484124795864718\\
57.01	0.00484125115319854\\
58.01	0.00484125441429345\\
59.01	0.00484125774331519\\
60.01	0.00484126114167586\\
61.01	0.00484126461081708\\
62.01	0.00484126815221\\
63.01	0.00484127176735651\\
64.01	0.00484127545778942\\
65.01	0.00484127922507353\\
66.01	0.004841283070806\\
67.01	0.00484128699661708\\
68.01	0.00484129100417076\\
69.01	0.00484129509516563\\
70.01	0.00484129927133535\\
71.01	0.00484130353444964\\
72.01	0.00484130788631471\\
73.01	0.00484131232877434\\
74.01	0.00484131686371049\\
75.01	0.00484132149304406\\
76.01	0.00484132621873572\\
77.01	0.00484133104278675\\
78.01	0.00484133596723996\\
79.01	0.00484134099418035\\
80.01	0.00484134612573614\\
81.01	0.00484135136407957\\
82.01	0.00484135671142785\\
83.01	0.00484136217004423\\
84.01	0.00484136774223851\\
85.01	0.00484137343036879\\
86.01	0.00484137923684153\\
87.01	0.00484138516411299\\
88.01	0.00484139121469049\\
89.01	0.00484139739113305\\
90.01	0.00484140369605274\\
91.01	0.00484141013211564\\
92.01	0.00484141670204284\\
93.01	0.00484142340861186\\
94.01	0.00484143025465771\\
95.01	0.00484143724307372\\
96.01	0.00484144437681321\\
97.01	0.0048414516588909\\
98.01	0.00484145909238325\\
99.01	0.00484146668043069\\
100.01	0.00484147442623818\\
101.01	0.00484148233307742\\
102.01	0.00484149040428737\\
103.01	0.00484149864327595\\
104.01	0.00484150705352175\\
105.01	0.00484151563857499\\
106.01	0.00484152440205915\\
107.01	0.00484153334767284\\
108.01	0.00484154247919084\\
109.01	0.00484155180046579\\
110.01	0.00484156131542998\\
111.01	0.00484157102809701\\
112.01	0.00484158094256308\\
113.01	0.00484159106300888\\
114.01	0.00484160139370177\\
115.01	0.00484161193899676\\
116.01	0.00484162270333881\\
117.01	0.00484163369126454\\
118.01	0.0048416449074042\\
119.01	0.00484165635648359\\
120.01	0.00484166804332589\\
121.01	0.00484167997285365\\
122.01	0.00484169215009089\\
123.01	0.00484170458016531\\
124.01	0.00484171726831029\\
125.01	0.00484173021986679\\
126.01	0.00484174344028601\\
127.01	0.00484175693513141\\
128.01	0.00484177071008135\\
129.01	0.00484178477093053\\
130.01	0.00484179912359343\\
131.01	0.00484181377410595\\
132.01	0.00484182872862858\\
133.01	0.00484184399344847\\
134.01	0.00484185957498212\\
135.01	0.00484187547977799\\
136.01	0.00484189171451951\\
137.01	0.0048419082860273\\
138.01	0.00484192520126239\\
139.01	0.00484194246732876\\
140.01	0.00484196009147685\\
141.01	0.0048419780811057\\
142.01	0.00484199644376682\\
143.01	0.00484201518716645\\
144.01	0.00484203431916946\\
145.01	0.00484205384780233\\
146.01	0.00484207378125599\\
147.01	0.00484209412788991\\
148.01	0.00484211489623504\\
149.01	0.00484213609499745\\
150.01	0.00484215773306196\\
151.01	0.00484217981949567\\
152.01	0.00484220236355148\\
153.01	0.00484222537467247\\
154.01	0.00484224886249482\\
155.01	0.00484227283685254\\
156.01	0.00484229730778137\\
157.01	0.00484232228552252\\
158.01	0.00484234778052693\\
159.01	0.00484237380345987\\
160.01	0.00484240036520472\\
161.01	0.00484242747686796\\
162.01	0.00484245514978336\\
163.01	0.00484248339551639\\
164.01	0.00484251222586927\\
165.01	0.00484254165288583\\
166.01	0.00484257168885596\\
167.01	0.00484260234632095\\
168.01	0.00484263363807816\\
169.01	0.00484266557718686\\
170.01	0.00484269817697271\\
171.01	0.00484273145103367\\
172.01	0.00484276541324545\\
173.01	0.00484280007776675\\
174.01	0.00484283545904539\\
175.01	0.0048428715718237\\
176.01	0.00484290843114478\\
177.01	0.00484294605235853\\
178.01	0.00484298445112744\\
179.01	0.00484302364343339\\
180.01	0.0048430636455835\\
181.01	0.00484310447421708\\
182.01	0.00484314614631209\\
183.01	0.00484318867919209\\
184.01	0.00484323209053285\\
185.01	0.00484327639836964\\
186.01	0.00484332162110407\\
187.01	0.00484336777751196\\
188.01	0.00484341488675034\\
189.01	0.00484346296836541\\
190.01	0.00484351204229985\\
191.01	0.0048435621289009\\
192.01	0.00484361324892868\\
193.01	0.004843665423564\\
194.01	0.00484371867441682\\
195.01	0.00484377302353495\\
196.01	0.00484382849341226\\
197.01	0.00484388510699811\\
198.01	0.00484394288770604\\
199.01	0.00484400185942281\\
200.01	0.00484406204651801\\
201.01	0.00484412347385346\\
202.01	0.00484418616679283\\
203.01	0.0048442501512117\\
204.01	0.00484431545350764\\
205.01	0.00484438210060993\\
206.01	0.00484445011999082\\
207.01	0.00484451953967542\\
208.01	0.00484459038825309\\
209.01	0.00484466269488806\\
210.01	0.00484473648933093\\
211.01	0.00484481180192987\\
212.01	0.00484488866364231\\
213.01	0.00484496710604693\\
214.01	0.00484504716135574\\
215.01	0.00484512886242598\\
216.01	0.00484521224277311\\
217.01	0.00484529733658319\\
218.01	0.00484538417872594\\
219.01	0.00484547280476801\\
220.01	0.00484556325098617\\
221.01	0.00484565555438125\\
222.01	0.00484574975269163\\
223.01	0.00484584588440766\\
224.01	0.00484594398878603\\
225.01	0.00484604410586413\\
226.01	0.00484614627647524\\
227.01	0.00484625054226349\\
228.01	0.00484635694569925\\
229.01	0.00484646553009466\\
230.01	0.00484657633961966\\
231.01	0.00484668941931833\\
232.01	0.00484680481512492\\
233.01	0.00484692257388074\\
234.01	0.00484704274335125\\
235.01	0.00484716537224299\\
236.01	0.00484729051022151\\
237.01	0.00484741820792921\\
238.01	0.00484754851700277\\
239.01	0.00484768149009221\\
240.01	0.00484781718087934\\
241.01	0.00484795564409644\\
242.01	0.00484809693554623\\
243.01	0.00484824111212054\\
244.01	0.00484838823182043\\
245.01	0.00484853835377692\\
246.01	0.00484869153827039\\
247.01	0.00484884784675199\\
248.01	0.00484900734186457\\
249.01	0.00484917008746368\\
250.01	0.00484933614863962\\
251.01	0.00484950559173884\\
252.01	0.00484967848438662\\
253.01	0.00484985489550946\\
254.01	0.00485003489535742\\
255.01	0.00485021855552795\\
256.01	0.00485040594898864\\
257.01	0.00485059715010121\\
258.01	0.00485079223464575\\
259.01	0.0048509912798443\\
260.01	0.00485119436438579\\
261.01	0.00485140156845081\\
262.01	0.00485161297373658\\
263.01	0.00485182866348256\\
264.01	0.00485204872249606\\
265.01	0.0048522732371775\\
266.01	0.00485250229554766\\
267.01	0.00485273598727294\\
268.01	0.00485297440369263\\
269.01	0.00485321763784555\\
270.01	0.00485346578449737\\
271.01	0.00485371894016734\\
272.01	0.00485397720315629\\
273.01	0.00485424067357403\\
274.01	0.00485450945336741\\
275.01	0.004854783646348\\
276.01	0.00485506335822079\\
277.01	0.00485534869661168\\
278.01	0.00485563977109645\\
279.01	0.00485593669322879\\
280.01	0.00485623957656975\\
281.01	0.00485654853671559\\
282.01	0.0048568636913264\\
283.01	0.00485718516015521\\
284.01	0.00485751306507655\\
285.01	0.00485784753011481\\
286.01	0.0048581886814732\\
287.01	0.00485853664756201\\
288.01	0.00485889155902688\\
289.01	0.00485925354877727\\
290.01	0.00485962275201414\\
291.01	0.00485999930625809\\
292.01	0.00486038335137674\\
293.01	0.00486077502961195\\
294.01	0.00486117448560672\\
295.01	0.00486158186643137\\
296.01	0.00486199732161045\\
297.01	0.00486242100314759\\
298.01	0.00486285306555097\\
299.01	0.00486329366585744\\
300.01	0.00486374296365687\\
301.01	0.00486420112111491\\
302.01	0.0048646683029957\\
303.01	0.00486514467668332\\
304.01	0.00486563041220268\\
305.01	0.00486612568223942\\
306.01	0.00486663066215824\\
307.01	0.00486714553002138\\
308.01	0.00486767046660457\\
309.01	0.00486820565541281\\
310.01	0.00486875128269441\\
311.01	0.00486930753745329\\
312.01	0.00486987461146118\\
313.01	0.00487045269926614\\
314.01	0.00487104199820167\\
315.01	0.00487164270839258\\
316.01	0.00487225503275988\\
317.01	0.00487287917702328\\
318.01	0.00487351534970219\\
319.01	0.0048741637621145\\
320.01	0.00487482462837312\\
321.01	0.00487549816538051\\
322.01	0.00487618459282051\\
323.01	0.00487688413314849\\
324.01	0.00487759701157832\\
325.01	0.00487832345606769\\
326.01	0.00487906369729975\\
327.01	0.00487981796866284\\
328.01	0.00488058650622751\\
329.01	0.00488136954872072\\
330.01	0.00488216733749693\\
331.01	0.00488298011650676\\
332.01	0.00488380813226291\\
333.01	0.00488465163380299\\
334.01	0.00488551087264984\\
335.01	0.00488638610276886\\
336.01	0.0048872775805229\\
337.01	0.0048881855646245\\
338.01	0.00488911031608599\\
339.01	0.00489005209816659\\
340.01	0.00489101117631845\\
341.01	0.00489198781812954\\
342.01	0.00489298229326658\\
343.01	0.00489399487341473\\
344.01	0.0048950258322186\\
345.01	0.0048960754452206\\
346.01	0.00489714398980129\\
347.01	0.00489823174511918\\
348.01	0.00489933899205165\\
349.01	0.00490046601313895\\
350.01	0.00490161309252951\\
351.01	0.0049027805159302\\
352.01	0.00490396857056039\\
353.01	0.00490517754511256\\
354.01	0.00490640772971881\\
355.01	0.00490765941592694\\
356.01	0.00490893289668548\\
357.01	0.0049102284663405\\
358.01	0.00491154642064512\\
359.01	0.00491288705678504\\
360.01	0.00491425067341986\\
361.01	0.00491563757074398\\
362.01	0.0049170480505699\\
363.01	0.0049184824164339\\
364.01	0.00491994097372978\\
365.01	0.0049214240298706\\
366.01	0.00492293189448334\\
367.01	0.00492446487963805\\
368.01	0.00492602330011557\\
369.01	0.00492760747371546\\
370.01	0.00492921772160859\\
371.01	0.00493085436873687\\
372.01	0.00493251774426251\\
373.01	0.00493420818207001\\
374.01	0.00493592602132382\\
375.01	0.00493767160708272\\
376.01	0.00493944529097344\\
377.01	0.00494124743192449\\
378.01	0.00494307839696045\\
379.01	0.0049449385620557\\
380.01	0.00494682831304756\\
381.01	0.00494874804660422\\
382.01	0.00495069817124415\\
383.01	0.00495267910840111\\
384.01	0.0049546912935258\\
385.01	0.004956735177215\\
386.01	0.00495881122635457\\
387.01	0.00496091992526212\\
388.01	0.00496306177680927\\
389.01	0.00496523730350368\\
390.01	0.00496744704850431\\
391.01	0.00496969157654296\\
392.01	0.00497197147472009\\
393.01	0.00497428735314021\\
394.01	0.00497663984534968\\
395.01	0.00497902960853638\\
396.01	0.00498145732345294\\
397.01	0.00498392369401963\\
398.01	0.00498642944657121\\
399.01	0.00498897532871037\\
400.01	0.00499156210774107\\
401.01	0.0049941905686622\\
402.01	0.00499686151171602\\
403.01	0.00499957574950577\\
404.01	0.00500233410371484\\
405.01	0.00500513740149072\\
406.01	0.00500798647158629\\
407.01	0.00501088214038635\\
408.01	0.00501382522798316\\
409.01	0.00501681654450414\\
410.01	0.00501985688692159\\
411.01	0.00502294703660079\\
412.01	0.00502608775784397\\
413.01	0.00502927979766606\\
414.01	0.00503252388697983\\
415.01	0.00503582074326243\\
416.01	0.00503917107461341\\
417.01	0.00504257558490201\\
418.01	0.00504603497944772\\
419.01	0.0050495499704387\\
420.01	0.00505312128116506\\
421.01	0.00505674964850621\\
422.01	0.00506043582437861\\
423.01	0.00506418057710659\\
424.01	0.0050679846928981\\
425.01	0.00507184897743225\\
426.01	0.00507577425755941\\
427.01	0.00507976138311518\\
428.01	0.00508381122884881\\
429.01	0.00508792469646221\\
430.01	0.00509210271675953\\
431.01	0.00509634625190054\\
432.01	0.00510065629775254\\
433.01	0.00510503388633241\\
434.01	0.00510948008832686\\
435.01	0.00511399601568081\\
436.01	0.00511858282423445\\
437.01	0.00512324171639403\\
438.01	0.00512797394381401\\
439.01	0.00513278081006661\\
440.01	0.00513766367327212\\
441.01	0.00514262394866119\\
442.01	0.00514766311103754\\
443.01	0.00515278269710621\\
444.01	0.0051579843076337\\
445.01	0.00516326960940153\\
446.01	0.00516864033691951\\
447.01	0.00517409829386191\\
448.01	0.00517964535419518\\
449.01	0.00518528346296747\\
450.01	0.00519101463673807\\
451.01	0.00519684096363207\\
452.01	0.00520276460301565\\
453.01	0.00520878778479873\\
454.01	0.00521491280838859\\
455.01	0.00522114204133339\\
456.01	0.00522747791771257\\
457.01	0.00523392293635468\\
458.01	0.00524047965897778\\
459.01	0.00524715070837192\\
460.01	0.00525393876675957\\
461.01	0.00526084657447919\\
462.01	0.00526787692914715\\
463.01	0.00527503268544641\\
464.01	0.00528231675567659\\
465.01	0.00528973211116573\\
466.01	0.00529728178459876\\
467.01	0.00530496887325003\\
468.01	0.0053127965430252\\
469.01	0.00532076803312209\\
470.01	0.0053288866610266\\
471.01	0.005337155827472\\
472.01	0.00534557902094231\\
473.01	0.0053541598213048\\
474.01	0.00536290190225683\\
475.01	0.00537180903246989\\
476.01	0.0053808850756099\\
477.01	0.00539013398969207\\
478.01	0.00539955982617882\\
479.01	0.00540916672897367\\
480.01	0.00541895893336858\\
481.01	0.00542894076500924\\
482.01	0.00543911663894388\\
483.01	0.00544949105882525\\
484.01	0.00546006861633296\\
485.01	0.00547085399088189\\
486.01	0.0054818519496684\\
487.01	0.00549306734809753\\
488.01	0.00550450513061362\\
489.01	0.00551617033193466\\
490.01	0.00552806807866621\\
491.01	0.00554020359124267\\
492.01	0.00555258218611994\\
493.01	0.00556520927811989\\
494.01	0.00557809038281561\\
495.01	0.00559123111884452\\
496.01	0.00560463721005036\\
497.01	0.00561831448738806\\
498.01	0.00563226889057196\\
499.01	0.00564650646950696\\
500.01	0.00566103338559742\\
501.01	0.00567585591306603\\
502.01	0.00569098044040994\\
503.01	0.00570641347207064\\
504.01	0.00572216163033214\\
505.01	0.0057382316574332\\
506.01	0.00575463041787029\\
507.01	0.00577136490085814\\
508.01	0.00578844222291054\\
509.01	0.00580586963049919\\
510.01	0.00582365450275156\\
511.01	0.00584180435414818\\
512.01	0.00586032683719329\\
513.01	0.00587922974503828\\
514.01	0.00589852101405182\\
515.01	0.00591820872634428\\
516.01	0.00593830111225706\\
517.01	0.00595880655283618\\
518.01	0.00597973358230162\\
519.01	0.0060010908905138\\
520.01	0.00602288732542533\\
521.01	0.00604513189549417\\
522.01	0.0060678337720324\\
523.01	0.00609100229146371\\
524.01	0.00611464695746536\\
525.01	0.00613877744297096\\
526.01	0.00616340359201505\\
527.01	0.00618853542139793\\
528.01	0.00621418312215408\\
529.01	0.00624035706080232\\
530.01	0.00626706778035532\\
531.01	0.00629432600106118\\
532.01	0.00632214262084489\\
533.01	0.00635052871541156\\
534.01	0.00637949553797108\\
535.01	0.00640905451853623\\
536.01	0.00643921726274706\\
537.01	0.00646999555016789\\
538.01	0.00650140133200039\\
539.01	0.00653344672815072\\
540.01	0.00656614402358066\\
541.01	0.0065995056638683\\
542.01	0.00663354424989565\\
543.01	0.00666827253156867\\
544.01	0.00670370340047058\\
545.01	0.00673984988133659\\
546.01	0.00677672512222907\\
547.01	0.00681434238327824\\
548.01	0.00685271502384668\\
549.01	0.0068918564879546\\
550.01	0.00693178028779452\\
551.01	0.00697249998514508\\
552.01	0.00701402917047571\\
553.01	0.00705638143951728\\
554.01	0.00709957036705249\\
555.01	0.00714360947765906\\
556.01	0.0071885122131167\\
557.01	0.00723429189616345\\
558.01	0.00728096169026451\\
559.01	0.00732853455502859\\
560.01	0.00737702319688083\\
561.01	0.00742644001457306\\
562.01	0.00747679703908753\\
563.01	0.00752810586745972\\
564.01	0.00758037759002343\\
565.01	0.00763362271055724\\
566.01	0.00768785105879237\\
567.01	0.0077430716947288\\
568.01	0.0077992928041998\\
569.01	0.00785652158513176\\
570.01	0.00791476412396291\\
571.01	0.00797402526172421\\
572.01	0.00803430844934671\\
573.01	0.00809561559185461\\
574.01	0.00815794688123565\\
575.01	0.00822130061796784\\
576.01	0.0082856730214308\\
577.01	0.00835105802976322\\
578.01	0.00841744709016409\\
579.01	0.00848482894120108\\
580.01	0.00855318938941672\\
581.01	0.00862251108345263\\
582.01	0.00869277329009313\\
583.01	0.00876395167812282\\
584.01	0.0088360181177741\\
585.01	0.00890894050590256\\
586.01	0.00898268262998571\\
587.01	0.00905720408773334\\
588.01	0.00913246028370159\\
589.01	0.0092084025300289\\
590.01	0.00928497828552531\\
591.01	0.00936213157616888\\
592.01	0.00943980365099336\\
593.01	0.00951793394087812\\
594.01	0.00959646140447543\\
595.01	0.00967532636617276\\
596.01	0.00975447297649412\\
597.01	0.00983360599464411\\
598.01	0.00990826330299362\\
599.01	0.00997053306357289\\
599.02	0.00997104257304143\\
599.03	0.00997154904691472\\
599.04	0.00997205245565375\\
599.05	0.00997255276942872\\
599.06	0.00997304995811617\\
599.07	0.00997354399129606\\
599.08	0.00997403483824883\\
599.09	0.00997452246795248\\
599.1	0.00997500684907951\\
599.11	0.00997548794999398\\
599.12	0.00997596573874838\\
599.13	0.0099764401830806\\
599.14	0.0099769112504108\\
599.15	0.00997737890783826\\
599.16	0.00997784312213823\\
599.17	0.0099783038597587\\
599.18	0.00997876108681718\\
599.19	0.00997921476909742\\
599.2	0.00997966487204611\\
599.21	0.00998011136076955\\
599.22	0.00998055420003028\\
599.23	0.0099809933542437\\
599.24	0.00998142878747458\\
599.25	0.00998186046343367\\
599.26	0.00998228834359041\\
599.27	0.00998271238778644\\
599.28	0.00998313255546381\\
599.29	0.00998354880566111\\
599.3	0.00998396109700947\\
599.31	0.00998436938772849\\
599.32	0.00998477363562221\\
599.33	0.00998517379807499\\
599.34	0.00998556983204737\\
599.35	0.00998596169407188\\
599.36	0.00998634934024881\\
599.37	0.00998673272624192\\
599.38	0.00998711180727415\\
599.39	0.00998748653812326\\
599.4	0.00998785687311743\\
599.41	0.00998822276613081\\
599.42	0.00998858417057903\\
599.43	0.00998894103941468\\
599.44	0.00998929332512275\\
599.45	0.00998964097971596\\
599.46	0.00998998395473014\\
599.47	0.00999032220121951\\
599.48	0.0099906556697519\\
599.49	0.00999098431040396\\
599.5	0.00999130807275631\\
599.51	0.00999162690588863\\
599.52	0.00999194075837471\\
599.53	0.00999224957827746\\
599.54	0.00999255331314386\\
599.55	0.00999285190999987\\
599.56	0.00999314531534524\\
599.57	0.00999343347514839\\
599.58	0.00999371633484107\\
599.59	0.00999399383931314\\
599.6	0.00999426593290715\\
599.61	0.009994532559413\\
599.62	0.0099947936620624\\
599.63	0.00999504918352342\\
599.64	0.00999529906589491\\
599.65	0.00999554325070086\\
599.66	0.00999578167888471\\
599.67	0.00999601429080366\\
599.68	0.00999624102622283\\
599.69	0.00999646182430947\\
599.7	0.00999667662362697\\
599.71	0.00999688536212897\\
599.72	0.00999708797715332\\
599.73	0.00999728440541595\\
599.74	0.00999747458300482\\
599.75	0.0099976584453736\\
599.76	0.00999783592733551\\
599.77	0.00999800696305692\\
599.78	0.00999817148605102\\
599.79	0.00999832942917133\\
599.8	0.00999848072460517\\
599.81	0.00999862530386713\\
599.82	0.00999876309779239\\
599.83	0.00999889403653003\\
599.84	0.00999901804953622\\
599.85	0.00999913506556741\\
599.86	0.00999924501267341\\
599.87	0.00999934781819042\\
599.88	0.00999944340873394\\
599.89	0.0099995317101917\\
599.9	0.00999961264771648\\
599.91	0.00999968614571878\\
599.92	0.00999975212785958\\
599.93	0.00999981051704285\\
599.94	0.00999986123540817\\
599.95	0.00999990420432308\\
599.96	0.00999993934437554\\
599.97	0.00999996657536618\\
599.98	0.00999998581630055\\
599.99	0.00999999698538124\\
600	0.01\\
};
\addplot [color=black!50!mycolor20,solid,forget plot]
  table[row sep=crcr]{%
0.01	0.00488462543491382\\
1.01	0.00488462642133425\\
2.01	0.00488462742821015\\
3.01	0.00488462845596445\\
4.01	0.0048846295050286\\
5.01	0.00488463057584322\\
6.01	0.00488463166885768\\
7.01	0.00488463278453091\\
8.01	0.00488463392333116\\
9.01	0.00488463508573644\\
10.01	0.00488463627223445\\
11.01	0.00488463748332285\\
12.01	0.00488463871950981\\
13.01	0.00488463998131372\\
14.01	0.0048846412692637\\
15.01	0.00488464258389983\\
16.01	0.00488464392577312\\
17.01	0.00488464529544633\\
18.01	0.00488464669349324\\
19.01	0.00488464812049973\\
20.01	0.00488464957706371\\
21.01	0.00488465106379535\\
22.01	0.00488465258131733\\
23.01	0.00488465413026523\\
24.01	0.00488465571128756\\
25.01	0.00488465732504612\\
26.01	0.00488465897221652\\
27.01	0.00488466065348817\\
28.01	0.00488466236956452\\
29.01	0.00488466412116355\\
30.01	0.00488466590901793\\
31.01	0.0048846677338755\\
32.01	0.00488466959649934\\
33.01	0.00488467149766832\\
34.01	0.00488467343817704\\
35.01	0.00488467541883672\\
36.01	0.00488467744047501\\
37.01	0.00488467950393669\\
38.01	0.00488468161008374\\
39.01	0.00488468375979582\\
40.01	0.00488468595397068\\
41.01	0.00488468819352453\\
42.01	0.00488469047939238\\
43.01	0.00488469281252815\\
44.01	0.00488469519390572\\
45.01	0.00488469762451858\\
46.01	0.00488470010538083\\
47.01	0.00488470263752724\\
48.01	0.00488470522201373\\
49.01	0.00488470785991796\\
50.01	0.00488471055233954\\
51.01	0.00488471330040076\\
52.01	0.00488471610524679\\
53.01	0.00488471896804608\\
54.01	0.00488472188999137\\
55.01	0.00488472487229946\\
56.01	0.00488472791621249\\
57.01	0.00488473102299765\\
58.01	0.00488473419394793\\
59.01	0.0048847374303831\\
60.01	0.00488474073364972\\
61.01	0.00488474410512205\\
62.01	0.00488474754620237\\
63.01	0.00488475105832172\\
64.01	0.00488475464294018\\
65.01	0.00488475830154779\\
66.01	0.00488476203566512\\
67.01	0.00488476584684377\\
68.01	0.00488476973666695\\
69.01	0.00488477370675029\\
70.01	0.00488477775874228\\
71.01	0.0048847818943253\\
72.01	0.00488478611521584\\
73.01	0.00488479042316561\\
74.01	0.00488479481996197\\
75.01	0.00488479930742881\\
76.01	0.00488480388742725\\
77.01	0.00488480856185623\\
78.01	0.00488481333265349\\
79.01	0.00488481820179654\\
80.01	0.00488482317130275\\
81.01	0.00488482824323097\\
82.01	0.00488483341968188\\
83.01	0.004884838702799\\
84.01	0.00488484409476954\\
85.01	0.00488484959782505\\
86.01	0.00488485521424283\\
87.01	0.00488486094634642\\
88.01	0.00488486679650672\\
89.01	0.00488487276714292\\
90.01	0.00488487886072335\\
91.01	0.00488488507976655\\
92.01	0.00488489142684233\\
93.01	0.00488489790457287\\
94.01	0.00488490451563364\\
95.01	0.00488491126275441\\
96.01	0.00488491814872058\\
97.01	0.004884925176374\\
98.01	0.00488493234861456\\
99.01	0.00488493966840066\\
100.01	0.00488494713875114\\
101.01	0.00488495476274613\\
102.01	0.00488496254352806\\
103.01	0.00488497048430354\\
104.01	0.00488497858834375\\
105.01	0.00488498685898661\\
106.01	0.0048849952996378\\
107.01	0.00488500391377188\\
108.01	0.00488501270493392\\
109.01	0.0048850216767411\\
110.01	0.00488503083288364\\
111.01	0.0048850401771266\\
112.01	0.0048850497133114\\
113.01	0.00488505944535733\\
114.01	0.00488506937726288\\
115.01	0.00488507951310754\\
116.01	0.00488508985705377\\
117.01	0.00488510041334787\\
118.01	0.00488511118632219\\
119.01	0.00488512218039647\\
120.01	0.00488513340008019\\
121.01	0.00488514484997393\\
122.01	0.00488515653477133\\
123.01	0.0048851684592606\\
124.01	0.00488518062832694\\
125.01	0.0048851930469542\\
126.01	0.00488520572022699\\
127.01	0.00488521865333234\\
128.01	0.00488523185156209\\
129.01	0.00488524532031505\\
130.01	0.00488525906509854\\
131.01	0.00488527309153144\\
132.01	0.00488528740534548\\
133.01	0.00488530201238821\\
134.01	0.00488531691862487\\
135.01	0.00488533213014113\\
136.01	0.00488534765314502\\
137.01	0.00488536349396942\\
138.01	0.00488537965907505\\
139.01	0.00488539615505265\\
140.01	0.00488541298862519\\
141.01	0.00488543016665109\\
142.01	0.00488544769612662\\
143.01	0.00488546558418873\\
144.01	0.00488548383811775\\
145.01	0.00488550246534004\\
146.01	0.00488552147343151\\
147.01	0.00488554087011993\\
148.01	0.00488556066328823\\
149.01	0.00488558086097736\\
150.01	0.00488560147138963\\
151.01	0.00488562250289179\\
152.01	0.00488564396401828\\
153.01	0.00488566586347448\\
154.01	0.00488568821014031\\
155.01	0.00488571101307322\\
156.01	0.00488573428151214\\
157.01	0.00488575802488071\\
158.01	0.00488578225279108\\
159.01	0.00488580697504766\\
160.01	0.00488583220165079\\
161.01	0.00488585794280036\\
162.01	0.00488588420890027\\
163.01	0.00488591101056195\\
164.01	0.0048859383586087\\
165.01	0.00488596626407946\\
166.01	0.0048859947382336\\
167.01	0.00488602379255464\\
168.01	0.00488605343875502\\
169.01	0.00488608368878034\\
170.01	0.00488611455481398\\
171.01	0.00488614604928185\\
172.01	0.00488617818485682\\
173.01	0.00488621097446391\\
174.01	0.00488624443128476\\
175.01	0.004886278568763\\
176.01	0.00488631340060888\\
177.01	0.00488634894080489\\
178.01	0.00488638520361084\\
179.01	0.00488642220356926\\
180.01	0.00488645995551076\\
181.01	0.00488649847455971\\
182.01	0.0048865377761399\\
183.01	0.00488657787597988\\
184.01	0.00488661879011984\\
185.01	0.00488666053491649\\
186.01	0.00488670312704981\\
187.01	0.00488674658352915\\
188.01	0.00488679092169943\\
189.01	0.00488683615924733\\
190.01	0.00488688231420836\\
191.01	0.00488692940497344\\
192.01	0.00488697745029516\\
193.01	0.00488702646929522\\
194.01	0.00488707648147119\\
195.01	0.00488712750670386\\
196.01	0.00488717956526429\\
197.01	0.00488723267782148\\
198.01	0.00488728686544965\\
199.01	0.00488734214963593\\
200.01	0.00488739855228837\\
201.01	0.00488745609574408\\
202.01	0.00488751480277642\\
203.01	0.00488757469660416\\
204.01	0.00488763580089938\\
205.01	0.00488769813979624\\
206.01	0.00488776173789938\\
207.01	0.00488782662029302\\
208.01	0.00488789281254931\\
209.01	0.00488796034073828\\
210.01	0.00488802923143664\\
211.01	0.00488809951173697\\
212.01	0.00488817120925797\\
213.01	0.00488824435215351\\
214.01	0.00488831896912283\\
215.01	0.00488839508942079\\
216.01	0.00488847274286774\\
217.01	0.00488855195986022\\
218.01	0.00488863277138125\\
219.01	0.00488871520901125\\
220.01	0.00488879930493913\\
221.01	0.00488888509197286\\
222.01	0.00488897260355128\\
223.01	0.00488906187375519\\
224.01	0.00488915293731899\\
225.01	0.00488924582964294\\
226.01	0.00488934058680437\\
227.01	0.00488943724557078\\
228.01	0.0048895358434112\\
229.01	0.00488963641850972\\
230.01	0.00488973900977757\\
231.01	0.00488984365686602\\
232.01	0.00488995040018013\\
233.01	0.00489005928089155\\
234.01	0.0048901703409521\\
235.01	0.00489028362310753\\
236.01	0.00489039917091151\\
237.01	0.00489051702873944\\
238.01	0.00489063724180322\\
239.01	0.00489075985616554\\
240.01	0.00489088491875445\\
241.01	0.00489101247737807\\
242.01	0.00489114258074019\\
243.01	0.0048912752784554\\
244.01	0.00489141062106455\\
245.01	0.00489154866005015\\
246.01	0.00489168944785268\\
247.01	0.00489183303788655\\
248.01	0.00489197948455636\\
249.01	0.00489212884327332\\
250.01	0.00489228117047229\\
251.01	0.00489243652362804\\
252.01	0.00489259496127287\\
253.01	0.00489275654301351\\
254.01	0.00489292132954898\\
255.01	0.00489308938268763\\
256.01	0.0048932607653657\\
257.01	0.00489343554166455\\
258.01	0.00489361377682952\\
259.01	0.00489379553728768\\
260.01	0.00489398089066691\\
261.01	0.00489416990581426\\
262.01	0.00489436265281511\\
263.01	0.00489455920301173\\
264.01	0.00489475962902271\\
265.01	0.00489496400476247\\
266.01	0.00489517240546028\\
267.01	0.00489538490768029\\
268.01	0.00489560158934112\\
269.01	0.00489582252973529\\
270.01	0.00489604780955007\\
271.01	0.00489627751088685\\
272.01	0.00489651171728148\\
273.01	0.00489675051372462\\
274.01	0.00489699398668199\\
275.01	0.00489724222411494\\
276.01	0.00489749531550084\\
277.01	0.00489775335185344\\
278.01	0.00489801642574396\\
279.01	0.00489828463132136\\
280.01	0.00489855806433266\\
281.01	0.00489883682214419\\
282.01	0.0048991210037619\\
283.01	0.00489941070985198\\
284.01	0.00489970604276144\\
285.01	0.00490000710653865\\
286.01	0.00490031400695382\\
287.01	0.00490062685151932\\
288.01	0.00490094574951033\\
289.01	0.00490127081198434\\
290.01	0.00490160215180185\\
291.01	0.00490193988364593\\
292.01	0.0049022841240422\\
293.01	0.00490263499137808\\
294.01	0.00490299260592256\\
295.01	0.00490335708984539\\
296.01	0.00490372856723544\\
297.01	0.00490410716412007\\
298.01	0.00490449300848338\\
299.01	0.00490488623028426\\
300.01	0.00490528696147452\\
301.01	0.00490569533601581\\
302.01	0.00490611148989772\\
303.01	0.00490653556115422\\
304.01	0.00490696768988012\\
305.01	0.00490740801824727\\
306.01	0.00490785669052059\\
307.01	0.00490831385307283\\
308.01	0.00490877965440047\\
309.01	0.00490925424513745\\
310.01	0.00490973777806981\\
311.01	0.00491023040814934\\
312.01	0.00491073229250645\\
313.01	0.00491124359046391\\
314.01	0.00491176446354842\\
315.01	0.00491229507550346\\
316.01	0.00491283559230044\\
317.01	0.00491338618215047\\
318.01	0.00491394701551483\\
319.01	0.00491451826511623\\
320.01	0.00491510010594903\\
321.01	0.00491569271528909\\
322.01	0.0049162962727043\\
323.01	0.0049169109600639\\
324.01	0.00491753696154876\\
325.01	0.0049181744636609\\
326.01	0.00491882365523369\\
327.01	0.00491948472744183\\
328.01	0.00492015787381168\\
329.01	0.00492084329023253\\
330.01	0.00492154117496778\\
331.01	0.00492225172866764\\
332.01	0.00492297515438152\\
333.01	0.00492371165757287\\
334.01	0.00492446144613423\\
335.01	0.00492522473040473\\
336.01	0.00492600172318833\\
337.01	0.00492679263977508\\
338.01	0.0049275976979638\\
339.01	0.00492841711808818\\
340.01	0.00492925112304513\\
341.01	0.0049300999383268\\
342.01	0.00493096379205611\\
343.01	0.00493184291502724\\
344.01	0.00493273754074887\\
345.01	0.00493364790549432\\
346.01	0.00493457424835596\\
347.01	0.0049355168113058\\
348.01	0.00493647583926354\\
349.01	0.00493745158017071\\
350.01	0.00493844428507326\\
351.01	0.0049394542082119\\
352.01	0.004940481607122\\
353.01	0.00494152674274304\\
354.01	0.00494258987953809\\
355.01	0.00494367128562485\\
356.01	0.00494477123291819\\
357.01	0.00494588999728481\\
358.01	0.00494702785871175\\
359.01	0.00494818510148777\\
360.01	0.0049493620143998\\
361.01	0.00495055889094456\\
362.01	0.0049517760295547\\
363.01	0.00495301373384249\\
364.01	0.00495427231285812\\
365.01	0.00495555208136646\\
366.01	0.00495685336013935\\
367.01	0.0049581764762656\\
368.01	0.0049595217634773\\
369.01	0.00496088956249255\\
370.01	0.00496228022137377\\
371.01	0.0049636940959015\\
372.01	0.00496513154996125\\
373.01	0.00496659295594318\\
374.01	0.00496807869515233\\
375.01	0.00496958915822666\\
376.01	0.00497112474556143\\
377.01	0.00497268586773587\\
378.01	0.00497427294593912\\
379.01	0.00497588641239095\\
380.01	0.00497752671075352\\
381.01	0.00497919429652868\\
382.01	0.004980889637435\\
383.01	0.00498261321375904\\
384.01	0.00498436551867423\\
385.01	0.00498614705852136\\
386.01	0.00498795835304229\\
387.01	0.00498979993556056\\
388.01	0.00499167235310256\\
389.01	0.00499357616645112\\
390.01	0.00499551195012558\\
391.01	0.00499748029228411\\
392.01	0.00499948179454097\\
393.01	0.0050015170716996\\
394.01	0.0050035867513961\\
395.01	0.00500569147365945\\
396.01	0.00500783189038718\\
397.01	0.00501000866474974\\
398.01	0.00501222247052978\\
399.01	0.0050144739914159\\
400.01	0.00501676392026873\\
401.01	0.00501909295838479\\
402.01	0.00502146181478867\\
403.01	0.0050238712055853\\
404.01	0.00502632185341148\\
405.01	0.00502881448702619\\
406.01	0.00503134984107985\\
407.01	0.00503392865610204\\
408.01	0.00503655167874308\\
409.01	0.00503921966229595\\
410.01	0.00504193336751546\\
411.01	0.00504469356373378\\
412.01	0.00504750103025557\\
413.01	0.00505035655799117\\
414.01	0.0050532609512678\\
415.01	0.0050562150297354\\
416.01	0.005059219630274\\
417.01	0.00506227560880644\\
418.01	0.0050653838419379\\
419.01	0.00506854522838324\\
420.01	0.00507176069020351\\
421.01	0.00507503117393805\\
422.01	0.00507835765173228\\
423.01	0.00508174112249618\\
424.01	0.00508518261309696\\
425.01	0.00508868317958147\\
426.01	0.00509224390842564\\
427.01	0.0050958659178061\\
428.01	0.00509955035888979\\
429.01	0.00510329841713758\\
430.01	0.00510711131361349\\
431.01	0.00511099030629548\\
432.01	0.00511493669137993\\
433.01	0.00511895180457242\\
434.01	0.00512303702235845\\
435.01	0.00512719376324269\\
436.01	0.0051314234889528\\
437.01	0.00513572770559581\\
438.01	0.00514010796476017\\
439.01	0.00514456586455563\\
440.01	0.00514910305058303\\
441.01	0.00515372121682685\\
442.01	0.00515842210646476\\
443.01	0.00516320751259082\\
444.01	0.00516807927884735\\
445.01	0.00517303929996674\\
446.01	0.00517808952222386\\
447.01	0.0051832319438026\\
448.01	0.00518846861508475\\
449.01	0.00519380163887069\\
450.01	0.00519923317054667\\
451.01	0.00520476541821373\\
452.01	0.0052104006428003\\
453.01	0.00521614115818127\\
454.01	0.00522198933132952\\
455.01	0.00522794758252718\\
456.01	0.0052340183856666\\
457.01	0.00524020426866761\\
458.01	0.00524650781403839\\
459.01	0.00525293165960437\\
460.01	0.00525947849941792\\
461.01	0.0052661510848642\\
462.01	0.00527295222595822\\
463.01	0.00527988479282504\\
464.01	0.00528695171733751\\
465.01	0.00529415599487579\\
466.01	0.00530150068616073\\
467.01	0.00530898891910096\\
468.01	0.00531662389059037\\
469.01	0.00532440886819223\\
470.01	0.00533234719165125\\
471.01	0.00534044227419347\\
472.01	0.00534869760359381\\
473.01	0.00535711674302511\\
474.01	0.00536570333172771\\
475.01	0.00537446108556734\\
476.01	0.00538339379755706\\
477.01	0.0053925053384032\\
478.01	0.00540179965711304\\
479.01	0.00541128078168127\\
480.01	0.00542095281987184\\
481.01	0.0054308199601092\\
482.01	0.00544088647248854\\
483.01	0.00545115670991292\\
484.01	0.00546163510936148\\
485.01	0.00547232619328618\\
486.01	0.00548323457113223\\
487.01	0.00549436494097295\\
488.01	0.00550572209124164\\
489.01	0.00551731090254502\\
490.01	0.00552913634953665\\
491.01	0.00554120350283076\\
492.01	0.00555351753093536\\
493.01	0.00556608370219074\\
494.01	0.00557890738670204\\
495.01	0.00559199405826316\\
496.01	0.00560534929627592\\
497.01	0.00561897878767768\\
498.01	0.0056328883288938\\
499.01	0.00564708382783537\\
500.01	0.00566157130595962\\
501.01	0.00567635690040588\\
502.01	0.00569144686620832\\
503.01	0.00570684757858236\\
504.01	0.00572256553527355\\
505.01	0.00573860735895959\\
506.01	0.00575497979969389\\
507.01	0.00577168973737842\\
508.01	0.00578874418425757\\
509.01	0.00580615028742262\\
510.01	0.00582391533132044\\
511.01	0.00584204674026063\\
512.01	0.00586055208091783\\
513.01	0.00587943906482583\\
514.01	0.00589871555086248\\
515.01	0.00591838954772095\\
516.01	0.00593846921636589\\
517.01	0.00595896287246597\\
518.01	0.00597987898879571\\
519.01	0.00600122619759545\\
520.01	0.00602301329287536\\
521.01	0.00604524923265283\\
522.01	0.0060679431411064\\
523.01	0.00609110431063371\\
524.01	0.0061147422037966\\
525.01	0.00613886645513884\\
526.01	0.00616348687285774\\
527.01	0.00618861344031034\\
528.01	0.0062142563173349\\
529.01	0.00624042584136023\\
530.01	0.00626713252828011\\
531.01	0.00629438707306058\\
532.01	0.00632220035004875\\
533.01	0.00635058341294649\\
534.01	0.00637954749440924\\
535.01	0.00640910400522809\\
536.01	0.00643926453304539\\
537.01	0.00647004084055382\\
538.01	0.00650144486311954\\
539.01	0.00653348870576834\\
540.01	0.00656618463946283\\
541.01	0.00659954509659628\\
542.01	0.00663358266561771\\
543.01	0.00666831008469636\\
544.01	0.00670374023432455\\
545.01	0.00673988612874671\\
546.01	0.00677676090609454\\
547.01	0.00681437781709397\\
548.01	0.00685275021219706\\
549.01	0.00689189152698148\\
550.01	0.00693181526564102\\
551.01	0.00697253498237767\\
552.01	0.0070140642604869\\
553.01	0.00705641668891045\\
554.01	0.00709960583601042\\
555.01	0.00714364522029591\\
556.01	0.00718854827781448\\
557.01	0.00723432832589305\\
558.01	0.00728099852289051\\
559.01	0.00732857182359682\\
560.01	0.007377060929888\\
561.01	0.00742647823621727\\
562.01	0.00747683576949768\\
563.01	0.00752814512290237\\
564.01	0.00758041738308592\\
565.01	0.00763366305030459\\
566.01	0.00768789195089741\\
567.01	0.00774311314157326\\
568.01	0.00779933480494592\\
569.01	0.00785656413576291\\
570.01	0.00791480721729327\\
571.01	0.0079740688873769\\
572.01	0.00803435259370153\\
573.01	0.008095660237966\\
574.01	0.008157992008723\\
575.01	0.00822134620288103\\
576.01	0.00828571903609403\\
577.01	0.00835110444260261\\
578.01	0.00841749386552518\\
579.01	0.00848487603916375\\
580.01	0.00855323676561756\\
581.01	0.00862255868892708\\
582.01	0.0086928210711525\\
583.01	0.00876399957628497\\
584.01	0.00883606606977153\\
585.01	0.00890898844379669\\
586.01	0.00898273048142365\\
587.01	0.00905725177639212\\
588.01	0.00913250772997538\\
589.01	0.0092084496520285\\
590.01	0.00928502500047535\\
591.01	0.00936217780230859\\
592.01	0.00943984931011133\\
593.01	0.0095179789616418\\
594.01	0.00959650572675306\\
595.01	0.00967536994658929\\
596.01	0.00975451579551667\\
597.01	0.00983362572953704\\
598.01	0.00990826330299362\\
599.01	0.00997053306357289\\
599.02	0.00997104257304143\\
599.03	0.00997154904691472\\
599.04	0.00997205245565375\\
599.05	0.00997255276942872\\
599.06	0.00997304995811617\\
599.07	0.00997354399129606\\
599.08	0.00997403483824883\\
599.09	0.00997452246795248\\
599.1	0.00997500684907952\\
599.11	0.00997548794999398\\
599.12	0.00997596573874838\\
599.13	0.0099764401830806\\
599.14	0.0099769112504108\\
599.15	0.00997737890783826\\
599.16	0.00997784312213823\\
599.17	0.0099783038597587\\
599.18	0.00997876108681718\\
599.19	0.00997921476909742\\
599.2	0.00997966487204611\\
599.21	0.00998011136076955\\
599.22	0.00998055420003028\\
599.23	0.00998099335424369\\
599.24	0.00998142878747458\\
599.25	0.00998186046343367\\
599.26	0.00998228834359041\\
599.27	0.00998271238778643\\
599.28	0.0099831325554638\\
599.29	0.00998354880566111\\
599.3	0.00998396109700947\\
599.31	0.00998436938772849\\
599.32	0.00998477363562221\\
599.33	0.00998517379807499\\
599.34	0.00998556983204737\\
599.35	0.00998596169407188\\
599.36	0.00998634934024881\\
599.37	0.00998673272624192\\
599.38	0.00998711180727415\\
599.39	0.00998748653812326\\
599.4	0.00998785687311743\\
599.41	0.00998822276613081\\
599.42	0.00998858417057903\\
599.43	0.00998894103941468\\
599.44	0.00998929332512275\\
599.45	0.00998964097971596\\
599.46	0.00998998395473014\\
599.47	0.00999032220121951\\
599.48	0.0099906556697519\\
599.49	0.00999098431040396\\
599.5	0.00999130807275631\\
599.51	0.00999162690588863\\
599.52	0.00999194075837471\\
599.53	0.00999224957827746\\
599.54	0.00999255331314387\\
599.55	0.00999285190999987\\
599.56	0.00999314531534524\\
599.57	0.00999343347514839\\
599.58	0.00999371633484107\\
599.59	0.00999399383931314\\
599.6	0.00999426593290715\\
599.61	0.009994532559413\\
599.62	0.0099947936620624\\
599.63	0.00999504918352342\\
599.64	0.00999529906589491\\
599.65	0.00999554325070086\\
599.66	0.00999578167888471\\
599.67	0.00999601429080366\\
599.68	0.00999624102622283\\
599.69	0.00999646182430947\\
599.7	0.00999667662362697\\
599.71	0.00999688536212897\\
599.72	0.00999708797715331\\
599.73	0.00999728440541596\\
599.74	0.00999747458300482\\
599.75	0.0099976584453736\\
599.76	0.00999783592733551\\
599.77	0.00999800696305692\\
599.78	0.00999817148605102\\
599.79	0.00999832942917133\\
599.8	0.00999848072460517\\
599.81	0.00999862530386713\\
599.82	0.00999876309779239\\
599.83	0.00999889403653003\\
599.84	0.00999901804953622\\
599.85	0.00999913506556741\\
599.86	0.00999924501267341\\
599.87	0.00999934781819042\\
599.88	0.00999944340873394\\
599.89	0.0099995317101917\\
599.9	0.00999961264771648\\
599.91	0.00999968614571878\\
599.92	0.00999975212785957\\
599.93	0.00999981051704285\\
599.94	0.00999986123540817\\
599.95	0.00999990420432308\\
599.96	0.00999993934437554\\
599.97	0.00999996657536618\\
599.98	0.00999998581630055\\
599.99	0.00999999698538124\\
600	0.01\\
};
\addplot [color=black!60!mycolor21,solid,forget plot]
  table[row sep=crcr]{%
0.01	0.00491031510145137\\
1.01	0.00491031607159087\\
2.01	0.00491031706172967\\
3.01	0.00491031807227832\\
4.01	0.00491031910365577\\
5.01	0.00491032015628892\\
6.01	0.004910321230614\\
7.01	0.004910322327076\\
8.01	0.00491032344612882\\
9.01	0.00491032458823537\\
10.01	0.00491032575386857\\
11.01	0.00491032694351048\\
12.01	0.0049103281576533\\
13.01	0.00491032939679925\\
14.01	0.00491033066146054\\
15.01	0.00491033195215984\\
16.01	0.00491033326943088\\
17.01	0.00491033461381757\\
18.01	0.0049103359858754\\
19.01	0.00491033738617094\\
20.01	0.00491033881528236\\
21.01	0.00491034027379961\\
22.01	0.00491034176232466\\
23.01	0.0049103432814715\\
24.01	0.00491034483186684\\
25.01	0.00491034641415014\\
26.01	0.00491034802897369\\
27.01	0.00491034967700305\\
28.01	0.00491035135891746\\
29.01	0.00491035307540957\\
30.01	0.0049103548271868\\
31.01	0.00491035661497022\\
32.01	0.00491035843949589\\
33.01	0.00491036030151454\\
34.01	0.00491036220179267\\
35.01	0.00491036414111198\\
36.01	0.00491036612026965\\
37.01	0.00491036814007967\\
38.01	0.00491037020137211\\
39.01	0.00491037230499405\\
40.01	0.0049103744518097\\
41.01	0.0049103766427008\\
42.01	0.00491037887856721\\
43.01	0.00491038116032647\\
44.01	0.0049103834889149\\
45.01	0.00491038586528794\\
46.01	0.00491038829042034\\
47.01	0.00491039076530628\\
48.01	0.00491039329096052\\
49.01	0.00491039586841802\\
50.01	0.00491039849873481\\
51.01	0.00491040118298822\\
52.01	0.00491040392227707\\
53.01	0.00491040671772291\\
54.01	0.00491040957046965\\
55.01	0.00491041248168456\\
56.01	0.00491041545255808\\
57.01	0.0049104184843049\\
58.01	0.00491042157816446\\
59.01	0.00491042473540079\\
60.01	0.00491042795730384\\
61.01	0.00491043124518914\\
62.01	0.00491043460039894\\
63.01	0.00491043802430274\\
64.01	0.00491044151829753\\
65.01	0.00491044508380858\\
66.01	0.00491044872228966\\
67.01	0.00491045243522376\\
68.01	0.00491045622412378\\
69.01	0.00491046009053339\\
70.01	0.0049104640360273\\
71.01	0.00491046806221143\\
72.01	0.00491047217072429\\
73.01	0.0049104763632376\\
74.01	0.0049104806414566\\
75.01	0.0049104850071206\\
76.01	0.00491048946200376\\
77.01	0.00491049400791638\\
78.01	0.00491049864670459\\
79.01	0.00491050338025184\\
80.01	0.00491050821047984\\
81.01	0.00491051313934815\\
82.01	0.00491051816885599\\
83.01	0.00491052330104297\\
84.01	0.00491052853798876\\
85.01	0.00491053388181567\\
86.01	0.00491053933468807\\
87.01	0.00491054489881396\\
88.01	0.00491055057644535\\
89.01	0.00491055636987944\\
90.01	0.00491056228145914\\
91.01	0.00491056831357481\\
92.01	0.00491057446866447\\
93.01	0.00491058074921474\\
94.01	0.00491058715776202\\
95.01	0.00491059369689356\\
96.01	0.00491060036924834\\
97.01	0.0049106071775179\\
98.01	0.00491061412444754\\
99.01	0.00491062121283775\\
100.01	0.00491062844554466\\
101.01	0.00491063582548121\\
102.01	0.00491064335561875\\
103.01	0.0049106510389878\\
104.01	0.00491065887867959\\
105.01	0.00491066687784653\\
106.01	0.00491067503970385\\
107.01	0.00491068336753123\\
108.01	0.00491069186467374\\
109.01	0.00491070053454246\\
110.01	0.00491070938061638\\
111.01	0.00491071840644464\\
112.01	0.00491072761564595\\
113.01	0.0049107370119115\\
114.01	0.00491074659900581\\
115.01	0.00491075638076785\\
116.01	0.00491076636111287\\
117.01	0.00491077654403418\\
118.01	0.0049107869336045\\
119.01	0.00491079753397677\\
120.01	0.0049108083493869\\
121.01	0.0049108193841543\\
122.01	0.00491083064268422\\
123.01	0.00491084212946899\\
124.01	0.00491085384909024\\
125.01	0.00491086580622008\\
126.01	0.00491087800562304\\
127.01	0.00491089045215781\\
128.01	0.00491090315077972\\
129.01	0.00491091610654132\\
130.01	0.00491092932459558\\
131.01	0.00491094281019688\\
132.01	0.00491095656870355\\
133.01	0.00491097060557918\\
134.01	0.00491098492639591\\
135.01	0.0049109995368353\\
136.01	0.00491101444269074\\
137.01	0.0049110296498705\\
138.01	0.00491104516439833\\
139.01	0.00491106099241695\\
140.01	0.00491107714018996\\
141.01	0.00491109361410449\\
142.01	0.00491111042067266\\
143.01	0.00491112756653495\\
144.01	0.00491114505846163\\
145.01	0.00491116290335708\\
146.01	0.00491118110826022\\
147.01	0.00491119968034831\\
148.01	0.00491121862693927\\
149.01	0.00491123795549456\\
150.01	0.00491125767362191\\
151.01	0.00491127778907771\\
152.01	0.00491129830977041\\
153.01	0.00491131924376292\\
154.01	0.00491134059927609\\
155.01	0.00491136238469103\\
156.01	0.00491138460855275\\
157.01	0.00491140727957315\\
158.01	0.00491143040663408\\
159.01	0.00491145399879039\\
160.01	0.00491147806527351\\
161.01	0.00491150261549497\\
162.01	0.00491152765904926\\
163.01	0.00491155320571765\\
164.01	0.00491157926547167\\
165.01	0.00491160584847666\\
166.01	0.00491163296509542\\
167.01	0.00491166062589185\\
168.01	0.00491168884163526\\
169.01	0.0049117176233032\\
170.01	0.00491174698208674\\
171.01	0.0049117769293931\\
172.01	0.00491180747685066\\
173.01	0.00491183863631256\\
174.01	0.00491187041986135\\
175.01	0.00491190283981335\\
176.01	0.00491193590872194\\
177.01	0.00491196963938337\\
178.01	0.00491200404484085\\
179.01	0.00491203913838839\\
180.01	0.00491207493357659\\
181.01	0.00491211144421663\\
182.01	0.00491214868438547\\
183.01	0.0049121866684308\\
184.01	0.00491222541097569\\
185.01	0.00491226492692406\\
186.01	0.0049123052314658\\
187.01	0.00491234634008211\\
188.01	0.00491238826855028\\
189.01	0.00491243103295011\\
190.01	0.00491247464966887\\
191.01	0.0049125191354068\\
192.01	0.00491256450718325\\
193.01	0.00491261078234234\\
194.01	0.00491265797855914\\
195.01	0.00491270611384517\\
196.01	0.00491275520655504\\
197.01	0.00491280527539256\\
198.01	0.00491285633941707\\
199.01	0.00491290841804976\\
200.01	0.00491296153108048\\
201.01	0.00491301569867408\\
202.01	0.0049130709413778\\
203.01	0.00491312728012738\\
204.01	0.00491318473625466\\
205.01	0.00491324333149432\\
206.01	0.00491330308799115\\
207.01	0.00491336402830791\\
208.01	0.00491342617543236\\
209.01	0.00491348955278451\\
210.01	0.00491355418422532\\
211.01	0.00491362009406349\\
212.01	0.00491368730706396\\
213.01	0.0049137558484561\\
214.01	0.0049138257439415\\
215.01	0.0049138970197028\\
216.01	0.00491396970241173\\
217.01	0.00491404381923788\\
218.01	0.00491411939785774\\
219.01	0.00491419646646279\\
220.01	0.00491427505376946\\
221.01	0.00491435518902755\\
222.01	0.00491443690202978\\
223.01	0.00491452022312133\\
224.01	0.00491460518320924\\
225.01	0.00491469181377202\\
226.01	0.00491478014686977\\
227.01	0.00491487021515369\\
228.01	0.00491496205187711\\
229.01	0.00491505569090489\\
230.01	0.00491515116672395\\
231.01	0.00491524851445427\\
232.01	0.00491534776985988\\
233.01	0.00491544896935836\\
234.01	0.00491555215003323\\
235.01	0.00491565734964514\\
236.01	0.00491576460664206\\
237.01	0.00491587396017187\\
238.01	0.0049159854500935\\
239.01	0.00491609911698887\\
240.01	0.00491621500217457\\
241.01	0.00491633314771489\\
242.01	0.00491645359643308\\
243.01	0.00491657639192389\\
244.01	0.00491670157856686\\
245.01	0.00491682920153858\\
246.01	0.00491695930682534\\
247.01	0.00491709194123705\\
248.01	0.0049172271524198\\
249.01	0.00491736498886938\\
250.01	0.00491750549994455\\
251.01	0.00491764873588155\\
252.01	0.00491779474780762\\
253.01	0.00491794358775449\\
254.01	0.00491809530867326\\
255.01	0.00491824996444892\\
256.01	0.00491840760991388\\
257.01	0.00491856830086376\\
258.01	0.00491873209407118\\
259.01	0.004918899047302\\
260.01	0.00491906921932911\\
261.01	0.00491924266994892\\
262.01	0.00491941945999599\\
263.01	0.00491959965135927\\
264.01	0.00491978330699741\\
265.01	0.00491997049095511\\
266.01	0.00492016126837869\\
267.01	0.00492035570553284\\
268.01	0.00492055386981688\\
269.01	0.00492075582978119\\
270.01	0.00492096165514341\\
271.01	0.00492117141680656\\
272.01	0.00492138518687485\\
273.01	0.00492160303867145\\
274.01	0.00492182504675528\\
275.01	0.00492205128693896\\
276.01	0.00492228183630566\\
277.01	0.00492251677322811\\
278.01	0.00492275617738483\\
279.01	0.00492300012977911\\
280.01	0.00492324871275682\\
281.01	0.00492350201002506\\
282.01	0.00492376010667026\\
283.01	0.00492402308917672\\
284.01	0.00492429104544514\\
285.01	0.00492456406481206\\
286.01	0.00492484223806858\\
287.01	0.00492512565747927\\
288.01	0.00492541441680122\\
289.01	0.00492570861130454\\
290.01	0.00492600833779094\\
291.01	0.00492631369461391\\
292.01	0.00492662478169853\\
293.01	0.00492694170056152\\
294.01	0.0049272645543319\\
295.01	0.00492759344777097\\
296.01	0.00492792848729334\\
297.01	0.00492826978098768\\
298.01	0.0049286174386374\\
299.01	0.00492897157174266\\
300.01	0.00492933229354131\\
301.01	0.00492969971903126\\
302.01	0.00493007396499201\\
303.01	0.00493045515000734\\
304.01	0.00493084339448832\\
305.01	0.00493123882069521\\
306.01	0.0049316415527626\\
307.01	0.00493205171672173\\
308.01	0.00493246944052533\\
309.01	0.00493289485407258\\
310.01	0.00493332808923319\\
311.01	0.00493376927987479\\
312.01	0.00493421856188775\\
313.01	0.00493467607321296\\
314.01	0.00493514195386935\\
315.01	0.00493561634598177\\
316.01	0.00493609939381048\\
317.01	0.00493659124378132\\
318.01	0.00493709204451662\\
319.01	0.0049376019468666\\
320.01	0.00493812110394312\\
321.01	0.0049386496711529\\
322.01	0.00493918780623445\\
323.01	0.00493973566929337\\
324.01	0.00494029342284161\\
325.01	0.00494086123183635\\
326.01	0.00494143926372242\\
327.01	0.00494202768847507\\
328.01	0.00494262667864553\\
329.01	0.00494323640940805\\
330.01	0.00494385705860945\\
331.01	0.00494448880682036\\
332.01	0.00494513183739081\\
333.01	0.00494578633650644\\
334.01	0.00494645249324882\\
335.01	0.00494713049965773\\
336.01	0.00494782055079787\\
337.01	0.00494852284482808\\
338.01	0.0049492375830745\\
339.01	0.00494996497010755\\
340.01	0.00495070521382246\\
341.01	0.00495145852552478\\
342.01	0.00495222512001933\\
343.01	0.00495300521570466\\
344.01	0.00495379903467205\\
345.01	0.00495460680280871\\
346.01	0.0049554287499079\\
347.01	0.00495626510978264\\
348.01	0.00495711612038646\\
349.01	0.00495798202393943\\
350.01	0.00495886306705925\\
351.01	0.00495975950090055\\
352.01	0.00496067158129826\\
353.01	0.00496159956891859\\
354.01	0.00496254372941649\\
355.01	0.00496350433359866\\
356.01	0.00496448165759408\\
357.01	0.00496547598303038\\
358.01	0.00496648759721674\\
359.01	0.0049675167933336\\
360.01	0.00496856387062783\\
361.01	0.00496962913461405\\
362.01	0.00497071289728146\\
363.01	0.00497181547730605\\
364.01	0.00497293720026756\\
365.01	0.00497407839887118\\
366.01	0.0049752394131716\\
367.01	0.00497642059080092\\
368.01	0.0049776222871997\\
369.01	0.00497884486584635\\
370.01	0.00498008869849034\\
371.01	0.00498135416538278\\
372.01	0.00498264165550477\\
373.01	0.00498395156679385\\
374.01	0.00498528430636566\\
375.01	0.00498664029072932\\
376.01	0.00498801994599571\\
377.01	0.0049894237080782\\
378.01	0.00499085202288227\\
379.01	0.00499230534648409\\
380.01	0.00499378414529664\\
381.01	0.00499528889622119\\
382.01	0.00499682008678526\\
383.01	0.00499837821526289\\
384.01	0.00499996379077947\\
385.01	0.00500157733340012\\
386.01	0.00500321937420012\\
387.01	0.00500489045532101\\
388.01	0.00500659113001053\\
389.01	0.00500832196264892\\
390.01	0.00501008352876542\\
391.01	0.0050118764150463\\
392.01	0.00501370121933842\\
393.01	0.00501555855065305\\
394.01	0.00501744902917711\\
395.01	0.00501937328629451\\
396.01	0.00502133196462709\\
397.01	0.00502332571810254\\
398.01	0.00502535521205555\\
399.01	0.00502742112337201\\
400.01	0.00502952414068444\\
401.01	0.00503166496462518\\
402.01	0.00503384430814549\\
403.01	0.00503606289690747\\
404.01	0.00503832146975167\\
405.01	0.00504062077924375\\
406.01	0.00504296159230044\\
407.01	0.00504534469089246\\
408.01	0.00504777087281655\\
409.01	0.00505024095252996\\
410.01	0.00505275576203379\\
411.01	0.00505531615179194\\
412.01	0.00505792299166824\\
413.01	0.00506057717186573\\
414.01	0.0050632796038543\\
415.01	0.00506603122127117\\
416.01	0.00506883298078781\\
417.01	0.00507168586294349\\
418.01	0.00507459087294607\\
419.01	0.00507754904145216\\
420.01	0.0050805614253398\\
421.01	0.0050836291084804\\
422.01	0.00508675320251795\\
423.01	0.00508993484765241\\
424.01	0.00509317521342523\\
425.01	0.00509647549950665\\
426.01	0.0050998369364815\\
427.01	0.00510326078663176\\
428.01	0.00510674834471328\\
429.01	0.00511030093872443\\
430.01	0.00511391993066553\\
431.01	0.00511760671728603\\
432.01	0.00512136273081784\\
433.01	0.00512518943969296\\
434.01	0.00512908834924385\\
435.01	0.00513306100238582\\
436.01	0.00513710898028007\\
437.01	0.00514123390297673\\
438.01	0.00514543743003947\\
439.01	0.00514972126114947\\
440.01	0.00515408713669316\\
441.01	0.00515853683833289\\
442.01	0.00516307218956499\\
443.01	0.00516769505626638\\
444.01	0.00517240734723573\\
445.01	0.00517721101473299\\
446.01	0.00518210805502062\\
447.01	0.00518710050891603\\
448.01	0.00519219046235775\\
449.01	0.00519738004699438\\
450.01	0.0052026714408009\\
451.01	0.00520806686873124\\
452.01	0.00521356860341134\\
453.01	0.00521917896587916\\
454.01	0.00522490032637621\\
455.01	0.00523073510519395\\
456.01	0.00523668577357699\\
457.01	0.00524275485468374\\
458.01	0.00524894492460183\\
459.01	0.00525525861341437\\
460.01	0.005261698606313\\
461.01	0.00526826764474517\\
462.01	0.00527496852759134\\
463.01	0.00528180411235646\\
464.01	0.00528877731636648\\
465.01	0.00529589111795879\\
466.01	0.00530314855765453\\
467.01	0.00531055273930738\\
468.01	0.00531810683122312\\
469.01	0.00532581406724998\\
470.01	0.00533367774784416\\
471.01	0.00534170124111938\\
472.01	0.00534988798389208\\
473.01	0.00535824148273772\\
474.01	0.0053667653150736\\
475.01	0.00537546313027987\\
476.01	0.00538433865087064\\
477.01	0.00539339567371967\\
478.01	0.00540263807134377\\
479.01	0.0054120697932465\\
480.01	0.00542169486732282\\
481.01	0.00543151740132276\\
482.01	0.00544154158437429\\
483.01	0.00545177168856248\\
484.01	0.00546221207056212\\
485.01	0.00547286717331909\\
486.01	0.00548374152777777\\
487.01	0.00549483975464801\\
488.01	0.00550616656620974\\
489.01	0.00551772676815016\\
490.01	0.00552952526143159\\
491.01	0.00554156704418657\\
492.01	0.00555385721364253\\
493.01	0.00556640096807356\\
494.01	0.00557920360878396\\
495.01	0.00559227054212341\\
496.01	0.00560560728153912\\
497.01	0.00561921944966617\\
498.01	0.00563311278045858\\
499.01	0.00564729312136132\\
500.01	0.00566176643552345\\
501.01	0.00567653880404938\\
502.01	0.0056916164282855\\
503.01	0.00570700563213769\\
504.01	0.0057227128644179\\
505.01	0.00573874470121358\\
506.01	0.00575510784827761\\
507.01	0.00577180914343571\\
508.01	0.00578885555900677\\
509.01	0.00580625420423381\\
510.01	0.00582401232772269\\
511.01	0.0058421373198836\\
512.01	0.00586063671537361\\
513.01	0.00587951819553455\\
514.01	0.00589878959082177\\
515.01	0.00591845888321874\\
516.01	0.00593853420862921\\
517.01	0.00595902385924149\\
518.01	0.0059799362858544\\
519.01	0.00600128010015649\\
520.01	0.00602306407694934\\
521.01	0.00604529715630196\\
522.01	0.00606798844562566\\
523.01	0.00609114722165505\\
524.01	0.00611478293232257\\
525.01	0.00613890519850736\\
526.01	0.00616352381564304\\
527.01	0.00618864875516505\\
528.01	0.00621429016577262\\
529.01	0.00624045837448448\\
530.01	0.00626716388746024\\
531.01	0.00629441739055772\\
532.01	0.00632222974959469\\
533.01	0.00635061201027849\\
534.01	0.00637957539776493\\
535.01	0.00640913131580215\\
536.01	0.00643929134541321\\
537.01	0.00647006724306409\\
538.01	0.00650147093825934\\
539.01	0.00653351453050154\\
540.01	0.00656621028554653\\
541.01	0.00659957063087552\\
542.01	0.00663360815030105\\
543.01	0.00666833557761387\\
544.01	0.00670376578917004\\
545.01	0.00673991179530625\\
546.01	0.00677678673046123\\
547.01	0.00681440384187\\
548.01	0.00685277647668575\\
549.01	0.00689191806736764\\
550.01	0.00693184211516124\\
551.01	0.00697256217148033\\
552.01	0.00701409181698227\\
553.01	0.00705644463811021\\
554.01	0.00709963420085635\\
555.01	0.00714367402147857\\
556.01	0.00718857753387962\\
557.01	0.00723435805333637\\
558.01	0.00728102873623945\\
559.01	0.00732860253547936\\
560.01	0.0073770921510864\\
561.01	0.00742650997570748\\
562.01	0.0074768680344723\\
563.01	0.00752817791877722\\
564.01	0.00758045071348906\\
565.01	0.00763369691704793\\
566.01	0.00768792635392892\\
567.01	0.00774314807891066\\
568.01	0.00779937027259147\\
569.01	0.0078566001275986\\
570.01	0.00791484372495724\\
571.01	0.00797410590012169\\
572.01	0.00803439009823456\\
573.01	0.00809569821827383\\
574.01	0.00815803044588143\\
575.01	0.00822138507485295\\
576.01	0.0082857583175189\\
577.01	0.00835114410458111\\
578.01	0.00841753387540446\\
579.01	0.00848491636033002\\
580.01	0.00855327735730464\\
581.01	0.00862259950605086\\
582.01	0.00869286206418521\\
583.01	0.0087640406911855\\
584.01	0.00883610724799195\\
585.01	0.00890902962239154\\
586.01	0.00898277159329311\\
587.01	0.00905729275069911\\
588.01	0.00913254849278597\\
589.01	0.00920849012723675\\
590.01	0.00928506511108912\\
591.01	0.0093622174721911\\
592.01	0.0094398884662976\\
593.01	0.00951801753737781\\
594.01	0.00959654366544385\\
595.01	0.0096754072068902\\
596.01	0.00975455235786353\\
597.01	0.00983363199586728\\
598.01	0.00990826330299362\\
599.01	0.00997053306357289\\
599.02	0.00997104257304143\\
599.03	0.00997154904691472\\
599.04	0.00997205245565375\\
599.05	0.00997255276942872\\
599.06	0.00997304995811617\\
599.07	0.00997354399129606\\
599.08	0.00997403483824883\\
599.09	0.00997452246795248\\
599.1	0.00997500684907951\\
599.11	0.00997548794999398\\
599.12	0.00997596573874838\\
599.13	0.0099764401830806\\
599.14	0.0099769112504108\\
599.15	0.00997737890783826\\
599.16	0.00997784312213823\\
599.17	0.0099783038597587\\
599.18	0.00997876108681718\\
599.19	0.00997921476909742\\
599.2	0.00997966487204611\\
599.21	0.00998011136076955\\
599.22	0.00998055420003028\\
599.23	0.00998099335424369\\
599.24	0.00998142878747458\\
599.25	0.00998186046343367\\
599.26	0.00998228834359041\\
599.27	0.00998271238778643\\
599.28	0.00998313255546381\\
599.29	0.00998354880566111\\
599.3	0.00998396109700947\\
599.31	0.00998436938772849\\
599.32	0.00998477363562221\\
599.33	0.00998517379807499\\
599.34	0.00998556983204737\\
599.35	0.00998596169407188\\
599.36	0.00998634934024881\\
599.37	0.00998673272624192\\
599.38	0.00998711180727415\\
599.39	0.00998748653812326\\
599.4	0.00998785687311743\\
599.41	0.00998822276613081\\
599.42	0.00998858417057903\\
599.43	0.00998894103941468\\
599.44	0.00998929332512275\\
599.45	0.00998964097971596\\
599.46	0.00998998395473014\\
599.47	0.00999032220121951\\
599.48	0.0099906556697519\\
599.49	0.00999098431040396\\
599.5	0.00999130807275631\\
599.51	0.00999162690588862\\
599.52	0.00999194075837471\\
599.53	0.00999224957827746\\
599.54	0.00999255331314386\\
599.55	0.00999285190999987\\
599.56	0.00999314531534524\\
599.57	0.00999343347514839\\
599.58	0.00999371633484107\\
599.59	0.00999399383931314\\
599.6	0.00999426593290715\\
599.61	0.009994532559413\\
599.62	0.0099947936620624\\
599.63	0.00999504918352342\\
599.64	0.00999529906589491\\
599.65	0.00999554325070086\\
599.66	0.00999578167888471\\
599.67	0.00999601429080366\\
599.68	0.00999624102622284\\
599.69	0.00999646182430947\\
599.7	0.00999667662362697\\
599.71	0.00999688536212897\\
599.72	0.00999708797715332\\
599.73	0.00999728440541595\\
599.74	0.00999747458300482\\
599.75	0.0099976584453736\\
599.76	0.00999783592733551\\
599.77	0.00999800696305692\\
599.78	0.00999817148605102\\
599.79	0.00999832942917133\\
599.8	0.00999848072460517\\
599.81	0.00999862530386713\\
599.82	0.00999876309779239\\
599.83	0.00999889403653003\\
599.84	0.00999901804953622\\
599.85	0.00999913506556741\\
599.86	0.00999924501267341\\
599.87	0.00999934781819042\\
599.88	0.00999944340873394\\
599.89	0.0099995317101917\\
599.9	0.00999961264771648\\
599.91	0.00999968614571878\\
599.92	0.00999975212785957\\
599.93	0.00999981051704285\\
599.94	0.00999986123540817\\
599.95	0.00999990420432308\\
599.96	0.00999993934437554\\
599.97	0.00999996657536618\\
599.98	0.00999998581630055\\
599.99	0.00999999698538124\\
600	0.01\\
};
\addplot [color=black!80!mycolor21,solid,forget plot]
  table[row sep=crcr]{%
0.01	0.00492490667902437\\
1.01	0.00492490763710551\\
2.01	0.00492490861484399\\
3.01	0.00492490961264044\\
4.01	0.00492491063090369\\
5.01	0.00492491167005141\\
6.01	0.00492491273050893\\
7.01	0.00492491381271079\\
8.01	0.0049249149170997\\
9.01	0.00492491604412815\\
10.01	0.00492491719425703\\
11.01	0.004924918367957\\
12.01	0.00492491956570795\\
13.01	0.00492492078799942\\
14.01	0.00492492203533126\\
15.01	0.00492492330821293\\
16.01	0.00492492460716443\\
17.01	0.00492492593271618\\
18.01	0.00492492728540914\\
19.01	0.00492492866579558\\
20.01	0.00492493007443867\\
21.01	0.00492493151191287\\
22.01	0.00492493297880414\\
23.01	0.00492493447571052\\
24.01	0.00492493600324184\\
25.01	0.0049249375620205\\
26.01	0.00492493915268086\\
27.01	0.004924940775871\\
28.01	0.00492494243225092\\
29.01	0.00492494412249517\\
30.01	0.00492494584729053\\
31.01	0.00492494760733822\\
32.01	0.00492494940335392\\
33.01	0.00492495123606726\\
34.01	0.00492495310622224\\
35.01	0.00492495501457851\\
36.01	0.0049249569619105\\
37.01	0.00492495894900827\\
38.01	0.00492496097667808\\
39.01	0.00492496304574183\\
40.01	0.00492496515703827\\
41.01	0.00492496731142281\\
42.01	0.00492496950976787\\
43.01	0.00492497175296369\\
44.01	0.00492497404191825\\
45.01	0.00492497637755762\\
46.01	0.00492497876082659\\
47.01	0.00492498119268858\\
48.01	0.00492498367412647\\
49.01	0.00492498620614293\\
50.01	0.00492498878976041\\
51.01	0.0049249914260221\\
52.01	0.00492499411599209\\
53.01	0.0049249968607552\\
54.01	0.00492499966141868\\
55.01	0.00492500251911115\\
56.01	0.00492500543498428\\
57.01	0.00492500841021268\\
58.01	0.00492501144599419\\
59.01	0.00492501454355068\\
60.01	0.00492501770412851\\
61.01	0.00492502092899852\\
62.01	0.00492502421945762\\
63.01	0.00492502757682772\\
64.01	0.0049250310024575\\
65.01	0.00492503449772236\\
66.01	0.00492503806402525\\
67.01	0.00492504170279701\\
68.01	0.00492504541549699\\
69.01	0.0049250492036132\\
70.01	0.00492505306866356\\
71.01	0.00492505701219631\\
72.01	0.00492506103579035\\
73.01	0.00492506514105567\\
74.01	0.00492506932963435\\
75.01	0.00492507360320107\\
76.01	0.00492507796346411\\
77.01	0.0049250824121653\\
78.01	0.00492508695108115\\
79.01	0.00492509158202337\\
80.01	0.00492509630683907\\
81.01	0.00492510112741271\\
82.01	0.00492510604566573\\
83.01	0.00492511106355731\\
84.01	0.00492511618308633\\
85.01	0.00492512140629026\\
86.01	0.00492512673524723\\
87.01	0.00492513217207638\\
88.01	0.0049251377189389\\
89.01	0.00492514337803851\\
90.01	0.00492514915162243\\
91.01	0.00492515504198224\\
92.01	0.00492516105145491\\
93.01	0.0049251671824235\\
94.01	0.00492517343731774\\
95.01	0.00492517981861551\\
96.01	0.00492518632884326\\
97.01	0.00492519297057759\\
98.01	0.00492519974644573\\
99.01	0.00492520665912602\\
100.01	0.00492521371135005\\
101.01	0.00492522090590311\\
102.01	0.00492522824562525\\
103.01	0.004925235733412\\
104.01	0.004925243372216\\
105.01	0.00492525116504766\\
106.01	0.00492525911497682\\
107.01	0.00492526722513315\\
108.01	0.00492527549870739\\
109.01	0.00492528393895361\\
110.01	0.00492529254918914\\
111.01	0.00492530133279617\\
112.01	0.00492531029322338\\
113.01	0.0049253194339865\\
114.01	0.00492532875867024\\
115.01	0.00492533827092964\\
116.01	0.00492534797449087\\
117.01	0.00492535787315251\\
118.01	0.00492536797078758\\
119.01	0.00492537827134487\\
120.01	0.00492538877884963\\
121.01	0.00492539949740589\\
122.01	0.00492541043119779\\
123.01	0.00492542158449069\\
124.01	0.00492543296163298\\
125.01	0.00492544456705721\\
126.01	0.00492545640528305\\
127.01	0.00492546848091738\\
128.01	0.00492548079865618\\
129.01	0.00492549336328715\\
130.01	0.00492550617969062\\
131.01	0.00492551925284163\\
132.01	0.00492553258781181\\
133.01	0.00492554618977081\\
134.01	0.00492556006398823\\
135.01	0.00492557421583584\\
136.01	0.00492558865078938\\
137.01	0.00492560337442985\\
138.01	0.00492561839244698\\
139.01	0.00492563371063967\\
140.01	0.00492564933491876\\
141.01	0.00492566527130894\\
142.01	0.0049256815259516\\
143.01	0.00492569810510612\\
144.01	0.00492571501515268\\
145.01	0.00492573226259324\\
146.01	0.00492574985405583\\
147.01	0.00492576779629533\\
148.01	0.00492578609619646\\
149.01	0.00492580476077628\\
150.01	0.00492582379718604\\
151.01	0.00492584321271464\\
152.01	0.0049258630147902\\
153.01	0.00492588321098314\\
154.01	0.00492590380900829\\
155.01	0.00492592481672902\\
156.01	0.00492594624215847\\
157.01	0.00492596809346252\\
158.01	0.00492599037896313\\
159.01	0.00492601310714125\\
160.01	0.0049260362866391\\
161.01	0.00492605992626372\\
162.01	0.00492608403498956\\
163.01	0.00492610862196233\\
164.01	0.00492613369650069\\
165.01	0.004926159268101\\
166.01	0.00492618534643985\\
167.01	0.00492621194137713\\
168.01	0.00492623906295953\\
169.01	0.00492626672142471\\
170.01	0.00492629492720278\\
171.01	0.00492632369092266\\
172.01	0.00492635302341332\\
173.01	0.00492638293570827\\
174.01	0.0049264134390494\\
175.01	0.00492644454489044\\
176.01	0.00492647626490114\\
177.01	0.0049265086109711\\
178.01	0.00492654159521297\\
179.01	0.00492657522996773\\
180.01	0.00492660952780785\\
181.01	0.0049266445015419\\
182.01	0.00492668016421846\\
183.01	0.00492671652913104\\
184.01	0.00492675360982156\\
185.01	0.00492679142008565\\
186.01	0.00492682997397659\\
187.01	0.00492686928581007\\
188.01	0.00492690937016915\\
189.01	0.00492695024190834\\
190.01	0.00492699191615945\\
191.01	0.0049270344083352\\
192.01	0.00492707773413564\\
193.01	0.00492712190955205\\
194.01	0.00492716695087228\\
195.01	0.00492721287468698\\
196.01	0.00492725969789357\\
197.01	0.00492730743770259\\
198.01	0.00492735611164244\\
199.01	0.00492740573756573\\
200.01	0.00492745633365438\\
201.01	0.00492750791842581\\
202.01	0.00492756051073809\\
203.01	0.00492761412979656\\
204.01	0.00492766879515991\\
205.01	0.00492772452674589\\
206.01	0.00492778134483804\\
207.01	0.00492783927009083\\
208.01	0.00492789832353788\\
209.01	0.00492795852659678\\
210.01	0.00492801990107683\\
211.01	0.00492808246918577\\
212.01	0.00492814625353585\\
213.01	0.00492821127715112\\
214.01	0.0049282775634748\\
215.01	0.00492834513637611\\
216.01	0.00492841402015756\\
217.01	0.00492848423956252\\
218.01	0.00492855581978241\\
219.01	0.00492862878646477\\
220.01	0.00492870316571992\\
221.01	0.00492877898413019\\
222.01	0.00492885626875733\\
223.01	0.00492893504715017\\
224.01	0.00492901534735302\\
225.01	0.00492909719791415\\
226.01	0.00492918062789444\\
227.01	0.00492926566687554\\
228.01	0.00492935234496852\\
229.01	0.00492944069282267\\
230.01	0.00492953074163544\\
231.01	0.00492962252315998\\
232.01	0.00492971606971512\\
233.01	0.00492981141419513\\
234.01	0.00492990859007864\\
235.01	0.00493000763143803\\
236.01	0.00493010857294976\\
237.01	0.00493021144990413\\
238.01	0.0049303162982155\\
239.01	0.0049304231544316\\
240.01	0.00493053205574517\\
241.01	0.00493064304000346\\
242.01	0.00493075614571922\\
243.01	0.00493087141208165\\
244.01	0.00493098887896681\\
245.01	0.00493110858694948\\
246.01	0.00493123057731407\\
247.01	0.00493135489206567\\
248.01	0.00493148157394202\\
249.01	0.00493161066642522\\
250.01	0.00493174221375398\\
251.01	0.00493187626093508\\
252.01	0.00493201285375597\\
253.01	0.00493215203879693\\
254.01	0.00493229386344446\\
255.01	0.00493243837590296\\
256.01	0.00493258562520849\\
257.01	0.00493273566124178\\
258.01	0.00493288853474158\\
259.01	0.00493304429731806\\
260.01	0.00493320300146705\\
261.01	0.00493336470058298\\
262.01	0.00493352944897399\\
263.01	0.00493369730187609\\
264.01	0.00493386831546727\\
265.01	0.00493404254688304\\
266.01	0.00493422005423044\\
267.01	0.00493440089660444\\
268.01	0.00493458513410247\\
269.01	0.00493477282784047\\
270.01	0.00493496403996954\\
271.01	0.00493515883369031\\
272.01	0.00493535727327132\\
273.01	0.00493555942406446\\
274.01	0.00493576535252299\\
275.01	0.00493597512621743\\
276.01	0.0049361888138548\\
277.01	0.00493640648529475\\
278.01	0.00493662821156854\\
279.01	0.00493685406489774\\
280.01	0.0049370841187128\\
281.01	0.00493731844767115\\
282.01	0.00493755712767796\\
283.01	0.00493780023590526\\
284.01	0.00493804785081245\\
285.01	0.00493830005216619\\
286.01	0.00493855692106168\\
287.01	0.0049388185399438\\
288.01	0.00493908499262932\\
289.01	0.00493935636432841\\
290.01	0.00493963274166756\\
291.01	0.0049399142127125\\
292.01	0.00494020086699201\\
293.01	0.004940492795522\\
294.01	0.00494079009082943\\
295.01	0.00494109284697816\\
296.01	0.00494140115959471\\
297.01	0.00494171512589406\\
298.01	0.00494203484470715\\
299.01	0.00494236041650806\\
300.01	0.00494269194344221\\
301.01	0.0049430295293558\\
302.01	0.00494337327982519\\
303.01	0.00494372330218756\\
304.01	0.00494407970557192\\
305.01	0.00494444260093238\\
306.01	0.00494481210107928\\
307.01	0.00494518832071475\\
308.01	0.00494557137646728\\
309.01	0.00494596138692713\\
310.01	0.00494635847268471\\
311.01	0.00494676275636632\\
312.01	0.00494717436267597\\
313.01	0.00494759341843476\\
314.01	0.00494802005262207\\
315.01	0.00494845439641964\\
316.01	0.00494889658325581\\
317.01	0.00494934674885122\\
318.01	0.00494980503126588\\
319.01	0.00495027157094885\\
320.01	0.00495074651078839\\
321.01	0.00495122999616479\\
322.01	0.00495172217500356\\
323.01	0.00495222319783234\\
324.01	0.00495273321783835\\
325.01	0.0049532523909282\\
326.01	0.00495378087578954\\
327.01	0.00495431883395647\\
328.01	0.00495486642987386\\
329.01	0.00495542383096773\\
330.01	0.00495599120771539\\
331.01	0.00495656873371975\\
332.01	0.00495715658578436\\
333.01	0.00495775494399275\\
334.01	0.00495836399179034\\
335.01	0.00495898391606756\\
336.01	0.00495961490724806\\
337.01	0.00496025715937801\\
338.01	0.00496091087021955\\
339.01	0.00496157624134719\\
340.01	0.00496225347824647\\
341.01	0.0049629427904172\\
342.01	0.00496364439147895\\
343.01	0.00496435849928029\\
344.01	0.00496508533601028\\
345.01	0.00496582512831606\\
346.01	0.00496657810741996\\
347.01	0.00496734450924372\\
348.01	0.00496812457453369\\
349.01	0.00496891854898956\\
350.01	0.00496972668339865\\
351.01	0.00497054923377022\\
352.01	0.00497138646147524\\
353.01	0.004972238633389\\
354.01	0.00497310602203585\\
355.01	0.00497398890573808\\
356.01	0.00497488756876668\\
357.01	0.00497580230149569\\
358.01	0.00497673340055799\\
359.01	0.00497768116900435\\
360.01	0.00497864591646405\\
361.01	0.00497962795930765\\
362.01	0.00498062762081122\\
363.01	0.00498164523132263\\
364.01	0.00498268112842817\\
365.01	0.00498373565711971\\
366.01	0.00498480916996449\\
367.01	0.00498590202727345\\
368.01	0.0049870145972693\\
369.01	0.0049881472562571\\
370.01	0.00498930038879045\\
371.01	0.00499047438783992\\
372.01	0.00499166965496023\\
373.01	0.00499288660045505\\
374.01	0.00499412564354196\\
375.01	0.00499538721251511\\
376.01	0.00499667174490859\\
377.01	0.00499797968765625\\
378.01	0.00499931149725183\\
379.01	0.00500066763990911\\
380.01	0.00500204859172093\\
381.01	0.00500345483881856\\
382.01	0.00500488687753237\\
383.01	0.00500634521455415\\
384.01	0.00500783036710345\\
385.01	0.0050093428630954\\
386.01	0.00501088324131688\\
387.01	0.00501245205160566\\
388.01	0.00501404985504153\\
389.01	0.00501567722414542\\
390.01	0.00501733474309138\\
391.01	0.00501902300792978\\
392.01	0.00502074262683022\\
393.01	0.00502249422033935\\
394.01	0.00502427842165828\\
395.01	0.00502609587694272\\
396.01	0.00502794724562558\\
397.01	0.0050298332007632\\
398.01	0.00503175442940865\\
399.01	0.00503371163300942\\
400.01	0.00503570552783239\\
401.01	0.0050377368454147\\
402.01	0.00503980633304035\\
403.01	0.00504191475423932\\
404.01	0.00504406288930961\\
405.01	0.00504625153585783\\
406.01	0.00504848150935616\\
407.01	0.00505075364371188\\
408.01	0.00505306879184558\\
409.01	0.00505542782627566\\
410.01	0.00505783163970333\\
411.01	0.00506028114559842\\
412.01	0.00506277727877949\\
413.01	0.00506532099599163\\
414.01	0.00506791327647585\\
415.01	0.00507055512253482\\
416.01	0.0050732475600952\\
417.01	0.00507599163926481\\
418.01	0.00507878843489198\\
419.01	0.00508163904712511\\
420.01	0.00508454460197319\\
421.01	0.00508750625187029\\
422.01	0.00509052517624065\\
423.01	0.00509360258206507\\
424.01	0.00509673970445076\\
425.01	0.00509993780720003\\
426.01	0.0051031981833782\\
427.01	0.00510652215588461\\
428.01	0.00510991107802156\\
429.01	0.00511336633406412\\
430.01	0.00511688933983065\\
431.01	0.00512048154325123\\
432.01	0.00512414442494041\\
433.01	0.0051278794987677\\
434.01	0.00513168831243274\\
435.01	0.00513557244804125\\
436.01	0.00513953352268609\\
437.01	0.00514357318903322\\
438.01	0.00514769313591301\\
439.01	0.00515189508892069\\
440.01	0.0051561808110252\\
441.01	0.00516055210319047\\
442.01	0.00516501080500903\\
443.01	0.00516955879535103\\
444.01	0.00517419799303032\\
445.01	0.00517893035748884\\
446.01	0.00518375788950279\\
447.01	0.00518868263191022\\
448.01	0.00519370667036323\\
449.01	0.0051988321341048\\
450.01	0.00520406119677222\\
451.01	0.00520939607722745\\
452.01	0.00521483904041384\\
453.01	0.00522039239824043\\
454.01	0.00522605851049271\\
455.01	0.00523183978576761\\
456.01	0.00523773868243391\\
457.01	0.00524375770961405\\
458.01	0.00524989942818671\\
459.01	0.00525616645180813\\
460.01	0.00526256144794826\\
461.01	0.00526908713894422\\
462.01	0.00527574630306341\\
463.01	0.00528254177558157\\
464.01	0.00528947644986957\\
465.01	0.00529655327849493\\
466.01	0.00530377527433369\\
467.01	0.00531114551169888\\
468.01	0.00531866712748578\\
469.01	0.00532634332233806\\
470.01	0.00533417736183868\\
471.01	0.00534217257772819\\
472.01	0.00535033236915525\\
473.01	0.00535866020396041\\
474.01	0.00536715961999752\\
475.01	0.00537583422649196\\
476.01	0.00538468770543755\\
477.01	0.0053937238130322\\
478.01	0.00540294638115267\\
479.01	0.00541235931886618\\
480.01	0.00542196661398017\\
481.01	0.00543177233462867\\
482.01	0.00544178063089473\\
483.01	0.00545199573646733\\
484.01	0.00546242197033285\\
485.01	0.00547306373849976\\
486.01	0.00548392553575595\\
487.01	0.00549501194745909\\
488.01	0.005506327651358\\
489.01	0.00551787741944686\\
490.01	0.00552966611985088\\
491.01	0.0055416987187458\\
492.01	0.00555398028230869\\
493.01	0.00556651597870308\\
494.01	0.00557931108009724\\
495.01	0.00559237096471669\\
496.01	0.00560570111892965\\
497.01	0.00561930713936619\\
498.01	0.00563319473506921\\
499.01	0.00564736972967747\\
500.01	0.0056618380636375\\
501.01	0.00567660579644481\\
502.01	0.00569167910891125\\
503.01	0.00570706430545774\\
504.01	0.00572276781642914\\
505.01	0.00573879620043109\\
506.01	0.00575515614668512\\
507.01	0.00577185447740057\\
508.01	0.00578889815016059\\
509.01	0.00580629426031899\\
510.01	0.00582405004340483\\
511.01	0.00584217287753226\\
512.01	0.00586067028580932\\
513.01	0.00587954993874287\\
514.01	0.00589881965663439\\
515.01	0.00591848741195934\\
516.01	0.00593856133172524\\
517.01	0.00595904969980052\\
518.01	0.00597996095920626\\
519.01	0.00600130371436242\\
520.01	0.00602308673327795\\
521.01	0.00604531894967439\\
522.01	0.00606800946503175\\
523.01	0.00609116755054227\\
524.01	0.00611480264895806\\
525.01	0.00613892437631648\\
526.01	0.00616354252352511\\
527.01	0.00618866705778603\\
528.01	0.00621430812383886\\
529.01	0.00624047604499726\\
530.01	0.00626718132395188\\
531.01	0.00629443464331172\\
532.01	0.0063222468658508\\
533.01	0.00635062903442334\\
534.01	0.00637959237151045\\
535.01	0.00640914827835252\\
536.01	0.00643930833362073\\
537.01	0.00647008429157556\\
538.01	0.00650148807965305\\
539.01	0.00653353179541709\\
540.01	0.00656622770280604\\
541.01	0.00659958822759805\\
542.01	0.00663362595201053\\
543.01	0.00666835360834141\\
544.01	0.0067037840715494\\
545.01	0.00673993035066367\\
546.01	0.00677680557889973\\
547.01	0.00681442300234893\\
548.01	0.00685279596709369\\
549.01	0.0068919379045899\\
550.01	0.00693186231514151\\
551.01	0.0069725827492761\\
552.01	0.00701411278681359\\
553.01	0.00705646601340192\\
554.01	0.00709965599427295\\
555.01	0.00714369624495094\\
556.01	0.00718860019862366\\
557.01	0.00723438116986311\\
558.01	0.00728105231435536\\
559.01	0.00732862658427608\\
560.01	0.00737711667892064\\
561.01	0.0074265349901686\\
562.01	0.00747689354233801\\
563.01	0.00752820392595671\\
564.01	0.00758047722495186\\
565.01	0.00763372393673838\\
566.01	0.0076879538846658\\
567.01	0.00774317612227109\\
568.01	0.00779939882877782\\
569.01	0.00785662919528888\\
570.01	0.00791487330113772\\
571.01	0.00797413597990184\\
572.01	0.00803442067464424\\
573.01	0.00809572928204298\\
574.01	0.00815806198520298\\
575.01	0.00822141707512997\\
576.01	0.00828579076109853\\
577.01	0.00835117697047761\\
578.01	0.00841756713901588\\
579.01	0.00848494999315436\\
580.01	0.00855331132666273\\
581.01	0.00862263377482674\\
582.01	0.00869289659059665\\
583.01	0.00876407542860217\\
584.01	0.00883614214482343\\
585.01	0.00890906462207175\\
586.01	0.0089828066343969\\
587.01	0.0090573277672331\\
588.01	0.00913258341470725\\
589.01	0.00920852488126394\\
590.01	0.00928509962188423\\
591.01	0.00936225166400636\\
592.01	0.00943992226519836\\
593.01	0.0095180508741716\\
594.01	0.00959657647946483\\
595.01	0.00967543945080618\\
596.01	0.00975458400368758\\
597.01	0.00983363394660122\\
598.01	0.00990826330299362\\
599.01	0.00997053306357289\\
599.02	0.00997104257304143\\
599.03	0.00997154904691472\\
599.04	0.00997205245565375\\
599.05	0.00997255276942872\\
599.06	0.00997304995811617\\
599.07	0.00997354399129606\\
599.08	0.00997403483824883\\
599.09	0.00997452246795248\\
599.1	0.00997500684907952\\
599.11	0.00997548794999398\\
599.12	0.00997596573874838\\
599.13	0.0099764401830806\\
599.14	0.0099769112504108\\
599.15	0.00997737890783826\\
599.16	0.00997784312213823\\
599.17	0.0099783038597587\\
599.18	0.00997876108681718\\
599.19	0.00997921476909742\\
599.2	0.00997966487204611\\
599.21	0.00998011136076955\\
599.22	0.00998055420003028\\
599.23	0.00998099335424369\\
599.24	0.00998142878747458\\
599.25	0.00998186046343367\\
599.26	0.00998228834359041\\
599.27	0.00998271238778643\\
599.28	0.00998313255546381\\
599.29	0.00998354880566111\\
599.3	0.00998396109700947\\
599.31	0.00998436938772849\\
599.32	0.00998477363562221\\
599.33	0.00998517379807499\\
599.34	0.00998556983204737\\
599.35	0.00998596169407188\\
599.36	0.00998634934024881\\
599.37	0.00998673272624192\\
599.38	0.00998711180727415\\
599.39	0.00998748653812326\\
599.4	0.00998785687311743\\
599.41	0.00998822276613081\\
599.42	0.00998858417057903\\
599.43	0.00998894103941468\\
599.44	0.00998929332512275\\
599.45	0.00998964097971596\\
599.46	0.00998998395473014\\
599.47	0.00999032220121951\\
599.48	0.0099906556697519\\
599.49	0.00999098431040396\\
599.5	0.00999130807275631\\
599.51	0.00999162690588863\\
599.52	0.00999194075837471\\
599.53	0.00999224957827746\\
599.54	0.00999255331314387\\
599.55	0.00999285190999987\\
599.56	0.00999314531534524\\
599.57	0.00999343347514839\\
599.58	0.00999371633484107\\
599.59	0.00999399383931314\\
599.6	0.00999426593290715\\
599.61	0.009994532559413\\
599.62	0.0099947936620624\\
599.63	0.00999504918352342\\
599.64	0.00999529906589491\\
599.65	0.00999554325070086\\
599.66	0.00999578167888471\\
599.67	0.00999601429080366\\
599.68	0.00999624102622283\\
599.69	0.00999646182430947\\
599.7	0.00999667662362697\\
599.71	0.00999688536212897\\
599.72	0.00999708797715332\\
599.73	0.00999728440541596\\
599.74	0.00999747458300482\\
599.75	0.0099976584453736\\
599.76	0.0099978359273355\\
599.77	0.00999800696305692\\
599.78	0.00999817148605102\\
599.79	0.00999832942917133\\
599.8	0.00999848072460517\\
599.81	0.00999862530386713\\
599.82	0.00999876309779239\\
599.83	0.00999889403653003\\
599.84	0.00999901804953622\\
599.85	0.00999913506556741\\
599.86	0.00999924501267341\\
599.87	0.00999934781819042\\
599.88	0.00999944340873394\\
599.89	0.0099995317101917\\
599.9	0.00999961264771648\\
599.91	0.00999968614571878\\
599.92	0.00999975212785958\\
599.93	0.00999981051704285\\
599.94	0.00999986123540817\\
599.95	0.00999990420432308\\
599.96	0.00999993934437554\\
599.97	0.00999996657536618\\
599.98	0.00999998581630055\\
599.99	0.00999999698538124\\
600	0.01\\
};
\addplot [color=black,solid,forget plot]
  table[row sep=crcr]{%
0.01	0.00493182287751867\\
1.01	0.0049318238290299\\
2.01	0.00493182480000867\\
3.01	0.00493182579085036\\
4.01	0.00493182680195881\\
5.01	0.00493182783374529\\
6.01	0.00493182888663021\\
7.01	0.00493182996104166\\
8.01	0.00493183105741671\\
9.01	0.00493183217620111\\
10.01	0.00493183331784941\\
11.01	0.00493183448282594\\
12.01	0.00493183567160338\\
13.01	0.00493183688466478\\
14.01	0.00493183812250247\\
15.01	0.00493183938561867\\
16.01	0.00493184067452582\\
17.01	0.00493184198974639\\
18.01	0.00493184333181385\\
19.01	0.00493184470127195\\
20.01	0.00493184609867525\\
21.01	0.00493184752458969\\
22.01	0.00493184897959259\\
23.01	0.00493185046427261\\
24.01	0.00493185197923021\\
25.01	0.00493185352507805\\
26.01	0.00493185510244132\\
27.01	0.00493185671195676\\
28.01	0.00493185835427515\\
29.01	0.00493186003005908\\
30.01	0.00493186173998522\\
31.01	0.00493186348474337\\
32.01	0.00493186526503714\\
33.01	0.00493186708158444\\
34.01	0.00493186893511746\\
35.01	0.00493187082638238\\
36.01	0.00493187275614124\\
37.01	0.00493187472517079\\
38.01	0.0049318767342634\\
39.01	0.00493187878422696\\
40.01	0.00493188087588575\\
41.01	0.00493188301008055\\
42.01	0.00493188518766854\\
43.01	0.00493188740952449\\
44.01	0.00493188967654021\\
45.01	0.00493189198962524\\
46.01	0.00493189434970737\\
47.01	0.0049318967577329\\
48.01	0.00493189921466686\\
49.01	0.00493190172149338\\
50.01	0.00493190427921622\\
51.01	0.00493190688885906\\
52.01	0.00493190955146586\\
53.01	0.00493191226810191\\
54.01	0.00493191503985274\\
55.01	0.00493191786782596\\
56.01	0.00493192075315094\\
57.01	0.00493192369697931\\
58.01	0.00493192670048592\\
59.01	0.00493192976486884\\
60.01	0.00493193289134933\\
61.01	0.00493193608117393\\
62.01	0.00493193933561248\\
63.01	0.00493194265596109\\
64.01	0.00493194604354083\\
65.01	0.00493194949969913\\
66.01	0.00493195302580993\\
67.01	0.00493195662327456\\
68.01	0.00493196029352166\\
69.01	0.00493196403800866\\
70.01	0.00493196785822145\\
71.01	0.00493197175567481\\
72.01	0.00493197573191395\\
73.01	0.00493197978851434\\
74.01	0.00493198392708238\\
75.01	0.00493198814925644\\
76.01	0.00493199245670695\\
77.01	0.00493199685113693\\
78.01	0.00493200133428354\\
79.01	0.00493200590791718\\
80.01	0.00493201057384435\\
81.01	0.00493201533390594\\
82.01	0.00493202018997954\\
83.01	0.0049320251439792\\
84.01	0.00493203019785679\\
85.01	0.00493203535360265\\
86.01	0.00493204061324592\\
87.01	0.00493204597885555\\
88.01	0.00493205145254083\\
89.01	0.00493205703645261\\
90.01	0.00493206273278426\\
91.01	0.00493206854377126\\
92.01	0.00493207447169297\\
93.01	0.00493208051887367\\
94.01	0.00493208668768283\\
95.01	0.00493209298053607\\
96.01	0.00493209939989656\\
97.01	0.00493210594827531\\
98.01	0.00493211262823195\\
99.01	0.00493211944237681\\
100.01	0.00493212639337046\\
101.01	0.00493213348392549\\
102.01	0.00493214071680723\\
103.01	0.00493214809483475\\
104.01	0.00493215562088206\\
105.01	0.00493216329787916\\
106.01	0.00493217112881278\\
107.01	0.00493217911672769\\
108.01	0.0049321872647278\\
109.01	0.00493219557597735\\
110.01	0.00493220405370171\\
111.01	0.00493221270118868\\
112.01	0.00493222152178996\\
113.01	0.00493223051892233\\
114.01	0.00493223969606799\\
115.01	0.00493224905677733\\
116.01	0.00493225860466878\\
117.01	0.00493226834343105\\
118.01	0.0049322782768236\\
119.01	0.00493228840867871\\
120.01	0.00493229874290274\\
121.01	0.00493230928347724\\
122.01	0.00493232003446017\\
123.01	0.00493233099998788\\
124.01	0.00493234218427618\\
125.01	0.0049323535916225\\
126.01	0.00493236522640594\\
127.01	0.00493237709309033\\
128.01	0.00493238919622524\\
129.01	0.00493240154044733\\
130.01	0.00493241413048207\\
131.01	0.00493242697114588\\
132.01	0.00493244006734664\\
133.01	0.00493245342408735\\
134.01	0.00493246704646585\\
135.01	0.00493248093967752\\
136.01	0.00493249510901711\\
137.01	0.0049325095598807\\
138.01	0.00493252429776672\\
139.01	0.00493253932827917\\
140.01	0.00493255465712827\\
141.01	0.00493257029013333\\
142.01	0.00493258623322426\\
143.01	0.00493260249244373\\
144.01	0.00493261907394901\\
145.01	0.004932635984015\\
146.01	0.00493265322903502\\
147.01	0.00493267081552407\\
148.01	0.00493268875012062\\
149.01	0.00493270703958856\\
150.01	0.00493272569082001\\
151.01	0.00493274471083699\\
152.01	0.00493276410679494\\
153.01	0.00493278388598445\\
154.01	0.00493280405583318\\
155.01	0.00493282462390918\\
156.01	0.00493284559792298\\
157.01	0.00493286698573062\\
158.01	0.00493288879533524\\
159.01	0.00493291103489127\\
160.01	0.00493293371270617\\
161.01	0.00493295683724287\\
162.01	0.00493298041712356\\
163.01	0.00493300446113168\\
164.01	0.00493302897821582\\
165.01	0.00493305397749164\\
166.01	0.00493307946824523\\
167.01	0.00493310545993636\\
168.01	0.00493313196220119\\
169.01	0.00493315898485566\\
170.01	0.00493318653789934\\
171.01	0.00493321463151686\\
172.01	0.0049332432760836\\
173.01	0.00493327248216704\\
174.01	0.00493330226053137\\
175.01	0.00493333262214005\\
176.01	0.00493336357816088\\
177.01	0.00493339513996749\\
178.01	0.00493342731914478\\
179.01	0.0049334601274918\\
180.01	0.00493349357702577\\
181.01	0.00493352767998562\\
182.01	0.00493356244883617\\
183.01	0.00493359789627182\\
184.01	0.00493363403522082\\
185.01	0.00493367087884955\\
186.01	0.00493370844056579\\
187.01	0.00493374673402433\\
188.01	0.00493378577312975\\
189.01	0.00493382557204222\\
190.01	0.00493386614518054\\
191.01	0.00493390750722789\\
192.01	0.0049339496731351\\
193.01	0.00493399265812658\\
194.01	0.00493403647770455\\
195.01	0.00493408114765315\\
196.01	0.00493412668404436\\
197.01	0.00493417310324204\\
198.01	0.004934220421908\\
199.01	0.00493426865700619\\
200.01	0.00493431782580808\\
201.01	0.00493436794589777\\
202.01	0.00493441903517816\\
203.01	0.00493447111187581\\
204.01	0.00493452419454568\\
205.01	0.00493457830207786\\
206.01	0.00493463345370304\\
207.01	0.00493468966899818\\
208.01	0.00493474696789159\\
209.01	0.00493480537067087\\
210.01	0.00493486489798639\\
211.01	0.00493492557085923\\
212.01	0.00493498741068712\\
213.01	0.00493505043925033\\
214.01	0.00493511467871835\\
215.01	0.00493518015165628\\
216.01	0.00493524688103181\\
217.01	0.00493531489022142\\
218.01	0.00493538420301784\\
219.01	0.00493545484363617\\
220.01	0.0049355268367218\\
221.01	0.00493560020735717\\
222.01	0.0049356749810687\\
223.01	0.0049357511838343\\
224.01	0.00493582884209182\\
225.01	0.00493590798274565\\
226.01	0.00493598863317379\\
227.01	0.00493607082123765\\
228.01	0.00493615457528845\\
229.01	0.00493623992417599\\
230.01	0.00493632689725595\\
231.01	0.00493641552439997\\
232.01	0.00493650583600228\\
233.01	0.00493659786298942\\
234.01	0.00493669163682811\\
235.01	0.00493678718953539\\
236.01	0.0049368845536864\\
237.01	0.00493698376242396\\
238.01	0.00493708484946778\\
239.01	0.0049371878491247\\
240.01	0.00493729279629676\\
241.01	0.00493739972649196\\
242.01	0.00493750867583373\\
243.01	0.00493761968107156\\
244.01	0.00493773277959098\\
245.01	0.00493784800942301\\
246.01	0.00493796540925598\\
247.01	0.00493808501844553\\
248.01	0.00493820687702576\\
249.01	0.00493833102572058\\
250.01	0.00493845750595384\\
251.01	0.00493858635986247\\
252.01	0.0049387176303064\\
253.01	0.00493885136088135\\
254.01	0.0049389875959312\\
255.01	0.00493912638055916\\
256.01	0.00493926776064105\\
257.01	0.0049394117828373\\
258.01	0.00493955849460703\\
259.01	0.00493970794421938\\
260.01	0.00493986018076875\\
261.01	0.00494001525418711\\
262.01	0.00494017321525833\\
263.01	0.00494033411563222\\
264.01	0.00494049800783891\\
265.01	0.00494066494530354\\
266.01	0.00494083498236114\\
267.01	0.00494100817427168\\
268.01	0.00494118457723583\\
269.01	0.00494136424841066\\
270.01	0.00494154724592571\\
271.01	0.00494173362889974\\
272.01	0.00494192345745711\\
273.01	0.00494211679274509\\
274.01	0.00494231369695136\\
275.01	0.00494251423332251\\
276.01	0.00494271846618066\\
277.01	0.00494292646094402\\
278.01	0.0049431382841448\\
279.01	0.00494335400344934\\
280.01	0.00494357368767709\\
281.01	0.0049437974068223\\
282.01	0.00494402523207391\\
283.01	0.00494425723583722\\
284.01	0.00494449349175594\\
285.01	0.0049447340747345\\
286.01	0.00494497906096072\\
287.01	0.00494522852792967\\
288.01	0.00494548255446772\\
289.01	0.00494574122075717\\
290.01	0.00494600460836153\\
291.01	0.00494627280025134\\
292.01	0.00494654588083177\\
293.01	0.00494682393596887\\
294.01	0.00494710705301846\\
295.01	0.00494739532085464\\
296.01	0.00494768882989916\\
297.01	0.00494798767215293\\
298.01	0.00494829194122615\\
299.01	0.00494860173237138\\
300.01	0.00494891714251669\\
301.01	0.00494923827029894\\
302.01	0.0049495652160996\\
303.01	0.00494989808208038\\
304.01	0.00495023697222066\\
305.01	0.00495058199235521\\
306.01	0.00495093325021403\\
307.01	0.00495129085546259\\
308.01	0.00495165491974356\\
309.01	0.00495202555671972\\
310.01	0.00495240288211798\\
311.01	0.00495278701377563\\
312.01	0.00495317807168652\\
313.01	0.00495357617804909\\
314.01	0.00495398145731751\\
315.01	0.00495439403625192\\
316.01	0.00495481404397116\\
317.01	0.00495524161200769\\
318.01	0.0049556768743641\\
319.01	0.00495611996756937\\
320.01	0.00495657103073976\\
321.01	0.00495703020563959\\
322.01	0.00495749763674413\\
323.01	0.00495797347130478\\
324.01	0.00495845785941534\\
325.01	0.0049589509540818\\
326.01	0.00495945291129252\\
327.01	0.00495996389009043\\
328.01	0.00496048405264951\\
329.01	0.00496101356434989\\
330.01	0.00496155259385811\\
331.01	0.00496210131320801\\
332.01	0.00496265989788433\\
333.01	0.00496322852690834\\
334.01	0.00496380738292523\\
335.01	0.00496439665229551\\
336.01	0.00496499652518606\\
337.01	0.0049656071956663\\
338.01	0.00496622886180392\\
339.01	0.00496686172576477\\
340.01	0.00496750599391538\\
341.01	0.00496816187692553\\
342.01	0.00496882958987546\\
343.01	0.00496950935236398\\
344.01	0.00497020138861987\\
345.01	0.00497090592761398\\
346.01	0.00497162320317568\\
347.01	0.00497235345410937\\
348.01	0.00497309692431502\\
349.01	0.00497385386290883\\
350.01	0.00497462452434724\\
351.01	0.00497540916855377\\
352.01	0.00497620806104546\\
353.01	0.00497702147306339\\
354.01	0.00497784968170376\\
355.01	0.00497869297005177\\
356.01	0.00497955162731625\\
357.01	0.00498042594896672\\
358.01	0.00498131623687226\\
359.01	0.00498222279944097\\
360.01	0.00498314595176261\\
361.01	0.00498408601575159\\
362.01	0.00498504332029251\\
363.01	0.00498601820138575\\
364.01	0.00498701100229715\\
365.01	0.00498802207370727\\
366.01	0.00498905177386316\\
367.01	0.0049901004687323\\
368.01	0.00499116853215789\\
369.01	0.00499225634601669\\
370.01	0.00499336430037978\\
371.01	0.00499449279367409\\
372.01	0.00499564223284896\\
373.01	0.00499681303354341\\
374.01	0.00499800562025859\\
375.01	0.00499922042653445\\
376.01	0.00500045789512835\\
377.01	0.00500171847820011\\
378.01	0.00500300263750234\\
379.01	0.00500431084457509\\
380.01	0.00500564358094832\\
381.01	0.00500700133834953\\
382.01	0.00500838461892072\\
383.01	0.00500979393544249\\
384.01	0.00501122981156529\\
385.01	0.00501269278205324\\
386.01	0.00501418339303464\\
387.01	0.00501570220226563\\
388.01	0.00501724977940263\\
389.01	0.00501882670628742\\
390.01	0.0050204335772433\\
391.01	0.00502207099938506\\
392.01	0.00502373959293897\\
393.01	0.0050254399915764\\
394.01	0.00502717284276052\\
395.01	0.00502893880810304\\
396.01	0.00503073856373509\\
397.01	0.00503257280068803\\
398.01	0.00503444222528482\\
399.01	0.00503634755954288\\
400.01	0.00503828954158442\\
401.01	0.0050402689260558\\
402.01	0.00504228648455377\\
403.01	0.00504434300605858\\
404.01	0.00504643929737017\\
405.01	0.00504857618354995\\
406.01	0.00505075450836626\\
407.01	0.00505297513473901\\
408.01	0.00505523894519064\\
409.01	0.00505754684229329\\
410.01	0.00505989974912169\\
411.01	0.00506229860970375\\
412.01	0.00506474438947371\\
413.01	0.00506723807572645\\
414.01	0.00506978067807337\\
415.01	0.00507237322890244\\
416.01	0.00507501678383912\\
417.01	0.0050777124222123\\
418.01	0.0050804612475229\\
419.01	0.00508326438791806\\
420.01	0.00508612299666758\\
421.01	0.00508903825264689\\
422.01	0.00509201136082305\\
423.01	0.00509504355274762\\
424.01	0.00509813608705151\\
425.01	0.00510129024994832\\
426.01	0.00510450735574314\\
427.01	0.00510778874734499\\
428.01	0.00511113579678996\\
429.01	0.00511454990576912\\
430.01	0.00511803250616452\\
431.01	0.00512158506059596\\
432.01	0.00512520906297331\\
433.01	0.00512890603906338\\
434.01	0.00513267754706359\\
435.01	0.00513652517819077\\
436.01	0.00514045055727982\\
437.01	0.00514445534339625\\
438.01	0.00514854123046399\\
439.01	0.00515270994790704\\
440.01	0.00515696326130713\\
441.01	0.00516130297307757\\
442.01	0.00516573092315437\\
443.01	0.00517024898970506\\
444.01	0.00517485908985496\\
445.01	0.00517956318043405\\
446.01	0.00518436325874022\\
447.01	0.00518926136332284\\
448.01	0.00519425957478664\\
449.01	0.00519936001661375\\
450.01	0.00520456485600515\\
451.01	0.00520987630474193\\
452.01	0.00521529662006501\\
453.01	0.00522082810557442\\
454.01	0.0052264731121468\\
455.01	0.00523223403887141\\
456.01	0.00523811333400456\\
457.01	0.00524411349594245\\
458.01	0.00525023707421238\\
459.01	0.00525648667048167\\
460.01	0.0052628649395878\\
461.01	0.0052693745905851\\
462.01	0.005276018387814\\
463.01	0.00528279915198995\\
464.01	0.00528971976131598\\
465.01	0.00529678315261656\\
466.01	0.00530399232249751\\
467.01	0.00531135032853136\\
468.01	0.00531886029046935\\
469.01	0.00532652539148226\\
470.01	0.00533434887942855\\
471.01	0.00534233406815415\\
472.01	0.00535048433882165\\
473.01	0.00535880314127121\\
474.01	0.00536729399541064\\
475.01	0.00537596049263875\\
476.01	0.00538480629729859\\
477.01	0.00539383514816266\\
478.01	0.00540305085994849\\
479.01	0.00541245732486612\\
480.01	0.00542205851419633\\
481.01	0.00543185847989999\\
482.01	0.00544186135625857\\
483.01	0.00545207136154552\\
484.01	0.00546249279972985\\
485.01	0.00547313006220942\\
486.01	0.0054839876295774\\
487.01	0.00549507007341975\\
488.01	0.00550638205814426\\
489.01	0.00551792834284205\\
490.01	0.00552971378318126\\
491.01	0.00554174333333225\\
492.01	0.00555402204792588\\
493.01	0.0055665550840433\\
494.01	0.00557934770323765\\
495.01	0.00559240527358693\\
496.01	0.00560573327177863\\
497.01	0.00561933728522382\\
498.01	0.00563322301420169\\
499.01	0.00564739627403231\\
500.01	0.00566186299727794\\
501.01	0.0056766292359703\\
502.01	0.00569170116386418\\
503.01	0.00570708507871467\\
504.01	0.00572278740457782\\
505.01	0.00573881469413159\\
506.01	0.00575517363101588\\
507.01	0.00577187103218983\\
508.01	0.00578891385030207\\
509.01	0.00580630917607387\\
510.01	0.00582406424068908\\
511.01	0.005842186418189\\
512.01	0.00586068322786752\\
513.01	0.00587956233666231\\
514.01	0.00589883156153569\\
515.01	0.00591849887184132\\
516.01	0.00593857239166883\\
517.01	0.00595906040216021\\
518.01	0.00597997134378925\\
519.01	0.00600131381859617\\
520.01	0.0060230965923669\\
521.01	0.00604532859674715\\
522.01	0.00606801893127805\\
523.01	0.00609117686534137\\
524.01	0.00611481183999849\\
525.01	0.00613893346970859\\
526.01	0.00616355154390595\\
527.01	0.00618867602841776\\
528.01	0.00621431706670075\\
529.01	0.00624048498087197\\
530.01	0.00626719027250712\\
531.01	0.00629444362317652\\
532.01	0.00632225589468721\\
533.01	0.00635063812899544\\
534.01	0.00637960154774825\\
535.01	0.00640915755141296\\
536.01	0.00643931771794654\\
537.01	0.00647009380094982\\
538.01	0.00650149772725282\\
539.01	0.0065335415938632\\
540.01	0.00656623766421105\\
541.01	0.00659959836361262\\
542.01	0.00663363627386721\\
543.01	0.00666836412689591\\
544.01	0.00670379479732074\\
545.01	0.00673994129387191\\
546.01	0.00677681674950216\\
547.01	0.00681443441007326\\
548.01	0.0068528076214695\\
549.01	0.00689194981497837\\
550.01	0.00693187449076199\\
551.01	0.00697259519923017\\
552.01	0.00701412552010712\\
553.01	0.00705647903896367\\
554.01	0.00709966932096989\\
555.01	0.00714370988160053\\
556.01	0.00718861415400322\\
557.01	0.00723439545271368\\
558.01	0.00728106693338249\\
559.01	0.00732864154814654\\
560.01	0.00737713199625294\\
561.01	0.00742655066951931\\
562.01	0.00747690959218157\\
563.01	0.00752822035465922\\
564.01	0.00758049404073814\\
565.01	0.00763374114765142\\
566.01	0.00768797149851824\\
567.01	0.0077431941465875\\
568.01	0.00779941727072736\\
569.01	0.00785664806160745\\
570.01	0.00791489259803824\\
571.01	0.00797415571297161\\
572.01	0.00803444084872872\\
573.01	0.00809574990111406\\
574.01	0.00815808305221095\\
575.01	0.0082214385918378\\
576.01	0.00828581272789679\\
577.01	0.00835119938617968\\
578.01	0.00841759000063182\\
579.01	0.00848497329564366\\
580.01	0.00855333506266526\\
581.01	0.00862265793437383\\
582.01	0.00869292116080432\\
583.01	0.008764100393351\\
584.01	0.00883616748443228\\
585.01	0.00890909031297696\\
586.01	0.00898283264885292\\
587.01	0.00905735407305973\\
588.01	0.00913260997511742\\
589.01	0.00920855165482164\\
590.01	0.00928512656266121\\
591.01	0.00936227872203362\\
592.01	0.00943994938734479\\
593.01	0.00951807800563368\\
594.01	0.00959660356611897\\
595.01	0.0096754664427708\\
596.01	0.00975461086056989\\
597.01	0.00983363553942939\\
598.01	0.00990826330299362\\
599.01	0.00997053306357288\\
599.02	0.00997104257304143\\
599.03	0.00997154904691472\\
599.04	0.00997205245565375\\
599.05	0.00997255276942872\\
599.06	0.00997304995811617\\
599.07	0.00997354399129606\\
599.08	0.00997403483824883\\
599.09	0.00997452246795248\\
599.1	0.00997500684907951\\
599.11	0.00997548794999398\\
599.12	0.00997596573874838\\
599.13	0.0099764401830806\\
599.14	0.0099769112504108\\
599.15	0.00997737890783826\\
599.16	0.00997784312213823\\
599.17	0.0099783038597587\\
599.18	0.00997876108681718\\
599.19	0.00997921476909742\\
599.2	0.00997966487204611\\
599.21	0.00998011136076955\\
599.22	0.00998055420003028\\
599.23	0.00998099335424369\\
599.24	0.00998142878747458\\
599.25	0.00998186046343367\\
599.26	0.00998228834359041\\
599.27	0.00998271238778643\\
599.28	0.00998313255546381\\
599.29	0.00998354880566111\\
599.3	0.00998396109700947\\
599.31	0.00998436938772849\\
599.32	0.00998477363562221\\
599.33	0.00998517379807499\\
599.34	0.00998556983204737\\
599.35	0.00998596169407188\\
599.36	0.00998634934024881\\
599.37	0.00998673272624192\\
599.38	0.00998711180727415\\
599.39	0.00998748653812326\\
599.4	0.00998785687311743\\
599.41	0.00998822276613081\\
599.42	0.00998858417057903\\
599.43	0.00998894103941468\\
599.44	0.00998929332512275\\
599.45	0.00998964097971596\\
599.46	0.00998998395473014\\
599.47	0.00999032220121951\\
599.48	0.0099906556697519\\
599.49	0.00999098431040396\\
599.5	0.00999130807275631\\
599.51	0.00999162690588863\\
599.52	0.00999194075837471\\
599.53	0.00999224957827746\\
599.54	0.00999255331314386\\
599.55	0.00999285190999987\\
599.56	0.00999314531534524\\
599.57	0.00999343347514839\\
599.58	0.00999371633484107\\
599.59	0.00999399383931314\\
599.6	0.00999426593290715\\
599.61	0.009994532559413\\
599.62	0.0099947936620624\\
599.63	0.00999504918352342\\
599.64	0.00999529906589491\\
599.65	0.00999554325070086\\
599.66	0.00999578167888471\\
599.67	0.00999601429080366\\
599.68	0.00999624102622283\\
599.69	0.00999646182430947\\
599.7	0.00999667662362697\\
599.71	0.00999688536212897\\
599.72	0.00999708797715331\\
599.73	0.00999728440541595\\
599.74	0.00999747458300482\\
599.75	0.0099976584453736\\
599.76	0.00999783592733551\\
599.77	0.00999800696305692\\
599.78	0.00999817148605102\\
599.79	0.00999832942917133\\
599.8	0.00999848072460517\\
599.81	0.00999862530386713\\
599.82	0.00999876309779239\\
599.83	0.00999889403653003\\
599.84	0.00999901804953622\\
599.85	0.00999913506556741\\
599.86	0.00999924501267341\\
599.87	0.00999934781819042\\
599.88	0.00999944340873394\\
599.89	0.0099995317101917\\
599.9	0.00999961264771648\\
599.91	0.00999968614571878\\
599.92	0.00999975212785957\\
599.93	0.00999981051704285\\
599.94	0.00999986123540817\\
599.95	0.00999990420432308\\
599.96	0.00999993934437554\\
599.97	0.00999996657536618\\
599.98	0.00999998581630055\\
599.99	0.00999999698538124\\
600	0.01\\
};
\end{axis}
\end{tikzpicture}%
  \caption{Continuous Time}
\end{subfigure}%
\hfill%
\begin{subfigure}{.45\linewidth}
  \centering
  \setlength\figureheight{\linewidth} 
  \setlength\figurewidth{\linewidth}
  \tikzsetnextfilename{dp_colorbar/dp_dscr_z8}
  % This file was created by matlab2tikz.
%
%The latest updates can be retrieved from
%  http://www.mathworks.com/matlabcentral/fileexchange/22022-matlab2tikz-matlab2tikz
%where you can also make suggestions and rate matlab2tikz.
%
\definecolor{mycolor1}{rgb}{0.00000,1.00000,0.14286}%
\definecolor{mycolor2}{rgb}{0.00000,1.00000,0.28571}%
\definecolor{mycolor3}{rgb}{0.00000,1.00000,0.42857}%
\definecolor{mycolor4}{rgb}{0.00000,1.00000,0.57143}%
\definecolor{mycolor5}{rgb}{0.00000,1.00000,0.71429}%
\definecolor{mycolor6}{rgb}{0.00000,1.00000,0.85714}%
\definecolor{mycolor7}{rgb}{0.00000,1.00000,1.00000}%
\definecolor{mycolor8}{rgb}{0.00000,0.87500,1.00000}%
\definecolor{mycolor9}{rgb}{0.00000,0.62500,1.00000}%
\definecolor{mycolor10}{rgb}{0.12500,0.00000,1.00000}%
\definecolor{mycolor11}{rgb}{0.25000,0.00000,1.00000}%
\definecolor{mycolor12}{rgb}{0.37500,0.00000,1.00000}%
\definecolor{mycolor13}{rgb}{0.50000,0.00000,1.00000}%
\definecolor{mycolor14}{rgb}{0.62500,0.00000,1.00000}%
\definecolor{mycolor15}{rgb}{0.75000,0.00000,1.00000}%
\definecolor{mycolor16}{rgb}{0.87500,0.00000,1.00000}%
\definecolor{mycolor17}{rgb}{1.00000,0.00000,1.00000}%
\definecolor{mycolor18}{rgb}{1.00000,0.00000,0.87500}%
\definecolor{mycolor19}{rgb}{1.00000,0.00000,0.62500}%
\definecolor{mycolor20}{rgb}{0.85714,0.00000,0.00000}%
\definecolor{mycolor21}{rgb}{0.71429,0.00000,0.00000}%
%
\begin{tikzpicture}

\begin{axis}[%
width=4.1in,
height=3.803in,
at={(0.809in,0.513in)},
scale only axis,
point meta min=0,
point meta max=1,
every outer x axis line/.append style={black},
every x tick label/.append style={font=\color{black}},
xmin=0,
xmax=600,
every outer y axis line/.append style={black},
every y tick label/.append style={font=\color{black}},
ymin=0,
ymax=0.01,
axis background/.style={fill=white},
axis x line*=bottom,
axis y line*=left,
colormap={mymap}{[1pt] rgb(0pt)=(0,1,0); rgb(7pt)=(0,1,1); rgb(15pt)=(0,0,1); rgb(23pt)=(1,0,1); rgb(31pt)=(1,0,0); rgb(38pt)=(0,0,0)},
colorbar,
colorbar style={separate axis lines,every outer x axis line/.append style={black},every x tick label/.append style={font=\color{black}},every outer y axis line/.append style={black},every y tick label/.append style={font=\color{black}},yticklabels={{-19},{-17},{-15},{-13},{-11},{-9},{-7},{-5},{-3},{-1},{1},{3},{5},{7},{9},{11},{13},{15},{17},{19}}}
]
\addplot [color=green,solid,forget plot]
  table[row sep=crcr]{%
1	0.00547780019338487\\
2	0.00547779836203247\\
3	0.00547779649665558\\
4	0.00547779459661717\\
5	0.00547779266126814\\
6	0.00547779068994721\\
7	0.00547778868198065\\
8	0.00547778663668193\\
9	0.00547778455335156\\
10	0.00547778243127691\\
11	0.00547778026973184\\
12	0.00547777806797653\\
13	0.00547777582525721\\
14	0.0054777735408058\\
15	0.00547777121383979\\
16	0.00547776884356175\\
17	0.00547776642915928\\
18	0.00547776396980463\\
19	0.00547776146465437\\
20	0.00547775891284912\\
21	0.00547775631351337\\
22	0.00547775366575492\\
23	0.00547775096866472\\
24	0.00547774822131646\\
25	0.00547774542276647\\
26	0.00547774257205312\\
27	0.00547773966819658\\
28	0.00547773671019848\\
29	0.00547773369704161\\
30	0.00547773062768945\\
31	0.00547772750108587\\
32	0.00547772431615465\\
33	0.00547772107179931\\
34	0.00547771776690257\\
35	0.00547771440032607\\
36	0.00547771097090967\\
37	0.00547770747747142\\
38	0.00547770391880689\\
39	0.00547770029368877\\
40	0.00547769660086674\\
41	0.00547769283906642\\
42	0.00547768900698951\\
43	0.00547768510331311\\
44	0.005477681126689\\
45	0.00547767707574356\\
46	0.00547767294907713\\
47	0.00547766874526338\\
48	0.00547766446284898\\
49	0.005477660100353\\
50	0.00547765565626646\\
51	0.00547765112905163\\
52	0.00547764651714156\\
53	0.00547764181893974\\
54	0.00547763703281917\\
55	0.00547763215712194\\
56	0.00547762719015876\\
57	0.00547762213020816\\
58	0.00547761697551608\\
59	0.00547761172429486\\
60	0.00547760637472321\\
61	0.00547760092494489\\
62	0.00547759537306858\\
63	0.00547758971716693\\
64	0.00547758395527595\\
65	0.00547757808539432\\
66	0.00547757210548267\\
67	0.00547756601346271\\
68	0.0054775598072167\\
69	0.00547755348458673\\
70	0.00547754704337352\\
71	0.00547754048133619\\
72	0.00547753379619117\\
73	0.00547752698561129\\
74	0.00547752004722511\\
75	0.0054775129786161\\
76	0.00547750577732178\\
77	0.0054774984408325\\
78	0.00547749096659111\\
79	0.00547748335199146\\
80	0.00547747559437789\\
81	0.00547746769104432\\
82	0.00547745963923266\\
83	0.00547745143613261\\
84	0.00547744307888013\\
85	0.00547743456455674\\
86	0.00547742589018804\\
87	0.00547741705274312\\
88	0.00547740804913314\\
89	0.00547739887621048\\
90	0.00547738953076738\\
91	0.00547738000953478\\
92	0.00547737030918148\\
93	0.00547736042631254\\
94	0.00547735035746826\\
95	0.00547734009912296\\
96	0.00547732964768352\\
97	0.00547731899948825\\
98	0.00547730815080578\\
99	0.00547729709783312\\
100	0.00547728583669486\\
101	0.0054772743634416\\
102	0.00547726267404826\\
103	0.0054772507644131\\
104	0.00547723863035594\\
105	0.00547722626761666\\
106	0.00547721367185374\\
107	0.00547720083864259\\
108	0.00547718776347421\\
109	0.00547717444175333\\
110	0.00547716086879675\\
111	0.00547714703983176\\
112	0.00547713294999438\\
113	0.00547711859432751\\
114	0.00547710396777942\\
115	0.00547708906520158\\
116	0.00547707388134694\\
117	0.00547705841086813\\
118	0.00547704264831543\\
119	0.00547702658813468\\
120	0.00547701022466556\\
121	0.00547699355213936\\
122	0.00547697656467691\\
123	0.00547695925628634\\
124	0.00547694162086135\\
125	0.00547692365217836\\
126	0.00547690534389472\\
127	0.00547688668954638\\
128	0.00547686768254531\\
129	0.00547684831617732\\
130	0.00547682858359955\\
131	0.00547680847783804\\
132	0.00547678799178508\\
133	0.00547676711819685\\
134	0.00547674584969067\\
135	0.00547672417874215\\
136	0.00547670209768282\\
137	0.00547667959869716\\
138	0.00547665667381978\\
139	0.00547663331493242\\
140	0.00547660951376113\\
141	0.00547658526187342\\
142	0.00547656055067495\\
143	0.00547653537140648\\
144	0.00547650971514078\\
145	0.00547648357277933\\
146	0.0054764569350489\\
147	0.00547642979249845\\
148	0.00547640213549555\\
149	0.00547637395422284\\
150	0.00547634523867466\\
151	0.00547631597865313\\
152	0.00547628616376466\\
153	0.0054762557834163\\
154	0.00547622482681152\\
155	0.00547619328294673\\
156	0.00547616114060697\\
157	0.00547612838836183\\
158	0.0054760950145617\\
159	0.00547606100733317\\
160	0.00547602635457487\\
161	0.00547599104395312\\
162	0.00547595506289734\\
163	0.0054759183985957\\
164	0.00547588103799036\\
165	0.0054758429677727\\
166	0.00547580417437866\\
167	0.0054757646439838\\
168	0.00547572436249831\\
169	0.00547568331556171\\
170	0.00547564148853824\\
171	0.00547559886651095\\
172	0.00547555543427678\\
173	0.00547551117634084\\
174	0.00547546607691097\\
175	0.00547542011989205\\
176	0.00547537328888027\\
177	0.00547532556715714\\
178	0.00547527693768367\\
179	0.00547522738309414\\
180	0.00547517688568993\\
181	0.00547512542743321\\
182	0.00547507298994053\\
183	0.00547501955447631\\
184	0.00547496510194596\\
185	0.00547490961288938\\
186	0.00547485306747379\\
187	0.00547479544548693\\
188	0.00547473672632967\\
189	0.00547467688900864\\
190	0.00547461591212905\\
191	0.00547455377388681\\
192	0.0054744904520609\\
193	0.00547442592400535\\
194	0.00547436016664137\\
195	0.00547429315644886\\
196	0.00547422486945841\\
197	0.00547415528124248\\
198	0.0054740843669055\\
199	0.00547401210107523\\
200	0.00547393845789294\\
201	0.00547386341100487\\
202	0.00547378693355224\\
203	0.00547370899816192\\
204	0.00547362957693638\\
205	0.00547354864144361\\
206	0.00547346616270701\\
207	0.00547338211119487\\
208	0.0054732964568095\\
209	0.00547320916887667\\
210	0.00547312021613425\\
211	0.0054730295667209\\
212	0.00547293718816452\\
213	0.0054728430473704\\
214	0.00547274711060898\\
215	0.00547264934350355\\
216	0.00547254971101772\\
217	0.00547244817744224\\
218	0.00547234470638199\\
219	0.0054722392607422\\
220	0.00547213180271464\\
221	0.00547202229376333\\
222	0.00547191069460997\\
223	0.00547179696521914\\
224	0.0054716810647825\\
225	0.00547156295170338\\
226	0.0054714425835802\\
227	0.00547131991719004\\
228	0.00547119490847116\\
229	0.00547106751250553\\
230	0.00547093768350016\\
231	0.00547080537476862\\
232	0.00547067053871114\\
233	0.00547053312679474\\
234	0.00547039308953201\\
235	0.00547025037645972\\
236	0.00547010493611618\\
237	0.00546995671601781\\
238	0.00546980566263517\\
239	0.00546965172136738\\
240	0.00546949483651604\\
241	0.0054693349512578\\
242	0.00546917200761569\\
243	0.00546900594642939\\
244	0.00546883670732378\\
245	0.0054686642286767\\
246	0.00546848844758473\\
247	0.00546830929982718\\
248	0.0054681267198294\\
249	0.00546794064062302\\
250	0.00546775099380574\\
251	0.00546755770949818\\
252	0.00546736071629964\\
253	0.00546715994124117\\
254	0.00546695530973744\\
255	0.00546674674553633\\
256	0.00546653417066647\\
257	0.00546631750538349\\
258	0.00546609666811435\\
259	0.00546587157540045\\
260	0.00546564214183956\\
261	0.00546540828002758\\
262	0.00546516990049957\\
263	0.00546492691167163\\
264	0.00546467921978485\\
265	0.00546442672885071\\
266	0.00546416934060098\\
267	0.00546390695444281\\
268	0.00546363946742059\\
269	0.00546336677418728\\
270	0.0054630887669867\\
271	0.00546280533565021\\
272	0.00546251636760866\\
273	0.00546222174792165\\
274	0.00546192135932096\\
275	0.00546161508225876\\
276	0.0054613027949394\\
277	0.00546098437332235\\
278	0.00546065969126471\\
279	0.00546032862209576\\
280	0.00545999103929571\\
281	0.00545964681381874\\
282	0.00545929581404179\\
283	0.00545893790571285\\
284	0.00545857295189752\\
285	0.00545820081292533\\
286	0.005457821346334\\
287	0.00545743440681342\\
288	0.00545703984614783\\
289	0.00545663751315714\\
290	0.00545622725363689\\
291	0.00545580891029723\\
292	0.00545538232270017\\
293	0.005454947327196\\
294	0.005454503756858\\
295	0.00545405144141609\\
296	0.00545359020718901\\
297	0.0054531198770149\\
298	0.00545264027018071\\
299	0.0054521512023498\\
300	0.00545165248548837\\
301	0.00545114392779012\\
302	0.00545062533359921\\
303	0.00545009650333209\\
304	0.00544955723339684\\
305	0.00544900731611164\\
306	0.00544844653962099\\
307	0.00544787468781012\\
308	0.00544729154021798\\
309	0.00544669687194783\\
310	0.0054460904535762\\
311	0.00544547205105978\\
312	0.00544484142564023\\
313	0.0054441983337471\\
314	0.00544354252689835\\
315	0.00544287375159881\\
316	0.00544219174923633\\
317	0.00544149625597578\\
318	0.00544078700265033\\
319	0.00544006371465047\\
320	0.0054393261118108\\
321	0.00543857390829369\\
322	0.0054378068124707\\
323	0.00543702452680095\\
324	0.00543622674770699\\
325	0.00543541316544767\\
326	0.00543458346398793\\
327	0.00543373732086547\\
328	0.0054328744070547\\
329	0.00543199438682682\\
330	0.00543109691760711\\
331	0.0054301816498285\\
332	0.00542924822678164\\
333	0.00542829628446135\\
334	0.00542732545140942\\
335	0.00542633534855325\\
336	0.00542532558904098\\
337	0.00542429577807218\\
338	0.00542324551272441\\
339	0.00542217438177564\\
340	0.00542108196552196\\
341	0.00541996783559111\\
342	0.00541883155475096\\
343	0.00541767267671351\\
344	0.00541649074593372\\
345	0.00541528529740356\\
346	0.00541405585644107\\
347	0.00541280193847348\\
348	0.00541152304881607\\
349	0.00541021868244466\\
350	0.00540888832376318\\
351	0.00540753144636543\\
352	0.00540614751279144\\
353	0.00540473597427764\\
354	0.00540329627050202\\
355	0.00540182782932307\\
356	0.00540033006651343\\
357	0.00539880238548735\\
358	0.00539724417702318\\
359	0.00539565481897998\\
360	0.00539403367600851\\
361	0.00539238009925782\\
362	0.00539069342607582\\
363	0.00538897297970559\\
364	0.00538721806897695\\
365	0.00538542798799348\\
366	0.00538360201581567\\
367	0.00538173941614002\\
368	0.00537983943697468\\
369	0.00537790131031196\\
370	0.00537592425179702\\
371	0.00537390746039434\\
372	0.00537185011805059\\
373	0.00536975138935397\\
374	0.0053676104211907\\
375	0.00536542634239583\\
376	0.0053631982633991\\
377	0.00536092527586302\\
378	0.00535860645231078\\
379	0.0053562408457403\\
380	0.00535382748921942\\
381	0.00535136539545445\\
382	0.00534885355632346\\
383	0.00534629094236728\\
384	0.00534367650224867\\
385	0.00534100916223877\\
386	0.00533828782582163\\
387	0.00533551137291279\\
388	0.00533267865854607\\
389	0.00532978851220988\\
390	0.00532683973715953\\
391	0.00532383110970253\\
392	0.00532076137845693\\
393	0.00531762926357949\\
394	0.0053144334559635\\
395	0.00531117261640312\\
396	0.00530784537472329\\
397	0.00530445032887193\\
398	0.00530098604397334\\
399	0.00529745105133906\\
400	0.0052938438474343\\
401	0.00529016289279669\\
402	0.00528640661090413\\
403	0.00528257338698845\\
404	0.00527866156679076\\
405	0.00527466945525515\\
406	0.00527059531515573\\
407	0.00526643736565234\\
408	0.0052621937807702\\
409	0.00525786268779764\\
410	0.00525344216559556\\
411	0.00524893024281307\\
412	0.00524432489600113\\
413	0.00523962404761742\\
414	0.00523482556391416\\
415	0.0052299272526993\\
416	0.00522492686096276\\
417	0.00521982207235701\\
418	0.00521461050452102\\
419	0.00520928970623665\\
420	0.00520385715440529\\
421	0.00519831025083372\\
422	0.00519264631881968\\
423	0.00518686259952791\\
424	0.00518095624813875\\
425	0.00517492432971706\\
426	0.00516876381481588\\
427	0.00516247157484287\\
428	0.00515604437712275\\
429	0.00514947887963317\\
430	0.00514277162539182\\
431	0.00513591903646833\\
432	0.00512891740759394\\
433	0.00512176289934059\\
434	0.00511445153083856\\
435	0.005106979171999\\
436	0.00509934153520695\\
437	0.00509153416644783\\
438	0.00508355243582635\\
439	0.00507539152743719\\
440	0.0050670464285423\\
441	0.00505851191800767\\
442	0.00504978255395004\\
443	0.00504085266054141\\
444	0.00503171631391671\\
445	0.00502236732712819\\
446	0.00501279923408703\\
447	0.00500300527243186\\
448	0.00499297836526077\\
449	0.00498271110166135\\
450	0.00497219571597312\\
451	0.00496142406572608\\
452	0.00495038760818735\\
453	0.0049390773754565\\
454	0.00492748394805629\\
455	0.004915597426971\\
456	0.00490340740409602\\
457	0.00489090293107581\\
458	0.00487807248652442\\
459	0.0048649039416482\\
460	0.00485138452431932\\
461	0.00483750078168705\\
462	0.00482323854146321\\
463	0.00480858287207625\\
464	0.0047935180419613\\
465	0.00477802747833561\\
466	0.00476209372589299\\
467	0.00474569840590054\\
468	0.00472882217609919\\
469	0.00471144469131011\\
470	0.00469354456297209\\
471	0.00467509931101878\\
472	0.00465608528883763\\
473	0.0046364775315969\\
474	0.00461624941539552\\
475	0.00459537193149305\\
476	0.00457381089568\\
477	0.00455151205921459\\
478	0.00452843715885329\\
479	0.00450455081417934\\
480	0.00447981684303206\\
481	0.00445419803128292\\
482	0.00442765630872568\\
483	0.00440015300515289\\
484	0.00437164919286085\\
485	0.00434210613156405\\
486	0.00431148583748201\\
487	0.00427975180317468\\
488	0.00424686989968002\\
489	0.00421280949829499\\
490	0.00417754485596785\\
491	0.00414105681581911\\
492	0.00410333488287071\\
493	0.00406437974484788\\
494	0.00402420631952435\\
495	0.00398284742522583\\
496	0.00394035819421055\\
497	0.00389682138917866\\
498	0.00385235385275624\\
499	0.00380711438400887\\
500	0.00376131301138053\\
501	0.00371521852825999\\
502	0.00366915875495774\\
503	0.00362356040215765\\
504	0.00357912772682816\\
505	0.0035387837955858\\
506	0.00350347152895402\\
507	0.00347337562112852\\
508	0.0034477007156403\\
509	0.00342321915374529\\
510	0.0033995558606626\\
511	0.00337624383061699\\
512	0.00335267912856479\\
513	0.00332876510735656\\
514	0.00330442430717686\\
515	0.00327960483711814\\
516	0.00325427310653981\\
517	0.00322840785924007\\
518	0.00320198625777614\\
519	0.00317498308822855\\
520	0.00314738455198443\\
521	0.00311917684391942\\
522	0.00309034579133459\\
523	0.00306087673592382\\
524	0.00303075439888956\\
525	0.00299996271838372\\
526	0.00296848470340479\\
527	0.00293630226281886\\
528	0.00290339600729777\\
529	0.00286974502351109\\
530	0.00283532660238758\\
531	0.00280011576890602\\
532	0.00276406939748543\\
533	0.00272712121191253\\
534	0.0026911965166374\\
535	0.00265655106163712\\
536	0.00262214033891497\\
537	0.00258722058161971\\
538	0.00255170106648381\\
539	0.00251557462242589\\
540	0.00247883838157633\\
541	0.00244149232061441\\
542	0.00240353787403877\\
543	0.00236497792929109\\
544	0.00232581699903547\\
545	0.00228606137899959\\
546	0.00224571921278943\\
547	0.0022048002200509\\
548	0.00216331439627576\\
549	0.00212127465109859\\
550	0.0020787029667516\\
551	0.00203562649819421\\
552	0.00199207845654765\\
553	0.00194934483622062\\
554	0.0019076605974282\\
555	0.00186553102545023\\
556	0.00182295585535523\\
557	0.00177995162674704\\
558	0.00173653654153092\\
559	0.00169273037654859\\
560	0.00164855438448207\\
561	0.0016040311335613\\
562	0.00155918428001908\\
563	0.00151403828525429\\
564	0.0014686181370549\\
565	0.0014229492435685\\
566	0.00137720469839254\\
567	0.00133166599318628\\
568	0.0012858376589304\\
569	0.00123974504677254\\
570	0.00119341585869796\\
571	0.00114688025839211\\
572	0.00110017097212519\\
573	0.00105332337464662\\
574	0.00100637555373387\\
575	0.000959368345375567\\
576	0.00091234532951171\\
577	0.000865352773713771\\
578	0.000818439509057376\\
579	0.000771656718587056\\
580	0.000725057614040402\\
581	0.000678696970709246\\
582	0.000632630483283924\\
583	0.000586913897120322\\
584	0.000541601859672152\\
585	0.000496746426604043\\
586	0.000452395149029594\\
587	0.000408588670365815\\
588	0.000365357795666475\\
589	0.00032272012337526\\
590	0.000280676711614416\\
591	0.000239312817487405\\
592	0.000198788644961841\\
593	0.000159293651685586\\
594	0.000121079888727804\\
595	8.45520570083908e-05\\
596	5.05092148680373e-05\\
597	2.07908715710836e-05\\
598	0\\
599	0\\
600	0\\
};
\addplot [color=mycolor1,solid,forget plot]
  table[row sep=crcr]{%
1	0.00548095199615839\\
2	0.00548094992070687\\
3	0.00548094780715951\\
4	0.00548094565481181\\
5	0.00548094346294612\\
6	0.00548094123083137\\
7	0.00548093895772277\\
8	0.00548093664286176\\
9	0.00548093428547568\\
10	0.00548093188477729\\
11	0.00548092943996483\\
12	0.00548092695022149\\
13	0.00548092441471523\\
14	0.00548092183259856\\
15	0.00548091920300802\\
16	0.00548091652506419\\
17	0.00548091379787118\\
18	0.00548091102051634\\
19	0.00548090819207006\\
20	0.00548090531158525\\
21	0.00548090237809721\\
22	0.00548089939062314\\
23	0.005480896348162\\
24	0.00548089324969399\\
25	0.00548089009418024\\
26	0.00548088688056246\\
27	0.00548088360776255\\
28	0.00548088027468232\\
29	0.00548087688020299\\
30	0.00548087342318489\\
31	0.0054808699024671\\
32	0.00548086631686685\\
33	0.00548086266517931\\
34	0.00548085894617706\\
35	0.00548085515860977\\
36	0.0054808513012037\\
37	0.0054808473726612\\
38	0.00548084337166045\\
39	0.00548083929685487\\
40	0.00548083514687246\\
41	0.00548083092031579\\
42	0.00548082661576104\\
43	0.00548082223175787\\
44	0.00548081776682869\\
45	0.00548081321946827\\
46	0.00548080858814314\\
47	0.00548080387129106\\
48	0.00548079906732057\\
49	0.00548079417461039\\
50	0.00548078919150867\\
51	0.00548078411633274\\
52	0.00548077894736843\\
53	0.00548077368286925\\
54	0.00548076832105606\\
55	0.0054807628601163\\
56	0.00548075729820337\\
57	0.00548075163343604\\
58	0.00548074586389776\\
59	0.00548073998763612\\
60	0.00548073400266186\\
61	0.00548072790694861\\
62	0.00548072169843166\\
63	0.00548071537500777\\
64	0.00548070893453403\\
65	0.00548070237482737\\
66	0.00548069569366378\\
67	0.00548068888877726\\
68	0.00548068195785944\\
69	0.00548067489855835\\
70	0.00548066770847799\\
71	0.00548066038517733\\
72	0.00548065292616929\\
73	0.00548064532892022\\
74	0.00548063759084875\\
75	0.00548062970932491\\
76	0.00548062168166931\\
77	0.00548061350515221\\
78	0.00548060517699245\\
79	0.00548059669435663\\
80	0.00548058805435795\\
81	0.00548057925405533\\
82	0.00548057029045237\\
83	0.00548056116049621\\
84	0.00548055186107659\\
85	0.00548054238902466\\
86	0.00548053274111202\\
87	0.00548052291404934\\
88	0.0054805129044855\\
89	0.00548050270900618\\
90	0.00548049232413272\\
91	0.00548048174632115\\
92	0.00548047097196047\\
93	0.00548045999737188\\
94	0.00548044881880714\\
95	0.00548043743244753\\
96	0.00548042583440221\\
97	0.00548041402070723\\
98	0.00548040198732382\\
99	0.00548038973013709\\
100	0.0054803772449546\\
101	0.00548036452750497\\
102	0.00548035157343624\\
103	0.00548033837831449\\
104	0.00548032493762212\\
105	0.00548031124675644\\
106	0.00548029730102794\\
107	0.00548028309565878\\
108	0.00548026862578087\\
109	0.00548025388643441\\
110	0.00548023887256618\\
111	0.00548022357902757\\
112	0.0054802080005728\\
113	0.00548019213185738\\
114	0.00548017596743578\\
115	0.00548015950175985\\
116	0.00548014272917685\\
117	0.00548012564392743\\
118	0.00548010824014352\\
119	0.00548009051184649\\
120	0.00548007245294482\\
121	0.00548005405723204\\
122	0.00548003531838467\\
123	0.00548001622996009\\
124	0.00547999678539385\\
125	0.00547997697799801\\
126	0.00547995680095824\\
127	0.00547993624733152\\
128	0.00547991531004419\\
129	0.00547989398188898\\
130	0.00547987225552268\\
131	0.00547985012346352\\
132	0.00547982757808871\\
133	0.00547980461163155\\
134	0.00547978121617879\\
135	0.00547975738366809\\
136	0.00547973310588491\\
137	0.00547970837445984\\
138	0.00547968318086549\\
139	0.00547965751641371\\
140	0.00547963137225254\\
141	0.00547960473936312\\
142	0.00547957760855628\\
143	0.0054795499704698\\
144	0.00547952181556483\\
145	0.00547949313412261\\
146	0.00547946391624125\\
147	0.00547943415183217\\
148	0.00547940383061655\\
149	0.00547937294212185\\
150	0.00547934147567809\\
151	0.00547930942041437\\
152	0.00547927676525471\\
153	0.00547924349891452\\
154	0.00547920960989667\\
155	0.00547917508648734\\
156	0.00547913991675218\\
157	0.00547910408853202\\
158	0.00547906758943872\\
159	0.00547903040685087\\
160	0.0054789925279096\\
161	0.0054789539395139\\
162	0.00547891462831629\\
163	0.00547887458071825\\
164	0.00547883378286538\\
165	0.00547879222064283\\
166	0.0054787498796703\\
167	0.00547870674529745\\
168	0.00547866280259838\\
169	0.00547861803636722\\
170	0.00547857243111221\\
171	0.00547852597105114\\
172	0.0054784786401056\\
173	0.00547843042189563\\
174	0.00547838129973434\\
175	0.0054783312566222\\
176	0.00547828027524145\\
177	0.00547822833795021\\
178	0.00547817542677671\\
179	0.00547812152341338\\
180	0.00547806660921088\\
181	0.0054780106651719\\
182	0.00547795367194498\\
183	0.00547789560981841\\
184	0.00547783645871397\\
185	0.00547777619817999\\
186	0.00547771480738572\\
187	0.00547765226511421\\
188	0.00547758854975598\\
189	0.00547752363930238\\
190	0.00547745751133886\\
191	0.00547739014303824\\
192	0.00547732151115371\\
193	0.00547725159201235\\
194	0.00547718036150766\\
195	0.00547710779509384\\
196	0.00547703386778093\\
197	0.00547695855413523\\
198	0.00547688182828359\\
199	0.00547680366388242\\
200	0.00547672403410897\\
201	0.00547664291165308\\
202	0.00547656026870942\\
203	0.00547647607696921\\
204	0.00547639030761201\\
205	0.00547630293129755\\
206	0.00547621391815725\\
207	0.00547612323778595\\
208	0.00547603085923324\\
209	0.00547593675099499\\
210	0.00547584088100468\\
211	0.00547574321662483\\
212	0.00547564372463806\\
213	0.0054755423712384\\
214	0.00547543912202255\\
215	0.00547533394198096\\
216	0.00547522679548875\\
217	0.00547511764629692\\
218	0.00547500645752335\\
219	0.00547489319164394\\
220	0.00547477781048339\\
221	0.00547466027520632\\
222	0.00547454054630832\\
223	0.00547441858360674\\
224	0.00547429434623182\\
225	0.00547416779261774\\
226	0.00547403888049354\\
227	0.00547390756687405\\
228	0.00547377380805125\\
229	0.00547363755958502\\
230	0.00547349877629423\\
231	0.00547335741224781\\
232	0.005473213420756\\
233	0.00547306675436082\\
234	0.00547291736482763\\
235	0.0054727652031355\\
236	0.00547261021946829\\
237	0.00547245236320512\\
238	0.0054722915829106\\
239	0.00547212782632512\\
240	0.00547196104035452\\
241	0.00547179117105935\\
242	0.00547161816364347\\
243	0.00547144196244214\\
244	0.00547126251090899\\
245	0.00547107975160189\\
246	0.00547089362616763\\
247	0.00547070407532488\\
248	0.00547051103884505\\
249	0.00547031445553121\\
250	0.00547011426319354\\
251	0.00546991039862183\\
252	0.00546970279755407\\
253	0.00546949139464002\\
254	0.00546927612339942\\
255	0.00546905691617357\\
256	0.0054688337040694\\
257	0.00546860641689464\\
258	0.00546837498308231\\
259	0.00546813932960382\\
260	0.00546789938186764\\
261	0.00546765506360236\\
262	0.00546740629672147\\
263	0.00546715300116732\\
264	0.00546689509473118\\
265	0.00546663249284685\\
266	0.00546636510835429\\
267	0.00546609285122961\\
268	0.00546581562827825\\
269	0.00546553334278746\\
270	0.00546524589413507\\
271	0.00546495317735124\\
272	0.00546465508262966\\
273	0.0054643514947835\\
274	0.00546404229263197\\
275	0.00546372734826881\\
276	0.00546340652603965\\
277	0.0054630796806094\\
278	0.00546274665192147\\
279	0.00546240724915325\\
280	0.00546206128609614\\
281	0.00546170863564718\\
282	0.00546134916823953\\
283	0.00546098275179508\\
284	0.00546060925167596\\
285	0.00546022853063473\\
286	0.00545984044876465\\
287	0.00545944486344781\\
288	0.00545904162930304\\
289	0.00545863059813262\\
290	0.00545821161886802\\
291	0.00545778453751446\\
292	0.00545734919709457\\
293	0.00545690543759074\\
294	0.00545645309588682\\
295	0.00545599200570826\\
296	0.00545552199756122\\
297	0.00545504289867078\\
298	0.0054545545329177\\
299	0.0054540567207738\\
300	0.00545354927923665\\
301	0.00545303202176261\\
302	0.00545250475819857\\
303	0.00545196729471259\\
304	0.00545141943372308\\
305	0.00545086097382662\\
306	0.00545029170972414\\
307	0.00544971143214647\\
308	0.00544911992777732\\
309	0.00544851697917555\\
310	0.00544790236469576\\
311	0.00544727585840716\\
312	0.0054466372300107\\
313	0.00544598624475502\\
314	0.00544532266335037\\
315	0.00544464624188075\\
316	0.00544395673171442\\
317	0.00544325387941254\\
318	0.00544253742663585\\
319	0.00544180711004963\\
320	0.00544106266122603\\
321	0.00544030380654516\\
322	0.00543953026709337\\
323	0.00543874175855965\\
324	0.00543793799112967\\
325	0.00543711866937707\\
326	0.0054362834921529\\
327	0.00543543215247208\\
328	0.005434564337397\\
329	0.00543367972791881\\
330	0.00543277799883512\\
331	0.00543185881862527\\
332	0.0054309218493221\\
333	0.00542996674638014\\
334	0.00542899315854046\\
335	0.00542800072769201\\
336	0.00542698908872839\\
337	0.00542595786940121\\
338	0.0054249066901685\\
339	0.00542383516403881\\
340	0.00542274289641018\\
341	0.0054216294849042\\
342	0.00542049451919431\\
343	0.00541933758082834\\
344	0.00541815824304497\\
345	0.00541695607058338\\
346	0.00541573061948593\\
347	0.00541448143689398\\
348	0.00541320806083489\\
349	0.00541191002000126\\
350	0.00541058683352134\\
351	0.00540923801071954\\
352	0.00540786305086729\\
353	0.00540646144292367\\
354	0.00540503266526402\\
355	0.00540357618539771\\
356	0.00540209145967306\\
357	0.00540057793296957\\
358	0.0053990350383768\\
359	0.00539746219685888\\
360	0.00539585881690556\\
361	0.00539422429416723\\
362	0.00539255801107627\\
363	0.00539085933645289\\
364	0.00538912762509577\\
365	0.0053873622173592\\
366	0.00538556243871585\\
367	0.00538372759930749\\
368	0.00538185699348492\\
369	0.0053799498993387\\
370	0.00537800557822563\\
371	0.00537602327429234\\
372	0.0053740022140025\\
373	0.00537194160567334\\
374	0.00536984063902765\\
375	0.00536769848477218\\
376	0.00536551429421052\\
377	0.00536328719890381\\
378	0.00536101631039253\\
379	0.00535870071999319\\
380	0.00535633949868242\\
381	0.00535393169707442\\
382	0.00535147634547554\\
383	0.00534897245395576\\
384	0.00534641901230657\\
385	0.00534381498980157\\
386	0.00534115933575377\\
387	0.00533845098860062\\
388	0.00533568888069135\\
389	0.00533287192107836\\
390	0.00532999899488533\\
391	0.00532706896265217\\
392	0.00532408065965528\\
393	0.00532103289520269\\
394	0.00531792445190184\\
395	0.00531475408490083\\
396	0.00531152052110063\\
397	0.00530822245833811\\
398	0.00530485856453716\\
399	0.00530142747682666\\
400	0.00529792780062444\\
401	0.00529435810868445\\
402	0.00529071694010634\\
403	0.00528700279930482\\
404	0.00528321415493732\\
405	0.00527934943878715\\
406	0.00527540704460026\\
407	0.00527138532687276\\
408	0.00526728259958638\\
409	0.00526309713488923\\
410	0.00525882716171782\\
411	0.00525447086435603\\
412	0.0052500263809267\\
413	0.00524549180180995\\
414	0.00524086516798206\\
415	0.00523614446926672\\
416	0.00523132764248807\\
417	0.00522641256951514\\
418	0.00522139707518676\\
419	0.00521627892509709\\
420	0.00521105582321913\\
421	0.00520572540934116\\
422	0.00520028525629376\\
423	0.00519473286698471\\
424	0.0051890656713553\\
425	0.00518328102341414\\
426	0.00517737619710569\\
427	0.00517134838138553\\
428	0.0051651946761483\\
429	0.00515891208789345\\
430	0.00515249752510843\\
431	0.00514594779334966\\
432	0.00513925959000025\\
433	0.00513242949867982\\
434	0.00512545398328218\\
435	0.00511832938161421\\
436	0.00511105189860727\\
437	0.00510361759907029\\
438	0.00509602239995316\\
439	0.00508826206208469\\
440	0.00508033218134922\\
441	0.00507222817926275\\
442	0.00506394529290762\\
443	0.00505547856418199\\
444	0.005046822828319\\
445	0.00503797270162797\\
446	0.00502892256840949\\
447	0.00501966656699718\\
448	0.00501019857488201\\
449	0.00500051219286966\\
450	0.00499060072818567\\
451	0.0049804571764038\\
452	0.00497007420233899\\
453	0.00495944411973041\\
454	0.00494855886965451\\
455	0.00493740999761189\\
456	0.00492598862923287\\
457	0.00491428544455216\\
458	0.00490229065080857\\
459	0.00488999395373417\\
460	0.00487738452730974\\
461	0.00486445098197741\\
462	0.00485118133132173\\
463	0.00483756295725451\\
464	0.00482358257377012\\
465	0.00480922618937484\\
466	0.00479447906833696\\
467	0.00477932569094731\\
468	0.00476374971300784\\
469	0.00474773392474334\\
470	0.00473126020916734\\
471	0.00471430949936953\\
472	0.00469686173269478\\
473	0.00467889579597215\\
474	0.00466038944590183\\
475	0.0046413191586662\\
476	0.00462165983497246\\
477	0.0046013846366147\\
478	0.00458046289322046\\
479	0.00455885105141006\\
480	0.00453650300946854\\
481	0.00451338434121667\\
482	0.00448945977358611\\
483	0.00446469277837288\\
484	0.0044390455330161\\
485	0.00441247901676345\\
486	0.00438495317490672\\
487	0.00435642714567625\\
488	0.00432685956560797\\
489	0.00429620897197185\\
490	0.00426443432577805\\
491	0.00423149568450701\\
492	0.00419735506082228\\
493	0.00416197751252801\\
494	0.00412533252053638\\
495	0.00408739572641924\\
496	0.00404815112021496\\
497	0.00400759379329441\\
498	0.00396573339945054\\
499	0.00392259849595016\\
500	0.00387824197248343\\
501	0.00383274799840133\\
502	0.00378624154179356\\
503	0.00373890021273193\\
504	0.00369096392134772\\
505	0.00364273454391488\\
506	0.00359457071864176\\
507	0.00354702042024983\\
508	0.00350131809110213\\
509	0.00346061049916525\\
510	0.00342502183026241\\
511	0.00339439861452269\\
512	0.00336708322036617\\
513	0.00334086691234794\\
514	0.00331534730435857\\
515	0.00329008181911496\\
516	0.00326469884107946\\
517	0.00323892834194342\\
518	0.00321269198886212\\
519	0.00318593354574616\\
520	0.00315860607848959\\
521	0.00313068753007561\\
522	0.00310216113399003\\
523	0.00307301064759331\\
524	0.00304322031745423\\
525	0.00301277483353352\\
526	0.00298165833580482\\
527	0.00294985424466898\\
528	0.00291734506982389\\
529	0.00288411216856215\\
530	0.00285013547640853\\
531	0.00281539321418112\\
532	0.00277986195570541\\
533	0.00274351740193557\\
534	0.00270633319570252\\
535	0.00266822518666434\\
536	0.00263037485873978\\
537	0.0025936689644631\\
538	0.00255802540455848\\
539	0.00252190230147721\\
540	0.00248520925445124\\
541	0.00244790403567086\\
542	0.00240998217519735\\
543	0.00237144483485123\\
544	0.00233229481254299\\
545	0.00229253661599072\\
546	0.0022521764177109\\
547	0.00221122188822311\\
548	0.00216968132057519\\
549	0.00212756246687727\\
550	0.00208488317727384\\
551	0.00204166550220163\\
552	0.00199793669477371\\
553	0.0019537309880139\\
554	0.00190965922034648\\
555	0.00186720834870374\\
556	0.00182456308654889\\
557	0.00178148227067175\\
558	0.0017379831919425\\
559	0.0016940856717289\\
560	0.00164981119638615\\
561	0.00160518279977741\\
562	0.00156022488653992\\
563	0.00151496298230589\\
564	0.00146942342078429\\
565	0.00142363302680362\\
566	0.00137761899412085\\
567	0.00133166599321896\\
568	0.00128583765893305\\
569	0.0012397450467737\\
570	0.00119341585869848\\
571	0.00114688025839233\\
572	0.00110017097212528\\
573	0.00105332337464666\\
574	0.00100637555373387\\
575	0.000959368345375562\\
576	0.000912345329511703\\
577	0.000865352773713766\\
578	0.000818439509057374\\
579	0.00077165671858706\\
580	0.00072505761404041\\
581	0.000678696970709248\\
582	0.000632630483283932\\
583	0.000586913897120327\\
584	0.000541601859672158\\
585	0.00049674642660404\\
586	0.000452395149029591\\
587	0.000408588670365814\\
588	0.000365357795666474\\
589	0.000322720123375262\\
590	0.000280676711614417\\
591	0.000239312817487403\\
592	0.000198788644961841\\
593	0.000159293651685586\\
594	0.000121079888727805\\
595	8.45520570083909e-05\\
596	5.05092148680371e-05\\
597	2.07908715710836e-05\\
598	0\\
599	0\\
600	0\\
};
\addplot [color=mycolor2,solid,forget plot]
  table[row sep=crcr]{%
1	0.0054891968694182\\
2	0.00548919435537706\\
3	0.00548919179592685\\
4	0.00548918919024169\\
5	0.00548918653748061\\
6	0.00548918383678725\\
7	0.00548918108728961\\
8	0.00548917828809958\\
9	0.00548917543831285\\
10	0.00548917253700857\\
11	0.00548916958324894\\
12	0.00548916657607902\\
13	0.00548916351452631\\
14	0.00548916039760048\\
15	0.00548915722429305\\
16	0.00548915399357712\\
17	0.00548915070440682\\
18	0.00548914735571718\\
19	0.00548914394642359\\
20	0.0054891404754216\\
21	0.00548913694158649\\
22	0.00548913334377289\\
23	0.00548912968081434\\
24	0.00548912595152302\\
25	0.00548912215468925\\
26	0.00548911828908118\\
27	0.00548911435344429\\
28	0.00548911034650102\\
29	0.0054891062669503\\
30	0.00548910211346729\\
31	0.00548909788470254\\
32	0.005489093579282\\
33	0.00548908919580631\\
34	0.00548908473285036\\
35	0.0054890801889628\\
36	0.00548907556266564\\
37	0.00548907085245371\\
38	0.00548906605679403\\
39	0.00548906117412548\\
40	0.0054890562028583\\
41	0.00548905114137334\\
42	0.00548904598802173\\
43	0.00548904074112418\\
44	0.00548903539897048\\
45	0.00548902995981902\\
46	0.00548902442189594\\
47	0.00548901878339485\\
48	0.00548901304247607\\
49	0.00548900719726596\\
50	0.00548900124585643\\
51	0.00548899518630413\\
52	0.00548898901662991\\
53	0.0054889827348182\\
54	0.00548897633881622\\
55	0.00548896982653331\\
56	0.00548896319584039\\
57	0.00548895644456892\\
58	0.00548894957051057\\
59	0.005488942571416\\
60	0.00548893544499462\\
61	0.00548892818891339\\
62	0.00548892080079643\\
63	0.00548891327822374\\
64	0.00548890561873086\\
65	0.00548889781980776\\
66	0.00548888987889793\\
67	0.00548888179339792\\
68	0.00548887356065595\\
69	0.00548886517797149\\
70	0.00548885664259395\\
71	0.00548884795172195\\
72	0.0054888391025024\\
73	0.00548883009202948\\
74	0.00548882091734358\\
75	0.00548881157543053\\
76	0.0054888020632204\\
77	0.00548879237758641\\
78	0.00548878251534416\\
79	0.00548877247325024\\
80	0.00548876224800123\\
81	0.00548875183623264\\
82	0.00548874123451786\\
83	0.0054887304393668\\
84	0.00548871944722482\\
85	0.00548870825447156\\
86	0.00548869685741956\\
87	0.00548868525231328\\
88	0.00548867343532759\\
89	0.00548866140256658\\
90	0.00548864915006227\\
91	0.00548863667377318\\
92	0.00548862396958306\\
93	0.00548861103329949\\
94	0.00548859786065239\\
95	0.00548858444729253\\
96	0.00548857078879037\\
97	0.00548855688063409\\
98	0.00548854271822849\\
99	0.00548852829689328\\
100	0.0054885136118614\\
101	0.00548849865827747\\
102	0.00548848343119637\\
103	0.00548846792558115\\
104	0.00548845213630171\\
105	0.00548843605813288\\
106	0.0054884196857527\\
107	0.00548840301374054\\
108	0.00548838603657553\\
109	0.0054883687486344\\
110	0.00548835114418964\\
111	0.00548833321740775\\
112	0.00548831496234715\\
113	0.00548829637295601\\
114	0.00548827744307067\\
115	0.00548825816641306\\
116	0.00548823853658886\\
117	0.00548821854708523\\
118	0.00548819819126881\\
119	0.00548817746238323\\
120	0.00548815635354697\\
121	0.00548813485775122\\
122	0.00548811296785711\\
123	0.0054880906765936\\
124	0.00548806797655505\\
125	0.00548804486019857\\
126	0.00548802131984158\\
127	0.00548799734765942\\
128	0.00548797293568226\\
129	0.00548794807579285\\
130	0.00548792275972361\\
131	0.00548789697905397\\
132	0.00548787072520739\\
133	0.00548784398944858\\
134	0.00548781676288066\\
135	0.00548778903644202\\
136	0.00548776080090342\\
137	0.00548773204686483\\
138	0.00548770276475235\\
139	0.00548767294481489\\
140	0.00548764257712121\\
141	0.00548761165155607\\
142	0.00548758015781746\\
143	0.00548754808541279\\
144	0.00548751542365553\\
145	0.00548748216166155\\
146	0.00548744828834564\\
147	0.00548741379241772\\
148	0.00548737866237903\\
149	0.00548734288651851\\
150	0.00548730645290866\\
151	0.00548726934940173\\
152	0.00548723156362574\\
153	0.00548719308298007\\
154	0.00548715389463163\\
155	0.00548711398551041\\
156	0.00548707334230508\\
157	0.00548703195145868\\
158	0.00548698979916422\\
159	0.0054869468713599\\
160	0.00548690315372438\\
161	0.00548685863167226\\
162	0.0054868132903491\\
163	0.00548676711462645\\
164	0.00548672008909692\\
165	0.00548667219806894\\
166	0.00548662342556174\\
167	0.00548657375529968\\
168	0.00548652317070723\\
169	0.00548647165490311\\
170	0.00548641919069484\\
171	0.00548636576057296\\
172	0.00548631134670515\\
173	0.00548625593093042\\
174	0.00548619949475278\\
175	0.00548614201933525\\
176	0.00548608348549331\\
177	0.00548602387368881\\
178	0.00548596316402304\\
179	0.00548590133623029\\
180	0.00548583836967079\\
181	0.00548577424332394\\
182	0.00548570893578119\\
183	0.00548564242523852\\
184	0.00548557468948928\\
185	0.00548550570591667\\
186	0.00548543545148583\\
187	0.00548536390273596\\
188	0.00548529103577244\\
189	0.00548521682625827\\
190	0.00548514124940548\\
191	0.0054850642799659\\
192	0.00548498589222079\\
193	0.00548490605996795\\
194	0.00548482475650521\\
195	0.00548474195460491\\
196	0.00548465762647844\\
197	0.0054845717437398\\
198	0.00548448427744685\\
199	0.00548439519858112\\
200	0.0054843044776251\\
201	0.00548421208453498\\
202	0.00548411798873232\\
203	0.00548402215909528\\
204	0.00548392456395014\\
205	0.00548382517106224\\
206	0.00548372394762747\\
207	0.00548362086026317\\
208	0.0054835158749993\\
209	0.00548340895726937\\
210	0.00548330007190167\\
211	0.00548318918310984\\
212	0.0054830762544842\\
213	0.00548296124898257\\
214	0.00548284412892102\\
215	0.00548272485596526\\
216	0.00548260339112148\\
217	0.00548247969472741\\
218	0.0054823537264438\\
219	0.00548222544524543\\
220	0.00548209480941283\\
221	0.00548196177652399\\
222	0.00548182630344604\\
223	0.00548168834632761\\
224	0.00548154786059131\\
225	0.00548140480092645\\
226	0.00548125912128235\\
227	0.00548111077486217\\
228	0.00548095971411708\\
229	0.00548080589074135\\
230	0.00548064925566799\\
231	0.00548048975906534\\
232	0.00548032735033443\\
233	0.00548016197810786\\
234	0.00547999359024942\\
235	0.00547982213385538\\
236	0.00547964755525738\\
237	0.00547946980002722\\
238	0.00547928881298338\\
239	0.00547910453820034\\
240	0.00547891691901986\\
241	0.00547872589806595\\
242	0.00547853141726252\\
243	0.00547833341785528\\
244	0.00547813184043755\\
245	0.00547792662498099\\
246	0.00547771771087146\\
247	0.00547750503695109\\
248	0.00547728854156713\\
249	0.00547706816262813\\
250	0.00547684383766866\\
251	0.00547661550392378\\
252	0.00547638309841347\\
253	0.00547614655803969\\
254	0.00547590581969587\\
255	0.00547566082039141\\
256	0.0054754114973923\\
257	0.0054751577883796\\
258	0.00547489963162761\\
259	0.00547463696620386\\
260	0.00547436973219279\\
261	0.00547409787094485\\
262	0.00547382132535371\\
263	0.00547354004016329\\
264	0.00547325396230583\\
265	0.00547296304127366\\
266	0.00547266722952463\\
267	0.00547236648292219\\
268	0.00547206076120906\\
269	0.00547175002851286\\
270	0.00547143425387946\\
271	0.00547111341182841\\
272	0.00547078748292199\\
273	0.00547045645433792\\
274	0.00547012032043892\\
275	0.00546977908334958\\
276	0.00546943275361923\\
277	0.00546908135127342\\
278	0.00546872490828844\\
279	0.00546836347592409\\
280	0.00546799639134473\\
281	0.00546762240864155\\
282	0.00546724139884568\\
283	0.00546685323060331\\
284	0.00546645777013235\\
285	0.00546605488117901\\
286	0.00546564442497286\\
287	0.00546522626018168\\
288	0.00546480024286546\\
289	0.00546436622642938\\
290	0.00546392406157631\\
291	0.00546347359625828\\
292	0.00546301467562746\\
293	0.00546254714198608\\
294	0.0054620708347355\\
295	0.00546158559032467\\
296	0.00546109124219773\\
297	0.00546058762074055\\
298	0.00546007455322658\\
299	0.00545955186376227\\
300	0.00545901937323075\\
301	0.00545847689923528\\
302	0.00545792425604187\\
303	0.00545736125452067\\
304	0.00545678770208683\\
305	0.00545620340264023\\
306	0.00545560815650464\\
307	0.00545500176036559\\
308	0.0054543840072078\\
309	0.00545375468625155\\
310	0.0054531135828881\\
311	0.00545246047861439\\
312	0.00545179515096701\\
313	0.00545111737345481\\
314	0.00545042691549136\\
315	0.00544972354232614\\
316	0.00544900701497484\\
317	0.0054482770901493\\
318	0.00544753352018626\\
319	0.00544677605297551\\
320	0.00544600443188722\\
321	0.00544521839569857\\
322	0.00544441767851983\\
323	0.00544360200971929\\
324	0.00544277111384802\\
325	0.00544192471056379\\
326	0.00544106251455431\\
327	0.00544018423545993\\
328	0.00543928957779588\\
329	0.00543837824087373\\
330	0.00543744991872266\\
331	0.00543650430001013\\
332	0.00543554106796177\\
333	0.0054345599002812\\
334	0.00543356046906949\\
335	0.00543254244074374\\
336	0.00543150547595554\\
337	0.00543044922950901\\
338	0.00542937335027815\\
339	0.00542827748112374\\
340	0.00542716125880973\\
341	0.00542602431391878\\
342	0.00542486627076723\\
343	0.00542368674731854\\
344	0.00542248535509619\\
345	0.00542126169909461\\
346	0.00542001537768867\\
347	0.00541874598254035\\
348	0.00541745309850346\\
349	0.00541613630352446\\
350	0.00541479516853939\\
351	0.00541342925736604\\
352	0.00541203812658974\\
353	0.00541062132544212\\
354	0.00540917839567121\\
355	0.00540770887140096\\
356	0.00540621227897795\\
357	0.00540468813680368\\
358	0.00540313595514848\\
359	0.00540155523594473\\
360	0.00539994547255471\\
361	0.00539830614950943\\
362	0.00539663674221262\\
363	0.00539493671660397\\
364	0.00539320552877565\\
365	0.00539144262453276\\
366	0.00538964743889032\\
367	0.00538781939549552\\
368	0.0053859579059644\\
369	0.00538406236912007\\
370	0.00538213217011732\\
371	0.00538016667943867\\
372	0.00537816525174382\\
373	0.00537612722455306\\
374	0.00537405191674505\\
375	0.00537193862684648\\
376	0.00536978663109177\\
377	0.00536759518123094\\
378	0.00536536350206276\\
379	0.00536309078867397\\
380	0.00536077620336287\\
381	0.00535841887221443\\
382	0.00535601788124313\\
383	0.00535357227181194\\
384	0.00535108103429341\\
385	0.0053485430962889\\
386	0.00534595729230035\\
387	0.00534332226783449\\
388	0.00534063667519442\\
389	0.00533789953008658\\
390	0.00533510982741454\\
391	0.0053322665406314\\
392	0.00532936862105515\\
393	0.00532641499714516\\
394	0.00532340457373629\\
395	0.00532033623122087\\
396	0.00531720882469156\\
397	0.00531402118304865\\
398	0.00531077210806497\\
399	0.00530746037340651\\
400	0.00530408472360693\\
401	0.00530064387299334\\
402	0.0052971365045622\\
403	0.00529356126880294\\
404	0.00528991678246438\\
405	0.00528620162727035\\
406	0.00528241434857942\\
407	0.00527855345397782\\
408	0.00527461741181547\\
409	0.0052706046496879\\
410	0.00526651355286785\\
411	0.00526234246269237\\
412	0.00525808967491305\\
413	0.00525375343801949\\
414	0.00524933195154946\\
415	0.00524482336440346\\
416	0.00524022577318039\\
417	0.00523553722053481\\
418	0.00523075569352813\\
419	0.00522587912211942\\
420	0.00522090537774521\\
421	0.00521583227193822\\
422	0.00521065755485279\\
423	0.00520537891343096\\
424	0.00519999396908422\\
425	0.00519450027717447\\
426	0.00518889534523272\\
427	0.00518317663173477\\
428	0.00517734151593173\\
429	0.00517138729411216\\
430	0.00516531117563831\\
431	0.00515911027874116\\
432	0.00515278162605946\\
433	0.00514632213990524\\
434	0.00513972863723734\\
435	0.00513299782432283\\
436	0.00512612629106462\\
437	0.0051191105049721\\
438	0.00511194680474882\\
439	0.00510463139346928\\
440	0.00509716033131396\\
441	0.0050895295278284\\
442	0.00508173473366834\\
443	0.00507377153178841\\
444	0.00506563532802464\\
445	0.00505732134101474\\
446	0.00504882459138982\\
447	0.00504013989016983\\
448	0.00503126182632091\\
449	0.00502218475354018\\
450	0.00501290277652464\\
451	0.00500340973646137\\
452	0.0049936991921593\\
453	0.00498376440288637\\
454	0.00497359831009461\\
455	0.00496319351796841\\
456	0.00495254227273068\\
457	0.00494163644064148\\
458	0.00493046748462266\\
459	0.0049190264394442\\
460	0.00490730388540931\\
461	0.00489528992047967\\
462	0.00488297413078798\\
463	0.00487034555949376\\
464	0.00485739267395021\\
465	0.004844103331166\\
466	0.00483046474156964\\
467	0.00481646343111338\\
468	0.00480208520178408\\
469	0.00478731509058795\\
470	0.00477213732701297\\
471	0.00475653528923159\\
472	0.00474049145901888\\
473	0.00472398737499751\\
474	0.00470700358311135\\
475	0.00468951958190851\\
476	0.00467151375681195\\
477	0.00465296326746879\\
478	0.00463384386379001\\
479	0.00461412987626572\\
480	0.00459379379728429\\
481	0.00457280301484697\\
482	0.00455110433656644\\
483	0.00452865953063785\\
484	0.00450543286302288\\
485	0.00448138762584243\\
486	0.0044564855239955\\
487	0.00443068671594453\\
488	0.00440394987873579\\
489	0.00437623233356901\\
490	0.00434749022961062\\
491	0.00431767880121355\\
492	0.00428675271652006\\
493	0.00425466653997079\\
494	0.00422137533634864\\
495	0.00418683545021244\\
496	0.00415100550209224\\
497	0.00411384765185985\\
498	0.00407532919063057\\
499	0.00403542453652763\\
500	0.00399411772952133\\
501	0.00395140554229041\\
502	0.00390730132316084\\
503	0.00386183971978261\\
504	0.00381508261869486\\
505	0.00376712721209702\\
506	0.00371811743437225\\
507	0.0036682546917953\\
508	0.00361780571383835\\
509	0.00356710002107475\\
510	0.00351655779440234\\
511	0.00346685990014391\\
512	0.00342018933600302\\
513	0.00337865550221436\\
514	0.00334233331140309\\
515	0.00331092069354877\\
516	0.00328172789297185\\
517	0.00325356226126486\\
518	0.00322599612173834\\
519	0.00319856266008969\\
520	0.00317113359201397\\
521	0.00314330028321649\\
522	0.00311494865759424\\
523	0.00308604010602848\\
524	0.00305653561648579\\
525	0.00302639862942524\\
526	0.00299561110095096\\
527	0.00296415546625934\\
528	0.00293201443441139\\
529	0.00289917111057011\\
530	0.00286560835076743\\
531	0.00283130815884521\\
532	0.00279625143981421\\
533	0.0027604176819128\\
534	0.00272378461722965\\
535	0.00268632921308865\\
536	0.00264802742428661\\
537	0.00260881862072326\\
538	0.00256896387323768\\
539	0.00253012203665061\\
540	0.00249255684260618\\
541	0.00245517426589838\\
542	0.00241727949581605\\
543	0.00237877084369842\\
544	0.00233964426046532\\
545	0.00229989957797116\\
546	0.00225954118691839\\
547	0.00221857506323328\\
548	0.00217700799179227\\
549	0.00213484522472573\\
550	0.00209210022016971\\
551	0.00204879104734812\\
552	0.00200493976528336\\
553	0.00196057319704059\\
554	0.00191572431648802\\
555	0.00187042054078887\\
556	0.00182648374393835\\
557	0.00178330361132209\\
558	0.00173971373924418\\
559	0.00169571747444497\\
560	0.00165133624243625\\
561	0.00160659337207\\
562	0.00156151382896341\\
563	0.00151612401849032\\
564	0.00147045151570613\\
565	0.00142452473202305\\
566	0.00137837259118908\\
567	0.00133202429968801\\
568	0.00128583765932925\\
569	0.00123974504679409\\
570	0.00119341585870728\\
571	0.00114688025839642\\
572	0.0011001709721271\\
573	0.00105332337464741\\
574	0.00100637555373416\\
575	0.000959368345375665\\
576	0.000912345329511744\\
577	0.000865352773713785\\
578	0.000818439509057384\\
579	0.00077165671858706\\
580	0.000725057614040401\\
581	0.000678696970709242\\
582	0.000632630483283921\\
583	0.000586913897120319\\
584	0.000541601859672149\\
585	0.00049674642660404\\
586	0.000452395149029591\\
587	0.000408588670365815\\
588	0.000365357795666476\\
589	0.00032272012337526\\
590	0.000280676711614416\\
591	0.000239312817487405\\
592	0.000198788644961842\\
593	0.000159293651685586\\
594	0.000121079888727804\\
595	8.45520570083913e-05\\
596	5.05092148680373e-05\\
597	2.07908715710836e-05\\
598	0\\
599	0\\
600	0\\
};
\addplot [color=mycolor3,solid,forget plot]
  table[row sep=crcr]{%
1	0.00550477682330948\\
2	0.00550477425642561\\
3	0.0055047716434672\\
4	0.00550476898360203\\
5	0.00550476627598275\\
6	0.00550476351974651\\
7	0.00550476071401469\\
8	0.00550475785789298\\
9	0.00550475495047054\\
10	0.00550475199082008\\
11	0.00550474897799734\\
12	0.00550474591104091\\
13	0.00550474278897191\\
14	0.00550473961079362\\
15	0.0055047363754912\\
16	0.00550473308203119\\
17	0.0055047297293615\\
18	0.00550472631641071\\
19	0.00550472284208796\\
20	0.00550471930528255\\
21	0.00550471570486342\\
22	0.00550471203967905\\
23	0.00550470830855674\\
24	0.00550470451030261\\
25	0.00550470064370086\\
26	0.0055046967075136\\
27	0.00550469270048034\\
28	0.00550468862131753\\
29	0.00550468446871832\\
30	0.00550468024135183\\
31	0.00550467593786316\\
32	0.00550467155687248\\
33	0.00550466709697481\\
34	0.00550466255673955\\
35	0.00550465793470994\\
36	0.00550465322940267\\
37	0.00550464843930729\\
38	0.00550464356288576\\
39	0.00550463859857201\\
40	0.00550463354477119\\
41	0.00550462839985948\\
42	0.00550462316218334\\
43	0.00550461783005889\\
44	0.00550461240177163\\
45	0.00550460687557555\\
46	0.00550460124969282\\
47	0.00550459552231303\\
48	0.00550458969159265\\
49	0.00550458375565442\\
50	0.0055045777125868\\
51	0.0055045715604432\\
52	0.00550456529724142\\
53	0.005504558920963\\
54	0.00550455242955239\\
55	0.00550454582091652\\
56	0.00550453909292396\\
57	0.00550453224340419\\
58	0.00550452527014696\\
59	0.0055045181709016\\
60	0.00550451094337607\\
61	0.00550450358523637\\
62	0.00550449609410563\\
63	0.00550448846756352\\
64	0.0055044807031453\\
65	0.00550447279834096\\
66	0.00550446475059457\\
67	0.00550445655730317\\
68	0.00550444821581615\\
69	0.00550443972343412\\
70	0.00550443107740831\\
71	0.00550442227493934\\
72	0.00550441331317651\\
73	0.00550440418921677\\
74	0.00550439490010382\\
75	0.0055043854428269\\
76	0.00550437581432005\\
77	0.00550436601146093\\
78	0.00550435603106987\\
79	0.00550434586990864\\
80	0.00550433552467965\\
81	0.00550432499202459\\
82	0.00550431426852338\\
83	0.00550430335069311\\
84	0.00550429223498672\\
85	0.00550428091779194\\
86	0.00550426939543011\\
87	0.00550425766415471\\
88	0.00550424572015037\\
89	0.00550423355953152\\
90	0.00550422117834099\\
91	0.0055042085725488\\
92	0.00550419573805069\\
93	0.00550418267066679\\
94	0.00550416936614034\\
95	0.00550415582013608\\
96	0.0055041420282389\\
97	0.00550412798595239\\
98	0.00550411368869705\\
99	0.00550409913180917\\
100	0.00550408431053899\\
101	0.00550406922004924\\
102	0.0055040538554133\\
103	0.00550403821161376\\
104	0.00550402228354073\\
105	0.00550400606598996\\
106	0.00550398955366117\\
107	0.00550397274115646\\
108	0.00550395562297808\\
109	0.00550393819352694\\
110	0.00550392044710062\\
111	0.00550390237789137\\
112	0.0055038839799842\\
113	0.00550386524735506\\
114	0.00550384617386849\\
115	0.00550382675327583\\
116	0.00550380697921296\\
117	0.00550378684519831\\
118	0.0055037663446305\\
119	0.00550374547078629\\
120	0.00550372421681824\\
121	0.00550370257575223\\
122	0.00550368054048561\\
123	0.00550365810378426\\
124	0.00550363525828055\\
125	0.00550361199647067\\
126	0.00550358831071218\\
127	0.0055035641932214\\
128	0.00550353963607094\\
129	0.00550351463118682\\
130	0.005503489170346\\
131	0.00550346324517338\\
132	0.00550343684713932\\
133	0.00550340996755649\\
134	0.00550338259757716\\
135	0.00550335472819009\\
136	0.00550332635021771\\
137	0.00550329745431279\\
138	0.0055032680309557\\
139	0.00550323807045099\\
140	0.00550320756292402\\
141	0.00550317649831816\\
142	0.005503144866391\\
143	0.00550311265671116\\
144	0.00550307985865477\\
145	0.00550304646140206\\
146	0.00550301245393358\\
147	0.00550297782502669\\
148	0.00550294256325193\\
149	0.00550290665696907\\
150	0.00550287009432344\\
151	0.00550283286324184\\
152	0.0055027949514287\\
153	0.00550275634636214\\
154	0.00550271703528952\\
155	0.00550267700522361\\
156	0.00550263624293808\\
157	0.00550259473496338\\
158	0.00550255246758201\\
159	0.00550250942682436\\
160	0.00550246559846392\\
161	0.00550242096801271\\
162	0.00550237552071655\\
163	0.00550232924155014\\
164	0.00550228211521213\\
165	0.00550223412612033\\
166	0.00550218525840629\\
167	0.00550213549591032\\
168	0.00550208482217605\\
169	0.00550203322044512\\
170	0.00550198067365169\\
171	0.00550192716441669\\
172	0.00550187267504231\\
173	0.0055018171875059\\
174	0.00550176068345432\\
175	0.00550170314419761\\
176	0.00550164455070308\\
177	0.00550158488358875\\
178	0.00550152412311714\\
179	0.00550146224918863\\
180	0.00550139924133485\\
181	0.00550133507871183\\
182	0.00550126974009321\\
183	0.00550120320386315\\
184	0.00550113544800915\\
185	0.00550106645011499\\
186	0.00550099618735303\\
187	0.00550092463647722\\
188	0.00550085177381503\\
189	0.00550077757526\\
190	0.00550070201626403\\
191	0.00550062507182938\\
192	0.00550054671650088\\
193	0.00550046692435791\\
194	0.00550038566900639\\
195	0.00550030292357135\\
196	0.00550021866068941\\
197	0.00550013285250246\\
198	0.00550004547065237\\
199	0.00549995648627275\\
200	0.00549986586996504\\
201	0.00549977359178749\\
202	0.00549967962124463\\
203	0.00549958392727645\\
204	0.00549948647824725\\
205	0.00549938724193426\\
206	0.005499286185516\\
207	0.00549918327556026\\
208	0.00549907847801172\\
209	0.00549897175817939\\
210	0.00549886308072349\\
211	0.0054987524096421\\
212	0.0054986397082573\\
213	0.00549852493920084\\
214	0.0054984080643997\\
215	0.00549828904506062\\
216	0.00549816784165441\\
217	0.00549804441390017\\
218	0.00549791872074772\\
219	0.00549779072036074\\
220	0.00549766037009842\\
221	0.00549752762649653\\
222	0.00549739244524797\\
223	0.00549725478118225\\
224	0.00549711458824402\\
225	0.00549697181947082\\
226	0.00549682642696964\\
227	0.00549667836189246\\
228	0.00549652757441025\\
229	0.00549637401368581\\
230	0.00549621762784532\\
231	0.00549605836394773\\
232	0.00549589616795276\\
233	0.0054957309846868\\
234	0.00549556275780674\\
235	0.00549539142976125\\
236	0.00549521694174974\\
237	0.00549503923367794\\
238	0.00549485824411096\\
239	0.00549467391022222\\
240	0.00549448616773887\\
241	0.00549429495088275\\
242	0.00549410019230668\\
243	0.0054939018230254\\
244	0.00549369977234063\\
245	0.00549349396775976\\
246	0.00549328433490728\\
247	0.00549307079742838\\
248	0.00549285327688355\\
249	0.00549263169263335\\
250	0.00549240596171277\\
251	0.00549217599869314\\
252	0.00549194171553131\\
253	0.00549170302140385\\
254	0.00549145982252539\\
255	0.00549121202194953\\
256	0.0054909595193502\\
257	0.0054907022107821\\
258	0.00549043998841833\\
259	0.00549017274026306\\
260	0.00548990034983755\\
261	0.00548962269583773\\
262	0.00548933965176146\\
263	0.00548905108550404\\
264	0.00548875685892081\\
265	0.00548845682735624\\
266	0.0054881508391397\\
267	0.00548783873504886\\
268	0.00548752034774364\\
269	0.00548719550117455\\
270	0.00548686400997268\\
271	0.00548652567883073\\
272	0.0054861803018893\\
273	0.00548582766214855\\
274	0.00548546753093547\\
275	0.00548509966748068\\
276	0.00548472381871943\\
277	0.00548433971962728\\
278	0.00548394709507049\\
279	0.00548354566654718\\
280	0.00548313585038502\\
281	0.00548271859907244\\
282	0.00548229377981785\\
283	0.0054818612575201\\
284	0.00548142089472977\\
285	0.00548097255161\\
286	0.00548051608589664\\
287	0.00548005135285767\\
288	0.0054795782052522\\
289	0.00547909649328861\\
290	0.00547860606458215\\
291	0.00547810676411174\\
292	0.00547759843417632\\
293	0.00547708091435022\\
294	0.005476554041438\\
295	0.00547601764942866\\
296	0.00547547156944895\\
297	0.00547491562971612\\
298	0.00547434965548969\\
299	0.00547377346902287\\
300	0.00547318688951282\\
301	0.00547258973305047\\
302	0.00547198181256938\\
303	0.00547136293779389\\
304	0.00547073291518661\\
305	0.005470091547895\\
306	0.00546943863569711\\
307	0.00546877397494682\\
308	0.00546809735851799\\
309	0.00546740857574802\\
310	0.00546670741238051\\
311	0.00546599365050735\\
312	0.00546526706850985\\
313	0.00546452744099923\\
314	0.00546377453875629\\
315	0.00546300812867045\\
316	0.00546222797367823\\
317	0.00546143383270067\\
318	0.00546062546058052\\
319	0.00545980260801873\\
320	0.00545896502151018\\
321	0.00545811244327932\\
322	0.00545724461121499\\
323	0.00545636125880529\\
324	0.00545546211507171\\
325	0.00545454690450366\\
326	0.00545361534699242\\
327	0.0054526671577656\\
328	0.00545170204732146\\
329	0.00545071972136402\\
330	0.00544971988073853\\
331	0.00544870222136763\\
332	0.00544766643418897\\
333	0.00544661220509387\\
334	0.00544553921486785\\
335	0.00544444713913311\\
336	0.00544333564829348\\
337	0.00544220440748248\\
338	0.0054410530765146\\
339	0.00543988130984097\\
340	0.0054386887565096\\
341	0.00543747506013127\\
342	0.0054362398588518\\
343	0.00543498278533194\\
344	0.00543370346673545\\
345	0.00543240152472745\\
346	0.00543107657548345\\
347	0.00542972822971189\\
348	0.00542835609269072\\
349	0.00542695976432092\\
350	0.00542553883919896\\
351	0.00542409290671077\\
352	0.00542262155115011\\
353	0.00542112435186449\\
354	0.00541960088343242\\
355	0.00541805071587583\\
356	0.00541647341491245\\
357	0.00541486854225276\\
358	0.00541323565594737\\
359	0.00541157431079114\\
360	0.00540988405879056\\
361	0.0054081644497021\\
362	0.00540641503164977\\
363	0.00540463535183105\\
364	0.0054028249573207\\
365	0.00540098339598347\\
366	0.00539911021750652\\
367	0.00539720497456394\\
368	0.00539526722412573\\
369	0.00539329652892395\\
370	0.00539129245908901\\
371	0.00538925459396853\\
372	0.00538718252414036\\
373	0.00538507585362938\\
374	0.00538293420233504\\
375	0.00538075720867236\\
376	0.00537854453242384\\
377	0.00537629585779\\
378	0.00537401089661826\\
379	0.00537168939177402\\
380	0.00536933112060602\\
381	0.00536693589844766\\
382	0.00536450358211477\\
383	0.00536203407346954\\
384	0.00535952732353251\\
385	0.00535698333899241\\
386	0.00535440219739541\\
387	0.00535178409178646\\
388	0.00534912514988319\\
389	0.00534641761903743\\
390	0.00534366061596057\\
391	0.00534085324265133\\
392	0.00533799458610111\\
393	0.00533508371795533\\
394	0.00533211969415246\\
395	0.00532910155454309\\
396	0.00532602832211091\\
397	0.00532289900203223\\
398	0.00531971258080888\\
399	0.00531646802530294\\
400	0.00531316428168546\\
401	0.00530980027430732\\
402	0.00530637490449251\\
403	0.00530288704928074\\
404	0.00529933556009359\\
405	0.00529571926117679\\
406	0.00529203694800453\\
407	0.00528828738570286\\
408	0.00528446930703295\\
409	0.00528058141012162\\
410	0.00527662235590618\\
411	0.00527259076525703\\
412	0.00526848521573851\\
413	0.00526430423796721\\
414	0.00526004631153165\\
415	0.00525570986046398\\
416	0.00525129324831951\\
417	0.00524679477302037\\
418	0.00524221266141137\\
419	0.00523754506071366\\
420	0.00523279003052249\\
421	0.00522794553410434\\
422	0.0052230094281619\\
423	0.00521797944891619\\
424	0.00521285318686776\\
425	0.00520762802306626\\
426	0.00520230092936529\\
427	0.00519686899937804\\
428	0.00519132985949917\\
429	0.00518568105926974\\
430	0.00517992006754721\\
431	0.00517404426844573\\
432	0.00516805095702335\\
433	0.00516193733472785\\
434	0.00515570050460202\\
435	0.00514933746622776\\
436	0.00514284511040285\\
437	0.00513622021354724\\
438	0.00512945943183804\\
439	0.0051225592950757\\
440	0.00511551620027982\\
441	0.00510832640502191\\
442	0.00510098602050871\\
443	0.00509349100442846\\
444	0.00508583715357047\\
445	0.00507802009621341\\
446	0.00507003528423538\\
447	0.00506187798479317\\
448	0.00505354327122278\\
449	0.00504502601268046\\
450	0.00503632086333021\\
451	0.00502742226201914\\
452	0.00501832447056875\\
453	0.00500902149371753\\
454	0.00499950706396044\\
455	0.00498977462543773\\
456	0.00497981731679798\\
457	0.00496962795297015\\
458	0.00495919900577868\\
459	0.00494852258333314\\
460	0.00493759040812209\\
461	0.00492639379373232\\
462	0.00491492362011725\\
463	0.00490317030733283\\
464	0.00489112378765591\\
465	0.00487877347599587\\
466	0.00486610823851244\\
467	0.00485311635938098\\
468	0.00483978550576249\\
469	0.0048261026913033\\
470	0.00481205423849246\\
471	0.00479762573586072\\
472	0.00478280199444234\\
473	0.00476756700393987\\
474	0.00475190388728252\\
475	0.00473579485348131\\
476	0.00471922114823902\\
477	0.0047021630016523\\
478	0.00468459957115042\\
479	0.00466650885784214\\
480	0.0046478675515963\\
481	0.0046286508102488\\
482	0.00460883243742465\\
483	0.00458838400496198\\
484	0.00456726999448731\\
485	0.00454543469310754\\
486	0.00452284311214705\\
487	0.00449945871076504\\
488	0.00447524373166766\\
489	0.00445015868630056\\
490	0.00442416236753183\\
491	0.00439721188603949\\
492	0.00436926275841055\\
493	0.00434026904545963\\
494	0.00431018355398794\\
495	0.00427895811764693\\
496	0.00424654397658071\\
497	0.00421289228003259\\
498	0.00417795474171066\\
499	0.00414168448465006\\
500	0.0041040371207246\\
501	0.0040649721200045\\
502	0.00402445453844905\\
503	0.00398245719006864\\
504	0.00393896337156799\\
505	0.00389397025230831\\
506	0.00384749302944713\\
507	0.00379957011758446\\
508	0.00375026981833159\\
509	0.00369969955863891\\
510	0.0036480175997573\\
511	0.00359544297495342\\
512	0.0035422627542797\\
513	0.00348883084154451\\
514	0.0034356354815106\\
515	0.00338346436248934\\
516	0.00333529941397275\\
517	0.00329239584335947\\
518	0.00325479594604639\\
519	0.00322212145566726\\
520	0.00319081861398368\\
521	0.0031604552962268\\
522	0.00313060409131133\\
523	0.0031007791385642\\
524	0.00307090240615044\\
525	0.00304081945040074\\
526	0.00301018748031608\\
527	0.00297896731824318\\
528	0.0029471211890936\\
529	0.00291460839813878\\
530	0.00288140050348125\\
531	0.00284747823876823\\
532	0.00281282247445835\\
533	0.00277741396691225\\
534	0.00274123362774897\\
535	0.0027042612251743\\
536	0.00266647498145895\\
537	0.00262785203732332\\
538	0.00258836871834692\\
539	0.00254799039987889\\
540	0.00250663194236034\\
541	0.00246552657985651\\
542	0.00242551775773468\\
543	0.00238679174078739\\
544	0.00234758968952664\\
545	0.00230783756843862\\
546	0.00226746688232517\\
547	0.00222647633461372\\
548	0.00218486861377273\\
549	0.00214264741712259\\
550	0.00209982152218929\\
551	0.00205640582806659\\
552	0.00201241871448934\\
553	0.00196788261038474\\
554	0.001922824759329\\
555	0.00187727861552058\\
556	0.00183128530984319\\
557	0.00178571858457715\\
558	0.00174175370183844\\
559	0.00169765189645918\\
560	0.00165315928882859\\
561	0.00160829620918484\\
562	0.0015630878421284\\
563	0.00151756120705137\\
564	0.00147174485938914\\
565	0.00142566860063263\\
566	0.00137936312148276\\
567	0.00133285966646703\\
568	0.00128618971348929\\
569	0.00123974505182\\
570	0.00119341585886474\\
571	0.00114688025846023\\
572	0.00110017097215749\\
573	0.00105332337466137\\
574	0.00100637555374019\\
575	0.000959368345378089\\
576	0.00091234532951263\\
577	0.00086535277371407\\
578	0.00081843950905746\\
579	0.000771656718587084\\
580	0.000725057614040413\\
581	0.000678696970709247\\
582	0.000632630483283924\\
583	0.00058691389712032\\
584	0.000541601859672149\\
585	0.000496746426604037\\
586	0.000452395149029591\\
587	0.000408588670365813\\
588	0.000365357795666473\\
589	0.000322720123375262\\
590	0.000280676711614414\\
591	0.000239312817487402\\
592	0.000198788644961839\\
593	0.000159293651685585\\
594	0.000121079888727804\\
595	8.45520570083908e-05\\
596	5.05092148680372e-05\\
597	2.07908715710836e-05\\
598	0\\
599	0\\
600	0\\
};
\addplot [color=mycolor4,solid,forget plot]
  table[row sep=crcr]{%
1	0.00552236449378588\\
2	0.00552236161741465\\
3	0.00552235869004543\\
4	0.00552235571076934\\
5	0.00552235267866136\\
6	0.00552234959277986\\
7	0.00552234645216632\\
8	0.00552234325584501\\
9	0.0055223400028227\\
10	0.00552233669208844\\
11	0.0055223333226131\\
12	0.00552232989334907\\
13	0.00552232640323002\\
14	0.00552232285117041\\
15	0.00552231923606535\\
16	0.00552231555678995\\
17	0.00552231181219927\\
18	0.0055223080011278\\
19	0.0055223041223891\\
20	0.00552230017477544\\
21	0.0055222961570574\\
22	0.00552229206798352\\
23	0.00552228790627984\\
24	0.00552228367064954\\
25	0.00552227935977247\\
26	0.00552227497230478\\
27	0.00552227050687855\\
28	0.00552226596210119\\
29	0.00552226133655511\\
30	0.00552225662879733\\
31	0.00552225183735878\\
32	0.0055222469607441\\
33	0.00552224199743105\\
34	0.00552223694586996\\
35	0.00552223180448328\\
36	0.00552222657166512\\
37	0.00552222124578073\\
38	0.00552221582516592\\
39	0.00552221030812658\\
40	0.00552220469293806\\
41	0.00552219897784472\\
42	0.0055221931610593\\
43	0.00552218724076237\\
44	0.00552218121510173\\
45	0.00552217508219179\\
46	0.00552216884011311\\
47	0.00552216248691159\\
48	0.00552215602059788\\
49	0.00552214943914689\\
50	0.00552214274049698\\
51	0.00552213592254938\\
52	0.00552212898316735\\
53	0.00552212192017577\\
54	0.00552211473136027\\
55	0.00552210741446656\\
56	0.00552209996719957\\
57	0.00552209238722291\\
58	0.00552208467215801\\
59	0.00552207681958337\\
60	0.00552206882703378\\
61	0.0055220606919995\\
62	0.00552205241192548\\
63	0.00552204398421052\\
64	0.00552203540620642\\
65	0.00552202667521712\\
66	0.00552201778849787\\
67	0.00552200874325425\\
68	0.00552199953664139\\
69	0.00552199016576291\\
70	0.00552198062767006\\
71	0.00552197091936075\\
72	0.00552196103777858\\
73	0.00552195097981178\\
74	0.00552194074229234\\
75	0.00552193032199479\\
76	0.00552191971563531\\
77	0.00552190891987057\\
78	0.00552189793129664\\
79	0.00552188674644797\\
80	0.00552187536179611\\
81	0.00552186377374865\\
82	0.00552185197864809\\
83	0.00552183997277049\\
84	0.00552182775232437\\
85	0.00552181531344947\\
86	0.00552180265221537\\
87	0.00552178976462037\\
88	0.00552177664659001\\
89	0.00552176329397577\\
90	0.00552174970255381\\
91	0.00552173586802346\\
92	0.00552172178600583\\
93	0.00552170745204245\\
94	0.00552169286159359\\
95	0.00552167801003702\\
96	0.00552166289266629\\
97	0.0055216475046892\\
98	0.00552163184122631\\
99	0.00552161589730922\\
100	0.00552159966787898\\
101	0.00552158314778434\\
102	0.00552156633178012\\
103	0.0055215492145254\\
104	0.00552153179058173\\
105	0.00552151405441141\\
106	0.00552149600037559\\
107	0.0055214776227323\\
108	0.00552145891563467\\
109	0.00552143987312886\\
110	0.00552142048915216\\
111	0.00552140075753084\\
112	0.00552138067197822\\
113	0.00552136022609242\\
114	0.00552133941335427\\
115	0.00552131822712517\\
116	0.00552129666064476\\
117	0.00552127470702868\\
118	0.00552125235926623\\
119	0.00552122961021801\\
120	0.00552120645261356\\
121	0.00552118287904884\\
122	0.0055211588819837\\
123	0.00552113445373936\\
124	0.00552110958649586\\
125	0.00552108427228929\\
126	0.00552105850300921\\
127	0.00552103227039581\\
128	0.00552100556603709\\
129	0.00552097838136615\\
130	0.00552095070765808\\
131	0.00552092253602712\\
132	0.00552089385742359\\
133	0.0055208646626308\\
134	0.00552083494226198\\
135	0.00552080468675704\\
136	0.00552077388637924\\
137	0.00552074253121211\\
138	0.00552071061115585\\
139	0.00552067811592396\\
140	0.00552064503503989\\
141	0.00552061135783322\\
142	0.00552057707343636\\
143	0.00552054217078058\\
144	0.0055205066385925\\
145	0.00552047046539015\\
146	0.00552043363947916\\
147	0.00552039614894883\\
148	0.00552035798166811\\
149	0.00552031912528151\\
150	0.00552027956720514\\
151	0.00552023929462236\\
152	0.00552019829447949\\
153	0.00552015655348165\\
154	0.00552011405808831\\
155	0.00552007079450878\\
156	0.00552002674869782\\
157	0.00551998190635094\\
158	0.00551993625289978\\
159	0.00551988977350748\\
160	0.00551984245306379\\
161	0.00551979427618033\\
162	0.00551974522718557\\
163	0.00551969529011997\\
164	0.0055196444487308\\
165	0.00551959268646715\\
166	0.00551953998647461\\
167	0.00551948633159015\\
168	0.0055194317043366\\
169	0.0055193760869174\\
170	0.00551931946121092\\
171	0.00551926180876504\\
172	0.00551920311079126\\
173	0.00551914334815907\\
174	0.00551908250138998\\
175	0.00551902055065147\\
176	0.00551895747575083\\
177	0.00551889325612899\\
178	0.00551882787085396\\
179	0.0055187612986142\\
180	0.00551869351771199\\
181	0.00551862450605631\\
182	0.00551855424115581\\
183	0.00551848270011119\\
184	0.00551840985960772\\
185	0.00551833569590721\\
186	0.00551826018483992\\
187	0.00551818330179588\\
188	0.00551810502171628\\
189	0.00551802531908424\\
190	0.00551794416791541\\
191	0.00551786154174813\\
192	0.00551777741363352\\
193	0.00551769175612491\\
194	0.00551760454126753\\
195	0.00551751574058748\\
196	0.00551742532508085\\
197	0.00551733326520237\\
198	0.00551723953085419\\
199	0.00551714409137402\\
200	0.00551704691552371\\
201	0.00551694797147738\\
202	0.00551684722680929\\
203	0.00551674464848142\\
204	0.00551664020283073\\
205	0.00551653385555592\\
206	0.00551642557170416\\
207	0.0055163153156571\\
208	0.00551620305111673\\
209	0.00551608874109072\\
210	0.00551597234787748\\
211	0.00551585383305065\\
212	0.00551573315744316\\
213	0.00551561028113103\\
214	0.00551548516341644\\
215	0.00551535776281035\\
216	0.00551522803701485\\
217	0.00551509594290453\\
218	0.00551496143650786\\
219	0.00551482447298749\\
220	0.0055146850066202\\
221	0.00551454299077638\\
222	0.00551439837789852\\
223	0.00551425111947928\\
224	0.00551410116603904\\
225	0.00551394846710242\\
226	0.00551379297117448\\
227	0.00551363462571579\\
228	0.00551347337711732\\
229	0.00551330917067408\\
230	0.00551314195055836\\
231	0.00551297165979225\\
232	0.0055127982402193\\
233	0.00551262163247557\\
234	0.00551244177596003\\
235	0.00551225860880428\\
236	0.00551207206784177\\
237	0.00551188208857648\\
238	0.00551168860515112\\
239	0.00551149155031508\\
240	0.00551129085539215\\
241	0.00551108645024829\\
242	0.00551087826325943\\
243	0.00551066622127962\\
244	0.00551045024961022\\
245	0.00551023027196952\\
246	0.00551000621046425\\
247	0.0055097779855625\\
248	0.00550954551606927\\
249	0.00550930871910496\\
250	0.00550906751008725\\
251	0.00550882180271794\\
252	0.0055085715089747\\
253	0.00550831653910962\\
254	0.00550805680165543\\
255	0.00550779220344061\\
256	0.00550752264961524\\
257	0.00550724804368904\\
258	0.00550696828758372\\
259	0.00550668328170141\\
260	0.00550639292501186\\
261	0.00550609711516044\\
262	0.00550579574859972\\
263	0.00550548872074722\\
264	0.0055051759261724\\
265	0.00550485725881497\\
266	0.00550453261223783\\
267	0.00550420187991627\\
268	0.00550386495556571\\
269	0.00550352173350871\\
270	0.00550317210908153\\
271	0.00550281597907872\\
272	0.00550245324223288\\
273	0.00550208379972421\\
274	0.00550170755571362\\
275	0.00550132441789157\\
276	0.00550093429803821\\
277	0.00550053711259858\\
278	0.00550013278327558\\
279	0.0054997212375169\\
280	0.00549930240779239\\
281	0.00549887620663984\\
282	0.00549844250829231\\
283	0.00549800118495234\\
284	0.00549755210676195\\
285	0.00549709514177206\\
286	0.00549663015591132\\
287	0.00549615701295485\\
288	0.00549567557449205\\
289	0.00549518569989425\\
290	0.00549468724628167\\
291	0.00549418006848997\\
292	0.00549366401903615\\
293	0.00549313894808387\\
294	0.00549260470340838\\
295	0.00549206113036053\\
296	0.00549150807183033\\
297	0.00549094536820982\\
298	0.00549037285735526\\
299	0.00548979037454844\\
300	0.00548919775245742\\
301	0.00548859482109638\\
302	0.00548798140778461\\
303	0.00548735733710453\\
304	0.00548672243085905\\
305	0.00548607650802752\\
306	0.00548541938472111\\
307	0.00548475087413662\\
308	0.00548407078650947\\
309	0.00548337892906525\\
310	0.00548267510597001\\
311	0.00548195911827908\\
312	0.0054812307638845\\
313	0.0054804898374607\\
314	0.00547973613040869\\
315	0.00547896943079811\\
316	0.00547818952330758\\
317	0.0054773961891628\\
318	0.00547658920607254\\
319	0.00547576834816195\\
320	0.00547493338590362\\
321	0.00547408408604557\\
322	0.00547322021153632\\
323	0.00547234152144663\\
324	0.00547144777088796\\
325	0.00547053871092677\\
326	0.00546961408849504\\
327	0.00546867364629624\\
328	0.00546771712270641\\
329	0.00546674425167016\\
330	0.00546575476259082\\
331	0.00546474838021479\\
332	0.0054637248245088\\
333	0.0054626838105302\\
334	0.00546162504828915\\
335	0.00546054824260222\\
336	0.00545945309293683\\
337	0.00545833929324494\\
338	0.00545720653178605\\
339	0.00545605449093745\\
340	0.00545488284699143\\
341	0.00545369126993761\\
342	0.00545247942322917\\
343	0.00545124696353159\\
344	0.00544999354045185\\
345	0.00544871879624645\\
346	0.00544742236550625\\
347	0.00544610387481529\\
348	0.00544476294238176\\
349	0.00544339917763805\\
350	0.00544201218080623\\
351	0.0054406015424265\\
352	0.00543916684284406\\
353	0.00543770765165072\\
354	0.00543622352707626\\
355	0.00543471401532517\\
356	0.0054331786498525\\
357	0.00543161695057351\\
358	0.00543002842300023\\
359	0.00542841255729794\\
360	0.0054267688272538\\
361	0.00542509668914959\\
362	0.00542339558052954\\
363	0.00542166491885402\\
364	0.00541990410002906\\
365	0.00541811249680166\\
366	0.00541628945701016\\
367	0.00541443430167937\\
368	0.00541254632294991\\
369	0.00541062478183212\\
370	0.00540866890577628\\
371	0.00540667788605188\\
372	0.00540465087493226\\
373	0.00540258698268411\\
374	0.00540048527436669\\
375	0.00539834476645274\\
376	0.00539616442329166\\
377	0.00539394315344812\\
378	0.00539167980596377\\
379	0.00538937316660993\\
380	0.00538702195422511\\
381	0.00538462481726917\\
382	0.00538218033078822\\
383	0.00537968699412042\\
384	0.00537714323003581\\
385	0.00537454738714601\\
386	0.00537189775132708\\
387	0.00536919258590505\\
388	0.00536643411745236\\
389	0.00536362814878825\\
390	0.00536077380558616\\
391	0.00535787020184325\\
392	0.00535491644144934\\
393	0.00535191161956328\\
394	0.00534885482403642\\
395	0.0053457451376249\\
396	0.00534258164487129\\
397	0.00533936343356682\\
398	0.00533608959274483\\
399	0.00533275921349833\\
400	0.00532937138914247\\
401	0.00532592521468404\\
402	0.00532241978561147\\
403	0.00531885419614399\\
404	0.0053152275385854\\
405	0.0053115389049386\\
406	0.00530778738568527\\
407	0.00530397206919219\\
408	0.00530009204781425\\
409	0.00529614642026316\\
410	0.00529213429423897\\
411	0.00528805478932205\\
412	0.00528390704009951\\
413	0.00527969019945088\\
414	0.00527540344183438\\
415	0.00527104596630378\\
416	0.00526661699899993\\
417	0.00526211579574787\\
418	0.00525754165017734\\
419	0.00525289393257562\\
420	0.00524817207014708\\
421	0.00524337554790461\\
422	0.00523850391366673\\
423	0.00523355678417246\\
424	0.0052285338562323\\
425	0.00522343493609123\\
426	0.00521826003033973\\
427	0.00521299981013176\\
428	0.00520764154984799\\
429	0.00520218316507152\\
430	0.00519662250553933\\
431	0.00519095735143663\\
432	0.00518518540943985\\
433	0.00517930430786009\\
434	0.00517331159122543\\
435	0.00516720471464439\\
436	0.00516098103757801\\
437	0.00515463781694451\\
438	0.00514817219947595\\
439	0.0051415812132719\\
440	0.00513486175855821\\
441	0.00512801059742921\\
442	0.0051210243424309\\
443	0.00511389944394695\\
444	0.00510663217628548\\
445	0.00509921862234511\\
446	0.00509165465665163\\
447	0.00508393592620462\\
448	0.00507605782727588\\
449	0.00506801547165309\\
450	0.0050598036192859\\
451	0.00505141649510841\\
452	0.0050428475396761\\
453	0.00503409144145339\\
454	0.0050251426509999\\
455	0.00501599536655582\\
456	0.00500664351914084\\
457	0.00499708075683417\\
458	0.00498730042815378\\
459	0.00497729556449069\\
460	0.0049670588615872\\
461	0.00495658266011656\\
462	0.00494585892524127\\
463	0.0049348792251699\\
464	0.00492363470868343\\
465	0.00491211608156425\\
466	0.00490031358175842\\
467	0.00488821695288661\\
468	0.00487581541544252\\
469	0.00486309763559506\\
470	0.00485005169845369\\
471	0.0048366651328826\\
472	0.00482292483719772\\
473	0.00480881700809003\\
474	0.00479432709932754\\
475	0.00477943977858097\\
476	0.0047641388823978\\
477	0.00474840736927814\\
478	0.00473222727046199\\
479	0.00471557963788382\\
480	0.00469844448912965\\
481	0.00468080074667346\\
482	0.0046626261375146\\
483	0.00464389700735168\\
484	0.00462458811794882\\
485	0.0046046727571455\\
486	0.00458412118468098\\
487	0.00456289381669408\\
488	0.00454093863433154\\
489	0.00451821998828621\\
490	0.00449470064864536\\
491	0.00447034196370148\\
492	0.00444510340289893\\
493	0.00441894254081404\\
494	0.00439181506085209\\
495	0.00436367480329994\\
496	0.0043344738555302\\
497	0.00430416269614973\\
498	0.00427269040674317\\
499	0.00424000496854713\\
500	0.00420605366530414\\
501	0.00417078361847794\\
502	0.00413414248746009\\
503	0.00409607937443961\\
504	0.00405654598266416\\
505	0.00401549808831925\\
506	0.00397289740224313\\
507	0.00392871391491754\\
508	0.00388292883819276\\
509	0.00383553825499621\\
510	0.00378655763043521\\
511	0.00373602750432496\\
512	0.00368402082627437\\
513	0.00363065307189473\\
514	0.00357609408201811\\
515	0.00352058662871403\\
516	0.00346443837690163\\
517	0.00340801376574661\\
518	0.00335185318789953\\
519	0.00329683024981812\\
520	0.00324653985142069\\
521	0.00320161003127924\\
522	0.00316207202960114\\
523	0.00312749854380489\\
524	0.00309395292311824\\
525	0.00306108207870875\\
526	0.00302867931543196\\
527	0.00299627525953052\\
528	0.00296378678854382\\
529	0.00293115317034936\\
530	0.00289809271373717\\
531	0.00286441613276463\\
532	0.00283008516840576\\
533	0.00279506205124351\\
534	0.00275930312882306\\
535	0.00272278664427699\\
536	0.00268549207316304\\
537	0.00264739866875622\\
538	0.00260848525570127\\
539	0.00256873024649069\\
540	0.00252811165960377\\
541	0.00248660681497579\\
542	0.00244415586141974\\
543	0.00240076947053404\\
544	0.00235831976722801\\
545	0.00231705559205897\\
546	0.00227650183143643\\
547	0.00223546034490059\\
548	0.00219384736307803\\
549	0.00215161604661321\\
550	0.00210876510575541\\
551	0.00206530472400923\\
552	0.00202124965549764\\
553	0.00197661822767365\\
554	0.00193143270313793\\
555	0.00188571981291925\\
556	0.00183951158664442\\
557	0.00179284685067574\\
558	0.00174575367523509\\
559	0.00169999208744112\\
560	0.00165532117678849\\
561	0.00161033080979638\\
562	0.00156498746152975\\
563	0.00151931648608894\\
564	0.00147334703390769\\
565	0.00142710998207863\\
566	0.00138063755587401\\
567	0.00133396294340406\\
568	0.00128711993625959\\
569	0.00124014252333356\\
570	0.00119341592573713\\
571	0.00114688025971738\\
572	0.00110017097260895\\
573	0.0010533233748797\\
574	0.00100637555384331\\
575	0.000959368345424142\\
576	0.000912345329531775\\
577	0.000865352773721365\\
578	0.000818439509059961\\
579	0.000771656718587823\\
580	0.000725057614040593\\
581	0.000678696970709286\\
582	0.000632630483283935\\
583	0.000586913897120328\\
584	0.000541601859672155\\
585	0.000496746426604042\\
586	0.000452395149029594\\
587	0.000408588670365815\\
588	0.000365357795666473\\
589	0.000322720123375259\\
590	0.000280676711614414\\
591	0.000239312817487403\\
592	0.000198788644961842\\
593	0.000159293651685587\\
594	0.000121079888727805\\
595	8.45520570083912e-05\\
596	5.05092148680371e-05\\
597	2.07908715710836e-05\\
598	0\\
599	0\\
600	0\\
};
\addplot [color=mycolor5,solid,forget plot]
  table[row sep=crcr]{%
1	0.0055565662029982\\
2	0.0055565625234433\\
3	0.00555655877973134\\
4	0.00555655497074048\\
5	0.00555655109532921\\
6	0.00555654715233594\\
7	0.00555654314057877\\
8	0.00555653905885504\\
9	0.00555653490594107\\
10	0.0055565306805916\\
11	0.00555652638153959\\
12	0.00555652200749573\\
13	0.00555651755714814\\
14	0.00555651302916188\\
15	0.00555650842217858\\
16	0.0055565037348161\\
17	0.00555649896566798\\
18	0.00555649411330313\\
19	0.00555648917626525\\
20	0.00555648415307261\\
21	0.00555647904221737\\
22	0.00555647384216524\\
23	0.00555646855135501\\
24	0.00555646316819805\\
25	0.00555645769107788\\
26	0.00555645211834957\\
27	0.00555644644833935\\
28	0.00555644067934405\\
29	0.0055564348096306\\
30	0.00555642883743542\\
31	0.00555642276096407\\
32	0.00555641657839049\\
33	0.00555641028785659\\
34	0.00555640388747161\\
35	0.00555639737531164\\
36	0.00555639074941891\\
37	0.00555638400780123\\
38	0.00555637714843142\\
39	0.00555637016924665\\
40	0.00555636306814791\\
41	0.00555635584299926\\
42	0.00555634849162714\\
43	0.00555634101181982\\
44	0.00555633340132675\\
45	0.00555632565785775\\
46	0.00555631777908234\\
47	0.0055563097626291\\
48	0.00555630160608489\\
49	0.00555629330699411\\
50	0.00555628486285798\\
51	0.00555627627113374\\
52	0.00555626752923394\\
53	0.00555625863452551\\
54	0.00555624958432911\\
55	0.00555624037591822\\
56	0.00555623100651832\\
57	0.00555622147330605\\
58	0.00555621177340834\\
59	0.00555620190390153\\
60	0.00555619186181047\\
61	0.00555618164410758\\
62	0.005556171247712\\
63	0.00555616066948858\\
64	0.0055561499062469\\
65	0.00555613895474034\\
66	0.00555612781166499\\
67	0.00555611647365882\\
68	0.00555610493730038\\
69	0.00555609319910802\\
70	0.00555608125553862\\
71	0.00555606910298657\\
72	0.00555605673778266\\
73	0.00555604415619292\\
74	0.00555603135441743\\
75	0.00555601832858933\\
76	0.00555600507477335\\
77	0.0055559915889648\\
78	0.00555597786708824\\
79	0.00555596390499621\\
80	0.00555594969846801\\
81	0.00555593524320828\\
82	0.00555592053484576\\
83	0.00555590556893193\\
84	0.00555589034093952\\
85	0.00555587484626118\\
86	0.00555585908020805\\
87	0.00555584303800822\\
88	0.00555582671480538\\
89	0.00555581010565713\\
90	0.00555579320553354\\
91	0.00555577600931549\\
92	0.00555575851179316\\
93	0.00555574070766432\\
94	0.00555572259153268\\
95	0.00555570415790615\\
96	0.00555568540119518\\
97	0.00555566631571092\\
98	0.00555564689566349\\
99	0.00555562713516005\\
100	0.00555560702820297\\
101	0.00555558656868798\\
102	0.00555556575040211\\
103	0.0055555445670218\\
104	0.00555552301211084\\
105	0.00555550107911823\\
106	0.00555547876137623\\
107	0.00555545605209811\\
108	0.00555543294437596\\
109	0.00555540943117851\\
110	0.00555538550534878\\
111	0.00555536115960185\\
112	0.00555533638652235\\
113	0.00555531117856221\\
114	0.0055552855280381\\
115	0.00555525942712889\\
116	0.00555523286787309\\
117	0.00555520584216623\\
118	0.00555517834175824\\
119	0.00555515035825068\\
120	0.00555512188309391\\
121	0.00555509290758422\\
122	0.00555506342286103\\
123	0.00555503341990391\\
124	0.00555500288952936\\
125	0.0055549718223879\\
126	0.00555494020896089\\
127	0.00555490803955718\\
128	0.00555487530430996\\
129	0.00555484199317324\\
130	0.0055548080959185\\
131	0.00555477360213113\\
132	0.00555473850120685\\
133	0.005554702782348\\
134	0.00555466643455983\\
135	0.00555462944664663\\
136	0.00555459180720787\\
137	0.005554553504634\\
138	0.00555451452710263\\
139	0.00555447486257415\\
140	0.00555443449878744\\
141	0.00555439342325572\\
142	0.00555435162326185\\
143	0.00555430908585392\\
144	0.00555426579784046\\
145	0.00555422174578586\\
146	0.00555417691600541\\
147	0.00555413129456037\\
148	0.00555408486725289\\
149	0.00555403761962097\\
150	0.00555398953693303\\
151	0.0055539406041827\\
152	0.00555389080608334\\
153	0.00555384012706248\\
154	0.00555378855125606\\
155	0.00555373606250292\\
156	0.00555368264433869\\
157	0.0055536282799901\\
158	0.00555357295236878\\
159	0.00555351664406516\\
160	0.00555345933734259\\
161	0.00555340101413062\\
162	0.00555334165601911\\
163	0.00555328124425156\\
164	0.00555321975971893\\
165	0.00555315718295301\\
166	0.00555309349412002\\
167	0.00555302867301414\\
168	0.00555296269905097\\
169	0.00555289555126099\\
170	0.00555282720828328\\
171	0.00555275764835887\\
172	0.00555268684932458\\
173	0.00555261478860662\\
174	0.00555254144321438\\
175	0.00555246678973436\\
176	0.00555239080432414\\
177	0.00555231346270656\\
178	0.00555223474016394\\
179	0.00555215461153247\\
180	0.00555207305119675\\
181	0.00555199003308438\\
182	0.00555190553066068\\
183	0.00555181951692356\\
184	0.0055517319643983\\
185	0.00555164284513227\\
186	0.00555155213068955\\
187	0.00555145979214541\\
188	0.00555136580008011\\
189	0.00555127012457244\\
190	0.00555117273519229\\
191	0.00555107360099227\\
192	0.00555097269049799\\
193	0.00555086997169687\\
194	0.00555076541202491\\
195	0.00555065897835187\\
196	0.00555055063696405\\
197	0.00555044035354526\\
198	0.00555032809315718\\
199	0.00555021382022778\\
200	0.00555009749853958\\
201	0.00554997909121792\\
202	0.00554985856071892\\
203	0.00554973586881732\\
204	0.00554961097659401\\
205	0.00554948384442358\\
206	0.00554935443196139\\
207	0.00554922269813061\\
208	0.00554908860110916\\
209	0.00554895209831617\\
210	0.0055488131463986\\
211	0.00554867170121735\\
212	0.00554852771783352\\
213	0.00554838115049422\\
214	0.00554823195261837\\
215	0.0055480800767825\\
216	0.00554792547470603\\
217	0.00554776809723678\\
218	0.00554760789433616\\
219	0.00554744481506445\\
220	0.00554727880756594\\
221	0.00554710981905395\\
222	0.00554693779579595\\
223	0.0055467626830986\\
224	0.00554658442529286\\
225	0.00554640296571936\\
226	0.00554621824671349\\
227	0.00554603020959108\\
228	0.00554583879463392\\
229	0.00554564394107592\\
230	0.00554544558708915\\
231	0.0055452436697707\\
232	0.00554503812512965\\
233	0.00554482888807474\\
234	0.00554461589240254\\
235	0.00554439907078644\\
236	0.00554417835476615\\
237	0.00554395367473844\\
238	0.00554372495994837\\
239	0.00554349213848204\\
240	0.00554325513726026\\
241	0.0055430138820335\\
242	0.0055427682973783\\
243	0.00554251830669526\\
244	0.00554226383220837\\
245	0.00554200479496647\\
246	0.00554174111484607\\
247	0.0055414727105564\\
248	0.00554119949964632\\
249	0.00554092139851309\\
250	0.0055406383224134\\
251	0.00554035018547619\\
252	0.00554005690071743\\
253	0.00553975838005681\\
254	0.00553945453433581\\
255	0.00553914527333715\\
256	0.00553883050580524\\
257	0.00553851013946685\\
258	0.00553818408105177\\
259	0.00553785223631224\\
260	0.00553751451004054\\
261	0.00553717080608326\\
262	0.00553682102735139\\
263	0.005536465075824\\
264	0.00553610285254412\\
265	0.00553573425760475\\
266	0.00553535919012243\\
267	0.0055349775481963\\
268	0.00553458922884979\\
269	0.00553419412795238\\
270	0.00553379214011849\\
271	0.00553338315858182\\
272	0.00553296707504204\\
273	0.00553254377948372\\
274	0.00553211315996623\\
275	0.00553167510238606\\
276	0.00553122949021451\\
277	0.00553077620421378\\
278	0.00553031512213377\\
279	0.00552984611838764\\
280	0.0055293690637428\\
281	0.00552888382583964\\
282	0.00552839027033731\\
283	0.0055278882608906\\
284	0.0055273776591266\\
285	0.00552685832462144\\
286	0.00552633011487688\\
287	0.005525792885297\\
288	0.0055252464891646\\
289	0.00552469077761789\\
290	0.00552412559962685\\
291	0.00552355080196984\\
292	0.00552296622920996\\
293	0.00552237172367168\\
294	0.00552176712541721\\
295	0.00552115227222303\\
296	0.00552052699955636\\
297	0.00551989114055175\\
298	0.00551924452598748\\
299	0.00551858698426229\\
300	0.00551791834137189\\
301	0.00551723842088557\\
302	0.00551654704392292\\
303	0.0055158440291304\\
304	0.00551512919265814\\
305	0.00551440234813669\\
306	0.0055136633066536\\
307	0.00551291187673046\\
308	0.00551214786429946\\
309	0.00551137107268013\\
310	0.00551058130255629\\
311	0.0055097783519526\\
312	0.00550896201621118\\
313	0.00550813208796837\\
314	0.00550728835713096\\
315	0.00550643061085276\\
316	0.00550555863351065\\
317	0.00550467220668062\\
318	0.00550377110911339\\
319	0.00550285511670998\\
320	0.00550192400249677\\
321	0.00550097753660009\\
322	0.00550001548622049\\
323	0.00549903761560637\\
324	0.00549804368602693\\
325	0.00549703345574455\\
326	0.00549600667998613\\
327	0.00549496311091374\\
328	0.00549390249759393\\
329	0.00549282458596611\\
330	0.00549172911880951\\
331	0.0054906158357086\\
332	0.00548948447301702\\
333	0.00548833476381949\\
334	0.00548716643789195\\
335	0.00548597922165947\\
336	0.00548477283815144\\
337	0.00548354700695471\\
338	0.00548230144416329\\
339	0.00548103586232541\\
340	0.00547974997038695\\
341	0.0054784434736314\\
342	0.00547711607361598\\
343	0.00547576746810338\\
344	0.0054743973509896\\
345	0.00547300541222673\\
346	0.005471591337741\\
347	0.00547015480934577\\
348	0.00546869550464924\\
349	0.00546721309695638\\
350	0.00546570725516585\\
351	0.0054641776436606\\
352	0.00546262392219336\\
353	0.00546104574576619\\
354	0.00545944276450474\\
355	0.0054578146235275\\
356	0.00545616096281012\\
357	0.00545448141704604\\
358	0.00545277561550345\\
359	0.00545104318188011\\
360	0.00544928373415703\\
361	0.00544749688445256\\
362	0.00544568223887847\\
363	0.00544383939739993\\
364	0.00544196795370227\\
365	0.00544006749506626\\
366	0.00543813760225537\\
367	0.0054361778494178\\
368	0.00543418780400641\\
369	0.00543216702671998\\
370	0.00543011507146868\\
371	0.00542803148536672\\
372	0.00542591580875399\\
373	0.00542376757524806\\
374	0.00542158631182604\\
375	0.00541937153893374\\
376	0.00541712277061681\\
377	0.0054148395146646\\
378	0.00541252127275313\\
379	0.00541016754056764\\
380	0.00540777780787916\\
381	0.00540535155854355\\
382	0.00540288827038902\\
383	0.00540038741496609\\
384	0.00539784845716456\\
385	0.00539527085476927\\
386	0.00539265405805357\\
387	0.00538999750881576\\
388	0.00538730063262149\\
389	0.00538456269424022\\
390	0.00538178267450218\\
391	0.00537895948912136\\
392	0.00537609198820947\\
393	0.00537317895757003\\
394	0.00537021912229371\\
395	0.00536721115319308\\
396	0.00536415367653561\\
397	0.00536104528726815\\
398	0.00535788456594081\\
399	0.00535467009890918\\
400	0.00535140050032142\\
401	0.00534807443324338\\
402	0.00534469062506345\\
403	0.00534124786628992\\
404	0.00533774496506\\
405	0.00533418068466903\\
406	0.00533055374014991\\
407	0.00532686279465179\\
408	0.00532310645558164\\
409	0.00531928327028568\\
410	0.0053153917213833\\
411	0.0053114302217447\\
412	0.00530739710910915\\
413	0.00530329064035132\\
414	0.00529910898541757\\
415	0.00529485022097572\\
416	0.00529051232385152\\
417	0.00528609316436\\
418	0.00528159049964099\\
419	0.00527700196694627\\
420	0.00527232507615669\\
421	0.00526755720374208\\
422	0.00526269558811525\\
423	0.00525773732758168\\
424	0.0052526793845131\\
425	0.00524751860727432\\
426	0.00524225180987519\\
427	0.00523688478169057\\
428	0.00523142625771105\\
429	0.00522587457125705\\
430	0.00522022801988917\\
431	0.00521448486504337\\
432	0.00520864333315186\\
433	0.0052027016247223\\
434	0.00519665791541612\\
435	0.00519051035238131\\
436	0.00518425705670917\\
437	0.00517789612644616\\
438	0.005171425640055\\
439	0.00516484365994742\\
440	0.00515814823576862\\
441	0.00515133741063655\\
442	0.00514440922838081\\
443	0.00513736174019266\\
444	0.00513019301163666\\
445	0.00512290112979097\\
446	0.00511548421024411\\
447	0.00510794040379761\\
448	0.00510026790340343\\
449	0.00509246495436486\\
450	0.00508452987913126\\
451	0.00507646115502242\\
452	0.00506825492819732\\
453	0.00505988116691834\\
454	0.00505133523997187\\
455	0.00504261231016473\\
456	0.00503370731012738\\
457	0.00502461492443078\\
458	0.00501532957139311\\
459	0.00500584538340648\\
460	0.0049961561850006\\
461	0.00498625546799501\\
462	0.00497613636666769\\
463	0.0049657916299069\\
464	0.004955213591057\\
465	0.00494439413520791\\
466	0.00493332466344369\\
467	0.00492199605246461\\
468	0.00491039860409967\\
469	0.00489852196541588\\
470	0.00488635495113334\\
471	0.00487388502444875\\
472	0.00486110016872916\\
473	0.00484798846091057\\
474	0.00483453733968041\\
475	0.00482073356715673\\
476	0.00480656318852167\\
477	0.00479201148947316\\
478	0.00477706295186214\\
479	0.0047617012074535\\
480	0.00474590898969505\\
481	0.0047296680830029\\
482	0.00471295926900949\\
483	0.00469576226956938\\
484	0.00467805567959101\\
485	0.00465981685182614\\
486	0.00464102171153467\\
487	0.00462164470289859\\
488	0.00460165873227493\\
489	0.00458103299112202\\
490	0.00455972502689214\\
491	0.0045376851652642\\
492	0.00451487713810451\\
493	0.00449126302561625\\
494	0.00446680334652431\\
495	0.00444145660496162\\
496	0.00441517925139808\\
497	0.00438792565232686\\
498	0.00435964810079153\\
499	0.00433029686131218\\
500	0.0042998202602825\\
501	0.00426816483262271\\
502	0.00423527553509954\\
503	0.00420109605439452\\
504	0.0041655692281557\\
505	0.00412863760719055\\
506	0.00409024419353347\\
507	0.00405033339713849\\
508	0.00400885226367352\\
509	0.00396575203901821\\
510	0.0039209901516933\\
511	0.00387453271320261\\
512	0.00382635766291394\\
513	0.00377645868634219\\
514	0.00372485008048032\\
515	0.00367157241905168\\
516	0.0036167007092409\\
517	0.00356035948260369\\
518	0.00350273687447986\\
519	0.00344408652000935\\
520	0.00338472437097065\\
521	0.00332502096721129\\
522	0.00326555364965393\\
523	0.00320726580269645\\
524	0.00315410754475045\\
525	0.00310637335838345\\
526	0.00306411528959599\\
527	0.00302688819689068\\
528	0.00299070228713309\\
529	0.00295512081968544\\
530	0.00291990283029932\\
531	0.00288470889343574\\
532	0.00284941466422512\\
533	0.00281394805285373\\
534	0.00277826688557104\\
535	0.00274196152703712\\
536	0.00270497313438378\\
537	0.00266726288515937\\
538	0.0026287901198124\\
539	0.00258951542707469\\
540	0.0025494161112003\\
541	0.00250846903797176\\
542	0.00246665102411426\\
543	0.00242393945402507\\
544	0.00238030871636102\\
545	0.00233567897679858\\
546	0.00229069692309558\\
547	0.00224669736708306\\
548	0.00220394841017211\\
549	0.00216151286075968\\
550	0.00211858016627274\\
551	0.00207506500891316\\
552	0.00203094114340961\\
553	0.00198621792500414\\
554	0.00194091184632872\\
555	0.00189504487825128\\
556	0.00184864360533295\\
557	0.00180173986240179\\
558	0.00175437208562482\\
559	0.00170658633949408\\
560	0.00165892004054196\\
561	0.00161280548965377\\
562	0.00156727794385113\\
563	0.00152145163057755\\
564	0.00147531828604501\\
565	0.0014289079242206\\
566	0.00138225391875487\\
567	0.00133539122235288\\
568	0.00128835585617032\\
569	0.00124118449461087\\
570	0.00119391396189252\\
571	0.00114688118876282\\
572	0.00110017098334024\\
573	0.00105332337801152\\
574	0.00100637555536751\\
575	0.000959368346163701\\
576	0.000912345329873068\\
577	0.000865352773868407\\
578	0.00081843950911817\\
579	0.000771656718608598\\
580	0.000725057614047103\\
581	0.000678696970711006\\
582	0.000632630483284291\\
583	0.000586913897120376\\
584	0.000541601859672158\\
585	0.000496746426604039\\
586	0.000452395149029589\\
587	0.000408588670365811\\
588	0.000365357795666471\\
589	0.000322720123375259\\
590	0.000280676711614414\\
591	0.000239312817487403\\
592	0.00019878864496184\\
593	0.000159293651685586\\
594	0.000121079888727804\\
595	8.45520570083909e-05\\
596	5.05092148680371e-05\\
597	2.07908715710836e-05\\
598	0\\
599	0\\
600	0\\
};
\addplot [color=mycolor6,solid,forget plot]
  table[row sep=crcr]{%
1	0.00564414169708746\\
2	0.00564413625517687\\
3	0.00564413071987849\\
4	0.00564412508958927\\
5	0.00564411936267863\\
6	0.00564411353748807\\
7	0.00564410761233051\\
8	0.00564410158549004\\
9	0.00564409545522124\\
10	0.00564408921974873\\
11	0.00564408287726674\\
12	0.00564407642593847\\
13	0.0056440698638956\\
14	0.00564406318923781\\
15	0.00564405640003209\\
16	0.00564404949431236\\
17	0.00564404247007876\\
18	0.00564403532529714\\
19	0.00564402805789849\\
20	0.00564402066577821\\
21	0.00564401314679571\\
22	0.00564400549877365\\
23	0.00564399771949732\\
24	0.00564398980671406\\
25	0.00564398175813259\\
26	0.00564397357142226\\
27	0.00564396524421251\\
28	0.00564395677409218\\
29	0.00564394815860864\\
30	0.00564393939526732\\
31	0.00564393048153081\\
32	0.00564392141481821\\
33	0.00564391219250438\\
34	0.00564390281191916\\
35	0.00564389327034665\\
36	0.00564388356502433\\
37	0.00564387369314236\\
38	0.00564386365184279\\
39	0.00564385343821856\\
40	0.00564384304931287\\
41	0.00564383248211816\\
42	0.00564382173357545\\
43	0.00564381080057326\\
44	0.00564379967994678\\
45	0.00564378836847701\\
46	0.00564377686288969\\
47	0.00564376515985457\\
48	0.00564375325598425\\
49	0.00564374114783338\\
50	0.00564372883189747\\
51	0.00564371630461208\\
52	0.00564370356235165\\
53	0.00564369060142858\\
54	0.00564367741809204\\
55	0.0056436640085269\\
56	0.0056436503688528\\
57	0.00564363649512285\\
58	0.00564362238332254\\
59	0.00564360802936866\\
60	0.00564359342910799\\
61	0.00564357857831631\\
62	0.00564356347269692\\
63	0.0056435481078796\\
64	0.00564353247941936\\
65	0.00564351658279498\\
66	0.00564350041340793\\
67	0.00564348396658084\\
68	0.00564346723755637\\
69	0.00564345022149561\\
70	0.00564343291347689\\
71	0.00564341530849428\\
72	0.00564339740145613\\
73	0.00564337918718361\\
74	0.00564336066040927\\
75	0.00564334181577543\\
76	0.00564332264783279\\
77	0.0056433031510387\\
78	0.00564328331975563\\
79	0.0056432631482496\\
80	0.00564324263068842\\
81	0.00564322176114014\\
82	0.00564320053357121\\
83	0.0056431789418448\\
84	0.00564315697971908\\
85	0.00564313464084535\\
86	0.00564311191876622\\
87	0.00564308880691379\\
88	0.00564306529860771\\
89	0.00564304138705329\\
90	0.00564301706533944\\
91	0.00564299232643684\\
92	0.00564296716319575\\
93	0.00564294156834404\\
94	0.00564291553448503\\
95	0.00564288905409542\\
96	0.005642862119523\\
97	0.00564283472298454\\
98	0.00564280685656343\\
99	0.00564277851220748\\
100	0.00564274968172658\\
101	0.00564272035679014\\
102	0.00564269052892486\\
103	0.00564266018951221\\
104	0.00564262932978591\\
105	0.00564259794082932\\
106	0.00564256601357286\\
107	0.00564253353879135\\
108	0.00564250050710135\\
109	0.00564246690895835\\
110	0.00564243273465392\\
111	0.0056423979743129\\
112	0.00564236261789058\\
113	0.00564232665516947\\
114	0.00564229007575648\\
115	0.00564225286907973\\
116	0.00564221502438551\\
117	0.00564217653073491\\
118	0.00564213737700059\\
119	0.00564209755186348\\
120	0.00564205704380929\\
121	0.00564201584112511\\
122	0.00564197393189578\\
123	0.00564193130400025\\
124	0.0056418879451079\\
125	0.00564184384267478\\
126	0.00564179898393962\\
127	0.00564175335591998\\
128	0.0056417069454081\\
129	0.00564165973896684\\
130	0.00564161172292541\\
131	0.00564156288337497\\
132	0.00564151320616429\\
133	0.00564146267689518\\
134	0.00564141128091767\\
135	0.00564135900332546\\
136	0.00564130582895089\\
137	0.00564125174235987\\
138	0.00564119672784683\\
139	0.00564114076942943\\
140	0.00564108385084303\\
141	0.00564102595553514\\
142	0.00564096706665966\\
143	0.005640907167071\\
144	0.00564084623931801\\
145	0.00564078426563768\\
146	0.00564072122794868\\
147	0.00564065710784477\\
148	0.00564059188658799\\
149	0.00564052554510143\\
150	0.0056404580639621\\
151	0.0056403894233934\\
152	0.00564031960325727\\
153	0.00564024858304622\\
154	0.00564017634187505\\
155	0.00564010285847223\\
156	0.00564002811117116\\
157	0.00563995207790084\\
158	0.00563987473617662\\
159	0.00563979606309024\\
160	0.00563971603529977\\
161	0.00563963462901919\\
162	0.0056395518200075\\
163	0.00563946758355766\\
164	0.00563938189448501\\
165	0.00563929472711542\\
166	0.00563920605527308\\
167	0.00563911585226782\\
168	0.00563902409088234\\
169	0.00563893074335885\\
170	0.00563883578138554\\
171	0.00563873917608285\\
172	0.00563864089798935\\
173	0.00563854091704757\\
174	0.00563843920258979\\
175	0.00563833572332355\\
176	0.00563823044731771\\
177	0.0056381233419882\\
178	0.00563801437408465\\
179	0.00563790350967728\\
180	0.00563779071414467\\
181	0.00563767595216268\\
182	0.00563755918769461\\
183	0.00563744038398316\\
184	0.00563731950354441\\
185	0.00563719650816463\\
186	0.00563707135889997\\
187	0.00563694401607988\\
188	0.00563681443931492\\
189	0.00563668258750926\\
190	0.00563654841887897\\
191	0.0056364118909759\\
192	0.00563627296071807\\
193	0.00563613158442602\\
194	0.00563598771786444\\
195	0.00563584131628652\\
196	0.00563569233447492\\
197	0.00563554072676165\\
198	0.00563538644697248\\
199	0.00563522944814338\\
200	0.00563506968249567\\
201	0.00563490710142288\\
202	0.00563474165547733\\
203	0.00563457329435654\\
204	0.00563440196688957\\
205	0.00563422762102303\\
206	0.00563405020380734\\
207	0.00563386966138245\\
208	0.00563368593896365\\
209	0.00563349898082722\\
210	0.00563330873029588\\
211	0.00563311512972439\\
212	0.0056329181204847\\
213	0.00563271764295124\\
214	0.00563251363648603\\
215	0.00563230603942387\\
216	0.0056320947890571\\
217	0.00563187982162077\\
218	0.00563166107227743\\
219	0.00563143847510205\\
220	0.00563121196306666\\
221	0.00563098146802532\\
222	0.00563074692069891\\
223	0.00563050825065989\\
224	0.00563026538631715\\
225	0.00563001825490073\\
226	0.00562976678244683\\
227	0.0056295108937827\\
228	0.00562925051251158\\
229	0.00562898556099765\\
230	0.00562871596035131\\
231	0.0056284416304141\\
232	0.00562816248974406\\
233	0.00562787845560098\\
234	0.00562758944393172\\
235	0.0056272953693556\\
236	0.00562699614514988\\
237	0.00562669168323515\\
238	0.00562638189416079\\
239	0.0056260666870904\\
240	0.00562574596978704\\
241	0.00562541964859858\\
242	0.00562508762844261\\
243	0.0056247498127913\\
244	0.00562440610365575\\
245	0.00562405640157011\\
246	0.00562370060557519\\
247	0.00562333861320116\\
248	0.00562297032044985\\
249	0.00562259562177602\\
250	0.00562221441006732\\
251	0.00562182657662356\\
252	0.00562143201113391\\
253	0.00562103060165317\\
254	0.00562062223457563\\
255	0.00562020679460738\\
256	0.00561978416473611\\
257	0.00561935422619877\\
258	0.00561891685844633\\
259	0.0056184719391061\\
260	0.00561801934394091\\
261	0.00561755894680571\\
262	0.00561709061960082\\
263	0.00561661423222263\\
264	0.00561612965251152\\
265	0.00561563674619742\\
266	0.00561513537684337\\
267	0.00561462540578792\\
268	0.00561410669208684\\
269	0.00561357909245545\\
270	0.00561304246121241\\
271	0.00561249665022646\\
272	0.005611941508868\\
273	0.00561137688396609\\
274	0.00561080261977376\\
275	0.00561021855794219\\
276	0.00560962453750519\\
277	0.00560902039487498\\
278	0.00560840596384905\\
279	0.00560778107562823\\
280	0.00560714555884466\\
281	0.00560649923957224\\
282	0.00560584194129897\\
283	0.00560517348489965\\
284	0.00560449368860877\\
285	0.00560380236799386\\
286	0.00560309933592902\\
287	0.00560238440256894\\
288	0.0056016573753236\\
289	0.00560091805883311\\
290	0.00560016625494342\\
291	0.00559940176268257\\
292	0.00559862437823763\\
293	0.00559783389493254\\
294	0.00559703010320651\\
295	0.00559621279059369\\
296	0.0055953817417038\\
297	0.00559453673820359\\
298	0.00559367755879988\\
299	0.00559280397922369\\
300	0.00559191577221573\\
301	0.00559101270751374\\
302	0.00559009455184119\\
303	0.00558916106889795\\
304	0.00558821201935285\\
305	0.00558724716083838\\
306	0.00558626624794766\\
307	0.00558526903223386\\
308	0.00558425526221214\\
309	0.00558322468336484\\
310	0.00558217703814907\\
311	0.00558111206600827\\
312	0.00558002950338697\\
313	0.00557892908374936\\
314	0.00557781053760209\\
315	0.00557667359252137\\
316	0.00557551797318496\\
317	0.00557434340140912\\
318	0.00557314959619112\\
319	0.00557193627375752\\
320	0.00557070314761891\\
321	0.00556944992863151\\
322	0.00556817632506571\\
323	0.00556688204268277\\
324	0.0055655667848199\\
325	0.0055642302524841\\
326	0.00556287214445611\\
327	0.00556149215740428\\
328	0.00556008998601007\\
329	0.00555866532310511\\
330	0.0055572178598215\\
331	0.00555574728575555\\
332	0.00555425328914656\\
333	0.00555273555707142\\
334	0.00555119377565612\\
335	0.00554962763030557\\
336	0.00554803680595303\\
337	0.00554642098733027\\
338	0.00554477985926037\\
339	0.00554311310697427\\
340	0.00554142041645326\\
341	0.00553970147479857\\
342	0.00553795597063039\\
343	0.00553618359451826\\
344	0.00553438403944448\\
345	0.0055325570013031\\
346	0.00553070217943665\\
347	0.00552881927721283\\
348	0.00552690800264369\\
349	0.00552496806904948\\
350	0.00552299919577002\\
351	0.00552100110892575\\
352	0.00551897354223103\\
353	0.00551691623786212\\
354	0.00551482894738188\\
355	0.00551271143272327\\
356	0.00551056346723349\\
357	0.00550838483677973\\
358	0.00550617534091777\\
359	0.00550393479412282\\
360	0.00550166302708257\\
361	0.00549935988804942\\
362	0.00549702524424956\\
363	0.00549465898334295\\
364	0.00549226101492717\\
365	0.00548983127207504\\
366	0.00548736971289305\\
367	0.00548487632208314\\
368	0.00548235111248626\\
369	0.00547979412658058\\
370	0.00547720543789952\\
371	0.00547458515232831\\
372	0.00547193340922715\\
373	0.00546925038231827\\
374	0.0054665362802615\\
375	0.00546379134682676\\
376	0.00546101586055519\\
377	0.00545821013377916\\
378	0.00545537451084841\\
379	0.00545250936538243\\
380	0.00544961509633896\\
381	0.00544669212265489\\
382	0.00544374087617907\\
383	0.00544076179257658\\
384	0.0054377552998406\\
385	0.0054347218039998\\
386	0.0054316616715437\\
387	0.00542857520801726\\
388	0.00542546263241958\\
389	0.00542232405135473\\
390	0.00541915943173047\\
391	0.00541596856053574\\
392	0.00541275099852629\\
393	0.00540950602802968\\
394	0.00540623259578367\\
395	0.00540292925287511\\
396	0.00539959409575478\\
397	0.00539622471454812\\
398	0.00539281815841277\\
399	0.00538937093322477\\
400	0.00538587905482985\\
401	0.00538233819374411\\
402	0.00537874396997767\\
403	0.00537509250059577\\
404	0.00537138136169903\\
405	0.00536760963869151\\
406	0.00536377640670455\\
407	0.00535988073120857\\
408	0.00535592166875451\\
409	0.00535189826786303\\
410	0.00534780957007553\\
411	0.00534365461117891\\
412	0.0053394324226149\\
413	0.00533514203308011\\
414	0.00533078247031827\\
415	0.0053263527630979\\
416	0.00532185194335856\\
417	0.0053172790484941\\
418	0.0053126331237232\\
419	0.00530791322447705\\
420	0.00530311841874507\\
421	0.00529824778923973\\
422	0.00529330043525395\\
423	0.00528827547412149\\
424	0.00528317204227609\\
425	0.00527798929587917\\
426	0.00527272640940693\\
427	0.00526738255837269\\
428	0.00526195660210042\\
429	0.00525644697598015\\
430	0.00525085202010332\\
431	0.00524516996856736\\
432	0.00523939893776977\\
433	0.00523353691359344\\
434	0.00522758173720634\\
435	0.00522153108974944\\
436	0.00521538247611547\\
437	0.00520913320773026\\
438	0.00520278038449278\\
439	0.0051963208761216\\
440	0.00518975130328123\\
441	0.0051830680190074\\
442	0.00517626709101539\\
443	0.00516934428574013\\
444	0.00516229505523642\\
445	0.00515511452830036\\
446	0.00514779750747866\\
447	0.00514033847401999\\
448	0.00513273160350065\\
449	0.00512497079664492\\
450	0.00511704973592452\\
451	0.00510896200076135\\
452	0.00510070378862286\\
453	0.00509229594263602\\
454	0.00508373464752929\\
455	0.00507501624632246\\
456	0.00506613720024991\\
457	0.00505709388351448\\
458	0.00504788255170718\\
459	0.00503849933846256\\
460	0.00502894026034941\\
461	0.00501920122093163\\
462	0.00500927797509956\\
463	0.00499916613887364\\
464	0.00498886118326885\\
465	0.0049783584277492\\
466	0.0049676530324458\\
467	0.00495673999020768\\
468	0.00494561412153549\\
469	0.00493427008262226\\
470	0.00492270241860566\\
471	0.00491090576179196\\
472	0.00489884744474724\\
473	0.00488650547011184\\
474	0.00487386914060477\\
475	0.0048609271690671\\
476	0.00484766764446001\\
477	0.00483407799596508\\
478	0.00482014494740563\\
479	0.004805854470123\\
480	0.00479119173146963\\
481	0.00477614103689769\\
482	0.00476068576490874\\
483	0.00474480829222686\\
484	0.00472848990354356\\
485	0.00471171066811643\\
486	0.00469444922446338\\
487	0.00467668224998897\\
488	0.00465838538681553\\
489	0.00463953438165116\\
490	0.00462010347189625\\
491	0.00460006522725015\\
492	0.00457938821124362\\
493	0.00455802863850385\\
494	0.00453593713953987\\
495	0.00451307676840705\\
496	0.00448940875107804\\
497	0.00446489274774366\\
498	0.00443948628249789\\
499	0.00441314468988377\\
500	0.00438582105601528\\
501	0.00435746622138655\\
502	0.0043280288572434\\
503	0.00429745539131718\\
504	0.00426569009173831\\
505	0.00423267521069181\\
506	0.00419835120334439\\
507	0.00416265704191008\\
508	0.00412553064833676\\
509	0.00408690947531639\\
510	0.00404673127247722\\
511	0.0040049350833286\\
512	0.00396146252888379\\
513	0.00391625944696082\\
514	0.00386927797080916\\
515	0.00382047918193405\\
516	0.00376983645483958\\
517	0.00371733938793335\\
518	0.00366299856508242\\
519	0.0036068522353349\\
520	0.00354898254480159\\
521	0.00348952212899389\\
522	0.00342866087146759\\
523	0.00336665476859641\\
524	0.00330382152996528\\
525	0.00324053311458391\\
526	0.0031773849182677\\
527	0.00311536749133956\\
528	0.00305847479549083\\
529	0.00300703652234188\\
530	0.0029611558647825\\
531	0.00292045584447661\\
532	0.00288119041155718\\
533	0.00284263737455762\\
534	0.00280433019479108\\
535	0.00276613286193941\\
536	0.00272783883980301\\
537	0.00268936526453775\\
538	0.00265065306074691\\
539	0.00261155905477273\\
540	0.00257176095863499\\
541	0.0025312176445629\\
542	0.0024898888746985\\
543	0.00244772979898243\\
544	0.00240470570191919\\
545	0.00236079274626476\\
546	0.00231596758188079\\
547	0.0022701870295366\\
548	0.00222336772233065\\
549	0.0021766009736321\\
550	0.00213084111050196\\
551	0.00208636429435196\\
552	0.00204199580507713\\
553	0.00199714708603788\\
554	0.00195173779208016\\
555	0.00190574754438444\\
556	0.00185919235620512\\
557	0.0018120959308981\\
558	0.00176448917511599\\
559	0.00171640898605244\\
560	0.00166789981490248\\
561	0.00161900323277768\\
562	0.00157077130801954\\
563	0.00152412909216968\\
564	0.00147778367616564\\
565	0.00143118179454771\\
566	0.00138432726713373\\
567	0.00133725430873315\\
568	0.00129000081580688\\
569	0.00124260613446559\\
570	0.00119511043658036\\
571	0.0011475541064327\\
572	0.00110018398429605\\
573	0.00105332347969338\\
574	0.00100637557683123\\
575	0.000959368356499549\\
576	0.000912345335012991\\
577	0.000865352776319937\\
578	0.000818439510213676\\
579	0.000771656719059582\\
580	0.000725057614214845\\
581	0.000678696970765909\\
582	0.000632630483299512\\
583	0.000586913897123733\\
584	0.00054160185967268\\
585	0.000496746426604084\\
586	0.000452395149029592\\
587	0.000408588670365816\\
588	0.000365357795666476\\
589	0.000322720123375263\\
590	0.000280676711614417\\
591	0.000239312817487403\\
592	0.00019878864496184\\
593	0.000159293651685585\\
594	0.000121079888727804\\
595	8.45520570083909e-05\\
596	5.05092148680374e-05\\
597	2.07908715710837e-05\\
598	0\\
599	0\\
600	0\\
};
\addplot [color=mycolor7,solid,forget plot]
  table[row sep=crcr]{%
1	0.00588219123805133\\
2	0.00588218208726686\\
3	0.00588217278113544\\
4	0.00588216331702408\\
5	0.00588215369225527\\
6	0.00588214390410623\\
7	0.00588213394980824\\
8	0.00588212382654572\\
9	0.00588211353145555\\
10	0.00588210306162625\\
11	0.00588209241409704\\
12	0.00588208158585732\\
13	0.00588207057384544\\
14	0.00588205937494817\\
15	0.00588204798599965\\
16	0.00588203640378059\\
17	0.00588202462501728\\
18	0.00588201264638077\\
19	0.00588200046448591\\
20	0.00588198807589044\\
21	0.00588197547709392\\
22	0.00588196266453694\\
23	0.00588194963459998\\
24	0.00588193638360238\\
25	0.00588192290780144\\
26	0.00588190920339136\\
27	0.00588189526650207\\
28	0.00588188109319825\\
29	0.00588186667947823\\
30	0.00588185202127291\\
31	0.00588183711444447\\
32	0.0058818219547854\\
33	0.00588180653801729\\
34	0.00588179085978959\\
35	0.00588177491567842\\
36	0.00588175870118537\\
37	0.00588174221173632\\
38	0.00588172544267996\\
39	0.00588170838928676\\
40	0.00588169104674747\\
41	0.00588167341017188\\
42	0.00588165547458745\\
43	0.00588163723493796\\
44	0.00588161868608199\\
45	0.00588159982279169\\
46	0.00588158063975125\\
47	0.00588156113155537\\
48	0.00588154129270789\\
49	0.00588152111762016\\
50	0.00588150060060958\\
51	0.00588147973589798\\
52	0.00588145851761012\\
53	0.00588143693977191\\
54	0.00588141499630895\\
55	0.00588139268104478\\
56	0.00588136998769915\\
57	0.00588134690988632\\
58	0.0058813234411134\\
59	0.00588129957477834\\
60	0.00588127530416842\\
61	0.00588125062245813\\
62	0.00588122552270752\\
63	0.00588119999786017\\
64	0.00588117404074132\\
65	0.00588114764405584\\
66	0.00588112080038635\\
67	0.00588109350219113\\
68	0.00588106574180199\\
69	0.0058810375114224\\
70	0.00588100880312507\\
71	0.00588097960885001\\
72	0.00588094992040228\\
73	0.00588091972944976\\
74	0.00588088902752087\\
75	0.00588085780600227\\
76	0.00588082605613652\\
77	0.00588079376901975\\
78	0.00588076093559925\\
79	0.00588072754667089\\
80	0.00588069359287686\\
81	0.00588065906470292\\
82	0.00588062395247593\\
83	0.00588058824636128\\
84	0.00588055193636014\\
85	0.00588051501230688\\
86	0.00588047746386625\\
87	0.00588043928053062\\
88	0.00588040045161711\\
89	0.00588036096626489\\
90	0.00588032081343208\\
91	0.00588027998189295\\
92	0.00588023846023473\\
93	0.00588019623685472\\
94	0.00588015329995716\\
95	0.00588010963755001\\
96	0.00588006523744187\\
97	0.00588002008723857\\
98	0.00587997417434008\\
99	0.005879927485937\\
100	0.00587988000900718\\
101	0.0058798317303124\\
102	0.00587978263639473\\
103	0.00587973271357297\\
104	0.00587968194793907\\
105	0.00587963032535454\\
106	0.0058795778314466\\
107	0.00587952445160447\\
108	0.00587947017097546\\
109	0.00587941497446109\\
110	0.00587935884671325\\
111	0.00587930177212999\\
112	0.00587924373485153\\
113	0.00587918471875616\\
114	0.00587912470745592\\
115	0.00587906368429241\\
116	0.0058790016323324\\
117	0.00587893853436342\\
118	0.0058788743728894\\
119	0.00587880913012591\\
120	0.00587874278799578\\
121	0.00587867532812425\\
122	0.00587860673183428\\
123	0.00587853698014171\\
124	0.00587846605375035\\
125	0.00587839393304693\\
126	0.00587832059809621\\
127	0.00587824602863568\\
128	0.00587817020407035\\
129	0.0058780931034675\\
130	0.00587801470555129\\
131	0.0058779349886972\\
132	0.00587785393092651\\
133	0.00587777150990062\\
134	0.00587768770291526\\
135	0.00587760248689474\\
136	0.00587751583838573\\
137	0.00587742773355146\\
138	0.00587733814816538\\
139	0.00587724705760487\\
140	0.00587715443684479\\
141	0.00587706026045104\\
142	0.00587696450257372\\
143	0.00587686713694038\\
144	0.00587676813684901\\
145	0.00587666747516092\\
146	0.00587656512429337\\
147	0.00587646105621216\\
148	0.00587635524242384\\
149	0.00587624765396788\\
150	0.00587613826140854\\
151	0.00587602703482644\\
152	0.00587591394381001\\
153	0.00587579895744645\\
154	0.00587568204431256\\
155	0.00587556317246514\\
156	0.00587544230943093\\
157	0.00587531942219636\\
158	0.00587519447719651\\
159	0.00587506744030385\\
160	0.00587493827681619\\
161	0.00587480695144415\\
162	0.0058746734282978\\
163	0.0058745376708726\\
164	0.00587439964203436\\
165	0.00587425930400317\\
166	0.00587411661833645\\
167	0.00587397154591045\\
168	0.00587382404690054\\
169	0.00587367408075979\\
170	0.00587352160619589\\
171	0.00587336658114587\\
172	0.00587320896274888\\
173	0.00587304870731608\\
174	0.00587288577029793\\
175	0.00587272010624825\\
176	0.0058725516687845\\
177	0.0058723804105444\\
178	0.0058722062831377\\
179	0.00587202923709321\\
180	0.00587184922180032\\
181	0.00587166618544422\\
182	0.00587148007493482\\
183	0.005871290835828\\
184	0.0058710984122396\\
185	0.00587090274675077\\
186	0.00587070378030529\\
187	0.00587050145209808\\
188	0.00587029569945591\\
189	0.00587008645771111\\
190	0.0058698736600702\\
191	0.00586965723748165\\
192	0.00586943711850872\\
193	0.0058692132292203\\
194	0.00586898549312512\\
195	0.0058687538312082\\
196	0.00586851816221758\\
197	0.00586827840359564\\
198	0.00586803447412705\\
199	0.00586778630120221\\
200	0.00586753381139697\\
201	0.00586727693003613\\
202	0.0058670155811731\\
203	0.00586674968756943\\
204	0.00586647917067383\\
205	0.00586620395060098\\
206	0.00586592394611\\
207	0.00586563907458275\\
208	0.00586534925200168\\
209	0.00586505439292744\\
210	0.00586475441047631\\
211	0.00586444921629699\\
212	0.00586413872054743\\
213	0.00586382283187114\\
214	0.00586350145737323\\
215	0.00586317450259602\\
216	0.0058628418714946\\
217	0.00586250346641158\\
218	0.00586215918805206\\
219	0.00586180893545772\\
220	0.00586145260598089\\
221	0.00586109009525804\\
222	0.00586072129718302\\
223	0.00586034610387987\\
224	0.00585996440567515\\
225	0.00585957609106991\\
226	0.00585918104671132\\
227	0.00585877915736358\\
228	0.00585837030587866\\
229	0.00585795437316637\\
230	0.00585753123816403\\
231	0.00585710077780552\\
232	0.0058566628669898\\
233	0.00585621737854895\\
234	0.00585576418321546\\
235	0.005855303149589\\
236	0.00585483414410254\\
237	0.00585435703098766\\
238	0.00585387167223928\\
239	0.00585337792757956\\
240	0.00585287565442103\\
241	0.00585236470782888\\
242	0.00585184494048238\\
243	0.00585131620263546\\
244	0.00585077834207637\\
245	0.00585023120408629\\
246	0.00584967463139692\\
247	0.00584910846414732\\
248	0.00584853253983935\\
249	0.0058479466932923\\
250	0.00584735075659636\\
251	0.00584674455906493\\
252	0.00584612792718618\\
253	0.005845500684573\\
254	0.00584486265191258\\
255	0.0058442136469145\\
256	0.0058435534842581\\
257	0.00584288197553906\\
258	0.00584219892921499\\
259	0.00584150415055066\\
260	0.00584079744156258\\
261	0.00584007860096303\\
262	0.00583934742410421\\
263	0.00583860370292198\\
264	0.00583784722588\\
265	0.0058370777779139\\
266	0.00583629514037624\\
267	0.00583549909098182\\
268	0.00583468940375375\\
269	0.00583386584897045\\
270	0.00583302819311345\\
271	0.00583217619881607\\
272	0.00583130962481249\\
273	0.00583042822588784\\
274	0.00582953175282786\\
275	0.00582861995236874\\
276	0.00582769256714601\\
277	0.00582674933564253\\
278	0.00582578999213452\\
279	0.00582481426663568\\
280	0.00582382188483897\\
281	0.00582281256805669\\
282	0.00582178603316027\\
283	0.00582074199251977\\
284	0.00581968015394315\\
285	0.0058186002206152\\
286	0.00581750189103652\\
287	0.00581638485896217\\
288	0.00581524881334033\\
289	0.00581409343825085\\
290	0.0058129184128441\\
291	0.00581172341127963\\
292	0.00581050810266517\\
293	0.00580927215099572\\
294	0.00580801521509329\\
295	0.0058067369485468\\
296	0.00580543699965241\\
297	0.00580411501135488\\
298	0.00580277062118932\\
299	0.0058014034612239\\
300	0.00580001315800364\\
301	0.00579859933249527\\
302	0.00579716160003328\\
303	0.00579569957026767\\
304	0.00579421284711314\\
305	0.00579270102870023\\
306	0.00579116370732854\\
307	0.00578960046942194\\
308	0.00578801089548685\\
309	0.00578639456007275\\
310	0.00578475103173618\\
311	0.00578307987300787\\
312	0.00578138064036386\\
313	0.00577965288420035\\
314	0.00577789614881355\\
315	0.00577610997238415\\
316	0.00577429388696708\\
317	0.0057724474184875\\
318	0.00577057008674308\\
319	0.00576866140541335\\
320	0.00576672088207668\\
321	0.00576474801823573\\
322	0.00576274230935219\\
323	0.00576070324489125\\
324	0.00575863030837713\\
325	0.00575652297746055\\
326	0.00575438072399893\\
327	0.00575220301415095\\
328	0.00574998930848624\\
329	0.00574773906211215\\
330	0.00574545172481836\\
331	0.00574312674124184\\
332	0.00574076355105323\\
333	0.00573836158916711\\
334	0.00573592028597793\\
335	0.00573343906762434\\
336	0.00573091735628411\\
337	0.00572835457050286\\
338	0.00572575012555948\\
339	0.00572310343387171\\
340	0.00572041390544555\\
341	0.00571768094837294\\
342	0.00571490396938198\\
343	0.00571208237444478\\
344	0.00570921556944856\\
345	0.00570630296093632\\
346	0.00570334395692322\\
347	0.00570033796779702\\
348	0.00569728440731028\\
349	0.00569418269367389\\
350	0.00569103225076202\\
351	0.00568783250943989\\
352	0.00568458290902718\\
353	0.00568128289891095\\
354	0.00567793194032404\\
355	0.00567452950830646\\
356	0.00567107509386924\\
357	0.00566756820638286\\
358	0.00566400837621448\\
359	0.00566039515764186\\
360	0.00565672813207408\\
361	0.00565300691161411\\
362	0.00564923114300126\\
363	0.00564540051197693\\
364	0.00564151474812193\\
365	0.00563757363021977\\
366	0.00563357699220581\\
367	0.00562952472977064\\
368	0.00562541680769257\\
369	0.00562125326798283\\
370	0.00561703423893659\\
371	0.00561275994519176\\
372	0.0056084307189075\\
373	0.00560404701218465\\
374	0.00559960941085824\\
375	0.00559511864980179\\
376	0.00559057562988734\\
377	0.00558598143674747\\
378	0.00558133736148083\\
379	0.00557664492342907\\
380	0.00557190589512597\\
381	0.00556712232947278\\
382	0.00556229658911968\\
383	0.00555743137792129\\
384	0.00555252977416729\\
385	0.00554759526504962\\
386	0.00554263178149162\\
387	0.00553764373200467\\
388	0.00553263603363741\\
389	0.00552761413726744\\
390	0.00552258404403517\\
391	0.00551755230432516\\
392	0.00551252599174899\\
393	0.00550751264045885\\
394	0.00550252012886844\\
395	0.00549755648547518\\
396	0.0054926295808739\\
397	0.00548774667063801\\
398	0.00548291373303975\\
399	0.00547813451476601\\
400	0.00547340916938308\\
401	0.00546873232430192\\
402	0.00546409034107289\\
403	0.00545945746060361\\
404	0.00545479065452195\\
405	0.00545005260128847\\
406	0.00544524264586333\\
407	0.00544036015771497\\
408	0.00543540453319037\\
409	0.00543037519796206\\
410	0.00542527160953924\\
411	0.00542009325982682\\
412	0.00541483967771263\\
413	0.0054095104316578\\
414	0.00540410513226049\\
415	0.00539862343475789\\
416	0.0053930650414277\\
417	0.00538742970385511\\
418	0.00538171722506037\\
419	0.00537592746158062\\
420	0.00537006032535792\\
421	0.00536411578537901\\
422	0.00535809386903167\\
423	0.00535199466314406\\
424	0.00534581831466087\\
425	0.0053395650308661\\
426	0.00533323507899149\\
427	0.00532682878551841\\
428	0.00532034654460782\\
429	0.00531378882878005\\
430	0.00530715618751142\\
431	0.00530044924399701\\
432	0.00529366868970234\\
433	0.00528681527658234\\
434	0.00527988980632675\\
435	0.00527289311561445\\
436	0.00526582605648935\\
437	0.00525868947083078\\
438	0.00525148415773978\\
439	0.00524421083250835\\
440	0.0052368700757049\\
441	0.00522946227082878\\
442	0.00522198752898299\\
443	0.00521444559914102\\
444	0.00520683576294551\\
445	0.00519915671370076\\
446	0.00519140642050471\\
447	0.0051835819805921\\
448	0.00517567946634349\\
449	0.00516769377863308\\
450	0.00515961852592683\\
451	0.005151445958989\\
452	0.00514316699999214\\
453	0.0051347713274009\\
454	0.00512624707099367\\
455	0.00511758257154256\\
456	0.00510876815839775\\
457	0.00509979921378732\\
458	0.0050906715587827\\
459	0.00508138073314177\\
460	0.00507192196861216\\
461	0.00506229016033015\\
462	0.00505247983613435\\
463	0.00504248512515146\\
464	0.00503229972450469\\
465	0.00502191686506552\\
466	0.00501132927698835\\
467	0.00500052915644913\\
468	0.00498950813675828\\
469	0.00497825727254909\\
470	0.0049667670647291\\
471	0.00495502762238985\\
472	0.0049430541617577\\
473	0.00493085021822894\\
474	0.00491840729180021\\
475	0.00490571632057168\\
476	0.00489276766813617\\
477	0.00487955113536917\\
478	0.00486605606432649\\
479	0.00485227131562047\\
480	0.00483818530089091\\
481	0.00482378604549152\\
482	0.00480906125528886\\
483	0.00479399839257722\\
484	0.00477858473533605\\
485	0.00476280739141535\\
486	0.00474665320711469\\
487	0.00473010861408686\\
488	0.00471314021482741\\
489	0.00469570132323347\\
490	0.00467777053771616\\
491	0.00465932506538153\\
492	0.00464034054077538\\
493	0.00462079104619254\\
494	0.00460064903615265\\
495	0.00457988331032009\\
496	0.00455845203088011\\
497	0.00453630139290828\\
498	0.00451339335556852\\
499	0.00448968753890518\\
500	0.00446514170582074\\
501	0.00443971084097204\\
502	0.0044133469625095\\
503	0.0043860022132521\\
504	0.00435762631213689\\
505	0.00432816648945773\\
506	0.00429756746269201\\
507	0.00426577144806584\\
508	0.00423271823782212\\
509	0.0041983453407199\\
510	0.00416258819980435\\
511	0.00412538050824811\\
512	0.00408665464676958\\
513	0.0040463422770657\\
514	0.00400437516965353\\
515	0.00396068628906017\\
516	0.00391521098887378\\
517	0.00386788870705819\\
518	0.00381866515000677\\
519	0.00376749505137355\\
520	0.00371434520876362\\
521	0.0036591985190329\\
522	0.00360205971707007\\
523	0.00354297258640321\\
524	0.00348201672566389\\
525	0.00341931897600911\\
526	0.00335506276957203\\
527	0.00328949729187063\\
528	0.00322293267212333\\
529	0.00315573339194042\\
530	0.00308848525482472\\
531	0.00302213661827374\\
532	0.00296050629345285\\
533	0.00290430468982034\\
534	0.00285370670007441\\
535	0.00280842771725674\\
536	0.00276558846072333\\
537	0.00272363262543577\\
538	0.00268211738834541\\
539	0.00264064638487863\\
540	0.00259910961814965\\
541	0.00255741339571434\\
542	0.00251548061943308\\
543	0.00247326495309667\\
544	0.00243050165550552\\
545	0.00238698086892067\\
546	0.00234265994943839\\
547	0.00229749658658826\\
548	0.00225144245025375\\
549	0.0022044690067269\\
550	0.00215651952593421\\
551	0.00210750156462815\\
552	0.00205873951659417\\
553	0.0020109931608278\\
554	0.00196454992977493\\
555	0.00191823942837985\\
556	0.00187148664771676\\
557	0.00182422356319178\\
558	0.00177642326600128\\
559	0.0017281096045976\\
560	0.00167931427601956\\
561	0.00163007921572574\\
562	0.00158045429977808\\
563	0.00153047646411673\\
564	0.00148151315013013\\
565	0.00143416985390059\\
566	0.00138705937387004\\
567	0.00133975888995393\\
568	0.00129226939301679\\
569	0.0012446284113432\\
570	0.00119687824718637\\
571	0.0011490625211639\\
572	0.00110122520101772\\
573	0.0010534821435598\\
574	0.00100637667104439\\
575	0.000959368504257833\\
576	0.000912345403221071\\
577	0.000865352810900999\\
578	0.000818439527258546\\
579	0.000771656726968869\\
580	0.00072505761760655\\
581	0.000678696972083882\\
582	0.000632630483751538\\
583	0.000586913897255451\\
584	0.000541601859703357\\
585	0.000496746426609176\\
586	0.000452395149030046\\
587	0.000408588670365811\\
588	0.000365357795666471\\
589	0.000322720123375259\\
590	0.000280676711614415\\
591	0.000239312817487403\\
592	0.000198788644961841\\
593	0.000159293651685586\\
594	0.000121079888727804\\
595	8.45520570083908e-05\\
596	5.05092148680372e-05\\
597	2.07908715710836e-05\\
598	0\\
599	0\\
600	0\\
};
\addplot [color=mycolor8,solid,forget plot]
  table[row sep=crcr]{%
1	0.00647624004771694\\
2	0.00647622387146877\\
3	0.00647620742204131\\
4	0.00647619069483262\\
5	0.00647617368516355\\
6	0.00647615638827646\\
7	0.0064761387993338\\
8	0.0064761209134169\\
9	0.00647610272552456\\
10	0.00647608423057166\\
11	0.00647606542338788\\
12	0.00647604629871608\\
13	0.00647602685121104\\
14	0.00647600707543784\\
15	0.0064759869658705\\
16	0.00647596651689038\\
17	0.00647594572278466\\
18	0.00647592457774483\\
19	0.00647590307586496\\
20	0.0064758812111402\\
21	0.00647585897746516\\
22	0.00647583636863205\\
23	0.0064758133783292\\
24	0.00647579000013921\\
25	0.00647576622753725\\
26	0.00647574205388915\\
27	0.00647571747244984\\
28	0.00647569247636121\\
29	0.00647566705865046\\
30	0.00647564121222805\\
31	0.00647561492988589\\
32	0.00647558820429525\\
33	0.00647556102800488\\
34	0.00647553339343885\\
35	0.0064755052928946\\
36	0.00647547671854082\\
37	0.00647544766241516\\
38	0.00647541811642238\\
39	0.00647538807233187\\
40	0.00647535752177552\\
41	0.00647532645624541\\
42	0.00647529486709158\\
43	0.00647526274551955\\
44	0.0064752300825881\\
45	0.00647519686920668\\
46	0.00647516309613301\\
47	0.00647512875397063\\
48	0.00647509383316633\\
49	0.00647505832400752\\
50	0.0064750222166196\\
51	0.00647498550096336\\
52	0.00647494816683225\\
53	0.00647491020384958\\
54	0.00647487160146572\\
55	0.00647483234895532\\
56	0.00647479243541441\\
57	0.00647475184975742\\
58	0.00647471058071415\\
59	0.0064746686168269\\
60	0.00647462594644722\\
61	0.00647458255773297\\
62	0.00647453843864493\\
63	0.00647449357694374\\
64	0.0064744479601865\\
65	0.00647440157572358\\
66	0.00647435441069509\\
67	0.00647430645202757\\
68	0.00647425768643044\\
69	0.00647420810039235\\
70	0.00647415768017784\\
71	0.00647410641182344\\
72	0.00647405428113408\\
73	0.00647400127367924\\
74	0.00647394737478918\\
75	0.00647389256955111\\
76	0.00647383684280508\\
77	0.00647378017914003\\
78	0.00647372256288978\\
79	0.00647366397812885\\
80	0.00647360440866827\\
81	0.00647354383805118\\
82	0.00647348224954876\\
83	0.00647341962615564\\
84	0.00647335595058546\\
85	0.00647329120526639\\
86	0.00647322537233646\\
87	0.00647315843363897\\
88	0.00647309037071768\\
89	0.006473021164812\\
90	0.00647295079685217\\
91	0.0064728792474542\\
92	0.00647280649691494\\
93	0.00647273252520689\\
94	0.00647265731197301\\
95	0.00647258083652153\\
96	0.00647250307782057\\
97	0.00647242401449275\\
98	0.00647234362480961\\
99	0.00647226188668611\\
100	0.00647217877767497\\
101	0.0064720942749609\\
102	0.00647200835535479\\
103	0.00647192099528782\\
104	0.0064718321708054\\
105	0.00647174185756121\\
106	0.006471650030811\\
107	0.00647155666540625\\
108	0.00647146173578796\\
109	0.0064713652159802\\
110	0.00647126707958353\\
111	0.00647116729976853\\
112	0.00647106584926904\\
113	0.0064709627003754\\
114	0.0064708578249275\\
115	0.00647075119430803\\
116	0.00647064277943524\\
117	0.00647053255075598\\
118	0.00647042047823828\\
119	0.00647030653136422\\
120	0.00647019067912238\\
121	0.00647007289000054\\
122	0.00646995313197789\\
123	0.00646983137251749\\
124	0.00646970757855844\\
125	0.0064695817165082\\
126	0.00646945375223438\\
127	0.00646932365105696\\
128	0.00646919137774018\\
129	0.00646905689648426\\
130	0.00646892017091722\\
131	0.00646878116408651\\
132	0.00646863983845066\\
133	0.00646849615587075\\
134	0.00646835007760198\\
135	0.00646820156428492\\
136	0.00646805057593706\\
137	0.0064678970719439\\
138	0.00646774101105035\\
139	0.00646758235135188\\
140	0.00646742105028574\\
141	0.00646725706462213\\
142	0.00646709035045533\\
143	0.0064669208631949\\
144	0.00646674855755685\\
145	0.00646657338755481\\
146	0.00646639530649136\\
147	0.00646621426694923\\
148	0.00646603022078284\\
149	0.00646584311910966\\
150	0.00646565291230193\\
151	0.00646545954997842\\
152	0.00646526298099641\\
153	0.00646506315344396\\
154	0.00646486001463229\\
155	0.00646465351108868\\
156	0.0064644435885495\\
157	0.00646423019195383\\
158	0.00646401326543751\\
159	0.00646379275232764\\
160	0.00646356859513789\\
161	0.00646334073556431\\
162	0.00646310911448212\\
163	0.00646287367194317\\
164	0.00646263434717474\\
165	0.00646239107857922\\
166	0.00646214380373511\\
167	0.00646189245939954\\
168	0.00646163698151231\\
169	0.00646137730520183\\
170	0.00646111336479263\\
171	0.00646084509381557\\
172	0.00646057242501987\\
173	0.00646029529038804\\
174	0.00646001362115334\\
175	0.00645972734782021\\
176	0.00645943640018777\\
177	0.00645914070737652\\
178	0.00645884019785802\\
179	0.00645853479948794\\
180	0.00645822443954161\\
181	0.00645790904475235\\
182	0.00645758854135083\\
183	0.00645726285510531\\
184	0.00645693191136017\\
185	0.00645659563507051\\
186	0.006456253950829\\
187	0.00645590678287998\\
188	0.00645555405511283\\
189	0.00645519569102329\\
190	0.00645483161362372\\
191	0.00645446174526723\\
192	0.00645408600731315\\
193	0.00645370431946377\\
194	0.0064533165983436\\
195	0.0064529227541831\\
196	0.00645252268249495\\
197	0.00645211624207562\\
198	0.00645170319487064\\
199	0.00645128303787878\\
200	0.00645085563272166\\
201	0.0064504208551466\\
202	0.00644997857879551\\
203	0.00644952867516996\\
204	0.00644907101359553\\
205	0.00644860546118575\\
206	0.00644813188280542\\
207	0.00644765014103328\\
208	0.00644716009612396\\
209	0.00644666160596958\\
210	0.00644615452606039\\
211	0.00644563870944506\\
212	0.00644511400669005\\
213	0.0064445802658385\\
214	0.00644403733236834\\
215	0.00644348504914973\\
216	0.00644292325640165\\
217	0.00644235179164807\\
218	0.00644177048967304\\
219	0.00644117918247519\\
220	0.0064405776992215\\
221	0.00643996586620005\\
222	0.00643934350677232\\
223	0.0064387104413243\\
224	0.00643806648721708\\
225	0.00643741145873626\\
226	0.00643674516704073\\
227	0.00643606742011057\\
228	0.00643537802269375\\
229	0.00643467677625234\\
230	0.00643396347890729\\
231	0.00643323792538263\\
232	0.00643249990694855\\
233	0.00643174921136343\\
234	0.00643098562281489\\
235	0.00643020892185986\\
236	0.00642941888536348\\
237	0.00642861528643711\\
238	0.00642779789437499\\
239	0.00642696647459007\\
240	0.00642612078854862\\
241	0.00642526059370366\\
242	0.00642438564342734\\
243	0.00642349568694218\\
244	0.00642259046925103\\
245	0.00642166973106612\\
246	0.00642073320873672\\
247	0.00641978063417571\\
248	0.00641881173478507\\
249	0.00641782623338007\\
250	0.00641682384811246\\
251	0.00641580429239246\\
252	0.00641476727480949\\
253	0.00641371249905202\\
254	0.00641263966382602\\
255	0.00641154846277254\\
256	0.00641043858438417\\
257	0.00640930971192017\\
258	0.00640816152332094\\
259	0.00640699369112094\\
260	0.00640580588236107\\
261	0.00640459775849952\\
262	0.00640336897532169\\
263	0.00640211918284917\\
264	0.00640084802524742\\
265	0.00639955514073233\\
266	0.00639824016147552\\
267	0.00639690271350842\\
268	0.00639554241662507\\
269	0.00639415888428342\\
270	0.00639275172350507\\
271	0.00639132053477362\\
272	0.00638986491193117\\
273	0.00638838444207315\\
274	0.00638687870544136\\
275	0.00638534727531509\\
276	0.00638378971790053\\
277	0.00638220559221794\\
278	0.00638059444998737\\
279	0.00637895583551238\\
280	0.00637728928556208\\
281	0.00637559432925129\\
282	0.00637387048791925\\
283	0.00637211727500619\\
284	0.00637033419592853\\
285	0.00636852074795187\\
286	0.00636667642006241\\
287	0.00636480069283646\\
288	0.00636289303830807\\
289	0.0063609529198348\\
290	0.0063589797919614\\
291	0.00635697310028176\\
292	0.00635493228129887\\
293	0.00635285676228271\\
294	0.00635074596112604\\
295	0.00634859928619845\\
296	0.00634641613619815\\
297	0.00634419590000159\\
298	0.00634193795651116\\
299	0.00633964167450065\\
300	0.00633730641245856\\
301	0.00633493151842919\\
302	0.00633251632985156\\
303	0.00633006017339608\\
304	0.00632756236479882\\
305	0.00632502220869358\\
306	0.00632243899844158\\
307	0.00631981201595874\\
308	0.00631714053154054\\
309	0.00631442380368444\\
310	0.00631166107890984\\
311	0.00630885159157541\\
312	0.00630599456369383\\
313	0.00630308920474416\\
314	0.00630013471148108\\
315	0.00629713026774185\\
316	0.00629407504425036\\
317	0.00629096819841817\\
318	0.00628780887414304\\
319	0.00628459620160424\\
320	0.006281329297055\\
321	0.00627800726261194\\
322	0.00627462918604145\\
323	0.00627119414054285\\
324	0.00626770118452838\\
325	0.00626414936139994\\
326	0.00626053769932257\\
327	0.00625686521099455\\
328	0.00625313089341405\\
329	0.00624933372764253\\
330	0.00624547267856455\\
331	0.00624154669464419\\
332	0.00623755470767811\\
333	0.00623349563254495\\
334	0.00622936836695184\\
335	0.00622517179117711\\
336	0.00622090476781047\\
337	0.0062165661414898\\
338	0.00621215473863561\\
339	0.00620766936718299\\
340	0.00620310881631192\\
341	0.00619847185617595\\
342	0.00619375723763056\\
343	0.00618896369196163\\
344	0.00618408993061516\\
345	0.00617913464492978\\
346	0.00617409650587355\\
347	0.00616897416378686\\
348	0.00616376624813409\\
349	0.00615847136726692\\
350	0.00615308810820268\\
351	0.00614761503642237\\
352	0.00614205069569307\\
353	0.00613639360792149\\
354	0.0061306422730459\\
355	0.0061247951689757\\
356	0.00611885075158952\\
357	0.00611280745480499\\
358	0.00610666369073642\\
359	0.00610041784995903\\
360	0.00609406830190338\\
361	0.00608761339540712\\
362	0.00608105145945854\\
363	0.00607438080417158\\
364	0.00606759972204199\\
365	0.00606070648954351\\
366	0.00605369936913638\\
367	0.00604657661177525\\
368	0.00603933646002237\\
369	0.00603197715189596\\
370	0.00602449692561097\\
371	0.00601689402540503\\
372	0.00600916670868642\\
373	0.0060013132547934\\
374	0.00599333197572258\\
375	0.00598522122926663\\
376	0.00597697943510633\\
377	0.00596860509453244\\
378	0.0059600968146372\\
379	0.00595145333802245\\
380	0.0059426735793323\\
381	0.00593375667024947\\
382	0.00592470201501394\\
383	0.00591550935905417\\
384	0.00590617887399451\\
385	0.00589671126314147\\
386	0.00588710789256086\\
387	0.0058773709539363\\
388	0.00586750366609288\\
389	0.00585751052076736\\
390	0.00584739756964675\\
391	0.00583717286767827\\
392	0.00582684699909627\\
393	0.00581643373854259\\
394	0.00580595090300746\\
395	0.00579542146859583\\
396	0.00578487508013719\\
397	0.00577434973701437\\
398	0.00576389390029359\\
399	0.00575356936161976\\
400	0.00574345493461472\\
401	0.00573365119175744\\
402	0.00572428646084609\\
403	0.00571552405381711\\
404	0.00570756948526384\\
405	0.00570020151858036\\
406	0.00569271599057066\\
407	0.00568511155554668\\
408	0.00567738689906436\\
409	0.00566954074321656\\
410	0.00566157185251424\\
411	0.00565347904043678\\
412	0.00564526117674768\\
413	0.00563691719568569\\
414	0.00562844610515465\\
415	0.00561984699703459\\
416	0.00561111905869675\\
417	0.00560226158565433\\
418	0.00559327399484952\\
419	0.00558415583698435\\
420	0.00557490681356992\\
421	0.00556552679656875\\
422	0.00555601585045792\\
423	0.00554637425700223\\
424	0.0055366025430567\\
425	0.00552670151176883\\
426	0.00551667227767325\\
427	0.0055065163065078\\
428	0.00549623546132956\\
429	0.00548583205295491\\
430	0.0054753088941555\\
431	0.00546466935862506\\
432	0.0054539174425402\\
433	0.00544305782176041\\
434	0.00543209591680835\\
435	0.00542103797090364\\
436	0.0054098911331517\\
437	0.00539866354596811\\
438	0.00538736443515098\\
439	0.00537600420007305\\
440	0.00536459450004577\\
441	0.0053531483307078\\
442	0.00534168008187833\\
443	0.00533020556475153\\
444	0.00531874199109124\\
445	0.00530730788008513\\
446	0.00529592285891626\\
447	0.00528460730995834\\
448	0.00527338179949224\\
449	0.0052622661981414\\
450	0.00525127836917647\\
451	0.0052404322533988\\
452	0.0052297351121516\\
453	0.00521918359548272\\
454	0.00520875817253184\\
455	0.00519841518453535\\
456	0.00518807572061038\\
457	0.00517761143626688\\
458	0.0051670056819433\\
459	0.00515625832843616\\
460	0.00514536935435214\\
461	0.00513433882988236\\
462	0.00512316690726056\\
463	0.00511185380605984\\
464	0.00510039979199753\\
465	0.00508880514764454\\
466	0.00507707013315591\\
467	0.00506519493487367\\
468	0.00505317959946346\\
469	0.00504102395116322\\
470	0.00502872748930082\\
471	0.00501628925885373\\
472	0.00500370765351237\\
473	0.0049909793991094\\
474	0.00497809968549204\\
475	0.00496506227722912\\
476	0.00495185923263095\\
477	0.00493848060039775\\
478	0.00492491410648339\\
479	0.00491114485115891\\
480	0.00489715505779419\\
481	0.00488292393090441\\
482	0.00486842771334284\\
483	0.00485364008282515\\
484	0.00483853311333963\\
485	0.00482307919412716\\
486	0.00480725458589019\\
487	0.00479104299851726\\
488	0.00477444558520379\\
489	0.00475747358007265\\
490	0.00474011146481785\\
491	0.00472234262990966\\
492	0.00470414929833351\\
493	0.00468551243506386\\
494	0.00466641161320032\\
495	0.00464682485774645\\
496	0.00462672867982591\\
497	0.00460609842400275\\
498	0.00458490755158626\\
499	0.00456312446528458\\
500	0.00454069397053237\\
501	0.0045175760745551\\
502	0.00449372477696406\\
503	0.00446905426462991\\
504	0.00444351900407773\\
505	0.00441707207804545\\
506	0.00438966452483288\\
507	0.00436124514501922\\
508	0.00433175993730267\\
509	0.00430115198250383\\
510	0.00426936140119424\\
511	0.0042363253111851\\
512	0.00420197783560677\\
513	0.00416625008663321\\
514	0.00412906987623692\\
515	0.00409036252097511\\
516	0.00405005428318096\\
517	0.00400807037188265\\
518	0.00396433572973611\\
519	0.00391877605958476\\
520	0.00387131917511654\\
521	0.00382189674995734\\
522	0.00377044651661206\\
523	0.00371691446152693\\
524	0.00366125824878971\\
525	0.00360345349321673\\
526	0.00354350693164753\\
527	0.0034814518998612\\
528	0.00341735621285175\\
529	0.00335133319318523\\
530	0.00328355130589948\\
531	0.00321424345000754\\
532	0.00314370305522421\\
533	0.00307227580608966\\
534	0.00300050952196754\\
535	0.00292929211302619\\
536	0.00286178178697941\\
537	0.00279960223019647\\
538	0.00274303581167043\\
539	0.00269196434323486\\
540	0.00264490546750042\\
541	0.00259895181245598\\
542	0.00255370632164094\\
543	0.0025086849952371\\
544	0.00246359209360616\\
545	0.0024183959307389\\
546	0.00237300335466916\\
547	0.00232734162646493\\
548	0.00228137378200997\\
549	0.00223471812268854\\
550	0.00218726473464073\\
551	0.00213896836534945\\
552	0.00208977713704287\\
553	0.00203960532090274\\
554	0.00198836693274954\\
555	0.00193738925710515\\
556	0.00188742405447135\\
557	0.00183875487424147\\
558	0.00179049315303823\\
559	0.00174185184830918\\
560	0.00169278433015013\\
561	0.00164324280767594\\
562	0.0015932603719237\\
563	0.00154287867702145\\
564	0.00149215207571757\\
565	0.00144111678732867\\
566	0.00139123663602677\\
567	0.00134300099050608\\
568	0.00129517846879981\\
569	0.00124726049348067\\
570	0.00119922945498401\\
571	0.00115112545879614\\
572	0.00110299534370064\\
573	0.00105488697841206\\
574	0.00100684692524753\\
575	0.000959381521014027\\
576	0.000912346458482514\\
577	0.000865353251381383\\
578	0.000818439752416602\\
579	0.000771656841517049\\
580	0.000725057672851425\\
581	0.000678696996801308\\
582	0.000632630493807236\\
583	0.000586913900878002\\
584	0.000541601860815896\\
585	0.000496746426883291\\
586	0.000452395149078408\\
587	0.000408588670370421\\
588	0.000365357795666472\\
589	0.00032272012337526\\
590	0.000280676711614416\\
591	0.000239312817487404\\
592	0.000198788644961842\\
593	0.000159293651685588\\
594	0.000121079888727806\\
595	8.45520570083917e-05\\
596	5.05092148680373e-05\\
597	2.07908715710836e-05\\
598	0\\
599	0\\
600	0\\
};
\addplot [color=blue!25!mycolor7,solid,forget plot]
  table[row sep=crcr]{%
1	0.00682152599929207\\
2	0.00682151808048622\\
3	0.00682151002797638\\
4	0.00682150183951254\\
5	0.006821493512807\\
6	0.00682148504553382\\
7	0.00682147643532816\\
8	0.00682146767978563\\
9	0.00682145877646161\\
10	0.00682144972287067\\
11	0.00682144051648577\\
12	0.00682143115473767\\
13	0.00682142163501422\\
14	0.00682141195465958\\
15	0.00682140211097362\\
16	0.00682139210121112\\
17	0.00682138192258097\\
18	0.00682137157224552\\
19	0.00682136104731981\\
20	0.00682135034487067\\
21	0.00682133946191601\\
22	0.00682132839542409\\
23	0.00682131714231252\\
24	0.00682130569944763\\
25	0.00682129406364346\\
26	0.00682128223166108\\
27	0.00682127020020752\\
28	0.00682125796593495\\
29	0.00682124552543993\\
30	0.00682123287526229\\
31	0.00682122001188433\\
32	0.00682120693172984\\
33	0.0068211936311631\\
34	0.00682118010648805\\
35	0.00682116635394711\\
36	0.00682115236972032\\
37	0.00682113814992434\\
38	0.00682112369061126\\
39	0.00682110898776762\\
40	0.00682109403731346\\
41	0.00682107883510114\\
42	0.00682106337691419\\
43	0.00682104765846632\\
44	0.00682103167540016\\
45	0.00682101542328615\\
46	0.00682099889762152\\
47	0.00682098209382884\\
48	0.00682096500725499\\
49	0.00682094763316988\\
50	0.00682092996676526\\
51	0.00682091200315343\\
52	0.00682089373736587\\
53	0.00682087516435213\\
54	0.00682085627897833\\
55	0.00682083707602595\\
56	0.00682081755019037\\
57	0.00682079769607956\\
58	0.00682077750821269\\
59	0.00682075698101862\\
60	0.00682073610883456\\
61	0.00682071488590459\\
62	0.0068206933063781\\
63	0.00682067136430838\\
64	0.00682064905365105\\
65	0.00682062636826243\\
66	0.00682060330189814\\
67	0.00682057984821136\\
68	0.00682055600075125\\
69	0.00682053175296142\\
70	0.00682050709817799\\
71	0.00682048202962821\\
72	0.00682045654042859\\
73	0.00682043062358315\\
74	0.00682040427198172\\
75	0.00682037747839815\\
76	0.00682035023548839\\
77	0.00682032253578886\\
78	0.00682029437171437\\
79	0.00682026573555639\\
80	0.006820236619481\\
81	0.00682020701552719\\
82	0.00682017691560461\\
83	0.00682014631149174\\
84	0.0068201151948338\\
85	0.00682008355714087\\
86	0.00682005138978555\\
87	0.00682001868400113\\
88	0.00681998543087931\\
89	0.00681995162136805\\
90	0.00681991724626948\\
91	0.00681988229623762\\
92	0.00681984676177617\\
93	0.00681981063323628\\
94	0.00681977390081422\\
95	0.00681973655454912\\
96	0.0068196985843205\\
97	0.00681965997984607\\
98	0.00681962073067918\\
99	0.00681958082620655\\
100	0.0068195402556457\\
101	0.00681949900804251\\
102	0.00681945707226868\\
103	0.00681941443701926\\
104	0.00681937109081009\\
105	0.0068193270219751\\
106	0.00681928221866379\\
107	0.00681923666883861\\
108	0.00681919036027223\\
109	0.0068191432805449\\
110	0.00681909541704164\\
111	0.00681904675694962\\
112	0.00681899728725532\\
113	0.00681894699474166\\
114	0.00681889586598545\\
115	0.00681884388735419\\
116	0.00681879104500348\\
117	0.00681873732487395\\
118	0.00681868271268856\\
119	0.00681862719394947\\
120	0.00681857075393517\\
121	0.00681851337769744\\
122	0.00681845505005851\\
123	0.0068183957556079\\
124	0.00681833547869944\\
125	0.0068182742034482\\
126	0.00681821191372749\\
127	0.00681814859316567\\
128	0.00681808422514313\\
129	0.00681801879278911\\
130	0.00681795227897859\\
131	0.00681788466632918\\
132	0.00681781593719792\\
133	0.00681774607367807\\
134	0.00681767505759591\\
135	0.00681760287050757\\
136	0.00681752949369577\\
137	0.0068174549081666\\
138	0.00681737909464619\\
139	0.00681730203357736\\
140	0.00681722370511652\\
141	0.00681714408913013\\
142	0.00681706316519127\\
143	0.00681698091257642\\
144	0.00681689731026185\\
145	0.00681681233692014\\
146	0.00681672597091659\\
147	0.00681663819030566\\
148	0.00681654897282697\\
149	0.00681645829590186\\
150	0.00681636613662908\\
151	0.00681627247178097\\
152	0.00681617727779895\\
153	0.0068160805307893\\
154	0.00681598220651826\\
155	0.00681588228040731\\
156	0.00681578072752773\\
157	0.00681567752259502\\
158	0.00681557263996284\\
159	0.00681546605361657\\
160	0.00681535773716594\\
161	0.00681524766383737\\
162	0.00681513580646522\\
163	0.00681502213748237\\
164	0.00681490662890945\\
165	0.00681478925234313\\
166	0.00681466997894275\\
167	0.00681454877941545\\
168	0.00681442562399916\\
169	0.0068143004824438\\
170	0.00681417332398954\\
171	0.00681404411734251\\
172	0.00681391283064702\\
173	0.00681377943145406\\
174	0.00681364388668554\\
175	0.00681350616259372\\
176	0.00681336622471505\\
177	0.00681322403781779\\
178	0.00681307956584296\\
179	0.00681293277183706\\
180	0.00681278361787651\\
181	0.00681263206498246\\
182	0.00681247807302526\\
183	0.0068123216006177\\
184	0.00681216260499597\\
185	0.00681200104188856\\
186	0.00681183686537187\\
187	0.00681167002771319\\
188	0.00681150047920165\\
189	0.00681132816796832\\
190	0.00681115303979744\\
191	0.00681097503793067\\
192	0.00681079410286415\\
193	0.00681061017213128\\
194	0.0068104231800471\\
195	0.00681023305736157\\
196	0.00681003973076177\\
197	0.0068098431224084\\
198	0.00680964315129954\\
199	0.00680943974594813\\
200	0.00680923284747863\\
201	0.00680902239660873\\
202	0.00680880833307165\\
203	0.00680859059560025\\
204	0.00680836912191124\\
205	0.00680814384868866\\
206	0.00680791471156739\\
207	0.00680768164511646\\
208	0.00680744458282179\\
209	0.00680720345706904\\
210	0.00680695819912599\\
211	0.00680670873912447\\
212	0.00680645500604247\\
213	0.0068061969276855\\
214	0.00680593443066793\\
215	0.0068056674403939\\
216	0.0068053958810381\\
217	0.00680511967552597\\
218	0.00680483874551379\\
219	0.00680455301136853\\
220	0.00680426239214711\\
221	0.00680396680557559\\
222	0.00680366616802786\\
223	0.0068033603945041\\
224	0.00680304939860875\\
225	0.00680273309252847\\
226	0.00680241138700929\\
227	0.00680208419133379\\
228	0.00680175141329776\\
229	0.00680141295918634\\
230	0.00680106873375031\\
231	0.00680071864018141\\
232	0.0068003625800876\\
233	0.00680000045346802\\
234	0.0067996321586873\\
235	0.00679925759244981\\
236	0.00679887664977322\\
237	0.00679848922396195\\
238	0.00679809520658008\\
239	0.006797694487424\\
240	0.00679728695449443\\
241	0.00679687249396846\\
242	0.00679645099017089\\
243	0.00679602232554523\\
244	0.00679558638062453\\
245	0.00679514303400161\\
246	0.006794692162299\\
247	0.00679423364013861\\
248	0.00679376734011086\\
249	0.00679329313274358\\
250	0.00679281088647048\\
251	0.00679232046759924\\
252	0.00679182174027928\\
253	0.00679131456646923\\
254	0.00679079880590386\\
255	0.00679027431606076\\
256	0.00678974095212658\\
257	0.00678919856696295\\
258	0.00678864701107205\\
259	0.00678808613256156\\
260	0.00678751577710937\\
261	0.00678693578792792\\
262	0.00678634600572787\\
263	0.00678574626868148\\
264	0.0067851364123854\\
265	0.00678451626982306\\
266	0.00678388567132639\\
267	0.00678324444453717\\
268	0.00678259241436765\\
269	0.00678192940296065\\
270	0.00678125522964914\\
271	0.00678056971091495\\
272	0.00677987266034734\\
273	0.00677916388860037\\
274	0.00677844320335\\
275	0.00677771040925065\\
276	0.00677696530789062\\
277	0.00677620769774744\\
278	0.00677543737414224\\
279	0.00677465412919359\\
280	0.00677385775177052\\
281	0.00677304802744507\\
282	0.00677222473844398\\
283	0.00677138766359968\\
284	0.0067705365783005\\
285	0.00676967125444025\\
286	0.00676879146036689\\
287	0.00676789696083024\\
288	0.00676698751692908\\
289	0.00676606288605736\\
290	0.00676512282184922\\
291	0.00676416707412344\\
292	0.00676319538882644\\
293	0.00676220750797481\\
294	0.00676120316959627\\
295	0.00676018210766969\\
296	0.00675914405206401\\
297	0.00675808872847585\\
298	0.00675701585836574\\
299	0.00675592515889312\\
300	0.00675481634284983\\
301	0.00675368911859217\\
302	0.00675254318997126\\
303	0.00675137825626169\\
304	0.00675019401208859\\
305	0.00674899014735249\\
306	0.00674776634715239\\
307	0.00674652229170682\\
308	0.00674525765627203\\
309	0.00674397211105854\\
310	0.00674266532114452\\
311	0.00674133694638666\\
312	0.00673998664132796\\
313	0.00673861405510251\\
314	0.00673721883133672\\
315	0.00673580060804694\\
316	0.00673435901753325\\
317	0.00673289368626896\\
318	0.00673140423478562\\
319	0.0067298902775532\\
320	0.0067283514228549\\
321	0.00672678727265644\\
322	0.006725197422469\\
323	0.00672358146120589\\
324	0.00672193897103177\\
325	0.00672026952720406\\
326	0.00671857269790636\\
327	0.00671684804407221\\
328	0.00671509511919944\\
329	0.00671331346915329\\
330	0.00671150263195827\\
331	0.00670966213757684\\
332	0.00670779150767453\\
333	0.00670589025536974\\
334	0.00670395788496691\\
335	0.00670199389167166\\
336	0.00669999776128625\\
337	0.00669796896988296\\
338	0.00669590698345397\\
339	0.00669381125753483\\
340	0.00669168123679933\\
341	0.00668951635462255\\
342	0.00668731603260905\\
343	0.00668507968008221\\
344	0.0066828066935313\\
345	0.00668049645601082\\
346	0.0066781483364877\\
347	0.00667576168913012\\
348	0.00667333585253143\\
349	0.00667087014886187\\
350	0.00666836388293925\\
351	0.00666581634120933\\
352	0.00666322679062493\\
353	0.00666059447741077\\
354	0.0066579186257001\\
355	0.00665519843602668\\
356	0.00665243308365263\\
357	0.00664962171671082\\
358	0.00664676345413608\\
359	0.00664385738335635\\
360	0.00664090255770907\\
361	0.00663789799354379\\
362	0.006634842666964\\
363	0.00663173551015453\\
364	0.00662857540723024\\
365	0.00662536118953125\\
366	0.00662209163027598\\
367	0.00661876543846735\\
368	0.00661538125192753\\
369	0.00661193762931299\\
370	0.00660843304093303\\
371	0.00660486585816018\\
372	0.00660123434117727\\
373	0.00659753662475583\\
374	0.00659377070169413\\
375	0.00658993440346675\\
376	0.00658602537753923\\
377	0.0065820410606818\\
378	0.00657797864746696\\
379	0.00657383505295076\\
380	0.00656960686830603\\
381	0.00656529030789022\\
382	0.00656088114587788\\
383	0.00655637464017199\\
384	0.006551765440869\\
385	0.00654704748027803\\
386	0.00654221384207036\\
387	0.00653725661091762\\
388	0.00653216671896052\\
389	0.0065269338587564\\
390	0.00652154671652445\\
391	0.00651599095147768\\
392	0.00651024932391503\\
393	0.00650430118488542\\
394	0.00649812145499418\\
395	0.00649167893127105\\
396	0.00648493299469873\\
397	0.00647783680310319\\
398	0.00647033688183818\\
399	0.00646236779727463\\
400	0.00645384908975378\\
401	0.00644468139013384\\
402	0.00643474159157995\\
403	0.00642387719790623\\
404	0.00641190104780991\\
405	0.00639903657200411\\
406	0.00638596078425052\\
407	0.00637267027765513\\
408	0.00635916159825583\\
409	0.00634543124535949\\
410	0.00633147567191987\\
411	0.0063172912849489\\
412	0.00630287444596532\\
413	0.00628822147152613\\
414	0.00627332863398959\\
415	0.0062581921629295\\
416	0.00624280824829613\\
417	0.00622717304810649\\
418	0.00621128270763228\\
419	0.0061951334073186\\
420	0.00617872130624063\\
421	0.00616204253850911\\
422	0.00614509322036607\\
423	0.00612786945910534\\
424	0.0061103673640725\\
425	0.00609258305964295\\
426	0.00607451269871913\\
427	0.00605615247067725\\
428	0.00603749858282038\\
429	0.00601854731459214\\
430	0.00599929511417933\\
431	0.00597973868856861\\
432	0.00595987516664406\\
433	0.00593970243760798\\
434	0.00591921918218595\\
435	0.00589842481180852\\
436	0.0058773196493913\\
437	0.00585590515351625\\
438	0.00583418419697106\\
439	0.00581216141364589\\
440	0.00578984363321967\\
441	0.00576724043148619\\
442	0.00574436482077435\\
443	0.00572123411658393\\
444	0.00569787103223775\\
445	0.00567430506515952\\
446	0.00565057425709745\\
447	0.00562672743507575\\
448	0.00560282707190043\\
449	0.00557895294706662\\
450	0.00555520684395962\\
451	0.00553171859080939\\
452	0.00550865384334736\\
453	0.00548622411073188\\
454	0.00546469960050526\\
455	0.00544442531408781\\
456	0.0054258396714864\\
457	0.00540948983742446\\
458	0.00539323260504457\\
459	0.00537676311249133\\
460	0.00536008469878835\\
461	0.00534320198965536\\
462	0.00532612059192507\\
463	0.00530884720770051\\
464	0.00529138975884361\\
465	0.00527375752160495\\
466	0.00525596127074597\\
467	0.00523801343165346\\
468	0.00521992823775862\\
469	0.00520172188888132\\
470	0.00518341270376497\\
471	0.00516502125709558\\
472	0.0051465704860549\\
473	0.00512808576765448\\
474	0.0051095948999598\\
475	0.00509112794187953\\
476	0.00507271685943956\\
477	0.00505439488960137\\
478	0.00503619549586247\\
479	0.00501815076360548\\
480	0.0050002889997187\\
481	0.00498263121570112\\
482	0.00496518604990143\\
483	0.0049479424870087\\
484	0.00493085946334262\\
485	0.00491385109885935\\
486	0.00489676634709877\\
487	0.00487941046645583\\
488	0.00486177900139084\\
489	0.00484386633386189\\
490	0.00482566519217253\\
491	0.00480716726418363\\
492	0.00478836297873783\\
493	0.00476924124746257\\
494	0.00474978916359166\\
495	0.00472999165560429\\
496	0.00470983108666331\\
497	0.00468928677138484\\
498	0.0046683344028814\\
499	0.00464694545929378\\
500	0.00462508734092348\\
501	0.00460272350259559\\
502	0.00457982238022318\\
503	0.00455638069580096\\
504	0.00453233403861286\\
505	0.00450760479277492\\
506	0.00448213612080047\\
507	0.00445587381878527\\
508	0.00442877360691682\\
509	0.00440078970698855\\
510	0.00437187371198864\\
511	0.00434197524437358\\
512	0.00431104092335962\\
513	0.00427901432034292\\
514	0.00424583630966117\\
515	0.0042114290035927\\
516	0.00417567760871252\\
517	0.00413850706469992\\
518	0.00409983938562309\\
519	0.00405959407839367\\
520	0.00401768872169933\\
521	0.0039740397359373\\
522	0.00392856330490904\\
523	0.00388117643313738\\
524	0.00383179797078545\\
525	0.00378034940225419\\
526	0.0037267559301864\\
527	0.0036709496971085\\
528	0.00361287876217814\\
529	0.00355251646179829\\
530	0.00348985500560026\\
531	0.00342491096846352\\
532	0.00335773241447154\\
533	0.00328840965250917\\
534	0.00321708462727485\\
535	0.00314395991947484\\
536	0.00306929731610841\\
537	0.00299340958820004\\
538	0.0029167674290013\\
539	0.00284013624520599\\
540	0.00276555084315231\\
541	0.0026960778195408\\
542	0.00263212286836089\\
543	0.00257378503688891\\
544	0.00252085379215805\\
545	0.00247009708931051\\
546	0.00242037269702207\\
547	0.00237125755908936\\
548	0.00232224243204798\\
549	0.00227316453779023\\
550	0.00222398371594873\\
551	0.00217460623235058\\
552	0.00212495896211347\\
553	0.00207497488170052\\
554	0.00202421243490735\\
555	0.00197262615657274\\
556	0.0019201375163468\\
557	0.00186664188289284\\
558	0.00181318150960817\\
559	0.00176071773665684\\
560	0.00170951692853472\\
561	0.00165928064744219\\
562	0.00160875603182242\\
563	0.00155792315883206\\
564	0.00150671918854937\\
565	0.00145517384907741\\
566	0.00140334093225183\\
567	0.00135125632661799\\
568	0.00130024976398429\\
569	0.00125090910566476\\
570	0.00120241934549694\\
571	0.00115396094926996\\
572	0.00110548107662132\\
573	0.00105702283547423\\
574	0.00100863570447616\\
575	0.000960371942245695\\
576	0.000912495014597589\\
577	0.000865361483000015\\
578	0.000818442575994003\\
579	0.000771658262692563\\
580	0.000725058416501592\\
581	0.000678697369762489\\
582	0.000632630668187501\\
583	0.000586913975324799\\
584	0.000541601889067355\\
585	0.000496746436058626\\
586	0.000452395151479575\\
587	0.000408588670822602\\
588	0.000365357795712724\\
589	0.000322720123375259\\
590	0.000280676711614414\\
591	0.000239312817487402\\
592	0.000198788644961838\\
593	0.000159293651685583\\
594	0.000121079888727803\\
595	8.45520570083905e-05\\
596	5.05092148680371e-05\\
597	2.07908715710836e-05\\
598	0\\
599	0\\
600	0\\
};
\addplot [color=mycolor9,solid,forget plot]
  table[row sep=crcr]{%
1	0.00701212116330883\\
2	0.0070121158585532\\
3	0.00701211046433695\\
4	0.00701210497915749\\
5	0.00701209940148715\\
6	0.0070120937297728\\
7	0.00701208796243541\\
8	0.00701208209786965\\
9	0.00701207613444345\\
10	0.00701207007049751\\
11	0.00701206390434495\\
12	0.00701205763427078\\
13	0.00701205125853145\\
14	0.00701204477535449\\
15	0.00701203818293786\\
16	0.00701203147944953\\
17	0.00701202466302706\\
18	0.00701201773177704\\
19	0.00701201068377461\\
20	0.00701200351706295\\
21	0.00701199622965271\\
22	0.00701198881952148\\
23	0.00701198128461338\\
24	0.00701197362283837\\
25	0.00701196583207182\\
26	0.0070119579101537\\
27	0.00701194985488829\\
28	0.00701194166404353\\
29	0.00701193333535033\\
30	0.00701192486650201\\
31	0.00701191625515377\\
32	0.007011907498922\\
33	0.00701189859538364\\
34	0.00701188954207558\\
35	0.00701188033649402\\
36	0.00701187097609382\\
37	0.00701186145828771\\
38	0.00701185178044581\\
39	0.00701184193989485\\
40	0.00701183193391744\\
41	0.00701182175975135\\
42	0.00701181141458891\\
43	0.00701180089557619\\
44	0.00701179019981224\\
45	0.00701177932434842\\
46	0.00701176826618755\\
47	0.00701175702228318\\
48	0.00701174558953885\\
49	0.00701173396480721\\
50	0.00701172214488927\\
51	0.00701171012653347\\
52	0.0070116979064351\\
53	0.00701168548123514\\
54	0.00701167284751962\\
55	0.00701166000181868\\
56	0.00701164694060568\\
57	0.00701163366029628\\
58	0.00701162015724765\\
59	0.00701160642775744\\
60	0.00701159246806282\\
61	0.00701157827433967\\
62	0.00701156384270145\\
63	0.00701154916919833\\
64	0.00701153424981618\\
65	0.00701151908047565\\
66	0.00701150365703097\\
67	0.00701148797526907\\
68	0.00701147203090845\\
69	0.00701145581959819\\
70	0.00701143933691689\\
71	0.0070114225783714\\
72	0.00701140553939595\\
73	0.00701138821535091\\
74	0.00701137060152166\\
75	0.0070113526931174\\
76	0.00701133448527019\\
77	0.00701131597303348\\
78	0.00701129715138104\\
79	0.0070112780152059\\
80	0.00701125855931881\\
81	0.00701123877844722\\
82	0.00701121866723396\\
83	0.00701119822023592\\
84	0.00701117743192289\\
85	0.00701115629667599\\
86	0.00701113480878663\\
87	0.00701111296245492\\
88	0.00701109075178841\\
89	0.00701106817080076\\
90	0.00701104521341026\\
91	0.00701102187343835\\
92	0.00701099814460837\\
93	0.00701097402054388\\
94	0.00701094949476734\\
95	0.00701092456069852\\
96	0.00701089921165302\\
97	0.00701087344084075\\
98	0.00701084724136438\\
99	0.00701082060621768\\
100	0.00701079352828397\\
101	0.00701076600033452\\
102	0.00701073801502681\\
103	0.00701070956490305\\
104	0.00701068064238827\\
105	0.00701065123978876\\
106	0.00701062134929025\\
107	0.00701059096295618\\
108	0.00701056007272589\\
109	0.0070105286704128\\
110	0.0070104967477027\\
111	0.00701046429615158\\
112	0.00701043130718397\\
113	0.00701039777209098\\
114	0.00701036368202821\\
115	0.00701032902801386\\
116	0.00701029380092672\\
117	0.00701025799150396\\
118	0.0070102215903391\\
119	0.00701018458787989\\
120	0.00701014697442606\\
121	0.0070101087401271\\
122	0.00701006987497999\\
123	0.00701003036882673\\
124	0.00700999021135221\\
125	0.0070099493920815\\
126	0.00700990790037743\\
127	0.00700986572543801\\
128	0.00700982285629382\\
129	0.00700977928180526\\
130	0.00700973499065963\\
131	0.00700968997136834\\
132	0.00700964421226397\\
133	0.00700959770149705\\
134	0.00700955042703295\\
135	0.00700950237664855\\
136	0.00700945353792868\\
137	0.00700940389826273\\
138	0.00700935344484077\\
139	0.00700930216464976\\
140	0.00700925004446939\\
141	0.00700919707086783\\
142	0.00700914323019746\\
143	0.00700908850858996\\
144	0.00700903289195151\\
145	0.00700897636595761\\
146	0.00700891891604771\\
147	0.00700886052741928\\
148	0.00700880118502204\\
149	0.00700874087355126\\
150	0.00700867957744124\\
151	0.00700861728085784\\
152	0.0070085539676912\\
153	0.00700848962154737\\
154	0.00700842422573996\\
155	0.00700835776328083\\
156	0.0070082902168706\\
157	0.00700822156888834\\
158	0.00700815180138065\\
159	0.00700808089604997\\
160	0.00700800883424242\\
161	0.00700793559693475\\
162	0.00700786116472039\\
163	0.00700778551779494\\
164	0.00700770863594062\\
165	0.00700763049850998\\
166	0.00700755108440853\\
167	0.00700747037207685\\
168	0.00700738833947156\\
169	0.00700730496404541\\
170	0.00700722022272688\\
171	0.00700713409189867\\
172	0.0070070465473757\\
173	0.00700695756438277\\
174	0.0070068671175316\\
175	0.00700677518079791\\
176	0.0070066817274985\\
177	0.00700658673026886\\
178	0.00700649016104153\\
179	0.00700639199102593\\
180	0.00700629219068988\\
181	0.00700619072974375\\
182	0.00700608757712771\\
183	0.00700598270100331\\
184	0.00700587606875001\\
185	0.00700576764696783\\
186	0.00700565740148734\\
187	0.00700554529738814\\
188	0.007005431299027\\
189	0.00700531537007689\\
190	0.00700519747357758\\
191	0.00700507757199851\\
192	0.00700495562731353\\
193	0.00700483160108602\\
194	0.00700470545456245\\
195	0.00700457714877309\\
196	0.00700444664464774\\
197	0.0070043139031749\\
198	0.00700417888562068\\
199	0.00700404155337168\\
200	0.00700390186720293\\
201	0.00700375978723681\\
202	0.00700361527293231\\
203	0.00700346828307445\\
204	0.00700331877576297\\
205	0.00700316670840151\\
206	0.00700301203768601\\
207	0.00700285471959327\\
208	0.00700269470936933\\
209	0.0070025319615174\\
210	0.0070023664297858\\
211	0.00700219806715577\\
212	0.00700202682582877\\
213	0.00700185265721398\\
214	0.00700167551191515\\
215	0.0070014953397177\\
216	0.00700131208957524\\
217	0.00700112570959605\\
218	0.0070009361470293\\
219	0.00700074334825093\\
220	0.00700054725874964\\
221	0.00700034782311216\\
222	0.00700014498500878\\
223	0.0069999386871782\\
224	0.00699972887141246\\
225	0.00699951547854149\\
226	0.00699929844841746\\
227	0.00699907771989871\\
228	0.00699885323083372\\
229	0.00699862491804464\\
230	0.0069983927173104\\
231	0.0069981565633499\\
232	0.00699791638980476\\
233	0.00699767212922168\\
234	0.0069974237130348\\
235	0.00699717107154743\\
236	0.00699691413391379\\
237	0.00699665282812034\\
238	0.0069963870809668\\
239	0.00699611681804689\\
240	0.00699584196372884\\
241	0.0069955624411354\\
242	0.00699527817212372\\
243	0.0069949890772649\\
244	0.00699469507582308\\
245	0.00699439608573423\\
246	0.00699409202358484\\
247	0.00699378280458979\\
248	0.00699346834257033\\
249	0.00699314854993146\\
250	0.00699282333763886\\
251	0.00699249261519567\\
252	0.00699215629061865\\
253	0.006991814270414\\
254	0.00699146645955284\\
255	0.00699111276144603\\
256	0.00699075307791886\\
257	0.00699038730918494\\
258	0.00699001535381969\\
259	0.00698963710873354\\
260	0.00698925246914425\\
261	0.00698886132854892\\
262	0.00698846357869551\\
263	0.00698805910955341\\
264	0.0069876478092838\\
265	0.0069872295642091\\
266	0.00698680425878206\\
267	0.00698637177555383\\
268	0.00698593199514149\\
269	0.00698548479619506\\
270	0.00698503005536338\\
271	0.00698456764725966\\
272	0.00698409744442576\\
273	0.00698361931729606\\
274	0.00698313313416024\\
275	0.00698263876112519\\
276	0.00698213606207617\\
277	0.00698162489863678\\
278	0.00698110513012813\\
279	0.00698057661352673\\
280	0.00698003920342169\\
281	0.00697949275197032\\
282	0.00697893710885287\\
283	0.00697837212122592\\
284	0.00697779763367445\\
285	0.00697721348816269\\
286	0.00697661952398332\\
287	0.00697601557770547\\
288	0.00697540148312096\\
289	0.00697477707118888\\
290	0.00697414216997879\\
291	0.00697349660461161\\
292	0.00697284019719909\\
293	0.00697217276678098\\
294	0.00697149412926034\\
295	0.0069708040973366\\
296	0.00697010248043605\\
297	0.00696938908464039\\
298	0.00696866371261231\\
299	0.00696792616351857\\
300	0.00696717623295016\\
301	0.00696641371283952\\
302	0.00696563839137457\\
303	0.00696485005290942\\
304	0.00696404847787139\\
305	0.0069632334426646\\
306	0.00696240471956947\\
307	0.00696156207663779\\
308	0.006960705277584\\
309	0.00695983408167112\\
310	0.00695894824359223\\
311	0.00695804751334643\\
312	0.00695713163610941\\
313	0.00695620035209805\\
314	0.00695525339642884\\
315	0.00695429049896952\\
316	0.00695331138418405\\
317	0.00695231577096971\\
318	0.00695130337248641\\
319	0.00695027389597768\\
320	0.00694922704258255\\
321	0.00694816250713802\\
322	0.00694707997797123\\
323	0.00694597913668085\\
324	0.00694485965790696\\
325	0.00694372120908883\\
326	0.00694256345020909\\
327	0.00694138603352449\\
328	0.00694018860328137\\
329	0.0069389707954152\\
330	0.0069377322372331\\
331	0.00693647254707809\\
332	0.00693519133397384\\
333	0.0069338881972483\\
334	0.00693256272613508\\
335	0.00693121449935059\\
336	0.00692984308464523\\
337	0.00692844803832713\\
338	0.00692702890475551\\
339	0.00692558521580226\\
340	0.00692411649027866\\
341	0.00692262223332462\\
342	0.00692110193575778\\
343	0.00691955507337899\\
344	0.00691798110623043\\
345	0.00691637947780302\\
346	0.00691474961418832\\
347	0.00691309092317046\\
348	0.00691140279325312\\
349	0.00690968459261531\\
350	0.00690793566799061\\
351	0.00690615534346191\\
352	0.00690434291916476\\
353	0.00690249766989062\\
354	0.00690061884358072\\
355	0.00689870565969982\\
356	0.00689675730747893\\
357	0.00689477294401382\\
358	0.00689275169220508\\
359	0.00689069263852404\\
360	0.00688859483058697\\
361	0.00688645727451776\\
362	0.0068842789320776\\
363	0.00688205871753681\\
364	0.00687979549426217\\
365	0.00687748807098947\\
366	0.00687513519774751\\
367	0.00687273556139631\\
368	0.00687028778073843\\
369	0.006867790401157\\
370	0.00686524188873027\\
371	0.00686264062376682\\
372	0.00685998489370048\\
373	0.00685727288527866\\
374	0.00685450267597319\\
375	0.00685167222453705\\
376	0.00684877936062798\\
377	0.00684582177341692\\
378	0.0068427969991017\\
379	0.00683970240724956\\
380	0.0068365351859054\\
381	0.0068332923254236\\
382	0.00682997060101846\\
383	0.0068265665540891\\
384	0.00682307647247982\\
385	0.00681949637001059\\
386	0.00681582196590169\\
387	0.00681204866509814\\
388	0.00680817154044249\\
389	0.00680418531386084\\
390	0.00680008431010625\\
391	0.00679586243472893\\
392	0.00679151321162811\\
393	0.00678702981014269\\
394	0.00678240509034287\\
395	0.00677763170285273\\
396	0.00677270236258949\\
397	0.00676761019656155\\
398	0.00676234896814914\\
399	0.00675691344267549\\
400	0.00675130005788015\\
401	0.00674550787862871\\
402	0.00673953995275049\\
403	0.00673340522722306\\
404	0.00672712117799294\\
405	0.00672071625804746\\
406	0.00671421349619451\\
407	0.00670761146295683\\
408	0.00670090870046873\\
409	0.00669410372057442\\
410	0.00668719500258211\\
411	0.00668018099058624\\
412	0.00667306009025135\\
413	0.00666583066493078\\
414	0.0066584910309769\\
415	0.00665103945208354\\
416	0.00664347413248083\\
417	0.00663579320869548\\
418	0.00662799473913034\\
419	0.00662007668933002\\
420	0.00661203691626788\\
421	0.00660387315302666\\
422	0.00659558299162864\\
423	0.00658716386409457\\
424	0.00657861302397764\\
425	0.00656992753692295\\
426	0.00656110430995516\\
427	0.00655214026025386\\
428	0.00654303296443749\\
429	0.00653377909572568\\
430	0.00652437365427828\\
431	0.00651481037914996\\
432	0.00650508044441175\\
433	0.00649516897538018\\
434	0.00648506248822431\\
435	0.00647475124985164\\
436	0.0064642242365699\\
437	0.00645346889958397\\
438	0.00644247087869849\\
439	0.00643121364884752\\
440	0.00641967806504588\\
441	0.00640784181270222\\
442	0.00639567874715611\\
443	0.00638315807704225\\
444	0.00637024335089058\\
445	0.00635689119486041\\
446	0.00634304973467485\\
447	0.00632865661554167\\
448	0.00631363650875949\\
449	0.00629789796108431\\
450	0.00628132940073313\\
451	0.00626379406023011\\
452	0.00624512351136802\\
453	0.00622510944104158\\
454	0.00620349327436457\\
455	0.00617995345149095\\
456	0.00615409125076077\\
457	0.00612542075730385\\
458	0.00609601646693627\\
459	0.00606616241725746\\
460	0.00603584092572867\\
461	0.00600502899017431\\
462	0.00597372174869766\\
463	0.00594191505454013\\
464	0.00590960567751194\\
465	0.00587679155557354\\
466	0.0058434721069465\\
467	0.00580964861900482\\
468	0.00577532473337828\\
469	0.00574050705142841\\
470	0.00570520588925315\\
471	0.00566943621445993\\
472	0.0056332188542572\\
473	0.0055965820362056\\
474	0.00555956331100378\\
475	0.00552221199590847\\
476	0.00548459229280957\\
477	0.00544678728978496\\
478	0.00540890413927234\\
479	0.00537108040097315\\
480	0.00533349233870938\\
481	0.00529636564331368\\
482	0.00525998922738064\\
483	0.00522473313857186\\
484	0.00519107140812409\\
485	0.00515961030441914\\
486	0.00513111942658377\\
487	0.00510583050143825\\
488	0.00508025809679198\\
489	0.00505441758935908\\
490	0.00502832694705126\\
491	0.00500200695922378\\
492	0.00497548144923012\\
493	0.00494877744918957\\
494	0.0049219253078496\\
495	0.00489495868990997\\
496	0.00486791440817347\\
497	0.0048408320076661\\
498	0.00481375299114852\\
499	0.00478671953457748\\
500	0.00475977245909883\\
501	0.00473294813251668\\
502	0.00470627391279249\\
503	0.00467976119434337\\
504	0.00465339454787544\\
505	0.00462711890340711\\
506	0.00460081918406089\\
507	0.00457429104784732\\
508	0.00454720443578428\\
509	0.00451949920381706\\
510	0.00449112857439873\\
511	0.00446202316154678\\
512	0.00443214253187289\\
513	0.00440144139335974\\
514	0.00436986892632396\\
515	0.00433738351339268\\
516	0.00430396765942999\\
517	0.00426955412733077\\
518	0.00423406767181877\\
519	0.00419742583052031\\
520	0.00415953875905356\\
521	0.00412030917420201\\
522	0.00407963455826062\\
523	0.00403741149826289\\
524	0.00399354342326266\\
525	0.00394794794373237\\
526	0.00390054155404816\\
527	0.00385122407791419\\
528	0.0037998565790712\\
529	0.00374634929783748\\
530	0.00369061540493826\\
531	0.00363257431252558\\
532	0.00357216559047094\\
533	0.00350934272525963\\
534	0.00344407564145591\\
535	0.00337635592798696\\
536	0.00330620316315285\\
537	0.00323367434338437\\
538	0.00315887169747382\\
539	0.00308195591005351\\
540	0.00300314669175381\\
541	0.00292271428666918\\
542	0.00284102108688311\\
543	0.00275866609982855\\
544	0.00267662723432266\\
545	0.00259850455214081\\
546	0.00252561474725806\\
547	0.00245828060502498\\
548	0.00239657932141828\\
549	0.00233997269311372\\
550	0.00228472809978635\\
551	0.00223052266454992\\
552	0.00217693151171248\\
553	0.00212346363709513\\
554	0.00207004191030077\\
555	0.00201655731486821\\
556	0.00196290412541734\\
557	0.00190901167585809\\
558	0.00185475671516\\
559	0.00179972684135543\\
560	0.00174381965237555\\
561	0.00168751872449357\\
562	0.00163222103135658\\
563	0.00157815070953452\\
564	0.00152562817316902\\
565	0.00147318530183207\\
566	0.00142057936312707\\
567	0.00136778956338732\\
568	0.00131479124306664\\
569	0.0012616240709895\\
570	0.00120923369631625\\
571	0.00115852724686328\\
572	0.00110937816652107\\
573	0.00106041064718068\\
574	0.00101157345479646\\
575	0.000962869504412621\\
576	0.000914347675446341\\
577	0.000866058972727197\\
578	0.000818515558008548\\
579	0.000771676801487267\\
580	0.000725067167815385\\
581	0.000678702038941237\\
582	0.000632633102614749\\
583	0.000586915166466385\\
584	0.000541602424099719\\
585	0.000496746650654584\\
586	0.000452395225472513\\
587	0.000408588691486123\\
588	0.000365357799889246\\
589	0.000322720123837111\\
590	0.000280676711614414\\
591	0.000239312817487403\\
592	0.000198788644961841\\
593	0.000159293651685586\\
594	0.000121079888727805\\
595	8.45520570083913e-05\\
596	5.05092148680373e-05\\
597	2.07908715710836e-05\\
598	0\\
599	0\\
600	0\\
};
\addplot [color=blue!50!mycolor7,solid,forget plot]
  table[row sep=crcr]{%
1	0.0074068299566467\\
2	0.00740682441785876\\
3	0.00740681878582221\\
4	0.00740681305897183\\
5	0.00740680723571628\\
6	0.00740680131443758\\
7	0.00740679529349066\\
8	0.00740678917120309\\
9	0.00740678294587443\\
10	0.00740677661577581\\
11	0.00740677017914957\\
12	0.00740676363420853\\
13	0.0074067569791359\\
14	0.00740675021208434\\
15	0.00740674333117578\\
16	0.00740673633450079\\
17	0.00740672922011808\\
18	0.00740672198605392\\
19	0.00740671463030165\\
20	0.00740670715082115\\
21	0.00740669954553828\\
22	0.00740669181234433\\
23	0.00740668394909534\\
24	0.00740667595361162\\
25	0.00740666782367715\\
26	0.00740665955703898\\
27	0.00740665115140663\\
28	0.00740664260445139\\
29	0.00740663391380575\\
30	0.00740662507706283\\
31	0.00740661609177554\\
32	0.00740660695545612\\
33	0.00740659766557532\\
34	0.00740658821956183\\
35	0.00740657861480151\\
36	0.00740656884863661\\
37	0.00740655891836529\\
38	0.00740654882124067\\
39	0.00740653855447019\\
40	0.0074065281152148\\
41	0.0074065175005883\\
42	0.00740650670765637\\
43	0.007406495733436\\
44	0.00740648457489452\\
45	0.00740647322894892\\
46	0.00740646169246489\\
47	0.00740644996225608\\
48	0.00740643803508317\\
49	0.00740642590765303\\
50	0.00740641357661781\\
51	0.00740640103857418\\
52	0.00740638829006216\\
53	0.00740637532756442\\
54	0.00740636214750523\\
55	0.00740634874624953\\
56	0.00740633512010191\\
57	0.00740632126530572\\
58	0.00740630717804195\\
59	0.00740629285442831\\
60	0.00740627829051807\\
61	0.00740626348229912\\
62	0.00740624842569288\\
63	0.00740623311655316\\
64	0.00740621755066502\\
65	0.00740620172374375\\
66	0.00740618563143371\\
67	0.00740616926930701\\
68	0.00740615263286254\\
69	0.00740613571752465\\
70	0.00740611851864188\\
71	0.00740610103148595\\
72	0.00740608325125019\\
73	0.00740606517304846\\
74	0.00740604679191379\\
75	0.00740602810279702\\
76	0.00740600910056551\\
77	0.00740598978000174\\
78	0.00740597013580194\\
79	0.00740595016257458\\
80	0.00740592985483905\\
81	0.00740590920702419\\
82	0.00740588821346668\\
83	0.00740586686840964\\
84	0.00740584516600103\\
85	0.00740582310029207\\
86	0.00740580066523575\\
87	0.00740577785468505\\
88	0.00740575466239144\\
89	0.00740573108200305\\
90	0.0074057071070631\\
91	0.00740568273100809\\
92	0.00740565794716598\\
93	0.00740563274875449\\
94	0.00740560712887922\\
95	0.00740558108053179\\
96	0.00740555459658784\\
97	0.00740552766980521\\
98	0.00740550029282195\\
99	0.00740547245815421\\
100	0.00740544415819427\\
101	0.00740541538520844\\
102	0.00740538613133496\\
103	0.00740535638858165\\
104	0.00740532614882389\\
105	0.00740529540380227\\
106	0.00740526414512026\\
107	0.00740523236424184\\
108	0.00740520005248921\\
109	0.00740516720104009\\
110	0.00740513380092542\\
111	0.00740509984302671\\
112	0.00740506531807335\\
113	0.00740503021664007\\
114	0.00740499452914394\\
115	0.00740495824584187\\
116	0.00740492135682741\\
117	0.00740488385202798\\
118	0.00740484572120179\\
119	0.00740480695393475\\
120	0.00740476753963729\\
121	0.007404727467541\\
122	0.00740468672669541\\
123	0.00740464530596449\\
124	0.00740460319402304\\
125	0.00740456037935315\\
126	0.0074045168502405\\
127	0.00740447259477036\\
128	0.00740442760082377\\
129	0.00740438185607332\\
130	0.00740433534797922\\
131	0.00740428806378474\\
132	0.00740423999051178\\
133	0.00740419111495641\\
134	0.007404141423684\\
135	0.0074040909030245\\
136	0.00740403953906722\\
137	0.00740398731765573\\
138	0.00740393422438254\\
139	0.00740388024458343\\
140	0.00740382536333186\\
141	0.007403769565433\\
142	0.00740371283541765\\
143	0.00740365515753591\\
144	0.00740359651575079\\
145	0.00740353689373137\\
146	0.00740347627484593\\
147	0.00740341464215491\\
148	0.00740335197840341\\
149	0.00740328826601369\\
150	0.00740322348707738\\
151	0.00740315762334745\\
152	0.0074030906562299\\
153	0.00740302256677547\\
154	0.0074029533356708\\
155	0.00740288294322967\\
156	0.00740281136938397\\
157	0.00740273859367428\\
158	0.00740266459524067\\
159	0.00740258935281312\\
160	0.00740251284470192\\
161	0.00740243504878779\\
162	0.00740235594251239\\
163	0.00740227550286832\\
164	0.00740219370638955\\
165	0.00740211052914166\\
166	0.0074020259467126\\
167	0.00740193993420315\\
168	0.00740185246621825\\
169	0.00740176351685822\\
170	0.0074016730597107\\
171	0.00740158106784308\\
172	0.00740148751379556\\
173	0.00740139236957476\\
174	0.0074012956066485\\
175	0.0074011971959411\\
176	0.00740109710782996\\
177	0.00740099531214307\\
178	0.00740089177815762\\
179	0.00740078647459987\\
180	0.00740067936964604\\
181	0.00740057043092449\\
182	0.00740045962551893\\
183	0.00740034691997241\\
184	0.00740023228029238\\
185	0.00740011567195596\\
186	0.00739999705991545\\
187	0.0073998764086036\\
188	0.00739975368193796\\
189	0.00739962884332372\\
190	0.00739950185565448\\
191	0.00739937268130986\\
192	0.00739924128214912\\
193	0.00739910761950039\\
194	0.00739897165414423\\
195	0.0073988333462918\\
196	0.00739869265555768\\
197	0.00739854954092654\\
198	0.00739840396070977\\
199	0.00739825587250141\\
200	0.00739810523316437\\
201	0.00739795199881759\\
202	0.00739779612482297\\
203	0.0073976375657721\\
204	0.00739747627547275\\
205	0.00739731220693504\\
206	0.0073971453123574\\
207	0.00739697554311232\\
208	0.00739680284973175\\
209	0.00739662718189224\\
210	0.00739644848839992\\
211	0.007396266717175\\
212	0.00739608181523616\\
213	0.00739589372868453\\
214	0.0073957024026875\\
215	0.00739550778146202\\
216	0.00739530980825783\\
217	0.00739510842534015\\
218	0.00739490357397224\\
219	0.00739469519439751\\
220	0.00739448322582124\\
221	0.00739426760639218\\
222	0.00739404827318354\\
223	0.00739382516217373\\
224	0.00739359820822686\\
225	0.00739336734507254\\
226	0.00739313250528566\\
227	0.00739289362026561\\
228	0.00739265062021495\\
229	0.00739240343411795\\
230	0.00739215198971861\\
231	0.00739189621349811\\
232	0.007391636030652\\
233	0.00739137136506691\\
234	0.00739110213929665\\
235	0.00739082827453812\\
236	0.00739054969060642\\
237	0.00739026630590975\\
238	0.00738997803742359\\
239	0.00738968480066449\\
240	0.0073893865096632\\
241	0.00738908307693744\\
242	0.007388774413464\\
243	0.00738846042865016\\
244	0.00738814103030472\\
245	0.00738781612460839\\
246	0.00738748561608332\\
247	0.00738714940756228\\
248	0.00738680740015708\\
249	0.00738645949322613\\
250	0.00738610558434173\\
251	0.0073857455692561\\
252	0.00738537934186714\\
253	0.00738500679418307\\
254	0.00738462781628656\\
255	0.0073842422962979\\
256	0.0073838501203373\\
257	0.00738345117248648\\
258	0.00738304533474929\\
259	0.00738263248701129\\
260	0.00738221250699872\\
261	0.0073817852702362\\
262	0.00738135065000343\\
263	0.00738090851729121\\
264	0.00738045874075587\\
265	0.00738000118667315\\
266	0.0073795357188904\\
267	0.00737906219877818\\
268	0.00737858048518024\\
269	0.00737809043436237\\
270	0.00737759189996016\\
271	0.00737708473292503\\
272	0.00737656878146932\\
273	0.0073760438910098\\
274	0.00737550990410958\\
275	0.00737496666041871\\
276	0.00737441399661314\\
277	0.00737385174633213\\
278	0.00737327974011379\\
279	0.00737269780532916\\
280	0.00737210576611426\\
281	0.00737150344330047\\
282	0.00737089065434291\\
283	0.00737026721324675\\
284	0.00736963293049165\\
285	0.00736898761295384\\
286	0.00736833106382621\\
287	0.00736766308253593\\
288	0.00736698346465981\\
289	0.0073662920018371\\
290	0.00736558848167974\\
291	0.0073648726876801\\
292	0.00736414439911577\\
293	0.00736340339095163\\
294	0.00736264943373898\\
295	0.00736188229351134\\
296	0.00736110173167753\\
297	0.00736030750491094\\
298	0.00735949936503572\\
299	0.00735867705890934\\
300	0.00735784032830132\\
301	0.00735698890976823\\
302	0.00735612253452471\\
303	0.00735524092831032\\
304	0.00735434381125222\\
305	0.00735343089772323\\
306	0.00735250189619534\\
307	0.0073515565090885\\
308	0.00735059443261443\\
309	0.00734961535661524\\
310	0.00734861896439659\\
311	0.00734760493255559\\
312	0.00734657293080274\\
313	0.00734552262177798\\
314	0.00734445366086052\\
315	0.00734336569597255\\
316	0.00734225836737576\\
317	0.00734113130746157\\
318	0.00733998414053407\\
319	0.00733881648258547\\
320	0.00733762794106414\\
321	0.00733641811463464\\
322	0.00733518659292982\\
323	0.0073339329562943\\
324	0.00733265677551952\\
325	0.00733135761156964\\
326	0.00733003501529837\\
327	0.00732868852715621\\
328	0.00732731767688786\\
329	0.0073259219832197\\
330	0.00732450095353673\\
331	0.00732305408354902\\
332	0.00732158085694693\\
333	0.00732008074504546\\
334	0.00731855320641677\\
335	0.00731699768651116\\
336	0.00731541361726578\\
337	0.00731380041670108\\
338	0.00731215748850482\\
339	0.00731048422160312\\
340	0.00730877998971845\\
341	0.00730704415091459\\
342	0.00730527604712798\\
343	0.00730347500368556\\
344	0.00730164032880894\\
345	0.00729977131310471\\
346	0.00729786722904094\\
347	0.00729592733040972\\
348	0.00729395085177594\\
349	0.00729193700791217\\
350	0.00728988499322004\\
351	0.00728779398113814\\
352	0.00728566312353681\\
353	0.00728349155010023\\
354	0.00728127836769626\\
355	0.00727902265973467\\
356	0.00727672348551477\\
357	0.00727437987956289\\
358	0.00727199085096141\\
359	0.00726955538267042\\
360	0.00726707243084388\\
361	0.00726454092414224\\
362	0.007261959763044\\
363	0.00725932781915937\\
364	0.00725664393454936\\
365	0.00725390692105469\\
366	0.00725111555964022\\
367	0.00724826859976054\\
368	0.00724536475875515\\
369	0.00724240272128197\\
370	0.00723938113880114\\
371	0.00723629862912245\\
372	0.00723315377603356\\
373	0.0072299451290294\\
374	0.00722667120316761\\
375	0.00722333047908035\\
376	0.00721992140317944\\
377	0.00721644238809939\\
378	0.0072128918134324\\
379	0.00720926802682098\\
380	0.00720556934548708\\
381	0.0072017940582922\\
382	0.0071979404284423\\
383	0.00719400669697306\\
384	0.00718999108717504\\
385	0.00718589181014539\\
386	0.00718170707167325\\
387	0.00717743508066228\\
388	0.00717307405923942\\
389	0.00716862225468668\\
390	0.00716407795460443\\
391	0.00715943950628516\\
392	0.00715470533720593\\
393	0.00714987397793557\\
394	0.00714494408796095\\
395	0.00713991448424189\\
396	0.0071347841672599\\
397	0.00712955233752372\\
398	0.00712421840955929\\
399	0.00711878202108357\\
400	0.00711324302348959\\
401	0.00710760144058647\\
402	0.00710185737522207\\
403	0.00709601083121614\\
404	0.0070900613992831\\
405	0.00708400778027166\\
406	0.00707784782427909\\
407	0.00707157929726241\\
408	0.00706519989025344\\
409	0.00705870721664789\\
410	0.00705209880977829\\
411	0.00704537212084409\\
412	0.00703852451725753\\
413	0.00703155328144083\\
414	0.00702445561010178\\
415	0.00701722861410645\\
416	0.00700986931951129\\
417	0.00700237467173069\\
418	0.00699474154818311\\
419	0.00698696679845117\\
420	0.00697904714820445\\
421	0.00697097917369077\\
422	0.00696275928721133\\
423	0.00695438372097599\\
424	0.00694584850944636\\
425	0.0069371494704971\\
426	0.0069282821853252\\
427	0.00691924197185974\\
428	0.00691002381607948\\
429	0.00690062229619514\\
430	0.00689103157279853\\
431	0.0068812453785263\\
432	0.00687125700243555\\
433	0.00686105947035154\\
434	0.00685064572278754\\
435	0.00684000842741724\\
436	0.00682913979427269\\
437	0.00681803157463512\\
438	0.00680667509094791\\
439	0.00679506134536192\\
440	0.00678318167415435\\
441	0.00677102761914462\\
442	0.00675859057904313\\
443	0.00674586191458853\\
444	0.00673283310171506\\
445	0.00671949595028459\\
446	0.00670584291204887\\
447	0.00669186750977668\\
448	0.00667756493066131\\
449	0.00666293284232163\\
450	0.00664797251059849\\
451	0.00663269032774426\\
452	0.0066170999029803\\
453	0.00660122493712663\\
454	0.00658510323055887\\
455	0.00656879244459335\\
456	0.00655237888856633\\
457	0.00653598164334163\\
458	0.00651975971387527\\
459	0.00650381850604284\\
460	0.00648827189993001\\
461	0.00647325382423801\\
462	0.00645794555767362\\
463	0.00644233367387723\\
464	0.00642640309573541\\
465	0.00641013679445086\\
466	0.00639351544169336\\
467	0.00637651698950613\\
468	0.00635911616107912\\
469	0.00634128383119084\\
470	0.00632298626971447\\
471	0.0063041842149933\\
472	0.00628483173380949\\
473	0.00626487481204921\\
474	0.00624424960687539\\
475	0.00622288027233064\\
476	0.00620067624437277\\
477	0.00617752883802573\\
478	0.00615330696433514\\
479	0.00612785173133297\\
480	0.00610096961666709\\
481	0.00607242379360694\\
482	0.00604192306370216\\
483	0.00600910805190212\\
484	0.00597353391254416\\
485	0.00593464965252985\\
486	0.00589177709624734\\
487	0.0058447826614934\\
488	0.00579706219262404\\
489	0.00574862086705563\\
490	0.00569946786929883\\
491	0.00564961737971868\\
492	0.00559908980549789\\
493	0.00554791331535837\\
494	0.00549612575580097\\
495	0.00544377704720198\\
496	0.00539093218421757\\
497	0.00533767499801058\\
498	0.00528411287952206\\
499	0.00523038271546018\\
500	0.00517665835607147\\
501	0.0051231600238064\\
502	0.0050701661987078\\
503	0.00501802872194789\\
504	0.00496721929228177\\
505	0.00491837790758348\\
506	0.0048723283960852\\
507	0.0048301257606557\\
508	0.00479309668846519\\
509	0.00475625955239109\\
510	0.00471898918517486\\
511	0.00468131797440956\\
512	0.0046432840722822\\
513	0.00460493031991499\\
514	0.004566303879972\\
515	0.00452745570207619\\
516	0.00448844008980965\\
517	0.0044493097997433\\
518	0.00441008687171728\\
519	0.00437074144775284\\
520	0.00433122959196758\\
521	0.00429149784774394\\
522	0.00425142897253515\\
523	0.00421081080778946\\
524	0.00416929313380805\\
525	0.00412643381319458\\
526	0.00408215596336811\\
527	0.00403639275687767\\
528	0.00398910860078957\\
529	0.00394021841245417\\
530	0.00388963168405301\\
531	0.00383725090119883\\
532	0.00378297102842801\\
533	0.00372667994895513\\
534	0.00366826052558962\\
535	0.00360759499538104\\
536	0.00354458408089595\\
537	0.00347914048606964\\
538	0.00341120344786853\\
539	0.00334068582296219\\
540	0.00326757016964767\\
541	0.00319187772760867\\
542	0.0031136686903834\\
543	0.00303305075136004\\
544	0.00295018712023015\\
545	0.0028652920901408\\
546	0.00277862134303654\\
547	0.00269059274570993\\
548	0.00260190152765235\\
549	0.00251390181794879\\
550	0.00243061929734025\\
551	0.00235257346475307\\
552	0.00228003875241692\\
553	0.00221317257157155\\
554	0.00215112287025263\\
555	0.00209052503954314\\
556	0.00203108611377265\\
557	0.00197240575620483\\
558	0.00191405466630919\\
559	0.0018558391868882\\
560	0.001797667721791\\
561	0.00173945129258813\\
562	0.00168110421909231\\
563	0.00162243118677364\\
564	0.00156298187657357\\
565	0.00150437000468756\\
566	0.00144703158843678\\
567	0.00139122030448227\\
568	0.00133673591860897\\
569	0.00128228716789082\\
570	0.00122788747298109\\
571	0.00117350595556176\\
572	0.00111939032316793\\
573	0.00106700721614886\\
574	0.00101651208218248\\
575	0.000967001075013799\\
576	0.000917838202573339\\
577	0.000868968269894002\\
578	0.000820423355365856\\
579	0.000772347639745716\\
580	0.000725198081162504\\
581	0.000678755482625295\\
582	0.000632661594736207\\
583	0.000586930577994372\\
584	0.000541610321954773\\
585	0.00049675039409852\\
586	0.000452396819030694\\
587	0.000408589277753164\\
588	0.000365357975688249\\
589	0.000322720162279343\\
590	0.000280676716267942\\
591	0.000239312817487404\\
592	0.000198788644961841\\
593	0.000159293651685586\\
594	0.000121079888727805\\
595	8.45520570083912e-05\\
596	5.05092148680374e-05\\
597	2.07908715710836e-05\\
598	0\\
599	0\\
600	0\\
};
\addplot [color=blue!40!mycolor9,solid,forget plot]
  table[row sep=crcr]{%
1	0.0096074075755447\\
2	0.00960739244151761\\
3	0.00960737705294697\\
4	0.00960736140555528\\
5	0.00960734549499303\\
6	0.0096073293168376\\
7	0.00960731286659196\\
8	0.00960729613968336\\
9	0.00960727913146209\\
10	0.00960726183720023\\
11	0.00960724425209022\\
12	0.00960722637124369\\
13	0.00960720818968978\\
14	0.00960718970237404\\
15	0.0096071709041569\\
16	0.00960715178981218\\
17	0.00960713235402568\\
18	0.0096071125913937\\
19	0.00960709249642146\\
20	0.00960707206352154\\
21	0.00960705128701242\\
22	0.00960703016111673\\
23	0.00960700867995977\\
24	0.00960698683756772\\
25	0.00960696462786599\\
26	0.00960694204467757\\
27	0.00960691908172121\\
28	0.00960689573260967\\
29	0.00960687199084789\\
30	0.0096068478498312\\
31	0.00960682330284341\\
32	0.00960679834305482\\
33	0.00960677296352051\\
34	0.00960674715717811\\
35	0.00960672091684596\\
36	0.00960669423522108\\
37	0.00960666710487688\\
38	0.0096066395182613\\
39	0.00960661146769448\\
40	0.0096065829453667\\
41	0.00960655394333601\\
42	0.00960652445352608\\
43	0.00960649446772381\\
44	0.00960646397757705\\
45	0.0096064329745921\\
46	0.00960640145013138\\
47	0.00960636939541089\\
48	0.0096063368014977\\
49	0.00960630365930738\\
50	0.00960626995960139\\
51	0.00960623569298438\\
52	0.00960620084990154\\
53	0.00960616542063577\\
54	0.00960612939530495\\
55	0.00960609276385898\\
56	0.00960605551607695\\
57	0.00960601764156417\\
58	0.00960597912974904\\
59	0.0096059399698801\\
60	0.00960590015102288\\
61	0.00960585966205662\\
62	0.00960581849167114\\
63	0.00960577662836343\\
64	0.00960573406043442\\
65	0.00960569077598536\\
66	0.00960564676291445\\
67	0.00960560200891331\\
68	0.00960555650146327\\
69	0.00960551022783172\\
70	0.00960546317506837\\
71	0.00960541533000136\\
72	0.00960536667923342\\
73	0.00960531720913784\\
74	0.00960526690585453\\
75	0.00960521575528577\\
76	0.00960516374309205\\
77	0.00960511085468785\\
78	0.00960505707523722\\
79	0.00960500238964931\\
80	0.00960494678257397\\
81	0.00960489023839697\\
82	0.00960483274123542\\
83	0.00960477427493294\\
84	0.00960471482305481\\
85	0.00960465436888295\\
86	0.00960459289541087\\
87	0.00960453038533852\\
88	0.00960446682106704\\
89	0.00960440218469337\\
90	0.00960433645800474\\
91	0.0096042696224732\\
92	0.00960420165924989\\
93	0.00960413254915918\\
94	0.00960406227269294\\
95	0.0096039908100043\\
96	0.00960391814090174\\
97	0.00960384424484273\\
98	0.00960376910092729\\
99	0.00960369268789166\\
100	0.00960361498410159\\
101	0.00960353596754553\\
102	0.00960345561582784\\
103	0.00960337390616179\\
104	0.00960329081536231\\
105	0.00960320631983879\\
106	0.00960312039558761\\
107	0.00960303301818457\\
108	0.00960294416277709\\
109	0.00960285380407652\\
110	0.00960276191634992\\
111	0.00960266847341198\\
112	0.00960257344861666\\
113	0.00960247681484864\\
114	0.00960237854451474\\
115	0.00960227860953499\\
116	0.00960217698133369\\
117	0.00960207363083024\\
118	0.00960196852842973\\
119	0.00960186164401339\\
120	0.009601752946929\\
121	0.00960164240598088\\
122	0.00960152998941985\\
123	0.00960141566493296\\
124	0.00960129939963303\\
125	0.00960118116004798\\
126	0.00960106091210998\\
127	0.00960093862114444\\
128	0.00960081425185868\\
129	0.00960068776833064\\
130	0.009600559133997\\
131	0.00960042831164155\\
132	0.00960029526338298\\
133	0.00960015995066272\\
134	0.00960002233423241\\
135	0.00959988237414115\\
136	0.00959974002972281\\
137	0.00959959525958278\\
138	0.0095994480215847\\
139	0.00959929827283704\\
140	0.00959914596967927\\
141	0.00959899106766809\\
142	0.00959883352156315\\
143	0.00959867328531301\\
144	0.00959851031204037\\
145	0.00959834455402753\\
146	0.00959817596270155\\
147	0.0095980044886191\\
148	0.00959783008145125\\
149	0.00959765268996817\\
150	0.00959747226202343\\
151	0.00959728874453834\\
152	0.00959710208348609\\
153	0.0095969122238756\\
154	0.00959671910973556\\
155	0.00959652268409789\\
156	0.00959632288898141\\
157	0.00959611966537533\\
158	0.00959591295322249\\
159	0.00959570269140253\\
160	0.00959548881771502\\
161	0.00959527126886251\\
162	0.00959504998043319\\
163	0.00959482488688383\\
164	0.00959459592152232\\
165	0.00959436301649017\\
166	0.00959412610274484\\
167	0.00959388511004209\\
168	0.00959363996691783\\
169	0.00959339060067005\\
170	0.00959313693734044\\
171	0.00959287890169572\\
172	0.00959261641720862\\
173	0.00959234940603876\\
174	0.00959207778901272\\
175	0.00959180148560398\\
176	0.00959152041391221\\
177	0.00959123449064189\\
178	0.00959094363108033\\
179	0.00959064774907484\\
180	0.00959034675700905\\
181	0.0095900405657784\\
182	0.00958972908476431\\
183	0.00958941222180756\\
184	0.00958908988318017\\
185	0.00958876197355619\\
186	0.00958842839598119\\
187	0.00958808905184028\\
188	0.00958774384082494\\
189	0.00958739266089855\\
190	0.00958703540826065\\
191	0.00958667197731019\\
192	0.00958630226060791\\
193	0.00958592614883796\\
194	0.00958554353076913\\
195	0.00958515429321589\\
196	0.00958475832099956\\
197	0.00958435549690961\\
198	0.00958394570166566\\
199	0.00958352881387976\\
200	0.00958310471001807\\
201	0.00958267326436181\\
202	0.00958223434896743\\
203	0.00958178783362597\\
204	0.00958133358582178\\
205	0.00958087147069034\\
206	0.00958040135097536\\
207	0.00957992308698488\\
208	0.00957943653654678\\
209	0.00957894155496329\\
210	0.00957843799496454\\
211	0.00957792570666142\\
212	0.00957740453749741\\
213	0.00957687433219943\\
214	0.00957633493272788\\
215	0.00957578617822562\\
216	0.00957522790496593\\
217	0.00957465994629961\\
218	0.00957408213260086\\
219	0.00957349429121224\\
220	0.0095728962463885\\
221	0.00957228781923943\\
222	0.00957166882767131\\
223	0.00957103908632757\\
224	0.00957039840652806\\
225	0.00956974659620724\\
226	0.00956908345985101\\
227	0.00956840879843246\\
228	0.00956772240934639\\
229	0.00956702408634227\\
230	0.00956631361945617\\
231	0.00956559079494125\\
232	0.00956485539519678\\
233	0.00956410719869591\\
234	0.00956334597991188\\
235	0.00956257150924284\\
236	0.00956178355293522\\
237	0.0095609818730054\\
238	0.00956016622716001\\
239	0.00955933636871454\\
240	0.00955849204651045\\
241	0.00955763300483037\\
242	0.00955675898331186\\
243	0.0095558697168594\\
244	0.00955496493555446\\
245	0.00955404436456391\\
246	0.0095531077240466\\
247	0.00955215472905803\\
248	0.009551185089453\\
249	0.00955019850978655\\
250	0.00954919468921262\\
251	0.00954817332138089\\
252	0.00954713409433151\\
253	0.00954607669038755\\
254	0.00954500078604549\\
255	0.00954390605186338\\
256	0.00954279215234682\\
257	0.00954165874583254\\
258	0.00954050548436971\\
259	0.00953933201359898\\
260	0.00953813797262877\\
261	0.00953692299390935\\
262	0.00953568670310429\\
263	0.00953442871895927\\
264	0.00953314865316835\\
265	0.00953184611023756\\
266	0.00953052068734558\\
267	0.0095291719742018\\
268	0.00952779955290148\\
269	0.00952640299777804\\
270	0.00952498187525226\\
271	0.00952353574367869\\
272	0.0095220641531888\\
273	0.00952056664553099\\
274	0.00951904275390769\\
275	0.00951749200280888\\
276	0.00951591390784251\\
277	0.00951430797556143\\
278	0.00951267370328713\\
279	0.00951101057892972\\
280	0.00950931808080451\\
281	0.00950759567744505\\
282	0.00950584282741231\\
283	0.00950405897910038\\
284	0.00950224357053821\\
285	0.00950039602918757\\
286	0.00949851577173702\\
287	0.00949660220389208\\
288	0.00949465472016117\\
289	0.00949267270363763\\
290	0.00949065552577744\\
291	0.0094886025461728\\
292	0.00948651311232144\\
293	0.00948438655939151\\
294	0.00948222220998218\\
295	0.00948001937387975\\
296	0.00947777734780916\\
297	0.00947549541518103\\
298	0.00947317284583406\\
299	0.00947080889577266\\
300	0.00946840280689992\\
301	0.00946595380674576\\
302	0.00946346110819021\\
303	0.00946092390918188\\
304	0.00945834139245133\\
305	0.00945571272521959\\
306	0.00945303705890157\\
307	0.00945031352880442\\
308	0.00944754125382065\\
309	0.00944471933611625\\
310	0.0094418468608135\\
311	0.00943892289566851\\
312	0.00943594649074358\\
313	0.00943291667807412\\
314	0.0094298324713303\\
315	0.00942669286547325\\
316	0.00942349683640599\\
317	0.00942024334061878\\
318	0.00941693131482899\\
319	0.00941355967561562\\
320	0.00941012731904821\\
321	0.00940663312031021\\
322	0.00940307593331675\\
323	0.00939945459032695\\
324	0.00939576790155048\\
325	0.00939201465474842\\
326	0.00938819361482865\\
327	0.00938430352343525\\
328	0.0093803430985324\\
329	0.00937631103398228\\
330	0.00937220599911714\\
331	0.00936802663830553\\
332	0.00936377157051248\\
333	0.00935943938885354\\
334	0.00935502866014277\\
335	0.00935053792443428\\
336	0.00934596569455752\\
337	0.00934131045564591\\
338	0.00933657066465869\\
339	0.00933174474989592\\
340	0.00932683111050627\\
341	0.00932182811598741\\
342	0.00931673410567867\\
343	0.00931154738824574\\
344	0.00930626624115684\\
345	0.00930088891015026\\
346	0.00929541360869242\\
347	0.00928983851742644\\
348	0.00928416178361004\\
349	0.00927838152054283\\
350	0.00927249580698165\\
351	0.0092665026865438\\
352	0.00926040016709684\\
353	0.00925418622013451\\
354	0.00924785878013752\\
355	0.00924141574391836\\
356	0.00923485496994897\\
357	0.00922817427767003\\
358	0.00922137144678084\\
359	0.00921444421650832\\
360	0.00920739028485377\\
361	0.00920020730781615\\
362	0.00919289289859011\\
363	0.00918544462673757\\
364	0.00917786001733097\\
365	0.00917013655006702\\
366	0.0091622716583488\\
367	0.00915426272833516\\
368	0.00914610709795526\\
369	0.00913780205588702\\
370	0.00912934484049751\\
371	0.00912073263874376\\
372	0.00911196258503221\\
373	0.00910303176003477\\
374	0.00909393718945951\\
375	0.00908467584277375\\
376	0.00907524463187639\\
377	0.00906564040971642\\
378	0.00905585996885271\\
379	0.00904590003994982\\
380	0.00903575729020174\\
381	0.0090254283216738\\
382	0.00901490966954928\\
383	0.00900419780026284\\
384	0.00899328910949718\\
385	0.0089821799200117\\
386	0.00897086647926193\\
387	0.00895934495675723\\
388	0.0089476114410925\\
389	0.00893566193658549\\
390	0.00892349235939025\\
391	0.00891109853290295\\
392	0.00889847618237991\\
393	0.00888562092862454\\
394	0.00887252828055634\\
395	0.00885919362642484\\
396	0.0088456122235615\\
397	0.00883177918680417\\
398	0.00881768947542669\\
399	0.00880333787845024\\
400	0.00878871899876059\\
401	0.00877382723711889\\
402	0.00875865677849773\\
403	0.00874320158609038\\
404	0.00872745541512584\\
405	0.00871141187025567\\
406	0.00869506458123908\\
407	0.00867840695760704\\
408	0.00866143169002759\\
409	0.00864413122629998\\
410	0.00862649776265295\\
411	0.00860852323537416\\
412	0.00859019931259046\\
413	0.00857151738521099\\
414	0.00855246855384727\\
415	0.00853304360289888\\
416	0.0085132329392359\\
417	0.00849302644087801\\
418	0.00847241309164893\\
419	0.00845137969852336\\
420	0.00842991615910179\\
421	0.00840801263579948\\
422	0.00838565906355578\\
423	0.00836284514951068\\
424	0.00833956037385398\\
425	0.00831579399204311\\
426	0.00829153503852154\\
427	0.00826677233202398\\
428	0.00824149448407581\\
429	0.00821568991226852\\
430	0.00818934685571328\\
431	0.00816245339275658\\
432	0.00813499746363264\\
433	0.00810696689224114\\
434	0.00807834940349541\\
435	0.00804913265733362\\
436	0.00801930431272337\\
437	0.00798885212316119\\
438	0.00795776403013172\\
439	0.00792602794353914\\
440	0.00789361383292988\\
441	0.0078604925639844\\
442	0.00782664425161163\\
443	0.00779204802704287\\
444	0.00775668192794685\\
445	0.00772052276554722\\
446	0.00768354596203074\\
447	0.00764572534928105\\
448	0.00760703291693255\\
449	0.00756743849364348\\
450	0.0075269093399834\\
451	0.00748540962396113\\
452	0.00744289974033823\\
453	0.00739933542136051\\
454	0.00735466656663158\\
455	0.0073088356848356\\
456	0.00726177573290427\\
457	0.00721340736359842\\
458	0.00716363577891605\\
459	0.00711235074722052\\
460	0.00705942633796879\\
461	0.00700506357462199\\
462	0.0069897888830494\\
463	0.00697415177545719\\
464	0.00695813781822587\\
465	0.00694173199708212\\
466	0.0069249183178362\\
467	0.00690767973509524\\
468	0.0068899980807137\\
469	0.00687185399387683\\
470	0.00685322685424857\\
471	0.00683409471452101\\
472	0.00681443428282061\\
473	0.00679422095417364\\
474	0.0067734288573931\\
475	0.00675203096749532\\
476	0.00672999931676739\\
477	0.00670730535859358\\
478	0.00668392059335595\\
479	0.00665981715701593\\
480	0.00663496905067333\\
481	0.00660935439457848\\
482	0.00658296001510484\\
483	0.00655577770972172\\
484	0.00652780995818072\\
485	0.00649907640909796\\
486	0.00646961933516901\\
487	0.00643950390821288\\
488	0.00640880524000931\\
489	0.00637750451331813\\
490	0.00634558129860116\\
491	0.00631301334562971\\
492	0.00627977635446164\\
493	0.00624584372821354\\
494	0.00621118631284592\\
495	0.00617577213330074\\
496	0.00613956614119091\\
497	0.00610252999712457\\
498	0.00606462192026863\\
499	0.00602579664634754\\
500	0.00598600553441714\\
501	0.00594519683091755\\
502	0.00590331599908454\\
503	0.0058603058449187\\
504	0.00581513748167444\\
505	0.0057661407166878\\
506	0.00571256209853997\\
507	0.00565347260740609\\
508	0.00558774672291453\\
509	0.00552027502761627\\
510	0.00545171558099382\\
511	0.00538210725278225\\
512	0.00531150662435504\\
513	0.00523999271506372\\
514	0.00516767325099425\\
515	0.00509469267449204\\
516	0.00502124230610663\\
517	0.00494757317034403\\
518	0.004874013063796\\
519	0.00480099012756452\\
520	0.00472906273088725\\
521	0.00465895572787164\\
522	0.00459160707580159\\
523	0.00452822708277632\\
524	0.00447036487618838\\
525	0.0044184375290259\\
526	0.00436587397357472\\
527	0.0043126885610599\\
528	0.00425885476479472\\
529	0.00420427192507372\\
530	0.00414892816415185\\
531	0.00409285714575569\\
532	0.00403608410604617\\
533	0.00397861679527939\\
534	0.00392043183348241\\
535	0.0038614544566571\\
536	0.00380152811035017\\
537	0.00374037288718971\\
538	0.00367753023728315\\
539	0.00361247081341678\\
540	0.00354512799959542\\
541	0.00347543947168619\\
542	0.00340334641225652\\
543	0.0033287955821686\\
544	0.00325174240432155\\
545	0.00317215630959154\\
546	0.00309002880541974\\
547	0.00300538262148774\\
548	0.00291828382082769\\
549	0.00282884679169599\\
550	0.00273717375698181\\
551	0.00264351117793987\\
552	0.0025483095260621\\
553	0.00245231860330577\\
554	0.00235722079604287\\
555	0.00226672926843008\\
556	0.00218134599997664\\
557	0.00210146237082903\\
558	0.00202732470075\\
559	0.00195859464071399\\
560	0.00189147544706247\\
561	0.0018257439348815\\
562	0.00176104830254656\\
563	0.00169700040817372\\
564	0.0016334359172862\\
565	0.00157013355595559\\
566	0.0015069742191685\\
567	0.00144382934319225\\
568	0.00138100221627712\\
569	0.00131954993805358\\
570	0.0012596888460467\\
571	0.00120169720130309\\
572	0.00114504677832975\\
573	0.00108874011920619\\
574	0.00103277243214585\\
575	0.00097811364792387\\
576	0.000925437814226222\\
577	0.0008749039059792\\
578	0.000825214412511029\\
579	0.000776118395576129\\
580	0.000727556856375548\\
581	0.000679739350145835\\
582	0.00063299572419274\\
583	0.000587101587693234\\
584	0.000541705757972209\\
585	0.000496801587980551\\
586	0.000452422511210272\\
587	0.000408600951836586\\
588	0.000365362590765301\\
589	0.000322721662279121\\
590	0.000280677076068218\\
591	0.000239312865627478\\
592	0.00019878864496184\\
593	0.000159293651685586\\
594	0.000121079888727804\\
595	8.45520570083909e-05\\
596	5.0509214868037e-05\\
597	2.07908715710836e-05\\
598	0\\
599	0\\
600	0\\
};
\addplot [color=blue!75!mycolor7,solid,forget plot]
  table[row sep=crcr]{%
1	0.00983598579693763\\
2	0.00983598411159307\\
3	0.00983598239790594\\
4	0.00983598065539993\\
5	0.00983597888359073\\
6	0.00983597708198586\\
7	0.00983597525008457\\
8	0.00983597338737765\\
9	0.00983597149334735\\
10	0.00983596956746719\\
11	0.00983596760920181\\
12	0.00983596561800683\\
13	0.00983596359332872\\
14	0.00983596153460462\\
15	0.00983595944126217\\
16	0.00983595731271935\\
17	0.00983595514838439\\
18	0.00983595294765546\\
19	0.00983595070992064\\
20	0.00983594843455768\\
21	0.00983594612093382\\
22	0.00983594376840563\\
23	0.00983594137631882\\
24	0.00983593894400807\\
25	0.00983593647079682\\
26	0.00983593395599707\\
27	0.00983593139890923\\
28	0.00983592879882189\\
29	0.0098359261550116\\
30	0.00983592346674269\\
31	0.00983592073326708\\
32	0.00983591795382402\\
33	0.0098359151276399\\
34	0.00983591225392803\\
35	0.0098359093318884\\
36	0.00983590636070748\\
37	0.00983590333955796\\
38	0.00983590026759852\\
39	0.00983589714397359\\
40	0.00983589396781312\\
41	0.00983589073823231\\
42	0.00983588745433135\\
43	0.0098358841151952\\
44	0.00983588071989328\\
45	0.00983587726747922\\
46	0.00983587375699062\\
47	0.0098358701874487\\
48	0.0098358665578581\\
49	0.00983586286720652\\
50	0.00983585911446448\\
51	0.00983585529858498\\
52	0.00983585141850324\\
53	0.00983584747313635\\
54	0.00983584346138297\\
55	0.00983583938212302\\
56	0.00983583523421735\\
57	0.00983583101650739\\
58	0.00983582672781482\\
59	0.00983582236694124\\
60	0.0098358179326678\\
61	0.00983581342375487\\
62	0.00983580883894161\\
63	0.00983580417694569\\
64	0.00983579943646286\\
65	0.00983579461616655\\
66	0.00983578971470752\\
67	0.00983578473071342\\
68	0.00983577966278844\\
69	0.00983577450951282\\
70	0.00983576926944251\\
71	0.00983576394110868\\
72	0.00983575852301727\\
73	0.00983575301364863\\
74	0.00983574741145698\\
75	0.00983574171486998\\
76	0.00983573592228828\\
77	0.00983573003208498\\
78	0.0098357240426052\\
79	0.00983571795216557\\
80	0.00983571175905367\\
81	0.00983570546152761\\
82	0.00983569905781539\\
83	0.00983569254611445\\
84	0.00983568592459109\\
85	0.0098356791913799\\
86	0.00983567234458321\\
87	0.00983566538227052\\
88	0.00983565830247785\\
89	0.00983565110320724\\
90	0.00983564378242602\\
91	0.00983563633806631\\
92	0.00983562876802426\\
93	0.00983562107015949\\
94	0.00983561324229438\\
95	0.00983560528221342\\
96	0.00983559718766253\\
97	0.00983558895634832\\
98	0.00983558058593741\\
99	0.00983557207405573\\
100	0.00983556341828769\\
101	0.00983555461617552\\
102	0.00983554566521845\\
103	0.00983553656287193\\
104	0.00983552730654685\\
105	0.00983551789360869\\
106	0.00983550832137675\\
107	0.00983549858712324\\
108	0.00983548868807246\\
109	0.00983547862139991\\
110	0.0098354683842314\\
111	0.00983545797364213\\
112	0.00983544738665577\\
113	0.00983543662024351\\
114	0.00983542567132311\\
115	0.00983541453675792\\
116	0.00983540321335585\\
117	0.0098353916978684\\
118	0.00983537998698958\\
119	0.00983536807735489\\
120	0.00983535596554026\\
121	0.00983534364806088\\
122	0.0098353311213702\\
123	0.00983531838185871\\
124	0.00983530542585284\\
125	0.00983529224961377\\
126	0.00983527884933623\\
127	0.00983526522114734\\
128	0.00983525136110529\\
129	0.00983523726519819\\
130	0.00983522292934274\\
131	0.00983520834938293\\
132	0.00983519352108877\\
133	0.00983517844015494\\
134	0.0098351631021994\\
135	0.00983514750276208\\
136	0.00983513163730345\\
137	0.00983511550120311\\
138	0.00983509908975833\\
139	0.00983508239818266\\
140	0.00983506542160439\\
141	0.00983504815506511\\
142	0.00983503059351812\\
143	0.009835012731827\\
144	0.00983499456476397\\
145	0.00983497608700835\\
146	0.009834957293145\\
147	0.00983493817766266\\
148	0.00983491873495236\\
149	0.00983489895930575\\
150	0.0098348788449135\\
151	0.00983485838586352\\
152	0.00983483757613938\\
153	0.00983481640961852\\
154	0.00983479488007056\\
155	0.00983477298115552\\
156	0.00983475070642213\\
157	0.00983472804930598\\
158	0.00983470500312777\\
159	0.00983468156109151\\
160	0.00983465771628264\\
161	0.00983463346166626\\
162	0.00983460879008519\\
163	0.00983458369425816\\
164	0.00983455816677784\\
165	0.00983453220010894\\
166	0.00983450578658627\\
167	0.00983447891841271\\
168	0.00983445158765724\\
169	0.00983442378625285\\
170	0.00983439550599447\\
171	0.00983436673853685\\
172	0.00983433747539234\\
173	0.00983430770792866\\
174	0.00983427742736665\\
175	0.00983424662477785\\
176	0.00983421529108211\\
177	0.00983418341704506\\
178	0.00983415099327551\\
179	0.00983411801022278\\
180	0.00983408445817392\\
181	0.00983405032725079\\
182	0.0098340156074071\\
183	0.00983398028842531\\
184	0.00983394435991335\\
185	0.00983390781130135\\
186	0.00983387063183816\\
187	0.00983383281058781\\
188	0.00983379433642582\\
189	0.00983375519803545\\
190	0.00983371538390387\\
191	0.00983367488231827\\
192	0.00983363368136185\\
193	0.00983359176890989\\
194	0.00983354913262567\\
195	0.00983350575995649\\
196	0.00983346163812967\\
197	0.00983341675414844\\
198	0.00983337109478805\\
199	0.00983332464659158\\
200	0.0098332773958658\\
201	0.00983322932867695\\
202	0.00983318043084641\\
203	0.00983313068794629\\
204	0.00983308008529495\\
205	0.00983302860795242\\
206	0.00983297624071581\\
207	0.00983292296811446\\
208	0.00983286877440525\\
209	0.00983281364356758\\
210	0.00983275755929841\\
211	0.00983270050500715\\
212	0.00983264246381048\\
213	0.00983258341852703\\
214	0.00983252335167203\\
215	0.00983246224545177\\
216	0.00983240008175807\\
217	0.00983233684216253\\
218	0.00983227250791074\\
219	0.00983220705991638\\
220	0.00983214047875521\\
221	0.0098320727446589\\
222	0.00983200383750881\\
223	0.00983193373682962\\
224	0.00983186242178282\\
225	0.00983178987116015\\
226	0.00983171606337684\\
227	0.00983164097646476\\
228	0.00983156458806543\\
229	0.00983148687542292\\
230	0.00983140781537661\\
231	0.00983132738435378\\
232	0.0098312455583621\\
233	0.009831162312982\\
234	0.00983107762335881\\
235	0.00983099146419485\\
236	0.00983090380974132\\
237	0.00983081463379006\\
238	0.0098307239096651\\
239	0.00983063161021417\\
240	0.00983053770779993\\
241	0.00983044217429109\\
242	0.00983034498105338\\
243	0.00983024609894031\\
244	0.0098301454982838\\
245	0.0098300431488846\\
246	0.00982993902000258\\
247	0.00982983308034675\\
248	0.00982972529806523\\
249	0.00982961564073488\\
250	0.00982950407535088\\
251	0.00982939056831603\\
252	0.00982927508542986\\
253	0.00982915759187762\\
254	0.00982903805221893\\
255	0.00982891643037636\\
256	0.00982879268962371\\
257	0.00982866679257412\\
258	0.00982853870116795\\
259	0.00982840837666048\\
260	0.00982827577960929\\
261	0.00982814086986154\\
262	0.00982800360654092\\
263	0.00982786394803442\\
264	0.00982772185197886\\
265	0.00982757727524719\\
266	0.0098274301739345\\
267	0.00982728050334386\\
268	0.00982712821797182\\
269	0.00982697327149379\\
270	0.00982681561674903\\
271	0.00982665520572546\\
272	0.00982649198954422\\
273	0.00982632591844391\\
274	0.00982615694176462\\
275	0.00982598500793166\\
276	0.00982581006443908\\
277	0.0098256320578328\\
278	0.00982545093369359\\
279	0.00982526663661972\\
280	0.0098250791102093\\
281	0.00982488829704241\\
282	0.00982469413866293\\
283	0.00982449657556001\\
284	0.00982429554714937\\
285	0.00982409099175425\\
286	0.00982388284658605\\
287	0.00982367104772479\\
288	0.00982345553009914\\
289	0.0098232362274663\\
290	0.00982301307239145\\
291	0.00982278599622708\\
292	0.00982255492909187\\
293	0.00982231979984942\\
294	0.00982208053608657\\
295	0.00982183706409158\\
296	0.00982158930883191\\
297	0.00982133719393176\\
298	0.00982108064164936\\
299	0.00982081957285399\\
300	0.00982055390700268\\
301	0.00982028356211668\\
302	0.0098200084547577\\
303	0.00981972850000382\\
304	0.00981944361142525\\
305	0.00981915370105972\\
306	0.00981885867938778\\
307	0.00981855845530773\\
308	0.0098182529361104\\
309	0.00981794202745373\\
310	0.00981762563333706\\
311	0.00981730365607531\\
312	0.00981697599627289\\
313	0.0098166425527975\\
314	0.00981630322275366\\
315	0.00981595790145617\\
316	0.00981560648240336\\
317	0.00981524885725017\\
318	0.00981488491578119\\
319	0.00981451454588345\\
320	0.00981413763351916\\
321	0.00981375406269833\\
322	0.0098133637154513\\
323	0.00981296647180112\\
324	0.00981256220973592\\
325	0.00981215080518117\\
326	0.00981173213197181\\
327	0.00981130606182442\\
328	0.0098108724643092\\
329	0.00981043120682202\\
330	0.00980998215455628\\
331	0.00980952517047473\\
332	0.00980906011528128\\
333	0.00980858684739264\\
334	0.0098081052229099\\
335	0.00980761509558998\\
336	0.00980711631681693\\
337	0.00980660873557315\\
338	0.00980609219841025\\
339	0.00980556654941983\\
340	0.00980503163020391\\
341	0.00980448727984503\\
342	0.00980393333487598\\
343	0.00980336962924906\\
344	0.00980279599430481\\
345	0.0098022122587401\\
346	0.00980161824857549\\
347	0.00980101378712182\\
348	0.00980039869494577\\
349	0.00979977278983434\\
350	0.00979913588675813\\
351	0.00979848779783319\\
352	0.00979782833228123\\
353	0.00979715729638818\\
354	0.00979647449346058\\
355	0.00979577972377991\\
356	0.00979507278455434\\
357	0.00979435346986779\\
358	0.00979362157062594\\
359	0.009792876874499\\
360	0.00979211916586069\\
361	0.00979134822572343\\
362	0.00979056383166907\\
363	0.00978976575777495\\
364	0.00978895377453493\\
365	0.00978812764877493\\
366	0.00978728714356253\\
367	0.00978643201811032\\
368	0.0097855620276725\\
369	0.00978467692343428\\
370	0.00978377645239367\\
371	0.00978286035723521\\
372	0.00978192837619511\\
373	0.00978098024291748\\
374	0.00978001568630101\\
375	0.00977903443033571\\
376	0.00977803619392928\\
377	0.00977702069072246\\
378	0.00977598762889288\\
379	0.00977493671094685\\
380	0.00977386763349839\\
381	0.00977278008703469\\
382	0.00977167375566731\\
383	0.00977054831686783\\
384	0.0097694034411868\\
385	0.00976823879195447\\
386	0.00976705402496097\\
387	0.00976584878811385\\
388	0.00976462272106999\\
389	0.00976337545483856\\
390	0.00976210661134679\\
391	0.00976081580296672\\
392	0.00975950263199832\\
393	0.00975816669010216\\
394	0.00975680755767175\\
395	0.00975542480313566\\
396	0.009754017982191\\
397	0.00975258663695195\\
398	0.00975113029498769\\
399	0.0097496484682359\\
400	0.00974814065177603\\
401	0.0097466063224438\\
402	0.0097450449372837\\
403	0.00974345593202622\\
404	0.00974183872085087\\
405	0.00974019270298737\\
406	0.00973851726202865\\
407	0.00973681173897762\\
408	0.00973507542762574\\
409	0.00973330759298671\\
410	0.00973150746984965\\
411	0.00972967426148173\\
412	0.00972780713863298\\
413	0.00972590523909215\\
414	0.00972396766814842\\
415	0.0097219935001699\\
416	0.00971998177979321\\
417	0.00971793151220656\\
418	0.00971584159280819\\
419	0.00971371092465262\\
420	0.009711538524179\\
421	0.009709323390324\\
422	0.00970706448206493\\
423	0.00970476071462334\\
424	0.00970241095497063\\
425	0.00970001401646347\\
426	0.00969756865238058\\
427	0.00969507354811213\\
428	0.0096925273117392\\
429	0.00968992846244233\\
430	0.00968727541617079\\
431	0.00968456646808999\\
432	0.00968179977106074\\
433	0.00967897330958293\\
434	0.00967608487168244\\
435	0.00967313202846759\\
436	0.00967011215823784\\
437	0.00966702265224127\\
438	0.00966386179836363\\
439	0.00966063213953145\\
440	0.0096581438166461\\
441	0.00965618471398823\\
442	0.00965419083958201\\
443	0.0096521612180493\\
444	0.0096500947584985\\
445	0.00964799022959298\\
446	0.00964584622868649\\
447	0.00964366114348987\\
448	0.00964143310430262\\
449	0.00963915992428517\\
450	0.00963683902452495\\
451	0.00963446733971605\\
452	0.00963204119906393\\
453	0.00962955617537492\\
454	0.00962700689239675\\
455	0.00962438677223191\\
456	0.00962168769901288\\
457	0.00961889955818363\\
458	0.00961600960843108\\
459	0.00961300116654827\\
460	0.00960985036340646\\
461	0.0096061970184462\\
462	0.00956347884637286\\
463	0.00951990473209833\\
464	0.00947545229190016\\
465	0.00943009062697348\\
466	0.00938379558812048\\
467	0.00933654208400312\\
468	0.00928830404964439\\
469	0.00923905444750061\\
470	0.00918876536193582\\
471	0.00913740851543912\\
472	0.00908495400076351\\
473	0.00903136941236922\\
474	0.00897662092565435\\
475	0.00892067309233651\\
476	0.00886348839234452\\
477	0.00880502621294514\\
478	0.00874523943928828\\
479	0.00868408740739547\\
480	0.00862152976468595\\
481	0.00855752423148799\\
482	0.00849202612298176\\
483	0.00842498784328819\\
484	0.00835635834929183\\
485	0.00828608224596591\\
486	0.00821409847920065\\
487	0.0081403390213053\\
488	0.0080647285432332\\
489	0.00798718641478394\\
490	0.00790762535960176\\
491	0.0078259505817493\\
492	0.00774205873614955\\
493	0.0076558367082527\\
494	0.00756716015957302\\
495	0.00747589178508248\\
496	0.00738187921625173\\
497	0.00728495249213396\\
498	0.00718492101848958\\
499	0.00708156996690433\\
500	0.00697465620733147\\
501	0.00686390433814216\\
502	0.0067490048182002\\
503	0.00662962124945524\\
504	0.00654282147243754\\
505	0.00650416080218482\\
506	0.00646417271853324\\
507	0.00642286895962343\\
508	0.006380290798672\\
509	0.00633649164171396\\
510	0.00629140003101015\\
511	0.00624491050299828\\
512	0.00619690040501686\\
513	0.00614722597860324\\
514	0.00609571778460128\\
515	0.00604217500531807\\
516	0.0059863583248483\\
517	0.00592798101679594\\
518	0.00586669774697961\\
519	0.00580209039815476\\
520	0.00573365010006878\\
521	0.00566075463826947\\
522	0.0055826405375136\\
523	0.00549837040510512\\
524	0.00540680211042718\\
525	0.00530801300573983\\
526	0.00520877949944434\\
527	0.00510955840050037\\
528	0.0050109498653376\\
529	0.00491374327105498\\
530	0.00481743395479156\\
531	0.00472041401216693\\
532	0.0046230730277967\\
533	0.00452592907234172\\
534	0.00442966844522136\\
535	0.00433519904004857\\
536	0.00424372441285738\\
537	0.00415679393801028\\
538	0.00407632456672\\
539	0.00400303780149818\\
540	0.00392833741499244\\
541	0.00385228073333916\\
542	0.0037749313033303\\
543	0.00369635469524612\\
544	0.00361661131706036\\
545	0.00353574439772463\\
546	0.00345377407636575\\
547	0.0033706573271108\\
548	0.00328623817180589\\
549	0.00320020881435913\\
550	0.00311226168746383\\
551	0.00302174557633384\\
552	0.0029285721704536\\
553	0.00283284845112247\\
554	0.00273472165432939\\
555	0.0026343666970143\\
556	0.00253200294034252\\
557	0.00242802872893287\\
558	0.00232311566314357\\
559	0.00221851657094213\\
560	0.00211831907397233\\
561	0.00202299700719578\\
562	0.00193303203249592\\
563	0.00184882022133786\\
564	0.00177042797505062\\
565	0.00169520130901441\\
566	0.00162172309513774\\
567	0.00154972673714696\\
568	0.00147890019079677\\
569	0.00140915704425968\\
570	0.00133996649982656\\
571	0.00127121916543033\\
572	0.00120361310685485\\
573	0.00113828975820201\\
574	0.0010750352614193\\
575	0.00101406573453819\\
576	0.00095501450658787\\
577	0.000896782871369601\\
578	0.000840716170556225\\
579	0.0007870993751659\\
580	0.000735978059148706\\
581	0.000686067843775887\\
582	0.000637128970025702\\
583	0.000589268610066713\\
584	0.000542736806562125\\
585	0.000497389550407257\\
586	0.000452751633483821\\
587	0.000408776510568709\\
588	0.000365448460457315\\
589	0.000322758538660347\\
590	0.000280690270321202\\
591	0.000239316379010231\\
592	0.000198789174864552\\
593	0.000159293651685586\\
594	0.000121079888727805\\
595	8.45520570083912e-05\\
596	5.05092148680373e-05\\
597	2.07908715710836e-05\\
598	0\\
599	0\\
600	0\\
};
\addplot [color=blue!80!mycolor9,solid,forget plot]
  table[row sep=crcr]{%
1	0.00995024965342724\\
2	0.0099502494926092\\
3	0.00995024932908664\\
4	0.00995024916281409\\
5	0.0099502489937453\\
6	0.00995024882183327\\
7	0.00995024864703018\\
8	0.00995024846928742\\
9	0.00995024828855554\\
10	0.00995024810478429\\
11	0.00995024791792253\\
12	0.00995024772791829\\
13	0.00995024753471872\\
14	0.00995024733827005\\
15	0.00995024713851763\\
16	0.00995024693540588\\
17	0.00995024672887828\\
18	0.00995024651887735\\
19	0.00995024630534464\\
20	0.00995024608822072\\
21	0.00995024586744516\\
22	0.00995024564295647\\
23	0.00995024541469219\\
24	0.00995024518258872\\
25	0.00995024494658145\\
26	0.00995024470660464\\
27	0.00995024446259145\\
28	0.00995024421447391\\
29	0.0099502439621829\\
30	0.00995024370564812\\
31	0.00995024344479807\\
32	0.00995024317956006\\
33	0.00995024290986016\\
34	0.00995024263562317\\
35	0.00995024235677264\\
36	0.00995024207323081\\
37	0.00995024178491859\\
38	0.00995024149175555\\
39	0.0099502411936599\\
40	0.00995024089054847\\
41	0.00995024058233666\\
42	0.00995024026893843\\
43	0.00995023995026628\\
44	0.00995023962623122\\
45	0.00995023929674274\\
46	0.00995023896170881\\
47	0.00995023862103581\\
48	0.00995023827462851\\
49	0.0099502379223901\\
50	0.00995023756422208\\
51	0.00995023720002429\\
52	0.00995023682969484\\
53	0.00995023645313012\\
54	0.00995023607022472\\
55	0.00995023568087147\\
56	0.00995023528496134\\
57	0.00995023488238342\\
58	0.00995023447302493\\
59	0.00995023405677114\\
60	0.00995023363350536\\
61	0.0099502332031089\\
62	0.00995023276546102\\
63	0.00995023232043893\\
64	0.00995023186791771\\
65	0.00995023140777031\\
66	0.00995023093986748\\
67	0.00995023046407775\\
68	0.00995022998026739\\
69	0.00995022948830038\\
70	0.00995022898803832\\
71	0.00995022847934045\\
72	0.0099502279620636\\
73	0.00995022743606207\\
74	0.0099502269011877\\
75	0.00995022635728974\\
76	0.00995022580421483\\
77	0.00995022524180698\\
78	0.00995022466990747\\
79	0.00995022408835485\\
80	0.00995022349698486\\
81	0.00995022289563039\\
82	0.00995022228412143\\
83	0.00995022166228502\\
84	0.00995022102994519\\
85	0.00995022038692289\\
86	0.00995021973303601\\
87	0.00995021906809919\\
88	0.00995021839192389\\
89	0.00995021770431828\\
90	0.00995021700508715\\
91	0.0099502162940319\\
92	0.00995021557095047\\
93	0.00995021483563724\\
94	0.00995021408788301\\
95	0.0099502133274749\\
96	0.00995021255419632\\
97	0.00995021176782685\\
98	0.00995021096814223\\
99	0.00995021015491425\\
100	0.00995020932791068\\
101	0.00995020848689522\\
102	0.0099502076316274\\
103	0.00995020676186253\\
104	0.00995020587735159\\
105	0.00995020497784118\\
106	0.00995020406307343\\
107	0.00995020313278591\\
108	0.00995020218671155\\
109	0.00995020122457859\\
110	0.00995020024611043\\
111	0.00995019925102558\\
112	0.00995019823903758\\
113	0.00995019720985489\\
114	0.00995019616318081\\
115	0.00995019509871337\\
116	0.00995019401614525\\
117	0.00995019291516367\\
118	0.00995019179545031\\
119	0.00995019065668119\\
120	0.00995018949852657\\
121	0.00995018832065087\\
122	0.00995018712271253\\
123	0.00995018590436392\\
124	0.00995018466525123\\
125	0.00995018340501437\\
126	0.00995018212328684\\
127	0.00995018081969562\\
128	0.00995017949386106\\
129	0.00995017814539676\\
130	0.00995017677390944\\
131	0.00995017537899884\\
132	0.00995017396025759\\
133	0.00995017251727106\\
134	0.00995017104961728\\
135	0.00995016955686677\\
136	0.00995016803858243\\
137	0.00995016649431942\\
138	0.00995016492362499\\
139	0.00995016332603836\\
140	0.00995016170109063\\
141	0.00995016004830457\\
142	0.00995015836719451\\
143	0.00995015665726621\\
144	0.00995015491801671\\
145	0.00995015314893415\\
146	0.0099501513494977\\
147	0.00995014951917732\\
148	0.00995014765743366\\
149	0.00995014576371792\\
150	0.00995014383747166\\
151	0.00995014187812665\\
152	0.00995013988510474\\
153	0.00995013785781765\\
154	0.00995013579566686\\
155	0.00995013369804341\\
156	0.00995013156432773\\
157	0.00995012939388952\\
158	0.0099501271860875\\
159	0.00995012494026931\\
160	0.00995012265577127\\
161	0.00995012033191825\\
162	0.00995011796802346\\
163	0.00995011556338825\\
164	0.00995011311730196\\
165	0.00995011062904168\\
166	0.00995010809787208\\
167	0.00995010552304521\\
168	0.00995010290380026\\
169	0.00995010023936337\\
170	0.00995009752894742\\
171	0.00995009477175178\\
172	0.00995009196696209\\
173	0.00995008911375003\\
174	0.00995008621127306\\
175	0.00995008325867418\\
176	0.00995008025508167\\
177	0.00995007719960882\\
178	0.00995007409135364\\
179	0.00995007092939862\\
180	0.00995006771281036\\
181	0.00995006444063938\\
182	0.00995006111191971\\
183	0.00995005772566862\\
184	0.0099500542808863\\
185	0.00995005077655552\\
186	0.00995004721164129\\
187	0.00995004358509052\\
188	0.00995003989583166\\
189	0.00995003614277436\\
190	0.00995003232480913\\
191	0.00995002844080692\\
192	0.0099500244896188\\
193	0.0099500204700756\\
194	0.0099500163809875\\
195	0.00995001222114367\\
196	0.00995000798931191\\
197	0.00995000368423822\\
198	0.00994999930464644\\
199	0.00994999484923785\\
200	0.0099499903166907\\
201	0.00994998570565989\\
202	0.00994998101477646\\
203	0.00994997624264718\\
204	0.00994997138785415\\
205	0.00994996644895427\\
206	0.00994996142447887\\
207	0.00994995631293317\\
208	0.00994995111279585\\
209	0.00994994582251851\\
210	0.00994994044052526\\
211	0.00994993496521215\\
212	0.00994992939494666\\
213	0.0099499237280672\\
214	0.00994991796288258\\
215	0.00994991209767143\\
216	0.0099499061306817\\
217	0.00994990006013004\\
218	0.00994989388420129\\
219	0.0099498876010478\\
220	0.00994988120878893\\
221	0.0099498747055104\\
222	0.00994986808926366\\
223	0.00994986135806527\\
224	0.00994985450989627\\
225	0.00994984754270148\\
226	0.00994984045438891\\
227	0.00994983324282898\\
228	0.00994982590585392\\
229	0.00994981844125699\\
230	0.00994981084679182\\
231	0.00994980312017161\\
232	0.00994979525906845\\
233	0.00994978726111251\\
234	0.00994977912389129\\
235	0.00994977084494881\\
236	0.00994976242178481\\
237	0.00994975385185393\\
238	0.00994974513256489\\
239	0.00994973626127961\\
240	0.00994972723531233\\
241	0.0099497180519288\\
242	0.00994970870834526\\
243	0.00994969920172764\\
244	0.00994968952919054\\
245	0.00994967968779633\\
246	0.00994966967455414\\
247	0.00994965948641888\\
248	0.00994964912029027\\
249	0.00994963857301175\\
250	0.00994962784136949\\
251	0.00994961692209132\\
252	0.00994960581184558\\
253	0.00994959450724013\\
254	0.00994958300482111\\
255	0.00994957130107188\\
256	0.00994955939241183\\
257	0.00994954727519518\\
258	0.00994953494570976\\
259	0.00994952240017586\\
260	0.00994950963474487\\
261	0.00994949664549808\\
262	0.00994948342844539\\
263	0.00994946997952393\\
264	0.0099494562945968\\
265	0.00994944236945163\\
266	0.00994942819979924\\
267	0.00994941378127223\\
268	0.00994939910942352\\
269	0.0099493841797249\\
270	0.00994936898756556\\
271	0.00994935352825057\\
272	0.00994933779699935\\
273	0.00994932178894409\\
274	0.00994930549912823\\
275	0.00994928892250477\\
276	0.0099492720539347\\
277	0.00994925488818531\\
278	0.00994923741992849\\
279	0.00994921964373906\\
280	0.00994920155409301\\
281	0.00994918314536572\\
282	0.00994916441183021\\
283	0.00994914534765528\\
284	0.00994912594690369\\
285	0.0099491062035303\\
286	0.00994908611138015\\
287	0.00994906566418656\\
288	0.00994904485556918\\
289	0.00994902367903201\\
290	0.00994900212796138\\
291	0.00994898019562399\\
292	0.00994895787516478\\
293	0.0099489351596049\\
294	0.00994891204183962\\
295	0.00994888851463616\\
296	0.00994886457063157\\
297	0.00994884020233053\\
298	0.00994881540210321\\
299	0.00994879016218295\\
300	0.00994876447466411\\
301	0.00994873833149973\\
302	0.00994871172449927\\
303	0.00994868464532629\\
304	0.0099486570854961\\
305	0.00994862903637343\\
306	0.00994860048917002\\
307	0.00994857143494221\\
308	0.00994854186458856\\
309	0.00994851176884737\\
310	0.00994848113829424\\
311	0.00994844996333959\\
312	0.0099484182342261\\
313	0.00994838594102629\\
314	0.00994835307363989\\
315	0.00994831962179133\\
316	0.00994828557502712\\
317	0.0099482509227133\\
318	0.00994821565403277\\
319	0.00994817975798267\\
320	0.00994814322337174\\
321	0.00994810603881756\\
322	0.00994806819274393\\
323	0.00994802967337804\\
324	0.0099479904687478\\
325	0.00994795056667894\\
326	0.00994790995479227\\
327	0.00994786862050074\\
328	0.00994782655100661\\
329	0.00994778373329841\\
330	0.00994774015414801\\
331	0.00994769580010754\\
332	0.00994765065750626\\
333	0.0099476047124474\\
334	0.00994755795080488\\
335	0.00994751035821997\\
336	0.00994746192009786\\
337	0.00994741262160411\\
338	0.00994736244766097\\
339	0.00994731138294361\\
340	0.00994725941187616\\
341	0.00994720651862764\\
342	0.00994715268710765\\
343	0.00994709790096189\\
344	0.00994704214356749\\
345	0.00994698539802803\\
346	0.0099469276471684\\
347	0.00994686887352922\\
348	0.00994680905936108\\
349	0.00994674818661835\\
350	0.00994668623695262\\
351	0.00994662319170575\\
352	0.00994655903190239\\
353	0.00994649373824212\\
354	0.00994642729109096\\
355	0.00994635967047233\\
356	0.00994629085605737\\
357	0.00994622082715467\\
358	0.00994614956269912\\
359	0.00994607704124017\\
360	0.0099460032409291\\
361	0.00994592813950552\\
362	0.00994585171428288\\
363	0.00994577394213301\\
364	0.00994569479946957\\
365	0.00994561426223041\\
366	0.00994553230585879\\
367	0.00994544890528327\\
368	0.00994536403489642\\
369	0.00994527766853207\\
370	0.0099451897794412\\
371	0.00994510034026634\\
372	0.00994500932301441\\
373	0.0099449166990279\\
374	0.00994482243895451\\
375	0.00994472651271482\\
376	0.00994462888946838\\
377	0.00994452953757768\\
378	0.0099444284245702\\
379	0.00994432551709836\\
380	0.00994422078089722\\
381	0.00994411418073977\\
382	0.00994400568038974\\
383	0.00994389524255166\\
384	0.00994378282881784\\
385	0.00994366839961223\\
386	0.00994355191413059\\
387	0.00994343333027662\\
388	0.00994331260459367\\
389	0.00994318969219126\\
390	0.0099430645466659\\
391	0.00994293712001525\\
392	0.00994280736254448\\
393	0.00994267522276367\\
394	0.0099425406472743\\
395	0.00994240358064276\\
396	0.00994226396525818\\
397	0.00994212174117097\\
398	0.00994197684590766\\
399	0.00994182921425599\\
400	0.00994167877801249\\
401	0.00994152546568373\\
402	0.00994136920213928\\
403	0.00994120990824239\\
404	0.00994104750053987\\
405	0.00994088189096524\\
406	0.00994071298604871\\
407	0.00994054068635598\\
408	0.00994036488696237\\
409	0.00994018547708384\\
410	0.0099400023397103\\
411	0.00993981535125655\\
412	0.00993962438125417\\
413	0.00993942929211299\\
414	0.00993922993896757\\
415	0.0099390261695445\\
416	0.00993881782376216\\
417	0.00993860473254393\\
418	0.00993838671728155\\
419	0.00993816359336768\\
420	0.00993793516339354\\
421	0.00993770121470024\\
422	0.00993746151751359\\
423	0.00993721582279131\\
424	0.00993696385973073\\
425	0.00993670533287484\\
426	0.00993643991874117\\
427	0.00993616726188308\\
428	0.00993588697026948\\
429	0.00993559860983968\\
430	0.00993530169804049\\
431	0.00993499569604213\\
432	0.00993467999905065\\
433	0.00993435392334768\\
434	0.0099340166861986\\
435	0.00993366736666794\\
436	0.00993330480829898\\
437	0.00993292733222369\\
438	0.00993253180907329\\
439	0.00993211051292307\\
440	0.00993090909648561\\
441	0.00992913499893833\\
442	0.00992731780245534\\
443	0.00992545602322119\\
444	0.0099235481080655\\
445	0.00992159243265715\\
446	0.00991958730045363\\
447	0.00991753094274116\\
448	0.00991542152021376\\
449	0.00991325712668699\\
450	0.00991103579573368\\
451	0.00990875551127558\\
452	0.00990641422346764\\
453	0.00990400987154726\\
454	0.00990154041558171\\
455	0.0098990038789983\\
456	0.00989639840298146\\
457	0.00989372230993868\\
458	0.00989097416419154\\
459	0.00988815280269732\\
460	0.00988525731382974\\
461	0.00988227107081666\\
462	0.00987726586402313\\
463	0.00987217644234873\\
464	0.00986701232349152\\
465	0.00986207582144072\\
466	0.00985706019359379\\
467	0.0098519638048553\\
468	0.00984678493520374\\
469	0.00984152177448637\\
470	0.00983617244261739\\
471	0.00983073497250655\\
472	0.00982520722452214\\
473	0.00981958689007658\\
474	0.00981387154405269\\
475	0.00980805864121547\\
476	0.00980214548286862\\
477	0.00979612902455971\\
478	0.00979000614332297\\
479	0.0097837739541957\\
480	0.00977742945990819\\
481	0.00977096943423777\\
482	0.00976439038006545\\
483	0.00975768849293456\\
484	0.00975085960756883\\
485	0.00974389912357521\\
486	0.00973680192042395\\
487	0.00972956229998946\\
488	0.00972217402843945\\
489	0.00971463020511259\\
490	0.0097069231534983\\
491	0.00969904429054335\\
492	0.00969098396942824\\
493	0.00968273128975784\\
494	0.00967427386756375\\
495	0.00966559755544607\\
496	0.00965668610019438\\
497	0.00964752072027568\\
498	0.00963807957556147\\
499	0.00962833707685462\\
500	0.00961826291146477\\
501	0.00960782043746308\\
502	0.00959696336322506\\
503	0.00958562636004133\\
504	0.00953824077459935\\
505	0.0094428667253592\\
506	0.00934773776944827\\
507	0.00925021459251212\\
508	0.00915018834629194\\
509	0.00904753857441189\\
510	0.00894213426567151\\
511	0.00883382755758078\\
512	0.0087224673322712\\
513	0.00860790122505926\\
514	0.00848996516568321\\
515	0.00836848236237575\\
516	0.0082432622479476\\
517	0.00811409941774295\\
518	0.00798077260987465\\
519	0.00784304381493639\\
520	0.00770065766270751\\
521	0.00755334132295126\\
522	0.00740080524889721\\
523	0.00724274490892458\\
524	0.00707883679535215\\
525	0.00690873609418503\\
526	0.00673202972081162\\
527	0.00654799664263\\
528	0.00635577039827619\\
529	0.00615428802740013\\
530	0.00599822874844659\\
531	0.0059229785927858\\
532	0.00584393196698481\\
533	0.0057605063025082\\
534	0.00567197500609394\\
535	0.00557742595482749\\
536	0.00547570712171949\\
537	0.00536527523908269\\
538	0.00524427765188107\\
539	0.00511240149795213\\
540	0.00497826976213729\\
541	0.00484217428987451\\
542	0.00470452003423817\\
543	0.00456585911209548\\
544	0.00442693566006206\\
545	0.00428874727498958\\
546	0.00415262655058662\\
547	0.00402035366075448\\
548	0.00389417407788781\\
549	0.00377651184121573\\
550	0.00366528682586713\\
551	0.00356355257474437\\
552	0.00346211700310143\\
553	0.00335854913922197\\
554	0.00325299748522605\\
555	0.00314563583395675\\
556	0.00303666170785842\\
557	0.00292628271907932\\
558	0.00281469808724814\\
559	0.00270207149648971\\
560	0.00258855827107614\\
561	0.00247403834706246\\
562	0.00235818448456371\\
563	0.00224107956118409\\
564	0.00212364440435233\\
565	0.00200907458233774\\
566	0.00189926355595736\\
567	0.00179479801060058\\
568	0.00169626418907342\\
569	0.00160400293221254\\
570	0.00151783759986445\\
571	0.00143390318578877\\
572	0.00135203502647469\\
573	0.00127200640621873\\
574	0.00119387836966914\\
575	0.00111724688276997\\
576	0.0010422787258352\\
577	0.000970248500825921\\
578	0.000901376380147175\\
579	0.000835676581260966\\
580	0.000773155683964323\\
581	0.000713512969864073\\
582	0.000657039252267262\\
583	0.000603820761156831\\
584	0.000552908173124348\\
585	0.000503686775016135\\
586	0.0004564261874486\\
587	0.000410922180362966\\
588	0.000366669791782508\\
589	0.000323409636595769\\
590	0.000280998940489726\\
591	0.000239439664031353\\
592	0.000198826448456301\\
593	0.000159300178734201\\
594	0.000121079888727804\\
595	8.45520570083909e-05\\
596	5.05092148680373e-05\\
597	2.07908715710836e-05\\
598	0\\
599	0\\
600	0\\
};
\addplot [color=blue,solid,forget plot]
  table[row sep=crcr]{%
1	0.00996996602909277\\
2	0.00996996602247315\\
3	0.0099699660157422\\
4	0.00996996600889807\\
5	0.00996996600193883\\
6	0.00996996599486257\\
7	0.00996996598766729\\
8	0.00996996598035102\\
9	0.00996996597291171\\
10	0.00996996596534729\\
11	0.00996996595765566\\
12	0.00996996594983469\\
13	0.00996996594188218\\
14	0.00996996593379593\\
15	0.00996996592557369\\
16	0.00996996591721317\\
17	0.00996996590871205\\
18	0.00996996590006795\\
19	0.00996996589127847\\
20	0.00996996588234117\\
21	0.00996996587325356\\
22	0.0099699658640131\\
23	0.00996996585461723\\
24	0.00996996584506332\\
25	0.00996996583534872\\
26	0.00996996582547072\\
27	0.00996996581542656\\
28	0.00996996580521345\\
29	0.00996996579482855\\
30	0.00996996578426895\\
31	0.00996996577353172\\
32	0.00996996576261385\\
33	0.00996996575151232\\
34	0.00996996574022401\\
35	0.00996996572874579\\
36	0.00996996571707446\\
37	0.00996996570520674\\
38	0.00996996569313934\\
39	0.00996996568086888\\
40	0.00996996566839194\\
41	0.00996996565570504\\
42	0.00996996564280464\\
43	0.00996996562968712\\
44	0.00996996561634883\\
45	0.00996996560278604\\
46	0.00996996558899495\\
47	0.00996996557497172\\
48	0.00996996556071242\\
49	0.00996996554621307\\
50	0.0099699655314696\\
51	0.0099699655164779\\
52	0.00996996550123376\\
53	0.00996996548573293\\
54	0.00996996546997105\\
55	0.00996996545394371\\
56	0.00996996543764643\\
57	0.00996996542107463\\
58	0.00996996540422368\\
59	0.00996996538708884\\
60	0.00996996536966531\\
61	0.00996996535194819\\
62	0.00996996533393252\\
63	0.00996996531561324\\
64	0.0099699652969852\\
65	0.00996996527804317\\
66	0.00996996525878183\\
67	0.00996996523919575\\
68	0.00996996521927943\\
69	0.00996996519902727\\
70	0.00996996517843356\\
71	0.0099699651574925\\
72	0.0099699651361982\\
73	0.00996996511454465\\
74	0.00996996509252575\\
75	0.00996996507013529\\
76	0.00996996504736695\\
77	0.0099699650242143\\
78	0.0099699650006708\\
79	0.0099699649767298\\
80	0.00996996495238452\\
81	0.0099699649276281\\
82	0.00996996490245351\\
83	0.00996996487685363\\
84	0.00996996485082121\\
85	0.00996996482434888\\
86	0.00996996479742911\\
87	0.00996996477005428\\
88	0.00996996474221662\\
89	0.00996996471390822\\
90	0.00996996468512103\\
91	0.00996996465584687\\
92	0.00996996462607742\\
93	0.00996996459580418\\
94	0.00996996456501855\\
95	0.00996996453371174\\
96	0.00996996450187483\\
97	0.00996996446949874\\
98	0.00996996443657422\\
99	0.00996996440309185\\
100	0.00996996436904208\\
101	0.00996996433441515\\
102	0.00996996429920115\\
103	0.00996996426339\\
104	0.00996996422697143\\
105	0.009969964189935\\
106	0.00996996415227006\\
107	0.0099699641139658\\
108	0.00996996407501121\\
109	0.0099699640353951\\
110	0.00996996399510604\\
111	0.00996996395413243\\
112	0.00996996391246248\\
113	0.00996996387008414\\
114	0.0099699638269852\\
115	0.0099699637831532\\
116	0.00996996373857546\\
117	0.00996996369323909\\
118	0.00996996364713098\\
119	0.00996996360023775\\
120	0.00996996355254582\\
121	0.00996996350404134\\
122	0.00996996345471023\\
123	0.00996996340453816\\
124	0.00996996335351054\\
125	0.00996996330161252\\
126	0.009969963248829\\
127	0.00996996319514459\\
128	0.00996996314054364\\
129	0.00996996308501022\\
130	0.00996996302852812\\
131	0.00996996297108083\\
132	0.00996996291265156\\
133	0.00996996285322321\\
134	0.00996996279277839\\
135	0.00996996273129941\\
136	0.00996996266876823\\
137	0.00996996260516652\\
138	0.00996996254047562\\
139	0.00996996247467654\\
140	0.00996996240774995\\
141	0.00996996233967618\\
142	0.00996996227043522\\
143	0.0099699622000067\\
144	0.0099699621283699\\
145	0.00996996205550372\\
146	0.00996996198138669\\
147	0.00996996190599699\\
148	0.00996996182931239\\
149	0.00996996175131028\\
150	0.00996996167196766\\
151	0.00996996159126112\\
152	0.00996996150916683\\
153	0.00996996142566058\\
154	0.00996996134071771\\
155	0.00996996125431315\\
156	0.00996996116642136\\
157	0.00996996107701641\\
158	0.00996996098607187\\
159	0.0099699608935609\\
160	0.00996996079945615\\
161	0.00996996070372984\\
162	0.00996996060635368\\
163	0.00996996050729892\\
164	0.00996996040653629\\
165	0.00996996030403604\\
166	0.00996996019976788\\
167	0.00996996009370104\\
168	0.00996995998580419\\
169	0.00996995987604547\\
170	0.00996995976439248\\
171	0.00996995965081226\\
172	0.00996995953527128\\
173	0.00996995941773546\\
174	0.00996995929817009\\
175	0.0099699591765399\\
176	0.009969959052809\\
177	0.00996995892694088\\
178	0.0099699587988984\\
179	0.00996995866864378\\
180	0.00996995853613859\\
181	0.00996995840134372\\
182	0.00996995826421941\\
183	0.00996995812472517\\
184	0.00996995798281983\\
185	0.00996995783846149\\
186	0.00996995769160753\\
187	0.00996995754221457\\
188	0.00996995739023846\\
189	0.00996995723563431\\
190	0.00996995707835639\\
191	0.00996995691835821\\
192	0.00996995675559244\\
193	0.0099699565900109\\
194	0.0099699564215646\\
195	0.00996995625020365\\
196	0.00996995607587729\\
197	0.00996995589853388\\
198	0.00996995571812082\\
199	0.00996995553458463\\
200	0.00996995534787086\\
201	0.0099699551579241\\
202	0.00996995496468796\\
203	0.00996995476810505\\
204	0.00996995456811695\\
205	0.00996995436466423\\
206	0.00996995415768639\\
207	0.00996995394712184\\
208	0.00996995373290793\\
209	0.00996995351498086\\
210	0.00996995329327574\\
211	0.00996995306772648\\
212	0.00996995283826584\\
213	0.00996995260482538\\
214	0.00996995236733544\\
215	0.0099699521257251\\
216	0.00996995187992221\\
217	0.00996995162985329\\
218	0.0099699513754436\\
219	0.00996995111661702\\
220	0.00996995085329609\\
221	0.00996995058540196\\
222	0.00996995031285437\\
223	0.00996995003557163\\
224	0.00996994975347059\\
225	0.00996994946646659\\
226	0.00996994917447347\\
227	0.00996994887740352\\
228	0.00996994857516746\\
229	0.00996994826767441\\
230	0.00996994795483184\\
231	0.00996994763654557\\
232	0.00996994731271975\\
233	0.00996994698325675\\
234	0.00996994664805725\\
235	0.00996994630702009\\
236	0.0099699459600423\\
237	0.00996994560701908\\
238	0.00996994524784372\\
239	0.00996994488240757\\
240	0.00996994451060006\\
241	0.00996994413230859\\
242	0.00996994374741854\\
243	0.00996994335581322\\
244	0.00996994295737383\\
245	0.00996994255197941\\
246	0.00996994213950683\\
247	0.00996994171983073\\
248	0.00996994129282346\\
249	0.00996994085835508\\
250	0.0099699404162933\\
251	0.00996993996650341\\
252	0.00996993950884827\\
253	0.00996993904318826\\
254	0.00996993856938121\\
255	0.00996993808728239\\
256	0.00996993759674444\\
257	0.00996993709761731\\
258	0.00996993658974824\\
259	0.00996993607298169\\
260	0.00996993554715929\\
261	0.00996993501211981\\
262	0.00996993446769908\\
263	0.00996993391372993\\
264	0.00996993335004219\\
265	0.00996993277646257\\
266	0.00996993219281463\\
267	0.00996993159891875\\
268	0.009969930994592\\
269	0.00996993037964818\\
270	0.00996992975389767\\
271	0.00996992911714741\\
272	0.00996992846920084\\
273	0.00996992780985784\\
274	0.00996992713891465\\
275	0.00996992645616379\\
276	0.00996992576139404\\
277	0.00996992505439036\\
278	0.00996992433493378\\
279	0.00996992360280138\\
280	0.0099699228577662\\
281	0.00996992209959719\\
282	0.00996992132805907\\
283	0.00996992054291237\\
284	0.00996991974391325\\
285	0.00996991893081348\\
286	0.00996991810336036\\
287	0.00996991726129664\\
288	0.00996991640436042\\
289	0.00996991553228512\\
290	0.00996991464479934\\
291	0.00996991374162682\\
292	0.00996991282248637\\
293	0.00996991188709173\\
294	0.00996991093515156\\
295	0.0099699099663693\\
296	0.00996990898044311\\
297	0.00996990797706577\\
298	0.00996990695592461\\
299	0.00996990591670143\\
300	0.00996990485907237\\
301	0.00996990378270787\\
302	0.00996990268727253\\
303	0.00996990157242509\\
304	0.00996990043781826\\
305	0.00996989928309867\\
306	0.00996989810790679\\
307	0.0099698969118768\\
308	0.00996989569463652\\
309	0.00996989445580732\\
310	0.009969893195004\\
311	0.00996989191183472\\
312	0.00996989060590088\\
313	0.00996988927679704\\
314	0.00996988792411083\\
315	0.00996988654742281\\
316	0.00996988514630644\\
317	0.00996988372032789\\
318	0.00996988226904603\\
319	0.00996988079201227\\
320	0.00996987928877048\\
321	0.00996987775885688\\
322	0.00996987620179994\\
323	0.00996987461712028\\
324	0.00996987300433055\\
325	0.00996987136293531\\
326	0.009969869692431\\
327	0.0099698679923057\\
328	0.00996986626203913\\
329	0.0099698645011025\\
330	0.00996986270895836\\
331	0.00996986088506052\\
332	0.00996985902885392\\
333	0.0099698571397745\\
334	0.00996985521724908\\
335	0.0099698532606952\\
336	0.00996985126952101\\
337	0.00996984924312513\\
338	0.00996984718089648\\
339	0.00996984508221414\\
340	0.00996984294644718\\
341	0.0099698407729545\\
342	0.00996983856108466\\
343	0.00996983631017565\\
344	0.00996983401955474\\
345	0.00996983168853827\\
346	0.00996982931643136\\
347	0.00996982690252778\\
348	0.00996982444610959\\
349	0.00996982194644695\\
350	0.00996981940279781\\
351	0.00996981681440757\\
352	0.00996981418050881\\
353	0.0099698115003209\\
354	0.00996980877304962\\
355	0.00996980599788678\\
356	0.00996980317400977\\
357	0.00996980030058113\\
358	0.00996979737674798\\
359	0.00996979440164157\\
360	0.00996979137437667\\
361	0.00996978829405096\\
362	0.00996978515974438\\
363	0.00996978197051844\\
364	0.00996977872541548\\
365	0.00996977542345787\\
366	0.00996977206364713\\
367	0.00996976864496313\\
368	0.00996976516636301\\
369	0.00996976162678025\\
370	0.00996975802512356\\
371	0.00996975436027575\\
372	0.00996975063109251\\
373	0.00996974683640113\\
374	0.0099697429749992\\
375	0.00996973904565309\\
376	0.00996973504709655\\
377	0.00996973097802906\\
378	0.00996972683711417\\
379	0.00996972262297775\\
380	0.00996971833420612\\
381	0.00996971396934407\\
382	0.00996970952689278\\
383	0.00996970500530764\\
384	0.00996970040299582\\
385	0.00996969571831386\\
386	0.0099696909495649\\
387	0.00996968609499584\\
388	0.0099696811527942\\
389	0.00996967612108474\\
390	0.0099696709979258\\
391	0.00996966578130518\\
392	0.00996966046913576\\
393	0.00996965505925043\\
394	0.00996964954939657\\
395	0.00996964393722969\\
396	0.00996963822030625\\
397	0.00996963239607535\\
398	0.00996962646186914\\
399	0.00996962041489166\\
400	0.00996961425220605\\
401	0.00996960797072057\\
402	0.00996960156717452\\
403	0.00996959503812491\\
404	0.00996958837992792\\
405	0.00996958158870555\\
406	0.00996957466032922\\
407	0.00996956759043277\\
408	0.00996956037439588\\
409	0.00996955300732745\\
410	0.00996954548404943\\
411	0.00996953779908189\\
412	0.00996952994662935\\
413	0.0099695219205671\\
414	0.00996951371442238\\
415	0.00996950532134258\\
416	0.00996949673405511\\
417	0.00996948794487979\\
418	0.00996947894579393\\
419	0.00996946972817297\\
420	0.00996946028268595\\
421	0.00996945059920828\\
422	0.00996944066672143\\
423	0.00996943047319725\\
424	0.0099694200054641\\
425	0.00996940924905151\\
426	0.00996939818800903\\
427	0.00996938680469331\\
428	0.00996937507951417\\
429	0.00996936299062253\\
430	0.00996935051350271\\
431	0.00996933762037826\\
432	0.00996932427919697\\
433	0.00996931045158034\\
434	0.00996929608813639\\
435	0.00996928111709898\\
436	0.00996926541677184\\
437	0.00996924875236517\\
438	0.00996923065227521\\
439	0.00996921027240247\\
440	0.00996914722470959\\
441	0.00996905323791148\\
442	0.00996895700353201\\
443	0.00996885844626655\\
444	0.00996875748756669\\
445	0.00996865404562082\\
446	0.00996854803539017\\
447	0.00996843936872157\\
448	0.00996832795456464\\
449	0.0099682136993294\\
450	0.00996809650743038\\
451	0.00996797628207344\\
452	0.0099678529263493\\
453	0.00996772634469452\\
454	0.00996759644474977\\
455	0.00996746313955286\\
456	0.0099673263497839\\
457	0.00996718600542639\\
458	0.00996704204584876\\
459	0.00996689441745486\\
460	0.00996674306799945\\
461	0.0099665879168168\\
462	0.00996642858931903\\
463	0.00996626287390056\\
464	0.00996608086039723\\
465	0.00996560070924794\\
466	0.00996511141802948\\
467	0.00996461271605404\\
468	0.00996410431613691\\
469	0.00996358591377835\\
470	0.00996305718637086\\
471	0.00996251779086151\\
472	0.00996196736363268\\
473	0.00996140552254573\\
474	0.00996083186701568\\
475	0.0099602459774223\\
476	0.00995964741248944\\
477	0.0099590357087881\\
478	0.00995841039055374\\
479	0.00995777095424221\\
480	0.00995711686219497\\
481	0.00995644753924431\\
482	0.00995576236905725\\
483	0.00995506068982423\\
484	0.00995434178917141\\
485	0.00995360489833252\\
486	0.0099528491857619\\
487	0.00995207374943311\\
488	0.00995127760706633\\
489	0.00995045968546667\\
490	0.00994961880800696\\
491	0.00994875367986187\\
492	0.00994786287049847\\
493	0.00994694479278383\\
494	0.00994599767784186\\
495	0.00994501954436793\\
496	0.00994400816019456\\
497	0.00994296099166491\\
498	0.00994187513049822\\
499	0.00994074717187218\\
500	0.00993957297357205\\
501	0.00993834710585245\\
502	0.00993706147747492\\
503	0.00993570179724759\\
504	0.00993248398813303\\
505	0.00992458450554222\\
506	0.00991413653284674\\
507	0.00990352052705566\\
508	0.00989272929216558\\
509	0.00988175512175933\\
510	0.00987058941679399\\
511	0.0098592228462381\\
512	0.00984764596295923\\
513	0.00983584888235029\\
514	0.00982382077557695\\
515	0.00981154976657298\\
516	0.00979902281971108\\
517	0.00978622561948705\\
518	0.00977314244811886\\
519	0.00975975608224493\\
520	0.0097460477821495\\
521	0.00973199762945084\\
522	0.00971758611197099\\
523	0.00970280013215884\\
524	0.0096879112403344\\
525	0.00967331433680058\\
526	0.00965827364290924\\
527	0.0096427331866361\\
528	0.00962661829896217\\
529	0.00960982055197028\\
530	0.00953928920580111\\
531	0.009381629884998\\
532	0.00921843775153357\\
533	0.00904940266084434\\
534	0.00887420844912711\\
535	0.00869254825095905\\
536	0.00850414847764258\\
537	0.00831267373605276\\
538	0.00811927026652865\\
539	0.00791867309489294\\
540	0.00771042776501912\\
541	0.00749378093925573\\
542	0.00726785763693795\\
543	0.00703163999036617\\
544	0.00678393139256419\\
545	0.00652326853684657\\
546	0.00624786368936832\\
547	0.00595554554291631\\
548	0.00564387510515887\\
549	0.00532233053630014\\
550	0.00517288644773442\\
551	0.00500843477872852\\
552	0.00483831416493209\\
553	0.00466529191159948\\
554	0.00448978070172018\\
555	0.00431233796658526\\
556	0.00413367673565331\\
557	0.00395483603638486\\
558	0.00377714582892456\\
559	0.00360215714364037\\
560	0.00343164512815057\\
561	0.00326867563951183\\
562	0.00311755320915099\\
563	0.00297470414778135\\
564	0.00283388527610961\\
565	0.0026973744658845\\
566	0.00255897925763545\\
567	0.00241906878442113\\
568	0.002278315210554\\
569	0.00213766925746306\\
570	0.00199855167123961\\
571	0.00186534421375662\\
572	0.00173829019138898\\
573	0.00161722417034892\\
574	0.00150255682998357\\
575	0.00139486353128288\\
576	0.00129406798244129\\
577	0.0011966074677817\\
578	0.001102369942485\\
579	0.00101118175468484\\
580	0.00092283933127548\\
581	0.000838824294993086\\
582	0.000758927142791253\\
583	0.000683310297993498\\
584	0.00061327566441274\\
585	0.000548715386105192\\
586	0.000488688640616261\\
587	0.00043319358073894\\
588	0.000381096578785599\\
589	0.000332240260312541\\
590	0.000286210565888725\\
591	0.000242202348687757\\
592	0.000200085409992\\
593	0.000159745073714005\\
594	0.000121173935986304\\
595	8.45520570083912e-05\\
596	5.05092148680373e-05\\
597	2.07908715710836e-05\\
598	0\\
599	0\\
600	0\\
};
\addplot [color=mycolor10,solid,forget plot]
  table[row sep=crcr]{%
1	0.00997084548698741\\
2	0.00997084548691379\\
3	0.00997084548683892\\
4	0.0099708454867628\\
5	0.00997084548668539\\
6	0.00997084548660669\\
7	0.00997084548652666\\
8	0.00997084548644529\\
9	0.00997084548636254\\
10	0.00997084548627841\\
11	0.00997084548619286\\
12	0.00997084548610587\\
13	0.00997084548601742\\
14	0.00997084548592748\\
15	0.00997084548583603\\
16	0.00997084548574304\\
17	0.00997084548564849\\
18	0.00997084548555234\\
19	0.00997084548545458\\
20	0.00997084548535518\\
21	0.0099708454852541\\
22	0.00997084548515133\\
23	0.00997084548504682\\
24	0.00997084548494056\\
25	0.0099708454848325\\
26	0.00997084548472264\\
27	0.00997084548461093\\
28	0.00997084548449733\\
29	0.00997084548438182\\
30	0.00997084548426437\\
31	0.00997084548414495\\
32	0.00997084548402352\\
33	0.00997084548390004\\
34	0.00997084548377448\\
35	0.00997084548364682\\
36	0.009970845483517\\
37	0.00997084548338501\\
38	0.00997084548325079\\
39	0.00997084548311431\\
40	0.00997084548297553\\
41	0.00997084548283442\\
42	0.00997084548269094\\
43	0.00997084548254504\\
44	0.00997084548239668\\
45	0.00997084548224583\\
46	0.00997084548209243\\
47	0.00997084548193646\\
48	0.00997084548177786\\
49	0.00997084548161659\\
50	0.0099708454814526\\
51	0.00997084548128586\\
52	0.0099708454811163\\
53	0.00997084548094389\\
54	0.00997084548076858\\
55	0.00997084548059031\\
56	0.00997084548040904\\
57	0.00997084548022471\\
58	0.00997084548003728\\
59	0.00997084547984669\\
60	0.00997084547965289\\
61	0.00997084547945583\\
62	0.00997084547925545\\
63	0.00997084547905168\\
64	0.00997084547884449\\
65	0.00997084547863379\\
66	0.00997084547841955\\
67	0.0099708454782017\\
68	0.00997084547798016\\
69	0.0099708454777549\\
70	0.00997084547752584\\
71	0.00997084547729291\\
72	0.00997084547705604\\
73	0.00997084547681519\\
74	0.00997084547657027\\
75	0.00997084547632122\\
76	0.00997084547606796\\
77	0.00997084547581043\\
78	0.00997084547554855\\
79	0.00997084547528224\\
80	0.00997084547501144\\
81	0.00997084547473606\\
82	0.00997084547445604\\
83	0.00997084547417127\\
84	0.0099708454738817\\
85	0.00997084547358724\\
86	0.00997084547328779\\
87	0.00997084547298328\\
88	0.00997084547267362\\
89	0.00997084547235873\\
90	0.00997084547203851\\
91	0.00997084547171287\\
92	0.00997084547138171\\
93	0.00997084547104496\\
94	0.0099708454707025\\
95	0.00997084547035424\\
96	0.00997084547000008\\
97	0.00997084546963992\\
98	0.00997084546927367\\
99	0.0099708454689012\\
100	0.00997084546852242\\
101	0.00997084546813722\\
102	0.00997084546774548\\
103	0.0099708454673471\\
104	0.00997084546694196\\
105	0.00997084546652994\\
106	0.00997084546611093\\
107	0.00997084546568481\\
108	0.00997084546525145\\
109	0.00997084546481072\\
110	0.00997084546436251\\
111	0.00997084546390668\\
112	0.0099708454634431\\
113	0.00997084546297163\\
114	0.00997084546249215\\
115	0.00997084546200451\\
116	0.00997084546150856\\
117	0.00997084546100418\\
118	0.0099708454604912\\
119	0.00997084545996949\\
120	0.00997084545943889\\
121	0.00997084545889924\\
122	0.00997084545835039\\
123	0.00997084545779219\\
124	0.00997084545722446\\
125	0.00997084545664704\\
126	0.00997084545605977\\
127	0.00997084545546246\\
128	0.00997084545485496\\
129	0.00997084545423708\\
130	0.00997084545360864\\
131	0.00997084545296945\\
132	0.00997084545231933\\
133	0.0099708454516581\\
134	0.00997084545098555\\
135	0.00997084545030149\\
136	0.00997084544960571\\
137	0.00997084544889802\\
138	0.0099708454481782\\
139	0.00997084544744605\\
140	0.00997084544670135\\
141	0.00997084544594388\\
142	0.00997084544517342\\
143	0.00997084544438973\\
144	0.0099708454435926\\
145	0.00997084544278178\\
146	0.00997084544195704\\
147	0.00997084544111813\\
148	0.00997084544026481\\
149	0.00997084543939683\\
150	0.00997084543851392\\
151	0.00997084543761583\\
152	0.00997084543670229\\
153	0.00997084543577304\\
154	0.00997084543482779\\
155	0.00997084543386628\\
156	0.00997084543288821\\
157	0.0099708454318933\\
158	0.00997084543088126\\
159	0.00997084542985178\\
160	0.00997084542880455\\
161	0.00997084542773929\\
162	0.00997084542665566\\
163	0.00997084542555335\\
164	0.00997084542443202\\
165	0.00997084542329137\\
166	0.00997084542213103\\
167	0.00997084542095068\\
168	0.00997084541974996\\
169	0.00997084541852853\\
170	0.00997084541728601\\
171	0.00997084541602204\\
172	0.00997084541473626\\
173	0.00997084541342828\\
174	0.00997084541209771\\
175	0.00997084541074417\\
176	0.00997084540936725\\
177	0.00997084540796655\\
178	0.00997084540654165\\
179	0.00997084540509214\\
180	0.00997084540361759\\
181	0.00997084540211755\\
182	0.00997084540059161\\
183	0.00997084539903929\\
184	0.00997084539746014\\
185	0.0099708453958537\\
186	0.00997084539421949\\
187	0.00997084539255703\\
188	0.00997084539086584\\
189	0.00997084538914539\\
190	0.0099708453873952\\
191	0.00997084538561475\\
192	0.0099708453838035\\
193	0.00997084538196092\\
194	0.00997084538008647\\
195	0.00997084537817959\\
196	0.00997084537623971\\
197	0.00997084537426627\\
198	0.00997084537225868\\
199	0.00997084537021633\\
200	0.00997084536813864\\
201	0.00997084536602497\\
202	0.00997084536387471\\
203	0.00997084536168721\\
204	0.00997084535946183\\
205	0.0099708453571979\\
206	0.00997084535489475\\
207	0.0099708453525517\\
208	0.00997084535016805\\
209	0.00997084534774309\\
210	0.00997084534527609\\
211	0.00997084534276634\\
212	0.00997084534021306\\
213	0.00997084533761551\\
214	0.00997084533497291\\
215	0.00997084533228447\\
216	0.00997084532954939\\
217	0.00997084532676685\\
218	0.00997084532393602\\
219	0.00997084532105606\\
220	0.00997084531812609\\
221	0.00997084531514526\\
222	0.00997084531211265\\
223	0.00997084530902738\\
224	0.0099708453058885\\
225	0.00997084530269509\\
226	0.00997084529944617\\
227	0.00997084529614078\\
228	0.00997084529277792\\
229	0.00997084528935659\\
230	0.00997084528587576\\
231	0.00997084528233437\\
232	0.00997084527873137\\
233	0.00997084527506566\\
234	0.00997084527133614\\
235	0.0099708452675417\\
236	0.00997084526368117\\
237	0.00997084525975341\\
238	0.00997084525575722\\
239	0.0099708452516914\\
240	0.00997084524755472\\
241	0.00997084524334592\\
242	0.00997084523906373\\
243	0.00997084523470686\\
244	0.00997084523027398\\
245	0.00997084522576376\\
246	0.00997084522117482\\
247	0.00997084521650577\\
248	0.00997084521175519\\
249	0.00997084520692164\\
250	0.00997084520200365\\
251	0.00997084519699972\\
252	0.00997084519190834\\
253	0.00997084518672794\\
254	0.00997084518145695\\
255	0.00997084517609377\\
256	0.00997084517063675\\
257	0.00997084516508423\\
258	0.00997084515943451\\
259	0.00997084515368588\\
260	0.00997084514783656\\
261	0.00997084514188476\\
262	0.00997084513582868\\
263	0.00997084512966644\\
264	0.00997084512339617\\
265	0.00997084511701593\\
266	0.00997084511052378\\
267	0.00997084510391771\\
268	0.0099708450971957\\
269	0.00997084509035568\\
270	0.00997084508339555\\
271	0.00997084507631316\\
272	0.00997084506910634\\
273	0.00997084506177287\\
274	0.00997084505431048\\
275	0.00997084504671688\\
276	0.00997084503898972\\
277	0.00997084503112663\\
278	0.00997084502312516\\
279	0.00997084501498286\\
280	0.0099708450066972\\
281	0.00997084499826563\\
282	0.00997084498968554\\
283	0.00997084498095427\\
284	0.00997084497206913\\
285	0.00997084496302736\\
286	0.00997084495382617\\
287	0.00997084494446271\\
288	0.00997084493493408\\
289	0.00997084492523733\\
290	0.00997084491536945\\
291	0.00997084490532739\\
292	0.00997084489510803\\
293	0.00997084488470822\\
294	0.00997084487412471\\
295	0.00997084486335425\\
296	0.00997084485239349\\
297	0.00997084484123903\\
298	0.00997084482988741\\
299	0.00997084481833513\\
300	0.0099708448065786\\
301	0.00997084479461418\\
302	0.00997084478243817\\
303	0.00997084477004681\\
304	0.00997084475743624\\
305	0.00997084474460259\\
306	0.00997084473154187\\
307	0.00997084471825006\\
308	0.00997084470472304\\
309	0.00997084469095665\\
310	0.00997084467694664\\
311	0.00997084466268868\\
312	0.00997084464817839\\
313	0.00997084463341131\\
314	0.00997084461838288\\
315	0.00997084460308849\\
316	0.00997084458752345\\
317	0.00997084457168299\\
318	0.00997084455556225\\
319	0.00997084453915629\\
320	0.00997084452246011\\
321	0.0099708445054686\\
322	0.00997084448817659\\
323	0.00997084447057881\\
324	0.0099708444526699\\
325	0.00997084443444444\\
326	0.00997084441589689\\
327	0.00997084439702164\\
328	0.00997084437781298\\
329	0.00997084435826512\\
330	0.00997084433837215\\
331	0.00997084431812812\\
332	0.00997084429752692\\
333	0.00997084427656238\\
334	0.00997084425522824\\
335	0.00997084423351811\\
336	0.00997084421142552\\
337	0.00997084418894389\\
338	0.00997084416606655\\
339	0.00997084414278671\\
340	0.00997084411909746\\
341	0.00997084409499181\\
342	0.00997084407046265\\
343	0.00997084404550273\\
344	0.00997084402010473\\
345	0.00997084399426118\\
346	0.00997084396796449\\
347	0.00997084394120695\\
348	0.00997084391398074\\
349	0.0099708438862779\\
350	0.00997084385809033\\
351	0.0099708438294098\\
352	0.00997084380022794\\
353	0.00997084377053624\\
354	0.00997084374032603\\
355	0.00997084370958851\\
356	0.0099708436783147\\
357	0.00997084364649547\\
358	0.0099708436141215\\
359	0.00997084358118333\\
360	0.0099708435476713\\
361	0.00997084351357554\\
362	0.00997084347888602\\
363	0.00997084344359248\\
364	0.00997084340768447\\
365	0.0099708433711513\\
366	0.00997084333398205\\
367	0.00997084329616555\\
368	0.00997084325769042\\
369	0.00997084321854495\\
370	0.0099708431787172\\
371	0.00997084313819494\\
372	0.0099708430969656\\
373	0.00997084305501633\\
374	0.00997084301233393\\
375	0.00997084296890484\\
376	0.00997084292471516\\
377	0.00997084287975059\\
378	0.00997084283399642\\
379	0.00997084278743753\\
380	0.00997084274005834\\
381	0.00997084269184282\\
382	0.00997084264277444\\
383	0.00997084259283615\\
384	0.00997084254201035\\
385	0.00997084249027889\\
386	0.00997084243762299\\
387	0.00997084238402323\\
388	0.00997084232945953\\
389	0.00997084227391108\\
390	0.00997084221735631\\
391	0.00997084215977281\\
392	0.00997084210113733\\
393	0.00997084204142566\\
394	0.00997084198061258\\
395	0.00997084191867175\\
396	0.00997084185557563\\
397	0.00997084179129535\\
398	0.00997084172580054\\
399	0.0099708416590592\\
400	0.0099708415910376\\
401	0.00997084152170017\\
402	0.00997084145100952\\
403	0.00997084137892621\\
404	0.00997084130540826\\
405	0.00997084123041089\\
406	0.0099708411538868\\
407	0.00997084107578596\\
408	0.00997084099605557\\
409	0.00997084091463992\\
410	0.00997084083148032\\
411	0.00997084074651508\\
412	0.0099708406596792\\
413	0.00997084057090389\\
414	0.00997084048011545\\
415	0.00997084038723444\\
416	0.00997084029217644\\
417	0.00997084019485359\\
418	0.00997084009517133\\
419	0.00997083999302708\\
420	0.00997083988830929\\
421	0.0099708397808962\\
422	0.00997083967065452\\
423	0.00997083955743784\\
424	0.00997083944108465\\
425	0.00997083932141605\\
426	0.00997083919823265\\
427	0.00997083907131013\\
428	0.00997083894039184\\
429	0.0099708388051747\\
430	0.00997083866527911\\
431	0.00997083852018113\\
432	0.00997083836905726\\
433	0.0099708382104362\\
434	0.00997083804145571\\
435	0.00997083785641904\\
436	0.00997083764446549\\
437	0.0099708373873598\\
438	0.00997083706262313\\
439	0.00997083666448949\\
440	0.00997083623128341\\
441	0.00997083578821133\\
442	0.00997083533497104\\
443	0.00997083487124972\\
444	0.00997083439672429\\
445	0.00997083391106196\\
446	0.00997083341392111\\
447	0.0099708329049525\\
448	0.00997083238380088\\
449	0.0099708318501067\\
450	0.00997083130350745\\
451	0.00997083074363683\\
452	0.00997083017011757\\
453	0.00997082958253852\\
454	0.00997082898039505\\
455	0.00997082836294783\\
456	0.00997082772890378\\
457	0.00997082707570273\\
458	0.00997082639786802\\
459	0.00997082568288892\\
460	0.0099708249000621\\
461	0.00997082396978854\\
462	0.00997082269080592\\
463	0.00997082061716523\\
464	0.00997081682871155\\
465	0.00997079688302175\\
466	0.00997077655963949\\
467	0.00997075584701171\\
468	0.00997073473290042\\
469	0.00997071320433725\\
470	0.00997069124754055\\
471	0.00997066884793057\\
472	0.00997064599020059\\
473	0.00997062265830179\\
474	0.00997059883539591\\
475	0.00997057450379251\\
476	0.00997054964505322\\
477	0.00997052424024599\\
478	0.00997049826936458\\
479	0.00997047171110278\\
480	0.00997044454273607\\
481	0.00997041673999051\\
482	0.00997038827689002\\
483	0.0099703591255796\\
484	0.0099703292561263\\
485	0.00997029863629874\\
486	0.00997026723130749\\
487	0.00997023500349532\\
488	0.00997020191199549\\
489	0.00997016791233528\\
490	0.0099701329559731\\
491	0.00997009698975231\\
492	0.00997005995524691\\
493	0.0099700217879548\\
494	0.00996998241625147\\
495	0.00996994175991042\\
496	0.00996989972773478\\
497	0.00996985621320054\\
498	0.00996981108546361\\
499	0.00996976416950573\\
500	0.00996971520147044\\
501	0.0099696637305697\\
502	0.00996960891877847\\
503	0.00996954919283904\\
504	0.00996948186685571\\
505	0.00996929001369385\\
506	0.00996896204731497\\
507	0.0099686269049131\\
508	0.00996828423231784\\
509	0.00996793364018603\\
510	0.0099675747004568\\
511	0.00996720696758362\\
512	0.00996682997231482\\
513	0.00996644320053974\\
514	0.00996604608795879\\
515	0.00996563801379749\\
516	0.00996521829280044\\
517	0.00996478616328325\\
518	0.00996434076440996\\
519	0.00996388108095988\\
520	0.0099634057846906\\
521	0.00996291273618887\\
522	0.00996239734618991\\
523	0.0099618470340882\\
524	0.00996098584201838\\
525	0.00995942607081101\\
526	0.00995781208910305\\
527	0.00995613837293259\\
528	0.00995439742492012\\
529	0.00995257893419314\\
530	0.00994806151716236\\
531	0.00993936864397242\\
532	0.00993061374323875\\
533	0.00992179613327257\\
534	0.00991291338686796\\
535	0.00990395593179488\\
536	0.0098948917788269\\
537	0.00988209889720862\\
538	0.00986432247477629\\
539	0.00984622759948875\\
540	0.00982778327047255\\
541	0.00980895423599394\\
542	0.00978970060060886\\
543	0.00976997744200252\\
544	0.00974973268212212\\
545	0.00972890408895865\\
546	0.00970741598062481\\
547	0.00968517836419248\\
548	0.00966207204382259\\
549	0.00962617862383391\\
550	0.00940147602621652\\
551	0.00916775487436897\\
552	0.00892443714625995\\
553	0.0086706446418252\\
554	0.00840532921897203\\
555	0.00812729915250492\\
556	0.00783671321482161\\
557	0.00753063773729663\\
558	0.0072069709116426\\
559	0.00686361645109179\\
560	0.00650839030900701\\
561	0.00613192561547077\\
562	0.00572661969500788\\
563	0.0052961591784016\\
564	0.00484649400971892\\
565	0.00437548564659083\\
566	0.00415383614871475\\
567	0.00393383894687057\\
568	0.00371213006982874\\
569	0.00349020085871587\\
570	0.00326984259083614\\
571	0.00305370983431872\\
572	0.00284548539780111\\
573	0.00265016852435452\\
574	0.00245848661210091\\
575	0.00226662545412475\\
576	0.00207713061461257\\
577	0.00189587208535604\\
578	0.00172554605519975\\
579	0.00156853045100365\\
580	0.00142575417691301\\
581	0.00129100157684195\\
582	0.00116453672973894\\
583	0.00104320185244136\\
584	0.000925111851658451\\
585	0.000811581194832413\\
586	0.0007041690715661\\
587	0.000603390290013832\\
588	0.000510932493051772\\
589	0.00042719620397733\\
590	0.00035239839743971\\
591	0.000285859088514837\\
592	0.00022641344495182\\
593	0.000173745915387586\\
594	0.00012714455469262\\
595	8.61467099905492e-05\\
596	5.05092148680371e-05\\
597	2.07908715710836e-05\\
598	0\\
599	0\\
600	0\\
};
\addplot [color=mycolor11,solid,forget plot]
  table[row sep=crcr]{%
1	0.00997158250207167\\
2	0.00997158250206864\\
3	0.00997158250206557\\
4	0.00997158250206244\\
5	0.00997158250205925\\
6	0.00997158250205602\\
7	0.00997158250205273\\
8	0.00997158250204938\\
9	0.00997158250204598\\
10	0.00997158250204253\\
11	0.00997158250203901\\
12	0.00997158250203543\\
13	0.0099715825020318\\
14	0.0099715825020281\\
15	0.00997158250202434\\
16	0.00997158250202052\\
17	0.00997158250201663\\
18	0.00997158250201268\\
19	0.00997158250200866\\
20	0.00997158250200458\\
21	0.00997158250200042\\
22	0.0099715825019962\\
23	0.0099715825019919\\
24	0.00997158250198753\\
25	0.00997158250198309\\
26	0.00997158250197857\\
27	0.00997158250197398\\
28	0.00997158250196931\\
29	0.00997158250196457\\
30	0.00997158250195974\\
31	0.00997158250195483\\
32	0.00997158250194984\\
33	0.00997158250194476\\
34	0.0099715825019396\\
35	0.00997158250193435\\
36	0.00997158250192902\\
37	0.00997158250192359\\
38	0.00997158250191807\\
39	0.00997158250191246\\
40	0.00997158250190676\\
41	0.00997158250190096\\
42	0.00997158250189506\\
43	0.00997158250188906\\
44	0.00997158250188296\\
45	0.00997158250187676\\
46	0.00997158250187046\\
47	0.00997158250186404\\
48	0.00997158250185753\\
49	0.0099715825018509\\
50	0.00997158250184415\\
51	0.0099715825018373\\
52	0.00997158250183033\\
53	0.00997158250182324\\
54	0.00997158250181604\\
55	0.00997158250180871\\
56	0.00997158250180125\\
57	0.00997158250179368\\
58	0.00997158250178598\\
59	0.00997158250177814\\
60	0.00997158250177017\\
61	0.00997158250176207\\
62	0.00997158250175384\\
63	0.00997158250174546\\
64	0.00997158250173694\\
65	0.00997158250172828\\
66	0.00997158250171947\\
67	0.00997158250171052\\
68	0.00997158250170141\\
69	0.00997158250169215\\
70	0.00997158250168273\\
71	0.00997158250167316\\
72	0.00997158250166342\\
73	0.00997158250165352\\
74	0.00997158250164345\\
75	0.00997158250163321\\
76	0.0099715825016228\\
77	0.00997158250161221\\
78	0.00997158250160145\\
79	0.0099715825015905\\
80	0.00997158250157936\\
81	0.00997158250156805\\
82	0.00997158250155653\\
83	0.00997158250154483\\
84	0.00997158250153292\\
85	0.00997158250152082\\
86	0.0099715825015085\\
87	0.00997158250149599\\
88	0.00997158250148326\\
89	0.00997158250147031\\
90	0.00997158250145714\\
91	0.00997158250144376\\
92	0.00997158250143014\\
93	0.0099715825014163\\
94	0.00997158250140222\\
95	0.0099715825013879\\
96	0.00997158250137334\\
97	0.00997158250135853\\
98	0.00997158250134347\\
99	0.00997158250132816\\
100	0.00997158250131259\\
101	0.00997158250129675\\
102	0.00997158250128064\\
103	0.00997158250126427\\
104	0.00997158250124761\\
105	0.00997158250123067\\
106	0.00997158250121344\\
107	0.00997158250119592\\
108	0.0099715825011781\\
109	0.00997158250115998\\
110	0.00997158250114155\\
111	0.00997158250112281\\
112	0.00997158250110375\\
113	0.00997158250108436\\
114	0.00997158250106465\\
115	0.0099715825010446\\
116	0.0099715825010242\\
117	0.00997158250100346\\
118	0.00997158250098237\\
119	0.00997158250096092\\
120	0.0099715825009391\\
121	0.00997158250091691\\
122	0.00997158250089434\\
123	0.00997158250087139\\
124	0.00997158250084804\\
125	0.0099715825008243\\
126	0.00997158250080015\\
127	0.00997158250077559\\
128	0.0099715825007506\\
129	0.0099715825007252\\
130	0.00997158250069935\\
131	0.00997158250067307\\
132	0.00997158250064633\\
133	0.00997158250061914\\
134	0.00997158250059148\\
135	0.00997158250056335\\
136	0.00997158250053474\\
137	0.00997158250050563\\
138	0.00997158250047603\\
139	0.00997158250044592\\
140	0.0099715825004153\\
141	0.00997158250038414\\
142	0.00997158250035246\\
143	0.00997158250032023\\
144	0.00997158250028744\\
145	0.0099715825002541\\
146	0.00997158250022018\\
147	0.00997158250018568\\
148	0.00997158250015058\\
149	0.00997158250011488\\
150	0.00997158250007857\\
151	0.00997158250004163\\
152	0.00997158250000406\\
153	0.00997158249996584\\
154	0.00997158249992696\\
155	0.00997158249988742\\
156	0.00997158249984719\\
157	0.00997158249980627\\
158	0.00997158249976464\\
159	0.0099715824997223\\
160	0.00997158249967923\\
161	0.00997158249963541\\
162	0.00997158249959084\\
163	0.0099715824995455\\
164	0.00997158249949938\\
165	0.00997158249945247\\
166	0.00997158249940474\\
167	0.00997158249935619\\
168	0.0099715824993068\\
169	0.00997158249925656\\
170	0.00997158249920546\\
171	0.00997158249915346\\
172	0.00997158249910057\\
173	0.00997158249904678\\
174	0.00997158249899205\\
175	0.00997158249893637\\
176	0.00997158249887974\\
177	0.00997158249882212\\
178	0.00997158249876351\\
179	0.00997158249870389\\
180	0.00997158249864324\\
181	0.00997158249858154\\
182	0.00997158249851877\\
183	0.00997158249845492\\
184	0.00997158249838997\\
185	0.00997158249832389\\
186	0.00997158249825667\\
187	0.00997158249818829\\
188	0.00997158249811872\\
189	0.00997158249804795\\
190	0.00997158249797596\\
191	0.00997158249790273\\
192	0.00997158249782822\\
193	0.00997158249775243\\
194	0.00997158249767533\\
195	0.00997158249759689\\
196	0.00997158249751709\\
197	0.00997158249743592\\
198	0.00997158249735334\\
199	0.00997158249726933\\
200	0.00997158249718386\\
201	0.00997158249709692\\
202	0.00997158249700847\\
203	0.00997158249691849\\
204	0.00997158249682695\\
205	0.00997158249673382\\
206	0.00997158249663908\\
207	0.0099715824965427\\
208	0.00997158249644465\\
209	0.00997158249634489\\
210	0.00997158249624341\\
211	0.00997158249614017\\
212	0.00997158249603514\\
213	0.00997158249592829\\
214	0.00997158249581958\\
215	0.009971582495709\\
216	0.00997158249559648\\
217	0.00997158249548202\\
218	0.00997158249536557\\
219	0.0099715824952471\\
220	0.00997158249512657\\
221	0.00997158249500395\\
222	0.0099715824948792\\
223	0.00997158249475227\\
224	0.00997158249462315\\
225	0.00997158249449178\\
226	0.00997158249435813\\
227	0.00997158249422215\\
228	0.00997158249408381\\
229	0.00997158249394306\\
230	0.00997158249379987\\
231	0.00997158249365418\\
232	0.00997158249350596\\
233	0.00997158249335515\\
234	0.00997158249320173\\
235	0.00997158249304563\\
236	0.00997158249288681\\
237	0.00997158249272522\\
238	0.00997158249256082\\
239	0.00997158249239355\\
240	0.00997158249222337\\
241	0.00997158249205022\\
242	0.00997158249187405\\
243	0.00997158249169481\\
244	0.00997158249151243\\
245	0.00997158249132688\\
246	0.00997158249113808\\
247	0.00997158249094599\\
248	0.00997158249075055\\
249	0.00997158249055169\\
250	0.00997158249034935\\
251	0.00997158249014348\\
252	0.00997158248993401\\
253	0.00997158248972088\\
254	0.00997158248950401\\
255	0.00997158248928336\\
256	0.00997158248905884\\
257	0.00997158248883039\\
258	0.00997158248859794\\
259	0.00997158248836142\\
260	0.00997158248812076\\
261	0.00997158248787588\\
262	0.00997158248762671\\
263	0.00997158248737317\\
264	0.00997158248711518\\
265	0.00997158248685267\\
266	0.00997158248658555\\
267	0.00997158248631374\\
268	0.00997158248603716\\
269	0.00997158248575573\\
270	0.00997158248546934\\
271	0.00997158248517794\\
272	0.0099715824848814\\
273	0.00997158248457966\\
274	0.00997158248427261\\
275	0.00997158248396016\\
276	0.00997158248364221\\
277	0.00997158248331867\\
278	0.00997158248298943\\
279	0.0099715824826544\\
280	0.00997158248231347\\
281	0.00997158248196653\\
282	0.00997158248161348\\
283	0.00997158248125421\\
284	0.0099715824808886\\
285	0.00997158248051655\\
286	0.00997158248013794\\
287	0.00997158247975265\\
288	0.00997158247936056\\
289	0.00997158247896156\\
290	0.00997158247855551\\
291	0.0099715824781423\\
292	0.00997158247772179\\
293	0.00997158247729384\\
294	0.00997158247685835\\
295	0.00997158247641516\\
296	0.00997158247596413\\
297	0.00997158247550514\\
298	0.00997158247503804\\
299	0.00997158247456268\\
300	0.00997158247407891\\
301	0.00997158247358659\\
302	0.00997158247308556\\
303	0.00997158247257567\\
304	0.00997158247205676\\
305	0.00997158247152868\\
306	0.00997158247099126\\
307	0.00997158247044432\\
308	0.00997158246988771\\
309	0.00997158246932125\\
310	0.00997158246874477\\
311	0.00997158246815809\\
312	0.00997158246756103\\
313	0.00997158246695341\\
314	0.00997158246633504\\
315	0.00997158246570573\\
316	0.00997158246506528\\
317	0.00997158246441351\\
318	0.00997158246375021\\
319	0.00997158246307519\\
320	0.00997158246238822\\
321	0.00997158246168912\\
322	0.00997158246097766\\
323	0.00997158246025362\\
324	0.0099715824595168\\
325	0.00997158245876695\\
326	0.00997158245800387\\
327	0.00997158245722731\\
328	0.00997158245643705\\
329	0.00997158245563285\\
330	0.00997158245481445\\
331	0.00997158245398163\\
332	0.00997158245313413\\
333	0.0099715824522717\\
334	0.00997158245139408\\
335	0.00997158245050101\\
336	0.00997158244959222\\
337	0.00997158244866744\\
338	0.00997158244772641\\
339	0.00997158244676884\\
340	0.00997158244579445\\
341	0.00997158244480295\\
342	0.00997158244379405\\
343	0.00997158244276747\\
344	0.00997158244172288\\
345	0.00997158244066\\
346	0.0099715824395785\\
347	0.00997158243847807\\
348	0.0099715824373584\\
349	0.00997158243621915\\
350	0.00997158243505999\\
351	0.00997158243388059\\
352	0.0099715824326806\\
353	0.00997158243145967\\
354	0.00997158243021745\\
355	0.00997158242895357\\
356	0.00997158242766766\\
357	0.00997158242635936\\
358	0.00997158242502828\\
359	0.00997158242367402\\
360	0.00997158242229619\\
361	0.00997158242089439\\
362	0.00997158241946819\\
363	0.00997158241801718\\
364	0.00997158241654094\\
365	0.009971582415039\\
366	0.00997158241351092\\
367	0.00997158241195625\\
368	0.0099715824103745\\
369	0.00997158240876521\\
370	0.00997158240712786\\
371	0.00997158240546195\\
372	0.00997158240376696\\
373	0.00997158240204237\\
374	0.00997158240028761\\
375	0.00997158239850212\\
376	0.00997158239668534\\
377	0.00997158239483664\\
378	0.00997158239295544\\
379	0.00997158239104109\\
380	0.00997158238909294\\
381	0.00997158238711033\\
382	0.00997158238509255\\
383	0.00997158238303891\\
384	0.00997158238094865\\
385	0.00997158237882102\\
386	0.00997158237665522\\
387	0.00997158237445045\\
388	0.00997158237220584\\
389	0.00997158236992053\\
390	0.00997158236759361\\
391	0.00997158236522412\\
392	0.00997158236281107\\
393	0.00997158236035344\\
394	0.00997158235785015\\
395	0.00997158235530007\\
396	0.00997158235270203\\
397	0.00997158235005476\\
398	0.00997158234735696\\
399	0.00997158234460724\\
400	0.00997158234180415\\
401	0.00997158233894615\\
402	0.00997158233603162\\
403	0.00997158233305882\\
404	0.00997158233002592\\
405	0.00997158232693097\\
406	0.00997158232377192\\
407	0.00997158232054661\\
408	0.00997158231725276\\
409	0.00997158231388795\\
410	0.00997158231044967\\
411	0.00997158230693523\\
412	0.0099715823033418\\
413	0.00997158229966632\\
414	0.0099715822959055\\
415	0.0099715822920559\\
416	0.00997158228811384\\
417	0.00997158228407539\\
418	0.00997158227993627\\
419	0.00997158227569181\\
420	0.0099715822713369\\
421	0.00997158226686594\\
422	0.00997158226227277\\
423	0.00997158225755055\\
424	0.00997158225269167\\
425	0.00997158224768756\\
426	0.00997158224252837\\
427	0.00997158223720239\\
428	0.00997158223169493\\
429	0.00997158222598576\\
430	0.00997158222004395\\
431	0.00997158221381703\\
432	0.00997158220721015\\
433	0.00997158220004925\\
434	0.00997158219202694\\
435	0.0099715821826513\\
436	0.00997158217127756\\
437	0.00997158215739005\\
438	0.00997158214122857\\
439	0.00997158212401503\\
440	0.00997158210640971\\
441	0.00997158208840071\\
442	0.00997158206997567\\
443	0.00997158205112189\\
444	0.00997158203182631\\
445	0.00997158201207553\\
446	0.00997158199185592\\
447	0.00997158197115361\\
448	0.00997158194995454\\
449	0.00997158192824437\\
450	0.00997158190600811\\
451	0.00997158188322912\\
452	0.00997158185988651\\
453	0.00997158183594896\\
454	0.0099715818113606\\
455	0.00997158178600892\\
456	0.00997158175965185\\
457	0.00997158173175193\\
458	0.0099715817011006\\
459	0.00997158166499934\\
460	0.00997158161765242\\
461	0.00997158154774152\\
462	0.00997158143699461\\
463	0.00997158126622755\\
464	0.00997158104189281\\
465	0.00997158081343181\\
466	0.00997158058071974\\
467	0.00997158034362549\\
468	0.00997158010201132\\
469	0.00997157985573162\\
470	0.00997157960463319\\
471	0.00997157934855614\\
472	0.00997157908733262\\
473	0.0099715788207843\\
474	0.00997157854872028\\
475	0.0099715782709403\\
476	0.0099715779872406\\
477	0.00997157769740609\\
478	0.00997157740120733\\
479	0.00997157709839908\\
480	0.00997157678871876\\
481	0.00997157647188453\\
482	0.00997157614759318\\
483	0.00997157581551775\\
484	0.0099715754753051\\
485	0.00997157512657271\\
486	0.00997157476890483\\
487	0.00997157440184818\\
488	0.00997157402490682\\
489	0.00997157363753587\\
490	0.00997157323913343\\
491	0.00997157282902933\\
492	0.00997157240646753\\
493	0.00997157197057501\\
494	0.00997157152030013\\
495	0.00997157105428204\\
496	0.00997157057056529\\
497	0.00997157006597804\\
498	0.00997156953481639\\
499	0.00997156896621494\\
500	0.00997156833939104\\
501	0.00997156761655168\\
502	0.00997156673684214\\
503	0.00997156562584849\\
504	0.00997156425665817\\
505	0.00997156274604111\\
506	0.0099715612080005\\
507	0.00997155964131668\\
508	0.00997155804437759\\
509	0.00997155641517961\\
510	0.00997155475184757\\
511	0.00997155305262294\\
512	0.00997155131554636\\
513	0.00997154953838315\\
514	0.00997154771845011\\
515	0.00997154585217563\\
516	0.00997154393393788\\
517	0.00997154195295965\\
518	0.00997153988504464\\
519	0.00997153767096723\\
520	0.00997153516186568\\
521	0.00997153198982832\\
522	0.00997152729962883\\
523	0.00997151937369225\\
524	0.00997149318604075\\
525	0.00997142978152294\\
526	0.00997136434858027\\
527	0.00997129665617338\\
528	0.00997122642403624\\
529	0.00997115341161725\\
530	0.00997107758492663\\
531	0.00997099889826753\\
532	0.00997091691634657\\
533	0.00997083090031376\\
534	0.00997073943226779\\
535	0.00997063970133814\\
536	0.00997052669160612\\
537	0.0099702082159012\\
538	0.00996962328454001\\
539	0.00996902078936174\\
540	0.00996839922675705\\
541	0.00996775689666558\\
542	0.00996709187429563\\
543	0.00996640193383221\\
544	0.00996568437542882\\
545	0.00996493579503008\\
546	0.00996415169729546\\
547	0.00996332579684075\\
548	0.00996244843394521\\
549	0.00996092208856274\\
550	0.00995025740715425\\
551	0.00993968336057744\\
552	0.00992920641838787\\
553	0.00991882893957265\\
554	0.00990854606370902\\
555	0.00989832445380637\\
556	0.00988667401057027\\
557	0.00987483205160817\\
558	0.00986299957738175\\
559	0.00985111088440256\\
560	0.00982940086753967\\
561	0.00980373496254183\\
562	0.00977743457905058\\
563	0.00975058793269213\\
564	0.00972322653700024\\
565	0.00969510362383588\\
566	0.00940580285354155\\
567	0.00909740060809503\\
568	0.00877252040548564\\
569	0.00842896046466404\\
570	0.00806413126271723\\
571	0.00767497738991288\\
572	0.0072578893771046\\
573	0.00680862291380522\\
574	0.00633647419335779\\
575	0.00584531199636737\\
576	0.00533342520441997\\
577	0.00479868293791651\\
578	0.00423933523304903\\
579	0.00367550052531207\\
580	0.00317524304856927\\
581	0.00291649390778095\\
582	0.00266264787724582\\
583	0.0024212958895209\\
584	0.00219950504465807\\
585	0.00198101316635129\\
586	0.0017652741331012\\
587	0.00155340735392807\\
588	0.0013468096195542\\
589	0.00114664925084055\\
590	0.000951372357338311\\
591	0.000764343087165558\\
592	0.00058935041161929\\
593	0.000429546140184966\\
594	0.000289033664663948\\
595	0.000172252650723403\\
596	8.12550264159866e-05\\
597	2.07908715710836e-05\\
598	0\\
599	0\\
600	0\\
};
\addplot [color=mycolor12,solid,forget plot]
  table[row sep=crcr]{%
1	0.00998476612601138\\
2	0.00998476612601125\\
3	0.00998476612601112\\
4	0.00998476612601099\\
5	0.00998476612601085\\
6	0.00998476612601071\\
7	0.00998476612601057\\
8	0.00998476612601043\\
9	0.00998476612601028\\
10	0.00998476612601013\\
11	0.00998476612600998\\
12	0.00998476612600983\\
13	0.00998476612600967\\
14	0.00998476612600952\\
15	0.00998476612600935\\
16	0.00998476612600919\\
17	0.00998476612600903\\
18	0.00998476612600886\\
19	0.00998476612600869\\
20	0.00998476612600851\\
21	0.00998476612600833\\
22	0.00998476612600815\\
23	0.00998476612600797\\
24	0.00998476612600778\\
25	0.00998476612600759\\
26	0.0099847661260074\\
27	0.0099847661260072\\
28	0.009984766126007\\
29	0.0099847661260068\\
30	0.00998476612600659\\
31	0.00998476612600638\\
32	0.00998476612600617\\
33	0.00998476612600595\\
34	0.00998476612600573\\
35	0.00998476612600551\\
36	0.00998476612600528\\
37	0.00998476612600505\\
38	0.00998476612600481\\
39	0.00998476612600457\\
40	0.00998476612600433\\
41	0.00998476612600408\\
42	0.00998476612600383\\
43	0.00998476612600357\\
44	0.00998476612600331\\
45	0.00998476612600305\\
46	0.00998476612600278\\
47	0.0099847661260025\\
48	0.00998476612600223\\
49	0.00998476612600194\\
50	0.00998476612600165\\
51	0.00998476612600136\\
52	0.00998476612600106\\
53	0.00998476612600076\\
54	0.00998476612600045\\
55	0.00998476612600014\\
56	0.00998476612599982\\
57	0.0099847661259995\\
58	0.00998476612599917\\
59	0.00998476612599883\\
60	0.00998476612599849\\
61	0.00998476612599815\\
62	0.00998476612599779\\
63	0.00998476612599744\\
64	0.00998476612599707\\
65	0.0099847661259967\\
66	0.00998476612599633\\
67	0.00998476612599594\\
68	0.00998476612599555\\
69	0.00998476612599516\\
70	0.00998476612599476\\
71	0.00998476612599435\\
72	0.00998476612599393\\
73	0.00998476612599351\\
74	0.00998476612599307\\
75	0.00998476612599264\\
76	0.00998476612599219\\
77	0.00998476612599174\\
78	0.00998476612599128\\
79	0.00998476612599081\\
80	0.00998476612599034\\
81	0.00998476612598985\\
82	0.00998476612598936\\
83	0.00998476612598886\\
84	0.00998476612598835\\
85	0.00998476612598783\\
86	0.00998476612598731\\
87	0.00998476612598677\\
88	0.00998476612598623\\
89	0.00998476612598567\\
90	0.00998476612598511\\
91	0.00998476612598454\\
92	0.00998476612598396\\
93	0.00998476612598337\\
94	0.00998476612598276\\
95	0.00998476612598215\\
96	0.00998476612598153\\
97	0.00998476612598089\\
98	0.00998476612598025\\
99	0.0099847661259796\\
100	0.00998476612597893\\
101	0.00998476612597826\\
102	0.00998476612597757\\
103	0.00998476612597687\\
104	0.00998476612597616\\
105	0.00998476612597543\\
106	0.00998476612597469\\
107	0.00998476612597395\\
108	0.00998476612597318\\
109	0.00998476612597241\\
110	0.00998476612597162\\
111	0.00998476612597082\\
112	0.00998476612597\\
113	0.00998476612596918\\
114	0.00998476612596833\\
115	0.00998476612596747\\
116	0.0099847661259666\\
117	0.00998476612596572\\
118	0.00998476612596482\\
119	0.00998476612596389\\
120	0.00998476612596296\\
121	0.00998476612596201\\
122	0.00998476612596105\\
123	0.00998476612596007\\
124	0.00998476612595907\\
125	0.00998476612595805\\
126	0.00998476612595702\\
127	0.00998476612595597\\
128	0.0099847661259549\\
129	0.00998476612595382\\
130	0.00998476612595271\\
131	0.00998476612595159\\
132	0.00998476612595045\\
133	0.00998476612594928\\
134	0.0099847661259481\\
135	0.0099847661259469\\
136	0.00998476612594567\\
137	0.00998476612594443\\
138	0.00998476612594316\\
139	0.00998476612594187\\
140	0.00998476612594057\\
141	0.00998476612593923\\
142	0.00998476612593788\\
143	0.0099847661259365\\
144	0.0099847661259351\\
145	0.00998476612593367\\
146	0.00998476612593222\\
147	0.00998476612593075\\
148	0.00998476612592925\\
149	0.00998476612592772\\
150	0.00998476612592616\\
151	0.00998476612592459\\
152	0.00998476612592298\\
153	0.00998476612592135\\
154	0.00998476612591968\\
155	0.00998476612591799\\
156	0.00998476612591627\\
157	0.00998476612591452\\
158	0.00998476612591274\\
159	0.00998476612591093\\
160	0.00998476612590909\\
161	0.00998476612590721\\
162	0.00998476612590531\\
163	0.00998476612590336\\
164	0.0099847661259014\\
165	0.00998476612589939\\
166	0.00998476612589735\\
167	0.00998476612589527\\
168	0.00998476612589316\\
169	0.00998476612589101\\
170	0.00998476612588882\\
171	0.0099847661258866\\
172	0.00998476612588434\\
173	0.00998476612588204\\
174	0.0099847661258797\\
175	0.00998476612587731\\
176	0.00998476612587489\\
177	0.00998476612587243\\
178	0.00998476612586992\\
179	0.00998476612586737\\
180	0.00998476612586478\\
181	0.00998476612586214\\
182	0.00998476612585945\\
183	0.00998476612585672\\
184	0.00998476612585395\\
185	0.00998476612585112\\
186	0.00998476612584824\\
187	0.00998476612584532\\
188	0.00998476612584234\\
189	0.00998476612583932\\
190	0.00998476612583624\\
191	0.00998476612583311\\
192	0.00998476612582992\\
193	0.00998476612582668\\
194	0.00998476612582338\\
195	0.00998476612582003\\
196	0.00998476612581661\\
197	0.00998476612581314\\
198	0.00998476612580961\\
199	0.00998476612580602\\
200	0.00998476612580236\\
201	0.00998476612579864\\
202	0.00998476612579486\\
203	0.00998476612579101\\
204	0.00998476612578709\\
205	0.00998476612578311\\
206	0.00998476612577906\\
207	0.00998476612577493\\
208	0.00998476612577074\\
209	0.00998476612576647\\
210	0.00998476612576213\\
211	0.00998476612575772\\
212	0.00998476612575322\\
213	0.00998476612574865\\
214	0.009984766125744\\
215	0.00998476612573928\\
216	0.00998476612573446\\
217	0.00998476612572956\\
218	0.00998476612572458\\
219	0.00998476612571952\\
220	0.00998476612571436\\
221	0.00998476612570912\\
222	0.00998476612570378\\
223	0.00998476612569835\\
224	0.00998476612569283\\
225	0.00998476612568721\\
226	0.00998476612568149\\
227	0.00998476612567567\\
228	0.00998476612566975\\
229	0.00998476612566373\\
230	0.00998476612565761\\
231	0.00998476612565138\\
232	0.00998476612564504\\
233	0.00998476612563859\\
234	0.00998476612563202\\
235	0.00998476612562534\\
236	0.00998476612561855\\
237	0.00998476612561164\\
238	0.0099847661256046\\
239	0.00998476612559745\\
240	0.00998476612559017\\
241	0.00998476612558276\\
242	0.00998476612557522\\
243	0.00998476612556756\\
244	0.00998476612555975\\
245	0.00998476612555182\\
246	0.00998476612554374\\
247	0.00998476612553552\\
248	0.00998476612552716\\
249	0.00998476612551865\\
250	0.00998476612551\\
251	0.00998476612550119\\
252	0.00998476612549223\\
253	0.00998476612548311\\
254	0.00998476612547383\\
255	0.00998476612546439\\
256	0.00998476612545479\\
257	0.00998476612544501\\
258	0.00998476612543507\\
259	0.00998476612542495\\
260	0.00998476612541465\\
261	0.00998476612540418\\
262	0.00998476612539351\\
263	0.00998476612538267\\
264	0.00998476612537163\\
265	0.0099847661253604\\
266	0.00998476612534897\\
267	0.00998476612533734\\
268	0.00998476612532551\\
269	0.00998476612531347\\
270	0.00998476612530121\\
271	0.00998476612528874\\
272	0.00998476612527606\\
273	0.00998476612526314\\
274	0.00998476612525001\\
275	0.00998476612523664\\
276	0.00998476612522303\\
277	0.00998476612520919\\
278	0.00998476612519511\\
279	0.00998476612518077\\
280	0.00998476612516618\\
281	0.00998476612515134\\
282	0.00998476612513623\\
283	0.00998476612512086\\
284	0.00998476612510522\\
285	0.00998476612508929\\
286	0.0099847661250731\\
287	0.00998476612505661\\
288	0.00998476612503983\\
289	0.00998476612502276\\
290	0.00998476612500538\\
291	0.0099847661249877\\
292	0.00998476612496971\\
293	0.0099847661249514\\
294	0.00998476612493276\\
295	0.0099847661249138\\
296	0.0099847661248945\\
297	0.00998476612487486\\
298	0.00998476612485487\\
299	0.00998476612483453\\
300	0.00998476612481383\\
301	0.00998476612479277\\
302	0.00998476612477133\\
303	0.00998476612474951\\
304	0.00998476612472731\\
305	0.00998476612470471\\
306	0.00998476612468171\\
307	0.00998476612465831\\
308	0.00998476612463449\\
309	0.00998476612461025\\
310	0.00998476612458559\\
311	0.00998476612456048\\
312	0.00998476612453494\\
313	0.00998476612450894\\
314	0.00998476612448248\\
315	0.00998476612445555\\
316	0.00998476612442815\\
317	0.00998476612440026\\
318	0.00998476612437188\\
319	0.009984766124343\\
320	0.00998476612431361\\
321	0.0099847661242837\\
322	0.00998476612425326\\
323	0.00998476612422228\\
324	0.00998476612419075\\
325	0.00998476612415867\\
326	0.00998476612412602\\
327	0.00998476612409281\\
328	0.009984766124059\\
329	0.00998476612402459\\
330	0.00998476612398958\\
331	0.00998476612395395\\
332	0.0099847661239177\\
333	0.0099847661238808\\
334	0.00998476612384326\\
335	0.00998476612380506\\
336	0.00998476612376618\\
337	0.00998476612372663\\
338	0.00998476612368637\\
339	0.00998476612364541\\
340	0.00998476612360373\\
341	0.00998476612356133\\
342	0.00998476612351818\\
343	0.00998476612347427\\
344	0.00998476612342959\\
345	0.00998476612338413\\
346	0.00998476612333788\\
347	0.00998476612329082\\
348	0.00998476612324294\\
349	0.00998476612319422\\
350	0.00998476612314465\\
351	0.00998476612309421\\
352	0.0099847661230429\\
353	0.00998476612299069\\
354	0.00998476612293757\\
355	0.00998476612288353\\
356	0.00998476612282855\\
357	0.00998476612277261\\
358	0.00998476612271569\\
359	0.00998476612265779\\
360	0.00998476612259888\\
361	0.00998476612253895\\
362	0.00998476612247797\\
363	0.00998476612241593\\
364	0.00998476612235281\\
365	0.0099847661222886\\
366	0.00998476612222327\\
367	0.0099847661221568\\
368	0.00998476612208917\\
369	0.00998476612202037\\
370	0.00998476612195036\\
371	0.00998476612187914\\
372	0.00998476612180667\\
373	0.00998476612173293\\
374	0.0099847661216579\\
375	0.00998476612158155\\
376	0.00998476612150387\\
377	0.00998476612142482\\
378	0.00998476612134437\\
379	0.0099847661212625\\
380	0.00998476612117919\\
381	0.00998476612109439\\
382	0.00998476612100809\\
383	0.00998476612092025\\
384	0.00998476612083083\\
385	0.0099847661207398\\
386	0.00998476612064715\\
387	0.00998476612055281\\
388	0.00998476612045675\\
389	0.00998476612035895\\
390	0.00998476612025935\\
391	0.00998476612015792\\
392	0.00998476612005461\\
393	0.00998476611994938\\
394	0.00998476611984217\\
395	0.00998476611973294\\
396	0.00998476611962164\\
397	0.0099847661195082\\
398	0.00998476611939257\\
399	0.00998476611927468\\
400	0.00998476611915448\\
401	0.00998476611903188\\
402	0.00998476611890682\\
403	0.00998476611877922\\
404	0.00998476611864899\\
405	0.00998476611851605\\
406	0.00998476611838029\\
407	0.00998476611824164\\
408	0.00998476611809998\\
409	0.0099847661179552\\
410	0.0099847661178072\\
411	0.00998476611765584\\
412	0.00998476611750099\\
413	0.00998476611734252\\
414	0.00998476611718027\\
415	0.00998476611701409\\
416	0.0099847661168438\\
417	0.00998476611666922\\
418	0.00998476611649014\\
419	0.00998476611630634\\
420	0.00998476611611757\\
421	0.00998476611592355\\
422	0.00998476611572399\\
423	0.00998476611551854\\
424	0.0099847661153068\\
425	0.00998476611508829\\
426	0.00998476611486241\\
427	0.00998476611462834\\
428	0.00998476611438484\\
429	0.0099847661141299\\
430	0.0099847661138601\\
431	0.00998476611356959\\
432	0.00998476611324869\\
433	0.0099847661128826\\
434	0.00998476611245178\\
435	0.00998476611193696\\
436	0.00998476611133118\\
437	0.00998476611065367\\
438	0.00998476610994262\\
439	0.00998476610921538\\
440	0.00998476610847146\\
441	0.00998476610771034\\
442	0.00998476610693152\\
443	0.00998476610613445\\
444	0.00998476610531858\\
445	0.00998476610448337\\
446	0.00998476610362824\\
447	0.00998476610275261\\
448	0.00998476610185584\\
449	0.0099847661009372\\
450	0.00998476609999576\\
451	0.00998476609903\\
452	0.00998476609803725\\
453	0.00998476609701211\\
454	0.00998476609594338\\
455	0.00998476609480752\\
456	0.00998476609355561\\
457	0.00998476609208882\\
458	0.00998476609021732\\
459	0.00998476608760555\\
460	0.00998476608373925\\
461	0.00998476607801881\\
462	0.00998476607014061\\
463	0.00998476606073503\\
464	0.00998476605115606\\
465	0.00998476604139841\\
466	0.00998476603145654\\
467	0.00998476602132461\\
468	0.00998476601099649\\
469	0.00998476600046571\\
470	0.00998476598972551\\
471	0.00998476597876875\\
472	0.00998476596758783\\
473	0.00998476595617469\\
474	0.00998476594452093\\
475	0.00998476593261786\\
476	0.00998476592045632\\
477	0.0099847659080265\\
478	0.00998476589531791\\
479	0.00998476588231931\\
480	0.00998476586901861\\
481	0.0099847658554028\\
482	0.00998476584145781\\
483	0.00998476582716846\\
484	0.00998476581251822\\
485	0.00998476579748913\\
486	0.00998476578206153\\
487	0.00998476576621385\\
488	0.00998476574992225\\
489	0.00998476573316013\\
490	0.00998476571589736\\
491	0.0099847656980989\\
492	0.00998476567972207\\
493	0.00998476566071112\\
494	0.00998476564098614\\
495	0.00998476562042082\\
496	0.00998476559879951\\
497	0.0099847655757395\\
498	0.00998476555056474\\
499	0.00998476552213733\\
500	0.00998476548872632\\
501	0.00998476544816116\\
502	0.00998476539873531\\
503	0.00998476534112569\\
504	0.00998476527941104\\
505	0.00998476521657973\\
506	0.00998476515257641\\
507	0.00998476508733086\\
508	0.00998476502076274\\
509	0.00998476495279559\\
510	0.00998476488335207\\
511	0.00998476481234044\\
512	0.00998476473964256\\
513	0.00998476466508539\\
514	0.00998476458837248\\
515	0.0099847645089216\\
516	0.00998476442548878\\
517	0.00998476433532989\\
518	0.00998476423243209\\
519	0.00998476410410157\\
520	0.00998476392536108\\
521	0.00998476365277688\\
522	0.00998476322724478\\
523	0.00998476261074155\\
524	0.00998476185929397\\
525	0.00998476107951758\\
526	0.00998476026608737\\
527	0.00998475941368839\\
528	0.00998475851976169\\
529	0.00998475758590979\\
530	0.00998475660914964\\
531	0.00998475557133194\\
532	0.00998475443618455\\
533	0.00998475313530915\\
534	0.00998475155561873\\
535	0.00998474955614854\\
536	0.00998474707681101\\
537	0.00998474431322528\\
538	0.00998474148463942\\
539	0.00998473858561031\\
540	0.00998473560989778\\
541	0.00998473254958401\\
542	0.00998472939203223\\
543	0.0099847261133409\\
544	0.00998472266845293\\
545	0.00998471897603467\\
546	0.00998471489917038\\
547	0.00998471023443736\\
548	0.00998470478673389\\
549	0.009984698629892\\
550	0.0099846922355815\\
551	0.00998468539895788\\
552	0.00998467761634069\\
553	0.00998466766639784\\
554	0.00998465284881902\\
555	0.00998462839384316\\
556	0.00998451835371763\\
557	0.00998438713828728\\
558	0.00998424091014161\\
559	0.00998407048417542\\
560	0.00998337245231295\\
561	0.0099824607661268\\
562	0.00998150752370361\\
563	0.00998050497714601\\
564	0.00997944065250886\\
565	0.00997829668708648\\
566	0.00996445051854514\\
567	0.00995056165887344\\
568	0.00993686172377713\\
569	0.00992336079454392\\
570	0.00991007448540605\\
571	0.0098970254209339\\
572	0.0098842472847136\\
573	0.00987180223381704\\
574	0.00986010348666058\\
575	0.00984931761907658\\
576	0.00983942691884809\\
577	0.00983036670881545\\
578	0.00982193005125355\\
579	0.0097931048823283\\
580	0.00967661252408177\\
581	0.00930277543452092\\
582	0.00890533237643708\\
583	0.00848053657077329\\
584	0.00802397360009435\\
585	0.00755205906109305\\
586	0.00706559415643771\\
587	0.00656389548252311\\
588	0.00604646101757425\\
589	0.00551317570106527\\
590	0.00496674996040426\\
591	0.00440647859325226\\
592	0.00383129818654383\\
593	0.00323992338914098\\
594	0.0026306372037739\\
595	0.00200104994316786\\
596	0.00135174293718153\\
597	0.000683662792068022\\
598	0\\
599	0\\
600	0\\
};
\addplot [color=mycolor13,solid,forget plot]
  table[row sep=crcr]{%
1	0.000484240466903288\\
2	0.000484240466903338\\
3	0.000484240466903389\\
4	0.000484240466903441\\
5	0.000484240466903493\\
6	0.000484240466903547\\
7	0.000484240466903602\\
8	0.000484240466903656\\
9	0.000484240466903713\\
10	0.000484240466903771\\
11	0.000484240466903829\\
12	0.000484240466903889\\
13	0.00048424046690395\\
14	0.000484240466904013\\
15	0.000484240466904075\\
16	0.000484240466904139\\
17	0.000484240466904205\\
18	0.000484240466904272\\
19	0.000484240466904339\\
20	0.000484240466904408\\
21	0.000484240466904478\\
22	0.000484240466904549\\
23	0.000484240466904622\\
24	0.000484240466904695\\
25	0.000484240466904771\\
26	0.000484240466904848\\
27	0.000484240466904927\\
28	0.000484240466905006\\
29	0.000484240466905086\\
30	0.000484240466905168\\
31	0.000484240466905252\\
32	0.000484240466905337\\
33	0.000484240466905425\\
34	0.000484240466905513\\
35	0.000484240466905602\\
36	0.000484240466905693\\
37	0.000484240466905788\\
38	0.000484240466905882\\
39	0.000484240466905979\\
40	0.000484240466906078\\
41	0.000484240466906178\\
42	0.000484240466906279\\
43	0.000484240466906384\\
44	0.000484240466906488\\
45	0.000484240466906596\\
46	0.000484240466906706\\
47	0.000484240466906817\\
48	0.000484240466906931\\
49	0.000484240466907046\\
50	0.000484240466907165\\
51	0.000484240466907284\\
52	0.000484240466907405\\
53	0.000484240466907529\\
54	0.000484240466907655\\
55	0.000484240466907784\\
56	0.000484240466907914\\
57	0.000484240466908047\\
58	0.000484240466908183\\
59	0.000484240466908321\\
60	0.000484240466908462\\
61	0.000484240466908604\\
62	0.000484240466908751\\
63	0.000484240466908898\\
64	0.000484240466909049\\
65	0.000484240466909203\\
66	0.000484240466909359\\
67	0.000484240466909518\\
68	0.00048424046690968\\
69	0.000484240466909845\\
70	0.000484240466910012\\
71	0.000484240466910183\\
72	0.000484240466910357\\
73	0.000484240466910534\\
74	0.000484240466910714\\
75	0.000484240466910899\\
76	0.000484240466911085\\
77	0.000484240466911275\\
78	0.000484240466911469\\
79	0.000484240466911665\\
80	0.000484240466911866\\
81	0.00048424046691207\\
82	0.000484240466912277\\
83	0.000484240466912489\\
84	0.000484240466912703\\
85	0.000484240466912923\\
86	0.000484240466913147\\
87	0.000484240466913373\\
88	0.000484240466913604\\
89	0.00048424046691384\\
90	0.000484240466914079\\
91	0.000484240466914324\\
92	0.000484240466914572\\
93	0.000484240466914824\\
94	0.000484240466915081\\
95	0.000484240466915342\\
96	0.00048424046691561\\
97	0.00048424046691588\\
98	0.000484240466916157\\
99	0.000484240466916438\\
100	0.000484240466916723\\
101	0.000484240466917015\\
102	0.000484240466917311\\
103	0.000484240466917613\\
104	0.00048424046691792\\
105	0.000484240466918233\\
106	0.000484240466918552\\
107	0.000484240466918875\\
108	0.000484240466919205\\
109	0.00048424046691954\\
110	0.000484240466919883\\
111	0.000484240466920231\\
112	0.000484240466920584\\
113	0.000484240466920946\\
114	0.000484240466921311\\
115	0.000484240466921684\\
116	0.000484240466922066\\
117	0.000484240466922452\\
118	0.000484240466922847\\
119	0.000484240466923247\\
120	0.000484240466923655\\
121	0.000484240466924071\\
122	0.000484240466924494\\
123	0.000484240466924925\\
124	0.000484240466925363\\
125	0.000484240466925809\\
126	0.000484240466926262\\
127	0.000484240466926725\\
128	0.000484240466927195\\
129	0.000484240466927674\\
130	0.000484240466928163\\
131	0.000484240466928658\\
132	0.000484240466929163\\
133	0.000484240466929677\\
134	0.000484240466930202\\
135	0.000484240466930734\\
136	0.000484240466931277\\
137	0.000484240466931828\\
138	0.000484240466932391\\
139	0.000484240466932962\\
140	0.000484240466933544\\
141	0.000484240466934137\\
142	0.00048424046693474\\
143	0.000484240466935354\\
144	0.00048424046693598\\
145	0.000484240466936615\\
146	0.000484240466937264\\
147	0.000484240466937923\\
148	0.000484240466938593\\
149	0.000484240466939276\\
150	0.000484240466939972\\
151	0.00048424046694068\\
152	0.000484240466941401\\
153	0.000484240466942133\\
154	0.00048424046694288\\
155	0.000484240466943641\\
156	0.000484240466944414\\
157	0.000484240466945202\\
158	0.000484240466946003\\
159	0.000484240466946819\\
160	0.000484240466947649\\
161	0.000484240466948494\\
162	0.000484240466949355\\
163	0.000484240466950231\\
164	0.000484240466951122\\
165	0.00048424046695203\\
166	0.000484240466952954\\
167	0.000484240466953896\\
168	0.000484240466954853\\
169	0.000484240466955828\\
170	0.000484240466956819\\
171	0.000484240466957829\\
172	0.000484240466958857\\
173	0.000484240466959902\\
174	0.000484240466960968\\
175	0.000484240466962053\\
176	0.000484240466963156\\
177	0.000484240466964281\\
178	0.000484240466965423\\
179	0.000484240466966587\\
180	0.000484240466967772\\
181	0.000484240466968979\\
182	0.000484240466970208\\
183	0.000484240466971458\\
184	0.00048424046697273\\
185	0.000484240466974026\\
186	0.000484240466975344\\
187	0.000484240466976687\\
188	0.000484240466978053\\
189	0.000484240466979444\\
190	0.00048424046698086\\
191	0.000484240466982302\\
192	0.000484240466983769\\
193	0.000484240466985264\\
194	0.000484240466986784\\
195	0.000484240466988332\\
196	0.000484240466989907\\
197	0.000484240466991513\\
198	0.000484240466993146\\
199	0.000484240466994807\\
200	0.000484240466996501\\
201	0.000484240466998223\\
202	0.000484240466999977\\
203	0.000484240467001762\\
204	0.00048424046700358\\
205	0.00048424046700543\\
206	0.000484240467007315\\
207	0.000484240467009232\\
208	0.000484240467011185\\
209	0.000484240467013172\\
210	0.000484240467015197\\
211	0.000484240467017256\\
212	0.000484240467019352\\
213	0.000484240467021488\\
214	0.000484240467023662\\
215	0.000484240467025873\\
216	0.000484240467028126\\
217	0.00048424046703042\\
218	0.000484240467032754\\
219	0.000484240467035132\\
220	0.000484240467037552\\
221	0.000484240467040016\\
222	0.000484240467042524\\
223	0.000484240467045077\\
224	0.000484240467047678\\
225	0.000484240467050325\\
226	0.00048424046705302\\
227	0.000484240467055764\\
228	0.000484240467058556\\
229	0.0004842404670614\\
230	0.000484240467064296\\
231	0.000484240467067244\\
232	0.000484240467070247\\
233	0.000484240467073302\\
234	0.000484240467076414\\
235	0.000484240467079583\\
236	0.000484240467082807\\
237	0.000484240467086091\\
238	0.000484240467089436\\
239	0.00048424046709284\\
240	0.000484240467096307\\
241	0.000484240467099838\\
242	0.000484240467103431\\
243	0.000484240467107091\\
244	0.000484240467110817\\
245	0.000484240467114611\\
246	0.000484240467118475\\
247	0.000484240467122408\\
248	0.000484240467126413\\
249	0.000484240467130492\\
250	0.000484240467134645\\
251	0.000484240467138874\\
252	0.00048424046714318\\
253	0.000484240467147566\\
254	0.00048424046715203\\
255	0.000484240467156577\\
256	0.000484240467161208\\
257	0.000484240467165922\\
258	0.000484240467170724\\
259	0.000484240467175612\\
260	0.000484240467180592\\
261	0.000484240467185661\\
262	0.000484240467190824\\
263	0.000484240467196083\\
264	0.000484240467201438\\
265	0.00048424046720689\\
266	0.000484240467212444\\
267	0.000484240467218099\\
268	0.000484240467223858\\
269	0.000484240467229725\\
270	0.000484240467235697\\
271	0.000484240467241782\\
272	0.000484240467247977\\
273	0.000484240467254287\\
274	0.000484240467260715\\
275	0.000484240467267261\\
276	0.000484240467273926\\
277	0.000484240467280716\\
278	0.000484240467287631\\
279	0.000484240467294674\\
280	0.000484240467301846\\
281	0.000484240467309154\\
282	0.000484240467316596\\
283	0.000484240467324175\\
284	0.000484240467331895\\
285	0.000484240467339758\\
286	0.000484240467347768\\
287	0.000484240467355925\\
288	0.000484240467364235\\
289	0.000484240467372699\\
290	0.000484240467381321\\
291	0.000484240467390103\\
292	0.000484240467399049\\
293	0.000484240467408161\\
294	0.000484240467417442\\
295	0.000484240467426897\\
296	0.000484240467436528\\
297	0.000484240467446339\\
298	0.000484240467456333\\
299	0.000484240467466513\\
300	0.000484240467476884\\
301	0.000484240467487449\\
302	0.00048424046749821\\
303	0.000484240467509173\\
304	0.00048424046752034\\
305	0.000484240467531717\\
306	0.000484240467543307\\
307	0.000484240467555115\\
308	0.000484240467567143\\
309	0.000484240467579396\\
310	0.000484240467591879\\
311	0.000484240467604594\\
312	0.000484240467617549\\
313	0.000484240467630747\\
314	0.000484240467644192\\
315	0.00048424046765789\\
316	0.000484240467671845\\
317	0.00048424046768606\\
318	0.000484240467700543\\
319	0.000484240467715297\\
320	0.000484240467730329\\
321	0.000484240467745641\\
322	0.000484240467761242\\
323	0.000484240467777135\\
324	0.000484240467793326\\
325	0.000484240467809823\\
326	0.000484240467826628\\
327	0.000484240467843747\\
328	0.000484240467861189\\
329	0.000484240467878957\\
330	0.000484240467897058\\
331	0.0004842404679155\\
332	0.000484240467934286\\
333	0.000484240467953425\\
334	0.000484240467972922\\
335	0.000484240467992787\\
336	0.000484240468013021\\
337	0.000484240468033635\\
338	0.000484240468054636\\
339	0.000484240468076028\\
340	0.000484240468097822\\
341	0.000484240468120023\\
342	0.000484240468142638\\
343	0.000484240468165678\\
344	0.000484240468189149\\
345	0.000484240468213057\\
346	0.000484240468237412\\
347	0.000484240468262221\\
348	0.000484240468287495\\
349	0.000484240468313239\\
350	0.000484240468339464\\
351	0.000484240468366179\\
352	0.000484240468393391\\
353	0.000484240468421111\\
354	0.000484240468449347\\
355	0.00048424046847811\\
356	0.000484240468507407\\
357	0.000484240468537252\\
358	0.000484240468567653\\
359	0.000484240468598618\\
360	0.000484240468630161\\
361	0.000484240468662293\\
362	0.000484240468695021\\
363	0.000484240468728362\\
364	0.000484240468762321\\
365	0.000484240468796915\\
366	0.000484240468832154\\
367	0.00048424046886805\\
368	0.000484240468904617\\
369	0.000484240468941867\\
370	0.000484240468979814\\
371	0.000484240469018471\\
372	0.000484240469057852\\
373	0.000484240469097974\\
374	0.000484240469138849\\
375	0.000484240469180494\\
376	0.000484240469222925\\
377	0.00048424046926616\\
378	0.000484240469310213\\
379	0.000484240469355105\\
380	0.000484240469400852\\
381	0.000484240469447475\\
382	0.000484240469494994\\
383	0.000484240469543428\\
384	0.000484240469592801\\
385	0.000484240469643136\\
386	0.000484240469694458\\
387	0.000484240469746791\\
388	0.000484240469800161\\
389	0.000484240469854599\\
390	0.000484240469910134\\
391	0.000484240469966797\\
392	0.00048424047002462\\
393	0.00048424047008364\\
394	0.000484240470143896\\
395	0.000484240470205425\\
396	0.000484240470268271\\
397	0.000484240470332477\\
398	0.000484240470398088\\
399	0.000484240470465156\\
400	0.000484240470533727\\
401	0.000484240470603858\\
402	0.000484240470675606\\
403	0.00048424047074903\\
404	0.000484240470824197\\
405	0.000484240470901179\\
406	0.000484240470980048\\
407	0.000484240471060881\\
408	0.000484240471143761\\
409	0.000484240471228782\\
410	0.000484240471316043\\
411	0.000484240471405652\\
412	0.000484240471497727\\
413	0.000484240471592404\\
414	0.000484240471689826\\
415	0.000484240471790159\\
416	0.000484240471893589\\
417	0.000484240472000334\\
418	0.00048424047211065\\
419	0.000484240472224868\\
420	0.00048424047234344\\
421	0.000484240472467045\\
422	0.000484240472596777\\
423	0.000484240472734483\\
424	0.000484240472883296\\
425	0.000484240473048316\\
426	0.000484240473237145\\
427	0.000484240473459371\\
428	0.000484240473723402\\
429	0.000484240474029596\\
430	0.000484240474363645\\
431	0.000484240474704589\\
432	0.000484240475052085\\
433	0.000484240475405585\\
434	0.000484240475764482\\
435	0.000484240476128623\\
436	0.000484240476499017\\
437	0.000484240476877848\\
438	0.000484240477266483\\
439	0.000484240477665547\\
440	0.000484240478076063\\
441	0.000484240478499937\\
442	0.000484240478941073\\
443	0.000484240479407776\\
444	0.00048424047991768\\
445	0.000484240480507274\\
446	0.000484240481248529\\
447	0.000484240482273253\\
448	0.000484240483796312\\
449	0.000484240486103035\\
450	0.0004842404894243\\
451	0.000484240493635419\\
452	0.000484240498062313\\
453	0.000484240502573705\\
454	0.000484240507169916\\
455	0.000484240511850491\\
456	0.000484240516613272\\
457	0.000484240521452686\\
458	0.000484240526357104\\
459	0.000484240531306371\\
460	0.000484240536273769\\
461	0.000484240541240519\\
462	0.000484240546228037\\
463	0.000484240551311235\\
464	0.000484240556492951\\
465	0.000484240561776135\\
466	0.000484240567163889\\
467	0.000484240572659478\\
468	0.000484240578266315\\
469	0.000484240583988024\\
470	0.000484240589828514\\
471	0.000484240595792018\\
472	0.000484240601882934\\
473	0.000484240608105622\\
474	0.000484240614464737\\
475	0.000484240620965255\\
476	0.000484240627612496\\
477	0.000484240634412163\\
478	0.000484240641370363\\
479	0.000484240648493667\\
480	0.000484240655789144\\
481	0.000484240663264423\\
482	0.000484240670927752\\
483	0.000484240678788071\\
484	0.000484240686855094\\
485	0.00048424069513941\\
486	0.000484240703652612\\
487	0.000484240712407421\\
488	0.000484240721417891\\
489	0.000484240730699638\\
490	0.000484240740270217\\
491	0.000484240750149749\\
492	0.00048424076036209\\
493	0.000484240770937182\\
494	0.000484240781915921\\
495	0.00048424079336023\\
496	0.000484240805373368\\
497	0.000484240818138671\\
498	0.000484240831986899\\
499	0.000484240847495088\\
500	0.000484240865582196\\
501	0.000484240887462403\\
502	0.00048424091413874\\
503	0.000484240945128151\\
504	0.000484240977435586\\
505	0.000484241010360868\\
506	0.000484241043948233\\
507	0.000484241078241602\\
508	0.000484241113276388\\
509	0.000484241149094674\\
510	0.000484241185749511\\
511	0.000484241223317212\\
512	0.000484241261926852\\
513	0.000484241301830343\\
514	0.000484241343566185\\
515	0.000484241388329228\\
516	0.000484241438764512\\
517	0.000484241500542035\\
518	0.000484241585077113\\
519	0.000484241712968724\\
520	0.000484241914290421\\
521	0.000484242213187589\\
522	0.000484242580237095\\
523	0.000484242952253886\\
524	0.000484243336263891\\
525	0.000484243733975993\\
526	0.000484244147872262\\
527	0.000484244581210334\\
528	0.000484245037161741\\
529	0.000484245516419022\\
530	0.000484246016790978\\
531	0.000484246545701493\\
532	0.000484247119218224\\
533	0.000484247767406508\\
534	0.000484248541306853\\
535	0.000484249512882124\\
536	0.000484250731440822\\
537	0.000484252078883766\\
538	0.000484253458944545\\
539	0.000484254874471593\\
540	0.000484256328862317\\
541	0.000484257826378404\\
542	0.000484259373029604\\
543	0.000484260979213294\\
544	0.00048426266514276\\
545	0.000484264468980179\\
546	0.000484266455953145\\
547	0.000484268725433269\\
548	0.00048427138412627\\
549	0.000484274421921086\\
550	0.000484277630733646\\
551	0.000484281192722443\\
552	0.000484285569045597\\
553	0.000484291838485129\\
554	0.000484302055719857\\
555	0.000484318648444575\\
556	0.00048433717149597\\
557	0.000484358312049961\\
558	0.000484384433837813\\
559	0.000484419741214524\\
560	0.000484467972308084\\
561	0.000484518384319466\\
562	0.000484571833803974\\
563	0.00048463043821816\\
564	0.000484698204292027\\
565	0.000484780473870265\\
566	0.000484872232915441\\
567	0.000484965833244364\\
568	0.000485062002730881\\
569	0.00048516191722124\\
570	0.000485267787354841\\
571	0.000485383204690684\\
572	0.000485511922586414\\
573	0.000485642971766619\\
574	0.000485775812999213\\
575	0.000485914470174083\\
576	0.000486084229404065\\
577	0.000486346843596949\\
578	0.000486874060292007\\
579	0.000499332124294703\\
580	0.000549606817473229\\
581	0.00088240546943094\\
582	0.00124000461749179\\
583	0.00162646793862394\\
584	0.0020440485140991\\
585	0.00247787686417299\\
586	0.00292892988002479\\
587	0.00339798959748976\\
588	0.00388566154600113\\
589	0.00439214851677062\\
590	0.00491474520087469\\
591	0.00545443621534336\\
592	0.00601233132542992\\
593	0.00658977217077654\\
594	0.00718858827196951\\
595	0.00781118959413535\\
596	0.00845712341615153\\
597	0.00912599392411371\\
598	0.00981478197180822\\
599	0\\
600	0\\
};
\addplot [color=mycolor14,solid,forget plot]
  table[row sep=crcr]{%
1	2.808615381101e-05\\
2	2.80861538121573e-05\\
3	2.8086153813325e-05\\
4	2.80861538145132e-05\\
5	2.80861538157218e-05\\
6	2.80861538169526e-05\\
7	2.80861538182056e-05\\
8	2.80861538194807e-05\\
9	2.80861538207797e-05\\
10	2.80861538221009e-05\\
11	2.80861538234459e-05\\
12	2.8086153824813e-05\\
13	2.80861538262075e-05\\
14	2.80861538276258e-05\\
15	2.80861538290697e-05\\
16	2.80861538305392e-05\\
17	2.80861538320342e-05\\
18	2.80861538335565e-05\\
19	2.80861538351061e-05\\
20	2.80861538366846e-05\\
21	2.80861538382905e-05\\
22	2.80861538399236e-05\\
23	2.80861538415874e-05\\
24	2.80861538432801e-05\\
25	2.80861538450036e-05\\
26	2.80861538467577e-05\\
27	2.80861538485443e-05\\
28	2.80861538503615e-05\\
29	2.80861538522111e-05\\
30	2.80861538540948e-05\\
31	2.80861538560109e-05\\
32	2.80861538579611e-05\\
33	2.80861538599471e-05\\
34	2.80861538619689e-05\\
35	2.80861538640265e-05\\
36	2.80861538661198e-05\\
37	2.80861538682524e-05\\
38	2.80861538704208e-05\\
39	2.80861538726301e-05\\
40	2.80861538748769e-05\\
41	2.80861538771646e-05\\
42	2.80861538794933e-05\\
43	2.80861538818645e-05\\
44	2.80861538842767e-05\\
45	2.80861538867332e-05\\
46	2.80861538892323e-05\\
47	2.80861538917774e-05\\
48	2.80861538943669e-05\\
49	2.80861538970023e-05\\
50	2.80861538996855e-05\\
51	2.80861539024165e-05\\
52	2.80861539051969e-05\\
53	2.8086153908025e-05\\
54	2.80861539109059e-05\\
55	2.80861539138363e-05\\
56	2.80861539168213e-05\\
57	2.80861539198574e-05\\
58	2.8086153922948e-05\\
59	2.80861539260949e-05\\
60	2.80861539292981e-05\\
61	2.80861539325574e-05\\
62	2.80861539358748e-05\\
63	2.80861539392518e-05\\
64	2.80861539426902e-05\\
65	2.80861539461883e-05\\
66	2.80861539497494e-05\\
67	2.80861539533754e-05\\
68	2.80861539570643e-05\\
69	2.80861539608198e-05\\
70	2.80861539646418e-05\\
71	2.80861539685319e-05\\
72	2.80861539724919e-05\\
73	2.80861539765236e-05\\
74	2.80861539806268e-05\\
75	2.80861539848016e-05\\
76	2.80861539890532e-05\\
77	2.80861539933798e-05\\
78	2.8086153997783e-05\\
79	2.80861540022647e-05\\
80	2.80861540068265e-05\\
81	2.80861540114701e-05\\
82	2.80861540161973e-05\\
83	2.8086154021008e-05\\
84	2.80861540259039e-05\\
85	2.80861540308885e-05\\
86	2.80861540359617e-05\\
87	2.80861540411252e-05\\
88	2.80861540463809e-05\\
89	2.80861540517302e-05\\
90	2.80861540571751e-05\\
91	2.80861540627171e-05\\
92	2.8086154068358e-05\\
93	2.80861540740994e-05\\
94	2.80861540799432e-05\\
95	2.80861540858909e-05\\
96	2.8086154091946e-05\\
97	2.80861540981085e-05\\
98	2.80861541043802e-05\\
99	2.80861541107643e-05\\
100	2.80861541172626e-05\\
101	2.80861541238769e-05\\
102	2.80861541306088e-05\\
103	2.80861541374617e-05\\
104	2.80861541444357e-05\\
105	2.80861541515341e-05\\
106	2.80861541587604e-05\\
107	2.80861541661145e-05\\
108	2.80861541735999e-05\\
109	2.80861541812199e-05\\
110	2.80861541889746e-05\\
111	2.80861541968674e-05\\
112	2.80861542049017e-05\\
113	2.80861542130792e-05\\
114	2.80861542214033e-05\\
115	2.80861542298757e-05\\
116	2.80861542384981e-05\\
117	2.80861542472757e-05\\
118	2.80861542562083e-05\\
119	2.8086154265303e-05\\
120	2.80861542745578e-05\\
121	2.80861542839781e-05\\
122	2.80861542935653e-05\\
123	2.80861543033248e-05\\
124	2.80861543132598e-05\\
125	2.80861543233705e-05\\
126	2.80861543336618e-05\\
127	2.80861543441355e-05\\
128	2.80861543547967e-05\\
129	2.80861543656489e-05\\
130	2.80861543766955e-05\\
131	2.8086154387938e-05\\
132	2.808615439938e-05\\
133	2.80861544110283e-05\\
134	2.80861544228828e-05\\
135	2.80861544349505e-05\\
136	2.80861544472329e-05\\
137	2.80861544597335e-05\\
138	2.80861544724575e-05\\
139	2.80861544854098e-05\\
140	2.80861544985924e-05\\
141	2.80861545120101e-05\\
142	2.80861545256665e-05\\
143	2.80861545395684e-05\\
144	2.80861545537175e-05\\
145	2.80861545681189e-05\\
146	2.80861545827777e-05\\
147	2.80861545976973e-05\\
148	2.80861546128846e-05\\
149	2.80861546283429e-05\\
150	2.80861546440756e-05\\
151	2.80861546600896e-05\\
152	2.80861546763918e-05\\
153	2.8086154692982e-05\\
154	2.80861547098706e-05\\
155	2.80861547270591e-05\\
156	2.80861547445563e-05\\
157	2.80861547623654e-05\\
158	2.80861547804915e-05\\
159	2.80861547989416e-05\\
160	2.80861548177224e-05\\
161	2.80861548368372e-05\\
162	2.80861548562948e-05\\
163	2.80861548761001e-05\\
164	2.80861548962583e-05\\
165	2.80861549167762e-05\\
166	2.80861549376605e-05\\
167	2.808615495892e-05\\
168	2.80861549805578e-05\\
169	2.80861550025827e-05\\
170	2.80861550250013e-05\\
171	2.80861550478205e-05\\
172	2.80861550710472e-05\\
173	2.80861550946898e-05\\
174	2.80861551187552e-05\\
175	2.80861551432501e-05\\
176	2.80861551681832e-05\\
177	2.80861551935629e-05\\
178	2.80861552193961e-05\\
179	2.80861552456912e-05\\
180	2.80861552724568e-05\\
181	2.80861552997015e-05\\
182	2.80861553274337e-05\\
183	2.8086155355662e-05\\
184	2.80861553843966e-05\\
185	2.80861554136442e-05\\
186	2.80861554434153e-05\\
187	2.80861554737199e-05\\
188	2.80861555045666e-05\\
189	2.80861555359673e-05\\
190	2.80861555679289e-05\\
191	2.80861556004633e-05\\
192	2.80861556335807e-05\\
193	2.80861556672913e-05\\
194	2.80861557016054e-05\\
195	2.80861557365348e-05\\
196	2.80861557720917e-05\\
197	2.80861558082843e-05\\
198	2.80861558451265e-05\\
199	2.80861558826301e-05\\
200	2.80861559208071e-05\\
201	2.80861559596676e-05\\
202	2.80861559992254e-05\\
203	2.80861560394923e-05\\
204	2.80861560804838e-05\\
205	2.80861561222099e-05\\
206	2.80861561646862e-05\\
207	2.80861562079244e-05\\
208	2.808615625194e-05\\
209	2.80861562967465e-05\\
210	2.80861563423577e-05\\
211	2.80861563887889e-05\\
212	2.80861564360554e-05\\
213	2.80861564841725e-05\\
214	2.80861565331522e-05\\
215	2.80861565830149e-05\\
216	2.80861566337761e-05\\
217	2.80861566854492e-05\\
218	2.80861567380514e-05\\
219	2.80861567916031e-05\\
220	2.8086156846118e-05\\
221	2.80861569016166e-05\\
222	2.8086156958114e-05\\
223	2.80861570156292e-05\\
224	2.80861570741826e-05\\
225	2.80861571337911e-05\\
226	2.80861571944754e-05\\
227	2.8086157256254e-05\\
228	2.80861573191475e-05\\
229	2.80861573831781e-05\\
230	2.80861574483645e-05\\
231	2.80861575147271e-05\\
232	2.80861575822899e-05\\
233	2.80861576510749e-05\\
234	2.80861577211026e-05\\
235	2.8086157792397e-05\\
236	2.80861578649818e-05\\
237	2.80861579388793e-05\\
238	2.80861580141149e-05\\
239	2.80861580907143e-05\\
240	2.80861581686996e-05\\
241	2.80861582480999e-05\\
242	2.80861583289389e-05\\
243	2.8086158411244e-05\\
244	2.80861584950424e-05\\
245	2.80861585803614e-05\\
246	2.808615866723e-05\\
247	2.80861587556754e-05\\
248	2.80861588457284e-05\\
249	2.80861589374196e-05\\
250	2.80861590307763e-05\\
251	2.80861591258309e-05\\
252	2.80861592226158e-05\\
253	2.80861593211634e-05\\
254	2.80861594215043e-05\\
255	2.80861595236743e-05\\
256	2.80861596277059e-05\\
257	2.80861597336349e-05\\
258	2.80861598414953e-05\\
259	2.80861599513246e-05\\
260	2.80861600631604e-05\\
261	2.80861601770367e-05\\
262	2.80861602929962e-05\\
263	2.80861604110746e-05\\
264	2.80861605313111e-05\\
265	2.80861606537502e-05\\
266	2.80861607784293e-05\\
267	2.8086160905391e-05\\
268	2.80861610346796e-05\\
269	2.80861611663394e-05\\
270	2.80861613004132e-05\\
271	2.8086161436945e-05\\
272	2.80861615759846e-05\\
273	2.80861617175778e-05\\
274	2.80861618617707e-05\\
275	2.80861620086161e-05\\
276	2.80861621581618e-05\\
277	2.80861623104571e-05\\
278	2.80861624655585e-05\\
279	2.80861626235135e-05\\
280	2.80861627843802e-05\\
281	2.8086162948213e-05\\
282	2.80861631150649e-05\\
283	2.80861632849972e-05\\
284	2.80861634580662e-05\\
285	2.80861636343298e-05\\
286	2.80861638138511e-05\\
287	2.80861639966915e-05\\
288	2.80861641829124e-05\\
289	2.80861643725784e-05\\
290	2.80861645657562e-05\\
291	2.80861647625104e-05\\
292	2.80861649629093e-05\\
293	2.80861651670228e-05\\
294	2.80861653749207e-05\\
295	2.80861655866747e-05\\
296	2.80861658023598e-05\\
297	2.80861660220475e-05\\
298	2.80861662458163e-05\\
299	2.80861664737411e-05\\
300	2.80861667059039e-05\\
301	2.80861669423831e-05\\
302	2.80861671832603e-05\\
303	2.8086167428621e-05\\
304	2.80861676785469e-05\\
305	2.80861679331266e-05\\
306	2.80861681924488e-05\\
307	2.80861684566022e-05\\
308	2.80861687256771e-05\\
309	2.80861689997707e-05\\
310	2.80861692789733e-05\\
311	2.8086169563382e-05\\
312	2.80861698530975e-05\\
313	2.80861701482204e-05\\
314	2.80861704488495e-05\\
315	2.80861707550905e-05\\
316	2.80861710670491e-05\\
317	2.80861713848309e-05\\
318	2.80861717085486e-05\\
319	2.80861720383129e-05\\
320	2.80861723742346e-05\\
321	2.8086172716433e-05\\
322	2.80861730650224e-05\\
323	2.80861734201238e-05\\
324	2.80861737818582e-05\\
325	2.80861741503501e-05\\
326	2.80861745257256e-05\\
327	2.80861749081126e-05\\
328	2.80861752976423e-05\\
329	2.80861756944477e-05\\
330	2.808617609866e-05\\
331	2.80861765104226e-05\\
332	2.80861769298699e-05\\
333	2.8086177357147e-05\\
334	2.80861777923987e-05\\
335	2.80861782357716e-05\\
336	2.80861786874158e-05\\
337	2.8086179147483e-05\\
338	2.80861796161282e-05\\
339	2.80861800935101e-05\\
340	2.80861805797872e-05\\
341	2.80861810751247e-05\\
342	2.80861815796899e-05\\
343	2.8086182093648e-05\\
344	2.80861826171747e-05\\
345	2.80861831504437e-05\\
346	2.80861836936324e-05\\
347	2.8086184246925e-05\\
348	2.80861848105038e-05\\
349	2.80861853845562e-05\\
350	2.80861859692751e-05\\
351	2.80861865648563e-05\\
352	2.8086187171496e-05\\
353	2.80861877893987e-05\\
354	2.8086188418769e-05\\
355	2.80861890598182e-05\\
356	2.80861897127595e-05\\
357	2.80861903778127e-05\\
358	2.80861910551978e-05\\
359	2.80861917451431e-05\\
360	2.80861924478824e-05\\
361	2.80861931636507e-05\\
362	2.80861938926884e-05\\
363	2.80861946352462e-05\\
364	2.80861953915747e-05\\
365	2.80861961619311e-05\\
366	2.80861969465816e-05\\
367	2.80861977457971e-05\\
368	2.80861985598556e-05\\
369	2.80861993890399e-05\\
370	2.80862002336433e-05\\
371	2.80862010939642e-05\\
372	2.80862019703127e-05\\
373	2.80862028630044e-05\\
374	2.80862037723647e-05\\
375	2.80862046987312e-05\\
376	2.808620564245e-05\\
377	2.80862066038774e-05\\
378	2.80862075833849e-05\\
379	2.80862085813546e-05\\
380	2.80862095981818e-05\\
381	2.80862106342774e-05\\
382	2.8086211690071e-05\\
383	2.80862127660042e-05\\
384	2.80862138625407e-05\\
385	2.80862149801645e-05\\
386	2.80862161193821e-05\\
387	2.8086217280727e-05\\
388	2.80862184647584e-05\\
389	2.80862196720662e-05\\
390	2.80862209032726e-05\\
391	2.80862221590288e-05\\
392	2.80862234400219e-05\\
393	2.80862247469712e-05\\
394	2.80862260806475e-05\\
395	2.80862274418707e-05\\
396	2.80862288315067e-05\\
397	2.80862302504661e-05\\
398	2.80862316997153e-05\\
399	2.80862331802619e-05\\
400	2.80862346931698e-05\\
401	2.80862362395503e-05\\
402	2.80862378205697e-05\\
403	2.80862394374588e-05\\
404	2.8086241091532e-05\\
405	2.80862427842111e-05\\
406	2.80862445170288e-05\\
407	2.80862462915791e-05\\
408	2.80862481095242e-05\\
409	2.80862499726541e-05\\
410	2.80862518829092e-05\\
411	2.80862538424e-05\\
412	2.80862558534264e-05\\
413	2.8086257918505e-05\\
414	2.80862600403959e-05\\
415	2.80862622221511e-05\\
416	2.80862644671482e-05\\
417	2.80862667791621e-05\\
418	2.80862691624525e-05\\
419	2.80862716219223e-05\\
420	2.80862741634101e-05\\
421	2.80862767942748e-05\\
422	2.80862795246284e-05\\
423	2.80862823699214e-05\\
424	2.8086285356212e-05\\
425	2.8086288530267e-05\\
426	2.80862919770322e-05\\
427	2.80862958444663e-05\\
428	2.80863003639551e-05\\
429	2.80863058241207e-05\\
430	2.80863124129024e-05\\
431	2.80863198919898e-05\\
432	2.80863275459902e-05\\
433	2.80863353806416e-05\\
434	2.80863434019857e-05\\
435	2.80863516163779e-05\\
436	2.80863600305164e-05\\
437	2.80863686514935e-05\\
438	2.80863774868853e-05\\
439	2.8086386544978e-05\\
440	2.80863958353203e-05\\
441	2.80864053701281e-05\\
442	2.80864151678357e-05\\
443	2.80864252619863e-05\\
444	2.80864357231286e-05\\
445	2.8086446711237e-05\\
446	2.80864585968064e-05\\
447	2.80864722254694e-05\\
448	2.80864894474804e-05\\
449	2.80865140184409e-05\\
450	2.80865526116246e-05\\
451	2.80866141754087e-05\\
452	2.80867023239014e-05\\
453	2.80867965835178e-05\\
454	2.80868926758453e-05\\
455	2.80869906251254e-05\\
456	2.80870904593219e-05\\
457	2.80871922118918e-05\\
458	2.8087295922761e-05\\
459	2.80874016360984e-05\\
460	2.8087509391849e-05\\
461	2.80876192130461e-05\\
462	2.808773110846e-05\\
463	2.80878451184784e-05\\
464	2.80879613047944e-05\\
465	2.80880797319544e-05\\
466	2.80882004670643e-05\\
467	2.80883235804694e-05\\
468	2.80884491466583e-05\\
469	2.80885772429902e-05\\
470	2.80887079505498e-05\\
471	2.80888413562767e-05\\
472	2.80889775553677e-05\\
473	2.80891166493581e-05\\
474	2.80892587363316e-05\\
475	2.80894039207413e-05\\
476	2.80895523139946e-05\\
477	2.80897040350871e-05\\
478	2.8089859211319e-05\\
479	2.80900179791046e-05\\
480	2.80901804848662e-05\\
481	2.80903468860667e-05\\
482	2.80905173523679e-05\\
483	2.8090692066955e-05\\
484	2.80908712280712e-05\\
485	2.80910550508182e-05\\
486	2.80912437692606e-05\\
487	2.80914376388795e-05\\
488	2.80916369395449e-05\\
489	2.80918419790192e-05\\
490	2.80920530972932e-05\\
491	2.80922706722094e-05\\
492	2.80924951278461e-05\\
493	2.80927269487464e-05\\
494	2.80929667067045e-05\\
495	2.80932151172259e-05\\
496	2.80934731650202e-05\\
497	2.80937423869787e-05\\
498	2.80940254996891e-05\\
499	2.80943277284858e-05\\
500	2.80946593912327e-05\\
501	2.8095040150302e-05\\
502	2.80955034844796e-05\\
503	2.80960927022307e-05\\
504	2.80968231164462e-05\\
505	2.80975931042638e-05\\
506	2.80983776764648e-05\\
507	2.80991778365475e-05\\
508	2.80999946916829e-05\\
509	2.81008290390136e-05\\
510	2.81016817629114e-05\\
511	2.81025538510951e-05\\
512	2.8103446463201e-05\\
513	2.81043611010734e-05\\
514	2.81053000457361e-05\\
515	2.81062675292016e-05\\
516	2.81072728711379e-05\\
517	2.81083387599474e-05\\
518	2.81095225156801e-05\\
519	2.81109679733623e-05\\
520	2.81130200084217e-05\\
521	2.81164217137343e-05\\
522	2.8122372809078e-05\\
523	2.81307328451471e-05\\
524	2.81393374669202e-05\\
525	2.8148210089866e-05\\
526	2.81573853040879e-05\\
527	2.81669176615914e-05\\
528	2.81768901519144e-05\\
529	2.81874046473829e-05\\
530	2.81985071666067e-05\\
531	2.82100322058695e-05\\
532	2.82220592867267e-05\\
533	2.82348190650448e-05\\
534	2.82488249099873e-05\\
535	2.82651675017933e-05\\
536	2.82860473335031e-05\\
537	2.83145323942167e-05\\
538	2.83472442182761e-05\\
539	2.8380749358613e-05\\
540	2.84151133427996e-05\\
541	2.84504107378315e-05\\
542	2.8486726090546e-05\\
543	2.85241574117138e-05\\
544	2.85628504144372e-05\\
545	2.8603106863314e-05\\
546	2.86455858784958e-05\\
547	2.86916048197475e-05\\
548	2.87437943798303e-05\\
549	2.88064993319595e-05\\
550	2.88810108054738e-05\\
551	2.89574258384307e-05\\
552	2.90366136841843e-05\\
553	2.9121818143783e-05\\
554	2.92262971949298e-05\\
555	2.93947309519714e-05\\
556	3.11587393730253e-05\\
557	3.31624962695448e-05\\
558	3.52871923224828e-05\\
559	3.75823663834745e-05\\
560	4.1712988773412e-05\\
561	5.80622218823199e-05\\
562	7.5201694526181e-05\\
563	9.32340626589157e-05\\
564	0.000112306196431779\\
565	0.000132645131255909\\
566	0.00030164397021752\\
567	0.000607768803823587\\
568	0.000930822380175926\\
569	0.00127311620720435\\
570	0.00163736940748546\\
571	0.00202680492261195\\
572	0.00244524712103414\\
573	0.00289722057015775\\
574	0.0033697985617141\\
575	0.00386159633509131\\
576	0.00437431846692315\\
577	0.00491016184443514\\
578	0.00547110656601669\\
579	0.00604483533067061\\
580	0.00659986535839362\\
581	0.00686411121575563\\
582	0.00712289222269977\\
583	0.007367743343057\\
584	0.00759482125548611\\
585	0.00782073949169679\\
586	0.00804446638930281\\
587	0.00826483417025616\\
588	0.00848064034958728\\
589	0.00869044844652259\\
590	0.00889569740473372\\
591	0.00909380142313226\\
592	0.00928132122564964\\
593	0.00945516715155288\\
594	0.00961172798547802\\
595	0.00974741425137617\\
596	0.00986072654667908\\
597	0.00994757959173143\\
598	0.0099999191923403\\
599	0\\
600	0\\
};
\addplot [color=mycolor15,solid,forget plot]
  table[row sep=crcr]{%
1	2.90931793024638e-05\\
2	2.9093179327587e-05\\
3	2.90931793531601e-05\\
4	2.90931793791901e-05\\
5	2.90931794056854e-05\\
6	2.90931794326563e-05\\
7	2.90931794601061e-05\\
8	2.90931794880486e-05\\
9	2.90931795164905e-05\\
10	2.90931795454421e-05\\
11	2.90931795749101e-05\\
12	2.90931796049031e-05\\
13	2.90931796354348e-05\\
14	2.90931796665119e-05\\
15	2.9093179698143e-05\\
16	2.909317973034e-05\\
17	2.90931797631131e-05\\
18	2.90931797964726e-05\\
19	2.90931798304269e-05\\
20	2.90931798649881e-05\\
21	2.90931799001681e-05\\
22	2.90931799359752e-05\\
23	2.90931799724233e-05\\
24	2.90931800095226e-05\\
25	2.90931800472849e-05\\
26	2.90931800857222e-05\\
27	2.90931801248464e-05\\
28	2.90931801646695e-05\\
29	2.90931802052034e-05\\
30	2.90931802464617e-05\\
31	2.9093180288458e-05\\
32	2.90931803312044e-05\\
33	2.90931803747144e-05\\
34	2.90931804190017e-05\\
35	2.90931804640798e-05\\
36	2.90931805099642e-05\\
37	2.90931805566668e-05\\
38	2.90931806042046e-05\\
39	2.90931806525913e-05\\
40	2.90931807018439e-05\\
41	2.90931807519743e-05\\
42	2.90931808030013e-05\\
43	2.90931808549385e-05\\
44	2.90931809078047e-05\\
45	2.90931809616153e-05\\
46	2.90931810163854e-05\\
47	2.90931810721357e-05\\
48	2.90931811288797e-05\\
49	2.90931811866379e-05\\
50	2.90931812454274e-05\\
51	2.90931813052686e-05\\
52	2.90931813661768e-05\\
53	2.90931814281724e-05\\
54	2.90931814912744e-05\\
55	2.90931815555047e-05\\
56	2.90931816208822e-05\\
57	2.90931816874256e-05\\
58	2.9093181755157e-05\\
59	2.90931818240987e-05\\
60	2.90931818942728e-05\\
61	2.9093181965698e-05\\
62	2.90931820383982e-05\\
63	2.90931821123955e-05\\
64	2.90931821877155e-05\\
65	2.90931822643805e-05\\
66	2.90931823424124e-05\\
67	2.90931824218387e-05\\
68	2.90931825026832e-05\\
69	2.90931825849697e-05\\
70	2.90931826687255e-05\\
71	2.90931827539762e-05\\
72	2.9093182840749e-05\\
73	2.90931829290696e-05\\
74	2.90931830189685e-05\\
75	2.90931831104697e-05\\
76	2.90931832036056e-05\\
77	2.90931832984034e-05\\
78	2.90931833948938e-05\\
79	2.90931834931058e-05\\
80	2.909318359307e-05\\
81	2.90931836948189e-05\\
82	2.90931837983831e-05\\
83	2.90931839037968e-05\\
84	2.90931840110905e-05\\
85	2.90931841202984e-05\\
86	2.90931842314564e-05\\
87	2.90931843445984e-05\\
88	2.90931844597586e-05\\
89	2.90931845769727e-05\\
90	2.909318469628e-05\\
91	2.90931848177163e-05\\
92	2.90931849413191e-05\\
93	2.90931850671258e-05\\
94	2.90931851951791e-05\\
95	2.90931853255164e-05\\
96	2.90931854581805e-05\\
97	2.90931855932105e-05\\
98	2.90931857306489e-05\\
99	2.9093185870542e-05\\
100	2.90931860129287e-05\\
101	2.90931861578569e-05\\
102	2.90931863053709e-05\\
103	2.90931864555167e-05\\
104	2.90931866083403e-05\\
105	2.90931867638929e-05\\
106	2.90931869222187e-05\\
107	2.9093187083369e-05\\
108	2.90931872473948e-05\\
109	2.90931874143473e-05\\
110	2.90931875842777e-05\\
111	2.90931877572404e-05\\
112	2.909318793329e-05\\
113	2.90931881124794e-05\\
114	2.90931882948648e-05\\
115	2.90931884805041e-05\\
116	2.90931886694554e-05\\
117	2.90931888617782e-05\\
118	2.90931890575323e-05\\
119	2.9093189256779e-05\\
120	2.90931894595796e-05\\
121	2.9093189665999e-05\\
122	2.90931898761001e-05\\
123	2.90931900899495e-05\\
124	2.90931903076154e-05\\
125	2.90931905291642e-05\\
126	2.90931907546642e-05\\
127	2.90931909841902e-05\\
128	2.90931912178088e-05\\
129	2.90931914555968e-05\\
130	2.90931916976273e-05\\
131	2.90931919439771e-05\\
132	2.90931921947213e-05\\
133	2.90931924499398e-05\\
134	2.90931927097111e-05\\
135	2.90931929741188e-05\\
136	2.90931932432446e-05\\
137	2.90931935171721e-05\\
138	2.90931937959883e-05\\
139	2.90931940797799e-05\\
140	2.90931943686358e-05\\
141	2.90931946626461e-05\\
142	2.90931949619031e-05\\
143	2.9093195266502e-05\\
144	2.90931955765367e-05\\
145	2.90931958921044e-05\\
146	2.90931962133055e-05\\
147	2.9093196540239e-05\\
148	2.90931968730071e-05\\
149	2.90931972117172e-05\\
150	2.90931975564716e-05\\
151	2.90931979073828e-05\\
152	2.90931982645565e-05\\
153	2.90931986281085e-05\\
154	2.90931989981497e-05\\
155	2.90931993747994e-05\\
156	2.90931997581735e-05\\
157	2.9093200148393e-05\\
158	2.90932005455807e-05\\
159	2.90932009498626e-05\\
160	2.9093201361365e-05\\
161	2.90932017802156e-05\\
162	2.90932022065474e-05\\
163	2.90932026404951e-05\\
164	2.9093203082195e-05\\
165	2.90932035317853e-05\\
166	2.9093203989409e-05\\
167	2.90932044552076e-05\\
168	2.90932049293313e-05\\
169	2.90932054119247e-05\\
170	2.90932059031448e-05\\
171	2.90932064031433e-05\\
172	2.90932069120787e-05\\
173	2.90932074301129e-05\\
174	2.90932079574061e-05\\
175	2.90932084941271e-05\\
176	2.90932090404447e-05\\
177	2.90932095965309e-05\\
178	2.90932101625632e-05\\
179	2.90932107387169e-05\\
180	2.9093211325178e-05\\
181	2.90932119221289e-05\\
182	2.90932125297603e-05\\
183	2.90932131482633e-05\\
184	2.90932137778338e-05\\
185	2.90932144186731e-05\\
186	2.90932150709804e-05\\
187	2.90932157349655e-05\\
188	2.90932164108381e-05\\
189	2.90932170988112e-05\\
190	2.90932177991047e-05\\
191	2.90932185119401e-05\\
192	2.90932192375442e-05\\
193	2.90932199761454e-05\\
194	2.90932207279824e-05\\
195	2.90932214932903e-05\\
196	2.90932222723164e-05\\
197	2.90932230653042e-05\\
198	2.90932238725096e-05\\
199	2.90932246941883e-05\\
200	2.90932255306027e-05\\
201	2.90932263820186e-05\\
202	2.9093227248709e-05\\
203	2.90932281309498e-05\\
204	2.90932290290242e-05\\
205	2.90932299432148e-05\\
206	2.90932308738202e-05\\
207	2.90932318211317e-05\\
208	2.90932327854579e-05\\
209	2.9093233767104e-05\\
210	2.90932347663886e-05\\
211	2.9093235783627e-05\\
212	2.909323681915e-05\\
213	2.90932378732866e-05\\
214	2.90932389463793e-05\\
215	2.90932400387691e-05\\
216	2.90932411508105e-05\\
217	2.90932422828615e-05\\
218	2.90932434352834e-05\\
219	2.90932446084512e-05\\
220	2.90932458027433e-05\\
221	2.90932470185433e-05\\
222	2.90932482562432e-05\\
223	2.90932495162469e-05\\
224	2.90932507989602e-05\\
225	2.90932521047973e-05\\
226	2.90932534341825e-05\\
227	2.90932547875471e-05\\
228	2.90932561653293e-05\\
229	2.90932575679772e-05\\
230	2.90932589959478e-05\\
231	2.90932604497046e-05\\
232	2.90932619297197e-05\\
233	2.90932634364791e-05\\
234	2.90932649704702e-05\\
235	2.90932665321958e-05\\
236	2.90932681221656e-05\\
237	2.90932697409011e-05\\
238	2.90932713889309e-05\\
239	2.90932730667934e-05\\
240	2.9093274775041e-05\\
241	2.90932765142328e-05\\
242	2.90932782849433e-05\\
243	2.90932800877501e-05\\
244	2.90932819232483e-05\\
245	2.90932837920446e-05\\
246	2.90932856947527e-05\\
247	2.90932876320032e-05\\
248	2.90932896044337e-05\\
249	2.9093291612697e-05\\
250	2.90932936574579e-05\\
251	2.90932957393947e-05\\
252	2.90932978591995e-05\\
253	2.90933000175729e-05\\
254	2.90933022152342e-05\\
255	2.9093304452913e-05\\
256	2.90933067313574e-05\\
257	2.90933090513244e-05\\
258	2.90933114135895e-05\\
259	2.90933138189419e-05\\
260	2.90933162681879e-05\\
261	2.90933187621439e-05\\
262	2.90933213016469e-05\\
263	2.9093323887549e-05\\
264	2.90933265207198e-05\\
265	2.90933292020454e-05\\
266	2.90933319324242e-05\\
267	2.90933347127801e-05\\
268	2.90933375440488e-05\\
269	2.90933404271866e-05\\
270	2.90933433631668e-05\\
271	2.90933463529849e-05\\
272	2.90933493976515e-05\\
273	2.90933524981998e-05\\
274	2.90933556556813e-05\\
275	2.90933588711683e-05\\
276	2.9093362145755e-05\\
277	2.90933654805545e-05\\
278	2.90933688767037e-05\\
279	2.90933723353618e-05\\
280	2.90933758577065e-05\\
281	2.90933794449447e-05\\
282	2.90933830983003e-05\\
283	2.90933868190243e-05\\
284	2.90933906083918e-05\\
285	2.90933944677015e-05\\
286	2.90933983982781e-05\\
287	2.90934024014714e-05\\
288	2.90934064786569e-05\\
289	2.90934106312359e-05\\
290	2.90934148606385e-05\\
291	2.90934191683202e-05\\
292	2.90934235557693e-05\\
293	2.90934280244941e-05\\
294	2.90934325760407e-05\\
295	2.90934372119791e-05\\
296	2.90934419339097e-05\\
297	2.90934467434655e-05\\
298	2.90934516423083e-05\\
299	2.90934566321344e-05\\
300	2.9093461714672e-05\\
301	2.90934668916767e-05\\
302	2.90934721649436e-05\\
303	2.90934775362964e-05\\
304	2.90934830075999e-05\\
305	2.90934885807462e-05\\
306	2.90934942576664e-05\\
307	2.90935000403294e-05\\
308	2.90935059307362e-05\\
309	2.90935119309289e-05\\
310	2.90935180429871e-05\\
311	2.90935242690242e-05\\
312	2.90935306111966e-05\\
313	2.90935370717015e-05\\
314	2.90935436527717e-05\\
315	2.90935503566846e-05\\
316	2.90935571857566e-05\\
317	2.90935641423467e-05\\
318	2.90935712288566e-05\\
319	2.90935784477308e-05\\
320	2.90935858014594e-05\\
321	2.90935932925737e-05\\
322	2.90936009236545e-05\\
323	2.90936086973232e-05\\
324	2.90936166162525e-05\\
325	2.90936246831596e-05\\
326	2.90936329008091e-05\\
327	2.90936412720169e-05\\
328	2.90936497996484e-05\\
329	2.90936584866132e-05\\
330	2.90936673358806e-05\\
331	2.90936763504659e-05\\
332	2.90936855334373e-05\\
333	2.90936948879192e-05\\
334	2.90937044170888e-05\\
335	2.90937141241779e-05\\
336	2.90937240124763e-05\\
337	2.909373408533e-05\\
338	2.9093744346143e-05\\
339	2.90937547983787e-05\\
340	2.9093765445564e-05\\
341	2.90937762912835e-05\\
342	2.90937873391848e-05\\
343	2.90937985929839e-05\\
344	2.90938100564562e-05\\
345	2.9093821733449e-05\\
346	2.9093833627874e-05\\
347	2.90938457437149e-05\\
348	2.90938580850267e-05\\
349	2.90938706559326e-05\\
350	2.90938834606379e-05\\
351	2.90938965034156e-05\\
352	2.9093909788626e-05\\
353	2.90939233207027e-05\\
354	2.9093937104166e-05\\
355	2.90939511436199e-05\\
356	2.90939654437536e-05\\
357	2.90939800093517e-05\\
358	2.90939948452909e-05\\
359	2.909400995654e-05\\
360	2.90940253481715e-05\\
361	2.9094041025364e-05\\
362	2.90940569933997e-05\\
363	2.90940732576754e-05\\
364	2.90940898237051e-05\\
365	2.90941066971224e-05\\
366	2.90941238836903e-05\\
367	2.90941413893049e-05\\
368	2.90941592200037e-05\\
369	2.90941773819708e-05\\
370	2.90941958815436e-05\\
371	2.90942147252217e-05\\
372	2.90942339196784e-05\\
373	2.90942534717711e-05\\
374	2.90942733885413e-05\\
375	2.90942936772436e-05\\
376	2.90943143453424e-05\\
377	2.90943354005319e-05\\
378	2.90943568507555e-05\\
379	2.90943787042138e-05\\
380	2.90944009693871e-05\\
381	2.90944236550557e-05\\
382	2.90944467703202e-05\\
383	2.90944703246275e-05\\
384	2.9094494327796e-05\\
385	2.90945187900515e-05\\
386	2.9094543722063e-05\\
387	2.90945691349873e-05\\
388	2.90945950405195e-05\\
389	2.90946214509568e-05\\
390	2.9094648379261e-05\\
391	2.9094675839103e-05\\
392	2.90947038448594e-05\\
393	2.90947324115871e-05\\
394	2.9094761555059e-05\\
395	2.90947912921322e-05\\
396	2.9094821640816e-05\\
397	2.90948526201269e-05\\
398	2.90948842501369e-05\\
399	2.90949165520221e-05\\
400	2.90949495480855e-05\\
401	2.9094983261762e-05\\
402	2.90950177176112e-05\\
403	2.90950529413199e-05\\
404	2.9095088959847e-05\\
405	2.90951258017984e-05\\
406	2.90951634980899e-05\\
407	2.90952020822798e-05\\
408	2.90952415895099e-05\\
409	2.90952820557793e-05\\
410	2.90953235198916e-05\\
411	2.90953660238076e-05\\
412	2.9095409613039e-05\\
413	2.90954543371108e-05\\
414	2.90955002500981e-05\\
415	2.90955474112519e-05\\
416	2.90955958857304e-05\\
417	2.90956457454964e-05\\
418	2.90956970703915e-05\\
419	2.90957499495869e-05\\
420	2.90958044836598e-05\\
421	2.90958607880629e-05\\
422	2.90959189997387e-05\\
423	2.90959792911682e-05\\
424	2.90960419019107e-05\\
425	2.9096107210368e-05\\
426	2.90961758938051e-05\\
427	2.90962492668808e-05\\
428	2.90963299303535e-05\\
429	2.90964227872562e-05\\
430	2.90965358853274e-05\\
431	2.90966784828055e-05\\
432	2.90968501766455e-05\\
433	2.90970258792738e-05\\
434	2.90972057215506e-05\\
435	2.90973898409771e-05\\
436	2.90975783821051e-05\\
437	2.90977714969531e-05\\
438	2.90979693455173e-05\\
439	2.9098172096481e-05\\
440	2.90983799285325e-05\\
441	2.90985930334534e-05\\
442	2.90988116241307e-05\\
443	2.90990359564183e-05\\
444	2.9099266389285e-05\\
445	2.9099503550225e-05\\
446	2.90997487864562e-05\\
447	2.91000053791739e-05\\
448	2.91002817408142e-05\\
449	2.91005995366793e-05\\
450	2.91010130430726e-05\\
451	2.9101649662234e-05\\
452	2.91027682805687e-05\\
453	2.91047020870107e-05\\
454	2.91068190737854e-05\\
455	2.91089775935504e-05\\
456	2.91111781752481e-05\\
457	2.91134214173341e-05\\
458	2.91157080402251e-05\\
459	2.9118038945718e-05\\
460	2.91204152310904e-05\\
461	2.91228380245918e-05\\
462	2.91253079910998e-05\\
463	2.91278249038759e-05\\
464	2.91303894186689e-05\\
465	2.91330029185693e-05\\
466	2.91356668570307e-05\\
467	2.91383827440669e-05\\
468	2.91411521584032e-05\\
469	2.91439767852071e-05\\
470	2.91468583702275e-05\\
471	2.91497987253e-05\\
472	2.91527997675997e-05\\
473	2.91558636058589e-05\\
474	2.91589925727011e-05\\
475	2.91621888539417e-05\\
476	2.91654547756575e-05\\
477	2.91687928170124e-05\\
478	2.91722056241405e-05\\
479	2.91756960256899e-05\\
480	2.91792670502333e-05\\
481	2.91829219459032e-05\\
482	2.9186664202492e-05\\
483	2.91904975764392e-05\\
484	2.91944261191551e-05\\
485	2.91984542096033e-05\\
486	2.92025865924738e-05\\
487	2.92068284229079e-05\\
488	2.92111853178327e-05\\
489	2.92156634186546e-05\\
490	2.92202694634806e-05\\
491	2.92250108729358e-05\\
492	2.92298958493688e-05\\
493	2.92349335061599e-05\\
494	2.9240134058643e-05\\
495	2.92455091202278e-05\\
496	2.92510722606742e-05\\
497	2.92568402214188e-05\\
498	2.92628358171017e-05\\
499	2.92690951603317e-05\\
500	2.92756857450048e-05\\
501	2.92827506305816e-05\\
502	2.92906101766628e-05\\
503	2.92999676784904e-05\\
504	2.93121927531563e-05\\
505	2.93289709976244e-05\\
506	2.9346987131976e-05\\
507	2.93653466747014e-05\\
508	2.93840716575043e-05\\
509	2.94031900535272e-05\\
510	2.94227202101169e-05\\
511	2.94426823864227e-05\\
512	2.94630985001884e-05\\
513	2.9483992550852e-05\\
514	2.95053915277041e-05\\
515	2.95273272121389e-05\\
516	2.95498412891083e-05\\
517	2.95730008941682e-05\\
518	2.95969477534526e-05\\
519	2.96220575361834e-05\\
520	2.96494673391088e-05\\
521	2.96828537393044e-05\\
522	2.97345261078044e-05\\
523	3.01021053709156e-05\\
524	3.10823745284106e-05\\
525	3.20921802992323e-05\\
526	3.31334756897174e-05\\
527	3.42085687022956e-05\\
528	3.53203790258592e-05\\
529	3.64729006401969e-05\\
530	3.76717780476027e-05\\
531	3.89235042501472e-05\\
532	4.02246349365916e-05\\
533	4.15798328242147e-05\\
534	4.29965025234212e-05\\
535	4.44859984380255e-05\\
536	4.60685486942687e-05\\
537	4.77956073358063e-05\\
538	5.46164785367609e-05\\
539	6.53176031175611e-05\\
540	7.63862330248849e-05\\
541	8.78552200880326e-05\\
542	9.9762079827913e-05\\
543	0.000112149796931475\\
544	0.000125067565715275\\
545	0.000138572067831741\\
546	0.00015273059037222\\
547	0.000167626377004082\\
548	0.000183363584696547\\
549	0.000200089867442313\\
550	0.000259094817245577\\
551	0.000488015275408033\\
552	0.000726542020146932\\
553	0.000975629408224585\\
554	0.00123636589032539\\
555	0.00150999973544108\\
556	0.00179648290113427\\
557	0.00209887107636958\\
558	0.00241929322761919\\
559	0.00276003182189189\\
560	0.00312210450741178\\
561	0.00349818716082299\\
562	0.00390375972898349\\
563	0.00433457550024097\\
564	0.00478459859119485\\
565	0.00525596815664969\\
566	0.00560062013606284\\
567	0.00582740021304894\\
568	0.00605605899399815\\
569	0.00628510015539956\\
570	0.00651276544529824\\
571	0.00673634212359634\\
572	0.0069520380325136\\
573	0.0071547202173806\\
574	0.00735717499191341\\
575	0.00756064332953579\\
576	0.0077619994239701\\
577	0.00795658404551624\\
578	0.00814198173030071\\
579	0.00831546642819351\\
580	0.00847403078801572\\
581	0.00861655664322103\\
582	0.00875071443575307\\
583	0.00887968108874371\\
584	0.00900452361312009\\
585	0.00912444387945188\\
586	0.0092381478141785\\
587	0.00934511367152276\\
588	0.009443897894158\\
589	0.0095340619381977\\
590	0.00961580002830814\\
591	0.00968902107613021\\
592	0.00975458929242902\\
593	0.00981291410532457\\
594	0.00986442025097488\\
595	0.00990933460668555\\
596	0.00994779104554705\\
597	0.00997906286423442\\
598	0.0099999191923403\\
599	0\\
600	0\\
};
\addplot [color=mycolor16,solid,forget plot]
  table[row sep=crcr]{%
1	2.93043087117404e-05\\
2	2.93043092703793e-05\\
3	2.93043098390084e-05\\
4	2.93043104178049e-05\\
5	2.93043110069512e-05\\
6	2.93043116066297e-05\\
7	2.93043122170296e-05\\
8	2.93043128383435e-05\\
9	2.9304313470764e-05\\
10	2.93043141144905e-05\\
11	2.93043147697259e-05\\
12	2.93043154366748e-05\\
13	2.93043161155449e-05\\
14	2.93043168065513e-05\\
15	2.93043175099102e-05\\
16	2.93043182258399e-05\\
17	2.93043189545671e-05\\
18	2.93043196963202e-05\\
19	2.93043204513311e-05\\
20	2.93043212198348e-05\\
21	2.93043220020752e-05\\
22	2.93043227982943e-05\\
23	2.93043236087444e-05\\
24	2.9304324433676e-05\\
25	2.93043252733517e-05\\
26	2.93043261280323e-05\\
27	2.93043269979852e-05\\
28	2.93043278834833e-05\\
29	2.93043287848044e-05\\
30	2.93043297022315e-05\\
31	2.93043306360509e-05\\
32	2.9304331586554e-05\\
33	2.93043325540409e-05\\
34	2.93043335388133e-05\\
35	2.93043345411797e-05\\
36	2.93043355614537e-05\\
37	2.93043365999557e-05\\
38	2.93043376570097e-05\\
39	2.9304338732948e-05\\
40	2.93043398281064e-05\\
41	2.93043409428274e-05\\
42	2.93043420774606e-05\\
43	2.93043432323622e-05\\
44	2.93043444078918e-05\\
45	2.93043456044193e-05\\
46	2.93043468223162e-05\\
47	2.93043480619679e-05\\
48	2.93043493237578e-05\\
49	2.93043506080848e-05\\
50	2.93043519153494e-05\\
51	2.9304353245959e-05\\
52	2.93043546003329e-05\\
53	2.93043559788922e-05\\
54	2.93043573820698e-05\\
55	2.93043588103037e-05\\
56	2.93043602640422e-05\\
57	2.93043617437405e-05\\
58	2.93043632498588e-05\\
59	2.93043647828691e-05\\
60	2.93043663432522e-05\\
61	2.93043679314956e-05\\
62	2.9304369548097e-05\\
63	2.93043711935609e-05\\
64	2.93043728684021e-05\\
65	2.93043745731439e-05\\
66	2.93043763083197e-05\\
67	2.93043780744734e-05\\
68	2.93043798721572e-05\\
69	2.93043817019318e-05\\
70	2.930438356437e-05\\
71	2.93043854600548e-05\\
72	2.93043873895775e-05\\
73	2.93043893535434e-05\\
74	2.93043913525642e-05\\
75	2.93043933872674e-05\\
76	2.93043954582886e-05\\
77	2.93043975662756e-05\\
78	2.93043997118847e-05\\
79	2.9304401895789e-05\\
80	2.93044041186721e-05\\
81	2.93044063812278e-05\\
82	2.93044086841617e-05\\
83	2.93044110281965e-05\\
84	2.93044134140635e-05\\
85	2.93044158425092e-05\\
86	2.93044183142939e-05\\
87	2.93044208301863e-05\\
88	2.93044233909772e-05\\
89	2.93044259974645e-05\\
90	2.93044286504645e-05\\
91	2.9304431350804e-05\\
92	2.930443409933e-05\\
93	2.93044368969e-05\\
94	2.93044397443884e-05\\
95	2.93044426426867e-05\\
96	2.93044455926998e-05\\
97	2.93044485953501e-05\\
98	2.93044516515749e-05\\
99	2.93044547623304e-05\\
100	2.93044579285901e-05\\
101	2.93044611513407e-05\\
102	2.93044644315931e-05\\
103	2.93044677703717e-05\\
104	2.93044711687196e-05\\
105	2.93044746276987e-05\\
106	2.93044781483897e-05\\
107	2.93044817318955e-05\\
108	2.93044853793341e-05\\
109	2.93044890918477e-05\\
110	2.93044928705969e-05\\
111	2.93044967167614e-05\\
112	2.93045006315444e-05\\
113	2.93045046161699e-05\\
114	2.93045086718838e-05\\
115	2.93045127999542e-05\\
116	2.93045170016716e-05\\
117	2.93045212783518e-05\\
118	2.93045256313294e-05\\
119	2.93045300619681e-05\\
120	2.93045345716518e-05\\
121	2.9304539161792e-05\\
122	2.93045438338238e-05\\
123	2.93045485892098e-05\\
124	2.93045534294344e-05\\
125	2.93045583560166e-05\\
126	2.93045633704937e-05\\
127	2.93045684744355e-05\\
128	2.9304573669441e-05\\
129	2.93045789571328e-05\\
130	2.93045843391694e-05\\
131	2.93045898172315e-05\\
132	2.93045953930355e-05\\
133	2.93046010683268e-05\\
134	2.93046068448816e-05\\
135	2.93046127245083e-05\\
136	2.93046187090495e-05\\
137	2.93046248003767e-05\\
138	2.9304631000399e-05\\
139	2.93046373110579e-05\\
140	2.93046437343306e-05\\
141	2.93046502722301e-05\\
142	2.93046569268019e-05\\
143	2.93046637001356e-05\\
144	2.930467059435e-05\\
145	2.93046776116064e-05\\
146	2.9304684754107e-05\\
147	2.93046920240916e-05\\
148	2.93046994238357e-05\\
149	2.93047069556625e-05\\
150	2.93047146219363e-05\\
151	2.93047224250586e-05\\
152	2.93047303674788e-05\\
153	2.93047384516907e-05\\
154	2.93047466802305e-05\\
155	2.93047550556807e-05\\
156	2.93047635806729e-05\\
157	2.93047722578833e-05\\
158	2.93047810900391e-05\\
159	2.93047900799118e-05\\
160	2.93047992303276e-05\\
161	2.93048085441635e-05\\
162	2.93048180243463e-05\\
163	2.93048276738572e-05\\
164	2.93048374957319e-05\\
165	2.93048474930573e-05\\
166	2.93048576689833e-05\\
167	2.93048680267076e-05\\
168	2.93048785694943e-05\\
169	2.93048893006622e-05\\
170	2.93049002235912e-05\\
171	2.93049113417229e-05\\
172	2.93049226585599e-05\\
173	2.93049341776698e-05\\
174	2.93049459026849e-05\\
175	2.93049578373041e-05\\
176	2.93049699852926e-05\\
177	2.93049823504821e-05\\
178	2.93049949367794e-05\\
179	2.93050077481577e-05\\
180	2.93050207886653e-05\\
181	2.93050340624202e-05\\
182	2.93050475736206e-05\\
183	2.93050613265381e-05\\
184	2.93050753255261e-05\\
185	2.9305089575011e-05\\
186	2.93051040795048e-05\\
187	2.93051188436029e-05\\
188	2.93051338719807e-05\\
189	2.93051491694023e-05\\
190	2.93051647407171e-05\\
191	2.93051805908648e-05\\
192	2.93051967248771e-05\\
193	2.93052131478743e-05\\
194	2.93052298650723e-05\\
195	2.93052468817857e-05\\
196	2.93052642034196e-05\\
197	2.93052818354881e-05\\
198	2.93052997835972e-05\\
199	2.9305318053464e-05\\
200	2.93053366509077e-05\\
201	2.93053555818531e-05\\
202	2.93053748523359e-05\\
203	2.93053944685061e-05\\
204	2.93054144366209e-05\\
205	2.93054347630603e-05\\
206	2.93054554543167e-05\\
207	2.93054765170072e-05\\
208	2.93054979578662e-05\\
209	2.93055197837596e-05\\
210	2.93055420016758e-05\\
211	2.93055646187347e-05\\
212	2.93055876421871e-05\\
213	2.93056110794205e-05\\
214	2.93056349379604e-05\\
215	2.93056592254701e-05\\
216	2.93056839497564e-05\\
217	2.93057091187743e-05\\
218	2.93057347406236e-05\\
219	2.93057608235576e-05\\
220	2.93057873759846e-05\\
221	2.93058144064681e-05\\
222	2.93058419237318e-05\\
223	2.93058699366683e-05\\
224	2.93058984543284e-05\\
225	2.93059274859404e-05\\
226	2.9305957040903e-05\\
227	2.93059871287905e-05\\
228	2.93060177593578e-05\\
229	2.93060489425456e-05\\
230	2.93060806884822e-05\\
231	2.93061130074815e-05\\
232	2.93061459100605e-05\\
233	2.93061794069286e-05\\
234	2.93062135089981e-05\\
235	2.93062482273944e-05\\
236	2.93062835734457e-05\\
237	2.93063195586984e-05\\
238	2.93063561949187e-05\\
239	2.93063934940963e-05\\
240	2.93064314684473e-05\\
241	2.93064701304198e-05\\
242	2.93065094927023e-05\\
243	2.93065495682236e-05\\
244	2.9306590370156e-05\\
245	2.93066319119261e-05\\
246	2.93066742072177e-05\\
247	2.93067172699755e-05\\
248	2.930676111441e-05\\
249	2.93068057550044e-05\\
250	2.930685120652e-05\\
251	2.93068974840007e-05\\
252	2.93069446027787e-05\\
253	2.93069925784809e-05\\
254	2.93070414270343e-05\\
255	2.93070911646745e-05\\
256	2.93071418079455e-05\\
257	2.93071933737134e-05\\
258	2.93072458791667e-05\\
259	2.93072993418294e-05\\
260	2.9307353779558e-05\\
261	2.93074092105568e-05\\
262	2.93074656533828e-05\\
263	2.93075231269477e-05\\
264	2.93075816505345e-05\\
265	2.93076412437948e-05\\
266	2.93077019267599e-05\\
267	2.93077637198516e-05\\
268	2.93078266438872e-05\\
269	2.93078907200863e-05\\
270	2.93079559700773e-05\\
271	2.93080224159134e-05\\
272	2.93080900800703e-05\\
273	2.93081589854635e-05\\
274	2.93082291554465e-05\\
275	2.93083006138332e-05\\
276	2.93083733848958e-05\\
277	2.93084474933771e-05\\
278	2.93085229644989e-05\\
279	2.93085998239753e-05\\
280	2.93086780980152e-05\\
281	2.93087578133384e-05\\
282	2.93088389971796e-05\\
283	2.93089216773039e-05\\
284	2.93090058820111e-05\\
285	2.9309091640152e-05\\
286	2.93091789811311e-05\\
287	2.93092679349241e-05\\
288	2.93093585320843e-05\\
289	2.93094508037568e-05\\
290	2.93095447816828e-05\\
291	2.93096404982194e-05\\
292	2.93097379863436e-05\\
293	2.93098372796669e-05\\
294	2.93099384124448e-05\\
295	2.93100414195927e-05\\
296	2.93101463366906e-05\\
297	2.93102532000003e-05\\
298	2.93103620464757e-05\\
299	2.93104729137747e-05\\
300	2.93105858402726e-05\\
301	2.93107008650693e-05\\
302	2.93108180280113e-05\\
303	2.93109373696914e-05\\
304	2.93110589314767e-05\\
305	2.93111827555043e-05\\
306	2.93113088847092e-05\\
307	2.93114373628293e-05\\
308	2.93115682344172e-05\\
309	2.93117015448571e-05\\
310	2.93118373403806e-05\\
311	2.93119756680697e-05\\
312	2.93121165758844e-05\\
313	2.93122601126605e-05\\
314	2.93124063281393e-05\\
315	2.93125552729684e-05\\
316	2.93127069987215e-05\\
317	2.93128615579127e-05\\
318	2.93130190040092e-05\\
319	2.93131793914446e-05\\
320	2.93133427756377e-05\\
321	2.93135092130009e-05\\
322	2.93136787609609e-05\\
323	2.93138514779671e-05\\
324	2.93140274235173e-05\\
325	2.93142066581623e-05\\
326	2.93143892435288e-05\\
327	2.93145752423309e-05\\
328	2.93147647183921e-05\\
329	2.93149577366578e-05\\
330	2.9315154363217e-05\\
331	2.93153546653141e-05\\
332	2.93155587113752e-05\\
333	2.93157665710192e-05\\
334	2.9315978315082e-05\\
335	2.93161940156423e-05\\
336	2.93164137460263e-05\\
337	2.93166375808508e-05\\
338	2.93168655960277e-05\\
339	2.93170978687985e-05\\
340	2.93173344777514e-05\\
341	2.93175755028535e-05\\
342	2.93178210254676e-05\\
343	2.93180711283923e-05\\
344	2.93183258958727e-05\\
345	2.93185854136492e-05\\
346	2.93188497689687e-05\\
347	2.93191190506294e-05\\
348	2.93193933490078e-05\\
349	2.93196727560963e-05\\
350	2.93199573655373e-05\\
351	2.9320247272664e-05\\
352	2.93205425745397e-05\\
353	2.93208433700023e-05\\
354	2.93211497597115e-05\\
355	2.93214618461989e-05\\
356	2.93217797339203e-05\\
357	2.93221035293106e-05\\
358	2.93224333408535e-05\\
359	2.93227692791394e-05\\
360	2.93231114569356e-05\\
361	2.93234599892746e-05\\
362	2.93238149935226e-05\\
363	2.93241765894779e-05\\
364	2.93245448994635e-05\\
365	2.93249200484307e-05\\
366	2.93253021640719e-05\\
367	2.93256913769481e-05\\
368	2.93260878206101e-05\\
369	2.93264916317505e-05\\
370	2.9326902950348e-05\\
371	2.93273219198469e-05\\
372	2.9327748687329e-05\\
373	2.93281834037115e-05\\
374	2.93286262239667e-05\\
375	2.93290773073439e-05\\
376	2.93295368176299e-05\\
377	2.93300049234186e-05\\
378	2.93304817984106e-05\\
379	2.9330967621743e-05\\
380	2.93314625783413e-05\\
381	2.93319668593139e-05\\
382	2.93324806623867e-05\\
383	2.93330041923679e-05\\
384	2.93335376616909e-05\\
385	2.93340812909987e-05\\
386	2.93346353098155e-05\\
387	2.93351999573295e-05\\
388	2.93357754832963e-05\\
389	2.93363621491538e-05\\
390	2.9336960229354e-05\\
391	2.9337570012906e-05\\
392	2.93381918048516e-05\\
393	2.93388259268588e-05\\
394	2.9339472715923e-05\\
395	2.93401325229076e-05\\
396	2.93408057222732e-05\\
397	2.93414927158237e-05\\
398	2.93421939278154e-05\\
399	2.93429098063191e-05\\
400	2.9343640824428e-05\\
401	2.93443874811739e-05\\
402	2.93451503019373e-05\\
403	2.9345929838172e-05\\
404	2.93467266666941e-05\\
405	2.93475413900327e-05\\
406	2.93483746418173e-05\\
407	2.93492271025257e-05\\
408	2.93500995193379e-05\\
409	2.93509926816469e-05\\
410	2.93519073836872e-05\\
411	2.93528444802147e-05\\
412	2.93538048938463e-05\\
413	2.93547896235088e-05\\
414	2.93557997541904e-05\\
415	2.93568364682357e-05\\
416	2.93579010584453e-05\\
417	2.9358994943368e-05\\
418	2.93601196852197e-05\\
419	2.93612770111704e-05\\
420	2.93624688391245e-05\\
421	2.93636973104335e-05\\
422	2.93649648350278e-05\\
423	2.93662741629849e-05\\
424	2.93676285195353e-05\\
425	2.93690319025862e-05\\
426	2.93704898071103e-05\\
427	2.9372011068031e-05\\
428	2.93736125641227e-05\\
429	2.93753308720823e-05\\
430	2.93772491091177e-05\\
431	2.937954906549e-05\\
432	2.93825649121727e-05\\
433	2.93865647993176e-05\\
434	2.93906580715017e-05\\
435	2.93948477653989e-05\\
436	2.93991370711979e-05\\
437	2.94035293415676e-05\\
438	2.94080281007108e-05\\
439	2.94126370534127e-05\\
440	2.94173600941866e-05\\
441	2.94222013171684e-05\\
442	2.94271650294862e-05\\
443	2.94322557769839e-05\\
444	2.94374784106811e-05\\
445	2.94428382829006e-05\\
446	2.94483418539612e-05\\
447	2.94539986059908e-05\\
448	2.94598271691304e-05\\
449	2.9465875226017e-05\\
450	2.9472285212237e-05\\
451	2.94795147135854e-05\\
452	2.94890877018201e-05\\
453	2.95061946780885e-05\\
454	2.97137528650851e-05\\
455	2.99643177527541e-05\\
456	3.02198907130476e-05\\
457	3.04805329288714e-05\\
458	3.07463097120086e-05\\
459	3.10172934108331e-05\\
460	3.12935673858202e-05\\
461	3.15752306426986e-05\\
462	3.18623999270261e-05\\
463	3.21551991485585e-05\\
464	3.24537260405704e-05\\
465	3.2758119079863e-05\\
466	3.3068553554098e-05\\
467	3.33852152092756e-05\\
468	3.37083005507875e-05\\
469	3.40380176623664e-05\\
470	3.43745878109453e-05\\
471	3.47182432694875e-05\\
472	3.50692271046537e-05\\
473	3.54277931937303e-05\\
474	3.57942075380966e-05\\
475	3.61687510396599e-05\\
476	3.65517064958356e-05\\
477	3.69433747872977e-05\\
478	3.73440764709209e-05\\
479	3.77541535261776e-05\\
480	3.817397129973e-05\\
481	3.86039206799129e-05\\
482	3.90444205384928e-05\\
483	3.9495920483335e-05\\
484	3.99589039722751e-05\\
485	4.04338918478452e-05\\
486	4.09214463707476e-05\\
487	4.14221758634367e-05\\
488	4.19367400822611e-05\\
489	4.24658563771034e-05\\
490	4.30103069572825e-05\\
491	4.3570947296266e-05\\
492	4.41487160335469e-05\\
493	4.47446463784222e-05\\
494	4.53598795319635e-05\\
495	4.59956810291058e-05\\
496	4.66534600498183e-05\\
497	4.73347933366168e-05\\
498	4.80414565567155e-05\\
499	4.87754703249452e-05\\
500	4.9539181367542e-05\\
501	5.033544070285e-05\\
502	5.11680726456097e-05\\
503	5.20432562481787e-05\\
504	5.29738500783278e-05\\
505	5.39938867634016e-05\\
506	5.87673186764253e-05\\
507	6.4606815017149e-05\\
508	7.0590274701469e-05\\
509	7.67252049983633e-05\\
510	8.3019714326634e-05\\
511	8.9481993182314e-05\\
512	9.61209472369967e-05\\
513	0.000102946251576546\\
514	0.000109968443091134\\
515	0.000117199021355419\\
516	0.000124650541318546\\
517	0.000132336703579475\\
518	0.000140272419398281\\
519	0.000148473773838568\\
520	0.000156957625963504\\
521	0.000165739941223944\\
522	0.000174829676150907\\
523	0.000183950961874965\\
524	0.000192825752632742\\
525	0.000202059447372597\\
526	0.000211688618436699\\
527	0.000221755970778601\\
528	0.000232312204801424\\
529	0.000243420847026636\\
530	0.000255169619523371\\
531	0.000267702889893964\\
532	0.000412746538030032\\
533	0.000576195888250494\\
534	0.000745765199194563\\
535	0.000921788564508206\\
536	0.00110457820815963\\
537	0.0012943489400593\\
538	0.00148633721617432\\
539	0.00168191902455894\\
540	0.00188511074741496\\
541	0.00209666253199608\\
542	0.00231744598211593\\
543	0.00254847920196742\\
544	0.00279095733672176\\
545	0.00304628750794034\\
546	0.00331615021568388\\
547	0.0036026079092931\\
548	0.00390820819911889\\
549	0.00423590450669407\\
550	0.00454732922748713\\
551	0.00471530322788151\\
552	0.00489204389082255\\
553	0.00507209484026282\\
554	0.00525506600206778\\
555	0.00544042211683775\\
556	0.00562744957463521\\
557	0.00581514029179541\\
558	0.0060021940002762\\
559	0.00618708573928175\\
560	0.00636777611433338\\
561	0.00654145229614157\\
562	0.00670414805441769\\
563	0.00686023270116887\\
564	0.00701594208881994\\
565	0.00716944104891064\\
566	0.00731902377334951\\
567	0.00746707784675243\\
568	0.00761602253010009\\
569	0.00776477719611707\\
570	0.00791099048273329\\
571	0.00805137960956685\\
572	0.0081854601127217\\
573	0.00831318779882302\\
574	0.0084344350629761\\
575	0.00854874215821512\\
576	0.00865665372359101\\
577	0.00876045240599831\\
578	0.0088600806646344\\
579	0.00895569983895539\\
580	0.00904732770420615\\
581	0.00913419155627037\\
582	0.00921661474496478\\
583	0.00929486551170983\\
584	0.00936833250999321\\
585	0.00943643823988877\\
586	0.00949976142125213\\
587	0.00955824735515365\\
588	0.00961252025824673\\
589	0.00966324058718606\\
590	0.00971085603739681\\
591	0.00975602317557434\\
592	0.00979895351145169\\
593	0.00983980399718307\\
594	0.00987864202528269\\
595	0.00991535020799166\\
596	0.00994937493726701\\
597	0.00997906286423442\\
598	0.0099999191923403\\
599	0\\
600	0\\
};
\addplot [color=mycolor17,solid,forget plot]
  table[row sep=crcr]{%
1	4.85195228051361e-05\\
2	4.85195905472055e-05\\
3	4.85196595005982e-05\\
4	4.85197296869432e-05\\
5	4.85198011282528e-05\\
6	4.85198738469381e-05\\
7	4.85199478658032e-05\\
8	4.85200232080614e-05\\
9	4.85200998973416e-05\\
10	4.8520177957688e-05\\
11	4.85202574135813e-05\\
12	4.85203382899344e-05\\
13	4.85204206121018e-05\\
14	4.85205044058957e-05\\
15	4.85205896975854e-05\\
16	4.85206765139064e-05\\
17	4.8520764882073e-05\\
18	4.85208548297851e-05\\
19	4.85209463852382e-05\\
20	4.85210395771233e-05\\
21	4.85211344346477e-05\\
22	4.85212309875416e-05\\
23	4.85213292660615e-05\\
24	4.85214293010004e-05\\
25	4.85215311237068e-05\\
26	4.85216347660825e-05\\
27	4.85217402606018e-05\\
28	4.8521847640313e-05\\
29	4.85219569388589e-05\\
30	4.85220681904767e-05\\
31	4.85221814300167e-05\\
32	4.85222966929513e-05\\
33	4.85224140153793e-05\\
34	4.85225334340487e-05\\
35	4.85226549863615e-05\\
36	4.85227787103824e-05\\
37	4.85229046448591e-05\\
38	4.85230328292256e-05\\
39	4.85231633036214e-05\\
40	4.85232961088994e-05\\
41	4.85234312866416e-05\\
42	4.85235688791694e-05\\
43	4.85237089295637e-05\\
44	4.85238514816636e-05\\
45	4.85239965800966e-05\\
46	4.85241442702842e-05\\
47	4.85242945984534e-05\\
48	4.85244476116607e-05\\
49	4.85246033577975e-05\\
50	4.85247618856084e-05\\
51	4.85249232447049e-05\\
52	4.85250874855878e-05\\
53	4.85252546596556e-05\\
54	4.85254248192179e-05\\
55	4.85255980175213e-05\\
56	4.85257743087629e-05\\
57	4.85259537480985e-05\\
58	4.85261363916739e-05\\
59	4.85263222966311e-05\\
60	4.85265115211309e-05\\
61	4.85267041243712e-05\\
62	4.85269001666046e-05\\
63	4.85270997091599e-05\\
64	4.85273028144527e-05\\
65	4.85275095460144e-05\\
66	4.85277199685125e-05\\
67	4.85279341477591e-05\\
68	4.8528152150745e-05\\
69	4.85283740456517e-05\\
70	4.85285999018786e-05\\
71	4.852882979006e-05\\
72	4.85290637820906e-05\\
73	4.85293019511444e-05\\
74	4.85295443717018e-05\\
75	4.85297911195685e-05\\
76	4.85300422719043e-05\\
77	4.85302979072419e-05\\
78	4.85305581055176e-05\\
79	4.85308229480864e-05\\
80	4.85310925177616e-05\\
81	4.85313668988247e-05\\
82	4.85316461770697e-05\\
83	4.85319304398101e-05\\
84	4.85322197759229e-05\\
85	4.85325142758694e-05\\
86	4.85328140317205e-05\\
87	4.85331191371924e-05\\
88	4.85334296876743e-05\\
89	4.85337457802515e-05\\
90	4.85340675137454e-05\\
91	4.85343949887367e-05\\
92	4.85347283076049e-05\\
93	4.85350675745539e-05\\
94	4.8535412895644e-05\\
95	4.85357643788334e-05\\
96	4.85361221340029e-05\\
97	4.85364862729943e-05\\
98	4.85368569096487e-05\\
99	4.85372341598347e-05\\
100	4.85376181414916e-05\\
101	4.85380089746661e-05\\
102	4.85384067815443e-05\\
103	4.85388116864975e-05\\
104	4.85392238161132e-05\\
105	4.85396432992459e-05\\
106	4.85400702670465e-05\\
107	4.8540504853011e-05\\
108	4.85409471930187e-05\\
109	4.85413974253778e-05\\
110	4.85418556908663e-05\\
111	4.85423221327831e-05\\
112	4.85427968969806e-05\\
113	4.85432801319221e-05\\
114	4.85437719887268e-05\\
115	4.85442726212134e-05\\
116	4.854478218595e-05\\
117	4.85453008423083e-05\\
118	4.8545828752508e-05\\
119	4.85463660816713e-05\\
120	4.85469129978757e-05\\
121	4.85474696722051e-05\\
122	4.85480362788079e-05\\
123	4.85486129949458e-05\\
124	4.85492000010577e-05\\
125	4.85497974808099e-05\\
126	4.85504056211634e-05\\
127	4.85510246124225e-05\\
128	4.85516546483036e-05\\
129	4.85522959259959e-05\\
130	4.85529486462182e-05\\
131	4.85536130132935e-05\\
132	4.85542892352019e-05\\
133	4.8554977523657e-05\\
134	4.85556780941695e-05\\
135	4.85563911661148e-05\\
136	4.85571169628064e-05\\
137	4.85578557115676e-05\\
138	4.85586076437961e-05\\
139	4.85593729950541e-05\\
140	4.85601520051265e-05\\
141	4.85609449181111e-05\\
142	4.85617519824902e-05\\
143	4.85625734512119e-05\\
144	4.85634095817727e-05\\
145	4.85642606363002e-05\\
146	4.85651268816352e-05\\
147	4.85660085894219e-05\\
148	4.8566906036195e-05\\
149	4.85678195034628e-05\\
150	4.85687492778094e-05\\
151	4.85696956509753e-05\\
152	4.85706589199607e-05\\
153	4.85716393871228e-05\\
154	4.85726373602646e-05\\
155	4.85736531527433e-05\\
156	4.85746870835716e-05\\
157	4.85757394775177e-05\\
158	4.85768106652127e-05\\
159	4.85779009832611e-05\\
160	4.85790107743469e-05\\
161	4.8580140387351e-05\\
162	4.85812901774596e-05\\
163	4.85824605062826e-05\\
164	4.85836517419741e-05\\
165	4.85848642593533e-05\\
166	4.85860984400237e-05\\
167	4.85873546725047e-05\\
168	4.85886333523554e-05\\
169	4.8589934882306e-05\\
170	4.85912596723923e-05\\
171	4.85926081400904e-05\\
172	4.85939807104564e-05\\
173	4.85953778162622e-05\\
174	4.85967998981494e-05\\
175	4.85982474047637e-05\\
176	4.85997207929129e-05\\
177	4.86012205277192e-05\\
178	4.86027470827684e-05\\
179	4.86043009402722e-05\\
180	4.86058825912316e-05\\
181	4.86074925355984e-05\\
182	4.86091312824445e-05\\
183	4.86107993501316e-05\\
184	4.86124972664885e-05\\
185	4.86142255689852e-05\\
186	4.86159848049198e-05\\
187	4.86177755315993e-05\\
188	4.86195983165268e-05\\
189	4.86214537376046e-05\\
190	4.86233423833176e-05\\
191	4.86252648529382e-05\\
192	4.8627221756737e-05\\
193	4.8629213716181e-05\\
194	4.86312413641528e-05\\
195	4.8633305345169e-05\\
196	4.86354063155928e-05\\
197	4.86375449438675e-05\\
198	4.8639721910745e-05\\
199	4.86419379095203e-05\\
200	4.8644193646269e-05\\
201	4.86464898400934e-05\\
202	4.86488272233737e-05\\
203	4.86512065420214e-05\\
204	4.86536285557313e-05\\
205	4.86560940382629e-05\\
206	4.86586037776941e-05\\
207	4.86611585767051e-05\\
208	4.86637592528588e-05\\
209	4.86664066388893e-05\\
210	4.86691015829889e-05\\
211	4.86718449491142e-05\\
212	4.86746376172859e-05\\
213	4.8677480483906e-05\\
214	4.86803744620675e-05\\
215	4.86833204818831e-05\\
216	4.86863194908192e-05\\
217	4.86893724540281e-05\\
218	4.86924803546937e-05\\
219	4.86956441943839e-05\\
220	4.86988649934139e-05\\
221	4.87021437912051e-05\\
222	4.87054816466617e-05\\
223	4.87088796385576e-05\\
224	4.87123388659159e-05\\
225	4.87158604484162e-05\\
226	4.87194455267965e-05\\
227	4.87230952632638e-05\\
228	4.87268108419249e-05\\
229	4.87305934692125e-05\\
230	4.87344443743226e-05\\
231	4.87383648096746e-05\\
232	4.87423560513576e-05\\
233	4.87464193996125e-05\\
234	4.87505561792985e-05\\
235	4.87547677403838e-05\\
236	4.87590554584498e-05\\
237	4.87634207351953e-05\\
238	4.87678649989508e-05\\
239	4.87723897052186e-05\\
240	4.87769963372058e-05\\
241	4.87816864063762e-05\\
242	4.87864614530179e-05\\
243	4.87913230468166e-05\\
244	4.87962727874388e-05\\
245	4.88013123051327e-05\\
246	4.88064432613451e-05\\
247	4.88116673493306e-05\\
248	4.88169862948016e-05\\
249	4.88224018565728e-05\\
250	4.8827915827221e-05\\
251	4.88335300337648e-05\\
252	4.88392463383471e-05\\
253	4.8845066638951e-05\\
254	4.88509928701049e-05\\
255	4.88570270036215e-05\\
256	4.88631710493406e-05\\
257	4.88694270558925e-05\\
258	4.88757971114794e-05\\
259	4.88822833446537e-05\\
260	4.88888879251433e-05\\
261	4.88956130646577e-05\\
262	4.89024610177392e-05\\
263	4.89094340826172e-05\\
264	4.89165346020737e-05\\
265	4.89237649643376e-05\\
266	4.89311276039891e-05\\
267	4.89386250028848e-05\\
268	4.89462596910926e-05\\
269	4.89540342478559e-05\\
270	4.89619513025609e-05\\
271	4.89700135357422e-05\\
272	4.89782236800782e-05\\
273	4.89865845214382e-05\\
274	4.89950988999169e-05\\
275	4.90037697109065e-05\\
276	4.90125999061813e-05\\
277	4.90215924950037e-05\\
278	4.90307505452409e-05\\
279	4.90400771845188e-05\\
280	4.9049575601366e-05\\
281	4.90592490464081e-05\\
282	4.90691008335636e-05\\
283	4.9079134341253e-05\\
284	4.90893530136492e-05\\
285	4.90997603619277e-05\\
286	4.9110359965539e-05\\
287	4.91211554735117e-05\\
288	4.91321506057572e-05\\
289	4.91433491544102e-05\\
290	4.91547549851743e-05\\
291	4.91663720386933e-05\\
292	4.9178204331943e-05\\
293	4.91902559596341e-05\\
294	4.9202531095637e-05\\
295	4.92150339944272e-05\\
296	4.9227768992544e-05\\
297	4.92407405100684e-05\\
298	4.92539530521209e-05\\
299	4.92674112103702e-05\\
300	4.92811196645611e-05\\
301	4.92950831840634e-05\\
302	4.93093066294312e-05\\
303	4.93237949539777e-05\\
304	4.93385532053724e-05\\
305	4.93535865272458e-05\\
306	4.93689001608142e-05\\
307	4.93844994465169e-05\\
308	4.94003898256681e-05\\
309	4.941657684213e-05\\
310	4.94330661439883e-05\\
311	4.9449863485256e-05\\
312	4.94669747275817e-05\\
313	4.94844058419817e-05\\
314	4.95021629105794e-05\\
315	4.95202521283679e-05\\
316	4.9538679804987e-05\\
317	4.9557452366518e-05\\
318	4.95765763572965e-05\\
319	4.95960584417466e-05\\
320	4.96159054062375e-05\\
321	4.96361241609624e-05\\
322	4.9656721741848e-05\\
323	4.96777053124757e-05\\
324	4.96990821660607e-05\\
325	4.97208597274403e-05\\
326	4.97430455551105e-05\\
327	4.9765647343306e-05\\
328	4.97886729241193e-05\\
329	4.98121302696654e-05\\
330	4.98360274943216e-05\\
331	4.98603728569967e-05\\
332	4.98851747634897e-05\\
333	4.99104417689053e-05\\
334	4.99361825801509e-05\\
335	4.99624060585073e-05\\
336	4.9989121222297e-05\\
337	5.00163372496359e-05\\
338	5.00440634812877e-05\\
339	5.00723094236053e-05\\
340	5.01010847516042e-05\\
341	5.01303993121218e-05\\
342	5.01602631270978e-05\\
343	5.01906863969831e-05\\
344	5.02216795042634e-05\\
345	5.02532530171252e-05\\
346	5.02854176932574e-05\\
347	5.03181844838271e-05\\
348	5.03515645375995e-05\\
349	5.03855692052767e-05\\
350	5.04202100440215e-05\\
351	5.04554988222424e-05\\
352	5.04914475246349e-05\\
353	5.05280683575287e-05\\
354	5.05653737545819e-05\\
355	5.06033763828397e-05\\
356	5.06420891492121e-05\\
357	5.06815252073879e-05\\
358	5.07216979652491e-05\\
359	5.07626210928243e-05\\
360	5.08043085308115e-05\\
361	5.08467744997615e-05\\
362	5.08900335099431e-05\\
363	5.09341003719842e-05\\
364	5.09789902083251e-05\\
365	5.10247184655912e-05\\
366	5.10713009279394e-05\\
367	5.11187537314744e-05\\
368	5.11670933798314e-05\\
369	5.12163367610239e-05\\
370	5.12665011656658e-05\\
371	5.13176043066897e-05\\
372	5.13696643406997e-05\\
373	5.14226998910996e-05\\
374	5.14767300731561e-05\\
375	5.15317745211822e-05\\
376	5.15878534180362e-05\\
377	5.1644987527163e-05\\
378	5.17031982274314e-05\\
379	5.17625075510554e-05\\
380	5.18229382249418e-05\\
381	5.18845137158304e-05\\
382	5.19472582797052e-05\\
383	5.20111970159642e-05\\
384	5.20763559270271e-05\\
385	5.2142761984092e-05\\
386	5.22104431999679e-05\\
387	5.22794287101352e-05\\
388	5.23497488634706e-05\\
389	5.24214353245409e-05\\
390	5.24945211900758e-05\\
391	5.25690411231054e-05\\
392	5.26450315089373e-05\\
393	5.27225306351912e-05\\
394	5.28015788873556e-05\\
395	5.28822189220311e-05\\
396	5.29644957750493e-05\\
397	5.30484573727744e-05\\
398	5.31341547007723e-05\\
399	5.3221641628035e-05\\
400	5.33109751041869e-05\\
401	5.34022153600259e-05\\
402	5.34954261059868e-05\\
403	5.35906747198619e-05\\
404	5.36880324110592e-05\\
405	5.37875743466039e-05\\
406	5.38893797348472e-05\\
407	5.39935319156555e-05\\
408	5.41001186388268e-05\\
409	5.42092327617706e-05\\
410	5.4320971769912e-05\\
411	5.44354371495247e-05\\
412	5.45527380881619e-05\\
413	5.46729923971985e-05\\
414	5.47963275720986e-05\\
415	5.49228820134563e-05\\
416	5.50528064366033e-05\\
417	5.51862655029529e-05\\
418	5.53234397127437e-05\\
419	5.54645276068821e-05\\
420	5.56097483355481e-05\\
421	5.57593446642293e-05\\
422	5.5913586505215e-05\\
423	5.60727750903251e-05\\
424	5.6237247952476e-05\\
425	5.6407385006861e-05\\
426	5.65836163645633e-05\\
427	5.67664335585839e-05\\
428	5.69564092393166e-05\\
429	5.71542416273935e-05\\
430	5.73608781483743e-05\\
431	5.75779043230569e-05\\
432	5.78088449023424e-05\\
433	5.81528195365713e-05\\
434	5.944600411787e-05\\
435	6.07708016525669e-05\\
436	6.21282928284642e-05\\
437	6.35196126365089e-05\\
438	6.4945952967713e-05\\
439	6.64085650019074e-05\\
440	6.7908761246117e-05\\
441	6.94479170315175e-05\\
442	7.10274712130791e-05\\
443	7.26489257282991e-05\\
444	7.4313843546243e-05\\
445	7.60238443407499e-05\\
446	7.77805968532678e-05\\
447	7.95858060671009e-05\\
448	8.14411910082956e-05\\
449	8.33484419114667e-05\\
450	8.53091223231798e-05\\
451	8.7324403025868e-05\\
452	8.93942409883346e-05\\
453	9.15146481566733e-05\\
454	9.35035181938711e-05\\
455	9.55063189739746e-05\\
456	9.75619116404759e-05\\
457	9.96717437119328e-05\\
458	0.000101837158122837\\
459	0.000104059334890113\\
460	0.000106339220398147\\
461	0.000108677450218538\\
462	0.000111074278278899\\
463	0.000113529500974515\\
464	0.000116042112437598\\
465	0.000118607941916725\\
466	0.000121225878689891\\
467	0.000123897386685402\\
468	0.000126624013333797\\
469	0.000129407381553128\\
470	0.000132249194157849\\
471	0.000135151309566943\\
472	0.000138115684211088\\
473	0.000141144369207877\\
474	0.000144239505589599\\
475	0.000147403349520146\\
476	0.000150638493100372\\
477	0.000153947246360344\\
478	0.000157332063585502\\
479	0.000160795558711385\\
480	0.000164340522107563\\
481	0.000167969939549219\\
482	0.000171687013726675\\
483	0.00017549518871605\\
484	0.000179398177915557\\
485	0.000183399996029068\\
486	0.000187504995758643\\
487	0.000191717910133522\\
488	0.000196043902545426\\
489	0.00020048862747069\\
490	0.000205058300007985\\
491	0.000209759786026714\\
492	0.000214600706085218\\
493	0.000219589567095355\\
494	0.000224735913396954\\
495	0.000230050508790892\\
496	0.000235545583825447\\
497	0.000241235118228053\\
498	0.000247135186952318\\
499	0.00025326438473167\\
500	0.00025964434531677\\
501	0.000266300365657899\\
502	0.000273262109020108\\
503	0.000280564206036033\\
504	0.000288245964501537\\
505	0.000296341669277401\\
506	0.000308549128404662\\
507	0.000401301320030103\\
508	0.000496495979642583\\
509	0.00059424841886353\\
510	0.000694694016538126\\
511	0.000797978055842614\\
512	0.000904240215439014\\
513	0.00101363016259423\\
514	0.00112630667846547\\
515	0.00124243940537317\\
516	0.00136221071779773\\
517	0.00148581664666537\\
518	0.00161346774579206\\
519	0.0017453898123079\\
520	0.00188182432228271\\
521	0.00202302843785282\\
522	0.00216927474552162\\
523	0.00232085412302543\\
524	0.0024780776132213\\
525	0.00264125873975925\\
526	0.00281073686921825\\
527	0.00298714697226363\\
528	0.00317122809954563\\
529	0.00336384916330986\\
530	0.00356606087372385\\
531	0.00377911073125441\\
532	0.00387101444814595\\
533	0.00395496570181997\\
534	0.00404411861485679\\
535	0.0041393541562261\\
536	0.00424177554848268\\
537	0.0043527725456736\\
538	0.00447412533780575\\
539	0.00460796840269658\\
540	0.00474871272796527\\
541	0.00489195363004134\\
542	0.00503735684616973\\
543	0.00518445717205239\\
544	0.00533261860337901\\
545	0.00548098202337772\\
546	0.00562839635531924\\
547	0.00577331960216846\\
548	0.00591379893350599\\
549	0.00604742138754942\\
550	0.00617093021530702\\
551	0.00628179603343684\\
552	0.00639010919318062\\
553	0.00650075390556414\\
554	0.00661355826910363\\
555	0.00672831871640315\\
556	0.00684480418700696\\
557	0.00696276543345935\\
558	0.00708195602380279\\
559	0.00720220504478184\\
560	0.007323419518955\\
561	0.00744556801198611\\
562	0.00756874620780185\\
563	0.00769285619681815\\
564	0.00781677314052897\\
565	0.00793628932779018\\
566	0.00805088736053623\\
567	0.00816001629554882\\
568	0.00826313047911087\\
569	0.00835993460475097\\
570	0.00845129433017152\\
571	0.00853958906049813\\
572	0.00862495123378327\\
573	0.00870767807453262\\
574	0.00878809121736665\\
575	0.00886661500378345\\
576	0.00894286737083422\\
577	0.00901626113058295\\
578	0.0090866482592975\\
579	0.00915386945533433\\
580	0.00921808471957653\\
581	0.00927992511337708\\
582	0.00933845256431125\\
583	0.00939359549483182\\
584	0.0094456284309534\\
585	0.00949555685866655\\
586	0.00954333369779573\\
587	0.00958916916241128\\
588	0.00963370188398624\\
589	0.00967711629413547\\
590	0.00971958295016231\\
591	0.00976110413435579\\
592	0.00980161112056955\\
593	0.00984099058333643\\
594	0.00987905095617487\\
595	0.00991543381058343\\
596	0.00994937493726701\\
597	0.00997906286423442\\
598	0.0099999191923403\\
599	0\\
600	0\\
};
\addplot [color=mycolor18,solid,forget plot]
  table[row sep=crcr]{%
1	0.000169140775378202\\
2	0.000169141360870221\\
3	0.000169141956833748\\
4	0.000169142563455799\\
5	0.000169143180926731\\
6	0.000169143809440294\\
7	0.000169144449193693\\
8	0.00016914510038765\\
9	0.000169145763226465\\
10	0.000169146437918081\\
11	0.000169147124674152\\
12	0.0001691478237101\\
13	0.000169148535245194\\
14	0.000169149259502606\\
15	0.000169149996709493\\
16	0.000169150747097061\\
17	0.00016915151090063\\
18	0.000169152288359729\\
19	0.000169153079718147\\
20	0.000169153885224021\\
21	0.000169154705129912\\
22	0.000169155539692885\\
23	0.000169156389174587\\
24	0.000169157253841328\\
25	0.000169158133964165\\
26	0.000169159029818986\\
27	0.000169159941686601\\
28	0.00016916086985282\\
29	0.00016916181460855\\
30	0.000169162776249881\\
31	0.000169163755078186\\
32	0.000169164751400206\\
33	0.000169165765528146\\
34	0.000169166797779781\\
35	0.000169167848478545\\
36	0.000169168917953642\\
37	0.000169170006540139\\
38	0.000169171114579074\\
39	0.000169172242417566\\
40	0.000169173390408922\\
41	0.000169174558912741\\
42	0.000169175748295031\\
43	0.000169176958928325\\
44	0.000169178191191797\\
45	0.000169179445471374\\
46	0.000169180722159865\\
47	0.000169182021657074\\
48	0.000169183344369936\\
49	0.000169184690712634\\
50	0.000169186061106733\\
51	0.000169187455981313\\
52	0.000169188875773099\\
53	0.000169190320926595\\
54	0.000169191791894236\\
55	0.00016919328913651\\
56	0.00016919481312212\\
57	0.000169196364328115\\
58	0.00016919794324005\\
59	0.000169199550352132\\
60	0.000169201186167376\\
61	0.00016920285119776\\
62	0.00016920454596439\\
63	0.000169206270997654\\
64	0.000169208026837402\\
65	0.000169209814033093\\
66	0.000169211633143991\\
67	0.000169213484739319\\
68	0.000169215369398452\\
69	0.000169217287711087\\
70	0.000169219240277436\\
71	0.000169221227708407\\
72	0.000169223250625795\\
73	0.000169225309662487\\
74	0.000169227405462643\\
75	0.00016922953868191\\
76	0.000169231709987623\\
77	0.000169233920059015\\
78	0.000169236169587427\\
79	0.000169238459276524\\
80	0.00016924078984252\\
81	0.000169243162014397\\
82	0.000169245576534136\\
83	0.000169248034156946\\
84	0.000169250535651508\\
85	0.000169253081800205\\
86	0.000169255673399376\\
87	0.000169258311259563\\
88	0.000169260996205759\\
89	0.000169263729077678\\
90	0.000169266510730009\\
91	0.000169269342032687\\
92	0.000169272223871166\\
93	0.000169275157146695\\
94	0.000169278142776606\\
95	0.000169281181694597\\
96	0.000169284274851023\\
97	0.000169287423213203\\
98	0.000169290627765713\\
99	0.000169293889510708\\
100	0.000169297209468223\\
101	0.000169300588676507\\
102	0.000169304028192337\\
103	0.000169307529091361\\
104	0.000169311092468429\\
105	0.000169314719437943\\
106	0.000169318411134204\\
107	0.00016932216871177\\
108	0.000169325993345824\\
109	0.000169329886232535\\
110	0.000169333848589447\\
111	0.000169337881655853\\
112	0.000169341986693192\\
113	0.000169346164985444\\
114	0.000169350417839538\\
115	0.000169354746585762\\
116	0.000169359152578186\\
117	0.000169363637195088\\
118	0.000169368201839394\\
119	0.000169372847939113\\
120	0.000169377576947802\\
121	0.000169382390345014\\
122	0.000169387289636777\\
123	0.000169392276356063\\
124	0.000169397352063277\\
125	0.000169402518346758\\
126	0.000169407776823276\\
127	0.000169413129138547\\
128	0.00016941857696776\\
129	0.000169424122016103\\
130	0.000169429766019315\\
131	0.000169435510744231\\
132	0.000169441357989345\\
133	0.000169447309585389\\
134	0.000169453367395912\\
135	0.000169459533317877\\
136	0.000169465809282265\\
137	0.000169472197254695\\
138	0.000169478699236051\\
139	0.000169485317263119\\
140	0.000169492053409246\\
141	0.000169498909785002\\
142	0.000169505888538853\\
143	0.000169512991857853\\
144	0.000169520221968349\\
145	0.000169527581136695\\
146	0.000169535071669976\\
147	0.000169542695916759\\
148	0.000169550456267846\\
149	0.000169558355157043\\
150	0.000169566395061947\\
151	0.000169574578504747\\
152	0.000169582908053038\\
153	0.000169591386320653\\
154	0.000169600015968509\\
155	0.000169608799705471\\
156	0.000169617740289225\\
157	0.000169626840527181\\
158	0.00016963610327738\\
159	0.000169645531449434\\
160	0.000169655128005455\\
161	0.000169664895961041\\
162	0.000169674838386244\\
163	0.000169684958406587\\
164	0.000169695259204072\\
165	0.000169705744018236\\
166	0.000169716416147201\\
167	0.000169727278948765\\
168	0.000169738335841502\\
169	0.000169749590305881\\
170	0.000169761045885419\\
171	0.000169772706187853\\
172	0.000169784574886315\\
173	0.000169796655720564\\
174	0.000169808952498212\\
175	0.000169821469095988\\
176	0.00016983420946103\\
177	0.000169847177612184\\
178	0.000169860377641356\\
179	0.000169873813714861\\
180	0.000169887490074822\\
181	0.000169901411040579\\
182	0.000169915581010145\\
183	0.000169930004461665\\
184	0.000169944685954929\\
185	0.000169959630132898\\
186	0.00016997484172327\\
187	0.000169990325540069\\
188	0.000170006086485271\\
189	0.000170022129550458\\
190	0.000170038459818513\\
191	0.000170055082465341\\
192	0.000170072002761624\\
193	0.000170089226074614\\
194	0.000170106757869964\\
195	0.000170124603713592\\
196	0.000170142769273583\\
197	0.000170161260322129\\
198	0.00017018008273751\\
199	0.000170199242506112\\
200	0.000170218745724491\\
201	0.000170238598601468\\
202	0.000170258807460283\\
203	0.000170279378740776\\
204	0.000170300319001619\\
205	0.0001703216349226\\
206	0.00017034333330694\\
207	0.000170365421083671\\
208	0.00017038790531005\\
209	0.000170410793174038\\
210	0.000170434091996806\\
211	0.000170457809235325\\
212	0.000170481952484971\\
213	0.00017050652948222\\
214	0.000170531548107377\\
215	0.000170557016387361\\
216	0.000170582942498561\\
217	0.000170609334769738\\
218	0.000170636201684998\\
219	0.000170663551886816\\
220	0.000170691394179128\\
221	0.00017071973753049\\
222	0.000170748591077301\\
223	0.000170777964127089\\
224	0.000170807866161872\\
225	0.000170838306841585\\
226	0.00017086929600758\\
227	0.0001709008436862\\
228	0.000170932960092429\\
229	0.000170965655633617\\
230	0.000170998940913282\\
231	0.000171032826734991\\
232	0.000171067324106335\\
233	0.00017110244424297\\
234	0.000171138198572752\\
235	0.000171174598739968\\
236	0.000171211656609632\\
237	0.000171249384271903\\
238	0.000171287794046572\\
239	0.000171326898487654\\
240	0.000171366710388079\\
241	0.000171407242784477\\
242	0.00017144850896207\\
243	0.000171490522459664\\
244	0.000171533297074748\\
245	0.000171576846868694\\
246	0.000171621186172091\\
247	0.000171666329590157\\
248	0.000171712292008301\\
249	0.00017175908859777\\
250	0.000171806734821448\\
251	0.00017185524643975\\
252	0.000171904639516657\\
253	0.000171954930425878\\
254	0.000172006135857134\\
255	0.000172058272822577\\
256	0.000172111358663353\\
257	0.000172165411056294\\
258	0.000172220448020755\\
259	0.000172276487925589\\
260	0.000172333549496276\\
261	0.000172391651822199\\
262	0.000172450814364063\\
263	0.00017251105696149\\
264	0.000172572399840738\\
265	0.000172634863622617\\
266	0.000172698469330543\\
267	0.000172763238398767\\
268	0.000172829192680768\\
269	0.000172896354457827\\
270	0.000172964746447769\\
271	0.000173034391813874\\
272	0.00017310531417399\\
273	0.000173177537609807\\
274	0.000173251086676319\\
275	0.000173325986411495\\
276	0.000173402262346116\\
277	0.00017347994051381\\
278	0.000173559047461299\\
279	0.000173639610258825\\
280	0.000173721656510783\\
281	0.000173805214366563\\
282	0.000173890312531585\\
283	0.000173976980278556\\
284	0.000174065247458914\\
285	0.000174155144514509\\
286	0.000174246702489475\\
287	0.000174339953042325\\
288	0.000174434928458261\\
289	0.000174531661661694\\
290	0.000174630186228997\\
291	0.000174730536401446\\
292	0.000174832747098415\\
293	0.000174936853930769\\
294	0.000175042893214468\\
295	0.000175150901984418\\
296	0.000175260918008512\\
297	0.000175372979801904\\
298	0.0001754871266415\\
299	0.000175603398580655\\
300	0.000175721836464091\\
301	0.000175842481943035\\
302	0.000175965377490548\\
303	0.00017609056641709\\
304	0.000176218092886261\\
305	0.000176348001930775\\
306	0.000176480339468613\\
307	0.000176615152319388\\
308	0.000176752488220905\\
309	0.000176892395845903\\
310	0.000177034924819003\\
311	0.000177180125733827\\
312	0.000177328050170325\\
313	0.000177478750712254\\
314	0.000177632280964872\\
315	0.00017778869557278\\
316	0.000177948050237974\\
317	0.000178110401738062\\
318	0.000178275807944658\\
319	0.00017844432784198\\
320	0.000178616021545623\\
321	0.000178790950321544\\
322	0.000178969176605242\\
323	0.000179150764021163\\
324	0.000179335777402335\\
325	0.000179524282810263\\
326	0.000179716347555077\\
327	0.000179912040215995\\
328	0.00018011143066211\\
329	0.000180314590073516\\
330	0.000180521590962852\\
331	0.000180732507197275\\
332	0.000180947414020906\\
333	0.000181166388077831\\
334	0.000181389507435652\\
335	0.000181616851609706\\
336	0.000181848501587966\\
337	0.000182084539856667\\
338	0.000182325050426761\\
339	0.000182570118861175\\
340	0.00018281983230296\\
341	0.000183074279504315\\
342	0.000183333550856513\\
343	0.000183597738420683\\
344	0.000183866935959448\\
345	0.000184141238969377\\
346	0.000184420744714248\\
347	0.000184705552259115\\
348	0.000184995762505134\\
349	0.000185291478225168\\
350	0.000185592804100197\\
351	0.000185899846756542\\
352	0.000186212714804042\\
353	0.000186531518875363\\
354	0.000186856371666636\\
355	0.000187187387979784\\
356	0.000187524684766725\\
357	0.000187868381175616\\
358	0.000188218598599337\\
359	0.000188575460726427\\
360	0.000188939093594694\\
361	0.000189309625647759\\
362	0.000189687187794814\\
363	0.00019007191347385\\
364	0.000190463938718729\\
365	0.000190863402230402\\
366	0.000191270445452653\\
367	0.000191685212652803\\
368	0.000192107851007744\\
369	0.000192538510695839\\
370	0.000192977344995141\\
371	0.000193424510388485\\
372	0.000193880166676053\\
373	0.000194344477096035\\
374	0.000194817608454052\\
375	0.00019529973126214\\
376	0.000195791019888057\\
377	0.000196291652715871\\
378	0.0001968018123188\\
379	0.000197321685645456\\
380	0.000197851464220781\\
381	0.000198391344363156\\
382	0.000198941527419401\\
383	0.000199502220019748\\
384	0.000200073634355187\\
385	0.000200655988480237\\
386	0.000201249506644805\\
387	0.000201854419659851\\
388	0.000202470965303061\\
389	0.000203099388773038\\
390	0.000203739943204837\\
391	0.000204392890268078\\
392	0.000205058500885782\\
393	0.000205737056143686\\
394	0.000206428848497399\\
395	0.000207134183307596\\
396	0.000207853379823844\\
397	0.000208586765521004\\
398	0.00020933468860833\\
399	0.000210097524724835\\
400	0.000210875667369174\\
401	0.000211669529028189\\
402	0.000212479542287857\\
403	0.000213306160867271\\
404	0.000214149860468807\\
405	0.000215011139248253\\
406	0.000215890517559442\\
407	0.000216788536482469\\
408	0.000217705755060487\\
409	0.000218642749987742\\
410	0.000219600141048629\\
411	0.000220578570894087\\
412	0.000221578619178366\\
413	0.000222600889698149\\
414	0.00022364601233429\\
415	0.00022471464532483\\
416	0.000225807477920701\\
417	0.000226925233513964\\
418	0.000228068673351751\\
419	0.000229238600978757\\
420	0.000230435867590063\\
421	0.000231661378522649\\
422	0.000232916101182086\\
423	0.000234201074767163\\
424	0.000235517422258308\\
425	0.000236866365179842\\
426	0.000238249241555478\\
427	0.000239667526799369\\
428	0.000241122854473137\\
429	0.000242617023424316\\
430	0.000244151940366315\\
431	0.000245729312811729\\
432	0.000247349424765246\\
433	0.000248917206812989\\
434	0.000249596544315368\\
435	0.000250289392039495\\
436	0.000250995965316014\\
437	0.000251716480921113\\
438	0.00025245115865765\\
439	0.000253200223695467\\
440	0.00025396390992625\\
441	0.000254742464665859\\
442	0.000255536155136975\\
443	0.00025634527729162\\
444	0.000257170167690267\\
445	0.000258011219336887\\
446	0.000258868902549362\\
447	0.000259743792028783\\
448	0.000260636601032502\\
449	0.000261548222420013\\
450	0.000262479773609716\\
451	0.00026343263996428\\
452	0.000264408529489691\\
453	0.000265409712920401\\
454	0.000266440324185177\\
455	0.000267507390817175\\
456	0.000268615613600356\\
457	0.000269769795927175\\
458	0.00027097573957645\\
459	0.000272240500972844\\
460	0.00027357274626769\\
461	0.000274983305147449\\
462	0.000276486214505931\\
463	0.000278101230395547\\
464	0.000279861166060325\\
465	0.000307524637827358\\
466	0.000349969645631211\\
467	0.000393328689223382\\
468	0.000437627384730763\\
469	0.000482892213005026\\
470	0.000529150187553064\\
471	0.000576429124412446\\
472	0.0006247596323477\\
473	0.000674173667795424\\
474	0.000724704758816065\\
475	0.00077638852461577\\
476	0.000829263833541199\\
477	0.000883376879913921\\
478	0.000938764033621671\\
479	0.000995463115227499\\
480	0.00105351352101643\\
481	0.00111295636342444\\
482	0.00117383465337102\\
483	0.0012361935426771\\
484	0.00130008065241501\\
485	0.00136554652389328\\
486	0.00143264524472401\\
487	0.00150143532629143\\
488	0.00157198094668955\\
489	0.00164435372963181\\
490	0.00171863436274021\\
491	0.00179490999916231\\
492	0.00187327516151905\\
493	0.00195383234317257\\
494	0.00203669287158261\\
495	0.00212197728380178\\
496	0.00220981618775743\\
497	0.0023003531839373\\
498	0.00239374683710717\\
499	0.00249017303845386\\
500	0.00258982783074779\\
501	0.00269293074595561\\
502	0.00279972860438639\\
503	0.00291049941708504\\
504	0.00302555520514576\\
505	0.00314524552958512\\
506	0.00326263227805496\\
507	0.00330343466190086\\
508	0.00334563719849741\\
509	0.00338919372248778\\
510	0.00343401933015739\\
511	0.00348021493093214\\
512	0.00352790103705132\\
513	0.00357722775633646\\
514	0.0036283968779377\\
515	0.00368161901404434\\
516	0.00373712168333583\\
517	0.00379517532398462\\
518	0.00385610310005531\\
519	0.00392029325315597\\
520	0.0039882148043472\\
521	0.00406043735756756\\
522	0.00413765605181866\\
523	0.00422072260381219\\
524	0.00431068243548318\\
525	0.00440881296717743\\
526	0.00451553428790778\\
527	0.00462318184507296\\
528	0.00473134252956375\\
529	0.00483946652908471\\
530	0.00494682918447378\\
531	0.00505248079026667\\
532	0.00515562250187475\\
533	0.00525876642613349\\
534	0.00536161268074051\\
535	0.00546336034902462\\
536	0.00556294739333485\\
537	0.00565897620022263\\
538	0.00574971498516013\\
539	0.00583295559956373\\
540	0.00591379869531271\\
541	0.00599621185148525\\
542	0.00608011835818136\\
543	0.00616543130002583\\
544	0.00625205770730007\\
545	0.00633990646259004\\
546	0.00642888543393893\\
547	0.00651894091697697\\
548	0.00661009698930814\\
549	0.00670250472605515\\
550	0.0067965135636925\\
551	0.00689273545573893\\
552	0.00699161192885386\\
553	0.00709320002231934\\
554	0.00719740780199425\\
555	0.00730401456018862\\
556	0.0074126633149158\\
557	0.00752281634470193\\
558	0.00763362999187877\\
559	0.00774222432184838\\
560	0.00784624175801257\\
561	0.0079451173684526\\
562	0.00803836138267193\\
563	0.00812565253839744\\
564	0.00820701294573663\\
565	0.00828582198432094\\
566	0.00836215658040431\\
567	0.00843620586934715\\
568	0.00850832721502562\\
569	0.00857901013151536\\
570	0.00864906440997031\\
571	0.00871885032529967\\
572	0.00878725608766623\\
573	0.00885362539487147\\
574	0.00891781359239266\\
575	0.00897959328019709\\
576	0.00903975342437911\\
577	0.00909898426363027\\
578	0.0091567304087957\\
579	0.00921190939223783\\
580	0.00926446940119685\\
581	0.00931498012486593\\
582	0.00936431107260272\\
583	0.00941219245211477\\
584	0.00945866243584944\\
585	0.00950405575984369\\
586	0.00954865718510014\\
587	0.0095925304780904\\
588	0.00963572111751646\\
589	0.00967825788510043\\
590	0.00972017907935068\\
591	0.00976137872239597\\
592	0.00980171668827597\\
593	0.00984102099863122\\
594	0.00987905595166784\\
595	0.00991543381058343\\
596	0.00994937493726701\\
597	0.00997906286423442\\
598	0.0099999191923403\\
599	0\\
600	0\\
};
\addplot [color=red!25!mycolor17,solid,forget plot]
  table[row sep=crcr]{%
1	0.000375855361567015\\
2	0.000375864735362743\\
3	0.000375874276850767\\
4	0.000375883989026931\\
5	0.000375893874940512\\
6	0.000375903937695174\\
7	0.000375914180449942\\
8	0.000375924606420201\\
9	0.000375935218878671\\
10	0.000375946021156463\\
11	0.0003759570166441\\
12	0.000375968208792553\\
13	0.000375979601114359\\
14	0.000375991197184693\\
15	0.000376003000642513\\
16	0.000376015015191641\\
17	0.000376027244601968\\
18	0.000376039692710604\\
19	0.00037605236342309\\
20	0.000376065260714613\\
21	0.000376078388631249\\
22	0.000376091751291189\\
23	0.000376105352886097\\
24	0.000376119197682364\\
25	0.000376133290022439\\
26	0.000376147634326219\\
27	0.000376162235092381\\
28	0.000376177096899843\\
29	0.000376192224409123\\
30	0.000376207622363863\\
31	0.000376223295592248\\
32	0.000376239249008547\\
33	0.000376255487614668\\
34	0.000376272016501651\\
35	0.000376288840851316\\
36	0.000376305965937858\\
37	0.000376323397129494\\
38	0.000376341139890124\\
39	0.000376359199781053\\
40	0.000376377582462738\\
41	0.000376396293696532\\
42	0.000376415339346471\\
43	0.000376434725381129\\
44	0.000376454457875476\\
45	0.000376474543012747\\
46	0.000376494987086415\\
47	0.000376515796502134\\
48	0.000376536977779709\\
49	0.000376558537555178\\
50	0.00037658048258284\\
51	0.000376602819737422\\
52	0.000376625556016146\\
53	0.000376648698540977\\
54	0.000376672254560806\\
55	0.000376696231453725\\
56	0.000376720636729335\\
57	0.000376745478031095\\
58	0.000376770763138686\\
59	0.000376796499970441\\
60	0.000376822696585871\\
61	0.000376849361188093\\
62	0.000376876502126454\\
63	0.000376904127899108\\
64	0.000376932247155693\\
65	0.000376960868699994\\
66	0.000376990001492742\\
67	0.000377019654654379\\
68	0.000377049837467919\\
69	0.00037708055938181\\
70	0.000377111830012952\\
71	0.000377143659149655\\
72	0.000377176056754711\\
73	0.0003772090329685\\
74	0.000377242598112152\\
75	0.000377276762690801\\
76	0.000377311537396846\\
77	0.000377346933113272\\
78	0.000377382960917113\\
79	0.000377419632082857\\
80	0.000377456958085995\\
81	0.000377494950606607\\
82	0.000377533621533028\\
83	0.000377572982965524\\
84	0.000377613047220143\\
85	0.000377653826832503\\
86	0.000377695334561765\\
87	0.0003777375833946\\
88	0.000377780586549253\\
89	0.000377824357479701\\
90	0.00037786890987985\\
91	0.0003779142576878\\
92	0.000377960415090283\\
93	0.000378007396527034\\
94	0.000378055216695353\\
95	0.000378103890554693\\
96	0.000378153433331373\\
97	0.00037820386052334\\
98	0.000378255187905027\\
99	0.000378307431532289\\
100	0.00037836060774748\\
101	0.000378414733184553\\
102	0.00037846982477425\\
103	0.000378525899749498\\
104	0.000378582975650768\\
105	0.000378641070331584\\
106	0.000378700201964145\\
107	0.000378760389045049\\
108	0.000378821650401074\\
109	0.000378884005195147\\
110	0.000378947472932331\\
111	0.000379012073465995\\
112	0.000379077827004037\\
113	0.000379144754115263\\
114	0.000379212875735858\\
115	0.000379282213175986\\
116	0.000379352788126547\\
117	0.000379424622665918\\
118	0.000379497739267039\\
119	0.000379572160804407\\
120	0.000379647910561355\\
121	0.000379725012237393\\
122	0.000379803489955662\\
123	0.00037988336827062\\
124	0.000379964672175753\\
125	0.000380047427111495\\
126	0.000380131658973323\\
127	0.000380217394119883\\
128	0.000380304659381396\\
129	0.000380393482068136\\
130	0.000380483889979131\\
131	0.000380575911410879\\
132	0.000380669575166452\\
133	0.000380764910564579\\
134	0.000380861947448975\\
135	0.000380960716197793\\
136	0.000381061247733372\\
137	0.000381163573531995\\
138	0.00038126772563395\\
139	0.000381373736653725\\
140	0.000381481639790436\\
141	0.000381591468838373\\
142	0.000381703258197825\\
143	0.000381817042886077\\
144	0.00038193285854853\\
145	0.000382050741470206\\
146	0.000382170728587262\\
147	0.000382292857498925\\
148	0.000382417166479432\\
149	0.000382543694490394\\
150	0.000382672481193246\\
151	0.00038280356696203\\
152	0.000382936992896346\\
153	0.000383072800834601\\
154	0.000383211033367421\\
155	0.000383351733851479\\
156	0.00038349494642337\\
157	0.000383640716013898\\
158	0.000383789088362624\\
159	0.000383940110032628\\
160	0.000384093828425572\\
161	0.000384250291797077\\
162	0.00038440954927231\\
163	0.000384571650862024\\
164	0.000384736647478714\\
165	0.000384904590953235\\
166	0.000385075534051627\\
167	0.000385249530492313\\
168	0.000385426634963654\\
169	0.000385606903141773\\
170	0.000385790391708767\\
171	0.00038597715837124\\
172	0.000386167261879223\\
173	0.000386360762045411\\
174	0.000386557719764833\\
175	0.000386758197034831\\
176	0.00038696225697549\\
177	0.000387169963850431\\
178	0.000387381383087991\\
179	0.000387596581302871\\
180	0.000387815626318109\\
181	0.000388038587187575\\
182	0.000388265534218859\\
183	0.000388496538996624\\
184	0.000388731674406369\\
185	0.000388971014658706\\
186	0.000389214635314107\\
187	0.000389462613308146\\
188	0.000389715026977176\\
189	0.000389971956084584\\
190	0.000390233481847511\\
191	0.000390499686964138\\
192	0.00039077065564149\\
193	0.0003910464736238\\
194	0.000391327228221434\\
195	0.000391613008340391\\
196	0.000391903904512368\\
197	0.000392200008925458\\
198	0.000392501415455497\\
199	0.000392808219697883\\
200	0.000393120519000252\\
201	0.000393438412495624\\
202	0.000393762001136319\\
203	0.000394091387728522\\
204	0.000394426676967558\\
205	0.000394767975473868\\
206	0.000395115391829715\\
207	0.000395469036616637\\
208	0.000395829022453655\\
209	0.000396195464036341\\
210	0.000396568478176487\\
211	0.000396948183842786\\
212	0.000397334702202259\\
213	0.000397728156662532\\
214	0.000398128672915005\\
215	0.000398536378978847\\
216	0.000398951405246032\\
217	0.00039937388452712\\
218	0.000399803952098157\\
219	0.000400241745748425\\
220	0.000400687405829319\\
221	0.000401141075304038\\
222	0.000401602899798527\\
223	0.000402073027653355\\
224	0.000402551609976692\\
225	0.000403038800698443\\
226	0.000403534756625491\\
227	0.0004040396374981\\
228	0.000404553606047494\\
229	0.000405076828054671\\
230	0.00040560947241045\\
231	0.000406151711176854\\
232	0.000406703719649673\\
233	0.000407265676422422\\
234	0.000407837763451732\\
235	0.000408420166124007\\
236	0.000409013073323558\\
237	0.000409616677502251\\
238	0.000410231174750549\\
239	0.000410856764870171\\
240	0.000411493651448245\\
241	0.000412142041933094\\
242	0.000412802147711648\\
243	0.000413474184188537\\
244	0.000414158370866868\\
245	0.000414854931430798\\
246	0.000415564093829837\\
247	0.000416286090365026\\
248	0.000417021157777013\\
249	0.000417769537335968\\
250	0.000418531474933557\\
251	0.000419307221176814\\
252	0.000420097031484151\\
253	0.000420901166183438\\
254	0.000421719890612251\\
255	0.000422553475220229\\
256	0.000423402195673846\\
257	0.000424266332963319\\
258	0.000425146173511911\\
259	0.000426042009287706\\
260	0.000426954137917746\\
261	0.000427882862804754\\
262	0.000428828493246336\\
263	0.000429791344556945\\
264	0.000430771738192403\\
265	0.000431770001877172\\
266	0.000432786469734556\\
267	0.000433821482419583\\
268	0.000434875387254931\\
269	0.000435948538369868\\
270	0.000437041296842104\\
271	0.000438154030842877\\
272	0.000439287115785182\\
273	0.000440440934475291\\
274	0.000441615877267434\\
275	0.000442812342222102\\
276	0.00044403073526753\\
277	0.000445271470364928\\
278	0.000446534969677154\\
279	0.000447821663741041\\
280	0.000449131991643555\\
281	0.000450466401201604\\
282	0.000451825349145801\\
283	0.000453209301308104\\
284	0.000454618732813421\\
285	0.000456054128275351\\
286	0.00045751598199604\\
287	0.000459004798170102\\
288	0.000460521091092953\\
289	0.000462065385373391\\
290	0.000463638216150475\\
291	0.00046524012931502\\
292	0.000466871681735374\\
293	0.000468533441487851\\
294	0.000470225988091772\\
295	0.000471949912749025\\
296	0.000473705818588461\\
297	0.000475494320914845\\
298	0.000477316047462662\\
299	0.000479171638654685\\
300	0.000481061747865299\\
301	0.000482987041688683\\
302	0.000484948200211826\\
303	0.000486945917292301\\
304	0.000488980900840993\\
305	0.000491053873109538\\
306	0.000493165570982603\\
307	0.000495316746274937\\
308	0.000497508166033209\\
309	0.000499740612842413\\
310	0.000502014885137115\\
311	0.000504331797517111\\
312	0.000506692181067717\\
313	0.000509096883684502\\
314	0.000511546770402322\\
315	0.000514042723728773\\
316	0.000516585643981767\\
317	0.000519176449631222\\
318	0.00052181607764491\\
319	0.000524505483838112\\
320	0.00052724564322733\\
321	0.000530037550387755\\
322	0.00053288221981458\\
323	0.000535780686288141\\
324	0.000538734005242898\\
325	0.000541743253140319\\
326	0.00054480952784587\\
327	0.000547933949010139\\
328	0.000551117658454588\\
329	0.000554361820562086\\
330	0.000557667622672797\\
331	0.000561036275485951\\
332	0.000564469013468215\\
333	0.000567967095269487\\
334	0.000571531804147124\\
335	0.000575164448399749\\
336	0.000578866361811827\\
337	0.000582638904110663\\
338	0.000586483461437091\\
339	0.000590401446831726\\
340	0.000594394300738327\\
341	0.000598463491525742\\
342	0.000602610516029878\\
343	0.000606836900116855\\
344	0.000611144199267906\\
345	0.000615533999183235\\
346	0.000620007916404401\\
347	0.00062456759895429\\
348	0.0006292147269931\\
349	0.000633951013488447\\
350	0.000638778204896862\\
351	0.00064369808185397\\
352	0.000648712459870477\\
353	0.000653823190031695\\
354	0.00065903215969827\\
355	0.000664341293204633\\
356	0.000669752552572866\\
357	0.000675267938244582\\
358	0.000680889489831862\\
359	0.000686619286888755\\
360	0.000692459449704616\\
361	0.000698412140121118\\
362	0.000704479562374511\\
363	0.000710663963965251\\
364	0.000716967636556976\\
365	0.000723392916907245\\
366	0.000729942187832629\\
367	0.000736617879210691\\
368	0.000743422469022132\\
369	0.00075035848443618\\
370	0.000757428502942744\\
371	0.000764635153535262\\
372	0.000771981117948212\\
373	0.000779469131953892\\
374	0.000787101986723156\\
375	0.000794882530255428\\
376	0.000802813668883572\\
377	0.000810898368859764\\
378	0.000819139658029086\\
379	0.000827540627598236\\
380	0.000836104434007321\\
381	0.000844834300913916\\
382	0.000853733521299567\\
383	0.000862805459710351\\
384	0.000872053554644853\\
385	0.000881481321105537\\
386	0.000891092353332046\\
387	0.000900890327738918\\
388	0.000910879006084138\\
389	0.000921062238898888\\
390	0.000931443969208682\\
391	0.000942028236562785\\
392	0.000952819181333663\\
393	0.000963821049074246\\
394	0.000975038194216569\\
395	0.000986475081000435\\
396	0.00099813627578383\\
397	0.00101002641521094\\
398	0.00102215005977609\\
399	0.00103451205535606\\
400	0.00104711769728795\\
401	0.00105997245280715\\
402	0.00107308197031745\\
403	0.00108645208931192\\
404	0.00110008885116143\\
405	0.00111399851140646\\
406	0.00112818755533907\\
407	0.00114266272165957\\
408	0.00115743104644096\\
409	0.00117249995723704\\
410	0.00118787748593639\\
411	0.00120357290884931\\
412	0.00121959580838612\\
413	0.00123595364884107\\
414	0.00125265406994354\\
415	0.00126970487381734\\
416	0.0012871140242872\\
417	0.00130488964536087\\
418	0.00132304001870451\\
419	0.00134157357987641\\
420	0.00136049891310906\\
421	0.00137982474455152\\
422	0.00139955993346553\\
423	0.00141971346091733\\
424	0.00144029441416247\\
425	0.00146131196750383\\
426	0.00148277535892334\\
427	0.00150469386122702\\
428	0.00152707674509668\\
429	0.00154993322871793\\
430	0.00157327240515157\\
431	0.00159710314396872\\
432	0.00162143403281746\\
433	0.00164627419956294\\
434	0.00167164021505231\\
435	0.00169756119205294\\
436	0.00172405154377306\\
437	0.00175112621257999\\
438	0.00177880070542449\\
439	0.00180709113343772\\
440	0.00183601425639072\\
441	0.00186558753285368\\
442	0.0018958291770815\\
443	0.00192675822389243\\
444	0.00195839460311546\\
445	0.00199075922557841\\
446	0.00202387408312123\\
447	0.0020577623657961\\
448	0.002092448600366\\
449	0.00212795881569221\\
450	0.00216432074310062\\
451	0.00220156406288743\\
452	0.00223972071917512\\
453	0.0022788253353444\\
454	0.0023189156084635\\
455	0.00236003293586921\\
456	0.00240222330492044\\
457	0.00244553836795654\\
458	0.00249003681886893\\
459	0.00253578618256766\\
460	0.00258286512841324\\
461	0.00263136639903981\\
462	0.00268140027958883\\
463	0.00273309391864708\\
464	0.00278658892957238\\
465	0.0028160858344251\\
466	0.00283229364123304\\
467	0.00284891162233504\\
468	0.00286595707549019\\
469	0.00288344845123734\\
470	0.00290140543312626\\
471	0.00291984904478575\\
472	0.00293880188699185\\
473	0.00295828798435947\\
474	0.00297833269726963\\
475	0.00299896215397996\\
476	0.00302020306229267\\
477	0.00304208316074912\\
478	0.00306463106400083\\
479	0.00308787617611319\\
480	0.0031118482766772\\
481	0.0031365768954682\\
482	0.00316209039491253\\
483	0.00318841464993482\\
484	0.00321557117399093\\
485	0.00324357448262726\\
486	0.00327242840310902\\
487	0.00330212091969892\\
488	0.00333261699691251\\
489	0.00336384880815839\\
490	0.00339572914874768\\
491	0.00342827959547576\\
492	0.00346152324687747\\
493	0.00349549303595669\\
494	0.00353023096274955\\
495	0.00356579818865329\\
496	0.00360226589595257\\
497	0.00363968661879014\\
498	0.00367811966898888\\
499	0.00371763218444223\\
500	0.00375830031168834\\
501	0.00380021052783213\\
502	0.00384346111023267\\
503	0.00388816378050593\\
504	0.00393444559791164\\
505	0.00398245123033693\\
506	0.00403235920962256\\
507	0.00408472396073087\\
508	0.00414228410426581\\
509	0.00420609627069447\\
510	0.00427692741442519\\
511	0.00434901585665425\\
512	0.00442233189612585\\
513	0.00449682806344451\\
514	0.0045724331827926\\
515	0.00464904554710942\\
516	0.00472652549801599\\
517	0.00480468550780751\\
518	0.0048832768550998\\
519	0.00496197116355689\\
520	0.00504033495355125\\
521	0.00511780127935372\\
522	0.00519363221594222\\
523	0.00526686946768327\\
524	0.00533627214472673\\
525	0.00540024436774731\\
526	0.00545788363371357\\
527	0.00551627335090793\\
528	0.00557539167135386\\
529	0.00563527859893137\\
530	0.00569594513862986\\
531	0.00575739841589084\\
532	0.00581967989167963\\
533	0.00588276119286136\\
534	0.00594662385371704\\
535	0.0060112831455332\\
536	0.00607681286106642\\
537	0.00614337892796407\\
538	0.00621128615237045\\
539	0.00628104207503862\\
540	0.00635312254556396\\
541	0.0064277223001837\\
542	0.0065048984881829\\
543	0.00658470064598824\\
544	0.00666716561147777\\
545	0.00675231227196042\\
546	0.00684007090564941\\
547	0.00693038605348422\\
548	0.00702318517624303\\
549	0.00711841038944991\\
550	0.00721590978540083\\
551	0.00731534307738251\\
552	0.00741610746377254\\
553	0.00751723386195233\\
554	0.007615025649278\\
555	0.00770805129765681\\
556	0.00779580222598891\\
557	0.00787788560954736\\
558	0.00795385979339356\\
559	0.00802573665129025\\
560	0.00809530374867697\\
561	0.00816276262131606\\
562	0.00822843087997214\\
563	0.00829273829880889\\
564	0.00835656690377135\\
565	0.0084203077058799\\
566	0.0084841452589513\\
567	0.00854828119591742\\
568	0.00861187435009632\\
569	0.00867400404508977\\
570	0.00873448446449719\\
571	0.00879303057564046\\
572	0.00885059500516906\\
573	0.00890770640378388\\
574	0.00896437197501948\\
575	0.00902054698487434\\
576	0.00907467179264195\\
577	0.00912662893845292\\
578	0.00917681195253149\\
579	0.00922620839379051\\
580	0.00927487032649829\\
581	0.00932247740245366\\
582	0.0093690234246846\\
583	0.00941489058069364\\
584	0.00946020548959901\\
585	0.00950497683508797\\
586	0.009549192402668\\
587	0.00959283081692013\\
588	0.00963587774208467\\
589	0.0096783320190804\\
590	0.0097202097033969\\
591	0.00976138909231767\\
592	0.00980171925872612\\
593	0.00984102135019415\\
594	0.00987905595166784\\
595	0.00991543381058343\\
596	0.00994937493726701\\
597	0.00997906286423442\\
598	0.0099999191923403\\
599	0\\
600	0\\
};
\addplot [color=mycolor19,solid,forget plot]
  table[row sep=crcr]{%
1	0.00243322575185515\\
2	0.00243322940690112\\
3	0.002433233127433\\
4	0.00243323691462267\\
5	0.00243324076966301\\
6	0.00243324469376817\\
7	0.00243324868817403\\
8	0.00243325275413852\\
9	0.00243325689294207\\
10	0.00243326110588798\\
11	0.00243326539430283\\
12	0.00243326975953691\\
13	0.00243327420296463\\
14	0.00243327872598494\\
15	0.0024332833300218\\
16	0.0024332880165246\\
17	0.0024332927869686\\
18	0.00243329764285543\\
19	0.00243330258571356\\
20	0.00243330761709872\\
21	0.00243331273859443\\
22	0.00243331795181255\\
23	0.00243332325839364\\
24	0.00243332866000762\\
25	0.00243333415835421\\
26	0.00243333975516345\\
27	0.00243334545219633\\
28	0.00243335125124526\\
29	0.00243335715413463\\
30	0.00243336316272144\\
31	0.00243336927889581\\
32	0.00243337550458164\\
33	0.00243338184173715\\
34	0.00243338829235553\\
35	0.00243339485846553\\
36	0.00243340154213214\\
37	0.0024334083454572\\
38	0.00243341527058006\\
39	0.0024334223196783\\
40	0.00243342949496824\\
41	0.0024334367987059\\
42	0.00243344423318748\\
43	0.00243345180075018\\
44	0.0024334595037729\\
45	0.00243346734467702\\
46	0.00243347532592709\\
47	0.00243348345003162\\
48	0.00243349171954392\\
49	0.00243350013706283\\
50	0.00243350870523359\\
51	0.00243351742674857\\
52	0.00243352630434822\\
53	0.00243353534082182\\
54	0.00243354453900849\\
55	0.00243355390179791\\
56	0.00243356343213137\\
57	0.00243357313300256\\
58	0.00243358300745861\\
59	0.00243359305860097\\
60	0.00243360328958637\\
61	0.00243361370362788\\
62	0.00243362430399586\\
63	0.00243363509401895\\
64	0.00243364607708517\\
65	0.00243365725664295\\
66	0.0024336686362022\\
67	0.0024336802193354\\
68	0.00243369200967867\\
69	0.00243370401093303\\
70	0.00243371622686545\\
71	0.00243372866131001\\
72	0.00243374131816915\\
73	0.00243375420141487\\
74	0.00243376731508999\\
75	0.00243378066330931\\
76	0.00243379425026102\\
77	0.00243380808020794\\
78	0.0024338221574888\\
79	0.00243383648651969\\
80	0.00243385107179539\\
81	0.00243386591789073\\
82	0.00243388102946204\\
83	0.00243389641124867\\
84	0.00243391206807432\\
85	0.00243392800484867\\
86	0.00243394422656882\\
87	0.00243396073832091\\
88	0.00243397754528165\\
89	0.00243399465271997\\
90	0.00243401206599862\\
91	0.00243402979057587\\
92	0.00243404783200714\\
93	0.00243406619594686\\
94	0.00243408488815007\\
95	0.00243410391447433\\
96	0.00243412328088146\\
97	0.00243414299343945\\
98	0.00243416305832428\\
99	0.00243418348182193\\
100	0.00243420427033023\\
101	0.00243422543036089\\
102	0.00243424696854157\\
103	0.00243426889161785\\
104	0.00243429120645541\\
105	0.00243431392004206\\
106	0.00243433703949003\\
107	0.00243436057203805\\
108	0.00243438452505372\\
109	0.00243440890603569\\
110	0.00243443372261605\\
111	0.00243445898256268\\
112	0.00243448469378165\\
113	0.00243451086431968\\
114	0.00243453750236664\\
115	0.00243456461625812\\
116	0.00243459221447791\\
117	0.00243462030566075\\
118	0.00243464889859492\\
119	0.00243467800222503\\
120	0.00243470762565476\\
121	0.00243473777814967\\
122	0.00243476846914011\\
123	0.00243479970822408\\
124	0.00243483150517026\\
125	0.00243486386992106\\
126	0.0024348968125956\\
127	0.00243493034349291\\
128	0.00243496447309514\\
129	0.00243499921207075\\
130	0.00243503457127785\\
131	0.00243507056176759\\
132	0.00243510719478751\\
133	0.00243514448178507\\
134	0.00243518243441118\\
135	0.00243522106452384\\
136	0.00243526038419174\\
137	0.00243530040569808\\
138	0.00243534114154432\\
139	0.00243538260445403\\
140	0.00243542480737691\\
141	0.00243546776349272\\
142	0.00243551148621539\\
143	0.00243555598919714\\
144	0.0024356012863328\\
145	0.00243564739176398\\
146	0.00243569431988355\\
147	0.00243574208534001\\
148	0.00243579070304211\\
149	0.00243584018816341\\
150	0.00243589055614697\\
151	0.00243594182271013\\
152	0.00243599400384942\\
153	0.00243604711584545\\
154	0.00243610117526801\\
155	0.00243615619898108\\
156	0.00243621220414821\\
157	0.00243626920823775\\
158	0.00243632722902821\\
159	0.00243638628461381\\
160	0.0024364463934101\\
161	0.00243650757415964\\
162	0.00243656984593782\\
163	0.00243663322815867\\
164	0.002436697740581\\
165	0.00243676340331443\\
166	0.00243683023682566\\
167	0.00243689826194482\\
168	0.00243696749987184\\
169	0.00243703797218308\\
170	0.00243710970083805\\
171	0.00243718270818609\\
172	0.00243725701697337\\
173	0.00243733265035\\
174	0.00243740963187707\\
175	0.00243748798553408\\
176	0.00243756773572627\\
177	0.00243764890729223\\
178	0.00243773152551165\\
179	0.00243781561611305\\
180	0.0024379012052819\\
181	0.00243798831966862\\
182	0.00243807698639691\\
183	0.00243816723307218\\
184	0.00243825908779009\\
185	0.00243835257914535\\
186	0.00243844773624056\\
187	0.00243854458869515\\
188	0.00243864316665478\\
189	0.00243874350080061\\
190	0.00243884562235886\\
191	0.00243894956311053\\
192	0.00243905535540133\\
193	0.00243916303215172\\
194	0.00243927262686716\\
195	0.0024393841736486\\
196	0.00243949770720314\\
197	0.00243961326285478\\
198	0.00243973087655549\\
199	0.0024398505848965\\
200	0.00243997242511965\\
201	0.00244009643512912\\
202	0.00244022265350323\\
203	0.0024403511195066\\
204	0.00244048187310239\\
205	0.00244061495496486\\
206	0.00244075040649215\\
207	0.0024408882698193\\
208	0.00244102858783148\\
209	0.00244117140417743\\
210	0.00244131676328333\\
211	0.00244146471036672\\
212	0.0024416152914508\\
213	0.00244176855337893\\
214	0.00244192454382946\\
215	0.00244208331133091\\
216	0.00244224490527716\\
217	0.0024424093759433\\
218	0.0024425767745015\\
219	0.00244274715303737\\
220	0.00244292056456639\\
221	0.002443097063051\\
222	0.00244327670341765\\
223	0.00244345954157455\\
224	0.00244364563442941\\
225	0.00244383503990779\\
226	0.00244402781697168\\
227	0.00244422402563843\\
228	0.00244442372700015\\
229	0.00244462698324345\\
230	0.00244483385766953\\
231	0.00244504441471467\\
232	0.00244525871997119\\
233	0.00244547684020884\\
234	0.00244569884339648\\
235	0.00244592479872436\\
236	0.00244615477662682\\
237	0.00244638884880532\\
238	0.0024466270882522\\
239	0.00244686956927463\\
240	0.00244711636751926\\
241	0.00244736755999737\\
242	0.00244762322511045\\
243	0.00244788344267635\\
244	0.00244814829395612\\
245	0.00244841786168116\\
246	0.00244869223008125\\
247	0.00244897148491293\\
248	0.00244925571348868\\
249	0.00244954500470662\\
250	0.00244983944908089\\
251	0.0024501391387728\\
252	0.00245044416762254\\
253	0.0024507546311817\\
254	0.00245107062674643\\
255	0.00245139225339151\\
256	0.00245171961200502\\
257	0.00245205280532398\\
258	0.00245239193797073\\
259	0.00245273711649021\\
260	0.00245308844938809\\
261	0.00245344604716983\\
262	0.0024538100223808\\
263	0.00245418048964712\\
264	0.00245455756571778\\
265	0.00245494136950771\\
266	0.00245533202214183\\
267	0.00245572964700042\\
268	0.00245613436976553\\
269	0.00245654631846854\\
270	0.00245696562353919\\
271	0.00245739241785566\\
272	0.00245782683679614\\
273	0.00245826901829169\\
274	0.00245871910288073\\
275	0.00245917723376485\\
276	0.00245964355686627\\
277	0.00246011822088686\\
278	0.00246060137736899\\
279	0.002461093180758\\
280	0.00246159378846649\\
281	0.0024621033609407\\
282	0.00246262206172861\\
283	0.00246315005755033\\
284	0.00246368751837049\\
285	0.00246423461747296\\
286	0.00246479153153784\\
287	0.00246535844072099\\
288	0.00246593552873593\\
289	0.00246652298293855\\
290	0.00246712099441454\\
291	0.00246772975806956\\
292	0.00246834947272263\\
293	0.00246898034120261\\
294	0.0024696225704477\\
295	0.00247027637160877\\
296	0.00247094196015583\\
297	0.00247161955598855\\
298	0.00247230938355034\\
299	0.00247301167194677\\
300	0.00247372665506791\\
301	0.00247445457171518\\
302	0.00247519566573273\\
303	0.0024759501861435\\
304	0.00247671838729021\\
305	0.00247750052898146\\
306	0.00247829687664303\\
307	0.00247910770147479\\
308	0.00247993328061304\\
309	0.00248077389729889\\
310	0.00248162984105248\\
311	0.00248250140785344\\
312	0.00248338890032766\\
313	0.00248429262794039\\
314	0.00248521290719616\\
315	0.00248615006184504\\
316	0.00248710442309585\\
317	0.00248807632983603\\
318	0.00248906612885809\\
319	0.00249007417509303\\
320	0.0024911008318499\\
321	0.00249214647106188\\
322	0.00249321147353832\\
323	0.00249429622922244\\
324	0.0024954011374541\\
325	0.00249652660723728\\
326	0.00249767305751127\\
327	0.00249884091742482\\
328	0.00250003062661243\\
329	0.00250124263547115\\
330	0.00250247740543719\\
331	0.00250373540926026\\
332	0.00250501713127448\\
333	0.00250632306766386\\
334	0.00250765372672073\\
335	0.00250900962909522\\
336	0.0025103913080341\\
337	0.00251179930960745\\
338	0.00251323419292223\\
339	0.00251469653032239\\
340	0.00251618690757661\\
341	0.00251770592405601\\
342	0.00251925419290494\\
343	0.00252083234120518\\
344	0.0025224410101529\\
345	0.00252408085532325\\
346	0.00252575254696737\\
347	0.00252745677035158\\
348	0.00252919422614939\\
349	0.00253096563089699\\
350	0.00253277171752206\\
351	0.00253461323595275\\
352	0.00253649095380898\\
353	0.00253840565717714\\
354	0.00254035815148238\\
355	0.00254234926247657\\
356	0.00254437983683534\\
357	0.00254645074273674\\
358	0.00254856287045189\\
359	0.00255071713294689\\
360	0.00255291446649582\\
361	0.00255515583130394\\
362	0.00255744221214044\\
363	0.0025597746189799\\
364	0.00256215408765162\\
365	0.00256458168049537\\
366	0.00256705848702252\\
367	0.0025695856245811\\
368	0.00257216423902266\\
369	0.00257479550536971\\
370	0.0025774806284808\\
371	0.00258022084371129\\
372	0.00258301741756675\\
373	0.00258587164834578\\
374	0.00258878486676855\\
375	0.00259175843658691\\
376	0.00259479375517109\\
377	0.00259789225406755\\
378	0.00260105539952129\\
379	0.00260428469295505\\
380	0.00260758167139656\\
381	0.00261094790784304\\
382	0.00261438501155048\\
383	0.00261789462823249\\
384	0.00262147844015029\\
385	0.0026251381660719\\
386	0.00262887556107286\\
387	0.00263269241614548\\
388	0.00263659055757467\\
389	0.00264057184602923\\
390	0.0026446381753039\\
391	0.0026487914706317\\
392	0.00265303368646489\\
393	0.00265736680359865\\
394	0.00266179282548564\\
395	0.00266631377357758\\
396	0.00267093168157545\\
397	0.00267564858866405\\
398	0.00268046653230534\\
399	0.00268538753867355\\
400	0.00269041360363406\\
401	0.00269554667067543\\
402	0.00270078861131059\\
403	0.00270614120321678\\
404	0.00271160610671546\\
405	0.00271718484080559\\
406	0.00272287876090144\\
407	0.00272868904181218\\
408	0.00273461667146946\\
409	0.00274066246357437\\
410	0.00274682710084091\\
411	0.00275311122544019\\
412	0.00275951561264378\\
413	0.00276604151076391\\
414	0.0027726912044118\\
415	0.00277946746198407\\
416	0.00278637314424618\\
417	0.00279341120986781\\
418	0.00280058472191354\\
419	0.00280789685554711\\
420	0.00281535090508996\\
421	0.00282295028744672\\
422	0.00283069855175062\\
423	0.00283859939608647\\
424	0.00284665671753488\\
425	0.0028548746330132\\
426	0.00286325750122126\\
427	0.00287180994486809\\
428	0.00288053687188287\\
429	0.00288944349364547\\
430	0.00289853533701717\\
431	0.00290781824351838\\
432	0.00291729833865878\\
433	0.00292698192475248\\
434	0.00293687522341037\\
435	0.00294698438892954\\
436	0.00295731584817982\\
437	0.00296787631184095\\
438	0.00297867278412394\\
439	0.00298971257008401\\
440	0.00300100327931023\\
441	0.00301255282434944\\
442	0.00302436941164805\\
443	0.00303646152202934\\
444	0.00304883787669455\\
445	0.00306150738335215\\
446	0.00307447905521328\\
447	0.00308776189309536\\
448	0.00310136471771375\\
449	0.00311529593617701\\
450	0.00312956322845109\\
451	0.00314417316807397\\
452	0.0031591306794131\\
453	0.00317443806932127\\
454	0.00319009365038383\\
455	0.00320608971468185\\
456	0.00322241117922144\\
457	0.00323903284133796\\
458	0.00325591568031399\\
459	0.00327300186627504\\
460	0.00329020802013293\\
461	0.00330741618752921\\
462	0.00332446235535646\\
463	0.0033412318864999\\
464	0.00335758742685534\\
465	0.00337338760836127\\
466	0.00338918042814833\\
467	0.00340530655874068\\
468	0.00342178366260316\\
469	0.00343863177671142\\
470	0.00345587363475759\\
471	0.00347353476768834\\
472	0.0034916426578092\\
473	0.00351023430286914\\
474	0.00352935638325661\\
475	0.00354907225021517\\
476	0.00356945357161289\\
477	0.00359057112268743\\
478	0.00361250894201974\\
479	0.00363536731737758\\
480	0.00365926652319814\\
481	0.00368435151566849\\
482	0.00371079785126699\\
483	0.00373881917193254\\
484	0.00376867669766305\\
485	0.00380069128279977\\
486	0.0038352586996068\\
487	0.00387286879664353\\
488	0.0039141285910975\\
489	0.00395978649863041\\
490	0.00400999994379843\\
491	0.00406104221269004\\
492	0.00411290889303081\\
493	0.00416559058337188\\
494	0.00421907124212074\\
495	0.00427332600109312\\
496	0.00432831847470865\\
497	0.00438399836381636\\
498	0.0044402986351139\\
499	0.00449713119941296\\
500	0.00455438145906978\\
501	0.00461190142112322\\
502	0.00466950098744828\\
503	0.00472693692844685\\
504	0.00478389889952752\\
505	0.00483999165498312\\
506	0.00489471185943743\\
507	0.00494740805108051\\
508	0.00499718385826722\\
509	0.00504289057825752\\
510	0.00508359944924854\\
511	0.00512486917326703\\
512	0.0051666669189476\\
513	0.00520895418874585\\
514	0.00525168675647541\\
515	0.00529481481413926\\
516	0.00533828304376811\\
517	0.00538202974207427\\
518	0.00542599384801781\\
519	0.00547014462970504\\
520	0.00551450866905051\\
521	0.00555909764201829\\
522	0.00560393937996247\\
523	0.00564911930132595\\
524	0.00569481048431436\\
525	0.00574131614048781\\
526	0.0057890876497553\\
527	0.00583843272153866\\
528	0.00588943896141111\\
529	0.00594220010200307\\
530	0.00599681393333867\\
531	0.00605338332978803\\
532	0.00611201645377706\\
533	0.00617282830798182\\
534	0.00623593928816131\\
535	0.00630147018111279\\
536	0.00636949234990505\\
537	0.00644006104353074\\
538	0.00651325385171948\\
539	0.0065891142941634\\
540	0.00666763728431354\\
541	0.00674877479497815\\
542	0.00683243735517887\\
543	0.00691848790243981\\
544	0.00700678142595555\\
545	0.00709710460458226\\
546	0.00718914909923042\\
547	0.00728233011489576\\
548	0.00737567760376649\\
549	0.00746556707184357\\
550	0.00755055199800488\\
551	0.00763014193807071\\
552	0.00770374944381809\\
553	0.00777122664892783\\
554	0.00783558854786247\\
555	0.00789783794406934\\
556	0.00795825232183846\\
557	0.0080172196760148\\
558	0.00807556562364243\\
559	0.00813380648727646\\
560	0.00819215789604154\\
561	0.00825081058869847\\
562	0.00830995726729622\\
563	0.00836978176436801\\
564	0.00843009923217756\\
565	0.00848930666147047\\
566	0.00854719485207219\\
567	0.00860348792501928\\
568	0.00865888706570304\\
569	0.0087141507280509\\
570	0.00876926249803578\\
571	0.00882426452232611\\
572	0.00887920168507198\\
573	0.00893313527028819\\
574	0.00898517449966207\\
575	0.00903522942231062\\
576	0.00908474529495557\\
577	0.00913376135612379\\
578	0.00918233514225318\\
579	0.00923013735242427\\
580	0.00927705205247275\\
581	0.00932349070899567\\
582	0.00936953410212466\\
583	0.00941516998195261\\
584	0.00946036084630153\\
585	0.00950506346562379\\
586	0.00954923819722844\\
587	0.0095928530922261\\
588	0.00963588745565751\\
589	0.00967833566889083\\
590	0.0097202108171807\\
591	0.00976138933709939\\
592	0.00980171928775089\\
593	0.00984102135019415\\
594	0.00987905595166784\\
595	0.00991543381058343\\
596	0.00994937493726701\\
597	0.00997906286423442\\
598	0.0099999191923403\\
599	0\\
600	0\\
};
\addplot [color=red!50!mycolor17,solid,forget plot]
  table[row sep=crcr]{%
1	0.00283870092861365\\
2	0.00283870433655056\\
3	0.00283870780564254\\
4	0.00283871133698597\\
5	0.00283871493169682\\
6	0.0028387185909111\\
7	0.00283872231578512\\
8	0.00283872610749591\\
9	0.00283872996724156\\
10	0.00283873389624159\\
11	0.00283873789573735\\
12	0.00283874196699244\\
13	0.00283874611129304\\
14	0.00283875032994839\\
15	0.00283875462429111\\
16	0.00283875899567769\\
17	0.00283876344548891\\
18	0.00283876797513025\\
19	0.00283877258603231\\
20	0.00283877727965132\\
21	0.00283878205746957\\
22	0.00283878692099581\\
23	0.00283879187176586\\
24	0.00283879691134296\\
25	0.00283880204131834\\
26	0.0028388072633117\\
27	0.00283881257897167\\
28	0.00283881798997637\\
29	0.00283882349803399\\
30	0.00283882910488322\\
31	0.00283883481229385\\
32	0.00283884062206733\\
33	0.00283884653603731\\
34	0.00283885255607024\\
35	0.0028388586840659\\
36	0.0028388649219581\\
37	0.00283887127171516\\
38	0.00283887773534063\\
39	0.00283888431487382\\
40	0.00283889101239058\\
41	0.00283889783000374\\
42	0.00283890476986399\\
43	0.00283891183416043\\
44	0.00283891902512122\\
45	0.00283892634501442\\
46	0.00283893379614855\\
47	0.00283894138087341\\
48	0.00283894910158078\\
49	0.00283895696070514\\
50	0.00283896496072446\\
51	0.00283897310416097\\
52	0.00283898139358199\\
53	0.00283898983160067\\
54	0.00283899842087676\\
55	0.00283900716411757\\
56	0.00283901606407876\\
57	0.00283902512356513\\
58	0.00283903434543156\\
59	0.00283904373258396\\
60	0.00283905328798007\\
61	0.00283906301463037\\
62	0.00283907291559913\\
63	0.00283908299400531\\
64	0.00283909325302345\\
65	0.00283910369588481\\
66	0.00283911432587824\\
67	0.0028391251463513\\
68	0.00283913616071125\\
69	0.00283914737242611\\
70	0.00283915878502576\\
71	0.00283917040210304\\
72	0.00283918222731482\\
73	0.0028391942643832\\
74	0.0028392065170966\\
75	0.00283921898931104\\
76	0.00283923168495119\\
77	0.00283924460801173\\
78	0.00283925776255847\\
79	0.00283927115272973\\
80	0.00283928478273752\\
81	0.00283929865686889\\
82	0.00283931277948727\\
83	0.00283932715503378\\
84	0.00283934178802866\\
85	0.00283935668307262\\
86	0.0028393718448483\\
87	0.0028393872781217\\
88	0.00283940298774367\\
89	0.00283941897865134\\
90	0.00283943525586976\\
91	0.0028394518245134\\
92	0.00283946868978772\\
93	0.00283948585699075\\
94	0.00283950333151479\\
95	0.00283952111884801\\
96	0.0028395392245762\\
97	0.00283955765438446\\
98	0.00283957641405893\\
99	0.00283959550948862\\
100	0.00283961494666721\\
101	0.0028396347316949\\
102	0.00283965487078025\\
103	0.00283967537024211\\
104	0.00283969623651155\\
105	0.00283971747613387\\
106	0.0028397390957706\\
107	0.00283976110220152\\
108	0.00283978350232674\\
109	0.00283980630316888\\
110	0.00283982951187513\\
111	0.00283985313571945\\
112	0.00283987718210492\\
113	0.00283990165856587\\
114	0.00283992657277022\\
115	0.00283995193252184\\
116	0.0028399777457629\\
117	0.00284000402057639\\
118	0.00284003076518851\\
119	0.00284005798797113\\
120	0.00284008569744448\\
121	0.00284011390227963\\
122	0.00284014261130115\\
123	0.00284017183348989\\
124	0.00284020157798559\\
125	0.00284023185408969\\
126	0.00284026267126823\\
127	0.00284029403915462\\
128	0.00284032596755265\\
129	0.0028403584664394\\
130	0.00284039154596828\\
131	0.00284042521647216\\
132	0.00284045948846641\\
133	0.00284049437265214\\
134	0.00284052987991943\\
135	0.00284056602135055\\
136	0.00284060280822343\\
137	0.00284064025201492\\
138	0.00284067836440439\\
139	0.00284071715727713\\
140	0.00284075664272799\\
141	0.002840796833065\\
142	0.00284083774081305\\
143	0.00284087937871771\\
144	0.00284092175974898\\
145	0.00284096489710522\\
146	0.00284100880421708\\
147	0.00284105349475156\\
148	0.00284109898261603\\
149	0.00284114528196243\\
150	0.0028411924071915\\
151	0.00284124037295709\\
152	0.00284128919417041\\
153	0.00284133888600459\\
154	0.00284138946389919\\
155	0.00284144094356474\\
156	0.00284149334098739\\
157	0.00284154667243366\\
158	0.00284160095445534\\
159	0.00284165620389425\\
160	0.00284171243788735\\
161	0.00284176967387167\\
162	0.00284182792958957\\
163	0.00284188722309395\\
164	0.0028419475727535\\
165	0.00284200899725816\\
166	0.00284207151562455\\
167	0.00284213514720164\\
168	0.00284219991167635\\
169	0.00284226582907936\\
170	0.00284233291979086\\
171	0.00284240120454669\\
172	0.00284247070444422\\
173	0.00284254144094859\\
174	0.00284261343589891\\
175	0.00284268671151462\\
176	0.00284276129040196\\
177	0.0028428371955605\\
178	0.00284291445038981\\
179	0.00284299307869625\\
180	0.00284307310469975\\
181	0.00284315455304094\\
182	0.00284323744878815\\
183	0.00284332181744459\\
184	0.00284340768495583\\
185	0.00284349507771704\\
186	0.00284358402258066\\
187	0.00284367454686415\\
188	0.00284376667835761\\
189	0.00284386044533187\\
190	0.00284395587654643\\
191	0.00284405300125773\\
192	0.00284415184922738\\
193	0.00284425245073059\\
194	0.00284435483656481\\
195	0.0028444590380584\\
196	0.00284456508707942\\
197	0.00284467301604467\\
198	0.00284478285792881\\
199	0.00284489464627353\\
200	0.00284500841519708\\
201	0.00284512419940369\\
202	0.00284524203419337\\
203	0.00284536195547169\\
204	0.00284548399975977\\
205	0.00284560820420451\\
206	0.00284573460658881\\
207	0.00284586324534205\\
208	0.00284599415955079\\
209	0.00284612738896945\\
210	0.00284626297403133\\
211	0.00284640095585972\\
212	0.00284654137627915\\
213	0.00284668427782694\\
214	0.00284682970376474\\
215	0.00284697769809036\\
216	0.00284712830554981\\
217	0.00284728157164946\\
218	0.0028474375426683\\
219	0.00284759626567062\\
220	0.00284775778851866\\
221	0.00284792215988554\\
222	0.00284808942926841\\
223	0.00284825964700172\\
224	0.00284843286427077\\
225	0.00284860913312545\\
226	0.00284878850649403\\
227	0.00284897103819749\\
228	0.00284915678296373\\
229	0.00284934579644215\\
230	0.00284953813521847\\
231	0.00284973385682968\\
232	0.00284993301977929\\
233	0.00285013568355279\\
234	0.00285034190863333\\
235	0.00285055175651761\\
236	0.00285076528973209\\
237	0.00285098257184941\\
238	0.0028512036675049\\
239	0.00285142864241372\\
240	0.00285165756338779\\
241	0.00285189049835336\\
242	0.00285212751636863\\
243	0.00285236868764177\\
244	0.00285261408354902\\
245	0.00285286377665339\\
246	0.00285311784072332\\
247	0.00285337635075185\\
248	0.00285363938297592\\
249	0.00285390701489617\\
250	0.0028541793252969\\
251	0.00285445639426639\\
252	0.00285473830321754\\
253	0.00285502513490887\\
254	0.00285531697346588\\
255	0.00285561390440267\\
256	0.00285591601464397\\
257	0.0028562233925476\\
258	0.00285653612792724\\
259	0.00285685431207555\\
260	0.00285717803778778\\
261	0.00285750739938577\\
262	0.00285784249274233\\
263	0.00285818341530614\\
264	0.00285853026612701\\
265	0.00285888314588166\\
266	0.00285924215690008\\
267	0.00285960740319219\\
268	0.00285997899047519\\
269	0.00286035702620138\\
270	0.00286074161958659\\
271	0.0028611328816392\\
272	0.00286153092518973\\
273	0.00286193586492109\\
274	0.00286234781739955\\
275	0.00286276690110634\\
276	0.00286319323647004\\
277	0.00286362694589979\\
278	0.00286406815381909\\
279	0.00286451698670082\\
280	0.0028649735731028\\
281	0.0028654380437045\\
282	0.00286591053134476\\
283	0.00286639117106043\\
284	0.00286688010012632\\
285	0.00286737745809612\\
286	0.00286788338684473\\
287	0.0028683980306119\\
288	0.00286892153604727\\
289	0.00286945405225691\\
290	0.00286999573085146\\
291	0.00287054672599602\\
292	0.00287110719446185\\
293	0.00287167729567994\\
294	0.0028722571917969\\
295	0.00287284704773271\\
296	0.00287344703124124\\
297	0.00287405731297303\\
298	0.00287467806654102\\
299	0.00287530946858908\\
300	0.00287595169886375\\
301	0.00287660494028946\\
302	0.0028772693790472\\
303	0.00287794520465731\\
304	0.00287863261006642\\
305	0.00287933179173891\\
306	0.00288004294975335\\
307	0.00288076628790415\\
308	0.00288150201380895\\
309	0.00288225033902219\\
310	0.0028830114791553\\
311	0.002883785654004\\
312	0.00288457308768344\\
313	0.00288537400877162\\
314	0.00288618865046179\\
315	0.00288701725072479\\
316	0.00288786005248165\\
317	0.00288871730378771\\
318	0.00288958925802896\\
319	0.00289047617413145\\
320	0.0028913783167851\\
321	0.00289229595668272\\
322	0.00289322937077541\\
323	0.0028941788425457\\
324	0.00289514466229953\\
325	0.00289612712747834\\
326	0.0028971265429927\\
327	0.00289814322157858\\
328	0.00289917748417769\\
329	0.00290022966034285\\
330	0.00290130008866944\\
331	0.00290238911725369\\
332	0.00290349710417811\\
333	0.00290462441802381\\
334	0.00290577143840861\\
335	0.00290693855654915\\
336	0.00290812617584297\\
337	0.00290933471246505\\
338	0.00291056459596982\\
339	0.00291181626988715\\
340	0.00291309019229849\\
341	0.00291438683638664\\
342	0.0029157066909734\\
343	0.00291705026105032\\
344	0.00291841806776943\\
345	0.00291981064685911\\
346	0.0029212285484882\\
347	0.00292267233700952\\
348	0.00292414259058094\\
349	0.00292563990067231\\
350	0.00292716487147907\\
351	0.002928718119271\\
352	0.00293030027168571\\
353	0.00293191196688974\\
354	0.00293355385245548\\
355	0.0029352265849003\\
356	0.00293693084065037\\
357	0.00293866731783082\\
358	0.00294043673738679\\
359	0.0029422398442847\\
360	0.00294407740880065\\
361	0.00294595022790378\\
362	0.00294785912674309\\
363	0.00294980496024692\\
364	0.00295178861484537\\
365	0.00295381101032717\\
366	0.00295587310184337\\
367	0.00295797588207172\\
368	0.00296012038355706\\
369	0.00296230768124428\\
370	0.00296453889522294\\
371	0.00296681519370352\\
372	0.00296913779624821\\
373	0.00297150797728118\\
374	0.00297392706990552\\
375	0.0029763964700572\\
376	0.00297891764102881\\
377	0.00298149211839925\\
378	0.00298412151540846\\
379	0.00298680752881971\\
380	0.00298955194531476\\
381	0.00299235664847091\\
382	0.0029952236263704\\
383	0.00299815497989572\\
384	0.0030011529317642\\
385	0.00300421983635425\\
386	0.00300735819037249\\
387	0.00301057064440294\\
388	0.003013860015367\\
389	0.00301722929990288\\
390	0.00302068168864234\\
391	0.00302422058131805\\
392	0.00302784960257062\\
393	0.00303157261823363\\
394	0.00303539375174873\\
395	0.00303931740018923\\
396	0.00304334824912741\\
397	0.00304749128523309\\
398	0.00305175180497379\\
399	0.00305613541720487\\
400	0.00306064803684511\\
401	0.00306529586542185\\
402	0.00307008535243514\\
403	0.00307502312943827\\
404	0.00308011590570261\\
405	0.00308537031020121\\
406	0.00309079265898533\\
407	0.00309638861924857\\
408	0.0031021627305885\\
409	0.00310811772883947\\
410	0.00311425359633487\\
411	0.00312056623196952\\
412	0.00312704559457391\\
413	0.00313367314836349\\
414	0.00314041859936968\\
415	0.00314727139920554\\
416	0.00315423369614006\\
417	0.00316130776674751\\
418	0.0031684960276916\\
419	0.0031758010557658\\
420	0.00318322565622549\\
421	0.00319077296565129\\
422	0.00319844631403471\\
423	0.0032062490418238\\
424	0.003214183896715\\
425	0.00322225377701107\\
426	0.00323046174710021\\
427	0.00323881105488027\\
428	0.00324730515158386\\
429	0.00325594771461001\\
430	0.00326474267408322\\
431	0.00327369424394737\\
432	0.00328280695904933\\
433	0.00329208572344072\\
434	0.00330153588294981\\
435	0.00331116331548502\\
436	0.00332097451830581\\
437	0.00333097671219466\\
438	0.00334117796620837\\
439	0.00335158734756848\\
440	0.00336221510235997\\
441	0.00337307287410827\\
442	0.00338417396907542\\
443	0.00339553367935669\\
444	0.00340716967768397\\
445	0.0034191025013505\\
446	0.00343135614685067\\
447	0.0034439588011638\\
448	0.0034569437378115\\
449	0.00347035039750513\\
450	0.00348422562530908\\
451	0.00349862484951891\\
452	0.00351361518620998\\
453	0.00352928220505041\\
454	0.00354573489592987\\
455	0.0035631115355953\\
456	0.0035815654237004\\
457	0.00360128540371144\\
458	0.00362250512451155\\
459	0.00364551475566122\\
460	0.00367067542333025\\
461	0.00369843548734482\\
462	0.00372934297113895\\
463	0.00376095621363585\\
464	0.00379309105106504\\
465	0.00382577263716753\\
466	0.00385901484731637\\
467	0.00389282359395008\\
468	0.00392720376105065\\
469	0.00396215890131907\\
470	0.00399769086849669\\
471	0.00403379936012597\\
472	0.00407048133580928\\
473	0.00410773020443479\\
474	0.00414553463487091\\
475	0.00418387698318528\\
476	0.00422273139401959\\
477	0.00426206186955287\\
478	0.00430182020803441\\
479	0.00434194273055568\\
480	0.0043823461022077\\
481	0.00442292198487946\\
482	0.00446353017886259\\
483	0.00450398980586727\\
484	0.00454406795040861\\
485	0.00458346500642036\\
486	0.00462179578739872\\
487	0.00465856535452245\\
488	0.00469313896240186\\
489	0.00472470844453819\\
490	0.0047530155955093\\
491	0.00478167072697865\\
492	0.00481065613380342\\
493	0.00483995091478616\\
494	0.00486953070053233\\
495	0.0048993674315073\\
496	0.00492942922729178\\
497	0.00495968036195307\\
498	0.00499008137895352\\
499	0.00502058946022384\\
500	0.00505115916793818\\
501	0.00508174373103509\\
502	0.00511229716796209\\
503	0.00514277748512175\\
504	0.00517315143935459\\
505	0.00520340138261903\\
506	0.00523353332555396\\
507	0.00526359232558662\\
508	0.005293685216728\\
509	0.00532401058150215\\
510	0.00535487809816213\\
511	0.00538650320169562\\
512	0.00541896601711208\\
513	0.0054523077533292\\
514	0.00548657443382623\\
515	0.00552181785933801\\
516	0.00555809814807075\\
517	0.00559548305717446\\
518	0.00563404906956839\\
519	0.00567388228072107\\
520	0.00571507818832832\\
521	0.00575774030131697\\
522	0.00580198076408421\\
523	0.00584791935643933\\
524	0.00589567988748809\\
525	0.00594538341309121\\
526	0.00599709568295925\\
527	0.00605087853310064\\
528	0.00610682270070741\\
529	0.00616501674035226\\
530	0.00622554406078374\\
531	0.00628847479267532\\
532	0.00635385821429033\\
533	0.00642172861966806\\
534	0.00649209994009733\\
535	0.00656495654067599\\
536	0.006640281558746\\
537	0.00671802543277983\\
538	0.00679805038301329\\
539	0.00688017893851313\\
540	0.00696422364181383\\
541	0.00704985310511505\\
542	0.00713651715111697\\
543	0.0072233324844939\\
544	0.00730733886327977\\
545	0.00738638960302661\\
546	0.00745982539142606\\
547	0.00752713644055451\\
548	0.00758836051141061\\
549	0.00764653857858205\\
550	0.00770271529057265\\
551	0.00775720579591524\\
552	0.00781071095053047\\
553	0.00786393070041828\\
554	0.00791723492564974\\
555	0.00797081310434071\\
556	0.0080248179508191\\
557	0.00807939150536912\\
558	0.00813468278734686\\
559	0.00819081823622241\\
560	0.0082479087802371\\
561	0.00830477765193133\\
562	0.00836059325980605\\
563	0.00841510282450088\\
564	0.00846832031068026\\
565	0.00852161182516532\\
566	0.00857497379044075\\
567	0.00862842855953914\\
568	0.00868200623738066\\
569	0.00873571853626056\\
570	0.00878953117589831\\
571	0.0088418987105902\\
572	0.00889242540907162\\
573	0.00894187213284206\\
574	0.00899097596863721\\
575	0.00903979684709103\\
576	0.0090883736771921\\
577	0.00913644094746051\\
578	0.00918373619157009\\
579	0.00923062808335025\\
580	0.0092772493210446\\
581	0.00932358230216704\\
582	0.00936958177756908\\
583	0.00941519618323153\\
584	0.00946037485366822\\
585	0.00950507047593196\\
586	0.00954924140305753\\
587	0.00959285439625311\\
588	0.00963588790901514\\
589	0.0096783357954337\\
590	0.00972021084245027\\
591	0.00976138933978722\\
592	0.00980171928775089\\
593	0.00984102135019415\\
594	0.00987905595166784\\
595	0.00991543381058343\\
596	0.00994937493726701\\
597	0.00997906286423442\\
598	0.0099999191923403\\
599	0\\
600	0\\
};
\addplot [color=red!40!mycolor19,solid,forget plot]
  table[row sep=crcr]{%
1	0.0030112771484627\\
2	0.00301128212363723\\
3	0.00301128718813487\\
4	0.00301129234355798\\
5	0.00301129759153769\\
6	0.00301130293373427\\
7	0.0030113083718378\\
8	0.00301131390756862\\
9	0.00301131954267788\\
10	0.00301132527894816\\
11	0.00301133111819388\\
12	0.00301133706226203\\
13	0.00301134311303272\\
14	0.00301134927241964\\
15	0.00301135554237081\\
16	0.00301136192486916\\
17	0.00301136842193308\\
18	0.00301137503561719\\
19	0.00301138176801284\\
20	0.00301138862124889\\
21	0.00301139559749225\\
22	0.00301140269894872\\
23	0.00301140992786355\\
24	0.00301141728652219\\
25	0.00301142477725101\\
26	0.00301143240241807\\
27	0.00301144016443379\\
28	0.00301144806575179\\
29	0.00301145610886957\\
30	0.00301146429632934\\
31	0.00301147263071887\\
32	0.00301148111467217\\
33	0.00301148975087046\\
34	0.00301149854204294\\
35	0.00301150749096766\\
36	0.00301151660047236\\
37	0.00301152587343541\\
38	0.00301153531278668\\
39	0.00301154492150848\\
40	0.00301155470263645\\
41	0.00301156465926061\\
42	0.0030115747945262\\
43	0.00301158511163475\\
44	0.00301159561384511\\
45	0.00301160630447436\\
46	0.00301161718689893\\
47	0.00301162826455568\\
48	0.0030116395409429\\
49	0.0030116510196215\\
50	0.00301166270421601\\
51	0.00301167459841581\\
52	0.00301168670597626\\
53	0.00301169903071984\\
54	0.00301171157653739\\
55	0.00301172434738933\\
56	0.00301173734730686\\
57	0.00301175058039324\\
58	0.0030117640508251\\
59	0.00301177776285367\\
60	0.00301179172080623\\
61	0.00301180592908736\\
62	0.00301182039218036\\
63	0.0030118351146486\\
64	0.00301185010113706\\
65	0.00301186535637363\\
66	0.00301188088517075\\
67	0.00301189669242676\\
68	0.00301191278312753\\
69	0.00301192916234797\\
70	0.00301194583525363\\
71	0.0030119628071023\\
72	0.0030119800832457\\
73	0.00301199766913105\\
74	0.00301201557030285\\
75	0.00301203379240457\\
76	0.00301205234118046\\
77	0.00301207122247726\\
78	0.00301209044224609\\
79	0.00301211000654424\\
80	0.00301212992153708\\
81	0.00301215019350007\\
82	0.00301217082882056\\
83	0.00301219183399986\\
84	0.00301221321565528\\
85	0.00301223498052212\\
86	0.00301225713545579\\
87	0.003012279687434\\
88	0.00301230264355882\\
89	0.00301232601105897\\
90	0.00301234979729207\\
91	0.00301237400974681\\
92	0.0030123986560454\\
93	0.00301242374394588\\
94	0.0030124492813445\\
95	0.00301247527627825\\
96	0.00301250173692723\\
97	0.00301252867161732\\
98	0.00301255608882265\\
99	0.00301258399716825\\
100	0.00301261240543275\\
101	0.00301264132255103\\
102	0.00301267075761706\\
103	0.00301270071988666\\
104	0.00301273121878037\\
105	0.00301276226388639\\
106	0.00301279386496348\\
107	0.00301282603194397\\
108	0.00301285877493694\\
109	0.00301289210423115\\
110	0.00301292603029838\\
111	0.00301296056379656\\
112	0.00301299571557307\\
113	0.00301303149666809\\
114	0.00301306791831796\\
115	0.0030131049919587\\
116	0.00301314272922949\\
117	0.0030131811419762\\
118	0.00301322024225507\\
119	0.00301326004233647\\
120	0.00301330055470852\\
121	0.00301334179208103\\
122	0.00301338376738934\\
123	0.00301342649379836\\
124	0.00301346998470643\\
125	0.00301351425374962\\
126	0.00301355931480574\\
127	0.00301360518199869\\
128	0.00301365186970269\\
129	0.00301369939254675\\
130	0.00301374776541901\\
131	0.00301379700347138\\
132	0.00301384712212412\\
133	0.00301389813707056\\
134	0.0030139500642818\\
135	0.00301400292001168\\
136	0.00301405672080159\\
137	0.00301411148348558\\
138	0.00301416722519542\\
139	0.00301422396336584\\
140	0.00301428171573976\\
141	0.00301434050037368\\
142	0.00301440033564319\\
143	0.00301446124024844\\
144	0.00301452323321978\\
145	0.00301458633392366\\
146	0.00301465056206828\\
147	0.00301471593770963\\
148	0.0030147824812575\\
149	0.00301485021348162\\
150	0.00301491915551792\\
151	0.00301498932887487\\
152	0.00301506075543991\\
153	0.00301513345748602\\
154	0.00301520745767843\\
155	0.00301528277908132\\
156	0.00301535944516481\\
157	0.00301543747981191\\
158	0.00301551690732563\\
159	0.00301559775243631\\
160	0.00301568004030886\\
161	0.00301576379655038\\
162	0.00301584904721765\\
163	0.00301593581882494\\
164	0.00301602413835183\\
165	0.00301611403325125\\
166	0.00301620553145757\\
167	0.00301629866139481\\
168	0.00301639345198515\\
169	0.00301648993265734\\
170	0.00301658813335542\\
171	0.00301668808454757\\
172	0.00301678981723495\\
173	0.00301689336296089\\
174	0.00301699875382012\\
175	0.00301710602246815\\
176	0.00301721520213082\\
177	0.00301732632661396\\
178	0.00301743943031334\\
179	0.00301755454822457\\
180	0.00301767171595336\\
181	0.00301779096972585\\
182	0.00301791234639907\\
183	0.00301803588347162\\
184	0.00301816161909453\\
185	0.00301828959208228\\
186	0.00301841984192393\\
187	0.00301855240879453\\
188	0.00301868733356669\\
189	0.00301882465782218\\
190	0.00301896442386401\\
191	0.00301910667472834\\
192	0.00301925145419692\\
193	0.0030193988068095\\
194	0.00301954877787644\\
195	0.00301970141349166\\
196	0.00301985676054566\\
197	0.00302001486673878\\
198	0.0030201757805947\\
199	0.00302033955147407\\
200	0.00302050622958839\\
201	0.00302067586601417\\
202	0.00302084851270721\\
203	0.00302102422251701\\
204	0.00302120304920175\\
205	0.00302138504744302\\
206	0.00302157027286114\\
207	0.00302175878203055\\
208	0.00302195063249546\\
209	0.00302214588278573\\
210	0.00302234459243299\\
211	0.00302254682198701\\
212	0.00302275263303224\\
213	0.00302296208820471\\
214	0.0030231752512091\\
215	0.00302339218683602\\
216	0.0030236129609797\\
217	0.00302383764065569\\
218	0.00302406629401913\\
219	0.00302429899038295\\
220	0.00302453580023652\\
221	0.00302477679526461\\
222	0.00302502204836646\\
223	0.00302527163367521\\
224	0.0030255256265776\\
225	0.00302578410373393\\
226	0.00302604714309833\\
227	0.00302631482393923\\
228	0.00302658722686019\\
229	0.00302686443382102\\
230	0.00302714652815906\\
231	0.00302743359461096\\
232	0.00302772571933458\\
233	0.00302802298993118\\
234	0.00302832549546807\\
235	0.00302863332650143\\
236	0.0030289465750994\\
237	0.00302926533486556\\
238	0.00302958970096271\\
239	0.00302991977013685\\
240	0.00303025564074161\\
241	0.00303059741276294\\
242	0.00303094518784404\\
243	0.00303129906931069\\
244	0.00303165916219689\\
245	0.00303202557327074\\
246	0.0030323984110607\\
247	0.00303277778588221\\
248	0.00303316380986452\\
249	0.00303355659697799\\
250	0.00303395626306149\\
251	0.00303436292585038\\
252	0.0030347767050046\\
253	0.00303519772213726\\
254	0.0030356261008434\\
255	0.0030360619667292\\
256	0.00303650544744144\\
257	0.0030369566726973\\
258	0.00303741577431452\\
259	0.00303788288624183\\
260	0.00303835814458976\\
261	0.00303884168766177\\
262	0.00303933365598564\\
263	0.00303983419234523\\
264	0.00304034344181264\\
265	0.00304086155178052\\
266	0.00304138867199482\\
267	0.00304192495458787\\
268	0.00304247055411169\\
269	0.00304302562757174\\
270	0.00304359033446077\\
271	0.00304416483679328\\
272	0.00304474929914\\
273	0.00304534388866292\\
274	0.00304594877515046\\
275	0.00304656413105293\\
276	0.00304719013151849\\
277	0.00304782695442915\\
278	0.0030484747804373\\
279	0.0030491337930023\\
280	0.00304980417842765\\
281	0.00305048612589808\\
282	0.00305117982751735\\
283	0.00305188547834593\\
284	0.00305260327643924\\
285	0.00305333342288612\\
286	0.00305407612184748\\
287	0.00305483158059534\\
288	0.00305560000955211\\
289	0.00305638162233016\\
290	0.00305717663577175\\
291	0.00305798526998912\\
292	0.00305880774840501\\
293	0.00305964429779349\\
294	0.00306049514832087\\
295	0.00306136053358738\\
296	0.0030622406906688\\
297	0.0030631358601587\\
298	0.00306404628621093\\
299	0.00306497221658262\\
300	0.00306591390267762\\
301	0.00306687159959031\\
302	0.00306784556615013\\
303	0.00306883606496656\\
304	0.00306984336247477\\
305	0.00307086772898195\\
306	0.00307190943871459\\
307	0.00307296876986641\\
308	0.00307404600464744\\
309	0.0030751414293342\\
310	0.00307625533432112\\
311	0.00307738801417344\\
312	0.00307853976768165\\
313	0.00307971089791798\\
314	0.00308090171229491\\
315	0.00308211252262612\\
316	0.00308334364519038\\
317	0.00308459540079854\\
318	0.00308586811486459\\
319	0.00308716211748075\\
320	0.00308847774349799\\
321	0.00308981533261227\\
322	0.00309117522945808\\
323	0.00309255778370995\\
324	0.00309396335019373\\
325	0.00309539228900961\\
326	0.00309684496566861\\
327	0.00309832175124565\\
328	0.00309982302255223\\
329	0.00310134916233282\\
330	0.00310290055949007\\
331	0.00310447760934495\\
332	0.00310608071393943\\
333	0.00310771028239149\\
334	0.00310936673131448\\
335	0.0031110504853147\\
336	0.00311276197758505\\
337	0.00311450165061221\\
338	0.00311626995701265\\
339	0.00311806736049661\\
340	0.00311989433692556\\
341	0.00312175137539228\\
342	0.00312363897943671\\
343	0.003125557670164\\
344	0.0031275080020705\\
345	0.00312949059349016\\
346	0.00313150609232436\\
347	0.00313355517864743\\
348	0.00313563856738285\\
349	0.0031377570110202\\
350	0.00313991130239194\\
351	0.00314210227763664\\
352	0.00314433081954981\\
353	0.00314659786080227\\
354	0.00314890438113493\\
355	0.00315125136603214\\
356	0.00315363958019717\\
357	0.00315606978981859\\
358	0.00315854278111835\\
359	0.00316105936144412\\
360	0.00316362036046411\\
361	0.00316622663147663\\
362	0.00316887905284827\\
363	0.00317157852959675\\
364	0.00317432599513677\\
365	0.00317712241320975\\
366	0.00317996878002174\\
367	0.00318286612661715\\
368	0.00318581552152086\\
369	0.00318881807368557\\
370	0.00319187493578809\\
371	0.0031949873079246\\
372	0.00319815644176414\\
373	0.00320138364522874\\
374	0.0032046702877813\\
375	0.00320801780641599\\
376	0.00321142771246325\\
377	0.00321490159934192\\
378	0.00321844115141562\\
379	0.0032220481541401\\
380	0.00322572450572489\\
381	0.00322947223057536\\
382	0.003233293494836\\
383	0.00323719062441954\\
384	0.00324116612598733\\
385	0.0032452227114445\\
386	0.00324936332663442\\
387	0.00325359118506684\\
388	0.00325790980770063\\
389	0.00326232307003289\\
390	0.00326683525803616\\
391	0.00327145113484695\\
392	0.00327617602056441\\
393	0.00328101588809075\\
394	0.00328597747866928\\
395	0.00329106844169038\\
396	0.00329629750449335\\
397	0.00330167467934666\\
398	0.00330721151660432\\
399	0.00331292141582399\\
400	0.0033188200096813\\
401	0.00332492563927259\\
402	0.00333125994479976\\
403	0.00333784860241919\\
404	0.00334472224685625\\
405	0.00335191763085759\\
406	0.00335947908746879\\
407	0.00336746038041965\\
408	0.00337592705245904\\
409	0.00338495941124689\\
410	0.00339465632310242\\
411	0.00340513999615007\\
412	0.0034165618451223\\
413	0.00342910904764097\\
414	0.00344300944672684\\
415	0.00345755801444051\\
416	0.00347234515899185\\
417	0.00348737409996265\\
418	0.00350264800731524\\
419	0.00351816984114086\\
420	0.0035339418067238\\
421	0.00354996568105268\\
422	0.00356624732542498\\
423	0.00358279709431792\\
424	0.00359963373631932\\
425	0.00361676185217926\\
426	0.00363418607612658\\
427	0.00365191106702642\\
428	0.00366994149738172\\
429	0.00368828203972282\\
430	0.00370693734984093\\
431	0.00372591204624422\\
432	0.00374521068514317\\
433	0.00376483772985987\\
434	0.0037847975113347\\
435	0.00380509417336685\\
436	0.00382573160994369\\
437	0.00384671338743754\\
438	0.00386804264795385\\
439	0.00388972198915353\\
440	0.00391175331465244\\
441	0.0039341376475423\\
442	0.00395687489758588\\
443	0.00397996357008652\\
444	0.00400340040115312\\
445	0.00402717989986627\\
446	0.00405129377242333\\
447	0.00407573019637187\\
448	0.0041004729041679\\
449	0.00412550002427073\\
450	0.00415078261515364\\
451	0.00417628281527402\\
452	0.00420195147286436\\
453	0.00422772500064694\\
454	0.00425352122695221\\
455	0.00427923405601563\\
456	0.00430472705272688\\
457	0.00432982539830094\\
458	0.00435430509663297\\
459	0.00437787912462104\\
460	0.00440018001228618\\
461	0.00442073945979756\\
462	0.00443897091856598\\
463	0.00445725960360944\\
464	0.00447580055336444\\
465	0.00449458924557552\\
466	0.00451361983613378\\
467	0.00453288517536736\\
468	0.00455237665129548\\
469	0.00457208401945484\\
470	0.00459199521965731\\
471	0.00461209618082414\\
472	0.00463237061681782\\
473	0.00465279982181396\\
474	0.00467336248284928\\
475	0.00469403452956161\\
476	0.0047147890479902\\
477	0.00473559627477015\\
478	0.00475642369917354\\
479	0.00477723634335092\\
480	0.00479799731346455\\
481	0.00481866874103088\\
482	0.00483921327983798\\
483	0.00485959638845276\\
484	0.00487978971827209\\
485	0.00489977605615691\\
486	0.00491955646208567\\
487	0.00493916053055873\\
488	0.00495866108989554\\
489	0.00497819474011869\\
490	0.00499796145418809\\
491	0.0050180776608813\\
492	0.00503855184396394\\
493	0.00505939348956257\\
494	0.0050806133032376\\
495	0.00510222347856746\\
496	0.00512423801255947\\
497	0.00514667307048559\\
498	0.00516954748368211\\
499	0.00519288328983599\\
500	0.00521670633080219\\
501	0.00524104668724238\\
502	0.00526593805116734\\
503	0.00529141966021656\\
504	0.00531753775955137\\
505	0.00534435061159844\\
506	0.00537195477964907\\
507	0.0054004263733106\\
508	0.00542983977475685\\
509	0.00546026936773654\\
510	0.00549178321330305\\
511	0.00552444340442616\\
512	0.00555831555240889\\
513	0.00559346761621968\\
514	0.00562997120760527\\
515	0.00566789562440533\\
516	0.00570725572049871\\
517	0.00574812831483058\\
518	0.0057905930740293\\
519	0.0058347319712251\\
520	0.00588062854546756\\
521	0.0059283669909499\\
522	0.00597803091455968\\
523	0.00602970178255804\\
524	0.0060834574252944\\
525	0.00613936985774463\\
526	0.00619754555422584\\
527	0.00625807035375208\\
528	0.00632097800128714\\
529	0.00638628421195292\\
530	0.00645398122847025\\
531	0.00652402827496695\\
532	0.00659634488034116\\
533	0.00667080369120464\\
534	0.00674722577632755\\
535	0.00682543605429951\\
536	0.00690517067081394\\
537	0.006985979879753\\
538	0.0070671353684765\\
539	0.00714694473789928\\
540	0.00722171707236833\\
541	0.00729072305803338\\
542	0.00735364306074771\\
543	0.00741047186832885\\
544	0.00746351245621795\\
545	0.00751458096933631\\
546	0.0075642019222949\\
547	0.00761304925249949\\
548	0.00766168557552592\\
549	0.00771048316522664\\
550	0.00775963560327215\\
551	0.00780930056370565\\
552	0.00785961270762311\\
553	0.00791065670204328\\
554	0.00796248760120481\\
555	0.00801519227973258\\
556	0.00806886380586354\\
557	0.00812360765985678\\
558	0.0081776104431567\\
559	0.00823054447228591\\
560	0.00828212481938252\\
561	0.00833330018496336\\
562	0.00838468146777848\\
563	0.00843628943921433\\
564	0.0084881762700586\\
565	0.00854036375010845\\
566	0.00859280326717225\\
567	0.00864547347042305\\
568	0.00869831647947506\\
569	0.00874951291989383\\
570	0.0087989032234281\\
571	0.00884783920707862\\
572	0.00889659845102266\\
573	0.00894523117842181\\
574	0.0089937388115468\\
575	0.00904204227672353\\
576	0.00908965434686499\\
577	0.00913682916240663\\
578	0.0091838363449705\\
579	0.00923066223904429\\
580	0.00927726451044059\\
581	0.00932359040395203\\
582	0.00936958612525249\\
583	0.00941519840888499\\
584	0.00946037591227052\\
585	0.00950507093368084\\
586	0.00954924157818994\\
587	0.00959285445321423\\
588	0.00963588792379213\\
589	0.00967833579815023\\
590	0.00972021084271538\\
591	0.00976138933978722\\
592	0.00980171928775088\\
593	0.00984102135019415\\
594	0.00987905595166784\\
595	0.00991543381058343\\
596	0.00994937493726701\\
597	0.00997906286423442\\
598	0.0099999191923403\\
599	0\\
600	0\\
};
\addplot [color=red!75!mycolor17,solid,forget plot]
  table[row sep=crcr]{%
1	0.00330282272054889\\
2	0.00330283342163309\\
3	0.00330284431478508\\
4	0.00330285540344943\\
5	0.00330286669113244\\
6	0.00330287818140321\\
7	0.00330288987789482\\
8	0.0033029017843054\\
9	0.00330291390439932\\
10	0.00330292624200841\\
11	0.00330293880103316\\
12	0.00330295158544386\\
13	0.00330296459928195\\
14	0.00330297784666129\\
15	0.00330299133176942\\
16	0.00330300505886883\\
17	0.00330301903229842\\
18	0.00330303325647477\\
19	0.00330304773589353\\
20	0.00330306247513091\\
21	0.00330307747884505\\
22	0.00330309275177754\\
23	0.0033031082987548\\
24	0.00330312412468978\\
25	0.00330314023458331\\
26	0.00330315663352584\\
27	0.00330317332669894\\
28	0.00330319031937693\\
29	0.00330320761692861\\
30	0.00330322522481887\\
31	0.00330324314861042\\
32	0.00330326139396562\\
33	0.00330327996664815\\
34	0.0033032988725249\\
35	0.00330331811756777\\
36	0.00330333770785555\\
37	0.00330335764957587\\
38	0.00330337794902708\\
39	0.00330339861262027\\
40	0.0033034196468813\\
41	0.00330344105845283\\
42	0.00330346285409639\\
43	0.00330348504069449\\
44	0.00330350762525285\\
45	0.00330353061490255\\
46	0.00330355401690227\\
47	0.00330357783864058\\
48	0.00330360208763825\\
49	0.00330362677155063\\
50	0.00330365189817003\\
51	0.0033036774754282\\
52	0.00330370351139872\\
53	0.0033037300142997\\
54	0.00330375699249623\\
55	0.00330378445450304\\
56	0.00330381240898714\\
57	0.00330384086477062\\
58	0.00330386983083331\\
59	0.00330389931631573\\
60	0.0033039293305218\\
61	0.00330395988292187\\
62	0.00330399098315565\\
63	0.00330402264103528\\
64	0.00330405486654828\\
65	0.00330408766986083\\
66	0.00330412106132083\\
67	0.00330415505146126\\
68	0.00330418965100335\\
69	0.00330422487086005\\
70	0.0033042607221394\\
71	0.00330429721614801\\
72	0.00330433436439457\\
73	0.00330437217859355\\
74	0.00330441067066876\\
75	0.00330444985275714\\
76	0.00330448973721252\\
77	0.00330453033660952\\
78	0.00330457166374751\\
79	0.0033046137316545\\
80	0.00330465655359143\\
81	0.003304700143056\\
82	0.00330474451378721\\
83	0.00330478967976943\\
84	0.00330483565523691\\
85	0.00330488245467818\\
86	0.00330493009284055\\
87	0.00330497858473473\\
88	0.00330502794563957\\
89	0.00330507819110677\\
90	0.00330512933696575\\
91	0.00330518139932865\\
92	0.00330523439459526\\
93	0.00330528833945823\\
94	0.00330534325090826\\
95	0.00330539914623935\\
96	0.00330545604305425\\
97	0.00330551395926991\\
98	0.00330557291312314\\
99	0.00330563292317622\\
100	0.00330569400832267\\
101	0.00330575618779331\\
102	0.003305819481162\\
103	0.00330588390835195\\
104	0.00330594948964182\\
105	0.00330601624567204\\
106	0.0033060841974513\\
107	0.00330615336636303\\
108	0.00330622377417209\\
109	0.00330629544303151\\
110	0.00330636839548943\\
111	0.00330644265449607\\
112	0.00330651824341091\\
113	0.00330659518600989\\
114	0.00330667350649289\\
115	0.00330675322949116\\
116	0.00330683438007506\\
117	0.00330691698376181\\
118	0.00330700106652339\\
119	0.00330708665479459\\
120	0.00330717377548134\\
121	0.00330726245596888\\
122	0.00330735272413047\\
123	0.0033074446083358\\
124	0.00330753813745996\\
125	0.00330763334089236\\
126	0.0033077302485458\\
127	0.00330782889086571\\
128	0.00330792929883973\\
129	0.00330803150400712\\
130	0.00330813553846866\\
131	0.00330824143489649\\
132	0.00330834922654435\\
133	0.00330845894725767\\
134	0.00330857063148426\\
135	0.00330868431428477\\
136	0.00330880003134363\\
137	0.00330891781898009\\
138	0.00330903771415932\\
139	0.00330915975450399\\
140	0.00330928397830574\\
141	0.00330941042453704\\
142	0.00330953913286325\\
143	0.00330967014365478\\
144	0.00330980349799959\\
145	0.00330993923771584\\
146	0.00331007740536477\\
147	0.00331021804426372\\
148	0.00331036119849963\\
149	0.00331050691294246\\
150	0.00331065523325904\\
151	0.0033108062059271\\
152	0.00331095987824959\\
153	0.00331111629836915\\
154	0.00331127551528291\\
155	0.00331143757885755\\
156	0.00331160253984459\\
157	0.00331177044989592\\
158	0.00331194136157969\\
159	0.00331211532839636\\
160	0.0033122924047951\\
161	0.00331247264619046\\
162	0.00331265610897938\\
163	0.00331284285055829\\
164	0.00331303292934082\\
165	0.00331322640477553\\
166	0.00331342333736411\\
167	0.00331362378867986\\
168	0.00331382782138642\\
169	0.00331403549925693\\
170	0.00331424688719343\\
171	0.0033144620512466\\
172	0.00331468105863596\\
173	0.00331490397777025\\
174	0.00331513087826821\\
175	0.00331536183097977\\
176	0.00331559690800757\\
177	0.00331583618272889\\
178	0.00331607972981778\\
179	0.00331632762526781\\
180	0.0033165799464151\\
181	0.00331683677196163\\
182	0.00331709818199918\\
183	0.00331736425803347\\
184	0.00331763508300882\\
185	0.00331791074133321\\
186	0.00331819131890377\\
187	0.00331847690313258\\
188	0.00331876758297316\\
189	0.0033190634489472\\
190	0.00331936459317178\\
191	0.00331967110938712\\
192	0.00331998309298471\\
193	0.00332030064103599\\
194	0.00332062385232146\\
195	0.00332095282736034\\
196	0.00332128766844062\\
197	0.00332162847964977\\
198	0.0033219753669058\\
199	0.00332232843798895\\
200	0.00332268780257393\\
201	0.00332305357226255\\
202	0.00332342586061704\\
203	0.00332380478319396\\
204	0.00332419045757847\\
205	0.00332458300341937\\
206	0.00332498254246467\\
207	0.00332538919859769\\
208	0.00332580309787378\\
209	0.00332622436855776\\
210	0.00332665314116178\\
211	0.003327089548484\\
212	0.00332753372564779\\
213	0.00332798581014162\\
214	0.00332844594185967\\
215	0.0033289142631429\\
216	0.00332939091882108\\
217	0.0033298760562552\\
218	0.00333036982538093\\
219	0.00333087237875247\\
220	0.00333138387158732\\
221	0.00333190446181167\\
222	0.0033324343101066\\
223	0.00333297357995497\\
224	0.00333352243768914\\
225	0.00333408105253948\\
226	0.00333464959668346\\
227	0.00333522824529588\\
228	0.00333581717659961\\
229	0.00333641657191728\\
230	0.00333702661572385\\
231	0.0033376474956999\\
232	0.00333827940278585\\
233	0.0033389225312371\\
234	0.00333957707867991\\
235	0.00334024324616826\\
236	0.00334092123824171\\
237	0.00334161126298399\\
238	0.00334231353208262\\
239	0.00334302826088954\\
240	0.00334375566848259\\
241	0.00334449597772797\\
242	0.00334524941534377\\
243	0.0033460162119644\\
244	0.00334679660220614\\
245	0.0033475908247336\\
246	0.00334839912232735\\
247	0.00334922174195242\\
248	0.00335005893482809\\
249	0.00335091095649854\\
250	0.00335177806690484\\
251	0.00335266053045776\\
252	0.00335355861611198\\
253	0.00335447259744131\\
254	0.00335540275271499\\
255	0.00335634936497527\\
256	0.00335731272211621\\
257	0.0033582931169635\\
258	0.00335929084735563\\
259	0.00336030621622627\\
260	0.00336133953168787\\
261	0.00336239110711638\\
262	0.0033634612612376\\
263	0.00336455031821432\\
264	0.00336565860773516\\
265	0.0033667864651045\\
266	0.0033679342313338\\
267	0.00336910225323429\\
268	0.00337029088351086\\
269	0.00337150048085747\\
270	0.00337273141005392\\
271	0.00337398404206385\\
272	0.00337525875413432\\
273	0.00337655592989667\\
274	0.00337787595946882\\
275	0.00337921923955908\\
276	0.00338058617357127\\
277	0.00338197717171143\\
278	0.00338339265109577\\
279	0.00338483303586036\\
280	0.00338629875727196\\
281	0.00338779025384068\\
282	0.00338930797143373\\
283	0.00339085236339104\\
284	0.00339242389064208\\
285	0.00339402302182419\\
286	0.0033956502334026\\
287	0.00339730600979159\\
288	0.00339899084347746\\
289	0.00340070523514277\\
290	0.00340244969379195\\
291	0.00340422473687872\\
292	0.00340603089043444\\
293	0.00340786868919834\\
294	0.00340973867674884\\
295	0.0034116414056363\\
296	0.00341357743751731\\
297	0.00341554734328991\\
298	0.00341755170323048\\
299	0.00341959110713156\\
300	0.00342166615444105\\
301	0.00342377745440246\\
302	0.00342592562619619\\
303	0.0034281112990819\\
304	0.00343033511254174\\
305	0.00343259771642438\\
306	0.00343489977108982\\
307	0.00343724194755478\\
308	0.00343962492763856\\
309	0.00344204940410927\\
310	0.00344451608083008\\
311	0.00344702567290555\\
312	0.00344957890682761\\
313	0.00345217652062094\\
314	0.00345481926398756\\
315	0.00345750789845017\\
316	0.00346024319749367\\
317	0.00346302594670484\\
318	0.00346585694390895\\
319	0.00346873699930295\\
320	0.00347166693558459\\
321	0.00347464758807583\\
322	0.00347767980484001\\
323	0.00348076444679074\\
324	0.00348390238779121\\
325	0.00348709451474115\\
326	0.00349034172764914\\
327	0.00349364493968681\\
328	0.0034970050772207\\
329	0.0035004230798166\\
330	0.00350389990020994\\
331	0.00350743650423407\\
332	0.00351103387069592\\
333	0.00351469299118633\\
334	0.00351841486980706\\
335	0.00352220052279181\\
336	0.00352605097798613\\
337	0.00352996727412774\\
338	0.00353395045980457\\
339	0.00353800159178545\\
340	0.00354212173185045\\
341	0.00354631193944075\\
342	0.00355057325157021\\
343	0.00355490662209339\\
344	0.00355931272808235\\
345	0.00356379189166622\\
346	0.00356834515469216\\
347	0.00357297356679901\\
348	0.00357767818776288\\
349	0.00358246009188216\\
350	0.00358732037653211\\
351	0.00359226018023511\\
352	0.00359728072439287\\
353	0.00360238341730273\\
354	0.00360757012792604\\
355	0.00361284393236545\\
356	0.00361821119603823\\
357	0.00362367390982745\\
358	0.00362923375132823\\
359	0.00363489242754245\\
360	0.00364065167534496\\
361	0.00364651326193577\\
362	0.00365247898527317\\
363	0.00365855067448048\\
364	0.00366473019021816\\
365	0.00367101942501133\\
366	0.0036774203035199\\
367	0.00368393478273699\\
368	0.00369056485209686\\
369	0.003697312533471\\
370	0.00370417988102549\\
371	0.00371116898090746\\
372	0.00371828195072195\\
373	0.00372552093875149\\
374	0.00373288812286142\\
375	0.00374038570902078\\
376	0.00374801592935452\\
377	0.00375578103962323\\
378	0.00376368331600495\\
379	0.00377172505102493\\
380	0.00377990854844547\\
381	0.00378823611688516\\
382	0.0037967100618844\\
383	0.00380533267606928\\
384	0.00381410622698426\\
385	0.00382303294206365\\
386	0.00383211499008518\\
387	0.00384135445829135\\
388	0.00385075332416488\\
389	0.00386031342059559\\
390	0.00387003639285955\\
391	0.00387992364543375\\
392	0.0038899762761649\\
393	0.0039001949946742\\
394	0.00391058002108038\\
395	0.00392113096013745\\
396	0.00393184664472379\\
397	0.00394272494144054\\
398	0.00395376251048509\\
399	0.00396495449586735\\
400	0.00397629413257958\\
401	0.00398777225813496\\
402	0.00399937669498494\\
403	0.00401109146763229\\
404	0.00402289580756077\\
405	0.00403476288509919\\
406	0.00404665818896517\\
407	0.00405853745010709\\
408	0.00407034397486529\\
409	0.00408200521173058\\
410	0.0040934283265161\\
411	0.00410449451422482\\
412	0.00411505185245532\\
413	0.00412490688083491\\
414	0.00413381680457718\\
415	0.00414246359764626\\
416	0.00415125134065076\\
417	0.00416018176530774\\
418	0.00416925659035271\\
419	0.00417847751893866\\
420	0.00418784624035994\\
421	0.00419736444267878\\
422	0.00420703380198752\\
423	0.00421685584956833\\
424	0.00422683181202162\\
425	0.00423696240349508\\
426	0.00424724815917203\\
427	0.00425768941180764\\
428	0.00426828626551413\\
429	0.00427903856647297\\
430	0.00428994587021759\\
431	0.00430100740509318\\
432	0.00431222203145255\\
433	0.00432358819609439\\
434	0.00433510388148189\\
435	0.00434676654945613\\
436	0.00435857307901268\\
437	0.00437051969760218\\
438	0.00438260190559301\\
439	0.00439481439365558\\
440	0.00440715095302605\\
441	0.00441960437891222\\
442	0.00443216636776108\\
443	0.00444482740977402\\
444	0.00445757667901259\\
445	0.00447040192479705\\
446	0.00448328937001098\\
447	0.00449622362459359\\
448	0.00450918762620207\\
449	0.00452216262508911\\
450	0.00453512823714873\\
451	0.00454806259823822\\
452	0.00456094266652204\\
453	0.00457374474231479\\
454	0.00458644530390559\\
455	0.00459902229070082\\
456	0.00461145699868897\\
457	0.00462373681301679\\
458	0.00463585913954865\\
459	0.00464783703643459\\
460	0.00465970727157742\\
461	0.00467154172123509\\
462	0.00468346221395803\\
463	0.00469554963119461\\
464	0.00470781029851564\\
465	0.00472024469819738\\
466	0.00473285328017813\\
467	0.00474563649024712\\
468	0.00475859480719211\\
469	0.00477172879083086\\
470	0.00478503914318833\\
471	0.00479852678546889\\
472	0.00481219295390058\\
473	0.0048260393178048\\
474	0.00484006812319904\\
475	0.00485428236509176\\
476	0.0048686859910726\\
477	0.00488328413879259\\
478	0.00489808340948448\\
479	0.00491309217733719\\
480	0.00492832090057559\\
481	0.0049437824638965\\
482	0.00495949254182109\\
483	0.0049754699354219\\
484	0.00499173683140299\\
485	0.00500831890355531\\
486	0.00502524513177611\\
487	0.00504254713582935\\
488	0.00506025766691289\\
489	0.0050784076222585\\
490	0.00509702183527401\\
491	0.00511612026777347\\
492	0.00513572427144415\\
493	0.00515585814725386\\
494	0.00517654827885316\\
495	0.005197823043026\\
496	0.0052197129940454\\
497	0.00524225100489788\\
498	0.00526547092274689\\
499	0.00528940925192003\\
500	0.00531410682208839\\
501	0.00533961427475088\\
502	0.00536600662699488\\
503	0.00539333449201868\\
504	0.00542164384722157\\
505	0.00545093317709244\\
506	0.00548125584165916\\
507	0.00551266631417104\\
508	0.00554522086648375\\
509	0.00557897776063199\\
510	0.00561399747215752\\
511	0.00565034295582015\\
512	0.00568807961924574\\
513	0.00572727514130953\\
514	0.0057679989351695\\
515	0.00581032699449293\\
516	0.00585439423092452\\
517	0.00590027932527563\\
518	0.00594806052659907\\
519	0.00599781429473797\\
520	0.00604961359373404\\
521	0.00610352620502252\\
522	0.00615961228747102\\
523	0.00621792029876914\\
524	0.0062784771031797\\
525	0.0063412833477263\\
526	0.00640631260834304\\
527	0.00647349981735246\\
528	0.00654273540940123\\
529	0.00661386712290187\\
530	0.0066866853889961\\
531	0.00676099618491231\\
532	0.00683645871329546\\
533	0.00691251557173578\\
534	0.00698820960614814\\
535	0.0070600184570712\\
536	0.00712619182055303\\
537	0.00718635518837399\\
538	0.00724038537103614\\
539	0.0072891926067398\\
540	0.00733611246284593\\
541	0.00738175075357763\\
542	0.00742656905747836\\
543	0.00747111430771151\\
544	0.00751581455378255\\
545	0.00756089746622922\\
546	0.00760651824603813\\
547	0.00765279954637135\\
548	0.00769981995261393\\
549	0.00774763076508289\\
550	0.00779627233522115\\
551	0.00784578270503273\\
552	0.00789623511963474\\
553	0.00794771179525126\\
554	0.00800011671826545\\
555	0.00805165496701063\\
556	0.00810209881618632\\
557	0.0081511534873175\\
558	0.00820039187507906\\
559	0.00824993373640694\\
560	0.00829982916153586\\
561	0.00835013282549448\\
562	0.00840084161324492\\
563	0.00845191747075763\\
564	0.00850331152133759\\
565	0.0085549989209369\\
566	0.00860690126516819\\
567	0.00865718321445381\\
568	0.00870571147056739\\
569	0.00875412101410349\\
570	0.00880247830324436\\
571	0.00885082904818314\\
572	0.00889914614462142\\
573	0.00894739710985454\\
574	0.00899525183449353\\
575	0.00904253590053291\\
576	0.00908973605534923\\
577	0.0091368486117437\\
578	0.00918384152338781\\
579	0.00923066477496741\\
580	0.00927726586124061\\
581	0.00932359110588857\\
582	0.00936958646951639\\
583	0.0094151985649821\\
584	0.00946037597636478\\
585	0.00950507095687452\\
586	0.00954924158529672\\
587	0.0095928544549437\\
588	0.00963588792408897\\
589	0.00967833579817709\\
590	0.00972021084271539\\
591	0.00976138933978722\\
592	0.00980171928775089\\
593	0.00984102135019415\\
594	0.00987905595166785\\
595	0.00991543381058343\\
596	0.00994937493726701\\
597	0.00997906286423442\\
598	0.0099999191923403\\
599	0\\
600	0\\
};
\addplot [color=red!80!mycolor19,solid,forget plot]
  table[row sep=crcr]{%
1	0.00391309393964753\\
2	0.00391310035145049\\
3	0.00391310687863073\\
4	0.00391311352326403\\
5	0.0039131202874635\\
6	0.00391312717338031\\
7	0.00391313418320423\\
8	0.00391314131916451\\
9	0.00391314858353044\\
10	0.00391315597861217\\
11	0.00391316350676137\\
12	0.00391317117037203\\
13	0.00391317897188121\\
14	0.00391318691376976\\
15	0.00391319499856321\\
16	0.00391320322883247\\
17	0.00391321160719473\\
18	0.00391322013631422\\
19	0.0039132288189031\\
20	0.00391323765772234\\
21	0.00391324665558253\\
22	0.00391325581534482\\
23	0.00391326513992186\\
24	0.00391327463227862\\
25	0.00391328429543347\\
26	0.003913294132459\\
27	0.00391330414648312\\
28	0.00391331434068997\\
29	0.00391332471832095\\
30	0.00391333528267575\\
31	0.00391334603711346\\
32	0.00391335698505355\\
33	0.00391336812997696\\
34	0.00391337947542727\\
35	0.00391339102501179\\
36	0.00391340278240267\\
37	0.00391341475133813\\
38	0.00391342693562365\\
39	0.00391343933913308\\
40	0.00391345196580999\\
41	0.00391346481966886\\
42	0.00391347790479635\\
43	0.00391349122535264\\
44	0.00391350478557267\\
45	0.00391351858976759\\
46	0.0039135326423261\\
47	0.00391354694771575\\
48	0.00391356151048447\\
49	0.00391357633526195\\
50	0.00391359142676115\\
51	0.0039136067897798\\
52	0.00391362242920189\\
53	0.0039136383499992\\
54	0.00391365455723295\\
55	0.00391367105605535\\
56	0.00391368785171129\\
57	0.00391370494953988\\
58	0.00391372235497636\\
59	0.00391374007355357\\
60	0.00391375811090393\\
61	0.00391377647276107\\
62	0.00391379516496177\\
63	0.00391381419344772\\
64	0.00391383356426739\\
65	0.00391385328357808\\
66	0.00391387335764775\\
67	0.00391389379285705\\
68	0.00391391459570137\\
69	0.00391393577279287\\
70	0.00391395733086256\\
71	0.00391397927676248\\
72	0.00391400161746784\\
73	0.00391402436007925\\
74	0.00391404751182495\\
75	0.00391407108006316\\
76	0.00391409507228431\\
77	0.00391411949611354\\
78	0.00391414435931302\\
79	0.00391416966978445\\
80	0.00391419543557158\\
81	0.00391422166486273\\
82	0.00391424836599338\\
83	0.00391427554744888\\
84	0.00391430321786703\\
85	0.00391433138604094\\
86	0.00391436006092166\\
87	0.00391438925162124\\
88	0.00391441896741536\\
89	0.00391444921774651\\
90	0.00391448001222679\\
91	0.00391451136064106\\
92	0.00391454327295003\\
93	0.00391457575929336\\
94	0.00391460882999292\\
95	0.00391464249555611\\
96	0.00391467676667905\\
97	0.00391471165425006\\
98	0.0039147471693531\\
99	0.00391478332327129\\
100	0.00391482012749046\\
101	0.00391485759370267\\
102	0.00391489573381023\\
103	0.00391493455992908\\
104	0.00391497408439289\\
105	0.00391501431975687\\
106	0.00391505527880174\\
107	0.0039150969745378\\
108	0.00391513942020896\\
109	0.00391518262929707\\
110	0.00391522661552606\\
111	0.00391527139286634\\
112	0.00391531697553919\\
113	0.00391536337802129\\
114	0.00391541061504923\\
115	0.00391545870162422\\
116	0.00391550765301681\\
117	0.00391555748477168\\
118	0.00391560821271267\\
119	0.00391565985294758\\
120	0.00391571242187345\\
121	0.00391576593618165\\
122	0.00391582041286306\\
123	0.00391587586921361\\
124	0.0039159323228396\\
125	0.00391598979166334\\
126	0.00391604829392865\\
127	0.00391610784820688\\
128	0.00391616847340245\\
129	0.00391623018875904\\
130	0.00391629301386557\\
131	0.00391635696866237\\
132	0.00391642207344744\\
133	0.00391648834888292\\
134	0.0039165558160015\\
135	0.00391662449621307\\
136	0.0039166944113115\\
137	0.00391676558348142\\
138	0.0039168380353053\\
139	0.00391691178977044\\
140	0.00391698687027623\\
141	0.00391706330064156\\
142	0.0039171411051122\\
143	0.00391722030836855\\
144	0.00391730093553329\\
145	0.0039173830121792\\
146	0.00391746656433737\\
147	0.00391755161850521\\
148	0.00391763820165484\\
149	0.00391772634124151\\
150	0.00391781606521217\\
151	0.00391790740201436\\
152	0.00391800038060491\\
153	0.00391809503045919\\
154	0.00391819138158023\\
155	0.00391828946450815\\
156	0.00391838931032967\\
157	0.00391849095068785\\
158	0.00391859441779201\\
159	0.00391869974442765\\
160	0.00391880696396692\\
161	0.0039189161103788\\
162	0.0039190272182398\\
163	0.0039191403227448\\
164	0.00391925545971783\\
165	0.00391937266562344\\
166	0.0039194919775779\\
167	0.0039196134333608\\
168	0.00391973707142682\\
169	0.00391986293091758\\
170	0.003919991051674\\
171	0.00392012147424852\\
172	0.00392025423991759\\
173	0.0039203893906947\\
174	0.00392052696934325\\
175	0.00392066701938984\\
176	0.00392080958513773\\
177	0.00392095471168059\\
178	0.00392110244491633\\
179	0.00392125283156142\\
180	0.00392140591916523\\
181	0.00392156175612471\\
182	0.00392172039169928\\
183	0.00392188187602615\\
184	0.00392204626013556\\
185	0.00392221359596661\\
186	0.00392238393638313\\
187	0.00392255733519008\\
188	0.00392273384714987\\
189	0.00392291352799926\\
190	0.00392309643446646\\
191	0.0039232826242884\\
192	0.00392347215622846\\
193	0.00392366509009446\\
194	0.00392386148675686\\
195	0.00392406140816729\\
196	0.00392426491737753\\
197	0.00392447207855869\\
198	0.00392468295702064\\
199	0.00392489761923196\\
200	0.00392511613284003\\
201	0.00392533856669158\\
202	0.00392556499085358\\
203	0.00392579547663438\\
204	0.00392603009660527\\
205	0.00392626892462241\\
206	0.00392651203584904\\
207	0.00392675950677819\\
208	0.00392701141525566\\
209	0.00392726784050333\\
210	0.00392752886314305\\
211	0.00392779456522076\\
212	0.00392806503023093\\
213	0.00392834034314166\\
214	0.00392862059041988\\
215	0.00392890586005726\\
216	0.00392919624159628\\
217	0.00392949182615693\\
218	0.00392979270646361\\
219	0.00393009897687269\\
220	0.00393041073340037\\
221	0.00393072807375105\\
222	0.00393105109734605\\
223	0.00393137990535299\\
224	0.00393171460071542\\
225	0.00393205528818294\\
226	0.00393240207434204\\
227	0.00393275506764708\\
228	0.00393311437845195\\
229	0.00393348011904223\\
230	0.00393385240366776\\
231	0.00393423134857578\\
232	0.00393461707204458\\
233	0.00393500969441757\\
234	0.00393540933813807\\
235	0.00393581612778442\\
236	0.0039362301901057\\
237	0.0039366516540581\\
238	0.00393708065084169\\
239	0.00393751731393771\\
240	0.00393796177914665\\
241	0.0039384141846267\\
242	0.00393887467093278\\
243	0.00393934338105621\\
244	0.00393982046046497\\
245	0.00394030605714449\\
246	0.00394080032163905\\
247	0.00394130340709382\\
248	0.00394181546929747\\
249	0.00394233666672541\\
250	0.00394286716058364\\
251	0.00394340711485321\\
252	0.00394395669633537\\
253	0.00394451607469714\\
254	0.00394508542251787\\
255	0.00394566491533617\\
256	0.00394625473169753\\
257	0.00394685505320269\\
258	0.00394746606455656\\
259	0.00394808795361791\\
260	0.00394872091144956\\
261	0.00394936513236943\\
262	0.00395002081400208\\
263	0.00395068815733114\\
264	0.00395136736675211\\
265	0.0039520586501262\\
266	0.00395276221883456\\
267	0.00395347828783326\\
268	0.00395420707570917\\
269	0.0039549488047362\\
270	0.00395570370093242\\
271	0.00395647199411784\\
272	0.00395725391797283\\
273	0.0039580497100973\\
274	0.00395885961207044\\
275	0.00395968386951128\\
276	0.00396052273213981\\
277	0.00396137645383877\\
278	0.00396224529271628\\
279	0.00396312951116883\\
280	0.00396402937594523\\
281	0.00396494515821093\\
282	0.00396587713361319\\
283	0.00396682558234666\\
284	0.00396779078921978\\
285	0.00396877304372167\\
286	0.00396977264008943\\
287	0.00397078987737639\\
288	0.00397182505952052\\
289	0.00397287849541347\\
290	0.00397395049897024\\
291	0.00397504138919915\\
292	0.00397615149027226\\
293	0.00397728113159628\\
294	0.0039784306478837\\
295	0.00397960037922457\\
296	0.00398079067115801\\
297	0.00398200187474465\\
298	0.00398323434663867\\
299	0.00398448844916045\\
300	0.00398576455036895\\
301	0.00398706302413444\\
302	0.003988384250211\\
303	0.00398972861430908\\
304	0.00399109650816784\\
305	0.00399248832962729\\
306	0.0039939044827002\\
307	0.00399534537764357\\
308	0.00399681143102979\\
309	0.00399830306581716\\
310	0.00399982071141991\\
311	0.0040013648037776\\
312	0.00400293578542367\\
313	0.0040045341055533\\
314	0.00400616022009033\\
315	0.00400781459175329\\
316	0.00400949769012048\\
317	0.00401120999169397\\
318	0.00401295197996263\\
319	0.00401472414546422\\
320	0.00401652698584636\\
321	0.00401836100592681\\
322	0.00402022671775282\\
323	0.00402212464066014\\
324	0.00402405530133158\\
325	0.00402601923385591\\
326	0.00402801697978755\\
327	0.00403004908820754\\
328	0.00403211611578721\\
329	0.00403421862685557\\
330	0.00403635719347254\\
331	0.00403853239550991\\
332	0.00404074482074375\\
333	0.00404299506496178\\
334	0.00404528373209185\\
335	0.00404761143435832\\
336	0.00404997879247634\\
337	0.00405238643589678\\
338	0.00405483500311943\\
339	0.00405732514209878\\
340	0.00405985751078104\\
341	0.00406243277784526\\
342	0.00406505162382405\\
343	0.00406771474309292\\
344	0.00407042284814956\\
345	0.0040731766793176\\
346	0.00407597701430165\\
347	0.00407882465138177\\
348	0.00408172041067304\\
349	0.00408466513531158\\
350	0.00408765969244745\\
351	0.00409070497383421\\
352	0.00409380189560034\\
353	0.00409695139626494\\
354	0.00410015443066837\\
355	0.00410341195380954\\
356	0.00410672487967\\
357	0.00411009400125772\\
358	0.00411352010226292\\
359	0.00411700396460171\\
360	0.00412054636708436\\
361	0.00412414808393454\\
362	0.004127809883143\\
363	0.00413153252463749\\
364	0.0041353167582489\\
365	0.00413916332145065\\
366	0.0041430729368464\\
367	0.00414704630937751\\
368	0.0041510841232191\\
369	0.00415518703832901\\
370	0.00415935568661078\\
371	0.00416359066764624\\
372	0.00416789254394891\\
373	0.00417226183568325\\
374	0.00417669901478839\\
375	0.00418120449843865\\
376	0.00418577864176438\\
377	0.00419042172974946\\
378	0.00419513396821128\\
379	0.00419991547376057\\
380	0.00420476626262653\\
381	0.00420968623822302\\
382	0.00421467517731951\\
383	0.00421973271466958\\
384	0.00422485832593976\\
385	0.0042300513087723\\
386	0.0042353107618099\\
387	0.00424063556150867\\
388	0.00424602433657202\\
389	0.00425147543985365\\
390	0.00425698691761027\\
391	0.00426255647603806\\
392	0.00426818144511206\\
393	0.00427385873987608\\
394	0.00427958481952026\\
395	0.00428535564485324\\
396	0.00429116663515453\\
397	0.00429701262590292\\
398	0.00430288782952713\\
399	0.00430878580264344\\
400	0.00431469942497536\\
401	0.00432062089684156\\
402	0.00432654176479418\\
403	0.00433245298898998\\
404	0.00433834507115593\\
405	0.00434420826923577\\
406	0.00435003293470858\\
407	0.00435581002220494\\
408	0.00436153184002012\\
409	0.00436719313706548\\
410	0.00437279266132182\\
411	0.00437833538472884\\
412	0.00438383567545592\\
413	0.00438932177878079\\
414	0.00439484173526621\\
415	0.00440043693996867\\
416	0.00440611959258895\\
417	0.00441189048639354\\
418	0.00441775037617439\\
419	0.00442369997286494\\
420	0.00442973993741127\\
421	0.00443587087335271\\
422	0.00444209331871319\\
423	0.00444840774111909\\
424	0.00445481453771689\\
425	0.0044613140414695\\
426	0.00446790651697064\\
427	0.0044745921563337\\
428	0.00448137107521975\\
429	0.0044882433090829\\
430	0.00449520880974954\\
431	0.00450226744247634\\
432	0.00450941898366786\\
433	0.00451666311947818\\
434	0.00452399944557325\\
435	0.00453142746838399\\
436	0.00453894660824531\\
437	0.00454655620490727\\
438	0.00455425552600115\\
439	0.00456204377916066\\
440	0.00456992012862854\\
441	0.00457788371733072\\
442	0.00458593369556622\\
443	0.00459406925764097\\
444	0.00460228968796137\\
445	0.00461059441828207\\
446	0.00461898309795326\\
447	0.00462745567909243\\
448	0.00463601251855382\\
449	0.00464465449905457\\
450	0.00465338317019809\\
451	0.00466220090898406\\
452	0.00467111109790705\\
453	0.00468011831560007\\
454	0.00468922852994733\\
455	0.00469844927614577\\
456	0.00470778979203129\\
457	0.0047172610686496\\
458	0.00472687575038563\\
459	0.00473664778038361\\
460	0.00474659161162588\\
461	0.00475672065104738\\
462	0.00476704443458528\\
463	0.00477756856208024\\
464	0.00478829864180216\\
465	0.00479924063984959\\
466	0.00481040087052541\\
467	0.00482178602364282\\
468	0.00483340319515331\\
469	0.00484525992006689\\
470	0.00485736420771812\\
471	0.004869724579392\\
472	0.00488235010823822\\
473	0.00489525046131628\\
474	0.00490843594349121\\
475	0.00492191754271023\\
476	0.00493570697592564\\
477	0.00494981673455072\\
478	0.00496426012906006\\
479	0.00497905133959719\\
480	0.00499420641614639\\
481	0.00500974301389922\\
482	0.00502567987587271\\
483	0.00504203681804972\\
484	0.00505883469393796\\
485	0.00507609533954526\\
486	0.00509384150344459\\
487	0.0051120967723867\\
488	0.00513088551504203\\
489	0.00515023291574502\\
490	0.00517016521560303\\
491	0.00519070995746875\\
492	0.00521189574655456\\
493	0.00523370568230926\\
494	0.00525615797345135\\
495	0.00527928085048286\\
496	0.00530310526724121\\
497	0.00532766854668698\\
498	0.00535303735342734\\
499	0.00537925241866236\\
500	0.0054063556515977\\
501	0.00543439127309023\\
502	0.00546340557950195\\
503	0.00549344603574776\\
504	0.00552456889504518\\
505	0.00555688280214329\\
506	0.00559044812484436\\
507	0.00562532780659876\\
508	0.00566158726887005\\
509	0.00569929423683336\\
510	0.00573851846062438\\
511	0.00577933131156622\\
512	0.00582180523855337\\
513	0.0058660130791357\\
514	0.00591202733005102\\
515	0.00595991936941512\\
516	0.00600975518857396\\
517	0.00606159220849871\\
518	0.00611547879735186\\
519	0.00617145111727593\\
520	0.00622952827222556\\
521	0.00628970070576879\\
522	0.00635192430572301\\
523	0.00641612600430055\\
524	0.00648218785696528\\
525	0.00654993917002101\\
526	0.0066191621183003\\
527	0.00668963241587316\\
528	0.00676094353650896\\
529	0.00683211407028415\\
530	0.00690211112710542\\
531	0.00696671445942419\\
532	0.00702546965301922\\
533	0.00707812424877317\\
534	0.00712481587446659\\
535	0.00716849059922027\\
536	0.00721077696089307\\
537	0.00725210040069124\\
538	0.00729298041879735\\
539	0.00733394341899069\\
540	0.00737527202531301\\
541	0.00741711243910963\\
542	0.00745958050016931\\
543	0.00750275904121561\\
544	0.00754670141587454\\
545	0.00759144767135124\\
546	0.00763703054234583\\
547	0.00768347622976199\\
548	0.00773080823398957\\
549	0.00777909452938988\\
550	0.00782841534682607\\
551	0.00787862213957222\\
552	0.00792795218358609\\
553	0.00797616861463886\\
554	0.00802315876012582\\
555	0.00807047849253736\\
556	0.00811816536939713\\
557	0.0081662917589018\\
558	0.00821491588109375\\
559	0.00826402041714991\\
560	0.00831357679383578\\
561	0.00836354672564337\\
562	0.00841388288403202\\
563	0.00846455795573843\\
564	0.00851554903587025\\
565	0.00856510622084549\\
566	0.00861296704327435\\
567	0.00866079323060213\\
568	0.0087086623118578\\
569	0.00875662026486645\\
570	0.00880463250812337\\
571	0.0088526615516325\\
572	0.00890067654903162\\
573	0.00894813872224893\\
574	0.00899535238494104\\
575	0.00904255767963824\\
576	0.00908973840022921\\
577	0.00913684942824384\\
578	0.00918384194182773\\
579	0.00923066499370698\\
580	0.00927726597097645\\
581	0.00932359115751704\\
582	0.00936958649189013\\
583	0.00941519857373444\\
584	0.00946037597937255\\
585	0.00950507095774694\\
586	0.00954924158549703\\
587	0.00959285445497603\\
588	0.00963588792409171\\
589	0.00967833579817709\\
590	0.00972021084271539\\
591	0.00976138933978722\\
592	0.00980171928775089\\
593	0.00984102135019415\\
594	0.00987905595166784\\
595	0.00991543381058343\\
596	0.00994937493726701\\
597	0.00997906286423442\\
598	0.0099999191923403\\
599	0\\
600	0\\
};
\addplot [color=red,solid,forget plot]
  table[row sep=crcr]{%
1	0.00417141729594836\\
2	0.0041714212375612\\
3	0.00417142525044982\\
4	0.00417142933590415\\
5	0.00417143349523751\\
6	0.00417143772978697\\
7	0.00417144204091389\\
8	0.00417144643000426\\
9	0.00417145089846921\\
10	0.00417145544774538\\
11	0.00417146007929553\\
12	0.00417146479460882\\
13	0.00417146959520149\\
14	0.00417147448261719\\
15	0.00417147945842758\\
16	0.00417148452423281\\
17	0.004171489681662\\
18	0.00417149493237377\\
19	0.00417150027805687\\
20	0.0041715057204306\\
21	0.00417151126124547\\
22	0.00417151690228366\\
23	0.00417152264535971\\
24	0.00417152849232102\\
25	0.00417153444504848\\
26	0.00417154050545709\\
27	0.00417154667549652\\
28	0.0041715529571518\\
29	0.00417155935244397\\
30	0.00417156586343074\\
31	0.00417157249220702\\
32	0.00417157924090577\\
33	0.00417158611169864\\
34	0.00417159310679659\\
35	0.0041716002284507\\
36	0.0041716074789529\\
37	0.00417161486063662\\
38	0.00417162237587761\\
39	0.00417163002709472\\
40	0.00417163781675066\\
41	0.00417164574735278\\
42	0.00417165382145392\\
43	0.00417166204165321\\
44	0.00417167041059691\\
45	0.0041716789309793\\
46	0.00417168760554341\\
47	0.00417169643708219\\
48	0.00417170542843912\\
49	0.00417171458250933\\
50	0.00417172390224041\\
51	0.00417173339063343\\
52	0.00417174305074389\\
53	0.00417175288568275\\
54	0.00417176289861736\\
55	0.00417177309277255\\
56	0.00417178347143162\\
57	0.00417179403793749\\
58	0.00417180479569368\\
59	0.00417181574816551\\
60	0.00417182689888108\\
61	0.0041718382514326\\
62	0.00417184980947735\\
63	0.00417186157673903\\
64	0.00417187355700894\\
65	0.00417188575414711\\
66	0.00417189817208365\\
67	0.00417191081481996\\
68	0.00417192368643009\\
69	0.00417193679106193\\
70	0.00417195013293873\\
71	0.00417196371636035\\
72	0.0041719775457047\\
73	0.00417199162542913\\
74	0.00417200596007185\\
75	0.00417202055425349\\
76	0.00417203541267852\\
77	0.0041720505401368\\
78	0.00417206594150512\\
79	0.00417208162174879\\
80	0.00417209758592323\\
81	0.00417211383917564\\
82	0.00417213038674667\\
83	0.00417214723397208\\
84	0.00417216438628448\\
85	0.00417218184921508\\
86	0.00417219962839556\\
87	0.00417221772955976\\
88	0.00417223615854567\\
89	0.00417225492129723\\
90	0.00417227402386632\\
91	0.00417229347241466\\
92	0.00417231327321582\\
93	0.00417233343265732\\
94	0.00417235395724267\\
95	0.00417237485359332\\
96	0.00417239612845109\\
97	0.00417241778868015\\
98	0.00417243984126929\\
99	0.00417246229333417\\
100	0.00417248515211969\\
101	0.00417250842500232\\
102	0.00417253211949238\\
103	0.00417255624323666\\
104	0.00417258080402071\\
105	0.00417260580977149\\
106	0.0041726312685599\\
107	0.00417265718860336\\
108	0.00417268357826855\\
109	0.00417271044607403\\
110	0.00417273780069307\\
111	0.00417276565095637\\
112	0.00417279400585506\\
113	0.00417282287454349\\
114	0.00417285226634222\\
115	0.00417288219074112\\
116	0.00417291265740226\\
117	0.00417294367616329\\
118	0.00417297525704035\\
119	0.00417300741023158\\
120	0.00417304014612021\\
121	0.004173073475278\\
122	0.00417310740846868\\
123	0.00417314195665143\\
124	0.00417317713098439\\
125	0.00417321294282827\\
126	0.00417324940375013\\
127	0.00417328652552689\\
128	0.0041733243201494\\
129	0.00417336279982615\\
130	0.0041734019769873\\
131	0.00417344186428863\\
132	0.00417348247461566\\
133	0.00417352382108783\\
134	0.00417356591706271\\
135	0.00417360877614035\\
136	0.00417365241216765\\
137	0.00417369683924282\\
138	0.00417374207171994\\
139	0.00417378812421355\\
140	0.00417383501160351\\
141	0.00417388274903964\\
142	0.00417393135194666\\
143	0.00417398083602919\\
144	0.00417403121727675\\
145	0.00417408251196902\\
146	0.00417413473668098\\
147	0.00417418790828828\\
148	0.00417424204397271\\
149	0.00417429716122767\\
150	0.00417435327786387\\
151	0.00417441041201495\\
152	0.00417446858214343\\
153	0.0041745278070466\\
154	0.00417458810586256\\
155	0.00417464949807632\\
156	0.00417471200352611\\
157	0.00417477564240979\\
158	0.00417484043529119\\
159	0.00417490640310686\\
160	0.00417497356717269\\
161	0.00417504194919076\\
162	0.00417511157125627\\
163	0.00417518245586462\\
164	0.00417525462591869\\
165	0.00417532810473602\\
166	0.00417540291605636\\
167	0.00417547908404922\\
168	0.0041755566333216\\
169	0.00417563558892583\\
170	0.00417571597636753\\
171	0.00417579782161376\\
172	0.00417588115110138\\
173	0.00417596599174526\\
174	0.00417605237094702\\
175	0.00417614031660364\\
176	0.00417622985711639\\
177	0.00417632102139971\\
178	0.00417641383889057\\
179	0.00417650833955761\\
180	0.00417660455391075\\
181	0.00417670251301075\\
182	0.00417680224847913\\
183	0.004176903792508\\
184	0.00417700717787039\\
185	0.00417711243793043\\
186	0.00417721960665394\\
187	0.00417732871861905\\
188	0.00417743980902709\\
189	0.00417755291371368\\
190	0.00417766806915986\\
191	0.0041777853125036\\
192	0.00417790468155139\\
193	0.00417802621479001\\
194	0.00417814995139857\\
195	0.00417827593126074\\
196	0.00417840419497708\\
197	0.00417853478387773\\
198	0.00417866774003521\\
199	0.00417880310627744\\
200	0.00417894092620098\\
201	0.00417908124418459\\
202	0.00417922410540279\\
203	0.00417936955583983\\
204	0.00417951764230381\\
205	0.00417966841244109\\
206	0.00417982191475086\\
207	0.00417997819859995\\
208	0.00418013731423793\\
209	0.00418029931281241\\
210	0.00418046424638457\\
211	0.00418063216794502\\
212	0.00418080313142981\\
213	0.00418097719173671\\
214	0.00418115440474184\\
215	0.00418133482731642\\
216	0.00418151851734385\\
217	0.00418170553373708\\
218	0.00418189593645622\\
219	0.00418208978652635\\
220	0.00418228714605576\\
221	0.0041824880782543\\
222	0.0041826926474521\\
223	0.00418290091911855\\
224	0.00418311295988156\\
225	0.00418332883754708\\
226	0.00418354862111898\\
227	0.00418377238081907\\
228	0.00418400018810767\\
229	0.00418423211570418\\
230	0.00418446823760813\\
231	0.00418470862912047\\
232	0.00418495336686518\\
233	0.0041852025288112\\
234	0.00418545619429455\\
235	0.00418571444404094\\
236	0.00418597736018854\\
237	0.00418624502631107\\
238	0.00418651752744131\\
239	0.00418679495009489\\
240	0.00418707738229421\\
241	0.0041873649135929\\
242	0.00418765763510055\\
243	0.00418795563950764\\
244	0.00418825902111096\\
245	0.00418856787583913\\
246	0.00418888230127873\\
247	0.00418920239670044\\
248	0.0041895282630857\\
249	0.0041898600031536\\
250	0.00419019772138816\\
251	0.00419054152406586\\
252	0.0041908915192835\\
253	0.00419124781698645\\
254	0.00419161052899709\\
255	0.00419197976904371\\
256	0.00419235565278956\\
257	0.00419273829786239\\
258	0.00419312782388419\\
259	0.00419352435250128\\
260	0.00419392800741462\\
261	0.00419433891441072\\
262	0.00419475720139237\\
263	0.00419518299841011\\
264	0.00419561643769389\\
265	0.00419605765368487\\
266	0.00419650678306768\\
267	0.00419696396480297\\
268	0.00419742934016015\\
269	0.00419790305275047\\
270	0.0041983852485605\\
271	0.00419887607598572\\
272	0.00419937568586451\\
273	0.00419988423151231\\
274	0.00420040186875631\\
275	0.00420092875596998\\
276	0.00420146505410838\\
277	0.00420201092674335\\
278	0.00420256654009912\\
279	0.00420313206308821\\
280	0.00420370766734756\\
281	0.00420429352727493\\
282	0.00420488982006546\\
283	0.00420549672574871\\
284	0.00420611442722573\\
285	0.00420674311030644\\
286	0.00420738296374742\\
287	0.00420803417928973\\
288	0.00420869695169706\\
289	0.00420937147879415\\
290	0.00421005796150549\\
291	0.00421075660389411\\
292	0.00421146761320083\\
293	0.00421219119988342\\
294	0.00421292757765646\\
295	0.00421367696353093\\
296	0.00421443957785442\\
297	0.00421521564435124\\
298	0.00421600539016304\\
299	0.00421680904588942\\
300	0.00421762684562882\\
301	0.00421845902701961\\
302	0.00421930583128124\\
303	0.00422016750325577\\
304	0.0042210442914492\\
305	0.00422193644807332\\
306	0.00422284422908731\\
307	0.00422376789423958\\
308	0.00422470770710954\\
309	0.00422566393514946\\
310	0.00422663684972616\\
311	0.00422762672616268\\
312	0.00422863384377971\\
313	0.00422965848593693\\
314	0.0042307009400738\\
315	0.00423176149775028\\
316	0.00423284045468683\\
317	0.00423393811080408\\
318	0.00423505477026159\\
319	0.00423619074149609\\
320	0.00423734633725874\\
321	0.0042385218746514\\
322	0.00423971767516177\\
323	0.0042409340646973\\
324	0.00424217137361771\\
325	0.00424342993676598\\
326	0.00424471009349763\\
327	0.00424601218770806\\
328	0.00424733656785803\\
329	0.00424868358699681\\
330	0.00425005360278272\\
331	0.0042514469775011\\
332	0.00425286407807892\\
333	0.00425430527609568\\
334	0.00425577094779001\\
335	0.00425726147406045\\
336	0.0042587772404593\\
337	0.00426031863717721\\
338	0.00426188605901504\\
339	0.0042634799053389\\
340	0.00426510058001185\\
341	0.00426674849129432\\
342	0.00426842405170062\\
343	0.00427012767778342\\
344	0.00427185978975912\\
345	0.00427362081076799\\
346	0.00427541116576958\\
347	0.00427723128085833\\
348	0.00427908158246891\\
349	0.00428096249646347\\
350	0.0042828744470972\\
351	0.00428481785586465\\
352	0.00428679314023817\\
353	0.00428880071232879\\
354	0.00429084097756436\\
355	0.00429291433371471\\
356	0.0042950211712544\\
357	0.00429716187620316\\
358	0.00429933682967659\\
359	0.00430154640713625\\
360	0.00430379097758998\\
361	0.00430607090273958\\
362	0.00430838653607358\\
363	0.00431073822190244\\
364	0.00431312629433371\\
365	0.00431555107618505\\
366	0.00431801287783277\\
367	0.0043205119959941\\
368	0.00432304871244172\\
369	0.00432562329264953\\
370	0.00432823598436941\\
371	0.00433088701613951\\
372	0.00433357659572557\\
373	0.00433630490849874\\
374	0.00433907211575428\\
375	0.00434187835297895\\
376	0.0043447237280762\\
377	0.00434760831956328\\
378	0.0043505321747576\\
379	0.00435349530797551\\
380	0.00435649769877283\\
381	0.00435953929026478\\
382	0.00436261998757205\\
383	0.00436573965645217\\
384	0.00436889812218877\\
385	0.00437209516882886\\
386	0.00437533053887853\\
387	0.00437860393359149\\
388	0.00438191501401419\\
389	0.00438526340298479\\
390	0.0043886486883227\\
391	0.00439207042749119\\
392	0.00439552815406753\\
393	0.0043990213864128\\
394	0.00440254963899731\\
395	0.00440611243690426\\
396	0.00440970933410266\\
397	0.00441333993614392\\
398	0.00441700392799374\\
399	0.00442070110772882\\
400	0.00442443142677545\\
401	0.00442819503726806\\
402	0.00443199234687599\\
403	0.00443582408101143\\
404	0.00443969135159781\\
405	0.00444359573040631\\
406	0.00444753932316088\\
407	0.0044515248379043\\
408	0.00445555563713975\\
409	0.00445963575746238\\
410	0.0044637698717059\\
411	0.00446796315433237\\
412	0.00447222098358322\\
413	0.00447654835910707\\
414	0.00448094884530292\\
415	0.00448542420369106\\
416	0.00448997560015105\\
417	0.00449460422115946\\
418	0.00449931127504153\\
419	0.00450409799335963\\
420	0.00450896563245742\\
421	0.00451391547521329\\
422	0.00451894883305504\\
423	0.00452406704818085\\
424	0.00452927149581646\\
425	0.00453456358633038\\
426	0.00453994476757048\\
427	0.00454541652748043\\
428	0.00455098039742705\\
429	0.0045566379564885\\
430	0.00456239083633436\\
431	0.00456824072677151\\
432	0.00457418938203479\\
433	0.00458023862790652\\
434	0.00458639036975255\\
435	0.00459264660156371\\
436	0.00459900941609112\\
437	0.00460548101615734\\
438	0.00461206372721509\\
439	0.00461876001120466\\
440	0.00462557248173187\\
441	0.00463250392054477\\
442	0.00463955729522559\\
443	0.00464673577793208\\
444	0.00465404276491351\\
445	0.00466148189639597\\
446	0.0046690570762898\\
447	0.00467677249104362\\
448	0.00468463262666304\\
449	0.00469264226005057\\
450	0.00470080645466231\\
451	0.00470913056139528\\
452	0.00471762021278094\\
453	0.00472628130882881\\
454	0.00473511999293577\\
455	0.00474414261666887\\
456	0.00475335569311759\\
457	0.00476276583994915\\
458	0.00477237971546353\\
459	0.00478220395394506\\
460	0.00479224511277774\\
461	0.00480250966573844\\
462	0.0048130041410784\\
463	0.00482373531637173\\
464	0.00483471025564672\\
465	0.00484593634938176\\
466	0.00485742262810758\\
467	0.00486917868065496\\
468	0.0048812146512215\\
469	0.0048935412725203\\
470	0.00490616989892637\\
471	0.00491911253884136\\
472	0.00493238188599012\\
473	0.00494599134833053\\
474	0.00495995507289613\\
475	0.00497428796445237\\
476	0.00498900569361151\\
477	0.00500412468545673\\
478	0.00501966204991506\\
479	0.00503563534260202\\
480	0.00505203203850175\\
481	0.00506885035320858\\
482	0.00508610647732088\\
483	0.00510381737352443\\
484	0.00512200081946964\\
485	0.00514067545717838\\
486	0.00515986085050313\\
487	0.0051795775543532\\
488	0.0051998472524993\\
489	0.00522069291948455\\
490	0.00524213893576288\\
491	0.00526421036697117\\
492	0.0052869337710113\\
493	0.00531038708338049\\
494	0.00533463065977775\\
495	0.00535972170666477\\
496	0.00538570357433848\\
497	0.00541262217930969\\
498	0.00544052593988315\\
499	0.00546946488375426\\
500	0.0054994915755224\\
501	0.0055306611806234\\
502	0.00556303151660603\\
503	0.00559666327700001\\
504	0.00563162031738346\\
505	0.00566796721508761\\
506	0.00570576840772046\\
507	0.00574508942737241\\
508	0.00578599628828108\\
509	0.00582855467622107\\
510	0.00587282886678455\\
511	0.00591888013540978\\
512	0.00596676496370957\\
513	0.0060165327491921\\
514	0.00606822295732727\\
515	0.00612186193541779\\
516	0.00617745923946957\\
517	0.00623501086570649\\
518	0.0062944782534695\\
519	0.00635577778146541\\
520	0.0064187788043303\\
521	0.00648329513000825\\
522	0.00654909551745368\\
523	0.00661556809570386\\
524	0.00668225778840485\\
525	0.00674845423256209\\
526	0.00681265527495289\\
527	0.00687114274886836\\
528	0.00692366259412663\\
529	0.00697041755680642\\
530	0.00701168306378662\\
531	0.00705128720921838\\
532	0.00708973073153022\\
533	0.00712748667855347\\
534	0.00716510576662421\\
535	0.00720298425452938\\
536	0.00724131062521522\\
537	0.00728020567755491\\
538	0.0073197621304377\\
539	0.00736003760201659\\
540	0.00740107026007002\\
541	0.00744289189206673\\
542	0.00748552922793119\\
543	0.00752900542539635\\
544	0.00757334165858363\\
545	0.00761855794325576\\
546	0.0076647133488703\\
547	0.0077118861491722\\
548	0.00776011644032672\\
549	0.00780747688911366\\
550	0.00785373177371214\\
551	0.00789882269053892\\
552	0.00794427584846579\\
553	0.00799013221423805\\
554	0.00803647165748248\\
555	0.00808335531207435\\
556	0.00813077466079379\\
557	0.0081787090079221\\
558	0.00822712878473145\\
559	0.00827599893382302\\
560	0.00832527752852308\\
561	0.00837493010777871\\
562	0.00842494111645729\\
563	0.00847397024240702\\
564	0.00852129441202751\\
565	0.00856849529392683\\
566	0.00861580505555088\\
567	0.00866327323530527\\
568	0.00871086822438768\\
569	0.00875855220172137\\
570	0.0088062920664252\\
571	0.00885392879390287\\
572	0.00890103813867015\\
573	0.00894817183198183\\
574	0.00899535443329664\\
575	0.00904255798659048\\
576	0.00908973853069441\\
577	0.00913684949557439\\
578	0.0091838419761039\\
579	0.009230665010298\\
580	0.00927726597847558\\
581	0.00932359116063004\\
582	0.00936958649305349\\
583	0.00941519857411533\\
584	0.00946037597947751\\
585	0.00950507095776978\\
586	0.00954924158550052\\
587	0.00959285445497631\\
588	0.00963588792409171\\
589	0.00967833579817709\\
590	0.00972021084271539\\
591	0.00976138933978722\\
592	0.00980171928775089\\
593	0.00984102135019415\\
594	0.00987905595166784\\
595	0.00991543381058343\\
596	0.00994937493726701\\
597	0.00997906286423442\\
598	0.0099999191923403\\
599	0\\
600	0\\
};
\addplot [color=mycolor20,solid,forget plot]
  table[row sep=crcr]{%
1	0.00426663331528449\\
2	0.00426663605446198\\
3	0.00426663884350637\\
4	0.00426664168332735\\
5	0.00426664457485123\\
6	0.0042666475190213\\
7	0.00426665051679803\\
8	0.00426665356915952\\
9	0.00426665667710172\\
10	0.00426665984163881\\
11	0.00426666306380356\\
12	0.00426666634464759\\
13	0.00426666968524176\\
14	0.00426667308667659\\
15	0.00426667655006247\\
16	0.00426668007653015\\
17	0.00426668366723108\\
18	0.0042666873233378\\
19	0.00426669104604424\\
20	0.00426669483656624\\
21	0.00426669869614191\\
22	0.004266702626032\\
23	0.00426670662752032\\
24	0.0042667107019142\\
25	0.00426671485054495\\
26	0.00426671907476817\\
27	0.00426672337596439\\
28	0.00426672775553932\\
29	0.00426673221492451\\
30	0.00426673675557764\\
31	0.00426674137898315\\
32	0.00426674608665263\\
33	0.00426675088012538\\
34	0.00426675576096894\\
35	0.0042667607307795\\
36	0.00426676579118251\\
37	0.00426677094383328\\
38	0.00426677619041735\\
39	0.00426678153265121\\
40	0.00426678697228281\\
41	0.00426679251109208\\
42	0.00426679815089161\\
43	0.00426680389352723\\
44	0.0042668097408786\\
45	0.00426681569485981\\
46	0.00426682175742013\\
47	0.00426682793054448\\
48	0.00426683421625422\\
49	0.00426684061660772\\
50	0.00426684713370119\\
51	0.00426685376966924\\
52	0.00426686052668561\\
53	0.00426686740696394\\
54	0.00426687441275847\\
55	0.0042668815463648\\
56	0.00426688881012064\\
57	0.00426689620640662\\
58	0.004266903737647\\
59	0.00426691140631057\\
60	0.0042669192149114\\
61	0.00426692716600971\\
62	0.00426693526221271\\
63	0.00426694350617553\\
64	0.00426695190060191\\
65	0.00426696044824531\\
66	0.00426696915190976\\
67	0.00426697801445072\\
68	0.00426698703877609\\
69	0.00426699622784719\\
70	0.00426700558467973\\
71	0.00426701511234478\\
72	0.00426702481396984\\
73	0.00426703469273984\\
74	0.00426704475189825\\
75	0.00426705499474805\\
76	0.00426706542465298\\
77	0.00426707604503858\\
78	0.0042670868593933\\
79	0.0042670978712697\\
80	0.00426710908428565\\
81	0.00426712050212553\\
82	0.00426713212854137\\
83	0.00426714396735423\\
84	0.00426715602245535\\
85	0.00426716829780761\\
86	0.00426718079744667\\
87	0.00426719352548243\\
88	0.00426720648610035\\
89	0.00426721968356286\\
90	0.00426723312221078\\
91	0.00426724680646477\\
92	0.00426726074082681\\
93	0.00426727492988168\\
94	0.0042672893782985\\
95	0.00426730409083235\\
96	0.0042673190723257\\
97	0.00426733432771018\\
98	0.0042673498620082\\
99	0.00426736568033454\\
100	0.00426738178789813\\
101	0.00426739819000379\\
102	0.00426741489205398\\
103	0.00426743189955058\\
104	0.0042674492180968\\
105	0.00426746685339901\\
106	0.00426748481126861\\
107	0.00426750309762402\\
108	0.00426752171849266\\
109	0.00426754068001293\\
110	0.00426755998843633\\
111	0.00426757965012944\\
112	0.00426759967157611\\
113	0.00426762005937972\\
114	0.00426764082026519\\
115	0.00426766196108137\\
116	0.00426768348880335\\
117	0.00426770541053467\\
118	0.00426772773350981\\
119	0.00426775046509656\\
120	0.00426777361279847\\
121	0.0042677971842574\\
122	0.00426782118725611\\
123	0.00426784562972072\\
124	0.00426787051972356\\
125	0.00426789586548568\\
126	0.00426792167537974\\
127	0.00426794795793283\\
128	0.00426797472182915\\
129	0.00426800197591313\\
130	0.00426802972919219\\
131	0.00426805799083993\\
132	0.00426808677019914\\
133	0.00426811607678491\\
134	0.00426814592028779\\
135	0.00426817631057712\\
136	0.00426820725770426\\
137	0.00426823877190602\\
138	0.004268270863608\\
139	0.00426830354342821\\
140	0.0042683368221805\\
141	0.00426837071087827\\
142	0.00426840522073809\\
143	0.00426844036318351\\
144	0.00426847614984887\\
145	0.00426851259258323\\
146	0.00426854970345422\\
147	0.00426858749475229\\
148	0.00426862597899463\\
149	0.00426866516892941\\
150	0.00426870507754012\\
151	0.00426874571804986\\
152	0.00426878710392581\\
153	0.00426882924888365\\
154	0.00426887216689225\\
155	0.00426891587217831\\
156	0.00426896037923111\\
157	0.00426900570280735\\
158	0.00426905185793613\\
159	0.00426909885992393\\
160	0.00426914672435976\\
161	0.00426919546712036\\
162	0.00426924510437553\\
163	0.00426929565259355\\
164	0.00426934712854658\\
165	0.0042693995493164\\
166	0.00426945293230002\\
167	0.00426950729521554\\
168	0.00426956265610805\\
169	0.00426961903335571\\
170	0.00426967644567576\\
171	0.00426973491213084\\
172	0.00426979445213538\\
173	0.00426985508546202\\
174	0.00426991683224824\\
175	0.00426997971300302\\
176	0.00427004374861366\\
177	0.00427010896035288\\
178	0.00427017536988565\\
179	0.00427024299927665\\
180	0.00427031187099741\\
181	0.00427038200793391\\
182	0.00427045343339406\\
183	0.00427052617111558\\
184	0.00427060024527374\\
185	0.00427067568048948\\
186	0.00427075250183746\\
187	0.00427083073485445\\
188	0.0042709104055478\\
189	0.00427099154040393\\
190	0.00427107416639718\\
191	0.00427115831099867\\
192	0.00427124400218538\\
193	0.00427133126844935\\
194	0.00427142013880714\\
195	0.00427151064280929\\
196	0.00427160281055008\\
197	0.0042716966726774\\
198	0.00427179226040283\\
199	0.00427188960551184\\
200	0.00427198874037426\\
201	0.00427208969795479\\
202	0.00427219251182382\\
203	0.00427229721616838\\
204	0.00427240384580332\\
205	0.00427251243618258\\
206	0.00427262302341075\\
207	0.0042727356442548\\
208	0.004272850336156\\
209	0.00427296713724205\\
210	0.00427308608633941\\
211	0.00427320722298579\\
212	0.00427333058744295\\
213	0.00427345622070965\\
214	0.00427358416453478\\
215	0.00427371446143077\\
216	0.00427384715468722\\
217	0.00427398228838469\\
218	0.0042741199074088\\
219	0.00427426005746445\\
220	0.0042744027850904\\
221	0.00427454813767394\\
222	0.00427469616346598\\
223	0.00427484691159612\\
224	0.00427500043208823\\
225	0.00427515677587616\\
226	0.00427531599481954\\
227	0.00427547814172013\\
228	0.00427564327033817\\
229	0.00427581143540913\\
230	0.00427598269266059\\
231	0.00427615709882952\\
232	0.00427633471167965\\
233	0.00427651559001924\\
234	0.00427669979371907\\
235	0.00427688738373068\\
236	0.00427707842210479\\
237	0.00427727297201019\\
238	0.00427747109775273\\
239	0.00427767286479459\\
240	0.00427787833977384\\
241	0.00427808759052437\\
242	0.00427830068609593\\
243	0.00427851769677457\\
244	0.00427873869410317\\
245	0.00427896375090256\\
246	0.00427919294129249\\
247	0.00427942634071331\\
248	0.00427966402594756\\
249	0.00427990607514204\\
250	0.00428015256783011\\
251	0.00428040358495417\\
252	0.00428065920888856\\
253	0.00428091952346254\\
254	0.00428118461398376\\
255	0.00428145456726173\\
256	0.00428172947163182\\
257	0.00428200941697935\\
258	0.00428229449476391\\
259	0.00428258479804408\\
260	0.00428288042150233\\
261	0.00428318146147007\\
262	0.00428348801595315\\
263	0.00428380018465741\\
264	0.00428411806901457\\
265	0.00428444177220832\\
266	0.00428477139920069\\
267	0.00428510705675852\\
268	0.00428544885348034\\
269	0.00428579689982334\\
270	0.00428615130813052\\
271	0.00428651219265822\\
272	0.00428687966960365\\
273	0.00428725385713282\\
274	0.00428763487540848\\
275	0.00428802284661846\\
276	0.00428841789500398\\
277	0.00428882014688836\\
278	0.00428922973070582\\
279	0.00428964677703041\\
280	0.00429007141860525\\
281	0.00429050379037187\\
282	0.00429094402949989\\
283	0.00429139227541663\\
284	0.00429184866983709\\
285	0.00429231335679421\\
286	0.0042927864826692\\
287	0.00429326819622204\\
288	0.00429375864862242\\
289	0.00429425799348073\\
290	0.00429476638687946\\
291	0.00429528398740469\\
292	0.00429581095617808\\
293	0.00429634745688905\\
294	0.00429689365582736\\
295	0.00429744972191603\\
296	0.00429801582674476\\
297	0.00429859214460362\\
298	0.00429917885251751\\
299	0.00429977613028076\\
300	0.00430038416049273\\
301	0.00430100312859371\\
302	0.00430163322290172\\
303	0.00430227463464979\\
304	0.00430292755802435\\
305	0.00430359219020419\\
306	0.00430426873140046\\
307	0.0043049573848976\\
308	0.00430565835709522\\
309	0.00430637185755113\\
310	0.00430709809902548\\
311	0.00430783729752581\\
312	0.0043085896723536\\
313	0.00430935544615174\\
314	0.00431013484495313\\
315	0.00431092809823047\\
316	0.00431173543894718\\
317	0.00431255710360897\\
318	0.00431339333231663\\
319	0.00431424436881902\\
320	0.00431511046056672\\
321	0.00431599185876574\\
322	0.00431688881843094\\
323	0.00431780159843884\\
324	0.00431873046157911\\
325	0.00431967567460451\\
326	0.00432063750827823\\
327	0.0043216162374183\\
328	0.00432261214093769\\
329	0.00432362550187977\\
330	0.00432465660744775\\
331	0.00432570574902713\\
332	0.00432677322220024\\
333	0.00432785932675161\\
334	0.00432896436666306\\
335	0.00433008865009771\\
336	0.00433123248937182\\
337	0.00433239620091378\\
338	0.0043335801052101\\
339	0.00433478452673791\\
340	0.00433600979388494\\
341	0.00433725623885781\\
342	0.00433852419758035\\
343	0.00433981400958369\\
344	0.00434112601789247\\
345	0.00434246056892186\\
346	0.00434381801239764\\
347	0.00434519870128648\\
348	0.00434660299173398\\
349	0.00434803124301478\\
350	0.00434948381749918\\
351	0.00435096108064044\\
352	0.00435246340098642\\
353	0.00435399115021984\\
354	0.00435554470323253\\
355	0.00435712443823616\\
356	0.00435873073685524\\
357	0.00436036398412274\\
358	0.00436202456847492\\
359	0.00436371288175468\\
360	0.00436542931922488\\
361	0.00436717427959442\\
362	0.00436894816505918\\
363	0.00437075138136093\\
364	0.00437258433786724\\
365	0.00437444744767627\\
366	0.00437634112775016\\
367	0.00437826579908203\\
368	0.00438022188690141\\
369	0.00438220982092401\\
370	0.00438423003565191\\
371	0.00438628297073149\\
372	0.00438836907137671\\
373	0.00439048878886601\\
374	0.00439264258112255\\
375	0.00439483091338743\\
376	0.0043970542589972\\
377	0.00439931310027703\\
378	0.00440160792956237\\
379	0.00440393925036222\\
380	0.00440630757867798\\
381	0.00440871344449215\\
382	0.00441115739344192\\
383	0.00441363998869198\\
384	0.00441616181302129\\
385	0.00441872347113729\\
386	0.00442132559222974\\
387	0.00442396883277432\\
388	0.00442665387959276\\
389	0.00442938145317225\\
390	0.00443215231124075\\
391	0.00443496725258767\\
392	0.00443782712110927\\
393	0.0044407328100463\\
394	0.00444368526636549\\
395	0.00444668549521858\\
396	0.00444973456438823\\
397	0.00445283360860144\\
398	0.00445598383354885\\
399	0.00445918651955122\\
400	0.00446244302487293\\
401	0.0044657547879608\\
402	0.00446912332816642\\
403	0.00447255024442683\\
404	0.00447603721130479\\
405	0.00447958597174322\\
406	0.00448319832590475\\
407	0.004486876115589\\
408	0.0044906212040159\\
409	0.00449443545129221\\
410	0.00449832068667244\\
411	0.00450227867983526\\
412	0.00450631111553174\\
413	0.00451041958314296\\
414	0.00451460561420282\\
415	0.00451887075326674\\
416	0.00452321658618162\\
417	0.0045276447430362\\
418	0.0045321569014872\\
419	0.00453675479048509\\
420	0.00454144019441523\\
421	0.00454621495765489\\
422	0.00455108098952513\\
423	0.00455604026958924\\
424	0.00456109485322701\\
425	0.00456624687739753\\
426	0.00457149856649096\\
427	0.00457685223810688\\
428	0.00458231029759972\\
429	0.004587875226996\\
430	0.0045935495869901\\
431	0.00459933601879077\\
432	0.0046052372457871\\
433	0.00461125607500289\\
434	0.00461739539831013\\
435	0.00462365819337468\\
436	0.00463004752431581\\
437	0.00463656654207527\\
438	0.00464321848450911\\
439	0.0046500066762436\\
440	0.00465693452837336\\
441	0.00466400553812897\\
442	0.00467122328870246\\
443	0.00467859144948759\\
444	0.00468611377705478\\
445	0.00469379411721445\\
446	0.00470163640858135\\
447	0.00470964468879945\\
448	0.00471782311025046\\
449	0.00472617665871337\\
450	0.00473471089905645\\
451	0.00474343165936944\\
452	0.00475234504327954\\
453	0.00476145744292708\\
454	0.00477077555288365\\
455	0.00478030638543103\\
456	0.0047900572865475\\
457	0.00480003595306472\\
458	0.00481025045074162\\
459	0.00482070923265264\\
460	0.00483142115711013\\
461	0.00484239550330722\\
462	0.00485364197314648\\
463	0.00486517066339799\\
464	0.0048769919772247\\
465	0.00488911634640063\\
466	0.00490151423351223\\
467	0.00491419268996534\\
468	0.00492716026276567\\
469	0.0049404258593231\\
470	0.00495399877491506\\
471	0.00496788872794736\\
472	0.00498210588125886\\
473	0.00499666087416632\\
474	0.00501156486292523\\
475	0.005026829556054\\
476	0.00504246724598849\\
477	0.00505849076745665\\
478	0.00507491360227769\\
479	0.00509174979338502\\
480	0.00510904461151792\\
481	0.00512683513835379\\
482	0.00514514294413699\\
483	0.00516399086837915\\
484	0.00518340310039239\\
485	0.00520340532154351\\
486	0.00522402486784549\\
487	0.00524529086619148\\
488	0.00526723353178775\\
489	0.00528988555479908\\
490	0.00531328551978419\\
491	0.00533749217269366\\
492	0.00536255852004745\\
493	0.00538852951181075\\
494	0.00541545080092527\\
495	0.00544336957331949\\
496	0.00547233467708075\\
497	0.00550239710763532\\
498	0.00553360994472241\\
499	0.00556602827457043\\
500	0.00559970902539875\\
501	0.00563471072458662\\
502	0.00567109315732001\\
503	0.00570891688837819\\
504	0.00574824251891552\\
505	0.00578912967368312\\
506	0.00583163601149896\\
507	0.00587581585992625\\
508	0.0059217185016282\\
509	0.00596938601077104\\
510	0.00601885147197273\\
511	0.00607014279388628\\
512	0.00612327173424099\\
513	0.00617822896877258\\
514	0.00623497651937254\\
515	0.00629343208266863\\
516	0.00635345162141576\\
517	0.00641450018738264\\
518	0.00647630306501683\\
519	0.00653856759307269\\
520	0.00660083177351316\\
521	0.0066623734845899\\
522	0.00672148584669011\\
523	0.00677514287299271\\
524	0.00682314710642925\\
525	0.00686550063474744\\
526	0.00690303785427555\\
527	0.006939183141758\\
528	0.0069743533931855\\
529	0.00700904633276673\\
530	0.00704380336489818\\
531	0.00707891357147736\\
532	0.00711450782547904\\
533	0.00715069050118957\\
534	0.00718753256758812\\
535	0.00722507731585913\\
536	0.00726335660395916\\
537	0.00730239682932776\\
538	0.00734222033833203\\
539	0.00738284789789318\\
540	0.00742430017052845\\
541	0.00746659733277799\\
542	0.00750975936517875\\
543	0.00755383226995456\\
544	0.00759889240011881\\
545	0.00764503248014944\\
546	0.00769063100310164\\
547	0.00773515726105684\\
548	0.00777837421757474\\
549	0.00782197210231666\\
550	0.00786599669861675\\
551	0.00791053055057518\\
552	0.0079556338739196\\
553	0.00800129915646446\\
554	0.00804751442201892\\
555	0.00809426408080562\\
556	0.00814152538695643\\
557	0.00818926789268914\\
558	0.00823745553953724\\
559	0.00828604517572598\\
560	0.00833502635030947\\
561	0.0083836844121862\\
562	0.00843068273194334\\
563	0.00847720045237796\\
564	0.00852387037291765\\
565	0.0085707516393333\\
566	0.00861782349227564\\
567	0.00866505204116376\\
568	0.00871240258683382\\
569	0.00875984726743983\\
570	0.00880711445205011\\
571	0.00885403140777324\\
572	0.00890105211970034\\
573	0.00894817202601561\\
574	0.00899535447720905\\
575	0.00904255800719692\\
576	0.00908973854120521\\
577	0.00913684950076924\\
578	0.00918384197853101\\
579	0.00923066501135355\\
580	0.00927726597889613\\
581	0.00932359116078053\\
582	0.00936958649310055\\
583	0.00941519857412769\\
584	0.00946037597948007\\
585	0.00950507095777015\\
586	0.00954924158550054\\
587	0.00959285445497631\\
588	0.00963588792409171\\
589	0.00967833579817709\\
590	0.00972021084271539\\
591	0.00976138933978722\\
592	0.00980171928775089\\
593	0.00984102135019415\\
594	0.00987905595166784\\
595	0.00991543381058343\\
596	0.00994937493726701\\
597	0.00997906286423442\\
598	0.0099999191923403\\
599	0\\
600	0\\
};
\addplot [color=mycolor21,solid,forget plot]
  table[row sep=crcr]{%
1	0.00430148357374624\\
2	0.00430148576350371\\
3	0.00430148799339185\\
4	0.00430149026414835\\
5	0.00430149257652449\\
6	0.00430149493128541\\
7	0.00430149732921043\\
8	0.00430149977109314\\
9	0.00430150225774187\\
10	0.00430150478997984\\
11	0.00430150736864544\\
12	0.00430150999459258\\
13	0.00430151266869094\\
14	0.00430151539182618\\
15	0.00430151816490036\\
16	0.00430152098883223\\
17	0.00430152386455745\\
18	0.00430152679302899\\
19	0.00430152977521747\\
20	0.00430153281211134\\
21	0.00430153590471736\\
22	0.00430153905406089\\
23	0.00430154226118625\\
24	0.00430154552715704\\
25	0.0043015488530565\\
26	0.00430155223998792\\
27	0.00430155568907497\\
28	0.00430155920146212\\
29	0.00430156277831491\\
30	0.00430156642082051\\
31	0.00430157013018804\\
32	0.00430157390764895\\
33	0.0043015777544575\\
34	0.00430158167189106\\
35	0.00430158566125074\\
36	0.00430158972386165\\
37	0.0043015938610734\\
38	0.00430159807426065\\
39	0.00430160236482345\\
40	0.0043016067341877\\
41	0.00430161118380577\\
42	0.00430161571515693\\
43	0.00430162032974776\\
44	0.00430162502911277\\
45	0.00430162981481488\\
46	0.00430163468844592\\
47	0.00430163965162722\\
48	0.00430164470601016\\
49	0.00430164985327671\\
50	0.00430165509513993\\
51	0.00430166043334467\\
52	0.00430166586966809\\
53	0.00430167140592028\\
54	0.00430167704394488\\
55	0.00430168278561971\\
56	0.00430168863285737\\
57	0.0043016945876059\\
58	0.00430170065184949\\
59	0.00430170682760911\\
60	0.00430171311694324\\
61	0.00430171952194844\\
62	0.00430172604476025\\
63	0.00430173268755374\\
64	0.00430173945254442\\
65	0.00430174634198882\\
66	0.00430175335818536\\
67	0.00430176050347511\\
68	0.00430176778024267\\
69	0.00430177519091678\\
70	0.00430178273797132\\
71	0.00430179042392609\\
72	0.00430179825134768\\
73	0.00430180622285032\\
74	0.0043018143410968\\
75	0.00430182260879937\\
76	0.0043018310287207\\
77	0.0043018396036747\\
78	0.00430184833652761\\
79	0.00430185723019895\\
80	0.00430186628766246\\
81	0.00430187551194713\\
82	0.00430188490613833\\
83	0.00430189447337872\\
84	0.00430190421686942\\
85	0.00430191413987106\\
86	0.00430192424570493\\
87	0.00430193453775406\\
88	0.00430194501946442\\
89	0.00430195569434613\\
90	0.00430196656597458\\
91	0.00430197763799169\\
92	0.00430198891410719\\
93	0.00430200039809985\\
94	0.00430201209381875\\
95	0.00430202400518465\\
96	0.00430203613619134\\
97	0.00430204849090697\\
98	0.0043020610734754\\
99	0.00430207388811781\\
100	0.00430208693913393\\
101	0.00430210023090359\\
102	0.00430211376788829\\
103	0.00430212755463265\\
104	0.00430214159576601\\
105	0.00430215589600398\\
106	0.00430217046015014\\
107	0.00430218529309764\\
108	0.00430220039983083\\
109	0.0043022157854271\\
110	0.00430223145505849\\
111	0.00430224741399359\\
112	0.00430226366759927\\
113	0.0043022802213426\\
114	0.00430229708079265\\
115	0.00430231425162249\\
116	0.00430233173961106\\
117	0.00430234955064526\\
118	0.00430236769072188\\
119	0.00430238616594977\\
120	0.00430240498255189\\
121	0.00430242414686748\\
122	0.00430244366535419\\
123	0.00430246354459042\\
124	0.00430248379127747\\
125	0.00430250441224198\\
126	0.0043025254144382\\
127	0.00430254680495045\\
128	0.00430256859099555\\
129	0.00430259077992528\\
130	0.004302613379229\\
131	0.00430263639653623\\
132	0.00430265983961922\\
133	0.00430268371639573\\
134	0.00430270803493175\\
135	0.00430273280344428\\
136	0.00430275803030421\\
137	0.00430278372403919\\
138	0.00430280989333666\\
139	0.00430283654704676\\
140	0.00430286369418549\\
141	0.00430289134393785\\
142	0.00430291950566099\\
143	0.00430294818888749\\
144	0.00430297740332864\\
145	0.00430300715887788\\
146	0.00430303746561421\\
147	0.00430306833380574\\
148	0.00430309977391316\\
149	0.00430313179659353\\
150	0.00430316441270393\\
151	0.00430319763330525\\
152	0.00430323146966604\\
153	0.00430326593326646\\
154	0.00430330103580227\\
155	0.00430333678918899\\
156	0.00430337320556595\\
157	0.00430341029730062\\
158	0.00430344807699291\\
159	0.00430348655747963\\
160	0.00430352575183885\\
161	0.00430356567339461\\
162	0.00430360633572154\\
163	0.00430364775264963\\
164	0.00430368993826906\\
165	0.00430373290693513\\
166	0.00430377667327339\\
167	0.00430382125218463\\
168	0.00430386665885023\\
169	0.0043039129087374\\
170	0.00430396001760469\\
171	0.00430400800150752\\
172	0.00430405687680379\\
173	0.00430410666015964\\
174	0.0043041573685553\\
175	0.00430420901929114\\
176	0.00430426162999371\\
177	0.00430431521862189\\
178	0.00430436980347337\\
179	0.00430442540319092\\
180	0.00430448203676915\\
181	0.00430453972356102\\
182	0.00430459848328482\\
183	0.00430465833603102\\
184	0.00430471930226947\\
185	0.00430478140285652\\
186	0.00430484465904247\\
187	0.0043049090924791\\
188	0.00430497472522717\\
189	0.00430504157976439\\
190	0.00430510967899332\\
191	0.00430517904624945\\
192	0.00430524970530943\\
193	0.00430532168039956\\
194	0.00430539499620432\\
195	0.00430546967787514\\
196	0.00430554575103932\\
197	0.00430562324180901\\
198	0.00430570217679061\\
199	0.00430578258309413\\
200	0.00430586448834281\\
201	0.00430594792068292\\
202	0.00430603290879372\\
203	0.00430611948189766\\
204	0.00430620766977075\\
205	0.00430629750275308\\
206	0.00430638901175968\\
207	0.00430648222829146\\
208	0.00430657718444632\\
209	0.0043066739129306\\
210	0.00430677244707076\\
211	0.00430687282082509\\
212	0.0043069750687958\\
213	0.00430707922624141\\
214	0.00430718532908909\\
215	0.00430729341394761\\
216	0.00430740351812016\\
217	0.00430751567961771\\
218	0.00430762993717242\\
219	0.00430774633025148\\
220	0.004307864899071\\
221	0.00430798568461041\\
222	0.00430810872862685\\
223	0.00430823407367006\\
224	0.00430836176309745\\
225	0.00430849184108941\\
226	0.00430862435266503\\
227	0.00430875934369804\\
228	0.00430889686093293\\
229	0.00430903695200158\\
230	0.00430917966544005\\
231	0.00430932505070579\\
232	0.00430947315819503\\
233	0.00430962403926053\\
234	0.00430977774622978\\
235	0.00430993433242325\\
236	0.0043100938521734\\
237	0.00431025636084354\\
238	0.00431042191484736\\
239	0.0043105905716687\\
240	0.00431076238988176\\
241	0.00431093742917142\\
242	0.00431111575035422\\
243	0.00431129741539953\\
244	0.00431148248745111\\
245	0.00431167103084907\\
246	0.00431186311115226\\
247	0.00431205879516082\\
248	0.00431225815093942\\
249	0.00431246124784067\\
250	0.00431266815652895\\
251	0.00431287894900475\\
252	0.00431309369862925\\
253	0.00431331248014943\\
254	0.00431353536972344\\
255	0.00431376244494658\\
256	0.00431399378487743\\
257	0.00431422947006457\\
258	0.00431446958257367\\
259	0.00431471420601489\\
260	0.00431496342557085\\
261	0.00431521732802484\\
262	0.00431547600178957\\
263	0.00431573953693623\\
264	0.00431600802522396\\
265	0.00431628156012981\\
266	0.00431656023687893\\
267	0.00431684415247528\\
268	0.00431713340573277\\
269	0.00431742809730652\\
270	0.00431772832972479\\
271	0.00431803420742105\\
272	0.00431834583676664\\
273	0.00431866332610351\\
274	0.00431898678577741\\
275	0.00431931632817146\\
276	0.00431965206773999\\
277	0.00431999412104263\\
278	0.00432034260677865\\
279	0.00432069764582177\\
280	0.00432105936125497\\
281	0.00432142787840569\\
282	0.00432180332488109\\
283	0.00432218583060372\\
284	0.00432257552784727\\
285	0.00432297255127231\\
286	0.00432337703796235\\
287	0.00432378912746006\\
288	0.00432420896180337\\
289	0.00432463668556183\\
290	0.00432507244587283\\
291	0.0043255163924781\\
292	0.00432596867776004\\
293	0.00432642945677815\\
294	0.00432689888730534\\
295	0.00432737712986429\\
296	0.00432786434776395\\
297	0.00432836070713575\\
298	0.00432886637696996\\
299	0.00432938152915217\\
300	0.00432990633849959\\
301	0.00433044098279764\\
302	0.00433098564283659\\
303	0.00433154050244854\\
304	0.00433210574854453\\
305	0.00433268157115222\\
306	0.00433326816345408\\
307	0.00433386572182634\\
308	0.00433447444587875\\
309	0.00433509453849552\\
310	0.00433572620587763\\
311	0.00433636965758684\\
312	0.00433702510659156\\
313	0.00433769276931526\\
314	0.00433837286568757\\
315	0.00433906561919869\\
316	0.00433977125695748\\
317	0.00434049000975394\\
318	0.00434122211212661\\
319	0.00434196780243559\\
320	0.00434272732294176\\
321	0.004343500919893\\
322	0.00434428884361807\\
323	0.004345091348629\\
324	0.00434590869373256\\
325	0.00434674114215151\\
326	0.00434758896165619\\
327	0.00434845242470694\\
328	0.00434933180860762\\
329	0.00435022739567029\\
330	0.00435113947339081\\
331	0.00435206833463485\\
332	0.00435301427783332\\
333	0.00435397760718565\\
334	0.00435495863286911\\
335	0.00435595767125125\\
336	0.00435697504510243\\
337	0.00435801108380426\\
338	0.00435906612354941\\
339	0.00436014050752808\\
340	0.0043612345860951\\
341	0.0043623487169129\\
342	0.00436348326506536\\
343	0.00436463860313877\\
344	0.004365815111269\\
345	0.00436701317715537\\
346	0.00436823319604323\\
347	0.00436947557074711\\
348	0.0043707407118326\\
349	0.0043720290378046\\
350	0.00437334097530099\\
351	0.00437467695929117\\
352	0.00437603743327923\\
353	0.00437742284951051\\
354	0.00437883366918161\\
355	0.00438027036265104\\
356	0.00438173340965126\\
357	0.00438322329950548\\
358	0.00438474053134917\\
359	0.00438628561435644\\
360	0.00438785906797147\\
361	0.0043894614221449\\
362	0.00439109321757576\\
363	0.00439275500595847\\
364	0.00439444735023603\\
365	0.00439617082485916\\
366	0.00439792601605232\\
367	0.00439971352208733\\
368	0.00440153395356529\\
369	0.00440338793370829\\
370	0.00440527609866231\\
371	0.00440719909781317\\
372	0.0044091575941179\\
373	0.00441115226445455\\
374	0.00441318379999359\\
375	0.00441525290659536\\
376	0.00441736030523858\\
377	0.00441950673248561\\
378	0.00442169294099152\\
379	0.00442391970006501\\
380	0.00442618779629041\\
381	0.00442849803422117\\
382	0.00443085123715657\\
383	0.0044332482480148\\
384	0.00443568993031624\\
385	0.00443817716929183\\
386	0.004440710873132\\
387	0.00444329197439091\\
388	0.00444592143156016\\
389	0.00444860023082302\\
390	0.00445132938799669\\
391	0.00445410995066293\\
392	0.00445694300047828\\
393	0.00445982965564104\\
394	0.00446277107347559\\
395	0.00446576845307248\\
396	0.00446882303790564\\
397	0.004471936118343\\
398	0.00447510903394544\\
399	0.00447834317079437\\
400	0.00448163995012209\\
401	0.00448500082751085\\
402	0.00448842729181676\\
403	0.0044919208638358\\
404	0.00449548309475358\\
405	0.00449911556445268\\
406	0.00450281987978895\\
407	0.0045065976729882\\
408	0.00451045060034677\\
409	0.00451438034142587\\
410	0.00451838859889704\\
411	0.00452247709914652\\
412	0.00452664759381917\\
413	0.00453090186251676\\
414	0.0045352417147148\\
415	0.00453966898894157\\
416	0.0045441855506476\\
417	0.00454879328977864\\
418	0.0045534941180901\\
419	0.00455828996627101\\
420	0.00456318278098549\\
421	0.00456817452198913\\
422	0.00457326715952721\\
423	0.00457846267227382\\
424	0.00458376304609542\\
425	0.00458917027399014\\
426	0.00459468635791866\\
427	0.00460031331623116\\
428	0.00460605352979721\\
429	0.00461190990897305\\
430	0.00461788548949603\\
431	0.00462398344079992\\
432	0.00463020707486431\\
433	0.00463655985556917\\
434	0.0046430454085166\\
435	0.00464966753135385\\
436	0.00465643020445243\\
437	0.00466333760180588\\
438	0.00467039410194801\\
439	0.00467760429861155\\
440	0.00468497301074153\\
441	0.00469250529133314\\
442	0.00470020643436151\\
443	0.00470808197874832\\
444	0.00471613770768533\\
445	0.0047243796400989\\
446	0.00473281400643386\\
447	0.0047414471853645\\
448	0.0047502855217763\\
449	0.00475931359508267\\
450	0.00476852503652838\\
451	0.00477792411818879\\
452	0.00478751522745984\\
453	0.00479730286871586\\
454	0.00480729166239887\\
455	0.00481748633836206\\
456	0.00482789175655993\\
457	0.00483851291571316\\
458	0.00484935496444078\\
459	0.00486042321558921\\
460	0.00487172316447817\\
461	0.00488326051116053\\
462	0.00489504118541369\\
463	0.0049070713661606\\
464	0.0049193574708969\\
465	0.00493190604161025\\
466	0.00494476407015433\\
467	0.00495794401795656\\
468	0.00497145787261548\\
469	0.00498531831362211\\
470	0.00499953876316261\\
471	0.00501413343991552\\
472	0.00502911741621035\\
473	0.00504450667898729\\
474	0.00506031819431716\\
475	0.00507656997661637\\
476	0.00509328116762204\\
477	0.0051104721473742\\
478	0.0051281647361623\\
479	0.00514638269199349\\
480	0.00516515080193658\\
481	0.00518449389127852\\
482	0.0052044376264225\\
483	0.0052250091872611\\
484	0.00524623737495304\\
485	0.00526815187863189\\
486	0.00529078468564697\\
487	0.00531417357557784\\
488	0.00533837735097006\\
489	0.00536344414407026\\
490	0.0053894165439534\\
491	0.00541633916344982\\
492	0.00544425797132098\\
493	0.00547322049379098\\
494	0.00550327613013326\\
495	0.00553447607120864\\
496	0.00556687314828419\\
497	0.00560052158990553\\
498	0.00563547657178693\\
499	0.00567179360157377\\
500	0.00570952787723091\\
501	0.0057487334698313\\
502	0.00578946229353781\\
503	0.00583176282309453\\
504	0.00587568061430283\\
505	0.00592126098008407\\
506	0.00596854041180243\\
507	0.0060175431613355\\
508	0.00606827693345952\\
509	0.0061207277550409\\
510	0.0061748191704826\\
511	0.00623017318207606\\
512	0.00628662745140264\\
513	0.00634395484991725\\
514	0.00640187056429931\\
515	0.00646008917106267\\
516	0.00651816770353974\\
517	0.00657574688726379\\
518	0.00663128465556018\\
519	0.00668140030949382\\
520	0.0067258985133532\\
521	0.00676486664598173\\
522	0.00679941350434478\\
523	0.00683269133876112\\
524	0.00686511250829205\\
525	0.00689716691348905\\
526	0.00692935080768741\\
527	0.00696191476452744\\
528	0.00699497463284185\\
529	0.00702862070688934\\
530	0.00706290768641132\\
531	0.00709786900375971\\
532	0.00713353198173733\\
533	0.00716991911972827\\
534	0.00720705005775949\\
535	0.00724494357205851\\
536	0.00728361853919417\\
537	0.00732309453011616\\
538	0.00736339161534406\\
539	0.00740453024079764\\
540	0.00744653870738161\\
541	0.00748948972389208\\
542	0.00753347142049639\\
543	0.00757747381845585\\
544	0.00762044744386721\\
545	0.00766211550727995\\
546	0.00770387386088141\\
547	0.00774606208827088\\
548	0.00778876715738478\\
549	0.00783205956081914\\
550	0.00787593614332746\\
551	0.00792038862872709\\
552	0.00796540239658982\\
553	0.00801096018996022\\
554	0.00805704284211701\\
555	0.00810363228870382\\
556	0.00815069909831113\\
557	0.00819820719505273\\
558	0.00824613258604494\\
559	0.00829446918237396\\
560	0.00834131866989406\\
561	0.00838711839027356\\
562	0.00843309183235914\\
563	0.00847930989270675\\
564	0.00852577430147598\\
565	0.00857245512281394\\
566	0.00861932104340902\\
567	0.0086663403000851\\
568	0.00871348883275888\\
569	0.00876041921360522\\
570	0.00880715257596726\\
571	0.00885403364170714\\
572	0.00890105214066917\\
573	0.0089481720325389\\
574	0.00899535448039004\\
575	0.0090425580087869\\
576	0.00908973854196738\\
577	0.00913684950111323\\
578	0.00918384197867516\\
579	0.00923066501140877\\
580	0.0092772659789151\\
581	0.00932359116078622\\
582	0.00936958649310198\\
583	0.00941519857412797\\
584	0.00946037597948012\\
585	0.00950507095777015\\
586	0.00954924158550055\\
587	0.00959285445497632\\
588	0.00963588792409171\\
589	0.0096783357981771\\
590	0.00972021084271539\\
591	0.00976138933978722\\
592	0.00980171928775089\\
593	0.00984102135019415\\
594	0.00987905595166784\\
595	0.00991543381058343\\
596	0.00994937493726701\\
597	0.00997906286423442\\
598	0.0099999191923403\\
599	0\\
600	0\\
};
\addplot [color=black!20!mycolor21,solid,forget plot]
  table[row sep=crcr]{%
1	0.00431499694832056\\
2	0.00431499894077788\\
3	0.00431500096991465\\
4	0.00431500303640849\\
5	0.00431500514094959\\
6	0.00431500728424092\\
7	0.00431500946699855\\
8	0.00431501168995181\\
9	0.00431501395384356\\
10	0.00431501625943045\\
11	0.00431501860748317\\
12	0.00431502099878675\\
13	0.00431502343414068\\
14	0.00431502591435945\\
15	0.00431502844027261\\
16	0.00431503101272507\\
17	0.00431503363257744\\
18	0.00431503630070642\\
19	0.00431503901800483\\
20	0.00431504178538223\\
21	0.00431504460376497\\
22	0.00431504747409669\\
23	0.00431505039733851\\
24	0.00431505337446941\\
25	0.0043150564064866\\
26	0.00431505949440579\\
27	0.00431506263926162\\
28	0.0043150658421079\\
29	0.00431506910401811\\
30	0.00431507242608562\\
31	0.00431507580942415\\
32	0.0043150792551682\\
33	0.0043150827644733\\
34	0.0043150863385165\\
35	0.00431508997849671\\
36	0.00431509368563522\\
37	0.00431509746117602\\
38	0.0043151013063862\\
39	0.00431510522255646\\
40	0.0043151092110016\\
41	0.00431511327306083\\
42	0.00431511741009825\\
43	0.00431512162350343\\
44	0.00431512591469183\\
45	0.0043151302851052\\
46	0.0043151347362122\\
47	0.00431513926950884\\
48	0.00431514388651896\\
49	0.00431514858879484\\
50	0.0043151533779177\\
51	0.0043151582554982\\
52	0.00431516322317698\\
53	0.00431516828262534\\
54	0.00431517343554571\\
55	0.00431517868367225\\
56	0.00431518402877144\\
57	0.00431518947264272\\
58	0.00431519501711916\\
59	0.00431520066406793\\
60	0.00431520641539104\\
61	0.00431521227302606\\
62	0.00431521823894665\\
63	0.00431522431516331\\
64	0.00431523050372412\\
65	0.00431523680671531\\
66	0.00431524322626214\\
67	0.00431524976452953\\
68	0.00431525642372277\\
69	0.00431526320608839\\
70	0.00431527011391491\\
71	0.00431527714953353\\
72	0.00431528431531907\\
73	0.00431529161369067\\
74	0.00431529904711277\\
75	0.00431530661809576\\
76	0.00431531432919702\\
77	0.00431532218302178\\
78	0.00431533018222386\\
79	0.00431533832950687\\
80	0.00431534662762482\\
81	0.00431535507938334\\
82	0.00431536368764055\\
83	0.00431537245530801\\
84	0.00431538138535179\\
85	0.00431539048079349\\
86	0.00431539974471124\\
87	0.00431540918024076\\
88	0.00431541879057662\\
89	0.00431542857897303\\
90	0.00431543854874529\\
91	0.00431544870327075\\
92	0.00431545904599003\\
93	0.00431546958040823\\
94	0.00431548031009616\\
95	0.00431549123869157\\
96	0.00431550236990035\\
97	0.0043155137074979\\
98	0.00431552525533045\\
99	0.00431553701731626\\
100	0.00431554899744721\\
101	0.00431556119979001\\
102	0.0043155736284877\\
103	0.00431558628776109\\
104	0.00431559918191016\\
105	0.00431561231531569\\
106	0.00431562569244063\\
107	0.00431563931783176\\
108	0.00431565319612135\\
109	0.0043156673320286\\
110	0.00431568173036141\\
111	0.004315696396018\\
112	0.00431571133398872\\
113	0.00431572654935767\\
114	0.00431574204730453\\
115	0.00431575783310647\\
116	0.00431577391213986\\
117	0.00431579028988224\\
118	0.00431580697191418\\
119	0.00431582396392129\\
120	0.00431584127169621\\
121	0.00431585890114065\\
122	0.00431587685826738\\
123	0.0043158951492025\\
124	0.00431591378018745\\
125	0.0043159327575813\\
126	0.00431595208786295\\
127	0.0043159717776334\\
128	0.00431599183361809\\
129	0.00431601226266936\\
130	0.00431603307176863\\
131	0.00431605426802919\\
132	0.00431607585869843\\
133	0.00431609785116058\\
134	0.00431612025293926\\
135	0.00431614307170008\\
136	0.00431616631525346\\
137	0.00431618999155733\\
138	0.00431621410872\\
139	0.00431623867500296\\
140	0.00431626369882397\\
141	0.00431628918875977\\
142	0.00431631515354941\\
143	0.00431634160209725\\
144	0.00431636854347602\\
145	0.00431639598693029\\
146	0.00431642394187952\\
147	0.00431645241792153\\
148	0.00431648142483594\\
149	0.00431651097258764\\
150	0.00431654107133032\\
151	0.00431657173141015\\
152	0.00431660296336936\\
153	0.00431663477795024\\
154	0.00431666718609875\\
155	0.00431670019896855\\
156	0.00431673382792501\\
157	0.00431676808454926\\
158	0.00431680298064243\\
159	0.00431683852822968\\
160	0.00431687473956477\\
161	0.00431691162713437\\
162	0.0043169492036625\\
163	0.0043169874821152\\
164	0.00431702647570518\\
165	0.00431706619789658\\
166	0.00431710666240985\\
167	0.00431714788322675\\
168	0.00431718987459531\\
169	0.00431723265103512\\
170	0.00431727622734244\\
171	0.00431732061859577\\
172	0.00431736584016113\\
173	0.00431741190769776\\
174	0.00431745883716381\\
175	0.00431750664482215\\
176	0.00431755534724629\\
177	0.00431760496132637\\
178	0.00431765550427544\\
179	0.00431770699363569\\
180	0.00431775944728483\\
181	0.00431781288344283\\
182	0.00431786732067833\\
183	0.00431792277791573\\
184	0.00431797927444189\\
185	0.00431803682991343\\
186	0.00431809546436388\\
187	0.00431815519821102\\
188	0.00431821605226454\\
189	0.00431827804773369\\
190	0.00431834120623502\\
191	0.00431840554980058\\
192	0.00431847110088596\\
193	0.00431853788237864\\
194	0.0043186059176065\\
195	0.00431867523034657\\
196	0.00431874584483381\\
197	0.00431881778577029\\
198	0.00431889107833416\\
199	0.00431896574818935\\
200	0.004319041821495\\
201	0.0043191193249154\\
202	0.00431919828562992\\
203	0.00431927873134326\\
204	0.0043193606902959\\
205	0.0043194441912748\\
206	0.00431952926362411\\
207	0.00431961593725647\\
208	0.0043197042426642\\
209	0.00431979421093099\\
210	0.00431988587374356\\
211	0.00431997926340383\\
212	0.00432007441284126\\
213	0.00432017135562532\\
214	0.00432027012597845\\
215	0.00432037075878894\\
216	0.00432047328962461\\
217	0.00432057775474629\\
218	0.00432068419112179\\
219	0.00432079263644017\\
220	0.00432090312912635\\
221	0.00432101570835582\\
222	0.00432113041406999\\
223	0.00432124728699163\\
224	0.00432136636864066\\
225	0.00432148770135029\\
226	0.00432161132828377\\
227	0.00432173729345093\\
228	0.00432186564172562\\
229	0.00432199641886331\\
230	0.00432212967151902\\
231	0.00432226544726559\\
232	0.00432240379461269\\
233	0.00432254476302579\\
234	0.00432268840294584\\
235	0.00432283476580929\\
236	0.00432298390406842\\
237	0.0043231358712123\\
238	0.00432329072178821\\
239	0.00432344851142332\\
240	0.004323609296847\\
241	0.00432377313591361\\
242	0.00432394008762582\\
243	0.00432411021215819\\
244	0.00432428357088166\\
245	0.00432446022638823\\
246	0.00432464024251637\\
247	0.00432482368437691\\
248	0.0043250106183795\\
249	0.00432520111225972\\
250	0.00432539523510669\\
251	0.00432559305739126\\
252	0.0043257946509949\\
253	0.00432600008923925\\
254	0.00432620944691617\\
255	0.00432642280031853\\
256	0.00432664022727168\\
257	0.00432686180716552\\
258	0.00432708762098739\\
259	0.00432731775135558\\
260	0.00432755228255349\\
261	0.0043277913005647\\
262	0.00432803489310863\\
263	0.00432828314967698\\
264	0.00432853616157106\\
265	0.00432879402193971\\
266	0.00432905682581817\\
267	0.00432932467016762\\
268	0.00432959765391561\\
269	0.00432987587799733\\
270	0.00433015944539764\\
271	0.00433044846119397\\
272	0.00433074303260004\\
273	0.00433104326901036\\
274	0.00433134928204578\\
275	0.00433166118559969\\
276	0.00433197909588508\\
277	0.00433230313148254\\
278	0.00433263341338901\\
279	0.00433297006506743\\
280	0.00433331321249714\\
281	0.00433366298422504\\
282	0.00433401951141767\\
283	0.00433438292791386\\
284	0.00433475337027831\\
285	0.00433513097785564\\
286	0.00433551589282521\\
287	0.00433590826025656\\
288	0.00433630822816532\\
289	0.00433671594756963\\
290	0.00433713157254702\\
291	0.00433755526029147\\
292	0.00433798717117093\\
293	0.00433842746878474\\
294	0.00433887632002143\\
295	0.00433933389511596\\
296	0.00433980036770704\\
297	0.00434027591489401\\
298	0.00434076071729329\\
299	0.00434125495909364\\
300	0.00434175882811116\\
301	0.00434227251584266\\
302	0.00434279621751782\\
303	0.00434333013214985\\
304	0.00434387446258422\\
305	0.00434442941554523\\
306	0.0043449952016803\\
307	0.00434557203560147\\
308	0.00434616013592399\\
309	0.00434675972530172\\
310	0.00434737103045892\\
311	0.00434799428221822\\
312	0.00434862971552461\\
313	0.00434927756946509\\
314	0.00434993808728401\\
315	0.00435061151639384\\
316	0.00435129810838146\\
317	0.00435199811901004\\
318	0.00435271180821678\\
319	0.0043534394401069\\
320	0.00435418128294456\\
321	0.00435493760914164\\
322	0.00435570869524555\\
323	0.00435649482192759\\
324	0.0043572962739739\\
325	0.00435811334028154\\
326	0.00435894631386279\\
327	0.00435979549186134\\
328	0.00436066117558451\\
329	0.00436154367055733\\
330	0.0043624432866041\\
331	0.00436336033796421\\
332	0.00436429514345049\\
333	0.00436524802665853\\
334	0.00436621931623619\\
335	0.00436720934622373\\
336	0.00436821845647436\\
337	0.00436924699316567\\
338	0.00437029530941072\\
339	0.00437136376597633\\
340	0.00437245273211302\\
341	0.00437356258649627\\
342	0.00437469371827179\\
343	0.00437584652819098\\
344	0.00437702142981376\\
345	0.00437821885075589\\
346	0.00437943923394699\\
347	0.00438068303683586\\
348	0.0043819507271359\\
349	0.00438324278311871\\
350	0.00438455969391738\\
351	0.00438590195984028\\
352	0.00438727009269485\\
353	0.00438866461612229\\
354	0.00439008606594292\\
355	0.00439153499051145\\
356	0.00439301195107609\\
357	0.00439451752214569\\
358	0.00439605229186454\\
359	0.00439761686239407\\
360	0.00439921185030041\\
361	0.00440083788694632\\
362	0.00440249561888556\\
363	0.00440418570825825\\
364	0.00440590883318406\\
365	0.00440766568815095\\
366	0.00440945698439527\\
367	0.0044112834502697\\
368	0.00441314583159376\\
369	0.00441504489198145\\
370	0.00441698141313957\\
371	0.00441895619512892\\
372	0.00442097005658\\
373	0.00442302383485335\\
374	0.0044251183861332\\
375	0.00442725458544242\\
376	0.00442943332656388\\
377	0.00443165552185372\\
378	0.00443392210192886\\
379	0.00443623401521027\\
380	0.00443859222730249\\
381	0.00444099772018842\\
382	0.00444345149121776\\
383	0.00444595455186778\\
384	0.00444850792625582\\
385	0.00445111264938523\\
386	0.00445376976511048\\
387	0.00445648032381303\\
388	0.00445924537978965\\
389	0.00446206598836737\\
390	0.00446494320277788\\
391	0.00446787807084819\\
392	0.00447087163159433\\
393	0.00447392491184105\\
394	0.00447703892302631\\
395	0.00448021465837707\\
396	0.00448345309066581\\
397	0.00448675517093273\\
398	0.00449012182999356\\
399	0.00449355412269602\\
400	0.00449705347939348\\
401	0.00450062137353033\\
402	0.00450425932415012\\
403	0.00450796889867491\\
404	0.00451175171598503\\
405	0.00451560944982605\\
406	0.00451954383256431\\
407	0.00452355665930099\\
408	0.00452764979233868\\
409	0.00453182516598714\\
410	0.0045360847917572\\
411	0.00454043076433954\\
412	0.00454486526831602\\
413	0.00454939058452903\\
414	0.00455400909656978\\
415	0.00455872329750913\\
416	0.00456353579682076\\
417	0.00456844932732751\\
418	0.00457346675197657\\
419	0.00457859107016703\\
420	0.00458382542321563\\
421	0.00458917309840705\\
422	0.00459463753106349\\
423	0.00460022230356219\\
424	0.0046059311395983\\
425	0.00461176788800474\\
426	0.00461773648292344\\
427	0.00462384083473687\\
428	0.00463007454269391\\
429	0.00463642669478293\\
430	0.00464289947138564\\
431	0.00464949507804951\\
432	0.00465621574373993\\
433	0.00466306371942736\\
434	0.00467004127672492\\
435	0.00467715070359887\\
436	0.00468439430175754\\
437	0.00469177438409394\\
438	0.00469929327236011\\
439	0.0047069532953228\\
440	0.00471475678774227\\
441	0.00472270609062633\\
442	0.00473080355332531\\
443	0.00473905153808924\\
444	0.00474745242750651\\
445	0.00475600863417552\\
446	0.00476472260824388\\
447	0.00477359682704871\\
448	0.00478263371851145\\
449	0.00479185742060597\\
450	0.00480128388406291\\
451	0.00481091912788268\\
452	0.0048207694746903\\
453	0.00483084157395689\\
454	0.00484114242736373\\
455	0.00485167941703031\\
456	0.00486246033618055\\
457	0.00487349342185919\\
458	0.00488478739021973\\
459	0.00489635147453042\\
460	0.00490819546631836\\
461	0.00492032976123099\\
462	0.00493276541531882\\
463	0.00494551423176829\\
464	0.00495858894656423\\
465	0.00497200374370233\\
466	0.00498577276785185\\
467	0.00499990968924556\\
468	0.00501442887447785\\
469	0.00502934546358139\\
470	0.00504467541338836\\
471	0.00506043554226604\\
472	0.00507664357606649\\
473	0.00509331819510764\\
474	0.00511047908213691\\
475	0.00512814697166886\\
476	0.00514634370104921\\
477	0.00516509226166558\\
478	0.00518441684913845\\
479	0.00520434292710943\\
480	0.00522489733605872\\
481	0.00524610843829957\\
482	0.00526800545903931\\
483	0.00529061981117917\\
484	0.00531398865034318\\
485	0.0053381701578214\\
486	0.00536321129115987\\
487	0.00538915358114079\\
488	0.00541604042217931\\
489	0.0054439163752652\\
490	0.00547282747592509\\
491	0.00550282136238553\\
492	0.00553394694540943\\
493	0.0055662540850593\\
494	0.00559979322677082\\
495	0.00563461493277553\\
496	0.00567076928958796\\
497	0.00570830517638911\\
498	0.00574727309521228\\
499	0.00578772535756451\\
500	0.00582971021638085\\
501	0.00587326972706042\\
502	0.00591843697540275\\
503	0.00596523248465533\\
504	0.00601358359371006\\
505	0.00606322632772836\\
506	0.00611408959572197\\
507	0.00616606641138297\\
508	0.00621900369888828\\
509	0.00627268736679497\\
510	0.00632686701159178\\
511	0.00638156012032947\\
512	0.00643639486293027\\
513	0.00649077754730627\\
514	0.00654356250451543\\
515	0.00659100231749131\\
516	0.00663288736294835\\
517	0.00666934946135635\\
518	0.00670138719394604\\
519	0.00673220793885969\\
520	0.00676222399287684\\
521	0.00679192047183419\\
522	0.0068217634860095\\
523	0.00685198358162248\\
524	0.00688268742381229\\
525	0.00691395526121181\\
526	0.00694583429170407\\
527	0.0069783535437772\\
528	0.00701153670833231\\
529	0.00704540320024276\\
530	0.00707997042946935\\
531	0.00711525540894092\\
532	0.0071512752192859\\
533	0.0071880474870327\\
534	0.00722559083761262\\
535	0.0072639251102361\\
536	0.00730307067581672\\
537	0.00734304842914756\\
538	0.00738391399526787\\
539	0.00742574890892478\\
540	0.00746833575598918\\
541	0.00750995398553259\\
542	0.00755035018874908\\
543	0.00759032441223851\\
544	0.00763072324758412\\
545	0.00767162903141855\\
546	0.00771311969698628\\
547	0.00775520894382689\\
548	0.00779789355612366\\
549	0.00784116403536065\\
550	0.007885008292511\\
551	0.00792941148083906\\
552	0.00797435600912795\\
553	0.00801982170725804\\
554	0.00806578702538616\\
555	0.00811222764331581\\
556	0.0081591063939708\\
557	0.00820641693101756\\
558	0.00825328101622345\\
559	0.00829844873438657\\
560	0.00834367510390144\\
561	0.00838916204614102\\
562	0.00843494259992949\\
563	0.00848099134388042\\
564	0.00852727997864265\\
565	0.00857377962259063\\
566	0.008620461017887\\
567	0.00866730268255435\\
568	0.00871391830588996\\
569	0.00876044306491334\\
570	0.00880715274894682\\
571	0.00885403364426453\\
572	0.00890105214164188\\
573	0.00894817203301684\\
574	0.00899535448062322\\
575	0.00904255800889528\\
576	0.00908973854201466\\
577	0.00913684950113236\\
578	0.00918384197868222\\
579	0.0092306650114111\\
580	0.00927726597891577\\
581	0.00932359116078638\\
582	0.00936958649310202\\
583	0.00941519857412799\\
584	0.00946037597948012\\
585	0.00950507095777015\\
586	0.00954924158550055\\
587	0.00959285445497631\\
588	0.00963588792409171\\
589	0.00967833579817709\\
590	0.00972021084271539\\
591	0.00976138933978722\\
592	0.00980171928775089\\
593	0.00984102135019415\\
594	0.00987905595166784\\
595	0.00991543381058343\\
596	0.00994937493726701\\
597	0.00997906286423442\\
598	0.0099999191923403\\
599	0\\
600	0\\
};
\addplot [color=black!50!mycolor20,solid,forget plot]
  table[row sep=crcr]{%
1	0.00432156273776794\\
2	0.0043215647387563\\
3	0.00432156677664413\\
4	0.00432156885211437\\
5	0.00432157096586263\\
6	0.00432157311859742\\
7	0.00432157531104039\\
8	0.00432157754392669\\
9	0.00432157981800506\\
10	0.00432158213403821\\
11	0.00432158449280303\\
12	0.00432158689509076\\
13	0.00432158934170756\\
14	0.00432159183347438\\
15	0.0043215943712275\\
16	0.00432159695581882\\
17	0.00432159958811599\\
18	0.00432160226900288\\
19	0.00432160499937982\\
20	0.00432160778016382\\
21	0.00432161061228898\\
22	0.00432161349670679\\
23	0.00432161643438645\\
24	0.00432161942631524\\
25	0.00432162247349878\\
26	0.00432162557696142\\
27	0.00432162873774662\\
28	0.00432163195691715\\
29	0.00432163523555574\\
30	0.00432163857476522\\
31	0.00432164197566891\\
32	0.0043216454394111\\
33	0.00432164896715743\\
34	0.00432165256009517\\
35	0.00432165621943383\\
36	0.00432165994640539\\
37	0.00432166374226482\\
38	0.00432166760829047\\
39	0.00432167154578451\\
40	0.00432167555607341\\
41	0.00432167964050835\\
42	0.00432168380046576\\
43	0.00432168803734771\\
44	0.00432169235258235\\
45	0.0043216967476246\\
46	0.00432170122395643\\
47	0.00432170578308751\\
48	0.00432171042655564\\
49	0.0043217151559273\\
50	0.00432171997279828\\
51	0.00432172487879413\\
52	0.0043217298755707\\
53	0.00432173496481483\\
54	0.00432174014824476\\
55	0.00432174542761091\\
56	0.00432175080469635\\
57	0.00432175628131747\\
58	0.00432176185932452\\
59	0.00432176754060238\\
60	0.00432177332707118\\
61	0.00432177922068684\\
62	0.00432178522344185\\
63	0.00432179133736603\\
64	0.00432179756452703\\
65	0.00432180390703123\\
66	0.00432181036702437\\
67	0.00432181694669225\\
68	0.00432182364826164\\
69	0.00432183047400093\\
70	0.00432183742622088\\
71	0.0043218445072756\\
72	0.00432185171956308\\
73	0.00432185906552633\\
74	0.00432186654765398\\
75	0.00432187416848134\\
76	0.00432188193059106\\
77	0.00432188983661416\\
78	0.00432189788923104\\
79	0.00432190609117213\\
80	0.0043219144452191\\
81	0.00432192295420573\\
82	0.00432193162101884\\
83	0.00432194044859928\\
84	0.00432194943994308\\
85	0.00432195859810238\\
86	0.00432196792618659\\
87	0.00432197742736336\\
88	0.00432198710485967\\
89	0.00432199696196312\\
90	0.00432200700202283\\
91	0.00432201722845085\\
92	0.00432202764472316\\
93	0.00432203825438095\\
94	0.0043220490610319\\
95	0.00432206006835131\\
96	0.00432207128008351\\
97	0.00432208270004313\\
98	0.00432209433211635\\
99	0.00432210618026241\\
100	0.0043221182485148\\
101	0.00432213054098283\\
102	0.00432214306185296\\
103	0.00432215581539031\\
104	0.00432216880594014\\
105	0.00432218203792936\\
106	0.00432219551586814\\
107	0.00432220924435133\\
108	0.00432222322806028\\
109	0.00432223747176431\\
110	0.00432225198032239\\
111	0.00432226675868503\\
112	0.00432228181189574\\
113	0.00432229714509297\\
114	0.00432231276351192\\
115	0.00432232867248623\\
116	0.00432234487745005\\
117	0.0043223613839398\\
118	0.00432237819759618\\
119	0.00432239532416614\\
120	0.00432241276950487\\
121	0.00432243053957785\\
122	0.00432244864046308\\
123	0.00432246707835305\\
124	0.00432248585955702\\
125	0.00432250499050324\\
126	0.00432252447774112\\
127	0.00432254432794366\\
128	0.00432256454790976\\
129	0.00432258514456658\\
130	0.0043226061249721\\
131	0.00432262749631745\\
132	0.00432264926592956\\
133	0.00432267144127379\\
134	0.00432269402995643\\
135	0.0043227170397275\\
136	0.00432274047848354\\
137	0.00432276435427027\\
138	0.00432278867528554\\
139	0.00432281344988224\\
140	0.0043228386865711\\
141	0.00432286439402405\\
142	0.00432289058107703\\
143	0.00432291725673317\\
144	0.0043229444301661\\
145	0.00432297211072314\\
146	0.00432300030792858\\
147	0.00432302903148723\\
148	0.00432305829128777\\
149	0.00432308809740627\\
150	0.00432311846010989\\
151	0.00432314938986035\\
152	0.00432318089731803\\
153	0.00432321299334535\\
154	0.00432324568901105\\
155	0.00432327899559389\\
156	0.00432331292458686\\
157	0.00432334748770122\\
158	0.00432338269687067\\
159	0.00432341856425585\\
160	0.00432345510224846\\
161	0.00432349232347589\\
162	0.00432353024080571\\
163	0.0043235688673503\\
164	0.00432360821647165\\
165	0.00432364830178617\\
166	0.00432368913716954\\
167	0.0043237307367618\\
168	0.00432377311497254\\
169	0.00432381628648593\\
170	0.00432386026626634\\
171	0.00432390506956349\\
172	0.00432395071191814\\
173	0.00432399720916782\\
174	0.00432404457745243\\
175	0.00432409283322027\\
176	0.00432414199323388\\
177	0.00432419207457645\\
178	0.00432424309465783\\
179	0.00432429507122102\\
180	0.00432434802234867\\
181	0.00432440196646958\\
182	0.00432445692236575\\
183	0.00432451290917902\\
184	0.00432456994641837\\
185	0.00432462805396691\\
186	0.00432468725208936\\
187	0.00432474756143953\\
188	0.00432480900306796\\
189	0.00432487159842961\\
190	0.00432493536939208\\
191	0.0043250003382435\\
192	0.00432506652770101\\
193	0.00432513396091912\\
194	0.00432520266149843\\
195	0.00432527265349443\\
196	0.00432534396142653\\
197	0.00432541661028717\\
198	0.00432549062555138\\
199	0.00432556603318634\\
200	0.00432564285966097\\
201	0.00432572113195616\\
202	0.00432580087757486\\
203	0.00432588212455263\\
204	0.00432596490146808\\
205	0.00432604923745394\\
206	0.00432613516220803\\
207	0.00432622270600464\\
208	0.00432631189970619\\
209	0.00432640277477489\\
210	0.00432649536328497\\
211	0.00432658969793499\\
212	0.0043266858120604\\
213	0.00432678373964642\\
214	0.00432688351534124\\
215	0.00432698517446952\\
216	0.00432708875304603\\
217	0.00432719428778965\\
218	0.00432730181613792\\
219	0.00432741137626148\\
220	0.00432752300707917\\
221	0.00432763674827337\\
222	0.00432775264030556\\
223	0.00432787072443241\\
224	0.00432799104272207\\
225	0.00432811363807102\\
226	0.00432823855422097\\
227	0.00432836583577658\\
228	0.00432849552822319\\
229	0.00432862767794518\\
230	0.00432876233224459\\
231	0.00432889953936042\\
232	0.00432903934848804\\
233	0.00432918180979924\\
234	0.00432932697446275\\
235	0.00432947489466525\\
236	0.0043296256236327\\
237	0.0043297792156524\\
238	0.00432993572609534\\
239	0.00433009521143934\\
240	0.0043302577292924\\
241	0.004330423338417\\
242	0.00433059209875454\\
243	0.00433076407145074\\
244	0.00433093931888154\\
245	0.00433111790467953\\
246	0.00433129989376101\\
247	0.00433148535235388\\
248	0.00433167434802614\\
249	0.00433186694971496\\
250	0.00433206322775662\\
251	0.00433226325391725\\
252	0.00433246710142406\\
253	0.00433267484499761\\
254	0.00433288656088486\\
255	0.00433310232689295\\
256	0.00433332222242387\\
257	0.00433354632851018\\
258	0.00433377472785137\\
259	0.00433400750485138\\
260	0.00433424474565704\\
261	0.00433448653819752\\
262	0.00433473297222474\\
263	0.00433498413935495\\
264	0.00433524013311138\\
265	0.004335501048968\\
266	0.00433576698439459\\
267	0.00433603803890286\\
268	0.00433631431409392\\
269	0.00433659591370706\\
270	0.0043368829436698\\
271	0.00433717551214946\\
272	0.00433747372960603\\
273	0.00433777770884665\\
274	0.00433808756508153\\
275	0.00433840341598155\\
276	0.0043387253817374\\
277	0.00433905358512051\\
278	0.00433938815154574\\
279	0.00433972920913584\\
280	0.00434007688878776\\
281	0.00434043132424101\\
282	0.00434079265214799\\
283	0.00434116101214642\\
284	0.00434153654693376\\
285	0.004341919402344\\
286	0.00434230972742683\\
287	0.00434270767452879\\
288	0.00434311339937728\\
289	0.00434352706116669\\
290	0.00434394882264741\\
291	0.00434437885021715\\
292	0.00434481731401516\\
293	0.00434526438801898\\
294	0.00434572025014423\\
295	0.00434618508234692\\
296	0.00434665907072889\\
297	0.00434714240564587\\
298	0.00434763528181837\\
299	0.00434813789844575\\
300	0.00434865045932256\\
301	0.00434917317295794\\
302	0.00434970625269746\\
303	0.00435024991684725\\
304	0.00435080438880053\\
305	0.00435136989716599\\
306	0.00435194667589759\\
307	0.00435253496442565\\
308	0.00435313500778857\\
309	0.00435374705676422\\
310	0.00435437136800087\\
311	0.00435500820414609\\
312	0.00435565783397282\\
313	0.0043563205325014\\
314	0.0043569965811155\\
315	0.00435768626767073\\
316	0.00435838988659347\\
317	0.00435910773896732\\
318	0.0043598401326048\\
319	0.00436058738210031\\
320	0.0043613498088614\\
321	0.00436212774111298\\
322	0.00436292151387087\\
323	0.0043637314688778\\
324	0.00436455795449694\\
325	0.00436540132555528\\
326	0.00436626194312957\\
327	0.00436714017426647\\
328	0.00436803639162843\\
329	0.00436895097305518\\
330	0.00436988430103141\\
331	0.0043708367620511\\
332	0.00437180874586785\\
333	0.00437280064462305\\
334	0.00437381285184419\\
335	0.00437484576130809\\
336	0.0043758997657681\\
337	0.00437697525554858\\
338	0.00437807261701865\\
339	0.00437919223096592\\
340	0.00438033447090391\\
341	0.00438149970136107\\
342	0.00438268827621516\\
343	0.00438390053714678\\
344	0.00438513681229731\\
345	0.00438639741528997\\
346	0.00438768264538279\\
347	0.00438899284910283\\
348	0.00439032852186834\\
349	0.00439169016963117\\
350	0.00439307830916022\\
351	0.00439449346833624\\
352	0.00439593618645757\\
353	0.00439740701455486\\
354	0.00439890651571774\\
355	0.004400435265466\\
356	0.00440199385231873\\
357	0.00440358287824578\\
358	0.00440520295913732\\
359	0.00440685472530561\\
360	0.00440853882202268\\
361	0.00441025591009763\\
362	0.00441200666649775\\
363	0.0044137917850179\\
364	0.0044156119770036\\
365	0.00441746797213329\\
366	0.00441936051926643\\
367	0.00442129038736393\\
368	0.00442325836648904\\
369	0.00442526526889701\\
370	0.00442731193022267\\
371	0.0044293992107759\\
372	0.00443152799695614\\
373	0.00443369920279728\\
374	0.00443591377165537\\
375	0.00443817267805222\\
376	0.00444047692968832\\
377	0.0044428275696388\\
378	0.00444522567874573\\
379	0.00444767237822\\
380	0.00445016883246353\\
381	0.00445271625212137\\
382	0.00445531589736821\\
383	0.00445796908142958\\
384	0.00446067717432911\\
385	0.00446344160684308\\
386	0.00446626387462823\\
387	0.0044691455424694\\
388	0.00447208824856691\\
389	0.00447509370874885\\
390	0.00447816372044754\\
391	0.00448130016621814\\
392	0.00448450501649252\\
393	0.00448778033113468\\
394	0.00449112825913685\\
395	0.00449455103528687\\
396	0.00449805097120155\\
397	0.00450163043343578\\
398	0.00450529178476443\\
399	0.00450903312499081\\
400	0.00451284518155388\\
401	0.00451672918428431\\
402	0.00452068637426448\\
403	0.00452471800279135\\
404	0.0045288253302109\\
405	0.0045330096246486\\
406	0.00453727216070178\\
407	0.00454161421822174\\
408	0.00454603708133855\\
409	0.00455054203755669\\
410	0.00455513037477349\\
411	0.00455980336567712\\
412	0.0045645622530175\\
413	0.00456940826439893\\
414	0.00457434261038014\\
415	0.00457936647922931\\
416	0.00458448103142263\\
417	0.00458968739511137\\
418	0.00459498666169048\\
419	0.00460037988191109\\
420	0.0046058680639046\\
421	0.0046114521734614\\
422	0.00461713313071034\\
423	0.00462291180709291\\
424	0.00462878901320134\\
425	0.00463476551515811\\
426	0.00464084202686863\\
427	0.00464701916605228\\
428	0.00465330764650406\\
429	0.00465972294715053\\
430	0.00466626796949668\\
431	0.0046729457221835\\
432	0.00467975932905515\\
433	0.00468671203806247\\
434	0.00469380723103485\\
435	0.0047010484344903\\
436	0.0047084393316286\\
437	0.00471598377555301\\
438	0.00472368580381427\\
439	0.00473154965436415\\
440	0.004739579782992\\
441	0.00474778088230933\\
442	0.00475615790236735\\
443	0.00476471607315437\\
444	0.00477346092989589\\
445	0.00478239834456105\\
446	0.00479153457560125\\
447	0.00480087637727662\\
448	0.00481043130862172\\
449	0.00482020706581279\\
450	0.00483021084487188\\
451	0.00484044985114416\\
452	0.00485093163581722\\
453	0.00486166411637998\\
454	0.00487265559813321\\
455	0.00488391479673149\\
456	0.00489545086173582\\
457	0.00490727340119204\\
458	0.00491939250723813\\
459	0.00493181878277906\\
460	0.00494456336936779\\
461	0.00495763797663287\\
462	0.00497105491383159\\
463	0.0049848271235522\\
464	0.0049989682118395\\
465	0.0050134924396604\\
466	0.00502841475798565\\
467	0.00504375091022794\\
468	0.00505951747774275\\
469	0.00507573192480233\\
470	0.0050924126447223\\
471	0.00510957900694087\\
472	0.00512725140490503\\
473	0.00514545130482393\\
474	0.0051642012957185\\
475	0.0051835251411441\\
476	0.00520344785397583\\
477	0.00522399583307634\\
478	0.0052451969499694\\
479	0.00526708001994137\\
480	0.00528967582130181\\
481	0.00531302073050156\\
482	0.00533717128049394\\
483	0.00536217440954135\\
484	0.00538807060812187\\
485	0.00541490203822802\\
486	0.00544271156614922\\
487	0.00547154299835975\\
488	0.00550144111267412\\
489	0.00553245141674796\\
490	0.00556461982902692\\
491	0.00559799228696207\\
492	0.00563262021729175\\
493	0.00566855676452508\\
494	0.00570585375095233\\
495	0.00574456031946986\\
496	0.00578472113107927\\
497	0.00582637393627206\\
498	0.00586941332332261\\
499	0.00591367412563156\\
500	0.0059591312442232\\
501	0.00600573960009374\\
502	0.00605342869976447\\
503	0.00610209427571644\\
504	0.00615166353661207\\
505	0.00620218464799494\\
506	0.0062534348899334\\
507	0.00630517093441952\\
508	0.00635706597096826\\
509	0.00640860003944755\\
510	0.00645893456739127\\
511	0.0065044826325323\\
512	0.00654456380460978\\
513	0.00657922957752016\\
514	0.00660912384878171\\
515	0.00663781310127959\\
516	0.00666570035195224\\
517	0.00669325982984133\\
518	0.00672095204731527\\
519	0.0067490035419314\\
520	0.00677751497187324\\
521	0.00680656037357772\\
522	0.00683618266850091\\
523	0.00686640821868181\\
524	0.00689725841090671\\
525	0.0069287508654358\\
526	0.00696090155737811\\
527	0.00699372616758012\\
528	0.00702724047294954\\
529	0.00706146075026851\\
530	0.00709640408608181\\
531	0.00713208865138898\\
532	0.00716853402710913\\
533	0.00720576089941698\\
534	0.00724379067528484\\
535	0.00728265318316113\\
536	0.00732242355127311\\
537	0.00736319290804136\\
538	0.00740362713540518\\
539	0.00744294329917313\\
540	0.00748116645121112\\
541	0.00751979739156501\\
542	0.00755890721054429\\
543	0.00759857987564944\\
544	0.00763885345226277\\
545	0.00767972887409422\\
546	0.00772120151702209\\
547	0.00776326398074099\\
548	0.00780590660744546\\
549	0.00784911740722832\\
550	0.00789288179370365\\
551	0.00793718233108313\\
552	0.0079819985880279\\
553	0.00802730742623898\\
554	0.00807308398607837\\
555	0.00811930438610058\\
556	0.00816597124869282\\
557	0.00821131168944101\\
558	0.00825575379156772\\
559	0.00830045453103746\\
560	0.00834548332965024\\
561	0.00839082502862416\\
562	0.00843645437478884\\
563	0.00848234516615932\\
564	0.00852847072095564\\
565	0.00857480405939613\\
566	0.00862132528999987\\
567	0.00866764572460421\\
568	0.00871393573045342\\
569	0.00876044307747747\\
570	0.00880715274928313\\
571	0.0088540336444077\\
572	0.00890105214171166\\
573	0.00894817203305002\\
574	0.00899535448063817\\
575	0.00904255800890162\\
576	0.00908973854201715\\
577	0.00913684950113324\\
578	0.0091838419786825\\
579	0.00923066501141119\\
580	0.00927726597891579\\
581	0.00932359116078638\\
582	0.00936958649310201\\
583	0.00941519857412797\\
584	0.00946037597948012\\
585	0.00950507095777015\\
586	0.00954924158550055\\
587	0.00959285445497631\\
588	0.00963588792409171\\
589	0.00967833579817709\\
590	0.00972021084271538\\
591	0.00976138933978722\\
592	0.00980171928775089\\
593	0.00984102135019415\\
594	0.00987905595166784\\
595	0.00991543381058343\\
596	0.00994937493726701\\
597	0.00997906286423442\\
598	0.0099999191923403\\
599	0\\
600	0\\
};
\addplot [color=black!60!mycolor21,solid,forget plot]
  table[row sep=crcr]{%
1	0.00432695855327979\\
2	0.00432696069719906\\
3	0.00432696288063723\\
4	0.00432696510432514\\
5	0.00432696736900736\\
6	0.00432696967544223\\
7	0.0043269720244022\\
8	0.00432697441667397\\
9	0.004326976853059\\
10	0.00432697933437348\\
11	0.0043269818614489\\
12	0.00432698443513213\\
13	0.00432698705628577\\
14	0.00432698972578843\\
15	0.00432699244453511\\
16	0.0043269952134374\\
17	0.0043269980334239\\
18	0.00432700090544033\\
19	0.00432700383045009\\
20	0.00432700680943442\\
21	0.00432700984339293\\
22	0.00432701293334372\\
23	0.00432701608032385\\
24	0.00432701928538961\\
25	0.004327022549617\\
26	0.004327025874102\\
27	0.00432702925996095\\
28	0.00432703270833108\\
29	0.00432703622037062\\
30	0.0043270397972594\\
31	0.00432704344019931\\
32	0.0043270471504144\\
33	0.00432705092915171\\
34	0.00432705477768136\\
35	0.00432705869729714\\
36	0.00432706268931694\\
37	0.00432706675508314\\
38	0.00432707089596317\\
39	0.00432707511334992\\
40	0.00432707940866214\\
41	0.00432708378334511\\
42	0.00432708823887093\\
43	0.00432709277673917\\
44	0.00432709739847733\\
45	0.00432710210564138\\
46	0.00432710689981621\\
47	0.0043271117826163\\
48	0.00432711675568621\\
49	0.00432712182070122\\
50	0.00432712697936762\\
51	0.0043271322334237\\
52	0.00432713758464017\\
53	0.00432714303482059\\
54	0.00432714858580225\\
55	0.00432715423945664\\
56	0.00432715999769022\\
57	0.004327165862445\\
58	0.00432717183569916\\
59	0.00432717791946777\\
60	0.00432718411580356\\
61	0.0043271904267975\\
62	0.00432719685457959\\
63	0.00432720340131958\\
64	0.00432721006922769\\
65	0.00432721686055543\\
66	0.00432722377759633\\
67	0.0043272308226867\\
68	0.00432723799820648\\
69	0.00432724530658002\\
70	0.00432725275027702\\
71	0.00432726033181314\\
72	0.00432726805375125\\
73	0.00432727591870186\\
74	0.00432728392932442\\
75	0.00432729208832793\\
76	0.00432730039847212\\
77	0.0043273088625682\\
78	0.00432731748347992\\
79	0.0043273262641246\\
80	0.00432733520747408\\
81	0.00432734431655567\\
82	0.00432735359445329\\
83	0.00432736304430868\\
84	0.00432737266932216\\
85	0.00432738247275397\\
86	0.00432739245792525\\
87	0.00432740262821932\\
88	0.00432741298708276\\
89	0.00432742353802666\\
90	0.00432743428462778\\
91	0.00432744523052981\\
92	0.00432745637944468\\
93	0.00432746773515366\\
94	0.00432747930150892\\
95	0.00432749108243471\\
96	0.00432750308192877\\
97	0.00432751530406363\\
98	0.00432752775298816\\
99	0.00432754043292886\\
100	0.00432755334819144\\
101	0.00432756650316226\\
102	0.00432757990230986\\
103	0.00432759355018648\\
104	0.00432760745142971\\
105	0.00432762161076401\\
106	0.00432763603300237\\
107	0.00432765072304815\\
108	0.00432766568589645\\
109	0.00432768092663611\\
110	0.0043276964504514\\
111	0.00432771226262376\\
112	0.0043277283685338\\
113	0.00432774477366301\\
114	0.00432776148359568\\
115	0.00432777850402097\\
116	0.00432779584073471\\
117	0.00432781349964156\\
118	0.00432783148675699\\
119	0.00432784980820943\\
120	0.00432786847024234\\
121	0.00432788747921642\\
122	0.00432790684161185\\
123	0.00432792656403053\\
124	0.00432794665319832\\
125	0.00432796711596752\\
126	0.00432798795931918\\
127	0.00432800919036546\\
128	0.00432803081635228\\
129	0.00432805284466176\\
130	0.00432807528281481\\
131	0.00432809813847368\\
132	0.00432812141944487\\
133	0.00432814513368153\\
134	0.00432816928928657\\
135	0.00432819389451525\\
136	0.00432821895777821\\
137	0.00432824448764443\\
138	0.00432827049284404\\
139	0.00432829698227167\\
140	0.0043283239649894\\
141	0.00432835145022986\\
142	0.00432837944739962\\
143	0.00432840796608247\\
144	0.00432843701604282\\
145	0.00432846660722887\\
146	0.00432849674977659\\
147	0.00432852745401276\\
148	0.00432855873045901\\
149	0.0043285905898353\\
150	0.00432862304306379\\
151	0.00432865610127265\\
152	0.00432868977579992\\
153	0.00432872407819774\\
154	0.00432875902023616\\
155	0.00432879461390737\\
156	0.00432883087143009\\
157	0.00432886780525368\\
158	0.00432890542806266\\
159	0.00432894375278118\\
160	0.0043289827925775\\
161	0.0043290225608689\\
162	0.00432906307132606\\
163	0.00432910433787831\\
164	0.00432914637471825\\
165	0.00432918919630696\\
166	0.00432923281737907\\
167	0.00432927725294803\\
168	0.00432932251831144\\
169	0.0043293686290565\\
170	0.00432941560106546\\
171	0.0043294634505214\\
172	0.00432951219391403\\
173	0.00432956184804536\\
174	0.00432961243003595\\
175	0.00432966395733081\\
176	0.00432971644770586\\
177	0.00432976991927409\\
178	0.00432982439049204\\
179	0.00432987988016658\\
180	0.00432993640746147\\
181	0.00432999399190443\\
182	0.00433005265339393\\
183	0.00433011241220651\\
184	0.0043301732890039\\
185	0.00433023530484067\\
186	0.00433029848117162\\
187	0.00433036283985964\\
188	0.00433042840318345\\
189	0.00433049519384591\\
190	0.00433056323498191\\
191	0.00433063255016688\\
192	0.00433070316342541\\
193	0.00433077509923983\\
194	0.00433084838255915\\
195	0.00433092303880804\\
196	0.00433099909389616\\
197	0.0043310765742276\\
198	0.00433115550671049\\
199	0.00433123591876661\\
200	0.00433131783834173\\
201	0.00433140129391559\\
202	0.00433148631451246\\
203	0.00433157292971159\\
204	0.00433166116965828\\
205	0.00433175106507474\\
206	0.00433184264727161\\
207	0.00433193594815924\\
208	0.00433203100025962\\
209	0.00433212783671843\\
210	0.00433222649131709\\
211	0.00433232699848539\\
212	0.00433242939331433\\
213	0.00433253371156901\\
214	0.00433263998970192\\
215	0.00433274826486667\\
216	0.00433285857493167\\
217	0.00433297095849443\\
218	0.0043330854548959\\
219	0.00433320210423526\\
220	0.00433332094738493\\
221	0.00433344202600606\\
222	0.00433356538256405\\
223	0.00433369106034479\\
224	0.00433381910347088\\
225	0.00433394955691843\\
226	0.00433408246653407\\
227	0.00433421787905251\\
228	0.00433435584211419\\
229	0.00433449640428373\\
230	0.00433463961506829\\
231	0.00433478552493675\\
232	0.00433493418533899\\
233	0.00433508564872588\\
234	0.00433523996856943\\
235	0.00433539719938359\\
236	0.00433555739674543\\
237	0.00433572061731673\\
238	0.0043358869188662\\
239	0.00433605636029206\\
240	0.00433622900164514\\
241	0.00433640490415262\\
242	0.00433658413024219\\
243	0.00433676674356681\\
244	0.00433695280902993\\
245	0.00433714239281142\\
246	0.0043373355623941\\
247	0.00433753238659085\\
248	0.0043377329355722\\
249	0.00433793728089482\\
250	0.00433814549553049\\
251	0.00433835765389585\\
252	0.00433857383188282\\
253	0.00433879410688977\\
254	0.00433901855785339\\
255	0.00433924726528139\\
256	0.00433948031128586\\
257	0.00433971777961768\\
258	0.00433995975570167\\
259	0.00434020632667236\\
260	0.00434045758141111\\
261	0.0043407136105838\\
262	0.00434097450667962\\
263	0.00434124036405092\\
264	0.00434151127895385\\
265	0.00434178734959036\\
266	0.00434206867615101\\
267	0.00434235536085907\\
268	0.00434264750801582\\
269	0.00434294522404696\\
270	0.00434324861755036\\
271	0.00434355779934511\\
272	0.00434387288252188\\
273	0.00434419398249479\\
274	0.00434452121705471\\
275	0.00434485470642407\\
276	0.00434519457331332\\
277	0.00434554094297909\\
278	0.00434589394328401\\
279	0.00434625370475853\\
280	0.0043466203606644\\
281	0.00434699404706052\\
282	0.00434737490287053\\
283	0.00434776306995298\\
284	0.00434815869317363\\
285	0.00434856192048034\\
286	0.0043489729029806\\
287	0.00434939179502176\\
288	0.00434981875427422\\
289	0.00435025394181805\\
290	0.00435069752223251\\
291	0.00435114966368958\\
292	0.00435161053805094\\
293	0.00435208032096964\\
294	0.00435255919199538\\
295	0.00435304733468546\\
296	0.00435354493672011\\
297	0.00435405219002397\\
298	0.00435456929089321\\
299	0.00435509644012949\\
300	0.0043556338431809\\
301	0.00435618171029078\\
302	0.0043567402566551\\
303	0.00435730970258917\\
304	0.00435789027370451\\
305	0.0043584822010971\\
306	0.00435908572154804\\
307	0.00435970107773785\\
308	0.00436032851847588\\
309	0.0043609682989465\\
310	0.0043616206809735\\
311	0.00436228593330542\\
312	0.00436296433192296\\
313	0.00436365616037193\\
314	0.00436436171012386\\
315	0.00436508128096751\\
316	0.00436581518143451\\
317	0.00436656372926307\\
318	0.00436732725190336\\
319	0.00436810608706939\\
320	0.00436890058334199\\
321	0.00436971110082804\\
322	0.00437053801188148\\
323	0.0043713817018922\\
324	0.00437224257014867\\
325	0.0043731210307803\\
326	0.00437401751378662\\
327	0.00437493246615868\\
328	0.00437586635309825\\
329	0.0043768196593399\\
330	0.00437779289057866\\
331	0.00437878657500424\\
332	0.00437980126494006\\
333	0.00438083753858002\\
334	0.0043818960018099\\
335	0.00438297729009362\\
336	0.00438408207039105\\
337	0.0043852110430619\\
338	0.00438636494368967\\
339	0.00438754454473416\\
340	0.00438875065688568\\
341	0.00438998412993984\\
342	0.00439124585291623\\
343	0.00439253675292674\\
344	0.00439385779169215\\
345	0.00439520995665367\\
346	0.00439659423677513\\
347	0.00439800980650354\\
348	0.00439945288719747\\
349	0.00440092400728502\\
350	0.00440242370465192\\
351	0.00440395252680818\\
352	0.00440551103111611\\
353	0.00440709978509535\\
354	0.00440871936667319\\
355	0.00441037036353843\\
356	0.00441205336796628\\
357	0.00441376898130935\\
358	0.00441551781433659\\
359	0.00441730048719107\\
360	0.00441911762932339\\
361	0.00442096987939735\\
362	0.00442285788516469\\
363	0.00442478230330447\\
364	0.00442674379922321\\
365	0.0044287430468102\\
366	0.00443078072814295\\
367	0.00443285753313633\\
368	0.00443497415912887\\
369	0.00443713131039822\\
370	0.00443932969759767\\
371	0.00444157003710466\\
372	0.00444385305027066\\
373	0.00444617946256172\\
374	0.00444855000257832\\
375	0.00445096540094059\\
376	0.00445342638902633\\
377	0.00445593369754693\\
378	0.00445848805494721\\
379	0.00446109018561414\\
380	0.00446374080788017\\
381	0.00446644063180854\\
382	0.00446919035674984\\
383	0.00447199066866269\\
384	0.00447484223719717\\
385	0.00447774571254766\\
386	0.0044807017220934\\
387	0.00448371086686047\\
388	0.00448677371786047\\
389	0.00448989081239065\\
390	0.00449306265041508\\
391	0.00449628969119351\\
392	0.0044995723503721\\
393	0.00450291099778542\\
394	0.00450630595618257\\
395	0.00450975750081975\\
396	0.00451326585891238\\
397	0.00451683120515006\\
398	0.00452045364209266\\
399	0.00452413735820681\\
400	0.00452789394284761\\
401	0.00453172483387663\\
402	0.00453563150123215\\
403	0.00453961544838681\\
404	0.00454367821396767\\
405	0.00454782137356271\\
406	0.00455204654173728\\
407	0.00455635537427974\\
408	0.00456074957067161\\
409	0.00456523087672764\\
410	0.00456980108733492\\
411	0.0045744620497789\\
412	0.00457921566841893\\
413	0.00458406390985492\\
414	0.00458900880804196\\
415	0.00459405246996129\\
416	0.00459919708201783\\
417	0.00460444491720469\\
418	0.00460979834306404\\
419	0.00461525983049784\\
420	0.00462083196341752\\
421	0.00462651744911842\\
422	0.0046323191295337\\
423	0.00463823999399964\\
424	0.00464428319606497\\
425	0.00465045208049502\\
426	0.00465675024305133\\
427	0.00466318170211106\\
428	0.00466975068776097\\
429	0.00467646102269409\\
430	0.00468331626401164\\
431	0.00469032009822216\\
432	0.00469747634809923\\
433	0.00470478897990422\\
434	0.00471226211098159\\
435	0.00471990001772997\\
436	0.00472770714394326\\
437	0.00473568810951569\\
438	0.00474384771949964\\
439	0.00475219097350291\\
440	0.00476072307541075\\
441	0.0047694494434255\\
442	0.00477837572044345\\
443	0.00478750778485379\\
444	0.00479685176198382\\
445	0.0048064140365962\\
446	0.00481620126662683\\
447	0.00482622039523245\\
448	0.00483647864149911\\
449	0.004846983503795\\
450	0.00485774280977831\\
451	0.00486876475354069\\
452	0.00488005791751922\\
453	0.00489163129574941\\
454	0.00490349431852351\\
455	0.00491565687852088\\
456	0.00492812935847706\\
457	0.00494092266045717\\
458	0.00495404823679719\\
459	0.00496751812277545\\
460	0.00498134497106543\\
461	0.00499554208798379\\
462	0.00501012347143418\\
463	0.00502510385021868\\
464	0.00504049872440022\\
465	0.00505632440914279\\
466	0.00507259808091958\\
467	0.00508933782211824\\
468	0.0051065626661562\\
469	0.00512429264292472\\
470	0.00514254882450478\\
471	0.00516135337167488\\
472	0.00518072958309387\\
473	0.00520070195230599\\
474	0.00522129624785663\\
475	0.00524253966141403\\
476	0.00526446073178525\\
477	0.00528708936517332\\
478	0.00531046069907047\\
479	0.00533462727696618\\
480	0.00535963747295621\\
481	0.00538552963238027\\
482	0.00541234318243217\\
483	0.0054401178576874\\
484	0.00546889387848687\\
485	0.00549871271132115\\
486	0.00552962358923932\\
487	0.00556167658536844\\
488	0.00559492199409357\\
489	0.00562940946645967\\
490	0.00566518680759279\\
491	0.00570229826537744\\
492	0.00574057270663483\\
493	0.00577994483980594\\
494	0.00582041438361801\\
495	0.00586196977977204\\
496	0.00590458482482532\\
497	0.00594821370429102\\
498	0.0059929190006349\\
499	0.00603878586440403\\
500	0.00608571187789154\\
501	0.00613354908577968\\
502	0.00618209679919275\\
503	0.00623111975679493\\
504	0.0062803460795765\\
505	0.00632932117173365\\
506	0.00637730219787598\\
507	0.00642155840085762\\
508	0.00646045553143953\\
509	0.00649394778892431\\
510	0.00652243322939103\\
511	0.00654927054506817\\
512	0.00657526604504233\\
513	0.00660087923655145\\
514	0.00662658810284878\\
515	0.00665262797492567\\
516	0.00667909600829342\\
517	0.00670606353836837\\
518	0.00673357132505209\\
519	0.00676164372812885\\
520	0.0067903003794335\\
521	0.00681955743687604\\
522	0.00684942956662143\\
523	0.00687993114722263\\
524	0.00691107661826311\\
525	0.00694288081781127\\
526	0.00697535922344092\\
527	0.00700852813728366\\
528	0.00704240489215253\\
529	0.0070770080849942\\
530	0.00711235785255778\\
531	0.00714847517504492\\
532	0.0071853817036795\\
533	0.00722312753643244\\
534	0.00726179417464418\\
535	0.00730115804025518\\
536	0.00733950427705983\\
537	0.00737657424641395\\
538	0.007413491098651\\
539	0.00745084701001097\\
540	0.00748873240113763\\
541	0.00752721478980981\\
542	0.00756629878659511\\
543	0.00760598438697385\\
544	0.00764626809925852\\
545	0.00768714464315555\\
546	0.0077286069021184\\
547	0.00777064572670249\\
548	0.00781324967273179\\
549	0.00785640469595659\\
550	0.00790009381669526\\
551	0.00794429678697347\\
552	0.00798898986048007\\
553	0.00803414520393946\\
554	0.00807975463982383\\
555	0.00812542668900067\\
556	0.00816939338006928\\
557	0.00821329420027352\\
558	0.00825751363257453\\
559	0.00830208407653316\\
560	0.00834698368736043\\
561	0.00839218901342985\\
562	0.00843767581465922\\
563	0.00848341941476406\\
564	0.0085293949138881\\
565	0.00857558303702803\\
566	0.00862162837759405\\
567	0.00866765938100907\\
568	0.00871393573140663\\
569	0.00876044307752332\\
570	0.00880715274930379\\
571	0.00885403364441763\\
572	0.00890105214171626\\
573	0.00894817203305202\\
574	0.00899535448063901\\
575	0.00904255800890193\\
576	0.00908973854201726\\
577	0.00913684950113328\\
578	0.00918384197868252\\
579	0.00923066501141119\\
580	0.00927726597891579\\
581	0.00932359116078638\\
582	0.00936958649310202\\
583	0.00941519857412797\\
584	0.00946037597948011\\
585	0.00950507095777015\\
586	0.00954924158550055\\
587	0.00959285445497631\\
588	0.00963588792409171\\
589	0.00967833579817709\\
590	0.00972021084271538\\
591	0.00976138933978722\\
592	0.00980171928775089\\
593	0.00984102135019415\\
594	0.00987905595166784\\
595	0.00991543381058343\\
596	0.00994937493726701\\
597	0.00997906286423442\\
598	0.0099999191923403\\
599	0\\
600	0\\
};
\addplot [color=black!80!mycolor21,solid,forget plot]
  table[row sep=crcr]{%
1	0.00433439942612947\\
2	0.00433440181287078\\
3	0.00433440424353772\\
4	0.00433440671894131\\
5	0.00433440923990749\\
6	0.0043344118072775\\
7	0.00433441442190828\\
8	0.00433441708467256\\
9	0.00433441979645923\\
10	0.00433442255817373\\
11	0.00433442537073827\\
12	0.00433442823509211\\
13	0.00433443115219192\\
14	0.00433443412301212\\
15	0.0043344371485452\\
16	0.00433444022980203\\
17	0.00433444336781228\\
18	0.00433444656362468\\
19	0.00433444981830739\\
20	0.00433445313294842\\
21	0.00433445650865589\\
22	0.00433445994655854\\
23	0.00433446344780604\\
24	0.0043344670135694\\
25	0.0043344706450413\\
26	0.00433447434343657\\
27	0.00433447810999261\\
28	0.00433448194596979\\
29	0.00433448585265188\\
30	0.00433448983134642\\
31	0.0043344938833852\\
32	0.00433449801012487\\
33	0.00433450221294707\\
34	0.00433450649325921\\
35	0.00433451085249485\\
36	0.00433451529211415\\
37	0.00433451981360434\\
38	0.00433452441848027\\
39	0.00433452910828499\\
40	0.00433453388459013\\
41	0.00433453874899659\\
42	0.00433454370313495\\
43	0.00433454874866611\\
44	0.00433455388728183\\
45	0.00433455912070522\\
46	0.00433456445069154\\
47	0.00433456987902862\\
48	0.00433457540753754\\
49	0.00433458103807315\\
50	0.00433458677252486\\
51	0.00433459261281712\\
52	0.00433459856091015\\
53	0.0043346046188007\\
54	0.00433461078852263\\
55	0.00433461707214761\\
56	0.00433462347178577\\
57	0.00433462998958654\\
58	0.00433463662773934\\
59	0.0043346433884743\\
60	0.00433465027406297\\
61	0.00433465728681925\\
62	0.00433466442910008\\
63	0.00433467170330617\\
64	0.00433467911188301\\
65	0.00433468665732144\\
66	0.00433469434215876\\
67	0.00433470216897956\\
68	0.00433471014041645\\
69	0.00433471825915103\\
70	0.00433472652791482\\
71	0.00433473494949035\\
72	0.00433474352671167\\
73	0.00433475226246583\\
74	0.0043347611596935\\
75	0.00433477022139027\\
76	0.00433477945060735\\
77	0.00433478885045292\\
78	0.00433479842409299\\
79	0.00433480817475246\\
80	0.00433481810571649\\
81	0.00433482822033137\\
82	0.00433483852200576\\
83	0.00433484901421175\\
84	0.00433485970048622\\
85	0.00433487058443192\\
86	0.00433488166971885\\
87	0.00433489296008528\\
88	0.00433490445933922\\
89	0.00433491617135971\\
90	0.00433492810009815\\
91	0.00433494024957949\\
92	0.00433495262390377\\
93	0.00433496522724763\\
94	0.0043349780638655\\
95	0.00433499113809108\\
96	0.00433500445433906\\
97	0.00433501801710641\\
98	0.00433503183097401\\
99	0.00433504590060823\\
100	0.00433506023076251\\
101	0.00433507482627893\\
102	0.00433508969209002\\
103	0.0043351048332203\\
104	0.00433512025478811\\
105	0.00433513596200738\\
106	0.00433515196018933\\
107	0.00433516825474427\\
108	0.00433518485118366\\
109	0.00433520175512177\\
110	0.00433521897227784\\
111	0.00433523650847781\\
112	0.00433525436965644\\
113	0.00433527256185936\\
114	0.00433529109124517\\
115	0.00433530996408747\\
116	0.00433532918677717\\
117	0.00433534876582449\\
118	0.00433536870786138\\
119	0.00433538901964363\\
120	0.00433540970805338\\
121	0.00433543078010148\\
122	0.00433545224292965\\
123	0.00433547410381325\\
124	0.0043354963701636\\
125	0.00433551904953057\\
126	0.00433554214960529\\
127	0.00433556567822269\\
128	0.00433558964336422\\
129	0.00433561405316057\\
130	0.00433563891589457\\
131	0.00433566424000396\\
132	0.00433569003408434\\
133	0.00433571630689219\\
134	0.00433574306734774\\
135	0.00433577032453824\\
136	0.00433579808772094\\
137	0.00433582636632634\\
138	0.00433585516996155\\
139	0.00433588450841341\\
140	0.00433591439165201\\
141	0.00433594482983415\\
142	0.00433597583330677\\
143	0.00433600741261058\\
144	0.00433603957848358\\
145	0.00433607234186507\\
146	0.0043361057138991\\
147	0.00433613970593858\\
148	0.00433617432954894\\
149	0.00433620959651243\\
150	0.00433624551883205\\
151	0.00433628210873568\\
152	0.00433631937868039\\
153	0.00433635734135661\\
154	0.00433639600969271\\
155	0.0043364353968595\\
156	0.00433647551627451\\
157	0.0043365163816069\\
158	0.00433655800678218\\
159	0.00433660040598687\\
160	0.00433664359367367\\
161	0.00433668758456619\\
162	0.00433673239366435\\
163	0.00433677803624933\\
164	0.00433682452788904\\
165	0.00433687188444338\\
166	0.00433692012206996\\
167	0.00433696925722949\\
168	0.00433701930669161\\
169	0.00433707028754067\\
170	0.00433712221718179\\
171	0.0043371751133467\\
172	0.00433722899410006\\
173	0.00433728387784567\\
174	0.00433733978333297\\
175	0.00433739672966335\\
176	0.00433745473629706\\
177	0.00433751382305975\\
178	0.00433757401014956\\
179	0.00433763531814403\\
180	0.00433769776800725\\
181	0.00433776138109724\\
182	0.00433782617917334\\
183	0.00433789218440382\\
184	0.00433795941937356\\
185	0.00433802790709201\\
186	0.00433809767100114\\
187	0.00433816873498355\\
188	0.00433824112337095\\
189	0.00433831486095252\\
190	0.00433838997298366\\
191	0.00433846648519474\\
192	0.00433854442380009\\
193	0.00433862381550717\\
194	0.00433870468752586\\
195	0.00433878706757803\\
196	0.00433887098390719\\
197	0.00433895646528827\\
198	0.00433904354103794\\
199	0.00433913224102455\\
200	0.0043392225956787\\
201	0.00433931463600408\\
202	0.00433940839358797\\
203	0.0043395039006125\\
204	0.00433960118986595\\
205	0.00433970029475408\\
206	0.00433980124931195\\
207	0.00433990408821587\\
208	0.00434000884679544\\
209	0.004340115561046\\
210	0.00434022426764125\\
211	0.00434033500394618\\
212	0.00434044780802998\\
213	0.0043405627186797\\
214	0.00434067977541363\\
215	0.00434079901849514\\
216	0.00434092048894711\\
217	0.00434104422856615\\
218	0.00434117027993722\\
219	0.00434129868644876\\
220	0.00434142949230795\\
221	0.00434156274255614\\
222	0.00434169848308484\\
223	0.00434183676065168\\
224	0.00434197762289716\\
225	0.00434212111836126\\
226	0.00434226729650053\\
227	0.0043424162077056\\
228	0.00434256790331901\\
229	0.00434272243565309\\
230	0.00434287985800876\\
231	0.00434304022469402\\
232	0.00434320359104341\\
233	0.00434337001343736\\
234	0.00434353954932225\\
235	0.00434371225723068\\
236	0.00434388819680213\\
237	0.00434406742880416\\
238	0.00434425001515388\\
239	0.00434443601893982\\
240	0.00434462550444436\\
241	0.00434481853716655\\
242	0.00434501518384526\\
243	0.00434521551248277\\
244	0.00434541959236909\\
245	0.00434562749410651\\
246	0.00434583928963452\\
247	0.00434605505225551\\
248	0.00434627485666074\\
249	0.00434649877895702\\
250	0.00434672689669355\\
251	0.00434695928888968\\
252	0.0043471960360629\\
253	0.00434743722025764\\
254	0.00434768292507426\\
255	0.00434793323569891\\
256	0.004348188238934\\
257	0.00434844802322883\\
258	0.00434871267871118\\
259	0.00434898229721944\\
260	0.00434925697233537\\
261	0.00434953679941713\\
262	0.00434982187563352\\
263	0.00435011229999835\\
264	0.00435040817340566\\
265	0.00435070959866558\\
266	0.00435101668054085\\
267	0.00435132952578396\\
268	0.00435164824317491\\
269	0.00435197294355986\\
270	0.0043523037398901\\
271	0.00435264074726211\\
272	0.00435298408295787\\
273	0.00435333386648616\\
274	0.00435369021962445\\
275	0.00435405326646142\\
276	0.00435442313344009\\
277	0.00435479994940184\\
278	0.00435518384563063\\
279	0.00435557495589837\\
280	0.00435597341651037\\
281	0.00435637936635177\\
282	0.00435679294693444\\
283	0.00435721430244384\\
284	0.00435764357978722\\
285	0.00435808092864142\\
286	0.00435852650150141\\
287	0.00435898045372909\\
288	0.004359442943602\\
289	0.00435991413236213\\
290	0.00436039418426491\\
291	0.00436088326662764\\
292	0.0043613815498779\\
293	0.00436188920760098\\
294	0.00436240641658698\\
295	0.00436293335687661\\
296	0.0043634702118058\\
297	0.00436401716804851\\
298	0.00436457441565798\\
299	0.00436514214810526\\
300	0.00436572056231491\\
301	0.00436630985869759\\
302	0.00436691024117822\\
303	0.00436752191721976\\
304	0.00436814509784179\\
305	0.00436877999763203\\
306	0.00436942683475128\\
307	0.00437008583092947\\
308	0.00437075721145214\\
309	0.00437144120513579\\
310	0.00437213804429049\\
311	0.00437284796466764\\
312	0.0043735712053918\\
313	0.00437430800887281\\
314	0.00437505862069719\\
315	0.00437582328949509\\
316	0.00437660226677961\\
317	0.00437739580675559\\
318	0.00437820416609351\\
319	0.00437902760366431\\
320	0.00437986638023082\\
321	0.00438072075809017\\
322	0.0043815910006622\\
323	0.00438247737201782\\
324	0.00438338013634087\\
325	0.00438429955731751\\
326	0.00438523589744583\\
327	0.00438618941725928\\
328	0.00438716037445751\\
329	0.0043881490229384\\
330	0.0043891556117265\\
331	0.00439018038379473\\
332	0.00439122357477812\\
333	0.00439228541158204\\
334	0.00439336611089349\\
335	0.00439446587760887\\
336	0.00439558490320252\\
337	0.00439672336407297\\
338	0.00439788141991926\\
339	0.00439905921222226\\
340	0.00440025686292988\\
341	0.0044014744734705\\
342	0.00440271212422077\\
343	0.00440396987448398\\
344	0.00440524776275998\\
345	0.00440654580531069\\
346	0.00440786398489007\\
347	0.00440920402385365\\
348	0.00441057060935592\\
349	0.00441196426536201\\
350	0.00441338552577781\\
351	0.00441483493460724\\
352	0.00441631304610621\\
353	0.00441782042492295\\
354	0.0044193576462028\\
355	0.00442092529565036\\
356	0.00442252396978652\\
357	0.00442415427627943\\
358	0.00442581683414257\\
359	0.00442751227392775\\
360	0.00442924123792482\\
361	0.00443100438036846\\
362	0.00443280236765423\\
363	0.00443463587856507\\
364	0.00443650560451036\\
365	0.00443841224977974\\
366	0.00444035653181446\\
367	0.00444233918149978\\
368	0.00444436094348195\\
369	0.00444642257651406\\
370	0.00444852485383648\\
371	0.00445066856359745\\
372	0.00445285450932144\\
373	0.00445508351043292\\
374	0.00445735640284518\\
375	0.00445967403962522\\
376	0.00446203729174694\\
377	0.00446444704894753\\
378	0.00446690422070181\\
379	0.00446940973733432\\
380	0.00447196455128907\\
381	0.00447456963857968\\
382	0.00447722600044551\\
383	0.00447993466524188\\
384	0.0044826966905934\\
385	0.00448551316584321\\
386	0.0044883852148302\\
387	0.00449131399902813\\
388	0.00449430072107765\\
389	0.00449734662873975\\
390	0.00450045301929358\\
391	0.00450362124439519\\
392	0.00450685271541414\\
393	0.00451014890930527\\
394	0.00451351137525375\\
395	0.00451694174302952\\
396	0.0045204417364464\\
397	0.00452401320375389\\
398	0.00452765820524174\\
399	0.0045313789886601\\
400	0.00453517753936264\\
401	0.00453905557873385\\
402	0.0045430148741424\\
403	0.00454705724075805\\
404	0.00455118454346668\\
405	0.00455539869888527\\
406	0.00455970167748078\\
407	0.00456409550579306\\
408	0.00456858226876349\\
409	0.00457316411217185\\
410	0.00457784324519672\\
411	0.00458262194310348\\
412	0.00458750255001831\\
413	0.00459248748176176\\
414	0.00459757922875969\\
415	0.00460278035902185\\
416	0.00460809352116916\\
417	0.00461352144748655\\
418	0.00461906695697717\\
419	0.00462473295838905\\
420	0.00463052245318885\\
421	0.00463643853847783\\
422	0.0046424844098883\\
423	0.00464866336458066\\
424	0.00465497880455615\\
425	0.00466143424044806\\
426	0.00466803329447707\\
427	0.00467477969251494\\
428	0.00468167725361806\\
429	0.0046887299093657\\
430	0.00469594173033701\\
431	0.00470331693310338\\
432	0.00471085988765404\\
433	0.00471857512528687\\
434	0.00472646734699674\\
435	0.00473454143239654\\
436	0.00474280244921074\\
437	0.00475125566338333\\
438	0.00475990654984844\\
439	0.00476876080401437\\
440	0.00477782435402021\\
441	0.00478710337383024\\
442	0.00479660429724071\\
443	0.0048063338328793\\
444	0.00481629898027238\\
445	0.00482650704700604\\
446	0.00483696566688477\\
447	0.0048476828189708\\
448	0.0048586668488317\\
449	0.0048699264922354\\
450	0.00488147089918492\\
451	0.00489330965865071\\
452	0.00490545282471894\\
453	0.00491791094418666\\
454	0.00493069508562214\\
455	0.0049438168698928\\
456	0.00495728850214065\\
457	0.00497112280515963\\
458	0.00498533325409378\\
459	0.0049999340123333\\
460	0.00501493996843038\\
461	0.0050303667737904\\
462	0.00504623088081715\\
463	0.00506254958112901\\
464	0.00507934104342271\\
465	0.00509662435029209\\
466	0.0051144195332361\\
467	0.00513274760527605\\
468	0.00515163059078765\\
469	0.00517109155320189\\
470	0.00519115462429839\\
471	0.00521184504677305\\
472	0.00523318926021573\\
473	0.00525521509555144\\
474	0.00527795217275313\\
475	0.00530143265222818\\
476	0.00532569914663003\\
477	0.00535080864374956\\
478	0.00537679809147539\\
479	0.00540370560503476\\
480	0.00543157784115154\\
481	0.00546046220442261\\
482	0.00549040667370869\\
483	0.00552145904822377\\
484	0.00555366580937486\\
485	0.00558704762300578\\
486	0.00562138769391787\\
487	0.0056566986408806\\
488	0.00569298744985147\\
489	0.00573025396980975\\
490	0.00576848877383418\\
491	0.00580766931668448\\
492	0.00584796783373438\\
493	0.00588943448996365\\
494	0.0059320294334912\\
495	0.00597568959241816\\
496	0.00602032355118579\\
497	0.00606580644661143\\
498	0.00611196917249998\\
499	0.0061585824520033\\
500	0.00620541096103591\\
501	0.0062520924216836\\
502	0.00629802034333257\\
503	0.00634147450349103\\
504	0.00637971149189926\\
505	0.00641262592420366\\
506	0.00644048029382127\\
507	0.00646574725732371\\
508	0.00649010650081638\\
509	0.00651399622170918\\
510	0.00653789763179205\\
511	0.0065620801181893\\
512	0.00658665536465146\\
513	0.00661169510902583\\
514	0.006637239803336\\
515	0.00666331249094905\\
516	0.00668993163021992\\
517	0.00671711228438945\\
518	0.00674486803670351\\
519	0.00677321216013866\\
520	0.0068021579289145\\
521	0.00683171890503599\\
522	0.00686190912800818\\
523	0.00689274324833507\\
524	0.00692423666951791\\
525	0.00695640570141792\\
526	0.00698926773593825\\
527	0.00702284145500003\\
528	0.0070571470756616\\
529	0.00709220664141625\\
530	0.00712804440069717\\
531	0.00716472854050527\\
532	0.00720234880295647\\
533	0.00723985893619002\\
534	0.00727623466988971\\
535	0.00731151311131741\\
536	0.00734719353934899\\
537	0.0073833556106261\\
538	0.00742008300313997\\
539	0.00745740535755315\\
540	0.00749532669484875\\
541	0.00753384645106589\\
542	0.00757296262261381\\
543	0.00761267172446651\\
544	0.00765296869609351\\
545	0.00769384670138497\\
546	0.00773529689100766\\
547	0.00777730812572076\\
548	0.00781986665437991\\
549	0.00786295573820756\\
550	0.00790655521515773\\
551	0.00795064102098\\
552	0.00799518479699153\\
553	0.0080401859272667\\
554	0.00808464783813621\\
555	0.00812773859917388\\
556	0.00817111836314038\\
557	0.00821486580995575\\
558	0.00825897791658296\\
559	0.00830343402391313\\
560	0.00834821243243883\\
561	0.00839329070202066\\
562	0.00843864597211353\\
563	0.00848425526709758\\
564	0.00853009920226463\\
565	0.00857589049842109\\
566	0.00862164036385014\\
567	0.0086676593810865\\
568	0.00871393573141297\\
569	0.00876044307752625\\
570	0.00880715274930517\\
571	0.00885403364441827\\
572	0.00890105214171656\\
573	0.00894817203305216\\
574	0.00899535448063906\\
575	0.00904255800890195\\
576	0.00908973854201727\\
577	0.00913684950113329\\
578	0.00918384197868252\\
579	0.00923066501141119\\
580	0.00927726597891579\\
581	0.00932359116078638\\
582	0.00936958649310202\\
583	0.00941519857412798\\
584	0.00946037597948012\\
585	0.00950507095777016\\
586	0.00954924158550055\\
587	0.00959285445497631\\
588	0.00963588792409171\\
589	0.00967833579817709\\
590	0.00972021084271539\\
591	0.00976138933978722\\
592	0.00980171928775089\\
593	0.00984102135019415\\
594	0.00987905595166784\\
595	0.00991543381058343\\
596	0.00994937493726701\\
597	0.00997906286423442\\
598	0.0099999191923403\\
599	0\\
600	0\\
};
\addplot [color=black,solid,forget plot]
  table[row sep=crcr]{%
1	0.00434222595919898\\
2	0.00434222821860981\\
3	0.00434223051966135\\
4	0.00434223286312353\\
5	0.00434223524978059\\
6	0.00434223768043138\\
7	0.00434224015588943\\
8	0.00434224267698356\\
9	0.00434224524455786\\
10	0.00434224785947209\\
11	0.00434225052260194\\
12	0.00434225323483943\\
13	0.00434225599709309\\
14	0.00434225881028843\\
15	0.00434226167536802\\
16	0.00434226459329199\\
17	0.00434226756503826\\
18	0.00434227059160297\\
19	0.00434227367400076\\
20	0.00434227681326514\\
21	0.00434228001044872\\
22	0.00434228326662375\\
23	0.00434228658288232\\
24	0.00434228996033694\\
25	0.00434229340012076\\
26	0.00434229690338792\\
27	0.00434230047131408\\
28	0.0043423041050967\\
29	0.00434230780595548\\
30	0.0043423115751329\\
31	0.00434231541389449\\
32	0.0043423193235292\\
33	0.00434232330535016\\
34	0.00434232736069477\\
35	0.00434233149092533\\
36	0.00434233569742945\\
37	0.00434233998162055\\
38	0.00434234434493852\\
39	0.00434234878884976\\
40	0.00434235331484817\\
41	0.00434235792445522\\
42	0.00434236261922094\\
43	0.00434236740072407\\
44	0.00434237227057269\\
45	0.00434237723040493\\
46	0.00434238228188933\\
47	0.00434238742672547\\
48	0.00434239266664463\\
49	0.00434239800341021\\
50	0.00434240343881856\\
51	0.00434240897469947\\
52	0.00434241461291672\\
53	0.00434242035536886\\
54	0.00434242620398974\\
55	0.00434243216074926\\
56	0.00434243822765407\\
57	0.00434244440674809\\
58	0.00434245070011332\\
59	0.00434245710987066\\
60	0.00434246363818046\\
61	0.00434247028724327\\
62	0.00434247705930069\\
63	0.00434248395663615\\
64	0.00434249098157545\\
65	0.00434249813648803\\
66	0.00434250542378723\\
67	0.00434251284593148\\
68	0.00434252040542502\\
69	0.00434252810481881\\
70	0.00434253594671141\\
71	0.00434254393374964\\
72	0.00434255206862989\\
73	0.00434256035409873\\
74	0.00434256879295399\\
75	0.00434257738804563\\
76	0.00434258614227683\\
77	0.0043425950586049\\
78	0.00434260414004231\\
79	0.00434261338965775\\
80	0.00434262281057715\\
81	0.00434263240598475\\
82	0.00434264217912425\\
83	0.00434265213329983\\
84	0.00434266227187737\\
85	0.00434267259828563\\
86	0.00434268311601725\\
87	0.00434269382863009\\
88	0.00434270473974859\\
89	0.00434271585306465\\
90	0.00434272717233917\\
91	0.00434273870140327\\
92	0.00434275044415967\\
93	0.00434276240458394\\
94	0.00434277458672585\\
95	0.00434278699471084\\
96	0.00434279963274138\\
97	0.00434281250509854\\
98	0.00434282561614324\\
99	0.00434283897031795\\
100	0.00434285257214809\\
101	0.00434286642624372\\
102	0.00434288053730087\\
103	0.00434289491010356\\
104	0.00434290954952511\\
105	0.00434292446052985\\
106	0.00434293964817495\\
107	0.00434295511761211\\
108	0.00434297087408942\\
109	0.00434298692295295\\
110	0.00434300326964874\\
111	0.00434301991972472\\
112	0.00434303687883253\\
113	0.00434305415272947\\
114	0.00434307174728058\\
115	0.00434308966846045\\
116	0.00434310792235553\\
117	0.00434312651516604\\
118	0.00434314545320822\\
119	0.00434316474291647\\
120	0.00434318439084554\\
121	0.00434320440367275\\
122	0.00434322478820046\\
123	0.00434324555135824\\
124	0.00434326670020539\\
125	0.00434328824193327\\
126	0.00434331018386777\\
127	0.00434333253347188\\
128	0.00434335529834836\\
129	0.00434337848624228\\
130	0.0043434021050436\\
131	0.00434342616279007\\
132	0.00434345066766977\\
133	0.00434347562802419\\
134	0.00434350105235097\\
135	0.0043435269493068\\
136	0.00434355332771052\\
137	0.00434358019654606\\
138	0.0043436075649656\\
139	0.00434363544229271\\
140	0.0043436638380256\\
141	0.00434369276184034\\
142	0.00434372222359422\\
143	0.00434375223332918\\
144	0.00434378280127535\\
145	0.00434381393785427\\
146	0.00434384565368292\\
147	0.00434387795957702\\
148	0.00434391086655505\\
149	0.00434394438584184\\
150	0.00434397852887255\\
151	0.00434401330729659\\
152	0.00434404873298164\\
153	0.00434408481801783\\
154	0.00434412157472182\\
155	0.00434415901564095\\
156	0.00434419715355784\\
157	0.00434423600149462\\
158	0.00434427557271744\\
159	0.00434431588074112\\
160	0.00434435693933373\\
161	0.00434439876252149\\
162	0.00434444136459342\\
163	0.00434448476010633\\
164	0.00434452896389\\
165	0.00434457399105212\\
166	0.00434461985698352\\
167	0.00434466657736367\\
168	0.00434471416816581\\
169	0.00434476264566271\\
170	0.00434481202643204\\
171	0.0043448623273624\\
172	0.00434491356565887\\
173	0.00434496575884917\\
174	0.00434501892478946\\
175	0.00434507308167075\\
176	0.00434512824802498\\
177	0.00434518444273155\\
178	0.00434524168502374\\
179	0.00434529999449536\\
180	0.00434535939110761\\
181	0.00434541989519574\\
182	0.00434548152747634\\
183	0.00434554430905421\\
184	0.00434560826142982\\
185	0.00434567340650657\\
186	0.00434573976659853\\
187	0.00434580736443798\\
188	0.00434587622318318\\
189	0.00434594636642652\\
190	0.00434601781820248\\
191	0.00434609060299602\\
192	0.00434616474575091\\
193	0.00434624027187832\\
194	0.00434631720726564\\
195	0.00434639557828524\\
196	0.00434647541180373\\
197	0.00434655673519102\\
198	0.00434663957632959\\
199	0.00434672396362446\\
200	0.00434680992601267\\
201	0.00434689749297314\\
202	0.00434698669453694\\
203	0.00434707756129756\\
204	0.00434717012442129\\
205	0.00434726441565799\\
206	0.004347360467352\\
207	0.00434745831245296\\
208	0.00434755798452739\\
209	0.00434765951777001\\
210	0.00434776294701537\\
211	0.00434786830774987\\
212	0.00434797563612374\\
213	0.0043480849689634\\
214	0.00434819634378422\\
215	0.00434830979880302\\
216	0.00434842537295115\\
217	0.00434854310588775\\
218	0.00434866303801332\\
219	0.00434878521048335\\
220	0.0043489096652223\\
221	0.00434903644493781\\
222	0.00434916559313522\\
223	0.00434929715413229\\
224	0.00434943117307416\\
225	0.00434956769594852\\
226	0.00434970676960141\\
227	0.00434984844175275\\
228	0.0043499927610126\\
229	0.00435013977689738\\
230	0.0043502895398467\\
231	0.00435044210124015\\
232	0.00435059751341458\\
233	0.00435075582968171\\
234	0.00435091710434584\\
235	0.00435108139272219\\
236	0.00435124875115522\\
237	0.00435141923703743\\
238	0.00435159290882859\\
239	0.00435176982607499\\
240	0.00435195004942941\\
241	0.00435213364067103\\
242	0.00435232066272596\\
243	0.00435251117968812\\
244	0.0043527052568402\\
245	0.00435290296067521\\
246	0.00435310435891838\\
247	0.00435330952054942\\
248	0.00435351851582486\\
249	0.00435373141630114\\
250	0.00435394829485812\\
251	0.00435416922572236\\
252	0.0043543942844916\\
253	0.0043546235481589\\
254	0.00435485709513777\\
255	0.00435509500528721\\
256	0.00435533735993752\\
257	0.0043555842419162\\
258	0.00435583573557455\\
259	0.0043560919268145\\
260	0.00435635290311574\\
261	0.00435661875356367\\
262	0.00435688956887727\\
263	0.00435716544143785\\
264	0.0043574464653178\\
265	0.00435773273631016\\
266	0.00435802435195835\\
267	0.00435832141158653\\
268	0.00435862401633014\\
269	0.00435893226916712\\
270	0.00435924627494958\\
271	0.00435956614043557\\
272	0.00435989197432189\\
273	0.00436022388727688\\
274	0.00436056199197369\\
275	0.00436090640312437\\
276	0.00436125723751402\\
277	0.00436161461403568\\
278	0.00436197865372554\\
279	0.00436234947979873\\
280	0.00436272721768567\\
281	0.00436311199506868\\
282	0.00436350394191926\\
283	0.00436390319053608\\
284	0.00436430987558299\\
285	0.00436472413412836\\
286	0.00436514610568424\\
287	0.00436557593224635\\
288	0.00436601375833513\\
289	0.00436645973103691\\
290	0.00436691400004596\\
291	0.00436737671770755\\
292	0.00436784803906134\\
293	0.00436832812188603\\
294	0.0043688171267447\\
295	0.00436931521703131\\
296	0.00436982255901841\\
297	0.00437033932190591\\
298	0.00437086567787115\\
299	0.00437140180212078\\
300	0.00437194787294426\\
301	0.00437250407176894\\
302	0.00437307058321792\\
303	0.00437364759516981\\
304	0.00437423529882131\\
305	0.00437483388875333\\
306	0.0043754435630001\\
307	0.00437606452312268\\
308	0.00437669697428681\\
309	0.00437734112534621\\
310	0.00437799718893176\\
311	0.00437866538154727\\
312	0.0043793459236733\\
313	0.00438003903987961\\
314	0.00438074495894785\\
315	0.00438146391400548\\
316	0.00438219614267317\\
317	0.0043829418872268\\
318	0.00438370139477653\\
319	0.00438447491746537\\
320	0.00438526271268933\\
321	0.00438606504334291\\
322	0.00438688217809268\\
323	0.00438771439168233\\
324	0.0043885619652747\\
325	0.00438942518683345\\
326	0.00439030435155112\\
327	0.004391199762328\\
328	0.00439211173030835\\
329	0.00439304057548008\\
330	0.00439398662734469\\
331	0.00439495022566533\\
332	0.00439593172129862\\
333	0.00439693147711934\\
334	0.00439794986904292\\
335	0.00439898728715272\\
336	0.00440004413693626\\
337	0.00440112084063253\\
338	0.00440221783869206\\
339	0.00440333559135105\\
340	0.0044044745803329\\
341	0.00440563531073312\\
342	0.00440681831329135\\
343	0.00440802414769152\\
344	0.00440925340879721\\
345	0.00441050674127023\\
346	0.00441178487748705\\
347	0.00441308861159611\\
348	0.00441441859668612\\
349	0.00441577537239013\\
350	0.00441715949007091\\
351	0.00441857151312482\\
352	0.00442001201729709\\
353	0.00442148159100886\\
354	0.00442298083569901\\
355	0.00442451036618554\\
356	0.00442607081104137\\
357	0.00442766281297806\\
358	0.00442928702924466\\
359	0.00443094413204339\\
360	0.00443263480896246\\
361	0.00443435976342743\\
362	0.00443611971517158\\
363	0.00443791540072674\\
364	0.00443974757393486\\
365	0.00444161700648187\\
366	0.00444352448845459\\
367	0.00444547082892134\\
368	0.0044474568565372\\
369	0.0044494834201751\\
370	0.00445155138958286\\
371	0.00445366165606704\\
372	0.00445581513320393\\
373	0.00445801275757793\\
374	0.00446025548954678\\
375	0.00446254431403404\\
376	0.00446488024134691\\
377	0.00446726430801856\\
378	0.00446969757767323\\
379	0.00447218114191075\\
380	0.00447471612120721\\
381	0.00447730366582724\\
382	0.00447994495674243\\
383	0.0044826412065479\\
384	0.00448539366036974\\
385	0.00448820359675243\\
386	0.0044910723285143\\
387	0.00449400120355786\\
388	0.00449699160561887\\
389	0.00450004495493866\\
390	0.00450316270884218\\
391	0.00450634636220972\\
392	0.00450959744784129\\
393	0.00451291753673766\\
394	0.00451630823837839\\
395	0.00451977120114798\\
396	0.00452330811301538\\
397	0.00452692070170608\\
398	0.00453061072922442\\
399	0.00453437998069945\\
400	0.00453823027231605\\
401	0.00454216346841403\\
402	0.00454618148311681\\
403	0.00455028628203201\\
404	0.00455447988402687\\
405	0.00455876436308343\\
406	0.00456314185023713\\
407	0.004567614535605\\
408	0.00457218467050823\\
409	0.00457685456969654\\
410	0.00458162661367982\\
411	0.00458650325117443\\
412	0.0045914870016723\\
413	0.00459658045814428\\
414	0.00460178628988913\\
415	0.0046071072455413\\
416	0.00461254615625144\\
417	0.00461810593905845\\
418	0.00462378960047169\\
419	0.0046296002402883\\
420	0.00463554105567392\\
421	0.00464161534554071\\
422	0.0046478265152618\\
423	0.00465417808175562\\
424	0.00466067367894573\\
425	0.00466731706352769\\
426	0.00467411212094043\\
427	0.00468106287215399\\
428	0.00468817348186214\\
429	0.00469544826687676\\
430	0.00470289170423734\\
431	0.00471050843987783\\
432	0.00471830329789414\\
433	0.00472628129045722\\
434	0.00473444762842076\\
435	0.00474280773267571\\
436	0.00475136724630788\\
437	0.00476013204761824\\
438	0.00476910826406953\\
439	0.00477830228722649\\
440	0.0047877207887615\\
441	0.00479737073760044\\
442	0.00480725941828665\\
443	0.00481739445064306\\
444	0.00482778381081094\\
445	0.00483843585374646\\
446	0.00484935933727037\\
447	0.00486056344780547\\
448	0.00487205782786032\\
449	0.00488385260528553\\
450	0.00489595842442378\\
451	0.00490838647926189\\
452	0.00492114854865773\\
453	0.00493425703369915\\
454	0.00494772499723305\\
455	0.00496156620557198\\
456	0.00497579517234968\\
457	0.00499042720444371\\
458	0.00500547844981828\\
459	0.00502096594705182\\
460	0.005036907676203\\
461	0.00505332261052375\\
462	0.00507023076834851\\
463	0.00508765326425881\\
464	0.00510561235833505\\
465	0.00512413150199129\\
466	0.00514323537859757\\
467	0.0051629499370455\\
468	0.00518330241737538\\
469	0.00520432137194165\\
470	0.00522603670084335\\
471	0.0052484797701117\\
472	0.00527168384025851\\
473	0.00529568554045985\\
474	0.00532052975210367\\
475	0.00534628267470183\\
476	0.00537298517324241\\
477	0.00540067823671829\\
478	0.005429401877641\\
479	0.00545919466390937\\
480	0.00548981733005028\\
481	0.00552129955298564\\
482	0.00555367068432858\\
483	0.00558695979453451\\
484	0.00562119644582296\\
485	0.00565643318442142\\
486	0.00569293098925691\\
487	0.00573071750642869\\
488	0.00576981032505155\\
489	0.00581021341315327\\
490	0.00585191298560103\\
491	0.00589487354705255\\
492	0.00593902316472585\\
493	0.00598423743188394\\
494	0.00603033170296704\\
495	0.00607704681967084\\
496	0.00612402716256964\\
497	0.0061707902296417\\
498	0.00621667981092483\\
499	0.00626077931899491\\
500	0.00629986903605491\\
501	0.00633321627267928\\
502	0.00636114201732931\\
503	0.00638521509377619\\
504	0.00640821620985371\\
505	0.00643059378635421\\
506	0.00645283186455397\\
507	0.00647526889657669\\
508	0.00649805406722564\\
509	0.0065212637067607\\
510	0.00654494242954996\\
511	0.00656911485166403\\
512	0.00659379925237844\\
513	0.00661901035700388\\
514	0.00664476133373063\\
515	0.0066710650495984\\
516	0.00669793437851615\\
517	0.0067253824797126\\
518	0.00675342297778298\\
519	0.00678207008594943\\
520	0.00681133873692882\\
521	0.00684124472623264\\
522	0.00687180488058706\\
523	0.00690303726415001\\
524	0.00693496143280232\\
525	0.00696759874906755\\
526	0.00700097277136572\\
527	0.00703510972972449\\
528	0.00707003908897031\\
529	0.00710579415587524\\
530	0.00714241253002996\\
531	0.00717823387096147\\
532	0.00721273677147728\\
533	0.00724682109888564\\
534	0.00728132396461994\\
535	0.00731634289462122\\
536	0.00735194707806507\\
537	0.00738814406559645\\
538	0.00742493702546428\\
539	0.00746232678336369\\
540	0.00750031295800742\\
541	0.00753889396747831\\
542	0.00757806687614832\\
543	0.00761782721299181\\
544	0.00765816875730519\\
545	0.00769908328917873\\
546	0.00774056029745439\\
547	0.00778258663495929\\
548	0.00782514610529904\\
549	0.00786821895268939\\
550	0.00791178119219577\\
551	0.00795580361922691\\
552	0.00800025003764972\\
553	0.00804363618546738\\
554	0.00808620366337529\\
555	0.00812910677813416\\
556	0.00817240401823542\\
557	0.00821607813701697\\
558	0.00826010969053816\\
559	0.00830447822831884\\
560	0.0083491625665264\\
561	0.0083941411104164\\
562	0.00843939212066833\\
563	0.00848489356497313\\
564	0.00853045416626427\\
565	0.00857590279196591\\
566	0.0086216403638569\\
567	0.00866765938108735\\
568	0.00871393573141338\\
569	0.00876044307752645\\
570	0.00880715274930527\\
571	0.00885403364441831\\
572	0.00890105214171657\\
573	0.00894817203305217\\
574	0.00899535448063906\\
575	0.00904255800890197\\
576	0.00908973854201728\\
577	0.00913684950113329\\
578	0.00918384197868253\\
579	0.00923066501141121\\
580	0.00927726597891581\\
581	0.00932359116078639\\
582	0.00936958649310202\\
583	0.00941519857412798\\
584	0.00946037597948012\\
585	0.00950507095777015\\
586	0.00954924158550055\\
587	0.00959285445497631\\
588	0.00963588792409171\\
589	0.00967833579817709\\
590	0.00972021084271539\\
591	0.00976138933978722\\
592	0.00980171928775089\\
593	0.00984102135019415\\
594	0.00987905595166784\\
595	0.00991543381058343\\
596	0.00994937493726701\\
597	0.00997906286423442\\
598	0.0099999191923403\\
599	0\\
600	0\\
};
\end{axis}
\end{tikzpicture}% 
  \caption{Discrete Time}
\end{subfigure}\\
\vspace{1cm}
\begin{subfigure}{.45\linewidth}
  \centering
  \setlength\figureheight{\linewidth} 
  \setlength\figurewidth{\linewidth}
  \tikzsetnextfilename{dp_colorbar/dp_cts_nFPC_z8}
  % This file was created by matlab2tikz.
%
%The latest updates can be retrieved from
%  http://www.mathworks.com/matlabcentral/fileexchange/22022-matlab2tikz-matlab2tikz
%where you can also make suggestions and rate matlab2tikz.
%
\definecolor{mycolor1}{rgb}{1.00000,0.00000,1.00000}%
%
\begin{tikzpicture}[trim axis left, trim axis right]

\begin{axis}[%
width=\figurewidth,
height=\figureheight,
at={(0\figurewidth,0\figureheight)},
scale only axis,
every outer x axis line/.append style={black},
every x tick label/.append style={font=\color{black}},
xmin=0,
xmax=100,
xlabel={Time},
every outer y axis line/.append style={black},
every y tick label/.append style={font=\color{black}},
ymin=0,
ymax=0.015,
%ylabel={Depth $\delta^-$},
axis background/.style={fill=white},
axis x line*=bottom,
axis y line*=left,
yticklabel style={
        /pgf/number format/fixed,
        /pgf/number format/precision=3
},
scaled y ticks=false,
legend style={legend cell align=left,align=left,draw=black,font=\footnotesize, at={(0.98,0.02)},anchor=south east},
every axis legend/.code={\renewcommand\addlegendentry[2][]{}}  %ignore legend locally
]
\addplot [color=green,dashed]
  table[row sep=crcr]{%
0.01	0.00879904401680774\\
1.01	0.00878916708833785\\
2.01	0.0087787961888384\\
3.01	0.00876790513602132\\
4.01	0.00875646626029204\\
5.01	0.00874445029188437\\
6.01	0.00873182622823632\\
7.01	0.00871856117564894\\
8.01	0.00870462015755683\\
9.01	0.00868996587971709\\
10.01	0.00867455844037774\\
11.01	0.00865835497124509\\
12.01	0.00864130919332668\\
13.01	0.00862337087144434\\
14.01	0.00860448515415128\\
15.01	0.0085845917950726\\
16.01	0.008563624272698\\
17.01	0.00854150886751429\\
18.01	0.00851816383398183\\
19.01	0.00849349895135208\\
20.01	0.00846741536914966\\
21.01	0.00843980510464505\\
22.01	0.00841055053034011\\
23.01	0.00837952414414822\\
24.01	0.00834658885548674\\
25.01	0.00831159914291629\\
26.01	0.00827439088654129\\
27.01	0.00823474865363261\\
28.01	0.00819242029290907\\
29.01	0.00814711174474293\\
30.01	0.00809847572556894\\
31.01	0.008046098121038\\
32.01	0.00798948011169459\\
33.01	0.00792801450311603\\
34.01	0.00786095412805441\\
35.01	0.00778736932341551\\
36.01	0.00770609253118292\\
37.01	0.00761831095217872\\
38.01	0.00752567132623096\\
39.01	0.00742787799461891\\
40.01	0.00732461307718008\\
41.01	0.00721553022557604\\
42.01	0.00710024419862604\\
43.01	0.00697831362580073\\
44.01	0.00684921265854122\\
45.01	0.00671228646638744\\
46.01	0.00656673627739562\\
47.01	0.00641161044007972\\
48.01	0.00624575996949978\\
49.01	0.00611007732321344\\
50.01	0.00601299750052631\\
51.01	0.00591061552152229\\
52.01	0.00580267052463306\\
53.01	0.00568894582868561\\
54.01	0.00556930583078947\\
55.01	0.00544375195560288\\
56.01	0.00531250684963002\\
57.01	0.00517614047360847\\
58.01	0.00503575843718278\\
59.01	0.0048934475279031\\
60.01	0.00475210798939968\\
61.01	0.00461462362621046\\
62.01	0.00448473547667558\\
63.01	0.00435644228817179\\
64.01	0.00422613481601493\\
65.01	0.00409451963672095\\
66.01	0.00396245028281321\\
67.01	0.00383091501052414\\
68.01	0.00370099265560547\\
69.01	0.00357375159677462\\
70.01	0.00344997114751124\\
71.01	0.00333076902272894\\
72.01	0.00321880879105804\\
73.01	0.00310933418509031\\
74.01	0.00300092191586959\\
75.01	0.00289356421969468\\
76.01	0.00278657362826304\\
77.01	0.00267869253183736\\
78.01	0.00256960840121469\\
79.01	0.00245926901255094\\
80.01	0.00234772625485614\\
81.01	0.00223519869647231\\
82.01	0.00212189178782957\\
83.01	0.00200792736130027\\
84.01	0.00189327625168117\\
85.01	0.00177765665873484\\
86.01	0.00166041252095138\\
87.01	0.00154096936677342\\
88.01	0.00141928021253589\\
89.01	0.00129545664766734\\
90.01	0.00116963756502902\\
91.01	0.00104197157482483\\
92.01	0.000912617061273989\\
93.01	0.000781744839962994\\
94.01	0.0006495451184769\\
95.01	0.000516241746188426\\
96.01	0.000382118196883106\\
97.01	0.00024756148145675\\
98.01	0.000113204229188223\\
99.01	2.04676060806437e-05\\
99.02	1.98506175098923e-05\\
99.03	1.92413562266989e-05\\
99.04	1.86398922933494e-05\\
99.05	1.804629641931e-05\\
99.06	1.74606399672653e-05\\
99.07	1.68829949592029e-05\\
99.08	1.6313434082562e-05\\
99.09	1.57520306964323e-05\\
99.1	1.51988588378213e-05\\
99.11	1.46539932279633e-05\\
99.12	1.41175092787053e-05\\
99.13	1.35894830989459e-05\\
99.14	1.30699915011235e-05\\
99.15	1.25591120077907e-05\\
99.16	1.20569228582185e-05\\
99.17	1.15635030150888e-05\\
99.18	1.10789321712389e-05\\
99.19	1.06032981187169e-05\\
99.2	1.01366899886816e-05\\
99.21	9.67919775133295e-06\\
99.22	9.23091222374516e-06\\
99.23	8.79192507778105e-06\\
99.24	8.36232884808032e-06\\
99.25	7.94221694011034e-06\\
99.26	7.53168363830208e-06\\
99.27	7.13082411427438e-06\\
99.28	6.73973443509994e-06\\
99.29	6.3585115716875e-06\\
99.3	5.98725340720911e-06\\
99.31	5.62605874563499e-06\\
99.32	5.27502732032906e-06\\
99.33	4.93425980272778e-06\\
99.34	4.60385781110052e-06\\
99.35	4.28392391940528e-06\\
99.36	3.97456166620347e-06\\
99.37	3.67587556367177e-06\\
99.38	3.38797110669906e-06\\
99.39	3.11095478206652e-06\\
99.4	2.84493407771459e-06\\
99.41	2.59001749209654e-06\\
99.42	2.34631454362755e-06\\
99.43	2.11393578021697e-06\\
99.44	1.89299278887875e-06\\
99.45	1.68359820545798e-06\\
99.46	1.48586572441996e-06\\
99.47	1.29991010874159e-06\\
99.48	1.12584719991031e-06\\
99.49	9.63793927985512e-07\\
99.5	8.13868321781347e-07\\
99.51	6.76189519131093e-07\\
99.52	5.50877777248659e-07\\
99.53	4.38054483194172e-07\\
99.54	3.37842164419358e-07\\
99.55	2.50364499436093e-07\\
99.56	1.7574632856128e-07\\
99.57	1.14113664783158e-07\\
99.58	6.55937047039368e-08\\
99.59	3.03148396090663e-08\\
99.6	8.40666662844936e-09\\
99.61	0\\
99.62	0\\
99.63	0\\
99.64	0\\
99.65	0\\
99.66	0\\
99.67	0\\
99.68	0\\
99.69	0\\
99.7	0\\
99.71	0\\
99.72	0\\
99.73	0\\
99.74	0\\
99.75	0\\
99.76	0\\
99.77	0\\
99.78	0\\
99.79	0\\
99.8	0\\
99.81	0\\
99.82	0\\
99.83	0\\
99.84	0\\
99.85	0\\
99.86	0\\
99.87	0\\
99.88	0\\
99.89	0\\
99.9	0\\
99.91	0\\
99.92	0\\
99.93	0\\
99.94	0\\
99.95	0\\
99.96	0\\
99.97	0\\
99.98	0\\
99.99	0\\
100	0\\
};
\addlegendentry{$q=-4$};

\addplot [color=mycolor1,dashed]
  table[row sep=crcr]{%
0.01	0.01\\
1.01	0.01\\
2.01	0.01\\
3.01	0.01\\
4.01	0.01\\
5.01	0.01\\
6.01	0.01\\
7.01	0.01\\
8.01	0.01\\
9.01	0.01\\
10.01	0.01\\
11.01	0.01\\
12.01	0.01\\
13.01	0.01\\
14.01	0.01\\
15.01	0.01\\
16.01	0.01\\
17.01	0.01\\
18.01	0.01\\
19.01	0.01\\
20.01	0.01\\
21.01	0.01\\
22.01	0.01\\
23.01	0.01\\
24.01	0.01\\
25.01	0.01\\
26.01	0.01\\
27.01	0.01\\
28.01	0.01\\
29.01	0.01\\
30.01	0.01\\
31.01	0.01\\
32.01	0.01\\
33.01	0.01\\
34.01	0.01\\
35.01	0.01\\
36.01	0.01\\
37.01	0.01\\
38.01	0.01\\
39.01	0.01\\
40.01	0.01\\
41.01	0.01\\
42.01	0.01\\
43.01	0.01\\
44.01	0.01\\
45.01	0.01\\
46.01	0.01\\
47.01	0.01\\
48.01	0.01\\
49.01	0.00995806603276508\\
50.01	0.00986679348671141\\
51.01	0.0097696679127076\\
52.01	0.00966610471935848\\
53.01	0.0095554243848595\\
54.01	0.00943682937641714\\
55.01	0.00930937366575347\\
56.01	0.00917192202280119\\
57.01	0.00902309507088274\\
58.01	0.00886119432678425\\
59.01	0.00868409687832849\\
60.01	0.00848910618133506\\
61.01	0.00827281314552082\\
62.01	0.00803123443720172\\
63.01	0.00777023253276142\\
64.01	0.00749272709371373\\
65.01	0.0071972554037051\\
66.01	0.0068822478259438\\
67.01	0.00654605684832105\\
68.01	0.00618702135861113\\
69.01	0.00580360955010393\\
70.01	0.00539472343790348\\
71.01	0.00495878700373076\\
72.01	0.00459234396243271\\
73.01	0.00437901367131237\\
74.01	0.00416850570978012\\
75.01	0.00396572764659518\\
76.01	0.00377778805885348\\
77.01	0.00359727605438995\\
78.01	0.00341468896082242\\
79.01	0.00323034930536865\\
80.01	0.00304353727266926\\
81.01	0.00285483719727066\\
82.01	0.00266617877883341\\
83.01	0.00248016672353796\\
84.01	0.00230031846927878\\
85.01	0.00213137185966049\\
86.01	0.0019771061048414\\
87.01	0.00182511621036325\\
88.01	0.00167085897405416\\
89.01	0.00151419826491386\\
90.01	0.00135570882384954\\
91.01	0.00119608501287654\\
92.01	0.00103612616572766\\
93.01	0.000876729625283948\\
94.01	0.000718872071917943\\
95.01	0.000563572918176667\\
96.01	0.000411831375856974\\
97.01	0.000264532650272513\\
98.01	0.000122320814127108\\
99.01	2.08868531277918e-05\\
99.02	2.02551760214512e-05\\
99.03	1.96315401111164e-05\\
99.04	1.90160129319994e-05\\
99.05	1.84086626507998e-05\\
99.06	1.78095580716676e-05\\
99.07	1.7218768642243e-05\\
99.08	1.66363644597527e-05\\
99.09	1.60624162771558e-05\\
99.1	1.54969955093523e-05\\
99.11	1.49401742394594e-05\\
99.12	1.43920252251226e-05\\
99.13	1.38526219049216e-05\\
99.14	1.33220384048085e-05\\
99.15	1.28003495446158e-05\\
99.16	1.22876308446349e-05\\
99.17	1.1783958532248e-05\\
99.18	1.12894095486218e-05\\
99.19	1.08040673361641e-05\\
99.2	1.03280183761673e-05\\
99.21	9.86134997679744e-06\\
99.22	9.40415028095366e-06\\
99.23	8.95650827419971e-06\\
99.24	8.51851379275563e-06\\
99.25	8.09025753160421e-06\\
99.26	7.67183105262158e-06\\
99.27	7.26332679283626e-06\\
99.28	6.86483807273673e-06\\
99.29	6.47645910466059e-06\\
99.3	6.09828500128598e-06\\
99.31	5.73041178416993e-06\\
99.32	5.37293639239766e-06\\
99.33	5.02595669130655e-06\\
99.34	4.68957148128807e-06\\
99.35	4.36388050667827e-06\\
99.36	4.04898446473498e-06\\
99.37	3.74498501470345e-06\\
99.38	3.45198478696705e-06\\
99.39	3.17008739210423e-06\\
99.4	2.89939742239079e-06\\
99.41	2.64002046114487e-06\\
99.42	2.39206309212044e-06\\
99.43	2.15563290903095e-06\\
99.44	1.93083852514583e-06\\
99.45	1.71778958298931e-06\\
99.46	1.51659676413639e-06\\
99.47	1.32737179910601e-06\\
99.48	1.15022747734263e-06\\
99.49	9.85277657314029e-07\\
99.5	8.32637276701118e-07\\
99.51	6.92422362690015e-07\\
99.52	5.64750042368264e-07\\
99.53	4.49738553228579e-07\\
99.54	3.47507253785351e-07\\
99.55	2.58176634277893e-07\\
99.56	1.81868327514198e-07\\
99.57	1.18705119791021e-07\\
99.58	6.88109619700894e-08\\
99.59	3.23109806184968e-08\\
99.6	9.33148930348793e-09\\
99.61	0\\
99.62	0\\
99.63	0\\
99.64	0\\
99.65	0\\
99.66	0\\
99.67	0\\
99.68	0\\
99.69	0\\
99.7	0\\
99.71	0\\
99.72	0\\
99.73	0\\
99.74	0\\
99.75	0\\
99.76	0\\
99.77	0\\
99.78	0\\
99.79	0\\
99.8	0\\
99.81	0\\
99.82	0\\
99.83	0\\
99.84	0\\
99.85	0\\
99.86	0\\
99.87	0\\
99.88	0\\
99.89	0\\
99.9	0\\
99.91	0\\
99.92	0\\
99.93	0\\
99.94	0\\
99.95	0\\
99.96	0\\
99.97	0\\
99.98	0\\
99.99	0\\
100	0\\
};
\addlegendentry{$q=-3$};

\addplot [color=red,dashed]
  table[row sep=crcr]{%
0.01	0.01\\
1.01	0.01\\
2.01	0.01\\
3.01	0.01\\
4.01	0.01\\
5.01	0.01\\
6.01	0.01\\
7.01	0.01\\
8.01	0.01\\
9.01	0.01\\
10.01	0.01\\
11.01	0.01\\
12.01	0.01\\
13.01	0.01\\
14.01	0.01\\
15.01	0.01\\
16.01	0.01\\
17.01	0.01\\
18.01	0.01\\
19.01	0.01\\
20.01	0.01\\
21.01	0.01\\
22.01	0.01\\
23.01	0.01\\
24.01	0.01\\
25.01	0.01\\
26.01	0.01\\
27.01	0.01\\
28.01	0.01\\
29.01	0.01\\
30.01	0.01\\
31.01	0.01\\
32.01	0.01\\
33.01	0.01\\
34.01	0.01\\
35.01	0.01\\
36.01	0.01\\
37.01	0.01\\
38.01	0.01\\
39.01	0.01\\
40.01	0.01\\
41.01	0.01\\
42.01	0.01\\
43.01	0.01\\
44.01	0.01\\
45.01	0.01\\
46.01	0.01\\
47.01	0.01\\
48.01	0.01\\
49.01	0.01\\
50.01	0.01\\
51.01	0.01\\
52.01	0.01\\
53.01	0.01\\
54.01	0.01\\
55.01	0.01\\
56.01	0.01\\
57.01	0.01\\
58.01	0.01\\
59.01	0.01\\
60.01	0.01\\
61.01	0.01\\
62.01	0.01\\
63.01	0.01\\
64.01	0.01\\
65.01	0.01\\
66.01	0.01\\
67.01	0.01\\
68.01	0.01\\
69.01	0.01\\
70.01	0.01\\
71.01	0.01\\
72.01	0.00990057041713054\\
73.01	0.00962270905049018\\
74.01	0.00931729974228617\\
75.01	0.0089779304749674\\
76.01	0.00859677687383986\\
77.01	0.00818162924635808\\
78.01	0.00774136108451931\\
79.01	0.00727472771568591\\
80.01	0.00678140382444221\\
81.01	0.0062596052458434\\
82.01	0.00570627218954784\\
83.01	0.00511790658247763\\
84.01	0.00449042694295134\\
85.01	0.00381895042717371\\
86.01	0.00332077846995116\\
87.01	0.00307372686830099\\
88.01	0.00283493893904919\\
89.01	0.00258799776351812\\
90.01	0.00233284257025338\\
91.01	0.00207030359555413\\
92.01	0.00180161100528257\\
93.01	0.00152852324252531\\
94.01	0.00125350033934508\\
95.01	0.000979957596255637\\
96.01	0.000712585247351627\\
97.01	0.000457646608820146\\
98.01	0.00022337038221871\\
99.01	4.39236936352807e-05\\
99.02	4.27894660898227e-05\\
99.03	4.16677340412432e-05\\
99.04	4.05585809066453e-05\\
99.05	3.94620907495404e-05\\
99.06	3.83783482850279e-05\\
99.07	3.73074388850345e-05\\
99.08	3.62494485835439e-05\\
99.09	3.52044640818897e-05\\
99.1	3.41725727540567e-05\\
99.11	3.31538626520134e-05\\
99.12	3.21484225110687e-05\\
99.13	3.11563417552743e-05\\
99.14	3.01777105028277e-05\\
99.15	2.92126195715094e-05\\
99.16	2.82611604841623e-05\\
99.17	2.73234254741638e-05\\
99.18	2.63995074909502e-05\\
99.19	2.54895002014359e-05\\
99.2	2.45934979939289e-05\\
99.21	2.37115959836436e-05\\
99.22	2.28438900182325e-05\\
99.23	2.19904766833416e-05\\
99.24	2.11514533081832e-05\\
99.25	2.03269179711221e-05\\
99.26	1.95169695052736e-05\\
99.27	1.87217075041362e-05\\
99.28	1.79412323272118e-05\\
99.29	1.71756451056593e-05\\
99.3	1.64250477479484e-05\\
99.31	1.56895429455264e-05\\
99.32	1.49692341784944e-05\\
99.33	1.42642257212871e-05\\
99.34	1.35746226483605e-05\\
99.35	1.29005308398839e-05\\
99.36	1.22420569874312e-05\\
99.37	1.15993085996729e-05\\
99.38	1.09723940080689e-05\\
99.39	1.03614226602163e-05\\
99.4	9.76651714497574e-06\\
99.41	9.18780102560947e-06\\
99.42	8.62539884664143e-06\\
99.43	8.07943614066463e-06\\
99.44	7.55003943518281e-06\\
99.45	7.03733625940541e-06\\
99.46	6.5414551510528e-06\\
99.47	6.06252566313741e-06\\
99.48	5.60067837072735e-06\\
99.49	5.15604487768065e-06\\
99.5	4.72875782336381e-06\\
99.51	4.3189508893253e-06\\
99.52	3.92675880595342e-06\\
99.53	3.5523173590752e-06\\
99.54	3.19576339652752e-06\\
99.55	2.85723483467448e-06\\
99.56	2.53687066486923e-06\\
99.57	2.23481095987417e-06\\
99.58	1.95119688019101e-06\\
99.59	1.68617068034213e-06\\
99.6	1.4398757150879e-06\\
99.61	1.2124564455207e-06\\
99.62	1.00405963195105e-06\\
99.63	8.14835682361875e-07\\
99.64	6.449361741271e-07\\
99.65	4.94513860235801e-07\\
99.66	3.63722675407116e-07\\
99.67	2.52717742073305e-07\\
99.68	1.61655376246586e-07\\
99.69	9.069309325066e-08\\
99.7	3.99896132857042e-08\\
99.71	9.70486685111793e-09\\
99.72	0\\
99.73	0\\
99.74	0\\
99.75	0\\
99.76	0\\
99.77	0\\
99.78	0\\
99.79	0\\
99.8	0\\
99.81	0\\
99.82	0\\
99.83	0\\
99.84	0\\
99.85	0\\
99.86	0\\
99.87	0\\
99.88	0\\
99.89	0\\
99.9	0\\
99.91	0\\
99.92	0\\
99.93	0\\
99.94	0\\
99.95	0\\
99.96	0\\
99.97	0\\
99.98	0\\
99.99	0\\
100	0\\
};
\addlegendentry{$q=-2$};

\addplot [color=blue,dashed]
  table[row sep=crcr]{%
0.01	0.01\\
1.01	0.01\\
2.01	0.01\\
3.01	0.01\\
4.01	0.01\\
5.01	0.01\\
6.01	0.01\\
7.01	0.01\\
8.01	0.01\\
9.01	0.01\\
10.01	0.01\\
11.01	0.01\\
12.01	0.01\\
13.01	0.01\\
14.01	0.01\\
15.01	0.01\\
16.01	0.01\\
17.01	0.01\\
18.01	0.01\\
19.01	0.01\\
20.01	0.01\\
21.01	0.01\\
22.01	0.01\\
23.01	0.01\\
24.01	0.01\\
25.01	0.01\\
26.01	0.01\\
27.01	0.01\\
28.01	0.01\\
29.01	0.01\\
30.01	0.01\\
31.01	0.01\\
32.01	0.01\\
33.01	0.01\\
34.01	0.01\\
35.01	0.01\\
36.01	0.01\\
37.01	0.01\\
38.01	0.01\\
39.01	0.01\\
40.01	0.01\\
41.01	0.01\\
42.01	0.01\\
43.01	0.01\\
44.01	0.01\\
45.01	0.01\\
46.01	0.01\\
47.01	0.01\\
48.01	0.01\\
49.01	0.01\\
50.01	0.01\\
51.01	0.01\\
52.01	0.01\\
53.01	0.01\\
54.01	0.01\\
55.01	0.01\\
56.01	0.01\\
57.01	0.01\\
58.01	0.01\\
59.01	0.01\\
60.01	0.01\\
61.01	0.01\\
62.01	0.01\\
63.01	0.01\\
64.01	0.01\\
65.01	0.01\\
66.01	0.01\\
67.01	0.01\\
68.01	0.01\\
69.01	0.01\\
70.01	0.01\\
71.01	0.01\\
72.01	0.01\\
73.01	0.01\\
74.01	0.01\\
75.01	0.01\\
76.01	0.01\\
77.01	0.01\\
78.01	0.01\\
79.01	0.01\\
80.01	0.01\\
81.01	0.01\\
82.01	0.01\\
83.01	0.01\\
84.01	0.01\\
85.01	0.01\\
86.01	0.00977996835059351\\
87.01	0.0092759272733409\\
88.01	0.00873565013865457\\
89.01	0.00817534486985543\\
90.01	0.00759394957702919\\
91.01	0.00698952267376253\\
92.01	0.00635975923032467\\
93.01	0.00570178861138295\\
94.01	0.00501174575093118\\
95.01	0.00428383487608855\\
96.01	0.00351239596837763\\
97.01	0.00269284670310579\\
98.01	0.00182013337470615\\
99.01	0.000891311399442173\\
99.02	0.000881920709532172\\
99.03	0.0008725346602031\\
99.04	0.000863153327663143\\
99.05	0.000853776788903485\\
99.06	0.000844405121708091\\
99.07	0.000835038404663648\\
99.08	0.000825676717169661\\
99.09	0.000816320139448719\\
99.1	0.000806968752556916\\
99.11	0.000797622638394458\\
99.12	0.000788281879716437\\
99.13	0.000778946560143766\\
99.14	0.000769616764174317\\
99.15	0.000760292577194233\\
99.16	0.000750974085489424\\
99.17	0.000741661376257259\\
99.18	0.000732354537618451\\
99.19	0.000723053658629137\\
99.2	0.000713758829293178\\
99.21	0.000704470140574644\\
99.22	0.000695187684410533\\
99.23	0.000685911553723684\\
99.24	0.000676641842435923\\
99.25	0.000667378645481434\\
99.26	0.00065812205882035\\
99.27	0.000648872179452586\\
99.28	0.000639629105431912\\
99.29	0.000630392935880257\\
99.3	0.000621163771002279\\
99.31	0.000611941712100177\\
99.32	0.000602726861588772\\
99.33	0.000593519323010844\\
99.34	0.000584319201052751\\
99.35	0.000575126601560311\\
99.36	0.000565941631554984\\
99.37	0.00055676439925033\\
99.38	0.00054759501406877\\
99.39	0.000538433586658444\\
99.4	0.000529280228901823\\
99.41	0.000520135053933318\\
99.42	0.000510998176157183\\
99.43	0.000501869711265774\\
99.44	0.000492749776258126\\
99.45	0.000483638489458892\\
99.46	0.000474535970537619\\
99.47	0.000465442340528399\\
99.48	0.000456357721849881\\
99.49	0.000447282238325672\\
99.5	0.000438216015205108\\
99.51	0.000429159179184449\\
99.52	0.000420111858428454\\
99.53	0.000411074182592397\\
99.54	0.000402046282844485\\
99.55	0.000393028291888746\\
99.56	0.000384020343988333\\
99.57	0.000375022574989311\\
99.58	0.000366035122344904\\
99.59	0.000357058125140244\\
99.6	0.000348091724117582\\
99.61	0.000339136061702052\\
99.62	0.000330191282027605\\
99.63	0.000321257530963142\\
99.64	0.000312334956139852\\
99.65	0.000303423706979116\\
99.66	0.000294523934720993\\
99.67	0.000285635792453304\\
99.68	0.000276759435141329\\
99.69	0.000267895019658133\\
99.7	0.000259042704815544\\
99.71	0.000250202651395793\\
99.72	0.000241375022183844\\
99.73	0.000232559984946454\\
99.74	0.000223757712161201\\
99.75	0.000214968378467508\\
99.76	0.000206192160702477\\
99.77	0.000197429237937504\\
99.78	0.000188679791515737\\
99.79	0.00017994400509036\\
99.8	0.000171222064663754\\
99.81	0.000162514158627553\\
99.82	0.000153820477803632\\
99.83	0.000145141215486051\\
99.84	0.000136476567483968\\
99.85	0.00012782673216559\\
99.86	0.000119191910503162\\
99.87	0.000110572306119041\\
99.88	0.000101968125332889\\
99.89	9.33795772100343e-05\\
99.9	8.4806873611008e-05\\
99.91	7.62502292423421e-05\\
99.92	6.77098617086289e-05\\
99.93	5.91859915659125e-05\\
99.94	5.06788423764483e-05\\
99.95	4.21886407648911e-05\\
99.96	3.37156164759312e-05\\
99.97	2.52600024334866e-05\\
99.98	1.68220348014392e-05\\
99.99	8.40195304604129e-06\\
100	0\\
};
\addlegendentry{$q=-1$};

\addplot [color=black,solid]
  table[row sep=crcr]{%
0.01	0\\
1.01	0\\
2.01	0\\
3.01	0\\
4.01	0\\
5.01	0\\
6.01	0\\
7.01	0\\
8.01	0\\
9.01	0\\
10.01	0\\
11.01	0\\
12.01	0\\
13.01	0\\
14.01	0\\
15.01	0\\
16.01	0\\
17.01	0\\
18.01	0\\
19.01	0\\
20.01	0\\
21.01	0\\
22.01	0\\
23.01	0\\
24.01	0\\
25.01	0\\
26.01	0\\
27.01	0\\
28.01	0\\
29.01	0\\
30.01	0\\
31.01	0\\
32.01	0\\
33.01	0\\
34.01	0\\
35.01	0\\
36.01	0\\
37.01	0\\
38.01	0\\
39.01	0\\
40.01	0\\
41.01	0\\
42.01	0\\
43.01	0\\
44.01	0\\
45.01	0\\
46.01	0\\
47.01	0\\
48.01	0\\
49.01	0\\
50.01	0\\
51.01	0\\
52.01	0\\
53.01	0\\
54.01	0\\
55.01	0\\
56.01	0\\
57.01	0\\
58.01	0\\
59.01	0\\
60.01	0\\
61.01	0\\
62.01	0\\
63.01	0\\
64.01	0\\
65.01	0\\
66.01	0\\
67.01	0\\
68.01	0\\
69.01	0\\
70.01	0\\
71.01	0\\
72.01	0\\
73.01	0\\
74.01	0\\
75.01	0\\
76.01	0\\
77.01	0\\
78.01	0\\
79.01	0\\
80.01	0\\
81.01	0\\
82.01	0\\
83.01	0\\
84.01	0\\
85.01	0\\
86.01	0\\
87.01	9.33319065741164e-05\\
88.01	0.00059435419498903\\
89.01	0.00114736171693115\\
90.01	0.00174962640650023\\
91.01	0.00238295213405392\\
92.01	0.00304664764442438\\
93.01	0.00374309473172899\\
94.01	0.00447679439357722\\
95.01	0.00525497275216101\\
96.01	0.00608691999137679\\
97.01	0.00697905984819611\\
98.01	0.00793791965569993\\
99.01	0.00896782660176875\\
99.02	0.00897835004419935\\
99.03	0.00898887365066564\\
99.04	0.00899939737136215\\
99.05	0.00900992115593992\\
99.06	0.00902044495349986\\
99.07	0.00903096871258595\\
99.08	0.00904149238117846\\
99.09	0.00905201590668712\\
99.1	0.009062539235944\\
99.11	0.00907306231519623\\
99.12	0.0090835850903821\\
99.13	0.00909410750691204\\
99.14	0.00910462950959158\\
99.15	0.00911515104261371\\
99.16	0.00912567204955099\\
99.17	0.00913619247334767\\
99.18	0.00914671225631155\\
99.19	0.00915723134010582\\
99.2	0.00916774966574069\\
99.21	0.00917826717356491\\
99.22	0.00918878380325717\\
99.23	0.00919929949381728\\
99.24	0.00920981418355733\\
99.25	0.00922032781009259\\
99.26	0.0092308403103323\\
99.27	0.00924135162047032\\
99.28	0.00925186167597559\\
99.29	0.00926237041158246\\
99.3	0.00927287776128084\\
99.31	0.00928338365830616\\
99.32	0.00929388803512921\\
99.33	0.00930439082344577\\
99.34	0.00931489195416604\\
99.35	0.00932539135740396\\
99.36	0.00933588896246625\\
99.37	0.00934638469784135\\
99.38	0.00935687849118809\\
99.39	0.00936737026932421\\
99.4	0.00937785995821467\\
99.41	0.00938834748295972\\
99.42	0.00939883276778278\\
99.43	0.00940931573601812\\
99.44	0.00941979631009832\\
99.45	0.00943027441154143\\
99.46	0.00944074996093798\\
99.47	0.00945122287793775\\
99.48	0.00946169308123622\\
99.49	0.00947216048856088\\
99.5	0.0094826250166572\\
99.51	0.00949308658127434\\
99.52	0.00950354509715069\\
99.53	0.00951400047799902\\
99.54	0.00952445263649142\\
99.55	0.00953490148424391\\
99.56	0.00954534693180085\\
99.57	0.00955578888861889\\
99.58	0.00956622726305078\\
99.59	0.00957666196232875\\
99.6	0.00958709289254763\\
99.61	0.0095975199586476\\
99.62	0.00960794306439661\\
99.63	0.00961836211237247\\
99.64	0.00962877700394458\\
99.65	0.00963918763925525\\
99.66	0.00964959391720069\\
99.67	0.00965999573541164\\
99.68	0.00967039299023348\\
99.69	0.0096807855767061\\
99.7	0.00969117338854325\\
99.71	0.00970155631812073\\
99.72	0.0097119342564558\\
99.73	0.00972230709318531\\
99.74	0.00973267471654341\\
99.75	0.00974303701333878\\
99.76	0.00975339386893135\\
99.77	0.00976374516720854\\
99.78	0.00977409079056095\\
99.79	0.00978443061985756\\
99.8	0.0097947645344203\\
99.81	0.00980509241199814\\
99.82	0.00981541412874054\\
99.83	0.00982572955917033\\
99.84	0.00983603857615593\\
99.85	0.00984634105088294\\
99.86	0.00985663685282512\\
99.87	0.00986692584971462\\
99.88	0.00987720790751152\\
99.89	0.0098874828903727\\
99.9	0.00989775066061986\\
99.91	0.00990801107870688\\
99.92	0.00991826400318625\\
99.93	0.00992850929067483\\
99.94	0.00993874679581858\\
99.95	0.00994897637125655\\
99.96	0.00995919786758387\\
99.97	0.0099694111333138\\
99.98	0.00997961601483888\\
99.99	0.0099898123563909\\
100	0.01\\
};
\addlegendentry{$q=0$};

\addplot [color=blue,solid]
  table[row sep=crcr]{%
0.01	0\\
1.01	0\\
2.01	0\\
3.01	0\\
4.01	0\\
5.01	0\\
6.01	0\\
7.01	0\\
8.01	0\\
9.01	0\\
10.01	0\\
11.01	0\\
12.01	0\\
13.01	0\\
14.01	0\\
15.01	0\\
16.01	0\\
17.01	0\\
18.01	0\\
19.01	0\\
20.01	0\\
21.01	0\\
22.01	0\\
23.01	0\\
24.01	0\\
25.01	0\\
26.01	0\\
27.01	0\\
28.01	0\\
29.01	0\\
30.01	0\\
31.01	0\\
32.01	0\\
33.01	0\\
34.01	0\\
35.01	0\\
36.01	0\\
37.01	0\\
38.01	0\\
39.01	0\\
40.01	0\\
41.01	0\\
42.01	0\\
43.01	0\\
44.01	0\\
45.01	0\\
46.01	0\\
47.01	0\\
48.01	0\\
49.01	0\\
50.01	0\\
51.01	0\\
52.01	0\\
53.01	0\\
54.01	0\\
55.01	0\\
56.01	0\\
57.01	0\\
58.01	0\\
59.01	0\\
60.01	0\\
61.01	0\\
62.01	0\\
63.01	0\\
64.01	0\\
65.01	0\\
66.01	0\\
67.01	0\\
68.01	0\\
69.01	0\\
70.01	0\\
71.01	0\\
72.01	0\\
73.01	0\\
74.01	0\\
75.01	0\\
76.01	0.000157971794062944\\
77.01	0.000497402840502613\\
78.01	0.000868732816274718\\
79.01	0.00127798997886026\\
80.01	0.00173267500751677\\
81.01	0.00223731746694362\\
82.01	0.00277948353736682\\
83.01	0.00336076721073516\\
84.01	0.00398508455809201\\
85.01	0.00465541316784714\\
86.01	0.0053750082596734\\
87.01	0.00605767964090931\\
88.01	0.00638058741292995\\
89.01	0.00669288600691645\\
90.01	0.0069952637881693\\
91.01	0.00730567053966473\\
92.01	0.00762610637409276\\
93.01	0.00795595548448326\\
94.01	0.00829271951313568\\
95.01	0.0086326909036133\\
96.01	0.00897044149421249\\
97.01	0.00929838445085943\\
98.01	0.0096063332737912\\
99.01	0.0098623535996167\\
99.02	0.00986437411420948\\
99.03	0.00986638419216177\\
99.04	0.0098683837718627\\
99.05	0.00987037279127147\\
99.06	0.00987235118791395\\
99.07	0.00987431889887921\\
99.08	0.00987627586053859\\
99.09	0.00987822200859097\\
99.1	0.00988015727829076\\
99.11	0.00988208160444435\\
99.12	0.00988399492112197\\
99.13	0.00988589716186539\\
99.14	0.00988778825975373\\
99.15	0.00988966814739988\\
99.16	0.00989153675694688\\
99.17	0.00989339402006423\\
99.18	0.00989523986794429\\
99.19	0.00989707423129856\\
99.2	0.00989889704035396\\
99.21	0.00990070822484916\\
99.22	0.00990250771403083\\
99.23	0.00990429543664984\\
99.24	0.00990607132095756\\
99.25	0.00990783529470205\\
99.26	0.00990958728512422\\
99.27	0.00991132721895409\\
99.28	0.0099130550224069\\
99.29	0.00991477062117929\\
99.3	0.00991647394044544\\
99.31	0.00991816490485323\\
99.32	0.00991984343852032\\
99.33	0.00992150946503027\\
99.34	0.00992316290742868\\
99.35	0.00992480368821925\\
99.36	0.00992643172935984\\
99.37	0.00992804695225863\\
99.38	0.00992964927777009\\
99.39	0.00993123862619115\\
99.4	0.00993281491725719\\
99.41	0.00993437807013812\\
99.42	0.00993592800343448\\
99.43	0.00993746463517346\\
99.44	0.00993898788280498\\
99.45	0.00994049766319776\\
99.46	0.00994199389263541\\
99.47	0.00994347648681246\\
99.48	0.00994494536083052\\
99.49	0.00994640042919431\\
99.5	0.00994784160580784\\
99.51	0.00994926880397047\\
99.52	0.00995068193637311\\
99.53	0.00995208091509435\\
99.54	0.00995346565159667\\
99.55	0.00995483605672261\\
99.56	0.00995619204069105\\
99.57	0.00995753351309345\\
99.58	0.00995886038289013\\
99.59	0.00996017255840666\\
99.6	0.00996146994733017\\
99.61	0.00996275245671045\\
99.62	0.00996401999295883\\
99.63	0.00996527246184481\\
99.64	0.00996650976849268\\
99.65	0.00996773181737827\\
99.66	0.0099689385123257\\
99.67	0.00997012975650425\\
99.68	0.00997130545242531\\
99.69	0.00997246550193929\\
99.7	0.00997360980623282\\
99.71	0.009974738248558\\
99.72	0.00997585070992424\\
99.73	0.00997694707046292\\
99.74	0.00997802720942307\\
99.75	0.00997909100516722\\
99.76	0.00998013833516726\\
99.77	0.00998116907600054\\
99.78	0.00998218310334597\\
99.79	0.00998318029198038\\
99.8	0.00998416051577496\\
99.81	0.00998512364769184\\
99.82	0.00998606955978089\\
99.83	0.00998699812317667\\
99.84	0.00998790920809558\\
99.85	0.00998880268383318\\
99.86	0.0099896784187618\\
99.87	0.00999053628032826\\
99.88	0.00999137613505194\\
99.89	0.00999219784852308\\
99.9	0.00999300128540131\\
99.91	0.0099937863094145\\
99.92	0.00999455278335794\\
99.93	0.00999530056909381\\
99.94	0.009996029527551\\
99.95	0.00999673951872531\\
99.96	0.00999743040168006\\
99.97	0.00999810203454702\\
99.98	0.00999875427452794\\
99.99	0.00999938697789635\\
100	0.01\\
};
\addlegendentry{$q=1$};

\addplot [color=red,solid]
  table[row sep=crcr]{%
0.01	0\\
1.01	0\\
2.01	0\\
3.01	0\\
4.01	0\\
5.01	0\\
6.01	0\\
7.01	0\\
8.01	0\\
9.01	0\\
10.01	0\\
11.01	0\\
12.01	0\\
13.01	0\\
14.01	0\\
15.01	0\\
16.01	0\\
17.01	0\\
18.01	0\\
19.01	0\\
20.01	0\\
21.01	0\\
22.01	0\\
23.01	0\\
24.01	0\\
25.01	0\\
26.01	0\\
27.01	0\\
28.01	0\\
29.01	0\\
30.01	0\\
31.01	0\\
32.01	0\\
33.01	0\\
34.01	0\\
35.01	0\\
36.01	0\\
37.01	0\\
38.01	0\\
39.01	0\\
40.01	0\\
41.01	0\\
42.01	0\\
43.01	0\\
44.01	0\\
45.01	0\\
46.01	0\\
47.01	0\\
48.01	0\\
49.01	0\\
50.01	0\\
51.01	0\\
52.01	0\\
53.01	0\\
54.01	0\\
55.01	0\\
56.01	0\\
57.01	0\\
58.01	0\\
59.01	0\\
60.01	0\\
61.01	8.12348871340116e-05\\
62.01	0.000266803100728522\\
63.01	0.000465190390940062\\
64.01	0.00067797855544325\\
65.01	0.000907073550808404\\
66.01	0.0011547455705739\\
67.01	0.00142371516634133\\
68.01	0.00171724410815382\\
69.01	0.00203920083310826\\
70.01	0.00239432008939969\\
71.01	0.00277840888262248\\
72.01	0.003187457880224\\
73.01	0.00362456692764628\\
74.01	0.00409352079991442\\
75.01	0.00459918226766583\\
76.01	0.00498833685124197\\
77.01	0.00523196975578501\\
78.01	0.00547844433181472\\
79.01	0.00572303242577908\\
80.01	0.00595956126638655\\
81.01	0.00618383648916674\\
82.01	0.00640903628688978\\
83.01	0.00663500495559349\\
84.01	0.00685912686680056\\
85.01	0.00707947183678679\\
86.01	0.00729337640664342\\
87.01	0.00749391939642775\\
88.01	0.00768901014755165\\
89.01	0.00788544157167353\\
90.01	0.00808501675136655\\
91.01	0.0082882727431725\\
92.01	0.00849454843487404\\
93.01	0.00870271239060721\\
94.01	0.00891144151874779\\
95.01	0.00911932188514614\\
96.01	0.00932498131322101\\
97.01	0.00952730884353645\\
98.01	0.00972576718884031\\
99.01	0.00989322850663147\\
99.02	0.00989465349182635\\
99.03	0.00989607320763906\\
99.04	0.00989748761156016\\
99.05	0.00989889666066474\\
99.06	0.00990030031160832\\
99.07	0.00990169852062259\\
99.08	0.00990309124351133\\
99.09	0.0099044784356463\\
99.1	0.00990586005196288\\
99.11	0.00990723604695566\\
99.12	0.00990860637467526\\
99.13	0.00990997098872422\\
99.14	0.00991132984225253\\
99.15	0.00991268288795305\\
99.16	0.009914030078057\\
99.17	0.00991537136432928\\
99.18	0.00991670669806387\\
99.19	0.00991803603007907\\
99.2	0.00991935931071277\\
99.21	0.00992067648981765\\
99.22	0.0099219875167563\\
99.23	0.00992329234039635\\
99.24	0.00992459090910551\\
99.25	0.00992588317074659\\
99.26	0.00992716907267241\\
99.27	0.00992844856172076\\
99.28	0.0099297215842092\\
99.29	0.0099309880859299\\
99.3	0.00993224801214435\\
99.31	0.0099335013075781\\
99.32	0.00993474791641538\\
99.33	0.00993598778229369\\
99.34	0.00993722084829831\\
99.35	0.00993844705695684\\
99.36	0.00993966635023357\\
99.37	0.00994087866952389\\
99.38	0.00994208395564854\\
99.39	0.00994328214884795\\
99.4	0.00994447318877637\\
99.41	0.00994565701449602\\
99.42	0.00994683356447118\\
99.43	0.00994800277656223\\
99.44	0.00994916458801953\\
99.45	0.00995031893547741\\
99.46	0.00995146575494794\\
99.47	0.00995260498181472\\
99.48	0.00995373655082658\\
99.49	0.00995486039609122\\
99.5	0.0099559764510688\\
99.51	0.00995708464856543\\
99.52	0.00995818492072661\\
99.53	0.00995927719903064\\
99.54	0.00996036141428188\\
99.55	0.00996143749660399\\
99.56	0.00996250537543313\\
99.57	0.00996356497951101\\
99.58	0.00996461623687793\\
99.59	0.00996565907486572\\
99.6	0.00996669342009061\\
99.61	0.00996771918980673\\
99.62	0.00996873629602636\\
99.63	0.00996974464986599\\
99.64	0.00997074416153733\\
99.65	0.00997173474033815\\
99.66	0.00997271629464304\\
99.67	0.00997368873189412\\
99.68	0.00997465195859159\\
99.69	0.00997560588028419\\
99.7	0.00997655040155958\\
99.71	0.00997748542604711\\
99.72	0.00997841085640911\\
99.73	0.00997932659433115\\
99.74	0.00998023254051213\\
99.75	0.00998112859465432\\
99.76	0.0099820146554532\\
99.77	0.00998289062058732\\
99.78	0.00998375638670789\\
99.79	0.00998461184942835\\
99.8	0.00998545690331381\\
99.81	0.00998629144187034\\
99.82	0.00998711535753411\\
99.83	0.00998792854166048\\
99.84	0.00998873088451288\\
99.85	0.00998952227525159\\
99.86	0.00999030260192242\\
99.87	0.00999107175144514\\
99.88	0.00999182960960189\\
99.89	0.00999257606102539\\
99.9	0.00999331098918694\\
99.91	0.00999403427638442\\
99.92	0.00999474580372994\\
99.93	0.0099954454511375\\
99.94	0.00999613309731035\\
99.95	0.0099968086197283\\
99.96	0.00999747189463473\\
99.97	0.00999812279702354\\
99.98	0.00999876120062581\\
99.99	0.00999938697789635\\
100	0.01\\
};
\addlegendentry{$q=2$};

\addplot [color=mycolor1,solid]
  table[row sep=crcr]{%
0.01	0\\
1.01	0\\
2.01	0\\
3.01	0\\
4.01	0\\
5.01	0\\
6.01	0\\
7.01	0\\
8.01	0\\
9.01	0\\
10.01	0\\
11.01	0\\
12.01	0\\
13.01	0\\
14.01	0\\
15.01	0\\
16.01	0\\
17.01	0\\
18.01	0\\
19.01	0\\
20.01	0\\
21.01	0\\
22.01	0\\
23.01	0\\
24.01	0\\
25.01	0\\
26.01	0\\
27.01	0\\
28.01	0\\
29.01	0\\
30.01	0\\
31.01	0\\
32.01	0\\
33.01	5.09644672509399e-05\\
34.01	0.000109794444065885\\
35.01	0.00017151533121116\\
36.01	0.000236298073143903\\
37.01	0.000304316035931007\\
38.01	0.000375738239777754\\
39.01	0.000450718319040512\\
40.01	0.00052938742339783\\
41.01	0.000611950742902587\\
42.01	0.000698694363709187\\
43.01	0.000789941291687961\\
44.01	0.000886057053765751\\
45.01	0.000987458085999988\\
46.01	0.00109462229690098\\
47.01	0.0012081024668774\\
48.01	0.00132854339560494\\
49.01	0.00145670404379396\\
50.01	0.00159348638652697\\
51.01	0.00173997336230275\\
52.01	0.00189747925447799\\
53.01	0.00206761721379348\\
54.01	0.00225239062531717\\
55.01	0.00245431795358011\\
56.01	0.00267590725002329\\
57.01	0.00291282995301112\\
58.01	0.00316479531533496\\
59.01	0.00343368670551544\\
60.01	0.00372185027368629\\
61.01	0.00395022031197872\\
62.01	0.00409293588869374\\
63.01	0.00424183573526733\\
64.01	0.00439676674185464\\
65.01	0.00455733023107847\\
66.01	0.00472276548092922\\
67.01	0.00489177668395449\\
68.01	0.00506227999851957\\
69.01	0.00523040716100169\\
70.01	0.00539063353106456\\
71.01	0.00554740124099169\\
72.01	0.00570587964539781\\
73.01	0.0058642310340034\\
74.01	0.00601985027425704\\
75.01	0.0061690628622802\\
76.01	0.00630852010183723\\
77.01	0.0064449175239508\\
78.01	0.00658079677745805\\
79.01	0.00671645879885539\\
80.01	0.00685206299611372\\
81.01	0.0069887554420211\\
82.01	0.00712734193524032\\
83.01	0.00726779158239517\\
84.01	0.00741019680706147\\
85.01	0.00755477469004441\\
86.01	0.00770168193423851\\
87.01	0.00785138131254818\\
88.01	0.00800463383062428\\
89.01	0.00816174771874713\\
90.01	0.00832282605401112\\
91.01	0.0084877615335843\\
92.01	0.00865636593366449\\
93.01	0.00882843564353477\\
94.01	0.00900378484830787\\
95.01	0.00918226226366763\\
96.01	0.00936376041375512\\
97.01	0.00954820718766945\\
98.01	0.00973490290657201\\
99.01	0.00989397488312613\\
99.02	0.00989537779255572\\
99.03	0.00989677585830473\\
99.04	0.00989816903435277\\
99.05	0.00989955727424447\\
99.06	0.00990094053108541\\
99.07	0.00990231875753784\\
99.08	0.00990369190581653\\
99.09	0.00990505992768447\\
99.1	0.00990642277444855\\
99.11	0.00990778039695523\\
99.12	0.00990913274558614\\
99.13	0.00991047977025362\\
99.14	0.0099118214203963\\
99.15	0.00991315764497448\\
99.16	0.00991448839246565\\
99.17	0.00991581361085984\\
99.18	0.00991713324765496\\
99.19	0.00991844724985212\\
99.2	0.00991975556395086\\
99.21	0.00992105813594439\\
99.22	0.00992235491131472\\
99.23	0.0099236458350278\\
99.24	0.0099249308515286\\
99.25	0.00992620990473608\\
99.26	0.00992748293803825\\
99.27	0.00992874989428702\\
99.28	0.00993001071579311\\
99.29	0.00993126534432089\\
99.3	0.00993251372108315\\
99.31	0.00993375578673581\\
99.32	0.00993499148137264\\
99.33	0.00993622074451984\\
99.34	0.00993744351513067\\
99.35	0.00993865973157995\\
99.36	0.00993986933165849\\
99.37	0.0099410722525676\\
99.38	0.00994226843091338\\
99.39	0.00994345780270105\\
99.4	0.00994464030332924\\
99.41	0.00994581586758415\\
99.42	0.00994698442963372\\
99.43	0.00994814592302171\\
99.44	0.00994930028066174\\
99.45	0.00995044743483127\\
99.46	0.00995158731716552\\
99.47	0.0099527198586513\\
99.48	0.00995384498962087\\
99.49	0.00995496263974562\\
99.5	0.0099560727380298\\
99.51	0.00995717521280411\\
99.52	0.00995826999171929\\
99.53	0.00995935700173958\\
99.54	0.0099604361691362\\
99.55	0.00996150741948069\\
99.56	0.00996257067763823\\
99.57	0.00996362586776092\\
99.58	0.00996467291328088\\
99.59	0.00996571173690344\\
99.6	0.00996674226060019\\
99.61	0.00996776439644508\\
99.62	0.00996877805189459\\
99.63	0.00996978313349925\\
99.64	0.0099707795468949\\
99.65	0.00997176719679398\\
99.66	0.00997274598697662\\
99.67	0.00997371582028172\\
99.68	0.00997467659859792\\
99.69	0.00997562822285446\\
99.7	0.00997657059301199\\
99.71	0.0099775036080533\\
99.72	0.00997842716597387\\
99.73	0.00997934116377243\\
99.74	0.0099802454974414\\
99.75	0.00998114006195719\\
99.76	0.00998202475127047\\
99.77	0.00998289945829632\\
99.78	0.00998376407490432\\
99.79	0.00998461849190844\\
99.8	0.00998546259905699\\
99.81	0.00998629628502236\\
99.82	0.00998711943739068\\
99.83	0.00998793194265142\\
99.84	0.00998873368618689\\
99.85	0.00998952455226156\\
99.86	0.0099903044240114\\
99.87	0.00999107318343302\\
99.88	0.00999183071137277\\
99.89	0.00999257688751569\\
99.9	0.00999331159037439\\
99.91	0.00999403469727783\\
99.92	0.00999474608435993\\
99.93	0.00999544562654817\\
99.94	0.00999613319755202\\
99.95	0.00999680866985125\\
99.96	0.0099974719146842\\
99.97	0.00999812280203583\\
99.98	0.00999876120062581\\
99.99	0.00999938697789635\\
100	0.01\\
};
\addlegendentry{$q=3$};

\addplot [color=green,solid]
  table[row sep=crcr]{%
0.01	0.00157538227128272\\
1.01	0.00159330354651341\\
2.01	0.00161198537576968\\
3.01	0.00163146474644173\\
4.01	0.00165178130380222\\
5.01	0.00167297744204932\\
6.01	0.00169509857820879\\
7.01	0.00171819349773444\\
8.01	0.00174231475891689\\
9.01	0.0017675191685306\\
10.01	0.00179386834469968\\
11.01	0.00182142938721429\\
12.01	0.00185027568107371\\
13.01	0.00188048786631688\\
14.01	0.00191215501683038\\
15.01	0.00194537608363874\\
16.01	0.00198026167534923\\
17.01	0.00201693627157198\\
18.01	0.00205554099655589\\
19.01	0.00209623712319749\\
20.01	0.00213921053658141\\
21.01	0.002184677467822\\
22.01	0.00223289192256133\\
23.01	0.0022841553875158\\
24.01	0.00233882962246737\\
25.01	0.00239735366247747\\
26.01	0.00246026660746542\\
27.01	0.00252823842498746\\
28.01	0.00260211192820974\\
29.01	0.00268242605954877\\
30.01	0.00276757166883638\\
31.01	0.00285772953510862\\
32.01	0.00295348971141785\\
33.01	0.00300372734832802\\
34.01	0.00305074524799789\\
35.01	0.003099746021628\\
36.01	0.00315080568957947\\
37.01	0.00320401012188151\\
38.01	0.00325946384392634\\
39.01	0.00331730508574213\\
40.01	0.00337772026057272\\
41.01	0.00344085589109503\\
42.01	0.00350682458589178\\
43.01	0.00357575608053452\\
44.01	0.00364777979727941\\
45.01	0.0037230187089124\\
46.01	0.00380158242461726\\
47.01	0.00388355697266633\\
48.01	0.00396898979951969\\
49.01	0.00405786783034105\\
50.01	0.00415008544349357\\
51.01	0.00424539774580612\\
52.01	0.0043433523618149\\
53.01	0.00444318971346495\\
54.01	0.00454369692583007\\
55.01	0.00464299322119982\\
56.01	0.00473892097650941\\
57.01	0.00483613780119834\\
58.01	0.00493510252095684\\
59.01	0.00503411480971749\\
60.01	0.00513164758415562\\
61.01	0.00522627916595729\\
62.01	0.00532151384207943\\
63.01	0.00541782343389308\\
64.01	0.0055146740077758\\
65.01	0.00561145400197507\\
66.01	0.00570750901710846\\
67.01	0.00580221735540401\\
68.01	0.0058951352324198\\
69.01	0.00598689973384306\\
70.01	0.00607893169971058\\
71.01	0.0061719996330892\\
72.01	0.0062661112646304\\
73.01	0.00636111892574771\\
74.01	0.00645698629557117\\
75.01	0.00655387851398136\\
76.01	0.00665228205931369\\
77.01	0.00675276780062926\\
78.01	0.00685569246110767\\
79.01	0.00696131053188288\\
80.01	0.00706992105889258\\
81.01	0.0071818163679376\\
82.01	0.0072972050433229\\
83.01	0.00741628152993219\\
84.01	0.0075392471567647\\
85.01	0.00766629502929004\\
86.01	0.00779761327606906\\
87.01	0.00793338849602983\\
88.01	0.00807377339546965\\
89.01	0.00821887633982523\\
90.01	0.00836876794752141\\
91.01	0.00852348913886508\\
92.01	0.00868305782989654\\
93.01	0.00884746537033053\\
94.01	0.0090166661308453\\
95.01	0.00919056134976411\\
96.01	0.0093689767617065\\
97.01	0.00955163360046357\\
98.01	0.00973566740022747\\
99.01	0.00989398858246139\\
99.02	0.00989539094476144\\
99.03	0.00989678847973188\\
99.04	0.00989818114103547\\
99.05	0.00989956888190258\\
99.06	0.00990095165512703\\
99.07	0.00990232941306191\\
99.08	0.00990370210761538\\
99.09	0.00990506969024643\\
99.1	0.00990643211196056\\
99.11	0.00990778932330546\\
99.12	0.00990914127436665\\
99.13	0.00991048791476302\\
99.14	0.0099118291936424\\
99.15	0.00991316505967706\\
99.16	0.00991449546105915\\
99.17	0.00991582034549609\\
99.18	0.00991713966020598\\
99.19	0.00991845335191288\\
99.2	0.00991976136684211\\
99.21	0.00992106365071546\\
99.22	0.00992236014874638\\
99.23	0.00992365080563514\\
99.24	0.00992493556556388\\
99.25	0.00992621437219167\\
99.26	0.00992748716864954\\
99.27	0.00992875389753535\\
99.28	0.00993001450090877\\
99.29	0.00993126892028608\\
99.3	0.00993251709663497\\
99.31	0.00993375897036932\\
99.32	0.00993499448134388\\
99.33	0.00993622356884892\\
99.34	0.00993744617160479\\
99.35	0.00993866222775656\\
99.36	0.0099398716748684\\
99.37	0.00994107444991809\\
99.38	0.0099422704892914\\
99.39	0.00994345972877637\\
99.4	0.00994464210355766\\
99.41	0.00994581754821071\\
99.42	0.00994698599669593\\
99.43	0.00994814738235283\\
99.44	0.00994930163789402\\
99.45	0.00995044869539926\\
99.46	0.00995158848630938\\
99.47	0.00995272094142012\\
99.48	0.00995384599087603\\
99.49	0.00995496356416414\\
99.5	0.00995607359010774\\
99.51	0.00995717599685997\\
99.52	0.00995827071189741\\
99.53	0.0099593576620136\\
99.54	0.00996043677331251\\
99.55	0.0099615079712019\\
99.56	0.00996257118038667\\
99.57	0.00996362632486211\\
99.58	0.00996467332790712\\
99.59	0.00996571211207732\\
99.6	0.00996674259919812\\
99.61	0.00996776470120001\\
99.62	0.00996877832540117\\
99.63	0.00996978337821723\\
99.64	0.00997077976515262\\
99.65	0.00997176739079183\\
99.66	0.0099727461587905\\
99.67	0.00997371597186654\\
99.68	0.00997467673179107\\
99.69	0.00997562833937934\\
99.7	0.00997657069448153\\
99.71	0.00997750369597346\\
99.72	0.00997842724174722\\
99.73	0.0099793412287017\\
99.74	0.00998024555273308\\
99.75	0.00998114010872511\\
99.76	0.00998202479053945\\
99.77	0.00998289949100579\\
99.78	0.00998376410191195\\
99.79	0.00998461851399383\\
99.8	0.00998546261692534\\
99.81	0.00998629629930815\\
99.82	0.0099871194486614\\
99.83	0.00998793195141125\\
99.84	0.00998873369288044\\
99.85	0.00998952455727761\\
99.86	0.00999030442768662\\
99.87	0.00999107318605574\\
99.88	0.0099918307131867\\
99.89	0.00999257688872371\\
99.9	0.00999331159114228\\
99.91	0.00999403469773802\\
99.92	0.00999474608461527\\
99.93	0.00999544562667568\\
99.94	0.00999613319760659\\
99.95	0.00999680866986942\\
99.96	0.00999747191468782\\
99.97	0.00999812280203583\\
99.98	0.00999876120062581\\
99.99	0.00999938697789635\\
100	0.01\\
};
\addlegendentry{$q=4$};

\end{axis}
\end{tikzpicture}%

  \caption{Continuous Time}
\end{subfigure}%
\hfill%
\begin{subfigure}{.45\linewidth}
  \centering
  \setlength\figureheight{\linewidth} 
  \setlength\figurewidth{\linewidth}
  \tikzsetnextfilename{dp_colorbar/dp_dscr_nFPC_z8}
  % This file was created by matlab2tikz.
%
%The latest updates can be retrieved from
%  http://www.mathworks.com/matlabcentral/fileexchange/22022-matlab2tikz-matlab2tikz
%where you can also make suggestions and rate matlab2tikz.
%
\definecolor{mycolor1}{rgb}{0.00000,1.00000,0.14286}%
\definecolor{mycolor2}{rgb}{0.00000,1.00000,0.28571}%
\definecolor{mycolor3}{rgb}{0.00000,1.00000,0.42857}%
\definecolor{mycolor4}{rgb}{0.00000,1.00000,0.57143}%
\definecolor{mycolor5}{rgb}{0.00000,1.00000,0.71429}%
\definecolor{mycolor6}{rgb}{0.00000,1.00000,0.85714}%
\definecolor{mycolor7}{rgb}{0.00000,1.00000,1.00000}%
\definecolor{mycolor8}{rgb}{0.00000,0.87500,1.00000}%
\definecolor{mycolor9}{rgb}{0.00000,0.62500,1.00000}%
\definecolor{mycolor10}{rgb}{0.12500,0.00000,1.00000}%
\definecolor{mycolor11}{rgb}{0.25000,0.00000,1.00000}%
\definecolor{mycolor12}{rgb}{0.37500,0.00000,1.00000}%
\definecolor{mycolor13}{rgb}{0.50000,0.00000,1.00000}%
\definecolor{mycolor14}{rgb}{0.62500,0.00000,1.00000}%
\definecolor{mycolor15}{rgb}{0.75000,0.00000,1.00000}%
\definecolor{mycolor16}{rgb}{0.87500,0.00000,1.00000}%
\definecolor{mycolor17}{rgb}{1.00000,0.00000,1.00000}%
\definecolor{mycolor18}{rgb}{1.00000,0.00000,0.87500}%
\definecolor{mycolor19}{rgb}{1.00000,0.00000,0.62500}%
\definecolor{mycolor20}{rgb}{0.85714,0.00000,0.00000}%
\definecolor{mycolor21}{rgb}{0.71429,0.00000,0.00000}%
%
\begin{tikzpicture}[trim axis left, trim axis right]

\begin{axis}[%
width=\figurewidth,
height=\figureheight,
at={(0\figurewidth,0\figureheight)},
scale only axis,
every outer x axis line/.append style={black},
every x tick label/.append style={font=\color{black}},
xmin=0,
xmax=600,
every outer y axis line/.append style={black},
every y tick label/.append style={font=\color{black}},
ymin=0,
ymax=0.014,
axis background/.style={fill=white},
axis x line*=bottom,
axis y line*=left,
yticklabel style={
        /pgf/number format/fixed,
        /pgf/number format/precision=3
},
scaled y ticks=false
]
\addplot [color=green,solid,forget plot]
  table[row sep=crcr]{%
1	0.00590226758762085\\
2	0.00590219569436802\\
3	0.00590212217829792\\
4	0.00590204700301902\\
5	0.00590197013133593\\
6	0.0059018915252321\\
7	0.00590181114585226\\
8	0.00590172895348463\\
9	0.00590164490754284\\
10	0.00590155896654729\\
11	0.00590147108810626\\
12	0.00590138122889673\\
13	0.00590128934464471\\
14	0.00590119539010542\\
15	0.00590109931904272\\
16	0.00590100108420855\\
17	0.00590090063732162\\
18	0.00590079792904614\\
19	0.00590069290896984\\
20	0.00590058552558166\\
21	0.00590047572624894\\
22	0.00590036345719459\\
23	0.00590024866347334\\
24	0.00590013128894799\\
25	0.00590001127626487\\
26	0.00589988856682939\\
27	0.00589976310078054\\
28	0.00589963481696547\\
29	0.00589950365291344\\
30	0.00589936954480913\\
31	0.0058992324274661\\
32	0.00589909223429924\\
33	0.00589894889729688\\
34	0.00589880234699279\\
35	0.00589865251243731\\
36	0.00589849932116842\\
37	0.00589834269918213\\
38	0.00589818257090226\\
39	0.00589801885915011\\
40	0.00589785148511377\\
41	0.00589768036831639\\
42	0.00589750542658482\\
43	0.0058973265760169\\
44	0.00589714373094918\\
45	0.00589695680392344\\
46	0.00589676570565338\\
47	0.00589657034499055\\
48	0.00589637062888976\\
49	0.00589616646237447\\
50	0.00589595774850121\\
51	0.00589574438832412\\
52	0.00589552628085854\\
53	0.00589530332304463\\
54	0.00589507540971039\\
55	0.00589484243353415\\
56	0.00589460428500702\\
57	0.00589436085239466\\
58	0.00589411202169841\\
59	0.00589385767661683\\
60	0.00589359769850622\\
61	0.00589333196634099\\
62	0.00589306035667342\\
63	0.00589278274359378\\
64	0.0058924989986892\\
65	0.00589220899100278\\
66	0.00589191258699223\\
67	0.00589160965048836\\
68	0.00589130004265269\\
69	0.00589098362193565\\
70	0.00589066024403374\\
71	0.00589032976184663\\
72	0.00588999202543418\\
73	0.00588964688197296\\
74	0.00588929417571254\\
75	0.00588893374793158\\
76	0.00588856543689365\\
77	0.00588818907780258\\
78	0.0058878045027581\\
79	0.00588741154071082\\
80	0.00588701001741674\\
81	0.00588659975539223\\
82	0.00588618057386828\\
83	0.00588575228874461\\
84	0.00588531471254356\\
85	0.00588486765436407\\
86	0.00588441091983501\\
87	0.00588394431106872\\
88	0.00588346762661411\\
89	0.00588298066140999\\
90	0.00588248320673816\\
91	0.00588197505017642\\
92	0.0058814559755516\\
93	0.00588092576289306\\
94	0.00588038418838609\\
95	0.00587983102432578\\
96	0.00587926603907173\\
97	0.00587868899700308\\
98	0.00587809965847495\\
99	0.00587749777977601\\
100	0.00587688311308808\\
101	0.00587625540644743\\
102	0.00587561440370964\\
103	0.00587495984451722\\
104	0.00587429146427208\\
105	0.00587360899411308\\
106	0.00587291216090031\\
107	0.00587220068720716\\
108	0.00587147429132234\\
109	0.0058707326872633\\
110	0.00586997558480418\\
111	0.00586920268952053\\
112	0.00586841370285426\\
113	0.00586760832220309\\
114	0.00586678624103878\\
115	0.00586594714905966\\
116	0.0058650907323843\\
117	0.00586421667379264\\
118	0.00586332465302427\\
119	0.00586241434714379\\
120	0.00586148543098398\\
121	0.00586053757768146\\
122	0.00585957045931951\\
123	0.0058585837476963\\
124	0.0058575771152385\\
125	0.00585655023608422\\
126	0.00585550278736069\\
127	0.00585443445068751\\
128	0.00585334491393736\\
129	0.00585223387329249\\
130	0.00585110103563626\\
131	0.00584994612132499\\
132	0.00584876886738457\\
133	0.005847569031183\\
134	0.00584634639462503\\
135	0.00584510076891578\\
136	0.00584383199993321\\
137	0.00584253997423657\\
138	0.00584122462571889\\
139	0.00583988594288035\\
140	0.00583852397665295\\
141	0.00583713884863892\\
142	0.00583573075952868\\
143	0.00583429999732783\\
144	0.00583284694483269\\
145	0.00583137208553207\\
146	0.00582987600675632\\
147	0.00582835939840925\\
148	0.00582682304496935\\
149	0.00582526780758558\\
150	0.00582369459197112\\
151	0.00582210429627641\\
152	0.0058204977296022\\
153	0.00581887545880713\\
154	0.00581723739739895\\
155	0.00581558340849058\\
156	0.00581391335407126\\
157	0.00581222709496928\\
158	0.00581052449081065\\
159	0.00580880539997442\\
160	0.00580706967954358\\
161	0.0058053171852519\\
162	0.00580354777142536\\
163	0.00580176129091901\\
164	0.00579995759504783\\
165	0.00579813653351179\\
166	0.00579629795431418\\
167	0.00579444170367297\\
168	0.00579256762592458\\
169	0.00579067556341947\\
170	0.00578876535640884\\
171	0.00578683684292163\\
172	0.00578488985863155\\
173	0.00578292423671281\\
174	0.00578093980768388\\
175	0.00577893639923868\\
176	0.00577691383606331\\
177	0.00577487193963837\\
178	0.00577281052802476\\
179	0.00577072941563249\\
180	0.00576862841297023\\
181	0.00576650732637528\\
182	0.00576436595772126\\
183	0.00576220410410249\\
184	0.00576002155749306\\
185	0.00575781810437872\\
186	0.00575559352535908\\
187	0.00575334759471802\\
188	0.00575108007995981\\
189	0.00574879074130821\\
190	0.00574647933116541\\
191	0.00574414559352756\\
192	0.00574178926335378\\
193	0.00573941006588399\\
194	0.00573700771590237\\
195	0.00573458191694115\\
196	0.0057321323604202\\
197	0.00572965872471653\\
198	0.00572716067415846\\
199	0.00572463785793738\\
200	0.00572208990893019\\
201	0.00571951644242463\\
202	0.00571691705473904\\
203	0.00571429132172704\\
204	0.00571163879715717\\
205	0.00570895901095526\\
206	0.00570625146729825\\
207	0.00570351564254499\\
208	0.00570075098298861\\
209	0.00569795690241481\\
210	0.0056951327794465\\
211	0.00569227795465521\\
212	0.00568939172741659\\
213	0.00568647335248552\\
214	0.00568352203626303\\
215	0.00568053693272564\\
216	0.00567751713898401\\
217	0.00567446169043394\\
218	0.00567136955546088\\
219	0.00566823962965385\\
220	0.00566507072948129\\
221	0.0056618615853776\\
222	0.00565861083418448\\
223	0.00565531701088754\\
224	0.00565197853958499\\
225	0.00564859372362192\\
226	0.00564516073482158\\
227	0.00564167760174593\\
228	0.00563814219692024\\
229	0.00563455222296846\\
230	0.00563090519763434\\
231	0.00562719843773486\\
232	0.00562342904227852\\
233	0.00561959387539725\\
234	0.00561568954947589\\
235	0.00561171238065216\\
236	0.00560765838085181\\
237	0.00560352323743865\\
238	0.00559930228887502\\
239	0.0055949904999797\\
240	0.00559058243743392\\
241	0.00558607224639563\\
242	0.00558145362942611\\
243	0.00557671982937909\\
244	0.00557186361849583\\
245	0.00556687729672331\\
246	0.00556175270328639\\
247	0.00555648124686407\\
248	0.0055510539614342\\
249	0.005545461597087\\
250	0.00553969475807989\\
251	0.00553374410485552\\
252	0.00552760064720504\\
253	0.00552125616713517\\
254	0.00551470375091368\\
255	0.00550793853492297\\
256	0.00550095871514837\\
257	0.00549376689797336\\
258	0.00548637190179406\\
259	0.00547880412473265\\
260	0.0054711615979969\\
261	0.00546344384532699\\
262	0.00545565041652165\\
263	0.00544778089136691\\
264	0.00543983488411992\\
265	0.00543181204862253\\
266	0.00542371208001076\\
267	0.00541553463770604\\
268	0.00540727939736451\\
269	0.00539894607785779\\
270	0.0053905344489206\\
271	0.00538204433742959\\
272	0.00537347563429518\\
273	0.00536482830200602\\
274	0.0053561023828637\\
275	0.00534729800793862\\
276	0.00533841540677069\\
277	0.00532945491783\\
278	0.00532041699977937\\
279	0.00531130224385107\\
280	0.00530211138967355\\
281	0.00529284536486983\\
282	0.00528350523427824\\
283	0.00527409219089737\\
284	0.00526460765797136\\
285	0.00525505332659508\\
286	0.00524543121741848\\
287	0.00523574385418239\\
288	0.00522599423971317\\
289	0.00521618584396532\\
290	0.00520632265207895\\
291	0.00519640921234668\\
292	0.00518645069507146\\
293	0.00517645328602592\\
294	0.00516642397989371\\
295	0.00515637066445414\\
296	0.00514630221035993\\
297	0.00513622856570469\\
298	0.00512616085401361\\
299	0.00511611147354042\\
300	0.00510609419480906\\
301	0.00509612425244915\\
302	0.00508621842822103\\
303	0.00507639513606668\\
304	0.00506667464224506\\
305	0.00505707780951913\\
306	0.0050476264979348\\
307	0.00503834342573031\\
308	0.00502925168468738\\
309	0.00502037401474721\\
310	0.00501173175224369\\
311	0.00500334333618516\\
312	0.00499522220916615\\
313	0.00498737383552014\\
314	0.00497979285788452\\
315	0.00497245932733467\\
316	0.00496517032215807\\
317	0.00495789415233938\\
318	0.00495063539936947\\
319	0.00494339888326918\\
320	0.00493618966568305\\
321	0.00492901305487639\\
322	0.00492187460699136\\
323	0.00491478012219509\\
324	0.00490773563673702\\
325	0.00490074740982068\\
326	0.00489382190401122\\
327	0.00488696575826997\\
328	0.00488018574497965\\
329	0.00487348871923138\\
330	0.00486688156124533\\
331	0.0048603711051543\\
332	0.00485396405196309\\
333	0.00484766686390582\\
334	0.00484148564706391\\
335	0.004835426041604\\
336	0.00482949307007449\\
337	0.00482369096615793\\
338	0.00481802298469154\\
339	0.00481249119501329\\
340	0.00480709626105165\\
341	0.0048018371760404\\
342	0.00479671096199709\\
343	0.0047917124319064\\
344	0.00478683399505208\\
345	0.00478206554455709\\
346	0.00477739448361568\\
347	0.00477280597085663\\
348	0.00476828349815441\\
349	0.00476380995973181\\
350	0.00475937041137946\\
351	0.00475496584128219\\
352	0.00475059704889023\\
353	0.00474626461898813\\
354	0.00474196889476297\\
355	0.00473770995058193\\
356	0.00473348756479409\\
357	0.00472930119310447\\
358	0.00472514994262298\\
359	0.00472103254748688\\
360	0.00471694734743485\\
361	0.00471289227064709\\
362	0.00470886482249196\\
363	0.00470486208307431\\
364	0.00470088071707579\\
365	0.00469691699727165\\
366	0.00469296684452168\\
367	0.00468902588702934\\
368	0.0046850895413815\\
369	0.00468115311715026\\
370	0.00467721194541925\\
371	0.00467326152914154\\
372	0.00466929770959023\\
373	0.00466531683699681\\
374	0.00466131592437641\\
375	0.00465729275009499\\
376	0.00465324578873874\\
377	0.00464917342082965\\
378	0.00464507393782156\\
379	0.00464094554882346\\
380	0.00463678638918679\\
381	0.00463259453106235\\
382	0.00462836799593486\\
383	0.00462410476907713\\
384	0.00461980281578399\\
385	0.00461546009911589\\
386	0.00461107459872934\\
387	0.0046066443301807\\
388	0.0046021673638745\\
389	0.00459764184259331\\
390	0.00459306599629965\\
391	0.00458843815268068\\
392	0.00458375674176487\\
393	0.00457902029295402\\
394	0.00457422742312289\\
395	0.00456937681524134\\
396	0.00456446718854771\\
397	0.00455949726408454\\
398	0.00455446576244348\\
399	0.00454937140691227\\
400	0.00454421292645538\\
401	0.00453898905842823\\
402	0.00453369855091595\\
403	0.00452834016458548\\
404	0.00452291267394101\\
405	0.00451741486788391\\
406	0.00451184554949909\\
407	0.00450620353502427\\
408	0.00450048765201002\\
409	0.00449469673674516\\
410	0.004488829631108\\
411	0.00448288517909952\\
412	0.00447686222341574\\
413	0.00447075960249672\\
414	0.00446457614851732\\
415	0.00445831068669633\\
416	0.00445196203510796\\
417	0.00444552900437213\\
418	0.00443901039722519\\
419	0.00443240500797667\\
420	0.00442571162186341\\
421	0.00441892901431833\\
422	0.0044120559501766\\
423	0.00440509118284664\\
424	0.00439803345347797\\
425	0.00439088149015869\\
426	0.00438363400717377\\
427	0.0043762897043491\\
428	0.0043688472664947\\
429	0.00436130536294375\\
430	0.00435366264716264\\
431	0.00434591775638611\\
432	0.0043380693112426\\
433	0.00433011591537352\\
434	0.00432205615505058\\
435	0.00431388859879552\\
436	0.00430561179700604\\
437	0.00429722428159108\\
438	0.0042887245656184\\
439	0.00428011114297546\\
440	0.00427138248804422\\
441	0.00426253705538868\\
442	0.00425357327945279\\
443	0.00424448957426615\\
444	0.00423528433315402\\
445	0.00422595592844986\\
446	0.00421650271120965\\
447	0.00420692301092892\\
448	0.00419721513526305\\
449	0.0041873773697511\\
450	0.00417740797754377\\
451	0.00416730519913552\\
452	0.00415706725210115\\
453	0.0041466923308371\\
454	0.00413617860630727\\
455	0.00412552422579394\\
456	0.00411472731265397\\
457	0.00410378596608072\\
458	0.00409269826087246\\
459	0.00408146224720821\\
460	0.00407007595043173\\
461	0.00405853737084471\\
462	0.00404684448351028\\
463	0.00403499523806797\\
464	0.00402298755856135\\
465	0.00401081934327958\\
466	0.00399848846461431\\
467	0.00398599276893367\\
468	0.00397333007647451\\
469	0.00396049818125501\\
470	0.00394749485100917\\
471	0.00393431782714499\\
472	0.00392096482472816\\
473	0.00390743353249293\\
474	0.00389372161288206\\
475	0.0038798267021169\\
476	0.00386574641029963\\
477	0.00385147832154774\\
478	0.00383701999416199\\
479	0.00382236896082727\\
480	0.00380752272884562\\
481	0.00379247878039903\\
482	0.00377723457283913\\
483	0.00376178753899777\\
484	0.00374613508751146\\
485	0.00373027460314929\\
486	0.00371420344713001\\
487	0.0036979189574103\\
488	0.00368141844892043\\
489	0.00366469921371667\\
490	0.00364775852101219\\
491	0.0036305936170375\\
492	0.00361320172467033\\
493	0.00359558004275919\\
494	0.00357772574504836\\
495	0.00355963597859005\\
496	0.00354130786150495\\
497	0.00352273847992189\\
498	0.00350392488389229\\
499	0.00348486408203332\\
500	0.00346555303460428\\
501	0.00344598864466502\\
502	0.00342616774689879\\
503	0.0034060870936082\\
504	0.00338574333730855\\
505	0.00336513300924976\\
506	0.00334425249309668\\
507	0.0033230979928903\\
508	0.00330166549430296\\
509	0.00327995071809673\\
510	0.00325794906454704\\
511	0.00323565558805765\\
512	0.00321306503419404\\
513	0.00319017544354297\\
514	0.00316698274648975\\
515	0.00314347361015554\\
516	0.00311963450968671\\
517	0.00309544530579971\\
518	0.00307089148905672\\
519	0.00304596426033527\\
520	0.00302065473786325\\
521	0.00299495569260285\\
522	0.00296886135966797\\
523	0.00294236719344355\\
524	0.00291547045767632\\
525	0.00288817079499245\\
526	0.00286047088236643\\
527	0.00283237770487081\\
528	0.00280390829254135\\
529	0.00277516730460763\\
530	0.00274609835995322\\
531	0.00271660490486484\\
532	0.00268653048771086\\
533	0.00265582621920599\\
534	0.00262444011296189\\
535	0.00259232902272056\\
536	0.0025594459900875\\
537	0.00252574097051207\\
538	0.00249115886685006\\
539	0.00245563912095192\\
540	0.00241911485691729\\
541	0.00238151650599522\\
542	0.00234276913476679\\
543	0.00230278422282756\\
544	0.00226144060937377\\
545	0.00222094444173137\\
546	0.00218284040994203\\
547	0.00214486939655888\\
548	0.00210633730549874\\
549	0.00206719310578365\\
550	0.00202744380742546\\
551	0.0019871081064355\\
552	0.00194621198469961\\
553	0.00190478906376779\\
554	0.00186288216891816\\
555	0.0018205457624277\\
556	0.0017778495907487\\
557	0.00173488030913016\\
558	0.00169240265022286\\
559	0.00165074091200184\\
560	0.0016087876624426\\
561	0.00156654726965046\\
562	0.00152404814044937\\
563	0.00148131883357367\\
564	0.00143838736743784\\
565	0.00139539549877649\\
566	0.00135261018668069\\
567	0.00130946073240556\\
568	0.001265963685663\\
569	0.001222137707383\\
570	0.00117800354514696\\
571	0.0011335841768572\\
572	0.00108890495847541\\
573	0.00104399377313653\\
574	0.000998881179097747\\
575	0.000953600553169377\\
576	0.00090818822522013\\
577	0.000862683598018209\\
578	0.000817129244997436\\
579	0.000771570976441667\\
580	0.000726057861961392\\
581	0.000680642193874463\\
582	0.000635379372066651\\
583	0.000590327685971717\\
584	0.000545547963424768\\
585	0.000501103049522702\\
586	0.000457057072276927\\
587	0.000413474448977722\\
588	0.000370418597003136\\
589	0.000327950360562704\\
590	0.000286126318718799\\
591	0.000244997582319803\\
592	0.000204610904245313\\
593	0.000165017211836692\\
594	0.000126301460681321\\
595	8.88161203105196e-05\\
596	5.34134895574392e-05\\
597	2.21100055488405e-05\\
598	0\\
599	0\\
600	0\\
};
\addplot [color=mycolor1,solid,forget plot]
  table[row sep=crcr]{%
1	0.00590251934398063\\
2	0.00590245528621135\\
3	0.00590238985445012\\
4	0.0059023230203767\\
5	0.00590225475513416\\
6	0.00590218502932082\\
7	0.00590211381298212\\
8	0.00590204107560234\\
9	0.00590196678609648\\
10	0.00590189091280194\\
11	0.00590181342347044\\
12	0.00590173428525967\\
13	0.00590165346472526\\
14	0.00590157092781229\\
15	0.00590148663984745\\
16	0.00590140056553063\\
17	0.005901312668927\\
18	0.00590122291345906\\
19	0.00590113126189837\\
20	0.0059010376763578\\
21	0.00590094211828385\\
22	0.00590084454844871\\
23	0.00590074492694275\\
24	0.00590064321316698\\
25	0.00590053936582584\\
26	0.00590043334291956\\
27	0.00590032510173767\\
28	0.00590021459885176\\
29	0.00590010179010885\\
30	0.00589998663062511\\
31	0.00589986907477924\\
32	0.00589974907620682\\
33	0.00589962658779437\\
34	0.00589950156167406\\
35	0.00589937394921843\\
36	0.00589924370103549\\
37	0.00589911076696443\\
38	0.00589897509607151\\
39	0.00589883663664628\\
40	0.0058986953361982\\
41	0.00589855114145387\\
42	0.00589840399835472\\
43	0.00589825385205506\\
44	0.00589810064692058\\
45	0.00589794432652779\\
46	0.00589778483366375\\
47	0.00589762211032613\\
48	0.00589745609772473\\
49	0.00589728673628285\\
50	0.00589711396563975\\
51	0.00589693772465387\\
52	0.00589675795140676\\
53	0.00589657458320775\\
54	0.00589638755659954\\
55	0.00589619680736474\\
56	0.00589600227053305\\
57	0.00589580388038979\\
58	0.00589560157048523\\
59	0.0058953952736447\\
60	0.00589518492198015\\
61	0.00589497044690265\\
62	0.00589475177913608\\
63	0.0058945288487318\\
64	0.00589430158508479\\
65	0.00589406991695089\\
66	0.00589383377246554\\
67	0.00589359307916327\\
68	0.0058933477639994\\
69	0.00589309775337223\\
70	0.00589284297314741\\
71	0.00589258334868332\\
72	0.00589231880485801\\
73	0.00589204926609763\\
74	0.00589177465640655\\
75	0.00589149489939882\\
76	0.00589120991833141\\
77	0.00589091963613875\\
78	0.00589062397546922\\
79	0.00589032285872279\\
80	0.00589001620809103\\
81	0.00588970394559791\\
82	0.00588938599314266\\
83	0.0058890622725441\\
84	0.00588873270558701\\
85	0.00588839721406927\\
86	0.00588805571985134\\
87	0.00588770814490695\\
88	0.00588735441137535\\
89	0.00588699444161509\\
90	0.0058866281582593\\
91	0.00588625548427224\\
92	0.0058858763430074\\
93	0.00588549065826699\\
94	0.0058850983543624\\
95	0.00588469935617632\\
96	0.00588429358922564\\
97	0.00588388097972579\\
98	0.00588346145465596\\
99	0.00588303494182551\\
100	0.00588260136994127\\
101	0.00588216066867578\\
102	0.00588171276873639\\
103	0.00588125760193554\\
104	0.00588079510126159\\
105	0.0058803252009507\\
106	0.00587984783655952\\
107	0.0058793629450391\\
108	0.00587887046480937\\
109	0.00587837033583518\\
110	0.00587786249970313\\
111	0.00587734689969997\\
112	0.00587682348089286\\
113	0.00587629219021066\\
114	0.00587575297652769\\
115	0.00587520579074966\\
116	0.00587465058590212\\
117	0.00587408731722169\\
118	0.0058735159422506\\
119	0.00587293642093432\\
120	0.00587234871572358\\
121	0.00587175279167982\\
122	0.00587114861658496\\
123	0.0058705361610553\\
124	0.00586991539865931\\
125	0.00586928630603887\\
126	0.00586864886303308\\
127	0.00586800305280335\\
128	0.00586734886195822\\
129	0.00586668628067415\\
130	0.0058660153028101\\
131	0.00586533592600947\\
132	0.00586464815178444\\
133	0.00586395198557304\\
134	0.00586324743675978\\
135	0.00586253451864659\\
136	0.00586181324835835\\
137	0.005861083646664\\
138	0.00586034573769071\\
139	0.00585959954850546\\
140	0.00585884510853362\\
141	0.00585808244878297\\
142	0.00585731160083764\\
143	0.00585653259558932\\
144	0.00585574546167559\\
145	0.0058549502236067\\
146	0.00585414689958255\\
147	0.00585333549903554\\
148	0.00585251601999084\\
149	0.00585168844641513\\
150	0.00585085274580252\\
151	0.00585000886715439\\
152	0.00584915673875567\\
153	0.00584829627335727\\
154	0.00584742738093528\\
155	0.00584654996957622\\
156	0.00584566394539258\\
157	0.00584476921243401\\
158	0.0058438656725945\\
159	0.00584295322551465\\
160	0.00584203176847918\\
161	0.00584110119630936\\
162	0.00584016140125032\\
163	0.00583921227285307\\
164	0.00583825369785029\\
165	0.00583728556002674\\
166	0.00583630774008337\\
167	0.00583532011549491\\
168	0.00583432256036083\\
169	0.00583331494524959\\
170	0.0058322971370354\\
171	0.00583126899872801\\
172	0.00583023038929416\\
173	0.00582918116347169\\
174	0.00582812117157468\\
175	0.0058270502592907\\
176	0.00582596826746862\\
177	0.00582487503189767\\
178	0.00582377038307673\\
179	0.00582265414597404\\
180	0.00582152613977678\\
181	0.00582038617763003\\
182	0.00581923406636559\\
183	0.00581806960621902\\
184	0.00581689259053609\\
185	0.00581570280546712\\
186	0.00581450002964959\\
187	0.0058132840338785\\
188	0.0058120545807642\\
189	0.00581081142437762\\
190	0.00580955430988192\\
191	0.00580828297315126\\
192	0.00580699714037604\\
193	0.00580569652765392\\
194	0.00580438084056727\\
195	0.00580304977374607\\
196	0.00580170301041663\\
197	0.00580034022193596\\
198	0.00579896106731123\\
199	0.00579756519270495\\
200	0.00579615223092535\\
201	0.00579472180090243\\
202	0.00579327350714958\\
203	0.0057918069392109\\
204	0.0057903216710948\\
205	0.0057888172606946\\
206	0.00578729324919572\\
207	0.00578574916047149\\
208	0.00578418450046743\\
209	0.00578259875657595\\
210	0.00578099139700239\\
211	0.00577936187012468\\
212	0.00577770960384813\\
213	0.00577603400495866\\
214	0.00577433445847713\\
215	0.00577261032701884\\
216	0.00577086095016265\\
217	0.00576908564383538\\
218	0.00576728369971782\\
219	0.00576545438468046\\
220	0.00576359694025803\\
221	0.00576171058217476\\
222	0.00575979449993266\\
223	0.00575784785647946\\
224	0.00575586978797509\\
225	0.00575385940367807\\
226	0.00575181578597957\\
227	0.00574973799061557\\
228	0.00574762504709556\\
229	0.00574547595939447\\
230	0.00574328970696595\\
231	0.00574106524615232\\
232	0.00573880151205814\\
233	0.00573649742074274\\
234	0.00573415187039852\\
235	0.00573176374606769\\
236	0.00572933192429559\\
237	0.00572685527840376\\
238	0.00572433268484067\\
239	0.00572176303077739\\
240	0.00571914522312208\\
241	0.00571647819914063\\
242	0.00571376093887343\\
243	0.0057109924795355\\
244	0.00570817193206658\\
245	0.00570529849995605\\
246	0.00570237150039545\\
247	0.00569939038768694\\
248	0.00569635477865656\\
249	0.00569326447956258\\
250	0.00569011951368892\\
251	0.00568692014856976\\
252	0.00568366692062209\\
253	0.00568036064986266\\
254	0.00567700244381534\\
255	0.0056735936822851\\
256	0.00567013597217538\\
257	0.00566663105804775\\
258	0.00566308066957925\\
259	0.00565948625958952\\
260	0.00565584841263152\\
261	0.00565216609102806\\
262	0.00564843820419253\\
263	0.00564466360485588\\
264	0.00564084108475608\\
265	0.00563696936879612\\
266	0.00563304710399896\\
267	0.00562907285885337\\
268	0.0056250451197385\\
269	0.00562096228485869\\
270	0.00561682265740823\\
271	0.00561262443812102\\
272	0.00560836571714789\\
273	0.00560404446520052\\
274	0.00559965852389711\\
275	0.00559520559524481\\
276	0.00559068323019928\\
277	0.00558608881626838\\
278	0.00558141956421164\\
279	0.00557667249411617\\
280	0.00557184442119394\\
281	0.00556693193155663\\
282	0.0055619313597477\\
283	0.00555683877510902\\
284	0.00555164996184468\\
285	0.00554636039859688\\
286	0.00554096524159194\\
287	0.0055354592881318\\
288	0.00552983693876181\\
289	0.00552409216144583\\
290	0.00551821845284976\\
291	0.00551220880162227\\
292	0.00550605567548976\\
293	0.0054997509562657\\
294	0.00549328589104347\\
295	0.00548665104689526\\
296	0.00547983626585418\\
297	0.00547283062171438\\
298	0.00546562238080637\\
299	0.00545819896974385\\
300	0.00545054695429415\\
301	0.00544265203515902\\
302	0.00543449906859606\\
303	0.0054260721179611\\
304	0.00541735445952625\\
305	0.00540832879420024\\
306	0.00539897752471909\\
307	0.00538928312778538\\
308	0.00537922868237661\\
309	0.00536879860955287\\
310	0.00535797969309182\\
311	0.00534676247257293\\
312	0.00533514313645609\\
313	0.00532312615580551\\
314	0.00531072776264848\\
315	0.00529798033230431\\
316	0.00528509108377234\\
317	0.00527209809328364\\
318	0.00525900434016779\\
319	0.00524581325619099\\
320	0.00523252881885486\\
321	0.00521915545066693\\
322	0.00520569811781198\\
323	0.00519216247741685\\
324	0.00517855496125716\\
325	0.00516488286843877\\
326	0.00515115446023739\\
327	0.00513737901426608\\
328	0.00512356725396317\\
329	0.00510973143323508\\
330	0.00509588529630315\\
331	0.0050820442322894\\
332	0.0050682254535027\\
333	0.00505444827216694\\
334	0.00504073406369509\\
335	0.00502710510947419\\
336	0.00501358570886181\\
337	0.00500020219050444\\
338	0.00498698284779547\\
339	0.00497395774552717\\
340	0.00496115823720337\\
341	0.0049486180502011\\
342	0.00493637445751942\\
343	0.00492446539907476\\
344	0.00491292833816373\\
345	0.00490179860145864\\
346	0.00489110702351452\\
347	0.00488087665843246\\
348	0.0048711182416854\\
349	0.00486182396433311\\
350	0.00485293806156923\\
351	0.00484412957686245\\
352	0.00483540736418802\\
353	0.00482678051127734\\
354	0.00481825828917079\\
355	0.00480985007568382\\
356	0.0048015652690991\\
357	0.00479341318551467\\
358	0.00478540296815216\\
359	0.00477754347686151\\
360	0.00476984314266119\\
361	0.00476230980045775\\
362	0.00475495049790426\\
363	0.0047477712358786\\
364	0.00474077662442931\\
365	0.00473396956968372\\
366	0.00472735093479879\\
367	0.00472091918377157\\
368	0.00471467002319526\\
369	0.00470859606602662\\
370	0.00470268655422674\\
371	0.00469692719506652\\
372	0.0046913001824281\\
373	0.00468578451720068\\
374	0.00468035678206487\\
375	0.00467499258685562\\
376	0.00466967071569014\\
377	0.00466439094468745\\
378	0.00465915263878804\\
379	0.0046539547154853\\
380	0.00464879561007354\\
381	0.00464367324380538\\
382	0.00463858499881921\\
383	0.00463352770164772\\
384	0.0046284976171737\\
385	0.00462349045605455\\
386	0.00461850139899356\\
387	0.00461352514149374\\
388	0.00460855596279575\\
389	0.00460358782241647\\
390	0.00459861448713568\\
391	0.00459362968977339\\
392	0.00458862731836186\\
393	0.00458360162990402\\
394	0.00457854747576893\\
395	0.00457346051477442\\
396	0.0045683373742081\\
397	0.00456317569562472\\
398	0.00455797318520075\\
399	0.0045527274656\\
400	0.00454743609094299\\
401	0.0045420965642578\\
402	0.00453670635728405\\
403	0.00453126293235265\\
404	0.00452576376587907\\
405	0.00452020637278369\\
406	0.00451458833088573\\
407	0.00450890730401451\\
408	0.004503161062266\\
409	0.00449734749752988\\
410	0.00449146463220514\\
411	0.00448551061897748\\
412	0.00447948372980187\\
413	0.00447338233305419\\
414	0.00446720485950733\\
415	0.00446094976085365\\
416	0.00445461549458166\\
417	0.00444820052744019\\
418	0.00444170333848689\\
419	0.00443512242158188\\
420	0.00442845628718682\\
421	0.0044217034633394\\
422	0.00441486249569465\\
423	0.00440793194656331\\
424	0.00440091039293376\\
425	0.0043937964235418\\
426	0.00438658863515213\\
427	0.00437928562833206\\
428	0.00437188600312198\\
429	0.00436438835511852\\
430	0.00435679127254411\\
431	0.00434909333481376\\
432	0.0043412931121523\\
433	0.00433338916506646\\
434	0.00432538004367972\\
435	0.00431726428694371\\
436	0.00430904042174816\\
437	0.00430070696195715\\
438	0.00429226240740638\\
439	0.0042837052429001\\
440	0.00427503393724809\\
441	0.00426624694238005\\
442	0.00425734269256671\\
443	0.0042483196037618\\
444	0.00423917607305786\\
445	0.00422991047822245\\
446	0.00422052117725603\\
447	0.00421100650794284\\
448	0.00420136478739952\\
449	0.00419159431162678\\
450	0.00418169335506834\\
451	0.00417166017018127\\
452	0.00416149298702058\\
453	0.0041511900128395\\
454	0.00414074943170557\\
455	0.00413016940413091\\
456	0.00411944806671382\\
457	0.00410858353178745\\
458	0.00409757388707241\\
459	0.00408641719532977\\
460	0.00407511149401484\\
461	0.00406365479493196\\
462	0.00405204508389014\\
463	0.00404028032035976\\
464	0.00402835843713015\\
465	0.0040162773399675\\
466	0.00400403490727281\\
467	0.00399162898973887\\
468	0.00397905741000597\\
469	0.00396631796231549\\
470	0.00395340841216087\\
471	0.00394032649593539\\
472	0.00392706992057615\\
473	0.00391363636320374\\
474	0.0039000234707567\\
475	0.00388622885961998\\
476	0.00387225011524586\\
477	0.00385808479176625\\
478	0.00384373041159462\\
479	0.00382918446501518\\
480	0.00381444440975717\\
481	0.00379950767055129\\
482	0.0037843716386642\\
483	0.00376903367140757\\
484	0.00375349109161576\\
485	0.00373774118708615\\
486	0.00372178120997454\\
487	0.00370560837613651\\
488	0.0036892198644035\\
489	0.00367261281578044\\
490	0.00365578433254883\\
491	0.00363873147725569\\
492	0.00362145127156468\\
493	0.00360394069494109\\
494	0.00358619668313547\\
495	0.00356821612642407\\
496	0.00354999586755376\\
497	0.00353153269932913\\
498	0.00351282336176382\\
499	0.00349386453870145\\
500	0.00347465285378928\\
501	0.00345518486566005\\
502	0.00343545706214343\\
503	0.00341546585328494\\
504	0.0033952075628955\\
505	0.00337467841828577\\
506	0.00335387453775167\\
507	0.0033327919152645\\
508	0.00331142640167784\\
509	0.00328977368157769\\
510	0.00326782924466955\\
511	0.00324558834792906\\
512	0.00322304596628121\\
513	0.00320019673530399\\
514	0.0031770349151287\\
515	0.00315355442308624\\
516	0.0031297488046761\\
517	0.00310561647545263\\
518	0.00308114887875852\\
519	0.00305633122473592\\
520	0.00303114847792327\\
521	0.00300557837230024\\
522	0.00297960563559337\\
523	0.00295321995832703\\
524	0.00292641113572341\\
525	0.00289917060189006\\
526	0.00287149176898304\\
527	0.00284336974872904\\
528	0.00281480202683791\\
529	0.00278578925927946\\
530	0.00275633684383654\\
531	0.00272647973443768\\
532	0.0026962839166467\\
533	0.00266568859564292\\
534	0.00263458148689847\\
535	0.00260281655666955\\
536	0.00257033981782914\\
537	0.00253709333153114\\
538	0.00250302847963458\\
539	0.00246809247793383\\
540	0.00243223002778302\\
541	0.00239538281627027\\
542	0.00235749417763278\\
543	0.00231849814622264\\
544	0.00227831752274385\\
545	0.00223686152525975\\
546	0.00219399064469931\\
547	0.00215238099906579\\
548	0.00211316779448829\\
549	0.00207391789185168\\
550	0.00203412739027735\\
551	0.00199374047816936\\
552	0.0019527641731985\\
553	0.00191122226322775\\
554	0.00186914804010064\\
555	0.00182658423727254\\
556	0.00178358468438323\\
557	0.00174021795680351\\
558	0.00169656963512825\\
559	0.00165305721170727\\
560	0.0016107056490028\\
561	0.00156824435218013\\
562	0.00152551367306318\\
563	0.0014825422894469\\
564	0.00143936045003831\\
565	0.0013959977150654\\
566	0.00135261153787214\\
567	0.00130946074191574\\
568	0.00126596368633984\\
569	0.00122213770752465\\
570	0.00117800354519282\\
571	0.0011335841768772\\
572	0.00108890495848566\\
573	0.00104399377314213\\
574	0.000998881179100837\\
575	0.000953600553171082\\
576	0.000908188225221049\\
577	0.00086268359801868\\
578	0.000817129244997649\\
579	0.000771570976441746\\
580	0.000726057861961415\\
581	0.000680642193874474\\
582	0.000635379372066665\\
583	0.000590327685971726\\
584	0.000545547963424773\\
585	0.000501103049522709\\
586	0.000457057072276937\\
587	0.000413474448977734\\
588	0.000370418597003141\\
589	0.000327950360562709\\
590	0.000286126318718803\\
591	0.000244997582319807\\
592	0.000204610904245316\\
593	0.000165017211836693\\
594	0.000126301460681324\\
595	8.8816120310521e-05\\
596	5.341348955744e-05\\
597	2.21100055488408e-05\\
598	0\\
599	0\\
600	0\\
};
\addplot [color=mycolor2,solid,forget plot]
  table[row sep=crcr]{%
1	0.00590289298809553\\
2	0.00590283980942159\\
3	0.00590278557316541\\
4	0.00590273026010499\\
5	0.00590267385073519\\
6	0.00590261632526622\\
7	0.00590255766362219\\
8	0.0059024978454396\\
9	0.00590243685006619\\
10	0.00590237465655978\\
11	0.00590231124368723\\
12	0.00590224658992364\\
13	0.00590218067345159\\
14	0.00590211347216063\\
15	0.00590204496364685\\
16	0.00590197512521264\\
17	0.00590190393386664\\
18	0.0059018313663239\\
19	0.00590175739900615\\
20	0.00590168200804236\\
21	0.00590160516926949\\
22	0.00590152685823341\\
23	0.00590144705018995\\
24	0.00590136572010657\\
25	0.00590128284266371\\
26	0.00590119839225697\\
27	0.00590111234299898\\
28	0.00590102466872212\\
29	0.00590093534298097\\
30	0.00590084433905542\\
31	0.00590075162995403\\
32	0.00590065718841745\\
33	0.00590056098692247\\
34	0.00590046299768619\\
35	0.00590036319267056\\
36	0.00590026154358736\\
37	0.0059001580219034\\
38	0.00590005259884612\\
39	0.00589994524540961\\
40	0.00589983593236088\\
41	0.00589972463024672\\
42	0.00589961130940069\\
43	0.0058994959399507\\
44	0.00589937849182696\\
45	0.00589925893477018\\
46	0.00589913723834041\\
47	0.00589901337192616\\
48	0.00589888730475396\\
49	0.0058987590058984\\
50	0.00589862844429249\\
51	0.00589849558873845\\
52	0.00589836040791912\\
53	0.00589822287040956\\
54	0.0058980829446892\\
55	0.00589794059915432\\
56	0.00589779580213108\\
57	0.00589764852188881\\
58	0.00589749872665372\\
59	0.00589734638462316\\
60	0.0058971914639801\\
61	0.00589703393290788\\
62	0.00589687375960556\\
63	0.00589671091230352\\
64	0.00589654535927913\\
65	0.00589637706887303\\
66	0.0058962060095055\\
67	0.00589603214969311\\
68	0.00589585545806554\\
69	0.00589567590338246\\
70	0.00589549345455087\\
71	0.00589530808064215\\
72	0.00589511975090933\\
73	0.00589492843480444\\
74	0.00589473410199556\\
75	0.00589453672238405\\
76	0.00589433626612137\\
77	0.00589413270362589\\
78	0.0058939260055991\\
79	0.00589371614304192\\
80	0.00589350308727014\\
81	0.00589328680992961\\
82	0.00589306728301068\\
83	0.00589284447886212\\
84	0.0058926183702043\\
85	0.00589238893014128\\
86	0.00589215613217215\\
87	0.00589191995020146\\
88	0.00589168035854821\\
89	0.00589143733195379\\
90	0.00589119084558862\\
91	0.00589094087505732\\
92	0.00589068739640216\\
93	0.00589043038610515\\
94	0.00589016982108816\\
95	0.00588990567871135\\
96	0.00588963793676946\\
97	0.0058893665734861\\
98	0.00588909156750609\\
99	0.00588881289788516\\
100	0.00588853054407754\\
101	0.00588824448592126\\
102	0.00588795470362059\\
103	0.00588766117772615\\
104	0.00588736388911231\\
105	0.00588706281895193\\
106	0.0058867579486883\\
107	0.00588644926000442\\
108	0.00588613673478923\\
109	0.00588582035510137\\
110	0.00588550010312994\\
111	0.0058851759611525\\
112	0.00588484791149027\\
113	0.00588451593646074\\
114	0.00588418001832737\\
115	0.00588384013924701\\
116	0.00588349628121432\\
117	0.00588314842600411\\
118	0.00588279655511109\\
119	0.00588244064968738\\
120	0.00588208069047766\\
121	0.00588171665775221\\
122	0.00588134853123793\\
123	0.00588097629004714\\
124	0.00588059991260446\\
125	0.00588021937657165\\
126	0.00587983465877019\\
127	0.00587944573510217\\
128	0.00587905258046838\\
129	0.00587865516868465\\
130	0.00587825347239518\\
131	0.00587784746298339\\
132	0.00587743711047974\\
133	0.00587702238346651\\
134	0.00587660324897912\\
135	0.00587617967240413\\
136	0.00587575161737366\\
137	0.00587531904565667\\
138	0.00587488191704753\\
139	0.00587444018925267\\
140	0.0058739938177769\\
141	0.00587354275581134\\
142	0.00587308695412598\\
143	0.00587262636097098\\
144	0.00587216092199153\\
145	0.00587169058016306\\
146	0.00587121527575362\\
147	0.00587073494632138\\
148	0.00587024952675232\\
149	0.00586975894933288\\
150	0.00586926314382588\\
151	0.00586876203750937\\
152	0.00586825555561693\\
153	0.00586774362158243\\
154	0.00586722615700598\\
155	0.00586670308159826\\
156	0.0058661743131236\\
157	0.00586563976734193\\
158	0.00586509935794921\\
159	0.00586455299651673\\
160	0.00586400059242919\\
161	0.0058634420528215\\
162	0.00586287728251447\\
163	0.00586230618394927\\
164	0.00586172865712087\\
165	0.00586114459951034\\
166	0.00586055390601624\\
167	0.00585995646888486\\
168	0.00585935217763975\\
169	0.00585874091901054\\
170	0.00585812257686064\\
171	0.00585749703211461\\
172	0.0058568641626848\\
173	0.00585622384339763\\
174	0.00585557594591947\\
175	0.00585492033868218\\
176	0.00585425688680876\\
177	0.00585358545203869\\
178	0.00585290589265376\\
179	0.00585221806340392\\
180	0.00585152181543385\\
181	0.0058508169962099\\
182	0.005850103449448\\
183	0.00584938101504254\\
184	0.00584864952899625\\
185	0.00584790882335182\\
186	0.00584715872612471\\
187	0.00584639906123796\\
188	0.00584562964845907\\
189	0.00584485030333891\\
190	0.00584406083715339\\
191	0.0058432610568476\\
192	0.00584245076498301\\
193	0.00584162975968797\\
194	0.00584079783461143\\
195	0.00583995477888071\\
196	0.00583910037706289\\
197	0.00583823440913059\\
198	0.00583735665043217\\
199	0.00583646687166668\\
200	0.00583556483886362\\
201	0.0058346503133681\\
202	0.00583372305183121\\
203	0.0058327828062062\\
204	0.00583182932375047\\
205	0.00583086234703337\\
206	0.00582988161395089\\
207	0.00582888685774623\\
208	0.00582787780703734\\
209	0.00582685418585128\\
210	0.00582581571366544\\
211	0.00582476210545613\\
212	0.0058236930717545\\
213	0.00582260831870985\\
214	0.00582150754816091\\
215	0.00582039045771465\\
216	0.00581925674083349\\
217	0.00581810608693054\\
218	0.00581693818147348\\
219	0.00581575270609708\\
220	0.00581454933872472\\
221	0.00581332775369906\\
222	0.00581208762192244\\
223	0.00581082861100702\\
224	0.00580955038543486\\
225	0.00580825260672872\\
226	0.00580693493363345\\
227	0.0058055970223081\\
228	0.00580423852652999\\
229	0.00580285909790977\\
230	0.00580145838611824\\
231	0.00580003603911752\\
232	0.00579859170336768\\
233	0.00579712502394831\\
234	0.0057956356448991\\
235	0.00579412320947861\\
236	0.00579258736038246\\
237	0.0057910277399373\\
238	0.00578944399025673\\
239	0.00578783575334226\\
240	0.00578620267110513\\
241	0.00578454438528075\\
242	0.00578286053719846\\
243	0.005781150767361\\
244	0.00577941471477828\\
245	0.00577765201598826\\
246	0.00577586230368574\\
247	0.00577404520486917\\
248	0.00577220033840756\\
249	0.00577032731193253\\
250	0.00576842571797302\\
251	0.00576649512919416\\
252	0.00576453509242091\\
253	0.00576254512176232\\
254	0.00576052469084514\\
255	0.00575847322438624\\
256	0.00575639008963429\\
257	0.0057542745884814\\
258	0.00575212595030449\\
259	0.00574994332611573\\
260	0.00574772580515928\\
261	0.00574547244434873\\
262	0.00574318226705595\\
263	0.00574085426181224\\
264	0.00573848738082318\\
265	0.00573608053819416\\
266	0.0057336326090271\\
267	0.00573114242822554\\
268	0.00572860878907956\\
269	0.00572603044179799\\
270	0.00572340609201073\\
271	0.00572073439924577\\
272	0.00571801397538682\\
273	0.00571524338311896\\
274	0.00571242113437221\\
275	0.00570954568877621\\
276	0.00570661545214432\\
277	0.0057036287750148\\
278	0.00570058395127895\\
279	0.00569747921684793\\
280	0.00569431274780722\\
281	0.00569108265911777\\
282	0.00568778700431337\\
283	0.00568442377512202\\
284	0.00568099090152077\\
285	0.00567748625235599\\
286	0.005673907634917\\
287	0.00567025279523492\\
288	0.00566651941947826\\
289	0.00566270513625548\\
290	0.00565880752048726\\
291	0.00565482409980961\\
292	0.00565075235857844\\
293	0.00564658974480492\\
294	0.00564233367934105\\
295	0.00563798156718042\\
296	0.00563353081119124\\
297	0.00562897882862025\\
298	0.00562432307071964\\
299	0.00561956104584457\\
300	0.0056146903463208\\
301	0.00560970867915591\\
302	0.00560461389963621\\
303	0.00559940404330536\\
304	0.0055940773734678\\
305	0.00558863242917835\\
306	0.00558306807195979\\
307	0.00557738353058685\\
308	0.00557157843999121\\
309	0.00556565286800116\\
310	0.00555960732095007\\
311	0.00555344271627002\\
312	0.00554716030791843\\
313	0.00554076152678147\\
314	0.00553424768685379\\
315	0.00552761954265008\\
316	0.00552087636392247\\
317	0.00551401375131407\\
318	0.0055070263749434\\
319	0.00549990849730939\\
320	0.00549265392265208\\
321	0.00548525595894545\\
322	0.00547770738267522\\
323	0.00547000039348945\\
324	0.00546212656341064\\
325	0.00545407678091484\\
326	0.00544584119101884\\
327	0.00543740916390555\\
328	0.00542876923151705\\
329	0.00541990900561854\\
330	0.00541081511062264\\
331	0.00540147312116253\\
332	0.00539186750342838\\
333	0.00538198152648837\\
334	0.00537179712695583\\
335	0.00536129497651668\\
336	0.0053504545039735\\
337	0.00533925394283664\\
338	0.00532767043787667\\
339	0.00531568024517913\\
340	0.00530325917823541\\
341	0.00529038316818771\\
342	0.00527702871586357\\
343	0.00526317376585702\\
344	0.0052487989607292\\
345	0.00523388935754593\\
346	0.0052184367554935\\
347	0.00520244283120165\\
348	0.00518592334981263\\
349	0.00516891387837378\\
350	0.00515149700577763\\
351	0.00513401777571863\\
352	0.00511648800722797\\
353	0.00509892097993325\\
354	0.0050813315129457\\
355	0.00506373666185014\\
356	0.00504615572162247\\
357	0.00502861053340402\\
358	0.00501112428391625\\
359	0.00499372163377343\\
360	0.0049764294647856\\
361	0.00495927686281922\\
362	0.00494229485614734\\
363	0.00492551797085755\\
364	0.00490898678737586\\
365	0.00489274506618564\\
366	0.00487683956578182\\
367	0.00486131966195579\\
368	0.00484623668923569\\
369	0.0048316428977439\\
370	0.00481758987942017\\
371	0.00480412626590053\\
372	0.00479129461440321\\
373	0.00477912694617862\\
374	0.00476763859951386\\
375	0.00475681984758731\\
376	0.00474658792067708\\
377	0.00473650845189604\\
378	0.00472659321758804\\
379	0.00471685387017981\\
380	0.00470730179998657\\
381	0.00469794797234422\\
382	0.00468880263451389\\
383	0.00467987500589663\\
384	0.00467117296451996\\
385	0.00466270269153025\\
386	0.00465446827488359\\
387	0.00464647127685829\\
388	0.00463871027524485\\
389	0.00463118039627614\\
390	0.00462387286196124\\
391	0.00461677459336521\\
392	0.00460986793314097\\
393	0.00460313057695795\\
394	0.00459653584704915\\
395	0.00459005349898099\\
396	0.00458365131539413\\
397	0.00457729784580636\\
398	0.00457098698920859\\
399	0.004564716294561\\
400	0.00455848264931132\\
401	0.00455228225476963\\
402	0.00454611061339167\\
403	0.00453996253161096\\
404	0.00453383214232365\\
405	0.00452771295147087\\
406	0.00452159791349731\\
407	0.00451547954042655\\
408	0.00450935004856175\\
409	0.00450320154540282\\
410	0.00449702625590228\\
411	0.00449081678246951\\
412	0.00448456638528331\\
413	0.00447826925709325\\
414	0.00447192074868736\\
415	0.00446551747447454\\
416	0.00445905650867842\\
417	0.00445253485727306\\
418	0.00444594948123969\\
419	0.0044392973225393\\
420	0.00443257533232113\\
421	0.0044257805006135\\
422	0.00441890988641957\\
423	0.00441196064676609\\
424	0.00440493006287965\\
425	0.00439781556128897\\
426	0.00439061472735234\\
427	0.00438332530857979\\
428	0.00437594520532298\\
429	0.00436847244719166\\
430	0.004360905155297\\
431	0.00435324149367675\\
432	0.00434547963732502\\
433	0.00433761777577844\\
434	0.00432965411601971\\
435	0.00432158688452897\\
436	0.00431341432832667\\
437	0.0043051347148801\\
438	0.00429674633079376\\
439	0.0042882474792748\\
440	0.00427963647646285\\
441	0.00427091164683658\\
442	0.00426207131805377\\
443	0.00425311381573304\\
444	0.00424403745881409\\
445	0.00423484055619014\\
446	0.00422552140519882\\
447	0.00421607829096292\\
448	0.00420650948556624\\
449	0.00419681324708218\\
450	0.00418698781848135\\
451	0.00417703142645265\\
452	0.00416694228017938\\
453	0.00415671857011792\\
454	0.00414635846682665\\
455	0.00413586011988973\\
456	0.00412522165696883\\
457	0.00411444118299726\\
458	0.00410351677950335\\
459	0.0040924465040181\\
460	0.00408122838949152\\
461	0.00406986044370007\\
462	0.00405834064865096\\
463	0.00404666695998803\\
464	0.00403483730640359\\
465	0.00402284958905925\\
466	0.00401070168101698\\
467	0.00399839142668\\
468	0.00398591664124104\\
469	0.00397327511013314\\
470	0.00396046458847795\\
471	0.00394748280052504\\
472	0.00393432743907793\\
473	0.00392099616490454\\
474	0.00390748660613059\\
475	0.00389379635761438\\
476	0.0038799229803005\\
477	0.00386586400054992\\
478	0.00385161690944312\\
479	0.00383717916205258\\
480	0.00382254817668082\\
481	0.00380772133405904\\
482	0.00379269597650179\\
483	0.0037774694070118\\
484	0.00376203888832935\\
485	0.00374640164191922\\
486	0.00373055484688748\\
487	0.00371449563881903\\
488	0.00369822110852578\\
489	0.00368172830069326\\
490	0.00366501421241198\\
491	0.00364807579157742\\
492	0.00363090993514002\\
493	0.00361351348718326\\
494	0.00359588323680422\\
495	0.00357801591576661\\
496	0.00355990819589096\\
497	0.00354155668614011\\
498	0.00352295792935061\\
499	0.00350410839855148\\
500	0.00348500449280061\\
501	0.00346564253245645\\
502	0.00344601875378601\\
503	0.00342612930279188\\
504	0.00340597022811835\\
505	0.00338553747286873\\
506	0.00336482686513437\\
507	0.00334383410699678\\
508	0.00332255476171738\\
509	0.00330098423877584\\
510	0.00327911777635182\\
511	0.00325695042085306\\
512	0.00323447700299123\\
513	0.00321169210960435\\
514	0.00318859004860879\\
515	0.00316516480378879\\
516	0.00314140998102817\\
517	0.00311731874960481\\
518	0.00309288381940531\\
519	0.00306809749335543\\
520	0.00304295159572964\\
521	0.00301744315427394\\
522	0.00299156088140972\\
523	0.00296528798680148\\
524	0.00293860735335825\\
525	0.00291149560615242\\
526	0.00288393397029197\\
527	0.00285591009287589\\
528	0.00282741204123086\\
529	0.00279842919924932\\
530	0.00276895377662112\\
531	0.0027389801810752\\
532	0.00270850589336545\\
533	0.00267753399434044\\
534	0.00264610521054587\\
535	0.00261426666192251\\
536	0.00258195392813423\\
537	0.00254905409497617\\
538	0.00251541422986857\\
539	0.00248097774510772\\
540	0.00244568021648018\\
541	0.00240947031427079\\
542	0.00237230079634654\\
543	0.00233412157856374\\
544	0.0022948750819944\\
545	0.00225449432082486\\
546	0.00221290115191247\\
547	0.00217000353342318\\
548	0.00212565426290639\\
549	0.0020826812330069\\
550	0.00204203122193233\\
551	0.00200145631970456\\
552	0.00196040402445154\\
553	0.00191878027993334\\
554	0.00187658805460734\\
555	0.00183385596587102\\
556	0.00179062510621794\\
557	0.00174694704334947\\
558	0.00170288693239941\\
559	0.00165852630810456\\
560	0.001613963685285\\
561	0.0015704372970884\\
562	0.00152743720717162\\
563	0.00148420001088167\\
564	0.00144074240204645\\
565	0.00139709605779737\\
566	0.00135329267834897\\
567	0.00130947115127387\\
568	0.00126596376036952\\
569	0.00122213771256506\\
570	0.00117800354620471\\
571	0.00113358417719133\\
572	0.00108890495861866\\
573	0.00104399377320935\\
574	0.000998881179137393\\
575	0.000953600553191194\\
576	0.000908188225232209\\
577	0.000862683598024717\\
578	0.000817129245000729\\
579	0.000771570976443156\\
580	0.000726057861961947\\
581	0.000680642193874616\\
582	0.000635379372066682\\
583	0.00059032768597172\\
584	0.00054554796342477\\
585	0.000501103049522705\\
586	0.00045705707227693\\
587	0.000413474448977723\\
588	0.000370418597003136\\
589	0.000327950360562706\\
590	0.000286126318718799\\
591	0.000244997582319804\\
592	0.000204610904245314\\
593	0.000165017211836693\\
594	0.000126301460681322\\
595	8.88161203105204e-05\\
596	5.34134895574396e-05\\
597	2.21100055488407e-05\\
598	0\\
599	0\\
600	0\\
};
\addplot [color=mycolor3,solid,forget plot]
  table[row sep=crcr]{%
1	0.00590333806059412\\
2	0.00590329678128906\\
3	0.00590325475384485\\
4	0.00590321196662225\\
5	0.00590316840786256\\
6	0.00590312406568831\\
7	0.00590307892810431\\
8	0.00590303298299859\\
9	0.00590298621814341\\
10	0.00590293862119661\\
11	0.00590289017970282\\
12	0.00590284088109476\\
13	0.00590279071269491\\
14	0.00590273966171686\\
15	0.00590268771526706\\
16	0.00590263486034665\\
17	0.00590258108385331\\
18	0.00590252637258319\\
19	0.00590247071323301\\
20	0.0059024140924023\\
21	0.00590235649659557\\
22	0.00590229791222486\\
23	0.00590223832561214\\
24	0.00590217772299197\\
25	0.00590211609051422\\
26	0.00590205341424684\\
27	0.00590198968017892\\
28	0.00590192487422354\\
29	0.00590185898222115\\
30	0.00590179198994261\\
31	0.00590172388309263\\
32	0.00590165464731318\\
33	0.00590158426818708\\
34	0.00590151273124164\\
35	0.00590144002195226\\
36	0.00590136612574644\\
37	0.0059012910280075\\
38	0.00590121471407871\\
39	0.00590113716926716\\
40	0.00590105837884802\\
41	0.00590097832806862\\
42	0.00590089700215268\\
43	0.00590081438630474\\
44	0.00590073046571417\\
45	0.00590064522555982\\
46	0.00590055865101416\\
47	0.0059004707272478\\
48	0.00590038143943372\\
49	0.00590029077275173\\
50	0.00590019871239278\\
51	0.00590010524356324\\
52	0.00590001035148916\\
53	0.00589991402142042\\
54	0.0058998162386348\\
55	0.00589971698844211\\
56	0.0058996162561879\\
57	0.0058995140272573\\
58	0.00589941028707861\\
59	0.00589930502112669\\
60	0.0058991982149262\\
61	0.00589908985405471\\
62	0.00589897992414526\\
63	0.00589886841088904\\
64	0.0058987553000376\\
65	0.00589864057740471\\
66	0.00589852422886803\\
67	0.00589840624037018\\
68	0.00589828659791972\\
69	0.00589816528759147\\
70	0.00589804229552654\\
71	0.00589791760793174\\
72	0.0058977912110786\\
73	0.00589766309130188\\
74	0.00589753323499729\\
75	0.00589740162861898\\
76	0.00589726825867608\\
77	0.00589713311172873\\
78	0.00589699617438341\\
79	0.00589685743328763\\
80	0.00589671687512365\\
81	0.00589657448660168\\
82	0.00589643025445222\\
83	0.00589628416541746\\
84	0.00589613620624197\\
85	0.00589598636366242\\
86	0.00589583462439656\\
87	0.00589568097513107\\
88	0.00589552540250864\\
89	0.00589536789311421\\
90	0.00589520843345995\\
91	0.00589504700996954\\
92	0.0058948836089615\\
93	0.00589471821663132\\
94	0.00589455081903294\\
95	0.00589438140205905\\
96	0.00589420995142053\\
97	0.00589403645262513\\
98	0.00589386089095484\\
99	0.00589368325144289\\
100	0.00589350351884961\\
101	0.00589332167763755\\
102	0.00589313771194584\\
103	0.0058929516055639\\
104	0.00589276334190443\\
105	0.00589257290397592\\
106	0.00589238027435446\\
107	0.00589218543515516\\
108	0.00589198836800336\\
109	0.00589178905400529\\
110	0.00589158747371858\\
111	0.00589138360712279\\
112	0.00589117743358958\\
113	0.0058909689318533\\
114	0.0058907580799815\\
115	0.00589054485534555\\
116	0.00589032923459194\\
117	0.00589011119361365\\
118	0.00588989070752221\\
119	0.00588966775062016\\
120	0.00588944229637443\\
121	0.00588921431739025\\
122	0.00588898378538588\\
123	0.00588875067116826\\
124	0.00588851494460939\\
125	0.00588827657462387\\
126	0.00588803552914733\\
127	0.00588779177511562\\
128	0.00588754527844537\\
129	0.00588729600401512\\
130	0.00588704391564782\\
131	0.00588678897609389\\
132	0.00588653114701554\\
133	0.0058862703889718\\
134	0.00588600666140452\\
135	0.00588573992262503\\
136	0.00588547012980185\\
137	0.00588519723894899\\
138	0.00588492120491518\\
139	0.0058846419813737\\
140	0.00588435952081299\\
141	0.00588407377452813\\
142	0.00588378469261278\\
143	0.00588349222395189\\
144	0.00588319631621476\\
145	0.00588289691584855\\
146	0.0058825939680713\\
147	0.00588228741686388\\
148	0.00588197720495874\\
149	0.00588166327382249\\
150	0.00588134556363449\\
151	0.0058810240132927\\
152	0.00588069856039914\\
153	0.00588036914123539\\
154	0.00588003569073719\\
155	0.00587969814246919\\
156	0.0058793564285995\\
157	0.00587901047987424\\
158	0.00587866022559217\\
159	0.00587830559357929\\
160	0.0058779465101634\\
161	0.00587758290014899\\
162	0.00587721468679199\\
163	0.00587684179177486\\
164	0.00587646413518159\\
165	0.00587608163547312\\
166	0.00587569420946294\\
167	0.00587530177229281\\
168	0.00587490423740891\\
169	0.00587450151653802\\
170	0.00587409351966432\\
171	0.00587368015500627\\
172	0.00587326132899405\\
173	0.00587283694624714\\
174	0.00587240690955239\\
175	0.00587197111984255\\
176	0.00587152947617501\\
177	0.00587108187571106\\
178	0.00587062821369553\\
179	0.00587016838343667\\
180	0.00586970227628657\\
181	0.00586922978162183\\
182	0.00586875078682447\\
183	0.00586826517726333\\
184	0.00586777283627547\\
185	0.0058672736451478\\
186	0.00586676748309892\\
187	0.00586625422726093\\
188	0.00586573375266123\\
189	0.00586520593220404\\
190	0.00586467063665185\\
191	0.00586412773460643\\
192	0.00586357709248929\\
193	0.00586301857452158\\
194	0.00586245204270305\\
195	0.00586187735679009\\
196	0.00586129437427256\\
197	0.00586070295034923\\
198	0.00586010293790159\\
199	0.0058594941874658\\
200	0.00585887654720249\\
201	0.00585824986286415\\
202	0.0058576139777599\\
203	0.0058569687327172\\
204	0.00585631396604036\\
205	0.00585564951346543\\
206	0.00585497520811081\\
207	0.00585429088042399\\
208	0.00585359635812338\\
209	0.00585289146613499\\
210	0.00585217602652403\\
211	0.00585144985842028\\
212	0.00585071277793774\\
213	0.00584996459808745\\
214	0.00584920512868351\\
215	0.00584843417624176\\
216	0.00584765154387117\\
217	0.0058468570311569\\
218	0.0058460504340353\\
219	0.00584523154466043\\
220	0.00584440015126168\\
221	0.00584355603799238\\
222	0.00584269898476903\\
223	0.00584182876710115\\
224	0.00584094515591158\\
225	0.00584004791734668\\
226	0.00583913681257685\\
227	0.00583821159758714\\
228	0.00583727202295753\\
229	0.00583631783363322\\
230	0.00583534876868337\\
231	0.00583436456104625\\
232	0.00583336493725985\\
233	0.00583234961719841\\
234	0.00583131831379131\\
235	0.00583027073272855\\
236	0.00582920657215461\\
237	0.00582812552235046\\
238	0.00582702726540339\\
239	0.00582591147486398\\
240	0.00582477781539051\\
241	0.00582362594237958\\
242	0.00582245550158399\\
243	0.00582126612871744\\
244	0.00582005744904719\\
245	0.00581882907697657\\
246	0.00581758061562036\\
247	0.00581631165637793\\
248	0.00581502177851218\\
249	0.00581371054874442\\
250	0.00581237752087128\\
251	0.00581102223540446\\
252	0.00580964421927844\\
253	0.00580824298564178\\
254	0.00580681803375392\\
255	0.00580536884900539\\
256	0.00580389490304745\\
257	0.00580239565391754\\
258	0.00580087054616907\\
259	0.00579931901198088\\
260	0.00579774047163374\\
261	0.00579613433334658\\
262	0.00579449999310545\\
263	0.00579283683447723\\
264	0.00579114422841492\\
265	0.00578942153313777\\
266	0.00578766809398388\\
267	0.00578588324324393\\
268	0.00578406629999092\\
269	0.00578221656990779\\
270	0.00578033334511233\\
271	0.00577841590397929\\
272	0.00577646351095954\\
273	0.00577447541639589\\
274	0.00577245085633587\\
275	0.00577038905234164\\
276	0.0057682892112973\\
277	0.00576615052521205\\
278	0.00576397217100841\\
279	0.00576175331026504\\
280	0.00575949308900137\\
281	0.00575719063752106\\
282	0.0057548450702292\\
283	0.0057524554854554\\
284	0.00575002096526562\\
285	0.00574754057514018\\
286	0.00574501336366976\\
287	0.0057424383622854\\
288	0.00573981458499559\\
289	0.00573714102816268\\
290	0.00573441667030899\\
291	0.00573164047160095\\
292	0.00572881137339478\\
293	0.00572592829774257\\
294	0.0057229901467965\\
295	0.00571999580207755\\
296	0.00571694412356491\\
297	0.0057138339485516\\
298	0.00571066409019515\\
299	0.00570743333566944\\
300	0.00570414044377602\\
301	0.00570078414177516\\
302	0.00569736312114746\\
303	0.00569387603342288\\
304	0.00569032148463817\\
305	0.0056866980282371\\
306	0.00568300415638837\\
307	0.00567923828954731\\
308	0.0056753987641211\\
309	0.00567148381823785\\
310	0.00566749157589968\\
311	0.00566342003013474\\
312	0.00565926702490346\\
313	0.00565503023655459\\
314	0.0056507071588506\\
315	0.00564629508481615\\
316	0.00564179112620737\\
317	0.00563719227535485\\
318	0.00563249541323885\\
319	0.0056276973060651\\
320	0.00562279460355457\\
321	0.00561778383818147\\
322	0.00561266142449403\\
323	0.00560742365907418\\
324	0.00560206672146222\\
325	0.00559658667665739\\
326	0.00559097948114829\\
327	0.00558524098817996\\
328	0.00557936695390648\\
329	0.00557335304724402\\
330	0.00556719486278295\\
331	0.00556088793634252\\
332	0.00555442776079585\\
333	0.0055478098039952\\
334	0.00554102954609717\\
335	0.00553408251539891\\
336	0.00552696432800334\\
337	0.00551967073358674\\
338	0.00551219766920138\\
339	0.00550454132719009\\
340	0.00549669821790575\\
341	0.00548866520831701\\
342	0.00548043956988406\\
343	0.00547201901752715\\
344	0.00546340172581629\\
345	0.00545458630710211\\
346	0.00544557172993664\\
347	0.00543635714778947\\
348	0.00542694159604296\\
349	0.00541732346631383\\
350	0.00540749972686067\\
351	0.00539746421660027\\
352	0.00538720469915132\\
353	0.0053767079213199\\
354	0.00536595958104545\\
355	0.00535494423409109\\
356	0.00534364519367905\\
357	0.00533204433172414\\
358	0.00532012201216823\\
359	0.00530785709942542\\
360	0.00529522692650257\\
361	0.0052822072972082\\
362	0.00526877268711841\\
363	0.00525489651118935\\
364	0.0052405510446857\\
365	0.00522570766390407\\
366	0.00521033727861089\\
367	0.00519441096669814\\
368	0.00517790088003339\\
369	0.00516078151427541\\
370	0.00514303147787896\\
371	0.00512463595368377\\
372	0.00510558951866255\\
373	0.0050859006581032\\
374	0.00506559748035046\\
375	0.00504473529900209\\
376	0.00502344156360004\\
377	0.00500218022872259\\
378	0.0049809760460786\\
379	0.00495985578832219\\
380	0.00493884842349805\\
381	0.00491798468283812\\
382	0.00489730159774379\\
383	0.004876841736052\\
384	0.0048566516451108\\
385	0.0048367819424855\\
386	0.00481728729511061\\
387	0.00479822623413121\\
388	0.0047796607313178\\
389	0.00476165542284863\\
390	0.00474427648300055\\
391	0.00472758988880617\\
392	0.00471165883337791\\
393	0.00469654004973934\\
394	0.00468227857998984\\
395	0.00466890043389443\\
396	0.00465640265863999\\
397	0.00464473984308537\\
398	0.00463338476898287\\
399	0.00462226273154278\\
400	0.00461138633557096\\
401	0.00460076702303838\\
402	0.00459041470085884\\
403	0.00458033731668045\\
404	0.00457054038341459\\
405	0.00456102645771964\\
406	0.00455179457840555\\
407	0.00454283968002161\\
408	0.00453415201224889\\
409	0.00452571660673242\\
410	0.00451751287505113\\
411	0.0045095144341123\\
412	0.00450168930621192\\
413	0.00449400071203626\\
414	0.00448640874644793\\
415	0.00447887334794408\\
416	0.00447137650171153\\
417	0.0044639138489833\\
418	0.00445648019262798\\
419	0.00444906949342142\\
420	0.00444167488844161\\
421	0.00443428873727616\\
422	0.00442690270202452\\
423	0.00441950786725293\\
424	0.0044120949047246\\
425	0.00440465428644279\\
426	0.00439717654633382\\
427	0.00438965258573034\\
428	0.00438207400929819\\
429	0.0043744334648178\\
430	0.00436672494026864\\
431	0.00435894394214415\\
432	0.00435108693653489\\
433	0.00434315033914831\\
434	0.00433513054650948\\
435	0.00432702396978867\\
436	0.00431882707029175\\
437	0.00431053639527444\\
438	0.00430214861230272\\
439	0.00429366053991516\\
440	0.00428506917189402\\
441	0.00427637169210207\\
442	0.00426756547672204\\
443	0.00425864808105266\\
444	0.00424961720908999\\
445	0.00424047066643495\\
446	0.00423120630132512\\
447	0.00422182197827159\\
448	0.00421231558137371\\
449	0.00420268501660574\\
450	0.00419292821288889\\
451	0.00418304312179867\\
452	0.00417302771581684\\
453	0.00416287998512681\\
454	0.00415259793307238\\
455	0.00414217957055166\\
456	0.00413162290979472\\
457	0.00412092595815515\\
458	0.00411008671269693\\
459	0.00409910315640782\\
460	0.00408797325670467\\
461	0.0040766949644316\\
462	0.0040652662126725\\
463	0.00405368491540735\\
464	0.00404194896605094\\
465	0.00403005623592155\\
466	0.00401800457269473\\
467	0.00400579179889832\\
468	0.00399341571050184\\
469	0.00398087407564079\\
470	0.00396816463349411\\
471	0.00395528509330173\\
472	0.00394223313347\\
473	0.00392900640067599\\
474	0.00391560250894252\\
475	0.00390201903868778\\
476	0.00388825353575286\\
477	0.00387430351040873\\
478	0.00386016643634252\\
479	0.00384583974961992\\
480	0.00383132084761834\\
481	0.00381660708792198\\
482	0.00380169578716801\\
483	0.00378658421983103\\
484	0.00377126961693291\\
485	0.00375574916466674\\
486	0.00374002000292527\\
487	0.00372407922372343\\
488	0.0037079238695024\\
489	0.00369155093130099\\
490	0.00367495734677861\\
491	0.00365813999807156\\
492	0.00364109570946233\\
493	0.00362382124483922\\
494	0.00360631330492075\\
495	0.0035885685242167\\
496	0.00357058346769363\\
497	0.00355235462710943\\
498	0.00353387841697602\\
499	0.00351515117010466\\
500	0.00349616913268174\\
501	0.00347692845881597\\
502	0.00345742520449019\\
503	0.00343765532084158\\
504	0.00341761464668331\\
505	0.00339729890016923\\
506	0.00337670366948799\\
507	0.00335582440245784\\
508	0.00333465639487335\\
509	0.00331319477743365\\
510	0.00329143450105606\\
511	0.00326937032034516\\
512	0.00324699677495018\\
513	0.00322430816850468\\
514	0.00320129854485634\\
515	0.00317796166130206\\
516	0.00315429095836913\\
517	0.00313027952542767\\
518	0.0031059200590184\\
519	0.00308120481164158\\
520	0.00305612553426673\\
521	0.00303067341577268\\
522	0.00300483906420901\\
523	0.00297861256684646\\
524	0.00295198341446589\\
525	0.00292494538011159\\
526	0.00289748602609065\\
527	0.00286958592024448\\
528	0.00284122506151952\\
529	0.00281237967021063\\
530	0.00278302467603092\\
531	0.00275314503776596\\
532	0.00272272663175835\\
533	0.00269175613661856\\
534	0.00266022413796544\\
535	0.00262812442108877\\
536	0.00259545648095494\\
537	0.00256225561626944\\
538	0.0025285702644702\\
539	0.00249433480316095\\
540	0.00245945308195854\\
541	0.00242374914526642\\
542	0.00238717222854202\\
543	0.00234966450884324\\
544	0.00231117872502218\\
545	0.00227166332173057\\
546	0.00223105961431705\\
547	0.00218930071425861\\
548	0.00214630848721728\\
549	0.00210199113162199\\
550	0.00205619994680202\\
551	0.00201158132110446\\
552	0.0019691256828508\\
553	0.00192715271475019\\
554	0.00188480931366883\\
555	0.0018419273240984\\
556	0.00179850167657468\\
557	0.00175457044864958\\
558	0.00171018229188805\\
559	0.0016654007279087\\
560	0.00162030670009296\\
561	0.00157499906801893\\
562	0.00153015667117478\\
563	0.00148647803572806\\
564	0.00144270425365799\\
565	0.00139873860085343\\
566	0.00135460699404303\\
567	0.00131034339611248\\
568	0.00126603637621005\\
569	0.00122213828276059\\
570	0.00117800358348419\\
571	0.00113358418438274\\
572	0.00108890496075918\\
573	0.00104399377408744\\
574	0.000998881179574082\\
575	0.000953600553427397\\
576	0.000908188225361615\\
577	0.000862683598096681\\
578	0.000817129245039902\\
579	0.000771570976463344\\
580	0.00072605786197129\\
581	0.000680642193878206\\
582	0.000635379372067672\\
583	0.000590327685971876\\
584	0.000545547963424773\\
585	0.000501103049522706\\
586	0.000457057072276932\\
587	0.00041347444897773\\
588	0.000370418597003138\\
589	0.000327950360562707\\
590	0.0002861263187188\\
591	0.000244997582319805\\
592	0.000204610904245315\\
593	0.000165017211836693\\
594	0.000126301460681323\\
595	8.88161203105201e-05\\
596	5.34134895574397e-05\\
597	2.21100055488406e-05\\
598	0\\
599	0\\
600	0\\
};
\addplot [color=mycolor4,solid,forget plot]
  table[row sep=crcr]{%
1	0.00590379526347467\\
2	0.00590376498787609\\
3	0.00590373421525883\\
4	0.00590370293902819\\
5	0.00590367115254342\\
6	0.00590363884911854\\
7	0.00590360602202297\\
8	0.00590357266448247\\
9	0.00590353876967977\\
10	0.00590350433075548\\
11	0.00590346934080889\\
12	0.00590343379289877\\
13	0.00590339768004422\\
14	0.00590336099522564\\
15	0.00590332373138558\\
16	0.00590328588142959\\
17	0.00590324743822709\\
18	0.0059032083946124\\
19	0.00590316874338557\\
20	0.00590312847731336\\
21	0.00590308758913008\\
22	0.00590304607153859\\
23	0.00590300391721115\\
24	0.00590296111879034\\
25	0.00590291766889001\\
26	0.00590287356009605\\
27	0.00590282878496737\\
28	0.00590278333603661\\
29	0.00590273720581107\\
30	0.00590269038677339\\
31	0.00590264287138242\\
32	0.00590259465207385\\
33	0.00590254572126088\\
34	0.00590249607133481\\
35	0.00590244569466576\\
36	0.00590239458360299\\
37	0.00590234273047549\\
38	0.00590229012759227\\
39	0.00590223676724273\\
40	0.00590218264169687\\
41	0.00590212774320538\\
42	0.00590207206399979\\
43	0.00590201559629226\\
44	0.00590195833227557\\
45	0.00590190026412281\\
46	0.00590184138398693\\
47	0.00590178168400025\\
48	0.00590172115627383\\
49	0.00590165979289661\\
50	0.00590159758593454\\
51	0.00590153452742941\\
52	0.00590147060939759\\
53	0.00590140582382857\\
54	0.0059013401626834\\
55	0.00590127361789277\\
56	0.00590120618135503\\
57	0.00590113784493395\\
58	0.00590106860045635\\
59	0.00590099843970934\\
60	0.00590092735443741\\
61	0.0059008553363394\\
62	0.00590078237706507\\
63	0.00590070846821142\\
64	0.00590063360131881\\
65	0.00590055776786686\\
66	0.00590048095926989\\
67	0.00590040316687229\\
68	0.00590032438194349\\
69	0.00590024459567266\\
70	0.00590016379916304\\
71	0.00590008198342608\\
72	0.0058999991393753\\
73	0.00589991525781958\\
74	0.00589983032945659\\
75	0.00589974434486537\\
76	0.00589965729449901\\
77	0.00589956916867689\\
78	0.0058994799575765\\
79	0.00589938965122506\\
80	0.00589929823949085\\
81	0.00589920571207409\\
82	0.00589911205849771\\
83	0.00589901726809768\\
84	0.00589892133001309\\
85	0.00589882423317598\\
86	0.00589872596630087\\
87	0.00589862651787407\\
88	0.00589852587614271\\
89	0.00589842402910346\\
90	0.0058983209644912\\
91	0.00589821666976742\\
92	0.00589811113210831\\
93	0.00589800433839292\\
94	0.00589789627519096\\
95	0.00589778692875054\\
96	0.00589767628498594\\
97	0.005897564329465\\
98	0.00589745104739679\\
99	0.00589733642361903\\
100	0.00589722044258555\\
101	0.00589710308835377\\
102	0.00589698434457225\\
103	0.00589686419446828\\
104	0.00589674262083548\\
105	0.00589661960602164\\
106	0.00589649513191666\\
107	0.0058963691799406\\
108	0.00589624173103197\\
109	0.00589611276563616\\
110	0.0058959822636941\\
111	0.00589585020463123\\
112	0.00589571656734662\\
113	0.0058955813302024\\
114	0.00589544447101334\\
115	0.00589530596703697\\
116	0.00589516579496371\\
117	0.0058950239309073\\
118	0.00589488035039563\\
119	0.00589473502836173\\
120	0.00589458793913506\\
121	0.00589443905643291\\
122	0.00589428835335214\\
123	0.00589413580236104\\
124	0.00589398137529138\\
125	0.00589382504333054\\
126	0.00589366677701375\\
127	0.00589350654621647\\
128	0.00589334432014663\\
129	0.00589318006733699\\
130	0.00589301375563747\\
131	0.00589284535220734\\
132	0.00589267482350716\\
133	0.00589250213529082\\
134	0.00589232725259709\\
135	0.00589215013974123\\
136	0.00589197076030609\\
137	0.0058917890771331\\
138	0.00589160505231279\\
139	0.00589141864717509\\
140	0.00589122982227927\\
141	0.00589103853740335\\
142	0.00589084475153345\\
143	0.00589064842285231\\
144	0.0058904495087279\\
145	0.00589024796570103\\
146	0.00589004374947293\\
147	0.00588983681489199\\
148	0.00588962711593984\\
149	0.00588941460571741\\
150	0.00588919923643322\\
151	0.0058889809593899\\
152	0.00588875972496993\\
153	0.00588853548262112\\
154	0.00588830818084222\\
155	0.00588807776716816\\
156	0.00588784418815529\\
157	0.00588760738936659\\
158	0.00588736731535644\\
159	0.00588712390965559\\
160	0.00588687711475577\\
161	0.00588662687209412\\
162	0.00588637312203766\\
163	0.00588611580386718\\
164	0.00588585485576133\\
165	0.00588559021478023\\
166	0.00588532181684884\\
167	0.00588504959674015\\
168	0.00588477348805809\\
169	0.00588449342321997\\
170	0.00588420933343877\\
171	0.00588392114870485\\
172	0.00588362879776743\\
173	0.00588333220811543\\
174	0.0058830313059581\\
175	0.00588272601620475\\
176	0.00588241626244428\\
177	0.00588210196692384\\
178	0.00588178305052682\\
179	0.00588145943275045\\
180	0.00588113103168215\\
181	0.00588079776397541\\
182	0.00588045954482466\\
183	0.00588011628793912\\
184	0.0058797679055157\\
185	0.00587941430821089\\
186	0.00587905540511135\\
187	0.00587869110370336\\
188	0.00587832130984095\\
189	0.00587794592771265\\
190	0.0058775648598068\\
191	0.00587717800687543\\
192	0.00587678526789627\\
193	0.00587638654003338\\
194	0.00587598171859583\\
195	0.00587557069699474\\
196	0.00587515336669815\\
197	0.00587472961718414\\
198	0.00587429933589181\\
199	0.00587386240817019\\
200	0.00587341871722503\\
201	0.00587296814406326\\
202	0.00587251056743539\\
203	0.00587204586377532\\
204	0.0058715739071381\\
205	0.00587109456913512\\
206	0.00587060771886728\\
207	0.00587011322285538\\
208	0.00586961094496848\\
209	0.00586910074634974\\
210	0.00586858248534013\\
211	0.00586805601739993\\
212	0.00586752119502793\\
213	0.00586697786767869\\
214	0.00586642588167781\\
215	0.00586586508013543\\
216	0.00586529530285779\\
217	0.0058647163862575\\
218	0.00586412816326234\\
219	0.00586353046322277\\
220	0.00586292311181854\\
221	0.00586230593096449\\
222	0.00586167873871565\\
223	0.00586104134917222\\
224	0.0058603935723842\\
225	0.00585973521425634\\
226	0.0058590660764534\\
227	0.00585838595630602\\
228	0.00585769464671747\\
229	0.00585699193607131\\
230	0.00585627760814004\\
231	0.0058555514419952\\
232	0.00585481321192058\\
233	0.00585406268732634\\
234	0.00585329963266529\\
235	0.00585252380735102\\
236	0.00585173496567821\\
237	0.00585093285674526\\
238	0.00585011722437906\\
239	0.00584928780706206\\
240	0.00584844433786135\\
241	0.00584758654436012\\
242	0.00584671414859089\\
243	0.00584582686697076\\
244	0.00584492441023848\\
245	0.00584400648339321\\
246	0.00584307278563476\\
247	0.00584212301030537\\
248	0.00584115684483277\\
249	0.00584017397067318\\
250	0.0058391740632536\\
251	0.00583815679191384\\
252	0.00583712181984754\\
253	0.00583606880403983\\
254	0.00583499739519842\\
255	0.00583390723767195\\
256	0.00583279796934581\\
257	0.00583166922152828\\
258	0.00583052061889368\\
259	0.00582935177936182\\
260	0.00582816231394888\\
261	0.00582695182660875\\
262	0.00582571991406411\\
263	0.00582446616562789\\
264	0.00582319016302086\\
265	0.00582189148017541\\
266	0.00582056968302588\\
267	0.00581922432928647\\
268	0.00581785496821623\\
269	0.00581646114037047\\
270	0.0058150423773379\\
271	0.00581359820146302\\
272	0.0058121281255533\\
273	0.00581063165257026\\
274	0.00580910827530425\\
275	0.00580755747603235\\
276	0.00580597872615834\\
277	0.00580437148583363\\
278	0.00580273520355715\\
279	0.00580106931576133\\
280	0.00579937324638386\\
281	0.00579764640641807\\
282	0.00579588819344376\\
283	0.00579409799113489\\
284	0.00579227516873575\\
285	0.00579041908051904\\
286	0.00578852906522668\\
287	0.00578660444549092\\
288	0.00578464452723698\\
289	0.00578264859906228\\
290	0.0057806159315693\\
291	0.00577854577668241\\
292	0.00577643736693986\\
293	0.00577428991475673\\
294	0.00577210261165694\\
295	0.00576987462747252\\
296	0.0057676051095085\\
297	0.00576529318167107\\
298	0.0057629379435557\\
299	0.00576053846948955\\
300	0.00575809380752012\\
301	0.0057556029783565\\
302	0.0057530649743625\\
303	0.00575047875849766\\
304	0.00574784326322206\\
305	0.00574515738939738\\
306	0.00574242000520803\\
307	0.00573962994513303\\
308	0.00573678600901041\\
309	0.00573388696124936\\
310	0.00573093153023273\\
311	0.00572791840784602\\
312	0.00572484624910994\\
313	0.00572171367197512\\
314	0.00571851925676098\\
315	0.00571526154757381\\
316	0.00571193905303247\\
317	0.00570855024551227\\
318	0.00570509356005693\\
319	0.0057015673933592\\
320	0.00569797010272115\\
321	0.00569430000491495\\
322	0.00569055537497388\\
323	0.00568673444492674\\
324	0.00568283540250686\\
325	0.00567885638989284\\
326	0.00567479550211841\\
327	0.00567065078526745\\
328	0.00566642023460038\\
329	0.00566210179246317\\
330	0.00565769334581952\\
331	0.00565319272315991\\
332	0.00564859769103376\\
333	0.00564390595123254\\
334	0.00563911513671801\\
335	0.00563422280663334\\
336	0.0056292264404622\\
337	0.00562412343135971\\
338	0.00561891107868596\\
339	0.00561358657779683\\
340	0.00560814700602255\\
341	0.00560258930768038\\
342	0.00559691027684483\\
343	0.00559110653741519\\
344	0.00558517452050841\\
345	0.00557911043953141\\
346	0.00557291026378581\\
347	0.00556656969196654\\
348	0.00556008412633157\\
349	0.00555344865423694\\
350	0.00554665802886981\\
351	0.00553970671752919\\
352	0.00553258900139072\\
353	0.00552529898234968\\
354	0.00551783058723264\\
355	0.00551017757296908\\
356	0.00550233352881645\\
357	0.00549429189436004\\
358	0.00548604598673111\\
359	0.00547758902805796\\
360	0.00546891418086784\\
361	0.0054600146012915\\
362	0.00545088349244723\\
363	0.00544151413085002\\
364	0.00543189991725684\\
365	0.00542203443641144\\
366	0.0054119115189281\\
367	0.00540152530146267\\
368	0.00539087027925888\\
369	0.00537994134238283\\
370	0.00536873377960162\\
371	0.00535724320129957\\
372	0.00534546542353583\\
373	0.00533339621655065\\
374	0.00532103087005047\\
375	0.00530836350864288\\
376	0.00529538607474241\\
377	0.00528208573941149\\
378	0.00526844138642053\\
379	0.00525443023495017\\
380	0.00524002780960136\\
381	0.00522520825949498\\
382	0.00520994425757827\\
383	0.0051942068560803\\
384	0.00517796557513657\\
385	0.00516118861285438\\
386	0.00514384318728325\\
387	0.00512589605889187\\
388	0.00510731430512192\\
389	0.00508806649590293\\
390	0.00506812393535674\\
391	0.00504746272422302\\
392	0.00502606664883736\\
393	0.00500393098676999\\
394	0.00498106767928082\\
395	0.00495751007937781\\
396	0.00493332304891941\\
397	0.00490861433986502\\
398	0.0048839535690663\\
399	0.00485946909780619\\
400	0.00483520667819876\\
401	0.00481121659496473\\
402	0.00478755387758255\\
403	0.00476427843156136\\
404	0.00474145504198391\\
405	0.00471915317202083\\
406	0.0046974465831148\\
407	0.0046764126419548\\
408	0.00465613109236378\\
409	0.00463668225776951\\
410	0.00461814400985736\\
411	0.00460058760661798\\
412	0.00458407183862374\\
413	0.00456863464115532\\
414	0.00455428182232229\\
415	0.00454097170777687\\
416	0.00452823578507634\\
417	0.0045157922743315\\
418	0.00450365324477773\\
419	0.00449182858417429\\
420	0.00448032544899743\\
421	0.00446914764735966\\
422	0.00445829496546078\\
423	0.00444776245068988\\
424	0.0044375397014027\\
425	0.00442761020495439\\
426	0.00441795080391462\\
427	0.00440853139863911\\
428	0.00439931504829813\\
429	0.00439025870190537\\
430	0.00438131488528355\\
431	0.00437243479842817\\
432	0.00436358607575379\\
433	0.00435476249912737\\
434	0.00434595686158091\\
435	0.00433716099199814\\
436	0.00432836581422454\\
437	0.00431956144712783\\
438	0.00431073735238494\\
439	0.00430188253598267\\
440	0.00429298580733041\\
441	0.00428403609592639\\
442	0.00427502281877005\\
443	0.00426593628086591\\
444	0.00425676807438704\\
445	0.00424751141678241\\
446	0.00423816133103309\\
447	0.00422871365154483\\
448	0.00421916419305247\\
449	0.00420950879043587\\
450	0.00419974334040273\\
451	0.00418986384340222\\
452	0.00417986644360157\\
453	0.00416974746419458\\
454	0.00415950343478122\\
455	0.00414913110715767\\
456	0.00413862745575603\\
457	0.00412798965942506\\
458	0.00411721506266893\\
459	0.0041063011174757\\
460	0.00409524531237804\\
461	0.00408404515595559\\
462	0.00407269817964037\\
463	0.00406120193911939\\
464	0.00404955401414548\\
465	0.00403775200663431\\
466	0.0040257935370263\\
467	0.00401367623903292\\
468	0.00400139775306435\\
469	0.00398895571884476\\
470	0.00397634776793954\\
471	0.00396357151710422\\
472	0.00395062456343933\\
473	0.00393750448216406\\
474	0.00392420882510719\\
475	0.003910735118964\\
476	0.00389708086335479\\
477	0.00388324352873227\\
478	0.00386922055419525\\
479	0.00385500934527123\\
480	0.0038406072717301\\
481	0.00382601166548202\\
482	0.00381121981859191\\
483	0.00379622898141109\\
484	0.00378103636078207\\
485	0.00376563911822554\\
486	0.00375003436800754\\
487	0.00373421917508312\\
488	0.00371819055291034\\
489	0.00370194546112614\\
490	0.00368548080307145\\
491	0.00366879342314951\\
492	0.00365188010399561\\
493	0.00363473756343218\\
494	0.00361736245117754\\
495	0.00359975134527311\\
496	0.00358190074819062\\
497	0.00356380708258075\\
498	0.0035454666866246\\
499	0.00352687580894453\\
500	0.00350803060302674\\
501	0.0034889271211014\\
502	0.00346956130742128\\
503	0.00344992899087267\\
504	0.0034300258768456\\
505	0.00340984753828324\\
506	0.00338938940582154\\
507	0.00336864675692231\\
508	0.0033476147038928\\
509	0.00332628818067455\\
510	0.0033046619282728\\
511	0.00328273047868595\\
512	0.00326048813718142\\
513	0.00323792896275126\\
514	0.00321504674656414\\
515	0.00319183498821309\\
516	0.00316828686954246\\
517	0.00314439522582985\\
518	0.0031201525141719\\
519	0.00309555077894304\\
520	0.00307058161405028\\
521	0.00304523612154709\\
522	0.00301950486370087\\
523	0.0029933778069748\\
524	0.00296684426210323\\
525	0.00293989282401477\\
526	0.00291251135176726\\
527	0.0028846870297126\\
528	0.00285640633169834\\
529	0.00282765792198794\\
530	0.0027984296333008\\
531	0.00276869876328051\\
532	0.00273844180218386\\
533	0.00270763530767341\\
534	0.00267624608035157\\
535	0.00264425550756493\\
536	0.00261164692964258\\
537	0.00257840551553682\\
538	0.00254451994288678\\
539	0.00250998594321005\\
540	0.00247482286706217\\
541	0.00243910249701945\\
542	0.00240276234870822\\
543	0.0023657389254302\\
544	0.00232785832055229\\
545	0.00228905529707634\\
546	0.00224927220259275\\
547	0.00220844791110207\\
548	0.00216652471415998\\
549	0.00212343499341002\\
550	0.00207910199776249\\
551	0.00203343572283267\\
552	0.00198629221434771\\
553	0.00193978333544741\\
554	0.0018951895419994\\
555	0.00185179939928115\\
556	0.00180818798665882\\
557	0.00176406420818124\\
558	0.00171944480456451\\
559	0.00167435555765107\\
560	0.00162885181046918\\
561	0.00158300584892994\\
562	0.00153690735164017\\
563	0.00149066339943845\\
564	0.00144551975349939\\
565	0.00140111675669786\\
566	0.0013566170163797\\
567	0.00131197910736022\\
568	0.00126723396538844\\
569	0.00122241508511204\\
570	0.00117800790590505\\
571	0.00113358445788582\\
572	0.00108890501157333\\
573	0.00104399378859552\\
574	0.000998881185332254\\
575	0.000953600556240819\\
576	0.000908188226875188\\
577	0.000862683598921914\\
578	0.00081712924549955\\
579	0.000771570976715098\\
580	0.000726057862102571\\
581	0.000680642193939919\\
582	0.0006353793720919\\
583	0.000590327685978706\\
584	0.000545547963425862\\
585	0.000501103049522703\\
586	0.00045705707227693\\
587	0.000413474448977723\\
588	0.000370418597003137\\
589	0.000327950360562706\\
590	0.000286126318718802\\
591	0.000244997582319806\\
592	0.000204610904245315\\
593	0.000165017211836693\\
594	0.000126301460681323\\
595	8.88161203105208e-05\\
596	5.34134895574398e-05\\
597	2.21100055488407e-05\\
598	0\\
599	0\\
600	0\\
};
\addplot [color=mycolor5,solid,forget plot]
  table[row sep=crcr]{%
1	0.00590422946128678\\
2	0.00590420816105343\\
3	0.00590418654129558\\
4	0.00590416459823672\\
5	0.00590414232807503\\
6	0.00590411972698365\\
7	0.00590409679111047\\
8	0.00590407351657813\\
9	0.00590404989948405\\
10	0.00590402593590015\\
11	0.00590400162187299\\
12	0.00590397695342349\\
13	0.00590395192654684\\
14	0.00590392653721233\\
15	0.00590390078136317\\
16	0.00590387465491617\\
17	0.00590384815376174\\
18	0.00590382127376325\\
19	0.00590379401075699\\
20	0.00590376636055167\\
21	0.00590373831892813\\
22	0.00590370988163874\\
23	0.00590368104440714\\
24	0.00590365180292757\\
25	0.00590362215286433\\
26	0.0059035920898512\\
27	0.00590356160949076\\
28	0.00590353070735368\\
29	0.00590349937897794\\
30	0.00590346761986798\\
31	0.00590343542549384\\
32	0.00590340279129017\\
33	0.00590336971265528\\
34	0.00590333618494998\\
35	0.00590330220349647\\
36	0.00590326776357712\\
37	0.00590323286043313\\
38	0.00590319748926315\\
39	0.00590316164522192\\
40	0.00590312532341855\\
41	0.00590308851891512\\
42	0.00590305122672476\\
43	0.00590301344180997\\
44	0.00590297515908073\\
45	0.00590293637339243\\
46	0.00590289707954391\\
47	0.00590285727227515\\
48	0.00590281694626508\\
49	0.00590277609612918\\
50	0.00590273471641702\\
51	0.00590269280160959\\
52	0.00590265034611664\\
53	0.00590260734427394\\
54	0.00590256379034027\\
55	0.00590251967849441\\
56	0.00590247500283201\\
57	0.0059024297573624\\
58	0.00590238393600504\\
59	0.00590233753258622\\
60	0.00590229054083537\\
61	0.00590224295438132\\
62	0.00590219476674854\\
63	0.00590214597135311\\
64	0.00590209656149873\\
65	0.00590204653037243\\
66	0.00590199587104044\\
67	0.00590194457644361\\
68	0.00590189263939302\\
69	0.00590184005256525\\
70	0.00590178680849777\\
71	0.005901732899584\\
72	0.00590167831806837\\
73	0.00590162305604139\\
74	0.00590156710543432\\
75	0.00590151045801415\\
76	0.00590145310537817\\
77	0.00590139503894853\\
78	0.0059013362499669\\
79	0.00590127672948873\\
80	0.00590121646837771\\
81	0.00590115545730012\\
82	0.00590109368671895\\
83	0.00590103114688817\\
84	0.00590096782784679\\
85	0.00590090371941303\\
86	0.00590083881117837\\
87	0.00590077309250141\\
88	0.005900706552502\\
89	0.0059006391800552\\
90	0.00590057096378514\\
91	0.00590050189205903\\
92	0.00590043195298099\\
93	0.00590036113438613\\
94	0.00590028942383433\\
95	0.00590021680860435\\
96	0.00590014327568756\\
97	0.00590006881178215\\
98	0.00589999340328701\\
99	0.00589991703629577\\
100	0.00589983969659083\\
101	0.00589976136963753\\
102	0.00589968204057814\\
103	0.00589960169422615\\
104	0.00589952031506033\\
105	0.005899437887219\\
106	0.00589935439449429\\
107	0.00589926982032635\\
108	0.00589918414779768\\
109	0.00589909735962746\\
110	0.00589900943816588\\
111	0.0058989203653884\\
112	0.00589883012289022\\
113	0.00589873869188047\\
114	0.00589864605317668\\
115	0.00589855218719898\\
116	0.00589845707396431\\
117	0.00589836069308083\\
118	0.0058982630237419\\
119	0.00589816404472036\\
120	0.00589806373436234\\
121	0.00589796207058135\\
122	0.00589785903085215\\
123	0.00589775459220429\\
124	0.00589764873121586\\
125	0.00589754142400693\\
126	0.00589743264623284\\
127	0.00589732237307732\\
128	0.00589721057924555\\
129	0.00589709723895697\\
130	0.00589698232593792\\
131	0.00589686581341395\\
132	0.00589674767410229\\
133	0.00589662788020372\\
134	0.00589650640339444\\
135	0.00589638321481786\\
136	0.00589625828507575\\
137	0.00589613158421963\\
138	0.00589600308174176\\
139	0.00589587274656581\\
140	0.00589574054703761\\
141	0.00589560645091531\\
142	0.00589547042535966\\
143	0.00589533243692397\\
144	0.00589519245154357\\
145	0.00589505043452572\\
146	0.00589490635053855\\
147	0.00589476016360021\\
148	0.00589461183706771\\
149	0.00589446133362589\\
150	0.00589430861527592\\
151	0.00589415364332347\\
152	0.00589399637836669\\
153	0.00589383678028406\\
154	0.00589367480822181\\
155	0.00589351042058109\\
156	0.00589334357500514\\
157	0.00589317422836573\\
158	0.00589300233674966\\
159	0.00589282785544471\\
160	0.0058926507389255\\
161	0.00589247094083871\\
162	0.00589228841398823\\
163	0.00589210311031981\\
164	0.00589191498090522\\
165	0.00589172397592632\\
166	0.0058915300446583\\
167	0.00589133313545283\\
168	0.00589113319572062\\
169	0.00589093017191344\\
170	0.00589072400950573\\
171	0.00589051465297586\\
172	0.00589030204578648\\
173	0.00589008613036476\\
174	0.0058898668480817\\
175	0.00588964413923115\\
176	0.00588941794300805\\
177	0.00588918819748602\\
178	0.0058889548395947\\
179	0.00588871780509582\\
180	0.00588847702855905\\
181	0.0058882324433371\\
182	0.00588798398153998\\
183	0.00588773157400866\\
184	0.0058874751502879\\
185	0.0058872146385985\\
186	0.00588694996580856\\
187	0.00588668105740421\\
188	0.00588640783745937\\
189	0.00588613022860486\\
190	0.00588584815199671\\
191	0.00588556152728348\\
192	0.00588527027257313\\
193	0.00588497430439884\\
194	0.00588467353768407\\
195	0.00588436788570691\\
196	0.00588405726006361\\
197	0.00588374157063139\\
198	0.00588342072553046\\
199	0.00588309463108524\\
200	0.00588276319178499\\
201	0.00588242631024372\\
202	0.00588208388715927\\
203	0.00588173582127211\\
204	0.00588138200932304\\
205	0.0058810223460108\\
206	0.00588065672394878\\
207	0.0058802850336214\\
208	0.00587990716334001\\
209	0.00587952299919832\\
210	0.00587913242502751\\
211	0.00587873532235092\\
212	0.00587833157033847\\
213	0.00587792104576093\\
214	0.00587750362294375\\
215	0.00587707917372098\\
216	0.00587664756738887\\
217	0.00587620867065931\\
218	0.00587576234761343\\
219	0.00587530845965489\\
220	0.00587484686546326\\
221	0.00587437742094727\\
222	0.0058738999791982\\
223	0.00587341439044296\\
224	0.00587292050199739\\
225	0.00587241815821919\\
226	0.00587190720046095\\
227	0.00587138746702281\\
228	0.00587085879310497\\
229	0.00587032101075973\\
230	0.00586977394884321\\
231	0.00586921743296657\\
232	0.00586865128544643\\
233	0.00586807532525476\\
234	0.00586748936796734\\
235	0.00586689322571157\\
236	0.00586628670711273\\
237	0.00586566961723889\\
238	0.00586504175754402\\
239	0.0058644029258093\\
240	0.00586375291608242\\
241	0.00586309151861443\\
242	0.00586241851979424\\
243	0.00586173370208021\\
244	0.00586103684392907\\
245	0.00586032771972136\\
246	0.00585960609968369\\
247	0.00585887174980757\\
248	0.00585812443176398\\
249	0.00585736390281433\\
250	0.00585658991571682\\
251	0.0058558022186289\\
252	0.00585500055500476\\
253	0.00585418466348795\\
254	0.0058533542777985\\
255	0.00585250912661443\\
256	0.00585164893344955\\
257	0.00585077341653201\\
258	0.00584988228867367\\
259	0.00584897525713348\\
260	0.00584805202347618\\
261	0.00584711228342581\\
262	0.00584615572671486\\
263	0.00584518203692854\\
264	0.00584419089134428\\
265	0.00584318196076571\\
266	0.00584215490935187\\
267	0.00584110939444136\\
268	0.00584004506637146\\
269	0.00583896156829252\\
270	0.00583785853597729\\
271	0.00583673559762553\\
272	0.00583559237366397\\
273	0.00583442847654169\\
274	0.00583324351052077\\
275	0.00583203707146267\\
276	0.00583080874661024\\
277	0.00582955811436523\\
278	0.0058282847440624\\
279	0.00582698819574003\\
280	0.00582566801990627\\
281	0.00582432375730181\\
282	0.00582295493865824\\
283	0.00582156108445193\\
284	0.00582014170465437\\
285	0.00581869629847926\\
286	0.005817224354126\\
287	0.00581572534851926\\
288	0.00581419874704411\\
289	0.0058126440032749\\
290	0.00581106055869989\\
291	0.00580944784244093\\
292	0.00580780527096708\\
293	0.00580613224780166\\
294	0.00580442816322209\\
295	0.00580269239395184\\
296	0.00580092430284353\\
297	0.00579912323855193\\
298	0.00579728853519571\\
299	0.00579541951200662\\
300	0.00579351547296668\\
301	0.00579157570643923\\
302	0.00578959948478311\\
303	0.00578758606394963\\
304	0.00578553468306317\\
305	0.00578344456398434\\
306	0.00578131491085511\\
307	0.00577914490962455\\
308	0.00577693372755404\\
309	0.00577468051269665\\
310	0.0057723843933358\\
311	0.00577004447737454\\
312	0.00576765985167197\\
313	0.00576522958131156\\
314	0.00576275270896784\\
315	0.00576022825415838\\
316	0.00575765521239794\\
317	0.00575503255430145\\
318	0.00575235922465439\\
319	0.00574963414144133\\
320	0.00574685619482496\\
321	0.00574402424607719\\
322	0.00574113712646314\\
323	0.0057381936360782\\
324	0.00573519254263726\\
325	0.00573213258018739\\
326	0.0057290124477552\\
327	0.00572583080793719\\
328	0.00572258628541859\\
329	0.00571927746540623\\
330	0.00571590289196394\\
331	0.00571246106628896\\
332	0.00570895044499687\\
333	0.00570536943826348\\
334	0.00570171640787011\\
335	0.00569798966517718\\
336	0.00569418746904691\\
337	0.00569030802371435\\
338	0.00568634947646598\\
339	0.00568230991512193\\
340	0.00567818736560189\\
341	0.00567397978950855\\
342	0.00566968508174158\\
343	0.0056653010681927\\
344	0.00566082550358261\\
345	0.00565625606949576\\
346	0.00565159037262873\\
347	0.0056468259431447\\
348	0.00564196023320115\\
349	0.00563699061497836\\
350	0.00563191438172003\\
351	0.00562672874685184\\
352	0.00562143084105162\\
353	0.0056160177086346\\
354	0.00561048630343991\\
355	0.00560483348405053\\
356	0.0055990560097217\\
357	0.0055931505361707\\
358	0.0055871136104212\\
359	0.00558094166527853\\
360	0.00557463101373626\\
361	0.00556817784129923\\
362	0.00556157819448539\\
363	0.00555482796952004\\
364	0.00554792289961393\\
365	0.00554085854011185\\
366	0.0055336302511991\\
367	0.00552623317787879\\
368	0.00551866222696334\\
369	0.00551091204059506\\
370	0.00550297696512918\\
371	0.00549485102078616\\
372	0.00548652786910693\\
373	0.0054780007812667\\
374	0.00546926261160376\\
375	0.00546030577965059\\
376	0.00545112223515048\\
377	0.0054417035439922\\
378	0.00543204103315766\\
379	0.00542212581691445\\
380	0.00541194884783419\\
381	0.00540150093385621\\
382	0.00539077275646738\\
383	0.00537975491110892\\
384	0.0053684379573355\\
385	0.00535681247564071\\
386	0.00534486912991896\\
387	0.00533259873379562\\
388	0.00531999231713941\\
389	0.00530704116034601\\
390	0.00529373682989795\\
391	0.00528007118827396\\
392	0.00526603633846951\\
393	0.00525162446004344\\
394	0.00523682738499222\\
395	0.00522163623423634\\
396	0.00520604069801403\\
397	0.00519002813110791\\
398	0.00517358147230424\\
399	0.00515667352214649\\
400	0.0051392739056539\\
401	0.00512135042052818\\
402	0.00510286916788613\\
403	0.00508379479328046\\
404	0.00506409088727603\\
405	0.0050437206611807\\
406	0.00502264760897073\\
407	0.00500083663684043\\
408	0.00497825615655347\\
409	0.00495487846400973\\
410	0.00493068335478271\\
411	0.00490566247336895\\
412	0.00487982473353056\\
413	0.00485321070322321\\
414	0.00482589774349329\\
415	0.00479801353163928\\
416	0.0047700960633323\\
417	0.00474249402177076\\
418	0.00471527022582038\\
419	0.00468849313252759\\
420	0.00466223684002734\\
421	0.00463658110994893\\
422	0.00461161107985043\\
423	0.00458741670597206\\
424	0.00456409123097715\\
425	0.00454172905936182\\
426	0.00452042247842255\\
427	0.00450025711351624\\
428	0.00448130550552234\\
429	0.00446361815966903\\
430	0.0044472112158671\\
431	0.0044320496150116\\
432	0.00441776627010051\\
433	0.00440382560375862\\
434	0.00439023849119318\\
435	0.00437701253035674\\
436	0.00436415129386796\\
437	0.00435165353474997\\
438	0.00433951235928766\\
439	0.0043277144036239\\
440	0.00431623907618547\\
441	0.00430505796026616\\
442	0.00429413451603094\\
443	0.00428342428335162\\
444	0.00427287587232464\\
445	0.00426243314551698\\
446	0.00425203914870832\\
447	0.00424166417643199\\
448	0.00423129980799707\\
449	0.00422093651362871\\
450	0.00421056372715538\\
451	0.0042001699674827\\
452	0.00418974301691701\\
453	0.00417927016324874\\
454	0.00416873850982556\\
455	0.00415813535282318\\
456	0.00414744861648833\\
457	0.0041366673238222\\
458	0.00412578205906265\\
459	0.00411478534664403\\
460	0.00410367182457272\\
461	0.00409243667364076\\
462	0.00408107508771231\\
463	0.00406958232175505\\
464	0.00405795374048849\\
465	0.00404618486518619\\
466	0.00403427141549785\\
467	0.00402220934250697\\
468	0.00400999484873307\\
469	0.00399762439062231\\
470	0.00398509465953596\\
471	0.00397240253880971\\
472	0.00395954503782476\\
473	0.0039465192102554\\
474	0.00393332212963109\\
475	0.00391995089169573\\
476	0.00390640261499791\\
477	0.00389267443951113\\
478	0.0038787635231776\\
479	0.00386466703641068\\
480	0.00385038215477614\\
481	0.00383590605030284\\
482	0.00382123588213187\\
483	0.00380636878746327\\
484	0.00379130187393502\\
485	0.00377603221454406\\
486	0.003760556845066\\
487	0.00374487276109667\\
488	0.00372897691474455\\
489	0.00371286621101696\\
490	0.00369653750395411\\
491	0.00367998759257086\\
492	0.00366321321666703\\
493	0.00364621105255499\\
494	0.00362897770872998\\
495	0.00361150972146927\\
496	0.00359380355029271\\
497	0.00357585557315651\\
498	0.00355766208122395\\
499	0.00353921927317872\\
500	0.00352052324904254\\
501	0.00350157000345198\\
502	0.00348235541834172\\
503	0.00346287525497366\\
504	0.00344312514524119\\
505	0.00342310058216884\\
506	0.00340279690951775\\
507	0.00338220931039916\\
508	0.00336133279479212\\
509	0.00334016218585757\\
510	0.00331869210493753\\
511	0.00329691695511973\\
512	0.00327483090324128\\
513	0.00325242786019738\\
514	0.00322970145941672\\
515	0.00320664503336107\\
516	0.00318325158790684\\
517	0.00315951377446843\\
518	0.00313542385972952\\
519	0.00311097369285948\\
520	0.00308615467011548\\
521	0.00306095769677305\\
522	0.00303537314646925\\
523	0.00300939081813563\\
524	0.00298299989065061\\
525	0.00295618887532174\\
526	0.00292894556434193\\
527	0.00290125697427681\\
528	0.00287310928924201\\
529	0.00284448780932446\\
530	0.00281537693173909\\
531	0.00278576021769529\\
532	0.00275562044994857\\
533	0.00272493954434936\\
534	0.00269370660143718\\
535	0.00266189593130707\\
536	0.00262948053655158\\
537	0.00259643339538549\\
538	0.0025627243467191\\
539	0.00252832388409115\\
540	0.00249321388509144\\
541	0.00245737974754387\\
542	0.00242081245681518\\
543	0.00238351439691289\\
544	0.00234558287630616\\
545	0.00230699193288679\\
546	0.00226768390732465\\
547	0.00222753786807174\\
548	0.00218642148282379\\
549	0.00214427776531739\\
550	0.00210103341167732\\
551	0.00205662112099617\\
552	0.0020109660400507\\
553	0.00196397980144884\\
554	0.0019155303457963\\
555	0.00186682514613946\\
556	0.00181969807274834\\
557	0.00177480133178205\\
558	0.00172970914963858\\
559	0.00168431053725827\\
560	0.00163846944513358\\
561	0.00159221208723224\\
562	0.00154559597045618\\
563	0.00149870412838683\\
564	0.00145164000108015\\
565	0.00140490722431574\\
566	0.00135932596872896\\
567	0.00131422382038051\\
568	0.00126906825494275\\
569	0.00122384286451314\\
570	0.00117857875099172\\
571	0.00113361617673813\\
572	0.00108890699726297\\
573	0.00104399414514463\\
574	0.000998881283084228\\
575	0.000953600593730909\\
576	0.000908188244841367\\
577	0.000862683608535594\\
578	0.000817129250711461\\
579	0.000771570979621272\\
580	0.000726057863703377\\
581	0.000680642194784237\\
582	0.000635379372495318\\
583	0.000590327686140574\\
584	0.000545547963472517\\
585	0.000501103049530357\\
586	0.000457057072276932\\
587	0.00041347444897773\\
588	0.000370418597003137\\
589	0.000327950360562706\\
590	0.000286126318718801\\
591	0.000244997582319805\\
592	0.000204610904245315\\
593	0.000165017211836693\\
594	0.000126301460681322\\
595	8.88161203105204e-05\\
596	5.34134895574397e-05\\
597	2.21100055488407e-05\\
598	0\\
599	0\\
600	0\\
};
\addplot [color=mycolor6,solid,forget plot]
  table[row sep=crcr]{%
1	0.00590468340290409\\
2	0.00590466875793651\\
3	0.00590465390703613\\
4	0.00590463884782199\\
5	0.00590462357788963\\
6	0.0059046080948106\\
7	0.00590459239613193\\
8	0.00590457647937567\\
9	0.00590456034203833\\
10	0.00590454398159037\\
11	0.00590452739547542\\
12	0.00590451058110989\\
13	0.00590449353588219\\
14	0.0059044762571521\\
15	0.00590445874225004\\
16	0.00590444098847636\\
17	0.00590442299310058\\
18	0.0059044047533605\\
19	0.00590438626646154\\
20	0.00590436752957581\\
21	0.00590434853984111\\
22	0.00590432929436019\\
23	0.00590430979019961\\
24	0.0059042900243888\\
25	0.0059042699939191\\
26	0.00590424969574257\\
27	0.00590422912677091\\
28	0.00590420828387435\\
29	0.00590418716388034\\
30	0.00590416576357244\\
31	0.00590414407968894\\
32	0.00590412210892162\\
33	0.00590409984791431\\
34	0.00590407729326157\\
35	0.00590405444150706\\
36	0.00590403128914223\\
37	0.00590400783260469\\
38	0.00590398406827658\\
39	0.00590395999248299\\
40	0.00590393560149027\\
41	0.00590391089150424\\
42	0.00590388585866846\\
43	0.00590386049906233\\
44	0.00590383480869931\\
45	0.0059038087835249\\
46	0.00590378241941467\\
47	0.00590375571217231\\
48	0.00590372865752741\\
49	0.00590370125113351\\
50	0.00590367348856571\\
51	0.00590364536531862\\
52	0.00590361687680403\\
53	0.00590358801834854\\
54	0.00590355878519121\\
55	0.00590352917248113\\
56	0.00590349917527505\\
57	0.00590346878853472\\
58	0.00590343800712437\\
59	0.00590340682580811\\
60	0.0059033752392473\\
61	0.00590334324199781\\
62	0.00590331082850727\\
63	0.00590327799311232\\
64	0.00590324473003571\\
65	0.00590321103338355\\
66	0.00590317689714224\\
67	0.00590314231517563\\
68	0.00590310728122199\\
69	0.00590307178889096\\
70	0.00590303583166055\\
71	0.00590299940287396\\
72	0.00590296249573647\\
73	0.00590292510331232\\
74	0.00590288721852144\\
75	0.00590284883413628\\
76	0.00590280994277845\\
77	0.00590277053691555\\
78	0.00590273060885776\\
79	0.0059026901507546\\
80	0.00590264915459143\\
81	0.00590260761218613\\
82	0.00590256551518568\\
83	0.00590252285506267\\
84	0.00590247962311196\\
85	0.00590243581044699\\
86	0.00590239140799638\\
87	0.00590234640650043\\
88	0.00590230079650753\\
89	0.00590225456837046\\
90	0.00590220771224303\\
91	0.00590216021807623\\
92	0.00590211207561474\\
93	0.00590206327439308\\
94	0.00590201380373216\\
95	0.00590196365273521\\
96	0.00590191281028439\\
97	0.00590186126503672\\
98	0.00590180900542042\\
99	0.00590175601963092\\
100	0.00590170229562716\\
101	0.00590164782112746\\
102	0.00590159258360568\\
103	0.00590153657028705\\
104	0.00590147976814426\\
105	0.00590142216389328\\
106	0.00590136374398914\\
107	0.00590130449462175\\
108	0.00590124440171157\\
109	0.00590118345090519\\
110	0.00590112162757103\\
111	0.00590105891679464\\
112	0.0059009953033743\\
113	0.00590093077181617\\
114	0.00590086530632965\\
115	0.00590079889082242\\
116	0.00590073150889556\\
117	0.00590066314383845\\
118	0.00590059377862367\\
119	0.00590052339590161\\
120	0.00590045197799523\\
121	0.00590037950689453\\
122	0.00590030596425081\\
123	0.00590023133137115\\
124	0.00590015558921236\\
125	0.00590007871837515\\
126	0.00590000069909783\\
127	0.00589992151125026\\
128	0.00589984113432733\\
129	0.00589975954744251\\
130	0.00589967672932106\\
131	0.00589959265829342\\
132	0.00589950731228811\\
133	0.00589942066882459\\
134	0.00589933270500623\\
135	0.00589924339751263\\
136	0.00589915272259224\\
137	0.00589906065605457\\
138	0.00589896717326231\\
139	0.00589887224912327\\
140	0.00589877585808217\\
141	0.00589867797411222\\
142	0.00589857857070645\\
143	0.00589847762086901\\
144	0.00589837509710623\\
145	0.00589827097141713\\
146	0.0058981652152841\\
147	0.00589805779966319\\
148	0.00589794869497398\\
149	0.00589783787108906\\
150	0.00589772529732377\\
151	0.00589761094242572\\
152	0.0058974947745641\\
153	0.00589737676131866\\
154	0.00589725686966849\\
155	0.00589713506598066\\
156	0.00589701131599841\\
157	0.00589688558482921\\
158	0.0058967578369325\\
159	0.00589662803610724\\
160	0.00589649614547907\\
161	0.0058963621274872\\
162	0.00589622594387109\\
163	0.0058960875556567\\
164	0.00589594692314268\\
165	0.00589580400588577\\
166	0.00589565876268648\\
167	0.00589551115157401\\
168	0.00589536112979095\\
169	0.00589520865377784\\
170	0.00589505367915706\\
171	0.00589489616071657\\
172	0.00589473605239332\\
173	0.0058945733072562\\
174	0.00589440787748867\\
175	0.0058942397143711\\
176	0.00589406876826257\\
177	0.00589389498858256\\
178	0.0058937183237918\\
179	0.00589353872137333\\
180	0.00589335612781283\\
181	0.00589317048857853\\
182	0.00589298174810084\\
183	0.00589278984975164\\
184	0.00589259473582314\\
185	0.00589239634750617\\
186	0.00589219462486845\\
187	0.00589198950683209\\
188	0.00589178093115097\\
189	0.00589156883438758\\
190	0.00589135315188949\\
191	0.00589113381776557\\
192	0.00589091076486166\\
193	0.00589068392473583\\
194	0.0058904532276335\\
195	0.00589021860246186\\
196	0.00588997997676424\\
197	0.00588973727669382\\
198	0.0058894904269871\\
199	0.00588923935093704\\
200	0.00588898397036569\\
201	0.00588872420559656\\
202	0.00588845997542659\\
203	0.00588819119709763\\
204	0.0058879177862678\\
205	0.00588763965698221\\
206	0.0058873567216434\\
207	0.0058870688909814\\
208	0.00588677607402335\\
209	0.00588647817806277\\
210	0.00588617510862836\\
211	0.00588586676945228\\
212	0.00588555306243819\\
213	0.00588523388762864\\
214	0.00588490914317202\\
215	0.00588457872528902\\
216	0.00588424252823844\\
217	0.0058839004442827\\
218	0.00588355236365237\\
219	0.00588319817451027\\
220	0.00588283776291499\\
221	0.00588247101278344\\
222	0.00588209780585283\\
223	0.00588171802164172\\
224	0.00588133153741025\\
225	0.00588093822811953\\
226	0.00588053796638988\\
227	0.00588013062245819\\
228	0.00587971606413418\\
229	0.0058792941567554\\
230	0.00587886476314115\\
231	0.00587842774354511\\
232	0.0058779829556065\\
233	0.00587753025430005\\
234	0.00587706949188437\\
235	0.00587660051784882\\
236	0.00587612317885884\\
237	0.00587563731869965\\
238	0.00587514277821815\\
239	0.00587463939526331\\
240	0.00587412700462452\\
241	0.00587360543796824\\
242	0.00587307452377284\\
243	0.0058725340872614\\
244	0.00587198395033262\\
245	0.00587142393148978\\
246	0.00587085384576781\\
247	0.00587027350465801\\
248	0.00586968271603117\\
249	0.00586908128405838\\
250	0.00586846900912979\\
251	0.00586784568777148\\
252	0.00586721111255999\\
253	0.00586656507203495\\
254	0.00586590735060949\\
255	0.00586523772847881\\
256	0.00586455598152726\\
257	0.00586386188123273\\
258	0.00586315519456908\\
259	0.0058624356839065\\
260	0.00586170310690962\\
261	0.00586095721643395\\
262	0.00586019776041985\\
263	0.0058594244817849\\
264	0.00585863711831371\\
265	0.00585783540254617\\
266	0.00585701906166307\\
267	0.00585618781737013\\
268	0.00585534138577965\\
269	0.0058544794772901\\
270	0.00585360179646345\\
271	0.00585270804190059\\
272	0.00585179790611422\\
273	0.0058508710753995\\
274	0.0058499272297025\\
275	0.00584896604248601\\
276	0.00584798718059275\\
277	0.00584699030410621\\
278	0.00584597506620858\\
279	0.00584494111303573\\
280	0.00584388808352895\\
281	0.00584281560928368\\
282	0.00584172331439418\\
283	0.00584061081529492\\
284	0.00583947772059793\\
285	0.00583832363092568\\
286	0.00583714813873986\\
287	0.00583595082816481\\
288	0.00583473127480586\\
289	0.0058334890455621\\
290	0.00583222369843335\\
291	0.00583093478232062\\
292	0.00582962183681989\\
293	0.00582828439200871\\
294	0.00582692196822549\\
295	0.00582553407584027\\
296	0.00582412021501751\\
297	0.00582267987546966\\
298	0.00582121253620148\\
299	0.00581971766524496\\
300	0.0058181947193846\\
301	0.00581664314387221\\
302	0.00581506237213053\\
303	0.00581345182544603\\
304	0.00581181091264998\\
305	0.00581013902978819\\
306	0.00580843555977847\\
307	0.00580669987205628\\
308	0.00580493132220713\\
309	0.00580312925158503\\
310	0.00580129298691616\\
311	0.00579942183988817\\
312	0.00579751510672676\\
313	0.00579557206777081\\
314	0.00579359198702791\\
315	0.00579157411171272\\
316	0.00578951767177127\\
317	0.00578742187939196\\
318	0.00578528592850206\\
319	0.00578310899424838\\
320	0.00578089023246223\\
321	0.00577862877910772\\
322	0.0057763237497129\\
323	0.00577397423878248\\
324	0.00577157931918913\\
325	0.00576913804154369\\
326	0.00576664943354335\\
327	0.00576411249929563\\
328	0.00576152621861568\\
329	0.00575888954629525\\
330	0.00575620141134653\\
331	0.00575346071622257\\
332	0.00575066633600049\\
333	0.00574781711752994\\
334	0.0057449118785474\\
335	0.0057419494067554\\
336	0.00573892845886111\\
337	0.00573584775955973\\
338	0.00573270600046262\\
339	0.00572950183898877\\
340	0.00572623389720769\\
341	0.00572290076062817\\
342	0.00571950097692899\\
343	0.0057160330546261\\
344	0.0057124954616689\\
345	0.00570888662395184\\
346	0.00570520492372011\\
347	0.00570144869786388\\
348	0.00569761623608626\\
349	0.00569370577918557\\
350	0.00568971551707695\\
351	0.00568564358665135\\
352	0.00568148806948192\\
353	0.00567724698939166\\
354	0.00567291830988646\\
355	0.00566849993155864\\
356	0.00566398968937646\\
357	0.00565938534980228\\
358	0.00565468460779411\\
359	0.00564988508369076\\
360	0.00564498431980329\\
361	0.00563997977664507\\
362	0.0056348688291516\\
363	0.00562964876276981\\
364	0.0056243167693902\\
365	0.00561886994313852\\
366	0.00561330527605117\\
367	0.00560761965366104\\
368	0.00560180985050265\\
369	0.00559587252554999\\
370	0.00558980421797721\\
371	0.00558360134298044\\
372	0.00557726018776682\\
373	0.00557077690767884\\
374	0.00556414752193276\\
375	0.00555736790745322\\
376	0.00555043379876151\\
377	0.00554334078460109\\
378	0.00553608430173897\\
379	0.00552865962902432\\
380	0.00552106187709049\\
381	0.00551328597684666\\
382	0.00550532666814631\\
383	0.0054971784872554\\
384	0.00548883575262141\\
385	0.00548029254866431\\
386	0.00547154270728018\\
387	0.00546257978652851\\
388	0.005453397044808\\
389	0.00544398741351953\\
390	0.00543434346693309\\
391	0.0054244573877434\\
392	0.00541432092740871\\
393	0.00540392535923411\\
394	0.00539326145515441\\
395	0.00538231946491431\\
396	0.0053710891204171\\
397	0.00535955963133878\\
398	0.00534771972949598\\
399	0.0053355578458907\\
400	0.00532306216854998\\
401	0.00531022068392572\\
402	0.00529702122647642\\
403	0.00528345153681775\\
404	0.00526949932806628\\
405	0.00525515233608147\\
406	0.00524039837866732\\
407	0.00522522542145117\\
408	0.00520962147480793\\
409	0.0051935746538616\\
410	0.00517707320074113\\
411	0.00516010544766586\\
412	0.00514266003806422\\
413	0.00512472503489213\\
414	0.005106287077587\\
415	0.00508733024375371\\
416	0.00506783371395862\\
417	0.00504776645706047\\
418	0.00502709125794974\\
419	0.00500576929221536\\
420	0.00498376072912799\\
421	0.00496102502375408\\
422	0.00493752090317557\\
423	0.00491320597827768\\
424	0.00488804049162593\\
425	0.00486199133725965\\
426	0.00483503887956531\\
427	0.00480717787557458\\
428	0.00477842379352288\\
429	0.0047488212357013\\
430	0.00471845512139913\\
431	0.0046874654866577\\
432	0.00465631307862017\\
433	0.00462561156414618\\
434	0.00459543983872804\\
435	0.00456588365269817\\
436	0.00453703556819766\\
437	0.00450899410422867\\
438	0.00448186264312013\\
439	0.00445574775678373\\
440	0.00443075661614078\\
441	0.00440699314951797\\
442	0.00438455250599638\\
443	0.00436351323469214\\
444	0.00434392639820229\\
445	0.00432580057556547\\
446	0.00430908150370947\\
447	0.00429318860155691\\
448	0.00427768535355789\\
449	0.00426257968591244\\
450	0.00424787480931541\\
451	0.0042335682976457\\
452	0.00421965112901356\\
453	0.00420610673759469\\
454	0.00419291015256491\\
455	0.00418002734009607\\
456	0.00416741491750858\\
457	0.00415502047892401\\
458	0.00414278388644718\\
459	0.00413064002251907\\
460	0.00411852369359921\\
461	0.00410641085200688\\
462	0.00409429084603178\\
463	0.00408215182083046\\
464	0.00406998084925039\\
465	0.00405776412744179\\
466	0.00404548724370351\\
467	0.00403313552625417\\
468	0.00402069446996158\\
469	0.00400815023234441\\
470	0.00399549017400645\\
471	0.0039827033951195\\
472	0.00396978118393515\\
473	0.00395671724057395\\
474	0.00394350602644476\\
475	0.00393014203339979\\
476	0.00391661983939307\\
477	0.00390293416378062\\
478	0.00388907991902475\\
479	0.00387505225478129\\
480	0.00386084658963687\\
481	0.00384645862530778\\
482	0.00383188433818874\\
483	0.00381711994419265\\
484	0.00380216183552631\\
485	0.00378700649341192\\
486	0.0037716504093325\\
487	0.00375609008721532\\
488	0.00374032204350607\\
489	0.00372434280486503\\
490	0.00370814890332141\\
491	0.00369173686887879\\
492	0.00367510321977642\\
493	0.00365824445087879\\
494	0.00364115702097106\\
495	0.00362383734004319\\
496	0.00360628175787332\\
497	0.00358848655523246\\
498	0.00357044793801671\\
499	0.00355216203063861\\
500	0.00353362486868388\\
501	0.00351483239085247\\
502	0.00349578043021364\\
503	0.00347646470480874\\
504	0.00345688080763063\\
505	0.00343702419598986\\
506	0.00341689018024293\\
507	0.00339647391180406\\
508	0.00337577037029195\\
509	0.0033547743495868\\
510	0.00333348044258833\\
511	0.00331188302457041\\
512	0.00328997623502093\\
513	0.00326775395784681\\
514	0.00324520979981679\\
515	0.00322233706710684\\
516	0.00319912873980865\\
517	0.00317557744425976\\
518	0.00315167542305859\\
519	0.00312741450263957\\
520	0.00310278605830833\\
521	0.00307778097667462\\
522	0.00305238961546403\\
523	0.00302660176074863\\
524	0.00300040658172168\\
525	0.00297379258326178\\
526	0.00294674755676089\\
527	0.00291925852992653\\
528	0.00289131171640068\\
529	0.00286289246622052\\
530	0.00283398521723053\\
531	0.00280457344714429\\
532	0.00277463963122367\\
533	0.00274416521425818\\
534	0.00271313060469573\\
535	0.0026815152615177\\
536	0.00264929790275776\\
537	0.00261645655312388\\
538	0.00258297183329461\\
539	0.0025488234470146\\
540	0.00251398192579926\\
541	0.00247841834093031\\
542	0.00244210564234407\\
543	0.00240501362073586\\
544	0.00236712119477741\\
545	0.00232842771539922\\
546	0.00228894492744627\\
547	0.00224873067773624\\
548	0.00220782828421262\\
549	0.00216618570376261\\
550	0.00212373698039233\\
551	0.00208030553153823\\
552	0.00203581006666402\\
553	0.00199018245213893\\
554	0.00194333659588346\\
555	0.00189519383286964\\
556	0.00184564302198585\\
557	0.00179463488713487\\
558	0.00174495241332814\\
559	0.00169716065825053\\
560	0.0016505243091927\\
561	0.00160375639864616\\
562	0.00155667487689093\\
563	0.00150924837142255\\
564	0.00146152330798457\\
565	0.00141358649116827\\
566	0.00136554815600325\\
567	0.00131817961727847\\
568	0.00127200212309758\\
569	0.00122621830059046\\
570	0.00118045788475918\\
571	0.00113470471685007\\
572	0.00108911748476318\\
573	0.00104400833163835\\
574	0.000998883762037553\\
575	0.00095360124786219\\
576	0.000908188487090748\\
577	0.000862683722203797\\
578	0.000817129311225428\\
579	0.000771571012213757\\
580	0.000726057881889793\\
581	0.000680642204855582\\
582	0.000635379377867654\\
583	0.000590327688750045\\
584	0.00054554796454357\\
585	0.000501103049846204\\
586	0.000457057072330234\\
587	0.000413474448977726\\
588	0.000370418597003137\\
589	0.000327950360562707\\
590	0.000286126318718801\\
591	0.000244997582319805\\
592	0.000204610904245315\\
593	0.000165017211836693\\
594	0.000126301460681323\\
595	8.88161203105205e-05\\
596	5.34134895574399e-05\\
597	2.21100055488407e-05\\
598	0\\
599	0\\
600	0\\
};
\addplot [color=mycolor7,solid,forget plot]
  table[row sep=crcr]{%
1	0.00590546841052787\\
2	0.00590545833457366\\
3	0.00590544812051456\\
4	0.00590543776662942\\
5	0.00590542727117264\\
6	0.00590541663237348\\
7	0.00590540584843553\\
8	0.00590539491753609\\
9	0.00590538383782547\\
10	0.00590537260742631\\
11	0.00590536122443303\\
12	0.00590534968691108\\
13	0.00590533799289611\\
14	0.00590532614039341\\
15	0.00590531412737702\\
16	0.00590530195178901\\
17	0.00590528961153871\\
18	0.00590527710450184\\
19	0.0059052644285197\\
20	0.00590525158139828\\
21	0.00590523856090745\\
22	0.00590522536477998\\
23	0.00590521199071064\\
24	0.00590519843635526\\
25	0.00590518469932979\\
26	0.00590517077720924\\
27	0.0059051566675267\\
28	0.0059051423677723\\
29	0.00590512787539216\\
30	0.00590511318778731\\
31	0.00590509830231249\\
32	0.00590508321627515\\
33	0.00590506792693418\\
34	0.00590505243149878\\
35	0.00590503672712734\\
36	0.005905020810926\\
37	0.00590500467994756\\
38	0.00590498833119019\\
39	0.00590497176159597\\
40	0.00590495496804984\\
41	0.00590493794737794\\
42	0.00590492069634647\\
43	0.00590490321166016\\
44	0.00590488548996088\\
45	0.00590486752782616\\
46	0.00590484932176777\\
47	0.00590483086823016\\
48	0.00590481216358893\\
49	0.00590479320414936\\
50	0.00590477398614471\\
51	0.00590475450573475\\
52	0.00590473475900402\\
53	0.00590471474196023\\
54	0.00590469445053262\\
55	0.00590467388057021\\
56	0.00590465302784006\\
57	0.00590463188802554\\
58	0.00590461045672465\\
59	0.0059045887294481\\
60	0.00590456670161748\\
61	0.00590454436856355\\
62	0.00590452172552423\\
63	0.00590449876764281\\
64	0.00590447548996593\\
65	0.00590445188744177\\
66	0.00590442795491788\\
67	0.00590440368713947\\
68	0.00590437907874711\\
69	0.00590435412427487\\
70	0.00590432881814813\\
71	0.00590430315468164\\
72	0.00590427712807726\\
73	0.00590425073242188\\
74	0.00590422396168519\\
75	0.00590419680971755\\
76	0.00590416927024773\\
77	0.00590414133688066\\
78	0.00590411300309512\\
79	0.00590408426224143\\
80	0.00590405510753915\\
81	0.00590402553207467\\
82	0.00590399552879873\\
83	0.00590396509052408\\
84	0.00590393420992294\\
85	0.00590390287952453\\
86	0.0059038710917125\\
87	0.00590383883872226\\
88	0.00590380611263852\\
89	0.00590377290539246\\
90	0.00590373920875905\\
91	0.00590370501435438\\
92	0.00590367031363274\\
93	0.00590363509788389\\
94	0.00590359935822993\\
95	0.0059035630856227\\
96	0.00590352627084043\\
97	0.00590348890448488\\
98	0.00590345097697815\\
99	0.00590341247855951\\
100	0.00590337339928215\\
101	0.00590333372900982\\
102	0.00590329345741355\\
103	0.00590325257396822\\
104	0.00590321106794894\\
105	0.00590316892842748\\
106	0.00590312614426865\\
107	0.00590308270412653\\
108	0.00590303859644063\\
109	0.00590299380943207\\
110	0.00590294833109937\\
111	0.00590290214921464\\
112	0.00590285525131915\\
113	0.00590280762471929\\
114	0.00590275925648192\\
115	0.00590271013343025\\
116	0.00590266024213901\\
117	0.00590260956892993\\
118	0.00590255809986686\\
119	0.0059025058207511\\
120	0.00590245271711619\\
121	0.00590239877422297\\
122	0.0059023439770544\\
123	0.00590228831031021\\
124	0.00590223175840156\\
125	0.00590217430544544\\
126	0.0059021159352592\\
127	0.0059020566313547\\
128	0.00590199637693254\\
129	0.00590193515487609\\
130	0.00590187294774557\\
131	0.00590180973777175\\
132	0.00590174550684978\\
133	0.00590168023653298\\
134	0.00590161390802616\\
135	0.00590154650217936\\
136	0.00590147799948121\\
137	0.00590140838005219\\
138	0.00590133762363802\\
139	0.0059012657096031\\
140	0.00590119261692367\\
141	0.00590111832418158\\
142	0.00590104280955797\\
143	0.00590096605082757\\
144	0.00590088802535353\\
145	0.00590080871008369\\
146	0.00590072808154783\\
147	0.00590064611585666\\
148	0.00590056278870112\\
149	0.00590047807534445\\
150	0.00590039195058673\\
151	0.00590030438875595\\
152	0.00590021536369892\\
153	0.00590012484877209\\
154	0.00590003281683201\\
155	0.00589993924022577\\
156	0.00589984409078112\\
157	0.00589974733979652\\
158	0.00589964895803088\\
159	0.00589954891569313\\
160	0.00589944718243163\\
161	0.00589934372732335\\
162	0.00589923851886275\\
163	0.00589913152495061\\
164	0.00589902271288244\\
165	0.00589891204933697\\
166	0.0058987995003641\\
167	0.00589868503137265\\
168	0.00589856860711831\\
169	0.00589845019169064\\
170	0.00589832974850041\\
171	0.00589820724026663\\
172	0.00589808262900302\\
173	0.00589795587600448\\
174	0.00589782694183337\\
175	0.00589769578630525\\
176	0.00589756236847475\\
177	0.00589742664662077\\
178	0.00589728857823184\\
179	0.00589714811999084\\
180	0.00589700522775965\\
181	0.0058968598565635\\
182	0.00589671196057508\\
183	0.00589656149309826\\
184	0.00589640840655157\\
185	0.00589625265245151\\
186	0.00589609418139532\\
187	0.00589593294304377\\
188	0.00589576888610337\\
189	0.00589560195830838\\
190	0.00589543210640261\\
191	0.00589525927612065\\
192	0.00589508341216907\\
193	0.00589490445820704\\
194	0.00589472235682681\\
195	0.00589453704953374\\
196	0.00589434847672596\\
197	0.00589415657767367\\
198	0.00589396129049823\\
199	0.00589376255215059\\
200	0.00589356029838965\\
201	0.00589335446375991\\
202	0.00589314498156892\\
203	0.0058929317838643\\
204	0.00589271480141018\\
205	0.00589249396366328\\
206	0.0058922691987486\\
207	0.00589204043343456\\
208	0.00589180759310755\\
209	0.00589157060174617\\
210	0.00589132938189475\\
211	0.00589108385463654\\
212	0.00589083393956618\\
213	0.00589057955476151\\
214	0.00589032061675508\\
215	0.00589005704050472\\
216	0.00588978873936371\\
217	0.0058895156250501\\
218	0.00588923760761547\\
219	0.00588895459541305\\
220	0.00588866649506475\\
221	0.005888373211428\\
222	0.00588807464756129\\
223	0.00588777070468935\\
224	0.00588746128216713\\
225	0.00588714627744318\\
226	0.00588682558602215\\
227	0.00588649910142634\\
228	0.00588616671515635\\
229	0.00588582831665084\\
230	0.00588548379324532\\
231	0.00588513303013002\\
232	0.00588477591030671\\
233	0.0058844123145445\\
234	0.00588404212133474\\
235	0.00588366520684475\\
236	0.00588328144487065\\
237	0.00588289070678877\\
238	0.00588249286150653\\
239	0.00588208777541169\\
240	0.0058816753123208\\
241	0.00588125533342637\\
242	0.00588082769724298\\
243	0.00588039225955217\\
244	0.00587994887334623\\
245	0.0058794973887707\\
246	0.00587903765306576\\
247	0.00587856951050643\\
248	0.00587809280234142\\
249	0.00587760736673078\\
250	0.00587711303868243\\
251	0.00587660964998708\\
252	0.00587609702915229\\
253	0.00587557500133476\\
254	0.00587504338827168\\
255	0.00587450200821038\\
256	0.00587395067583698\\
257	0.00587338920220323\\
258	0.00587281739465211\\
259	0.00587223505674193\\
260	0.00587164198816886\\
261	0.00587103798468795\\
262	0.00587042283803267\\
263	0.00586979633583266\\
264	0.00586915826152999\\
265	0.00586850839429349\\
266	0.00586784650893159\\
267	0.00586717237580318\\
268	0.00586648576072665\\
269	0.00586578642488698\\
270	0.00586507412474107\\
271	0.00586434861192062\\
272	0.00586360963313332\\
273	0.00586285693006161\\
274	0.00586209023925931\\
275	0.00586130929204601\\
276	0.00586051381439918\\
277	0.00585970352684377\\
278	0.00585887814433981\\
279	0.00585803737616813\\
280	0.00585718092581171\\
281	0.00585630849083357\\
282	0.00585541976275199\\
283	0.00585451442691236\\
284	0.00585359216235555\\
285	0.00585265264168327\\
286	0.00585169553091936\\
287	0.00585072048936791\\
288	0.00584972716946724\\
289	0.00584871521664013\\
290	0.00584768426914011\\
291	0.00584663395789347\\
292	0.00584556390633723\\
293	0.00584447373025259\\
294	0.00584336303759399\\
295	0.0058422314283138\\
296	0.00584107849418192\\
297	0.00583990381860107\\
298	0.00583870697641697\\
299	0.00583748753372358\\
300	0.00583624504766346\\
301	0.00583497906622257\\
302	0.00583368912802002\\
303	0.0058323747620922\\
304	0.00583103548767137\\
305	0.00582967081395859\\
306	0.00582828023989054\\
307	0.0058268632539002\\
308	0.00582541933367116\\
309	0.00582394794588494\\
310	0.00582244854596126\\
311	0.0058209205777915\\
312	0.00581936347346473\\
313	0.00581777665298484\\
314	0.005816159523978\\
315	0.00581451148138946\\
316	0.00581283190716716\\
317	0.00581112016994243\\
318	0.00580937562470352\\
319	0.00580759761245936\\
320	0.00580578545989234\\
321	0.00580393847899998\\
322	0.00580205596672471\\
323	0.00580013720457074\\
324	0.00579818145820796\\
325	0.00579618797706165\\
326	0.00579415599388751\\
327	0.00579208472433111\\
328	0.00578997336647088\\
329	0.00578782110034475\\
330	0.00578562708745906\\
331	0.00578339047027857\\
332	0.00578111037169665\\
333	0.00577878589448524\\
334	0.00577641612072356\\
335	0.00577400011120417\\
336	0.00577153690481451\\
337	0.00576902551789336\\
338	0.0057664649435628\\
339	0.00576385415103406\\
340	0.00576119208488562\\
341	0.00575847766431253\\
342	0.00575570978234539\\
343	0.00575288730503758\\
344	0.00575000907061851\\
345	0.00574707388861067\\
346	0.00574408053890996\\
347	0.00574102777083051\\
348	0.00573791430212861\\
349	0.00573473881797358\\
350	0.0057314999698727\\
351	0.00572819637454793\\
352	0.00572482661276381\\
353	0.00572138922810651\\
354	0.00571788272571942\\
355	0.00571430557098448\\
356	0.00571065618814261\\
357	0.00570693295885431\\
358	0.00570313422069467\\
359	0.00569925826556355\\
360	0.00569530333800685\\
361	0.00569126763347323\\
362	0.00568714929649077\\
363	0.00568294641875502\\
364	0.00567865703712379\\
365	0.00567427913151291\\
366	0.00566981062268538\\
367	0.00566524936992456\\
368	0.00566059316858385\\
369	0.00565583974752377\\
370	0.00565098676639879\\
371	0.00564603181277991\\
372	0.00564097239908317\\
373	0.00563580595924434\\
374	0.00563052984511503\\
375	0.005625141323137\\
376	0.00561963757057203\\
377	0.00561401567146891\\
378	0.00560827261242815\\
379	0.00560240527782194\\
380	0.00559641044477364\\
381	0.00559028477800437\\
382	0.00558402482443303\\
383	0.00557762700750355\\
384	0.00557108762123924\\
385	0.00556440282402394\\
386	0.00555756863209419\\
387	0.00555058091269631\\
388	0.00554343537716995\\
389	0.00553612757386892\\
390	0.00552865288080679\\
391	0.00552100649796893\\
392	0.00551318343945385\\
393	0.00550517852756039\\
394	0.00549698638669302\\
395	0.00548860143780738\\
396	0.00548001789009351\\
397	0.00547122973279356\\
398	0.00546223072678853\\
399	0.0054530143922605\\
400	0.0054435739945392\\
401	0.00543390252840922\\
402	0.00542399270063053\\
403	0.00541383691020215\\
404	0.0054034272247367\\
405	0.00539275535468412\\
406	0.00538181262392792\\
407	0.00537058992780932\\
408	0.00535907770488959\\
409	0.00534726590909281\\
410	0.00533514398475938\\
411	0.00532270085996921\\
412	0.00530992487134049\\
413	0.00529680372629114\\
414	0.00528332448794985\\
415	0.00526947355874959\\
416	0.00525523674381165\\
417	0.00524059944589341\\
418	0.005225546794567\\
419	0.00521006370752857\\
420	0.00519413491090905\\
421	0.00517774491376843\\
422	0.00516087799672983\\
423	0.00514351850418792\\
424	0.00512565111900571\\
425	0.00510726111113186\\
426	0.00508833397563259\\
427	0.00506885526078299\\
428	0.00504881031382298\\
429	0.00502818381973569\\
430	0.00500695904268403\\
431	0.00498511668385148\\
432	0.0049626327749291\\
433	0.00493947399325718\\
434	0.00491559676440567\\
435	0.00489095299498545\\
436	0.00486549206433014\\
437	0.0048391676680193\\
438	0.00481193842453155\\
439	0.00478376779435718\\
440	0.00475462720371412\\
441	0.00472450027553666\\
442	0.0046933885180738\\
443	0.00466131892900443\\
444	0.0046283541132556\\
445	0.004594605694368\\
446	0.00456025212193193\\
447	0.00452597762992022\\
448	0.00449231447466387\\
449	0.00445936026029413\\
450	0.0044272197346658\\
451	0.00439600413019116\\
452	0.00436582992860276\\
453	0.00433681682754361\\
454	0.00430908461065308\\
455	0.00428274851459142\\
456	0.00425791259423312\\
457	0.00423466049179356\\
458	0.00421304246245616\\
459	0.00419305744549281\\
460	0.00417462853232605\\
461	0.00415690944873209\\
462	0.00413962019997972\\
463	0.00412276501290849\\
464	0.00410634185751169\\
465	0.00409034135471902\\
466	0.00407474568594373\\
467	0.00405952758531734\\
468	0.00404464954627945\\
469	0.00403006343814548\\
470	0.00401571080411634\\
471	0.00400152422913765\\
472	0.00398743032716485\\
473	0.00397335511442664\\
474	0.00395926769111637\\
475	0.00394515546851621\\
476	0.00393100453794348\\
477	0.0039167998492336\\
478	0.00390252546881891\\
479	0.00388816492575132\\
480	0.00387370164967247\\
481	0.00385911949642118\\
482	0.00384440334324048\\
483	0.00382953971373353\\
484	0.00381451735932588\\
485	0.00379932767405619\\
486	0.00378396425980192\\
487	0.00376842072626709\\
488	0.00375269075365844\\
489	0.00373676815550082\\
490	0.00372064693795249\\
491	0.00370432135103731\\
492	0.00368778592633672\\
493	0.00367103549507012\\
494	0.0036540651804571\\
495	0.00363687035928918\\
496	0.00361944659051906\\
497	0.00360178951453377\\
498	0.00358389475219438\\
499	0.00356575790433221\\
500	0.00354737454845765\\
501	0.00352874023234101\\
502	0.00350985046425289\\
503	0.00349070069984416\\
504	0.00347128632590631\\
505	0.00345160264157714\\
506	0.003431644837923\\
507	0.00341140797718071\\
508	0.00339088697317599\\
509	0.0033700765743615\\
510	0.00334897134842124\\
511	0.00332756566518767\\
512	0.00330585367781566\\
513	0.00328382930217281\\
514	0.0032614861944159\\
515	0.00323881772672662\\
516	0.003215816961173\\
517	0.00319247662163953\\
518	0.00316878906372772\\
519	0.00314474624246813\\
520	0.00312033967761277\\
521	0.0030955604162116\\
522	0.00307039899235162\\
523	0.0030448453840286\\
524	0.00301888896717868\\
525	0.00299251846697522\\
526	0.00296572190659903\\
527	0.00293848655382617\\
528	0.00291079886596397\\
529	0.00288264443391518\\
530	0.00285400792652315\\
531	0.00282487303684166\\
532	0.0027952224324324\\
533	0.00276503771232443\\
534	0.00273429937400198\\
535	0.00270298679154721\\
536	0.00267107821079603\\
537	0.00263855077512195\\
538	0.00260538059463947\\
539	0.00257154289936966\\
540	0.00253701234040634\\
541	0.0025017633546709\\
542	0.00246577049818809\\
543	0.00242901399438544\\
544	0.00239146970466376\\
545	0.00235310987813035\\
546	0.00231391248249973\\
547	0.00227387357085289\\
548	0.00223298399921975\\
549	0.00219124395363196\\
550	0.00214866854507098\\
551	0.00210535359027717\\
552	0.00206127734338405\\
553	0.00201637958665321\\
554	0.00197057891757341\\
555	0.00192367985441363\\
556	0.00187561454417354\\
557	0.00182630117890887\\
558	0.00177563615587418\\
559	0.0017234803562672\\
560	0.0016710561887658\\
561	0.00162005618808052\\
562	0.00157114435097566\\
563	0.00152278495823474\\
564	0.00147437765394238\\
565	0.00142573906631173\\
566	0.00137688342679055\\
567	0.00132788753056811\\
568	0.00127885728545426\\
569	0.00123068072321142\\
570	0.00118373042443368\\
571	0.00113727730782157\\
572	0.0010909595635236\\
573	0.00104473750684888\\
574	0.000998981829455881\\
575	0.00095361824078903\\
576	0.000908192826501344\\
577	0.000862685274585634\\
578	0.000817130023225066\\
579	0.000771571389571413\\
580	0.000726058083632912\\
581	0.000680642317466521\\
582	0.000635379440543511\\
583	0.000590327722563027\\
584	0.000545547981242051\\
585	0.000501103056863606\\
586	0.000457057074448636\\
587	0.000413474449346032\\
588	0.000370418597003137\\
589	0.000327950360562705\\
590	0.0002861263187188\\
591	0.000244997582319805\\
592	0.000204610904245314\\
593	0.000165017211836693\\
594	0.000126301460681323\\
595	8.88161203105205e-05\\
596	5.34134895574397e-05\\
597	2.21100055488407e-05\\
598	0\\
599	0\\
600	0\\
};
\addplot [color=mycolor8,solid,forget plot]
  table[row sep=crcr]{%
1	0.00590802353428763\\
2	0.00590801630497429\\
3	0.00590800897473085\\
4	0.00590800154213922\\
5	0.00590799400575707\\
6	0.00590798636411732\\
7	0.00590797861572751\\
8	0.00590797075906941\\
9	0.00590796279259819\\
10	0.00590795471474214\\
11	0.00590794652390183\\
12	0.00590793821844956\\
13	0.00590792979672887\\
14	0.00590792125705373\\
15	0.00590791259770804\\
16	0.0059079038169449\\
17	0.00590789491298587\\
18	0.00590788588402048\\
19	0.00590787672820538\\
20	0.00590786744366364\\
21	0.00590785802848411\\
22	0.00590784848072059\\
23	0.0059078387983912\\
24	0.00590782897947751\\
25	0.00590781902192376\\
26	0.00590780892363614\\
27	0.00590779868248203\\
28	0.00590778829628906\\
29	0.00590777776284428\\
30	0.00590776707989345\\
31	0.0059077562451401\\
32	0.00590774525624463\\
33	0.00590773411082346\\
34	0.0059077228064481\\
35	0.00590771134064428\\
36	0.005907699710891\\
37	0.00590768791461956\\
38	0.00590767594921261\\
39	0.00590766381200326\\
40	0.00590765150027391\\
41	0.00590763901125545\\
42	0.00590762634212604\\
43	0.00590761349001027\\
44	0.00590760045197793\\
45	0.00590758722504313\\
46	0.00590757380616302\\
47	0.00590756019223683\\
48	0.00590754638010476\\
49	0.00590753236654675\\
50	0.00590751814828145\\
51	0.00590750372196487\\
52	0.00590748908418953\\
53	0.00590747423148292\\
54	0.0059074591603065\\
55	0.00590744386705438\\
56	0.00590742834805213\\
57	0.00590741259955545\\
58	0.00590739661774893\\
59	0.00590738039874481\\
60	0.00590736393858155\\
61	0.0059073472332225\\
62	0.00590733027855464\\
63	0.00590731307038715\\
64	0.00590729560445\\
65	0.00590727787639254\\
66	0.00590725988178212\\
67	0.00590724161610255\\
68	0.00590722307475266\\
69	0.00590720425304483\\
70	0.00590718514620346\\
71	0.00590716574936337\\
72	0.00590714605756831\\
73	0.00590712606576926\\
74	0.00590710576882298\\
75	0.00590708516149026\\
76	0.00590706423843425\\
77	0.00590704299421882\\
78	0.00590702142330676\\
79	0.00590699952005819\\
80	0.00590697727872856\\
81	0.00590695469346705\\
82	0.00590693175831464\\
83	0.00590690846720231\\
84	0.00590688481394906\\
85	0.00590686079225999\\
86	0.00590683639572437\\
87	0.00590681161781369\\
88	0.00590678645187956\\
89	0.00590676089115165\\
90	0.00590673492873556\\
91	0.00590670855761073\\
92	0.00590668177062822\\
93	0.00590665456050847\\
94	0.00590662691983911\\
95	0.00590659884107245\\
96	0.00590657031652335\\
97	0.00590654133836662\\
98	0.00590651189863468\\
99	0.00590648198921498\\
100	0.00590645160184743\\
101	0.00590642072812184\\
102	0.00590638935947516\\
103	0.00590635748718875\\
104	0.00590632510238557\\
105	0.00590629219602741\\
106	0.00590625875891178\\
107	0.00590622478166902\\
108	0.00590619025475909\\
109	0.00590615516846852\\
110	0.00590611951290706\\
111	0.00590608327800438\\
112	0.00590604645350658\\
113	0.00590600902897263\\
114	0.00590597099377086\\
115	0.00590593233707497\\
116	0.00590589304786028\\
117	0.00590585311489977\\
118	0.00590581252675971\\
119	0.00590577127179557\\
120	0.00590572933814746\\
121	0.0059056867137355\\
122	0.005905643386255\\
123	0.00590559934317145\\
124	0.00590555457171525\\
125	0.0059055090588763\\
126	0.00590546279139811\\
127	0.00590541575577188\\
128	0.00590536793822994\\
129	0.00590531932473914\\
130	0.00590526990099337\\
131	0.00590521965240594\\
132	0.00590516856410093\\
133	0.00590511662090402\\
134	0.00590506380733209\\
135	0.00590501010758167\\
136	0.00590495550551558\\
137	0.00590489998464754\\
138	0.005904843528124\\
139	0.00590478611870195\\
140	0.00590472773872228\\
141	0.00590466837007603\\
142	0.00590460799416253\\
143	0.00590454659183592\\
144	0.00590448414333835\\
145	0.00590442062821777\\
146	0.00590435602523644\\
147	0.00590429031230012\\
148	0.005904223466523\\
149	0.00590415546486037\\
150	0.00590408628607652\\
151	0.0059040159085373\\
152	0.00590394431020254\\
153	0.00590387146861833\\
154	0.00590379736090922\\
155	0.00590372196377013\\
156	0.00590364525345832\\
157	0.00590356720578494\\
158	0.00590348779610667\\
159	0.00590340699931706\\
160	0.00590332478983781\\
161	0.0059032411416097\\
162	0.00590315602808361\\
163	0.00590306942221118\\
164	0.00590298129643536\\
165	0.0059028916226808\\
166	0.00590280037234397\\
167	0.00590270751628335\\
168	0.00590261302480897\\
169	0.00590251686767241\\
170	0.00590241901405598\\
171	0.005902319432562\\
172	0.00590221809120199\\
173	0.0059021149573854\\
174	0.00590200999790831\\
175	0.00590190317894187\\
176	0.0059017944660205\\
177	0.00590168382402991\\
178	0.00590157121719488\\
179	0.00590145660906684\\
180	0.00590133996251117\\
181	0.00590122123969432\\
182	0.00590110040207058\\
183	0.00590097741036876\\
184	0.00590085222457861\\
185	0.00590072480393673\\
186	0.00590059510691262\\
187	0.00590046309119406\\
188	0.00590032871367252\\
189	0.00590019193042812\\
190	0.00590005269671445\\
191	0.00589991096694289\\
192	0.00589976669466679\\
193	0.00589961983256537\\
194	0.00589947033242719\\
195	0.00589931814513329\\
196	0.00589916322064012\\
197	0.00589900550796218\\
198	0.00589884495515392\\
199	0.0058986815092919\\
200	0.00589851511645607\\
201	0.00589834572171095\\
202	0.00589817326908631\\
203	0.00589799770155759\\
204	0.00589781896102576\\
205	0.00589763698829691\\
206	0.00589745172306129\\
207	0.00589726310387215\\
208	0.00589707106812389\\
209	0.00589687555202988\\
210	0.00589667649059989\\
211	0.00589647381761683\\
212	0.00589626746561332\\
213	0.00589605736584751\\
214	0.00589584344827859\\
215	0.00589562564154159\\
216	0.0058954038729219\\
217	0.00589517806832906\\
218	0.0058949481522701\\
219	0.0058947140478223\\
220	0.00589447567660545\\
221	0.00589423295875337\\
222	0.00589398581288512\\
223	0.00589373415607526\\
224	0.00589347790382388\\
225	0.00589321697002569\\
226	0.00589295126693872\\
227	0.00589268070515217\\
228	0.00589240519355378\\
229	0.0058921246392965\\
230	0.00589183894776449\\
231	0.00589154802253821\\
232	0.00589125176535926\\
233	0.00589095007609415\\
234	0.00589064285269744\\
235	0.00589032999117428\\
236	0.00589001138554192\\
237	0.00588968692779095\\
238	0.0058893565078451\\
239	0.0058890200135209\\
240	0.00588867733048607\\
241	0.00588832834221746\\
242	0.00588797292995786\\
243	0.00588761097267214\\
244	0.00588724234700246\\
245	0.00588686692722262\\
246	0.00588648458519144\\
247	0.00588609519030535\\
248	0.0058856986094498\\
249	0.00588529470695005\\
250	0.00588488334452054\\
251	0.00588446438121353\\
252	0.00588403767336654\\
253	0.00588360307454875\\
254	0.0058831604355063\\
255	0.0058827096041064\\
256	0.00588225042528007\\
257	0.00588178274096392\\
258	0.00588130639004054\\
259	0.0058808212082773\\
260	0.00588032702826426\\
261	0.00587982367935019\\
262	0.00587931098757728\\
263	0.0058787887756143\\
264	0.0058782568626881\\
265	0.00587771506451347\\
266	0.00587716319322116\\
267	0.00587660105728395\\
268	0.00587602846144095\\
269	0.00587544520661932\\
270	0.00587485108985398\\
271	0.00587424590420423\\
272	0.00587362943866753\\
273	0.00587300147808908\\
274	0.00587236180306697\\
275	0.0058717101898511\\
276	0.00587104641023446\\
277	0.00587037023143491\\
278	0.0058696814159676\\
279	0.00586897972152364\\
280	0.00586826490097686\\
281	0.00586753670228778\\
282	0.00586679486839854\\
283	0.00586603913712516\\
284	0.00586526924104771\\
285	0.00586448490739712\\
286	0.00586368585793983\\
287	0.00586287180885939\\
288	0.00586204247063553\\
289	0.00586119754792022\\
290	0.00586033673941081\\
291	0.00585945973772049\\
292	0.00585856622924569\\
293	0.00585765589403042\\
294	0.00585672840562798\\
295	0.00585578343095918\\
296	0.00585482063016807\\
297	0.0058538396564742\\
298	0.00585284015602224\\
299	0.0058518217677283\\
300	0.00585078412312331\\
301	0.00584972684619354\\
302	0.00584864955321813\\
303	0.00584755185260379\\
304	0.00584643334471714\\
305	0.00584529362171465\\
306	0.00584413226737116\\
307	0.00584294885690771\\
308	0.00584174295682031\\
309	0.0058405141247117\\
310	0.00583926190912996\\
311	0.00583798584941846\\
312	0.005836685475584\\
313	0.00583536030819091\\
314	0.00583400985828665\\
315	0.0058326336273461\\
316	0.00583123110714031\\
317	0.00582980177906359\\
318	0.00582834511371602\\
319	0.00582686057065276\\
320	0.00582534759812624\\
321	0.00582380563282054\\
322	0.00582223409957815\\
323	0.00582063241111845\\
324	0.00581899996774759\\
325	0.00581733615705988\\
326	0.00581564035362977\\
327	0.00581391191869461\\
328	0.00581215019982756\\
329	0.00581035453060042\\
330	0.00580852423023593\\
331	0.00580665860324909\\
332	0.00580475693907734\\
333	0.00580281851169891\\
334	0.00580084257923887\\
335	0.00579882838356263\\
336	0.00579677514985606\\
337	0.00579468208619208\\
338	0.00579254838308299\\
339	0.00579037321301806\\
340	0.00578815572998549\\
341	0.00578589506897827\\
342	0.00578359034548299\\
343	0.00578124065495101\\
344	0.00577884507225076\\
345	0.00577640265110077\\
346	0.00577391242348277\\
347	0.00577137339903472\\
348	0.00576878456442045\\
349	0.00576614488267504\\
350	0.0057634532925253\\
351	0.00576070870768393\\
352	0.00575791001611674\\
353	0.00575505607928187\\
354	0.00575214573133892\\
355	0.00574917777832662\\
356	0.00574615099730748\\
357	0.00574306413547791\\
358	0.00573991590924122\\
359	0.00573670500324364\\
360	0.00573343006937158\\
361	0.00573008972569926\\
362	0.00572668255539077\\
363	0.00572320710555429\\
364	0.0057196618860459\\
365	0.00571604536821996\\
366	0.00571235598362319\\
367	0.00570859212262963\\
368	0.00570475213301496\\
369	0.00570083431846369\\
370	0.00569683693700552\\
371	0.00569275819937571\\
372	0.00568859626729262\\
373	0.00568434925165601\\
374	0.0056800152106995\\
375	0.00567559214803389\\
376	0.00567107801059165\\
377	0.00566647068646577\\
378	0.00566176800261187\\
379	0.00565696772243323\\
380	0.00565206754324283\\
381	0.00564706509362413\\
382	0.00564195793067386\\
383	0.00563674353710629\\
384	0.00563141931820979\\
385	0.0056259825986447\\
386	0.00562043061907116\\
387	0.0056147605326124\\
388	0.0056089694011292\\
389	0.00560305419127803\\
390	0.00559701177033551\\
391	0.00559083890180946\\
392	0.00558453224096251\\
393	0.00557808833003779\\
394	0.00557150359318984\\
395	0.00556477433087966\\
396	0.00555789671398568\\
397	0.00555086677762797\\
398	0.00554368041457364\\
399	0.00553633336827129\\
400	0.00552882122551482\\
401	0.00552113940871941\\
402	0.00551328316777066\\
403	0.00550524757136001\\
404	0.00549702749793597\\
405	0.00548861762611267\\
406	0.00548001242419854\\
407	0.0054712061408935\\
408	0.00546219279608239\\
409	0.00545296617182449\\
410	0.00544351980363581\\
411	0.00543384696606244\\
412	0.00542394066082811\\
413	0.00541379360523075\\
414	0.00540339821817247\\
415	0.00539274660862177\\
416	0.00538183056359896\\
417	0.0053706415307633\\
418	0.00535917059686725\\
419	0.00534740845962554\\
420	0.00533534539380005\\
421	0.00532297122059882\\
422	0.00531027529883268\\
423	0.00529724651062267\\
424	0.00528387323452439\\
425	0.00527014327336349\\
426	0.0052560438045496\\
427	0.00524156133578352\\
428	0.0052266816635043\\
429	0.00521138983961713\\
430	0.00519567015480811\\
431	0.00517950613184211\\
432	0.00516288056946737\\
433	0.00514577563265416\\
434	0.00512817288665004\\
435	0.0051100535863273\\
436	0.00509139925137241\\
437	0.00507219164355963\\
438	0.00505241258518674\\
439	0.00503204397804265\\
440	0.00501106781927507\\
441	0.00498946615169442\\
442	0.00496722091004136\\
443	0.00494431361570609\\
444	0.00492072489886094\\
445	0.00489643408364169\\
446	0.00487141414822884\\
447	0.00484562979888623\\
448	0.00481903642213972\\
449	0.00479158537430365\\
450	0.004763228102476\\
451	0.00473391756542243\\
452	0.00470361022057111\\
453	0.0046722687614305\\
454	0.00463986584641357\\
455	0.0046063891318761\\
456	0.00457184809079168\\
457	0.00453628306388062\\
458	0.00449977717874501\\
459	0.00446247211174984\\
460	0.00442458897286291\\
461	0.00438708476217384\\
462	0.00435035239775994\\
463	0.00431450735850626\\
464	0.00427967241906488\\
465	0.00424597631785417\\
466	0.00421355160993508\\
467	0.00418253137301872\\
468	0.00415304418914572\\
469	0.00412520677216004\\
470	0.00409911365408149\\
471	0.00407482289056936\\
472	0.0040523363874517\\
473	0.0040315730449379\\
474	0.00401164881165539\\
475	0.00399218739683764\\
476	0.00397319059741844\\
477	0.00395465264206548\\
478	0.00393655893484832\\
479	0.00391888484667759\\
480	0.00390159466504802\\
481	0.00388464087834956\\
482	0.00386796405063881\\
483	0.00385149365669067\\
484	0.00383515040115759\\
485	0.00381885075552387\\
486	0.00380252636992823\\
487	0.00378616337875535\\
488	0.00376974638153455\\
489	0.00375325862944356\\
490	0.00373668230125755\\
491	0.0037199988792966\\
492	0.00370318963112153\\
493	0.00368623619386584\\
494	0.00366912124307526\\
495	0.00365182920404782\\
496	0.00363434692661208\\
497	0.00361666418912785\\
498	0.00359877344348116\\
499	0.00358066712062533\\
500	0.00356233769956089\\
501	0.00354377777586562\\
502	0.00352498012534427\\
503	0.00350593775725361\\
504	0.00348664395054972\\
505	0.00346709226595765\\
506	0.00344727652677869\\
507	0.00342719076289304\\
508	0.00340682911637355\\
509	0.00338618571499653\\
510	0.00336525457896681\\
511	0.00334402961160912\\
512	0.00332250458591081\\
513	0.00330067312646816\\
514	0.00327852868656125\\
515	0.00325606452033717\\
516	0.00323327365042335\\
517	0.00321014883171252\\
518	0.00318668251252003\\
519	0.00316286679473023\\
520	0.00313869339476105\\
521	0.00311415360690661\\
522	0.00308923826480123\\
523	0.00306393769958115\\
524	0.0030382416947482\\
525	0.00301213943781063\\
526	0.00298561946885924\\
527	0.00295866962633007\\
528	0.00293127699030934\\
529	0.00290342782385294\\
530	0.00287510751292521\\
531	0.00284630050572302\\
532	0.00281699025237423\\
533	0.00278715914642115\\
534	0.00275678847024927\\
535	0.0027258583474368\\
536	0.00269434770585814\\
537	0.00266223425631255\\
538	0.00262949449273195\\
539	0.00259610372036823\\
540	0.00256203611983489\\
541	0.00252726486215598\\
542	0.00249176229581887\\
543	0.00245550022948785\\
544	0.00241845037920276\\
545	0.00238058505390565\\
546	0.00234187788129203\\
547	0.00230230436425532\\
548	0.00226185053906073\\
549	0.00222051365330463\\
550	0.00217828765766402\\
551	0.00213516904957736\\
552	0.00209115827729499\\
553	0.00204625694083962\\
554	0.00200047918608743\\
555	0.00195392585632444\\
556	0.00190654874064295\\
557	0.00185828393241978\\
558	0.00180903905475592\\
559	0.00175861688358186\\
560	0.0017069408381786\\
561	0.00165390077486336\\
562	0.00159933339332958\\
563	0.00154519250351498\\
564	0.00149251652556966\\
565	0.00144199233608514\\
566	0.00139193664277865\\
567	0.00134193923371883\\
568	0.00129189696277558\\
569	0.00124179721745673\\
570	0.0011917551532044\\
571	0.00114258430627697\\
572	0.00109468008503176\\
573	0.0010475786939303\\
574	0.0010007642291051\\
575	0.000954230162850314\\
576	0.000908306034116644\\
577	0.000862713691105962\\
578	0.000817139873524682\\
579	0.000771575799886298\\
580	0.00072606041348764\\
581	0.000680643552963479\\
582	0.000635380130251366\\
583	0.000590328108249356\\
584	0.000545548191687311\\
585	0.000501103162543998\\
586	0.000457057119965888\\
587	0.000413474463421802\\
588	0.000370418599527697\\
589	0.000327950360562708\\
590	0.000286126318718801\\
591	0.000244997582319806\\
592	0.000204610904245315\\
593	0.000165017211836693\\
594	0.000126301460681323\\
595	8.88161203105207e-05\\
596	5.34134895574399e-05\\
597	2.21100055488407e-05\\
598	0\\
599	0\\
600	0\\
};
\addplot [color=blue!25!mycolor7,solid,forget plot]
  table[row sep=crcr]{%
1	0.0059188651418757\\
2	0.00591885908155997\\
3	0.00591885293182187\\
4	0.00591884669126303\\
5	0.0059188403584599\\
6	0.00591883393196333\\
7	0.00591882741029797\\
8	0.00591882079196169\\
9	0.00591881407542529\\
10	0.00591880725913178\\
11	0.00591880034149585\\
12	0.00591879332090345\\
13	0.0059187861957111\\
14	0.00591877896424534\\
15	0.00591877162480226\\
16	0.00591876417564675\\
17	0.0059187566150121\\
18	0.00591874894109924\\
19	0.00591874115207617\\
20	0.00591873324607738\\
21	0.00591872522120314\\
22	0.00591871707551892\\
23	0.00591870880705471\\
24	0.00591870041380436\\
25	0.0059186918937249\\
26	0.00591868324473587\\
27	0.00591867446471858\\
28	0.00591866555151549\\
29	0.00591865650292945\\
30	0.00591864731672289\\
31	0.00591863799061722\\
32	0.00591862852229202\\
33	0.00591861890938424\\
34	0.00591860914948753\\
35	0.00591859924015131\\
36	0.00591858917888011\\
37	0.00591857896313272\\
38	0.00591856859032132\\
39	0.00591855805781074\\
40	0.00591854736291759\\
41	0.00591853650290934\\
42	0.00591852547500355\\
43	0.00591851427636694\\
44	0.0059185029041145\\
45	0.00591849135530861\\
46	0.00591847962695814\\
47	0.00591846771601745\\
48	0.00591845561938551\\
49	0.00591844333390496\\
50	0.00591843085636111\\
51	0.00591841818348097\\
52	0.00591840531193227\\
53	0.00591839223832242\\
54	0.00591837895919755\\
55	0.00591836547104141\\
56	0.00591835177027443\\
57	0.00591833785325253\\
58	0.00591832371626611\\
59	0.005918309355539\\
60	0.00591829476722731\\
61	0.00591827994741838\\
62	0.00591826489212959\\
63	0.00591824959730721\\
64	0.00591823405882531\\
65	0.00591821827248458\\
66	0.00591820223401108\\
67	0.00591818593905516\\
68	0.00591816938319019\\
69	0.00591815256191135\\
70	0.00591813547063441\\
71	0.00591811810469449\\
72	0.00591810045934478\\
73	0.00591808252975533\\
74	0.00591806431101174\\
75	0.00591804579811384\\
76	0.00591802698597446\\
77	0.00591800786941809\\
78	0.00591798844317966\\
79	0.00591796870190301\\
80	0.0059179486401398\\
81	0.00591792825234804\\
82	0.00591790753289084\\
83	0.00591788647603501\\
84	0.00591786507594974\\
85	0.00591784332670531\\
86	0.00591782122227177\\
87	0.0059177987565175\\
88	0.00591777592320803\\
89	0.00591775271600468\\
90	0.0059177291284633\\
91	0.00591770515403296\\
92	0.00591768078605472\\
93	0.00591765601776043\\
94	0.00591763084227152\\
95	0.00591760525259787\\
96	0.00591757924163657\\
97	0.00591755280217107\\
98	0.00591752592687001\\
99	0.00591749860828638\\
100	0.00591747083885655\\
101	0.00591744261089953\\
102	0.00591741391661631\\
103	0.0059173847480892\\
104	0.00591735509728145\\
105	0.00591732495603683\\
106	0.00591729431607956\\
107	0.00591726316901414\\
108	0.00591723150632576\\
109	0.00591719931938059\\
110	0.00591716659942644\\
111	0.00591713333759381\\
112	0.00591709952489709\\
113	0.00591706515223625\\
114	0.00591703021039891\\
115	0.00591699469006286\\
116	0.00591695858179918\\
117	0.0059169218760759\\
118	0.0059168845632626\\
119	0.00591684663363541\\
120	0.00591680807738351\\
121	0.00591676888461626\\
122	0.00591672904537201\\
123	0.0059166885496281\\
124	0.00591664738731295\\
125	0.00591660554832\\
126	0.00591656302252438\\
127	0.00591651979980229\\
128	0.00591647587005437\\
129	0.00591643122323283\\
130	0.00591638584937416\\
131	0.00591633973863793\\
132	0.00591629288135331\\
133	0.00591624526807472\\
134	0.00591619688964922\\
135	0.00591614773729764\\
136	0.00591609780271336\\
137	0.00591604707818248\\
138	0.00591599555673046\\
139	0.005915943232302\\
140	0.00591589009998083\\
141	0.00591583615625805\\
142	0.00591578139935393\\
143	0.0059157258295878\\
144	0.00591566944975112\\
145	0.00591561226531382\\
146	0.00591555428390163\\
147	0.00591549551221862\\
148	0.00591543594429587\\
149	0.00591537551860708\\
150	0.00591531402421091\\
151	0.00591525144149972\\
152	0.00591518775050186\\
153	0.005915122930875\\
154	0.00591505696189918\\
155	0.0059149898224698\\
156	0.00591492149109051\\
157	0.00591485194586587\\
158	0.00591478116449396\\
159	0.00591470912425885\\
160	0.00591463580202286\\
161	0.0059145611742188\\
162	0.00591448521684186\\
163	0.00591440790544167\\
164	0.0059143292151139\\
165	0.00591424912049185\\
166	0.00591416759573793\\
167	0.00591408461453486\\
168	0.00591400015007682\\
169	0.0059139141750603\\
170	0.00591382666167498\\
171	0.00591373758159426\\
172	0.00591364690596571\\
173	0.00591355460540129\\
174	0.00591346064996747\\
175	0.00591336500917504\\
176	0.00591326765196887\\
177	0.00591316854671734\\
178	0.00591306766120182\\
179	0.00591296496260548\\
180	0.00591286041750253\\
181	0.00591275399184665\\
182	0.0059126456509597\\
183	0.00591253535951979\\
184	0.00591242308154943\\
185	0.00591230878040346\\
186	0.00591219241875638\\
187	0.00591207395859003\\
188	0.00591195336118055\\
189	0.00591183058708528\\
190	0.00591170559612939\\
191	0.00591157834739234\\
192	0.00591144879919404\\
193	0.00591131690908058\\
194	0.00591118263380986\\
195	0.0059110459293371\\
196	0.00591090675079973\\
197	0.00591076505250216\\
198	0.0059106207879004\\
199	0.00591047390958608\\
200	0.00591032436927049\\
201	0.00591017211776811\\
202	0.00591001710497983\\
203	0.00590985927987601\\
204	0.00590969859047909\\
205	0.00590953498384588\\
206	0.00590936840604958\\
207	0.00590919880216138\\
208	0.0059090261162317\\
209	0.00590885029127128\\
210	0.00590867126923168\\
211	0.00590848899098555\\
212	0.00590830339630636\\
213	0.00590811442384809\\
214	0.0059079220111241\\
215	0.00590772609448603\\
216	0.005907526609102\\
217	0.00590732348893462\\
218	0.00590711666671837\\
219	0.00590690607393682\\
220	0.00590669164079936\\
221	0.0059064732962173\\
222	0.00590625096777978\\
223	0.00590602458172922\\
224	0.0059057940629362\\
225	0.00590555933487407\\
226	0.00590532031959288\\
227	0.00590507693769307\\
228	0.00590482910829859\\
229	0.0059045767490296\\
230	0.00590431977597459\\
231	0.00590405810366222\\
232	0.00590379164503247\\
233	0.00590352031140745\\
234	0.00590324401246165\\
235	0.00590296265619171\\
236	0.00590267614888578\\
237	0.00590238439509216\\
238	0.00590208729758769\\
239	0.00590178475734554\\
240	0.00590147667350251\\
241	0.00590116294332576\\
242	0.00590084346217916\\
243	0.00590051812348901\\
244	0.00590018681870945\\
245	0.00589984943728723\\
246	0.00589950586662604\\
247	0.00589915599205045\\
248	0.00589879969676935\\
249	0.00589843686183872\\
250	0.00589806736612446\\
251	0.00589769108626435\\
252	0.00589730789662985\\
253	0.00589691766928763\\
254	0.00589652027396045\\
255	0.00589611557798811\\
256	0.00589570344628814\\
257	0.00589528374131621\\
258	0.00589485632302661\\
259	0.00589442104883306\\
260	0.00589397777356926\\
261	0.00589352634945068\\
262	0.00589306662603661\\
263	0.00589259845019399\\
264	0.00589212166606282\\
265	0.0058916361150245\\
266	0.00589114163567389\\
267	0.00589063806379673\\
268	0.00589012523235447\\
269	0.00588960297147957\\
270	0.00588907110848436\\
271	0.00588852946788946\\
272	0.00588797787147783\\
273	0.00588741613838466\\
274	0.00588684408523326\\
275	0.00588626152633054\\
276	0.00588566827392798\\
277	0.00588506413851412\\
278	0.00588444892891336\\
279	0.00588382245100376\\
280	0.00588318449807008\\
281	0.0058825348590128\\
282	0.00588187331880002\\
283	0.00588119965838552\\
284	0.00588051365462389\\
285	0.00587981508018288\\
286	0.00587910370345234\\
287	0.00587837928844987\\
288	0.00587764159472272\\
289	0.00587689037724563\\
290	0.00587612538631439\\
291	0.00587534636743452\\
292	0.00587455306120481\\
293	0.00587374520319499\\
294	0.0058729225238168\\
295	0.00587208474818797\\
296	0.00587123159598753\\
297	0.00587036278130146\\
298	0.00586947801245687\\
299	0.00586857699184229\\
300	0.00586765941571126\\
301	0.00586672497396502\\
302	0.00586577334990901\\
303	0.0058648042199761\\
304	0.00586381725340672\\
305	0.00586281211187248\\
306	0.00586178844902646\\
307	0.00586074590995554\\
308	0.0058596841305036\\
309	0.00585860273642446\\
310	0.00585750134231237\\
311	0.00585637955025261\\
312	0.00585523694814955\\
313	0.00585407310778303\\
314	0.00585288758299598\\
315	0.00585167990964557\\
316	0.00585044961334749\\
317	0.00584919624883876\\
318	0.00584791937657553\\
319	0.00584661854835003\\
320	0.00584529330710308\\
321	0.00584394318673143\\
322	0.00584256771188981\\
323	0.00584116639778782\\
324	0.00583973874998108\\
325	0.00583828426415636\\
326	0.00583680242591091\\
327	0.00583529271052531\\
328	0.00583375458272986\\
329	0.00583218749646402\\
330	0.00583059089462881\\
331	0.0058289642088318\\
332	0.0058273068591242\\
333	0.00582561825372994\\
334	0.00582389778876628\\
335	0.00582214484795558\\
336	0.00582035880232746\\
337	0.00581853900991181\\
338	0.00581668481542106\\
339	0.00581479554992205\\
340	0.00581287053049673\\
341	0.0058109090598908\\
342	0.00580891042615028\\
343	0.00580687390224495\\
344	0.00580479874567809\\
345	0.00580268419808228\\
346	0.00580052948479973\\
347	0.00579833381444694\\
348	0.00579609637846199\\
349	0.0057938163506341\\
350	0.00579149288661333\\
351	0.00578912512339917\\
352	0.00578671217880552\\
353	0.00578425315089892\\
354	0.00578174711740561\\
355	0.00577919313508162\\
356	0.0057765902390379\\
357	0.00577393744201062\\
358	0.00577123373357035\\
359	0.00576847807929635\\
360	0.00576566942017733\\
361	0.00576280667229116\\
362	0.00575988872605111\\
363	0.00575691444542336\\
364	0.00575388266711421\\
365	0.00575079219972624\\
366	0.00574764182288094\\
367	0.00574443028630436\\
368	0.00574115630888115\\
369	0.00573781857767396\\
370	0.00573441574690804\\
371	0.00573094643692255\\
372	0.00572740923309333\\
373	0.00572380268473599\\
374	0.005720125303995\\
375	0.00571637556473634\\
376	0.00571255190146616\\
377	0.00570865270830226\\
378	0.00570467633802012\\
379	0.00570062110112619\\
380	0.00569648526460439\\
381	0.00569226704858611\\
382	0.00568796462368904\\
383	0.00568357610907727\\
384	0.00567909957042097\\
385	0.00567453301774997\\
386	0.00566987440319761\\
387	0.00566512161862799\\
388	0.00566027249314091\\
389	0.00565532479045079\\
390	0.00565027620614079\\
391	0.00564512436479847\\
392	0.00563986681701453\\
393	0.00563450103624374\\
394	0.00562902441551655\\
395	0.00562343426402935\\
396	0.00561772780362139\\
397	0.0056119021651399\\
398	0.00560595438471072\\
399	0.00559988139993504\\
400	0.00559368004603577\\
401	0.00558734705198143\\
402	0.00558087903661795\\
403	0.00557427250485708\\
404	0.00556752384394927\\
405	0.00556062931988138\\
406	0.00555358507408224\\
407	0.00554638712035304\\
408	0.00553903134196755\\
409	0.00553151348872566\\
410	0.00552382917336607\\
411	0.00551597386761887\\
412	0.00550794289708719\\
413	0.00549973143413929\\
414	0.00549133448834178\\
415	0.00548274688664091\\
416	0.00547396325814046\\
417	0.00546497802373885\\
418	0.00545578538484463\\
419	0.00544637931141403\\
420	0.00543675353033366\\
421	0.00542690151522444\\
422	0.00541681647523832\\
423	0.00540649134188126\\
424	0.00539591875216336\\
425	0.0053850910335577\\
426	0.00537400018914585\\
427	0.00536263788242526\\
428	0.00535099542177299\\
429	0.00533906374426414\\
430	0.00532683339727622\\
431	0.00531429451844637\\
432	0.00530143680895152\\
433	0.00528824950213971\\
434	0.00527472136143819\\
435	0.00526084068756722\\
436	0.00524659526987646\\
437	0.00523197232446155\\
438	0.0052169584486447\\
439	0.00520153957555673\\
440	0.00518570092578128\\
441	0.00516942695750495\\
442	0.00515270131816908\\
443	0.00513550680251643\\
444	0.00511782530865607\\
445	0.0050996375612257\\
446	0.005080923273698\\
447	0.00506166172231842\\
448	0.00504183203136729\\
449	0.00502141304891374\\
450	0.00500038337249701\\
451	0.00497872142081816\\
452	0.00495640549556105\\
453	0.00493341382583834\\
454	0.00490972461720514\\
455	0.00488531639942946\\
456	0.00486016471433154\\
457	0.0048342424059789\\
458	0.00480752277340168\\
459	0.00477997995509787\\
460	0.00475158503908517\\
461	0.0047223031056269\\
462	0.00469208532795192\\
463	0.00466087903083142\\
464	0.00462863329802819\\
465	0.00459530120078491\\
466	0.00456084286403987\\
467	0.00452522958610005\\
468	0.00448844925421451\\
469	0.00445051372491344\\
470	0.00441146866877086\\
471	0.00437140664415404\\
472	0.00433048439716472\\
473	0.0042889456962915\\
474	0.00424780188765614\\
475	0.00420755420416029\\
476	0.0041683333579541\\
477	0.00413027769285024\\
478	0.00409353132043151\\
479	0.00405824096408036\\
480	0.00402455137992824\\
481	0.00399259863899142\\
482	0.00396250056459678\\
483	0.00393434330753264\\
484	0.00390816275628304\\
485	0.00388391906252844\\
486	0.00386123616913274\\
487	0.0038390417191476\\
488	0.00381733894561227\\
489	0.00379612284227517\\
490	0.00377537874906163\\
491	0.00375508096906127\\
492	0.00373519154191179\\
493	0.00371565936210906\\
494	0.00369641992156969\\
495	0.00367739607739938\\
496	0.00365850042077436\\
497	0.00363964004349435\\
498	0.00362073356826921\\
499	0.00360176517069187\\
500	0.00358271728286732\\
501	0.00356357081159216\\
502	0.00354430546077433\\
503	0.00352490016954723\\
504	0.00350533367146163\\
505	0.00348558516873753\\
506	0.00346563509676915\\
507	0.00344546592364189\\
508	0.00342506288277991\\
509	0.00340441446709452\\
510	0.00338351130775128\\
511	0.00336234394263013\\
512	0.00334090288803782\\
513	0.00331917870814738\\
514	0.00329716207651345\\
515	0.00327484382265145\\
516	0.00325221495551864\\
517	0.00322926665511696\\
518	0.00320599022398332\\
519	0.00318237699293373\\
520	0.00315841818145577\\
521	0.00313410472472062\\
522	0.00310942721126966\\
523	0.00308437585158505\\
524	0.00305894044027125\\
525	0.00303311031139261\\
526	0.00300687428679977\\
527	0.00298022061768556\\
528	0.00295313692014422\\
529	0.00292561010614763\\
530	0.00289762631203977\\
531	0.00286917082725164\\
532	0.00284022802620434\\
533	0.00281078130393891\\
534	0.00278081301155708\\
535	0.00275030439324347\\
536	0.00271923552725015\\
537	0.00268758527399107\\
538	0.00265533123535152\\
539	0.00262244973055008\\
540	0.0025889157954173\\
541	0.00255470321368202\\
542	0.00251978459095741\\
543	0.0024841314847975\\
544	0.00244771460558413\\
545	0.00241050410851801\\
546	0.00237247000965317\\
547	0.0023335827737848\\
548	0.00229381404395241\\
549	0.00225313741932586\\
550	0.00221154059211673\\
551	0.00216901289169036\\
552	0.002125545741278\\
553	0.00208113833035888\\
554	0.00203579361568159\\
555	0.00198950880324453\\
556	0.00194228488943636\\
557	0.00189413021836552\\
558	0.00184507109150061\\
559	0.0017951892598509\\
560	0.00174443657204966\\
561	0.00169274885086732\\
562	0.00164004493249296\\
563	0.0015861209178538\\
564	0.0015308512570822\\
565	0.00147408828014922\\
566	0.00141791177863419\\
567	0.00136317271247085\\
568	0.00131050219331458\\
569	0.00125878059438045\\
570	0.00120724494964626\\
571	0.00115595954309104\\
572	0.00110479684501904\\
573	0.00105438977026886\\
574	0.00100530531250217\\
575	0.000957529082998364\\
576	0.000910184678218123\\
577	0.000863406211270714\\
578	0.000817321342927877\\
579	0.000771637462967062\\
580	0.000726087369012586\\
581	0.000680657779075544\\
582	0.000635387610447342\\
583	0.000590332284282012\\
584	0.000545550537540452\\
585	0.000501104457276143\\
586	0.000457057781190827\\
587	0.000413474755637532\\
588	0.000370418692160746\\
589	0.000327950377726591\\
590	0.000286126318718799\\
591	0.000244997582319803\\
592	0.000204610904245313\\
593	0.000165017211836692\\
594	0.000126301460681322\\
595	8.88161203105201e-05\\
596	5.34134895574395e-05\\
597	2.21100055488407e-05\\
598	0\\
599	0\\
600	0\\
};
\addplot [color=mycolor9,solid,forget plot]
  table[row sep=crcr]{%
1	0.00596698032876224\\
2	0.00596697428696311\\
3	0.0059669681495499\\
4	0.00596696191492659\\
5	0.00596695558146875\\
6	0.00596694914752286\\
7	0.00596694261140592\\
8	0.00596693597140488\\
9	0.00596692922577601\\
10	0.00596692237274445\\
11	0.00596691541050369\\
12	0.00596690833721486\\
13	0.00596690115100621\\
14	0.0059668938499726\\
15	0.00596688643217482\\
16	0.00596687889563902\\
17	0.00596687123835603\\
18	0.00596686345828079\\
19	0.00596685555333174\\
20	0.00596684752139008\\
21	0.0059668393602992\\
22	0.00596683106786396\\
23	0.00596682264185003\\
24	0.00596681407998321\\
25	0.0059668053799487\\
26	0.00596679653939043\\
27	0.00596678755591023\\
28	0.0059667784270672\\
29	0.00596676915037696\\
30	0.00596675972331082\\
31	0.00596675014329509\\
32	0.00596674040771014\\
33	0.00596673051388984\\
34	0.00596672045912049\\
35	0.00596671024064024\\
36	0.00596669985563803\\
37	0.00596668930125289\\
38	0.00596667857457304\\
39	0.00596666767263494\\
40	0.00596665659242247\\
41	0.005966645330866\\
42	0.00596663388484146\\
43	0.0059666222511694\\
44	0.00596661042661409\\
45	0.00596659840788246\\
46	0.00596658619162313\\
47	0.00596657377442547\\
48	0.00596656115281855\\
49	0.00596654832327008\\
50	0.00596653528218533\\
51	0.0059665220259062\\
52	0.00596650855070997\\
53	0.00596649485280829\\
54	0.005966480928346\\
55	0.00596646677340007\\
56	0.00596645238397834\\
57	0.00596643775601838\\
58	0.00596642288538634\\
59	0.00596640776787563\\
60	0.0059663923992058\\
61	0.00596637677502112\\
62	0.00596636089088945\\
63	0.00596634474230088\\
64	0.00596632832466638\\
65	0.00596631163331648\\
66	0.0059662946634999\\
67	0.00596627741038214\\
68	0.00596625986904409\\
69	0.00596624203448054\\
70	0.00596622390159879\\
71	0.00596620546521708\\
72	0.00596618672006311\\
73	0.00596616766077254\\
74	0.00596614828188739\\
75	0.00596612857785442\\
76	0.00596610854302357\\
77	0.00596608817164626\\
78	0.00596606745787369\\
79	0.0059660463957553\\
80	0.00596602497923682\\
81	0.00596600320215869\\
82	0.00596598105825415\\
83	0.00596595854114744\\
84	0.00596593564435203\\
85	0.00596591236126865\\
86	0.00596588868518346\\
87	0.00596586460926599\\
88	0.00596584012656725\\
89	0.00596581523001769\\
90	0.00596578991242516\\
91	0.00596576416647284\\
92	0.00596573798471713\\
93	0.00596571135958542\\
94	0.005965684283374\\
95	0.00596565674824579\\
96	0.00596562874622822\\
97	0.00596560026921068\\
98	0.00596557130894236\\
99	0.00596554185702984\\
100	0.00596551190493465\\
101	0.00596548144397089\\
102	0.00596545046530268\\
103	0.00596541895994166\\
104	0.0059653869187444\\
105	0.00596535433240978\\
106	0.00596532119147637\\
107	0.00596528748631974\\
108	0.00596525320714972\\
109	0.00596521834400758\\
110	0.00596518288676329\\
111	0.00596514682511265\\
112	0.00596511014857434\\
113	0.00596507284648711\\
114	0.00596503490800669\\
115	0.0059649963221029\\
116	0.0059649570775566\\
117	0.00596491716295662\\
118	0.0059648765666967\\
119	0.00596483527697249\\
120	0.00596479328177839\\
121	0.0059647505689045\\
122	0.00596470712593359\\
123	0.00596466294023813\\
124	0.00596461799897737\\
125	0.0059645722890943\\
126	0.0059645257973131\\
127	0.00596447851013654\\
128	0.00596443041384336\\
129	0.00596438149448628\\
130	0.00596433173789011\\
131	0.00596428112965005\\
132	0.00596422965513067\\
133	0.00596417729946522\\
134	0.00596412404755542\\
135	0.00596406988407196\\
136	0.00596401479345549\\
137	0.00596395875991829\\
138	0.00596390176744615\\
139	0.00596384379980049\\
140	0.00596378484051931\\
141	0.00596372487291455\\
142	0.00596366388005987\\
143	0.00596360184475242\\
144	0.00596353874940599\\
145	0.00596347457576359\\
146	0.00596340930414213\\
147	0.00596334291153906\\
148	0.00596327536756221\\
149	0.0059632066316546\\
150	0.00596313667910292\\
151	0.00596306548798833\\
152	0.00596299303599557\\
153	0.00596291930040575\\
154	0.00596284425808905\\
155	0.00596276788549745\\
156	0.005962690158657\\
157	0.00596261105316029\\
158	0.0059625305441587\\
159	0.0059624486063543\\
160	0.00596236521399185\\
161	0.00596228034085064\\
162	0.005962193960236\\
163	0.00596210604497084\\
164	0.00596201656738697\\
165	0.00596192549931628\\
166	0.00596183281208175\\
167	0.00596173847648834\\
168	0.00596164246281363\\
169	0.00596154474079839\\
170	0.00596144527963697\\
171	0.00596134404796742\\
172	0.00596124101386157\\
173	0.00596113614481485\\
174	0.00596102940773601\\
175	0.00596092076893647\\
176	0.00596081019411975\\
177	0.00596069764837057\\
178	0.00596058309614354\\
179	0.00596046650125228\\
180	0.00596034782685749\\
181	0.00596022703545563\\
182	0.00596010408886674\\
183	0.00595997894822261\\
184	0.00595985157395424\\
185	0.00595972192577939\\
186	0.00595958996268988\\
187	0.00595945564293847\\
188	0.00595931892402574\\
189	0.00595917976268663\\
190	0.00595903811487684\\
191	0.00595889393575869\\
192	0.00595874717968713\\
193	0.00595859780019526\\
194	0.00595844574997971\\
195	0.00595829098088561\\
196	0.00595813344389141\\
197	0.00595797308909345\\
198	0.00595780986569022\\
199	0.00595764372196629\\
200	0.00595747460527606\\
201	0.00595730246202706\\
202	0.00595712723766321\\
203	0.00595694887664747\\
204	0.00595676732244446\\
205	0.00595658251750257\\
206	0.00595639440323586\\
207	0.00595620292000564\\
208	0.00595600800710164\\
209	0.00595580960272282\\
210	0.00595560764395797\\
211	0.00595540206676586\\
212	0.00595519280595509\\
213	0.00595497979516345\\
214	0.00595476296683703\\
215	0.00595454225220883\\
216	0.00595431758127702\\
217	0.00595408888278293\\
218	0.00595385608418831\\
219	0.00595361911165241\\
220	0.00595337789000854\\
221	0.00595313234274014\\
222	0.00595288239195639\\
223	0.00595262795836741\\
224	0.00595236896125876\\
225	0.00595210531846558\\
226	0.00595183694634617\\
227	0.005951563759755\\
228	0.00595128567201496\\
229	0.00595100259488926\\
230	0.00595071443855242\\
231	0.0059504211115608\\
232	0.00595012252082237\\
233	0.00594981857156558\\
234	0.00594950916730758\\
235	0.00594919420982151\\
236	0.00594887359910305\\
237	0.00594854723333576\\
238	0.00594821500885565\\
239	0.00594787682011443\\
240	0.00594753255964174\\
241	0.00594718211800609\\
242	0.00594682538377413\\
243	0.00594646224346886\\
244	0.00594609258152561\\
245	0.00594571628024663\\
246	0.00594533321975339\\
247	0.00594494327793646\\
248	0.00594454633040312\\
249	0.00594414225042176\\
250	0.00594373090886309\\
251	0.00594331217413756\\
252	0.00594288591212829\\
253	0.00594245198611909\\
254	0.00594201025671653\\
255	0.00594156058176518\\
256	0.00594110281625461\\
257	0.00594063681221694\\
258	0.00594016241861275\\
259	0.00593967948120345\\
260	0.00593918784240694\\
261	0.00593868734113341\\
262	0.00593817781259664\\
263	0.00593765908809572\\
264	0.00593713099476015\\
265	0.0059365933552501\\
266	0.00593604598740111\\
267	0.00593548870379985\\
268	0.00593492131127443\\
269	0.00593434361027776\\
270	0.00593375539413795\\
271	0.00593315644814377\\
272	0.00593254654842749\\
273	0.00593192546061117\\
274	0.00593129293821519\\
275	0.00593064872098129\\
276	0.00592999253387784\\
277	0.00592932409005896\\
278	0.00592864311207371\\
279	0.00592794944143\\
280	0.00592724366041662\\
281	0.00592652562069577\\
282	0.00592579512551166\\
283	0.00592505197612543\\
284	0.00592429597190641\\
285	0.00592352691043863\\
286	0.00592274458764527\\
287	0.00592194879793301\\
288	0.00592113933436002\\
289	0.00592031598883094\\
290	0.00591947855232387\\
291	0.00591862681515412\\
292	0.00591776056728166\\
293	0.00591687959867023\\
294	0.00591598369970732\\
295	0.00591507266169724\\
296	0.0059141462774414\\
297	0.00591320434192362\\
298	0.00591224665312267\\
299	0.00591127301297869\\
300	0.00591028322854776\\
301	0.00590927711338638\\
302	0.00590825448921893\\
303	0.00590721518795306\\
304	0.00590615905412585\\
305	0.00590508594788322\\
306	0.00590399574861847\\
307	0.00590288835942188\\
308	0.00590176371251024\\
309	0.00590062177578279\\
310	0.00589946256049717\\
311	0.00589828612948728\\
312	0.00589709260348951\\
313	0.00589588215729965\\
314	0.00589465497888111\\
315	0.00589341110283245\\
316	0.00589214980670914\\
317	0.00589086730518115\\
318	0.00588956197479014\\
319	0.00588823342016629\\
320	0.00588688123950195\\
321	0.00588550502445978\\
322	0.00588410436008018\\
323	0.00588267882468772\\
324	0.00588122798979675\\
325	0.00587975142001631\\
326	0.00587824867295422\\
327	0.00587671929912034\\
328	0.00587516284182927\\
329	0.00587357883710227\\
330	0.00587196681356858\\
331	0.00587032629236615\\
332	0.00586865678704194\\
333	0.00586695780345159\\
334	0.00586522883965904\\
335	0.00586346938583569\\
336	0.00586167892415979\\
337	0.00585985692871575\\
338	0.0058580028653944\\
339	0.00585611619179378\\
340	0.00585419635712175\\
341	0.00585224280210114\\
342	0.00585025495887806\\
343	0.00584823225093598\\
344	0.0058461740930169\\
345	0.00584407989105306\\
346	0.00584194904211389\\
347	0.00583978093437328\\
348	0.00583757494710622\\
349	0.00583533045072542\\
350	0.00583304680687375\\
351	0.00583072336859305\\
352	0.00582835948059781\\
353	0.00582595447969194\\
354	0.0058235076953776\\
355	0.0058210184507158\\
356	0.00581848606349089\\
357	0.00581590984764366\\
358	0.00581328911447787\\
359	0.00581062317086183\\
360	0.00580791129842352\\
361	0.00580515273253551\\
362	0.00580234669481404\\
363	0.00579949239258684\\
364	0.00579658901829688\\
365	0.00579363574883313\\
366	0.00579063174477445\\
367	0.00578757614948704\\
368	0.00578446808787763\\
369	0.0057813066650461\\
370	0.00577809096464617\\
371	0.00577482004685582\\
372	0.00577149294583371\\
373	0.00576810866649863\\
374	0.00576466618042918\\
375	0.00576116442065325\\
376	0.00575760227514969\\
377	0.00575397857922289\\
378	0.00575029210821432\\
379	0.00574654157658063\\
380	0.00574272566568746\\
381	0.00573884316894899\\
382	0.00573489292066993\\
383	0.00573087372860639\\
384	0.00572678437235225\\
385	0.00572262360151242\\
386	0.00571839013363018\\
387	0.00571408265182969\\
388	0.00570969980212954\\
389	0.00570524019037783\\
390	0.00570070237875122\\
391	0.00569608488175161\\
392	0.00569138616162681\\
393	0.00568660462313372\\
394	0.00568173860755651\\
395	0.00567678638588469\\
396	0.00567174615104976\\
397	0.00566661600911928\\
398	0.00566139396935263\\
399	0.00565607793303559\\
400	0.00565066568103991\\
401	0.00564515486010003\\
402	0.00563954296787925\\
403	0.00563382733701319\\
404	0.00562800511850132\\
405	0.00562207326509037\\
406	0.00561602851566418\\
407	0.00560986738226857\\
408	0.00560358614274056\\
409	0.00559718084344612\\
410	0.00559064731396248\\
411	0.00558398120219288\\
412	0.00557717803971416\\
413	0.00557023335131169\\
414	0.00556314283044125\\
415	0.00555590262676456\\
416	0.00554850901419348\\
417	0.00554095814152592\\
418	0.0055332460268363\\
419	0.00552536855176885\\
420	0.00551732145581221\\
421	0.00550910033037316\\
422	0.00550070061258381\\
423	0.00549211757871309\\
424	0.00548334633706524\\
425	0.00547438181743129\\
426	0.0054652187589276\\
427	0.00545585170022723\\
428	0.0054462749691268\\
429	0.00543648267129829\\
430	0.00542646867823506\\
431	0.00541622661415981\\
432	0.00540574984263258\\
433	0.00539503145523869\\
434	0.00538406425960559\\
435	0.00537284076209295\\
436	0.00536135314921188\\
437	0.00534959327003797\\
438	0.00533755261815991\\
439	0.00532522231287002\\
440	0.00531259307962802\\
441	0.00529965522977213\\
442	0.00528639863899811\\
443	0.00527281272170294\\
444	0.00525888639097232\\
445	0.00524460805509669\\
446	0.00522996563264719\\
447	0.00521494652217509\\
448	0.00519953754023312\\
449	0.00518372487686434\\
450	0.00516749405216731\\
451	0.00515082986910345\\
452	0.00513371636310139\\
453	0.00511613674955181\\
454	0.00509807335976727\\
455	0.00507950736007831\\
456	0.00506041883433673\\
457	0.00504078711048057\\
458	0.00502059082109013\\
459	0.00499980766035087\\
460	0.00497841434972291\\
461	0.00495638673588621\\
462	0.00493370007258682\\
463	0.00491032918825705\\
464	0.00488624881909362\\
465	0.00486143208115949\\
466	0.00483584970052833\\
467	0.00480947088826041\\
468	0.00478226938885918\\
469	0.00475421917490449\\
470	0.00472529398634746\\
471	0.00469546662444491\\
472	0.00466470802727958\\
473	0.00463298578218709\\
474	0.0046002615361796\\
475	0.00456648303841523\\
476	0.00453159402478002\\
477	0.00449554146674696\\
478	0.00445827840448508\\
479	0.00441976782313848\\
480	0.00437998790630702\\
481	0.00433893908938266\\
482	0.00429665347640327\\
483	0.00425320735615181\\
484	0.00420873777828003\\
485	0.00416346444776063\\
486	0.00411793338147825\\
487	0.00407334441145914\\
488	0.00402983908360743\\
489	0.00398756754637643\\
490	0.00394668653099172\\
491	0.00390735617353304\\
492	0.00386973517480984\\
493	0.00383397370225649\\
494	0.00380020320070834\\
495	0.00376852208426059\\
496	0.00373897579237152\\
497	0.00371152955435554\\
498	0.00368586419896074\\
499	0.00366070892853893\\
500	0.00363606621761742\\
501	0.0036119291710621\\
502	0.00358827992768946\\
503	0.00356508812130673\\
504	0.00354230954779571\\
505	0.00351988526989542\\
506	0.00349774149255958\\
507	0.0034757907019673\\
508	0.00345393476507764\\
509	0.00343207096241981\\
510	0.00341012764353683\\
511	0.00338808604648802\\
512	0.00336592541463887\\
513	0.003343623263393\\
514	0.00332115577071465\\
515	0.00329849830394357\\
516	0.00327562608630737\\
517	0.00325251499140129\\
518	0.00322914242719214\\
519	0.00320548822984379\\
520	0.00318153542522883\\
521	0.00315727062184529\\
522	0.00313268099904592\\
523	0.00310775343241039\\
524	0.00308247455671109\\
525	0.00305683082287246\\
526	0.00303080854165051\\
527	0.00300439390522109\\
528	0.00297757297671185\\
529	0.00295033163747356\\
530	0.00292265548344591\\
531	0.00289452966666671\\
532	0.00286593868799494\\
533	0.00283686621062073\\
534	0.00280729500462495\\
535	0.00277720688756282\\
536	0.00274658266198832\\
537	0.00271540205159736\\
538	0.00268364363870369\\
539	0.00265128480712363\\
540	0.00261830169627242\\
541	0.00258466917438718\\
542	0.00255036084124864\\
543	0.00251534907343426\\
544	0.00247960512775518\\
545	0.0024430993137522\\
546	0.00240580125358572\\
547	0.00236768025517169\\
548	0.00232870583345605\\
549	0.00228884842711842\\
550	0.00224808016815168\\
551	0.0022063863306714\\
552	0.00216375294517026\\
553	0.00212016698020766\\
554	0.00207561654051804\\
555	0.00203009119300053\\
556	0.00198358244425289\\
557	0.0019360841537035\\
558	0.00188759332586823\\
559	0.00183812199659996\\
560	0.00178767181241054\\
561	0.00173625130079578\\
562	0.00168387822104933\\
563	0.00163065515449775\\
564	0.00157654741904633\\
565	0.0015214872420157\\
566	0.00146539784434777\\
567	0.00140810574272265\\
568	0.00134939730533792\\
569	0.00129085071884086\\
570	0.00123365690412139\\
571	0.0011782816837452\\
572	0.0011249450961202\\
573	0.00107192215633457\\
574	0.00101936903435207\\
575	0.000967409323524144\\
576	0.000916845628028934\\
577	0.000867784394304618\\
578	0.000819851503148485\\
579	0.000772715621978611\\
580	0.000726464313101914\\
581	0.000680819564535681\\
582	0.000635473298127815\\
583	0.000590376990638551\\
584	0.000545575509395754\\
585	0.000501118550186597\\
586	0.000457065651499505\\
587	0.000413478843923962\\
588	0.000370420548387954\\
589	0.000327950981361874\\
590	0.000286126434448213\\
591	0.000244997582319806\\
592	0.000204610904245315\\
593	0.000165017211836694\\
594	0.000126301460681323\\
595	8.88161203105206e-05\\
596	5.34134895574398e-05\\
597	2.21100055488407e-05\\
598	0\\
599	0\\
600	0\\
};
\addplot [color=blue!50!mycolor7,solid,forget plot]
  table[row sep=crcr]{%
1	0.00616796378288477\\
2	0.00616795498629279\\
3	0.00616794604287643\\
4	0.00616793695012169\\
5	0.00616792770547062\\
6	0.00616791830632054\\
7	0.00616790875002333\\
8	0.00616789903388457\\
9	0.00616788915516275\\
10	0.00616787911106843\\
11	0.00616786889876348\\
12	0.00616785851536016\\
13	0.00616784795792029\\
14	0.00616783722345434\\
15	0.0061678263089205\\
16	0.00616781521122393\\
17	0.00616780392721567\\
18	0.00616779245369177\\
19	0.00616778078739232\\
20	0.00616776892500048\\
21	0.00616775686314153\\
22	0.00616774459838177\\
23	0.00616773212722757\\
24	0.00616771944612435\\
25	0.0061677065514554\\
26	0.00616769343954091\\
27	0.00616768010663687\\
28	0.00616766654893387\\
29	0.00616765276255607\\
30	0.00616763874355997\\
31	0.00616762448793324\\
32	0.00616760999159356\\
33	0.00616759525038742\\
34	0.00616758026008878\\
35	0.00616756501639784\\
36	0.00616754951493993\\
37	0.00616753375126397\\
38	0.00616751772084123\\
39	0.00616750141906402\\
40	0.00616748484124429\\
41	0.0061674679826122\\
42	0.00616745083831474\\
43	0.00616743340341421\\
44	0.00616741567288683\\
45	0.00616739764162118\\
46	0.00616737930441671\\
47	0.00616736065598214\\
48	0.00616734169093393\\
49	0.0061673224037946\\
50	0.00616730278899121\\
51	0.00616728284085353\\
52	0.00616726255361243\\
53	0.00616724192139813\\
54	0.0061672209382385\\
55	0.00616719959805714\\
56	0.00616717789467163\\
57	0.00616715582179169\\
58	0.00616713337301722\\
59	0.0061671105418365\\
60	0.00616708732162401\\
61	0.00616706370563864\\
62	0.00616703968702152\\
63	0.00616701525879401\\
64	0.0061669904138556\\
65	0.00616696514498168\\
66	0.00616693944482146\\
67	0.00616691330589566\\
68	0.00616688672059423\\
69	0.00616685968117411\\
70	0.00616683217975678\\
71	0.00616680420832595\\
72	0.00616677575872509\\
73	0.00616674682265486\\
74	0.00616671739167068\\
75	0.00616668745718011\\
76	0.00616665701044015\\
77	0.0061666260425547\\
78	0.00616659454447171\\
79	0.0061665625069804\\
80	0.00616652992070855\\
81	0.00616649677611941\\
82	0.00616646306350893\\
83	0.00616642877300271\\
84	0.00616639389455295\\
85	0.00616635841793529\\
86	0.00616632233274569\\
87	0.0061662856283972\\
88	0.0061662482941167\\
89	0.0061662103189415\\
90	0.00616617169171596\\
91	0.00616613240108794\\
92	0.0061660924355054\\
93	0.00616605178321268\\
94	0.00616601043224692\\
95	0.00616596837043422\\
96	0.00616592558538583\\
97	0.00616588206449441\\
98	0.00616583779492999\\
99	0.00616579276363588\\
100	0.00616574695732473\\
101	0.00616570036247421\\
102	0.00616565296532289\\
103	0.0061656047518658\\
104	0.00616555570785015\\
105	0.0061655058187708\\
106	0.0061654550698657\\
107	0.00616540344611128\\
108	0.00616535093221767\\
109	0.006165297512624\\
110	0.00616524317149349\\
111	0.00616518789270843\\
112	0.00616513165986524\\
113	0.00616507445626923\\
114	0.00616501626492945\\
115	0.00616495706855343\\
116	0.00616489684954168\\
117	0.00616483558998232\\
118	0.00616477327164549\\
119	0.00616470987597776\\
120	0.00616464538409635\\
121	0.00616457977678329\\
122	0.00616451303447963\\
123	0.00616444513727944\\
124	0.00616437606492364\\
125	0.0061643057967941\\
126	0.00616423431190723\\
127	0.0061641615889078\\
128	0.00616408760606264\\
129	0.0061640123412541\\
130	0.00616393577197366\\
131	0.00616385787531536\\
132	0.00616377862796928\\
133	0.00616369800621478\\
134	0.00616361598591388\\
135	0.00616353254250459\\
136	0.00616344765099389\\
137	0.00616336128595101\\
138	0.00616327342150043\\
139	0.0061631840313144\\
140	0.00616309308860505\\
141	0.00616300056611451\\
142	0.00616290643610077\\
143	0.00616281067031201\\
144	0.00616271323993481\\
145	0.00616261411548337\\
146	0.00616251326658348\\
147	0.00616241066167364\\
148	0.0061623062681566\\
149	0.00616220005437358\\
150	0.00616209198842411\\
151	0.00616198203784909\\
152	0.00616187016962098\\
153	0.00616175635013395\\
154	0.00616164054519381\\
155	0.00616152272000782\\
156	0.00616140283917421\\
157	0.00616128086667171\\
158	0.00616115676584856\\
159	0.00616103049941183\\
160	0.00616090202941606\\
161	0.00616077131725206\\
162	0.0061606383236354\\
163	0.0061605030085946\\
164	0.00616036533145928\\
165	0.00616022525084805\\
166	0.00616008272465619\\
167	0.00615993771004306\\
168	0.00615979016341949\\
169	0.00615964004043468\\
170	0.00615948729596306\\
171	0.00615933188409106\\
172	0.00615917375810327\\
173	0.0061590128704687\\
174	0.00615884917282671\\
175	0.00615868261597266\\
176	0.00615851314984341\\
177	0.00615834072350244\\
178	0.00615816528512489\\
179	0.00615798678198224\\
180	0.00615780516042676\\
181	0.00615762036587572\\
182	0.00615743234279532\\
183	0.0061572410346844\\
184	0.00615704638405779\\
185	0.00615684833242945\\
186	0.00615664682029542\\
187	0.00615644178711623\\
188	0.00615623317129941\\
189	0.00615602091018129\\
190	0.0061558049400088\\
191	0.00615558519592096\\
192	0.00615536161192984\\
193	0.00615513412090156\\
194	0.00615490265453664\\
195	0.00615466714335028\\
196	0.00615442751665224\\
197	0.00615418370252641\\
198	0.00615393562780995\\
199	0.00615368321807235\\
200	0.00615342639759385\\
201	0.00615316508934384\\
202	0.00615289921495866\\
203	0.00615262869471919\\
204	0.00615235344752803\\
205	0.00615207339088641\\
206	0.00615178844087065\\
207	0.00615149851210829\\
208	0.00615120351775392\\
209	0.00615090336946461\\
210	0.00615059797737486\\
211	0.00615028725007127\\
212	0.00614997109456685\\
213	0.00614964941627499\\
214	0.00614932211898281\\
215	0.00614898910482444\\
216	0.00614865027425364\\
217	0.00614830552601614\\
218	0.00614795475712158\\
219	0.0061475978628151\\
220	0.00614723473654822\\
221	0.00614686526994981\\
222	0.00614648935279618\\
223	0.00614610687298098\\
224	0.00614571771648471\\
225	0.00614532176734357\\
226	0.00614491890761819\\
227	0.00614450901736168\\
228	0.00614409197458738\\
229	0.0061436676552361\\
230	0.00614323593314302\\
231	0.00614279668000399\\
232	0.00614234976534144\\
233	0.00614189505646999\\
234	0.00614143241846134\\
235	0.00614096171410899\\
236	0.00614048280389222\\
237	0.00613999554593982\\
238	0.00613949979599332\\
239	0.00613899540736961\\
240	0.00613848223092319\\
241	0.006137960115008\\
242	0.00613742890543874\\
243	0.00613688844545148\\
244	0.00613633857566415\\
245	0.00613577913403619\\
246	0.00613520995582788\\
247	0.00613463087355897\\
248	0.00613404171696693\\
249	0.0061334423129645\\
250	0.00613283248559666\\
251	0.00613221205599688\\
252	0.00613158084234291\\
253	0.00613093865981151\\
254	0.00613028532053265\\
255	0.00612962063354266\\
256	0.00612894440473639\\
257	0.00612825643681849\\
258	0.00612755652925316\\
259	0.00612684447821274\\
260	0.00612612007652472\\
261	0.00612538311361694\\
262	0.00612463337546085\\
263	0.00612387064451269\\
264	0.00612309469965206\\
265	0.00612230531611807\\
266	0.00612150226544255\\
267	0.00612068531538084\\
268	0.00611985422983949\\
269	0.00611900876880259\\
270	0.00611814868825768\\
271	0.00611727374012557\\
272	0.0061163836722045\\
273	0.00611547822815719\\
274	0.00611455714762405\\
275	0.00611362016671604\\
276	0.00611266701966769\\
277	0.00611169744400973\\
278	0.00611071119554222\\
279	0.00610970807905205\\
280	0.00610868783412906\\
281	0.00610765017968261\\
282	0.00610659482966852\\
283	0.00610552149380714\\
284	0.00610442987754155\\
285	0.00610331968199689\\
286	0.00610219060394079\\
287	0.00610104233574528\\
288	0.00609987456535038\\
289	0.00609868697623001\\
290	0.00609747924736\\
291	0.00609625105318948\\
292	0.00609500206361547\\
293	0.00609373194396177\\
294	0.00609244035496291\\
295	0.00609112695275381\\
296	0.00608979138886644\\
297	0.00608843331023451\\
298	0.00608705235920753\\
299	0.00608564817357558\\
300	0.00608422038660648\\
301	0.00608276862709649\\
302	0.00608129251943599\\
303	0.00607979168369105\\
304	0.00607826573570055\\
305	0.00607671428718683\\
306	0.0060751369458734\\
307	0.00607353331559405\\
308	0.00607190299635369\\
309	0.00607024558424393\\
310	0.00606856067096013\\
311	0.00606684784225487\\
312	0.00606510667356318\\
313	0.00606333671819089\\
314	0.00606153747673036\\
315	0.00605970832521404\\
316	0.00605784840435603\\
317	0.00605595704761991\\
318	0.00605403373875352\\
319	0.00605207797540534\\
320	0.00605008924888612\\
321	0.00604806704414971\\
322	0.00604601083977554\\
323	0.00604392010795301\\
324	0.00604179431446763\\
325	0.0060396329186885\\
326	0.00603743537355697\\
327	0.00603520112557593\\
328	0.00603292961479949\\
329	0.00603062027482197\\
330	0.00602827253276559\\
331	0.00602588580926559\\
332	0.00602345951845144\\
333	0.00602099306792202\\
334	0.0060184858587127\\
335	0.00601593728525108\\
336	0.00601334673529779\\
337	0.00601071358986723\\
338	0.00600803722312264\\
339	0.00600531700223735\\
340	0.0060025522872127\\
341	0.00599974243064031\\
342	0.00599688677739309\\
343	0.00599398466422511\\
344	0.00599103541925553\\
345	0.00598803836130458\\
346	0.00598499279904132\\
347	0.00598189802989213\\
348	0.0059787533386445\\
349	0.00597555799566378\\
350	0.00597231125461864\\
351	0.00596901234958503\\
352	0.00596566049137276\\
353	0.00596225486290824\\
354	0.00595879461357244\\
355	0.00595527885276398\\
356	0.00595170664450247\\
357	0.0059480770111792\\
358	0.00594438898170422\\
359	0.0059406418518164\\
360	0.00593683662340024\\
361	0.00593297577655792\\
362	0.00592905894473821\\
363	0.00592508580091501\\
364	0.00592105606328772\\
365	0.00591696950153508\\
366	0.00591282594413886\\
367	0.00590862528963745\\
368	0.0059043675281043\\
369	0.00590005274550033\\
370	0.00589568113934188\\
371	0.00589125303809417\\
372	0.00588676892447935\\
373	0.0058822294632954\\
374	0.00587763553379345\\
375	0.00587298826458306\\
376	0.00586828906221889\\
377	0.0058635396030339\\
378	0.0058587416888934\\
379	0.00585389663900076\\
380	0.00584900306679337\\
381	0.00584404835726201\\
382	0.00583902671542171\\
383	0.00583393813557483\\
384	0.00582878269393236\\
385	0.0058235605564739\\
386	0.00581827198738759\\
387	0.00581291735809316\\
388	0.00580749715683386\\
389	0.00580201199879914\\
390	0.00579646263670784\\
391	0.00579084997173673\\
392	0.00578517506461892\\
393	0.0057794391466551\\
394	0.00577364363027226\\
395	0.00576779011861874\\
396	0.0057618804134934\\
397	0.00575591652065193\\
398	0.00574990065119917\\
399	0.00574383521733511\\
400	0.00573772282014266\\
401	0.00573156622634407\\
402	0.00572536832994877\\
403	0.00571913209338799\\
404	0.00571286046093951\\
405	0.00570655623470178\\
406	0.00570022189923676\\
407	0.00569385937178882\\
408	0.00568746962131479\\
409	0.00568105210058922\\
410	0.0056746042595674\\
411	0.00566812073197445\\
412	0.00566159225447525\\
413	0.00565500423479394\\
414	0.00564833486260923\\
415	0.00564155264144871\\
416	0.00563464346015069\\
417	0.00562760468649372\\
418	0.00562043359053898\\
419	0.00561312733264331\\
420	0.00560568294873898\\
421	0.00559809733294775\\
422	0.00559036721966646\\
423	0.00558248917463631\\
424	0.00557445963074975\\
425	0.00556627510944276\\
426	0.0055579322708286\\
427	0.00554942767253612\\
428	0.00554075776446712\\
429	0.00553191888318161\\
430	0.00552290724588343\\
431	0.00551371894406478\\
432	0.00550434993691797\\
433	0.00549479604417364\\
434	0.00548505293799717\\
435	0.00547511613427119\\
436	0.00546498098337248\\
437	0.00545464266024167\\
438	0.00544409615363794\\
439	0.00543333625448267\\
440	0.00542235754316521\\
441	0.00541115437560315\\
442	0.00539972086767413\\
443	0.00538805087773602\\
444	0.00537613799088753\\
445	0.00536397550227631\\
446	0.00535155639435993\\
447	0.00533887331012354\\
448	0.00532591852573199\\
449	0.00531268392138899\\
450	0.00529916095340206\\
451	0.00528534062188356\\
452	0.00527121343455982\\
453	0.00525676936553943\\
454	0.00524199780072244\\
455	0.00522688751487832\\
456	0.00521142666218862\\
457	0.00519560273943266\\
458	0.00517940252820087\\
459	0.00516281197690765\\
460	0.00514581611704362\\
461	0.00512839899095527\\
462	0.00511054357688084\\
463	0.00509223171262275\\
464	0.00507344390265162\\
465	0.00505415925670797\\
466	0.00503435567621611\\
467	0.00501401036798015\\
468	0.00499309963100182\\
469	0.0049715990315815\\
470	0.00494948384062078\\
471	0.00492672976637657\\
472	0.00490331133257622\\
473	0.00487920215532924\\
474	0.00485437358386235\\
475	0.00482879420512406\\
476	0.00480242978742587\\
477	0.00477525017568199\\
478	0.00474722552552484\\
479	0.00471832626772328\\
480	0.00468852275574725\\
481	0.00465778522425869\\
482	0.00462608361112403\\
483	0.00459338717082247\\
484	0.00455966382745061\\
485	0.00452487901905876\\
486	0.00448899374288225\\
487	0.00445195974803545\\
488	0.00441371527226019\\
489	0.00437420191016548\\
490	0.0043333676522765\\
491	0.0042911710405434\\
492	0.00424758678823924\\
493	0.00420261332515753\\
494	0.00415628287011242\\
495	0.00410867481589535\\
496	0.00405993345448968\\
497	0.00401029138671468\\
498	0.00396025901850412\\
499	0.0039112447521658\\
500	0.00386340321922209\\
501	0.00381689842882749\\
502	0.00377190122476033\\
503	0.00372858540682304\\
504	0.00368712197668675\\
505	0.00364767067695465\\
506	0.00361036806645511\\
507	0.00357531064274824\\
508	0.00354253139428384\\
509	0.00351196768246299\\
510	0.00348294412958198\\
511	0.0034544445612314\\
512	0.00342646859927395\\
513	0.00339900494372885\\
514	0.00337202956824401\\
515	0.00334550400835951\\
516	0.00331937394815773\\
517	0.00329356839018511\\
518	0.00326799986285606\\
519	0.00324256629797655\\
520	0.0032171554602406\\
521	0.00319165318127545\\
522	0.00316601664016738\\
523	0.00314022231224942\\
524	0.00311424430564812\\
525	0.003088054691934\\
526	0.00306162398333896\\
527	0.00303492176756812\\
528	0.00300791749839272\\
529	0.00298058141801307\\
530	0.00295288555025977\\
531	0.00292480464640869\\
532	0.00289631687292284\\
533	0.00286740285777628\\
534	0.00283804236074501\\
535	0.00280821432287764\\
536	0.00277789691726086\\
537	0.00274706759525826\\
538	0.00271570312083671\\
539	0.0026837795843368\\
540	0.00265127238661691\\
541	0.00261815618565458\\
542	0.00258440480192056\\
543	0.00254999108823549\\
544	0.00251488678752385\\
545	0.00247906258232332\\
546	0.00244248826276903\\
547	0.0024051329724945\\
548	0.0023669655562731\\
549	0.00232795503983998\\
550	0.00228807128655577\\
551	0.00224728557924605\\
552	0.00220558181446146\\
553	0.00216294491653439\\
554	0.00211936041432799\\
555	0.00207481457162041\\
556	0.00202929454834635\\
557	0.00198278858773779\\
558	0.00193528623440087\\
559	0.00188677858576229\\
560	0.00183725850665937\\
561	0.00178672112391201\\
562	0.001735164201167\\
563	0.00168258854553136\\
564	0.001629003756701\\
565	0.00157442816337441\\
566	0.00151887965724378\\
567	0.00146242148369835\\
568	0.00140510286482678\\
569	0.00134687888471046\\
570	0.00128766019442617\\
571	0.00122731519619431\\
572	0.0011661227516964\\
573	0.0011061623275438\\
574	0.00104779478575355\\
575	0.000991600938932895\\
576	0.000937139387801049\\
577	0.000883303309683172\\
578	0.000830704566292631\\
579	0.00077988188673798\\
580	0.000730735365002536\\
581	0.000682984289216812\\
582	0.000636415229577655\\
583	0.000590882705834239\\
584	0.000545838398608977\\
585	0.000501265719758254\\
586	0.000457149164974275\\
587	0.000413526070867918\\
588	0.000370445508788355\\
589	0.000327962642665688\\
590	0.000286130327584013\\
591	0.00024499835594388\\
592	0.000204610904245314\\
593	0.000165017211836693\\
594	0.000126301460681323\\
595	8.88161203105204e-05\\
596	5.34134895574397e-05\\
597	2.21100055488407e-05\\
598	0\\
599	0\\
600	0\\
};
\addplot [color=blue!40!mycolor9,solid,forget plot]
  table[row sep=crcr]{%
1	0.00694684913621286\\
2	0.00694683074131371\\
3	0.00694681203309307\\
4	0.0069467930061662\\
5	0.00694677365505543\\
6	0.00694675397418873\\
7	0.00694673395789778\\
8	0.00694671360041654\\
9	0.0069466928958795\\
10	0.00694667183832005\\
11	0.00694665042166856\\
12	0.00694662863975075\\
13	0.00694660648628583\\
14	0.00694658395488472\\
15	0.00694656103904817\\
16	0.00694653773216479\\
17	0.00694651402750923\\
18	0.00694648991824014\\
19	0.00694646539739823\\
20	0.00694644045790419\\
21	0.00694641509255667\\
22	0.0069463892940301\\
23	0.00694636305487271\\
24	0.00694633636750413\\
25	0.00694630922421341\\
26	0.00694628161715664\\
27	0.00694625353835466\\
28	0.00694622497969081\\
29	0.00694619593290846\\
30	0.00694616638960869\\
31	0.00694613634124777\\
32	0.00694610577913478\\
33	0.00694607469442889\\
34	0.00694604307813698\\
35	0.0069460109211109\\
36	0.00694597821404477\\
37	0.00694594494747236\\
38	0.00694591111176434\\
39	0.00694587669712533\\
40	0.00694584169359116\\
41	0.00694580609102596\\
42	0.00694576987911914\\
43	0.0069457330473824\\
44	0.00694569558514672\\
45	0.0069456574815592\\
46	0.00694561872557987\\
47	0.00694557930597856\\
48	0.00694553921133148\\
49	0.00694549843001798\\
50	0.00694545695021716\\
51	0.00694541475990437\\
52	0.00694537184684776\\
53	0.00694532819860465\\
54	0.00694528380251787\\
55	0.00694523864571218\\
56	0.00694519271509042\\
57	0.00694514599732968\\
58	0.00694509847887745\\
59	0.00694505014594761\\
60	0.00694500098451646\\
61	0.00694495098031857\\
62	0.00694490011884265\\
63	0.00694484838532735\\
64	0.00694479576475676\\
65	0.00694474224185629\\
66	0.00694468780108797\\
67	0.00694463242664606\\
68	0.00694457610245239\\
69	0.00694451881215163\\
70	0.00694446053910653\\
71	0.00694440126639309\\
72	0.00694434097679557\\
73	0.0069442796528015\\
74	0.00694421727659651\\
75	0.0069441538300592\\
76	0.00694408929475578\\
77	0.00694402365193476\\
78	0.00694395688252141\\
79	0.00694388896711222\\
80	0.00694381988596922\\
81	0.00694374961901428\\
82	0.00694367814582314\\
83	0.00694360544561959\\
84	0.00694353149726931\\
85	0.00694345627927373\\
86	0.00694337976976384\\
87	0.00694330194649369\\
88	0.00694322278683403\\
89	0.00694314226776572\\
90	0.00694306036587285\\
91	0.00694297705733624\\
92	0.00694289231792625\\
93	0.00694280612299587\\
94	0.0069427184474735\\
95	0.00694262926585578\\
96	0.00694253855220005\\
97	0.00694244628011693\\
98	0.00694235242276263\\
99	0.00694225695283123\\
100	0.00694215984254669\\
101	0.00694206106365491\\
102	0.00694196058741549\\
103	0.00694185838459349\\
104	0.00694175442545098\\
105	0.00694164867973846\\
106	0.0069415411166861\\
107	0.00694143170499497\\
108	0.00694132041282801\\
109	0.0069412072078008\\
110	0.00694109205697233\\
111	0.00694097492683563\\
112	0.00694085578330795\\
113	0.00694073459172116\\
114	0.00694061131681176\\
115	0.00694048592271083\\
116	0.00694035837293372\\
117	0.00694022863036972\\
118	0.00694009665727136\\
119	0.00693996241524365\\
120	0.00693982586523321\\
121	0.00693968696751714\\
122	0.00693954568169164\\
123	0.00693940196666051\\
124	0.00693925578062355\\
125	0.00693910708106462\\
126	0.0069389558247396\\
127	0.00693880196766406\\
128	0.00693864546510077\\
129	0.00693848627154717\\
130	0.00693832434072222\\
131	0.00693815962555352\\
132	0.00693799207816383\\
133	0.00693782164985762\\
134	0.00693764829110722\\
135	0.00693747195153883\\
136	0.00693729257991836\\
137	0.00693711012413684\\
138	0.00693692453119565\\
139	0.00693673574719145\\
140	0.00693654371730047\\
141	0.00693634838576177\\
142	0.00693614969585849\\
143	0.00693594758989551\\
144	0.00693574200916975\\
145	0.00693553289393244\\
146	0.00693532018335594\\
147	0.00693510381555999\\
148	0.00693488372774939\\
149	0.00693465985608184\\
150	0.0069344321356242\\
151	0.0069342005003339\\
152	0.00693396488304016\\
153	0.00693372521542484\\
154	0.00693348142800299\\
155	0.00693323345010296\\
156	0.00693298120984649\\
157	0.00693272463412804\\
158	0.00693246364859417\\
159	0.00693219817762225\\
160	0.00693192814429906\\
161	0.00693165347039893\\
162	0.00693137407636147\\
163	0.00693108988126901\\
164	0.00693080080282362\\
165	0.00693050675732383\\
166	0.00693020765964077\\
167	0.00692990342319414\\
168	0.00692959395992757\\
169	0.00692927918028385\\
170	0.00692895899317939\\
171	0.00692863330597857\\
172	0.00692830202446747\\
173	0.00692796505282729\\
174	0.0069276222936072\\
175	0.00692727364769687\\
176	0.00692691901429847\\
177	0.00692655829089819\\
178	0.00692619137323748\\
179	0.00692581815528343\\
180	0.00692543852919908\\
181	0.00692505238531296\\
182	0.00692465961208832\\
183	0.00692426009609168\\
184	0.00692385372196094\\
185	0.00692344037237305\\
186	0.00692301992801094\\
187	0.00692259226753029\\
188	0.00692215726752522\\
189	0.00692171480249389\\
190	0.0069212647448033\\
191	0.00692080696465362\\
192	0.00692034133004186\\
193	0.00691986770672502\\
194	0.00691938595818254\\
195	0.00691889594557832\\
196	0.00691839752772193\\
197	0.00691789056102931\\
198	0.00691737489948283\\
199	0.00691685039459065\\
200	0.00691631689534553\\
201	0.00691577424818281\\
202	0.00691522229693786\\
203	0.00691466088280287\\
204	0.00691408984428273\\
205	0.00691350901715058\\
206	0.00691291823440226\\
207	0.00691231732621025\\
208	0.0069117061198769\\
209	0.00691108443978675\\
210	0.00691045210735832\\
211	0.00690980894099496\\
212	0.00690915475603499\\
213	0.0069084893647011\\
214	0.00690781257604885\\
215	0.00690712419591445\\
216	0.00690642402686185\\
217	0.00690571186812849\\
218	0.00690498751557098\\
219	0.00690425076160927\\
220	0.00690350139517041\\
221	0.00690273920163103\\
222	0.00690196396275935\\
223	0.00690117545665597\\
224	0.00690037345769399\\
225	0.00689955773645808\\
226	0.00689872805968258\\
227	0.00689788419018891\\
228	0.0068970258868217\\
229	0.00689615290438418\\
230	0.00689526499357249\\
231	0.00689436190090919\\
232	0.00689344336867545\\
233	0.00689250913484253\\
234	0.00689155893300214\\
235	0.0068905924922957\\
236	0.00688960953734268\\
237	0.0068886097881678\\
238	0.00688759296012722\\
239	0.0068865587638336\\
240	0.0068855069050802\\
241	0.00688443708476369\\
242	0.00688334899880603\\
243	0.00688224233807519\\
244	0.0068811167883046\\
245	0.00687997203001178\\
246	0.00687880773841547\\
247	0.00687762358335181\\
248	0.00687641922918933\\
249	0.00687519433474276\\
250	0.00687394855318545\\
251	0.00687268153196102\\
252	0.00687139291269337\\
253	0.00687008233109573\\
254	0.00686874941687837\\
255	0.00686739379365499\\
256	0.00686601507884814\\
257	0.00686461288359297\\
258	0.00686318681264006\\
259	0.00686173646425668\\
260	0.00686026143012689\\
261	0.00685876129525027\\
262	0.00685723563783935\\
263	0.00685568402921559\\
264	0.00685410603370409\\
265	0.0068525012085269\\
266	0.00685086910369517\\
267	0.00684920926189953\\
268	0.00684752121839974\\
269	0.00684580450091296\\
270	0.00684405862950203\\
271	0.00684228311646515\\
272	0.00684047746623336\\
273	0.00683864117529172\\
274	0.00683677373216894\\
275	0.00683487461760879\\
276	0.00683294330516728\\
277	0.0068309792625111\\
278	0.00682898195220097\\
279	0.00682695082218309\\
280	0.00682488531004993\\
281	0.00682278484442637\\
282	0.00682064884492553\\
283	0.00681847672203204\\
284	0.00681626787698549\\
285	0.00681402170166375\\
286	0.00681173757846667\\
287	0.00680941488020006\\
288	0.00680705296996019\\
289	0.00680465120101892\\
290	0.00680220891670961\\
291	0.00679972545031397\\
292	0.00679720012495015\\
293	0.00679463225346215\\
294	0.00679202113831082\\
295	0.00678936607146664\\
296	0.00678666633430457\\
297	0.00678392119750127\\
298	0.00678112992093475\\
299	0.00677829175358691\\
300	0.0067754059334494\\
301	0.0067724716874326\\
302	0.00676948823127832\\
303	0.0067664547694763\\
304	0.00676337049518445\\
305	0.00676023459015205\\
306	0.0067570462246441\\
307	0.00675380455736031\\
308	0.00675050873533282\\
309	0.00674715789375852\\
310	0.00674375115565688\\
311	0.00674028763108845\\
312	0.00673676641535544\\
313	0.00673318658517247\\
314	0.00672954719226556\\
315	0.00672584726090572\\
316	0.00672208582459313\\
317	0.00671826191934218\\
318	0.00671437457069753\\
319	0.0067104227917692\\
320	0.00670640558318577\\
321	0.00670232193305088\\
322	0.00669817081690297\\
323	0.00669395119767772\\
324	0.00668966202567304\\
325	0.00668530223851645\\
326	0.00668087076113417\\
327	0.00667636650572151\\
328	0.00667178837171383\\
329	0.00666713524575737\\
330	0.00666240600167895\\
331	0.00665759950045361\\
332	0.00665271459016837\\
333	0.00664775010598122\\
334	0.00664270487007293\\
335	0.00663757769158991\\
336	0.00663236736657532\\
337	0.00662707267788572\\
338	0.00662169239508965\\
339	0.00661622527434456\\
340	0.00661067005824696\\
341	0.00660502547565139\\
342	0.00659929024145186\\
343	0.00659346305631933\\
344	0.00658754260638799\\
345	0.00658152756288226\\
346	0.00657541658167648\\
347	0.00656920830277777\\
348	0.00656290134972417\\
349	0.00655649432889036\\
350	0.00654998582869663\\
351	0.00654337441872457\\
352	0.00653665864876187\\
353	0.0065298370478523\\
354	0.00652290812358438\\
355	0.00651587036233241\\
356	0.00650872223266671\\
357	0.00650146219875987\\
358	0.00649408876322378\\
359	0.00648660057352709\\
360	0.00647899633611881\\
361	0.00647127446191862\\
362	0.00646343328816032\\
363	0.00645547112456574\\
364	0.00644738625466931\\
365	0.00643917693731825\\
366	0.00643084140625058\\
367	0.00642237786411884\\
368	0.00641378448435023\\
369	0.00640505963384136\\
370	0.0063962016968313\\
371	0.00638720907496289\\
372	0.00637808019165857\\
373	0.00636881349640315\\
374	0.00635940746804279\\
375	0.00634986061467753\\
376	0.00634017146360102\\
377	0.0063303385239127\\
378	0.00632036017812918\\
379	0.00631023441153605\\
380	0.00629995835099525\\
381	0.00628952983903434\\
382	0.00627894748838681\\
383	0.00626821011798545\\
384	0.00625731668427991\\
385	0.00624626630671213\\
386	0.00623505829751204\\
387	0.00622369219660639\\
388	0.00621216781259234\\
389	0.00620048527092107\\
390	0.00618864507067233\\
391	0.00617664815158981\\
392	0.00616449597340108\\
393	0.00615219060988033\\
394	0.00613973486064801\\
395	0.00612713238436226\\
396	0.00611438785777488\\
397	0.00610150716613895\\
398	0.00608849763171694\\
399	0.00607536828871431\\
400	0.00606213021494645\\
401	0.00604879693308199\\
402	0.0060353848976741\\
403	0.00602191408910535\\
404	0.00600840874430738\\
405	0.00599489827521582\\
406	0.00598141849593739\\
407	0.00596801358005025\\
408	0.00595474082519926\\
409	0.00594167761009698\\
410	0.00592890464599032\\
411	0.00591651994719634\\
412	0.00590464274849586\\
413	0.0058934183122294\\
414	0.00588302352550035\\
415	0.005873672292017\\
416	0.00586463262703322\\
417	0.00585546218744932\\
418	0.00584616175105737\\
419	0.0058367325413083\\
420	0.00582717628869936\\
421	0.00581749523685951\\
422	0.00580769193367492\\
423	0.00579776827942576\\
424	0.00578772198825286\\
425	0.00577753309913231\\
426	0.00576717950020707\\
427	0.00575665945030968\\
428	0.00574597124855693\\
429	0.00573511323967497\\
430	0.00572408381979775\\
431	0.00571288144278806\\
432	0.00570150462709614\\
433	0.00568995196318385\\
434	0.00567822212160286\\
435	0.00566631386179493\\
436	0.00565422604166051\\
437	0.00564195762795269\\
438	0.00562950770755671\\
439	0.00561687549971017\\
440	0.00560406036921005\\
441	0.00559106184064171\\
442	0.00557787961370345\\
443	0.00556451357991259\\
444	0.00555096384037302\\
445	0.00553723072412042\\
446	0.00552331480693291\\
447	0.00550921692951282\\
448	0.00549493821106663\\
449	0.00548048001355052\\
450	0.00546584365931318\\
451	0.00545103054659524\\
452	0.00543604216524697\\
453	0.00542088000505086\\
454	0.00540554542385716\\
455	0.00539003945985792\\
456	0.00537436257255373\\
457	0.00535851431654713\\
458	0.00534249302310106\\
459	0.005326300620607\\
460	0.00530993977572947\\
461	0.00529341246413642\\
462	0.00527671948422658\\
463	0.00525985977606376\\
464	0.0052428295268324\\
465	0.00522562099366093\\
466	0.00520822093174952\\
467	0.00519060842437592\\
468	0.00517275203084588\\
469	0.00515460601916424\\
470	0.00513610538144768\\
471	0.00511716187763846\\
472	0.00509776069838698\\
473	0.00507788617190998\\
474	0.00505752168482359\\
475	0.00503664974569963\\
476	0.00501525245848585\\
477	0.00499331115423009\\
478	0.00497080623605254\\
479	0.00494771706670877\\
480	0.00492402188656566\\
481	0.00489969775395363\\
482	0.00487472058900362\\
483	0.0048490652129133\\
484	0.00482270245235811\\
485	0.0047956005757536\\
486	0.00476772442343782\\
487	0.00473904035393458\\
488	0.00470951624803794\\
489	0.0046791196018706\\
490	0.00464781759877594\\
491	0.0046155769116861\\
492	0.00458236369881093\\
493	0.00454814351465856\\
494	0.00451288108072529\\
495	0.00447653991846318\\
496	0.00443908164873674\\
497	0.00440046483465721\\
498	0.00436064307353843\\
499	0.00431956066780517\\
500	0.00427714610387846\\
501	0.00423333184872459\\
502	0.00418805823184039\\
503	0.00414127884382855\\
504	0.00409296795400067\\
505	0.00404313061070491\\
506	0.00399181187508489\\
507	0.00393911226723767\\
508	0.00388520834453271\\
509	0.00383037974367361\\
510	0.00377549415595658\\
511	0.00372172196201362\\
512	0.00366923277375448\\
513	0.00361820554168301\\
514	0.0035688256063494\\
515	0.00352128004985227\\
516	0.00347575048911292\\
517	0.00343240284597587\\
518	0.00339137233319415\\
519	0.00335274239791009\\
520	0.00331651583374602\\
521	0.00328257512187733\\
522	0.00324954687571363\\
523	0.00321703873827858\\
524	0.0031850442592495\\
525	0.00315354400458046\\
526	0.00312250357931197\\
527	0.00309187183781254\\
528	0.00306157953849304\\
529	0.00303153882359366\\
530	0.00300164409444535\\
531	0.00297177508822339\\
532	0.00294180349620389\\
533	0.00291162737945269\\
534	0.00288121816240077\\
535	0.00285054396232949\\
536	0.0028195698273891\\
537	0.00278825813558\\
538	0.00275656917596266\\
539	0.00272446192340949\\
540	0.00269189499921097\\
541	0.00265882778302623\\
542	0.00262522158590738\\
543	0.00259104071058111\\
544	0.00255625310260784\\
545	0.00252082680627605\\
546	0.00248472875678594\\
547	0.0024479250553087\\
548	0.00241038132282204\\
549	0.00237206314687042\\
550	0.00233293663899378\\
551	0.00229296913515159\\
552	0.00225212977229066\\
553	0.00221039882459138\\
554	0.0021677601511439\\
555	0.00212419835623349\\
556	0.00207969864940423\\
557	0.00203424697558643\\
558	0.00198783015632776\\
559	0.00194043604283886\\
560	0.0018920536825513\\
561	0.00184267349774045\\
562	0.0017922874812342\\
563	0.00174088941369864\\
564	0.00168847509589551\\
565	0.00163504253595368\\
566	0.00158059220452231\\
567	0.00152512740094253\\
568	0.00146865479755459\\
569	0.00141118627857556\\
570	0.00135274699387529\\
571	0.0012933730569172\\
572	0.00123315339091021\\
573	0.00117210960961848\\
574	0.0011101693316119\\
575	0.0010472300279469\\
576	0.000984276860620152\\
577	0.00092279293671995\\
578	0.000863072420869007\\
579	0.000805675803180638\\
580	0.000750592684941863\\
581	0.000696982038873612\\
582	0.000645396014974021\\
583	0.000595986134186247\\
584	0.000548695373864638\\
585	0.000502773121282149\\
586	0.000457999343816736\\
587	0.000414012697924258\\
588	0.000370724689282775\\
589	0.0003281128921034\\
590	0.00028620269797633\\
591	0.00024502317858971\\
592	0.00020461602593791\\
593	0.000165017211836693\\
594	0.000126301460681323\\
595	8.88161203105206e-05\\
596	5.34134895574398e-05\\
597	2.21100055488407e-05\\
598	0\\
599	0\\
600	0\\
};
\addplot [color=blue!75!mycolor7,solid,forget plot]
  table[row sep=crcr]{%
1	0.0074924042369101\\
2	0.00749239506213014\\
3	0.00749238573072458\\
4	0.00749237624000059\\
5	0.00749236658721893\\
6	0.00749235676959312\\
7	0.00749234678428867\\
8	0.00749233662842226\\
9	0.00749232629906093\\
10	0.00749231579322112\\
11	0.00749230510786791\\
12	0.00749229423991404\\
13	0.00749228318621916\\
14	0.00749227194358871\\
15	0.00749226050877319\\
16	0.00749224887846705\\
17	0.00749223704930785\\
18	0.00749222501787518\\
19	0.00749221278068979\\
20	0.00749220033421243\\
21	0.00749218767484292\\
22	0.00749217479891907\\
23	0.00749216170271557\\
24	0.00749214838244303\\
25	0.00749213483424673\\
26	0.00749212105420562\\
27	0.00749210703833106\\
28	0.00749209278256576\\
29	0.00749207828278255\\
30	0.00749206353478315\\
31	0.00749204853429706\\
32	0.00749203327698016\\
33	0.00749201775841356\\
34	0.00749200197410226\\
35	0.00749198591947388\\
36	0.0074919695898773\\
37	0.00749195298058131\\
38	0.0074919360867732\\
39	0.00749191890355744\\
40	0.00749190142595414\\
41	0.00749188364889773\\
42	0.00749186556723538\\
43	0.00749184717572557\\
44	0.00749182846903647\\
45	0.00749180944174449\\
46	0.00749179008833266\\
47	0.007491770403189\\
48	0.00749175038060492\\
49	0.00749173001477353\\
50	0.007491709299788\\
51	0.00749168822963974\\
52	0.00749166679821676\\
53	0.00749164499930182\\
54	0.00749162282657067\\
55	0.00749160027359014\\
56	0.00749157733381634\\
57	0.00749155400059271\\
58	0.0074915302671481\\
59	0.00749150612659481\\
60	0.00749148157192654\\
61	0.00749145659601643\\
62	0.00749143119161491\\
63	0.0074914053513476\\
64	0.00749137906771326\\
65	0.00749135233308147\\
66	0.00749132513969053\\
67	0.00749129747964512\\
68	0.00749126934491408\\
69	0.00749124072732802\\
70	0.00749121161857695\\
71	0.0074911820102079\\
72	0.00749115189362249\\
73	0.00749112126007436\\
74	0.00749109010066671\\
75	0.00749105840634964\\
76	0.00749102616791765\\
77	0.00749099337600681\\
78	0.0074909600210922\\
79	0.00749092609348508\\
80	0.00749089158333009\\
81	0.00749085648060244\\
82	0.00749082077510497\\
83	0.00749078445646523\\
84	0.00749074751413248\\
85	0.00749070993737465\\
86	0.00749067171527522\\
87	0.00749063283673014\\
88	0.00749059329044456\\
89	0.00749055306492962\\
90	0.00749051214849914\\
91	0.00749047052926625\\
92	0.00749042819513999\\
93	0.00749038513382188\\
94	0.00749034133280228\\
95	0.00749029677935688\\
96	0.00749025146054308\\
97	0.00749020536319627\\
98	0.00749015847392597\\
99	0.00749011077911215\\
100	0.00749006226490123\\
101	0.00749001291720217\\
102	0.00748996272168245\\
103	0.00748991166376396\\
104	0.00748985972861885\\
105	0.00748980690116533\\
106	0.00748975316606336\\
107	0.00748969850771025\\
108	0.00748964291023628\\
109	0.00748958635750016\\
110	0.00748952883308447\\
111	0.00748947032029089\\
112	0.00748941080213562\\
113	0.00748935026134442\\
114	0.0074892886803478\\
115	0.00748922604127593\\
116	0.0074891623259537\\
117	0.00748909751589544\\
118	0.00748903159229973\\
119	0.00748896453604408\\
120	0.00748889632767947\\
121	0.00748882694742483\\
122	0.00748875637516146\\
123	0.00748868459042729\\
124	0.00748861157241107\\
125	0.00748853729994645\\
126	0.00748846175150601\\
127	0.00748838490519511\\
128	0.00748830673874567\\
129	0.00748822722950986\\
130	0.00748814635445365\\
131	0.00748806409015029\\
132	0.00748798041277362\\
133	0.00748789529809124\\
134	0.00748780872145778\\
135	0.00748772065780771\\
136	0.00748763108164834\\
137	0.0074875399670525\\
138	0.00748744728765113\\
139	0.00748735301662569\\
140	0.0074872571267003\\
141	0.00748715959013352\\
142	0.00748706037870942\\
143	0.00748695946372785\\
144	0.00748685681599415\\
145	0.00748675240581058\\
146	0.00748664620297454\\
147	0.00748653817678317\\
148	0.00748642829601296\\
149	0.00748631652890841\\
150	0.00748620284317294\\
151	0.00748608720595981\\
152	0.00748596958386282\\
153	0.00748584994290679\\
154	0.00748572824853804\\
155	0.00748560446561454\\
156	0.00748547855839603\\
157	0.00748535049053385\\
158	0.00748522022506078\\
159	0.00748508772438051\\
160	0.00748495295025705\\
161	0.00748481586380396\\
162	0.00748467642547336\\
163	0.00748453459504479\\
164	0.0074843903316139\\
165	0.00748424359358089\\
166	0.00748409433863882\\
167	0.00748394252376171\\
168	0.0074837881051924\\
169	0.00748363103843031\\
170	0.00748347127821889\\
171	0.00748330877853291\\
172	0.00748314349256557\\
173	0.00748297537271532\\
174	0.00748280437057253\\
175	0.00748263043690597\\
176	0.00748245352164894\\
177	0.0074822735738853\\
178	0.00748209054183523\\
179	0.00748190437284069\\
180	0.00748171501335078\\
181	0.00748152240890672\\
182	0.00748132650412671\\
183	0.00748112724269037\\
184	0.00748092456732317\\
185	0.00748071841978038\\
186	0.00748050874083089\\
187	0.00748029547024069\\
188	0.00748007854675619\\
189	0.00747985790808712\\
190	0.00747963349088927\\
191	0.00747940523074695\\
192	0.00747917306215502\\
193	0.00747893691850082\\
194	0.00747869673204577\\
195	0.00747845243390651\\
196	0.00747820395403595\\
197	0.00747795122120389\\
198	0.00747769416297732\\
199	0.00747743270570058\\
200	0.00747716677447484\\
201	0.00747689629313772\\
202	0.00747662118424218\\
203	0.0074763413690353\\
204	0.00747605676743666\\
205	0.00747576729801631\\
206	0.00747547287797256\\
207	0.00747517342310925\\
208	0.0074748688478127\\
209	0.00747455906502837\\
210	0.00747424398623706\\
211	0.00747392352143078\\
212	0.00747359757908827\\
213	0.00747326606615006\\
214	0.00747292888799319\\
215	0.00747258594840554\\
216	0.00747223714955969\\
217	0.00747188239198657\\
218	0.00747152157454836\\
219	0.00747115459441131\\
220	0.00747078134701797\\
221	0.007470401726059\\
222	0.0074700156234446\\
223	0.00746962292927545\\
224	0.0074692235318132\\
225	0.00746881731745064\\
226	0.00746840417068127\\
227	0.0074679839740684\\
228	0.00746755660821392\\
229	0.00746712195172656\\
230	0.0074666798811896\\
231	0.00746623027112818\\
232	0.00746577299397611\\
233	0.00746530792004223\\
234	0.00746483491747617\\
235	0.00746435385223381\\
236	0.00746386458804207\\
237	0.00746336698636328\\
238	0.00746286090635916\\
239	0.00746234620485403\\
240	0.0074618227362978\\
241	0.00746129035272837\\
242	0.00746074890373346\\
243	0.00746019823641203\\
244	0.00745963819533521\\
245	0.00745906862250666\\
246	0.00745848935732256\\
247	0.00745790023653101\\
248	0.00745730109419103\\
249	0.00745669176163109\\
250	0.00745607206740713\\
251	0.00745544183726021\\
252	0.00745480089407363\\
253	0.00745414905782975\\
254	0.00745348614556628\\
255	0.00745281197133225\\
256	0.00745212634614359\\
257	0.00745142907793836\\
258	0.00745071997153162\\
259	0.00744999882857008\\
260	0.00744926544748631\\
261	0.00744851962345285\\
262	0.00744776114833602\\
263	0.00744698981064949\\
264	0.00744620539550787\\
265	0.00744540768457994\\
266	0.00744459645604195\\
267	0.00744377148453098\\
268	0.00744293254109817\\
269	0.00744207939316252\\
270	0.0074412118044653\\
271	0.00744032953502675\\
272	0.00743943234110788\\
273	0.00743851997518487\\
274	0.00743759218595036\\
275	0.00743664871836117\\
276	0.00743568931371634\\
277	0.00743471370954964\\
278	0.00743372163875567\\
279	0.00743271283000344\\
280	0.00743168700784525\\
281	0.00743064389268382\\
282	0.00742958320073343\\
283	0.00742850464398215\\
284	0.00742740793015531\\
285	0.00742629276268059\\
286	0.00742515884065443\\
287	0.0074240058588103\\
288	0.0074228335074887\\
289	0.00742164147260929\\
290	0.00742042943564502\\
291	0.00741919707359877\\
292	0.00741794405898229\\
293	0.00741667005979812\\
294	0.00741537473952427\\
295	0.00741405775710208\\
296	0.0074127187669274\\
297	0.00741135741884556\\
298	0.00740997335815006\\
299	0.00740856622558549\\
300	0.00740713565735493\\
301	0.00740568128513199\\
302	0.0074042027360781\\
303	0.00740269963286488\\
304	0.00740117159370228\\
305	0.00739961823237191\\
306	0.00739803915826482\\
307	0.0073964339764209\\
308	0.00739480228756206\\
309	0.00739314368810112\\
310	0.00739145777008878\\
311	0.00738974412103208\\
312	0.0073880023235208\\
313	0.00738623195480151\\
314	0.00738443258727678\\
315	0.0073826037919998\\
316	0.00738074513790797\\
317	0.00737885619067737\\
318	0.00737693651269443\\
319	0.00737498566322692\\
320	0.00737300319861045\\
321	0.0073709886724509\\
322	0.00736894163584407\\
323	0.00736686163761355\\
324	0.00736474822456766\\
325	0.00736260094177692\\
326	0.00736041933287318\\
327	0.00735820294037166\\
328	0.00735595130601743\\
329	0.00735366397115779\\
330	0.00735134047714198\\
331	0.00734898036575014\\
332	0.00734658317965315\\
333	0.00734414846290497\\
334	0.00734167576146975\\
335	0.00733916462378496\\
336	0.00733661460136281\\
337	0.00733402524943113\\
338	0.007331396127615\\
339	0.00732872680066001\\
340	0.00732601683919667\\
341	0.00732326582054489\\
342	0.00732047332955525\\
343	0.00731763895948154\\
344	0.00731476231287552\\
345	0.00731184300249034\\
346	0.00730888065217183\\
347	0.00730587489770941\\
348	0.00730282538760534\\
349	0.00729973178370637\\
350	0.00729659376162193\\
351	0.00729341101082845\\
352	0.00729018323433446\\
353	0.00728691014776416\\
354	0.00728359147774521\\
355	0.00728022695963563\\
356	0.00727681633502597\\
357	0.00727335934990114\\
358	0.00726985575162433\\
359	0.00726630526080623\\
360	0.0072627075219497\\
361	0.00725906209845442\\
362	0.00725536844057738\\
363	0.00725162583561816\\
364	0.00724783333997737\\
365	0.00724398968401482\\
366	0.00724009312976099\\
367	0.00723614122745122\\
368	0.00723213030331952\\
369	0.00722805028703685\\
370	0.00722389889421287\\
371	0.0072196743740852\\
372	0.00721537490020484\\
373	0.0072109985683676\\
374	0.00720654339513614\\
375	0.00720200731676553\\
376	0.00719738818823209\\
377	0.00719268378442597\\
378	0.00718789182452313\\
379	0.0071830101526967\\
380	0.00717803786455369\\
381	0.00717297302174104\\
382	0.00716781263677195\\
383	0.00716255349826023\\
384	0.0071571921445855\\
385	0.00715172483369407\\
386	0.00714614750836237\\
387	0.00714045575611459\\
388	0.00713464476282308\\
389	0.00712870925881882\\
390	0.00712264345609405\\
391	0.00711644097487716\\
392	0.00711009475748809\\
393	0.00710359696692507\\
394	0.00709693886706633\\
395	0.00709011068066894\\
396	0.00708310142047591\\
397	0.00707589868765737\\
398	0.00706848843045399\\
399	0.00706085465417936\\
400	0.00705297907154527\\
401	0.00704484067937505\\
402	0.00703641524368279\\
403	0.00702767466862863\\
404	0.00701858621263323\\
405	0.0070091114860898\\
406	0.00699920507810039\\
407	0.00698881230796572\\
408	0.00697786340322368\\
409	0.00696628449370698\\
410	0.00695399082909338\\
411	0.00694088181989411\\
412	0.00692683759102407\\
413	0.00691171474897144\\
414	0.00689534123322774\\
415	0.00687751107176523\\
416	0.0068589445373444\\
417	0.0068400736467418\\
418	0.00682089451339999\\
419	0.00680140327978297\\
420	0.00678159610050093\\
421	0.00676146908036924\\
422	0.00674101809379129\\
423	0.00672023832728233\\
424	0.00669912345187745\\
425	0.00667766734831997\\
426	0.00665586601704147\\
427	0.00663371588571305\\
428	0.00661121352443312\\
429	0.0065883556680858\\
430	0.00656513924150641\\
431	0.00654156138785944\\
432	0.00651761950070601\\
433	0.00649331126032545\\
434	0.00646863467495198\\
435	0.00644358812770502\\
436	0.006418170430136\\
437	0.00639238088348852\\
438	0.00636621934897878\\
439	0.00633968632865795\\
440	0.00631278305872948\\
441	0.00628551161758354\\
442	0.00625787505132119\\
443	0.00622987752022658\\
444	0.00620152447109456\\
445	0.00617282284439145\\
446	0.00614378133980875\\
447	0.0061144108289088\\
448	0.00608472516573098\\
449	0.00605474503763231\\
450	0.00602451632089729\\
451	0.0059940757788918\\
452	0.00596345983494581\\
453	0.00593271253265712\\
454	0.00590188708135116\\
455	0.00587104757514936\\
456	0.00584027051878397\\
457	0.00580964478849715\\
458	0.00577926518114491\\
459	0.00574884459712267\\
460	0.00571834681854514\\
461	0.00568784185090121\\
462	0.00565741516602163\\
463	0.00562717144134524\\
464	0.00559723828658416\\
465	0.00556777100765139\\
466	0.00553895881724486\\
467	0.00551103263774442\\
468	0.00548427489164653\\
469	0.00545903143612402\\
470	0.00543572631584556\\
471	0.0054147965624325\\
472	0.00539349444727599\\
473	0.00537181420825687\\
474	0.00534975006497606\\
475	0.00532729626326827\\
476	0.00530444705284925\\
477	0.00528119668588837\\
478	0.00525753942186825\\
479	0.00523346952832216\\
480	0.00520898123931054\\
481	0.005184068476006\\
482	0.00515872469670289\\
483	0.00513294295240333\\
484	0.00510671600976015\\
485	0.00508003638604027\\
486	0.00505289675152728\\
487	0.00502528991797591\\
488	0.00499720876771142\\
489	0.00496864706955647\\
490	0.00493960006460045\\
491	0.00491006342328762\\
492	0.00488003257272219\\
493	0.00484950249627489\\
494	0.00481846789260173\\
495	0.00478691941683505\\
496	0.00475484403321968\\
497	0.00472222377036806\\
498	0.00468903406153541\\
499	0.00465524054947011\\
500	0.00462080662780665\\
501	0.00458568465853478\\
502	0.00454981067668868\\
503	0.00451309773425715\\
504	0.00447542720897012\\
505	0.00443663751685369\\
506	0.00439667946359437\\
507	0.00435550655755052\\
508	0.00431307046188243\\
509	0.00426932728172256\\
510	0.00422422752017184\\
511	0.00417770960408217\\
512	0.00412970135996104\\
513	0.00408013656419233\\
514	0.00402895935181766\\
515	0.00397613019795216\\
516	0.00392163360410665\\
517	0.00386548864943913\\
518	0.00380776307374029\\
519	0.00374859197637785\\
520	0.00368820257472482\\
521	0.00362694688926678\\
522	0.00356637012747609\\
523	0.00350703592601809\\
524	0.00344912621670985\\
525	0.00339283064665058\\
526	0.0033383426009943\\
527	0.00328585325766154\\
528	0.00323554292911896\\
529	0.00318756877619778\\
530	0.0031420473407322\\
531	0.00309903033854259\\
532	0.00305846404358267\\
533	0.00301973398722696\\
534	0.00298147907057372\\
535	0.00294369779238631\\
536	0.0029063754048515\\
537	0.00286948159293979\\
538	0.0028329682403081\\
539	0.00279676752725694\\
540	0.00276079073611847\\
541	0.00272492799658533\\
542	0.00268904999554552\\
543	0.00265301287973426\\
544	0.00261666780847322\\
545	0.00257994640352295\\
546	0.00254280720659853\\
547	0.00250520494078379\\
548	0.0024670910246756\\
549	0.00242841434684367\\
550	0.00238912233411181\\
551	0.00234916233066769\\
552	0.00230848328248133\\
553	0.00226703747237023\\
554	0.00222478701931318\\
555	0.0021817075326122\\
556	0.00213778081363613\\
557	0.00209298897830381\\
558	0.00204731468235649\\
559	0.00200074137081413\\
560	0.00195325354663635\\
561	0.00190483705770766\\
562	0.00185547938081325\\
563	0.00180516988557633\\
564	0.00175390005863065\\
565	0.0017016636684699\\
566	0.00164845685395718\\
567	0.00159427812763442\\
568	0.00153912832808868\\
569	0.00148301086881223\\
570	0.00142593197227525\\
571	0.00136790084687\\
572	0.00130892989353384\\
573	0.00124903539994602\\
574	0.00118823985767079\\
575	0.00112657419000161\\
576	0.00106408186145155\\
577	0.00100084678722425\\
578	0.000936906460728108\\
579	0.00087219575385531\\
580	0.000807381068522308\\
581	0.000744253919298644\\
582	0.000683149208921055\\
583	0.000624409451839995\\
584	0.00056899987841953\\
585	0.000516346813304098\\
586	0.000466156175058029\\
587	0.000418738807028305\\
588	0.000373486291036665\\
589	0.000329728904611942\\
590	0.000287090770953892\\
591	0.000245465410968095\\
592	0.000204771950375222\\
593	0.000165050678943705\\
594	0.000126301460681323\\
595	8.88161203105205e-05\\
596	5.34134895574398e-05\\
597	2.21100055488407e-05\\
598	0\\
599	0\\
600	0\\
};
\addplot [color=blue!80!mycolor9,solid,forget plot]
  table[row sep=crcr]{%
1	0.00965497018695193\\
2	0.00965495204258113\\
3	0.00965493358761741\\
4	0.00965491481673022\\
5	0.00965489572449764\\
6	0.00965487630540482\\
7	0.0096548565538424\\
8	0.0096548364641049\\
9	0.00965481603038905\\
10	0.00965479524679217\\
11	0.00965477410731042\\
12	0.00965475260583716\\
13	0.00965473073616105\\
14	0.00965470849196445\\
15	0.00965468586682145\\
16	0.0096546628541961\\
17	0.00965463944744055\\
18	0.00965461563979307\\
19	0.00965459142437615\\
20	0.00965456679419457\\
21	0.00965454174213332\\
22	0.00965451626095559\\
23	0.00965449034330074\\
24	0.00965446398168209\\
25	0.00965443716848485\\
26	0.00965440989596393\\
27	0.0096543821562417\\
28	0.00965435394130576\\
29	0.00965432524300659\\
30	0.00965429605305531\\
31	0.0096542663630212\\
32	0.00965423616432937\\
33	0.00965420544825828\\
34	0.00965417420593723\\
35	0.00965414242834384\\
36	0.00965411010630147\\
37	0.00965407723047657\\
38	0.00965404379137607\\
39	0.00965400977934462\\
40	0.00965397518456185\\
41	0.00965393999703954\\
42	0.00965390420661882\\
43	0.00965386780296721\\
44	0.00965383077557574\\
45	0.00965379311375591\\
46	0.00965375480663662\\
47	0.0096537158431611\\
48	0.00965367621208379\\
49	0.0096536359019671\\
50	0.0096535949011781\\
51	0.00965355319788532\\
52	0.00965351078005532\\
53	0.00965346763544924\\
54	0.00965342375161937\\
55	0.00965337911590559\\
56	0.00965333371543175\\
57	0.00965328753710205\\
58	0.0096532405675973\\
59	0.00965319279337109\\
60	0.00965314420064604\\
61	0.0096530947754098\\
62	0.00965304450341113\\
63	0.00965299337015583\\
64	0.00965294136090262\\
65	0.00965288846065902\\
66	0.009652834654177\\
67	0.00965277992594876\\
68	0.00965272426020226\\
69	0.0096526676408968\\
70	0.0096526100517185\\
71	0.00965255147607562\\
72	0.00965249189709388\\
73	0.00965243129761175\\
74	0.00965236966017552\\
75	0.00965230696703442\\
76	0.00965224320013557\\
77	0.00965217834111894\\
78	0.00965211237131208\\
79	0.00965204527172493\\
80	0.00965197702304439\\
81	0.00965190760562893\\
82	0.00965183699950305\\
83	0.00965176518435158\\
84	0.00965169213951404\\
85	0.00965161784397877\\
86	0.00965154227637701\\
87	0.00965146541497694\\
88	0.00965138723767745\\
89	0.00965130772200203\\
90	0.00965122684509239\\
91	0.00965114458370204\\
92	0.00965106091418975\\
93	0.00965097581251287\\
94	0.00965088925422066\\
95	0.00965080121444731\\
96	0.00965071166790502\\
97	0.00965062058887685\\
98	0.00965052795120954\\
99	0.00965043372830612\\
100	0.00965033789311843\\
101	0.00965024041813962\\
102	0.00965014127539626\\
103	0.00965004043644066\\
104	0.00964993787234274\\
105	0.00964983355368198\\
106	0.00964972745053914\\
107	0.00964961953248789\\
108	0.0096495097685862\\
109	0.00964939812736771\\
110	0.00964928457683289\\
111	0.00964916908444004\\
112	0.00964905161709613\\
113	0.0096489321411476\\
114	0.0096488106223708\\
115	0.00964868702596246\\
116	0.00964856131652987\\
117	0.009648433458081\\
118	0.00964830341401434\\
119	0.00964817114710863\\
120	0.00964803661951248\\
121	0.00964789979273363\\
122	0.00964776062762822\\
123	0.00964761908438977\\
124	0.00964747512253805\\
125	0.00964732870090764\\
126	0.00964717977763642\\
127	0.0096470283101538\\
128	0.0096468742551688\\
129	0.00964671756865782\\
130	0.00964655820585237\\
131	0.00964639612122644\\
132	0.00964623126848375\\
133	0.00964606360054474\\
134	0.00964589306953336\\
135	0.00964571962676362\\
136	0.00964554322272591\\
137	0.00964536380707311\\
138	0.00964518132860641\\
139	0.00964499573526083\\
140	0.00964480697409056\\
141	0.00964461499125392\\
142	0.00964441973199811\\
143	0.00964422114064373\\
144	0.00964401916056945\\
145	0.00964381373419684\\
146	0.0096436048029754\\
147	0.00964339230736474\\
148	0.00964317618681762\\
149	0.00964295637976299\\
150	0.00964273282358866\\
151	0.00964250545462378\\
152	0.00964227420812089\\
153	0.00964203901823771\\
154	0.00964179981801877\\
155	0.00964155653937645\\
156	0.00964130911307192\\
157	0.00964105746869568\\
158	0.00964080153464767\\
159	0.00964054123811728\\
160	0.00964027650506273\\
161	0.00964000726019028\\
162	0.00963973342693307\\
163	0.00963945492742951\\
164	0.00963917168250134\\
165	0.00963888361163138\\
166	0.00963859063294074\\
167	0.00963829266316583\\
168	0.0096379896176348\\
169	0.00963768141024367\\
170	0.00963736795343198\\
171	0.00963704915815813\\
172	0.00963672493387414\\
173	0.00963639518850007\\
174	0.00963605982839801\\
175	0.00963571875834549\\
176	0.00963537188150858\\
177	0.00963501909941442\\
178	0.0096346603119233\\
179	0.00963429541720022\\
180	0.00963392431168603\\
181	0.00963354689006795\\
182	0.00963316304524962\\
183	0.00963277266832066\\
184	0.00963237564852566\\
185	0.00963197187323258\\
186	0.0096315612279006\\
187	0.00963114359604752\\
188	0.00963071885921635\\
189	0.00963028689694149\\
190	0.00962984758671429\\
191	0.00962940080394782\\
192	0.00962894642194125\\
193	0.00962848431184342\\
194	0.00962801434261575\\
195	0.00962753638099462\\
196	0.00962705029145289\\
197	0.00962655593616085\\
198	0.00962605317494643\\
199	0.00962554186525456\\
200	0.00962502186210601\\
201	0.0096244930180553\\
202	0.00962395518314787\\
203	0.00962340820487649\\
204	0.00962285192813687\\
205	0.00962228619518239\\
206	0.00962171084557805\\
207	0.00962112571615347\\
208	0.00962053064095517\\
209	0.00961992545119777\\
210	0.00961930997521443\\
211	0.00961868403840623\\
212	0.00961804746319063\\
213	0.009617400068949\\
214	0.00961674167197312\\
215	0.00961607208541063\\
216	0.00961539111920946\\
217	0.00961469858006117\\
218	0.00961399427134329\\
219	0.00961327799306033\\
220	0.00961254954178392\\
221	0.00961180871059156\\
222	0.00961105528900429\\
223	0.0096102890629231\\
224	0.00960950981456411\\
225	0.00960871732239239\\
226	0.00960791136105457\\
227	0.00960709170131012\\
228	0.0096062581099611\\
229	0.00960541034978067\\
230	0.00960454817944007\\
231	0.00960367135343416\\
232	0.00960277962200546\\
233	0.00960187273106658\\
234	0.00960095042212111\\
235	0.00960001243218291\\
236	0.00959905849369366\\
237	0.00959808833443883\\
238	0.00959710167746167\\
239	0.00959609824097572\\
240	0.00959507773827518\\
241	0.00959403987764359\\
242	0.00959298436226045\\
243	0.009591910890106\\
244	0.00959081915386375\\
245	0.00958970884082113\\
246	0.00958857963276779\\
247	0.00958743120589187\\
248	0.0095862632306739\\
249	0.00958507537177835\\
250	0.00958386728794284\\
251	0.00958263863186483\\
252	0.00958138905008581\\
253	0.00958011818287291\\
254	0.00957882566409769\\
255	0.0095775111211123\\
256	0.00957617417462274\\
257	0.0095748144385592\\
258	0.00957343151994328\\
259	0.00957202501875225\\
260	0.00957059452777998\\
261	0.00956913963249458\\
262	0.00956765991089259\\
263	0.00956615493334974\\
264	0.00956462426246793\\
265	0.00956306745291855\\
266	0.00956148405128192\\
267	0.0095598735958828\\
268	0.00955823561662191\\
269	0.00955656963480352\\
270	0.00955487516295943\\
271	0.00955315170466955\\
272	0.00955139875437975\\
273	0.0095496157972168\\
274	0.0095478023087985\\
275	0.00954595775503347\\
276	0.00954408159189874\\
277	0.00954217326514716\\
278	0.00954023221014498\\
279	0.00953825785166094\\
280	0.00953624960363585\\
281	0.00953420686894513\\
282	0.00953212903915469\\
283	0.00953001549426992\\
284	0.00952786560247765\\
285	0.00952567871988064\\
286	0.00952345419022464\\
287	0.00952119134461741\\
288	0.00951888950123982\\
289	0.00951654796504829\\
290	0.00951416602746885\\
291	0.00951174296608198\\
292	0.00950927804429836\\
293	0.00950677051102498\\
294	0.00950421960032135\\
295	0.00950162453104562\\
296	0.0094989845064902\\
297	0.0094962987140064\\
298	0.00949356632461807\\
299	0.00949078649262375\\
300	0.00948795835518685\\
301	0.00948508103191377\\
302	0.00948215362441926\\
303	0.00947917521587892\\
304	0.00947614487056771\\
305	0.00947306163338395\\
306	0.00946992452935717\\
307	0.00946673256313739\\
308	0.00946348471846284\\
309	0.0094601799576027\\
310	0.00945681722077536\\
311	0.00945339542555257\\
312	0.00944991346627829\\
313	0.00944637021356014\\
314	0.00944276451398482\\
315	0.00943909518935328\\
316	0.0094353610358999\\
317	0.00943156082361616\\
318	0.00942769329557353\\
319	0.00942375716722943\\
320	0.00941975112571598\\
321	0.00941567382911174\\
322	0.00941152390569652\\
323	0.00940729995318932\\
324	0.00940300053796974\\
325	0.00939862419428304\\
326	0.00939416942342937\\
327	0.00938963469293767\\
328	0.00938501843572465\\
329	0.00938031904923997\\
330	0.00937553489459816\\
331	0.00937066429569856\\
332	0.00936570553833459\\
333	0.0093606568692938\\
334	0.00935551649545057\\
335	0.00935028258285381\\
336	0.00934495325581231\\
337	0.00933952659598115\\
338	0.00933400064145329\\
339	0.00932837338586146\\
340	0.00932264277749711\\
341	0.00931680671845429\\
342	0.00931086306380939\\
343	0.00930480962084981\\
344	0.00929864414836947\\
345	0.00929236435605318\\
346	0.00928596790397956\\
347	0.0092794524022805\\
348	0.00927281541100742\\
349	0.00926605444027035\\
350	0.00925916695073759\\
351	0.00925215035461403\\
352	0.00924500201725672\\
353	0.0092377192596408\\
354	0.00923029936194912\\
355	0.00922273956860483\\
356	0.0092150370950777\\
357	0.00920718913680887\\
358	0.00919919287924184\\
359	0.00919104551082314\\
360	0.00918274424628183\\
361	0.0091742863547925\\
362	0.00916566919767199\\
363	0.00915689028159032\\
364	0.00914794733950183\\
365	0.00913883847019818\\
366	0.0091295624239009\\
367	0.00912011931536049\\
368	0.00911051298327061\\
369	0.00910117833255604\\
370	0.00909171561842355\\
371	0.00908209058493244\\
372	0.00907230016985511\\
373	0.00906234124237296\\
374	0.00905221060126851\\
375	0.00904190497370628\\
376	0.00903142101565248\\
377	0.00902075531438558\\
378	0.00900990438881151\\
379	0.00899886475236882\\
380	0.00898763268450551\\
381	0.00897620422810316\\
382	0.00896457528579671\\
383	0.00895274162548049\\
384	0.00894069887176215\\
385	0.00892844249664058\\
386	0.00891596780934495\\
387	0.00890326994527436\\
388	0.008890343853981\\
389	0.00887718428614845\\
390	0.00886378577953019\\
391	0.00885014264383539\\
392	0.00883624894458003\\
393	0.00882209848596725\\
394	0.00880768479292399\\
395	0.0087930010925074\\
396	0.00877804029500976\\
397	0.00876279497523994\\
398	0.00874725735464402\\
399	0.00873141928513815\\
400	0.0087152722357321\\
401	0.00869880728314632\\
402	0.00868201510755678\\
403	0.00866488599434183\\
404	0.00864740984332204\\
405	0.00862957619448917\\
406	0.008611374305736\\
407	0.00859279306337955\\
408	0.00857382229199742\\
409	0.00855445255551324\\
410	0.00853467508333412\\
411	0.00851448206690299\\
412	0.00849386721740041\\
413	0.00847282553754913\\
414	0.00845135629138582\\
415	0.00842946837858887\\
416	0.00840717303111885\\
417	0.00838446855188986\\
418	0.00836134607464309\\
419	0.00833779644702568\\
420	0.00831381020081693\\
421	0.00828937751239539\\
422	0.00826448815811504\\
423	0.00823913146378406\\
424	0.00821329646866346\\
425	0.00818697205617378\\
426	0.00816014675232015\\
427	0.00813280867112936\\
428	0.00810494548958678\\
429	0.00807654442026407\\
430	0.00804759218136667\\
431	0.00801807496389312\\
432	0.00798797839555728\\
433	0.00795728750107564\\
434	0.00792598665836684\\
435	0.00789405955014713\\
436	0.00786148911033843\\
437	0.0078282574646376\\
438	0.00779434586455267\\
439	0.0077597346142501\\
440	0.00772440298984031\\
441	0.00768832915167115\\
442	0.00765149005284808\\
443	0.00761386135416194\\
444	0.00757541737269823\\
445	0.00753613112674139\\
446	0.00749597454924612\\
447	0.00745491813766635\\
448	0.00741293710877575\\
449	0.00736999563629485\\
450	0.00732605222905261\\
451	0.00728105983545202\\
452	0.00723496581642318\\
453	0.00718771099863757\\
454	0.00713922838691379\\
455	0.00708944162447838\\
456	0.00703826340693689\\
457	0.00698559495438312\\
458	0.00693133047916984\\
459	0.00690529055976151\\
460	0.0068841207630141\\
461	0.00686208024413438\\
462	0.00683905710526833\\
463	0.0068149054819794\\
464	0.0067894708922581\\
465	0.00676258301758642\\
466	0.00673403986817408\\
467	0.00670360081697532\\
468	0.00667097525108228\\
469	0.00663582092278856\\
470	0.00659773011524339\\
471	0.00655629509722116\\
472	0.00651421305901012\\
473	0.00647147989328616\\
474	0.00642809231747627\\
475	0.00638404796618308\\
476	0.00633934549180825\\
477	0.00629398469173378\\
478	0.00624796680193247\\
479	0.00620129537540108\\
480	0.0061539799181788\\
481	0.00610605450934546\\
482	0.00605756163724939\\
483	0.00600852154552289\\
484	0.00595895779064099\\
485	0.00590889731784291\\
486	0.00585836996046816\\
487	0.00580740626716363\\
488	0.00575603018181687\\
489	0.00570421180804988\\
490	0.00565187157863738\\
491	0.00559901569161241\\
492	0.00554570427623523\\
493	0.00549201187304566\\
494	0.00543803029105876\\
495	0.00538387254887218\\
496	0.00532967764954309\\
497	0.00527561646363064\\
498	0.00522189901646822\\
499	0.00516878428684203\\
500	0.00511659104976579\\
501	0.00506571200740636\\
502	0.00501663098046658\\
503	0.00496994308152165\\
504	0.0049263810243382\\
505	0.00488684674767849\\
506	0.00484726967063035\\
507	0.00480753330695101\\
508	0.00476753172025539\\
509	0.00472663807761456\\
510	0.00468482920714398\\
511	0.00464208086863117\\
512	0.00459836793622333\\
513	0.00455366403367335\\
514	0.0045079407329001\\
515	0.00446116550987426\\
516	0.00441331715078829\\
517	0.00436437751689232\\
518	0.00431432977304112\\
519	0.00426315727662655\\
520	0.0042108415062387\\
521	0.00415735964308127\\
522	0.0041026801451013\\
523	0.00404675126036695\\
524	0.00398951582042739\\
525	0.00393091740233218\\
526	0.00387090184799251\\
527	0.00380941952584333\\
528	0.00374642829474921\\
529	0.00368189691490911\\
530	0.00361580972351305\\
531	0.00354817300613794\\
532	0.00347928394531365\\
533	0.00340998619139048\\
534	0.00334181294001288\\
535	0.00327495586372995\\
536	0.00320961667718388\\
537	0.00314600321252046\\
538	0.00308432322384668\\
539	0.00302477504282014\\
540	0.00296753390652657\\
541	0.00291275062540001\\
542	0.00286052641625685\\
543	0.00281087172974648\\
544	0.00276366019386304\\
545	0.00271737865858413\\
546	0.00267149001838159\\
547	0.00262597823294146\\
548	0.00258081087012544\\
549	0.0025359370152983\\
550	0.00249128575749629\\
551	0.00244676577299991\\
552	0.00240226677489789\\
553	0.00235766189430851\\
554	0.00231280486886916\\
555	0.0022675368226262\\
556	0.00222170449377727\\
557	0.00217528619547\\
558	0.00212825770808596\\
559	0.00208059251779431\\
560	0.00203226220776163\\
561	0.00198323669429928\\
562	0.00193348516259038\\
563	0.00188297732214686\\
564	0.00183168498165631\\
565	0.00177958389218695\\
566	0.00172665571247692\\
567	0.00167288999768657\\
568	0.00161828474516008\\
569	0.00156284015785779\\
570	0.00150655873420948\\
571	0.00144944562230146\\
572	0.00139150887414687\\
573	0.00133275960045463\\
574	0.001273212338925\\
575	0.00121288536990015\\
576	0.00115180094160388\\
577	0.00108998546890569\\
578	0.00102746988043389\\
579	0.000964291042797229\\
580	0.000900493476418601\\
581	0.000836121682920783\\
582	0.000771207498915577\\
583	0.000705846288818108\\
584	0.000640030929677252\\
585	0.000575449812676461\\
586	0.000512984914594378\\
587	0.000453002036723499\\
588	0.000396493011046006\\
589	0.000344400514679134\\
590	0.000296031958987211\\
591	0.000250550941554444\\
592	0.000207406150564994\\
593	0.000166007309561553\\
594	0.000126517918855203\\
595	8.88161203105205e-05\\
596	5.34134895574398e-05\\
597	2.21100055488407e-05\\
598	0\\
599	0\\
600	0\\
};
\addplot [color=blue,solid,forget plot]
  table[row sep=crcr]{%
1	0.00994728939738958\\
2	0.0099472887672835\\
3	0.00994728812639287\\
4	0.00994728747453268\\
5	0.00994728681151475\\
6	0.00994728613714766\\
7	0.00994728545123673\\
8	0.00994728475358395\\
9	0.0099472840439879\\
10	0.00994728332224372\\
11	0.00994728258814306\\
12	0.00994728184147398\\
13	0.00994728108202093\\
14	0.00994728030956467\\
15	0.00994727952388218\\
16	0.00994727872474666\\
17	0.00994727791192743\\
18	0.00994727708518983\\
19	0.00994727624429521\\
20	0.00994727538900085\\
21	0.00994727451905984\\
22	0.0099472736342211\\
23	0.0099472727342292\\
24	0.00994727181882439\\
25	0.00994727088774246\\
26	0.00994726994071468\\
27	0.00994726897746774\\
28	0.00994726799772364\\
29	0.00994726700119965\\
30	0.00994726598760819\\
31	0.00994726495665679\\
32	0.00994726390804796\\
33	0.00994726284147914\\
34	0.00994726175664261\\
35	0.00994726065322539\\
36	0.00994725953090913\\
37	0.0099472583893701\\
38	0.00994725722827899\\
39	0.0099472560473009\\
40	0.0099472548460952\\
41	0.00994725362431546\\
42	0.00994725238160932\\
43	0.00994725111761844\\
44	0.00994724983197833\\
45	0.0099472485243183\\
46	0.00994724719426134\\
47	0.00994724584142402\\
48	0.00994724446541636\\
49	0.00994724306584174\\
50	0.00994724164229676\\
51	0.00994724019437117\\
52	0.00994723872164771\\
53	0.00994723722370203\\
54	0.00994723570010253\\
55	0.00994723415041029\\
56	0.00994723257417887\\
57	0.00994723097095426\\
58	0.00994722934027473\\
59	0.00994722768167065\\
60	0.00994722599466443\\
61	0.00994722427877035\\
62	0.0099472225334944\\
63	0.0099472207583342\\
64	0.00994721895277881\\
65	0.00994721711630858\\
66	0.00994721524839507\\
67	0.00994721334850081\\
68	0.00994721141607922\\
69	0.00994720945057442\\
70	0.0099472074514211\\
71	0.00994720541804432\\
72	0.0099472033498594\\
73	0.00994720124627171\\
74	0.00994719910667653\\
75	0.00994719693045889\\
76	0.00994719471699334\\
77	0.00994719246564387\\
78	0.00994719017576362\\
79	0.00994718784669481\\
80	0.00994718547776846\\
81	0.00994718306830426\\
82	0.00994718061761038\\
83	0.00994717812498324\\
84	0.00994717558970735\\
85	0.00994717301105508\\
86	0.00994717038828649\\
87	0.00994716772064909\\
88	0.00994716500737764\\
89	0.00994716224769397\\
90	0.00994715944080671\\
91	0.0099471565859111\\
92	0.00994715368218877\\
93	0.00994715072880748\\
94	0.00994714772492094\\
95	0.00994714466966854\\
96	0.00994714156217508\\
97	0.00994713840155061\\
98	0.00994713518689011\\
99	0.00994713191727327\\
100	0.00994712859176422\\
101	0.00994712520941129\\
102	0.00994712176924673\\
103	0.00994711827028644\\
104	0.00994711471152971\\
105	0.00994711109195893\\
106	0.0099471074105393\\
107	0.00994710366621859\\
108	0.00994709985792678\\
109	0.00994709598457581\\
110	0.00994709204505926\\
111	0.00994708803825206\\
112	0.00994708396301017\\
113	0.00994707981817023\\
114	0.0099470756025493\\
115	0.0099470713149445\\
116	0.00994706695413266\\
117	0.00994706251887001\\
118	0.00994705800789183\\
119	0.0099470534199121\\
120	0.00994704875362312\\
121	0.0099470440076952\\
122	0.00994703918077621\\
123	0.00994703427149133\\
124	0.00994702927844254\\
125	0.00994702420020832\\
126	0.00994701903534321\\
127	0.00994701378237747\\
128	0.00994700843981659\\
129	0.00994700300614096\\
130	0.00994699747980538\\
131	0.0099469918592387\\
132	0.00994698614284335\\
133	0.00994698032899489\\
134	0.00994697441604159\\
135	0.00994696840230396\\
136	0.00994696228607432\\
137	0.00994695606561627\\
138	0.00994694973916428\\
139	0.00994694330492319\\
140	0.00994693676106768\\
141	0.00994693010574182\\
142	0.00994692333705845\\
143	0.00994691645309862\\
144	0.00994690945191107\\
145	0.00994690233151202\\
146	0.00994689508988443\\
147	0.00994688772497744\\
148	0.00994688023470578\\
149	0.00994687261694925\\
150	0.00994686486955207\\
151	0.00994685699032234\\
152	0.00994684897703144\\
153	0.00994684082741339\\
154	0.00994683253916425\\
155	0.0099468241099415\\
156	0.00994681553736337\\
157	0.00994680681900821\\
158	0.00994679795241383\\
159	0.00994678893507682\\
160	0.00994677976445188\\
161	0.00994677043795109\\
162	0.00994676095294323\\
163	0.0099467513067531\\
164	0.00994674149666068\\
165	0.00994673151990052\\
166	0.00994672137366089\\
167	0.00994671105508305\\
168	0.00994670056126047\\
169	0.00994668988923804\\
170	0.00994667903601124\\
171	0.00994666799852535\\
172	0.00994665677367461\\
173	0.00994664535830136\\
174	0.00994663374919521\\
175	0.00994662194309211\\
176	0.00994660993667353\\
177	0.00994659772656549\\
178	0.0099465853093377\\
179	0.00994657268150257\\
180	0.0099465598395143\\
181	0.0099465467797679\\
182	0.00994653349859821\\
183	0.00994651999227888\\
184	0.00994650625702139\\
185	0.009946492288974\\
186	0.00994647808422069\\
187	0.00994646363878009\\
188	0.00994644894860445\\
189	0.00994643400957845\\
190	0.00994641881751813\\
191	0.00994640336816973\\
192	0.00994638765720855\\
193	0.00994637168023776\\
194	0.00994635543278717\\
195	0.00994633891031205\\
196	0.00994632210819187\\
197	0.00994630502172903\\
198	0.00994628764614757\\
199	0.00994626997659191\\
200	0.00994625200812545\\
201	0.00994623373572928\\
202	0.00994621515430077\\
203	0.00994619625865216\\
204	0.00994617704350918\\
205	0.00994615750350956\\
206	0.0099461376332016\\
207	0.00994611742704261\\
208	0.00994609687939744\\
209	0.00994607598453692\\
210	0.00994605473663625\\
211	0.00994603312977342\\
212	0.00994601115792758\\
213	0.00994598881497736\\
214	0.00994596609469918\\
215	0.00994594299076553\\
216	0.00994591949674325\\
217	0.00994589560609167\\
218	0.00994587131216087\\
219	0.00994584660818978\\
220	0.00994582148730435\\
221	0.00994579594251557\\
222	0.00994576996671757\\
223	0.00994574355268562\\
224	0.0099457166930741\\
225	0.00994568938041447\\
226	0.00994566160711314\\
227	0.00994563336544935\\
228	0.009945604647573\\
229	0.00994557544550243\\
230	0.00994554575112222\\
231	0.00994551555618077\\
232	0.00994548485228811\\
233	0.00994545363091341\\
234	0.0099454218833826\\
235	0.0099453896008759\\
236	0.00994535677442528\\
237	0.00994532339491192\\
238	0.00994528945306356\\
239	0.00994525493945187\\
240	0.00994521984448973\\
241	0.00994518415842843\\
242	0.00994514787135489\\
243	0.00994511097318873\\
244	0.0099450734536794\\
245	0.00994503530240314\\
246	0.00994499650875996\\
247	0.00994495706197049\\
248	0.00994491695107285\\
249	0.00994487616491938\\
250	0.00994483469217337\\
251	0.00994479252130564\\
252	0.00994474964059116\\
253	0.00994470603810546\\
254	0.00994466170172113\\
255	0.0099446166191041\\
256	0.00994457077770993\\
257	0.00994452416477997\\
258	0.00994447676733747\\
259	0.00994442857218362\\
260	0.0099443795658934\\
261	0.00994432973481152\\
262	0.00994427906504804\\
263	0.00994422754247412\\
264	0.00994417515271752\\
265	0.00994412188115801\\
266	0.00994406771292274\\
267	0.00994401263288139\\
268	0.00994395662564126\\
269	0.0099438996755421\\
270	0.0099438417666508\\
271	0.00994378288275576\\
272	0.00994372300736148\\
273	0.00994366212368418\\
274	0.00994360021465019\\
275	0.00994353726289344\\
276	0.00994347325073415\\
277	0.00994340816017812\\
278	0.00994334197291253\\
279	0.00994327467029953\\
280	0.00994320623336957\\
281	0.00994313664281463\\
282	0.00994306587898105\\
283	0.00994299392186234\\
284	0.00994292075109171\\
285	0.00994284634593433\\
286	0.00994277068527944\\
287	0.00994269374763214\\
288	0.00994261551110497\\
289	0.00994253595340922\\
290	0.00994245505184591\\
291	0.00994237278329653\\
292	0.00994228912421344\\
293	0.00994220405060995\\
294	0.00994211753805004\\
295	0.00994202956163778\\
296	0.00994194009600626\\
297	0.00994184911530623\\
298	0.00994175659319425\\
299	0.00994166250282041\\
300	0.00994156681681557\\
301	0.00994146950727814\\
302	0.00994137054576027\\
303	0.00994126990325364\\
304	0.00994116755017456\\
305	0.00994106345634881\\
306	0.00994095759099596\\
307	0.00994084992271344\\
308	0.00994074041946005\\
309	0.00994062904853703\\
310	0.00994051577656212\\
311	0.0099404005694313\\
312	0.00994028339228341\\
313	0.00994016420953451\\
314	0.0099400429848504\\
315	0.00993991968110448\\
316	0.00993979426034843\\
317	0.00993966668378385\\
318	0.0099395369117322\\
319	0.00993940490360296\\
320	0.00993927061785956\\
321	0.00993913401198321\\
322	0.00993899504243423\\
323	0.00993885366461058\\
324	0.00993870983280351\\
325	0.00993856350014987\\
326	0.00993841461858081\\
327	0.00993826313876641\\
328	0.00993810901005589\\
329	0.00993795218041293\\
330	0.0099377925963454\\
331	0.00993763020282909\\
332	0.00993746494322466\\
333	0.009937296759187\\
334	0.00993712559056614\\
335	0.0099369513752989\\
336	0.00993677404928993\\
337	0.00993659354628105\\
338	0.00993640979770759\\
339	0.00993622273253997\\
340	0.00993603227710894\\
341	0.00993583835491259\\
342	0.00993564088640272\\
343	0.00993543978874843\\
344	0.00993523497557408\\
345	0.00993502635666861\\
346	0.00993481383766298\\
347	0.00993459731967199\\
348	0.00993437669889627\\
349	0.00993415186617975\\
350	0.00993392270651697\\
351	0.0099336890985035\\
352	0.00993345091372144\\
353	0.00993320801605457\\
354	0.0099329602609451\\
355	0.00993270749465525\\
356	0.00993244955361171\\
357	0.00993218626336552\\
358	0.00993191743696515\\
359	0.00993164287405719\\
360	0.00993136235969612\\
361	0.00993107566302746\\
362	0.00993078253599338\\
363	0.00993048271233565\\
364	0.00993017590777067\\
365	0.00992986182381064\\
366	0.00992954015840349\\
367	0.00992921057257196\\
368	0.00992887161019262\\
369	0.00992810194969619\\
370	0.00992728902741637\\
371	0.0099264631145681\\
372	0.00992562400856727\\
373	0.00992477150386479\\
374	0.0099239053917256\\
375	0.00992302545946369\\
376	0.00992213148860753\\
377	0.00992122325503199\\
378	0.00992030055431135\\
379	0.00991936317258418\\
380	0.00991841087754756\\
381	0.00991744342998338\\
382	0.00991646058560671\\
383	0.00991546209471297\\
384	0.00991444770177994\\
385	0.00991341714501988\\
386	0.00991237015587637\\
387	0.00991130645846048\\
388	0.00991022576892032\\
389	0.0099091277947384\\
390	0.00990801223395065\\
391	0.00990687877428166\\
392	0.00990572709219101\\
393	0.0099045568518263\\
394	0.00990336770388\\
395	0.00990215928434925\\
396	0.00990093121320017\\
397	0.0098996830929426\\
398	0.00989841450712726\\
399	0.00989712501879033\\
400	0.00989581416890412\\
401	0.00989448147499027\\
402	0.0098931264303108\\
403	0.00989174850451013\\
404	0.00989034714578065\\
405	0.00988892177088828\\
406	0.00988747166429707\\
407	0.00988599614891656\\
408	0.00988449457131745\\
409	0.00988296627242109\\
410	0.00988141060728983\\
411	0.00987982697629938\\
412	0.00987821450495692\\
413	0.00987657213037055\\
414	0.00987489960185651\\
415	0.00987319675954955\\
416	0.00987146305496015\\
417	0.00986969776315128\\
418	0.00986790013262936\\
419	0.00986606938492638\\
420	0.00986420471345122\\
421	0.00986230527599937\\
422	0.00986037017056468\\
423	0.00985839844667363\\
424	0.00985638913465386\\
425	0.00985434122410222\\
426	0.00985225365688332\\
427	0.00985012532347765\\
428	0.00984795505893773\\
429	0.0098457416383978\\
430	0.00984348377207433\\
431	0.00984118009968394\\
432	0.00983882918419298\\
433	0.00983642950479868\\
434	0.00983397944902411\\
435	0.00983147730378909\\
436	0.00982892124529488\\
437	0.00982630932753172\\
438	0.00982363946918497\\
439	0.00982090943867187\\
440	0.00981811683697266\\
441	0.00981525907773534\\
442	0.0098123333634594\\
443	0.00980933665384108\\
444	0.00980626561203159\\
445	0.00980311648509642\\
446	0.00979988490903435\\
447	0.00979656751367569\\
448	0.00979315984879927\\
449	0.00978965665875299\\
450	0.00978605196421807\\
451	0.00978233904989328\\
452	0.009778510358334\\
453	0.00977455735752918\\
454	0.0097704703965325\\
455	0.00976623854629632\\
456	0.00976184932712826\\
457	0.00975728762879463\\
458	0.00975253094102019\\
459	0.00971839771449009\\
460	0.00967783551590882\\
461	0.00963638307702326\\
462	0.00959401571846844\\
463	0.00955071116987698\\
464	0.009506449746934\\
465	0.00946121356433595\\
466	0.00941498989763655\\
467	0.00936777382588069\\
468	0.00931982649739658\\
469	0.00927097724594474\\
470	0.00922111249492722\\
471	0.00917024171997016\\
472	0.00911836431980632\\
473	0.00906544098017759\\
474	0.00901142986803239\\
475	0.00895628650202312\\
476	0.0088999636513926\\
477	0.00884241151725245\\
478	0.00878357579389252\\
479	0.00872339937401684\\
480	0.00866181904152994\\
481	0.0085987855990807\\
482	0.008534244330055\\
483	0.00846813739739182\\
484	0.008400404229598\\
485	0.00833098180268491\\
486	0.00825980565040134\\
487	0.00818681356724348\\
488	0.00811196041225784\\
489	0.00803757241513387\\
490	0.00796229125054936\\
491	0.00788495233018119\\
492	0.0078054261504511\\
493	0.00772356675223783\\
494	0.00763920884209486\\
495	0.00755216429112926\\
496	0.00746221784295183\\
497	0.00736912176415375\\
498	0.00727258914291985\\
499	0.00717228510848414\\
500	0.00706781684756801\\
501	0.0069587135245635\\
502	0.00684442693088627\\
503	0.00672436104868497\\
504	0.00659781673954426\\
505	0.00646397029198241\\
506	0.00632679729735914\\
507	0.00618639402032289\\
508	0.00606171878598091\\
509	0.00599096059071272\\
510	0.00591912655344071\\
511	0.00584623680915749\\
512	0.00577231931384684\\
513	0.00569741081960089\\
514	0.00562155638887639\\
515	0.00554480421508631\\
516	0.0054671773277444\\
517	0.00538855941456101\\
518	0.00530897412214534\\
519	0.00522850802198831\\
520	0.00514727042913968\\
521	0.00506539966618615\\
522	0.00498306956464646\\
523	0.00490049748841788\\
524	0.00481795446526811\\
525	0.00473577534449757\\
526	0.0046543761075988\\
527	0.00457426204420224\\
528	0.00449604712090942\\
529	0.00442050433839845\\
530	0.00434859482926174\\
531	0.00428150109757077\\
532	0.00421321273900356\\
533	0.00414348913715137\\
534	0.00407226318348304\\
535	0.0039995517841578\\
536	0.00392540198290194\\
537	0.00384990165189182\\
538	0.00377319371463011\\
539	0.00369549476055221\\
540	0.00361711914268737\\
541	0.00353742125391003\\
542	0.00345601782437688\\
543	0.00337319025389791\\
544	0.003289365302518\\
545	0.00320628170397525\\
546	0.00312468842652381\\
547	0.00304480981507298\\
548	0.00296686662900341\\
549	0.00289106424411822\\
550	0.00281757556151098\\
551	0.00274651671417479\\
552	0.00267791305917642\\
553	0.00261171652587721\\
554	0.00254802634911223\\
555	0.00248683843879565\\
556	0.0024279600872744\\
557	0.00236932352608795\\
558	0.00231094340623584\\
559	0.00225282286035417\\
560	0.00219495151987034\\
561	0.00213732035400731\\
562	0.00207989833733159\\
563	0.00202262868797722\\
564	0.00196542589818876\\
565	0.00190817449924115\\
566	0.00185073167445929\\
567	0.00179292770605968\\
568	0.00173462363221089\\
569	0.00167581051444986\\
570	0.00161648382026782\\
571	0.00155664018532771\\
572	0.00149628178551331\\
573	0.00143541952223136\\
574	0.00137406514768928\\
575	0.00131223203833818\\
576	0.00124993644289524\\
577	0.00118719902170031\\
578	0.00112404659450915\\
579	0.00106051390222132\\
580	0.000996645105392352\\
581	0.000932492781884864\\
582	0.000868111167276619\\
583	0.000803555608238753\\
584	0.000738882285574681\\
585	0.000674149269925966\\
586	0.000609397230982037\\
587	0.000544655405819526\\
588	0.000479944735464092\\
589	0.000415300425128081\\
590	0.000352022048369845\\
591	0.000291263299764878\\
592	0.000233865201392129\\
593	0.000180687335852113\\
594	0.000132203256552347\\
595	9.01789376102564e-05\\
596	5.34134895574398e-05\\
597	2.21100055488407e-05\\
598	0\\
599	0\\
600	0\\
};
\addplot [color=mycolor10,solid,forget plot]
  table[row sep=crcr]{%
1	0.00997061354577094\\
2	0.00997061352463255\\
3	0.00997061350313236\\
4	0.00997061348126415\\
5	0.00997061345902163\\
6	0.00997061343639837\\
7	0.00997061341338782\\
8	0.00997061338998336\\
9	0.00997061336617822\\
10	0.00997061334196553\\
11	0.00997061331733829\\
12	0.00997061329228941\\
13	0.00997061326681163\\
14	0.00997061324089762\\
15	0.00997061321453988\\
16	0.0099706131877308\\
17	0.00997061316046265\\
18	0.00997061313272754\\
19	0.00997061310451748\\
20	0.00997061307582432\\
21	0.00997061304663977\\
22	0.00997061301695541\\
23	0.00997061298676266\\
24	0.00997061295605282\\
25	0.00997061292481701\\
26	0.00997061289304622\\
27	0.00997061286073127\\
28	0.00997061282786284\\
29	0.00997061279443144\\
30	0.00997061276042742\\
31	0.00997061272584096\\
32	0.00997061269066209\\
33	0.00997061265488064\\
34	0.0099706126184863\\
35	0.00997061258146856\\
36	0.00997061254381673\\
37	0.00997061250551995\\
38	0.00997061246656718\\
39	0.00997061242694716\\
40	0.00997061238664848\\
41	0.0099706123456595\\
42	0.0099706123039684\\
43	0.00997061226156315\\
44	0.00997061221843151\\
45	0.00997061217456107\\
46	0.00997061212993914\\
47	0.00997061208455288\\
48	0.00997061203838918\\
49	0.00997061199143474\\
50	0.00997061194367601\\
51	0.00997061189509924\\
52	0.00997061184569041\\
53	0.00997061179543529\\
54	0.00997061174431937\\
55	0.00997061169232795\\
56	0.00997061163944602\\
57	0.00997061158565835\\
58	0.00997061153094944\\
59	0.00997061147530353\\
60	0.00997061141870458\\
61	0.0099706113611363\\
62	0.0099706113025821\\
63	0.00997061124302512\\
64	0.0099706111824482\\
65	0.0099706111208339\\
66	0.00997061105816449\\
67	0.00997061099442191\\
68	0.00997061092958783\\
69	0.00997061086364357\\
70	0.00997061079657017\\
71	0.00997061072834832\\
72	0.00997061065895839\\
73	0.00997061058838042\\
74	0.00997061051659411\\
75	0.0099706104435788\\
76	0.00997061036931351\\
77	0.00997061029377688\\
78	0.00997061021694718\\
79	0.00997061013880234\\
80	0.00997061005931989\\
81	0.00997060997847699\\
82	0.00997060989625042\\
83	0.00997060981261653\\
84	0.00997060972755132\\
85	0.00997060964103035\\
86	0.00997060955302877\\
87	0.0099706094635213\\
88	0.00997060937248227\\
89	0.00997060927988552\\
90	0.00997060918570449\\
91	0.00997060908991215\\
92	0.00997060899248102\\
93	0.00997060889338314\\
94	0.00997060879259009\\
95	0.00997060869007298\\
96	0.00997060858580239\\
97	0.00997060847974846\\
98	0.00997060837188076\\
99	0.0099706082621684\\
100	0.00997060815057993\\
101	0.00997060803708339\\
102	0.00997060792164628\\
103	0.00997060780423553\\
104	0.00997060768481753\\
105	0.0099706075633581\\
106	0.00997060743982246\\
107	0.00997060731417527\\
108	0.00997060718638058\\
109	0.00997060705640184\\
110	0.00997060692420186\\
111	0.00997060678974285\\
112	0.00997060665298638\\
113	0.00997060651389334\\
114	0.009970606372424\\
115	0.00997060622853792\\
116	0.00997060608219402\\
117	0.00997060593335048\\
118	0.00997060578196481\\
119	0.00997060562799379\\
120	0.00997060547139348\\
121	0.00997060531211917\\
122	0.00997060515012543\\
123	0.00997060498536604\\
124	0.00997060481779402\\
125	0.00997060464736158\\
126	0.00997060447402013\\
127	0.00997060429772027\\
128	0.00997060411841176\\
129	0.00997060393604351\\
130	0.00997060375056356\\
131	0.00997060356191911\\
132	0.00997060337005643\\
133	0.0099706031749209\\
134	0.00997060297645698\\
135	0.00997060277460821\\
136	0.00997060256931717\\
137	0.00997060236052546\\
138	0.00997060214817371\\
139	0.00997060193220158\\
140	0.00997060171254768\\
141	0.00997060148914957\\
142	0.00997060126194376\\
143	0.00997060103086568\\
144	0.00997060079584969\\
145	0.00997060055682905\\
146	0.00997060031373591\\
147	0.00997060006650126\\
148	0.00997059981505495\\
149	0.00997059955932563\\
150	0.00997059929924076\\
151	0.00997059903472658\\
152	0.00997059876570809\\
153	0.00997059849210903\\
154	0.00997059821385185\\
155	0.00997059793085772\\
156	0.00997059764304644\\
157	0.00997059735033651\\
158	0.00997059705264504\\
159	0.00997059674988774\\
160	0.0099705964419789\\
161	0.00997059612883142\\
162	0.00997059581035666\\
163	0.00997059548646454\\
164	0.00997059515706347\\
165	0.00997059482206029\\
166	0.00997059448136031\\
167	0.00997059413486723\\
168	0.00997059378248313\\
169	0.00997059342410845\\
170	0.00997059305964198\\
171	0.00997059268898078\\
172	0.0099705923120202\\
173	0.00997059192865382\\
174	0.00997059153877344\\
175	0.00997059114226905\\
176	0.0099705907390288\\
177	0.00997059032893893\\
178	0.00997058991188381\\
179	0.00997058948774585\\
180	0.00997058905640547\\
181	0.00997058861774113\\
182	0.00997058817162919\\
183	0.009970587717944\\
184	0.00997058725655774\\
185	0.00997058678734048\\
186	0.00997058631016011\\
187	0.00997058582488228\\
188	0.00997058533137041\\
189	0.00997058482948562\\
190	0.00997058431908669\\
191	0.00997058380003003\\
192	0.00997058327216965\\
193	0.0099705827353571\\
194	0.00997058218944143\\
195	0.00997058163426919\\
196	0.00997058106968431\\
197	0.00997058049552813\\
198	0.00997057991163929\\
199	0.00997057931785376\\
200	0.00997057871400473\\
201	0.0099705780999226\\
202	0.00997057747543489\\
203	0.00997057684036625\\
204	0.00997057619453839\\
205	0.00997057553776999\\
206	0.00997057486987669\\
207	0.00997057419067105\\
208	0.00997057349996245\\
209	0.00997057279755708\\
210	0.00997057208325785\\
211	0.00997057135686437\\
212	0.00997057061817286\\
213	0.00997056986697611\\
214	0.00997056910306343\\
215	0.00997056832622056\\
216	0.00997056753622963\\
217	0.0099705667328691\\
218	0.0099705659159137\\
219	0.00997056508513435\\
220	0.00997056424029808\\
221	0.00997056338116801\\
222	0.00997056250750326\\
223	0.00997056161905885\\
224	0.00997056071558568\\
225	0.00997055979683041\\
226	0.00997055886253545\\
227	0.00997055791243879\\
228	0.00997055694627403\\
229	0.00997055596377021\\
230	0.00997055496465181\\
231	0.0099705539486386\\
232	0.00997055291544561\\
233	0.009970551864783\\
234	0.00997055079635602\\
235	0.00997054970986491\\
236	0.00997054860500477\\
237	0.00997054748146552\\
238	0.0099705463389318\\
239	0.00997054517708285\\
240	0.00997054399559243\\
241	0.00997054279412871\\
242	0.00997054157235421\\
243	0.00997054032992564\\
244	0.00997053906649382\\
245	0.0099705377817036\\
246	0.0099705364751937\\
247	0.00997053514659665\\
248	0.00997053379553863\\
249	0.00997053242163939\\
250	0.00997053102451211\\
251	0.0099705296037633\\
252	0.00997052815899263\\
253	0.00997052668979288\\
254	0.00997052519574974\\
255	0.0099705236764417\\
256	0.00997052213143995\\
257	0.00997052056030817\\
258	0.00997051896260246\\
259	0.00997051733787115\\
260	0.00997051568565469\\
261	0.00997051400548545\\
262	0.0099705122968876\\
263	0.00997051055937694\\
264	0.00997050879246074\\
265	0.00997050699563754\\
266	0.00997050516839702\\
267	0.00997050331021974\\
268	0.00997050142057697\\
269	0.00997049949893044\\
270	0.00997049754473208\\
271	0.00997049555742399\\
272	0.00997049353643848\\
273	0.00997049148119846\\
274	0.00997048939111724\\
275	0.00997048726559648\\
276	0.00997048510402675\\
277	0.00997048290578748\\
278	0.0099704806702468\\
279	0.00997047839676128\\
280	0.00997047608467566\\
281	0.00997047373332259\\
282	0.00997047134202239\\
283	0.00997046891008274\\
284	0.0099704664367984\\
285	0.00997046392145093\\
286	0.00997046136330836\\
287	0.00997045876162489\\
288	0.00997045611564053\\
289	0.00997045342458081\\
290	0.00997045068765636\\
291	0.00997044790406261\\
292	0.00997044507297935\\
293	0.00997044219357035\\
294	0.00997043926498294\\
295	0.00997043628634762\\
296	0.00997043325677753\\
297	0.00997043017536806\\
298	0.00997042704119628\\
299	0.0099704238533205\\
300	0.00997042061077969\\
301	0.00997041731259295\\
302	0.00997041395775895\\
303	0.00997041054525537\\
304	0.00997040707403837\\
305	0.00997040354304208\\
306	0.00997039995117815\\
307	0.00997039629733503\\
308	0.00997039258037674\\
309	0.00997038879914037\\
310	0.00997038495243257\\
311	0.00997038103902861\\
312	0.00997037705767839\\
313	0.00997037300710451\\
314	0.00997036888599895\\
315	0.00997036469302148\\
316	0.00997036042679827\\
317	0.00997035608592045\\
318	0.00997035166894253\\
319	0.00997034717438066\\
320	0.00997034260071082\\
321	0.00997033794636685\\
322	0.00997033320973826\\
323	0.00997032838916791\\
324	0.00997032348294946\\
325	0.00997031848932463\\
326	0.00997031340648017\\
327	0.00997030823254455\\
328	0.00997030296558441\\
329	0.00997029760360056\\
330	0.00997029214452369\\
331	0.00997028658620959\\
332	0.00997028092643395\\
333	0.00997027516288656\\
334	0.009970269293165\\
335	0.00997026331476755\\
336	0.00997025722508549\\
337	0.00997025102139443\\
338	0.00997024470084481\\
339	0.00997023826045129\\
340	0.00997023169708096\\
341	0.00997022500744026\\
342	0.00997021818806036\\
343	0.0099702112352809\\
344	0.00997020414523177\\
345	0.00997019691381281\\
346	0.00997018953667097\\
347	0.00997018200917466\\
348	0.00997017432638478\\
349	0.00997016648302173\\
350	0.00997015847342776\\
351	0.00997015029152418\\
352	0.00997014193076433\\
353	0.0099701333840868\\
354	0.00997012464387614\\
355	0.00997011570192416\\
356	0.00997010654933114\\
357	0.00997009717635317\\
358	0.00997008757231958\\
359	0.00997007772546795\\
360	0.00997006762263018\\
361	0.00997005724853111\\
362	0.0099700465839626\\
363	0.00997003560059562\\
364	0.00997002424554486\\
365	0.00997001239440149\\
366	0.00996999970837929\\
367	0.00996998522494726\\
368	0.00996996657780003\\
369	0.00996993646474205\\
370	0.00996990507936311\\
371	0.00996987319490772\\
372	0.0099698408035988\\
373	0.00996980789731911\\
374	0.00996977446739821\\
375	0.00996974050453822\\
376	0.00996970599978222\\
377	0.0099696709475187\\
378	0.0099696353414128\\
379	0.00996959917324234\\
380	0.00996956243436619\\
381	0.00996952511598974\\
382	0.00996948720915478\\
383	0.00996944870472804\\
384	0.00996940959338796\\
385	0.00996936986560978\\
386	0.00996932951164845\\
387	0.00996928852151937\\
388	0.00996924688497654\\
389	0.00996920459148802\\
390	0.00996916163020821\\
391	0.00996911798994705\\
392	0.00996907365913566\\
393	0.00996902862578872\\
394	0.00996898287746428\\
395	0.00996893640122287\\
396	0.00996888918359111\\
397	0.00996884121054134\\
398	0.00996879246751142\\
399	0.00996874293951026\\
400	0.00996869261137801\\
401	0.00996864146824745\\
402	0.00996858949600124\\
403	0.009968536680564\\
404	0.00996848300283412\\
405	0.00996842842818678\\
406	0.0099683729291681\\
407	0.00996831648483095\\
408	0.00996825907823155\\
409	0.00996820069884713\\
410	0.00996814133719446\\
411	0.00996808093013593\\
412	0.00996801938085864\\
413	0.00996795668555509\\
414	0.00996789284271058\\
415	0.00996782782906769\\
416	0.00996776161727656\\
417	0.00996769417936158\\
418	0.00996762548679398\\
419	0.00996755551015033\\
420	0.00996748421766628\\
421	0.00996741157296322\\
422	0.00996733753749477\\
423	0.00996726207391997\\
424	0.00996718514376581\\
425	0.00996710670680078\\
426	0.00996702672090777\\
427	0.00996694514194422\\
428	0.00996686192358747\\
429	0.00996677701716359\\
430	0.00996669037145715\\
431	0.00996660193249929\\
432	0.00996651164333072\\
433	0.00996641944373556\\
434	0.00996632526994054\\
435	0.0099662290542713\\
436	0.00996613072475194\\
437	0.00996603020461965\\
438	0.00996592741169297\\
439	0.00996582225745127\\
440	0.00996571464549786\\
441	0.00996560446869608\\
442	0.0099654916036704\\
443	0.0099653759013705\\
444	0.00996525717809596\\
445	0.00996513524235298\\
446	0.00996501006455883\\
447	0.00996488153408937\\
448	0.00996474948263035\\
449	0.00996461370973543\\
450	0.00996447399127729\\
451	0.00996433007723816\\
452	0.00996418168428262\\
453	0.00996402847544085\\
454	0.00996386999537104\\
455	0.00996370546495979\\
456	0.00996353318254629\\
457	0.00996334905208016\\
458	0.00996314400097686\\
459	0.00996221523625862\\
460	0.00996113519056949\\
461	0.00996004041913434\\
462	0.00995893062995004\\
463	0.00995780564366182\\
464	0.00995666508847368\\
465	0.00995550896793464\\
466	0.00995433689561372\\
467	0.00995314343639058\\
468	0.0099516644700794\\
469	0.00995005647906279\\
470	0.00994842701693517\\
471	0.00994677631784331\\
472	0.00994510343433459\\
473	0.00994340732390194\\
474	0.00994168683820687\\
475	0.0099399407083655\\
476	0.0099381675040042\\
477	0.00993636533629464\\
478	0.00993454594009915\\
479	0.00993269413198386\\
480	0.00993080771174991\\
481	0.00992888502425512\\
482	0.00992692420088466\\
483	0.00992492320323511\\
484	0.00992287979615216\\
485	0.00992079144033754\\
486	0.00991865480706367\\
487	0.00991646356327589\\
488	0.009914197330135\\
489	0.00990953231637944\\
490	0.00990371933406626\\
491	0.00989777880307087\\
492	0.00989170464998114\\
493	0.00988549019416691\\
494	0.00987912805813988\\
495	0.00987261006145199\\
496	0.0098659271028743\\
497	0.00985906907976534\\
498	0.00985202488750755\\
499	0.00984478251775438\\
500	0.00983732599335023\\
501	0.00982963322013123\\
502	0.00982168145822714\\
503	0.00981348801526836\\
504	0.00980504427665915\\
505	0.00979627543354369\\
506	0.00978720324055375\\
507	0.00977780324281051\\
508	0.00974946762103066\\
509	0.00966496048588523\\
510	0.00957847197711245\\
511	0.00948989807925203\\
512	0.00939912432654791\\
513	0.00930602386821295\\
514	0.00921045412905488\\
515	0.00911225690659132\\
516	0.00901124405373172\\
517	0.00890721372196417\\
518	0.0087999486559805\\
519	0.00868920757694372\\
520	0.0085747290257097\\
521	0.0084562062354396\\
522	0.00833329598139413\\
523	0.00820561825086714\\
524	0.00807272727380087\\
525	0.00793418036646053\\
526	0.00778945875051984\\
527	0.00763796023128488\\
528	0.00747977264159258\\
529	0.00731359410387879\\
530	0.00713811110598407\\
531	0.00695215354600507\\
532	0.0067614980714633\\
533	0.00656930143035399\\
534	0.00637632138398905\\
535	0.0061785466664545\\
536	0.00597574754103314\\
537	0.00576767079421171\\
538	0.00555403531562184\\
539	0.00533452942614049\\
540	0.00510882425846582\\
541	0.00495120491619428\\
542	0.00483771765206894\\
543	0.00472283222744187\\
544	0.00460649821460026\\
545	0.00448896402189749\\
546	0.00437052540171803\\
547	0.00425156557335448\\
548	0.00413258864285608\\
549	0.0040142527263529\\
550	0.00389741170323932\\
551	0.00378316824563911\\
552	0.00367294001524462\\
553	0.00356681767329889\\
554	0.00345826409295106\\
555	0.0033476284425787\\
556	0.00323546794452343\\
557	0.00312412750013787\\
558	0.00301391693601337\\
559	0.00290518133434445\\
560	0.00279830352166498\\
561	0.0026937038752877\\
562	0.00259183838832754\\
563	0.00249319226631342\\
564	0.0023982743494106\\
565	0.00230757743610891\\
566	0.00222145668415517\\
567	0.00214031063684839\\
568	0.00206192599598129\\
569	0.00198419548821309\\
570	0.00190703933181228\\
571	0.00183039326121471\\
572	0.00175407930742514\\
573	0.00167786725530715\\
574	0.00160180056386005\\
575	0.00152591978290154\\
576	0.00145025069519184\\
577	0.00137480311354071\\
578	0.00129957149013983\\
579	0.00122453819501638\\
580	0.00114967844732551\\
581	0.00107500225175364\\
582	0.00100064326835064\\
583	0.000926739340948194\\
584	0.000853430677943719\\
585	0.000780857329077455\\
586	0.000709156049726809\\
587	0.000638455314416028\\
588	0.000568870748519688\\
589	0.000500499906204788\\
590	0.000433416517188362\\
591	0.000367650704584345\\
592	0.000303181785186723\\
593	0.000239933768152231\\
594	0.000177768097533602\\
595	0.000116479891379525\\
596	6.11665484105478e-05\\
597	2.21100055488407e-05\\
598	0\\
599	0\\
600	0\\
};
\addplot [color=mycolor11,solid,forget plot]
  table[row sep=crcr]{%
1	0.00999886160658231\\
2	0.00999886160551195\\
3	0.00999886160442327\\
4	0.00999886160331596\\
5	0.00999886160218969\\
6	0.00999886160104414\\
7	0.00999886159987898\\
8	0.00999886159869388\\
9	0.00999886159748848\\
10	0.00999886159626245\\
11	0.00999886159501543\\
12	0.00999886159374705\\
13	0.00999886159245696\\
14	0.00999886159114478\\
15	0.00999886158981012\\
16	0.00999886158845262\\
17	0.00999886158707186\\
18	0.00999886158566747\\
19	0.00999886158423902\\
20	0.00999886158278612\\
21	0.00999886158130832\\
22	0.00999886157980522\\
23	0.00999886157827638\\
24	0.00999886157672136\\
25	0.0099988615751397\\
26	0.00999886157353096\\
27	0.00999886157189466\\
28	0.00999886157023033\\
29	0.0099988615685375\\
30	0.00999886156681567\\
31	0.00999886156506436\\
32	0.00999886156328304\\
33	0.00999886156147122\\
34	0.00999886155962837\\
35	0.00999886155775394\\
36	0.00999886155584741\\
37	0.00999886155390823\\
38	0.00999886155193583\\
39	0.00999886154992965\\
40	0.0099988615478891\\
41	0.0099988615458136\\
42	0.00999886154370255\\
43	0.00999886154155535\\
44	0.00999886153937136\\
45	0.00999886153714997\\
46	0.00999886153489053\\
47	0.00999886153259239\\
48	0.00999886153025489\\
49	0.00999886152787735\\
50	0.00999886152545909\\
51	0.00999886152299941\\
52	0.0099988615204976\\
53	0.00999886151795295\\
54	0.00999886151536472\\
55	0.00999886151273216\\
56	0.00999886151005451\\
57	0.009998861507331\\
58	0.00999886150456086\\
59	0.00999886150174328\\
60	0.00999886149887744\\
61	0.00999886149596253\\
62	0.00999886149299771\\
63	0.00999886148998212\\
64	0.00999886148691489\\
65	0.00999886148379514\\
66	0.00999886148062197\\
67	0.00999886147739448\\
68	0.00999886147411172\\
69	0.00999886147077277\\
70	0.00999886146737665\\
71	0.00999886146392239\\
72	0.00999886146040899\\
73	0.00999886145683546\\
74	0.00999886145320075\\
75	0.00999886144950382\\
76	0.00999886144574362\\
77	0.00999886144191905\\
78	0.00999886143802903\\
79	0.00999886143407244\\
80	0.00999886143004813\\
81	0.00999886142595495\\
82	0.00999886142179173\\
83	0.00999886141755727\\
84	0.00999886141325035\\
85	0.00999886140886975\\
86	0.00999886140441419\\
87	0.00999886139988241\\
88	0.0099988613952731\\
89	0.00999886139058494\\
90	0.00999886138581659\\
91	0.00999886138096668\\
92	0.00999886137603382\\
93	0.00999886137101659\\
94	0.00999886136591356\\
95	0.00999886136072326\\
96	0.0099988613554442\\
97	0.00999886135007489\\
98	0.00999886134461377\\
99	0.00999886133905928\\
100	0.00999886133340984\\
101	0.00999886132766382\\
102	0.00999886132181958\\
103	0.00999886131587545\\
104	0.00999886130982973\\
105	0.00999886130368069\\
106	0.00999886129742657\\
107	0.00999886129106559\\
108	0.00999886128459592\\
109	0.00999886127801571\\
110	0.00999886127132309\\
111	0.00999886126451614\\
112	0.00999886125759292\\
113	0.00999886125055145\\
114	0.00999886124338973\\
115	0.0099988612361057\\
116	0.00999886122869729\\
117	0.00999886122116239\\
118	0.00999886121349884\\
119	0.00999886120570446\\
120	0.00999886119777703\\
121	0.00999886118971429\\
122	0.00999886118151393\\
123	0.00999886117317363\\
124	0.009998861164691\\
125	0.00999886115606363\\
126	0.00999886114728907\\
127	0.00999886113836481\\
128	0.00999886112928831\\
129	0.009998861120057\\
130	0.00999886111066824\\
131	0.00999886110111936\\
132	0.00999886109140766\\
133	0.00999886108153036\\
134	0.00999886107148465\\
135	0.0099988610612677\\
136	0.00999886105087658\\
137	0.00999886104030835\\
138	0.00999886102956001\\
139	0.0099988610186285\\
140	0.00999886100751073\\
141	0.00999886099620353\\
142	0.0099988609847037\\
143	0.00999886097300798\\
144	0.00999886096111304\\
145	0.00999886094901551\\
146	0.00999886093671197\\
147	0.00999886092419893\\
148	0.00999886091147283\\
149	0.00999886089853008\\
150	0.00999886088536701\\
151	0.00999886087197989\\
152	0.00999886085836493\\
153	0.00999886084451827\\
154	0.00999886083043601\\
155	0.00999886081611415\\
156	0.00999886080154864\\
157	0.00999886078673536\\
158	0.00999886077167013\\
159	0.00999886075634869\\
160	0.0099988607407667\\
161	0.00999886072491978\\
162	0.00999886070880342\\
163	0.0099988606924131\\
164	0.00999886067574417\\
165	0.00999886065879194\\
166	0.00999886064155162\\
167	0.00999886062401834\\
168	0.00999886060618717\\
169	0.00999886058805307\\
170	0.00999886056961092\\
171	0.00999886055085554\\
172	0.00999886053178163\\
173	0.00999886051238381\\
174	0.00999886049265664\\
175	0.00999886047259455\\
176	0.00999886045219188\\
177	0.00999886043144291\\
178	0.00999886041034178\\
179	0.00999886038888257\\
180	0.00999886036705923\\
181	0.00999886034486562\\
182	0.00999886032229551\\
183	0.00999886029934255\\
184	0.00999886027600029\\
185	0.00999886025226217\\
186	0.00999886022812152\\
187	0.00999886020357156\\
188	0.0099988601786054\\
189	0.00999886015321602\\
190	0.0099988601273963\\
191	0.00999886010113899\\
192	0.00999886007443672\\
193	0.00999886004728201\\
194	0.00999886001966723\\
195	0.00999885999158465\\
196	0.00999885996302637\\
197	0.00999885993398442\\
198	0.00999885990445063\\
199	0.00999885987441673\\
200	0.00999885984387432\\
201	0.00999885981281483\\
202	0.00999885978122956\\
203	0.00999885974910967\\
204	0.00999885971644617\\
205	0.00999885968322991\\
206	0.00999885964945159\\
207	0.00999885961510175\\
208	0.0099988595801708\\
209	0.00999885954464895\\
210	0.00999885950852626\\
211	0.00999885947179264\\
212	0.0099988594344378\\
213	0.0099988593964513\\
214	0.00999885935782252\\
215	0.00999885931854066\\
216	0.00999885927859473\\
217	0.00999885923797358\\
218	0.00999885919666584\\
219	0.00999885915465996\\
220	0.00999885911194422\\
221	0.00999885906850666\\
222	0.00999885902433516\\
223	0.00999885897941737\\
224	0.00999885893374073\\
225	0.00999885888729249\\
226	0.00999885884005967\\
227	0.00999885879202906\\
228	0.00999885874318726\\
229	0.0099988586935206\\
230	0.00999885864301522\\
231	0.009998858591657\\
232	0.00999885853943159\\
233	0.00999885848632439\\
234	0.00999885843232057\\
235	0.00999885837740502\\
236	0.00999885832156241\\
237	0.00999885826477711\\
238	0.00999885820703325\\
239	0.00999885814831469\\
240	0.00999885808860501\\
241	0.0099988580278875\\
242	0.00999885796614518\\
243	0.00999885790336077\\
244	0.00999885783951671\\
245	0.00999885777459512\\
246	0.00999885770857783\\
247	0.00999885764144635\\
248	0.00999885757318187\\
249	0.00999885750376527\\
250	0.00999885743317708\\
251	0.00999885736139753\\
252	0.00999885728840647\\
253	0.00999885721418343\\
254	0.00999885713870757\\
255	0.00999885706195769\\
256	0.00999885698391225\\
257	0.0099988569045493\\
258	0.00999885682384652\\
259	0.00999885674178121\\
260	0.00999885665833028\\
261	0.00999885657347022\\
262	0.00999885648717712\\
263	0.00999885639942663\\
264	0.009998856310194\\
265	0.00999885621945404\\
266	0.0099988561271811\\
267	0.00999885603334908\\
268	0.00999885593793144\\
269	0.00999885584090112\\
270	0.00999885574223063\\
271	0.00999885564189193\\
272	0.00999885553985654\\
273	0.00999885543609545\\
274	0.00999885533057917\\
275	0.00999885522327762\\
276	0.00999885511416021\\
277	0.00999885500319577\\
278	0.00999885489035256\\
279	0.00999885477559827\\
280	0.0099988546589\\
281	0.00999885454022423\\
282	0.00999885441953683\\
283	0.00999885429680303\\
284	0.00999885417198742\\
285	0.00999885404505391\\
286	0.00999885391596574\\
287	0.00999885378468545\\
288	0.00999885365117487\\
289	0.0099988535153951\\
290	0.00999885337730647\\
291	0.00999885323686857\\
292	0.00999885309404018\\
293	0.00999885294877928\\
294	0.009998852801043\\
295	0.00999885265078764\\
296	0.0099988524979686\\
297	0.00999885234254039\\
298	0.00999885218445659\\
299	0.00999885202366979\\
300	0.00999885186013163\\
301	0.00999885169379272\\
302	0.00999885152460262\\
303	0.0099988513525098\\
304	0.00999885117746163\\
305	0.00999885099940431\\
306	0.00999885081828287\\
307	0.00999885063404112\\
308	0.00999885044662156\\
309	0.00999885025596534\\
310	0.00999885006201211\\
311	0.00999884986469999\\
312	0.00999884966396568\\
313	0.00999884945974437\\
314	0.0099988492519696\\
315	0.00999884904057318\\
316	0.00999884882548513\\
317	0.00999884860663358\\
318	0.00999884838394468\\
319	0.00999884815734249\\
320	0.00999884792674886\\
321	0.00999884769208335\\
322	0.00999884745326304\\
323	0.00999884721020244\\
324	0.00999884696281328\\
325	0.0099988467110044\\
326	0.00999884645468148\\
327	0.00999884619374692\\
328	0.00999884592809954\\
329	0.00999884565763438\\
330	0.0099988453822424\\
331	0.00999884510181016\\
332	0.00999884481621952\\
333	0.00999884452534724\\
334	0.00999884422906459\\
335	0.00999884392723685\\
336	0.00999884361972287\\
337	0.00999884330637445\\
338	0.00999884298703574\\
339	0.00999884266154254\\
340	0.00999884232972153\\
341	0.00999884199138937\\
342	0.00999884164635183\\
343	0.00999884129440259\\
344	0.00999884093532213\\
345	0.00999884056887634\\
346	0.00999884019481503\\
347	0.00999883981287024\\
348	0.00999883942275431\\
349	0.00999883902415775\\
350	0.00999883861674684\\
351	0.00999883820016082\\
352	0.00999883777400891\\
353	0.00999883733786695\\
354	0.00999883689127398\\
355	0.00999883643372824\\
356	0.00999883596468079\\
357	0.00999883548352641\\
358	0.009998834989593\\
359	0.00999883448211997\\
360	0.00999883396021254\\
361	0.0099988334227364\\
362	0.00999883286806642\\
363	0.00999883229348891\\
364	0.00999883169382415\\
365	0.00999883105845305\\
366	0.00999883036571932\\
367	0.00999882957683947\\
368	0.00999882867617437\\
369	0.0099988277547941\\
370	0.00999882681921332\\
371	0.00999882586922991\\
372	0.00999882490463685\\
373	0.00999882392521911\\
374	0.00999882293074936\\
375	0.00999882192098873\\
376	0.00999882089571304\\
377	0.00999881985478002\\
378	0.00999881879801241\\
379	0.00999881772519101\\
380	0.00999881663609405\\
381	0.00999881553049709\\
382	0.00999881440817278\\
383	0.00999881326889071\\
384	0.00999881211241709\\
385	0.00999881093851453\\
386	0.00999880974694162\\
387	0.00999880853745256\\
388	0.00999880730979668\\
389	0.0099988060637179\\
390	0.00999880479895408\\
391	0.00999880351523635\\
392	0.00999880221228826\\
393	0.00999880088982494\\
394	0.00999879954755211\\
395	0.0099987981851652\\
396	0.0099987968023486\\
397	0.00999879539877554\\
398	0.00999879397410944\\
399	0.00999879252800816\\
400	0.0099987910601337\\
401	0.0099987895701702\\
402	0.00999878805784939\\
403	0.00999878652296952\\
404	0.00999878496537107\\
405	0.00999878338492933\\
406	0.00999878178253733\\
407	0.00999878016043559\\
408	0.00999877852213052\\
409	0.00999877687002016\\
410	0.00999877519870064\\
411	0.00999877349877619\\
412	0.00999877176810963\\
413	0.00999877000698754\\
414	0.00999876821525015\\
415	0.0099987663922394\\
416	0.0099987645372807\\
417	0.00999876264968584\\
418	0.00999876072875439\\
419	0.00999875877376427\\
420	0.00999875678393502\\
421	0.00999875475837485\\
422	0.00999875269615624\\
423	0.00999875059639194\\
424	0.00999874845815901\\
425	0.00999874628049661\\
426	0.00999874406240368\\
427	0.00999874180283633\\
428	0.00999873950070511\\
429	0.00999873715487195\\
430	0.0099987347641469\\
431	0.00999873232728452\\
432	0.00999872984297999\\
433	0.00999872730986472\\
434	0.00999872472650155\\
435	0.00999872209137928\\
436	0.00999871940290623\\
437	0.0099987166594021\\
438	0.00999871385908642\\
439	0.00999871100005988\\
440	0.00999870808027008\\
441	0.00999870509744464\\
442	0.00999870204896453\\
443	0.0099986989316628\\
444	0.00999869574169428\\
445	0.00999869247534774\\
446	0.00999868913309026\\
447	0.0099986857126465\\
448	0.00999868221092105\\
449	0.00999867862430996\\
450	0.00999867494887379\\
451	0.00999867118020179\\
452	0.00999866731305387\\
453	0.00999866334034507\\
454	0.00999865925026592\\
455	0.00999865501853829\\
456	0.00999865059012545\\
457	0.00999864584946905\\
458	0.00999864064092852\\
459	0.00999863528250578\\
460	0.00999862983254051\\
461	0.00999862426219808\\
462	0.0099986185225333\\
463	0.0099986125415341\\
464	0.00999860627307679\\
465	0.00999859953783219\\
466	0.00999859156724826\\
467	0.0099985804685139\\
468	0.00999856122720195\\
469	0.00999853882104121\\
470	0.00999851599136136\\
471	0.00999849271598111\\
472	0.00999846897430738\\
473	0.00999844475242065\\
474	0.00999842005282698\\
475	0.00999839490691326\\
476	0.00999836936563406\\
477	0.00999834337238239\\
478	0.0099983043124748\\
479	0.0099982634295283\\
480	0.00999822129941152\\
481	0.00999817782978874\\
482	0.00999813290841915\\
483	0.00999808638303712\\
484	0.00999803800808412\\
485	0.00999798730127267\\
486	0.00999793315762983\\
487	0.00999787285584563\\
488	0.00999780014810592\\
489	0.00999766218044758\\
490	0.00999749457900609\\
491	0.0099973232727647\\
492	0.00999714810679892\\
493	0.0099969689184178\\
494	0.00999678554375719\\
495	0.0099965978334026\\
496	0.00999640568167148\\
497	0.00999620905808177\\
498	0.00999600794429296\\
499	0.00999580187601628\\
500	0.0099955898615952\\
501	0.00999537133399394\\
502	0.00999514566352727\\
503	0.00999487198030658\\
504	0.00999453197877233\\
505	0.00999417733984137\\
506	0.00999380368710282\\
507	0.00999339824171929\\
508	0.00999249863479512\\
509	0.00999024138993689\\
510	0.009987969291116\\
511	0.00998568193103925\\
512	0.00998337888068322\\
513	0.00998105967078947\\
514	0.0099787245288255\\
515	0.00997637157505996\\
516	0.0099739989746714\\
517	0.0099716055794803\\
518	0.00996919032502216\\
519	0.00996675219624949\\
520	0.00996428303999693\\
521	0.00996178664909822\\
522	0.0099592643678395\\
523	0.00995670698165256\\
524	0.00995411392853342\\
525	0.00995148465139348\\
526	0.00994881150400438\\
527	0.0099460695764821\\
528	0.00994247452732493\\
529	0.00993846045303648\\
530	0.00993432248483253\\
531	0.00993006061103056\\
532	0.00992571852724962\\
533	0.00991822796978299\\
534	0.00990672774350531\\
535	0.00989493303834666\\
536	0.00988281219865795\\
537	0.00987032853565887\\
538	0.00985743862467483\\
539	0.00984408667533638\\
540	0.00983017945278123\\
541	0.00974272020075207\\
542	0.00960566597158402\\
543	0.00946373262367878\\
544	0.00931647391436078\\
545	0.00916338120362373\\
546	0.00900387131459727\\
547	0.00883727334291648\\
548	0.00866281236164742\\
549	0.00847958965623579\\
550	0.00828655898260977\\
551	0.00808249893165934\\
552	0.00786598465982756\\
553	0.00763699834573866\\
554	0.00740195069507757\\
555	0.00716052774043377\\
556	0.00691239029212287\\
557	0.00665715505285347\\
558	0.00639441629849354\\
559	0.00612378737998776\\
560	0.00584478903828285\\
561	0.00555693540262497\\
562	0.00525977774490818\\
563	0.00495305641811948\\
564	0.0046363667826996\\
565	0.00431107070286592\\
566	0.00398614249697624\\
567	0.00366456760228114\\
568	0.00349609316333513\\
569	0.00333187726868771\\
570	0.0031732896727137\\
571	0.00302187273812389\\
572	0.00287974362373755\\
573	0.0027468448869553\\
574	0.00261504272331021\\
575	0.00248398048505642\\
576	0.00235392470609242\\
577	0.00222512595041651\\
578	0.00209779280040771\\
579	0.00197206281766885\\
580	0.00184806598976503\\
581	0.00172537279860785\\
582	0.00160240632595472\\
583	0.00147944176473777\\
584	0.00135679239519785\\
585	0.00123481760980238\\
586	0.00111393050009632\\
587	0.000994646016777647\\
588	0.000877530773009266\\
589	0.00076320155692426\\
590	0.00065232074510787\\
591	0.000545589588426686\\
592	0.000443742268922828\\
593	0.00034753281156088\\
594	0.000257714783085832\\
595	0.000175011746328155\\
596	0.000100073878205332\\
597	3.33540724222573e-05\\
598	0\\
599	0\\
600	0\\
};
\addplot [color=mycolor12,solid,forget plot]
  table[row sep=crcr]{%
1	0.00999970439610195\\
2	0.0099997043960851\\
3	0.00999970439606796\\
4	0.00999970439605053\\
5	0.0099997043960328\\
6	0.00999970439601477\\
7	0.00999970439599642\\
8	0.00999970439597777\\
9	0.00999970439595879\\
10	0.00999970439593949\\
11	0.00999970439591986\\
12	0.00999970439589989\\
13	0.00999970439587958\\
14	0.00999970439585892\\
15	0.00999970439583791\\
16	0.00999970439581654\\
17	0.0099997043957948\\
18	0.00999970439577269\\
19	0.0099997043957502\\
20	0.00999970439572733\\
21	0.00999970439570406\\
22	0.0099997043956804\\
23	0.00999970439565633\\
24	0.00999970439563185\\
25	0.00999970439560695\\
26	0.00999970439558162\\
27	0.00999970439555586\\
28	0.00999970439552966\\
29	0.00999970439550301\\
30	0.0099997043954759\\
31	0.00999970439544833\\
32	0.00999970439542029\\
33	0.00999970439539176\\
34	0.00999970439536275\\
35	0.00999970439533324\\
36	0.00999970439530323\\
37	0.0099997043952727\\
38	0.00999970439524165\\
39	0.00999970439521006\\
40	0.00999970439517794\\
41	0.00999970439514526\\
42	0.00999970439511203\\
43	0.00999970439507823\\
44	0.00999970439504384\\
45	0.00999970439500887\\
46	0.0099997043949733\\
47	0.00999970439493712\\
48	0.00999970439490032\\
49	0.00999970439486289\\
50	0.00999970439482482\\
51	0.0099997043947861\\
52	0.00999970439474671\\
53	0.00999970439470665\\
54	0.0099997043946659\\
55	0.00999970439462446\\
56	0.00999970439458231\\
57	0.00999970439453943\\
58	0.00999970439449582\\
59	0.00999970439445146\\
60	0.00999970439440635\\
61	0.00999970439436046\\
62	0.00999970439431378\\
63	0.00999970439426631\\
64	0.00999970439421802\\
65	0.00999970439416891\\
66	0.00999970439411895\\
67	0.00999970439406814\\
68	0.00999970439401646\\
69	0.0099997043939639\\
70	0.00999970439391043\\
71	0.00999970439385606\\
72	0.00999970439380075\\
73	0.00999970439374449\\
74	0.00999970439368727\\
75	0.00999970439362907\\
76	0.00999970439356987\\
77	0.00999970439350967\\
78	0.00999970439344843\\
79	0.00999970439338614\\
80	0.00999970439332279\\
81	0.00999970439325835\\
82	0.00999970439319281\\
83	0.00999970439312615\\
84	0.00999970439305835\\
85	0.00999970439298939\\
86	0.00999970439291925\\
87	0.00999970439284791\\
88	0.00999970439277535\\
89	0.00999970439270155\\
90	0.00999970439262648\\
91	0.00999970439255013\\
92	0.00999970439247248\\
93	0.0099997043923935\\
94	0.00999970439231317\\
95	0.00999970439223146\\
96	0.00999970439214836\\
97	0.00999970439206384\\
98	0.00999970439197787\\
99	0.00999970439189043\\
100	0.0099997043918015\\
101	0.00999970439171105\\
102	0.00999970439161905\\
103	0.00999970439152548\\
104	0.00999970439143031\\
105	0.00999970439133352\\
106	0.00999970439123507\\
107	0.00999970439113494\\
108	0.00999970439103309\\
109	0.00999970439092951\\
110	0.00999970439082416\\
111	0.00999970439071701\\
112	0.00999970439060803\\
113	0.00999970439049719\\
114	0.00999970439038446\\
115	0.0099997043902698\\
116	0.00999970439015319\\
117	0.00999970439003458\\
118	0.00999970438991395\\
119	0.00999970438979126\\
120	0.00999970438966647\\
121	0.00999970438953956\\
122	0.00999970438941048\\
123	0.0099997043892792\\
124	0.00999970438914568\\
125	0.00999970438900988\\
126	0.00999970438887176\\
127	0.00999970438873129\\
128	0.00999970438858843\\
129	0.00999970438844312\\
130	0.00999970438829534\\
131	0.00999970438814504\\
132	0.00999970438799218\\
133	0.00999970438783671\\
134	0.00999970438767859\\
135	0.00999970438751778\\
136	0.00999970438735422\\
137	0.00999970438718788\\
138	0.00999970438701871\\
139	0.00999970438684665\\
140	0.00999970438667166\\
141	0.00999970438649369\\
142	0.00999970438631269\\
143	0.00999970438612861\\
144	0.00999970438594139\\
145	0.00999970438575099\\
146	0.00999970438555734\\
147	0.0099997043853604\\
148	0.0099997043851601\\
149	0.00999970438495639\\
150	0.00999970438474922\\
151	0.00999970438453853\\
152	0.00999970438432425\\
153	0.00999970438410632\\
154	0.00999970438388469\\
155	0.00999970438365928\\
156	0.00999970438343004\\
157	0.00999970438319691\\
158	0.00999970438295981\\
159	0.00999970438271868\\
160	0.00999970438247345\\
161	0.00999970438222405\\
162	0.00999970438197042\\
163	0.00999970438171247\\
164	0.00999970438145014\\
165	0.00999970438118335\\
166	0.00999970438091203\\
167	0.0099997043806361\\
168	0.00999970438035549\\
169	0.00999970438007011\\
170	0.00999970437977988\\
171	0.00999970437948472\\
172	0.00999970437918456\\
173	0.0099997043788793\\
174	0.00999970437856885\\
175	0.00999970437825314\\
176	0.00999970437793208\\
177	0.00999970437760556\\
178	0.0099997043772735\\
179	0.00999970437693581\\
180	0.0099997043765924\\
181	0.00999970437624316\\
182	0.00999970437588799\\
183	0.00999970437552681\\
184	0.0099997043751595\\
185	0.00999970437478596\\
186	0.00999970437440609\\
187	0.00999970437401979\\
188	0.00999970437362694\\
189	0.00999970437322743\\
190	0.00999970437282115\\
191	0.00999970437240799\\
192	0.00999970437198783\\
193	0.00999970437156055\\
194	0.00999970437112604\\
195	0.00999970437068418\\
196	0.00999970437023483\\
197	0.00999970436977787\\
198	0.00999970436931317\\
199	0.00999970436884062\\
200	0.00999970436836006\\
201	0.00999970436787137\\
202	0.00999970436737442\\
203	0.00999970436686905\\
204	0.00999970436635514\\
205	0.00999970436583253\\
206	0.00999970436530108\\
207	0.00999970436476065\\
208	0.00999970436421108\\
209	0.00999970436365222\\
210	0.0099997043630839\\
211	0.00999970436250598\\
212	0.0099997043619183\\
213	0.00999970436132067\\
214	0.00999970436071295\\
215	0.00999970436009496\\
216	0.00999970435946653\\
217	0.00999970435882748\\
218	0.00999970435817764\\
219	0.00999970435751681\\
220	0.00999970435684483\\
221	0.0099997043561615\\
222	0.00999970435546662\\
223	0.00999970435476001\\
224	0.00999970435404147\\
225	0.0099997043533108\\
226	0.0099997043525678\\
227	0.00999970435181225\\
228	0.00999970435104395\\
229	0.00999970435026268\\
230	0.00999970434946822\\
231	0.00999970434866035\\
232	0.00999970434783885\\
233	0.00999970434700349\\
234	0.00999970434615404\\
235	0.00999970434529025\\
236	0.00999970434441188\\
237	0.0099997043435187\\
238	0.00999970434261045\\
239	0.00999970434168687\\
240	0.00999970434074772\\
241	0.00999970433979272\\
242	0.00999970433882161\\
243	0.00999970433783413\\
244	0.00999970433682998\\
245	0.0099997043358089\\
246	0.0099997043347706\\
247	0.00999970433371478\\
248	0.00999970433264115\\
249	0.00999970433154941\\
250	0.00999970433043926\\
251	0.00999970432931038\\
252	0.00999970432816247\\
253	0.00999970432699518\\
254	0.00999970432580821\\
255	0.00999970432460121\\
256	0.00999970432337385\\
257	0.00999970432212578\\
258	0.00999970432085666\\
259	0.00999970431956612\\
260	0.0099997043182538\\
261	0.00999970431691934\\
262	0.00999970431556236\\
263	0.00999970431418247\\
264	0.00999970431277928\\
265	0.00999970431135241\\
266	0.00999970430990145\\
267	0.00999970430842599\\
268	0.0099997043069256\\
269	0.00999970430539988\\
270	0.00999970430384837\\
271	0.00999970430227065\\
272	0.00999970430066627\\
273	0.00999970429903477\\
274	0.00999970429737569\\
275	0.00999970429568854\\
276	0.00999970429397287\\
277	0.00999970429222817\\
278	0.00999970429045394\\
279	0.00999970428864967\\
280	0.00999970428681486\\
281	0.00999970428494897\\
282	0.00999970428305146\\
283	0.0099997042811218\\
284	0.00999970427915942\\
285	0.00999970427716375\\
286	0.00999970427513421\\
287	0.00999970427307023\\
288	0.00999970427097119\\
289	0.00999970426883648\\
290	0.00999970426666549\\
291	0.00999970426445756\\
292	0.00999970426221207\\
293	0.00999970425992833\\
294	0.00999970425760568\\
295	0.00999970425524343\\
296	0.00999970425284088\\
297	0.0099997042503973\\
298	0.00999970424791197\\
299	0.00999970424538413\\
300	0.00999970424281304\\
301	0.00999970424019789\\
302	0.00999970423753791\\
303	0.00999970423483226\\
304	0.00999970423208013\\
305	0.00999970422928066\\
306	0.00999970422643297\\
307	0.00999970422353618\\
308	0.00999970422058937\\
309	0.00999970421759161\\
310	0.00999970421454195\\
311	0.0099997042114394\\
312	0.00999970420828295\\
313	0.00999970420507158\\
314	0.00999970420180422\\
315	0.00999970419847978\\
316	0.00999970419509715\\
317	0.00999970419165518\\
318	0.00999970418815268\\
319	0.00999970418458844\\
320	0.0099997041809612\\
321	0.00999970417726965\\
322	0.00999970417351248\\
323	0.00999970416968828\\
324	0.00999970416579564\\
325	0.00999970416183308\\
326	0.00999970415779904\\
327	0.00999970415369196\\
328	0.00999970414951016\\
329	0.00999970414525193\\
330	0.00999970414091547\\
331	0.00999970413649891\\
332	0.00999970413200029\\
333	0.00999970412741755\\
334	0.00999970412274857\\
335	0.00999970411799107\\
336	0.00999970411314268\\
337	0.00999970410820091\\
338	0.0099997041031631\\
339	0.00999970409802647\\
340	0.00999970409278805\\
341	0.00999970408744469\\
342	0.00999970408199303\\
343	0.00999970407642951\\
344	0.0099997040707503\\
345	0.0099997040649513\\
346	0.00999970405902812\\
347	0.00999970405297602\\
348	0.00999970404678991\\
349	0.00999970404046427\\
350	0.00999970403399313\\
351	0.00999970402737001\\
352	0.00999970402058787\\
353	0.00999970401363903\\
354	0.0099997040065151\\
355	0.00999970399920691\\
356	0.00999970399170435\\
357	0.00999970398399627\\
358	0.0099997039760702\\
359	0.0099997039679119\\
360	0.00999970395950438\\
361	0.00999970395082565\\
362	0.00999970394184347\\
363	0.00999970393250351\\
364	0.00999970392270427\\
365	0.00999970391225176\\
366	0.00999970390080888\\
367	0.00999970388800135\\
368	0.00999970387491621\\
369	0.00999970386163212\\
370	0.00999970384814638\\
371	0.00999970383445628\\
372	0.00999970382055909\\
373	0.00999970380645204\\
374	0.00999970379213237\\
375	0.00999970377759739\\
376	0.00999970376284445\\
377	0.00999970374787081\\
378	0.00999970373267359\\
379	0.00999970371724991\\
380	0.00999970370159686\\
381	0.00999970368571152\\
382	0.00999970366959092\\
383	0.0099997036532321\\
384	0.00999970363663206\\
385	0.00999970361978774\\
386	0.00999970360269609\\
387	0.00999970358535398\\
388	0.00999970356775827\\
389	0.00999970354990574\\
390	0.00999970353179314\\
391	0.00999970351341711\\
392	0.00999970349477425\\
393	0.00999970347586105\\
394	0.00999970345667391\\
395	0.00999970343720909\\
396	0.00999970341746275\\
397	0.00999970339743088\\
398	0.00999970337710936\\
399	0.00999970335649391\\
400	0.00999970333558022\\
401	0.00999970331436403\\
402	0.00999970329284148\\
403	0.00999970327101007\\
404	0.00999970324887127\\
405	0.00999970322643686\\
406	0.0099997032037377\\
407	0.00999970318083527\\
408	0.00999970315782051\\
409	0.00999970313475182\\
410	0.00999970311148934\\
411	0.00999970308782666\\
412	0.00999970306375765\\
413	0.0099997030392748\\
414	0.00999970301436928\\
415	0.009999702989032\\
416	0.00999970296325354\\
417	0.00999970293702415\\
418	0.00999970291033376\\
419	0.00999970288317189\\
420	0.00999970285552771\\
421	0.00999970282738997\\
422	0.00999970279874702\\
423	0.00999970276958678\\
424	0.00999970273989666\\
425	0.00999970270966362\\
426	0.00999970267887407\\
427	0.00999970264751387\\
428	0.00999970261556831\\
429	0.00999970258302206\\
430	0.00999970254985913\\
431	0.00999970251606284\\
432	0.00999970248161578\\
433	0.00999970244649979\\
434	0.00999970241069586\\
435	0.00999970237418414\\
436	0.00999970233694383\\
437	0.00999970229895314\\
438	0.00999970226018924\\
439	0.0099997022206281\\
440	0.00999970218024436\\
441	0.00999970213901119\\
442	0.00999970209690025\\
443	0.00999970205388205\\
444	0.00999970200992762\\
445	0.00999970196501004\\
446	0.00999970191909571\\
447	0.00999970187214897\\
448	0.00999970182413208\\
449	0.00999970177500489\\
450	0.00999970172472381\\
451	0.00999970167323947\\
452	0.00999970162049073\\
453	0.00999970156639009\\
454	0.00999970151079001\\
455	0.00999970145341547\\
456	0.00999970139377821\\
457	0.00999970133134347\\
458	0.00999970126761638\\
459	0.00999970120280487\\
460	0.0099997011366592\\
461	0.00999970106863759\\
462	0.00999970099764383\\
463	0.0099997009219837\\
464	0.00999970084079433\\
465	0.00999970074964362\\
466	0.00999970062952272\\
467	0.00999970043629822\\
468	0.00999970021503002\\
469	0.00999969999009975\\
470	0.009999699761335\\
471	0.00999969952858772\\
472	0.00999969929179549\\
473	0.00999969905111619\\
474	0.00999969880718207\\
475	0.00999969856142491\\
476	0.00999969831586726\\
477	0.00999969807009733\\
478	0.00999969749713267\\
479	0.00999969689104109\\
480	0.00999969626755884\\
481	0.0099996956253699\\
482	0.00999969496275721\\
483	0.00999969427710351\\
484	0.00999969356356345\\
485	0.00999969281150664\\
486	0.00999969199506879\\
487	0.00999969104944278\\
488	0.00999968982966578\\
489	0.00999968849158797\\
490	0.00999968712942908\\
491	0.00999968574244676\\
492	0.00999968432990924\\
493	0.00999968289118196\\
494	0.00999968142593702\\
495	0.00999967993459879\\
496	0.0099996784191395\\
497	0.00999967688390242\\
498	0.00999967533397796\\
499	0.00999967376351339\\
500	0.00999967215553076\\
501	0.00999967050724057\\
502	0.00999966881438454\\
503	0.00999966602971879\\
504	0.00999966169299949\\
505	0.00999965715544551\\
506	0.00999965232498029\\
507	0.00999964689477532\\
508	0.00999964070608822\\
509	0.00999963444375094\\
510	0.00999962810684547\\
511	0.00999962169542267\\
512	0.00999961521061403\\
513	0.00999960865463481\\
514	0.00999960205048303\\
515	0.00999959537010204\\
516	0.00999958859516865\\
517	0.00999958172256433\\
518	0.00999957475061019\\
519	0.00999956768088965\\
520	0.0099995603333951\\
521	0.00999955283804126\\
522	0.00999954526454885\\
523	0.00999953741587508\\
524	0.00999952932201664\\
525	0.0099995209888636\\
526	0.00999951225116848\\
527	0.00999950251419232\\
528	0.00999947180552544\\
529	0.00999943130832168\\
530	0.00999938862521251\\
531	0.00999934381240677\\
532	0.00999929575636018\\
533	0.00999916658108661\\
534	0.00999893491907451\\
535	0.00999869511603763\\
536	0.00999844639114465\\
537	0.00999818784741408\\
538	0.00999791843179201\\
539	0.00999763679861141\\
540	0.00999734069394201\\
541	0.00999523880893332\\
542	0.00999199346653391\\
543	0.00998872650467276\\
544	0.00998543582765006\\
545	0.00998211912727778\\
546	0.00997877390987515\\
547	0.00997539751394421\\
548	0.00997198713834896\\
549	0.00996853989228927\\
550	0.00996505288167449\\
551	0.00996152336179517\\
552	0.00995794903948198\\
553	0.00995435390004079\\
554	0.00995085585706864\\
555	0.00994745821629246\\
556	0.00994416444601642\\
557	0.00994097784367608\\
558	0.00993790230813352\\
559	0.00993494256110365\\
560	0.00993210290653603\\
561	0.00992938828563122\\
562	0.00992680457965828\\
563	0.00992435954354065\\
564	0.0099220597293271\\
565	0.00991813705201656\\
566	0.0099034258963405\\
567	0.00987473672518173\\
568	0.00968562218198496\\
569	0.00948588057623802\\
570	0.00927415985470431\\
571	0.00904887865168875\\
572	0.00880817797106934\\
573	0.00855229866101873\\
574	0.00828890720014428\\
575	0.00801779203946057\\
576	0.00773867093809746\\
577	0.00745130180414575\\
578	0.00715572070649497\\
579	0.00685181162239703\\
580	0.00653965469332596\\
581	0.0062206368942548\\
582	0.00589643732241719\\
583	0.00556694102676614\\
584	0.00523207996332671\\
585	0.004891739957962\\
586	0.00454581215895712\\
587	0.00419420467885093\\
588	0.0038368720520721\\
589	0.00347386032432953\\
590	0.0031053639618755\\
591	0.00273166653538783\\
592	0.00235288756260034\\
593	0.00196919002164565\\
594	0.00158084090714511\\
595	0.00118829559758769\\
596	0.000792401457307563\\
597	0.00039492475451582\\
598	0\\
599	0\\
600	0\\
};
\addplot [color=mycolor13,solid,forget plot]
  table[row sep=crcr]{%
1	0.000241064050983103\\
2	0.000241064050983103\\
3	0.000241064050983103\\
4	0.000241064050983103\\
5	0.000241064050983103\\
6	0.000241064050983103\\
7	0.000241064050983103\\
8	0.000241064050983103\\
9	0.000241064050983103\\
10	0.000241064050983103\\
11	0.000241064050983103\\
12	0.000241064050983103\\
13	0.000241064050983103\\
14	0.000241064050983103\\
15	0.000241064050983103\\
16	0.000241064050983103\\
17	0.000241064050983103\\
18	0.000241064050983103\\
19	0.000241064050983103\\
20	0.000241064050983103\\
21	0.000241064050983103\\
22	0.000241064050983103\\
23	0.000241064050983103\\
24	0.000241064050983103\\
25	0.000241064050983103\\
26	0.000241064050983103\\
27	0.000241064050983103\\
28	0.000241064050983103\\
29	0.000241064050983103\\
30	0.000241064050983103\\
31	0.000241064050983103\\
32	0.000241064050983103\\
33	0.000241064050983103\\
34	0.000241064050983103\\
35	0.000241064050983103\\
36	0.000241064050983103\\
37	0.000241064050983103\\
38	0.000241064050983103\\
39	0.000241064050983103\\
40	0.000241064050983103\\
41	0.000241064050983103\\
42	0.000241064050983103\\
43	0.000241064050983103\\
44	0.000241064050983103\\
45	0.000241064050983103\\
46	0.000241064050983103\\
47	0.000241064050983103\\
48	0.000241064050983103\\
49	0.000241064050983103\\
50	0.000241064050983103\\
51	0.000241064050983103\\
52	0.000241064050983103\\
53	0.000241064050983103\\
54	0.000241064050983103\\
55	0.000241064050983103\\
56	0.000241064050983103\\
57	0.000241064050983103\\
58	0.000241064050983103\\
59	0.000241064050983103\\
60	0.000241064050983103\\
61	0.000241064050983103\\
62	0.000241064050983103\\
63	0.000241064050983103\\
64	0.000241064050983103\\
65	0.000241064050983103\\
66	0.000241064050983103\\
67	0.000241064050983103\\
68	0.000241064050983103\\
69	0.000241064050983103\\
70	0.000241064050983103\\
71	0.000241064050983103\\
72	0.000241064050983103\\
73	0.000241064050983103\\
74	0.000241064050983103\\
75	0.000241064050983103\\
76	0.000241064050983103\\
77	0.000241064050983103\\
78	0.000241064050983103\\
79	0.000241064050983103\\
80	0.000241064050983103\\
81	0.000241064050983103\\
82	0.000241064050983103\\
83	0.000241064050983103\\
84	0.000241064050983103\\
85	0.000241064050983103\\
86	0.000241064050983103\\
87	0.000241064050983103\\
88	0.000241064050983103\\
89	0.000241064050983103\\
90	0.000241064050983103\\
91	0.000241064050983103\\
92	0.000241064050983103\\
93	0.000241064050983103\\
94	0.000241064050983103\\
95	0.000241064050983103\\
96	0.000241064050983103\\
97	0.000241064050983103\\
98	0.000241064050983103\\
99	0.000241064050983103\\
100	0.000241064050983103\\
101	0.000241064050983103\\
102	0.000241064050983103\\
103	0.000241064050983103\\
104	0.000241064050983103\\
105	0.000241064050983103\\
106	0.000241064050983103\\
107	0.000241064050983103\\
108	0.000241064050983103\\
109	0.000241064050983103\\
110	0.000241064050983103\\
111	0.000241064050983103\\
112	0.000241064050983103\\
113	0.000241064050983103\\
114	0.000241064050983103\\
115	0.000241064050983103\\
116	0.000241064050983103\\
117	0.000241064050983103\\
118	0.000241064050983103\\
119	0.000241064050983103\\
120	0.000241064050983103\\
121	0.000241064050983103\\
122	0.000241064050983103\\
123	0.000241064050983103\\
124	0.000241064050983103\\
125	0.000241064050983103\\
126	0.000241064050983103\\
127	0.000241064050983103\\
128	0.000241064050983103\\
129	0.000241064050983103\\
130	0.000241064050983103\\
131	0.000241064050983103\\
132	0.000241064050983103\\
133	0.000241064050983103\\
134	0.000241064050983103\\
135	0.000241064050983103\\
136	0.000241064050983103\\
137	0.000241064050983103\\
138	0.000241064050983103\\
139	0.000241064050983103\\
140	0.000241064050983103\\
141	0.000241064050983103\\
142	0.000241064050983103\\
143	0.000241064050983103\\
144	0.000241064050983103\\
145	0.000241064050983103\\
146	0.000241064050983103\\
147	0.000241064050983103\\
148	0.000241064050983103\\
149	0.000241064050983103\\
150	0.000241064050983103\\
151	0.000241064050983103\\
152	0.000241064050983103\\
153	0.000241064050983103\\
154	0.000241064050983103\\
155	0.000241064050983103\\
156	0.000241064050983103\\
157	0.000241064050983103\\
158	0.000241064050983103\\
159	0.000241064050983103\\
160	0.000241064050983103\\
161	0.000241064050983103\\
162	0.000241064050983103\\
163	0.000241064050983103\\
164	0.000241064050983103\\
165	0.000241064050983103\\
166	0.000241064050983103\\
167	0.000241064050983103\\
168	0.000241064050983103\\
169	0.000241064050983103\\
170	0.000241064050983103\\
171	0.000241064050983103\\
172	0.000241064050983103\\
173	0.000241064050983103\\
174	0.000241064050983103\\
175	0.000241064050983103\\
176	0.000241064050983103\\
177	0.000241064050983103\\
178	0.000241064050983103\\
179	0.000241064050983103\\
180	0.000241064050983103\\
181	0.000241064050983103\\
182	0.000241064050983103\\
183	0.000241064050983103\\
184	0.000241064050983103\\
185	0.000241064050983103\\
186	0.000241064050983103\\
187	0.000241064050983103\\
188	0.000241064050983103\\
189	0.000241064050983103\\
190	0.000241064050983103\\
191	0.000241064050983103\\
192	0.000241064050983103\\
193	0.000241064050983103\\
194	0.000241064050983103\\
195	0.000241064050983103\\
196	0.000241064050983103\\
197	0.000241064050983103\\
198	0.000241064050983103\\
199	0.000241064050983103\\
200	0.000241064050983103\\
201	0.000241064050983103\\
202	0.000241064050983103\\
203	0.000241064050983103\\
204	0.000241064050983103\\
205	0.000241064050983103\\
206	0.000241064050983103\\
207	0.000241064050983103\\
208	0.000241064050983103\\
209	0.000241064050983103\\
210	0.000241064050983103\\
211	0.000241064050983103\\
212	0.000241064050983103\\
213	0.000241064050983103\\
214	0.000241064050983103\\
215	0.000241064050983103\\
216	0.000241064050983103\\
217	0.000241064050983103\\
218	0.000241064050983103\\
219	0.000241064050983103\\
220	0.000241064050983103\\
221	0.000241064050983103\\
222	0.000241064050983103\\
223	0.000241064050983103\\
224	0.000241064050983103\\
225	0.000241064050983103\\
226	0.000241064050983103\\
227	0.000241064050983103\\
228	0.000241064050983103\\
229	0.000241064050983103\\
230	0.000241064050983103\\
231	0.000241064050983103\\
232	0.000241064050983103\\
233	0.000241064050983103\\
234	0.000241064050983103\\
235	0.000241064050983103\\
236	0.000241064050983103\\
237	0.000241064050983103\\
238	0.000241064050983103\\
239	0.000241064050983103\\
240	0.000241064050983103\\
241	0.000241064050983103\\
242	0.000241064050983103\\
243	0.000241064050983103\\
244	0.000241064050983103\\
245	0.000241064050983103\\
246	0.000241064050983103\\
247	0.000241064050983103\\
248	0.000241064050983103\\
249	0.000241064050983103\\
250	0.000241064050983103\\
251	0.000241064050983103\\
252	0.000241064050983103\\
253	0.000241064050983103\\
254	0.000241064050983103\\
255	0.000241064050983103\\
256	0.000241064050983103\\
257	0.000241064050983103\\
258	0.000241064050983103\\
259	0.000241064050983103\\
260	0.000241064050983103\\
261	0.000241064050983103\\
262	0.000241064050983103\\
263	0.000241064050983103\\
264	0.000241064050983103\\
265	0.000241064050983103\\
266	0.000241064050983103\\
267	0.000241064050983103\\
268	0.000241064050983103\\
269	0.000241064050983103\\
270	0.000241064050983103\\
271	0.000241064050983103\\
272	0.000241064050983103\\
273	0.000241064050983103\\
274	0.000241064050983103\\
275	0.000241064050983103\\
276	0.000241064050983103\\
277	0.000241064050983103\\
278	0.000241064050983103\\
279	0.000241064050983103\\
280	0.000241064050983103\\
281	0.000241064050983103\\
282	0.000241064050983103\\
283	0.000241064050983103\\
284	0.000241064050983103\\
285	0.000241064050983103\\
286	0.000241064050983103\\
287	0.000241064050983103\\
288	0.000241064050983103\\
289	0.000241064050983103\\
290	0.000241064050983103\\
291	0.000241064050983103\\
292	0.000241064050983103\\
293	0.000241064050983103\\
294	0.000241064050983103\\
295	0.000241064050983103\\
296	0.000241064050983103\\
297	0.000241064050983103\\
298	0.000241064050983103\\
299	0.000241064050983103\\
300	0.000241064050983103\\
301	0.000241064050983103\\
302	0.000241064050983103\\
303	0.000241064050983103\\
304	0.000241064050983103\\
305	0.000241064050983103\\
306	0.000241064050983103\\
307	0.000241064050983103\\
308	0.000241064050983103\\
309	0.000241064050983103\\
310	0.000241064050983103\\
311	0.000241064050983103\\
312	0.000241064050983103\\
313	0.000241064050983103\\
314	0.000241064050983103\\
315	0.000241064050983103\\
316	0.000241064050983103\\
317	0.000241064050983103\\
318	0.000241064050983103\\
319	0.000241064050983103\\
320	0.000241064050983103\\
321	0.000241064050983103\\
322	0.000241064050983103\\
323	0.000241064050983103\\
324	0.000241064050983103\\
325	0.000241064050983103\\
326	0.000241064050983103\\
327	0.000241064050983103\\
328	0.000241064050983103\\
329	0.000241064050983103\\
330	0.000241064050983103\\
331	0.000241064050983103\\
332	0.000241064050983103\\
333	0.000241064050983103\\
334	0.000241064050983103\\
335	0.000241064050983103\\
336	0.000241064050983103\\
337	0.000241064050983103\\
338	0.000241064050983103\\
339	0.000241064050983103\\
340	0.000241064050983103\\
341	0.000241064050983103\\
342	0.000241064050983103\\
343	0.000241064050983103\\
344	0.000241064050983103\\
345	0.000241064050983103\\
346	0.000241064050983103\\
347	0.000241064050983103\\
348	0.000241064050983103\\
349	0.000241064050983103\\
350	0.000241064050983103\\
351	0.000241064050983103\\
352	0.000241064050983103\\
353	0.000241064050983103\\
354	0.000241064050983103\\
355	0.000241064050983103\\
356	0.000241064050983103\\
357	0.000241064050983103\\
358	0.000241064050983103\\
359	0.000241064050983103\\
360	0.000241064050983103\\
361	0.000241064050983103\\
362	0.000241064050983103\\
363	0.000241064050983103\\
364	0.000241064050983103\\
365	0.000241064050983103\\
366	0.000241064050983103\\
367	0.000241064050983103\\
368	0.000241064050983103\\
369	0.000241064050983103\\
370	0.000241064050983103\\
371	0.000241064050983103\\
372	0.000241064050983103\\
373	0.000241064050983103\\
374	0.000241064050983103\\
375	0.000241064050983103\\
376	0.000241064050983103\\
377	0.000241064050983103\\
378	0.000241064050983103\\
379	0.000241064050983103\\
380	0.000241064050983103\\
381	0.000241064050983103\\
382	0.000241064050983103\\
383	0.000241064050983103\\
384	0.000241064050983103\\
385	0.000241064050983103\\
386	0.000241064050983103\\
387	0.000241064050983103\\
388	0.000241064050983103\\
389	0.000241064050983103\\
390	0.000241064050983103\\
391	0.000241064050983103\\
392	0.000241064050983103\\
393	0.000241064050983103\\
394	0.000241064050983103\\
395	0.000241064050983103\\
396	0.000241064050983103\\
397	0.000241064050983103\\
398	0.000241064050983103\\
399	0.000241064050983103\\
400	0.000241064050983103\\
401	0.000241064050983103\\
402	0.000241064050983103\\
403	0.000241064050983103\\
404	0.000241064050983103\\
405	0.000241064050983103\\
406	0.000241064050983103\\
407	0.000241064050983103\\
408	0.000241064050983103\\
409	0.000241064050983103\\
410	0.000241064050983103\\
411	0.000241064050983103\\
412	0.000241064050983103\\
413	0.000241064050983103\\
414	0.000241064050983103\\
415	0.000241064050983103\\
416	0.000241064050983103\\
417	0.000241064050983103\\
418	0.000241064050983103\\
419	0.000241064050983103\\
420	0.000241064050983103\\
421	0.000241064050983103\\
422	0.000241064050983103\\
423	0.000241064050983103\\
424	0.000241064050983103\\
425	0.000241064050983103\\
426	0.000241064050983103\\
427	0.000241064050983103\\
428	0.000241064050983103\\
429	0.000241064050983103\\
430	0.000241064050983103\\
431	0.000241064050983103\\
432	0.000241064050983103\\
433	0.000241064050983103\\
434	0.000241064050983103\\
435	0.000241064050983103\\
436	0.000241064050983103\\
437	0.000241064050983103\\
438	0.000241064050983103\\
439	0.000241064050983103\\
440	0.000241064050983103\\
441	0.000241064050983103\\
442	0.000241064050983103\\
443	0.000241064050983103\\
444	0.000241064050983103\\
445	0.000241064050983103\\
446	0.000241064050983103\\
447	0.000241064050983103\\
448	0.000241064050983103\\
449	0.000241064050983103\\
450	0.000241064050983103\\
451	0.000241064050983103\\
452	0.000241064050983103\\
453	0.000241064050983103\\
454	0.000241064050983103\\
455	0.000241064050983103\\
456	0.000241064050983103\\
457	0.000241064050983103\\
458	0.000241064050983103\\
459	0.000241064050983103\\
460	0.000241064050983103\\
461	0.000241064050983103\\
462	0.000241064050983103\\
463	0.000241064050983103\\
464	0.000241064050983103\\
465	0.000241064050983103\\
466	0.000241064050983103\\
467	0.000241064050983103\\
468	0.000241064050983103\\
469	0.000241064050983103\\
470	0.000241064050983103\\
471	0.000241064050983103\\
472	0.000241064050983103\\
473	0.000241064050983103\\
474	0.000241064050983103\\
475	0.000241064050983103\\
476	0.000241064050983103\\
477	0.000241064050983103\\
478	0.000241064050983103\\
479	0.000241064050983103\\
480	0.000241064050983103\\
481	0.000241064050983103\\
482	0.000241064050983103\\
483	0.000241064050983103\\
484	0.000241064050983103\\
485	0.000241064050983103\\
486	0.000241064050983103\\
487	0.000241064050983103\\
488	0.000241064050983103\\
489	0.000241064050983103\\
490	0.000241064050983103\\
491	0.000241064050983103\\
492	0.000241064050983103\\
493	0.000241064050983103\\
494	0.000241064050983103\\
495	0.000241064050983103\\
496	0.000241064050983103\\
497	0.000241064050983103\\
498	0.000241064050983103\\
499	0.000241064050983103\\
500	0.000241064050983103\\
501	0.000241064050983103\\
502	0.000241064050983103\\
503	0.000241064050983103\\
504	0.000241064050983103\\
505	0.000241064050983103\\
506	0.000241064050983103\\
507	0.000241064050983103\\
508	0.000241064050983103\\
509	0.000241064050983103\\
510	0.000241064050983103\\
511	0.000241064050983103\\
512	0.000241064050983103\\
513	0.000241064050983103\\
514	0.000241064050983103\\
515	0.000241064050983103\\
516	0.000241064050983103\\
517	0.000241064050983103\\
518	0.000241064050983103\\
519	0.000241064050983103\\
520	0.000241064050983103\\
521	0.000241064050983103\\
522	0.000241064050983103\\
523	0.000241064050983103\\
524	0.000241064050983103\\
525	0.000241064050983103\\
526	0.000241064050983103\\
527	0.000241064050983103\\
528	0.000241064050983103\\
529	0.000241064050983103\\
530	0.000241064050983103\\
531	0.000241064050983103\\
532	0.000241064050983103\\
533	0.000241064050983103\\
534	0.000241064050983103\\
535	0.000241064050983103\\
536	0.000241064050983103\\
537	0.000241064050983103\\
538	0.000241064050983103\\
539	0.000241064050983103\\
540	0.000241064050983103\\
541	0.000241064050983103\\
542	0.000241064050983103\\
543	0.000241064050983103\\
544	0.000241064050983103\\
545	0.000241064050983103\\
546	0.000241064050983103\\
547	0.000241064050983103\\
548	0.000241064050983103\\
549	0.000241064050983103\\
550	0.000241064050983103\\
551	0.000241064050983103\\
552	0.000241064050983103\\
553	0.000241064050983103\\
554	0.000241064050983103\\
555	0.000241064050983103\\
556	0.000241064050983103\\
557	0.000241064050983103\\
558	0.000241064050983103\\
559	0.000241064050983103\\
560	0.000241064050983103\\
561	0.000241064050983103\\
562	0.000241064050983103\\
563	0.000241064050983103\\
564	0.000241064050983103\\
565	0.000241749063135778\\
566	0.000253463059233083\\
567	0.000265313220815038\\
568	0.000410305514745056\\
569	0.000594445605831459\\
570	0.000790504526718181\\
571	0.0010000730134328\\
572	0.00122502642682604\\
573	0.00146605907646788\\
574	0.0017151074599206\\
575	0.00197243134443579\\
576	0.00223836387852356\\
577	0.00251314268478639\\
578	0.00279686556808013\\
579	0.00308975724537851\\
580	0.0033913089943072\\
581	0.00370111136261089\\
582	0.00401734174175409\\
583	0.00433970225536034\\
584	0.00466828634959776\\
585	0.00500320818845317\\
586	0.00534458915269589\\
587	0.00569252505204087\\
588	0.0060470515584461\\
589	0.00640808252825402\\
590	0.00677541915308552\\
591	0.00714888978102035\\
592	0.00752838693633574\\
593	0.00791376136525452\\
594	0.00830476600218674\\
595	0.0087009814905431\\
596	0.00910162822194027\\
597	0.0095050880145213\\
598	0.0099080685081414\\
599	0\\
600	0\\
};
\addplot [color=mycolor14,solid,forget plot]
  table[row sep=crcr]{%
1	0\\
2	0\\
3	0\\
4	0\\
5	0\\
6	0\\
7	0\\
8	0\\
9	0\\
10	0\\
11	0\\
12	0\\
13	0\\
14	0\\
15	0\\
16	0\\
17	0\\
18	0\\
19	0\\
20	0\\
21	0\\
22	0\\
23	0\\
24	0\\
25	0\\
26	0\\
27	0\\
28	0\\
29	0\\
30	0\\
31	0\\
32	0\\
33	0\\
34	0\\
35	0\\
36	0\\
37	0\\
38	0\\
39	0\\
40	0\\
41	0\\
42	0\\
43	0\\
44	0\\
45	0\\
46	0\\
47	0\\
48	0\\
49	0\\
50	0\\
51	0\\
52	0\\
53	0\\
54	0\\
55	0\\
56	0\\
57	0\\
58	0\\
59	0\\
60	0\\
61	0\\
62	0\\
63	0\\
64	0\\
65	0\\
66	0\\
67	0\\
68	0\\
69	0\\
70	0\\
71	0\\
72	0\\
73	0\\
74	0\\
75	0\\
76	0\\
77	0\\
78	0\\
79	0\\
80	0\\
81	0\\
82	0\\
83	0\\
84	0\\
85	0\\
86	0\\
87	0\\
88	0\\
89	0\\
90	0\\
91	0\\
92	0\\
93	0\\
94	0\\
95	0\\
96	0\\
97	0\\
98	0\\
99	0\\
100	0\\
101	0\\
102	0\\
103	0\\
104	0\\
105	0\\
106	0\\
107	0\\
108	0\\
109	0\\
110	0\\
111	0\\
112	0\\
113	0\\
114	0\\
115	0\\
116	0\\
117	0\\
118	0\\
119	0\\
120	0\\
121	0\\
122	0\\
123	0\\
124	0\\
125	0\\
126	0\\
127	0\\
128	0\\
129	0\\
130	0\\
131	0\\
132	0\\
133	0\\
134	0\\
135	0\\
136	0\\
137	0\\
138	0\\
139	0\\
140	0\\
141	0\\
142	0\\
143	0\\
144	0\\
145	0\\
146	0\\
147	0\\
148	0\\
149	0\\
150	0\\
151	0\\
152	0\\
153	0\\
154	0\\
155	0\\
156	0\\
157	0\\
158	0\\
159	0\\
160	0\\
161	0\\
162	0\\
163	0\\
164	0\\
165	0\\
166	0\\
167	0\\
168	0\\
169	0\\
170	0\\
171	0\\
172	0\\
173	0\\
174	0\\
175	0\\
176	0\\
177	0\\
178	0\\
179	0\\
180	0\\
181	0\\
182	0\\
183	0\\
184	0\\
185	0\\
186	0\\
187	0\\
188	0\\
189	0\\
190	0\\
191	0\\
192	0\\
193	0\\
194	0\\
195	0\\
196	0\\
197	0\\
198	0\\
199	0\\
200	0\\
201	0\\
202	0\\
203	0\\
204	0\\
205	0\\
206	0\\
207	0\\
208	0\\
209	0\\
210	0\\
211	0\\
212	0\\
213	0\\
214	0\\
215	0\\
216	0\\
217	0\\
218	0\\
219	0\\
220	0\\
221	0\\
222	0\\
223	0\\
224	0\\
225	0\\
226	0\\
227	0\\
228	0\\
229	0\\
230	0\\
231	0\\
232	0\\
233	0\\
234	0\\
235	0\\
236	0\\
237	0\\
238	0\\
239	0\\
240	0\\
241	0\\
242	0\\
243	0\\
244	0\\
245	0\\
246	0\\
247	0\\
248	0\\
249	0\\
250	0\\
251	0\\
252	0\\
253	0\\
254	0\\
255	0\\
256	0\\
257	0\\
258	0\\
259	0\\
260	0\\
261	0\\
262	0\\
263	0\\
264	0\\
265	0\\
266	0\\
267	0\\
268	0\\
269	0\\
270	0\\
271	0\\
272	0\\
273	0\\
274	0\\
275	0\\
276	0\\
277	0\\
278	0\\
279	0\\
280	0\\
281	0\\
282	0\\
283	0\\
284	0\\
285	0\\
286	0\\
287	0\\
288	0\\
289	0\\
290	0\\
291	0\\
292	0\\
293	0\\
294	0\\
295	0\\
296	0\\
297	0\\
298	0\\
299	0\\
300	0\\
301	0\\
302	0\\
303	0\\
304	0\\
305	0\\
306	0\\
307	0\\
308	0\\
309	0\\
310	0\\
311	0\\
312	0\\
313	0\\
314	0\\
315	0\\
316	0\\
317	0\\
318	0\\
319	0\\
320	0\\
321	0\\
322	0\\
323	0\\
324	0\\
325	0\\
326	0\\
327	0\\
328	0\\
329	0\\
330	0\\
331	0\\
332	0\\
333	0\\
334	5.34395650947204e-10\\
335	1.15405752732983e-09\\
336	1.78490839335517e-09\\
337	2.42718702688798e-09\\
338	3.08097622702292e-09\\
339	3.7462704053938e-09\\
340	4.42325923253163e-09\\
341	5.11214007990728e-09\\
342	5.81312728863281e-09\\
343	6.52647217814512e-09\\
344	7.25250069845883e-09\\
345	7.99166527067748e-09\\
346	8.74454929752424e-09\\
347	9.51159970801252e-09\\
348	1.02923500181668e-08\\
349	1.10862998942372e-08\\
350	1.18936706872483e-08\\
351	1.27146872222528e-08\\
352	1.35495778275436e-08\\
353	1.43985743639362e-08\\
354	1.52619122280985e-08\\
355	1.61398303870155e-08\\
356	1.70325714449766e-08\\
357	1.7940381750214e-08\\
358	1.88635113449878e-08\\
359	1.98022140017196e-08\\
360	2.07567472539151e-08\\
361	2.17273724305487e-08\\
362	2.27143546956379e-08\\
363	2.37179630756574e-08\\
364	2.4738470512949e-08\\
365	2.5776153896551e-08\\
366	2.68312941138062e-08\\
367	2.79041761053957e-08\\
368	2.8995088923791e-08\\
369	3.01043257864434e-08\\
370	3.12321841618619e-08\\
371	3.23789658417251e-08\\
372	3.35449770371859e-08\\
373	3.4730528463067e-08\\
374	3.59359353516285e-08\\
375	3.71615171534724e-08\\
376	3.84075963875673e-08\\
377	3.96744958152216e-08\\
378	4.09625329188538e-08\\
379	4.22720072520436e-08\\
380	4.36031754664621e-08\\
381	4.4956208310274e-08\\
382	4.63311348769375e-08\\
383	4.77278246214462e-08\\
384	4.91461651080398e-08\\
385	5.05866681360059e-08\\
386	5.20510761902004e-08\\
387	5.35401061011209e-08\\
388	5.50541788448286e-08\\
389	5.65937301735987e-08\\
390	5.81592121148706e-08\\
391	5.97510944375333e-08\\
392	6.13698657709378e-08\\
393	6.30160336134883e-08\\
394	6.46901213915346e-08\\
395	6.63926583226211e-08\\
396	6.81241528093417e-08\\
397	6.98850304655078e-08\\
398	7.1675503009099e-08\\
399	7.34953235246137e-08\\
400	7.53434266942629e-08\\
401	7.72177101885758e-08\\
402	7.91160011730615e-08\\
403	8.10402617750092e-08\\
404	8.30013124105549e-08\\
405	8.49998533169951e-08\\
406	8.70366282523623e-08\\
407	8.91124417830267e-08\\
408	9.12281765834018e-08\\
409	9.33847579100734e-08\\
410	9.55829274308186e-08\\
411	9.78228790172858e-08\\
412	1.00105585709825e-07\\
413	1.02432054835268e-07\\
414	1.04803329834153e-07\\
415	1.07220492206593e-07\\
416	1.09684663600859e-07\\
417	1.12197008109884e-07\\
418	1.14758734959682e-07\\
419	1.17371101876897e-07\\
420	1.20035419157529e-07\\
421	1.22753053347217e-07\\
422	1.25525427992502e-07\\
423	1.2835402409624e-07\\
424	1.31240388033559e-07\\
425	1.34186136086394e-07\\
426	1.37192959643411e-07\\
427	1.40262631489629e-07\\
428	1.43397014318821e-07\\
429	1.46598074148246e-07\\
430	1.49867905271133e-07\\
431	1.5320878232588e-07\\
432	1.56623274031272e-07\\
433	1.60114486841128e-07\\
434	1.63686540787919e-07\\
435	1.67345301663662e-07\\
436	1.71098764635964e-07\\
437	1.74954381160609e-07\\
438	1.78909154900545e-07\\
439	1.8295637287902e-07\\
440	1.87099605226687e-07\\
441	1.91342669503603e-07\\
442	1.95689656232806e-07\\
443	2.00144955729809e-07\\
444	2.04713282322897e-07\\
445	2.0939968622252e-07\\
446	2.14209534980528e-07\\
447	2.1914844253657e-07\\
448	2.24222122469994e-07\\
449	2.294359902385e-07\\
450	2.34794627981946e-07\\
451	2.40301934758161e-07\\
452	2.45965557771628e-07\\
453	2.5181662750504e-07\\
454	2.57965005822826e-07\\
455	2.64673797422747e-07\\
456	2.72410190911762e-07\\
457	2.81919231545979e-07\\
458	2.95772609912789e-07\\
459	3.10830772923359e-07\\
460	3.26185236841913e-07\\
461	3.41872439971896e-07\\
462	3.57922725856886e-07\\
463	3.74252066657456e-07\\
464	3.90866982913329e-07\\
465	4.07778149527845e-07\\
466	4.24997868336965e-07\\
467	4.42540711323325e-07\\
468	4.60423884675653e-07\\
469	4.78664199523387e-07\\
470	4.97262783721133e-07\\
471	5.16231892943477e-07\\
472	5.35584126137011e-07\\
473	5.55321885639264e-07\\
474	5.75426806261948e-07\\
475	5.95852950483929e-07\\
476	6.16567409574145e-07\\
477	6.3774354115478e-07\\
478	6.59575471449224e-07\\
479	6.82115872210608e-07\\
480	7.05438046054695e-07\\
481	7.29663740069737e-07\\
482	7.55037848234536e-07\\
483	7.82129100839374e-07\\
484	8.12347945901534e-07\\
485	8.48979433404414e-07\\
486	8.9425410391506e-07\\
487	9.40380960404758e-07\\
488	9.87385880167998e-07\\
489	1.03529459935483e-06\\
490	1.08413279222023e-06\\
491	1.13392403680387e-06\\
492	1.18468621498411e-06\\
493	1.24720906048516e-06\\
494	1.37006706325445e-06\\
495	1.49530305006788e-06\\
496	1.62279503189096e-06\\
497	1.75276173524047e-06\\
498	1.88607654705186e-06\\
499	2.02293411752196e-06\\
500	2.16357235569838e-06\\
501	2.30830980253248e-06\\
502	2.45762566848134e-06\\
503	2.6123006020255e-06\\
504	2.77347904379284e-06\\
505	2.94138237846309e-06\\
506	3.11739139183146e-06\\
507	3.30697665297456e-06\\
508	3.52424931125228e-06\\
509	3.76136854442694e-06\\
510	4.00167482907179e-06\\
511	4.24520184548467e-06\\
512	4.49197310366559e-06\\
513	4.74194860801026e-06\\
514	4.99356484026411e-06\\
515	5.24824522229138e-06\\
516	5.50687831805842e-06\\
517	5.76960685637947e-06\\
518	6.03654845807816e-06\\
519	6.30776522352361e-06\\
520	6.58327641762038e-06\\
521	6.87117288010451e-06\\
522	7.1678050440652e-06\\
523	7.47354024641004e-06\\
524	7.78880422125584e-06\\
525	8.12047792029648e-06\\
526	8.4942466400466e-06\\
527	9.78409477878136e-06\\
528	1.12481044670916e-05\\
529	1.27785154645931e-05\\
530	1.43861785462902e-05\\
531	1.60869638998932e-05\\
532	1.79245359223213e-05\\
533	2.34482844773654e-05\\
534	3.20899932668833e-05\\
535	4.10594471663407e-05\\
536	5.03858487580915e-05\\
537	6.01048232126996e-05\\
538	7.02571673572263e-05\\
539	8.08905515896843e-05\\
540	9.20670474081498e-05\\
541	0.00010622148232629\\
542	0.000237379372258858\\
543	0.000373368428114856\\
544	0.000514625762215635\\
545	0.000661651087225234\\
546	0.000815016828351423\\
547	0.000975380642210555\\
548	0.00114350123968401\\
549	0.00132025756814093\\
550	0.00150667201706074\\
551	0.00170393802307672\\
552	0.00191345060473266\\
553	0.00213682969392011\\
554	0.00236798736638595\\
555	0.00260563925080853\\
556	0.00285012933485717\\
557	0.00310182473985042\\
558	0.00336110730877499\\
559	0.00362843493984692\\
560	0.00390430453510286\\
561	0.00418922119559004\\
562	0.0044836504231414\\
563	0.00478785164046071\\
564	0.00510231617182818\\
565	0.00542675629214628\\
566	0.00575057379768084\\
567	0.00608512390105982\\
568	0.00628707380955198\\
569	0.00645757355205645\\
570	0.00662281053244697\\
571	0.00678105595816712\\
572	0.00693022715913866\\
573	0.00706963889082102\\
574	0.0072085378349189\\
575	0.00734681055555072\\
576	0.00748417292309207\\
577	0.00762034995855645\\
578	0.00775509740275521\\
579	0.00788823117439556\\
580	0.00801957808260108\\
581	0.00814908754750536\\
582	0.0082784986130116\\
583	0.00840796926358845\\
584	0.00853718929971185\\
585	0.00866579861787599\\
586	0.00879338692183546\\
587	0.00891946099316878\\
588	0.00904345529322948\\
589	0.00916475724618028\\
590	0.00928271236240283\\
591	0.00939663262290929\\
592	0.00950580711476238\\
593	0.009609516829347\\
594	0.00970705750680575\\
595	0.00979777178715996\\
596	0.00988114378468959\\
597	0.00995689746872328\\
598	0.00999970795535495\\
599	0\\
600	0\\
};
\addplot [color=mycolor15,solid,forget plot]
  table[row sep=crcr]{%
1	2.97404688413887e-05\\
2	2.97404862481491e-05\\
3	2.97405039699636e-05\\
4	2.97405220125235e-05\\
5	2.97405403816232e-05\\
6	2.9740559083164e-05\\
7	2.97405781231503e-05\\
8	2.9740597507695e-05\\
9	2.97406172430227e-05\\
10	2.97406373354716e-05\\
11	2.974065779149e-05\\
12	2.97406786176433e-05\\
13	2.97406998206171e-05\\
14	2.97407214072178e-05\\
15	2.974074338437e-05\\
16	2.97407657591278e-05\\
17	2.97407885386708e-05\\
18	2.97408117303039e-05\\
19	2.97408353414699e-05\\
20	2.97408593797424e-05\\
21	2.9740883852829e-05\\
22	2.97409087685803e-05\\
23	2.97409341349844e-05\\
24	2.97409599601773e-05\\
25	2.97409862524363e-05\\
26	2.97410130201917e-05\\
27	2.97410402720233e-05\\
28	2.97410680166644e-05\\
29	2.9741096263008e-05\\
30	2.97411250201055e-05\\
31	2.97411542971701e-05\\
32	2.97411841035834e-05\\
33	2.97412144488905e-05\\
34	2.9741245342814e-05\\
35	2.97412767952481e-05\\
36	2.97413088162646e-05\\
37	2.97413414161156e-05\\
38	2.97413746052409e-05\\
39	2.97414083942646e-05\\
40	2.97414427940014e-05\\
41	2.97414778154606e-05\\
42	2.97415134698528e-05\\
43	2.97415497685847e-05\\
44	2.97415867232712e-05\\
45	2.97416243457336e-05\\
46	2.97416626480066e-05\\
47	2.97417016423399e-05\\
48	2.97417413412035e-05\\
49	2.97417817572927e-05\\
50	2.974182290353e-05\\
51	2.97418647930666e-05\\
52	2.97419074392912e-05\\
53	2.97419508558368e-05\\
54	2.97419950565738e-05\\
55	2.9742040055629e-05\\
56	2.97420858673734e-05\\
57	2.97421325064466e-05\\
58	2.97421799877391e-05\\
59	2.9742228326417e-05\\
60	2.97422775379144e-05\\
61	2.97423276379445e-05\\
62	2.97423786424973e-05\\
63	2.97424305678536e-05\\
64	2.97424834305866e-05\\
65	2.97425372475604e-05\\
66	2.9742592035947e-05\\
67	2.97426478132246e-05\\
68	2.97427045971846e-05\\
69	2.97427624059331e-05\\
70	2.97428212579068e-05\\
71	2.97428811718655e-05\\
72	2.97429421669117e-05\\
73	2.9743004262485e-05\\
74	2.97430674783741e-05\\
75	2.97431318347205e-05\\
76	2.97431973520306e-05\\
77	2.97432640511736e-05\\
78	2.97433319533922e-05\\
79	2.97434010803108e-05\\
80	2.97434714539377e-05\\
81	2.97435430966787e-05\\
82	2.97436160313366e-05\\
83	2.97436902811259e-05\\
84	2.97437658696703e-05\\
85	2.97438428210202e-05\\
86	2.97439211596561e-05\\
87	2.97440009104921e-05\\
88	2.97440820988914e-05\\
89	2.97441647506696e-05\\
90	2.97442488921e-05\\
91	2.97443345499306e-05\\
92	2.97444217513827e-05\\
93	2.97445105241662e-05\\
94	2.97446008964847e-05\\
95	2.97446928970458e-05\\
96	2.97447865550702e-05\\
97	2.97448819002994e-05\\
98	2.97449789630034e-05\\
99	2.97450777739974e-05\\
100	2.9745178364644e-05\\
101	2.97452807668645e-05\\
102	2.97453850131519e-05\\
103	2.97454911365751e-05\\
104	2.9745599170797e-05\\
105	2.97457091500787e-05\\
106	2.97458211092926e-05\\
107	2.97459350839336e-05\\
108	2.97460511101297e-05\\
109	2.97461692246509e-05\\
110	2.97462894649258e-05\\
111	2.97464118690457e-05\\
112	2.97465364757846e-05\\
113	2.97466633246085e-05\\
114	2.97467924556816e-05\\
115	2.97469239098874e-05\\
116	2.97470577288371e-05\\
117	2.97471939548834e-05\\
118	2.97473326311327e-05\\
119	2.97474738014603e-05\\
120	2.9747617510521e-05\\
121	2.97477638037643e-05\\
122	2.97479127274519e-05\\
123	2.97480643286677e-05\\
124	2.97482186553321e-05\\
125	2.97483757562218e-05\\
126	2.97485356809794e-05\\
127	2.97486984801297e-05\\
128	2.97488642050993e-05\\
129	2.97490329082298e-05\\
130	2.97492046427922e-05\\
131	2.97493794630035e-05\\
132	2.97495574240511e-05\\
133	2.97497385820997e-05\\
134	2.97499229943135e-05\\
135	2.97501107188717e-05\\
136	2.9750301814993e-05\\
137	2.97504963429452e-05\\
138	2.97506943640664e-05\\
139	2.97508959407906e-05\\
140	2.97511011366547e-05\\
141	2.97513100163311e-05\\
142	2.97515226456378e-05\\
143	2.97517390915647e-05\\
144	2.97519594222922e-05\\
145	2.97521837072102e-05\\
146	2.97524120169437e-05\\
147	2.97526444233722e-05\\
148	2.97528809996517e-05\\
149	2.97531218202386e-05\\
150	2.97533669609094e-05\\
151	2.97536164987907e-05\\
152	2.97538705123772e-05\\
153	2.97541290815569e-05\\
154	2.97543922876374e-05\\
155	2.97546602133715e-05\\
156	2.97549329429775e-05\\
157	2.97552105621761e-05\\
158	2.97554931582052e-05\\
159	2.97557808198495e-05\\
160	2.97560736374782e-05\\
161	2.9756371703057e-05\\
162	2.97566751101879e-05\\
163	2.9756983954135e-05\\
164	2.97572983318532e-05\\
165	2.97576183420199e-05\\
166	2.97579440850588e-05\\
167	2.97582756631843e-05\\
168	2.97586131804212e-05\\
169	2.97589567426381e-05\\
170	2.97593064575862e-05\\
171	2.97596624349314e-05\\
172	2.9760024786282e-05\\
173	2.97603936252319e-05\\
174	2.97607690673841e-05\\
175	2.97611512304013e-05\\
176	2.97615402340333e-05\\
177	2.97619362001507e-05\\
178	2.97623392527923e-05\\
179	2.9762749518199e-05\\
180	2.97631671248483e-05\\
181	2.9763592203504e-05\\
182	2.97640248872477e-05\\
183	2.97644653115247e-05\\
184	2.97649136141839e-05\\
185	2.97653699355256e-05\\
186	2.97658344183348e-05\\
187	2.97663072079409e-05\\
188	2.97667884522436e-05\\
189	2.97672783017785e-05\\
190	2.97677769097478e-05\\
191	2.97682844320806e-05\\
192	2.97688010274693e-05\\
193	2.97693268574293e-05\\
194	2.97698620863476e-05\\
195	2.97704068815274e-05\\
196	2.97709614132446e-05\\
197	2.97715258548053e-05\\
198	2.97721003825931e-05\\
199	2.97726851761247e-05\\
200	2.97732804181117e-05\\
201	2.9773886294514e-05\\
202	2.97745029945912e-05\\
203	2.97751307109786e-05\\
204	2.97757696397298e-05\\
205	2.97764199803913e-05\\
206	2.97770819360553e-05\\
207	2.97777557134302e-05\\
208	2.97784415229013e-05\\
209	2.97791395786011e-05\\
210	2.97798500984726e-05\\
211	2.97805733043387e-05\\
212	2.97813094219759e-05\\
213	2.97820586811778e-05\\
214	2.97828213158347e-05\\
215	2.97835975640036e-05\\
216	2.97843876679877e-05\\
217	2.97851918744051e-05\\
218	2.97860104342731e-05\\
219	2.97868436030874e-05\\
220	2.97876916408994e-05\\
221	2.97885548124024e-05\\
222	2.97894333870122e-05\\
223	2.97903276389551e-05\\
224	2.97912378473574e-05\\
225	2.97921642963259e-05\\
226	2.97931072750464e-05\\
227	2.97940670778709e-05\\
228	2.97950440044181e-05\\
229	2.97960383596589e-05\\
230	2.97970504540231e-05\\
231	2.97980806034926e-05\\
232	2.97991291297078e-05\\
233	2.9800196360064e-05\\
234	2.98012826278215e-05\\
235	2.98023882722125e-05\\
236	2.98035136385473e-05\\
237	2.98046590783267e-05\\
238	2.9805824949355e-05\\
239	2.98070116158573e-05\\
240	2.98082194485931e-05\\
241	2.98094488249783e-05\\
242	2.98107001292055e-05\\
243	2.98119737523718e-05\\
244	2.98132700926038e-05\\
245	2.98145895551836e-05\\
246	2.98159325526829e-05\\
247	2.9817299505097e-05\\
248	2.98186908399811e-05\\
249	2.98201069925909e-05\\
250	2.9821548406019e-05\\
251	2.98230155313476e-05\\
252	2.98245088277847e-05\\
253	2.98260287628275e-05\\
254	2.98275758124032e-05\\
255	2.9829150461033e-05\\
256	2.98307532019827e-05\\
257	2.98323845374322e-05\\
258	2.98340449786399e-05\\
259	2.98357350461015e-05\\
260	2.98374552697339e-05\\
261	2.98392061890384e-05\\
262	2.98409883532887e-05\\
263	2.98428023217061e-05\\
264	2.98446486636503e-05\\
265	2.98465279588142e-05\\
266	2.98484407974301e-05\\
267	2.98503877804866e-05\\
268	2.9852369519957e-05\\
269	2.98543866390391e-05\\
270	2.98564397723334e-05\\
271	2.9858529565897e-05\\
272	2.98606566771132e-05\\
273	2.98628217747086e-05\\
274	2.98650255396937e-05\\
275	2.98672686660919e-05\\
276	2.98695518605617e-05\\
277	2.9871875842632e-05\\
278	2.98742413449366e-05\\
279	2.98766491134666e-05\\
280	2.98790999078097e-05\\
281	2.98815945014001e-05\\
282	2.98841336817878e-05\\
283	2.98867182508817e-05\\
284	2.98893490252302e-05\\
285	2.98920268362769e-05\\
286	2.98947525306485e-05\\
287	2.98975269704211e-05\\
288	2.99003510334061e-05\\
289	2.99032256134422e-05\\
290	2.99061516206816e-05\\
291	2.99091299818855e-05\\
292	2.99121616407267e-05\\
293	2.99152475580896e-05\\
294	2.99183887123827e-05\\
295	2.99215860998452e-05\\
296	2.99248407348631e-05\\
297	2.99281536502918e-05\\
298	2.99315258977662e-05\\
299	2.99349585480358e-05\\
300	2.99384526912784e-05\\
301	2.99420094374227e-05\\
302	2.99456299164607e-05\\
303	2.9949315278739e-05\\
304	2.99530666952098e-05\\
305	2.99568853576494e-05\\
306	2.99607724787912e-05\\
307	2.99647292924973e-05\\
308	2.99687570541305e-05\\
309	2.99728570416238e-05\\
310	2.99770305575876e-05\\
311	2.99812789314266e-05\\
312	2.99856035174778e-05\\
313	2.99900056885582e-05\\
314	2.99944868414278e-05\\
315	2.99990483970627e-05\\
316	3.00036918009157e-05\\
317	3.00084185231493e-05\\
318	3.00132300588652e-05\\
319	3.00181279282848e-05\\
320	3.00231136769208e-05\\
321	3.00281888757031e-05\\
322	3.00333551210753e-05\\
323	3.00386140350458e-05\\
324	3.00439672651796e-05\\
325	3.00494164845499e-05\\
326	3.00549633916248e-05\\
327	3.0060609710099e-05\\
328	3.00663571886809e-05\\
329	3.00722076009072e-05\\
330	3.00781627451161e-05\\
331	3.00842244450142e-05\\
332	3.00903945518141e-05\\
333	3.00966749503805e-05\\
334	3.01030675748446e-05\\
335	3.0109574444204e-05\\
336	3.0116197730633e-05\\
337	3.01229398372392e-05\\
338	3.01298032596678e-05\\
339	3.013678958593e-05\\
340	3.01438994298522e-05\\
341	3.01511347989999e-05\\
342	3.01584977206469e-05\\
343	3.01659902579481e-05\\
344	3.01736145562055e-05\\
345	3.01813729659352e-05\\
346	3.01892683393417e-05\\
347	3.0197304563206e-05\\
348	3.02054864102969e-05\\
349	3.02138128579057e-05\\
350	3.02222814728764e-05\\
351	3.02308947007519e-05\\
352	3.02396550330103e-05\\
353	3.02485650094424e-05\\
354	3.02576272166048e-05\\
355	3.02668442751239e-05\\
356	3.02762188145479e-05\\
357	3.02857534755208e-05\\
358	3.02954509977257e-05\\
359	3.03053141661812e-05\\
360	3.03153458119551e-05\\
361	3.03255488128955e-05\\
362	3.03359260944162e-05\\
363	3.03464806303157e-05\\
364	3.03572154436405e-05\\
365	3.03681336075773e-05\\
366	3.03792382463267e-05\\
367	3.03905325358621e-05\\
368	3.04020197043202e-05\\
369	3.04137030316267e-05\\
370	3.04255858476218e-05\\
371	3.04376715279285e-05\\
372	3.04499634877806e-05\\
373	3.04624651781903e-05\\
374	3.04751800990019e-05\\
375	3.04881118558866e-05\\
376	3.0501264266483e-05\\
377	3.05146413700678e-05\\
378	3.05282470375935e-05\\
379	3.05420851967666e-05\\
380	3.05561598031099e-05\\
381	3.05704747643933e-05\\
382	3.0585033775017e-05\\
383	3.05998400019483e-05\\
384	3.06148956558714e-05\\
385	3.06302020512804e-05\\
386	3.0645762569589e-05\\
387	3.06615913132755e-05\\
388	3.06776955832804e-05\\
389	3.06940806272026e-05\\
390	3.07107518965827e-05\\
391	3.07277150663531e-05\\
392	3.07449760558192e-05\\
393	3.07625410507499e-05\\
394	3.07804165253829e-05\\
395	3.0798609261259e-05\\
396	3.08171263552236e-05\\
397	3.08359751975293e-05\\
398	3.08551633733923e-05\\
399	3.08746983763119e-05\\
400	3.08945868818401e-05\\
401	3.09148331006111e-05\\
402	3.0935435757847e-05\\
403	3.09563860515573e-05\\
404	3.09776846867058e-05\\
405	3.09994090938901e-05\\
406	3.10215667480998e-05\\
407	3.10441679631196e-05\\
408	3.10672252442965e-05\\
409	3.10907548151211e-05\\
410	3.11147761810167e-05\\
411	3.11392994784912e-05\\
412	3.11642925247853e-05\\
413	3.11897662786693e-05\\
414	3.12157320612875e-05\\
415	3.12422015485712e-05\\
416	3.12691867359848e-05\\
417	3.1296699857704e-05\\
418	3.13247532557774e-05\\
419	3.13533592720863e-05\\
420	3.13825304572305e-05\\
421	3.14122807053136e-05\\
422	3.14426273511496e-05\\
423	3.14735901304213e-05\\
424	3.15051857807041e-05\\
425	3.15374317852405e-05\\
426	3.1570346425234e-05\\
427	3.16039488375239e-05\\
428	3.16382590789578e-05\\
429	3.16732981999255e-05\\
430	3.17090883325516e-05\\
431	3.17456528069055e-05\\
432	3.17830163293211e-05\\
433	3.18212053122779e-05\\
434	3.18602485916399e-05\\
435	3.19001791325837e-05\\
436	3.19410380561905e-05\\
437	3.19828824300761e-05\\
438	3.20257874025019e-05\\
439	3.20697658936037e-05\\
440	3.21147872590803e-05\\
441	3.21608935475674e-05\\
442	3.22081304390101e-05\\
443	3.2256548233606e-05\\
444	3.23062034443293e-05\\
445	3.23571608694045e-05\\
446	3.2409494039847e-05\\
447	3.24632763419494e-05\\
448	3.25185558668801e-05\\
449	3.25754032640866e-05\\
450	3.26339064499488e-05\\
451	3.2694162978103e-05\\
452	3.27562886805595e-05\\
453	3.28204574425456e-05\\
454	3.28870757674982e-05\\
455	3.295747725203e-05\\
456	3.30356563701028e-05\\
457	3.31302861934239e-05\\
458	3.32312889433352e-05\\
459	3.35644576772812e-05\\
460	3.3961516616606e-05\\
461	3.43665591525194e-05\\
462	3.47799253877007e-05\\
463	3.52021194582698e-05\\
464	3.56328498599616e-05\\
465	3.60724779356961e-05\\
466	3.65214328788453e-05\\
467	3.69802077471269e-05\\
468	3.74493896981799e-05\\
469	3.79297032815118e-05\\
470	3.84219729875345e-05\\
471	3.89262553057394e-05\\
472	3.94431589242761e-05\\
473	3.99735557862203e-05\\
474	4.05184070659492e-05\\
475	4.10787506897888e-05\\
476	4.16556414179785e-05\\
477	4.22500802699654e-05\\
478	4.28644809064331e-05\\
479	4.35015912728323e-05\\
480	4.4162779676883e-05\\
481	4.48497831672469e-05\\
482	4.55647937682293e-05\\
483	4.63105714109174e-05\\
484	4.70913852015442e-05\\
485	4.79174489215605e-05\\
486	4.94466246545089e-05\\
487	5.32610171649985e-05\\
488	5.71682557891886e-05\\
489	6.11734633723027e-05\\
490	6.52809737951096e-05\\
491	6.94956323432319e-05\\
492	7.38234983612785e-05\\
493	7.82615429808505e-05\\
494	8.27641291275647e-05\\
495	8.73950339060839e-05\\
496	9.21611414068531e-05\\
497	9.70695443509984e-05\\
498	0.000102129174520471\\
499	0.000107354891571254\\
500	0.000112757932422414\\
501	0.000118350994752327\\
502	0.00012414856212374\\
503	0.000130167454479554\\
504	0.000136428089358144\\
505	0.000142958330389988\\
506	0.00014977873458778\\
507	0.000156913664545248\\
508	0.000164400529141022\\
509	0.000208637551692032\\
510	0.000289365828409982\\
511	0.000372128538179114\\
512	0.000457035657703233\\
513	0.000544208739511396\\
514	0.000633784648101589\\
515	0.000725912880920369\\
516	0.000820776210335949\\
517	0.000918566704049611\\
518	0.00101949032647088\\
519	0.00112377558945576\\
520	0.00123167657983085\\
521	0.00134346807944416\\
522	0.00145947086589884\\
523	0.00158004110496509\\
524	0.0017055721376809\\
525	0.00183648333108853\\
526	0.00197325147179061\\
527	0.0021155721982587\\
528	0.00226487144985893\\
529	0.00242206314622936\\
530	0.00258807701420806\\
531	0.00276399348220323\\
532	0.0029492594674755\\
533	0.00313575936505194\\
534	0.00332415734137589\\
535	0.00351740715760901\\
536	0.00371573957706861\\
537	0.00391940576669202\\
538	0.00412868429533327\\
539	0.00434388767246155\\
540	0.00456536780885139\\
541	0.00479115779066909\\
542	0.00490665714545878\\
543	0.00502388027007063\\
544	0.00514281816252538\\
545	0.00526324563336852\\
546	0.00538489858802557\\
547	0.00550743548486917\\
548	0.00563040291106781\\
549	0.00575320531854967\\
550	0.00587506655315862\\
551	0.0059949809594261\\
552	0.00611165189103131\\
553	0.00622341901143749\\
554	0.00633639651244715\\
555	0.00645190759163161\\
556	0.00656953595970948\\
557	0.00668824396759888\\
558	0.00680607065679266\\
559	0.0069226896100025\\
560	0.0070377375858648\\
561	0.00715081464281374\\
562	0.0072614836895415\\
563	0.00736927191235638\\
564	0.00747368111557365\\
565	0.0075741982814712\\
566	0.00767034259470333\\
567	0.0077615946326214\\
568	0.00784765855962678\\
569	0.00793115728065736\\
570	0.00801393498895354\\
571	0.00809600623835434\\
572	0.00817749098008832\\
573	0.0082586155742962\\
574	0.00833945963331221\\
575	0.00841996297313883\\
576	0.00850007764627636\\
577	0.0085797729284561\\
578	0.00865903629153023\\
579	0.00873787217500842\\
580	0.00881629777701114\\
581	0.00889433337966777\\
582	0.00897189890365491\\
583	0.0090488514123376\\
584	0.00912503765204509\\
585	0.00920030513112112\\
586	0.00927450589979287\\
587	0.00934750117970699\\
588	0.00941916757473226\\
589	0.0094894043972367\\
590	0.00955815482124445\\
591	0.00962541204610133\\
592	0.00969122229181295\\
593	0.00975569306111539\\
594	0.00981900263092351\\
595	0.0098813590240843\\
596	0.00993714890633491\\
597	0.00997691198549166\\
598	0.00999970795535495\\
599	0\\
600	0\\
};
\addplot [color=mycolor16,solid,forget plot]
  table[row sep=crcr]{%
1	5.65686140171787e-05\\
2	5.65692926888515e-05\\
3	5.65699836428933e-05\\
4	5.6570687101196e-05\\
5	5.65714032896419e-05\\
6	5.65721324381866e-05\\
7	5.65728747809237e-05\\
8	5.6573630556163e-05\\
9	5.65744000065106e-05\\
10	5.65751833789346e-05\\
11	5.65759809248528e-05\\
12	5.6576792900208e-05\\
13	5.65776195655527e-05\\
14	5.6578461186128e-05\\
15	5.65793180319476e-05\\
16	5.65801903778843e-05\\
17	5.65810785037623e-05\\
18	5.65819826944348e-05\\
19	5.6582903239884e-05\\
20	5.65838404353065e-05\\
21	5.65847945812099e-05\\
22	5.65857659835094e-05\\
23	5.65867549536181e-05\\
24	5.65877618085544e-05\\
25	5.6588786871031e-05\\
26	5.65898304695701e-05\\
27	5.65908929385982e-05\\
28	5.65919746185507e-05\\
29	5.65930758559855e-05\\
30	5.65941970036895e-05\\
31	5.6595338420789e-05\\
32	5.65965004728662e-05\\
33	5.6597683532071e-05\\
34	5.65988879772381e-05\\
35	5.66001141940107e-05\\
36	5.66013625749592e-05\\
37	5.66026335197047e-05\\
38	5.66039274350468e-05\\
39	5.66052447350884e-05\\
40	5.66065858413722e-05\\
41	5.66079511830075e-05\\
42	5.6609341196806e-05\\
43	5.66107563274232e-05\\
44	5.6612197027499e-05\\
45	5.66136637577918e-05\\
46	5.66151569873301e-05\\
47	5.66166771935637e-05\\
48	5.66182248625042e-05\\
49	5.66198004888857e-05\\
50	5.66214045763156e-05\\
51	5.66230376374363e-05\\
52	5.66247001940868e-05\\
53	5.66263927774575e-05\\
54	5.66281159282758e-05\\
55	5.66298701969525e-05\\
56	5.66316561437757e-05\\
57	5.66334743390744e-05\\
58	5.66353253634023e-05\\
59	5.66372098077133e-05\\
60	5.6639128273561e-05\\
61	5.66410813732722e-05\\
62	5.66430697301463e-05\\
63	5.66450939786532e-05\\
64	5.66471547646258e-05\\
65	5.66492527454679e-05\\
66	5.66513885903539e-05\\
67	5.66535629804485e-05\\
68	5.66557766091084e-05\\
69	5.66580301821122e-05\\
70	5.66603244178669e-05\\
71	5.66626600476451e-05\\
72	5.66650378158088e-05\\
73	5.66674584800427e-05\\
74	5.66699228115921e-05\\
75	5.66724315954994e-05\\
76	5.66749856308612e-05\\
77	5.66775857310647e-05\\
78	5.66802327240513e-05\\
79	5.66829274525708e-05\\
80	5.66856707744515e-05\\
81	5.66884635628547e-05\\
82	5.66913067065632e-05\\
83	5.66942011102497e-05\\
84	5.66971476947592e-05\\
85	5.67001473974056e-05\\
86	5.67032011722475e-05\\
87	5.67063099904077e-05\\
88	5.67094748403569e-05\\
89	5.67126967282335e-05\\
90	5.6715976678153e-05\\
91	5.67193157325317e-05\\
92	5.67227149524056e-05\\
93	5.67261754177648e-05\\
94	5.67296982278936e-05\\
95	5.67332845017073e-05\\
96	5.67369353781049e-05\\
97	5.67406520163216e-05\\
98	5.67444355962932e-05\\
99	5.67482873190168e-05\\
100	5.67522084069385e-05\\
101	5.67562001043188e-05\\
102	5.67602636776284e-05\\
103	5.67644004159454e-05\\
104	5.67686116313471e-05\\
105	5.67728986593281e-05\\
106	5.67772628592091e-05\\
107	5.67817056145653e-05\\
108	5.67862283336528e-05\\
109	5.67908324498519e-05\\
110	5.67955194221043e-05\\
111	5.68002907353785e-05\\
112	5.68051479011222e-05\\
113	5.68100924577353e-05\\
114	5.68151259710528e-05\\
115	5.68202500348262e-05\\
116	5.68254662712205e-05\\
117	5.68307763313213e-05\\
118	5.68361818956422e-05\\
119	5.68416846746523e-05\\
120	5.68472864093111e-05\\
121	5.68529888715999e-05\\
122	5.68587938650786e-05\\
123	5.68647032254462e-05\\
124	5.68707188211138e-05\\
125	5.68768425537746e-05\\
126	5.68830763590099e-05\\
127	5.68894222068801e-05\\
128	5.68958821025368e-05\\
129	5.69024580868522e-05\\
130	5.69091522370484e-05\\
131	5.69159666673453e-05\\
132	5.69229035296142e-05\\
133	5.69299650140502e-05\\
134	5.69371533498509e-05\\
135	5.69444708059061e-05\\
136	5.69519196915097e-05\\
137	5.69595023570693e-05\\
138	5.69672211948445e-05\\
139	5.69750786396823e-05\\
140	5.69830771697804e-05\\
141	5.69912193074509e-05\\
142	5.69995076199075e-05\\
143	5.70079447200686e-05\\
144	5.70165332673602e-05\\
145	5.70252759685443e-05\\
146	5.7034175578567e-05\\
147	5.70432349014073e-05\\
148	5.70524567909548e-05\\
149	5.70618441518888e-05\\
150	5.70713999405958e-05\\
151	5.70811271660697e-05\\
152	5.70910288908665e-05\\
153	5.71011082320461e-05\\
154	5.71113683621456e-05\\
155	5.71218125101629e-05\\
156	5.71324439625708e-05\\
157	5.71432660643297e-05\\
158	5.71542822199352e-05\\
159	5.71654958944762e-05\\
160	5.71769106147174e-05\\
161	5.71885299701938e-05\\
162	5.72003576143319e-05\\
163	5.72123972655898e-05\\
164	5.72246527086122e-05\\
165	5.72371277954205e-05\\
166	5.72498264465989e-05\\
167	5.72627526525341e-05\\
168	5.72759104746552e-05\\
169	5.7289304046695e-05\\
170	5.7302937575994e-05\\
171	5.73168153448047e-05\\
172	5.73309417116368e-05\\
173	5.734532111262e-05\\
174	5.73599580628864e-05\\
175	5.73748571579912e-05\\
176	5.73900230753458e-05\\
177	5.74054605756886e-05\\
178	5.74211745045715e-05\\
179	5.74371697938855e-05\\
180	5.74534514634026e-05\\
181	5.74700246223558e-05\\
182	5.74868944710366e-05\\
183	5.7504066302442e-05\\
184	5.7521545503925e-05\\
185	5.75393375588965e-05\\
186	5.75574480485537e-05\\
187	5.75758826536282e-05\\
188	5.75946471561863e-05\\
189	5.76137474414499e-05\\
190	5.76331894996494e-05\\
191	5.76529794279174e-05\\
192	5.76731234322205e-05\\
193	5.76936278293164e-05\\
194	5.77144990487512e-05\\
195	5.77357436348993e-05\\
196	5.77573682490412e-05\\
197	5.77793796714665e-05\\
198	5.78017848036321e-05\\
199	5.7824590670357e-05\\
200	5.78478044220528e-05\\
201	5.78714333369946e-05\\
202	5.78954848236525e-05\\
203	5.79199664230385e-05\\
204	5.7944885811122e-05\\
205	5.79702508012799e-05\\
206	5.7996069346793e-05\\
207	5.80223495433889e-05\\
208	5.80490996318398e-05\\
209	5.80763280006069e-05\\
210	5.8104043188524e-05\\
211	5.81322538875468e-05\\
212	5.81609689455523e-05\\
213	5.81901973691784e-05\\
214	5.82199483267351e-05\\
215	5.82502311511634e-05\\
216	5.82810553430406e-05\\
217	5.83124305736681e-05\\
218	5.8344366688192e-05\\
219	5.83768737088011e-05\\
220	5.84099618379744e-05\\
221	5.84436414618047e-05\\
222	5.8477923153372e-05\\
223	5.85128176761903e-05\\
224	5.85483359877188e-05\\
225	5.85844892429447e-05\\
226	5.86212887980249e-05\\
227	5.86587462140113e-05\\
228	5.86968732606373e-05\\
229	5.87356819201838e-05\\
230	5.87751843914258e-05\\
231	5.88153930936375e-05\\
232	5.88563206706936e-05\\
233	5.88979799952467e-05\\
234	5.89403841729776e-05\\
235	5.89835465469382e-05\\
236	5.90274807019734e-05\\
237	5.90722004692365e-05\\
238	5.91177199307823e-05\\
239	5.91640534242598e-05\\
240	5.9211215547695e-05\\
241	5.92592211643669e-05\\
242	5.93080854077724e-05\\
243	5.93578236867041e-05\\
244	5.94084516904164e-05\\
245	5.94599853938932e-05\\
246	5.95124410632314e-05\\
247	5.95658352611133e-05\\
248	5.96201848524043e-05\\
249	5.96755070098495e-05\\
250	5.97318192198896e-05\\
251	5.97891392885874e-05\\
252	5.98474853476729e-05\\
253	5.9906875860719e-05\\
254	5.99673296294185e-05\\
255	6.00288658000058e-05\\
256	6.0091503869797e-05\\
257	6.01552636938625e-05\\
258	6.02201654918267e-05\\
259	6.02862298548174e-05\\
260	6.03534777525321e-05\\
261	6.04219305404695e-05\\
262	6.04916099672857e-05\\
263	6.05625381823067e-05\\
264	6.0634737743187e-05\\
265	6.07082316237365e-05\\
266	6.07830432218907e-05\\
267	6.08591963678865e-05\\
268	6.09367153326126e-05\\
269	6.10156248362292e-05\\
270	6.10959500570144e-05\\
271	6.11777166403687e-05\\
272	6.12609507074481e-05\\
273	6.13456788624173e-05\\
274	6.14319281988293e-05\\
275	6.15197263128553e-05\\
276	6.16091013176736e-05\\
277	6.1700081849194e-05\\
278	6.17926970761037e-05\\
279	6.1886976710078e-05\\
280	6.19829510162074e-05\\
281	6.20806508235941e-05\\
282	6.21801075361556e-05\\
283	6.22813531436073e-05\\
284	6.23844202326519e-05\\
285	6.24893419983553e-05\\
286	6.25961522557251e-05\\
287	6.27048854514626e-05\\
288	6.28155766759293e-05\\
289	6.29282616752975e-05\\
290	6.30429768638782e-05\\
291	6.31597593366452e-05\\
292	6.32786468819367e-05\\
293	6.33996779943343e-05\\
294	6.35228918877071e-05\\
295	6.3648328508428e-05\\
296	6.37760285487269e-05\\
297	6.39060334602184e-05\\
298	6.4038385467523e-05\\
299	6.4173127582044e-05\\
300	6.43103036158209e-05\\
301	6.44499581954836e-05\\
302	6.45921367762464e-05\\
303	6.47368856559318e-05\\
304	6.48842519889479e-05\\
305	6.50342838001458e-05\\
306	6.5187029998385e-05\\
307	6.5342540389643e-05\\
308	6.55008656895513e-05\\
309	6.5662057535781e-05\\
310	6.58261685024823e-05\\
311	6.59932521214869e-05\\
312	6.61633629120389e-05\\
313	6.63365563934328e-05\\
314	6.65128890395619e-05\\
315	6.66924183253693e-05\\
316	6.68752027372904e-05\\
317	6.70613017828279e-05\\
318	6.72507759991253e-05\\
319	6.74436869603686e-05\\
320	6.76400972838365e-05\\
321	6.78400706344218e-05\\
322	6.80436717273548e-05\\
323	6.82509663289394e-05\\
324	6.8462021254971e-05\\
325	6.86769043665899e-05\\
326	6.88956845632034e-05\\
327	6.91184317721492e-05\\
328	6.93452169347089e-05\\
329	6.9576111988072e-05\\
330	6.98111898428439e-05\\
331	7.00505243556354e-05\\
332	7.02941902964806e-05\\
333	7.05422633113625e-05\\
334	7.07948198825919e-05\\
335	7.10519372980586e-05\\
336	7.13136936645921e-05\\
337	7.15801680536252e-05\\
338	7.18514408826577e-05\\
339	7.21275938905929e-05\\
340	7.24087042430501e-05\\
341	7.26948428587868e-05\\
342	7.29860910020485e-05\\
343	7.32825302042284e-05\\
344	7.35842422087068e-05\\
345	7.38913089471398e-05\\
346	7.42038126507727e-05\\
347	7.4521836511912e-05\\
348	7.48454672176253e-05\\
349	7.51748004463351e-05\\
350	7.5509902440845e-05\\
351	7.5850838820205e-05\\
352	7.61977131217195e-05\\
353	7.65506307205999e-05\\
354	7.69096988742068e-05\\
355	7.72750267650979e-05\\
356	7.7646725496567e-05\\
357	7.80249079431435e-05\\
358	7.84096885959737e-05\\
359	7.88011843126841e-05\\
360	7.91995139765483e-05\\
361	7.96047985307576e-05\\
362	8.00171610146243e-05\\
363	8.04367266021808e-05\\
364	8.08636226435882e-05\\
365	8.12979787099416e-05\\
366	8.17399266420663e-05\\
367	8.21896006039804e-05\\
368	8.26471371416283e-05\\
369	8.31126752473286e-05\\
370	8.35863564296393e-05\\
371	8.40683247869657e-05\\
372	8.45587270806473e-05\\
373	8.50577128009715e-05\\
374	8.55654342253399e-05\\
375	8.60820465095733e-05\\
376	8.66077079884137e-05\\
377	8.71425810522433e-05\\
378	8.76868332892592e-05\\
379	8.82406345271868e-05\\
380	8.88041587279447e-05\\
381	8.93775842893488e-05\\
382	8.99610943089507e-05\\
383	9.0554876694332e-05\\
384	9.11591238130058e-05\\
385	9.17740312406845e-05\\
386	9.23997965054899e-05\\
387	9.30366290780722e-05\\
388	9.36848187472196e-05\\
389	9.43446116472078e-05\\
390	9.50162470261731e-05\\
391	9.56999747748655e-05\\
392	9.63960564241728e-05\\
393	9.71047662166858e-05\\
394	9.78263922479534e-05\\
395	9.85612376699937e-05\\
396	9.9309621946302e-05\\
397	0.000100071882143825\\
398	0.000100848374240193\\
399	0.000101639474398394\\
400	0.000102445580053366\\
401	0.000103267110253841\\
402	0.00010410450340359\\
403	0.000104958207218767\\
404	0.000105828653439658\\
405	0.000106716280661696\\
406	0.00010762201554143\\
407	0.000108546250093435\\
408	0.00010948940796019\\
409	0.000110451937993203\\
410	0.000111434327441338\\
411	0.000112437128199277\\
412	0.000113460940360323\\
413	0.00011450600917254\\
414	0.000115572906888593\\
415	0.000116662229101265\\
416	0.000117774596096609\\
417	0.000118910654227204\\
418	0.000120071077201777\\
419	0.00012125656710526\\
420	0.000122467854998132\\
421	0.000123705701775496\\
422	0.000124970903507071\\
423	0.00012626431100586\\
424	0.000127586837555735\\
425	0.000128939414976839\\
426	0.00013032302322817\\
427	0.000131738694130578\\
428	0.00013318751547734\\
429	0.000134670635581798\\
430	0.00013618926831784\\
431	0.000137744698715181\\
432	0.000139338289178032\\
433	0.000140971486410228\\
434	0.000142645829191509\\
435	0.000144362957433858\\
436	0.000146124624135332\\
437	0.000147932716261748\\
438	0.000149789303711434\\
439	0.000151696737210097\\
440	0.000153657119450931\\
441	0.000155672411296715\\
442	0.000157745331427691\\
443	0.000159878840783038\\
444	0.000162076176610412\\
445	0.000164340895066859\\
446	0.000166676927524447\\
447	0.000169088654855352\\
448	0.000171580961846792\\
449	0.000174159011220003\\
450	0.00017682868879919\\
451	0.000179596682747477\\
452	0.000182470525348742\\
453	0.000185458702239571\\
454	0.000188570720172352\\
455	0.000191817131472747\\
456	0.000195210848300859\\
457	0.000199307462198322\\
458	0.000234024009405493\\
459	0.00026928460136731\\
460	0.000305276150154545\\
461	0.000342075490123044\\
462	0.000379704063110204\\
463	0.000418183341215599\\
464	0.000457534982089145\\
465	0.000497777696872637\\
466	0.000538934363692127\\
467	0.000581026754456863\\
468	0.000624074492834437\\
469	0.000668093813393791\\
470	0.00071309587762147\\
471	0.000759084974786108\\
472	0.000806052990455533\\
473	0.000854028376390557\\
474	0.000903051642598929\\
475	0.000953165761908892\\
476	0.00100441633532496\\
477	0.00105685177698614\\
478	0.00111052282832288\\
479	0.00116548595276004\\
480	0.0012217935582322\\
481	0.0012794881130283\\
482	0.0013386153471524\\
483	0.00139922231598797\\
484	0.00146135345217606\\
485	0.00152504279172324\\
486	0.00158966854592528\\
487	0.00165368135681658\\
488	0.00171933201283993\\
489	0.00178671392434228\\
490	0.00185593334775801\\
491	0.00192709610727309\\
492	0.00200031965459637\\
493	0.00207573480513079\\
494	0.00215349542970393\\
495	0.00223377358343795\\
496	0.0023167659741897\\
497	0.00240270009880341\\
498	0.00249184103455956\\
499	0.00258450361043243\\
500	0.00268106407111543\\
501	0.00278190576531213\\
502	0.0028874736431148\\
503	0.0029982859704351\\
504	0.00311494759397139\\
505	0.0032381634133137\\
506	0.00336876465412971\\
507	0.00350338215803569\\
508	0.00364131750666185\\
509	0.00374611681805963\\
510	0.00381773951732628\\
511	0.00389051142076527\\
512	0.00396441256935567\\
513	0.00403941524333922\\
514	0.00411548449939267\\
515	0.00419258349552635\\
516	0.00427069909279331\\
517	0.00434994898963162\\
518	0.0044303370723928\\
519	0.0045117966287361\\
520	0.00459424109225291\\
521	0.00467756014622477\\
522	0.00476161321976009\\
523	0.00484622259021371\\
524	0.00493116498255079\\
525	0.00501616200872132\\
526	0.00510086597846347\\
527	0.00518485278179517\\
528	0.0052675936890152\\
529	0.00534842588310246\\
530	0.00542652963648317\\
531	0.00550089386889214\\
532	0.00557211176756093\\
533	0.0056449527050025\\
534	0.00571947466363871\\
535	0.00579570983248068\\
536	0.00587364085006118\\
537	0.00595321734125755\\
538	0.00603434412927874\\
539	0.00611686570736999\\
540	0.00620054569846214\\
541	0.00628504569438276\\
542	0.00637016506052658\\
543	0.00645710221151984\\
544	0.00654554616480435\\
545	0.00663503398630233\\
546	0.00672368199808921\\
547	0.00681075256412289\\
548	0.00689600642609548\\
549	0.00697921067175644\\
550	0.00706015269502789\\
551	0.00713866037434254\\
552	0.00721463073987145\\
553	0.00728807012257132\\
554	0.00735895954660089\\
555	0.00742723438282898\\
556	0.00749292400561867\\
557	0.00755666941165881\\
558	0.00762007127688854\\
559	0.0076831066443758\\
560	0.00774576403788845\\
561	0.00780802957268453\\
562	0.00786990696145242\\
563	0.00793142454002589\\
564	0.00799263908789431\\
565	0.00805363888320774\\
566	0.00811454494091608\\
567	0.00817550922357873\\
568	0.00823671564158016\\
569	0.00829828412878018\\
570	0.00836024274084549\\
571	0.00842259141986931\\
572	0.00848532757207436\\
573	0.00854844118564728\\
574	0.00861191607274561\\
575	0.00867573559690069\\
576	0.00873988162299363\\
577	0.0088043330464253\\
578	0.0088690640387695\\
579	0.0089340421116632\\
580	0.00899922608544475\\
581	0.00906456465846054\\
582	0.00913000140892399\\
583	0.00919547944080252\\
584	0.0092609411779913\\
585	0.00932634987444189\\
586	0.00939167533016675\\
587	0.00945690191585901\\
588	0.00952202325045493\\
589	0.00958699516677063\\
590	0.00965014598496389\\
591	0.00971019945887051\\
592	0.00976660115966509\\
593	0.00981892128442635\\
594	0.00986674062690153\\
595	0.00990834109159665\\
596	0.00994554766474265\\
597	0.0099771937715668\\
598	0.00999970795535495\\
599	0\\
600	0\\
};
\addplot [color=mycolor17,solid,forget plot]
  table[row sep=crcr]{%
1	0.00060710056306544\\
2	0.000607111173053\\
3	0.000607121975141467\\
4	0.000607132972803783\\
5	0.000607144169575505\\
6	0.000607155569055962\\
7	0.000607167174909374\\
8	0.000607178990866045\\
9	0.000607191020723523\\
10	0.000607203268347803\\
11	0.000607215737674604\\
12	0.000607228432710565\\
13	0.000607241357534549\\
14	0.000607254516298919\\
15	0.000607267913230871\\
16	0.000607281552633782\\
17	0.000607295438888552\\
18	0.00060730957645503\\
19	0.000607323969873404\\
20	0.000607338623765637\\
21	0.000607353542836953\\
22	0.00060736873187732\\
23	0.000607384195762969\\
24	0.000607399939457934\\
25	0.000607415968015658\\
26	0.000607432286580546\\
27	0.000607448900389626\\
28	0.000607465814774214\\
29	0.000607483035161575\\
30	0.000607500567076671\\
31	0.000607518416143883\\
32	0.000607536588088827\\
33	0.000607555088740134\\
34	0.000607573924031305\\
35	0.000607593100002599\\
36	0.000607612622802933\\
37	0.000607632498691823\\
38	0.000607652734041384\\
39	0.000607673335338322\\
40	0.000607694309185988\\
41	0.000607715662306484\\
42	0.000607737401542763\\
43	0.000607759533860813\\
44	0.000607782066351834\\
45	0.000607805006234513\\
46	0.000607828360857252\\
47	0.000607852137700522\\
48	0.000607876344379248\\
49	0.000607900988645143\\
50	0.000607926078389216\\
51	0.000607951621644233\\
52	0.000607977626587251\\
53	0.000608004101542216\\
54	0.000608031054982551\\
55	0.000608058495533872\\
56	0.000608086431976664\\
57	0.000608114873249078\\
58	0.00060814382844972\\
59	0.000608173306840546\\
60	0.000608203317849766\\
61	0.000608233871074806\\
62	0.000608264976285318\\
63	0.000608296643426294\\
64	0.000608328882621162\\
65	0.000608361704174973\\
66	0.000608395118577665\\
67	0.000608429136507334\\
68	0.000608463768833609\\
69	0.000608499026621057\\
70	0.000608534921132678\\
71	0.00060857146383343\\
72	0.00060860866639383\\
73	0.00060864654069363\\
74	0.000608685098825536\\
75	0.000608724353099022\\
76	0.00060876431604417\\
77	0.000608805000415634\\
78	0.000608846419196619\\
79	0.000608888585602958\\
80	0.000608931513087274\\
81	0.000608975215343184\\
82	0.00060901970630959\\
83	0.000609065000175061\\
84	0.000609111111382292\\
85	0.000609158054632594\\
86	0.000609205844890553\\
87	0.000609254497388657\\
88	0.000609304027632135\\
89	0.000609354451403769\\
90	0.00060940578476883\\
91	0.000609458044080163\\
92	0.000609511245983242\\
93	0.000609565407421429\\
94	0.000609620545641255\\
95	0.000609676678197812\\
96	0.000609733822960262\\
97	0.000609791998117441\\
98	0.000609851222183499\\
99	0.000609911514003726\\
100	0.000609972892760453\\
101	0.000610035377979026\\
102	0.000610098989533918\\
103	0.000610163747654957\\
104	0.000610229672933634\\
105	0.000610296786329516\\
106	0.000610365109176851\\
107	0.000610434663191176\\
108	0.000610505470476148\\
109	0.0006105775535304\\
110	0.00061065093525462\\
111	0.000610725638958657\\
112	0.000610801688368834\\
113	0.000610879107635327\\
114	0.000610957921339723\\
115	0.000611038154502675\\
116	0.000611119832591736\\
117	0.000611202981529285\\
118	0.00061128762770063\\
119	0.000611373797962243\\
120	0.000611461519650115\\
121	0.000611550820588303\\
122	0.000611641729097622\\
123	0.000611734274004447\\
124	0.000611828484649738\\
125	0.00061192439089816\\
126	0.000612022023147421\\
127	0.000612121412337731\\
128	0.000612222589961464\\
129	0.000612325588072974\\
130	0.000612430439298598\\
131	0.000612537176846808\\
132	0.000612645834518587\\
133	0.000612756446717948\\
134	0.000612869048462679\\
135	0.000612983675395235\\
136	0.000613100363793881\\
137	0.000613219150583975\\
138	0.0006133400733495\\
139	0.000613463170344769\\
140	0.000613588480506343\\
141	0.00061371604346519\\
142	0.00061384589955904\\
143	0.000613978089844943\\
144	0.000614112656112076\\
145	0.000614249640894796\\
146	0.000614389087485876\\
147	0.000614531039950022\\
148	0.000614675543137608\\
149	0.000614822642698651\\
150	0.000614972385097093\\
151	0.000615124817625259\\
152	0.000615279988418607\\
153	0.000615437946470778\\
154	0.000615598741648863\\
155	0.000615762424708964\\
156	0.000615929047312047\\
157	0.000616098662040062\\
158	0.000616271322412393\\
159	0.000616447082902506\\
160	0.000616625998955027\\
161	0.00061680812700302\\
162	0.000616993524485646\\
163	0.000617182249866102\\
164	0.000617374362649899\\
165	0.000617569923403457\\
166	0.000617768993773077\\
167	0.000617971636504181\\
168	0.000618177915460988\\
169	0.000618387895646483\\
170	0.000618601643222742\\
171	0.000618819225531706\\
172	0.000619040711116234\\
173	0.00061926616974161\\
174	0.000619495672417392\\
175	0.000619729291419688\\
176	0.000619967100313847\\
177	0.000620209173977507\\
178	0.000620455588624137\\
179	0.000620706421826964\\
180	0.000620961752543351\\
181	0.000621221661139611\\
182	0.000621486229416293\\
183	0.000621755540633897\\
184	0.000622029679539122\\
185	0.000622308732391533\\
186	0.000622592786990749\\
187	0.000622881932704141\\
188	0.000623176260495003\\
189	0.000623475862951279\\
190	0.000623780834314821\\
191	0.000624091270511146\\
192	0.000624407269179785\\
193	0.000624728929705209\\
194	0.000625056353248261\\
195	0.000625389642778231\\
196	0.000625728903105525\\
197	0.000626074240914882\\
198	0.000626425764799321\\
199	0.000626783585294594\\
200	0.000627147814914393\\
201	0.000627518568186147\\
202	0.000627895961687519\\
203	0.00062828011408358\\
204	0.000628671146164665\\
205	0.000629069180885006\\
206	0.000629474343402009\\
207	0.000629886761116316\\
208	0.00063030656371266\\
209	0.000630733883201393\\
210	0.000631168853960897\\
211	0.000631611612780782\\
212	0.000632062298905836\\
213	0.000632521054080933\\
214	0.000632988022596688\\
215	0.000633463351336034\\
216	0.000633947189821688\\
217	0.000634439690264526\\
218	0.000634941007612866\\
219	0.000635451299602746\\
220	0.00063597072680911\\
221	0.000636499452698023\\
222	0.000637037643679888\\
223	0.000637585469163665\\
224	0.00063814310161219\\
225	0.000638710716598526\\
226	0.000639288492863424\\
227	0.000639876612373886\\
228	0.000640475260382908\\
229	0.000641084625490354\\
230	0.000641704899705041\\
231	0.000642336278508026\\
232	0.000642978960917117\\
233	0.000643633149552712\\
234	0.000644299050704883\\
235	0.00064497687440177\\
236	0.000645666834479395\\
237	0.000646369148652763\\
238	0.000647084038588451\\
239	0.000647811729978585\\
240	0.000648552452616331\\
241	0.000649306440472815\\
242	0.000650073931775659\\
243	0.000650855169089061\\
244	0.000651650399395419\\
245	0.000652459874178667\\
246	0.000653283849509245\\
247	0.000654122586130791\\
248	0.000654976349548549\\
249	0.000655845410119596\\
250	0.000656730043144877\\
251	0.000657630528963128\\
252	0.000658547153046668\\
253	0.000659480206099143\\
254	0.000660429984155319\\
255	0.000661396788682836\\
256	0.00066238092668611\\
257	0.000663382710812323\\
258	0.000664402459459641\\
259	0.000665440496887614\\
260	0.000666497153329897\\
261	0.000667572765109307\\
262	0.000668667674755246\\
263	0.000669782231123588\\
264	0.000670916789519063\\
265	0.000672071711820202\\
266	0.000673247366606806\\
267	0.000674444129290063\\
268	0.000675662382245147\\
269	0.000676902514946222\\
270	0.000678164924103469\\
271	0.000679450013801791\\
272	0.000680758195640947\\
273	0.000682089888877918\\
274	0.000683445520574564\\
275	0.000684825525757863\\
276	0.000686230347613509\\
277	0.00068766043765968\\
278	0.000689116255873479\\
279	0.00069059827085856\\
280	0.000692106960016313\\
281	0.000693642809720723\\
282	0.000695206315496843\\
283	0.000696797982203112\\
284	0.000698418324217433\\
285	0.000700067865627164\\
286	0.000701747140423026\\
287	0.000703456692697003\\
288	0.000705197076844282\\
289	0.000706968857769262\\
290	0.000708772611095645\\
291	0.000710608923380657\\
292	0.000712478392333387\\
293	0.00071438162703725\\
294	0.000716319248176551\\
295	0.000718291888267079\\
296	0.000720300191890713\\
297	0.000722344815933939\\
298	0.000724426429830121\\
299	0.000726545715805459\\
300	0.000728703369128351\\
301	0.000730900098361999\\
302	0.000733136625620009\\
303	0.000735413686824708\\
304	0.000737732031967852\\
305	0.000740092425373718\\
306	0.000742495645964528\\
307	0.000744942487528862\\
308	0.00074743375899431\\
309	0.000749970284705992\\
310	0.000752552904711658\\
311	0.000755182475048859\\
312	0.000757859868016764\\
313	0.000760585972395715\\
314	0.000763361693543721\\
315	0.000766187953351628\\
316	0.000769065690777267\\
317	0.000771995862038124\\
318	0.000774979440783901\\
319	0.000778017418244921\\
320	0.000781110803352149\\
321	0.000784260622823736\\
322	0.000787467921212758\\
323	0.000790733760909778\\
324	0.000794059222093577\\
325	0.000797445402622664\\
326	0.000800893417859552\\
327	0.000804404400419329\\
328	0.000807979499833827\\
329	0.000811619882122375\\
330	0.000815326729260301\\
331	0.000819101238536467\\
332	0.000822944621790806\\
333	0.000826858104520196\\
334	0.000830842924832522\\
335	0.000834900332203871\\
336	0.000839031585935278\\
337	0.000843237953087477\\
338	0.000847520705524725\\
339	0.000851881115892226\\
340	0.000856320454048436\\
341	0.00086083997460567\\
342	0.000865440950954022\\
343	0.000870124765520062\\
344	0.000874892813357461\\
345	0.000879746502099697\\
346	0.000884687251878435\\
347	0.000889716494637998\\
348	0.000894835671465534\\
349	0.000900046226458471\\
350	0.00090534961427422\\
351	0.000910747165660721\\
352	0.000916240365161452\\
353	0.000921831016938411\\
354	0.000927520964625549\\
355	0.000933312092390991\\
356	0.00093920632597568\\
357	0.000945205633768997\\
358	0.00095131202810527\\
359	0.000957527567481472\\
360	0.000963854362196069\\
361	0.000970294572684496\\
362	0.000976850411488681\\
363	0.000983524145387163\\
364	0.000990318097706229\\
365	0.000997234650834975\\
366	0.00100427624897077\\
367	0.00101144540112513\\
368	0.00101874468442492\\
369	0.00102617674774945\\
370	0.00103374431575295\\
371	0.00104145019333466\\
372	0.00104929727064038\\
373	0.00105728852871219\\
374	0.00106542704594637\\
375	0.00107371600554792\\
376	0.00108215870409779\\
377	0.00109075856102227\\
378	0.00109951912815803\\
379	0.00110844409773115\\
380	0.00111753729548966\\
381	0.00112680270845646\\
382	0.00113624449881478\\
383	0.0011458670197193\\
384	0.00115567483428913\\
385	0.00116567273972835\\
386	0.00117586579861875\\
387	0.00118625937439456\\
388	0.00119685915185121\\
389	0.00120767126340748\\
390	0.00121870143312448\\
391	0.0012299554664848\\
392	0.0012414394300102\\
393	0.00125315966563248\\
394	0.0012651228044699\\
395	0.00127733577938584\\
396	0.00128980583549718\\
397	0.00130254053754614\\
398	0.00131554777276393\\
399	0.00132883574759273\\
400	0.00134241297654273\\
401	0.0013562882619089\\
402	0.00137047066481572\\
403	0.00138496947217772\\
404	0.00139979416905491\\
405	0.00141495440053258\\
406	0.00143045987581483\\
407	0.00144632158239907\\
408	0.001462545299621\\
409	0.00147913585335357\\
410	0.0014960965559549\\
411	0.00151342860016098\\
412	0.00153113025192362\\
413	0.00154919585542169\\
414	0.00156761450204312\\
415	0.0015863940812741\\
416	0.00160554271232852\\
417	0.0016250687544487\\
418	0.00164498081875182\\
419	0.00166528778274928\\
420	0.00168599880951421\\
421	0.00170712337422873\\
422	0.00172867129979596\\
423	0.00175065279766935\\
424	0.00177307850796426\\
425	0.0017959594132778\\
426	0.00181930659254416\\
427	0.00184313149916327\\
428	0.00186744597859563\\
429	0.00189226228667248\\
430	0.00191759310851943\\
431	0.00194345157791757\\
432	0.00196985129679493\\
433	0.00199680635427785\\
434	0.00202433134415148\\
435	0.00205244137820251\\
436	0.00208115208955569\\
437	0.00211047961200326\\
438	0.00214044050392997\\
439	0.00217105157111254\\
440	0.0022023298338896\\
441	0.00223429036131121\\
442	0.00226694938910766\\
443	0.00230032951286536\\
444	0.00233445494323543\\
445	0.00236935172915597\\
446	0.00240504800650734\\
447	0.00244157425986033\\
448	0.00247896359019233\\
449	0.00251725205283156\\
450	0.00255647831692348\\
451	0.00259668741809526\\
452	0.00263793096407801\\
453	0.00268026763373766\\
454	0.00272376469874542\\
455	0.00276849970976649\\
456	0.0028145610197631\\
457	0.00286150848974166\\
458	0.00287941697542049\\
459	0.00289800783767426\\
460	0.00291735065292944\\
461	0.00293752028578222\\
462	0.00295860304139865\\
463	0.00298069934473722\\
464	0.0030039265272458\\
465	0.00302842225403747\\
466	0.00305434867011648\\
467	0.00308189747994569\\
468	0.00311129624208877\\
469	0.0031428161229582\\
470	0.0031767814049533\\
471	0.00321358089606086\\
472	0.00325368035722807\\
473	0.00329501424169953\\
474	0.00333702469320655\\
475	0.00337971622038544\\
476	0.00342309259553974\\
477	0.00346715681227577\\
478	0.00351191100308439\\
479	0.00355735611816516\\
480	0.0036034912120651\\
481	0.00365030944749394\\
482	0.00369777262859418\\
483	0.00374585873342885\\
484	0.00379455808611841\\
485	0.00384386220384708\\
486	0.00389376869842156\\
487	0.00394428392090048\\
488	0.0039953968683074\\
489	0.00404708303801088\\
490	0.00409935611848499\\
491	0.00415228610492657\\
492	0.0042058298844828\\
493	0.00425993285968233\\
494	0.00431452645753851\\
495	0.00436952534095501\\
496	0.00442482334969621\\
497	0.00448028878141799\\
498	0.00453575849129831\\
499	0.00459103030891929\\
500	0.00464585317456217\\
501	0.00469991774731134\\
502	0.00475284339576551\\
503	0.00480416148072935\\
504	0.00485329454992126\\
505	0.00489953057909042\\
506	0.00494199189982736\\
507	0.00498403014859703\\
508	0.00502631851116879\\
509	0.00506877894432613\\
510	0.00511180870801828\\
511	0.00515583199383242\\
512	0.0052008761733455\\
513	0.00524696989751277\\
514	0.00529414375492416\\
515	0.00534243186387451\\
516	0.00539186417563767\\
517	0.00544246409415458\\
518	0.00549425230082413\\
519	0.00554724840675153\\
520	0.00560147249534651\\
521	0.00565694642515098\\
522	0.00571369586871487\\
523	0.00577175851596171\\
524	0.00583119248510434\\
525	0.00589205603176269\\
526	0.00595440650781836\\
527	0.00601829894926148\\
528	0.00608378416842955\\
529	0.00615090640577239\\
530	0.00621970099943915\\
531	0.006290191220047\\
532	0.00636234493983808\\
533	0.00643593470847046\\
534	0.0065104055149119\\
535	0.00658372831899706\\
536	0.00665569931798794\\
537	0.00672610360775146\\
538	0.00679471924588918\\
539	0.00686132371717709\\
540	0.0069257037777008\\
541	0.00698766973345227\\
542	0.00704707286005866\\
543	0.00710380131076663\\
544	0.00715783708505831\\
545	0.00720930379890342\\
546	0.00725978899536815\\
547	0.00730978161638203\\
548	0.00735929416828949\\
549	0.00740835724334546\\
550	0.00745702181705139\\
551	0.00750536083932739\\
552	0.007553469486322\\
553	0.00760146317781643\\
554	0.00764947698034132\\
555	0.00769766562106882\\
556	0.00774618802061691\\
557	0.00779517789609317\\
558	0.00784468709298625\\
559	0.00789473898348791\\
560	0.00794535972575758\\
561	0.00799657787168881\\
562	0.0080484239854243\\
563	0.00810092960386494\\
564	0.00815412576667316\\
565	0.00820804117363287\\
566	0.00826270003308565\\
567	0.00831811978243239\\
568	0.00837430874090948\\
569	0.00843126714425778\\
570	0.008488991119373\\
571	0.0085474742880391\\
572	0.00860670740247375\\
573	0.00866667814978815\\
574	0.00872737096574959\\
575	0.00878876667075254\\
576	0.00885084194403563\\
577	0.00891356973300979\\
578	0.00897691638806212\\
579	0.00904084037040419\\
580	0.00910530852313021\\
581	0.00917028946955943\\
582	0.009235735599458\\
583	0.00930157560444774\\
584	0.0093677995709556\\
585	0.00943209978059338\\
586	0.00949409750521473\\
587	0.00955271158918268\\
588	0.00960758154467065\\
589	0.00965794928725074\\
590	0.00970496351641899\\
591	0.00974966222411227\\
592	0.00979242953925062\\
593	0.0098334977384428\\
594	0.00987276071723054\\
595	0.00991035431098038\\
596	0.00994573390547615\\
597	0.0099771937715668\\
598	0.00999970795535495\\
599	0\\
600	0\\
};
\addplot [color=mycolor18,solid,forget plot]
  table[row sep=crcr]{%
1	0.00244237844548261\\
2	0.0024423835258782\\
3	0.00244238869846641\\
4	0.00244239396492018\\
5	0.00244239932694278\\
6	0.00244240478626838\\
7	0.00244241034466266\\
8	0.00244241600392328\\
9	0.00244242176588055\\
10	0.00244242763239799\\
11	0.00244243360537293\\
12	0.0024424396867371\\
13	0.00244244587845728\\
14	0.00244245218253596\\
15	0.00244245860101195\\
16	0.00244246513596102\\
17	0.00244247178949661\\
18	0.0024424785637705\\
19	0.00244248546097347\\
20	0.00244249248333609\\
21	0.00244249963312933\\
22	0.00244250691266538\\
23	0.00244251432429838\\
24	0.00244252187042512\\
25	0.00244252955348586\\
26	0.00244253737596515\\
27	0.00244254534039255\\
28	0.00244255344934349\\
29	0.00244256170544011\\
30	0.00244257011135209\\
31	0.0024425786697975\\
32	0.00244258738354368\\
33	0.00244259625540815\\
34	0.00244260528825948\\
35	0.00244261448501826\\
36	0.002442623848658\\
37	0.0024426333822061\\
38	0.00244264308874481\\
39	0.00244265297141227\\
40	0.00244266303340347\\
41	0.00244267327797128\\
42	0.00244268370842751\\
43	0.00244269432814396\\
44	0.00244270514055357\\
45	0.00244271614915138\\
46	0.00244272735749581\\
47	0.00244273876920971\\
48	0.00244275038798153\\
49	0.00244276221756652\\
50	0.00244277426178797\\
51	0.00244278652453837\\
52	0.00244279900978069\\
53	0.00244281172154965\\
54	0.00244282466395305\\
55	0.002442837841173\\
56	0.00244285125746735\\
57	0.00244286491717099\\
58	0.00244287882469729\\
59	0.00244289298453949\\
60	0.00244290740127211\\
61	0.00244292207955247\\
62	0.00244293702412215\\
63	0.00244295223980852\\
64	0.00244296773152627\\
65	0.002442983504279\\
66	0.00244299956316078\\
67	0.00244301591335784\\
68	0.00244303256015021\\
69	0.00244304950891338\\
70	0.00244306676512002\\
71	0.00244308433434176\\
72	0.00244310222225098\\
73	0.00244312043462255\\
74	0.00244313897733575\\
75	0.00244315785637612\\
76	0.00244317707783735\\
77	0.00244319664792325\\
78	0.00244321657294974\\
79	0.00244323685934682\\
80	0.00244325751366067\\
81	0.0024432785425557\\
82	0.0024432999528167\\
83	0.002443321751351\\
84	0.00244334394519064\\
85	0.00244336654149466\\
86	0.00244338954755132\\
87	0.0024434129707805\\
88	0.00244343681873596\\
89	0.00244346109910785\\
90	0.00244348581972508\\
91	0.00244351098855784\\
92	0.00244353661372015\\
93	0.00244356270347239\\
94	0.00244358926622398\\
95	0.00244361631053604\\
96	0.00244364384512406\\
97	0.00244367187886075\\
98	0.00244370042077877\\
99	0.00244372948007369\\
100	0.00244375906610683\\
101	0.00244378918840829\\
102	0.00244381985667991\\
103	0.00244385108079841\\
104	0.00244388287081846\\
105	0.00244391523697594\\
106	0.00244394818969112\\
107	0.00244398173957197\\
108	0.00244401589741758\\
109	0.00244405067422149\\
110	0.00244408608117523\\
111	0.00244412212967185\\
112	0.0024441588313095\\
113	0.00244419619789514\\
114	0.00244423424144822\\
115	0.00244427297420454\\
116	0.00244431240862009\\
117	0.00244435255737496\\
118	0.00244439343337737\\
119	0.00244443504976774\\
120	0.00244447741992283\\
121	0.00244452055746\\
122	0.00244456447624145\\
123	0.00244460919037862\\
124	0.00244465471423664\\
125	0.00244470106243884\\
126	0.00244474824987139\\
127	0.00244479629168795\\
128	0.00244484520331446\\
129	0.00244489500045402\\
130	0.00244494569909176\\
131	0.00244499731549994\\
132	0.00244504986624304\\
133	0.00244510336818296\\
134	0.00244515783848432\\
135	0.00244521329461989\\
136	0.002445269754376\\
137	0.00244532723585818\\
138	0.00244538575749684\\
139	0.00244544533805303\\
140	0.00244550599662434\\
141	0.00244556775265087\\
142	0.00244563062592133\\
143	0.00244569463657923\\
144	0.00244575980512923\\
145	0.00244582615244347\\
146	0.00244589369976821\\
147	0.00244596246873035\\
148	0.00244603248134433\\
149	0.00244610376001889\\
150	0.00244617632756412\\
151	0.0024462502071986\\
152	0.00244632542255664\\
153	0.00244640199769559\\
154	0.00244647995710343\\
155	0.00244655932570635\\
156	0.00244664012887657\\
157	0.00244672239244016\\
158	0.00244680614268512\\
159	0.00244689140636959\\
160	0.00244697821073016\\
161	0.00244706658349031\\
162	0.00244715655286902\\
163	0.00244724814758958\\
164	0.00244734139688848\\
165	0.00244743633052448\\
166	0.00244753297878786\\
167	0.00244763137250975\\
168	0.00244773154307178\\
169	0.00244783352241572\\
170	0.00244793734305341\\
171	0.00244804303807678\\
172	0.00244815064116814\\
173	0.00244826018661054\\
174	0.00244837170929836\\
175	0.00244848524474814\\
176	0.00244860082910945\\
177	0.00244871849917614\\
178	0.0024488382923976\\
179	0.00244896024689035\\
180	0.00244908440144973\\
181	0.0024492107955619\\
182	0.00244933946941591\\
183	0.00244947046391613\\
184	0.00244960382069475\\
185	0.00244973958212462\\
186	0.00244987779133223\\
187	0.00245001849221088\\
188	0.00245016172943426\\
189	0.00245030754847002\\
190	0.00245045599559375\\
191	0.00245060711790313\\
192	0.00245076096333232\\
193	0.0024509175806666\\
194	0.00245107701955733\\
195	0.00245123933053703\\
196	0.00245140456503486\\
197	0.00245157277539225\\
198	0.00245174401487891\\
199	0.00245191833770901\\
200	0.00245209579905768\\
201	0.00245227645507782\\
202	0.00245246036291716\\
203	0.00245264758073561\\
204	0.00245283816772298\\
205	0.00245303218411688\\
206	0.00245322969122101\\
207	0.0024534307514238\\
208	0.00245363542821723\\
209	0.00245384378621613\\
210	0.00245405589117771\\
211	0.00245427181002146\\
212	0.00245449161084941\\
213	0.00245471536296667\\
214	0.0024549431369024\\
215	0.00245517500443111\\
216	0.00245541103859433\\
217	0.00245565131372263\\
218	0.00245589590545805\\
219	0.00245614489077693\\
220	0.00245639834801303\\
221	0.00245665635688126\\
222	0.00245691899850154\\
223	0.00245718635542333\\
224	0.00245745851165044\\
225	0.00245773555266628\\
226	0.00245801756545962\\
227	0.0024583046385507\\
228	0.0024585968620179\\
229	0.00245889432752477\\
230	0.00245919712834759\\
231	0.00245950535940338\\
232	0.0024598191172785\\
233	0.00246013850025754\\
234	0.00246046360835289\\
235	0.00246079454333482\\
236	0.00246113140876195\\
237	0.00246147431001243\\
238	0.0024618233543155\\
239	0.00246217865078373\\
240	0.00246254031044571\\
241	0.00246290844627947\\
242	0.00246328317324632\\
243	0.00246366460832532\\
244	0.00246405287054844\\
245	0.00246444808103628\\
246	0.00246485036303433\\
247	0.00246525984195002\\
248	0.00246567664539031\\
249	0.00246610090319997\\
250	0.00246653274750046\\
251	0.00246697231272962\\
252	0.00246741973568192\\
253	0.00246787515554949\\
254	0.00246833871396379\\
255	0.00246881055503808\\
256	0.00246929082541056\\
257	0.0024697796742883\\
258	0.00247027725349187\\
259	0.00247078371750084\\
260	0.00247129922349995\\
261	0.00247182393142617\\
262	0.00247235800401653\\
263	0.00247290160685679\\
264	0.00247345490843098\\
265	0.0024740180801717\\
266	0.00247459129651147\\
267	0.00247517473493476\\
268	0.00247576857603087\\
269	0.00247637300354733\\
270	0.00247698820444326\\
271	0.00247761436894093\\
272	0.00247825169057141\\
273	0.00247890036620627\\
274	0.00247956059606223\\
275	0.00248023258367863\\
276	0.00248091653595876\\
277	0.00248161266358066\\
278	0.00248232118125036\\
279	0.00248304230741867\\
280	0.00248377626434689\\
281	0.00248452327817387\\
282	0.00248528357898464\\
283	0.0024860574008804\\
284	0.00248684498205017\\
285	0.00248764656484421\\
286	0.00248846239584902\\
287	0.00248929272596439\\
288	0.00249013781048225\\
289	0.00249099790916781\\
290	0.00249187328634269\\
291	0.00249276421097064\\
292	0.00249367095674565\\
293	0.00249459380218278\\
294	0.00249553303071185\\
295	0.00249648893077427\\
296	0.00249746179592313\\
297	0.00249845192492681\\
298	0.00249945962187647\\
299	0.00250048519629757\\
300	0.00250152896326585\\
301	0.00250259124352806\\
302	0.00250367236362781\\
303	0.00250477265603699\\
304	0.00250589245929339\\
305	0.00250703211814503\\
306	0.00250819198370252\\
307	0.00250937241360174\\
308	0.00251057377218133\\
309	0.00251179643068557\\
310	0.00251304076751458\\
311	0.00251430716856387\\
312	0.00251559602771501\\
313	0.00251690774749291\\
314	0.00251824273957381\\
315	0.00251960142391174\\
316	0.00252098422590553\\
317	0.00252239157967583\\
318	0.00252382392834611\\
319	0.00252528172433526\\
320	0.00252676542966019\\
321	0.00252827551624632\\
322	0.00252981246624239\\
323	0.00253137677233462\\
324	0.00253296893805325\\
325	0.00253458947806148\\
326	0.00253623891841334\\
327	0.00253791779676254\\
328	0.00253962666249777\\
329	0.00254136607677239\\
330	0.00254313661238575\\
331	0.00254493885346014\\
332	0.00254677339483983\\
333	0.00254864084111551\\
334	0.00255054180514683\\
335	0.00255247690591094\\
336	0.00255444676543193\\
337	0.00255645200439617\\
338	0.00255849323565539\\
339	0.00256057105346809\\
340	0.00256268601093812\\
341	0.00256483855260878\\
342	0.00256702871621453\\
343	0.00256925630532907\\
344	0.00257152194378727\\
345	0.00257382626851024\\
346	0.0025761699304349\\
347	0.00257855359553404\\
348	0.00258097794590763\\
349	0.00258344368094375\\
350	0.00258595151874106\\
351	0.00258850219942977\\
352	0.00259109648663455\\
353	0.00259373516443518\\
354	0.00259641903576101\\
355	0.00259914892306405\\
356	0.00260192566865458\\
357	0.00260475013406767\\
358	0.00260762319724591\\
359	0.0026105457471206\\
360	0.00261351868709786\\
361	0.0026165429802981\\
362	0.00261961961851274\\
363	0.00262274962406199\\
364	0.0026259340518677\\
365	0.00262917399177813\\
366	0.00263247057118866\\
367	0.0026358249580127\\
368	0.00263923836406946\\
369	0.00264271204897125\\
370	0.00264624732461443\\
371	0.00264984556040834\\
372	0.00265350818942373\\
373	0.00265723671572618\\
374	0.00266103272332598\\
375	0.00266489788753106\\
376	0.00266883399024505\\
377	0.00267284294219539\\
378	0.00267692681683241\\
379	0.00268108789732648\\
380	0.00268532870158226\\
381	0.00268965181341293\\
382	0.00269406006841038\\
383	0.00269855660434867\\
384	0.00270314492357118\\
385	0.00270782897478867\\
386	0.00271261326797898\\
387	0.00271750305810833\\
388	0.00272250470440484\\
389	0.0027276265667844\\
390	0.00273288190946128\\
391	0.00273827968754821\\
392	0.00274382822816893\\
393	0.00274953677257988\\
394	0.00275541561043726\\
395	0.00276147623771513\\
396	0.00276773154307394\\
397	0.00277419602856395\\
398	0.00278088607189821\\
399	0.00278782023921147\\
400	0.00279501965932281\\
401	0.00280250847313758\\
402	0.00281031437506767\\
403	0.00281846926726797\\
404	0.00282701005185636\\
405	0.00283597958983118\\
406	0.00284542785300111\\
407	0.00285541327511914\\
408	0.00286600432764904\\
409	0.00287728139424408\\
410	0.00288934128852139\\
411	0.00290229948848919\\
412	0.00291629425870046\\
413	0.00293149153708745\\
414	0.00294803052623377\\
415	0.0029648603126777\\
416	0.00298198513718147\\
417	0.00299940923279649\\
418	0.00301713681878248\\
419	0.00303517209573159\\
420	0.00305351924500449\\
421	0.00307218244096976\\
422	0.0030911658988944\\
423	0.00311047401714231\\
424	0.00313011174362107\\
425	0.00315008528389045\\
426	0.00317040135010141\\
427	0.00319106397692754\\
428	0.00321207708620355\\
429	0.00323344446751591\\
430	0.00325516975644509\\
431	0.00327725641013362\\
432	0.00329970767980168\\
433	0.00332252657977064\\
434	0.00334571585248872\\
435	0.00336927792898328\\
436	0.00339321488410088\\
437	0.00341752838589096\\
438	0.00344221963877208\\
439	0.00346728932151394\\
440	0.00349273752717384\\
441	0.00351856374366612\\
442	0.00354476677042555\\
443	0.00357134447724801\\
444	0.00359829358026841\\
445	0.00362560943043688\\
446	0.00365328575182862\\
447	0.00368131429903288\\
448	0.0037096843372865\\
449	0.00373838156662635\\
450	0.00376738468966009\\
451	0.00379664872239489\\
452	0.00382613472175291\\
453	0.00385580328726611\\
454	0.00388560615171628\\
455	0.00391548438820794\\
456	0.00394536663700935\\
457	0.00397516890657617\\
458	0.00400484313749156\\
459	0.00403468393460625\\
460	0.00406469774778422\\
461	0.00409485723653956\\
462	0.00412509802986709\\
463	0.00415534158024108\\
464	0.00418549210258833\\
465	0.00421543283122459\\
466	0.00424502122529108\\
467	0.00427408315193109\\
468	0.00430240566781353\\
469	0.00432972804576949\\
470	0.00435573062331471\\
471	0.00438002101124508\\
472	0.00440211747433547\\
473	0.00442406126361349\\
474	0.00444641432147683\\
475	0.00446918347720507\\
476	0.0044923755831371\\
477	0.00451599754607169\\
478	0.00454005632998642\\
479	0.004564558948784\\
480	0.00458951245540178\\
481	0.00461492393113333\\
482	0.00464080055425519\\
483	0.00466715003796089\\
484	0.00469398055662879\\
485	0.00472130034626765\\
486	0.00474911750556943\\
487	0.00477743919031227\\
488	0.00480627190263308\\
489	0.00483562226005485\\
490	0.00486549691479821\\
491	0.0048959019463123\\
492	0.00492684204823402\\
493	0.00495832198138339\\
494	0.00499034668114623\\
495	0.00502292107957287\\
496	0.00505605489047175\\
497	0.00508976080508257\\
498	0.00512405585633382\\
499	0.00515896350551215\\
500	0.00519451530265017\\
501	0.00523074291961656\\
502	0.00526768924415036\\
503	0.00530541283468658\\
504	0.00534399391726228\\
505	0.00538354245377115\\
506	0.00542420895843863\\
507	0.00546611266267793\\
508	0.00550931551993746\\
509	0.00555387293732511\\
510	0.00559983890393659\\
511	0.00564726579385787\\
512	0.00569622118682937\\
513	0.00574678272157384\\
514	0.00579902281922955\\
515	0.0058530045542137\\
516	0.00590877612135214\\
517	0.00596636353451982\\
518	0.0060257608323764\\
519	0.0060869167639242\\
520	0.00614971762405756\\
521	0.00621396516303915\\
522	0.00627934613861281\\
523	0.00634499542158637\\
524	0.00640940816470904\\
525	0.00647239290823549\\
526	0.00653374883032473\\
527	0.00659326934269996\\
528	0.00665074765908659\\
529	0.00670598518451775\\
530	0.0067588033458027\\
531	0.0068090601230743\\
532	0.00685667403304762\\
533	0.00690166053748132\\
534	0.00694428556288202\\
535	0.00698633836920417\\
536	0.00702781416657799\\
537	0.00706872243385485\\
538	0.00710908957047251\\
539	0.00714896149944049\\
540	0.00718840596287116\\
541	0.0072275141172655\\
542	0.00726640087898807\\
543	0.00730520379680146\\
544	0.0073440783511678\\
545	0.0073831883662304\\
546	0.00742263433169856\\
547	0.0074624700724885\\
548	0.00750274440711923\\
549	0.00754350978707902\\
550	0.00758482129058975\\
551	0.0076267352212644\\
552	0.00766930755017648\\
553	0.00771259113568549\\
554	0.00775662320431986\\
555	0.00780143604039559\\
556	0.00784705550342595\\
557	0.00789350133934143\\
558	0.00794079089526178\\
559	0.00798894090630713\\
560	0.00803796719697048\\
561	0.00808788435778896\\
562	0.00813870540146843\\
563	0.00819044142189942\\
564	0.00824310127831327\\
565	0.00829669132896895\\
566	0.00835121523792377\\
567	0.00840667387413305\\
568	0.00846306536310321\\
569	0.00852038513578629\\
570	0.00857862579519724\\
571	0.00863777707688469\\
572	0.0086978253363676\\
573	0.00875875063967371\\
574	0.00882052518607372\\
575	0.00888311124128023\\
576	0.00894648323118318\\
577	0.00901053857990508\\
578	0.00907531874573529\\
579	0.00914090526783061\\
580	0.009205549681425\\
581	0.00926835372003995\\
582	0.00932896255875889\\
583	0.00938624167980904\\
584	0.00943944829201353\\
585	0.00949040788106645\\
586	0.0095387431908683\\
587	0.00958509003453771\\
588	0.00962947324275186\\
589	0.0096724489502006\\
590	0.00971437214798795\\
591	0.00975535399573538\\
592	0.00979545792277785\\
593	0.0098347066133119\\
594	0.00987312381469506\\
595	0.00991038882774525\\
596	0.00994573390547615\\
597	0.0099771937715668\\
598	0.00999970795535495\\
599	0\\
600	0\\
};
\addplot [color=red!25!mycolor17,solid,forget plot]
  table[row sep=crcr]{%
1	0.00302839379222758\\
2	0.00302840505615055\\
3	0.00302841652450542\\
4	0.00302842820100256\\
5	0.00302844008941975\\
6	0.0030284521936034\\
7	0.0030284645174697\\
8	0.00302847706500599\\
9	0.00302848984027202\\
10	0.00302850284740127\\
11	0.00302851609060222\\
12	0.00302852957415983\\
13	0.00302854330243684\\
14	0.00302855727987524\\
15	0.0030285715109976\\
16	0.00302858600040869\\
17	0.00302860075279686\\
18	0.00302861577293557\\
19	0.00302863106568497\\
20	0.00302864663599342\\
21	0.00302866248889916\\
22	0.0030286786295319\\
23	0.00302869506311444\\
24	0.00302871179496444\\
25	0.00302872883049607\\
26	0.00302874617522179\\
27	0.00302876383475419\\
28	0.00302878181480763\\
29	0.0030288001212003\\
30	0.00302881875985596\\
31	0.00302883773680588\\
32	0.00302885705819082\\
33	0.00302887673026299\\
34	0.0030288967593881\\
35	0.00302891715204735\\
36	0.00302893791483957\\
37	0.00302895905448338\\
38	0.00302898057781929\\
39	0.00302900249181194\\
40	0.00302902480355235\\
41	0.00302904752026022\\
42	0.00302907064928623\\
43	0.00302909419811444\\
44	0.00302911817436465\\
45	0.00302914258579495\\
46	0.00302916744030412\\
47	0.00302919274593427\\
48	0.00302921851087336\\
49	0.00302924474345787\\
50	0.00302927145217546\\
51	0.00302929864566777\\
52	0.00302932633273314\\
53	0.00302935452232944\\
54	0.00302938322357701\\
55	0.00302941244576156\\
56	0.00302944219833715\\
57	0.00302947249092926\\
58	0.00302950333333786\\
59	0.00302953473554059\\
60	0.00302956670769596\\
61	0.00302959926014661\\
62	0.00302963240342263\\
63	0.00302966614824497\\
64	0.00302970050552887\\
65	0.00302973548638735\\
66	0.00302977110213484\\
67	0.00302980736429077\\
68	0.00302984428458324\\
69	0.00302988187495286\\
70	0.00302992014755654\\
71	0.0030299591147714\\
72	0.00302999878919874\\
73	0.0030300391836681\\
74	0.00303008031124134\\
75	0.00303012218521686\\
76	0.00303016481913389\\
77	0.0030302082267768\\
78	0.00303025242217946\\
79	0.00303029741962987\\
80	0.00303034323367461\\
81	0.00303038987912361\\
82	0.0030304373710548\\
83	0.00303048572481903\\
84	0.00303053495604487\\
85	0.00303058508064373\\
86	0.0030306361148149\\
87	0.00303068807505071\\
88	0.00303074097814188\\
89	0.00303079484118279\\
90	0.00303084968157703\\
91	0.00303090551704294\\
92	0.00303096236561924\\
93	0.00303102024567085\\
94	0.00303107917589472\\
95	0.00303113917532584\\
96	0.00303120026334325\\
97	0.0030312624596763\\
98	0.00303132578441093\\
99	0.00303139025799603\\
100	0.00303145590125002\\
101	0.00303152273536749\\
102	0.00303159078192597\\
103	0.00303166006289272\\
104	0.00303173060063185\\
105	0.0030318024179114\\
106	0.00303187553791054\\
107	0.00303194998422709\\
108	0.00303202578088487\\
109	0.00303210295234151\\
110	0.00303218152349608\\
111	0.00303226151969716\\
112	0.00303234296675079\\
113	0.00303242589092876\\
114	0.00303251031897693\\
115	0.00303259627812372\\
116	0.00303268379608879\\
117	0.00303277290109185\\
118	0.00303286362186161\\
119	0.00303295598764493\\
120	0.00303305002821607\\
121	0.00303314577388621\\
122	0.00303324325551295\\
123	0.00303334250451026\\
124	0.00303344355285829\\
125	0.00303354643311366\\
126	0.00303365117841963\\
127	0.00303375782251677\\
128	0.0030338663997535\\
129	0.00303397694509707\\
130	0.00303408949414461\\
131	0.00303420408313441\\
132	0.00303432074895733\\
133	0.00303443952916859\\
134	0.00303456046199954\\
135	0.00303468358636983\\
136	0.00303480894189968\\
137	0.00303493656892239\\
138	0.00303506650849714\\
139	0.00303519880242191\\
140	0.0030353334932467\\
141	0.003035470624287\\
142	0.0030356102396374\\
143	0.00303575238418558\\
144	0.00303589710362637\\
145	0.00303604444447623\\
146	0.00303619445408791\\
147	0.0030363471806654\\
148	0.00303650267327904\\
149	0.00303666098188105\\
150	0.00303682215732119\\
151	0.00303698625136286\\
152	0.00303715331669929\\
153	0.00303732340697017\\
154	0.00303749657677851\\
155	0.00303767288170779\\
156	0.00303785237833944\\
157	0.0030380351242706\\
158	0.00303822117813223\\
159	0.00303841059960749\\
160	0.00303860344945046\\
161	0.00303879978950521\\
162	0.00303899968272522\\
163	0.00303920319319303\\
164	0.00303941038614044\\
165	0.0030396213279688\\
166	0.00303983608626987\\
167	0.00304005472984703\\
168	0.00304027732873667\\
169	0.00304050395423017\\
170	0.00304073467889618\\
171	0.00304096957660327\\
172	0.00304120872254299\\
173	0.00304145219325335\\
174	0.00304170006664276\\
175	0.00304195242201417\\
176	0.00304220934008994\\
177	0.00304247090303695\\
178	0.00304273719449214\\
179	0.00304300829958858\\
180	0.00304328430498196\\
181	0.00304356529887753\\
182	0.0030438513710575\\
183	0.00304414261290895\\
184	0.00304443911745221\\
185	0.00304474097936965\\
186	0.00304504829503519\\
187	0.00304536116254405\\
188	0.00304567968174319\\
189	0.00304600395426218\\
190	0.00304633408354477\\
191	0.00304667017488073\\
192	0.00304701233543851\\
193	0.00304736067429829\\
194	0.00304771530248569\\
195	0.00304807633300599\\
196	0.00304844388087902\\
197	0.00304881806317457\\
198	0.00304919899904846\\
199	0.00304958680977917\\
200	0.00304998161880524\\
201	0.0030503835517631\\
202	0.00305079273652573\\
203	0.0030512093032419\\
204	0.00305163338437605\\
205	0.00305206511474893\\
206	0.00305250463157887\\
207	0.00305295207452379\\
208	0.00305340758572393\\
209	0.00305387130984524\\
210	0.00305434339412361\\
211	0.00305482398840977\\
212	0.00305531324521497\\
213	0.00305581131975746\\
214	0.00305631837000973\\
215	0.00305683455674661\\
216	0.003057360043594\\
217	0.0030578949970787\\
218	0.00305843958667887\\
219	0.00305899398487538\\
220	0.00305955836720408\\
221	0.0030601329123089\\
222	0.0030607178019959\\
223	0.00306131322128813\\
224	0.00306191935848148\\
225	0.00306253640520146\\
226	0.00306316455646089\\
227	0.00306380401071862\\
228	0.00306445496993916\\
229	0.00306511763965336\\
230	0.00306579222902001\\
231	0.00306647895088855\\
232	0.00306717802186287\\
233	0.003067889662366\\
234	0.00306861409670598\\
235	0.00306935155314283\\
236	0.00307010226395657\\
237	0.00307086646551644\\
238	0.0030716443983511\\
239	0.00307243630722022\\
240	0.00307324244118701\\
241	0.00307406305369208\\
242	0.00307489840262847\\
243	0.00307574875041794\\
244	0.00307661436408841\\
245	0.00307749551535274\\
246	0.00307839248068879\\
247	0.00307930554142075\\
248	0.00308023498380176\\
249	0.003081181099098\\
250	0.00308214418367396\\
251	0.00308312453907921\\
252	0.00308412247213652\\
253	0.00308513829503142\\
254	0.00308617232540319\\
255	0.00308722488643737\\
256	0.00308829630695964\\
257	0.00308938692153136\\
258	0.00309049707054654\\
259	0.00309162710033039\\
260	0.00309277736323956\\
261	0.0030939482177638\\
262	0.00309514002862941\\
263	0.00309635316690439\\
264	0.00309758801010511\\
265	0.00309884494230478\\
266	0.00310012435424364\\
267	0.00310142664344084\\
268	0.00310275221430782\\
269	0.00310410147826308\\
270	0.00310547485384705\\
271	0.00310687276683415\\
272	0.00310829565033333\\
273	0.00310974394485335\\
274	0.00311121809827172\\
275	0.00311271856556988\\
276	0.00311424580814105\\
277	0.00311580029304386\\
278	0.0031173824963509\\
279	0.00311899290588784\\
280	0.00312063201780131\\
281	0.00312230033668589\\
282	0.00312399837571202\\
283	0.00312572665675453\\
284	0.00312748571052208\\
285	0.00312927607668701\\
286	0.00313109830401566\\
287	0.00313295295049911\\
288	0.00313484058348401\\
289	0.00313676177980353\\
290	0.00313871712590803\\
291	0.00314070721799548\\
292	0.00314273266214129\\
293	0.00314479407442735\\
294	0.00314689208107\\
295	0.00314902731854667\\
296	0.00315120043372098\\
297	0.00315341208396572\\
298	0.00315566293728366\\
299	0.00315795367242572\\
300	0.00316028497900588\\
301	0.00316265755761281\\
302	0.0031650721199175\\
303	0.0031675293887765\\
304	0.00317003009833068\\
305	0.00317257499409899\\
306	0.00317516483306818\\
307	0.00317780038378063\\
308	0.00318048242642767\\
309	0.00318321175296873\\
310	0.00318598916732919\\
311	0.00318881548580932\\
312	0.00319169153801165\\
313	0.00319461816890939\\
314	0.00319759624290387\\
315	0.00320062664899632\\
316	0.0032037102951036\\
317	0.00320684807037445\\
318	0.00321004087627448\\
319	0.0032132896266077\\
320	0.00321659524753004\\
321	0.00321995867755515\\
322	0.00322338086755313\\
323	0.00322686278074274\\
324	0.00323040539267859\\
325	0.00323400969123477\\
326	0.0032376766765872\\
327	0.00324140736119793\\
328	0.00324520276980519\\
329	0.00324906393942445\\
330	0.00325299191936726\\
331	0.00325698777128638\\
332	0.00326105256925825\\
333	0.0032651873999171\\
334	0.00326939336265826\\
335	0.00327367156993406\\
336	0.00327802314767201\\
337	0.00328244923585473\\
338	0.00328695098931985\\
339	0.00329152957888654\\
340	0.0032961861930979\\
341	0.00330092204169816\\
342	0.00330573836608922\\
343	0.0033106364500463\\
344	0.00331561760769122\\
345	0.0033206831751721\\
346	0.00332583451126905\\
347	0.00333107299801417\\
348	0.00333640004138324\\
349	0.00334181707225653\\
350	0.00334732554826556\\
351	0.00335292695829203\\
352	0.00335862282034505\\
353	0.00336441467612646\\
354	0.00337030409122422\\
355	0.00337629265504074\\
356	0.00338238197997807\\
357	0.00338857369842472\\
358	0.00339486945307576\\
359	0.00340127086753418\\
360	0.00340777946832079\\
361	0.00341439661982143\\
362	0.00342112404744642\\
363	0.00342796349887596\\
364	0.00343491674355327\\
365	0.00344198557200682\\
366	0.00344917179497051\\
367	0.00345647724226561\\
368	0.00346390376140449\\
369	0.00347145321587229\\
370	0.00347912748304138\\
371	0.00348692845167971\\
372	0.00349485801903882\\
373	0.00350291808758462\\
374	0.0035111105616546\\
375	0.00351943734493847\\
376	0.00352790034136671\\
377	0.00353650146654665\\
378	0.00354524268871292\\
379	0.00355412614573979\\
380	0.00356315442598909\\
381	0.00357233094853862\\
382	0.00358165751084972\\
383	0.00359113580335864\\
384	0.00360076739601845\\
385	0.00361055370673163\\
386	0.00362049596272862\\
387	0.00363059515242785\\
388	0.00364085196232736\\
389	0.00365126668323878\\
390	0.0036618390373958\\
391	0.00367256814840234\\
392	0.00368345253156515\\
393	0.00369449000001941\\
394	0.00370567753454716\\
395	0.00371701112855533\\
396	0.00372848560301563\\
397	0.00374009438497698\\
398	0.00375182924177285\\
399	0.00376367996118231\\
400	0.00377563396545737\\
401	0.00378767584415013\\
402	0.00379978678682925\\
403	0.00381194389164227\\
404	0.00382411931833971\\
405	0.00383627924249317\\
406	0.00384838254476245\\
407	0.00386037912688761\\
408	0.00387220770166266\\
409	0.00388379162842935\\
410	0.00389502823215165\\
411	0.0039058053770272\\
412	0.00391598628688462\\
413	0.00392540432342172\\
414	0.00393391703171463\\
415	0.00394256687755733\\
416	0.00395135437222172\\
417	0.00396027978814989\\
418	0.00396934311056532\\
419	0.00397854398002191\\
420	0.00398788162754897\\
421	0.00399735481235831\\
422	0.00400696179881135\\
423	0.00401670049483978\\
424	0.00402656915525881\\
425	0.00403656904293239\\
426	0.00404671444866124\\
427	0.00405703284297276\\
428	0.00406752627570502\\
429	0.00407819676557926\\
430	0.00408904629520965\\
431	0.00410007680565582\\
432	0.00411129019047086\\
433	0.00412268828918791\\
434	0.00413427288017187\\
435	0.0041460456727693\\
436	0.00415800829871378\\
437	0.00417016230271938\\
438	0.00418250913206429\\
439	0.00419505012502878\\
440	0.00420778649786106\\
441	0.00422071932930983\\
442	0.00423384954243987\\
443	0.00424717788626631\\
444	0.00426070491824429\\
445	0.00427443098595149\\
446	0.00428835620833312\\
447	0.00430248045747204\\
448	0.00431680334269483\\
449	0.00433132419952804\\
450	0.00434604209504468\\
451	0.00436095589945023\\
452	0.004376064595495\\
453	0.00439136721676721\\
454	0.00440686288610909\\
455	0.00442255099897275\\
456	0.004438431480289\\
457	0.00445450512939587\\
458	0.00447077370408318\\
459	0.00448723486789741\\
460	0.00450388630057895\\
461	0.00452072524652081\\
462	0.00453774909302376\\
463	0.00455495611894913\\
464	0.00457234607125751\\
465	0.00458992095339337\\
466	0.00460768605450424\\
467	0.00462565133655255\\
468	0.0046438333617516\\
469	0.00466225783770557\\
470	0.00468096298585217\\
471	0.00470000401649422\\
472	0.00471945906677093\\
473	0.00473938767759429\\
474	0.00475981409024856\\
475	0.00478075575623904\\
476	0.00480223100153094\\
477	0.00482425856778606\\
478	0.0048468580439876\\
479	0.00487005002816076\\
480	0.00489385620104787\\
481	0.00491829940194202\\
482	0.00494340368502003\\
483	0.00496919421239836\\
484	0.00499569775861788\\
485	0.00502294511464872\\
486	0.00505096981172665\\
487	0.0050798091614902\\
488	0.00510949778355719\\
489	0.0051400699821511\\
490	0.00517156102005021\\
491	0.00520400707139461\\
492	0.0052374451943808\\
493	0.00527191332472441\\
494	0.00530745031577766\\
495	0.00534409604514044\\
496	0.00538189161588839\\
497	0.00542087969115504\\
498	0.00546110512910798\\
499	0.0055026159982025\\
500	0.00554546521133257\\
501	0.0055897269789056\\
502	0.00563547697008804\\
503	0.00568278516608635\\
504	0.00573171148097553\\
505	0.00578229947832214\\
506	0.00583456775120688\\
507	0.00588849998078259\\
508	0.00594403332644779\\
509	0.0060010412253308\\
510	0.00605931006271971\\
511	0.00611851027633337\\
512	0.00617747791830166\\
513	0.00623520586116127\\
514	0.00629150581262778\\
515	0.00634618044132797\\
516	0.0063990274318587\\
517	0.00644984572907789\\
518	0.00649844476803919\\
519	0.00654465776229582\\
520	0.00658836033287184\\
521	0.00662949619949569\\
522	0.00666811281657877\\
523	0.00670481814336649\\
524	0.00674090081281494\\
525	0.00677635894202173\\
526	0.00681120429792292\\
527	0.00684546463234754\\
528	0.00687918591593404\\
529	0.00691243422022874\\
530	0.00694529688123025\\
531	0.00697788240868995\\
532	0.00701031833327823\\
533	0.00704274576437311\\
534	0.00707530468489793\\
535	0.00710805919650755\\
536	0.00714104422029622\\
537	0.00717429851413554\\
538	0.00720786425663497\\
539	0.00724178641245666\\
540	0.00727611185933979\\
541	0.00731088827640736\\
542	0.00734616282644195\\
543	0.00738198070843317\\
544	0.00741838376052503\\
545	0.00745540942216023\\
546	0.00749309307772818\\
547	0.00753147042725658\\
548	0.00757057761092794\\
549	0.00761045061691577\\
550	0.0076511245205872\\
551	0.00769263289885326\\
552	0.00773499993678999\\
553	0.00777824368278787\\
554	0.00782238079408664\\
555	0.00786742681565397\\
556	0.00791339629805104\\
557	0.00796030292073503\\
558	0.00800815943839331\\
559	0.00805697750780174\\
560	0.00810676749818279\\
561	0.00815753828243189\\
562	0.00820929699943151\\
563	0.0082620488236965\\
564	0.00831579673842313\\
565	0.00837054130588479\\
566	0.00842628054133109\\
567	0.00848300995809075\\
568	0.00854071929184885\\
569	0.00859939067377092\\
570	0.00865899639130592\\
571	0.00871947851713435\\
572	0.00878076518693766\\
573	0.00884291437470965\\
574	0.00890600938256045\\
575	0.00897017009872411\\
576	0.0090329788948814\\
577	0.00909416288432404\\
578	0.00915329402632274\\
579	0.00920930753183795\\
580	0.00926288552407399\\
581	0.00931447511044903\\
582	0.00936389576540479\\
583	0.00941168254089778\\
584	0.00945797801032644\\
585	0.00950288536810165\\
586	0.00954684401204475\\
587	0.00959001465926593\\
588	0.009632483040377\\
589	0.00967428778528623\\
590	0.00971542443559021\\
591	0.00975588428269145\\
592	0.00979566256029117\\
593	0.0098347658551079\\
594	0.00987312984006299\\
595	0.00991038882774525\\
596	0.00994573390547615\\
597	0.0099771937715668\\
598	0.00999970795535495\\
599	0\\
600	0\\
};
\addplot [color=mycolor19,solid,forget plot]
  table[row sep=crcr]{%
1	0.00370774725980159\\
2	0.00370775285074728\\
3	0.00370775854349141\\
4	0.0037077643398901\\
5	0.00370777024183357\\
6	0.00370777625124661\\
7	0.00370778237008943\\
8	0.00370778860035815\\
9	0.00370779494408551\\
10	0.00370780140334163\\
11	0.00370780798023459\\
12	0.00370781467691122\\
13	0.00370782149555769\\
14	0.00370782843840041\\
15	0.00370783550770665\\
16	0.00370784270578533\\
17	0.00370785003498786\\
18	0.00370785749770877\\
19	0.00370786509638668\\
20	0.00370787283350505\\
21	0.00370788071159295\\
22	0.00370788873322599\\
23	0.00370789690102714\\
24	0.00370790521766767\\
25	0.00370791368586793\\
26	0.00370792230839837\\
27	0.00370793108808039\\
28	0.00370794002778736\\
29	0.00370794913044552\\
30	0.00370795839903501\\
31	0.00370796783659081\\
32	0.0037079774462038\\
33	0.00370798723102183\\
34	0.00370799719425068\\
35	0.00370800733915524\\
36	0.00370801766906051\\
37	0.00370802818735279\\
38	0.00370803889748079\\
39	0.00370804980295677\\
40	0.00370806090735776\\
41	0.00370807221432673\\
42	0.00370808372757378\\
43	0.00370809545087748\\
44	0.00370810738808608\\
45	0.00370811954311878\\
46	0.0037081319199671\\
47	0.00370814452269614\\
48	0.00370815735544609\\
49	0.00370817042243348\\
50	0.00370818372795265\\
51	0.0037081972763772\\
52	0.00370821107216143\\
53	0.00370822511984191\\
54	0.0037082394240389\\
55	0.003708253989458\\
56	0.00370826882089166\\
57	0.0037082839232208\\
58	0.00370829930141651\\
59	0.00370831496054165\\
60	0.0037083309057526\\
61	0.00370834714230094\\
62	0.00370836367553529\\
63	0.00370838051090304\\
64	0.0037083976539522\\
65	0.00370841511033331\\
66	0.00370843288580124\\
67	0.00370845098621722\\
68	0.00370846941755074\\
69	0.00370848818588162\\
70	0.00370850729740203\\
71	0.0037085267584185\\
72	0.00370854657535414\\
73	0.00370856675475073\\
74	0.00370858730327099\\
75	0.00370860822770072\\
76	0.00370862953495113\\
77	0.00370865123206115\\
78	0.00370867332619983\\
79	0.00370869582466869\\
80	0.00370871873490419\\
81	0.00370874206448024\\
82	0.00370876582111075\\
83	0.00370879001265215\\
84	0.00370881464710616\\
85	0.00370883973262228\\
86	0.00370886527750067\\
87	0.00370889129019495\\
88	0.00370891777931489\\
89	0.00370894475362944\\
90	0.00370897222206957\\
91	0.00370900019373133\\
92	0.00370902867787887\\
93	0.00370905768394751\\
94	0.00370908722154696\\
95	0.00370911730046449\\
96	0.00370914793066823\\
97	0.0037091791223105\\
98	0.00370921088573119\\
99	0.00370924323146131\\
100	0.00370927617022635\\
101	0.00370930971294998\\
102	0.00370934387075767\\
103	0.00370937865498044\\
104	0.00370941407715861\\
105	0.00370945014904564\\
106	0.00370948688261215\\
107	0.00370952429004976\\
108	0.00370956238377534\\
109	0.003709601176435\\
110	0.0037096406809084\\
111	0.00370968091031304\\
112	0.00370972187800857\\
113	0.00370976359760137\\
114	0.00370980608294895\\
115	0.00370984934816467\\
116	0.00370989340762235\\
117	0.00370993827596115\\
118	0.00370998396809042\\
119	0.00371003049919463\\
120	0.00371007788473846\\
121	0.0037101261404719\\
122	0.00371017528243558\\
123	0.00371022532696602\\
124	0.00371027629070111\\
125	0.00371032819058558\\
126	0.00371038104387678\\
127	0.00371043486815016\\
128	0.00371048968130542\\
129	0.00371054550157219\\
130	0.00371060234751623\\
131	0.00371066023804553\\
132	0.00371071919241662\\
133	0.00371077923024095\\
134	0.00371084037149136\\
135	0.00371090263650874\\
136	0.00371096604600877\\
137	0.00371103062108875\\
138	0.00371109638323462\\
139	0.00371116335432808\\
140	0.0037112315566538\\
141	0.00371130101290682\\
142	0.00371137174620002\\
143	0.00371144378007182\\
144	0.00371151713849395\\
145	0.00371159184587932\\
146	0.00371166792709015\\
147	0.00371174540744613\\
148	0.00371182431273281\\
149	0.00371190466921006\\
150	0.00371198650362085\\
151	0.0037120698431999\\
152	0.00371215471568282\\
153	0.0037122411493151\\
154	0.00371232917286155\\
155	0.00371241881561567\\
156	0.00371251010740934\\
157	0.00371260307862261\\
158	0.00371269776019371\\
159	0.00371279418362919\\
160	0.00371289238101432\\
161	0.00371299238502354\\
162	0.00371309422893128\\
163	0.00371319794662282\\
164	0.00371330357260537\\
165	0.00371341114201949\\
166	0.00371352069065049\\
167	0.00371363225494017\\
168	0.00371374587199876\\
169	0.00371386157961711\\
170	0.0037139794162789\\
171	0.00371409942117336\\
172	0.00371422163420798\\
173	0.0037143460960215\\
174	0.00371447284799725\\
175	0.00371460193227658\\
176	0.0037147333917726\\
177	0.00371486727018411\\
178	0.00371500361200984\\
179	0.00371514246256291\\
180	0.00371528386798552\\
181	0.00371542787526392\\
182	0.00371557453224369\\
183	0.00371572388764516\\
184	0.00371587599107926\\
185	0.00371603089306352\\
186	0.00371618864503836\\
187	0.00371634929938376\\
188	0.00371651290943616\\
189	0.00371667952950557\\
190	0.0037168492148931\\
191	0.00371702202190877\\
192	0.00371719800788955\\
193	0.00371737723121779\\
194	0.00371755975133997\\
195	0.00371774562878565\\
196	0.00371793492518691\\
197	0.00371812770329801\\
198	0.00371832402701543\\
199	0.00371852396139824\\
200	0.00371872757268879\\
201	0.00371893492833376\\
202	0.00371914609700563\\
203	0.00371936114862442\\
204	0.00371958015437981\\
205	0.00371980318675368\\
206	0.00372003031954299\\
207	0.00372026162788308\\
208	0.00372049718827116\\
209	0.00372073707859057\\
210	0.00372098137813506\\
211	0.00372123016763367\\
212	0.003721483529276\\
213	0.0037217415467378\\
214	0.00372200430520705\\
215	0.0037222718914105\\
216	0.00372254439364055\\
217	0.00372282190178258\\
218	0.00372310450734271\\
219	0.00372339230347611\\
220	0.00372368538501556\\
221	0.00372398384850068\\
222	0.0037242877922074\\
223	0.00372459731617809\\
224	0.00372491252225198\\
225	0.00372523351409626\\
226	0.00372556039723741\\
227	0.00372589327909318\\
228	0.00372623226900502\\
229	0.00372657747827098\\
230	0.00372692902017913\\
231	0.00372728701004149\\
232	0.00372765156522834\\
233	0.00372802280520326\\
234	0.00372840085155847\\
235	0.00372878582805087\\
236	0.00372917786063841\\
237	0.00372957707751717\\
238	0.00372998360915887\\
239	0.00373039758834892\\
240	0.00373081915022508\\
241	0.00373124843231657\\
242	0.00373168557458385\\
243	0.00373213071945883\\
244	0.00373258401188581\\
245	0.0037330455993628\\
246	0.00373351563198372\\
247	0.00373399426248091\\
248	0.00373448164626847\\
249	0.00373497794148611\\
250	0.0037354833090438\\
251	0.00373599791266707\\
252	0.00373652191894298\\
253	0.00373705549736696\\
254	0.00373759882039051\\
255	0.00373815206346966\\
256	0.00373871540511456\\
257	0.00373928902693999\\
258	0.00373987311371715\\
259	0.00374046785342664\\
260	0.00374107343731284\\
261	0.00374169005993994\\
262	0.00374231791924943\\
263	0.00374295721661959\\
264	0.00374360815692694\\
265	0.00374427094860952\\
266	0.00374494580373236\\
267	0.0037456329380547\\
268	0.00374633257109861\\
269	0.00374704492621789\\
270	0.00374777023066485\\
271	0.00374850871564893\\
272	0.00374926061637186\\
273	0.00375002617199477\\
274	0.00375080562540618\\
275	0.00375159922239119\\
276	0.00375240720900068\\
277	0.00375322982386502\\
278	0.00375406728101007\\
279	0.00375491980008241\\
280	0.00375578764372609\\
281	0.00375667107882098\\
282	0.00375757037653577\\
283	0.00375848581238061\\
284	0.00375941766625949\\
285	0.00376036622252202\\
286	0.00376133177001474\\
287	0.0037623146021315\\
288	0.00376331501686291\\
289	0.00376433331684438\\
290	0.00376536980940285\\
291	0.00376642480660149\\
292	0.00376749862528214\\
293	0.00376859158710497\\
294	0.00376970401858495\\
295	0.0037708362511243\\
296	0.00377198862104032\\
297	0.0037731614695878\\
298	0.00377435514297501\\
299	0.00377556999237241\\
300	0.00377680637391298\\
301	0.00377806464868323\\
302	0.00377934518270398\\
303	0.00378064834690048\\
304	0.00378197451706164\\
305	0.00378332407378998\\
306	0.0037846974024458\\
307	0.00378609489309412\\
308	0.00378751694047521\\
309	0.00378896394404684\\
310	0.00379043630822041\\
311	0.00379193444310449\\
312	0.00379345876658225\\
313	0.00379500970989577\\
314	0.00379658773227283\\
315	0.00379819335722322\\
316	0.00379982724749577\\
317	0.00380149022344091\\
318	0.00380318277068881\\
319	0.00380490538161127\\
320	0.0038066585553709\\
321	0.00380844279797141\\
322	0.00381025862230934\\
323	0.00381210654822796\\
324	0.00381398710257377\\
325	0.00381590081925616\\
326	0.00381784823931141\\
327	0.00381982991097145\\
328	0.00382184638973867\\
329	0.00382389823846755\\
330	0.00382598602745455\\
331	0.00382811033453736\\
332	0.00383027174520502\\
333	0.00383247085272003\\
334	0.00383470825825402\\
335	0.00383698457103804\\
336	0.00383930040852735\\
337	0.00384165639657982\\
338	0.00384405316964468\\
339	0.00384649137096169\\
340	0.00384897165279837\\
341	0.00385149467681659\\
342	0.00385406111430279\\
343	0.00385667164583049\\
344	0.00385932696127703\\
345	0.00386202776021929\\
346	0.00386477475244102\\
347	0.00386756865858514\\
348	0.00387041021099452\\
349	0.00387330015479153\\
350	0.00387623924923149\\
351	0.00387922826935019\\
352	0.00388226800808929\\
353	0.00388535927917676\\
354	0.00388850292087259\\
355	0.00389169980062951\\
356	0.00389495082056812\\
357	0.00389825692272083\\
358	0.00390161908920078\\
359	0.00390503831660652\\
360	0.00390851547084551\\
361	0.00391205052607056\\
362	0.00391564062560171\\
363	0.0039192862141833\\
364	0.00392298769992046\\
365	0.00392674544836104\\
366	0.00393055977558642\\
367	0.00393443094010376\\
368	0.00393835913328275\\
369	0.00394234446801822\\
370	0.00394638696522411\\
371	0.00395048653767655\\
372	0.00395464297063303\\
373	0.00395885589860197\\
374	0.00396312477775013\\
375	0.00396744885411017\\
376	0.00397182713018891\\
377	0.0039762583404722\\
378	0.00398074097125978\\
379	0.00398527344009837\\
380	0.00398985481702349\\
381	0.00399448744356757\\
382	0.00399918702415652\\
383	0.00400395374011795\\
384	0.00400878755654688\\
385	0.00401368833832755\\
386	0.0040186558402524\\
387	0.00402368969664548\\
388	0.00402878941069664\\
389	0.00403395434399141\\
390	0.0040391837075818\\
391	0.00404447655377464\\
392	0.00404983176584848\\
393	0.00405524804770437\\
394	0.00406072391450447\\
395	0.00406625768498511\\
396	0.00407184747637465\\
397	0.00407749120316849\\
398	0.00408318658143646\\
399	0.00408893114089407\\
400	0.00409472224769767\\
401	0.00410055714188188\\
402	0.00410643299461499\\
403	0.00411234699210283\\
404	0.00411829645515665\\
405	0.00412427900635643\\
406	0.00413029280069661\\
407	0.00413633684087422\\
408	0.00414241140317777\\
409	0.00414851861368006\\
410	0.00415466324757146\\
411	0.00416085389820221\\
412	0.0041671041987995\\
413	0.00417343462776469\\
414	0.00417987382238871\\
415	0.0041864383085485\\
416	0.004193130669954\\
417	0.00419995356930276\\
418	0.00420690975524765\\
419	0.00421400207096304\\
420	0.00422123346479872\\
421	0.00422860700361297\\
422	0.00423612588937687\\
423	0.00424379347919162\\
424	0.0042516133069473\\
425	0.00425958909775831\\
426	0.00426772474066566\\
427	0.00427602409957658\\
428	0.00428449080175241\\
429	0.00429312858546607\\
430	0.00430194130554165\\
431	0.00431093293928129\\
432	0.00432010759281875\\
433	0.00432946950794833\\
434	0.00433902306946049\\
435	0.00434877281287252\\
436	0.00435872343253877\\
437	0.00436887979053711\\
438	0.00437924692684946\\
439	0.00438983006992397\\
440	0.0044006346481212\\
441	0.00441166630217564\\
442	0.00442293089885492\\
443	0.00443443454599071\\
444	0.00444618360914368\\
445	0.00445818473027138\\
446	0.00447044484713513\\
447	0.00448297121265312\\
448	0.00449577141647535\\
449	0.00450885341594247\\
450	0.00452222556599645\\
451	0.00453589665078687\\
452	0.00454987591430312\\
453	0.00456417309517411\\
454	0.0045787984650985\\
455	0.00459376287157062\\
456	0.00460907779050298\\
457	0.00462475533350661\\
458	0.00464080825653633\\
459	0.0046572500923446\\
460	0.00467409523580149\\
461	0.00469135902580906\\
462	0.00470905782699198\\
463	0.00472720910829467\\
464	0.00474583151982263\\
465	0.004764945007161\\
466	0.00478457105546634\\
467	0.00480473250036142\\
468	0.00482545305244929\\
469	0.00484675749753368\\
470	0.00486867165465498\\
471	0.00489122204384054\\
472	0.00491443527086963\\
473	0.00493833810209281\\
474	0.0049629581911663\\
475	0.00498832557828856\\
476	0.00501447391319709\\
477	0.00504144009062676\\
478	0.00506925741142093\\
479	0.00509795886157181\\
480	0.00512757795960136\\
481	0.00515814867236376\\
482	0.00518970535665398\\
483	0.00522228275738427\\
484	0.00525591610606085\\
485	0.00529064133492093\\
486	0.00532649549571177\\
487	0.00536351754362962\\
488	0.00540174964165694\\
489	0.00544124498503406\\
490	0.00548206991791641\\
491	0.00552428741703136\\
492	0.00556795394712316\\
493	0.00561311510864539\\
494	0.0056597999935744\\
495	0.00570801366404955\\
496	0.00575772704840013\\
497	0.00580886355070153\\
498	0.00586128135389376\\
499	0.00591475005480695\\
500	0.00596891965519479\\
501	0.00602224078756516\\
502	0.00607430541547943\\
503	0.00612493655162792\\
504	0.00617394793527647\\
505	0.00622115025590455\\
506	0.00626635772595491\\
507	0.00630939751224373\\
508	0.00635012319693875\\
509	0.0063884336455997\\
510	0.006424299242059\\
511	0.00645779751263656\\
512	0.00648985887878153\\
513	0.0065212909655835\\
514	0.00655209447491636\\
515	0.00658228317133356\\
516	0.00661188597703505\\
517	0.0066409489001781\\
518	0.00666953653677782\\
519	0.00669773275867687\\
520	0.00672564002782952\\
521	0.00675337654140902\\
522	0.00678107006430719\\
523	0.0068088302391249\\
524	0.00683670581356637\\
525	0.00686472515311536\\
526	0.00689291942367655\\
527	0.0069213221216492\\
528	0.00694996841419969\\
529	0.00697889428131792\\
530	0.00700813547176763\\
531	0.00703772631932867\\
532	0.00706769852177207\\
533	0.00709808007400404\\
534	0.0071288948681267\\
535	0.00716016583566274\\
536	0.00719191687720358\\
537	0.00722417277820445\\
538	0.00725695912297007\\
539	0.0072903022146856\\
540	0.00732422901093675\\
541	0.0073587670848767\\
542	0.00739394462109925\\
543	0.00742979045103049\\
544	0.0074663341223274\\
545	0.00750360597705983\\
546	0.00754163706134009\\
547	0.00758045881103562\\
548	0.00762010257783498\\
549	0.00766059897626249\\
550	0.00770197741878033\\
551	0.00774425545375649\\
552	0.0077874484398888\\
553	0.00783157106649422\\
554	0.00787663735344774\\
555	0.00792266052771767\\
556	0.00796965286556778\\
557	0.00801762549231267\\
558	0.00806658813049182\\
559	0.00811654885776477\\
560	0.00816751384335804\\
561	0.0082194870649378\\
562	0.00827247057381776\\
563	0.0083264621733809\\
564	0.00838145312782495\\
565	0.0084374259061988\\
566	0.00849434126395024\\
567	0.00855209945312733\\
568	0.00861074676539497\\
569	0.00867036466855077\\
570	0.0087310632366218\\
571	0.00879301394938548\\
572	0.00885417706724721\\
573	0.00891386297593653\\
574	0.00897164060929281\\
575	0.00902671138084478\\
576	0.00908054156668914\\
577	0.00913262846411269\\
578	0.00918281655212943\\
579	0.0092315474226628\\
580	0.00927941704386785\\
581	0.00932608920321522\\
582	0.00937167546725341\\
583	0.00941645926763461\\
584	0.00946071491381077\\
585	0.0095044908654121\\
586	0.00954778452593938\\
587	0.0095905700628325\\
588	0.00963281167868831\\
589	0.00967447054637906\\
590	0.00971551367588456\\
591	0.00975591786564725\\
592	0.00979567177304212\\
593	0.00983476678232329\\
594	0.00987312984006299\\
595	0.00991038882774525\\
596	0.00994573390547615\\
597	0.0099771937715668\\
598	0.00999970795535495\\
599	0\\
600	0\\
};
\addplot [color=red!50!mycolor17,solid,forget plot]
  table[row sep=crcr]{%
1	0.00389736634792342\\
2	0.0038973698205401\\
3	0.00389737335663963\\
4	0.0038973769573813\\
5	0.00389738062394577\\
6	0.0038973843575356\\
7	0.00389738815937552\\
8	0.00389739203071294\\
9	0.00389739597281837\\
10	0.0038973999869858\\
11	0.00389740407453324\\
12	0.00389740823680307\\
13	0.00389741247516264\\
14	0.00389741679100462\\
15	0.00389742118574756\\
16	0.00389742566083632\\
17	0.00389743021774259\\
18	0.00389743485796549\\
19	0.003897439583032\\
20	0.00389744439449745\\
21	0.00389744929394619\\
22	0.00389745428299206\\
23	0.003897459363279\\
24	0.00389746453648154\\
25	0.00389746980430549\\
26	0.00389747516848852\\
27	0.00389748063080071\\
28	0.00389748619304524\\
29	0.00389749185705896\\
30	0.00389749762471306\\
31	0.00389750349791381\\
32	0.0038975094786031\\
33	0.0038975155687592\\
34	0.00389752177039746\\
35	0.00389752808557101\\
36	0.0038975345163715\\
37	0.0038975410649298\\
38	0.00389754773341678\\
39	0.00389755452404413\\
40	0.00389756143906508\\
41	0.00389756848077517\\
42	0.00389757565151321\\
43	0.00389758295366194\\
44	0.00389759038964901\\
45	0.00389759796194776\\
46	0.00389760567307816\\
47	0.00389761352560769\\
48	0.00389762152215224\\
49	0.00389762966537704\\
50	0.00389763795799768\\
51	0.00389764640278098\\
52	0.00389765500254609\\
53	0.0038976637601654\\
54	0.0038976726785656\\
55	0.0038976817607288\\
56	0.00389769100969351\\
57	0.00389770042855577\\
58	0.00389771002047026\\
59	0.00389771978865146\\
60	0.0038977297363747\\
61	0.0038977398669775\\
62	0.00389775018386061\\
63	0.00389776069048938\\
64	0.0038977713903949\\
65	0.00389778228717531\\
66	0.00389779338449712\\
67	0.00389780468609649\\
68	0.00389781619578063\\
69	0.0038978279174291\\
70	0.00389783985499525\\
71	0.00389785201250772\\
72	0.00389786439407172\\
73	0.00389787700387069\\
74	0.00389788984616773\\
75	0.00389790292530709\\
76	0.0038979162457159\\
77	0.00389792981190555\\
78	0.00389794362847351\\
79	0.00389795770010493\\
80	0.0038979720315743\\
81	0.00389798662774719\\
82	0.00389800149358205\\
83	0.00389801663413194\\
84	0.00389803205454643\\
85	0.00389804776007337\\
86	0.00389806375606092\\
87	0.00389808004795931\\
88	0.00389809664132298\\
89	0.00389811354181252\\
90	0.00389813075519668\\
91	0.0038981482873545\\
92	0.0038981661442774\\
93	0.00389818433207144\\
94	0.00389820285695938\\
95	0.00389822172528308\\
96	0.00389824094350567\\
97	0.00389826051821396\\
98	0.00389828045612083\\
99	0.00389830076406753\\
100	0.00389832144902633\\
101	0.00389834251810289\\
102	0.00389836397853896\\
103	0.00389838583771488\\
104	0.00389840810315226\\
105	0.00389843078251676\\
106	0.00389845388362081\\
107	0.00389847741442639\\
108	0.00389850138304799\\
109	0.00389852579775546\\
110	0.00389855066697705\\
111	0.00389857599930235\\
112	0.00389860180348553\\
113	0.00389862808844831\\
114	0.00389865486328337\\
115	0.00389868213725741\\
116	0.0038987099198147\\
117	0.00389873822058033\\
118	0.00389876704936367\\
119	0.00389879641616201\\
120	0.00389882633116396\\
121	0.00389885680475334\\
122	0.00389888784751269\\
123	0.00389891947022726\\
124	0.0038989516838887\\
125	0.00389898449969912\\
126	0.00389901792907507\\
127	0.00389905198365164\\
128	0.00389908667528656\\
129	0.00389912201606456\\
130	0.00389915801830162\\
131	0.00389919469454937\\
132	0.00389923205759959\\
133	0.00389927012048882\\
134	0.00389930889650297\\
135	0.00389934839918209\\
136	0.00389938864232519\\
137	0.00389942963999514\\
138	0.00389947140652378\\
139	0.00389951395651692\\
140	0.0038995573048596\\
141	0.00389960146672138\\
142	0.00389964645756178\\
143	0.00389969229313573\\
144	0.00389973898949917\\
145	0.00389978656301486\\
146	0.00389983503035809\\
147	0.00389988440852261\\
148	0.00389993471482679\\
149	0.00389998596691965\\
150	0.00390003818278713\\
151	0.00390009138075858\\
152	0.00390014557951309\\
153	0.00390020079808633\\
154	0.00390025705587712\\
155	0.00390031437265439\\
156	0.00390037276856414\\
157	0.00390043226413663\\
158	0.00390049288029356\\
159	0.0039005546383556\\
160	0.00390061756004982\\
161	0.00390068166751742\\
162	0.00390074698332155\\
163	0.0039008135304553\\
164	0.00390088133234975\\
165	0.00390095041288234\\
166	0.00390102079638526\\
167	0.003901092507654\\
168	0.00390116557195617\\
169	0.00390124001504028\\
170	0.00390131586314495\\
171	0.0039013931430081\\
172	0.00390147188187634\\
173	0.00390155210751463\\
174	0.00390163384821593\\
175	0.00390171713281131\\
176	0.00390180199067997\\
177	0.00390188845175957\\
178	0.00390197654655685\\
179	0.00390206630615819\\
180	0.00390215776224066\\
181	0.00390225094708309\\
182	0.00390234589357738\\
183	0.00390244263524002\\
184	0.00390254120622388\\
185	0.00390264164133013\\
186	0.00390274397602053\\
187	0.00390284824642968\\
188	0.00390295448937779\\
189	0.0039030627423836\\
190	0.00390317304367732\\
191	0.00390328543221417\\
192	0.00390339994768793\\
193	0.00390351663054479\\
194	0.00390363552199747\\
195	0.00390375666403967\\
196	0.00390388009946066\\
197	0.00390400587186024\\
198	0.0039041340256639\\
199	0.00390426460613837\\
200	0.00390439765940735\\
201	0.00390453323246763\\
202	0.00390467137320542\\
203	0.00390481213041306\\
204	0.00390495555380594\\
205	0.00390510169403992\\
206	0.00390525060272893\\
207	0.00390540233246284\\
208	0.00390555693682597\\
209	0.00390571447041555\\
210	0.00390587498886083\\
211	0.00390603854884248\\
212	0.00390620520811224\\
213	0.00390637502551312\\
214	0.00390654806099997\\
215	0.00390672437566022\\
216	0.00390690403173535\\
217	0.00390708709264261\\
218	0.00390727362299718\\
219	0.00390746368863474\\
220	0.00390765735663467\\
221	0.00390785469534347\\
222	0.0039080557743988\\
223	0.00390826066475405\\
224	0.0039084694387034\\
225	0.00390868216990731\\
226	0.00390889893341869\\
227	0.00390911980570959\\
228	0.00390934486469853\\
229	0.0039095741897783\\
230	0.00390980786184463\\
231	0.00391004596332535\\
232	0.00391028857821035\\
233	0.00391053579208221\\
234	0.00391078769214775\\
235	0.00391104436727017\\
236	0.0039113059080023\\
237	0.00391157240662061\\
238	0.00391184395716029\\
239	0.00391212065545136\\
240	0.0039124025991559\\
241	0.00391268988780645\\
242	0.00391298262284592\\
243	0.00391328090766864\\
244	0.00391358484766326\\
245	0.00391389455025719\\
246	0.00391421012496301\\
247	0.00391453168342704\\
248	0.00391485933948022\\
249	0.0039151932091917\\
250	0.00391553341092541\\
251	0.00391588006540011\\
252	0.00391623329575343\\
253	0.00391659322761038\\
254	0.00391695998915727\\
255	0.00391733371122172\\
256	0.00391771452736018\\
257	0.00391810257395385\\
258	0.00391849799031516\\
259	0.00391890091880645\\
260	0.00391931150497363\\
261	0.00391972989769786\\
262	0.00392015624936923\\
263	0.00392059071608725\\
264	0.0039210334578942\\
265	0.00392148463904928\\
266	0.00392194442835299\\
267	0.00392241299953364\\
268	0.00392289053171176\\
269	0.00392337720996081\\
270	0.00392387322598818\\
271	0.0039243787789641\\
272	0.00392489407652749\\
273	0.00392541933598408\\
274	0.0039259547856397\\
275	0.00392650066591704\\
276	0.00392705722871356\\
277	0.0039276247283476\\
278	0.00392820337258427\\
279	0.00392879305377505\\
280	0.00392939356584512\\
281	0.0039300050948801\\
282	0.00393062782927172\\
283	0.00393126195966689\\
284	0.00393190767890577\\
285	0.00393256518194811\\
286	0.00393323466578536\\
287	0.00393391632933697\\
288	0.0039346103733282\\
289	0.00393531700014664\\
290	0.00393603641367378\\
291	0.00393676881908754\\
292	0.00393751442263075\\
293	0.00393827343133943\\
294	0.00393904605272357\\
295	0.00393983249439113\\
296	0.00394063296360454\\
297	0.00394144766675594\\
298	0.00394227680874452\\
299	0.00394312059223514\\
300	0.00394397921677281\\
301	0.00394485287772095\\
302	0.0039457417649836\\
303	0.00394664606146187\\
304	0.0039475659411825\\
305	0.00394850156702071\\
306	0.00394945308792055\\
307	0.00395042063549458\\
308	0.00395140431986339\\
309	0.00395240422458772\\
310	0.00395342040059171\\
311	0.00395445285919665\\
312	0.0039555015651512\\
313	0.0039565664330002\\
314	0.00395764733789326\\
315	0.00395874417703582\\
316	0.00395985710401096\\
317	0.00396098738962125\\
318	0.00396213886171391\\
319	0.00396331190676099\\
320	0.00396450691791733\\
321	0.00396572429512668\\
322	0.00396696444522947\\
323	0.00396822778207173\\
324	0.00396951472661552\\
325	0.00397082570705071\\
326	0.00397216115890792\\
327	0.00397352152517256\\
328	0.00397490725639991\\
329	0.00397631881083108\\
330	0.00397775665450973\\
331	0.0039792212613991\\
332	0.00398071311349925\\
333	0.00398223270096437\\
334	0.00398378052221951\\
335	0.00398535708407654\\
336	0.00398696290184869\\
337	0.00398859849946344\\
338	0.00399026440957306\\
339	0.00399196117366264\\
340	0.003993689342155\\
341	0.00399544947451154\\
342	0.00399724213932856\\
343	0.00399906791443525\\
344	0.0040009273869971\\
345	0.00400282115361412\\
346	0.00400474982041147\\
347	0.00400671400311934\\
348	0.00400871432713766\\
349	0.00401075142757921\\
350	0.00401282594928307\\
351	0.00401493854678837\\
352	0.00401708988425414\\
353	0.00401928063530098\\
354	0.00402151148274127\\
355	0.00402378311815706\\
356	0.00402609624126987\\
357	0.0040284515590339\\
358	0.00403084978438118\\
359	0.0040332916346042\\
360	0.00403577782969322\\
361	0.00403830909258915\\
362	0.00404088616161536\\
363	0.00404350983386068\\
364	0.00404618092083107\\
365	0.00404890024897001\\
366	0.0040516686602516\\
367	0.00405448701286639\\
368	0.00405735618202696\\
369	0.00406027706092961\\
370	0.00406325056192143\\
371	0.00406627761793908\\
372	0.00406935918430876\\
373	0.00407249624102802\\
374	0.00407568979568942\\
375	0.0040789408872557\\
376	0.00408225059094776\\
377	0.00408562002453609\\
378	0.00408905035623915\\
379	0.00409254281392852\\
380	0.00409609869342913\\
381	0.00409971935696407\\
382	0.00410340618832885\\
383	0.00410716038661171\\
384	0.00411098317744976\\
385	0.00411487581673965\\
386	0.00411883959294418\\
387	0.004122875829743\\
388	0.00412698588907731\\
389	0.00413117117464473\\
390	0.00413543313586004\\
391	0.00413977327232583\\
392	0.00414419313893618\\
393	0.00414869435171167\\
394	0.00415327859445866\\
395	0.00415794762635095\\
396	0.00416270329053206\\
397	0.00416754752383153\\
398	0.00417248236767648\\
399	0.0041775099802546\\
400	0.00418263264993889\\
401	0.00418785280991241\\
402	0.00419317305381311\\
403	0.0041985961520556\\
404	0.00420412506823816\\
405	0.00420976297468375\\
406	0.00421551326565563\\
407	0.00422137956604833\\
408	0.00422736573238066\\
409	0.00423347584149412\\
410	0.00423971415986722\\
411	0.00424608508279133\\
412	0.00425259303372716\\
413	0.00425924229985253\\
414	0.00426603677859521\\
415	0.00427298008943125\\
416	0.00428007596905633\\
417	0.00428732827654689\\
418	0.00429474099877965\\
419	0.00430231825611065\\
420	0.0043100643083079\\
421	0.00431798356070496\\
422	0.00432608057051945\\
423	0.00433436005331877\\
424	0.004342826889352\\
425	0.00435148612981159\\
426	0.00436034300384433\\
427	0.00436940292925008\\
428	0.00437867152652871\\
429	0.00438815462916805\\
430	0.00439785829455999\\
431	0.00440778881559047\\
432	0.00441795273296464\\
433	0.00442835684837378\\
434	0.00443900823873352\\
435	0.00444991427180255\\
436	0.00446108262154964\\
437	0.0044725212817723\\
438	0.00448423858094976\\
439	0.0044962432057614\\
440	0.00450854422092761\\
441	0.00452115109014408\\
442	0.00453407369812363\\
443	0.00454732237379214\\
444	0.00456090791487675\\
445	0.00457484161525172\\
446	0.00458913530067627\\
447	0.00460380135787383\\
448	0.00461885273974556\\
449	0.00463430295530163\\
450	0.004650166150056\\
451	0.00466645714472242\\
452	0.00468319147353427\\
453	0.00470038543059159\\
454	0.00471805611894368\\
455	0.0047362215021036\\
456	0.00475490047618871\\
457	0.00477411317125081\\
458	0.00479388076546922\\
459	0.00481422512485293\\
460	0.00483516901356313\\
461	0.00485673633658239\\
462	0.00487895217184936\\
463	0.00490184277702585\\
464	0.00492543555181921\\
465	0.00494975875678549\\
466	0.00497484138331397\\
467	0.00500071589192295\\
468	0.00502741721143315\\
469	0.00505497690788504\\
470	0.00508342542731814\\
471	0.00511279344080281\\
472	0.00514311193797709\\
473	0.00517441246313275\\
474	0.00520672755744614\\
475	0.00524009148975446\\
476	0.0052745414091518\\
477	0.0053101217940235\\
478	0.00534689398902134\\
479	0.00538491714432695\\
480	0.00542424577839099\\
481	0.00546492646755991\\
482	0.0055069932302287\\
483	0.00555046131896994\\
484	0.0055953189682578\\
485	0.00564151663495771\\
486	0.00568895261741899\\
487	0.00573745407419114\\
488	0.00578675199310947\\
489	0.00583602825905697\\
490	0.0058842010330947\\
491	0.00593110897483594\\
492	0.00597658330801327\\
493	0.00602045039887398\\
494	0.00606253484819726\\
495	0.0061026665856415\\
496	0.00614069144522305\\
497	0.00617648509989499\\
498	0.00620997250803433\\
499	0.0062411547515585\\
500	0.00627014583967686\\
501	0.00629829006743282\\
502	0.00632581759750108\\
503	0.00635273293191337\\
504	0.00637905302861716\\
505	0.00640480916376602\\
506	0.00643004854353724\\
507	0.00645483543617274\\
508	0.00647925144017759\\
509	0.00650339433073105\\
510	0.00652737468658162\\
511	0.00655130919171871\\
512	0.00657528119528521\\
513	0.00659932731884868\\
514	0.00662347217653659\\
515	0.00664774250460606\\
516	0.00667216666117621\\
517	0.00669677395535563\\
518	0.00672159380448917\\
519	0.00674665474004682\\
520	0.0067719833180942\\
521	0.00679760304546858\\
522	0.00682353351760438\\
523	0.00684979079643919\\
524	0.006876389979067\\
525	0.00690334649439269\\
526	0.00693067601823747\\
527	0.00695839439858376\\
528	0.00698651760125394\\
529	0.00701506168821314\\
530	0.00704404284182867\\
531	0.00707347744796229\\
532	0.0071033822473147\\
533	0.00713377455577878\\
534	0.00716467252906144\\
535	0.00719609530995188\\
536	0.00722806308380102\\
537	0.00726059713458046\\
538	0.00729371989872221\\
539	0.00732745501242289\\
540	0.00736182734608918\\
541	0.00739686301702372\\
542	0.0074325893682471\\
543	0.00746903489751596\\
544	0.00750622911624425\\
545	0.00754420231346191\\
546	0.00758298520129518\\
547	0.00762260841386055\\
548	0.00766310181951272\\
549	0.00770449391757828\\
550	0.00774679995318002\\
551	0.00779003428858121\\
552	0.00783421045383369\\
553	0.00787934095957066\\
554	0.00792543702246905\\
555	0.00797250829035165\\
556	0.00802056253023774\\
557	0.00806960527968463\\
558	0.00811964027290157\\
559	0.00817066577568384\\
560	0.00822267263500106\\
561	0.00827564182636987\\
562	0.00832948684670028\\
563	0.00838419951871032\\
564	0.00843983032047843\\
565	0.00849646590164287\\
566	0.0085542417435169\\
567	0.00861340416126296\\
568	0.00867301874355139\\
569	0.00873133908695524\\
570	0.00878796637590921\\
571	0.00884218788307898\\
572	0.00889563206259737\\
573	0.00894812718337587\\
574	0.00899896376434958\\
575	0.00904832370413948\\
576	0.00909665414515263\\
577	0.00914434879159263\\
578	0.00919144228304781\\
579	0.00923764611023506\\
580	0.00928301640959322\\
581	0.0093279784285072\\
582	0.00937262878471426\\
583	0.00941698222330296\\
584	0.00946100780116802\\
585	0.00950465824199754\\
586	0.00954788190473341\\
587	0.00959062658943052\\
588	0.00963284242838222\\
589	0.0096744851752674\\
590	0.00971551906172689\\
591	0.00975591929412051\\
592	0.00979567191274874\\
593	0.00983476678232329\\
594	0.00987312984006299\\
595	0.00991038882774525\\
596	0.00994573390547615\\
597	0.0099771937715668\\
598	0.00999970795535495\\
599	0\\
600	0\\
};
\addplot [color=red!40!mycolor19,solid,forget plot]
  table[row sep=crcr]{%
1	0.00394496355807239\\
2	0.00394496646483332\\
3	0.0039449694249811\\
4	0.00394497243948291\\
5	0.00394497550932361\\
6	0.00394497863550595\\
7	0.00394498181905096\\
8	0.00394498506099832\\
9	0.00394498836240667\\
10	0.00394499172435401\\
11	0.00394499514793801\\
12	0.00394499863427644\\
13	0.00394500218450758\\
14	0.00394500579979047\\
15	0.00394500948130545\\
16	0.00394501323025453\\
17	0.0039450170478618\\
18	0.0039450209353738\\
19	0.00394502489406003\\
20	0.00394502892521339\\
21	0.00394503303015056\\
22	0.0039450372102125\\
23	0.00394504146676493\\
24	0.00394504580119881\\
25	0.00394505021493081\\
26	0.00394505470940376\\
27	0.0039450592860873\\
28	0.00394506394647825\\
29	0.00394506869210123\\
30	0.00394507352450914\\
31	0.0039450784452838\\
32	0.00394508345603639\\
33	0.00394508855840812\\
34	0.00394509375407081\\
35	0.00394509904472744\\
36	0.00394510443211276\\
37	0.00394510991799403\\
38	0.00394511550417151\\
39	0.00394512119247915\\
40	0.00394512698478532\\
41	0.00394513288299344\\
42	0.00394513888904266\\
43	0.00394514500490856\\
44	0.00394515123260396\\
45	0.00394515757417947\\
46	0.00394516403172446\\
47	0.00394517060736772\\
48	0.00394517730327818\\
49	0.00394518412166581\\
50	0.00394519106478241\\
51	0.00394519813492242\\
52	0.0039452053344238\\
53	0.00394521266566886\\
54	0.00394522013108519\\
55	0.00394522773314651\\
56	0.00394523547437364\\
57	0.00394524335733541\\
58	0.00394525138464964\\
59	0.00394525955898407\\
60	0.00394526788305749\\
61	0.00394527635964063\\
62	0.00394528499155725\\
63	0.0039452937816852\\
64	0.00394530273295754\\
65	0.00394531184836359\\
66	0.00394532113095009\\
67	0.00394533058382237\\
68	0.00394534021014553\\
69	0.00394535001314561\\
70	0.00394535999611084\\
71	0.00394537016239289\\
72	0.00394538051540814\\
73	0.00394539105863905\\
74	0.00394540179563535\\
75	0.00394541273001555\\
76	0.00394542386546822\\
77	0.00394543520575351\\
78	0.00394544675470449\\
79	0.00394545851622864\\
80	0.00394547049430942\\
81	0.00394548269300778\\
82	0.00394549511646364\\
83	0.00394550776889761\\
84	0.00394552065461258\\
85	0.00394553377799536\\
86	0.00394554714351835\\
87	0.00394556075574136\\
88	0.00394557461931328\\
89	0.00394558873897392\\
90	0.00394560311955589\\
91	0.00394561776598641\\
92	0.00394563268328927\\
93	0.00394564787658675\\
94	0.0039456633511016\\
95	0.00394567911215911\\
96	0.00394569516518915\\
97	0.00394571151572828\\
98	0.00394572816942192\\
99	0.00394574513202657\\
100	0.00394576240941198\\
101	0.00394578000756347\\
102	0.00394579793258432\\
103	0.003945816190698\\
104	0.00394583478825078\\
105	0.00394585373171405\\
106	0.00394587302768688\\
107	0.00394589268289868\\
108	0.00394591270421163\\
109	0.00394593309862358\\
110	0.00394595387327058\\
111	0.00394597503542981\\
112	0.00394599659252227\\
113	0.00394601855211579\\
114	0.00394604092192787\\
115	0.00394606370982878\\
116	0.00394608692384456\\
117	0.00394611057216016\\
118	0.00394613466312262\\
119	0.00394615920524426\\
120	0.00394618420720612\\
121	0.00394620967786115\\
122	0.00394623562623785\\
123	0.00394626206154353\\
124	0.00394628899316814\\
125	0.0039463164306877\\
126	0.0039463443838681\\
127	0.00394637286266892\\
128	0.00394640187724722\\
129	0.00394643143796143\\
130	0.00394646155537548\\
131	0.0039464922402628\\
132	0.00394652350361047\\
133	0.00394655535662353\\
134	0.0039465878107292\\
135	0.00394662087758137\\
136	0.00394665456906512\\
137	0.00394668889730117\\
138	0.00394672387465066\\
139	0.00394675951371986\\
140	0.00394679582736501\\
141	0.0039468328286973\\
142	0.00394687053108786\\
143	0.00394690894817291\\
144	0.003946948093859\\
145	0.00394698798232838\\
146	0.0039470286280443\\
147	0.00394707004575672\\
148	0.00394711225050783\\
149	0.00394715525763789\\
150	0.003947199082791\\
151	0.0039472437419212\\
152	0.00394728925129846\\
153	0.00394733562751496\\
154	0.00394738288749136\\
155	0.00394743104848332\\
156	0.00394748012808804\\
157	0.00394753014425094\\
158	0.00394758111527261\\
159	0.00394763305981562\\
160	0.00394768599691171\\
161	0.00394773994596905\\
162	0.00394779492677958\\
163	0.00394785095952653\\
164	0.00394790806479215\\
165	0.00394796626356545\\
166	0.00394802557725021\\
167	0.00394808602767313\\
168	0.00394814763709206\\
169	0.0039482104282045\\
170	0.00394827442415617\\
171	0.00394833964854986\\
172	0.00394840612545423\\
173	0.00394847387941314\\
174	0.0039485429354548\\
175	0.00394861331910132\\
176	0.00394868505637834\\
177	0.00394875817382495\\
178	0.00394883269850372\\
179	0.00394890865801092\\
180	0.00394898608048703\\
181	0.00394906499462736\\
182	0.00394914542969294\\
183	0.00394922741552157\\
184	0.00394931098253916\\
185	0.00394939616177125\\
186	0.00394948298485471\\
187	0.00394957148404979\\
188	0.00394966169225226\\
189	0.00394975364300596\\
190	0.00394984737051543\\
191	0.00394994290965888\\
192	0.00395004029600142\\
193	0.00395013956580852\\
194	0.0039502407560597\\
195	0.00395034390446258\\
196	0.0039504490494672\\
197	0.00395055623028044\\
198	0.00395066548688106\\
199	0.00395077686003469\\
200	0.00395089039130935\\
201	0.0039510061230911\\
202	0.00395112409860024\\
203	0.00395124436190747\\
204	0.00395136695795083\\
205	0.0039514919325525\\
206	0.00395161933243626\\
207	0.00395174920524514\\
208	0.00395188159955945\\
209	0.0039520165649152\\
210	0.00395215415182285\\
211	0.00395229441178633\\
212	0.00395243739732263\\
213	0.00395258316198166\\
214	0.00395273176036637\\
215	0.00395288324815356\\
216	0.00395303768211484\\
217	0.00395319512013803\\
218	0.00395335562124911\\
219	0.00395351924563447\\
220	0.00395368605466358\\
221	0.00395385611091217\\
222	0.00395402947818585\\
223	0.0039542062215441\\
224	0.0039543864073248\\
225	0.00395457010316907\\
226	0.00395475737804693\\
227	0.003954948302283\\
228	0.00395514294758295\\
229	0.00395534138706046\\
230	0.00395554369526455\\
231	0.00395574994820742\\
232	0.00395596022339293\\
233	0.00395617459984554\\
234	0.00395639315813964\\
235	0.00395661598042968\\
236	0.00395684315048058\\
237	0.00395707475369883\\
238	0.00395731087716406\\
239	0.00395755160966121\\
240	0.00395779704171326\\
241	0.00395804726561444\\
242	0.00395830237546404\\
243	0.00395856246720084\\
244	0.00395882763863805\\
245	0.00395909798949871\\
246	0.00395937362145188\\
247	0.00395965463814919\\
248	0.00395994114526192\\
249	0.00396023325051887\\
250	0.00396053106374449\\
251	0.00396083469689774\\
252	0.00396114426411135\\
253	0.00396145988173162\\
254	0.00396178166835869\\
255	0.00396210974488731\\
256	0.00396244423454795\\
257	0.00396278526294831\\
258	0.00396313295811507\\
259	0.00396348745053593\\
260	0.00396384887320159\\
261	0.00396421736164758\\
262	0.00396459305399591\\
263	0.00396497609099558\\
264	0.00396536661606227\\
265	0.00396576477531575\\
266	0.00396617071761449\\
267	0.00396658459458613\\
268	0.00396700656065184\\
269	0.00396743677304213\\
270	0.00396787539180087\\
271	0.00396832257977263\\
272	0.0039687785025673\\
273	0.00396924332849335\\
274	0.00396971722844904\\
275	0.00397020037575863\\
276	0.00397069294594217\\
277	0.00397119511642956\\
278	0.00397170706633817\\
279	0.00397222897697963\\
280	0.003972761036643\\
281	0.00397330344269698\\
282	0.0039738563963747\\
283	0.00397442010285586\\
284	0.00397499477135162\\
285	0.003975580615192\\
286	0.00397617785191623\\
287	0.00397678670336624\\
288	0.00397740739578341\\
289	0.00397804015990856\\
290	0.00397868523108615\\
291	0.00397934284937203\\
292	0.00398001325964574\\
293	0.00398069671172766\\
294	0.00398139346050093\\
295	0.00398210376603946\\
296	0.00398282789374209\\
297	0.00398356611447407\\
298	0.00398431870471683\\
299	0.00398508594672751\\
300	0.00398586812871031\\
301	0.00398666554500194\\
302	0.00398747849627502\\
303	0.00398830728976378\\
304	0.00398915223951882\\
305	0.00399001366669935\\
306	0.00399089189991507\\
307	0.00399178727563383\\
308	0.00399270013867693\\
309	0.00399363084283214\\
310	0.00399457975162358\\
311	0.00399554723929041\\
312	0.00399653369203804\\
313	0.00399753950962834\\
314	0.00399856510734339\\
315	0.00399961091819241\\
316	0.00400067739463\\
317	0.00400176500691815\\
318	0.00400287422750616\\
319	0.00400400549062357\\
320	0.00400515923946962\\
321	0.00400633592641326\\
322	0.00400753601319825\\
323	0.00400875997115389\\
324	0.00401000828141145\\
325	0.00401128143512629\\
326	0.00401257993370635\\
327	0.00401390428904663\\
328	0.0040152550237706\\
329	0.00401663267147813\\
330	0.00401803777700066\\
331	0.00401947089666371\\
332	0.00402093259855716\\
333	0.0040224234628134\\
334	0.00402394408189395\\
335	0.00402549506088478\\
336	0.00402707701780082\\
337	0.00402869058389997\\
338	0.00403033640400725\\
339	0.00403201513684932\\
340	0.00403372745540015\\
341	0.00403547404723831\\
342	0.00403725561491625\\
343	0.00403907287634269\\
344	0.00404092656517789\\
345	0.00404281743124264\\
346	0.00404474624094188\\
347	0.00404671377770299\\
348	0.00404872084243032\\
349	0.00405076825397612\\
350	0.00405285684962921\\
351	0.00405498748562241\\
352	0.00405716103766\\
353	0.0040593784014669\\
354	0.00406164049336234\\
355	0.0040639482508606\\
356	0.00406630263330409\\
357	0.00406870462253483\\
358	0.00407115522361375\\
359	0.0040736554656001\\
360	0.0040762064024049\\
361	0.00407880911370457\\
362	0.00408146470571022\\
363	0.00408417431140515\\
364	0.00408693909135697\\
365	0.00408976023456167\\
366	0.00409263895932078\\
367	0.00409557651415383\\
368	0.00409857417874719\\
369	0.00410163326494096\\
370	0.00410475511775484\\
371	0.00410794111645349\\
372	0.00411119267565054\\
373	0.00411451124644887\\
374	0.00411789831761287\\
375	0.00412135541676298\\
376	0.0041248841115783\\
377	0.00412848601098433\\
378	0.0041321627662925\\
379	0.00413591607225034\\
380	0.00413974766796995\\
381	0.0041436593378207\\
382	0.00414765291309146\\
383	0.00415173027645591\\
384	0.00415589336397858\\
385	0.00416014416719945\\
386	0.00416448473532126\\
387	0.00416891717750316\\
388	0.00417344366526517\\
389	0.00417806643500328\\
390	0.00418278779061785\\
391	0.00418761010625828\\
392	0.00419253582918275\\
393	0.00419756748273102\\
394	0.00420270766940556\\
395	0.00420795907405323\\
396	0.00421332446713702\\
397	0.00421880670808039\\
398	0.00422440874865449\\
399	0.00423013363637588\\
400	0.00423598451790212\\
401	0.00424196464238815\\
402	0.00424807736478989\\
403	0.00425432614902129\\
404	0.00426071457093349\\
405	0.00426724632107828\\
406	0.004273925207205\\
407	0.00428075515657952\\
408	0.00428774021817981\\
409	0.00429488456494814\\
410	0.00430219249646694\\
411	0.00430966844256718\\
412	0.00431731696826001\\
413	0.00432514278095873\\
414	0.00433315074798983\\
415	0.00434134591320038\\
416	0.00434973350581722\\
417	0.00435831894954741\\
418	0.00436710787222604\\
419	0.00437610611605775\\
420	0.00438531974852342\\
421	0.00439475507406703\\
422	0.00440441864655328\\
423	0.00441431728231955\\
424	0.00442445807483061\\
425	0.00443484840843557\\
426	0.00444549597140659\\
427	0.00445640876951495\\
428	0.00446759514619427\\
429	0.00447906380106728\\
430	0.0044908238084109\\
431	0.00450288463640696\\
432	0.00451525616716112\\
433	0.00452794871752132\\
434	0.00454097306099056\\
435	0.00455434045246367\\
436	0.00456806266293691\\
437	0.00458215200805808\\
438	0.0045966213459003\\
439	0.00461148406469206\\
440	0.00462675417512067\\
441	0.00464244635062386\\
442	0.00465857597089339\\
443	0.00467515916830442\\
444	0.00469221287659104\\
445	0.00470975488028099\\
446	0.00472780386788214\\
447	0.00474637959009092\\
448	0.00476550297957607\\
449	0.00478519598648636\\
450	0.00480548097141639\\
451	0.00482638121590674\\
452	0.00484792084856823\\
453	0.00487012456396235\\
454	0.00489301757738932\\
455	0.00491662557462582\\
456	0.00494097461056348\\
457	0.00496609074537327\\
458	0.00499200351336739\\
459	0.00501874473018187\\
460	0.00504634372167624\\
461	0.00507482848338156\\
462	0.00510422816602658\\
463	0.00513457392687377\\
464	0.00516590026784464\\
465	0.00519825241076104\\
466	0.00523168729608508\\
467	0.00526626029038589\\
468	0.00530202299552629\\
469	0.00533902031828007\\
470	0.00537728661793321\\
471	0.00541684044183194\\
472	0.00545767738756752\\
473	0.00549976060088042\\
474	0.00554300816069849\\
475	0.00558727665285665\\
476	0.00563233935889654\\
477	0.00567766537066826\\
478	0.00572200969788633\\
479	0.00576522661355051\\
480	0.00580716209985042\\
481	0.00584765617745405\\
482	0.00588654709499107\\
483	0.00592367616574599\\
484	0.00595889481100118\\
485	0.00599207473592931\\
486	0.00602312453221166\\
487	0.0060520110078312\\
488	0.00607878851709771\\
489	0.00610406937245691\\
490	0.00612875844950554\\
491	0.00615285424514161\\
492	0.00617636559083878\\
493	0.00619931332170027\\
494	0.0062217318742031\\
495	0.00624367062664039\\
496	0.00626519468351297\\
497	0.00628638471090834\\
498	0.00630733526897909\\
499	0.00632815083990122\\
500	0.00634893842542317\\
501	0.00636975308506636\\
502	0.00639061925824366\\
503	0.00641155928072116\\
504	0.00643259712800896\\
505	0.00645375790660415\\
506	0.00647506719390883\\
507	0.00649655023111479\\
508	0.0065182309953332\\
509	0.00654013121257779\\
510	0.00656226942791659\\
511	0.00658466033084912\\
512	0.00660731580898516\\
513	0.00663024701649982\\
514	0.00665346511383199\\
515	0.00667698117480523\\
516	0.00670080610463867\\
517	0.00672495057828269\\
518	0.00674942500995607\\
519	0.00677423956536453\\
520	0.00679940422702772\\
521	0.00682492891916385\\
522	0.00685082368968336\\
523	0.00687709889966683\\
524	0.00690376531885638\\
525	0.00693083416580522\\
526	0.00695831715653323\\
527	0.00698622656210869\\
528	0.00701457527508552\\
529	0.00704337688405048\\
530	0.00707264575467542\\
531	0.0071023971146663\\
532	0.00713264713896589\\
533	0.00716341303073712\\
534	0.00719471309383995\\
535	0.00722656679893333\\
536	0.00725899484423924\\
537	0.00729201920698204\\
538	0.00732566318026771\\
539	0.00735995138860808\\
540	0.00739490977335536\\
541	0.00743056553690132\\
542	0.00746694703150045\\
543	0.00750408357484665\\
544	0.00754200516988861\\
545	0.00758074210050495\\
546	0.00762032436740506\\
547	0.00766078090917343\\
548	0.0077021386134001\\
549	0.00774441090401645\\
550	0.00778760972370493\\
551	0.00783174548080916\\
552	0.0078768267041794\\
553	0.00792285966441682\\
554	0.00796984870473029\\
555	0.00801779233817997\\
556	0.00806668120835328\\
557	0.00811649542837718\\
558	0.00816712404026765\\
559	0.00821860615966\\
560	0.00827100021602301\\
561	0.00832438113928522\\
562	0.0083789392287531\\
563	0.00843483361816912\\
564	0.00849221069336367\\
565	0.00854933671311469\\
566	0.00860506706316596\\
567	0.00865884633410658\\
568	0.0087110393692905\\
569	0.0087629231690006\\
570	0.00881433000902838\\
571	0.00886444287004998\\
572	0.00891325983660414\\
573	0.00896106468256766\\
574	0.00900845651393775\\
575	0.00905548628705952\\
576	0.00910215844577083\\
577	0.00914809200862895\\
578	0.0091933703120837\\
579	0.00923845293548966\\
580	0.00928338532988187\\
581	0.00932815453170577\\
582	0.00937272078824421\\
583	0.00941703320141838\\
584	0.00946103673208299\\
585	0.00950467492731733\\
586	0.0095478914410834\\
587	0.00959063167473112\\
588	0.00963284478933166\\
589	0.00967448602529565\\
590	0.00971551927992669\\
591	0.00975591931512916\\
592	0.00979567191274874\\
593	0.00983476678232329\\
594	0.00987312984006299\\
595	0.00991038882774525\\
596	0.00994573390547615\\
597	0.0099771937715668\\
598	0.00999970795535495\\
599	0\\
600	0\\
};
\addplot [color=red!75!mycolor17,solid,forget plot]
  table[row sep=crcr]{%
1	0.00395602082978746\\
2	0.00395602405595148\\
3	0.00395602734201359\\
4	0.00395603068905089\\
5	0.00395603409815943\\
6	0.00395603757045459\\
7	0.00395604110707146\\
8	0.00395604470916515\\
9	0.00395604837791114\\
10	0.00395605211450571\\
11	0.00395605592016628\\
12	0.00395605979613175\\
13	0.00395606374366297\\
14	0.00395606776404308\\
15	0.00395607185857799\\
16	0.0039560760285967\\
17	0.0039560802754518\\
18	0.00395608460051981\\
19	0.00395608900520178\\
20	0.00395609349092355\\
21	0.00395609805913636\\
22	0.00395610271131725\\
23	0.00395610744896952\\
24	0.00395611227362331\\
25	0.00395611718683595\\
26	0.00395612219019262\\
27	0.00395612728530677\\
28	0.0039561324738207\\
29	0.00395613775740604\\
30	0.00395614313776441\\
31	0.00395614861662786\\
32	0.00395615419575955\\
33	0.00395615987695427\\
34	0.00395616566203905\\
35	0.00395617155287377\\
36	0.00395617755135186\\
37	0.0039561836594008\\
38	0.00395618987898292\\
39	0.00395619621209593\\
40	0.0039562026607737\\
41	0.00395620922708692\\
42	0.00395621591314377\\
43	0.00395622272109076\\
44	0.00395622965311331\\
45	0.00395623671143668\\
46	0.00395624389832659\\
47	0.00395625121609007\\
48	0.00395625866707629\\
49	0.00395626625367737\\
50	0.0039562739783292\\
51	0.0039562818435123\\
52	0.00395628985175273\\
53	0.00395629800562295\\
54	0.0039563063077428\\
55	0.00395631476078034\\
56	0.00395632336745288\\
57	0.00395633213052796\\
58	0.00395634105282425\\
59	0.00395635013721278\\
60	0.00395635938661775\\
61	0.00395636880401769\\
62	0.00395637839244663\\
63	0.0039563881549951\\
64	0.00395639809481132\\
65	0.0039564082151023\\
66	0.00395641851913513\\
67	0.00395642901023816\\
68	0.00395643969180219\\
69	0.0039564505672818\\
70	0.00395646164019659\\
71	0.00395647291413252\\
72	0.00395648439274334\\
73	0.00395649607975182\\
74	0.00395650797895131\\
75	0.00395652009420712\\
76	0.00395653242945801\\
77	0.00395654498871764\\
78	0.00395655777607619\\
79	0.00395657079570189\\
80	0.00395658405184268\\
81	0.00395659754882772\\
82	0.00395661129106929\\
83	0.00395662528306431\\
84	0.00395663952939613\\
85	0.0039566540347364\\
86	0.00395666880384683\\
87	0.00395668384158114\\
88	0.00395669915288682\\
89	0.00395671474280725\\
90	0.0039567306164836\\
91	0.00395674677915684\\
92	0.00395676323616991\\
93	0.00395677999296979\\
94	0.00395679705510968\\
95	0.00395681442825122\\
96	0.00395683211816676\\
97	0.00395685013074166\\
98	0.00395686847197662\\
99	0.00395688714799018\\
100	0.00395690616502109\\
101	0.00395692552943095\\
102	0.0039569452477066\\
103	0.00395696532646291\\
104	0.00395698577244533\\
105	0.00395700659253277\\
106	0.00395702779374025\\
107	0.00395704938322182\\
108	0.00395707136827348\\
109	0.0039570937563361\\
110	0.0039571165549985\\
111	0.00395713977200056\\
112	0.00395716341523631\\
113	0.00395718749275724\\
114	0.00395721201277556\\
115	0.0039572369836675\\
116	0.00395726241397684\\
117	0.00395728831241832\\
118	0.00395731468788126\\
119	0.00395734154943326\\
120	0.00395736890632381\\
121	0.00395739676798813\\
122	0.00395742514405103\\
123	0.00395745404433089\\
124	0.00395748347884368\\
125	0.00395751345780703\\
126	0.00395754399164451\\
127	0.00395757509098976\\
128	0.00395760676669098\\
129	0.00395763902981539\\
130	0.00395767189165368\\
131	0.00395770536372469\\
132	0.00395773945778006\\
133	0.00395777418580917\\
134	0.00395780956004397\\
135	0.00395784559296401\\
136	0.00395788229730144\\
137	0.00395791968604639\\
138	0.00395795777245217\\
139	0.00395799657004068\\
140	0.00395803609260799\\
141	0.00395807635422989\\
142	0.00395811736926766\\
143	0.0039581591523739\\
144	0.00395820171849852\\
145	0.00395824508289477\\
146	0.00395828926112551\\
147	0.00395833426906939\\
148	0.00395838012292743\\
149	0.0039584268392295\\
150	0.00395847443484102\\
151	0.00395852292696975\\
152	0.0039585723331728\\
153	0.0039586226713637\\
154	0.00395867395981956\\
155	0.00395872621718849\\
156	0.00395877946249704\\
157	0.00395883371515793\\
158	0.00395888899497776\\
159	0.00395894532216502\\
160	0.00395900271733813\\
161	0.00395906120153377\\
162	0.00395912079621516\\
163	0.00395918152328077\\
164	0.00395924340507306\\
165	0.00395930646438725\\
166	0.00395937072448059\\
167	0.00395943620908148\\
168	0.003959502942399\\
169	0.00395957094913248\\
170	0.00395964025448138\\
171	0.00395971088415519\\
172	0.00395978286438382\\
173	0.00395985622192772\\
174	0.00395993098408878\\
175	0.00396000717872095\\
176	0.00396008483424138\\
177	0.00396016397964162\\
178	0.00396024464449905\\
179	0.00396032685898871\\
180	0.00396041065389509\\
181	0.00396049606062437\\
182	0.00396058311121681\\
183	0.00396067183835945\\
184	0.00396076227539898\\
185	0.00396085445635494\\
186	0.00396094841593315\\
187	0.00396104418953945\\
188	0.00396114181329363\\
189	0.00396124132404376\\
190	0.00396134275938071\\
191	0.0039614461576531\\
192	0.00396155155798237\\
193	0.00396165900027826\\
194	0.00396176852525467\\
195	0.00396188017444574\\
196	0.00396199399022223\\
197	0.00396211001580845\\
198	0.00396222829529927\\
199	0.00396234887367768\\
200	0.00396247179683259\\
201	0.00396259711157713\\
202	0.00396272486566713\\
203	0.00396285510782021\\
204	0.0039629878877351\\
205	0.00396312325611148\\
206	0.00396326126467012\\
207	0.00396340196617354\\
208	0.00396354541444703\\
209	0.00396369166440016\\
210	0.00396384077204873\\
211	0.00396399279453712\\
212	0.00396414779016123\\
213	0.00396430581839175\\
214	0.00396446693989807\\
215	0.00396463121657251\\
216	0.00396479871155532\\
217	0.00396496948925985\\
218	0.00396514361539862\\
219	0.00396532115700966\\
220	0.0039655021824835\\
221	0.00396568676159081\\
222	0.00396587496551043\\
223	0.00396606686685812\\
224	0.0039662625397159\\
225	0.00396646205966207\\
226	0.00396666550380156\\
227	0.00396687295079733\\
228	0.00396708448090212\\
229	0.003967300175991\\
230	0.00396752011959459\\
231	0.00396774439693293\\
232	0.00396797309495013\\
233	0.00396820630234957\\
234	0.00396844410963019\\
235	0.00396868660912315\\
236	0.00396893389502942\\
237	0.00396918606345832\\
238	0.0039694432124666\\
239	0.00396970544209855\\
240	0.00396997285442672\\
241	0.00397024555359395\\
242	0.00397052364585571\\
243	0.00397080723962386\\
244	0.00397109644551091\\
245	0.00397139137637569\\
246	0.00397169214736955\\
247	0.00397199887598384\\
248	0.00397231168209841\\
249	0.00397263068803103\\
250	0.00397295601858803\\
251	0.00397328780111598\\
252	0.00397362616555453\\
253	0.0039739712444905\\
254	0.00397432317321316\\
255	0.00397468208977065\\
256	0.00397504813502792\\
257	0.00397542145272575\\
258	0.0039758021895414\\
259	0.00397619049515047\\
260	0.00397658652229027\\
261	0.00397699042682486\\
262	0.00397740236781135\\
263	0.00397782250756811\\
264	0.0039782510117444\\
265	0.00397868804939187\\
266	0.00397913379303773\\
267	0.0039795884187599\\
268	0.00398005210626397\\
269	0.00398052503896246\\
270	0.00398100740405609\\
271	0.00398149939261786\\
272	0.00398200119967981\\
273	0.00398251302432341\\
274	0.00398303506977423\\
275	0.00398356754350188\\
276	0.00398411065732733\\
277	0.00398466462753955\\
278	0.00398522967502303\\
279	0.00398580602538462\\
280	0.00398639390902049\\
281	0.00398699356115706\\
282	0.00398760522195791\\
283	0.00398822913663347\\
284	0.00398886555555311\\
285	0.00398951473436007\\
286	0.00399017693408895\\
287	0.00399085242128627\\
288	0.00399154146813359\\
289	0.00399224435257379\\
290	0.00399296135844012\\
291	0.00399369277558839\\
292	0.00399443890003218\\
293	0.00399520003408121\\
294	0.00399597648648278\\
295	0.00399676857256657\\
296	0.00399757661439261\\
297	0.00399840094090254\\
298	0.00399924188807439\\
299	0.0040000997990806\\
300	0.00400097502444961\\
301	0.00400186792223084\\
302	0.00400277885816342\\
303	0.00400370820584827\\
304	0.00400465634692377\\
305	0.00400562367124485\\
306	0.00400661057706534\\
307	0.004007617471223\\
308	0.00400864476932682\\
309	0.0040096928959449\\
310	0.0040107622847916\\
311	0.00401185337891045\\
312	0.00401296663084853\\
313	0.00401410250281579\\
314	0.00401526146681993\\
315	0.00401644400476658\\
316	0.00401765060851909\\
317	0.00401888177995787\\
318	0.00402013803126919\\
319	0.00402141988576657\\
320	0.00402272787815363\\
321	0.00402406255479396\\
322	0.00402542447398887\\
323	0.00402681420626239\\
324	0.00402823233465483\\
325	0.0040296794550242\\
326	0.00403115617635655\\
327	0.00403266312108521\\
328	0.00403420092541916\\
329	0.00403577023968131\\
330	0.00403737172865651\\
331	0.00403900607195019\\
332	0.00404067396435763\\
333	0.00404237611624463\\
334	0.00404411325393971\\
335	0.00404588612013855\\
336	0.00404769547432123\\
337	0.00404954209318242\\
338	0.00405142677107538\\
339	0.00405335032047024\\
340	0.00405531357242699\\
341	0.00405731737708404\\
342	0.00405936260416318\\
343	0.00406145014349098\\
344	0.00406358090553805\\
345	0.00406575582197638\\
346	0.00406797584625572\\
347	0.00407024195419969\\
348	0.0040725551446227\\
349	0.00407491643996836\\
350	0.00407732688697043\\
351	0.00407978755733752\\
352	0.00408229954846239\\
353	0.00408486398415702\\
354	0.00408748201541441\\
355	0.00409015482119849\\
356	0.00409288360926343\\
357	0.0040956696170037\\
358	0.0040985141123362\\
359	0.0041014183946158\\
360	0.00410438379558483\\
361	0.00410741168035432\\
362	0.00411050344842638\\
363	0.00411366053476552\\
364	0.00411688441091478\\
365	0.00412017658615775\\
366	0.00412353860872746\\
367	0.00412697206706691\\
368	0.00413047859114477\\
369	0.0041340598538287\\
370	0.00413771757232021\\
371	0.0041414535096554\\
372	0.00414526947627356\\
373	0.00414916733165619\\
374	0.00415314898604155\\
375	0.00415721640221952\\
376	0.00416137159741164\\
377	0.00416561664524115\\
378	0.00416995367779931\\
379	0.00417438488781448\\
380	0.00417891253093526\\
381	0.00418353892815308\\
382	0.00418826646833857\\
383	0.00419309761086729\\
384	0.00419803488836444\\
385	0.00420308090958143\\
386	0.00420823836242274\\
387	0.00421351001712873\\
388	0.00421889872961684\\
389	0.00422440744500477\\
390	0.00423003920130374\\
391	0.00423579713330029\\
392	0.00424168447663842\\
393	0.0042477045721139\\
394	0.0042538608701944\\
395	0.00426015693578099\\
396	0.00426659645323247\\
397	0.00427318323168068\\
398	0.00427992121066044\\
399	0.00428681446598744\\
400	0.00429386721579037\\
401	0.00430108382694307\\
402	0.00430846882187104\\
403	0.00431602688620221\\
404	0.00432376287672331\\
405	0.00433168182983377\\
406	0.00433978897059622\\
407	0.0043480897219022\\
408	0.00435658971459941\\
409	0.00436529479824546\\
410	0.00437421105255016\\
411	0.00438334479965116\\
412	0.00439270261748483\\
413	0.00440229135313236\\
414	0.00441211813466922\\
415	0.00442219038346729\\
416	0.0044325158284914\\
417	0.00444310252466623\\
418	0.00445395886868408\\
419	0.00446509361532809\\
420	0.00447651589436813\\
421	0.00448823522833583\\
422	0.00450026155237366\\
423	0.00451260523649623\\
424	0.00452527710718072\\
425	0.00453828848728478\\
426	0.00455165122226101\\
427	0.0045653776909434\\
428	0.00457948080298924\\
429	0.00459397407331674\\
430	0.00460887167322002\\
431	0.00462418847383528\\
432	0.00463994009095224\\
433	0.00465614292922633\\
434	0.00467281422244718\\
435	0.00468997206427308\\
436	0.00470763542724128\\
437	0.00472582428256058\\
438	0.00474455966549337\\
439	0.0047638632012953\\
440	0.00478375629291361\\
441	0.00480426069748669\\
442	0.00482539846900143\\
443	0.004847191910015\\
444	0.00486966354231972\\
445	0.00489283611036386\\
446	0.00491673261793214\\
447	0.00494137625657943\\
448	0.00496679181892853\\
449	0.0049930078615359\\
450	0.00502005721140985\\
451	0.00504797014935873\\
452	0.00507678366299867\\
453	0.00510655036703916\\
454	0.00513732270574328\\
455	0.00516915145734446\\
456	0.00520208365406883\\
457	0.00523615972776357\\
458	0.00527140961974777\\
459	0.00530784752197285\\
460	0.00534546502349508\\
461	0.00538422192609985\\
462	0.00542403408627614\\
463	0.00546475724299661\\
464	0.00550616574427743\\
465	0.00554754372534509\\
466	0.00558802536025232\\
467	0.00562747746474042\\
468	0.00566575906656747\\
469	0.0057027233202975\\
470	0.00573822038525274\\
471	0.00577210227021862\\
472	0.00580422976681203\\
473	0.00583448158778131\\
474	0.00586276724356554\\
475	0.00588904432465679\\
476	0.00591334489936595\\
477	0.00593600660055911\\
478	0.0059581084388347\\
479	0.00597964601036323\\
480	0.00600062369861628\\
481	0.00602105616318666\\
482	0.00604096978447627\\
483	0.00606040393537579\\
484	0.00607941189302934\\
485	0.00609806108721539\\
486	0.00611643223438726\\
487	0.00613461672470033\\
488	0.00615271141358175\\
489	0.00617079307180954\\
490	0.00618889452339401\\
491	0.0062070348678562\\
492	0.00622523483041168\\
493	0.00624351637917005\\
494	0.00626190221165521\\
495	0.00628041510624132\\
496	0.00629907714893966\\
497	0.0063179088687222\\
498	0.00633692835028307\\
499	0.0063561504481978\\
500	0.00637558630873905\\
501	0.0063952452307703\\
502	0.00641513623636046\\
503	0.00643526821137997\\
504	0.00645564981709753\\
505	0.00647628941324281\\
506	0.00649719500095732\\
507	0.006518374195067\\
508	0.00653983423523254\\
509	0.00656158204397307\\
510	0.00658362433509701\\
511	0.00660596776689138\\
512	0.006628619069477\\
513	0.00665158509273979\\
514	0.00667487282270193\\
515	0.00669848940419556\\
516	0.00672244217019867\\
517	0.00674673867782084\\
518	0.00677138675043223\\
519	0.00679639452483013\\
520	0.00682177050169216\\
521	0.0068475235970019\\
522	0.00687366319189161\\
523	0.00690019918016826\\
524	0.00692714201721569\\
525	0.00695450277284056\\
526	0.00698229318789062\\
527	0.0070105257343359\\
528	0.00703921367834333\\
529	0.00706837114569149\\
530	0.00709801318866338\\
531	0.00712815585331523\\
532	0.00715881624573353\\
533	0.00719001259553259\\
534	0.00722176431433398\\
535	0.00725409204598056\\
536	0.00728701770400715\\
537	0.0073205644905423\\
538	0.00735475688910968\\
539	0.00738962062164505\\
540	0.00742518255733732\\
541	0.0074614705575005\\
542	0.00749851323640489\\
543	0.00753633961264411\\
544	0.00757497861769079\\
545	0.00761445842043134\\
546	0.00765480549641281\\
547	0.00769604373580787\\
548	0.0077381821411131\\
549	0.00778122713602009\\
550	0.00782518306720235\\
551	0.0078700487002368\\
552	0.00791581480582034\\
553	0.00796245996525486\\
554	0.00800987316217017\\
555	0.00805809948019964\\
556	0.00810719919027359\\
557	0.00815724841861093\\
558	0.00820844826237103\\
559	0.00826089347488161\\
560	0.00831470945785028\\
561	0.00837006021856648\\
562	0.00842504692476101\\
563	0.00847857196042984\\
564	0.00852998377507575\\
565	0.00858051179727193\\
566	0.00863082934811707\\
567	0.00868092347506556\\
568	0.00873049486019678\\
569	0.00877890731150697\\
570	0.00882609246461658\\
571	0.00887301158364894\\
572	0.00891969493973841\\
573	0.00896620472866063\\
574	0.00901252443128679\\
575	0.00905824660811991\\
576	0.00910340646184474\\
577	0.00914848298145622\\
578	0.00919352650797757\\
579	0.00923851799560512\\
580	0.00928341551328938\\
581	0.00932817023841046\\
582	0.00937272947451715\\
583	0.00941703811786192\\
584	0.0094610395478402\\
585	0.00950467651425775\\
586	0.00954789227157245\\
587	0.0095906320514706\\
588	0.00963284492201702\\
589	0.00967448605830613\\
590	0.00971551928305088\\
591	0.00975591931512916\\
592	0.00979567191274874\\
593	0.00983476678232329\\
594	0.00987312984006299\\
595	0.00991038882774525\\
596	0.00994573390547615\\
597	0.0099771937715668\\
598	0.00999970795535495\\
599	0\\
600	0\\
};
\addplot [color=red!80!mycolor19,solid,forget plot]
  table[row sep=crcr]{%
1	0.00395863302433954\\
2	0.00395863704254213\\
3	0.00395864113696591\\
4	0.0039586453089917\\
5	0.00395864956002412\\
6	0.00395865389149205\\
7	0.00395865830484892\\
8	0.00395866280157322\\
9	0.00395866738316888\\
10	0.00395867205116569\\
11	0.00395867680711977\\
12	0.00395868165261395\\
13	0.00395868658925829\\
14	0.0039586916186905\\
15	0.00395869674257641\\
16	0.00395870196261045\\
17	0.00395870728051606\\
18	0.00395871269804637\\
19	0.00395871821698453\\
20	0.00395872383914418\\
21	0.00395872956637022\\
22	0.0039587354005391\\
23	0.00395874134355941\\
24	0.00395874739737249\\
25	0.00395875356395299\\
26	0.00395875984530938\\
27	0.00395876624348452\\
28	0.00395877276055638\\
29	0.0039587793986385\\
30	0.00395878615988069\\
31	0.00395879304646962\\
32	0.00395880006062943\\
33	0.00395880720462245\\
34	0.00395881448074988\\
35	0.00395882189135237\\
36	0.00395882943881079\\
37	0.00395883712554691\\
38	0.00395884495402408\\
39	0.00395885292674809\\
40	0.00395886104626777\\
41	0.00395886931517584\\
42	0.00395887773610968\\
43	0.00395888631175211\\
44	0.00395889504483215\\
45	0.00395890393812599\\
46	0.00395891299445771\\
47	0.00395892221670016\\
48	0.00395893160777593\\
49	0.00395894117065815\\
50	0.00395895090837146\\
51	0.00395896082399292\\
52	0.003958970920653\\
53	0.00395898120153658\\
54	0.00395899166988384\\
55	0.00395900232899139\\
56	0.00395901318221332\\
57	0.00395902423296218\\
58	0.00395903548471019\\
59	0.00395904694099019\\
60	0.00395905860539693\\
61	0.00395907048158818\\
62	0.00395908257328592\\
63	0.0039590948842775\\
64	0.003959107418417\\
65	0.00395912017962644\\
66	0.00395913317189703\\
67	0.00395914639929055\\
68	0.00395915986594074\\
69	0.00395917357605458\\
70	0.00395918753391386\\
71	0.00395920174387651\\
72	0.00395921621037813\\
73	0.00395923093793355\\
74	0.00395924593113826\\
75	0.00395926119467016\\
76	0.00395927673329107\\
77	0.0039592925518484\\
78	0.00395930865527695\\
79	0.00395932504860054\\
80	0.0039593417369338\\
81	0.00395935872548412\\
82	0.0039593760195533\\
83	0.00395939362453965\\
84	0.00395941154593984\\
85	0.00395942978935093\\
86	0.00395944836047241\\
87	0.00395946726510826\\
88	0.00395948650916915\\
89	0.00395950609867453\\
90	0.00395952603975498\\
91	0.00395954633865443\\
92	0.00395956700173253\\
93	0.00395958803546704\\
94	0.00395960944645631\\
95	0.00395963124142173\\
96	0.00395965342721036\\
97	0.00395967601079753\\
98	0.00395969899928957\\
99	0.00395972239992647\\
100	0.00395974622008484\\
101	0.00395977046728063\\
102	0.00395979514917221\\
103	0.00395982027356331\\
104	0.00395984584840615\\
105	0.00395987188180457\\
106	0.00395989838201725\\
107	0.00395992535746104\\
108	0.00395995281671437\\
109	0.0039599807685206\\
110	0.00396000922179166\\
111	0.00396003818561157\\
112	0.00396006766924031\\
113	0.00396009768211736\\
114	0.00396012823386576\\
115	0.00396015933429598\\
116	0.00396019099341\\
117	0.00396022322140539\\
118	0.00396025602867958\\
119	0.00396028942583416\\
120	0.00396032342367933\\
121	0.00396035803323832\\
122	0.00396039326575217\\
123	0.00396042913268429\\
124	0.00396046564572532\\
125	0.00396050281679812\\
126	0.00396054065806268\\
127	0.00396057918192141\\
128	0.00396061840102427\\
129	0.00396065832827417\\
130	0.00396069897683245\\
131	0.0039607403601245\\
132	0.0039607824918455\\
133	0.00396082538596611\\
134	0.00396086905673861\\
135	0.00396091351870289\\
136	0.00396095878669272\\
137	0.003961004875842\\
138	0.0039610518015913\\
139	0.00396109957969451\\
140	0.00396114822622545\\
141	0.00396119775758495\\
142	0.00396124819050762\\
143	0.00396129954206929\\
144	0.0039613518296941\\
145	0.00396140507116206\\
146	0.00396145928461665\\
147	0.00396151448857259\\
148	0.00396157070192373\\
149	0.00396162794395106\\
150	0.00396168623433112\\
151	0.00396174559314424\\
152	0.00396180604088311\\
153	0.00396186759846154\\
154	0.00396193028722337\\
155	0.00396199412895147\\
156	0.0039620591458771\\
157	0.00396212536068919\\
158	0.00396219279654399\\
159	0.00396226147707491\\
160	0.0039623314264024\\
161	0.00396240266914417\\
162	0.00396247523042552\\
163	0.0039625491358899\\
164	0.00396262441170959\\
165	0.00396270108459674\\
166	0.00396277918181448\\
167	0.00396285873118826\\
168	0.00396293976111745\\
169	0.00396302230058718\\
170	0.00396310637918021\\
171	0.00396319202708932\\
172	0.0039632792751297\\
173	0.00396336815475165\\
174	0.00396345869805349\\
175	0.00396355093779486\\
176	0.00396364490740993\\
177	0.00396374064102128\\
178	0.00396383817345372\\
179	0.00396393754024847\\
180	0.00396403877767768\\
181	0.00396414192275912\\
182	0.0039642470132712\\
183	0.00396435408776826\\
184	0.00396446318559614\\
185	0.00396457434690808\\
186	0.00396468761268086\\
187	0.00396480302473139\\
188	0.00396492062573343\\
189	0.00396504045923474\\
190	0.00396516256967458\\
191	0.00396528700240149\\
192	0.00396541380369148\\
193	0.00396554302076652\\
194	0.00396567470181345\\
195	0.00396580889600322\\
196	0.00396594565351054\\
197	0.00396608502553389\\
198	0.00396622706431605\\
199	0.00396637182316486\\
200	0.00396651935647459\\
201	0.00396666971974754\\
202	0.00396682296961639\\
203	0.00396697916386666\\
204	0.00396713836145993\\
205	0.0039673006225573\\
206	0.00396746600854364\\
207	0.00396763458205207\\
208	0.00396780640698901\\
209	0.00396798154855998\\
210	0.0039681600732957\\
211	0.00396834204907883\\
212	0.00396852754517139\\
213	0.00396871663224261\\
214	0.00396890938239744\\
215	0.0039691058692058\\
216	0.00396930616773226\\
217	0.00396951035456654\\
218	0.00396971850785459\\
219	0.00396993070733036\\
220	0.00397014703434829\\
221	0.0039703675719165\\
222	0.00397059240473073\\
223	0.00397082161920898\\
224	0.00397105530352697\\
225	0.00397129354765433\\
226	0.00397153644339162\\
227	0.00397178408440814\\
228	0.00397203656628065\\
229	0.00397229398653273\\
230	0.00397255644467531\\
231	0.00397282404224783\\
232	0.00397309688286044\\
233	0.00397337507223707\\
234	0.00397365871825952\\
235	0.0039739479310123\\
236	0.00397424282282885\\
237	0.00397454350833825\\
238	0.0039748501045135\\
239	0.00397516273072029\\
240	0.00397548150876739\\
241	0.00397580656295768\\
242	0.0039761380201406\\
243	0.00397647600976556\\
244	0.00397682066393664\\
245	0.00397717211746841\\
246	0.00397753050794297\\
247	0.00397789597576834\\
248	0.003978268664238\\
249	0.00397864871959184\\
250	0.0039790362910784\\
251	0.00397943153101849\\
252	0.00397983459487025\\
253	0.0039802456412957\\
254	0.00398066483222859\\
255	0.00398109233294403\\
256	0.00398152831212948\\
257	0.00398197294195742\\
258	0.00398242639815966\\
259	0.00398288886010329\\
260	0.0039833605108685\\
261	0.00398384153732786\\
262	0.00398433213022782\\
263	0.00398483248427173\\
264	0.00398534279820492\\
265	0.00398586327490178\\
266	0.00398639412145473\\
267	0.00398693554926535\\
268	0.0039874877741375\\
269	0.00398805101637271\\
270	0.00398862550086776\\
271	0.00398921145721438\\
272	0.00398980911980147\\
273	0.00399041872791959\\
274	0.00399104052586768\\
275	0.00399167476306257\\
276	0.00399232169415058\\
277	0.00399298157912189\\
278	0.00399365468342665\\
279	0.00399434127809343\\
280	0.00399504163985089\\
281	0.0039957560512532\\
282	0.00399648480080847\\
283	0.0039972281831101\\
284	0.00399798649897121\\
285	0.00399876005556228\\
286	0.00399954916655213\\
287	0.00400035415225188\\
288	0.00400117533976269\\
289	0.0040020130631268\\
290	0.00400286766348202\\
291	0.00400373948922019\\
292	0.00400462889614912\\
293	0.0040055362476586\\
294	0.00400646191489025\\
295	0.00400740627691153\\
296	0.004008369720894\\
297	0.00400935264229589\\
298	0.00401035544504916\\
299	0.00401137854175129\\
300	0.00401242235386175\\
301	0.00401348731190372\\
302	0.0040145738556706\\
303	0.00401568243443828\\
304	0.00401681350718271\\
305	0.00401796754280328\\
306	0.00401914502035211\\
307	0.00402034642926983\\
308	0.00402157226962755\\
309	0.00402282305237587\\
310	0.00402409929960078\\
311	0.00402540154478704\\
312	0.00402673033308928\\
313	0.00402808622161128\\
314	0.00402946977969398\\
315	0.00403088158921293\\
316	0.00403232224488714\\
317	0.00403379235460287\\
318	0.00403529253974487\\
319	0.00403682343553007\\
320	0.0040383856913515\\
321	0.00403997997113249\\
322	0.00404160695369182\\
323	0.0040432673331202\\
324	0.00404496181916823\\
325	0.00404669113764653\\
326	0.00404845603083848\\
327	0.00405025725792615\\
328	0.0040520955954296\\
329	0.00405397183766052\\
330	0.00405588679719062\\
331	0.00405784130533546\\
332	0.00405983621265435\\
333	0.00406187238946691\\
334	0.00406395072638708\\
335	0.00406607213487544\\
336	0.00406823754781072\\
337	0.00407044792008143\\
338	0.00407270422919885\\
339	0.00407500747593193\\
340	0.00407735868496444\\
341	0.00407975890557602\\
342	0.00408220921234976\\
343	0.00408471070590678\\
344	0.00408726451366925\\
345	0.00408987179065329\\
346	0.00409253372029332\\
347	0.00409525151529924\\
348	0.00409802641854714\\
349	0.00410085970400705\\
350	0.00410375267770889\\
351	0.00410670667874872\\
352	0.00410972308033793\\
353	0.00411280329089896\\
354	0.00411594875520805\\
355	0.00411916095558662\\
356	0.00412244141314466\\
357	0.00412579168907941\\
358	0.00412921338603191\\
359	0.00413270814950452\\
360	0.0041362776693427\\
361	0.00413992368128507\\
362	0.00414364796858523\\
363	0.00414745236371024\\
364	0.00415133875012142\\
365	0.00415530906414446\\
366	0.00415936529693058\\
367	0.0041635094964848\\
368	0.00416774376979062\\
369	0.00417207028504644\\
370	0.00417649127400975\\
371	0.00418100903445991\\
372	0.00418562593280709\\
373	0.00419034440683301\\
374	0.00419516696855612\\
375	0.00420009620724449\\
376	0.00420513479258485\\
377	0.00421028547801785\\
378	0.0042155511042497\\
379	0.00422093460295191\\
380	0.00422643900066343\\
381	0.00423206742290854\\
382	0.00423782309855356\\
383	0.00424370936443308\\
384	0.0042497296702617\\
385	0.00425588758372511\\
386	0.00426218679580812\\
387	0.00426863112651516\\
388	0.00427522453098988\\
389	0.00428197110597659\\
390	0.00428887509684302\\
391	0.00429594090488185\\
392	0.00430317309500478\\
393	0.0043105764038424\\
394	0.00431815574826033\\
395	0.00432591623430028\\
396	0.0043338631665573\\
397	0.00434200205802946\\
398	0.0043503386405642\\
399	0.00435887887614446\\
400	0.00436762896824028\\
401	0.00437659537135883\\
402	0.00438578480136786\\
403	0.0043952042449959\\
404	0.00440486097482426\\
405	0.00441476256290941\\
406	0.00442491689482738\\
407	0.00443533218625601\\
408	0.00444601699533134\\
409	0.00445698024307586\\
410	0.00446823123569972\\
411	0.0044797796892213\\
412	0.00449163575809743\\
413	0.00450381007425677\\
414	0.00451631378139361\\
415	0.00452915855083714\\
416	0.00454235659688374\\
417	0.0045559207028187\\
418	0.00456986429629113\\
419	0.00458420148799741\\
420	0.00459894710593212\\
421	0.00461411671981738\\
422	0.00462972664797165\\
423	0.00464579394609775\\
424	0.00466233634044626\\
425	0.00467937197221599\\
426	0.00469691950764049\\
427	0.0047149980975147\\
428	0.00473362714628462\\
429	0.00475282570763295\\
430	0.00477261282404043\\
431	0.00479300762252036\\
432	0.00481402934237911\\
433	0.00483569743900408\\
434	0.00485803180366222\\
435	0.00488105315331135\\
436	0.00490478364174819\\
437	0.00492924758940434\\
438	0.00495447426193388\\
439	0.00498050972171603\\
440	0.0050074088097979\\
441	0.00503522025973298\\
442	0.00506399181217181\\
443	0.00509376848891395\\
444	0.00512459022536719\\
445	0.00515648863985227\\
446	0.00518948267405278\\
447	0.00522357274918004\\
448	0.00525873296887809\\
449	0.0052949008123832\\
450	0.00533196344552232\\
451	0.00536973970926647\\
452	0.00540780491526894\\
453	0.00544513164960255\\
454	0.00548160413227004\\
455	0.00551709878836009\\
456	0.00555148537765958\\
457	0.00558462892299446\\
458	0.00561639269307804\\
459	0.00564664270068478\\
460	0.00567525409529587\\
461	0.00570212134197286\\
462	0.0057271707917798\\
463	0.00575037869247606\\
464	0.00577179501858945\\
465	0.00579196631115484\\
466	0.00581161088317639\\
467	0.00583072395100217\\
468	0.00584930851683075\\
469	0.00586737670239619\\
470	0.0058849510561947\\
471	0.00590206572157465\\
472	0.00591876729326882\\
473	0.00593511512012822\\
474	0.00595118070046808\\
475	0.0059670456533217\\
476	0.00598279752866531\\
477	0.00599851500023344\\
478	0.0060142312473829\\
479	0.00602996287092893\\
480	0.0060457280007058\\
481	0.0060615459934407\\
482	0.006077437020876\\
483	0.00609342154158974\\
484	0.00610951965913373\\
485	0.00612575038568053\\
486	0.00614213085734679\\
487	0.00615867558796312\\
488	0.00617539590799603\\
489	0.00619230048075224\\
490	0.00620939706712024\\
491	0.00622669330845284\\
492	0.00624419664227171\\
493	0.00626191422403095\\
494	0.00627985286150051\\
495	0.00629801896944802\\
496	0.00631641855290834\\
497	0.00633505722697534\\
498	0.00635394027895524\\
499	0.00637307277368215\\
500	0.00639245969301985\\
501	0.00641210601372115\\
502	0.00643201671985407\\
503	0.00645219680927015\\
504	0.00647265130469233\\
505	0.0064933852695478\\
506	0.00651440382832887\\
507	0.0065357121908277\\
508	0.00655731567908339\\
509	0.00657921975536935\\
510	0.00660143004915995\\
511	0.00662395238098873\\
512	0.00664679278386983\\
513	0.00666995752547013\\
514	0.00669345313259201\\
515	0.00671728641798934\\
516	0.00674146450951709\\
517	0.00676599488160156\\
518	0.00679088538902184\\
519	0.00681614430301679\\
520	0.00684178034978294\\
521	0.00686780275150313\\
522	0.00689422127013374\\
523	0.00692104625418669\\
524	0.00694828868854222\\
525	0.00697596024712747\\
526	0.00700407334816857\\
527	0.00703264121155625\\
528	0.00706167791764577\\
529	0.00709119846652551\\
530	0.00712121883641764\\
531	0.00715175603939151\\
532	0.00718282817194962\\
533	0.00721445445724499\\
534	0.00724665527466789\\
535	0.00727945217125151\\
536	0.00731286784772672\\
537	0.00734692611002578\\
538	0.00738165177447921\\
539	0.00741707051178375\\
540	0.00745320861031404\\
541	0.00749009263365716\\
542	0.00752774894035735\\
543	0.00756620302093191\\
544	0.00760547867133597\\
545	0.00764559690202095\\
546	0.00768657544560344\\
547	0.00772841462132849\\
548	0.0077711062852949\\
549	0.00781463193735336\\
550	0.00785888938985476\\
551	0.00790390269594911\\
552	0.00794972965265113\\
553	0.00799644535724796\\
554	0.00804424767396752\\
555	0.0080932033127342\\
556	0.00814339871409729\\
557	0.00819497083172656\\
558	0.00824806852407355\\
559	0.00830122220873903\\
560	0.00835292013228987\\
561	0.00840249754434815\\
562	0.00845135966565546\\
563	0.00850010708446728\\
564	0.00854884919147689\\
565	0.00859762193092735\\
566	0.00864573461590588\\
567	0.00869275682914903\\
568	0.0087390226911887\\
569	0.00878512832909994\\
570	0.00883118017564783\\
571	0.0088771673823357\\
572	0.00892306508132109\\
573	0.00896858605394561\\
574	0.00901356557619045\\
575	0.0090584968281205\\
576	0.00910348269220726\\
577	0.00914850897773023\\
578	0.00919353728823878\\
579	0.009238523032971\\
580	0.00928341814856172\\
581	0.00932817169731824\\
582	0.00937273029909558\\
583	0.0094170385870981\\
584	0.00946103980882924\\
585	0.00950467664840494\\
586	0.00954789233109838\\
587	0.00959063207199068\\
588	0.00963284492697341\\
589	0.0096744860587673\\
590	0.00971551928305088\\
591	0.00975591931512916\\
592	0.00979567191274874\\
593	0.00983476678232329\\
594	0.00987312984006299\\
595	0.00991038882774525\\
596	0.00994573390547615\\
597	0.0099771937715668\\
598	0.00999970795535495\\
599	0\\
600	0\\
};
\addplot [color=red,solid,forget plot]
  table[row sep=crcr]{%
1	0.00395928562838554\\
2	0.0039592907711727\\
3	0.00395929601464776\\
4	0.00395930136068423\\
5	0.00395930681118812\\
6	0.00395931236809866\\
7	0.00395931803338862\\
8	0.0039593238090649\\
9	0.00395932969716917\\
10	0.00395933569977824\\
11	0.0039593418190048\\
12	0.00395934805699787\\
13	0.00395935441594341\\
14	0.00395936089806494\\
15	0.00395936750562407\\
16	0.00395937424092114\\
17	0.00395938110629588\\
18	0.0039593881041279\\
19	0.00395939523683741\\
20	0.00395940250688591\\
21	0.00395940991677666\\
22	0.00395941746905555\\
23	0.00395942516631167\\
24	0.00395943301117795\\
25	0.00395944100633191\\
26	0.00395944915449638\\
27	0.00395945745844014\\
28	0.00395946592097865\\
29	0.00395947454497488\\
30	0.00395948333333992\\
31	0.00395949228903388\\
32	0.00395950141506656\\
33	0.00395951071449829\\
34	0.00395952019044062\\
35	0.00395952984605729\\
36	0.00395953968456493\\
37	0.00395954970923394\\
38	0.00395955992338934\\
39	0.00395957033041161\\
40	0.00395958093373761\\
41	0.00395959173686141\\
42	0.00395960274333528\\
43	0.00395961395677051\\
44	0.00395962538083848\\
45	0.00395963701927149\\
46	0.0039596488758638\\
47	0.00395966095447259\\
48	0.00395967325901903\\
49	0.00395968579348925\\
50	0.00395969856193534\\
51	0.00395971156847653\\
52	0.00395972481730017\\
53	0.00395973831266285\\
54	0.00395975205889159\\
55	0.00395976606038494\\
56	0.0039597803216141\\
57	0.00395979484712418\\
58	0.0039598096415354\\
59	0.00395982470954423\\
60	0.00395984005592484\\
61	0.00395985568553015\\
62	0.0039598716032933\\
63	0.00395988781422897\\
64	0.00395990432343463\\
65	0.00395992113609207\\
66	0.00395993825746877\\
67	0.00395995569291926\\
68	0.00395997344788672\\
69	0.00395999152790442\\
70	0.00396000993859723\\
71	0.0039600286856833\\
72	0.00396004777497558\\
73	0.00396006721238343\\
74	0.00396008700391443\\
75	0.00396010715567588\\
76	0.00396012767387676\\
77	0.00396014856482946\\
78	0.00396016983495148\\
79	0.00396019149076748\\
80	0.0039602135389111\\
81	0.00396023598612686\\
82	0.00396025883927226\\
83	0.00396028210531983\\
84	0.00396030579135909\\
85	0.00396032990459887\\
86	0.00396035445236935\\
87	0.00396037944212437\\
88	0.00396040488144378\\
89	0.00396043077803566\\
90	0.00396045713973882\\
91	0.00396048397452527\\
92	0.0039605112905027\\
93	0.00396053909591705\\
94	0.00396056739915523\\
95	0.00396059620874778\\
96	0.00396062553337168\\
97	0.00396065538185319\\
98	0.00396068576317069\\
99	0.00396071668645779\\
100	0.00396074816100637\\
101	0.0039607801962697\\
102	0.00396081280186557\\
103	0.00396084598757979\\
104	0.0039608797633694\\
105	0.00396091413936617\\
106	0.00396094912588029\\
107	0.00396098473340384\\
108	0.00396102097261462\\
109	0.00396105785437995\\
110	0.00396109538976068\\
111	0.00396113359001513\\
112	0.00396117246660317\\
113	0.00396121203119068\\
114	0.00396125229565367\\
115	0.00396129327208287\\
116	0.00396133497278819\\
117	0.00396137741030352\\
118	0.00396142059739151\\
119	0.00396146454704841\\
120	0.00396150927250922\\
121	0.00396155478725283\\
122	0.00396160110500732\\
123	0.00396164823975543\\
124	0.00396169620574011\\
125	0.00396174501747029\\
126	0.00396179468972671\\
127	0.00396184523756794\\
128	0.00396189667633653\\
129	0.00396194902166531\\
130	0.00396200228948398\\
131	0.00396205649602557\\
132	0.00396211165783331\\
133	0.00396216779176769\\
134	0.00396222491501343\\
135	0.00396228304508681\\
136	0.00396234219984329\\
137	0.00396240239748507\\
138	0.0039624636565689\\
139	0.00396252599601417\\
140	0.00396258943511117\\
141	0.00396265399352939\\
142	0.00396271969132629\\
143	0.003962786548956\\
144	0.00396285458727841\\
145	0.00396292382756833\\
146	0.00396299429152501\\
147	0.00396306600128176\\
148	0.00396313897941584\\
149	0.00396321324895854\\
150	0.00396328883340551\\
151	0.00396336575672731\\
152	0.00396344404338017\\
153	0.0039635237183171\\
154	0.00396360480699906\\
155	0.00396368733540644\\
156	0.0039637713300509\\
157	0.0039638568179873\\
158	0.0039639438268259\\
159	0.00396403238474498\\
160	0.00396412252050344\\
161	0.00396421426345396\\
162	0.00396430764355618\\
163	0.0039644026913903\\
164	0.00396449943817088\\
165	0.00396459791576088\\
166	0.00396469815668607\\
167	0.00396480019414965\\
168	0.00396490406204721\\
169	0.00396500979498179\\
170	0.00396511742827964\\
171	0.0039652269980057\\
172	0.00396533854097983\\
173	0.00396545209479315\\
174	0.00396556769782472\\
175	0.00396568538925843\\
176	0.00396580520910035\\
177	0.00396592719819628\\
178	0.00396605139824954\\
179	0.00396617785183936\\
180	0.00396630660243916\\
181	0.00396643769443564\\
182	0.00396657117314766\\
183	0.00396670708484592\\
184	0.00396684547677272\\
185	0.00396698639716208\\
186	0.00396712989526034\\
187	0.0039672760213468\\
188	0.00396742482675508\\
189	0.0039675763638946\\
190	0.0039677306862725\\
191	0.00396788784851592\\
192	0.00396804790639455\\
193	0.00396821091684379\\
194	0.00396837693798812\\
195	0.0039685460291649\\
196	0.00396871825094857\\
197	0.0039688936651753\\
198	0.00396907233496811\\
199	0.00396925432476223\\
200	0.00396943970033113\\
201	0.00396962852881294\\
202	0.0039698208787372\\
203	0.00397001682005237\\
204	0.00397021642415344\\
205	0.00397041976391054\\
206	0.00397062691369768\\
207	0.00397083794942215\\
208	0.0039710529485546\\
209	0.00397127199015954\\
210	0.00397149515492657\\
211	0.00397172252520202\\
212	0.00397195418502149\\
213	0.00397219022014285\\
214	0.00397243071808008\\
215	0.0039726757681377\\
216	0.00397292546144579\\
217	0.00397317989099619\\
218	0.00397343915167897\\
219	0.00397370334032017\\
220	0.00397397255571993\\
221	0.00397424689869178\\
222	0.00397452647210274\\
223	0.00397481138091419\\
224	0.00397510173222396\\
225	0.00397539763530904\\
226	0.00397569920166957\\
227	0.00397600654507365\\
228	0.00397631978160335\\
229	0.00397663902970174\\
230	0.0039769644102209\\
231	0.00397729604647134\\
232	0.0039776340642723\\
233	0.0039779785920034\\
234	0.00397832976065741\\
235	0.00397868770389442\\
236	0.00397905255809703\\
237	0.00397942446242707\\
238	0.00397980355888353\\
239	0.00398018999236181\\
240	0.00398058391071445\\
241	0.00398098546481303\\
242	0.00398139480861178\\
243	0.00398181209921244\\
244	0.00398223749693064\\
245	0.00398267116536377\\
246	0.00398311327146042\\
247	0.00398356398559121\\
248	0.0039840234816214\\
249	0.00398449193698499\\
250	0.00398496953276036\\
251	0.00398545645374786\\
252	0.00398595288854874\\
253	0.00398645902964598\\
254	0.00398697507348706\\
255	0.00398750122056822\\
256	0.00398803767552061\\
257	0.00398858464719871\\
258	0.00398914234877018\\
259	0.00398971099780807\\
260	0.00399029081638484\\
261	0.00399088203116879\\
262	0.00399148487352214\\
263	0.00399209957960171\\
264	0.00399272639046191\\
265	0.00399336555215997\\
266	0.00399401731586366\\
267	0.00399468193796166\\
268	0.00399535968017629\\
269	0.00399605080967924\\
270	0.00399675559920986\\
271	0.00399747432719653\\
272	0.00399820727788084\\
273	0.00399895474144503\\
274	0.00399971701414254\\
275	0.00400049439843196\\
276	0.00400128720311434\\
277	0.00400209574347409\\
278	0.00400292034142352\\
279	0.00400376132565124\\
280	0.00400461903177454\\
281	0.00400549380249566\\
282	0.00400638598776239\\
283	0.00400729594493301\\
284	0.00400822403894548\\
285	0.0040091706424914\\
286	0.0040101361361946\\
287	0.00401112090879458\\
288	0.00401212535733477\\
289	0.00401314988735613\\
290	0.00401419491309591\\
291	0.00401526085769171\\
292	0.00401634815339112\\
293	0.004017457241767\\
294	0.00401858857393864\\
295	0.00401974261079879\\
296	0.0040209198232469\\
297	0.0040221206924286\\
298	0.00402334570998154\\
299	0.00402459537828786\\
300	0.00402587021073369\\
301	0.00402717073197557\\
302	0.00402849747821428\\
303	0.00402985099747633\\
304	0.00403123184990356\\
305	0.00403264060805057\\
306	0.0040340778571902\\
307	0.00403554419562737\\
308	0.00403704023502279\\
309	0.00403856660072659\\
310	0.00404012393212164\\
311	0.00404171288297786\\
312	0.00404333412181746\\
313	0.00404498833229179\\
314	0.00404667621357041\\
315	0.00404839848074283\\
316	0.00405015586523428\\
317	0.00405194911523509\\
318	0.00405377899614491\\
319	0.00405564629103263\\
320	0.00405755180111261\\
321	0.00405949634623842\\
322	0.00406148076541443\\
323	0.00406350591732685\\
324	0.00406557268089453\\
325	0.0040676819558404\\
326	0.00406983466328365\\
327	0.00407203174635561\\
328	0.00407427417083935\\
329	0.00407656292583373\\
330	0.00407889902444354\\
331	0.00408128350449685\\
332	0.00408371742929218\\
333	0.00408620188837577\\
334	0.00408873799834865\\
335	0.00409132690370618\\
336	0.0040939697777115\\
337	0.00409666782330487\\
338	0.00409942227405102\\
339	0.00410223439512729\\
340	0.00410510548435461\\
341	0.00410803687326293\\
342	0.00411102992819091\\
343	0.0041140860514452\\
344	0.00411720668250834\\
345	0.00412039329929913\\
346	0.00412364741948896\\
347	0.00412697060187794\\
348	0.00413036444783145\\
349	0.0041338306027698\\
350	0.00413737075773042\\
351	0.00414098665100229\\
352	0.00414468006983396\\
353	0.00414845285222457\\
354	0.00415230688882021\\
355	0.00415624412490494\\
356	0.00416026656247631\\
357	0.00416437626242426\\
358	0.0041685753468203\\
359	0.00417286600132388\\
360	0.00417725047771393\\
361	0.00418173109655441\\
362	0.00418631025000467\\
363	0.00419099040478617\\
364	0.00419577410532349\\
365	0.00420066397708865\\
366	0.00420566273020264\\
367	0.0042107731633396\\
368	0.00421599816758439\\
369	0.00422134073049449\\
370	0.00422680394048904\\
371	0.00423239099144796\\
372	0.00423810518753114\\
373	0.00424394994851576\\
374	0.0042499288154414\\
375	0.00425604545637169\\
376	0.00426230367243822\\
377	0.00426870740416766\\
378	0.00427526073808825\\
379	0.00428196791360697\\
380	0.00428883333014062\\
381	0.00429586155448043\\
382	0.00430305732837217\\
383	0.00431042557632768\\
384	0.00431797141379822\\
385	0.00432570015593234\\
386	0.00433361732537465\\
387	0.00434172865924067\\
388	0.00435004011691807\\
389	0.00435855788887977\\
390	0.00436728840590924\\
391	0.00437623835386574\\
392	0.00438541468866873\\
393	0.00439482465276074\\
394	0.00440447579325377\\
395	0.00441437598197943\\
396	0.00442453343766441\\
397	0.00443495675045583\\
398	0.00444565490911844\\
399	0.00445663733203324\\
400	0.00446791390676606\\
401	0.00447949503317697\\
402	0.00449139164201457\\
403	0.00450361522064092\\
404	0.00451617781124177\\
405	0.00452909206636683\\
406	0.0045423712713968\\
407	0.00455602933868359\\
408	0.00457008078322276\\
409	0.0045845404776838\\
410	0.00459942361483797\\
411	0.00461474566295675\\
412	0.00463052231443713\\
413	0.00464676943419859\\
414	0.0046635031225445\\
415	0.00468073975085366\\
416	0.00469849576919983\\
417	0.00471678747600711\\
418	0.00473563082646425\\
419	0.00475504198679849\\
420	0.00477503761567556\\
421	0.00479563534401681\\
422	0.00481685453896025\\
423	0.00483871745262209\\
424	0.00486125290232378\\
425	0.0048845032981711\\
426	0.00490851216879996\\
427	0.00493332536941689\\
428	0.00495899115142616\\
429	0.00498555865790608\\
430	0.0050130692192738\\
431	0.00504155731314359\\
432	0.00507104835382907\\
433	0.00510155417785983\\
434	0.00513306701456907\\
435	0.00516555145828589\\
436	0.00519893382830121\\
437	0.00523308810507563\\
438	0.00526781736556338\\
439	0.00530230611598739\\
440	0.00533614661794569\\
441	0.0053692354747199\\
442	0.00540146212332761\\
443	0.00543270970304172\\
444	0.00546285657450566\\
445	0.00549177886011247\\
446	0.00551935417542613\\
447	0.00554546706659329\\
448	0.00557001671615367\\
449	0.00559292758404455\\
450	0.00561416479378256\\
451	0.00563375464816682\\
452	0.00565196719975264\\
453	0.00566969520827555\\
454	0.00568693068663521\\
455	0.00570367118878301\\
456	0.00571992156845385\\
457	0.00573569516086255\\
458	0.00575101491884332\\
459	0.00576591439055096\\
460	0.00578043836880817\\
461	0.00579464296308082\\
462	0.00580859474280781\\
463	0.00582236847192784\\
464	0.0058360427354226\\
465	0.00584967793912761\\
466	0.00586329941713264\\
467	0.00587692197482909\\
468	0.0058905617956592\\
469	0.00590423618122079\\
470	0.00591796319562236\\
471	0.0059317612080084\\
472	0.00594564833497666\\
473	0.00595964179743497\\
474	0.00597375722733093\\
475	0.00598800799244652\\
476	0.0060024046573183\\
477	0.00601695506085117\\
478	0.00603166601973468\\
479	0.00604654425438506\\
480	0.0060615963131767\\
481	0.00607682850015123\\
482	0.00609224681136822\\
483	0.00610785688606976\\
484	0.006123663979571\\
485	0.00613967296488364\\
486	0.00615588836895017\\
487	0.00617231444619037\\
488	0.00618895528558699\\
489	0.00620581490958981\\
490	0.00622289730832914\\
491	0.00624020643976903\\
492	0.00625774623333333\\
493	0.00627552059720033\\
494	0.0062935334292164\\
495	0.00631178863105106\\
496	0.00633029012481586\\
497	0.00634904187092652\\
498	0.0063680478855733\\
499	0.0063873122559112\\
500	0.00640683915122753\\
501	0.0064266328321379\\
502	0.00644669766046583\\
503	0.00646703811018547\\
504	0.00648765877938356\\
505	0.0065085644031848\\
506	0.00652975986758595\\
507	0.00655125022416271\\
508	0.00657304070565595\\
509	0.00659513674251247\\
510	0.00661754398055078\\
511	0.00664026830002966\\
512	0.00666331583640131\\
513	0.00668669300292377\\
514	0.00671040651525138\\
515	0.00673446341812398\\
516	0.00675887111427401\\
517	0.00678363739566562\\
518	0.0068087704771671\\
519	0.00683427903273516\\
520	0.00686017223415273\\
521	0.00688645979230361\\
522	0.00691315200088348\\
523	0.00694025978233218\\
524	0.00696779473562672\\
525	0.00699576918539001\\
526	0.0070241962315293\\
527	0.00705308979830633\\
528	0.00708246468134048\\
529	0.00711233659053203\\
530	0.00714272218623722\\
531	0.0071736391051927\\
532	0.00720510597162972\\
533	0.00723714238767914\\
534	0.0072697688954951\\
535	0.00730300690141519\\
536	0.00733687854989586\\
537	0.00737140653083732\\
538	0.00740661379942434\\
539	0.0074425231780382\\
540	0.00747915683284826\\
541	0.00751653561730621\\
542	0.00755467826988819\\
543	0.00759360095216047\\
544	0.00763331238082942\\
545	0.00767381095643044\\
546	0.007715025844359\\
547	0.0077569285435327\\
548	0.0077995714378718\\
549	0.00784302124100872\\
550	0.00788745658537758\\
551	0.00793295460546916\\
552	0.00797958458081587\\
553	0.00802743516966814\\
554	0.00807661277614705\\
555	0.00812725401186005\\
556	0.00817887856530284\\
557	0.0082291129550715\\
558	0.00827735421505801\\
559	0.0083245490587445\\
560	0.00837172262866089\\
561	0.00841901681308049\\
562	0.0084664844422458\\
563	0.00851408246692262\\
564	0.00856102546121805\\
565	0.00860690495770123\\
566	0.00865232480719928\\
567	0.00869771432457174\\
568	0.00874313870516814\\
569	0.00878858866114348\\
570	0.0088340308653561\\
571	0.00887936447173519\\
572	0.00892418360391415\\
573	0.00896884111831802\\
574	0.0090136165177729\\
575	0.00905850904675177\\
576	0.0091034868427784\\
577	0.0091485107227293\\
578	0.00919353811604447\\
579	0.00923852346941265\\
580	0.00928341839068734\\
581	0.0093281718340145\\
582	0.00937273037639527\\
583	0.0094170386295485\\
584	0.00946103983027298\\
585	0.00950467665772169\\
586	0.0095478923342439\\
587	0.00959063207272952\\
588	0.00963284492704104\\
589	0.0096744860587673\\
590	0.00971551928305088\\
591	0.00975591931512916\\
592	0.00979567191274874\\
593	0.00983476678232329\\
594	0.00987312984006299\\
595	0.00991038882774525\\
596	0.00994573390547615\\
597	0.0099771937715668\\
598	0.00999970795535495\\
599	0\\
600	0\\
};
\addplot [color=mycolor20,solid,forget plot]
  table[row sep=crcr]{%
1	0.00395948440225352\\
2	0.00395949092226044\\
3	0.00395949757490838\\
4	0.00395950436277181\\
5	0.00395951128847175\\
6	0.00395951835467654\\
7	0.00395952556410272\\
8	0.00395953291951572\\
9	0.00395954042373065\\
10	0.00395954807961318\\
11	0.00395955589008015\\
12	0.00395956385810068\\
13	0.00395957198669679\\
14	0.00395958027894421\\
15	0.00395958873797342\\
16	0.00395959736697037\\
17	0.00395960616917737\\
18	0.00395961514789404\\
19	0.00395962430647817\\
20	0.00395963364834655\\
21	0.00395964317697609\\
22	0.00395965289590453\\
23	0.00395966280873147\\
24	0.00395967291911941\\
25	0.0039596832307946\\
26	0.00395969374754803\\
27	0.00395970447323651\\
28	0.00395971541178364\\
29	0.00395972656718072\\
30	0.00395973794348798\\
31	0.00395974954483542\\
32	0.003959761375424\\
33	0.00395977343952666\\
34	0.00395978574148945\\
35	0.00395979828573252\\
36	0.00395981107675135\\
37	0.0039598241191178\\
38	0.00395983741748127\\
39	0.00395985097656992\\
40	0.00395986480119162\\
41	0.00395987889623541\\
42	0.00395989326667255\\
43	0.0039599079175577\\
44	0.00395992285403025\\
45	0.00395993808131548\\
46	0.00395995360472583\\
47	0.00395996942966224\\
48	0.00395998556161533\\
49	0.00396000200616676\\
50	0.00396001876899063\\
51	0.00396003585585468\\
52	0.00396005327262173\\
53	0.00396007102525107\\
54	0.00396008911979978\\
55	0.00396010756242421\\
56	0.0039601263593814\\
57	0.00396014551703048\\
58	0.00396016504183415\\
59	0.00396018494036029\\
60	0.00396020521928323\\
61	0.00396022588538556\\
62	0.00396024694555945\\
63	0.00396026840680833\\
64	0.00396029027624851\\
65	0.00396031256111069\\
66	0.0039603352687417\\
67	0.00396035840660617\\
68	0.00396038198228812\\
69	0.00396040600349278\\
70	0.00396043047804829\\
71	0.00396045541390743\\
72	0.0039604808191495\\
73	0.00396050670198204\\
74	0.00396053307074279\\
75	0.00396055993390153\\
76	0.00396058730006192\\
77	0.00396061517796358\\
78	0.00396064357648399\\
79	0.00396067250464044\\
80	0.0039607019715922\\
81	0.00396073198664259\\
82	0.003960762559241\\
83	0.00396079369898509\\
84	0.00396082541562309\\
85	0.00396085771905589\\
86	0.00396089061933936\\
87	0.00396092412668679\\
88	0.00396095825147106\\
89	0.00396099300422723\\
90	0.00396102839565496\\
91	0.00396106443662097\\
92	0.00396110113816164\\
93	0.00396113851148566\\
94	0.00396117656797667\\
95	0.00396121531919602\\
96	0.00396125477688555\\
97	0.0039612949529704\\
98	0.00396133585956208\\
99	0.00396137750896132\\
100	0.00396141991366111\\
101	0.00396146308634993\\
102	0.00396150703991493\\
103	0.00396155178744514\\
104	0.00396159734223496\\
105	0.0039616437177874\\
106	0.00396169092781778\\
107	0.00396173898625735\\
108	0.00396178790725683\\
109	0.00396183770519044\\
110	0.00396188839465959\\
111	0.00396193999049702\\
112	0.00396199250777091\\
113	0.00396204596178899\\
114	0.00396210036810296\\
115	0.00396215574251297\\
116	0.00396221210107214\\
117	0.00396226946009127\\
118	0.00396232783614365\\
119	0.00396238724607009\\
120	0.00396244770698396\\
121	0.00396250923627652\\
122	0.00396257185162219\\
123	0.00396263557098424\\
124	0.00396270041262049\\
125	0.00396276639508912\\
126	0.00396283353725475\\
127	0.00396290185829468\\
128	0.00396297137770534\\
129	0.00396304211530887\\
130	0.00396311409125984\\
131	0.00396318732605236\\
132	0.00396326184052723\\
133	0.00396333765587933\\
134	0.00396341479366529\\
135	0.00396349327581143\\
136	0.00396357312462168\\
137	0.00396365436278606\\
138	0.00396373701338924\\
139	0.00396382109991931\\
140	0.003963906646277\\
141	0.00396399367678494\\
142	0.00396408221619734\\
143	0.00396417228970994\\
144	0.00396426392297022\\
145	0.00396435714208795\\
146	0.00396445197364597\\
147	0.00396454844471137\\
148	0.00396464658284689\\
149	0.0039647464161228\\
150	0.003964847973129\\
151	0.00396495128298749\\
152	0.00396505637536522\\
153	0.0039651632804872\\
154	0.00396527202915017\\
155	0.0039653826527365\\
156	0.00396549518322848\\
157	0.00396560965322306\\
158	0.00396572609594695\\
159	0.00396584454527215\\
160	0.00396596503573188\\
161	0.00396608760253688\\
162	0.00396621228159229\\
163	0.00396633910951471\\
164	0.00396646812364995\\
165	0.00396659936209115\\
166	0.00396673286369724\\
167	0.00396686866811202\\
168	0.00396700681578351\\
169	0.00396714734798408\\
170	0.00396729030683059\\
171	0.00396743573530554\\
172	0.00396758367727822\\
173	0.00396773417752669\\
174	0.00396788728176\\
175	0.00396804303664116\\
176	0.00396820148981032\\
177	0.00396836268990858\\
178	0.00396852668660243\\
179	0.00396869353060839\\
180	0.0039688632737184\\
181	0.00396903596882557\\
182	0.00396921166995054\\
183	0.00396939043226819\\
184	0.00396957231213497\\
185	0.00396975736711667\\
186	0.00396994565601665\\
187	0.00397013723890466\\
188	0.00397033217714608\\
189	0.00397053053343162\\
190	0.00397073237180753\\
191	0.00397093775770633\\
192	0.00397114675797796\\
193	0.00397135944092138\\
194	0.0039715758763167\\
195	0.00397179613545778\\
196	0.00397202029118517\\
197	0.00397224841791972\\
198	0.00397248059169632\\
199	0.00397271689019849\\
200	0.00397295739279313\\
201	0.00397320218056577\\
202	0.00397345133635636\\
203	0.00397370494479541\\
204	0.00397396309234069\\
205	0.0039742258673142\\
206	0.00397449335993978\\
207	0.00397476566238107\\
208	0.00397504286878006\\
209	0.00397532507529596\\
210	0.00397561238014456\\
211	0.00397590488363834\\
212	0.00397620268822686\\
213	0.00397650589853769\\
214	0.00397681462141795\\
215	0.00397712896597649\\
216	0.0039774490436266\\
217	0.00397777496812941\\
218	0.00397810685563781\\
219	0.00397844482474121\\
220	0.00397878899651109\\
221	0.0039791394945472\\
222	0.0039794964450246\\
223	0.00397985997674165\\
224	0.00398023022116887\\
225	0.00398060731249876\\
226	0.00398099138769676\\
227	0.00398138258655321\\
228	0.0039817810517365\\
229	0.00398218692884742\\
230	0.00398260036647489\\
231	0.00398302151625291\\
232	0.00398345053291904\\
233	0.00398388757437447\\
234	0.00398433280174537\\
235	0.00398478637944622\\
236	0.00398524847524468\\
237	0.00398571926032826\\
238	0.0039861989093728\\
239	0.00398668760061308\\
240	0.00398718551591517\\
241	0.00398769284085103\\
242	0.0039882097647751\\
243	0.0039887364809031\\
244	0.00398927318639304\\
245	0.00398982008242858\\
246	0.00399037737430452\\
247	0.00399094527151491\\
248	0.0039915239878433\\
249	0.00399211374145552\\
250	0.00399271475499502\\
251	0.00399332725568042\\
252	0.00399395147540588\\
253	0.00399458765084366\\
254	0.00399523602354945\\
255	0.00399589684007014\\
256	0.00399657035205428\\
257	0.00399725681636487\\
258	0.00399795649519504\\
259	0.00399866965618633\\
260	0.00399939657254961\\
261	0.00400013752318873\\
262	0.00400089279282669\\
263	0.00400166267213369\\
264	0.00400244745785881\\
265	0.00400324745296446\\
266	0.00400406296676403\\
267	0.00400489431506205\\
268	0.00400574182029793\\
269	0.00400660581169249\\
270	0.00400748662539804\\
271	0.00400838460465164\\
272	0.00400930009993214\\
273	0.00401023346912073\\
274	0.00401118507766548\\
275	0.00401215529874979\\
276	0.00401314451346527\\
277	0.00401415311098896\\
278	0.00401518148876539\\
279	0.00401623005269339\\
280	0.0040172992173184\\
281	0.00401838940603019\\
282	0.00401950105126649\\
283	0.00402063459472275\\
284	0.00402179048756856\\
285	0.00402296919067083\\
286	0.00402417117482429\\
287	0.00402539692098974\\
288	0.00402664692054038\\
289	0.00402792167551637\\
290	0.00402922169888853\\
291	0.00403054751483122\\
292	0.00403189965900493\\
293	0.00403327867884902\\
294	0.00403468513388493\\
295	0.00403611959603054\\
296	0.00403758264992584\\
297	0.00403907489327053\\
298	0.00404059693717357\\
299	0.0040421494065143\\
300	0.00404373294031623\\
301	0.00404534819213448\\
302	0.00404699583045704\\
303	0.00404867653911984\\
304	0.00405039101773675\\
305	0.00405213998214593\\
306	0.00405392416487165\\
307	0.00405574431559477\\
308	0.00405760120163317\\
309	0.00405949560844524\\
310	0.00406142834014913\\
311	0.00406340022005844\\
312	0.00406541209123613\\
313	0.0040674648170667\\
314	0.00406955928184599\\
315	0.00407169639139072\\
316	0.00407387707366838\\
317	0.0040761022794489\\
318	0.00407837298297942\\
319	0.00408069018268334\\
320	0.0040830549018849\\
321	0.00408546818956152\\
322	0.00408793112112511\\
323	0.00409044479923499\\
324	0.00409301035464481\\
325	0.00409562894708605\\
326	0.00409830176618714\\
327	0.00410103003241982\\
328	0.00410381499809582\\
329	0.00410665794840662\\
330	0.00410956020250733\\
331	0.00411252311464709\\
332	0.00411554807535199\\
333	0.00411863651267666\\
334	0.00412178989352366\\
335	0.00412500972501356\\
336	0.0041282975559241\\
337	0.00413165497820438\\
338	0.00413508362857238\\
339	0.00413858519020842\\
340	0.00414216139456864\\
341	0.00414581402334682\\
342	0.00414954491048962\\
343	0.00415335594420658\\
344	0.00415724906924379\\
345	0.00416122628927412\\
346	0.00416528966941353\\
347	0.00416944133887728\\
348	0.00417368349379616\\
349	0.00417801840018998\\
350	0.00418244839695065\\
351	0.00418697589901957\\
352	0.00419160340072899\\
353	0.00419633347925919\\
354	0.0042011687982348\\
355	0.00420611211169295\\
356	0.00421116626827441\\
357	0.0042163342154261\\
358	0.00422161900375006\\
359	0.00422702379147992\\
360	0.00423255184905963\\
361	0.00423820656379134\\
362	0.00424399144451153\\
363	0.00424991012625111\\
364	0.00425596637483545\\
365	0.00426216409140471\\
366	0.00426850731693583\\
367	0.00427500023721365\\
368	0.00428164718949901\\
369	0.00428845266672812\\
370	0.00429542132320961\\
371	0.00430255798251003\\
372	0.00430986764586282\\
373	0.0043173555005868\\
374	0.00432502693245215\\
375	0.00433288754011261\\
376	0.0043409431491925\\
377	0.00434919982807049\\
378	0.00435766390559816\\
379	0.0043663419909905\\
380	0.0043752409961034\\
381	0.00438436816024748\\
382	0.00439373107756318\\
383	0.00440333772678778\\
384	0.00441319650309811\\
385	0.00442331625241044\\
386	0.00443370631256225\\
387	0.0044443765401406\\
388	0.00445533731535654\\
389	0.00446659953479586\\
390	0.00447817458043723\\
391	0.00449007413241409\\
392	0.00450231015283447\\
393	0.00451489485466701\\
394	0.00452784066455827\\
395	0.00454116018000212\\
396	0.00455486612197178\\
397	0.00456897128510276\\
398	0.00458348848880526\\
399	0.00459843053440882\\
400	0.00461381017938842\\
401	0.00462964021231349\\
402	0.00464593365578756\\
403	0.00466270372799138\\
404	0.00467996405031999\\
405	0.00469772858203319\\
406	0.00471601260218599\\
407	0.00473483384719461\\
408	0.00475421481069545\\
409	0.00477419142017473\\
410	0.00479480144025236\\
411	0.0048160842116969\\
412	0.00483808022758728\\
413	0.00486083046362899\\
414	0.00488437519333277\\
415	0.00490875381440622\\
416	0.00493400459263738\\
417	0.00496016133776183\\
418	0.00498724818555258\\
419	0.00501527111677807\\
420	0.00504421592126254\\
421	0.00507404009149779\\
422	0.00510466208601866\\
423	0.00513594713979381\\
424	0.00516755735354668\\
425	0.00519871705977875\\
426	0.00522934403357216\\
427	0.00525934920615354\\
428	0.00528863680824175\\
429	0.00531710488513274\\
430	0.00534464641881114\\
431	0.00537115119851654\\
432	0.00539650857693217\\
433	0.005420611496474\\
434	0.00544336227568943\\
435	0.0054646808461695\\
436	0.00548451611396553\\
437	0.00550286155930679\\
438	0.00551977644864224\\
439	0.0055359496527542\\
440	0.00555167583479705\\
441	0.00556694691578936\\
442	0.00558176016901808\\
443	0.00559611929440776\\
444	0.00561003529505826\\
445	0.00562352682800227\\
446	0.00563662179526778\\
447	0.00564935798415351\\
448	0.00566178325469104\\
449	0.00567395498949571\\
450	0.00568593839927066\\
451	0.00569780311094088\\
452	0.00570961156569204\\
453	0.00572138941486999\\
454	0.00573314893998529\\
455	0.00574490374263054\\
456	0.00575666857816241\\
457	0.00576845910233506\\
458	0.00578029153242253\\
459	0.00579218221910404\\
460	0.00580414713343451\\
461	0.00581620128659636\\
462	0.00582835812063487\\
463	0.00584062893949173\\
464	0.00585302249683192\\
465	0.00586554548556271\\
466	0.00587820392422858\\
467	0.00589100374351042\\
468	0.00590395071910557\\
469	0.00591705040727884\\
470	0.00593030808745506\\
471	0.00594372871708371\\
472	0.00595731690465106\\
473	0.00597107690688334\\
474	0.00598501265543397\\
475	0.00599912781599312\\
476	0.00601342587775641\\
477	0.00602791025067907\\
478	0.00604258430320072\\
479	0.0060574513602477\\
480	0.00607251470399852\\
481	0.00608777757761638\\
482	0.00610324319197112\\
483	0.00611891473512403\\
484	0.00613479538403393\\
485	0.00615088831758091\\
486	0.0061671967296386\\
487	0.00618372384064446\\
488	0.00620047290607825\\
489	0.00621744722177797\\
490	0.00623465012840185\\
491	0.00625208501643772\\
492	0.00626975533169465\\
493	0.0062876645811952\\
494	0.00630581633937912\\
495	0.00632421425453326\\
496	0.00634286205538325\\
497	0.006361763557825\\
498	0.00638092267183858\\
499	0.00640034340871236\\
500	0.00642002988879402\\
501	0.00643998634992998\\
502	0.00646021715666276\\
503	0.00648072681024703\\
504	0.00650151995955475\\
505	0.0065226014129498\\
506	0.00654397615122305\\
507	0.00656564934168963\\
508	0.0065876263535596\\
509	0.00660991277469965\\
510	0.00663251442990518\\
511	0.00665543740079762\\
512	0.00667868804745377\\
513	0.00670227303186607\\
514	0.00672619934332242\\
515	0.00675047432577689\\
516	0.00677510570725678\\
517	0.00680010163131477\\
518	0.00682547069048237\\
519	0.00685122196161199\\
520	0.00687736504289977\\
521	0.00690391009225805\\
522	0.00693086786654527\\
523	0.00695824976095131\\
524	0.00698606784756793\\
525	0.007014334911828\\
526	0.0070430644850565\\
527	0.00707227087081502\\
528	0.00710196916200891\\
529	0.00713217524482474\\
530	0.00716290578442511\\
531	0.00719417818590626\\
532	0.0072260105222357\\
533	0.00725842141872715\\
534	0.00729142987987181\\
535	0.00732505504062249\\
536	0.00735931581314322\\
537	0.00739423044572161\\
538	0.00742981597050931\\
539	0.00746608786335217\\
540	0.00750305817500797\\
541	0.00754073159948061\\
542	0.00757909781928992\\
543	0.00761806654300156\\
544	0.00765769760109196\\
545	0.00769805034120224\\
546	0.00773925844101698\\
547	0.00778143801340248\\
548	0.00782464807394301\\
549	0.00786896435375635\\
550	0.00791445734509906\\
551	0.00796120660613397\\
552	0.00800933710507936\\
553	0.00805900257046526\\
554	0.0081081314115405\\
555	0.00815551517192216\\
556	0.00820105298443216\\
557	0.00824664627952422\\
558	0.00829240598531143\\
559	0.00833844424324396\\
560	0.00838475291434543\\
561	0.00843125792622228\\
562	0.00847722774953948\\
563	0.00852215806893428\\
564	0.00856674750998803\\
565	0.00861139972405399\\
566	0.00865616050071849\\
567	0.0087010135091651\\
568	0.00874592342992904\\
569	0.00879085950398949\\
570	0.00883551799252388\\
571	0.00887980405825824\\
572	0.00892423816410628\\
573	0.00896885007363452\\
574	0.00901361838983874\\
575	0.00905850969697006\\
576	0.00910348712146237\\
577	0.0091485108571932\\
578	0.00919353818752705\\
579	0.00923852350916325\\
580	0.00928341841310162\\
581	0.00932817184660867\\
582	0.00937273038322628\\
583	0.00941703863294158\\
584	0.00946103983171765\\
585	0.00950467665819966\\
586	0.00954789233435323\\
587	0.00959063207273936\\
588	0.00963284492704104\\
589	0.0096744860587673\\
590	0.00971551928305087\\
591	0.00975591931512916\\
592	0.00979567191274874\\
593	0.00983476678232329\\
594	0.00987312984006299\\
595	0.00991038882774525\\
596	0.00994573390547615\\
597	0.0099771937715668\\
598	0.00999970795535495\\
599	0\\
600	0\\
};
\addplot [color=mycolor21,solid,forget plot]
  table[row sep=crcr]{%
1	0.0039595779923929\\
2	0.00395958604364415\\
3	0.00395959426546396\\
4	0.00395960266133263\\
5	0.00395961123479755\\
6	0.00395961998947425\\
7	0.00395962892904768\\
8	0.00395963805727339\\
9	0.00395964737797872\\
10	0.00395965689506409\\
11	0.00395966661250426\\
12	0.00395967653434954\\
13	0.00395968666472715\\
14	0.00395969700784251\\
15	0.00395970756798054\\
16	0.00395971834950707\\
17	0.00395972935687016\\
18	0.0039597405946015\\
19	0.00395975206731777\\
20	0.00395976377972216\\
21	0.00395977573660569\\
22	0.00395978794284877\\
23	0.00395980040342258\\
24	0.00395981312339065\\
25	0.00395982610791038\\
26	0.00395983936223449\\
27	0.00395985289171264\\
28	0.00395986670179298\\
29	0.00395988079802378\\
30	0.00395989518605494\\
31	0.00395990987163979\\
32	0.00395992486063655\\
33	0.00395994015901015\\
34	0.00395995577283383\\
35	0.0039599717082909\\
36	0.00395998797167647\\
37	0.00396000456939915\\
38	0.00396002150798283\\
39	0.0039600387940685\\
40	0.00396005643441613\\
41	0.00396007443590634\\
42	0.0039600928055423\\
43	0.00396011155045166\\
44	0.0039601306778884\\
45	0.00396015019523475\\
46	0.00396017011000306\\
47	0.00396019042983783\\
48	0.00396021116251764\\
49	0.00396023231595714\\
50	0.00396025389820902\\
51	0.00396027591746612\\
52	0.00396029838206338\\
53	0.00396032130047999\\
54	0.00396034468134141\\
55	0.00396036853342153\\
56	0.00396039286564475\\
57	0.00396041768708813\\
58	0.00396044300698361\\
59	0.00396046883472013\\
60	0.00396049517984587\\
61	0.0039605220520704\\
62	0.00396054946126702\\
63	0.00396057741747498\\
64	0.00396060593090174\\
65	0.00396063501192532\\
66	0.00396066467109651\\
67	0.00396069491914137\\
68	0.00396072576696344\\
69	0.00396075722564619\\
70	0.00396078930645542\\
71	0.00396082202084159\\
72	0.00396085538044239\\
73	0.00396088939708511\\
74	0.00396092408278909\\
75	0.00396095944976823\\
76	0.00396099551043356\\
77	0.00396103227739571\\
78	0.00396106976346744\\
79	0.00396110798166629\\
80	0.00396114694521711\\
81	0.00396118666755466\\
82	0.0039612271623263\\
83	0.00396126844339455\\
84	0.00396131052483993\\
85	0.00396135342096341\\
86	0.00396139714628938\\
87	0.00396144171556819\\
88	0.00396148714377905\\
89	0.0039615334461327\\
90	0.00396158063807436\\
91	0.00396162873528638\\
92	0.00396167775369127\\
93	0.0039617277094545\\
94	0.00396177861898743\\
95	0.00396183049895023\\
96	0.0039618833662549\\
97	0.00396193723806822\\
98	0.00396199213181484\\
99	0.00396204806518025\\
100	0.00396210505611397\\
101	0.00396216312283263\\
102	0.00396222228382321\\
103	0.00396228255784615\\
104	0.00396234396393865\\
105	0.00396240652141803\\
106	0.00396247024988497\\
107	0.0039625351692269\\
108	0.00396260129962157\\
109	0.00396266866154032\\
110	0.00396273727575188\\
111	0.00396280716332576\\
112	0.00396287834563605\\
113	0.00396295084436504\\
114	0.00396302468150717\\
115	0.0039630998793727\\
116	0.00396317646059185\\
117	0.00396325444811873\\
118	0.00396333386523545\\
119	0.00396341473555636\\
120	0.00396349708303233\\
121	0.00396358093195515\\
122	0.00396366630696207\\
123	0.00396375323304033\\
124	0.00396384173553193\\
125	0.00396393184013854\\
126	0.00396402357292645\\
127	0.00396411696033168\\
128	0.00396421202916528\\
129	0.00396430880661874\\
130	0.00396440732026953\\
131	0.00396450759808701\\
132	0.00396460966843818\\
133	0.00396471356009394\\
134	0.00396481930223533\\
135	0.00396492692446012\\
136	0.00396503645678954\\
137	0.00396514792967523\\
138	0.00396526137400653\\
139	0.00396537682111794\\
140	0.0039654943027968\\
141	0.00396561385129142\\
142	0.00396573549931932\\
143	0.00396585928007585\\
144	0.0039659852272432\\
145	0.00396611337499961\\
146	0.00396624375802904\\
147	0.00396637641153117\\
148	0.00396651137123178\\
149	0.00396664867339348\\
150	0.00396678835482697\\
151	0.00396693045290267\\
152	0.00396707500556277\\
153	0.00396722205133386\\
154	0.00396737162933989\\
155	0.00396752377931578\\
156	0.0039676785416215\\
157	0.00396783595725674\\
158	0.00396799606787609\\
159	0.00396815891580486\\
160	0.00396832454405552\\
161	0.00396849299634472\\
162	0.00396866431711102\\
163	0.0039688385515333\\
164	0.00396901574554988\\
165	0.00396919594587829\\
166	0.00396937920003592\\
167	0.00396956555636119\\
168	0.00396975506403595\\
169	0.00396994777310817\\
170	0.00397014373451592\\
171	0.00397034300011188\\
172	0.00397054562268897\\
173	0.00397075165600669\\
174	0.00397096115481857\\
175	0.0039711741749004\\
176	0.00397139077307949\\
177	0.00397161100726493\\
178	0.00397183493647885\\
179	0.00397206262088863\\
180	0.00397229412184027\\
181	0.00397252950189264\\
182	0.00397276882485293\\
183	0.00397301215581311\\
184	0.00397325956118748\\
185	0.00397351110875133\\
186	0.00397376686768065\\
187	0.00397402690859303\\
188	0.00397429130358959\\
189	0.003974560126298\\
190	0.00397483345191671\\
191	0.00397511135726011\\
192	0.00397539392080506\\
193	0.00397568122273813\\
194	0.00397597334500428\\
195	0.00397627037135636\\
196	0.00397657238740573\\
197	0.00397687948067391\\
198	0.00397719174064521\\
199	0.00397750925882038\\
200	0.00397783212877106\\
201	0.00397816044619533\\
202	0.00397849430897405\\
203	0.003978833817228\\
204	0.00397917907337592\\
205	0.00397953018219317\\
206	0.00397988725087129\\
207	0.0039802503890781\\
208	0.00398061970901842\\
209	0.00398099532549547\\
210	0.00398137735597267\\
211	0.00398176592063605\\
212	0.003982161142457\\
213	0.00398256314725538\\
214	0.0039829720637631\\
215	0.00398338802368782\\
216	0.00398381116177707\\
217	0.00398424161588237\\
218	0.00398467952702383\\
219	0.00398512503945458\\
220	0.00398557830072549\\
221	0.00398603946175003\\
222	0.0039865086768691\\
223	0.00398698610391609\\
224	0.00398747190428195\\
225	0.00398796624298054\\
226	0.00398846928871391\\
227	0.003988981213938\\
228	0.00398950219492847\\
229	0.00399003241184696\\
230	0.00399057204880772\\
231	0.00399112129394473\\
232	0.00399168033947957\\
233	0.00399224938178988\\
234	0.00399282862147892\\
235	0.0039934182634461\\
236	0.00399401851695871\\
237	0.00399462959572526\\
238	0.00399525171797025\\
239	0.00399588510651104\\
240	0.0039965299888365\\
241	0.00399718659718832\\
242	0.00399785516864467\\
243	0.0039985359452067\\
244	0.00399922917388823\\
245	0.00399993510680861\\
246	0.0040006540012893\\
247	0.00400138611995427\\
248	0.00400213173083437\\
249	0.00400289110747621\\
250	0.00400366452905542\\
251	0.00400445228049482\\
252	0.00400525465258755\\
253	0.00400607194212532\\
254	0.00400690445203214\\
255	0.0040077524915033\\
256	0.00400861637615048\\
257	0.00400949642815204\\
258	0.00401039297640977\\
259	0.00401130635671139\\
260	0.00401223691189968\\
261	0.00401318499204848\\
262	0.00401415095464581\\
263	0.00401513516478111\\
264	0.00401613799532841\\
265	0.00401715982714415\\
266	0.00401820104927012\\
267	0.00401926205914198\\
268	0.00402034326280314\\
269	0.00402144507512384\\
270	0.0040225679200258\\
271	0.00402371223071199\\
272	0.00402487844990196\\
273	0.00402606703007211\\
274	0.00402727843370165\\
275	0.00402851313352362\\
276	0.00402977161278135\\
277	0.00403105436549025\\
278	0.00403236189670511\\
279	0.00403369472279303\\
280	0.00403505337171201\\
281	0.00403643838329548\\
282	0.00403785030954319\\
283	0.00403928971491835\\
284	0.00404075717665173\\
285	0.0040422532850531\\
286	0.00404377864383033\\
287	0.00404533387041699\\
288	0.00404691959630863\\
289	0.00404853646740905\\
290	0.00405018514438676\\
291	0.00405186630304276\\
292	0.00405358063469048\\
293	0.00405532884654885\\
294	0.00405711166214952\\
295	0.00405892982175949\\
296	0.00406078408282071\\
297	0.00406267522040868\\
298	0.00406460402771199\\
299	0.00406657131653281\\
300	0.00406857791780154\\
301	0.00407062468211074\\
302	0.00407271248028372\\
303	0.00407484220397271\\
304	0.00407701476629087\\
305	0.00407923110248637\\
306	0.00408149217067528\\
307	0.00408379895264865\\
308	0.00408615245466971\\
309	0.00408855370825715\\
310	0.00409100377110257\\
311	0.00409350372802893\\
312	0.0040960546919937\\
313	0.00409865780514698\\
314	0.00410131423994946\\
315	0.00410402520032743\\
316	0.00410679192288071\\
317	0.00410961567814609\\
318	0.00411249777191874\\
319	0.00411543954663444\\
320	0.00411844238281576\\
321	0.00412150770058498\\
322	0.00412463696124773\\
323	0.00412783166895115\\
324	0.00413109337242268\\
325	0.00413442366680018\\
326	0.00413782419556743\\
327	0.00414129665258353\\
328	0.00414484278404729\\
329	0.00414846439064309\\
330	0.00415216332975872\\
331	0.00415594151775714\\
332	0.00415980093227251\\
333	0.00416374361452242\\
334	0.00416777167176996\\
335	0.00417188727990086\\
336	0.0041760926858126\\
337	0.00418039020972506\\
338	0.00418478224736796\\
339	0.00418927127200665\\
340	0.00419385983630205\\
341	0.00419855057414037\\
342	0.00420334620293092\\
343	0.00420824952551619\\
344	0.00421326343108173\\
345	0.00421839089884848\\
346	0.00422363500205102\\
347	0.00422899891226\\
348	0.00423448590416088\\
349	0.00424009936103744\\
350	0.00424584278125957\\
351	0.00425171978374354\\
352	0.00425773411458507\\
353	0.00426388965471359\\
354	0.00427019042807163\\
355	0.00427664061022658\\
356	0.00428324454059173\\
357	0.00429000673728246\\
358	0.00429693191220291\\
359	0.00430402498825019\\
360	0.00431129111883286\\
361	0.00431873570982133\\
362	0.00432636444389655\\
363	0.00433418330700377\\
364	0.00434219861619004\\
365	0.00435041704742423\\
366	0.00435884566096149\\
367	0.00436749192047545\\
368	0.00437636370227233\\
369	0.0043854693070619\\
370	0.00439481733945904\\
371	0.00440441665298403\\
372	0.00441427632905858\\
373	0.00442440564977241\\
374	0.00443481404823458\\
375	0.0044455110846419\\
376	0.00445650643044544\\
377	0.00446780982669083\\
378	0.0044794310382552\\
379	0.00449137980585808\\
380	0.0045036657990461\\
381	0.00451629857524127\\
382	0.00452928755259884\\
383	0.00454264200807975\\
384	0.00455637111704512\\
385	0.00457048405738908\\
386	0.00458499021645212\\
387	0.00459989966492497\\
388	0.0046152237597833\\
389	0.00463097589253995\\
390	0.00464717271358021\\
391	0.00466384263704998\\
392	0.00468101605901409\\
393	0.00469872520378388\\
394	0.00471700401685769\\
395	0.0047358879622072\\
396	0.00475541368900525\\
397	0.00477561852173576\\
398	0.00479653971262268\\
399	0.00481821337350695\\
400	0.00484067296282239\\
401	0.00486394705273863\\
402	0.00488805700361895\\
403	0.00491301544051242\\
404	0.00493881990764823\\
405	0.00496544710475881\\
406	0.00499283797409538\\
407	0.00502088860658765\\
408	0.0050493821509829\\
409	0.00507758634680132\\
410	0.00510543767386323\\
411	0.0051328667390106\\
412	0.00515979817508445\\
413	0.00518615071814976\\
414	0.00521183754728799\\
415	0.00523676696628491\\
416	0.00526084355604888\\
417	0.00528397003915966\\
418	0.00530605018083721\\
419	0.00532699312582212\\
420	0.00534671949017847\\
421	0.00536516976463816\\
422	0.00538231592593325\\
423	0.0053981773002428\\
424	0.00541297614030603\\
425	0.00542738618063539\\
426	0.00544139449392937\\
427	0.00545499164732757\\
428	0.00546817248068545\\
429	0.00548093692074996\\
430	0.00549329120131213\\
431	0.00550524851055586\\
432	0.00551682983204708\\
433	0.00552806467282283\\
434	0.00553899123328039\\
435	0.00554965577016223\\
436	0.00556011217760635\\
437	0.00557041976927555\\
438	0.0055806392335274\\
439	0.0055908064169165\\
440	0.00560093578277078\\
441	0.0056110380886052\\
442	0.00562112525464462\\
443	0.00563121021799256\\
444	0.00564130672562363\\
445	0.00565142906153454\\
446	0.00566159169729224\\
447	0.00567180885636397\\
448	0.0056820940158248\\
449	0.00569245937415726\\
450	0.00570291533896705\\
451	0.00571347012593252\\
452	0.00572412983063176\\
453	0.00573489977471728\\
454	0.00574578523871036\\
455	0.00575679140386329\\
456	0.0057679232939351\\
457	0.00577918572038726\\
458	0.00579058323498188\\
459	0.00580212009446097\\
460	0.00581380024243527\\
461	0.00582562731353743\\
462	0.0058376046639026\\
463	0.00584973542950027\\
464	0.00586202260878462\\
465	0.00587446913550529\\
466	0.00588707790151843\\
467	0.00589985175451819\\
468	0.00591279349809862\\
469	0.00592590589432517\\
470	0.00593919166884342\\
471	0.00595265351834822\\
472	0.00596629411997525\\
473	0.00598011614187412\\
474	0.00599412225390815\\
475	0.00600831513716732\\
476	0.00602269749089862\\
477	0.00603727203620461\\
478	0.00605204151852332\\
479	0.00606700871045744\\
480	0.00608217641488946\\
481	0.00609754746830291\\
482	0.00611312474422012\\
483	0.00612891115666342\\
484	0.00614490966355826\\
485	0.00616112327002145\\
486	0.00617755503152311\\
487	0.00619420805697205\\
488	0.00621108551184649\\
489	0.00622819062151137\\
490	0.0062455266747875\\
491	0.0062630970277878\\
492	0.0062809051080401\\
493	0.00629895441892298\\
494	0.00631724854444748\\
495	0.0063357911544256\\
496	0.00635458601007473\\
497	0.00637363697011413\\
498	0.00639294799741503\\
499	0.00641252316626832\\
500	0.00643236667033256\\
501	0.00645248283132593\\
502	0.00647287610852873\\
503	0.00649355110916894\\
504	0.00651451259976566\\
505	0.00653576551850926\\
506	0.00655731498875853\\
507	0.00657916633373493\\
508	0.0066013250924929\\
509	0.00662379703723859\\
510	0.00664658819206253\\
511	0.00666970485313681\\
512	0.00669315361040883\\
513	0.00671694137079573\\
514	0.00674107538284555\\
515	0.00676556326277949\\
516	0.00679041302176111\\
517	0.006815633094147\\
518	0.00684123236635596\\
519	0.0068672202058387\\
520	0.00689360648943386\\
521	0.00692040163014122\\
522	0.00694761660101961\\
523	0.0069752629545051\\
524	0.0070033528349217\\
525	0.00703189898129568\\
526	0.00706091471674855\\
527	0.00709041391968895\\
528	0.0071204109707071\\
529	0.00715092066742947\\
530	0.00718195809760305\\
531	0.00721353845722821\\
532	0.00724567679700885\\
533	0.00727838766877932\\
534	0.00731168469820507\\
535	0.00734558005552928\\
536	0.00738008435104086\\
537	0.00741520385143412\\
538	0.00745093704720049\\
539	0.00748723869970754\\
540	0.00752407422169693\\
541	0.0075615108104564\\
542	0.00759963420054334\\
543	0.00763863727699673\\
544	0.00767856978708636\\
545	0.00771949303797662\\
546	0.00776147010594045\\
547	0.00780456248016654\\
548	0.00784884021700096\\
549	0.00789439433353439\\
550	0.00794135701250253\\
551	0.00798960912808038\\
552	0.00803637094793776\\
553	0.00808097007159319\\
554	0.00812496792776143\\
555	0.00816909891588888\\
556	0.00821358388140258\\
557	0.00825844490010318\\
558	0.00830363112481854\\
559	0.00834907049081818\\
560	0.00839423278867568\\
561	0.00843838050845754\\
562	0.00848214519909988\\
563	0.00852603665066133\\
564	0.00857009983934468\\
565	0.00861431729034347\\
566	0.0086586558688049\\
567	0.00870308161489387\\
568	0.0087475675617029\\
569	0.00879162789679202\\
570	0.00883561106037741\\
571	0.00887981727907396\\
572	0.00892423949543059\\
573	0.00896885035521918\\
574	0.00901361849050742\\
575	0.00905850974103567\\
576	0.00910348714308695\\
577	0.00914851086878221\\
578	0.0091935381939862\\
579	0.00923852351279987\\
580	0.00928341841513161\\
581	0.00932817184769639\\
582	0.0093727303837578\\
583	0.00941703863316352\\
584	0.00946103983178964\\
585	0.00950467665821572\\
586	0.00954789233435466\\
587	0.00959063207273936\\
588	0.00963284492704104\\
589	0.0096744860587673\\
590	0.00971551928305088\\
591	0.00975591931512916\\
592	0.00979567191274874\\
593	0.00983476678232329\\
594	0.00987312984006299\\
595	0.00991038882774525\\
596	0.00994573390547615\\
597	0.0099771937715668\\
598	0.00999970795535495\\
599	0\\
600	0\\
};
\addplot [color=black!20!mycolor21,solid,forget plot]
  table[row sep=crcr]{%
1	0.00395964375163837\\
2	0.00395965334766334\\
3	0.00395966315493915\\
4	0.00395967317799734\\
5	0.00395968342146281\\
6	0.00395969389005562\\
7	0.00395970458859279\\
8	0.00395971552199028\\
9	0.00395972669526469\\
10	0.00395973811353531\\
11	0.00395974978202612\\
12	0.00395976170606758\\
13	0.00395977389109891\\
14	0.00395978634266988\\
15	0.0039597990664431\\
16	0.00395981206819604\\
17	0.00395982535382315\\
18	0.00395983892933814\\
19	0.0039598528008762\\
20	0.0039598669746961\\
21	0.00395988145718269\\
22	0.0039598962548491\\
23	0.00395991137433913\\
24	0.00395992682242971\\
25	0.00395994260603318\\
26	0.00395995873219995\\
27	0.00395997520812093\\
28	0.00395999204113003\\
29	0.00396000923870677\\
30	0.00396002680847904\\
31	0.00396004475822548\\
32	0.00396006309587852\\
33	0.00396008182952683\\
34	0.00396010096741822\\
35	0.0039601205179625\\
36	0.00396014048973427\\
37	0.00396016089147582\\
38	0.00396018173210012\\
39	0.00396020302069375\\
40	0.00396022476651985\\
41	0.00396024697902132\\
42	0.00396026966782382\\
43	0.00396029284273896\\
44	0.00396031651376742\\
45	0.00396034069110217\\
46	0.00396036538513182\\
47	0.00396039060644379\\
48	0.00396041636582772\\
49	0.00396044267427887\\
50	0.00396046954300149\\
51	0.00396049698341223\\
52	0.0039605250071438\\
53	0.00396055362604837\\
54	0.00396058285220117\\
55	0.00396061269790415\\
56	0.00396064317568966\\
57	0.00396067429832404\\
58	0.00396070607881142\\
59	0.00396073853039752\\
60	0.00396077166657339\\
61	0.00396080550107938\\
62	0.00396084004790885\\
63	0.0039608753213123\\
64	0.00396091133580114\\
65	0.00396094810615178\\
66	0.00396098564740967\\
67	0.00396102397489332\\
68	0.00396106310419847\\
69	0.00396110305120211\\
70	0.00396114383206675\\
71	0.00396118546324461\\
72	0.00396122796148171\\
73	0.00396127134382233\\
74	0.00396131562761313\\
75	0.00396136083050761\\
76	0.0039614069704703\\
77	0.00396145406578124\\
78	0.00396150213504033\\
79	0.00396155119717176\\
80	0.00396160127142844\\
81	0.0039616523773965\\
82	0.00396170453499971\\
83	0.00396175776450402\\
84	0.00396181208652208\\
85	0.00396186752201777\\
86	0.00396192409231075\\
87	0.00396198181908104\\
88	0.00396204072437357\\
89	0.00396210083060279\\
90	0.00396216216055722\\
91	0.00396222473740417\\
92	0.00396228858469422\\
93	0.00396235372636587\\
94	0.0039624201867502\\
95	0.00396248799057547\\
96	0.00396255716297167\\
97	0.00396262772947524\\
98	0.00396269971603358\\
99	0.00396277314900974\\
100	0.00396284805518691\\
101	0.00396292446177313\\
102	0.00396300239640572\\
103	0.00396308188715596\\
104	0.00396316296253354\\
105	0.00396324565149113\\
106	0.00396332998342891\\
107	0.00396341598819903\\
108	0.00396350369610999\\
109	0.00396359313793127\\
110	0.00396368434489753\\
111	0.00396377734871314\\
112	0.00396387218155648\\
113	0.00396396887608433\\
114	0.00396406746543607\\
115	0.00396416798323804\\
116	0.00396427046360772\\
117	0.00396437494115797\\
118	0.00396448145100114\\
119	0.00396459002875327\\
120	0.00396470071053813\\
121	0.00396481353299136\\
122	0.00396492853326446\\
123	0.00396504574902881\\
124	0.00396516521847957\\
125	0.00396528698033972\\
126	0.00396541107386386\\
127	0.00396553753884221\\
128	0.00396566641560436\\
129	0.00396579774502315\\
130	0.00396593156851853\\
131	0.00396606792806125\\
132	0.00396620686617672\\
133	0.00396634842594879\\
134	0.00396649265102352\\
135	0.00396663958561298\\
136	0.003966789274499\\
137	0.00396694176303706\\
138	0.00396709709716001\\
139	0.00396725532338214\\
140	0.00396741648880294\\
141	0.00396758064111113\\
142	0.0039677478285888\\
143	0.0039679181001154\\
144	0.00396809150517207\\
145	0.0039682680938459\\
146	0.00396844791683451\\
147	0.00396863102545051\\
148	0.00396881747162637\\
149	0.00396900730791938\\
150	0.00396920058751668\\
151	0.00396939736424077\\
152	0.00396959769255505\\
153	0.00396980162756976\\
154	0.00397000922504807\\
155	0.00397022054141285\\
156	0.00397043563375329\\
157	0.00397065455983234\\
158	0.00397087737809437\\
159	0.00397110414767336\\
160	0.00397133492840156\\
161	0.00397156978081874\\
162	0.00397180876618193\\
163	0.00397205194647591\\
164	0.00397229938442428\\
165	0.00397255114350131\\
166	0.00397280728794447\\
167	0.00397306788276805\\
168	0.00397333299377718\\
169	0.00397360268758339\\
170	0.0039738770316206\\
171	0.00397415609416259\\
172	0.00397443994434141\\
173	0.00397472865216697\\
174	0.0039750222885479\\
175	0.00397532092531391\\
176	0.00397562463523926\\
177	0.00397593349206792\\
178	0.00397624757054026\\
179	0.00397656694642131\\
180	0.00397689169653071\\
181	0.00397722189877456\\
182	0.0039775576321791\\
183	0.00397789897692628\\
184	0.00397824601439143\\
185	0.00397859882718305\\
186	0.00397895749918482\\
187	0.00397932211559987\\
188	0.00397969276299737\\
189	0.00398006952936187\\
190	0.00398045250414462\\
191	0.00398084177831814\\
192	0.00398123744443297\\
193	0.00398163959667756\\
194	0.00398204833094071\\
195	0.00398246374487709\\
196	0.00398288593797575\\
197	0.00398331501163142\\
198	0.00398375106921916\\
199	0.00398419421617194\\
200	0.00398464456006148\\
201	0.00398510221068217\\
202	0.00398556728013829\\
203	0.00398603988293433\\
204	0.00398652013606856\\
205	0.00398700815912995\\
206	0.00398750407439797\\
207	0.0039880080069457\\
208	0.00398852008474601\\
209	0.00398904043878078\\
210	0.00398956920315294\\
211	0.00399010651520159\\
212	0.00399065251561945\\
213	0.00399120734857345\\
214	0.00399177116182738\\
215	0.00399234410686703\\
216	0.00399292633902742\\
217	0.00399351801762209\\
218	0.00399411930607387\\
219	0.00399473037204754\\
220	0.00399535138758358\\
221	0.00399598252923296\\
222	0.00399662397819282\\
223	0.00399727592044269\\
224	0.00399793854688094\\
225	0.00399861205346116\\
226	0.00399929664132846\\
227	0.003999992516955\\
228	0.00400069989227484\\
229	0.00400141898481762\\
230	0.00400215001784093\\
231	0.00400289322046105\\
232	0.00400364882778198\\
233	0.00400441708102244\\
234	0.00400519822764072\\
235	0.00400599252145714\\
236	0.00400680022277436\\
237	0.00400762159849498\\
238	0.00400845692223696\\
239	0.00400930647444638\\
240	0.00401017054250808\\
241	0.00401104942085394\\
242	0.00401194341106922\\
243	0.00401285282199717\\
244	0.00401377796984214\\
245	0.00401471917827151\\
246	0.00401567677851693\\
247	0.00401665110947515\\
248	0.0040176425178092\\
249	0.00401865135804993\\
250	0.004019677992699\\
251	0.00402072279233356\\
252	0.00402178613571327\\
253	0.00402286840989043\\
254	0.00402397001032393\\
255	0.00402509134099774\\
256	0.0040262328145449\\
257	0.00402739485237837\\
258	0.00402857788482955\\
259	0.00402978235129683\\
260	0.00403100870040635\\
261	0.00403225739018976\\
262	0.00403352888828749\\
263	0.00403482367218943\\
264	0.00403614222949775\\
265	0.0040374850580941\\
266	0.00403885266642762\\
267	0.00404024557381834\\
268	0.00404166431077653\\
269	0.00404310941933892\\
270	0.00404458145342213\\
271	0.00404608097919428\\
272	0.00404760857546515\\
273	0.00404916483409596\\
274	0.00405075036042868\\
275	0.00405236577373619\\
276	0.00405401170769354\\
277	0.00405568881087088\\
278	0.00405739774724882\\
279	0.00405913919675669\\
280	0.00406091385583455\\
281	0.00406272243801931\\
282	0.0040645656745559\\
283	0.00406644431503428\\
284	0.004068359128053\\
285	0.00407031090191012\\
286	0.00407230044532286\\
287	0.00407432858817622\\
288	0.00407639618230251\\
289	0.00407850410229188\\
290	0.00408065324633533\\
291	0.00408284453710049\\
292	0.00408507892264047\\
293	0.00408735737733538\\
294	0.00408968090286549\\
295	0.00409205052921337\\
296	0.00409446731569204\\
297	0.00409693235199474\\
298	0.00409944675926425\\
299	0.00410201169118099\\
300	0.00410462833505344\\
301	0.00410729791278782\\
302	0.00411002168171443\\
303	0.00411280093540496\\
304	0.00411563700435423\\
305	0.00411853125649502\\
306	0.0041214850975437\\
307	0.00412449997128679\\
308	0.0041275773601546\\
309	0.00413071878522574\\
310	0.00413392580577175\\
311	0.00413720002038836\\
312	0.00414054306816125\\
313	0.00414395662987838\\
314	0.00414744242938025\\
315	0.00415100223514791\\
316	0.00415463786182732\\
317	0.00415835117185727\\
318	0.00416214407721573\\
319	0.00416601854130468\\
320	0.00416997658099756\\
321	0.00417402026887962\\
322	0.00417815173571903\\
323	0.00418237317321583\\
324	0.00418668683708922\\
325	0.00419109505058904\\
326	0.00419560020857598\\
327	0.00420020478245296\\
328	0.00420491132624466\\
329	0.00420972248165688\\
330	0.00421464098528164\\
331	0.0042196696769845\\
332	0.00422481150958617\\
333	0.00423006955981566\\
334	0.00423544704048069\\
335	0.00424094731580934\\
336	0.0042465739205694\\
337	0.00425233057956957\\
338	0.00425822122893274\\
339	0.00426425003844867\\
340	0.00427042143368057\\
341	0.004276740115593\\
342	0.00428321107475444\\
343	0.00428983960159714\\
344	0.0042966312552854\\
345	0.0043035917806431\\
346	0.00431072710551373\\
347	0.00431804333603102\\
348	0.00432554674944444\\
349	0.00433324378435994\\
350	0.00434114102969718\\
351	0.00434924521904846\\
352	0.0043575632011128\\
353	0.00436610191015336\\
354	0.00437486833446154\\
355	0.0043838694757048\\
356	0.00439311228965902\\
357	0.00440260364595343\\
358	0.00441235030060594\\
359	0.00442235884979655\\
360	0.00443263568766444\\
361	0.00444318697423391\\
362	0.0044540186225565\\
363	0.0044651363182996\\
364	0.00447654559072606\\
365	0.00448825196180528\\
366	0.00450026121066164\\
367	0.00451257980410656\\
368	0.00452521555989052\\
369	0.00453817862473681\\
370	0.00455148762462533\\
371	0.00456516460123408\\
372	0.00457923330368143\\
373	0.0045937192018997\\
374	0.00460864948914859\\
375	0.0046240527664869\\
376	0.00463995899883539\\
377	0.00465639960751397\\
378	0.00467340722466519\\
379	0.00469101530764693\\
380	0.00470925756498926\\
381	0.00472816713042042\\
382	0.00474777540177328\\
383	0.00476811043563431\\
384	0.00478919475461603\\
385	0.00481104237623984\\
386	0.00483365479123261\\
387	0.00485701536763692\\
388	0.00488108356522784\\
389	0.00490578651630885\\
390	0.00493100584675148\\
391	0.0049561050571859\\
392	0.00498102513730994\\
393	0.00500571703987493\\
394	0.00503012688162011\\
395	0.00505419574628497\\
396	0.00507785958478666\\
397	0.00510104926199273\\
398	0.0051236908171125\\
399	0.00514570602802744\\
400	0.00516701340075003\\
401	0.00518752974652354\\
402	0.00520717254565949\\
403	0.00522586334149184\\
404	0.00524353259235539\\
405	0.00526012649631629\\
406	0.00527561655099673\\
407	0.00529001263926006\\
408	0.00530343563728342\\
409	0.00531653331693652\\
410	0.00532929093024752\\
411	0.00534169586148326\\
412	0.00535373820700891\\
413	0.00536541142763497\\
414	0.00537671304665619\\
415	0.00538764540158245\\
416	0.00539821639072226\\
417	0.00540844024765905\\
418	0.0054183381926166\\
419	0.00542793885684519\\
420	0.00543727842000505\\
421	0.00544640026552884\\
422	0.00545535349965632\\
423	0.00546419035229756\\
424	0.0054729565826893\\
425	0.00548166983600437\\
426	0.0054903379177783\\
427	0.00549896976093038\\
428	0.00550757536486095\\
429	0.00551616568808048\\
430	0.00552475248546753\\
431	0.00553334808417513\\
432	0.00554196509939006\\
433	0.00555061608625226\\
434	0.00555931313637169\\
435	0.00556806743863489\\
436	0.00557688883323977\\
437	0.00558578541253118\\
438	0.00559476327897139\\
439	0.00560382735036957\\
440	0.00561298229679523\\
441	0.00562223275505113\\
442	0.00563158327885558\\
443	0.00564103828900022\\
444	0.00565060202589308\\
445	0.00566027850746577\\
446	0.00567007149644886\\
447	0.00567998448170495\\
448	0.00569002067801832\\
449	0.00570018304808882\\
450	0.00571047434862302\\
451	0.00572089719862039\\
452	0.00573145415290164\\
453	0.00574214772961181\\
454	0.00575298040693971\\
455	0.0057639546216638\\
456	0.00577507276974429\\
457	0.00578633720907419\\
458	0.00579775026435995\\
459	0.00580931423391469\\
460	0.00582103139791401\\
461	0.00583290402740387\\
462	0.00584493439309434\\
463	0.0058571247727867\\
464	0.00586947745628239\\
465	0.00588199474784275\\
466	0.005894678967924\\
467	0.00590753245516925\\
468	0.00592055756859933\\
469	0.00593375668992994\\
470	0.0059471322259331\\
471	0.00596068661075727\\
472	0.00597442230812716\\
473	0.00598834181336401\\
474	0.00600244765520103\\
475	0.00601674239741815\\
476	0.00603122864038028\\
477	0.00604590902259925\\
478	0.00606078622237516\\
479	0.00607586295951612\\
480	0.00609114199713636\\
481	0.0061066261435377\\
482	0.00612231825418213\\
483	0.00613822123376957\\
484	0.00615433803843859\\
485	0.00617067167811424\\
486	0.00618722521902902\\
487	0.00620400178644644\\
488	0.00622100456761418\\
489	0.00623823681497238\\
490	0.00625570184964363\\
491	0.00627340306523325\\
492	0.00629134393197285\\
493	0.00630952800124148\\
494	0.00632795891050328\\
495	0.00634664038870283\\
496	0.00636557626216234\\
497	0.0063847704610277\\
498	0.00640422702631292\\
499	0.00642395011759481\\
500	0.00644394402141205\\
501	0.00646421316042427\\
502	0.00648476210338836\\
503	0.00650559557600805\\
504	0.00652671847271206\\
505	0.00654813586941164\\
506	0.00656985303728164\\
507	0.00659187545759923\\
508	0.0066142088376587\\
509	0.00663685912776054\\
510	0.00665983253924229\\
511	0.0066831355634807\\
512	0.0067067749917419\\
513	0.00673075793568732\\
514	0.00675509184825398\\
515	0.0067797845445121\\
516	0.00680484422195338\\
517	0.00683027947947309\\
518	0.00685609933406564\\
519	0.00688231323394464\\
520	0.00690893106640585\\
521	0.00693596315825738\\
522	0.0069634202660161\\
523	0.0069913135522828\\
524	0.0070196545437194\\
525	0.00704845506480828\\
526	0.00707772714003761\\
527	0.0071074828552847\\
528	0.00713773416616653\\
529	0.00716849263744683\\
530	0.00719976908686505\\
531	0.00723157315713454\\
532	0.00726391279674431\\
533	0.00729679425204214\\
534	0.00733021895576369\\
535	0.00736418027017706\\
536	0.00739861050727177\\
537	0.00743350659545982\\
538	0.00746893739763071\\
539	0.00750502999446856\\
540	0.00754192715135258\\
541	0.00757969109057501\\
542	0.00761839596233787\\
543	0.00765809882046372\\
544	0.00769885502590327\\
545	0.0077407273894827\\
546	0.00778378744306659\\
547	0.00782813502988159\\
548	0.00787391113933587\\
549	0.00792022278387469\\
550	0.00796483486832117\\
551	0.00800724731079562\\
552	0.0080497803406812\\
553	0.00809262919199778\\
554	0.00813591216228803\\
555	0.00817962687777911\\
556	0.00822371846461619\\
557	0.00826811691466122\\
558	0.00831262787810168\\
559	0.00835614799688869\\
560	0.00839908507898075\\
561	0.00844218055789354\\
562	0.00848549450963725\\
563	0.008529013494506\\
564	0.00857270633871265\\
565	0.00861654004187527\\
566	0.00866048475963239\\
567	0.00870443353843061\\
568	0.00874795159195782\\
569	0.00879165244561201\\
570	0.00883561303603307\\
571	0.0088798174691251\\
572	0.00892423953733394\\
573	0.0089688503706678\\
574	0.00901361849741638\\
575	0.00905850974448163\\
576	0.00910348714494772\\
577	0.00914851086982138\\
578	0.00919353819457018\\
579	0.00923852351312367\\
580	0.00928341841530305\\
581	0.00932817184777885\\
582	0.00937273038379159\\
583	0.00941703863317428\\
584	0.00946103983179199\\
585	0.00950467665821593\\
586	0.00954789233435466\\
587	0.00959063207273936\\
588	0.00963284492704104\\
589	0.0096744860587673\\
590	0.00971551928305088\\
591	0.00975591931512916\\
592	0.00979567191274874\\
593	0.00983476678232329\\
594	0.00987312984006299\\
595	0.00991038882774525\\
596	0.00994573390547615\\
597	0.0099771937715668\\
598	0.00999970795535495\\
599	0\\
600	0\\
};
\addplot [color=black!50!mycolor20,solid,forget plot]
  table[row sep=crcr]{%
1	0.00395969554893294\\
2	0.00395970653642755\\
3	0.00395971777385418\\
4	0.00395972926681449\\
5	0.00395974102103281\\
6	0.00395975304235881\\
7	0.00395976533677007\\
8	0.00395977791037461\\
9	0.00395979076941396\\
10	0.00395980392026571\\
11	0.00395981736944642\\
12	0.00395983112361459\\
13	0.00395984518957348\\
14	0.00395985957427428\\
15	0.00395987428481911\\
16	0.00395988932846415\\
17	0.00395990471262283\\
18	0.00395992044486914\\
19	0.00395993653294086\\
20	0.00395995298474307\\
21	0.00395996980835143\\
22	0.00395998701201582\\
23	0.00396000460416384\\
24	0.0039600225934045\\
25	0.00396004098853185\\
26	0.00396005979852891\\
27	0.00396007903257125\\
28	0.00396009870003126\\
29	0.00396011881048184\\
30	0.00396013937370056\\
31	0.00396016039967391\\
32	0.00396018189860124\\
33	0.00396020388089935\\
34	0.00396022635720663\\
35	0.00396024933838761\\
36	0.00396027283553743\\
37	0.00396029685998643\\
38	0.00396032142330488\\
39	0.00396034653730766\\
40	0.00396037221405919\\
41	0.00396039846587827\\
42	0.00396042530534317\\
43	0.00396045274529659\\
44	0.00396048079885099\\
45	0.0039605094793938\\
46	0.0039605388005927\\
47	0.00396056877640119\\
48	0.00396059942106404\\
49	0.0039606307491229\\
50	0.00396066277542204\\
51	0.00396069551511427\\
52	0.00396072898366659\\
53	0.00396076319686642\\
54	0.00396079817082748\\
55	0.0039608339219962\\
56	0.00396087046715776\\
57	0.00396090782344264\\
58	0.00396094600833301\\
59	0.00396098503966929\\
60	0.00396102493565688\\
61	0.00396106571487285\\
62	0.0039611073962728\\
63	0.00396114999919789\\
64	0.00396119354338179\\
65	0.00396123804895797\\
66	0.00396128353646685\\
67	0.00396133002686319\\
68	0.00396137754152353\\
69	0.00396142610225383\\
70	0.00396147573129697\\
71	0.0039615264513406\\
72	0.00396157828552509\\
73	0.00396163125745124\\
74	0.00396168539118849\\
75	0.00396174071128306\\
76	0.00396179724276621\\
77	0.00396185501116245\\
78	0.00396191404249807\\
79	0.00396197436330974\\
80	0.00396203600065297\\
81	0.00396209898211091\\
82	0.00396216333580316\\
83	0.00396222909039467\\
84	0.00396229627510462\\
85	0.00396236491971563\\
86	0.00396243505458278\\
87	0.00396250671064294\\
88	0.00396257991942401\\
89	0.00396265471305432\\
90	0.00396273112427213\\
91	0.00396280918643512\\
92	0.00396288893352997\\
93	0.00396297040018213\\
94	0.00396305362166542\\
95	0.00396313863391186\\
96	0.00396322547352158\\
97	0.0039633141777726\\
98	0.00396340478463087\\
99	0.00396349733276019\\
100	0.00396359186153224\\
101	0.00396368841103668\\
102	0.00396378702209112\\
103	0.00396388773625135\\
104	0.00396399059582142\\
105	0.00396409564386375\\
106	0.00396420292420921\\
107	0.00396431248146741\\
108	0.00396442436103667\\
109	0.00396453860911422\\
110	0.00396465527270628\\
111	0.00396477439963812\\
112	0.00396489603856402\\
113	0.00396502023897732\\
114	0.00396514705122035\\
115	0.00396527652649425\\
116	0.00396540871686871\\
117	0.0039655436752918\\
118	0.00396568145559948\\
119	0.0039658221125252\\
120	0.0039659657017092\\
121	0.00396611227970794\\
122	0.00396626190400309\\
123	0.00396641463301061\\
124	0.00396657052608967\\
125	0.00396672964355128\\
126	0.00396689204666676\\
127	0.00396705779767616\\
128	0.00396722695979628\\
129	0.00396739959722869\\
130	0.00396757577516734\\
131	0.00396775555980596\\
132	0.00396793901834539\\
133	0.00396812621900038\\
134	0.00396831723100634\\
135	0.00396851212462558\\
136	0.00396871097115349\\
137	0.0039689138429242\\
138	0.00396912081331608\\
139	0.00396933195675666\\
140	0.00396954734872749\\
141	0.00396976706576844\\
142	0.00396999118548161\\
143	0.00397021978653505\\
144	0.00397045294866576\\
145	0.00397069075268268\\
146	0.00397093328046883\\
147	0.00397118061498339\\
148	0.00397143284026303\\
149	0.00397169004142311\\
150	0.00397195230465819\\
151	0.00397221971724211\\
152	0.00397249236752781\\
153	0.00397277034494633\\
154	0.00397305374000582\\
155	0.00397334264428955\\
156	0.00397363715045398\\
157	0.003973937352226\\
158	0.00397424334439992\\
159	0.00397455522283397\\
160	0.00397487308444631\\
161	0.00397519702721084\\
162	0.00397552715015242\\
163	0.00397586355334175\\
164	0.00397620633789004\\
165	0.00397655560594325\\
166	0.00397691146067616\\
167	0.00397727400628604\\
168	0.00397764334798637\\
169	0.00397801959200017\\
170	0.00397840284555344\\
171	0.00397879321686846\\
172	0.00397919081515721\\
173	0.00397959575061479\\
174	0.00398000813441317\\
175	0.00398042807869503\\
176	0.00398085569656832\\
177	0.00398129110210079\\
178	0.00398173441031555\\
179	0.00398218573718711\\
180	0.00398264519963834\\
181	0.00398311291553826\\
182	0.00398358900370114\\
183	0.00398407358388667\\
184	0.00398456677680181\\
185	0.00398506870410398\\
186	0.0039855794884062\\
187	0.00398609925328423\\
188	0.00398662812328593\\
189	0.0039871662239429\\
190	0.00398771368178504\\
191	0.00398827062435772\\
192	0.00398883718024241\\
193	0.00398941347908048\\
194	0.00398999965160087\\
195	0.00399059582965163\\
196	0.00399120214623592\\
197	0.00399181873555242\\
198	0.00399244573304076\\
199	0.0039930832754322\\
200	0.00399373150080589\\
201	0.00399439054865097\\
202	0.00399506055993499\\
203	0.00399574167717908\\
204	0.00399643404453986\\
205	0.00399713780789878\\
206	0.00399785311495922\\
207	0.0039985801153516\\
208	0.00399931896074677\\
209	0.00400006980497827\\
210	0.00400083280417361\\
211	0.00400160811689487\\
212	0.00400239590428916\\
213	0.0040031963302489\\
214	0.00400400956158237\\
215	0.00400483576819493\\
216	0.00400567512328072\\
217	0.00400652780352546\\
218	0.00400739398932024\\
219	0.00400827386498645\\
220	0.00400916761901202\\
221	0.00401007544429893\\
222	0.00401099753842209\\
223	0.00401193410389937\\
224	0.00401288534847289\\
225	0.00401385148540112\\
226	0.00401483273376201\\
227	0.00401582931876653\\
228	0.00401684147208253\\
229	0.00401786943216855\\
230	0.00401891344461715\\
231	0.00401997376250759\\
232	0.00402105064676699\\
233	0.00402214436654\\
234	0.00402325519956623\\
235	0.00402438343256498\\
236	0.00402552936162713\\
237	0.00402669329261319\\
238	0.00402787554155787\\
239	0.00402907643508009\\
240	0.00403029631079869\\
241	0.00403153551775326\\
242	0.00403279441683012\\
243	0.0040340733811933\\
244	0.00403537279672018\\
245	0.00403669306244212\\
246	0.00403803459098952\\
247	0.0040393978090413\\
248	0.00404078315777801\\
249	0.004042191093338\\
250	0.00404362208727489\\
251	0.00404507662701405\\
252	0.00404655521630505\\
253	0.00404805837566494\\
254	0.00404958664280591\\
255	0.00405114057303865\\
256	0.0040527207396397\\
257	0.00405432773416886\\
258	0.00405596216671877\\
259	0.00405762466607665\\
260	0.00405931587977623\\
261	0.00406103647402001\\
262	0.00406278713346629\\
263	0.0040645685609234\\
264	0.00406638147710659\\
265	0.00406822662055874\\
266	0.00407010474613871\\
267	0.00407201662523554\\
268	0.00407396304598531\\
269	0.00407594481349066\\
270	0.00407796275004411\\
271	0.00408001769535605\\
272	0.00408211050678803\\
273	0.00408424205959241\\
274	0.0040864132471601\\
275	0.00408862498127698\\
276	0.00409087819239134\\
277	0.00409317382989399\\
278	0.00409551286241395\\
279	0.00409789627813235\\
280	0.00410032508511883\\
281	0.00410280031169531\\
282	0.00410532300683364\\
283	0.00410789424059481\\
284	0.00411051510462028\\
285	0.00411318671268872\\
286	0.00411591020135412\\
287	0.00411868673068691\\
288	0.00412151748514391\\
289	0.00412440367460032\\
290	0.00412734653558421\\
291	0.00413034733276421\\
292	0.00413340736075139\\
293	0.00413652794628979\\
294	0.00413971045092405\\
295	0.00414295627424901\\
296	0.00414626685786274\\
297	0.00414964369016378\\
298	0.00415308831215686\\
299	0.0041566023244764\\
300	0.00416018739593252\\
301	0.00416384527394439\\
302	0.00416757779579823\\
303	0.0041713869004178\\
304	0.00417527464252895\\
305	0.00417924320775439\\
306	0.00418329492776416\\
307	0.0041874322940093\\
308	0.00419165796832062\\
309	0.00419597479204264\\
310	0.00420038576183341\\
311	0.00420489398253474\\
312	0.00420950266927981\\
313	0.00421421514900673\\
314	0.00421903486116254\\
315	0.00422396535825689\\
316	0.00422901030784202\\
317	0.00423417349193308\\
318	0.00423945880541624\\
319	0.00424487025324035\\
320	0.00425041194616065\\
321	0.00425608809477158\\
322	0.00426190300153614\\
323	0.00426786105048812\\
324	0.00427396669425726\\
325	0.00428022443804854\\
326	0.00428663882023624\\
327	0.00429321438951463\\
328	0.00429995568010478\\
329	0.00430686719176091\\
330	0.00431395334176155\\
331	0.00432121842565339\\
332	0.00432866657480124\\
333	0.00433630171164908\\
334	0.00434412750283828\\
335	0.00435214730700318\\
336	0.00436036414236211\\
337	0.00436878069589284\\
338	0.00437739934436666\\
339	0.0043862222235715\\
340	0.00439525137003741\\
341	0.00440448896883744\\
342	0.00441393775264034\\
343	0.00442360160977393\\
344	0.0044334879031296\\
345	0.00444360919024763\\
346	0.00445397903540017\\
347	0.00446461208546004\\
348	0.00447552414697224\\
349	0.0044867322621642\\
350	0.00449825478041501\\
351	0.004510111427792\\
352	0.00452232351700233\\
353	0.00453491390347039\\
354	0.0045479069566887\\
355	0.00456132853268967\\
356	0.00457520586966399\\
357	0.00458956718628757\\
358	0.00460444140180532\\
359	0.00461985793956428\\
360	0.00463584612954487\\
361	0.00465243435886602\\
362	0.00466964888774392\\
363	0.0046875122230492\\
364	0.00470604090849864\\
365	0.0047252425474306\\
366	0.00474511181771898\\
367	0.00476562516388812\\
368	0.00478673374934612\\
369	0.00480835408631128\\
370	0.00483005801812664\\
371	0.00485172198632208\\
372	0.00487331742568788\\
373	0.0048948135360401\\
374	0.00491617687233518\\
375	0.00493737172833845\\
376	0.00495835663584292\\
377	0.00497908363036898\\
378	0.00499950049407906\\
379	0.00501955077361669\\
380	0.00503917396544525\\
381	0.00505830593989146\\
382	0.00507687969917407\\
383	0.00509482660604844\\
384	0.00511207823753275\\
385	0.0051285690954679\\
386	0.0051442404718571\\
387	0.00515904586356221\\
388	0.00517295842467018\\
389	0.00518598113225567\\
390	0.00519816064651042\\
391	0.00521007320582187\\
392	0.00522171956990806\\
393	0.00523308554388016\\
394	0.00524415826380914\\
395	0.00525492661785436\\
396	0.00526538172482074\\
397	0.00527551746887198\\
398	0.00528533108466823\\
399	0.00529482378052394\\
400	0.00530400137743908\\
401	0.00531287492783769\\
402	0.00532146125856364\\
403	0.00532978335668494\\
404	0.00533787047707073\\
405	0.00534575779758709\\
406	0.00535348537323425\\
407	0.00536109604768382\\
408	0.00536862988912369\\
409	0.00537610170564491\\
410	0.00538351681248603\\
411	0.00539088150686385\\
412	0.00539820306368182\\
413	0.0054054897055481\\
414	0.0054127506373611\\
415	0.00541999584889401\\
416	0.00542723601052474\\
417	0.00543448186342449\\
418	0.0054417442293014\\
419	0.00544903371352326\\
420	0.00545636033618458\\
421	0.00546373314084725\\
422	0.00547115982365393\\
423	0.00547864645414766\\
424	0.00548619757398796\\
425	0.00549381721138449\\
426	0.00550150942358831\\
427	0.00550927825615538\\
428	0.00551712769995871\\
429	0.00552506164751555\\
430	0.00553308385076977\\
431	0.00554119788289693\\
432	0.00554940710707964\\
433	0.00555771465570698\\
434	0.00556612342353803\\
435	0.00557463607806808\\
436	0.00558325508963432\\
437	0.00559198278159137\\
438	0.00560082139614385\\
439	0.005609773136161\\
440	0.00561884017156667\\
441	0.00562802463637245\\
442	0.00563732862723041\\
443	0.00564675420368448\\
444	0.00565630339023131\\
445	0.00566597818020127\\
446	0.00567578054131269\\
447	0.00568571242255137\\
448	0.00569577576180889\\
449	0.00570597249349151\\
450	0.00571630455513656\\
451	0.00572677389202858\\
452	0.00573738245935717\\
453	0.00574813222342954\\
454	0.00575902516311368\\
455	0.00577006327147047\\
456	0.0057812485575194\\
457	0.00579258304807187\\
458	0.00580406878955898\\
459	0.00581570784978016\\
460	0.00582750231950734\\
461	0.00583945431389915\\
462	0.00585156597371133\\
463	0.00586383946633356\\
464	0.00587627698673253\\
465	0.00588888075838796\\
466	0.00590165303425179\\
467	0.00591459609772419\\
468	0.00592771226364155\\
469	0.00594100387927433\\
470	0.00595447332533495\\
471	0.00596812301699974\\
472	0.00598195540495295\\
473	0.00599597297646308\\
474	0.00601017825650559\\
475	0.0060245738089458\\
476	0.00603916223779494\\
477	0.00605394618854894\\
478	0.00606892834961896\\
479	0.00608411145386366\\
480	0.00609949828023481\\
481	0.00611509165554907\\
482	0.00613089445640049\\
483	0.00614690961122951\\
484	0.00616314010256575\\
485	0.00617958896946317\\
486	0.00619625931014782\\
487	0.00621315428489966\\
488	0.00623027711919161\\
489	0.00624763110711127\\
490	0.0062652196150922\\
491	0.00628304608598458\\
492	0.00630111404349607\\
493	0.0063194270970369\\
494	0.0063379889470043\\
495	0.00635680339054377\\
496	0.00637587432782535\\
497	0.00639520576887495\\
498	0.00641480184100072\\
499	0.00643466679685369\\
500	0.00645480502316044\\
501	0.00647522105016233\\
502	0.00649591956178959\\
503	0.00651690540659079\\
504	0.00653818360942569\\
505	0.00655975938391251\\
506	0.00658163814559771\\
507	0.00660382552578562\\
508	0.00662632738592403\\
509	0.00664914983238842\\
510	0.00667229923143845\\
511	0.00669578222402991\\
512	0.00671960574005042\\
513	0.0067437770114002\\
514	0.00676830358315132\\
515	0.00679319332178079\\
516	0.00681845441917239\\
517	0.00684409539070084\\
518	0.00687012506523294\\
519	0.006896552564277\\
520	0.00692338726675418\\
521	0.00695063875491424\\
522	0.00697831673572706\\
523	0.00700643093058926\\
524	0.00703499092432012\\
525	0.00706400596220947\\
526	0.00709348467960366\\
527	0.00712343474003986\\
528	0.0071538623899978\\
529	0.00718477193429298\\
530	0.00721616568394696\\
531	0.00724804108720154\\
532	0.00728038731950989\\
533	0.00731312613157621\\
534	0.00734626297854407\\
535	0.00737986506719777\\
536	0.00741408122042639\\
537	0.00744902119843883\\
538	0.00748474690796401\\
539	0.00752132339765042\\
540	0.00755880916842859\\
541	0.0075972666895293\\
542	0.00763676226561044\\
543	0.00767735965754053\\
544	0.00771913114226334\\
545	0.0077621788010122\\
546	0.00780664666479424\\
547	0.00785119985749986\\
548	0.00789382387062527\\
549	0.00793483981570633\\
550	0.00797601076828931\\
551	0.00801762780379761\\
552	0.00805973722232243\\
553	0.00810231501629268\\
554	0.00814531085129842\\
555	0.00818865547804408\\
556	0.00823227074941743\\
557	0.00827530602472503\\
558	0.00831741047380532\\
559	0.00835967933656371\\
560	0.00840220305870095\\
561	0.00844497776753068\\
562	0.00848797514844882\\
563	0.00853116460971669\\
564	0.00857451574226307\\
565	0.00861800248723491\\
566	0.00866143203040683\\
567	0.00870455612303824\\
568	0.00874796048963319\\
569	0.00879165271721747\\
570	0.00883561306273407\\
571	0.00887981747531298\\
572	0.00892423953968708\\
573	0.00896885037174267\\
574	0.00901361849796072\\
575	0.00905850974477759\\
576	0.00910348714511327\\
577	0.0091485108699142\\
578	0.00919353819462128\\
579	0.0092385235131504\\
580	0.0092834184153157\\
581	0.00932817184778395\\
582	0.00937273038379318\\
583	0.00941703863317463\\
584	0.00946103983179202\\
585	0.00950467665821593\\
586	0.00954789233435466\\
587	0.00959063207273937\\
588	0.00963284492704105\\
589	0.0096744860587673\\
590	0.00971551928305088\\
591	0.00975591931512916\\
592	0.00979567191274874\\
593	0.00983476678232329\\
594	0.00987312984006299\\
595	0.00991038882774525\\
596	0.00994573390547615\\
597	0.0099771937715668\\
598	0.00999970795535495\\
599	0\\
600	0\\
};
\addplot [color=black!60!mycolor21,solid,forget plot]
  table[row sep=crcr]{%
1	0.00395973365081125\\
2	0.00395974572637439\\
3	0.00395975808349903\\
4	0.00395977072871211\\
5	0.00395978366869045\\
6	0.00395979691026415\\
7	0.00395981046042001\\
8	0.00395982432630529\\
9	0.00395983851523096\\
10	0.00395985303467575\\
11	0.00395986789228969\\
12	0.00395988309589811\\
13	0.00395989865350548\\
14	0.00395991457329953\\
15	0.00395993086365537\\
16	0.00395994753313958\\
17	0.00395996459051466\\
18	0.00395998204474336\\
19	0.00395999990499313\\
20	0.00396001818064074\\
21	0.00396003688127698\\
22	0.00396005601671148\\
23	0.00396007559697752\\
24	0.00396009563233706\\
25	0.00396011613328586\\
26	0.00396013711055862\\
27	0.00396015857513445\\
28	0.00396018053824212\\
29	0.00396020301136581\\
30	0.00396022600625061\\
31	0.00396024953490839\\
32	0.00396027360962373\\
33	0.00396029824295994\\
34	0.00396032344776523\\
35	0.00396034923717902\\
36	0.00396037562463832\\
37	0.00396040262388439\\
38	0.00396043024896929\\
39	0.00396045851426296\\
40	0.00396048743445999\\
41	0.00396051702458686\\
42	0.00396054730000915\\
43	0.00396057827643913\\
44	0.00396060996994314\\
45	0.00396064239694945\\
46	0.0039606755742562\\
47	0.00396070951903935\\
48	0.00396074424886096\\
49	0.00396077978167765\\
50	0.00396081613584909\\
51	0.00396085333014676\\
52	0.00396089138376294\\
53	0.00396093031631961\\
54	0.00396097014787804\\
55	0.00396101089894803\\
56	0.0039610525904976\\
57	0.0039610952439629\\
58	0.00396113888125826\\
59	0.00396118352478639\\
60	0.00396122919744883\\
61	0.00396127592265659\\
62	0.00396132372434114\\
63	0.00396137262696529\\
64	0.00396142265553464\\
65	0.00396147383560898\\
66	0.00396152619331406\\
67	0.0039615797553535\\
68	0.00396163454902111\\
69	0.00396169060221312\\
70	0.0039617479434409\\
71	0.00396180660184387\\
72	0.00396186660720257\\
73	0.00396192798995211\\
74	0.00396199078119566\\
75	0.00396205501271844\\
76	0.00396212071700173\\
77	0.00396218792723732\\
78	0.00396225667734212\\
79	0.00396232700197303\\
80	0.00396239893654209\\
81	0.00396247251723202\\
82	0.00396254778101174\\
83	0.00396262476565246\\
84	0.00396270350974392\\
85	0.0039627840527109\\
86	0.00396286643482994\\
87	0.00396295069724662\\
88	0.00396303688199272\\
89	0.00396312503200407\\
90	0.00396321519113836\\
91	0.00396330740419339\\
92	0.00396340171692572\\
93	0.00396349817606933\\
94	0.00396359682935493\\
95	0.00396369772552925\\
96	0.00396380091437469\\
97	0.00396390644672952\\
98	0.003964014374508\\
99	0.00396412475072116\\
100	0.00396423762949755\\
101	0.00396435306610452\\
102	0.00396447111696974\\
103	0.00396459183970292\\
104	0.00396471529311793\\
105	0.00396484153725517\\
106	0.0039649706334043\\
107	0.00396510264412705\\
108	0.00396523763328059\\
109	0.00396537566604099\\
110	0.00396551680892709\\
111	0.00396566112982445\\
112	0.00396580869800976\\
113	0.0039659595841755\\
114	0.00396611386045458\\
115	0.00396627160044565\\
116	0.00396643287923846\\
117	0.00396659777343926\\
118	0.00396676636119673\\
119	0.00396693872222802\\
120	0.00396711493784497\\
121	0.00396729509098044\\
122	0.00396747926621499\\
123	0.00396766754980375\\
124	0.00396786002970323\\
125	0.00396805679559842\\
126	0.0039682579389301\\
127	0.00396846355292208\\
128	0.00396867373260866\\
129	0.00396888857486212\\
130	0.0039691081784203\\
131	0.00396933264391418\\
132	0.00396956207389549\\
133	0.00396979657286437\\
134	0.00397003624729687\\
135	0.00397028120567255\\
136	0.00397053155850191\\
137	0.0039707874183537\\
138	0.0039710488998821\\
139	0.00397131611985376\\
140	0.00397158919717467\\
141	0.0039718682529167\\
142	0.00397215341034385\\
143	0.00397244479493844\\
144	0.0039727425344266\\
145	0.00397304675880361\\
146	0.00397335760035877\\
147	0.00397367519369977\\
148	0.00397399967577657\\
149	0.00397433118590468\\
150	0.00397466986578781\\
151	0.00397501585954004\\
152	0.00397536931370701\\
153	0.00397573037728663\\
154	0.00397609920174876\\
155	0.00397647594105407\\
156	0.00397686075167213\\
157	0.00397725379259818\\
158	0.00397765522536923\\
159	0.00397806521407888\\
160	0.00397848392539086\\
161	0.00397891152855152\\
162	0.00397934819540101\\
163	0.00397979410038307\\
164	0.00398024942055329\\
165	0.00398071433558605\\
166	0.00398118902777987\\
167	0.00398167368206101\\
168	0.0039821684859856\\
169	0.00398267362973997\\
170	0.00398318930613895\\
171	0.00398371571062265\\
172	0.00398425304125103\\
173	0.00398480149869675\\
174	0.00398536128623569\\
175	0.00398593260973566\\
176	0.00398651567764272\\
177	0.00398711070096562\\
178	0.00398771789325776\\
179	0.003988337470597\\
180	0.00398896965156327\\
181	0.00398961465721376\\
182	0.00399027271105594\\
183	0.00399094403901817\\
184	0.00399162886941805\\
185	0.00399232743292847\\
186	0.00399303996254159\\
187	0.00399376669353023\\
188	0.00399450786340751\\
189	0.00399526371188408\\
190	0.00399603448082356\\
191	0.00399682041419608\\
192	0.00399762175802983\\
193	0.0039984387603612\\
194	0.00399927167118327\\
195	0.00400012074239307\\
196	0.00400098622773767\\
197	0.00400186838275947\\
198	0.00400276746474071\\
199	0.00400368373264778\\
200	0.00400461744707524\\
201	0.00400556887019027\\
202	0.00400653826567762\\
203	0.00400752589868547\\
204	0.00400853203577279\\
205	0.00400955694485846\\
206	0.00401060089517248\\
207	0.00401166415721008\\
208	0.00401274700268913\\
209	0.00401384970451091\\
210	0.0040149725367256\\
211	0.00401611577450251\\
212	0.00401727969410559\\
213	0.00401846457287518\\
214	0.00401967068921618\\
215	0.00402089832259371\\
216	0.00402214775353635\\
217	0.0040234192636479\\
218	0.00402471313562826\\
219	0.00402602965330369\\
220	0.00402736910166737\\
221	0.00402873176693055\\
222	0.00403011793658471\\
223	0.00403152789947538\\
224	0.00403296194588792\\
225	0.00403442036764567\\
226	0.00403590345822042\\
227	0.00403741151285638\\
228	0.0040389448287067\\
229	0.00404050370498376\\
230	0.00404208844312312\\
231	0.00404369934696128\\
232	0.00404533672292807\\
233	0.00404700088025384\\
234	0.00404869213119245\\
235	0.00405041079126148\\
236	0.00405215717950076\\
237	0.00405393161875217\\
238	0.0040557344359636\\
239	0.00405756596252151\\
240	0.00405942653461807\\
241	0.00406131649366064\\
242	0.00406323618673343\\
243	0.00406518596712463\\
244	0.00406716619493544\\
245	0.0040691772377917\\
246	0.00407121947168417\\
247	0.00407329328196941\\
248	0.00407539906457014\\
249	0.00407753722742205\\
250	0.00407970819222244\\
251	0.00408191239654666\\
252	0.00408415029640648\\
253	0.00408642236933493\\
254	0.00408872911808953\\
255	0.00409107107506929\\
256	0.00409344880753876\\
257	0.0040958629237395\\
258	0.00409831407993879\\
259	0.00410080298840983\\
260	0.00410333042624129\\
261	0.00410589724472036\\
262	0.00410850437879744\\
263	0.00411115285581545\\
264	0.0041138438024045\\
265	0.00411657844896079\\
266	0.00411935813113533\\
267	0.00412218423674385\\
268	0.00412505820739792\\
269	0.00412798154011819\\
270	0.00413095578891538\\
271	0.00413398256632245\\
272	0.00413706354485806\\
273	0.00414020045839793\\
274	0.00414339510342495\\
275	0.00414664934012569\\
276	0.00414996509329252\\
277	0.00415334435298507\\
278	0.00415678917489552\\
279	0.0041603016803534\\
280	0.00416388405589335\\
281	0.00416753855229801\\
282	0.00417126748301288\\
283	0.0041750732218148\\
284	0.00417895819959755\\
285	0.00418292490011852\\
286	0.00418697585452994\\
287	0.00419111363449593\\
288	0.00419534084367495\\
289	0.0041996601073263\\
290	0.00420407405978126\\
291	0.00420858532950783\\
292	0.00421319652149639\\
293	0.00421791019670725\\
294	0.00422272884835977\\
295	0.00422765487491709\\
296	0.00423269054974617\\
297	0.0042378379876333\\
298	0.00424309910864331\\
299	0.00424847560029406\\
300	0.00425396887987042\\
301	0.0042595800606655\\
302	0.00426530993112059\\
303	0.00427115894010199\\
304	0.00427712718852631\\
305	0.00428321446444065\\
306	0.00428942032659151\\
307	0.00429574425911853\\
308	0.00430218592757397\\
309	0.00430874557718916\\
310	0.00431542576113523\\
311	0.00432223194998269\\
312	0.0043291700187598\\
313	0.0043362462806597\\
314	0.00434346752159857\\
315	0.00435084103257978\\
316	0.00435837464522503\\
317	0.00436607679851565\\
318	0.00437395658936828\\
319	0.00438202382640046\\
320	0.00439028908668731\\
321	0.00439876377509387\\
322	0.00440746018550648\\
323	0.0044163915629326\\
324	0.00442557216495628\\
325	0.00443501732036787\\
326	0.00444474348182515\\
327	0.00445476826794441\\
328	0.00446511048838201\\
329	0.00447579015169591\\
330	0.00448682859803789\\
331	0.0044982482850376\\
332	0.00451007264947188\\
333	0.00452232588076041\\
334	0.00453503257412249\\
335	0.00454821720993343\\
336	0.00456190327868517\\
337	0.00457611217701565\\
338	0.00459086203911056\\
339	0.00460616586057752\\
340	0.00462202897244422\\
341	0.00463844566146682\\
342	0.00465539466877314\\
343	0.00467283320530182\\
344	0.00469060059314415\\
345	0.00470844266884853\\
346	0.0047263464874706\\
347	0.00474429746913023\\
348	0.00476227922607206\\
349	0.00478027337450774\\
350	0.0047982593258573\\
351	0.00481621402500397\\
352	0.00483411139612893\\
353	0.00485192400943727\\
354	0.00486962229425243\\
355	0.00488717408487719\\
356	0.00490454469905291\\
357	0.00492169736146272\\
358	0.00493859069784189\\
359	0.00495517802660223\\
360	0.00497140977455703\\
361	0.00498723402977358\\
362	0.00500259743945583\\
363	0.00501744657798227\\
364	0.005031729956633\\
365	0.005045400896116\\
366	0.00505842155578471\\
367	0.00507076850744154\\
368	0.00508244036750977\\
369	0.00509346818626211\\
370	0.00510423335285532\\
371	0.00511481721267507\\
372	0.00512520656178234\\
373	0.00513538847250518\\
374	0.00514535047457036\\
375	0.00515508077750276\\
376	0.00516456855361846\\
377	0.00517380431640475\\
378	0.0051827803356665\\
379	0.00519149106327959\\
380	0.00519993359963097\\
381	0.0052081081912844\\
382	0.0052160187428736\\
383	0.00522367331509532\\
384	0.00523108456575669\\
385	0.00523827006945351\\
386	0.00524525242173173\\
387	0.00525205899284521\\
388	0.00525872114211429\\
389	0.00526527263128898\\
390	0.00527174687682881\\
391	0.00527815583816508\\
392	0.00528450241978055\\
393	0.00529079005792261\\
394	0.0052970229970351\\
395	0.00530320630372358\\
396	0.00530934586209489\\
397	0.00531544834562946\\
398	0.00532152116033423\\
399	0.0053275723537607\\
400	0.00533361048473719\\
401	0.00533964444962932\\
402	0.00534568326299393\\
403	0.00535173579411485\\
404	0.00535781046704782\\
405	0.0053639149414637\\
406	0.00537005580621256\\
407	0.00537623833788493\\
408	0.00538246645925789\\
409	0.00538874363358309\\
410	0.00539507350836763\\
411	0.00540145973500464\\
412	0.00540790581700941\\
413	0.00541441522440572\\
414	0.00542099135453166\\
415	0.00542763749337966\\
416	0.005434356781826\\
417	0.00544115218914853\\
418	0.00544802649591731\\
419	0.00545498228253614\\
420	0.00546202193096005\\
421	0.0054691476426214\\
422	0.00547636147363726\\
423	0.00548366538546617\\
424	0.00549106129752603\\
425	0.00549855110550918\\
426	0.00550613667730672\\
427	0.00551381984997456\\
428	0.00552160242794284\\
429	0.0055294861826372\\
430	0.00553747285361932\\
431	0.00554556415127188\\
432	0.00555376176094407\\
433	0.00556206734832942\\
434	0.00557048256568327\\
435	0.00557900905830964\\
436	0.00558764847057428\\
437	0.00559640245059426\\
438	0.00560527265282975\\
439	0.005614260739245\\
440	0.00562336838029298\\
441	0.00563259725609319\\
442	0.00564194905776808\\
443	0.00565142548889317\\
444	0.00566102826700626\\
445	0.00567075912511421\\
446	0.0056806198131339\\
447	0.00569061209920956\\
448	0.00570073777086308\\
449	0.00571099863595957\\
450	0.00572139652350465\\
451	0.00573193328433221\\
452	0.00574261079176623\\
453	0.00575343094229183\\
454	0.00576439565622845\\
455	0.00577550687839866\\
456	0.00578676657878726\\
457	0.00579817675318735\\
458	0.00580973942383269\\
459	0.00582145664001769\\
460	0.00583333047871008\\
461	0.00584536304516276\\
462	0.00585755647353328\\
463	0.00586991292751904\\
464	0.00588243460101435\\
465	0.00589512371879264\\
466	0.00590798253721685\\
467	0.00592101334498117\\
468	0.00593421846388834\\
469	0.00594760024966681\\
470	0.00596116109283389\\
471	0.00597490341961021\\
472	0.00598882969289202\\
473	0.00600294241328854\\
474	0.00601724412023184\\
475	0.00603173739316653\\
476	0.00604642485282856\\
477	0.00606130916262169\\
478	0.00607639303010246\\
479	0.00609167920858412\\
480	0.00610717049887266\\
481	0.0061228697511475\\
482	0.00613877986700211\\
483	0.00615490380165982\\
484	0.00617124456638256\\
485	0.00618780523109075\\
486	0.00620458892721452\\
487	0.0062215988507978\\
488	0.00623883826587823\\
489	0.00625631050816762\\
490	0.00627401898905844\\
491	0.00629196719998375\\
492	0.00631015871715825\\
493	0.00632859720672935\\
494	0.00634728643036652\\
495	0.00636623025131708\\
496	0.00638543264095454\\
497	0.0064048976858429\\
498	0.00642462959533532\\
499	0.00644463270971899\\
500	0.00646491150890807\\
501	0.00648547062167295\\
502	0.00650631483537603\\
503	0.00652744910615984\\
504	0.0065488785695011\\
505	0.00657060855100292\\
506	0.0065926445772432\\
507	0.00661499238642729\\
508	0.0066376579385048\\
509	0.00666064742429596\\
510	0.00668396727302896\\
511	0.00670762415750627\\
512	0.006731624995887\\
513	0.00675597694877953\\
514	0.00678068740997081\\
515	0.0068057639886559\\
516	0.00683121448044888\\
517	0.00685704682372615\\
518	0.0068832690369369\\
519	0.00690988913136679\\
520	0.00693691499240218\\
521	0.00696435422053964\\
522	0.00699221392113048\\
523	0.00702050042902386\\
524	0.00704921895073827\\
525	0.00707837309196509\\
526	0.00710796432099803\\
527	0.00713799174457201\\
528	0.00716845010365891\\
529	0.00719932589383715\\
530	0.00723054248467109\\
531	0.00726209276444554\\
532	0.00729404052716598\\
533	0.00732653921734199\\
534	0.0073596836632091\\
535	0.00739353211339375\\
536	0.0074281402754436\\
537	0.00746356041024936\\
538	0.0074998485267378\\
539	0.00753706529569083\\
540	0.00757527798571959\\
541	0.00761456132309162\\
542	0.00765499591701934\\
543	0.00769668136422686\\
544	0.00773976131539374\\
545	0.00778279111614491\\
546	0.00782374860307764\\
547	0.00786341518638422\\
548	0.0079033333892121\\
549	0.00794373252387418\\
550	0.00798465944434795\\
551	0.00802608825170205\\
552	0.00806797470045948\\
553	0.00811025901270837\\
554	0.00815287259048969\\
555	0.00819554138159801\\
556	0.00823719242952156\\
557	0.00827861275638618\\
558	0.00832031456782766\\
559	0.00836230830113041\\
560	0.00840456921017423\\
561	0.00844706889092748\\
562	0.00848977886278278\\
563	0.00853267111690916\\
564	0.0085757222039079\\
565	0.00861868515414452\\
566	0.00866146567320772\\
567	0.00870455834676335\\
568	0.0087479605259397\\
569	0.00879165272093579\\
570	0.00883561306364211\\
571	0.00887981747566901\\
572	0.00892423953985309\\
573	0.00896885037182793\\
574	0.00901361849800735\\
575	0.00905850974480371\\
576	0.00910348714512788\\
577	0.0091485108699222\\
578	0.00919353819462542\\
579	0.00923852351315233\\
580	0.00928341841531647\\
581	0.00932817184778417\\
582	0.00937273038379323\\
583	0.00941703863317462\\
584	0.00946103983179201\\
585	0.00950467665821592\\
586	0.00954789233435465\\
587	0.00959063207273936\\
588	0.00963284492704104\\
589	0.0096744860587673\\
590	0.00971551928305087\\
591	0.00975591931512916\\
592	0.00979567191274874\\
593	0.00983476678232329\\
594	0.00987312984006299\\
595	0.00991038882774525\\
596	0.00994573390547615\\
597	0.0099771937715668\\
598	0.00999970795535495\\
599	0\\
600	0\\
};
\addplot [color=black!80!mycolor21,solid,forget plot]
  table[row sep=crcr]{%
1	0.00395975731968208\\
2	0.00395977009500059\\
3	0.00395978317291841\\
4	0.00395979656060401\\
5	0.00395981026539567\\
6	0.00395982429480537\\
7	0.00395983865652317\\
8	0.00395985335842107\\
9	0.00395986840855766\\
10	0.00395988381518224\\
11	0.00395989958673951\\
12	0.00395991573187418\\
13	0.0039599322594356\\
14	0.0039599491784827\\
15	0.00395996649828889\\
16	0.00395998422834712\\
17	0.00396000237837513\\
18	0.00396002095832076\\
19	0.00396003997836725\\
20	0.00396005944893901\\
21	0.00396007938070724\\
22	0.00396009978459562\\
23	0.00396012067178656\\
24	0.003960142053727\\
25	0.00396016394213487\\
26	0.00396018634900549\\
27	0.003960209286618\\
28	0.00396023276754217\\
29	0.00396025680464509\\
30	0.00396028141109849\\
31	0.00396030660038564\\
32	0.00396033238630884\\
33	0.00396035878299698\\
34	0.00396038580491315\\
35	0.00396041346686255\\
36	0.00396044178400063\\
37	0.00396047077184126\\
38	0.00396050044626527\\
39	0.00396053082352899\\
40	0.00396056192027318\\
41	0.00396059375353214\\
42	0.00396062634074285\\
43	0.00396065969975455\\
44	0.00396069384883843\\
45	0.00396072880669763\\
46	0.00396076459247724\\
47	0.0039608012257749\\
48	0.00396083872665135\\
49	0.00396087711564128\\
50	0.00396091641376467\\
51	0.00396095664253795\\
52	0.00396099782398588\\
53	0.00396103998065347\\
54	0.0039610831356181\\
55	0.0039611273125021\\
56	0.00396117253548557\\
57	0.00396121882931947\\
58	0.00396126621933903\\
59	0.00396131473147746\\
60	0.00396136439227993\\
61	0.00396141522891801\\
62	0.00396146726920432\\
63	0.00396152054160757\\
64	0.00396157507526792\\
65	0.00396163090001279\\
66	0.00396168804637274\\
67	0.00396174654559832\\
68	0.0039618064296765\\
69	0.00396186773134815\\
70	0.00396193048412568\\
71	0.00396199472231097\\
72	0.00396206048101383\\
73	0.00396212779617093\\
74	0.00396219670456516\\
75	0.00396226724384521\\
76	0.00396233945254585\\
77	0.00396241337010868\\
78	0.00396248903690322\\
79	0.00396256649424842\\
80	0.00396264578443488\\
81	0.00396272695074746\\
82	0.00396281003748834\\
83	0.00396289509000075\\
84	0.00396298215469307\\
85	0.00396307127906364\\
86	0.00396316251172606\\
87	0.00396325590243491\\
88	0.00396335150211243\\
89	0.00396344936287535\\
90	0.00396354953806262\\
91	0.00396365208226378\\
92	0.00396375705134773\\
93	0.00396386450249235\\
94	0.00396397449421459\\
95	0.00396408708640154\\
96	0.00396420234034195\\
97	0.00396432031875841\\
98	0.00396444108584042\\
99	0.00396456470727807\\
100	0.00396469125029649\\
101	0.00396482078369114\\
102	0.00396495337786379\\
103	0.00396508910485923\\
104	0.0039652280384029\\
105	0.00396537025393936\\
106	0.00396551582867142\\
107	0.00396566484160037\\
108	0.0039658173735669\\
109	0.00396597350729287\\
110	0.00396613332742416\\
111	0.00396629692057432\\
112	0.00396646437536921\\
113	0.00396663578249238\\
114	0.00396681123473194\\
115	0.00396699082702781\\
116	0.00396717465652026\\
117	0.00396736282259961\\
118	0.00396755542695665\\
119	0.00396775257363438\\
120	0.0039679543690807\\
121	0.00396816092220223\\
122	0.00396837234441927\\
123	0.00396858874972176\\
124	0.00396881025472663\\
125	0.00396903697873607\\
126	0.00396926904379712\\
127	0.00396950657476246\\
128	0.00396974969935238\\
129	0.00396999854821802\\
130	0.00397025325500585\\
131	0.00397051395642351\\
132	0.00397078079230673\\
133	0.00397105390568787\\
134	0.00397133344286555\\
135	0.00397161955347573\\
136	0.00397191239056412\\
137	0.00397221211066002\\
138	0.0039725188738516\\
139	0.00397283284386243\\
140	0.00397315418812967\\
141	0.00397348307788353\\
142	0.00397381968822831\\
143	0.00397416419822478\\
144	0.0039745167909744\\
145	0.00397487765370463\\
146	0.00397524697785602\\
147	0.00397562495917076\\
148	0.00397601179778292\\
149	0.0039764076983101\\
150	0.00397681286994673\\
151	0.00397722752655897\\
152	0.00397765188678133\\
153	0.00397808617411459\\
154	0.00397853061702575\\
155	0.00397898544904938\\
156	0.00397945090889058\\
157	0.00397992724053004\\
158	0.00398041469332998\\
159	0.00398091352214262\\
160	0.00398142398741975\\
161	0.00398194635532434\\
162	0.00398248089784353\\
163	0.00398302789290365\\
164	0.00398358762448686\\
165	0.00398416038274931\\
166	0.00398474646414138\\
167	0.00398534617152946\\
168	0.00398595981431937\\
169	0.00398658770858174\\
170	0.00398723017717919\\
171	0.00398788754989488\\
172	0.00398856016356348\\
173	0.00398924836220324\\
174	0.0039899524971503\\
175	0.0039906729271948\\
176	0.00399141001871862\\
177	0.00399216414583495\\
178	0.00399293569053003\\
179	0.00399372504280643\\
180	0.00399453260082846\\
181	0.00399535877106935\\
182	0.00399620396846057\\
183	0.00399706861654289\\
184	0.00399795314761986\\
185	0.00399885800291297\\
186	0.00399978363271927\\
187	0.00400073049657103\\
188	0.00400169906339758\\
189	0.00400268981168967\\
190	0.00400370322966614\\
191	0.00400473981544289\\
192	0.00400580007720474\\
193	0.00400688453337968\\
194	0.00400799371281597\\
195	0.00400912815496207\\
196	0.00401028841004964\\
197	0.00401147503927936\\
198	0.00401268861501043\\
199	0.00401392972095257\\
200	0.00401519895236227\\
201	0.00401649691624196\\
202	0.00401782423154308\\
203	0.00401918152937262\\
204	0.00402056945320336\\
205	0.00402198865908801\\
206	0.00402343981587675\\
207	0.00402492360543842\\
208	0.00402644072288514\\
209	0.00402799187680004\\
210	0.0040295777894678\\
211	0.00403119919710726\\
212	0.00403285685010575\\
213	0.00403455151325396\\
214	0.0040362839659807\\
215	0.00403805500258511\\
216	0.00403986543246583\\
217	0.00404171608034403\\
218	0.00404360778647767\\
219	0.00404554140686416\\
220	0.00404751781342695\\
221	0.00404953789418143\\
222	0.00405160255337412\\
223	0.00405371271158865\\
224	0.00405586930580977\\
225	0.00405807328943585\\
226	0.00406032563222836\\
227	0.00406262732018404\\
228	0.00406497935531465\\
229	0.00406738275531439\\
230	0.00406983855309343\\
231	0.00407234779615168\\
232	0.00407491154576282\\
233	0.00407753087593371\\
234	0.00408020687209914\\
235	0.00408294062950564\\
236	0.00408573325123097\\
237	0.00408858584577917\\
238	0.00409149952418159\\
239	0.00409447539652615\\
240	0.00409751456782683\\
241	0.0041006181331355\\
242	0.004103787171788\\
243	0.00410702274066561\\
244	0.00411032586634469\\
245	0.00411369753599901\\
246	0.00411713868691669\\
247	0.00412065019449218\\
248	0.00412423285856576\\
249	0.00412788738800069\\
250	0.00413161438342674\\
251	0.00413541431813813\\
252	0.00413928751722765\\
253	0.00414323413517606\\
254	0.00414725413231579\\
255	0.00415134725087171\\
256	0.00415551299167899\\
257	0.00415975059322887\\
258	0.00416405901544871\\
259	0.00416843693165244\\
260	0.00417288273349081\\
261	0.00417739455560111\\
262	0.00418197032914141\\
263	0.00418660787665542\\
264	0.00419130506492994\\
265	0.00419606003845378\\
266	0.00420087166060245\\
267	0.00420574168753451\\
268	0.00421067200318543\\
269	0.0042156646302065\\
270	0.0042207217418873\\
271	0.00422584567515237\\
272	0.00423103894472913\\
273	0.00423630425859268\\
274	0.00424164453479921\\
275	0.00424706291982672\\
276	0.00425256280854856\\
277	0.00425814786596869\\
278	0.00426382205085089\\
279	0.00426958964137228\\
280	0.00427545526293053\\
281	0.00428142391822107\\
282	0.00428750101967601\\
283	0.00429369242432835\\
284	0.00430000447111757\\
285	0.00430644402058436\\
286	0.00431301849680697\\
287	0.00431973593129833\\
288	0.00432660500840334\\
289	0.0043336351114907\\
290	0.00434083636890846\\
291	0.00434821969823636\\
292	0.00435579684679416\\
293	0.00436358042560654\\
294	0.00437158393302906\\
295	0.00437982176293492\\
296	0.00438830919065439\\
297	0.00439706232762599\\
298	0.0044060980327956\\
299	0.00441543376495375\\
300	0.00442508735511623\\
301	0.0044350766715519\\
302	0.00444541914624127\\
303	0.00445613120437171\\
304	0.00446722741201072\\
305	0.0044787191116309\\
306	0.00449061275563124\\
307	0.00450290767796568\\
308	0.00451559312880382\\
309	0.00452864432839761\\
310	0.004541947907067\\
311	0.0045553588873206\\
312	0.0045688748456693\\
313	0.00458249292291672\\
314	0.00459620976519901\\
315	0.00461002144510762\\
316	0.00462392331109815\\
317	0.00463790976470962\\
318	0.00465197439747579\\
319	0.00466610990037951\\
320	0.00468030796267168\\
321	0.00469455916112629\\
322	0.00470885283943766\\
323	0.00472317697770581\\
324	0.00473751805229825\\
325	0.00475186088683822\\
326	0.00476618849557077\\
327	0.00478048192037877\\
328	0.00479472005920495\\
329	0.00480887945664696\\
330	0.00482293383031546\\
331	0.00483685605757499\\
332	0.0048506168205456\\
333	0.0048641847244187\\
334	0.0048775265556112\\
335	0.00489060775715463\\
336	0.00490339331893577\\
337	0.00491584724707331\\
338	0.00492793212758714\\
339	0.00493961304147446\\
340	0.00495086008862659\\
341	0.00496165189698724\\
342	0.00497198044421259\\
343	0.00498185762882403\\
344	0.00499141439933949\\
345	0.00500088028159732\\
346	0.00501024535832554\\
347	0.00501949946955782\\
348	0.00502863227167289\\
349	0.00503763331231683\\
350	0.00504649212381208\\
351	0.00505519833767343\\
352	0.00506374182204377\\
353	0.00507211283478288\\
354	0.00508030218103914\\
355	0.0050883014106292\\
356	0.00509610304984313\\
357	0.00510370086514399\\
358	0.00511109017558575\\
359	0.00511826823791135\\
360	0.00512523465220046\\
361	0.00513199174797994\\
362	0.00513854496156206\\
363	0.00514490317304922\\
364	0.00515107895563593\\
365	0.00515708866805653\\
366	0.00516295229114501\\
367	0.00516869286883802\\
368	0.00517433535983105\\
369	0.0051799046354389\\
370	0.00518541152707971\\
371	0.00519085787461229\\
372	0.00519624456272691\\
373	0.00520157302184255\\
374	0.00520684526534517\\
375	0.00521206392179455\\
376	0.00521723225921125\\
377	0.00522235419718806\\
378	0.005227434303618\\
379	0.00523247777304402\\
380	0.00523749038250713\\
381	0.00524247842062639\\
382	0.00524744858582909\\
383	0.00525240785037693\\
384	0.00525736328838141\\
385	0.00526232186879773\\
386	0.00526729021905066\\
387	0.00527227437234063\\
388	0.00527727952288001\\
389	0.00528230982886046\\
390	0.00528736831585664\\
391	0.0052924576227457\\
392	0.00529758048059644\\
393	0.00530273970726266\\
394	0.0053079381861204\\
395	0.00531317884193692\\
396	0.00531846461422133\\
397	0.00532379842863412\\
398	0.00532918316730713\\
399	0.00533462163924448\\
400	0.0053401165523255\\
401	0.00534567048879429\\
402	0.00535128588645114\\
403	0.00535696502800076\\
404	0.00536271004104953\\
405	0.00536852291091192\\
406	0.00537440550722331\\
407	0.00538035962001042\\
408	0.00538638704634926\\
409	0.00539248949611286\\
410	0.0053986685962717\\
411	0.00540492594510756\\
412	0.00541126311200561\\
413	0.00541768163795106\\
414	0.00542418303487263\\
415	0.00543076878631851\\
416	0.0054374403494222\\
417	0.00544419915799453\\
418	0.00545104662659633\\
419	0.00545798415547701\\
420	0.00546501313597188\\
421	0.00547213495578448\\
422	0.00547935100346694\\
423	0.00548666267140053\\
424	0.00549407135701656\\
425	0.00550157846336842\\
426	0.00550918539989171\\
427	0.00551689358333569\\
428	0.0055247044388393\\
429	0.00553261940111636\\
430	0.00554063991570623\\
431	0.00554876744024031\\
432	0.00555700344567168\\
433	0.00556534941741769\\
434	0.00557380685637413\\
435	0.0055823772797757\\
436	0.00559106222190345\\
437	0.00559986323467231\\
438	0.00560878188816548\\
439	0.00561781977116567\\
440	0.00562697849169054\\
441	0.00563625967752554\\
442	0.00564566497674826\\
443	0.00565519605823902\\
444	0.00566485461217416\\
445	0.00567464235050029\\
446	0.00568456100738965\\
447	0.00569461233967939\\
448	0.00570479812729925\\
449	0.00571512017369344\\
450	0.0057255803062425\\
451	0.00573618037668963\\
452	0.00574692226157277\\
453	0.00575780786266332\\
454	0.0057688391074119\\
455	0.00578001794940259\\
456	0.00579134636881688\\
457	0.00580282637290954\\
458	0.00581445999649804\\
459	0.0058262493024685\\
460	0.00583819638230009\\
461	0.00585030335661083\\
462	0.00586257237572729\\
463	0.00587500562028092\\
464	0.00588760530183387\\
465	0.00590037366353742\\
466	0.00591331298082653\\
467	0.00592642556215416\\
468	0.00593971374977001\\
469	0.00595317992054803\\
470	0.00596682648686794\\
471	0.00598065589755636\\
472	0.00599467063889399\\
473	0.0060088732356952\\
474	0.00602326625246796\\
475	0.00603785229466203\\
476	0.00605263401001434\\
477	0.00606761409000184\\
478	0.0060827952714119\\
479	0.00609818033804263\\
480	0.00611377212254541\\
481	0.0061295735084239\\
482	0.00614558743220402\\
483	0.00616181688579138\\
484	0.00617826491903308\\
485	0.00619493464250248\\
486	0.00621182923052606\\
487	0.00622895192447276\\
488	0.00624630603632682\\
489	0.0062638949525654\\
490	0.00628172213836242\\
491	0.00629979114213942\\
492	0.00631810560048313\\
493	0.00633666924344644\\
494	0.00635548590024656\\
495	0.00637455950536808\\
496	0.00639389410507143\\
497	0.00641349386429583\\
498	0.00643336307393241\\
499	0.00645350615842317\\
500	0.00647392768361698\\
501	0.00649463236478072\\
502	0.00651562507462131\\
503	0.0065369108511203\\
504	0.00655849490491298\\
505	0.00658038262585548\\
506	0.00660257958831128\\
507	0.00662509155454699\\
508	0.00664792447544836\\
509	0.00667108448754348\\
510	0.00669457790503862\\
511	0.0067184112052194\\
512	0.00674259100513055\\
513	0.00676712402689802\\
514	0.0067920170483731\\
515	0.00681727683492816\\
516	0.00684291004717574\\
517	0.0068689231180722\\
518	0.00689532209124132\\
519	0.00692211241034079\\
520	0.00694929864680671\\
521	0.00697688415023615\\
522	0.00700487060187458\\
523	0.00703325744699845\\
524	0.00706204117622466\\
525	0.0070912153098665\\
526	0.00712076579362154\\
527	0.00715063072912295\\
528	0.00718076948847036\\
529	0.00721124154541624\\
530	0.00724218981273631\\
531	0.00727371306630571\\
532	0.00730586429774683\\
533	0.00733869159646911\\
534	0.007372240221459\\
535	0.00740655837439732\\
536	0.00744169881164493\\
537	0.00747772044052915\\
538	0.00751468935496618\\
539	0.00755268009366983\\
540	0.00759177697910282\\
541	0.00763207998339044\\
542	0.00767373445957806\\
543	0.00771550707605202\\
544	0.00775514922292376\\
545	0.00779354939627914\\
546	0.00783225659391049\\
547	0.00787147428152859\\
548	0.00791124023747355\\
549	0.00795153400233813\\
550	0.00799231827609853\\
551	0.00803354342957163\\
552	0.00807514462435062\\
553	0.00811705216563942\\
554	0.00815850355717977\\
555	0.00819910905867986\\
556	0.00823996787570166\\
557	0.0082811455538887\\
558	0.00832262966246162\\
559	0.00836439422555056\\
560	0.00840641248997081\\
561	0.00844865781267785\\
562	0.00849110408425233\\
563	0.00853372943125429\\
564	0.00857626868344534\\
565	0.00861870126158591\\
566	0.00866146599263486\\
567	0.008704558351549\\
568	0.0087479605264543\\
569	0.00879165272106828\\
570	0.00883561306369562\\
571	0.00887981747569446\\
572	0.00892423953986631\\
573	0.00896885037183521\\
574	0.00901361849801143\\
575	0.00905850974480598\\
576	0.0091034871451291\\
577	0.00914851086992281\\
578	0.00919353819462571\\
579	0.00923852351315245\\
580	0.0092834184153165\\
581	0.00932817184778419\\
582	0.00937273038379324\\
583	0.00941703863317463\\
584	0.00946103983179202\\
585	0.00950467665821593\\
586	0.00954789233435466\\
587	0.00959063207273937\\
588	0.00963284492704104\\
589	0.0096744860587673\\
590	0.00971551928305088\\
591	0.00975591931512916\\
592	0.00979567191274874\\
593	0.00983476678232329\\
594	0.00987312984006299\\
595	0.00991038882774525\\
596	0.00994573390547615\\
597	0.0099771937715668\\
598	0.00999970795535495\\
599	0\\
600	0\\
};
\addplot [color=black,solid,forget plot]
  table[row sep=crcr]{%
1	0.0039597677845085\\
2	0.00395978087530682\\
3	0.00395979427839161\\
4	0.00395980800123459\\
5	0.00395982205148686\\
6	0.00395983643698346\\
7	0.00395985116574756\\
8	0.00395986624599527\\
9	0.00395988168614005\\
10	0.0039598974947977\\
11	0.00395991368079114\\
12	0.00395993025315539\\
13	0.00395994722114278\\
14	0.00395996459422822\\
15	0.00395998238211445\\
16	0.00396000059473777\\
17	0.00396001924227345\\
18	0.00396003833514168\\
19	0.00396005788401349\\
20	0.00396007789981684\\
21	0.00396009839374268\\
22	0.00396011937725156\\
23	0.00396014086207997\\
24	0.00396016286024721\\
25	0.00396018538406209\\
26	0.00396020844613001\\
27	0.0039602320593602\\
28	0.00396025623697302\\
29	0.00396028099250763\\
30	0.00396030633982961\\
31	0.00396033229313887\\
32	0.00396035886697801\\
33	0.00396038607624027\\
34	0.00396041393617837\\
35	0.00396044246241313\\
36	0.00396047167094243\\
37	0.00396050157815039\\
38	0.00396053220081675\\
39	0.00396056355612649\\
40	0.00396059566167981\\
41	0.00396062853550198\\
42	0.0039606621960541\\
43	0.00396069666224333\\
44	0.00396073195343398\\
45	0.00396076808945855\\
46	0.00396080509062931\\
47	0.00396084297774973\\
48	0.00396088177212679\\
49	0.00396092149558308\\
50	0.00396096217046938\\
51	0.00396100381967768\\
52	0.00396104646665429\\
53	0.00396109013541348\\
54	0.00396113485055132\\
55	0.0039611806372599\\
56	0.00396122752134198\\
57	0.00396127552922588\\
58	0.00396132468798085\\
59	0.00396137502533265\\
60	0.00396142656967982\\
61	0.00396147935011013\\
62	0.00396153339641734\\
63	0.00396158873911868\\
64	0.00396164540947253\\
65	0.00396170343949658\\
66	0.00396176286198673\\
67	0.00396182371053586\\
68	0.00396188601955362\\
69	0.00396194982428658\\
70	0.00396201516083861\\
71	0.0039620820661922\\
72	0.00396215057823004\\
73	0.00396222073575712\\
74	0.00396229257852349\\
75	0.00396236614724766\\
76	0.00396244148364046\\
77	0.00396251863042954\\
78	0.00396259763138431\\
79	0.00396267853134196\\
80	0.00396276137623372\\
81	0.00396284621311189\\
82	0.00396293309017775\\
83	0.00396302205680982\\
84	0.00396311316359311\\
85	0.00396320646234906\\
86	0.00396330200616623\\
87	0.00396339984943172\\
88	0.00396350004786328\\
89	0.00396360265854262\\
90	0.00396370773994926\\
91	0.00396381535199517\\
92	0.0039639255560606\\
93	0.00396403841503058\\
94	0.00396415399333251\\
95	0.00396427235697452\\
96	0.00396439357358506\\
97	0.00396451771245327\\
98	0.00396464484457065\\
99	0.00396477504267351\\
100	0.00396490838128676\\
101	0.00396504493676871\\
102	0.00396518478735701\\
103	0.0039653280132159\\
104	0.00396547469648452\\
105	0.00396562492132671\\
106	0.00396577877398183\\
107	0.00396593634281711\\
108	0.00396609771838132\\
109	0.00396626299345977\\
110	0.00396643226313084\\
111	0.00396660562482398\\
112	0.00396678317837919\\
113	0.00396696502610827\\
114	0.00396715127285725\\
115	0.00396734202607095\\
116	0.00396753739585896\\
117	0.0039677374950636\\
118	0.00396794243932951\\
119	0.00396815234717506\\
120	0.00396836734006598\\
121	0.0039685875424906\\
122	0.00396881308203743\\
123	0.00396904408947463\\
124	0.00396928069883171\\
125	0.0039695230474835\\
126	0.00396977127623636\\
127	0.00397002552941675\\
128	0.0039702859549623\\
129	0.00397055270451526\\
130	0.00397082593351874\\
131	0.00397110580131541\\
132	0.00397139247124912\\
133	0.00397168611076932\\
134	0.00397198689153836\\
135	0.00397229498954192\\
136	0.00397261058520255\\
137	0.00397293386349641\\
138	0.00397326501407326\\
139	0.00397360423138033\\
140	0.00397395171478916\\
141	0.00397430766872666\\
142	0.00397467230280983\\
143	0.00397504583198428\\
144	0.00397542847666717\\
145	0.00397582046289416\\
146	0.00397622202247093\\
147	0.00397663339312926\\
148	0.00397705481868773\\
149	0.00397748654921761\\
150	0.00397792884121358\\
151	0.00397838195776999\\
152	0.00397884616876252\\
153	0.00397932175103576\\
154	0.00397980898859666\\
155	0.00398030817281407\\
156	0.00398081960262511\\
157	0.00398134358474793\\
158	0.00398188043390201\\
159	0.00398243047303523\\
160	0.00398299403355913\\
161	0.00398357145559186\\
162	0.00398416308820985\\
163	0.00398476928970786\\
164	0.00398539042786867\\
165	0.00398602688024207\\
166	0.00398667903443408\\
167	0.00398734728840669\\
168	0.00398803205078902\\
169	0.00398873374119959\\
170	0.0039894527905814\\
171	0.00399018964154978\\
172	0.00399094474875364\\
173	0.00399171857925138\\
174	0.00399251161290171\\
175	0.00399332434277046\\
176	0.00399415727555434\\
177	0.00399501093202239\\
178	0.00399588584747633\\
179	0.00399678257223094\\
180	0.0039977016721154\\
181	0.00399864372899717\\
182	0.00399960934132978\\
183	0.00400059912472559\\
184	0.00400161371255571\\
185	0.0040026537565782\\
186	0.00400371992759688\\
187	0.00400481291615253\\
188	0.00400593343324845\\
189	0.00400708221111312\\
190	0.00400826000400188\\
191	0.00400946758904113\\
192	0.00401070576711679\\
193	0.00401197536381151\\
194	0.00401327723039289\\
195	0.00401461224485727\\
196	0.00401598131303215\\
197	0.00401738536974248\\
198	0.00401882538004461\\
199	0.0040203023405339\\
200	0.00402181728073037\\
201	0.004023371264549\\
202	0.00402496539186076\\
203	0.00402660080015172\\
204	0.00402827866628696\\
205	0.00403000020838805\\
206	0.00403176668783252\\
207	0.00403357941138521\\
208	0.00403543973347098\\
209	0.00403734905860098\\
210	0.0040393088439634\\
211	0.00404132060219247\\
212	0.00404338590432935\\
213	0.0040455063829901\\
214	0.00404768373575706\\
215	0.0040499197288111\\
216	0.00405221620082364\\
217	0.00405457506712875\\
218	0.00405699832419687\\
219	0.00405948805443328\\
220	0.0040620464313263\\
221	0.00406467572497081\\
222	0.00406737830799553\\
223	0.00407015666192263\\
224	0.00407301338399044\\
225	0.0040759511944706\\
226	0.00407897294451281\\
227	0.00408208162454912\\
228	0.00408528037329207\\
229	0.00408857248735747\\
230	0.00409196143154372\\
231	0.00409545084979412\\
232	0.00409904457686569\\
233	0.00410274665072023\\
234	0.0041065613256449\\
235	0.00411049308609495\\
236	0.00411454666123576\\
237	0.0041187270401359\\
238	0.00412303948753413\\
239	0.00412748956006253\\
240	0.00413208312275696\\
241	0.00413682636561934\\
242	0.00414172581991221\\
243	0.00414678837375734\\
244	0.00415202128647276\\
245	0.00415743220090805\\
246	0.00416302915281537\\
247	0.00416882057601468\\
248	0.00417481530175669\\
249	0.00418102255024025\\
250	0.00418745191167736\\
251	0.00419411331358745\\
252	0.00420101697010822\\
253	0.00420817330798137\\
254	0.00421559286245153\\
255	0.00422328613452883\\
256	0.00423126339881372\\
257	0.00423953444824777\\
258	0.00424810825858506\\
259	0.0042569925508814\\
260	0.00426619322463004\\
261	0.00427571362701491\\
262	0.00428555361468141\\
263	0.00429570835280999\\
264	0.00430616678094397\\
265	0.00431690965254892\\
266	0.00432790142762682\\
267	0.00433898891337839\\
268	0.00435017149105044\\
269	0.00436144839634799\\
270	0.00437281870481662\\
271	0.00438428131576857\\
272	0.00439583493461043\\
273	0.00440747805341768\\
274	0.00441920892959058\\
275	0.00443102556241362\\
276	0.00444292566732974\\
277	0.00445490664773285\\
278	0.00446696556407647\\
279	0.00447909910010651\\
280	0.00449130352585494\\
281	0.00450357465720858\\
282	0.00451590781213734\\
283	0.00452829776324319\\
284	0.00454073868650704\\
285	0.00455322410617472\\
286	0.00456574683581455\\
287	0.00457829891571173\\
288	0.00459087154694168\\
289	0.00460345502271227\\
290	0.0046160386578906\\
291	0.00462861071806991\\
292	0.00464115835011131\\
293	0.00465366751685637\\
294	0.00466612293970611\\
295	0.00467850805406123\\
296	0.0046908049843093\\
297	0.00470299454722566\\
298	0.00471505629544852\\
299	0.00472696861618528\\
300	0.00473870890436941\\
301	0.00475025383266734\\
302	0.00476157973286847\\
303	0.00477266299195848\\
304	0.00478348176758782\\
305	0.00479401819324593\\
306	0.00480425880689436\\
307	0.00481419678173815\\
308	0.00482383488528588\\
309	0.00483318939227866\\
310	0.00484236411110974\\
311	0.00485149449666169\\
312	0.00486057507628651\\
313	0.00486960039861545\\
314	0.00487856510697842\\
315	0.0048874640255473\\
316	0.00489629226121847\\
317	0.00490504469125553\\
318	0.00491371377281994\\
319	0.00492229177784234\\
320	0.00493077082428824\\
321	0.00493914291552824\\
322	0.00494739998910972\\
323	0.00495553397636532\\
324	0.00496353687442899\\
325	0.00497140083234837\\
326	0.00497911825306907\\
327	0.00498668191308824\\
328	0.00499408510146902\\
329	0.00500132177949659\\
330	0.00500838676088834\\
331	0.00501527590213421\\
332	0.00502198629166413\\
333	0.00502851647319545\\
334	0.00503486667732287\\
335	0.00504103905245901\\
336	0.00504703788330689\\
337	0.00505286978844215\\
338	0.00505854388973855\\
339	0.00506407188869534\\
340	0.00506946795207431\\
341	0.00507474835771505\\
342	0.00507993079070375\\
343	0.00508503314178086\\
344	0.00509006849295808\\
345	0.00509503944889471\\
346	0.00509994524652635\\
347	0.00510478542672716\\
348	0.00510955986527552\\
349	0.00511426880380323\\
350	0.00511891288001494\\
351	0.00512349315628538\\
352	0.0051280111454621\\
353	0.00513246883287037\\
354	0.00513686869355852\\
355	0.00514121370276509\\
356	0.00514550733748316\\
357	0.00514975356694455\\
358	0.00515395682903118\\
359	0.00515812198868018\\
360	0.00516225427555822\\
361	0.00516635919917049\\
362	0.00517044243948941\\
363	0.00517450971242414\\
364	0.00517856661157374\\
365	0.005182618431124\\
366	0.00518666998001815\\
367	0.00519072540531589\\
368	0.00519478805272838\\
369	0.0051988603984944\\
370	0.00520294453539825\\
371	0.0052070425919397\\
372	0.00521115679769662\\
373	0.00521528947165547\\
374	0.00521944300822667\\
375	0.00522361986090933\\
376	0.00522782252366814\\
377	0.00523205351024857\\
378	0.00523631533179417\\
379	0.00524061047329331\\
380	0.0052449413695931\\
381	0.00524931038196096\\
382	0.00525371977643563\\
383	0.00525817170547221\\
384	0.0052626681946173\\
385	0.00526721113609796\\
386	0.00527180229117754\\
387	0.00527644330274793\\
388	0.00528113571851391\\
389	0.00528588102298737\\
390	0.00529068068003026\\
391	0.00529553614885919\\
392	0.0053004488806253\\
393	0.00530542031478252\\
394	0.00531045187588167\\
395	0.00531554497092085\\
396	0.00532070098737936\\
397	0.00532592129205031\\
398	0.00533120723076478\\
399	0.00533656012905985\\
400	0.00534198129378797\\
401	0.0053474720155889\\
402	0.0053530335720522\\
403	0.00535866723129939\\
404	0.00536437425565266\\
405	0.00537015590522182\\
406	0.00537601344262375\\
407	0.00538194814960723\\
408	0.00538796129411446\\
409	0.00539405405128865\\
410	0.0054002275981115\\
411	0.0054064831151994\\
412	0.0054128217877218\\
413	0.00541924480637303\\
414	0.0054257533684365\\
415	0.00543234867890174\\
416	0.00543903195159316\\
417	0.00544580441027162\\
418	0.00545266728967045\\
419	0.00545962183642861\\
420	0.00546666930989435\\
421	0.00547381098279216\\
422	0.00548104814177019\\
423	0.0054883820878729\\
424	0.00549581413699891\\
425	0.00550334562036698\\
426	0.00551097788498487\\
427	0.00551871229411495\\
428	0.00552655022773119\\
429	0.00553449308296273\\
430	0.00554254227452093\\
431	0.00555069923510744\\
432	0.00555896541580314\\
433	0.00556734228643954\\
434	0.00557583133595569\\
435	0.00558443407274487\\
436	0.00559315202499602\\
437	0.00560198674103388\\
438	0.00561093978965945\\
439	0.00562001276049151\\
440	0.00562920726430852\\
441	0.00563852493339155\\
442	0.00564796742186839\\
443	0.00565753640605954\\
444	0.00566723358482715\\
445	0.00567706067992759\\
446	0.00568701943636906\\
447	0.00569711162277522\\
448	0.00570733903175618\\
449	0.00571770348028746\\
450	0.00572820681009836\\
451	0.00573885088807027\\
452	0.00574963760664607\\
453	0.00576056888425166\\
454	0.00577164666573088\\
455	0.00578287292279483\\
456	0.00579424965448764\\
457	0.00580577888766963\\
458	0.00581746267752005\\
459	0.00582930310806115\\
460	0.00584130229270553\\
461	0.00585346237482923\\
462	0.00586578552837295\\
463	0.00587827395847425\\
464	0.00589092990213355\\
465	0.00590375562891754\\
466	0.00591675344170365\\
467	0.00592992567746969\\
468	0.00594327470813304\\
469	0.00595680294144458\\
470	0.00597051282194282\\
471	0.00598440683197411\\
472	0.00599848749278587\\
473	0.00601275736570005\\
474	0.00602721905337455\\
475	0.00604187520116198\\
476	0.00605672849857493\\
477	0.00607178168086813\\
478	0.0060870375307495\\
479	0.00610249888023157\\
480	0.00611816861263728\\
481	0.00613404966477412\\
482	0.00615014502929169\\
483	0.00616645775723915\\
484	0.00618299096083938\\
485	0.00619974781649722\\
486	0.00621673156806019\\
487	0.0062339455303493\\
488	0.0062513930929778\\
489	0.00626907772447446\\
490	0.00628700297672548\\
491	0.00630517248974692\\
492	0.00632358999679346\\
493	0.006342259329804\\
494	0.00636118442517365\\
495	0.00638036932982981\\
496	0.00639981820757165\\
497	0.00641953534561057\\
498	0.00643952516121694\\
499	0.00645979220834055\\
500	0.00648034118401903\\
501	0.00650117693432292\\
502	0.00652230445949956\\
503	0.00654372891786793\\
504	0.00656545562787597\\
505	0.0065874900675518\\
506	0.00660983787035161\\
507	0.00663250481611568\\
508	0.0066554968154751\\
509	0.00667881988558296\\
510	0.00670248011445096\\
511	0.00672648361042255\\
512	0.00675083643236417\\
513	0.00677554449495819\\
514	0.00680061344196787\\
515	0.00682604847843297\\
516	0.00685185415034493\\
517	0.00687803405730904\\
518	0.0069045904798648\\
519	0.00693152389830011\\
520	0.00695883237369566\\
521	0.00698651075424847\\
522	0.00701454966022969\\
523	0.00704293418871195\\
524	0.00707164226379601\\
525	0.00710056610684932\\
526	0.00712974350113507\\
527	0.00715929329431106\\
528	0.00718934245787863\\
529	0.00721994484589615\\
530	0.0072511468419841\\
531	0.00728298780461047\\
532	0.00731550855386092\\
533	0.00734875362171091\\
534	0.00738277285511294\\
535	0.00741762248749887\\
536	0.00745336635521868\\
537	0.00749007722424768\\
538	0.00752783839423504\\
539	0.00756674560848086\\
540	0.00760690935443786\\
541	0.00764808540321859\\
542	0.00768701508418313\\
543	0.00772433072101707\\
544	0.00776188162433868\\
545	0.0077999527261911\\
546	0.00783859087838171\\
547	0.00787778165752892\\
548	0.00791749614938111\\
549	0.00795769533073896\\
550	0.00799832857317591\\
551	0.00803932989916792\\
552	0.00808060984117103\\
553	0.00812101779645035\\
554	0.00816104474617506\\
555	0.00820139887366875\\
556	0.00824209241508463\\
557	0.0082831028968625\\
558	0.00832440529286259\\
559	0.00836597422744438\\
560	0.0084077846659062\\
561	0.00844981241998907\\
562	0.00849203327218037\\
563	0.00853420740548651\\
564	0.00857628031085049\\
565	0.00861870130330938\\
566	0.00866146599325849\\
567	0.00870455835161939\\
568	0.00874796052647332\\
569	0.00879165272107617\\
570	0.00883561306369943\\
571	0.00887981747569646\\
572	0.00892423953986743\\
573	0.00896885037183584\\
574	0.0090136184980118\\
575	0.00905850974480618\\
576	0.00910348714512921\\
577	0.00914851086992285\\
578	0.00919353819462572\\
579	0.00923852351315244\\
580	0.0092834184153165\\
581	0.00932817184778418\\
582	0.00937273038379323\\
583	0.00941703863317462\\
584	0.00946103983179201\\
585	0.00950467665821592\\
586	0.00954789233435465\\
587	0.00959063207273936\\
588	0.00963284492704104\\
589	0.0096744860587673\\
590	0.00971551928305088\\
591	0.00975591931512916\\
592	0.00979567191274874\\
593	0.00983476678232329\\
594	0.00987312984006299\\
595	0.00991038882774525\\
596	0.00994573390547615\\
597	0.0099771937715668\\
598	0.00999970795535495\\
599	0\\
600	0\\
};
\end{axis}
\end{tikzpicture}%
 
  \caption{Discrete Time}
\end{subfigure}\\

\leavevmode\smash{\makebox[0pt]{\hspace{-7em}% HORIZONTAL POSITION           
  \rotatebox[origin=l]{90}{\hspace{20em}% VERTICAL POSITION
    Depth $\delta^+$}%
}}\hspace{0pt plus 1filll}\null

Time (s)

\vspace{1cm}
\begin{subfigure}{\linewidth}
  \centering
  \tikzsetnextfilename{altdeltalegend}
  \definecolor{mycolor1}{rgb}{0.00000,1.00000,0.14286}%
\definecolor{mycolor2}{rgb}{0.00000,1.00000,0.28571}%
\definecolor{mycolor3}{rgb}{0.00000,1.00000,0.42857}%
\definecolor{mycolor4}{rgb}{0.00000,1.00000,0.57143}%
\definecolor{mycolor5}{rgb}{0.00000,1.00000,0.71429}%
\definecolor{mycolor6}{rgb}{0.00000,1.00000,0.85714}%
\definecolor{mycolor7}{rgb}{0.00000,1.00000,1.00000}%
\definecolor{mycolor8}{rgb}{0.00000,0.87500,1.00000}%
\definecolor{mycolor9}{rgb}{0.00000,0.62500,1.00000}%
\definecolor{mycolor10}{rgb}{0.12500,0.00000,1.00000}%
\definecolor{mycolor11}{rgb}{0.25000,0.00000,1.00000}%
\definecolor{mycolor12}{rgb}{0.37500,0.00000,1.00000}%
\definecolor{mycolor13}{rgb}{0.50000,0.00000,1.00000}%
\definecolor{mycolor14}{rgb}{0.62500,0.00000,1.00000}%
\definecolor{mycolor15}{rgb}{0.75000,0.00000,1.00000}%
\definecolor{mycolor16}{rgb}{0.87500,0.00000,1.00000}%
\definecolor{mycolor17}{rgb}{1.00000,0.00000,1.00000}%
\definecolor{mycolor18}{rgb}{1.00000,0.00000,0.87500}%
\definecolor{mycolor19}{rgb}{1.00000,0.00000,0.62500}%
\definecolor{mycolor20}{rgb}{0.85714,0.00000,0.00000}%
\definecolor{mycolor21}{rgb}{0.71429,0.00000,0.00000}%
%[trim axis left, trim axis right]
\begin{tikzpicture}
\begin{axis}[%
    hide axis,
    scale only axis,
    height=0pt,
    width=0pt,
    point meta min=-19,
    point meta max=19,
    colormap={mymap}{[1pt] rgb(0pt)=(0,1,0); rgb(7pt)=(0,1,1); rgb(15pt)=(0,0,1); rgb(23pt)=(1,0,1); rgb(31pt)=(1,0,0); rgb(38pt)=(0,0,0)},
    colorbar horizontal,
    colorbar style={width=15cm,xtick={{-15},{-10},{-5},{0},{5},{10},{15}}}
    %colorbar style={separate axis lines,every outer x axis line/.append style={black},every x tick label/.append style={font=\color{black}},every outer y axis line/.append style={black},every y tick label/.append style={font=\color{black}},yticklabels={{-19},{-17},{-15},{-13},{-11},{-9},{-7},{-5},{-3},{-1},{1},{3},{5},{7},{9},{11},{13},{15},{17},{19}}}
]%
    \addplot [draw=none] coordinates {(0,0)};
\end{axis}
\end{tikzpicture}
 
\end{subfigure}%
  \caption{Optimal buy depths $\delta^+$ for Markov state $Z=(\rho = 0, \Delta S = 0)$, implying neutral imbalance and no previous price change. We expect no change in midprice.}
  \label{fig:comp_dp_z8}
\end{figure}

\begin{figure}
\centering
\begin{subfigure}{.45\linewidth}
  \centering
  \setlength\figureheight{\linewidth} 
  \setlength\figurewidth{\linewidth}
  \tikzsetnextfilename{dp_colorbar/dp_cts_z15}
  % This file was created by matlab2tikz.
%
%The latest updates can be retrieved from
%  http://www.mathworks.com/matlabcentral/fileexchange/22022-matlab2tikz-matlab2tikz
%where you can also make suggestions and rate matlab2tikz.
%
\definecolor{mycolor1}{rgb}{1.00000,0.00000,1.00000}%
%
\begin{tikzpicture}

\begin{axis}[%
width=4.564in,
height=3.803in,
at={(1.067in,0.513in)},
scale only axis,
every outer x axis line/.append style={black},
every x tick label/.append style={font=\color{black}},
xmin=0,
xmax=100,
xlabel={Time},
every outer y axis line/.append style={black},
every y tick label/.append style={font=\color{black}},
ymin=0,
ymax=0.012,
ylabel={Depth $\delta$},
axis background/.style={fill=white},
title={Z=15},
axis x line*=bottom,
axis y line*=left,
legend style={legend cell align=left,align=left,draw=black}
]
\addplot [color=green,dashed,forget plot]
  table[row sep=crcr]{%
0.01	0.01\\
0.02	0.01\\
0.03	0.01\\
0.04	0.01\\
0.05	0.01\\
0.06	0.01\\
0.07	0.01\\
0.08	0.01\\
0.09	0.01\\
0.1	0.01\\
0.11	0.01\\
0.12	0.01\\
0.13	0.01\\
0.14	0.01\\
0.15	0.01\\
0.16	0.01\\
0.17	0.01\\
0.18	0.01\\
0.19	0.01\\
0.2	0.01\\
0.21	0.01\\
0.22	0.01\\
0.23	0.01\\
0.24	0.01\\
0.25	0.01\\
0.26	0.01\\
0.27	0.01\\
0.28	0.01\\
0.29	0.01\\
0.3	0.01\\
0.31	0.01\\
0.32	0.01\\
0.33	0.01\\
0.34	0.01\\
0.35	0.01\\
0.36	0.01\\
0.37	0.01\\
0.38	0.01\\
0.39	0.01\\
0.4	0.01\\
0.41	0.01\\
0.42	0.01\\
0.43	0.01\\
0.44	0.01\\
0.45	0.01\\
0.46	0.01\\
0.47	0.01\\
0.48	0.01\\
0.49	0.01\\
0.5	0.01\\
0.51	0.01\\
0.52	0.01\\
0.53	0.01\\
0.54	0.01\\
0.55	0.01\\
0.56	0.01\\
0.57	0.01\\
0.58	0.01\\
0.59	0.01\\
0.6	0.01\\
0.61	0.01\\
0.62	0.01\\
0.63	0.01\\
0.64	0.01\\
0.65	0.01\\
0.66	0.01\\
0.67	0.01\\
0.68	0.01\\
0.69	0.01\\
0.7	0.01\\
0.71	0.01\\
0.72	0.01\\
0.73	0.01\\
0.74	0.01\\
0.75	0.01\\
0.76	0.01\\
0.77	0.01\\
0.78	0.01\\
0.79	0.01\\
0.8	0.01\\
0.81	0.01\\
0.82	0.01\\
0.83	0.01\\
0.84	0.01\\
0.85	0.01\\
0.86	0.01\\
0.87	0.01\\
0.88	0.01\\
0.89	0.01\\
0.9	0.01\\
0.91	0.01\\
0.92	0.01\\
0.93	0.01\\
0.94	0.01\\
0.95	0.01\\
0.96	0.01\\
0.97	0.01\\
0.98	0.01\\
0.99	0.01\\
1	0.01\\
1.01	0.01\\
1.02	0.01\\
1.03	0.01\\
1.04	0.01\\
1.05	0.01\\
1.06	0.01\\
1.07	0.01\\
1.08	0.01\\
1.09	0.01\\
1.1	0.01\\
1.11	0.01\\
1.12	0.01\\
1.13	0.01\\
1.14	0.01\\
1.15	0.01\\
1.16	0.01\\
1.17	0.01\\
1.18	0.01\\
1.19	0.01\\
1.2	0.01\\
1.21	0.01\\
1.22	0.01\\
1.23	0.01\\
1.24	0.01\\
1.25	0.01\\
1.26	0.01\\
1.27	0.01\\
1.28	0.01\\
1.29	0.01\\
1.3	0.01\\
1.31	0.01\\
1.32	0.01\\
1.33	0.01\\
1.34	0.01\\
1.35	0.01\\
1.36	0.01\\
1.37	0.01\\
1.38	0.01\\
1.39	0.01\\
1.4	0.01\\
1.41	0.01\\
1.42	0.01\\
1.43	0.01\\
1.44	0.01\\
1.45	0.01\\
1.46	0.01\\
1.47	0.01\\
1.48	0.01\\
1.49	0.01\\
1.5	0.01\\
1.51	0.01\\
1.52	0.01\\
1.53	0.01\\
1.54	0.01\\
1.55	0.01\\
1.56	0.01\\
1.57	0.01\\
1.58	0.01\\
1.59	0.01\\
1.6	0.01\\
1.61	0.01\\
1.62	0.01\\
1.63	0.01\\
1.64	0.01\\
1.65	0.01\\
1.66	0.01\\
1.67	0.01\\
1.68	0.01\\
1.69	0.01\\
1.7	0.01\\
1.71	0.01\\
1.72	0.01\\
1.73	0.01\\
1.74	0.01\\
1.75	0.01\\
1.76	0.01\\
1.77	0.01\\
1.78	0.01\\
1.79	0.01\\
1.8	0.01\\
1.81	0.01\\
1.82	0.01\\
1.83	0.01\\
1.84	0.01\\
1.85	0.01\\
1.86	0.01\\
1.87	0.01\\
1.88	0.01\\
1.89	0.01\\
1.9	0.01\\
1.91	0.01\\
1.92	0.01\\
1.93	0.01\\
1.94	0.01\\
1.95	0.01\\
1.96	0.01\\
1.97	0.01\\
1.98	0.01\\
1.99	0.01\\
2	0.01\\
2.01	0.01\\
2.02	0.01\\
2.03	0.01\\
2.04	0.01\\
2.05	0.01\\
2.06	0.01\\
2.07	0.01\\
2.08	0.01\\
2.09	0.01\\
2.1	0.01\\
2.11	0.01\\
2.12	0.01\\
2.13	0.01\\
2.14	0.01\\
2.15	0.01\\
2.16	0.01\\
2.17	0.01\\
2.18	0.01\\
2.19	0.01\\
2.2	0.01\\
2.21	0.01\\
2.22	0.01\\
2.23	0.01\\
2.24	0.01\\
2.25	0.01\\
2.26	0.01\\
2.27	0.01\\
2.28	0.01\\
2.29	0.01\\
2.3	0.01\\
2.31	0.01\\
2.32	0.01\\
2.33	0.01\\
2.34	0.01\\
2.35	0.01\\
2.36	0.01\\
2.37	0.01\\
2.38	0.01\\
2.39	0.01\\
2.4	0.01\\
2.41	0.01\\
2.42	0.01\\
2.43	0.01\\
2.44	0.01\\
2.45	0.01\\
2.46	0.01\\
2.47	0.01\\
2.48	0.01\\
2.49	0.01\\
2.5	0.01\\
2.51	0.01\\
2.52	0.01\\
2.53	0.01\\
2.54	0.01\\
2.55	0.01\\
2.56	0.01\\
2.57	0.01\\
2.58	0.01\\
2.59	0.01\\
2.6	0.01\\
2.61	0.01\\
2.62	0.01\\
2.63	0.01\\
2.64	0.01\\
2.65	0.01\\
2.66	0.01\\
2.67	0.01\\
2.68	0.01\\
2.69	0.01\\
2.7	0.01\\
2.71	0.01\\
2.72	0.01\\
2.73	0.01\\
2.74	0.01\\
2.75	0.01\\
2.76	0.01\\
2.77	0.01\\
2.78	0.01\\
2.79	0.01\\
2.8	0.01\\
2.81	0.01\\
2.82	0.01\\
2.83	0.01\\
2.84	0.01\\
2.85	0.01\\
2.86	0.01\\
2.87	0.01\\
2.88	0.01\\
2.89	0.01\\
2.9	0.01\\
2.91	0.01\\
2.92	0.01\\
2.93	0.01\\
2.94	0.01\\
2.95	0.01\\
2.96	0.01\\
2.97	0.01\\
2.98	0.01\\
2.99	0.01\\
3	0.01\\
3.01	0.01\\
3.02	0.01\\
3.03	0.01\\
3.04	0.01\\
3.05	0.01\\
3.06	0.01\\
3.07	0.01\\
3.08	0.01\\
3.09	0.01\\
3.1	0.01\\
3.11	0.01\\
3.12	0.01\\
3.13	0.01\\
3.14	0.01\\
3.15	0.01\\
3.16	0.01\\
3.17	0.01\\
3.18	0.01\\
3.19	0.01\\
3.2	0.01\\
3.21	0.01\\
3.22	0.01\\
3.23	0.01\\
3.24	0.01\\
3.25	0.01\\
3.26	0.01\\
3.27	0.01\\
3.28	0.01\\
3.29	0.01\\
3.3	0.01\\
3.31	0.01\\
3.32	0.01\\
3.33	0.01\\
3.34	0.01\\
3.35	0.01\\
3.36	0.01\\
3.37	0.01\\
3.38	0.01\\
3.39	0.01\\
3.4	0.01\\
3.41	0.01\\
3.42	0.01\\
3.43	0.01\\
3.44	0.01\\
3.45	0.01\\
3.46	0.01\\
3.47	0.01\\
3.48	0.01\\
3.49	0.01\\
3.5	0.01\\
3.51	0.01\\
3.52	0.01\\
3.53	0.01\\
3.54	0.01\\
3.55	0.01\\
3.56	0.01\\
3.57	0.01\\
3.58	0.01\\
3.59	0.01\\
3.6	0.01\\
3.61	0.01\\
3.62	0.01\\
3.63	0.01\\
3.64	0.01\\
3.65	0.01\\
3.66	0.01\\
3.67	0.01\\
3.68	0.01\\
3.69	0.01\\
3.7	0.01\\
3.71	0.01\\
3.72	0.01\\
3.73	0.01\\
3.74	0.01\\
3.75	0.01\\
3.76	0.01\\
3.77	0.01\\
3.78	0.01\\
3.79	0.01\\
3.8	0.01\\
3.81	0.01\\
3.82	0.01\\
3.83	0.01\\
3.84	0.01\\
3.85	0.01\\
3.86	0.01\\
3.87	0.01\\
3.88	0.01\\
3.89	0.01\\
3.9	0.01\\
3.91	0.01\\
3.92	0.01\\
3.93	0.01\\
3.94	0.01\\
3.95	0.01\\
3.96	0.01\\
3.97	0.01\\
3.98	0.01\\
3.99	0.01\\
4	0.01\\
4.01	0.01\\
4.02	0.01\\
4.03	0.01\\
4.04	0.01\\
4.05	0.01\\
4.06	0.01\\
4.07	0.01\\
4.08	0.01\\
4.09	0.01\\
4.1	0.01\\
4.11	0.01\\
4.12	0.01\\
4.13	0.01\\
4.14	0.01\\
4.15	0.01\\
4.16	0.01\\
4.17	0.01\\
4.18	0.01\\
4.19	0.01\\
4.2	0.01\\
4.21	0.01\\
4.22	0.01\\
4.23	0.01\\
4.24	0.01\\
4.25	0.01\\
4.26	0.01\\
4.27	0.01\\
4.28	0.01\\
4.29	0.01\\
4.3	0.01\\
4.31	0.01\\
4.32	0.01\\
4.33	0.01\\
4.34	0.01\\
4.35	0.01\\
4.36	0.01\\
4.37	0.01\\
4.38	0.01\\
4.39	0.01\\
4.4	0.01\\
4.41	0.01\\
4.42	0.01\\
4.43	0.01\\
4.44	0.01\\
4.45	0.01\\
4.46	0.01\\
4.47	0.01\\
4.48	0.01\\
4.49	0.01\\
4.5	0.01\\
4.51	0.01\\
4.52	0.01\\
4.53	0.01\\
4.54	0.01\\
4.55	0.01\\
4.56	0.01\\
4.57	0.01\\
4.58	0.01\\
4.59	0.01\\
4.6	0.01\\
4.61	0.01\\
4.62	0.01\\
4.63	0.01\\
4.64	0.01\\
4.65	0.01\\
4.66	0.01\\
4.67	0.01\\
4.68	0.01\\
4.69	0.01\\
4.7	0.01\\
4.71	0.01\\
4.72	0.01\\
4.73	0.01\\
4.74	0.01\\
4.75	0.01\\
4.76	0.01\\
4.77	0.01\\
4.78	0.01\\
4.79	0.01\\
4.8	0.01\\
4.81	0.01\\
4.82	0.01\\
4.83	0.01\\
4.84	0.01\\
4.85	0.01\\
4.86	0.01\\
4.87	0.01\\
4.88	0.01\\
4.89	0.01\\
4.9	0.01\\
4.91	0.01\\
4.92	0.01\\
4.93	0.01\\
4.94	0.01\\
4.95	0.01\\
4.96	0.01\\
4.97	0.01\\
4.98	0.01\\
4.99	0.01\\
5	0.01\\
5.01	0.01\\
5.02	0.01\\
5.03	0.01\\
5.04	0.01\\
5.05	0.01\\
5.06	0.01\\
5.07	0.01\\
5.08	0.01\\
5.09	0.01\\
5.1	0.01\\
5.11	0.01\\
5.12	0.01\\
5.13	0.01\\
5.14	0.01\\
5.15	0.01\\
5.16	0.01\\
5.17	0.01\\
5.18	0.01\\
5.19	0.01\\
5.2	0.01\\
5.21	0.01\\
5.22	0.01\\
5.23	0.01\\
5.24	0.01\\
5.25	0.01\\
5.26	0.01\\
5.27	0.01\\
5.28	0.01\\
5.29	0.01\\
5.3	0.01\\
5.31	0.01\\
5.32	0.01\\
5.33	0.01\\
5.34	0.01\\
5.35	0.01\\
5.36	0.01\\
5.37	0.01\\
5.38	0.01\\
5.39	0.01\\
5.4	0.01\\
5.41	0.01\\
5.42	0.01\\
5.43	0.01\\
5.44	0.01\\
5.45	0.01\\
5.46	0.01\\
5.47	0.01\\
5.48	0.01\\
5.49	0.01\\
5.5	0.01\\
5.51	0.01\\
5.52	0.01\\
5.53	0.01\\
5.54	0.01\\
5.55	0.01\\
5.56	0.01\\
5.57	0.01\\
5.58	0.01\\
5.59	0.01\\
5.6	0.01\\
5.61	0.01\\
5.62	0.01\\
5.63	0.01\\
5.64	0.01\\
5.65	0.01\\
5.66	0.01\\
5.67	0.01\\
5.68	0.01\\
5.69	0.01\\
5.7	0.01\\
5.71	0.01\\
5.72	0.01\\
5.73	0.01\\
5.74	0.01\\
5.75	0.01\\
5.76	0.01\\
5.77	0.01\\
5.78	0.01\\
5.79	0.01\\
5.8	0.01\\
5.81	0.01\\
5.82	0.01\\
5.83	0.01\\
5.84	0.01\\
5.85	0.01\\
5.86	0.01\\
5.87	0.01\\
5.88	0.01\\
5.89	0.01\\
5.9	0.01\\
5.91	0.01\\
5.92	0.01\\
5.93	0.01\\
5.94	0.01\\
5.95	0.01\\
5.96	0.01\\
5.97	0.01\\
5.98	0.01\\
5.99	0.01\\
6	0.01\\
6.01	0.01\\
6.02	0.01\\
6.03	0.01\\
6.04	0.01\\
6.05	0.01\\
6.06	0.01\\
6.07	0.01\\
6.08	0.01\\
6.09	0.01\\
6.1	0.01\\
6.11	0.01\\
6.12	0.01\\
6.13	0.01\\
6.14	0.01\\
6.15	0.01\\
6.16	0.01\\
6.17	0.01\\
6.18	0.01\\
6.19	0.01\\
6.2	0.01\\
6.21	0.01\\
6.22	0.01\\
6.23	0.01\\
6.24	0.01\\
6.25	0.01\\
6.26	0.01\\
6.27	0.01\\
6.28	0.01\\
6.29	0.01\\
6.3	0.01\\
6.31	0.01\\
6.32	0.01\\
6.33	0.01\\
6.34	0.01\\
6.35	0.01\\
6.36	0.01\\
6.37	0.01\\
6.38	0.01\\
6.39	0.01\\
6.4	0.01\\
6.41	0.01\\
6.42	0.01\\
6.43	0.01\\
6.44	0.01\\
6.45	0.01\\
6.46	0.01\\
6.47	0.01\\
6.48	0.01\\
6.49	0.01\\
6.5	0.01\\
6.51	0.01\\
6.52	0.01\\
6.53	0.01\\
6.54	0.01\\
6.55	0.01\\
6.56	0.01\\
6.57	0.01\\
6.58	0.01\\
6.59	0.01\\
6.6	0.01\\
6.61	0.01\\
6.62	0.01\\
6.63	0.01\\
6.64	0.01\\
6.65	0.01\\
6.66	0.01\\
6.67	0.01\\
6.68	0.01\\
6.69	0.01\\
6.7	0.01\\
6.71	0.01\\
6.72	0.01\\
6.73	0.01\\
6.74	0.01\\
6.75	0.01\\
6.76	0.01\\
6.77	0.01\\
6.78	0.01\\
6.79	0.01\\
6.8	0.01\\
6.81	0.01\\
6.82	0.01\\
6.83	0.01\\
6.84	0.01\\
6.85	0.01\\
6.86	0.01\\
6.87	0.01\\
6.88	0.01\\
6.89	0.01\\
6.9	0.01\\
6.91	0.01\\
6.92	0.01\\
6.93	0.01\\
6.94	0.01\\
6.95	0.01\\
6.96	0.01\\
6.97	0.01\\
6.98	0.01\\
6.99	0.01\\
7	0.01\\
7.01	0.01\\
7.02	0.01\\
7.03	0.01\\
7.04	0.01\\
7.05	0.01\\
7.06	0.01\\
7.07	0.01\\
7.08	0.01\\
7.09	0.01\\
7.1	0.01\\
7.11	0.01\\
7.12	0.01\\
7.13	0.01\\
7.14	0.01\\
7.15	0.01\\
7.16	0.01\\
7.17	0.01\\
7.18	0.01\\
7.19	0.01\\
7.2	0.01\\
7.21	0.01\\
7.22	0.01\\
7.23	0.01\\
7.24	0.01\\
7.25	0.01\\
7.26	0.01\\
7.27	0.01\\
7.28	0.01\\
7.29	0.01\\
7.3	0.01\\
7.31	0.01\\
7.32	0.01\\
7.33	0.01\\
7.34	0.01\\
7.35	0.01\\
7.36	0.01\\
7.37	0.01\\
7.38	0.01\\
7.39	0.01\\
7.4	0.01\\
7.41	0.01\\
7.42	0.01\\
7.43	0.01\\
7.44	0.01\\
7.45	0.01\\
7.46	0.01\\
7.47	0.01\\
7.48	0.01\\
7.49	0.01\\
7.5	0.01\\
7.51	0.01\\
7.52	0.01\\
7.53	0.01\\
7.54	0.01\\
7.55	0.01\\
7.56	0.01\\
7.57	0.01\\
7.58	0.01\\
7.59	0.01\\
7.6	0.01\\
7.61	0.01\\
7.62	0.01\\
7.63	0.01\\
7.64	0.01\\
7.65	0.01\\
7.66	0.01\\
7.67	0.01\\
7.68	0.01\\
7.69	0.01\\
7.7	0.01\\
7.71	0.01\\
7.72	0.01\\
7.73	0.01\\
7.74	0.01\\
7.75	0.01\\
7.76	0.01\\
7.77	0.01\\
7.78	0.01\\
7.79	0.01\\
7.8	0.01\\
7.81	0.01\\
7.82	0.01\\
7.83	0.01\\
7.84	0.01\\
7.85	0.01\\
7.86	0.01\\
7.87	0.01\\
7.88	0.01\\
7.89	0.01\\
7.9	0.01\\
7.91	0.01\\
7.92	0.01\\
7.93	0.01\\
7.94	0.01\\
7.95	0.01\\
7.96	0.01\\
7.97	0.01\\
7.98	0.01\\
7.99	0.01\\
8	0.01\\
8.01	0.01\\
8.02	0.01\\
8.03	0.01\\
8.04	0.01\\
8.05	0.01\\
8.06	0.01\\
8.07	0.01\\
8.08	0.01\\
8.09	0.01\\
8.1	0.01\\
8.11	0.01\\
8.12	0.01\\
8.13	0.01\\
8.14	0.01\\
8.15	0.01\\
8.16	0.01\\
8.17	0.01\\
8.18	0.01\\
8.19	0.01\\
8.2	0.01\\
8.21	0.01\\
8.22	0.01\\
8.23	0.01\\
8.24	0.01\\
8.25	0.01\\
8.26	0.01\\
8.27	0.01\\
8.28	0.01\\
8.29	0.01\\
8.3	0.01\\
8.31	0.01\\
8.32	0.01\\
8.33	0.01\\
8.34	0.01\\
8.35	0.01\\
8.36	0.01\\
8.37	0.01\\
8.38	0.01\\
8.39	0.01\\
8.4	0.01\\
8.41	0.01\\
8.42	0.01\\
8.43	0.01\\
8.44	0.01\\
8.45	0.01\\
8.46	0.01\\
8.47	0.01\\
8.48	0.01\\
8.49	0.01\\
8.5	0.01\\
8.51	0.01\\
8.52	0.01\\
8.53	0.01\\
8.54	0.01\\
8.55	0.01\\
8.56	0.01\\
8.57	0.01\\
8.58	0.01\\
8.59	0.01\\
8.6	0.01\\
8.61	0.01\\
8.62	0.01\\
8.63	0.01\\
8.64	0.01\\
8.65	0.01\\
8.66	0.01\\
8.67	0.01\\
8.68	0.01\\
8.69	0.01\\
8.7	0.01\\
8.71	0.01\\
8.72	0.01\\
8.73	0.01\\
8.74	0.01\\
8.75	0.01\\
8.76	0.01\\
8.77	0.01\\
8.78	0.01\\
8.79	0.01\\
8.8	0.01\\
8.81	0.01\\
8.82	0.01\\
8.83	0.01\\
8.84	0.01\\
8.85	0.01\\
8.86	0.01\\
8.87	0.01\\
8.88	0.01\\
8.89	0.01\\
8.9	0.01\\
8.91	0.01\\
8.92	0.01\\
8.93	0.01\\
8.94	0.01\\
8.95	0.01\\
8.96	0.01\\
8.97	0.01\\
8.98	0.01\\
8.99	0.01\\
9	0.01\\
9.01	0.01\\
9.02	0.01\\
9.03	0.01\\
9.04	0.01\\
9.05	0.01\\
9.06	0.01\\
9.07	0.01\\
9.08	0.01\\
9.09	0.01\\
9.1	0.01\\
9.11	0.01\\
9.12	0.01\\
9.13	0.01\\
9.14	0.01\\
9.15	0.01\\
9.16	0.01\\
9.17	0.01\\
9.18	0.01\\
9.19	0.01\\
9.2	0.01\\
9.21	0.01\\
9.22	0.01\\
9.23	0.01\\
9.24	0.01\\
9.25	0.01\\
9.26	0.01\\
9.27	0.01\\
9.28	0.01\\
9.29	0.01\\
9.3	0.01\\
9.31	0.01\\
9.32	0.01\\
9.33	0.01\\
9.34	0.01\\
9.35	0.01\\
9.36	0.01\\
9.37	0.01\\
9.38	0.01\\
9.39	0.01\\
9.4	0.01\\
9.41	0.01\\
9.42	0.01\\
9.43	0.01\\
9.44	0.01\\
9.45	0.01\\
9.46	0.01\\
9.47	0.01\\
9.48	0.01\\
9.49	0.01\\
9.5	0.01\\
9.51	0.01\\
9.52	0.01\\
9.53	0.01\\
9.54	0.01\\
9.55	0.01\\
9.56	0.01\\
9.57	0.01\\
9.58	0.01\\
9.59	0.01\\
9.6	0.01\\
9.61	0.01\\
9.62	0.01\\
9.63	0.01\\
9.64	0.01\\
9.65	0.01\\
9.66	0.01\\
9.67	0.01\\
9.68	0.01\\
9.69	0.01\\
9.7	0.01\\
9.71	0.01\\
9.72	0.01\\
9.73	0.01\\
9.74	0.01\\
9.75	0.01\\
9.76	0.01\\
9.77	0.01\\
9.78	0.01\\
9.79	0.01\\
9.8	0.01\\
9.81	0.01\\
9.82	0.01\\
9.83	0.01\\
9.84	0.01\\
9.85	0.01\\
9.86	0.01\\
9.87	0.01\\
9.88	0.01\\
9.89	0.01\\
9.9	0.01\\
9.91	0.01\\
9.92	0.01\\
9.93	0.01\\
9.94	0.01\\
9.95	0.01\\
9.96	0.01\\
9.97	0.01\\
9.98	0.01\\
9.99	0.01\\
10	0.01\\
10.01	0.01\\
10.02	0.01\\
10.03	0.01\\
10.04	0.01\\
10.05	0.01\\
10.06	0.01\\
10.07	0.01\\
10.08	0.01\\
10.09	0.01\\
10.1	0.01\\
10.11	0.01\\
10.12	0.01\\
10.13	0.01\\
10.14	0.01\\
10.15	0.01\\
10.16	0.01\\
10.17	0.01\\
10.18	0.01\\
10.19	0.01\\
10.2	0.01\\
10.21	0.01\\
10.22	0.01\\
10.23	0.01\\
10.24	0.01\\
10.25	0.01\\
10.26	0.01\\
10.27	0.01\\
10.28	0.01\\
10.29	0.01\\
10.3	0.01\\
10.31	0.01\\
10.32	0.01\\
10.33	0.01\\
10.34	0.01\\
10.35	0.01\\
10.36	0.01\\
10.37	0.01\\
10.38	0.01\\
10.39	0.01\\
10.4	0.01\\
10.41	0.01\\
10.42	0.01\\
10.43	0.01\\
10.44	0.01\\
10.45	0.01\\
10.46	0.01\\
10.47	0.01\\
10.48	0.01\\
10.49	0.01\\
10.5	0.01\\
10.51	0.01\\
10.52	0.01\\
10.53	0.01\\
10.54	0.01\\
10.55	0.01\\
10.56	0.01\\
10.57	0.01\\
10.58	0.01\\
10.59	0.01\\
10.6	0.01\\
10.61	0.01\\
10.62	0.01\\
10.63	0.01\\
10.64	0.01\\
10.65	0.01\\
10.66	0.01\\
10.67	0.01\\
10.68	0.01\\
10.69	0.01\\
10.7	0.01\\
10.71	0.01\\
10.72	0.01\\
10.73	0.01\\
10.74	0.01\\
10.75	0.01\\
10.76	0.01\\
10.77	0.01\\
10.78	0.01\\
10.79	0.01\\
10.8	0.01\\
10.81	0.01\\
10.82	0.01\\
10.83	0.01\\
10.84	0.01\\
10.85	0.01\\
10.86	0.01\\
10.87	0.01\\
10.88	0.01\\
10.89	0.01\\
10.9	0.01\\
10.91	0.01\\
10.92	0.01\\
10.93	0.01\\
10.94	0.01\\
10.95	0.01\\
10.96	0.01\\
10.97	0.01\\
10.98	0.01\\
10.99	0.01\\
11	0.01\\
11.01	0.01\\
11.02	0.01\\
11.03	0.01\\
11.04	0.01\\
11.05	0.01\\
11.06	0.01\\
11.07	0.01\\
11.08	0.01\\
11.09	0.01\\
11.1	0.01\\
11.11	0.01\\
11.12	0.01\\
11.13	0.01\\
11.14	0.01\\
11.15	0.01\\
11.16	0.01\\
11.17	0.01\\
11.18	0.01\\
11.19	0.01\\
11.2	0.01\\
11.21	0.01\\
11.22	0.01\\
11.23	0.01\\
11.24	0.01\\
11.25	0.01\\
11.26	0.01\\
11.27	0.01\\
11.28	0.01\\
11.29	0.01\\
11.3	0.01\\
11.31	0.01\\
11.32	0.01\\
11.33	0.01\\
11.34	0.01\\
11.35	0.01\\
11.36	0.01\\
11.37	0.01\\
11.38	0.01\\
11.39	0.01\\
11.4	0.01\\
11.41	0.01\\
11.42	0.01\\
11.43	0.01\\
11.44	0.01\\
11.45	0.01\\
11.46	0.01\\
11.47	0.01\\
11.48	0.01\\
11.49	0.01\\
11.5	0.01\\
11.51	0.01\\
11.52	0.01\\
11.53	0.01\\
11.54	0.01\\
11.55	0.01\\
11.56	0.01\\
11.57	0.01\\
11.58	0.01\\
11.59	0.01\\
11.6	0.01\\
11.61	0.01\\
11.62	0.01\\
11.63	0.01\\
11.64	0.01\\
11.65	0.01\\
11.66	0.01\\
11.67	0.01\\
11.68	0.01\\
11.69	0.01\\
11.7	0.01\\
11.71	0.01\\
11.72	0.01\\
11.73	0.01\\
11.74	0.01\\
11.75	0.01\\
11.76	0.01\\
11.77	0.01\\
11.78	0.01\\
11.79	0.01\\
11.8	0.01\\
11.81	0.01\\
11.82	0.01\\
11.83	0.01\\
11.84	0.01\\
11.85	0.01\\
11.86	0.01\\
11.87	0.01\\
11.88	0.01\\
11.89	0.01\\
11.9	0.01\\
11.91	0.01\\
11.92	0.01\\
11.93	0.01\\
11.94	0.01\\
11.95	0.01\\
11.96	0.01\\
11.97	0.01\\
11.98	0.01\\
11.99	0.01\\
12	0.01\\
12.01	0.01\\
12.02	0.01\\
12.03	0.01\\
12.04	0.01\\
12.05	0.01\\
12.06	0.01\\
12.07	0.01\\
12.08	0.01\\
12.09	0.01\\
12.1	0.01\\
12.11	0.01\\
12.12	0.01\\
12.13	0.01\\
12.14	0.01\\
12.15	0.01\\
12.16	0.01\\
12.17	0.01\\
12.18	0.01\\
12.19	0.01\\
12.2	0.01\\
12.21	0.01\\
12.22	0.01\\
12.23	0.01\\
12.24	0.01\\
12.25	0.01\\
12.26	0.01\\
12.27	0.01\\
12.28	0.01\\
12.29	0.01\\
12.3	0.01\\
12.31	0.01\\
12.32	0.01\\
12.33	0.01\\
12.34	0.01\\
12.35	0.01\\
12.36	0.01\\
12.37	0.01\\
12.38	0.01\\
12.39	0.01\\
12.4	0.01\\
12.41	0.01\\
12.42	0.01\\
12.43	0.01\\
12.44	0.01\\
12.45	0.01\\
12.46	0.01\\
12.47	0.01\\
12.48	0.01\\
12.49	0.01\\
12.5	0.01\\
12.51	0.01\\
12.52	0.01\\
12.53	0.01\\
12.54	0.01\\
12.55	0.01\\
12.56	0.01\\
12.57	0.01\\
12.58	0.01\\
12.59	0.01\\
12.6	0.01\\
12.61	0.01\\
12.62	0.01\\
12.63	0.01\\
12.64	0.01\\
12.65	0.01\\
12.66	0.01\\
12.67	0.01\\
12.68	0.01\\
12.69	0.01\\
12.7	0.01\\
12.71	0.01\\
12.72	0.01\\
12.73	0.01\\
12.74	0.01\\
12.75	0.01\\
12.76	0.01\\
12.77	0.01\\
12.78	0.01\\
12.79	0.01\\
12.8	0.01\\
12.81	0.01\\
12.82	0.01\\
12.83	0.01\\
12.84	0.01\\
12.85	0.01\\
12.86	0.01\\
12.87	0.01\\
12.88	0.01\\
12.89	0.01\\
12.9	0.01\\
12.91	0.01\\
12.92	0.01\\
12.93	0.01\\
12.94	0.01\\
12.95	0.01\\
12.96	0.01\\
12.97	0.01\\
12.98	0.01\\
12.99	0.01\\
13	0.01\\
13.01	0.01\\
13.02	0.01\\
13.03	0.01\\
13.04	0.01\\
13.05	0.01\\
13.06	0.01\\
13.07	0.01\\
13.08	0.01\\
13.09	0.01\\
13.1	0.01\\
13.11	0.01\\
13.12	0.01\\
13.13	0.01\\
13.14	0.01\\
13.15	0.01\\
13.16	0.01\\
13.17	0.01\\
13.18	0.01\\
13.19	0.01\\
13.2	0.01\\
13.21	0.01\\
13.22	0.01\\
13.23	0.01\\
13.24	0.01\\
13.25	0.01\\
13.26	0.01\\
13.27	0.01\\
13.28	0.01\\
13.29	0.01\\
13.3	0.01\\
13.31	0.01\\
13.32	0.01\\
13.33	0.01\\
13.34	0.01\\
13.35	0.01\\
13.36	0.01\\
13.37	0.01\\
13.38	0.01\\
13.39	0.01\\
13.4	0.01\\
13.41	0.01\\
13.42	0.01\\
13.43	0.01\\
13.44	0.01\\
13.45	0.01\\
13.46	0.01\\
13.47	0.01\\
13.48	0.01\\
13.49	0.01\\
13.5	0.01\\
13.51	0.01\\
13.52	0.01\\
13.53	0.01\\
13.54	0.01\\
13.55	0.01\\
13.56	0.01\\
13.57	0.01\\
13.58	0.01\\
13.59	0.01\\
13.6	0.01\\
13.61	0.01\\
13.62	0.01\\
13.63	0.01\\
13.64	0.01\\
13.65	0.01\\
13.66	0.01\\
13.67	0.01\\
13.68	0.01\\
13.69	0.01\\
13.7	0.01\\
13.71	0.01\\
13.72	0.01\\
13.73	0.01\\
13.74	0.01\\
13.75	0.01\\
13.76	0.01\\
13.77	0.01\\
13.78	0.01\\
13.79	0.01\\
13.8	0.01\\
13.81	0.01\\
13.82	0.01\\
13.83	0.01\\
13.84	0.01\\
13.85	0.01\\
13.86	0.01\\
13.87	0.01\\
13.88	0.01\\
13.89	0.01\\
13.9	0.01\\
13.91	0.01\\
13.92	0.01\\
13.93	0.01\\
13.94	0.01\\
13.95	0.01\\
13.96	0.01\\
13.97	0.01\\
13.98	0.01\\
13.99	0.01\\
14	0.01\\
14.01	0.01\\
14.02	0.01\\
14.03	0.01\\
14.04	0.01\\
14.05	0.01\\
14.06	0.01\\
14.07	0.01\\
14.08	0.01\\
14.09	0.01\\
14.1	0.01\\
14.11	0.01\\
14.12	0.01\\
14.13	0.01\\
14.14	0.01\\
14.15	0.01\\
14.16	0.01\\
14.17	0.01\\
14.18	0.01\\
14.19	0.01\\
14.2	0.01\\
14.21	0.01\\
14.22	0.01\\
14.23	0.01\\
14.24	0.01\\
14.25	0.01\\
14.26	0.01\\
14.27	0.01\\
14.28	0.01\\
14.29	0.01\\
14.3	0.01\\
14.31	0.01\\
14.32	0.01\\
14.33	0.01\\
14.34	0.01\\
14.35	0.01\\
14.36	0.01\\
14.37	0.01\\
14.38	0.01\\
14.39	0.01\\
14.4	0.01\\
14.41	0.01\\
14.42	0.01\\
14.43	0.01\\
14.44	0.01\\
14.45	0.01\\
14.46	0.01\\
14.47	0.01\\
14.48	0.01\\
14.49	0.01\\
14.5	0.01\\
14.51	0.01\\
14.52	0.01\\
14.53	0.01\\
14.54	0.01\\
14.55	0.01\\
14.56	0.01\\
14.57	0.01\\
14.58	0.01\\
14.59	0.01\\
14.6	0.01\\
14.61	0.01\\
14.62	0.01\\
14.63	0.01\\
14.64	0.01\\
14.65	0.01\\
14.66	0.01\\
14.67	0.01\\
14.68	0.01\\
14.69	0.01\\
14.7	0.01\\
14.71	0.01\\
14.72	0.01\\
14.73	0.01\\
14.74	0.01\\
14.75	0.01\\
14.76	0.01\\
14.77	0.01\\
14.78	0.01\\
14.79	0.01\\
14.8	0.01\\
14.81	0.01\\
14.82	0.01\\
14.83	0.01\\
14.84	0.01\\
14.85	0.01\\
14.86	0.01\\
14.87	0.01\\
14.88	0.01\\
14.89	0.01\\
14.9	0.01\\
14.91	0.01\\
14.92	0.01\\
14.93	0.01\\
14.94	0.01\\
14.95	0.01\\
14.96	0.01\\
14.97	0.01\\
14.98	0.01\\
14.99	0.01\\
15	0.01\\
15.01	0.01\\
15.02	0.01\\
15.03	0.01\\
15.04	0.01\\
15.05	0.01\\
15.06	0.01\\
15.07	0.01\\
15.08	0.01\\
15.09	0.01\\
15.1	0.01\\
15.11	0.01\\
15.12	0.01\\
15.13	0.01\\
15.14	0.01\\
15.15	0.01\\
15.16	0.01\\
15.17	0.01\\
15.18	0.01\\
15.19	0.01\\
15.2	0.01\\
15.21	0.01\\
15.22	0.01\\
15.23	0.01\\
15.24	0.01\\
15.25	0.01\\
15.26	0.01\\
15.27	0.01\\
15.28	0.01\\
15.29	0.01\\
15.3	0.01\\
15.31	0.01\\
15.32	0.01\\
15.33	0.01\\
15.34	0.01\\
15.35	0.01\\
15.36	0.01\\
15.37	0.01\\
15.38	0.01\\
15.39	0.01\\
15.4	0.01\\
15.41	0.01\\
15.42	0.01\\
15.43	0.01\\
15.44	0.01\\
15.45	0.01\\
15.46	0.01\\
15.47	0.01\\
15.48	0.01\\
15.49	0.01\\
15.5	0.01\\
15.51	0.01\\
15.52	0.01\\
15.53	0.01\\
15.54	0.01\\
15.55	0.01\\
15.56	0.01\\
15.57	0.01\\
15.58	0.01\\
15.59	0.01\\
15.6	0.01\\
15.61	0.01\\
15.62	0.01\\
15.63	0.01\\
15.64	0.01\\
15.65	0.01\\
15.66	0.01\\
15.67	0.01\\
15.68	0.01\\
15.69	0.01\\
15.7	0.01\\
15.71	0.01\\
15.72	0.01\\
15.73	0.01\\
15.74	0.01\\
15.75	0.01\\
15.76	0.01\\
15.77	0.01\\
15.78	0.01\\
15.79	0.01\\
15.8	0.01\\
15.81	0.01\\
15.82	0.01\\
15.83	0.01\\
15.84	0.01\\
15.85	0.01\\
15.86	0.01\\
15.87	0.01\\
15.88	0.01\\
15.89	0.01\\
15.9	0.01\\
15.91	0.01\\
15.92	0.01\\
15.93	0.01\\
15.94	0.01\\
15.95	0.01\\
15.96	0.01\\
15.97	0.01\\
15.98	0.01\\
15.99	0.01\\
16	0.01\\
16.01	0.01\\
16.02	0.01\\
16.03	0.01\\
16.04	0.01\\
16.05	0.01\\
16.06	0.01\\
16.07	0.01\\
16.08	0.01\\
16.09	0.01\\
16.1	0.01\\
16.11	0.01\\
16.12	0.01\\
16.13	0.01\\
16.14	0.01\\
16.15	0.01\\
16.16	0.01\\
16.17	0.01\\
16.18	0.01\\
16.19	0.01\\
16.2	0.01\\
16.21	0.01\\
16.22	0.01\\
16.23	0.01\\
16.24	0.01\\
16.25	0.01\\
16.26	0.01\\
16.27	0.01\\
16.28	0.01\\
16.29	0.01\\
16.3	0.01\\
16.31	0.01\\
16.32	0.01\\
16.33	0.01\\
16.34	0.01\\
16.35	0.01\\
16.36	0.01\\
16.37	0.01\\
16.38	0.01\\
16.39	0.01\\
16.4	0.01\\
16.41	0.01\\
16.42	0.01\\
16.43	0.01\\
16.44	0.01\\
16.45	0.01\\
16.46	0.01\\
16.47	0.01\\
16.48	0.01\\
16.49	0.01\\
16.5	0.01\\
16.51	0.01\\
16.52	0.01\\
16.53	0.01\\
16.54	0.01\\
16.55	0.01\\
16.56	0.01\\
16.57	0.01\\
16.58	0.01\\
16.59	0.01\\
16.6	0.01\\
16.61	0.01\\
16.62	0.01\\
16.63	0.01\\
16.64	0.01\\
16.65	0.01\\
16.66	0.01\\
16.67	0.01\\
16.68	0.01\\
16.69	0.01\\
16.7	0.01\\
16.71	0.01\\
16.72	0.01\\
16.73	0.01\\
16.74	0.01\\
16.75	0.01\\
16.76	0.01\\
16.77	0.01\\
16.78	0.01\\
16.79	0.01\\
16.8	0.01\\
16.81	0.01\\
16.82	0.01\\
16.83	0.01\\
16.84	0.01\\
16.85	0.01\\
16.86	0.01\\
16.87	0.01\\
16.88	0.01\\
16.89	0.01\\
16.9	0.01\\
16.91	0.01\\
16.92	0.01\\
16.93	0.01\\
16.94	0.01\\
16.95	0.01\\
16.96	0.01\\
16.97	0.01\\
16.98	0.01\\
16.99	0.01\\
17	0.01\\
17.01	0.01\\
17.02	0.01\\
17.03	0.01\\
17.04	0.01\\
17.05	0.01\\
17.06	0.01\\
17.07	0.01\\
17.08	0.01\\
17.09	0.01\\
17.1	0.01\\
17.11	0.01\\
17.12	0.01\\
17.13	0.01\\
17.14	0.01\\
17.15	0.01\\
17.16	0.01\\
17.17	0.01\\
17.18	0.01\\
17.19	0.01\\
17.2	0.01\\
17.21	0.01\\
17.22	0.01\\
17.23	0.01\\
17.24	0.01\\
17.25	0.01\\
17.26	0.01\\
17.27	0.01\\
17.28	0.01\\
17.29	0.01\\
17.3	0.01\\
17.31	0.01\\
17.32	0.01\\
17.33	0.01\\
17.34	0.01\\
17.35	0.01\\
17.36	0.01\\
17.37	0.01\\
17.38	0.01\\
17.39	0.01\\
17.4	0.01\\
17.41	0.01\\
17.42	0.01\\
17.43	0.01\\
17.44	0.01\\
17.45	0.01\\
17.46	0.01\\
17.47	0.01\\
17.48	0.01\\
17.49	0.01\\
17.5	0.01\\
17.51	0.01\\
17.52	0.01\\
17.53	0.01\\
17.54	0.01\\
17.55	0.01\\
17.56	0.01\\
17.57	0.01\\
17.58	0.01\\
17.59	0.01\\
17.6	0.01\\
17.61	0.01\\
17.62	0.01\\
17.63	0.01\\
17.64	0.01\\
17.65	0.01\\
17.66	0.01\\
17.67	0.01\\
17.68	0.01\\
17.69	0.01\\
17.7	0.01\\
17.71	0.01\\
17.72	0.01\\
17.73	0.01\\
17.74	0.01\\
17.75	0.01\\
17.76	0.01\\
17.77	0.01\\
17.78	0.01\\
17.79	0.01\\
17.8	0.01\\
17.81	0.01\\
17.82	0.01\\
17.83	0.01\\
17.84	0.01\\
17.85	0.01\\
17.86	0.01\\
17.87	0.01\\
17.88	0.01\\
17.89	0.01\\
17.9	0.01\\
17.91	0.01\\
17.92	0.01\\
17.93	0.01\\
17.94	0.01\\
17.95	0.01\\
17.96	0.01\\
17.97	0.01\\
17.98	0.01\\
17.99	0.01\\
18	0.01\\
18.01	0.01\\
18.02	0.01\\
18.03	0.01\\
18.04	0.01\\
18.05	0.01\\
18.06	0.01\\
18.07	0.01\\
18.08	0.01\\
18.09	0.01\\
18.1	0.01\\
18.11	0.01\\
18.12	0.01\\
18.13	0.01\\
18.14	0.01\\
18.15	0.01\\
18.16	0.01\\
18.17	0.01\\
18.18	0.01\\
18.19	0.01\\
18.2	0.01\\
18.21	0.01\\
18.22	0.01\\
18.23	0.01\\
18.24	0.01\\
18.25	0.01\\
18.26	0.01\\
18.27	0.01\\
18.28	0.01\\
18.29	0.01\\
18.3	0.01\\
18.31	0.01\\
18.32	0.01\\
18.33	0.01\\
18.34	0.01\\
18.35	0.01\\
18.36	0.01\\
18.37	0.01\\
18.38	0.01\\
18.39	0.01\\
18.4	0.01\\
18.41	0.01\\
18.42	0.01\\
18.43	0.01\\
18.44	0.01\\
18.45	0.01\\
18.46	0.01\\
18.47	0.01\\
18.48	0.01\\
18.49	0.01\\
18.5	0.01\\
18.51	0.01\\
18.52	0.01\\
18.53	0.01\\
18.54	0.01\\
18.55	0.01\\
18.56	0.01\\
18.57	0.01\\
18.58	0.01\\
18.59	0.01\\
18.6	0.01\\
18.61	0.01\\
18.62	0.01\\
18.63	0.01\\
18.64	0.01\\
18.65	0.01\\
18.66	0.01\\
18.67	0.01\\
18.68	0.01\\
18.69	0.01\\
18.7	0.01\\
18.71	0.01\\
18.72	0.01\\
18.73	0.01\\
18.74	0.01\\
18.75	0.01\\
18.76	0.01\\
18.77	0.01\\
18.78	0.01\\
18.79	0.01\\
18.8	0.01\\
18.81	0.01\\
18.82	0.01\\
18.83	0.01\\
18.84	0.01\\
18.85	0.01\\
18.86	0.01\\
18.87	0.01\\
18.88	0.01\\
18.89	0.01\\
18.9	0.01\\
18.91	0.01\\
18.92	0.01\\
18.93	0.01\\
18.94	0.01\\
18.95	0.01\\
18.96	0.01\\
18.97	0.01\\
18.98	0.01\\
18.99	0.01\\
19	0.01\\
19.01	0.01\\
19.02	0.01\\
19.03	0.01\\
19.04	0.01\\
19.05	0.01\\
19.06	0.01\\
19.07	0.01\\
19.08	0.01\\
19.09	0.01\\
19.1	0.01\\
19.11	0.01\\
19.12	0.01\\
19.13	0.01\\
19.14	0.01\\
19.15	0.01\\
19.16	0.01\\
19.17	0.01\\
19.18	0.01\\
19.19	0.01\\
19.2	0.01\\
19.21	0.01\\
19.22	0.01\\
19.23	0.01\\
19.24	0.01\\
19.25	0.01\\
19.26	0.01\\
19.27	0.01\\
19.28	0.01\\
19.29	0.01\\
19.3	0.01\\
19.31	0.01\\
19.32	0.01\\
19.33	0.01\\
19.34	0.01\\
19.35	0.01\\
19.36	0.01\\
19.37	0.01\\
19.38	0.01\\
19.39	0.01\\
19.4	0.01\\
19.41	0.01\\
19.42	0.01\\
19.43	0.01\\
19.44	0.01\\
19.45	0.01\\
19.46	0.01\\
19.47	0.01\\
19.48	0.01\\
19.49	0.01\\
19.5	0.01\\
19.51	0.01\\
19.52	0.01\\
19.53	0.01\\
19.54	0.01\\
19.55	0.01\\
19.56	0.01\\
19.57	0.01\\
19.58	0.01\\
19.59	0.01\\
19.6	0.01\\
19.61	0.01\\
19.62	0.01\\
19.63	0.01\\
19.64	0.01\\
19.65	0.01\\
19.66	0.01\\
19.67	0.01\\
19.68	0.01\\
19.69	0.01\\
19.7	0.01\\
19.71	0.01\\
19.72	0.01\\
19.73	0.01\\
19.74	0.01\\
19.75	0.01\\
19.76	0.01\\
19.77	0.01\\
19.78	0.01\\
19.79	0.01\\
19.8	0.01\\
19.81	0.01\\
19.82	0.01\\
19.83	0.01\\
19.84	0.01\\
19.85	0.01\\
19.86	0.01\\
19.87	0.01\\
19.88	0.01\\
19.89	0.01\\
19.9	0.01\\
19.91	0.01\\
19.92	0.01\\
19.93	0.01\\
19.94	0.01\\
19.95	0.01\\
19.96	0.01\\
19.97	0.01\\
19.98	0.01\\
19.99	0.01\\
20	0.01\\
20.01	0.01\\
20.02	0.01\\
20.03	0.01\\
20.04	0.01\\
20.05	0.01\\
20.06	0.01\\
20.07	0.01\\
20.08	0.01\\
20.09	0.01\\
20.1	0.01\\
20.11	0.01\\
20.12	0.01\\
20.13	0.01\\
20.14	0.01\\
20.15	0.01\\
20.16	0.01\\
20.17	0.01\\
20.18	0.01\\
20.19	0.01\\
20.2	0.01\\
20.21	0.01\\
20.22	0.01\\
20.23	0.01\\
20.24	0.01\\
20.25	0.01\\
20.26	0.01\\
20.27	0.01\\
20.28	0.01\\
20.29	0.01\\
20.3	0.01\\
20.31	0.01\\
20.32	0.01\\
20.33	0.01\\
20.34	0.01\\
20.35	0.01\\
20.36	0.01\\
20.37	0.01\\
20.38	0.01\\
20.39	0.01\\
20.4	0.01\\
20.41	0.01\\
20.42	0.01\\
20.43	0.01\\
20.44	0.01\\
20.45	0.01\\
20.46	0.01\\
20.47	0.01\\
20.48	0.01\\
20.49	0.01\\
20.5	0.01\\
20.51	0.01\\
20.52	0.01\\
20.53	0.01\\
20.54	0.01\\
20.55	0.01\\
20.56	0.01\\
20.57	0.01\\
20.58	0.01\\
20.59	0.01\\
20.6	0.01\\
20.61	0.01\\
20.62	0.01\\
20.63	0.01\\
20.64	0.01\\
20.65	0.01\\
20.66	0.01\\
20.67	0.01\\
20.68	0.01\\
20.69	0.01\\
20.7	0.01\\
20.71	0.01\\
20.72	0.01\\
20.73	0.01\\
20.74	0.01\\
20.75	0.01\\
20.76	0.01\\
20.77	0.01\\
20.78	0.01\\
20.79	0.01\\
20.8	0.01\\
20.81	0.01\\
20.82	0.01\\
20.83	0.01\\
20.84	0.01\\
20.85	0.01\\
20.86	0.01\\
20.87	0.01\\
20.88	0.01\\
20.89	0.01\\
20.9	0.01\\
20.91	0.01\\
20.92	0.01\\
20.93	0.01\\
20.94	0.01\\
20.95	0.01\\
20.96	0.01\\
20.97	0.01\\
20.98	0.01\\
20.99	0.01\\
21	0.01\\
21.01	0.01\\
21.02	0.01\\
21.03	0.01\\
21.04	0.01\\
21.05	0.01\\
21.06	0.01\\
21.07	0.01\\
21.08	0.01\\
21.09	0.01\\
21.1	0.01\\
21.11	0.01\\
21.12	0.01\\
21.13	0.01\\
21.14	0.01\\
21.15	0.01\\
21.16	0.01\\
21.17	0.01\\
21.18	0.01\\
21.19	0.01\\
21.2	0.01\\
21.21	0.01\\
21.22	0.01\\
21.23	0.01\\
21.24	0.01\\
21.25	0.01\\
21.26	0.01\\
21.27	0.01\\
21.28	0.01\\
21.29	0.01\\
21.3	0.01\\
21.31	0.01\\
21.32	0.01\\
21.33	0.01\\
21.34	0.01\\
21.35	0.01\\
21.36	0.01\\
21.37	0.01\\
21.38	0.01\\
21.39	0.01\\
21.4	0.01\\
21.41	0.01\\
21.42	0.01\\
21.43	0.01\\
21.44	0.01\\
21.45	0.01\\
21.46	0.01\\
21.47	0.01\\
21.48	0.01\\
21.49	0.01\\
21.5	0.01\\
21.51	0.01\\
21.52	0.01\\
21.53	0.01\\
21.54	0.01\\
21.55	0.01\\
21.56	0.01\\
21.57	0.01\\
21.58	0.01\\
21.59	0.01\\
21.6	0.01\\
21.61	0.01\\
21.62	0.01\\
21.63	0.01\\
21.64	0.01\\
21.65	0.01\\
21.66	0.01\\
21.67	0.01\\
21.68	0.01\\
21.69	0.01\\
21.7	0.01\\
21.71	0.01\\
21.72	0.01\\
21.73	0.01\\
21.74	0.01\\
21.75	0.01\\
21.76	0.01\\
21.77	0.01\\
21.78	0.01\\
21.79	0.01\\
21.8	0.01\\
21.81	0.01\\
21.82	0.01\\
21.83	0.01\\
21.84	0.01\\
21.85	0.01\\
21.86	0.01\\
21.87	0.01\\
21.88	0.01\\
21.89	0.01\\
21.9	0.01\\
21.91	0.01\\
21.92	0.01\\
21.93	0.01\\
21.94	0.01\\
21.95	0.01\\
21.96	0.01\\
21.97	0.01\\
21.98	0.01\\
21.99	0.01\\
22	0.01\\
22.01	0.01\\
22.02	0.01\\
22.03	0.01\\
22.04	0.01\\
22.05	0.01\\
22.06	0.01\\
22.07	0.01\\
22.08	0.01\\
22.09	0.01\\
22.1	0.01\\
22.11	0.01\\
22.12	0.01\\
22.13	0.01\\
22.14	0.01\\
22.15	0.01\\
22.16	0.01\\
22.17	0.01\\
22.18	0.01\\
22.19	0.01\\
22.2	0.01\\
22.21	0.01\\
22.22	0.01\\
22.23	0.01\\
22.24	0.01\\
22.25	0.01\\
22.26	0.01\\
22.27	0.01\\
22.28	0.01\\
22.29	0.01\\
22.3	0.01\\
22.31	0.01\\
22.32	0.01\\
22.33	0.01\\
22.34	0.01\\
22.35	0.01\\
22.36	0.01\\
22.37	0.01\\
22.38	0.01\\
22.39	0.01\\
22.4	0.01\\
22.41	0.01\\
22.42	0.01\\
22.43	0.01\\
22.44	0.01\\
22.45	0.01\\
22.46	0.01\\
22.47	0.01\\
22.48	0.01\\
22.49	0.01\\
22.5	0.01\\
22.51	0.01\\
22.52	0.01\\
22.53	0.01\\
22.54	0.01\\
22.55	0.01\\
22.56	0.01\\
22.57	0.01\\
22.58	0.01\\
22.59	0.01\\
22.6	0.01\\
22.61	0.01\\
22.62	0.01\\
22.63	0.01\\
22.64	0.01\\
22.65	0.01\\
22.66	0.01\\
22.67	0.01\\
22.68	0.01\\
22.69	0.01\\
22.7	0.01\\
22.71	0.01\\
22.72	0.01\\
22.73	0.01\\
22.74	0.01\\
22.75	0.01\\
22.76	0.01\\
22.77	0.01\\
22.78	0.01\\
22.79	0.01\\
22.8	0.01\\
22.81	0.01\\
22.82	0.01\\
22.83	0.01\\
22.84	0.01\\
22.85	0.01\\
22.86	0.01\\
22.87	0.01\\
22.88	0.01\\
22.89	0.01\\
22.9	0.01\\
22.91	0.01\\
22.92	0.01\\
22.93	0.01\\
22.94	0.01\\
22.95	0.01\\
22.96	0.01\\
22.97	0.01\\
22.98	0.01\\
22.99	0.01\\
23	0.01\\
23.01	0.01\\
23.02	0.01\\
23.03	0.01\\
23.04	0.01\\
23.05	0.01\\
23.06	0.01\\
23.07	0.01\\
23.08	0.01\\
23.09	0.01\\
23.1	0.01\\
23.11	0.01\\
23.12	0.01\\
23.13	0.01\\
23.14	0.01\\
23.15	0.01\\
23.16	0.01\\
23.17	0.01\\
23.18	0.01\\
23.19	0.01\\
23.2	0.01\\
23.21	0.01\\
23.22	0.01\\
23.23	0.01\\
23.24	0.01\\
23.25	0.01\\
23.26	0.01\\
23.27	0.01\\
23.28	0.01\\
23.29	0.01\\
23.3	0.01\\
23.31	0.01\\
23.32	0.01\\
23.33	0.01\\
23.34	0.01\\
23.35	0.01\\
23.36	0.01\\
23.37	0.01\\
23.38	0.01\\
23.39	0.01\\
23.4	0.01\\
23.41	0.01\\
23.42	0.01\\
23.43	0.01\\
23.44	0.01\\
23.45	0.01\\
23.46	0.01\\
23.47	0.01\\
23.48	0.01\\
23.49	0.01\\
23.5	0.01\\
23.51	0.01\\
23.52	0.01\\
23.53	0.01\\
23.54	0.01\\
23.55	0.01\\
23.56	0.01\\
23.57	0.01\\
23.58	0.01\\
23.59	0.01\\
23.6	0.01\\
23.61	0.01\\
23.62	0.01\\
23.63	0.01\\
23.64	0.01\\
23.65	0.01\\
23.66	0.01\\
23.67	0.01\\
23.68	0.01\\
23.69	0.01\\
23.7	0.01\\
23.71	0.01\\
23.72	0.01\\
23.73	0.01\\
23.74	0.01\\
23.75	0.01\\
23.76	0.01\\
23.77	0.01\\
23.78	0.01\\
23.79	0.01\\
23.8	0.01\\
23.81	0.01\\
23.82	0.01\\
23.83	0.01\\
23.84	0.01\\
23.85	0.01\\
23.86	0.01\\
23.87	0.01\\
23.88	0.01\\
23.89	0.01\\
23.9	0.01\\
23.91	0.01\\
23.92	0.01\\
23.93	0.01\\
23.94	0.01\\
23.95	0.01\\
23.96	0.01\\
23.97	0.01\\
23.98	0.01\\
23.99	0.01\\
24	0.01\\
24.01	0.01\\
24.02	0.01\\
24.03	0.01\\
24.04	0.01\\
24.05	0.01\\
24.06	0.01\\
24.07	0.01\\
24.08	0.01\\
24.09	0.01\\
24.1	0.01\\
24.11	0.01\\
24.12	0.01\\
24.13	0.01\\
24.14	0.01\\
24.15	0.01\\
24.16	0.01\\
24.17	0.01\\
24.18	0.01\\
24.19	0.01\\
24.2	0.01\\
24.21	0.01\\
24.22	0.01\\
24.23	0.01\\
24.24	0.01\\
24.25	0.01\\
24.26	0.01\\
24.27	0.01\\
24.28	0.01\\
24.29	0.01\\
24.3	0.01\\
24.31	0.01\\
24.32	0.01\\
24.33	0.01\\
24.34	0.01\\
24.35	0.01\\
24.36	0.01\\
24.37	0.01\\
24.38	0.01\\
24.39	0.01\\
24.4	0.01\\
24.41	0.01\\
24.42	0.01\\
24.43	0.01\\
24.44	0.01\\
24.45	0.01\\
24.46	0.01\\
24.47	0.01\\
24.48	0.01\\
24.49	0.01\\
24.5	0.01\\
24.51	0.01\\
24.52	0.01\\
24.53	0.01\\
24.54	0.01\\
24.55	0.01\\
24.56	0.01\\
24.57	0.01\\
24.58	0.01\\
24.59	0.01\\
24.6	0.01\\
24.61	0.01\\
24.62	0.01\\
24.63	0.01\\
24.64	0.01\\
24.65	0.01\\
24.66	0.01\\
24.67	0.01\\
24.68	0.01\\
24.69	0.01\\
24.7	0.01\\
24.71	0.01\\
24.72	0.01\\
24.73	0.01\\
24.74	0.01\\
24.75	0.01\\
24.76	0.01\\
24.77	0.01\\
24.78	0.01\\
24.79	0.01\\
24.8	0.01\\
24.81	0.01\\
24.82	0.01\\
24.83	0.01\\
24.84	0.01\\
24.85	0.01\\
24.86	0.01\\
24.87	0.01\\
24.88	0.01\\
24.89	0.01\\
24.9	0.01\\
24.91	0.01\\
24.92	0.01\\
24.93	0.01\\
24.94	0.01\\
24.95	0.01\\
24.96	0.01\\
24.97	0.01\\
24.98	0.01\\
24.99	0.01\\
25	0.01\\
25.01	0.01\\
25.02	0.01\\
25.03	0.01\\
25.04	0.01\\
25.05	0.01\\
25.06	0.01\\
25.07	0.01\\
25.08	0.01\\
25.09	0.01\\
25.1	0.01\\
25.11	0.01\\
25.12	0.01\\
25.13	0.01\\
25.14	0.01\\
25.15	0.01\\
25.16	0.01\\
25.17	0.01\\
25.18	0.01\\
25.19	0.01\\
25.2	0.01\\
25.21	0.01\\
25.22	0.01\\
25.23	0.01\\
25.24	0.01\\
25.25	0.01\\
25.26	0.01\\
25.27	0.01\\
25.28	0.01\\
25.29	0.01\\
25.3	0.01\\
25.31	0.01\\
25.32	0.01\\
25.33	0.01\\
25.34	0.01\\
25.35	0.01\\
25.36	0.01\\
25.37	0.01\\
25.38	0.01\\
25.39	0.01\\
25.4	0.01\\
25.41	0.01\\
25.42	0.01\\
25.43	0.01\\
25.44	0.01\\
25.45	0.01\\
25.46	0.01\\
25.47	0.01\\
25.48	0.01\\
25.49	0.01\\
25.5	0.01\\
25.51	0.01\\
25.52	0.01\\
25.53	0.01\\
25.54	0.01\\
25.55	0.01\\
25.56	0.01\\
25.57	0.01\\
25.58	0.01\\
25.59	0.01\\
25.6	0.01\\
25.61	0.01\\
25.62	0.01\\
25.63	0.01\\
25.64	0.01\\
25.65	0.01\\
25.66	0.01\\
25.67	0.01\\
25.68	0.01\\
25.69	0.01\\
25.7	0.01\\
25.71	0.01\\
25.72	0.01\\
25.73	0.01\\
25.74	0.01\\
25.75	0.01\\
25.76	0.01\\
25.77	0.01\\
25.78	0.01\\
25.79	0.01\\
25.8	0.01\\
25.81	0.01\\
25.82	0.01\\
25.83	0.01\\
25.84	0.01\\
25.85	0.01\\
25.86	0.01\\
25.87	0.01\\
25.88	0.01\\
25.89	0.01\\
25.9	0.01\\
25.91	0.01\\
25.92	0.01\\
25.93	0.01\\
25.94	0.01\\
25.95	0.01\\
25.96	0.01\\
25.97	0.01\\
25.98	0.01\\
25.99	0.01\\
26	0.01\\
26.01	0.01\\
26.02	0.01\\
26.03	0.01\\
26.04	0.01\\
26.05	0.01\\
26.06	0.01\\
26.07	0.01\\
26.08	0.01\\
26.09	0.01\\
26.1	0.01\\
26.11	0.01\\
26.12	0.01\\
26.13	0.01\\
26.14	0.01\\
26.15	0.01\\
26.16	0.01\\
26.17	0.01\\
26.18	0.01\\
26.19	0.01\\
26.2	0.01\\
26.21	0.01\\
26.22	0.01\\
26.23	0.01\\
26.24	0.01\\
26.25	0.01\\
26.26	0.01\\
26.27	0.01\\
26.28	0.01\\
26.29	0.01\\
26.3	0.01\\
26.31	0.01\\
26.32	0.01\\
26.33	0.01\\
26.34	0.01\\
26.35	0.01\\
26.36	0.01\\
26.37	0.01\\
26.38	0.01\\
26.39	0.01\\
26.4	0.01\\
26.41	0.01\\
26.42	0.01\\
26.43	0.01\\
26.44	0.01\\
26.45	0.01\\
26.46	0.01\\
26.47	0.01\\
26.48	0.01\\
26.49	0.01\\
26.5	0.01\\
26.51	0.01\\
26.52	0.01\\
26.53	0.01\\
26.54	0.01\\
26.55	0.01\\
26.56	0.01\\
26.57	0.01\\
26.58	0.01\\
26.59	0.01\\
26.6	0.01\\
26.61	0.01\\
26.62	0.01\\
26.63	0.01\\
26.64	0.01\\
26.65	0.01\\
26.66	0.01\\
26.67	0.01\\
26.68	0.01\\
26.69	0.01\\
26.7	0.01\\
26.71	0.01\\
26.72	0.01\\
26.73	0.01\\
26.74	0.01\\
26.75	0.01\\
26.76	0.01\\
26.77	0.01\\
26.78	0.01\\
26.79	0.01\\
26.8	0.01\\
26.81	0.01\\
26.82	0.01\\
26.83	0.01\\
26.84	0.01\\
26.85	0.01\\
26.86	0.01\\
26.87	0.01\\
26.88	0.01\\
26.89	0.01\\
26.9	0.01\\
26.91	0.01\\
26.92	0.01\\
26.93	0.01\\
26.94	0.01\\
26.95	0.01\\
26.96	0.01\\
26.97	0.01\\
26.98	0.01\\
26.99	0.01\\
27	0.01\\
27.01	0.01\\
27.02	0.01\\
27.03	0.01\\
27.04	0.01\\
27.05	0.01\\
27.06	0.01\\
27.07	0.01\\
27.08	0.01\\
27.09	0.01\\
27.1	0.01\\
27.11	0.01\\
27.12	0.01\\
27.13	0.01\\
27.14	0.01\\
27.15	0.01\\
27.16	0.01\\
27.17	0.01\\
27.18	0.01\\
27.19	0.01\\
27.2	0.01\\
27.21	0.01\\
27.22	0.01\\
27.23	0.01\\
27.24	0.01\\
27.25	0.01\\
27.26	0.01\\
27.27	0.01\\
27.28	0.01\\
27.29	0.01\\
27.3	0.01\\
27.31	0.01\\
27.32	0.01\\
27.33	0.01\\
27.34	0.01\\
27.35	0.01\\
27.36	0.01\\
27.37	0.01\\
27.38	0.01\\
27.39	0.01\\
27.4	0.01\\
27.41	0.01\\
27.42	0.01\\
27.43	0.01\\
27.44	0.01\\
27.45	0.01\\
27.46	0.01\\
27.47	0.01\\
27.48	0.01\\
27.49	0.01\\
27.5	0.01\\
27.51	0.01\\
27.52	0.01\\
27.53	0.01\\
27.54	0.01\\
27.55	0.01\\
27.56	0.01\\
27.57	0.01\\
27.58	0.01\\
27.59	0.01\\
27.6	0.01\\
27.61	0.01\\
27.62	0.01\\
27.63	0.01\\
27.64	0.01\\
27.65	0.01\\
27.66	0.01\\
27.67	0.01\\
27.68	0.01\\
27.69	0.01\\
27.7	0.01\\
27.71	0.01\\
27.72	0.01\\
27.73	0.01\\
27.74	0.01\\
27.75	0.01\\
27.76	0.01\\
27.77	0.01\\
27.78	0.01\\
27.79	0.01\\
27.8	0.01\\
27.81	0.01\\
27.82	0.01\\
27.83	0.01\\
27.84	0.01\\
27.85	0.01\\
27.86	0.01\\
27.87	0.01\\
27.88	0.01\\
27.89	0.01\\
27.9	0.01\\
27.91	0.01\\
27.92	0.01\\
27.93	0.01\\
27.94	0.01\\
27.95	0.01\\
27.96	0.01\\
27.97	0.01\\
27.98	0.01\\
27.99	0.01\\
28	0.01\\
28.01	0.01\\
28.02	0.01\\
28.03	0.01\\
28.04	0.01\\
28.05	0.01\\
28.06	0.01\\
28.07	0.01\\
28.08	0.01\\
28.09	0.01\\
28.1	0.01\\
28.11	0.01\\
28.12	0.01\\
28.13	0.01\\
28.14	0.01\\
28.15	0.01\\
28.16	0.01\\
28.17	0.01\\
28.18	0.01\\
28.19	0.01\\
28.2	0.01\\
28.21	0.01\\
28.22	0.01\\
28.23	0.01\\
28.24	0.01\\
28.25	0.01\\
28.26	0.01\\
28.27	0.01\\
28.28	0.01\\
28.29	0.01\\
28.3	0.01\\
28.31	0.01\\
28.32	0.01\\
28.33	0.01\\
28.34	0.01\\
28.35	0.01\\
28.36	0.01\\
28.37	0.01\\
28.38	0.01\\
28.39	0.01\\
28.4	0.01\\
28.41	0.01\\
28.42	0.01\\
28.43	0.01\\
28.44	0.01\\
28.45	0.01\\
28.46	0.01\\
28.47	0.01\\
28.48	0.01\\
28.49	0.01\\
28.5	0.01\\
28.51	0.01\\
28.52	0.01\\
28.53	0.01\\
28.54	0.01\\
28.55	0.01\\
28.56	0.01\\
28.57	0.01\\
28.58	0.01\\
28.59	0.01\\
28.6	0.01\\
28.61	0.01\\
28.62	0.01\\
28.63	0.01\\
28.64	0.01\\
28.65	0.01\\
28.66	0.01\\
28.67	0.01\\
28.68	0.01\\
28.69	0.01\\
28.7	0.01\\
28.71	0.01\\
28.72	0.01\\
28.73	0.01\\
28.74	0.01\\
28.75	0.01\\
28.76	0.01\\
28.77	0.01\\
28.78	0.01\\
28.79	0.01\\
28.8	0.01\\
28.81	0.01\\
28.82	0.01\\
28.83	0.01\\
28.84	0.01\\
28.85	0.01\\
28.86	0.01\\
28.87	0.01\\
28.88	0.01\\
28.89	0.01\\
28.9	0.01\\
28.91	0.01\\
28.92	0.01\\
28.93	0.01\\
28.94	0.01\\
28.95	0.01\\
28.96	0.01\\
28.97	0.01\\
28.98	0.01\\
28.99	0.01\\
29	0.01\\
29.01	0.01\\
29.02	0.01\\
29.03	0.01\\
29.04	0.01\\
29.05	0.01\\
29.06	0.01\\
29.07	0.01\\
29.08	0.01\\
29.09	0.01\\
29.1	0.01\\
29.11	0.01\\
29.12	0.01\\
29.13	0.01\\
29.14	0.01\\
29.15	0.01\\
29.16	0.01\\
29.17	0.01\\
29.18	0.01\\
29.19	0.01\\
29.2	0.01\\
29.21	0.01\\
29.22	0.01\\
29.23	0.01\\
29.24	0.01\\
29.25	0.01\\
29.26	0.01\\
29.27	0.01\\
29.28	0.01\\
29.29	0.01\\
29.3	0.01\\
29.31	0.01\\
29.32	0.01\\
29.33	0.01\\
29.34	0.01\\
29.35	0.01\\
29.36	0.01\\
29.37	0.01\\
29.38	0.01\\
29.39	0.01\\
29.4	0.01\\
29.41	0.01\\
29.42	0.01\\
29.43	0.01\\
29.44	0.01\\
29.45	0.01\\
29.46	0.01\\
29.47	0.01\\
29.48	0.01\\
29.49	0.01\\
29.5	0.01\\
29.51	0.01\\
29.52	0.01\\
29.53	0.01\\
29.54	0.01\\
29.55	0.01\\
29.56	0.01\\
29.57	0.01\\
29.58	0.01\\
29.59	0.01\\
29.6	0.01\\
29.61	0.01\\
29.62	0.01\\
29.63	0.01\\
29.64	0.01\\
29.65	0.01\\
29.66	0.01\\
29.67	0.01\\
29.68	0.01\\
29.69	0.01\\
29.7	0.01\\
29.71	0.01\\
29.72	0.01\\
29.73	0.01\\
29.74	0.01\\
29.75	0.01\\
29.76	0.01\\
29.77	0.01\\
29.78	0.01\\
29.79	0.01\\
29.8	0.01\\
29.81	0.01\\
29.82	0.01\\
29.83	0.01\\
29.84	0.01\\
29.85	0.01\\
29.86	0.01\\
29.87	0.01\\
29.88	0.01\\
29.89	0.01\\
29.9	0.01\\
29.91	0.01\\
29.92	0.01\\
29.93	0.01\\
29.94	0.01\\
29.95	0.01\\
29.96	0.01\\
29.97	0.01\\
29.98	0.01\\
29.99	0.01\\
30	0.01\\
30.01	0.01\\
30.02	0.01\\
30.03	0.01\\
30.04	0.01\\
30.05	0.01\\
30.06	0.01\\
30.07	0.01\\
30.08	0.01\\
30.09	0.01\\
30.1	0.01\\
30.11	0.01\\
30.12	0.01\\
30.13	0.01\\
30.14	0.01\\
30.15	0.01\\
30.16	0.01\\
30.17	0.01\\
30.18	0.01\\
30.19	0.01\\
30.2	0.01\\
30.21	0.01\\
30.22	0.01\\
30.23	0.01\\
30.24	0.01\\
30.25	0.01\\
30.26	0.01\\
30.27	0.01\\
30.28	0.01\\
30.29	0.01\\
30.3	0.01\\
30.31	0.01\\
30.32	0.01\\
30.33	0.01\\
30.34	0.01\\
30.35	0.01\\
30.36	0.01\\
30.37	0.01\\
30.38	0.01\\
30.39	0.01\\
30.4	0.01\\
30.41	0.01\\
30.42	0.01\\
30.43	0.01\\
30.44	0.01\\
30.45	0.01\\
30.46	0.01\\
30.47	0.01\\
30.48	0.01\\
30.49	0.01\\
30.5	0.01\\
30.51	0.01\\
30.52	0.01\\
30.53	0.01\\
30.54	0.01\\
30.55	0.01\\
30.56	0.01\\
30.57	0.01\\
30.58	0.01\\
30.59	0.01\\
30.6	0.01\\
30.61	0.01\\
30.62	0.01\\
30.63	0.01\\
30.64	0.01\\
30.65	0.01\\
30.66	0.01\\
30.67	0.01\\
30.68	0.01\\
30.69	0.01\\
30.7	0.01\\
30.71	0.01\\
30.72	0.01\\
30.73	0.01\\
30.74	0.01\\
30.75	0.01\\
30.76	0.01\\
30.77	0.01\\
30.78	0.01\\
30.79	0.01\\
30.8	0.01\\
30.81	0.01\\
30.82	0.01\\
30.83	0.01\\
30.84	0.01\\
30.85	0.01\\
30.86	0.01\\
30.87	0.01\\
30.88	0.01\\
30.89	0.01\\
30.9	0.01\\
30.91	0.01\\
30.92	0.01\\
30.93	0.01\\
30.94	0.01\\
30.95	0.01\\
30.96	0.01\\
30.97	0.01\\
30.98	0.01\\
30.99	0.01\\
31	0.01\\
31.01	0.01\\
31.02	0.01\\
31.03	0.01\\
31.04	0.01\\
31.05	0.01\\
31.06	0.01\\
31.07	0.01\\
31.08	0.01\\
31.09	0.01\\
31.1	0.01\\
31.11	0.01\\
31.12	0.01\\
31.13	0.01\\
31.14	0.01\\
31.15	0.01\\
31.16	0.01\\
31.17	0.01\\
31.18	0.01\\
31.19	0.01\\
31.2	0.01\\
31.21	0.01\\
31.22	0.01\\
31.23	0.01\\
31.24	0.01\\
31.25	0.01\\
31.26	0.01\\
31.27	0.01\\
31.28	0.01\\
31.29	0.01\\
31.3	0.01\\
31.31	0.01\\
31.32	0.01\\
31.33	0.01\\
31.34	0.01\\
31.35	0.01\\
31.36	0.01\\
31.37	0.01\\
31.38	0.01\\
31.39	0.01\\
31.4	0.01\\
31.41	0.01\\
31.42	0.01\\
31.43	0.01\\
31.44	0.01\\
31.45	0.01\\
31.46	0.01\\
31.47	0.01\\
31.48	0.01\\
31.49	0.01\\
31.5	0.01\\
31.51	0.01\\
31.52	0.01\\
31.53	0.01\\
31.54	0.01\\
31.55	0.01\\
31.56	0.01\\
31.57	0.01\\
31.58	0.01\\
31.59	0.01\\
31.6	0.01\\
31.61	0.01\\
31.62	0.01\\
31.63	0.01\\
31.64	0.01\\
31.65	0.01\\
31.66	0.01\\
31.67	0.01\\
31.68	0.01\\
31.69	0.01\\
31.7	0.01\\
31.71	0.01\\
31.72	0.01\\
31.73	0.01\\
31.74	0.01\\
31.75	0.01\\
31.76	0.01\\
31.77	0.01\\
31.78	0.01\\
31.79	0.01\\
31.8	0.01\\
31.81	0.01\\
31.82	0.01\\
31.83	0.01\\
31.84	0.01\\
31.85	0.01\\
31.86	0.01\\
31.87	0.01\\
31.88	0.01\\
31.89	0.01\\
31.9	0.01\\
31.91	0.01\\
31.92	0.01\\
31.93	0.01\\
31.94	0.01\\
31.95	0.01\\
31.96	0.01\\
31.97	0.01\\
31.98	0.01\\
31.99	0.01\\
32	0.01\\
32.01	0.01\\
32.02	0.01\\
32.03	0.01\\
32.04	0.01\\
32.05	0.01\\
32.06	0.01\\
32.07	0.01\\
32.08	0.01\\
32.09	0.01\\
32.1	0.01\\
32.11	0.01\\
32.12	0.01\\
32.13	0.01\\
32.14	0.01\\
32.15	0.01\\
32.16	0.01\\
32.17	0.01\\
32.18	0.01\\
32.19	0.01\\
32.2	0.01\\
32.21	0.01\\
32.22	0.01\\
32.23	0.01\\
32.24	0.01\\
32.25	0.01\\
32.26	0.01\\
32.27	0.01\\
32.28	0.01\\
32.29	0.01\\
32.3	0.01\\
32.31	0.01\\
32.32	0.01\\
32.33	0.01\\
32.34	0.01\\
32.35	0.01\\
32.36	0.01\\
32.37	0.01\\
32.38	0.01\\
32.39	0.01\\
32.4	0.01\\
32.41	0.01\\
32.42	0.01\\
32.43	0.01\\
32.44	0.01\\
32.45	0.01\\
32.46	0.01\\
32.47	0.01\\
32.48	0.01\\
32.49	0.01\\
32.5	0.01\\
32.51	0.01\\
32.52	0.01\\
32.53	0.01\\
32.54	0.01\\
32.55	0.01\\
32.56	0.01\\
32.57	0.01\\
32.58	0.01\\
32.59	0.01\\
32.6	0.01\\
32.61	0.01\\
32.62	0.01\\
32.63	0.01\\
32.64	0.01\\
32.65	0.01\\
32.66	0.01\\
32.67	0.01\\
32.68	0.01\\
32.69	0.01\\
32.7	0.01\\
32.71	0.01\\
32.72	0.01\\
32.73	0.01\\
32.74	0.01\\
32.75	0.01\\
32.76	0.01\\
32.77	0.01\\
32.78	0.01\\
32.79	0.01\\
32.8	0.01\\
32.81	0.01\\
32.82	0.01\\
32.83	0.01\\
32.84	0.01\\
32.85	0.01\\
32.86	0.01\\
32.87	0.01\\
32.88	0.01\\
32.89	0.01\\
32.9	0.01\\
32.91	0.01\\
32.92	0.01\\
32.93	0.01\\
32.94	0.01\\
32.95	0.01\\
32.96	0.01\\
32.97	0.01\\
32.98	0.01\\
32.99	0.01\\
33	0.01\\
33.01	0.01\\
33.02	0.01\\
33.03	0.01\\
33.04	0.01\\
33.05	0.01\\
33.06	0.01\\
33.07	0.01\\
33.08	0.01\\
33.09	0.01\\
33.1	0.01\\
33.11	0.01\\
33.12	0.01\\
33.13	0.01\\
33.14	0.01\\
33.15	0.01\\
33.16	0.01\\
33.17	0.01\\
33.18	0.01\\
33.19	0.01\\
33.2	0.01\\
33.21	0.01\\
33.22	0.01\\
33.23	0.01\\
33.24	0.01\\
33.25	0.01\\
33.26	0.01\\
33.27	0.01\\
33.28	0.01\\
33.29	0.01\\
33.3	0.01\\
33.31	0.01\\
33.32	0.01\\
33.33	0.01\\
33.34	0.01\\
33.35	0.01\\
33.36	0.01\\
33.37	0.01\\
33.38	0.01\\
33.39	0.01\\
33.4	0.01\\
33.41	0.01\\
33.42	0.01\\
33.43	0.01\\
33.44	0.01\\
33.45	0.01\\
33.46	0.01\\
33.47	0.01\\
33.48	0.01\\
33.49	0.01\\
33.5	0.01\\
33.51	0.01\\
33.52	0.01\\
33.53	0.01\\
33.54	0.01\\
33.55	0.01\\
33.56	0.01\\
33.57	0.01\\
33.58	0.01\\
33.59	0.01\\
33.6	0.01\\
33.61	0.01\\
33.62	0.01\\
33.63	0.01\\
33.64	0.01\\
33.65	0.01\\
33.66	0.01\\
33.67	0.01\\
33.68	0.01\\
33.69	0.01\\
33.7	0.01\\
33.71	0.01\\
33.72	0.01\\
33.73	0.01\\
33.74	0.01\\
33.75	0.01\\
33.76	0.01\\
33.77	0.01\\
33.78	0.01\\
33.79	0.01\\
33.8	0.01\\
33.81	0.01\\
33.82	0.01\\
33.83	0.01\\
33.84	0.01\\
33.85	0.01\\
33.86	0.01\\
33.87	0.01\\
33.88	0.01\\
33.89	0.01\\
33.9	0.01\\
33.91	0.01\\
33.92	0.01\\
33.93	0.01\\
33.94	0.01\\
33.95	0.01\\
33.96	0.01\\
33.97	0.01\\
33.98	0.01\\
33.99	0.01\\
34	0.01\\
34.01	0.01\\
34.02	0.01\\
34.03	0.01\\
34.04	0.01\\
34.05	0.01\\
34.06	0.01\\
34.07	0.01\\
34.08	0.01\\
34.09	0.01\\
34.1	0.01\\
34.11	0.01\\
34.12	0.01\\
34.13	0.01\\
34.14	0.01\\
34.15	0.01\\
34.16	0.01\\
34.17	0.01\\
34.18	0.01\\
34.19	0.01\\
34.2	0.01\\
34.21	0.01\\
34.22	0.01\\
34.23	0.01\\
34.24	0.01\\
34.25	0.01\\
34.26	0.01\\
34.27	0.01\\
34.28	0.01\\
34.29	0.01\\
34.3	0.01\\
34.31	0.01\\
34.32	0.01\\
34.33	0.01\\
34.34	0.01\\
34.35	0.01\\
34.36	0.01\\
34.37	0.01\\
34.38	0.01\\
34.39	0.01\\
34.4	0.01\\
34.41	0.01\\
34.42	0.01\\
34.43	0.01\\
34.44	0.01\\
34.45	0.01\\
34.46	0.01\\
34.47	0.01\\
34.48	0.01\\
34.49	0.01\\
34.5	0.01\\
34.51	0.01\\
34.52	0.01\\
34.53	0.01\\
34.54	0.01\\
34.55	0.01\\
34.56	0.01\\
34.57	0.01\\
34.58	0.01\\
34.59	0.01\\
34.6	0.01\\
34.61	0.01\\
34.62	0.01\\
34.63	0.01\\
34.64	0.01\\
34.65	0.01\\
34.66	0.01\\
34.67	0.01\\
34.68	0.01\\
34.69	0.01\\
34.7	0.01\\
34.71	0.01\\
34.72	0.01\\
34.73	0.01\\
34.74	0.01\\
34.75	0.01\\
34.76	0.01\\
34.77	0.01\\
34.78	0.01\\
34.79	0.01\\
34.8	0.01\\
34.81	0.01\\
34.82	0.01\\
34.83	0.01\\
34.84	0.01\\
34.85	0.01\\
34.86	0.01\\
34.87	0.01\\
34.88	0.01\\
34.89	0.01\\
34.9	0.01\\
34.91	0.01\\
34.92	0.01\\
34.93	0.01\\
34.94	0.01\\
34.95	0.01\\
34.96	0.01\\
34.97	0.01\\
34.98	0.01\\
34.99	0.01\\
35	0.01\\
35.01	0.01\\
35.02	0.01\\
35.03	0.01\\
35.04	0.01\\
35.05	0.01\\
35.06	0.01\\
35.07	0.01\\
35.08	0.01\\
35.09	0.01\\
35.1	0.01\\
35.11	0.01\\
35.12	0.01\\
35.13	0.01\\
35.14	0.01\\
35.15	0.01\\
35.16	0.01\\
35.17	0.01\\
35.18	0.01\\
35.19	0.01\\
35.2	0.01\\
35.21	0.01\\
35.22	0.01\\
35.23	0.01\\
35.24	0.01\\
35.25	0.01\\
35.26	0.01\\
35.27	0.01\\
35.28	0.01\\
35.29	0.01\\
35.3	0.01\\
35.31	0.01\\
35.32	0.01\\
35.33	0.01\\
35.34	0.01\\
35.35	0.01\\
35.36	0.01\\
35.37	0.01\\
35.38	0.01\\
35.39	0.01\\
35.4	0.01\\
35.41	0.01\\
35.42	0.01\\
35.43	0.01\\
35.44	0.01\\
35.45	0.01\\
35.46	0.01\\
35.47	0.01\\
35.48	0.01\\
35.49	0.01\\
35.5	0.01\\
35.51	0.01\\
35.52	0.01\\
35.53	0.01\\
35.54	0.01\\
35.55	0.01\\
35.56	0.01\\
35.57	0.01\\
35.58	0.01\\
35.59	0.01\\
35.6	0.01\\
35.61	0.01\\
35.62	0.01\\
35.63	0.01\\
35.64	0.01\\
35.65	0.01\\
35.66	0.01\\
35.67	0.01\\
35.68	0.01\\
35.69	0.01\\
35.7	0.01\\
35.71	0.01\\
35.72	0.01\\
35.73	0.01\\
35.74	0.01\\
35.75	0.01\\
35.76	0.01\\
35.77	0.01\\
35.78	0.01\\
35.79	0.01\\
35.8	0.01\\
35.81	0.01\\
35.82	0.01\\
35.83	0.01\\
35.84	0.01\\
35.85	0.01\\
35.86	0.01\\
35.87	0.01\\
35.88	0.01\\
35.89	0.01\\
35.9	0.01\\
35.91	0.01\\
35.92	0.01\\
35.93	0.01\\
35.94	0.01\\
35.95	0.01\\
35.96	0.01\\
35.97	0.01\\
35.98	0.01\\
35.99	0.01\\
36	0.01\\
36.01	0.01\\
36.02	0.01\\
36.03	0.01\\
36.04	0.01\\
36.05	0.01\\
36.06	0.01\\
36.07	0.01\\
36.08	0.01\\
36.09	0.01\\
36.1	0.01\\
36.11	0.01\\
36.12	0.01\\
36.13	0.01\\
36.14	0.01\\
36.15	0.01\\
36.16	0.01\\
36.17	0.01\\
36.18	0.01\\
36.19	0.01\\
36.2	0.01\\
36.21	0.01\\
36.22	0.01\\
36.23	0.01\\
36.24	0.01\\
36.25	0.01\\
36.26	0.01\\
36.27	0.01\\
36.28	0.01\\
36.29	0.01\\
36.3	0.01\\
36.31	0.01\\
36.32	0.01\\
36.33	0.01\\
36.34	0.01\\
36.35	0.01\\
36.36	0.01\\
36.37	0.01\\
36.38	0.01\\
36.39	0.01\\
36.4	0.01\\
36.41	0.01\\
36.42	0.01\\
36.43	0.01\\
36.44	0.01\\
36.45	0.01\\
36.46	0.01\\
36.47	0.01\\
36.48	0.01\\
36.49	0.01\\
36.5	0.01\\
36.51	0.01\\
36.52	0.01\\
36.53	0.01\\
36.54	0.01\\
36.55	0.01\\
36.56	0.01\\
36.57	0.01\\
36.58	0.01\\
36.59	0.01\\
36.6	0.01\\
36.61	0.01\\
36.62	0.01\\
36.63	0.01\\
36.64	0.01\\
36.65	0.01\\
36.66	0.01\\
36.67	0.01\\
36.68	0.01\\
36.69	0.01\\
36.7	0.01\\
36.71	0.01\\
36.72	0.01\\
36.73	0.01\\
36.74	0.01\\
36.75	0.01\\
36.76	0.01\\
36.77	0.01\\
36.78	0.01\\
36.79	0.01\\
36.8	0.01\\
36.81	0.01\\
36.82	0.01\\
36.83	0.01\\
36.84	0.01\\
36.85	0.01\\
36.86	0.01\\
36.87	0.01\\
36.88	0.01\\
36.89	0.01\\
36.9	0.01\\
36.91	0.01\\
36.92	0.01\\
36.93	0.01\\
36.94	0.01\\
36.95	0.01\\
36.96	0.01\\
36.97	0.01\\
36.98	0.01\\
36.99	0.01\\
37	0.01\\
37.01	0.01\\
37.02	0.01\\
37.03	0.01\\
37.04	0.01\\
37.05	0.01\\
37.06	0.01\\
37.07	0.01\\
37.08	0.01\\
37.09	0.01\\
37.1	0.01\\
37.11	0.01\\
37.12	0.01\\
37.13	0.01\\
37.14	0.01\\
37.15	0.01\\
37.16	0.01\\
37.17	0.01\\
37.18	0.01\\
37.19	0.01\\
37.2	0.01\\
37.21	0.01\\
37.22	0.01\\
37.23	0.01\\
37.24	0.01\\
37.25	0.01\\
37.26	0.01\\
37.27	0.01\\
37.28	0.01\\
37.29	0.01\\
37.3	0.01\\
37.31	0.01\\
37.32	0.01\\
37.33	0.01\\
37.34	0.01\\
37.35	0.01\\
37.36	0.01\\
37.37	0.01\\
37.38	0.01\\
37.39	0.01\\
37.4	0.01\\
37.41	0.01\\
37.42	0.01\\
37.43	0.01\\
37.44	0.01\\
37.45	0.01\\
37.46	0.01\\
37.47	0.01\\
37.48	0.01\\
37.49	0.01\\
37.5	0.01\\
37.51	0.01\\
37.52	0.01\\
37.53	0.01\\
37.54	0.01\\
37.55	0.01\\
37.56	0.01\\
37.57	0.01\\
37.58	0.01\\
37.59	0.01\\
37.6	0.01\\
37.61	0.01\\
37.62	0.01\\
37.63	0.01\\
37.64	0.01\\
37.65	0.01\\
37.66	0.01\\
37.67	0.01\\
37.68	0.01\\
37.69	0.01\\
37.7	0.01\\
37.71	0.01\\
37.72	0.01\\
37.73	0.01\\
37.74	0.01\\
37.75	0.01\\
37.76	0.01\\
37.77	0.01\\
37.78	0.01\\
37.79	0.01\\
37.8	0.01\\
37.81	0.01\\
37.82	0.01\\
37.83	0.01\\
37.84	0.01\\
37.85	0.01\\
37.86	0.01\\
37.87	0.01\\
37.88	0.01\\
37.89	0.01\\
37.9	0.01\\
37.91	0.01\\
37.92	0.01\\
37.93	0.01\\
37.94	0.01\\
37.95	0.01\\
37.96	0.01\\
37.97	0.01\\
37.98	0.01\\
37.99	0.01\\
38	0.01\\
38.01	0.01\\
38.02	0.01\\
38.03	0.01\\
38.04	0.01\\
38.05	0.01\\
38.06	0.01\\
38.07	0.01\\
38.08	0.01\\
38.09	0.01\\
38.1	0.01\\
38.11	0.01\\
38.12	0.01\\
38.13	0.01\\
38.14	0.01\\
38.15	0.01\\
38.16	0.01\\
38.17	0.01\\
38.18	0.01\\
38.19	0.01\\
38.2	0.01\\
38.21	0.01\\
38.22	0.01\\
38.23	0.01\\
38.24	0.01\\
38.25	0.01\\
38.26	0.01\\
38.27	0.01\\
38.28	0.01\\
38.29	0.01\\
38.3	0.01\\
38.31	0.01\\
38.32	0.01\\
38.33	0.01\\
38.34	0.01\\
38.35	0.01\\
38.36	0.01\\
38.37	0.01\\
38.38	0.01\\
38.39	0.01\\
38.4	0.01\\
38.41	0.01\\
38.42	0.01\\
38.43	0.01\\
38.44	0.01\\
38.45	0.01\\
38.46	0.01\\
38.47	0.01\\
38.48	0.01\\
38.49	0.01\\
38.5	0.01\\
38.51	0.01\\
38.52	0.01\\
38.53	0.01\\
38.54	0.01\\
38.55	0.01\\
38.56	0.01\\
38.57	0.01\\
38.58	0.01\\
38.59	0.01\\
38.6	0.01\\
38.61	0.01\\
38.62	0.01\\
38.63	0.01\\
38.64	0.01\\
38.65	0.01\\
38.66	0.01\\
38.67	0.01\\
38.68	0.01\\
38.69	0.01\\
38.7	0.01\\
38.71	0.01\\
38.72	0.01\\
38.73	0.01\\
38.74	0.01\\
38.75	0.01\\
38.76	0.01\\
38.77	0.01\\
38.78	0.01\\
38.79	0.01\\
38.8	0.01\\
38.81	0.01\\
38.82	0.01\\
38.83	0.01\\
38.84	0.01\\
38.85	0.01\\
38.86	0.01\\
38.87	0.01\\
38.88	0.01\\
38.89	0.01\\
38.9	0.01\\
38.91	0.01\\
38.92	0.01\\
38.93	0.01\\
38.94	0.01\\
38.95	0.01\\
38.96	0.01\\
38.97	0.01\\
38.98	0.01\\
38.99	0.01\\
39	0.01\\
39.01	0.01\\
39.02	0.01\\
39.03	0.01\\
39.04	0.01\\
39.05	0.01\\
39.06	0.01\\
39.07	0.01\\
39.08	0.01\\
39.09	0.01\\
39.1	0.01\\
39.11	0.01\\
39.12	0.01\\
39.13	0.01\\
39.14	0.01\\
39.15	0.01\\
39.16	0.01\\
39.17	0.01\\
39.18	0.01\\
39.19	0.01\\
39.2	0.01\\
39.21	0.01\\
39.22	0.01\\
39.23	0.01\\
39.24	0.01\\
39.25	0.01\\
39.26	0.01\\
39.27	0.01\\
39.28	0.01\\
39.29	0.01\\
39.3	0.01\\
39.31	0.01\\
39.32	0.01\\
39.33	0.01\\
39.34	0.01\\
39.35	0.01\\
39.36	0.01\\
39.37	0.01\\
39.38	0.01\\
39.39	0.01\\
39.4	0.01\\
39.41	0.01\\
39.42	0.01\\
39.43	0.01\\
39.44	0.01\\
39.45	0.01\\
39.46	0.01\\
39.47	0.01\\
39.48	0.01\\
39.49	0.01\\
39.5	0.01\\
39.51	0.01\\
39.52	0.01\\
39.53	0.01\\
39.54	0.01\\
39.55	0.01\\
39.56	0.01\\
39.57	0.01\\
39.58	0.01\\
39.59	0.01\\
39.6	0.01\\
39.61	0.01\\
39.62	0.01\\
39.63	0.01\\
39.64	0.01\\
39.65	0.01\\
39.66	0.01\\
39.67	0.01\\
39.68	0.01\\
39.69	0.01\\
39.7	0.01\\
39.71	0.01\\
39.72	0.01\\
39.73	0.01\\
39.74	0.01\\
39.75	0.01\\
39.76	0.01\\
39.77	0.01\\
39.78	0.01\\
39.79	0.01\\
39.8	0.01\\
39.81	0.01\\
39.82	0.01\\
39.83	0.01\\
39.84	0.01\\
39.85	0.01\\
39.86	0.01\\
39.87	0.01\\
39.88	0.01\\
39.89	0.01\\
39.9	0.01\\
39.91	0.01\\
39.92	0.01\\
39.93	0.01\\
39.94	0.01\\
39.95	0.01\\
39.96	0.01\\
39.97	0.01\\
39.98	0.01\\
39.99	0.01\\
40	0.01\\
40.01	0.01\\
};
\addplot [color=green,dashed,forget plot]
  table[row sep=crcr]{%
40.01	0.01\\
40.02	0.01\\
40.03	0.01\\
40.04	0.01\\
40.05	0.01\\
40.06	0.01\\
40.07	0.01\\
40.08	0.01\\
40.09	0.01\\
40.1	0.01\\
40.11	0.01\\
40.12	0.01\\
40.13	0.01\\
40.14	0.01\\
40.15	0.01\\
40.16	0.01\\
40.17	0.01\\
40.18	0.01\\
40.19	0.01\\
40.2	0.01\\
40.21	0.01\\
40.22	0.01\\
40.23	0.01\\
40.24	0.01\\
40.25	0.01\\
40.26	0.01\\
40.27	0.01\\
40.28	0.01\\
40.29	0.01\\
40.3	0.01\\
40.31	0.01\\
40.32	0.01\\
40.33	0.01\\
40.34	0.01\\
40.35	0.01\\
40.36	0.01\\
40.37	0.01\\
40.38	0.01\\
40.39	0.01\\
40.4	0.01\\
40.41	0.01\\
40.42	0.01\\
40.43	0.01\\
40.44	0.01\\
40.45	0.01\\
40.46	0.01\\
40.47	0.01\\
40.48	0.01\\
40.49	0.01\\
40.5	0.01\\
40.51	0.01\\
40.52	0.01\\
40.53	0.01\\
40.54	0.01\\
40.55	0.01\\
40.56	0.01\\
40.57	0.01\\
40.58	0.01\\
40.59	0.01\\
40.6	0.01\\
40.61	0.01\\
40.62	0.01\\
40.63	0.01\\
40.64	0.01\\
40.65	0.01\\
40.66	0.01\\
40.67	0.01\\
40.68	0.01\\
40.69	0.01\\
40.7	0.01\\
40.71	0.01\\
40.72	0.01\\
40.73	0.01\\
40.74	0.01\\
40.75	0.01\\
40.76	0.01\\
40.77	0.01\\
40.78	0.01\\
40.79	0.01\\
40.8	0.01\\
40.81	0.01\\
40.82	0.01\\
40.83	0.01\\
40.84	0.01\\
40.85	0.01\\
40.86	0.01\\
40.87	0.01\\
40.88	0.01\\
40.89	0.01\\
40.9	0.01\\
40.91	0.01\\
40.92	0.01\\
40.93	0.01\\
40.94	0.01\\
40.95	0.01\\
40.96	0.01\\
40.97	0.01\\
40.98	0.01\\
40.99	0.01\\
41	0.01\\
41.01	0.01\\
41.02	0.01\\
41.03	0.01\\
41.04	0.01\\
41.05	0.01\\
41.06	0.01\\
41.07	0.01\\
41.08	0.01\\
41.09	0.01\\
41.1	0.01\\
41.11	0.01\\
41.12	0.01\\
41.13	0.01\\
41.14	0.01\\
41.15	0.01\\
41.16	0.01\\
41.17	0.01\\
41.18	0.01\\
41.19	0.01\\
41.2	0.01\\
41.21	0.01\\
41.22	0.01\\
41.23	0.01\\
41.24	0.01\\
41.25	0.01\\
41.26	0.01\\
41.27	0.01\\
41.28	0.01\\
41.29	0.01\\
41.3	0.01\\
41.31	0.01\\
41.32	0.01\\
41.33	0.01\\
41.34	0.01\\
41.35	0.01\\
41.36	0.01\\
41.37	0.01\\
41.38	0.01\\
41.39	0.01\\
41.4	0.01\\
41.41	0.01\\
41.42	0.01\\
41.43	0.01\\
41.44	0.01\\
41.45	0.01\\
41.46	0.01\\
41.47	0.01\\
41.48	0.01\\
41.49	0.01\\
41.5	0.01\\
41.51	0.01\\
41.52	0.01\\
41.53	0.01\\
41.54	0.01\\
41.55	0.01\\
41.56	0.01\\
41.57	0.01\\
41.58	0.01\\
41.59	0.01\\
41.6	0.01\\
41.61	0.01\\
41.62	0.01\\
41.63	0.01\\
41.64	0.01\\
41.65	0.01\\
41.66	0.01\\
41.67	0.01\\
41.68	0.01\\
41.69	0.01\\
41.7	0.01\\
41.71	0.01\\
41.72	0.01\\
41.73	0.01\\
41.74	0.01\\
41.75	0.01\\
41.76	0.01\\
41.77	0.01\\
41.78	0.01\\
41.79	0.01\\
41.8	0.01\\
41.81	0.01\\
41.82	0.01\\
41.83	0.01\\
41.84	0.01\\
41.85	0.01\\
41.86	0.01\\
41.87	0.01\\
41.88	0.01\\
41.89	0.01\\
41.9	0.01\\
41.91	0.01\\
41.92	0.01\\
41.93	0.01\\
41.94	0.01\\
41.95	0.01\\
41.96	0.01\\
41.97	0.01\\
41.98	0.01\\
41.99	0.01\\
42	0.01\\
42.01	0.01\\
42.02	0.01\\
42.03	0.01\\
42.04	0.01\\
42.05	0.01\\
42.06	0.01\\
42.07	0.01\\
42.08	0.01\\
42.09	0.01\\
42.1	0.01\\
42.11	0.01\\
42.12	0.01\\
42.13	0.01\\
42.14	0.01\\
42.15	0.01\\
42.16	0.01\\
42.17	0.01\\
42.18	0.01\\
42.19	0.01\\
42.2	0.01\\
42.21	0.01\\
42.22	0.01\\
42.23	0.01\\
42.24	0.01\\
42.25	0.01\\
42.26	0.01\\
42.27	0.01\\
42.28	0.01\\
42.29	0.01\\
42.3	0.01\\
42.31	0.01\\
42.32	0.01\\
42.33	0.01\\
42.34	0.01\\
42.35	0.01\\
42.36	0.01\\
42.37	0.01\\
42.38	0.01\\
42.39	0.01\\
42.4	0.01\\
42.41	0.01\\
42.42	0.01\\
42.43	0.01\\
42.44	0.01\\
42.45	0.01\\
42.46	0.01\\
42.47	0.01\\
42.48	0.01\\
42.49	0.01\\
42.5	0.01\\
42.51	0.01\\
42.52	0.01\\
42.53	0.01\\
42.54	0.01\\
42.55	0.01\\
42.56	0.01\\
42.57	0.01\\
42.58	0.01\\
42.59	0.01\\
42.6	0.01\\
42.61	0.01\\
42.62	0.01\\
42.63	0.01\\
42.64	0.01\\
42.65	0.01\\
42.66	0.01\\
42.67	0.01\\
42.68	0.01\\
42.69	0.01\\
42.7	0.01\\
42.71	0.01\\
42.72	0.01\\
42.73	0.01\\
42.74	0.01\\
42.75	0.01\\
42.76	0.01\\
42.77	0.01\\
42.78	0.01\\
42.79	0.01\\
42.8	0.01\\
42.81	0.01\\
42.82	0.01\\
42.83	0.01\\
42.84	0.01\\
42.85	0.01\\
42.86	0.01\\
42.87	0.01\\
42.88	0.01\\
42.89	0.01\\
42.9	0.01\\
42.91	0.01\\
42.92	0.01\\
42.93	0.01\\
42.94	0.01\\
42.95	0.01\\
42.96	0.01\\
42.97	0.01\\
42.98	0.01\\
42.99	0.01\\
43	0.01\\
43.01	0.01\\
43.02	0.01\\
43.03	0.01\\
43.04	0.01\\
43.05	0.01\\
43.06	0.01\\
43.07	0.01\\
43.08	0.01\\
43.09	0.01\\
43.1	0.01\\
43.11	0.01\\
43.12	0.01\\
43.13	0.01\\
43.14	0.01\\
43.15	0.01\\
43.16	0.01\\
43.17	0.01\\
43.18	0.01\\
43.19	0.01\\
43.2	0.01\\
43.21	0.01\\
43.22	0.01\\
43.23	0.01\\
43.24	0.01\\
43.25	0.01\\
43.26	0.01\\
43.27	0.01\\
43.28	0.01\\
43.29	0.01\\
43.3	0.01\\
43.31	0.01\\
43.32	0.01\\
43.33	0.01\\
43.34	0.01\\
43.35	0.01\\
43.36	0.01\\
43.37	0.01\\
43.38	0.01\\
43.39	0.01\\
43.4	0.01\\
43.41	0.01\\
43.42	0.01\\
43.43	0.01\\
43.44	0.01\\
43.45	0.01\\
43.46	0.01\\
43.47	0.01\\
43.48	0.01\\
43.49	0.01\\
43.5	0.01\\
43.51	0.01\\
43.52	0.01\\
43.53	0.01\\
43.54	0.01\\
43.55	0.01\\
43.56	0.01\\
43.57	0.01\\
43.58	0.01\\
43.59	0.01\\
43.6	0.01\\
43.61	0.01\\
43.62	0.01\\
43.63	0.01\\
43.64	0.01\\
43.65	0.01\\
43.66	0.01\\
43.67	0.01\\
43.68	0.01\\
43.69	0.01\\
43.7	0.01\\
43.71	0.01\\
43.72	0.01\\
43.73	0.01\\
43.74	0.01\\
43.75	0.01\\
43.76	0.01\\
43.77	0.01\\
43.78	0.01\\
43.79	0.01\\
43.8	0.01\\
43.81	0.01\\
43.82	0.01\\
43.83	0.01\\
43.84	0.01\\
43.85	0.01\\
43.86	0.01\\
43.87	0.01\\
43.88	0.01\\
43.89	0.01\\
43.9	0.01\\
43.91	0.01\\
43.92	0.01\\
43.93	0.01\\
43.94	0.01\\
43.95	0.01\\
43.96	0.01\\
43.97	0.01\\
43.98	0.01\\
43.99	0.01\\
44	0.01\\
44.01	0.01\\
44.02	0.01\\
44.03	0.01\\
44.04	0.01\\
44.05	0.01\\
44.06	0.01\\
44.07	0.01\\
44.08	0.01\\
44.09	0.01\\
44.1	0.01\\
44.11	0.01\\
44.12	0.01\\
44.13	0.01\\
44.14	0.01\\
44.15	0.01\\
44.16	0.01\\
44.17	0.01\\
44.18	0.01\\
44.19	0.01\\
44.2	0.01\\
44.21	0.01\\
44.22	0.01\\
44.23	0.01\\
44.24	0.01\\
44.25	0.01\\
44.26	0.01\\
44.27	0.01\\
44.28	0.01\\
44.29	0.01\\
44.3	0.01\\
44.31	0.01\\
44.32	0.01\\
44.33	0.01\\
44.34	0.01\\
44.35	0.01\\
44.36	0.01\\
44.37	0.01\\
44.38	0.01\\
44.39	0.01\\
44.4	0.01\\
44.41	0.01\\
44.42	0.01\\
44.43	0.01\\
44.44	0.01\\
44.45	0.01\\
44.46	0.01\\
44.47	0.01\\
44.48	0.01\\
44.49	0.01\\
44.5	0.01\\
44.51	0.01\\
44.52	0.01\\
44.53	0.01\\
44.54	0.01\\
44.55	0.01\\
44.56	0.01\\
44.57	0.01\\
44.58	0.01\\
44.59	0.01\\
44.6	0.01\\
44.61	0.01\\
44.62	0.01\\
44.63	0.01\\
44.64	0.01\\
44.65	0.01\\
44.66	0.01\\
44.67	0.01\\
44.68	0.01\\
44.69	0.01\\
44.7	0.01\\
44.71	0.01\\
44.72	0.01\\
44.73	0.01\\
44.74	0.01\\
44.75	0.01\\
44.76	0.01\\
44.77	0.01\\
44.78	0.01\\
44.79	0.01\\
44.8	0.01\\
44.81	0.01\\
44.82	0.01\\
44.83	0.01\\
44.84	0.01\\
44.85	0.01\\
44.86	0.01\\
44.87	0.01\\
44.88	0.01\\
44.89	0.01\\
44.9	0.01\\
44.91	0.01\\
44.92	0.01\\
44.93	0.01\\
44.94	0.01\\
44.95	0.01\\
44.96	0.01\\
44.97	0.01\\
44.98	0.01\\
44.99	0.01\\
45	0.01\\
45.01	0.01\\
45.02	0.01\\
45.03	0.01\\
45.04	0.01\\
45.05	0.01\\
45.06	0.01\\
45.07	0.01\\
45.08	0.01\\
45.09	0.01\\
45.1	0.01\\
45.11	0.01\\
45.12	0.01\\
45.13	0.01\\
45.14	0.01\\
45.15	0.01\\
45.16	0.01\\
45.17	0.01\\
45.18	0.01\\
45.19	0.01\\
45.2	0.01\\
45.21	0.01\\
45.22	0.01\\
45.23	0.01\\
45.24	0.01\\
45.25	0.01\\
45.26	0.01\\
45.27	0.01\\
45.28	0.01\\
45.29	0.01\\
45.3	0.01\\
45.31	0.01\\
45.32	0.01\\
45.33	0.01\\
45.34	0.01\\
45.35	0.01\\
45.36	0.01\\
45.37	0.01\\
45.38	0.01\\
45.39	0.01\\
45.4	0.01\\
45.41	0.01\\
45.42	0.01\\
45.43	0.01\\
45.44	0.01\\
45.45	0.01\\
45.46	0.01\\
45.47	0.01\\
45.48	0.01\\
45.49	0.01\\
45.5	0.01\\
45.51	0.01\\
45.52	0.01\\
45.53	0.01\\
45.54	0.01\\
45.55	0.01\\
45.56	0.01\\
45.57	0.01\\
45.58	0.01\\
45.59	0.01\\
45.6	0.01\\
45.61	0.01\\
45.62	0.01\\
45.63	0.01\\
45.64	0.01\\
45.65	0.01\\
45.66	0.01\\
45.67	0.01\\
45.68	0.01\\
45.69	0.01\\
45.7	0.01\\
45.71	0.01\\
45.72	0.01\\
45.73	0.01\\
45.74	0.01\\
45.75	0.01\\
45.76	0.01\\
45.77	0.01\\
45.78	0.01\\
45.79	0.01\\
45.8	0.01\\
45.81	0.01\\
45.82	0.01\\
45.83	0.01\\
45.84	0.01\\
45.85	0.01\\
45.86	0.01\\
45.87	0.01\\
45.88	0.01\\
45.89	0.01\\
45.9	0.01\\
45.91	0.01\\
45.92	0.01\\
45.93	0.01\\
45.94	0.01\\
45.95	0.01\\
45.96	0.01\\
45.97	0.01\\
45.98	0.01\\
45.99	0.01\\
46	0.01\\
46.01	0.01\\
46.02	0.01\\
46.03	0.01\\
46.04	0.01\\
46.05	0.01\\
46.06	0.01\\
46.07	0.01\\
46.08	0.01\\
46.09	0.01\\
46.1	0.01\\
46.11	0.01\\
46.12	0.01\\
46.13	0.01\\
46.14	0.01\\
46.15	0.01\\
46.16	0.01\\
46.17	0.01\\
46.18	0.01\\
46.19	0.01\\
46.2	0.01\\
46.21	0.01\\
46.22	0.01\\
46.23	0.01\\
46.24	0.01\\
46.25	0.01\\
46.26	0.01\\
46.27	0.01\\
46.28	0.01\\
46.29	0.01\\
46.3	0.01\\
46.31	0.01\\
46.32	0.01\\
46.33	0.01\\
46.34	0.01\\
46.35	0.01\\
46.36	0.01\\
46.37	0.01\\
46.38	0.01\\
46.39	0.01\\
46.4	0.01\\
46.41	0.01\\
46.42	0.01\\
46.43	0.01\\
46.44	0.01\\
46.45	0.01\\
46.46	0.01\\
46.47	0.01\\
46.48	0.01\\
46.49	0.01\\
46.5	0.01\\
46.51	0.01\\
46.52	0.01\\
46.53	0.01\\
46.54	0.01\\
46.55	0.01\\
46.56	0.01\\
46.57	0.01\\
46.58	0.01\\
46.59	0.01\\
46.6	0.01\\
46.61	0.01\\
46.62	0.01\\
46.63	0.01\\
46.64	0.01\\
46.65	0.01\\
46.66	0.01\\
46.67	0.01\\
46.68	0.01\\
46.69	0.01\\
46.7	0.01\\
46.71	0.01\\
46.72	0.01\\
46.73	0.01\\
46.74	0.01\\
46.75	0.01\\
46.76	0.01\\
46.77	0.01\\
46.78	0.01\\
46.79	0.01\\
46.8	0.01\\
46.81	0.01\\
46.82	0.01\\
46.83	0.01\\
46.84	0.01\\
46.85	0.01\\
46.86	0.01\\
46.87	0.01\\
46.88	0.01\\
46.89	0.01\\
46.9	0.01\\
46.91	0.01\\
46.92	0.01\\
46.93	0.01\\
46.94	0.01\\
46.95	0.01\\
46.96	0.01\\
46.97	0.01\\
46.98	0.01\\
46.99	0.01\\
47	0.01\\
47.01	0.01\\
47.02	0.01\\
47.03	0.01\\
47.04	0.01\\
47.05	0.01\\
47.06	0.01\\
47.07	0.01\\
47.08	0.01\\
47.09	0.01\\
47.1	0.01\\
47.11	0.01\\
47.12	0.01\\
47.13	0.01\\
47.14	0.01\\
47.15	0.01\\
47.16	0.01\\
47.17	0.01\\
47.18	0.01\\
47.19	0.01\\
47.2	0.01\\
47.21	0.01\\
47.22	0.01\\
47.23	0.01\\
47.24	0.01\\
47.25	0.01\\
47.26	0.01\\
47.27	0.01\\
47.28	0.01\\
47.29	0.01\\
47.3	0.01\\
47.31	0.01\\
47.32	0.01\\
47.33	0.01\\
47.34	0.01\\
47.35	0.01\\
47.36	0.01\\
47.37	0.01\\
47.38	0.01\\
47.39	0.01\\
47.4	0.01\\
47.41	0.01\\
47.42	0.01\\
47.43	0.01\\
47.44	0.01\\
47.45	0.01\\
47.46	0.01\\
47.47	0.01\\
47.48	0.01\\
47.49	0.01\\
47.5	0.01\\
47.51	0.01\\
47.52	0.01\\
47.53	0.01\\
47.54	0.01\\
47.55	0.01\\
47.56	0.01\\
47.57	0.01\\
47.58	0.01\\
47.59	0.01\\
47.6	0.01\\
47.61	0.01\\
47.62	0.01\\
47.63	0.01\\
47.64	0.01\\
47.65	0.01\\
47.66	0.01\\
47.67	0.01\\
47.68	0.01\\
47.69	0.01\\
47.7	0.01\\
47.71	0.01\\
47.72	0.01\\
47.73	0.01\\
47.74	0.01\\
47.75	0.01\\
47.76	0.01\\
47.77	0.01\\
47.78	0.01\\
47.79	0.01\\
47.8	0.01\\
47.81	0.01\\
47.82	0.01\\
47.83	0.01\\
47.84	0.01\\
47.85	0.01\\
47.86	0.01\\
47.87	0.01\\
47.88	0.01\\
47.89	0.01\\
47.9	0.01\\
47.91	0.01\\
47.92	0.01\\
47.93	0.01\\
47.94	0.01\\
47.95	0.01\\
47.96	0.01\\
47.97	0.01\\
47.98	0.01\\
47.99	0.01\\
48	0.01\\
48.01	0.01\\
48.02	0.01\\
48.03	0.01\\
48.04	0.01\\
48.05	0.01\\
48.06	0.01\\
48.07	0.01\\
48.08	0.01\\
48.09	0.01\\
48.1	0.01\\
48.11	0.01\\
48.12	0.01\\
48.13	0.01\\
48.14	0.01\\
48.15	0.01\\
48.16	0.01\\
48.17	0.01\\
48.18	0.01\\
48.19	0.01\\
48.2	0.01\\
48.21	0.01\\
48.22	0.01\\
48.23	0.01\\
48.24	0.01\\
48.25	0.01\\
48.26	0.01\\
48.27	0.01\\
48.28	0.01\\
48.29	0.01\\
48.3	0.01\\
48.31	0.01\\
48.32	0.01\\
48.33	0.01\\
48.34	0.01\\
48.35	0.01\\
48.36	0.01\\
48.37	0.01\\
48.38	0.01\\
48.39	0.01\\
48.4	0.01\\
48.41	0.01\\
48.42	0.01\\
48.43	0.01\\
48.44	0.01\\
48.45	0.01\\
48.46	0.01\\
48.47	0.01\\
48.48	0.01\\
48.49	0.01\\
48.5	0.01\\
48.51	0.01\\
48.52	0.01\\
48.53	0.01\\
48.54	0.01\\
48.55	0.01\\
48.56	0.01\\
48.57	0.01\\
48.58	0.01\\
48.59	0.01\\
48.6	0.01\\
48.61	0.01\\
48.62	0.01\\
48.63	0.01\\
48.64	0.01\\
48.65	0.01\\
48.66	0.01\\
48.67	0.01\\
48.68	0.01\\
48.69	0.01\\
48.7	0.01\\
48.71	0.01\\
48.72	0.01\\
48.73	0.01\\
48.74	0.01\\
48.75	0.01\\
48.76	0.01\\
48.77	0.01\\
48.78	0.01\\
48.79	0.01\\
48.8	0.01\\
48.81	0.01\\
48.82	0.01\\
48.83	0.01\\
48.84	0.01\\
48.85	0.01\\
48.86	0.01\\
48.87	0.01\\
48.88	0.01\\
48.89	0.01\\
48.9	0.01\\
48.91	0.01\\
48.92	0.01\\
48.93	0.01\\
48.94	0.01\\
48.95	0.01\\
48.96	0.01\\
48.97	0.01\\
48.98	0.01\\
48.99	0.01\\
49	0.01\\
49.01	0.01\\
49.02	0.01\\
49.03	0.01\\
49.04	0.01\\
49.05	0.01\\
49.06	0.01\\
49.07	0.01\\
49.08	0.01\\
49.09	0.01\\
49.1	0.01\\
49.11	0.01\\
49.12	0.01\\
49.13	0.01\\
49.14	0.01\\
49.15	0.01\\
49.16	0.01\\
49.17	0.01\\
49.18	0.01\\
49.19	0.01\\
49.2	0.01\\
49.21	0.01\\
49.22	0.01\\
49.23	0.01\\
49.24	0.01\\
49.25	0.01\\
49.26	0.01\\
49.27	0.01\\
49.28	0.01\\
49.29	0.01\\
49.3	0.01\\
49.31	0.01\\
49.32	0.01\\
49.33	0.01\\
49.34	0.01\\
49.35	0.01\\
49.36	0.01\\
49.37	0.01\\
49.38	0.01\\
49.39	0.01\\
49.4	0.01\\
49.41	0.01\\
49.42	0.01\\
49.43	0.01\\
49.44	0.01\\
49.45	0.01\\
49.46	0.01\\
49.47	0.01\\
49.48	0.01\\
49.49	0.01\\
49.5	0.01\\
49.51	0.01\\
49.52	0.01\\
49.53	0.01\\
49.54	0.01\\
49.55	0.01\\
49.56	0.01\\
49.57	0.01\\
49.58	0.01\\
49.59	0.01\\
49.6	0.01\\
49.61	0.01\\
49.62	0.01\\
49.63	0.01\\
49.64	0.01\\
49.65	0.01\\
49.66	0.01\\
49.67	0.01\\
49.68	0.01\\
49.69	0.01\\
49.7	0.01\\
49.71	0.01\\
49.72	0.01\\
49.73	0.01\\
49.74	0.01\\
49.75	0.01\\
49.76	0.01\\
49.77	0.01\\
49.78	0.01\\
49.79	0.01\\
49.8	0.01\\
49.81	0.01\\
49.82	0.01\\
49.83	0.01\\
49.84	0.01\\
49.85	0.01\\
49.86	0.01\\
49.87	0.01\\
49.88	0.01\\
49.89	0.01\\
49.9	0.01\\
49.91	0.01\\
49.92	0.01\\
49.93	0.01\\
49.94	0.01\\
49.95	0.01\\
49.96	0.01\\
49.97	0.01\\
49.98	0.01\\
49.99	0.01\\
50	0.01\\
50.01	0.01\\
50.02	0.01\\
50.03	0.01\\
50.04	0.01\\
50.05	0.01\\
50.06	0.01\\
50.07	0.01\\
50.08	0.01\\
50.09	0.01\\
50.1	0.01\\
50.11	0.01\\
50.12	0.01\\
50.13	0.01\\
50.14	0.01\\
50.15	0.01\\
50.16	0.01\\
50.17	0.01\\
50.18	0.01\\
50.19	0.01\\
50.2	0.01\\
50.21	0.01\\
50.22	0.01\\
50.23	0.01\\
50.24	0.01\\
50.25	0.01\\
50.26	0.01\\
50.27	0.01\\
50.28	0.01\\
50.29	0.01\\
50.3	0.01\\
50.31	0.01\\
50.32	0.01\\
50.33	0.01\\
50.34	0.01\\
50.35	0.01\\
50.36	0.01\\
50.37	0.01\\
50.38	0.01\\
50.39	0.01\\
50.4	0.01\\
50.41	0.01\\
50.42	0.01\\
50.43	0.01\\
50.44	0.01\\
50.45	0.01\\
50.46	0.01\\
50.47	0.01\\
50.48	0.01\\
50.49	0.01\\
50.5	0.01\\
50.51	0.01\\
50.52	0.01\\
50.53	0.01\\
50.54	0.01\\
50.55	0.01\\
50.56	0.01\\
50.57	0.01\\
50.58	0.01\\
50.59	0.01\\
50.6	0.01\\
50.61	0.01\\
50.62	0.01\\
50.63	0.01\\
50.64	0.01\\
50.65	0.01\\
50.66	0.01\\
50.67	0.01\\
50.68	0.01\\
50.69	0.01\\
50.7	0.01\\
50.71	0.01\\
50.72	0.01\\
50.73	0.01\\
50.74	0.01\\
50.75	0.01\\
50.76	0.01\\
50.77	0.01\\
50.78	0.01\\
50.79	0.01\\
50.8	0.01\\
50.81	0.01\\
50.82	0.01\\
50.83	0.01\\
50.84	0.01\\
50.85	0.01\\
50.86	0.01\\
50.87	0.01\\
50.88	0.01\\
50.89	0.01\\
50.9	0.01\\
50.91	0.01\\
50.92	0.01\\
50.93	0.01\\
50.94	0.01\\
50.95	0.01\\
50.96	0.01\\
50.97	0.01\\
50.98	0.01\\
50.99	0.01\\
51	0.01\\
51.01	0.01\\
51.02	0.01\\
51.03	0.01\\
51.04	0.01\\
51.05	0.01\\
51.06	0.01\\
51.07	0.01\\
51.08	0.01\\
51.09	0.01\\
51.1	0.01\\
51.11	0.01\\
51.12	0.01\\
51.13	0.01\\
51.14	0.01\\
51.15	0.01\\
51.16	0.01\\
51.17	0.01\\
51.18	0.01\\
51.19	0.01\\
51.2	0.01\\
51.21	0.01\\
51.22	0.01\\
51.23	0.01\\
51.24	0.01\\
51.25	0.01\\
51.26	0.01\\
51.27	0.01\\
51.28	0.01\\
51.29	0.01\\
51.3	0.01\\
51.31	0.01\\
51.32	0.01\\
51.33	0.01\\
51.34	0.01\\
51.35	0.01\\
51.36	0.01\\
51.37	0.01\\
51.38	0.01\\
51.39	0.01\\
51.4	0.01\\
51.41	0.01\\
51.42	0.01\\
51.43	0.01\\
51.44	0.01\\
51.45	0.01\\
51.46	0.01\\
51.47	0.01\\
51.48	0.01\\
51.49	0.01\\
51.5	0.01\\
51.51	0.01\\
51.52	0.01\\
51.53	0.01\\
51.54	0.01\\
51.55	0.01\\
51.56	0.01\\
51.57	0.01\\
51.58	0.01\\
51.59	0.01\\
51.6	0.01\\
51.61	0.01\\
51.62	0.01\\
51.63	0.01\\
51.64	0.01\\
51.65	0.01\\
51.66	0.01\\
51.67	0.01\\
51.68	0.01\\
51.69	0.01\\
51.7	0.01\\
51.71	0.01\\
51.72	0.01\\
51.73	0.01\\
51.74	0.01\\
51.75	0.01\\
51.76	0.01\\
51.77	0.01\\
51.78	0.01\\
51.79	0.01\\
51.8	0.01\\
51.81	0.01\\
51.82	0.01\\
51.83	0.01\\
51.84	0.01\\
51.85	0.01\\
51.86	0.01\\
51.87	0.01\\
51.88	0.01\\
51.89	0.01\\
51.9	0.01\\
51.91	0.01\\
51.92	0.01\\
51.93	0.01\\
51.94	0.01\\
51.95	0.01\\
51.96	0.01\\
51.97	0.01\\
51.98	0.01\\
51.99	0.01\\
52	0.01\\
52.01	0.01\\
52.02	0.01\\
52.03	0.01\\
52.04	0.01\\
52.05	0.01\\
52.06	0.01\\
52.07	0.01\\
52.08	0.01\\
52.09	0.01\\
52.1	0.01\\
52.11	0.01\\
52.12	0.01\\
52.13	0.01\\
52.14	0.01\\
52.15	0.01\\
52.16	0.01\\
52.17	0.01\\
52.18	0.01\\
52.19	0.01\\
52.2	0.01\\
52.21	0.01\\
52.22	0.01\\
52.23	0.01\\
52.24	0.01\\
52.25	0.01\\
52.26	0.01\\
52.27	0.01\\
52.28	0.01\\
52.29	0.01\\
52.3	0.01\\
52.31	0.01\\
52.32	0.01\\
52.33	0.01\\
52.34	0.01\\
52.35	0.01\\
52.36	0.01\\
52.37	0.01\\
52.38	0.01\\
52.39	0.01\\
52.4	0.01\\
52.41	0.01\\
52.42	0.01\\
52.43	0.01\\
52.44	0.01\\
52.45	0.01\\
52.46	0.01\\
52.47	0.01\\
52.48	0.01\\
52.49	0.01\\
52.5	0.01\\
52.51	0.01\\
52.52	0.01\\
52.53	0.01\\
52.54	0.01\\
52.55	0.01\\
52.56	0.01\\
52.57	0.01\\
52.58	0.01\\
52.59	0.01\\
52.6	0.01\\
52.61	0.01\\
52.62	0.01\\
52.63	0.01\\
52.64	0.01\\
52.65	0.01\\
52.66	0.01\\
52.67	0.01\\
52.68	0.01\\
52.69	0.01\\
52.7	0.01\\
52.71	0.01\\
52.72	0.01\\
52.73	0.01\\
52.74	0.01\\
52.75	0.01\\
52.76	0.01\\
52.77	0.01\\
52.78	0.01\\
52.79	0.01\\
52.8	0.01\\
52.81	0.01\\
52.82	0.01\\
52.83	0.01\\
52.84	0.01\\
52.85	0.01\\
52.86	0.01\\
52.87	0.01\\
52.88	0.01\\
52.89	0.01\\
52.9	0.01\\
52.91	0.01\\
52.92	0.01\\
52.93	0.01\\
52.94	0.01\\
52.95	0.01\\
52.96	0.01\\
52.97	0.01\\
52.98	0.01\\
52.99	0.01\\
53	0.01\\
53.01	0.01\\
53.02	0.01\\
53.03	0.01\\
53.04	0.01\\
53.05	0.01\\
53.06	0.01\\
53.07	0.01\\
53.08	0.01\\
53.09	0.01\\
53.1	0.01\\
53.11	0.01\\
53.12	0.01\\
53.13	0.01\\
53.14	0.01\\
53.15	0.01\\
53.16	0.01\\
53.17	0.01\\
53.18	0.01\\
53.19	0.01\\
53.2	0.01\\
53.21	0.01\\
53.22	0.01\\
53.23	0.01\\
53.24	0.01\\
53.25	0.01\\
53.26	0.01\\
53.27	0.01\\
53.28	0.01\\
53.29	0.01\\
53.3	0.01\\
53.31	0.01\\
53.32	0.01\\
53.33	0.01\\
53.34	0.01\\
53.35	0.01\\
53.36	0.01\\
53.37	0.01\\
53.38	0.01\\
53.39	0.01\\
53.4	0.01\\
53.41	0.01\\
53.42	0.01\\
53.43	0.01\\
53.44	0.01\\
53.45	0.01\\
53.46	0.01\\
53.47	0.01\\
53.48	0.01\\
53.49	0.01\\
53.5	0.01\\
53.51	0.01\\
53.52	0.01\\
53.53	0.01\\
53.54	0.01\\
53.55	0.01\\
53.56	0.01\\
53.57	0.01\\
53.58	0.01\\
53.59	0.01\\
53.6	0.01\\
53.61	0.01\\
53.62	0.01\\
53.63	0.01\\
53.64	0.01\\
53.65	0.01\\
53.66	0.01\\
53.67	0.01\\
53.68	0.01\\
53.69	0.01\\
53.7	0.01\\
53.71	0.01\\
53.72	0.01\\
53.73	0.01\\
53.74	0.01\\
53.75	0.01\\
53.76	0.01\\
53.77	0.01\\
53.78	0.01\\
53.79	0.01\\
53.8	0.01\\
53.81	0.01\\
53.82	0.01\\
53.83	0.01\\
53.84	0.01\\
53.85	0.01\\
53.86	0.01\\
53.87	0.01\\
53.88	0.01\\
53.89	0.01\\
53.9	0.01\\
53.91	0.01\\
53.92	0.01\\
53.93	0.01\\
53.94	0.01\\
53.95	0.01\\
53.96	0.01\\
53.97	0.01\\
53.98	0.01\\
53.99	0.01\\
54	0.01\\
54.01	0.01\\
54.02	0.01\\
54.03	0.01\\
54.04	0.01\\
54.05	0.01\\
54.06	0.01\\
54.07	0.01\\
54.08	0.01\\
54.09	0.01\\
54.1	0.01\\
54.11	0.01\\
54.12	0.01\\
54.13	0.01\\
54.14	0.01\\
54.15	0.01\\
54.16	0.01\\
54.17	0.01\\
54.18	0.01\\
54.19	0.01\\
54.2	0.01\\
54.21	0.01\\
54.22	0.01\\
54.23	0.01\\
54.24	0.01\\
54.25	0.01\\
54.26	0.01\\
54.27	0.01\\
54.28	0.01\\
54.29	0.01\\
54.3	0.01\\
54.31	0.01\\
54.32	0.01\\
54.33	0.01\\
54.34	0.01\\
54.35	0.01\\
54.36	0.01\\
54.37	0.01\\
54.38	0.01\\
54.39	0.01\\
54.4	0.01\\
54.41	0.01\\
54.42	0.01\\
54.43	0.01\\
54.44	0.01\\
54.45	0.01\\
54.46	0.01\\
54.47	0.01\\
54.48	0.01\\
54.49	0.01\\
54.5	0.01\\
54.51	0.01\\
54.52	0.01\\
54.53	0.01\\
54.54	0.01\\
54.55	0.01\\
54.56	0.01\\
54.57	0.01\\
54.58	0.01\\
54.59	0.01\\
54.6	0.01\\
54.61	0.01\\
54.62	0.01\\
54.63	0.01\\
54.64	0.01\\
54.65	0.01\\
54.66	0.01\\
54.67	0.01\\
54.68	0.01\\
54.69	0.01\\
54.7	0.01\\
54.71	0.01\\
54.72	0.01\\
54.73	0.01\\
54.74	0.01\\
54.75	0.01\\
54.76	0.01\\
54.77	0.01\\
54.78	0.01\\
54.79	0.01\\
54.8	0.01\\
54.81	0.01\\
54.82	0.01\\
54.83	0.01\\
54.84	0.01\\
54.85	0.01\\
54.86	0.01\\
54.87	0.01\\
54.88	0.01\\
54.89	0.01\\
54.9	0.01\\
54.91	0.01\\
54.92	0.01\\
54.93	0.01\\
54.94	0.01\\
54.95	0.01\\
54.96	0.01\\
54.97	0.01\\
54.98	0.01\\
54.99	0.01\\
55	0.01\\
55.01	0.01\\
55.02	0.01\\
55.03	0.01\\
55.04	0.01\\
55.05	0.01\\
55.06	0.01\\
55.07	0.01\\
55.08	0.01\\
55.09	0.01\\
55.1	0.01\\
55.11	0.01\\
55.12	0.01\\
55.13	0.01\\
55.14	0.01\\
55.15	0.01\\
55.16	0.01\\
55.17	0.01\\
55.18	0.01\\
55.19	0.01\\
55.2	0.01\\
55.21	0.01\\
55.22	0.01\\
55.23	0.01\\
55.24	0.01\\
55.25	0.01\\
55.26	0.01\\
55.27	0.01\\
55.28	0.01\\
55.29	0.01\\
55.3	0.01\\
55.31	0.01\\
55.32	0.01\\
55.33	0.01\\
55.34	0.01\\
55.35	0.01\\
55.36	0.01\\
55.37	0.01\\
55.38	0.01\\
55.39	0.01\\
55.4	0.01\\
55.41	0.01\\
55.42	0.01\\
55.43	0.01\\
55.44	0.01\\
55.45	0.01\\
55.46	0.01\\
55.47	0.01\\
55.48	0.01\\
55.49	0.01\\
55.5	0.01\\
55.51	0.01\\
55.52	0.01\\
55.53	0.01\\
55.54	0.01\\
55.55	0.01\\
55.56	0.01\\
55.57	0.01\\
55.58	0.01\\
55.59	0.01\\
55.6	0.01\\
55.61	0.01\\
55.62	0.01\\
55.63	0.01\\
55.64	0.01\\
55.65	0.01\\
55.66	0.01\\
55.67	0.01\\
55.68	0.01\\
55.69	0.01\\
55.7	0.01\\
55.71	0.01\\
55.72	0.01\\
55.73	0.01\\
55.74	0.01\\
55.75	0.01\\
55.76	0.01\\
55.77	0.01\\
55.78	0.01\\
55.79	0.01\\
55.8	0.01\\
55.81	0.01\\
55.82	0.01\\
55.83	0.01\\
55.84	0.01\\
55.85	0.01\\
55.86	0.01\\
55.87	0.01\\
55.88	0.01\\
55.89	0.01\\
55.9	0.01\\
55.91	0.01\\
55.92	0.01\\
55.93	0.01\\
55.94	0.01\\
55.95	0.01\\
55.96	0.01\\
55.97	0.01\\
55.98	0.01\\
55.99	0.01\\
56	0.01\\
56.01	0.01\\
56.02	0.01\\
56.03	0.01\\
56.04	0.01\\
56.05	0.01\\
56.06	0.01\\
56.07	0.01\\
56.08	0.01\\
56.09	0.01\\
56.1	0.01\\
56.11	0.01\\
56.12	0.01\\
56.13	0.01\\
56.14	0.01\\
56.15	0.01\\
56.16	0.01\\
56.17	0.01\\
56.18	0.01\\
56.19	0.01\\
56.2	0.01\\
56.21	0.01\\
56.22	0.01\\
56.23	0.01\\
56.24	0.01\\
56.25	0.01\\
56.26	0.01\\
56.27	0.01\\
56.28	0.01\\
56.29	0.01\\
56.3	0.01\\
56.31	0.01\\
56.32	0.01\\
56.33	0.01\\
56.34	0.01\\
56.35	0.01\\
56.36	0.01\\
56.37	0.01\\
56.38	0.01\\
56.39	0.01\\
56.4	0.01\\
56.41	0.01\\
56.42	0.01\\
56.43	0.01\\
56.44	0.01\\
56.45	0.01\\
56.46	0.01\\
56.47	0.01\\
56.48	0.01\\
56.49	0.01\\
56.5	0.01\\
56.51	0.01\\
56.52	0.01\\
56.53	0.01\\
56.54	0.01\\
56.55	0.01\\
56.56	0.01\\
56.57	0.01\\
56.58	0.01\\
56.59	0.01\\
56.6	0.01\\
56.61	0.01\\
56.62	0.01\\
56.63	0.01\\
56.64	0.01\\
56.65	0.01\\
56.66	0.01\\
56.67	0.01\\
56.68	0.01\\
56.69	0.01\\
56.7	0.01\\
56.71	0.01\\
56.72	0.01\\
56.73	0.01\\
56.74	0.01\\
56.75	0.01\\
56.76	0.01\\
56.77	0.01\\
56.78	0.01\\
56.79	0.01\\
56.8	0.01\\
56.81	0.01\\
56.82	0.01\\
56.83	0.01\\
56.84	0.01\\
56.85	0.01\\
56.86	0.01\\
56.87	0.01\\
56.88	0.01\\
56.89	0.01\\
56.9	0.01\\
56.91	0.01\\
56.92	0.01\\
56.93	0.01\\
56.94	0.01\\
56.95	0.01\\
56.96	0.01\\
56.97	0.01\\
56.98	0.01\\
56.99	0.01\\
57	0.01\\
57.01	0.01\\
57.02	0.01\\
57.03	0.01\\
57.04	0.01\\
57.05	0.01\\
57.06	0.01\\
57.07	0.01\\
57.08	0.01\\
57.09	0.01\\
57.1	0.01\\
57.11	0.01\\
57.12	0.01\\
57.13	0.01\\
57.14	0.01\\
57.15	0.01\\
57.16	0.01\\
57.17	0.01\\
57.18	0.01\\
57.19	0.01\\
57.2	0.01\\
57.21	0.01\\
57.22	0.01\\
57.23	0.01\\
57.24	0.01\\
57.25	0.01\\
57.26	0.01\\
57.27	0.01\\
57.28	0.01\\
57.29	0.01\\
57.3	0.01\\
57.31	0.01\\
57.32	0.01\\
57.33	0.01\\
57.34	0.01\\
57.35	0.01\\
57.36	0.01\\
57.37	0.01\\
57.38	0.01\\
57.39	0.01\\
57.4	0.01\\
57.41	0.01\\
57.42	0.01\\
57.43	0.01\\
57.44	0.01\\
57.45	0.01\\
57.46	0.01\\
57.47	0.01\\
57.48	0.01\\
57.49	0.01\\
57.5	0.01\\
57.51	0.01\\
57.52	0.01\\
57.53	0.01\\
57.54	0.01\\
57.55	0.01\\
57.56	0.01\\
57.57	0.01\\
57.58	0.01\\
57.59	0.01\\
57.6	0.01\\
57.61	0.01\\
57.62	0.01\\
57.63	0.01\\
57.64	0.01\\
57.65	0.01\\
57.66	0.01\\
57.67	0.01\\
57.68	0.01\\
57.69	0.01\\
57.7	0.01\\
57.71	0.01\\
57.72	0.01\\
57.73	0.01\\
57.74	0.01\\
57.75	0.01\\
57.76	0.01\\
57.77	0.01\\
57.78	0.01\\
57.79	0.01\\
57.8	0.01\\
57.81	0.01\\
57.82	0.01\\
57.83	0.01\\
57.84	0.01\\
57.85	0.01\\
57.86	0.01\\
57.87	0.01\\
57.88	0.01\\
57.89	0.01\\
57.9	0.01\\
57.91	0.01\\
57.92	0.01\\
57.93	0.01\\
57.94	0.01\\
57.95	0.01\\
57.96	0.01\\
57.97	0.01\\
57.98	0.01\\
57.99	0.01\\
58	0.01\\
58.01	0.01\\
58.02	0.01\\
58.03	0.01\\
58.04	0.01\\
58.05	0.01\\
58.06	0.01\\
58.07	0.01\\
58.08	0.01\\
58.09	0.01\\
58.1	0.01\\
58.11	0.01\\
58.12	0.01\\
58.13	0.01\\
58.14	0.01\\
58.15	0.01\\
58.16	0.01\\
58.17	0.01\\
58.18	0.01\\
58.19	0.01\\
58.2	0.01\\
58.21	0.01\\
58.22	0.01\\
58.23	0.01\\
58.24	0.01\\
58.25	0.01\\
58.26	0.01\\
58.27	0.01\\
58.28	0.01\\
58.29	0.01\\
58.3	0.01\\
58.31	0.01\\
58.32	0.01\\
58.33	0.01\\
58.34	0.01\\
58.35	0.01\\
58.36	0.01\\
58.37	0.01\\
58.38	0.01\\
58.39	0.01\\
58.4	0.01\\
58.41	0.01\\
58.42	0.01\\
58.43	0.01\\
58.44	0.01\\
58.45	0.01\\
58.46	0.01\\
58.47	0.01\\
58.48	0.01\\
58.49	0.01\\
58.5	0.01\\
58.51	0.01\\
58.52	0.01\\
58.53	0.01\\
58.54	0.01\\
58.55	0.01\\
58.56	0.01\\
58.57	0.01\\
58.58	0.01\\
58.59	0.01\\
58.6	0.01\\
58.61	0.01\\
58.62	0.01\\
58.63	0.01\\
58.64	0.01\\
58.65	0.01\\
58.66	0.01\\
58.67	0.01\\
58.68	0.01\\
58.69	0.01\\
58.7	0.01\\
58.71	0.01\\
58.72	0.01\\
58.73	0.01\\
58.74	0.01\\
58.75	0.01\\
58.76	0.01\\
58.77	0.01\\
58.78	0.01\\
58.79	0.01\\
58.8	0.01\\
58.81	0.01\\
58.82	0.01\\
58.83	0.01\\
58.84	0.01\\
58.85	0.01\\
58.86	0.01\\
58.87	0.01\\
58.88	0.01\\
58.89	0.01\\
58.9	0.01\\
58.91	0.01\\
58.92	0.01\\
58.93	0.01\\
58.94	0.01\\
58.95	0.01\\
58.96	0.01\\
58.97	0.01\\
58.98	0.01\\
58.99	0.01\\
59	0.01\\
59.01	0.01\\
59.02	0.01\\
59.03	0.01\\
59.04	0.01\\
59.05	0.01\\
59.06	0.01\\
59.07	0.01\\
59.08	0.01\\
59.09	0.01\\
59.1	0.01\\
59.11	0.01\\
59.12	0.01\\
59.13	0.01\\
59.14	0.01\\
59.15	0.01\\
59.16	0.01\\
59.17	0.01\\
59.18	0.01\\
59.19	0.01\\
59.2	0.01\\
59.21	0.01\\
59.22	0.01\\
59.23	0.01\\
59.24	0.01\\
59.25	0.01\\
59.26	0.01\\
59.27	0.01\\
59.28	0.01\\
59.29	0.01\\
59.3	0.01\\
59.31	0.01\\
59.32	0.01\\
59.33	0.01\\
59.34	0.01\\
59.35	0.01\\
59.36	0.01\\
59.37	0.01\\
59.38	0.01\\
59.39	0.01\\
59.4	0.01\\
59.41	0.01\\
59.42	0.01\\
59.43	0.01\\
59.44	0.01\\
59.45	0.01\\
59.46	0.01\\
59.47	0.01\\
59.48	0.01\\
59.49	0.01\\
59.5	0.01\\
59.51	0.01\\
59.52	0.01\\
59.53	0.01\\
59.54	0.01\\
59.55	0.01\\
59.56	0.01\\
59.57	0.01\\
59.58	0.01\\
59.59	0.01\\
59.6	0.01\\
59.61	0.01\\
59.62	0.01\\
59.63	0.01\\
59.64	0.01\\
59.65	0.01\\
59.66	0.01\\
59.67	0.01\\
59.68	0.01\\
59.69	0.01\\
59.7	0.01\\
59.71	0.01\\
59.72	0.01\\
59.73	0.01\\
59.74	0.01\\
59.75	0.01\\
59.76	0.01\\
59.77	0.01\\
59.78	0.01\\
59.79	0.01\\
59.8	0.01\\
59.81	0.01\\
59.82	0.01\\
59.83	0.01\\
59.84	0.01\\
59.85	0.01\\
59.86	0.01\\
59.87	0.01\\
59.88	0.01\\
59.89	0.01\\
59.9	0.01\\
59.91	0.01\\
59.92	0.01\\
59.93	0.01\\
59.94	0.01\\
59.95	0.01\\
59.96	0.01\\
59.97	0.01\\
59.98	0.01\\
59.99	0.01\\
60	0.01\\
60.01	0.01\\
60.02	0.01\\
60.03	0.01\\
60.04	0.01\\
60.05	0.01\\
60.06	0.01\\
60.07	0.01\\
60.08	0.01\\
60.09	0.01\\
60.1	0.01\\
60.11	0.01\\
60.12	0.01\\
60.13	0.01\\
60.14	0.01\\
60.15	0.01\\
60.16	0.01\\
60.17	0.01\\
60.18	0.01\\
60.19	0.01\\
60.2	0.01\\
60.21	0.01\\
60.22	0.01\\
60.23	0.01\\
60.24	0.01\\
60.25	0.01\\
60.26	0.01\\
60.27	0.01\\
60.28	0.01\\
60.29	0.01\\
60.3	0.01\\
60.31	0.01\\
60.32	0.01\\
60.33	0.01\\
60.34	0.01\\
60.35	0.01\\
60.36	0.01\\
60.37	0.01\\
60.38	0.01\\
60.39	0.01\\
60.4	0.01\\
60.41	0.01\\
60.42	0.01\\
60.43	0.01\\
60.44	0.01\\
60.45	0.01\\
60.46	0.01\\
60.47	0.01\\
60.48	0.01\\
60.49	0.01\\
60.5	0.01\\
60.51	0.01\\
60.52	0.01\\
60.53	0.01\\
60.54	0.01\\
60.55	0.01\\
60.56	0.01\\
60.57	0.01\\
60.58	0.01\\
60.59	0.01\\
60.6	0.01\\
60.61	0.01\\
60.62	0.01\\
60.63	0.01\\
60.64	0.01\\
60.65	0.01\\
60.66	0.01\\
60.67	0.01\\
60.68	0.01\\
60.69	0.01\\
60.7	0.01\\
60.71	0.01\\
60.72	0.01\\
60.73	0.01\\
60.74	0.01\\
60.75	0.01\\
60.76	0.01\\
60.77	0.01\\
60.78	0.01\\
60.79	0.01\\
60.8	0.01\\
60.81	0.01\\
60.82	0.01\\
60.83	0.01\\
60.84	0.01\\
60.85	0.01\\
60.86	0.01\\
60.87	0.01\\
60.88	0.01\\
60.89	0.01\\
60.9	0.01\\
60.91	0.01\\
60.92	0.01\\
60.93	0.01\\
60.94	0.01\\
60.95	0.01\\
60.96	0.01\\
60.97	0.01\\
60.98	0.01\\
60.99	0.01\\
61	0.01\\
61.01	0.01\\
61.02	0.01\\
61.03	0.01\\
61.04	0.01\\
61.05	0.01\\
61.06	0.01\\
61.07	0.01\\
61.08	0.01\\
61.09	0.01\\
61.1	0.01\\
61.11	0.01\\
61.12	0.01\\
61.13	0.01\\
61.14	0.01\\
61.15	0.01\\
61.16	0.01\\
61.17	0.01\\
61.18	0.01\\
61.19	0.01\\
61.2	0.01\\
61.21	0.01\\
61.22	0.01\\
61.23	0.01\\
61.24	0.01\\
61.25	0.01\\
61.26	0.01\\
61.27	0.01\\
61.28	0.01\\
61.29	0.01\\
61.3	0.01\\
61.31	0.01\\
61.32	0.01\\
61.33	0.01\\
61.34	0.01\\
61.35	0.01\\
61.36	0.01\\
61.37	0.01\\
61.38	0.01\\
61.39	0.01\\
61.4	0.01\\
61.41	0.01\\
61.42	0.01\\
61.43	0.01\\
61.44	0.01\\
61.45	0.01\\
61.46	0.01\\
61.47	0.01\\
61.48	0.01\\
61.49	0.01\\
61.5	0.01\\
61.51	0.01\\
61.52	0.01\\
61.53	0.01\\
61.54	0.01\\
61.55	0.01\\
61.56	0.01\\
61.57	0.01\\
61.58	0.01\\
61.59	0.01\\
61.6	0.01\\
61.61	0.01\\
61.62	0.01\\
61.63	0.01\\
61.64	0.01\\
61.65	0.01\\
61.66	0.01\\
61.67	0.01\\
61.68	0.01\\
61.69	0.01\\
61.7	0.01\\
61.71	0.01\\
61.72	0.01\\
61.73	0.01\\
61.74	0.01\\
61.75	0.01\\
61.76	0.01\\
61.77	0.01\\
61.78	0.01\\
61.79	0.01\\
61.8	0.01\\
61.81	0.01\\
61.82	0.01\\
61.83	0.01\\
61.84	0.01\\
61.85	0.01\\
61.86	0.01\\
61.87	0.01\\
61.88	0.01\\
61.89	0.01\\
61.9	0.01\\
61.91	0.01\\
61.92	0.01\\
61.93	0.01\\
61.94	0.01\\
61.95	0.01\\
61.96	0.01\\
61.97	0.01\\
61.98	0.01\\
61.99	0.01\\
62	0.01\\
62.01	0.01\\
62.02	0.01\\
62.03	0.01\\
62.04	0.01\\
62.05	0.01\\
62.06	0.01\\
62.07	0.01\\
62.08	0.01\\
62.09	0.01\\
62.1	0.01\\
62.11	0.01\\
62.12	0.01\\
62.13	0.01\\
62.14	0.01\\
62.15	0.01\\
62.16	0.01\\
62.17	0.01\\
62.18	0.01\\
62.19	0.01\\
62.2	0.01\\
62.21	0.01\\
62.22	0.01\\
62.23	0.01\\
62.24	0.01\\
62.25	0.01\\
62.26	0.01\\
62.27	0.01\\
62.28	0.01\\
62.29	0.01\\
62.3	0.01\\
62.31	0.01\\
62.32	0.01\\
62.33	0.01\\
62.34	0.01\\
62.35	0.01\\
62.36	0.01\\
62.37	0.01\\
62.38	0.01\\
62.39	0.01\\
62.4	0.01\\
62.41	0.01\\
62.42	0.01\\
62.43	0.01\\
62.44	0.01\\
62.45	0.01\\
62.46	0.01\\
62.47	0.01\\
62.48	0.01\\
62.49	0.01\\
62.5	0.01\\
62.51	0.01\\
62.52	0.01\\
62.53	0.01\\
62.54	0.01\\
62.55	0.01\\
62.56	0.01\\
62.57	0.01\\
62.58	0.01\\
62.59	0.01\\
62.6	0.01\\
62.61	0.01\\
62.62	0.01\\
62.63	0.01\\
62.64	0.01\\
62.65	0.01\\
62.66	0.01\\
62.67	0.01\\
62.68	0.01\\
62.69	0.01\\
62.7	0.01\\
62.71	0.01\\
62.72	0.01\\
62.73	0.01\\
62.74	0.01\\
62.75	0.01\\
62.76	0.01\\
62.77	0.01\\
62.78	0.01\\
62.79	0.01\\
62.8	0.01\\
62.81	0.01\\
62.82	0.01\\
62.83	0.01\\
62.84	0.01\\
62.85	0.01\\
62.86	0.01\\
62.87	0.01\\
62.88	0.01\\
62.89	0.01\\
62.9	0.01\\
62.91	0.01\\
62.92	0.01\\
62.93	0.01\\
62.94	0.01\\
62.95	0.01\\
62.96	0.01\\
62.97	0.01\\
62.98	0.01\\
62.99	0.01\\
63	0.01\\
63.01	0.01\\
63.02	0.01\\
63.03	0.01\\
63.04	0.01\\
63.05	0.01\\
63.06	0.01\\
63.07	0.01\\
63.08	0.01\\
63.09	0.01\\
63.1	0.01\\
63.11	0.01\\
63.12	0.01\\
63.13	0.01\\
63.14	0.01\\
63.15	0.01\\
63.16	0.01\\
63.17	0.01\\
63.18	0.01\\
63.19	0.01\\
63.2	0.01\\
63.21	0.01\\
63.22	0.01\\
63.23	0.01\\
63.24	0.01\\
63.25	0.01\\
63.26	0.01\\
63.27	0.01\\
63.28	0.01\\
63.29	0.01\\
63.3	0.01\\
63.31	0.01\\
63.32	0.01\\
63.33	0.01\\
63.34	0.01\\
63.35	0.01\\
63.36	0.01\\
63.37	0.01\\
63.38	0.01\\
63.39	0.01\\
63.4	0.01\\
63.41	0.01\\
63.42	0.01\\
63.43	0.01\\
63.44	0.01\\
63.45	0.01\\
63.46	0.01\\
63.47	0.01\\
63.48	0.01\\
63.49	0.01\\
63.5	0.01\\
63.51	0.01\\
63.52	0.01\\
63.53	0.01\\
63.54	0.01\\
63.55	0.01\\
63.56	0.01\\
63.57	0.01\\
63.58	0.01\\
63.59	0.01\\
63.6	0.01\\
63.61	0.01\\
63.62	0.01\\
63.63	0.01\\
63.64	0.01\\
63.65	0.01\\
63.66	0.01\\
63.67	0.01\\
63.68	0.01\\
63.69	0.01\\
63.7	0.01\\
63.71	0.01\\
63.72	0.01\\
63.73	0.01\\
63.74	0.01\\
63.75	0.01\\
63.76	0.01\\
63.77	0.01\\
63.78	0.01\\
63.79	0.01\\
63.8	0.01\\
63.81	0.01\\
63.82	0.01\\
63.83	0.01\\
63.84	0.01\\
63.85	0.01\\
63.86	0.01\\
63.87	0.01\\
63.88	0.01\\
63.89	0.01\\
63.9	0.01\\
63.91	0.01\\
63.92	0.01\\
63.93	0.01\\
63.94	0.01\\
63.95	0.01\\
63.96	0.01\\
63.97	0.01\\
63.98	0.01\\
63.99	0.01\\
64	0.01\\
64.01	0.01\\
64.02	0.01\\
64.03	0.01\\
64.04	0.01\\
64.05	0.01\\
64.06	0.01\\
64.07	0.01\\
64.08	0.01\\
64.09	0.01\\
64.1	0.01\\
64.11	0.01\\
64.12	0.01\\
64.13	0.01\\
64.14	0.01\\
64.15	0.01\\
64.16	0.01\\
64.17	0.01\\
64.18	0.01\\
64.19	0.01\\
64.2	0.01\\
64.21	0.01\\
64.22	0.01\\
64.23	0.01\\
64.24	0.01\\
64.25	0.01\\
64.26	0.01\\
64.27	0.01\\
64.28	0.01\\
64.29	0.01\\
64.3	0.01\\
64.31	0.01\\
64.32	0.01\\
64.33	0.01\\
64.34	0.01\\
64.35	0.01\\
64.36	0.01\\
64.37	0.01\\
64.38	0.01\\
64.39	0.01\\
64.4	0.01\\
64.41	0.01\\
64.42	0.01\\
64.43	0.01\\
64.44	0.01\\
64.45	0.01\\
64.46	0.01\\
64.47	0.01\\
64.48	0.01\\
64.49	0.01\\
64.5	0.01\\
64.51	0.01\\
64.52	0.01\\
64.53	0.01\\
64.54	0.01\\
64.55	0.01\\
64.56	0.01\\
64.57	0.01\\
64.58	0.01\\
64.59	0.01\\
64.6	0.01\\
64.61	0.01\\
64.62	0.01\\
64.63	0.01\\
64.64	0.01\\
64.65	0.01\\
64.66	0.01\\
64.67	0.01\\
64.68	0.01\\
64.69	0.01\\
64.7	0.01\\
64.71	0.01\\
64.72	0.01\\
64.73	0.01\\
64.74	0.01\\
64.75	0.01\\
64.76	0.01\\
64.77	0.01\\
64.78	0.01\\
64.79	0.01\\
64.8	0.01\\
64.81	0.01\\
64.82	0.01\\
64.83	0.01\\
64.84	0.01\\
64.85	0.01\\
64.86	0.01\\
64.87	0.01\\
64.88	0.01\\
64.89	0.01\\
64.9	0.01\\
64.91	0.01\\
64.92	0.01\\
64.93	0.01\\
64.94	0.01\\
64.95	0.01\\
64.96	0.01\\
64.97	0.01\\
64.98	0.01\\
64.99	0.01\\
65	0.01\\
65.01	0.01\\
65.02	0.01\\
65.03	0.01\\
65.04	0.01\\
65.05	0.01\\
65.06	0.01\\
65.07	0.01\\
65.08	0.01\\
65.09	0.01\\
65.1	0.01\\
65.11	0.01\\
65.12	0.01\\
65.13	0.01\\
65.14	0.01\\
65.15	0.01\\
65.16	0.01\\
65.17	0.01\\
65.18	0.01\\
65.19	0.01\\
65.2	0.01\\
65.21	0.01\\
65.22	0.01\\
65.23	0.01\\
65.24	0.01\\
65.25	0.01\\
65.26	0.01\\
65.27	0.01\\
65.28	0.01\\
65.29	0.01\\
65.3	0.01\\
65.31	0.01\\
65.32	0.01\\
65.33	0.01\\
65.34	0.01\\
65.35	0.01\\
65.36	0.01\\
65.37	0.01\\
65.38	0.01\\
65.39	0.01\\
65.4	0.01\\
65.41	0.01\\
65.42	0.01\\
65.43	0.01\\
65.44	0.01\\
65.45	0.01\\
65.46	0.01\\
65.47	0.01\\
65.48	0.01\\
65.49	0.01\\
65.5	0.01\\
65.51	0.01\\
65.52	0.01\\
65.53	0.01\\
65.54	0.01\\
65.55	0.01\\
65.56	0.01\\
65.57	0.01\\
65.58	0.01\\
65.59	0.01\\
65.6	0.01\\
65.61	0.01\\
65.62	0.01\\
65.63	0.01\\
65.64	0.01\\
65.65	0.01\\
65.66	0.01\\
65.67	0.01\\
65.68	0.01\\
65.69	0.01\\
65.7	0.01\\
65.71	0.01\\
65.72	0.01\\
65.73	0.01\\
65.74	0.01\\
65.75	0.01\\
65.76	0.01\\
65.77	0.01\\
65.78	0.01\\
65.79	0.01\\
65.8	0.01\\
65.81	0.01\\
65.82	0.01\\
65.83	0.01\\
65.84	0.01\\
65.85	0.01\\
65.86	0.01\\
65.87	0.01\\
65.88	0.01\\
65.89	0.01\\
65.9	0.01\\
65.91	0.01\\
65.92	0.01\\
65.93	0.01\\
65.94	0.01\\
65.95	0.01\\
65.96	0.01\\
65.97	0.01\\
65.98	0.01\\
65.99	0.01\\
66	0.01\\
66.01	0.01\\
66.02	0.01\\
66.03	0.01\\
66.04	0.01\\
66.05	0.01\\
66.06	0.01\\
66.07	0.01\\
66.08	0.01\\
66.09	0.01\\
66.1	0.01\\
66.11	0.01\\
66.12	0.01\\
66.13	0.01\\
66.14	0.01\\
66.15	0.01\\
66.16	0.01\\
66.17	0.01\\
66.18	0.01\\
66.19	0.01\\
66.2	0.01\\
66.21	0.01\\
66.22	0.01\\
66.23	0.01\\
66.24	0.01\\
66.25	0.01\\
66.26	0.01\\
66.27	0.01\\
66.28	0.01\\
66.29	0.01\\
66.3	0.01\\
66.31	0.01\\
66.32	0.01\\
66.33	0.01\\
66.34	0.01\\
66.35	0.01\\
66.36	0.01\\
66.37	0.01\\
66.38	0.01\\
66.39	0.01\\
66.4	0.01\\
66.41	0.01\\
66.42	0.01\\
66.43	0.01\\
66.44	0.01\\
66.45	0.01\\
66.46	0.01\\
66.47	0.01\\
66.48	0.01\\
66.49	0.01\\
66.5	0.01\\
66.51	0.01\\
66.52	0.01\\
66.53	0.01\\
66.54	0.01\\
66.55	0.01\\
66.56	0.01\\
66.57	0.01\\
66.58	0.01\\
66.59	0.01\\
66.6	0.01\\
66.61	0.01\\
66.62	0.01\\
66.63	0.01\\
66.64	0.01\\
66.65	0.01\\
66.66	0.01\\
66.67	0.01\\
66.68	0.01\\
66.69	0.01\\
66.7	0.01\\
66.71	0.01\\
66.72	0.01\\
66.73	0.01\\
66.74	0.01\\
66.75	0.01\\
66.76	0.01\\
66.77	0.01\\
66.78	0.01\\
66.79	0.01\\
66.8	0.01\\
66.81	0.01\\
66.82	0.01\\
66.83	0.01\\
66.84	0.01\\
66.85	0.01\\
66.86	0.01\\
66.87	0.01\\
66.88	0.01\\
66.89	0.01\\
66.9	0.01\\
66.91	0.01\\
66.92	0.01\\
66.93	0.01\\
66.94	0.01\\
66.95	0.01\\
66.96	0.01\\
66.97	0.01\\
66.98	0.01\\
66.99	0.01\\
67	0.01\\
67.01	0.01\\
67.02	0.01\\
67.03	0.01\\
67.04	0.01\\
67.05	0.01\\
67.06	0.01\\
67.07	0.01\\
67.08	0.01\\
67.09	0.01\\
67.1	0.01\\
67.11	0.01\\
67.12	0.01\\
67.13	0.01\\
67.14	0.01\\
67.15	0.01\\
67.16	0.01\\
67.17	0.01\\
67.18	0.01\\
67.19	0.01\\
67.2	0.01\\
67.21	0.01\\
67.22	0.01\\
67.23	0.01\\
67.24	0.01\\
67.25	0.01\\
67.26	0.01\\
67.27	0.01\\
67.28	0.01\\
67.29	0.01\\
67.3	0.01\\
67.31	0.01\\
67.32	0.01\\
67.33	0.01\\
67.34	0.01\\
67.35	0.01\\
67.36	0.01\\
67.37	0.01\\
67.38	0.01\\
67.39	0.01\\
67.4	0.01\\
67.41	0.01\\
67.42	0.01\\
67.43	0.01\\
67.44	0.01\\
67.45	0.01\\
67.46	0.01\\
67.47	0.01\\
67.48	0.01\\
67.49	0.01\\
67.5	0.01\\
67.51	0.01\\
67.52	0.01\\
67.53	0.01\\
67.54	0.01\\
67.55	0.01\\
67.56	0.01\\
67.57	0.01\\
67.58	0.01\\
67.59	0.01\\
67.6	0.01\\
67.61	0.01\\
67.62	0.01\\
67.63	0.01\\
67.64	0.01\\
67.65	0.01\\
67.66	0.01\\
67.67	0.01\\
67.68	0.01\\
67.69	0.01\\
67.7	0.01\\
67.71	0.01\\
67.72	0.01\\
67.73	0.01\\
67.74	0.01\\
67.75	0.01\\
67.76	0.01\\
67.77	0.01\\
67.78	0.01\\
67.79	0.01\\
67.8	0.01\\
67.81	0.01\\
67.82	0.01\\
67.83	0.01\\
67.84	0.01\\
67.85	0.01\\
67.86	0.01\\
67.87	0.01\\
67.88	0.01\\
67.89	0.01\\
67.9	0.01\\
67.91	0.01\\
67.92	0.01\\
67.93	0.01\\
67.94	0.01\\
67.95	0.01\\
67.96	0.01\\
67.97	0.01\\
67.98	0.01\\
67.99	0.01\\
68	0.01\\
68.01	0.01\\
68.02	0.01\\
68.03	0.01\\
68.04	0.01\\
68.05	0.01\\
68.06	0.01\\
68.07	0.01\\
68.08	0.01\\
68.09	0.01\\
68.1	0.01\\
68.11	0.01\\
68.12	0.01\\
68.13	0.01\\
68.14	0.01\\
68.15	0.01\\
68.16	0.01\\
68.17	0.01\\
68.18	0.01\\
68.19	0.01\\
68.2	0.01\\
68.21	0.01\\
68.22	0.01\\
68.23	0.01\\
68.24	0.01\\
68.25	0.01\\
68.26	0.01\\
68.27	0.01\\
68.28	0.01\\
68.29	0.01\\
68.3	0.01\\
68.31	0.01\\
68.32	0.01\\
68.33	0.01\\
68.34	0.01\\
68.35	0.01\\
68.36	0.01\\
68.37	0.01\\
68.38	0.01\\
68.39	0.01\\
68.4	0.01\\
68.41	0.01\\
68.42	0.01\\
68.43	0.01\\
68.44	0.01\\
68.45	0.01\\
68.46	0.01\\
68.47	0.01\\
68.48	0.01\\
68.49	0.01\\
68.5	0.01\\
68.51	0.01\\
68.52	0.01\\
68.53	0.01\\
68.54	0.01\\
68.55	0.01\\
68.56	0.01\\
68.57	0.01\\
68.58	0.01\\
68.59	0.01\\
68.6	0.01\\
68.61	0.01\\
68.62	0.01\\
68.63	0.01\\
68.64	0.01\\
68.65	0.01\\
68.66	0.01\\
68.67	0.01\\
68.68	0.01\\
68.69	0.01\\
68.7	0.01\\
68.71	0.01\\
68.72	0.01\\
68.73	0.01\\
68.74	0.01\\
68.75	0.01\\
68.76	0.01\\
68.77	0.01\\
68.78	0.01\\
68.79	0.01\\
68.8	0.01\\
68.81	0.01\\
68.82	0.01\\
68.83	0.01\\
68.84	0.01\\
68.85	0.01\\
68.86	0.01\\
68.87	0.01\\
68.88	0.01\\
68.89	0.01\\
68.9	0.01\\
68.91	0.01\\
68.92	0.01\\
68.93	0.01\\
68.94	0.01\\
68.95	0.01\\
68.96	0.01\\
68.97	0.01\\
68.98	0.01\\
68.99	0.01\\
69	0.01\\
69.01	0.01\\
69.02	0.01\\
69.03	0.01\\
69.04	0.01\\
69.05	0.01\\
69.06	0.01\\
69.07	0.01\\
69.08	0.01\\
69.09	0.01\\
69.1	0.01\\
69.11	0.01\\
69.12	0.01\\
69.13	0.01\\
69.14	0.01\\
69.15	0.01\\
69.16	0.01\\
69.17	0.01\\
69.18	0.01\\
69.19	0.01\\
69.2	0.01\\
69.21	0.01\\
69.22	0.01\\
69.23	0.01\\
69.24	0.01\\
69.25	0.01\\
69.26	0.01\\
69.27	0.01\\
69.28	0.01\\
69.29	0.01\\
69.3	0.01\\
69.31	0.01\\
69.32	0.01\\
69.33	0.01\\
69.34	0.01\\
69.35	0.01\\
69.36	0.01\\
69.37	0.01\\
69.38	0.01\\
69.39	0.01\\
69.4	0.01\\
69.41	0.01\\
69.42	0.01\\
69.43	0.01\\
69.44	0.01\\
69.45	0.01\\
69.46	0.01\\
69.47	0.01\\
69.48	0.01\\
69.49	0.01\\
69.5	0.01\\
69.51	0.01\\
69.52	0.01\\
69.53	0.01\\
69.54	0.01\\
69.55	0.01\\
69.56	0.01\\
69.57	0.01\\
69.58	0.01\\
69.59	0.01\\
69.6	0.01\\
69.61	0.01\\
69.62	0.01\\
69.63	0.01\\
69.64	0.01\\
69.65	0.01\\
69.66	0.01\\
69.67	0.01\\
69.68	0.01\\
69.69	0.01\\
69.7	0.01\\
69.71	0.01\\
69.72	0.01\\
69.73	0.01\\
69.74	0.01\\
69.75	0.01\\
69.76	0.01\\
69.77	0.01\\
69.78	0.01\\
69.79	0.01\\
69.8	0.01\\
69.81	0.01\\
69.82	0.01\\
69.83	0.01\\
69.84	0.01\\
69.85	0.01\\
69.86	0.01\\
69.87	0.01\\
69.88	0.01\\
69.89	0.01\\
69.9	0.01\\
69.91	0.01\\
69.92	0.01\\
69.93	0.01\\
69.94	0.01\\
69.95	0.01\\
69.96	0.01\\
69.97	0.01\\
69.98	0.01\\
69.99	0.01\\
70	0.01\\
70.01	0.01\\
70.02	0.01\\
70.03	0.01\\
70.04	0.01\\
70.05	0.01\\
70.06	0.01\\
70.07	0.01\\
70.08	0.01\\
70.09	0.01\\
70.1	0.01\\
70.11	0.01\\
70.12	0.01\\
70.13	0.01\\
70.14	0.01\\
70.15	0.01\\
70.16	0.01\\
70.17	0.01\\
70.18	0.01\\
70.19	0.01\\
70.2	0.01\\
70.21	0.01\\
70.22	0.01\\
70.23	0.01\\
70.24	0.01\\
70.25	0.01\\
70.26	0.01\\
70.27	0.01\\
70.28	0.01\\
70.29	0.01\\
70.3	0.01\\
70.31	0.01\\
70.32	0.01\\
70.33	0.01\\
70.34	0.01\\
70.35	0.01\\
70.36	0.01\\
70.37	0.01\\
70.38	0.01\\
70.39	0.01\\
70.4	0.01\\
70.41	0.01\\
70.42	0.01\\
70.43	0.01\\
70.44	0.01\\
70.45	0.01\\
70.46	0.01\\
70.47	0.01\\
70.48	0.01\\
70.49	0.01\\
70.5	0.01\\
70.51	0.01\\
70.52	0.01\\
70.53	0.01\\
70.54	0.01\\
70.55	0.01\\
70.56	0.01\\
70.57	0.01\\
70.58	0.01\\
70.59	0.01\\
70.6	0.01\\
70.61	0.01\\
70.62	0.01\\
70.63	0.01\\
70.64	0.01\\
70.65	0.01\\
70.66	0.01\\
70.67	0.01\\
70.68	0.01\\
70.69	0.01\\
70.7	0.01\\
70.71	0.01\\
70.72	0.01\\
70.73	0.01\\
70.74	0.01\\
70.75	0.01\\
70.76	0.01\\
70.77	0.01\\
70.78	0.01\\
70.79	0.01\\
70.8	0.01\\
70.81	0.01\\
70.82	0.01\\
70.83	0.01\\
70.84	0.01\\
70.85	0.01\\
70.86	0.01\\
70.87	0.01\\
70.88	0.01\\
70.89	0.01\\
70.9	0.01\\
70.91	0.01\\
70.92	0.01\\
70.93	0.01\\
70.94	0.01\\
70.95	0.01\\
70.96	0.01\\
70.97	0.01\\
70.98	0.01\\
70.99	0.01\\
71	0.01\\
71.01	0.01\\
71.02	0.01\\
71.03	0.01\\
71.04	0.01\\
71.05	0.01\\
71.06	0.01\\
71.07	0.01\\
71.08	0.01\\
71.09	0.01\\
71.1	0.01\\
71.11	0.01\\
71.12	0.01\\
71.13	0.01\\
71.14	0.01\\
71.15	0.01\\
71.16	0.01\\
71.17	0.01\\
71.18	0.01\\
71.19	0.01\\
71.2	0.01\\
71.21	0.01\\
71.22	0.01\\
71.23	0.01\\
71.24	0.01\\
71.25	0.01\\
71.26	0.01\\
71.27	0.01\\
71.28	0.01\\
71.29	0.01\\
71.3	0.01\\
71.31	0.01\\
71.32	0.01\\
71.33	0.01\\
71.34	0.01\\
71.35	0.01\\
71.36	0.01\\
71.37	0.01\\
71.38	0.01\\
71.39	0.01\\
71.4	0.01\\
71.41	0.01\\
71.42	0.01\\
71.43	0.01\\
71.44	0.01\\
71.45	0.01\\
71.46	0.01\\
71.47	0.01\\
71.48	0.01\\
71.49	0.01\\
71.5	0.01\\
71.51	0.01\\
71.52	0.01\\
71.53	0.01\\
71.54	0.01\\
71.55	0.01\\
71.56	0.01\\
71.57	0.01\\
71.58	0.01\\
71.59	0.01\\
71.6	0.01\\
71.61	0.01\\
71.62	0.01\\
71.63	0.01\\
71.64	0.01\\
71.65	0.01\\
71.66	0.01\\
71.67	0.01\\
71.68	0.01\\
71.69	0.01\\
71.7	0.01\\
71.71	0.01\\
71.72	0.01\\
71.73	0.01\\
71.74	0.01\\
71.75	0.01\\
71.76	0.01\\
71.77	0.01\\
71.78	0.01\\
71.79	0.01\\
71.8	0.01\\
71.81	0.01\\
71.82	0.01\\
71.83	0.01\\
71.84	0.01\\
71.85	0.01\\
71.86	0.01\\
71.87	0.01\\
71.88	0.01\\
71.89	0.01\\
71.9	0.01\\
71.91	0.01\\
71.92	0.01\\
71.93	0.01\\
71.94	0.01\\
71.95	0.01\\
71.96	0.01\\
71.97	0.01\\
71.98	0.01\\
71.99	0.01\\
72	0.01\\
72.01	0.01\\
72.02	0.01\\
72.03	0.01\\
72.04	0.01\\
72.05	0.01\\
72.06	0.01\\
72.07	0.01\\
72.08	0.01\\
72.09	0.01\\
72.1	0.01\\
72.11	0.01\\
72.12	0.01\\
72.13	0.01\\
72.14	0.01\\
72.15	0.01\\
72.16	0.01\\
72.17	0.01\\
72.18	0.01\\
72.19	0.01\\
72.2	0.01\\
72.21	0.01\\
72.22	0.01\\
72.23	0.01\\
72.24	0.01\\
72.25	0.01\\
72.26	0.01\\
72.27	0.01\\
72.28	0.01\\
72.29	0.01\\
72.3	0.01\\
72.31	0.01\\
72.32	0.01\\
72.33	0.01\\
72.34	0.01\\
72.35	0.01\\
72.36	0.01\\
72.37	0.01\\
72.38	0.01\\
72.39	0.01\\
72.4	0.01\\
72.41	0.01\\
72.42	0.01\\
72.43	0.01\\
72.44	0.01\\
72.45	0.01\\
72.46	0.01\\
72.47	0.01\\
72.48	0.01\\
72.49	0.01\\
72.5	0.01\\
72.51	0.01\\
72.52	0.01\\
72.53	0.01\\
72.54	0.01\\
72.55	0.01\\
72.56	0.01\\
72.57	0.01\\
72.58	0.01\\
72.59	0.01\\
72.6	0.01\\
72.61	0.01\\
72.62	0.01\\
72.63	0.01\\
72.64	0.01\\
72.65	0.01\\
72.66	0.01\\
72.67	0.01\\
72.68	0.01\\
72.69	0.01\\
72.7	0.01\\
72.71	0.01\\
72.72	0.01\\
72.73	0.01\\
72.74	0.01\\
72.75	0.01\\
72.76	0.01\\
72.77	0.01\\
72.78	0.01\\
72.79	0.01\\
72.8	0.01\\
72.81	0.01\\
72.82	0.01\\
72.83	0.01\\
72.84	0.01\\
72.85	0.01\\
72.86	0.01\\
72.87	0.01\\
72.88	0.01\\
72.89	0.01\\
72.9	0.01\\
72.91	0.01\\
72.92	0.01\\
72.93	0.01\\
72.94	0.01\\
72.95	0.01\\
72.96	0.01\\
72.97	0.01\\
72.98	0.01\\
72.99	0.01\\
73	0.01\\
73.01	0.01\\
73.02	0.01\\
73.03	0.01\\
73.04	0.01\\
73.05	0.01\\
73.06	0.01\\
73.07	0.01\\
73.08	0.01\\
73.09	0.01\\
73.1	0.01\\
73.11	0.01\\
73.12	0.01\\
73.13	0.01\\
73.14	0.01\\
73.15	0.01\\
73.16	0.01\\
73.17	0.01\\
73.18	0.01\\
73.19	0.01\\
73.2	0.01\\
73.21	0.01\\
73.22	0.01\\
73.23	0.01\\
73.24	0.01\\
73.25	0.01\\
73.26	0.01\\
73.27	0.01\\
73.28	0.01\\
73.29	0.01\\
73.3	0.01\\
73.31	0.01\\
73.32	0.01\\
73.33	0.01\\
73.34	0.01\\
73.35	0.01\\
73.36	0.01\\
73.37	0.01\\
73.38	0.01\\
73.39	0.01\\
73.4	0.01\\
73.41	0.01\\
73.42	0.01\\
73.43	0.01\\
73.44	0.01\\
73.45	0.01\\
73.46	0.01\\
73.47	0.01\\
73.48	0.01\\
73.49	0.01\\
73.5	0.01\\
73.51	0.01\\
73.52	0.01\\
73.53	0.01\\
73.54	0.01\\
73.55	0.01\\
73.56	0.01\\
73.57	0.01\\
73.58	0.01\\
73.59	0.01\\
73.6	0.01\\
73.61	0.01\\
73.62	0.01\\
73.63	0.01\\
73.64	0.01\\
73.65	0.01\\
73.66	0.01\\
73.67	0.01\\
73.68	0.01\\
73.69	0.01\\
73.7	0.01\\
73.71	0.01\\
73.72	0.01\\
73.73	0.01\\
73.74	0.01\\
73.75	0.01\\
73.76	0.01\\
73.77	0.01\\
73.78	0.01\\
73.79	0.01\\
73.8	0.01\\
73.81	0.01\\
73.82	0.01\\
73.83	0.01\\
73.84	0.01\\
73.85	0.01\\
73.86	0.01\\
73.87	0.01\\
73.88	0.01\\
73.89	0.01\\
73.9	0.01\\
73.91	0.01\\
73.92	0.01\\
73.93	0.01\\
73.94	0.01\\
73.95	0.01\\
73.96	0.01\\
73.97	0.01\\
73.98	0.01\\
73.99	0.01\\
74	0.01\\
74.01	0.01\\
74.02	0.01\\
74.03	0.01\\
74.04	0.01\\
74.05	0.01\\
74.06	0.01\\
74.07	0.01\\
74.08	0.01\\
74.09	0.01\\
74.1	0.01\\
74.11	0.01\\
74.12	0.01\\
74.13	0.01\\
74.14	0.01\\
74.15	0.01\\
74.16	0.01\\
74.17	0.01\\
74.18	0.01\\
74.19	0.01\\
74.2	0.01\\
74.21	0.01\\
74.22	0.01\\
74.23	0.01\\
74.24	0.01\\
74.25	0.01\\
74.26	0.01\\
74.27	0.01\\
74.28	0.01\\
74.29	0.01\\
74.3	0.01\\
74.31	0.01\\
74.32	0.01\\
74.33	0.01\\
74.34	0.01\\
74.35	0.01\\
74.36	0.01\\
74.37	0.01\\
74.38	0.01\\
74.39	0.01\\
74.4	0.01\\
74.41	0.01\\
74.42	0.01\\
74.43	0.01\\
74.44	0.01\\
74.45	0.01\\
74.46	0.01\\
74.47	0.01\\
74.48	0.01\\
74.49	0.01\\
74.5	0.01\\
74.51	0.01\\
74.52	0.01\\
74.53	0.01\\
74.54	0.01\\
74.55	0.01\\
74.56	0.01\\
74.57	0.01\\
74.58	0.01\\
74.59	0.01\\
74.6	0.01\\
74.61	0.01\\
74.62	0.01\\
74.63	0.01\\
74.64	0.01\\
74.65	0.01\\
74.66	0.01\\
74.67	0.01\\
74.68	0.01\\
74.69	0.01\\
74.7	0.01\\
74.71	0.01\\
74.72	0.01\\
74.73	0.01\\
74.74	0.01\\
74.75	0.01\\
74.76	0.01\\
74.77	0.01\\
74.78	0.01\\
74.79	0.01\\
74.8	0.01\\
74.81	0.01\\
74.82	0.01\\
74.83	0.01\\
74.84	0.01\\
74.85	0.01\\
74.86	0.01\\
74.87	0.01\\
74.88	0.01\\
74.89	0.01\\
74.9	0.01\\
74.91	0.01\\
74.92	0.01\\
74.93	0.01\\
74.94	0.01\\
74.95	0.01\\
74.96	0.01\\
74.97	0.01\\
74.98	0.01\\
74.99	0.01\\
75	0.01\\
75.01	0.01\\
75.02	0.01\\
75.03	0.01\\
75.04	0.01\\
75.05	0.01\\
75.06	0.01\\
75.07	0.01\\
75.08	0.01\\
75.09	0.01\\
75.1	0.01\\
75.11	0.01\\
75.12	0.01\\
75.13	0.01\\
75.14	0.01\\
75.15	0.01\\
75.16	0.01\\
75.17	0.01\\
75.18	0.01\\
75.19	0.01\\
75.2	0.01\\
75.21	0.01\\
75.22	0.01\\
75.23	0.01\\
75.24	0.01\\
75.25	0.01\\
75.26	0.01\\
75.27	0.01\\
75.28	0.01\\
75.29	0.01\\
75.3	0.01\\
75.31	0.01\\
75.32	0.01\\
75.33	0.01\\
75.34	0.01\\
75.35	0.01\\
75.36	0.01\\
75.37	0.01\\
75.38	0.01\\
75.39	0.01\\
75.4	0.01\\
75.41	0.01\\
75.42	0.01\\
75.43	0.01\\
75.44	0.01\\
75.45	0.01\\
75.46	0.01\\
75.47	0.01\\
75.48	0.01\\
75.49	0.01\\
75.5	0.01\\
75.51	0.01\\
75.52	0.01\\
75.53	0.01\\
75.54	0.01\\
75.55	0.01\\
75.56	0.01\\
75.57	0.01\\
75.58	0.01\\
75.59	0.01\\
75.6	0.01\\
75.61	0.01\\
75.62	0.01\\
75.63	0.01\\
75.64	0.01\\
75.65	0.01\\
75.66	0.01\\
75.67	0.01\\
75.68	0.01\\
75.69	0.01\\
75.7	0.01\\
75.71	0.01\\
75.72	0.01\\
75.73	0.01\\
75.74	0.01\\
75.75	0.01\\
75.76	0.01\\
75.77	0.01\\
75.78	0.01\\
75.79	0.01\\
75.8	0.01\\
75.81	0.01\\
75.82	0.01\\
75.83	0.01\\
75.84	0.01\\
75.85	0.01\\
75.86	0.01\\
75.87	0.01\\
75.88	0.01\\
75.89	0.01\\
75.9	0.01\\
75.91	0.01\\
75.92	0.01\\
75.93	0.01\\
75.94	0.01\\
75.95	0.01\\
75.96	0.01\\
75.97	0.01\\
75.98	0.01\\
75.99	0.01\\
76	0.01\\
76.01	0.01\\
76.02	0.01\\
76.03	0.01\\
76.04	0.01\\
76.05	0.01\\
76.06	0.01\\
76.07	0.01\\
76.08	0.01\\
76.09	0.01\\
76.1	0.01\\
76.11	0.01\\
76.12	0.01\\
76.13	0.01\\
76.14	0.01\\
76.15	0.01\\
76.16	0.01\\
76.17	0.01\\
76.18	0.01\\
76.19	0.01\\
76.2	0.01\\
76.21	0.01\\
76.22	0.01\\
76.23	0.01\\
76.24	0.01\\
76.25	0.01\\
76.26	0.01\\
76.27	0.01\\
76.28	0.01\\
76.29	0.01\\
76.3	0.01\\
76.31	0.01\\
76.32	0.01\\
76.33	0.01\\
76.34	0.01\\
76.35	0.01\\
76.36	0.01\\
76.37	0.01\\
76.38	0.01\\
76.39	0.01\\
76.4	0.01\\
76.41	0.01\\
76.42	0.01\\
76.43	0.01\\
76.44	0.01\\
76.45	0.01\\
76.46	0.01\\
76.47	0.01\\
76.48	0.01\\
76.49	0.01\\
76.5	0.01\\
76.51	0.01\\
76.52	0.01\\
76.53	0.01\\
76.54	0.01\\
76.55	0.01\\
76.56	0.01\\
76.57	0.01\\
76.58	0.01\\
76.59	0.01\\
76.6	0.01\\
76.61	0.01\\
76.62	0.01\\
76.63	0.01\\
76.64	0.01\\
76.65	0.01\\
76.66	0.01\\
76.67	0.01\\
76.68	0.01\\
76.69	0.01\\
76.7	0.01\\
76.71	0.01\\
76.72	0.01\\
76.73	0.01\\
76.74	0.01\\
76.75	0.01\\
76.76	0.01\\
76.77	0.01\\
76.78	0.01\\
76.79	0.01\\
76.8	0.01\\
76.81	0.01\\
76.82	0.01\\
76.83	0.01\\
76.84	0.01\\
76.85	0.01\\
76.86	0.01\\
76.87	0.01\\
76.88	0.01\\
76.89	0.01\\
76.9	0.01\\
76.91	0.01\\
76.92	0.01\\
76.93	0.01\\
76.94	0.01\\
76.95	0.01\\
76.96	0.01\\
76.97	0.01\\
76.98	0.01\\
76.99	0.01\\
77	0.01\\
77.01	0.01\\
77.02	0.01\\
77.03	0.01\\
77.04	0.01\\
77.05	0.01\\
77.06	0.01\\
77.07	0.01\\
77.08	0.01\\
77.09	0.01\\
77.1	0.01\\
77.11	0.01\\
77.12	0.01\\
77.13	0.01\\
77.14	0.01\\
77.15	0.01\\
77.16	0.01\\
77.17	0.01\\
77.18	0.01\\
77.19	0.01\\
77.2	0.01\\
77.21	0.01\\
77.22	0.01\\
77.23	0.01\\
77.24	0.01\\
77.25	0.01\\
77.26	0.01\\
77.27	0.01\\
77.28	0.01\\
77.29	0.01\\
77.3	0.01\\
77.31	0.01\\
77.32	0.01\\
77.33	0.01\\
77.34	0.01\\
77.35	0.01\\
77.36	0.01\\
77.37	0.01\\
77.38	0.01\\
77.39	0.01\\
77.4	0.01\\
77.41	0.01\\
77.42	0.01\\
77.43	0.01\\
77.44	0.01\\
77.45	0.01\\
77.46	0.01\\
77.47	0.01\\
77.48	0.01\\
77.49	0.01\\
77.5	0.01\\
77.51	0.01\\
77.52	0.01\\
77.53	0.01\\
77.54	0.01\\
77.55	0.01\\
77.56	0.01\\
77.57	0.01\\
77.58	0.01\\
77.59	0.01\\
77.6	0.01\\
77.61	0.01\\
77.62	0.01\\
77.63	0.01\\
77.64	0.01\\
77.65	0.01\\
77.66	0.01\\
77.67	0.01\\
77.68	0.01\\
77.69	0.01\\
77.7	0.01\\
77.71	0.01\\
77.72	0.01\\
77.73	0.01\\
77.74	0.01\\
77.75	0.01\\
77.76	0.01\\
77.77	0.01\\
77.78	0.01\\
77.79	0.01\\
77.8	0.01\\
77.81	0.01\\
77.82	0.01\\
77.83	0.01\\
77.84	0.01\\
77.85	0.01\\
77.86	0.01\\
77.87	0.01\\
77.88	0.01\\
77.89	0.01\\
77.9	0.01\\
77.91	0.01\\
77.92	0.01\\
77.93	0.01\\
77.94	0.01\\
77.95	0.01\\
77.96	0.01\\
77.97	0.01\\
77.98	0.01\\
77.99	0.01\\
78	0.01\\
78.01	0.01\\
78.02	0.01\\
78.03	0.01\\
78.04	0.01\\
78.05	0.01\\
78.06	0.01\\
78.07	0.01\\
78.08	0.01\\
78.09	0.01\\
78.1	0.01\\
78.11	0.01\\
78.12	0.01\\
78.13	0.01\\
78.14	0.01\\
78.15	0.01\\
78.16	0.01\\
78.17	0.01\\
78.18	0.01\\
78.19	0.01\\
78.2	0.01\\
78.21	0.01\\
78.22	0.01\\
78.23	0.01\\
78.24	0.01\\
78.25	0.01\\
78.26	0.01\\
78.27	0.01\\
78.28	0.01\\
78.29	0.01\\
78.3	0.01\\
78.31	0.01\\
78.32	0.01\\
78.33	0.01\\
78.34	0.01\\
78.35	0.01\\
78.36	0.01\\
78.37	0.01\\
78.38	0.01\\
78.39	0.01\\
78.4	0.01\\
78.41	0.01\\
78.42	0.01\\
78.43	0.01\\
78.44	0.01\\
78.45	0.01\\
78.46	0.01\\
78.47	0.01\\
78.48	0.01\\
78.49	0.01\\
78.5	0.01\\
78.51	0.01\\
78.52	0.01\\
78.53	0.01\\
78.54	0.01\\
78.55	0.01\\
78.56	0.01\\
78.57	0.01\\
78.58	0.01\\
78.59	0.01\\
78.6	0.01\\
78.61	0.01\\
78.62	0.01\\
78.63	0.01\\
78.64	0.01\\
78.65	0.01\\
78.66	0.01\\
78.67	0.01\\
78.68	0.01\\
78.69	0.01\\
78.7	0.01\\
78.71	0.01\\
78.72	0.01\\
78.73	0.01\\
78.74	0.01\\
78.75	0.01\\
78.76	0.01\\
78.77	0.01\\
78.78	0.01\\
78.79	0.01\\
78.8	0.01\\
78.81	0.01\\
78.82	0.01\\
78.83	0.01\\
78.84	0.01\\
78.85	0.01\\
78.86	0.01\\
78.87	0.01\\
78.88	0.01\\
78.89	0.01\\
78.9	0.01\\
78.91	0.01\\
78.92	0.01\\
78.93	0.01\\
78.94	0.01\\
78.95	0.01\\
78.96	0.01\\
78.97	0.01\\
78.98	0.01\\
78.99	0.01\\
79	0.01\\
79.01	0.01\\
79.02	0.01\\
79.03	0.01\\
79.04	0.01\\
79.05	0.01\\
79.06	0.01\\
79.07	0.01\\
79.08	0.01\\
79.09	0.01\\
79.1	0.01\\
79.11	0.01\\
79.12	0.01\\
79.13	0.01\\
79.14	0.01\\
79.15	0.01\\
79.16	0.01\\
79.17	0.01\\
79.18	0.01\\
79.19	0.01\\
79.2	0.01\\
79.21	0.01\\
79.22	0.01\\
79.23	0.01\\
79.24	0.01\\
79.25	0.01\\
79.26	0.01\\
79.27	0.01\\
79.28	0.01\\
79.29	0.01\\
79.3	0.01\\
79.31	0.01\\
79.32	0.01\\
79.33	0.01\\
79.34	0.01\\
79.35	0.01\\
79.36	0.01\\
79.37	0.01\\
79.38	0.01\\
79.39	0.01\\
79.4	0.01\\
79.41	0.01\\
79.42	0.01\\
79.43	0.01\\
79.44	0.01\\
79.45	0.01\\
79.46	0.01\\
79.47	0.01\\
79.48	0.01\\
79.49	0.01\\
79.5	0.01\\
79.51	0.01\\
79.52	0.01\\
79.53	0.01\\
79.54	0.01\\
79.55	0.01\\
79.56	0.01\\
79.57	0.01\\
79.58	0.01\\
79.59	0.01\\
79.6	0.01\\
79.61	0.01\\
79.62	0.01\\
79.63	0.01\\
79.64	0.01\\
79.65	0.01\\
79.66	0.01\\
79.67	0.01\\
79.68	0.01\\
79.69	0.01\\
79.7	0.01\\
79.71	0.01\\
79.72	0.01\\
79.73	0.01\\
79.74	0.01\\
79.75	0.01\\
79.76	0.01\\
79.77	0.01\\
79.78	0.01\\
79.79	0.01\\
79.8	0.01\\
79.81	0.01\\
79.82	0.01\\
79.83	0.01\\
79.84	0.01\\
79.85	0.01\\
79.86	0.01\\
79.87	0.01\\
79.88	0.01\\
79.89	0.01\\
79.9	0.01\\
79.91	0.01\\
79.92	0.01\\
79.93	0.01\\
79.94	0.01\\
79.95	0.01\\
79.96	0.01\\
79.97	0.01\\
79.98	0.01\\
79.99	0.01\\
80	0.01\\
80.01	0.01\\
};
\addplot [color=green,dashed]
  table[row sep=crcr]{%
80.01	0.01\\
80.02	0.01\\
80.03	0.01\\
80.04	0.01\\
80.05	0.01\\
80.06	0.01\\
80.07	0.01\\
80.08	0.01\\
80.09	0.01\\
80.1	0.01\\
80.11	0.01\\
80.12	0.01\\
80.13	0.01\\
80.14	0.01\\
80.15	0.01\\
80.16	0.01\\
80.17	0.01\\
80.18	0.01\\
80.19	0.01\\
80.2	0.01\\
80.21	0.01\\
80.22	0.01\\
80.23	0.01\\
80.24	0.01\\
80.25	0.01\\
80.26	0.01\\
80.27	0.01\\
80.28	0.01\\
80.29	0.01\\
80.3	0.01\\
80.31	0.01\\
80.32	0.01\\
80.33	0.01\\
80.34	0.01\\
80.35	0.01\\
80.36	0.01\\
80.37	0.01\\
80.38	0.01\\
80.39	0.01\\
80.4	0.01\\
80.41	0.01\\
80.42	0.01\\
80.43	0.01\\
80.44	0.01\\
80.45	0.01\\
80.46	0.01\\
80.47	0.01\\
80.48	0.01\\
80.49	0.01\\
80.5	0.01\\
80.51	0.01\\
80.52	0.01\\
80.53	0.01\\
80.54	0.01\\
80.55	0.01\\
80.56	0.01\\
80.57	0.01\\
80.58	0.01\\
80.59	0.01\\
80.6	0.01\\
80.61	0.01\\
80.62	0.01\\
80.63	0.01\\
80.64	0.01\\
80.65	0.01\\
80.66	0.01\\
80.67	0.01\\
80.68	0.01\\
80.69	0.01\\
80.7	0.01\\
80.71	0.01\\
80.72	0.01\\
80.73	0.01\\
80.74	0.01\\
80.75	0.01\\
80.76	0.01\\
80.77	0.01\\
80.78	0.01\\
80.79	0.01\\
80.8	0.01\\
80.81	0.01\\
80.82	0.01\\
80.83	0.01\\
80.84	0.01\\
80.85	0.01\\
80.86	0.01\\
80.87	0.01\\
80.88	0.01\\
80.89	0.01\\
80.9	0.01\\
80.91	0.01\\
80.92	0.01\\
80.93	0.01\\
80.94	0.01\\
80.95	0.01\\
80.96	0.01\\
80.97	0.01\\
80.98	0.01\\
80.99	0.01\\
81	0.01\\
81.01	0.01\\
81.02	0.01\\
81.03	0.01\\
81.04	0.01\\
81.05	0.01\\
81.06	0.01\\
81.07	0.01\\
81.08	0.01\\
81.09	0.01\\
81.1	0.01\\
81.11	0.01\\
81.12	0.01\\
81.13	0.01\\
81.14	0.01\\
81.15	0.01\\
81.16	0.01\\
81.17	0.01\\
81.18	0.01\\
81.19	0.01\\
81.2	0.01\\
81.21	0.01\\
81.22	0.01\\
81.23	0.01\\
81.24	0.01\\
81.25	0.01\\
81.26	0.01\\
81.27	0.01\\
81.28	0.01\\
81.29	0.01\\
81.3	0.01\\
81.31	0.01\\
81.32	0.01\\
81.33	0.01\\
81.34	0.01\\
81.35	0.01\\
81.36	0.01\\
81.37	0.01\\
81.38	0.01\\
81.39	0.01\\
81.4	0.01\\
81.41	0.01\\
81.42	0.01\\
81.43	0.01\\
81.44	0.01\\
81.45	0.01\\
81.46	0.01\\
81.47	0.01\\
81.48	0.01\\
81.49	0.01\\
81.5	0.01\\
81.51	0.01\\
81.52	0.01\\
81.53	0.01\\
81.54	0.01\\
81.55	0.01\\
81.56	0.01\\
81.57	0.01\\
81.58	0.01\\
81.59	0.01\\
81.6	0.01\\
81.61	0.01\\
81.62	0.01\\
81.63	0.01\\
81.64	0.01\\
81.65	0.01\\
81.66	0.01\\
81.67	0.01\\
81.68	0.01\\
81.69	0.01\\
81.7	0.01\\
81.71	0.01\\
81.72	0.01\\
81.73	0.01\\
81.74	0.01\\
81.75	0.01\\
81.76	0.01\\
81.77	0.01\\
81.78	0.01\\
81.79	0.01\\
81.8	0.01\\
81.81	0.01\\
81.82	0.01\\
81.83	0.01\\
81.84	0.01\\
81.85	0.01\\
81.86	0.01\\
81.87	0.01\\
81.88	0.01\\
81.89	0.01\\
81.9	0.01\\
81.91	0.01\\
81.92	0.01\\
81.93	0.01\\
81.94	0.01\\
81.95	0.01\\
81.96	0.01\\
81.97	0.01\\
81.98	0.01\\
81.99	0.01\\
82	0.01\\
82.01	0.01\\
82.02	0.01\\
82.03	0.01\\
82.04	0.01\\
82.05	0.01\\
82.06	0.01\\
82.07	0.01\\
82.08	0.01\\
82.09	0.01\\
82.1	0.01\\
82.11	0.01\\
82.12	0.01\\
82.13	0.01\\
82.14	0.01\\
82.15	0.01\\
82.16	0.01\\
82.17	0.01\\
82.18	0.01\\
82.19	0.01\\
82.2	0.01\\
82.21	0.01\\
82.22	0.01\\
82.23	0.01\\
82.24	0.01\\
82.25	0.01\\
82.26	0.01\\
82.27	0.01\\
82.28	0.01\\
82.29	0.01\\
82.3	0.01\\
82.31	0.01\\
82.32	0.01\\
82.33	0.01\\
82.34	0.01\\
82.35	0.01\\
82.36	0.01\\
82.37	0.01\\
82.38	0.01\\
82.39	0.01\\
82.4	0.01\\
82.41	0.01\\
82.42	0.01\\
82.43	0.01\\
82.44	0.01\\
82.45	0.01\\
82.46	0.01\\
82.47	0.01\\
82.48	0.01\\
82.49	0.01\\
82.5	0.01\\
82.51	0.01\\
82.52	0.01\\
82.53	0.01\\
82.54	0.01\\
82.55	0.01\\
82.56	0.01\\
82.57	0.01\\
82.58	0.01\\
82.59	0.01\\
82.6	0.01\\
82.61	0.01\\
82.62	0.01\\
82.63	0.01\\
82.64	0.01\\
82.65	0.01\\
82.66	0.01\\
82.67	0.01\\
82.68	0.01\\
82.69	0.01\\
82.7	0.01\\
82.71	0.01\\
82.72	0.01\\
82.73	0.01\\
82.74	0.01\\
82.75	0.01\\
82.76	0.01\\
82.77	0.01\\
82.78	0.01\\
82.79	0.01\\
82.8	0.01\\
82.81	0.01\\
82.82	0.01\\
82.83	0.01\\
82.84	0.01\\
82.85	0.01\\
82.86	0.01\\
82.87	0.01\\
82.88	0.01\\
82.89	0.01\\
82.9	0.01\\
82.91	0.01\\
82.92	0.01\\
82.93	0.01\\
82.94	0.01\\
82.95	0.01\\
82.96	0.01\\
82.97	0.01\\
82.98	0.01\\
82.99	0.01\\
83	0.01\\
83.01	0.01\\
83.02	0.01\\
83.03	0.01\\
83.04	0.01\\
83.05	0.01\\
83.06	0.01\\
83.07	0.01\\
83.08	0.01\\
83.09	0.01\\
83.1	0.01\\
83.11	0.01\\
83.12	0.01\\
83.13	0.01\\
83.14	0.01\\
83.15	0.01\\
83.16	0.01\\
83.17	0.01\\
83.18	0.01\\
83.19	0.01\\
83.2	0.01\\
83.21	0.01\\
83.22	0.01\\
83.23	0.01\\
83.24	0.01\\
83.25	0.01\\
83.26	0.01\\
83.27	0.01\\
83.28	0.01\\
83.29	0.01\\
83.3	0.01\\
83.31	0.01\\
83.32	0.01\\
83.33	0.01\\
83.34	0.01\\
83.35	0.01\\
83.36	0.01\\
83.37	0.01\\
83.38	0.01\\
83.39	0.01\\
83.4	0.01\\
83.41	0.01\\
83.42	0.01\\
83.43	0.01\\
83.44	0.01\\
83.45	0.01\\
83.46	0.01\\
83.47	0.01\\
83.48	0.01\\
83.49	0.01\\
83.5	0.01\\
83.51	0.01\\
83.52	0.01\\
83.53	0.01\\
83.54	0.01\\
83.55	0.01\\
83.56	0.01\\
83.57	0.01\\
83.58	0.01\\
83.59	0.01\\
83.6	0.01\\
83.61	0.01\\
83.62	0.01\\
83.63	0.01\\
83.64	0.01\\
83.65	0.01\\
83.66	0.01\\
83.67	0.01\\
83.68	0.01\\
83.69	0.01\\
83.7	0.01\\
83.71	0.01\\
83.72	0.01\\
83.73	0.01\\
83.74	0.01\\
83.75	0.01\\
83.76	0.01\\
83.77	0.01\\
83.78	0.01\\
83.79	0.01\\
83.8	0.01\\
83.81	0.01\\
83.82	0.01\\
83.83	0.01\\
83.84	0.01\\
83.85	0.01\\
83.86	0.01\\
83.87	0.01\\
83.88	0.01\\
83.89	0.01\\
83.9	0.01\\
83.91	0.01\\
83.92	0.01\\
83.93	0.01\\
83.94	0.01\\
83.95	0.01\\
83.96	0.01\\
83.97	0.01\\
83.98	0.01\\
83.99	0.01\\
84	0.01\\
84.01	0.01\\
84.02	0.01\\
84.03	0.01\\
84.04	0.01\\
84.05	0.01\\
84.06	0.01\\
84.07	0.01\\
84.08	0.01\\
84.09	0.01\\
84.1	0.01\\
84.11	0.01\\
84.12	0.01\\
84.13	0.01\\
84.14	0.01\\
84.15	0.01\\
84.16	0.01\\
84.17	0.01\\
84.18	0.01\\
84.19	0.01\\
84.2	0.01\\
84.21	0.01\\
84.22	0.01\\
84.23	0.01\\
84.24	0.01\\
84.25	0.01\\
84.26	0.01\\
84.27	0.01\\
84.28	0.01\\
84.29	0.01\\
84.3	0.01\\
84.31	0.01\\
84.32	0.01\\
84.33	0.01\\
84.34	0.01\\
84.35	0.01\\
84.36	0.01\\
84.37	0.01\\
84.38	0.01\\
84.39	0.01\\
84.4	0.01\\
84.41	0.01\\
84.42	0.01\\
84.43	0.01\\
84.44	0.01\\
84.45	0.01\\
84.46	0.01\\
84.47	0.01\\
84.48	0.01\\
84.49	0.01\\
84.5	0.01\\
84.51	0.01\\
84.52	0.01\\
84.53	0.01\\
84.54	0.01\\
84.55	0.01\\
84.56	0.01\\
84.57	0.01\\
84.58	0.01\\
84.59	0.01\\
84.6	0.01\\
84.61	0.01\\
84.62	0.01\\
84.63	0.01\\
84.64	0.01\\
84.65	0.01\\
84.66	0.01\\
84.67	0.01\\
84.68	0.01\\
84.69	0.01\\
84.7	0.01\\
84.71	0.01\\
84.72	0.01\\
84.73	0.01\\
84.74	0.01\\
84.75	0.01\\
84.76	0.01\\
84.77	0.01\\
84.78	0.01\\
84.79	0.01\\
84.8	0.01\\
84.81	0.01\\
84.82	0.01\\
84.83	0.01\\
84.84	0.01\\
84.85	0.01\\
84.86	0.01\\
84.87	0.01\\
84.88	0.01\\
84.89	0.01\\
84.9	0.01\\
84.91	0.01\\
84.92	0.01\\
84.93	0.01\\
84.94	0.01\\
84.95	0.01\\
84.96	0.01\\
84.97	0.01\\
84.98	0.01\\
84.99	0.01\\
85	0.01\\
85.01	0.01\\
85.02	0.01\\
85.03	0.01\\
85.04	0.01\\
85.05	0.01\\
85.06	0.01\\
85.07	0.01\\
85.08	0.01\\
85.09	0.01\\
85.1	0.01\\
85.11	0.01\\
85.12	0.01\\
85.13	0.01\\
85.14	0.01\\
85.15	0.01\\
85.16	0.01\\
85.17	0.01\\
85.18	0.01\\
85.19	0.01\\
85.2	0.01\\
85.21	0.01\\
85.22	0.01\\
85.23	0.01\\
85.24	0.01\\
85.25	0.01\\
85.26	0.01\\
85.27	0.01\\
85.28	0.01\\
85.29	0.01\\
85.3	0.01\\
85.31	0.01\\
85.32	0.01\\
85.33	0.01\\
85.34	0.01\\
85.35	0.01\\
85.36	0.01\\
85.37	0.01\\
85.38	0.01\\
85.39	0.01\\
85.4	0.01\\
85.41	0.01\\
85.42	0.01\\
85.43	0.01\\
85.44	0.01\\
85.45	0.01\\
85.46	0.01\\
85.47	0.01\\
85.48	0.01\\
85.49	0.01\\
85.5	0.01\\
85.51	0.01\\
85.52	0.01\\
85.53	0.01\\
85.54	0.01\\
85.55	0.01\\
85.56	0.01\\
85.57	0.01\\
85.58	0.01\\
85.59	0.01\\
85.6	0.01\\
85.61	0.01\\
85.62	0.01\\
85.63	0.01\\
85.64	0.01\\
85.65	0.01\\
85.66	0.01\\
85.67	0.01\\
85.68	0.01\\
85.69	0.01\\
85.7	0.01\\
85.71	0.01\\
85.72	0.01\\
85.73	0.01\\
85.74	0.01\\
85.75	0.01\\
85.76	0.01\\
85.77	0.01\\
85.78	0.01\\
85.79	0.01\\
85.8	0.01\\
85.81	0.01\\
85.82	0.01\\
85.83	0.01\\
85.84	0.01\\
85.85	0.01\\
85.86	0.01\\
85.87	0.01\\
85.88	0.01\\
85.89	0.01\\
85.9	0.01\\
85.91	0.01\\
85.92	0.01\\
85.93	0.01\\
85.94	0.01\\
85.95	0.01\\
85.96	0.01\\
85.97	0.01\\
85.98	0.01\\
85.99	0.01\\
86	0.01\\
86.01	0.01\\
86.02	0.01\\
86.03	0.01\\
86.04	0.01\\
86.05	0.01\\
86.06	0.01\\
86.07	0.01\\
86.08	0.01\\
86.09	0.01\\
86.1	0.01\\
86.11	0.01\\
86.12	0.01\\
86.13	0.01\\
86.14	0.01\\
86.15	0.01\\
86.16	0.01\\
86.17	0.01\\
86.18	0.01\\
86.19	0.01\\
86.2	0.01\\
86.21	0.01\\
86.22	0.01\\
86.23	0.01\\
86.24	0.01\\
86.25	0.01\\
86.26	0.01\\
86.27	0.01\\
86.28	0.01\\
86.29	0.01\\
86.3	0.01\\
86.31	0.01\\
86.32	0.01\\
86.33	0.01\\
86.34	0.01\\
86.35	0.01\\
86.36	0.01\\
86.37	0.01\\
86.38	0.01\\
86.39	0.01\\
86.4	0.01\\
86.41	0.01\\
86.42	0.01\\
86.43	0.01\\
86.44	0.01\\
86.45	0.01\\
86.46	0.01\\
86.47	0.01\\
86.48	0.01\\
86.49	0.01\\
86.5	0.01\\
86.51	0.01\\
86.52	0.01\\
86.53	0.01\\
86.54	0.01\\
86.55	0.01\\
86.56	0.01\\
86.57	0.01\\
86.58	0.01\\
86.59	0.01\\
86.6	0.01\\
86.61	0.01\\
86.62	0.01\\
86.63	0.01\\
86.64	0.01\\
86.65	0.01\\
86.66	0.01\\
86.67	0.01\\
86.68	0.01\\
86.69	0.01\\
86.7	0.01\\
86.71	0.01\\
86.72	0.01\\
86.73	0.01\\
86.74	0.01\\
86.75	0.01\\
86.76	0.01\\
86.77	0.01\\
86.78	0.01\\
86.79	0.01\\
86.8	0.01\\
86.81	0.01\\
86.82	0.01\\
86.83	0.01\\
86.84	0.01\\
86.85	0.01\\
86.86	0.01\\
86.87	0.01\\
86.88	0.01\\
86.89	0.01\\
86.9	0.01\\
86.91	0.01\\
86.92	0.01\\
86.93	0.01\\
86.94	0.01\\
86.95	0.01\\
86.96	0.01\\
86.97	0.01\\
86.98	0.01\\
86.99	0.01\\
87	0.01\\
87.01	0.01\\
87.02	0.01\\
87.03	0.01\\
87.04	0.01\\
87.05	0.01\\
87.06	0.01\\
87.07	0.01\\
87.08	0.01\\
87.09	0.01\\
87.1	0.01\\
87.11	0.01\\
87.12	0.01\\
87.13	0.01\\
87.14	0.01\\
87.15	0.01\\
87.16	0.01\\
87.17	0.01\\
87.18	0.01\\
87.19	0.01\\
87.2	0.01\\
87.21	0.01\\
87.22	0.01\\
87.23	0.01\\
87.24	0.01\\
87.25	0.01\\
87.26	0.01\\
87.27	0.01\\
87.28	0.01\\
87.29	0.01\\
87.3	0.01\\
87.31	0.01\\
87.32	0.01\\
87.33	0.01\\
87.34	0.01\\
87.35	0.01\\
87.36	0.01\\
87.37	0.01\\
87.38	0.01\\
87.39	0.01\\
87.4	0.01\\
87.41	0.01\\
87.42	0.01\\
87.43	0.01\\
87.44	0.01\\
87.45	0.01\\
87.46	0.01\\
87.47	0.01\\
87.48	0.01\\
87.49	0.01\\
87.5	0.01\\
87.51	0.01\\
87.52	0.01\\
87.53	0.01\\
87.54	0.01\\
87.55	0.01\\
87.56	0.01\\
87.57	0.01\\
87.58	0.01\\
87.59	0.01\\
87.6	0.01\\
87.61	0.01\\
87.62	0.01\\
87.63	0.01\\
87.64	0.01\\
87.65	0.01\\
87.66	0.01\\
87.67	0.01\\
87.68	0.01\\
87.69	0.01\\
87.7	0.01\\
87.71	0.01\\
87.72	0.01\\
87.73	0.01\\
87.74	0.01\\
87.75	0.01\\
87.76	0.01\\
87.77	0.01\\
87.78	0.01\\
87.79	0.01\\
87.8	0.01\\
87.81	0.01\\
87.82	0.01\\
87.83	0.01\\
87.84	0.01\\
87.85	0.01\\
87.86	0.01\\
87.87	0.01\\
87.88	0.01\\
87.89	0.01\\
87.9	0.01\\
87.91	0.01\\
87.92	0.01\\
87.93	0.01\\
87.94	0.01\\
87.95	0.01\\
87.96	0.01\\
87.97	0.01\\
87.98	0.01\\
87.99	0.01\\
88	0.01\\
88.01	0.01\\
88.02	0.01\\
88.03	0.01\\
88.04	0.01\\
88.05	0.01\\
88.06	0.01\\
88.07	0.01\\
88.08	0.01\\
88.09	0.01\\
88.1	0.01\\
88.11	0.01\\
88.12	0.01\\
88.13	0.01\\
88.14	0.01\\
88.15	0.01\\
88.16	0.01\\
88.17	0.01\\
88.18	0.01\\
88.19	0.01\\
88.2	0.01\\
88.21	0.01\\
88.22	0.01\\
88.23	0.01\\
88.24	0.01\\
88.25	0.01\\
88.26	0.01\\
88.27	0.01\\
88.28	0.01\\
88.29	0.01\\
88.3	0.01\\
88.31	0.01\\
88.32	0.01\\
88.33	0.01\\
88.34	0.01\\
88.35	0.01\\
88.36	0.01\\
88.37	0.01\\
88.38	0.01\\
88.39	0.01\\
88.4	0.01\\
88.41	0.01\\
88.42	0.01\\
88.43	0.01\\
88.44	0.01\\
88.45	0.01\\
88.46	0.01\\
88.47	0.01\\
88.48	0.01\\
88.49	0.01\\
88.5	0.01\\
88.51	0.01\\
88.52	0.01\\
88.53	0.01\\
88.54	0.01\\
88.55	0.01\\
88.56	0.01\\
88.57	0.01\\
88.58	0.01\\
88.59	0.01\\
88.6	0.01\\
88.61	0.01\\
88.62	0.01\\
88.63	0.01\\
88.64	0.01\\
88.65	0.01\\
88.66	0.01\\
88.67	0.01\\
88.68	0.01\\
88.69	0.01\\
88.7	0.01\\
88.71	0.01\\
88.72	0.01\\
88.73	0.01\\
88.74	0.01\\
88.75	0.01\\
88.76	0.01\\
88.77	0.01\\
88.78	0.01\\
88.79	0.01\\
88.8	0.01\\
88.81	0.01\\
88.82	0.01\\
88.83	0.01\\
88.84	0.01\\
88.85	0.01\\
88.86	0.01\\
88.87	0.01\\
88.88	0.01\\
88.89	0.01\\
88.9	0.01\\
88.91	0.01\\
88.92	0.01\\
88.93	0.01\\
88.94	0.01\\
88.95	0.01\\
88.96	0.01\\
88.97	0.01\\
88.98	0.01\\
88.99	0.01\\
89	0.01\\
89.01	0.01\\
89.02	0.01\\
89.03	0.01\\
89.04	0.01\\
89.05	0.01\\
89.06	0.01\\
89.07	0.01\\
89.08	0.01\\
89.09	0.01\\
89.1	0.01\\
89.11	0.01\\
89.12	0.01\\
89.13	0.01\\
89.14	0.01\\
89.15	0.01\\
89.16	0.01\\
89.17	0.01\\
89.18	0.01\\
89.19	0.01\\
89.2	0.01\\
89.21	0.01\\
89.22	0.01\\
89.23	0.01\\
89.24	0.01\\
89.25	0.01\\
89.26	0.01\\
89.27	0.01\\
89.28	0.01\\
89.29	0.01\\
89.3	0.01\\
89.31	0.01\\
89.32	0.01\\
89.33	0.01\\
89.34	0.01\\
89.35	0.01\\
89.36	0.01\\
89.37	0.01\\
89.38	0.01\\
89.39	0.01\\
89.4	0.01\\
89.41	0.01\\
89.42	0.01\\
89.43	0.01\\
89.44	0.01\\
89.45	0.01\\
89.46	0.01\\
89.47	0.01\\
89.48	0.01\\
89.49	0.01\\
89.5	0.01\\
89.51	0.01\\
89.52	0.01\\
89.53	0.01\\
89.54	0.01\\
89.55	0.01\\
89.56	0.01\\
89.57	0.01\\
89.58	0.01\\
89.59	0.01\\
89.6	0.01\\
89.61	0.01\\
89.62	0.01\\
89.63	0.01\\
89.64	0.01\\
89.65	0.01\\
89.66	0.01\\
89.67	0.01\\
89.68	0.01\\
89.69	0.01\\
89.7	0.01\\
89.71	0.01\\
89.72	0.01\\
89.73	0.01\\
89.74	0.01\\
89.75	0.01\\
89.76	0.01\\
89.77	0.01\\
89.78	0.01\\
89.79	0.01\\
89.8	0.01\\
89.81	0.01\\
89.82	0.01\\
89.83	0.01\\
89.84	0.01\\
89.85	0.01\\
89.86	0.01\\
89.87	0.01\\
89.88	0.01\\
89.89	0.01\\
89.9	0.01\\
89.91	0.01\\
89.92	0.01\\
89.93	0.01\\
89.94	0.01\\
89.95	0.01\\
89.96	0.01\\
89.97	0.01\\
89.98	0.01\\
89.99	0.01\\
90	0.01\\
90.01	0.01\\
90.02	0.01\\
90.03	0.01\\
90.04	0.01\\
90.05	0.01\\
90.06	0.01\\
90.07	0.01\\
90.08	0.01\\
90.09	0.01\\
90.1	0.01\\
90.11	0.01\\
90.12	0.01\\
90.13	0.01\\
90.14	0.01\\
90.15	0.01\\
90.16	0.01\\
90.17	0.01\\
90.18	0.01\\
90.19	0.01\\
90.2	0.01\\
90.21	0.01\\
90.22	0.01\\
90.23	0.01\\
90.24	0.01\\
90.25	0.01\\
90.26	0.01\\
90.27	0.01\\
90.28	0.01\\
90.29	0.01\\
90.3	0.01\\
90.31	0.01\\
90.32	0.01\\
90.33	0.01\\
90.34	0.01\\
90.35	0.01\\
90.36	0.01\\
90.37	0.01\\
90.38	0.01\\
90.39	0.01\\
90.4	0.01\\
90.41	0.01\\
90.42	0.01\\
90.43	0.01\\
90.44	0.01\\
90.45	0.01\\
90.46	0.01\\
90.47	0.01\\
90.48	0.01\\
90.49	0.01\\
90.5	0.01\\
90.51	0.01\\
90.52	0.01\\
90.53	0.01\\
90.54	0.01\\
90.55	0.01\\
90.56	0.01\\
90.57	0.01\\
90.58	0.01\\
90.59	0.01\\
90.6	0.01\\
90.61	0.01\\
90.62	0.01\\
90.63	0.01\\
90.64	0.01\\
90.65	0.01\\
90.66	0.01\\
90.67	0.01\\
90.68	0.01\\
90.69	0.01\\
90.7	0.01\\
90.71	0.01\\
90.72	0.01\\
90.73	0.01\\
90.74	0.01\\
90.75	0.01\\
90.76	0.01\\
90.77	0.01\\
90.78	0.01\\
90.79	0.01\\
90.8	0.01\\
90.81	0.01\\
90.82	0.01\\
90.83	0.01\\
90.84	0.01\\
90.85	0.01\\
90.86	0.01\\
90.87	0.01\\
90.88	0.01\\
90.89	0.01\\
90.9	0.01\\
90.91	0.01\\
90.92	0.01\\
90.93	0.01\\
90.94	0.01\\
90.95	0.01\\
90.96	0.01\\
90.97	0.01\\
90.98	0.01\\
90.99	0.01\\
91	0.01\\
91.01	0.01\\
91.02	0.01\\
91.03	0.01\\
91.04	0.01\\
91.05	0.01\\
91.06	0.01\\
91.07	0.01\\
91.08	0.01\\
91.09	0.01\\
91.1	0.01\\
91.11	0.01\\
91.12	0.01\\
91.13	0.01\\
91.14	0.01\\
91.15	0.01\\
91.16	0.01\\
91.17	0.01\\
91.18	0.01\\
91.19	0.01\\
91.2	0.01\\
91.21	0.01\\
91.22	0.01\\
91.23	0.01\\
91.24	0.01\\
91.25	0.01\\
91.26	0.01\\
91.27	0.01\\
91.28	0.01\\
91.29	0.01\\
91.3	0.01\\
91.31	0.01\\
91.32	0.01\\
91.33	0.01\\
91.34	0.01\\
91.35	0.01\\
91.36	0.01\\
91.37	0.01\\
91.38	0.01\\
91.39	0.01\\
91.4	0.01\\
91.41	0.01\\
91.42	0.01\\
91.43	0.01\\
91.44	0.01\\
91.45	0.01\\
91.46	0.01\\
91.47	0.01\\
91.48	0.01\\
91.49	0.01\\
91.5	0.01\\
91.51	0.01\\
91.52	0.01\\
91.53	0.01\\
91.54	0.01\\
91.55	0.01\\
91.56	0.01\\
91.57	0.01\\
91.58	0.01\\
91.59	0.01\\
91.6	0.01\\
91.61	0.01\\
91.62	0.01\\
91.63	0.01\\
91.64	0.01\\
91.65	0.01\\
91.66	0.01\\
91.67	0.01\\
91.68	0.01\\
91.69	0.01\\
91.7	0.01\\
91.71	0.01\\
91.72	0.01\\
91.73	0.01\\
91.74	0.01\\
91.75	0.01\\
91.76	0.01\\
91.77	0.01\\
91.78	0.01\\
91.79	0.01\\
91.8	0.01\\
91.81	0.01\\
91.82	0.01\\
91.83	0.01\\
91.84	0.01\\
91.85	0.01\\
91.86	0.01\\
91.87	0.01\\
91.88	0.01\\
91.89	0.01\\
91.9	0.01\\
91.91	0.01\\
91.92	0.01\\
91.93	0.01\\
91.94	0.01\\
91.95	0.01\\
91.96	0.01\\
91.97	0.01\\
91.98	0.01\\
91.99	0.01\\
92	0.01\\
92.01	0.01\\
92.02	0.01\\
92.03	0.01\\
92.04	0.01\\
92.05	0.01\\
92.06	0.01\\
92.07	0.01\\
92.08	0.01\\
92.09	0.01\\
92.1	0.01\\
92.11	0.01\\
92.12	0.01\\
92.13	0.01\\
92.14	0.01\\
92.15	0.01\\
92.16	0.01\\
92.17	0.01\\
92.18	0.01\\
92.19	0.01\\
92.2	0.01\\
92.21	0.01\\
92.22	0.01\\
92.23	0.01\\
92.24	0.01\\
92.25	0.01\\
92.26	0.01\\
92.27	0.01\\
92.28	0.01\\
92.29	0.01\\
92.3	0.01\\
92.31	0.01\\
92.32	0.01\\
92.33	0.01\\
92.34	0.01\\
92.35	0.01\\
92.36	0.01\\
92.37	0.01\\
92.38	0.01\\
92.39	0.01\\
92.4	0.01\\
92.41	0.01\\
92.42	0.01\\
92.43	0.01\\
92.44	0.01\\
92.45	0.01\\
92.46	0.01\\
92.47	0.01\\
92.48	0.01\\
92.49	0.01\\
92.5	0.01\\
92.51	0.01\\
92.52	0.01\\
92.53	0.01\\
92.54	0.01\\
92.55	0.01\\
92.56	0.01\\
92.57	0.01\\
92.58	0.01\\
92.59	0.01\\
92.6	0.01\\
92.61	0.01\\
92.62	0.01\\
92.63	0.01\\
92.64	0.01\\
92.65	0.01\\
92.66	0.01\\
92.67	0.01\\
92.68	0.01\\
92.69	0.01\\
92.7	0.01\\
92.71	0.01\\
92.72	0.01\\
92.73	0.01\\
92.74	0.01\\
92.75	0.01\\
92.76	0.01\\
92.77	0.01\\
92.78	0.01\\
92.79	0.01\\
92.8	0.01\\
92.81	0.01\\
92.82	0.01\\
92.83	0.01\\
92.84	0.01\\
92.85	0.01\\
92.86	0.01\\
92.87	0.01\\
92.88	0.01\\
92.89	0.01\\
92.9	0.01\\
92.91	0.01\\
92.92	0.01\\
92.93	0.01\\
92.94	0.01\\
92.95	0.01\\
92.96	0.01\\
92.97	0.01\\
92.98	0.01\\
92.99	0.01\\
93	0.01\\
93.01	0.01\\
93.02	0.01\\
93.03	0.01\\
93.04	0.01\\
93.05	0.01\\
93.06	0.01\\
93.07	0.01\\
93.08	0.01\\
93.09	0.01\\
93.1	0.01\\
93.11	0.01\\
93.12	0.01\\
93.13	0.01\\
93.14	0.01\\
93.15	0.01\\
93.16	0.01\\
93.17	0.01\\
93.18	0.01\\
93.19	0.01\\
93.2	0.01\\
93.21	0.01\\
93.22	0.01\\
93.23	0.01\\
93.24	0.01\\
93.25	0.01\\
93.26	0.01\\
93.27	0.01\\
93.28	0.01\\
93.29	0.01\\
93.3	0.01\\
93.31	0.01\\
93.32	0.01\\
93.33	0.01\\
93.34	0.01\\
93.35	0.01\\
93.36	0.01\\
93.37	0.01\\
93.38	0.01\\
93.39	0.01\\
93.4	0.01\\
93.41	0.01\\
93.42	0.01\\
93.43	0.01\\
93.44	0.01\\
93.45	0.01\\
93.46	0.01\\
93.47	0.01\\
93.48	0.01\\
93.49	0.01\\
93.5	0.01\\
93.51	0.01\\
93.52	0.01\\
93.53	0.01\\
93.54	0.01\\
93.55	0.01\\
93.56	0.01\\
93.57	0.01\\
93.58	0.01\\
93.59	0.01\\
93.6	0.01\\
93.61	0.01\\
93.62	0.01\\
93.63	0.01\\
93.64	0.01\\
93.65	0.01\\
93.66	0.01\\
93.67	0.01\\
93.68	0.01\\
93.69	0.01\\
93.7	0.01\\
93.71	0.01\\
93.72	0.01\\
93.73	0.01\\
93.74	0.01\\
93.75	0.01\\
93.76	0.01\\
93.77	0.01\\
93.78	0.01\\
93.79	0.01\\
93.8	0.01\\
93.81	0.01\\
93.82	0.01\\
93.83	0.01\\
93.84	0.01\\
93.85	0.01\\
93.86	0.01\\
93.87	0.01\\
93.88	0.01\\
93.89	0.01\\
93.9	0.01\\
93.91	0.01\\
93.92	0.01\\
93.93	0.01\\
93.94	0.01\\
93.95	0.01\\
93.96	0.01\\
93.97	0.01\\
93.98	0.01\\
93.99	0.01\\
94	0.01\\
94.01	0.01\\
94.02	0.01\\
94.03	0.01\\
94.04	0.01\\
94.05	0.01\\
94.06	0.01\\
94.07	0.01\\
94.08	0.01\\
94.09	0.01\\
94.1	0.01\\
94.11	0.01\\
94.12	0.01\\
94.13	0.01\\
94.14	0.01\\
94.15	0.01\\
94.16	0.01\\
94.17	0.01\\
94.18	0.01\\
94.19	0.01\\
94.2	0.01\\
94.21	0.01\\
94.22	0.01\\
94.23	0.01\\
94.24	0.01\\
94.25	0.01\\
94.26	0.01\\
94.27	0.01\\
94.28	0.01\\
94.29	0.01\\
94.3	0.01\\
94.31	0.01\\
94.32	0.01\\
94.33	0.01\\
94.34	0.01\\
94.35	0.01\\
94.36	0.01\\
94.37	0.01\\
94.38	0.01\\
94.39	0.01\\
94.4	0.01\\
94.41	0.01\\
94.42	0.01\\
94.43	0.01\\
94.44	0.01\\
94.45	0.01\\
94.46	0.01\\
94.47	0.01\\
94.48	0.01\\
94.49	0.01\\
94.5	0.01\\
94.51	0.01\\
94.52	0.01\\
94.53	0.01\\
94.54	0.01\\
94.55	0.01\\
94.56	0.01\\
94.57	0.01\\
94.58	0.01\\
94.59	0.01\\
94.6	0.01\\
94.61	0.01\\
94.62	0.01\\
94.63	0.01\\
94.64	0.01\\
94.65	0.01\\
94.66	0.01\\
94.67	0.01\\
94.68	0.01\\
94.69	0.01\\
94.7	0.01\\
94.71	0.01\\
94.72	0.01\\
94.73	0.01\\
94.74	0.01\\
94.75	0.01\\
94.76	0.01\\
94.77	0.01\\
94.78	0.01\\
94.79	0.01\\
94.8	0.01\\
94.81	0.01\\
94.82	0.01\\
94.83	0.01\\
94.84	0.01\\
94.85	0.01\\
94.86	0.01\\
94.87	0.01\\
94.88	0.01\\
94.89	0.01\\
94.9	0.01\\
94.91	0.01\\
94.92	0.01\\
94.93	0.01\\
94.94	0.01\\
94.95	0.01\\
94.96	0.01\\
94.97	0.01\\
94.98	0.01\\
94.99	0.01\\
95	0.01\\
95.01	0.01\\
95.02	0.01\\
95.03	0.01\\
95.04	0.01\\
95.05	0.01\\
95.06	0.01\\
95.07	0.01\\
95.08	0.01\\
95.09	0.01\\
95.1	0.01\\
95.11	0.01\\
95.12	0.01\\
95.13	0.01\\
95.14	0.01\\
95.15	0.01\\
95.16	0.01\\
95.17	0.01\\
95.18	0.01\\
95.19	0.01\\
95.2	0.01\\
95.21	0.01\\
95.22	0.01\\
95.23	0.01\\
95.24	0.01\\
95.25	0.01\\
95.26	0.01\\
95.27	0.01\\
95.28	0.01\\
95.29	0.01\\
95.3	0.01\\
95.31	0.01\\
95.32	0.01\\
95.33	0.01\\
95.34	0.01\\
95.35	0.01\\
95.36	0.01\\
95.37	0.01\\
95.38	0.01\\
95.39	0.01\\
95.4	0.01\\
95.41	0.01\\
95.42	0.01\\
95.43	0.01\\
95.44	0.01\\
95.45	0.01\\
95.46	0.01\\
95.47	0.01\\
95.48	0.01\\
95.49	0.01\\
95.5	0.01\\
95.51	0.01\\
95.52	0.01\\
95.53	0.01\\
95.54	0.01\\
95.55	0.01\\
95.56	0.01\\
95.57	0.01\\
95.58	0.01\\
95.59	0.01\\
95.6	0.01\\
95.61	0.01\\
95.62	0.01\\
95.63	0.01\\
95.64	0.01\\
95.65	0.01\\
95.66	0.01\\
95.67	0.01\\
95.68	0.01\\
95.69	0.01\\
95.7	0.01\\
95.71	0.01\\
95.72	0.01\\
95.73	0.01\\
95.74	0.01\\
95.75	0.01\\
95.76	0.01\\
95.77	0.01\\
95.78	0.01\\
95.79	0.01\\
95.8	0.01\\
95.81	0.01\\
95.82	0.01\\
95.83	0.01\\
95.84	0.01\\
95.85	0.01\\
95.86	0.01\\
95.87	0.01\\
95.88	0.01\\
95.89	0.01\\
95.9	0.01\\
95.91	0.01\\
95.92	0.01\\
95.93	0.01\\
95.94	0.01\\
95.95	0.01\\
95.96	0.01\\
95.97	0.01\\
95.98	0.01\\
95.99	0.01\\
96	0.01\\
96.01	0.01\\
96.02	0.01\\
96.03	0.01\\
96.04	0.01\\
96.05	0.01\\
96.06	0.01\\
96.07	0.01\\
96.08	0.01\\
96.09	0.01\\
96.1	0.01\\
96.11	0.01\\
96.12	0.01\\
96.13	0.01\\
96.14	0.01\\
96.15	0.01\\
96.16	0.01\\
96.17	0.01\\
96.18	0.01\\
96.19	0.01\\
96.2	0.01\\
96.21	0.01\\
96.22	0.01\\
96.23	0.01\\
96.24	0.01\\
96.25	0.01\\
96.26	0.01\\
96.27	0.01\\
96.28	0.01\\
96.29	0.01\\
96.3	0.01\\
96.31	0.01\\
96.32	0.01\\
96.33	0.01\\
96.34	0.01\\
96.35	0.01\\
96.36	0.01\\
96.37	0.01\\
96.38	0.01\\
96.39	0.01\\
96.4	0.01\\
96.41	0.01\\
96.42	0.01\\
96.43	0.01\\
96.44	0.01\\
96.45	0.01\\
96.46	0.01\\
96.47	0.01\\
96.48	0.01\\
96.49	0.01\\
96.5	0.01\\
96.51	0.01\\
96.52	0.01\\
96.53	0.01\\
96.54	0.01\\
96.55	0.01\\
96.56	0.01\\
96.57	0.01\\
96.58	0.01\\
96.59	0.01\\
96.6	0.01\\
96.61	0.01\\
96.62	0.01\\
96.63	0.01\\
96.64	0.01\\
96.65	0.01\\
96.66	0.01\\
96.67	0.01\\
96.68	0.01\\
96.69	0.01\\
96.7	0.01\\
96.71	0.01\\
96.72	0.01\\
96.73	0.01\\
96.74	0.01\\
96.75	0.01\\
96.76	0.01\\
96.77	0.01\\
96.78	0.01\\
96.79	0.01\\
96.8	0.01\\
96.81	0.01\\
96.82	0.01\\
96.83	0.01\\
96.84	0.01\\
96.85	0.01\\
96.86	0.01\\
96.87	0.01\\
96.88	0.01\\
96.89	0.01\\
96.9	0.01\\
96.91	0.01\\
96.92	0.01\\
96.93	0.01\\
96.94	0.01\\
96.95	0.01\\
96.96	0.01\\
96.97	0.01\\
96.98	0.01\\
96.99	0.01\\
97	0.01\\
97.01	0.01\\
97.02	0.01\\
97.03	0.01\\
97.04	0.01\\
97.05	0.01\\
97.06	0.01\\
97.07	0.01\\
97.08	0.01\\
97.09	0.01\\
97.1	0.01\\
97.11	0.01\\
97.12	0.01\\
97.13	0.01\\
97.14	0.01\\
97.15	0.01\\
97.16	0.01\\
97.17	0.01\\
97.18	0.01\\
97.19	0.01\\
97.2	0.01\\
97.21	0.01\\
97.22	0.01\\
97.23	0.01\\
97.24	0.01\\
97.25	0.01\\
97.26	0.01\\
97.27	0.01\\
97.28	0.01\\
97.29	0.01\\
97.3	0.01\\
97.31	0.01\\
97.32	0.01\\
97.33	0.01\\
97.34	0.01\\
97.35	0.01\\
97.36	0.01\\
97.37	0.01\\
97.38	0.01\\
97.39	0.01\\
97.4	0.01\\
97.41	0.01\\
97.42	0.01\\
97.43	0.01\\
97.44	0.01\\
97.45	0.01\\
97.46	0.01\\
97.47	0.01\\
97.48	0.01\\
97.49	0.01\\
97.5	0.01\\
97.51	0.01\\
97.52	0.01\\
97.53	0.01\\
97.54	0.01\\
97.55	0.01\\
97.56	0.01\\
97.57	0.01\\
97.58	0.01\\
97.59	0.01\\
97.6	0.01\\
97.61	0.01\\
97.62	0.01\\
97.63	0.01\\
97.64	0.01\\
97.65	0.01\\
97.66	0.01\\
97.67	0.01\\
97.68	0.01\\
97.69	0.01\\
97.7	0.01\\
97.71	0.01\\
97.72	0.01\\
97.73	0.01\\
97.74	0.01\\
97.75	0.01\\
97.76	0.01\\
97.77	0.01\\
97.78	0.01\\
97.79	0.01\\
97.8	0.01\\
97.81	0.01\\
97.82	0.01\\
97.83	0.01\\
97.84	0.01\\
97.85	0.01\\
97.86	0.01\\
97.87	0.01\\
97.88	0.01\\
97.89	0.01\\
97.9	0.01\\
97.91	0.01\\
97.92	0.01\\
97.93	0.01\\
97.94	0.01\\
97.95	0.01\\
97.96	0.01\\
97.97	0.01\\
97.98	0.01\\
97.99	0.01\\
98	0.01\\
98.01	0.01\\
98.02	0.01\\
98.03	0.01\\
98.04	0.01\\
98.05	0.01\\
98.06	0.01\\
98.07	0.01\\
98.08	0.01\\
98.09	0.01\\
98.1	0.01\\
98.11	0.01\\
98.12	0.01\\
98.13	0.01\\
98.14	0.01\\
98.15	0.01\\
98.16	0.01\\
98.17	0.01\\
98.18	0.01\\
98.19	0.01\\
98.2	0.01\\
98.21	0.01\\
98.22	0.01\\
98.23	0.01\\
98.24	0.01\\
98.25	0.01\\
98.26	0.01\\
98.27	0.01\\
98.28	0.01\\
98.29	0.01\\
98.3	0.01\\
98.31	0.01\\
98.32	0.01\\
98.33	0.01\\
98.34	0.01\\
98.35	0.01\\
98.36	0.01\\
98.37	0.01\\
98.38	0.01\\
98.39	0.01\\
98.4	0.01\\
98.41	0.01\\
98.42	0.01\\
98.43	0.01\\
98.44	0.01\\
98.45	0.01\\
98.46	0.01\\
98.47	0.01\\
98.48	0.01\\
98.49	0.01\\
98.5	0.01\\
98.51	0.01\\
98.52	0.01\\
98.53	0.01\\
98.54	0.01\\
98.55	0.01\\
98.56	0.01\\
98.57	0.01\\
98.58	0.01\\
98.59	0.01\\
98.6	0.01\\
98.61	0.01\\
98.62	0.01\\
98.63	0.01\\
98.64	0.01\\
98.65	0.01\\
98.66	0.01\\
98.67	0.01\\
98.68	0.01\\
98.69	0.01\\
98.7	0.01\\
98.71	0.01\\
98.72	0.01\\
98.73	0.01\\
98.74	0.01\\
98.75	0.01\\
98.76	0.01\\
98.77	0.01\\
98.78	0.01\\
98.79	0.01\\
98.8	0.01\\
98.81	0.01\\
98.82	0.01\\
98.83	0.0098426521778193\\
98.84	0.00962536896190331\\
98.85	0.00940640932468859\\
98.86	0.00918574768776908\\
98.87	0.00896335769774644\\
98.88	0.00873921220349199\\
98.89	0.00851328321738781\\
98.9	0.00828554188254797\\
98.91	0.00805597995566326\\
98.92	0.0078246095817055\\
98.93	0.00759140113879057\\
98.94	0.00735632404879756\\
98.95	0.00711934673653893\\
98.96	0.00693395505590613\\
98.97	0.00688987882381456\\
98.98	0.00684559155289971\\
98.99	0.0068011007461696\\
99	0.0067564144209328\\
99.01	0.00671154109295583\\
99.02	0.00666648979566538\\
99.03	0.00662127010377893\\
99.04	0.00657589218059403\\
99.05	0.00653030932509738\\
99.06	0.00648453023571555\\
99.07	0.00643856490098297\\
99.08	0.00639242392068952\\
99.09	0.00634611853474451\\
99.1	0.00629966065349513\\
99.11	0.00625306288962129\\
99.12	0.00620633859169888\\
99.13	0.00615950187952902\\
99.14	0.0061125419965859\\
99.15	0.00606516434576647\\
99.16	0.00601736505574436\\
99.17	0.00596914019869767\\
99.18	0.0059204857881827\\
99.19	0.00587139777688372\\
99.2	0.00582187205430372\\
99.21	0.00577190444433596\\
99.22	0.00572149075281801\\
99.23	0.00567062689318552\\
99.24	0.0056193087241921\\
99.25	0.00556753204802381\\
99.26	0.00551529260831485\\
99.27	0.00546258608805913\\
99.28	0.00540940810741223\\
99.29	0.0053557542213778\\
99.3	0.00530161991737217\\
99.31	0.00524700061266076\\
99.32	0.00519189165165917\\
99.33	0.00513628830309197\\
99.34	0.00508018575700118\\
99.35	0.0050235791215973\\
99.36	0.00496646341994358\\
99.37	0.00490883358644935\\
99.38	0.0048506844630757\\
99.39	0.00479201079544656\\
99.4	0.00473280722875311\\
99.41	0.00467306830343948\\
99.42	0.00461278915159383\\
99.43	0.00455196486264461\\
99.44	0.00449059048163354\\
99.45	0.00442866100885806\\
99.46	0.0043661713995154\\
99.47	0.00430311656334906\\
99.48	0.0042394913639357\\
99.49	0.00417529061784133\\
99.5	0.00411050909416695\\
99.51	0.00404514151408925\\
99.52	0.00397918255039652\\
99.53	0.00391262682701961\\
99.54	0.00384546891855804\\
99.55	0.00377770334980126\\
99.56	0.00370932459524504\\
99.57	0.00364032707860304\\
99.58	0.00357070517231364\\
99.59	0.0035004531970421\\
99.6	0.00342956542117795\\
99.61	0.00335803606032796\\
99.62	0.0032858592768046\\
99.63	0.00321302917911018\\
99.64	0.00313953982141679\\
99.65	0.00306538520304226\\
99.66	0.0029905592679222\\
99.67	0.00291505590407856\\
99.68	0.00283886894308471\\
99.69	0.00276199217558516\\
99.7	0.00268441933711856\\
99.71	0.00260614410548349\\
99.72	0.0025271601001977\\
99.73	0.0024474608819558\\
99.74	0.00236703995208607\\
99.75	0.00228589075200679\\
99.76	0.00220400666268282\\
99.77	0.00212138100408299\\
99.78	0.00203800703463929\\
99.79	0.00195387795070856\\
99.8	0.00186898688603761\\
99.81	0.00178332691123306\\
99.82	0.00169689103323675\\
99.83	0.00160967219480833\\
99.84	0.00152166327401621\\
99.85	0.00143285708373862\\
99.86	0.00134324637117649\\
99.87	0.00125282381737996\\
99.88	0.00116158203679085\\
99.89	0.00106951357680327\\
99.9	0.00097661091734502\\
99.91	0.000882866470482559\\
99.92	0.000788272580052774\\
99.93	0.000692821521324887\\
99.94	0.000596505500696339\\
99.95	0.000499316655426806\\
99.96	0.000401247053414959\\
99.97	0.000302288693022996\\
99.98	0.000202433502954543\\
99.99	0.00010167334219198\\
100	0\\
};
\addlegendentry{$q=-4$};

\addplot [color=mycolor1,dashed,forget plot]
  table[row sep=crcr]{%
0.01	0.01\\
0.02	0.01\\
0.03	0.01\\
0.04	0.01\\
0.05	0.01\\
0.06	0.01\\
0.07	0.01\\
0.08	0.01\\
0.09	0.01\\
0.1	0.01\\
0.11	0.01\\
0.12	0.01\\
0.13	0.01\\
0.14	0.01\\
0.15	0.01\\
0.16	0.01\\
0.17	0.01\\
0.18	0.01\\
0.19	0.01\\
0.2	0.01\\
0.21	0.01\\
0.22	0.01\\
0.23	0.01\\
0.24	0.01\\
0.25	0.01\\
0.26	0.01\\
0.27	0.01\\
0.28	0.01\\
0.29	0.01\\
0.3	0.01\\
0.31	0.01\\
0.32	0.01\\
0.33	0.01\\
0.34	0.01\\
0.35	0.01\\
0.36	0.01\\
0.37	0.01\\
0.38	0.01\\
0.39	0.01\\
0.4	0.01\\
0.41	0.01\\
0.42	0.01\\
0.43	0.01\\
0.44	0.01\\
0.45	0.01\\
0.46	0.01\\
0.47	0.01\\
0.48	0.01\\
0.49	0.01\\
0.5	0.01\\
0.51	0.01\\
0.52	0.01\\
0.53	0.01\\
0.54	0.01\\
0.55	0.01\\
0.56	0.01\\
0.57	0.01\\
0.58	0.01\\
0.59	0.01\\
0.6	0.01\\
0.61	0.01\\
0.62	0.01\\
0.63	0.01\\
0.64	0.01\\
0.65	0.01\\
0.66	0.01\\
0.67	0.01\\
0.68	0.01\\
0.69	0.01\\
0.7	0.01\\
0.71	0.01\\
0.72	0.01\\
0.73	0.01\\
0.74	0.01\\
0.75	0.01\\
0.76	0.01\\
0.77	0.01\\
0.78	0.01\\
0.79	0.01\\
0.8	0.01\\
0.81	0.01\\
0.82	0.01\\
0.83	0.01\\
0.84	0.01\\
0.85	0.01\\
0.86	0.01\\
0.87	0.01\\
0.88	0.01\\
0.89	0.01\\
0.9	0.01\\
0.91	0.01\\
0.92	0.01\\
0.93	0.01\\
0.94	0.01\\
0.95	0.01\\
0.96	0.01\\
0.97	0.01\\
0.98	0.01\\
0.99	0.01\\
1	0.01\\
1.01	0.01\\
1.02	0.01\\
1.03	0.01\\
1.04	0.01\\
1.05	0.01\\
1.06	0.01\\
1.07	0.01\\
1.08	0.01\\
1.09	0.01\\
1.1	0.01\\
1.11	0.01\\
1.12	0.01\\
1.13	0.01\\
1.14	0.01\\
1.15	0.01\\
1.16	0.01\\
1.17	0.01\\
1.18	0.01\\
1.19	0.01\\
1.2	0.01\\
1.21	0.01\\
1.22	0.01\\
1.23	0.01\\
1.24	0.01\\
1.25	0.01\\
1.26	0.01\\
1.27	0.01\\
1.28	0.01\\
1.29	0.01\\
1.3	0.01\\
1.31	0.01\\
1.32	0.01\\
1.33	0.01\\
1.34	0.01\\
1.35	0.01\\
1.36	0.01\\
1.37	0.01\\
1.38	0.01\\
1.39	0.01\\
1.4	0.01\\
1.41	0.01\\
1.42	0.01\\
1.43	0.01\\
1.44	0.01\\
1.45	0.01\\
1.46	0.01\\
1.47	0.01\\
1.48	0.01\\
1.49	0.01\\
1.5	0.01\\
1.51	0.01\\
1.52	0.01\\
1.53	0.01\\
1.54	0.01\\
1.55	0.01\\
1.56	0.01\\
1.57	0.01\\
1.58	0.01\\
1.59	0.01\\
1.6	0.01\\
1.61	0.01\\
1.62	0.01\\
1.63	0.01\\
1.64	0.01\\
1.65	0.01\\
1.66	0.01\\
1.67	0.01\\
1.68	0.01\\
1.69	0.01\\
1.7	0.01\\
1.71	0.01\\
1.72	0.01\\
1.73	0.01\\
1.74	0.01\\
1.75	0.01\\
1.76	0.01\\
1.77	0.01\\
1.78	0.01\\
1.79	0.01\\
1.8	0.01\\
1.81	0.01\\
1.82	0.01\\
1.83	0.01\\
1.84	0.01\\
1.85	0.01\\
1.86	0.01\\
1.87	0.01\\
1.88	0.01\\
1.89	0.01\\
1.9	0.01\\
1.91	0.01\\
1.92	0.01\\
1.93	0.01\\
1.94	0.01\\
1.95	0.01\\
1.96	0.01\\
1.97	0.01\\
1.98	0.01\\
1.99	0.01\\
2	0.01\\
2.01	0.01\\
2.02	0.01\\
2.03	0.01\\
2.04	0.01\\
2.05	0.01\\
2.06	0.01\\
2.07	0.01\\
2.08	0.01\\
2.09	0.01\\
2.1	0.01\\
2.11	0.01\\
2.12	0.01\\
2.13	0.01\\
2.14	0.01\\
2.15	0.01\\
2.16	0.01\\
2.17	0.01\\
2.18	0.01\\
2.19	0.01\\
2.2	0.01\\
2.21	0.01\\
2.22	0.01\\
2.23	0.01\\
2.24	0.01\\
2.25	0.01\\
2.26	0.01\\
2.27	0.01\\
2.28	0.01\\
2.29	0.01\\
2.3	0.01\\
2.31	0.01\\
2.32	0.01\\
2.33	0.01\\
2.34	0.01\\
2.35	0.01\\
2.36	0.01\\
2.37	0.01\\
2.38	0.01\\
2.39	0.01\\
2.4	0.01\\
2.41	0.01\\
2.42	0.01\\
2.43	0.01\\
2.44	0.01\\
2.45	0.01\\
2.46	0.01\\
2.47	0.01\\
2.48	0.01\\
2.49	0.01\\
2.5	0.01\\
2.51	0.01\\
2.52	0.01\\
2.53	0.01\\
2.54	0.01\\
2.55	0.01\\
2.56	0.01\\
2.57	0.01\\
2.58	0.01\\
2.59	0.01\\
2.6	0.01\\
2.61	0.01\\
2.62	0.01\\
2.63	0.01\\
2.64	0.01\\
2.65	0.01\\
2.66	0.01\\
2.67	0.01\\
2.68	0.01\\
2.69	0.01\\
2.7	0.01\\
2.71	0.01\\
2.72	0.01\\
2.73	0.01\\
2.74	0.01\\
2.75	0.01\\
2.76	0.01\\
2.77	0.01\\
2.78	0.01\\
2.79	0.01\\
2.8	0.01\\
2.81	0.01\\
2.82	0.01\\
2.83	0.01\\
2.84	0.01\\
2.85	0.01\\
2.86	0.01\\
2.87	0.01\\
2.88	0.01\\
2.89	0.01\\
2.9	0.01\\
2.91	0.01\\
2.92	0.01\\
2.93	0.01\\
2.94	0.01\\
2.95	0.01\\
2.96	0.01\\
2.97	0.01\\
2.98	0.01\\
2.99	0.01\\
3	0.01\\
3.01	0.01\\
3.02	0.01\\
3.03	0.01\\
3.04	0.01\\
3.05	0.01\\
3.06	0.01\\
3.07	0.01\\
3.08	0.01\\
3.09	0.01\\
3.1	0.01\\
3.11	0.01\\
3.12	0.01\\
3.13	0.01\\
3.14	0.01\\
3.15	0.01\\
3.16	0.01\\
3.17	0.01\\
3.18	0.01\\
3.19	0.01\\
3.2	0.01\\
3.21	0.01\\
3.22	0.01\\
3.23	0.01\\
3.24	0.01\\
3.25	0.01\\
3.26	0.01\\
3.27	0.01\\
3.28	0.01\\
3.29	0.01\\
3.3	0.01\\
3.31	0.01\\
3.32	0.01\\
3.33	0.01\\
3.34	0.01\\
3.35	0.01\\
3.36	0.01\\
3.37	0.01\\
3.38	0.01\\
3.39	0.01\\
3.4	0.01\\
3.41	0.01\\
3.42	0.01\\
3.43	0.01\\
3.44	0.01\\
3.45	0.01\\
3.46	0.01\\
3.47	0.01\\
3.48	0.01\\
3.49	0.01\\
3.5	0.01\\
3.51	0.01\\
3.52	0.01\\
3.53	0.01\\
3.54	0.01\\
3.55	0.01\\
3.56	0.01\\
3.57	0.01\\
3.58	0.01\\
3.59	0.01\\
3.6	0.01\\
3.61	0.01\\
3.62	0.01\\
3.63	0.01\\
3.64	0.01\\
3.65	0.01\\
3.66	0.01\\
3.67	0.01\\
3.68	0.01\\
3.69	0.01\\
3.7	0.01\\
3.71	0.01\\
3.72	0.01\\
3.73	0.01\\
3.74	0.01\\
3.75	0.01\\
3.76	0.01\\
3.77	0.01\\
3.78	0.01\\
3.79	0.01\\
3.8	0.01\\
3.81	0.01\\
3.82	0.01\\
3.83	0.01\\
3.84	0.01\\
3.85	0.01\\
3.86	0.01\\
3.87	0.01\\
3.88	0.01\\
3.89	0.01\\
3.9	0.01\\
3.91	0.01\\
3.92	0.01\\
3.93	0.01\\
3.94	0.01\\
3.95	0.01\\
3.96	0.01\\
3.97	0.01\\
3.98	0.01\\
3.99	0.01\\
4	0.01\\
4.01	0.01\\
4.02	0.01\\
4.03	0.01\\
4.04	0.01\\
4.05	0.01\\
4.06	0.01\\
4.07	0.01\\
4.08	0.01\\
4.09	0.01\\
4.1	0.01\\
4.11	0.01\\
4.12	0.01\\
4.13	0.01\\
4.14	0.01\\
4.15	0.01\\
4.16	0.01\\
4.17	0.01\\
4.18	0.01\\
4.19	0.01\\
4.2	0.01\\
4.21	0.01\\
4.22	0.01\\
4.23	0.01\\
4.24	0.01\\
4.25	0.01\\
4.26	0.01\\
4.27	0.01\\
4.28	0.01\\
4.29	0.01\\
4.3	0.01\\
4.31	0.01\\
4.32	0.01\\
4.33	0.01\\
4.34	0.01\\
4.35	0.01\\
4.36	0.01\\
4.37	0.01\\
4.38	0.01\\
4.39	0.01\\
4.4	0.01\\
4.41	0.01\\
4.42	0.01\\
4.43	0.01\\
4.44	0.01\\
4.45	0.01\\
4.46	0.01\\
4.47	0.01\\
4.48	0.01\\
4.49	0.01\\
4.5	0.01\\
4.51	0.01\\
4.52	0.01\\
4.53	0.01\\
4.54	0.01\\
4.55	0.01\\
4.56	0.01\\
4.57	0.01\\
4.58	0.01\\
4.59	0.01\\
4.6	0.01\\
4.61	0.01\\
4.62	0.01\\
4.63	0.01\\
4.64	0.01\\
4.65	0.01\\
4.66	0.01\\
4.67	0.01\\
4.68	0.01\\
4.69	0.01\\
4.7	0.01\\
4.71	0.01\\
4.72	0.01\\
4.73	0.01\\
4.74	0.01\\
4.75	0.01\\
4.76	0.01\\
4.77	0.01\\
4.78	0.01\\
4.79	0.01\\
4.8	0.01\\
4.81	0.01\\
4.82	0.01\\
4.83	0.01\\
4.84	0.01\\
4.85	0.01\\
4.86	0.01\\
4.87	0.01\\
4.88	0.01\\
4.89	0.01\\
4.9	0.01\\
4.91	0.01\\
4.92	0.01\\
4.93	0.01\\
4.94	0.01\\
4.95	0.01\\
4.96	0.01\\
4.97	0.01\\
4.98	0.01\\
4.99	0.01\\
5	0.01\\
5.01	0.01\\
5.02	0.01\\
5.03	0.01\\
5.04	0.01\\
5.05	0.01\\
5.06	0.01\\
5.07	0.01\\
5.08	0.01\\
5.09	0.01\\
5.1	0.01\\
5.11	0.01\\
5.12	0.01\\
5.13	0.01\\
5.14	0.01\\
5.15	0.01\\
5.16	0.01\\
5.17	0.01\\
5.18	0.01\\
5.19	0.01\\
5.2	0.01\\
5.21	0.01\\
5.22	0.01\\
5.23	0.01\\
5.24	0.01\\
5.25	0.01\\
5.26	0.01\\
5.27	0.01\\
5.28	0.01\\
5.29	0.01\\
5.3	0.01\\
5.31	0.01\\
5.32	0.01\\
5.33	0.01\\
5.34	0.01\\
5.35	0.01\\
5.36	0.01\\
5.37	0.01\\
5.38	0.01\\
5.39	0.01\\
5.4	0.01\\
5.41	0.01\\
5.42	0.01\\
5.43	0.01\\
5.44	0.01\\
5.45	0.01\\
5.46	0.01\\
5.47	0.01\\
5.48	0.01\\
5.49	0.01\\
5.5	0.01\\
5.51	0.01\\
5.52	0.01\\
5.53	0.01\\
5.54	0.01\\
5.55	0.01\\
5.56	0.01\\
5.57	0.01\\
5.58	0.01\\
5.59	0.01\\
5.6	0.01\\
5.61	0.01\\
5.62	0.01\\
5.63	0.01\\
5.64	0.01\\
5.65	0.01\\
5.66	0.01\\
5.67	0.01\\
5.68	0.01\\
5.69	0.01\\
5.7	0.01\\
5.71	0.01\\
5.72	0.01\\
5.73	0.01\\
5.74	0.01\\
5.75	0.01\\
5.76	0.01\\
5.77	0.01\\
5.78	0.01\\
5.79	0.01\\
5.8	0.01\\
5.81	0.01\\
5.82	0.01\\
5.83	0.01\\
5.84	0.01\\
5.85	0.01\\
5.86	0.01\\
5.87	0.01\\
5.88	0.01\\
5.89	0.01\\
5.9	0.01\\
5.91	0.01\\
5.92	0.01\\
5.93	0.01\\
5.94	0.01\\
5.95	0.01\\
5.96	0.01\\
5.97	0.01\\
5.98	0.01\\
5.99	0.01\\
6	0.01\\
6.01	0.01\\
6.02	0.01\\
6.03	0.01\\
6.04	0.01\\
6.05	0.01\\
6.06	0.01\\
6.07	0.01\\
6.08	0.01\\
6.09	0.01\\
6.1	0.01\\
6.11	0.01\\
6.12	0.01\\
6.13	0.01\\
6.14	0.01\\
6.15	0.01\\
6.16	0.01\\
6.17	0.01\\
6.18	0.01\\
6.19	0.01\\
6.2	0.01\\
6.21	0.01\\
6.22	0.01\\
6.23	0.01\\
6.24	0.01\\
6.25	0.01\\
6.26	0.01\\
6.27	0.01\\
6.28	0.01\\
6.29	0.01\\
6.3	0.01\\
6.31	0.01\\
6.32	0.01\\
6.33	0.01\\
6.34	0.01\\
6.35	0.01\\
6.36	0.01\\
6.37	0.01\\
6.38	0.01\\
6.39	0.01\\
6.4	0.01\\
6.41	0.01\\
6.42	0.01\\
6.43	0.01\\
6.44	0.01\\
6.45	0.01\\
6.46	0.01\\
6.47	0.01\\
6.48	0.01\\
6.49	0.01\\
6.5	0.01\\
6.51	0.01\\
6.52	0.01\\
6.53	0.01\\
6.54	0.01\\
6.55	0.01\\
6.56	0.01\\
6.57	0.01\\
6.58	0.01\\
6.59	0.01\\
6.6	0.01\\
6.61	0.01\\
6.62	0.01\\
6.63	0.01\\
6.64	0.01\\
6.65	0.01\\
6.66	0.01\\
6.67	0.01\\
6.68	0.01\\
6.69	0.01\\
6.7	0.01\\
6.71	0.01\\
6.72	0.01\\
6.73	0.01\\
6.74	0.01\\
6.75	0.01\\
6.76	0.01\\
6.77	0.01\\
6.78	0.01\\
6.79	0.01\\
6.8	0.01\\
6.81	0.01\\
6.82	0.01\\
6.83	0.01\\
6.84	0.01\\
6.85	0.01\\
6.86	0.01\\
6.87	0.01\\
6.88	0.01\\
6.89	0.01\\
6.9	0.01\\
6.91	0.01\\
6.92	0.01\\
6.93	0.01\\
6.94	0.01\\
6.95	0.01\\
6.96	0.01\\
6.97	0.01\\
6.98	0.01\\
6.99	0.01\\
7	0.01\\
7.01	0.01\\
7.02	0.01\\
7.03	0.01\\
7.04	0.01\\
7.05	0.01\\
7.06	0.01\\
7.07	0.01\\
7.08	0.01\\
7.09	0.01\\
7.1	0.01\\
7.11	0.01\\
7.12	0.01\\
7.13	0.01\\
7.14	0.01\\
7.15	0.01\\
7.16	0.01\\
7.17	0.01\\
7.18	0.01\\
7.19	0.01\\
7.2	0.01\\
7.21	0.01\\
7.22	0.01\\
7.23	0.01\\
7.24	0.01\\
7.25	0.01\\
7.26	0.01\\
7.27	0.01\\
7.28	0.01\\
7.29	0.01\\
7.3	0.01\\
7.31	0.01\\
7.32	0.01\\
7.33	0.01\\
7.34	0.01\\
7.35	0.01\\
7.36	0.01\\
7.37	0.01\\
7.38	0.01\\
7.39	0.01\\
7.4	0.01\\
7.41	0.01\\
7.42	0.01\\
7.43	0.01\\
7.44	0.01\\
7.45	0.01\\
7.46	0.01\\
7.47	0.01\\
7.48	0.01\\
7.49	0.01\\
7.5	0.01\\
7.51	0.01\\
7.52	0.01\\
7.53	0.01\\
7.54	0.01\\
7.55	0.01\\
7.56	0.01\\
7.57	0.01\\
7.58	0.01\\
7.59	0.01\\
7.6	0.01\\
7.61	0.01\\
7.62	0.01\\
7.63	0.01\\
7.64	0.01\\
7.65	0.01\\
7.66	0.01\\
7.67	0.01\\
7.68	0.01\\
7.69	0.01\\
7.7	0.01\\
7.71	0.01\\
7.72	0.01\\
7.73	0.01\\
7.74	0.01\\
7.75	0.01\\
7.76	0.01\\
7.77	0.01\\
7.78	0.01\\
7.79	0.01\\
7.8	0.01\\
7.81	0.01\\
7.82	0.01\\
7.83	0.01\\
7.84	0.01\\
7.85	0.01\\
7.86	0.01\\
7.87	0.01\\
7.88	0.01\\
7.89	0.01\\
7.9	0.01\\
7.91	0.01\\
7.92	0.01\\
7.93	0.01\\
7.94	0.01\\
7.95	0.01\\
7.96	0.01\\
7.97	0.01\\
7.98	0.01\\
7.99	0.01\\
8	0.01\\
8.01	0.01\\
8.02	0.01\\
8.03	0.01\\
8.04	0.01\\
8.05	0.01\\
8.06	0.01\\
8.07	0.01\\
8.08	0.01\\
8.09	0.01\\
8.1	0.01\\
8.11	0.01\\
8.12	0.01\\
8.13	0.01\\
8.14	0.01\\
8.15	0.01\\
8.16	0.01\\
8.17	0.01\\
8.18	0.01\\
8.19	0.01\\
8.2	0.01\\
8.21	0.01\\
8.22	0.01\\
8.23	0.01\\
8.24	0.01\\
8.25	0.01\\
8.26	0.01\\
8.27	0.01\\
8.28	0.01\\
8.29	0.01\\
8.3	0.01\\
8.31	0.01\\
8.32	0.01\\
8.33	0.01\\
8.34	0.01\\
8.35	0.01\\
8.36	0.01\\
8.37	0.01\\
8.38	0.01\\
8.39	0.01\\
8.4	0.01\\
8.41	0.01\\
8.42	0.01\\
8.43	0.01\\
8.44	0.01\\
8.45	0.01\\
8.46	0.01\\
8.47	0.01\\
8.48	0.01\\
8.49	0.01\\
8.5	0.01\\
8.51	0.01\\
8.52	0.01\\
8.53	0.01\\
8.54	0.01\\
8.55	0.01\\
8.56	0.01\\
8.57	0.01\\
8.58	0.01\\
8.59	0.01\\
8.6	0.01\\
8.61	0.01\\
8.62	0.01\\
8.63	0.01\\
8.64	0.01\\
8.65	0.01\\
8.66	0.01\\
8.67	0.01\\
8.68	0.01\\
8.69	0.01\\
8.7	0.01\\
8.71	0.01\\
8.72	0.01\\
8.73	0.01\\
8.74	0.01\\
8.75	0.01\\
8.76	0.01\\
8.77	0.01\\
8.78	0.01\\
8.79	0.01\\
8.8	0.01\\
8.81	0.01\\
8.82	0.01\\
8.83	0.01\\
8.84	0.01\\
8.85	0.01\\
8.86	0.01\\
8.87	0.01\\
8.88	0.01\\
8.89	0.01\\
8.9	0.01\\
8.91	0.01\\
8.92	0.01\\
8.93	0.01\\
8.94	0.01\\
8.95	0.01\\
8.96	0.01\\
8.97	0.01\\
8.98	0.01\\
8.99	0.01\\
9	0.01\\
9.01	0.01\\
9.02	0.01\\
9.03	0.01\\
9.04	0.01\\
9.05	0.01\\
9.06	0.01\\
9.07	0.01\\
9.08	0.01\\
9.09	0.01\\
9.1	0.01\\
9.11	0.01\\
9.12	0.01\\
9.13	0.01\\
9.14	0.01\\
9.15	0.01\\
9.16	0.01\\
9.17	0.01\\
9.18	0.01\\
9.19	0.01\\
9.2	0.01\\
9.21	0.01\\
9.22	0.01\\
9.23	0.01\\
9.24	0.01\\
9.25	0.01\\
9.26	0.01\\
9.27	0.01\\
9.28	0.01\\
9.29	0.01\\
9.3	0.01\\
9.31	0.01\\
9.32	0.01\\
9.33	0.01\\
9.34	0.01\\
9.35	0.01\\
9.36	0.01\\
9.37	0.01\\
9.38	0.01\\
9.39	0.01\\
9.4	0.01\\
9.41	0.01\\
9.42	0.01\\
9.43	0.01\\
9.44	0.01\\
9.45	0.01\\
9.46	0.01\\
9.47	0.01\\
9.48	0.01\\
9.49	0.01\\
9.5	0.01\\
9.51	0.01\\
9.52	0.01\\
9.53	0.01\\
9.54	0.01\\
9.55	0.01\\
9.56	0.01\\
9.57	0.01\\
9.58	0.01\\
9.59	0.01\\
9.6	0.01\\
9.61	0.01\\
9.62	0.01\\
9.63	0.01\\
9.64	0.01\\
9.65	0.01\\
9.66	0.01\\
9.67	0.01\\
9.68	0.01\\
9.69	0.01\\
9.7	0.01\\
9.71	0.01\\
9.72	0.01\\
9.73	0.01\\
9.74	0.01\\
9.75	0.01\\
9.76	0.01\\
9.77	0.01\\
9.78	0.01\\
9.79	0.01\\
9.8	0.01\\
9.81	0.01\\
9.82	0.01\\
9.83	0.01\\
9.84	0.01\\
9.85	0.01\\
9.86	0.01\\
9.87	0.01\\
9.88	0.01\\
9.89	0.01\\
9.9	0.01\\
9.91	0.01\\
9.92	0.01\\
9.93	0.01\\
9.94	0.01\\
9.95	0.01\\
9.96	0.01\\
9.97	0.01\\
9.98	0.01\\
9.99	0.01\\
10	0.01\\
10.01	0.01\\
10.02	0.01\\
10.03	0.01\\
10.04	0.01\\
10.05	0.01\\
10.06	0.01\\
10.07	0.01\\
10.08	0.01\\
10.09	0.01\\
10.1	0.01\\
10.11	0.01\\
10.12	0.01\\
10.13	0.01\\
10.14	0.01\\
10.15	0.01\\
10.16	0.01\\
10.17	0.01\\
10.18	0.01\\
10.19	0.01\\
10.2	0.01\\
10.21	0.01\\
10.22	0.01\\
10.23	0.01\\
10.24	0.01\\
10.25	0.01\\
10.26	0.01\\
10.27	0.01\\
10.28	0.01\\
10.29	0.01\\
10.3	0.01\\
10.31	0.01\\
10.32	0.01\\
10.33	0.01\\
10.34	0.01\\
10.35	0.01\\
10.36	0.01\\
10.37	0.01\\
10.38	0.01\\
10.39	0.01\\
10.4	0.01\\
10.41	0.01\\
10.42	0.01\\
10.43	0.01\\
10.44	0.01\\
10.45	0.01\\
10.46	0.01\\
10.47	0.01\\
10.48	0.01\\
10.49	0.01\\
10.5	0.01\\
10.51	0.01\\
10.52	0.01\\
10.53	0.01\\
10.54	0.01\\
10.55	0.01\\
10.56	0.01\\
10.57	0.01\\
10.58	0.01\\
10.59	0.01\\
10.6	0.01\\
10.61	0.01\\
10.62	0.01\\
10.63	0.01\\
10.64	0.01\\
10.65	0.01\\
10.66	0.01\\
10.67	0.01\\
10.68	0.01\\
10.69	0.01\\
10.7	0.01\\
10.71	0.01\\
10.72	0.01\\
10.73	0.01\\
10.74	0.01\\
10.75	0.01\\
10.76	0.01\\
10.77	0.01\\
10.78	0.01\\
10.79	0.01\\
10.8	0.01\\
10.81	0.01\\
10.82	0.01\\
10.83	0.01\\
10.84	0.01\\
10.85	0.01\\
10.86	0.01\\
10.87	0.01\\
10.88	0.01\\
10.89	0.01\\
10.9	0.01\\
10.91	0.01\\
10.92	0.01\\
10.93	0.01\\
10.94	0.01\\
10.95	0.01\\
10.96	0.01\\
10.97	0.01\\
10.98	0.01\\
10.99	0.01\\
11	0.01\\
11.01	0.01\\
11.02	0.01\\
11.03	0.01\\
11.04	0.01\\
11.05	0.01\\
11.06	0.01\\
11.07	0.01\\
11.08	0.01\\
11.09	0.01\\
11.1	0.01\\
11.11	0.01\\
11.12	0.01\\
11.13	0.01\\
11.14	0.01\\
11.15	0.01\\
11.16	0.01\\
11.17	0.01\\
11.18	0.01\\
11.19	0.01\\
11.2	0.01\\
11.21	0.01\\
11.22	0.01\\
11.23	0.01\\
11.24	0.01\\
11.25	0.01\\
11.26	0.01\\
11.27	0.01\\
11.28	0.01\\
11.29	0.01\\
11.3	0.01\\
11.31	0.01\\
11.32	0.01\\
11.33	0.01\\
11.34	0.01\\
11.35	0.01\\
11.36	0.01\\
11.37	0.01\\
11.38	0.01\\
11.39	0.01\\
11.4	0.01\\
11.41	0.01\\
11.42	0.01\\
11.43	0.01\\
11.44	0.01\\
11.45	0.01\\
11.46	0.01\\
11.47	0.01\\
11.48	0.01\\
11.49	0.01\\
11.5	0.01\\
11.51	0.01\\
11.52	0.01\\
11.53	0.01\\
11.54	0.01\\
11.55	0.01\\
11.56	0.01\\
11.57	0.01\\
11.58	0.01\\
11.59	0.01\\
11.6	0.01\\
11.61	0.01\\
11.62	0.01\\
11.63	0.01\\
11.64	0.01\\
11.65	0.01\\
11.66	0.01\\
11.67	0.01\\
11.68	0.01\\
11.69	0.01\\
11.7	0.01\\
11.71	0.01\\
11.72	0.01\\
11.73	0.01\\
11.74	0.01\\
11.75	0.01\\
11.76	0.01\\
11.77	0.01\\
11.78	0.01\\
11.79	0.01\\
11.8	0.01\\
11.81	0.01\\
11.82	0.01\\
11.83	0.01\\
11.84	0.01\\
11.85	0.01\\
11.86	0.01\\
11.87	0.01\\
11.88	0.01\\
11.89	0.01\\
11.9	0.01\\
11.91	0.01\\
11.92	0.01\\
11.93	0.01\\
11.94	0.01\\
11.95	0.01\\
11.96	0.01\\
11.97	0.01\\
11.98	0.01\\
11.99	0.01\\
12	0.01\\
12.01	0.01\\
12.02	0.01\\
12.03	0.01\\
12.04	0.01\\
12.05	0.01\\
12.06	0.01\\
12.07	0.01\\
12.08	0.01\\
12.09	0.01\\
12.1	0.01\\
12.11	0.01\\
12.12	0.01\\
12.13	0.01\\
12.14	0.01\\
12.15	0.01\\
12.16	0.01\\
12.17	0.01\\
12.18	0.01\\
12.19	0.01\\
12.2	0.01\\
12.21	0.01\\
12.22	0.01\\
12.23	0.01\\
12.24	0.01\\
12.25	0.01\\
12.26	0.01\\
12.27	0.01\\
12.28	0.01\\
12.29	0.01\\
12.3	0.01\\
12.31	0.01\\
12.32	0.01\\
12.33	0.01\\
12.34	0.01\\
12.35	0.01\\
12.36	0.01\\
12.37	0.01\\
12.38	0.01\\
12.39	0.01\\
12.4	0.01\\
12.41	0.01\\
12.42	0.01\\
12.43	0.01\\
12.44	0.01\\
12.45	0.01\\
12.46	0.01\\
12.47	0.01\\
12.48	0.01\\
12.49	0.01\\
12.5	0.01\\
12.51	0.01\\
12.52	0.01\\
12.53	0.01\\
12.54	0.01\\
12.55	0.01\\
12.56	0.01\\
12.57	0.01\\
12.58	0.01\\
12.59	0.01\\
12.6	0.01\\
12.61	0.01\\
12.62	0.01\\
12.63	0.01\\
12.64	0.01\\
12.65	0.01\\
12.66	0.01\\
12.67	0.01\\
12.68	0.01\\
12.69	0.01\\
12.7	0.01\\
12.71	0.01\\
12.72	0.01\\
12.73	0.01\\
12.74	0.01\\
12.75	0.01\\
12.76	0.01\\
12.77	0.01\\
12.78	0.01\\
12.79	0.01\\
12.8	0.01\\
12.81	0.01\\
12.82	0.01\\
12.83	0.01\\
12.84	0.01\\
12.85	0.01\\
12.86	0.01\\
12.87	0.01\\
12.88	0.01\\
12.89	0.01\\
12.9	0.01\\
12.91	0.01\\
12.92	0.01\\
12.93	0.01\\
12.94	0.01\\
12.95	0.01\\
12.96	0.01\\
12.97	0.01\\
12.98	0.01\\
12.99	0.01\\
13	0.01\\
13.01	0.01\\
13.02	0.01\\
13.03	0.01\\
13.04	0.01\\
13.05	0.01\\
13.06	0.01\\
13.07	0.01\\
13.08	0.01\\
13.09	0.01\\
13.1	0.01\\
13.11	0.01\\
13.12	0.01\\
13.13	0.01\\
13.14	0.01\\
13.15	0.01\\
13.16	0.01\\
13.17	0.01\\
13.18	0.01\\
13.19	0.01\\
13.2	0.01\\
13.21	0.01\\
13.22	0.01\\
13.23	0.01\\
13.24	0.01\\
13.25	0.01\\
13.26	0.01\\
13.27	0.01\\
13.28	0.01\\
13.29	0.01\\
13.3	0.01\\
13.31	0.01\\
13.32	0.01\\
13.33	0.01\\
13.34	0.01\\
13.35	0.01\\
13.36	0.01\\
13.37	0.01\\
13.38	0.01\\
13.39	0.01\\
13.4	0.01\\
13.41	0.01\\
13.42	0.01\\
13.43	0.01\\
13.44	0.01\\
13.45	0.01\\
13.46	0.01\\
13.47	0.01\\
13.48	0.01\\
13.49	0.01\\
13.5	0.01\\
13.51	0.01\\
13.52	0.01\\
13.53	0.01\\
13.54	0.01\\
13.55	0.01\\
13.56	0.01\\
13.57	0.01\\
13.58	0.01\\
13.59	0.01\\
13.6	0.01\\
13.61	0.01\\
13.62	0.01\\
13.63	0.01\\
13.64	0.01\\
13.65	0.01\\
13.66	0.01\\
13.67	0.01\\
13.68	0.01\\
13.69	0.01\\
13.7	0.01\\
13.71	0.01\\
13.72	0.01\\
13.73	0.01\\
13.74	0.01\\
13.75	0.01\\
13.76	0.01\\
13.77	0.01\\
13.78	0.01\\
13.79	0.01\\
13.8	0.01\\
13.81	0.01\\
13.82	0.01\\
13.83	0.01\\
13.84	0.01\\
13.85	0.01\\
13.86	0.01\\
13.87	0.01\\
13.88	0.01\\
13.89	0.01\\
13.9	0.01\\
13.91	0.01\\
13.92	0.01\\
13.93	0.01\\
13.94	0.01\\
13.95	0.01\\
13.96	0.01\\
13.97	0.01\\
13.98	0.01\\
13.99	0.01\\
14	0.01\\
14.01	0.01\\
14.02	0.01\\
14.03	0.01\\
14.04	0.01\\
14.05	0.01\\
14.06	0.01\\
14.07	0.01\\
14.08	0.01\\
14.09	0.01\\
14.1	0.01\\
14.11	0.01\\
14.12	0.01\\
14.13	0.01\\
14.14	0.01\\
14.15	0.01\\
14.16	0.01\\
14.17	0.01\\
14.18	0.01\\
14.19	0.01\\
14.2	0.01\\
14.21	0.01\\
14.22	0.01\\
14.23	0.01\\
14.24	0.01\\
14.25	0.01\\
14.26	0.01\\
14.27	0.01\\
14.28	0.01\\
14.29	0.01\\
14.3	0.01\\
14.31	0.01\\
14.32	0.01\\
14.33	0.01\\
14.34	0.01\\
14.35	0.01\\
14.36	0.01\\
14.37	0.01\\
14.38	0.01\\
14.39	0.01\\
14.4	0.01\\
14.41	0.01\\
14.42	0.01\\
14.43	0.01\\
14.44	0.01\\
14.45	0.01\\
14.46	0.01\\
14.47	0.01\\
14.48	0.01\\
14.49	0.01\\
14.5	0.01\\
14.51	0.01\\
14.52	0.01\\
14.53	0.01\\
14.54	0.01\\
14.55	0.01\\
14.56	0.01\\
14.57	0.01\\
14.58	0.01\\
14.59	0.01\\
14.6	0.01\\
14.61	0.01\\
14.62	0.01\\
14.63	0.01\\
14.64	0.01\\
14.65	0.01\\
14.66	0.01\\
14.67	0.01\\
14.68	0.01\\
14.69	0.01\\
14.7	0.01\\
14.71	0.01\\
14.72	0.01\\
14.73	0.01\\
14.74	0.01\\
14.75	0.01\\
14.76	0.01\\
14.77	0.01\\
14.78	0.01\\
14.79	0.01\\
14.8	0.01\\
14.81	0.01\\
14.82	0.01\\
14.83	0.01\\
14.84	0.01\\
14.85	0.01\\
14.86	0.01\\
14.87	0.01\\
14.88	0.01\\
14.89	0.01\\
14.9	0.01\\
14.91	0.01\\
14.92	0.01\\
14.93	0.01\\
14.94	0.01\\
14.95	0.01\\
14.96	0.01\\
14.97	0.01\\
14.98	0.01\\
14.99	0.01\\
15	0.01\\
15.01	0.01\\
15.02	0.01\\
15.03	0.01\\
15.04	0.01\\
15.05	0.01\\
15.06	0.01\\
15.07	0.01\\
15.08	0.01\\
15.09	0.01\\
15.1	0.01\\
15.11	0.01\\
15.12	0.01\\
15.13	0.01\\
15.14	0.01\\
15.15	0.01\\
15.16	0.01\\
15.17	0.01\\
15.18	0.01\\
15.19	0.01\\
15.2	0.01\\
15.21	0.01\\
15.22	0.01\\
15.23	0.01\\
15.24	0.01\\
15.25	0.01\\
15.26	0.01\\
15.27	0.01\\
15.28	0.01\\
15.29	0.01\\
15.3	0.01\\
15.31	0.01\\
15.32	0.01\\
15.33	0.01\\
15.34	0.01\\
15.35	0.01\\
15.36	0.01\\
15.37	0.01\\
15.38	0.01\\
15.39	0.01\\
15.4	0.01\\
15.41	0.01\\
15.42	0.01\\
15.43	0.01\\
15.44	0.01\\
15.45	0.01\\
15.46	0.01\\
15.47	0.01\\
15.48	0.01\\
15.49	0.01\\
15.5	0.01\\
15.51	0.01\\
15.52	0.01\\
15.53	0.01\\
15.54	0.01\\
15.55	0.01\\
15.56	0.01\\
15.57	0.01\\
15.58	0.01\\
15.59	0.01\\
15.6	0.01\\
15.61	0.01\\
15.62	0.01\\
15.63	0.01\\
15.64	0.01\\
15.65	0.01\\
15.66	0.01\\
15.67	0.01\\
15.68	0.01\\
15.69	0.01\\
15.7	0.01\\
15.71	0.01\\
15.72	0.01\\
15.73	0.01\\
15.74	0.01\\
15.75	0.01\\
15.76	0.01\\
15.77	0.01\\
15.78	0.01\\
15.79	0.01\\
15.8	0.01\\
15.81	0.01\\
15.82	0.01\\
15.83	0.01\\
15.84	0.01\\
15.85	0.01\\
15.86	0.01\\
15.87	0.01\\
15.88	0.01\\
15.89	0.01\\
15.9	0.01\\
15.91	0.01\\
15.92	0.01\\
15.93	0.01\\
15.94	0.01\\
15.95	0.01\\
15.96	0.01\\
15.97	0.01\\
15.98	0.01\\
15.99	0.01\\
16	0.01\\
16.01	0.01\\
16.02	0.01\\
16.03	0.01\\
16.04	0.01\\
16.05	0.01\\
16.06	0.01\\
16.07	0.01\\
16.08	0.01\\
16.09	0.01\\
16.1	0.01\\
16.11	0.01\\
16.12	0.01\\
16.13	0.01\\
16.14	0.01\\
16.15	0.01\\
16.16	0.01\\
16.17	0.01\\
16.18	0.01\\
16.19	0.01\\
16.2	0.01\\
16.21	0.01\\
16.22	0.01\\
16.23	0.01\\
16.24	0.01\\
16.25	0.01\\
16.26	0.01\\
16.27	0.01\\
16.28	0.01\\
16.29	0.01\\
16.3	0.01\\
16.31	0.01\\
16.32	0.01\\
16.33	0.01\\
16.34	0.01\\
16.35	0.01\\
16.36	0.01\\
16.37	0.01\\
16.38	0.01\\
16.39	0.01\\
16.4	0.01\\
16.41	0.01\\
16.42	0.01\\
16.43	0.01\\
16.44	0.01\\
16.45	0.01\\
16.46	0.01\\
16.47	0.01\\
16.48	0.01\\
16.49	0.01\\
16.5	0.01\\
16.51	0.01\\
16.52	0.01\\
16.53	0.01\\
16.54	0.01\\
16.55	0.01\\
16.56	0.01\\
16.57	0.01\\
16.58	0.01\\
16.59	0.01\\
16.6	0.01\\
16.61	0.01\\
16.62	0.01\\
16.63	0.01\\
16.64	0.01\\
16.65	0.01\\
16.66	0.01\\
16.67	0.01\\
16.68	0.01\\
16.69	0.01\\
16.7	0.01\\
16.71	0.01\\
16.72	0.01\\
16.73	0.01\\
16.74	0.01\\
16.75	0.01\\
16.76	0.01\\
16.77	0.01\\
16.78	0.01\\
16.79	0.01\\
16.8	0.01\\
16.81	0.01\\
16.82	0.01\\
16.83	0.01\\
16.84	0.01\\
16.85	0.01\\
16.86	0.01\\
16.87	0.01\\
16.88	0.01\\
16.89	0.01\\
16.9	0.01\\
16.91	0.01\\
16.92	0.01\\
16.93	0.01\\
16.94	0.01\\
16.95	0.01\\
16.96	0.01\\
16.97	0.01\\
16.98	0.01\\
16.99	0.01\\
17	0.01\\
17.01	0.01\\
17.02	0.01\\
17.03	0.01\\
17.04	0.01\\
17.05	0.01\\
17.06	0.01\\
17.07	0.01\\
17.08	0.01\\
17.09	0.01\\
17.1	0.01\\
17.11	0.01\\
17.12	0.01\\
17.13	0.01\\
17.14	0.01\\
17.15	0.01\\
17.16	0.01\\
17.17	0.01\\
17.18	0.01\\
17.19	0.01\\
17.2	0.01\\
17.21	0.01\\
17.22	0.01\\
17.23	0.01\\
17.24	0.01\\
17.25	0.01\\
17.26	0.01\\
17.27	0.01\\
17.28	0.01\\
17.29	0.01\\
17.3	0.01\\
17.31	0.01\\
17.32	0.01\\
17.33	0.01\\
17.34	0.01\\
17.35	0.01\\
17.36	0.01\\
17.37	0.01\\
17.38	0.01\\
17.39	0.01\\
17.4	0.01\\
17.41	0.01\\
17.42	0.01\\
17.43	0.01\\
17.44	0.01\\
17.45	0.01\\
17.46	0.01\\
17.47	0.01\\
17.48	0.01\\
17.49	0.01\\
17.5	0.01\\
17.51	0.01\\
17.52	0.01\\
17.53	0.01\\
17.54	0.01\\
17.55	0.01\\
17.56	0.01\\
17.57	0.01\\
17.58	0.01\\
17.59	0.01\\
17.6	0.01\\
17.61	0.01\\
17.62	0.01\\
17.63	0.01\\
17.64	0.01\\
17.65	0.01\\
17.66	0.01\\
17.67	0.01\\
17.68	0.01\\
17.69	0.01\\
17.7	0.01\\
17.71	0.01\\
17.72	0.01\\
17.73	0.01\\
17.74	0.01\\
17.75	0.01\\
17.76	0.01\\
17.77	0.01\\
17.78	0.01\\
17.79	0.01\\
17.8	0.01\\
17.81	0.01\\
17.82	0.01\\
17.83	0.01\\
17.84	0.01\\
17.85	0.01\\
17.86	0.01\\
17.87	0.01\\
17.88	0.01\\
17.89	0.01\\
17.9	0.01\\
17.91	0.01\\
17.92	0.01\\
17.93	0.01\\
17.94	0.01\\
17.95	0.01\\
17.96	0.01\\
17.97	0.01\\
17.98	0.01\\
17.99	0.01\\
18	0.01\\
18.01	0.01\\
18.02	0.01\\
18.03	0.01\\
18.04	0.01\\
18.05	0.01\\
18.06	0.01\\
18.07	0.01\\
18.08	0.01\\
18.09	0.01\\
18.1	0.01\\
18.11	0.01\\
18.12	0.01\\
18.13	0.01\\
18.14	0.01\\
18.15	0.01\\
18.16	0.01\\
18.17	0.01\\
18.18	0.01\\
18.19	0.01\\
18.2	0.01\\
18.21	0.01\\
18.22	0.01\\
18.23	0.01\\
18.24	0.01\\
18.25	0.01\\
18.26	0.01\\
18.27	0.01\\
18.28	0.01\\
18.29	0.01\\
18.3	0.01\\
18.31	0.01\\
18.32	0.01\\
18.33	0.01\\
18.34	0.01\\
18.35	0.01\\
18.36	0.01\\
18.37	0.01\\
18.38	0.01\\
18.39	0.01\\
18.4	0.01\\
18.41	0.01\\
18.42	0.01\\
18.43	0.01\\
18.44	0.01\\
18.45	0.01\\
18.46	0.01\\
18.47	0.01\\
18.48	0.01\\
18.49	0.01\\
18.5	0.01\\
18.51	0.01\\
18.52	0.01\\
18.53	0.01\\
18.54	0.01\\
18.55	0.01\\
18.56	0.01\\
18.57	0.01\\
18.58	0.01\\
18.59	0.01\\
18.6	0.01\\
18.61	0.01\\
18.62	0.01\\
18.63	0.01\\
18.64	0.01\\
18.65	0.01\\
18.66	0.01\\
18.67	0.01\\
18.68	0.01\\
18.69	0.01\\
18.7	0.01\\
18.71	0.01\\
18.72	0.01\\
18.73	0.01\\
18.74	0.01\\
18.75	0.01\\
18.76	0.01\\
18.77	0.01\\
18.78	0.01\\
18.79	0.01\\
18.8	0.01\\
18.81	0.01\\
18.82	0.01\\
18.83	0.01\\
18.84	0.01\\
18.85	0.01\\
18.86	0.01\\
18.87	0.01\\
18.88	0.01\\
18.89	0.01\\
18.9	0.01\\
18.91	0.01\\
18.92	0.01\\
18.93	0.01\\
18.94	0.01\\
18.95	0.01\\
18.96	0.01\\
18.97	0.01\\
18.98	0.01\\
18.99	0.01\\
19	0.01\\
19.01	0.01\\
19.02	0.01\\
19.03	0.01\\
19.04	0.01\\
19.05	0.01\\
19.06	0.01\\
19.07	0.01\\
19.08	0.01\\
19.09	0.01\\
19.1	0.01\\
19.11	0.01\\
19.12	0.01\\
19.13	0.01\\
19.14	0.01\\
19.15	0.01\\
19.16	0.01\\
19.17	0.01\\
19.18	0.01\\
19.19	0.01\\
19.2	0.01\\
19.21	0.01\\
19.22	0.01\\
19.23	0.01\\
19.24	0.01\\
19.25	0.01\\
19.26	0.01\\
19.27	0.01\\
19.28	0.01\\
19.29	0.01\\
19.3	0.01\\
19.31	0.01\\
19.32	0.01\\
19.33	0.01\\
19.34	0.01\\
19.35	0.01\\
19.36	0.01\\
19.37	0.01\\
19.38	0.01\\
19.39	0.01\\
19.4	0.01\\
19.41	0.01\\
19.42	0.01\\
19.43	0.01\\
19.44	0.01\\
19.45	0.01\\
19.46	0.01\\
19.47	0.01\\
19.48	0.01\\
19.49	0.01\\
19.5	0.01\\
19.51	0.01\\
19.52	0.01\\
19.53	0.01\\
19.54	0.01\\
19.55	0.01\\
19.56	0.01\\
19.57	0.01\\
19.58	0.01\\
19.59	0.01\\
19.6	0.01\\
19.61	0.01\\
19.62	0.01\\
19.63	0.01\\
19.64	0.01\\
19.65	0.01\\
19.66	0.01\\
19.67	0.01\\
19.68	0.01\\
19.69	0.01\\
19.7	0.01\\
19.71	0.01\\
19.72	0.01\\
19.73	0.01\\
19.74	0.01\\
19.75	0.01\\
19.76	0.01\\
19.77	0.01\\
19.78	0.01\\
19.79	0.01\\
19.8	0.01\\
19.81	0.01\\
19.82	0.01\\
19.83	0.01\\
19.84	0.01\\
19.85	0.01\\
19.86	0.01\\
19.87	0.01\\
19.88	0.01\\
19.89	0.01\\
19.9	0.01\\
19.91	0.01\\
19.92	0.01\\
19.93	0.01\\
19.94	0.01\\
19.95	0.01\\
19.96	0.01\\
19.97	0.01\\
19.98	0.01\\
19.99	0.01\\
20	0.01\\
20.01	0.01\\
20.02	0.01\\
20.03	0.01\\
20.04	0.01\\
20.05	0.01\\
20.06	0.01\\
20.07	0.01\\
20.08	0.01\\
20.09	0.01\\
20.1	0.01\\
20.11	0.01\\
20.12	0.01\\
20.13	0.01\\
20.14	0.01\\
20.15	0.01\\
20.16	0.01\\
20.17	0.01\\
20.18	0.01\\
20.19	0.01\\
20.2	0.01\\
20.21	0.01\\
20.22	0.01\\
20.23	0.01\\
20.24	0.01\\
20.25	0.01\\
20.26	0.01\\
20.27	0.01\\
20.28	0.01\\
20.29	0.01\\
20.3	0.01\\
20.31	0.01\\
20.32	0.01\\
20.33	0.01\\
20.34	0.01\\
20.35	0.01\\
20.36	0.01\\
20.37	0.01\\
20.38	0.01\\
20.39	0.01\\
20.4	0.01\\
20.41	0.01\\
20.42	0.01\\
20.43	0.01\\
20.44	0.01\\
20.45	0.01\\
20.46	0.01\\
20.47	0.01\\
20.48	0.01\\
20.49	0.01\\
20.5	0.01\\
20.51	0.01\\
20.52	0.01\\
20.53	0.01\\
20.54	0.01\\
20.55	0.01\\
20.56	0.01\\
20.57	0.01\\
20.58	0.01\\
20.59	0.01\\
20.6	0.01\\
20.61	0.01\\
20.62	0.01\\
20.63	0.01\\
20.64	0.01\\
20.65	0.01\\
20.66	0.01\\
20.67	0.01\\
20.68	0.01\\
20.69	0.01\\
20.7	0.01\\
20.71	0.01\\
20.72	0.01\\
20.73	0.01\\
20.74	0.01\\
20.75	0.01\\
20.76	0.01\\
20.77	0.01\\
20.78	0.01\\
20.79	0.01\\
20.8	0.01\\
20.81	0.01\\
20.82	0.01\\
20.83	0.01\\
20.84	0.01\\
20.85	0.01\\
20.86	0.01\\
20.87	0.01\\
20.88	0.01\\
20.89	0.01\\
20.9	0.01\\
20.91	0.01\\
20.92	0.01\\
20.93	0.01\\
20.94	0.01\\
20.95	0.01\\
20.96	0.01\\
20.97	0.01\\
20.98	0.01\\
20.99	0.01\\
21	0.01\\
21.01	0.01\\
21.02	0.01\\
21.03	0.01\\
21.04	0.01\\
21.05	0.01\\
21.06	0.01\\
21.07	0.01\\
21.08	0.01\\
21.09	0.01\\
21.1	0.01\\
21.11	0.01\\
21.12	0.01\\
21.13	0.01\\
21.14	0.01\\
21.15	0.01\\
21.16	0.01\\
21.17	0.01\\
21.18	0.01\\
21.19	0.01\\
21.2	0.01\\
21.21	0.01\\
21.22	0.01\\
21.23	0.01\\
21.24	0.01\\
21.25	0.01\\
21.26	0.01\\
21.27	0.01\\
21.28	0.01\\
21.29	0.01\\
21.3	0.01\\
21.31	0.01\\
21.32	0.01\\
21.33	0.01\\
21.34	0.01\\
21.35	0.01\\
21.36	0.01\\
21.37	0.01\\
21.38	0.01\\
21.39	0.01\\
21.4	0.01\\
21.41	0.01\\
21.42	0.01\\
21.43	0.01\\
21.44	0.01\\
21.45	0.01\\
21.46	0.01\\
21.47	0.01\\
21.48	0.01\\
21.49	0.01\\
21.5	0.01\\
21.51	0.01\\
21.52	0.01\\
21.53	0.01\\
21.54	0.01\\
21.55	0.01\\
21.56	0.01\\
21.57	0.01\\
21.58	0.01\\
21.59	0.01\\
21.6	0.01\\
21.61	0.01\\
21.62	0.01\\
21.63	0.01\\
21.64	0.01\\
21.65	0.01\\
21.66	0.01\\
21.67	0.01\\
21.68	0.01\\
21.69	0.01\\
21.7	0.01\\
21.71	0.01\\
21.72	0.01\\
21.73	0.01\\
21.74	0.01\\
21.75	0.01\\
21.76	0.01\\
21.77	0.01\\
21.78	0.01\\
21.79	0.01\\
21.8	0.01\\
21.81	0.01\\
21.82	0.01\\
21.83	0.01\\
21.84	0.01\\
21.85	0.01\\
21.86	0.01\\
21.87	0.01\\
21.88	0.01\\
21.89	0.01\\
21.9	0.01\\
21.91	0.01\\
21.92	0.01\\
21.93	0.01\\
21.94	0.01\\
21.95	0.01\\
21.96	0.01\\
21.97	0.01\\
21.98	0.01\\
21.99	0.01\\
22	0.01\\
22.01	0.01\\
22.02	0.01\\
22.03	0.01\\
22.04	0.01\\
22.05	0.01\\
22.06	0.01\\
22.07	0.01\\
22.08	0.01\\
22.09	0.01\\
22.1	0.01\\
22.11	0.01\\
22.12	0.01\\
22.13	0.01\\
22.14	0.01\\
22.15	0.01\\
22.16	0.01\\
22.17	0.01\\
22.18	0.01\\
22.19	0.01\\
22.2	0.01\\
22.21	0.01\\
22.22	0.01\\
22.23	0.01\\
22.24	0.01\\
22.25	0.01\\
22.26	0.01\\
22.27	0.01\\
22.28	0.01\\
22.29	0.01\\
22.3	0.01\\
22.31	0.01\\
22.32	0.01\\
22.33	0.01\\
22.34	0.01\\
22.35	0.01\\
22.36	0.01\\
22.37	0.01\\
22.38	0.01\\
22.39	0.01\\
22.4	0.01\\
22.41	0.01\\
22.42	0.01\\
22.43	0.01\\
22.44	0.01\\
22.45	0.01\\
22.46	0.01\\
22.47	0.01\\
22.48	0.01\\
22.49	0.01\\
22.5	0.01\\
22.51	0.01\\
22.52	0.01\\
22.53	0.01\\
22.54	0.01\\
22.55	0.01\\
22.56	0.01\\
22.57	0.01\\
22.58	0.01\\
22.59	0.01\\
22.6	0.01\\
22.61	0.01\\
22.62	0.01\\
22.63	0.01\\
22.64	0.01\\
22.65	0.01\\
22.66	0.01\\
22.67	0.01\\
22.68	0.01\\
22.69	0.01\\
22.7	0.01\\
22.71	0.01\\
22.72	0.01\\
22.73	0.01\\
22.74	0.01\\
22.75	0.01\\
22.76	0.01\\
22.77	0.01\\
22.78	0.01\\
22.79	0.01\\
22.8	0.01\\
22.81	0.01\\
22.82	0.01\\
22.83	0.01\\
22.84	0.01\\
22.85	0.01\\
22.86	0.01\\
22.87	0.01\\
22.88	0.01\\
22.89	0.01\\
22.9	0.01\\
22.91	0.01\\
22.92	0.01\\
22.93	0.01\\
22.94	0.01\\
22.95	0.01\\
22.96	0.01\\
22.97	0.01\\
22.98	0.01\\
22.99	0.01\\
23	0.01\\
23.01	0.01\\
23.02	0.01\\
23.03	0.01\\
23.04	0.01\\
23.05	0.01\\
23.06	0.01\\
23.07	0.01\\
23.08	0.01\\
23.09	0.01\\
23.1	0.01\\
23.11	0.01\\
23.12	0.01\\
23.13	0.01\\
23.14	0.01\\
23.15	0.01\\
23.16	0.01\\
23.17	0.01\\
23.18	0.01\\
23.19	0.01\\
23.2	0.01\\
23.21	0.01\\
23.22	0.01\\
23.23	0.01\\
23.24	0.01\\
23.25	0.01\\
23.26	0.01\\
23.27	0.01\\
23.28	0.01\\
23.29	0.01\\
23.3	0.01\\
23.31	0.01\\
23.32	0.01\\
23.33	0.01\\
23.34	0.01\\
23.35	0.01\\
23.36	0.01\\
23.37	0.01\\
23.38	0.01\\
23.39	0.01\\
23.4	0.01\\
23.41	0.01\\
23.42	0.01\\
23.43	0.01\\
23.44	0.01\\
23.45	0.01\\
23.46	0.01\\
23.47	0.01\\
23.48	0.01\\
23.49	0.01\\
23.5	0.01\\
23.51	0.01\\
23.52	0.01\\
23.53	0.01\\
23.54	0.01\\
23.55	0.01\\
23.56	0.01\\
23.57	0.01\\
23.58	0.01\\
23.59	0.01\\
23.6	0.01\\
23.61	0.01\\
23.62	0.01\\
23.63	0.01\\
23.64	0.01\\
23.65	0.01\\
23.66	0.01\\
23.67	0.01\\
23.68	0.01\\
23.69	0.01\\
23.7	0.01\\
23.71	0.01\\
23.72	0.01\\
23.73	0.01\\
23.74	0.01\\
23.75	0.01\\
23.76	0.01\\
23.77	0.01\\
23.78	0.01\\
23.79	0.01\\
23.8	0.01\\
23.81	0.01\\
23.82	0.01\\
23.83	0.01\\
23.84	0.01\\
23.85	0.01\\
23.86	0.01\\
23.87	0.01\\
23.88	0.01\\
23.89	0.01\\
23.9	0.01\\
23.91	0.01\\
23.92	0.01\\
23.93	0.01\\
23.94	0.01\\
23.95	0.01\\
23.96	0.01\\
23.97	0.01\\
23.98	0.01\\
23.99	0.01\\
24	0.01\\
24.01	0.01\\
24.02	0.01\\
24.03	0.01\\
24.04	0.01\\
24.05	0.01\\
24.06	0.01\\
24.07	0.01\\
24.08	0.01\\
24.09	0.01\\
24.1	0.01\\
24.11	0.01\\
24.12	0.01\\
24.13	0.01\\
24.14	0.01\\
24.15	0.01\\
24.16	0.01\\
24.17	0.01\\
24.18	0.01\\
24.19	0.01\\
24.2	0.01\\
24.21	0.01\\
24.22	0.01\\
24.23	0.01\\
24.24	0.01\\
24.25	0.01\\
24.26	0.01\\
24.27	0.01\\
24.28	0.01\\
24.29	0.01\\
24.3	0.01\\
24.31	0.01\\
24.32	0.01\\
24.33	0.01\\
24.34	0.01\\
24.35	0.01\\
24.36	0.01\\
24.37	0.01\\
24.38	0.01\\
24.39	0.01\\
24.4	0.01\\
24.41	0.01\\
24.42	0.01\\
24.43	0.01\\
24.44	0.01\\
24.45	0.01\\
24.46	0.01\\
24.47	0.01\\
24.48	0.01\\
24.49	0.01\\
24.5	0.01\\
24.51	0.01\\
24.52	0.01\\
24.53	0.01\\
24.54	0.01\\
24.55	0.01\\
24.56	0.01\\
24.57	0.01\\
24.58	0.01\\
24.59	0.01\\
24.6	0.01\\
24.61	0.01\\
24.62	0.01\\
24.63	0.01\\
24.64	0.01\\
24.65	0.01\\
24.66	0.01\\
24.67	0.01\\
24.68	0.01\\
24.69	0.01\\
24.7	0.01\\
24.71	0.01\\
24.72	0.01\\
24.73	0.01\\
24.74	0.01\\
24.75	0.01\\
24.76	0.01\\
24.77	0.01\\
24.78	0.01\\
24.79	0.01\\
24.8	0.01\\
24.81	0.01\\
24.82	0.01\\
24.83	0.01\\
24.84	0.01\\
24.85	0.01\\
24.86	0.01\\
24.87	0.01\\
24.88	0.01\\
24.89	0.01\\
24.9	0.01\\
24.91	0.01\\
24.92	0.01\\
24.93	0.01\\
24.94	0.01\\
24.95	0.01\\
24.96	0.01\\
24.97	0.01\\
24.98	0.01\\
24.99	0.01\\
25	0.01\\
25.01	0.01\\
25.02	0.01\\
25.03	0.01\\
25.04	0.01\\
25.05	0.01\\
25.06	0.01\\
25.07	0.01\\
25.08	0.01\\
25.09	0.01\\
25.1	0.01\\
25.11	0.01\\
25.12	0.01\\
25.13	0.01\\
25.14	0.01\\
25.15	0.01\\
25.16	0.01\\
25.17	0.01\\
25.18	0.01\\
25.19	0.01\\
25.2	0.01\\
25.21	0.01\\
25.22	0.01\\
25.23	0.01\\
25.24	0.01\\
25.25	0.01\\
25.26	0.01\\
25.27	0.01\\
25.28	0.01\\
25.29	0.01\\
25.3	0.01\\
25.31	0.01\\
25.32	0.01\\
25.33	0.01\\
25.34	0.01\\
25.35	0.01\\
25.36	0.01\\
25.37	0.01\\
25.38	0.01\\
25.39	0.01\\
25.4	0.01\\
25.41	0.01\\
25.42	0.01\\
25.43	0.01\\
25.44	0.01\\
25.45	0.01\\
25.46	0.01\\
25.47	0.01\\
25.48	0.01\\
25.49	0.01\\
25.5	0.01\\
25.51	0.01\\
25.52	0.01\\
25.53	0.01\\
25.54	0.01\\
25.55	0.01\\
25.56	0.01\\
25.57	0.01\\
25.58	0.01\\
25.59	0.01\\
25.6	0.01\\
25.61	0.01\\
25.62	0.01\\
25.63	0.01\\
25.64	0.01\\
25.65	0.01\\
25.66	0.01\\
25.67	0.01\\
25.68	0.01\\
25.69	0.01\\
25.7	0.01\\
25.71	0.01\\
25.72	0.01\\
25.73	0.01\\
25.74	0.01\\
25.75	0.01\\
25.76	0.01\\
25.77	0.01\\
25.78	0.01\\
25.79	0.01\\
25.8	0.01\\
25.81	0.01\\
25.82	0.01\\
25.83	0.01\\
25.84	0.01\\
25.85	0.01\\
25.86	0.01\\
25.87	0.01\\
25.88	0.01\\
25.89	0.01\\
25.9	0.01\\
25.91	0.01\\
25.92	0.01\\
25.93	0.01\\
25.94	0.01\\
25.95	0.01\\
25.96	0.01\\
25.97	0.01\\
25.98	0.01\\
25.99	0.01\\
26	0.01\\
26.01	0.01\\
26.02	0.01\\
26.03	0.01\\
26.04	0.01\\
26.05	0.01\\
26.06	0.01\\
26.07	0.01\\
26.08	0.01\\
26.09	0.01\\
26.1	0.01\\
26.11	0.01\\
26.12	0.01\\
26.13	0.01\\
26.14	0.01\\
26.15	0.01\\
26.16	0.01\\
26.17	0.01\\
26.18	0.01\\
26.19	0.01\\
26.2	0.01\\
26.21	0.01\\
26.22	0.01\\
26.23	0.01\\
26.24	0.01\\
26.25	0.01\\
26.26	0.01\\
26.27	0.01\\
26.28	0.01\\
26.29	0.01\\
26.3	0.01\\
26.31	0.01\\
26.32	0.01\\
26.33	0.01\\
26.34	0.01\\
26.35	0.01\\
26.36	0.01\\
26.37	0.01\\
26.38	0.01\\
26.39	0.01\\
26.4	0.01\\
26.41	0.01\\
26.42	0.01\\
26.43	0.01\\
26.44	0.01\\
26.45	0.01\\
26.46	0.01\\
26.47	0.01\\
26.48	0.01\\
26.49	0.01\\
26.5	0.01\\
26.51	0.01\\
26.52	0.01\\
26.53	0.01\\
26.54	0.01\\
26.55	0.01\\
26.56	0.01\\
26.57	0.01\\
26.58	0.01\\
26.59	0.01\\
26.6	0.01\\
26.61	0.01\\
26.62	0.01\\
26.63	0.01\\
26.64	0.01\\
26.65	0.01\\
26.66	0.01\\
26.67	0.01\\
26.68	0.01\\
26.69	0.01\\
26.7	0.01\\
26.71	0.01\\
26.72	0.01\\
26.73	0.01\\
26.74	0.01\\
26.75	0.01\\
26.76	0.01\\
26.77	0.01\\
26.78	0.01\\
26.79	0.01\\
26.8	0.01\\
26.81	0.01\\
26.82	0.01\\
26.83	0.01\\
26.84	0.01\\
26.85	0.01\\
26.86	0.01\\
26.87	0.01\\
26.88	0.01\\
26.89	0.01\\
26.9	0.01\\
26.91	0.01\\
26.92	0.01\\
26.93	0.01\\
26.94	0.01\\
26.95	0.01\\
26.96	0.01\\
26.97	0.01\\
26.98	0.01\\
26.99	0.01\\
27	0.01\\
27.01	0.01\\
27.02	0.01\\
27.03	0.01\\
27.04	0.01\\
27.05	0.01\\
27.06	0.01\\
27.07	0.01\\
27.08	0.01\\
27.09	0.01\\
27.1	0.01\\
27.11	0.01\\
27.12	0.01\\
27.13	0.01\\
27.14	0.01\\
27.15	0.01\\
27.16	0.01\\
27.17	0.01\\
27.18	0.01\\
27.19	0.01\\
27.2	0.01\\
27.21	0.01\\
27.22	0.01\\
27.23	0.01\\
27.24	0.01\\
27.25	0.01\\
27.26	0.01\\
27.27	0.01\\
27.28	0.01\\
27.29	0.01\\
27.3	0.01\\
27.31	0.01\\
27.32	0.01\\
27.33	0.01\\
27.34	0.01\\
27.35	0.01\\
27.36	0.01\\
27.37	0.01\\
27.38	0.01\\
27.39	0.01\\
27.4	0.01\\
27.41	0.01\\
27.42	0.01\\
27.43	0.01\\
27.44	0.01\\
27.45	0.01\\
27.46	0.01\\
27.47	0.01\\
27.48	0.01\\
27.49	0.01\\
27.5	0.01\\
27.51	0.01\\
27.52	0.01\\
27.53	0.01\\
27.54	0.01\\
27.55	0.01\\
27.56	0.01\\
27.57	0.01\\
27.58	0.01\\
27.59	0.01\\
27.6	0.01\\
27.61	0.01\\
27.62	0.01\\
27.63	0.01\\
27.64	0.01\\
27.65	0.01\\
27.66	0.01\\
27.67	0.01\\
27.68	0.01\\
27.69	0.01\\
27.7	0.01\\
27.71	0.01\\
27.72	0.01\\
27.73	0.01\\
27.74	0.01\\
27.75	0.01\\
27.76	0.01\\
27.77	0.01\\
27.78	0.01\\
27.79	0.01\\
27.8	0.01\\
27.81	0.01\\
27.82	0.01\\
27.83	0.01\\
27.84	0.01\\
27.85	0.01\\
27.86	0.01\\
27.87	0.01\\
27.88	0.01\\
27.89	0.01\\
27.9	0.01\\
27.91	0.01\\
27.92	0.01\\
27.93	0.01\\
27.94	0.01\\
27.95	0.01\\
27.96	0.01\\
27.97	0.01\\
27.98	0.01\\
27.99	0.01\\
28	0.01\\
28.01	0.01\\
28.02	0.01\\
28.03	0.01\\
28.04	0.01\\
28.05	0.01\\
28.06	0.01\\
28.07	0.01\\
28.08	0.01\\
28.09	0.01\\
28.1	0.01\\
28.11	0.01\\
28.12	0.01\\
28.13	0.01\\
28.14	0.01\\
28.15	0.01\\
28.16	0.01\\
28.17	0.01\\
28.18	0.01\\
28.19	0.01\\
28.2	0.01\\
28.21	0.01\\
28.22	0.01\\
28.23	0.01\\
28.24	0.01\\
28.25	0.01\\
28.26	0.01\\
28.27	0.01\\
28.28	0.01\\
28.29	0.01\\
28.3	0.01\\
28.31	0.01\\
28.32	0.01\\
28.33	0.01\\
28.34	0.01\\
28.35	0.01\\
28.36	0.01\\
28.37	0.01\\
28.38	0.01\\
28.39	0.01\\
28.4	0.01\\
28.41	0.01\\
28.42	0.01\\
28.43	0.01\\
28.44	0.01\\
28.45	0.01\\
28.46	0.01\\
28.47	0.01\\
28.48	0.01\\
28.49	0.01\\
28.5	0.01\\
28.51	0.01\\
28.52	0.01\\
28.53	0.01\\
28.54	0.01\\
28.55	0.01\\
28.56	0.01\\
28.57	0.01\\
28.58	0.01\\
28.59	0.01\\
28.6	0.01\\
28.61	0.01\\
28.62	0.01\\
28.63	0.01\\
28.64	0.01\\
28.65	0.01\\
28.66	0.01\\
28.67	0.01\\
28.68	0.01\\
28.69	0.01\\
28.7	0.01\\
28.71	0.01\\
28.72	0.01\\
28.73	0.01\\
28.74	0.01\\
28.75	0.01\\
28.76	0.01\\
28.77	0.01\\
28.78	0.01\\
28.79	0.01\\
28.8	0.01\\
28.81	0.01\\
28.82	0.01\\
28.83	0.01\\
28.84	0.01\\
28.85	0.01\\
28.86	0.01\\
28.87	0.01\\
28.88	0.01\\
28.89	0.01\\
28.9	0.01\\
28.91	0.01\\
28.92	0.01\\
28.93	0.01\\
28.94	0.01\\
28.95	0.01\\
28.96	0.01\\
28.97	0.01\\
28.98	0.01\\
28.99	0.01\\
29	0.01\\
29.01	0.01\\
29.02	0.01\\
29.03	0.01\\
29.04	0.01\\
29.05	0.01\\
29.06	0.01\\
29.07	0.01\\
29.08	0.01\\
29.09	0.01\\
29.1	0.01\\
29.11	0.01\\
29.12	0.01\\
29.13	0.01\\
29.14	0.01\\
29.15	0.01\\
29.16	0.01\\
29.17	0.01\\
29.18	0.01\\
29.19	0.01\\
29.2	0.01\\
29.21	0.01\\
29.22	0.01\\
29.23	0.01\\
29.24	0.01\\
29.25	0.01\\
29.26	0.01\\
29.27	0.01\\
29.28	0.01\\
29.29	0.01\\
29.3	0.01\\
29.31	0.01\\
29.32	0.01\\
29.33	0.01\\
29.34	0.01\\
29.35	0.01\\
29.36	0.01\\
29.37	0.01\\
29.38	0.01\\
29.39	0.01\\
29.4	0.01\\
29.41	0.01\\
29.42	0.01\\
29.43	0.01\\
29.44	0.01\\
29.45	0.01\\
29.46	0.01\\
29.47	0.01\\
29.48	0.01\\
29.49	0.01\\
29.5	0.01\\
29.51	0.01\\
29.52	0.01\\
29.53	0.01\\
29.54	0.01\\
29.55	0.01\\
29.56	0.01\\
29.57	0.01\\
29.58	0.01\\
29.59	0.01\\
29.6	0.01\\
29.61	0.01\\
29.62	0.01\\
29.63	0.01\\
29.64	0.01\\
29.65	0.01\\
29.66	0.01\\
29.67	0.01\\
29.68	0.01\\
29.69	0.01\\
29.7	0.01\\
29.71	0.01\\
29.72	0.01\\
29.73	0.01\\
29.74	0.01\\
29.75	0.01\\
29.76	0.01\\
29.77	0.01\\
29.78	0.01\\
29.79	0.01\\
29.8	0.01\\
29.81	0.01\\
29.82	0.01\\
29.83	0.01\\
29.84	0.01\\
29.85	0.01\\
29.86	0.01\\
29.87	0.01\\
29.88	0.01\\
29.89	0.01\\
29.9	0.01\\
29.91	0.01\\
29.92	0.01\\
29.93	0.01\\
29.94	0.01\\
29.95	0.01\\
29.96	0.01\\
29.97	0.01\\
29.98	0.01\\
29.99	0.01\\
30	0.01\\
30.01	0.01\\
30.02	0.01\\
30.03	0.01\\
30.04	0.01\\
30.05	0.01\\
30.06	0.01\\
30.07	0.01\\
30.08	0.01\\
30.09	0.01\\
30.1	0.01\\
30.11	0.01\\
30.12	0.01\\
30.13	0.01\\
30.14	0.01\\
30.15	0.01\\
30.16	0.01\\
30.17	0.01\\
30.18	0.01\\
30.19	0.01\\
30.2	0.01\\
30.21	0.01\\
30.22	0.01\\
30.23	0.01\\
30.24	0.01\\
30.25	0.01\\
30.26	0.01\\
30.27	0.01\\
30.28	0.01\\
30.29	0.01\\
30.3	0.01\\
30.31	0.01\\
30.32	0.01\\
30.33	0.01\\
30.34	0.01\\
30.35	0.01\\
30.36	0.01\\
30.37	0.01\\
30.38	0.01\\
30.39	0.01\\
30.4	0.01\\
30.41	0.01\\
30.42	0.01\\
30.43	0.01\\
30.44	0.01\\
30.45	0.01\\
30.46	0.01\\
30.47	0.01\\
30.48	0.01\\
30.49	0.01\\
30.5	0.01\\
30.51	0.01\\
30.52	0.01\\
30.53	0.01\\
30.54	0.01\\
30.55	0.01\\
30.56	0.01\\
30.57	0.01\\
30.58	0.01\\
30.59	0.01\\
30.6	0.01\\
30.61	0.01\\
30.62	0.01\\
30.63	0.01\\
30.64	0.01\\
30.65	0.01\\
30.66	0.01\\
30.67	0.01\\
30.68	0.01\\
30.69	0.01\\
30.7	0.01\\
30.71	0.01\\
30.72	0.01\\
30.73	0.01\\
30.74	0.01\\
30.75	0.01\\
30.76	0.01\\
30.77	0.01\\
30.78	0.01\\
30.79	0.01\\
30.8	0.01\\
30.81	0.01\\
30.82	0.01\\
30.83	0.01\\
30.84	0.01\\
30.85	0.01\\
30.86	0.01\\
30.87	0.01\\
30.88	0.01\\
30.89	0.01\\
30.9	0.01\\
30.91	0.01\\
30.92	0.01\\
30.93	0.01\\
30.94	0.01\\
30.95	0.01\\
30.96	0.01\\
30.97	0.01\\
30.98	0.01\\
30.99	0.01\\
31	0.01\\
31.01	0.01\\
31.02	0.01\\
31.03	0.01\\
31.04	0.01\\
31.05	0.01\\
31.06	0.01\\
31.07	0.01\\
31.08	0.01\\
31.09	0.01\\
31.1	0.01\\
31.11	0.01\\
31.12	0.01\\
31.13	0.01\\
31.14	0.01\\
31.15	0.01\\
31.16	0.01\\
31.17	0.01\\
31.18	0.01\\
31.19	0.01\\
31.2	0.01\\
31.21	0.01\\
31.22	0.01\\
31.23	0.01\\
31.24	0.01\\
31.25	0.01\\
31.26	0.01\\
31.27	0.01\\
31.28	0.01\\
31.29	0.01\\
31.3	0.01\\
31.31	0.01\\
31.32	0.01\\
31.33	0.01\\
31.34	0.01\\
31.35	0.01\\
31.36	0.01\\
31.37	0.01\\
31.38	0.01\\
31.39	0.01\\
31.4	0.01\\
31.41	0.01\\
31.42	0.01\\
31.43	0.01\\
31.44	0.01\\
31.45	0.01\\
31.46	0.01\\
31.47	0.01\\
31.48	0.01\\
31.49	0.01\\
31.5	0.01\\
31.51	0.01\\
31.52	0.01\\
31.53	0.01\\
31.54	0.01\\
31.55	0.01\\
31.56	0.01\\
31.57	0.01\\
31.58	0.01\\
31.59	0.01\\
31.6	0.01\\
31.61	0.01\\
31.62	0.01\\
31.63	0.01\\
31.64	0.01\\
31.65	0.01\\
31.66	0.01\\
31.67	0.01\\
31.68	0.01\\
31.69	0.01\\
31.7	0.01\\
31.71	0.01\\
31.72	0.01\\
31.73	0.01\\
31.74	0.01\\
31.75	0.01\\
31.76	0.01\\
31.77	0.01\\
31.78	0.01\\
31.79	0.01\\
31.8	0.01\\
31.81	0.01\\
31.82	0.01\\
31.83	0.01\\
31.84	0.01\\
31.85	0.01\\
31.86	0.01\\
31.87	0.01\\
31.88	0.01\\
31.89	0.01\\
31.9	0.01\\
31.91	0.01\\
31.92	0.01\\
31.93	0.01\\
31.94	0.01\\
31.95	0.01\\
31.96	0.01\\
31.97	0.01\\
31.98	0.01\\
31.99	0.01\\
32	0.01\\
32.01	0.01\\
32.02	0.01\\
32.03	0.01\\
32.04	0.01\\
32.05	0.01\\
32.06	0.01\\
32.07	0.01\\
32.08	0.01\\
32.09	0.01\\
32.1	0.01\\
32.11	0.01\\
32.12	0.01\\
32.13	0.01\\
32.14	0.01\\
32.15	0.01\\
32.16	0.01\\
32.17	0.01\\
32.18	0.01\\
32.19	0.01\\
32.2	0.01\\
32.21	0.01\\
32.22	0.01\\
32.23	0.01\\
32.24	0.01\\
32.25	0.01\\
32.26	0.01\\
32.27	0.01\\
32.28	0.01\\
32.29	0.01\\
32.3	0.01\\
32.31	0.01\\
32.32	0.01\\
32.33	0.01\\
32.34	0.01\\
32.35	0.01\\
32.36	0.01\\
32.37	0.01\\
32.38	0.01\\
32.39	0.01\\
32.4	0.01\\
32.41	0.01\\
32.42	0.01\\
32.43	0.01\\
32.44	0.01\\
32.45	0.01\\
32.46	0.01\\
32.47	0.01\\
32.48	0.01\\
32.49	0.01\\
32.5	0.01\\
32.51	0.01\\
32.52	0.01\\
32.53	0.01\\
32.54	0.01\\
32.55	0.01\\
32.56	0.01\\
32.57	0.01\\
32.58	0.01\\
32.59	0.01\\
32.6	0.01\\
32.61	0.01\\
32.62	0.01\\
32.63	0.01\\
32.64	0.01\\
32.65	0.01\\
32.66	0.01\\
32.67	0.01\\
32.68	0.01\\
32.69	0.01\\
32.7	0.01\\
32.71	0.01\\
32.72	0.01\\
32.73	0.01\\
32.74	0.01\\
32.75	0.01\\
32.76	0.01\\
32.77	0.01\\
32.78	0.01\\
32.79	0.01\\
32.8	0.01\\
32.81	0.01\\
32.82	0.01\\
32.83	0.01\\
32.84	0.01\\
32.85	0.01\\
32.86	0.01\\
32.87	0.01\\
32.88	0.01\\
32.89	0.01\\
32.9	0.01\\
32.91	0.01\\
32.92	0.01\\
32.93	0.01\\
32.94	0.01\\
32.95	0.01\\
32.96	0.01\\
32.97	0.01\\
32.98	0.01\\
32.99	0.01\\
33	0.01\\
33.01	0.01\\
33.02	0.01\\
33.03	0.01\\
33.04	0.01\\
33.05	0.01\\
33.06	0.01\\
33.07	0.01\\
33.08	0.01\\
33.09	0.01\\
33.1	0.01\\
33.11	0.01\\
33.12	0.01\\
33.13	0.01\\
33.14	0.01\\
33.15	0.01\\
33.16	0.01\\
33.17	0.01\\
33.18	0.01\\
33.19	0.01\\
33.2	0.01\\
33.21	0.01\\
33.22	0.01\\
33.23	0.01\\
33.24	0.01\\
33.25	0.01\\
33.26	0.01\\
33.27	0.01\\
33.28	0.01\\
33.29	0.01\\
33.3	0.01\\
33.31	0.01\\
33.32	0.01\\
33.33	0.01\\
33.34	0.01\\
33.35	0.01\\
33.36	0.01\\
33.37	0.01\\
33.38	0.01\\
33.39	0.01\\
33.4	0.01\\
33.41	0.01\\
33.42	0.01\\
33.43	0.01\\
33.44	0.01\\
33.45	0.01\\
33.46	0.01\\
33.47	0.01\\
33.48	0.01\\
33.49	0.01\\
33.5	0.01\\
33.51	0.01\\
33.52	0.01\\
33.53	0.01\\
33.54	0.01\\
33.55	0.01\\
33.56	0.01\\
33.57	0.01\\
33.58	0.01\\
33.59	0.01\\
33.6	0.01\\
33.61	0.01\\
33.62	0.01\\
33.63	0.01\\
33.64	0.01\\
33.65	0.01\\
33.66	0.01\\
33.67	0.01\\
33.68	0.01\\
33.69	0.01\\
33.7	0.01\\
33.71	0.01\\
33.72	0.01\\
33.73	0.01\\
33.74	0.01\\
33.75	0.01\\
33.76	0.01\\
33.77	0.01\\
33.78	0.01\\
33.79	0.01\\
33.8	0.01\\
33.81	0.01\\
33.82	0.01\\
33.83	0.01\\
33.84	0.01\\
33.85	0.01\\
33.86	0.01\\
33.87	0.01\\
33.88	0.01\\
33.89	0.01\\
33.9	0.01\\
33.91	0.01\\
33.92	0.01\\
33.93	0.01\\
33.94	0.01\\
33.95	0.01\\
33.96	0.01\\
33.97	0.01\\
33.98	0.01\\
33.99	0.01\\
34	0.01\\
34.01	0.01\\
34.02	0.01\\
34.03	0.01\\
34.04	0.01\\
34.05	0.01\\
34.06	0.01\\
34.07	0.01\\
34.08	0.01\\
34.09	0.01\\
34.1	0.01\\
34.11	0.01\\
34.12	0.01\\
34.13	0.01\\
34.14	0.01\\
34.15	0.01\\
34.16	0.01\\
34.17	0.01\\
34.18	0.01\\
34.19	0.01\\
34.2	0.01\\
34.21	0.01\\
34.22	0.01\\
34.23	0.01\\
34.24	0.01\\
34.25	0.01\\
34.26	0.01\\
34.27	0.01\\
34.28	0.01\\
34.29	0.01\\
34.3	0.01\\
34.31	0.01\\
34.32	0.01\\
34.33	0.01\\
34.34	0.01\\
34.35	0.01\\
34.36	0.01\\
34.37	0.01\\
34.38	0.01\\
34.39	0.01\\
34.4	0.01\\
34.41	0.01\\
34.42	0.01\\
34.43	0.01\\
34.44	0.01\\
34.45	0.01\\
34.46	0.01\\
34.47	0.01\\
34.48	0.01\\
34.49	0.01\\
34.5	0.01\\
34.51	0.01\\
34.52	0.01\\
34.53	0.01\\
34.54	0.01\\
34.55	0.01\\
34.56	0.01\\
34.57	0.01\\
34.58	0.01\\
34.59	0.01\\
34.6	0.01\\
34.61	0.01\\
34.62	0.01\\
34.63	0.01\\
34.64	0.01\\
34.65	0.01\\
34.66	0.01\\
34.67	0.01\\
34.68	0.01\\
34.69	0.01\\
34.7	0.01\\
34.71	0.01\\
34.72	0.01\\
34.73	0.01\\
34.74	0.01\\
34.75	0.01\\
34.76	0.01\\
34.77	0.01\\
34.78	0.01\\
34.79	0.01\\
34.8	0.01\\
34.81	0.01\\
34.82	0.01\\
34.83	0.01\\
34.84	0.01\\
34.85	0.01\\
34.86	0.01\\
34.87	0.01\\
34.88	0.01\\
34.89	0.01\\
34.9	0.01\\
34.91	0.01\\
34.92	0.01\\
34.93	0.01\\
34.94	0.01\\
34.95	0.01\\
34.96	0.01\\
34.97	0.01\\
34.98	0.01\\
34.99	0.01\\
35	0.01\\
35.01	0.01\\
35.02	0.01\\
35.03	0.01\\
35.04	0.01\\
35.05	0.01\\
35.06	0.01\\
35.07	0.01\\
35.08	0.01\\
35.09	0.01\\
35.1	0.01\\
35.11	0.01\\
35.12	0.01\\
35.13	0.01\\
35.14	0.01\\
35.15	0.01\\
35.16	0.01\\
35.17	0.01\\
35.18	0.01\\
35.19	0.01\\
35.2	0.01\\
35.21	0.01\\
35.22	0.01\\
35.23	0.01\\
35.24	0.01\\
35.25	0.01\\
35.26	0.01\\
35.27	0.01\\
35.28	0.01\\
35.29	0.01\\
35.3	0.01\\
35.31	0.01\\
35.32	0.01\\
35.33	0.01\\
35.34	0.01\\
35.35	0.01\\
35.36	0.01\\
35.37	0.01\\
35.38	0.01\\
35.39	0.01\\
35.4	0.01\\
35.41	0.01\\
35.42	0.01\\
35.43	0.01\\
35.44	0.01\\
35.45	0.01\\
35.46	0.01\\
35.47	0.01\\
35.48	0.01\\
35.49	0.01\\
35.5	0.01\\
35.51	0.01\\
35.52	0.01\\
35.53	0.01\\
35.54	0.01\\
35.55	0.01\\
35.56	0.01\\
35.57	0.01\\
35.58	0.01\\
35.59	0.01\\
35.6	0.01\\
35.61	0.01\\
35.62	0.01\\
35.63	0.01\\
35.64	0.01\\
35.65	0.01\\
35.66	0.01\\
35.67	0.01\\
35.68	0.01\\
35.69	0.01\\
35.7	0.01\\
35.71	0.01\\
35.72	0.01\\
35.73	0.01\\
35.74	0.01\\
35.75	0.01\\
35.76	0.01\\
35.77	0.01\\
35.78	0.01\\
35.79	0.01\\
35.8	0.01\\
35.81	0.01\\
35.82	0.01\\
35.83	0.01\\
35.84	0.01\\
35.85	0.01\\
35.86	0.01\\
35.87	0.01\\
35.88	0.01\\
35.89	0.01\\
35.9	0.01\\
35.91	0.01\\
35.92	0.01\\
35.93	0.01\\
35.94	0.01\\
35.95	0.01\\
35.96	0.01\\
35.97	0.01\\
35.98	0.01\\
35.99	0.01\\
36	0.01\\
36.01	0.01\\
36.02	0.01\\
36.03	0.01\\
36.04	0.01\\
36.05	0.01\\
36.06	0.01\\
36.07	0.01\\
36.08	0.01\\
36.09	0.01\\
36.1	0.01\\
36.11	0.01\\
36.12	0.01\\
36.13	0.01\\
36.14	0.01\\
36.15	0.01\\
36.16	0.01\\
36.17	0.01\\
36.18	0.01\\
36.19	0.01\\
36.2	0.01\\
36.21	0.01\\
36.22	0.01\\
36.23	0.01\\
36.24	0.01\\
36.25	0.01\\
36.26	0.01\\
36.27	0.01\\
36.28	0.01\\
36.29	0.01\\
36.3	0.01\\
36.31	0.01\\
36.32	0.01\\
36.33	0.01\\
36.34	0.01\\
36.35	0.01\\
36.36	0.01\\
36.37	0.01\\
36.38	0.01\\
36.39	0.01\\
36.4	0.01\\
36.41	0.01\\
36.42	0.01\\
36.43	0.01\\
36.44	0.01\\
36.45	0.01\\
36.46	0.01\\
36.47	0.01\\
36.48	0.01\\
36.49	0.01\\
36.5	0.01\\
36.51	0.01\\
36.52	0.01\\
36.53	0.01\\
36.54	0.01\\
36.55	0.01\\
36.56	0.01\\
36.57	0.01\\
36.58	0.01\\
36.59	0.01\\
36.6	0.01\\
36.61	0.01\\
36.62	0.01\\
36.63	0.01\\
36.64	0.01\\
36.65	0.01\\
36.66	0.01\\
36.67	0.01\\
36.68	0.01\\
36.69	0.01\\
36.7	0.01\\
36.71	0.01\\
36.72	0.01\\
36.73	0.01\\
36.74	0.01\\
36.75	0.01\\
36.76	0.01\\
36.77	0.01\\
36.78	0.01\\
36.79	0.01\\
36.8	0.01\\
36.81	0.01\\
36.82	0.01\\
36.83	0.01\\
36.84	0.01\\
36.85	0.01\\
36.86	0.01\\
36.87	0.01\\
36.88	0.01\\
36.89	0.01\\
36.9	0.01\\
36.91	0.01\\
36.92	0.01\\
36.93	0.01\\
36.94	0.01\\
36.95	0.01\\
36.96	0.01\\
36.97	0.01\\
36.98	0.01\\
36.99	0.01\\
37	0.01\\
37.01	0.01\\
37.02	0.01\\
37.03	0.01\\
37.04	0.01\\
37.05	0.01\\
37.06	0.01\\
37.07	0.01\\
37.08	0.01\\
37.09	0.01\\
37.1	0.01\\
37.11	0.01\\
37.12	0.01\\
37.13	0.01\\
37.14	0.01\\
37.15	0.01\\
37.16	0.01\\
37.17	0.01\\
37.18	0.01\\
37.19	0.01\\
37.2	0.01\\
37.21	0.01\\
37.22	0.01\\
37.23	0.01\\
37.24	0.01\\
37.25	0.01\\
37.26	0.01\\
37.27	0.01\\
37.28	0.01\\
37.29	0.01\\
37.3	0.01\\
37.31	0.01\\
37.32	0.01\\
37.33	0.01\\
37.34	0.01\\
37.35	0.01\\
37.36	0.01\\
37.37	0.01\\
37.38	0.01\\
37.39	0.01\\
37.4	0.01\\
37.41	0.01\\
37.42	0.01\\
37.43	0.01\\
37.44	0.01\\
37.45	0.01\\
37.46	0.01\\
37.47	0.01\\
37.48	0.01\\
37.49	0.01\\
37.5	0.01\\
37.51	0.01\\
37.52	0.01\\
37.53	0.01\\
37.54	0.01\\
37.55	0.01\\
37.56	0.01\\
37.57	0.01\\
37.58	0.01\\
37.59	0.01\\
37.6	0.01\\
37.61	0.01\\
37.62	0.01\\
37.63	0.01\\
37.64	0.01\\
37.65	0.01\\
37.66	0.01\\
37.67	0.01\\
37.68	0.01\\
37.69	0.01\\
37.7	0.01\\
37.71	0.01\\
37.72	0.01\\
37.73	0.01\\
37.74	0.01\\
37.75	0.01\\
37.76	0.01\\
37.77	0.01\\
37.78	0.01\\
37.79	0.01\\
37.8	0.01\\
37.81	0.01\\
37.82	0.01\\
37.83	0.01\\
37.84	0.01\\
37.85	0.01\\
37.86	0.01\\
37.87	0.01\\
37.88	0.01\\
37.89	0.01\\
37.9	0.01\\
37.91	0.01\\
37.92	0.01\\
37.93	0.01\\
37.94	0.01\\
37.95	0.01\\
37.96	0.01\\
37.97	0.01\\
37.98	0.01\\
37.99	0.01\\
38	0.01\\
38.01	0.01\\
38.02	0.01\\
38.03	0.01\\
38.04	0.01\\
38.05	0.01\\
38.06	0.01\\
38.07	0.01\\
38.08	0.01\\
38.09	0.01\\
38.1	0.01\\
38.11	0.01\\
38.12	0.01\\
38.13	0.01\\
38.14	0.01\\
38.15	0.01\\
38.16	0.01\\
38.17	0.01\\
38.18	0.01\\
38.19	0.01\\
38.2	0.01\\
38.21	0.01\\
38.22	0.01\\
38.23	0.01\\
38.24	0.01\\
38.25	0.01\\
38.26	0.01\\
38.27	0.01\\
38.28	0.01\\
38.29	0.01\\
38.3	0.01\\
38.31	0.01\\
38.32	0.01\\
38.33	0.01\\
38.34	0.01\\
38.35	0.01\\
38.36	0.01\\
38.37	0.01\\
38.38	0.01\\
38.39	0.01\\
38.4	0.01\\
38.41	0.01\\
38.42	0.01\\
38.43	0.01\\
38.44	0.01\\
38.45	0.01\\
38.46	0.01\\
38.47	0.01\\
38.48	0.01\\
38.49	0.01\\
38.5	0.01\\
38.51	0.01\\
38.52	0.01\\
38.53	0.01\\
38.54	0.01\\
38.55	0.01\\
38.56	0.01\\
38.57	0.01\\
38.58	0.01\\
38.59	0.01\\
38.6	0.01\\
38.61	0.01\\
38.62	0.01\\
38.63	0.01\\
38.64	0.01\\
38.65	0.01\\
38.66	0.01\\
38.67	0.01\\
38.68	0.01\\
38.69	0.01\\
38.7	0.01\\
38.71	0.01\\
38.72	0.01\\
38.73	0.01\\
38.74	0.01\\
38.75	0.01\\
38.76	0.01\\
38.77	0.01\\
38.78	0.01\\
38.79	0.01\\
38.8	0.01\\
38.81	0.01\\
38.82	0.01\\
38.83	0.01\\
38.84	0.01\\
38.85	0.01\\
38.86	0.01\\
38.87	0.01\\
38.88	0.01\\
38.89	0.01\\
38.9	0.01\\
38.91	0.01\\
38.92	0.01\\
38.93	0.01\\
38.94	0.01\\
38.95	0.01\\
38.96	0.01\\
38.97	0.01\\
38.98	0.01\\
38.99	0.01\\
39	0.01\\
39.01	0.01\\
39.02	0.01\\
39.03	0.01\\
39.04	0.01\\
39.05	0.01\\
39.06	0.01\\
39.07	0.01\\
39.08	0.01\\
39.09	0.01\\
39.1	0.01\\
39.11	0.01\\
39.12	0.01\\
39.13	0.01\\
39.14	0.01\\
39.15	0.01\\
39.16	0.01\\
39.17	0.01\\
39.18	0.01\\
39.19	0.01\\
39.2	0.01\\
39.21	0.01\\
39.22	0.01\\
39.23	0.01\\
39.24	0.01\\
39.25	0.01\\
39.26	0.01\\
39.27	0.01\\
39.28	0.01\\
39.29	0.01\\
39.3	0.01\\
39.31	0.01\\
39.32	0.01\\
39.33	0.01\\
39.34	0.01\\
39.35	0.01\\
39.36	0.01\\
39.37	0.01\\
39.38	0.01\\
39.39	0.01\\
39.4	0.01\\
39.41	0.01\\
39.42	0.01\\
39.43	0.01\\
39.44	0.01\\
39.45	0.01\\
39.46	0.01\\
39.47	0.01\\
39.48	0.01\\
39.49	0.01\\
39.5	0.01\\
39.51	0.01\\
39.52	0.01\\
39.53	0.01\\
39.54	0.01\\
39.55	0.01\\
39.56	0.01\\
39.57	0.01\\
39.58	0.01\\
39.59	0.01\\
39.6	0.01\\
39.61	0.01\\
39.62	0.01\\
39.63	0.01\\
39.64	0.01\\
39.65	0.01\\
39.66	0.01\\
39.67	0.01\\
39.68	0.01\\
39.69	0.01\\
39.7	0.01\\
39.71	0.01\\
39.72	0.01\\
39.73	0.01\\
39.74	0.01\\
39.75	0.01\\
39.76	0.01\\
39.77	0.01\\
39.78	0.01\\
39.79	0.01\\
39.8	0.01\\
39.81	0.01\\
39.82	0.01\\
39.83	0.01\\
39.84	0.01\\
39.85	0.01\\
39.86	0.01\\
39.87	0.01\\
39.88	0.01\\
39.89	0.01\\
39.9	0.01\\
39.91	0.01\\
39.92	0.01\\
39.93	0.01\\
39.94	0.01\\
39.95	0.01\\
39.96	0.01\\
39.97	0.01\\
39.98	0.01\\
39.99	0.01\\
40	0.01\\
40.01	0.01\\
};
\addplot [color=mycolor1,dashed,forget plot]
  table[row sep=crcr]{%
40.01	0.01\\
40.02	0.01\\
40.03	0.01\\
40.04	0.01\\
40.05	0.01\\
40.06	0.01\\
40.07	0.01\\
40.08	0.01\\
40.09	0.01\\
40.1	0.01\\
40.11	0.01\\
40.12	0.01\\
40.13	0.01\\
40.14	0.01\\
40.15	0.01\\
40.16	0.01\\
40.17	0.01\\
40.18	0.01\\
40.19	0.01\\
40.2	0.01\\
40.21	0.01\\
40.22	0.01\\
40.23	0.01\\
40.24	0.01\\
40.25	0.01\\
40.26	0.01\\
40.27	0.01\\
40.28	0.01\\
40.29	0.01\\
40.3	0.01\\
40.31	0.01\\
40.32	0.01\\
40.33	0.01\\
40.34	0.01\\
40.35	0.01\\
40.36	0.01\\
40.37	0.01\\
40.38	0.01\\
40.39	0.01\\
40.4	0.01\\
40.41	0.01\\
40.42	0.01\\
40.43	0.01\\
40.44	0.01\\
40.45	0.01\\
40.46	0.01\\
40.47	0.01\\
40.48	0.01\\
40.49	0.01\\
40.5	0.01\\
40.51	0.01\\
40.52	0.01\\
40.53	0.01\\
40.54	0.01\\
40.55	0.01\\
40.56	0.01\\
40.57	0.01\\
40.58	0.01\\
40.59	0.01\\
40.6	0.01\\
40.61	0.01\\
40.62	0.01\\
40.63	0.01\\
40.64	0.01\\
40.65	0.01\\
40.66	0.01\\
40.67	0.01\\
40.68	0.01\\
40.69	0.01\\
40.7	0.01\\
40.71	0.01\\
40.72	0.01\\
40.73	0.01\\
40.74	0.01\\
40.75	0.01\\
40.76	0.01\\
40.77	0.01\\
40.78	0.01\\
40.79	0.01\\
40.8	0.01\\
40.81	0.01\\
40.82	0.01\\
40.83	0.01\\
40.84	0.01\\
40.85	0.01\\
40.86	0.01\\
40.87	0.01\\
40.88	0.01\\
40.89	0.01\\
40.9	0.01\\
40.91	0.01\\
40.92	0.01\\
40.93	0.01\\
40.94	0.01\\
40.95	0.01\\
40.96	0.01\\
40.97	0.01\\
40.98	0.01\\
40.99	0.01\\
41	0.01\\
41.01	0.01\\
41.02	0.01\\
41.03	0.01\\
41.04	0.01\\
41.05	0.01\\
41.06	0.01\\
41.07	0.01\\
41.08	0.01\\
41.09	0.01\\
41.1	0.01\\
41.11	0.01\\
41.12	0.01\\
41.13	0.01\\
41.14	0.01\\
41.15	0.01\\
41.16	0.01\\
41.17	0.01\\
41.18	0.01\\
41.19	0.01\\
41.2	0.01\\
41.21	0.01\\
41.22	0.01\\
41.23	0.01\\
41.24	0.01\\
41.25	0.01\\
41.26	0.01\\
41.27	0.01\\
41.28	0.01\\
41.29	0.01\\
41.3	0.01\\
41.31	0.01\\
41.32	0.01\\
41.33	0.01\\
41.34	0.01\\
41.35	0.01\\
41.36	0.01\\
41.37	0.01\\
41.38	0.01\\
41.39	0.01\\
41.4	0.01\\
41.41	0.01\\
41.42	0.01\\
41.43	0.01\\
41.44	0.01\\
41.45	0.01\\
41.46	0.01\\
41.47	0.01\\
41.48	0.01\\
41.49	0.01\\
41.5	0.01\\
41.51	0.01\\
41.52	0.01\\
41.53	0.01\\
41.54	0.01\\
41.55	0.01\\
41.56	0.01\\
41.57	0.01\\
41.58	0.01\\
41.59	0.01\\
41.6	0.01\\
41.61	0.01\\
41.62	0.01\\
41.63	0.01\\
41.64	0.01\\
41.65	0.01\\
41.66	0.01\\
41.67	0.01\\
41.68	0.01\\
41.69	0.01\\
41.7	0.01\\
41.71	0.01\\
41.72	0.01\\
41.73	0.01\\
41.74	0.01\\
41.75	0.01\\
41.76	0.01\\
41.77	0.01\\
41.78	0.01\\
41.79	0.01\\
41.8	0.01\\
41.81	0.01\\
41.82	0.01\\
41.83	0.01\\
41.84	0.01\\
41.85	0.01\\
41.86	0.01\\
41.87	0.01\\
41.88	0.01\\
41.89	0.01\\
41.9	0.01\\
41.91	0.01\\
41.92	0.01\\
41.93	0.01\\
41.94	0.01\\
41.95	0.01\\
41.96	0.01\\
41.97	0.01\\
41.98	0.01\\
41.99	0.01\\
42	0.01\\
42.01	0.01\\
42.02	0.01\\
42.03	0.01\\
42.04	0.01\\
42.05	0.01\\
42.06	0.01\\
42.07	0.01\\
42.08	0.01\\
42.09	0.01\\
42.1	0.01\\
42.11	0.01\\
42.12	0.01\\
42.13	0.01\\
42.14	0.01\\
42.15	0.01\\
42.16	0.01\\
42.17	0.01\\
42.18	0.01\\
42.19	0.01\\
42.2	0.01\\
42.21	0.01\\
42.22	0.01\\
42.23	0.01\\
42.24	0.01\\
42.25	0.01\\
42.26	0.01\\
42.27	0.01\\
42.28	0.01\\
42.29	0.01\\
42.3	0.01\\
42.31	0.01\\
42.32	0.01\\
42.33	0.01\\
42.34	0.01\\
42.35	0.01\\
42.36	0.01\\
42.37	0.01\\
42.38	0.01\\
42.39	0.01\\
42.4	0.01\\
42.41	0.01\\
42.42	0.01\\
42.43	0.01\\
42.44	0.01\\
42.45	0.01\\
42.46	0.01\\
42.47	0.01\\
42.48	0.01\\
42.49	0.01\\
42.5	0.01\\
42.51	0.01\\
42.52	0.01\\
42.53	0.01\\
42.54	0.01\\
42.55	0.01\\
42.56	0.01\\
42.57	0.01\\
42.58	0.01\\
42.59	0.01\\
42.6	0.01\\
42.61	0.01\\
42.62	0.01\\
42.63	0.01\\
42.64	0.01\\
42.65	0.01\\
42.66	0.01\\
42.67	0.01\\
42.68	0.01\\
42.69	0.01\\
42.7	0.01\\
42.71	0.01\\
42.72	0.01\\
42.73	0.01\\
42.74	0.01\\
42.75	0.01\\
42.76	0.01\\
42.77	0.01\\
42.78	0.01\\
42.79	0.01\\
42.8	0.01\\
42.81	0.01\\
42.82	0.01\\
42.83	0.01\\
42.84	0.01\\
42.85	0.01\\
42.86	0.01\\
42.87	0.01\\
42.88	0.01\\
42.89	0.01\\
42.9	0.01\\
42.91	0.01\\
42.92	0.01\\
42.93	0.01\\
42.94	0.01\\
42.95	0.01\\
42.96	0.01\\
42.97	0.01\\
42.98	0.01\\
42.99	0.01\\
43	0.01\\
43.01	0.01\\
43.02	0.01\\
43.03	0.01\\
43.04	0.01\\
43.05	0.01\\
43.06	0.01\\
43.07	0.01\\
43.08	0.01\\
43.09	0.01\\
43.1	0.01\\
43.11	0.01\\
43.12	0.01\\
43.13	0.01\\
43.14	0.01\\
43.15	0.01\\
43.16	0.01\\
43.17	0.01\\
43.18	0.01\\
43.19	0.01\\
43.2	0.01\\
43.21	0.01\\
43.22	0.01\\
43.23	0.01\\
43.24	0.01\\
43.25	0.01\\
43.26	0.01\\
43.27	0.01\\
43.28	0.01\\
43.29	0.01\\
43.3	0.01\\
43.31	0.01\\
43.32	0.01\\
43.33	0.01\\
43.34	0.01\\
43.35	0.01\\
43.36	0.01\\
43.37	0.01\\
43.38	0.01\\
43.39	0.01\\
43.4	0.01\\
43.41	0.01\\
43.42	0.01\\
43.43	0.01\\
43.44	0.01\\
43.45	0.01\\
43.46	0.01\\
43.47	0.01\\
43.48	0.01\\
43.49	0.01\\
43.5	0.01\\
43.51	0.01\\
43.52	0.01\\
43.53	0.01\\
43.54	0.01\\
43.55	0.01\\
43.56	0.01\\
43.57	0.01\\
43.58	0.01\\
43.59	0.01\\
43.6	0.01\\
43.61	0.01\\
43.62	0.01\\
43.63	0.01\\
43.64	0.01\\
43.65	0.01\\
43.66	0.01\\
43.67	0.01\\
43.68	0.01\\
43.69	0.01\\
43.7	0.01\\
43.71	0.01\\
43.72	0.01\\
43.73	0.01\\
43.74	0.01\\
43.75	0.01\\
43.76	0.01\\
43.77	0.01\\
43.78	0.01\\
43.79	0.01\\
43.8	0.01\\
43.81	0.01\\
43.82	0.01\\
43.83	0.01\\
43.84	0.01\\
43.85	0.01\\
43.86	0.01\\
43.87	0.01\\
43.88	0.01\\
43.89	0.01\\
43.9	0.01\\
43.91	0.01\\
43.92	0.01\\
43.93	0.01\\
43.94	0.01\\
43.95	0.01\\
43.96	0.01\\
43.97	0.01\\
43.98	0.01\\
43.99	0.01\\
44	0.01\\
44.01	0.01\\
44.02	0.01\\
44.03	0.01\\
44.04	0.01\\
44.05	0.01\\
44.06	0.01\\
44.07	0.01\\
44.08	0.01\\
44.09	0.01\\
44.1	0.01\\
44.11	0.01\\
44.12	0.01\\
44.13	0.01\\
44.14	0.01\\
44.15	0.01\\
44.16	0.01\\
44.17	0.01\\
44.18	0.01\\
44.19	0.01\\
44.2	0.01\\
44.21	0.01\\
44.22	0.01\\
44.23	0.01\\
44.24	0.01\\
44.25	0.01\\
44.26	0.01\\
44.27	0.01\\
44.28	0.01\\
44.29	0.01\\
44.3	0.01\\
44.31	0.01\\
44.32	0.01\\
44.33	0.01\\
44.34	0.01\\
44.35	0.01\\
44.36	0.01\\
44.37	0.01\\
44.38	0.01\\
44.39	0.01\\
44.4	0.01\\
44.41	0.01\\
44.42	0.01\\
44.43	0.01\\
44.44	0.01\\
44.45	0.01\\
44.46	0.01\\
44.47	0.01\\
44.48	0.01\\
44.49	0.01\\
44.5	0.01\\
44.51	0.01\\
44.52	0.01\\
44.53	0.01\\
44.54	0.01\\
44.55	0.01\\
44.56	0.01\\
44.57	0.01\\
44.58	0.01\\
44.59	0.01\\
44.6	0.01\\
44.61	0.01\\
44.62	0.01\\
44.63	0.01\\
44.64	0.01\\
44.65	0.01\\
44.66	0.01\\
44.67	0.01\\
44.68	0.01\\
44.69	0.01\\
44.7	0.01\\
44.71	0.01\\
44.72	0.01\\
44.73	0.01\\
44.74	0.01\\
44.75	0.01\\
44.76	0.01\\
44.77	0.01\\
44.78	0.01\\
44.79	0.01\\
44.8	0.01\\
44.81	0.01\\
44.82	0.01\\
44.83	0.01\\
44.84	0.01\\
44.85	0.01\\
44.86	0.01\\
44.87	0.01\\
44.88	0.01\\
44.89	0.01\\
44.9	0.01\\
44.91	0.01\\
44.92	0.01\\
44.93	0.01\\
44.94	0.01\\
44.95	0.01\\
44.96	0.01\\
44.97	0.01\\
44.98	0.01\\
44.99	0.01\\
45	0.01\\
45.01	0.01\\
45.02	0.01\\
45.03	0.01\\
45.04	0.01\\
45.05	0.01\\
45.06	0.01\\
45.07	0.01\\
45.08	0.01\\
45.09	0.01\\
45.1	0.01\\
45.11	0.01\\
45.12	0.01\\
45.13	0.01\\
45.14	0.01\\
45.15	0.01\\
45.16	0.01\\
45.17	0.01\\
45.18	0.01\\
45.19	0.01\\
45.2	0.01\\
45.21	0.01\\
45.22	0.01\\
45.23	0.01\\
45.24	0.01\\
45.25	0.01\\
45.26	0.01\\
45.27	0.01\\
45.28	0.01\\
45.29	0.01\\
45.3	0.01\\
45.31	0.01\\
45.32	0.01\\
45.33	0.01\\
45.34	0.01\\
45.35	0.01\\
45.36	0.01\\
45.37	0.01\\
45.38	0.01\\
45.39	0.01\\
45.4	0.01\\
45.41	0.01\\
45.42	0.01\\
45.43	0.01\\
45.44	0.01\\
45.45	0.01\\
45.46	0.01\\
45.47	0.01\\
45.48	0.01\\
45.49	0.01\\
45.5	0.01\\
45.51	0.01\\
45.52	0.01\\
45.53	0.01\\
45.54	0.01\\
45.55	0.01\\
45.56	0.01\\
45.57	0.01\\
45.58	0.01\\
45.59	0.01\\
45.6	0.01\\
45.61	0.01\\
45.62	0.01\\
45.63	0.01\\
45.64	0.01\\
45.65	0.01\\
45.66	0.01\\
45.67	0.01\\
45.68	0.01\\
45.69	0.01\\
45.7	0.01\\
45.71	0.01\\
45.72	0.01\\
45.73	0.01\\
45.74	0.01\\
45.75	0.01\\
45.76	0.01\\
45.77	0.01\\
45.78	0.01\\
45.79	0.01\\
45.8	0.01\\
45.81	0.01\\
45.82	0.01\\
45.83	0.01\\
45.84	0.01\\
45.85	0.01\\
45.86	0.01\\
45.87	0.01\\
45.88	0.01\\
45.89	0.01\\
45.9	0.01\\
45.91	0.01\\
45.92	0.01\\
45.93	0.01\\
45.94	0.01\\
45.95	0.01\\
45.96	0.01\\
45.97	0.01\\
45.98	0.01\\
45.99	0.01\\
46	0.01\\
46.01	0.01\\
46.02	0.01\\
46.03	0.01\\
46.04	0.01\\
46.05	0.01\\
46.06	0.01\\
46.07	0.01\\
46.08	0.01\\
46.09	0.01\\
46.1	0.01\\
46.11	0.01\\
46.12	0.01\\
46.13	0.01\\
46.14	0.01\\
46.15	0.01\\
46.16	0.01\\
46.17	0.01\\
46.18	0.01\\
46.19	0.01\\
46.2	0.01\\
46.21	0.01\\
46.22	0.01\\
46.23	0.01\\
46.24	0.01\\
46.25	0.01\\
46.26	0.01\\
46.27	0.01\\
46.28	0.01\\
46.29	0.01\\
46.3	0.01\\
46.31	0.01\\
46.32	0.01\\
46.33	0.01\\
46.34	0.01\\
46.35	0.01\\
46.36	0.01\\
46.37	0.01\\
46.38	0.01\\
46.39	0.01\\
46.4	0.01\\
46.41	0.01\\
46.42	0.01\\
46.43	0.01\\
46.44	0.01\\
46.45	0.01\\
46.46	0.01\\
46.47	0.01\\
46.48	0.01\\
46.49	0.01\\
46.5	0.01\\
46.51	0.01\\
46.52	0.01\\
46.53	0.01\\
46.54	0.01\\
46.55	0.01\\
46.56	0.01\\
46.57	0.01\\
46.58	0.01\\
46.59	0.01\\
46.6	0.01\\
46.61	0.01\\
46.62	0.01\\
46.63	0.01\\
46.64	0.01\\
46.65	0.01\\
46.66	0.01\\
46.67	0.01\\
46.68	0.01\\
46.69	0.01\\
46.7	0.01\\
46.71	0.01\\
46.72	0.01\\
46.73	0.01\\
46.74	0.01\\
46.75	0.01\\
46.76	0.01\\
46.77	0.01\\
46.78	0.01\\
46.79	0.01\\
46.8	0.01\\
46.81	0.01\\
46.82	0.01\\
46.83	0.01\\
46.84	0.01\\
46.85	0.01\\
46.86	0.01\\
46.87	0.01\\
46.88	0.01\\
46.89	0.01\\
46.9	0.01\\
46.91	0.01\\
46.92	0.01\\
46.93	0.01\\
46.94	0.01\\
46.95	0.01\\
46.96	0.01\\
46.97	0.01\\
46.98	0.01\\
46.99	0.01\\
47	0.01\\
47.01	0.01\\
47.02	0.01\\
47.03	0.01\\
47.04	0.01\\
47.05	0.01\\
47.06	0.01\\
47.07	0.01\\
47.08	0.01\\
47.09	0.01\\
47.1	0.01\\
47.11	0.01\\
47.12	0.01\\
47.13	0.01\\
47.14	0.01\\
47.15	0.01\\
47.16	0.01\\
47.17	0.01\\
47.18	0.01\\
47.19	0.01\\
47.2	0.01\\
47.21	0.01\\
47.22	0.01\\
47.23	0.01\\
47.24	0.01\\
47.25	0.01\\
47.26	0.01\\
47.27	0.01\\
47.28	0.01\\
47.29	0.01\\
47.3	0.01\\
47.31	0.01\\
47.32	0.01\\
47.33	0.01\\
47.34	0.01\\
47.35	0.01\\
47.36	0.01\\
47.37	0.01\\
47.38	0.01\\
47.39	0.01\\
47.4	0.01\\
47.41	0.01\\
47.42	0.01\\
47.43	0.01\\
47.44	0.01\\
47.45	0.01\\
47.46	0.01\\
47.47	0.01\\
47.48	0.01\\
47.49	0.01\\
47.5	0.01\\
47.51	0.01\\
47.52	0.01\\
47.53	0.01\\
47.54	0.01\\
47.55	0.01\\
47.56	0.01\\
47.57	0.01\\
47.58	0.01\\
47.59	0.01\\
47.6	0.01\\
47.61	0.01\\
47.62	0.01\\
47.63	0.01\\
47.64	0.01\\
47.65	0.01\\
47.66	0.01\\
47.67	0.01\\
47.68	0.01\\
47.69	0.01\\
47.7	0.01\\
47.71	0.01\\
47.72	0.01\\
47.73	0.01\\
47.74	0.01\\
47.75	0.01\\
47.76	0.01\\
47.77	0.01\\
47.78	0.01\\
47.79	0.01\\
47.8	0.01\\
47.81	0.01\\
47.82	0.01\\
47.83	0.01\\
47.84	0.01\\
47.85	0.01\\
47.86	0.01\\
47.87	0.01\\
47.88	0.01\\
47.89	0.01\\
47.9	0.01\\
47.91	0.01\\
47.92	0.01\\
47.93	0.01\\
47.94	0.01\\
47.95	0.01\\
47.96	0.01\\
47.97	0.01\\
47.98	0.01\\
47.99	0.01\\
48	0.01\\
48.01	0.01\\
48.02	0.01\\
48.03	0.01\\
48.04	0.01\\
48.05	0.01\\
48.06	0.01\\
48.07	0.01\\
48.08	0.01\\
48.09	0.01\\
48.1	0.01\\
48.11	0.01\\
48.12	0.01\\
48.13	0.01\\
48.14	0.01\\
48.15	0.01\\
48.16	0.01\\
48.17	0.01\\
48.18	0.01\\
48.19	0.01\\
48.2	0.01\\
48.21	0.01\\
48.22	0.01\\
48.23	0.01\\
48.24	0.01\\
48.25	0.01\\
48.26	0.01\\
48.27	0.01\\
48.28	0.01\\
48.29	0.01\\
48.3	0.01\\
48.31	0.01\\
48.32	0.01\\
48.33	0.01\\
48.34	0.01\\
48.35	0.01\\
48.36	0.01\\
48.37	0.01\\
48.38	0.01\\
48.39	0.01\\
48.4	0.01\\
48.41	0.01\\
48.42	0.01\\
48.43	0.01\\
48.44	0.01\\
48.45	0.01\\
48.46	0.01\\
48.47	0.01\\
48.48	0.01\\
48.49	0.01\\
48.5	0.01\\
48.51	0.01\\
48.52	0.01\\
48.53	0.01\\
48.54	0.01\\
48.55	0.01\\
48.56	0.01\\
48.57	0.01\\
48.58	0.01\\
48.59	0.01\\
48.6	0.01\\
48.61	0.01\\
48.62	0.01\\
48.63	0.01\\
48.64	0.01\\
48.65	0.01\\
48.66	0.01\\
48.67	0.01\\
48.68	0.01\\
48.69	0.01\\
48.7	0.01\\
48.71	0.01\\
48.72	0.01\\
48.73	0.01\\
48.74	0.01\\
48.75	0.01\\
48.76	0.01\\
48.77	0.01\\
48.78	0.01\\
48.79	0.01\\
48.8	0.01\\
48.81	0.01\\
48.82	0.01\\
48.83	0.01\\
48.84	0.01\\
48.85	0.01\\
48.86	0.01\\
48.87	0.01\\
48.88	0.01\\
48.89	0.01\\
48.9	0.01\\
48.91	0.01\\
48.92	0.01\\
48.93	0.01\\
48.94	0.01\\
48.95	0.01\\
48.96	0.01\\
48.97	0.01\\
48.98	0.01\\
48.99	0.01\\
49	0.01\\
49.01	0.01\\
49.02	0.01\\
49.03	0.01\\
49.04	0.01\\
49.05	0.01\\
49.06	0.01\\
49.07	0.01\\
49.08	0.01\\
49.09	0.01\\
49.1	0.01\\
49.11	0.01\\
49.12	0.01\\
49.13	0.01\\
49.14	0.01\\
49.15	0.01\\
49.16	0.01\\
49.17	0.01\\
49.18	0.01\\
49.19	0.01\\
49.2	0.01\\
49.21	0.01\\
49.22	0.01\\
49.23	0.01\\
49.24	0.01\\
49.25	0.01\\
49.26	0.01\\
49.27	0.01\\
49.28	0.01\\
49.29	0.01\\
49.3	0.01\\
49.31	0.01\\
49.32	0.01\\
49.33	0.01\\
49.34	0.01\\
49.35	0.01\\
49.36	0.01\\
49.37	0.01\\
49.38	0.01\\
49.39	0.01\\
49.4	0.01\\
49.41	0.01\\
49.42	0.01\\
49.43	0.01\\
49.44	0.01\\
49.45	0.01\\
49.46	0.01\\
49.47	0.01\\
49.48	0.01\\
49.49	0.01\\
49.5	0.01\\
49.51	0.01\\
49.52	0.01\\
49.53	0.01\\
49.54	0.01\\
49.55	0.01\\
49.56	0.01\\
49.57	0.01\\
49.58	0.01\\
49.59	0.01\\
49.6	0.01\\
49.61	0.01\\
49.62	0.01\\
49.63	0.01\\
49.64	0.01\\
49.65	0.01\\
49.66	0.01\\
49.67	0.01\\
49.68	0.01\\
49.69	0.01\\
49.7	0.01\\
49.71	0.01\\
49.72	0.01\\
49.73	0.01\\
49.74	0.01\\
49.75	0.01\\
49.76	0.01\\
49.77	0.01\\
49.78	0.01\\
49.79	0.01\\
49.8	0.01\\
49.81	0.01\\
49.82	0.01\\
49.83	0.01\\
49.84	0.01\\
49.85	0.01\\
49.86	0.01\\
49.87	0.01\\
49.88	0.01\\
49.89	0.01\\
49.9	0.01\\
49.91	0.01\\
49.92	0.01\\
49.93	0.01\\
49.94	0.01\\
49.95	0.01\\
49.96	0.01\\
49.97	0.01\\
49.98	0.01\\
49.99	0.01\\
50	0.01\\
50.01	0.01\\
50.02	0.01\\
50.03	0.01\\
50.04	0.01\\
50.05	0.01\\
50.06	0.01\\
50.07	0.01\\
50.08	0.01\\
50.09	0.01\\
50.1	0.01\\
50.11	0.01\\
50.12	0.01\\
50.13	0.01\\
50.14	0.01\\
50.15	0.01\\
50.16	0.01\\
50.17	0.01\\
50.18	0.01\\
50.19	0.01\\
50.2	0.01\\
50.21	0.01\\
50.22	0.01\\
50.23	0.01\\
50.24	0.01\\
50.25	0.01\\
50.26	0.01\\
50.27	0.01\\
50.28	0.01\\
50.29	0.01\\
50.3	0.01\\
50.31	0.01\\
50.32	0.01\\
50.33	0.01\\
50.34	0.01\\
50.35	0.01\\
50.36	0.01\\
50.37	0.01\\
50.38	0.01\\
50.39	0.01\\
50.4	0.01\\
50.41	0.01\\
50.42	0.01\\
50.43	0.01\\
50.44	0.01\\
50.45	0.01\\
50.46	0.01\\
50.47	0.01\\
50.48	0.01\\
50.49	0.01\\
50.5	0.01\\
50.51	0.01\\
50.52	0.01\\
50.53	0.01\\
50.54	0.01\\
50.55	0.01\\
50.56	0.01\\
50.57	0.01\\
50.58	0.01\\
50.59	0.01\\
50.6	0.01\\
50.61	0.01\\
50.62	0.01\\
50.63	0.01\\
50.64	0.01\\
50.65	0.01\\
50.66	0.01\\
50.67	0.01\\
50.68	0.01\\
50.69	0.01\\
50.7	0.01\\
50.71	0.01\\
50.72	0.01\\
50.73	0.01\\
50.74	0.01\\
50.75	0.01\\
50.76	0.01\\
50.77	0.01\\
50.78	0.01\\
50.79	0.01\\
50.8	0.01\\
50.81	0.01\\
50.82	0.01\\
50.83	0.01\\
50.84	0.01\\
50.85	0.01\\
50.86	0.01\\
50.87	0.01\\
50.88	0.01\\
50.89	0.01\\
50.9	0.01\\
50.91	0.01\\
50.92	0.01\\
50.93	0.01\\
50.94	0.01\\
50.95	0.01\\
50.96	0.01\\
50.97	0.01\\
50.98	0.01\\
50.99	0.01\\
51	0.01\\
51.01	0.01\\
51.02	0.01\\
51.03	0.01\\
51.04	0.01\\
51.05	0.01\\
51.06	0.01\\
51.07	0.01\\
51.08	0.01\\
51.09	0.01\\
51.1	0.01\\
51.11	0.01\\
51.12	0.01\\
51.13	0.01\\
51.14	0.01\\
51.15	0.01\\
51.16	0.01\\
51.17	0.01\\
51.18	0.01\\
51.19	0.01\\
51.2	0.01\\
51.21	0.01\\
51.22	0.01\\
51.23	0.01\\
51.24	0.01\\
51.25	0.01\\
51.26	0.01\\
51.27	0.01\\
51.28	0.01\\
51.29	0.01\\
51.3	0.01\\
51.31	0.01\\
51.32	0.01\\
51.33	0.01\\
51.34	0.01\\
51.35	0.01\\
51.36	0.01\\
51.37	0.01\\
51.38	0.01\\
51.39	0.01\\
51.4	0.01\\
51.41	0.01\\
51.42	0.01\\
51.43	0.01\\
51.44	0.01\\
51.45	0.01\\
51.46	0.01\\
51.47	0.01\\
51.48	0.01\\
51.49	0.01\\
51.5	0.01\\
51.51	0.01\\
51.52	0.01\\
51.53	0.01\\
51.54	0.01\\
51.55	0.01\\
51.56	0.01\\
51.57	0.01\\
51.58	0.01\\
51.59	0.01\\
51.6	0.01\\
51.61	0.01\\
51.62	0.01\\
51.63	0.01\\
51.64	0.01\\
51.65	0.01\\
51.66	0.01\\
51.67	0.01\\
51.68	0.01\\
51.69	0.01\\
51.7	0.01\\
51.71	0.01\\
51.72	0.01\\
51.73	0.01\\
51.74	0.01\\
51.75	0.01\\
51.76	0.01\\
51.77	0.01\\
51.78	0.01\\
51.79	0.01\\
51.8	0.01\\
51.81	0.01\\
51.82	0.01\\
51.83	0.01\\
51.84	0.01\\
51.85	0.01\\
51.86	0.01\\
51.87	0.01\\
51.88	0.01\\
51.89	0.01\\
51.9	0.01\\
51.91	0.01\\
51.92	0.01\\
51.93	0.01\\
51.94	0.01\\
51.95	0.01\\
51.96	0.01\\
51.97	0.01\\
51.98	0.01\\
51.99	0.01\\
52	0.01\\
52.01	0.01\\
52.02	0.01\\
52.03	0.01\\
52.04	0.01\\
52.05	0.01\\
52.06	0.01\\
52.07	0.01\\
52.08	0.01\\
52.09	0.01\\
52.1	0.01\\
52.11	0.01\\
52.12	0.01\\
52.13	0.01\\
52.14	0.01\\
52.15	0.01\\
52.16	0.01\\
52.17	0.01\\
52.18	0.01\\
52.19	0.01\\
52.2	0.01\\
52.21	0.01\\
52.22	0.01\\
52.23	0.01\\
52.24	0.01\\
52.25	0.01\\
52.26	0.01\\
52.27	0.01\\
52.28	0.01\\
52.29	0.01\\
52.3	0.01\\
52.31	0.01\\
52.32	0.01\\
52.33	0.01\\
52.34	0.01\\
52.35	0.01\\
52.36	0.01\\
52.37	0.01\\
52.38	0.01\\
52.39	0.01\\
52.4	0.01\\
52.41	0.01\\
52.42	0.01\\
52.43	0.01\\
52.44	0.01\\
52.45	0.01\\
52.46	0.01\\
52.47	0.01\\
52.48	0.01\\
52.49	0.01\\
52.5	0.01\\
52.51	0.01\\
52.52	0.01\\
52.53	0.01\\
52.54	0.01\\
52.55	0.01\\
52.56	0.01\\
52.57	0.01\\
52.58	0.01\\
52.59	0.01\\
52.6	0.01\\
52.61	0.01\\
52.62	0.01\\
52.63	0.01\\
52.64	0.01\\
52.65	0.01\\
52.66	0.01\\
52.67	0.01\\
52.68	0.01\\
52.69	0.01\\
52.7	0.01\\
52.71	0.01\\
52.72	0.01\\
52.73	0.01\\
52.74	0.01\\
52.75	0.01\\
52.76	0.01\\
52.77	0.01\\
52.78	0.01\\
52.79	0.01\\
52.8	0.01\\
52.81	0.01\\
52.82	0.01\\
52.83	0.01\\
52.84	0.01\\
52.85	0.01\\
52.86	0.01\\
52.87	0.01\\
52.88	0.01\\
52.89	0.01\\
52.9	0.01\\
52.91	0.01\\
52.92	0.01\\
52.93	0.01\\
52.94	0.01\\
52.95	0.01\\
52.96	0.01\\
52.97	0.01\\
52.98	0.01\\
52.99	0.01\\
53	0.01\\
53.01	0.01\\
53.02	0.01\\
53.03	0.01\\
53.04	0.01\\
53.05	0.01\\
53.06	0.01\\
53.07	0.01\\
53.08	0.01\\
53.09	0.01\\
53.1	0.01\\
53.11	0.01\\
53.12	0.01\\
53.13	0.01\\
53.14	0.01\\
53.15	0.01\\
53.16	0.01\\
53.17	0.01\\
53.18	0.01\\
53.19	0.01\\
53.2	0.01\\
53.21	0.01\\
53.22	0.01\\
53.23	0.01\\
53.24	0.01\\
53.25	0.01\\
53.26	0.01\\
53.27	0.01\\
53.28	0.01\\
53.29	0.01\\
53.3	0.01\\
53.31	0.01\\
53.32	0.01\\
53.33	0.01\\
53.34	0.01\\
53.35	0.01\\
53.36	0.01\\
53.37	0.01\\
53.38	0.01\\
53.39	0.01\\
53.4	0.01\\
53.41	0.01\\
53.42	0.01\\
53.43	0.01\\
53.44	0.01\\
53.45	0.01\\
53.46	0.01\\
53.47	0.01\\
53.48	0.01\\
53.49	0.01\\
53.5	0.01\\
53.51	0.01\\
53.52	0.01\\
53.53	0.01\\
53.54	0.01\\
53.55	0.01\\
53.56	0.01\\
53.57	0.01\\
53.58	0.01\\
53.59	0.01\\
53.6	0.01\\
53.61	0.01\\
53.62	0.01\\
53.63	0.01\\
53.64	0.01\\
53.65	0.01\\
53.66	0.01\\
53.67	0.01\\
53.68	0.01\\
53.69	0.01\\
53.7	0.01\\
53.71	0.01\\
53.72	0.01\\
53.73	0.01\\
53.74	0.01\\
53.75	0.01\\
53.76	0.01\\
53.77	0.01\\
53.78	0.01\\
53.79	0.01\\
53.8	0.01\\
53.81	0.01\\
53.82	0.01\\
53.83	0.01\\
53.84	0.01\\
53.85	0.01\\
53.86	0.01\\
53.87	0.01\\
53.88	0.01\\
53.89	0.01\\
53.9	0.01\\
53.91	0.01\\
53.92	0.01\\
53.93	0.01\\
53.94	0.01\\
53.95	0.01\\
53.96	0.01\\
53.97	0.01\\
53.98	0.01\\
53.99	0.01\\
54	0.01\\
54.01	0.01\\
54.02	0.01\\
54.03	0.01\\
54.04	0.01\\
54.05	0.01\\
54.06	0.01\\
54.07	0.01\\
54.08	0.01\\
54.09	0.01\\
54.1	0.01\\
54.11	0.01\\
54.12	0.01\\
54.13	0.01\\
54.14	0.01\\
54.15	0.01\\
54.16	0.01\\
54.17	0.01\\
54.18	0.01\\
54.19	0.01\\
54.2	0.01\\
54.21	0.01\\
54.22	0.01\\
54.23	0.01\\
54.24	0.01\\
54.25	0.01\\
54.26	0.01\\
54.27	0.01\\
54.28	0.01\\
54.29	0.01\\
54.3	0.01\\
54.31	0.01\\
54.32	0.01\\
54.33	0.01\\
54.34	0.01\\
54.35	0.01\\
54.36	0.01\\
54.37	0.01\\
54.38	0.01\\
54.39	0.01\\
54.4	0.01\\
54.41	0.01\\
54.42	0.01\\
54.43	0.01\\
54.44	0.01\\
54.45	0.01\\
54.46	0.01\\
54.47	0.01\\
54.48	0.01\\
54.49	0.01\\
54.5	0.01\\
54.51	0.01\\
54.52	0.01\\
54.53	0.01\\
54.54	0.01\\
54.55	0.01\\
54.56	0.01\\
54.57	0.01\\
54.58	0.01\\
54.59	0.01\\
54.6	0.01\\
54.61	0.01\\
54.62	0.01\\
54.63	0.01\\
54.64	0.01\\
54.65	0.01\\
54.66	0.01\\
54.67	0.01\\
54.68	0.01\\
54.69	0.01\\
54.7	0.01\\
54.71	0.01\\
54.72	0.01\\
54.73	0.01\\
54.74	0.01\\
54.75	0.01\\
54.76	0.01\\
54.77	0.01\\
54.78	0.01\\
54.79	0.01\\
54.8	0.01\\
54.81	0.01\\
54.82	0.01\\
54.83	0.01\\
54.84	0.01\\
54.85	0.01\\
54.86	0.01\\
54.87	0.01\\
54.88	0.01\\
54.89	0.01\\
54.9	0.01\\
54.91	0.01\\
54.92	0.01\\
54.93	0.01\\
54.94	0.01\\
54.95	0.01\\
54.96	0.01\\
54.97	0.01\\
54.98	0.01\\
54.99	0.01\\
55	0.01\\
55.01	0.01\\
55.02	0.01\\
55.03	0.01\\
55.04	0.01\\
55.05	0.01\\
55.06	0.01\\
55.07	0.01\\
55.08	0.01\\
55.09	0.01\\
55.1	0.01\\
55.11	0.01\\
55.12	0.01\\
55.13	0.01\\
55.14	0.01\\
55.15	0.01\\
55.16	0.01\\
55.17	0.01\\
55.18	0.01\\
55.19	0.01\\
55.2	0.01\\
55.21	0.01\\
55.22	0.01\\
55.23	0.01\\
55.24	0.01\\
55.25	0.01\\
55.26	0.01\\
55.27	0.01\\
55.28	0.01\\
55.29	0.01\\
55.3	0.01\\
55.31	0.01\\
55.32	0.01\\
55.33	0.01\\
55.34	0.01\\
55.35	0.01\\
55.36	0.01\\
55.37	0.01\\
55.38	0.01\\
55.39	0.01\\
55.4	0.01\\
55.41	0.01\\
55.42	0.01\\
55.43	0.01\\
55.44	0.01\\
55.45	0.01\\
55.46	0.01\\
55.47	0.01\\
55.48	0.01\\
55.49	0.01\\
55.5	0.01\\
55.51	0.01\\
55.52	0.01\\
55.53	0.01\\
55.54	0.01\\
55.55	0.01\\
55.56	0.01\\
55.57	0.01\\
55.58	0.01\\
55.59	0.01\\
55.6	0.01\\
55.61	0.01\\
55.62	0.01\\
55.63	0.01\\
55.64	0.01\\
55.65	0.01\\
55.66	0.01\\
55.67	0.01\\
55.68	0.01\\
55.69	0.01\\
55.7	0.01\\
55.71	0.01\\
55.72	0.01\\
55.73	0.01\\
55.74	0.01\\
55.75	0.01\\
55.76	0.01\\
55.77	0.01\\
55.78	0.01\\
55.79	0.01\\
55.8	0.01\\
55.81	0.01\\
55.82	0.01\\
55.83	0.01\\
55.84	0.01\\
55.85	0.01\\
55.86	0.01\\
55.87	0.01\\
55.88	0.01\\
55.89	0.01\\
55.9	0.01\\
55.91	0.01\\
55.92	0.01\\
55.93	0.01\\
55.94	0.01\\
55.95	0.01\\
55.96	0.01\\
55.97	0.01\\
55.98	0.01\\
55.99	0.01\\
56	0.01\\
56.01	0.01\\
56.02	0.01\\
56.03	0.01\\
56.04	0.01\\
56.05	0.01\\
56.06	0.01\\
56.07	0.01\\
56.08	0.01\\
56.09	0.01\\
56.1	0.01\\
56.11	0.01\\
56.12	0.01\\
56.13	0.01\\
56.14	0.01\\
56.15	0.01\\
56.16	0.01\\
56.17	0.01\\
56.18	0.01\\
56.19	0.01\\
56.2	0.01\\
56.21	0.01\\
56.22	0.01\\
56.23	0.01\\
56.24	0.01\\
56.25	0.01\\
56.26	0.01\\
56.27	0.01\\
56.28	0.01\\
56.29	0.01\\
56.3	0.01\\
56.31	0.01\\
56.32	0.01\\
56.33	0.01\\
56.34	0.01\\
56.35	0.01\\
56.36	0.01\\
56.37	0.01\\
56.38	0.01\\
56.39	0.01\\
56.4	0.01\\
56.41	0.01\\
56.42	0.01\\
56.43	0.01\\
56.44	0.01\\
56.45	0.01\\
56.46	0.01\\
56.47	0.01\\
56.48	0.01\\
56.49	0.01\\
56.5	0.01\\
56.51	0.01\\
56.52	0.01\\
56.53	0.01\\
56.54	0.01\\
56.55	0.01\\
56.56	0.01\\
56.57	0.01\\
56.58	0.01\\
56.59	0.01\\
56.6	0.01\\
56.61	0.01\\
56.62	0.01\\
56.63	0.01\\
56.64	0.01\\
56.65	0.01\\
56.66	0.01\\
56.67	0.01\\
56.68	0.01\\
56.69	0.01\\
56.7	0.01\\
56.71	0.01\\
56.72	0.01\\
56.73	0.01\\
56.74	0.01\\
56.75	0.01\\
56.76	0.01\\
56.77	0.01\\
56.78	0.01\\
56.79	0.01\\
56.8	0.01\\
56.81	0.01\\
56.82	0.01\\
56.83	0.01\\
56.84	0.01\\
56.85	0.01\\
56.86	0.01\\
56.87	0.01\\
56.88	0.01\\
56.89	0.01\\
56.9	0.01\\
56.91	0.01\\
56.92	0.01\\
56.93	0.01\\
56.94	0.01\\
56.95	0.01\\
56.96	0.01\\
56.97	0.01\\
56.98	0.01\\
56.99	0.01\\
57	0.01\\
57.01	0.01\\
57.02	0.01\\
57.03	0.01\\
57.04	0.01\\
57.05	0.01\\
57.06	0.01\\
57.07	0.01\\
57.08	0.01\\
57.09	0.01\\
57.1	0.01\\
57.11	0.01\\
57.12	0.01\\
57.13	0.01\\
57.14	0.01\\
57.15	0.01\\
57.16	0.01\\
57.17	0.01\\
57.18	0.01\\
57.19	0.01\\
57.2	0.01\\
57.21	0.01\\
57.22	0.01\\
57.23	0.01\\
57.24	0.01\\
57.25	0.01\\
57.26	0.01\\
57.27	0.01\\
57.28	0.01\\
57.29	0.01\\
57.3	0.01\\
57.31	0.01\\
57.32	0.01\\
57.33	0.01\\
57.34	0.01\\
57.35	0.01\\
57.36	0.01\\
57.37	0.01\\
57.38	0.01\\
57.39	0.01\\
57.4	0.01\\
57.41	0.01\\
57.42	0.01\\
57.43	0.01\\
57.44	0.01\\
57.45	0.01\\
57.46	0.01\\
57.47	0.01\\
57.48	0.01\\
57.49	0.01\\
57.5	0.01\\
57.51	0.01\\
57.52	0.01\\
57.53	0.01\\
57.54	0.01\\
57.55	0.01\\
57.56	0.01\\
57.57	0.01\\
57.58	0.01\\
57.59	0.01\\
57.6	0.01\\
57.61	0.01\\
57.62	0.01\\
57.63	0.01\\
57.64	0.01\\
57.65	0.01\\
57.66	0.01\\
57.67	0.01\\
57.68	0.01\\
57.69	0.01\\
57.7	0.01\\
57.71	0.01\\
57.72	0.01\\
57.73	0.01\\
57.74	0.01\\
57.75	0.01\\
57.76	0.01\\
57.77	0.01\\
57.78	0.01\\
57.79	0.01\\
57.8	0.01\\
57.81	0.01\\
57.82	0.01\\
57.83	0.01\\
57.84	0.01\\
57.85	0.01\\
57.86	0.01\\
57.87	0.01\\
57.88	0.01\\
57.89	0.01\\
57.9	0.01\\
57.91	0.01\\
57.92	0.01\\
57.93	0.01\\
57.94	0.01\\
57.95	0.01\\
57.96	0.01\\
57.97	0.01\\
57.98	0.01\\
57.99	0.01\\
58	0.01\\
58.01	0.01\\
58.02	0.01\\
58.03	0.01\\
58.04	0.01\\
58.05	0.01\\
58.06	0.01\\
58.07	0.01\\
58.08	0.01\\
58.09	0.01\\
58.1	0.01\\
58.11	0.01\\
58.12	0.01\\
58.13	0.01\\
58.14	0.01\\
58.15	0.01\\
58.16	0.01\\
58.17	0.01\\
58.18	0.01\\
58.19	0.01\\
58.2	0.01\\
58.21	0.01\\
58.22	0.01\\
58.23	0.01\\
58.24	0.01\\
58.25	0.01\\
58.26	0.01\\
58.27	0.01\\
58.28	0.01\\
58.29	0.01\\
58.3	0.01\\
58.31	0.01\\
58.32	0.01\\
58.33	0.01\\
58.34	0.01\\
58.35	0.01\\
58.36	0.01\\
58.37	0.01\\
58.38	0.01\\
58.39	0.01\\
58.4	0.01\\
58.41	0.01\\
58.42	0.01\\
58.43	0.01\\
58.44	0.01\\
58.45	0.01\\
58.46	0.01\\
58.47	0.01\\
58.48	0.01\\
58.49	0.01\\
58.5	0.01\\
58.51	0.01\\
58.52	0.01\\
58.53	0.01\\
58.54	0.01\\
58.55	0.01\\
58.56	0.01\\
58.57	0.01\\
58.58	0.01\\
58.59	0.01\\
58.6	0.01\\
58.61	0.01\\
58.62	0.01\\
58.63	0.01\\
58.64	0.01\\
58.65	0.01\\
58.66	0.01\\
58.67	0.01\\
58.68	0.01\\
58.69	0.01\\
58.7	0.01\\
58.71	0.01\\
58.72	0.01\\
58.73	0.01\\
58.74	0.01\\
58.75	0.01\\
58.76	0.01\\
58.77	0.01\\
58.78	0.01\\
58.79	0.01\\
58.8	0.01\\
58.81	0.01\\
58.82	0.01\\
58.83	0.01\\
58.84	0.01\\
58.85	0.01\\
58.86	0.01\\
58.87	0.01\\
58.88	0.01\\
58.89	0.01\\
58.9	0.01\\
58.91	0.01\\
58.92	0.01\\
58.93	0.01\\
58.94	0.01\\
58.95	0.01\\
58.96	0.01\\
58.97	0.01\\
58.98	0.01\\
58.99	0.01\\
59	0.01\\
59.01	0.01\\
59.02	0.01\\
59.03	0.01\\
59.04	0.01\\
59.05	0.01\\
59.06	0.01\\
59.07	0.01\\
59.08	0.01\\
59.09	0.01\\
59.1	0.01\\
59.11	0.01\\
59.12	0.01\\
59.13	0.01\\
59.14	0.01\\
59.15	0.01\\
59.16	0.01\\
59.17	0.01\\
59.18	0.01\\
59.19	0.01\\
59.2	0.01\\
59.21	0.01\\
59.22	0.01\\
59.23	0.01\\
59.24	0.01\\
59.25	0.01\\
59.26	0.01\\
59.27	0.01\\
59.28	0.01\\
59.29	0.01\\
59.3	0.01\\
59.31	0.01\\
59.32	0.01\\
59.33	0.01\\
59.34	0.01\\
59.35	0.01\\
59.36	0.01\\
59.37	0.01\\
59.38	0.01\\
59.39	0.01\\
59.4	0.01\\
59.41	0.01\\
59.42	0.01\\
59.43	0.01\\
59.44	0.01\\
59.45	0.01\\
59.46	0.01\\
59.47	0.01\\
59.48	0.01\\
59.49	0.01\\
59.5	0.01\\
59.51	0.01\\
59.52	0.01\\
59.53	0.01\\
59.54	0.01\\
59.55	0.01\\
59.56	0.01\\
59.57	0.01\\
59.58	0.01\\
59.59	0.01\\
59.6	0.01\\
59.61	0.01\\
59.62	0.01\\
59.63	0.01\\
59.64	0.01\\
59.65	0.01\\
59.66	0.01\\
59.67	0.01\\
59.68	0.01\\
59.69	0.01\\
59.7	0.01\\
59.71	0.01\\
59.72	0.01\\
59.73	0.01\\
59.74	0.01\\
59.75	0.01\\
59.76	0.01\\
59.77	0.01\\
59.78	0.01\\
59.79	0.01\\
59.8	0.01\\
59.81	0.01\\
59.82	0.01\\
59.83	0.01\\
59.84	0.01\\
59.85	0.01\\
59.86	0.01\\
59.87	0.01\\
59.88	0.01\\
59.89	0.01\\
59.9	0.01\\
59.91	0.01\\
59.92	0.01\\
59.93	0.01\\
59.94	0.01\\
59.95	0.01\\
59.96	0.01\\
59.97	0.01\\
59.98	0.01\\
59.99	0.01\\
60	0.01\\
60.01	0.01\\
60.02	0.01\\
60.03	0.01\\
60.04	0.01\\
60.05	0.01\\
60.06	0.01\\
60.07	0.01\\
60.08	0.01\\
60.09	0.01\\
60.1	0.01\\
60.11	0.01\\
60.12	0.01\\
60.13	0.01\\
60.14	0.01\\
60.15	0.01\\
60.16	0.01\\
60.17	0.01\\
60.18	0.01\\
60.19	0.01\\
60.2	0.01\\
60.21	0.01\\
60.22	0.01\\
60.23	0.01\\
60.24	0.01\\
60.25	0.01\\
60.26	0.01\\
60.27	0.01\\
60.28	0.01\\
60.29	0.01\\
60.3	0.01\\
60.31	0.01\\
60.32	0.01\\
60.33	0.01\\
60.34	0.01\\
60.35	0.01\\
60.36	0.01\\
60.37	0.01\\
60.38	0.01\\
60.39	0.01\\
60.4	0.01\\
60.41	0.01\\
60.42	0.01\\
60.43	0.01\\
60.44	0.01\\
60.45	0.01\\
60.46	0.01\\
60.47	0.01\\
60.48	0.01\\
60.49	0.01\\
60.5	0.01\\
60.51	0.01\\
60.52	0.01\\
60.53	0.01\\
60.54	0.01\\
60.55	0.01\\
60.56	0.01\\
60.57	0.01\\
60.58	0.01\\
60.59	0.01\\
60.6	0.01\\
60.61	0.01\\
60.62	0.01\\
60.63	0.01\\
60.64	0.01\\
60.65	0.01\\
60.66	0.01\\
60.67	0.01\\
60.68	0.01\\
60.69	0.01\\
60.7	0.01\\
60.71	0.01\\
60.72	0.01\\
60.73	0.01\\
60.74	0.01\\
60.75	0.01\\
60.76	0.01\\
60.77	0.01\\
60.78	0.01\\
60.79	0.01\\
60.8	0.01\\
60.81	0.01\\
60.82	0.01\\
60.83	0.01\\
60.84	0.01\\
60.85	0.01\\
60.86	0.01\\
60.87	0.01\\
60.88	0.01\\
60.89	0.01\\
60.9	0.01\\
60.91	0.01\\
60.92	0.01\\
60.93	0.01\\
60.94	0.01\\
60.95	0.01\\
60.96	0.01\\
60.97	0.01\\
60.98	0.01\\
60.99	0.01\\
61	0.01\\
61.01	0.01\\
61.02	0.01\\
61.03	0.01\\
61.04	0.01\\
61.05	0.01\\
61.06	0.01\\
61.07	0.01\\
61.08	0.01\\
61.09	0.01\\
61.1	0.01\\
61.11	0.01\\
61.12	0.01\\
61.13	0.01\\
61.14	0.01\\
61.15	0.01\\
61.16	0.01\\
61.17	0.01\\
61.18	0.01\\
61.19	0.01\\
61.2	0.01\\
61.21	0.01\\
61.22	0.01\\
61.23	0.01\\
61.24	0.01\\
61.25	0.01\\
61.26	0.01\\
61.27	0.01\\
61.28	0.01\\
61.29	0.01\\
61.3	0.01\\
61.31	0.01\\
61.32	0.01\\
61.33	0.01\\
61.34	0.01\\
61.35	0.01\\
61.36	0.01\\
61.37	0.01\\
61.38	0.01\\
61.39	0.01\\
61.4	0.01\\
61.41	0.01\\
61.42	0.01\\
61.43	0.01\\
61.44	0.01\\
61.45	0.01\\
61.46	0.01\\
61.47	0.01\\
61.48	0.01\\
61.49	0.01\\
61.5	0.01\\
61.51	0.01\\
61.52	0.01\\
61.53	0.01\\
61.54	0.01\\
61.55	0.01\\
61.56	0.01\\
61.57	0.01\\
61.58	0.01\\
61.59	0.01\\
61.6	0.01\\
61.61	0.01\\
61.62	0.01\\
61.63	0.01\\
61.64	0.01\\
61.65	0.01\\
61.66	0.01\\
61.67	0.01\\
61.68	0.01\\
61.69	0.01\\
61.7	0.01\\
61.71	0.01\\
61.72	0.01\\
61.73	0.01\\
61.74	0.01\\
61.75	0.01\\
61.76	0.01\\
61.77	0.01\\
61.78	0.01\\
61.79	0.01\\
61.8	0.01\\
61.81	0.01\\
61.82	0.01\\
61.83	0.01\\
61.84	0.01\\
61.85	0.01\\
61.86	0.01\\
61.87	0.01\\
61.88	0.01\\
61.89	0.01\\
61.9	0.01\\
61.91	0.01\\
61.92	0.01\\
61.93	0.01\\
61.94	0.01\\
61.95	0.01\\
61.96	0.01\\
61.97	0.01\\
61.98	0.01\\
61.99	0.01\\
62	0.01\\
62.01	0.01\\
62.02	0.01\\
62.03	0.01\\
62.04	0.01\\
62.05	0.01\\
62.06	0.01\\
62.07	0.01\\
62.08	0.01\\
62.09	0.01\\
62.1	0.01\\
62.11	0.01\\
62.12	0.01\\
62.13	0.01\\
62.14	0.01\\
62.15	0.01\\
62.16	0.01\\
62.17	0.01\\
62.18	0.01\\
62.19	0.01\\
62.2	0.01\\
62.21	0.01\\
62.22	0.01\\
62.23	0.01\\
62.24	0.01\\
62.25	0.01\\
62.26	0.01\\
62.27	0.01\\
62.28	0.01\\
62.29	0.01\\
62.3	0.01\\
62.31	0.01\\
62.32	0.01\\
62.33	0.01\\
62.34	0.01\\
62.35	0.01\\
62.36	0.01\\
62.37	0.01\\
62.38	0.01\\
62.39	0.01\\
62.4	0.01\\
62.41	0.01\\
62.42	0.01\\
62.43	0.01\\
62.44	0.01\\
62.45	0.01\\
62.46	0.01\\
62.47	0.01\\
62.48	0.01\\
62.49	0.01\\
62.5	0.01\\
62.51	0.01\\
62.52	0.01\\
62.53	0.01\\
62.54	0.01\\
62.55	0.01\\
62.56	0.01\\
62.57	0.01\\
62.58	0.01\\
62.59	0.01\\
62.6	0.01\\
62.61	0.01\\
62.62	0.01\\
62.63	0.01\\
62.64	0.01\\
62.65	0.01\\
62.66	0.01\\
62.67	0.01\\
62.68	0.01\\
62.69	0.01\\
62.7	0.01\\
62.71	0.01\\
62.72	0.01\\
62.73	0.01\\
62.74	0.01\\
62.75	0.01\\
62.76	0.01\\
62.77	0.01\\
62.78	0.01\\
62.79	0.01\\
62.8	0.01\\
62.81	0.01\\
62.82	0.01\\
62.83	0.01\\
62.84	0.01\\
62.85	0.01\\
62.86	0.01\\
62.87	0.01\\
62.88	0.01\\
62.89	0.01\\
62.9	0.01\\
62.91	0.01\\
62.92	0.01\\
62.93	0.01\\
62.94	0.01\\
62.95	0.01\\
62.96	0.01\\
62.97	0.01\\
62.98	0.01\\
62.99	0.01\\
63	0.01\\
63.01	0.01\\
63.02	0.01\\
63.03	0.01\\
63.04	0.01\\
63.05	0.01\\
63.06	0.01\\
63.07	0.01\\
63.08	0.01\\
63.09	0.01\\
63.1	0.01\\
63.11	0.01\\
63.12	0.01\\
63.13	0.01\\
63.14	0.01\\
63.15	0.01\\
63.16	0.01\\
63.17	0.01\\
63.18	0.01\\
63.19	0.01\\
63.2	0.01\\
63.21	0.01\\
63.22	0.01\\
63.23	0.01\\
63.24	0.01\\
63.25	0.01\\
63.26	0.01\\
63.27	0.01\\
63.28	0.01\\
63.29	0.01\\
63.3	0.01\\
63.31	0.01\\
63.32	0.01\\
63.33	0.01\\
63.34	0.01\\
63.35	0.01\\
63.36	0.01\\
63.37	0.01\\
63.38	0.01\\
63.39	0.01\\
63.4	0.01\\
63.41	0.01\\
63.42	0.01\\
63.43	0.01\\
63.44	0.01\\
63.45	0.01\\
63.46	0.01\\
63.47	0.01\\
63.48	0.01\\
63.49	0.01\\
63.5	0.01\\
63.51	0.01\\
63.52	0.01\\
63.53	0.01\\
63.54	0.01\\
63.55	0.01\\
63.56	0.01\\
63.57	0.01\\
63.58	0.01\\
63.59	0.01\\
63.6	0.01\\
63.61	0.01\\
63.62	0.01\\
63.63	0.01\\
63.64	0.01\\
63.65	0.01\\
63.66	0.01\\
63.67	0.01\\
63.68	0.01\\
63.69	0.01\\
63.7	0.01\\
63.71	0.01\\
63.72	0.01\\
63.73	0.01\\
63.74	0.01\\
63.75	0.01\\
63.76	0.01\\
63.77	0.01\\
63.78	0.01\\
63.79	0.01\\
63.8	0.01\\
63.81	0.01\\
63.82	0.01\\
63.83	0.01\\
63.84	0.01\\
63.85	0.01\\
63.86	0.01\\
63.87	0.01\\
63.88	0.01\\
63.89	0.01\\
63.9	0.01\\
63.91	0.01\\
63.92	0.01\\
63.93	0.01\\
63.94	0.01\\
63.95	0.01\\
63.96	0.01\\
63.97	0.01\\
63.98	0.01\\
63.99	0.01\\
64	0.01\\
64.01	0.01\\
64.02	0.01\\
64.03	0.01\\
64.04	0.01\\
64.05	0.01\\
64.06	0.01\\
64.07	0.01\\
64.08	0.01\\
64.09	0.01\\
64.1	0.01\\
64.11	0.01\\
64.12	0.01\\
64.13	0.01\\
64.14	0.01\\
64.15	0.01\\
64.16	0.01\\
64.17	0.01\\
64.18	0.01\\
64.19	0.01\\
64.2	0.01\\
64.21	0.01\\
64.22	0.01\\
64.23	0.01\\
64.24	0.01\\
64.25	0.01\\
64.26	0.01\\
64.27	0.01\\
64.28	0.01\\
64.29	0.01\\
64.3	0.01\\
64.31	0.01\\
64.32	0.01\\
64.33	0.01\\
64.34	0.01\\
64.35	0.01\\
64.36	0.01\\
64.37	0.01\\
64.38	0.01\\
64.39	0.01\\
64.4	0.01\\
64.41	0.01\\
64.42	0.01\\
64.43	0.01\\
64.44	0.01\\
64.45	0.01\\
64.46	0.01\\
64.47	0.01\\
64.48	0.01\\
64.49	0.01\\
64.5	0.01\\
64.51	0.01\\
64.52	0.01\\
64.53	0.01\\
64.54	0.01\\
64.55	0.01\\
64.56	0.01\\
64.57	0.01\\
64.58	0.01\\
64.59	0.01\\
64.6	0.01\\
64.61	0.01\\
64.62	0.01\\
64.63	0.01\\
64.64	0.01\\
64.65	0.01\\
64.66	0.01\\
64.67	0.01\\
64.68	0.01\\
64.69	0.01\\
64.7	0.01\\
64.71	0.01\\
64.72	0.01\\
64.73	0.01\\
64.74	0.01\\
64.75	0.01\\
64.76	0.01\\
64.77	0.01\\
64.78	0.01\\
64.79	0.01\\
64.8	0.01\\
64.81	0.01\\
64.82	0.01\\
64.83	0.01\\
64.84	0.01\\
64.85	0.01\\
64.86	0.01\\
64.87	0.01\\
64.88	0.01\\
64.89	0.01\\
64.9	0.01\\
64.91	0.01\\
64.92	0.01\\
64.93	0.01\\
64.94	0.01\\
64.95	0.01\\
64.96	0.01\\
64.97	0.01\\
64.98	0.01\\
64.99	0.01\\
65	0.01\\
65.01	0.01\\
65.02	0.01\\
65.03	0.01\\
65.04	0.01\\
65.05	0.01\\
65.06	0.01\\
65.07	0.01\\
65.08	0.01\\
65.09	0.01\\
65.1	0.01\\
65.11	0.01\\
65.12	0.01\\
65.13	0.01\\
65.14	0.01\\
65.15	0.01\\
65.16	0.01\\
65.17	0.01\\
65.18	0.01\\
65.19	0.01\\
65.2	0.01\\
65.21	0.01\\
65.22	0.01\\
65.23	0.01\\
65.24	0.01\\
65.25	0.01\\
65.26	0.01\\
65.27	0.01\\
65.28	0.01\\
65.29	0.01\\
65.3	0.01\\
65.31	0.01\\
65.32	0.01\\
65.33	0.01\\
65.34	0.01\\
65.35	0.01\\
65.36	0.01\\
65.37	0.01\\
65.38	0.01\\
65.39	0.01\\
65.4	0.01\\
65.41	0.01\\
65.42	0.01\\
65.43	0.01\\
65.44	0.01\\
65.45	0.01\\
65.46	0.01\\
65.47	0.01\\
65.48	0.01\\
65.49	0.01\\
65.5	0.01\\
65.51	0.01\\
65.52	0.01\\
65.53	0.01\\
65.54	0.01\\
65.55	0.01\\
65.56	0.01\\
65.57	0.01\\
65.58	0.01\\
65.59	0.01\\
65.6	0.01\\
65.61	0.01\\
65.62	0.01\\
65.63	0.01\\
65.64	0.01\\
65.65	0.01\\
65.66	0.01\\
65.67	0.01\\
65.68	0.01\\
65.69	0.01\\
65.7	0.01\\
65.71	0.01\\
65.72	0.01\\
65.73	0.01\\
65.74	0.01\\
65.75	0.01\\
65.76	0.01\\
65.77	0.01\\
65.78	0.01\\
65.79	0.01\\
65.8	0.01\\
65.81	0.01\\
65.82	0.01\\
65.83	0.01\\
65.84	0.01\\
65.85	0.01\\
65.86	0.01\\
65.87	0.01\\
65.88	0.01\\
65.89	0.01\\
65.9	0.01\\
65.91	0.01\\
65.92	0.01\\
65.93	0.01\\
65.94	0.01\\
65.95	0.01\\
65.96	0.01\\
65.97	0.01\\
65.98	0.01\\
65.99	0.01\\
66	0.01\\
66.01	0.01\\
66.02	0.01\\
66.03	0.01\\
66.04	0.01\\
66.05	0.01\\
66.06	0.01\\
66.07	0.01\\
66.08	0.01\\
66.09	0.01\\
66.1	0.01\\
66.11	0.01\\
66.12	0.01\\
66.13	0.01\\
66.14	0.01\\
66.15	0.01\\
66.16	0.01\\
66.17	0.01\\
66.18	0.01\\
66.19	0.01\\
66.2	0.01\\
66.21	0.01\\
66.22	0.01\\
66.23	0.01\\
66.24	0.01\\
66.25	0.01\\
66.26	0.01\\
66.27	0.01\\
66.28	0.01\\
66.29	0.01\\
66.3	0.01\\
66.31	0.01\\
66.32	0.01\\
66.33	0.01\\
66.34	0.01\\
66.35	0.01\\
66.36	0.01\\
66.37	0.01\\
66.38	0.01\\
66.39	0.01\\
66.4	0.01\\
66.41	0.01\\
66.42	0.01\\
66.43	0.01\\
66.44	0.01\\
66.45	0.01\\
66.46	0.01\\
66.47	0.01\\
66.48	0.01\\
66.49	0.01\\
66.5	0.01\\
66.51	0.01\\
66.52	0.01\\
66.53	0.01\\
66.54	0.01\\
66.55	0.01\\
66.56	0.01\\
66.57	0.01\\
66.58	0.01\\
66.59	0.01\\
66.6	0.01\\
66.61	0.01\\
66.62	0.01\\
66.63	0.01\\
66.64	0.01\\
66.65	0.01\\
66.66	0.01\\
66.67	0.01\\
66.68	0.01\\
66.69	0.01\\
66.7	0.01\\
66.71	0.01\\
66.72	0.01\\
66.73	0.01\\
66.74	0.01\\
66.75	0.01\\
66.76	0.01\\
66.77	0.01\\
66.78	0.01\\
66.79	0.01\\
66.8	0.01\\
66.81	0.01\\
66.82	0.01\\
66.83	0.01\\
66.84	0.01\\
66.85	0.01\\
66.86	0.01\\
66.87	0.01\\
66.88	0.01\\
66.89	0.01\\
66.9	0.01\\
66.91	0.01\\
66.92	0.01\\
66.93	0.01\\
66.94	0.01\\
66.95	0.01\\
66.96	0.01\\
66.97	0.01\\
66.98	0.01\\
66.99	0.01\\
67	0.01\\
67.01	0.01\\
67.02	0.01\\
67.03	0.01\\
67.04	0.01\\
67.05	0.01\\
67.06	0.01\\
67.07	0.01\\
67.08	0.01\\
67.09	0.01\\
67.1	0.01\\
67.11	0.01\\
67.12	0.01\\
67.13	0.01\\
67.14	0.01\\
67.15	0.01\\
67.16	0.01\\
67.17	0.01\\
67.18	0.01\\
67.19	0.01\\
67.2	0.01\\
67.21	0.01\\
67.22	0.01\\
67.23	0.01\\
67.24	0.01\\
67.25	0.01\\
67.26	0.01\\
67.27	0.01\\
67.28	0.01\\
67.29	0.01\\
67.3	0.01\\
67.31	0.01\\
67.32	0.01\\
67.33	0.01\\
67.34	0.01\\
67.35	0.01\\
67.36	0.01\\
67.37	0.01\\
67.38	0.01\\
67.39	0.01\\
67.4	0.01\\
67.41	0.01\\
67.42	0.01\\
67.43	0.01\\
67.44	0.01\\
67.45	0.01\\
67.46	0.01\\
67.47	0.01\\
67.48	0.01\\
67.49	0.01\\
67.5	0.01\\
67.51	0.01\\
67.52	0.01\\
67.53	0.01\\
67.54	0.01\\
67.55	0.01\\
67.56	0.01\\
67.57	0.01\\
67.58	0.01\\
67.59	0.01\\
67.6	0.01\\
67.61	0.01\\
67.62	0.01\\
67.63	0.01\\
67.64	0.01\\
67.65	0.01\\
67.66	0.01\\
67.67	0.01\\
67.68	0.01\\
67.69	0.01\\
67.7	0.01\\
67.71	0.01\\
67.72	0.01\\
67.73	0.01\\
67.74	0.01\\
67.75	0.01\\
67.76	0.01\\
67.77	0.01\\
67.78	0.01\\
67.79	0.01\\
67.8	0.01\\
67.81	0.01\\
67.82	0.01\\
67.83	0.01\\
67.84	0.01\\
67.85	0.01\\
67.86	0.01\\
67.87	0.01\\
67.88	0.01\\
67.89	0.01\\
67.9	0.01\\
67.91	0.01\\
67.92	0.01\\
67.93	0.01\\
67.94	0.01\\
67.95	0.01\\
67.96	0.01\\
67.97	0.01\\
67.98	0.01\\
67.99	0.01\\
68	0.01\\
68.01	0.01\\
68.02	0.01\\
68.03	0.01\\
68.04	0.01\\
68.05	0.01\\
68.06	0.01\\
68.07	0.01\\
68.08	0.01\\
68.09	0.01\\
68.1	0.01\\
68.11	0.01\\
68.12	0.01\\
68.13	0.01\\
68.14	0.01\\
68.15	0.01\\
68.16	0.01\\
68.17	0.01\\
68.18	0.01\\
68.19	0.01\\
68.2	0.01\\
68.21	0.01\\
68.22	0.01\\
68.23	0.01\\
68.24	0.01\\
68.25	0.01\\
68.26	0.01\\
68.27	0.01\\
68.28	0.01\\
68.29	0.01\\
68.3	0.01\\
68.31	0.01\\
68.32	0.01\\
68.33	0.01\\
68.34	0.01\\
68.35	0.01\\
68.36	0.01\\
68.37	0.01\\
68.38	0.01\\
68.39	0.01\\
68.4	0.01\\
68.41	0.01\\
68.42	0.01\\
68.43	0.01\\
68.44	0.01\\
68.45	0.01\\
68.46	0.01\\
68.47	0.01\\
68.48	0.01\\
68.49	0.01\\
68.5	0.01\\
68.51	0.01\\
68.52	0.01\\
68.53	0.01\\
68.54	0.01\\
68.55	0.01\\
68.56	0.01\\
68.57	0.01\\
68.58	0.01\\
68.59	0.01\\
68.6	0.01\\
68.61	0.01\\
68.62	0.01\\
68.63	0.01\\
68.64	0.01\\
68.65	0.01\\
68.66	0.01\\
68.67	0.01\\
68.68	0.01\\
68.69	0.01\\
68.7	0.01\\
68.71	0.01\\
68.72	0.01\\
68.73	0.01\\
68.74	0.01\\
68.75	0.01\\
68.76	0.01\\
68.77	0.01\\
68.78	0.01\\
68.79	0.01\\
68.8	0.01\\
68.81	0.01\\
68.82	0.01\\
68.83	0.01\\
68.84	0.01\\
68.85	0.01\\
68.86	0.01\\
68.87	0.01\\
68.88	0.01\\
68.89	0.01\\
68.9	0.01\\
68.91	0.01\\
68.92	0.01\\
68.93	0.01\\
68.94	0.01\\
68.95	0.01\\
68.96	0.01\\
68.97	0.01\\
68.98	0.01\\
68.99	0.01\\
69	0.01\\
69.01	0.01\\
69.02	0.01\\
69.03	0.01\\
69.04	0.01\\
69.05	0.01\\
69.06	0.01\\
69.07	0.01\\
69.08	0.01\\
69.09	0.01\\
69.1	0.01\\
69.11	0.01\\
69.12	0.01\\
69.13	0.01\\
69.14	0.01\\
69.15	0.01\\
69.16	0.01\\
69.17	0.01\\
69.18	0.01\\
69.19	0.01\\
69.2	0.01\\
69.21	0.01\\
69.22	0.01\\
69.23	0.01\\
69.24	0.01\\
69.25	0.01\\
69.26	0.01\\
69.27	0.01\\
69.28	0.01\\
69.29	0.01\\
69.3	0.01\\
69.31	0.01\\
69.32	0.01\\
69.33	0.01\\
69.34	0.01\\
69.35	0.01\\
69.36	0.01\\
69.37	0.01\\
69.38	0.01\\
69.39	0.01\\
69.4	0.01\\
69.41	0.01\\
69.42	0.01\\
69.43	0.01\\
69.44	0.01\\
69.45	0.01\\
69.46	0.01\\
69.47	0.01\\
69.48	0.01\\
69.49	0.01\\
69.5	0.01\\
69.51	0.01\\
69.52	0.01\\
69.53	0.01\\
69.54	0.01\\
69.55	0.01\\
69.56	0.01\\
69.57	0.01\\
69.58	0.01\\
69.59	0.01\\
69.6	0.01\\
69.61	0.01\\
69.62	0.01\\
69.63	0.01\\
69.64	0.01\\
69.65	0.01\\
69.66	0.01\\
69.67	0.01\\
69.68	0.01\\
69.69	0.01\\
69.7	0.01\\
69.71	0.01\\
69.72	0.01\\
69.73	0.01\\
69.74	0.01\\
69.75	0.01\\
69.76	0.01\\
69.77	0.01\\
69.78	0.01\\
69.79	0.01\\
69.8	0.01\\
69.81	0.01\\
69.82	0.01\\
69.83	0.01\\
69.84	0.01\\
69.85	0.01\\
69.86	0.01\\
69.87	0.01\\
69.88	0.01\\
69.89	0.01\\
69.9	0.01\\
69.91	0.01\\
69.92	0.01\\
69.93	0.01\\
69.94	0.01\\
69.95	0.01\\
69.96	0.01\\
69.97	0.01\\
69.98	0.01\\
69.99	0.01\\
70	0.01\\
70.01	0.01\\
70.02	0.01\\
70.03	0.01\\
70.04	0.01\\
70.05	0.01\\
70.06	0.01\\
70.07	0.01\\
70.08	0.01\\
70.09	0.01\\
70.1	0.01\\
70.11	0.01\\
70.12	0.01\\
70.13	0.01\\
70.14	0.01\\
70.15	0.01\\
70.16	0.01\\
70.17	0.01\\
70.18	0.01\\
70.19	0.01\\
70.2	0.01\\
70.21	0.01\\
70.22	0.01\\
70.23	0.01\\
70.24	0.01\\
70.25	0.01\\
70.26	0.01\\
70.27	0.01\\
70.28	0.01\\
70.29	0.01\\
70.3	0.01\\
70.31	0.01\\
70.32	0.01\\
70.33	0.01\\
70.34	0.01\\
70.35	0.01\\
70.36	0.01\\
70.37	0.01\\
70.38	0.01\\
70.39	0.01\\
70.4	0.01\\
70.41	0.01\\
70.42	0.01\\
70.43	0.01\\
70.44	0.01\\
70.45	0.01\\
70.46	0.01\\
70.47	0.01\\
70.48	0.01\\
70.49	0.01\\
70.5	0.01\\
70.51	0.01\\
70.52	0.01\\
70.53	0.01\\
70.54	0.01\\
70.55	0.01\\
70.56	0.01\\
70.57	0.01\\
70.58	0.01\\
70.59	0.01\\
70.6	0.01\\
70.61	0.01\\
70.62	0.01\\
70.63	0.01\\
70.64	0.01\\
70.65	0.01\\
70.66	0.01\\
70.67	0.01\\
70.68	0.01\\
70.69	0.01\\
70.7	0.01\\
70.71	0.01\\
70.72	0.01\\
70.73	0.01\\
70.74	0.01\\
70.75	0.01\\
70.76	0.01\\
70.77	0.01\\
70.78	0.01\\
70.79	0.01\\
70.8	0.01\\
70.81	0.01\\
70.82	0.01\\
70.83	0.01\\
70.84	0.01\\
70.85	0.01\\
70.86	0.01\\
70.87	0.01\\
70.88	0.01\\
70.89	0.01\\
70.9	0.01\\
70.91	0.01\\
70.92	0.01\\
70.93	0.01\\
70.94	0.01\\
70.95	0.01\\
70.96	0.01\\
70.97	0.01\\
70.98	0.01\\
70.99	0.01\\
71	0.01\\
71.01	0.01\\
71.02	0.01\\
71.03	0.01\\
71.04	0.01\\
71.05	0.01\\
71.06	0.01\\
71.07	0.01\\
71.08	0.01\\
71.09	0.01\\
71.1	0.01\\
71.11	0.01\\
71.12	0.01\\
71.13	0.01\\
71.14	0.01\\
71.15	0.01\\
71.16	0.01\\
71.17	0.01\\
71.18	0.01\\
71.19	0.01\\
71.2	0.01\\
71.21	0.01\\
71.22	0.01\\
71.23	0.01\\
71.24	0.01\\
71.25	0.01\\
71.26	0.01\\
71.27	0.01\\
71.28	0.01\\
71.29	0.01\\
71.3	0.01\\
71.31	0.01\\
71.32	0.01\\
71.33	0.01\\
71.34	0.01\\
71.35	0.01\\
71.36	0.01\\
71.37	0.01\\
71.38	0.01\\
71.39	0.01\\
71.4	0.01\\
71.41	0.01\\
71.42	0.01\\
71.43	0.01\\
71.44	0.01\\
71.45	0.01\\
71.46	0.01\\
71.47	0.01\\
71.48	0.01\\
71.49	0.01\\
71.5	0.01\\
71.51	0.01\\
71.52	0.01\\
71.53	0.01\\
71.54	0.01\\
71.55	0.01\\
71.56	0.01\\
71.57	0.01\\
71.58	0.01\\
71.59	0.01\\
71.6	0.01\\
71.61	0.01\\
71.62	0.01\\
71.63	0.01\\
71.64	0.01\\
71.65	0.01\\
71.66	0.01\\
71.67	0.01\\
71.68	0.01\\
71.69	0.01\\
71.7	0.01\\
71.71	0.01\\
71.72	0.01\\
71.73	0.01\\
71.74	0.01\\
71.75	0.01\\
71.76	0.01\\
71.77	0.01\\
71.78	0.01\\
71.79	0.01\\
71.8	0.01\\
71.81	0.01\\
71.82	0.01\\
71.83	0.01\\
71.84	0.01\\
71.85	0.01\\
71.86	0.01\\
71.87	0.01\\
71.88	0.01\\
71.89	0.01\\
71.9	0.01\\
71.91	0.01\\
71.92	0.01\\
71.93	0.01\\
71.94	0.01\\
71.95	0.01\\
71.96	0.01\\
71.97	0.01\\
71.98	0.01\\
71.99	0.01\\
72	0.01\\
72.01	0.01\\
72.02	0.01\\
72.03	0.01\\
72.04	0.01\\
72.05	0.01\\
72.06	0.01\\
72.07	0.01\\
72.08	0.01\\
72.09	0.01\\
72.1	0.01\\
72.11	0.01\\
72.12	0.01\\
72.13	0.01\\
72.14	0.01\\
72.15	0.01\\
72.16	0.01\\
72.17	0.01\\
72.18	0.01\\
72.19	0.01\\
72.2	0.01\\
72.21	0.01\\
72.22	0.01\\
72.23	0.01\\
72.24	0.01\\
72.25	0.01\\
72.26	0.01\\
72.27	0.01\\
72.28	0.01\\
72.29	0.01\\
72.3	0.01\\
72.31	0.01\\
72.32	0.01\\
72.33	0.01\\
72.34	0.01\\
72.35	0.01\\
72.36	0.01\\
72.37	0.01\\
72.38	0.01\\
72.39	0.01\\
72.4	0.01\\
72.41	0.01\\
72.42	0.01\\
72.43	0.01\\
72.44	0.01\\
72.45	0.01\\
72.46	0.01\\
72.47	0.01\\
72.48	0.01\\
72.49	0.01\\
72.5	0.01\\
72.51	0.01\\
72.52	0.01\\
72.53	0.01\\
72.54	0.01\\
72.55	0.01\\
72.56	0.01\\
72.57	0.01\\
72.58	0.01\\
72.59	0.01\\
72.6	0.01\\
72.61	0.01\\
72.62	0.01\\
72.63	0.01\\
72.64	0.01\\
72.65	0.01\\
72.66	0.01\\
72.67	0.01\\
72.68	0.01\\
72.69	0.01\\
72.7	0.01\\
72.71	0.01\\
72.72	0.01\\
72.73	0.01\\
72.74	0.01\\
72.75	0.01\\
72.76	0.01\\
72.77	0.01\\
72.78	0.01\\
72.79	0.01\\
72.8	0.01\\
72.81	0.01\\
72.82	0.01\\
72.83	0.01\\
72.84	0.01\\
72.85	0.01\\
72.86	0.01\\
72.87	0.01\\
72.88	0.01\\
72.89	0.01\\
72.9	0.01\\
72.91	0.01\\
72.92	0.01\\
72.93	0.01\\
72.94	0.01\\
72.95	0.01\\
72.96	0.01\\
72.97	0.01\\
72.98	0.01\\
72.99	0.01\\
73	0.01\\
73.01	0.01\\
73.02	0.01\\
73.03	0.01\\
73.04	0.01\\
73.05	0.01\\
73.06	0.01\\
73.07	0.01\\
73.08	0.01\\
73.09	0.01\\
73.1	0.01\\
73.11	0.01\\
73.12	0.01\\
73.13	0.01\\
73.14	0.01\\
73.15	0.01\\
73.16	0.01\\
73.17	0.01\\
73.18	0.01\\
73.19	0.01\\
73.2	0.01\\
73.21	0.01\\
73.22	0.01\\
73.23	0.01\\
73.24	0.01\\
73.25	0.01\\
73.26	0.01\\
73.27	0.01\\
73.28	0.01\\
73.29	0.01\\
73.3	0.01\\
73.31	0.01\\
73.32	0.01\\
73.33	0.01\\
73.34	0.01\\
73.35	0.01\\
73.36	0.01\\
73.37	0.01\\
73.38	0.01\\
73.39	0.01\\
73.4	0.01\\
73.41	0.01\\
73.42	0.01\\
73.43	0.01\\
73.44	0.01\\
73.45	0.01\\
73.46	0.01\\
73.47	0.01\\
73.48	0.01\\
73.49	0.01\\
73.5	0.01\\
73.51	0.01\\
73.52	0.01\\
73.53	0.01\\
73.54	0.01\\
73.55	0.01\\
73.56	0.01\\
73.57	0.01\\
73.58	0.01\\
73.59	0.01\\
73.6	0.01\\
73.61	0.01\\
73.62	0.01\\
73.63	0.01\\
73.64	0.01\\
73.65	0.01\\
73.66	0.01\\
73.67	0.01\\
73.68	0.01\\
73.69	0.01\\
73.7	0.01\\
73.71	0.01\\
73.72	0.01\\
73.73	0.01\\
73.74	0.01\\
73.75	0.01\\
73.76	0.01\\
73.77	0.01\\
73.78	0.01\\
73.79	0.01\\
73.8	0.01\\
73.81	0.01\\
73.82	0.01\\
73.83	0.01\\
73.84	0.01\\
73.85	0.01\\
73.86	0.01\\
73.87	0.01\\
73.88	0.01\\
73.89	0.01\\
73.9	0.01\\
73.91	0.01\\
73.92	0.01\\
73.93	0.01\\
73.94	0.01\\
73.95	0.01\\
73.96	0.01\\
73.97	0.01\\
73.98	0.01\\
73.99	0.01\\
74	0.01\\
74.01	0.01\\
74.02	0.01\\
74.03	0.01\\
74.04	0.01\\
74.05	0.01\\
74.06	0.01\\
74.07	0.01\\
74.08	0.01\\
74.09	0.01\\
74.1	0.01\\
74.11	0.01\\
74.12	0.01\\
74.13	0.01\\
74.14	0.01\\
74.15	0.01\\
74.16	0.01\\
74.17	0.01\\
74.18	0.01\\
74.19	0.01\\
74.2	0.01\\
74.21	0.01\\
74.22	0.01\\
74.23	0.01\\
74.24	0.01\\
74.25	0.01\\
74.26	0.01\\
74.27	0.01\\
74.28	0.01\\
74.29	0.01\\
74.3	0.01\\
74.31	0.01\\
74.32	0.01\\
74.33	0.01\\
74.34	0.01\\
74.35	0.01\\
74.36	0.01\\
74.37	0.01\\
74.38	0.01\\
74.39	0.01\\
74.4	0.01\\
74.41	0.01\\
74.42	0.01\\
74.43	0.01\\
74.44	0.01\\
74.45	0.01\\
74.46	0.01\\
74.47	0.01\\
74.48	0.01\\
74.49	0.01\\
74.5	0.01\\
74.51	0.01\\
74.52	0.01\\
74.53	0.01\\
74.54	0.01\\
74.55	0.01\\
74.56	0.01\\
74.57	0.01\\
74.58	0.01\\
74.59	0.01\\
74.6	0.01\\
74.61	0.01\\
74.62	0.01\\
74.63	0.01\\
74.64	0.01\\
74.65	0.01\\
74.66	0.01\\
74.67	0.01\\
74.68	0.01\\
74.69	0.01\\
74.7	0.01\\
74.71	0.01\\
74.72	0.01\\
74.73	0.01\\
74.74	0.01\\
74.75	0.01\\
74.76	0.01\\
74.77	0.01\\
74.78	0.01\\
74.79	0.01\\
74.8	0.01\\
74.81	0.01\\
74.82	0.01\\
74.83	0.01\\
74.84	0.01\\
74.85	0.01\\
74.86	0.01\\
74.87	0.01\\
74.88	0.01\\
74.89	0.01\\
74.9	0.01\\
74.91	0.01\\
74.92	0.01\\
74.93	0.01\\
74.94	0.01\\
74.95	0.01\\
74.96	0.01\\
74.97	0.01\\
74.98	0.01\\
74.99	0.01\\
75	0.01\\
75.01	0.01\\
75.02	0.01\\
75.03	0.01\\
75.04	0.01\\
75.05	0.01\\
75.06	0.01\\
75.07	0.01\\
75.08	0.01\\
75.09	0.01\\
75.1	0.01\\
75.11	0.01\\
75.12	0.01\\
75.13	0.01\\
75.14	0.01\\
75.15	0.01\\
75.16	0.01\\
75.17	0.01\\
75.18	0.01\\
75.19	0.01\\
75.2	0.01\\
75.21	0.01\\
75.22	0.01\\
75.23	0.01\\
75.24	0.01\\
75.25	0.01\\
75.26	0.01\\
75.27	0.01\\
75.28	0.01\\
75.29	0.01\\
75.3	0.01\\
75.31	0.01\\
75.32	0.01\\
75.33	0.01\\
75.34	0.01\\
75.35	0.01\\
75.36	0.01\\
75.37	0.01\\
75.38	0.01\\
75.39	0.01\\
75.4	0.01\\
75.41	0.01\\
75.42	0.01\\
75.43	0.01\\
75.44	0.01\\
75.45	0.01\\
75.46	0.01\\
75.47	0.01\\
75.48	0.01\\
75.49	0.01\\
75.5	0.01\\
75.51	0.01\\
75.52	0.01\\
75.53	0.01\\
75.54	0.01\\
75.55	0.01\\
75.56	0.01\\
75.57	0.01\\
75.58	0.01\\
75.59	0.01\\
75.6	0.01\\
75.61	0.01\\
75.62	0.01\\
75.63	0.01\\
75.64	0.01\\
75.65	0.01\\
75.66	0.01\\
75.67	0.01\\
75.68	0.01\\
75.69	0.01\\
75.7	0.01\\
75.71	0.01\\
75.72	0.01\\
75.73	0.01\\
75.74	0.01\\
75.75	0.01\\
75.76	0.01\\
75.77	0.01\\
75.78	0.01\\
75.79	0.01\\
75.8	0.01\\
75.81	0.01\\
75.82	0.01\\
75.83	0.01\\
75.84	0.01\\
75.85	0.01\\
75.86	0.01\\
75.87	0.01\\
75.88	0.01\\
75.89	0.01\\
75.9	0.01\\
75.91	0.01\\
75.92	0.01\\
75.93	0.01\\
75.94	0.01\\
75.95	0.01\\
75.96	0.01\\
75.97	0.01\\
75.98	0.01\\
75.99	0.01\\
76	0.01\\
76.01	0.01\\
76.02	0.01\\
76.03	0.01\\
76.04	0.01\\
76.05	0.01\\
76.06	0.01\\
76.07	0.01\\
76.08	0.01\\
76.09	0.01\\
76.1	0.01\\
76.11	0.01\\
76.12	0.01\\
76.13	0.01\\
76.14	0.01\\
76.15	0.01\\
76.16	0.01\\
76.17	0.01\\
76.18	0.01\\
76.19	0.01\\
76.2	0.01\\
76.21	0.01\\
76.22	0.01\\
76.23	0.01\\
76.24	0.01\\
76.25	0.01\\
76.26	0.01\\
76.27	0.01\\
76.28	0.01\\
76.29	0.01\\
76.3	0.01\\
76.31	0.01\\
76.32	0.01\\
76.33	0.01\\
76.34	0.01\\
76.35	0.01\\
76.36	0.01\\
76.37	0.01\\
76.38	0.01\\
76.39	0.01\\
76.4	0.01\\
76.41	0.01\\
76.42	0.01\\
76.43	0.01\\
76.44	0.01\\
76.45	0.01\\
76.46	0.01\\
76.47	0.01\\
76.48	0.01\\
76.49	0.01\\
76.5	0.01\\
76.51	0.01\\
76.52	0.01\\
76.53	0.01\\
76.54	0.01\\
76.55	0.01\\
76.56	0.01\\
76.57	0.01\\
76.58	0.01\\
76.59	0.01\\
76.6	0.01\\
76.61	0.01\\
76.62	0.01\\
76.63	0.01\\
76.64	0.01\\
76.65	0.01\\
76.66	0.01\\
76.67	0.01\\
76.68	0.01\\
76.69	0.01\\
76.7	0.01\\
76.71	0.01\\
76.72	0.01\\
76.73	0.01\\
76.74	0.01\\
76.75	0.01\\
76.76	0.01\\
76.77	0.01\\
76.78	0.01\\
76.79	0.01\\
76.8	0.01\\
76.81	0.01\\
76.82	0.01\\
76.83	0.01\\
76.84	0.01\\
76.85	0.01\\
76.86	0.01\\
76.87	0.01\\
76.88	0.01\\
76.89	0.01\\
76.9	0.01\\
76.91	0.01\\
76.92	0.01\\
76.93	0.01\\
76.94	0.01\\
76.95	0.01\\
76.96	0.01\\
76.97	0.01\\
76.98	0.01\\
76.99	0.01\\
77	0.01\\
77.01	0.01\\
77.02	0.01\\
77.03	0.01\\
77.04	0.01\\
77.05	0.01\\
77.06	0.01\\
77.07	0.01\\
77.08	0.01\\
77.09	0.01\\
77.1	0.01\\
77.11	0.01\\
77.12	0.01\\
77.13	0.01\\
77.14	0.01\\
77.15	0.01\\
77.16	0.01\\
77.17	0.01\\
77.18	0.01\\
77.19	0.01\\
77.2	0.01\\
77.21	0.01\\
77.22	0.01\\
77.23	0.01\\
77.24	0.01\\
77.25	0.01\\
77.26	0.01\\
77.27	0.01\\
77.28	0.01\\
77.29	0.01\\
77.3	0.01\\
77.31	0.01\\
77.32	0.01\\
77.33	0.01\\
77.34	0.01\\
77.35	0.01\\
77.36	0.01\\
77.37	0.01\\
77.38	0.01\\
77.39	0.01\\
77.4	0.01\\
77.41	0.01\\
77.42	0.01\\
77.43	0.01\\
77.44	0.01\\
77.45	0.01\\
77.46	0.01\\
77.47	0.01\\
77.48	0.01\\
77.49	0.01\\
77.5	0.01\\
77.51	0.01\\
77.52	0.01\\
77.53	0.01\\
77.54	0.01\\
77.55	0.01\\
77.56	0.01\\
77.57	0.01\\
77.58	0.01\\
77.59	0.01\\
77.6	0.01\\
77.61	0.01\\
77.62	0.01\\
77.63	0.01\\
77.64	0.01\\
77.65	0.01\\
77.66	0.01\\
77.67	0.01\\
77.68	0.01\\
77.69	0.01\\
77.7	0.01\\
77.71	0.01\\
77.72	0.01\\
77.73	0.01\\
77.74	0.01\\
77.75	0.01\\
77.76	0.01\\
77.77	0.01\\
77.78	0.01\\
77.79	0.01\\
77.8	0.01\\
77.81	0.01\\
77.82	0.01\\
77.83	0.01\\
77.84	0.01\\
77.85	0.01\\
77.86	0.01\\
77.87	0.01\\
77.88	0.01\\
77.89	0.01\\
77.9	0.01\\
77.91	0.01\\
77.92	0.01\\
77.93	0.01\\
77.94	0.01\\
77.95	0.01\\
77.96	0.01\\
77.97	0.01\\
77.98	0.01\\
77.99	0.01\\
78	0.01\\
78.01	0.01\\
78.02	0.01\\
78.03	0.01\\
78.04	0.01\\
78.05	0.01\\
78.06	0.01\\
78.07	0.01\\
78.08	0.01\\
78.09	0.01\\
78.1	0.01\\
78.11	0.01\\
78.12	0.01\\
78.13	0.01\\
78.14	0.01\\
78.15	0.01\\
78.16	0.01\\
78.17	0.01\\
78.18	0.01\\
78.19	0.01\\
78.2	0.01\\
78.21	0.01\\
78.22	0.01\\
78.23	0.01\\
78.24	0.01\\
78.25	0.01\\
78.26	0.01\\
78.27	0.01\\
78.28	0.01\\
78.29	0.01\\
78.3	0.01\\
78.31	0.01\\
78.32	0.01\\
78.33	0.01\\
78.34	0.01\\
78.35	0.01\\
78.36	0.01\\
78.37	0.01\\
78.38	0.01\\
78.39	0.01\\
78.4	0.01\\
78.41	0.01\\
78.42	0.01\\
78.43	0.01\\
78.44	0.01\\
78.45	0.01\\
78.46	0.01\\
78.47	0.01\\
78.48	0.01\\
78.49	0.01\\
78.5	0.01\\
78.51	0.01\\
78.52	0.01\\
78.53	0.01\\
78.54	0.01\\
78.55	0.01\\
78.56	0.01\\
78.57	0.01\\
78.58	0.01\\
78.59	0.01\\
78.6	0.01\\
78.61	0.01\\
78.62	0.01\\
78.63	0.01\\
78.64	0.01\\
78.65	0.01\\
78.66	0.01\\
78.67	0.01\\
78.68	0.01\\
78.69	0.01\\
78.7	0.01\\
78.71	0.01\\
78.72	0.01\\
78.73	0.01\\
78.74	0.01\\
78.75	0.01\\
78.76	0.01\\
78.77	0.01\\
78.78	0.01\\
78.79	0.01\\
78.8	0.01\\
78.81	0.01\\
78.82	0.01\\
78.83	0.01\\
78.84	0.01\\
78.85	0.01\\
78.86	0.01\\
78.87	0.01\\
78.88	0.01\\
78.89	0.01\\
78.9	0.01\\
78.91	0.01\\
78.92	0.01\\
78.93	0.01\\
78.94	0.01\\
78.95	0.01\\
78.96	0.01\\
78.97	0.01\\
78.98	0.01\\
78.99	0.01\\
79	0.01\\
79.01	0.01\\
79.02	0.01\\
79.03	0.01\\
79.04	0.01\\
79.05	0.01\\
79.06	0.01\\
79.07	0.01\\
79.08	0.01\\
79.09	0.01\\
79.1	0.01\\
79.11	0.01\\
79.12	0.01\\
79.13	0.01\\
79.14	0.01\\
79.15	0.01\\
79.16	0.01\\
79.17	0.01\\
79.18	0.01\\
79.19	0.01\\
79.2	0.01\\
79.21	0.01\\
79.22	0.01\\
79.23	0.01\\
79.24	0.01\\
79.25	0.01\\
79.26	0.01\\
79.27	0.01\\
79.28	0.01\\
79.29	0.01\\
79.3	0.01\\
79.31	0.01\\
79.32	0.01\\
79.33	0.01\\
79.34	0.01\\
79.35	0.01\\
79.36	0.01\\
79.37	0.01\\
79.38	0.01\\
79.39	0.01\\
79.4	0.01\\
79.41	0.01\\
79.42	0.01\\
79.43	0.01\\
79.44	0.01\\
79.45	0.01\\
79.46	0.01\\
79.47	0.01\\
79.48	0.01\\
79.49	0.01\\
79.5	0.01\\
79.51	0.01\\
79.52	0.01\\
79.53	0.01\\
79.54	0.01\\
79.55	0.01\\
79.56	0.01\\
79.57	0.01\\
79.58	0.01\\
79.59	0.01\\
79.6	0.01\\
79.61	0.01\\
79.62	0.01\\
79.63	0.01\\
79.64	0.01\\
79.65	0.01\\
79.66	0.01\\
79.67	0.01\\
79.68	0.01\\
79.69	0.01\\
79.7	0.01\\
79.71	0.01\\
79.72	0.01\\
79.73	0.01\\
79.74	0.01\\
79.75	0.01\\
79.76	0.01\\
79.77	0.01\\
79.78	0.01\\
79.79	0.01\\
79.8	0.01\\
79.81	0.01\\
79.82	0.01\\
79.83	0.01\\
79.84	0.01\\
79.85	0.01\\
79.86	0.01\\
79.87	0.01\\
79.88	0.01\\
79.89	0.01\\
79.9	0.01\\
79.91	0.01\\
79.92	0.01\\
79.93	0.01\\
79.94	0.01\\
79.95	0.01\\
79.96	0.01\\
79.97	0.01\\
79.98	0.01\\
79.99	0.01\\
80	0.01\\
80.01	0.01\\
};
\addplot [color=mycolor1,dashed]
  table[row sep=crcr]{%
80.01	0.01\\
80.02	0.01\\
80.03	0.01\\
80.04	0.01\\
80.05	0.01\\
80.06	0.01\\
80.07	0.01\\
80.08	0.01\\
80.09	0.01\\
80.1	0.01\\
80.11	0.01\\
80.12	0.01\\
80.13	0.01\\
80.14	0.01\\
80.15	0.01\\
80.16	0.01\\
80.17	0.01\\
80.18	0.01\\
80.19	0.01\\
80.2	0.01\\
80.21	0.01\\
80.22	0.01\\
80.23	0.01\\
80.24	0.01\\
80.25	0.01\\
80.26	0.01\\
80.27	0.01\\
80.28	0.01\\
80.29	0.01\\
80.3	0.01\\
80.31	0.01\\
80.32	0.01\\
80.33	0.01\\
80.34	0.01\\
80.35	0.01\\
80.36	0.01\\
80.37	0.01\\
80.38	0.01\\
80.39	0.01\\
80.4	0.01\\
80.41	0.01\\
80.42	0.01\\
80.43	0.01\\
80.44	0.01\\
80.45	0.01\\
80.46	0.01\\
80.47	0.01\\
80.48	0.01\\
80.49	0.01\\
80.5	0.01\\
80.51	0.01\\
80.52	0.01\\
80.53	0.01\\
80.54	0.01\\
80.55	0.01\\
80.56	0.01\\
80.57	0.01\\
80.58	0.01\\
80.59	0.01\\
80.6	0.01\\
80.61	0.01\\
80.62	0.01\\
80.63	0.01\\
80.64	0.01\\
80.65	0.01\\
80.66	0.01\\
80.67	0.01\\
80.68	0.01\\
80.69	0.01\\
80.7	0.01\\
80.71	0.01\\
80.72	0.01\\
80.73	0.01\\
80.74	0.01\\
80.75	0.01\\
80.76	0.01\\
80.77	0.01\\
80.78	0.01\\
80.79	0.01\\
80.8	0.01\\
80.81	0.01\\
80.82	0.01\\
80.83	0.01\\
80.84	0.01\\
80.85	0.01\\
80.86	0.01\\
80.87	0.01\\
80.88	0.01\\
80.89	0.01\\
80.9	0.01\\
80.91	0.01\\
80.92	0.01\\
80.93	0.01\\
80.94	0.01\\
80.95	0.01\\
80.96	0.01\\
80.97	0.01\\
80.98	0.01\\
80.99	0.01\\
81	0.01\\
81.01	0.01\\
81.02	0.01\\
81.03	0.01\\
81.04	0.01\\
81.05	0.01\\
81.06	0.01\\
81.07	0.01\\
81.08	0.01\\
81.09	0.01\\
81.1	0.01\\
81.11	0.01\\
81.12	0.01\\
81.13	0.01\\
81.14	0.01\\
81.15	0.01\\
81.16	0.01\\
81.17	0.01\\
81.18	0.01\\
81.19	0.01\\
81.2	0.01\\
81.21	0.01\\
81.22	0.01\\
81.23	0.01\\
81.24	0.01\\
81.25	0.01\\
81.26	0.01\\
81.27	0.01\\
81.28	0.01\\
81.29	0.01\\
81.3	0.01\\
81.31	0.01\\
81.32	0.01\\
81.33	0.01\\
81.34	0.01\\
81.35	0.01\\
81.36	0.01\\
81.37	0.01\\
81.38	0.01\\
81.39	0.01\\
81.4	0.01\\
81.41	0.01\\
81.42	0.01\\
81.43	0.01\\
81.44	0.01\\
81.45	0.01\\
81.46	0.01\\
81.47	0.01\\
81.48	0.01\\
81.49	0.01\\
81.5	0.01\\
81.51	0.01\\
81.52	0.01\\
81.53	0.01\\
81.54	0.01\\
81.55	0.01\\
81.56	0.01\\
81.57	0.01\\
81.58	0.01\\
81.59	0.01\\
81.6	0.01\\
81.61	0.01\\
81.62	0.01\\
81.63	0.01\\
81.64	0.01\\
81.65	0.01\\
81.66	0.01\\
81.67	0.01\\
81.68	0.01\\
81.69	0.01\\
81.7	0.01\\
81.71	0.01\\
81.72	0.01\\
81.73	0.01\\
81.74	0.01\\
81.75	0.01\\
81.76	0.01\\
81.77	0.01\\
81.78	0.01\\
81.79	0.01\\
81.8	0.01\\
81.81	0.01\\
81.82	0.01\\
81.83	0.01\\
81.84	0.01\\
81.85	0.01\\
81.86	0.01\\
81.87	0.01\\
81.88	0.01\\
81.89	0.01\\
81.9	0.01\\
81.91	0.01\\
81.92	0.01\\
81.93	0.01\\
81.94	0.01\\
81.95	0.01\\
81.96	0.01\\
81.97	0.01\\
81.98	0.01\\
81.99	0.01\\
82	0.01\\
82.01	0.01\\
82.02	0.01\\
82.03	0.01\\
82.04	0.01\\
82.05	0.01\\
82.06	0.01\\
82.07	0.01\\
82.08	0.01\\
82.09	0.01\\
82.1	0.01\\
82.11	0.01\\
82.12	0.01\\
82.13	0.01\\
82.14	0.01\\
82.15	0.01\\
82.16	0.01\\
82.17	0.01\\
82.18	0.01\\
82.19	0.01\\
82.2	0.01\\
82.21	0.01\\
82.22	0.01\\
82.23	0.01\\
82.24	0.01\\
82.25	0.01\\
82.26	0.01\\
82.27	0.01\\
82.28	0.01\\
82.29	0.01\\
82.3	0.01\\
82.31	0.01\\
82.32	0.01\\
82.33	0.01\\
82.34	0.01\\
82.35	0.01\\
82.36	0.01\\
82.37	0.01\\
82.38	0.01\\
82.39	0.01\\
82.4	0.01\\
82.41	0.01\\
82.42	0.01\\
82.43	0.01\\
82.44	0.01\\
82.45	0.01\\
82.46	0.01\\
82.47	0.01\\
82.48	0.01\\
82.49	0.01\\
82.5	0.01\\
82.51	0.01\\
82.52	0.01\\
82.53	0.01\\
82.54	0.01\\
82.55	0.01\\
82.56	0.01\\
82.57	0.01\\
82.58	0.01\\
82.59	0.01\\
82.6	0.01\\
82.61	0.01\\
82.62	0.01\\
82.63	0.01\\
82.64	0.01\\
82.65	0.01\\
82.66	0.01\\
82.67	0.01\\
82.68	0.01\\
82.69	0.01\\
82.7	0.01\\
82.71	0.01\\
82.72	0.01\\
82.73	0.01\\
82.74	0.01\\
82.75	0.01\\
82.76	0.01\\
82.77	0.01\\
82.78	0.01\\
82.79	0.01\\
82.8	0.01\\
82.81	0.01\\
82.82	0.01\\
82.83	0.01\\
82.84	0.01\\
82.85	0.01\\
82.86	0.01\\
82.87	0.01\\
82.88	0.01\\
82.89	0.01\\
82.9	0.01\\
82.91	0.01\\
82.92	0.01\\
82.93	0.01\\
82.94	0.01\\
82.95	0.01\\
82.96	0.01\\
82.97	0.01\\
82.98	0.01\\
82.99	0.01\\
83	0.01\\
83.01	0.01\\
83.02	0.01\\
83.03	0.01\\
83.04	0.01\\
83.05	0.01\\
83.06	0.01\\
83.07	0.01\\
83.08	0.01\\
83.09	0.01\\
83.1	0.01\\
83.11	0.01\\
83.12	0.01\\
83.13	0.01\\
83.14	0.01\\
83.15	0.01\\
83.16	0.01\\
83.17	0.01\\
83.18	0.01\\
83.19	0.01\\
83.2	0.01\\
83.21	0.01\\
83.22	0.01\\
83.23	0.01\\
83.24	0.01\\
83.25	0.01\\
83.26	0.01\\
83.27	0.01\\
83.28	0.01\\
83.29	0.01\\
83.3	0.01\\
83.31	0.01\\
83.32	0.01\\
83.33	0.01\\
83.34	0.01\\
83.35	0.01\\
83.36	0.01\\
83.37	0.01\\
83.38	0.01\\
83.39	0.01\\
83.4	0.01\\
83.41	0.01\\
83.42	0.01\\
83.43	0.01\\
83.44	0.01\\
83.45	0.01\\
83.46	0.01\\
83.47	0.01\\
83.48	0.01\\
83.49	0.01\\
83.5	0.01\\
83.51	0.01\\
83.52	0.01\\
83.53	0.01\\
83.54	0.01\\
83.55	0.01\\
83.56	0.01\\
83.57	0.01\\
83.58	0.01\\
83.59	0.01\\
83.6	0.01\\
83.61	0.01\\
83.62	0.01\\
83.63	0.01\\
83.64	0.01\\
83.65	0.01\\
83.66	0.01\\
83.67	0.01\\
83.68	0.01\\
83.69	0.01\\
83.7	0.01\\
83.71	0.01\\
83.72	0.01\\
83.73	0.01\\
83.74	0.01\\
83.75	0.01\\
83.76	0.01\\
83.77	0.01\\
83.78	0.01\\
83.79	0.01\\
83.8	0.01\\
83.81	0.01\\
83.82	0.01\\
83.83	0.01\\
83.84	0.01\\
83.85	0.01\\
83.86	0.01\\
83.87	0.01\\
83.88	0.01\\
83.89	0.01\\
83.9	0.01\\
83.91	0.01\\
83.92	0.01\\
83.93	0.01\\
83.94	0.01\\
83.95	0.01\\
83.96	0.01\\
83.97	0.01\\
83.98	0.01\\
83.99	0.01\\
84	0.01\\
84.01	0.01\\
84.02	0.01\\
84.03	0.01\\
84.04	0.01\\
84.05	0.01\\
84.06	0.01\\
84.07	0.01\\
84.08	0.01\\
84.09	0.01\\
84.1	0.01\\
84.11	0.01\\
84.12	0.01\\
84.13	0.01\\
84.14	0.01\\
84.15	0.01\\
84.16	0.01\\
84.17	0.01\\
84.18	0.01\\
84.19	0.01\\
84.2	0.01\\
84.21	0.01\\
84.22	0.01\\
84.23	0.01\\
84.24	0.01\\
84.25	0.01\\
84.26	0.01\\
84.27	0.01\\
84.28	0.01\\
84.29	0.01\\
84.3	0.01\\
84.31	0.01\\
84.32	0.01\\
84.33	0.01\\
84.34	0.01\\
84.35	0.01\\
84.36	0.01\\
84.37	0.01\\
84.38	0.01\\
84.39	0.01\\
84.4	0.01\\
84.41	0.01\\
84.42	0.01\\
84.43	0.01\\
84.44	0.01\\
84.45	0.01\\
84.46	0.01\\
84.47	0.01\\
84.48	0.01\\
84.49	0.01\\
84.5	0.01\\
84.51	0.01\\
84.52	0.01\\
84.53	0.01\\
84.54	0.01\\
84.55	0.01\\
84.56	0.01\\
84.57	0.01\\
84.58	0.01\\
84.59	0.01\\
84.6	0.01\\
84.61	0.01\\
84.62	0.01\\
84.63	0.01\\
84.64	0.01\\
84.65	0.01\\
84.66	0.01\\
84.67	0.01\\
84.68	0.01\\
84.69	0.01\\
84.7	0.01\\
84.71	0.01\\
84.72	0.01\\
84.73	0.01\\
84.74	0.01\\
84.75	0.01\\
84.76	0.01\\
84.77	0.01\\
84.78	0.01\\
84.79	0.01\\
84.8	0.01\\
84.81	0.01\\
84.82	0.01\\
84.83	0.01\\
84.84	0.01\\
84.85	0.01\\
84.86	0.01\\
84.87	0.01\\
84.88	0.01\\
84.89	0.01\\
84.9	0.01\\
84.91	0.01\\
84.92	0.01\\
84.93	0.01\\
84.94	0.01\\
84.95	0.01\\
84.96	0.01\\
84.97	0.01\\
84.98	0.01\\
84.99	0.01\\
85	0.01\\
85.01	0.01\\
85.02	0.01\\
85.03	0.01\\
85.04	0.01\\
85.05	0.01\\
85.06	0.01\\
85.07	0.01\\
85.08	0.01\\
85.09	0.01\\
85.1	0.01\\
85.11	0.01\\
85.12	0.01\\
85.13	0.01\\
85.14	0.01\\
85.15	0.01\\
85.16	0.01\\
85.17	0.01\\
85.18	0.01\\
85.19	0.01\\
85.2	0.01\\
85.21	0.01\\
85.22	0.01\\
85.23	0.01\\
85.24	0.01\\
85.25	0.01\\
85.26	0.01\\
85.27	0.01\\
85.28	0.01\\
85.29	0.01\\
85.3	0.01\\
85.31	0.01\\
85.32	0.01\\
85.33	0.01\\
85.34	0.01\\
85.35	0.01\\
85.36	0.01\\
85.37	0.01\\
85.38	0.01\\
85.39	0.01\\
85.4	0.01\\
85.41	0.01\\
85.42	0.01\\
85.43	0.01\\
85.44	0.01\\
85.45	0.01\\
85.46	0.01\\
85.47	0.01\\
85.48	0.01\\
85.49	0.01\\
85.5	0.01\\
85.51	0.01\\
85.52	0.01\\
85.53	0.01\\
85.54	0.01\\
85.55	0.01\\
85.56	0.01\\
85.57	0.01\\
85.58	0.01\\
85.59	0.01\\
85.6	0.01\\
85.61	0.01\\
85.62	0.01\\
85.63	0.01\\
85.64	0.01\\
85.65	0.01\\
85.66	0.01\\
85.67	0.01\\
85.68	0.01\\
85.69	0.01\\
85.7	0.01\\
85.71	0.01\\
85.72	0.01\\
85.73	0.01\\
85.74	0.01\\
85.75	0.01\\
85.76	0.01\\
85.77	0.01\\
85.78	0.01\\
85.79	0.01\\
85.8	0.01\\
85.81	0.01\\
85.82	0.01\\
85.83	0.01\\
85.84	0.01\\
85.85	0.01\\
85.86	0.01\\
85.87	0.01\\
85.88	0.01\\
85.89	0.01\\
85.9	0.01\\
85.91	0.01\\
85.92	0.01\\
85.93	0.01\\
85.94	0.01\\
85.95	0.01\\
85.96	0.01\\
85.97	0.01\\
85.98	0.01\\
85.99	0.01\\
86	0.01\\
86.01	0.01\\
86.02	0.01\\
86.03	0.01\\
86.04	0.01\\
86.05	0.01\\
86.06	0.01\\
86.07	0.01\\
86.08	0.01\\
86.09	0.01\\
86.1	0.01\\
86.11	0.01\\
86.12	0.01\\
86.13	0.01\\
86.14	0.01\\
86.15	0.01\\
86.16	0.01\\
86.17	0.01\\
86.18	0.01\\
86.19	0.01\\
86.2	0.01\\
86.21	0.01\\
86.22	0.01\\
86.23	0.01\\
86.24	0.01\\
86.25	0.01\\
86.26	0.01\\
86.27	0.01\\
86.28	0.01\\
86.29	0.01\\
86.3	0.01\\
86.31	0.01\\
86.32	0.01\\
86.33	0.01\\
86.34	0.01\\
86.35	0.01\\
86.36	0.01\\
86.37	0.01\\
86.38	0.01\\
86.39	0.01\\
86.4	0.01\\
86.41	0.01\\
86.42	0.01\\
86.43	0.01\\
86.44	0.01\\
86.45	0.01\\
86.46	0.01\\
86.47	0.01\\
86.48	0.01\\
86.49	0.01\\
86.5	0.01\\
86.51	0.01\\
86.52	0.01\\
86.53	0.01\\
86.54	0.01\\
86.55	0.01\\
86.56	0.01\\
86.57	0.01\\
86.58	0.01\\
86.59	0.01\\
86.6	0.01\\
86.61	0.01\\
86.62	0.01\\
86.63	0.01\\
86.64	0.01\\
86.65	0.01\\
86.66	0.01\\
86.67	0.01\\
86.68	0.01\\
86.69	0.01\\
86.7	0.01\\
86.71	0.01\\
86.72	0.01\\
86.73	0.01\\
86.74	0.01\\
86.75	0.01\\
86.76	0.01\\
86.77	0.01\\
86.78	0.01\\
86.79	0.01\\
86.8	0.01\\
86.81	0.01\\
86.82	0.01\\
86.83	0.01\\
86.84	0.01\\
86.85	0.01\\
86.86	0.01\\
86.87	0.01\\
86.88	0.01\\
86.89	0.01\\
86.9	0.01\\
86.91	0.01\\
86.92	0.01\\
86.93	0.01\\
86.94	0.01\\
86.95	0.01\\
86.96	0.01\\
86.97	0.01\\
86.98	0.01\\
86.99	0.01\\
87	0.01\\
87.01	0.01\\
87.02	0.01\\
87.03	0.01\\
87.04	0.01\\
87.05	0.01\\
87.06	0.01\\
87.07	0.01\\
87.08	0.01\\
87.09	0.01\\
87.1	0.01\\
87.11	0.01\\
87.12	0.01\\
87.13	0.01\\
87.14	0.01\\
87.15	0.01\\
87.16	0.01\\
87.17	0.01\\
87.18	0.01\\
87.19	0.01\\
87.2	0.01\\
87.21	0.01\\
87.22	0.01\\
87.23	0.01\\
87.24	0.01\\
87.25	0.01\\
87.26	0.01\\
87.27	0.01\\
87.28	0.01\\
87.29	0.01\\
87.3	0.01\\
87.31	0.01\\
87.32	0.01\\
87.33	0.01\\
87.34	0.01\\
87.35	0.01\\
87.36	0.01\\
87.37	0.01\\
87.38	0.01\\
87.39	0.01\\
87.4	0.01\\
87.41	0.01\\
87.42	0.01\\
87.43	0.01\\
87.44	0.01\\
87.45	0.01\\
87.46	0.01\\
87.47	0.01\\
87.48	0.01\\
87.49	0.01\\
87.5	0.01\\
87.51	0.01\\
87.52	0.01\\
87.53	0.01\\
87.54	0.01\\
87.55	0.01\\
87.56	0.01\\
87.57	0.01\\
87.58	0.01\\
87.59	0.01\\
87.6	0.01\\
87.61	0.01\\
87.62	0.01\\
87.63	0.01\\
87.64	0.01\\
87.65	0.01\\
87.66	0.01\\
87.67	0.01\\
87.68	0.01\\
87.69	0.01\\
87.7	0.01\\
87.71	0.01\\
87.72	0.01\\
87.73	0.01\\
87.74	0.01\\
87.75	0.01\\
87.76	0.01\\
87.77	0.01\\
87.78	0.01\\
87.79	0.01\\
87.8	0.01\\
87.81	0.01\\
87.82	0.01\\
87.83	0.01\\
87.84	0.01\\
87.85	0.01\\
87.86	0.01\\
87.87	0.01\\
87.88	0.01\\
87.89	0.01\\
87.9	0.01\\
87.91	0.01\\
87.92	0.01\\
87.93	0.01\\
87.94	0.01\\
87.95	0.01\\
87.96	0.01\\
87.97	0.01\\
87.98	0.01\\
87.99	0.01\\
88	0.01\\
88.01	0.01\\
88.02	0.01\\
88.03	0.01\\
88.04	0.01\\
88.05	0.01\\
88.06	0.01\\
88.07	0.01\\
88.08	0.01\\
88.09	0.01\\
88.1	0.01\\
88.11	0.01\\
88.12	0.01\\
88.13	0.01\\
88.14	0.01\\
88.15	0.01\\
88.16	0.01\\
88.17	0.01\\
88.18	0.01\\
88.19	0.01\\
88.2	0.01\\
88.21	0.01\\
88.22	0.01\\
88.23	0.01\\
88.24	0.01\\
88.25	0.01\\
88.26	0.01\\
88.27	0.01\\
88.28	0.01\\
88.29	0.01\\
88.3	0.01\\
88.31	0.01\\
88.32	0.01\\
88.33	0.01\\
88.34	0.01\\
88.35	0.01\\
88.36	0.01\\
88.37	0.01\\
88.38	0.01\\
88.39	0.01\\
88.4	0.01\\
88.41	0.01\\
88.42	0.01\\
88.43	0.01\\
88.44	0.01\\
88.45	0.01\\
88.46	0.01\\
88.47	0.01\\
88.48	0.01\\
88.49	0.01\\
88.5	0.01\\
88.51	0.01\\
88.52	0.01\\
88.53	0.01\\
88.54	0.01\\
88.55	0.01\\
88.56	0.01\\
88.57	0.01\\
88.58	0.01\\
88.59	0.01\\
88.6	0.01\\
88.61	0.01\\
88.62	0.01\\
88.63	0.01\\
88.64	0.01\\
88.65	0.01\\
88.66	0.01\\
88.67	0.01\\
88.68	0.01\\
88.69	0.01\\
88.7	0.01\\
88.71	0.01\\
88.72	0.01\\
88.73	0.01\\
88.74	0.01\\
88.75	0.01\\
88.76	0.01\\
88.77	0.01\\
88.78	0.01\\
88.79	0.01\\
88.8	0.01\\
88.81	0.01\\
88.82	0.01\\
88.83	0.01\\
88.84	0.01\\
88.85	0.01\\
88.86	0.01\\
88.87	0.01\\
88.88	0.01\\
88.89	0.01\\
88.9	0.01\\
88.91	0.01\\
88.92	0.01\\
88.93	0.01\\
88.94	0.01\\
88.95	0.01\\
88.96	0.01\\
88.97	0.01\\
88.98	0.01\\
88.99	0.01\\
89	0.01\\
89.01	0.01\\
89.02	0.01\\
89.03	0.01\\
89.04	0.01\\
89.05	0.01\\
89.06	0.01\\
89.07	0.01\\
89.08	0.01\\
89.09	0.01\\
89.1	0.01\\
89.11	0.01\\
89.12	0.01\\
89.13	0.01\\
89.14	0.01\\
89.15	0.01\\
89.16	0.01\\
89.17	0.01\\
89.18	0.01\\
89.19	0.01\\
89.2	0.01\\
89.21	0.01\\
89.22	0.01\\
89.23	0.01\\
89.24	0.01\\
89.25	0.01\\
89.26	0.01\\
89.27	0.01\\
89.28	0.01\\
89.29	0.01\\
89.3	0.01\\
89.31	0.01\\
89.32	0.01\\
89.33	0.01\\
89.34	0.01\\
89.35	0.01\\
89.36	0.01\\
89.37	0.01\\
89.38	0.01\\
89.39	0.01\\
89.4	0.01\\
89.41	0.01\\
89.42	0.01\\
89.43	0.01\\
89.44	0.01\\
89.45	0.01\\
89.46	0.01\\
89.47	0.01\\
89.48	0.01\\
89.49	0.01\\
89.5	0.01\\
89.51	0.01\\
89.52	0.01\\
89.53	0.01\\
89.54	0.01\\
89.55	0.01\\
89.56	0.01\\
89.57	0.01\\
89.58	0.01\\
89.59	0.01\\
89.6	0.01\\
89.61	0.01\\
89.62	0.01\\
89.63	0.01\\
89.64	0.01\\
89.65	0.01\\
89.66	0.01\\
89.67	0.01\\
89.68	0.01\\
89.69	0.01\\
89.7	0.01\\
89.71	0.01\\
89.72	0.01\\
89.73	0.01\\
89.74	0.01\\
89.75	0.01\\
89.76	0.01\\
89.77	0.01\\
89.78	0.01\\
89.79	0.01\\
89.8	0.01\\
89.81	0.01\\
89.82	0.01\\
89.83	0.01\\
89.84	0.01\\
89.85	0.01\\
89.86	0.01\\
89.87	0.01\\
89.88	0.01\\
89.89	0.01\\
89.9	0.01\\
89.91	0.01\\
89.92	0.01\\
89.93	0.01\\
89.94	0.01\\
89.95	0.01\\
89.96	0.01\\
89.97	0.01\\
89.98	0.01\\
89.99	0.01\\
90	0.01\\
90.01	0.01\\
90.02	0.01\\
90.03	0.01\\
90.04	0.01\\
90.05	0.01\\
90.06	0.01\\
90.07	0.01\\
90.08	0.01\\
90.09	0.01\\
90.1	0.01\\
90.11	0.01\\
90.12	0.01\\
90.13	0.01\\
90.14	0.01\\
90.15	0.01\\
90.16	0.01\\
90.17	0.01\\
90.18	0.01\\
90.19	0.01\\
90.2	0.01\\
90.21	0.01\\
90.22	0.01\\
90.23	0.01\\
90.24	0.01\\
90.25	0.01\\
90.26	0.01\\
90.27	0.01\\
90.28	0.01\\
90.29	0.01\\
90.3	0.01\\
90.31	0.01\\
90.32	0.01\\
90.33	0.01\\
90.34	0.01\\
90.35	0.01\\
90.36	0.01\\
90.37	0.01\\
90.38	0.01\\
90.39	0.01\\
90.4	0.01\\
90.41	0.01\\
90.42	0.01\\
90.43	0.01\\
90.44	0.01\\
90.45	0.01\\
90.46	0.01\\
90.47	0.01\\
90.48	0.01\\
90.49	0.01\\
90.5	0.01\\
90.51	0.01\\
90.52	0.01\\
90.53	0.01\\
90.54	0.01\\
90.55	0.01\\
90.56	0.01\\
90.57	0.01\\
90.58	0.01\\
90.59	0.01\\
90.6	0.01\\
90.61	0.01\\
90.62	0.01\\
90.63	0.01\\
90.64	0.01\\
90.65	0.01\\
90.66	0.01\\
90.67	0.01\\
90.68	0.01\\
90.69	0.01\\
90.7	0.01\\
90.71	0.01\\
90.72	0.01\\
90.73	0.01\\
90.74	0.01\\
90.75	0.01\\
90.76	0.01\\
90.77	0.01\\
90.78	0.01\\
90.79	0.01\\
90.8	0.01\\
90.81	0.01\\
90.82	0.01\\
90.83	0.01\\
90.84	0.01\\
90.85	0.01\\
90.86	0.01\\
90.87	0.01\\
90.88	0.01\\
90.89	0.01\\
90.9	0.01\\
90.91	0.01\\
90.92	0.01\\
90.93	0.01\\
90.94	0.01\\
90.95	0.01\\
90.96	0.01\\
90.97	0.01\\
90.98	0.01\\
90.99	0.01\\
91	0.01\\
91.01	0.01\\
91.02	0.01\\
91.03	0.01\\
91.04	0.01\\
91.05	0.01\\
91.06	0.01\\
91.07	0.01\\
91.08	0.01\\
91.09	0.01\\
91.1	0.01\\
91.11	0.01\\
91.12	0.01\\
91.13	0.01\\
91.14	0.01\\
91.15	0.01\\
91.16	0.01\\
91.17	0.01\\
91.18	0.01\\
91.19	0.01\\
91.2	0.01\\
91.21	0.01\\
91.22	0.01\\
91.23	0.01\\
91.24	0.01\\
91.25	0.01\\
91.26	0.01\\
91.27	0.01\\
91.28	0.01\\
91.29	0.01\\
91.3	0.01\\
91.31	0.01\\
91.32	0.01\\
91.33	0.01\\
91.34	0.01\\
91.35	0.01\\
91.36	0.01\\
91.37	0.01\\
91.38	0.01\\
91.39	0.01\\
91.4	0.01\\
91.41	0.01\\
91.42	0.01\\
91.43	0.01\\
91.44	0.01\\
91.45	0.01\\
91.46	0.01\\
91.47	0.01\\
91.48	0.01\\
91.49	0.01\\
91.5	0.01\\
91.51	0.01\\
91.52	0.01\\
91.53	0.01\\
91.54	0.01\\
91.55	0.01\\
91.56	0.01\\
91.57	0.01\\
91.58	0.01\\
91.59	0.01\\
91.6	0.01\\
91.61	0.01\\
91.62	0.01\\
91.63	0.01\\
91.64	0.01\\
91.65	0.01\\
91.66	0.01\\
91.67	0.01\\
91.68	0.01\\
91.69	0.01\\
91.7	0.01\\
91.71	0.01\\
91.72	0.01\\
91.73	0.01\\
91.74	0.01\\
91.75	0.01\\
91.76	0.01\\
91.77	0.01\\
91.78	0.01\\
91.79	0.01\\
91.8	0.01\\
91.81	0.01\\
91.82	0.01\\
91.83	0.01\\
91.84	0.01\\
91.85	0.01\\
91.86	0.01\\
91.87	0.01\\
91.88	0.01\\
91.89	0.01\\
91.9	0.01\\
91.91	0.01\\
91.92	0.01\\
91.93	0.01\\
91.94	0.01\\
91.95	0.01\\
91.96	0.01\\
91.97	0.01\\
91.98	0.01\\
91.99	0.01\\
92	0.01\\
92.01	0.01\\
92.02	0.01\\
92.03	0.01\\
92.04	0.01\\
92.05	0.01\\
92.06	0.01\\
92.07	0.01\\
92.08	0.01\\
92.09	0.01\\
92.1	0.01\\
92.11	0.01\\
92.12	0.01\\
92.13	0.01\\
92.14	0.01\\
92.15	0.01\\
92.16	0.01\\
92.17	0.01\\
92.18	0.01\\
92.19	0.01\\
92.2	0.01\\
92.21	0.01\\
92.22	0.01\\
92.23	0.01\\
92.24	0.01\\
92.25	0.01\\
92.26	0.01\\
92.27	0.01\\
92.28	0.01\\
92.29	0.01\\
92.3	0.01\\
92.31	0.01\\
92.32	0.01\\
92.33	0.01\\
92.34	0.01\\
92.35	0.01\\
92.36	0.01\\
92.37	0.01\\
92.38	0.01\\
92.39	0.01\\
92.4	0.01\\
92.41	0.01\\
92.42	0.01\\
92.43	0.01\\
92.44	0.01\\
92.45	0.01\\
92.46	0.01\\
92.47	0.01\\
92.48	0.01\\
92.49	0.01\\
92.5	0.01\\
92.51	0.01\\
92.52	0.01\\
92.53	0.01\\
92.54	0.01\\
92.55	0.01\\
92.56	0.01\\
92.57	0.01\\
92.58	0.01\\
92.59	0.01\\
92.6	0.01\\
92.61	0.01\\
92.62	0.01\\
92.63	0.01\\
92.64	0.01\\
92.65	0.01\\
92.66	0.01\\
92.67	0.01\\
92.68	0.01\\
92.69	0.01\\
92.7	0.01\\
92.71	0.01\\
92.72	0.01\\
92.73	0.01\\
92.74	0.01\\
92.75	0.01\\
92.76	0.01\\
92.77	0.01\\
92.78	0.01\\
92.79	0.01\\
92.8	0.01\\
92.81	0.01\\
92.82	0.01\\
92.83	0.01\\
92.84	0.01\\
92.85	0.01\\
92.86	0.01\\
92.87	0.01\\
92.88	0.01\\
92.89	0.01\\
92.9	0.01\\
92.91	0.01\\
92.92	0.01\\
92.93	0.01\\
92.94	0.01\\
92.95	0.01\\
92.96	0.01\\
92.97	0.01\\
92.98	0.01\\
92.99	0.01\\
93	0.01\\
93.01	0.01\\
93.02	0.01\\
93.03	0.01\\
93.04	0.01\\
93.05	0.01\\
93.06	0.01\\
93.07	0.01\\
93.08	0.01\\
93.09	0.01\\
93.1	0.01\\
93.11	0.01\\
93.12	0.01\\
93.13	0.01\\
93.14	0.01\\
93.15	0.01\\
93.16	0.01\\
93.17	0.01\\
93.18	0.01\\
93.19	0.01\\
93.2	0.01\\
93.21	0.01\\
93.22	0.01\\
93.23	0.01\\
93.24	0.01\\
93.25	0.01\\
93.26	0.01\\
93.27	0.01\\
93.28	0.01\\
93.29	0.01\\
93.3	0.01\\
93.31	0.01\\
93.32	0.01\\
93.33	0.01\\
93.34	0.01\\
93.35	0.01\\
93.36	0.01\\
93.37	0.01\\
93.38	0.01\\
93.39	0.01\\
93.4	0.01\\
93.41	0.01\\
93.42	0.01\\
93.43	0.01\\
93.44	0.01\\
93.45	0.01\\
93.46	0.01\\
93.47	0.01\\
93.48	0.01\\
93.49	0.01\\
93.5	0.01\\
93.51	0.01\\
93.52	0.01\\
93.53	0.01\\
93.54	0.01\\
93.55	0.01\\
93.56	0.01\\
93.57	0.01\\
93.58	0.01\\
93.59	0.01\\
93.6	0.01\\
93.61	0.01\\
93.62	0.01\\
93.63	0.01\\
93.64	0.01\\
93.65	0.01\\
93.66	0.01\\
93.67	0.01\\
93.68	0.01\\
93.69	0.01\\
93.7	0.01\\
93.71	0.01\\
93.72	0.01\\
93.73	0.01\\
93.74	0.01\\
93.75	0.01\\
93.76	0.01\\
93.77	0.01\\
93.78	0.01\\
93.79	0.01\\
93.8	0.01\\
93.81	0.01\\
93.82	0.01\\
93.83	0.01\\
93.84	0.01\\
93.85	0.01\\
93.86	0.01\\
93.87	0.01\\
93.88	0.01\\
93.89	0.01\\
93.9	0.01\\
93.91	0.01\\
93.92	0.01\\
93.93	0.01\\
93.94	0.01\\
93.95	0.01\\
93.96	0.01\\
93.97	0.01\\
93.98	0.01\\
93.99	0.01\\
94	0.01\\
94.01	0.01\\
94.02	0.01\\
94.03	0.01\\
94.04	0.01\\
94.05	0.01\\
94.06	0.01\\
94.07	0.01\\
94.08	0.01\\
94.09	0.01\\
94.1	0.01\\
94.11	0.01\\
94.12	0.01\\
94.13	0.01\\
94.14	0.01\\
94.15	0.01\\
94.16	0.01\\
94.17	0.01\\
94.18	0.01\\
94.19	0.01\\
94.2	0.01\\
94.21	0.01\\
94.22	0.01\\
94.23	0.01\\
94.24	0.01\\
94.25	0.01\\
94.26	0.01\\
94.27	0.01\\
94.28	0.01\\
94.29	0.01\\
94.3	0.01\\
94.31	0.01\\
94.32	0.01\\
94.33	0.01\\
94.34	0.01\\
94.35	0.01\\
94.36	0.01\\
94.37	0.01\\
94.38	0.01\\
94.39	0.01\\
94.4	0.01\\
94.41	0.01\\
94.42	0.01\\
94.43	0.01\\
94.44	0.01\\
94.45	0.01\\
94.46	0.01\\
94.47	0.01\\
94.48	0.01\\
94.49	0.01\\
94.5	0.01\\
94.51	0.01\\
94.52	0.01\\
94.53	0.01\\
94.54	0.01\\
94.55	0.01\\
94.56	0.01\\
94.57	0.01\\
94.58	0.01\\
94.59	0.01\\
94.6	0.01\\
94.61	0.01\\
94.62	0.01\\
94.63	0.01\\
94.64	0.01\\
94.65	0.01\\
94.66	0.01\\
94.67	0.01\\
94.68	0.01\\
94.69	0.01\\
94.7	0.01\\
94.71	0.01\\
94.72	0.01\\
94.73	0.01\\
94.74	0.01\\
94.75	0.01\\
94.76	0.01\\
94.77	0.01\\
94.78	0.01\\
94.79	0.01\\
94.8	0.01\\
94.81	0.01\\
94.82	0.01\\
94.83	0.01\\
94.84	0.01\\
94.85	0.01\\
94.86	0.01\\
94.87	0.01\\
94.88	0.01\\
94.89	0.01\\
94.9	0.01\\
94.91	0.01\\
94.92	0.01\\
94.93	0.01\\
94.94	0.01\\
94.95	0.01\\
94.96	0.01\\
94.97	0.01\\
94.98	0.01\\
94.99	0.01\\
95	0.01\\
95.01	0.01\\
95.02	0.01\\
95.03	0.01\\
95.04	0.01\\
95.05	0.01\\
95.06	0.01\\
95.07	0.01\\
95.08	0.01\\
95.09	0.01\\
95.1	0.01\\
95.11	0.01\\
95.12	0.01\\
95.13	0.01\\
95.14	0.01\\
95.15	0.01\\
95.16	0.01\\
95.17	0.01\\
95.18	0.01\\
95.19	0.01\\
95.2	0.01\\
95.21	0.01\\
95.22	0.01\\
95.23	0.01\\
95.24	0.01\\
95.25	0.01\\
95.26	0.01\\
95.27	0.01\\
95.28	0.01\\
95.29	0.01\\
95.3	0.01\\
95.31	0.01\\
95.32	0.01\\
95.33	0.01\\
95.34	0.01\\
95.35	0.01\\
95.36	0.01\\
95.37	0.01\\
95.38	0.01\\
95.39	0.01\\
95.4	0.01\\
95.41	0.01\\
95.42	0.01\\
95.43	0.01\\
95.44	0.01\\
95.45	0.01\\
95.46	0.01\\
95.47	0.01\\
95.48	0.01\\
95.49	0.01\\
95.5	0.01\\
95.51	0.01\\
95.52	0.01\\
95.53	0.01\\
95.54	0.01\\
95.55	0.01\\
95.56	0.01\\
95.57	0.01\\
95.58	0.01\\
95.59	0.01\\
95.6	0.01\\
95.61	0.01\\
95.62	0.01\\
95.63	0.01\\
95.64	0.01\\
95.65	0.01\\
95.66	0.01\\
95.67	0.01\\
95.68	0.01\\
95.69	0.01\\
95.7	0.01\\
95.71	0.01\\
95.72	0.01\\
95.73	0.01\\
95.74	0.01\\
95.75	0.01\\
95.76	0.01\\
95.77	0.01\\
95.78	0.01\\
95.79	0.01\\
95.8	0.01\\
95.81	0.01\\
95.82	0.01\\
95.83	0.01\\
95.84	0.01\\
95.85	0.01\\
95.86	0.01\\
95.87	0.01\\
95.88	0.01\\
95.89	0.01\\
95.9	0.01\\
95.91	0.01\\
95.92	0.01\\
95.93	0.01\\
95.94	0.01\\
95.95	0.01\\
95.96	0.01\\
95.97	0.01\\
95.98	0.01\\
95.99	0.01\\
96	0.01\\
96.01	0.01\\
96.02	0.01\\
96.03	0.01\\
96.04	0.01\\
96.05	0.01\\
96.06	0.01\\
96.07	0.01\\
96.08	0.01\\
96.09	0.01\\
96.1	0.01\\
96.11	0.01\\
96.12	0.01\\
96.13	0.01\\
96.14	0.01\\
96.15	0.01\\
96.16	0.01\\
96.17	0.01\\
96.18	0.01\\
96.19	0.01\\
96.2	0.01\\
96.21	0.01\\
96.22	0.01\\
96.23	0.01\\
96.24	0.01\\
96.25	0.01\\
96.26	0.01\\
96.27	0.01\\
96.28	0.01\\
96.29	0.01\\
96.3	0.01\\
96.31	0.01\\
96.32	0.01\\
96.33	0.01\\
96.34	0.01\\
96.35	0.01\\
96.36	0.01\\
96.37	0.01\\
96.38	0.01\\
96.39	0.01\\
96.4	0.01\\
96.41	0.01\\
96.42	0.01\\
96.43	0.01\\
96.44	0.01\\
96.45	0.01\\
96.46	0.01\\
96.47	0.01\\
96.48	0.01\\
96.49	0.01\\
96.5	0.01\\
96.51	0.01\\
96.52	0.01\\
96.53	0.01\\
96.54	0.01\\
96.55	0.01\\
96.56	0.01\\
96.57	0.01\\
96.58	0.01\\
96.59	0.01\\
96.6	0.01\\
96.61	0.01\\
96.62	0.01\\
96.63	0.01\\
96.64	0.01\\
96.65	0.01\\
96.66	0.01\\
96.67	0.01\\
96.68	0.01\\
96.69	0.01\\
96.7	0.01\\
96.71	0.01\\
96.72	0.01\\
96.73	0.01\\
96.74	0.01\\
96.75	0.01\\
96.76	0.01\\
96.77	0.01\\
96.78	0.01\\
96.79	0.01\\
96.8	0.01\\
96.81	0.01\\
96.82	0.01\\
96.83	0.01\\
96.84	0.01\\
96.85	0.01\\
96.86	0.01\\
96.87	0.01\\
96.88	0.01\\
96.89	0.01\\
96.9	0.01\\
96.91	0.01\\
96.92	0.01\\
96.93	0.01\\
96.94	0.01\\
96.95	0.01\\
96.96	0.01\\
96.97	0.01\\
96.98	0.01\\
96.99	0.01\\
97	0.01\\
97.01	0.01\\
97.02	0.01\\
97.03	0.01\\
97.04	0.01\\
97.05	0.01\\
97.06	0.01\\
97.07	0.01\\
97.08	0.01\\
97.09	0.01\\
97.1	0.01\\
97.11	0.01\\
97.12	0.01\\
97.13	0.01\\
97.14	0.01\\
97.15	0.01\\
97.16	0.01\\
97.17	0.01\\
97.18	0.01\\
97.19	0.01\\
97.2	0.01\\
97.21	0.01\\
97.22	0.01\\
97.23	0.01\\
97.24	0.01\\
97.25	0.01\\
97.26	0.01\\
97.27	0.01\\
97.28	0.01\\
97.29	0.01\\
97.3	0.01\\
97.31	0.01\\
97.32	0.01\\
97.33	0.01\\
97.34	0.01\\
97.35	0.01\\
97.36	0.01\\
97.37	0.01\\
97.38	0.01\\
97.39	0.01\\
97.4	0.01\\
97.41	0.01\\
97.42	0.01\\
97.43	0.01\\
97.44	0.01\\
97.45	0.01\\
97.46	0.01\\
97.47	0.01\\
97.48	0.01\\
97.49	0.01\\
97.5	0.01\\
97.51	0.01\\
97.52	0.01\\
97.53	0.01\\
97.54	0.01\\
97.55	0.01\\
97.56	0.01\\
97.57	0.01\\
97.58	0.01\\
97.59	0.01\\
97.6	0.01\\
97.61	0.01\\
97.62	0.01\\
97.63	0.01\\
97.64	0.01\\
97.65	0.01\\
97.66	0.01\\
97.67	0.01\\
97.68	0.01\\
97.69	0.01\\
97.7	0.01\\
97.71	0.01\\
97.72	0.01\\
97.73	0.01\\
97.74	0.01\\
97.75	0.01\\
97.76	0.01\\
97.77	0.01\\
97.78	0.01\\
97.79	0.01\\
97.8	0.01\\
97.81	0.01\\
97.82	0.01\\
97.83	0.01\\
97.84	0.01\\
97.85	0.01\\
97.86	0.01\\
97.87	0.01\\
97.88	0.01\\
97.89	0.01\\
97.9	0.01\\
97.91	0.01\\
97.92	0.01\\
97.93	0.01\\
97.94	0.01\\
97.95	0.01\\
97.96	0.01\\
97.97	0.01\\
97.98	0.01\\
97.99	0.01\\
98	0.01\\
98.01	0.01\\
98.02	0.01\\
98.03	0.01\\
98.04	0.01\\
98.05	0.01\\
98.06	0.01\\
98.07	0.01\\
98.08	0.01\\
98.09	0.01\\
98.1	0.01\\
98.11	0.01\\
98.12	0.01\\
98.13	0.01\\
98.14	0.01\\
98.15	0.01\\
98.16	0.01\\
98.17	0.01\\
98.18	0.01\\
98.19	0.01\\
98.2	0.01\\
98.21	0.01\\
98.22	0.01\\
98.23	0.01\\
98.24	0.01\\
98.25	0.01\\
98.26	0.01\\
98.27	0.01\\
98.28	0.01\\
98.29	0.01\\
98.3	0.01\\
98.31	0.01\\
98.32	0.01\\
98.33	0.01\\
98.34	0.01\\
98.35	0.01\\
98.36	0.01\\
98.37	0.01\\
98.38	0.01\\
98.39	0.01\\
98.4	0.01\\
98.41	0.01\\
98.42	0.01\\
98.43	0.01\\
98.44	0.01\\
98.45	0.01\\
98.46	0.01\\
98.47	0.01\\
98.48	0.01\\
98.49	0.01\\
98.5	0.01\\
98.51	0.01\\
98.52	0.01\\
98.53	0.01\\
98.54	0.01\\
98.55	0.01\\
98.56	0.01\\
98.57	0.01\\
98.58	0.01\\
98.59	0.01\\
98.6	0.01\\
98.61	0.01\\
98.62	0.01\\
98.63	0.01\\
98.64	0.01\\
98.65	0.01\\
98.66	0.01\\
98.67	0.01\\
98.68	0.01\\
98.69	0.01\\
98.7	0.01\\
98.71	0.01\\
98.72	0.01\\
98.73	0.01\\
98.74	0.01\\
98.75	0.01\\
98.76	0.01\\
98.77	0.01\\
98.78	0.01\\
98.79	0.01\\
98.8	0.01\\
98.81	0.01\\
98.82	0.01\\
98.83	0.01\\
98.84	0.01\\
98.85	0.01\\
98.86	0.01\\
98.87	0.01\\
98.88	0.01\\
98.89	0.01\\
98.9	0.01\\
98.91	0.01\\
98.92	0.01\\
98.93	0.01\\
98.94	0.01\\
98.95	0.01\\
98.96	0.00994657662760442\\
98.97	0.00975022967615308\\
98.98	0.00955247302775352\\
98.99	0.00935328647093454\\
99	0.00915264917735695\\
99.01	0.00895053971824736\\
99.02	0.00874693604588305\\
99.03	0.00854181547068092\\
99.04	0.0083351546369417\\
99.05	0.00812698671253183\\
99.06	0.00791728924939752\\
99.07	0.00770603839157192\\
99.08	0.00749320955439558\\
99.09	0.00727877739613037\\
99.1	0.007062715780018\\
99.11	0.0068449977483514\\
99.12	0.00662559549257636\\
99.13	0.00640448031894984\\
99.14	0.00619506749001899\\
99.15	0.00614343065536976\\
99.16	0.00609144762524035\\
99.17	0.00603912153677965\\
99.18	0.00598645582119131\\
99.19	0.00593345421943499\\
99.2	0.00588012078333322\\
99.21	0.0058264598893813\\
99.22	0.00577246463597955\\
99.23	0.00571809972977372\\
99.24	0.00566336885383304\\
99.25	0.0056082760216254\\
99.26	0.00555282559207382\\
99.27	0.00549702228530884\\
99.28	0.0054408711991545\\
99.29	0.00538437782638891\\
99.3	0.00532754807282254\\
99.31	0.00527038827624041\\
99.32	0.0052129052262574\\
99.33	0.0051551061851392\\
99.34	0.00509699890964516\\
99.35	0.00503859167365097\\
99.36	0.00497989329218159\\
99.37	0.00492091315045911\\
99.38	0.0048616612557091\\
99.39	0.00480214823897229\\
99.4	0.00474238538462658\\
99.41	0.00468238466146454\\
99.42	0.00462184456317968\\
99.43	0.00456075958847703\\
99.44	0.00449912481928159\\
99.45	0.00443693528457883\\
99.46	0.00437418595912791\\
99.47	0.00431087176214988\\
99.48	0.00424698765185965\\
99.49	0.00418252865380583\\
99.5	0.00411748975146728\\
99.51	0.00405186588565974\\
99.52	0.00398565195391432\\
99.53	0.00391884280982608\\
99.54	0.00385143326237089\\
99.55	0.00378341807518831\\
99.56	0.00371479196582851\\
99.57	0.00364554960496075\\
99.58	0.00357568561554109\\
99.59	0.00350519457193654\\
99.6	0.00343407099900291\\
99.61	0.00336230937111332\\
99.62	0.00328990411113425\\
99.63	0.00321684958934553\\
99.64	0.00314314012230076\\
99.65	0.00306876997162424\\
99.66	0.00299373334274013\\
99.67	0.00291802438352953\\
99.68	0.00284163718291057\\
99.69	0.00276456578468188\\
99.7	0.00268680417387409\\
99.71	0.00260834627216765\\
99.72	0.00252918593610012\\
99.73	0.00244931695514438\\
99.74	0.00236873304971111\\
99.75	0.00228742786903057\\
99.76	0.00220539498890514\\
99.77	0.00212262790932414\\
99.78	0.00203912005193141\\
99.79	0.0019548647573356\\
99.8	0.00186985528225254\\
99.81	0.00178408479646787\\
99.82	0.00169754637960787\\
99.83	0.00161023301770489\\
99.84	0.00152213759954336\\
99.85	0.001433252912771\\
99.86	0.00134357163975866\\
99.87	0.00125308635319149\\
99.88	0.0011617895113721\\
99.89	0.00106967345321565\\
99.9	0.000976730392914901\\
99.91	0.000882952414253254\\
99.92	0.000788331464541591\\
99.93	0.000692859348149801\\
99.94	0.000596527719603999\\
99.95	0.000499328076217975\\
99.96	0.000401251750225215\\
99.97	0.000302289900375147\\
99.98	0.000202433502954543\\
99.99	0.000101673342191978\\
100	0\\
};
\addlegendentry{$q=-3$};

\addplot [color=red,dashed,forget plot]
  table[row sep=crcr]{%
0.01	0.01\\
0.02	0.01\\
0.03	0.01\\
0.04	0.01\\
0.05	0.01\\
0.06	0.01\\
0.07	0.01\\
0.08	0.01\\
0.09	0.01\\
0.1	0.01\\
0.11	0.01\\
0.12	0.01\\
0.13	0.01\\
0.14	0.01\\
0.15	0.01\\
0.16	0.01\\
0.17	0.01\\
0.18	0.01\\
0.19	0.01\\
0.2	0.01\\
0.21	0.01\\
0.22	0.01\\
0.23	0.01\\
0.24	0.01\\
0.25	0.01\\
0.26	0.01\\
0.27	0.01\\
0.28	0.01\\
0.29	0.01\\
0.3	0.01\\
0.31	0.01\\
0.32	0.01\\
0.33	0.01\\
0.34	0.01\\
0.35	0.01\\
0.36	0.01\\
0.37	0.01\\
0.38	0.01\\
0.39	0.01\\
0.4	0.01\\
0.41	0.01\\
0.42	0.01\\
0.43	0.01\\
0.44	0.01\\
0.45	0.01\\
0.46	0.01\\
0.47	0.01\\
0.48	0.01\\
0.49	0.01\\
0.5	0.01\\
0.51	0.01\\
0.52	0.01\\
0.53	0.01\\
0.54	0.01\\
0.55	0.01\\
0.56	0.01\\
0.57	0.01\\
0.58	0.01\\
0.59	0.01\\
0.6	0.01\\
0.61	0.01\\
0.62	0.01\\
0.63	0.01\\
0.64	0.01\\
0.65	0.01\\
0.66	0.01\\
0.67	0.01\\
0.68	0.01\\
0.69	0.01\\
0.7	0.01\\
0.71	0.01\\
0.72	0.01\\
0.73	0.01\\
0.74	0.01\\
0.75	0.01\\
0.76	0.01\\
0.77	0.01\\
0.78	0.01\\
0.79	0.01\\
0.8	0.01\\
0.81	0.01\\
0.82	0.01\\
0.83	0.01\\
0.84	0.01\\
0.85	0.01\\
0.86	0.01\\
0.87	0.01\\
0.88	0.01\\
0.89	0.01\\
0.9	0.01\\
0.91	0.01\\
0.92	0.01\\
0.93	0.01\\
0.94	0.01\\
0.95	0.01\\
0.96	0.01\\
0.97	0.01\\
0.98	0.01\\
0.99	0.01\\
1	0.01\\
1.01	0.01\\
1.02	0.01\\
1.03	0.01\\
1.04	0.01\\
1.05	0.01\\
1.06	0.01\\
1.07	0.01\\
1.08	0.01\\
1.09	0.01\\
1.1	0.01\\
1.11	0.01\\
1.12	0.01\\
1.13	0.01\\
1.14	0.01\\
1.15	0.01\\
1.16	0.01\\
1.17	0.01\\
1.18	0.01\\
1.19	0.01\\
1.2	0.01\\
1.21	0.01\\
1.22	0.01\\
1.23	0.01\\
1.24	0.01\\
1.25	0.01\\
1.26	0.01\\
1.27	0.01\\
1.28	0.01\\
1.29	0.01\\
1.3	0.01\\
1.31	0.01\\
1.32	0.01\\
1.33	0.01\\
1.34	0.01\\
1.35	0.01\\
1.36	0.01\\
1.37	0.01\\
1.38	0.01\\
1.39	0.01\\
1.4	0.01\\
1.41	0.01\\
1.42	0.01\\
1.43	0.01\\
1.44	0.01\\
1.45	0.01\\
1.46	0.01\\
1.47	0.01\\
1.48	0.01\\
1.49	0.01\\
1.5	0.01\\
1.51	0.01\\
1.52	0.01\\
1.53	0.01\\
1.54	0.01\\
1.55	0.01\\
1.56	0.01\\
1.57	0.01\\
1.58	0.01\\
1.59	0.01\\
1.6	0.01\\
1.61	0.01\\
1.62	0.01\\
1.63	0.01\\
1.64	0.01\\
1.65	0.01\\
1.66	0.01\\
1.67	0.01\\
1.68	0.01\\
1.69	0.01\\
1.7	0.01\\
1.71	0.01\\
1.72	0.01\\
1.73	0.01\\
1.74	0.01\\
1.75	0.01\\
1.76	0.01\\
1.77	0.01\\
1.78	0.01\\
1.79	0.01\\
1.8	0.01\\
1.81	0.01\\
1.82	0.01\\
1.83	0.01\\
1.84	0.01\\
1.85	0.01\\
1.86	0.01\\
1.87	0.01\\
1.88	0.01\\
1.89	0.01\\
1.9	0.01\\
1.91	0.01\\
1.92	0.01\\
1.93	0.01\\
1.94	0.01\\
1.95	0.01\\
1.96	0.01\\
1.97	0.01\\
1.98	0.01\\
1.99	0.01\\
2	0.01\\
2.01	0.01\\
2.02	0.01\\
2.03	0.01\\
2.04	0.01\\
2.05	0.01\\
2.06	0.01\\
2.07	0.01\\
2.08	0.01\\
2.09	0.01\\
2.1	0.01\\
2.11	0.01\\
2.12	0.01\\
2.13	0.01\\
2.14	0.01\\
2.15	0.01\\
2.16	0.01\\
2.17	0.01\\
2.18	0.01\\
2.19	0.01\\
2.2	0.01\\
2.21	0.01\\
2.22	0.01\\
2.23	0.01\\
2.24	0.01\\
2.25	0.01\\
2.26	0.01\\
2.27	0.01\\
2.28	0.01\\
2.29	0.01\\
2.3	0.01\\
2.31	0.01\\
2.32	0.01\\
2.33	0.01\\
2.34	0.01\\
2.35	0.01\\
2.36	0.01\\
2.37	0.01\\
2.38	0.01\\
2.39	0.01\\
2.4	0.01\\
2.41	0.01\\
2.42	0.01\\
2.43	0.01\\
2.44	0.01\\
2.45	0.01\\
2.46	0.01\\
2.47	0.01\\
2.48	0.01\\
2.49	0.01\\
2.5	0.01\\
2.51	0.01\\
2.52	0.01\\
2.53	0.01\\
2.54	0.01\\
2.55	0.01\\
2.56	0.01\\
2.57	0.01\\
2.58	0.01\\
2.59	0.01\\
2.6	0.01\\
2.61	0.01\\
2.62	0.01\\
2.63	0.01\\
2.64	0.01\\
2.65	0.01\\
2.66	0.01\\
2.67	0.01\\
2.68	0.01\\
2.69	0.01\\
2.7	0.01\\
2.71	0.01\\
2.72	0.01\\
2.73	0.01\\
2.74	0.01\\
2.75	0.01\\
2.76	0.01\\
2.77	0.01\\
2.78	0.01\\
2.79	0.01\\
2.8	0.01\\
2.81	0.01\\
2.82	0.01\\
2.83	0.01\\
2.84	0.01\\
2.85	0.01\\
2.86	0.01\\
2.87	0.01\\
2.88	0.01\\
2.89	0.01\\
2.9	0.01\\
2.91	0.01\\
2.92	0.01\\
2.93	0.01\\
2.94	0.01\\
2.95	0.01\\
2.96	0.01\\
2.97	0.01\\
2.98	0.01\\
2.99	0.01\\
3	0.01\\
3.01	0.01\\
3.02	0.01\\
3.03	0.01\\
3.04	0.01\\
3.05	0.01\\
3.06	0.01\\
3.07	0.01\\
3.08	0.01\\
3.09	0.01\\
3.1	0.01\\
3.11	0.01\\
3.12	0.01\\
3.13	0.01\\
3.14	0.01\\
3.15	0.01\\
3.16	0.01\\
3.17	0.01\\
3.18	0.01\\
3.19	0.01\\
3.2	0.01\\
3.21	0.01\\
3.22	0.01\\
3.23	0.01\\
3.24	0.01\\
3.25	0.01\\
3.26	0.01\\
3.27	0.01\\
3.28	0.01\\
3.29	0.01\\
3.3	0.01\\
3.31	0.01\\
3.32	0.01\\
3.33	0.01\\
3.34	0.01\\
3.35	0.01\\
3.36	0.01\\
3.37	0.01\\
3.38	0.01\\
3.39	0.01\\
3.4	0.01\\
3.41	0.01\\
3.42	0.01\\
3.43	0.01\\
3.44	0.01\\
3.45	0.01\\
3.46	0.01\\
3.47	0.01\\
3.48	0.01\\
3.49	0.01\\
3.5	0.01\\
3.51	0.01\\
3.52	0.01\\
3.53	0.01\\
3.54	0.01\\
3.55	0.01\\
3.56	0.01\\
3.57	0.01\\
3.58	0.01\\
3.59	0.01\\
3.6	0.01\\
3.61	0.01\\
3.62	0.01\\
3.63	0.01\\
3.64	0.01\\
3.65	0.01\\
3.66	0.01\\
3.67	0.01\\
3.68	0.01\\
3.69	0.01\\
3.7	0.01\\
3.71	0.01\\
3.72	0.01\\
3.73	0.01\\
3.74	0.01\\
3.75	0.01\\
3.76	0.01\\
3.77	0.01\\
3.78	0.01\\
3.79	0.01\\
3.8	0.01\\
3.81	0.01\\
3.82	0.01\\
3.83	0.01\\
3.84	0.01\\
3.85	0.01\\
3.86	0.01\\
3.87	0.01\\
3.88	0.01\\
3.89	0.01\\
3.9	0.01\\
3.91	0.01\\
3.92	0.01\\
3.93	0.01\\
3.94	0.01\\
3.95	0.01\\
3.96	0.01\\
3.97	0.01\\
3.98	0.01\\
3.99	0.01\\
4	0.01\\
4.01	0.01\\
4.02	0.01\\
4.03	0.01\\
4.04	0.01\\
4.05	0.01\\
4.06	0.01\\
4.07	0.01\\
4.08	0.01\\
4.09	0.01\\
4.1	0.01\\
4.11	0.01\\
4.12	0.01\\
4.13	0.01\\
4.14	0.01\\
4.15	0.01\\
4.16	0.01\\
4.17	0.01\\
4.18	0.01\\
4.19	0.01\\
4.2	0.01\\
4.21	0.01\\
4.22	0.01\\
4.23	0.01\\
4.24	0.01\\
4.25	0.01\\
4.26	0.01\\
4.27	0.01\\
4.28	0.01\\
4.29	0.01\\
4.3	0.01\\
4.31	0.01\\
4.32	0.01\\
4.33	0.01\\
4.34	0.01\\
4.35	0.01\\
4.36	0.01\\
4.37	0.01\\
4.38	0.01\\
4.39	0.01\\
4.4	0.01\\
4.41	0.01\\
4.42	0.01\\
4.43	0.01\\
4.44	0.01\\
4.45	0.01\\
4.46	0.01\\
4.47	0.01\\
4.48	0.01\\
4.49	0.01\\
4.5	0.01\\
4.51	0.01\\
4.52	0.01\\
4.53	0.01\\
4.54	0.01\\
4.55	0.01\\
4.56	0.01\\
4.57	0.01\\
4.58	0.01\\
4.59	0.01\\
4.6	0.01\\
4.61	0.01\\
4.62	0.01\\
4.63	0.01\\
4.64	0.01\\
4.65	0.01\\
4.66	0.01\\
4.67	0.01\\
4.68	0.01\\
4.69	0.01\\
4.7	0.01\\
4.71	0.01\\
4.72	0.01\\
4.73	0.01\\
4.74	0.01\\
4.75	0.01\\
4.76	0.01\\
4.77	0.01\\
4.78	0.01\\
4.79	0.01\\
4.8	0.01\\
4.81	0.01\\
4.82	0.01\\
4.83	0.01\\
4.84	0.01\\
4.85	0.01\\
4.86	0.01\\
4.87	0.01\\
4.88	0.01\\
4.89	0.01\\
4.9	0.01\\
4.91	0.01\\
4.92	0.01\\
4.93	0.01\\
4.94	0.01\\
4.95	0.01\\
4.96	0.01\\
4.97	0.01\\
4.98	0.01\\
4.99	0.01\\
5	0.01\\
5.01	0.01\\
5.02	0.01\\
5.03	0.01\\
5.04	0.01\\
5.05	0.01\\
5.06	0.01\\
5.07	0.01\\
5.08	0.01\\
5.09	0.01\\
5.1	0.01\\
5.11	0.01\\
5.12	0.01\\
5.13	0.01\\
5.14	0.01\\
5.15	0.01\\
5.16	0.01\\
5.17	0.01\\
5.18	0.01\\
5.19	0.01\\
5.2	0.01\\
5.21	0.01\\
5.22	0.01\\
5.23	0.01\\
5.24	0.01\\
5.25	0.01\\
5.26	0.01\\
5.27	0.01\\
5.28	0.01\\
5.29	0.01\\
5.3	0.01\\
5.31	0.01\\
5.32	0.01\\
5.33	0.01\\
5.34	0.01\\
5.35	0.01\\
5.36	0.01\\
5.37	0.01\\
5.38	0.01\\
5.39	0.01\\
5.4	0.01\\
5.41	0.01\\
5.42	0.01\\
5.43	0.01\\
5.44	0.01\\
5.45	0.01\\
5.46	0.01\\
5.47	0.01\\
5.48	0.01\\
5.49	0.01\\
5.5	0.01\\
5.51	0.01\\
5.52	0.01\\
5.53	0.01\\
5.54	0.01\\
5.55	0.01\\
5.56	0.01\\
5.57	0.01\\
5.58	0.01\\
5.59	0.01\\
5.6	0.01\\
5.61	0.01\\
5.62	0.01\\
5.63	0.01\\
5.64	0.01\\
5.65	0.01\\
5.66	0.01\\
5.67	0.01\\
5.68	0.01\\
5.69	0.01\\
5.7	0.01\\
5.71	0.01\\
5.72	0.01\\
5.73	0.01\\
5.74	0.01\\
5.75	0.01\\
5.76	0.01\\
5.77	0.01\\
5.78	0.01\\
5.79	0.01\\
5.8	0.01\\
5.81	0.01\\
5.82	0.01\\
5.83	0.01\\
5.84	0.01\\
5.85	0.01\\
5.86	0.01\\
5.87	0.01\\
5.88	0.01\\
5.89	0.01\\
5.9	0.01\\
5.91	0.01\\
5.92	0.01\\
5.93	0.01\\
5.94	0.01\\
5.95	0.01\\
5.96	0.01\\
5.97	0.01\\
5.98	0.01\\
5.99	0.01\\
6	0.01\\
6.01	0.01\\
6.02	0.01\\
6.03	0.01\\
6.04	0.01\\
6.05	0.01\\
6.06	0.01\\
6.07	0.01\\
6.08	0.01\\
6.09	0.01\\
6.1	0.01\\
6.11	0.01\\
6.12	0.01\\
6.13	0.01\\
6.14	0.01\\
6.15	0.01\\
6.16	0.01\\
6.17	0.01\\
6.18	0.01\\
6.19	0.01\\
6.2	0.01\\
6.21	0.01\\
6.22	0.01\\
6.23	0.01\\
6.24	0.01\\
6.25	0.01\\
6.26	0.01\\
6.27	0.01\\
6.28	0.01\\
6.29	0.01\\
6.3	0.01\\
6.31	0.01\\
6.32	0.01\\
6.33	0.01\\
6.34	0.01\\
6.35	0.01\\
6.36	0.01\\
6.37	0.01\\
6.38	0.01\\
6.39	0.01\\
6.4	0.01\\
6.41	0.01\\
6.42	0.01\\
6.43	0.01\\
6.44	0.01\\
6.45	0.01\\
6.46	0.01\\
6.47	0.01\\
6.48	0.01\\
6.49	0.01\\
6.5	0.01\\
6.51	0.01\\
6.52	0.01\\
6.53	0.01\\
6.54	0.01\\
6.55	0.01\\
6.56	0.01\\
6.57	0.01\\
6.58	0.01\\
6.59	0.01\\
6.6	0.01\\
6.61	0.01\\
6.62	0.01\\
6.63	0.01\\
6.64	0.01\\
6.65	0.01\\
6.66	0.01\\
6.67	0.01\\
6.68	0.01\\
6.69	0.01\\
6.7	0.01\\
6.71	0.01\\
6.72	0.01\\
6.73	0.01\\
6.74	0.01\\
6.75	0.01\\
6.76	0.01\\
6.77	0.01\\
6.78	0.01\\
6.79	0.01\\
6.8	0.01\\
6.81	0.01\\
6.82	0.01\\
6.83	0.01\\
6.84	0.01\\
6.85	0.01\\
6.86	0.01\\
6.87	0.01\\
6.88	0.01\\
6.89	0.01\\
6.9	0.01\\
6.91	0.01\\
6.92	0.01\\
6.93	0.01\\
6.94	0.01\\
6.95	0.01\\
6.96	0.01\\
6.97	0.01\\
6.98	0.01\\
6.99	0.01\\
7	0.01\\
7.01	0.01\\
7.02	0.01\\
7.03	0.01\\
7.04	0.01\\
7.05	0.01\\
7.06	0.01\\
7.07	0.01\\
7.08	0.01\\
7.09	0.01\\
7.1	0.01\\
7.11	0.01\\
7.12	0.01\\
7.13	0.01\\
7.14	0.01\\
7.15	0.01\\
7.16	0.01\\
7.17	0.01\\
7.18	0.01\\
7.19	0.01\\
7.2	0.01\\
7.21	0.01\\
7.22	0.01\\
7.23	0.01\\
7.24	0.01\\
7.25	0.01\\
7.26	0.01\\
7.27	0.01\\
7.28	0.01\\
7.29	0.01\\
7.3	0.01\\
7.31	0.01\\
7.32	0.01\\
7.33	0.01\\
7.34	0.01\\
7.35	0.01\\
7.36	0.01\\
7.37	0.01\\
7.38	0.01\\
7.39	0.01\\
7.4	0.01\\
7.41	0.01\\
7.42	0.01\\
7.43	0.01\\
7.44	0.01\\
7.45	0.01\\
7.46	0.01\\
7.47	0.01\\
7.48	0.01\\
7.49	0.01\\
7.5	0.01\\
7.51	0.01\\
7.52	0.01\\
7.53	0.01\\
7.54	0.01\\
7.55	0.01\\
7.56	0.01\\
7.57	0.01\\
7.58	0.01\\
7.59	0.01\\
7.6	0.01\\
7.61	0.01\\
7.62	0.01\\
7.63	0.01\\
7.64	0.01\\
7.65	0.01\\
7.66	0.01\\
7.67	0.01\\
7.68	0.01\\
7.69	0.01\\
7.7	0.01\\
7.71	0.01\\
7.72	0.01\\
7.73	0.01\\
7.74	0.01\\
7.75	0.01\\
7.76	0.01\\
7.77	0.01\\
7.78	0.01\\
7.79	0.01\\
7.8	0.01\\
7.81	0.01\\
7.82	0.01\\
7.83	0.01\\
7.84	0.01\\
7.85	0.01\\
7.86	0.01\\
7.87	0.01\\
7.88	0.01\\
7.89	0.01\\
7.9	0.01\\
7.91	0.01\\
7.92	0.01\\
7.93	0.01\\
7.94	0.01\\
7.95	0.01\\
7.96	0.01\\
7.97	0.01\\
7.98	0.01\\
7.99	0.01\\
8	0.01\\
8.01	0.01\\
8.02	0.01\\
8.03	0.01\\
8.04	0.01\\
8.05	0.01\\
8.06	0.01\\
8.07	0.01\\
8.08	0.01\\
8.09	0.01\\
8.1	0.01\\
8.11	0.01\\
8.12	0.01\\
8.13	0.01\\
8.14	0.01\\
8.15	0.01\\
8.16	0.01\\
8.17	0.01\\
8.18	0.01\\
8.19	0.01\\
8.2	0.01\\
8.21	0.01\\
8.22	0.01\\
8.23	0.01\\
8.24	0.01\\
8.25	0.01\\
8.26	0.01\\
8.27	0.01\\
8.28	0.01\\
8.29	0.01\\
8.3	0.01\\
8.31	0.01\\
8.32	0.01\\
8.33	0.01\\
8.34	0.01\\
8.35	0.01\\
8.36	0.01\\
8.37	0.01\\
8.38	0.01\\
8.39	0.01\\
8.4	0.01\\
8.41	0.01\\
8.42	0.01\\
8.43	0.01\\
8.44	0.01\\
8.45	0.01\\
8.46	0.01\\
8.47	0.01\\
8.48	0.01\\
8.49	0.01\\
8.5	0.01\\
8.51	0.01\\
8.52	0.01\\
8.53	0.01\\
8.54	0.01\\
8.55	0.01\\
8.56	0.01\\
8.57	0.01\\
8.58	0.01\\
8.59	0.01\\
8.6	0.01\\
8.61	0.01\\
8.62	0.01\\
8.63	0.01\\
8.64	0.01\\
8.65	0.01\\
8.66	0.01\\
8.67	0.01\\
8.68	0.01\\
8.69	0.01\\
8.7	0.01\\
8.71	0.01\\
8.72	0.01\\
8.73	0.01\\
8.74	0.01\\
8.75	0.01\\
8.76	0.01\\
8.77	0.01\\
8.78	0.01\\
8.79	0.01\\
8.8	0.01\\
8.81	0.01\\
8.82	0.01\\
8.83	0.01\\
8.84	0.01\\
8.85	0.01\\
8.86	0.01\\
8.87	0.01\\
8.88	0.01\\
8.89	0.01\\
8.9	0.01\\
8.91	0.01\\
8.92	0.01\\
8.93	0.01\\
8.94	0.01\\
8.95	0.01\\
8.96	0.01\\
8.97	0.01\\
8.98	0.01\\
8.99	0.01\\
9	0.01\\
9.01	0.01\\
9.02	0.01\\
9.03	0.01\\
9.04	0.01\\
9.05	0.01\\
9.06	0.01\\
9.07	0.01\\
9.08	0.01\\
9.09	0.01\\
9.1	0.01\\
9.11	0.01\\
9.12	0.01\\
9.13	0.01\\
9.14	0.01\\
9.15	0.01\\
9.16	0.01\\
9.17	0.01\\
9.18	0.01\\
9.19	0.01\\
9.2	0.01\\
9.21	0.01\\
9.22	0.01\\
9.23	0.01\\
9.24	0.01\\
9.25	0.01\\
9.26	0.01\\
9.27	0.01\\
9.28	0.01\\
9.29	0.01\\
9.3	0.01\\
9.31	0.01\\
9.32	0.01\\
9.33	0.01\\
9.34	0.01\\
9.35	0.01\\
9.36	0.01\\
9.37	0.01\\
9.38	0.01\\
9.39	0.01\\
9.4	0.01\\
9.41	0.01\\
9.42	0.01\\
9.43	0.01\\
9.44	0.01\\
9.45	0.01\\
9.46	0.01\\
9.47	0.01\\
9.48	0.01\\
9.49	0.01\\
9.5	0.01\\
9.51	0.01\\
9.52	0.01\\
9.53	0.01\\
9.54	0.01\\
9.55	0.01\\
9.56	0.01\\
9.57	0.01\\
9.58	0.01\\
9.59	0.01\\
9.6	0.01\\
9.61	0.01\\
9.62	0.01\\
9.63	0.01\\
9.64	0.01\\
9.65	0.01\\
9.66	0.01\\
9.67	0.01\\
9.68	0.01\\
9.69	0.01\\
9.7	0.01\\
9.71	0.01\\
9.72	0.01\\
9.73	0.01\\
9.74	0.01\\
9.75	0.01\\
9.76	0.01\\
9.77	0.01\\
9.78	0.01\\
9.79	0.01\\
9.8	0.01\\
9.81	0.01\\
9.82	0.01\\
9.83	0.01\\
9.84	0.01\\
9.85	0.01\\
9.86	0.01\\
9.87	0.01\\
9.88	0.01\\
9.89	0.01\\
9.9	0.01\\
9.91	0.01\\
9.92	0.01\\
9.93	0.01\\
9.94	0.01\\
9.95	0.01\\
9.96	0.01\\
9.97	0.01\\
9.98	0.01\\
9.99	0.01\\
10	0.01\\
10.01	0.01\\
10.02	0.01\\
10.03	0.01\\
10.04	0.01\\
10.05	0.01\\
10.06	0.01\\
10.07	0.01\\
10.08	0.01\\
10.09	0.01\\
10.1	0.01\\
10.11	0.01\\
10.12	0.01\\
10.13	0.01\\
10.14	0.01\\
10.15	0.01\\
10.16	0.01\\
10.17	0.01\\
10.18	0.01\\
10.19	0.01\\
10.2	0.01\\
10.21	0.01\\
10.22	0.01\\
10.23	0.01\\
10.24	0.01\\
10.25	0.01\\
10.26	0.01\\
10.27	0.01\\
10.28	0.01\\
10.29	0.01\\
10.3	0.01\\
10.31	0.01\\
10.32	0.01\\
10.33	0.01\\
10.34	0.01\\
10.35	0.01\\
10.36	0.01\\
10.37	0.01\\
10.38	0.01\\
10.39	0.01\\
10.4	0.01\\
10.41	0.01\\
10.42	0.01\\
10.43	0.01\\
10.44	0.01\\
10.45	0.01\\
10.46	0.01\\
10.47	0.01\\
10.48	0.01\\
10.49	0.01\\
10.5	0.01\\
10.51	0.01\\
10.52	0.01\\
10.53	0.01\\
10.54	0.01\\
10.55	0.01\\
10.56	0.01\\
10.57	0.01\\
10.58	0.01\\
10.59	0.01\\
10.6	0.01\\
10.61	0.01\\
10.62	0.01\\
10.63	0.01\\
10.64	0.01\\
10.65	0.01\\
10.66	0.01\\
10.67	0.01\\
10.68	0.01\\
10.69	0.01\\
10.7	0.01\\
10.71	0.01\\
10.72	0.01\\
10.73	0.01\\
10.74	0.01\\
10.75	0.01\\
10.76	0.01\\
10.77	0.01\\
10.78	0.01\\
10.79	0.01\\
10.8	0.01\\
10.81	0.01\\
10.82	0.01\\
10.83	0.01\\
10.84	0.01\\
10.85	0.01\\
10.86	0.01\\
10.87	0.01\\
10.88	0.01\\
10.89	0.01\\
10.9	0.01\\
10.91	0.01\\
10.92	0.01\\
10.93	0.01\\
10.94	0.01\\
10.95	0.01\\
10.96	0.01\\
10.97	0.01\\
10.98	0.01\\
10.99	0.01\\
11	0.01\\
11.01	0.01\\
11.02	0.01\\
11.03	0.01\\
11.04	0.01\\
11.05	0.01\\
11.06	0.01\\
11.07	0.01\\
11.08	0.01\\
11.09	0.01\\
11.1	0.01\\
11.11	0.01\\
11.12	0.01\\
11.13	0.01\\
11.14	0.01\\
11.15	0.01\\
11.16	0.01\\
11.17	0.01\\
11.18	0.01\\
11.19	0.01\\
11.2	0.01\\
11.21	0.01\\
11.22	0.01\\
11.23	0.01\\
11.24	0.01\\
11.25	0.01\\
11.26	0.01\\
11.27	0.01\\
11.28	0.01\\
11.29	0.01\\
11.3	0.01\\
11.31	0.01\\
11.32	0.01\\
11.33	0.01\\
11.34	0.01\\
11.35	0.01\\
11.36	0.01\\
11.37	0.01\\
11.38	0.01\\
11.39	0.01\\
11.4	0.01\\
11.41	0.01\\
11.42	0.01\\
11.43	0.01\\
11.44	0.01\\
11.45	0.01\\
11.46	0.01\\
11.47	0.01\\
11.48	0.01\\
11.49	0.01\\
11.5	0.01\\
11.51	0.01\\
11.52	0.01\\
11.53	0.01\\
11.54	0.01\\
11.55	0.01\\
11.56	0.01\\
11.57	0.01\\
11.58	0.01\\
11.59	0.01\\
11.6	0.01\\
11.61	0.01\\
11.62	0.01\\
11.63	0.01\\
11.64	0.01\\
11.65	0.01\\
11.66	0.01\\
11.67	0.01\\
11.68	0.01\\
11.69	0.01\\
11.7	0.01\\
11.71	0.01\\
11.72	0.01\\
11.73	0.01\\
11.74	0.01\\
11.75	0.01\\
11.76	0.01\\
11.77	0.01\\
11.78	0.01\\
11.79	0.01\\
11.8	0.01\\
11.81	0.01\\
11.82	0.01\\
11.83	0.01\\
11.84	0.01\\
11.85	0.01\\
11.86	0.01\\
11.87	0.01\\
11.88	0.01\\
11.89	0.01\\
11.9	0.01\\
11.91	0.01\\
11.92	0.01\\
11.93	0.01\\
11.94	0.01\\
11.95	0.01\\
11.96	0.01\\
11.97	0.01\\
11.98	0.01\\
11.99	0.01\\
12	0.01\\
12.01	0.01\\
12.02	0.01\\
12.03	0.01\\
12.04	0.01\\
12.05	0.01\\
12.06	0.01\\
12.07	0.01\\
12.08	0.01\\
12.09	0.01\\
12.1	0.01\\
12.11	0.01\\
12.12	0.01\\
12.13	0.01\\
12.14	0.01\\
12.15	0.01\\
12.16	0.01\\
12.17	0.01\\
12.18	0.01\\
12.19	0.01\\
12.2	0.01\\
12.21	0.01\\
12.22	0.01\\
12.23	0.01\\
12.24	0.01\\
12.25	0.01\\
12.26	0.01\\
12.27	0.01\\
12.28	0.01\\
12.29	0.01\\
12.3	0.01\\
12.31	0.01\\
12.32	0.01\\
12.33	0.01\\
12.34	0.01\\
12.35	0.01\\
12.36	0.01\\
12.37	0.01\\
12.38	0.01\\
12.39	0.01\\
12.4	0.01\\
12.41	0.01\\
12.42	0.01\\
12.43	0.01\\
12.44	0.01\\
12.45	0.01\\
12.46	0.01\\
12.47	0.01\\
12.48	0.01\\
12.49	0.01\\
12.5	0.01\\
12.51	0.01\\
12.52	0.01\\
12.53	0.01\\
12.54	0.01\\
12.55	0.01\\
12.56	0.01\\
12.57	0.01\\
12.58	0.01\\
12.59	0.01\\
12.6	0.01\\
12.61	0.01\\
12.62	0.01\\
12.63	0.01\\
12.64	0.01\\
12.65	0.01\\
12.66	0.01\\
12.67	0.01\\
12.68	0.01\\
12.69	0.01\\
12.7	0.01\\
12.71	0.01\\
12.72	0.01\\
12.73	0.01\\
12.74	0.01\\
12.75	0.01\\
12.76	0.01\\
12.77	0.01\\
12.78	0.01\\
12.79	0.01\\
12.8	0.01\\
12.81	0.01\\
12.82	0.01\\
12.83	0.01\\
12.84	0.01\\
12.85	0.01\\
12.86	0.01\\
12.87	0.01\\
12.88	0.01\\
12.89	0.01\\
12.9	0.01\\
12.91	0.01\\
12.92	0.01\\
12.93	0.01\\
12.94	0.01\\
12.95	0.01\\
12.96	0.01\\
12.97	0.01\\
12.98	0.01\\
12.99	0.01\\
13	0.01\\
13.01	0.01\\
13.02	0.01\\
13.03	0.01\\
13.04	0.01\\
13.05	0.01\\
13.06	0.01\\
13.07	0.01\\
13.08	0.01\\
13.09	0.01\\
13.1	0.01\\
13.11	0.01\\
13.12	0.01\\
13.13	0.01\\
13.14	0.01\\
13.15	0.01\\
13.16	0.01\\
13.17	0.01\\
13.18	0.01\\
13.19	0.01\\
13.2	0.01\\
13.21	0.01\\
13.22	0.01\\
13.23	0.01\\
13.24	0.01\\
13.25	0.01\\
13.26	0.01\\
13.27	0.01\\
13.28	0.01\\
13.29	0.01\\
13.3	0.01\\
13.31	0.01\\
13.32	0.01\\
13.33	0.01\\
13.34	0.01\\
13.35	0.01\\
13.36	0.01\\
13.37	0.01\\
13.38	0.01\\
13.39	0.01\\
13.4	0.01\\
13.41	0.01\\
13.42	0.01\\
13.43	0.01\\
13.44	0.01\\
13.45	0.01\\
13.46	0.01\\
13.47	0.01\\
13.48	0.01\\
13.49	0.01\\
13.5	0.01\\
13.51	0.01\\
13.52	0.01\\
13.53	0.01\\
13.54	0.01\\
13.55	0.01\\
13.56	0.01\\
13.57	0.01\\
13.58	0.01\\
13.59	0.01\\
13.6	0.01\\
13.61	0.01\\
13.62	0.01\\
13.63	0.01\\
13.64	0.01\\
13.65	0.01\\
13.66	0.01\\
13.67	0.01\\
13.68	0.01\\
13.69	0.01\\
13.7	0.01\\
13.71	0.01\\
13.72	0.01\\
13.73	0.01\\
13.74	0.01\\
13.75	0.01\\
13.76	0.01\\
13.77	0.01\\
13.78	0.01\\
13.79	0.01\\
13.8	0.01\\
13.81	0.01\\
13.82	0.01\\
13.83	0.01\\
13.84	0.01\\
13.85	0.01\\
13.86	0.01\\
13.87	0.01\\
13.88	0.01\\
13.89	0.01\\
13.9	0.01\\
13.91	0.01\\
13.92	0.01\\
13.93	0.01\\
13.94	0.01\\
13.95	0.01\\
13.96	0.01\\
13.97	0.01\\
13.98	0.01\\
13.99	0.01\\
14	0.01\\
14.01	0.01\\
14.02	0.01\\
14.03	0.01\\
14.04	0.01\\
14.05	0.01\\
14.06	0.01\\
14.07	0.01\\
14.08	0.01\\
14.09	0.01\\
14.1	0.01\\
14.11	0.01\\
14.12	0.01\\
14.13	0.01\\
14.14	0.01\\
14.15	0.01\\
14.16	0.01\\
14.17	0.01\\
14.18	0.01\\
14.19	0.01\\
14.2	0.01\\
14.21	0.01\\
14.22	0.01\\
14.23	0.01\\
14.24	0.01\\
14.25	0.01\\
14.26	0.01\\
14.27	0.01\\
14.28	0.01\\
14.29	0.01\\
14.3	0.01\\
14.31	0.01\\
14.32	0.01\\
14.33	0.01\\
14.34	0.01\\
14.35	0.01\\
14.36	0.01\\
14.37	0.01\\
14.38	0.01\\
14.39	0.01\\
14.4	0.01\\
14.41	0.01\\
14.42	0.01\\
14.43	0.01\\
14.44	0.01\\
14.45	0.01\\
14.46	0.01\\
14.47	0.01\\
14.48	0.01\\
14.49	0.01\\
14.5	0.01\\
14.51	0.01\\
14.52	0.01\\
14.53	0.01\\
14.54	0.01\\
14.55	0.01\\
14.56	0.01\\
14.57	0.01\\
14.58	0.01\\
14.59	0.01\\
14.6	0.01\\
14.61	0.01\\
14.62	0.01\\
14.63	0.01\\
14.64	0.01\\
14.65	0.01\\
14.66	0.01\\
14.67	0.01\\
14.68	0.01\\
14.69	0.01\\
14.7	0.01\\
14.71	0.01\\
14.72	0.01\\
14.73	0.01\\
14.74	0.01\\
14.75	0.01\\
14.76	0.01\\
14.77	0.01\\
14.78	0.01\\
14.79	0.01\\
14.8	0.01\\
14.81	0.01\\
14.82	0.01\\
14.83	0.01\\
14.84	0.01\\
14.85	0.01\\
14.86	0.01\\
14.87	0.01\\
14.88	0.01\\
14.89	0.01\\
14.9	0.01\\
14.91	0.01\\
14.92	0.01\\
14.93	0.01\\
14.94	0.01\\
14.95	0.01\\
14.96	0.01\\
14.97	0.01\\
14.98	0.01\\
14.99	0.01\\
15	0.01\\
15.01	0.01\\
15.02	0.01\\
15.03	0.01\\
15.04	0.01\\
15.05	0.01\\
15.06	0.01\\
15.07	0.01\\
15.08	0.01\\
15.09	0.01\\
15.1	0.01\\
15.11	0.01\\
15.12	0.01\\
15.13	0.01\\
15.14	0.01\\
15.15	0.01\\
15.16	0.01\\
15.17	0.01\\
15.18	0.01\\
15.19	0.01\\
15.2	0.01\\
15.21	0.01\\
15.22	0.01\\
15.23	0.01\\
15.24	0.01\\
15.25	0.01\\
15.26	0.01\\
15.27	0.01\\
15.28	0.01\\
15.29	0.01\\
15.3	0.01\\
15.31	0.01\\
15.32	0.01\\
15.33	0.01\\
15.34	0.01\\
15.35	0.01\\
15.36	0.01\\
15.37	0.01\\
15.38	0.01\\
15.39	0.01\\
15.4	0.01\\
15.41	0.01\\
15.42	0.01\\
15.43	0.01\\
15.44	0.01\\
15.45	0.01\\
15.46	0.01\\
15.47	0.01\\
15.48	0.01\\
15.49	0.01\\
15.5	0.01\\
15.51	0.01\\
15.52	0.01\\
15.53	0.01\\
15.54	0.01\\
15.55	0.01\\
15.56	0.01\\
15.57	0.01\\
15.58	0.01\\
15.59	0.01\\
15.6	0.01\\
15.61	0.01\\
15.62	0.01\\
15.63	0.01\\
15.64	0.01\\
15.65	0.01\\
15.66	0.01\\
15.67	0.01\\
15.68	0.01\\
15.69	0.01\\
15.7	0.01\\
15.71	0.01\\
15.72	0.01\\
15.73	0.01\\
15.74	0.01\\
15.75	0.01\\
15.76	0.01\\
15.77	0.01\\
15.78	0.01\\
15.79	0.01\\
15.8	0.01\\
15.81	0.01\\
15.82	0.01\\
15.83	0.01\\
15.84	0.01\\
15.85	0.01\\
15.86	0.01\\
15.87	0.01\\
15.88	0.01\\
15.89	0.01\\
15.9	0.01\\
15.91	0.01\\
15.92	0.01\\
15.93	0.01\\
15.94	0.01\\
15.95	0.01\\
15.96	0.01\\
15.97	0.01\\
15.98	0.01\\
15.99	0.01\\
16	0.01\\
16.01	0.01\\
16.02	0.01\\
16.03	0.01\\
16.04	0.01\\
16.05	0.01\\
16.06	0.01\\
16.07	0.01\\
16.08	0.01\\
16.09	0.01\\
16.1	0.01\\
16.11	0.01\\
16.12	0.01\\
16.13	0.01\\
16.14	0.01\\
16.15	0.01\\
16.16	0.01\\
16.17	0.01\\
16.18	0.01\\
16.19	0.01\\
16.2	0.01\\
16.21	0.01\\
16.22	0.01\\
16.23	0.01\\
16.24	0.01\\
16.25	0.01\\
16.26	0.01\\
16.27	0.01\\
16.28	0.01\\
16.29	0.01\\
16.3	0.01\\
16.31	0.01\\
16.32	0.01\\
16.33	0.01\\
16.34	0.01\\
16.35	0.01\\
16.36	0.01\\
16.37	0.01\\
16.38	0.01\\
16.39	0.01\\
16.4	0.01\\
16.41	0.01\\
16.42	0.01\\
16.43	0.01\\
16.44	0.01\\
16.45	0.01\\
16.46	0.01\\
16.47	0.01\\
16.48	0.01\\
16.49	0.01\\
16.5	0.01\\
16.51	0.01\\
16.52	0.01\\
16.53	0.01\\
16.54	0.01\\
16.55	0.01\\
16.56	0.01\\
16.57	0.01\\
16.58	0.01\\
16.59	0.01\\
16.6	0.01\\
16.61	0.01\\
16.62	0.01\\
16.63	0.01\\
16.64	0.01\\
16.65	0.01\\
16.66	0.01\\
16.67	0.01\\
16.68	0.01\\
16.69	0.01\\
16.7	0.01\\
16.71	0.01\\
16.72	0.01\\
16.73	0.01\\
16.74	0.01\\
16.75	0.01\\
16.76	0.01\\
16.77	0.01\\
16.78	0.01\\
16.79	0.01\\
16.8	0.01\\
16.81	0.01\\
16.82	0.01\\
16.83	0.01\\
16.84	0.01\\
16.85	0.01\\
16.86	0.01\\
16.87	0.01\\
16.88	0.01\\
16.89	0.01\\
16.9	0.01\\
16.91	0.01\\
16.92	0.01\\
16.93	0.01\\
16.94	0.01\\
16.95	0.01\\
16.96	0.01\\
16.97	0.01\\
16.98	0.01\\
16.99	0.01\\
17	0.01\\
17.01	0.01\\
17.02	0.01\\
17.03	0.01\\
17.04	0.01\\
17.05	0.01\\
17.06	0.01\\
17.07	0.01\\
17.08	0.01\\
17.09	0.01\\
17.1	0.01\\
17.11	0.01\\
17.12	0.01\\
17.13	0.01\\
17.14	0.01\\
17.15	0.01\\
17.16	0.01\\
17.17	0.01\\
17.18	0.01\\
17.19	0.01\\
17.2	0.01\\
17.21	0.01\\
17.22	0.01\\
17.23	0.01\\
17.24	0.01\\
17.25	0.01\\
17.26	0.01\\
17.27	0.01\\
17.28	0.01\\
17.29	0.01\\
17.3	0.01\\
17.31	0.01\\
17.32	0.01\\
17.33	0.01\\
17.34	0.01\\
17.35	0.01\\
17.36	0.01\\
17.37	0.01\\
17.38	0.01\\
17.39	0.01\\
17.4	0.01\\
17.41	0.01\\
17.42	0.01\\
17.43	0.01\\
17.44	0.01\\
17.45	0.01\\
17.46	0.01\\
17.47	0.01\\
17.48	0.01\\
17.49	0.01\\
17.5	0.01\\
17.51	0.01\\
17.52	0.01\\
17.53	0.01\\
17.54	0.01\\
17.55	0.01\\
17.56	0.01\\
17.57	0.01\\
17.58	0.01\\
17.59	0.01\\
17.6	0.01\\
17.61	0.01\\
17.62	0.01\\
17.63	0.01\\
17.64	0.01\\
17.65	0.01\\
17.66	0.01\\
17.67	0.01\\
17.68	0.01\\
17.69	0.01\\
17.7	0.01\\
17.71	0.01\\
17.72	0.01\\
17.73	0.01\\
17.74	0.01\\
17.75	0.01\\
17.76	0.01\\
17.77	0.01\\
17.78	0.01\\
17.79	0.01\\
17.8	0.01\\
17.81	0.01\\
17.82	0.01\\
17.83	0.01\\
17.84	0.01\\
17.85	0.01\\
17.86	0.01\\
17.87	0.01\\
17.88	0.01\\
17.89	0.01\\
17.9	0.01\\
17.91	0.01\\
17.92	0.01\\
17.93	0.01\\
17.94	0.01\\
17.95	0.01\\
17.96	0.01\\
17.97	0.01\\
17.98	0.01\\
17.99	0.01\\
18	0.01\\
18.01	0.01\\
18.02	0.01\\
18.03	0.01\\
18.04	0.01\\
18.05	0.01\\
18.06	0.01\\
18.07	0.01\\
18.08	0.01\\
18.09	0.01\\
18.1	0.01\\
18.11	0.01\\
18.12	0.01\\
18.13	0.01\\
18.14	0.01\\
18.15	0.01\\
18.16	0.01\\
18.17	0.01\\
18.18	0.01\\
18.19	0.01\\
18.2	0.01\\
18.21	0.01\\
18.22	0.01\\
18.23	0.01\\
18.24	0.01\\
18.25	0.01\\
18.26	0.01\\
18.27	0.01\\
18.28	0.01\\
18.29	0.01\\
18.3	0.01\\
18.31	0.01\\
18.32	0.01\\
18.33	0.01\\
18.34	0.01\\
18.35	0.01\\
18.36	0.01\\
18.37	0.01\\
18.38	0.01\\
18.39	0.01\\
18.4	0.01\\
18.41	0.01\\
18.42	0.01\\
18.43	0.01\\
18.44	0.01\\
18.45	0.01\\
18.46	0.01\\
18.47	0.01\\
18.48	0.01\\
18.49	0.01\\
18.5	0.01\\
18.51	0.01\\
18.52	0.01\\
18.53	0.01\\
18.54	0.01\\
18.55	0.01\\
18.56	0.01\\
18.57	0.01\\
18.58	0.01\\
18.59	0.01\\
18.6	0.01\\
18.61	0.01\\
18.62	0.01\\
18.63	0.01\\
18.64	0.01\\
18.65	0.01\\
18.66	0.01\\
18.67	0.01\\
18.68	0.01\\
18.69	0.01\\
18.7	0.01\\
18.71	0.01\\
18.72	0.01\\
18.73	0.01\\
18.74	0.01\\
18.75	0.01\\
18.76	0.01\\
18.77	0.01\\
18.78	0.01\\
18.79	0.01\\
18.8	0.01\\
18.81	0.01\\
18.82	0.01\\
18.83	0.01\\
18.84	0.01\\
18.85	0.01\\
18.86	0.01\\
18.87	0.01\\
18.88	0.01\\
18.89	0.01\\
18.9	0.01\\
18.91	0.01\\
18.92	0.01\\
18.93	0.01\\
18.94	0.01\\
18.95	0.01\\
18.96	0.01\\
18.97	0.01\\
18.98	0.01\\
18.99	0.01\\
19	0.01\\
19.01	0.01\\
19.02	0.01\\
19.03	0.01\\
19.04	0.01\\
19.05	0.01\\
19.06	0.01\\
19.07	0.01\\
19.08	0.01\\
19.09	0.01\\
19.1	0.01\\
19.11	0.01\\
19.12	0.01\\
19.13	0.01\\
19.14	0.01\\
19.15	0.01\\
19.16	0.01\\
19.17	0.01\\
19.18	0.01\\
19.19	0.01\\
19.2	0.01\\
19.21	0.01\\
19.22	0.01\\
19.23	0.01\\
19.24	0.01\\
19.25	0.01\\
19.26	0.01\\
19.27	0.01\\
19.28	0.01\\
19.29	0.01\\
19.3	0.01\\
19.31	0.01\\
19.32	0.01\\
19.33	0.01\\
19.34	0.01\\
19.35	0.01\\
19.36	0.01\\
19.37	0.01\\
19.38	0.01\\
19.39	0.01\\
19.4	0.01\\
19.41	0.01\\
19.42	0.01\\
19.43	0.01\\
19.44	0.01\\
19.45	0.01\\
19.46	0.01\\
19.47	0.01\\
19.48	0.01\\
19.49	0.01\\
19.5	0.01\\
19.51	0.01\\
19.52	0.01\\
19.53	0.01\\
19.54	0.01\\
19.55	0.01\\
19.56	0.01\\
19.57	0.01\\
19.58	0.01\\
19.59	0.01\\
19.6	0.01\\
19.61	0.01\\
19.62	0.01\\
19.63	0.01\\
19.64	0.01\\
19.65	0.01\\
19.66	0.01\\
19.67	0.01\\
19.68	0.01\\
19.69	0.01\\
19.7	0.01\\
19.71	0.01\\
19.72	0.01\\
19.73	0.01\\
19.74	0.01\\
19.75	0.01\\
19.76	0.01\\
19.77	0.01\\
19.78	0.01\\
19.79	0.01\\
19.8	0.01\\
19.81	0.01\\
19.82	0.01\\
19.83	0.01\\
19.84	0.01\\
19.85	0.01\\
19.86	0.01\\
19.87	0.01\\
19.88	0.01\\
19.89	0.01\\
19.9	0.01\\
19.91	0.01\\
19.92	0.01\\
19.93	0.01\\
19.94	0.01\\
19.95	0.01\\
19.96	0.01\\
19.97	0.01\\
19.98	0.01\\
19.99	0.01\\
20	0.01\\
20.01	0.01\\
20.02	0.01\\
20.03	0.01\\
20.04	0.01\\
20.05	0.01\\
20.06	0.01\\
20.07	0.01\\
20.08	0.01\\
20.09	0.01\\
20.1	0.01\\
20.11	0.01\\
20.12	0.01\\
20.13	0.01\\
20.14	0.01\\
20.15	0.01\\
20.16	0.01\\
20.17	0.01\\
20.18	0.01\\
20.19	0.01\\
20.2	0.01\\
20.21	0.01\\
20.22	0.01\\
20.23	0.01\\
20.24	0.01\\
20.25	0.01\\
20.26	0.01\\
20.27	0.01\\
20.28	0.01\\
20.29	0.01\\
20.3	0.01\\
20.31	0.01\\
20.32	0.01\\
20.33	0.01\\
20.34	0.01\\
20.35	0.01\\
20.36	0.01\\
20.37	0.01\\
20.38	0.01\\
20.39	0.01\\
20.4	0.01\\
20.41	0.01\\
20.42	0.01\\
20.43	0.01\\
20.44	0.01\\
20.45	0.01\\
20.46	0.01\\
20.47	0.01\\
20.48	0.01\\
20.49	0.01\\
20.5	0.01\\
20.51	0.01\\
20.52	0.01\\
20.53	0.01\\
20.54	0.01\\
20.55	0.01\\
20.56	0.01\\
20.57	0.01\\
20.58	0.01\\
20.59	0.01\\
20.6	0.01\\
20.61	0.01\\
20.62	0.01\\
20.63	0.01\\
20.64	0.01\\
20.65	0.01\\
20.66	0.01\\
20.67	0.01\\
20.68	0.01\\
20.69	0.01\\
20.7	0.01\\
20.71	0.01\\
20.72	0.01\\
20.73	0.01\\
20.74	0.01\\
20.75	0.01\\
20.76	0.01\\
20.77	0.01\\
20.78	0.01\\
20.79	0.01\\
20.8	0.01\\
20.81	0.01\\
20.82	0.01\\
20.83	0.01\\
20.84	0.01\\
20.85	0.01\\
20.86	0.01\\
20.87	0.01\\
20.88	0.01\\
20.89	0.01\\
20.9	0.01\\
20.91	0.01\\
20.92	0.01\\
20.93	0.01\\
20.94	0.01\\
20.95	0.01\\
20.96	0.01\\
20.97	0.01\\
20.98	0.01\\
20.99	0.01\\
21	0.01\\
21.01	0.01\\
21.02	0.01\\
21.03	0.01\\
21.04	0.01\\
21.05	0.01\\
21.06	0.01\\
21.07	0.01\\
21.08	0.01\\
21.09	0.01\\
21.1	0.01\\
21.11	0.01\\
21.12	0.01\\
21.13	0.01\\
21.14	0.01\\
21.15	0.01\\
21.16	0.01\\
21.17	0.01\\
21.18	0.01\\
21.19	0.01\\
21.2	0.01\\
21.21	0.01\\
21.22	0.01\\
21.23	0.01\\
21.24	0.01\\
21.25	0.01\\
21.26	0.01\\
21.27	0.01\\
21.28	0.01\\
21.29	0.01\\
21.3	0.01\\
21.31	0.01\\
21.32	0.01\\
21.33	0.01\\
21.34	0.01\\
21.35	0.01\\
21.36	0.01\\
21.37	0.01\\
21.38	0.01\\
21.39	0.01\\
21.4	0.01\\
21.41	0.01\\
21.42	0.01\\
21.43	0.01\\
21.44	0.01\\
21.45	0.01\\
21.46	0.01\\
21.47	0.01\\
21.48	0.01\\
21.49	0.01\\
21.5	0.01\\
21.51	0.01\\
21.52	0.01\\
21.53	0.01\\
21.54	0.01\\
21.55	0.01\\
21.56	0.01\\
21.57	0.01\\
21.58	0.01\\
21.59	0.01\\
21.6	0.01\\
21.61	0.01\\
21.62	0.01\\
21.63	0.01\\
21.64	0.01\\
21.65	0.01\\
21.66	0.01\\
21.67	0.01\\
21.68	0.01\\
21.69	0.01\\
21.7	0.01\\
21.71	0.01\\
21.72	0.01\\
21.73	0.01\\
21.74	0.01\\
21.75	0.01\\
21.76	0.01\\
21.77	0.01\\
21.78	0.01\\
21.79	0.01\\
21.8	0.01\\
21.81	0.01\\
21.82	0.01\\
21.83	0.01\\
21.84	0.01\\
21.85	0.01\\
21.86	0.01\\
21.87	0.01\\
21.88	0.01\\
21.89	0.01\\
21.9	0.01\\
21.91	0.01\\
21.92	0.01\\
21.93	0.01\\
21.94	0.01\\
21.95	0.01\\
21.96	0.01\\
21.97	0.01\\
21.98	0.01\\
21.99	0.01\\
22	0.01\\
22.01	0.01\\
22.02	0.01\\
22.03	0.01\\
22.04	0.01\\
22.05	0.01\\
22.06	0.01\\
22.07	0.01\\
22.08	0.01\\
22.09	0.01\\
22.1	0.01\\
22.11	0.01\\
22.12	0.01\\
22.13	0.01\\
22.14	0.01\\
22.15	0.01\\
22.16	0.01\\
22.17	0.01\\
22.18	0.01\\
22.19	0.01\\
22.2	0.01\\
22.21	0.01\\
22.22	0.01\\
22.23	0.01\\
22.24	0.01\\
22.25	0.01\\
22.26	0.01\\
22.27	0.01\\
22.28	0.01\\
22.29	0.01\\
22.3	0.01\\
22.31	0.01\\
22.32	0.01\\
22.33	0.01\\
22.34	0.01\\
22.35	0.01\\
22.36	0.01\\
22.37	0.01\\
22.38	0.01\\
22.39	0.01\\
22.4	0.01\\
22.41	0.01\\
22.42	0.01\\
22.43	0.01\\
22.44	0.01\\
22.45	0.01\\
22.46	0.01\\
22.47	0.01\\
22.48	0.01\\
22.49	0.01\\
22.5	0.01\\
22.51	0.01\\
22.52	0.01\\
22.53	0.01\\
22.54	0.01\\
22.55	0.01\\
22.56	0.01\\
22.57	0.01\\
22.58	0.01\\
22.59	0.01\\
22.6	0.01\\
22.61	0.01\\
22.62	0.01\\
22.63	0.01\\
22.64	0.01\\
22.65	0.01\\
22.66	0.01\\
22.67	0.01\\
22.68	0.01\\
22.69	0.01\\
22.7	0.01\\
22.71	0.01\\
22.72	0.01\\
22.73	0.01\\
22.74	0.01\\
22.75	0.01\\
22.76	0.01\\
22.77	0.01\\
22.78	0.01\\
22.79	0.01\\
22.8	0.01\\
22.81	0.01\\
22.82	0.01\\
22.83	0.01\\
22.84	0.01\\
22.85	0.01\\
22.86	0.01\\
22.87	0.01\\
22.88	0.01\\
22.89	0.01\\
22.9	0.01\\
22.91	0.01\\
22.92	0.01\\
22.93	0.01\\
22.94	0.01\\
22.95	0.01\\
22.96	0.01\\
22.97	0.01\\
22.98	0.01\\
22.99	0.01\\
23	0.01\\
23.01	0.01\\
23.02	0.01\\
23.03	0.01\\
23.04	0.01\\
23.05	0.01\\
23.06	0.01\\
23.07	0.01\\
23.08	0.01\\
23.09	0.01\\
23.1	0.01\\
23.11	0.01\\
23.12	0.01\\
23.13	0.01\\
23.14	0.01\\
23.15	0.01\\
23.16	0.01\\
23.17	0.01\\
23.18	0.01\\
23.19	0.01\\
23.2	0.01\\
23.21	0.01\\
23.22	0.01\\
23.23	0.01\\
23.24	0.01\\
23.25	0.01\\
23.26	0.01\\
23.27	0.01\\
23.28	0.01\\
23.29	0.01\\
23.3	0.01\\
23.31	0.01\\
23.32	0.01\\
23.33	0.01\\
23.34	0.01\\
23.35	0.01\\
23.36	0.01\\
23.37	0.01\\
23.38	0.01\\
23.39	0.01\\
23.4	0.01\\
23.41	0.01\\
23.42	0.01\\
23.43	0.01\\
23.44	0.01\\
23.45	0.01\\
23.46	0.01\\
23.47	0.01\\
23.48	0.01\\
23.49	0.01\\
23.5	0.01\\
23.51	0.01\\
23.52	0.01\\
23.53	0.01\\
23.54	0.01\\
23.55	0.01\\
23.56	0.01\\
23.57	0.01\\
23.58	0.01\\
23.59	0.01\\
23.6	0.01\\
23.61	0.01\\
23.62	0.01\\
23.63	0.01\\
23.64	0.01\\
23.65	0.01\\
23.66	0.01\\
23.67	0.01\\
23.68	0.01\\
23.69	0.01\\
23.7	0.01\\
23.71	0.01\\
23.72	0.01\\
23.73	0.01\\
23.74	0.01\\
23.75	0.01\\
23.76	0.01\\
23.77	0.01\\
23.78	0.01\\
23.79	0.01\\
23.8	0.01\\
23.81	0.01\\
23.82	0.01\\
23.83	0.01\\
23.84	0.01\\
23.85	0.01\\
23.86	0.01\\
23.87	0.01\\
23.88	0.01\\
23.89	0.01\\
23.9	0.01\\
23.91	0.01\\
23.92	0.01\\
23.93	0.01\\
23.94	0.01\\
23.95	0.01\\
23.96	0.01\\
23.97	0.01\\
23.98	0.01\\
23.99	0.01\\
24	0.01\\
24.01	0.01\\
24.02	0.01\\
24.03	0.01\\
24.04	0.01\\
24.05	0.01\\
24.06	0.01\\
24.07	0.01\\
24.08	0.01\\
24.09	0.01\\
24.1	0.01\\
24.11	0.01\\
24.12	0.01\\
24.13	0.01\\
24.14	0.01\\
24.15	0.01\\
24.16	0.01\\
24.17	0.01\\
24.18	0.01\\
24.19	0.01\\
24.2	0.01\\
24.21	0.01\\
24.22	0.01\\
24.23	0.01\\
24.24	0.01\\
24.25	0.01\\
24.26	0.01\\
24.27	0.01\\
24.28	0.01\\
24.29	0.01\\
24.3	0.01\\
24.31	0.01\\
24.32	0.01\\
24.33	0.01\\
24.34	0.01\\
24.35	0.01\\
24.36	0.01\\
24.37	0.01\\
24.38	0.01\\
24.39	0.01\\
24.4	0.01\\
24.41	0.01\\
24.42	0.01\\
24.43	0.01\\
24.44	0.01\\
24.45	0.01\\
24.46	0.01\\
24.47	0.01\\
24.48	0.01\\
24.49	0.01\\
24.5	0.01\\
24.51	0.01\\
24.52	0.01\\
24.53	0.01\\
24.54	0.01\\
24.55	0.01\\
24.56	0.01\\
24.57	0.01\\
24.58	0.01\\
24.59	0.01\\
24.6	0.01\\
24.61	0.01\\
24.62	0.01\\
24.63	0.01\\
24.64	0.01\\
24.65	0.01\\
24.66	0.01\\
24.67	0.01\\
24.68	0.01\\
24.69	0.01\\
24.7	0.01\\
24.71	0.01\\
24.72	0.01\\
24.73	0.01\\
24.74	0.01\\
24.75	0.01\\
24.76	0.01\\
24.77	0.01\\
24.78	0.01\\
24.79	0.01\\
24.8	0.01\\
24.81	0.01\\
24.82	0.01\\
24.83	0.01\\
24.84	0.01\\
24.85	0.01\\
24.86	0.01\\
24.87	0.01\\
24.88	0.01\\
24.89	0.01\\
24.9	0.01\\
24.91	0.01\\
24.92	0.01\\
24.93	0.01\\
24.94	0.01\\
24.95	0.01\\
24.96	0.01\\
24.97	0.01\\
24.98	0.01\\
24.99	0.01\\
25	0.01\\
25.01	0.01\\
25.02	0.01\\
25.03	0.01\\
25.04	0.01\\
25.05	0.01\\
25.06	0.01\\
25.07	0.01\\
25.08	0.01\\
25.09	0.01\\
25.1	0.01\\
25.11	0.01\\
25.12	0.01\\
25.13	0.01\\
25.14	0.01\\
25.15	0.01\\
25.16	0.01\\
25.17	0.01\\
25.18	0.01\\
25.19	0.01\\
25.2	0.01\\
25.21	0.01\\
25.22	0.01\\
25.23	0.01\\
25.24	0.01\\
25.25	0.01\\
25.26	0.01\\
25.27	0.01\\
25.28	0.01\\
25.29	0.01\\
25.3	0.01\\
25.31	0.01\\
25.32	0.01\\
25.33	0.01\\
25.34	0.01\\
25.35	0.01\\
25.36	0.01\\
25.37	0.01\\
25.38	0.01\\
25.39	0.01\\
25.4	0.01\\
25.41	0.01\\
25.42	0.01\\
25.43	0.01\\
25.44	0.01\\
25.45	0.01\\
25.46	0.01\\
25.47	0.01\\
25.48	0.01\\
25.49	0.01\\
25.5	0.01\\
25.51	0.01\\
25.52	0.01\\
25.53	0.01\\
25.54	0.01\\
25.55	0.01\\
25.56	0.01\\
25.57	0.01\\
25.58	0.01\\
25.59	0.01\\
25.6	0.01\\
25.61	0.01\\
25.62	0.01\\
25.63	0.01\\
25.64	0.01\\
25.65	0.01\\
25.66	0.01\\
25.67	0.01\\
25.68	0.01\\
25.69	0.01\\
25.7	0.01\\
25.71	0.01\\
25.72	0.01\\
25.73	0.01\\
25.74	0.01\\
25.75	0.01\\
25.76	0.01\\
25.77	0.01\\
25.78	0.01\\
25.79	0.01\\
25.8	0.01\\
25.81	0.01\\
25.82	0.01\\
25.83	0.01\\
25.84	0.01\\
25.85	0.01\\
25.86	0.01\\
25.87	0.01\\
25.88	0.01\\
25.89	0.01\\
25.9	0.01\\
25.91	0.01\\
25.92	0.01\\
25.93	0.01\\
25.94	0.01\\
25.95	0.01\\
25.96	0.01\\
25.97	0.01\\
25.98	0.01\\
25.99	0.01\\
26	0.01\\
26.01	0.01\\
26.02	0.01\\
26.03	0.01\\
26.04	0.01\\
26.05	0.01\\
26.06	0.01\\
26.07	0.01\\
26.08	0.01\\
26.09	0.01\\
26.1	0.01\\
26.11	0.01\\
26.12	0.01\\
26.13	0.01\\
26.14	0.01\\
26.15	0.01\\
26.16	0.01\\
26.17	0.01\\
26.18	0.01\\
26.19	0.01\\
26.2	0.01\\
26.21	0.01\\
26.22	0.01\\
26.23	0.01\\
26.24	0.01\\
26.25	0.01\\
26.26	0.01\\
26.27	0.01\\
26.28	0.01\\
26.29	0.01\\
26.3	0.01\\
26.31	0.01\\
26.32	0.01\\
26.33	0.01\\
26.34	0.01\\
26.35	0.01\\
26.36	0.01\\
26.37	0.01\\
26.38	0.01\\
26.39	0.01\\
26.4	0.01\\
26.41	0.01\\
26.42	0.01\\
26.43	0.01\\
26.44	0.01\\
26.45	0.01\\
26.46	0.01\\
26.47	0.01\\
26.48	0.01\\
26.49	0.01\\
26.5	0.01\\
26.51	0.01\\
26.52	0.01\\
26.53	0.01\\
26.54	0.01\\
26.55	0.01\\
26.56	0.01\\
26.57	0.01\\
26.58	0.01\\
26.59	0.01\\
26.6	0.01\\
26.61	0.01\\
26.62	0.01\\
26.63	0.01\\
26.64	0.01\\
26.65	0.01\\
26.66	0.01\\
26.67	0.01\\
26.68	0.01\\
26.69	0.01\\
26.7	0.01\\
26.71	0.01\\
26.72	0.01\\
26.73	0.01\\
26.74	0.01\\
26.75	0.01\\
26.76	0.01\\
26.77	0.01\\
26.78	0.01\\
26.79	0.01\\
26.8	0.01\\
26.81	0.01\\
26.82	0.01\\
26.83	0.01\\
26.84	0.01\\
26.85	0.01\\
26.86	0.01\\
26.87	0.01\\
26.88	0.01\\
26.89	0.01\\
26.9	0.01\\
26.91	0.01\\
26.92	0.01\\
26.93	0.01\\
26.94	0.01\\
26.95	0.01\\
26.96	0.01\\
26.97	0.01\\
26.98	0.01\\
26.99	0.01\\
27	0.01\\
27.01	0.01\\
27.02	0.01\\
27.03	0.01\\
27.04	0.01\\
27.05	0.01\\
27.06	0.01\\
27.07	0.01\\
27.08	0.01\\
27.09	0.01\\
27.1	0.01\\
27.11	0.01\\
27.12	0.01\\
27.13	0.01\\
27.14	0.01\\
27.15	0.01\\
27.16	0.01\\
27.17	0.01\\
27.18	0.01\\
27.19	0.01\\
27.2	0.01\\
27.21	0.01\\
27.22	0.01\\
27.23	0.01\\
27.24	0.01\\
27.25	0.01\\
27.26	0.01\\
27.27	0.01\\
27.28	0.01\\
27.29	0.01\\
27.3	0.01\\
27.31	0.01\\
27.32	0.01\\
27.33	0.01\\
27.34	0.01\\
27.35	0.01\\
27.36	0.01\\
27.37	0.01\\
27.38	0.01\\
27.39	0.01\\
27.4	0.01\\
27.41	0.01\\
27.42	0.01\\
27.43	0.01\\
27.44	0.01\\
27.45	0.01\\
27.46	0.01\\
27.47	0.01\\
27.48	0.01\\
27.49	0.01\\
27.5	0.01\\
27.51	0.01\\
27.52	0.01\\
27.53	0.01\\
27.54	0.01\\
27.55	0.01\\
27.56	0.01\\
27.57	0.01\\
27.58	0.01\\
27.59	0.01\\
27.6	0.01\\
27.61	0.01\\
27.62	0.01\\
27.63	0.01\\
27.64	0.01\\
27.65	0.01\\
27.66	0.01\\
27.67	0.01\\
27.68	0.01\\
27.69	0.01\\
27.7	0.01\\
27.71	0.01\\
27.72	0.01\\
27.73	0.01\\
27.74	0.01\\
27.75	0.01\\
27.76	0.01\\
27.77	0.01\\
27.78	0.01\\
27.79	0.01\\
27.8	0.01\\
27.81	0.01\\
27.82	0.01\\
27.83	0.01\\
27.84	0.01\\
27.85	0.01\\
27.86	0.01\\
27.87	0.01\\
27.88	0.01\\
27.89	0.01\\
27.9	0.01\\
27.91	0.01\\
27.92	0.01\\
27.93	0.01\\
27.94	0.01\\
27.95	0.01\\
27.96	0.01\\
27.97	0.01\\
27.98	0.01\\
27.99	0.01\\
28	0.01\\
28.01	0.01\\
28.02	0.01\\
28.03	0.01\\
28.04	0.01\\
28.05	0.01\\
28.06	0.01\\
28.07	0.01\\
28.08	0.01\\
28.09	0.01\\
28.1	0.01\\
28.11	0.01\\
28.12	0.01\\
28.13	0.01\\
28.14	0.01\\
28.15	0.01\\
28.16	0.01\\
28.17	0.01\\
28.18	0.01\\
28.19	0.01\\
28.2	0.01\\
28.21	0.01\\
28.22	0.01\\
28.23	0.01\\
28.24	0.01\\
28.25	0.01\\
28.26	0.01\\
28.27	0.01\\
28.28	0.01\\
28.29	0.01\\
28.3	0.01\\
28.31	0.01\\
28.32	0.01\\
28.33	0.01\\
28.34	0.01\\
28.35	0.01\\
28.36	0.01\\
28.37	0.01\\
28.38	0.01\\
28.39	0.01\\
28.4	0.01\\
28.41	0.01\\
28.42	0.01\\
28.43	0.01\\
28.44	0.01\\
28.45	0.01\\
28.46	0.01\\
28.47	0.01\\
28.48	0.01\\
28.49	0.01\\
28.5	0.01\\
28.51	0.01\\
28.52	0.01\\
28.53	0.01\\
28.54	0.01\\
28.55	0.01\\
28.56	0.01\\
28.57	0.01\\
28.58	0.01\\
28.59	0.01\\
28.6	0.01\\
28.61	0.01\\
28.62	0.01\\
28.63	0.01\\
28.64	0.01\\
28.65	0.01\\
28.66	0.01\\
28.67	0.01\\
28.68	0.01\\
28.69	0.01\\
28.7	0.01\\
28.71	0.01\\
28.72	0.01\\
28.73	0.01\\
28.74	0.01\\
28.75	0.01\\
28.76	0.01\\
28.77	0.01\\
28.78	0.01\\
28.79	0.01\\
28.8	0.01\\
28.81	0.01\\
28.82	0.01\\
28.83	0.01\\
28.84	0.01\\
28.85	0.01\\
28.86	0.01\\
28.87	0.01\\
28.88	0.01\\
28.89	0.01\\
28.9	0.01\\
28.91	0.01\\
28.92	0.01\\
28.93	0.01\\
28.94	0.01\\
28.95	0.01\\
28.96	0.01\\
28.97	0.01\\
28.98	0.01\\
28.99	0.01\\
29	0.01\\
29.01	0.01\\
29.02	0.01\\
29.03	0.01\\
29.04	0.01\\
29.05	0.01\\
29.06	0.01\\
29.07	0.01\\
29.08	0.01\\
29.09	0.01\\
29.1	0.01\\
29.11	0.01\\
29.12	0.01\\
29.13	0.01\\
29.14	0.01\\
29.15	0.01\\
29.16	0.01\\
29.17	0.01\\
29.18	0.01\\
29.19	0.01\\
29.2	0.01\\
29.21	0.01\\
29.22	0.01\\
29.23	0.01\\
29.24	0.01\\
29.25	0.01\\
29.26	0.01\\
29.27	0.01\\
29.28	0.01\\
29.29	0.01\\
29.3	0.01\\
29.31	0.01\\
29.32	0.01\\
29.33	0.01\\
29.34	0.01\\
29.35	0.01\\
29.36	0.01\\
29.37	0.01\\
29.38	0.01\\
29.39	0.01\\
29.4	0.01\\
29.41	0.01\\
29.42	0.01\\
29.43	0.01\\
29.44	0.01\\
29.45	0.01\\
29.46	0.01\\
29.47	0.01\\
29.48	0.01\\
29.49	0.01\\
29.5	0.01\\
29.51	0.01\\
29.52	0.01\\
29.53	0.01\\
29.54	0.01\\
29.55	0.01\\
29.56	0.01\\
29.57	0.01\\
29.58	0.01\\
29.59	0.01\\
29.6	0.01\\
29.61	0.01\\
29.62	0.01\\
29.63	0.01\\
29.64	0.01\\
29.65	0.01\\
29.66	0.01\\
29.67	0.01\\
29.68	0.01\\
29.69	0.01\\
29.7	0.01\\
29.71	0.01\\
29.72	0.01\\
29.73	0.01\\
29.74	0.01\\
29.75	0.01\\
29.76	0.01\\
29.77	0.01\\
29.78	0.01\\
29.79	0.01\\
29.8	0.01\\
29.81	0.01\\
29.82	0.01\\
29.83	0.01\\
29.84	0.01\\
29.85	0.01\\
29.86	0.01\\
29.87	0.01\\
29.88	0.01\\
29.89	0.01\\
29.9	0.01\\
29.91	0.01\\
29.92	0.01\\
29.93	0.01\\
29.94	0.01\\
29.95	0.01\\
29.96	0.01\\
29.97	0.01\\
29.98	0.01\\
29.99	0.01\\
30	0.01\\
30.01	0.01\\
30.02	0.01\\
30.03	0.01\\
30.04	0.01\\
30.05	0.01\\
30.06	0.01\\
30.07	0.01\\
30.08	0.01\\
30.09	0.01\\
30.1	0.01\\
30.11	0.01\\
30.12	0.01\\
30.13	0.01\\
30.14	0.01\\
30.15	0.01\\
30.16	0.01\\
30.17	0.01\\
30.18	0.01\\
30.19	0.01\\
30.2	0.01\\
30.21	0.01\\
30.22	0.01\\
30.23	0.01\\
30.24	0.01\\
30.25	0.01\\
30.26	0.01\\
30.27	0.01\\
30.28	0.01\\
30.29	0.01\\
30.3	0.01\\
30.31	0.01\\
30.32	0.01\\
30.33	0.01\\
30.34	0.01\\
30.35	0.01\\
30.36	0.01\\
30.37	0.01\\
30.38	0.01\\
30.39	0.01\\
30.4	0.01\\
30.41	0.01\\
30.42	0.01\\
30.43	0.01\\
30.44	0.01\\
30.45	0.01\\
30.46	0.01\\
30.47	0.01\\
30.48	0.01\\
30.49	0.01\\
30.5	0.01\\
30.51	0.01\\
30.52	0.01\\
30.53	0.01\\
30.54	0.01\\
30.55	0.01\\
30.56	0.01\\
30.57	0.01\\
30.58	0.01\\
30.59	0.01\\
30.6	0.01\\
30.61	0.01\\
30.62	0.01\\
30.63	0.01\\
30.64	0.01\\
30.65	0.01\\
30.66	0.01\\
30.67	0.01\\
30.68	0.01\\
30.69	0.01\\
30.7	0.01\\
30.71	0.01\\
30.72	0.01\\
30.73	0.01\\
30.74	0.01\\
30.75	0.01\\
30.76	0.01\\
30.77	0.01\\
30.78	0.01\\
30.79	0.01\\
30.8	0.01\\
30.81	0.01\\
30.82	0.01\\
30.83	0.01\\
30.84	0.01\\
30.85	0.01\\
30.86	0.01\\
30.87	0.01\\
30.88	0.01\\
30.89	0.01\\
30.9	0.01\\
30.91	0.01\\
30.92	0.01\\
30.93	0.01\\
30.94	0.01\\
30.95	0.01\\
30.96	0.01\\
30.97	0.01\\
30.98	0.01\\
30.99	0.01\\
31	0.01\\
31.01	0.01\\
31.02	0.01\\
31.03	0.01\\
31.04	0.01\\
31.05	0.01\\
31.06	0.01\\
31.07	0.01\\
31.08	0.01\\
31.09	0.01\\
31.1	0.01\\
31.11	0.01\\
31.12	0.01\\
31.13	0.01\\
31.14	0.01\\
31.15	0.01\\
31.16	0.01\\
31.17	0.01\\
31.18	0.01\\
31.19	0.01\\
31.2	0.01\\
31.21	0.01\\
31.22	0.01\\
31.23	0.01\\
31.24	0.01\\
31.25	0.01\\
31.26	0.01\\
31.27	0.01\\
31.28	0.01\\
31.29	0.01\\
31.3	0.01\\
31.31	0.01\\
31.32	0.01\\
31.33	0.01\\
31.34	0.01\\
31.35	0.01\\
31.36	0.01\\
31.37	0.01\\
31.38	0.01\\
31.39	0.01\\
31.4	0.01\\
31.41	0.01\\
31.42	0.01\\
31.43	0.01\\
31.44	0.01\\
31.45	0.01\\
31.46	0.01\\
31.47	0.01\\
31.48	0.01\\
31.49	0.01\\
31.5	0.01\\
31.51	0.01\\
31.52	0.01\\
31.53	0.01\\
31.54	0.01\\
31.55	0.01\\
31.56	0.01\\
31.57	0.01\\
31.58	0.01\\
31.59	0.01\\
31.6	0.01\\
31.61	0.01\\
31.62	0.01\\
31.63	0.01\\
31.64	0.01\\
31.65	0.01\\
31.66	0.01\\
31.67	0.01\\
31.68	0.01\\
31.69	0.01\\
31.7	0.01\\
31.71	0.01\\
31.72	0.01\\
31.73	0.01\\
31.74	0.01\\
31.75	0.01\\
31.76	0.01\\
31.77	0.01\\
31.78	0.01\\
31.79	0.01\\
31.8	0.01\\
31.81	0.01\\
31.82	0.01\\
31.83	0.01\\
31.84	0.01\\
31.85	0.01\\
31.86	0.01\\
31.87	0.01\\
31.88	0.01\\
31.89	0.01\\
31.9	0.01\\
31.91	0.01\\
31.92	0.01\\
31.93	0.01\\
31.94	0.01\\
31.95	0.01\\
31.96	0.01\\
31.97	0.01\\
31.98	0.01\\
31.99	0.01\\
32	0.01\\
32.01	0.01\\
32.02	0.01\\
32.03	0.01\\
32.04	0.01\\
32.05	0.01\\
32.06	0.01\\
32.07	0.01\\
32.08	0.01\\
32.09	0.01\\
32.1	0.01\\
32.11	0.01\\
32.12	0.01\\
32.13	0.01\\
32.14	0.01\\
32.15	0.01\\
32.16	0.01\\
32.17	0.01\\
32.18	0.01\\
32.19	0.01\\
32.2	0.01\\
32.21	0.01\\
32.22	0.01\\
32.23	0.01\\
32.24	0.01\\
32.25	0.01\\
32.26	0.01\\
32.27	0.01\\
32.28	0.01\\
32.29	0.01\\
32.3	0.01\\
32.31	0.01\\
32.32	0.01\\
32.33	0.01\\
32.34	0.01\\
32.35	0.01\\
32.36	0.01\\
32.37	0.01\\
32.38	0.01\\
32.39	0.01\\
32.4	0.01\\
32.41	0.01\\
32.42	0.01\\
32.43	0.01\\
32.44	0.01\\
32.45	0.01\\
32.46	0.01\\
32.47	0.01\\
32.48	0.01\\
32.49	0.01\\
32.5	0.01\\
32.51	0.01\\
32.52	0.01\\
32.53	0.01\\
32.54	0.01\\
32.55	0.01\\
32.56	0.01\\
32.57	0.01\\
32.58	0.01\\
32.59	0.01\\
32.6	0.01\\
32.61	0.01\\
32.62	0.01\\
32.63	0.01\\
32.64	0.01\\
32.65	0.01\\
32.66	0.01\\
32.67	0.01\\
32.68	0.01\\
32.69	0.01\\
32.7	0.01\\
32.71	0.01\\
32.72	0.01\\
32.73	0.01\\
32.74	0.01\\
32.75	0.01\\
32.76	0.01\\
32.77	0.01\\
32.78	0.01\\
32.79	0.01\\
32.8	0.01\\
32.81	0.01\\
32.82	0.01\\
32.83	0.01\\
32.84	0.01\\
32.85	0.01\\
32.86	0.01\\
32.87	0.01\\
32.88	0.01\\
32.89	0.01\\
32.9	0.01\\
32.91	0.01\\
32.92	0.01\\
32.93	0.01\\
32.94	0.01\\
32.95	0.01\\
32.96	0.01\\
32.97	0.01\\
32.98	0.01\\
32.99	0.01\\
33	0.01\\
33.01	0.01\\
33.02	0.01\\
33.03	0.01\\
33.04	0.01\\
33.05	0.01\\
33.06	0.01\\
33.07	0.01\\
33.08	0.01\\
33.09	0.01\\
33.1	0.01\\
33.11	0.01\\
33.12	0.01\\
33.13	0.01\\
33.14	0.01\\
33.15	0.01\\
33.16	0.01\\
33.17	0.01\\
33.18	0.01\\
33.19	0.01\\
33.2	0.01\\
33.21	0.01\\
33.22	0.01\\
33.23	0.01\\
33.24	0.01\\
33.25	0.01\\
33.26	0.01\\
33.27	0.01\\
33.28	0.01\\
33.29	0.01\\
33.3	0.01\\
33.31	0.01\\
33.32	0.01\\
33.33	0.01\\
33.34	0.01\\
33.35	0.01\\
33.36	0.01\\
33.37	0.01\\
33.38	0.01\\
33.39	0.01\\
33.4	0.01\\
33.41	0.01\\
33.42	0.01\\
33.43	0.01\\
33.44	0.01\\
33.45	0.01\\
33.46	0.01\\
33.47	0.01\\
33.48	0.01\\
33.49	0.01\\
33.5	0.01\\
33.51	0.01\\
33.52	0.01\\
33.53	0.01\\
33.54	0.01\\
33.55	0.01\\
33.56	0.01\\
33.57	0.01\\
33.58	0.01\\
33.59	0.01\\
33.6	0.01\\
33.61	0.01\\
33.62	0.01\\
33.63	0.01\\
33.64	0.01\\
33.65	0.01\\
33.66	0.01\\
33.67	0.01\\
33.68	0.01\\
33.69	0.01\\
33.7	0.01\\
33.71	0.01\\
33.72	0.01\\
33.73	0.01\\
33.74	0.01\\
33.75	0.01\\
33.76	0.01\\
33.77	0.01\\
33.78	0.01\\
33.79	0.01\\
33.8	0.01\\
33.81	0.01\\
33.82	0.01\\
33.83	0.01\\
33.84	0.01\\
33.85	0.01\\
33.86	0.01\\
33.87	0.01\\
33.88	0.01\\
33.89	0.01\\
33.9	0.01\\
33.91	0.01\\
33.92	0.01\\
33.93	0.01\\
33.94	0.01\\
33.95	0.01\\
33.96	0.01\\
33.97	0.01\\
33.98	0.01\\
33.99	0.01\\
34	0.01\\
34.01	0.01\\
34.02	0.01\\
34.03	0.01\\
34.04	0.01\\
34.05	0.01\\
34.06	0.01\\
34.07	0.01\\
34.08	0.01\\
34.09	0.01\\
34.1	0.01\\
34.11	0.01\\
34.12	0.01\\
34.13	0.01\\
34.14	0.01\\
34.15	0.01\\
34.16	0.01\\
34.17	0.01\\
34.18	0.01\\
34.19	0.01\\
34.2	0.01\\
34.21	0.01\\
34.22	0.01\\
34.23	0.01\\
34.24	0.01\\
34.25	0.01\\
34.26	0.01\\
34.27	0.01\\
34.28	0.01\\
34.29	0.01\\
34.3	0.01\\
34.31	0.01\\
34.32	0.01\\
34.33	0.01\\
34.34	0.01\\
34.35	0.01\\
34.36	0.01\\
34.37	0.01\\
34.38	0.01\\
34.39	0.01\\
34.4	0.01\\
34.41	0.01\\
34.42	0.01\\
34.43	0.01\\
34.44	0.01\\
34.45	0.01\\
34.46	0.01\\
34.47	0.01\\
34.48	0.01\\
34.49	0.01\\
34.5	0.01\\
34.51	0.01\\
34.52	0.01\\
34.53	0.01\\
34.54	0.01\\
34.55	0.01\\
34.56	0.01\\
34.57	0.01\\
34.58	0.01\\
34.59	0.01\\
34.6	0.01\\
34.61	0.01\\
34.62	0.01\\
34.63	0.01\\
34.64	0.01\\
34.65	0.01\\
34.66	0.01\\
34.67	0.01\\
34.68	0.01\\
34.69	0.01\\
34.7	0.01\\
34.71	0.01\\
34.72	0.01\\
34.73	0.01\\
34.74	0.01\\
34.75	0.01\\
34.76	0.01\\
34.77	0.01\\
34.78	0.01\\
34.79	0.01\\
34.8	0.01\\
34.81	0.01\\
34.82	0.01\\
34.83	0.01\\
34.84	0.01\\
34.85	0.01\\
34.86	0.01\\
34.87	0.01\\
34.88	0.01\\
34.89	0.01\\
34.9	0.01\\
34.91	0.01\\
34.92	0.01\\
34.93	0.01\\
34.94	0.01\\
34.95	0.01\\
34.96	0.01\\
34.97	0.01\\
34.98	0.01\\
34.99	0.01\\
35	0.01\\
35.01	0.01\\
35.02	0.01\\
35.03	0.01\\
35.04	0.01\\
35.05	0.01\\
35.06	0.01\\
35.07	0.01\\
35.08	0.01\\
35.09	0.01\\
35.1	0.01\\
35.11	0.01\\
35.12	0.01\\
35.13	0.01\\
35.14	0.01\\
35.15	0.01\\
35.16	0.01\\
35.17	0.01\\
35.18	0.01\\
35.19	0.01\\
35.2	0.01\\
35.21	0.01\\
35.22	0.01\\
35.23	0.01\\
35.24	0.01\\
35.25	0.01\\
35.26	0.01\\
35.27	0.01\\
35.28	0.01\\
35.29	0.01\\
35.3	0.01\\
35.31	0.01\\
35.32	0.01\\
35.33	0.01\\
35.34	0.01\\
35.35	0.01\\
35.36	0.01\\
35.37	0.01\\
35.38	0.01\\
35.39	0.01\\
35.4	0.01\\
35.41	0.01\\
35.42	0.01\\
35.43	0.01\\
35.44	0.01\\
35.45	0.01\\
35.46	0.01\\
35.47	0.01\\
35.48	0.01\\
35.49	0.01\\
35.5	0.01\\
35.51	0.01\\
35.52	0.01\\
35.53	0.01\\
35.54	0.01\\
35.55	0.01\\
35.56	0.01\\
35.57	0.01\\
35.58	0.01\\
35.59	0.01\\
35.6	0.01\\
35.61	0.01\\
35.62	0.01\\
35.63	0.01\\
35.64	0.01\\
35.65	0.01\\
35.66	0.01\\
35.67	0.01\\
35.68	0.01\\
35.69	0.01\\
35.7	0.01\\
35.71	0.01\\
35.72	0.01\\
35.73	0.01\\
35.74	0.01\\
35.75	0.01\\
35.76	0.01\\
35.77	0.01\\
35.78	0.01\\
35.79	0.01\\
35.8	0.01\\
35.81	0.01\\
35.82	0.01\\
35.83	0.01\\
35.84	0.01\\
35.85	0.01\\
35.86	0.01\\
35.87	0.01\\
35.88	0.01\\
35.89	0.01\\
35.9	0.01\\
35.91	0.01\\
35.92	0.01\\
35.93	0.01\\
35.94	0.01\\
35.95	0.01\\
35.96	0.01\\
35.97	0.01\\
35.98	0.01\\
35.99	0.01\\
36	0.01\\
36.01	0.01\\
36.02	0.01\\
36.03	0.01\\
36.04	0.01\\
36.05	0.01\\
36.06	0.01\\
36.07	0.01\\
36.08	0.01\\
36.09	0.01\\
36.1	0.01\\
36.11	0.01\\
36.12	0.01\\
36.13	0.01\\
36.14	0.01\\
36.15	0.01\\
36.16	0.01\\
36.17	0.01\\
36.18	0.01\\
36.19	0.01\\
36.2	0.01\\
36.21	0.01\\
36.22	0.01\\
36.23	0.01\\
36.24	0.01\\
36.25	0.01\\
36.26	0.01\\
36.27	0.01\\
36.28	0.01\\
36.29	0.01\\
36.3	0.01\\
36.31	0.01\\
36.32	0.01\\
36.33	0.01\\
36.34	0.01\\
36.35	0.01\\
36.36	0.01\\
36.37	0.01\\
36.38	0.01\\
36.39	0.01\\
36.4	0.01\\
36.41	0.01\\
36.42	0.01\\
36.43	0.01\\
36.44	0.01\\
36.45	0.01\\
36.46	0.01\\
36.47	0.01\\
36.48	0.01\\
36.49	0.01\\
36.5	0.01\\
36.51	0.01\\
36.52	0.01\\
36.53	0.01\\
36.54	0.01\\
36.55	0.01\\
36.56	0.01\\
36.57	0.01\\
36.58	0.01\\
36.59	0.01\\
36.6	0.01\\
36.61	0.01\\
36.62	0.01\\
36.63	0.01\\
36.64	0.01\\
36.65	0.01\\
36.66	0.01\\
36.67	0.01\\
36.68	0.01\\
36.69	0.01\\
36.7	0.01\\
36.71	0.01\\
36.72	0.01\\
36.73	0.01\\
36.74	0.01\\
36.75	0.01\\
36.76	0.01\\
36.77	0.01\\
36.78	0.01\\
36.79	0.01\\
36.8	0.01\\
36.81	0.01\\
36.82	0.01\\
36.83	0.01\\
36.84	0.01\\
36.85	0.01\\
36.86	0.01\\
36.87	0.01\\
36.88	0.01\\
36.89	0.01\\
36.9	0.01\\
36.91	0.01\\
36.92	0.01\\
36.93	0.01\\
36.94	0.01\\
36.95	0.01\\
36.96	0.01\\
36.97	0.01\\
36.98	0.01\\
36.99	0.01\\
37	0.01\\
37.01	0.01\\
37.02	0.01\\
37.03	0.01\\
37.04	0.01\\
37.05	0.01\\
37.06	0.01\\
37.07	0.01\\
37.08	0.01\\
37.09	0.01\\
37.1	0.01\\
37.11	0.01\\
37.12	0.01\\
37.13	0.01\\
37.14	0.01\\
37.15	0.01\\
37.16	0.01\\
37.17	0.01\\
37.18	0.01\\
37.19	0.01\\
37.2	0.01\\
37.21	0.01\\
37.22	0.01\\
37.23	0.01\\
37.24	0.01\\
37.25	0.01\\
37.26	0.01\\
37.27	0.01\\
37.28	0.01\\
37.29	0.01\\
37.3	0.01\\
37.31	0.01\\
37.32	0.01\\
37.33	0.01\\
37.34	0.01\\
37.35	0.01\\
37.36	0.01\\
37.37	0.01\\
37.38	0.01\\
37.39	0.01\\
37.4	0.01\\
37.41	0.01\\
37.42	0.01\\
37.43	0.01\\
37.44	0.01\\
37.45	0.01\\
37.46	0.01\\
37.47	0.01\\
37.48	0.01\\
37.49	0.01\\
37.5	0.01\\
37.51	0.01\\
37.52	0.01\\
37.53	0.01\\
37.54	0.01\\
37.55	0.01\\
37.56	0.01\\
37.57	0.01\\
37.58	0.01\\
37.59	0.01\\
37.6	0.01\\
37.61	0.01\\
37.62	0.01\\
37.63	0.01\\
37.64	0.01\\
37.65	0.01\\
37.66	0.01\\
37.67	0.01\\
37.68	0.01\\
37.69	0.01\\
37.7	0.01\\
37.71	0.01\\
37.72	0.01\\
37.73	0.01\\
37.74	0.01\\
37.75	0.01\\
37.76	0.01\\
37.77	0.01\\
37.78	0.01\\
37.79	0.01\\
37.8	0.01\\
37.81	0.01\\
37.82	0.01\\
37.83	0.01\\
37.84	0.01\\
37.85	0.01\\
37.86	0.01\\
37.87	0.01\\
37.88	0.01\\
37.89	0.01\\
37.9	0.01\\
37.91	0.01\\
37.92	0.01\\
37.93	0.01\\
37.94	0.01\\
37.95	0.01\\
37.96	0.01\\
37.97	0.01\\
37.98	0.01\\
37.99	0.01\\
38	0.01\\
38.01	0.01\\
38.02	0.01\\
38.03	0.01\\
38.04	0.01\\
38.05	0.01\\
38.06	0.01\\
38.07	0.01\\
38.08	0.01\\
38.09	0.01\\
38.1	0.01\\
38.11	0.01\\
38.12	0.01\\
38.13	0.01\\
38.14	0.01\\
38.15	0.01\\
38.16	0.01\\
38.17	0.01\\
38.18	0.01\\
38.19	0.01\\
38.2	0.01\\
38.21	0.01\\
38.22	0.01\\
38.23	0.01\\
38.24	0.01\\
38.25	0.01\\
38.26	0.01\\
38.27	0.01\\
38.28	0.01\\
38.29	0.01\\
38.3	0.01\\
38.31	0.01\\
38.32	0.01\\
38.33	0.01\\
38.34	0.01\\
38.35	0.01\\
38.36	0.01\\
38.37	0.01\\
38.38	0.01\\
38.39	0.01\\
38.4	0.01\\
38.41	0.01\\
38.42	0.01\\
38.43	0.01\\
38.44	0.01\\
38.45	0.01\\
38.46	0.01\\
38.47	0.01\\
38.48	0.01\\
38.49	0.01\\
38.5	0.01\\
38.51	0.01\\
38.52	0.01\\
38.53	0.01\\
38.54	0.01\\
38.55	0.01\\
38.56	0.01\\
38.57	0.01\\
38.58	0.01\\
38.59	0.01\\
38.6	0.01\\
38.61	0.01\\
38.62	0.01\\
38.63	0.01\\
38.64	0.01\\
38.65	0.01\\
38.66	0.01\\
38.67	0.01\\
38.68	0.01\\
38.69	0.01\\
38.7	0.01\\
38.71	0.01\\
38.72	0.01\\
38.73	0.01\\
38.74	0.01\\
38.75	0.01\\
38.76	0.01\\
38.77	0.01\\
38.78	0.01\\
38.79	0.01\\
38.8	0.01\\
38.81	0.01\\
38.82	0.01\\
38.83	0.01\\
38.84	0.01\\
38.85	0.01\\
38.86	0.01\\
38.87	0.01\\
38.88	0.01\\
38.89	0.01\\
38.9	0.01\\
38.91	0.01\\
38.92	0.01\\
38.93	0.01\\
38.94	0.01\\
38.95	0.01\\
38.96	0.01\\
38.97	0.01\\
38.98	0.01\\
38.99	0.01\\
39	0.01\\
39.01	0.01\\
39.02	0.01\\
39.03	0.01\\
39.04	0.01\\
39.05	0.01\\
39.06	0.01\\
39.07	0.01\\
39.08	0.01\\
39.09	0.01\\
39.1	0.01\\
39.11	0.01\\
39.12	0.01\\
39.13	0.01\\
39.14	0.01\\
39.15	0.01\\
39.16	0.01\\
39.17	0.01\\
39.18	0.01\\
39.19	0.01\\
39.2	0.01\\
39.21	0.01\\
39.22	0.01\\
39.23	0.01\\
39.24	0.01\\
39.25	0.01\\
39.26	0.01\\
39.27	0.01\\
39.28	0.01\\
39.29	0.01\\
39.3	0.01\\
39.31	0.01\\
39.32	0.01\\
39.33	0.01\\
39.34	0.01\\
39.35	0.01\\
39.36	0.01\\
39.37	0.01\\
39.38	0.01\\
39.39	0.01\\
39.4	0.01\\
39.41	0.01\\
39.42	0.01\\
39.43	0.01\\
39.44	0.01\\
39.45	0.01\\
39.46	0.01\\
39.47	0.01\\
39.48	0.01\\
39.49	0.01\\
39.5	0.01\\
39.51	0.01\\
39.52	0.01\\
39.53	0.01\\
39.54	0.01\\
39.55	0.01\\
39.56	0.01\\
39.57	0.01\\
39.58	0.01\\
39.59	0.01\\
39.6	0.01\\
39.61	0.01\\
39.62	0.01\\
39.63	0.01\\
39.64	0.01\\
39.65	0.01\\
39.66	0.01\\
39.67	0.01\\
39.68	0.01\\
39.69	0.01\\
39.7	0.01\\
39.71	0.01\\
39.72	0.01\\
39.73	0.01\\
39.74	0.01\\
39.75	0.01\\
39.76	0.01\\
39.77	0.01\\
39.78	0.01\\
39.79	0.01\\
39.8	0.01\\
39.81	0.01\\
39.82	0.01\\
39.83	0.01\\
39.84	0.01\\
39.85	0.01\\
39.86	0.01\\
39.87	0.01\\
39.88	0.01\\
39.89	0.01\\
39.9	0.01\\
39.91	0.01\\
39.92	0.01\\
39.93	0.01\\
39.94	0.01\\
39.95	0.01\\
39.96	0.01\\
39.97	0.01\\
39.98	0.01\\
39.99	0.01\\
40	0.01\\
40.01	0.01\\
};
\addplot [color=red,dashed,forget plot]
  table[row sep=crcr]{%
40.01	0.01\\
40.02	0.01\\
40.03	0.01\\
40.04	0.01\\
40.05	0.01\\
40.06	0.01\\
40.07	0.01\\
40.08	0.01\\
40.09	0.01\\
40.1	0.01\\
40.11	0.01\\
40.12	0.01\\
40.13	0.01\\
40.14	0.01\\
40.15	0.01\\
40.16	0.01\\
40.17	0.01\\
40.18	0.01\\
40.19	0.01\\
40.2	0.01\\
40.21	0.01\\
40.22	0.01\\
40.23	0.01\\
40.24	0.01\\
40.25	0.01\\
40.26	0.01\\
40.27	0.01\\
40.28	0.01\\
40.29	0.01\\
40.3	0.01\\
40.31	0.01\\
40.32	0.01\\
40.33	0.01\\
40.34	0.01\\
40.35	0.01\\
40.36	0.01\\
40.37	0.01\\
40.38	0.01\\
40.39	0.01\\
40.4	0.01\\
40.41	0.01\\
40.42	0.01\\
40.43	0.01\\
40.44	0.01\\
40.45	0.01\\
40.46	0.01\\
40.47	0.01\\
40.48	0.01\\
40.49	0.01\\
40.5	0.01\\
40.51	0.01\\
40.52	0.01\\
40.53	0.01\\
40.54	0.01\\
40.55	0.01\\
40.56	0.01\\
40.57	0.01\\
40.58	0.01\\
40.59	0.01\\
40.6	0.01\\
40.61	0.01\\
40.62	0.01\\
40.63	0.01\\
40.64	0.01\\
40.65	0.01\\
40.66	0.01\\
40.67	0.01\\
40.68	0.01\\
40.69	0.01\\
40.7	0.01\\
40.71	0.01\\
40.72	0.01\\
40.73	0.01\\
40.74	0.01\\
40.75	0.01\\
40.76	0.01\\
40.77	0.01\\
40.78	0.01\\
40.79	0.01\\
40.8	0.01\\
40.81	0.01\\
40.82	0.01\\
40.83	0.01\\
40.84	0.01\\
40.85	0.01\\
40.86	0.01\\
40.87	0.01\\
40.88	0.01\\
40.89	0.01\\
40.9	0.01\\
40.91	0.01\\
40.92	0.01\\
40.93	0.01\\
40.94	0.01\\
40.95	0.01\\
40.96	0.01\\
40.97	0.01\\
40.98	0.01\\
40.99	0.01\\
41	0.01\\
41.01	0.01\\
41.02	0.01\\
41.03	0.01\\
41.04	0.01\\
41.05	0.01\\
41.06	0.01\\
41.07	0.01\\
41.08	0.01\\
41.09	0.01\\
41.1	0.01\\
41.11	0.01\\
41.12	0.01\\
41.13	0.01\\
41.14	0.01\\
41.15	0.01\\
41.16	0.01\\
41.17	0.01\\
41.18	0.01\\
41.19	0.01\\
41.2	0.01\\
41.21	0.01\\
41.22	0.01\\
41.23	0.01\\
41.24	0.01\\
41.25	0.01\\
41.26	0.01\\
41.27	0.01\\
41.28	0.01\\
41.29	0.01\\
41.3	0.01\\
41.31	0.01\\
41.32	0.01\\
41.33	0.01\\
41.34	0.01\\
41.35	0.01\\
41.36	0.01\\
41.37	0.01\\
41.38	0.01\\
41.39	0.01\\
41.4	0.01\\
41.41	0.01\\
41.42	0.01\\
41.43	0.01\\
41.44	0.01\\
41.45	0.01\\
41.46	0.01\\
41.47	0.01\\
41.48	0.01\\
41.49	0.01\\
41.5	0.01\\
41.51	0.01\\
41.52	0.01\\
41.53	0.01\\
41.54	0.01\\
41.55	0.01\\
41.56	0.01\\
41.57	0.01\\
41.58	0.01\\
41.59	0.01\\
41.6	0.01\\
41.61	0.01\\
41.62	0.01\\
41.63	0.01\\
41.64	0.01\\
41.65	0.01\\
41.66	0.01\\
41.67	0.01\\
41.68	0.01\\
41.69	0.01\\
41.7	0.01\\
41.71	0.01\\
41.72	0.01\\
41.73	0.01\\
41.74	0.01\\
41.75	0.01\\
41.76	0.01\\
41.77	0.01\\
41.78	0.01\\
41.79	0.01\\
41.8	0.01\\
41.81	0.01\\
41.82	0.01\\
41.83	0.01\\
41.84	0.01\\
41.85	0.01\\
41.86	0.01\\
41.87	0.01\\
41.88	0.01\\
41.89	0.01\\
41.9	0.01\\
41.91	0.01\\
41.92	0.01\\
41.93	0.01\\
41.94	0.01\\
41.95	0.01\\
41.96	0.01\\
41.97	0.01\\
41.98	0.01\\
41.99	0.01\\
42	0.01\\
42.01	0.01\\
42.02	0.01\\
42.03	0.01\\
42.04	0.01\\
42.05	0.01\\
42.06	0.01\\
42.07	0.01\\
42.08	0.01\\
42.09	0.01\\
42.1	0.01\\
42.11	0.01\\
42.12	0.01\\
42.13	0.01\\
42.14	0.01\\
42.15	0.01\\
42.16	0.01\\
42.17	0.01\\
42.18	0.01\\
42.19	0.01\\
42.2	0.01\\
42.21	0.01\\
42.22	0.01\\
42.23	0.01\\
42.24	0.01\\
42.25	0.01\\
42.26	0.01\\
42.27	0.01\\
42.28	0.01\\
42.29	0.01\\
42.3	0.01\\
42.31	0.01\\
42.32	0.01\\
42.33	0.01\\
42.34	0.01\\
42.35	0.01\\
42.36	0.01\\
42.37	0.01\\
42.38	0.01\\
42.39	0.01\\
42.4	0.01\\
42.41	0.01\\
42.42	0.01\\
42.43	0.01\\
42.44	0.01\\
42.45	0.01\\
42.46	0.01\\
42.47	0.01\\
42.48	0.01\\
42.49	0.01\\
42.5	0.01\\
42.51	0.01\\
42.52	0.01\\
42.53	0.01\\
42.54	0.01\\
42.55	0.01\\
42.56	0.01\\
42.57	0.01\\
42.58	0.01\\
42.59	0.01\\
42.6	0.01\\
42.61	0.01\\
42.62	0.01\\
42.63	0.01\\
42.64	0.01\\
42.65	0.01\\
42.66	0.01\\
42.67	0.01\\
42.68	0.01\\
42.69	0.01\\
42.7	0.01\\
42.71	0.01\\
42.72	0.01\\
42.73	0.01\\
42.74	0.01\\
42.75	0.01\\
42.76	0.01\\
42.77	0.01\\
42.78	0.01\\
42.79	0.01\\
42.8	0.01\\
42.81	0.01\\
42.82	0.01\\
42.83	0.01\\
42.84	0.01\\
42.85	0.01\\
42.86	0.01\\
42.87	0.01\\
42.88	0.01\\
42.89	0.01\\
42.9	0.01\\
42.91	0.01\\
42.92	0.01\\
42.93	0.01\\
42.94	0.01\\
42.95	0.01\\
42.96	0.01\\
42.97	0.01\\
42.98	0.01\\
42.99	0.01\\
43	0.01\\
43.01	0.01\\
43.02	0.01\\
43.03	0.01\\
43.04	0.01\\
43.05	0.01\\
43.06	0.01\\
43.07	0.01\\
43.08	0.01\\
43.09	0.01\\
43.1	0.01\\
43.11	0.01\\
43.12	0.01\\
43.13	0.01\\
43.14	0.01\\
43.15	0.01\\
43.16	0.01\\
43.17	0.01\\
43.18	0.01\\
43.19	0.01\\
43.2	0.01\\
43.21	0.01\\
43.22	0.01\\
43.23	0.01\\
43.24	0.01\\
43.25	0.01\\
43.26	0.01\\
43.27	0.01\\
43.28	0.01\\
43.29	0.01\\
43.3	0.01\\
43.31	0.01\\
43.32	0.01\\
43.33	0.01\\
43.34	0.01\\
43.35	0.01\\
43.36	0.01\\
43.37	0.01\\
43.38	0.01\\
43.39	0.01\\
43.4	0.01\\
43.41	0.01\\
43.42	0.01\\
43.43	0.01\\
43.44	0.01\\
43.45	0.01\\
43.46	0.01\\
43.47	0.01\\
43.48	0.01\\
43.49	0.01\\
43.5	0.01\\
43.51	0.01\\
43.52	0.01\\
43.53	0.01\\
43.54	0.01\\
43.55	0.01\\
43.56	0.01\\
43.57	0.01\\
43.58	0.01\\
43.59	0.01\\
43.6	0.01\\
43.61	0.01\\
43.62	0.01\\
43.63	0.01\\
43.64	0.01\\
43.65	0.01\\
43.66	0.01\\
43.67	0.01\\
43.68	0.01\\
43.69	0.01\\
43.7	0.01\\
43.71	0.01\\
43.72	0.01\\
43.73	0.01\\
43.74	0.01\\
43.75	0.01\\
43.76	0.01\\
43.77	0.01\\
43.78	0.01\\
43.79	0.01\\
43.8	0.01\\
43.81	0.01\\
43.82	0.01\\
43.83	0.01\\
43.84	0.01\\
43.85	0.01\\
43.86	0.01\\
43.87	0.01\\
43.88	0.01\\
43.89	0.01\\
43.9	0.01\\
43.91	0.01\\
43.92	0.01\\
43.93	0.01\\
43.94	0.01\\
43.95	0.01\\
43.96	0.01\\
43.97	0.01\\
43.98	0.01\\
43.99	0.01\\
44	0.01\\
44.01	0.01\\
44.02	0.01\\
44.03	0.01\\
44.04	0.01\\
44.05	0.01\\
44.06	0.01\\
44.07	0.01\\
44.08	0.01\\
44.09	0.01\\
44.1	0.01\\
44.11	0.01\\
44.12	0.01\\
44.13	0.01\\
44.14	0.01\\
44.15	0.01\\
44.16	0.01\\
44.17	0.01\\
44.18	0.01\\
44.19	0.01\\
44.2	0.01\\
44.21	0.01\\
44.22	0.01\\
44.23	0.01\\
44.24	0.01\\
44.25	0.01\\
44.26	0.01\\
44.27	0.01\\
44.28	0.01\\
44.29	0.01\\
44.3	0.01\\
44.31	0.01\\
44.32	0.01\\
44.33	0.01\\
44.34	0.01\\
44.35	0.01\\
44.36	0.01\\
44.37	0.01\\
44.38	0.01\\
44.39	0.01\\
44.4	0.01\\
44.41	0.01\\
44.42	0.01\\
44.43	0.01\\
44.44	0.01\\
44.45	0.01\\
44.46	0.01\\
44.47	0.01\\
44.48	0.01\\
44.49	0.01\\
44.5	0.01\\
44.51	0.01\\
44.52	0.01\\
44.53	0.01\\
44.54	0.01\\
44.55	0.01\\
44.56	0.01\\
44.57	0.01\\
44.58	0.01\\
44.59	0.01\\
44.6	0.01\\
44.61	0.01\\
44.62	0.01\\
44.63	0.01\\
44.64	0.01\\
44.65	0.01\\
44.66	0.01\\
44.67	0.01\\
44.68	0.01\\
44.69	0.01\\
44.7	0.01\\
44.71	0.01\\
44.72	0.01\\
44.73	0.01\\
44.74	0.01\\
44.75	0.01\\
44.76	0.01\\
44.77	0.01\\
44.78	0.01\\
44.79	0.01\\
44.8	0.01\\
44.81	0.01\\
44.82	0.01\\
44.83	0.01\\
44.84	0.01\\
44.85	0.01\\
44.86	0.01\\
44.87	0.01\\
44.88	0.01\\
44.89	0.01\\
44.9	0.01\\
44.91	0.01\\
44.92	0.01\\
44.93	0.01\\
44.94	0.01\\
44.95	0.01\\
44.96	0.01\\
44.97	0.01\\
44.98	0.01\\
44.99	0.01\\
45	0.01\\
45.01	0.01\\
45.02	0.01\\
45.03	0.01\\
45.04	0.01\\
45.05	0.01\\
45.06	0.01\\
45.07	0.01\\
45.08	0.01\\
45.09	0.01\\
45.1	0.01\\
45.11	0.01\\
45.12	0.01\\
45.13	0.01\\
45.14	0.01\\
45.15	0.01\\
45.16	0.01\\
45.17	0.01\\
45.18	0.01\\
45.19	0.01\\
45.2	0.01\\
45.21	0.01\\
45.22	0.01\\
45.23	0.01\\
45.24	0.01\\
45.25	0.01\\
45.26	0.01\\
45.27	0.01\\
45.28	0.01\\
45.29	0.01\\
45.3	0.01\\
45.31	0.01\\
45.32	0.01\\
45.33	0.01\\
45.34	0.01\\
45.35	0.01\\
45.36	0.01\\
45.37	0.01\\
45.38	0.01\\
45.39	0.01\\
45.4	0.01\\
45.41	0.01\\
45.42	0.01\\
45.43	0.01\\
45.44	0.01\\
45.45	0.01\\
45.46	0.01\\
45.47	0.01\\
45.48	0.01\\
45.49	0.01\\
45.5	0.01\\
45.51	0.01\\
45.52	0.01\\
45.53	0.01\\
45.54	0.01\\
45.55	0.01\\
45.56	0.01\\
45.57	0.01\\
45.58	0.01\\
45.59	0.01\\
45.6	0.01\\
45.61	0.01\\
45.62	0.01\\
45.63	0.01\\
45.64	0.01\\
45.65	0.01\\
45.66	0.01\\
45.67	0.01\\
45.68	0.01\\
45.69	0.01\\
45.7	0.01\\
45.71	0.01\\
45.72	0.01\\
45.73	0.01\\
45.74	0.01\\
45.75	0.01\\
45.76	0.01\\
45.77	0.01\\
45.78	0.01\\
45.79	0.01\\
45.8	0.01\\
45.81	0.01\\
45.82	0.01\\
45.83	0.01\\
45.84	0.01\\
45.85	0.01\\
45.86	0.01\\
45.87	0.01\\
45.88	0.01\\
45.89	0.01\\
45.9	0.01\\
45.91	0.01\\
45.92	0.01\\
45.93	0.01\\
45.94	0.01\\
45.95	0.01\\
45.96	0.01\\
45.97	0.01\\
45.98	0.01\\
45.99	0.01\\
46	0.01\\
46.01	0.01\\
46.02	0.01\\
46.03	0.01\\
46.04	0.01\\
46.05	0.01\\
46.06	0.01\\
46.07	0.01\\
46.08	0.01\\
46.09	0.01\\
46.1	0.01\\
46.11	0.01\\
46.12	0.01\\
46.13	0.01\\
46.14	0.01\\
46.15	0.01\\
46.16	0.01\\
46.17	0.01\\
46.18	0.01\\
46.19	0.01\\
46.2	0.01\\
46.21	0.01\\
46.22	0.01\\
46.23	0.01\\
46.24	0.01\\
46.25	0.01\\
46.26	0.01\\
46.27	0.01\\
46.28	0.01\\
46.29	0.01\\
46.3	0.01\\
46.31	0.01\\
46.32	0.01\\
46.33	0.01\\
46.34	0.01\\
46.35	0.01\\
46.36	0.01\\
46.37	0.01\\
46.38	0.01\\
46.39	0.01\\
46.4	0.01\\
46.41	0.01\\
46.42	0.01\\
46.43	0.01\\
46.44	0.01\\
46.45	0.01\\
46.46	0.01\\
46.47	0.01\\
46.48	0.01\\
46.49	0.01\\
46.5	0.01\\
46.51	0.01\\
46.52	0.01\\
46.53	0.01\\
46.54	0.01\\
46.55	0.01\\
46.56	0.01\\
46.57	0.01\\
46.58	0.01\\
46.59	0.01\\
46.6	0.01\\
46.61	0.01\\
46.62	0.01\\
46.63	0.01\\
46.64	0.01\\
46.65	0.01\\
46.66	0.01\\
46.67	0.01\\
46.68	0.01\\
46.69	0.01\\
46.7	0.01\\
46.71	0.01\\
46.72	0.01\\
46.73	0.01\\
46.74	0.01\\
46.75	0.01\\
46.76	0.01\\
46.77	0.01\\
46.78	0.01\\
46.79	0.01\\
46.8	0.01\\
46.81	0.01\\
46.82	0.01\\
46.83	0.01\\
46.84	0.01\\
46.85	0.01\\
46.86	0.01\\
46.87	0.01\\
46.88	0.01\\
46.89	0.01\\
46.9	0.01\\
46.91	0.01\\
46.92	0.01\\
46.93	0.01\\
46.94	0.01\\
46.95	0.01\\
46.96	0.01\\
46.97	0.01\\
46.98	0.01\\
46.99	0.01\\
47	0.01\\
47.01	0.01\\
47.02	0.01\\
47.03	0.01\\
47.04	0.01\\
47.05	0.01\\
47.06	0.01\\
47.07	0.01\\
47.08	0.01\\
47.09	0.01\\
47.1	0.01\\
47.11	0.01\\
47.12	0.01\\
47.13	0.01\\
47.14	0.01\\
47.15	0.01\\
47.16	0.01\\
47.17	0.01\\
47.18	0.01\\
47.19	0.01\\
47.2	0.01\\
47.21	0.01\\
47.22	0.01\\
47.23	0.01\\
47.24	0.01\\
47.25	0.01\\
47.26	0.01\\
47.27	0.01\\
47.28	0.01\\
47.29	0.01\\
47.3	0.01\\
47.31	0.01\\
47.32	0.01\\
47.33	0.01\\
47.34	0.01\\
47.35	0.01\\
47.36	0.01\\
47.37	0.01\\
47.38	0.01\\
47.39	0.01\\
47.4	0.01\\
47.41	0.01\\
47.42	0.01\\
47.43	0.01\\
47.44	0.01\\
47.45	0.01\\
47.46	0.01\\
47.47	0.01\\
47.48	0.01\\
47.49	0.01\\
47.5	0.01\\
47.51	0.01\\
47.52	0.01\\
47.53	0.01\\
47.54	0.01\\
47.55	0.01\\
47.56	0.01\\
47.57	0.01\\
47.58	0.01\\
47.59	0.01\\
47.6	0.01\\
47.61	0.01\\
47.62	0.01\\
47.63	0.01\\
47.64	0.01\\
47.65	0.01\\
47.66	0.01\\
47.67	0.01\\
47.68	0.01\\
47.69	0.01\\
47.7	0.01\\
47.71	0.01\\
47.72	0.01\\
47.73	0.01\\
47.74	0.01\\
47.75	0.01\\
47.76	0.01\\
47.77	0.01\\
47.78	0.01\\
47.79	0.01\\
47.8	0.01\\
47.81	0.01\\
47.82	0.01\\
47.83	0.01\\
47.84	0.01\\
47.85	0.01\\
47.86	0.01\\
47.87	0.01\\
47.88	0.01\\
47.89	0.01\\
47.9	0.01\\
47.91	0.01\\
47.92	0.01\\
47.93	0.01\\
47.94	0.01\\
47.95	0.01\\
47.96	0.01\\
47.97	0.01\\
47.98	0.01\\
47.99	0.01\\
48	0.01\\
48.01	0.01\\
48.02	0.01\\
48.03	0.01\\
48.04	0.01\\
48.05	0.01\\
48.06	0.01\\
48.07	0.01\\
48.08	0.01\\
48.09	0.01\\
48.1	0.01\\
48.11	0.01\\
48.12	0.01\\
48.13	0.01\\
48.14	0.01\\
48.15	0.01\\
48.16	0.01\\
48.17	0.01\\
48.18	0.01\\
48.19	0.01\\
48.2	0.01\\
48.21	0.01\\
48.22	0.01\\
48.23	0.01\\
48.24	0.01\\
48.25	0.01\\
48.26	0.01\\
48.27	0.01\\
48.28	0.01\\
48.29	0.01\\
48.3	0.01\\
48.31	0.01\\
48.32	0.01\\
48.33	0.01\\
48.34	0.01\\
48.35	0.01\\
48.36	0.01\\
48.37	0.01\\
48.38	0.01\\
48.39	0.01\\
48.4	0.01\\
48.41	0.01\\
48.42	0.01\\
48.43	0.01\\
48.44	0.01\\
48.45	0.01\\
48.46	0.01\\
48.47	0.01\\
48.48	0.01\\
48.49	0.01\\
48.5	0.01\\
48.51	0.01\\
48.52	0.01\\
48.53	0.01\\
48.54	0.01\\
48.55	0.01\\
48.56	0.01\\
48.57	0.01\\
48.58	0.01\\
48.59	0.01\\
48.6	0.01\\
48.61	0.01\\
48.62	0.01\\
48.63	0.01\\
48.64	0.01\\
48.65	0.01\\
48.66	0.01\\
48.67	0.01\\
48.68	0.01\\
48.69	0.01\\
48.7	0.01\\
48.71	0.01\\
48.72	0.01\\
48.73	0.01\\
48.74	0.01\\
48.75	0.01\\
48.76	0.01\\
48.77	0.01\\
48.78	0.01\\
48.79	0.01\\
48.8	0.01\\
48.81	0.01\\
48.82	0.01\\
48.83	0.01\\
48.84	0.01\\
48.85	0.01\\
48.86	0.01\\
48.87	0.01\\
48.88	0.01\\
48.89	0.01\\
48.9	0.01\\
48.91	0.01\\
48.92	0.01\\
48.93	0.01\\
48.94	0.01\\
48.95	0.01\\
48.96	0.01\\
48.97	0.01\\
48.98	0.01\\
48.99	0.01\\
49	0.01\\
49.01	0.01\\
49.02	0.01\\
49.03	0.01\\
49.04	0.01\\
49.05	0.01\\
49.06	0.01\\
49.07	0.01\\
49.08	0.01\\
49.09	0.01\\
49.1	0.01\\
49.11	0.01\\
49.12	0.01\\
49.13	0.01\\
49.14	0.01\\
49.15	0.01\\
49.16	0.01\\
49.17	0.01\\
49.18	0.01\\
49.19	0.01\\
49.2	0.01\\
49.21	0.01\\
49.22	0.01\\
49.23	0.01\\
49.24	0.01\\
49.25	0.01\\
49.26	0.01\\
49.27	0.01\\
49.28	0.01\\
49.29	0.01\\
49.3	0.01\\
49.31	0.01\\
49.32	0.01\\
49.33	0.01\\
49.34	0.01\\
49.35	0.01\\
49.36	0.01\\
49.37	0.01\\
49.38	0.01\\
49.39	0.01\\
49.4	0.01\\
49.41	0.01\\
49.42	0.01\\
49.43	0.01\\
49.44	0.01\\
49.45	0.01\\
49.46	0.01\\
49.47	0.01\\
49.48	0.01\\
49.49	0.01\\
49.5	0.01\\
49.51	0.01\\
49.52	0.01\\
49.53	0.01\\
49.54	0.01\\
49.55	0.01\\
49.56	0.01\\
49.57	0.01\\
49.58	0.01\\
49.59	0.01\\
49.6	0.01\\
49.61	0.01\\
49.62	0.01\\
49.63	0.01\\
49.64	0.01\\
49.65	0.01\\
49.66	0.01\\
49.67	0.01\\
49.68	0.01\\
49.69	0.01\\
49.7	0.01\\
49.71	0.01\\
49.72	0.01\\
49.73	0.01\\
49.74	0.01\\
49.75	0.01\\
49.76	0.01\\
49.77	0.01\\
49.78	0.01\\
49.79	0.01\\
49.8	0.01\\
49.81	0.01\\
49.82	0.01\\
49.83	0.01\\
49.84	0.01\\
49.85	0.01\\
49.86	0.01\\
49.87	0.01\\
49.88	0.01\\
49.89	0.01\\
49.9	0.01\\
49.91	0.01\\
49.92	0.01\\
49.93	0.01\\
49.94	0.01\\
49.95	0.01\\
49.96	0.01\\
49.97	0.01\\
49.98	0.01\\
49.99	0.01\\
50	0.01\\
50.01	0.01\\
50.02	0.01\\
50.03	0.01\\
50.04	0.01\\
50.05	0.01\\
50.06	0.01\\
50.07	0.01\\
50.08	0.01\\
50.09	0.01\\
50.1	0.01\\
50.11	0.01\\
50.12	0.01\\
50.13	0.01\\
50.14	0.01\\
50.15	0.01\\
50.16	0.01\\
50.17	0.01\\
50.18	0.01\\
50.19	0.01\\
50.2	0.01\\
50.21	0.01\\
50.22	0.01\\
50.23	0.01\\
50.24	0.01\\
50.25	0.01\\
50.26	0.01\\
50.27	0.01\\
50.28	0.01\\
50.29	0.01\\
50.3	0.01\\
50.31	0.01\\
50.32	0.01\\
50.33	0.01\\
50.34	0.01\\
50.35	0.01\\
50.36	0.01\\
50.37	0.01\\
50.38	0.01\\
50.39	0.01\\
50.4	0.01\\
50.41	0.01\\
50.42	0.01\\
50.43	0.01\\
50.44	0.01\\
50.45	0.01\\
50.46	0.01\\
50.47	0.01\\
50.48	0.01\\
50.49	0.01\\
50.5	0.01\\
50.51	0.01\\
50.52	0.01\\
50.53	0.01\\
50.54	0.01\\
50.55	0.01\\
50.56	0.01\\
50.57	0.01\\
50.58	0.01\\
50.59	0.01\\
50.6	0.01\\
50.61	0.01\\
50.62	0.01\\
50.63	0.01\\
50.64	0.01\\
50.65	0.01\\
50.66	0.01\\
50.67	0.01\\
50.68	0.01\\
50.69	0.01\\
50.7	0.01\\
50.71	0.01\\
50.72	0.01\\
50.73	0.01\\
50.74	0.01\\
50.75	0.01\\
50.76	0.01\\
50.77	0.01\\
50.78	0.01\\
50.79	0.01\\
50.8	0.01\\
50.81	0.01\\
50.82	0.01\\
50.83	0.01\\
50.84	0.01\\
50.85	0.01\\
50.86	0.01\\
50.87	0.01\\
50.88	0.01\\
50.89	0.01\\
50.9	0.01\\
50.91	0.01\\
50.92	0.01\\
50.93	0.01\\
50.94	0.01\\
50.95	0.01\\
50.96	0.01\\
50.97	0.01\\
50.98	0.01\\
50.99	0.01\\
51	0.01\\
51.01	0.01\\
51.02	0.01\\
51.03	0.01\\
51.04	0.01\\
51.05	0.01\\
51.06	0.01\\
51.07	0.01\\
51.08	0.01\\
51.09	0.01\\
51.1	0.01\\
51.11	0.01\\
51.12	0.01\\
51.13	0.01\\
51.14	0.01\\
51.15	0.01\\
51.16	0.01\\
51.17	0.01\\
51.18	0.01\\
51.19	0.01\\
51.2	0.01\\
51.21	0.01\\
51.22	0.01\\
51.23	0.01\\
51.24	0.01\\
51.25	0.01\\
51.26	0.01\\
51.27	0.01\\
51.28	0.01\\
51.29	0.01\\
51.3	0.01\\
51.31	0.01\\
51.32	0.01\\
51.33	0.01\\
51.34	0.01\\
51.35	0.01\\
51.36	0.01\\
51.37	0.01\\
51.38	0.01\\
51.39	0.01\\
51.4	0.01\\
51.41	0.01\\
51.42	0.01\\
51.43	0.01\\
51.44	0.01\\
51.45	0.01\\
51.46	0.01\\
51.47	0.01\\
51.48	0.01\\
51.49	0.01\\
51.5	0.01\\
51.51	0.01\\
51.52	0.01\\
51.53	0.01\\
51.54	0.01\\
51.55	0.01\\
51.56	0.01\\
51.57	0.01\\
51.58	0.01\\
51.59	0.01\\
51.6	0.01\\
51.61	0.01\\
51.62	0.01\\
51.63	0.01\\
51.64	0.01\\
51.65	0.01\\
51.66	0.01\\
51.67	0.01\\
51.68	0.01\\
51.69	0.01\\
51.7	0.01\\
51.71	0.01\\
51.72	0.01\\
51.73	0.01\\
51.74	0.01\\
51.75	0.01\\
51.76	0.01\\
51.77	0.01\\
51.78	0.01\\
51.79	0.01\\
51.8	0.01\\
51.81	0.01\\
51.82	0.01\\
51.83	0.01\\
51.84	0.01\\
51.85	0.01\\
51.86	0.01\\
51.87	0.01\\
51.88	0.01\\
51.89	0.01\\
51.9	0.01\\
51.91	0.01\\
51.92	0.01\\
51.93	0.01\\
51.94	0.01\\
51.95	0.01\\
51.96	0.01\\
51.97	0.01\\
51.98	0.01\\
51.99	0.01\\
52	0.01\\
52.01	0.01\\
52.02	0.01\\
52.03	0.01\\
52.04	0.01\\
52.05	0.01\\
52.06	0.01\\
52.07	0.01\\
52.08	0.01\\
52.09	0.01\\
52.1	0.01\\
52.11	0.01\\
52.12	0.01\\
52.13	0.01\\
52.14	0.01\\
52.15	0.01\\
52.16	0.01\\
52.17	0.01\\
52.18	0.01\\
52.19	0.01\\
52.2	0.01\\
52.21	0.01\\
52.22	0.01\\
52.23	0.01\\
52.24	0.01\\
52.25	0.01\\
52.26	0.01\\
52.27	0.01\\
52.28	0.01\\
52.29	0.01\\
52.3	0.01\\
52.31	0.01\\
52.32	0.01\\
52.33	0.01\\
52.34	0.01\\
52.35	0.01\\
52.36	0.01\\
52.37	0.01\\
52.38	0.01\\
52.39	0.01\\
52.4	0.01\\
52.41	0.01\\
52.42	0.01\\
52.43	0.01\\
52.44	0.01\\
52.45	0.01\\
52.46	0.01\\
52.47	0.01\\
52.48	0.01\\
52.49	0.01\\
52.5	0.01\\
52.51	0.01\\
52.52	0.01\\
52.53	0.01\\
52.54	0.01\\
52.55	0.01\\
52.56	0.01\\
52.57	0.01\\
52.58	0.01\\
52.59	0.01\\
52.6	0.01\\
52.61	0.01\\
52.62	0.01\\
52.63	0.01\\
52.64	0.01\\
52.65	0.01\\
52.66	0.01\\
52.67	0.01\\
52.68	0.01\\
52.69	0.01\\
52.7	0.01\\
52.71	0.01\\
52.72	0.01\\
52.73	0.01\\
52.74	0.01\\
52.75	0.01\\
52.76	0.01\\
52.77	0.01\\
52.78	0.01\\
52.79	0.01\\
52.8	0.01\\
52.81	0.01\\
52.82	0.01\\
52.83	0.01\\
52.84	0.01\\
52.85	0.01\\
52.86	0.01\\
52.87	0.01\\
52.88	0.01\\
52.89	0.01\\
52.9	0.01\\
52.91	0.01\\
52.92	0.01\\
52.93	0.01\\
52.94	0.01\\
52.95	0.01\\
52.96	0.01\\
52.97	0.01\\
52.98	0.01\\
52.99	0.01\\
53	0.01\\
53.01	0.01\\
53.02	0.01\\
53.03	0.01\\
53.04	0.01\\
53.05	0.01\\
53.06	0.01\\
53.07	0.01\\
53.08	0.01\\
53.09	0.01\\
53.1	0.01\\
53.11	0.01\\
53.12	0.01\\
53.13	0.01\\
53.14	0.01\\
53.15	0.01\\
53.16	0.01\\
53.17	0.01\\
53.18	0.01\\
53.19	0.01\\
53.2	0.01\\
53.21	0.01\\
53.22	0.01\\
53.23	0.01\\
53.24	0.01\\
53.25	0.01\\
53.26	0.01\\
53.27	0.01\\
53.28	0.01\\
53.29	0.01\\
53.3	0.01\\
53.31	0.01\\
53.32	0.01\\
53.33	0.01\\
53.34	0.01\\
53.35	0.01\\
53.36	0.01\\
53.37	0.01\\
53.38	0.01\\
53.39	0.01\\
53.4	0.01\\
53.41	0.01\\
53.42	0.01\\
53.43	0.01\\
53.44	0.01\\
53.45	0.01\\
53.46	0.01\\
53.47	0.01\\
53.48	0.01\\
53.49	0.01\\
53.5	0.01\\
53.51	0.01\\
53.52	0.01\\
53.53	0.01\\
53.54	0.01\\
53.55	0.01\\
53.56	0.01\\
53.57	0.01\\
53.58	0.01\\
53.59	0.01\\
53.6	0.01\\
53.61	0.01\\
53.62	0.01\\
53.63	0.01\\
53.64	0.01\\
53.65	0.01\\
53.66	0.01\\
53.67	0.01\\
53.68	0.01\\
53.69	0.01\\
53.7	0.01\\
53.71	0.01\\
53.72	0.01\\
53.73	0.01\\
53.74	0.01\\
53.75	0.01\\
53.76	0.01\\
53.77	0.01\\
53.78	0.01\\
53.79	0.01\\
53.8	0.01\\
53.81	0.01\\
53.82	0.01\\
53.83	0.01\\
53.84	0.01\\
53.85	0.01\\
53.86	0.01\\
53.87	0.01\\
53.88	0.01\\
53.89	0.01\\
53.9	0.01\\
53.91	0.01\\
53.92	0.01\\
53.93	0.01\\
53.94	0.01\\
53.95	0.01\\
53.96	0.01\\
53.97	0.01\\
53.98	0.01\\
53.99	0.01\\
54	0.01\\
54.01	0.01\\
54.02	0.01\\
54.03	0.01\\
54.04	0.01\\
54.05	0.01\\
54.06	0.01\\
54.07	0.01\\
54.08	0.01\\
54.09	0.01\\
54.1	0.01\\
54.11	0.01\\
54.12	0.01\\
54.13	0.01\\
54.14	0.01\\
54.15	0.01\\
54.16	0.01\\
54.17	0.01\\
54.18	0.01\\
54.19	0.01\\
54.2	0.01\\
54.21	0.01\\
54.22	0.01\\
54.23	0.01\\
54.24	0.01\\
54.25	0.01\\
54.26	0.01\\
54.27	0.01\\
54.28	0.01\\
54.29	0.01\\
54.3	0.01\\
54.31	0.01\\
54.32	0.01\\
54.33	0.01\\
54.34	0.01\\
54.35	0.01\\
54.36	0.01\\
54.37	0.01\\
54.38	0.01\\
54.39	0.01\\
54.4	0.01\\
54.41	0.01\\
54.42	0.01\\
54.43	0.01\\
54.44	0.01\\
54.45	0.01\\
54.46	0.01\\
54.47	0.01\\
54.48	0.01\\
54.49	0.01\\
54.5	0.01\\
54.51	0.01\\
54.52	0.01\\
54.53	0.01\\
54.54	0.01\\
54.55	0.01\\
54.56	0.01\\
54.57	0.01\\
54.58	0.01\\
54.59	0.01\\
54.6	0.01\\
54.61	0.01\\
54.62	0.01\\
54.63	0.01\\
54.64	0.01\\
54.65	0.01\\
54.66	0.01\\
54.67	0.01\\
54.68	0.01\\
54.69	0.01\\
54.7	0.01\\
54.71	0.01\\
54.72	0.01\\
54.73	0.01\\
54.74	0.01\\
54.75	0.01\\
54.76	0.01\\
54.77	0.01\\
54.78	0.01\\
54.79	0.01\\
54.8	0.01\\
54.81	0.01\\
54.82	0.01\\
54.83	0.01\\
54.84	0.01\\
54.85	0.01\\
54.86	0.01\\
54.87	0.01\\
54.88	0.01\\
54.89	0.01\\
54.9	0.01\\
54.91	0.01\\
54.92	0.01\\
54.93	0.01\\
54.94	0.01\\
54.95	0.01\\
54.96	0.01\\
54.97	0.01\\
54.98	0.01\\
54.99	0.01\\
55	0.01\\
55.01	0.01\\
55.02	0.01\\
55.03	0.01\\
55.04	0.01\\
55.05	0.01\\
55.06	0.01\\
55.07	0.01\\
55.08	0.01\\
55.09	0.01\\
55.1	0.01\\
55.11	0.01\\
55.12	0.01\\
55.13	0.01\\
55.14	0.01\\
55.15	0.01\\
55.16	0.01\\
55.17	0.01\\
55.18	0.01\\
55.19	0.01\\
55.2	0.01\\
55.21	0.01\\
55.22	0.01\\
55.23	0.01\\
55.24	0.01\\
55.25	0.01\\
55.26	0.01\\
55.27	0.01\\
55.28	0.01\\
55.29	0.01\\
55.3	0.01\\
55.31	0.01\\
55.32	0.01\\
55.33	0.01\\
55.34	0.01\\
55.35	0.01\\
55.36	0.01\\
55.37	0.01\\
55.38	0.01\\
55.39	0.01\\
55.4	0.01\\
55.41	0.01\\
55.42	0.01\\
55.43	0.01\\
55.44	0.01\\
55.45	0.01\\
55.46	0.01\\
55.47	0.01\\
55.48	0.01\\
55.49	0.01\\
55.5	0.01\\
55.51	0.01\\
55.52	0.01\\
55.53	0.01\\
55.54	0.01\\
55.55	0.01\\
55.56	0.01\\
55.57	0.01\\
55.58	0.01\\
55.59	0.01\\
55.6	0.01\\
55.61	0.01\\
55.62	0.01\\
55.63	0.01\\
55.64	0.01\\
55.65	0.01\\
55.66	0.01\\
55.67	0.01\\
55.68	0.01\\
55.69	0.01\\
55.7	0.01\\
55.71	0.01\\
55.72	0.01\\
55.73	0.01\\
55.74	0.01\\
55.75	0.01\\
55.76	0.01\\
55.77	0.01\\
55.78	0.01\\
55.79	0.01\\
55.8	0.01\\
55.81	0.01\\
55.82	0.01\\
55.83	0.01\\
55.84	0.01\\
55.85	0.01\\
55.86	0.01\\
55.87	0.01\\
55.88	0.01\\
55.89	0.01\\
55.9	0.01\\
55.91	0.01\\
55.92	0.01\\
55.93	0.01\\
55.94	0.01\\
55.95	0.01\\
55.96	0.01\\
55.97	0.01\\
55.98	0.01\\
55.99	0.01\\
56	0.01\\
56.01	0.01\\
56.02	0.01\\
56.03	0.01\\
56.04	0.01\\
56.05	0.01\\
56.06	0.01\\
56.07	0.01\\
56.08	0.01\\
56.09	0.01\\
56.1	0.01\\
56.11	0.01\\
56.12	0.01\\
56.13	0.01\\
56.14	0.01\\
56.15	0.01\\
56.16	0.01\\
56.17	0.01\\
56.18	0.01\\
56.19	0.01\\
56.2	0.01\\
56.21	0.01\\
56.22	0.01\\
56.23	0.01\\
56.24	0.01\\
56.25	0.01\\
56.26	0.01\\
56.27	0.01\\
56.28	0.01\\
56.29	0.01\\
56.3	0.01\\
56.31	0.01\\
56.32	0.01\\
56.33	0.01\\
56.34	0.01\\
56.35	0.01\\
56.36	0.01\\
56.37	0.01\\
56.38	0.01\\
56.39	0.01\\
56.4	0.01\\
56.41	0.01\\
56.42	0.01\\
56.43	0.01\\
56.44	0.01\\
56.45	0.01\\
56.46	0.01\\
56.47	0.01\\
56.48	0.01\\
56.49	0.01\\
56.5	0.01\\
56.51	0.01\\
56.52	0.01\\
56.53	0.01\\
56.54	0.01\\
56.55	0.01\\
56.56	0.01\\
56.57	0.01\\
56.58	0.01\\
56.59	0.01\\
56.6	0.01\\
56.61	0.01\\
56.62	0.01\\
56.63	0.01\\
56.64	0.01\\
56.65	0.01\\
56.66	0.01\\
56.67	0.01\\
56.68	0.01\\
56.69	0.01\\
56.7	0.01\\
56.71	0.01\\
56.72	0.01\\
56.73	0.01\\
56.74	0.01\\
56.75	0.01\\
56.76	0.01\\
56.77	0.01\\
56.78	0.01\\
56.79	0.01\\
56.8	0.01\\
56.81	0.01\\
56.82	0.01\\
56.83	0.01\\
56.84	0.01\\
56.85	0.01\\
56.86	0.01\\
56.87	0.01\\
56.88	0.01\\
56.89	0.01\\
56.9	0.01\\
56.91	0.01\\
56.92	0.01\\
56.93	0.01\\
56.94	0.01\\
56.95	0.01\\
56.96	0.01\\
56.97	0.01\\
56.98	0.01\\
56.99	0.01\\
57	0.01\\
57.01	0.01\\
57.02	0.01\\
57.03	0.01\\
57.04	0.01\\
57.05	0.01\\
57.06	0.01\\
57.07	0.01\\
57.08	0.01\\
57.09	0.01\\
57.1	0.01\\
57.11	0.01\\
57.12	0.01\\
57.13	0.01\\
57.14	0.01\\
57.15	0.01\\
57.16	0.01\\
57.17	0.01\\
57.18	0.01\\
57.19	0.01\\
57.2	0.01\\
57.21	0.01\\
57.22	0.01\\
57.23	0.01\\
57.24	0.01\\
57.25	0.01\\
57.26	0.01\\
57.27	0.01\\
57.28	0.01\\
57.29	0.01\\
57.3	0.01\\
57.31	0.01\\
57.32	0.01\\
57.33	0.01\\
57.34	0.01\\
57.35	0.01\\
57.36	0.01\\
57.37	0.01\\
57.38	0.01\\
57.39	0.01\\
57.4	0.01\\
57.41	0.01\\
57.42	0.01\\
57.43	0.01\\
57.44	0.01\\
57.45	0.01\\
57.46	0.01\\
57.47	0.01\\
57.48	0.01\\
57.49	0.01\\
57.5	0.01\\
57.51	0.01\\
57.52	0.01\\
57.53	0.01\\
57.54	0.01\\
57.55	0.01\\
57.56	0.01\\
57.57	0.01\\
57.58	0.01\\
57.59	0.01\\
57.6	0.01\\
57.61	0.01\\
57.62	0.01\\
57.63	0.01\\
57.64	0.01\\
57.65	0.01\\
57.66	0.01\\
57.67	0.01\\
57.68	0.01\\
57.69	0.01\\
57.7	0.01\\
57.71	0.01\\
57.72	0.01\\
57.73	0.01\\
57.74	0.01\\
57.75	0.01\\
57.76	0.01\\
57.77	0.01\\
57.78	0.01\\
57.79	0.01\\
57.8	0.01\\
57.81	0.01\\
57.82	0.01\\
57.83	0.01\\
57.84	0.01\\
57.85	0.01\\
57.86	0.01\\
57.87	0.01\\
57.88	0.01\\
57.89	0.01\\
57.9	0.01\\
57.91	0.01\\
57.92	0.01\\
57.93	0.01\\
57.94	0.01\\
57.95	0.01\\
57.96	0.01\\
57.97	0.01\\
57.98	0.01\\
57.99	0.01\\
58	0.01\\
58.01	0.01\\
58.02	0.01\\
58.03	0.01\\
58.04	0.01\\
58.05	0.01\\
58.06	0.01\\
58.07	0.01\\
58.08	0.01\\
58.09	0.01\\
58.1	0.01\\
58.11	0.01\\
58.12	0.01\\
58.13	0.01\\
58.14	0.01\\
58.15	0.01\\
58.16	0.01\\
58.17	0.01\\
58.18	0.01\\
58.19	0.01\\
58.2	0.01\\
58.21	0.01\\
58.22	0.01\\
58.23	0.01\\
58.24	0.01\\
58.25	0.01\\
58.26	0.01\\
58.27	0.01\\
58.28	0.01\\
58.29	0.01\\
58.3	0.01\\
58.31	0.01\\
58.32	0.01\\
58.33	0.01\\
58.34	0.01\\
58.35	0.01\\
58.36	0.01\\
58.37	0.01\\
58.38	0.01\\
58.39	0.01\\
58.4	0.01\\
58.41	0.01\\
58.42	0.01\\
58.43	0.01\\
58.44	0.01\\
58.45	0.01\\
58.46	0.01\\
58.47	0.01\\
58.48	0.01\\
58.49	0.01\\
58.5	0.01\\
58.51	0.01\\
58.52	0.01\\
58.53	0.01\\
58.54	0.01\\
58.55	0.01\\
58.56	0.01\\
58.57	0.01\\
58.58	0.01\\
58.59	0.01\\
58.6	0.01\\
58.61	0.01\\
58.62	0.01\\
58.63	0.01\\
58.64	0.01\\
58.65	0.01\\
58.66	0.01\\
58.67	0.01\\
58.68	0.01\\
58.69	0.01\\
58.7	0.01\\
58.71	0.01\\
58.72	0.01\\
58.73	0.01\\
58.74	0.01\\
58.75	0.01\\
58.76	0.01\\
58.77	0.01\\
58.78	0.01\\
58.79	0.01\\
58.8	0.01\\
58.81	0.01\\
58.82	0.01\\
58.83	0.01\\
58.84	0.01\\
58.85	0.01\\
58.86	0.01\\
58.87	0.01\\
58.88	0.01\\
58.89	0.01\\
58.9	0.01\\
58.91	0.01\\
58.92	0.01\\
58.93	0.01\\
58.94	0.01\\
58.95	0.01\\
58.96	0.01\\
58.97	0.01\\
58.98	0.01\\
58.99	0.01\\
59	0.01\\
59.01	0.01\\
59.02	0.01\\
59.03	0.01\\
59.04	0.01\\
59.05	0.01\\
59.06	0.01\\
59.07	0.01\\
59.08	0.01\\
59.09	0.01\\
59.1	0.01\\
59.11	0.01\\
59.12	0.01\\
59.13	0.01\\
59.14	0.01\\
59.15	0.01\\
59.16	0.01\\
59.17	0.01\\
59.18	0.01\\
59.19	0.01\\
59.2	0.01\\
59.21	0.01\\
59.22	0.01\\
59.23	0.01\\
59.24	0.01\\
59.25	0.01\\
59.26	0.01\\
59.27	0.01\\
59.28	0.01\\
59.29	0.01\\
59.3	0.01\\
59.31	0.01\\
59.32	0.01\\
59.33	0.01\\
59.34	0.01\\
59.35	0.01\\
59.36	0.01\\
59.37	0.01\\
59.38	0.01\\
59.39	0.01\\
59.4	0.01\\
59.41	0.01\\
59.42	0.01\\
59.43	0.01\\
59.44	0.01\\
59.45	0.01\\
59.46	0.01\\
59.47	0.01\\
59.48	0.01\\
59.49	0.01\\
59.5	0.01\\
59.51	0.01\\
59.52	0.01\\
59.53	0.01\\
59.54	0.01\\
59.55	0.01\\
59.56	0.01\\
59.57	0.01\\
59.58	0.01\\
59.59	0.01\\
59.6	0.01\\
59.61	0.01\\
59.62	0.01\\
59.63	0.01\\
59.64	0.01\\
59.65	0.01\\
59.66	0.01\\
59.67	0.01\\
59.68	0.01\\
59.69	0.01\\
59.7	0.01\\
59.71	0.01\\
59.72	0.01\\
59.73	0.01\\
59.74	0.01\\
59.75	0.01\\
59.76	0.01\\
59.77	0.01\\
59.78	0.01\\
59.79	0.01\\
59.8	0.01\\
59.81	0.01\\
59.82	0.01\\
59.83	0.01\\
59.84	0.01\\
59.85	0.01\\
59.86	0.01\\
59.87	0.01\\
59.88	0.01\\
59.89	0.01\\
59.9	0.01\\
59.91	0.01\\
59.92	0.01\\
59.93	0.01\\
59.94	0.01\\
59.95	0.01\\
59.96	0.01\\
59.97	0.01\\
59.98	0.01\\
59.99	0.01\\
60	0.01\\
60.01	0.01\\
60.02	0.01\\
60.03	0.01\\
60.04	0.01\\
60.05	0.01\\
60.06	0.01\\
60.07	0.01\\
60.08	0.01\\
60.09	0.01\\
60.1	0.01\\
60.11	0.01\\
60.12	0.01\\
60.13	0.01\\
60.14	0.01\\
60.15	0.01\\
60.16	0.01\\
60.17	0.01\\
60.18	0.01\\
60.19	0.01\\
60.2	0.01\\
60.21	0.01\\
60.22	0.01\\
60.23	0.01\\
60.24	0.01\\
60.25	0.01\\
60.26	0.01\\
60.27	0.01\\
60.28	0.01\\
60.29	0.01\\
60.3	0.01\\
60.31	0.01\\
60.32	0.01\\
60.33	0.01\\
60.34	0.01\\
60.35	0.01\\
60.36	0.01\\
60.37	0.01\\
60.38	0.01\\
60.39	0.01\\
60.4	0.01\\
60.41	0.01\\
60.42	0.01\\
60.43	0.01\\
60.44	0.01\\
60.45	0.01\\
60.46	0.01\\
60.47	0.01\\
60.48	0.01\\
60.49	0.01\\
60.5	0.01\\
60.51	0.01\\
60.52	0.01\\
60.53	0.01\\
60.54	0.01\\
60.55	0.01\\
60.56	0.01\\
60.57	0.01\\
60.58	0.01\\
60.59	0.01\\
60.6	0.01\\
60.61	0.01\\
60.62	0.01\\
60.63	0.01\\
60.64	0.01\\
60.65	0.01\\
60.66	0.01\\
60.67	0.01\\
60.68	0.01\\
60.69	0.01\\
60.7	0.01\\
60.71	0.01\\
60.72	0.01\\
60.73	0.01\\
60.74	0.01\\
60.75	0.01\\
60.76	0.01\\
60.77	0.01\\
60.78	0.01\\
60.79	0.01\\
60.8	0.01\\
60.81	0.01\\
60.82	0.01\\
60.83	0.01\\
60.84	0.01\\
60.85	0.01\\
60.86	0.01\\
60.87	0.01\\
60.88	0.01\\
60.89	0.01\\
60.9	0.01\\
60.91	0.01\\
60.92	0.01\\
60.93	0.01\\
60.94	0.01\\
60.95	0.01\\
60.96	0.01\\
60.97	0.01\\
60.98	0.01\\
60.99	0.01\\
61	0.01\\
61.01	0.01\\
61.02	0.01\\
61.03	0.01\\
61.04	0.01\\
61.05	0.01\\
61.06	0.01\\
61.07	0.01\\
61.08	0.01\\
61.09	0.01\\
61.1	0.01\\
61.11	0.01\\
61.12	0.01\\
61.13	0.01\\
61.14	0.01\\
61.15	0.01\\
61.16	0.01\\
61.17	0.01\\
61.18	0.01\\
61.19	0.01\\
61.2	0.01\\
61.21	0.01\\
61.22	0.01\\
61.23	0.01\\
61.24	0.01\\
61.25	0.01\\
61.26	0.01\\
61.27	0.01\\
61.28	0.01\\
61.29	0.01\\
61.3	0.01\\
61.31	0.01\\
61.32	0.01\\
61.33	0.01\\
61.34	0.01\\
61.35	0.01\\
61.36	0.01\\
61.37	0.01\\
61.38	0.01\\
61.39	0.01\\
61.4	0.01\\
61.41	0.01\\
61.42	0.01\\
61.43	0.01\\
61.44	0.01\\
61.45	0.01\\
61.46	0.01\\
61.47	0.01\\
61.48	0.01\\
61.49	0.01\\
61.5	0.01\\
61.51	0.01\\
61.52	0.01\\
61.53	0.01\\
61.54	0.01\\
61.55	0.01\\
61.56	0.01\\
61.57	0.01\\
61.58	0.01\\
61.59	0.01\\
61.6	0.01\\
61.61	0.01\\
61.62	0.01\\
61.63	0.01\\
61.64	0.01\\
61.65	0.01\\
61.66	0.01\\
61.67	0.01\\
61.68	0.01\\
61.69	0.01\\
61.7	0.01\\
61.71	0.01\\
61.72	0.01\\
61.73	0.01\\
61.74	0.01\\
61.75	0.01\\
61.76	0.01\\
61.77	0.01\\
61.78	0.01\\
61.79	0.01\\
61.8	0.01\\
61.81	0.01\\
61.82	0.01\\
61.83	0.01\\
61.84	0.01\\
61.85	0.01\\
61.86	0.01\\
61.87	0.01\\
61.88	0.01\\
61.89	0.01\\
61.9	0.01\\
61.91	0.01\\
61.92	0.01\\
61.93	0.01\\
61.94	0.01\\
61.95	0.01\\
61.96	0.01\\
61.97	0.01\\
61.98	0.01\\
61.99	0.01\\
62	0.01\\
62.01	0.01\\
62.02	0.01\\
62.03	0.01\\
62.04	0.01\\
62.05	0.01\\
62.06	0.01\\
62.07	0.01\\
62.08	0.01\\
62.09	0.01\\
62.1	0.01\\
62.11	0.01\\
62.12	0.01\\
62.13	0.01\\
62.14	0.01\\
62.15	0.01\\
62.16	0.01\\
62.17	0.01\\
62.18	0.01\\
62.19	0.01\\
62.2	0.01\\
62.21	0.01\\
62.22	0.01\\
62.23	0.01\\
62.24	0.01\\
62.25	0.01\\
62.26	0.01\\
62.27	0.01\\
62.28	0.01\\
62.29	0.01\\
62.3	0.01\\
62.31	0.01\\
62.32	0.01\\
62.33	0.01\\
62.34	0.01\\
62.35	0.01\\
62.36	0.01\\
62.37	0.01\\
62.38	0.01\\
62.39	0.01\\
62.4	0.01\\
62.41	0.01\\
62.42	0.01\\
62.43	0.01\\
62.44	0.01\\
62.45	0.01\\
62.46	0.01\\
62.47	0.01\\
62.48	0.01\\
62.49	0.01\\
62.5	0.01\\
62.51	0.01\\
62.52	0.01\\
62.53	0.01\\
62.54	0.01\\
62.55	0.01\\
62.56	0.01\\
62.57	0.01\\
62.58	0.01\\
62.59	0.01\\
62.6	0.01\\
62.61	0.01\\
62.62	0.01\\
62.63	0.01\\
62.64	0.01\\
62.65	0.01\\
62.66	0.01\\
62.67	0.01\\
62.68	0.01\\
62.69	0.01\\
62.7	0.01\\
62.71	0.01\\
62.72	0.01\\
62.73	0.01\\
62.74	0.01\\
62.75	0.01\\
62.76	0.01\\
62.77	0.01\\
62.78	0.01\\
62.79	0.01\\
62.8	0.01\\
62.81	0.01\\
62.82	0.01\\
62.83	0.01\\
62.84	0.01\\
62.85	0.01\\
62.86	0.01\\
62.87	0.01\\
62.88	0.01\\
62.89	0.01\\
62.9	0.01\\
62.91	0.01\\
62.92	0.01\\
62.93	0.01\\
62.94	0.01\\
62.95	0.01\\
62.96	0.01\\
62.97	0.01\\
62.98	0.01\\
62.99	0.01\\
63	0.01\\
63.01	0.01\\
63.02	0.01\\
63.03	0.01\\
63.04	0.01\\
63.05	0.01\\
63.06	0.01\\
63.07	0.01\\
63.08	0.01\\
63.09	0.01\\
63.1	0.01\\
63.11	0.01\\
63.12	0.01\\
63.13	0.01\\
63.14	0.01\\
63.15	0.01\\
63.16	0.01\\
63.17	0.01\\
63.18	0.01\\
63.19	0.01\\
63.2	0.01\\
63.21	0.01\\
63.22	0.01\\
63.23	0.01\\
63.24	0.01\\
63.25	0.01\\
63.26	0.01\\
63.27	0.01\\
63.28	0.01\\
63.29	0.01\\
63.3	0.01\\
63.31	0.01\\
63.32	0.01\\
63.33	0.01\\
63.34	0.01\\
63.35	0.01\\
63.36	0.01\\
63.37	0.01\\
63.38	0.01\\
63.39	0.01\\
63.4	0.01\\
63.41	0.01\\
63.42	0.01\\
63.43	0.01\\
63.44	0.01\\
63.45	0.01\\
63.46	0.01\\
63.47	0.01\\
63.48	0.01\\
63.49	0.01\\
63.5	0.01\\
63.51	0.01\\
63.52	0.01\\
63.53	0.01\\
63.54	0.01\\
63.55	0.01\\
63.56	0.01\\
63.57	0.01\\
63.58	0.01\\
63.59	0.01\\
63.6	0.01\\
63.61	0.01\\
63.62	0.01\\
63.63	0.01\\
63.64	0.01\\
63.65	0.01\\
63.66	0.01\\
63.67	0.01\\
63.68	0.01\\
63.69	0.01\\
63.7	0.01\\
63.71	0.01\\
63.72	0.01\\
63.73	0.01\\
63.74	0.01\\
63.75	0.01\\
63.76	0.01\\
63.77	0.01\\
63.78	0.01\\
63.79	0.01\\
63.8	0.01\\
63.81	0.01\\
63.82	0.01\\
63.83	0.01\\
63.84	0.01\\
63.85	0.01\\
63.86	0.01\\
63.87	0.01\\
63.88	0.01\\
63.89	0.01\\
63.9	0.01\\
63.91	0.01\\
63.92	0.01\\
63.93	0.01\\
63.94	0.01\\
63.95	0.01\\
63.96	0.01\\
63.97	0.01\\
63.98	0.01\\
63.99	0.01\\
64	0.01\\
64.01	0.01\\
64.02	0.01\\
64.03	0.01\\
64.04	0.01\\
64.05	0.01\\
64.06	0.01\\
64.07	0.01\\
64.08	0.01\\
64.09	0.01\\
64.1	0.01\\
64.11	0.01\\
64.12	0.01\\
64.13	0.01\\
64.14	0.01\\
64.15	0.01\\
64.16	0.01\\
64.17	0.01\\
64.18	0.01\\
64.19	0.01\\
64.2	0.01\\
64.21	0.01\\
64.22	0.01\\
64.23	0.01\\
64.24	0.01\\
64.25	0.01\\
64.26	0.01\\
64.27	0.01\\
64.28	0.01\\
64.29	0.01\\
64.3	0.01\\
64.31	0.01\\
64.32	0.01\\
64.33	0.01\\
64.34	0.01\\
64.35	0.01\\
64.36	0.01\\
64.37	0.01\\
64.38	0.01\\
64.39	0.01\\
64.4	0.01\\
64.41	0.01\\
64.42	0.01\\
64.43	0.01\\
64.44	0.01\\
64.45	0.01\\
64.46	0.01\\
64.47	0.01\\
64.48	0.01\\
64.49	0.01\\
64.5	0.01\\
64.51	0.01\\
64.52	0.01\\
64.53	0.01\\
64.54	0.01\\
64.55	0.01\\
64.56	0.01\\
64.57	0.01\\
64.58	0.01\\
64.59	0.01\\
64.6	0.01\\
64.61	0.01\\
64.62	0.01\\
64.63	0.01\\
64.64	0.01\\
64.65	0.01\\
64.66	0.01\\
64.67	0.01\\
64.68	0.01\\
64.69	0.01\\
64.7	0.01\\
64.71	0.01\\
64.72	0.01\\
64.73	0.01\\
64.74	0.01\\
64.75	0.01\\
64.76	0.01\\
64.77	0.01\\
64.78	0.01\\
64.79	0.01\\
64.8	0.01\\
64.81	0.01\\
64.82	0.01\\
64.83	0.01\\
64.84	0.01\\
64.85	0.01\\
64.86	0.01\\
64.87	0.01\\
64.88	0.01\\
64.89	0.01\\
64.9	0.01\\
64.91	0.01\\
64.92	0.01\\
64.93	0.01\\
64.94	0.01\\
64.95	0.01\\
64.96	0.01\\
64.97	0.01\\
64.98	0.01\\
64.99	0.01\\
65	0.01\\
65.01	0.01\\
65.02	0.01\\
65.03	0.01\\
65.04	0.01\\
65.05	0.01\\
65.06	0.01\\
65.07	0.01\\
65.08	0.01\\
65.09	0.01\\
65.1	0.01\\
65.11	0.01\\
65.12	0.01\\
65.13	0.01\\
65.14	0.01\\
65.15	0.01\\
65.16	0.01\\
65.17	0.01\\
65.18	0.01\\
65.19	0.01\\
65.2	0.01\\
65.21	0.01\\
65.22	0.01\\
65.23	0.01\\
65.24	0.01\\
65.25	0.01\\
65.26	0.01\\
65.27	0.01\\
65.28	0.01\\
65.29	0.01\\
65.3	0.01\\
65.31	0.01\\
65.32	0.01\\
65.33	0.01\\
65.34	0.01\\
65.35	0.01\\
65.36	0.01\\
65.37	0.01\\
65.38	0.01\\
65.39	0.01\\
65.4	0.01\\
65.41	0.01\\
65.42	0.01\\
65.43	0.01\\
65.44	0.01\\
65.45	0.01\\
65.46	0.01\\
65.47	0.01\\
65.48	0.01\\
65.49	0.01\\
65.5	0.01\\
65.51	0.01\\
65.52	0.01\\
65.53	0.01\\
65.54	0.01\\
65.55	0.01\\
65.56	0.01\\
65.57	0.01\\
65.58	0.01\\
65.59	0.01\\
65.6	0.01\\
65.61	0.01\\
65.62	0.01\\
65.63	0.01\\
65.64	0.01\\
65.65	0.01\\
65.66	0.01\\
65.67	0.01\\
65.68	0.01\\
65.69	0.01\\
65.7	0.01\\
65.71	0.01\\
65.72	0.01\\
65.73	0.01\\
65.74	0.01\\
65.75	0.01\\
65.76	0.01\\
65.77	0.01\\
65.78	0.01\\
65.79	0.01\\
65.8	0.01\\
65.81	0.01\\
65.82	0.01\\
65.83	0.01\\
65.84	0.01\\
65.85	0.01\\
65.86	0.01\\
65.87	0.01\\
65.88	0.01\\
65.89	0.01\\
65.9	0.01\\
65.91	0.01\\
65.92	0.01\\
65.93	0.01\\
65.94	0.01\\
65.95	0.01\\
65.96	0.01\\
65.97	0.01\\
65.98	0.01\\
65.99	0.01\\
66	0.01\\
66.01	0.01\\
66.02	0.01\\
66.03	0.01\\
66.04	0.01\\
66.05	0.01\\
66.06	0.01\\
66.07	0.01\\
66.08	0.01\\
66.09	0.01\\
66.1	0.01\\
66.11	0.01\\
66.12	0.01\\
66.13	0.01\\
66.14	0.01\\
66.15	0.01\\
66.16	0.01\\
66.17	0.01\\
66.18	0.01\\
66.19	0.01\\
66.2	0.01\\
66.21	0.01\\
66.22	0.01\\
66.23	0.01\\
66.24	0.01\\
66.25	0.01\\
66.26	0.01\\
66.27	0.01\\
66.28	0.01\\
66.29	0.01\\
66.3	0.01\\
66.31	0.01\\
66.32	0.01\\
66.33	0.01\\
66.34	0.01\\
66.35	0.01\\
66.36	0.01\\
66.37	0.01\\
66.38	0.01\\
66.39	0.01\\
66.4	0.01\\
66.41	0.01\\
66.42	0.01\\
66.43	0.01\\
66.44	0.01\\
66.45	0.01\\
66.46	0.01\\
66.47	0.01\\
66.48	0.01\\
66.49	0.01\\
66.5	0.01\\
66.51	0.01\\
66.52	0.01\\
66.53	0.01\\
66.54	0.01\\
66.55	0.01\\
66.56	0.01\\
66.57	0.01\\
66.58	0.01\\
66.59	0.01\\
66.6	0.01\\
66.61	0.01\\
66.62	0.01\\
66.63	0.01\\
66.64	0.01\\
66.65	0.01\\
66.66	0.01\\
66.67	0.01\\
66.68	0.01\\
66.69	0.01\\
66.7	0.01\\
66.71	0.01\\
66.72	0.01\\
66.73	0.01\\
66.74	0.01\\
66.75	0.01\\
66.76	0.01\\
66.77	0.01\\
66.78	0.01\\
66.79	0.01\\
66.8	0.01\\
66.81	0.01\\
66.82	0.01\\
66.83	0.01\\
66.84	0.01\\
66.85	0.01\\
66.86	0.01\\
66.87	0.01\\
66.88	0.01\\
66.89	0.01\\
66.9	0.01\\
66.91	0.01\\
66.92	0.01\\
66.93	0.01\\
66.94	0.01\\
66.95	0.01\\
66.96	0.01\\
66.97	0.01\\
66.98	0.01\\
66.99	0.01\\
67	0.01\\
67.01	0.01\\
67.02	0.01\\
67.03	0.01\\
67.04	0.01\\
67.05	0.01\\
67.06	0.01\\
67.07	0.01\\
67.08	0.01\\
67.09	0.01\\
67.1	0.01\\
67.11	0.01\\
67.12	0.01\\
67.13	0.01\\
67.14	0.01\\
67.15	0.01\\
67.16	0.01\\
67.17	0.01\\
67.18	0.01\\
67.19	0.01\\
67.2	0.01\\
67.21	0.01\\
67.22	0.01\\
67.23	0.01\\
67.24	0.01\\
67.25	0.01\\
67.26	0.01\\
67.27	0.01\\
67.28	0.01\\
67.29	0.01\\
67.3	0.01\\
67.31	0.01\\
67.32	0.01\\
67.33	0.01\\
67.34	0.01\\
67.35	0.01\\
67.36	0.01\\
67.37	0.01\\
67.38	0.01\\
67.39	0.01\\
67.4	0.01\\
67.41	0.01\\
67.42	0.01\\
67.43	0.01\\
67.44	0.01\\
67.45	0.01\\
67.46	0.01\\
67.47	0.01\\
67.48	0.01\\
67.49	0.01\\
67.5	0.01\\
67.51	0.01\\
67.52	0.01\\
67.53	0.01\\
67.54	0.01\\
67.55	0.01\\
67.56	0.01\\
67.57	0.01\\
67.58	0.01\\
67.59	0.01\\
67.6	0.01\\
67.61	0.01\\
67.62	0.01\\
67.63	0.01\\
67.64	0.01\\
67.65	0.01\\
67.66	0.01\\
67.67	0.01\\
67.68	0.01\\
67.69	0.01\\
67.7	0.01\\
67.71	0.01\\
67.72	0.01\\
67.73	0.01\\
67.74	0.01\\
67.75	0.01\\
67.76	0.01\\
67.77	0.01\\
67.78	0.01\\
67.79	0.01\\
67.8	0.01\\
67.81	0.01\\
67.82	0.01\\
67.83	0.01\\
67.84	0.01\\
67.85	0.01\\
67.86	0.01\\
67.87	0.01\\
67.88	0.01\\
67.89	0.01\\
67.9	0.01\\
67.91	0.01\\
67.92	0.01\\
67.93	0.01\\
67.94	0.01\\
67.95	0.01\\
67.96	0.01\\
67.97	0.01\\
67.98	0.01\\
67.99	0.01\\
68	0.01\\
68.01	0.01\\
68.02	0.01\\
68.03	0.01\\
68.04	0.01\\
68.05	0.01\\
68.06	0.01\\
68.07	0.01\\
68.08	0.01\\
68.09	0.01\\
68.1	0.01\\
68.11	0.01\\
68.12	0.01\\
68.13	0.01\\
68.14	0.01\\
68.15	0.01\\
68.16	0.01\\
68.17	0.01\\
68.18	0.01\\
68.19	0.01\\
68.2	0.01\\
68.21	0.01\\
68.22	0.01\\
68.23	0.01\\
68.24	0.01\\
68.25	0.01\\
68.26	0.01\\
68.27	0.01\\
68.28	0.01\\
68.29	0.01\\
68.3	0.01\\
68.31	0.01\\
68.32	0.01\\
68.33	0.01\\
68.34	0.01\\
68.35	0.01\\
68.36	0.01\\
68.37	0.01\\
68.38	0.01\\
68.39	0.01\\
68.4	0.01\\
68.41	0.01\\
68.42	0.01\\
68.43	0.01\\
68.44	0.01\\
68.45	0.01\\
68.46	0.01\\
68.47	0.01\\
68.48	0.01\\
68.49	0.01\\
68.5	0.01\\
68.51	0.01\\
68.52	0.01\\
68.53	0.01\\
68.54	0.01\\
68.55	0.01\\
68.56	0.01\\
68.57	0.01\\
68.58	0.01\\
68.59	0.01\\
68.6	0.01\\
68.61	0.01\\
68.62	0.01\\
68.63	0.01\\
68.64	0.01\\
68.65	0.01\\
68.66	0.01\\
68.67	0.01\\
68.68	0.01\\
68.69	0.01\\
68.7	0.01\\
68.71	0.01\\
68.72	0.01\\
68.73	0.01\\
68.74	0.01\\
68.75	0.01\\
68.76	0.01\\
68.77	0.01\\
68.78	0.01\\
68.79	0.01\\
68.8	0.01\\
68.81	0.01\\
68.82	0.01\\
68.83	0.01\\
68.84	0.01\\
68.85	0.01\\
68.86	0.01\\
68.87	0.01\\
68.88	0.01\\
68.89	0.01\\
68.9	0.01\\
68.91	0.01\\
68.92	0.01\\
68.93	0.01\\
68.94	0.01\\
68.95	0.01\\
68.96	0.01\\
68.97	0.01\\
68.98	0.01\\
68.99	0.01\\
69	0.01\\
69.01	0.01\\
69.02	0.01\\
69.03	0.01\\
69.04	0.01\\
69.05	0.01\\
69.06	0.01\\
69.07	0.01\\
69.08	0.01\\
69.09	0.01\\
69.1	0.01\\
69.11	0.01\\
69.12	0.01\\
69.13	0.01\\
69.14	0.01\\
69.15	0.01\\
69.16	0.01\\
69.17	0.01\\
69.18	0.01\\
69.19	0.01\\
69.2	0.01\\
69.21	0.01\\
69.22	0.01\\
69.23	0.01\\
69.24	0.01\\
69.25	0.01\\
69.26	0.01\\
69.27	0.01\\
69.28	0.01\\
69.29	0.01\\
69.3	0.01\\
69.31	0.01\\
69.32	0.01\\
69.33	0.01\\
69.34	0.01\\
69.35	0.01\\
69.36	0.01\\
69.37	0.01\\
69.38	0.01\\
69.39	0.01\\
69.4	0.01\\
69.41	0.01\\
69.42	0.01\\
69.43	0.01\\
69.44	0.01\\
69.45	0.01\\
69.46	0.01\\
69.47	0.01\\
69.48	0.01\\
69.49	0.01\\
69.5	0.01\\
69.51	0.01\\
69.52	0.01\\
69.53	0.01\\
69.54	0.01\\
69.55	0.01\\
69.56	0.01\\
69.57	0.01\\
69.58	0.01\\
69.59	0.01\\
69.6	0.01\\
69.61	0.01\\
69.62	0.01\\
69.63	0.01\\
69.64	0.01\\
69.65	0.01\\
69.66	0.01\\
69.67	0.01\\
69.68	0.01\\
69.69	0.01\\
69.7	0.01\\
69.71	0.01\\
69.72	0.01\\
69.73	0.01\\
69.74	0.01\\
69.75	0.01\\
69.76	0.01\\
69.77	0.01\\
69.78	0.01\\
69.79	0.01\\
69.8	0.01\\
69.81	0.01\\
69.82	0.01\\
69.83	0.01\\
69.84	0.01\\
69.85	0.01\\
69.86	0.01\\
69.87	0.01\\
69.88	0.01\\
69.89	0.01\\
69.9	0.01\\
69.91	0.01\\
69.92	0.01\\
69.93	0.01\\
69.94	0.01\\
69.95	0.01\\
69.96	0.01\\
69.97	0.01\\
69.98	0.01\\
69.99	0.01\\
70	0.01\\
70.01	0.01\\
70.02	0.01\\
70.03	0.01\\
70.04	0.01\\
70.05	0.01\\
70.06	0.01\\
70.07	0.01\\
70.08	0.01\\
70.09	0.01\\
70.1	0.01\\
70.11	0.01\\
70.12	0.01\\
70.13	0.01\\
70.14	0.01\\
70.15	0.01\\
70.16	0.01\\
70.17	0.01\\
70.18	0.01\\
70.19	0.01\\
70.2	0.01\\
70.21	0.01\\
70.22	0.01\\
70.23	0.01\\
70.24	0.01\\
70.25	0.01\\
70.26	0.01\\
70.27	0.01\\
70.28	0.01\\
70.29	0.01\\
70.3	0.01\\
70.31	0.01\\
70.32	0.01\\
70.33	0.01\\
70.34	0.01\\
70.35	0.01\\
70.36	0.01\\
70.37	0.01\\
70.38	0.01\\
70.39	0.01\\
70.4	0.01\\
70.41	0.01\\
70.42	0.01\\
70.43	0.01\\
70.44	0.01\\
70.45	0.01\\
70.46	0.01\\
70.47	0.01\\
70.48	0.01\\
70.49	0.01\\
70.5	0.01\\
70.51	0.01\\
70.52	0.01\\
70.53	0.01\\
70.54	0.01\\
70.55	0.01\\
70.56	0.01\\
70.57	0.01\\
70.58	0.01\\
70.59	0.01\\
70.6	0.01\\
70.61	0.01\\
70.62	0.01\\
70.63	0.01\\
70.64	0.01\\
70.65	0.01\\
70.66	0.01\\
70.67	0.01\\
70.68	0.01\\
70.69	0.01\\
70.7	0.01\\
70.71	0.01\\
70.72	0.01\\
70.73	0.01\\
70.74	0.01\\
70.75	0.01\\
70.76	0.01\\
70.77	0.01\\
70.78	0.01\\
70.79	0.01\\
70.8	0.01\\
70.81	0.01\\
70.82	0.01\\
70.83	0.01\\
70.84	0.01\\
70.85	0.01\\
70.86	0.01\\
70.87	0.01\\
70.88	0.01\\
70.89	0.01\\
70.9	0.01\\
70.91	0.01\\
70.92	0.01\\
70.93	0.01\\
70.94	0.01\\
70.95	0.01\\
70.96	0.01\\
70.97	0.01\\
70.98	0.01\\
70.99	0.01\\
71	0.01\\
71.01	0.01\\
71.02	0.01\\
71.03	0.01\\
71.04	0.01\\
71.05	0.01\\
71.06	0.01\\
71.07	0.01\\
71.08	0.01\\
71.09	0.01\\
71.1	0.01\\
71.11	0.01\\
71.12	0.01\\
71.13	0.01\\
71.14	0.01\\
71.15	0.01\\
71.16	0.01\\
71.17	0.01\\
71.18	0.01\\
71.19	0.01\\
71.2	0.01\\
71.21	0.01\\
71.22	0.01\\
71.23	0.01\\
71.24	0.01\\
71.25	0.01\\
71.26	0.01\\
71.27	0.01\\
71.28	0.01\\
71.29	0.01\\
71.3	0.01\\
71.31	0.01\\
71.32	0.01\\
71.33	0.01\\
71.34	0.01\\
71.35	0.01\\
71.36	0.01\\
71.37	0.01\\
71.38	0.01\\
71.39	0.01\\
71.4	0.01\\
71.41	0.01\\
71.42	0.01\\
71.43	0.01\\
71.44	0.01\\
71.45	0.01\\
71.46	0.01\\
71.47	0.01\\
71.48	0.01\\
71.49	0.01\\
71.5	0.01\\
71.51	0.01\\
71.52	0.01\\
71.53	0.01\\
71.54	0.01\\
71.55	0.01\\
71.56	0.01\\
71.57	0.01\\
71.58	0.01\\
71.59	0.01\\
71.6	0.01\\
71.61	0.01\\
71.62	0.01\\
71.63	0.01\\
71.64	0.01\\
71.65	0.01\\
71.66	0.01\\
71.67	0.01\\
71.68	0.01\\
71.69	0.01\\
71.7	0.01\\
71.71	0.01\\
71.72	0.01\\
71.73	0.01\\
71.74	0.01\\
71.75	0.01\\
71.76	0.01\\
71.77	0.01\\
71.78	0.01\\
71.79	0.01\\
71.8	0.01\\
71.81	0.01\\
71.82	0.01\\
71.83	0.01\\
71.84	0.01\\
71.85	0.01\\
71.86	0.01\\
71.87	0.01\\
71.88	0.01\\
71.89	0.01\\
71.9	0.01\\
71.91	0.01\\
71.92	0.01\\
71.93	0.01\\
71.94	0.01\\
71.95	0.01\\
71.96	0.01\\
71.97	0.01\\
71.98	0.01\\
71.99	0.01\\
72	0.01\\
72.01	0.01\\
72.02	0.01\\
72.03	0.01\\
72.04	0.01\\
72.05	0.01\\
72.06	0.01\\
72.07	0.01\\
72.08	0.01\\
72.09	0.01\\
72.1	0.01\\
72.11	0.01\\
72.12	0.01\\
72.13	0.01\\
72.14	0.01\\
72.15	0.01\\
72.16	0.01\\
72.17	0.01\\
72.18	0.01\\
72.19	0.01\\
72.2	0.01\\
72.21	0.01\\
72.22	0.01\\
72.23	0.01\\
72.24	0.01\\
72.25	0.01\\
72.26	0.01\\
72.27	0.01\\
72.28	0.01\\
72.29	0.01\\
72.3	0.01\\
72.31	0.01\\
72.32	0.01\\
72.33	0.01\\
72.34	0.01\\
72.35	0.01\\
72.36	0.01\\
72.37	0.01\\
72.38	0.01\\
72.39	0.01\\
72.4	0.01\\
72.41	0.01\\
72.42	0.01\\
72.43	0.01\\
72.44	0.01\\
72.45	0.01\\
72.46	0.01\\
72.47	0.01\\
72.48	0.01\\
72.49	0.01\\
72.5	0.01\\
72.51	0.01\\
72.52	0.01\\
72.53	0.01\\
72.54	0.01\\
72.55	0.01\\
72.56	0.01\\
72.57	0.01\\
72.58	0.01\\
72.59	0.01\\
72.6	0.01\\
72.61	0.01\\
72.62	0.01\\
72.63	0.01\\
72.64	0.01\\
72.65	0.01\\
72.66	0.01\\
72.67	0.01\\
72.68	0.01\\
72.69	0.01\\
72.7	0.01\\
72.71	0.01\\
72.72	0.01\\
72.73	0.01\\
72.74	0.01\\
72.75	0.01\\
72.76	0.01\\
72.77	0.01\\
72.78	0.01\\
72.79	0.01\\
72.8	0.01\\
72.81	0.01\\
72.82	0.01\\
72.83	0.01\\
72.84	0.01\\
72.85	0.01\\
72.86	0.01\\
72.87	0.01\\
72.88	0.01\\
72.89	0.01\\
72.9	0.01\\
72.91	0.01\\
72.92	0.01\\
72.93	0.01\\
72.94	0.01\\
72.95	0.01\\
72.96	0.01\\
72.97	0.01\\
72.98	0.01\\
72.99	0.01\\
73	0.01\\
73.01	0.01\\
73.02	0.01\\
73.03	0.01\\
73.04	0.01\\
73.05	0.01\\
73.06	0.01\\
73.07	0.01\\
73.08	0.01\\
73.09	0.01\\
73.1	0.01\\
73.11	0.01\\
73.12	0.01\\
73.13	0.01\\
73.14	0.01\\
73.15	0.01\\
73.16	0.01\\
73.17	0.01\\
73.18	0.01\\
73.19	0.01\\
73.2	0.01\\
73.21	0.01\\
73.22	0.01\\
73.23	0.01\\
73.24	0.01\\
73.25	0.01\\
73.26	0.01\\
73.27	0.01\\
73.28	0.01\\
73.29	0.01\\
73.3	0.01\\
73.31	0.01\\
73.32	0.01\\
73.33	0.01\\
73.34	0.01\\
73.35	0.01\\
73.36	0.01\\
73.37	0.01\\
73.38	0.01\\
73.39	0.01\\
73.4	0.01\\
73.41	0.01\\
73.42	0.01\\
73.43	0.01\\
73.44	0.01\\
73.45	0.01\\
73.46	0.01\\
73.47	0.01\\
73.48	0.01\\
73.49	0.01\\
73.5	0.01\\
73.51	0.01\\
73.52	0.01\\
73.53	0.01\\
73.54	0.01\\
73.55	0.01\\
73.56	0.01\\
73.57	0.01\\
73.58	0.01\\
73.59	0.01\\
73.6	0.01\\
73.61	0.01\\
73.62	0.01\\
73.63	0.01\\
73.64	0.01\\
73.65	0.01\\
73.66	0.01\\
73.67	0.01\\
73.68	0.01\\
73.69	0.01\\
73.7	0.01\\
73.71	0.01\\
73.72	0.01\\
73.73	0.01\\
73.74	0.01\\
73.75	0.01\\
73.76	0.01\\
73.77	0.01\\
73.78	0.01\\
73.79	0.01\\
73.8	0.01\\
73.81	0.01\\
73.82	0.01\\
73.83	0.01\\
73.84	0.01\\
73.85	0.01\\
73.86	0.01\\
73.87	0.01\\
73.88	0.01\\
73.89	0.01\\
73.9	0.01\\
73.91	0.01\\
73.92	0.01\\
73.93	0.01\\
73.94	0.01\\
73.95	0.01\\
73.96	0.01\\
73.97	0.01\\
73.98	0.01\\
73.99	0.01\\
74	0.01\\
74.01	0.01\\
74.02	0.01\\
74.03	0.01\\
74.04	0.01\\
74.05	0.01\\
74.06	0.01\\
74.07	0.01\\
74.08	0.01\\
74.09	0.01\\
74.1	0.01\\
74.11	0.01\\
74.12	0.01\\
74.13	0.01\\
74.14	0.01\\
74.15	0.01\\
74.16	0.01\\
74.17	0.01\\
74.18	0.01\\
74.19	0.01\\
74.2	0.01\\
74.21	0.01\\
74.22	0.01\\
74.23	0.01\\
74.24	0.01\\
74.25	0.01\\
74.26	0.01\\
74.27	0.01\\
74.28	0.01\\
74.29	0.01\\
74.3	0.01\\
74.31	0.01\\
74.32	0.01\\
74.33	0.01\\
74.34	0.01\\
74.35	0.01\\
74.36	0.01\\
74.37	0.01\\
74.38	0.01\\
74.39	0.01\\
74.4	0.01\\
74.41	0.01\\
74.42	0.01\\
74.43	0.01\\
74.44	0.01\\
74.45	0.01\\
74.46	0.01\\
74.47	0.01\\
74.48	0.01\\
74.49	0.01\\
74.5	0.01\\
74.51	0.01\\
74.52	0.01\\
74.53	0.01\\
74.54	0.01\\
74.55	0.01\\
74.56	0.01\\
74.57	0.01\\
74.58	0.01\\
74.59	0.01\\
74.6	0.01\\
74.61	0.01\\
74.62	0.01\\
74.63	0.01\\
74.64	0.01\\
74.65	0.01\\
74.66	0.01\\
74.67	0.01\\
74.68	0.01\\
74.69	0.01\\
74.7	0.01\\
74.71	0.01\\
74.72	0.01\\
74.73	0.01\\
74.74	0.01\\
74.75	0.01\\
74.76	0.01\\
74.77	0.01\\
74.78	0.01\\
74.79	0.01\\
74.8	0.01\\
74.81	0.01\\
74.82	0.01\\
74.83	0.01\\
74.84	0.01\\
74.85	0.01\\
74.86	0.01\\
74.87	0.01\\
74.88	0.01\\
74.89	0.01\\
74.9	0.01\\
74.91	0.01\\
74.92	0.01\\
74.93	0.01\\
74.94	0.01\\
74.95	0.01\\
74.96	0.01\\
74.97	0.01\\
74.98	0.01\\
74.99	0.01\\
75	0.01\\
75.01	0.01\\
75.02	0.01\\
75.03	0.01\\
75.04	0.01\\
75.05	0.01\\
75.06	0.01\\
75.07	0.01\\
75.08	0.01\\
75.09	0.01\\
75.1	0.01\\
75.11	0.01\\
75.12	0.01\\
75.13	0.01\\
75.14	0.01\\
75.15	0.01\\
75.16	0.01\\
75.17	0.01\\
75.18	0.01\\
75.19	0.01\\
75.2	0.01\\
75.21	0.01\\
75.22	0.01\\
75.23	0.01\\
75.24	0.01\\
75.25	0.01\\
75.26	0.01\\
75.27	0.01\\
75.28	0.01\\
75.29	0.01\\
75.3	0.01\\
75.31	0.01\\
75.32	0.01\\
75.33	0.01\\
75.34	0.01\\
75.35	0.01\\
75.36	0.01\\
75.37	0.01\\
75.38	0.01\\
75.39	0.01\\
75.4	0.01\\
75.41	0.01\\
75.42	0.01\\
75.43	0.01\\
75.44	0.01\\
75.45	0.01\\
75.46	0.01\\
75.47	0.01\\
75.48	0.01\\
75.49	0.01\\
75.5	0.01\\
75.51	0.01\\
75.52	0.01\\
75.53	0.01\\
75.54	0.01\\
75.55	0.01\\
75.56	0.01\\
75.57	0.01\\
75.58	0.01\\
75.59	0.01\\
75.6	0.01\\
75.61	0.01\\
75.62	0.01\\
75.63	0.01\\
75.64	0.01\\
75.65	0.01\\
75.66	0.01\\
75.67	0.01\\
75.68	0.01\\
75.69	0.01\\
75.7	0.01\\
75.71	0.01\\
75.72	0.01\\
75.73	0.01\\
75.74	0.01\\
75.75	0.01\\
75.76	0.01\\
75.77	0.01\\
75.78	0.01\\
75.79	0.01\\
75.8	0.01\\
75.81	0.01\\
75.82	0.01\\
75.83	0.01\\
75.84	0.01\\
75.85	0.01\\
75.86	0.01\\
75.87	0.01\\
75.88	0.01\\
75.89	0.01\\
75.9	0.01\\
75.91	0.01\\
75.92	0.01\\
75.93	0.01\\
75.94	0.01\\
75.95	0.01\\
75.96	0.01\\
75.97	0.01\\
75.98	0.01\\
75.99	0.01\\
76	0.01\\
76.01	0.01\\
76.02	0.01\\
76.03	0.01\\
76.04	0.01\\
76.05	0.01\\
76.06	0.01\\
76.07	0.01\\
76.08	0.01\\
76.09	0.01\\
76.1	0.01\\
76.11	0.01\\
76.12	0.01\\
76.13	0.01\\
76.14	0.01\\
76.15	0.01\\
76.16	0.01\\
76.17	0.01\\
76.18	0.01\\
76.19	0.01\\
76.2	0.01\\
76.21	0.01\\
76.22	0.01\\
76.23	0.01\\
76.24	0.01\\
76.25	0.01\\
76.26	0.01\\
76.27	0.01\\
76.28	0.01\\
76.29	0.01\\
76.3	0.01\\
76.31	0.01\\
76.32	0.01\\
76.33	0.01\\
76.34	0.01\\
76.35	0.01\\
76.36	0.01\\
76.37	0.01\\
76.38	0.01\\
76.39	0.01\\
76.4	0.01\\
76.41	0.01\\
76.42	0.01\\
76.43	0.01\\
76.44	0.01\\
76.45	0.01\\
76.46	0.01\\
76.47	0.01\\
76.48	0.01\\
76.49	0.01\\
76.5	0.01\\
76.51	0.01\\
76.52	0.01\\
76.53	0.01\\
76.54	0.01\\
76.55	0.01\\
76.56	0.01\\
76.57	0.01\\
76.58	0.01\\
76.59	0.01\\
76.6	0.01\\
76.61	0.01\\
76.62	0.01\\
76.63	0.01\\
76.64	0.01\\
76.65	0.01\\
76.66	0.01\\
76.67	0.01\\
76.68	0.01\\
76.69	0.01\\
76.7	0.01\\
76.71	0.01\\
76.72	0.01\\
76.73	0.01\\
76.74	0.01\\
76.75	0.01\\
76.76	0.01\\
76.77	0.01\\
76.78	0.01\\
76.79	0.01\\
76.8	0.01\\
76.81	0.01\\
76.82	0.01\\
76.83	0.01\\
76.84	0.01\\
76.85	0.01\\
76.86	0.01\\
76.87	0.01\\
76.88	0.01\\
76.89	0.01\\
76.9	0.01\\
76.91	0.01\\
76.92	0.01\\
76.93	0.01\\
76.94	0.01\\
76.95	0.01\\
76.96	0.01\\
76.97	0.01\\
76.98	0.01\\
76.99	0.01\\
77	0.01\\
77.01	0.01\\
77.02	0.01\\
77.03	0.01\\
77.04	0.01\\
77.05	0.01\\
77.06	0.01\\
77.07	0.01\\
77.08	0.01\\
77.09	0.01\\
77.1	0.01\\
77.11	0.01\\
77.12	0.01\\
77.13	0.01\\
77.14	0.01\\
77.15	0.01\\
77.16	0.01\\
77.17	0.01\\
77.18	0.01\\
77.19	0.01\\
77.2	0.01\\
77.21	0.01\\
77.22	0.01\\
77.23	0.01\\
77.24	0.01\\
77.25	0.01\\
77.26	0.01\\
77.27	0.01\\
77.28	0.01\\
77.29	0.01\\
77.3	0.01\\
77.31	0.01\\
77.32	0.01\\
77.33	0.01\\
77.34	0.01\\
77.35	0.01\\
77.36	0.01\\
77.37	0.01\\
77.38	0.01\\
77.39	0.01\\
77.4	0.01\\
77.41	0.01\\
77.42	0.01\\
77.43	0.01\\
77.44	0.01\\
77.45	0.01\\
77.46	0.01\\
77.47	0.01\\
77.48	0.01\\
77.49	0.01\\
77.5	0.01\\
77.51	0.01\\
77.52	0.01\\
77.53	0.01\\
77.54	0.01\\
77.55	0.01\\
77.56	0.01\\
77.57	0.01\\
77.58	0.01\\
77.59	0.01\\
77.6	0.01\\
77.61	0.01\\
77.62	0.01\\
77.63	0.01\\
77.64	0.01\\
77.65	0.01\\
77.66	0.01\\
77.67	0.01\\
77.68	0.01\\
77.69	0.01\\
77.7	0.01\\
77.71	0.01\\
77.72	0.01\\
77.73	0.01\\
77.74	0.01\\
77.75	0.01\\
77.76	0.01\\
77.77	0.01\\
77.78	0.01\\
77.79	0.01\\
77.8	0.01\\
77.81	0.01\\
77.82	0.01\\
77.83	0.01\\
77.84	0.01\\
77.85	0.01\\
77.86	0.01\\
77.87	0.01\\
77.88	0.01\\
77.89	0.01\\
77.9	0.01\\
77.91	0.01\\
77.92	0.01\\
77.93	0.01\\
77.94	0.01\\
77.95	0.01\\
77.96	0.01\\
77.97	0.01\\
77.98	0.01\\
77.99	0.01\\
78	0.01\\
78.01	0.01\\
78.02	0.01\\
78.03	0.01\\
78.04	0.01\\
78.05	0.01\\
78.06	0.01\\
78.07	0.01\\
78.08	0.01\\
78.09	0.01\\
78.1	0.01\\
78.11	0.01\\
78.12	0.01\\
78.13	0.01\\
78.14	0.01\\
78.15	0.01\\
78.16	0.01\\
78.17	0.01\\
78.18	0.01\\
78.19	0.01\\
78.2	0.01\\
78.21	0.01\\
78.22	0.01\\
78.23	0.01\\
78.24	0.01\\
78.25	0.01\\
78.26	0.01\\
78.27	0.01\\
78.28	0.01\\
78.29	0.01\\
78.3	0.01\\
78.31	0.01\\
78.32	0.01\\
78.33	0.01\\
78.34	0.01\\
78.35	0.01\\
78.36	0.01\\
78.37	0.01\\
78.38	0.01\\
78.39	0.01\\
78.4	0.01\\
78.41	0.01\\
78.42	0.01\\
78.43	0.01\\
78.44	0.01\\
78.45	0.01\\
78.46	0.01\\
78.47	0.01\\
78.48	0.01\\
78.49	0.01\\
78.5	0.01\\
78.51	0.01\\
78.52	0.01\\
78.53	0.01\\
78.54	0.01\\
78.55	0.01\\
78.56	0.01\\
78.57	0.01\\
78.58	0.01\\
78.59	0.01\\
78.6	0.01\\
78.61	0.01\\
78.62	0.01\\
78.63	0.01\\
78.64	0.01\\
78.65	0.01\\
78.66	0.01\\
78.67	0.01\\
78.68	0.01\\
78.69	0.01\\
78.7	0.01\\
78.71	0.01\\
78.72	0.01\\
78.73	0.01\\
78.74	0.01\\
78.75	0.01\\
78.76	0.01\\
78.77	0.01\\
78.78	0.01\\
78.79	0.01\\
78.8	0.01\\
78.81	0.01\\
78.82	0.01\\
78.83	0.01\\
78.84	0.01\\
78.85	0.01\\
78.86	0.01\\
78.87	0.01\\
78.88	0.01\\
78.89	0.01\\
78.9	0.01\\
78.91	0.01\\
78.92	0.01\\
78.93	0.01\\
78.94	0.01\\
78.95	0.01\\
78.96	0.01\\
78.97	0.01\\
78.98	0.01\\
78.99	0.01\\
79	0.01\\
79.01	0.01\\
79.02	0.01\\
79.03	0.01\\
79.04	0.01\\
79.05	0.01\\
79.06	0.01\\
79.07	0.01\\
79.08	0.01\\
79.09	0.01\\
79.1	0.01\\
79.11	0.01\\
79.12	0.01\\
79.13	0.01\\
79.14	0.01\\
79.15	0.01\\
79.16	0.01\\
79.17	0.01\\
79.18	0.01\\
79.19	0.01\\
79.2	0.01\\
79.21	0.01\\
79.22	0.01\\
79.23	0.01\\
79.24	0.01\\
79.25	0.01\\
79.26	0.01\\
79.27	0.01\\
79.28	0.01\\
79.29	0.01\\
79.3	0.01\\
79.31	0.01\\
79.32	0.01\\
79.33	0.01\\
79.34	0.01\\
79.35	0.01\\
79.36	0.01\\
79.37	0.01\\
79.38	0.01\\
79.39	0.01\\
79.4	0.01\\
79.41	0.01\\
79.42	0.01\\
79.43	0.01\\
79.44	0.01\\
79.45	0.01\\
79.46	0.01\\
79.47	0.01\\
79.48	0.01\\
79.49	0.01\\
79.5	0.01\\
79.51	0.01\\
79.52	0.01\\
79.53	0.01\\
79.54	0.01\\
79.55	0.01\\
79.56	0.01\\
79.57	0.01\\
79.58	0.01\\
79.59	0.01\\
79.6	0.01\\
79.61	0.01\\
79.62	0.01\\
79.63	0.01\\
79.64	0.01\\
79.65	0.01\\
79.66	0.01\\
79.67	0.01\\
79.68	0.01\\
79.69	0.01\\
79.7	0.01\\
79.71	0.01\\
79.72	0.01\\
79.73	0.01\\
79.74	0.01\\
79.75	0.01\\
79.76	0.01\\
79.77	0.01\\
79.78	0.01\\
79.79	0.01\\
79.8	0.01\\
79.81	0.01\\
79.82	0.01\\
79.83	0.01\\
79.84	0.01\\
79.85	0.01\\
79.86	0.01\\
79.87	0.01\\
79.88	0.01\\
79.89	0.01\\
79.9	0.01\\
79.91	0.01\\
79.92	0.01\\
79.93	0.01\\
79.94	0.01\\
79.95	0.01\\
79.96	0.01\\
79.97	0.01\\
79.98	0.01\\
79.99	0.01\\
80	0.01\\
80.01	0.01\\
};
\addplot [color=red,dashed]
  table[row sep=crcr]{%
80.01	0.01\\
80.02	0.01\\
80.03	0.01\\
80.04	0.01\\
80.05	0.01\\
80.06	0.01\\
80.07	0.01\\
80.08	0.01\\
80.09	0.01\\
80.1	0.01\\
80.11	0.01\\
80.12	0.01\\
80.13	0.01\\
80.14	0.01\\
80.15	0.01\\
80.16	0.01\\
80.17	0.01\\
80.18	0.01\\
80.19	0.01\\
80.2	0.01\\
80.21	0.01\\
80.22	0.01\\
80.23	0.01\\
80.24	0.01\\
80.25	0.01\\
80.26	0.01\\
80.27	0.01\\
80.28	0.01\\
80.29	0.01\\
80.3	0.01\\
80.31	0.01\\
80.32	0.01\\
80.33	0.01\\
80.34	0.01\\
80.35	0.01\\
80.36	0.01\\
80.37	0.01\\
80.38	0.01\\
80.39	0.01\\
80.4	0.01\\
80.41	0.01\\
80.42	0.01\\
80.43	0.01\\
80.44	0.01\\
80.45	0.01\\
80.46	0.01\\
80.47	0.01\\
80.48	0.01\\
80.49	0.01\\
80.5	0.01\\
80.51	0.01\\
80.52	0.01\\
80.53	0.01\\
80.54	0.01\\
80.55	0.01\\
80.56	0.01\\
80.57	0.01\\
80.58	0.01\\
80.59	0.01\\
80.6	0.01\\
80.61	0.01\\
80.62	0.01\\
80.63	0.01\\
80.64	0.01\\
80.65	0.01\\
80.66	0.01\\
80.67	0.01\\
80.68	0.01\\
80.69	0.01\\
80.7	0.01\\
80.71	0.01\\
80.72	0.01\\
80.73	0.01\\
80.74	0.01\\
80.75	0.01\\
80.76	0.01\\
80.77	0.01\\
80.78	0.01\\
80.79	0.01\\
80.8	0.01\\
80.81	0.01\\
80.82	0.01\\
80.83	0.01\\
80.84	0.01\\
80.85	0.01\\
80.86	0.01\\
80.87	0.01\\
80.88	0.01\\
80.89	0.01\\
80.9	0.01\\
80.91	0.01\\
80.92	0.01\\
80.93	0.01\\
80.94	0.01\\
80.95	0.01\\
80.96	0.01\\
80.97	0.01\\
80.98	0.01\\
80.99	0.01\\
81	0.01\\
81.01	0.01\\
81.02	0.01\\
81.03	0.01\\
81.04	0.01\\
81.05	0.01\\
81.06	0.01\\
81.07	0.01\\
81.08	0.01\\
81.09	0.01\\
81.1	0.01\\
81.11	0.01\\
81.12	0.01\\
81.13	0.01\\
81.14	0.01\\
81.15	0.01\\
81.16	0.01\\
81.17	0.01\\
81.18	0.01\\
81.19	0.01\\
81.2	0.01\\
81.21	0.01\\
81.22	0.01\\
81.23	0.01\\
81.24	0.01\\
81.25	0.01\\
81.26	0.01\\
81.27	0.01\\
81.28	0.01\\
81.29	0.01\\
81.3	0.01\\
81.31	0.01\\
81.32	0.01\\
81.33	0.01\\
81.34	0.01\\
81.35	0.01\\
81.36	0.01\\
81.37	0.01\\
81.38	0.01\\
81.39	0.01\\
81.4	0.01\\
81.41	0.01\\
81.42	0.01\\
81.43	0.01\\
81.44	0.01\\
81.45	0.01\\
81.46	0.01\\
81.47	0.01\\
81.48	0.01\\
81.49	0.01\\
81.5	0.01\\
81.51	0.01\\
81.52	0.01\\
81.53	0.01\\
81.54	0.01\\
81.55	0.01\\
81.56	0.01\\
81.57	0.01\\
81.58	0.01\\
81.59	0.01\\
81.6	0.01\\
81.61	0.01\\
81.62	0.01\\
81.63	0.01\\
81.64	0.01\\
81.65	0.01\\
81.66	0.01\\
81.67	0.01\\
81.68	0.01\\
81.69	0.01\\
81.7	0.01\\
81.71	0.01\\
81.72	0.01\\
81.73	0.01\\
81.74	0.01\\
81.75	0.01\\
81.76	0.01\\
81.77	0.01\\
81.78	0.01\\
81.79	0.01\\
81.8	0.01\\
81.81	0.01\\
81.82	0.01\\
81.83	0.01\\
81.84	0.01\\
81.85	0.01\\
81.86	0.01\\
81.87	0.01\\
81.88	0.01\\
81.89	0.01\\
81.9	0.01\\
81.91	0.01\\
81.92	0.01\\
81.93	0.01\\
81.94	0.01\\
81.95	0.01\\
81.96	0.01\\
81.97	0.01\\
81.98	0.01\\
81.99	0.01\\
82	0.01\\
82.01	0.01\\
82.02	0.01\\
82.03	0.01\\
82.04	0.01\\
82.05	0.01\\
82.06	0.01\\
82.07	0.01\\
82.08	0.01\\
82.09	0.01\\
82.1	0.01\\
82.11	0.01\\
82.12	0.01\\
82.13	0.01\\
82.14	0.01\\
82.15	0.01\\
82.16	0.01\\
82.17	0.01\\
82.18	0.01\\
82.19	0.01\\
82.2	0.01\\
82.21	0.01\\
82.22	0.01\\
82.23	0.01\\
82.24	0.01\\
82.25	0.01\\
82.26	0.01\\
82.27	0.01\\
82.28	0.01\\
82.29	0.01\\
82.3	0.01\\
82.31	0.01\\
82.32	0.01\\
82.33	0.01\\
82.34	0.01\\
82.35	0.01\\
82.36	0.01\\
82.37	0.01\\
82.38	0.01\\
82.39	0.01\\
82.4	0.01\\
82.41	0.01\\
82.42	0.01\\
82.43	0.01\\
82.44	0.01\\
82.45	0.01\\
82.46	0.01\\
82.47	0.01\\
82.48	0.01\\
82.49	0.01\\
82.5	0.01\\
82.51	0.01\\
82.52	0.01\\
82.53	0.01\\
82.54	0.01\\
82.55	0.01\\
82.56	0.01\\
82.57	0.01\\
82.58	0.01\\
82.59	0.01\\
82.6	0.01\\
82.61	0.01\\
82.62	0.01\\
82.63	0.01\\
82.64	0.01\\
82.65	0.01\\
82.66	0.01\\
82.67	0.01\\
82.68	0.01\\
82.69	0.01\\
82.7	0.01\\
82.71	0.01\\
82.72	0.01\\
82.73	0.01\\
82.74	0.01\\
82.75	0.01\\
82.76	0.01\\
82.77	0.01\\
82.78	0.01\\
82.79	0.01\\
82.8	0.01\\
82.81	0.01\\
82.82	0.01\\
82.83	0.01\\
82.84	0.01\\
82.85	0.01\\
82.86	0.01\\
82.87	0.01\\
82.88	0.01\\
82.89	0.01\\
82.9	0.01\\
82.91	0.01\\
82.92	0.01\\
82.93	0.01\\
82.94	0.01\\
82.95	0.01\\
82.96	0.01\\
82.97	0.01\\
82.98	0.01\\
82.99	0.01\\
83	0.01\\
83.01	0.01\\
83.02	0.01\\
83.03	0.01\\
83.04	0.01\\
83.05	0.01\\
83.06	0.01\\
83.07	0.01\\
83.08	0.01\\
83.09	0.01\\
83.1	0.01\\
83.11	0.01\\
83.12	0.01\\
83.13	0.01\\
83.14	0.01\\
83.15	0.01\\
83.16	0.01\\
83.17	0.01\\
83.18	0.01\\
83.19	0.01\\
83.2	0.01\\
83.21	0.01\\
83.22	0.01\\
83.23	0.01\\
83.24	0.01\\
83.25	0.01\\
83.26	0.01\\
83.27	0.01\\
83.28	0.01\\
83.29	0.01\\
83.3	0.01\\
83.31	0.01\\
83.32	0.01\\
83.33	0.01\\
83.34	0.01\\
83.35	0.01\\
83.36	0.01\\
83.37	0.01\\
83.38	0.01\\
83.39	0.01\\
83.4	0.01\\
83.41	0.01\\
83.42	0.01\\
83.43	0.01\\
83.44	0.01\\
83.45	0.01\\
83.46	0.01\\
83.47	0.01\\
83.48	0.01\\
83.49	0.01\\
83.5	0.01\\
83.51	0.01\\
83.52	0.01\\
83.53	0.01\\
83.54	0.01\\
83.55	0.01\\
83.56	0.01\\
83.57	0.01\\
83.58	0.01\\
83.59	0.01\\
83.6	0.01\\
83.61	0.01\\
83.62	0.01\\
83.63	0.01\\
83.64	0.01\\
83.65	0.01\\
83.66	0.01\\
83.67	0.01\\
83.68	0.01\\
83.69	0.01\\
83.7	0.01\\
83.71	0.01\\
83.72	0.01\\
83.73	0.01\\
83.74	0.01\\
83.75	0.01\\
83.76	0.01\\
83.77	0.01\\
83.78	0.01\\
83.79	0.01\\
83.8	0.01\\
83.81	0.01\\
83.82	0.01\\
83.83	0.01\\
83.84	0.01\\
83.85	0.01\\
83.86	0.01\\
83.87	0.01\\
83.88	0.01\\
83.89	0.01\\
83.9	0.01\\
83.91	0.01\\
83.92	0.01\\
83.93	0.01\\
83.94	0.01\\
83.95	0.01\\
83.96	0.01\\
83.97	0.01\\
83.98	0.01\\
83.99	0.01\\
84	0.01\\
84.01	0.01\\
84.02	0.01\\
84.03	0.01\\
84.04	0.01\\
84.05	0.01\\
84.06	0.01\\
84.07	0.01\\
84.08	0.01\\
84.09	0.01\\
84.1	0.01\\
84.11	0.01\\
84.12	0.01\\
84.13	0.01\\
84.14	0.01\\
84.15	0.01\\
84.16	0.01\\
84.17	0.01\\
84.18	0.01\\
84.19	0.01\\
84.2	0.01\\
84.21	0.01\\
84.22	0.01\\
84.23	0.01\\
84.24	0.01\\
84.25	0.01\\
84.26	0.01\\
84.27	0.01\\
84.28	0.01\\
84.29	0.01\\
84.3	0.01\\
84.31	0.01\\
84.32	0.01\\
84.33	0.01\\
84.34	0.01\\
84.35	0.01\\
84.36	0.01\\
84.37	0.01\\
84.38	0.01\\
84.39	0.01\\
84.4	0.01\\
84.41	0.01\\
84.42	0.01\\
84.43	0.01\\
84.44	0.01\\
84.45	0.01\\
84.46	0.01\\
84.47	0.01\\
84.48	0.01\\
84.49	0.01\\
84.5	0.01\\
84.51	0.01\\
84.52	0.01\\
84.53	0.01\\
84.54	0.01\\
84.55	0.01\\
84.56	0.01\\
84.57	0.01\\
84.58	0.01\\
84.59	0.01\\
84.6	0.01\\
84.61	0.01\\
84.62	0.01\\
84.63	0.01\\
84.64	0.01\\
84.65	0.01\\
84.66	0.01\\
84.67	0.01\\
84.68	0.01\\
84.69	0.01\\
84.7	0.01\\
84.71	0.01\\
84.72	0.01\\
84.73	0.01\\
84.74	0.01\\
84.75	0.01\\
84.76	0.01\\
84.77	0.01\\
84.78	0.01\\
84.79	0.01\\
84.8	0.01\\
84.81	0.01\\
84.82	0.01\\
84.83	0.01\\
84.84	0.01\\
84.85	0.01\\
84.86	0.01\\
84.87	0.01\\
84.88	0.01\\
84.89	0.01\\
84.9	0.01\\
84.91	0.01\\
84.92	0.01\\
84.93	0.01\\
84.94	0.01\\
84.95	0.01\\
84.96	0.01\\
84.97	0.01\\
84.98	0.01\\
84.99	0.01\\
85	0.01\\
85.01	0.01\\
85.02	0.01\\
85.03	0.01\\
85.04	0.01\\
85.05	0.01\\
85.06	0.01\\
85.07	0.01\\
85.08	0.01\\
85.09	0.01\\
85.1	0.01\\
85.11	0.01\\
85.12	0.01\\
85.13	0.01\\
85.14	0.01\\
85.15	0.01\\
85.16	0.01\\
85.17	0.01\\
85.18	0.01\\
85.19	0.01\\
85.2	0.01\\
85.21	0.01\\
85.22	0.01\\
85.23	0.01\\
85.24	0.01\\
85.25	0.01\\
85.26	0.01\\
85.27	0.01\\
85.28	0.01\\
85.29	0.01\\
85.3	0.01\\
85.31	0.01\\
85.32	0.01\\
85.33	0.01\\
85.34	0.01\\
85.35	0.01\\
85.36	0.01\\
85.37	0.01\\
85.38	0.01\\
85.39	0.01\\
85.4	0.01\\
85.41	0.01\\
85.42	0.01\\
85.43	0.01\\
85.44	0.01\\
85.45	0.01\\
85.46	0.01\\
85.47	0.01\\
85.48	0.01\\
85.49	0.01\\
85.5	0.01\\
85.51	0.01\\
85.52	0.01\\
85.53	0.01\\
85.54	0.01\\
85.55	0.01\\
85.56	0.01\\
85.57	0.01\\
85.58	0.01\\
85.59	0.01\\
85.6	0.01\\
85.61	0.01\\
85.62	0.01\\
85.63	0.01\\
85.64	0.01\\
85.65	0.01\\
85.66	0.01\\
85.67	0.01\\
85.68	0.01\\
85.69	0.01\\
85.7	0.01\\
85.71	0.01\\
85.72	0.01\\
85.73	0.01\\
85.74	0.01\\
85.75	0.01\\
85.76	0.01\\
85.77	0.01\\
85.78	0.01\\
85.79	0.01\\
85.8	0.01\\
85.81	0.01\\
85.82	0.01\\
85.83	0.01\\
85.84	0.01\\
85.85	0.01\\
85.86	0.01\\
85.87	0.01\\
85.88	0.01\\
85.89	0.01\\
85.9	0.01\\
85.91	0.01\\
85.92	0.01\\
85.93	0.01\\
85.94	0.01\\
85.95	0.01\\
85.96	0.01\\
85.97	0.01\\
85.98	0.01\\
85.99	0.01\\
86	0.01\\
86.01	0.01\\
86.02	0.01\\
86.03	0.01\\
86.04	0.01\\
86.05	0.01\\
86.06	0.01\\
86.07	0.01\\
86.08	0.01\\
86.09	0.01\\
86.1	0.01\\
86.11	0.01\\
86.12	0.01\\
86.13	0.01\\
86.14	0.01\\
86.15	0.01\\
86.16	0.01\\
86.17	0.01\\
86.18	0.01\\
86.19	0.01\\
86.2	0.01\\
86.21	0.01\\
86.22	0.01\\
86.23	0.01\\
86.24	0.01\\
86.25	0.01\\
86.26	0.01\\
86.27	0.01\\
86.28	0.01\\
86.29	0.01\\
86.3	0.01\\
86.31	0.01\\
86.32	0.01\\
86.33	0.01\\
86.34	0.01\\
86.35	0.01\\
86.36	0.01\\
86.37	0.01\\
86.38	0.01\\
86.39	0.01\\
86.4	0.01\\
86.41	0.01\\
86.42	0.01\\
86.43	0.01\\
86.44	0.01\\
86.45	0.01\\
86.46	0.01\\
86.47	0.01\\
86.48	0.01\\
86.49	0.01\\
86.5	0.01\\
86.51	0.01\\
86.52	0.01\\
86.53	0.01\\
86.54	0.01\\
86.55	0.01\\
86.56	0.01\\
86.57	0.01\\
86.58	0.01\\
86.59	0.01\\
86.6	0.01\\
86.61	0.01\\
86.62	0.01\\
86.63	0.01\\
86.64	0.01\\
86.65	0.01\\
86.66	0.01\\
86.67	0.01\\
86.68	0.01\\
86.69	0.01\\
86.7	0.01\\
86.71	0.01\\
86.72	0.01\\
86.73	0.01\\
86.74	0.01\\
86.75	0.01\\
86.76	0.01\\
86.77	0.01\\
86.78	0.01\\
86.79	0.01\\
86.8	0.01\\
86.81	0.01\\
86.82	0.01\\
86.83	0.01\\
86.84	0.01\\
86.85	0.01\\
86.86	0.01\\
86.87	0.01\\
86.88	0.01\\
86.89	0.01\\
86.9	0.01\\
86.91	0.01\\
86.92	0.01\\
86.93	0.01\\
86.94	0.01\\
86.95	0.01\\
86.96	0.01\\
86.97	0.01\\
86.98	0.01\\
86.99	0.01\\
87	0.01\\
87.01	0.01\\
87.02	0.01\\
87.03	0.01\\
87.04	0.01\\
87.05	0.01\\
87.06	0.01\\
87.07	0.01\\
87.08	0.01\\
87.09	0.01\\
87.1	0.01\\
87.11	0.01\\
87.12	0.01\\
87.13	0.01\\
87.14	0.01\\
87.15	0.01\\
87.16	0.01\\
87.17	0.01\\
87.18	0.01\\
87.19	0.01\\
87.2	0.01\\
87.21	0.01\\
87.22	0.01\\
87.23	0.01\\
87.24	0.01\\
87.25	0.01\\
87.26	0.01\\
87.27	0.01\\
87.28	0.01\\
87.29	0.01\\
87.3	0.01\\
87.31	0.01\\
87.32	0.01\\
87.33	0.01\\
87.34	0.01\\
87.35	0.01\\
87.36	0.01\\
87.37	0.01\\
87.38	0.01\\
87.39	0.01\\
87.4	0.01\\
87.41	0.01\\
87.42	0.01\\
87.43	0.01\\
87.44	0.01\\
87.45	0.01\\
87.46	0.01\\
87.47	0.01\\
87.48	0.01\\
87.49	0.01\\
87.5	0.01\\
87.51	0.01\\
87.52	0.01\\
87.53	0.01\\
87.54	0.01\\
87.55	0.01\\
87.56	0.01\\
87.57	0.01\\
87.58	0.01\\
87.59	0.01\\
87.6	0.01\\
87.61	0.01\\
87.62	0.01\\
87.63	0.01\\
87.64	0.01\\
87.65	0.01\\
87.66	0.01\\
87.67	0.01\\
87.68	0.01\\
87.69	0.01\\
87.7	0.01\\
87.71	0.01\\
87.72	0.01\\
87.73	0.01\\
87.74	0.01\\
87.75	0.01\\
87.76	0.01\\
87.77	0.01\\
87.78	0.01\\
87.79	0.01\\
87.8	0.01\\
87.81	0.01\\
87.82	0.01\\
87.83	0.01\\
87.84	0.01\\
87.85	0.01\\
87.86	0.01\\
87.87	0.01\\
87.88	0.01\\
87.89	0.01\\
87.9	0.01\\
87.91	0.01\\
87.92	0.01\\
87.93	0.01\\
87.94	0.01\\
87.95	0.01\\
87.96	0.01\\
87.97	0.01\\
87.98	0.01\\
87.99	0.01\\
88	0.01\\
88.01	0.01\\
88.02	0.01\\
88.03	0.01\\
88.04	0.01\\
88.05	0.01\\
88.06	0.01\\
88.07	0.01\\
88.08	0.01\\
88.09	0.01\\
88.1	0.01\\
88.11	0.01\\
88.12	0.01\\
88.13	0.01\\
88.14	0.01\\
88.15	0.01\\
88.16	0.01\\
88.17	0.01\\
88.18	0.01\\
88.19	0.01\\
88.2	0.01\\
88.21	0.01\\
88.22	0.01\\
88.23	0.01\\
88.24	0.01\\
88.25	0.01\\
88.26	0.01\\
88.27	0.01\\
88.28	0.01\\
88.29	0.01\\
88.3	0.01\\
88.31	0.01\\
88.32	0.01\\
88.33	0.01\\
88.34	0.01\\
88.35	0.01\\
88.36	0.01\\
88.37	0.01\\
88.38	0.01\\
88.39	0.01\\
88.4	0.01\\
88.41	0.01\\
88.42	0.01\\
88.43	0.01\\
88.44	0.01\\
88.45	0.01\\
88.46	0.01\\
88.47	0.01\\
88.48	0.01\\
88.49	0.01\\
88.5	0.01\\
88.51	0.01\\
88.52	0.01\\
88.53	0.01\\
88.54	0.01\\
88.55	0.01\\
88.56	0.01\\
88.57	0.01\\
88.58	0.01\\
88.59	0.01\\
88.6	0.01\\
88.61	0.01\\
88.62	0.01\\
88.63	0.01\\
88.64	0.01\\
88.65	0.01\\
88.66	0.01\\
88.67	0.01\\
88.68	0.01\\
88.69	0.01\\
88.7	0.01\\
88.71	0.01\\
88.72	0.01\\
88.73	0.01\\
88.74	0.01\\
88.75	0.01\\
88.76	0.01\\
88.77	0.01\\
88.78	0.01\\
88.79	0.01\\
88.8	0.01\\
88.81	0.01\\
88.82	0.01\\
88.83	0.01\\
88.84	0.01\\
88.85	0.01\\
88.86	0.01\\
88.87	0.01\\
88.88	0.01\\
88.89	0.01\\
88.9	0.01\\
88.91	0.01\\
88.92	0.01\\
88.93	0.01\\
88.94	0.01\\
88.95	0.01\\
88.96	0.01\\
88.97	0.01\\
88.98	0.01\\
88.99	0.01\\
89	0.01\\
89.01	0.01\\
89.02	0.01\\
89.03	0.01\\
89.04	0.01\\
89.05	0.01\\
89.06	0.01\\
89.07	0.01\\
89.08	0.01\\
89.09	0.01\\
89.1	0.01\\
89.11	0.01\\
89.12	0.01\\
89.13	0.01\\
89.14	0.01\\
89.15	0.01\\
89.16	0.01\\
89.17	0.01\\
89.18	0.01\\
89.19	0.01\\
89.2	0.01\\
89.21	0.01\\
89.22	0.01\\
89.23	0.01\\
89.24	0.01\\
89.25	0.01\\
89.26	0.01\\
89.27	0.01\\
89.28	0.01\\
89.29	0.01\\
89.3	0.01\\
89.31	0.01\\
89.32	0.01\\
89.33	0.01\\
89.34	0.01\\
89.35	0.01\\
89.36	0.01\\
89.37	0.01\\
89.38	0.01\\
89.39	0.01\\
89.4	0.01\\
89.41	0.01\\
89.42	0.01\\
89.43	0.01\\
89.44	0.01\\
89.45	0.01\\
89.46	0.01\\
89.47	0.01\\
89.48	0.01\\
89.49	0.01\\
89.5	0.01\\
89.51	0.01\\
89.52	0.01\\
89.53	0.01\\
89.54	0.01\\
89.55	0.01\\
89.56	0.01\\
89.57	0.01\\
89.58	0.01\\
89.59	0.01\\
89.6	0.01\\
89.61	0.01\\
89.62	0.01\\
89.63	0.01\\
89.64	0.01\\
89.65	0.01\\
89.66	0.01\\
89.67	0.01\\
89.68	0.01\\
89.69	0.01\\
89.7	0.01\\
89.71	0.01\\
89.72	0.01\\
89.73	0.01\\
89.74	0.01\\
89.75	0.01\\
89.76	0.01\\
89.77	0.01\\
89.78	0.01\\
89.79	0.01\\
89.8	0.01\\
89.81	0.01\\
89.82	0.01\\
89.83	0.01\\
89.84	0.01\\
89.85	0.01\\
89.86	0.01\\
89.87	0.01\\
89.88	0.01\\
89.89	0.01\\
89.9	0.01\\
89.91	0.01\\
89.92	0.01\\
89.93	0.01\\
89.94	0.01\\
89.95	0.01\\
89.96	0.01\\
89.97	0.01\\
89.98	0.01\\
89.99	0.01\\
90	0.01\\
90.01	0.01\\
90.02	0.01\\
90.03	0.01\\
90.04	0.01\\
90.05	0.01\\
90.06	0.01\\
90.07	0.01\\
90.08	0.01\\
90.09	0.01\\
90.1	0.01\\
90.11	0.01\\
90.12	0.01\\
90.13	0.01\\
90.14	0.01\\
90.15	0.01\\
90.16	0.01\\
90.17	0.01\\
90.18	0.01\\
90.19	0.01\\
90.2	0.01\\
90.21	0.01\\
90.22	0.01\\
90.23	0.01\\
90.24	0.01\\
90.25	0.01\\
90.26	0.01\\
90.27	0.01\\
90.28	0.01\\
90.29	0.01\\
90.3	0.01\\
90.31	0.01\\
90.32	0.01\\
90.33	0.01\\
90.34	0.01\\
90.35	0.01\\
90.36	0.01\\
90.37	0.01\\
90.38	0.01\\
90.39	0.01\\
90.4	0.01\\
90.41	0.01\\
90.42	0.01\\
90.43	0.01\\
90.44	0.01\\
90.45	0.01\\
90.46	0.01\\
90.47	0.01\\
90.48	0.01\\
90.49	0.01\\
90.5	0.01\\
90.51	0.01\\
90.52	0.01\\
90.53	0.01\\
90.54	0.01\\
90.55	0.01\\
90.56	0.01\\
90.57	0.01\\
90.58	0.01\\
90.59	0.01\\
90.6	0.01\\
90.61	0.01\\
90.62	0.01\\
90.63	0.01\\
90.64	0.01\\
90.65	0.01\\
90.66	0.01\\
90.67	0.01\\
90.68	0.01\\
90.69	0.01\\
90.7	0.01\\
90.71	0.01\\
90.72	0.01\\
90.73	0.01\\
90.74	0.01\\
90.75	0.01\\
90.76	0.01\\
90.77	0.01\\
90.78	0.01\\
90.79	0.01\\
90.8	0.01\\
90.81	0.01\\
90.82	0.01\\
90.83	0.01\\
90.84	0.01\\
90.85	0.01\\
90.86	0.01\\
90.87	0.01\\
90.88	0.01\\
90.89	0.01\\
90.9	0.01\\
90.91	0.01\\
90.92	0.01\\
90.93	0.01\\
90.94	0.01\\
90.95	0.01\\
90.96	0.01\\
90.97	0.01\\
90.98	0.01\\
90.99	0.01\\
91	0.01\\
91.01	0.01\\
91.02	0.01\\
91.03	0.01\\
91.04	0.01\\
91.05	0.01\\
91.06	0.01\\
91.07	0.01\\
91.08	0.01\\
91.09	0.01\\
91.1	0.01\\
91.11	0.01\\
91.12	0.01\\
91.13	0.01\\
91.14	0.01\\
91.15	0.01\\
91.16	0.01\\
91.17	0.01\\
91.18	0.01\\
91.19	0.01\\
91.2	0.01\\
91.21	0.01\\
91.22	0.01\\
91.23	0.01\\
91.24	0.01\\
91.25	0.01\\
91.26	0.01\\
91.27	0.01\\
91.28	0.01\\
91.29	0.01\\
91.3	0.01\\
91.31	0.01\\
91.32	0.01\\
91.33	0.01\\
91.34	0.01\\
91.35	0.01\\
91.36	0.01\\
91.37	0.01\\
91.38	0.01\\
91.39	0.01\\
91.4	0.01\\
91.41	0.01\\
91.42	0.01\\
91.43	0.01\\
91.44	0.01\\
91.45	0.01\\
91.46	0.01\\
91.47	0.01\\
91.48	0.01\\
91.49	0.01\\
91.5	0.01\\
91.51	0.01\\
91.52	0.01\\
91.53	0.01\\
91.54	0.01\\
91.55	0.01\\
91.56	0.01\\
91.57	0.01\\
91.58	0.01\\
91.59	0.01\\
91.6	0.01\\
91.61	0.01\\
91.62	0.01\\
91.63	0.01\\
91.64	0.01\\
91.65	0.01\\
91.66	0.01\\
91.67	0.01\\
91.68	0.01\\
91.69	0.01\\
91.7	0.01\\
91.71	0.01\\
91.72	0.01\\
91.73	0.01\\
91.74	0.01\\
91.75	0.01\\
91.76	0.01\\
91.77	0.01\\
91.78	0.01\\
91.79	0.01\\
91.8	0.01\\
91.81	0.01\\
91.82	0.01\\
91.83	0.01\\
91.84	0.01\\
91.85	0.01\\
91.86	0.01\\
91.87	0.01\\
91.88	0.01\\
91.89	0.01\\
91.9	0.01\\
91.91	0.01\\
91.92	0.01\\
91.93	0.01\\
91.94	0.01\\
91.95	0.01\\
91.96	0.01\\
91.97	0.01\\
91.98	0.01\\
91.99	0.01\\
92	0.01\\
92.01	0.01\\
92.02	0.01\\
92.03	0.01\\
92.04	0.01\\
92.05	0.01\\
92.06	0.01\\
92.07	0.01\\
92.08	0.01\\
92.09	0.01\\
92.1	0.01\\
92.11	0.01\\
92.12	0.01\\
92.13	0.01\\
92.14	0.01\\
92.15	0.01\\
92.16	0.01\\
92.17	0.01\\
92.18	0.01\\
92.19	0.01\\
92.2	0.01\\
92.21	0.01\\
92.22	0.01\\
92.23	0.01\\
92.24	0.01\\
92.25	0.01\\
92.26	0.01\\
92.27	0.01\\
92.28	0.01\\
92.29	0.01\\
92.3	0.01\\
92.31	0.01\\
92.32	0.01\\
92.33	0.01\\
92.34	0.01\\
92.35	0.01\\
92.36	0.01\\
92.37	0.01\\
92.38	0.01\\
92.39	0.01\\
92.4	0.01\\
92.41	0.01\\
92.42	0.01\\
92.43	0.01\\
92.44	0.01\\
92.45	0.01\\
92.46	0.01\\
92.47	0.01\\
92.48	0.01\\
92.49	0.01\\
92.5	0.01\\
92.51	0.01\\
92.52	0.01\\
92.53	0.01\\
92.54	0.01\\
92.55	0.01\\
92.56	0.01\\
92.57	0.01\\
92.58	0.01\\
92.59	0.01\\
92.6	0.01\\
92.61	0.01\\
92.62	0.01\\
92.63	0.01\\
92.64	0.01\\
92.65	0.01\\
92.66	0.01\\
92.67	0.01\\
92.68	0.01\\
92.69	0.01\\
92.7	0.01\\
92.71	0.01\\
92.72	0.01\\
92.73	0.01\\
92.74	0.01\\
92.75	0.01\\
92.76	0.01\\
92.77	0.01\\
92.78	0.01\\
92.79	0.01\\
92.8	0.01\\
92.81	0.01\\
92.82	0.01\\
92.83	0.01\\
92.84	0.01\\
92.85	0.01\\
92.86	0.01\\
92.87	0.01\\
92.88	0.01\\
92.89	0.01\\
92.9	0.01\\
92.91	0.01\\
92.92	0.01\\
92.93	0.01\\
92.94	0.01\\
92.95	0.01\\
92.96	0.01\\
92.97	0.01\\
92.98	0.01\\
92.99	0.01\\
93	0.01\\
93.01	0.01\\
93.02	0.01\\
93.03	0.01\\
93.04	0.01\\
93.05	0.01\\
93.06	0.01\\
93.07	0.01\\
93.08	0.01\\
93.09	0.01\\
93.1	0.01\\
93.11	0.01\\
93.12	0.01\\
93.13	0.01\\
93.14	0.01\\
93.15	0.01\\
93.16	0.01\\
93.17	0.01\\
93.18	0.01\\
93.19	0.01\\
93.2	0.01\\
93.21	0.01\\
93.22	0.01\\
93.23	0.01\\
93.24	0.01\\
93.25	0.01\\
93.26	0.01\\
93.27	0.01\\
93.28	0.01\\
93.29	0.01\\
93.3	0.01\\
93.31	0.01\\
93.32	0.01\\
93.33	0.01\\
93.34	0.01\\
93.35	0.01\\
93.36	0.01\\
93.37	0.01\\
93.38	0.01\\
93.39	0.01\\
93.4	0.01\\
93.41	0.01\\
93.42	0.01\\
93.43	0.01\\
93.44	0.01\\
93.45	0.01\\
93.46	0.01\\
93.47	0.01\\
93.48	0.01\\
93.49	0.01\\
93.5	0.01\\
93.51	0.01\\
93.52	0.01\\
93.53	0.01\\
93.54	0.01\\
93.55	0.01\\
93.56	0.01\\
93.57	0.01\\
93.58	0.01\\
93.59	0.01\\
93.6	0.01\\
93.61	0.01\\
93.62	0.01\\
93.63	0.01\\
93.64	0.01\\
93.65	0.01\\
93.66	0.01\\
93.67	0.01\\
93.68	0.01\\
93.69	0.01\\
93.7	0.01\\
93.71	0.01\\
93.72	0.01\\
93.73	0.01\\
93.74	0.01\\
93.75	0.01\\
93.76	0.01\\
93.77	0.01\\
93.78	0.01\\
93.79	0.01\\
93.8	0.01\\
93.81	0.01\\
93.82	0.01\\
93.83	0.01\\
93.84	0.01\\
93.85	0.01\\
93.86	0.01\\
93.87	0.01\\
93.88	0.01\\
93.89	0.01\\
93.9	0.01\\
93.91	0.01\\
93.92	0.01\\
93.93	0.01\\
93.94	0.01\\
93.95	0.01\\
93.96	0.01\\
93.97	0.01\\
93.98	0.01\\
93.99	0.01\\
94	0.01\\
94.01	0.01\\
94.02	0.01\\
94.03	0.01\\
94.04	0.01\\
94.05	0.01\\
94.06	0.01\\
94.07	0.01\\
94.08	0.01\\
94.09	0.01\\
94.1	0.01\\
94.11	0.01\\
94.12	0.01\\
94.13	0.01\\
94.14	0.01\\
94.15	0.01\\
94.16	0.01\\
94.17	0.01\\
94.18	0.01\\
94.19	0.01\\
94.2	0.01\\
94.21	0.01\\
94.22	0.01\\
94.23	0.01\\
94.24	0.01\\
94.25	0.01\\
94.26	0.01\\
94.27	0.01\\
94.28	0.01\\
94.29	0.01\\
94.3	0.01\\
94.31	0.01\\
94.32	0.01\\
94.33	0.01\\
94.34	0.01\\
94.35	0.01\\
94.36	0.01\\
94.37	0.01\\
94.38	0.01\\
94.39	0.01\\
94.4	0.01\\
94.41	0.01\\
94.42	0.01\\
94.43	0.01\\
94.44	0.01\\
94.45	0.01\\
94.46	0.01\\
94.47	0.01\\
94.48	0.01\\
94.49	0.01\\
94.5	0.01\\
94.51	0.01\\
94.52	0.01\\
94.53	0.01\\
94.54	0.01\\
94.55	0.01\\
94.56	0.01\\
94.57	0.01\\
94.58	0.01\\
94.59	0.01\\
94.6	0.01\\
94.61	0.01\\
94.62	0.01\\
94.63	0.01\\
94.64	0.01\\
94.65	0.01\\
94.66	0.01\\
94.67	0.01\\
94.68	0.01\\
94.69	0.01\\
94.7	0.01\\
94.71	0.01\\
94.72	0.01\\
94.73	0.01\\
94.74	0.01\\
94.75	0.01\\
94.76	0.01\\
94.77	0.01\\
94.78	0.01\\
94.79	0.01\\
94.8	0.01\\
94.81	0.01\\
94.82	0.01\\
94.83	0.01\\
94.84	0.01\\
94.85	0.01\\
94.86	0.01\\
94.87	0.01\\
94.88	0.01\\
94.89	0.01\\
94.9	0.01\\
94.91	0.01\\
94.92	0.01\\
94.93	0.01\\
94.94	0.01\\
94.95	0.01\\
94.96	0.01\\
94.97	0.01\\
94.98	0.01\\
94.99	0.01\\
95	0.01\\
95.01	0.01\\
95.02	0.01\\
95.03	0.01\\
95.04	0.01\\
95.05	0.01\\
95.06	0.01\\
95.07	0.01\\
95.08	0.01\\
95.09	0.01\\
95.1	0.01\\
95.11	0.01\\
95.12	0.01\\
95.13	0.01\\
95.14	0.01\\
95.15	0.01\\
95.16	0.01\\
95.17	0.01\\
95.18	0.01\\
95.19	0.01\\
95.2	0.01\\
95.21	0.01\\
95.22	0.01\\
95.23	0.01\\
95.24	0.01\\
95.25	0.01\\
95.26	0.01\\
95.27	0.01\\
95.28	0.01\\
95.29	0.01\\
95.3	0.01\\
95.31	0.01\\
95.32	0.01\\
95.33	0.01\\
95.34	0.01\\
95.35	0.01\\
95.36	0.01\\
95.37	0.01\\
95.38	0.01\\
95.39	0.01\\
95.4	0.01\\
95.41	0.01\\
95.42	0.01\\
95.43	0.01\\
95.44	0.01\\
95.45	0.01\\
95.46	0.01\\
95.47	0.01\\
95.48	0.01\\
95.49	0.01\\
95.5	0.01\\
95.51	0.01\\
95.52	0.01\\
95.53	0.01\\
95.54	0.01\\
95.55	0.01\\
95.56	0.01\\
95.57	0.01\\
95.58	0.01\\
95.59	0.01\\
95.6	0.01\\
95.61	0.01\\
95.62	0.01\\
95.63	0.01\\
95.64	0.01\\
95.65	0.01\\
95.66	0.01\\
95.67	0.01\\
95.68	0.01\\
95.69	0.01\\
95.7	0.01\\
95.71	0.01\\
95.72	0.01\\
95.73	0.01\\
95.74	0.01\\
95.75	0.01\\
95.76	0.01\\
95.77	0.01\\
95.78	0.01\\
95.79	0.01\\
95.8	0.01\\
95.81	0.01\\
95.82	0.01\\
95.83	0.01\\
95.84	0.01\\
95.85	0.01\\
95.86	0.01\\
95.87	0.01\\
95.88	0.01\\
95.89	0.01\\
95.9	0.01\\
95.91	0.01\\
95.92	0.01\\
95.93	0.01\\
95.94	0.01\\
95.95	0.01\\
95.96	0.01\\
95.97	0.01\\
95.98	0.01\\
95.99	0.01\\
96	0.01\\
96.01	0.01\\
96.02	0.01\\
96.03	0.01\\
96.04	0.01\\
96.05	0.01\\
96.06	0.01\\
96.07	0.01\\
96.08	0.01\\
96.09	0.01\\
96.1	0.01\\
96.11	0.01\\
96.12	0.01\\
96.13	0.01\\
96.14	0.01\\
96.15	0.01\\
96.16	0.01\\
96.17	0.01\\
96.18	0.01\\
96.19	0.01\\
96.2	0.01\\
96.21	0.01\\
96.22	0.01\\
96.23	0.01\\
96.24	0.01\\
96.25	0.01\\
96.26	0.01\\
96.27	0.01\\
96.28	0.01\\
96.29	0.01\\
96.3	0.01\\
96.31	0.01\\
96.32	0.01\\
96.33	0.01\\
96.34	0.01\\
96.35	0.01\\
96.36	0.01\\
96.37	0.01\\
96.38	0.01\\
96.39	0.01\\
96.4	0.01\\
96.41	0.01\\
96.42	0.01\\
96.43	0.01\\
96.44	0.01\\
96.45	0.01\\
96.46	0.01\\
96.47	0.01\\
96.48	0.01\\
96.49	0.01\\
96.5	0.01\\
96.51	0.01\\
96.52	0.01\\
96.53	0.01\\
96.54	0.01\\
96.55	0.01\\
96.56	0.01\\
96.57	0.01\\
96.58	0.01\\
96.59	0.01\\
96.6	0.01\\
96.61	0.01\\
96.62	0.01\\
96.63	0.01\\
96.64	0.01\\
96.65	0.01\\
96.66	0.01\\
96.67	0.01\\
96.68	0.01\\
96.69	0.01\\
96.7	0.01\\
96.71	0.01\\
96.72	0.01\\
96.73	0.01\\
96.74	0.01\\
96.75	0.01\\
96.76	0.01\\
96.77	0.01\\
96.78	0.01\\
96.79	0.01\\
96.8	0.01\\
96.81	0.01\\
96.82	0.01\\
96.83	0.01\\
96.84	0.01\\
96.85	0.01\\
96.86	0.01\\
96.87	0.01\\
96.88	0.01\\
96.89	0.01\\
96.9	0.01\\
96.91	0.01\\
96.92	0.01\\
96.93	0.01\\
96.94	0.01\\
96.95	0.01\\
96.96	0.01\\
96.97	0.01\\
96.98	0.01\\
96.99	0.01\\
97	0.01\\
97.01	0.01\\
97.02	0.01\\
97.03	0.01\\
97.04	0.01\\
97.05	0.01\\
97.06	0.01\\
97.07	0.01\\
97.08	0.01\\
97.09	0.01\\
97.1	0.01\\
97.11	0.01\\
97.12	0.01\\
97.13	0.01\\
97.14	0.01\\
97.15	0.01\\
97.16	0.01\\
97.17	0.01\\
97.18	0.01\\
97.19	0.01\\
97.2	0.01\\
97.21	0.01\\
97.22	0.01\\
97.23	0.01\\
97.24	0.01\\
97.25	0.01\\
97.26	0.01\\
97.27	0.01\\
97.28	0.01\\
97.29	0.01\\
97.3	0.01\\
97.31	0.01\\
97.32	0.01\\
97.33	0.01\\
97.34	0.01\\
97.35	0.01\\
97.36	0.01\\
97.37	0.01\\
97.38	0.01\\
97.39	0.01\\
97.4	0.01\\
97.41	0.01\\
97.42	0.01\\
97.43	0.01\\
97.44	0.01\\
97.45	0.01\\
97.46	0.01\\
97.47	0.01\\
97.48	0.01\\
97.49	0.01\\
97.5	0.01\\
97.51	0.01\\
97.52	0.01\\
97.53	0.01\\
97.54	0.01\\
97.55	0.01\\
97.56	0.01\\
97.57	0.01\\
97.58	0.01\\
97.59	0.01\\
97.6	0.01\\
97.61	0.01\\
97.62	0.01\\
97.63	0.01\\
97.64	0.01\\
97.65	0.01\\
97.66	0.01\\
97.67	0.01\\
97.68	0.01\\
97.69	0.01\\
97.7	0.01\\
97.71	0.01\\
97.72	0.01\\
97.73	0.01\\
97.74	0.01\\
97.75	0.01\\
97.76	0.01\\
97.77	0.01\\
97.78	0.01\\
97.79	0.01\\
97.8	0.01\\
97.81	0.01\\
97.82	0.01\\
97.83	0.01\\
97.84	0.01\\
97.85	0.01\\
97.86	0.01\\
97.87	0.01\\
97.88	0.01\\
97.89	0.01\\
97.9	0.01\\
97.91	0.01\\
97.92	0.01\\
97.93	0.01\\
97.94	0.01\\
97.95	0.01\\
97.96	0.01\\
97.97	0.01\\
97.98	0.01\\
97.99	0.01\\
98	0.01\\
98.01	0.01\\
98.02	0.01\\
98.03	0.01\\
98.04	0.01\\
98.05	0.01\\
98.06	0.01\\
98.07	0.01\\
98.08	0.01\\
98.09	0.01\\
98.1	0.01\\
98.11	0.01\\
98.12	0.01\\
98.13	0.01\\
98.14	0.01\\
98.15	0.01\\
98.16	0.01\\
98.17	0.01\\
98.18	0.01\\
98.19	0.01\\
98.2	0.01\\
98.21	0.01\\
98.22	0.01\\
98.23	0.01\\
98.24	0.01\\
98.25	0.01\\
98.26	0.01\\
98.27	0.01\\
98.28	0.01\\
98.29	0.01\\
98.3	0.01\\
98.31	0.01\\
98.32	0.01\\
98.33	0.01\\
98.34	0.01\\
98.35	0.01\\
98.36	0.01\\
98.37	0.01\\
98.38	0.01\\
98.39	0.01\\
98.4	0.01\\
98.41	0.01\\
98.42	0.01\\
98.43	0.01\\
98.44	0.01\\
98.45	0.01\\
98.46	0.01\\
98.47	0.01\\
98.48	0.01\\
98.49	0.01\\
98.5	0.01\\
98.51	0.01\\
98.52	0.01\\
98.53	0.01\\
98.54	0.01\\
98.55	0.01\\
98.56	0.01\\
98.57	0.01\\
98.58	0.01\\
98.59	0.01\\
98.6	0.01\\
98.61	0.01\\
98.62	0.01\\
98.63	0.01\\
98.64	0.01\\
98.65	0.01\\
98.66	0.01\\
98.67	0.01\\
98.68	0.01\\
98.69	0.01\\
98.7	0.01\\
98.71	0.01\\
98.72	0.01\\
98.73	0.01\\
98.74	0.01\\
98.75	0.01\\
98.76	0.01\\
98.77	0.01\\
98.78	0.01\\
98.79	0.01\\
98.8	0.01\\
98.81	0.01\\
98.82	0.01\\
98.83	0.01\\
98.84	0.01\\
98.85	0.01\\
98.86	0.01\\
98.87	0.01\\
98.88	0.01\\
98.89	0.01\\
98.9	0.01\\
98.91	0.01\\
98.92	0.01\\
98.93	0.01\\
98.94	0.01\\
98.95	0.01\\
98.96	0.01\\
98.97	0.01\\
98.98	0.01\\
98.99	0.01\\
99	0.01\\
99.01	0.01\\
99.02	0.01\\
99.03	0.01\\
99.04	0.01\\
99.05	0.01\\
99.06	0.01\\
99.07	0.01\\
99.08	0.01\\
99.09	0.01\\
99.1	0.01\\
99.11	0.01\\
99.12	0.01\\
99.13	0.01\\
99.14	0.00998658074540667\\
99.15	0.00981394762269664\\
99.16	0.00964021100322028\\
99.17	0.00946535581611112\\
99.18	0.0092893668729081\\
99.19	0.00911222854862415\\
99.2	0.00893392479044062\\
99.21	0.00875443910369502\\
99.22	0.00857376609587646\\
99.23	0.00839192844111959\\
99.24	0.00820890970110557\\
99.25	0.00802469297024153\\
99.26	0.00783926085981311\\
99.27	0.00765259548150028\\
99.28	0.00746467843022284\\
99.29	0.00727549076627984\\
99.3	0.00708501299674452\\
99.31	0.00689322505607424\\
99.32	0.00670010628589174\\
99.33	0.00650563541389112\\
99.34	0.00630979053181893\\
99.35	0.00611254922952013\\
99.36	0.00591388841707558\\
99.37	0.00571378427464273\\
99.38	0.00551221206393733\\
99.39	0.00530914628219126\\
99.4	0.00510456061610345\\
99.41	0.00489842792758349\\
99.42	0.00483354096659049\\
99.43	0.00476839726764135\\
99.44	0.00470269466849563\\
99.45	0.00463643043392819\\
99.46	0.0045696019104531\\
99.47	0.00450220652384934\\
99.48	0.00443422230458845\\
99.49	0.0043656225605152\\
99.5	0.0042964036119007\\
99.51	0.00422656183753986\\
99.52	0.00415609367880683\\
99.53	0.00408499564389557\\
99.54	0.00401326431225524\\
99.55	0.00394089633923041\\
99.56	0.0038678884609169\\
99.57	0.00379423749924454\\
99.58	0.00371994036729902\\
99.59	0.00364499407489556\\
99.6	0.00356939573441804\\
99.61	0.00349314256693811\\
99.62	0.00341623190862955\\
99.63	0.00333866121749428\\
99.64	0.00326042808041742\\
99.65	0.00318153022056972\\
99.66	0.00310196550517733\\
99.67	0.00302173195367941\\
99.68	0.00294082774629635\\
99.69	0.00285925123288629\\
99.7	0.00277700094237731\\
99.71	0.00269407559271374\\
99.72	0.00261047410832228\\
99.73	0.00252619563924148\\
99.74	0.00244123955819906\\
99.75	0.00235560547282003\\
99.76	0.0022692932384859\\
99.77	0.00218230297188465\\
99.78	0.002094635065294\\
99.79	0.00200629020164378\\
99.8	0.00191726937040586\\
99.81	0.001827573884364\\
99.82	0.0017372053973193\\
99.83	0.00164616592279117\\
99.84	0.00155445785377785\\
99.85	0.00146208398364519\\
99.86	0.00136904752821746\\
99.87	0.00127535214914916\\
99.88	0.00118100197866295\\
99.89	0.00108600164574474\\
99.9	0.000990356303894199\\
99.91	0.000894071660039042\\
99.92	0.000797154004978352\\
99.93	0.000699610246319197\\
99.94	0.000601447943428838\\
99.95	0.000502675344543365\\
99.96	0.000403301426184725\\
99.97	0.000303335935050114\\
99.98	0.000202789432550832\\
99.99	0.00010167334219198\\
100	0\\
};
\addlegendentry{$q=-2$};

\addplot [color=blue,dashed,forget plot]
  table[row sep=crcr]{%
0.01	0.01\\
0.02	0.01\\
0.03	0.01\\
0.04	0.01\\
0.05	0.01\\
0.06	0.01\\
0.07	0.01\\
0.08	0.01\\
0.09	0.01\\
0.1	0.01\\
0.11	0.01\\
0.12	0.01\\
0.13	0.01\\
0.14	0.01\\
0.15	0.01\\
0.16	0.01\\
0.17	0.01\\
0.18	0.01\\
0.19	0.01\\
0.2	0.01\\
0.21	0.01\\
0.22	0.01\\
0.23	0.01\\
0.24	0.01\\
0.25	0.01\\
0.26	0.01\\
0.27	0.01\\
0.28	0.01\\
0.29	0.01\\
0.3	0.01\\
0.31	0.01\\
0.32	0.01\\
0.33	0.01\\
0.34	0.01\\
0.35	0.01\\
0.36	0.01\\
0.37	0.01\\
0.38	0.01\\
0.39	0.01\\
0.4	0.01\\
0.41	0.01\\
0.42	0.01\\
0.43	0.01\\
0.44	0.01\\
0.45	0.01\\
0.46	0.01\\
0.47	0.01\\
0.48	0.01\\
0.49	0.01\\
0.5	0.01\\
0.51	0.01\\
0.52	0.01\\
0.53	0.01\\
0.54	0.01\\
0.55	0.01\\
0.56	0.01\\
0.57	0.01\\
0.58	0.01\\
0.59	0.01\\
0.6	0.01\\
0.61	0.01\\
0.62	0.01\\
0.63	0.01\\
0.64	0.01\\
0.65	0.01\\
0.66	0.01\\
0.67	0.01\\
0.68	0.01\\
0.69	0.01\\
0.7	0.01\\
0.71	0.01\\
0.72	0.01\\
0.73	0.01\\
0.74	0.01\\
0.75	0.01\\
0.76	0.01\\
0.77	0.01\\
0.78	0.01\\
0.79	0.01\\
0.8	0.01\\
0.81	0.01\\
0.82	0.01\\
0.83	0.01\\
0.84	0.01\\
0.85	0.01\\
0.86	0.01\\
0.87	0.01\\
0.88	0.01\\
0.89	0.01\\
0.9	0.01\\
0.91	0.01\\
0.92	0.01\\
0.93	0.01\\
0.94	0.01\\
0.95	0.01\\
0.96	0.01\\
0.97	0.01\\
0.98	0.01\\
0.99	0.01\\
1	0.01\\
1.01	0.01\\
1.02	0.01\\
1.03	0.01\\
1.04	0.01\\
1.05	0.01\\
1.06	0.01\\
1.07	0.01\\
1.08	0.01\\
1.09	0.01\\
1.1	0.01\\
1.11	0.01\\
1.12	0.01\\
1.13	0.01\\
1.14	0.01\\
1.15	0.01\\
1.16	0.01\\
1.17	0.01\\
1.18	0.01\\
1.19	0.01\\
1.2	0.01\\
1.21	0.01\\
1.22	0.01\\
1.23	0.01\\
1.24	0.01\\
1.25	0.01\\
1.26	0.01\\
1.27	0.01\\
1.28	0.01\\
1.29	0.01\\
1.3	0.01\\
1.31	0.01\\
1.32	0.01\\
1.33	0.01\\
1.34	0.01\\
1.35	0.01\\
1.36	0.01\\
1.37	0.01\\
1.38	0.01\\
1.39	0.01\\
1.4	0.01\\
1.41	0.01\\
1.42	0.01\\
1.43	0.01\\
1.44	0.01\\
1.45	0.01\\
1.46	0.01\\
1.47	0.01\\
1.48	0.01\\
1.49	0.01\\
1.5	0.01\\
1.51	0.01\\
1.52	0.01\\
1.53	0.01\\
1.54	0.01\\
1.55	0.01\\
1.56	0.01\\
1.57	0.01\\
1.58	0.01\\
1.59	0.01\\
1.6	0.01\\
1.61	0.01\\
1.62	0.01\\
1.63	0.01\\
1.64	0.01\\
1.65	0.01\\
1.66	0.01\\
1.67	0.01\\
1.68	0.01\\
1.69	0.01\\
1.7	0.01\\
1.71	0.01\\
1.72	0.01\\
1.73	0.01\\
1.74	0.01\\
1.75	0.01\\
1.76	0.01\\
1.77	0.01\\
1.78	0.01\\
1.79	0.01\\
1.8	0.01\\
1.81	0.01\\
1.82	0.01\\
1.83	0.01\\
1.84	0.01\\
1.85	0.01\\
1.86	0.01\\
1.87	0.01\\
1.88	0.01\\
1.89	0.01\\
1.9	0.01\\
1.91	0.01\\
1.92	0.01\\
1.93	0.01\\
1.94	0.01\\
1.95	0.01\\
1.96	0.01\\
1.97	0.01\\
1.98	0.01\\
1.99	0.01\\
2	0.01\\
2.01	0.01\\
2.02	0.01\\
2.03	0.01\\
2.04	0.01\\
2.05	0.01\\
2.06	0.01\\
2.07	0.01\\
2.08	0.01\\
2.09	0.01\\
2.1	0.01\\
2.11	0.01\\
2.12	0.01\\
2.13	0.01\\
2.14	0.01\\
2.15	0.01\\
2.16	0.01\\
2.17	0.01\\
2.18	0.01\\
2.19	0.01\\
2.2	0.01\\
2.21	0.01\\
2.22	0.01\\
2.23	0.01\\
2.24	0.01\\
2.25	0.01\\
2.26	0.01\\
2.27	0.01\\
2.28	0.01\\
2.29	0.01\\
2.3	0.01\\
2.31	0.01\\
2.32	0.01\\
2.33	0.01\\
2.34	0.01\\
2.35	0.01\\
2.36	0.01\\
2.37	0.01\\
2.38	0.01\\
2.39	0.01\\
2.4	0.01\\
2.41	0.01\\
2.42	0.01\\
2.43	0.01\\
2.44	0.01\\
2.45	0.01\\
2.46	0.01\\
2.47	0.01\\
2.48	0.01\\
2.49	0.01\\
2.5	0.01\\
2.51	0.01\\
2.52	0.01\\
2.53	0.01\\
2.54	0.01\\
2.55	0.01\\
2.56	0.01\\
2.57	0.01\\
2.58	0.01\\
2.59	0.01\\
2.6	0.01\\
2.61	0.01\\
2.62	0.01\\
2.63	0.01\\
2.64	0.01\\
2.65	0.01\\
2.66	0.01\\
2.67	0.01\\
2.68	0.01\\
2.69	0.01\\
2.7	0.01\\
2.71	0.01\\
2.72	0.01\\
2.73	0.01\\
2.74	0.01\\
2.75	0.01\\
2.76	0.01\\
2.77	0.01\\
2.78	0.01\\
2.79	0.01\\
2.8	0.01\\
2.81	0.01\\
2.82	0.01\\
2.83	0.01\\
2.84	0.01\\
2.85	0.01\\
2.86	0.01\\
2.87	0.01\\
2.88	0.01\\
2.89	0.01\\
2.9	0.01\\
2.91	0.01\\
2.92	0.01\\
2.93	0.01\\
2.94	0.01\\
2.95	0.01\\
2.96	0.01\\
2.97	0.01\\
2.98	0.01\\
2.99	0.01\\
3	0.01\\
3.01	0.01\\
3.02	0.01\\
3.03	0.01\\
3.04	0.01\\
3.05	0.01\\
3.06	0.01\\
3.07	0.01\\
3.08	0.01\\
3.09	0.01\\
3.1	0.01\\
3.11	0.01\\
3.12	0.01\\
3.13	0.01\\
3.14	0.01\\
3.15	0.01\\
3.16	0.01\\
3.17	0.01\\
3.18	0.01\\
3.19	0.01\\
3.2	0.01\\
3.21	0.01\\
3.22	0.01\\
3.23	0.01\\
3.24	0.01\\
3.25	0.01\\
3.26	0.01\\
3.27	0.01\\
3.28	0.01\\
3.29	0.01\\
3.3	0.01\\
3.31	0.01\\
3.32	0.01\\
3.33	0.01\\
3.34	0.01\\
3.35	0.01\\
3.36	0.01\\
3.37	0.01\\
3.38	0.01\\
3.39	0.01\\
3.4	0.01\\
3.41	0.01\\
3.42	0.01\\
3.43	0.01\\
3.44	0.01\\
3.45	0.01\\
3.46	0.01\\
3.47	0.01\\
3.48	0.01\\
3.49	0.01\\
3.5	0.01\\
3.51	0.01\\
3.52	0.01\\
3.53	0.01\\
3.54	0.01\\
3.55	0.01\\
3.56	0.01\\
3.57	0.01\\
3.58	0.01\\
3.59	0.01\\
3.6	0.01\\
3.61	0.01\\
3.62	0.01\\
3.63	0.01\\
3.64	0.01\\
3.65	0.01\\
3.66	0.01\\
3.67	0.01\\
3.68	0.01\\
3.69	0.01\\
3.7	0.01\\
3.71	0.01\\
3.72	0.01\\
3.73	0.01\\
3.74	0.01\\
3.75	0.01\\
3.76	0.01\\
3.77	0.01\\
3.78	0.01\\
3.79	0.01\\
3.8	0.01\\
3.81	0.01\\
3.82	0.01\\
3.83	0.01\\
3.84	0.01\\
3.85	0.01\\
3.86	0.01\\
3.87	0.01\\
3.88	0.01\\
3.89	0.01\\
3.9	0.01\\
3.91	0.01\\
3.92	0.01\\
3.93	0.01\\
3.94	0.01\\
3.95	0.01\\
3.96	0.01\\
3.97	0.01\\
3.98	0.01\\
3.99	0.01\\
4	0.01\\
4.01	0.01\\
4.02	0.01\\
4.03	0.01\\
4.04	0.01\\
4.05	0.01\\
4.06	0.01\\
4.07	0.01\\
4.08	0.01\\
4.09	0.01\\
4.1	0.01\\
4.11	0.01\\
4.12	0.01\\
4.13	0.01\\
4.14	0.01\\
4.15	0.01\\
4.16	0.01\\
4.17	0.01\\
4.18	0.01\\
4.19	0.01\\
4.2	0.01\\
4.21	0.01\\
4.22	0.01\\
4.23	0.01\\
4.24	0.01\\
4.25	0.01\\
4.26	0.01\\
4.27	0.01\\
4.28	0.01\\
4.29	0.01\\
4.3	0.01\\
4.31	0.01\\
4.32	0.01\\
4.33	0.01\\
4.34	0.01\\
4.35	0.01\\
4.36	0.01\\
4.37	0.01\\
4.38	0.01\\
4.39	0.01\\
4.4	0.01\\
4.41	0.01\\
4.42	0.01\\
4.43	0.01\\
4.44	0.01\\
4.45	0.01\\
4.46	0.01\\
4.47	0.01\\
4.48	0.01\\
4.49	0.01\\
4.5	0.01\\
4.51	0.01\\
4.52	0.01\\
4.53	0.01\\
4.54	0.01\\
4.55	0.01\\
4.56	0.01\\
4.57	0.01\\
4.58	0.01\\
4.59	0.01\\
4.6	0.01\\
4.61	0.01\\
4.62	0.01\\
4.63	0.01\\
4.64	0.01\\
4.65	0.01\\
4.66	0.01\\
4.67	0.01\\
4.68	0.01\\
4.69	0.01\\
4.7	0.01\\
4.71	0.01\\
4.72	0.01\\
4.73	0.01\\
4.74	0.01\\
4.75	0.01\\
4.76	0.01\\
4.77	0.01\\
4.78	0.01\\
4.79	0.01\\
4.8	0.01\\
4.81	0.01\\
4.82	0.01\\
4.83	0.01\\
4.84	0.01\\
4.85	0.01\\
4.86	0.01\\
4.87	0.01\\
4.88	0.01\\
4.89	0.01\\
4.9	0.01\\
4.91	0.01\\
4.92	0.01\\
4.93	0.01\\
4.94	0.01\\
4.95	0.01\\
4.96	0.01\\
4.97	0.01\\
4.98	0.01\\
4.99	0.01\\
5	0.01\\
5.01	0.01\\
5.02	0.01\\
5.03	0.01\\
5.04	0.01\\
5.05	0.01\\
5.06	0.01\\
5.07	0.01\\
5.08	0.01\\
5.09	0.01\\
5.1	0.01\\
5.11	0.01\\
5.12	0.01\\
5.13	0.01\\
5.14	0.01\\
5.15	0.01\\
5.16	0.01\\
5.17	0.01\\
5.18	0.01\\
5.19	0.01\\
5.2	0.01\\
5.21	0.01\\
5.22	0.01\\
5.23	0.01\\
5.24	0.01\\
5.25	0.01\\
5.26	0.01\\
5.27	0.01\\
5.28	0.01\\
5.29	0.01\\
5.3	0.01\\
5.31	0.01\\
5.32	0.01\\
5.33	0.01\\
5.34	0.01\\
5.35	0.01\\
5.36	0.01\\
5.37	0.01\\
5.38	0.01\\
5.39	0.01\\
5.4	0.01\\
5.41	0.01\\
5.42	0.01\\
5.43	0.01\\
5.44	0.01\\
5.45	0.01\\
5.46	0.01\\
5.47	0.01\\
5.48	0.01\\
5.49	0.01\\
5.5	0.01\\
5.51	0.01\\
5.52	0.01\\
5.53	0.01\\
5.54	0.01\\
5.55	0.01\\
5.56	0.01\\
5.57	0.01\\
5.58	0.01\\
5.59	0.01\\
5.6	0.01\\
5.61	0.01\\
5.62	0.01\\
5.63	0.01\\
5.64	0.01\\
5.65	0.01\\
5.66	0.01\\
5.67	0.01\\
5.68	0.01\\
5.69	0.01\\
5.7	0.01\\
5.71	0.01\\
5.72	0.01\\
5.73	0.01\\
5.74	0.01\\
5.75	0.01\\
5.76	0.01\\
5.77	0.01\\
5.78	0.01\\
5.79	0.01\\
5.8	0.01\\
5.81	0.01\\
5.82	0.01\\
5.83	0.01\\
5.84	0.01\\
5.85	0.01\\
5.86	0.01\\
5.87	0.01\\
5.88	0.01\\
5.89	0.01\\
5.9	0.01\\
5.91	0.01\\
5.92	0.01\\
5.93	0.01\\
5.94	0.01\\
5.95	0.01\\
5.96	0.01\\
5.97	0.01\\
5.98	0.01\\
5.99	0.01\\
6	0.01\\
6.01	0.01\\
6.02	0.01\\
6.03	0.01\\
6.04	0.01\\
6.05	0.01\\
6.06	0.01\\
6.07	0.01\\
6.08	0.01\\
6.09	0.01\\
6.1	0.01\\
6.11	0.01\\
6.12	0.01\\
6.13	0.01\\
6.14	0.01\\
6.15	0.01\\
6.16	0.01\\
6.17	0.01\\
6.18	0.01\\
6.19	0.01\\
6.2	0.01\\
6.21	0.01\\
6.22	0.01\\
6.23	0.01\\
6.24	0.01\\
6.25	0.01\\
6.26	0.01\\
6.27	0.01\\
6.28	0.01\\
6.29	0.01\\
6.3	0.01\\
6.31	0.01\\
6.32	0.01\\
6.33	0.01\\
6.34	0.01\\
6.35	0.01\\
6.36	0.01\\
6.37	0.01\\
6.38	0.01\\
6.39	0.01\\
6.4	0.01\\
6.41	0.01\\
6.42	0.01\\
6.43	0.01\\
6.44	0.01\\
6.45	0.01\\
6.46	0.01\\
6.47	0.01\\
6.48	0.01\\
6.49	0.01\\
6.5	0.01\\
6.51	0.01\\
6.52	0.01\\
6.53	0.01\\
6.54	0.01\\
6.55	0.01\\
6.56	0.01\\
6.57	0.01\\
6.58	0.01\\
6.59	0.01\\
6.6	0.01\\
6.61	0.01\\
6.62	0.01\\
6.63	0.01\\
6.64	0.01\\
6.65	0.01\\
6.66	0.01\\
6.67	0.01\\
6.68	0.01\\
6.69	0.01\\
6.7	0.01\\
6.71	0.01\\
6.72	0.01\\
6.73	0.01\\
6.74	0.01\\
6.75	0.01\\
6.76	0.01\\
6.77	0.01\\
6.78	0.01\\
6.79	0.01\\
6.8	0.01\\
6.81	0.01\\
6.82	0.01\\
6.83	0.01\\
6.84	0.01\\
6.85	0.01\\
6.86	0.01\\
6.87	0.01\\
6.88	0.01\\
6.89	0.01\\
6.9	0.01\\
6.91	0.01\\
6.92	0.01\\
6.93	0.01\\
6.94	0.01\\
6.95	0.01\\
6.96	0.01\\
6.97	0.01\\
6.98	0.01\\
6.99	0.01\\
7	0.01\\
7.01	0.01\\
7.02	0.01\\
7.03	0.01\\
7.04	0.01\\
7.05	0.01\\
7.06	0.01\\
7.07	0.01\\
7.08	0.01\\
7.09	0.01\\
7.1	0.01\\
7.11	0.01\\
7.12	0.01\\
7.13	0.01\\
7.14	0.01\\
7.15	0.01\\
7.16	0.01\\
7.17	0.01\\
7.18	0.01\\
7.19	0.01\\
7.2	0.01\\
7.21	0.01\\
7.22	0.01\\
7.23	0.01\\
7.24	0.01\\
7.25	0.01\\
7.26	0.01\\
7.27	0.01\\
7.28	0.01\\
7.29	0.01\\
7.3	0.01\\
7.31	0.01\\
7.32	0.01\\
7.33	0.01\\
7.34	0.01\\
7.35	0.01\\
7.36	0.01\\
7.37	0.01\\
7.38	0.01\\
7.39	0.01\\
7.4	0.01\\
7.41	0.01\\
7.42	0.01\\
7.43	0.01\\
7.44	0.01\\
7.45	0.01\\
7.46	0.01\\
7.47	0.01\\
7.48	0.01\\
7.49	0.01\\
7.5	0.01\\
7.51	0.01\\
7.52	0.01\\
7.53	0.01\\
7.54	0.01\\
7.55	0.01\\
7.56	0.01\\
7.57	0.01\\
7.58	0.01\\
7.59	0.01\\
7.6	0.01\\
7.61	0.01\\
7.62	0.01\\
7.63	0.01\\
7.64	0.01\\
7.65	0.01\\
7.66	0.01\\
7.67	0.01\\
7.68	0.01\\
7.69	0.01\\
7.7	0.01\\
7.71	0.01\\
7.72	0.01\\
7.73	0.01\\
7.74	0.01\\
7.75	0.01\\
7.76	0.01\\
7.77	0.01\\
7.78	0.01\\
7.79	0.01\\
7.8	0.01\\
7.81	0.01\\
7.82	0.01\\
7.83	0.01\\
7.84	0.01\\
7.85	0.01\\
7.86	0.01\\
7.87	0.01\\
7.88	0.01\\
7.89	0.01\\
7.9	0.01\\
7.91	0.01\\
7.92	0.01\\
7.93	0.01\\
7.94	0.01\\
7.95	0.01\\
7.96	0.01\\
7.97	0.01\\
7.98	0.01\\
7.99	0.01\\
8	0.01\\
8.01	0.01\\
8.02	0.01\\
8.03	0.01\\
8.04	0.01\\
8.05	0.01\\
8.06	0.01\\
8.07	0.01\\
8.08	0.01\\
8.09	0.01\\
8.1	0.01\\
8.11	0.01\\
8.12	0.01\\
8.13	0.01\\
8.14	0.01\\
8.15	0.01\\
8.16	0.01\\
8.17	0.01\\
8.18	0.01\\
8.19	0.01\\
8.2	0.01\\
8.21	0.01\\
8.22	0.01\\
8.23	0.01\\
8.24	0.01\\
8.25	0.01\\
8.26	0.01\\
8.27	0.01\\
8.28	0.01\\
8.29	0.01\\
8.3	0.01\\
8.31	0.01\\
8.32	0.01\\
8.33	0.01\\
8.34	0.01\\
8.35	0.01\\
8.36	0.01\\
8.37	0.01\\
8.38	0.01\\
8.39	0.01\\
8.4	0.01\\
8.41	0.01\\
8.42	0.01\\
8.43	0.01\\
8.44	0.01\\
8.45	0.01\\
8.46	0.01\\
8.47	0.01\\
8.48	0.01\\
8.49	0.01\\
8.5	0.01\\
8.51	0.01\\
8.52	0.01\\
8.53	0.01\\
8.54	0.01\\
8.55	0.01\\
8.56	0.01\\
8.57	0.01\\
8.58	0.01\\
8.59	0.01\\
8.6	0.01\\
8.61	0.01\\
8.62	0.01\\
8.63	0.01\\
8.64	0.01\\
8.65	0.01\\
8.66	0.01\\
8.67	0.01\\
8.68	0.01\\
8.69	0.01\\
8.7	0.01\\
8.71	0.01\\
8.72	0.01\\
8.73	0.01\\
8.74	0.01\\
8.75	0.01\\
8.76	0.01\\
8.77	0.01\\
8.78	0.01\\
8.79	0.01\\
8.8	0.01\\
8.81	0.01\\
8.82	0.01\\
8.83	0.01\\
8.84	0.01\\
8.85	0.01\\
8.86	0.01\\
8.87	0.01\\
8.88	0.01\\
8.89	0.01\\
8.9	0.01\\
8.91	0.01\\
8.92	0.01\\
8.93	0.01\\
8.94	0.01\\
8.95	0.01\\
8.96	0.01\\
8.97	0.01\\
8.98	0.01\\
8.99	0.01\\
9	0.01\\
9.01	0.01\\
9.02	0.01\\
9.03	0.01\\
9.04	0.01\\
9.05	0.01\\
9.06	0.01\\
9.07	0.01\\
9.08	0.01\\
9.09	0.01\\
9.1	0.01\\
9.11	0.01\\
9.12	0.01\\
9.13	0.01\\
9.14	0.01\\
9.15	0.01\\
9.16	0.01\\
9.17	0.01\\
9.18	0.01\\
9.19	0.01\\
9.2	0.01\\
9.21	0.01\\
9.22	0.01\\
9.23	0.01\\
9.24	0.01\\
9.25	0.01\\
9.26	0.01\\
9.27	0.01\\
9.28	0.01\\
9.29	0.01\\
9.3	0.01\\
9.31	0.01\\
9.32	0.01\\
9.33	0.01\\
9.34	0.01\\
9.35	0.01\\
9.36	0.01\\
9.37	0.01\\
9.38	0.01\\
9.39	0.01\\
9.4	0.01\\
9.41	0.01\\
9.42	0.01\\
9.43	0.01\\
9.44	0.01\\
9.45	0.01\\
9.46	0.01\\
9.47	0.01\\
9.48	0.01\\
9.49	0.01\\
9.5	0.01\\
9.51	0.01\\
9.52	0.01\\
9.53	0.01\\
9.54	0.01\\
9.55	0.01\\
9.56	0.01\\
9.57	0.01\\
9.58	0.01\\
9.59	0.01\\
9.6	0.01\\
9.61	0.01\\
9.62	0.01\\
9.63	0.01\\
9.64	0.01\\
9.65	0.01\\
9.66	0.01\\
9.67	0.01\\
9.68	0.01\\
9.69	0.01\\
9.7	0.01\\
9.71	0.01\\
9.72	0.01\\
9.73	0.01\\
9.74	0.01\\
9.75	0.01\\
9.76	0.01\\
9.77	0.01\\
9.78	0.01\\
9.79	0.01\\
9.8	0.01\\
9.81	0.01\\
9.82	0.01\\
9.83	0.01\\
9.84	0.01\\
9.85	0.01\\
9.86	0.01\\
9.87	0.01\\
9.88	0.01\\
9.89	0.01\\
9.9	0.01\\
9.91	0.01\\
9.92	0.01\\
9.93	0.01\\
9.94	0.01\\
9.95	0.01\\
9.96	0.01\\
9.97	0.01\\
9.98	0.01\\
9.99	0.01\\
10	0.01\\
10.01	0.01\\
10.02	0.01\\
10.03	0.01\\
10.04	0.01\\
10.05	0.01\\
10.06	0.01\\
10.07	0.01\\
10.08	0.01\\
10.09	0.01\\
10.1	0.01\\
10.11	0.01\\
10.12	0.01\\
10.13	0.01\\
10.14	0.01\\
10.15	0.01\\
10.16	0.01\\
10.17	0.01\\
10.18	0.01\\
10.19	0.01\\
10.2	0.01\\
10.21	0.01\\
10.22	0.01\\
10.23	0.01\\
10.24	0.01\\
10.25	0.01\\
10.26	0.01\\
10.27	0.01\\
10.28	0.01\\
10.29	0.01\\
10.3	0.01\\
10.31	0.01\\
10.32	0.01\\
10.33	0.01\\
10.34	0.01\\
10.35	0.01\\
10.36	0.01\\
10.37	0.01\\
10.38	0.01\\
10.39	0.01\\
10.4	0.01\\
10.41	0.01\\
10.42	0.01\\
10.43	0.01\\
10.44	0.01\\
10.45	0.01\\
10.46	0.01\\
10.47	0.01\\
10.48	0.01\\
10.49	0.01\\
10.5	0.01\\
10.51	0.01\\
10.52	0.01\\
10.53	0.01\\
10.54	0.01\\
10.55	0.01\\
10.56	0.01\\
10.57	0.01\\
10.58	0.01\\
10.59	0.01\\
10.6	0.01\\
10.61	0.01\\
10.62	0.01\\
10.63	0.01\\
10.64	0.01\\
10.65	0.01\\
10.66	0.01\\
10.67	0.01\\
10.68	0.01\\
10.69	0.01\\
10.7	0.01\\
10.71	0.01\\
10.72	0.01\\
10.73	0.01\\
10.74	0.01\\
10.75	0.01\\
10.76	0.01\\
10.77	0.01\\
10.78	0.01\\
10.79	0.01\\
10.8	0.01\\
10.81	0.01\\
10.82	0.01\\
10.83	0.01\\
10.84	0.01\\
10.85	0.01\\
10.86	0.01\\
10.87	0.01\\
10.88	0.01\\
10.89	0.01\\
10.9	0.01\\
10.91	0.01\\
10.92	0.01\\
10.93	0.01\\
10.94	0.01\\
10.95	0.01\\
10.96	0.01\\
10.97	0.01\\
10.98	0.01\\
10.99	0.01\\
11	0.01\\
11.01	0.01\\
11.02	0.01\\
11.03	0.01\\
11.04	0.01\\
11.05	0.01\\
11.06	0.01\\
11.07	0.01\\
11.08	0.01\\
11.09	0.01\\
11.1	0.01\\
11.11	0.01\\
11.12	0.01\\
11.13	0.01\\
11.14	0.01\\
11.15	0.01\\
11.16	0.01\\
11.17	0.01\\
11.18	0.01\\
11.19	0.01\\
11.2	0.01\\
11.21	0.01\\
11.22	0.01\\
11.23	0.01\\
11.24	0.01\\
11.25	0.01\\
11.26	0.01\\
11.27	0.01\\
11.28	0.01\\
11.29	0.01\\
11.3	0.01\\
11.31	0.01\\
11.32	0.01\\
11.33	0.01\\
11.34	0.01\\
11.35	0.01\\
11.36	0.01\\
11.37	0.01\\
11.38	0.01\\
11.39	0.01\\
11.4	0.01\\
11.41	0.01\\
11.42	0.01\\
11.43	0.01\\
11.44	0.01\\
11.45	0.01\\
11.46	0.01\\
11.47	0.01\\
11.48	0.01\\
11.49	0.01\\
11.5	0.01\\
11.51	0.01\\
11.52	0.01\\
11.53	0.01\\
11.54	0.01\\
11.55	0.01\\
11.56	0.01\\
11.57	0.01\\
11.58	0.01\\
11.59	0.01\\
11.6	0.01\\
11.61	0.01\\
11.62	0.01\\
11.63	0.01\\
11.64	0.01\\
11.65	0.01\\
11.66	0.01\\
11.67	0.01\\
11.68	0.01\\
11.69	0.01\\
11.7	0.01\\
11.71	0.01\\
11.72	0.01\\
11.73	0.01\\
11.74	0.01\\
11.75	0.01\\
11.76	0.01\\
11.77	0.01\\
11.78	0.01\\
11.79	0.01\\
11.8	0.01\\
11.81	0.01\\
11.82	0.01\\
11.83	0.01\\
11.84	0.01\\
11.85	0.01\\
11.86	0.01\\
11.87	0.01\\
11.88	0.01\\
11.89	0.01\\
11.9	0.01\\
11.91	0.01\\
11.92	0.01\\
11.93	0.01\\
11.94	0.01\\
11.95	0.01\\
11.96	0.01\\
11.97	0.01\\
11.98	0.01\\
11.99	0.01\\
12	0.01\\
12.01	0.01\\
12.02	0.01\\
12.03	0.01\\
12.04	0.01\\
12.05	0.01\\
12.06	0.01\\
12.07	0.01\\
12.08	0.01\\
12.09	0.01\\
12.1	0.01\\
12.11	0.01\\
12.12	0.01\\
12.13	0.01\\
12.14	0.01\\
12.15	0.01\\
12.16	0.01\\
12.17	0.01\\
12.18	0.01\\
12.19	0.01\\
12.2	0.01\\
12.21	0.01\\
12.22	0.01\\
12.23	0.01\\
12.24	0.01\\
12.25	0.01\\
12.26	0.01\\
12.27	0.01\\
12.28	0.01\\
12.29	0.01\\
12.3	0.01\\
12.31	0.01\\
12.32	0.01\\
12.33	0.01\\
12.34	0.01\\
12.35	0.01\\
12.36	0.01\\
12.37	0.01\\
12.38	0.01\\
12.39	0.01\\
12.4	0.01\\
12.41	0.01\\
12.42	0.01\\
12.43	0.01\\
12.44	0.01\\
12.45	0.01\\
12.46	0.01\\
12.47	0.01\\
12.48	0.01\\
12.49	0.01\\
12.5	0.01\\
12.51	0.01\\
12.52	0.01\\
12.53	0.01\\
12.54	0.01\\
12.55	0.01\\
12.56	0.01\\
12.57	0.01\\
12.58	0.01\\
12.59	0.01\\
12.6	0.01\\
12.61	0.01\\
12.62	0.01\\
12.63	0.01\\
12.64	0.01\\
12.65	0.01\\
12.66	0.01\\
12.67	0.01\\
12.68	0.01\\
12.69	0.01\\
12.7	0.01\\
12.71	0.01\\
12.72	0.01\\
12.73	0.01\\
12.74	0.01\\
12.75	0.01\\
12.76	0.01\\
12.77	0.01\\
12.78	0.01\\
12.79	0.01\\
12.8	0.01\\
12.81	0.01\\
12.82	0.01\\
12.83	0.01\\
12.84	0.01\\
12.85	0.01\\
12.86	0.01\\
12.87	0.01\\
12.88	0.01\\
12.89	0.01\\
12.9	0.01\\
12.91	0.01\\
12.92	0.01\\
12.93	0.01\\
12.94	0.01\\
12.95	0.01\\
12.96	0.01\\
12.97	0.01\\
12.98	0.01\\
12.99	0.01\\
13	0.01\\
13.01	0.01\\
13.02	0.01\\
13.03	0.01\\
13.04	0.01\\
13.05	0.01\\
13.06	0.01\\
13.07	0.01\\
13.08	0.01\\
13.09	0.01\\
13.1	0.01\\
13.11	0.01\\
13.12	0.01\\
13.13	0.01\\
13.14	0.01\\
13.15	0.01\\
13.16	0.01\\
13.17	0.01\\
13.18	0.01\\
13.19	0.01\\
13.2	0.01\\
13.21	0.01\\
13.22	0.01\\
13.23	0.01\\
13.24	0.01\\
13.25	0.01\\
13.26	0.01\\
13.27	0.01\\
13.28	0.01\\
13.29	0.01\\
13.3	0.01\\
13.31	0.01\\
13.32	0.01\\
13.33	0.01\\
13.34	0.01\\
13.35	0.01\\
13.36	0.01\\
13.37	0.01\\
13.38	0.01\\
13.39	0.01\\
13.4	0.01\\
13.41	0.01\\
13.42	0.01\\
13.43	0.01\\
13.44	0.01\\
13.45	0.01\\
13.46	0.01\\
13.47	0.01\\
13.48	0.01\\
13.49	0.01\\
13.5	0.01\\
13.51	0.01\\
13.52	0.01\\
13.53	0.01\\
13.54	0.01\\
13.55	0.01\\
13.56	0.01\\
13.57	0.01\\
13.58	0.01\\
13.59	0.01\\
13.6	0.01\\
13.61	0.01\\
13.62	0.01\\
13.63	0.01\\
13.64	0.01\\
13.65	0.01\\
13.66	0.01\\
13.67	0.01\\
13.68	0.01\\
13.69	0.01\\
13.7	0.01\\
13.71	0.01\\
13.72	0.01\\
13.73	0.01\\
13.74	0.01\\
13.75	0.01\\
13.76	0.01\\
13.77	0.01\\
13.78	0.01\\
13.79	0.01\\
13.8	0.01\\
13.81	0.01\\
13.82	0.01\\
13.83	0.01\\
13.84	0.01\\
13.85	0.01\\
13.86	0.01\\
13.87	0.01\\
13.88	0.01\\
13.89	0.01\\
13.9	0.01\\
13.91	0.01\\
13.92	0.01\\
13.93	0.01\\
13.94	0.01\\
13.95	0.01\\
13.96	0.01\\
13.97	0.01\\
13.98	0.01\\
13.99	0.01\\
14	0.01\\
14.01	0.01\\
14.02	0.01\\
14.03	0.01\\
14.04	0.01\\
14.05	0.01\\
14.06	0.01\\
14.07	0.01\\
14.08	0.01\\
14.09	0.01\\
14.1	0.01\\
14.11	0.01\\
14.12	0.01\\
14.13	0.01\\
14.14	0.01\\
14.15	0.01\\
14.16	0.01\\
14.17	0.01\\
14.18	0.01\\
14.19	0.01\\
14.2	0.01\\
14.21	0.01\\
14.22	0.01\\
14.23	0.01\\
14.24	0.01\\
14.25	0.01\\
14.26	0.01\\
14.27	0.01\\
14.28	0.01\\
14.29	0.01\\
14.3	0.01\\
14.31	0.01\\
14.32	0.01\\
14.33	0.01\\
14.34	0.01\\
14.35	0.01\\
14.36	0.01\\
14.37	0.01\\
14.38	0.01\\
14.39	0.01\\
14.4	0.01\\
14.41	0.01\\
14.42	0.01\\
14.43	0.01\\
14.44	0.01\\
14.45	0.01\\
14.46	0.01\\
14.47	0.01\\
14.48	0.01\\
14.49	0.01\\
14.5	0.01\\
14.51	0.01\\
14.52	0.01\\
14.53	0.01\\
14.54	0.01\\
14.55	0.01\\
14.56	0.01\\
14.57	0.01\\
14.58	0.01\\
14.59	0.01\\
14.6	0.01\\
14.61	0.01\\
14.62	0.01\\
14.63	0.01\\
14.64	0.01\\
14.65	0.01\\
14.66	0.01\\
14.67	0.01\\
14.68	0.01\\
14.69	0.01\\
14.7	0.01\\
14.71	0.01\\
14.72	0.01\\
14.73	0.01\\
14.74	0.01\\
14.75	0.01\\
14.76	0.01\\
14.77	0.01\\
14.78	0.01\\
14.79	0.01\\
14.8	0.01\\
14.81	0.01\\
14.82	0.01\\
14.83	0.01\\
14.84	0.01\\
14.85	0.01\\
14.86	0.01\\
14.87	0.01\\
14.88	0.01\\
14.89	0.01\\
14.9	0.01\\
14.91	0.01\\
14.92	0.01\\
14.93	0.01\\
14.94	0.01\\
14.95	0.01\\
14.96	0.01\\
14.97	0.01\\
14.98	0.01\\
14.99	0.01\\
15	0.01\\
15.01	0.01\\
15.02	0.01\\
15.03	0.01\\
15.04	0.01\\
15.05	0.01\\
15.06	0.01\\
15.07	0.01\\
15.08	0.01\\
15.09	0.01\\
15.1	0.01\\
15.11	0.01\\
15.12	0.01\\
15.13	0.01\\
15.14	0.01\\
15.15	0.01\\
15.16	0.01\\
15.17	0.01\\
15.18	0.01\\
15.19	0.01\\
15.2	0.01\\
15.21	0.01\\
15.22	0.01\\
15.23	0.01\\
15.24	0.01\\
15.25	0.01\\
15.26	0.01\\
15.27	0.01\\
15.28	0.01\\
15.29	0.01\\
15.3	0.01\\
15.31	0.01\\
15.32	0.01\\
15.33	0.01\\
15.34	0.01\\
15.35	0.01\\
15.36	0.01\\
15.37	0.01\\
15.38	0.01\\
15.39	0.01\\
15.4	0.01\\
15.41	0.01\\
15.42	0.01\\
15.43	0.01\\
15.44	0.01\\
15.45	0.01\\
15.46	0.01\\
15.47	0.01\\
15.48	0.01\\
15.49	0.01\\
15.5	0.01\\
15.51	0.01\\
15.52	0.01\\
15.53	0.01\\
15.54	0.01\\
15.55	0.01\\
15.56	0.01\\
15.57	0.01\\
15.58	0.01\\
15.59	0.01\\
15.6	0.01\\
15.61	0.01\\
15.62	0.01\\
15.63	0.01\\
15.64	0.01\\
15.65	0.01\\
15.66	0.01\\
15.67	0.01\\
15.68	0.01\\
15.69	0.01\\
15.7	0.01\\
15.71	0.01\\
15.72	0.01\\
15.73	0.01\\
15.74	0.01\\
15.75	0.01\\
15.76	0.01\\
15.77	0.01\\
15.78	0.01\\
15.79	0.01\\
15.8	0.01\\
15.81	0.01\\
15.82	0.01\\
15.83	0.01\\
15.84	0.01\\
15.85	0.01\\
15.86	0.01\\
15.87	0.01\\
15.88	0.01\\
15.89	0.01\\
15.9	0.01\\
15.91	0.01\\
15.92	0.01\\
15.93	0.01\\
15.94	0.01\\
15.95	0.01\\
15.96	0.01\\
15.97	0.01\\
15.98	0.01\\
15.99	0.01\\
16	0.01\\
16.01	0.01\\
16.02	0.01\\
16.03	0.01\\
16.04	0.01\\
16.05	0.01\\
16.06	0.01\\
16.07	0.01\\
16.08	0.01\\
16.09	0.01\\
16.1	0.01\\
16.11	0.01\\
16.12	0.01\\
16.13	0.01\\
16.14	0.01\\
16.15	0.01\\
16.16	0.01\\
16.17	0.01\\
16.18	0.01\\
16.19	0.01\\
16.2	0.01\\
16.21	0.01\\
16.22	0.01\\
16.23	0.01\\
16.24	0.01\\
16.25	0.01\\
16.26	0.01\\
16.27	0.01\\
16.28	0.01\\
16.29	0.01\\
16.3	0.01\\
16.31	0.01\\
16.32	0.01\\
16.33	0.01\\
16.34	0.01\\
16.35	0.01\\
16.36	0.01\\
16.37	0.01\\
16.38	0.01\\
16.39	0.01\\
16.4	0.01\\
16.41	0.01\\
16.42	0.01\\
16.43	0.01\\
16.44	0.01\\
16.45	0.01\\
16.46	0.01\\
16.47	0.01\\
16.48	0.01\\
16.49	0.01\\
16.5	0.01\\
16.51	0.01\\
16.52	0.01\\
16.53	0.01\\
16.54	0.01\\
16.55	0.01\\
16.56	0.01\\
16.57	0.01\\
16.58	0.01\\
16.59	0.01\\
16.6	0.01\\
16.61	0.01\\
16.62	0.01\\
16.63	0.01\\
16.64	0.01\\
16.65	0.01\\
16.66	0.01\\
16.67	0.01\\
16.68	0.01\\
16.69	0.01\\
16.7	0.01\\
16.71	0.01\\
16.72	0.01\\
16.73	0.01\\
16.74	0.01\\
16.75	0.01\\
16.76	0.01\\
16.77	0.01\\
16.78	0.01\\
16.79	0.01\\
16.8	0.01\\
16.81	0.01\\
16.82	0.01\\
16.83	0.01\\
16.84	0.01\\
16.85	0.01\\
16.86	0.01\\
16.87	0.01\\
16.88	0.01\\
16.89	0.01\\
16.9	0.01\\
16.91	0.01\\
16.92	0.01\\
16.93	0.01\\
16.94	0.01\\
16.95	0.01\\
16.96	0.01\\
16.97	0.01\\
16.98	0.01\\
16.99	0.01\\
17	0.01\\
17.01	0.01\\
17.02	0.01\\
17.03	0.01\\
17.04	0.01\\
17.05	0.01\\
17.06	0.01\\
17.07	0.01\\
17.08	0.01\\
17.09	0.01\\
17.1	0.01\\
17.11	0.01\\
17.12	0.01\\
17.13	0.01\\
17.14	0.01\\
17.15	0.01\\
17.16	0.01\\
17.17	0.01\\
17.18	0.01\\
17.19	0.01\\
17.2	0.01\\
17.21	0.01\\
17.22	0.01\\
17.23	0.01\\
17.24	0.01\\
17.25	0.01\\
17.26	0.01\\
17.27	0.01\\
17.28	0.01\\
17.29	0.01\\
17.3	0.01\\
17.31	0.01\\
17.32	0.01\\
17.33	0.01\\
17.34	0.01\\
17.35	0.01\\
17.36	0.01\\
17.37	0.01\\
17.38	0.01\\
17.39	0.01\\
17.4	0.01\\
17.41	0.01\\
17.42	0.01\\
17.43	0.01\\
17.44	0.01\\
17.45	0.01\\
17.46	0.01\\
17.47	0.01\\
17.48	0.01\\
17.49	0.01\\
17.5	0.01\\
17.51	0.01\\
17.52	0.01\\
17.53	0.01\\
17.54	0.01\\
17.55	0.01\\
17.56	0.01\\
17.57	0.01\\
17.58	0.01\\
17.59	0.01\\
17.6	0.01\\
17.61	0.01\\
17.62	0.01\\
17.63	0.01\\
17.64	0.01\\
17.65	0.01\\
17.66	0.01\\
17.67	0.01\\
17.68	0.01\\
17.69	0.01\\
17.7	0.01\\
17.71	0.01\\
17.72	0.01\\
17.73	0.01\\
17.74	0.01\\
17.75	0.01\\
17.76	0.01\\
17.77	0.01\\
17.78	0.01\\
17.79	0.01\\
17.8	0.01\\
17.81	0.01\\
17.82	0.01\\
17.83	0.01\\
17.84	0.01\\
17.85	0.01\\
17.86	0.01\\
17.87	0.01\\
17.88	0.01\\
17.89	0.01\\
17.9	0.01\\
17.91	0.01\\
17.92	0.01\\
17.93	0.01\\
17.94	0.01\\
17.95	0.01\\
17.96	0.01\\
17.97	0.01\\
17.98	0.01\\
17.99	0.01\\
18	0.01\\
18.01	0.01\\
18.02	0.01\\
18.03	0.01\\
18.04	0.01\\
18.05	0.01\\
18.06	0.01\\
18.07	0.01\\
18.08	0.01\\
18.09	0.01\\
18.1	0.01\\
18.11	0.01\\
18.12	0.01\\
18.13	0.01\\
18.14	0.01\\
18.15	0.01\\
18.16	0.01\\
18.17	0.01\\
18.18	0.01\\
18.19	0.01\\
18.2	0.01\\
18.21	0.01\\
18.22	0.01\\
18.23	0.01\\
18.24	0.01\\
18.25	0.01\\
18.26	0.01\\
18.27	0.01\\
18.28	0.01\\
18.29	0.01\\
18.3	0.01\\
18.31	0.01\\
18.32	0.01\\
18.33	0.01\\
18.34	0.01\\
18.35	0.01\\
18.36	0.01\\
18.37	0.01\\
18.38	0.01\\
18.39	0.01\\
18.4	0.01\\
18.41	0.01\\
18.42	0.01\\
18.43	0.01\\
18.44	0.01\\
18.45	0.01\\
18.46	0.01\\
18.47	0.01\\
18.48	0.01\\
18.49	0.01\\
18.5	0.01\\
18.51	0.01\\
18.52	0.01\\
18.53	0.01\\
18.54	0.01\\
18.55	0.01\\
18.56	0.01\\
18.57	0.01\\
18.58	0.01\\
18.59	0.01\\
18.6	0.01\\
18.61	0.01\\
18.62	0.01\\
18.63	0.01\\
18.64	0.01\\
18.65	0.01\\
18.66	0.01\\
18.67	0.01\\
18.68	0.01\\
18.69	0.01\\
18.7	0.01\\
18.71	0.01\\
18.72	0.01\\
18.73	0.01\\
18.74	0.01\\
18.75	0.01\\
18.76	0.01\\
18.77	0.01\\
18.78	0.01\\
18.79	0.01\\
18.8	0.01\\
18.81	0.01\\
18.82	0.01\\
18.83	0.01\\
18.84	0.01\\
18.85	0.01\\
18.86	0.01\\
18.87	0.01\\
18.88	0.01\\
18.89	0.01\\
18.9	0.01\\
18.91	0.01\\
18.92	0.01\\
18.93	0.01\\
18.94	0.01\\
18.95	0.01\\
18.96	0.01\\
18.97	0.01\\
18.98	0.01\\
18.99	0.01\\
19	0.01\\
19.01	0.01\\
19.02	0.01\\
19.03	0.01\\
19.04	0.01\\
19.05	0.01\\
19.06	0.01\\
19.07	0.01\\
19.08	0.01\\
19.09	0.01\\
19.1	0.01\\
19.11	0.01\\
19.12	0.01\\
19.13	0.01\\
19.14	0.01\\
19.15	0.01\\
19.16	0.01\\
19.17	0.01\\
19.18	0.01\\
19.19	0.01\\
19.2	0.01\\
19.21	0.01\\
19.22	0.01\\
19.23	0.01\\
19.24	0.01\\
19.25	0.01\\
19.26	0.01\\
19.27	0.01\\
19.28	0.01\\
19.29	0.01\\
19.3	0.01\\
19.31	0.01\\
19.32	0.01\\
19.33	0.01\\
19.34	0.01\\
19.35	0.01\\
19.36	0.01\\
19.37	0.01\\
19.38	0.01\\
19.39	0.01\\
19.4	0.01\\
19.41	0.01\\
19.42	0.01\\
19.43	0.01\\
19.44	0.01\\
19.45	0.01\\
19.46	0.01\\
19.47	0.01\\
19.48	0.01\\
19.49	0.01\\
19.5	0.01\\
19.51	0.01\\
19.52	0.01\\
19.53	0.01\\
19.54	0.01\\
19.55	0.01\\
19.56	0.01\\
19.57	0.01\\
19.58	0.01\\
19.59	0.01\\
19.6	0.01\\
19.61	0.01\\
19.62	0.01\\
19.63	0.01\\
19.64	0.01\\
19.65	0.01\\
19.66	0.01\\
19.67	0.01\\
19.68	0.01\\
19.69	0.01\\
19.7	0.01\\
19.71	0.01\\
19.72	0.01\\
19.73	0.01\\
19.74	0.01\\
19.75	0.01\\
19.76	0.01\\
19.77	0.01\\
19.78	0.01\\
19.79	0.01\\
19.8	0.01\\
19.81	0.01\\
19.82	0.01\\
19.83	0.01\\
19.84	0.01\\
19.85	0.01\\
19.86	0.01\\
19.87	0.01\\
19.88	0.01\\
19.89	0.01\\
19.9	0.01\\
19.91	0.01\\
19.92	0.01\\
19.93	0.01\\
19.94	0.01\\
19.95	0.01\\
19.96	0.01\\
19.97	0.01\\
19.98	0.01\\
19.99	0.01\\
20	0.01\\
20.01	0.01\\
20.02	0.01\\
20.03	0.01\\
20.04	0.01\\
20.05	0.01\\
20.06	0.01\\
20.07	0.01\\
20.08	0.01\\
20.09	0.01\\
20.1	0.01\\
20.11	0.01\\
20.12	0.01\\
20.13	0.01\\
20.14	0.01\\
20.15	0.01\\
20.16	0.01\\
20.17	0.01\\
20.18	0.01\\
20.19	0.01\\
20.2	0.01\\
20.21	0.01\\
20.22	0.01\\
20.23	0.01\\
20.24	0.01\\
20.25	0.01\\
20.26	0.01\\
20.27	0.01\\
20.28	0.01\\
20.29	0.01\\
20.3	0.01\\
20.31	0.01\\
20.32	0.01\\
20.33	0.01\\
20.34	0.01\\
20.35	0.01\\
20.36	0.01\\
20.37	0.01\\
20.38	0.01\\
20.39	0.01\\
20.4	0.01\\
20.41	0.01\\
20.42	0.01\\
20.43	0.01\\
20.44	0.01\\
20.45	0.01\\
20.46	0.01\\
20.47	0.01\\
20.48	0.01\\
20.49	0.01\\
20.5	0.01\\
20.51	0.01\\
20.52	0.01\\
20.53	0.01\\
20.54	0.01\\
20.55	0.01\\
20.56	0.01\\
20.57	0.01\\
20.58	0.01\\
20.59	0.01\\
20.6	0.01\\
20.61	0.01\\
20.62	0.01\\
20.63	0.01\\
20.64	0.01\\
20.65	0.01\\
20.66	0.01\\
20.67	0.01\\
20.68	0.01\\
20.69	0.01\\
20.7	0.01\\
20.71	0.01\\
20.72	0.01\\
20.73	0.01\\
20.74	0.01\\
20.75	0.01\\
20.76	0.01\\
20.77	0.01\\
20.78	0.01\\
20.79	0.01\\
20.8	0.01\\
20.81	0.01\\
20.82	0.01\\
20.83	0.01\\
20.84	0.01\\
20.85	0.01\\
20.86	0.01\\
20.87	0.01\\
20.88	0.01\\
20.89	0.01\\
20.9	0.01\\
20.91	0.01\\
20.92	0.01\\
20.93	0.01\\
20.94	0.01\\
20.95	0.01\\
20.96	0.01\\
20.97	0.01\\
20.98	0.01\\
20.99	0.01\\
21	0.01\\
21.01	0.01\\
21.02	0.01\\
21.03	0.01\\
21.04	0.01\\
21.05	0.01\\
21.06	0.01\\
21.07	0.01\\
21.08	0.01\\
21.09	0.01\\
21.1	0.01\\
21.11	0.01\\
21.12	0.01\\
21.13	0.01\\
21.14	0.01\\
21.15	0.01\\
21.16	0.01\\
21.17	0.01\\
21.18	0.01\\
21.19	0.01\\
21.2	0.01\\
21.21	0.01\\
21.22	0.01\\
21.23	0.01\\
21.24	0.01\\
21.25	0.01\\
21.26	0.01\\
21.27	0.01\\
21.28	0.01\\
21.29	0.01\\
21.3	0.01\\
21.31	0.01\\
21.32	0.01\\
21.33	0.01\\
21.34	0.01\\
21.35	0.01\\
21.36	0.01\\
21.37	0.01\\
21.38	0.01\\
21.39	0.01\\
21.4	0.01\\
21.41	0.01\\
21.42	0.01\\
21.43	0.01\\
21.44	0.01\\
21.45	0.01\\
21.46	0.01\\
21.47	0.01\\
21.48	0.01\\
21.49	0.01\\
21.5	0.01\\
21.51	0.01\\
21.52	0.01\\
21.53	0.01\\
21.54	0.01\\
21.55	0.01\\
21.56	0.01\\
21.57	0.01\\
21.58	0.01\\
21.59	0.01\\
21.6	0.01\\
21.61	0.01\\
21.62	0.01\\
21.63	0.01\\
21.64	0.01\\
21.65	0.01\\
21.66	0.01\\
21.67	0.01\\
21.68	0.01\\
21.69	0.01\\
21.7	0.01\\
21.71	0.01\\
21.72	0.01\\
21.73	0.01\\
21.74	0.01\\
21.75	0.01\\
21.76	0.01\\
21.77	0.01\\
21.78	0.01\\
21.79	0.01\\
21.8	0.01\\
21.81	0.01\\
21.82	0.01\\
21.83	0.01\\
21.84	0.01\\
21.85	0.01\\
21.86	0.01\\
21.87	0.01\\
21.88	0.01\\
21.89	0.01\\
21.9	0.01\\
21.91	0.01\\
21.92	0.01\\
21.93	0.01\\
21.94	0.01\\
21.95	0.01\\
21.96	0.01\\
21.97	0.01\\
21.98	0.01\\
21.99	0.01\\
22	0.01\\
22.01	0.01\\
22.02	0.01\\
22.03	0.01\\
22.04	0.01\\
22.05	0.01\\
22.06	0.01\\
22.07	0.01\\
22.08	0.01\\
22.09	0.01\\
22.1	0.01\\
22.11	0.01\\
22.12	0.01\\
22.13	0.01\\
22.14	0.01\\
22.15	0.01\\
22.16	0.01\\
22.17	0.01\\
22.18	0.01\\
22.19	0.01\\
22.2	0.01\\
22.21	0.01\\
22.22	0.01\\
22.23	0.01\\
22.24	0.01\\
22.25	0.01\\
22.26	0.01\\
22.27	0.01\\
22.28	0.01\\
22.29	0.01\\
22.3	0.01\\
22.31	0.01\\
22.32	0.01\\
22.33	0.01\\
22.34	0.01\\
22.35	0.01\\
22.36	0.01\\
22.37	0.01\\
22.38	0.01\\
22.39	0.01\\
22.4	0.01\\
22.41	0.01\\
22.42	0.01\\
22.43	0.01\\
22.44	0.01\\
22.45	0.01\\
22.46	0.01\\
22.47	0.01\\
22.48	0.01\\
22.49	0.01\\
22.5	0.01\\
22.51	0.01\\
22.52	0.01\\
22.53	0.01\\
22.54	0.01\\
22.55	0.01\\
22.56	0.01\\
22.57	0.01\\
22.58	0.01\\
22.59	0.01\\
22.6	0.01\\
22.61	0.01\\
22.62	0.01\\
22.63	0.01\\
22.64	0.01\\
22.65	0.01\\
22.66	0.01\\
22.67	0.01\\
22.68	0.01\\
22.69	0.01\\
22.7	0.01\\
22.71	0.01\\
22.72	0.01\\
22.73	0.01\\
22.74	0.01\\
22.75	0.01\\
22.76	0.01\\
22.77	0.01\\
22.78	0.01\\
22.79	0.01\\
22.8	0.01\\
22.81	0.01\\
22.82	0.01\\
22.83	0.01\\
22.84	0.01\\
22.85	0.01\\
22.86	0.01\\
22.87	0.01\\
22.88	0.01\\
22.89	0.01\\
22.9	0.01\\
22.91	0.01\\
22.92	0.01\\
22.93	0.01\\
22.94	0.01\\
22.95	0.01\\
22.96	0.01\\
22.97	0.01\\
22.98	0.01\\
22.99	0.01\\
23	0.01\\
23.01	0.01\\
23.02	0.01\\
23.03	0.01\\
23.04	0.01\\
23.05	0.01\\
23.06	0.01\\
23.07	0.01\\
23.08	0.01\\
23.09	0.01\\
23.1	0.01\\
23.11	0.01\\
23.12	0.01\\
23.13	0.01\\
23.14	0.01\\
23.15	0.01\\
23.16	0.01\\
23.17	0.01\\
23.18	0.01\\
23.19	0.01\\
23.2	0.01\\
23.21	0.01\\
23.22	0.01\\
23.23	0.01\\
23.24	0.01\\
23.25	0.01\\
23.26	0.01\\
23.27	0.01\\
23.28	0.01\\
23.29	0.01\\
23.3	0.01\\
23.31	0.01\\
23.32	0.01\\
23.33	0.01\\
23.34	0.01\\
23.35	0.01\\
23.36	0.01\\
23.37	0.01\\
23.38	0.01\\
23.39	0.01\\
23.4	0.01\\
23.41	0.01\\
23.42	0.01\\
23.43	0.01\\
23.44	0.01\\
23.45	0.01\\
23.46	0.01\\
23.47	0.01\\
23.48	0.01\\
23.49	0.01\\
23.5	0.01\\
23.51	0.01\\
23.52	0.01\\
23.53	0.01\\
23.54	0.01\\
23.55	0.01\\
23.56	0.01\\
23.57	0.01\\
23.58	0.01\\
23.59	0.01\\
23.6	0.01\\
23.61	0.01\\
23.62	0.01\\
23.63	0.01\\
23.64	0.01\\
23.65	0.01\\
23.66	0.01\\
23.67	0.01\\
23.68	0.01\\
23.69	0.01\\
23.7	0.01\\
23.71	0.01\\
23.72	0.01\\
23.73	0.01\\
23.74	0.01\\
23.75	0.01\\
23.76	0.01\\
23.77	0.01\\
23.78	0.01\\
23.79	0.01\\
23.8	0.01\\
23.81	0.01\\
23.82	0.01\\
23.83	0.01\\
23.84	0.01\\
23.85	0.01\\
23.86	0.01\\
23.87	0.01\\
23.88	0.01\\
23.89	0.01\\
23.9	0.01\\
23.91	0.01\\
23.92	0.01\\
23.93	0.01\\
23.94	0.01\\
23.95	0.01\\
23.96	0.01\\
23.97	0.01\\
23.98	0.01\\
23.99	0.01\\
24	0.01\\
24.01	0.01\\
24.02	0.01\\
24.03	0.01\\
24.04	0.01\\
24.05	0.01\\
24.06	0.01\\
24.07	0.01\\
24.08	0.01\\
24.09	0.01\\
24.1	0.01\\
24.11	0.01\\
24.12	0.01\\
24.13	0.01\\
24.14	0.01\\
24.15	0.01\\
24.16	0.01\\
24.17	0.01\\
24.18	0.01\\
24.19	0.01\\
24.2	0.01\\
24.21	0.01\\
24.22	0.01\\
24.23	0.01\\
24.24	0.01\\
24.25	0.01\\
24.26	0.01\\
24.27	0.01\\
24.28	0.01\\
24.29	0.01\\
24.3	0.01\\
24.31	0.01\\
24.32	0.01\\
24.33	0.01\\
24.34	0.01\\
24.35	0.01\\
24.36	0.01\\
24.37	0.01\\
24.38	0.01\\
24.39	0.01\\
24.4	0.01\\
24.41	0.01\\
24.42	0.01\\
24.43	0.01\\
24.44	0.01\\
24.45	0.01\\
24.46	0.01\\
24.47	0.01\\
24.48	0.01\\
24.49	0.01\\
24.5	0.01\\
24.51	0.01\\
24.52	0.01\\
24.53	0.01\\
24.54	0.01\\
24.55	0.01\\
24.56	0.01\\
24.57	0.01\\
24.58	0.01\\
24.59	0.01\\
24.6	0.01\\
24.61	0.01\\
24.62	0.01\\
24.63	0.01\\
24.64	0.01\\
24.65	0.01\\
24.66	0.01\\
24.67	0.01\\
24.68	0.01\\
24.69	0.01\\
24.7	0.01\\
24.71	0.01\\
24.72	0.01\\
24.73	0.01\\
24.74	0.01\\
24.75	0.01\\
24.76	0.01\\
24.77	0.01\\
24.78	0.01\\
24.79	0.01\\
24.8	0.01\\
24.81	0.01\\
24.82	0.01\\
24.83	0.01\\
24.84	0.01\\
24.85	0.01\\
24.86	0.01\\
24.87	0.01\\
24.88	0.01\\
24.89	0.01\\
24.9	0.01\\
24.91	0.01\\
24.92	0.01\\
24.93	0.01\\
24.94	0.01\\
24.95	0.01\\
24.96	0.01\\
24.97	0.01\\
24.98	0.01\\
24.99	0.01\\
25	0.01\\
25.01	0.01\\
25.02	0.01\\
25.03	0.01\\
25.04	0.01\\
25.05	0.01\\
25.06	0.01\\
25.07	0.01\\
25.08	0.01\\
25.09	0.01\\
25.1	0.01\\
25.11	0.01\\
25.12	0.01\\
25.13	0.01\\
25.14	0.01\\
25.15	0.01\\
25.16	0.01\\
25.17	0.01\\
25.18	0.01\\
25.19	0.01\\
25.2	0.01\\
25.21	0.01\\
25.22	0.01\\
25.23	0.01\\
25.24	0.01\\
25.25	0.01\\
25.26	0.01\\
25.27	0.01\\
25.28	0.01\\
25.29	0.01\\
25.3	0.01\\
25.31	0.01\\
25.32	0.01\\
25.33	0.01\\
25.34	0.01\\
25.35	0.01\\
25.36	0.01\\
25.37	0.01\\
25.38	0.01\\
25.39	0.01\\
25.4	0.01\\
25.41	0.01\\
25.42	0.01\\
25.43	0.01\\
25.44	0.01\\
25.45	0.01\\
25.46	0.01\\
25.47	0.01\\
25.48	0.01\\
25.49	0.01\\
25.5	0.01\\
25.51	0.01\\
25.52	0.01\\
25.53	0.01\\
25.54	0.01\\
25.55	0.01\\
25.56	0.01\\
25.57	0.01\\
25.58	0.01\\
25.59	0.01\\
25.6	0.01\\
25.61	0.01\\
25.62	0.01\\
25.63	0.01\\
25.64	0.01\\
25.65	0.01\\
25.66	0.01\\
25.67	0.01\\
25.68	0.01\\
25.69	0.01\\
25.7	0.01\\
25.71	0.01\\
25.72	0.01\\
25.73	0.01\\
25.74	0.01\\
25.75	0.01\\
25.76	0.01\\
25.77	0.01\\
25.78	0.01\\
25.79	0.01\\
25.8	0.01\\
25.81	0.01\\
25.82	0.01\\
25.83	0.01\\
25.84	0.01\\
25.85	0.01\\
25.86	0.01\\
25.87	0.01\\
25.88	0.01\\
25.89	0.01\\
25.9	0.01\\
25.91	0.01\\
25.92	0.01\\
25.93	0.01\\
25.94	0.01\\
25.95	0.01\\
25.96	0.01\\
25.97	0.01\\
25.98	0.01\\
25.99	0.01\\
26	0.01\\
26.01	0.01\\
26.02	0.01\\
26.03	0.01\\
26.04	0.01\\
26.05	0.01\\
26.06	0.01\\
26.07	0.01\\
26.08	0.01\\
26.09	0.01\\
26.1	0.01\\
26.11	0.01\\
26.12	0.01\\
26.13	0.01\\
26.14	0.01\\
26.15	0.01\\
26.16	0.01\\
26.17	0.01\\
26.18	0.01\\
26.19	0.01\\
26.2	0.01\\
26.21	0.01\\
26.22	0.01\\
26.23	0.01\\
26.24	0.01\\
26.25	0.01\\
26.26	0.01\\
26.27	0.01\\
26.28	0.01\\
26.29	0.01\\
26.3	0.01\\
26.31	0.01\\
26.32	0.01\\
26.33	0.01\\
26.34	0.01\\
26.35	0.01\\
26.36	0.01\\
26.37	0.01\\
26.38	0.01\\
26.39	0.01\\
26.4	0.01\\
26.41	0.01\\
26.42	0.01\\
26.43	0.01\\
26.44	0.01\\
26.45	0.01\\
26.46	0.01\\
26.47	0.01\\
26.48	0.01\\
26.49	0.01\\
26.5	0.01\\
26.51	0.01\\
26.52	0.01\\
26.53	0.01\\
26.54	0.01\\
26.55	0.01\\
26.56	0.01\\
26.57	0.01\\
26.58	0.01\\
26.59	0.01\\
26.6	0.01\\
26.61	0.01\\
26.62	0.01\\
26.63	0.01\\
26.64	0.01\\
26.65	0.01\\
26.66	0.01\\
26.67	0.01\\
26.68	0.01\\
26.69	0.01\\
26.7	0.01\\
26.71	0.01\\
26.72	0.01\\
26.73	0.01\\
26.74	0.01\\
26.75	0.01\\
26.76	0.01\\
26.77	0.01\\
26.78	0.01\\
26.79	0.01\\
26.8	0.01\\
26.81	0.01\\
26.82	0.01\\
26.83	0.01\\
26.84	0.01\\
26.85	0.01\\
26.86	0.01\\
26.87	0.01\\
26.88	0.01\\
26.89	0.01\\
26.9	0.01\\
26.91	0.01\\
26.92	0.01\\
26.93	0.01\\
26.94	0.01\\
26.95	0.01\\
26.96	0.01\\
26.97	0.01\\
26.98	0.01\\
26.99	0.01\\
27	0.01\\
27.01	0.01\\
27.02	0.01\\
27.03	0.01\\
27.04	0.01\\
27.05	0.01\\
27.06	0.01\\
27.07	0.01\\
27.08	0.01\\
27.09	0.01\\
27.1	0.01\\
27.11	0.01\\
27.12	0.01\\
27.13	0.01\\
27.14	0.01\\
27.15	0.01\\
27.16	0.01\\
27.17	0.01\\
27.18	0.01\\
27.19	0.01\\
27.2	0.01\\
27.21	0.01\\
27.22	0.01\\
27.23	0.01\\
27.24	0.01\\
27.25	0.01\\
27.26	0.01\\
27.27	0.01\\
27.28	0.01\\
27.29	0.01\\
27.3	0.01\\
27.31	0.01\\
27.32	0.01\\
27.33	0.01\\
27.34	0.01\\
27.35	0.01\\
27.36	0.01\\
27.37	0.01\\
27.38	0.01\\
27.39	0.01\\
27.4	0.01\\
27.41	0.01\\
27.42	0.01\\
27.43	0.01\\
27.44	0.01\\
27.45	0.01\\
27.46	0.01\\
27.47	0.01\\
27.48	0.01\\
27.49	0.01\\
27.5	0.01\\
27.51	0.01\\
27.52	0.01\\
27.53	0.01\\
27.54	0.01\\
27.55	0.01\\
27.56	0.01\\
27.57	0.01\\
27.58	0.01\\
27.59	0.01\\
27.6	0.01\\
27.61	0.01\\
27.62	0.01\\
27.63	0.01\\
27.64	0.01\\
27.65	0.01\\
27.66	0.01\\
27.67	0.01\\
27.68	0.01\\
27.69	0.01\\
27.7	0.01\\
27.71	0.01\\
27.72	0.01\\
27.73	0.01\\
27.74	0.01\\
27.75	0.01\\
27.76	0.01\\
27.77	0.01\\
27.78	0.01\\
27.79	0.01\\
27.8	0.01\\
27.81	0.01\\
27.82	0.01\\
27.83	0.01\\
27.84	0.01\\
27.85	0.01\\
27.86	0.01\\
27.87	0.01\\
27.88	0.01\\
27.89	0.01\\
27.9	0.01\\
27.91	0.01\\
27.92	0.01\\
27.93	0.01\\
27.94	0.01\\
27.95	0.01\\
27.96	0.01\\
27.97	0.01\\
27.98	0.01\\
27.99	0.01\\
28	0.01\\
28.01	0.01\\
28.02	0.01\\
28.03	0.01\\
28.04	0.01\\
28.05	0.01\\
28.06	0.01\\
28.07	0.01\\
28.08	0.01\\
28.09	0.01\\
28.1	0.01\\
28.11	0.01\\
28.12	0.01\\
28.13	0.01\\
28.14	0.01\\
28.15	0.01\\
28.16	0.01\\
28.17	0.01\\
28.18	0.01\\
28.19	0.01\\
28.2	0.01\\
28.21	0.01\\
28.22	0.01\\
28.23	0.01\\
28.24	0.01\\
28.25	0.01\\
28.26	0.01\\
28.27	0.01\\
28.28	0.01\\
28.29	0.01\\
28.3	0.01\\
28.31	0.01\\
28.32	0.01\\
28.33	0.01\\
28.34	0.01\\
28.35	0.01\\
28.36	0.01\\
28.37	0.01\\
28.38	0.01\\
28.39	0.01\\
28.4	0.01\\
28.41	0.01\\
28.42	0.01\\
28.43	0.01\\
28.44	0.01\\
28.45	0.01\\
28.46	0.01\\
28.47	0.01\\
28.48	0.01\\
28.49	0.01\\
28.5	0.01\\
28.51	0.01\\
28.52	0.01\\
28.53	0.01\\
28.54	0.01\\
28.55	0.01\\
28.56	0.01\\
28.57	0.01\\
28.58	0.01\\
28.59	0.01\\
28.6	0.01\\
28.61	0.01\\
28.62	0.01\\
28.63	0.01\\
28.64	0.01\\
28.65	0.01\\
28.66	0.01\\
28.67	0.01\\
28.68	0.01\\
28.69	0.01\\
28.7	0.01\\
28.71	0.01\\
28.72	0.01\\
28.73	0.01\\
28.74	0.01\\
28.75	0.01\\
28.76	0.01\\
28.77	0.01\\
28.78	0.01\\
28.79	0.01\\
28.8	0.01\\
28.81	0.01\\
28.82	0.01\\
28.83	0.01\\
28.84	0.01\\
28.85	0.01\\
28.86	0.01\\
28.87	0.01\\
28.88	0.01\\
28.89	0.01\\
28.9	0.01\\
28.91	0.01\\
28.92	0.01\\
28.93	0.01\\
28.94	0.01\\
28.95	0.01\\
28.96	0.01\\
28.97	0.01\\
28.98	0.01\\
28.99	0.01\\
29	0.01\\
29.01	0.01\\
29.02	0.01\\
29.03	0.01\\
29.04	0.01\\
29.05	0.01\\
29.06	0.01\\
29.07	0.01\\
29.08	0.01\\
29.09	0.01\\
29.1	0.01\\
29.11	0.01\\
29.12	0.01\\
29.13	0.01\\
29.14	0.01\\
29.15	0.01\\
29.16	0.01\\
29.17	0.01\\
29.18	0.01\\
29.19	0.01\\
29.2	0.01\\
29.21	0.01\\
29.22	0.01\\
29.23	0.01\\
29.24	0.01\\
29.25	0.01\\
29.26	0.01\\
29.27	0.01\\
29.28	0.01\\
29.29	0.01\\
29.3	0.01\\
29.31	0.01\\
29.32	0.01\\
29.33	0.01\\
29.34	0.01\\
29.35	0.01\\
29.36	0.01\\
29.37	0.01\\
29.38	0.01\\
29.39	0.01\\
29.4	0.01\\
29.41	0.01\\
29.42	0.01\\
29.43	0.01\\
29.44	0.01\\
29.45	0.01\\
29.46	0.01\\
29.47	0.01\\
29.48	0.01\\
29.49	0.01\\
29.5	0.01\\
29.51	0.01\\
29.52	0.01\\
29.53	0.01\\
29.54	0.01\\
29.55	0.01\\
29.56	0.01\\
29.57	0.01\\
29.58	0.01\\
29.59	0.01\\
29.6	0.01\\
29.61	0.01\\
29.62	0.01\\
29.63	0.01\\
29.64	0.01\\
29.65	0.01\\
29.66	0.01\\
29.67	0.01\\
29.68	0.01\\
29.69	0.01\\
29.7	0.01\\
29.71	0.01\\
29.72	0.01\\
29.73	0.01\\
29.74	0.01\\
29.75	0.01\\
29.76	0.01\\
29.77	0.01\\
29.78	0.01\\
29.79	0.01\\
29.8	0.01\\
29.81	0.01\\
29.82	0.01\\
29.83	0.01\\
29.84	0.01\\
29.85	0.01\\
29.86	0.01\\
29.87	0.01\\
29.88	0.01\\
29.89	0.01\\
29.9	0.01\\
29.91	0.01\\
29.92	0.01\\
29.93	0.01\\
29.94	0.01\\
29.95	0.01\\
29.96	0.01\\
29.97	0.01\\
29.98	0.01\\
29.99	0.01\\
30	0.01\\
30.01	0.01\\
30.02	0.01\\
30.03	0.01\\
30.04	0.01\\
30.05	0.01\\
30.06	0.01\\
30.07	0.01\\
30.08	0.01\\
30.09	0.01\\
30.1	0.01\\
30.11	0.01\\
30.12	0.01\\
30.13	0.01\\
30.14	0.01\\
30.15	0.01\\
30.16	0.01\\
30.17	0.01\\
30.18	0.01\\
30.19	0.01\\
30.2	0.01\\
30.21	0.01\\
30.22	0.01\\
30.23	0.01\\
30.24	0.01\\
30.25	0.01\\
30.26	0.01\\
30.27	0.01\\
30.28	0.01\\
30.29	0.01\\
30.3	0.01\\
30.31	0.01\\
30.32	0.01\\
30.33	0.01\\
30.34	0.01\\
30.35	0.01\\
30.36	0.01\\
30.37	0.01\\
30.38	0.01\\
30.39	0.01\\
30.4	0.01\\
30.41	0.01\\
30.42	0.01\\
30.43	0.01\\
30.44	0.01\\
30.45	0.01\\
30.46	0.01\\
30.47	0.01\\
30.48	0.01\\
30.49	0.01\\
30.5	0.01\\
30.51	0.01\\
30.52	0.01\\
30.53	0.01\\
30.54	0.01\\
30.55	0.01\\
30.56	0.01\\
30.57	0.01\\
30.58	0.01\\
30.59	0.01\\
30.6	0.01\\
30.61	0.01\\
30.62	0.01\\
30.63	0.01\\
30.64	0.01\\
30.65	0.01\\
30.66	0.01\\
30.67	0.01\\
30.68	0.01\\
30.69	0.01\\
30.7	0.01\\
30.71	0.01\\
30.72	0.01\\
30.73	0.01\\
30.74	0.01\\
30.75	0.01\\
30.76	0.01\\
30.77	0.01\\
30.78	0.01\\
30.79	0.01\\
30.8	0.01\\
30.81	0.01\\
30.82	0.01\\
30.83	0.01\\
30.84	0.01\\
30.85	0.01\\
30.86	0.01\\
30.87	0.01\\
30.88	0.01\\
30.89	0.01\\
30.9	0.01\\
30.91	0.01\\
30.92	0.01\\
30.93	0.01\\
30.94	0.01\\
30.95	0.01\\
30.96	0.01\\
30.97	0.01\\
30.98	0.01\\
30.99	0.01\\
31	0.01\\
31.01	0.01\\
31.02	0.01\\
31.03	0.01\\
31.04	0.01\\
31.05	0.01\\
31.06	0.01\\
31.07	0.01\\
31.08	0.01\\
31.09	0.01\\
31.1	0.01\\
31.11	0.01\\
31.12	0.01\\
31.13	0.01\\
31.14	0.01\\
31.15	0.01\\
31.16	0.01\\
31.17	0.01\\
31.18	0.01\\
31.19	0.01\\
31.2	0.01\\
31.21	0.01\\
31.22	0.01\\
31.23	0.01\\
31.24	0.01\\
31.25	0.01\\
31.26	0.01\\
31.27	0.01\\
31.28	0.01\\
31.29	0.01\\
31.3	0.01\\
31.31	0.01\\
31.32	0.01\\
31.33	0.01\\
31.34	0.01\\
31.35	0.01\\
31.36	0.01\\
31.37	0.01\\
31.38	0.01\\
31.39	0.01\\
31.4	0.01\\
31.41	0.01\\
31.42	0.01\\
31.43	0.01\\
31.44	0.01\\
31.45	0.01\\
31.46	0.01\\
31.47	0.01\\
31.48	0.01\\
31.49	0.01\\
31.5	0.01\\
31.51	0.01\\
31.52	0.01\\
31.53	0.01\\
31.54	0.01\\
31.55	0.01\\
31.56	0.01\\
31.57	0.01\\
31.58	0.01\\
31.59	0.01\\
31.6	0.01\\
31.61	0.01\\
31.62	0.01\\
31.63	0.01\\
31.64	0.01\\
31.65	0.01\\
31.66	0.01\\
31.67	0.01\\
31.68	0.01\\
31.69	0.01\\
31.7	0.01\\
31.71	0.01\\
31.72	0.01\\
31.73	0.01\\
31.74	0.01\\
31.75	0.01\\
31.76	0.01\\
31.77	0.01\\
31.78	0.01\\
31.79	0.01\\
31.8	0.01\\
31.81	0.01\\
31.82	0.01\\
31.83	0.01\\
31.84	0.01\\
31.85	0.01\\
31.86	0.01\\
31.87	0.01\\
31.88	0.01\\
31.89	0.01\\
31.9	0.01\\
31.91	0.01\\
31.92	0.01\\
31.93	0.01\\
31.94	0.01\\
31.95	0.01\\
31.96	0.01\\
31.97	0.01\\
31.98	0.01\\
31.99	0.01\\
32	0.01\\
32.01	0.01\\
32.02	0.01\\
32.03	0.01\\
32.04	0.01\\
32.05	0.01\\
32.06	0.01\\
32.07	0.01\\
32.08	0.01\\
32.09	0.01\\
32.1	0.01\\
32.11	0.01\\
32.12	0.01\\
32.13	0.01\\
32.14	0.01\\
32.15	0.01\\
32.16	0.01\\
32.17	0.01\\
32.18	0.01\\
32.19	0.01\\
32.2	0.01\\
32.21	0.01\\
32.22	0.01\\
32.23	0.01\\
32.24	0.01\\
32.25	0.01\\
32.26	0.01\\
32.27	0.01\\
32.28	0.01\\
32.29	0.01\\
32.3	0.01\\
32.31	0.01\\
32.32	0.01\\
32.33	0.01\\
32.34	0.01\\
32.35	0.01\\
32.36	0.01\\
32.37	0.01\\
32.38	0.01\\
32.39	0.01\\
32.4	0.01\\
32.41	0.01\\
32.42	0.01\\
32.43	0.01\\
32.44	0.01\\
32.45	0.01\\
32.46	0.01\\
32.47	0.01\\
32.48	0.01\\
32.49	0.01\\
32.5	0.01\\
32.51	0.01\\
32.52	0.01\\
32.53	0.01\\
32.54	0.01\\
32.55	0.01\\
32.56	0.01\\
32.57	0.01\\
32.58	0.01\\
32.59	0.01\\
32.6	0.01\\
32.61	0.01\\
32.62	0.01\\
32.63	0.01\\
32.64	0.01\\
32.65	0.01\\
32.66	0.01\\
32.67	0.01\\
32.68	0.01\\
32.69	0.01\\
32.7	0.01\\
32.71	0.01\\
32.72	0.01\\
32.73	0.01\\
32.74	0.01\\
32.75	0.01\\
32.76	0.01\\
32.77	0.01\\
32.78	0.01\\
32.79	0.01\\
32.8	0.01\\
32.81	0.01\\
32.82	0.01\\
32.83	0.01\\
32.84	0.01\\
32.85	0.01\\
32.86	0.01\\
32.87	0.01\\
32.88	0.01\\
32.89	0.01\\
32.9	0.01\\
32.91	0.01\\
32.92	0.01\\
32.93	0.01\\
32.94	0.01\\
32.95	0.01\\
32.96	0.01\\
32.97	0.01\\
32.98	0.01\\
32.99	0.01\\
33	0.01\\
33.01	0.01\\
33.02	0.01\\
33.03	0.01\\
33.04	0.01\\
33.05	0.01\\
33.06	0.01\\
33.07	0.01\\
33.08	0.01\\
33.09	0.01\\
33.1	0.01\\
33.11	0.01\\
33.12	0.01\\
33.13	0.01\\
33.14	0.01\\
33.15	0.01\\
33.16	0.01\\
33.17	0.01\\
33.18	0.01\\
33.19	0.01\\
33.2	0.01\\
33.21	0.01\\
33.22	0.01\\
33.23	0.01\\
33.24	0.01\\
33.25	0.01\\
33.26	0.01\\
33.27	0.01\\
33.28	0.01\\
33.29	0.01\\
33.3	0.01\\
33.31	0.01\\
33.32	0.01\\
33.33	0.01\\
33.34	0.01\\
33.35	0.01\\
33.36	0.01\\
33.37	0.01\\
33.38	0.01\\
33.39	0.01\\
33.4	0.01\\
33.41	0.01\\
33.42	0.01\\
33.43	0.01\\
33.44	0.01\\
33.45	0.01\\
33.46	0.01\\
33.47	0.01\\
33.48	0.01\\
33.49	0.01\\
33.5	0.01\\
33.51	0.01\\
33.52	0.01\\
33.53	0.01\\
33.54	0.01\\
33.55	0.01\\
33.56	0.01\\
33.57	0.01\\
33.58	0.01\\
33.59	0.01\\
33.6	0.01\\
33.61	0.01\\
33.62	0.01\\
33.63	0.01\\
33.64	0.01\\
33.65	0.01\\
33.66	0.01\\
33.67	0.01\\
33.68	0.01\\
33.69	0.01\\
33.7	0.01\\
33.71	0.01\\
33.72	0.01\\
33.73	0.01\\
33.74	0.01\\
33.75	0.01\\
33.76	0.01\\
33.77	0.01\\
33.78	0.01\\
33.79	0.01\\
33.8	0.01\\
33.81	0.01\\
33.82	0.01\\
33.83	0.01\\
33.84	0.01\\
33.85	0.01\\
33.86	0.01\\
33.87	0.01\\
33.88	0.01\\
33.89	0.01\\
33.9	0.01\\
33.91	0.01\\
33.92	0.01\\
33.93	0.01\\
33.94	0.01\\
33.95	0.01\\
33.96	0.01\\
33.97	0.01\\
33.98	0.01\\
33.99	0.01\\
34	0.01\\
34.01	0.01\\
34.02	0.01\\
34.03	0.01\\
34.04	0.01\\
34.05	0.01\\
34.06	0.01\\
34.07	0.01\\
34.08	0.01\\
34.09	0.01\\
34.1	0.01\\
34.11	0.01\\
34.12	0.01\\
34.13	0.01\\
34.14	0.01\\
34.15	0.01\\
34.16	0.01\\
34.17	0.01\\
34.18	0.01\\
34.19	0.01\\
34.2	0.01\\
34.21	0.01\\
34.22	0.01\\
34.23	0.01\\
34.24	0.01\\
34.25	0.01\\
34.26	0.01\\
34.27	0.01\\
34.28	0.01\\
34.29	0.01\\
34.3	0.01\\
34.31	0.01\\
34.32	0.01\\
34.33	0.01\\
34.34	0.01\\
34.35	0.01\\
34.36	0.01\\
34.37	0.01\\
34.38	0.01\\
34.39	0.01\\
34.4	0.01\\
34.41	0.01\\
34.42	0.01\\
34.43	0.01\\
34.44	0.01\\
34.45	0.01\\
34.46	0.01\\
34.47	0.01\\
34.48	0.01\\
34.49	0.01\\
34.5	0.01\\
34.51	0.01\\
34.52	0.01\\
34.53	0.01\\
34.54	0.01\\
34.55	0.01\\
34.56	0.01\\
34.57	0.01\\
34.58	0.01\\
34.59	0.01\\
34.6	0.01\\
34.61	0.01\\
34.62	0.01\\
34.63	0.01\\
34.64	0.01\\
34.65	0.01\\
34.66	0.01\\
34.67	0.01\\
34.68	0.01\\
34.69	0.01\\
34.7	0.01\\
34.71	0.01\\
34.72	0.01\\
34.73	0.01\\
34.74	0.01\\
34.75	0.01\\
34.76	0.01\\
34.77	0.01\\
34.78	0.01\\
34.79	0.01\\
34.8	0.01\\
34.81	0.01\\
34.82	0.01\\
34.83	0.01\\
34.84	0.01\\
34.85	0.01\\
34.86	0.01\\
34.87	0.01\\
34.88	0.01\\
34.89	0.01\\
34.9	0.01\\
34.91	0.01\\
34.92	0.01\\
34.93	0.01\\
34.94	0.01\\
34.95	0.01\\
34.96	0.01\\
34.97	0.01\\
34.98	0.01\\
34.99	0.01\\
35	0.01\\
35.01	0.01\\
35.02	0.01\\
35.03	0.01\\
35.04	0.01\\
35.05	0.01\\
35.06	0.01\\
35.07	0.01\\
35.08	0.01\\
35.09	0.01\\
35.1	0.01\\
35.11	0.01\\
35.12	0.01\\
35.13	0.01\\
35.14	0.01\\
35.15	0.01\\
35.16	0.01\\
35.17	0.01\\
35.18	0.01\\
35.19	0.01\\
35.2	0.01\\
35.21	0.01\\
35.22	0.01\\
35.23	0.01\\
35.24	0.01\\
35.25	0.01\\
35.26	0.01\\
35.27	0.01\\
35.28	0.01\\
35.29	0.01\\
35.3	0.01\\
35.31	0.01\\
35.32	0.01\\
35.33	0.01\\
35.34	0.01\\
35.35	0.01\\
35.36	0.01\\
35.37	0.01\\
35.38	0.01\\
35.39	0.01\\
35.4	0.01\\
35.41	0.01\\
35.42	0.01\\
35.43	0.01\\
35.44	0.01\\
35.45	0.01\\
35.46	0.01\\
35.47	0.01\\
35.48	0.01\\
35.49	0.01\\
35.5	0.01\\
35.51	0.01\\
35.52	0.01\\
35.53	0.01\\
35.54	0.01\\
35.55	0.01\\
35.56	0.01\\
35.57	0.01\\
35.58	0.01\\
35.59	0.01\\
35.6	0.01\\
35.61	0.01\\
35.62	0.01\\
35.63	0.01\\
35.64	0.01\\
35.65	0.01\\
35.66	0.01\\
35.67	0.01\\
35.68	0.01\\
35.69	0.01\\
35.7	0.01\\
35.71	0.01\\
35.72	0.01\\
35.73	0.01\\
35.74	0.01\\
35.75	0.01\\
35.76	0.01\\
35.77	0.01\\
35.78	0.01\\
35.79	0.01\\
35.8	0.01\\
35.81	0.01\\
35.82	0.01\\
35.83	0.01\\
35.84	0.01\\
35.85	0.01\\
35.86	0.01\\
35.87	0.01\\
35.88	0.01\\
35.89	0.01\\
35.9	0.01\\
35.91	0.01\\
35.92	0.01\\
35.93	0.01\\
35.94	0.01\\
35.95	0.01\\
35.96	0.01\\
35.97	0.01\\
35.98	0.01\\
35.99	0.01\\
36	0.01\\
36.01	0.01\\
36.02	0.01\\
36.03	0.01\\
36.04	0.01\\
36.05	0.01\\
36.06	0.01\\
36.07	0.01\\
36.08	0.01\\
36.09	0.01\\
36.1	0.01\\
36.11	0.01\\
36.12	0.01\\
36.13	0.01\\
36.14	0.01\\
36.15	0.01\\
36.16	0.01\\
36.17	0.01\\
36.18	0.01\\
36.19	0.01\\
36.2	0.01\\
36.21	0.01\\
36.22	0.01\\
36.23	0.01\\
36.24	0.01\\
36.25	0.01\\
36.26	0.01\\
36.27	0.01\\
36.28	0.01\\
36.29	0.01\\
36.3	0.01\\
36.31	0.01\\
36.32	0.01\\
36.33	0.01\\
36.34	0.01\\
36.35	0.01\\
36.36	0.01\\
36.37	0.01\\
36.38	0.01\\
36.39	0.01\\
36.4	0.01\\
36.41	0.01\\
36.42	0.01\\
36.43	0.01\\
36.44	0.01\\
36.45	0.01\\
36.46	0.01\\
36.47	0.01\\
36.48	0.01\\
36.49	0.01\\
36.5	0.01\\
36.51	0.01\\
36.52	0.01\\
36.53	0.01\\
36.54	0.01\\
36.55	0.01\\
36.56	0.01\\
36.57	0.01\\
36.58	0.01\\
36.59	0.01\\
36.6	0.01\\
36.61	0.01\\
36.62	0.01\\
36.63	0.01\\
36.64	0.01\\
36.65	0.01\\
36.66	0.01\\
36.67	0.01\\
36.68	0.01\\
36.69	0.01\\
36.7	0.01\\
36.71	0.01\\
36.72	0.01\\
36.73	0.01\\
36.74	0.01\\
36.75	0.01\\
36.76	0.01\\
36.77	0.01\\
36.78	0.01\\
36.79	0.01\\
36.8	0.01\\
36.81	0.01\\
36.82	0.01\\
36.83	0.01\\
36.84	0.01\\
36.85	0.01\\
36.86	0.01\\
36.87	0.01\\
36.88	0.01\\
36.89	0.01\\
36.9	0.01\\
36.91	0.01\\
36.92	0.01\\
36.93	0.01\\
36.94	0.01\\
36.95	0.01\\
36.96	0.01\\
36.97	0.01\\
36.98	0.01\\
36.99	0.01\\
37	0.01\\
37.01	0.01\\
37.02	0.01\\
37.03	0.01\\
37.04	0.01\\
37.05	0.01\\
37.06	0.01\\
37.07	0.01\\
37.08	0.01\\
37.09	0.01\\
37.1	0.01\\
37.11	0.01\\
37.12	0.01\\
37.13	0.01\\
37.14	0.01\\
37.15	0.01\\
37.16	0.01\\
37.17	0.01\\
37.18	0.01\\
37.19	0.01\\
37.2	0.01\\
37.21	0.01\\
37.22	0.01\\
37.23	0.01\\
37.24	0.01\\
37.25	0.01\\
37.26	0.01\\
37.27	0.01\\
37.28	0.01\\
37.29	0.01\\
37.3	0.01\\
37.31	0.01\\
37.32	0.01\\
37.33	0.01\\
37.34	0.01\\
37.35	0.01\\
37.36	0.01\\
37.37	0.01\\
37.38	0.01\\
37.39	0.01\\
37.4	0.01\\
37.41	0.01\\
37.42	0.01\\
37.43	0.01\\
37.44	0.01\\
37.45	0.01\\
37.46	0.01\\
37.47	0.01\\
37.48	0.01\\
37.49	0.01\\
37.5	0.01\\
37.51	0.01\\
37.52	0.01\\
37.53	0.01\\
37.54	0.01\\
37.55	0.01\\
37.56	0.01\\
37.57	0.01\\
37.58	0.01\\
37.59	0.01\\
37.6	0.01\\
37.61	0.01\\
37.62	0.01\\
37.63	0.01\\
37.64	0.01\\
37.65	0.01\\
37.66	0.01\\
37.67	0.01\\
37.68	0.01\\
37.69	0.01\\
37.7	0.01\\
37.71	0.01\\
37.72	0.01\\
37.73	0.01\\
37.74	0.01\\
37.75	0.01\\
37.76	0.01\\
37.77	0.01\\
37.78	0.01\\
37.79	0.01\\
37.8	0.01\\
37.81	0.01\\
37.82	0.01\\
37.83	0.01\\
37.84	0.01\\
37.85	0.01\\
37.86	0.01\\
37.87	0.01\\
37.88	0.01\\
37.89	0.01\\
37.9	0.01\\
37.91	0.01\\
37.92	0.01\\
37.93	0.01\\
37.94	0.01\\
37.95	0.01\\
37.96	0.01\\
37.97	0.01\\
37.98	0.01\\
37.99	0.01\\
38	0.01\\
38.01	0.01\\
38.02	0.01\\
38.03	0.01\\
38.04	0.01\\
38.05	0.01\\
38.06	0.01\\
38.07	0.01\\
38.08	0.01\\
38.09	0.01\\
38.1	0.01\\
38.11	0.01\\
38.12	0.01\\
38.13	0.01\\
38.14	0.01\\
38.15	0.01\\
38.16	0.01\\
38.17	0.01\\
38.18	0.01\\
38.19	0.01\\
38.2	0.01\\
38.21	0.01\\
38.22	0.01\\
38.23	0.01\\
38.24	0.01\\
38.25	0.01\\
38.26	0.01\\
38.27	0.01\\
38.28	0.01\\
38.29	0.01\\
38.3	0.01\\
38.31	0.01\\
38.32	0.01\\
38.33	0.01\\
38.34	0.01\\
38.35	0.01\\
38.36	0.01\\
38.37	0.01\\
38.38	0.01\\
38.39	0.01\\
38.4	0.01\\
38.41	0.01\\
38.42	0.01\\
38.43	0.01\\
38.44	0.01\\
38.45	0.01\\
38.46	0.01\\
38.47	0.01\\
38.48	0.01\\
38.49	0.01\\
38.5	0.01\\
38.51	0.01\\
38.52	0.01\\
38.53	0.01\\
38.54	0.01\\
38.55	0.01\\
38.56	0.01\\
38.57	0.01\\
38.58	0.01\\
38.59	0.01\\
38.6	0.01\\
38.61	0.01\\
38.62	0.01\\
38.63	0.01\\
38.64	0.01\\
38.65	0.01\\
38.66	0.01\\
38.67	0.01\\
38.68	0.01\\
38.69	0.01\\
38.7	0.01\\
38.71	0.01\\
38.72	0.01\\
38.73	0.01\\
38.74	0.01\\
38.75	0.01\\
38.76	0.01\\
38.77	0.01\\
38.78	0.01\\
38.79	0.01\\
38.8	0.01\\
38.81	0.01\\
38.82	0.01\\
38.83	0.01\\
38.84	0.01\\
38.85	0.01\\
38.86	0.01\\
38.87	0.01\\
38.88	0.01\\
38.89	0.01\\
38.9	0.01\\
38.91	0.01\\
38.92	0.01\\
38.93	0.01\\
38.94	0.01\\
38.95	0.01\\
38.96	0.01\\
38.97	0.01\\
38.98	0.01\\
38.99	0.01\\
39	0.01\\
39.01	0.01\\
39.02	0.01\\
39.03	0.01\\
39.04	0.01\\
39.05	0.01\\
39.06	0.01\\
39.07	0.01\\
39.08	0.01\\
39.09	0.01\\
39.1	0.01\\
39.11	0.01\\
39.12	0.01\\
39.13	0.01\\
39.14	0.01\\
39.15	0.01\\
39.16	0.01\\
39.17	0.01\\
39.18	0.01\\
39.19	0.01\\
39.2	0.01\\
39.21	0.01\\
39.22	0.01\\
39.23	0.01\\
39.24	0.01\\
39.25	0.01\\
39.26	0.01\\
39.27	0.01\\
39.28	0.01\\
39.29	0.01\\
39.3	0.01\\
39.31	0.01\\
39.32	0.01\\
39.33	0.01\\
39.34	0.01\\
39.35	0.01\\
39.36	0.01\\
39.37	0.01\\
39.38	0.01\\
39.39	0.01\\
39.4	0.01\\
39.41	0.01\\
39.42	0.01\\
39.43	0.01\\
39.44	0.01\\
39.45	0.01\\
39.46	0.01\\
39.47	0.01\\
39.48	0.01\\
39.49	0.01\\
39.5	0.01\\
39.51	0.01\\
39.52	0.01\\
39.53	0.01\\
39.54	0.01\\
39.55	0.01\\
39.56	0.01\\
39.57	0.01\\
39.58	0.01\\
39.59	0.01\\
39.6	0.01\\
39.61	0.01\\
39.62	0.01\\
39.63	0.01\\
39.64	0.01\\
39.65	0.01\\
39.66	0.01\\
39.67	0.01\\
39.68	0.01\\
39.69	0.01\\
39.7	0.01\\
39.71	0.01\\
39.72	0.01\\
39.73	0.01\\
39.74	0.01\\
39.75	0.01\\
39.76	0.01\\
39.77	0.01\\
39.78	0.01\\
39.79	0.01\\
39.8	0.01\\
39.81	0.01\\
39.82	0.01\\
39.83	0.01\\
39.84	0.01\\
39.85	0.01\\
39.86	0.01\\
39.87	0.01\\
39.88	0.01\\
39.89	0.01\\
39.9	0.01\\
39.91	0.01\\
39.92	0.01\\
39.93	0.01\\
39.94	0.01\\
39.95	0.01\\
39.96	0.01\\
39.97	0.01\\
39.98	0.01\\
39.99	0.01\\
40	0.01\\
40.01	0.01\\
};
\addplot [color=blue,dashed,forget plot]
  table[row sep=crcr]{%
40.01	0.01\\
40.02	0.01\\
40.03	0.01\\
40.04	0.01\\
40.05	0.01\\
40.06	0.01\\
40.07	0.01\\
40.08	0.01\\
40.09	0.01\\
40.1	0.01\\
40.11	0.01\\
40.12	0.01\\
40.13	0.01\\
40.14	0.01\\
40.15	0.01\\
40.16	0.01\\
40.17	0.01\\
40.18	0.01\\
40.19	0.01\\
40.2	0.01\\
40.21	0.01\\
40.22	0.01\\
40.23	0.01\\
40.24	0.01\\
40.25	0.01\\
40.26	0.01\\
40.27	0.01\\
40.28	0.01\\
40.29	0.01\\
40.3	0.01\\
40.31	0.01\\
40.32	0.01\\
40.33	0.01\\
40.34	0.01\\
40.35	0.01\\
40.36	0.01\\
40.37	0.01\\
40.38	0.01\\
40.39	0.01\\
40.4	0.01\\
40.41	0.01\\
40.42	0.01\\
40.43	0.01\\
40.44	0.01\\
40.45	0.01\\
40.46	0.01\\
40.47	0.01\\
40.48	0.01\\
40.49	0.01\\
40.5	0.01\\
40.51	0.01\\
40.52	0.01\\
40.53	0.01\\
40.54	0.01\\
40.55	0.01\\
40.56	0.01\\
40.57	0.01\\
40.58	0.01\\
40.59	0.01\\
40.6	0.01\\
40.61	0.01\\
40.62	0.01\\
40.63	0.01\\
40.64	0.01\\
40.65	0.01\\
40.66	0.01\\
40.67	0.01\\
40.68	0.01\\
40.69	0.01\\
40.7	0.01\\
40.71	0.01\\
40.72	0.01\\
40.73	0.01\\
40.74	0.01\\
40.75	0.01\\
40.76	0.01\\
40.77	0.01\\
40.78	0.01\\
40.79	0.01\\
40.8	0.01\\
40.81	0.01\\
40.82	0.01\\
40.83	0.01\\
40.84	0.01\\
40.85	0.01\\
40.86	0.01\\
40.87	0.01\\
40.88	0.01\\
40.89	0.01\\
40.9	0.01\\
40.91	0.01\\
40.92	0.01\\
40.93	0.01\\
40.94	0.01\\
40.95	0.01\\
40.96	0.01\\
40.97	0.01\\
40.98	0.01\\
40.99	0.01\\
41	0.01\\
41.01	0.01\\
41.02	0.01\\
41.03	0.01\\
41.04	0.01\\
41.05	0.01\\
41.06	0.01\\
41.07	0.01\\
41.08	0.01\\
41.09	0.01\\
41.1	0.01\\
41.11	0.01\\
41.12	0.01\\
41.13	0.01\\
41.14	0.01\\
41.15	0.01\\
41.16	0.01\\
41.17	0.01\\
41.18	0.01\\
41.19	0.01\\
41.2	0.01\\
41.21	0.01\\
41.22	0.01\\
41.23	0.01\\
41.24	0.01\\
41.25	0.01\\
41.26	0.01\\
41.27	0.01\\
41.28	0.01\\
41.29	0.01\\
41.3	0.01\\
41.31	0.01\\
41.32	0.01\\
41.33	0.01\\
41.34	0.01\\
41.35	0.01\\
41.36	0.01\\
41.37	0.01\\
41.38	0.01\\
41.39	0.01\\
41.4	0.01\\
41.41	0.01\\
41.42	0.01\\
41.43	0.01\\
41.44	0.01\\
41.45	0.01\\
41.46	0.01\\
41.47	0.01\\
41.48	0.01\\
41.49	0.01\\
41.5	0.01\\
41.51	0.01\\
41.52	0.01\\
41.53	0.01\\
41.54	0.01\\
41.55	0.01\\
41.56	0.01\\
41.57	0.01\\
41.58	0.01\\
41.59	0.01\\
41.6	0.01\\
41.61	0.01\\
41.62	0.01\\
41.63	0.01\\
41.64	0.01\\
41.65	0.01\\
41.66	0.01\\
41.67	0.01\\
41.68	0.01\\
41.69	0.01\\
41.7	0.01\\
41.71	0.01\\
41.72	0.01\\
41.73	0.01\\
41.74	0.01\\
41.75	0.01\\
41.76	0.01\\
41.77	0.01\\
41.78	0.01\\
41.79	0.01\\
41.8	0.01\\
41.81	0.01\\
41.82	0.01\\
41.83	0.01\\
41.84	0.01\\
41.85	0.01\\
41.86	0.01\\
41.87	0.01\\
41.88	0.01\\
41.89	0.01\\
41.9	0.01\\
41.91	0.01\\
41.92	0.01\\
41.93	0.01\\
41.94	0.01\\
41.95	0.01\\
41.96	0.01\\
41.97	0.01\\
41.98	0.01\\
41.99	0.01\\
42	0.01\\
42.01	0.01\\
42.02	0.01\\
42.03	0.01\\
42.04	0.01\\
42.05	0.01\\
42.06	0.01\\
42.07	0.01\\
42.08	0.01\\
42.09	0.01\\
42.1	0.01\\
42.11	0.01\\
42.12	0.01\\
42.13	0.01\\
42.14	0.01\\
42.15	0.01\\
42.16	0.01\\
42.17	0.01\\
42.18	0.01\\
42.19	0.01\\
42.2	0.01\\
42.21	0.01\\
42.22	0.01\\
42.23	0.01\\
42.24	0.01\\
42.25	0.01\\
42.26	0.01\\
42.27	0.01\\
42.28	0.01\\
42.29	0.01\\
42.3	0.01\\
42.31	0.01\\
42.32	0.01\\
42.33	0.01\\
42.34	0.01\\
42.35	0.01\\
42.36	0.01\\
42.37	0.01\\
42.38	0.01\\
42.39	0.01\\
42.4	0.01\\
42.41	0.01\\
42.42	0.01\\
42.43	0.01\\
42.44	0.01\\
42.45	0.01\\
42.46	0.01\\
42.47	0.01\\
42.48	0.01\\
42.49	0.01\\
42.5	0.01\\
42.51	0.01\\
42.52	0.01\\
42.53	0.01\\
42.54	0.01\\
42.55	0.01\\
42.56	0.01\\
42.57	0.01\\
42.58	0.01\\
42.59	0.01\\
42.6	0.01\\
42.61	0.01\\
42.62	0.01\\
42.63	0.01\\
42.64	0.01\\
42.65	0.01\\
42.66	0.01\\
42.67	0.01\\
42.68	0.01\\
42.69	0.01\\
42.7	0.01\\
42.71	0.01\\
42.72	0.01\\
42.73	0.01\\
42.74	0.01\\
42.75	0.01\\
42.76	0.01\\
42.77	0.01\\
42.78	0.01\\
42.79	0.01\\
42.8	0.01\\
42.81	0.01\\
42.82	0.01\\
42.83	0.01\\
42.84	0.01\\
42.85	0.01\\
42.86	0.01\\
42.87	0.01\\
42.88	0.01\\
42.89	0.01\\
42.9	0.01\\
42.91	0.01\\
42.92	0.01\\
42.93	0.01\\
42.94	0.01\\
42.95	0.01\\
42.96	0.01\\
42.97	0.01\\
42.98	0.01\\
42.99	0.01\\
43	0.01\\
43.01	0.01\\
43.02	0.01\\
43.03	0.01\\
43.04	0.01\\
43.05	0.01\\
43.06	0.01\\
43.07	0.01\\
43.08	0.01\\
43.09	0.01\\
43.1	0.01\\
43.11	0.01\\
43.12	0.01\\
43.13	0.01\\
43.14	0.01\\
43.15	0.01\\
43.16	0.01\\
43.17	0.01\\
43.18	0.01\\
43.19	0.01\\
43.2	0.01\\
43.21	0.01\\
43.22	0.01\\
43.23	0.01\\
43.24	0.01\\
43.25	0.01\\
43.26	0.01\\
43.27	0.01\\
43.28	0.01\\
43.29	0.01\\
43.3	0.01\\
43.31	0.01\\
43.32	0.01\\
43.33	0.01\\
43.34	0.01\\
43.35	0.01\\
43.36	0.01\\
43.37	0.01\\
43.38	0.01\\
43.39	0.01\\
43.4	0.01\\
43.41	0.01\\
43.42	0.01\\
43.43	0.01\\
43.44	0.01\\
43.45	0.01\\
43.46	0.01\\
43.47	0.01\\
43.48	0.01\\
43.49	0.01\\
43.5	0.01\\
43.51	0.01\\
43.52	0.01\\
43.53	0.01\\
43.54	0.01\\
43.55	0.01\\
43.56	0.01\\
43.57	0.01\\
43.58	0.01\\
43.59	0.01\\
43.6	0.01\\
43.61	0.01\\
43.62	0.01\\
43.63	0.01\\
43.64	0.01\\
43.65	0.01\\
43.66	0.01\\
43.67	0.01\\
43.68	0.01\\
43.69	0.01\\
43.7	0.01\\
43.71	0.01\\
43.72	0.01\\
43.73	0.01\\
43.74	0.01\\
43.75	0.01\\
43.76	0.01\\
43.77	0.01\\
43.78	0.01\\
43.79	0.01\\
43.8	0.01\\
43.81	0.01\\
43.82	0.01\\
43.83	0.01\\
43.84	0.01\\
43.85	0.01\\
43.86	0.01\\
43.87	0.01\\
43.88	0.01\\
43.89	0.01\\
43.9	0.01\\
43.91	0.01\\
43.92	0.01\\
43.93	0.01\\
43.94	0.01\\
43.95	0.01\\
43.96	0.01\\
43.97	0.01\\
43.98	0.01\\
43.99	0.01\\
44	0.01\\
44.01	0.01\\
44.02	0.01\\
44.03	0.01\\
44.04	0.01\\
44.05	0.01\\
44.06	0.01\\
44.07	0.01\\
44.08	0.01\\
44.09	0.01\\
44.1	0.01\\
44.11	0.01\\
44.12	0.01\\
44.13	0.01\\
44.14	0.01\\
44.15	0.01\\
44.16	0.01\\
44.17	0.01\\
44.18	0.01\\
44.19	0.01\\
44.2	0.01\\
44.21	0.01\\
44.22	0.01\\
44.23	0.01\\
44.24	0.01\\
44.25	0.01\\
44.26	0.01\\
44.27	0.01\\
44.28	0.01\\
44.29	0.01\\
44.3	0.01\\
44.31	0.01\\
44.32	0.01\\
44.33	0.01\\
44.34	0.01\\
44.35	0.01\\
44.36	0.01\\
44.37	0.01\\
44.38	0.01\\
44.39	0.01\\
44.4	0.01\\
44.41	0.01\\
44.42	0.01\\
44.43	0.01\\
44.44	0.01\\
44.45	0.01\\
44.46	0.01\\
44.47	0.01\\
44.48	0.01\\
44.49	0.01\\
44.5	0.01\\
44.51	0.01\\
44.52	0.01\\
44.53	0.01\\
44.54	0.01\\
44.55	0.01\\
44.56	0.01\\
44.57	0.01\\
44.58	0.01\\
44.59	0.01\\
44.6	0.01\\
44.61	0.01\\
44.62	0.01\\
44.63	0.01\\
44.64	0.01\\
44.65	0.01\\
44.66	0.01\\
44.67	0.01\\
44.68	0.01\\
44.69	0.01\\
44.7	0.01\\
44.71	0.01\\
44.72	0.01\\
44.73	0.01\\
44.74	0.01\\
44.75	0.01\\
44.76	0.01\\
44.77	0.01\\
44.78	0.01\\
44.79	0.01\\
44.8	0.01\\
44.81	0.01\\
44.82	0.01\\
44.83	0.01\\
44.84	0.01\\
44.85	0.01\\
44.86	0.01\\
44.87	0.01\\
44.88	0.01\\
44.89	0.01\\
44.9	0.01\\
44.91	0.01\\
44.92	0.01\\
44.93	0.01\\
44.94	0.01\\
44.95	0.01\\
44.96	0.01\\
44.97	0.01\\
44.98	0.01\\
44.99	0.01\\
45	0.01\\
45.01	0.01\\
45.02	0.01\\
45.03	0.01\\
45.04	0.01\\
45.05	0.01\\
45.06	0.01\\
45.07	0.01\\
45.08	0.01\\
45.09	0.01\\
45.1	0.01\\
45.11	0.01\\
45.12	0.01\\
45.13	0.01\\
45.14	0.01\\
45.15	0.01\\
45.16	0.01\\
45.17	0.01\\
45.18	0.01\\
45.19	0.01\\
45.2	0.01\\
45.21	0.01\\
45.22	0.01\\
45.23	0.01\\
45.24	0.01\\
45.25	0.01\\
45.26	0.01\\
45.27	0.01\\
45.28	0.01\\
45.29	0.01\\
45.3	0.01\\
45.31	0.01\\
45.32	0.01\\
45.33	0.01\\
45.34	0.01\\
45.35	0.01\\
45.36	0.01\\
45.37	0.01\\
45.38	0.01\\
45.39	0.01\\
45.4	0.01\\
45.41	0.01\\
45.42	0.01\\
45.43	0.01\\
45.44	0.01\\
45.45	0.01\\
45.46	0.01\\
45.47	0.01\\
45.48	0.01\\
45.49	0.01\\
45.5	0.01\\
45.51	0.01\\
45.52	0.01\\
45.53	0.01\\
45.54	0.01\\
45.55	0.01\\
45.56	0.01\\
45.57	0.01\\
45.58	0.01\\
45.59	0.01\\
45.6	0.01\\
45.61	0.01\\
45.62	0.01\\
45.63	0.01\\
45.64	0.01\\
45.65	0.01\\
45.66	0.01\\
45.67	0.01\\
45.68	0.01\\
45.69	0.01\\
45.7	0.01\\
45.71	0.01\\
45.72	0.01\\
45.73	0.01\\
45.74	0.01\\
45.75	0.01\\
45.76	0.01\\
45.77	0.01\\
45.78	0.01\\
45.79	0.01\\
45.8	0.01\\
45.81	0.01\\
45.82	0.01\\
45.83	0.01\\
45.84	0.01\\
45.85	0.01\\
45.86	0.01\\
45.87	0.01\\
45.88	0.01\\
45.89	0.01\\
45.9	0.01\\
45.91	0.01\\
45.92	0.01\\
45.93	0.01\\
45.94	0.01\\
45.95	0.01\\
45.96	0.01\\
45.97	0.01\\
45.98	0.01\\
45.99	0.01\\
46	0.01\\
46.01	0.01\\
46.02	0.01\\
46.03	0.01\\
46.04	0.01\\
46.05	0.01\\
46.06	0.01\\
46.07	0.01\\
46.08	0.01\\
46.09	0.01\\
46.1	0.01\\
46.11	0.01\\
46.12	0.01\\
46.13	0.01\\
46.14	0.01\\
46.15	0.01\\
46.16	0.01\\
46.17	0.01\\
46.18	0.01\\
46.19	0.01\\
46.2	0.01\\
46.21	0.01\\
46.22	0.01\\
46.23	0.01\\
46.24	0.01\\
46.25	0.01\\
46.26	0.01\\
46.27	0.01\\
46.28	0.01\\
46.29	0.01\\
46.3	0.01\\
46.31	0.01\\
46.32	0.01\\
46.33	0.01\\
46.34	0.01\\
46.35	0.01\\
46.36	0.01\\
46.37	0.01\\
46.38	0.01\\
46.39	0.01\\
46.4	0.01\\
46.41	0.01\\
46.42	0.01\\
46.43	0.01\\
46.44	0.01\\
46.45	0.01\\
46.46	0.01\\
46.47	0.01\\
46.48	0.01\\
46.49	0.01\\
46.5	0.01\\
46.51	0.01\\
46.52	0.01\\
46.53	0.01\\
46.54	0.01\\
46.55	0.01\\
46.56	0.01\\
46.57	0.01\\
46.58	0.01\\
46.59	0.01\\
46.6	0.01\\
46.61	0.01\\
46.62	0.01\\
46.63	0.01\\
46.64	0.01\\
46.65	0.01\\
46.66	0.01\\
46.67	0.01\\
46.68	0.01\\
46.69	0.01\\
46.7	0.01\\
46.71	0.01\\
46.72	0.01\\
46.73	0.01\\
46.74	0.01\\
46.75	0.01\\
46.76	0.01\\
46.77	0.01\\
46.78	0.01\\
46.79	0.01\\
46.8	0.01\\
46.81	0.01\\
46.82	0.01\\
46.83	0.01\\
46.84	0.01\\
46.85	0.01\\
46.86	0.01\\
46.87	0.01\\
46.88	0.01\\
46.89	0.01\\
46.9	0.01\\
46.91	0.01\\
46.92	0.01\\
46.93	0.01\\
46.94	0.01\\
46.95	0.01\\
46.96	0.01\\
46.97	0.01\\
46.98	0.01\\
46.99	0.01\\
47	0.01\\
47.01	0.01\\
47.02	0.01\\
47.03	0.01\\
47.04	0.01\\
47.05	0.01\\
47.06	0.01\\
47.07	0.01\\
47.08	0.01\\
47.09	0.01\\
47.1	0.01\\
47.11	0.01\\
47.12	0.01\\
47.13	0.01\\
47.14	0.01\\
47.15	0.01\\
47.16	0.01\\
47.17	0.01\\
47.18	0.01\\
47.19	0.01\\
47.2	0.01\\
47.21	0.01\\
47.22	0.01\\
47.23	0.01\\
47.24	0.01\\
47.25	0.01\\
47.26	0.01\\
47.27	0.01\\
47.28	0.01\\
47.29	0.01\\
47.3	0.01\\
47.31	0.01\\
47.32	0.01\\
47.33	0.01\\
47.34	0.01\\
47.35	0.01\\
47.36	0.01\\
47.37	0.01\\
47.38	0.01\\
47.39	0.01\\
47.4	0.01\\
47.41	0.01\\
47.42	0.01\\
47.43	0.01\\
47.44	0.01\\
47.45	0.01\\
47.46	0.01\\
47.47	0.01\\
47.48	0.01\\
47.49	0.01\\
47.5	0.01\\
47.51	0.01\\
47.52	0.01\\
47.53	0.01\\
47.54	0.01\\
47.55	0.01\\
47.56	0.01\\
47.57	0.01\\
47.58	0.01\\
47.59	0.01\\
47.6	0.01\\
47.61	0.01\\
47.62	0.01\\
47.63	0.01\\
47.64	0.01\\
47.65	0.01\\
47.66	0.01\\
47.67	0.01\\
47.68	0.01\\
47.69	0.01\\
47.7	0.01\\
47.71	0.01\\
47.72	0.01\\
47.73	0.01\\
47.74	0.01\\
47.75	0.01\\
47.76	0.01\\
47.77	0.01\\
47.78	0.01\\
47.79	0.01\\
47.8	0.01\\
47.81	0.01\\
47.82	0.01\\
47.83	0.01\\
47.84	0.01\\
47.85	0.01\\
47.86	0.01\\
47.87	0.01\\
47.88	0.01\\
47.89	0.01\\
47.9	0.01\\
47.91	0.01\\
47.92	0.01\\
47.93	0.01\\
47.94	0.01\\
47.95	0.01\\
47.96	0.01\\
47.97	0.01\\
47.98	0.01\\
47.99	0.01\\
48	0.01\\
48.01	0.01\\
48.02	0.01\\
48.03	0.01\\
48.04	0.01\\
48.05	0.01\\
48.06	0.01\\
48.07	0.01\\
48.08	0.01\\
48.09	0.01\\
48.1	0.01\\
48.11	0.01\\
48.12	0.01\\
48.13	0.01\\
48.14	0.01\\
48.15	0.01\\
48.16	0.01\\
48.17	0.01\\
48.18	0.01\\
48.19	0.01\\
48.2	0.01\\
48.21	0.01\\
48.22	0.01\\
48.23	0.01\\
48.24	0.01\\
48.25	0.01\\
48.26	0.01\\
48.27	0.01\\
48.28	0.01\\
48.29	0.01\\
48.3	0.01\\
48.31	0.01\\
48.32	0.01\\
48.33	0.01\\
48.34	0.01\\
48.35	0.01\\
48.36	0.01\\
48.37	0.01\\
48.38	0.01\\
48.39	0.01\\
48.4	0.01\\
48.41	0.01\\
48.42	0.01\\
48.43	0.01\\
48.44	0.01\\
48.45	0.01\\
48.46	0.01\\
48.47	0.01\\
48.48	0.01\\
48.49	0.01\\
48.5	0.01\\
48.51	0.01\\
48.52	0.01\\
48.53	0.01\\
48.54	0.01\\
48.55	0.01\\
48.56	0.01\\
48.57	0.01\\
48.58	0.01\\
48.59	0.01\\
48.6	0.01\\
48.61	0.01\\
48.62	0.01\\
48.63	0.01\\
48.64	0.01\\
48.65	0.01\\
48.66	0.01\\
48.67	0.01\\
48.68	0.01\\
48.69	0.01\\
48.7	0.01\\
48.71	0.01\\
48.72	0.01\\
48.73	0.01\\
48.74	0.01\\
48.75	0.01\\
48.76	0.01\\
48.77	0.01\\
48.78	0.01\\
48.79	0.01\\
48.8	0.01\\
48.81	0.01\\
48.82	0.01\\
48.83	0.01\\
48.84	0.01\\
48.85	0.01\\
48.86	0.01\\
48.87	0.01\\
48.88	0.01\\
48.89	0.01\\
48.9	0.01\\
48.91	0.01\\
48.92	0.01\\
48.93	0.01\\
48.94	0.01\\
48.95	0.01\\
48.96	0.01\\
48.97	0.01\\
48.98	0.01\\
48.99	0.01\\
49	0.01\\
49.01	0.01\\
49.02	0.01\\
49.03	0.01\\
49.04	0.01\\
49.05	0.01\\
49.06	0.01\\
49.07	0.01\\
49.08	0.01\\
49.09	0.01\\
49.1	0.01\\
49.11	0.01\\
49.12	0.01\\
49.13	0.01\\
49.14	0.01\\
49.15	0.01\\
49.16	0.01\\
49.17	0.01\\
49.18	0.01\\
49.19	0.01\\
49.2	0.01\\
49.21	0.01\\
49.22	0.01\\
49.23	0.01\\
49.24	0.01\\
49.25	0.01\\
49.26	0.01\\
49.27	0.01\\
49.28	0.01\\
49.29	0.01\\
49.3	0.01\\
49.31	0.01\\
49.32	0.01\\
49.33	0.01\\
49.34	0.01\\
49.35	0.01\\
49.36	0.01\\
49.37	0.01\\
49.38	0.01\\
49.39	0.01\\
49.4	0.01\\
49.41	0.01\\
49.42	0.01\\
49.43	0.01\\
49.44	0.01\\
49.45	0.01\\
49.46	0.01\\
49.47	0.01\\
49.48	0.01\\
49.49	0.01\\
49.5	0.01\\
49.51	0.01\\
49.52	0.01\\
49.53	0.01\\
49.54	0.01\\
49.55	0.01\\
49.56	0.01\\
49.57	0.01\\
49.58	0.01\\
49.59	0.01\\
49.6	0.01\\
49.61	0.01\\
49.62	0.01\\
49.63	0.01\\
49.64	0.01\\
49.65	0.01\\
49.66	0.01\\
49.67	0.01\\
49.68	0.01\\
49.69	0.01\\
49.7	0.01\\
49.71	0.01\\
49.72	0.01\\
49.73	0.01\\
49.74	0.01\\
49.75	0.01\\
49.76	0.01\\
49.77	0.01\\
49.78	0.01\\
49.79	0.01\\
49.8	0.01\\
49.81	0.01\\
49.82	0.01\\
49.83	0.01\\
49.84	0.01\\
49.85	0.01\\
49.86	0.01\\
49.87	0.01\\
49.88	0.01\\
49.89	0.01\\
49.9	0.01\\
49.91	0.01\\
49.92	0.01\\
49.93	0.01\\
49.94	0.01\\
49.95	0.01\\
49.96	0.01\\
49.97	0.01\\
49.98	0.01\\
49.99	0.01\\
50	0.01\\
50.01	0.01\\
50.02	0.01\\
50.03	0.01\\
50.04	0.01\\
50.05	0.01\\
50.06	0.01\\
50.07	0.01\\
50.08	0.01\\
50.09	0.01\\
50.1	0.01\\
50.11	0.01\\
50.12	0.01\\
50.13	0.01\\
50.14	0.01\\
50.15	0.01\\
50.16	0.01\\
50.17	0.01\\
50.18	0.01\\
50.19	0.01\\
50.2	0.01\\
50.21	0.01\\
50.22	0.01\\
50.23	0.01\\
50.24	0.01\\
50.25	0.01\\
50.26	0.01\\
50.27	0.01\\
50.28	0.01\\
50.29	0.01\\
50.3	0.01\\
50.31	0.01\\
50.32	0.01\\
50.33	0.01\\
50.34	0.01\\
50.35	0.01\\
50.36	0.01\\
50.37	0.01\\
50.38	0.01\\
50.39	0.01\\
50.4	0.01\\
50.41	0.01\\
50.42	0.01\\
50.43	0.01\\
50.44	0.01\\
50.45	0.01\\
50.46	0.01\\
50.47	0.01\\
50.48	0.01\\
50.49	0.01\\
50.5	0.01\\
50.51	0.01\\
50.52	0.01\\
50.53	0.01\\
50.54	0.01\\
50.55	0.01\\
50.56	0.01\\
50.57	0.01\\
50.58	0.01\\
50.59	0.01\\
50.6	0.01\\
50.61	0.01\\
50.62	0.01\\
50.63	0.01\\
50.64	0.01\\
50.65	0.01\\
50.66	0.01\\
50.67	0.01\\
50.68	0.01\\
50.69	0.01\\
50.7	0.01\\
50.71	0.01\\
50.72	0.01\\
50.73	0.01\\
50.74	0.01\\
50.75	0.01\\
50.76	0.01\\
50.77	0.01\\
50.78	0.01\\
50.79	0.01\\
50.8	0.01\\
50.81	0.01\\
50.82	0.01\\
50.83	0.01\\
50.84	0.01\\
50.85	0.01\\
50.86	0.01\\
50.87	0.01\\
50.88	0.01\\
50.89	0.01\\
50.9	0.01\\
50.91	0.01\\
50.92	0.01\\
50.93	0.01\\
50.94	0.01\\
50.95	0.01\\
50.96	0.01\\
50.97	0.01\\
50.98	0.01\\
50.99	0.01\\
51	0.01\\
51.01	0.01\\
51.02	0.01\\
51.03	0.01\\
51.04	0.01\\
51.05	0.01\\
51.06	0.01\\
51.07	0.01\\
51.08	0.01\\
51.09	0.01\\
51.1	0.01\\
51.11	0.01\\
51.12	0.01\\
51.13	0.01\\
51.14	0.01\\
51.15	0.01\\
51.16	0.01\\
51.17	0.01\\
51.18	0.01\\
51.19	0.01\\
51.2	0.01\\
51.21	0.01\\
51.22	0.01\\
51.23	0.01\\
51.24	0.01\\
51.25	0.01\\
51.26	0.01\\
51.27	0.01\\
51.28	0.01\\
51.29	0.01\\
51.3	0.01\\
51.31	0.01\\
51.32	0.01\\
51.33	0.01\\
51.34	0.01\\
51.35	0.01\\
51.36	0.01\\
51.37	0.01\\
51.38	0.01\\
51.39	0.01\\
51.4	0.01\\
51.41	0.01\\
51.42	0.01\\
51.43	0.01\\
51.44	0.01\\
51.45	0.01\\
51.46	0.01\\
51.47	0.01\\
51.48	0.01\\
51.49	0.01\\
51.5	0.01\\
51.51	0.01\\
51.52	0.01\\
51.53	0.01\\
51.54	0.01\\
51.55	0.01\\
51.56	0.01\\
51.57	0.01\\
51.58	0.01\\
51.59	0.01\\
51.6	0.01\\
51.61	0.01\\
51.62	0.01\\
51.63	0.01\\
51.64	0.01\\
51.65	0.01\\
51.66	0.01\\
51.67	0.01\\
51.68	0.01\\
51.69	0.01\\
51.7	0.01\\
51.71	0.01\\
51.72	0.01\\
51.73	0.01\\
51.74	0.01\\
51.75	0.01\\
51.76	0.01\\
51.77	0.01\\
51.78	0.01\\
51.79	0.01\\
51.8	0.01\\
51.81	0.01\\
51.82	0.01\\
51.83	0.01\\
51.84	0.01\\
51.85	0.01\\
51.86	0.01\\
51.87	0.01\\
51.88	0.01\\
51.89	0.01\\
51.9	0.01\\
51.91	0.01\\
51.92	0.01\\
51.93	0.01\\
51.94	0.01\\
51.95	0.01\\
51.96	0.01\\
51.97	0.01\\
51.98	0.01\\
51.99	0.01\\
52	0.01\\
52.01	0.01\\
52.02	0.01\\
52.03	0.01\\
52.04	0.01\\
52.05	0.01\\
52.06	0.01\\
52.07	0.01\\
52.08	0.01\\
52.09	0.01\\
52.1	0.01\\
52.11	0.01\\
52.12	0.01\\
52.13	0.01\\
52.14	0.01\\
52.15	0.01\\
52.16	0.01\\
52.17	0.01\\
52.18	0.01\\
52.19	0.01\\
52.2	0.01\\
52.21	0.01\\
52.22	0.01\\
52.23	0.01\\
52.24	0.01\\
52.25	0.01\\
52.26	0.01\\
52.27	0.01\\
52.28	0.01\\
52.29	0.01\\
52.3	0.01\\
52.31	0.01\\
52.32	0.01\\
52.33	0.01\\
52.34	0.01\\
52.35	0.01\\
52.36	0.01\\
52.37	0.01\\
52.38	0.01\\
52.39	0.01\\
52.4	0.01\\
52.41	0.01\\
52.42	0.01\\
52.43	0.01\\
52.44	0.01\\
52.45	0.01\\
52.46	0.01\\
52.47	0.01\\
52.48	0.01\\
52.49	0.01\\
52.5	0.01\\
52.51	0.01\\
52.52	0.01\\
52.53	0.01\\
52.54	0.01\\
52.55	0.01\\
52.56	0.01\\
52.57	0.01\\
52.58	0.01\\
52.59	0.01\\
52.6	0.01\\
52.61	0.01\\
52.62	0.01\\
52.63	0.01\\
52.64	0.01\\
52.65	0.01\\
52.66	0.01\\
52.67	0.01\\
52.68	0.01\\
52.69	0.01\\
52.7	0.01\\
52.71	0.01\\
52.72	0.01\\
52.73	0.01\\
52.74	0.01\\
52.75	0.01\\
52.76	0.01\\
52.77	0.01\\
52.78	0.01\\
52.79	0.01\\
52.8	0.01\\
52.81	0.01\\
52.82	0.01\\
52.83	0.01\\
52.84	0.01\\
52.85	0.01\\
52.86	0.01\\
52.87	0.01\\
52.88	0.01\\
52.89	0.01\\
52.9	0.01\\
52.91	0.01\\
52.92	0.01\\
52.93	0.01\\
52.94	0.01\\
52.95	0.01\\
52.96	0.01\\
52.97	0.01\\
52.98	0.01\\
52.99	0.01\\
53	0.01\\
53.01	0.01\\
53.02	0.01\\
53.03	0.01\\
53.04	0.01\\
53.05	0.01\\
53.06	0.01\\
53.07	0.01\\
53.08	0.01\\
53.09	0.01\\
53.1	0.01\\
53.11	0.01\\
53.12	0.01\\
53.13	0.01\\
53.14	0.01\\
53.15	0.01\\
53.16	0.01\\
53.17	0.01\\
53.18	0.01\\
53.19	0.01\\
53.2	0.01\\
53.21	0.01\\
53.22	0.01\\
53.23	0.01\\
53.24	0.01\\
53.25	0.01\\
53.26	0.01\\
53.27	0.01\\
53.28	0.01\\
53.29	0.01\\
53.3	0.01\\
53.31	0.01\\
53.32	0.01\\
53.33	0.01\\
53.34	0.01\\
53.35	0.01\\
53.36	0.01\\
53.37	0.01\\
53.38	0.01\\
53.39	0.01\\
53.4	0.01\\
53.41	0.01\\
53.42	0.01\\
53.43	0.01\\
53.44	0.01\\
53.45	0.01\\
53.46	0.01\\
53.47	0.01\\
53.48	0.01\\
53.49	0.01\\
53.5	0.01\\
53.51	0.01\\
53.52	0.01\\
53.53	0.01\\
53.54	0.01\\
53.55	0.01\\
53.56	0.01\\
53.57	0.01\\
53.58	0.01\\
53.59	0.01\\
53.6	0.01\\
53.61	0.01\\
53.62	0.01\\
53.63	0.01\\
53.64	0.01\\
53.65	0.01\\
53.66	0.01\\
53.67	0.01\\
53.68	0.01\\
53.69	0.01\\
53.7	0.01\\
53.71	0.01\\
53.72	0.01\\
53.73	0.01\\
53.74	0.01\\
53.75	0.01\\
53.76	0.01\\
53.77	0.01\\
53.78	0.01\\
53.79	0.01\\
53.8	0.01\\
53.81	0.01\\
53.82	0.01\\
53.83	0.01\\
53.84	0.01\\
53.85	0.01\\
53.86	0.01\\
53.87	0.01\\
53.88	0.01\\
53.89	0.01\\
53.9	0.01\\
53.91	0.01\\
53.92	0.01\\
53.93	0.01\\
53.94	0.01\\
53.95	0.01\\
53.96	0.01\\
53.97	0.01\\
53.98	0.01\\
53.99	0.01\\
54	0.01\\
54.01	0.01\\
54.02	0.01\\
54.03	0.01\\
54.04	0.01\\
54.05	0.01\\
54.06	0.01\\
54.07	0.01\\
54.08	0.01\\
54.09	0.01\\
54.1	0.01\\
54.11	0.01\\
54.12	0.01\\
54.13	0.01\\
54.14	0.01\\
54.15	0.01\\
54.16	0.01\\
54.17	0.01\\
54.18	0.01\\
54.19	0.01\\
54.2	0.01\\
54.21	0.01\\
54.22	0.01\\
54.23	0.01\\
54.24	0.01\\
54.25	0.01\\
54.26	0.01\\
54.27	0.01\\
54.28	0.01\\
54.29	0.01\\
54.3	0.01\\
54.31	0.01\\
54.32	0.01\\
54.33	0.01\\
54.34	0.01\\
54.35	0.01\\
54.36	0.01\\
54.37	0.01\\
54.38	0.01\\
54.39	0.01\\
54.4	0.01\\
54.41	0.01\\
54.42	0.01\\
54.43	0.01\\
54.44	0.01\\
54.45	0.01\\
54.46	0.01\\
54.47	0.01\\
54.48	0.01\\
54.49	0.01\\
54.5	0.01\\
54.51	0.01\\
54.52	0.01\\
54.53	0.01\\
54.54	0.01\\
54.55	0.01\\
54.56	0.01\\
54.57	0.01\\
54.58	0.01\\
54.59	0.01\\
54.6	0.01\\
54.61	0.01\\
54.62	0.01\\
54.63	0.01\\
54.64	0.01\\
54.65	0.01\\
54.66	0.01\\
54.67	0.01\\
54.68	0.01\\
54.69	0.01\\
54.7	0.01\\
54.71	0.01\\
54.72	0.01\\
54.73	0.01\\
54.74	0.01\\
54.75	0.01\\
54.76	0.01\\
54.77	0.01\\
54.78	0.01\\
54.79	0.01\\
54.8	0.01\\
54.81	0.01\\
54.82	0.01\\
54.83	0.01\\
54.84	0.01\\
54.85	0.01\\
54.86	0.01\\
54.87	0.01\\
54.88	0.01\\
54.89	0.01\\
54.9	0.01\\
54.91	0.01\\
54.92	0.01\\
54.93	0.01\\
54.94	0.01\\
54.95	0.01\\
54.96	0.01\\
54.97	0.01\\
54.98	0.01\\
54.99	0.01\\
55	0.01\\
55.01	0.01\\
55.02	0.01\\
55.03	0.01\\
55.04	0.01\\
55.05	0.01\\
55.06	0.01\\
55.07	0.01\\
55.08	0.01\\
55.09	0.01\\
55.1	0.01\\
55.11	0.01\\
55.12	0.01\\
55.13	0.01\\
55.14	0.01\\
55.15	0.01\\
55.16	0.01\\
55.17	0.01\\
55.18	0.01\\
55.19	0.01\\
55.2	0.01\\
55.21	0.01\\
55.22	0.01\\
55.23	0.01\\
55.24	0.01\\
55.25	0.01\\
55.26	0.01\\
55.27	0.01\\
55.28	0.01\\
55.29	0.01\\
55.3	0.01\\
55.31	0.01\\
55.32	0.01\\
55.33	0.01\\
55.34	0.01\\
55.35	0.01\\
55.36	0.01\\
55.37	0.01\\
55.38	0.01\\
55.39	0.01\\
55.4	0.01\\
55.41	0.01\\
55.42	0.01\\
55.43	0.01\\
55.44	0.01\\
55.45	0.01\\
55.46	0.01\\
55.47	0.01\\
55.48	0.01\\
55.49	0.01\\
55.5	0.01\\
55.51	0.01\\
55.52	0.01\\
55.53	0.01\\
55.54	0.01\\
55.55	0.01\\
55.56	0.01\\
55.57	0.01\\
55.58	0.01\\
55.59	0.01\\
55.6	0.01\\
55.61	0.01\\
55.62	0.01\\
55.63	0.01\\
55.64	0.01\\
55.65	0.01\\
55.66	0.01\\
55.67	0.01\\
55.68	0.01\\
55.69	0.01\\
55.7	0.01\\
55.71	0.01\\
55.72	0.01\\
55.73	0.01\\
55.74	0.01\\
55.75	0.01\\
55.76	0.01\\
55.77	0.01\\
55.78	0.01\\
55.79	0.01\\
55.8	0.01\\
55.81	0.01\\
55.82	0.01\\
55.83	0.01\\
55.84	0.01\\
55.85	0.01\\
55.86	0.01\\
55.87	0.01\\
55.88	0.01\\
55.89	0.01\\
55.9	0.01\\
55.91	0.01\\
55.92	0.01\\
55.93	0.01\\
55.94	0.01\\
55.95	0.01\\
55.96	0.01\\
55.97	0.01\\
55.98	0.01\\
55.99	0.01\\
56	0.01\\
56.01	0.01\\
56.02	0.01\\
56.03	0.01\\
56.04	0.01\\
56.05	0.01\\
56.06	0.01\\
56.07	0.01\\
56.08	0.01\\
56.09	0.01\\
56.1	0.01\\
56.11	0.01\\
56.12	0.01\\
56.13	0.01\\
56.14	0.01\\
56.15	0.01\\
56.16	0.01\\
56.17	0.01\\
56.18	0.01\\
56.19	0.01\\
56.2	0.01\\
56.21	0.01\\
56.22	0.01\\
56.23	0.01\\
56.24	0.01\\
56.25	0.01\\
56.26	0.01\\
56.27	0.01\\
56.28	0.01\\
56.29	0.01\\
56.3	0.01\\
56.31	0.01\\
56.32	0.01\\
56.33	0.01\\
56.34	0.01\\
56.35	0.01\\
56.36	0.01\\
56.37	0.01\\
56.38	0.01\\
56.39	0.01\\
56.4	0.01\\
56.41	0.01\\
56.42	0.01\\
56.43	0.01\\
56.44	0.01\\
56.45	0.01\\
56.46	0.01\\
56.47	0.01\\
56.48	0.01\\
56.49	0.01\\
56.5	0.01\\
56.51	0.01\\
56.52	0.01\\
56.53	0.01\\
56.54	0.01\\
56.55	0.01\\
56.56	0.01\\
56.57	0.01\\
56.58	0.01\\
56.59	0.01\\
56.6	0.01\\
56.61	0.01\\
56.62	0.01\\
56.63	0.01\\
56.64	0.01\\
56.65	0.01\\
56.66	0.01\\
56.67	0.01\\
56.68	0.01\\
56.69	0.01\\
56.7	0.01\\
56.71	0.01\\
56.72	0.01\\
56.73	0.01\\
56.74	0.01\\
56.75	0.01\\
56.76	0.01\\
56.77	0.01\\
56.78	0.01\\
56.79	0.01\\
56.8	0.01\\
56.81	0.01\\
56.82	0.01\\
56.83	0.01\\
56.84	0.01\\
56.85	0.01\\
56.86	0.01\\
56.87	0.01\\
56.88	0.01\\
56.89	0.01\\
56.9	0.01\\
56.91	0.01\\
56.92	0.01\\
56.93	0.01\\
56.94	0.01\\
56.95	0.01\\
56.96	0.01\\
56.97	0.01\\
56.98	0.01\\
56.99	0.01\\
57	0.01\\
57.01	0.01\\
57.02	0.01\\
57.03	0.01\\
57.04	0.01\\
57.05	0.01\\
57.06	0.01\\
57.07	0.01\\
57.08	0.01\\
57.09	0.01\\
57.1	0.01\\
57.11	0.01\\
57.12	0.01\\
57.13	0.01\\
57.14	0.01\\
57.15	0.01\\
57.16	0.01\\
57.17	0.01\\
57.18	0.01\\
57.19	0.01\\
57.2	0.01\\
57.21	0.01\\
57.22	0.01\\
57.23	0.01\\
57.24	0.01\\
57.25	0.01\\
57.26	0.01\\
57.27	0.01\\
57.28	0.01\\
57.29	0.01\\
57.3	0.01\\
57.31	0.01\\
57.32	0.01\\
57.33	0.01\\
57.34	0.01\\
57.35	0.01\\
57.36	0.01\\
57.37	0.01\\
57.38	0.01\\
57.39	0.01\\
57.4	0.01\\
57.41	0.01\\
57.42	0.01\\
57.43	0.01\\
57.44	0.01\\
57.45	0.01\\
57.46	0.01\\
57.47	0.01\\
57.48	0.01\\
57.49	0.01\\
57.5	0.01\\
57.51	0.01\\
57.52	0.01\\
57.53	0.01\\
57.54	0.01\\
57.55	0.01\\
57.56	0.01\\
57.57	0.01\\
57.58	0.01\\
57.59	0.01\\
57.6	0.01\\
57.61	0.01\\
57.62	0.01\\
57.63	0.01\\
57.64	0.01\\
57.65	0.01\\
57.66	0.01\\
57.67	0.01\\
57.68	0.01\\
57.69	0.01\\
57.7	0.01\\
57.71	0.01\\
57.72	0.01\\
57.73	0.01\\
57.74	0.01\\
57.75	0.01\\
57.76	0.01\\
57.77	0.01\\
57.78	0.01\\
57.79	0.01\\
57.8	0.01\\
57.81	0.01\\
57.82	0.01\\
57.83	0.01\\
57.84	0.01\\
57.85	0.01\\
57.86	0.01\\
57.87	0.01\\
57.88	0.01\\
57.89	0.01\\
57.9	0.01\\
57.91	0.01\\
57.92	0.01\\
57.93	0.01\\
57.94	0.01\\
57.95	0.01\\
57.96	0.01\\
57.97	0.01\\
57.98	0.01\\
57.99	0.01\\
58	0.01\\
58.01	0.01\\
58.02	0.01\\
58.03	0.01\\
58.04	0.01\\
58.05	0.01\\
58.06	0.01\\
58.07	0.01\\
58.08	0.01\\
58.09	0.01\\
58.1	0.01\\
58.11	0.01\\
58.12	0.01\\
58.13	0.01\\
58.14	0.01\\
58.15	0.01\\
58.16	0.01\\
58.17	0.01\\
58.18	0.01\\
58.19	0.01\\
58.2	0.01\\
58.21	0.01\\
58.22	0.01\\
58.23	0.01\\
58.24	0.01\\
58.25	0.01\\
58.26	0.01\\
58.27	0.01\\
58.28	0.01\\
58.29	0.01\\
58.3	0.01\\
58.31	0.01\\
58.32	0.01\\
58.33	0.01\\
58.34	0.01\\
58.35	0.01\\
58.36	0.01\\
58.37	0.01\\
58.38	0.01\\
58.39	0.01\\
58.4	0.01\\
58.41	0.01\\
58.42	0.01\\
58.43	0.01\\
58.44	0.01\\
58.45	0.01\\
58.46	0.01\\
58.47	0.01\\
58.48	0.01\\
58.49	0.01\\
58.5	0.01\\
58.51	0.01\\
58.52	0.01\\
58.53	0.01\\
58.54	0.01\\
58.55	0.01\\
58.56	0.01\\
58.57	0.01\\
58.58	0.01\\
58.59	0.01\\
58.6	0.01\\
58.61	0.01\\
58.62	0.01\\
58.63	0.01\\
58.64	0.01\\
58.65	0.01\\
58.66	0.01\\
58.67	0.01\\
58.68	0.01\\
58.69	0.01\\
58.7	0.01\\
58.71	0.01\\
58.72	0.01\\
58.73	0.01\\
58.74	0.01\\
58.75	0.01\\
58.76	0.01\\
58.77	0.01\\
58.78	0.01\\
58.79	0.01\\
58.8	0.01\\
58.81	0.01\\
58.82	0.01\\
58.83	0.01\\
58.84	0.01\\
58.85	0.01\\
58.86	0.01\\
58.87	0.01\\
58.88	0.01\\
58.89	0.01\\
58.9	0.01\\
58.91	0.01\\
58.92	0.01\\
58.93	0.01\\
58.94	0.01\\
58.95	0.01\\
58.96	0.01\\
58.97	0.01\\
58.98	0.01\\
58.99	0.01\\
59	0.01\\
59.01	0.01\\
59.02	0.01\\
59.03	0.01\\
59.04	0.01\\
59.05	0.01\\
59.06	0.01\\
59.07	0.01\\
59.08	0.01\\
59.09	0.01\\
59.1	0.01\\
59.11	0.01\\
59.12	0.01\\
59.13	0.01\\
59.14	0.01\\
59.15	0.01\\
59.16	0.01\\
59.17	0.01\\
59.18	0.01\\
59.19	0.01\\
59.2	0.01\\
59.21	0.01\\
59.22	0.01\\
59.23	0.01\\
59.24	0.01\\
59.25	0.01\\
59.26	0.01\\
59.27	0.01\\
59.28	0.01\\
59.29	0.01\\
59.3	0.01\\
59.31	0.01\\
59.32	0.01\\
59.33	0.01\\
59.34	0.01\\
59.35	0.01\\
59.36	0.01\\
59.37	0.01\\
59.38	0.01\\
59.39	0.01\\
59.4	0.01\\
59.41	0.01\\
59.42	0.01\\
59.43	0.01\\
59.44	0.01\\
59.45	0.01\\
59.46	0.01\\
59.47	0.01\\
59.48	0.01\\
59.49	0.01\\
59.5	0.01\\
59.51	0.01\\
59.52	0.01\\
59.53	0.01\\
59.54	0.01\\
59.55	0.01\\
59.56	0.01\\
59.57	0.01\\
59.58	0.01\\
59.59	0.01\\
59.6	0.01\\
59.61	0.01\\
59.62	0.01\\
59.63	0.01\\
59.64	0.01\\
59.65	0.01\\
59.66	0.01\\
59.67	0.01\\
59.68	0.01\\
59.69	0.01\\
59.7	0.01\\
59.71	0.01\\
59.72	0.01\\
59.73	0.01\\
59.74	0.01\\
59.75	0.01\\
59.76	0.01\\
59.77	0.01\\
59.78	0.01\\
59.79	0.01\\
59.8	0.01\\
59.81	0.01\\
59.82	0.01\\
59.83	0.01\\
59.84	0.01\\
59.85	0.01\\
59.86	0.01\\
59.87	0.01\\
59.88	0.01\\
59.89	0.01\\
59.9	0.01\\
59.91	0.01\\
59.92	0.01\\
59.93	0.01\\
59.94	0.01\\
59.95	0.01\\
59.96	0.01\\
59.97	0.01\\
59.98	0.01\\
59.99	0.01\\
60	0.01\\
60.01	0.01\\
60.02	0.01\\
60.03	0.01\\
60.04	0.01\\
60.05	0.01\\
60.06	0.01\\
60.07	0.01\\
60.08	0.01\\
60.09	0.01\\
60.1	0.01\\
60.11	0.01\\
60.12	0.01\\
60.13	0.01\\
60.14	0.01\\
60.15	0.01\\
60.16	0.01\\
60.17	0.01\\
60.18	0.01\\
60.19	0.01\\
60.2	0.01\\
60.21	0.01\\
60.22	0.01\\
60.23	0.01\\
60.24	0.01\\
60.25	0.01\\
60.26	0.01\\
60.27	0.01\\
60.28	0.01\\
60.29	0.01\\
60.3	0.01\\
60.31	0.01\\
60.32	0.01\\
60.33	0.01\\
60.34	0.01\\
60.35	0.01\\
60.36	0.01\\
60.37	0.01\\
60.38	0.01\\
60.39	0.01\\
60.4	0.01\\
60.41	0.01\\
60.42	0.01\\
60.43	0.01\\
60.44	0.01\\
60.45	0.01\\
60.46	0.01\\
60.47	0.01\\
60.48	0.01\\
60.49	0.01\\
60.5	0.01\\
60.51	0.01\\
60.52	0.01\\
60.53	0.01\\
60.54	0.01\\
60.55	0.01\\
60.56	0.01\\
60.57	0.01\\
60.58	0.01\\
60.59	0.01\\
60.6	0.01\\
60.61	0.01\\
60.62	0.01\\
60.63	0.01\\
60.64	0.01\\
60.65	0.01\\
60.66	0.01\\
60.67	0.01\\
60.68	0.01\\
60.69	0.01\\
60.7	0.01\\
60.71	0.01\\
60.72	0.01\\
60.73	0.01\\
60.74	0.01\\
60.75	0.01\\
60.76	0.01\\
60.77	0.01\\
60.78	0.01\\
60.79	0.01\\
60.8	0.01\\
60.81	0.01\\
60.82	0.01\\
60.83	0.01\\
60.84	0.01\\
60.85	0.01\\
60.86	0.01\\
60.87	0.01\\
60.88	0.01\\
60.89	0.01\\
60.9	0.01\\
60.91	0.01\\
60.92	0.01\\
60.93	0.01\\
60.94	0.01\\
60.95	0.01\\
60.96	0.01\\
60.97	0.01\\
60.98	0.01\\
60.99	0.01\\
61	0.01\\
61.01	0.01\\
61.02	0.01\\
61.03	0.01\\
61.04	0.01\\
61.05	0.01\\
61.06	0.01\\
61.07	0.01\\
61.08	0.01\\
61.09	0.01\\
61.1	0.01\\
61.11	0.01\\
61.12	0.01\\
61.13	0.01\\
61.14	0.01\\
61.15	0.01\\
61.16	0.01\\
61.17	0.01\\
61.18	0.01\\
61.19	0.01\\
61.2	0.01\\
61.21	0.01\\
61.22	0.01\\
61.23	0.01\\
61.24	0.01\\
61.25	0.01\\
61.26	0.01\\
61.27	0.01\\
61.28	0.01\\
61.29	0.01\\
61.3	0.01\\
61.31	0.01\\
61.32	0.01\\
61.33	0.01\\
61.34	0.01\\
61.35	0.01\\
61.36	0.01\\
61.37	0.01\\
61.38	0.01\\
61.39	0.01\\
61.4	0.01\\
61.41	0.01\\
61.42	0.01\\
61.43	0.01\\
61.44	0.01\\
61.45	0.01\\
61.46	0.01\\
61.47	0.01\\
61.48	0.01\\
61.49	0.01\\
61.5	0.01\\
61.51	0.01\\
61.52	0.01\\
61.53	0.01\\
61.54	0.01\\
61.55	0.01\\
61.56	0.01\\
61.57	0.01\\
61.58	0.01\\
61.59	0.01\\
61.6	0.01\\
61.61	0.01\\
61.62	0.01\\
61.63	0.01\\
61.64	0.01\\
61.65	0.01\\
61.66	0.01\\
61.67	0.01\\
61.68	0.01\\
61.69	0.01\\
61.7	0.01\\
61.71	0.01\\
61.72	0.01\\
61.73	0.01\\
61.74	0.01\\
61.75	0.01\\
61.76	0.01\\
61.77	0.01\\
61.78	0.01\\
61.79	0.01\\
61.8	0.01\\
61.81	0.01\\
61.82	0.01\\
61.83	0.01\\
61.84	0.01\\
61.85	0.01\\
61.86	0.01\\
61.87	0.01\\
61.88	0.01\\
61.89	0.01\\
61.9	0.01\\
61.91	0.01\\
61.92	0.01\\
61.93	0.01\\
61.94	0.01\\
61.95	0.01\\
61.96	0.01\\
61.97	0.01\\
61.98	0.01\\
61.99	0.01\\
62	0.01\\
62.01	0.01\\
62.02	0.01\\
62.03	0.01\\
62.04	0.01\\
62.05	0.01\\
62.06	0.01\\
62.07	0.01\\
62.08	0.01\\
62.09	0.01\\
62.1	0.01\\
62.11	0.01\\
62.12	0.01\\
62.13	0.01\\
62.14	0.01\\
62.15	0.01\\
62.16	0.01\\
62.17	0.01\\
62.18	0.01\\
62.19	0.01\\
62.2	0.01\\
62.21	0.01\\
62.22	0.01\\
62.23	0.01\\
62.24	0.01\\
62.25	0.01\\
62.26	0.01\\
62.27	0.01\\
62.28	0.01\\
62.29	0.01\\
62.3	0.01\\
62.31	0.01\\
62.32	0.01\\
62.33	0.01\\
62.34	0.01\\
62.35	0.01\\
62.36	0.01\\
62.37	0.01\\
62.38	0.01\\
62.39	0.01\\
62.4	0.01\\
62.41	0.01\\
62.42	0.01\\
62.43	0.01\\
62.44	0.01\\
62.45	0.01\\
62.46	0.01\\
62.47	0.01\\
62.48	0.01\\
62.49	0.01\\
62.5	0.01\\
62.51	0.01\\
62.52	0.01\\
62.53	0.01\\
62.54	0.01\\
62.55	0.01\\
62.56	0.01\\
62.57	0.01\\
62.58	0.01\\
62.59	0.01\\
62.6	0.01\\
62.61	0.01\\
62.62	0.01\\
62.63	0.01\\
62.64	0.01\\
62.65	0.01\\
62.66	0.01\\
62.67	0.01\\
62.68	0.01\\
62.69	0.01\\
62.7	0.01\\
62.71	0.01\\
62.72	0.01\\
62.73	0.01\\
62.74	0.01\\
62.75	0.01\\
62.76	0.01\\
62.77	0.01\\
62.78	0.01\\
62.79	0.01\\
62.8	0.01\\
62.81	0.01\\
62.82	0.01\\
62.83	0.01\\
62.84	0.01\\
62.85	0.01\\
62.86	0.01\\
62.87	0.01\\
62.88	0.01\\
62.89	0.01\\
62.9	0.01\\
62.91	0.01\\
62.92	0.01\\
62.93	0.01\\
62.94	0.01\\
62.95	0.01\\
62.96	0.01\\
62.97	0.01\\
62.98	0.01\\
62.99	0.01\\
63	0.01\\
63.01	0.01\\
63.02	0.01\\
63.03	0.01\\
63.04	0.01\\
63.05	0.01\\
63.06	0.01\\
63.07	0.01\\
63.08	0.01\\
63.09	0.01\\
63.1	0.01\\
63.11	0.01\\
63.12	0.01\\
63.13	0.01\\
63.14	0.01\\
63.15	0.01\\
63.16	0.01\\
63.17	0.01\\
63.18	0.01\\
63.19	0.01\\
63.2	0.01\\
63.21	0.01\\
63.22	0.01\\
63.23	0.01\\
63.24	0.01\\
63.25	0.01\\
63.26	0.01\\
63.27	0.01\\
63.28	0.01\\
63.29	0.01\\
63.3	0.01\\
63.31	0.01\\
63.32	0.01\\
63.33	0.01\\
63.34	0.01\\
63.35	0.01\\
63.36	0.01\\
63.37	0.01\\
63.38	0.01\\
63.39	0.01\\
63.4	0.01\\
63.41	0.01\\
63.42	0.01\\
63.43	0.01\\
63.44	0.01\\
63.45	0.01\\
63.46	0.01\\
63.47	0.01\\
63.48	0.01\\
63.49	0.01\\
63.5	0.01\\
63.51	0.01\\
63.52	0.01\\
63.53	0.01\\
63.54	0.01\\
63.55	0.01\\
63.56	0.01\\
63.57	0.01\\
63.58	0.01\\
63.59	0.01\\
63.6	0.01\\
63.61	0.01\\
63.62	0.01\\
63.63	0.01\\
63.64	0.01\\
63.65	0.01\\
63.66	0.01\\
63.67	0.01\\
63.68	0.01\\
63.69	0.01\\
63.7	0.01\\
63.71	0.01\\
63.72	0.01\\
63.73	0.01\\
63.74	0.01\\
63.75	0.01\\
63.76	0.01\\
63.77	0.01\\
63.78	0.01\\
63.79	0.01\\
63.8	0.01\\
63.81	0.01\\
63.82	0.01\\
63.83	0.01\\
63.84	0.01\\
63.85	0.01\\
63.86	0.01\\
63.87	0.01\\
63.88	0.01\\
63.89	0.01\\
63.9	0.01\\
63.91	0.01\\
63.92	0.01\\
63.93	0.01\\
63.94	0.01\\
63.95	0.01\\
63.96	0.01\\
63.97	0.01\\
63.98	0.01\\
63.99	0.01\\
64	0.01\\
64.01	0.01\\
64.02	0.01\\
64.03	0.01\\
64.04	0.01\\
64.05	0.01\\
64.06	0.01\\
64.07	0.01\\
64.08	0.01\\
64.09	0.01\\
64.1	0.01\\
64.11	0.01\\
64.12	0.01\\
64.13	0.01\\
64.14	0.01\\
64.15	0.01\\
64.16	0.01\\
64.17	0.01\\
64.18	0.01\\
64.19	0.01\\
64.2	0.01\\
64.21	0.01\\
64.22	0.01\\
64.23	0.01\\
64.24	0.01\\
64.25	0.01\\
64.26	0.01\\
64.27	0.01\\
64.28	0.01\\
64.29	0.01\\
64.3	0.01\\
64.31	0.01\\
64.32	0.01\\
64.33	0.01\\
64.34	0.01\\
64.35	0.01\\
64.36	0.01\\
64.37	0.01\\
64.38	0.01\\
64.39	0.01\\
64.4	0.01\\
64.41	0.01\\
64.42	0.01\\
64.43	0.01\\
64.44	0.01\\
64.45	0.01\\
64.46	0.01\\
64.47	0.01\\
64.48	0.01\\
64.49	0.01\\
64.5	0.01\\
64.51	0.01\\
64.52	0.01\\
64.53	0.01\\
64.54	0.01\\
64.55	0.01\\
64.56	0.01\\
64.57	0.01\\
64.58	0.01\\
64.59	0.01\\
64.6	0.01\\
64.61	0.01\\
64.62	0.01\\
64.63	0.01\\
64.64	0.01\\
64.65	0.01\\
64.66	0.01\\
64.67	0.01\\
64.68	0.01\\
64.69	0.01\\
64.7	0.01\\
64.71	0.01\\
64.72	0.01\\
64.73	0.01\\
64.74	0.01\\
64.75	0.01\\
64.76	0.01\\
64.77	0.01\\
64.78	0.01\\
64.79	0.01\\
64.8	0.01\\
64.81	0.01\\
64.82	0.01\\
64.83	0.01\\
64.84	0.01\\
64.85	0.01\\
64.86	0.01\\
64.87	0.01\\
64.88	0.01\\
64.89	0.01\\
64.9	0.01\\
64.91	0.01\\
64.92	0.01\\
64.93	0.01\\
64.94	0.01\\
64.95	0.01\\
64.96	0.01\\
64.97	0.01\\
64.98	0.01\\
64.99	0.01\\
65	0.01\\
65.01	0.01\\
65.02	0.01\\
65.03	0.01\\
65.04	0.01\\
65.05	0.01\\
65.06	0.01\\
65.07	0.01\\
65.08	0.01\\
65.09	0.01\\
65.1	0.01\\
65.11	0.01\\
65.12	0.01\\
65.13	0.01\\
65.14	0.01\\
65.15	0.01\\
65.16	0.01\\
65.17	0.01\\
65.18	0.01\\
65.19	0.01\\
65.2	0.01\\
65.21	0.01\\
65.22	0.01\\
65.23	0.01\\
65.24	0.01\\
65.25	0.01\\
65.26	0.01\\
65.27	0.01\\
65.28	0.01\\
65.29	0.01\\
65.3	0.01\\
65.31	0.01\\
65.32	0.01\\
65.33	0.01\\
65.34	0.01\\
65.35	0.01\\
65.36	0.01\\
65.37	0.01\\
65.38	0.01\\
65.39	0.01\\
65.4	0.01\\
65.41	0.01\\
65.42	0.01\\
65.43	0.01\\
65.44	0.01\\
65.45	0.01\\
65.46	0.01\\
65.47	0.01\\
65.48	0.01\\
65.49	0.01\\
65.5	0.01\\
65.51	0.01\\
65.52	0.01\\
65.53	0.01\\
65.54	0.01\\
65.55	0.01\\
65.56	0.01\\
65.57	0.01\\
65.58	0.01\\
65.59	0.01\\
65.6	0.01\\
65.61	0.01\\
65.62	0.01\\
65.63	0.01\\
65.64	0.01\\
65.65	0.01\\
65.66	0.01\\
65.67	0.01\\
65.68	0.01\\
65.69	0.01\\
65.7	0.01\\
65.71	0.01\\
65.72	0.01\\
65.73	0.01\\
65.74	0.01\\
65.75	0.01\\
65.76	0.01\\
65.77	0.01\\
65.78	0.01\\
65.79	0.01\\
65.8	0.01\\
65.81	0.01\\
65.82	0.01\\
65.83	0.01\\
65.84	0.01\\
65.85	0.01\\
65.86	0.01\\
65.87	0.01\\
65.88	0.01\\
65.89	0.01\\
65.9	0.01\\
65.91	0.01\\
65.92	0.01\\
65.93	0.01\\
65.94	0.01\\
65.95	0.01\\
65.96	0.01\\
65.97	0.01\\
65.98	0.01\\
65.99	0.01\\
66	0.01\\
66.01	0.01\\
66.02	0.01\\
66.03	0.01\\
66.04	0.01\\
66.05	0.01\\
66.06	0.01\\
66.07	0.01\\
66.08	0.01\\
66.09	0.01\\
66.1	0.01\\
66.11	0.01\\
66.12	0.01\\
66.13	0.01\\
66.14	0.01\\
66.15	0.01\\
66.16	0.01\\
66.17	0.01\\
66.18	0.01\\
66.19	0.01\\
66.2	0.01\\
66.21	0.01\\
66.22	0.01\\
66.23	0.01\\
66.24	0.01\\
66.25	0.01\\
66.26	0.01\\
66.27	0.01\\
66.28	0.01\\
66.29	0.01\\
66.3	0.01\\
66.31	0.01\\
66.32	0.01\\
66.33	0.01\\
66.34	0.01\\
66.35	0.01\\
66.36	0.01\\
66.37	0.01\\
66.38	0.01\\
66.39	0.01\\
66.4	0.01\\
66.41	0.01\\
66.42	0.01\\
66.43	0.01\\
66.44	0.01\\
66.45	0.01\\
66.46	0.01\\
66.47	0.01\\
66.48	0.01\\
66.49	0.01\\
66.5	0.01\\
66.51	0.01\\
66.52	0.01\\
66.53	0.01\\
66.54	0.01\\
66.55	0.01\\
66.56	0.01\\
66.57	0.01\\
66.58	0.01\\
66.59	0.01\\
66.6	0.01\\
66.61	0.01\\
66.62	0.01\\
66.63	0.01\\
66.64	0.01\\
66.65	0.01\\
66.66	0.01\\
66.67	0.01\\
66.68	0.01\\
66.69	0.01\\
66.7	0.01\\
66.71	0.01\\
66.72	0.01\\
66.73	0.01\\
66.74	0.01\\
66.75	0.01\\
66.76	0.01\\
66.77	0.01\\
66.78	0.01\\
66.79	0.01\\
66.8	0.01\\
66.81	0.01\\
66.82	0.01\\
66.83	0.01\\
66.84	0.01\\
66.85	0.01\\
66.86	0.01\\
66.87	0.01\\
66.88	0.01\\
66.89	0.01\\
66.9	0.01\\
66.91	0.01\\
66.92	0.01\\
66.93	0.01\\
66.94	0.01\\
66.95	0.01\\
66.96	0.01\\
66.97	0.01\\
66.98	0.01\\
66.99	0.01\\
67	0.01\\
67.01	0.01\\
67.02	0.01\\
67.03	0.01\\
67.04	0.01\\
67.05	0.01\\
67.06	0.01\\
67.07	0.01\\
67.08	0.01\\
67.09	0.01\\
67.1	0.01\\
67.11	0.01\\
67.12	0.01\\
67.13	0.01\\
67.14	0.01\\
67.15	0.01\\
67.16	0.01\\
67.17	0.01\\
67.18	0.01\\
67.19	0.01\\
67.2	0.01\\
67.21	0.01\\
67.22	0.01\\
67.23	0.01\\
67.24	0.01\\
67.25	0.01\\
67.26	0.01\\
67.27	0.01\\
67.28	0.01\\
67.29	0.01\\
67.3	0.01\\
67.31	0.01\\
67.32	0.01\\
67.33	0.01\\
67.34	0.01\\
67.35	0.01\\
67.36	0.01\\
67.37	0.01\\
67.38	0.01\\
67.39	0.01\\
67.4	0.01\\
67.41	0.01\\
67.42	0.01\\
67.43	0.01\\
67.44	0.01\\
67.45	0.01\\
67.46	0.01\\
67.47	0.01\\
67.48	0.01\\
67.49	0.01\\
67.5	0.01\\
67.51	0.01\\
67.52	0.01\\
67.53	0.01\\
67.54	0.01\\
67.55	0.01\\
67.56	0.01\\
67.57	0.01\\
67.58	0.01\\
67.59	0.01\\
67.6	0.01\\
67.61	0.01\\
67.62	0.01\\
67.63	0.01\\
67.64	0.01\\
67.65	0.01\\
67.66	0.01\\
67.67	0.01\\
67.68	0.01\\
67.69	0.01\\
67.7	0.01\\
67.71	0.01\\
67.72	0.01\\
67.73	0.01\\
67.74	0.01\\
67.75	0.01\\
67.76	0.01\\
67.77	0.01\\
67.78	0.01\\
67.79	0.01\\
67.8	0.01\\
67.81	0.01\\
67.82	0.01\\
67.83	0.01\\
67.84	0.01\\
67.85	0.01\\
67.86	0.01\\
67.87	0.01\\
67.88	0.01\\
67.89	0.01\\
67.9	0.01\\
67.91	0.01\\
67.92	0.01\\
67.93	0.01\\
67.94	0.01\\
67.95	0.01\\
67.96	0.01\\
67.97	0.01\\
67.98	0.01\\
67.99	0.01\\
68	0.01\\
68.01	0.01\\
68.02	0.01\\
68.03	0.01\\
68.04	0.01\\
68.05	0.01\\
68.06	0.01\\
68.07	0.01\\
68.08	0.01\\
68.09	0.01\\
68.1	0.01\\
68.11	0.01\\
68.12	0.01\\
68.13	0.01\\
68.14	0.01\\
68.15	0.01\\
68.16	0.01\\
68.17	0.01\\
68.18	0.01\\
68.19	0.01\\
68.2	0.01\\
68.21	0.01\\
68.22	0.01\\
68.23	0.01\\
68.24	0.01\\
68.25	0.01\\
68.26	0.01\\
68.27	0.01\\
68.28	0.01\\
68.29	0.01\\
68.3	0.01\\
68.31	0.01\\
68.32	0.01\\
68.33	0.01\\
68.34	0.01\\
68.35	0.01\\
68.36	0.01\\
68.37	0.01\\
68.38	0.01\\
68.39	0.01\\
68.4	0.01\\
68.41	0.01\\
68.42	0.01\\
68.43	0.01\\
68.44	0.01\\
68.45	0.01\\
68.46	0.01\\
68.47	0.01\\
68.48	0.01\\
68.49	0.01\\
68.5	0.01\\
68.51	0.01\\
68.52	0.01\\
68.53	0.01\\
68.54	0.01\\
68.55	0.01\\
68.56	0.01\\
68.57	0.01\\
68.58	0.01\\
68.59	0.01\\
68.6	0.01\\
68.61	0.01\\
68.62	0.01\\
68.63	0.01\\
68.64	0.01\\
68.65	0.01\\
68.66	0.01\\
68.67	0.01\\
68.68	0.01\\
68.69	0.01\\
68.7	0.01\\
68.71	0.01\\
68.72	0.01\\
68.73	0.01\\
68.74	0.01\\
68.75	0.01\\
68.76	0.01\\
68.77	0.01\\
68.78	0.01\\
68.79	0.01\\
68.8	0.01\\
68.81	0.01\\
68.82	0.01\\
68.83	0.01\\
68.84	0.01\\
68.85	0.01\\
68.86	0.01\\
68.87	0.01\\
68.88	0.01\\
68.89	0.01\\
68.9	0.01\\
68.91	0.01\\
68.92	0.01\\
68.93	0.01\\
68.94	0.01\\
68.95	0.01\\
68.96	0.01\\
68.97	0.01\\
68.98	0.01\\
68.99	0.01\\
69	0.01\\
69.01	0.01\\
69.02	0.01\\
69.03	0.01\\
69.04	0.01\\
69.05	0.01\\
69.06	0.01\\
69.07	0.01\\
69.08	0.01\\
69.09	0.01\\
69.1	0.01\\
69.11	0.01\\
69.12	0.01\\
69.13	0.01\\
69.14	0.01\\
69.15	0.01\\
69.16	0.01\\
69.17	0.01\\
69.18	0.01\\
69.19	0.01\\
69.2	0.01\\
69.21	0.01\\
69.22	0.01\\
69.23	0.01\\
69.24	0.01\\
69.25	0.01\\
69.26	0.01\\
69.27	0.01\\
69.28	0.01\\
69.29	0.01\\
69.3	0.01\\
69.31	0.01\\
69.32	0.01\\
69.33	0.01\\
69.34	0.01\\
69.35	0.01\\
69.36	0.01\\
69.37	0.01\\
69.38	0.01\\
69.39	0.01\\
69.4	0.01\\
69.41	0.01\\
69.42	0.01\\
69.43	0.01\\
69.44	0.01\\
69.45	0.01\\
69.46	0.01\\
69.47	0.01\\
69.48	0.01\\
69.49	0.01\\
69.5	0.01\\
69.51	0.01\\
69.52	0.01\\
69.53	0.01\\
69.54	0.01\\
69.55	0.01\\
69.56	0.01\\
69.57	0.01\\
69.58	0.01\\
69.59	0.01\\
69.6	0.01\\
69.61	0.01\\
69.62	0.01\\
69.63	0.01\\
69.64	0.01\\
69.65	0.01\\
69.66	0.01\\
69.67	0.01\\
69.68	0.01\\
69.69	0.01\\
69.7	0.01\\
69.71	0.01\\
69.72	0.01\\
69.73	0.01\\
69.74	0.01\\
69.75	0.01\\
69.76	0.01\\
69.77	0.01\\
69.78	0.01\\
69.79	0.01\\
69.8	0.01\\
69.81	0.01\\
69.82	0.01\\
69.83	0.01\\
69.84	0.01\\
69.85	0.01\\
69.86	0.01\\
69.87	0.01\\
69.88	0.01\\
69.89	0.01\\
69.9	0.01\\
69.91	0.01\\
69.92	0.01\\
69.93	0.01\\
69.94	0.01\\
69.95	0.01\\
69.96	0.01\\
69.97	0.01\\
69.98	0.01\\
69.99	0.01\\
70	0.01\\
70.01	0.01\\
70.02	0.01\\
70.03	0.01\\
70.04	0.01\\
70.05	0.01\\
70.06	0.01\\
70.07	0.01\\
70.08	0.01\\
70.09	0.01\\
70.1	0.01\\
70.11	0.01\\
70.12	0.01\\
70.13	0.01\\
70.14	0.01\\
70.15	0.01\\
70.16	0.01\\
70.17	0.01\\
70.18	0.01\\
70.19	0.01\\
70.2	0.01\\
70.21	0.01\\
70.22	0.01\\
70.23	0.01\\
70.24	0.01\\
70.25	0.01\\
70.26	0.01\\
70.27	0.01\\
70.28	0.01\\
70.29	0.01\\
70.3	0.01\\
70.31	0.01\\
70.32	0.01\\
70.33	0.01\\
70.34	0.01\\
70.35	0.01\\
70.36	0.01\\
70.37	0.01\\
70.38	0.01\\
70.39	0.01\\
70.4	0.01\\
70.41	0.01\\
70.42	0.01\\
70.43	0.01\\
70.44	0.01\\
70.45	0.01\\
70.46	0.01\\
70.47	0.01\\
70.48	0.01\\
70.49	0.01\\
70.5	0.01\\
70.51	0.01\\
70.52	0.01\\
70.53	0.01\\
70.54	0.01\\
70.55	0.01\\
70.56	0.01\\
70.57	0.01\\
70.58	0.01\\
70.59	0.01\\
70.6	0.01\\
70.61	0.01\\
70.62	0.01\\
70.63	0.01\\
70.64	0.01\\
70.65	0.01\\
70.66	0.01\\
70.67	0.01\\
70.68	0.01\\
70.69	0.01\\
70.7	0.01\\
70.71	0.01\\
70.72	0.01\\
70.73	0.01\\
70.74	0.01\\
70.75	0.01\\
70.76	0.01\\
70.77	0.01\\
70.78	0.01\\
70.79	0.01\\
70.8	0.01\\
70.81	0.01\\
70.82	0.01\\
70.83	0.01\\
70.84	0.01\\
70.85	0.01\\
70.86	0.01\\
70.87	0.01\\
70.88	0.01\\
70.89	0.01\\
70.9	0.01\\
70.91	0.01\\
70.92	0.01\\
70.93	0.01\\
70.94	0.01\\
70.95	0.01\\
70.96	0.01\\
70.97	0.01\\
70.98	0.01\\
70.99	0.01\\
71	0.01\\
71.01	0.01\\
71.02	0.01\\
71.03	0.01\\
71.04	0.01\\
71.05	0.01\\
71.06	0.01\\
71.07	0.01\\
71.08	0.01\\
71.09	0.01\\
71.1	0.01\\
71.11	0.01\\
71.12	0.01\\
71.13	0.01\\
71.14	0.01\\
71.15	0.01\\
71.16	0.01\\
71.17	0.01\\
71.18	0.01\\
71.19	0.01\\
71.2	0.01\\
71.21	0.01\\
71.22	0.01\\
71.23	0.01\\
71.24	0.01\\
71.25	0.01\\
71.26	0.01\\
71.27	0.01\\
71.28	0.01\\
71.29	0.01\\
71.3	0.01\\
71.31	0.01\\
71.32	0.01\\
71.33	0.01\\
71.34	0.01\\
71.35	0.01\\
71.36	0.01\\
71.37	0.01\\
71.38	0.01\\
71.39	0.01\\
71.4	0.01\\
71.41	0.01\\
71.42	0.01\\
71.43	0.01\\
71.44	0.01\\
71.45	0.01\\
71.46	0.01\\
71.47	0.01\\
71.48	0.01\\
71.49	0.01\\
71.5	0.01\\
71.51	0.01\\
71.52	0.01\\
71.53	0.01\\
71.54	0.01\\
71.55	0.01\\
71.56	0.01\\
71.57	0.01\\
71.58	0.01\\
71.59	0.01\\
71.6	0.01\\
71.61	0.01\\
71.62	0.01\\
71.63	0.01\\
71.64	0.01\\
71.65	0.01\\
71.66	0.01\\
71.67	0.01\\
71.68	0.01\\
71.69	0.01\\
71.7	0.01\\
71.71	0.01\\
71.72	0.01\\
71.73	0.01\\
71.74	0.01\\
71.75	0.01\\
71.76	0.01\\
71.77	0.01\\
71.78	0.01\\
71.79	0.01\\
71.8	0.01\\
71.81	0.01\\
71.82	0.01\\
71.83	0.01\\
71.84	0.01\\
71.85	0.01\\
71.86	0.01\\
71.87	0.01\\
71.88	0.01\\
71.89	0.01\\
71.9	0.01\\
71.91	0.01\\
71.92	0.01\\
71.93	0.01\\
71.94	0.01\\
71.95	0.01\\
71.96	0.01\\
71.97	0.01\\
71.98	0.01\\
71.99	0.01\\
72	0.01\\
72.01	0.01\\
72.02	0.01\\
72.03	0.01\\
72.04	0.01\\
72.05	0.01\\
72.06	0.01\\
72.07	0.01\\
72.08	0.01\\
72.09	0.01\\
72.1	0.01\\
72.11	0.01\\
72.12	0.01\\
72.13	0.01\\
72.14	0.01\\
72.15	0.01\\
72.16	0.01\\
72.17	0.01\\
72.18	0.01\\
72.19	0.01\\
72.2	0.01\\
72.21	0.01\\
72.22	0.01\\
72.23	0.01\\
72.24	0.01\\
72.25	0.01\\
72.26	0.01\\
72.27	0.01\\
72.28	0.01\\
72.29	0.01\\
72.3	0.01\\
72.31	0.01\\
72.32	0.01\\
72.33	0.01\\
72.34	0.01\\
72.35	0.01\\
72.36	0.01\\
72.37	0.01\\
72.38	0.01\\
72.39	0.01\\
72.4	0.01\\
72.41	0.01\\
72.42	0.01\\
72.43	0.01\\
72.44	0.01\\
72.45	0.01\\
72.46	0.01\\
72.47	0.01\\
72.48	0.01\\
72.49	0.01\\
72.5	0.01\\
72.51	0.01\\
72.52	0.01\\
72.53	0.01\\
72.54	0.01\\
72.55	0.01\\
72.56	0.01\\
72.57	0.01\\
72.58	0.01\\
72.59	0.01\\
72.6	0.01\\
72.61	0.01\\
72.62	0.01\\
72.63	0.01\\
72.64	0.01\\
72.65	0.01\\
72.66	0.01\\
72.67	0.01\\
72.68	0.01\\
72.69	0.01\\
72.7	0.01\\
72.71	0.01\\
72.72	0.01\\
72.73	0.01\\
72.74	0.01\\
72.75	0.01\\
72.76	0.01\\
72.77	0.01\\
72.78	0.01\\
72.79	0.01\\
72.8	0.01\\
72.81	0.01\\
72.82	0.01\\
72.83	0.01\\
72.84	0.01\\
72.85	0.01\\
72.86	0.01\\
72.87	0.01\\
72.88	0.01\\
72.89	0.01\\
72.9	0.01\\
72.91	0.01\\
72.92	0.01\\
72.93	0.01\\
72.94	0.01\\
72.95	0.01\\
72.96	0.01\\
72.97	0.01\\
72.98	0.01\\
72.99	0.01\\
73	0.01\\
73.01	0.01\\
73.02	0.01\\
73.03	0.01\\
73.04	0.01\\
73.05	0.01\\
73.06	0.01\\
73.07	0.01\\
73.08	0.01\\
73.09	0.01\\
73.1	0.01\\
73.11	0.01\\
73.12	0.01\\
73.13	0.01\\
73.14	0.01\\
73.15	0.01\\
73.16	0.01\\
73.17	0.01\\
73.18	0.01\\
73.19	0.01\\
73.2	0.01\\
73.21	0.01\\
73.22	0.01\\
73.23	0.01\\
73.24	0.01\\
73.25	0.01\\
73.26	0.01\\
73.27	0.01\\
73.28	0.01\\
73.29	0.01\\
73.3	0.01\\
73.31	0.01\\
73.32	0.01\\
73.33	0.01\\
73.34	0.01\\
73.35	0.01\\
73.36	0.01\\
73.37	0.01\\
73.38	0.01\\
73.39	0.01\\
73.4	0.01\\
73.41	0.01\\
73.42	0.01\\
73.43	0.01\\
73.44	0.01\\
73.45	0.01\\
73.46	0.01\\
73.47	0.01\\
73.48	0.01\\
73.49	0.01\\
73.5	0.01\\
73.51	0.01\\
73.52	0.01\\
73.53	0.01\\
73.54	0.01\\
73.55	0.01\\
73.56	0.01\\
73.57	0.01\\
73.58	0.01\\
73.59	0.01\\
73.6	0.01\\
73.61	0.01\\
73.62	0.01\\
73.63	0.01\\
73.64	0.01\\
73.65	0.01\\
73.66	0.01\\
73.67	0.01\\
73.68	0.01\\
73.69	0.01\\
73.7	0.01\\
73.71	0.01\\
73.72	0.01\\
73.73	0.01\\
73.74	0.01\\
73.75	0.01\\
73.76	0.01\\
73.77	0.01\\
73.78	0.01\\
73.79	0.01\\
73.8	0.01\\
73.81	0.01\\
73.82	0.01\\
73.83	0.01\\
73.84	0.01\\
73.85	0.01\\
73.86	0.01\\
73.87	0.01\\
73.88	0.01\\
73.89	0.01\\
73.9	0.01\\
73.91	0.01\\
73.92	0.01\\
73.93	0.01\\
73.94	0.01\\
73.95	0.01\\
73.96	0.01\\
73.97	0.01\\
73.98	0.01\\
73.99	0.01\\
74	0.01\\
74.01	0.01\\
74.02	0.01\\
74.03	0.01\\
74.04	0.01\\
74.05	0.01\\
74.06	0.01\\
74.07	0.01\\
74.08	0.01\\
74.09	0.01\\
74.1	0.01\\
74.11	0.01\\
74.12	0.01\\
74.13	0.01\\
74.14	0.01\\
74.15	0.01\\
74.16	0.01\\
74.17	0.01\\
74.18	0.01\\
74.19	0.01\\
74.2	0.01\\
74.21	0.01\\
74.22	0.01\\
74.23	0.01\\
74.24	0.01\\
74.25	0.01\\
74.26	0.01\\
74.27	0.01\\
74.28	0.01\\
74.29	0.01\\
74.3	0.01\\
74.31	0.01\\
74.32	0.01\\
74.33	0.01\\
74.34	0.01\\
74.35	0.01\\
74.36	0.01\\
74.37	0.01\\
74.38	0.01\\
74.39	0.01\\
74.4	0.01\\
74.41	0.01\\
74.42	0.01\\
74.43	0.01\\
74.44	0.01\\
74.45	0.01\\
74.46	0.01\\
74.47	0.01\\
74.48	0.01\\
74.49	0.01\\
74.5	0.01\\
74.51	0.01\\
74.52	0.01\\
74.53	0.01\\
74.54	0.01\\
74.55	0.01\\
74.56	0.01\\
74.57	0.01\\
74.58	0.01\\
74.59	0.01\\
74.6	0.01\\
74.61	0.01\\
74.62	0.01\\
74.63	0.01\\
74.64	0.01\\
74.65	0.01\\
74.66	0.01\\
74.67	0.01\\
74.68	0.01\\
74.69	0.01\\
74.7	0.01\\
74.71	0.01\\
74.72	0.01\\
74.73	0.01\\
74.74	0.01\\
74.75	0.01\\
74.76	0.01\\
74.77	0.01\\
74.78	0.01\\
74.79	0.01\\
74.8	0.01\\
74.81	0.01\\
74.82	0.01\\
74.83	0.01\\
74.84	0.01\\
74.85	0.01\\
74.86	0.01\\
74.87	0.01\\
74.88	0.01\\
74.89	0.01\\
74.9	0.01\\
74.91	0.01\\
74.92	0.01\\
74.93	0.01\\
74.94	0.01\\
74.95	0.01\\
74.96	0.01\\
74.97	0.01\\
74.98	0.01\\
74.99	0.01\\
75	0.01\\
75.01	0.01\\
75.02	0.01\\
75.03	0.01\\
75.04	0.01\\
75.05	0.01\\
75.06	0.01\\
75.07	0.01\\
75.08	0.01\\
75.09	0.01\\
75.1	0.01\\
75.11	0.01\\
75.12	0.01\\
75.13	0.01\\
75.14	0.01\\
75.15	0.01\\
75.16	0.01\\
75.17	0.01\\
75.18	0.01\\
75.19	0.01\\
75.2	0.01\\
75.21	0.01\\
75.22	0.01\\
75.23	0.01\\
75.24	0.01\\
75.25	0.01\\
75.26	0.01\\
75.27	0.01\\
75.28	0.01\\
75.29	0.01\\
75.3	0.01\\
75.31	0.01\\
75.32	0.01\\
75.33	0.01\\
75.34	0.01\\
75.35	0.01\\
75.36	0.01\\
75.37	0.01\\
75.38	0.01\\
75.39	0.01\\
75.4	0.01\\
75.41	0.01\\
75.42	0.01\\
75.43	0.01\\
75.44	0.01\\
75.45	0.01\\
75.46	0.01\\
75.47	0.01\\
75.48	0.01\\
75.49	0.01\\
75.5	0.01\\
75.51	0.01\\
75.52	0.01\\
75.53	0.01\\
75.54	0.01\\
75.55	0.01\\
75.56	0.01\\
75.57	0.01\\
75.58	0.01\\
75.59	0.01\\
75.6	0.01\\
75.61	0.01\\
75.62	0.01\\
75.63	0.01\\
75.64	0.01\\
75.65	0.01\\
75.66	0.01\\
75.67	0.01\\
75.68	0.01\\
75.69	0.01\\
75.7	0.01\\
75.71	0.01\\
75.72	0.01\\
75.73	0.01\\
75.74	0.01\\
75.75	0.01\\
75.76	0.01\\
75.77	0.01\\
75.78	0.01\\
75.79	0.01\\
75.8	0.01\\
75.81	0.01\\
75.82	0.01\\
75.83	0.01\\
75.84	0.01\\
75.85	0.01\\
75.86	0.01\\
75.87	0.01\\
75.88	0.01\\
75.89	0.01\\
75.9	0.01\\
75.91	0.01\\
75.92	0.01\\
75.93	0.01\\
75.94	0.01\\
75.95	0.01\\
75.96	0.01\\
75.97	0.01\\
75.98	0.01\\
75.99	0.01\\
76	0.01\\
76.01	0.01\\
76.02	0.01\\
76.03	0.01\\
76.04	0.01\\
76.05	0.01\\
76.06	0.01\\
76.07	0.01\\
76.08	0.01\\
76.09	0.01\\
76.1	0.01\\
76.11	0.01\\
76.12	0.01\\
76.13	0.01\\
76.14	0.01\\
76.15	0.01\\
76.16	0.01\\
76.17	0.01\\
76.18	0.01\\
76.19	0.01\\
76.2	0.01\\
76.21	0.01\\
76.22	0.01\\
76.23	0.01\\
76.24	0.01\\
76.25	0.01\\
76.26	0.01\\
76.27	0.01\\
76.28	0.01\\
76.29	0.01\\
76.3	0.01\\
76.31	0.01\\
76.32	0.01\\
76.33	0.01\\
76.34	0.01\\
76.35	0.01\\
76.36	0.01\\
76.37	0.01\\
76.38	0.01\\
76.39	0.01\\
76.4	0.01\\
76.41	0.01\\
76.42	0.01\\
76.43	0.01\\
76.44	0.01\\
76.45	0.01\\
76.46	0.01\\
76.47	0.01\\
76.48	0.01\\
76.49	0.01\\
76.5	0.01\\
76.51	0.01\\
76.52	0.01\\
76.53	0.01\\
76.54	0.01\\
76.55	0.01\\
76.56	0.01\\
76.57	0.01\\
76.58	0.01\\
76.59	0.01\\
76.6	0.01\\
76.61	0.01\\
76.62	0.01\\
76.63	0.01\\
76.64	0.01\\
76.65	0.01\\
76.66	0.01\\
76.67	0.01\\
76.68	0.01\\
76.69	0.01\\
76.7	0.01\\
76.71	0.01\\
76.72	0.01\\
76.73	0.01\\
76.74	0.01\\
76.75	0.01\\
76.76	0.01\\
76.77	0.01\\
76.78	0.01\\
76.79	0.01\\
76.8	0.01\\
76.81	0.01\\
76.82	0.01\\
76.83	0.01\\
76.84	0.01\\
76.85	0.01\\
76.86	0.01\\
76.87	0.01\\
76.88	0.01\\
76.89	0.01\\
76.9	0.01\\
76.91	0.01\\
76.92	0.01\\
76.93	0.01\\
76.94	0.01\\
76.95	0.01\\
76.96	0.01\\
76.97	0.01\\
76.98	0.01\\
76.99	0.01\\
77	0.01\\
77.01	0.01\\
77.02	0.01\\
77.03	0.01\\
77.04	0.01\\
77.05	0.01\\
77.06	0.01\\
77.07	0.01\\
77.08	0.01\\
77.09	0.01\\
77.1	0.01\\
77.11	0.01\\
77.12	0.01\\
77.13	0.01\\
77.14	0.01\\
77.15	0.01\\
77.16	0.01\\
77.17	0.01\\
77.18	0.01\\
77.19	0.01\\
77.2	0.01\\
77.21	0.01\\
77.22	0.01\\
77.23	0.01\\
77.24	0.01\\
77.25	0.01\\
77.26	0.01\\
77.27	0.01\\
77.28	0.01\\
77.29	0.01\\
77.3	0.01\\
77.31	0.01\\
77.32	0.01\\
77.33	0.01\\
77.34	0.01\\
77.35	0.01\\
77.36	0.01\\
77.37	0.01\\
77.38	0.01\\
77.39	0.01\\
77.4	0.01\\
77.41	0.01\\
77.42	0.01\\
77.43	0.01\\
77.44	0.01\\
77.45	0.01\\
77.46	0.01\\
77.47	0.01\\
77.48	0.01\\
77.49	0.01\\
77.5	0.01\\
77.51	0.01\\
77.52	0.01\\
77.53	0.01\\
77.54	0.01\\
77.55	0.01\\
77.56	0.01\\
77.57	0.01\\
77.58	0.01\\
77.59	0.01\\
77.6	0.01\\
77.61	0.01\\
77.62	0.01\\
77.63	0.01\\
77.64	0.01\\
77.65	0.01\\
77.66	0.01\\
77.67	0.01\\
77.68	0.01\\
77.69	0.01\\
77.7	0.01\\
77.71	0.01\\
77.72	0.01\\
77.73	0.01\\
77.74	0.01\\
77.75	0.01\\
77.76	0.01\\
77.77	0.01\\
77.78	0.01\\
77.79	0.01\\
77.8	0.01\\
77.81	0.01\\
77.82	0.01\\
77.83	0.01\\
77.84	0.01\\
77.85	0.01\\
77.86	0.01\\
77.87	0.01\\
77.88	0.01\\
77.89	0.01\\
77.9	0.01\\
77.91	0.01\\
77.92	0.01\\
77.93	0.01\\
77.94	0.01\\
77.95	0.01\\
77.96	0.01\\
77.97	0.01\\
77.98	0.01\\
77.99	0.01\\
78	0.01\\
78.01	0.01\\
78.02	0.01\\
78.03	0.01\\
78.04	0.01\\
78.05	0.01\\
78.06	0.01\\
78.07	0.01\\
78.08	0.01\\
78.09	0.01\\
78.1	0.01\\
78.11	0.01\\
78.12	0.01\\
78.13	0.01\\
78.14	0.01\\
78.15	0.01\\
78.16	0.01\\
78.17	0.01\\
78.18	0.01\\
78.19	0.01\\
78.2	0.01\\
78.21	0.01\\
78.22	0.01\\
78.23	0.01\\
78.24	0.01\\
78.25	0.01\\
78.26	0.01\\
78.27	0.01\\
78.28	0.01\\
78.29	0.01\\
78.3	0.01\\
78.31	0.01\\
78.32	0.01\\
78.33	0.01\\
78.34	0.01\\
78.35	0.01\\
78.36	0.01\\
78.37	0.01\\
78.38	0.01\\
78.39	0.01\\
78.4	0.01\\
78.41	0.01\\
78.42	0.01\\
78.43	0.01\\
78.44	0.01\\
78.45	0.01\\
78.46	0.01\\
78.47	0.01\\
78.48	0.01\\
78.49	0.01\\
78.5	0.01\\
78.51	0.01\\
78.52	0.01\\
78.53	0.01\\
78.54	0.01\\
78.55	0.01\\
78.56	0.01\\
78.57	0.01\\
78.58	0.01\\
78.59	0.01\\
78.6	0.01\\
78.61	0.01\\
78.62	0.01\\
78.63	0.01\\
78.64	0.01\\
78.65	0.01\\
78.66	0.01\\
78.67	0.01\\
78.68	0.01\\
78.69	0.01\\
78.7	0.01\\
78.71	0.01\\
78.72	0.01\\
78.73	0.01\\
78.74	0.01\\
78.75	0.01\\
78.76	0.01\\
78.77	0.01\\
78.78	0.01\\
78.79	0.01\\
78.8	0.01\\
78.81	0.01\\
78.82	0.01\\
78.83	0.01\\
78.84	0.01\\
78.85	0.01\\
78.86	0.01\\
78.87	0.01\\
78.88	0.01\\
78.89	0.01\\
78.9	0.01\\
78.91	0.01\\
78.92	0.01\\
78.93	0.01\\
78.94	0.01\\
78.95	0.01\\
78.96	0.01\\
78.97	0.01\\
78.98	0.01\\
78.99	0.01\\
79	0.01\\
79.01	0.01\\
79.02	0.01\\
79.03	0.01\\
79.04	0.01\\
79.05	0.01\\
79.06	0.01\\
79.07	0.01\\
79.08	0.01\\
79.09	0.01\\
79.1	0.01\\
79.11	0.01\\
79.12	0.01\\
79.13	0.01\\
79.14	0.01\\
79.15	0.01\\
79.16	0.01\\
79.17	0.01\\
79.18	0.01\\
79.19	0.01\\
79.2	0.01\\
79.21	0.01\\
79.22	0.01\\
79.23	0.01\\
79.24	0.01\\
79.25	0.01\\
79.26	0.01\\
79.27	0.01\\
79.28	0.01\\
79.29	0.01\\
79.3	0.01\\
79.31	0.01\\
79.32	0.01\\
79.33	0.01\\
79.34	0.01\\
79.35	0.01\\
79.36	0.01\\
79.37	0.01\\
79.38	0.01\\
79.39	0.01\\
79.4	0.01\\
79.41	0.01\\
79.42	0.01\\
79.43	0.01\\
79.44	0.01\\
79.45	0.01\\
79.46	0.01\\
79.47	0.01\\
79.48	0.01\\
79.49	0.01\\
79.5	0.01\\
79.51	0.01\\
79.52	0.01\\
79.53	0.01\\
79.54	0.01\\
79.55	0.01\\
79.56	0.01\\
79.57	0.01\\
79.58	0.01\\
79.59	0.01\\
79.6	0.01\\
79.61	0.01\\
79.62	0.01\\
79.63	0.01\\
79.64	0.01\\
79.65	0.01\\
79.66	0.01\\
79.67	0.01\\
79.68	0.01\\
79.69	0.01\\
79.7	0.01\\
79.71	0.01\\
79.72	0.01\\
79.73	0.01\\
79.74	0.01\\
79.75	0.01\\
79.76	0.01\\
79.77	0.01\\
79.78	0.01\\
79.79	0.01\\
79.8	0.01\\
79.81	0.01\\
79.82	0.01\\
79.83	0.01\\
79.84	0.01\\
79.85	0.01\\
79.86	0.01\\
79.87	0.01\\
79.88	0.01\\
79.89	0.01\\
79.9	0.01\\
79.91	0.01\\
79.92	0.01\\
79.93	0.01\\
79.94	0.01\\
79.95	0.01\\
79.96	0.01\\
79.97	0.01\\
79.98	0.01\\
79.99	0.01\\
80	0.01\\
80.01	0.01\\
};
\addplot [color=blue,dashed]
  table[row sep=crcr]{%
80.01	0.01\\
80.02	0.01\\
80.03	0.01\\
80.04	0.01\\
80.05	0.01\\
80.06	0.01\\
80.07	0.01\\
80.08	0.01\\
80.09	0.01\\
80.1	0.01\\
80.11	0.01\\
80.12	0.01\\
80.13	0.01\\
80.14	0.01\\
80.15	0.01\\
80.16	0.01\\
80.17	0.01\\
80.18	0.01\\
80.19	0.01\\
80.2	0.01\\
80.21	0.01\\
80.22	0.01\\
80.23	0.01\\
80.24	0.01\\
80.25	0.01\\
80.26	0.01\\
80.27	0.01\\
80.28	0.01\\
80.29	0.01\\
80.3	0.01\\
80.31	0.01\\
80.32	0.01\\
80.33	0.01\\
80.34	0.01\\
80.35	0.01\\
80.36	0.01\\
80.37	0.01\\
80.38	0.01\\
80.39	0.01\\
80.4	0.01\\
80.41	0.01\\
80.42	0.01\\
80.43	0.01\\
80.44	0.01\\
80.45	0.01\\
80.46	0.01\\
80.47	0.01\\
80.48	0.01\\
80.49	0.01\\
80.5	0.01\\
80.51	0.01\\
80.52	0.01\\
80.53	0.01\\
80.54	0.01\\
80.55	0.01\\
80.56	0.01\\
80.57	0.01\\
80.58	0.01\\
80.59	0.01\\
80.6	0.01\\
80.61	0.01\\
80.62	0.01\\
80.63	0.01\\
80.64	0.01\\
80.65	0.01\\
80.66	0.01\\
80.67	0.01\\
80.68	0.01\\
80.69	0.01\\
80.7	0.01\\
80.71	0.01\\
80.72	0.01\\
80.73	0.01\\
80.74	0.01\\
80.75	0.01\\
80.76	0.01\\
80.77	0.01\\
80.78	0.01\\
80.79	0.01\\
80.8	0.01\\
80.81	0.01\\
80.82	0.01\\
80.83	0.01\\
80.84	0.01\\
80.85	0.01\\
80.86	0.01\\
80.87	0.01\\
80.88	0.01\\
80.89	0.01\\
80.9	0.01\\
80.91	0.01\\
80.92	0.01\\
80.93	0.01\\
80.94	0.01\\
80.95	0.01\\
80.96	0.01\\
80.97	0.01\\
80.98	0.01\\
80.99	0.01\\
81	0.01\\
81.01	0.01\\
81.02	0.01\\
81.03	0.01\\
81.04	0.01\\
81.05	0.01\\
81.06	0.01\\
81.07	0.01\\
81.08	0.01\\
81.09	0.01\\
81.1	0.01\\
81.11	0.01\\
81.12	0.01\\
81.13	0.01\\
81.14	0.01\\
81.15	0.01\\
81.16	0.01\\
81.17	0.01\\
81.18	0.01\\
81.19	0.01\\
81.2	0.01\\
81.21	0.01\\
81.22	0.01\\
81.23	0.01\\
81.24	0.01\\
81.25	0.01\\
81.26	0.01\\
81.27	0.01\\
81.28	0.01\\
81.29	0.01\\
81.3	0.01\\
81.31	0.01\\
81.32	0.01\\
81.33	0.01\\
81.34	0.01\\
81.35	0.01\\
81.36	0.01\\
81.37	0.01\\
81.38	0.01\\
81.39	0.01\\
81.4	0.01\\
81.41	0.01\\
81.42	0.01\\
81.43	0.01\\
81.44	0.01\\
81.45	0.01\\
81.46	0.01\\
81.47	0.01\\
81.48	0.01\\
81.49	0.01\\
81.5	0.01\\
81.51	0.01\\
81.52	0.01\\
81.53	0.01\\
81.54	0.01\\
81.55	0.01\\
81.56	0.01\\
81.57	0.01\\
81.58	0.01\\
81.59	0.01\\
81.6	0.01\\
81.61	0.01\\
81.62	0.01\\
81.63	0.01\\
81.64	0.01\\
81.65	0.01\\
81.66	0.01\\
81.67	0.01\\
81.68	0.01\\
81.69	0.01\\
81.7	0.01\\
81.71	0.01\\
81.72	0.01\\
81.73	0.01\\
81.74	0.01\\
81.75	0.01\\
81.76	0.01\\
81.77	0.01\\
81.78	0.01\\
81.79	0.01\\
81.8	0.01\\
81.81	0.01\\
81.82	0.01\\
81.83	0.01\\
81.84	0.01\\
81.85	0.01\\
81.86	0.01\\
81.87	0.01\\
81.88	0.01\\
81.89	0.01\\
81.9	0.01\\
81.91	0.01\\
81.92	0.01\\
81.93	0.01\\
81.94	0.01\\
81.95	0.01\\
81.96	0.01\\
81.97	0.01\\
81.98	0.01\\
81.99	0.01\\
82	0.01\\
82.01	0.01\\
82.02	0.01\\
82.03	0.01\\
82.04	0.01\\
82.05	0.01\\
82.06	0.01\\
82.07	0.01\\
82.08	0.01\\
82.09	0.01\\
82.1	0.01\\
82.11	0.01\\
82.12	0.01\\
82.13	0.01\\
82.14	0.01\\
82.15	0.01\\
82.16	0.01\\
82.17	0.01\\
82.18	0.01\\
82.19	0.01\\
82.2	0.01\\
82.21	0.01\\
82.22	0.01\\
82.23	0.01\\
82.24	0.01\\
82.25	0.01\\
82.26	0.01\\
82.27	0.01\\
82.28	0.01\\
82.29	0.01\\
82.3	0.01\\
82.31	0.01\\
82.32	0.01\\
82.33	0.01\\
82.34	0.01\\
82.35	0.01\\
82.36	0.01\\
82.37	0.01\\
82.38	0.01\\
82.39	0.01\\
82.4	0.01\\
82.41	0.01\\
82.42	0.01\\
82.43	0.01\\
82.44	0.01\\
82.45	0.01\\
82.46	0.01\\
82.47	0.01\\
82.48	0.01\\
82.49	0.01\\
82.5	0.01\\
82.51	0.01\\
82.52	0.01\\
82.53	0.01\\
82.54	0.01\\
82.55	0.01\\
82.56	0.01\\
82.57	0.01\\
82.58	0.01\\
82.59	0.01\\
82.6	0.01\\
82.61	0.01\\
82.62	0.01\\
82.63	0.01\\
82.64	0.01\\
82.65	0.01\\
82.66	0.01\\
82.67	0.01\\
82.68	0.01\\
82.69	0.01\\
82.7	0.01\\
82.71	0.01\\
82.72	0.01\\
82.73	0.01\\
82.74	0.01\\
82.75	0.01\\
82.76	0.01\\
82.77	0.01\\
82.78	0.01\\
82.79	0.01\\
82.8	0.01\\
82.81	0.01\\
82.82	0.01\\
82.83	0.01\\
82.84	0.01\\
82.85	0.01\\
82.86	0.01\\
82.87	0.01\\
82.88	0.01\\
82.89	0.01\\
82.9	0.01\\
82.91	0.01\\
82.92	0.01\\
82.93	0.01\\
82.94	0.01\\
82.95	0.01\\
82.96	0.01\\
82.97	0.01\\
82.98	0.01\\
82.99	0.01\\
83	0.01\\
83.01	0.01\\
83.02	0.01\\
83.03	0.01\\
83.04	0.01\\
83.05	0.01\\
83.06	0.01\\
83.07	0.01\\
83.08	0.01\\
83.09	0.01\\
83.1	0.01\\
83.11	0.01\\
83.12	0.01\\
83.13	0.01\\
83.14	0.01\\
83.15	0.01\\
83.16	0.01\\
83.17	0.01\\
83.18	0.01\\
83.19	0.01\\
83.2	0.01\\
83.21	0.01\\
83.22	0.01\\
83.23	0.01\\
83.24	0.01\\
83.25	0.01\\
83.26	0.01\\
83.27	0.01\\
83.28	0.01\\
83.29	0.01\\
83.3	0.01\\
83.31	0.01\\
83.32	0.01\\
83.33	0.01\\
83.34	0.01\\
83.35	0.01\\
83.36	0.01\\
83.37	0.01\\
83.38	0.01\\
83.39	0.01\\
83.4	0.01\\
83.41	0.01\\
83.42	0.01\\
83.43	0.01\\
83.44	0.01\\
83.45	0.01\\
83.46	0.01\\
83.47	0.01\\
83.48	0.01\\
83.49	0.01\\
83.5	0.01\\
83.51	0.01\\
83.52	0.01\\
83.53	0.01\\
83.54	0.01\\
83.55	0.01\\
83.56	0.01\\
83.57	0.01\\
83.58	0.01\\
83.59	0.01\\
83.6	0.01\\
83.61	0.01\\
83.62	0.01\\
83.63	0.01\\
83.64	0.01\\
83.65	0.01\\
83.66	0.01\\
83.67	0.01\\
83.68	0.01\\
83.69	0.01\\
83.7	0.01\\
83.71	0.01\\
83.72	0.01\\
83.73	0.01\\
83.74	0.01\\
83.75	0.01\\
83.76	0.01\\
83.77	0.01\\
83.78	0.01\\
83.79	0.01\\
83.8	0.01\\
83.81	0.01\\
83.82	0.01\\
83.83	0.01\\
83.84	0.01\\
83.85	0.01\\
83.86	0.01\\
83.87	0.01\\
83.88	0.01\\
83.89	0.01\\
83.9	0.01\\
83.91	0.01\\
83.92	0.01\\
83.93	0.01\\
83.94	0.01\\
83.95	0.01\\
83.96	0.01\\
83.97	0.01\\
83.98	0.01\\
83.99	0.01\\
84	0.01\\
84.01	0.01\\
84.02	0.01\\
84.03	0.01\\
84.04	0.01\\
84.05	0.01\\
84.06	0.01\\
84.07	0.01\\
84.08	0.01\\
84.09	0.01\\
84.1	0.01\\
84.11	0.01\\
84.12	0.01\\
84.13	0.01\\
84.14	0.01\\
84.15	0.01\\
84.16	0.01\\
84.17	0.01\\
84.18	0.01\\
84.19	0.01\\
84.2	0.01\\
84.21	0.01\\
84.22	0.01\\
84.23	0.01\\
84.24	0.01\\
84.25	0.01\\
84.26	0.01\\
84.27	0.01\\
84.28	0.01\\
84.29	0.01\\
84.3	0.01\\
84.31	0.01\\
84.32	0.01\\
84.33	0.01\\
84.34	0.01\\
84.35	0.01\\
84.36	0.01\\
84.37	0.01\\
84.38	0.01\\
84.39	0.01\\
84.4	0.01\\
84.41	0.01\\
84.42	0.01\\
84.43	0.01\\
84.44	0.01\\
84.45	0.01\\
84.46	0.01\\
84.47	0.01\\
84.48	0.01\\
84.49	0.01\\
84.5	0.01\\
84.51	0.01\\
84.52	0.01\\
84.53	0.01\\
84.54	0.01\\
84.55	0.01\\
84.56	0.01\\
84.57	0.01\\
84.58	0.01\\
84.59	0.01\\
84.6	0.01\\
84.61	0.01\\
84.62	0.01\\
84.63	0.01\\
84.64	0.01\\
84.65	0.01\\
84.66	0.01\\
84.67	0.01\\
84.68	0.01\\
84.69	0.01\\
84.7	0.01\\
84.71	0.01\\
84.72	0.01\\
84.73	0.01\\
84.74	0.01\\
84.75	0.01\\
84.76	0.01\\
84.77	0.01\\
84.78	0.01\\
84.79	0.01\\
84.8	0.01\\
84.81	0.01\\
84.82	0.01\\
84.83	0.01\\
84.84	0.01\\
84.85	0.01\\
84.86	0.01\\
84.87	0.01\\
84.88	0.01\\
84.89	0.01\\
84.9	0.01\\
84.91	0.01\\
84.92	0.01\\
84.93	0.01\\
84.94	0.01\\
84.95	0.01\\
84.96	0.01\\
84.97	0.01\\
84.98	0.01\\
84.99	0.01\\
85	0.01\\
85.01	0.01\\
85.02	0.01\\
85.03	0.01\\
85.04	0.01\\
85.05	0.01\\
85.06	0.01\\
85.07	0.01\\
85.08	0.01\\
85.09	0.01\\
85.1	0.01\\
85.11	0.01\\
85.12	0.01\\
85.13	0.01\\
85.14	0.01\\
85.15	0.01\\
85.16	0.01\\
85.17	0.01\\
85.18	0.01\\
85.19	0.01\\
85.2	0.01\\
85.21	0.01\\
85.22	0.01\\
85.23	0.01\\
85.24	0.01\\
85.25	0.01\\
85.26	0.01\\
85.27	0.01\\
85.28	0.01\\
85.29	0.01\\
85.3	0.01\\
85.31	0.01\\
85.32	0.01\\
85.33	0.01\\
85.34	0.01\\
85.35	0.01\\
85.36	0.01\\
85.37	0.01\\
85.38	0.01\\
85.39	0.01\\
85.4	0.01\\
85.41	0.01\\
85.42	0.01\\
85.43	0.01\\
85.44	0.01\\
85.45	0.01\\
85.46	0.01\\
85.47	0.01\\
85.48	0.01\\
85.49	0.01\\
85.5	0.01\\
85.51	0.01\\
85.52	0.01\\
85.53	0.01\\
85.54	0.01\\
85.55	0.01\\
85.56	0.01\\
85.57	0.01\\
85.58	0.01\\
85.59	0.01\\
85.6	0.01\\
85.61	0.01\\
85.62	0.01\\
85.63	0.01\\
85.64	0.01\\
85.65	0.01\\
85.66	0.01\\
85.67	0.01\\
85.68	0.01\\
85.69	0.01\\
85.7	0.01\\
85.71	0.01\\
85.72	0.01\\
85.73	0.01\\
85.74	0.01\\
85.75	0.01\\
85.76	0.01\\
85.77	0.01\\
85.78	0.01\\
85.79	0.01\\
85.8	0.01\\
85.81	0.01\\
85.82	0.01\\
85.83	0.01\\
85.84	0.01\\
85.85	0.01\\
85.86	0.01\\
85.87	0.01\\
85.88	0.01\\
85.89	0.01\\
85.9	0.01\\
85.91	0.01\\
85.92	0.01\\
85.93	0.01\\
85.94	0.01\\
85.95	0.01\\
85.96	0.01\\
85.97	0.01\\
85.98	0.01\\
85.99	0.01\\
86	0.01\\
86.01	0.01\\
86.02	0.01\\
86.03	0.01\\
86.04	0.01\\
86.05	0.01\\
86.06	0.01\\
86.07	0.01\\
86.08	0.01\\
86.09	0.01\\
86.1	0.01\\
86.11	0.01\\
86.12	0.01\\
86.13	0.01\\
86.14	0.01\\
86.15	0.01\\
86.16	0.01\\
86.17	0.01\\
86.18	0.01\\
86.19	0.01\\
86.2	0.01\\
86.21	0.01\\
86.22	0.01\\
86.23	0.01\\
86.24	0.01\\
86.25	0.01\\
86.26	0.01\\
86.27	0.01\\
86.28	0.01\\
86.29	0.01\\
86.3	0.01\\
86.31	0.01\\
86.32	0.01\\
86.33	0.01\\
86.34	0.01\\
86.35	0.01\\
86.36	0.01\\
86.37	0.01\\
86.38	0.01\\
86.39	0.01\\
86.4	0.01\\
86.41	0.01\\
86.42	0.01\\
86.43	0.01\\
86.44	0.01\\
86.45	0.01\\
86.46	0.01\\
86.47	0.01\\
86.48	0.01\\
86.49	0.01\\
86.5	0.01\\
86.51	0.01\\
86.52	0.01\\
86.53	0.01\\
86.54	0.01\\
86.55	0.01\\
86.56	0.01\\
86.57	0.01\\
86.58	0.01\\
86.59	0.01\\
86.6	0.01\\
86.61	0.01\\
86.62	0.01\\
86.63	0.01\\
86.64	0.01\\
86.65	0.01\\
86.66	0.01\\
86.67	0.01\\
86.68	0.01\\
86.69	0.01\\
86.7	0.01\\
86.71	0.01\\
86.72	0.01\\
86.73	0.01\\
86.74	0.01\\
86.75	0.01\\
86.76	0.01\\
86.77	0.01\\
86.78	0.01\\
86.79	0.01\\
86.8	0.01\\
86.81	0.01\\
86.82	0.01\\
86.83	0.01\\
86.84	0.01\\
86.85	0.01\\
86.86	0.01\\
86.87	0.01\\
86.88	0.01\\
86.89	0.01\\
86.9	0.01\\
86.91	0.01\\
86.92	0.01\\
86.93	0.01\\
86.94	0.01\\
86.95	0.01\\
86.96	0.01\\
86.97	0.01\\
86.98	0.01\\
86.99	0.01\\
87	0.01\\
87.01	0.01\\
87.02	0.01\\
87.03	0.01\\
87.04	0.01\\
87.05	0.01\\
87.06	0.01\\
87.07	0.01\\
87.08	0.01\\
87.09	0.01\\
87.1	0.01\\
87.11	0.01\\
87.12	0.01\\
87.13	0.01\\
87.14	0.01\\
87.15	0.01\\
87.16	0.01\\
87.17	0.01\\
87.18	0.01\\
87.19	0.01\\
87.2	0.01\\
87.21	0.01\\
87.22	0.01\\
87.23	0.01\\
87.24	0.01\\
87.25	0.01\\
87.26	0.01\\
87.27	0.01\\
87.28	0.01\\
87.29	0.01\\
87.3	0.01\\
87.31	0.01\\
87.32	0.01\\
87.33	0.01\\
87.34	0.01\\
87.35	0.01\\
87.36	0.01\\
87.37	0.01\\
87.38	0.01\\
87.39	0.01\\
87.4	0.01\\
87.41	0.01\\
87.42	0.01\\
87.43	0.01\\
87.44	0.01\\
87.45	0.01\\
87.46	0.01\\
87.47	0.01\\
87.48	0.01\\
87.49	0.01\\
87.5	0.01\\
87.51	0.01\\
87.52	0.01\\
87.53	0.01\\
87.54	0.01\\
87.55	0.01\\
87.56	0.01\\
87.57	0.01\\
87.58	0.01\\
87.59	0.01\\
87.6	0.01\\
87.61	0.01\\
87.62	0.01\\
87.63	0.01\\
87.64	0.01\\
87.65	0.01\\
87.66	0.01\\
87.67	0.01\\
87.68	0.01\\
87.69	0.01\\
87.7	0.01\\
87.71	0.01\\
87.72	0.01\\
87.73	0.01\\
87.74	0.01\\
87.75	0.01\\
87.76	0.01\\
87.77	0.01\\
87.78	0.01\\
87.79	0.01\\
87.8	0.01\\
87.81	0.01\\
87.82	0.01\\
87.83	0.01\\
87.84	0.01\\
87.85	0.01\\
87.86	0.01\\
87.87	0.01\\
87.88	0.01\\
87.89	0.01\\
87.9	0.01\\
87.91	0.01\\
87.92	0.01\\
87.93	0.01\\
87.94	0.01\\
87.95	0.01\\
87.96	0.01\\
87.97	0.01\\
87.98	0.01\\
87.99	0.01\\
88	0.01\\
88.01	0.01\\
88.02	0.01\\
88.03	0.01\\
88.04	0.01\\
88.05	0.01\\
88.06	0.01\\
88.07	0.01\\
88.08	0.01\\
88.09	0.01\\
88.1	0.01\\
88.11	0.01\\
88.12	0.01\\
88.13	0.01\\
88.14	0.01\\
88.15	0.01\\
88.16	0.01\\
88.17	0.01\\
88.18	0.01\\
88.19	0.01\\
88.2	0.01\\
88.21	0.01\\
88.22	0.01\\
88.23	0.01\\
88.24	0.01\\
88.25	0.01\\
88.26	0.01\\
88.27	0.01\\
88.28	0.01\\
88.29	0.01\\
88.3	0.01\\
88.31	0.01\\
88.32	0.01\\
88.33	0.01\\
88.34	0.01\\
88.35	0.01\\
88.36	0.01\\
88.37	0.01\\
88.38	0.01\\
88.39	0.01\\
88.4	0.01\\
88.41	0.01\\
88.42	0.01\\
88.43	0.01\\
88.44	0.01\\
88.45	0.01\\
88.46	0.01\\
88.47	0.01\\
88.48	0.01\\
88.49	0.01\\
88.5	0.01\\
88.51	0.01\\
88.52	0.01\\
88.53	0.01\\
88.54	0.01\\
88.55	0.01\\
88.56	0.01\\
88.57	0.01\\
88.58	0.01\\
88.59	0.01\\
88.6	0.01\\
88.61	0.01\\
88.62	0.01\\
88.63	0.01\\
88.64	0.01\\
88.65	0.01\\
88.66	0.01\\
88.67	0.01\\
88.68	0.01\\
88.69	0.01\\
88.7	0.01\\
88.71	0.01\\
88.72	0.01\\
88.73	0.01\\
88.74	0.01\\
88.75	0.01\\
88.76	0.01\\
88.77	0.01\\
88.78	0.01\\
88.79	0.01\\
88.8	0.01\\
88.81	0.01\\
88.82	0.01\\
88.83	0.01\\
88.84	0.01\\
88.85	0.01\\
88.86	0.01\\
88.87	0.01\\
88.88	0.01\\
88.89	0.01\\
88.9	0.01\\
88.91	0.01\\
88.92	0.01\\
88.93	0.01\\
88.94	0.01\\
88.95	0.01\\
88.96	0.01\\
88.97	0.01\\
88.98	0.01\\
88.99	0.01\\
89	0.01\\
89.01	0.01\\
89.02	0.01\\
89.03	0.01\\
89.04	0.01\\
89.05	0.01\\
89.06	0.01\\
89.07	0.01\\
89.08	0.01\\
89.09	0.01\\
89.1	0.01\\
89.11	0.01\\
89.12	0.01\\
89.13	0.01\\
89.14	0.01\\
89.15	0.01\\
89.16	0.01\\
89.17	0.01\\
89.18	0.01\\
89.19	0.01\\
89.2	0.01\\
89.21	0.01\\
89.22	0.01\\
89.23	0.01\\
89.24	0.01\\
89.25	0.01\\
89.26	0.01\\
89.27	0.01\\
89.28	0.01\\
89.29	0.01\\
89.3	0.01\\
89.31	0.01\\
89.32	0.01\\
89.33	0.01\\
89.34	0.01\\
89.35	0.01\\
89.36	0.01\\
89.37	0.01\\
89.38	0.01\\
89.39	0.01\\
89.4	0.01\\
89.41	0.01\\
89.42	0.01\\
89.43	0.01\\
89.44	0.01\\
89.45	0.01\\
89.46	0.01\\
89.47	0.01\\
89.48	0.01\\
89.49	0.01\\
89.5	0.01\\
89.51	0.01\\
89.52	0.01\\
89.53	0.01\\
89.54	0.01\\
89.55	0.01\\
89.56	0.01\\
89.57	0.01\\
89.58	0.01\\
89.59	0.01\\
89.6	0.01\\
89.61	0.01\\
89.62	0.01\\
89.63	0.01\\
89.64	0.01\\
89.65	0.01\\
89.66	0.01\\
89.67	0.01\\
89.68	0.01\\
89.69	0.01\\
89.7	0.01\\
89.71	0.01\\
89.72	0.01\\
89.73	0.01\\
89.74	0.01\\
89.75	0.01\\
89.76	0.01\\
89.77	0.01\\
89.78	0.01\\
89.79	0.01\\
89.8	0.01\\
89.81	0.01\\
89.82	0.01\\
89.83	0.01\\
89.84	0.01\\
89.85	0.01\\
89.86	0.01\\
89.87	0.01\\
89.88	0.01\\
89.89	0.01\\
89.9	0.01\\
89.91	0.01\\
89.92	0.01\\
89.93	0.01\\
89.94	0.01\\
89.95	0.01\\
89.96	0.01\\
89.97	0.01\\
89.98	0.01\\
89.99	0.01\\
90	0.01\\
90.01	0.01\\
90.02	0.01\\
90.03	0.01\\
90.04	0.01\\
90.05	0.01\\
90.06	0.01\\
90.07	0.01\\
90.08	0.01\\
90.09	0.01\\
90.1	0.01\\
90.11	0.01\\
90.12	0.01\\
90.13	0.01\\
90.14	0.01\\
90.15	0.01\\
90.16	0.01\\
90.17	0.01\\
90.18	0.01\\
90.19	0.01\\
90.2	0.01\\
90.21	0.01\\
90.22	0.01\\
90.23	0.01\\
90.24	0.01\\
90.25	0.01\\
90.26	0.01\\
90.27	0.01\\
90.28	0.01\\
90.29	0.01\\
90.3	0.01\\
90.31	0.01\\
90.32	0.01\\
90.33	0.01\\
90.34	0.01\\
90.35	0.01\\
90.36	0.01\\
90.37	0.01\\
90.38	0.01\\
90.39	0.01\\
90.4	0.01\\
90.41	0.01\\
90.42	0.01\\
90.43	0.01\\
90.44	0.01\\
90.45	0.01\\
90.46	0.01\\
90.47	0.01\\
90.48	0.01\\
90.49	0.01\\
90.5	0.01\\
90.51	0.01\\
90.52	0.01\\
90.53	0.01\\
90.54	0.01\\
90.55	0.01\\
90.56	0.01\\
90.57	0.01\\
90.58	0.01\\
90.59	0.01\\
90.6	0.01\\
90.61	0.01\\
90.62	0.01\\
90.63	0.01\\
90.64	0.01\\
90.65	0.01\\
90.66	0.01\\
90.67	0.01\\
90.68	0.01\\
90.69	0.01\\
90.7	0.01\\
90.71	0.01\\
90.72	0.01\\
90.73	0.01\\
90.74	0.01\\
90.75	0.01\\
90.76	0.01\\
90.77	0.01\\
90.78	0.01\\
90.79	0.01\\
90.8	0.01\\
90.81	0.01\\
90.82	0.01\\
90.83	0.01\\
90.84	0.01\\
90.85	0.01\\
90.86	0.01\\
90.87	0.01\\
90.88	0.01\\
90.89	0.01\\
90.9	0.01\\
90.91	0.01\\
90.92	0.01\\
90.93	0.01\\
90.94	0.01\\
90.95	0.01\\
90.96	0.01\\
90.97	0.01\\
90.98	0.01\\
90.99	0.01\\
91	0.01\\
91.01	0.01\\
91.02	0.01\\
91.03	0.01\\
91.04	0.01\\
91.05	0.01\\
91.06	0.01\\
91.07	0.01\\
91.08	0.01\\
91.09	0.01\\
91.1	0.01\\
91.11	0.01\\
91.12	0.01\\
91.13	0.01\\
91.14	0.01\\
91.15	0.01\\
91.16	0.01\\
91.17	0.01\\
91.18	0.01\\
91.19	0.01\\
91.2	0.01\\
91.21	0.01\\
91.22	0.01\\
91.23	0.01\\
91.24	0.01\\
91.25	0.01\\
91.26	0.01\\
91.27	0.01\\
91.28	0.01\\
91.29	0.01\\
91.3	0.01\\
91.31	0.01\\
91.32	0.01\\
91.33	0.01\\
91.34	0.01\\
91.35	0.01\\
91.36	0.01\\
91.37	0.01\\
91.38	0.01\\
91.39	0.01\\
91.4	0.01\\
91.41	0.01\\
91.42	0.01\\
91.43	0.01\\
91.44	0.01\\
91.45	0.01\\
91.46	0.01\\
91.47	0.01\\
91.48	0.01\\
91.49	0.01\\
91.5	0.01\\
91.51	0.01\\
91.52	0.01\\
91.53	0.01\\
91.54	0.01\\
91.55	0.01\\
91.56	0.01\\
91.57	0.01\\
91.58	0.01\\
91.59	0.01\\
91.6	0.01\\
91.61	0.01\\
91.62	0.01\\
91.63	0.01\\
91.64	0.01\\
91.65	0.01\\
91.66	0.01\\
91.67	0.01\\
91.68	0.01\\
91.69	0.01\\
91.7	0.01\\
91.71	0.01\\
91.72	0.01\\
91.73	0.01\\
91.74	0.01\\
91.75	0.01\\
91.76	0.01\\
91.77	0.01\\
91.78	0.01\\
91.79	0.01\\
91.8	0.01\\
91.81	0.01\\
91.82	0.01\\
91.83	0.01\\
91.84	0.01\\
91.85	0.01\\
91.86	0.01\\
91.87	0.01\\
91.88	0.01\\
91.89	0.01\\
91.9	0.01\\
91.91	0.01\\
91.92	0.01\\
91.93	0.01\\
91.94	0.01\\
91.95	0.01\\
91.96	0.01\\
91.97	0.01\\
91.98	0.01\\
91.99	0.01\\
92	0.01\\
92.01	0.01\\
92.02	0.01\\
92.03	0.01\\
92.04	0.01\\
92.05	0.01\\
92.06	0.01\\
92.07	0.01\\
92.08	0.01\\
92.09	0.01\\
92.1	0.01\\
92.11	0.01\\
92.12	0.01\\
92.13	0.01\\
92.14	0.01\\
92.15	0.01\\
92.16	0.01\\
92.17	0.01\\
92.18	0.01\\
92.19	0.01\\
92.2	0.01\\
92.21	0.01\\
92.22	0.01\\
92.23	0.01\\
92.24	0.01\\
92.25	0.01\\
92.26	0.01\\
92.27	0.01\\
92.28	0.01\\
92.29	0.01\\
92.3	0.01\\
92.31	0.01\\
92.32	0.01\\
92.33	0.01\\
92.34	0.01\\
92.35	0.01\\
92.36	0.01\\
92.37	0.01\\
92.38	0.01\\
92.39	0.01\\
92.4	0.01\\
92.41	0.01\\
92.42	0.01\\
92.43	0.01\\
92.44	0.01\\
92.45	0.01\\
92.46	0.01\\
92.47	0.01\\
92.48	0.01\\
92.49	0.01\\
92.5	0.01\\
92.51	0.01\\
92.52	0.01\\
92.53	0.01\\
92.54	0.01\\
92.55	0.01\\
92.56	0.01\\
92.57	0.01\\
92.58	0.01\\
92.59	0.01\\
92.6	0.01\\
92.61	0.01\\
92.62	0.01\\
92.63	0.01\\
92.64	0.01\\
92.65	0.01\\
92.66	0.01\\
92.67	0.01\\
92.68	0.01\\
92.69	0.01\\
92.7	0.01\\
92.71	0.01\\
92.72	0.01\\
92.73	0.01\\
92.74	0.01\\
92.75	0.01\\
92.76	0.01\\
92.77	0.01\\
92.78	0.01\\
92.79	0.01\\
92.8	0.01\\
92.81	0.01\\
92.82	0.01\\
92.83	0.01\\
92.84	0.01\\
92.85	0.01\\
92.86	0.01\\
92.87	0.01\\
92.88	0.01\\
92.89	0.01\\
92.9	0.01\\
92.91	0.01\\
92.92	0.01\\
92.93	0.01\\
92.94	0.01\\
92.95	0.01\\
92.96	0.01\\
92.97	0.01\\
92.98	0.01\\
92.99	0.01\\
93	0.01\\
93.01	0.01\\
93.02	0.01\\
93.03	0.01\\
93.04	0.01\\
93.05	0.01\\
93.06	0.01\\
93.07	0.01\\
93.08	0.01\\
93.09	0.01\\
93.1	0.01\\
93.11	0.01\\
93.12	0.01\\
93.13	0.01\\
93.14	0.01\\
93.15	0.01\\
93.16	0.01\\
93.17	0.01\\
93.18	0.01\\
93.19	0.01\\
93.2	0.01\\
93.21	0.01\\
93.22	0.01\\
93.23	0.01\\
93.24	0.01\\
93.25	0.01\\
93.26	0.01\\
93.27	0.01\\
93.28	0.01\\
93.29	0.01\\
93.3	0.01\\
93.31	0.01\\
93.32	0.01\\
93.33	0.01\\
93.34	0.01\\
93.35	0.01\\
93.36	0.01\\
93.37	0.01\\
93.38	0.01\\
93.39	0.01\\
93.4	0.01\\
93.41	0.01\\
93.42	0.01\\
93.43	0.01\\
93.44	0.01\\
93.45	0.01\\
93.46	0.01\\
93.47	0.01\\
93.48	0.01\\
93.49	0.01\\
93.5	0.01\\
93.51	0.01\\
93.52	0.01\\
93.53	0.01\\
93.54	0.01\\
93.55	0.01\\
93.56	0.01\\
93.57	0.01\\
93.58	0.01\\
93.59	0.01\\
93.6	0.01\\
93.61	0.01\\
93.62	0.01\\
93.63	0.01\\
93.64	0.01\\
93.65	0.01\\
93.66	0.01\\
93.67	0.01\\
93.68	0.01\\
93.69	0.01\\
93.7	0.01\\
93.71	0.01\\
93.72	0.01\\
93.73	0.01\\
93.74	0.01\\
93.75	0.01\\
93.76	0.01\\
93.77	0.01\\
93.78	0.01\\
93.79	0.01\\
93.8	0.01\\
93.81	0.01\\
93.82	0.01\\
93.83	0.01\\
93.84	0.01\\
93.85	0.01\\
93.86	0.01\\
93.87	0.01\\
93.88	0.01\\
93.89	0.01\\
93.9	0.01\\
93.91	0.01\\
93.92	0.01\\
93.93	0.01\\
93.94	0.01\\
93.95	0.01\\
93.96	0.01\\
93.97	0.01\\
93.98	0.01\\
93.99	0.01\\
94	0.01\\
94.01	0.01\\
94.02	0.01\\
94.03	0.01\\
94.04	0.01\\
94.05	0.01\\
94.06	0.01\\
94.07	0.01\\
94.08	0.01\\
94.09	0.01\\
94.1	0.01\\
94.11	0.01\\
94.12	0.01\\
94.13	0.01\\
94.14	0.01\\
94.15	0.01\\
94.16	0.01\\
94.17	0.01\\
94.18	0.01\\
94.19	0.01\\
94.2	0.01\\
94.21	0.01\\
94.22	0.01\\
94.23	0.01\\
94.24	0.01\\
94.25	0.01\\
94.26	0.01\\
94.27	0.01\\
94.28	0.01\\
94.29	0.01\\
94.3	0.01\\
94.31	0.01\\
94.32	0.01\\
94.33	0.01\\
94.34	0.01\\
94.35	0.01\\
94.36	0.01\\
94.37	0.01\\
94.38	0.01\\
94.39	0.01\\
94.4	0.01\\
94.41	0.01\\
94.42	0.01\\
94.43	0.01\\
94.44	0.01\\
94.45	0.01\\
94.46	0.01\\
94.47	0.01\\
94.48	0.01\\
94.49	0.01\\
94.5	0.01\\
94.51	0.01\\
94.52	0.01\\
94.53	0.01\\
94.54	0.01\\
94.55	0.01\\
94.56	0.01\\
94.57	0.01\\
94.58	0.01\\
94.59	0.01\\
94.6	0.01\\
94.61	0.01\\
94.62	0.01\\
94.63	0.01\\
94.64	0.01\\
94.65	0.01\\
94.66	0.01\\
94.67	0.01\\
94.68	0.01\\
94.69	0.01\\
94.7	0.01\\
94.71	0.01\\
94.72	0.01\\
94.73	0.01\\
94.74	0.01\\
94.75	0.01\\
94.76	0.01\\
94.77	0.01\\
94.78	0.01\\
94.79	0.01\\
94.8	0.01\\
94.81	0.01\\
94.82	0.01\\
94.83	0.01\\
94.84	0.01\\
94.85	0.01\\
94.86	0.01\\
94.87	0.01\\
94.88	0.01\\
94.89	0.01\\
94.9	0.01\\
94.91	0.01\\
94.92	0.01\\
94.93	0.01\\
94.94	0.01\\
94.95	0.01\\
94.96	0.01\\
94.97	0.01\\
94.98	0.01\\
94.99	0.01\\
95	0.01\\
95.01	0.01\\
95.02	0.01\\
95.03	0.01\\
95.04	0.01\\
95.05	0.01\\
95.06	0.01\\
95.07	0.01\\
95.08	0.01\\
95.09	0.01\\
95.1	0.01\\
95.11	0.01\\
95.12	0.01\\
95.13	0.01\\
95.14	0.01\\
95.15	0.01\\
95.16	0.01\\
95.17	0.01\\
95.18	0.01\\
95.19	0.01\\
95.2	0.01\\
95.21	0.01\\
95.22	0.01\\
95.23	0.01\\
95.24	0.01\\
95.25	0.01\\
95.26	0.01\\
95.27	0.01\\
95.28	0.01\\
95.29	0.01\\
95.3	0.01\\
95.31	0.01\\
95.32	0.01\\
95.33	0.01\\
95.34	0.01\\
95.35	0.01\\
95.36	0.01\\
95.37	0.01\\
95.38	0.01\\
95.39	0.01\\
95.4	0.01\\
95.41	0.01\\
95.42	0.01\\
95.43	0.01\\
95.44	0.01\\
95.45	0.01\\
95.46	0.01\\
95.47	0.01\\
95.48	0.01\\
95.49	0.01\\
95.5	0.01\\
95.51	0.01\\
95.52	0.01\\
95.53	0.01\\
95.54	0.01\\
95.55	0.01\\
95.56	0.01\\
95.57	0.01\\
95.58	0.01\\
95.59	0.01\\
95.6	0.01\\
95.61	0.01\\
95.62	0.01\\
95.63	0.01\\
95.64	0.01\\
95.65	0.01\\
95.66	0.01\\
95.67	0.01\\
95.68	0.01\\
95.69	0.01\\
95.7	0.01\\
95.71	0.01\\
95.72	0.01\\
95.73	0.01\\
95.74	0.01\\
95.75	0.01\\
95.76	0.01\\
95.77	0.01\\
95.78	0.01\\
95.79	0.01\\
95.8	0.01\\
95.81	0.01\\
95.82	0.01\\
95.83	0.01\\
95.84	0.01\\
95.85	0.01\\
95.86	0.01\\
95.87	0.01\\
95.88	0.01\\
95.89	0.01\\
95.9	0.01\\
95.91	0.01\\
95.92	0.01\\
95.93	0.01\\
95.94	0.01\\
95.95	0.01\\
95.96	0.01\\
95.97	0.01\\
95.98	0.01\\
95.99	0.01\\
96	0.01\\
96.01	0.01\\
96.02	0.01\\
96.03	0.01\\
96.04	0.01\\
96.05	0.01\\
96.06	0.01\\
96.07	0.01\\
96.08	0.01\\
96.09	0.01\\
96.1	0.01\\
96.11	0.01\\
96.12	0.01\\
96.13	0.01\\
96.14	0.01\\
96.15	0.01\\
96.16	0.01\\
96.17	0.01\\
96.18	0.01\\
96.19	0.01\\
96.2	0.01\\
96.21	0.01\\
96.22	0.01\\
96.23	0.01\\
96.24	0.01\\
96.25	0.01\\
96.26	0.01\\
96.27	0.01\\
96.28	0.01\\
96.29	0.01\\
96.3	0.01\\
96.31	0.01\\
96.32	0.01\\
96.33	0.01\\
96.34	0.01\\
96.35	0.01\\
96.36	0.01\\
96.37	0.01\\
96.38	0.01\\
96.39	0.01\\
96.4	0.01\\
96.41	0.01\\
96.42	0.01\\
96.43	0.01\\
96.44	0.01\\
96.45	0.01\\
96.46	0.01\\
96.47	0.01\\
96.48	0.01\\
96.49	0.01\\
96.5	0.01\\
96.51	0.01\\
96.52	0.01\\
96.53	0.01\\
96.54	0.01\\
96.55	0.01\\
96.56	0.01\\
96.57	0.01\\
96.58	0.01\\
96.59	0.01\\
96.6	0.01\\
96.61	0.01\\
96.62	0.01\\
96.63	0.01\\
96.64	0.01\\
96.65	0.01\\
96.66	0.01\\
96.67	0.01\\
96.68	0.01\\
96.69	0.01\\
96.7	0.01\\
96.71	0.01\\
96.72	0.01\\
96.73	0.01\\
96.74	0.01\\
96.75	0.01\\
96.76	0.01\\
96.77	0.01\\
96.78	0.01\\
96.79	0.01\\
96.8	0.01\\
96.81	0.01\\
96.82	0.01\\
96.83	0.01\\
96.84	0.01\\
96.85	0.01\\
96.86	0.01\\
96.87	0.01\\
96.88	0.01\\
96.89	0.01\\
96.9	0.01\\
96.91	0.01\\
96.92	0.01\\
96.93	0.01\\
96.94	0.01\\
96.95	0.01\\
96.96	0.01\\
96.97	0.01\\
96.98	0.01\\
96.99	0.01\\
97	0.01\\
97.01	0.01\\
97.02	0.01\\
97.03	0.01\\
97.04	0.01\\
97.05	0.01\\
97.06	0.01\\
97.07	0.01\\
97.08	0.01\\
97.09	0.01\\
97.1	0.01\\
97.11	0.01\\
97.12	0.01\\
97.13	0.01\\
97.14	0.01\\
97.15	0.01\\
97.16	0.01\\
97.17	0.01\\
97.18	0.01\\
97.19	0.01\\
97.2	0.01\\
97.21	0.01\\
97.22	0.01\\
97.23	0.01\\
97.24	0.01\\
97.25	0.01\\
97.26	0.01\\
97.27	0.01\\
97.28	0.01\\
97.29	0.01\\
97.3	0.01\\
97.31	0.01\\
97.32	0.01\\
97.33	0.01\\
97.34	0.01\\
97.35	0.01\\
97.36	0.01\\
97.37	0.01\\
97.38	0.01\\
97.39	0.01\\
97.4	0.01\\
97.41	0.01\\
97.42	0.01\\
97.43	0.01\\
97.44	0.01\\
97.45	0.01\\
97.46	0.01\\
97.47	0.01\\
97.48	0.01\\
97.49	0.01\\
97.5	0.01\\
97.51	0.01\\
97.52	0.01\\
97.53	0.01\\
97.54	0.01\\
97.55	0.01\\
97.56	0.01\\
97.57	0.01\\
97.58	0.01\\
97.59	0.01\\
97.6	0.01\\
97.61	0.01\\
97.62	0.01\\
97.63	0.01\\
97.64	0.01\\
97.65	0.01\\
97.66	0.01\\
97.67	0.01\\
97.68	0.01\\
97.69	0.01\\
97.7	0.01\\
97.71	0.01\\
97.72	0.01\\
97.73	0.01\\
97.74	0.01\\
97.75	0.01\\
97.76	0.01\\
97.77	0.01\\
97.78	0.01\\
97.79	0.01\\
97.8	0.01\\
97.81	0.01\\
97.82	0.01\\
97.83	0.01\\
97.84	0.01\\
97.85	0.01\\
97.86	0.01\\
97.87	0.01\\
97.88	0.01\\
97.89	0.01\\
97.9	0.01\\
97.91	0.01\\
97.92	0.01\\
97.93	0.01\\
97.94	0.01\\
97.95	0.01\\
97.96	0.01\\
97.97	0.01\\
97.98	0.01\\
97.99	0.01\\
98	0.01\\
98.01	0.01\\
98.02	0.01\\
98.03	0.01\\
98.04	0.01\\
98.05	0.01\\
98.06	0.01\\
98.07	0.01\\
98.08	0.01\\
98.09	0.01\\
98.1	0.01\\
98.11	0.01\\
98.12	0.01\\
98.13	0.01\\
98.14	0.01\\
98.15	0.01\\
98.16	0.01\\
98.17	0.01\\
98.18	0.01\\
98.19	0.01\\
98.2	0.01\\
98.21	0.01\\
98.22	0.01\\
98.23	0.01\\
98.24	0.01\\
98.25	0.01\\
98.26	0.01\\
98.27	0.01\\
98.28	0.01\\
98.29	0.01\\
98.3	0.01\\
98.31	0.01\\
98.32	0.01\\
98.33	0.01\\
98.34	0.01\\
98.35	0.01\\
98.36	0.01\\
98.37	0.01\\
98.38	0.01\\
98.39	0.01\\
98.4	0.01\\
98.41	0.01\\
98.42	0.01\\
98.43	0.01\\
98.44	0.01\\
98.45	0.01\\
98.46	0.01\\
98.47	0.01\\
98.48	0.01\\
98.49	0.01\\
98.5	0.01\\
98.51	0.01\\
98.52	0.01\\
98.53	0.01\\
98.54	0.01\\
98.55	0.01\\
98.56	0.01\\
98.57	0.01\\
98.58	0.01\\
98.59	0.01\\
98.6	0.01\\
98.61	0.01\\
98.62	0.01\\
98.63	0.01\\
98.64	0.01\\
98.65	0.01\\
98.66	0.01\\
98.67	0.01\\
98.68	0.01\\
98.69	0.01\\
98.7	0.01\\
98.71	0.01\\
98.72	0.01\\
98.73	0.01\\
98.74	0.01\\
98.75	0.01\\
98.76	0.01\\
98.77	0.01\\
98.78	0.01\\
98.79	0.01\\
98.8	0.01\\
98.81	0.01\\
98.82	0.01\\
98.83	0.01\\
98.84	0.01\\
98.85	0.01\\
98.86	0.01\\
98.87	0.01\\
98.88	0.01\\
98.89	0.01\\
98.9	0.01\\
98.91	0.01\\
98.92	0.01\\
98.93	0.01\\
98.94	0.01\\
98.95	0.01\\
98.96	0.01\\
98.97	0.01\\
98.98	0.01\\
98.99	0.01\\
99	0.01\\
99.01	0.01\\
99.02	0.01\\
99.03	0.01\\
99.04	0.01\\
99.05	0.01\\
99.06	0.01\\
99.07	0.01\\
99.08	0.01\\
99.09	0.01\\
99.1	0.01\\
99.11	0.01\\
99.12	0.01\\
99.13	0.01\\
99.14	0.01\\
99.15	0.01\\
99.16	0.01\\
99.17	0.01\\
99.18	0.01\\
99.19	0.01\\
99.2	0.01\\
99.21	0.01\\
99.22	0.01\\
99.23	0.01\\
99.24	0.01\\
99.25	0.01\\
99.26	0.01\\
99.27	0.01\\
99.28	0.01\\
99.29	0.01\\
99.3	0.01\\
99.31	0.01\\
99.32	0.01\\
99.33	0.01\\
99.34	0.01\\
99.35	0.01\\
99.36	0.01\\
99.37	0.01\\
99.38	0.01\\
99.39	0.01\\
99.4	0.01\\
99.41	0.01\\
99.42	0.0098574925152882\\
99.43	0.00971396905841077\\
99.44	0.00956972004081591\\
99.45	0.0094247362673596\\
99.46	0.0092790082849791\\
99.47	0.00913252639168261\\
99.48	0.00898529998637358\\
99.49	0.00883734285850463\\
99.5	0.00868864559125499\\
99.51	0.00853919851239646\\
99.52	0.00838899168717304\\
99.53	0.00823801491096786\\
99.54	0.00808625770174855\\
99.55	0.00793370929228211\\
99.56	0.00778035862210953\\
99.57	0.00762619432926998\\
99.58	0.00747120474176386\\
99.59	0.00731537786874324\\
99.6	0.00715870139141777\\
99.61	0.00700116265366315\\
99.62	0.00684274865231885\\
99.63	0.00668344602716063\\
99.64	0.00652324105053277\\
99.65	0.00636211961662405\\
99.66	0.00620006723037029\\
99.67	0.00603706899596548\\
99.68	0.0058731096049623\\
99.69	0.00570817332394313\\
99.7	0.00554224398173856\\
99.71	0.00537530495617044\\
99.72	0.00520733916022993\\
99.73	0.00503832902765711\\
99.74	0.00486825649810313\\
99.75	0.00469710300170934\\
99.76	0.00452484944307186\\
99.77	0.00435147618455827\\
99.78	0.00417696302894062\\
99.79	0.00400128920130679\\
99.8	0.00382443333020954\\
99.81	0.00364637342800988\\
99.82	0.00346708687036842\\
99.83	0.00328655037483507\\
99.84	0.00310473997848419\\
99.85	0.00292163101453843\\
99.86	0.00273719808792035\\
99.87	0.00255141504966703\\
99.88	0.00236425497013749\\
99.89	0.0021756901109384\\
99.9	0.00198569189548735\\
99.91	0.0017942310464843\\
99.92	0.00160127761552209\\
99.93	0.00140680072665657\\
99.94	0.00121076854085607\\
99.95	0.00101314821860146\\
99.96	0.000813905880513196\\
99.97	0.000613006565871897\\
99.98	0.000410414188888463\\
99.99	0.000206091492568098\\
100	0\\
};
\addlegendentry{$q=-1$};

\addplot [color=black,solid,forget plot]
  table[row sep=crcr]{%
0.01	0.000772572677978338\\
0.02	0.000772572809190752\\
0.03	0.000772572940773125\\
0.04	0.000772573072723257\\
0.05	0.000772573205038835\\
0.06	0.000772573337717434\\
0.07	0.000772573470756514\\
0.08	0.000772573604153424\\
0.09	0.000772573737905393\\
0.1	0.00077257387200954\\
0.11	0.000772574006462861\\
0.12	0.00077257414126224\\
0.13	0.000772574276404437\\
0.14	0.000772574411886101\\
0.15	0.000772574547703757\\
0.16	0.000772574683853814\\
0.17	0.000772574820332559\\
0.18	0.000772574957136164\\
0.19	0.00077257509426068\\
0.2	0.000772575231702044\\
0.21	0.000772575369456071\\
0.22	0.000772575507518461\\
0.23	0.000772575645884796\\
0.24	0.000772575784550549\\
0.25	0.000772575923511074\\
0.26	0.000772576062761616\\
0.27	0.00077257620229731\\
0.28	0.000772576342113182\\
0.29	0.000772576482204152\\
0.3	0.000772576622565036\\
0.31	0.000772576763190553\\
0.32	0.00077257690407532\\
0.33	0.000772577045213864\\
0.34	0.000772577186600619\\
0.35	0.000772577328229931\\
0.36	0.000772577470096068\\
0.37	0.00077257761219322\\
0.38	0.000772577754515502\\
0.39	0.000772577897056967\\
0.4	0.000772578039811603\\
0.41	0.000772578182773346\\
0.42	0.000772578325936087\\
0.43	0.000772578469293671\\
0.44	0.000772578612839919\\
0.45	0.000772578756568621\\
0.46	0.00077257890047356\\
0.47	0.000772579044548506\\
0.48	0.00077257918878724\\
0.49	0.000772579333183558\\
0.5	0.000772579477731282\\
0.51	0.000772579622424278\\
0.52	0.000772579767256459\\
0.53	0.000772579912221807\\
0.54	0.000772580057314386\\
0.55	0.000772580202528355\\
0.56	0.000772580347857987\\
0.57	0.000772580493297682\\
0.58	0.00077258063884199\\
0.59	0.000772580784485626\\
0.6	0.000772580930223494\\
0.61	0.000772581076050702\\
0.62	0.00077258122196259\\
0.63	0.000772581367954753\\
0.64	0.000772581514023063\\
0.65	0.000772581660163694\\
0.66	0.000772581806373152\\
0.67	0.000772581952648302\\
0.68	0.0007725820989864\\
0.69	0.000772582245385123\\
0.7	0.000772582391842599\\
0.71	0.000772582538357448\\
0.72	0.000772582684928812\\
0.73	0.000772582831556399\\
0.74	0.000772582978240231\\
0.75	0.00077258312498033\\
0.76	0.000772583271776721\\
0.77	0.000772583418629428\\
0.78	0.000772583565538474\\
0.79	0.000772583712503881\\
0.8	0.000772583859525675\\
0.81	0.000772584006603879\\
0.82	0.000772584153738515\\
0.83	0.000772584300929607\\
0.84	0.000772584448177181\\
0.85	0.000772584595481258\\
0.86	0.000772584742841862\\
0.87	0.000772584890259017\\
0.88	0.000772585037732747\\
0.89	0.000772585185263075\\
0.9	0.000772585332850025\\
0.91	0.00077258548049362\\
0.92	0.000772585628193885\\
0.93	0.000772585775950843\\
0.94	0.000772585923764517\\
0.95	0.000772586071634931\\
0.96	0.00077258621956211\\
0.97	0.000772586367546077\\
0.98	0.000772586515586855\\
0.99	0.000772586663684469\\
1	0.000772586811838942\\
1.01	0.000772586960050298\\
1.02	0.000772587108318561\\
1.03	0.000772587256643753\\
1.04	0.0007725874050259\\
1.05	0.000772587553465025\\
1.06	0.000772587701961152\\
1.07	0.000772587850514306\\
1.08	0.000772587999124509\\
1.09	0.000772588147791785\\
1.1	0.00077258829651616\\
1.11	0.000772588445297656\\
1.12	0.000772588594136297\\
1.13	0.000772588743032107\\
1.14	0.00077258889198511\\
1.15	0.000772589040995331\\
1.16	0.000772589190062793\\
1.17	0.000772589339187519\\
1.18	0.000772589488369535\\
1.19	0.000772589637608864\\
1.2	0.000772589786905529\\
1.21	0.000772589936259557\\
1.22	0.000772590085670969\\
1.23	0.00077259023513979\\
1.24	0.000772590384666044\\
1.25	0.000772590534249756\\
1.26	0.000772590683890949\\
1.27	0.000772590833589649\\
1.28	0.000772590983345877\\
1.29	0.00077259113315966\\
1.3	0.00077259128303102\\
1.31	0.000772591432959982\\
1.32	0.000772591582946571\\
1.33	0.00077259173299081\\
1.34	0.000772591883092724\\
1.35	0.000772592033252336\\
1.36	0.000772592183469671\\
1.37	0.000772592333744753\\
1.38	0.000772592484077607\\
1.39	0.000772592634468256\\
1.4	0.000772592784916724\\
1.41	0.000772592935423037\\
1.42	0.000772593085987218\\
1.43	0.000772593236609291\\
1.44	0.000772593387289282\\
1.45	0.000772593538027215\\
1.46	0.000772593688823112\\
1.47	0.000772593839676998\\
1.48	0.000772593990588899\\
1.49	0.000772594141558839\\
1.5	0.000772594292586842\\
1.51	0.000772594443672931\\
1.52	0.000772594594817133\\
1.53	0.000772594746019472\\
1.54	0.000772594897279971\\
1.55	0.000772595048598654\\
1.56	0.000772595199975547\\
1.57	0.000772595351410674\\
1.58	0.00077259550290406\\
1.59	0.000772595654455728\\
1.6	0.000772595806065705\\
1.61	0.000772595957734013\\
1.62	0.000772596109460677\\
1.63	0.000772596261245722\\
1.64	0.000772596413089173\\
1.65	0.000772596564991053\\
1.66	0.000772596716951388\\
1.67	0.000772596868970202\\
1.68	0.000772597021047519\\
1.69	0.000772597173183365\\
1.7	0.000772597325377765\\
1.71	0.00077259747763074\\
1.72	0.000772597629942319\\
1.73	0.000772597782312524\\
1.74	0.000772597934741381\\
1.75	0.000772598087228914\\
1.76	0.000772598239775149\\
1.77	0.000772598392380109\\
1.78	0.000772598545043818\\
1.79	0.000772598697766304\\
1.8	0.00077259885054759\\
1.81	0.0007725990033877\\
1.82	0.00077259915628666\\
1.83	0.000772599309244492\\
1.84	0.000772599462261223\\
1.85	0.000772599615336879\\
1.86	0.000772599768471483\\
1.87	0.000772599921665061\\
1.88	0.000772600074917636\\
1.89	0.000772600228229234\\
1.9	0.00077260038159988\\
1.91	0.000772600535029599\\
1.92	0.000772600688518416\\
1.93	0.000772600842066354\\
1.94	0.000772600995673441\\
1.95	0.0007726011493397\\
1.96	0.000772601303065156\\
1.97	0.000772601456849834\\
1.98	0.000772601610693759\\
1.99	0.000772601764596956\\
2	0.000772601918559452\\
2.01	0.000772602072581268\\
2.02	0.000772602226662433\\
2.03	0.00077260238080297\\
2.04	0.000772602535002904\\
2.05	0.000772602689262259\\
2.06	0.000772602843581062\\
2.07	0.000772602997959337\\
2.08	0.000772603152397111\\
2.09	0.000772603306894406\\
2.1	0.000772603461451249\\
2.11	0.000772603616067666\\
2.12	0.000772603770743681\\
2.13	0.000772603925479318\\
2.14	0.000772604080274604\\
2.15	0.000772604235129563\\
2.16	0.000772604390044221\\
2.17	0.000772604545018603\\
2.18	0.000772604700052734\\
2.19	0.000772604855146639\\
2.2	0.000772605010300343\\
2.21	0.000772605165513873\\
2.22	0.000772605320787252\\
2.23	0.000772605476120507\\
2.24	0.000772605631513662\\
2.25	0.000772605786966743\\
2.26	0.000772605942479776\\
2.27	0.000772606098052785\\
2.28	0.000772606253685797\\
2.29	0.000772606409378835\\
2.3	0.000772606565131926\\
2.31	0.000772606720945096\\
2.32	0.000772606876818368\\
2.33	0.00077260703275177\\
2.34	0.000772607188745325\\
2.35	0.000772607344799061\\
2.36	0.000772607500913002\\
2.37	0.000772607657087173\\
2.38	0.000772607813321599\\
2.39	0.000772607969616307\\
2.4	0.000772608125971323\\
2.41	0.000772608282386671\\
2.42	0.000772608438862376\\
2.43	0.000772608595398466\\
2.44	0.000772608751994966\\
2.45	0.0007726089086519\\
2.46	0.000772609065369294\\
2.47	0.000772609222147174\\
2.48	0.000772609378985566\\
2.49	0.000772609535884494\\
2.5	0.000772609692843984\\
2.51	0.000772609849864063\\
2.52	0.000772610006944757\\
2.53	0.00077261016408609\\
2.54	0.000772610321288088\\
2.55	0.000772610478550777\\
2.56	0.000772610635874183\\
2.57	0.000772610793258332\\
2.58	0.000772610950703249\\
2.59	0.000772611108208959\\
2.6	0.000772611265775488\\
2.61	0.000772611423402863\\
2.62	0.000772611581091109\\
2.63	0.000772611738840252\\
2.64	0.000772611896650319\\
2.65	0.000772612054521334\\
2.66	0.000772612212453323\\
2.67	0.000772612370446313\\
2.68	0.000772612528500329\\
2.69	0.000772612686615397\\
2.7	0.000772612844791543\\
2.71	0.000772613003028793\\
2.72	0.000772613161327172\\
2.73	0.000772613319686708\\
2.74	0.000772613478107424\\
2.75	0.000772613636589347\\
2.76	0.000772613795132505\\
2.77	0.000772613953736923\\
2.78	0.000772614112402625\\
2.79	0.000772614271129639\\
2.8	0.00077261442991799\\
2.81	0.000772614588767704\\
2.82	0.000772614747678809\\
2.83	0.000772614906651328\\
2.84	0.00077261506568529\\
2.85	0.000772615224780719\\
2.86	0.000772615383937643\\
2.87	0.000772615543156087\\
2.88	0.000772615702436075\\
2.89	0.000772615861777636\\
2.9	0.000772616021180795\\
2.91	0.00077261618064558\\
2.92	0.000772616340172016\\
2.93	0.000772616499760128\\
2.94	0.000772616659409942\\
2.95	0.000772616819121486\\
2.96	0.000772616978894787\\
2.97	0.000772617138729868\\
2.98	0.000772617298626757\\
2.99	0.00077261745858548\\
3	0.000772617618606065\\
3.01	0.000772617778688537\\
3.02	0.000772617938832921\\
3.03	0.000772618099039245\\
3.04	0.000772618259307535\\
3.05	0.000772618419637817\\
3.06	0.000772618580030117\\
3.07	0.000772618740484463\\
3.08	0.000772618901000881\\
3.09	0.000772619061579395\\
3.1	0.000772619222220034\\
3.11	0.000772619382922824\\
3.12	0.000772619543687791\\
3.13	0.000772619704514961\\
3.14	0.000772619865404361\\
3.15	0.000772620026356018\\
3.16	0.000772620187369958\\
3.17	0.000772620348446208\\
3.18	0.000772620509584793\\
3.19	0.000772620670785739\\
3.2	0.000772620832049075\\
3.21	0.000772620993374827\\
3.22	0.000772621154763021\\
3.23	0.000772621316213685\\
3.24	0.000772621477726844\\
3.25	0.000772621639302523\\
3.26	0.000772621800940752\\
3.27	0.000772621962641556\\
3.28	0.000772622124404963\\
3.29	0.000772622286230998\\
3.3	0.000772622448119688\\
3.31	0.00077262261007106\\
3.32	0.000772622772085141\\
3.33	0.000772622934161958\\
3.34	0.000772623096301538\\
3.35	0.000772623258503906\\
3.36	0.00077262342076909\\
3.37	0.000772623583097116\\
3.38	0.000772623745488012\\
3.39	0.000772623907941803\\
3.4	0.000772624070458517\\
3.41	0.00077262423303818\\
3.42	0.000772624395680821\\
3.43	0.000772624558386465\\
3.44	0.00077262472115514\\
3.45	0.000772624883986873\\
3.46	0.000772625046881689\\
3.47	0.000772625209839615\\
3.48	0.000772625372860681\\
3.49	0.000772625535944911\\
3.5	0.000772625699092333\\
3.51	0.000772625862302975\\
3.52	0.000772626025576862\\
3.53	0.000772626188914022\\
3.54	0.000772626352314482\\
3.55	0.000772626515778269\\
3.56	0.000772626679305411\\
3.57	0.000772626842895932\\
3.58	0.000772627006549863\\
3.59	0.000772627170267229\\
3.6	0.000772627334048056\\
3.61	0.000772627497892373\\
3.62	0.000772627661800206\\
3.63	0.000772627825771584\\
3.64	0.000772627989806533\\
3.65	0.000772628153905079\\
3.66	0.000772628318067251\\
3.67	0.000772628482293074\\
3.68	0.000772628646582578\\
3.69	0.000772628810935788\\
3.7	0.000772628975352734\\
3.71	0.00077262913983344\\
3.72	0.000772629304377935\\
3.73	0.000772629468986246\\
3.74	0.0007726296336584\\
3.75	0.000772629798394424\\
3.76	0.000772629963194346\\
3.77	0.000772630128058192\\
3.78	0.000772630292985993\\
3.79	0.000772630457977772\\
3.8	0.000772630623033559\\
3.81	0.000772630788153381\\
3.82	0.000772630953337265\\
3.83	0.000772631118585238\\
3.84	0.000772631283897329\\
3.85	0.000772631449273564\\
3.86	0.00077263161471397\\
3.87	0.000772631780218577\\
3.88	0.00077263194578741\\
3.89	0.000772632111420497\\
3.9	0.000772632277117866\\
3.91	0.000772632442879544\\
3.92	0.00077263260870556\\
3.93	0.00077263277459594\\
3.94	0.000772632940550712\\
3.95	0.000772633106569904\\
3.96	0.000772633272653544\\
3.97	0.000772633438801658\\
3.98	0.000772633605014275\\
3.99	0.000772633771291421\\
4	0.000772633937633126\\
4.01	0.000772634104039415\\
4.02	0.000772634270510318\\
4.03	0.000772634437045863\\
4.04	0.000772634603646075\\
4.05	0.000772634770310985\\
4.06	0.000772634937040618\\
4.07	0.000772635103835004\\
4.08	0.000772635270694169\\
4.09	0.000772635437618142\\
4.1	0.00077263560460695\\
4.11	0.000772635771660622\\
4.12	0.000772635938779184\\
4.13	0.000772636105962664\\
4.14	0.000772636273211093\\
4.15	0.000772636440524496\\
4.16	0.0007726366079029\\
4.17	0.000772636775346336\\
4.18	0.000772636942854829\\
4.19	0.000772637110428409\\
4.2	0.000772637278067104\\
4.21	0.000772637445770941\\
4.22	0.000772637613539948\\
4.23	0.000772637781374153\\
4.24	0.000772637949273583\\
4.25	0.000772638117238269\\
4.26	0.000772638285268235\\
4.27	0.000772638453363513\\
4.28	0.000772638621524128\\
4.29	0.00077263878975011\\
4.3	0.000772638958041487\\
4.31	0.000772639126398286\\
4.32	0.000772639294820536\\
4.33	0.000772639463308265\\
4.34	0.000772639631861501\\
4.35	0.000772639800480273\\
4.36	0.000772639969164607\\
4.37	0.000772640137914532\\
4.38	0.000772640306730079\\
4.39	0.000772640475611272\\
4.4	0.000772640644558143\\
4.41	0.000772640813570718\\
4.42	0.000772640982649026\\
4.43	0.000772641151793095\\
4.44	0.000772641321002953\\
4.45	0.000772641490278629\\
4.46	0.000772641659620151\\
4.47	0.000772641829027547\\
4.48	0.000772641998500847\\
4.49	0.000772642168040077\\
4.5	0.000772642337645268\\
4.51	0.000772642507316446\\
4.52	0.000772642677053641\\
4.53	0.00077264284685688\\
4.54	0.000772643016726193\\
4.55	0.000772643186661608\\
4.56	0.000772643356663152\\
4.57	0.000772643526730855\\
4.58	0.000772643696864746\\
4.59	0.000772643867064852\\
4.6	0.000772644037331202\\
4.61	0.000772644207663825\\
4.62	0.000772644378062749\\
4.63	0.000772644548528004\\
4.64	0.000772644719059617\\
4.65	0.000772644889657618\\
4.66	0.000772645060322034\\
4.67	0.000772645231052894\\
4.68	0.000772645401850229\\
4.69	0.000772645572714064\\
4.7	0.000772645743644431\\
4.71	0.000772645914641356\\
4.72	0.00077264608570487\\
4.73	0.000772646256835001\\
4.74	0.000772646428031777\\
4.75	0.000772646599295226\\
4.76	0.00077264677062538\\
4.77	0.000772646942022264\\
4.78	0.000772647113485909\\
4.79	0.000772647285016343\\
4.8	0.000772647456613595\\
4.81	0.000772647628277695\\
4.82	0.00077264780000867\\
4.83	0.00077264797180655\\
4.84	0.000772648143671364\\
4.85	0.00077264831560314\\
4.86	0.000772648487601908\\
4.87	0.000772648659667696\\
4.88	0.000772648831800534\\
4.89	0.00077264900400045\\
4.9	0.000772649176267473\\
4.91	0.000772649348601634\\
4.92	0.00077264952100296\\
4.93	0.00077264969347148\\
4.94	0.000772649866007225\\
4.95	0.000772650038610221\\
4.96	0.000772650211280498\\
4.97	0.000772650384018088\\
4.98	0.000772650556823016\\
4.99	0.000772650729695314\\
5	0.00077265090263501\\
5.01	0.000772651075642132\\
5.02	0.000772651248716713\\
5.03	0.000772651421858778\\
5.04	0.000772651595068358\\
5.05	0.000772651768345484\\
5.06	0.000772651941690182\\
5.07	0.000772652115102482\\
5.08	0.000772652288582414\\
5.09	0.000772652462130007\\
5.1	0.00077265263574529\\
5.11	0.000772652809428294\\
5.12	0.000772652983179047\\
5.13	0.000772653156997578\\
5.14	0.000772653330883917\\
5.15	0.000772653504838093\\
5.16	0.000772653678860136\\
5.17	0.000772653852950075\\
5.18	0.000772654027107939\\
5.19	0.000772654201333758\\
5.2	0.000772654375627561\\
5.21	0.000772654549989378\\
5.22	0.000772654724419237\\
5.23	0.000772654898917169\\
5.24	0.000772655073483204\\
5.25	0.000772655248117371\\
5.26	0.000772655422819698\\
5.27	0.000772655597590217\\
5.28	0.000772655772428956\\
5.29	0.000772655947335945\\
5.3	0.000772656122311213\\
5.31	0.00077265629735479\\
5.32	0.000772656472466706\\
5.33	0.000772656647646992\\
5.34	0.000772656822895675\\
5.35	0.000772656998212787\\
5.36	0.000772657173598356\\
5.37	0.000772657349052411\\
5.38	0.000772657524574984\\
5.39	0.000772657700166103\\
5.4	0.000772657875825799\\
5.41	0.0007726580515541\\
5.42	0.000772658227351038\\
5.43	0.000772658403216641\\
5.44	0.000772658579150941\\
5.45	0.000772658755153966\\
5.46	0.000772658931225747\\
5.47	0.000772659107366312\\
5.48	0.000772659283575693\\
5.49	0.000772659459853919\\
5.5	0.000772659636201021\\
5.51	0.000772659812617028\\
5.52	0.000772659989101969\\
5.53	0.000772660165655874\\
5.54	0.000772660342278774\\
5.55	0.000772660518970699\\
5.56	0.00077266069573168\\
5.57	0.000772660872561745\\
5.58	0.000772661049460924\\
5.59	0.000772661226429249\\
5.6	0.000772661403466749\\
5.61	0.000772661580573453\\
5.62	0.000772661757749394\\
5.63	0.000772661934994598\\
5.64	0.0007726621123091\\
5.65	0.000772662289692926\\
5.66	0.000772662467146109\\
5.67	0.000772662644668676\\
5.68	0.000772662822260661\\
5.69	0.000772662999922092\\
5.7	0.000772663177653\\
5.71	0.000772663355453415\\
5.72	0.000772663533323366\\
5.73	0.000772663711262885\\
5.74	0.000772663889272001\\
5.75	0.000772664067350746\\
5.76	0.00077266424549915\\
5.77	0.000772664423717241\\
5.78	0.000772664602005053\\
5.79	0.000772664780362614\\
5.8	0.000772664958789955\\
5.81	0.000772665137287106\\
5.82	0.000772665315854097\\
5.83	0.000772665494490959\\
5.84	0.000772665673197724\\
5.85	0.00077266585197442\\
5.86	0.00077266603082108\\
5.87	0.000772666209737732\\
5.88	0.000772666388724409\\
5.89	0.000772666567781138\\
5.9	0.000772666746907952\\
5.91	0.000772666926104882\\
5.92	0.000772667105371957\\
5.93	0.000772667284709209\\
5.94	0.000772667464116668\\
5.95	0.000772667643594365\\
5.96	0.00077266782314233\\
5.97	0.000772668002760594\\
5.98	0.000772668182449188\\
5.99	0.000772668362208143\\
6	0.000772668542037488\\
6.01	0.000772668721937257\\
6.02	0.000772668901907478\\
6.03	0.000772669081948183\\
6.04	0.000772669262059402\\
6.05	0.000772669442241166\\
6.06	0.000772669622493505\\
6.07	0.000772669802816451\\
6.08	0.000772669983210034\\
6.09	0.000772670163674286\\
6.1	0.000772670344209237\\
6.11	0.000772670524814919\\
6.12	0.000772670705491361\\
6.13	0.000772670886238596\\
6.14	0.000772671067056654\\
6.15	0.000772671247945566\\
6.16	0.000772671428905362\\
6.17	0.000772671609936075\\
6.18	0.000772671791037734\\
6.19	0.000772671972210372\\
6.2	0.000772672153454019\\
6.21	0.000772672334768706\\
6.22	0.000772672516154463\\
6.23	0.000772672697611323\\
6.24	0.000772672879139317\\
6.25	0.000772673060738475\\
6.26	0.000772673242408829\\
6.27	0.000772673424150409\\
6.28	0.000772673605963246\\
6.29	0.000772673787847374\\
6.3	0.000772673969802821\\
6.31	0.000772674151829621\\
6.32	0.000772674333927803\\
6.33	0.0007726745160974\\
6.34	0.000772674698338442\\
6.35	0.000772674880650961\\
6.36	0.000772675063034987\\
6.37	0.000772675245490554\\
6.38	0.000772675428017689\\
6.39	0.000772675610616428\\
6.4	0.0007726757932868\\
6.41	0.000772675976028836\\
6.42	0.000772676158842569\\
6.43	0.000772676341728029\\
6.44	0.000772676524685248\\
6.45	0.000772676707714259\\
6.46	0.00077267689081509\\
6.47	0.000772677073987775\\
6.48	0.000772677257232346\\
6.49	0.000772677440548833\\
6.5	0.000772677623937268\\
6.51	0.000772677807397683\\
6.52	0.00077267799093011\\
6.53	0.000772678174534578\\
6.54	0.000772678358211123\\
6.55	0.000772678541959773\\
6.56	0.00077267872578056\\
6.57	0.000772678909673516\\
6.58	0.000772679093638673\\
6.59	0.000772679277676065\\
6.6	0.000772679461785719\\
6.61	0.00077267964596767\\
6.62	0.000772679830221949\\
6.63	0.000772680014548588\\
6.64	0.000772680198947618\\
6.65	0.000772680383419072\\
6.66	0.000772680567962982\\
6.67	0.000772680752579377\\
6.68	0.000772680937268292\\
6.69	0.000772681122029757\\
6.7	0.000772681306863804\\
6.71	0.000772681491770467\\
6.72	0.000772681676749775\\
6.73	0.000772681861801763\\
6.74	0.000772682046926459\\
6.75	0.000772682232123898\\
6.76	0.000772682417394113\\
6.77	0.000772682602737133\\
6.78	0.000772682788152991\\
6.79	0.00077268297364172\\
6.8	0.00077268315920335\\
6.81	0.000772683344837915\\
6.82	0.000772683530545446\\
6.83	0.000772683716325976\\
6.84	0.000772683902179536\\
6.85	0.000772684088106159\\
6.86	0.000772684274105876\\
6.87	0.000772684460178721\\
6.88	0.000772684646324726\\
6.89	0.000772684832543922\\
6.9	0.000772685018836341\\
6.91	0.000772685205202017\\
6.92	0.00077268539164098\\
6.93	0.000772685578153265\\
6.94	0.000772685764738903\\
6.95	0.000772685951397924\\
6.96	0.000772686138130365\\
6.97	0.000772686324936255\\
6.98	0.000772686511815626\\
6.99	0.000772686698768512\\
7	0.000772686885794944\\
7.01	0.000772687072894957\\
7.02	0.000772687260068581\\
7.03	0.000772687447315849\\
7.04	0.000772687634636794\\
7.05	0.000772687822031447\\
7.06	0.000772688009499842\\
7.07	0.000772688197042012\\
7.08	0.000772688384657988\\
7.09	0.000772688572347803\\
7.1	0.00077268876011149\\
7.11	0.000772688947949081\\
7.12	0.000772689135860609\\
7.13	0.000772689323846106\\
7.14	0.000772689511905606\\
7.15	0.000772689700039141\\
7.16	0.000772689888246744\\
7.17	0.000772690076528446\\
7.18	0.000772690264884281\\
7.19	0.000772690453314281\\
7.2	0.00077269064181848\\
7.21	0.00077269083039691\\
7.22	0.000772691019049605\\
7.23	0.000772691207776595\\
7.24	0.000772691396577915\\
7.25	0.000772691585453598\\
7.26	0.000772691774403676\\
7.27	0.00077269196342818\\
7.28	0.000772692152527146\\
7.29	0.000772692341700605\\
7.3	0.000772692530948591\\
7.31	0.000772692720271136\\
7.32	0.000772692909668275\\
7.33	0.000772693099140038\\
7.34	0.00077269328868646\\
7.35	0.000772693478307574\\
7.36	0.000772693668003411\\
7.37	0.000772693857774006\\
7.38	0.000772694047619392\\
7.39	0.000772694237539602\\
7.4	0.000772694427534668\\
7.41	0.000772694617604625\\
7.42	0.000772694807749504\\
7.43	0.000772694997969338\\
7.44	0.000772695188264162\\
7.45	0.000772695378634008\\
7.46	0.000772695569078909\\
7.47	0.0007726957595989\\
7.48	0.000772695950194012\\
7.49	0.00077269614086428\\
7.5	0.000772696331609736\\
7.51	0.000772696522430413\\
7.52	0.000772696713326346\\
7.53	0.000772696904297565\\
7.54	0.000772697095344107\\
7.55	0.000772697286466003\\
7.56	0.000772697477663288\\
7.57	0.000772697668935994\\
7.58	0.000772697860284156\\
7.59	0.000772698051707805\\
7.6	0.000772698243206978\\
7.61	0.000772698434781704\\
7.62	0.00077269862643202\\
7.63	0.000772698818157959\\
7.64	0.000772699009959552\\
7.65	0.000772699201836835\\
7.66	0.000772699393789841\\
7.67	0.000772699585818604\\
7.68	0.000772699777923156\\
7.69	0.000772699970103532\\
7.7	0.000772700162359766\\
7.71	0.00077270035469189\\
7.72	0.000772700547099938\\
7.73	0.000772700739583944\\
7.74	0.000772700932143941\\
7.75	0.000772701124779965\\
7.76	0.000772701317492047\\
7.77	0.000772701510280221\\
7.78	0.000772701703144522\\
7.79	0.000772701896084983\\
7.8	0.000772702089101638\\
7.81	0.000772702282194521\\
7.82	0.000772702475363665\\
7.83	0.000772702668609106\\
7.84	0.000772702861930876\\
7.85	0.000772703055329008\\
7.86	0.000772703248803538\\
7.87	0.000772703442354498\\
7.88	0.000772703635981923\\
7.89	0.000772703829685847\\
7.9	0.000772704023466304\\
7.91	0.000772704217323327\\
7.92	0.000772704411256951\\
7.93	0.000772704605267209\\
7.94	0.000772704799354136\\
7.95	0.000772704993517765\\
7.96	0.000772705187758132\\
7.97	0.000772705382075269\\
7.98	0.000772705576469211\\
7.99	0.000772705770939991\\
8	0.000772705965487645\\
8.01	0.000772706160112206\\
8.02	0.000772706354813708\\
8.03	0.000772706549592186\\
8.04	0.000772706744447673\\
8.05	0.000772706939380204\\
8.06	0.000772707134389813\\
8.07	0.000772707329476534\\
8.08	0.000772707524640402\\
8.09	0.000772707719881451\\
8.1	0.000772707915199714\\
8.11	0.000772708110595226\\
8.12	0.000772708306068024\\
8.13	0.000772708501618138\\
8.14	0.000772708697245605\\
8.15	0.000772708892950459\\
8.16	0.000772709088732735\\
8.17	0.000772709284592466\\
8.18	0.000772709480529687\\
8.19	0.000772709676544431\\
8.2	0.000772709872636735\\
8.21	0.000772710068806632\\
8.22	0.000772710265054157\\
8.23	0.000772710461379344\\
8.24	0.000772710657782227\\
8.25	0.000772710854262842\\
8.26	0.000772711050821223\\
8.27	0.000772711247457404\\
8.28	0.000772711444171421\\
8.29	0.000772711640963307\\
8.3	0.000772711837833099\\
8.31	0.000772712034780828\\
8.32	0.000772712231806531\\
8.33	0.000772712428910242\\
8.34	0.000772712626091996\\
8.35	0.000772712823351828\\
8.36	0.000772713020689772\\
8.37	0.000772713218105863\\
8.38	0.000772713415600135\\
8.39	0.000772713613172625\\
8.4	0.000772713810823366\\
8.41	0.000772714008552394\\
8.42	0.000772714206359743\\
8.43	0.000772714404245447\\
8.44	0.000772714602209542\\
8.45	0.000772714800252062\\
8.46	0.000772714998373041\\
8.47	0.000772715196572517\\
8.48	0.000772715394850522\\
8.49	0.000772715593207093\\
8.5	0.000772715791642264\\
8.51	0.00077271599015607\\
8.52	0.000772716188748546\\
8.53	0.000772716387419726\\
8.54	0.000772716586169647\\
8.55	0.000772716784998342\\
8.56	0.000772716983905847\\
8.57	0.000772717182892197\\
8.58	0.000772717381957427\\
8.59	0.000772717581101572\\
8.6	0.000772717780324669\\
8.61	0.00077271797962675\\
8.62	0.000772718179007852\\
8.63	0.000772718378468009\\
8.64	0.000772718578007257\\
8.65	0.000772718777625631\\
8.66	0.000772718977323167\\
8.67	0.000772719177099899\\
8.68	0.000772719376955862\\
8.69	0.000772719576891092\\
8.7	0.000772719776905625\\
8.71	0.000772719976999495\\
8.72	0.000772720177172738\\
8.73	0.000772720377425389\\
8.74	0.000772720577757482\\
8.75	0.000772720778169055\\
8.76	0.00077272097866014\\
8.77	0.000772721179230775\\
8.78	0.000772721379880995\\
8.79	0.000772721580610836\\
8.8	0.000772721781420332\\
8.81	0.000772721982309519\\
8.82	0.000772722183278432\\
8.83	0.000772722384327107\\
8.84	0.000772722585455578\\
8.85	0.000772722786663882\\
8.86	0.000772722987952053\\
8.87	0.000772723189320128\\
8.88	0.000772723390768143\\
8.89	0.000772723592296132\\
8.9	0.000772723793904131\\
8.91	0.000772723995592176\\
8.92	0.000772724197360302\\
8.93	0.000772724399208545\\
8.94	0.000772724601136941\\
8.95	0.000772724803145525\\
8.96	0.000772725005234332\\
8.97	0.0007727252074034\\
8.98	0.000772725409652762\\
8.99	0.000772725611982454\\
9	0.000772725814392514\\
9.01	0.000772726016882976\\
9.02	0.000772726219453875\\
9.03	0.000772726422105246\\
9.04	0.000772726624837128\\
9.05	0.000772726827649554\\
9.06	0.000772727030542561\\
9.07	0.000772727233516185\\
9.08	0.000772727436570459\\
9.09	0.000772727639705423\\
9.1	0.00077272784292111\\
9.11	0.000772728046217558\\
9.12	0.0007727282495948\\
9.13	0.000772728453052873\\
9.14	0.000772728656591814\\
9.15	0.000772728860211657\\
9.16	0.000772729063912439\\
9.17	0.000772729267694196\\
9.18	0.000772729471556963\\
9.19	0.000772729675500777\\
9.2	0.000772729879525673\\
9.21	0.000772730083631688\\
9.22	0.000772730287818857\\
9.23	0.000772730492087216\\
9.24	0.0007727306964368\\
9.25	0.000772730900867647\\
9.26	0.000772731105379792\\
9.27	0.000772731309973271\\
9.28	0.000772731514648121\\
9.29	0.000772731719404377\\
9.3	0.000772731924242074\\
9.31	0.000772732129161249\\
9.32	0.000772732334161939\\
9.33	0.000772732539244179\\
9.34	0.000772732744408006\\
9.35	0.000772732949653455\\
9.36	0.000772733154980563\\
9.37	0.000772733360389365\\
9.38	0.000772733565879898\\
9.39	0.000772733771452197\\
9.4	0.000772733977106298\\
9.41	0.000772734182842238\\
9.42	0.000772734388660054\\
9.43	0.000772734594559782\\
9.44	0.000772734800541457\\
9.45	0.000772735006605116\\
9.46	0.000772735212750795\\
9.47	0.000772735418978529\\
9.48	0.000772735625288356\\
9.49	0.00077273583168031\\
9.5	0.000772736038154429\\
9.51	0.000772736244710748\\
9.52	0.000772736451349305\\
9.53	0.000772736658070134\\
9.54	0.000772736864873273\\
9.55	0.000772737071758758\\
9.56	0.000772737278726624\\
9.57	0.000772737485776909\\
9.58	0.000772737692909647\\
9.59	0.000772737900124876\\
9.6	0.000772738107422632\\
9.61	0.00077273831480295\\
9.62	0.000772738522265868\\
9.63	0.000772738729811422\\
9.64	0.000772738937439647\\
9.65	0.000772739145150581\\
9.66	0.000772739352944259\\
9.67	0.000772739560820718\\
9.68	0.000772739768779993\\
9.69	0.000772739976822121\\
9.7	0.00077274018494714\\
9.71	0.000772740393155084\\
9.72	0.000772740601445989\\
9.73	0.000772740809819894\\
9.74	0.000772741018276834\\
9.75	0.000772741226816843\\
9.76	0.000772741435439961\\
9.77	0.000772741644146223\\
9.78	0.000772741852935665\\
9.79	0.000772742061808324\\
9.8	0.000772742270764234\\
9.81	0.000772742479803433\\
9.82	0.000772742688925958\\
9.83	0.000772742898131845\\
9.84	0.000772743107421129\\
9.85	0.000772743316793847\\
9.86	0.000772743526250038\\
9.87	0.000772743735789734\\
9.88	0.000772743945412975\\
9.89	0.000772744155119794\\
9.9	0.000772744364910229\\
9.91	0.000772744574784316\\
9.92	0.000772744784742091\\
9.93	0.000772744994783593\\
9.94	0.000772745204908856\\
9.95	0.000772745415117916\\
9.96	0.000772745625410808\\
9.97	0.000772745835787571\\
9.98	0.000772746046248241\\
9.99	0.000772746256792854\\
10	0.000772746467421445\\
10.01	0.000772746678134052\\
10.02	0.00077274688893071\\
10.03	0.000772747099811455\\
10.04	0.000772747310776326\\
10.05	0.000772747521825356\\
10.06	0.000772747732958583\\
10.07	0.000772747944176044\\
10.08	0.000772748155477773\\
10.09	0.000772748366863807\\
10.1	0.000772748578334183\\
10.11	0.000772748789888936\\
10.12	0.000772749001528104\\
10.13	0.000772749213251721\\
10.14	0.000772749425059826\\
10.15	0.000772749636952452\\
10.16	0.000772749848929638\\
10.17	0.000772750060991419\\
10.18	0.000772750273137831\\
10.19	0.000772750485368909\\
10.2	0.000772750697684692\\
10.21	0.000772750910085215\\
10.22	0.000772751122570513\\
10.23	0.000772751335140622\\
10.24	0.00077275154779558\\
10.25	0.000772751760535421\\
10.26	0.000772751973360183\\
10.27	0.0007727521862699\\
10.28	0.000772752399264609\\
10.29	0.000772752612344347\\
10.3	0.000772752825509149\\
10.31	0.000772753038759051\\
10.32	0.000772753252094088\\
10.33	0.000772753465514298\\
10.34	0.000772753679019716\\
10.35	0.000772753892610379\\
10.36	0.000772754106286321\\
10.37	0.000772754320047577\\
10.38	0.000772754533894186\\
10.39	0.000772754747826183\\
10.4	0.000772754961843604\\
10.41	0.000772755175946483\\
10.42	0.000772755390134857\\
10.43	0.000772755604408761\\
10.44	0.000772755818768232\\
10.45	0.000772756033213305\\
10.46	0.000772756247744016\\
10.47	0.000772756462360401\\
10.48	0.000772756677062496\\
10.49	0.000772756891850334\\
10.5	0.000772757106723955\\
10.51	0.000772757321683391\\
10.52	0.000772757536728678\\
10.53	0.000772757751859853\\
10.54	0.00077275796707695\\
10.55	0.000772758182380007\\
10.56	0.000772758397769057\\
10.57	0.000772758613244136\\
10.58	0.00077275882880528\\
10.59	0.000772759044452523\\
10.6	0.000772759260185902\\
10.61	0.000772759476005452\\
10.62	0.000772759691911207\\
10.63	0.000772759907903204\\
10.64	0.000772760123981477\\
10.65	0.000772760340146061\\
10.66	0.000772760556396994\\
10.67	0.000772760772734308\\
10.68	0.000772760989158039\\
10.69	0.000772761205668222\\
10.7	0.000772761422264892\\
10.71	0.000772761638948084\\
10.72	0.000772761855717834\\
10.73	0.000772762072574176\\
10.74	0.000772762289517144\\
10.75	0.000772762506546775\\
10.76	0.000772762723663101\\
10.77	0.000772762940866159\\
10.78	0.000772763158155984\\
10.79	0.000772763375532609\\
10.8	0.00077276359299607\\
10.81	0.0007727638105464\\
10.82	0.000772764028183635\\
10.83	0.000772764245907808\\
10.84	0.000772764463718956\\
10.85	0.000772764681617112\\
10.86	0.00077276489960231\\
10.87	0.000772765117674585\\
10.88	0.000772765335833972\\
10.89	0.000772765554080504\\
10.9	0.000772765772414214\\
10.91	0.00077276599083514\\
10.92	0.000772766209343313\\
10.93	0.000772766427938767\\
10.94	0.000772766646621537\\
10.95	0.000772766865391657\\
10.96	0.000772767084249162\\
10.97	0.000772767303194083\\
10.98	0.000772767522226457\\
10.99	0.000772767741346315\\
11	0.000772767960553693\\
11.01	0.000772768179848622\\
11.02	0.000772768399231138\\
11.03	0.000772768618701273\\
11.04	0.000772768838259061\\
11.05	0.000772769057904537\\
11.06	0.000772769277637732\\
11.07	0.00077276949745868\\
11.08	0.000772769717367413\\
11.09	0.000772769937363967\\
11.1	0.000772770157448375\\
11.11	0.000772770377620666\\
11.12	0.000772770597880877\\
11.13	0.000772770818229041\\
11.14	0.000772771038665188\\
11.15	0.000772771259189352\\
11.16	0.000772771479801567\\
11.17	0.000772771700501863\\
11.18	0.000772771921290275\\
11.19	0.000772772142166835\\
11.2	0.000772772363131575\\
11.21	0.000772772584184527\\
11.22	0.000772772805325724\\
11.23	0.000772773026555198\\
11.24	0.000772773247872982\\
11.25	0.000772773469279106\\
11.26	0.000772773690773604\\
11.27	0.000772773912356507\\
11.28	0.000772774134027848\\
11.29	0.000772774355787658\\
11.3	0.000772774577635969\\
11.31	0.000772774799572811\\
11.32	0.000772775021598218\\
11.33	0.00077277524371222\\
11.34	0.00077277546591485\\
11.35	0.000772775688206136\\
11.36	0.000772775910586112\\
11.37	0.000772776133054808\\
11.38	0.000772776355612256\\
11.39	0.000772776578258487\\
11.4	0.000772776800993531\\
11.41	0.000772777023817419\\
11.42	0.000772777246730182\\
11.43	0.000772777469731851\\
11.44	0.000772777692822456\\
11.45	0.000772777916002028\\
11.46	0.000772778139270597\\
11.47	0.000772778362628193\\
11.48	0.000772778586074847\\
11.49	0.000772778809610587\\
11.5	0.000772779033235445\\
11.51	0.000772779256949451\\
11.52	0.000772779480752633\\
11.53	0.000772779704645024\\
11.54	0.000772779928626651\\
11.55	0.000772780152697543\\
11.56	0.000772780376857731\\
11.57	0.000772780601107245\\
11.58	0.000772780825446111\\
11.59	0.000772781049874362\\
11.6	0.000772781274392025\\
11.61	0.000772781498999129\\
11.62	0.000772781723695702\\
11.63	0.000772781948481776\\
11.64	0.000772782173357377\\
11.65	0.000772782398322534\\
11.66	0.000772782623377277\\
11.67	0.000772782848521631\\
11.68	0.000772783073755626\\
11.69	0.00077278329907929\\
11.7	0.000772783524492652\\
11.71	0.000772783749995738\\
11.72	0.000772783975588578\\
11.73	0.000772784201271198\\
11.74	0.000772784427043626\\
11.75	0.000772784652905891\\
11.76	0.000772784878858018\\
11.77	0.000772785104900035\\
11.78	0.000772785331031969\\
11.79	0.000772785557253849\\
11.8	0.000772785783565698\\
11.81	0.000772786009967547\\
11.82	0.000772786236459421\\
11.83	0.000772786463041346\\
11.84	0.000772786689713348\\
11.85	0.000772786916475456\\
11.86	0.000772787143327692\\
11.87	0.000772787370270086\\
11.88	0.000772787597302663\\
11.89	0.000772787824425449\\
11.9	0.000772788051638468\\
11.91	0.000772788278941746\\
11.92	0.00077278850633531\\
11.93	0.000772788733819186\\
11.94	0.000772788961393399\\
11.95	0.000772789189057972\\
11.96	0.000772789416812933\\
11.97	0.000772789644658304\\
11.98	0.000772789872594111\\
11.99	0.000772790100620381\\
12	0.000772790328737135\\
12.01	0.000772790556944401\\
12.02	0.000772790785242202\\
12.03	0.00077279101363056\\
12.04	0.000772791242109503\\
12.05	0.000772791470679051\\
12.06	0.000772791699339232\\
12.07	0.000772791928090068\\
12.08	0.000772792156931582\\
12.09	0.000772792385863799\\
12.1	0.00077279261488674\\
12.11	0.000772792844000432\\
12.12	0.000772793073204894\\
12.13	0.000772793302500152\\
12.14	0.00077279353188623\\
12.15	0.000772793761363149\\
12.16	0.000772793990930932\\
12.17	0.000772794220589601\\
12.18	0.000772794450339179\\
12.19	0.000772794680179689\\
12.2	0.000772794910111153\\
12.21	0.000772795140133594\\
12.22	0.000772795370247032\\
12.23	0.000772795600451492\\
12.24	0.000772795830746993\\
12.25	0.000772796061133559\\
12.26	0.000772796291611212\\
12.27	0.00077279652217997\\
12.28	0.000772796752839859\\
12.29	0.000772796983590897\\
12.3	0.000772797214433107\\
12.31	0.00077279744536651\\
12.32	0.000772797676391126\\
12.33	0.000772797907506978\\
12.34	0.000772798138714085\\
12.35	0.00077279837001247\\
12.36	0.000772798601402152\\
12.37	0.000772798832883151\\
12.38	0.000772799064455489\\
12.39	0.000772799296119186\\
12.4	0.000772799527874261\\
12.41	0.000772799759720737\\
12.42	0.000772799991658633\\
12.43	0.000772800223687968\\
12.44	0.000772800455808763\\
12.45	0.000772800688021038\\
12.46	0.000772800920324813\\
12.47	0.000772801152720107\\
12.48	0.000772801385206941\\
12.49	0.000772801617785333\\
12.5	0.000772801850455303\\
12.51	0.000772802083216872\\
12.52	0.000772802316070059\\
12.53	0.000772802549014882\\
12.54	0.000772802782051361\\
12.55	0.000772803015179516\\
12.56	0.000772803248399366\\
12.57	0.000772803481710931\\
12.58	0.000772803715114227\\
12.59	0.000772803948609277\\
12.6	0.000772804182196099\\
12.61	0.00077280441587471\\
12.62	0.000772804649645131\\
12.63	0.000772804883507382\\
12.64	0.000772805117461479\\
12.65	0.000772805351507444\\
12.66	0.000772805585645294\\
12.67	0.000772805819875049\\
12.68	0.000772806054196728\\
12.69	0.000772806288610348\\
12.7	0.00077280652311593\\
12.71	0.000772806757713492\\
12.72	0.000772806992403053\\
12.73	0.000772807227184633\\
12.74	0.00077280746205825\\
12.75	0.000772807697023923\\
12.76	0.000772807932081673\\
12.77	0.000772808167231515\\
12.78	0.000772808402473471\\
12.79	0.00077280863780756\\
12.8	0.000772808873233801\\
12.81	0.000772809108752213\\
12.82	0.000772809344362815\\
12.83	0.000772809580065627\\
12.84	0.000772809815860667\\
12.85	0.000772810051747955\\
12.86	0.000772810287727512\\
12.87	0.000772810523799357\\
12.88	0.000772810759963508\\
12.89	0.000772810996219985\\
12.9	0.000772811232568811\\
12.91	0.000772811469010004\\
12.92	0.000772811705543585\\
12.93	0.000772811942169573\\
12.94	0.000772812178887989\\
12.95	0.000772812415698853\\
12.96	0.000772812652602187\\
12.97	0.000772812889598011\\
12.98	0.000772813126686346\\
12.99	0.000772813363867215\\
13	0.000772813601140636\\
13.01	0.000772813838506633\\
13.02	0.000772814075965227\\
13.03	0.00077281431351644\\
13.04	0.000772814551160293\\
13.05	0.000772814788896809\\
13.06	0.000772815026726012\\
13.07	0.000772815264647921\\
13.08	0.000772815502662562\\
13.09	0.000772815740769957\\
13.1	0.00077281597897013\\
13.11	0.000772816217263104\\
13.12	0.000772816455648905\\
13.13	0.000772816694127555\\
13.14	0.000772816932699079\\
13.15	0.000772817171363502\\
13.16	0.000772817410120847\\
13.17	0.000772817648971143\\
13.18	0.000772817887914412\\
13.19	0.000772818126950684\\
13.2	0.000772818366079982\\
13.21	0.000772818605302334\\
13.22	0.000772818844617768\\
13.23	0.000772819084026309\\
13.24	0.000772819323527988\\
13.25	0.000772819563122831\\
13.26	0.000772819802810866\\
13.27	0.000772820042592123\\
13.28	0.000772820282466632\\
13.29	0.000772820522434421\\
13.3	0.00077282076249552\\
13.31	0.000772821002649962\\
13.32	0.000772821242897776\\
13.33	0.000772821483238996\\
13.34	0.00077282172367365\\
13.35	0.000772821964201774\\
13.36	0.000772822204823399\\
13.37	0.000772822445538559\\
13.38	0.000772822686347287\\
13.39	0.00077282292724962\\
13.4	0.000772823168245592\\
13.41	0.000772823409335236\\
13.42	0.000772823650518591\\
13.43	0.000772823891795692\\
13.44	0.000772824133166577\\
13.45	0.000772824374631283\\
13.46	0.000772824616189849\\
13.47	0.000772824857842315\\
13.48	0.000772825099588717\\
13.49	0.000772825341429098\\
13.5	0.000772825583363499\\
13.51	0.000772825825391961\\
13.52	0.000772826067514524\\
13.53	0.000772826309731233\\
13.54	0.000772826552042132\\
13.55	0.000772826794447262\\
13.56	0.00077282703694667\\
13.57	0.000772827279540401\\
13.58	0.000772827522228502\\
13.59	0.000772827765011018\\
13.6	0.000772828007887999\\
13.61	0.000772828250859491\\
13.62	0.000772828493925544\\
13.63	0.000772828737086208\\
13.64	0.000772828980341533\\
13.65	0.000772829223691571\\
13.66	0.000772829467136374\\
13.67	0.000772829710675995\\
13.68	0.000772829954310486\\
13.69	0.000772830198039903\\
13.7	0.000772830441864301\\
13.71	0.000772830685783736\\
13.72	0.000772830929798265\\
13.73	0.000772831173907945\\
13.74	0.000772831418112836\\
13.75	0.000772831662412996\\
13.76	0.000772831906808486\\
13.77	0.000772832151299366\\
13.78	0.000772832395885699\\
13.79	0.000772832640567546\\
13.8	0.000772832885344974\\
13.81	0.000772833130218043\\
13.82	0.000772833375186823\\
13.83	0.000772833620251379\\
13.84	0.000772833865411776\\
13.85	0.000772834110668082\\
13.86	0.000772834356020369\\
13.87	0.000772834601468705\\
13.88	0.000772834847013159\\
13.89	0.000772835092653806\\
13.9	0.000772835338390716\\
13.91	0.000772835584223962\\
13.92	0.000772835830153621\\
13.93	0.000772836076179765\\
13.94	0.000772836322302472\\
13.95	0.000772836568521817\\
13.96	0.00077283681483788\\
13.97	0.000772837061250739\\
13.98	0.000772837307760474\\
13.99	0.000772837554367163\\
14	0.00077283780107089\\
14.01	0.000772838047871736\\
14.02	0.000772838294769783\\
14.03	0.000772838541765117\\
14.04	0.000772838788857821\\
14.05	0.000772839036047981\\
14.06	0.000772839283335684\\
14.07	0.000772839530721017\\
14.08	0.000772839778204066\\
14.09	0.000772840025784923\\
14.1	0.000772840273463676\\
14.11	0.000772840521240414\\
14.12	0.00077284076911523\\
14.13	0.000772841017088216\\
14.14	0.000772841265159464\\
14.15	0.000772841513329067\\
14.16	0.000772841761597119\\
14.17	0.000772842009963715\\
14.18	0.000772842258428951\\
14.19	0.000772842506992923\\
14.2	0.000772842755655727\\
14.21	0.00077284300441746\\
14.22	0.000772843253278222\\
14.23	0.000772843502238109\\
14.24	0.000772843751297223\\
14.25	0.000772844000455662\\
14.26	0.000772844249713526\\
14.27	0.000772844499070915\\
14.28	0.000772844748527931\\
14.29	0.000772844998084676\\
14.3	0.000772845247741251\\
14.31	0.000772845497497759\\
14.32	0.000772845747354303\\
14.33	0.000772845997310985\\
14.34	0.00077284624736791\\
14.35	0.00077284649752518\\
14.36	0.0007728467477829\\
14.37	0.000772846998141175\\
14.38	0.000772847248600106\\
14.39	0.0007728474991598\\
14.4	0.000772847749820362\\
14.41	0.000772848000581897\\
14.42	0.000772848251444507\\
14.43	0.000772848502408299\\
14.44	0.000772848753473377\\
14.45	0.000772849004639847\\
14.46	0.000772849255907811\\
14.47	0.000772849507277377\\
14.48	0.000772849758748646\\
14.49	0.000772850010321725\\
14.5	0.000772850261996715\\
14.51	0.000772850513773722\\
14.52	0.000772850765652849\\
14.53	0.000772851017634198\\
14.54	0.000772851269717873\\
14.55	0.000772851521903975\\
14.56	0.000772851774192606\\
14.57	0.000772852026583868\\
14.58	0.000772852279077862\\
14.59	0.000772852531674688\\
14.6	0.000772852784374445\\
14.61	0.000772853037177232\\
14.62	0.000772853290083147\\
14.63	0.000772853543092289\\
14.64	0.000772853796204754\\
14.65	0.000772854049420638\\
14.66	0.000772854302740036\\
14.67	0.000772854556163044\\
14.68	0.000772854809689754\\
14.69	0.00077285506332026\\
14.7	0.000772855317054653\\
14.71	0.000772855570893023\\
14.72	0.000772855824835461\\
14.73	0.000772856078882058\\
14.74	0.000772856333032897\\
14.75	0.000772856587288068\\
14.76	0.000772856841647655\\
14.77	0.000772857096111744\\
14.78	0.000772857350680417\\
14.79	0.000772857605353757\\
14.8	0.000772857860131844\\
14.81	0.000772858115014759\\
14.82	0.000772858370002582\\
14.83	0.000772858625095388\\
14.84	0.000772858880293256\\
14.85	0.000772859135596261\\
14.86	0.000772859391004474\\
14.87	0.000772859646517972\\
14.88	0.000772859902136824\\
14.89	0.000772860157861101\\
14.9	0.000772860413690873\\
14.91	0.000772860669626209\\
14.92	0.000772860925667173\\
14.93	0.000772861181813835\\
14.94	0.000772861438066257\\
14.95	0.000772861694424504\\
14.96	0.000772861950888638\\
14.97	0.000772862207458723\\
14.98	0.000772862464134818\\
14.99	0.000772862720916985\\
15	0.000772862977805282\\
15.01	0.000772863234799767\\
15.02	0.000772863491900496\\
15.03	0.000772863749107528\\
15.04	0.000772864006420917\\
15.05	0.00077286426384072\\
15.06	0.000772864521366991\\
15.07	0.000772864778999783\\
15.08	0.000772865036739151\\
15.09	0.000772865294585147\\
15.1	0.000772865552537826\\
15.11	0.000772865810597238\\
15.12	0.000772866068763435\\
15.13	0.00077286632703647\\
15.14	0.000772866585416395\\
15.15	0.00077286684390326\\
15.16	0.000772867102497118\\
15.17	0.000772867361198019\\
15.18	0.000772867620006015\\
15.19	0.000772867878921158\\
15.2	0.000772868137943498\\
15.21	0.000772868397073087\\
15.22	0.000772868656309978\\
15.23	0.00077286891565422\\
15.24	0.000772869175105866\\
15.25	0.000772869434664966\\
15.26	0.000772869694331574\\
15.27	0.000772869954105738\\
15.28	0.000772870213987512\\
15.29	0.000772870473976948\\
15.3	0.000772870734074097\\
15.31	0.00077287099427901\\
15.32	0.000772871254591738\\
15.33	0.000772871515012334\\
15.34	0.00077287177554085\\
15.35	0.000772872036177338\\
15.36	0.000772872296921849\\
15.37	0.000772872557774435\\
15.38	0.000772872818735148\\
15.39	0.00077287307980404\\
15.4	0.000772873340981164\\
15.41	0.000772873602266569\\
15.42	0.000772873863660309\\
15.43	0.000772874125162436\\
15.44	0.000772874386773002\\
15.45	0.000772874648492058\\
15.46	0.000772874910319658\\
15.47	0.000772875172255853\\
15.48	0.000772875434300695\\
15.49	0.000772875696454237\\
15.5	0.000772875958716531\\
15.51	0.00077287622108763\\
15.52	0.000772876483567584\\
15.53	0.000772876746156448\\
15.54	0.000772877008854271\\
15.55	0.000772877271661108\\
15.56	0.000772877534577013\\
15.57	0.000772877797602035\\
15.58	0.000772878060736228\\
15.59	0.000772878323979645\\
15.6	0.000772878587332337\\
15.61	0.000772878850794358\\
15.62	0.000772879114365761\\
15.63	0.000772879378046596\\
15.64	0.000772879641836918\\
15.65	0.000772879905736778\\
15.66	0.000772880169746231\\
15.67	0.000772880433865329\\
15.68	0.000772880698094124\\
15.69	0.000772880962432668\\
15.7	0.000772881226881017\\
15.71	0.00077288149143922\\
15.72	0.000772881756107333\\
15.73	0.000772882020885407\\
15.74	0.000772882285773496\\
15.75	0.000772882550771653\\
15.76	0.00077288281587993\\
15.77	0.000772883081098382\\
15.78	0.000772883346427059\\
15.79	0.000772883611866016\\
15.8	0.000772883877415307\\
15.81	0.000772884143074984\\
15.82	0.0007728844088451\\
15.83	0.000772884674725709\\
15.84	0.000772884940716865\\
15.85	0.00077288520681862\\
15.86	0.000772885473031028\\
15.87	0.000772885739354141\\
15.88	0.000772886005788015\\
15.89	0.0007728862723327\\
15.9	0.000772886538988252\\
15.91	0.000772886805754724\\
15.92	0.000772887072632169\\
15.93	0.000772887339620641\\
15.94	0.000772887606720193\\
15.95	0.000772887873930879\\
15.96	0.000772888141252754\\
15.97	0.000772888408685869\\
15.98	0.000772888676230279\\
15.99	0.00077288894388604\\
16	0.000772889211653202\\
16.01	0.000772889479531821\\
16.02	0.000772889747521949\\
16.03	0.000772890015623643\\
16.04	0.000772890283836954\\
16.05	0.000772890552161936\\
16.06	0.000772890820598645\\
16.07	0.000772891089147134\\
16.08	0.000772891357807455\\
16.09	0.000772891626579664\\
16.1	0.000772891895463817\\
16.11	0.000772892164459965\\
16.12	0.000772892433568163\\
16.13	0.000772892702788465\\
16.14	0.000772892972120926\\
16.15	0.000772893241565601\\
16.16	0.000772893511122542\\
16.17	0.000772893780791805\\
16.18	0.000772894050573443\\
16.19	0.000772894320467512\\
16.2	0.000772894590474063\\
16.21	0.000772894860593155\\
16.22	0.00077289513082484\\
16.23	0.000772895401169172\\
16.24	0.000772895671626207\\
16.25	0.000772895942195999\\
16.26	0.000772896212878602\\
16.27	0.000772896483674072\\
16.28	0.000772896754582461\\
16.29	0.000772897025603826\\
16.3	0.000772897296738222\\
16.31	0.000772897567985703\\
16.32	0.000772897839346323\\
16.33	0.000772898110820137\\
16.34	0.000772898382407201\\
16.35	0.000772898654107568\\
16.36	0.000772898925921294\\
16.37	0.000772899197848434\\
16.38	0.000772899469889043\\
16.39	0.000772899742043176\\
16.4	0.000772900014310888\\
16.41	0.000772900286692234\\
16.42	0.000772900559187269\\
16.43	0.000772900831796049\\
16.44	0.000772901104518628\\
16.45	0.00077290137735506\\
16.46	0.000772901650305403\\
16.47	0.00077290192336971\\
16.48	0.000772902196548037\\
16.49	0.00077290246984044\\
16.5	0.000772902743246973\\
16.51	0.000772903016767693\\
16.52	0.000772903290402655\\
16.53	0.000772903564151913\\
16.54	0.000772903838015523\\
16.55	0.000772904111993543\\
16.56	0.000772904386086026\\
16.57	0.000772904660293026\\
16.58	0.000772904934614602\\
16.59	0.000772905209050809\\
16.6	0.000772905483601701\\
16.61	0.000772905758267335\\
16.62	0.000772906033047766\\
16.63	0.00077290630794305\\
16.64	0.000772906582953243\\
16.65	0.000772906858078402\\
16.66	0.00077290713331858\\
16.67	0.000772907408673834\\
16.68	0.000772907684144222\\
16.69	0.000772907959729797\\
16.7	0.000772908235430617\\
16.71	0.000772908511246737\\
16.72	0.000772908787178212\\
16.73	0.000772909063225101\\
16.74	0.000772909339387458\\
16.75	0.00077290961566534\\
16.76	0.000772909892058802\\
16.77	0.000772910168567903\\
16.78	0.000772910445192695\\
16.79	0.000772910721933237\\
16.8	0.000772910998789586\\
16.81	0.000772911275761796\\
16.82	0.000772911552849925\\
16.83	0.000772911830054028\\
16.84	0.000772912107374164\\
16.85	0.000772912384810387\\
16.86	0.000772912662362756\\
16.87	0.000772912940031324\\
16.88	0.000772913217816151\\
16.89	0.00077291349571729\\
16.9	0.0007729137737348\\
16.91	0.000772914051868738\\
16.92	0.000772914330119159\\
16.93	0.000772914608486122\\
16.94	0.000772914886969681\\
16.95	0.000772915165569895\\
16.96	0.00077291544428682\\
16.97	0.000772915723120512\\
16.98	0.000772916002071029\\
16.99	0.000772916281138429\\
17	0.000772916560322768\\
17.01	0.000772916839624101\\
17.02	0.000772917119042487\\
17.03	0.000772917398577983\\
17.04	0.000772917678230645\\
17.05	0.000772917958000531\\
17.06	0.000772918237887698\\
17.07	0.000772918517892204\\
17.08	0.000772918798014105\\
17.09	0.00077291907825346\\
17.1	0.000772919358610324\\
17.11	0.000772919639084756\\
17.12	0.000772919919676813\\
17.13	0.000772920200386551\\
17.14	0.000772920481214029\\
17.15	0.000772920762159303\\
17.16	0.000772921043222432\\
17.17	0.000772921324403472\\
17.18	0.000772921605702482\\
17.19	0.00077292188711952\\
17.2	0.000772922168654641\\
17.21	0.000772922450307906\\
17.22	0.00077292273207937\\
17.23	0.000772923013969092\\
17.24	0.000772923295977128\\
17.25	0.000772923578103539\\
17.26	0.000772923860348381\\
17.27	0.000772924142711712\\
17.28	0.000772924425193588\\
17.29	0.000772924707794069\\
17.3	0.000772924990513213\\
17.31	0.000772925273351077\\
17.32	0.000772925556307719\\
17.33	0.000772925839383199\\
17.34	0.000772926122577573\\
17.35	0.000772926405890899\\
17.36	0.000772926689323238\\
17.37	0.000772926972874644\\
17.38	0.000772927256545177\\
17.39	0.000772927540334896\\
17.4	0.00077292782424386\\
17.41	0.000772928108272125\\
17.42	0.00077292839241975\\
17.43	0.000772928676686794\\
17.44	0.000772928961073316\\
17.45	0.000772929245579373\\
17.46	0.000772929530205024\\
17.47	0.000772929814950328\\
17.48	0.000772930099815343\\
17.49	0.000772930384800129\\
17.5	0.000772930669904743\\
17.51	0.000772930955129243\\
17.52	0.000772931240473689\\
17.53	0.00077293152593814\\
17.54	0.000772931811522654\\
17.55	0.000772932097227289\\
17.56	0.000772932383052106\\
17.57	0.000772932668997162\\
17.58	0.000772932955062516\\
17.59	0.000772933241248228\\
17.6	0.000772933527554356\\
17.61	0.000772933813980959\\
17.62	0.000772934100528096\\
17.63	0.000772934387195827\\
17.64	0.00077293467398421\\
17.65	0.000772934960893305\\
17.66	0.00077293524792317\\
17.67	0.000772935535073866\\
17.68	0.000772935822345451\\
17.69	0.000772936109737984\\
17.7	0.000772936397251525\\
17.71	0.000772936684886132\\
17.72	0.000772936972641866\\
17.73	0.000772937260518785\\
17.74	0.000772937548516948\\
17.75	0.000772937836636418\\
17.76	0.00077293812487725\\
17.77	0.000772938413239506\\
17.78	0.000772938701723245\\
17.79	0.000772938990328527\\
17.8	0.000772939279055411\\
17.81	0.000772939567903957\\
17.82	0.000772939856874225\\
17.83	0.000772940145966274\\
17.84	0.000772940435180164\\
17.85	0.000772940724515956\\
17.86	0.000772941013973709\\
17.87	0.000772941303553482\\
17.88	0.000772941593255335\\
17.89	0.000772941883079329\\
17.9	0.000772942173025523\\
17.91	0.000772942463093979\\
17.92	0.000772942753284754\\
17.93	0.00077294304359791\\
17.94	0.000772943334033506\\
17.95	0.000772943624591602\\
17.96	0.000772943915272261\\
17.97	0.00077294420607554\\
17.98	0.0007729444970015\\
17.99	0.000772944788050203\\
18	0.000772945079221706\\
18.01	0.000772945370516073\\
18.02	0.000772945661933363\\
18.03	0.000772945953473636\\
18.04	0.000772946245136951\\
18.05	0.00077294653692337\\
18.06	0.000772946828832955\\
18.07	0.000772947120865765\\
18.08	0.000772947413021859\\
18.09	0.000772947705301301\\
18.1	0.00077294799770415\\
18.11	0.000772948290230465\\
18.12	0.000772948582880311\\
18.13	0.000772948875653745\\
18.14	0.000772949168550828\\
18.15	0.000772949461571623\\
18.16	0.000772949754716189\\
18.17	0.000772950047984588\\
18.18	0.00077295034137688\\
18.19	0.000772950634893126\\
18.2	0.000772950928533387\\
18.21	0.000772951222297725\\
18.22	0.000772951516186201\\
18.23	0.000772951810198874\\
18.24	0.000772952104335808\\
18.25	0.000772952398597062\\
18.26	0.000772952692982699\\
18.27	0.000772952987492779\\
18.28	0.000772953282127363\\
18.29	0.000772953576886514\\
18.3	0.000772953871770293\\
18.31	0.00077295416677876\\
18.32	0.000772954461911977\\
18.33	0.000772954757170005\\
18.34	0.000772955052552907\\
18.35	0.000772955348060744\\
18.36	0.000772955643693576\\
18.37	0.000772955939451467\\
18.38	0.000772956235334476\\
18.39	0.000772956531342666\\
18.4	0.0007729568274761\\
18.41	0.000772957123734838\\
18.42	0.000772957420118941\\
18.43	0.000772957716628473\\
18.44	0.000772958013263495\\
18.45	0.000772958310024068\\
18.46	0.000772958606910255\\
18.47	0.000772958903922118\\
18.48	0.000772959201059719\\
18.49	0.000772959498323119\\
18.5	0.000772959795712382\\
18.51	0.000772960093227568\\
18.52	0.00077296039086874\\
18.53	0.000772960688635959\\
18.54	0.000772960986529288\\
18.55	0.00077296128454879\\
18.56	0.000772961582694527\\
18.57	0.000772961880966561\\
18.58	0.000772962179364954\\
18.59	0.000772962477889768\\
18.6	0.000772962776541066\\
18.61	0.000772963075318911\\
18.62	0.000772963374223366\\
18.63	0.000772963673254492\\
18.64	0.000772963972412352\\
18.65	0.000772964271697008\\
18.66	0.000772964571108524\\
18.67	0.000772964870646962\\
18.68	0.000772965170312384\\
18.69	0.000772965470104855\\
18.7	0.000772965770024435\\
18.71	0.000772966070071188\\
18.72	0.000772966370245176\\
18.73	0.000772966670546464\\
18.74	0.000772966970975113\\
18.75	0.000772967271531186\\
18.76	0.000772967572214747\\
18.77	0.000772967873025858\\
18.78	0.000772968173964583\\
18.79	0.000772968475030984\\
18.8	0.000772968776225125\\
18.81	0.000772969077547068\\
18.82	0.000772969378996877\\
18.83	0.000772969680574616\\
18.84	0.000772969982280347\\
18.85	0.000772970284114135\\
18.86	0.000772970586076041\\
18.87	0.00077297088816613\\
18.88	0.000772971190384466\\
18.89	0.00077297149273111\\
18.9	0.000772971795206127\\
18.91	0.00077297209780958\\
18.92	0.000772972400541532\\
18.93	0.000772972703402049\\
18.94	0.000772973006391193\\
18.95	0.000772973309509027\\
18.96	0.000772973612755617\\
18.97	0.000772973916131023\\
18.98	0.000772974219635313\\
18.99	0.000772974523268547\\
19	0.000772974827030792\\
19.01	0.00077297513092211\\
19.02	0.000772975434942565\\
19.03	0.000772975739092222\\
19.04	0.000772976043371143\\
19.05	0.000772976347779394\\
19.06	0.000772976652317039\\
19.07	0.000772976956984141\\
19.08	0.000772977261780765\\
19.09	0.000772977566706974\\
19.1	0.000772977871762833\\
19.11	0.000772978176948406\\
19.12	0.000772978482263759\\
19.13	0.000772978787708954\\
19.14	0.000772979093284056\\
19.15	0.00077297939898913\\
19.16	0.000772979704824239\\
19.17	0.000772980010789449\\
19.18	0.000772980316884823\\
19.19	0.000772980623110427\\
19.2	0.000772980929466326\\
19.21	0.000772981235952583\\
19.22	0.000772981542569263\\
19.23	0.00077298184931643\\
19.24	0.000772982156194152\\
19.25	0.00077298246320249\\
19.26	0.000772982770341511\\
19.27	0.000772983077611278\\
19.28	0.000772983385011857\\
19.29	0.000772983692543314\\
19.3	0.000772984000205711\\
19.31	0.000772984307999115\\
19.32	0.000772984615923591\\
19.33	0.000772984923979203\\
19.34	0.000772985232166018\\
19.35	0.000772985540484099\\
19.36	0.000772985848933512\\
19.37	0.000772986157514323\\
19.38	0.000772986466226595\\
19.39	0.000772986775070397\\
19.4	0.00077298708404579\\
19.41	0.000772987393152843\\
19.42	0.00077298770239162\\
19.43	0.000772988011762185\\
19.44	0.000772988321264606\\
19.45	0.000772988630898946\\
19.46	0.000772988940665274\\
19.47	0.000772989250563651\\
19.48	0.000772989560594146\\
19.49	0.000772989870756823\\
19.5	0.000772990181051748\\
19.51	0.000772990491478987\\
19.52	0.000772990802038607\\
19.53	0.000772991112730671\\
19.54	0.000772991423555247\\
19.55	0.000772991734512399\\
19.56	0.000772992045602196\\
19.57	0.0007729923568247\\
19.58	0.00077299266817998\\
19.59	0.000772992979668102\\
19.6	0.00077299329128913\\
19.61	0.000772993603043131\\
19.62	0.000772993914930171\\
19.63	0.000772994226950317\\
19.64	0.000772994539103634\\
19.65	0.00077299485139019\\
19.66	0.000772995163810049\\
19.67	0.000772995476363279\\
19.68	0.000772995789049945\\
19.69	0.000772996101870116\\
19.7	0.000772996414823856\\
19.71	0.000772996727911231\\
19.72	0.00077299704113231\\
19.73	0.000772997354487159\\
19.74	0.000772997667975844\\
19.75	0.00077299798159843\\
19.76	0.000772998295354986\\
19.77	0.000772998609245576\\
19.78	0.00077299892327027\\
19.79	0.000772999237429134\\
19.8	0.000772999551722234\\
19.81	0.000772999866149636\\
19.82	0.000773000180711409\\
19.83	0.000773000495407617\\
19.84	0.000773000810238331\\
19.85	0.000773001125203614\\
19.86	0.000773001440303536\\
19.87	0.000773001755538162\\
19.88	0.000773002070907561\\
19.89	0.000773002386411798\\
19.9	0.000773002702050942\\
19.91	0.000773003017825059\\
19.92	0.000773003333734217\\
19.93	0.000773003649778484\\
19.94	0.000773003965957926\\
19.95	0.000773004282272612\\
19.96	0.000773004598722609\\
19.97	0.000773004915307983\\
19.98	0.000773005232028803\\
19.99	0.000773005548885135\\
20	0.000773005865877049\\
20.01	0.000773006183004609\\
20.02	0.000773006500267886\\
20.03	0.000773006817666946\\
20.04	0.000773007135201858\\
20.05	0.000773007452872688\\
20.06	0.000773007770679505\\
20.07	0.000773008088622379\\
20.08	0.000773008406701374\\
20.09	0.000773008724916559\\
20.1	0.000773009043268004\\
20.11	0.000773009361755774\\
20.12	0.00077300968037994\\
20.13	0.000773009999140569\\
20.14	0.000773010318037729\\
20.15	0.000773010637071489\\
20.16	0.000773010956241915\\
20.17	0.000773011275549077\\
20.18	0.000773011594993043\\
20.19	0.000773011914573881\\
20.2	0.000773012234291659\\
20.21	0.000773012554146447\\
20.22	0.000773012874138314\\
20.23	0.000773013194267326\\
20.24	0.000773013514533552\\
20.25	0.000773013834937062\\
20.26	0.000773014155477923\\
20.27	0.000773014476156205\\
20.28	0.000773014796971977\\
20.29	0.000773015117925307\\
20.3	0.000773015439016263\\
20.31	0.000773015760244914\\
20.32	0.00077301608161133\\
20.33	0.000773016403115579\\
20.34	0.000773016724757731\\
20.35	0.000773017046537854\\
20.36	0.000773017368456017\\
20.37	0.000773017690512289\\
20.38	0.00077301801270674\\
20.39	0.000773018335039439\\
20.4	0.000773018657510454\\
20.41	0.000773018980119855\\
20.42	0.000773019302867713\\
20.43	0.000773019625754096\\
20.44	0.000773019948779072\\
20.45	0.000773020271942712\\
20.46	0.000773020595245085\\
20.47	0.00077302091868626\\
20.48	0.000773021242266306\\
20.49	0.000773021565985296\\
20.5	0.000773021889843295\\
20.51	0.000773022213840376\\
20.52	0.000773022537976608\\
20.53	0.00077302286225206\\
20.54	0.000773023186666801\\
20.55	0.000773023511220902\\
20.56	0.000773023835914433\\
20.57	0.000773024160747463\\
20.58	0.000773024485720064\\
20.59	0.000773024810832303\\
20.6	0.000773025136084253\\
20.61	0.000773025461475983\\
20.62	0.000773025787007562\\
20.63	0.000773026112679062\\
20.64	0.000773026438490551\\
20.65	0.000773026764442101\\
20.66	0.000773027090533783\\
20.67	0.000773027416765664\\
20.68	0.000773027743137818\\
20.69	0.000773028069650314\\
20.7	0.000773028396303221\\
20.71	0.000773028723096612\\
20.72	0.000773029050030558\\
20.73	0.000773029377105126\\
20.74	0.00077302970432039\\
20.75	0.000773030031676419\\
20.76	0.000773030359173286\\
20.77	0.000773030686811059\\
20.78	0.000773031014589809\\
20.79	0.000773031342509608\\
20.8	0.000773031670570527\\
20.81	0.000773031998772636\\
20.82	0.000773032327116008\\
20.83	0.000773032655600712\\
20.84	0.000773032984226821\\
20.85	0.000773033312994404\\
20.86	0.000773033641903533\\
20.87	0.00077303397095428\\
20.88	0.000773034300146715\\
20.89	0.000773034629480911\\
20.9	0.000773034958956939\\
20.91	0.000773035288574869\\
20.92	0.000773035618334774\\
20.93	0.000773035948236725\\
20.94	0.000773036278280794\\
20.95	0.000773036608467051\\
20.96	0.000773036938795569\\
20.97	0.000773037269266419\\
20.98	0.000773037599879673\\
20.99	0.000773037930635403\\
21	0.00077303826153368\\
21.01	0.000773038592574578\\
21.02	0.000773038923758167\\
21.03	0.000773039255084521\\
21.04	0.00077303958655371\\
21.05	0.000773039918165807\\
21.06	0.000773040249920883\\
21.07	0.000773040581819012\\
21.08	0.000773040913860266\\
21.09	0.000773041246044716\\
21.1	0.000773041578372435\\
21.11	0.000773041910843496\\
21.12	0.000773042243457971\\
21.13	0.000773042576215931\\
21.14	0.00077304290911745\\
21.15	0.0007730432421626\\
21.16	0.000773043575351455\\
21.17	0.000773043908684087\\
21.18	0.00077304424216057\\
21.19	0.000773044575780974\\
21.2	0.000773044909545374\\
21.21	0.00077304524345384\\
21.22	0.000773045577506449\\
21.23	0.000773045911703272\\
21.24	0.000773046246044382\\
21.25	0.000773046580529853\\
21.26	0.000773046915159756\\
21.27	0.000773047249934166\\
21.28	0.000773047584853156\\
21.29	0.0007730479199168\\
21.3	0.000773048255125169\\
21.31	0.000773048590478338\\
21.32	0.000773048925976381\\
21.33	0.00077304926161937\\
21.34	0.00077304959740738\\
21.35	0.000773049933340484\\
21.36	0.000773050269418757\\
21.37	0.000773050605642272\\
21.38	0.0007730509420111\\
21.39	0.000773051278525319\\
21.4	0.000773051615185002\\
21.41	0.000773051951990221\\
21.42	0.000773052288941051\\
21.43	0.000773052626037567\\
21.44	0.000773052963279843\\
21.45	0.000773053300667952\\
21.46	0.000773053638201968\\
21.47	0.000773053975881968\\
21.48	0.000773054313708024\\
21.49	0.00077305465168021\\
21.5	0.000773054989798603\\
21.51	0.000773055328063276\\
21.52	0.000773055666474304\\
21.53	0.000773056005031763\\
21.54	0.000773056343735725\\
21.55	0.000773056682586266\\
21.56	0.000773057021583461\\
21.57	0.000773057360727384\\
21.58	0.000773057700018112\\
21.59	0.000773058039455718\\
21.6	0.000773058379040278\\
21.61	0.000773058718771868\\
21.62	0.000773059058650561\\
21.63	0.000773059398676435\\
21.64	0.000773059738849564\\
21.65	0.000773060079170024\\
21.66	0.000773060419637889\\
21.67	0.000773060760253236\\
21.68	0.00077306110101614\\
21.69	0.000773061441926676\\
21.7	0.000773061782984922\\
21.71	0.000773062124190952\\
21.72	0.000773062465544842\\
21.73	0.000773062807046669\\
21.74	0.000773063148696507\\
21.75	0.000773063490494434\\
21.76	0.000773063832440525\\
21.77	0.000773064174534857\\
21.78	0.000773064516777507\\
21.79	0.000773064859168549\\
21.8	0.000773065201708063\\
21.81	0.000773065544396124\\
21.82	0.000773065887232806\\
21.83	0.000773066230218189\\
21.84	0.000773066573352349\\
21.85	0.000773066916635362\\
21.86	0.000773067260067305\\
21.87	0.000773067603648255\\
21.88	0.000773067947378289\\
21.89	0.000773068291257485\\
21.9	0.000773068635285919\\
21.91	0.00077306897946367\\
21.92	0.000773069323790813\\
21.93	0.000773069668267427\\
21.94	0.00077307001289359\\
21.95	0.000773070357669377\\
21.96	0.000773070702594868\\
21.97	0.000773071047670141\\
21.98	0.000773071392895272\\
21.99	0.000773071738270341\\
22	0.000773072083795423\\
22.01	0.000773072429470599\\
22.02	0.000773072775295947\\
22.03	0.000773073121271544\\
22.04	0.000773073467397469\\
22.05	0.000773073813673799\\
22.06	0.000773074160100614\\
22.07	0.000773074506677992\\
22.08	0.000773074853406011\\
22.09	0.000773075200284752\\
22.1	0.000773075547314293\\
22.11	0.000773075894494711\\
22.12	0.000773076241826086\\
22.13	0.000773076589308499\\
22.14	0.000773076936942027\\
22.15	0.000773077284726751\\
22.16	0.000773077632662749\\
22.17	0.000773077980750101\\
22.18	0.000773078328988888\\
22.19	0.000773078677379187\\
22.2	0.000773079025921081\\
22.21	0.000773079374614648\\
22.22	0.000773079723459967\\
22.23	0.000773080072457119\\
22.24	0.000773080421606185\\
22.25	0.000773080770907245\\
22.26	0.00077308112036038\\
22.27	0.000773081469965669\\
22.28	0.000773081819723194\\
22.29	0.000773082169633035\\
22.3	0.000773082519695273\\
22.31	0.000773082869909987\\
22.32	0.000773083220277263\\
22.33	0.000773083570797178\\
22.34	0.000773083921469814\\
22.35	0.000773084272295254\\
22.36	0.000773084623273577\\
22.37	0.000773084974404866\\
22.38	0.000773085325689203\\
22.39	0.000773085677126669\\
22.4	0.000773086028717347\\
22.41	0.00077308638046132\\
22.42	0.000773086732358668\\
22.43	0.000773087084409475\\
22.44	0.000773087436613822\\
22.45	0.000773087788971793\\
22.46	0.000773088141483469\\
22.47	0.000773088494148933\\
22.48	0.00077308884696827\\
22.49	0.000773089199941561\\
22.5	0.00077308955306889\\
22.51	0.000773089906350341\\
22.52	0.000773090259785999\\
22.53	0.000773090613375944\\
22.54	0.000773090967120263\\
22.55	0.000773091321019038\\
22.56	0.000773091675072352\\
22.57	0.000773092029280293\\
22.58	0.000773092383642942\\
22.59	0.000773092738160385\\
22.6	0.000773093092832706\\
22.61	0.00077309344765999\\
22.62	0.000773093802642321\\
22.63	0.000773094157779786\\
22.64	0.000773094513072468\\
22.65	0.000773094868520455\\
22.66	0.000773095224123832\\
22.67	0.000773095579882683\\
22.68	0.000773095935797096\\
22.69	0.000773096291867155\\
22.7	0.000773096648092947\\
22.71	0.000773097004474558\\
22.72	0.000773097361012077\\
22.73	0.000773097717705586\\
22.74	0.000773098074555176\\
22.75	0.000773098431560933\\
22.76	0.000773098788722943\\
22.77	0.000773099146041295\\
22.78	0.000773099503516075\\
22.79	0.000773099861147373\\
22.8	0.000773100218935275\\
22.81	0.000773100576879869\\
22.82	0.000773100934981244\\
22.83	0.00077310129323949\\
22.84	0.000773101651654694\\
22.85	0.000773102010226944\\
22.86	0.00077310236895633\\
22.87	0.000773102727842943\\
22.88	0.000773103086886872\\
22.89	0.000773103446088205\\
22.9	0.000773103805447033\\
22.91	0.000773104164963446\\
22.92	0.000773104524637533\\
22.93	0.000773104884469387\\
22.94	0.000773105244459097\\
22.95	0.000773105604606755\\
22.96	0.000773105964912452\\
22.97	0.000773106325376279\\
22.98	0.000773106685998327\\
22.99	0.000773107046778689\\
23	0.000773107407717457\\
23.01	0.000773107768814724\\
23.02	0.000773108130070581\\
23.03	0.000773108491485121\\
23.04	0.00077310885305844\\
23.05	0.000773109214790629\\
23.06	0.000773109576681781\\
23.07	0.000773109938731991\\
23.08	0.000773110300941352\\
23.09	0.00077311066330996\\
23.1	0.000773111025837908\\
23.11	0.000773111388525293\\
23.12	0.000773111751372209\\
23.13	0.000773112114378751\\
23.14	0.000773112477545015\\
23.15	0.000773112840871098\\
23.16	0.000773113204357094\\
23.17	0.000773113568003102\\
23.18	0.000773113931809217\\
23.19	0.000773114295775538\\
23.2	0.000773114659902161\\
23.21	0.000773115024189185\\
23.22	0.000773115388636707\\
23.23	0.000773115753244827\\
23.24	0.000773116118013641\\
23.25	0.00077311648294325\\
23.26	0.000773116848033753\\
23.27	0.000773117213285251\\
23.28	0.000773117578697842\\
23.29	0.000773117944271626\\
23.3	0.000773118310006706\\
23.31	0.000773118675903182\\
23.32	0.000773119041961154\\
23.33	0.000773119408180726\\
23.34	0.000773119774561999\\
23.35	0.000773120141105076\\
23.36	0.00077312050781006\\
23.37	0.000773120874677053\\
23.38	0.00077312124170616\\
23.39	0.000773121608897484\\
23.4	0.00077312197625113\\
23.41	0.000773122343767202\\
23.42	0.000773122711445806\\
23.43	0.000773123079287048\\
23.44	0.000773123447291033\\
23.45	0.000773123815457867\\
23.46	0.000773124183787657\\
23.47	0.000773124552280511\\
23.48	0.000773124920936537\\
23.49	0.000773125289755844\\
23.5	0.000773125658738537\\
23.51	0.000773126027884728\\
23.52	0.000773126397194526\\
23.53	0.000773126766668041\\
23.54	0.000773127136305382\\
23.55	0.000773127506106662\\
23.56	0.000773127876071991\\
23.57	0.000773128246201482\\
23.58	0.000773128616495247\\
23.59	0.000773128986953399\\
23.6	0.00077312935757605\\
23.61	0.000773129728363315\\
23.62	0.000773130099315308\\
23.63	0.000773130470432145\\
23.64	0.00077313084171394\\
23.65	0.00077313121316081\\
23.66	0.000773131584772871\\
23.67	0.00077313195655024\\
23.68	0.000773132328493036\\
23.69	0.000773132700601375\\
23.7	0.000773133072875378\\
23.71	0.000773133445315164\\
23.72	0.000773133817920851\\
23.73	0.000773134190692562\\
23.74	0.000773134563630418\\
23.75	0.000773134936734541\\
23.76	0.000773135310005052\\
23.77	0.000773135683442077\\
23.78	0.000773136057045737\\
23.79	0.000773136430816158\\
23.8	0.000773136804753464\\
23.81	0.000773137178857781\\
23.82	0.000773137553129237\\
23.83	0.000773137927567957\\
23.84	0.00077313830217407\\
23.85	0.000773138676947704\\
23.86	0.00077313905188899\\
23.87	0.000773139426998058\\
23.88	0.000773139802275035\\
23.89	0.000773140177720058\\
23.9	0.000773140553333255\\
23.91	0.000773140929114761\\
23.92	0.000773141305064709\\
23.93	0.000773141681183236\\
23.94	0.000773142057470476\\
23.95	0.000773142433926564\\
23.96	0.000773142810551639\\
23.97	0.000773143187345839\\
23.98	0.000773143564309301\\
23.99	0.000773143941442166\\
24	0.000773144318744576\\
24.01	0.000773144696216669\\
24.02	0.000773145073858591\\
24.03	0.000773145451670483\\
24.04	0.00077314582965249\\
24.05	0.000773146207804757\\
24.06	0.00077314658612743\\
24.07	0.000773146964620657\\
24.08	0.000773147343284585\\
24.09	0.000773147722119363\\
24.1	0.000773148101125141\\
24.11	0.000773148480302072\\
24.12	0.000773148859650307\\
24.13	0.000773149239169999\\
24.14	0.000773149618861301\\
24.15	0.00077314999872437\\
24.16	0.000773150378759361\\
24.17	0.000773150758966433\\
24.18	0.000773151139345745\\
24.19	0.000773151519897455\\
24.2	0.000773151900621724\\
24.21	0.000773152281518717\\
24.22	0.000773152662588594\\
24.23	0.000773153043831522\\
24.24	0.000773153425247666\\
24.25	0.000773153806837191\\
24.26	0.000773154188600269\\
24.27	0.000773154570537067\\
24.28	0.000773154952647756\\
24.29	0.000773155334932508\\
24.3	0.000773155717391498\\
24.31	0.000773156100024899\\
24.32	0.000773156482832889\\
24.33	0.000773156865815644\\
24.34	0.000773157248973344\\
24.35	0.000773157632306169\\
24.36	0.000773158015814301\\
24.37	0.000773158399497922\\
24.38	0.000773158783357219\\
24.39	0.000773159167392377\\
24.4	0.000773159551603585\\
24.41	0.000773159935991032\\
24.42	0.000773160320554909\\
24.43	0.000773160705295407\\
24.44	0.000773161090212722\\
24.45	0.00077316147530705\\
24.46	0.000773161860578589\\
24.47	0.000773162246027536\\
24.48	0.000773162631654093\\
24.49	0.000773163017458462\\
24.5	0.000773163403440849\\
24.51	0.000773163789601459\\
24.52	0.000773164175940502\\
24.53	0.000773164562458184\\
24.54	0.00077316494915472\\
24.55	0.000773165336030323\\
24.56	0.000773165723085207\\
24.57	0.000773166110319591\\
24.58	0.000773166497733693\\
24.59	0.000773166885327737\\
24.6	0.000773167273101944\\
24.61	0.000773167661056539\\
24.62	0.000773168049191753\\
24.63	0.000773168437507812\\
24.64	0.000773168826004951\\
24.65	0.000773169214683402\\
24.66	0.000773169603543402\\
24.67	0.000773169992585188\\
24.68	0.000773170381809001\\
24.69	0.000773170771215085\\
24.7	0.000773171160803686\\
24.71	0.00077317155057505\\
24.72	0.000773171940529428\\
24.73	0.000773172330667073\\
24.74	0.000773172720988237\\
24.75	0.000773173111493181\\
24.76	0.000773173502182164\\
24.77	0.000773173893055448\\
24.78	0.000773174284113297\\
24.79	0.000773174675355981\\
24.8	0.000773175066783769\\
24.81	0.000773175458396934\\
24.82	0.000773175850195753\\
24.83	0.000773176242180505\\
24.84	0.000773176634351471\\
24.85	0.000773177026708935\\
24.86	0.000773177419253184\\
24.87	0.000773177811984509\\
24.88	0.000773178204903203\\
24.89	0.000773178598009563\\
24.9	0.000773178991303887\\
24.91	0.000773179384786479\\
24.92	0.000773179778457644\\
24.93	0.000773180172317691\\
24.94	0.000773180566366933\\
24.95	0.000773180960605684\\
24.96	0.000773181355034262\\
24.97	0.00077318174965299\\
24.98	0.000773182144462195\\
24.99	0.000773182539462205\\
25	0.000773182934653352\\
25.01	0.000773183330035973\\
25.02	0.000773183725610407\\
25.03	0.000773184121376998\\
25.04	0.000773184517336094\\
25.05	0.000773184913488045\\
25.06	0.000773185309833207\\
25.07	0.000773185706371938\\
25.08	0.0007731861031046\\
25.09	0.000773186500031562\\
25.1	0.000773186897153192\\
25.11	0.000773187294469867\\
25.12	0.000773187691981965\\
25.13	0.000773188089689869\\
25.14	0.000773188487593968\\
25.15	0.000773188885694652\\
25.16	0.00077318928399232\\
25.17	0.000773189682487371\\
25.18	0.000773190081180212\\
25.19	0.00077319048007125\\
25.2	0.000773190879160902\\
25.21	0.000773191278449588\\
25.22	0.00077319167793773\\
25.23	0.000773192077625759\\
25.24	0.000773192477514109\\
25.25	0.000773192877603218\\
25.26	0.00077319327789353\\
25.27	0.000773193678385494\\
25.28	0.000773194079079565\\
25.29	0.000773194479976204\\
25.3	0.000773194881075873\\
25.31	0.000773195282379045\\
25.32	0.000773195683886195\\
25.33	0.000773196085597805\\
25.34	0.000773196487514363\\
25.35	0.00077319688963636\\
25.36	0.000773197291964295\\
25.37	0.000773197694498674\\
25.38	0.000773198097240007\\
25.39	0.000773198500188811\\
25.4	0.000773198903345609\\
25.41	0.00077319930671093\\
25.42	0.000773199710285308\\
25.43	0.000773200114069287\\
25.44	0.000773200518063415\\
25.45	0.000773200922268246\\
25.46	0.000773201326684342\\
25.47	0.000773201731312271\\
25.48	0.000773202136152609\\
25.49	0.000773202541205937\\
25.5	0.000773202946472845\\
25.51	0.00077320335195393\\
25.52	0.000773203757649795\\
25.53	0.000773204163561051\\
25.54	0.000773204569688317\\
25.55	0.000773204976032218\\
25.56	0.000773205382593388\\
25.57	0.000773205789372469\\
25.58	0.00077320619637011\\
25.59	0.000773206603586967\\
25.6	0.000773207011023708\\
25.61	0.000773207418681005\\
25.62	0.000773207826559539\\
25.63	0.000773208234660001\\
25.64	0.00077320864298309\\
25.65	0.000773209051529514\\
25.66	0.000773209460299988\\
25.67	0.000773209869295237\\
25.68	0.000773210278515997\\
25.69	0.000773210687963009\\
25.7	0.000773211097637028\\
25.71	0.000773211507538814\\
25.72	0.00077321191766914\\
25.73	0.000773212328028785\\
25.74	0.000773212738618542\\
25.75	0.000773213149439211\\
25.76	0.000773213560491603\\
25.77	0.000773213971776538\\
25.78	0.000773214383294848\\
25.79	0.000773214795047376\\
25.8	0.000773215207034973\\
25.81	0.000773215619258503\\
25.82	0.00077321603171884\\
25.83	0.000773216444416868\\
25.84	0.000773216857353485\\
25.85	0.000773217270529598\\
25.86	0.000773217683946127\\
25.87	0.000773218097604002\\
25.88	0.000773218511504166\\
25.89	0.000773218925647573\\
25.9	0.000773219340035191\\
25.91	0.000773219754667999\\
25.92	0.000773220169546987\\
25.93	0.00077322058467316\\
25.94	0.000773221000047536\\
25.95	0.000773221415671144\\
25.96	0.000773221831545028\\
25.97	0.000773222247670244\\
25.98	0.000773222664047862\\
25.99	0.000773223080678965\\
26	0.000773223497564653\\
26.01	0.000773223914706035\\
26.02	0.000773224332104238\\
26.03	0.000773224749760403\\
26.04	0.000773225167675683\\
26.05	0.00077322558585125\\
26.06	0.000773226004288289\\
26.07	0.000773226422987997\\
26.08	0.000773226841951594\\
26.09	0.000773227261180308\\
26.1	0.000773227680675387\\
26.11	0.000773228100438096\\
26.12	0.000773228520469713\\
26.13	0.000773228940771534\\
26.14	0.000773229361344871\\
26.15	0.000773229782191055\\
26.16	0.000773230203311432\\
26.17	0.000773230624707365\\
26.18	0.000773231046380239\\
26.19	0.000773231468331453\\
26.2	0.000773231890562423\\
26.21	0.000773232313074586\\
26.22	0.000773232735869397\\
26.23	0.00077323315894833\\
26.24	0.000773233582312878\\
26.25	0.000773234005964553\\
26.26	0.000773234429904885\\
26.27	0.000773234854135428\\
26.28	0.000773235278657752\\
26.29	0.00077323570347345\\
26.3	0.000773236128584135\\
26.31	0.000773236553991438\\
26.32	0.000773236979697016\\
26.33	0.000773237405702546\\
26.34	0.000773237832009725\\
26.35	0.000773238258620274\\
26.36	0.000773238685535934\\
26.37	0.000773239112758473\\
26.38	0.000773239540289676\\
26.39	0.000773239968131357\\
26.4	0.000773240396285348\\
26.41	0.00077324082475351\\
26.42	0.000773241253537725\\
26.43	0.0007732416826399\\
26.44	0.000773242112061968\\
26.45	0.000773242541805885\\
26.46	0.000773242971873636\\
26.47	0.000773243402267226\\
26.48	0.000773243832988693\\
26.49	0.000773244264040098\\
26.5	0.000773244695423527\\
26.51	0.000773245127141097\\
26.52	0.000773245559194948\\
26.53	0.000773245991587252\\
26.54	0.000773246424320207\\
26.55	0.000773246857396041\\
26.56	0.000773247290817011\\
26.57	0.000773247724585402\\
26.58	0.000773248158703528\\
26.59	0.000773248593173737\\
26.6	0.000773249027998401\\
26.61	0.000773249463179929\\
26.62	0.000773249898720758\\
26.63	0.000773250334623358\\
26.64	0.000773250770890233\\
26.65	0.000773251207523916\\
26.66	0.000773251644526973\\
26.67	0.000773252081902005\\
26.68	0.000773252519651647\\
26.69	0.000773252957778567\\
26.7	0.000773253396285468\\
26.71	0.000773253835175087\\
26.72	0.0007732542744502\\
26.73	0.000773254714113616\\
26.74	0.000773255154168182\\
26.75	0.00077325559461678\\
26.76	0.000773256035462329\\
26.77	0.00077325647670779\\
26.78	0.000773256918356158\\
26.79	0.00077325736041047\\
26.8	0.0007732578028738\\
26.81	0.000773258245749262\\
26.82	0.000773258689040011\\
26.83	0.000773259132749244\\
26.84	0.000773259576880196\\
26.85	0.000773260021436149\\
26.86	0.000773260466420423\\
26.87	0.00077326091183638\\
26.88	0.000773261357687431\\
26.89	0.000773261803977026\\
26.9	0.000773262250708662\\
26.91	0.000773262697885881\\
26.92	0.00077326314551227\\
26.93	0.000773263593591462\\
26.94	0.000773264042127136\\
26.95	0.000773264491123022\\
26.96	0.000773264940582894\\
26.97	0.000773265390510577\\
26.98	0.000773265840909946\\
26.99	0.000773266291784922\\
27	0.000773266743139482\\
27.01	0.00077326719497765\\
27.02	0.0007732676473035\\
27.03	0.000773268100121164\\
27.04	0.000773268553434823\\
27.05	0.000773269007248714\\
27.06	0.000773269461567127\\
27.07	0.000773269916394407\\
27.08	0.000773270371734955\\
27.09	0.000773270827593228\\
27.1	0.000773271283973741\\
27.11	0.000773271740881066\\
27.12	0.000773272198319833\\
27.13	0.000773272656294732\\
27.14	0.000773273114810515\\
27.15	0.00077327357387199\\
27.16	0.000773274033484031\\
27.17	0.00077327449365157\\
27.18	0.000773274954379606\\
27.19	0.000773275415673199\\
27.2	0.000773275877537474\\
27.21	0.000773276339977622\\
27.22	0.0007732768029989\\
27.23	0.00077327726660663\\
27.24	0.000773277730806205\\
27.25	0.000773278195603083\\
27.26	0.000773278661002795\\
27.27	0.000773279127010938\\
27.28	0.000773279593633184\\
27.29	0.000773280060875275\\
27.3	0.000773280528743025\\
27.31	0.000773280997242323\\
27.32	0.000773281466379132\\
27.33	0.000773281936159491\\
27.34	0.000773282406589514\\
27.35	0.000773282877675393\\
27.36	0.000773283349423398\\
27.37	0.000773283821839879\\
27.38	0.000773284294931266\\
27.39	0.000773284768704068\\
27.4	0.000773285243164879\\
27.41	0.000773285718320372\\
27.42	0.000773286194177309\\
27.43	0.000773286670742532\\
27.44	0.000773287148022973\\
27.45	0.000773287626025647\\
27.46	0.000773288104757661\\
27.47	0.00077328858422621\\
27.48	0.000773289064438576\\
27.49	0.000773289545402137\\
27.5	0.000773290027124358\\
27.51	0.000773290509612803\\
27.52	0.000773290992875125\\
27.53	0.000773291476919078\\
27.54	0.000773291961752508\\
27.55	0.000773292447383363\\
27.56	0.000773292933819687\\
27.57	0.000773293421069623\\
27.58	0.00077329390914142\\
27.59	0.000773294398043425\\
27.6	0.000773294887784092\\
27.61	0.000773295378371976\\
27.62	0.000773295869815742\\
27.63	0.000773296362124162\\
27.64	0.000773296855306114\\
27.65	0.000773297349370589\\
27.66	0.000773297844326688\\
27.67	0.000773298340183621\\
27.68	0.000773298836950719\\
27.69	0.000773299334637422\\
27.7	0.000773299833253289\\
27.71	0.000773300332807996\\
27.72	0.000773300833311338\\
27.73	0.000773301334773232\\
27.74	0.000773301837203717\\
27.75	0.000773302340612951\\
27.76	0.000773302845011222\\
27.77	0.000773303350408943\\
27.78	0.000773303856816652\\
27.79	0.000773304364245017\\
27.8	0.00077330487270484\\
27.81	0.000773305382207051\\
27.82	0.000773305892762715\\
27.83	0.000773306404383032\\
27.84	0.000773306917079339\\
27.85	0.000773307430863112\\
27.86	0.000773307945745963\\
27.87	0.00077330846173965\\
27.88	0.000773308978856071\\
27.89	0.00077330949710727\\
27.9	0.000773310016505437\\
27.91	0.000773310537062909\\
27.92	0.000773311058792175\\
27.93	0.000773311581705871\\
27.94	0.000773312105816791\\
27.95	0.000773312631137878\\
27.96	0.000773313157682236\\
27.97	0.000773313685463127\\
27.98	0.000773314214493968\\
27.99	0.000773314744788346\\
28	0.000773315276360004\\
28.01	0.000773315809222854\\
28.02	0.000773316343390975\\
28.03	0.000773316878878614\\
28.04	0.000773317415700189\\
28.05	0.000773317953870293\\
28.06	0.000773318493403691\\
28.07	0.00077331903431533\\
28.08	0.000773319576620328\\
28.09	0.000773320120333989\\
28.1	0.000773320665471798\\
28.11	0.000773321212049427\\
28.12	0.000773321760082732\\
28.13	0.000773322309587758\\
28.14	0.000773322860580744\\
28.15	0.000773323413078118\\
28.16	0.000773323967096506\\
28.17	0.000773324522652732\\
28.18	0.000773325079763817\\
28.19	0.000773325638446985\\
28.2	0.000773326198719664\\
28.21	0.000773326760599487\\
28.22	0.000773327324104298\\
28.23	0.000773327889252148\\
28.24	0.000773328456061304\\
28.25	0.000773329024550247\\
28.26	0.000773329594737677\\
28.27	0.000773330166642511\\
28.28	0.000773330740283892\\
28.29	0.000773331315681186\\
28.3	0.000773331892853987\\
28.31	0.000773332471822116\\
28.32	0.000773333052605629\\
28.33	0.000773333635224818\\
28.34	0.000773334219700208\\
28.35	0.000773334806052567\\
28.36	0.000773335394302905\\
28.37	0.000773335984472474\\
28.38	0.000773336576582776\\
28.39	0.000773337170655563\\
28.4	0.000773337766712841\\
28.41	0.000773338364776869\\
28.42	0.000773338964870165\\
28.43	0.000773339567015509\\
28.44	0.000773340171235943\\
28.45	0.000773340777554778\\
28.46	0.000773341385995592\\
28.47	0.000773341996582235\\
28.48	0.000773342609338833\\
28.49	0.000773343224289789\\
28.5	0.000773343841459788\\
28.51	0.000773344460873798\\
28.52	0.000773345082557074\\
28.53	0.000773345706535158\\
28.54	0.000773346332833888\\
28.55	0.000773346961479396\\
28.56	0.00077334759249811\\
28.57	0.000773348225916763\\
28.58	0.000773348861762392\\
28.59	0.000773349500062342\\
28.6	0.000773350140844266\\
28.61	0.000773350784136134\\
28.62	0.000773351429966231\\
28.63	0.000773352078363165\\
28.64	0.000773352729355864\\
28.65	0.000773353382973584\\
28.66	0.000773354039245912\\
28.67	0.000773354698202769\\
28.68	0.000773355359874406\\
28.69	0.000773356024291421\\
28.7	0.000773356691484751\\
28.71	0.00077335736148568\\
28.72	0.000773358034325841\\
28.73	0.00077335871003722\\
28.74	0.000773359388652159\\
28.75	0.000773360070203358\\
28.76	0.000773360754723883\\
28.77	0.000773361442247163\\
28.78	0.000773362132806996\\
28.79	0.000773362826437555\\
28.8	0.000773363523173389\\
28.81	0.000773364223049425\\
28.82	0.000773364926100971\\
28.83	0.000773365632363724\\
28.84	0.00077336634187377\\
28.85	0.000773367054667587\\
28.86	0.00077336777078205\\
28.87	0.000773368490254431\\
28.88	0.000773369213122406\\
28.89	0.000773369939424057\\
28.9	0.000773370669197875\\
28.91	0.000773371402482763\\
28.92	0.000773372139318038\\
28.93	0.000773372879743437\\
28.94	0.000773373623799121\\
28.95	0.00077337437152567\\
28.96	0.000773375122964097\\
28.97	0.000773375878155843\\
28.98	0.000773376637142784\\
28.99	0.000773377399967232\\
29	0.000773378166671937\\
29.01	0.000773378937300093\\
29.02	0.000773379711895339\\
29.03	0.00077338049050176\\
29.04	0.000773381273163891\\
29.05	0.000773382059926719\\
29.06	0.000773382850835688\\
29.07	0.000773383645936696\\
29.08	0.000773384445276099\\
29.09	0.000773385248900719\\
29.1	0.000773386056857837\\
29.11	0.000773386869195198\\
29.12	0.000773387685961012\\
29.13	0.000773388507203963\\
29.14	0.000773389332973195\\
29.15	0.000773390163318328\\
29.16	0.00077339099828945\\
29.17	0.000773391837937122\\
29.18	0.000773392682312376\\
29.19	0.000773393531466715\\
29.2	0.000773394385452117\\
29.21	0.000773395244321028\\
29.22	0.00077339610812637\\
29.23	0.000773396976921532\\
29.24	0.000773397850760377\\
29.25	0.000773398729697233\\
29.26	0.000773399613786894\\
29.27	0.000773400503084623\\
29.28	0.000773401397646142\\
29.29	0.000773402297527632\\
29.3	0.000773403202785735\\
29.31	0.000773404113477542\\
29.32	0.000773405029660595\\
29.33	0.000773405951392877\\
29.34	0.000773406878732816\\
29.35	0.00077340781173927\\
29.36	0.000773408750471523\\
29.37	0.000773409694989286\\
29.38	0.000773410645352681\\
29.39	0.000773411601622236\\
29.4	0.000773412563858877\\
29.41	0.000773413532123918\\
29.42	0.000773414506479055\\
29.43	0.000773415486986345\\
29.44	0.000773416473708209\\
29.45	0.000773417466707406\\
29.46	0.000773418466047027\\
29.47	0.000773419471790483\\
29.48	0.000773420484001482\\
29.49	0.00077342150274402\\
29.5	0.000773422528082358\\
29.51	0.00077342356008101\\
29.52	0.000773424598804716\\
29.53	0.000773425644318428\\
29.54	0.000773426696687283\\
29.55	0.000773427755976584\\
29.56	0.000773428822251775\\
29.57	0.000773429895578428\\
29.58	0.000773430976022223\\
29.59	0.000773432063648922\\
29.6	0.000773433158524352\\
29.61	0.000773434260714386\\
29.62	0.000773435370284912\\
29.63	0.000773436487301814\\
29.64	0.000773437611830947\\
29.65	0.000773438743938109\\
29.66	0.000773439883689012\\
29.67	0.00077344103114926\\
29.68	0.000773442186384315\\
29.69	0.000773443349459465\\
29.7	0.000773444520439795\\
29.71	0.000773445699390155\\
29.72	0.000773446886375118\\
29.73	0.000773448081458957\\
29.74	0.000773449284705598\\
29.75	0.000773450496178585\\
29.76	0.000773451715941039\\
29.77	0.000773452944055618\\
29.78	0.000773454180584473\\
29.79	0.000773455425589202\\
29.8	0.000773456679130811\\
29.81	0.000773457941269656\\
29.82	0.000773459212065399\\
29.83	0.000773460491576956\\
29.84	0.000773461779862445\\
29.85	0.000773463076979128\\
29.86	0.000773464382983357\\
29.87	0.000773465697930509\\
29.88	0.000773467021874932\\
29.89	0.000773468354869879\\
29.9	0.000773469696967439\\
29.91	0.000773471048218474\\
29.92	0.000773472408672546\\
29.93	0.000773473778377845\\
29.94	0.000773475157381113\\
29.95	0.000773476545727566\\
29.96	0.000773477943460815\\
29.97	0.000773479350622783\\
29.98	0.000773480767253618\\
29.99	0.000773482193391604\\
30	0.000773483629073066\\
30.01	0.000773485074332282\\
30.02	0.000773486529201381\\
30.03	0.000773487993710239\\
30.04	0.000773489467886378\\
30.05	0.000773490951754857\\
30.06	0.000773492445338158\\
30.07	0.00077349394865607\\
30.08	0.000773495461725577\\
30.09	0.000773496984560723\\
30.1	0.000773498517172493\\
30.11	0.00077350005956868\\
30.12	0.000773501611753748\\
30.13	0.000773503173728691\\
30.14	0.000773504745490892\\
30.15	0.00077350632703397\\
30.16	0.000773507918347623\\
30.17	0.000773509519417482\\
30.18	0.000773511130224927\\
30.19	0.000773512750746939\\
30.2	0.000773514380955905\\
30.21	0.000773516020819454\\
30.22	0.000773517670300259\\
30.23	0.000773519329355854\\
30.24	0.000773520997938428\\
30.25	0.000773522675994625\\
30.26	0.000773524363465335\\
30.27	0.000773526060285473\\
30.28	0.000773527766383755\\
30.29	0.000773529481682468\\
30.3	0.000773531206097236\\
30.31	0.000773532939536761\\
30.32	0.000773534681902584\\
30.33	0.000773536433088815\\
30.34	0.000773538192981866\\
30.35	0.000773539961460174\\
30.36	0.000773541738393915\\
30.37	0.000773543523644704\\
30.38	0.000773545317065295\\
30.39	0.000773547118499265\\
30.4	0.000773548927780697\\
30.41	0.000773550744733833\\
30.42	0.000773552569172744\\
30.43	0.000773554400900968\\
30.44	0.000773556239711148\\
30.45	0.000773558085384658\\
30.46	0.000773559937691211\\
30.47	0.000773561796388459\\
30.48	0.000773563661221589\\
30.49	0.00077356553192289\\
30.5	0.000773567408211322\\
30.51	0.000773569289792064\\
30.52	0.00077357117635605\\
30.53	0.000773573067579494\\
30.54	0.000773574963123393\\
30.55	0.000773576862633022\\
30.56	0.000773578765737415\\
30.57	0.000773580672048822\\
30.58	0.000773582581162158\\
30.59	0.000773584492654432\\
30.6	0.000773586406084156\\
30.61	0.000773588320990743\\
30.62	0.000773590236893884\\
30.63	0.000773592153533174\\
30.64	0.000773594070908967\\
30.65	0.000773595989021615\\
30.66	0.000773597907871473\\
30.67	0.000773599827458898\\
30.68	0.000773601747784243\\
30.69	0.000773603668847864\\
30.7	0.000773605590650114\\
30.71	0.00077360751319135\\
30.72	0.000773609436471928\\
30.73	0.000773611360492204\\
30.74	0.000773613285252535\\
30.75	0.000773615210753276\\
30.76	0.000773617136994784\\
30.77	0.000773619063977419\\
30.78	0.000773620991701534\\
30.79	0.000773622920167489\\
30.8	0.000773624849375642\\
30.81	0.000773626779326349\\
30.82	0.000773628710019972\\
30.83	0.000773630641456868\\
30.84	0.000773632573637394\\
30.85	0.000773634506561911\\
30.86	0.000773636440230778\\
30.87	0.000773638374644356\\
30.88	0.000773640309803002\\
30.89	0.000773642245707078\\
30.9	0.000773644182356944\\
30.91	0.000773646119752962\\
30.92	0.000773648057895491\\
30.93	0.000773649996784892\\
30.94	0.000773651936421529\\
30.95	0.000773653876805761\\
30.96	0.000773655817937952\\
30.97	0.000773657759818462\\
30.98	0.000773659702447655\\
30.99	0.000773661645825895\\
31	0.000773663589953542\\
31.01	0.000773665534830962\\
31.02	0.000773667480458517\\
31.03	0.000773669426836572\\
31.04	0.000773671373965491\\
31.05	0.000773673321845636\\
31.06	0.000773675270477373\\
31.07	0.000773677219861067\\
31.08	0.000773679169997083\\
31.09	0.000773681120885788\\
31.1	0.000773683072527545\\
31.11	0.000773685024922722\\
31.12	0.000773686978071684\\
31.13	0.000773688931974797\\
31.14	0.000773690886632429\\
31.15	0.000773692842044947\\
31.16	0.000773694798212718\\
31.17	0.000773696755136109\\
31.18	0.000773698712815487\\
31.19	0.000773700671251222\\
31.2	0.000773702630443681\\
31.21	0.000773704590393233\\
31.22	0.000773706551100246\\
31.23	0.000773708512565089\\
31.24	0.000773710474788133\\
31.25	0.000773712437769747\\
31.26	0.000773714401510301\\
31.27	0.000773716366010164\\
31.28	0.000773718331269707\\
31.29	0.000773720297289301\\
31.3	0.000773722264069318\\
31.31	0.000773724231610127\\
31.32	0.000773726199912101\\
31.33	0.000773728168975613\\
31.34	0.000773730138801032\\
31.35	0.000773732109388732\\
31.36	0.000773734080739085\\
31.37	0.000773736052852466\\
31.38	0.000773738025729246\\
31.39	0.000773739999369799\\
31.4	0.0007737419737745\\
31.41	0.000773743948943721\\
31.42	0.000773745924877837\\
31.43	0.000773747901577223\\
31.44	0.000773749879042251\\
31.45	0.000773751857273299\\
31.46	0.000773753836270742\\
31.47	0.000773755816034956\\
31.48	0.000773757796566315\\
31.49	0.000773759777865196\\
31.5	0.000773761759931976\\
31.51	0.000773763742767032\\
31.52	0.000773765726370741\\
31.53	0.000773767710743478\\
31.54	0.000773769695885624\\
31.55	0.000773771681797554\\
31.56	0.000773773668479647\\
31.57	0.000773775655932282\\
31.58	0.000773777644155838\\
31.59	0.000773779633150692\\
31.6	0.000773781622917226\\
31.61	0.000773783613455818\\
31.62	0.000773785604766847\\
31.63	0.000773787596850694\\
31.64	0.000773789589707738\\
31.65	0.000773791583338361\\
31.66	0.000773793577742944\\
31.67	0.000773795572921867\\
31.68	0.000773797568875512\\
31.69	0.000773799565604262\\
31.7	0.000773801563108497\\
31.71	0.000773803561388601\\
31.72	0.000773805560444955\\
31.73	0.000773807560277943\\
31.74	0.000773809560887948\\
31.75	0.000773811562275352\\
31.76	0.000773813564440541\\
31.77	0.000773815567383897\\
31.78	0.000773817571105806\\
31.79	0.000773819575606653\\
31.8	0.00077382158088682\\
31.81	0.000773823586946694\\
31.82	0.000773825593786661\\
31.83	0.000773827601407106\\
31.84	0.000773829609808414\\
31.85	0.000773831618990972\\
31.86	0.000773833628955167\\
31.87	0.000773835639701386\\
31.88	0.000773837651230016\\
31.89	0.000773839663541444\\
31.9	0.000773841676636059\\
31.91	0.000773843690514246\\
31.92	0.000773845705176397\\
31.93	0.000773847720622898\\
31.94	0.000773849736854139\\
31.95	0.000773851753870508\\
31.96	0.000773853771672396\\
31.97	0.000773855790260193\\
31.98	0.000773857809634288\\
31.99	0.000773859829795071\\
32	0.000773861850742933\\
32.01	0.000773863872478265\\
32.02	0.000773865895001457\\
32.03	0.000773867918312902\\
32.04	0.000773869942412993\\
32.05	0.000773871967302119\\
32.06	0.000773873992980675\\
32.07	0.000773876019449051\\
32.08	0.000773878046707643\\
32.09	0.000773880074756842\\
32.1	0.000773882103597044\\
32.11	0.000773884133228641\\
32.12	0.000773886163652026\\
32.13	0.000773888194867595\\
32.14	0.000773890226875743\\
32.15	0.000773892259676865\\
32.16	0.000773894293271356\\
32.17	0.000773896327659611\\
32.18	0.000773898362842026\\
32.19	0.000773900398818998\\
32.2	0.000773902435590923\\
32.21	0.000773904473158199\\
32.22	0.000773906511521223\\
32.23	0.00077390855068039\\
32.24	0.0007739105906361\\
32.25	0.00077391263138875\\
32.26	0.000773914672938738\\
32.27	0.000773916715286463\\
32.28	0.000773918758432326\\
32.29	0.000773920802376723\\
32.3	0.000773922847120054\\
32.31	0.000773924892662722\\
32.32	0.000773926939005123\\
32.33	0.00077392898614766\\
32.34	0.000773931034090732\\
32.35	0.000773933082834742\\
32.36	0.000773935132380089\\
32.37	0.000773937182727178\\
32.38	0.000773939233876407\\
32.39	0.000773941285828181\\
32.4	0.000773943338582902\\
32.41	0.000773945392140972\\
32.42	0.000773947446502795\\
32.43	0.000773949501668775\\
32.44	0.000773951557639315\\
32.45	0.000773953614414818\\
32.46	0.00077395567199569\\
32.47	0.000773957730382335\\
32.48	0.000773959789575158\\
32.49	0.000773961849574565\\
32.5	0.000773963910380961\\
32.51	0.000773965971994752\\
32.52	0.000773968034416343\\
32.53	0.000773970097646142\\
32.54	0.000773972161684556\\
32.55	0.000773974226531991\\
32.56	0.000773976292188857\\
32.57	0.00077397835865556\\
32.58	0.000773980425932507\\
32.59	0.000773982494020107\\
32.6	0.000773984562918771\\
32.61	0.000773986632628905\\
32.62	0.000773988703150921\\
32.63	0.000773990774485226\\
32.64	0.000773992846632232\\
32.65	0.000773994919592349\\
32.66	0.000773996993365986\\
32.67	0.000773999067953555\\
32.68	0.000774001143355468\\
32.69	0.000774003219572135\\
32.7	0.00077400529660397\\
32.71	0.000774007374451383\\
32.72	0.000774009453114788\\
32.73	0.000774011532594597\\
32.74	0.000774013612891223\\
32.75	0.00077401569400508\\
32.76	0.000774017775936582\\
32.77	0.000774019858686144\\
32.78	0.000774021942254177\\
32.79	0.000774024026641099\\
32.8	0.000774026111847324\\
32.81	0.000774028197873266\\
32.82	0.000774030284719342\\
32.83	0.000774032372385968\\
32.84	0.000774034460873561\\
32.85	0.000774036550182535\\
32.86	0.000774038640313311\\
32.87	0.000774040731266303\\
32.88	0.000774042823041929\\
32.89	0.000774044915640608\\
32.9	0.000774047009062757\\
32.91	0.000774049103308796\\
32.92	0.000774051198379143\\
32.93	0.000774053294274216\\
32.94	0.000774055390994437\\
32.95	0.000774057488540226\\
32.96	0.000774059586912\\
32.97	0.000774061686110184\\
32.98	0.000774063786135195\\
32.99	0.000774065886987457\\
33	0.000774067988667389\\
33.01	0.000774070091175414\\
33.02	0.000774072194511954\\
33.03	0.000774074298677431\\
33.04	0.00077407640367227\\
33.05	0.000774078509496892\\
33.06	0.00077408061615172\\
33.07	0.000774082723637179\\
33.08	0.000774084831953693\\
33.09	0.000774086941101686\\
33.1	0.000774089051081584\\
33.11	0.00077409116189381\\
33.12	0.000774093273538792\\
33.13	0.000774095386016953\\
33.14	0.000774097499328722\\
33.15	0.000774099613474523\\
33.16	0.000774101728454784\\
33.17	0.000774103844269932\\
33.18	0.000774105960920394\\
33.19	0.000774108078406598\\
33.2	0.000774110196728973\\
33.21	0.000774112315887946\\
33.22	0.000774114435883946\\
33.23	0.000774116556717403\\
33.24	0.000774118678388747\\
33.25	0.000774120800898406\\
33.26	0.000774122924246811\\
33.27	0.000774125048434392\\
33.28	0.000774127173461581\\
33.29	0.000774129299328808\\
33.3	0.000774131426036506\\
33.31	0.000774133553585103\\
33.32	0.000774135681975034\\
33.33	0.000774137811206731\\
33.34	0.000774139941280628\\
33.35	0.000774142072197156\\
33.36	0.000774144203956752\\
33.37	0.000774146336559846\\
33.38	0.000774148470006874\\
33.39	0.000774150604298271\\
33.4	0.00077415273943447\\
33.41	0.000774154875415909\\
33.42	0.00077415701224302\\
33.43	0.000774159149916241\\
33.44	0.000774161288436008\\
33.45	0.000774163427802758\\
33.46	0.000774165568016929\\
33.47	0.000774167709078956\\
33.48	0.000774169850989277\\
33.49	0.000774171993748331\\
33.5	0.000774174137356556\\
33.51	0.000774176281814389\\
33.52	0.000774178427122272\\
33.53	0.000774180573280643\\
33.54	0.000774182720289941\\
33.55	0.000774184868150606\\
33.56	0.000774187016863079\\
33.57	0.000774189166427802\\
33.58	0.000774191316845214\\
33.59	0.000774193468115757\\
33.6	0.000774195620239875\\
33.61	0.000774197773218008\\
33.62	0.000774199927050597\\
33.63	0.000774202081738088\\
33.64	0.000774204237280923\\
33.65	0.000774206393679545\\
33.66	0.000774208550934399\\
33.67	0.000774210709045929\\
33.68	0.000774212868014578\\
33.69	0.000774215027840794\\
33.7	0.000774217188525019\\
33.71	0.0007742193500677\\
33.72	0.000774221512469284\\
33.73	0.000774223675730218\\
33.74	0.000774225839850946\\
33.75	0.000774228004831916\\
33.76	0.000774230170673577\\
33.77	0.000774232337376375\\
33.78	0.00077423450494076\\
33.79	0.000774236673367179\\
33.8	0.000774238842656082\\
33.81	0.000774241012807918\\
33.82	0.000774243183823136\\
33.83	0.000774245355702186\\
33.84	0.000774247528445519\\
33.85	0.000774249702053585\\
33.86	0.000774251876526836\\
33.87	0.000774254051865722\\
33.88	0.000774256228070695\\
33.89	0.00077425840514221\\
33.9	0.000774260583080716\\
33.91	0.000774262761886668\\
33.92	0.000774264941560517\\
33.93	0.000774267122102719\\
33.94	0.000774269303513726\\
33.95	0.000774271485793994\\
33.96	0.000774273668943976\\
33.97	0.000774275852964128\\
33.98	0.000774278037854906\\
33.99	0.000774280223616766\\
34	0.000774282410250163\\
34.01	0.000774284597755553\\
34.02	0.000774286786133395\\
34.03	0.000774288975384145\\
34.04	0.00077429116550826\\
34.05	0.000774293356506198\\
34.06	0.00077429554837842\\
34.07	0.000774297741125383\\
34.08	0.000774299934747546\\
34.09	0.000774302129245367\\
34.1	0.000774304324619309\\
34.11	0.000774306520869831\\
34.12	0.000774308717997392\\
34.13	0.000774310916002455\\
34.14	0.00077431311488548\\
34.15	0.000774315314646929\\
34.16	0.000774317515287264\\
34.17	0.00077431971680695\\
34.18	0.000774321919206445\\
34.19	0.000774324122486215\\
34.2	0.000774326326646726\\
34.21	0.000774328531688438\\
34.22	0.000774330737611817\\
34.23	0.000774332944417328\\
34.24	0.000774335152105435\\
34.25	0.000774337360676605\\
34.26	0.000774339570131302\\
34.27	0.000774341780469994\\
34.28	0.000774343991693146\\
34.29	0.000774346203801228\\
34.3	0.000774348416794704\\
34.31	0.000774350630674044\\
34.32	0.000774352845439715\\
34.33	0.000774355061092186\\
34.34	0.000774357277631926\\
34.35	0.000774359495059404\\
34.36	0.000774361713375089\\
34.37	0.000774363932579453\\
34.38	0.000774366152672965\\
34.39	0.000774368373656095\\
34.4	0.000774370595529315\\
34.41	0.000774372818293097\\
34.42	0.000774375041947913\\
34.43	0.000774377266494236\\
34.44	0.000774379491932538\\
34.45	0.000774381718263292\\
34.46	0.000774383945486972\\
34.47	0.000774386173604051\\
34.48	0.000774388402615005\\
34.49	0.000774390632520307\\
34.5	0.000774392863320433\\
34.51	0.000774395095015859\\
34.52	0.000774397327607059\\
34.53	0.000774399561094511\\
34.54	0.000774401795478693\\
34.55	0.000774404030760078\\
34.56	0.000774406266939146\\
34.57	0.000774408504016375\\
34.58	0.000774410741992242\\
34.59	0.000774412980867226\\
34.6	0.000774415220641807\\
34.61	0.000774417461316465\\
34.62	0.000774419702891679\\
34.63	0.000774421945367927\\
34.64	0.000774424188745692\\
34.65	0.000774426433025456\\
34.66	0.000774428678207697\\
34.67	0.0007744309242929\\
34.68	0.000774433171281546\\
34.69	0.000774435419174118\\
34.7	0.000774437667971098\\
34.71	0.000774439917672969\\
34.72	0.000774442168280216\\
34.73	0.000774444419793324\\
34.74	0.000774446672212777\\
34.75	0.000774448925539059\\
34.76	0.000774451179772656\\
34.77	0.000774453434914054\\
34.78	0.00077445569096374\\
34.79	0.000774457947922198\\
34.8	0.000774460205789917\\
34.81	0.000774462464567385\\
34.82	0.000774464724255087\\
34.83	0.000774466984853514\\
34.84	0.000774469246363155\\
34.85	0.000774471508784498\\
34.86	0.000774473772118032\\
34.87	0.000774476036364248\\
34.88	0.000774478301523635\\
34.89	0.000774480567596684\\
34.9	0.000774482834583886\\
34.91	0.000774485102485734\\
34.92	0.000774487371302717\\
34.93	0.00077448964103533\\
34.94	0.000774491911684065\\
34.95	0.000774494183249416\\
34.96	0.000774496455731875\\
34.97	0.000774498729131937\\
34.98	0.000774501003450095\\
34.99	0.000774503278686845\\
35	0.000774505554842682\\
35.01	0.000774507831918101\\
35.02	0.0007745101099136\\
35.03	0.000774512388829673\\
35.04	0.000774514668666819\\
35.05	0.000774516949425534\\
35.06	0.000774519231106316\\
35.07	0.000774521513709664\\
35.08	0.000774523797236075\\
35.09	0.00077452608168605\\
35.1	0.000774528367060085\\
35.11	0.000774530653358684\\
35.12	0.000774532940582344\\
35.13	0.000774535228731566\\
35.14	0.000774537517806853\\
35.15	0.000774539807808707\\
35.16	0.000774542098737627\\
35.17	0.000774544390594116\\
35.18	0.000774546683378678\\
35.19	0.000774548977091815\\
35.2	0.000774551271734033\\
35.21	0.000774553567305833\\
35.22	0.000774555863807721\\
35.23	0.000774558161240202\\
35.24	0.000774560459603781\\
35.25	0.000774562758898964\\
35.26	0.000774565059126257\\
35.27	0.000774567360286166\\
35.28	0.000774569662379199\\
35.29	0.000774571965405863\\
35.3	0.000774574269366666\\
35.31	0.000774576574262116\\
35.32	0.000774578880092722\\
35.33	0.000774581186858993\\
35.34	0.000774583494561439\\
35.35	0.00077458580320057\\
35.36	0.000774588112776897\\
35.37	0.00077459042329093\\
35.38	0.000774592734743181\\
35.39	0.000774595047134161\\
35.4	0.000774597360464382\\
35.41	0.000774599674734359\\
35.42	0.000774601989944603\\
35.43	0.000774604306095628\\
35.44	0.000774606623187948\\
35.45	0.000774608941222077\\
35.46	0.000774611260198531\\
35.47	0.000774613580117824\\
35.48	0.000774615900980472\\
35.49	0.000774618222786991\\
35.5	0.0007746205455379\\
35.51	0.000774622869233712\\
35.52	0.000774625193874947\\
35.53	0.000774627519462123\\
35.54	0.000774629845995757\\
35.55	0.000774632173476369\\
35.56	0.000774634501904478\\
35.57	0.000774636831280604\\
35.58	0.000774639161605266\\
35.59	0.000774641492878985\\
35.6	0.000774643825102284\\
35.61	0.000774646158275681\\
35.62	0.000774648492399702\\
35.63	0.000774650827474866\\
35.64	0.000774653163501698\\
35.65	0.00077465550048072\\
35.66	0.000774657838412456\\
35.67	0.00077466017729743\\
35.68	0.000774662517136168\\
35.69	0.000774664857929193\\
35.7	0.000774667199677032\\
35.71	0.000774669542380209\\
35.72	0.000774671886039253\\
35.73	0.000774674230654691\\
35.74	0.000774676576227047\\
35.75	0.000774678922756853\\
35.76	0.000774681270244633\\
35.77	0.00077468361869092\\
35.78	0.000774685968096239\\
35.79	0.000774688318461123\\
35.8	0.0007746906697861\\
35.81	0.000774693022071702\\
35.82	0.000774695375318459\\
35.83	0.000774697729526904\\
35.84	0.000774700084697566\\
35.85	0.00077470244083098\\
35.86	0.000774704797927678\\
35.87	0.000774707155988192\\
35.88	0.000774709515013058\\
35.89	0.000774711875002809\\
35.9	0.00077471423595798\\
35.91	0.000774716597879106\\
35.92	0.000774718960766722\\
35.93	0.000774721324621365\\
35.94	0.000774723689443571\\
35.95	0.000774726055233878\\
35.96	0.000774728421992821\\
35.97	0.000774730789720941\\
35.98	0.000774733158418774\\
35.99	0.000774735528086861\\
36	0.000774737898725739\\
36.01	0.000774740270335949\\
36.02	0.000774742642918031\\
36.03	0.000774745016472527\\
36.04	0.000774747390999976\\
36.05	0.000774749766500922\\
36.06	0.000774752142975906\\
36.07	0.000774754520425469\\
36.08	0.000774756898850157\\
36.09	0.000774759278250513\\
36.1	0.000774761658627081\\
36.11	0.000774764039980403\\
36.12	0.000774766422311028\\
36.13	0.000774768805619499\\
36.14	0.000774771189906362\\
36.15	0.000774773575172163\\
36.16	0.000774775961417449\\
36.17	0.000774778348642769\\
36.18	0.00077478073684867\\
36.19	0.0007747831260357\\
36.2	0.000774785516204409\\
36.21	0.000774787907355344\\
36.22	0.000774790299489058\\
36.23	0.000774792692606099\\
36.24	0.000774795086707018\\
36.25	0.000774797481792367\\
36.26	0.000774799877862697\\
36.27	0.00077480227491856\\
36.28	0.000774804672960509\\
36.29	0.000774807071989098\\
36.3	0.000774809472004879\\
36.31	0.000774811873008407\\
36.32	0.000774814275000237\\
36.33	0.000774816677980922\\
36.34	0.000774819081951022\\
36.35	0.000774821486911088\\
36.36	0.00077482389286168\\
36.37	0.000774826299803353\\
36.38	0.000774828707736666\\
36.39	0.000774831116662177\\
36.4	0.000774833526580443\\
36.41	0.000774835937492026\\
36.42	0.000774838349397485\\
36.43	0.000774840762297377\\
36.44	0.000774843176192265\\
36.45	0.000774845591082709\\
36.46	0.000774848006969271\\
36.47	0.000774850423852513\\
36.48	0.000774852841732997\\
36.49	0.000774855260611286\\
36.5	0.000774857680487944\\
36.51	0.000774860101363534\\
36.52	0.000774862523238622\\
36.53	0.000774864946113773\\
36.54	0.00077486736998955\\
36.55	0.000774869794866521\\
36.56	0.000774872220745253\\
36.57	0.000774874647626312\\
36.58	0.000774877075510265\\
36.59	0.000774879504397681\\
36.6	0.000774881934289129\\
36.61	0.000774884365185175\\
36.62	0.000774886797086392\\
36.63	0.000774889229993348\\
36.64	0.000774891663906615\\
36.65	0.000774894098826763\\
36.66	0.000774896534754363\\
36.67	0.000774898971689988\\
36.68	0.00077490140963421\\
36.69	0.000774903848587602\\
36.7	0.000774906288550738\\
36.71	0.000774908729524192\\
36.72	0.00077491117150854\\
36.73	0.000774913614504354\\
36.74	0.000774916058512211\\
36.75	0.000774918503532687\\
36.76	0.00077492094956636\\
36.77	0.000774923396613805\\
36.78	0.000774925844675601\\
36.79	0.000774928293752326\\
36.8	0.000774930743844559\\
36.81	0.000774933194952879\\
36.82	0.000774935647077864\\
36.83	0.000774938100220097\\
36.84	0.000774940554380156\\
36.85	0.000774943009558625\\
36.86	0.000774945465756084\\
36.87	0.000774947922973116\\
36.88	0.000774950381210303\\
36.89	0.000774952840468229\\
36.9	0.000774955300747478\\
36.91	0.000774957762048634\\
36.92	0.000774960224372283\\
36.93	0.000774962687719009\\
36.94	0.000774965152089399\\
36.95	0.000774967617484039\\
36.96	0.000774970083903516\\
36.97	0.000774972551348417\\
36.98	0.000774975019819331\\
36.99	0.000774977489316845\\
37	0.00077497995984155\\
37.01	0.000774982431394036\\
37.02	0.000774984903974891\\
37.03	0.000774987377584708\\
37.04	0.000774989852224076\\
37.05	0.000774992327893589\\
37.06	0.000774994804593837\\
37.07	0.000774997282325414\\
37.08	0.000774999761088914\\
37.09	0.00077500224088493\\
37.1	0.000775004721714058\\
37.11	0.000775007203576889\\
37.12	0.000775009686474022\\
37.13	0.000775012170406052\\
37.14	0.000775014655373576\\
37.15	0.000775017141377191\\
37.16	0.000775019628417493\\
37.17	0.000775022116495083\\
37.18	0.000775024605610557\\
37.19	0.000775027095764515\\
37.2	0.000775029586957559\\
37.21	0.000775032079190286\\
37.22	0.0007750345724633\\
37.23	0.000775037066777199\\
37.24	0.000775039562132587\\
37.25	0.000775042058530066\\
37.26	0.000775044555970239\\
37.27	0.000775047054453712\\
37.28	0.000775049553981085\\
37.29	0.000775052054552965\\
37.3	0.000775054556169956\\
37.31	0.000775057058832665\\
37.32	0.000775059562541699\\
37.33	0.000775062067297664\\
37.34	0.000775064573101166\\
37.35	0.000775067079952814\\
37.36	0.000775069587853217\\
37.37	0.000775072096802984\\
37.38	0.000775074606802723\\
37.39	0.000775077117853046\\
37.4	0.000775079629954564\\
37.41	0.000775082143107887\\
37.42	0.000775084657313626\\
37.43	0.000775087172572395\\
37.44	0.000775089688884806\\
37.45	0.000775092206251474\\
37.46	0.000775094724673011\\
37.47	0.000775097244150032\\
37.48	0.000775099764683153\\
37.49	0.000775102286272989\\
37.5	0.000775104808920157\\
37.51	0.000775107332625273\\
37.52	0.000775109857388954\\
37.53	0.00077511238321182\\
37.54	0.000775114910094486\\
37.55	0.000775117438037574\\
37.56	0.000775119967041703\\
37.57	0.000775122497107492\\
37.58	0.000775125028235563\\
37.59	0.000775127560426537\\
37.6	0.000775130093681036\\
37.61	0.000775132627999681\\
37.62	0.000775135163383097\\
37.63	0.000775137699831906\\
37.64	0.000775140237346733\\
37.65	0.000775142775928203\\
37.66	0.000775145315576939\\
37.67	0.000775147856293569\\
37.68	0.00077515039807872\\
37.69	0.000775152940933017\\
37.7	0.000775155484857088\\
37.71	0.000775158029851562\\
37.72	0.000775160575917066\\
37.73	0.000775163123054231\\
37.74	0.000775165671263685\\
37.75	0.000775168220546059\\
37.76	0.000775170770901983\\
37.77	0.000775173322332092\\
37.78	0.000775175874837015\\
37.79	0.000775178428417385\\
37.8	0.000775180983073836\\
37.81	0.000775183538807001\\
37.82	0.000775186095617516\\
37.83	0.000775188653506014\\
37.84	0.000775191212473133\\
37.85	0.000775193772519505\\
37.86	0.000775196333645771\\
37.87	0.000775198895852567\\
37.88	0.000775201459140529\\
37.89	0.000775204023510298\\
37.9	0.000775206588962511\\
37.91	0.000775209155497808\\
37.92	0.00077521172311683\\
37.93	0.000775214291820217\\
37.94	0.000775216861608611\\
37.95	0.000775219432482654\\
37.96	0.000775222004442988\\
37.97	0.000775224577490257\\
37.98	0.000775227151625103\\
37.99	0.000775229726848171\\
38	0.000775232303160105\\
38.01	0.000775234880561552\\
38.02	0.000775237459053158\\
38.03	0.000775240038635569\\
38.04	0.000775242619309431\\
38.05	0.000775245201075393\\
38.06	0.000775247783934105\\
38.07	0.000775250367886213\\
38.08	0.000775252952932369\\
38.09	0.000775255539073223\\
38.1	0.000775258126309424\\
38.11	0.000775260714641623\\
38.12	0.000775263304070475\\
38.13	0.00077526589459663\\
38.14	0.000775268486220742\\
38.15	0.000775271078943464\\
38.16	0.000775273672765451\\
38.17	0.000775276267687357\\
38.18	0.000775278863709839\\
38.19	0.000775281460833553\\
38.2	0.000775284059059154\\
38.21	0.000775286658387301\\
38.22	0.000775289258818651\\
38.23	0.000775291860353862\\
38.24	0.000775294462993594\\
38.25	0.000775297066738507\\
38.26	0.00077529967158926\\
38.27	0.000775302277546516\\
38.28	0.000775304884610934\\
38.29	0.000775307492783178\\
38.3	0.000775310102063911\\
38.31	0.000775312712453794\\
38.32	0.000775315323953493\\
38.33	0.000775317936563671\\
38.34	0.000775320550284996\\
38.35	0.000775323165118131\\
38.36	0.000775325781063744\\
38.37	0.000775328398122499\\
38.38	0.000775331016295067\\
38.39	0.000775333635582114\\
38.4	0.00077533625598431\\
38.41	0.000775338877502325\\
38.42	0.000775341500136827\\
38.43	0.000775344123888487\\
38.44	0.000775346748757977\\
38.45	0.00077534937474597\\
38.46	0.000775352001853137\\
38.47	0.000775354630080152\\
38.48	0.000775357259427687\\
38.49	0.000775359889896417\\
38.5	0.000775362521487017\\
38.51	0.000775365154200161\\
38.52	0.000775367788036529\\
38.53	0.000775370422996795\\
38.54	0.000775373059081636\\
38.55	0.000775375696291733\\
38.56	0.00077537833462776\\
38.57	0.000775380974090399\\
38.58	0.00077538361468033\\
38.59	0.000775386256398234\\
38.6	0.00077538889924479\\
38.61	0.000775391543220681\\
38.62	0.000775394188326589\\
38.63	0.000775396834563197\\
38.64	0.000775399481931189\\
38.65	0.000775402130431249\\
38.66	0.000775404780064062\\
38.67	0.000775407430830313\\
38.68	0.000775410082730688\\
38.69	0.000775412735765873\\
38.7	0.000775415389936559\\
38.71	0.000775418045243429\\
38.72	0.000775420701687174\\
38.73	0.000775423359268484\\
38.74	0.000775426017988048\\
38.75	0.000775428677846557\\
38.76	0.000775431338844701\\
38.77	0.000775434000983172\\
38.78	0.000775436664262663\\
38.79	0.000775439328683866\\
38.8	0.000775441994247476\\
38.81	0.000775444660954186\\
38.82	0.000775447328804692\\
38.83	0.000775449997799689\\
38.84	0.000775452667939873\\
38.85	0.00077545533922594\\
38.86	0.00077545801165859\\
38.87	0.000775460685238519\\
38.88	0.000775463359966427\\
38.89	0.000775466035843012\\
38.9	0.000775468712868974\\
38.91	0.000775471391045014\\
38.92	0.000775474070371835\\
38.93	0.000775476750850138\\
38.94	0.000775479432480623\\
38.95	0.000775482115263997\\
38.96	0.000775484799200962\\
38.97	0.000775487484292223\\
38.98	0.000775490170538485\\
38.99	0.000775492857940452\\
39	0.000775495546498833\\
39.01	0.000775498236214334\\
39.02	0.000775500927087662\\
39.03	0.000775503619119527\\
39.04	0.000775506312310638\\
39.05	0.000775509006661703\\
39.06	0.000775511702173434\\
39.07	0.000775514398846541\\
39.08	0.000775517096681735\\
39.09	0.000775519795679731\\
39.1	0.000775522495841239\\
39.11	0.000775525197166973\\
39.12	0.000775527899657649\\
39.13	0.00077553060331398\\
39.14	0.000775533308136683\\
39.15	0.000775536014126475\\
39.16	0.00077553872128407\\
39.17	0.000775541429610187\\
39.18	0.000775544139105546\\
39.19	0.000775546849770863\\
39.2	0.000775549561606859\\
39.21	0.000775552274614254\\
39.22	0.000775554988793769\\
39.23	0.000775557704146125\\
39.24	0.000775560420672044\\
39.25	0.000775563138372249\\
39.26	0.000775565857247465\\
39.27	0.000775568577298415\\
39.28	0.000775571298525823\\
39.29	0.000775574020930416\\
39.3	0.00077557674451292\\
39.31	0.000775579469274061\\
39.32	0.000775582195214567\\
39.33	0.000775584922335166\\
39.34	0.000775587650636587\\
39.35	0.00077559038011956\\
39.36	0.000775593110784815\\
39.37	0.000775595842633084\\
39.38	0.000775598575665096\\
39.39	0.000775601309881585\\
39.4	0.000775604045283283\\
39.41	0.000775606781870926\\
39.42	0.000775609519645245\\
39.43	0.000775612258606978\\
39.44	0.000775614998756859\\
39.45	0.000775617740095625\\
39.46	0.000775620482624011\\
39.47	0.000775623226342758\\
39.48	0.000775625971252603\\
39.49	0.000775628717354284\\
39.5	0.000775631464648542\\
39.51	0.000775634213136119\\
39.52	0.000775636962817754\\
39.53	0.000775639713694189\\
39.54	0.000775642465766167\\
39.55	0.000775645219034431\\
39.56	0.000775647973499726\\
39.57	0.000775650729162797\\
39.58	0.000775653486024389\\
39.59	0.000775656244085246\\
39.6	0.000775659003346118\\
39.61	0.000775661763807752\\
39.62	0.000775664525470893\\
39.63	0.000775667288336293\\
39.64	0.000775670052404701\\
39.65	0.000775672817676867\\
39.66	0.000775675584153543\\
39.67	0.000775678351835481\\
39.68	0.000775681120723432\\
39.69	0.000775683890818149\\
39.7	0.000775686662120387\\
39.71	0.000775689434630902\\
39.72	0.000775692208350448\\
39.73	0.000775694983279781\\
39.74	0.000775697759419658\\
39.75	0.000775700536770835\\
39.76	0.000775703315334072\\
39.77	0.000775706095110129\\
39.78	0.000775708876099764\\
39.79	0.000775711658303738\\
39.8	0.000775714441722813\\
39.81	0.000775717226357751\\
39.82	0.000775720012209314\\
39.83	0.000775722799278265\\
39.84	0.000775725587565369\\
39.85	0.000775728377071391\\
39.86	0.000775731167797097\\
39.87	0.000775733959743253\\
39.88	0.000775736752910625\\
39.89	0.000775739547299984\\
39.9	0.000775742342912096\\
39.91	0.000775745139747732\\
39.92	0.000775747937807662\\
39.93	0.000775750737092656\\
39.94	0.000775753537603487\\
39.95	0.000775756339340926\\
39.96	0.000775759142305748\\
39.97	0.000775761946498726\\
39.98	0.000775764751920637\\
39.99	0.000775767558572255\\
40	0.000775770366454357\\
40.01	0.000775773175567719\\
};
\addplot [color=black,solid,forget plot]
  table[row sep=crcr]{%
40.01	0.000775773175567719\\
40.02	0.00077577598591312\\
40.03	0.000775778797491338\\
40.04	0.000775781610303155\\
40.05	0.000775784424349349\\
40.06	0.000775787239630702\\
40.07	0.000775790056147996\\
40.08	0.000775792873902014\\
40.09	0.000775795692893539\\
40.1	0.000775798513123356\\
40.11	0.000775801334592249\\
40.12	0.000775804157301005\\
40.13	0.000775806981250411\\
40.14	0.000775809806441255\\
40.15	0.000775812632874325\\
40.16	0.000775815460550411\\
40.17	0.000775818289470303\\
40.18	0.000775821119634791\\
40.19	0.000775823951044667\\
40.2	0.000775826783700724\\
40.21	0.000775829617603757\\
40.22	0.000775832452754558\\
40.23	0.000775835289153924\\
40.24	0.000775838126802649\\
40.25	0.000775840965701532\\
40.26	0.00077584380585137\\
40.27	0.000775846647252961\\
40.28	0.000775849489907107\\
40.29	0.000775852333814605\\
40.3	0.000775855178976258\\
40.31	0.000775858025392869\\
40.32	0.00077586087306524\\
40.33	0.000775863721994174\\
40.34	0.000775866572180478\\
40.35	0.000775869423624957\\
40.36	0.000775872276328417\\
40.37	0.000775875130291667\\
40.38	0.000775877985515512\\
40.39	0.000775880842000766\\
40.4	0.000775883699748235\\
40.41	0.000775886558758734\\
40.42	0.000775889419033072\\
40.43	0.000775892280572064\\
40.44	0.000775895143376525\\
40.45	0.000775898007447267\\
40.46	0.000775900872785108\\
40.47	0.000775903739390862\\
40.48	0.000775906607265351\\
40.49	0.000775909476409391\\
40.5	0.000775912346823802\\
40.51	0.000775915218509407\\
40.52	0.000775918091467025\\
40.53	0.00077592096569748\\
40.54	0.000775923841201596\\
40.55	0.000775926717980196\\
40.56	0.000775929596034107\\
40.57	0.000775932475364156\\
40.58	0.000775935355971169\\
40.59	0.000775938237855977\\
40.6	0.000775941121019407\\
40.61	0.000775944005462293\\
40.62	0.000775946891185464\\
40.63	0.000775949778189754\\
40.64	0.000775952666475996\\
40.65	0.000775955556045026\\
40.66	0.000775958446897681\\
40.67	0.000775961339034795\\
40.68	0.000775964232457209\\
40.69	0.000775967127165759\\
40.7	0.00077597002316129\\
40.71	0.000775972920444639\\
40.72	0.00077597581901665\\
40.73	0.000775978718878167\\
40.74	0.000775981620030034\\
40.75	0.000775984522473098\\
40.76	0.000775987426208205\\
40.77	0.000775990331236202\\
40.78	0.000775993237557939\\
40.79	0.000775996145174268\\
40.8	0.000775999054086039\\
40.81	0.000776001964294105\\
40.82	0.000776004875799319\\
40.83	0.000776007788602536\\
40.84	0.000776010702704612\\
40.85	0.000776013618106405\\
40.86	0.000776016534808775\\
40.87	0.000776019452812578\\
40.88	0.000776022372118679\\
40.89	0.000776025292727937\\
40.9	0.000776028214641216\\
40.91	0.000776031137859382\\
40.92	0.0007760340623833\\
40.93	0.000776036988213838\\
40.94	0.000776039915351863\\
40.95	0.000776042843798244\\
40.96	0.000776045773553854\\
40.97	0.000776048704619564\\
40.98	0.000776051636996247\\
40.99	0.000776054570684779\\
41	0.000776057505686035\\
41.01	0.000776060442000893\\
41.02	0.00077606337963023\\
41.03	0.000776066318574929\\
41.04	0.000776069258835868\\
41.05	0.000776072200413933\\
41.06	0.000776075143310006\\
41.07	0.000776078087524973\\
41.08	0.00077608103305972\\
41.09	0.000776083979915135\\
41.1	0.000776086928092109\\
41.11	0.000776089877591531\\
41.12	0.000776092828414296\\
41.13	0.000776095780561297\\
41.14	0.000776098734033427\\
41.15	0.000776101688831585\\
41.16	0.000776104644956668\\
41.17	0.000776107602409575\\
41.18	0.000776110561191208\\
41.19	0.000776113521302468\\
41.2	0.00077611648274426\\
41.21	0.000776119445517488\\
41.22	0.00077612240962306\\
41.23	0.000776125375061885\\
41.24	0.000776128341834872\\
41.25	0.000776131309942932\\
41.26	0.000776134279386978\\
41.27	0.000776137250167926\\
41.28	0.000776140222286691\\
41.29	0.000776143195744188\\
41.3	0.00077614617054134\\
41.31	0.000776149146679065\\
41.32	0.000776152124158287\\
41.33	0.000776155102979927\\
41.34	0.000776158083144914\\
41.35	0.000776161064654172\\
41.36	0.00077616404750863\\
41.37	0.00077616703170922\\
41.38	0.000776170017256873\\
41.39	0.000776173004152522\\
41.4	0.000776175992397102\\
41.41	0.000776178981991551\\
41.42	0.000776181972936807\\
41.43	0.000776184965233807\\
41.44	0.000776187958883496\\
41.45	0.000776190953886816\\
41.46	0.000776193950244714\\
41.47	0.000776196947958135\\
41.48	0.000776199947028029\\
41.49	0.000776202947455343\\
41.5	0.000776205949241033\\
41.51	0.000776208952386051\\
41.52	0.000776211956891351\\
41.53	0.000776214962757892\\
41.54	0.00077621796998663\\
41.55	0.000776220978578528\\
41.56	0.000776223988534549\\
41.57	0.000776226999855655\\
41.58	0.000776230012542813\\
41.59	0.00077623302659699\\
41.6	0.000776236042019155\\
41.61	0.000776239058810281\\
41.62	0.000776242076971339\\
41.63	0.000776245096503305\\
41.64	0.000776248117407155\\
41.65	0.000776251139683867\\
41.66	0.000776254163334421\\
41.67	0.000776257188359801\\
41.68	0.000776260214760988\\
41.69	0.000776263242538967\\
41.7	0.000776266271694729\\
41.71	0.000776269302229259\\
41.72	0.000776272334143549\\
41.73	0.000776275367438595\\
41.74	0.000776278402115388\\
41.75	0.000776281438174925\\
41.76	0.000776284475618207\\
41.77	0.000776287514446229\\
41.78	0.000776290554659997\\
41.79	0.000776293596260514\\
41.8	0.000776296639248783\\
41.81	0.000776299683625815\\
41.82	0.000776302729392617\\
41.83	0.000776305776550201\\
41.84	0.000776308825099577\\
41.85	0.000776311875041763\\
41.86	0.000776314926377773\\
41.87	0.000776317979108626\\
41.88	0.000776321033235342\\
41.89	0.000776324088758943\\
41.9	0.000776327145680452\\
41.91	0.000776330204000895\\
41.92	0.000776333263721299\\
41.93	0.000776336324842692\\
41.94	0.000776339387366105\\
41.95	0.000776342451292572\\
41.96	0.000776345516623126\\
41.97	0.000776348583358802\\
41.98	0.00077635165150064\\
41.99	0.000776354721049676\\
42	0.000776357792006955\\
42.01	0.000776360864373517\\
42.02	0.000776363938150408\\
42.03	0.000776367013338674\\
42.04	0.000776370089939362\\
42.05	0.000776373167953521\\
42.06	0.000776376247382205\\
42.07	0.000776379328226466\\
42.08	0.000776382410487357\\
42.09	0.000776385494165933\\
42.1	0.000776388579263255\\
42.11	0.00077639166578038\\
42.12	0.000776394753718369\\
42.13	0.000776397843078284\\
42.14	0.000776400933861192\\
42.15	0.000776404026068154\\
42.16	0.000776407119700239\\
42.17	0.000776410214758516\\
42.18	0.000776413311244055\\
42.19	0.000776416409157925\\
42.2	0.0007764195085012\\
42.21	0.000776422609274954\\
42.22	0.000776425711480263\\
42.23	0.000776428815118201\\
42.24	0.000776431920189849\\
42.25	0.000776435026696286\\
42.26	0.000776438134638593\\
42.27	0.00077644124401785\\
42.28	0.000776444354835142\\
42.29	0.000776447467091552\\
42.3	0.000776450580788167\\
42.31	0.000776453695926072\\
42.32	0.000776456812506356\\
42.33	0.000776459930530106\\
42.34	0.000776463049998415\\
42.35	0.000776466170912373\\
42.36	0.000776469293273071\\
42.37	0.000776472417081602\\
42.38	0.000776475542339061\\
42.39	0.000776478669046543\\
42.4	0.000776481797205141\\
42.41	0.000776484926815955\\
42.42	0.00077648805788008\\
42.43	0.000776491190398616\\
42.44	0.000776494324372662\\
42.45	0.000776497459803316\\
42.46	0.00077650059669168\\
42.47	0.000776503735038854\\
42.48	0.000776506874845942\\
42.49	0.000776510016114043\\
42.5	0.000776513158844264\\
42.51	0.000776516303037706\\
42.52	0.000776519448695473\\
42.53	0.00077652259581867\\
42.54	0.000776525744408401\\
42.55	0.000776528894465773\\
42.56	0.000776532045991891\\
42.57	0.000776535198987861\\
42.58	0.00077653835345479\\
42.59	0.000776541509393784\\
42.6	0.000776544666805951\\
42.61	0.000776547825692397\\
42.62	0.00077655098605423\\
42.63	0.000776554147892558\\
42.64	0.000776557311208488\\
42.65	0.00077656047600313\\
42.66	0.00077656364227759\\
42.67	0.000776566810032976\\
42.68	0.000776569979270397\\
42.69	0.000776573149990962\\
42.7	0.000776576322195776\\
42.71	0.000776579495885952\\
42.72	0.000776582671062594\\
42.73	0.000776585847726812\\
42.74	0.000776589025879713\\
42.75	0.000776592205522406\\
42.76	0.000776595386655997\\
42.77	0.000776598569281596\\
42.78	0.000776601753400309\\
42.79	0.000776604939013244\\
42.8	0.000776608126121507\\
42.81	0.000776611314726207\\
42.82	0.00077661450482845\\
42.83	0.000776617696429343\\
42.84	0.000776620889529994\\
42.85	0.000776624084131506\\
42.86	0.000776627280234986\\
42.87	0.000776630477841543\\
42.88	0.000776633676952281\\
42.89	0.000776636877568307\\
42.9	0.000776640079690725\\
42.91	0.00077664328332064\\
42.92	0.000776646488459159\\
42.93	0.000776649695107387\\
42.94	0.000776652903266431\\
42.95	0.000776656112937394\\
42.96	0.000776659324121383\\
42.97	0.000776662536819501\\
42.98	0.000776665751032855\\
42.99	0.00077666896676255\\
43	0.00077667218400969\\
43.01	0.000776675402775383\\
43.02	0.000776678623060734\\
43.03	0.00077668184486685\\
43.04	0.000776685068194836\\
43.05	0.000776688293045799\\
43.06	0.000776691519420846\\
43.07	0.000776694747321085\\
43.08	0.000776697976747624\\
43.09	0.000776701207701569\\
43.1	0.000776704440184031\\
43.11	0.000776707674196121\\
43.12	0.000776710909738947\\
43.13	0.000776714146813622\\
43.14	0.000776717385421255\\
43.15	0.000776720625562961\\
43.16	0.000776723867239851\\
43.17	0.000776727110453042\\
43.18	0.000776730355203649\\
43.19	0.000776733601492788\\
43.2	0.000776736849321575\\
43.21	0.000776740098691131\\
43.22	0.000776743349602576\\
43.23	0.000776746602057029\\
43.24	0.000776749856055614\\
43.25	0.000776753111599452\\
43.26	0.000776756368689669\\
43.27	0.000776759627327392\\
43.28	0.000776762887513747\\
43.29	0.000776766149249862\\
43.3	0.000776769412536868\\
43.31	0.000776772677375893\\
43.32	0.000776775943768073\\
43.33	0.000776779211714538\\
43.34	0.000776782481216425\\
43.35	0.000776785752274866\\
43.36	0.000776789024891001\\
43.37	0.000776792299065968\\
43.38	0.000776795574800907\\
43.39	0.000776798852096958\\
43.4	0.000776802130955262\\
43.41	0.000776805411376962\\
43.42	0.000776808693363204\\
43.43	0.000776811976915133\\
43.44	0.000776815262033895\\
43.45	0.00077681854872064\\
43.46	0.000776821836976516\\
43.47	0.000776825126802674\\
43.48	0.000776828418200266\\
43.49	0.000776831711170446\\
43.5	0.000776835005714367\\
43.51	0.000776838301833184\\
43.52	0.000776841599528054\\
43.53	0.000776844898800136\\
43.54	0.000776848199650588\\
43.55	0.000776851502080572\\
43.56	0.00077685480609125\\
43.57	0.000776858111683782\\
43.58	0.000776861418859335\\
43.59	0.000776864727619073\\
43.6	0.000776868037964165\\
43.61	0.000776871349895776\\
43.62	0.000776874663415077\\
43.63	0.000776877978523237\\
43.64	0.000776881295221429\\
43.65	0.000776884613510826\\
43.66	0.0007768879333926\\
43.67	0.000776891254867927\\
43.68	0.000776894577937986\\
43.69	0.000776897902603952\\
43.7	0.000776901228867005\\
43.71	0.000776904556728325\\
43.72	0.000776907886189093\\
43.73	0.000776911217250492\\
43.74	0.000776914549913707\\
43.75	0.00077691788417992\\
43.76	0.000776921220050322\\
43.77	0.000776924557526096\\
43.78	0.000776927896608434\\
43.79	0.000776931237298523\\
43.8	0.000776934579597556\\
43.81	0.000776937923506726\\
43.82	0.000776941269027226\\
43.83	0.000776944616160249\\
43.84	0.000776947964906995\\
43.85	0.000776951315268658\\
43.86	0.000776954667246438\\
43.87	0.000776958020841535\\
43.88	0.000776961376055148\\
43.89	0.000776964732888481\\
43.9	0.000776968091342735\\
43.91	0.000776971451419117\\
43.92	0.000776974813118831\\
43.93	0.000776978176443085\\
43.94	0.000776981541393086\\
43.95	0.000776984907970044\\
43.96	0.000776988276175169\\
43.97	0.000776991646009672\\
43.98	0.000776995017474769\\
43.99	0.00077699839057167\\
44	0.000777001765301592\\
44.01	0.000777005141665751\\
44.02	0.000777008519665365\\
44.03	0.000777011899301653\\
44.04	0.000777015280575834\\
44.05	0.000777018663489129\\
44.06	0.000777022048042761\\
44.07	0.000777025434237954\\
44.08	0.000777028822075929\\
44.09	0.000777032211557916\\
44.1	0.000777035602685138\\
44.11	0.000777038995458825\\
44.12	0.000777042389880207\\
44.13	0.000777045785950512\\
44.14	0.000777049183670973\\
44.15	0.000777052583042823\\
44.16	0.000777055984067293\\
44.17	0.000777059386745622\\
44.18	0.000777062791079043\\
44.19	0.000777066197068793\\
44.2	0.000777069604716111\\
44.21	0.000777073014022234\\
44.22	0.000777076424988407\\
44.23	0.000777079837615867\\
44.24	0.000777083251905859\\
44.25	0.000777086667859625\\
44.26	0.000777090085478412\\
44.27	0.000777093504763465\\
44.28	0.00077709692571603\\
44.29	0.000777100348337355\\
44.3	0.00077710377262869\\
44.31	0.000777107198591285\\
44.32	0.000777110626226391\\
44.33	0.000777114055535261\\
44.34	0.000777117486519147\\
44.35	0.000777120919179305\\
44.36	0.000777124353516989\\
44.37	0.000777127789533455\\
44.38	0.000777131227229963\\
44.39	0.000777134666607769\\
44.4	0.000777138107668135\\
44.41	0.000777141550412318\\
44.42	0.000777144994841583\\
44.43	0.000777148440957191\\
44.44	0.000777151888760406\\
44.45	0.000777155338252492\\
44.46	0.000777158789434714\\
44.47	0.00077716224230834\\
44.48	0.000777165696874636\\
44.49	0.000777169153134871\\
44.5	0.000777172611090313\\
44.51	0.000777176070742236\\
44.52	0.000777179532091908\\
44.53	0.000777182995140603\\
44.54	0.000777186459889594\\
44.55	0.000777189926340155\\
44.56	0.00077719339449356\\
44.57	0.000777196864351085\\
44.58	0.000777200335914007\\
44.59	0.000777203809183604\\
44.6	0.000777207284161155\\
44.61	0.000777210760847937\\
44.62	0.000777214239245233\\
44.63	0.000777217719354323\\
44.64	0.000777221201176488\\
44.65	0.000777224684713012\\
44.66	0.000777228169965179\\
44.67	0.000777231656934272\\
44.68	0.000777235145621578\\
44.69	0.000777238636028382\\
44.7	0.000777242128155971\\
44.71	0.000777245622005632\\
44.72	0.000777249117578654\\
44.73	0.000777252614876326\\
44.74	0.000777256113899939\\
44.75	0.000777259614650781\\
44.76	0.000777263117130144\\
44.77	0.000777266621339322\\
44.78	0.000777270127279607\\
44.79	0.000777273634952292\\
44.8	0.000777277144358671\\
44.81	0.000777280655500038\\
44.82	0.00077728416837769\\
44.83	0.000777287682992922\\
44.84	0.000777291199347033\\
44.85	0.000777294717441318\\
44.86	0.000777298237277074\\
44.87	0.000777301758855603\\
44.88	0.000777305282178202\\
44.89	0.00077730880724617\\
44.9	0.000777312334060807\\
44.91	0.000777315862623416\\
44.92	0.000777319392935297\\
44.93	0.000777322924997752\\
44.94	0.000777326458812084\\
44.95	0.000777329994379595\\
44.96	0.000777333531701587\\
44.97	0.000777337070779366\\
44.98	0.000777340611614235\\
44.99	0.000777344154207498\\
45	0.000777347698560462\\
45.01	0.000777351244674431\\
45.02	0.00077735479255071\\
45.03	0.000777358342190606\\
45.04	0.000777361893595428\\
45.05	0.000777365446766478\\
45.06	0.000777369001705067\\
45.07	0.0007773725584125\\
45.08	0.000777376116890086\\
45.09	0.000777379677139131\\
45.1	0.000777383239160946\\
45.11	0.000777386802956838\\
45.12	0.000777390368528115\\
45.13	0.000777393935876086\\
45.14	0.00077739750500206\\
45.15	0.000777401075907347\\
45.16	0.000777404648593256\\
45.17	0.000777408223061096\\
45.18	0.000777411799312176\\
45.19	0.000777415377347806\\
45.2	0.000777418957169294\\
45.21	0.000777422538777951\\
45.22	0.000777426122175085\\
45.23	0.000777429707362009\\
45.24	0.000777433294340029\\
45.25	0.000777436883110456\\
45.26	0.000777440473674599\\
45.27	0.000777444066033766\\
45.28	0.000777447660189267\\
45.29	0.000777451256142411\\
45.3	0.000777454853894508\\
45.31	0.000777458453446863\\
45.32	0.000777462054800789\\
45.33	0.000777465657957589\\
45.34	0.000777469262918573\\
45.35	0.000777472869685051\\
45.36	0.000777476478258326\\
45.37	0.000777480088639707\\
45.38	0.0007774837008305\\
45.39	0.000777487314832011\\
45.4	0.000777490930645545\\
45.41	0.000777494548272408\\
45.42	0.000777498167713905\\
45.43	0.000777501788971338\\
45.44	0.000777505412046011\\
45.45	0.000777509036939229\\
45.46	0.000777512663652292\\
45.47	0.000777516292186503\\
45.48	0.000777519922543162\\
45.49	0.000777523554723571\\
45.5	0.000777527188729029\\
45.51	0.000777530824560833\\
45.52	0.000777534462220283\\
45.53	0.000777538101708676\\
45.54	0.000777541743027308\\
45.55	0.000777545386177473\\
45.56	0.000777549031160467\\
45.57	0.000777552677977585\\
45.58	0.000777556326630117\\
45.59	0.000777559977119356\\
45.6	0.000777563629446594\\
45.61	0.000777567283613119\\
45.62	0.000777570939620218\\
45.63	0.00077757459746918\\
45.64	0.00077757825716129\\
45.65	0.000777581918697832\\
45.66	0.000777585582080093\\
45.67	0.000777589247309352\\
45.68	0.000777592914386891\\
45.69	0.000777596583313989\\
45.7	0.000777600254091924\\
45.71	0.000777603926721972\\
45.72	0.000777607601205408\\
45.73	0.000777611277543507\\
45.74	0.000777614955737539\\
45.75	0.000777618635788775\\
45.76	0.000777622317698484\\
45.77	0.000777626001467933\\
45.78	0.000777629687098387\\
45.79	0.000777633374591108\\
45.8	0.000777637063947359\\
45.81	0.000777640755168398\\
45.82	0.000777644448255484\\
45.83	0.000777648143209873\\
45.84	0.000777651840032818\\
45.85	0.000777655538725571\\
45.86	0.000777659239289382\\
45.87	0.000777662941725495\\
45.88	0.000777666646035157\\
45.89	0.000777670352219612\\
45.9	0.000777674060280101\\
45.91	0.000777677770217861\\
45.92	0.000777681482034127\\
45.93	0.000777685195730134\\
45.94	0.00077768891130711\\
45.95	0.000777692628766286\\
45.96	0.000777696348108887\\
45.97	0.000777700069336135\\
45.98	0.000777703792449251\\
45.99	0.000777707517449454\\
46	0.000777711244337956\\
46.01	0.000777714973115972\\
46.02	0.000777718703784708\\
46.03	0.000777722436345371\\
46.04	0.000777726170799164\\
46.05	0.000777729907147286\\
46.06	0.000777733645390934\\
46.07	0.000777737385531302\\
46.08	0.000777741127569578\\
46.09	0.000777744871506952\\
46.1	0.000777748617344604\\
46.11	0.000777752365083716\\
46.12	0.000777756114725464\\
46.13	0.000777759866271021\\
46.14	0.000777763619721558\\
46.15	0.000777767375078238\\
46.16	0.000777771132342222\\
46.17	0.000777774891514671\\
46.18	0.000777778652596737\\
46.19	0.000777782415589571\\
46.2	0.000777786180494318\\
46.21	0.000777789947312122\\
46.22	0.00077779371604412\\
46.23	0.000777797486691446\\
46.24	0.000777801259255228\\
46.25	0.000777805033736593\\
46.26	0.000777808810136661\\
46.27	0.000777812588456547\\
46.28	0.000777816368697365\\
46.29	0.00077782015086022\\
46.3	0.000777823934946216\\
46.31	0.000777827720956449\\
46.32	0.000777831508892013\\
46.33	0.000777835298753995\\
46.34	0.000777839090543478\\
46.35	0.000777842884261542\\
46.36	0.000777846679909256\\
46.37	0.00077785047748769\\
46.38	0.000777854276997907\\
46.39	0.000777858078440963\\
46.4	0.000777861881817908\\
46.41	0.00077786568712979\\
46.42	0.000777869494377649\\
46.43	0.000777873303562519\\
46.44	0.000777877114685431\\
46.45	0.000777880927747407\\
46.46	0.000777884742749464\\
46.47	0.000777888559692616\\
46.48	0.000777892378577866\\
46.49	0.000777896199406214\\
46.5	0.000777900022178655\\
46.51	0.000777903846896173\\
46.52	0.000777907673559749\\
46.53	0.00077791150217036\\
46.54	0.00077791533272897\\
46.55	0.000777919165236542\\
46.56	0.000777922999694029\\
46.57	0.000777926836102382\\
46.58	0.000777930674462537\\
46.59	0.000777934514775432\\
46.6	0.00077793835704199\\
46.61	0.000777942201263135\\
46.62	0.000777946047439776\\
46.63	0.000777949895572821\\
46.64	0.000777953745663166\\
46.65	0.000777957597711702\\
46.66	0.000777961451719312\\
46.67	0.000777965307686871\\
46.68	0.000777969165615247\\
46.69	0.000777973025505301\\
46.7	0.000777976887357885\\
46.71	0.000777980751173841\\
46.72	0.000777984616954008\\
46.73	0.000777988484699212\\
46.74	0.000777992354410274\\
46.75	0.000777996226088004\\
46.76	0.000778000099733207\\
46.77	0.000778003975346676\\
46.78	0.000778007852929198\\
46.79	0.000778011732481551\\
46.8	0.000778015614004503\\
46.81	0.000778019497498816\\
46.82	0.000778023382965239\\
46.83	0.000778027270404514\\
46.84	0.000778031159817375\\
46.85	0.000778035051204546\\
46.86	0.000778038944566739\\
46.87	0.000778042839904663\\
46.88	0.000778046737219013\\
46.89	0.000778050636510473\\
46.9	0.000778054537779724\\
46.91	0.00077805844102743\\
46.92	0.000778062346254249\\
46.93	0.000778066253460828\\
46.94	0.000778070162647805\\
46.95	0.000778074073815809\\
46.96	0.000778077986965458\\
46.97	0.000778081902097357\\
46.98	0.000778085819212106\\
46.99	0.000778089738310289\\
47	0.000778093659392486\\
47.01	0.00077809758245926\\
47.02	0.000778101507511167\\
47.03	0.000778105434548752\\
47.04	0.00077810936357255\\
47.05	0.000778113294583082\\
47.06	0.000778117227580864\\
47.07	0.000778121162566394\\
47.08	0.000778125099540164\\
47.09	0.000778129038502653\\
47.1	0.000778132979454329\\
47.11	0.00077813692239565\\
47.12	0.000778140867327061\\
47.13	0.000778144814248996\\
47.14	0.000778148763161879\\
47.15	0.000778152714066118\\
47.16	0.000778156666962116\\
47.17	0.00077816062185026\\
47.18	0.000778164578730927\\
47.19	0.000778168537604483\\
47.2	0.000778172498471277\\
47.21	0.000778176461331654\\
47.22	0.000778180426185941\\
47.23	0.000778184393034455\\
47.24	0.000778188361877503\\
47.25	0.000778192332715377\\
47.26	0.000778196305548358\\
47.27	0.000778200280376715\\
47.28	0.000778204257200703\\
47.29	0.000778208236020568\\
47.3	0.000778212216836544\\
47.31	0.000778216199648848\\
47.32	0.000778220184457688\\
47.33	0.00077822417126326\\
47.34	0.000778228160065746\\
47.35	0.000778232150865315\\
47.36	0.000778236143662127\\
47.37	0.000778240138456327\\
47.38	0.000778244135248048\\
47.39	0.00077824813403741\\
47.4	0.000778252134824521\\
47.41	0.000778256137609477\\
47.42	0.000778260142392359\\
47.43	0.00077826414917324\\
47.44	0.000778268157952178\\
47.45	0.000778272168729215\\
47.46	0.000778276181504386\\
47.47	0.000778280196277712\\
47.48	0.0007782842130492\\
47.49	0.000778288231818847\\
47.5	0.000778292252586636\\
47.51	0.000778296275352537\\
47.52	0.000778300300116509\\
47.53	0.000778304326878499\\
47.54	0.000778308355638439\\
47.55	0.000778312386396253\\
47.56	0.00077831641915185\\
47.57	0.000778320453905128\\
47.58	0.000778324490655971\\
47.59	0.000778328529404255\\
47.6	0.000778332570149838\\
47.61	0.000778336612892574\\
47.62	0.0007783406576323\\
47.63	0.000778344704368843\\
47.64	0.000778348753102018\\
47.65	0.000778352803831629\\
47.66	0.000778356856557468\\
47.67	0.000778360911279316\\
47.68	0.000778364967996945\\
47.69	0.000778369026710113\\
47.7	0.00077837308741857\\
47.71	0.000778377150122052\\
47.72	0.000778381214820288\\
47.73	0.000778385281512995\\
47.74	0.00077838935019988\\
47.75	0.00077839342088064\\
47.76	0.000778397493554963\\
47.77	0.000778401568222524\\
47.78	0.000778405644882994\\
47.79	0.00077840972353603\\
47.8	0.000778413804181283\\
47.81	0.000778417886818393\\
47.82	0.000778421971446994\\
47.83	0.000778426058066708\\
47.84	0.000778430146677153\\
47.85	0.000778434237277937\\
47.86	0.000778438329868658\\
47.87	0.000778442424448912\\
47.88	0.000778446521018284\\
47.89	0.000778450619576355\\
47.9	0.000778454720122697\\
47.91	0.000778458822656876\\
47.92	0.000778462927178454\\
47.93	0.000778467033686987\\
47.94	0.000778471142182024\\
47.95	0.000778475252663113\\
47.96	0.000778479365129793\\
47.97	0.000778483479581602\\
47.98	0.000778487596018073\\
47.99	0.000778491714438738\\
48	0.000778495834843122\\
48.01	0.000778499957230751\\
48.02	0.000778504081601149\\
48.03	0.000778508207953835\\
48.04	0.000778512336288331\\
48.05	0.000778516466604155\\
48.06	0.000778520598900826\\
48.07	0.000778524733177865\\
48.08	0.00077852886943479\\
48.09	0.000778533007671124\\
48.1	0.00077853714788639\\
48.11	0.000778541290080113\\
48.12	0.000778545434251824\\
48.13	0.000778549580401054\\
48.14	0.000778553728527337\\
48.15	0.000778557878630216\\
48.16	0.000778562030709237\\
48.17	0.000778566184763952\\
48.18	0.000778570340793918\\
48.19	0.000778574498798703\\
48.2	0.000778578658777878\\
48.21	0.000778582820731026\\
48.22	0.000778586984657739\\
48.23	0.000778591150557615\\
48.24	0.00077859531843027\\
48.25	0.000778599488275323\\
48.26	0.000778603660092411\\
48.27	0.000778607833881182\\
48.28	0.000778612009641296\\
48.29	0.000778616187372431\\
48.3	0.000778620367074276\\
48.31	0.00077862454874654\\
48.32	0.000778628732388946\\
48.33	0.000778632918001237\\
48.34	0.000778637105583173\\
48.35	0.000778641295134535\\
48.36	0.000778645486655122\\
48.37	0.000778649680144758\\
48.38	0.000778653875603283\\
48.39	0.000778658073030567\\
48.4	0.000778662272426503\\
48.41	0.000778666473791004\\
48.42	0.000778670677124015\\
48.43	0.000778674882425506\\
48.44	0.000778679089695474\\
48.45	0.000778683298933946\\
48.46	0.00077868751014098\\
48.47	0.000778691723316663\\
48.48	0.000778695938461119\\
48.49	0.000778700155574501\\
48.5	0.000778704374656998\\
48.51	0.000778708595708836\\
48.52	0.000778712818730275\\
48.53	0.000778717043721617\\
48.54	0.000778721270683199\\
48.55	0.000778725499615402\\
48.56	0.000778729730518646\\
48.57	0.000778733963393395\\
48.58	0.000778738198240156\\
48.59	0.000778742435059482\\
48.6	0.000778746673851973\\
48.61	0.000778750914618273\\
48.62	0.000778755157359081\\
48.63	0.000778759402075143\\
48.64	0.000778763648767255\\
48.65	0.000778767897436269\\
48.66	0.000778772148083088\\
48.67	0.000778776400708671\\
48.68	0.000778780655314037\\
48.69	0.00077878491190026\\
48.7	0.000778789170468471\\
48.71	0.000778793431019869\\
48.72	0.000778797693555709\\
48.73	0.000778801958077311\\
48.74	0.00077880622458606\\
48.75	0.000778810493083407\\
48.76	0.000778814763570871\\
48.77	0.000778819036050038\\
48.78	0.000778823310522567\\
48.79	0.000778827586990187\\
48.8	0.0007788318654547\\
48.81	0.000778836145917982\\
48.82	0.000778840428381986\\
48.83	0.000778844712848741\\
48.84	0.000778848999320354\\
48.85	0.000778853287799012\\
48.86	0.000778857578286983\\
48.87	0.000778861870786619\\
48.88	0.000778866165300351\\
48.89	0.000778870461830701\\
48.9	0.000778874760380275\\
48.91	0.000778879060951764\\
48.92	0.00077888336354795\\
48.93	0.000778887668171704\\
48.94	0.000778891974825993\\
48.95	0.00077889628351387\\
48.96	0.000778900594238484\\
48.97	0.000778904907003079\\
48.98	0.000778909221810996\\
48.99	0.000778913538665674\\
49	0.000778917857570645\\
49.01	0.000778922178529547\\
49.02	0.000778926501546113\\
49.03	0.000778930826624181\\
49.04	0.000778935153767689\\
49.05	0.000778939482980679\\
49.06	0.000778943814267297\\
49.07	0.000778948147631792\\
49.08	0.000778952483078523\\
49.09	0.000778956820611951\\
49.1	0.000778961160236645\\
49.11	0.000778965501957284\\
49.12	0.000778969845778655\\
49.13	0.00077897419170565\\
49.14	0.000778978539743274\\
49.15	0.000778982889896642\\
49.16	0.000778987242170976\\
49.17	0.000778991596571613\\
49.18	0.000778995953103996\\
49.19	0.00077900031177368\\
49.2	0.000779004672586332\\
49.21	0.00077900903554773\\
49.22	0.000779013400663761\\
49.23	0.000779017767940425\\
49.24	0.000779022137383827\\
49.25	0.00077902650900019\\
49.26	0.00077903088279584\\
49.27	0.000779035258777214\\
49.28	0.000779039636950859\\
49.29	0.000779044017323427\\
49.3	0.000779048399901679\\
49.31	0.00077905278469248\\
49.32	0.000779057171702801\\
49.33	0.000779061560939716\\
49.34	0.000779065952410401\\
49.35	0.000779070346122134\\
49.36	0.000779074742082293\\
49.37	0.000779079140298351\\
49.38	0.000779083540777882\\
49.39	0.00077908794352855\\
49.4	0.000779092348558112\\
49.41	0.000779096755874415\\
49.42	0.000779101165485397\\
49.43	0.000779105577399077\\
49.44	0.000779109991623559\\
49.45	0.000779114408167027\\
49.46	0.000779118827037741\\
49.47	0.000779123248244034\\
49.48	0.000779127671794316\\
49.49	0.000779132097697059\\
49.5	0.000779136525960801\\
49.51	0.00077914095659414\\
49.52	0.000779145389605735\\
49.53	0.000779149825004292\\
49.54	0.000779154262798572\\
49.55	0.000779158702997374\\
49.56	0.000779163145609545\\
49.57	0.000779167590643962\\
49.58	0.000779172038109536\\
49.59	0.000779176488015204\\
49.6	0.000779180940369923\\
49.61	0.00077918539518267\\
49.62	0.000779189852462428\\
49.63	0.000779194312218191\\
49.64	0.000779198774458948\\
49.65	0.000779203239193688\\
49.66	0.000779207706431383\\
49.67	0.00077921217618099\\
49.68	0.000779216648451443\\
49.69	0.000779221123251647\\
49.7	0.000779225600590471\\
49.71	0.000779230080476742\\
49.72	0.000779234562919237\\
49.73	0.000779239047926679\\
49.74	0.000779243535507729\\
49.75	0.000779248025670981\\
49.76	0.000779252518424952\\
49.77	0.000779257013778081\\
49.78	0.000779261511738715\\
49.79	0.00077926601231511\\
49.8	0.000779270515515415\\
49.81	0.000779275021347678\\
49.82	0.000779279529819828\\
49.83	0.000779284040939673\\
49.84	0.000779288554714899\\
49.85	0.000779293071153051\\
49.86	0.000779297590261539\\
49.87	0.000779302112047629\\
49.88	0.000779306636518436\\
49.89	0.000779311163680917\\
49.9	0.000779315693541872\\
49.91	0.000779320226107934\\
49.92	0.000779324761385571\\
49.93	0.000779329299381071\\
49.94	0.000779333840100552\\
49.95	0.000779338383549951\\
49.96	0.000779342929735029\\
49.97	0.00077934747866136\\
49.98	0.000779352030334338\\
49.99	0.000779356584759174\\
50	0.000779361141940899\\
50.01	0.000779365701884367\\
50.02	0.000779370264594254\\
50.03	0.000779374830075064\\
50.04	0.000779379398331134\\
50.05	0.000779383969366642\\
50.06	0.000779388543185617\\
50.07	0.00077939311979194\\
50.08	0.000779397699189366\\
50.09	0.000779402281381528\\
50.1	0.000779406866371957\\
50.11	0.000779411454164096\\
50.12	0.000779416044761321\\
50.13	0.000779420638166958\\
50.14	0.000779425234384309\\
50.15	0.000779429833416679\\
50.16	0.000779434435267382\\
50.17	0.000779439039939735\\
50.18	0.00077944364743707\\
50.19	0.000779448257762723\\
50.2	0.000779452870920039\\
50.21	0.000779457486912374\\
50.22	0.000779462105743089\\
50.23	0.000779466727415558\\
50.24	0.00077947135193316\\
50.25	0.000779475979299287\\
50.26	0.000779480609517338\\
50.27	0.000779485242590723\\
50.28	0.00077948987852286\\
50.29	0.000779494517317177\\
50.3	0.000779499158977114\\
50.31	0.000779503803506118\\
50.32	0.000779508450907651\\
50.33	0.000779513101185181\\
50.34	0.000779517754342188\\
50.35	0.000779522410382165\\
50.36	0.000779527069308613\\
50.37	0.000779531731125048\\
50.38	0.000779536395834992\\
50.39	0.000779541063441984\\
50.4	0.000779545733949571\\
50.41	0.000779550407361313\\
50.42	0.000779555083680784\\
50.43	0.000779559762911568\\
50.44	0.000779564445057264\\
50.45	0.00077956913012148\\
50.46	0.000779573818107842\\
50.47	0.000779578509019984\\
50.48	0.000779583202861554\\
50.49	0.000779587899636216\\
50.5	0.000779592599347645\\
50.51	0.000779597301999531\\
50.52	0.000779602007595576\\
50.53	0.000779606716139497\\
50.54	0.000779611427635024\\
50.55	0.000779616142085902\\
50.56	0.000779620859495889\\
50.57	0.00077962557986876\\
50.58	0.000779630303208299\\
50.59	0.00077963502951831\\
50.6	0.000779639758802611\\
50.61	0.000779644491065034\\
50.62	0.000779649226309425\\
50.63	0.000779653964539646\\
50.64	0.000779658705759575\\
50.65	0.000779663449973106\\
50.66	0.000779668197184146\\
50.67	0.000779672947396623\\
50.68	0.000779677700614477\\
50.69	0.000779682456841664\\
50.7	0.000779687216082159\\
50.71	0.00077969197833995\\
50.72	0.000779696743619046\\
50.73	0.000779701511923471\\
50.74	0.000779706283257266\\
50.75	0.000779711057624489\\
50.76	0.000779715835029216\\
50.77	0.000779720615475541\\
50.78	0.000779725398967576\\
50.79	0.000779730185509451\\
50.8	0.000779734975105311\\
50.81	0.000779739767759326\\
50.82	0.00077974456347568\\
50.83	0.000779749362258575\\
50.84	0.000779754164112237\\
50.85	0.000779758969040907\\
50.86	0.000779763777048846\\
50.87	0.000779768588140339\\
50.88	0.000779773402319685\\
50.89	0.000779778219591208\\
50.9	0.000779783039959249\\
50.91	0.00077978786342817\\
50.92	0.000779792690002357\\
50.93	0.000779797519686215\\
50.94	0.00077980235248417\\
50.95	0.00077980718840067\\
50.96	0.000779812027440185\\
50.97	0.000779816869607207\\
50.98	0.000779821714906251\\
50.99	0.000779826563341853\\
51	0.000779831414918575\\
51.01	0.000779836269641\\
51.02	0.000779841127513734\\
51.03	0.000779845988541408\\
51.04	0.000779850852728676\\
51.05	0.000779855720080217\\
51.06	0.000779860590600734\\
51.07	0.000779865464294953\\
51.08	0.000779870341167629\\
51.09	0.000779875221223542\\
51.1	0.000779880104467496\\
51.11	0.000779884990904318\\
51.12	0.00077988988053887\\
51.13	0.000779894773376031\\
51.14	0.000779899669420714\\
51.15	0.000779904568677856\\
51.16	0.000779909471152421\\
51.17	0.000779914376849405\\
51.18	0.000779919285773829\\
51.19	0.000779924197930741\\
51.2	0.000779929113325223\\
51.21	0.000779934031962382\\
51.22	0.000779938953847356\\
51.23	0.000779943878985315\\
51.24	0.000779948807381456\\
51.25	0.000779953739041009\\
51.26	0.000779958673969235\\
51.27	0.000779963612171428\\
51.28	0.000779968553652909\\
51.29	0.000779973498419037\\
51.3	0.000779978446475199\\
51.31	0.00077998339782682\\
51.32	0.000779988352479355\\
51.33	0.000779993310438293\\
51.34	0.000779998271709159\\
51.35	0.000780003236297512\\
51.36	0.000780008204208947\\
51.37	0.000780013175449092\\
51.38	0.000780018150023615\\
51.39	0.000780023127938219\\
51.4	0.00078002810919864\\
51.41	0.000780033093810658\\
51.42	0.000780038081780088\\
51.43	0.000780043073112783\\
51.44	0.000780048067814634\\
51.45	0.000780053065891574\\
51.46	0.000780058067349576\\
51.47	0.000780063072194651\\
51.48	0.00078006808043285\\
51.49	0.000780073092070269\\
51.5	0.000780078107113044\\
51.51	0.000780083125567354\\
51.52	0.000780088147439422\\
51.53	0.000780093172735511\\
51.54	0.000780098201461932\\
51.55	0.000780103233625038\\
51.56	0.000780108269231229\\
51.57	0.00078011330828695\\
51.58	0.000780118350798692\\
51.59	0.000780123396772994\\
51.6	0.000780128446216442\\
51.61	0.000780133499135671\\
51.62	0.000780138555537362\\
51.63	0.000780143615428251\\
51.64	0.000780148678815117\\
51.65	0.000780153745704793\\
51.66	0.000780158816104168\\
51.67	0.000780163890020176\\
51.68	0.000780168967459806\\
51.69	0.000780174048430102\\
51.7	0.000780179132938159\\
51.71	0.000780184220991131\\
51.72	0.000780189312596226\\
51.73	0.000780194407760703\\
51.74	0.000780199506491887\\
51.75	0.000780204608797153\\
51.76	0.000780209714683937\\
51.77	0.000780214824159737\\
51.78	0.000780219937232107\\
51.79	0.000780225053908663\\
51.8	0.000780230174197084\\
51.81	0.000780235298105107\\
51.82	0.000780240425640538\\
51.83	0.000780245556811244\\
51.84	0.000780250691625154\\
51.85	0.000780255830090268\\
51.86	0.000780260972214649\\
51.87	0.000780266118006427\\
51.88	0.000780271267473804\\
51.89	0.000780276420625047\\
51.9	0.000780281577468494\\
51.91	0.000780286738012555\\
51.92	0.000780291902265712\\
51.93	0.000780297070236519\\
51.94	0.000780302241933603\\
51.95	0.000780307417365669\\
51.96	0.000780312596541493\\
51.97	0.00078031777946993\\
51.98	0.000780322966159914\\
51.99	0.000780328156620456\\
52	0.000780333350860647\\
52.01	0.00078033854888966\\
52.02	0.000780343750716748\\
52.03	0.00078034895635125\\
52.04	0.000780354165802584\\
52.05	0.000780359379080258\\
52.06	0.000780364596193864\\
52.07	0.000780369817153081\\
52.08	0.000780375041967677\\
52.09	0.000780380270647512\\
52.1	0.000780385503202535\\
52.11	0.000780390739642785\\
52.12	0.000780395979978398\\
52.13	0.000780401224219605\\
52.14	0.000780406472376727\\
52.15	0.000780411724460188\\
52.16	0.000780416980480508\\
52.17	0.000780422240448306\\
52.18	0.000780427504374305\\
52.19	0.000780432772269326\\
52.2	0.000780438044144297\\
52.21	0.00078044332001025\\
52.22	0.000780448599878322\\
52.23	0.00078045388375976\\
52.24	0.00078045917166592\\
52.25	0.000780464463608266\\
52.26	0.00078046975959838\\
52.27	0.000780475059647949\\
52.28	0.000780480363768784\\
52.29	0.000780485671972806\\
52.3	0.000780490984272058\\
52.31	0.000780496300678701\\
52.32	0.000780501621205019\\
52.33	0.000780506945863417\\
52.34	0.000780512274666426\\
52.35	0.000780517607626702\\
52.36	0.00078052294475703\\
52.37	0.000780528286070323\\
52.38	0.000780533631579627\\
52.39	0.000780538981298122\\
52.4	0.000780544335239119\\
52.41	0.000780549693416068\\
52.42	0.000780555055842557\\
52.43	0.000780560422532314\\
52.44	0.000780565793499211\\
52.45	0.000780571168757262\\
52.46	0.000780576548320625\\
52.47	0.000780581932203608\\
52.48	0.00078058732042067\\
52.49	0.00078059271298642\\
52.5	0.000780598109915619\\
52.51	0.000780603511223185\\
52.52	0.000780608916924195\\
52.53	0.000780614327033885\\
52.54	0.000780619741567653\\
52.55	0.00078062516054106\\
52.56	0.000780630583969833\\
52.57	0.000780636011869871\\
52.58	0.000780641444257237\\
52.59	0.000780646881148173\\
52.6	0.000780652322559092\\
52.61	0.000780657768506587\\
52.62	0.000780663219007431\\
52.63	0.000780668674078576\\
52.64	0.000780674133737161\\
52.65	0.00078067959800051\\
52.66	0.000780685066886139\\
52.67	0.000780690540411754\\
52.68	0.000780696018595255\\
52.69	0.000780701501454739\\
52.7	0.000780706989008503\\
52.71	0.000780712481275044\\
52.72	0.000780717978273067\\
52.73	0.000780723480021483\\
52.74	0.000780728986539412\\
52.75	0.000780734497846186\\
52.76	0.000780740013961355\\
52.77	0.000780745534904688\\
52.78	0.00078075106069617\\
52.79	0.000780756591356015\\
52.8	0.000780762126904663\\
52.81	0.000780767667362781\\
52.82	0.000780773212751273\\
52.83	0.000780778763091275\\
52.84	0.000780784318404166\\
52.85	0.000780789878711562\\
52.86	0.00078079544403533\\
52.87	0.00078080101439758\\
52.88	0.000780806589820675\\
52.89	0.000780812170327235\\
52.9	0.000780817755940138\\
52.91	0.000780823346682519\\
52.92	0.000780828942577782\\
52.93	0.000780834543649597\\
52.94	0.000780840149921906\\
52.95	0.000780845761418927\\
52.96	0.000780851378165155\\
52.97	0.000780857000185368\\
52.98	0.000780862627504632\\
52.99	0.000780868260148296\\
53	0.000780873898142008\\
53.01	0.000780879541511712\\
53.02	0.00078088519028365\\
53.03	0.000780890844484371\\
53.04	0.000780896504140729\\
53.05	0.000780902169279894\\
53.06	0.000780907839929351\\
53.07	0.000780913516116902\\
53.08	0.000780919197870678\\
53.09	0.000780924885219136\\
53.1	0.000780930578191065\\
53.11	0.000780936276815592\\
53.12	0.000780941981122184\\
53.13	0.000780947691140652\\
53.14	0.000780953406901159\\
53.15	0.000780959128434221\\
53.16	0.00078096485577071\\
53.17	0.000780970588941863\\
53.18	0.000780976327979285\\
53.19	0.000780982072914953\\
53.2	0.000780987823781217\\
53.21	0.000780993580610812\\
53.22	0.000780999343436856\\
53.23	0.000781005112292862\\
53.24	0.000781010887212734\\
53.25	0.00078101666823078\\
53.26	0.000781022455381711\\
53.27	0.000781028248700648\\
53.28	0.000781034048223132\\
53.29	0.000781039853985118\\
53.3	0.000781045666022991\\
53.31	0.000781051484373569\\
53.32	0.0007810573090741\\
53.33	0.000781063140162278\\
53.34	0.000781068977676243\\
53.35	0.000781074821654588\\
53.36	0.00078108067213636\\
53.37	0.000781086529161076\\
53.38	0.000781092392768719\\
53.39	0.000781098262999744\\
53.4	0.000781104139895092\\
53.41	0.000781110023496184\\
53.42	0.000781115913844938\\
53.43	0.000781121810983767\\
53.44	0.000781127714955591\\
53.45	0.000781133625803836\\
53.46	0.000781139543572448\\
53.47	0.000781145468305891\\
53.48	0.000781151400049163\\
53.49	0.000781157338847793\\
53.5	0.00078116328474785\\
53.51	0.000781169237795953\\
53.52	0.000781175198039273\\
53.53	0.000781181165525542\\
53.54	0.00078118714030306\\
53.55	0.000781193122420698\\
53.56	0.000781199111927909\\
53.57	0.000781205108874736\\
53.58	0.000781211113311808\\
53.59	0.000781217125290362\\
53.6	0.000781223144862238\\
53.61	0.000781229172079893\\
53.62	0.000781235206996406\\
53.63	0.000781241249665483\\
53.64	0.000781247300141468\\
53.65	0.000781253358479346\\
53.66	0.000781259424734754\\
53.67	0.000781265498963988\\
53.68	0.000781271581224006\\
53.69	0.000781277671572444\\
53.7	0.000781283770067615\\
53.71	0.000781289876768518\\
53.72	0.000781295991734855\\
53.73	0.000781302115027025\\
53.74	0.000781308246706141\\
53.75	0.000781314386834035\\
53.76	0.000781320535473267\\
53.77	0.000781326692687129\\
53.78	0.00078133285853966\\
53.79	0.000781339033095652\\
53.8	0.000781345216420651\\
53.81	0.000781351408580978\\
53.82	0.000781357609643721\\
53.83	0.000781363819676761\\
53.84	0.00078137003874877\\
53.85	0.000781376266929219\\
53.86	0.000781382504288391\\
53.87	0.000781388750897387\\
53.88	0.000781395006828136\\
53.89	0.000781401272153405\\
53.9	0.000781407546946802\\
53.91	0.000781413831282793\\
53.92	0.000781420125236703\\
53.93	0.00078142642888473\\
53.94	0.000781432742303953\\
53.95	0.000781439065572343\\
53.96	0.000781445398768765\\
53.97	0.000781451741972998\\
53.98	0.000781458095265732\\
53.99	0.000781464458728587\\
54	0.000781470832444122\\
54.01	0.000781477216495834\\
54.02	0.00078148361096818\\
54.03	0.00078149001594658\\
54.04	0.000781496431517427\\
54.05	0.000781502857768099\\
54.06	0.000781509294786965\\
54.07	0.000781515742663398\\
54.08	0.000781522201487781\\
54.09	0.000781528671351523\\
54.1	0.00078153515234706\\
54.11	0.000781541644567873\\
54.12	0.000781548148108492\\
54.13	0.00078155466306451\\
54.14	0.000781561189532589\\
54.15	0.000781567727610476\\
54.16	0.000781574277397007\\
54.17	0.000781580838992116\\
54.18	0.000781587412496853\\
54.19	0.000781593998013386\\
54.2	0.000781600595645015\\
54.21	0.000781607205496181\\
54.22	0.000781613827672478\\
54.23	0.000781620462280658\\
54.24	0.000781627109428648\\
54.25	0.000781633769225555\\
54.26	0.000781640441781677\\
54.27	0.000781647127208514\\
54.28	0.000781653825618782\\
54.29	0.000781660537126413\\
54.3	0.000781667261846575\\
54.31	0.00078167399989568\\
54.32	0.000781680751391388\\
54.33	0.000781687516452625\\
54.34	0.000781694295199588\\
54.35	0.000781701087753758\\
54.36	0.000781707894237909\\
54.37	0.000781714714776116\\
54.38	0.000781721549493767\\
54.39	0.000781728398517575\\
54.4	0.000781735261975584\\
54.41	0.000781742139997181\\
54.42	0.000781749032713103\\
54.43	0.000781755940255452\\
54.44	0.000781762862757702\\
54.45	0.000781769800354706\\
54.46	0.000781776753182709\\
54.47	0.000781783721379357\\
54.48	0.000781790705083704\\
54.49	0.000781797704436227\\
54.5	0.000781804719578828\\
54.51	0.000781811750654848\\
54.52	0.000781818797809074\\
54.53	0.000781825861187751\\
54.54	0.000781832940938587\\
54.55	0.000781840037210763\\
54.56	0.000781847150154943\\
54.57	0.000781854279923283\\
54.58	0.000781861426669434\\
54.59	0.00078186859054856\\
54.6	0.000781875771717334\\
54.61	0.000781882970333957\\
54.62	0.00078189018655816\\
54.63	0.000781897420551212\\
54.64	0.000781904672475931\\
54.65	0.000781911942496683\\
54.66	0.000781919230779402\\
54.67	0.000781926537491583\\
54.68	0.000781933862802299\\
54.69	0.000781941206882202\\
54.7	0.000781948569903535\\
54.71	0.00078195595204013\\
54.72	0.000781963353467419\\
54.73	0.000781970774362439\\
54.74	0.000781978214903837\\
54.75	0.000781985675271877\\
54.76	0.000781993155648442\\
54.77	0.000782000656217037\\
54.78	0.000782008177162801\\
54.79	0.0007820157186725\\
54.8	0.000782023280934544\\
54.81	0.000782030864138979\\
54.82	0.000782038468477493\\
54.83	0.000782046094143425\\
54.84	0.000782053741331759\\
54.85	0.000782061410239133\\
54.86	0.000782069101063836\\
54.87	0.00078207681400581\\
54.88	0.000782084549266657\\
54.89	0.00078209230704963\\
54.9	0.000782100087559643\\
54.91	0.000782107891003263\\
54.92	0.000782115717588716\\
54.93	0.00078212356752588\\
54.94	0.00078213144102629\\
54.95	0.000782139338303131\\
54.96	0.00078214725957124\\
54.97	0.000782155205047099\\
54.98	0.000782163174948836\\
54.99	0.000782171169496219\\
55	0.000782179188910654\\
55.01	0.000782187233415175\\
55.02	0.000782195303234447\\
55.03	0.000782203398594757\\
55.04	0.000782211519724001\\
55.05	0.000782219666851691\\
55.06	0.000782227840208935\\
55.07	0.000782236040028439\\
55.08	0.000782244266544489\\
55.09	0.00078225251999295\\
55.1	0.000782260800611253\\
55.11	0.000782269108638385\\
55.12	0.00078227744431488\\
55.13	0.000782285807882801\\
55.14	0.000782294199585741\\
55.15	0.000782302619668796\\
55.16	0.000782311068378559\\
55.17	0.000782319545963106\\
55.18	0.000782328052671982\\
55.19	0.000782336588756177\\
55.2	0.000782345154468123\\
55.21	0.000782353750061665\\
55.22	0.000782362375792054\\
55.23	0.00078237103191592\\
55.24	0.000782379718691259\\
55.25	0.000782388436377408\\
55.26	0.000782397185235031\\
55.27	0.00078240596552609\\
55.28	0.000782414777513826\\
55.29	0.000782423621462741\\
55.3	0.000782432497638569\\
55.31	0.000782441406308249\\
55.32	0.000782450347739909\\
55.33	0.00078245932220283\\
55.34	0.000782468329967426\\
55.35	0.000782477371305214\\
55.36	0.000782486446488786\\
55.37	0.000782495555791777\\
55.38	0.000782504699488838\\
55.39	0.000782513877855606\\
55.4	0.00078252309116867\\
55.41	0.000782532339705537\\
55.42	0.000782541623744606\\
55.43	0.000782550943565124\\
55.44	0.000782560299447158\\
55.45	0.000782569691671557\\
55.46	0.000782579120519917\\
55.47	0.000782588586274542\\
55.48	0.000782598089218406\\
55.49	0.000782607629635116\\
55.5	0.000782617207808872\\
55.51	0.000782626824024427\\
55.52	0.000782636478567043\\
55.53	0.000782646171722455\\
55.54	0.000782655903776826\\
55.55	0.000782665675016702\\
55.56	0.000782675485728974\\
55.57	0.000782685336200829\\
55.58	0.000782695226719707\\
55.59	0.000782705157573258\\
55.6	0.000782715129049292\\
55.61	0.000782725141435738\\
55.62	0.000782735195020597\\
55.63	0.000782745290091889\\
55.64	0.000782755426937615\\
55.65	0.000782765605845703\\
55.66	0.000782775827103965\\
55.67	0.000782786091000045\\
55.68	0.000782796397821375\\
55.69	0.000782806747855126\\
55.7	0.000782817141388158\\
55.71	0.000782827578706978\\
55.72	0.000782838060097686\\
55.73	0.000782848585845933\\
55.74	0.000782859156236873\\
55.75	0.000782869771555113\\
55.76	0.000782880432084673\\
55.77	0.00078289113810894\\
55.78	0.00078290188991062\\
55.79	0.000782912687771696\\
55.8	0.000782923531973389\\
55.81	0.000782934422796113\\
55.82	0.000782945360519441\\
55.83	0.000782956345422055\\
55.84	0.000782967377781723\\
55.85	0.000782978457875253\\
55.86	0.000782989585978467\\
55.87	0.000783000762366166\\
55.88	0.0007830119873121\\
55.89	0.000783023261088948\\
55.9	0.000783034583968289\\
55.91	0.00078304595622058\\
55.92	0.00078305737811514\\
55.93	0.000783068849920141\\
55.94	0.000783080371902589\\
55.95	0.00078309194432832\\
55.96	0.000783103567461998\\
55.97	0.000783115241567116\\
55.98	0.000783126966906003\\
55.99	0.000783138743739833\\
56	0.000783150572328643\\
56.01	0.000783162452931356\\
56.02	0.000783174385805809\\
56.03	0.000783186371208784\\
56.04	0.000783198409396056\\
56.05	0.000783210500622432\\
56.06	0.000783222645141819\\
56.07	0.000783234843207277\\
56.08	0.000783247095071099\\
56.09	0.000783259400984889\\
56.1	0.000783271761199656\\
56.11	0.000783284175965909\\
56.12	0.000783296645533775\\
56.13	0.00078330917015312\\
56.14	0.000783321750073684\\
56.15	0.000783334385545221\\
56.16	0.000783347076817669\\
56.17	0.000783359824141317\\
56.18	0.000783372627766992\\
56.19	0.000783385487946266\\
56.2	0.000783398404931669\\
56.21	0.000783411378976931\\
56.22	0.000783424410337226\\
56.23	0.000783437499269448\\
56.24	0.000783450646032505\\
56.25	0.000783463850887617\\
56.26	0.000783477114098659\\
56.27	0.000783490435928048\\
56.28	0.000783503816628239\\
56.29	0.000783517256441201\\
56.3	0.000783530755597889\\
56.31	0.000783544314317684\\
56.32	0.000783557932807816\\
56.33	0.00078357161126278\\
56.34	0.000783585349863716\\
56.35	0.000783599148777776\\
56.36	0.00078361300815747\\
56.37	0.000783626928139988\\
56.38	0.000783640908846495\\
56.39	0.000783654950381414\\
56.4	0.000783669052831676\\
56.41	0.000783683216265941\\
56.42	0.000783697440733807\\
56.43	0.00078371172626498\\
56.44	0.000783726072868425\\
56.45	0.000783740480531481\\
56.46	0.000783754949218955\\
56.47	0.000783769478872182\\
56.48	0.000783784069408051\\
56.49	0.000783798720718012\\
56.5	0.00078381343266703\\
56.51	0.000783828205092528\\
56.52	0.000783843037803274\\
56.53	0.000783857930578251\\
56.54	0.000783872883165476\\
56.55	0.000783887895280793\\
56.56	0.000783902966606612\\
56.57	0.000783918096790626\\
56.58	0.000783933285444472\\
56.59	0.000783948532142358\\
56.6	0.000783963836419635\\
56.61	0.000783979197771345\\
56.62	0.000783994615650695\\
56.63	0.000784010089467506\\
56.64	0.000784025618586599\\
56.65	0.000784041202326139\\
56.66	0.000784056839955916\\
56.67	0.00078407253069558\\
56.68	0.000784088273712824\\
56.69	0.000784104068121497\\
56.7	0.000784119912979667\\
56.71	0.000784135807287624\\
56.72	0.000784151749985815\\
56.73	0.000784167739952716\\
56.74	0.000784183776002643\\
56.75	0.000784199856883487\\
56.76	0.000784215981274384\\
56.77	0.000784232147783309\\
56.78	0.000784248354944598\\
56.79	0.000784264601216391\\
56.8	0.000784280884977999\\
56.81	0.000784297204527183\\
56.82	0.000784313558077359\\
56.83	0.000784329943754703\\
56.84	0.000784346359595185\\
56.85	0.000784362803541494\\
56.86	0.000784379273439875\\
56.87	0.000784395767036873\\
56.88	0.000784412281975976\\
56.89	0.000784428815794143\\
56.9	0.000784445365918245\\
56.91	0.000784461929661385\\
56.92	0.000784478504219102\\
56.93	0.000784495086665471\\
56.94	0.000784511676594733\\
56.95	0.000784528274011972\\
56.96	0.000784544878922277\\
56.97	0.000784561491330741\\
56.98	0.000784578111242464\\
56.99	0.000784594738662549\\
57	0.000784611373596106\\
57.01	0.00078462801604825\\
57.02	0.000784644666024101\\
57.03	0.000784661323528784\\
57.04	0.00078467798856743\\
57.05	0.000784694661145174\\
57.06	0.000784711341267157\\
57.07	0.000784728028938525\\
57.08	0.00078474472416443\\
57.09	0.000784761426950029\\
57.1	0.000784778137300483\\
57.11	0.00078479485522096\\
57.12	0.000784811580716633\\
57.13	0.000784828313792679\\
57.14	0.000784845054454282\\
57.15	0.000784861802706632\\
57.16	0.000784878558554922\\
57.17	0.000784895322004351\\
57.18	0.000784912093060124\\
57.19	0.000784928871727451\\
57.2	0.000784945658011548\\
57.21	0.000784962451917635\\
57.22	0.000784979253450939\\
57.23	0.000784996062616691\\
57.24	0.000785012879420127\\
57.25	0.000785029703866491\\
57.26	0.00078504653596103\\
57.27	0.000785063375708998\\
57.28	0.000785080223115653\\
57.29	0.00078509707818626\\
57.3	0.000785113940926086\\
57.31	0.000785130811340408\\
57.32	0.000785147689434508\\
57.33	0.000785164575213668\\
57.34	0.000785181468683182\\
57.35	0.000785198369848345\\
57.36	0.000785215278714462\\
57.37	0.000785232195286839\\
57.38	0.000785249119570789\\
57.39	0.000785266051571632\\
57.4	0.000785282991294692\\
57.41	0.000785299938745298\\
57.42	0.000785316893928786\\
57.43	0.000785333856850498\\
57.44	0.000785350827515779\\
57.45	0.000785367805929981\\
57.46	0.000785384792098462\\
57.47	0.000785401786026584\\
57.48	0.000785418787719717\\
57.49	0.000785435797183235\\
57.5	0.00078545281442252\\
57.51	0.000785469839442953\\
57.52	0.000785486872249929\\
57.53	0.000785503912848843\\
57.54	0.000785520961245099\\
57.55	0.000785538017444102\\
57.56	0.000785555081451268\\
57.57	0.000785572153272015\\
57.58	0.000785589232911768\\
57.59	0.000785606320375958\\
57.6	0.00078562341567002\\
57.61	0.000785640518799398\\
57.62	0.000785657629769539\\
57.63	0.000785674748585895\\
57.64	0.000785691875253926\\
57.65	0.000785709009779097\\
57.66	0.000785726152166877\\
57.67	0.000785743302422743\\
57.68	0.000785760460552178\\
57.69	0.000785777626560668\\
57.7	0.000785794800453706\\
57.71	0.000785811982236793\\
57.72	0.000785829171915433\\
57.73	0.000785846369495137\\
57.74	0.000785863574981421\\
57.75	0.000785880788379807\\
57.76	0.000785898009695822\\
57.77	0.000785915238935002\\
57.78	0.000785932476102884\\
57.79	0.000785949721205015\\
57.8	0.000785966974246946\\
57.81	0.000785984235234235\\
57.82	0.000786001504172443\\
57.83	0.000786018781067141\\
57.84	0.000786036065923902\\
57.85	0.000786053358748307\\
57.86	0.000786070659545944\\
57.87	0.000786087968322403\\
57.88	0.000786105285083282\\
57.89	0.000786122609834187\\
57.9	0.000786139942580727\\
57.91	0.000786157283328518\\
57.92	0.000786174632083182\\
57.93	0.000786191988850346\\
57.94	0.000786209353635643\\
57.95	0.000786226726444714\\
57.96	0.000786244107283205\\
57.97	0.000786261496156767\\
57.98	0.000786278893071056\\
57.99	0.000786296298031739\\
58	0.000786313711044482\\
58.01	0.000786331132114963\\
58.02	0.000786348561248863\\
58.03	0.000786365998451869\\
58.04	0.000786383443729674\\
58.05	0.00078640089708798\\
58.06	0.000786418358532491\\
58.07	0.000786435828068918\\
58.08	0.000786453305702978\\
58.09	0.000786470791440399\\
58.1	0.000786488285286907\\
58.11	0.000786505787248239\\
58.12	0.000786523297330137\\
58.13	0.000786540815538351\\
58.14	0.000786558341878633\\
58.15	0.000786575876356745\\
58.16	0.000786593418978453\\
58.17	0.000786610969749529\\
58.18	0.000786628528675754\\
58.19	0.00078664609576291\\
58.2	0.000786663671016791\\
58.21	0.000786681254443193\\
58.22	0.00078669884604792\\
58.23	0.00078671644583678\\
58.24	0.000786734053815592\\
58.25	0.000786751669990175\\
58.26	0.000786769294366359\\
58.27	0.00078678692694998\\
58.28	0.000786804567746876\\
58.29	0.000786822216762895\\
58.3	0.000786839874003891\\
58.31	0.000786857539475723\\
58.32	0.000786875213184258\\
58.33	0.000786892895135366\\
58.34	0.000786910585334928\\
58.35	0.000786928283788827\\
58.36	0.000786945990502955\\
58.37	0.000786963705483208\\
58.38	0.000786981428735491\\
58.39	0.000786999160265712\\
58.4	0.00078701690007979\\
58.41	0.000787034648183646\\
58.42	0.000787052404583211\\
58.43	0.000787070169284419\\
58.44	0.000787087942293211\\
58.45	0.000787105723615537\\
58.46	0.00078712351325735\\
58.47	0.000787141311224614\\
58.48	0.000787159117523292\\
58.49	0.000787176932159361\\
58.5	0.0007871947551388\\
58.51	0.000787212586467597\\
58.52	0.000787230426151744\\
58.53	0.00078724827419724\\
58.54	0.000787266130610094\\
58.55	0.000787283995396317\\
58.56	0.000787301868561928\\
58.57	0.000787319750112953\\
58.58	0.000787337640055423\\
58.59	0.000787355538395379\\
58.6	0.000787373445138864\\
58.61	0.00078739136029193\\
58.62	0.000787409283860637\\
58.63	0.000787427215851049\\
58.64	0.000787445156269237\\
58.65	0.000787463105121278\\
58.66	0.00078748106241326\\
58.67	0.00078749902815127\\
58.68	0.00078751700234141\\
58.69	0.000787534984989782\\
58.7	0.000787552976102496\\
58.71	0.000787570975685672\\
58.72	0.000787588983745435\\
58.73	0.000787607000287913\\
58.74	0.000787625025319245\\
58.75	0.000787643058845577\\
58.76	0.000787661100873058\\
58.77	0.000787679151407848\\
58.78	0.000787697210456108\\
58.79	0.000787715278024011\\
58.8	0.000787733354117736\\
58.81	0.000787751438743467\\
58.82	0.000787769531907394\\
58.83	0.000787787633615717\\
58.84	0.000787805743874642\\
58.85	0.000787823862690377\\
58.86	0.000787841990069144\\
58.87	0.000787860126017167\\
58.88	0.000787878270540677\\
58.89	0.000787896423645915\\
58.9	0.000787914585339125\\
58.91	0.000787932755626561\\
58.92	0.000787950934514482\\
58.93	0.000787969122009154\\
58.94	0.000787987318116851\\
58.95	0.000788005522843853\\
58.96	0.000788023736196446\\
58.97	0.000788041958180925\\
58.98	0.000788060188803589\\
58.99	0.000788078428070748\\
59	0.000788096675988717\\
59.01	0.000788114932563816\\
59.02	0.000788133197802373\\
59.03	0.000788151471710725\\
59.04	0.000788169754295215\\
59.05	0.000788188045562192\\
59.06	0.000788206345518011\\
59.07	0.000788224654169037\\
59.08	0.000788242971521641\\
59.09	0.000788261297582199\\
59.1	0.000788279632357097\\
59.11	0.000788297975852727\\
59.12	0.000788316328075488\\
59.13	0.000788334689031783\\
59.14	0.000788353058728028\\
59.15	0.000788371437170641\\
59.16	0.000788389824366052\\
59.17	0.000788408220320693\\
59.18	0.000788426625041005\\
59.19	0.000788445038533439\\
59.2	0.000788463460804448\\
59.21	0.000788481891860496\\
59.22	0.000788500331708054\\
59.23	0.000788518780353598\\
59.24	0.000788537237803612\\
59.25	0.000788555704064589\\
59.26	0.000788574179143029\\
59.27	0.000788592663045436\\
59.28	0.000788611155778324\\
59.29	0.000788629657348213\\
59.3	0.000788648167761632\\
59.31	0.000788666687025116\\
59.32	0.000788685215145208\\
59.33	0.000788703752128457\\
59.34	0.000788722297981419\\
59.35	0.00078874085271066\\
59.36	0.000788759416322752\\
59.37	0.000788777988824275\\
59.38	0.000788796570221814\\
59.39	0.000788815160521962\\
59.4	0.000788833759731322\\
59.41	0.000788852367856502\\
59.42	0.000788870984904117\\
59.43	0.000788889610880793\\
59.44	0.00078890824579316\\
59.45	0.000788926889647856\\
59.46	0.000788945542451527\\
59.47	0.000788964204210827\\
59.48	0.000788982874932415\\
59.49	0.000789001554622961\\
59.5	0.00078902024328914\\
59.51	0.000789038940937637\\
59.52	0.000789057647575142\\
59.53	0.000789076363208352\\
59.54	0.000789095087843975\\
59.55	0.000789113821488724\\
59.56	0.000789132564149319\\
59.57	0.00078915131583249\\
59.58	0.000789170076544974\\
59.59	0.000789188846293514\\
59.6	0.000789207625084862\\
59.61	0.000789226412925779\\
59.62	0.00078924520982303\\
59.63	0.00078926401578339\\
59.64	0.000789282830813641\\
59.65	0.000789301654920574\\
59.66	0.000789320488110985\\
59.67	0.000789339330391683\\
59.68	0.000789358181769478\\
59.69	0.000789377042251193\\
59.7	0.000789395911843656\\
59.71	0.000789414790553703\\
59.72	0.000789433678388178\\
59.73	0.000789452575353936\\
59.74	0.000789471481457834\\
59.75	0.000789490396706741\\
59.76	0.000789509321107534\\
59.77	0.000789528254667094\\
59.78	0.000789547197392314\\
59.79	0.000789566149290094\\
59.8	0.000789585110367341\\
59.81	0.000789604080630968\\
59.82	0.0007896230600879\\
59.83	0.000789642048745067\\
59.84	0.00078966104660941\\
59.85	0.000789680053687875\\
59.86	0.000789699069987417\\
59.87	0.000789718095514997\\
59.88	0.000789737130277589\\
59.89	0.000789756174282171\\
59.9	0.000789775227535729\\
59.91	0.00078979429004526\\
59.92	0.000789813361817767\\
59.93	0.00078983244286026\\
59.94	0.000789851533179758\\
59.95	0.000789870632783293\\
59.96	0.000789889741677896\\
59.97	0.000789908859870613\\
59.98	0.000789927987368495\\
59.99	0.000789947124178602\\
60	0.000789966270308005\\
60.01	0.000789985425763777\\
60.02	0.000790004590553004\\
60.03	0.000790023764682781\\
60.04	0.00079004294816021\\
60.05	0.000790062140992396\\
60.06	0.000790081343186462\\
60.07	0.000790100554749531\\
60.08	0.000790119775688738\\
60.09	0.000790139006011227\\
60.1	0.000790158245724149\\
60.11	0.000790177494834662\\
60.12	0.000790196753349936\\
60.13	0.000790216021277147\\
60.14	0.000790235298623479\\
60.15	0.000790254585396126\\
60.16	0.00079027388160229\\
60.17	0.000790293187249181\\
60.18	0.000790312502344017\\
60.19	0.000790331826894025\\
60.2	0.000790351160906443\\
60.21	0.000790370504388512\\
60.22	0.000790389857347487\\
60.23	0.000790409219790629\\
60.24	0.000790428591725208\\
60.25	0.000790447973158501\\
60.26	0.000790467364097797\\
60.27	0.00079048676455039\\
60.28	0.000790506174523586\\
60.29	0.000790525594024698\\
60.3	0.000790545023061046\\
60.31	0.000790564461639963\\
60.32	0.000790583909768787\\
60.33	0.000790603367454866\\
60.34	0.000790622834705556\\
60.35	0.000790642311528223\\
60.36	0.000790661797930243\\
60.37	0.000790681293918996\\
60.38	0.000790700799501876\\
60.39	0.000790720314686284\\
60.4	0.000790739839479629\\
60.41	0.000790759373889328\\
60.42	0.00079077891792281\\
60.43	0.000790798471587511\\
60.44	0.000790818034890877\\
60.45	0.000790837607840362\\
60.46	0.000790857190443429\\
60.47	0.000790876782707549\\
60.48	0.000790896384640204\\
60.49	0.000790915996248887\\
60.5	0.000790935617541094\\
60.51	0.000790955248524334\\
60.52	0.000790974889206125\\
60.53	0.000790994539593994\\
60.54	0.000791014199695477\\
60.55	0.000791033869518117\\
60.56	0.00079105354906947\\
60.57	0.000791073238357099\\
60.58	0.000791092937388577\\
60.59	0.000791112646171483\\
60.6	0.00079113236471341\\
60.61	0.00079115209302196\\
60.62	0.000791171831104739\\
60.63	0.000791191578969369\\
60.64	0.000791211336623476\\
60.65	0.000791231104074698\\
60.66	0.000791250881330684\\
60.67	0.000791270668399088\\
60.68	0.000791290465287578\\
60.69	0.000791310272003826\\
60.7	0.000791330088555519\\
60.71	0.000791349914950352\\
60.72	0.000791369751196027\\
60.73	0.000791389597300258\\
60.74	0.000791409453270768\\
60.75	0.000791429319115289\\
60.76	0.000791449194841564\\
60.77	0.000791469080457343\\
60.78	0.000791488975970389\\
60.79	0.000791508881388472\\
60.8	0.000791528796719374\\
60.81	0.000791548721970884\\
60.82	0.000791568657150802\\
60.83	0.000791588602266939\\
60.84	0.000791608557327114\\
60.85	0.000791628522339157\\
60.86	0.000791648497310907\\
60.87	0.000791668482250213\\
60.88	0.000791688477164933\\
60.89	0.000791708482062939\\
60.9	0.000791728496952106\\
60.91	0.000791748521840326\\
60.92	0.000791768556735498\\
60.93	0.00079178860164553\\
60.94	0.00079180865657834\\
60.95	0.000791828721541859\\
60.96	0.000791848796544025\\
60.97	0.000791868881592786\\
60.98	0.000791888976696103\\
60.99	0.000791909081861946\\
61	0.000791929197098295\\
61.01	0.000791949322413138\\
61.02	0.000791969457814476\\
61.03	0.00079198960331032\\
61.04	0.000792009758908691\\
61.05	0.000792029924617618\\
61.06	0.000792050100445146\\
61.07	0.000792070286399323\\
61.08	0.000792090482488215\\
61.09	0.000792110688719892\\
61.1	0.000792130905102439\\
61.11	0.000792151131643949\\
61.12	0.000792171368352525\\
61.13	0.000792191615236283\\
61.14	0.000792211872303349\\
61.15	0.000792232139561857\\
61.16	0.000792252417019955\\
61.17	0.000792272704685801\\
61.18	0.000792293002567562\\
61.19	0.000792313310673417\\
61.2	0.000792333629011556\\
61.21	0.000792353957590178\\
61.22	0.000792374296417497\\
61.23	0.000792394645501732\\
61.24	0.000792415004851117\\
61.25	0.000792435374473896\\
61.26	0.000792455754378323\\
61.27	0.000792476144572665\\
61.28	0.000792496545065198\\
61.29	0.000792516955864211\\
61.3	0.000792537376978\\
61.31	0.000792557808414878\\
61.32	0.000792578250183164\\
61.33	0.000792598702291194\\
61.34	0.000792619164747309\\
61.35	0.000792639637559864\\
61.36	0.000792660120737225\\
61.37	0.00079268061428777\\
61.38	0.000792701118219889\\
61.39	0.00079272163254198\\
61.4	0.000792742157262456\\
61.41	0.00079276269238974\\
61.42	0.000792783237932266\\
61.43	0.000792803793898482\\
61.44	0.000792824360296844\\
61.45	0.000792844937135821\\
61.46	0.000792865524423895\\
61.47	0.000792886122169559\\
61.48	0.000792906730381317\\
61.49	0.000792927349067687\\
61.5	0.000792947978237193\\
61.51	0.000792968617898378\\
61.52	0.000792989268059794\\
61.53	0.000793009928730003\\
61.54	0.000793030599917581\\
61.55	0.000793051281631115\\
61.56	0.000793071973879208\\
61.57	0.000793092676670467\\
61.58	0.00079311339001352\\
61.59	0.000793134113917\\
61.6	0.000793154848389557\\
61.61	0.000793175593439851\\
61.62	0.000793196349076555\\
61.63	0.000793217115308354\\
61.64	0.000793237892143944\\
61.65	0.000793258679592037\\
61.66	0.000793279477661354\\
61.67	0.000793300286360632\\
61.68	0.000793321105698616\\
61.69	0.000793341935684068\\
61.7	0.000793362776325759\\
61.71	0.000793383627632475\\
61.72	0.000793404489613014\\
61.73	0.000793425362276187\\
61.74	0.000793446245630818\\
61.75	0.000793467139685743\\
61.76	0.000793488044449812\\
61.77	0.000793508959931888\\
61.78	0.000793529886140844\\
61.79	0.000793550823085571\\
61.8	0.00079357177077497\\
61.81	0.000793592729217956\\
61.82	0.000793613698423457\\
61.83	0.000793634678400414\\
61.84	0.00079365566915778\\
61.85	0.000793676670704527\\
61.86	0.000793697683049634\\
61.87	0.000793718706202096\\
61.88	0.000793739740170921\\
61.89	0.000793760784965132\\
61.9	0.000793781840593765\\
61.91	0.000793802907065868\\
61.92	0.000793823984390506\\
61.93	0.000793845072576753\\
61.94	0.000793866171633703\\
61.95	0.000793887281570458\\
61.96	0.000793908402396138\\
61.97	0.000793929534119876\\
61.98	0.000793950676750817\\
61.99	0.000793971830298123\\
62	0.000793992994770969\\
62.01	0.000794014170178546\\
62.02	0.000794035356530056\\
62.03	0.000794056553834716\\
62.04	0.000794077762101759\\
62.05	0.000794098981340434\\
62.06	0.000794120211560001\\
62.07	0.000794141452769737\\
62.08	0.00079416270497893\\
62.09	0.000794183968196889\\
62.1	0.000794205242432932\\
62.11	0.000794226527696396\\
62.12	0.000794247823996629\\
62.13	0.000794269131342999\\
62.14	0.000794290449744885\\
62.15	0.000794311779211682\\
62.16	0.000794333119752801\\
62.17	0.000794354471377668\\
62.18	0.000794375834095725\\
62.19	0.000794397207916428\\
62.2	0.000794418592849249\\
62.21	0.000794439988903675\\
62.22	0.000794461396089212\\
62.23	0.000794482814415376\\
62.24	0.000794504243891705\\
62.25	0.000794525684527748\\
62.26	0.000794547136333071\\
62.27	0.000794568599317256\\
62.28	0.000794590073489905\\
62.29	0.000794611558860629\\
62.3	0.000794633055439061\\
62.31	0.000794654563234848\\
62.32	0.000794676082257655\\
62.33	0.000794697612517161\\
62.34	0.000794719154023061\\
62.35	0.00079474070678507\\
62.36	0.00079476227081292\\
62.37	0.000794783846116354\\
62.38	0.000794805432705138\\
62.39	0.000794827030589054\\
62.4	0.000794848639777896\\
62.41	0.000794870260281482\\
62.42	0.000794891892109642\\
62.43	0.000794913535272225\\
62.44	0.000794935189779101\\
62.45	0.00079495685564015\\
62.46	0.000794978532865276\\
62.47	0.000795000221464398\\
62.48	0.000795021921447451\\
62.49	0.000795043632824392\\
62.5	0.000795065355605192\\
62.51	0.000795087089799843\\
62.52	0.000795108835418354\\
62.53	0.000795130592470751\\
62.54	0.000795152360967081\\
62.55	0.000795174140917407\\
62.56	0.000795195932331811\\
62.57	0.000795217735220392\\
62.58	0.000795239549593272\\
62.59	0.000795261375460588\\
62.6	0.000795283212832497\\
62.61	0.000795305061719176\\
62.62	0.00079532692213082\\
62.63	0.000795348794077643\\
62.64	0.000795370677569879\\
62.65	0.00079539257261778\\
62.66	0.000795414479231619\\
62.67	0.000795436397421689\\
62.68	0.000795458327198302\\
62.69	0.000795480268571788\\
62.7	0.000795502221552499\\
62.71	0.000795524186150807\\
62.72	0.000795546162377106\\
62.73	0.000795568150241805\\
62.74	0.000795590149755339\\
62.75	0.000795612160928161\\
62.76	0.000795634183770743\\
62.77	0.00079565621829358\\
62.78	0.000795678264507188\\
62.79	0.000795700322422103\\
62.8	0.000795722392048882\\
62.81	0.000795744473398103\\
62.82	0.000795766566480368\\
62.83	0.000795788671306297\\
62.84	0.000795810787886535\\
62.85	0.000795832916231743\\
62.86	0.000795855056352612\\
62.87	0.000795877208259847\\
62.88	0.000795899371964181\\
62.89	0.000795921547476366\\
62.9	0.000795943734807179\\
62.91	0.000795965933967416\\
62.92	0.000795988144967898\\
62.93	0.000796010367819469\\
62.94	0.000796032602532994\\
62.95	0.000796054849119364\\
62.96	0.000796077107589492\\
62.97	0.000796099377954312\\
62.98	0.000796121660224785\\
62.99	0.000796143954411892\\
63	0.000796166260526643\\
63.01	0.000796188578580066\\
63.02	0.000796210908583216\\
63.03	0.000796233250547174\\
63.04	0.000796255604483043\\
63.05	0.00079627797040195\\
63.06	0.000796300348315048\\
63.07	0.000796322738233516\\
63.08	0.000796345140168554\\
63.09	0.000796367554131393\\
63.1	0.000796389980133283\\
63.11	0.000796412418185503\\
63.12	0.00079643486829936\\
63.13	0.000796457330486181\\
63.14	0.000796479804757323\\
63.15	0.000796502291124167\\
63.16	0.000796524789598123\\
63.17	0.000796547300190626\\
63.18	0.000796569822913136\\
63.19	0.000796592357777143\\
63.2	0.000796614904794163\\
63.21	0.000796637463975736\\
63.22	0.000796660035333433\\
63.23	0.000796682618878853\\
63.24	0.000796705214623619\\
63.25	0.000796727822579387\\
63.26	0.000796750442757836\\
63.27	0.000796773075170676\\
63.28	0.000796795719829645\\
63.29	0.00079681837674651\\
63.3	0.000796841045933066\\
63.31	0.000796863727401138\\
63.32	0.000796886421162579\\
63.33	0.000796909127229274\\
63.34	0.000796931845613134\\
63.35	0.000796954576326104\\
63.36	0.000796977319380153\\
63.37	0.000797000074787288\\
63.38	0.000797022842559541\\
63.39	0.000797045622708974\\
63.4	0.000797068415247685\\
63.41	0.000797091220187798\\
63.42	0.000797114037541471\\
63.43	0.000797136867320894\\
63.44	0.000797159709538288\\
63.45	0.000797182564205903\\
63.46	0.000797205431336025\\
63.47	0.000797228310940971\\
63.48	0.000797251203033091\\
63.49	0.000797274107624767\\
63.5	0.000797297024728416\\
63.51	0.000797319954356486\\
63.52	0.000797342896521459\\
63.53	0.000797365851235852\\
63.54	0.000797388818512216\\
63.55	0.000797411798363136\\
63.56	0.000797434790801228\\
63.57	0.000797457795839149\\
63.58	0.000797480813489586\\
63.59	0.000797503843765264\\
63.6	0.000797526886678942\\
63.61	0.000797549942243415\\
63.62	0.000797573010471515\\
63.63	0.000797596091376109\\
63.64	0.000797619184970103\\
63.65	0.000797642291266436\\
63.66	0.000797665410278085\\
63.67	0.000797688542018067\\
63.68	0.000797711686499433\\
63.69	0.000797734843735274\\
63.7	0.000797758013738719\\
63.71	0.000797781196522936\\
63.72	0.000797804392101129\\
63.73	0.000797827600486543\\
63.74	0.000797850821692463\\
63.75	0.000797874055732213\\
63.76	0.000797897302619154\\
63.77	0.00079792056236669\\
63.78	0.000797943834988265\\
63.79	0.000797967120497364\\
63.8	0.000797990418907511\\
63.81	0.000798013730232274\\
63.82	0.00079803705448526\\
63.83	0.00079806039168012\\
63.84	0.000798083741830546\\
63.85	0.000798107104950272\\
63.86	0.000798130481053075\\
63.87	0.000798153870152777\\
63.88	0.000798177272263242\\
63.89	0.000798200687398378\\
63.9	0.000798224115572137\\
63.91	0.000798247556798514\\
63.92	0.000798271011091551\\
63.93	0.000798294478465335\\
63.94	0.000798317958933994\\
63.95	0.00079834145251171\\
63.96	0.000798364959212702\\
63.97	0.000798388479051242\\
63.98	0.000798412012041646\\
63.99	0.000798435558198276\\
64	0.000798459117535545\\
64.01	0.000798482690067912\\
64.02	0.000798506275809883\\
64.03	0.000798529874776011\\
64.04	0.000798553486980904\\
64.05	0.000798577112439212\\
64.06	0.00079860075116564\\
64.07	0.000798624403174941\\
64.08	0.000798648068481917\\
64.09	0.000798671747101423\\
64.1	0.000798695439048364\\
64.11	0.000798719144337696\\
64.12	0.000798742862984427\\
64.13	0.000798766595003617\\
64.14	0.000798790340410379\\
64.15	0.000798814099219881\\
64.16	0.00079883787144734\\
64.17	0.000798861657108029\\
64.18	0.000798885456217277\\
64.19	0.000798909268790464\\
64.2	0.000798933094843029\\
64.21	0.000798956934390463\\
64.22	0.000798980787448313\\
64.23	0.000799004654032183\\
64.24	0.000799028534157736\\
64.25	0.000799052427840688\\
64.26	0.000799076335096814\\
64.27	0.000799100255941949\\
64.28	0.000799124190391982\\
64.29	0.000799148138462862\\
64.3	0.000799172100170602\\
64.31	0.000799196075531267\\
64.32	0.000799220064560988\\
64.33	0.000799244067275953\\
64.34	0.000799268083692411\\
64.35	0.000799292113826676\\
64.36	0.000799316157695119\\
64.37	0.000799340215314175\\
64.38	0.000799364286700343\\
64.39	0.000799388371870184\\
64.4	0.000799412470840323\\
64.41	0.000799436583627447\\
64.42	0.000799460710248312\\
64.43	0.000799484850719734\\
64.44	0.000799509005058596\\
64.45	0.000799533173281849\\
64.46	0.000799557355406509\\
64.47	0.000799581551449657\\
64.48	0.000799605761428445\\
64.49	0.00079962998536009\\
64.5	0.000799654223261878\\
64.51	0.000799678475151165\\
64.52	0.000799702741045372\\
64.53	0.000799727020961996\\
64.54	0.000799751314918599\\
64.55	0.000799775622932817\\
64.56	0.000799799945022356\\
64.57	0.000799824281204993\\
64.58	0.000799848631498578\\
64.59	0.000799872995921035\\
64.6	0.000799897374490358\\
64.61	0.000799921767224617\\
64.62	0.000799946174141957\\
64.63	0.000799970595260595\\
64.64	0.000799995030598825\\
64.65	0.000800019480175017\\
64.66	0.000800043944007617\\
64.67	0.000800068422115146\\
64.68	0.000800092914516205\\
64.69	0.000800117421229471\\
64.7	0.000800141942273701\\
64.71	0.000800166477667728\\
64.72	0.000800191027430468\\
64.73	0.000800215591580914\\
64.74	0.00080024017013814\\
64.75	0.000800264763121302\\
64.76	0.000800289370549637\\
64.77	0.000800313992442462\\
64.78	0.00080033862881918\\
64.79	0.000800363279699274\\
64.8	0.000800387945102314\\
64.81	0.00080041262504795\\
64.82	0.000800437319555918\\
64.83	0.00080046202864604\\
64.84	0.000800486752338226\\
64.85	0.000800511490652467\\
64.86	0.000800536243608842\\
64.87	0.000800561011227522\\
64.88	0.00080058579352876\\
64.89	0.000800610590532899\\
64.9	0.000800635402260372\\
64.91	0.0008006602287317\\
64.92	0.000800685069967496\\
64.93	0.00080070992598846\\
64.94	0.000800734796815385\\
64.95	0.000800759682469157\\
64.96	0.00080078458297075\\
64.97	0.000800809498341232\\
64.98	0.000800834428601765\\
64.99	0.000800859373773605\\
65	0.0008008843338781\\
65.01	0.000800909308936693\\
65.02	0.000800934298970921\\
65.03	0.000800959304002417\\
65.04	0.000800984324052914\\
65.05	0.000801009359144235\\
65.06	0.000801034409298303\\
65.07	0.000801059474537139\\
65.08	0.000801084554882859\\
65.09	0.00080110965035768\\
65.1	0.000801134760983918\\
65.11	0.000801159886783987\\
65.12	0.000801185027780401\\
65.13	0.000801210183995772\\
65.14	0.000801235355452816\\
65.15	0.000801260542174347\\
65.16	0.000801285744183283\\
65.17	0.000801310961502643\\
65.18	0.000801336194155546\\
65.19	0.000801361442165216\\
65.2	0.000801386705554982\\
65.21	0.000801411984348271\\
65.22	0.000801437278568618\\
65.23	0.000801462588239661\\
65.24	0.000801487913385142\\
65.25	0.000801513254028908\\
65.26	0.000801538610194911\\
65.27	0.000801563981907208\\
65.28	0.000801589369189963\\
65.29	0.000801614772067447\\
65.3	0.000801640190564034\\
65.31	0.000801665624704207\\
65.32	0.000801691074512556\\
65.33	0.000801716540013776\\
65.34	0.000801742021232671\\
65.35	0.000801767518194154\\
65.36	0.000801793030923243\\
65.37	0.000801818559445064\\
65.38	0.000801844103784853\\
65.39	0.000801869663967952\\
65.4	0.000801895240019813\\
65.41	0.000801920831965995\\
65.42	0.000801946439832169\\
65.43	0.000801972063644107\\
65.44	0.000801997703427697\\
65.45	0.000802023359208932\\
65.46	0.000802049031013916\\
65.47	0.000802074718868858\\
65.48	0.000802100422800078\\
65.49	0.000802126142834002\\
65.5	0.000802151878997166\\
65.51	0.000802177631316216\\
65.52	0.000802203399817901\\
65.53	0.000802229184529081\\
65.54	0.000802254985476723\\
65.55	0.000802280802687901\\
65.56	0.000802306636189796\\
65.57	0.000802332486009695\\
65.58	0.00080235835217499\\
65.59	0.000802384234713183\\
65.6	0.000802410133651876\\
65.61	0.000802436049018779\\
65.62	0.000802461980841704\\
65.63	0.00080248792914857\\
65.64	0.000802513893967396\\
65.65	0.000802539875326306\\
65.66	0.000802565873253523\\
65.67	0.000802591887777373\\
65.68	0.000802617918926281\\
65.69	0.000802643966728772\\
65.7	0.000802670031213469\\
65.71	0.000802696112409093\\
65.72	0.000802722210344459\\
65.73	0.000802748325048484\\
65.74	0.00080277445655017\\
65.75	0.000802800604878619\\
65.76	0.000802826770063022\\
65.77	0.000802852952132662\\
65.78	0.00080287915111691\\
65.79	0.000802905367045225\\
65.8	0.000802931599947153\\
65.81	0.00080295784985232\\
65.82	0.000802984116790443\\
65.83	0.000803010400791315\\
65.84	0.000803036701884807\\
65.85	0.000803063020100872\\
65.86	0.000803089355469536\\
65.87	0.000803115708020897\\
65.88	0.000803142077785129\\
65.89	0.000803168464792468\\
65.9	0.000803194869073225\\
65.91	0.000803221290657768\\
65.92	0.000803247729576531\\
65.93	0.000803274185860006\\
65.94	0.000803300659538739\\
65.95	0.000803327150643334\\
65.96	0.000803353659204442\\
65.97	0.000803380185252764\\
65.98	0.000803406728819043\\
65.99	0.000803433289934064\\
66	0.000803459868628651\\
66.01	0.000803486464933658\\
66.02	0.000803513078879974\\
66.03	0.000803539710498511\\
66.04	0.000803566359820205\\
66.05	0.00080359302687601\\
66.06	0.000803619711696894\\
66.07	0.000803646414313835\\
66.08	0.000803673134757816\\
66.09	0.000803699873059819\\
66.1	0.000803726629250823\\
66.11	0.000803753403361798\\
66.12	0.000803780195423699\\
66.13	0.000803807005467458\\
66.14	0.000803833833523986\\
66.15	0.000803860679624159\\
66.16	0.000803887543798815\\
66.17	0.00080391442607875\\
66.18	0.000803941326494709\\
66.19	0.000803968245077381\\
66.2	0.00080399518185739\\
66.21	0.000804022136865291\\
66.22	0.000804049110131561\\
66.23	0.00080407610168659\\
66.24	0.000804103111560678\\
66.25	0.00080413013978402\\
66.26	0.000804157186386703\\
66.27	0.000804184251398698\\
66.28	0.000804211334849848\\
66.29	0.00080423843676986\\
66.3	0.000804265557188297\\
66.31	0.000804292696134567\\
66.32	0.000804319853637912\\
66.33	0.0008043470297274\\
66.34	0.000804374224431916\\
66.35	0.000804401437780145\\
66.36	0.000804428669800569\\
66.37	0.000804455920521446\\
66.38	0.000804483189970806\\
66.39	0.000804510478176437\\
66.4	0.000804537785165867\\
66.41	0.00080456511096636\\
66.42	0.000804592455604892\\
66.43	0.000804619819108149\\
66.44	0.000804647201502501\\
66.45	0.000804674602813997\\
66.46	0.000804702023068343\\
66.47	0.000804729462290892\\
66.48	0.000804756920506623\\
66.49	0.000804784397740127\\
66.5	0.000804811894015588\\
66.51	0.000804839409356771\\
66.52	0.000804866943786998\\
66.53	0.00080489449732913\\
66.54	0.000804922070005551\\
66.55	0.000804949661838148\\
66.56	0.000804977272848286\\
66.57	0.000805004903056794\\
66.58	0.000805032552483941\\
66.59	0.000805060221149411\\
66.6	0.000805087909072282\\
66.61	0.000805115616271005\\
66.62	0.000805143342763378\\
66.63	0.000805171088566523\\
66.64	0.000805198853696855\\
66.65	0.000805226638170065\\
66.66	0.000805254442001085\\
66.67	0.000805282265204066\\
66.68	0.000805310107792346\\
66.69	0.000805337969778422\\
66.7	0.000805365851173918\\
66.71	0.000805393751989561\\
66.72	0.000805421672235138\\
66.73	0.000805449611919475\\
66.74	0.000805477571050397\\
66.75	0.000805505549634692\\
66.76	0.000805533547678079\\
66.77	0.000805561565185171\\
66.78	0.000805589602159439\\
66.79	0.000805617658603168\\
66.8	0.000805645734517423\\
66.81	0.000805673829902006\\
66.82	0.000805701944755413\\
66.83	0.000805730079074795\\
66.84	0.000805758232855908\\
66.85	0.000805786406093073\\
66.86	0.000805814598779127\\
66.87	0.000805842810905372\\
66.88	0.000805871042461532\\
66.89	0.0008058992934357\\
66.9	0.000805927563814283\\
66.91	0.00080595585358195\\
66.92	0.000805984162721582\\
66.93	0.000806012491214207\\
66.94	0.000806040839038947\\
66.95	0.00080606920617296\\
66.96	0.000806097592591379\\
66.97	0.000806125998267243\\
66.98	0.000806154423171437\\
66.99	0.000806182867272629\\
67	0.000806211330537196\\
67.01	0.000806239812929157\\
67.02	0.000806268314410099\\
67.03	0.000806296834939111\\
67.04	0.000806325374472694\\
67.05	0.000806353932964701\\
67.06	0.000806382510366244\\
67.07	0.000806411106625613\\
67.08	0.000806439721688201\\
67.09	0.000806468355496411\\
67.1	0.000806497008074848\\
67.11	0.00080652567946334\\
67.12	0.000806554369702072\\
67.13	0.000806583078831588\\
67.14	0.000806611806892798\\
67.15	0.000806640553926982\\
67.16	0.000806669319975795\\
67.17	0.000806698105081277\\
67.18	0.00080672690928585\\
67.19	0.000806755732632333\\
67.2	0.000806784575163937\\
67.21	0.000806813436924278\\
67.22	0.000806842317957381\\
67.23	0.000806871218307687\\
67.24	0.000806900138020053\\
67.25	0.000806929077139766\\
67.26	0.000806958035712544\\
67.27	0.00080698701378454\\
67.28	0.000807016011402355\\
67.29	0.00080704502861304\\
67.3	0.000807074065464101\\
67.31	0.00080710312200351\\
67.32	0.000807132198279707\\
67.33	0.000807161294341609\\
67.34	0.000807190410238615\\
67.35	0.000807219546020617\\
67.36	0.000807248701738001\\
67.37	0.00080727787744166\\
67.38	0.000807307073182993\\
67.39	0.000807336289013923\\
67.4	0.000807365524986897\\
67.41	0.000807394781154892\\
67.42	0.00080742405757143\\
67.43	0.000807453354290577\\
67.44	0.00080748267136696\\
67.45	0.000807512008855767\\
67.46	0.000807541366812756\\
67.47	0.00080757074529427\\
67.48	0.000807600144357235\\
67.49	0.000807629564059179\\
67.5	0.000807659004458229\\
67.51	0.000807688465613132\\
67.52	0.000807717947583253\\
67.53	0.00080774745042859\\
67.54	0.000807776974209778\\
67.55	0.000807806518988109\\
67.56	0.000807836084825524\\
67.57	0.000807865671784638\\
67.58	0.000807895279928741\\
67.59	0.00080792490932181\\
67.6	0.000807954560028521\\
67.61	0.000807984232114254\\
67.62	0.000808013925645107\\
67.63	0.000808043640687903\\
67.64	0.000808073377310205\\
67.65	0.000808103135580322\\
67.66	0.000808132915567321\\
67.67	0.00080816271734104\\
67.68	0.000808192540972094\\
67.69	0.000808222386531894\\
67.7	0.000808252254092647\\
67.71	0.000808282143727379\\
67.72	0.000808312055509939\\
67.73	0.000808341989515014\\
67.74	0.000808371945818141\\
67.75	0.000808401924495716\\
67.76	0.000808431925625011\\
67.77	0.000808461949284179\\
67.78	0.000808491995552277\\
67.79	0.000808522064509272\\
67.8	0.000808552156236054\\
67.81	0.000808582270814449\\
67.82	0.000808612408327239\\
67.83	0.000808642568858162\\
67.84	0.000808672752491942\\
67.85	0.000808702959314291\\
67.86	0.000808733189411928\\
67.87	0.000808763442872592\\
67.88	0.000808793719785057\\
67.89	0.000808824020239145\\
67.9	0.000808854344325746\\
67.91	0.000808884692136826\\
67.92	0.000808915063765448\\
67.93	0.000808945459305785\\
67.94	0.000808975878853136\\
67.95	0.000809006322503945\\
67.96	0.000809036790355809\\
67.97	0.000809067282507507\\
67.98	0.000809097799059005\\
67.99	0.000809128340111478\\
68	0.00080915890576733\\
68.01	0.000809189496130203\\
68.02	0.000809220111305006\\
68.03	0.000809250751397922\\
68.04	0.000809281416516433\\
68.05	0.000809312106769336\\
68.06	0.000809342822266763\\
68.07	0.000809373563120197\\
68.08	0.000809404329442496\\
68.09	0.000809435121347905\\
68.1	0.000809465938952085\\
68.11	0.000809496782372125\\
68.12	0.000809527651726568\\
68.13	0.000809558547135426\\
68.14	0.000809589468720206\\
68.15	0.000809620416603931\\
68.16	0.000809651390911154\\
68.17	0.000809682391767989\\
68.18	0.000809713419302128\\
68.19	0.000809744473642866\\
68.2	0.000809775554921118\\
68.21	0.000809806663269451\\
68.22	0.000809837798822098\\
68.23	0.000809868961714987\\
68.24	0.000809900152085766\\
68.25	0.000809931370073822\\
68.26	0.000809962615820309\\
68.27	0.000809993889468171\\
68.28	0.000810025191162172\\
68.29	0.000810056521048914\\
68.3	0.000810087879276869\\
68.31	0.000810119265996402\\
68.32	0.000810150681359798\\
68.33	0.000810182125521291\\
68.34	0.00081021359863709\\
68.35	0.000810245100865407\\
68.36	0.000810276632366482\\
68.37	0.000810308193302618\\
68.38	0.000810339783838201\\
68.39	0.000810371404139738\\
68.4	0.00081040305437588\\
68.41	0.000810434734717458\\
68.42	0.000810466445337506\\
68.43	0.000810498186411299\\
68.44	0.00081052995811638\\
68.45	0.000810561760632595\\
68.46	0.000810593594142116\\
68.47	0.000810625458829487\\
68.48	0.000810657354881646\\
68.49	0.000810689282487963\\
68.5	0.000810721241840276\\
68.51	0.000810753233132918\\
68.52	0.00081078525656276\\
68.53	0.000810817312329237\\
68.54	0.000810849400634394\\
68.55	0.000810881521682912\\
68.56	0.000810913675682151\\
68.57	0.000810945862842185\\
68.58	0.000810978083375841\\
68.59	0.000811010337498733\\
68.6	0.000811042625429302\\
68.61	0.000811074947388857\\
68.62	0.000811107303601614\\
68.63	0.000811139694294733\\
68.64	0.000811172119698363\\
68.65	0.000811204580045678\\
68.66	0.000811237075572919\\
68.67	0.000811269606519442\\
68.68	0.000811302173127755\\
68.69	0.000811334775643558\\
68.7	0.000811367414315797\\
68.71	0.000811400089396699\\
68.72	0.000811432801141817\\
68.73	0.000811465549810082\\
68.74	0.000811498335663843\\
68.75	0.000811531158968913\\
68.76	0.000811564019994621\\
68.77	0.000811596919013856\\
68.78	0.000811629856303113\\
68.79	0.000811662832142547\\
68.8	0.000811695846816018\\
68.81	0.000811728900611142\\
68.82	0.000811761993819346\\
68.83	0.000811795126735907\\
68.84	0.000811828299660018\\
68.85	0.000811861512894831\\
68.86	0.000811894766747517\\
68.87	0.000811928061529309\\
68.88	0.00081196139755557\\
68.89	0.000811994775145835\\
68.9	0.000812028194623874\\
68.91	0.000812061656317746\\
68.92	0.000812095160559858\\
68.93	0.000812128707687019\\
68.94	0.0008121622980405\\
68.95	0.000812195931966093\\
68.96	0.000812229609814171\\
68.97	0.000812263331939744\\
68.98	0.000812297098702527\\
68.99	0.000812330910466997\\
69	0.000812364767602455\\
69.01	0.00081239867048309\\
69.02	0.000812432619488044\\
69.03	0.000812466615001473\\
69.04	0.000812500657412616\\
69.05	0.000812534747115857\\
69.06	0.000812568884510796\\
69.07	0.000812603070002312\\
69.08	0.00081263730400063\\
69.09	0.000812671586921398\\
69.1	0.000812705919185747\\
69.11	0.000812740301220366\\
69.12	0.000812774733457571\\
69.13	0.000812809216335378\\
69.14	0.000812843750297575\\
69.15	0.000812878335793795\\
69.16	0.00081291297327959\\
69.17	0.000812947663216502\\
69.18	0.000812982406072148\\
69.19	0.000813017202320286\\
69.2	0.000813052052440899\\
69.21	0.000813086956920263\\
69.22	0.000813121916251039\\
69.23	0.000813156930932341\\
69.24	0.000813192001469822\\
69.25	0.000813227128375752\\
69.26	0.000813262312169102\\
69.27	0.000813297553375621\\
69.28	0.000813332852527925\\
69.29	0.000813368210165576\\
69.3	0.00081340362683517\\
69.31	0.000813439103090419\\
69.32	0.00081347463949224\\
69.33	0.000813510236608842\\
69.34	0.000813545895015807\\
69.35	0.000813581615296186\\
69.36	0.000813617398040586\\
69.37	0.000813653243847256\\
69.38	0.000813689153322181\\
69.39	0.00081372512707917\\
69.4	0.000813761165739953\\
69.41	0.000813797269934266\\
69.42	0.000813833440299953\\
69.43	0.000813869677483051\\
69.44	0.00081390598213789\\
69.45	0.00081394235492719\\
69.46	0.000813978796522147\\
69.47	0.000814015307602542\\
69.48	0.000814051888856832\\
69.49	0.000814088540982244\\
69.5	0.000814125264684884\\
69.51	0.000814162060679824\\
69.52	0.00081419892969121\\
69.53	0.000814235872452357\\
69.54	0.000814272889705854\\
69.55	0.000814309982203661\\
69.56	0.000814347150707218\\
69.57	0.000814384395987534\\
69.58	0.000814421718825304\\
69.59	0.000814459120011003\\
69.6	0.000814496600344994\\
69.61	0.00081453416063763\\
69.62	0.000814571801709359\\
69.63	0.000814609524390827\\
69.64	0.000814647329522985\\
69.65	0.000814685217957195\\
69.66	0.000814723190555334\\
69.67	0.0008147612481899\\
69.68	0.000814799391744118\\
69.69	0.000814837622112046\\
69.7	0.000814875940198686\\
69.71	0.000814914346920079\\
69.72	0.000814952843203426\\
69.73	0.000814991429987182\\
69.74	0.000815030108221171\\
69.75	0.000815068878866686\\
69.76	0.000815107742896603\\
69.77	0.000815146701295476\\
69.78	0.000815185755059655\\
69.79	0.000815224905197384\\
69.8	0.000815264152728907\\
69.81	0.000815303498686576\\
69.82	0.000815342944114953\\
69.83	0.000815382490070914\\
69.84	0.000815422137623754\\
69.85	0.000815461887855286\\
69.86	0.000815501741859948\\
69.87	0.000815541700744901\\
69.88	0.00081558176563013\\
69.89	0.000815621937648543\\
69.9	0.000815662217946073\\
69.91	0.000815702607681767\\
69.92	0.000815743108027895\\
69.93	0.00081578372017003\\
69.94	0.000815824445307152\\
69.95	0.000815865284651739\\
69.96	0.000815906239429853\\
69.97	0.00081594731088123\\
69.98	0.000815988500259371\\
69.99	0.000816029808831623\\
70	0.000816071237879263\\
70.01	0.00081611278869758\\
70.02	0.000816154462595954\\
70.03	0.000816196260897928\\
70.04	0.000816238184941288\\
70.05	0.00081628023607813\\
70.06	0.000816322415674932\\
70.07	0.000816364725112612\\
70.08	0.0008164071657866\\
70.09	0.000816449739106889\\
70.1	0.000816492446498096\\
70.11	0.00081653528939951\\
70.12	0.000816578269265143\\
70.13	0.000816621387563773\\
70.14	0.000816664645778982\\
70.15	0.000816708045409197\\
70.16	0.000816751587967714\\
70.17	0.00081679527498273\\
70.18	0.000816839107997363\\
70.19	0.000816883088569666\\
70.2	0.000816927218272643\\
70.21	0.000816971498694247\\
70.22	0.000817015931437381\\
70.23	0.000817060518119896\\
70.24	0.000817105260374569\\
70.25	0.000817150159849095\\
70.26	0.000817195218206045\\
70.27	0.000817240437122842\\
70.28	0.000817285818291713\\
70.29	0.000817331363419643\\
70.3	0.000817377074228314\\
70.31	0.000817422952454039\\
70.32	0.000817468999847686\\
70.33	0.000817515218174593\\
70.34	0.000817561609214472\\
70.35	0.000817608174761305\\
70.36	0.000817654916623231\\
70.37	0.000817701836622416\\
70.38	0.000817748936594915\\
70.39	0.000817796218390529\\
70.4	0.000817843683872639\\
70.41	0.000817891334918035\\
70.42	0.000817939173416726\\
70.43	0.000817987201271742\\
70.44	0.000818035420398917\\
70.45	0.000818083832726666\\
70.46	0.000818132440195735\\
70.47	0.000818181244758942\\
70.48	0.000818230248380904\\
70.49	0.000818279453037743\\
70.5	0.000818328860716772\\
70.51	0.000818378473416174\\
70.52	0.000818428293144649\\
70.53	0.000818478321921046\\
70.54	0.000818528561773984\\
70.55	0.000818579014741437\\
70.56	0.000818629682870308\\
70.57	0.000818680568215978\\
70.58	0.00081873167284183\\
70.59	0.000818782998818752\\
70.6	0.000818834548224612\\
70.61	0.000818886323143713\\
70.62	0.000818938325666211\\
70.63	0.00081899055788752\\
70.64	0.000819043021907672\\
70.65	0.000819095719830666\\
70.66	0.000819148653763769\\
70.67	0.000819201825816797\\
70.68	0.000819255238101361\\
70.69	0.000819308892730075\\
70.7	0.000819362791815737\\
70.71	0.000819416937470466\\
70.72	0.00081947133180481\\
70.73	0.000819525976926805\\
70.74	0.000819580874941005\\
70.75	0.000819636027947468\\
70.76	0.000819691438040691\\
70.77	0.000819747107308517\\
70.78	0.000819803037830981\\
70.79	0.00081985923167912\\
70.8	0.000819915690913732\\
70.81	0.000819972417584077\\
70.82	0.000820029413726544\\
70.83	0.000820086681363245\\
70.84	0.000820144222500565\\
70.85	0.000820202039127657\\
70.86	0.000820260133214869\\
70.87	0.000820318506712122\\
70.88	0.000820377161547214\\
70.89	0.000820436099624063\\
70.9	0.000820495322820891\\
70.91	0.000820554832988327\\
70.92	0.00082061463194745\\
70.93	0.000820674721487745\\
70.94	0.000820735103364998\\
70.95	0.000820795779299104\\
70.96	0.000820856750971792\\
70.97	0.000820918020024274\\
70.98	0.000820979588054806\\
70.99	0.000821041456616151\\
71	0.000821103627212964\\
71.01	0.000821166101299073\\
71.02	0.000821228880274666\\
71.03	0.00082129196548337\\
71.04	0.000821355358209243\\
71.05	0.000821419059673634\\
71.06	0.000821483071031956\\
71.07	0.000821547393370325\\
71.08	0.000821612027702099\\
71.09	0.000821676974964278\\
71.1	0.00082174223601379\\
71.11	0.000821807811623644\\
71.12	0.000821873702478946\\
71.13	0.000821939909172778\\
71.14	0.000822006432201944\\
71.15	0.000822073271962556\\
71.16	0.000822140428745472\\
71.17	0.000822207902731584\\
71.18	0.000822275693986936\\
71.19	0.000822343802457683\\
71.2	0.000822412227964876\\
71.21	0.000822480970199066\\
71.22	0.00082255002871473\\
71.23	0.000822619402924513\\
71.24	0.000822689092093262\\
71.25	0.000822759095331874\\
71.26	0.00082282941159093\\
71.27	0.000822900039654122\\
71.28	0.000822970978131448\\
71.29	0.000823042225452196\\
71.3	0.000823113779857687\\
71.31	0.000823185639393775\\
71.32	0.000823257801903102\\
71.33	0.000823330265017097\\
71.34	0.000823403026147709\\
71.35	0.000823476082478873\\
71.36	0.000823549430957684\\
71.37	0.000823623068285298\\
71.38	0.000823696990907516\\
71.39	0.000823771195005077\\
71.4	0.000823845676483624\\
71.41	0.000823920430963344\\
71.42	0.000823995453768274\\
71.43	0.000824070739915255\\
71.44	0.000824146284102538\\
71.45	0.000824222080697998\\
71.46	0.000824298123726995\\
71.47	0.000824374406859823\\
71.48	0.000824450923398759\\
71.49	0.000824527666264698\\
71.5	0.000824604627983352\\
71.51	0.000824681800671015\\
71.52	0.000824759176019851\\
71.53	0.000824836745282741\\
71.54	0.00082491449925761\\
71.55	0.000824992428271285\\
71.56	0.000825070522162807\\
71.57	0.000825148770266236\\
71.58	0.000825227162350679\\
71.59	0.000825305700266385\\
71.6	0.00082538438589441\\
71.61	0.000825463221147165\\
71.62	0.000825542207968964\\
71.63	0.000825621348336582\\
71.64	0.000825700644259833\\
71.65	0.000825780097782143\\
71.66	0.000825859710981151\\
71.67	0.000825939485969302\\
71.68	0.000826019424894475\\
71.69	0.000826099529940595\\
71.7	0.000826179803328284\\
71.71	0.000826260247315502\\
71.72	0.000826340864198213\\
71.73	0.00082642165631106\\
71.74	0.000826502626028049\\
71.75	0.000826583775763249\\
71.76	0.000826665107971513\\
71.77	0.000826746625149196\\
71.78	0.000826828329834899\\
71.79	0.000826910224610223\\
71.8	0.00082699231210054\\
71.81	0.000827074594975776\\
71.82	0.000827157075951206\\
71.83	0.00082723975778827\\
71.84	0.000827322643295405\\
71.85	0.000827405735328886\\
71.86	0.000827489036793689\\
71.87	0.000827572550644364\\
71.88	0.00082765627988594\\
71.89	0.000827740227574829\\
71.9	0.000827824396819754\\
71.91	0.000827908790782698\\
71.92	0.000827993412679877\\
71.93	0.000828078265782711\\
71.94	0.000828163353418836\\
71.95	0.000828248678973124\\
71.96	0.000828334245888728\\
71.97	0.000828420057668139\\
71.98	0.000828506117874278\\
71.99	0.000828592430131591\\
72	0.000828678998127183\\
72.01	0.000828765825611963\\
72.02	0.000828852916401815\\
72.03	0.000828940274378792\\
72.04	0.000829027903492336\\
72.05	0.000829115807760515\\
72.06	0.000829203991271291\\
72.07	0.000829292458183814\\
72.08	0.000829381212729732\\
72.09	0.000829470259214538\\
72.1	0.000829559602018939\\
72.11	0.000829649245600252\\
72.12	0.000829739194493832\\
72.13	0.000829829453314519\\
72.14	0.000829920026758121\\
72.15	0.000830010919602935\\
72.16	0.000830102136711273\\
72.17	0.000830193683031055\\
72.18	0.000830285563597396\\
72.19	0.000830377783534255\\
72.2	0.000830470348056101\\
72.21	0.00083056326246962\\
72.22	0.000830656532175454\\
72.23	0.000830750162669973\\
72.24	0.000830844159547088\\
72.25	0.000830938528500101\\
72.26	0.000831033275323587\\
72.27	0.000831128405915321\\
72.28	0.000831223926278244\\
72.29	0.000831319842522462\\
72.3	0.000831416160867298\\
72.31	0.000831512887643377\\
72.32	0.000831610029294759\\
72.33	0.000831707592381117\\
72.34	0.000831805583579955\\
72.35	0.000831904009688874\\
72.36	0.000832002877627896\\
72.37	0.00083210219444182\\
72.38	0.000832201967302635\\
72.39	0.00083230220351199\\
72.4	0.000832402910503707\\
72.41	0.000832504095846348\\
72.42	0.000832605767245842\\
72.43	0.000832707932548164\\
72.44	0.00083281059974207\\
72.45	0.000832913776961893\\
72.46	0.000833017472490399\\
72.47	0.000833121694761699\\
72.48	0.000833226452364231\\
72.49	0.000833331754043799\\
72.5	0.000833437608706685\\
72.51	0.00083354402542282\\
72.52	0.000833651013429033\\
72.53	0.000833758582132362\\
72.54	0.000833866741113439\\
72.55	0.000833975500129954\\
72.56	0.000834084869120191\\
72.57	0.000834194858206641\\
72.58	0.000834305477699693\\
72.59	0.000834416738101413\\
72.6	0.000834528650109398\\
72.61	0.00083464122462072\\
72.62	0.000834754472735962\\
72.63	0.000834868405763326\\
72.64	0.000834983035222862\\
72.65	0.00083509837285076\\
72.66	0.000835214430603764\\
72.67	0.000835331220663663\\
72.68	0.000835448755441904\\
72.69	0.000835567047584298\\
72.7	0.000835686109975829\\
72.71	0.00083580595574558\\
72.72	0.00083592659827177\\
72.73	0.000836048051186899\\
72.74	0.000836170328383026\\
72.75	0.000836293444017141\\
72.76	0.00083641741251664\\
72.77	0.000836542248584919\\
72.78	0.000836667967207093\\
72.79	0.000836794583655842\\
72.8	0.000836922113497391\\
72.81	0.000837050572597623\\
72.82	0.000837179977128335\\
72.83	0.00083731034357363\\
72.84	0.000837441688736456\\
72.85	0.000837574029745297\\
72.86	0.000837707384061014\\
72.87	0.000837841769483838\\
72.88	0.000837977204160532\\
72.89	0.000838113706591712\\
72.9	0.000838251295639331\\
72.91	0.000838389990534347\\
72.92	0.00083852981088456\\
72.93	0.000838670776682626\\
72.94	0.000838812908314267\\
72.95	0.000838956226566663\\
72.96	0.000839100752637042\\
72.97	0.000839246508141466\\
72.98	0.000839393515123832\\
72.99	0.000839540901050709\\
73	0.000839688399544802\\
73.01	0.000839836010926244\\
73.02	0.000839983735514726\\
73.03	0.000840131573629426\\
73.04	0.000840279525588952\\
73.05	0.000840427591711271\\
73.06	0.000840575772313652\\
73.07	0.000840724067712594\\
73.08	0.000840872478223762\\
73.09	0.000841021004161925\\
73.1	0.000841169645840882\\
73.11	0.000841318403573396\\
73.12	0.000841467277671129\\
73.13	0.000841616268444574\\
73.14	0.000841765376202981\\
73.15	0.000841914601254294\\
73.16	0.000842063943905078\\
73.17	0.000842213404460454\\
73.18	0.000842362983224029\\
73.19	0.000842512680497827\\
73.2	0.000842662496582221\\
73.21	0.000842812431775867\\
73.22	0.000842962486375637\\
73.23	0.000843112660676548\\
73.24	0.000843262954971705\\
73.25	0.00084341336955223\\
73.26	0.0008435639047072\\
73.27	0.000843714560723586\\
73.28	0.000843865337886194\\
73.29	0.000844016236477604\\
73.3	0.000844167256778114\\
73.31	0.000844318399065687\\
73.32	0.000844469663615892\\
73.33	0.000844621050701863\\
73.34	0.000844772560594247\\
73.35	0.000844924193561159\\
73.36	0.000845075949868143\\
73.37	0.000845227829778134\\
73.38	0.000845379833551424\\
73.39	0.000845531961445635\\
73.4	0.00084568421371569\\
73.41	0.000845836590613798\\
73.42	0.000845989092389433\\
73.43	0.000846141719289326\\
73.44	0.000846294471557468\\
73.45	0.000846447349435104\\
73.46	0.000846600353160748\\
73.47	0.000846753482970197\\
73.48	0.000846906739096554\\
73.49	0.000847060121770268\\
73.5	0.000847213631219163\\
73.51	0.0008473672676685\\
73.52	0.000847521031341029\\
73.53	0.000847674922457062\\
73.54	0.000847828941234556\\
73.55	0.000847983087889203\\
73.56	0.000848137362634536\\
73.57	0.000848291765682047\\
73.58	0.000848446297241319\\
73.59	0.000848600957520172\\
73.6	0.000848755746724824\\
73.61	0.000848910665060066\\
73.62	0.00084906571272946\\
73.63	0.000849220889935552\\
73.64	0.000849376196880094\\
73.65	0.000849531633764307\\
73.66	0.000849687200789141\\
73.67	0.000849842898155572\\
73.68	0.000849998726064923\\
73.69	0.000850154684719195\\
73.7	0.00085031077432144\\
73.71	0.000850466995076149\\
73.72	0.000850623347189673\\
73.73	0.000850779830870669\\
73.74	0.000850936446330582\\
73.75	0.000851093193784161\\
73.76	0.000851250073449996\\
73.77	0.000851407085551103\\
73.78	0.000851564230315544\\
73.79	0.000851721507977079\\
73.8	0.000851878918775867\\
73.81	0.000852036462959201\\
73.82	0.000852194140782296\\
73.83	0.000852351952509118\\
73.84	0.00085250989841326\\
73.85	0.000852667978778875\\
73.86	0.000852826193901658\\
73.87	0.00085298454408988\\
73.88	0.000853143029665493\\
73.89	0.000853301650965277\\
73.9	0.000853460408342069\\
73.91	0.000853619302166039\\
73.92	0.000853778332826051\\
73.93	0.000853937500731093\\
73.94	0.000854096806311771\\
73.95	0.000854256250021895\\
73.96	0.000854415832340142\\
73.97	0.000854575553771801\\
73.98	0.000854735414850609\\
73.99	0.000854895416140684\\
74	0.00085505555823855\\
74.01	0.000855215841775262\\
74.02	0.000855376267418641\\
74.03	0.000855536835875617\\
74.04	0.000855697547894683\\
74.05	0.00085585840426847\\
74.06	0.000856019405836449\\
74.07	0.000856180553487759\\
74.08	0.000856341848164167\\
74.09	0.000856503290863175\\
74.1	0.00085666488260152\\
74.11	0.000856826624411878\\
74.12	0.00085698851734318\\
74.13	0.000857150562460919\\
74.14	0.000857312760847478\\
74.15	0.000857475113602452\\
74.16	0.000857637621842995\\
74.17	0.000857800286704158\\
74.18	0.000857963109339239\\
74.19	0.000858126090920152\\
74.2	0.000858289232637789\\
74.21	0.000858452535702403\\
74.22	0.00085861600134399\\
74.23	0.000858779630812689\\
74.24	0.000858943425379181\\
74.25	0.000859107386335108\\
74.26	0.000859271514993497\\
74.27	0.000859435812689196\\
74.28	0.000859600280779312\\
74.29	0.000859764920643671\\
74.3	0.000859929733685288\\
74.31	0.000860094721330838\\
74.32	0.000860259885031144\\
74.33	0.000860425226261688\\
74.34	0.000860590746523112\\
74.35	0.000860756447341751\\
74.36	0.000860922330270169\\
74.37	0.000861088396887711\\
74.38	0.000861254648801069\\
74.39	0.000861421087644858\\
74.4	0.000861587715082213\\
74.41	0.000861754532805398\\
74.42	0.000861921542536425\\
74.43	0.000862088746027697\\
74.44	0.00086225614506266\\
74.45	0.000862423741456481\\
74.46	0.000862591537056723\\
74.47	0.000862759533744069\\
74.48	0.000862927733433031\\
74.49	0.000863096138072698\\
74.5	0.000863264749647499\\
74.51	0.00086343357017798\\
74.52	0.000863602601721608\\
74.53	0.00086377184637359\\
74.54	0.000863941306267716\\
74.55	0.000864110983577226\\
74.56	0.000864280880515694\\
74.57	0.000864450999337936\\
74.58	0.000864621342340954\\
74.59	0.000864791911864881\\
74.6	0.000864962710293975\\
74.61	0.000865133740057622\\
74.62	0.000865305003631375\\
74.63	0.000865476503538017\\
74.64	0.000865648242348652\\
74.65	0.000865820222683828\\
74.66	0.000865992447214685\\
74.67	0.00086616491866414\\
74.68	0.000866337639808097\\
74.69	0.000866510613476694\\
74.7	0.000866683842555584\\
74.71	0.00086685732998725\\
74.72	0.000867031078772347\\
74.73	0.0008672050919711\\
74.74	0.000867379372704723\\
74.75	0.000867553924156878\\
74.76	0.000867728749575185\\
74.77	0.000867903852272764\\
74.78	0.000868079235629822\\
74.79	0.00086825490309528\\
74.8	0.000868430858188456\\
74.81	0.000868607059687372\\
74.82	0.000868783397481305\\
74.83	0.000868959872187233\\
74.84	0.000869136484429952\\
74.85	0.000869313234842189\\
74.86	0.000869490124064726\\
74.87	0.000869667152746527\\
74.88	0.000869844321544862\\
74.89	0.000870021631125437\\
74.9	0.000870199082162524\\
74.91	0.000870376675339093\\
74.92	0.00087055441134695\\
74.93	0.000870732290886871\\
74.94	0.00087091031466874\\
74.95	0.000871088483411695\\
74.96	0.000871266797844269\\
74.97	0.000871445258704537\\
74.98	0.000871623866740263\\
74.99	0.000871802622709056\\
75	0.000871981527378517\\
75.01	0.000872160581526399\\
75.02	0.000872339785940765\\
75.03	0.00087251914142015\\
75.04	0.00087269864877372\\
75.05	0.000872878308821446\\
75.06	0.00087305812239427\\
75.07	0.000873238090334271\\
75.08	0.000873418213494853\\
75.09	0.000873598492740909\\
75.1	0.00087377892894901\\
75.11	0.000873959523007587\\
75.12	0.000874140275817115\\
75.13	0.000874321188290306\\
75.14	0.000874502261352302\\
75.15	0.000874683495940866\\
75.16	0.000874864893006588\\
75.17	0.000875046453513084\\
75.18	0.000875228178437201\\
75.19	0.000875410068769228\\
75.2	0.00087559212551311\\
75.21	0.000875774349686663\\
75.22	0.000875956742321791\\
75.23	0.000876139304464715\\
75.24	0.000876322037176198\\
75.25	0.000876504941531775\\
75.26	0.000876688018621986\\
75.27	0.000876871269552625\\
75.28	0.000877054695444966\\
75.29	0.000877238297436028\\
75.3	0.000877422076678814\\
75.31	0.000877606034342567\\
75.32	0.000877790171613036\\
75.33	0.000877974489692736\\
75.34	0.000878158989801215\\
75.35	0.000878343673175328\\
75.36	0.000878528541069514\\
75.37	0.000878713594756081\\
75.38	0.000878898835525487\\
75.39	0.000879084264686638\\
75.4	0.00087926988356718\\
75.41	0.000879455693513801\\
75.42	0.000879641695892542\\
75.43	0.0008798278920891\\
75.44	0.000880014283509157\\
75.45	0.000880200871578691\\
75.46	0.000880387657744309\\
75.47	0.000880574643473582\\
75.48	0.000880761830255383\\
75.49	0.000880949219600229\\
75.5	0.000881136813040638\\
75.51	0.000881324612131475\\
75.52	0.000881512618450328\\
75.53	0.00088170083359787\\
75.54	0.000881889259198233\\
75.55	0.00088207789689939\\
75.56	0.000882266748373549\\
75.57	0.00088245581531754\\
75.58	0.000882645099453223\\
75.59	0.000882834602527893\\
75.6	0.000883024326314697\\
75.61	0.000883214272613057\\
75.62	0.000883404443249098\\
75.63	0.000883594840076087\\
75.64	0.000883785464974876\\
75.65	0.00088397631985436\\
75.66	0.000884167406651931\\
75.67	0.00088435872733395\\
75.68	0.000884550283896222\\
75.69	0.000884742078364481\\
75.7	0.000884934112794883\\
75.71	0.00088512638927451\\
75.72	0.000885318909921879\\
75.73	0.000885511676887458\\
75.74	0.000885704692354199\\
75.75	0.000885897958538074\\
75.76	0.000886091477688622\\
75.77	0.000886285252089504\\
75.78	0.000886479284059074\\
75.79	0.000886673575950946\\
75.8	0.000886868130154593\\
75.81	0.000887062949095931\\
75.82	0.000887258035237934\\
75.83	0.000887453391081251\\
75.84	0.000887649019164828\\
75.85	0.000887844922066559\\
75.86	0.000888041102403925\\
75.87	0.000888237562834664\\
75.88	0.000888434306057445\\
75.89	0.00088863133481255\\
75.9	0.000888828651882575\\
75.91	0.000889026260093141\\
75.92	0.000889224162313615\\
75.93	0.000889422361457847\\
75.94	0.000889620860484922\\
75.95	0.00088981966239992\\
75.96	0.000890018770254689\\
75.97	0.000890218187148644\\
75.98	0.000890417916229562\\
75.99	0.000890617960694404\\
76	0.000890818323790151\\
76.01	0.000891019008814649\\
76.02	0.000891220019117471\\
76.03	0.000891421358100802\\
76.04	0.000891623029220327\\
76.05	0.000891825035986145\\
76.06	0.000892027381963699\\
76.07	0.000892230070774715\\
76.08	0.000892433106098166\\
76.09	0.000892636491671255\\
76.1	0.000892840231290399\\
76.11	0.000893044328812258\\
76.12	0.000893248788154758\\
76.13	0.000893453613298149\\
76.14	0.00089365880828607\\
76.15	0.000893864377226647\\
76.16	0.000894070324293596\\
76.17	0.000894276653727358\\
76.18	0.00089448336983625\\
76.19	0.000894690476997638\\
76.2	0.000894897979659129\\
76.21	0.000895105882339788\\
76.22	0.000895314189631381\\
76.23	0.00089552290619963\\
76.24	0.0008957320367855\\
76.25	0.000895941586206509\\
76.26	0.000896151559358059\\
76.27	0.000896361961214795\\
76.28	0.000896572796831984\\
76.29	0.00089678407134693\\
76.3	0.0008969957899804\\
76.31	0.000897207958038094\\
76.32	0.000897420580912127\\
76.33	0.000897633664082548\\
76.34	0.000897847213118884\\
76.35	0.000898061233681713\\
76.36	0.000898275731524272\\
76.37	0.000898490712494083\\
76.38	0.000898706182534622\\
76.39	0.000898922147687015\\
76.4	0.000899138614091761\\
76.41	0.000899355587990494\\
76.42	0.000899573075727781\\
76.43	0.000899791083752939\\
76.44	0.000900009618621911\\
76.45	0.00090022868699915\\
76.46	0.000900448295659561\\
76.47	0.000900668451490467\\
76.48	0.000900889161493619\\
76.49	0.000901110432787242\\
76.5	0.00090133227260812\\
76.51	0.000901554688313719\\
76.52	0.000901777687384361\\
76.53	0.000902001277425422\\
76.54	0.000902225466169589\\
76.55	0.00090245026147915\\
76.56	0.000902675671348333\\
76.57	0.000902901703905692\\
76.58	0.000903128367416531\\
76.59	0.000903355670285383\\
76.6	0.000903583621058534\\
76.61	0.000903812228426599\\
76.62	0.000904041501227141\\
76.63	0.000904271448447349\\
76.64	0.00090450207922676\\
76.65	0.000904733402860049\\
76.66	0.000904965428799854\\
76.67	0.00090519816665967\\
76.68	0.000905431626216795\\
76.69	0.000905665817415338\\
76.7	0.000905900750369282\\
76.71	0.000906136435365603\\
76.72	0.000906372882867467\\
76.73	0.000906610103517471\\
76.74	0.000906848108140958\\
76.75	0.0009070869077494\\
76.76	0.000907326513543841\\
76.77	0.00090756693691842\\
76.78	0.000907808189463944\\
76.79	0.000908050282971558\\
76.8	0.000908293229436467\\
76.81	0.000908537041061743\\
76.82	0.000908781730262209\\
76.83	0.000909027309668397\\
76.84	0.000909273792130582\\
76.85	0.000909521190722915\\
76.86	0.000909769518747612\\
76.87	0.000910018789739255\\
76.88	0.000910269017469158\\
76.89	0.000910520215949839\\
76.9	0.000910772399439572\\
76.91	0.000911025582447033\\
76.92	0.000911279779736042\\
76.93	0.00091153500633041\\
76.94	0.000911791277518866\\
76.95	0.000912048608860105\\
76.96	0.000912307016187929\\
76.97	0.000912566515616492\\
76.98	0.000912827123545661\\
76.99	0.000913088856666477\\
77	0.00091335173196674\\
77.01	0.000913615766736705\\
77.02	0.000913880978574887\\
77.03	0.000914147385394\\
77.04	0.00091441500542702\\
77.05	0.000914683857233353\\
77.06	0.00091495395970516\\
77.07	0.000915225332073796\\
77.08	0.000915497993916386\\
77.09	0.000915771965162545\\
77.1	0.000916047266101229\\
77.11	0.000916323917387746\\
77.12	0.000916601940050894\\
77.13	0.000916881355500268\\
77.14	0.000917162185533707\\
77.15	0.000917444452344913\\
77.16	0.000917728178531215\\
77.17	0.000918013387101511\\
77.18	0.000918300101484373\\
77.19	0.000918588345536327\\
77.2	0.000918878143550303\\
77.21	0.000919169520264279\\
77.22	0.000919462500870095\\
77.23	0.000919757111022467\\
77.24	0.00092005337684819\\
77.25	0.000920351324955535\\
77.26	0.000920650982443854\\
77.27	0.000920952376913399\\
77.28	0.000921255536475336\\
77.29	0.000921560489761988\\
77.3	0.000921867265937298\\
77.31	0.00092217589470751\\
77.32	0.00092248640633206\\
77.33	0.000922798831634711\\
77.34	0.000923113202014924\\
77.35	0.000923429549459471\\
77.36	0.000923747906554313\\
77.37	0.000924068306496726\\
77.38	0.000924390783107692\\
77.39	0.000924715370844566\\
77.4	0.000925042104814014\\
77.41	0.000925371020785236\\
77.42	0.000925702155203482\\
77.43	0.000926035545203862\\
77.44	0.000926371228625452\\
77.45	0.00092670924402573\\
77.46	0.000927049630695308\\
77.47	0.000927392428673006\\
77.48	0.000927737678761248\\
77.49	0.000928085422541807\\
77.5	0.000928435702391896\\
77.51	0.000928788561500611\\
77.52	0.000929144043885754\\
77.53	0.000929502194411014\\
77.54	0.000929863058803541\\
77.55	0.000930226683671917\\
77.56	0.000930593116524509\\
77.57	0.000930962405788261\\
77.58	0.000931334600827886\\
77.59	0.0009317097519655\\
77.6	0.000932087910500692\\
77.61	0.000932469128731054\\
77.62	0.000932853299300599\\
77.63	0.00093323764936913\\
77.64	0.000933622178979347\\
77.65	0.000934006888177354\\
77.66	0.000934391777012806\\
77.67	0.000934776845539071\\
77.68	0.000935162093813382\\
77.69	0.000935547521897008\\
77.7	0.000935933129855411\\
77.71	0.000936318917758425\\
77.72	0.000936704885680423\\
77.73	0.0009370910337005\\
77.74	0.000937477361902654\\
77.75	0.00093786387037597\\
77.76	0.000938250559214811\\
77.77	0.000938637428519016\\
77.78	0.000939024478394088\\
77.79	0.000939411708951409\\
77.8	0.000939799120308438\\
77.81	0.000940186712588924\\
77.82	0.000940574485923119\\
77.83	0.000940962440448001\\
77.84	0.000941350576307494\\
77.85	0.000941738893652696\\
77.86	0.000942127392642112\\
77.87	0.000942516073441887\\
77.88	0.000942904936226049\\
77.89	0.000943293981176754\\
77.9	0.000943683208484531\\
77.91	0.000944072618348538\\
77.92	0.000944462210976819\\
77.93	0.000944851986586564\\
77.94	0.000945241945404376\\
77.95	0.000945632087666539\\
77.96	0.000946022413619295\\
77.97	0.000946412923519114\\
77.98	0.000946803617632986\\
77.99	0.000947194496238701\\
78	0.000947585559625139\\
78.01	0.000947976808092561\\
78.02	0.000948368241952914\\
78.03	0.000948759861530122\\
78.04	0.000949151667160395\\
78.05	0.00094954365919254\\
78.06	0.000949935837988261\\
78.07	0.000950328203922487\\
78.08	0.000950720757383675\\
78.09	0.00095111349877414\\
78.1	0.000951506428510369\\
78.11	0.000951899547023355\\
78.12	0.000952292854758913\\
78.13	0.000952686352178019\\
78.14	0.000953080039757134\\
78.15	0.000953473917988537\\
78.16	0.000953867987380662\\
78.17	0.000954262248458428\\
78.18	0.000954656701763578\\
78.19	0.000955051347855011\\
78.2	0.000955446187309117\\
78.21	0.000955841220720116\\
78.22	0.00095623644870039\\
78.23	0.000956631871880814\\
78.24	0.000957027490911085\\
78.25	0.000957423306460065\\
78.26	0.000957819319216086\\
78.27	0.000958215529887294\\
78.28	0.000958611939201956\\
78.29	0.000959008547908778\\
78.3	0.000959405356777221\\
78.31	0.000959802366597796\\
78.32	0.000960199578182374\\
78.33	0.000960596992364469\\
78.34	0.000960994609999528\\
78.35	0.000961392431965202\\
78.36	0.000961790459161616\\
78.37	0.000962188692511622\\
78.38	0.000962587132961048\\
78.39	0.000962985781478925\\
78.4	0.000963384639057711\\
78.41	0.000963783706713493\\
78.42	0.000964182985486176\\
78.43	0.000964582476439662\\
78.44	0.000964982180661997\\
78.45	0.00096538209926551\\
78.46	0.000965782233386931\\
78.47	0.000966182584187481\\
78.48	0.00096658315285294\\
78.49	0.000966983940593696\\
78.5	0.000967384948644758\\
78.51	0.000967786178265748\\
78.52	0.000968187630740854\\
78.53	0.000968589307378765\\
78.54	0.000968991209512557\\
78.55	0.000969393338499545\\
78.56	0.000969795695721108\\
78.57	0.000970198282582456\\
78.58	0.000970601100512369\\
78.59	0.000971004150962879\\
78.6	0.000971407435408911\\
78.61	0.000971810955347867\\
78.62	0.00097221471229916\\
78.63	0.000972618707803689\\
78.64	0.000973022943423257\\
78.65	0.000973427420739918\\
78.66	0.000973832141355265\\
78.67	0.000974237106889646\\
78.68	0.000974642318981301\\
78.69	0.000975047779285428\\
78.7	0.000975453489473159\\
78.71	0.000975859451230462\\
78.72	0.000976265666256936\\
78.73	0.000976672136264525\\
78.74	0.000977078862976124\\
78.75	0.000977485848124083\\
78.76	0.000977893093448604\\
78.77	0.00097830060069601\\
78.78	0.000978708371616914\\
78.79	0.000979116407964237\\
78.8	0.000979524711491112\\
78.81	0.000979933283948634\\
78.82	0.00098034212708348\\
78.83	0.000980751242635358\\
78.84	0.00098116063233431\\
78.85	0.000981570297897836\\
78.86	0.000981980241027853\\
78.87	0.000982390463407458\\
78.88	0.000982800966697514\\
78.89	0.000983211752533023\\
78.9	0.000983622822519287\\
78.91	0.00098403417822786\\
78.92	0.000984445821192266\\
78.93	0.000984857752903469\\
78.94	0.000985269974805109\\
78.95	0.000985682488288451\\
78.96	0.000986095294687088\\
78.97	0.000986508395271337\\
78.98	0.000986921791242349\\
78.99	0.000987335483725902\\
79	0.000987749473837981\\
79.01	0.000988163762707569\\
79.02	0.000988578351476817\\
79.03	0.000988993241301223\\
79.04	0.000989408433349805\\
79.05	0.000989823928805285\\
79.06	0.000990239728864271\\
79.07	0.000990655834737439\\
79.08	0.000991072247649718\\
79.09	0.000991488968840479\\
79.1	0.000991905999563726\\
79.11	0.000992323341088289\\
79.12	0.000992740994698016\\
79.13	0.000993158961691968\\
79.14	0.000993577243384624\\
79.15	0.000993995841106073\\
79.16	0.000994414756202221\\
79.17	0.000994833990034999\\
79.18	0.000995253543982564\\
79.19	0.00099567341943951\\
79.2	0.00099609361781708\\
79.21	0.000996514140543381\\
79.22	0.000996934989063595\\
79.23	0.000997356164840198\\
79.24	0.000997777669353186\\
79.25	0.000998199504100286\\
79.26	0.000998621670597185\\
79.27	0.000999044170377762\\
79.28	0.000999467004994302\\
79.29	0.000999890176017744\\
79.3	0.0010003136850379\\
79.31	0.00100073753366368\\
79.32	0.00100116172352337\\
79.33	0.00100158625626481\\
79.34	0.0010020111335557\\
79.35	0.00100243635708378\\
79.36	0.00100286192855713\\
79.37	0.00100328784970436\\
79.38	0.00100371412227493\\
79.39	0.00100414074803931\\
79.4	0.00100456772878933\\
79.41	0.00100499506633835\\
79.42	0.00100542276252158\\
79.43	0.0010058508191963\\
79.44	0.00100627923824212\\
79.45	0.00100670802156126\\
79.46	0.00100713717107882\\
79.47	0.00100756668874302\\
79.48	0.00100799657652546\\
79.49	0.00100842683642144\\
79.5	0.00100885747045018\\
79.51	0.00100928848065509\\
79.52	0.0010097198691041\\
79.53	0.00101015163788986\\
79.54	0.00101058378913006\\
79.55	0.0010110163249677\\
79.56	0.00101144924757138\\
79.57	0.00101188255913555\\
79.58	0.00101231626188079\\
79.59	0.00101275035805415\\
79.6	0.00101318484992935\\
79.61	0.00101361973980713\\
79.62	0.00101405503001548\\
79.63	0.00101449072290999\\
79.64	0.00101492682087406\\
79.65	0.00101536332631925\\
79.66	0.00101580024168553\\
79.67	0.00101623756944157\\
79.68	0.00101667531208505\\
79.69	0.00101711347214291\\
79.7	0.00101755205217167\\
79.71	0.0010179910547577\\
79.72	0.00101843048251751\\
79.73	0.00101887033809804\\
79.74	0.00101931062417692\\
79.75	0.0010197513434628\\
79.76	0.00102019249869559\\
79.77	0.00102063409264676\\
79.78	0.00102107612811964\\
79.79	0.00102151860794967\\
79.8	0.00102196153500466\\
79.81	0.00102240491218515\\
79.82	0.00102284874242458\\
79.83	0.00102329302868963\\
79.84	0.00102373777398048\\
79.85	0.00102418298133103\\
79.86	0.00102462865380925\\
79.87	0.00102507479451735\\
79.88	0.0010255214065921\\
79.89	0.00102596849320504\\
79.9	0.00102641605756278\\
79.91	0.00102686410290719\\
79.92	0.00102731263251568\\
79.93	0.00102776164970142\\
79.94	0.00102821115781357\\
79.95	0.00102866116023751\\
79.96	0.00102911166039507\\
79.97	0.00102956266174472\\
79.98	0.00103001416778179\\
79.99	0.00103046618203866\\
80	0.001030918708085\\
80.01	0.00103137174952786\\
};
\addplot [color=black,solid]
  table[row sep=crcr]{%
80.01	0.00103137174952786\\
80.02	0.00103182531001195\\
80.03	0.00103227939321974\\
80.04	0.00103273400287165\\
80.05	0.00103318914272617\\
80.06	0.00103364481658007\\
80.07	0.00103410102826842\\
80.08	0.00103455778166481\\
80.09	0.00103501508068141\\
80.1	0.00103547292926907\\
80.11	0.00103593133141739\\
80.12	0.00103639029115482\\
80.13	0.0010368498125487\\
80.14	0.0010373098997053\\
80.15	0.00103777055676983\\
80.16	0.00103823178792649\\
80.17	0.00103869359739843\\
80.18	0.00103915598944773\\
80.19	0.00103961896837538\\
80.2	0.0010400825385212\\
80.21	0.00104054670426376\\
80.22	0.00104101147002031\\
80.23	0.00104147684024661\\
80.24	0.00104194281943683\\
80.25	0.00104240941212337\\
80.26	0.00104287662287665\\
80.27	0.00104334445630494\\
80.28	0.00104381291705408\\
80.29	0.00104428200980724\\
80.3	0.00104475173928458\\
80.31	0.00104522211024301\\
80.32	0.00104569312747575\\
80.33	0.00104616479581198\\
80.34	0.00104663712011644\\
80.35	0.00104711010528895\\
80.36	0.00104758375626394\\
80.37	0.00104805807800992\\
80.38	0.00104853307552892\\
80.39	0.00104900875385589\\
80.4	0.00104948511805809\\
80.41	0.00104996217323433\\
80.42	0.00105043992451434\\
80.43	0.00105091837705793\\
80.44	0.00105139753605422\\
80.45	0.00105187740672071\\
80.46	0.00105235799430243\\
80.47	0.00105283930407096\\
80.48	0.00105332134132337\\
80.49	0.00105380411138116\\
80.5	0.00105428761958914\\
80.51	0.00105477187131422\\
80.52	0.00105525687194416\\
80.53	0.00105574262688623\\
80.54	0.00105622914156584\\
80.55	0.00105671642142507\\
80.56	0.00105720447192115\\
80.57	0.00105769329852485\\
80.58	0.00105818290671882\\
80.59	0.00105867330199581\\
80.6	0.00105916448985684\\
80.61	0.00105965647580928\\
80.62	0.00106014926536485\\
80.63	0.00106064286403751\\
80.64	0.00106113727734125\\
80.65	0.00106163251078785\\
80.66	0.00106212856988445\\
80.67	0.00106262546013108\\
80.68	0.00106312318701804\\
80.69	0.00106362175602325\\
80.7	0.00106412117260936\\
80.71	0.00106462144222087\\
80.72	0.00106512257028106\\
80.73	0.00106562456218878\\
80.74	0.0010661274233152\\
80.75	0.00106663115900029\\
80.76	0.00106713577454932\\
80.77	0.0010676412752291\\
80.78	0.0010681476662641\\
80.79	0.00106865495283245\\
80.8	0.00106916314006178\\
80.81	0.00106967223302486\\
80.82	0.00107018223673511\\
80.83	0.00107069315614193\\
80.84	0.00107120499612585\\
80.85	0.00107171776149348\\
80.86	0.00107223145697233\\
80.87	0.00107274608720538\\
80.88	0.00107326165674543\\
80.89	0.00107377817004933\\
80.9	0.00107429563147193\\
80.91	0.00107481404525982\\
80.92	0.00107533341554484\\
80.93	0.00107585374633741\\
80.94	0.0010763750415195\\
80.95	0.00107689730483751\\
80.96	0.00107742053989475\\
80.97	0.00107794475014374\\
80.98	0.00107846993887819\\
80.99	0.00107899610922477\\
81	0.00107952326413447\\
81.01	0.00108005140637381\\
81.02	0.0010805805385156\\
81.03	0.00108111066292949\\
81.04	0.00108164178177212\\
81.05	0.00108217389697699\\
81.06	0.00108270701024393\\
81.07	0.00108324112302827\\
81.08	0.00108377623652955\\
81.09	0.00108431235167998\\
81.1	0.00108484946913237\\
81.11	0.00108538758924776\\
81.12	0.0010859267120826\\
81.13	0.00108646683737546\\
81.14	0.00108700796453335\\
81.15	0.0010875500926176\\
81.16	0.00108809322032918\\
81.17	0.00108863734599366\\
81.18	0.00108918246754556\\
81.19	0.00108972858251226\\
81.2	0.00109027568799734\\
81.21	0.00109082378066344\\
81.22	0.00109137285671444\\
81.23	0.00109192291187722\\
81.24	0.00109247394138267\\
81.25	0.00109302593994621\\
81.26	0.00109357890174761\\
81.27	0.00109413282041017\\
81.28	0.00109468768897928\\
81.29	0.00109524349990022\\
81.3	0.00109580024499531\\
81.31	0.00109635791544029\\
81.32	0.00109691650173997\\
81.33	0.00109747599370312\\
81.34	0.00109803638041651\\
81.35	0.00109859765021823\\
81.36	0.00109915979067001\\
81.37	0.00109972278852884\\
81.38	0.00110028662971761\\
81.39	0.00110085129929481\\
81.4	0.00110141678142332\\
81.41	0.00110198305933824\\
81.42	0.00110255011531366\\
81.43	0.00110311793062846\\
81.44	0.00110368648553098\\
81.45	0.00110425575920269\\
81.46	0.00110482572972059\\
81.47	0.00110539637401863\\
81.48	0.00110596766784778\\
81.49	0.00110653958573493\\
81.5	0.00110711210094061\\
81.51	0.00110768518541523\\
81.52	0.00110825883518859\\
81.53	0.00110883305497247\\
81.54	0.00110940784955213\\
81.55	0.00110998322378759\\
81.56	0.00111055918261489\\
81.57	0.00111113573104737\\
81.58	0.00111171287417704\\
81.59	0.00111229061717586\\
81.6	0.00111286896529715\\
81.61	0.00111344792387699\\
81.62	0.00111402749833559\\
81.63	0.00111460769417879\\
81.64	0.00111518851699948\\
81.65	0.00111576997247912\\
81.66	0.00111635206638927\\
81.67	0.00111693480459312\\
81.68	0.00111751819304706\\
81.69	0.00111810223780232\\
81.7	0.00111868694500658\\
81.71	0.00111927232090563\\
81.72	0.00111985837184508\\
81.73	0.00112044510427207\\
81.74	0.00112103252473706\\
81.75	0.00112162063989557\\
81.76	0.00112220945651002\\
81.77	0.0011227989814516\\
81.78	0.00112338922170216\\
81.79	0.00112398018435609\\
81.8	0.00112457187662233\\
81.81	0.00112516430582632\\
81.82	0.00112575747941207\\
81.83	0.00112635140494417\\
81.84	0.00112694609010998\\
81.85	0.00112754154272169\\
81.86	0.00112813777071854\\
81.87	0.00112873478216904\\
81.88	0.00112933258527326\\
81.89	0.00112993118836508\\
81.9	0.00113053059991458\\
81.91	0.00113113082853041\\
81.92	0.00113173188296225\\
81.93	0.00113233377210329\\
81.94	0.0011329365049927\\
81.95	0.00113354009081831\\
81.96	0.00113414453891912\\
81.97	0.00113474985878806\\
81.98	0.00113535606007465\\
81.99	0.0011359631525878\\
82	0.00113657114629866\\
82.01	0.00113718005134342\\
82.02	0.00113778987802633\\
82.03	0.0011384006368226\\
82.04	0.00113901233838153\\
82.05	0.00113962499352952\\
82.06	0.00114023861327332\\
82.07	0.00114085320880314\\
82.08	0.00114146879149602\\
82.09	0.00114208537291914\\
82.1	0.0011427029648332\\
82.11	0.0011433215791959\\
82.12	0.00114394122816549\\
82.13	0.00114456192410437\\
82.14	0.00114518367958272\\
82.15	0.00114580650738228\\
82.16	0.00114643042050015\\
82.17	0.00114705543215266\\
82.18	0.00114768155577934\\
82.19	0.00114830880504696\\
82.2	0.00114893719385362\\
82.21	0.00114956673633295\\
82.22	0.00115019744685838\\
82.23	0.00115082934004751\\
82.24	0.00115146243076651\\
82.25	0.00115209673413467\\
82.26	0.00115273226552901\\
82.27	0.00115336904058896\\
82.28	0.00115400707522116\\
82.29	0.00115464638560436\\
82.3	0.00115528698819437\\
82.31	0.00115592889972916\\
82.32	0.00115657213723402\\
82.33	0.00115721671802684\\
82.34	0.00115786265972349\\
82.35	0.00115850998024331\\
82.36	0.00115915869781467\\
82.37	0.00115980883098072\\
82.38	0.00116046039860519\\
82.39	0.00116111341987829\\
82.4	0.00116176791432278\\
82.41	0.00116242390180014\\
82.42	0.00116308140251685\\
82.43	0.00116374043703079\\
82.44	0.00116440102625781\\
82.45	0.00116506319147835\\
82.46	0.00116572695434431\\
82.47	0.00116639233688593\\
82.48	0.00116705936151888\\
82.49	0.00116772805105151\\
82.5	0.00116839842869217\\
82.51	0.00116907051805674\\
82.52	0.00116974434317631\\
82.53	0.00117041992850495\\
82.54	0.00117109729892773\\
82.55	0.00117177647976883\\
82.56	0.00117245749679984\\
82.57	0.00117314037624823\\
82.58	0.00117382514480598\\
82.59	0.00117451182963844\\
82.6	0.00117520045839323\\
82.61	0.00117589105920951\\
82.62	0.0011765836607273\\
82.63	0.00117727829209703\\
82.64	0.00117797498298932\\
82.65	0.00117867376360489\\
82.66	0.00117937466468476\\
82.67	0.00118007771752057\\
82.68	0.00118078295396519\\
82.69	0.00118149040644351\\
82.7	0.00118220010796344\\
82.71	0.00118291209212718\\
82.72	0.00118362639314266\\
82.73	0.00118434304583533\\
82.74	0.00118506208566005\\
82.75	0.00118578354871333\\
82.76	0.0011865074717458\\
82.77	0.00118723389217492\\
82.78	0.00118796284809799\\
82.79	0.00118869437830541\\
82.8	0.0011894285222942\\
82.81	0.00119016532028187\\
82.82	0.00119090481322048\\
82.83	0.00119164704281113\\
82.84	0.00119239205151861\\
82.85	0.00119313988258647\\
82.86	0.00119389058005238\\
82.87	0.00119464418876378\\
82.88	0.00119540075439388\\
82.89	0.00119616032345807\\
82.9	0.00119692294333056\\
82.91	0.00119768866226145\\
82.92	0.00119845752939418\\
82.93	0.00119922959478328\\
82.94	0.00120000490941258\\
82.95	0.00120078352521378\\
82.96	0.00120156549508538\\
82.97	0.00120235087291211\\
82.98	0.00120313971358466\\
82.99	0.00120393207301996\\
83	0.00120472800818183\\
83.01	0.00120552757710205\\
83.02	0.00120633083890199\\
83.03	0.0012071378538146\\
83.04	0.00120794868320696\\
83.05	0.00120876338960326\\
83.06	0.00120958203670836\\
83.07	0.00121040468943177\\
83.08	0.00121123141391226\\
83.09	0.00121206227754291\\
83.1	0.00121289734899678\\
83.11	0.00121373669825316\\
83.12	0.0012145803966243\\
83.13	0.00121542851678286\\
83.14	0.00121628113278987\\
83.15	0.00121713832012336\\
83.16	0.00121800015570763\\
83.17	0.00121886671794311\\
83.18	0.00121973808673699\\
83.19	0.00122061434353445\\
83.2	0.00122149557135061\\
83.21	0.00122238185480321\\
83.22	0.00122327328014602\\
83.23	0.001224169935303\\
83.24	0.00122507190990321\\
83.25	0.00122597929531654\\
83.26	0.00122689218469025\\
83.27	0.00122781067298629\\
83.28	0.00122873485701948\\
83.29	0.00122966483549664\\
83.3	0.00123060070905649\\
83.31	0.00123154258031051\\
83.32	0.00123249055388477\\
83.33	0.00123344473646265\\
83.34	0.00123440523682859\\
83.35	0.00123537216591278\\
83.36	0.001236345636837\\
83.37	0.00123732576496133\\
83.38	0.00123831266793213\\
83.39	0.00123930646573096\\
83.4	0.00124030728072478\\
83.41	0.00124131523771717\\
83.42	0.00124233046400085\\
83.43	0.00124335308941136\\
83.44	0.00124438324638201\\
83.45	0.00124542107000009\\
83.46	0.00124646669806443\\
83.47	0.00124752027114426\\
83.48	0.00124858193263948\\
83.49	0.00124965182884234\\
83.5	0.00125073010900052\\
83.51	0.00125181692538181\\
83.52	0.00125291243334014\\
83.53	0.00125401679138337\\
83.54	0.00125513016124248\\
83.55	0.00125625270794258\\
83.56	0.00125738459987546\\
83.57	0.00125852317353368\\
83.58	0.00125966227977767\\
83.59	0.0012608019191867\\
83.6	0.00126194209234181\\
83.61	0.00126308279982579\\
83.62	0.00126422404222321\\
83.63	0.00126536582012046\\
83.64	0.00126650813410573\\
83.65	0.00126765098476907\\
83.66	0.00126879437270237\\
83.67	0.00126993829849939\\
83.68	0.00127108276275582\\
83.69	0.00127222776606923\\
83.7	0.00127337330903915\\
83.71	0.00127451939226705\\
83.72	0.0012756660163564\\
83.73	0.00127681318191264\\
83.74	0.00127796088954326\\
83.75	0.00127910913985777\\
83.76	0.00128025793346775\\
83.77	0.00128140727098688\\
83.78	0.00128255715303094\\
83.79	0.00128370758021783\\
83.8	0.00128485855316764\\
83.81	0.00128601007250261\\
83.82	0.00128716213884719\\
83.83	0.00128831475282808\\
83.84	0.00128946791507421\\
83.85	0.00129062162621681\\
83.86	0.00129177588688941\\
83.87	0.00129293069772787\\
83.88	0.0012940860593704\\
83.89	0.00129524197245762\\
83.9	0.00129639843763255\\
83.91	0.00129755545554066\\
83.92	0.00129871302682987\\
83.93	0.00129987115215063\\
83.94	0.0013010298321559\\
83.95	0.00130218906750121\\
83.96	0.00130334885884467\\
83.97	0.00130450920684702\\
83.98	0.00130567011217164\\
83.99	0.00130683157548461\\
84	0.0013079935974547\\
84.01	0.00130915617875347\\
84.02	0.00131031932005522\\
84.03	0.00131148302203709\\
84.04	0.00131264728537907\\
84.05	0.00131381211076402\\
84.06	0.00131497749887775\\
84.07	0.001316143450409\\
84.08	0.00131730996604952\\
84.09	0.00131847704649408\\
84.1	0.00131964469244054\\
84.11	0.00132081290458985\\
84.12	0.00132198168364611\\
84.13	0.00132315103031661\\
84.14	0.00132432094531187\\
84.15	0.00132549142934568\\
84.16	0.00132666248313514\\
84.17	0.00132783410740069\\
84.18	0.0013290063028662\\
84.19	0.00133017907025894\\
84.2	0.00133135241030969\\
84.21	0.00133252632375275\\
84.22	0.001333700811326\\
84.23	0.00133487587377096\\
84.24	0.00133605151183278\\
84.25	0.00133722772626037\\
84.26	0.00133840451780638\\
84.27	0.0013395818872273\\
84.28	0.00134075983528346\\
84.29	0.00134193836273915\\
84.3	0.00134311747036259\\
84.31	0.00134429715892606\\
84.32	0.0013454774292059\\
84.33	0.0013466582819826\\
84.34	0.00134783971804082\\
84.35	0.00134902173816948\\
84.36	0.00135020434316182\\
84.37	0.00135138753381542\\
84.38	0.0013525713109323\\
84.39	0.00135375567531895\\
84.4	0.00135494062778642\\
84.41	0.00135612616915036\\
84.42	0.0013573123002311\\
84.43	0.00135849902185371\\
84.44	0.00135968633484804\\
84.45	0.00136087424004884\\
84.46	0.00136206273829578\\
84.47	0.00136325183043354\\
84.48	0.00136444151731187\\
84.49	0.0013656317997857\\
84.5	0.00136682267871515\\
84.51	0.00136801415496563\\
84.52	0.00136920622940795\\
84.53	0.00137039890291835\\
84.54	0.00137159217637858\\
84.55	0.00137278605067602\\
84.56	0.00137398052670371\\
84.57	0.00137517560536048\\
84.58	0.00137637128755097\\
84.59	0.00137756757418578\\
84.6	0.00137876446618153\\
84.61	0.00137996196446091\\
84.62	0.00138116006995283\\
84.63	0.00138235878359249\\
84.64	0.00138355810632143\\
84.65	0.00138475803908768\\
84.66	0.00138595858284583\\
84.67	0.00138715973855711\\
84.68	0.00138836150718952\\
84.69	0.00138956388971791\\
84.7	0.00139076688712408\\
84.71	0.00139197050039688\\
84.72	0.00139317473053235\\
84.73	0.00139437957853376\\
84.74	0.00139558504541177\\
84.75	0.00139679113218453\\
84.76	0.00139799783987778\\
84.77	0.00139920516952497\\
84.78	0.00140041312216736\\
84.79	0.00140162169885418\\
84.8	0.0014028309006427\\
84.81	0.00140404072859837\\
84.82	0.00140525118379495\\
84.83	0.00140646226731463\\
84.84	0.00140767398024815\\
84.85	0.00140888632369494\\
84.86	0.00141009929876325\\
84.87	0.00141131290657028\\
84.88	0.00141252714824231\\
84.89	0.00141374202491485\\
84.9	0.00141495753773278\\
84.91	0.00141617368785047\\
84.92	0.00141739047643198\\
84.93	0.00141860790465113\\
84.94	0.00141982597369172\\
84.95	0.00142104468474764\\
84.96	0.00142226403902305\\
84.97	0.00142348403773252\\
84.98	0.00142470468210121\\
84.99	0.001425925973365\\
85	0.00142714791277071\\
85.01	0.00142837050157621\\
85.02	0.00142959374105063\\
85.03	0.00143081763247452\\
85.04	0.00143204217714003\\
85.05	0.0014332673763511\\
85.06	0.00143449323142362\\
85.07	0.00143571974368564\\
85.08	0.00143694691447756\\
85.09	0.00143817474515231\\
85.1	0.00143940323707554\\
85.11	0.00144063239162585\\
85.12	0.00144186221019495\\
85.13	0.00144309269418792\\
85.14	0.00144432384502336\\
85.15	0.00144555566413367\\
85.16	0.00144678815296519\\
85.17	0.0014480213129785\\
85.18	0.00144925514564859\\
85.19	0.00145048965246511\\
85.2	0.00145172483493259\\
85.21	0.00145296069457069\\
85.22	0.00145419723291444\\
85.23	0.00145543445151448\\
85.24	0.00145667235193729\\
85.25	0.00145791093576548\\
85.26	0.00145915020459803\\
85.27	0.00146039016005053\\
85.28	0.00146163080375549\\
85.29	0.00146287213736256\\
85.3	0.00146411416253886\\
85.31	0.00146535688096921\\
85.32	0.00146660029435644\\
85.33	0.00146784440442167\\
85.34	0.00146908921290464\\
85.35	0.00147033472156394\\
85.36	0.00147158093217738\\
85.37	0.00147282784654225\\
85.38	0.00147407546647569\\
85.39	0.00147532379381497\\
85.4	0.0014765728304178\\
85.41	0.00147782257816271\\
85.42	0.00147907303894938\\
85.43	0.00148032421469892\\
85.44	0.00148157610735432\\
85.45	0.00148282871888072\\
85.46	0.00148408205126581\\
85.47	0.00148533610652019\\
85.48	0.00148659088667774\\
85.49	0.00148784639379601\\
85.5	0.0014891026299566\\
85.51	0.00149035959726556\\
85.52	0.00149161729785376\\
85.53	0.00149287573387735\\
85.54	0.00149413490751813\\
85.55	0.00149539482098397\\
85.56	0.0014966554765093\\
85.57	0.00149791687635544\\
85.58	0.00149917902281116\\
85.59	0.00150044191819303\\
85.6	0.00150170556484595\\
85.61	0.00150296996514358\\
85.62	0.00150423512148882\\
85.63	0.0015055010363143\\
85.64	0.00150676771208287\\
85.65	0.00150803515128811\\
85.66	0.00150930335645481\\
85.67	0.0015105723301395\\
85.68	0.001511842074931\\
85.69	0.00151311259345092\\
85.7	0.00151438388835423\\
85.71	0.00151565596232981\\
85.72	0.00151692881810099\\
85.73	0.00151820245842616\\
85.74	0.00151947688609934\\
85.75	0.00152075210395074\\
85.76	0.00152202811484742\\
85.77	0.00152330492169389\\
85.78	0.00152458252743269\\
85.79	0.0015258609350451\\
85.8	0.0015271401475517\\
85.81	0.00152842016801314\\
85.82	0.00152970099953068\\
85.83	0.00153098264524699\\
85.84	0.00153226510834678\\
85.85	0.0015335483920575\\
85.86	0.0015348324996501\\
85.87	0.00153611743443974\\
85.88	0.00153740319978652\\
85.89	0.00153868979909626\\
85.9	0.00153997723582124\\
85.91	0.00154126551346104\\
85.92	0.00154255463556327\\
85.93	0.00154384460572442\\
85.94	0.0015451354275907\\
85.95	0.00154642710485882\\
85.96	0.00154771964127692\\
85.97	0.00154901304064538\\
85.98	0.00155030730681776\\
85.99	0.00155160244370163\\
86	0.00155289845525956\\
86.01	0.00155419534551001\\
86.02	0.0015554931185283\\
86.03	0.00155679177844755\\
86.04	0.0015580913294597\\
86.05	0.00155939177581648\\
86.06	0.00156069312183045\\
86.07	0.00156199537187603\\
86.08	0.00156329853039054\\
86.09	0.0015646026018753\\
86.1	0.0015659075908967\\
86.11	0.00156721350208735\\
86.12	0.00156852034014716\\
86.13	0.00156982810984451\\
86.14	0.00157113681601746\\
86.15	0.0015724464635749\\
86.16	0.00157375705749778\\
86.17	0.00157506860284034\\
86.18	0.00157638110473138\\
86.19	0.00157769456837556\\
86.2	0.00157900899905464\\
86.21	0.00158032440212889\\
86.22	0.00158164078303837\\
86.23	0.00158295814730434\\
86.24	0.00158427650053067\\
86.25	0.00158559584840525\\
86.26	0.00158691619670142\\
86.27	0.0015882375512795\\
86.28	0.00158955991808825\\
86.29	0.00159088330316641\\
86.3	0.00159220771264425\\
86.31	0.0015935331527452\\
86.32	0.0015948596297874\\
86.33	0.0015961871501854\\
86.34	0.00159751572045179\\
86.35	0.00159884534719895\\
86.36	0.00160017603714073\\
86.37	0.00160150779709425\\
86.38	0.00160284063398172\\
86.39	0.0016041745548322\\
86.4	0.00160550956678355\\
86.41	0.00160684567708424\\
86.42	0.00160818289309537\\
86.43	0.00160952122229256\\
86.44	0.001610860672268\\
86.45	0.00161220125073247\\
86.46	0.00161354296551743\\
86.47	0.00161488582457713\\
86.48	0.00161622983599075\\
86.49	0.00161757500796459\\
86.5	0.00161892134883434\\
86.51	0.00162026886706732\\
86.52	0.00162161757126479\\
86.53	0.00162296747016433\\
86.54	0.00162431857264222\\
86.55	0.00162567088771593\\
86.56	0.00162702442454652\\
86.57	0.00162837919244127\\
86.58	0.00162973520085621\\
86.59	0.00163109245939875\\
86.6	0.00163245097783038\\
86.61	0.00163381076606939\\
86.62	0.00163517183419364\\
86.63	0.0016365341924434\\
86.64	0.0016378978512242\\
86.65	0.00163926282110982\\
86.66	0.00164062911284523\\
86.67	0.00164199673734967\\
86.68	0.00164336570571972\\
86.69	0.00164473602923247\\
86.7	0.00164610771934877\\
86.71	0.00164748078771645\\
86.72	0.00164885524617371\\
86.73	0.00165023110675247\\
86.74	0.00165160838168191\\
86.75	0.00165298708339191\\
86.76	0.00165436722451671\\
86.77	0.00165574881789855\\
86.78	0.0016571318765914\\
86.79	0.00165851641386478\\
86.8	0.00165990244320759\\
86.81	0.00166128997833212\\
86.82	0.001662679033178\\
86.83	0.00166406962191637\\
86.84	0.00166546175895397\\
86.85	0.00166685545893747\\
86.86	0.00166825073675775\\
86.87	0.00166964760755436\\
86.88	0.00167104608671997\\
86.89	0.00167244618990497\\
86.9	0.00167384793302217\\
86.91	0.00167525133225154\\
86.92	0.00167665640404502\\
86.93	0.00167806316513154\\
86.94	0.001679471632522\\
86.95	0.00168088182351441\\
86.96	0.00168229375569916\\
86.97	0.00168370744696431\\
86.98	0.00168512291550102\\
86.99	0.00168654017980912\\
87	0.00168795925870275\\
87.01	0.00168938017131609\\
87.02	0.00169080293710924\\
87.03	0.00169222757587418\\
87.04	0.00169365410774091\\
87.05	0.00169508255318358\\
87.06	0.0016965129330269\\
87.07	0.00169794526845254\\
87.08	0.00169937958100575\\
87.09	0.001700815892602\\
87.1	0.0017022542255339\\
87.11	0.00170369460247811\\
87.12	0.00170513704650246\\
87.13	0.00170658158107319\\
87.14	0.00170802823006235\\
87.15	0.00170947701775532\\
87.16	0.00171092796885849\\
87.17	0.00171238110850708\\
87.18	0.00171383646227312\\
87.19	0.00171529405617361\\
87.2	0.0017167539166788\\
87.21	0.00171821607072065\\
87.22	0.00171968054570147\\
87.23	0.00172114736950272\\
87.24	0.00172261657049397\\
87.25	0.00172408817754208\\
87.26	0.0017255622200205\\
87.27	0.00172703872781882\\
87.28	0.00172851773135245\\
87.29	0.00172999926157256\\
87.3	0.00173148334997613\\
87.31	0.00173297002861631\\
87.32	0.00173445933011288\\
87.33	0.00173595128766298\\
87.34	0.00173744593505208\\
87.35	0.00173894330666507\\
87.36	0.0017404434374977\\
87.37	0.00174194636316816\\
87.38	0.0017434521199289\\
87.39	0.00174496074467875\\
87.4	0.00174647227497524\\
87.41	0.00174798674904715\\
87.42	0.00174950420580735\\
87.43	0.00175102468486593\\
87.44	0.0017525482265435\\
87.45	0.00175407487188485\\
87.46	0.00175560466267288\\
87.47	0.00175713764144275\\
87.48	0.00175867385149638\\
87.49	0.00176021333691728\\
87.5	0.00176175614258554\\
87.51	0.00176330231419331\\
87.52	0.00176485189826049\\
87.53	0.00176640494215074\\
87.54	0.00176796149408787\\
87.55	0.00176952160317257\\
87.56	0.00177108531939942\\
87.57	0.00177265269367432\\
87.58	0.00177422377783227\\
87.59	0.0017757986246555\\
87.6	0.00177737728789199\\
87.61	0.00177895982227438\\
87.62	0.00178054628353926\\
87.63	0.00178213672844689\\
87.64	0.00178373121480132\\
87.65	0.00178532980147092\\
87.66	0.00178693254840939\\
87.67	0.00178853951667714\\
87.68	0.00179015076846316\\
87.69	0.00179176636710741\\
87.7	0.00179338637712356\\
87.71	0.00179501086422232\\
87.72	0.00179663989533522\\
87.73	0.00179827353863892\\
87.74	0.00179991186357999\\
87.75	0.00180155494090029\\
87.76	0.00180320284266281\\
87.77	0.00180485564227817\\
87.78	0.00180651341453159\\
87.79	0.00180817623561049\\
87.8	0.00180984418313267\\
87.81	0.00181151733617512\\
87.82	0.00181319577530343\\
87.83	0.00181487958260183\\
87.84	0.00181656884170392\\
87.85	0.00181826363782401\\
87.86	0.00181996405778924\\
87.87	0.00182167019007225\\
87.88	0.00182338212482473\\
87.89	0.00182509995391157\\
87.9	0.00182682377094585\\
87.91	0.00182855367132455\\
87.92	0.00183028975226504\\
87.93	0.00183203211284243\\
87.94	0.00183378085402768\\
87.95	0.00183553607872658\\
87.96	0.0018372978918196\\
87.97	0.00183906640020263\\
87.98	0.00184084171282858\\
87.99	0.00184262394074997\\
88	0.00184441319716245\\
88.01	0.00184620959744923\\
88.02	0.00184801325922662\\
88.03	0.00184982430239047\\
88.04	0.00185164284916378\\
88.05	0.00185346902414524\\
88.06	0.00185530295435899\\
88.07	0.00185714476930545\\
88.08	0.00185899460101327\\
88.09	0.00186085258409253\\
88.1	0.00186271885578906\\
88.11	0.0018645935560401\\
88.12	0.00186647682753113\\
88.13	0.00186836881575407\\
88.14	0.00187026966906676\\
88.15	0.00187217953875385\\
88.16	0.00187409857908901\\
88.17	0.00187602694739871\\
88.18	0.00187796480412729\\
88.19	0.00187991231290369\\
88.2	0.00188186964060964\\
88.21	0.00188383695744944\\
88.22	0.00188581443702137\\
88.23	0.00188780225639077\\
88.24	0.00188980059616478\\
88.25	0.00189180964056891\\
88.26	0.00189382957752529\\
88.27	0.00189586059873282\\
88.28	0.00189790289974923\\
88.29	0.00189995668007498\\
88.3	0.00190202214323918\\
88.31	0.0019040994968876\\
88.32	0.00190618895287265\\
88.33	0.00190829072734554\\
88.34	0.0019104050408507\\
88.35	0.00191253211842227\\
88.36	0.00191467218968309\\
88.37	0.00191682548894589\\
88.38	0.00191899225531702\\
88.39	0.00192117273280255\\
88.4	0.00192336717041704\\
88.41	0.0019255758222948\\
88.42	0.00192779894780395\\
88.43	0.00193003681166313\\
88.44	0.00193228968406108\\
88.45	0.00193455784077916\\
88.46	0.00193684156331674\\
88.47	0.00193914113901974\\
88.48	0.0019414568612122\\
88.49	0.00194378902933121\\
88.5	0.00194613794906498\\
88.51	0.00194850393249437\\
88.52	0.00195088729823793\\
88.53	0.00195328837160043\\
88.54	0.0019557074847251\\
88.55	0.00195814497674962\\
88.56	0.00196060119396597\\
88.57	0.00196307648998422\\
88.58	0.00196557122590046\\
88.59	0.00196808577046878\\
88.6	0.0019706205002777\\
88.61	0.00197316460559005\\
88.62	0.00197570969336572\\
88.63	0.00197825576441454\\
88.64	0.00198080281954738\\
88.65	0.00198335085957615\\
88.66	0.00198589988531384\\
88.67	0.00198844989757451\\
88.68	0.00199100089717326\\
88.69	0.00199355288492626\\
88.7	0.00199610586165076\\
88.71	0.00199865982816507\\
88.72	0.00200121478528856\\
88.73	0.0020037707338417\\
88.74	0.00200632767464599\\
88.75	0.00200888560852406\\
88.76	0.00201144453629957\\
88.77	0.00201400445879728\\
88.78	0.00201656537684304\\
88.79	0.00201912729126378\\
88.8	0.00202169020288749\\
88.81	0.00202425411254327\\
88.82	0.00202681902106132\\
88.83	0.0020293849292729\\
88.84	0.00203195183801038\\
88.85	0.00203451974810723\\
88.86	0.00203708866039801\\
88.87	0.00203965857571837\\
88.88	0.00204222949490507\\
88.89	0.00204480141879598\\
88.9	0.00204737434823005\\
88.91	0.00204994828404736\\
88.92	0.00205252322708909\\
88.93	0.00205509917819752\\
88.94	0.00205767613821605\\
88.95	0.0020602541079892\\
88.96	0.0020628330883626\\
88.97	0.00206541308018298\\
88.98	0.00206799408429823\\
88.99	0.00207057610155732\\
89	0.00207315913281036\\
89.01	0.0020757431789086\\
89.02	0.00207832824070439\\
89.03	0.00208091431905124\\
89.04	0.00208350141480375\\
89.05	0.00208608952881769\\
89.06	0.00208867866194995\\
89.07	0.00209126881505857\\
89.08	0.0020938599890027\\
89.09	0.00209645218464266\\
89.1	0.0020990454028399\\
89.11	0.00210163964445701\\
89.12	0.00210423491035774\\
89.13	0.00210683120140699\\
89.14	0.0021094285184708\\
89.15	0.00211202686241636\\
89.16	0.00211462623411203\\
89.17	0.00211722663442732\\
89.18	0.00211982806423289\\
89.19	0.00212243052440058\\
89.2	0.00212503401580337\\
89.21	0.00212763853931543\\
89.22	0.00213024409581207\\
89.23	0.0021328506861698\\
89.24	0.00213545831126627\\
89.25	0.00213806697198033\\
89.26	0.00214067666919199\\
89.27	0.00214328740378243\\
89.28	0.00214589917663404\\
89.29	0.00214851198863036\\
89.3	0.00215112584065612\\
89.31	0.00215374073359724\\
89.32	0.00215635666834084\\
89.33	0.00215897364577522\\
89.34	0.00216159166678985\\
89.35	0.00216421073227542\\
89.36	0.00216683084312382\\
89.37	0.00216945200022812\\
89.38	0.0021720742044826\\
89.39	0.00217469745678273\\
89.4	0.0021773217580252\\
89.41	0.00217994710910791\\
89.42	0.00218257351092994\\
89.43	0.0021852009643916\\
89.44	0.00218782947039443\\
89.45	0.00219045902984114\\
89.46	0.00219308964363571\\
89.47	0.00219572131268329\\
89.48	0.00219835403789028\\
89.49	0.00220098782016431\\
89.5	0.0022036226604142\\
89.51	0.00220625855955003\\
89.52	0.0022088955184831\\
89.53	0.00221153353812593\\
89.54	0.0022141726193923\\
89.55	0.0022168127631972\\
89.56	0.00221945397045687\\
89.57	0.00222209624208879\\
89.58	0.00222473957901169\\
89.59	0.00222738398214551\\
89.6	0.00223002945241149\\
89.61	0.00223267599073208\\
89.62	0.002235323598031\\
89.63	0.0022379722752332\\
89.64	0.00224062202326492\\
89.65	0.00224327284305362\\
89.66	0.00224592473552806\\
89.67	0.00224857770161823\\
89.68	0.00225123174225539\\
89.69	0.00225388685837208\\
89.7	0.00225654305090209\\
89.71	0.00225920032078051\\
89.72	0.00226185866894366\\
89.73	0.00226451809632917\\
89.74	0.00226717860387594\\
89.75	0.00226984019252414\\
89.76	0.00227250286321523\\
89.77	0.00227516661689195\\
89.78	0.00227783145449833\\
89.79	0.00228049737697967\\
89.8	0.0022831643852826\\
89.81	0.002285832480355\\
89.82	0.00228850166314608\\
89.83	0.00229117193460631\\
89.84	0.0022938432956875\\
89.85	0.00229651574734274\\
89.86	0.00229918929052641\\
89.87	0.00230186392619423\\
89.88	0.0023045396553032\\
89.89	0.00230721647881165\\
89.9	0.00230989439767921\\
89.91	0.00231257341286682\\
89.92	0.00231525352533676\\
89.93	0.0023179347360526\\
89.94	0.00232061704597927\\
89.95	0.00232330045608298\\
89.96	0.0023259849673313\\
89.97	0.00232867058069311\\
89.98	0.00233135729713863\\
89.99	0.00233404511763941\\
90	0.00233673404316833\\
90.01	0.00233942407469962\\
90.02	0.00234211521320885\\
90.03	0.0023448074596729\\
90.04	0.00234750081507004\\
90.05	0.00235019528037986\\
90.06	0.0023528908565833\\
90.07	0.00235558754466265\\
90.08	0.00235828534560157\\
90.09	0.00236098426038507\\
90.1	0.00236368428999949\\
90.11	0.00236638543543256\\
90.12	0.00236908769767337\\
90.13	0.00237179107771237\\
90.14	0.00237449557654136\\
90.15	0.00237720119515354\\
90.16	0.00237990793454346\\
90.17	0.00238261579570705\\
90.18	0.00238532477964162\\
90.19	0.00238803488734584\\
90.2	0.00239074611981979\\
90.21	0.0023934584780649\\
90.22	0.00239617196308401\\
90.23	0.00239888657588134\\
90.24	0.0024016023174625\\
90.25	0.00240431918883448\\
90.26	0.00240703719100568\\
90.27	0.00240975632498589\\
90.28	0.00241247659178629\\
90.29	0.00241519799241948\\
90.3	0.00241792052789944\\
90.31	0.00242064419924157\\
90.32	0.00242336900746268\\
90.33	0.00242609495358098\\
90.34	0.00242882203861609\\
90.35	0.00243155026358906\\
90.36	0.00243427962952233\\
90.37	0.00243701013743979\\
90.38	0.00243974178836673\\
90.39	0.00244247458332986\\
90.4	0.00244520852335733\\
90.41	0.00244794360947871\\
90.42	0.002450679842725\\
90.43	0.00245341722412863\\
90.44	0.00245615575472347\\
90.45	0.00245889543554482\\
90.46	0.00246163626762942\\
90.47	0.00246437825201545\\
90.48	0.00246712138974253\\
90.49	0.00246986568185174\\
90.5	0.00247261112938559\\
90.51	0.00247535773338804\\
90.52	0.00247810549490451\\
90.53	0.00248085441498188\\
90.54	0.00248360449466846\\
90.55	0.00248635573501404\\
90.56	0.00248910813706987\\
90.57	0.00249186170188865\\
90.58	0.00249461643052456\\
90.59	0.00249737232403322\\
90.6	0.00250012938347175\\
90.61	0.00250288760989871\\
90.62	0.00250564700437417\\
90.63	0.00250840756795964\\
90.64	0.00251116930171812\\
90.65	0.00251393220671409\\
90.66	0.00251669628401351\\
90.67	0.00251946153468383\\
90.68	0.00252222795979396\\
90.69	0.00252499556041434\\
90.7	0.00252776433761686\\
90.71	0.00253053429247491\\
90.72	0.00253330542606339\\
90.73	0.00253607773945868\\
90.74	0.00253885123373865\\
90.75	0.00254162590998269\\
90.76	0.00254440176927168\\
90.77	0.002547178812688\\
90.78	0.00254995704131553\\
90.79	0.00255273645623968\\
90.8	0.00255551705854734\\
90.81	0.00255829884932693\\
90.82	0.00256108182966837\\
90.83	0.0025638660006631\\
90.84	0.00256665136340408\\
90.85	0.00256943791898577\\
90.86	0.00257222566850418\\
90.87	0.00257501461305681\\
90.88	0.0025778047537427\\
90.89	0.00258059609166242\\
90.9	0.00258338862791804\\
90.91	0.00258618236361318\\
90.92	0.002588977299853\\
90.93	0.00259177343774416\\
90.94	0.00259457077839488\\
90.95	0.0025973693229149\\
90.96	0.0026001690724155\\
90.97	0.00260297002800949\\
90.98	0.00260577219081125\\
90.99	0.00260857556193666\\
91	0.00261138014250316\\
91.01	0.00261418593362975\\
91.02	0.00261699293643695\\
91.03	0.00261980115204684\\
91.04	0.00262261058158304\\
91.05	0.00262542122617072\\
91.06	0.00262823308693662\\
91.07	0.00263104616500901\\
91.08	0.00263386046151771\\
91.09	0.00263667597759412\\
91.1	0.00263949271437117\\
91.11	0.00264231067298337\\
91.12	0.00264512985456675\\
91.13	0.00264795026025895\\
91.14	0.00265077189119914\\
91.15	0.00265359474852805\\
91.16	0.00265641883338797\\
91.17	0.00265924414692278\\
91.18	0.00266207069027789\\
91.19	0.0026648984646003\\
91.2	0.00266772747103857\\
91.21	0.00267055771074281\\
91.22	0.00267338918486471\\
91.23	0.00267622189455754\\
91.24	0.00267905584097612\\
91.25	0.00268189102527685\\
91.26	0.0026847274486177\\
91.27	0.0026875651121582\\
91.28	0.00269040401705946\\
91.29	0.00269324416448415\\
91.3	0.00269608555559654\\
91.31	0.00269892819156244\\
91.32	0.00270177207354925\\
91.33	0.00270461720272594\\
91.34	0.00270746358026304\\
91.35	0.00271031120733268\\
91.36	0.00271316008510853\\
91.37	0.00271601021476586\\
91.38	0.0027188615974815\\
91.39	0.00272171423443386\\
91.4	0.00272456812680291\\
91.41	0.0027274232757702\\
91.42	0.00273027968251886\\
91.43	0.00273313734823356\\
91.44	0.00273599627410059\\
91.45	0.00273885646130778\\
91.46	0.00274171791104452\\
91.47	0.00274458062450179\\
91.48	0.00274744460287213\\
91.49	0.00275030984734965\\
91.5	0.00275317635913003\\
91.51	0.00275604413941051\\
91.52	0.00275891318938988\\
91.53	0.00276178351026853\\
91.54	0.00276465510324838\\
91.55	0.00276752796953292\\
91.56	0.0027704021103272\\
91.57	0.00277327752683783\\
91.58	0.00277615422027297\\
91.59	0.00277903219184234\\
91.6	0.00278191144275722\\
91.61	0.00278479197423041\\
91.62	0.00278767378747629\\
91.63	0.00279055688371078\\
91.64	0.00279344126415133\\
91.65	0.00279632693001694\\
91.66	0.00279921388252816\\
91.67	0.00280210212290706\\
91.68	0.00280499165237724\\
91.69	0.00280788247216386\\
91.7	0.00281077458349357\\
91.71	0.00281366798759457\\
91.72	0.00281656268569658\\
91.73	0.00281945867903083\\
91.74	0.00282235596883006\\
91.75	0.00282525455632854\\
91.76	0.00282815444276202\\
91.77	0.00283105562936779\\
91.78	0.0028339581173846\\
91.79	0.00283686190805273\\
91.8	0.00283976700261392\\
91.81	0.00284267340231141\\
91.82	0.00284558110838993\\
91.83	0.00284849012209567\\
91.84	0.00285140044467631\\
91.85	0.00285431207738098\\
91.86	0.00285722502146028\\
91.87	0.00286013927816625\\
91.88	0.00286305484875241\\
91.89	0.00286597173447369\\
91.9	0.00286888993658648\\
91.91	0.0028718094563486\\
91.92	0.00287473029501928\\
91.93	0.00287765245385918\\
91.94	0.00288057593413037\\
91.95	0.00288350073709632\\
91.96	0.0028864268640219\\
91.97	0.00288935431617337\\
91.98	0.00289228309481837\\
91.99	0.00289521320122591\\
92	0.00289814463666638\\
92.01	0.00290107740241149\\
92.02	0.00290401149973435\\
92.03	0.00290694692990936\\
92.04	0.00290988369421229\\
92.05	0.00291282179392019\\
92.06	0.00291576123031145\\
92.07	0.00291870200466576\\
92.08	0.00292164411826409\\
92.09	0.00292458757238868\\
92.1	0.00292753236832305\\
92.11	0.00293047850735199\\
92.12	0.00293342599076152\\
92.13	0.00293637481983889\\
92.14	0.00293932499587258\\
92.15	0.00294227652015229\\
92.16	0.00294522939396889\\
92.17	0.00294818361861446\\
92.18	0.00295113919538224\\
92.19	0.00295409612556662\\
92.2	0.00295705441046314\\
92.21	0.00296001405136848\\
92.22	0.00296297504958041\\
92.23	0.00296593740639782\\
92.24	0.00296890112312066\\
92.25	0.00297186620104996\\
92.26	0.00297483264148781\\
92.27	0.00297780044573732\\
92.28	0.00298076961510261\\
92.29	0.00298374015088883\\
92.3	0.00298671205440207\\
92.31	0.00298968532694941\\
92.32	0.00299265996983887\\
92.33	0.00299563598437938\\
92.34	0.0029986133718808\\
92.35	0.00300159213365386\\
92.36	0.00300457227101014\\
92.37	0.0030075537852621\\
92.38	0.00301053667772297\\
92.39	0.00301352094970683\\
92.4	0.0030165066025285\\
92.41	0.00301949363750357\\
92.42	0.00302248205594836\\
92.43	0.00302547185917989\\
92.44	0.00302846304851585\\
92.45	0.00303145562527461\\
92.46	0.00303444959077515\\
92.47	0.00303744494633704\\
92.48	0.00304044169328047\\
92.49	0.00304343983292613\\
92.5	0.00304643936659525\\
92.51	0.00304944029560956\\
92.52	0.00305244262129123\\
92.53	0.00305544634496288\\
92.54	0.0030584514679475\\
92.55	0.0030614579915685\\
92.56	0.00306446591714957\\
92.57	0.00306747524601473\\
92.58	0.00307048597948828\\
92.59	0.00307349811889473\\
92.6	0.00307651166555882\\
92.61	0.00307952662080542\\
92.62	0.00308254298595956\\
92.63	0.00308556076234634\\
92.64	0.00308857995129092\\
92.65	0.00309160055411848\\
92.66	0.00309462257215418\\
92.67	0.00309764600672309\\
92.68	0.00310067085915019\\
92.69	0.00310369713076032\\
92.7	0.00310672482287812\\
92.71	0.00310975393682799\\
92.72	0.00311278447393406\\
92.73	0.00311581643552013\\
92.74	0.00311884982290963\\
92.75	0.00312188463742558\\
92.76	0.00312492088039053\\
92.77	0.00312795855312651\\
92.78	0.003130997656955\\
92.79	0.00313403819319685\\
92.8	0.00313708016317225\\
92.81	0.00314012356820068\\
92.82	0.00314316840960084\\
92.83	0.00314621468869059\\
92.84	0.00314926240678694\\
92.85	0.00315231156520593\\
92.86	0.00315536216526262\\
92.87	0.00315841420827102\\
92.88	0.00316146769554402\\
92.89	0.00316452262839333\\
92.9	0.00316757900812943\\
92.91	0.00317063683606151\\
92.92	0.00317369611349739\\
92.93	0.00317675684174347\\
92.94	0.00317981902210464\\
92.95	0.00318288265588426\\
92.96	0.00318594774438405\\
92.97	0.00318901428890403\\
92.98	0.00319208229074247\\
92.99	0.00319515175119581\\
93	0.00319822267155855\\
93.01	0.00320129505312325\\
93.02	0.0032043688971804\\
93.03	0.00320744420501835\\
93.04	0.00321052097792328\\
93.05	0.00321359921717904\\
93.06	0.00321667892406717\\
93.07	0.00321976009986675\\
93.08	0.00322284274585433\\
93.09	0.00322592686330389\\
93.1	0.0032290124534867\\
93.11	0.00323209951767128\\
93.12	0.00323518805712331\\
93.13	0.00323827807310552\\
93.14	0.00324136956687763\\
93.15	0.00324446253969627\\
93.16	0.00324755699281485\\
93.17	0.00325065292748352\\
93.18	0.00325375034494905\\
93.19	0.00325684924645476\\
93.2	0.00325994963324042\\
93.21	0.00326305150654214\\
93.22	0.00326615486759233\\
93.23	0.00326925971761955\\
93.24	0.00327236605784845\\
93.25	0.00327547388949967\\
93.26	0.00327858321378973\\
93.27	0.00328169403193091\\
93.28	0.00328480634513113\\
93.29	0.00328792015459385\\
93.3	0.0032910354615179\\
93.31	0.00329415226709742\\
93.32	0.0032972705725217\\
93.33	0.00330039037897502\\
93.34	0.00330351168763658\\
93.35	0.00330663449968035\\
93.36	0.00330975881627488\\
93.37	0.00331288463858323\\
93.38	0.00331601196776281\\
93.39	0.00331914080496522\\
93.4	0.00332227115133611\\
93.41	0.00332540300801503\\
93.42	0.0033285363761353\\
93.43	0.00333167125682385\\
93.44	0.00333480765120101\\
93.45	0.00333794556038044\\
93.46	0.00334108498546891\\
93.47	0.00334422592756614\\
93.48	0.00334736838776464\\
93.49	0.00335051236714955\\
93.5	0.00335365786679845\\
93.51	0.0033568048877812\\
93.52	0.00335995343115974\\
93.53	0.00336310349798791\\
93.54	0.0033662550893113\\
93.55	0.003369408206167\\
93.56	0.00337256284958345\\
93.57	0.00337571902058026\\
93.58	0.00337887672016794\\
93.59	0.00338203594934779\\
93.6	0.00338519670911161\\
93.61	0.00338835900044156\\
93.62	0.00339152282430988\\
93.63	0.00339468818167875\\
93.64	0.00339785507349999\\
93.65	0.0034010235007149\\
93.66	0.003404193464254\\
93.67	0.00340736496503682\\
93.68	0.00341053800397162\\
93.69	0.00341371258195521\\
93.7	0.00341688869987268\\
93.71	0.00342006635859713\\
93.72	0.00342324555898948\\
93.73	0.00342642630189813\\
93.74	0.00342960858815878\\
93.75	0.00343279241859412\\
93.76	0.0034359777940136\\
93.77	0.0034391647152131\\
93.78	0.00344235318297471\\
93.79	0.00344554319806643\\
93.8	0.00344873476124186\\
93.81	0.00345192787323996\\
93.82	0.00345512253478471\\
93.83	0.00345831874658483\\
93.84	0.00346151650933348\\
93.85	0.00346471582370795\\
93.86	0.00346791669036934\\
93.87	0.00347111910996224\\
93.88	0.00347432308311441\\
93.89	0.00347752861043647\\
93.9	0.00348073569252154\\
93.91	0.00348394432994491\\
93.92	0.0034871545232637\\
93.93	0.00349036627301653\\
93.94	0.00349357957972313\\
93.95	0.00349679444388399\\
93.96	0.00350001086598002\\
93.97	0.00350322884647217\\
93.98	0.00350644838580103\\
93.99	0.00350966948438649\\
94	0.00351289214262733\\
94.01	0.00351611636090081\\
94.02	0.00351934213956231\\
94.03	0.00352256947894493\\
94.04	0.00352579837935904\\
94.05	0.00352902884109189\\
94.06	0.00353226086440719\\
94.07	0.0035354944495447\\
94.08	0.00353872959671976\\
94.09	0.00354196630612289\\
94.1	0.00354520457791931\\
94.11	0.00354844441224855\\
94.12	0.00355168580922391\\
94.13	0.00355492876893207\\
94.14	0.0035581732914326\\
94.15	0.00356141937675747\\
94.16	0.0035646670249106\\
94.17	0.00356791623586733\\
94.18	0.003571167009574\\
94.19	0.00357441934594738\\
94.2	0.00357767324487419\\
94.21	0.00358092870621063\\
94.22	0.00358418572978179\\
94.23	0.00358744431538119\\
94.24	0.00359070446277023\\
94.25	0.00359396617167766\\
94.26	0.00359722944179901\\
94.27	0.0036004942727961\\
94.28	0.00360376066429643\\
94.29	0.00360702861589267\\
94.3	0.00361029812714207\\
94.31	0.00361356919756589\\
94.32	0.00361684182664881\\
94.33	0.00362011601383841\\
94.34	0.0036233917585445\\
94.35	0.0036266690601386\\
94.36	0.00362994791795327\\
94.37	0.00363322833128158\\
94.38	0.00363651029937644\\
94.39	0.00363979382145003\\
94.4	0.00364307889667315\\
94.41	0.00364636552417461\\
94.42	0.0036496537030406\\
94.43	0.00365294343231403\\
94.44	0.00365623471099394\\
94.45	0.00365952753803482\\
94.46	0.00366282191234598\\
94.47	0.00366611783279086\\
94.48	0.00366941529818645\\
94.49	0.00367271430730255\\
94.5	0.00367601485886115\\
94.51	0.00367931695153576\\
94.52	0.00368262058395075\\
94.53	0.00368592575468064\\
94.54	0.00368923246224947\\
94.55	0.00369254070513008\\
94.56	0.0036958504817435\\
94.57	0.00369916179045819\\
94.58	0.00370247462958941\\
94.59	0.00370578899739853\\
94.6	0.00370910489209232\\
94.61	0.00371242231182233\\
94.62	0.00371574125468414\\
94.63	0.00371906171871672\\
94.64	0.00372238370190175\\
94.65	0.00372570720216291\\
94.66	0.00372903221736526\\
94.67	0.00373235874531453\\
94.68	0.00373568678375645\\
94.69	0.0037390163303761\\
94.7	0.00374234738279727\\
94.71	0.00374567993858176\\
94.72	0.00374901399522877\\
94.73	0.00375234955017426\\
94.74	0.00375568660079029\\
94.75	0.00375902514438443\\
94.76	0.00376236517819911\\
94.77	0.00376570669941106\\
94.78	0.00376904970513068\\
94.79	0.00377239419240149\\
94.8	0.00377574015819951\\
94.81	0.0037790875994328\\
94.82	0.00378243651294081\\
94.83	0.00378578689549397\\
94.84	0.00378913874379309\\
94.85	0.00379249205446895\\
94.86	0.00379584682408179\\
94.87	0.0037992030491209\\
94.88	0.00380256072600416\\
94.89	0.0038059198510777\\
94.9	0.00380928042061549\\
94.91	0.00381264243081901\\
94.92	0.00381600587781696\\
94.93	0.00381937075766495\\
94.94	0.00382273706634526\\
94.95	0.00382610479976667\\
94.96	0.00382947395376421\\
94.97	0.00383284452409909\\
94.98	0.00383621650645858\\
94.99	0.00383958989645595\\
95	0.00384296468963049\\
95.01	0.00384634088144753\\
95.02	0.00384971846729854\\
95.03	0.00385309744250129\\
95.04	0.00385647780230004\\
95.05	0.00385985954186578\\
95.06	0.00386324265629658\\
95.07	0.00386662714061794\\
95.08	0.00387001298978315\\
95.09	0.00387340019867388\\
95.1	0.00387678876210063\\
95.11	0.00388017867480346\\
95.12	0.00388356993145266\\
95.13	0.00388696252664955\\
95.14	0.00389035645492738\\
95.15	0.00389375171075228\\
95.16	0.00389714828852437\\
95.17	0.00390054618257889\\
95.18	0.00390394538718748\\
95.19	0.0039073458965596\\
95.2	0.00391074770484395\\
95.21	0.00391415080613014\\
95.22	0.00391755519445041\\
95.23	0.00392096086378148\\
95.24	0.00392436780804654\\
95.25	0.00392777602111741\\
95.26	0.00393118549681681\\
95.27	0.00393459622892078\\
95.28	0.0039380082111613\\
95.29	0.00394142143722901\\
95.3	0.00394483590077615\\
95.31	0.00394825159541971\\
95.32	0.00395166851474465\\
95.33	0.00395508665230744\\
95.34	0.00395850600163975\\
95.35	0.00396192655625233\\
95.36	0.00396534830963912\\
95.37	0.00396877125528166\\
95.38	0.00397219538665361\\
95.39	0.0039756206972256\\
95.4	0.00397904718047036\\
95.41	0.00398247482986804\\
95.42	0.00398590363891184\\
95.43	0.00398933360111399\\
95.44	0.0039927647100119\\
95.45	0.00399619695917478\\
95.46	0.00399963034221044\\
95.47	0.00400306485277252\\
95.48	0.00400650048456806\\
95.49	0.00400993723136539\\
95.5	0.00401337508700244\\
95.51	0.00401681404539544\\
95.52	0.00402025410054799\\
95.53	0.00402369524656064\\
95.54	0.0040271374776408\\
95.55	0.0040305807881132\\
95.56	0.00403402517243078\\
95.57	0.0040374706251861\\
95.58	0.00404091714112321\\
95.59	0.00404436471515014\\
95.6	0.00404781334235185\\
95.61	0.00405126301800383\\
95.62	0.00405471373758624\\
95.63	0.0040581654967987\\
95.64	0.00406161829157572\\
95.65	0.00406507211810276\\
95.66	0.00406852697283303\\
95.67	0.00407198285250498\\
95.68	0.00407543975416053\\
95.69	0.00407889767516405\\
95.7	0.00408235661322223\\
95.71	0.00408581656640464\\
95.72	0.00408927753316527\\
95.73	0.0040927395123649\\
95.74	0.00409620250329439\\
95.75	0.00409966650569898\\
95.76	0.00410313151980353\\
95.77	0.00410659754633881\\
95.78	0.00411006458656885\\
95.79	0.00411353264231943\\
95.8	0.00411700171600767\\
95.81	0.00412047181067283\\
95.82	0.00412394293000837\\
95.83	0.00412741507839523\\
95.84	0.00413088826093649\\
95.85	0.00413436248349338\\
95.86	0.00413783775272269\\
95.87	0.00414131407611571\\
95.88	0.00414479146203864\\
95.89	0.00414826991977459\\
95.9	0.00415174945956728\\
95.91	0.00415523009266638\\
95.92	0.0041587118313746\\
95.93	0.00416219468909666\\
95.94	0.00416567868039011\\
95.95	0.0041691638210181\\
95.96	0.00417265012800422\\
95.97	0.00417613761968942\\
95.98	0.00417962631579107\\
95.99	0.00418311623746432\\
96	0.00418660740736577\\
96.01	0.00419009984971961\\
96.02	0.00419359359038617\\
96.03	0.00419708865693313\\
96.04	0.00420058507870941\\
96.05	0.00420408288692186\\
96.06	0.00420758211471476\\
96.07	0.00421108279725236\\
96.08	0.00421458497180454\\
96.09	0.0042180886778356\\
96.1	0.0042215939570964\\
96.11	0.00422510085371993\\
96.12	0.00422860941432044\\
96.13	0.00423211968809624\\
96.14	0.00423563172693633\\
96.15	0.00423914558553093\\
96.16	0.00424266132148616\\
96.17	0.0042461789954429\\
96.18	0.00424969867120005\\
96.19	0.00425322041584236\\
96.2	0.00425674429987289\\
96.21	0.00426027039735038\\
96.22	0.00426379878603165\\
96.23	0.00426732954751923\\
96.24	0.00427086276741434\\
96.25	0.00427439853547551\\
96.26	0.00427793694578297\\
96.27	0.00428147809690903\\
96.28	0.0042850220920947\\
96.29	0.00428856898279929\\
96.3	0.00429211878638083\\
96.31	0.00429567152050528\\
96.32	0.00429922720315299\\
96.33	0.00430278585262515\\
96.34	0.00430634748755051\\
96.35	0.00430991212689213\\
96.36	0.00431347978995437\\
96.37	0.00431705049638999\\
96.38	0.00432062426620734\\
96.39	0.00432420111977782\\
96.4	0.00432778107784344\\
96.41	0.00433136416152448\\
96.42	0.00433495039232746\\
96.43	0.00433853979215311\\
96.44	0.00434213238330466\\
96.45	0.00434572818849622\\
96.46	0.00434932723086133\\
96.47	0.00435292953396178\\
96.48	0.00435653512179651\\
96.49	0.00436014401881079\\
96.5	0.00436375624990552\\
96.51	0.0043673718404468\\
96.52	0.00437099081627567\\
96.53	0.00437461320371802\\
96.54	0.00437823902959481\\
96.55	0.00438186832123239\\
96.56	0.00438550110647316\\
96.57	0.00438913741368635\\
96.58	0.0043927772717791\\
96.59	0.00439642071020778\\
96.6	0.0044000677589895\\
96.61	0.0044037184487139\\
96.62	0.00440737281055519\\
96.63	0.00441103087628448\\
96.64	0.00441469267828231\\
96.65	0.00441835824955147\\
96.66	0.00442202762373015\\
96.67	0.00442570083510529\\
96.68	0.00442937791862629\\
96.69	0.00443305890991894\\
96.7	0.00443674384529974\\
96.71	0.00444043276179043\\
96.72	0.00444412569713291\\
96.73	0.00444782268980444\\
96.74	0.00445152377903318\\
96.75	0.00445522900481407\\
96.76	0.00445893840792503\\
96.77	0.00446265202994354\\
96.78	0.00446636991326354\\
96.79	0.00447009210111275\\
96.8	0.00447381863757028\\
96.81	0.00447754956758471\\
96.82	0.0044812849369925\\
96.83	0.00448502479253681\\
96.84	0.00448876918188675\\
96.85	0.00449251815365699\\
96.86	0.00449627175742788\\
96.87	0.00450003004376591\\
96.88	0.00450379306424469\\
96.89	0.00450756087146632\\
96.9	0.00451133351908329\\
96.91	0.00451511106182076\\
96.92	0.00451889355549944\\
96.93	0.00452268105705886\\
96.94	0.0045264736245812\\
96.95	0.00453027131731562\\
96.96	0.00453407419570313\\
96.97	0.00453788232140196\\
96.98	0.00454169575731355\\
96.99	0.00454551456760903\\
97	0.0045493388177563\\
97.01	0.00455316857454773\\
97.02	0.00455700390612843\\
97.03	0.00456084488202511\\
97.04	0.00456469157317562\\
97.05	0.0045685440519591\\
97.06	0.0045724023922268\\
97.07	0.00457626666933354\\
97.08	0.00458013696016989\\
97.09	0.00458401334319503\\
97.1	0.00458789589847034\\
97.11	0.0045917847076937\\
97.12	0.00459567985423455\\
97.13	0.00459958142316972\\
97.14	0.00460348950132003\\
97.15	0.00460740417728766\\
97.16	0.00461132554149443\\
97.17	0.00461525368622079\\
97.18	0.00461918870564574\\
97.19	0.00462313069588762\\
97.2	0.00462707975504575\\
97.21	0.00463103598324301\\
97.22	0.00463499948266935\\
97.23	0.00463897035762623\\
97.24	0.00464294871457207\\
97.25	0.00464693466216862\\
97.26	0.00465092831132848\\
97.27	0.00465492977526348\\
97.28	0.0046589391695343\\
97.29	0.00466295661210103\\
97.3	0.00466698222337493\\
97.31	0.0046710161262713\\
97.32	0.0046750584462635\\
97.33	0.00467910931143814\\
97.34	0.00468316885255156\\
97.35	0.00468723720308742\\
97.36	0.00469131449931571\\
97.37	0.0046954008803529\\
97.38	0.00469949648822355\\
97.39	0.00470360146792318\\
97.4	0.0047077159674826\\
97.41	0.00471184013803356\\
97.42	0.00471597413387596\\
97.43	0.00472011811254644\\
97.44	0.00472427223488858\\
97.45	0.00472843666512455\\
97.46	0.00473261157092841\\
97.47	0.004736797123501\\
97.48	0.00474099349764651\\
97.49	0.00474520087185073\\
97.5	0.00474941942836098\\
97.51	0.00475364935326792\\
97.52	0.00475789083658906\\
97.53	0.0047621440723542\\
97.54	0.00476640925869272\\
97.55	0.0047706865979228\\
97.56	0.00477497629664268\\
97.57	0.00477927856582385\\
97.58	0.00478359362090641\\
97.59	0.00478792168189649\\
97.6	0.00479226297346583\\
97.61	0.00479661772505364\\
97.62	0.00480098617097063\\
97.63	0.00480536855050547\\
97.64	0.0048097651080335\\
97.65	0.00481417609312799\\
97.66	0.00481860176067379\\
97.67	0.00482304237098356\\
97.68	0.0048274981899166\\
97.69	0.0048319694890003\\
97.7	0.00483645654555951\\
97.71	0.00484095964284813\\
97.72	0.0048454790701796\\
97.73	0.00485001512306028\\
97.74	0.00485456810332584\\
97.75	0.00485913831928074\\
97.76	0.00486372608584086\\
97.77	0.00486833172467939\\
97.78	0.00487295556437595\\
97.79	0.00487759794056912\\
97.8	0.00488225919611246\\
97.81	0.00488693968123404\\
97.82	0.00489163975369956\\
97.83	0.00489635977897926\\
97.84	0.00490110013041851\\
97.85	0.00490586118941243\\
97.86	0.00491064334558433\\
97.87	0.00491544699696836\\
97.88	0.0049202725501962\\
97.89	0.00492512042068815\\
97.9	0.00492999103284847\\
97.91	0.00493488482026528\\
97.92	0.00493980222591496\\
97.93	0.00494474370237131\\
97.94	0.00494970971201947\\
97.95	0.00495470072727479\\
97.96	0.00495971723080667\\
97.97	0.00496475971576765\\
97.98	0.00496982868602769\\
97.99	0.00497492465641393\\
98	0.0049800481529559\\
98.01	0.00498519971313656\\
98.02	0.00499037988614894\\
98.03	0.00499558923315892\\
98.04	0.005000828327574\\
98.05	0.00500609775531837\\
98.06	0.00501139811511431\\
98.07	0.00501673001877028\\
98.08	0.00502209409147554\\
98.09	0.00502749097210186\\
98.1	0.00503292131351209\\
98.11	0.0050383857828761\\
98.12	0.00504388506199408\\
98.13	0.00504941984762739\\
98.14	0.00505499085183724\\
98.15	0.00506059880233135\\
98.16	0.00506624444281872\\
98.17	0.00507192853337287\\
98.18	0.00507765185080359\\
98.19	0.00508341518903751\\
98.2	0.00508921935950777\\
98.21	0.00509506519155282\\
98.22	0.00510095353282481\\
98.23	0.0051068852497077\\
98.24	0.00511286122774527\\
98.25	0.0051188823720795\\
98.26	0.00512494960789931\\
98.27	0.00513106388090016\\
98.28	0.00513722615775466\\
98.29	0.00514343742659451\\
98.3	0.00514969869750407\\
98.31	0.00515601100302584\\
98.32	0.00516237539867824\\
98.33	0.00516879296348582\\
98.34	0.00517526480052249\\
98.35	0.00518179203746792\\
98.36	0.00518837582717745\\
98.37	0.00519501734826607\\
98.38	0.00520171780570658\\
98.39	0.00520847843144248\\
98.4	0.00521530048501596\\
98.41	0.00522218525421132\\
98.42	0.00522913405571431\\
98.43	0.00523614823578784\\
98.44	0.00524322917096434\\
98.45	0.00525037826875549\\
98.46	0.00525759696837952\\
98.47	0.00526488674150674\\
98.48	0.00527224909302379\\
98.49	0.00527968556181697\\
98.5	0.0052871977215754\\
98.51	0.00529478718161438\\
98.52	0.00530245558771964\\
98.53	0.00531020462301297\\
98.54	0.00531803600883986\\
98.55	0.00532595150567984\\
98.56	0.00533395291408008\\
98.57	0.00534204207561288\\
98.58	0.00535022087385788\\
98.59	0.00535849123540955\\
98.6	0.00536685513091079\\
98.61	0.00537531457611329\\
98.62	0.00538387163296558\\
98.63	0.00539252841072941\\
98.64	0.00540128706712541\\
98.65	0.00541014980950888\\
98.66	0.0054191188960765\\
98.67	0.00542819663710504\\
98.68	0.00543738539622288\\
98.69	0.00544668759171546\\
98.7	0.00545610569786551\\
98.71	0.0054656422463293\\
98.72	0.00547529982754992\\
98.73	0.00548508109220868\\
98.74	0.00549498875271589\\
98.75	0.0055050255847422\\
98.76	0.00551519442879178\\
98.77	0.00552549819181854\\
98.78	0.005535939848887\\
98.79	0.0055465224448789\\
98.8	0.00555724909624725\\
98.81	0.00556812299281921\\
98.82	0.00557914739964944\\
98.83	0.00559032565892547\\
98.84	0.00560166119192689\\
98.85	0.00561315750103999\\
98.86	0.00562481817182983\\
98.87	0.00563664687517144\\
98.88	0.00564864736944234\\
98.89	0.00566082350281313\\
98.9	0.00567317921560775\\
98.91	0.00568571854269937\\
98.92	0.00569844561597666\\
98.93	0.00571136466688288\\
98.94	0.00572448002903031\\
98.95	0.00573779614089264\\
98.96	0.0057513175485781\\
98.97	0.00576504890868555\\
98.98	0.00577899499124521\\
98.99	0.00579316068275143\\
99	0.00580755098928834\\
99.01	0.00582217103975174\\
99.02	0.00583702608917074\\
99.03	0.00585212152213285\\
99.04	0.00586746285631636\\
99.05	0.00588305574613397\\
99.06	0.00589890598649193\\
99.07	0.00591501951666894\\
99.08	0.00593140242431951\\
99.09	0.00594806094960653\\
99.1	0.0059650014894658\\
99.11	0.00598223060201175\\
99.12	0.00599975501108825\\
99.13	0.00601758161096993\\
99.14	0.0060357174712204\\
99.15	0.00605416984171331\\
99.16	0.00607294615782251\\
99.17	0.00609205404578594\\
99.18	0.00611150132825341\\
99.19	0.00613129603002837\\
99.2	0.00615144638400911\\
99.21	0.00617196083733744\\
99.22	0.006192848057764\\
99.23	0.00621411694023918\\
99.24	0.0062357766137396\\
99.25	0.00625783644834031\\
99.26	0.00628030606254335\\
99.27	0.0063031953308743\\
99.28	0.00632651439175847\\
99.29	0.0063502736556895\\
99.3	0.00637448381370353\\
99.31	0.00639915586344442\\
99.32	0.00642430111417673\\
99.33	0.00644993118405764\\
99.34	0.00647605801001219\\
99.35	0.00650269385799353\\
99.36	0.00652985133364663\\
99.37	0.00655754339339478\\
99.38	0.00658574111450058\\
99.39	0.0066144340019461\\
99.4	0.00664363418934559\\
99.41	0.00667335415000014\\
99.42	0.00670360670988301\\
99.43	0.00673440507738807\\
99.44	0.00676574324691763\\
99.45	0.00679761637399577\\
99.46	0.0068300377389763\\
99.47	0.00686302100297037\\
99.48	0.00689658022086231\\
99.49	0.00693072985487318\\
99.5	0.00696548478869991\\
99.51	0.00700086034226031\\
99.52	0.00703687228707563\\
99.53	0.00707353686232452\\
99.54	0.00711087079160439\\
99.55	0.00714889130043846\\
99.56	0.00718761613456919\\
99.57	0.00722706357908156\\
99.58	0.00726725247840242\\
99.59	0.00730820225722517\\
99.6	0.00734993294241264\\
99.61	0.00739246518593405\\
99.62	0.00743582028889647\\
99.63	0.0074800202267347\\
99.64	0.00752508767562849\\
99.65	0.0075710460402206\\
99.66	0.00761791948271452\\
99.67	0.00766573295343635\\
99.68	0.00771451222295164\\
99.69	0.00776428391583597\\
99.7	0.00781507554620144\\
99.71	0.00786691555509222\\
99.72	0.00791983334987064\\
99.73	0.00797385934572439\\
99.74	0.00802902500943545\\
99.75	0.00808536290556256\\
99.76	0.00814290674520107\\
99.77	0.00820169143749746\\
99.78	0.00826175314411004\\
99.79	0.00832312933682349\\
99.8	0.00838585885854211\\
99.81	0.00844998198790592\\
99.82	0.00851554050779452\\
99.83	0.00858257777800699\\
99.84	0.00865113881243126\\
99.85	0.00872127036104468\\
99.86	0.00879302099711813\\
99.87	0.00886644121003055\\
99.88	0.00894158350413849\\
99.89	0.00901850250418713\\
99.9	0.00909725506779586\\
99.91	0.00917790040560339\\
99.92	0.00926050020971464\\
99.93	0.00934511879115603\\
99.94	0.0094318232271174\\
99.95	0.00952068351883868\\
99.96	0.00961177276108944\\
99.97	0.00970516732428955\\
99.98	0.00980094705043254\\
99.99	0.00989919546410015\\
100	0.01\\
};
\addlegendentry{$q=0$};

\addplot [color=blue,solid,forget plot]
  table[row sep=crcr]{%
0.01	0.00597637174049999\\
0.02	0.00597637265910895\\
0.03	0.00597637358204436\\
0.04	0.00597637450934478\\
0.05	0.00597637544104841\\
0.06	0.00597637637719302\\
0.07	0.00597637731781594\\
0.08	0.00597637826295398\\
0.09	0.00597637921264341\\
0.1	0.00597638016691989\\
0.11	0.00597638112581843\\
0.12	0.00597638208937334\\
0.13	0.00597638305761813\\
0.14	0.00597638403058552\\
0.15	0.00597638500830732\\
0.16	0.00597638599081439\\
0.17	0.00597638697813656\\
0.18	0.00597638797030259\\
0.19	0.00597638896734004\\
0.2	0.00597638996927523\\
0.21	0.00597639097613318\\
0.22	0.00597639198793746\\
0.23	0.00597639300471018\\
0.24	0.00597639402647183\\
0.25	0.00597639505324124\\
0.26	0.00597639608503544\\
0.27	0.00597639712186959\\
0.28	0.00597639816375683\\
0.29	0.00597639921070823\\
0.3	0.00597640026273261\\
0.31	0.00597640131983647\\
0.32	0.00597640238202383\\
0.33	0.00597640344929612\\
0.34	0.00597640452165204\\
0.35	0.00597640559908738\\
0.36	0.00597640668159496\\
0.37	0.00597640776916439\\
0.38	0.00597640886178196\\
0.39	0.00597640995943046\\
0.4	0.00597641106208899\\
0.41	0.00597641216973284\\
0.42	0.00597641328233323\\
0.43	0.00597641439985716\\
0.44	0.00597641552226723\\
0.45	0.00597641664952137\\
0.46	0.00597641778157266\\
0.47	0.00597641891836915\\
0.48	0.00597642005985354\\
0.49	0.005976421205963\\
0.5	0.00597642235662889\\
0.51	0.00597642351177652\\
0.52	0.00597642467132487\\
0.53	0.00597642583518631\\
0.54	0.00597642700326631\\
0.55	0.00597642817546314\\
0.56	0.00597642935166754\\
0.57	0.00597643053176243\\
0.58	0.00597643171562254\\
0.59	0.00597643290311408\\
0.6	0.00597643409409436\\
0.61	0.00597643528841144\\
0.62	0.00597643648590371\\
0.63	0.0059764376863995\\
0.64	0.00597643888971667\\
0.65	0.00597644009566217\\
0.66	0.00597644130403156\\
0.67	0.00597644251460858\\
0.68	0.00597644372716466\\
0.69	0.0059764449414584\\
0.7	0.00597644615723506\\
0.71	0.00597644737422602\\
0.72	0.00597644859214821\\
0.73	0.00597644981070357\\
0.74	0.00597645102972546\\
0.75	0.00597645224921406\\
0.76	0.00597645346916958\\
0.77	0.00597645468959222\\
0.78	0.00597645591048215\\
0.79	0.00597645713183958\\
0.8	0.0059764583536647\\
0.81	0.0059764595759577\\
0.82	0.00597646079871879\\
0.83	0.00597646202194815\\
0.84	0.00597646324564598\\
0.85	0.00597646446981247\\
0.86	0.00597646569444782\\
0.87	0.00597646691955223\\
0.88	0.00597646814512588\\
0.89	0.00597646937116898\\
0.9	0.00597647059768171\\
0.91	0.00597647182466428\\
0.92	0.00597647305211687\\
0.93	0.0059764742800397\\
0.94	0.00597647550843293\\
0.95	0.0059764767372968\\
0.96	0.00597647796663146\\
0.97	0.00597647919643713\\
0.98	0.00597648042671402\\
0.99	0.00597648165746229\\
1	0.00597648288868216\\
1.01	0.00597648412037382\\
1.02	0.00597648535253747\\
1.03	0.00597648658517331\\
1.04	0.00597648781828152\\
1.05	0.00597648905186231\\
1.06	0.00597649028591588\\
1.07	0.00597649152044241\\
1.08	0.00597649275544212\\
1.09	0.00597649399091518\\
1.1	0.00597649522686181\\
1.11	0.0059764964632822\\
1.12	0.00597649770017654\\
1.13	0.00597649893754503\\
1.14	0.00597650017538788\\
1.15	0.00597650141370527\\
1.16	0.00597650265249741\\
1.17	0.0059765038917645\\
1.18	0.00597650513150672\\
1.19	0.00597650637172428\\
1.2	0.00597650761241738\\
1.21	0.00597650885358622\\
1.22	0.00597651009523099\\
1.23	0.00597651133735189\\
1.24	0.00597651257994912\\
1.25	0.00597651382302288\\
1.26	0.00597651506657336\\
1.27	0.00597651631060078\\
1.28	0.00597651755510532\\
1.29	0.00597651880008718\\
1.3	0.00597652004554656\\
1.31	0.00597652129148367\\
1.32	0.00597652253789869\\
1.33	0.00597652378479184\\
1.34	0.00597652503216331\\
1.35	0.00597652628001329\\
1.36	0.005976527528342\\
1.37	0.00597652877714962\\
1.38	0.00597653002643637\\
1.39	0.00597653127620243\\
1.4	0.005976532526448\\
1.41	0.0059765337771733\\
1.42	0.00597653502837852\\
1.43	0.00597653628006385\\
1.44	0.0059765375322295\\
1.45	0.00597653878487567\\
1.46	0.00597654003800256\\
1.47	0.00597654129161037\\
1.48	0.00597654254569931\\
1.49	0.00597654380026957\\
1.5	0.00597654505532135\\
1.51	0.00597654631085485\\
1.52	0.00597654756687028\\
1.53	0.00597654882336783\\
1.54	0.00597655008034772\\
1.55	0.00597655133781013\\
1.56	0.00597655259575528\\
1.57	0.00597655385418335\\
1.58	0.00597655511309456\\
1.59	0.00597655637248911\\
1.6	0.0059765576323672\\
1.61	0.00597655889272902\\
1.62	0.00597656015357479\\
1.63	0.0059765614149047\\
1.64	0.00597656267671897\\
1.65	0.00597656393901778\\
1.66	0.00597656520180134\\
1.67	0.00597656646506986\\
1.68	0.00597656772882353\\
1.69	0.00597656899306256\\
1.7	0.00597657025778717\\
1.71	0.00597657152299753\\
1.72	0.00597657278869386\\
1.73	0.00597657405487637\\
1.74	0.00597657532154525\\
1.75	0.00597657658870072\\
1.76	0.00597657785634296\\
1.77	0.00597657912447219\\
1.78	0.00597658039308862\\
1.79	0.00597658166219243\\
1.8	0.00597658293178385\\
1.81	0.00597658420186306\\
1.82	0.00597658547243029\\
1.83	0.00597658674348573\\
1.84	0.00597658801502958\\
1.85	0.00597658928706205\\
1.86	0.00597659055958334\\
1.87	0.00597659183259366\\
1.88	0.00597659310609322\\
1.89	0.00597659438008221\\
1.9	0.00597659565456085\\
1.91	0.00597659692952934\\
1.92	0.00597659820498787\\
1.93	0.00597659948093667\\
1.94	0.00597660075737594\\
1.95	0.00597660203430587\\
1.96	0.00597660331172668\\
1.97	0.00597660458963857\\
1.98	0.00597660586804174\\
1.99	0.00597660714693641\\
2	0.00597660842632278\\
2.01	0.00597660970620105\\
2.02	0.00597661098657143\\
2.03	0.00597661226743412\\
2.04	0.00597661354878935\\
2.05	0.0059766148306373\\
2.06	0.00597661611297819\\
2.07	0.00597661739581222\\
2.08	0.0059766186791396\\
2.09	0.00597661996296054\\
2.1	0.00597662124727524\\
2.11	0.00597662253208392\\
2.12	0.00597662381738677\\
2.13	0.005976625103184\\
2.14	0.00597662638947583\\
2.15	0.00597662767626247\\
2.16	0.00597662896354411\\
2.17	0.00597663025132096\\
2.18	0.00597663153959323\\
2.19	0.00597663282836114\\
2.2	0.00597663411762489\\
2.21	0.00597663540738469\\
2.22	0.00597663669764075\\
2.23	0.00597663798839326\\
2.24	0.00597663927964245\\
2.25	0.00597664057138853\\
2.26	0.00597664186363169\\
2.27	0.00597664315637214\\
2.28	0.00597664444961011\\
2.29	0.0059766457433458\\
2.3	0.0059766470375794\\
2.31	0.00597664833231115\\
2.32	0.00597664962754124\\
2.33	0.00597665092326988\\
2.34	0.00597665221949728\\
2.35	0.00597665351622366\\
2.36	0.00597665481344921\\
2.37	0.00597665611117416\\
2.38	0.00597665740939872\\
2.39	0.00597665870812308\\
2.4	0.00597666000734747\\
2.41	0.00597666130707209\\
2.42	0.00597666260729715\\
2.43	0.00597666390802287\\
2.44	0.00597666520924944\\
2.45	0.00597666651097709\\
2.46	0.00597666781320603\\
2.47	0.00597666911593646\\
2.48	0.0059766704191686\\
2.49	0.00597667172290265\\
2.5	0.00597667302713884\\
2.51	0.00597667433187736\\
2.52	0.00597667563711844\\
2.53	0.00597667694286228\\
2.54	0.00597667824910909\\
2.55	0.00597667955585909\\
2.56	0.00597668086311249\\
2.57	0.00597668217086949\\
2.58	0.00597668347913032\\
2.59	0.00597668478789518\\
2.6	0.00597668609716429\\
2.61	0.00597668740693786\\
2.62	0.0059766887172161\\
2.63	0.00597669002799922\\
2.64	0.00597669133928744\\
2.65	0.00597669265108096\\
2.66	0.00597669396338001\\
2.67	0.00597669527618479\\
2.68	0.00597669658949552\\
2.69	0.00597669790331241\\
2.7	0.00597669921763568\\
2.71	0.00597670053246553\\
2.72	0.00597670184780218\\
2.73	0.00597670316364585\\
2.74	0.00597670447999675\\
2.75	0.00597670579685509\\
2.76	0.00597670711422108\\
2.77	0.00597670843209494\\
2.78	0.00597670975047689\\
2.79	0.00597671106936713\\
2.8	0.00597671238876589\\
2.81	0.00597671370867338\\
2.82	0.00597671502908981\\
2.83	0.00597671635001539\\
2.84	0.00597671767145035\\
2.85	0.00597671899339489\\
2.86	0.00597672031584923\\
2.87	0.00597672163881359\\
2.88	0.00597672296228818\\
2.89	0.00597672428627322\\
2.9	0.00597672561076892\\
2.91	0.00597672693577549\\
2.92	0.00597672826129316\\
2.93	0.00597672958732214\\
2.94	0.00597673091386264\\
2.95	0.00597673224091489\\
2.96	0.00597673356847909\\
2.97	0.00597673489655546\\
2.98	0.00597673622514422\\
2.99	0.00597673755424559\\
3	0.00597673888385978\\
3.01	0.00597674021398701\\
3.02	0.00597674154462749\\
3.03	0.00597674287578145\\
3.04	0.0059767442074491\\
3.05	0.00597674553963065\\
3.06	0.00597674687232633\\
3.07	0.00597674820553634\\
3.08	0.00597674953926092\\
3.09	0.00597675087350027\\
3.1	0.00597675220825461\\
3.11	0.00597675354352417\\
3.12	0.00597675487930915\\
3.13	0.00597675621560978\\
3.14	0.00597675755242627\\
3.15	0.00597675888975885\\
3.16	0.00597676022760773\\
3.17	0.00597676156597313\\
3.18	0.00597676290485526\\
3.19	0.00597676424425436\\
3.2	0.00597676558417063\\
3.21	0.00597676692460429\\
3.22	0.00597676826555556\\
3.23	0.00597676960702467\\
3.24	0.00597677094901183\\
3.25	0.00597677229151725\\
3.26	0.00597677363454117\\
3.27	0.0059767749780838\\
3.28	0.00597677632214535\\
3.29	0.00597677766672605\\
3.3	0.00597677901182612\\
3.31	0.00597678035744578\\
3.32	0.00597678170358525\\
3.33	0.00597678305024474\\
3.34	0.00597678439742448\\
3.35	0.0059767857451247\\
3.36	0.0059767870933456\\
3.37	0.00597678844208741\\
3.38	0.00597678979135035\\
3.39	0.00597679114113465\\
3.4	0.00597679249144051\\
3.41	0.00597679384226817\\
3.42	0.00597679519361784\\
3.43	0.00597679654548975\\
3.44	0.00597679789788412\\
3.45	0.00597679925080116\\
3.46	0.0059768006042411\\
3.47	0.00597680195820417\\
3.48	0.00597680331269057\\
3.49	0.00597680466770055\\
3.5	0.00597680602323431\\
3.51	0.00597680737929208\\
3.52	0.00597680873587408\\
3.53	0.00597681009298054\\
3.54	0.00597681145061167\\
3.55	0.0059768128087677\\
3.56	0.00597681416744885\\
3.57	0.00597681552665535\\
3.58	0.00597681688638742\\
3.59	0.00597681824664528\\
3.6	0.00597681960742915\\
3.61	0.00597682096873925\\
3.62	0.00597682233057583\\
3.63	0.00597682369293908\\
3.64	0.00597682505582924\\
3.65	0.00597682641924653\\
3.66	0.00597682778319118\\
3.67	0.0059768291476634\\
3.68	0.00597683051266343\\
3.69	0.00597683187819149\\
3.7	0.00597683324424779\\
3.71	0.00597683461083258\\
3.72	0.00597683597794606\\
3.73	0.00597683734558847\\
3.74	0.00597683871376002\\
3.75	0.00597684008246096\\
3.76	0.00597684145169149\\
3.77	0.00597684282145185\\
3.78	0.00597684419174226\\
3.79	0.00597684556256294\\
3.8	0.00597684693391412\\
3.81	0.00597684830579603\\
3.82	0.00597684967820889\\
3.83	0.00597685105115293\\
3.84	0.00597685242462837\\
3.85	0.00597685379863544\\
3.86	0.00597685517317437\\
3.87	0.00597685654824538\\
3.88	0.0059768579238487\\
3.89	0.00597685929998455\\
3.9	0.00597686067665316\\
3.91	0.00597686205385476\\
3.92	0.00597686343158958\\
3.93	0.00597686480985783\\
3.94	0.00597686618865976\\
3.95	0.00597686756799558\\
3.96	0.00597686894786552\\
3.97	0.00597687032826982\\
3.98	0.00597687170920869\\
3.99	0.00597687309068237\\
4	0.00597687447269108\\
4.01	0.00597687585523506\\
4.02	0.00597687723831451\\
4.03	0.00597687862192969\\
4.04	0.00597688000608082\\
4.05	0.00597688139076812\\
4.06	0.00597688277599182\\
4.07	0.00597688416175215\\
4.08	0.00597688554804934\\
4.09	0.00597688693488362\\
4.1	0.00597688832225521\\
4.11	0.00597688971016435\\
4.12	0.00597689109861127\\
4.13	0.00597689248759619\\
4.14	0.00597689387711934\\
4.15	0.00597689526718096\\
4.16	0.00597689665778127\\
4.17	0.0059768980489205\\
4.18	0.00597689944059889\\
4.19	0.00597690083281665\\
4.2	0.00597690222557403\\
4.21	0.00597690361887125\\
4.22	0.00597690501270855\\
4.23	0.00597690640708614\\
4.24	0.00597690780200427\\
4.25	0.00597690919746317\\
4.26	0.00597691059346306\\
4.27	0.00597691199000418\\
4.28	0.00597691338708675\\
4.29	0.00597691478471101\\
4.3	0.00597691618287719\\
4.31	0.00597691758158552\\
4.32	0.00597691898083623\\
4.33	0.00597692038062956\\
4.34	0.00597692178096574\\
4.35	0.00597692318184499\\
4.36	0.00597692458326755\\
4.37	0.00597692598523365\\
4.38	0.00597692738774353\\
4.39	0.00597692879079741\\
4.4	0.00597693019439553\\
4.41	0.00597693159853812\\
4.42	0.00597693300322541\\
4.43	0.00597693440845765\\
4.44	0.00597693581423504\\
4.45	0.00597693722055784\\
4.46	0.00597693862742628\\
4.47	0.00597694003484059\\
4.48	0.005976941442801\\
4.49	0.00597694285130774\\
4.5	0.00597694426036106\\
4.51	0.00597694566996117\\
4.52	0.00597694708010832\\
4.53	0.00597694849080274\\
4.54	0.00597694990204467\\
4.55	0.00597695131383433\\
4.56	0.00597695272617198\\
4.57	0.00597695413905783\\
4.58	0.00597695555249212\\
4.59	0.00597695696647509\\
4.6	0.00597695838100697\\
4.61	0.00597695979608799\\
4.62	0.0059769612117184\\
4.63	0.00597696262789843\\
4.64	0.00597696404462831\\
4.65	0.00597696546190827\\
4.66	0.00597696687973856\\
4.67	0.00597696829811941\\
4.68	0.00597696971705105\\
4.69	0.00597697113653373\\
4.7	0.00597697255656767\\
4.71	0.00597697397715312\\
4.72	0.0059769753982903\\
4.73	0.00597697681997946\\
4.74	0.00597697824222083\\
4.75	0.00597697966501465\\
4.76	0.00597698108836115\\
4.77	0.00597698251226058\\
4.78	0.00597698393671317\\
4.79	0.00597698536171916\\
4.8	0.00597698678727878\\
4.81	0.00597698821339226\\
4.82	0.00597698964005987\\
4.83	0.00597699106728181\\
4.84	0.00597699249505834\\
4.85	0.00597699392338969\\
4.86	0.0059769953522761\\
4.87	0.00597699678171781\\
4.88	0.00597699821171506\\
4.89	0.00597699964226808\\
4.9	0.00597700107337712\\
4.91	0.0059770025050424\\
4.92	0.00597700393726418\\
4.93	0.00597700537004269\\
4.94	0.00597700680337816\\
4.95	0.00597700823727084\\
4.96	0.00597700967172097\\
4.97	0.00597701110672878\\
4.98	0.00597701254229452\\
4.99	0.00597701397841842\\
5	0.00597701541510073\\
5.01	0.00597701685234169\\
5.02	0.00597701829014152\\
5.03	0.00597701972850049\\
5.04	0.00597702116741882\\
5.05	0.00597702260689675\\
5.06	0.00597702404693453\\
5.07	0.00597702548753239\\
5.08	0.00597702692869058\\
5.09	0.00597702837040934\\
5.1	0.00597702981268891\\
5.11	0.00597703125552953\\
5.12	0.00597703269893144\\
5.13	0.00597703414289488\\
5.14	0.00597703558742009\\
5.15	0.00597703703250733\\
5.16	0.00597703847815681\\
5.17	0.0059770399243688\\
5.18	0.00597704137114353\\
5.19	0.00597704281848124\\
5.2	0.00597704426638217\\
5.21	0.00597704571484657\\
5.22	0.00597704716387469\\
5.23	0.00597704861346675\\
5.24	0.00597705006362301\\
5.25	0.00597705151434371\\
5.26	0.00597705296562909\\
5.27	0.00597705441747939\\
5.28	0.00597705586989487\\
5.29	0.00597705732287574\\
5.3	0.00597705877642228\\
5.31	0.00597706023053471\\
5.32	0.00597706168521329\\
5.33	0.00597706314045825\\
5.34	0.00597706459626983\\
5.35	0.0059770660526483\\
5.36	0.00597706750959387\\
5.37	0.00597706896710682\\
5.38	0.00597707042518737\\
5.39	0.00597707188383576\\
5.4	0.00597707334305226\\
5.41	0.0059770748028371\\
5.42	0.00597707626319052\\
5.43	0.00597707772411277\\
5.44	0.0059770791856041\\
5.45	0.00597708064766475\\
5.46	0.00597708211029496\\
5.47	0.00597708357349499\\
5.48	0.00597708503726508\\
5.49	0.00597708650160547\\
5.5	0.00597708796651641\\
5.51	0.00597708943199815\\
5.52	0.00597709089805093\\
5.53	0.00597709236467499\\
5.54	0.0059770938318706\\
5.55	0.00597709529963798\\
5.56	0.0059770967679774\\
5.57	0.00597709823688909\\
5.58	0.0059770997063733\\
5.59	0.00597710117643028\\
5.6	0.00597710264706028\\
5.61	0.00597710411826355\\
5.62	0.00597710559004032\\
5.63	0.00597710706239086\\
5.64	0.0059771085353154\\
5.65	0.0059771100088142\\
5.66	0.0059771114828875\\
5.67	0.00597711295753556\\
5.68	0.00597711443275862\\
5.69	0.00597711590855692\\
5.7	0.00597711738493072\\
5.71	0.00597711886188027\\
5.72	0.00597712033940581\\
5.73	0.0059771218175076\\
5.74	0.00597712329618588\\
5.75	0.0059771247754409\\
5.76	0.00597712625527291\\
5.77	0.00597712773568216\\
5.78	0.0059771292166689\\
5.79	0.00597713069823339\\
5.8	0.00597713218037586\\
5.81	0.00597713366309657\\
5.82	0.00597713514639578\\
5.83	0.00597713663027373\\
5.84	0.00597713811473067\\
5.85	0.00597713959976684\\
5.86	0.00597714108538251\\
5.87	0.00597714257157793\\
5.88	0.00597714405835333\\
5.89	0.00597714554570899\\
5.9	0.00597714703364514\\
5.91	0.00597714852216203\\
5.92	0.00597715001125993\\
5.93	0.00597715150093908\\
5.94	0.00597715299119973\\
5.95	0.00597715448204213\\
5.96	0.00597715597346654\\
5.97	0.00597715746547321\\
5.98	0.00597715895806239\\
5.99	0.00597716045123434\\
6	0.00597716194498929\\
6.01	0.00597716343932752\\
6.02	0.00597716493424926\\
6.03	0.00597716642975478\\
6.04	0.00597716792584433\\
6.05	0.00597716942251816\\
6.06	0.00597717091977652\\
6.07	0.00597717241761967\\
6.08	0.00597717391604786\\
6.09	0.00597717541506134\\
6.1	0.00597717691466037\\
6.11	0.0059771784148452\\
6.12	0.0059771799156161\\
6.13	0.00597718141697329\\
6.14	0.00597718291891706\\
6.15	0.00597718442144764\\
6.16	0.0059771859245653\\
6.17	0.00597718742827029\\
6.18	0.00597718893256285\\
6.19	0.00597719043744326\\
6.2	0.00597719194291176\\
6.21	0.00597719344896861\\
6.22	0.00597719495561407\\
6.23	0.00597719646284839\\
6.24	0.00597719797067182\\
6.25	0.00597719947908462\\
6.26	0.00597720098808705\\
6.27	0.00597720249767936\\
6.28	0.00597720400786182\\
6.29	0.00597720551863466\\
6.3	0.00597720702999817\\
6.31	0.00597720854195258\\
6.32	0.00597721005449816\\
6.33	0.00597721156763517\\
6.34	0.00597721308136385\\
6.35	0.00597721459568447\\
6.36	0.00597721611059729\\
6.37	0.00597721762610255\\
6.38	0.00597721914220054\\
6.39	0.00597722065889148\\
6.4	0.00597722217617565\\
6.41	0.00597722369405331\\
6.42	0.00597722521252471\\
6.43	0.00597722673159011\\
6.44	0.00597722825124977\\
6.45	0.00597722977150394\\
6.46	0.00597723129235289\\
6.47	0.00597723281379688\\
6.48	0.00597723433583616\\
6.49	0.00597723585847099\\
6.5	0.00597723738170163\\
6.51	0.00597723890552835\\
6.52	0.00597724042995139\\
6.53	0.00597724195497103\\
6.54	0.00597724348058751\\
6.55	0.00597724500680111\\
6.56	0.00597724653361208\\
6.57	0.00597724806102067\\
6.58	0.00597724958902716\\
6.59	0.0059772511176318\\
6.6	0.00597725264683485\\
6.61	0.00597725417663658\\
6.62	0.00597725570703724\\
6.63	0.00597725723803709\\
6.64	0.0059772587696364\\
6.65	0.00597726030183542\\
6.66	0.00597726183463443\\
6.67	0.00597726336803368\\
6.68	0.00597726490203342\\
6.69	0.00597726643663393\\
6.7	0.00597726797183547\\
6.71	0.00597726950763829\\
6.72	0.00597727104404267\\
6.73	0.00597727258104885\\
6.74	0.00597727411865711\\
6.75	0.00597727565686771\\
6.76	0.00597727719568091\\
6.77	0.00597727873509698\\
6.78	0.00597728027511616\\
6.79	0.00597728181573874\\
6.8	0.00597728335696497\\
6.81	0.00597728489879513\\
6.82	0.00597728644122946\\
6.83	0.00597728798426823\\
6.84	0.00597728952791172\\
6.85	0.00597729107216017\\
6.86	0.00597729261701387\\
6.87	0.00597729416247306\\
6.88	0.00597729570853802\\
6.89	0.005977297255209\\
6.9	0.00597729880248629\\
6.91	0.00597730035037013\\
6.92	0.0059773018988608\\
6.93	0.00597730344795856\\
6.94	0.00597730499766367\\
6.95	0.00597730654797641\\
6.96	0.00597730809889702\\
6.97	0.0059773096504258\\
6.98	0.00597731120256299\\
6.99	0.00597731275530886\\
7	0.00597731430866369\\
7.01	0.00597731586262773\\
7.02	0.00597731741720126\\
7.03	0.00597731897238454\\
7.04	0.00597732052817783\\
7.05	0.00597732208458141\\
7.06	0.00597732364159553\\
7.07	0.00597732519922048\\
7.08	0.00597732675745651\\
7.09	0.00597732831630389\\
7.1	0.00597732987576289\\
7.11	0.00597733143583378\\
7.12	0.00597733299651683\\
7.13	0.0059773345578123\\
7.14	0.00597733611972046\\
7.15	0.00597733768224159\\
7.16	0.00597733924537594\\
7.17	0.00597734080912378\\
7.18	0.0059773423734854\\
7.19	0.00597734393846105\\
7.2	0.005977345504051\\
7.21	0.00597734707025553\\
7.22	0.0059773486370749\\
7.23	0.00597735020450938\\
7.24	0.00597735177255924\\
7.25	0.00597735334122476\\
7.26	0.0059773549105062\\
7.27	0.00597735648040382\\
7.28	0.00597735805091791\\
7.29	0.00597735962204873\\
7.3	0.00597736119379656\\
7.31	0.00597736276616165\\
7.32	0.00597736433914429\\
7.33	0.00597736591274475\\
7.34	0.00597736748696329\\
7.35	0.00597736906180018\\
7.36	0.00597737063725571\\
7.37	0.00597737221333014\\
7.38	0.00597737379002373\\
7.39	0.00597737536733677\\
7.4	0.00597737694526952\\
7.41	0.00597737852382226\\
7.42	0.00597738010299526\\
7.43	0.00597738168278879\\
7.44	0.00597738326320313\\
7.45	0.00597738484423854\\
7.46	0.0059773864258953\\
7.47	0.00597738800817369\\
7.48	0.00597738959107397\\
7.49	0.00597739117459642\\
7.5	0.0059773927587413\\
7.51	0.00597739434350891\\
7.52	0.0059773959288995\\
7.53	0.00597739751491336\\
7.54	0.00597739910155076\\
7.55	0.00597740068881196\\
7.56	0.00597740227669725\\
7.57	0.0059774038652069\\
7.58	0.00597740545434119\\
7.59	0.00597740704410038\\
7.6	0.00597740863448476\\
7.61	0.00597741022549459\\
7.62	0.00597741181713016\\
7.63	0.00597741340939174\\
7.64	0.0059774150022796\\
7.65	0.00597741659579402\\
7.66	0.00597741818993529\\
7.67	0.00597741978470365\\
7.68	0.00597742138009941\\
7.69	0.00597742297612283\\
7.7	0.00597742457277419\\
7.71	0.00597742617005377\\
7.72	0.00597742776796184\\
7.73	0.00597742936649868\\
7.74	0.00597743096566456\\
7.75	0.00597743256545977\\
7.76	0.00597743416588458\\
7.77	0.00597743576693927\\
7.78	0.00597743736862411\\
7.79	0.00597743897093939\\
7.8	0.00597744057388538\\
7.81	0.00597744217746236\\
7.82	0.0059774437816706\\
7.83	0.00597744538651039\\
7.84	0.005977446991982\\
7.85	0.00597744859808571\\
7.86	0.00597745020482181\\
7.87	0.00597745181219056\\
7.88	0.00597745342019225\\
7.89	0.00597745502882716\\
7.9	0.00597745663809557\\
7.91	0.00597745824799775\\
7.92	0.00597745985853399\\
7.93	0.00597746146970456\\
7.94	0.00597746308150975\\
7.95	0.00597746469394983\\
7.96	0.00597746630702509\\
7.97	0.0059774679207358\\
7.98	0.00597746953508225\\
7.99	0.00597747115006472\\
8	0.00597747276568349\\
8.01	0.00597747438193883\\
8.02	0.00597747599883103\\
8.03	0.00597747761636037\\
8.04	0.00597747923452713\\
8.05	0.0059774808533316\\
8.06	0.00597748247277405\\
8.07	0.00597748409285477\\
8.08	0.00597748571357403\\
8.09	0.00597748733493213\\
8.1	0.00597748895692934\\
8.11	0.00597749057956594\\
8.12	0.00597749220284223\\
8.13	0.00597749382675847\\
8.14	0.00597749545131496\\
8.15	0.00597749707651197\\
8.16	0.00597749870234979\\
8.17	0.0059775003288287\\
8.18	0.00597750195594899\\
8.19	0.00597750358371094\\
8.2	0.00597750521211484\\
8.21	0.00597750684116096\\
8.22	0.00597750847084959\\
8.23	0.00597751010118102\\
8.24	0.00597751173215553\\
8.25	0.0059775133637734\\
8.26	0.00597751499603492\\
8.27	0.00597751662894037\\
8.28	0.00597751826249004\\
8.29	0.00597751989668422\\
8.3	0.00597752153152318\\
8.31	0.00597752316700722\\
8.32	0.00597752480313662\\
8.33	0.00597752643991167\\
8.34	0.00597752807733264\\
8.35	0.00597752971539983\\
8.36	0.00597753135411353\\
8.37	0.00597753299347402\\
8.38	0.00597753463348158\\
8.39	0.00597753627413651\\
8.4	0.00597753791543908\\
8.41	0.00597753955738959\\
8.42	0.00597754119998833\\
8.43	0.00597754284323557\\
8.44	0.00597754448713161\\
8.45	0.00597754613167674\\
8.46	0.00597754777687125\\
8.47	0.00597754942271541\\
8.48	0.00597755106920952\\
8.49	0.00597755271635387\\
8.5	0.00597755436414875\\
8.51	0.00597755601259443\\
8.52	0.00597755766169122\\
8.53	0.0059775593114394\\
8.54	0.00597756096183926\\
8.55	0.00597756261289109\\
8.56	0.00597756426459518\\
8.57	0.00597756591695182\\
8.58	0.00597756756996129\\
8.59	0.00597756922362389\\
8.6	0.0059775708779399\\
8.61	0.00597757253290962\\
8.62	0.00597757418853334\\
8.63	0.00597757584481134\\
8.64	0.00597757750174392\\
8.65	0.00597757915933137\\
8.66	0.00597758081757397\\
8.67	0.00597758247647203\\
8.68	0.00597758413602582\\
8.69	0.00597758579623565\\
8.7	0.00597758745710179\\
8.71	0.00597758911862455\\
8.72	0.00597759078080422\\
8.73	0.00597759244364108\\
8.74	0.00597759410713543\\
8.75	0.00597759577128757\\
8.76	0.00597759743609777\\
8.77	0.00597759910156634\\
8.78	0.00597760076769357\\
8.79	0.00597760243447975\\
8.8	0.00597760410192518\\
8.81	0.00597760577003013\\
8.82	0.00597760743879492\\
8.83	0.00597760910821983\\
8.84	0.00597761077830516\\
8.85	0.0059776124490512\\
8.86	0.00597761412045824\\
8.87	0.00597761579252658\\
8.88	0.0059776174652565\\
8.89	0.00597761913864832\\
8.9	0.00597762081270231\\
8.91	0.00597762248741878\\
8.92	0.00597762416279802\\
8.93	0.00597762583884032\\
8.94	0.00597762751554597\\
8.95	0.00597762919291529\\
8.96	0.00597763087094855\\
8.97	0.00597763254964605\\
8.98	0.0059776342290081\\
8.99	0.00597763590903497\\
9	0.00597763758972698\\
9.01	0.00597763927108442\\
9.02	0.00597764095310758\\
9.03	0.00597764263579676\\
9.04	0.00597764431915225\\
9.05	0.00597764600317436\\
9.06	0.00597764768786337\\
9.07	0.0059776493732196\\
9.08	0.00597765105924332\\
9.09	0.00597765274593484\\
9.1	0.00597765443329446\\
9.11	0.00597765612132248\\
9.12	0.00597765781001919\\
9.13	0.00597765949938488\\
9.14	0.00597766118941987\\
9.15	0.00597766288012444\\
9.16	0.00597766457149889\\
9.17	0.00597766626354353\\
9.18	0.00597766795625865\\
9.19	0.00597766964964455\\
9.2	0.00597767134370153\\
9.21	0.00597767303842988\\
9.22	0.00597767473382992\\
9.23	0.00597767642990192\\
9.24	0.0059776781266462\\
9.25	0.00597767982406306\\
9.26	0.00597768152215279\\
9.27	0.00597768322091569\\
9.28	0.00597768492035207\\
9.29	0.00597768662046222\\
9.3	0.00597768832124645\\
9.31	0.00597769002270505\\
9.32	0.00597769172483832\\
9.33	0.00597769342764658\\
9.34	0.0059776951311301\\
9.35	0.00597769683528921\\
9.36	0.00597769854012419\\
9.37	0.00597770024563535\\
9.38	0.005977701951823\\
9.39	0.00597770365868743\\
9.4	0.00597770536622894\\
9.41	0.00597770707444784\\
9.42	0.00597770878334443\\
9.43	0.00597771049291901\\
9.44	0.00597771220317188\\
9.45	0.00597771391410335\\
9.46	0.00597771562571371\\
9.47	0.00597771733800328\\
9.48	0.00597771905097234\\
9.49	0.00597772076462122\\
9.5	0.0059777224789502\\
9.51	0.00597772419395959\\
9.52	0.0059777259096497\\
9.53	0.00597772762602083\\
9.54	0.00597772934307328\\
9.55	0.00597773106080736\\
9.56	0.00597773277922337\\
9.57	0.00597773449832162\\
9.58	0.0059777362181024\\
9.59	0.00597773793856603\\
9.6	0.0059777396597128\\
9.61	0.00597774138154303\\
9.62	0.00597774310405701\\
9.63	0.00597774482725506\\
9.64	0.00597774655113747\\
9.65	0.00597774827570455\\
9.66	0.00597775000095662\\
9.67	0.00597775172689396\\
9.68	0.00597775345351689\\
9.69	0.00597775518082572\\
9.7	0.00597775690882075\\
9.71	0.00597775863750228\\
9.72	0.00597776036687063\\
9.73	0.00597776209692608\\
9.74	0.00597776382766897\\
9.75	0.00597776555909959\\
9.76	0.00597776729121825\\
9.77	0.00597776902402525\\
9.78	0.0059777707575209\\
9.79	0.0059777724917055\\
9.8	0.00597777422657938\\
9.81	0.00597777596214282\\
9.82	0.00597777769839615\\
9.83	0.00597777943533966\\
9.84	0.00597778117297367\\
9.85	0.00597778291129848\\
9.86	0.00597778465031439\\
9.87	0.00597778639002173\\
9.88	0.0059777881304208\\
9.89	0.00597778987151189\\
9.9	0.00597779161329533\\
9.91	0.00597779335577143\\
9.92	0.00597779509894048\\
9.93	0.0059777968428028\\
9.94	0.00597779858735869\\
9.95	0.00597780033260848\\
9.96	0.00597780207855245\\
9.97	0.00597780382519093\\
9.98	0.00597780557252423\\
9.99	0.00597780732055265\\
10	0.0059778090692765\\
10.01	0.00597781081869609\\
10.02	0.00597781256881174\\
10.03	0.00597781431962374\\
10.04	0.00597781607113242\\
10.05	0.00597781782333808\\
10.06	0.00597781957624102\\
10.07	0.00597782132984157\\
10.08	0.00597782308414003\\
10.09	0.00597782483913672\\
10.1	0.00597782659483193\\
10.11	0.005977828351226\\
10.12	0.00597783010831921\\
10.13	0.00597783186611189\\
10.14	0.00597783362460434\\
10.15	0.00597783538379688\\
10.16	0.00597783714368982\\
10.17	0.00597783890428347\\
10.18	0.00597784066557814\\
10.19	0.00597784242757414\\
10.2	0.00597784419027178\\
10.21	0.00597784595367138\\
10.22	0.00597784771777325\\
10.23	0.00597784948257769\\
10.24	0.00597785124808502\\
10.25	0.00597785301429556\\
10.26	0.00597785478120961\\
10.27	0.00597785654882748\\
10.28	0.00597785831714949\\
10.29	0.00597786008617596\\
10.3	0.00597786185590718\\
10.31	0.00597786362634348\\
10.32	0.00597786539748517\\
10.33	0.00597786716933256\\
10.34	0.00597786894188596\\
10.35	0.00597787071514568\\
10.36	0.00597787248911204\\
10.37	0.00597787426378536\\
10.38	0.00597787603916593\\
10.39	0.00597787781525408\\
10.4	0.00597787959205013\\
10.41	0.00597788136955437\\
10.42	0.00597788314776713\\
10.43	0.00597788492668872\\
10.44	0.00597788670631945\\
10.45	0.00597788848665963\\
10.46	0.00597789026770958\\
10.47	0.00597789204946962\\
10.48	0.00597789383194005\\
10.49	0.00597789561512119\\
10.5	0.00597789739901334\\
10.51	0.00597789918361684\\
10.52	0.00597790096893198\\
10.53	0.00597790275495908\\
10.54	0.00597790454169847\\
10.55	0.00597790632915043\\
10.56	0.00597790811731531\\
10.57	0.0059779099061934\\
10.58	0.00597791169578502\\
10.59	0.00597791348609049\\
10.6	0.00597791527711011\\
10.61	0.00597791706884421\\
10.62	0.00597791886129309\\
10.63	0.00597792065445707\\
10.64	0.00597792244833646\\
10.65	0.00597792424293158\\
10.66	0.00597792603824274\\
10.67	0.00597792783427026\\
10.68	0.00597792963101444\\
10.69	0.00597793142847561\\
10.7	0.00597793322665407\\
10.71	0.00597793502555014\\
10.72	0.00597793682516413\\
10.73	0.00597793862549637\\
10.74	0.00597794042654715\\
10.75	0.00597794222831679\\
10.76	0.00597794403080562\\
10.77	0.00597794583401394\\
10.78	0.00597794763794206\\
10.79	0.0059779494425903\\
10.8	0.00597795124795898\\
10.81	0.0059779530540484\\
10.82	0.00597795486085888\\
10.83	0.00597795666839074\\
10.84	0.00597795847664428\\
10.85	0.00597796028561982\\
10.86	0.00597796209531767\\
10.87	0.00597796390573815\\
10.88	0.00597796571688157\\
10.89	0.00597796752874824\\
10.9	0.00597796934133847\\
10.91	0.00597797115465258\\
10.92	0.00597797296869089\\
10.93	0.00597797478345369\\
10.94	0.00597797659894131\\
10.95	0.00597797841515407\\
10.96	0.00597798023209226\\
10.97	0.0059779820497562\\
10.98	0.00597798386814622\\
10.99	0.00597798568726261\\
11	0.00597798750710569\\
11.01	0.00597798932767577\\
11.02	0.00597799114897316\\
11.03	0.00597799297099818\\
11.04	0.00597799479375114\\
11.05	0.00597799661723234\\
11.06	0.0059779984414421\\
11.07	0.00597800026638073\\
11.08	0.00597800209204855\\
11.09	0.00597800391844585\\
11.1	0.00597800574557296\\
11.11	0.00597800757343018\\
11.12	0.00597800940201782\\
11.13	0.0059780112313362\\
11.14	0.00597801306138562\\
11.15	0.00597801489216639\\
11.16	0.00597801672367882\\
11.17	0.00597801855592323\\
11.18	0.00597802038889991\\
11.19	0.00597802222260918\\
11.2	0.00597802405705136\\
11.21	0.00597802589222673\\
11.22	0.00597802772813563\\
11.23	0.00597802956477834\\
11.24	0.00597803140215519\\
11.25	0.00597803324026647\\
11.26	0.00597803507911249\\
11.27	0.00597803691869357\\
11.28	0.00597803875901001\\
11.29	0.00597804060006211\\
11.3	0.00597804244185019\\
11.31	0.00597804428437453\\
11.32	0.00597804612763547\\
11.33	0.00597804797163328\\
11.34	0.0059780498163683\\
11.35	0.0059780516618408\\
11.36	0.00597805350805111\\
11.37	0.00597805535499953\\
11.38	0.00597805720268635\\
11.39	0.00597805905111189\\
11.4	0.00597806090027644\\
11.41	0.00597806275018031\\
11.42	0.0059780646008238\\
11.43	0.00597806645220721\\
11.44	0.00597806830433084\\
11.45	0.00597807015719499\\
11.46	0.00597807201079997\\
11.47	0.00597807386514608\\
11.48	0.00597807572023361\\
11.49	0.00597807757606287\\
11.5	0.00597807943263414\\
11.51	0.00597808128994774\\
11.52	0.00597808314800396\\
11.53	0.00597808500680309\\
11.54	0.00597808686634543\\
11.55	0.00597808872663129\\
11.56	0.00597809058766096\\
11.57	0.00597809244943472\\
11.58	0.00597809431195289\\
11.59	0.00597809617521575\\
11.6	0.00597809803922358\\
11.61	0.0059780999039767\\
11.62	0.0059781017694754\\
11.63	0.00597810363571996\\
11.64	0.00597810550271067\\
11.65	0.00597810737044783\\
11.66	0.00597810923893173\\
11.67	0.00597811110816266\\
11.68	0.00597811297814091\\
11.69	0.00597811484886676\\
11.7	0.00597811672034051\\
11.71	0.00597811859256244\\
11.72	0.00597812046553284\\
11.73	0.00597812233925199\\
11.74	0.00597812421372019\\
11.75	0.00597812608893771\\
11.76	0.00597812796490484\\
11.77	0.00597812984162186\\
11.78	0.00597813171908907\\
11.79	0.00597813359730672\\
11.8	0.00597813547627512\\
11.81	0.00597813735599454\\
11.82	0.00597813923646525\\
11.83	0.00597814111768754\\
11.84	0.00597814299966169\\
11.85	0.00597814488238798\\
11.86	0.00597814676586667\\
11.87	0.00597814865009805\\
11.88	0.00597815053508238\\
11.89	0.00597815242081996\\
11.9	0.00597815430731103\\
11.91	0.00597815619455589\\
11.92	0.0059781580825548\\
11.93	0.00597815997130803\\
11.94	0.00597816186081584\\
11.95	0.00597816375107852\\
11.96	0.00597816564209632\\
11.97	0.00597816753386951\\
11.98	0.00597816942639836\\
11.99	0.00597817131968314\\
12	0.00597817321372409\\
12.01	0.00597817510852149\\
12.02	0.0059781770040756\\
12.03	0.00597817890038668\\
12.04	0.00597818079745499\\
12.05	0.00597818269528079\\
12.06	0.00597818459386432\\
12.07	0.00597818649320586\\
12.08	0.00597818839330564\\
12.09	0.00597819029416394\\
12.1	0.005978192195781\\
12.11	0.00597819409815706\\
12.12	0.00597819600129238\\
12.13	0.00597819790518722\\
12.14	0.00597819980984181\\
12.15	0.0059782017152564\\
12.16	0.00597820362143124\\
12.17	0.00597820552836657\\
12.18	0.00597820743606263\\
12.19	0.00597820934451967\\
12.2	0.00597821125373793\\
12.21	0.00597821316371763\\
12.22	0.00597821507445903\\
12.23	0.00597821698596235\\
12.24	0.00597821889822782\\
12.25	0.00597822081125569\\
12.26	0.00597822272504619\\
12.27	0.00597822463959953\\
12.28	0.00597822655491596\\
12.29	0.00597822847099569\\
12.3	0.00597823038783896\\
12.31	0.00597823230544598\\
12.32	0.00597823422381699\\
12.33	0.00597823614295219\\
12.34	0.00597823806285182\\
12.35	0.00597823998351607\\
12.36	0.00597824190494518\\
12.37	0.00597824382713936\\
12.38	0.00597824575009882\\
12.39	0.00597824767382377\\
12.4	0.00597824959831442\\
12.41	0.00597825152357097\\
12.42	0.00597825344959364\\
12.43	0.00597825537638262\\
12.44	0.00597825730393813\\
12.45	0.00597825923226035\\
12.46	0.00597826116134949\\
12.47	0.00597826309120575\\
12.48	0.00597826502182932\\
12.49	0.0059782669532204\\
12.5	0.00597826888537917\\
12.51	0.00597827081830582\\
12.52	0.00597827275200055\\
12.53	0.00597827468646354\\
12.54	0.00597827662169497\\
12.55	0.00597827855769502\\
12.56	0.00597828049446388\\
12.57	0.00597828243200172\\
12.58	0.00597828437030873\\
12.59	0.00597828630938506\\
12.6	0.0059782882492309\\
12.61	0.00597829018984641\\
12.62	0.00597829213123177\\
12.63	0.00597829407338713\\
12.64	0.00597829601631267\\
12.65	0.00597829796000854\\
12.66	0.00597829990447491\\
12.67	0.00597830184971192\\
12.68	0.00597830379571975\\
12.69	0.00597830574249853\\
12.7	0.00597830769004843\\
12.71	0.00597830963836958\\
12.72	0.00597831158746214\\
12.73	0.00597831353732626\\
12.74	0.00597831548796206\\
12.75	0.0059783174393697\\
12.76	0.00597831939154931\\
12.77	0.00597832134450103\\
12.78	0.00597832329822499\\
12.79	0.00597832525272133\\
12.8	0.00597832720799016\\
12.81	0.00597832916403162\\
12.82	0.00597833112084583\\
12.83	0.00597833307843292\\
12.84	0.005978335036793\\
12.85	0.0059783369959262\\
12.86	0.00597833895583262\\
12.87	0.00597834091651238\\
12.88	0.0059783428779656\\
12.89	0.00597834484019237\\
12.9	0.00597834680319281\\
12.91	0.00597834876696702\\
12.92	0.0059783507315151\\
12.93	0.00597835269683715\\
12.94	0.00597835466293327\\
12.95	0.00597835662980355\\
12.96	0.00597835859744808\\
12.97	0.00597836056586697\\
12.98	0.00597836253506028\\
12.99	0.00597836450502811\\
13	0.00597836647577054\\
13.01	0.00597836844728766\\
13.02	0.00597837041957953\\
13.03	0.00597837239264624\\
13.04	0.00597837436648787\\
13.05	0.00597837634110448\\
13.06	0.00597837831649615\\
13.07	0.00597838029266294\\
13.08	0.00597838226960491\\
13.09	0.00597838424732214\\
13.1	0.00597838622581468\\
13.11	0.0059783882050826\\
13.12	0.00597839018512594\\
13.13	0.00597839216594477\\
13.14	0.00597839414753914\\
13.15	0.00597839612990909\\
13.16	0.00597839811305469\\
13.17	0.00597840009697597\\
13.18	0.00597840208167298\\
13.19	0.00597840406714577\\
13.2	0.00597840605339437\\
13.21	0.00597840804041884\\
13.22	0.0059784100282192\\
13.23	0.00597841201679549\\
13.24	0.00597841400614775\\
13.25	0.005978415996276\\
13.26	0.00597841798718029\\
13.27	0.00597841997886064\\
13.28	0.00597842197131708\\
13.29	0.00597842396454963\\
13.3	0.00597842595855833\\
13.31	0.00597842795334321\\
13.32	0.00597842994890427\\
13.33	0.00597843194524154\\
13.34	0.00597843394235505\\
13.35	0.00597843594024482\\
13.36	0.00597843793891086\\
13.37	0.00597843993835319\\
13.38	0.00597844193857184\\
13.39	0.00597844393956681\\
13.4	0.00597844594133812\\
13.41	0.00597844794388579\\
13.42	0.00597844994720983\\
13.43	0.00597845195131027\\
13.44	0.00597845395618711\\
13.45	0.00597845596184036\\
13.46	0.00597845796827004\\
13.47	0.00597845997547617\\
13.48	0.00597846198345875\\
13.49	0.00597846399221781\\
13.5	0.00597846600175336\\
13.51	0.0059784680120654\\
13.52	0.00597847002315397\\
13.53	0.00597847203501907\\
13.54	0.00597847404766072\\
13.55	0.00597847606107894\\
13.56	0.00597847807527374\\
13.57	0.00597848009024516\\
13.58	0.0059784821059932\\
13.59	0.00597848412251789\\
13.6	0.00597848613981926\\
13.61	0.00597848815789733\\
13.62	0.00597849017675213\\
13.63	0.0059784921963837\\
13.64	0.00597849421679205\\
13.65	0.00597849623797723\\
13.66	0.00597849825993928\\
13.67	0.00597850028267822\\
13.68	0.00597850230619412\\
13.69	0.005978504330487\\
13.7	0.00597850635555692\\
13.71	0.00597850838140394\\
13.72	0.0059785104080281\\
13.73	0.00597851243542948\\
13.74	0.00597851446360812\\
13.75	0.0059785164925641\\
13.76	0.00597851852229749\\
13.77	0.00597852055280837\\
13.78	0.00597852258409681\\
13.79	0.00597852461616291\\
13.8	0.00597852664900675\\
13.81	0.00597852868262844\\
13.82	0.00597853071702807\\
13.83	0.00597853275220575\\
13.84	0.0059785347881616\\
13.85	0.00597853682489573\\
13.86	0.00597853886240828\\
13.87	0.00597854090069937\\
13.88	0.00597854293976915\\
13.89	0.00597854497961777\\
13.9	0.00597854702024537\\
13.91	0.00597854906165213\\
13.92	0.00597855110383821\\
13.93	0.0059785531468038\\
13.94	0.00597855519054907\\
13.95	0.00597855723507424\\
13.96	0.00597855928037949\\
13.97	0.00597856132646506\\
13.98	0.00597856337333115\\
13.99	0.00597856542097801\\
14	0.00597856746940587\\
14.01	0.005978569518615\\
14.02	0.00597857156860565\\
14.03	0.00597857361937811\\
14.04	0.00597857567093265\\
14.05	0.00597857772326959\\
14.06	0.00597857977638921\\
14.07	0.00597858183029186\\
14.08	0.00597858388497785\\
14.09	0.00597858594044754\\
14.1	0.00597858799670129\\
14.11	0.00597859005373946\\
14.12	0.00597859211156244\\
14.13	0.00597859417017063\\
14.14	0.00597859622956445\\
14.15	0.00597859828974431\\
14.16	0.00597860035071065\\
14.17	0.00597860241246394\\
14.18	0.00597860447500464\\
14.19	0.00597860653833324\\
14.2	0.00597860860245023\\
14.21	0.00597861066735613\\
14.22	0.00597861273305146\\
14.23	0.00597861479953679\\
14.24	0.00597861686681266\\
14.25	0.00597861893487966\\
14.26	0.00597862100373839\\
14.27	0.00597862307338944\\
14.28	0.00597862514383347\\
14.29	0.0059786272150711\\
14.3	0.005978629287103\\
14.31	0.00597863135992986\\
14.32	0.00597863343355236\\
14.33	0.00597863550797123\\
14.34	0.00597863758318721\\
14.35	0.00597863965920102\\
14.36	0.00597864173601346\\
14.37	0.0059786438136253\\
14.38	0.00597864589203736\\
14.39	0.00597864797125044\\
14.4	0.0059786500512654\\
14.41	0.00597865213208309\\
14.42	0.0059786542137044\\
14.43	0.0059786562961302\\
14.44	0.00597865837936143\\
14.45	0.00597866046339901\\
14.46	0.00597866254824388\\
14.47	0.00597866463389703\\
14.48	0.00597866672035942\\
14.49	0.00597866880763206\\
14.5	0.00597867089571598\\
14.51	0.00597867298461221\\
14.52	0.00597867507432179\\
14.53	0.00597867716484581\\
14.54	0.00597867925618534\\
14.55	0.0059786813483415\\
14.56	0.00597868344131538\\
14.57	0.00597868553510814\\
14.58	0.00597868762972092\\
14.59	0.00597868972515487\\
14.6	0.00597869182141117\\
14.61	0.00597869391849102\\
14.62	0.00597869601639562\\
14.63	0.00597869811512617\\
14.64	0.0059787002146839\\
14.65	0.00597870231507007\\
14.66	0.0059787044162859\\
14.67	0.00597870651833265\\
14.68	0.0059787086212116\\
14.69	0.00597871072492401\\
14.7	0.00597871282947117\\
14.71	0.00597871493485437\\
14.72	0.00597871704107488\\
14.73	0.00597871914813403\\
14.74	0.0059787212560331\\
14.75	0.0059787233647734\\
14.76	0.00597872547435624\\
14.77	0.00597872758478291\\
14.78	0.00597872969605473\\
14.79	0.005978731808173\\
14.8	0.00597873392113902\\
14.81	0.00597873603495408\\
14.82	0.00597873814961948\\
14.83	0.0059787402651365\\
14.84	0.00597874238150642\\
14.85	0.0059787444987305\\
14.86	0.00597874661681\\
14.87	0.00597874873574617\\
14.88	0.00597875085554024\\
14.89	0.00597875297619341\\
14.9	0.00597875509770691\\
14.91	0.0059787572200819\\
14.92	0.00597875934331957\\
14.93	0.00597876146742103\\
14.94	0.00597876359238744\\
14.95	0.00597876571821988\\
14.96	0.00597876784491944\\
14.97	0.00597876997248717\\
14.98	0.00597877210092409\\
14.99	0.0059787742302312\\
15	0.00597877636040949\\
15.01	0.00597877849145988\\
15.02	0.0059787806233833\\
15.03	0.00597878275618062\\
15.04	0.00597878488985269\\
15.05	0.00597878702440034\\
15.06	0.00597878915982434\\
15.07	0.00597879129612546\\
15.08	0.00597879343330441\\
15.09	0.00597879557136188\\
15.1	0.00597879771029853\\
15.11	0.00597879985011498\\
15.12	0.00597880199081183\\
15.13	0.00597880413238964\\
15.14	0.00597880627484894\\
15.15	0.00597880841819027\\
15.16	0.00597881056241409\\
15.17	0.00597881270752088\\
15.18	0.00597881485351108\\
15.19	0.00597881700038514\\
15.2	0.00597881914814348\\
15.21	0.00597882129678651\\
15.22	0.00597882344631466\\
15.23	0.00597882559672835\\
15.24	0.00597882774802799\\
15.25	0.00597882990021401\\
15.26	0.00597883205328682\\
15.27	0.00597883420724685\\
15.28	0.00597883636209451\\
15.29	0.00597883851783024\\
15.3	0.00597884067445444\\
15.31	0.00597884283196754\\
15.32	0.00597884499036997\\
15.33	0.00597884714966214\\
15.34	0.00597884930984448\\
15.35	0.00597885147091741\\
15.36	0.00597885363288135\\
15.37	0.00597885579573672\\
15.38	0.00597885795948395\\
15.39	0.00597886012412346\\
15.4	0.00597886228965567\\
15.41	0.00597886445608102\\
15.42	0.00597886662339991\\
15.43	0.00597886879161278\\
15.44	0.00597887096072005\\
15.45	0.00597887313072215\\
15.46	0.00597887530161951\\
15.47	0.00597887747341254\\
15.48	0.00597887964610166\\
15.49	0.00597888181968732\\
15.5	0.00597888399416993\\
15.51	0.00597888616954992\\
15.52	0.00597888834582772\\
15.53	0.00597889052300375\\
15.54	0.00597889270107844\\
15.55	0.00597889488005222\\
15.56	0.00597889705992552\\
15.57	0.00597889924069875\\
15.58	0.00597890142237236\\
15.59	0.00597890360494677\\
15.6	0.0059789057884224\\
15.61	0.00597890797279969\\
15.62	0.00597891015807906\\
15.63	0.00597891234426095\\
15.64	0.00597891453134578\\
15.65	0.00597891671933399\\
15.66	0.005978918908226\\
15.67	0.00597892109802224\\
15.68	0.00597892328872314\\
15.69	0.00597892548032914\\
15.7	0.00597892767284067\\
15.71	0.00597892986625815\\
15.72	0.00597893206058202\\
15.73	0.00597893425581271\\
15.74	0.00597893645195065\\
15.75	0.00597893864899627\\
15.76	0.00597894084695001\\
15.77	0.0059789430458123\\
15.78	0.00597894524558357\\
15.79	0.00597894744626425\\
15.8	0.00597894964785478\\
15.81	0.00597895185035559\\
15.82	0.00597895405376711\\
15.83	0.00597895625808979\\
15.84	0.00597895846332405\\
15.85	0.00597896066947032\\
15.86	0.00597896287652905\\
15.87	0.00597896508450067\\
15.88	0.00597896729338561\\
15.89	0.00597896950318431\\
15.9	0.0059789717138972\\
15.91	0.00597897392552473\\
15.92	0.00597897613806732\\
15.93	0.00597897835152542\\
15.94	0.00597898056589945\\
15.95	0.00597898278118986\\
15.96	0.00597898499739709\\
15.97	0.00597898721452157\\
15.98	0.00597898943256374\\
15.99	0.00597899165152403\\
16	0.0059789938714029\\
16.01	0.00597899609220077\\
16.02	0.00597899831391808\\
16.03	0.00597900053655527\\
16.04	0.00597900276011279\\
16.05	0.00597900498459107\\
16.06	0.00597900720999055\\
16.07	0.00597900943631167\\
16.08	0.00597901166355487\\
16.09	0.0059790138917206\\
16.1	0.00597901612080929\\
16.11	0.00597901835082138\\
16.12	0.00597902058175732\\
16.13	0.00597902281361754\\
16.14	0.00597902504640249\\
16.15	0.00597902728011262\\
16.16	0.00597902951474836\\
16.17	0.00597903175031014\\
16.18	0.00597903398679844\\
16.19	0.00597903622421367\\
16.2	0.00597903846255629\\
16.21	0.00597904070182673\\
16.22	0.00597904294202545\\
16.23	0.00597904518315288\\
16.24	0.00597904742520948\\
16.25	0.00597904966819568\\
16.26	0.00597905191211193\\
16.27	0.00597905415695867\\
16.28	0.00597905640273636\\
16.29	0.00597905864944543\\
16.3	0.00597906089708633\\
16.31	0.00597906314565951\\
16.32	0.00597906539516541\\
16.33	0.00597906764560449\\
16.34	0.00597906989697718\\
16.35	0.00597907214928394\\
16.36	0.00597907440252521\\
16.37	0.00597907665670143\\
16.38	0.00597907891181308\\
16.39	0.00597908116786057\\
16.4	0.00597908342484436\\
16.41	0.00597908568276491\\
16.42	0.00597908794162266\\
16.43	0.00597909020141806\\
16.44	0.00597909246215156\\
16.45	0.00597909472382361\\
16.46	0.00597909698643466\\
16.47	0.00597909924998516\\
16.48	0.00597910151447556\\
16.49	0.00597910377990631\\
16.5	0.00597910604627786\\
16.51	0.00597910831359066\\
16.52	0.00597911058184517\\
16.53	0.00597911285104184\\
16.54	0.00597911512118111\\
16.55	0.00597911739226344\\
16.56	0.00597911966428929\\
16.57	0.0059791219372591\\
16.58	0.00597912421117333\\
16.59	0.00597912648603243\\
16.6	0.00597912876183686\\
16.61	0.00597913103858706\\
16.62	0.0059791333162835\\
16.63	0.00597913559492662\\
16.64	0.00597913787451689\\
16.65	0.00597914015505475\\
16.66	0.00597914243654067\\
16.67	0.00597914471897509\\
16.68	0.00597914700235847\\
16.69	0.00597914928669127\\
16.7	0.00597915157197394\\
16.71	0.00597915385820695\\
16.72	0.00597915614539075\\
16.73	0.00597915843352578\\
16.74	0.00597916072261252\\
16.75	0.00597916301265142\\
16.76	0.00597916530364293\\
16.77	0.00597916759558752\\
16.78	0.00597916988848563\\
16.79	0.00597917218233774\\
16.8	0.0059791744771443\\
16.81	0.00597917677290576\\
16.82	0.0059791790696226\\
16.83	0.00597918136729526\\
16.84	0.0059791836659242\\
16.85	0.00597918596550989\\
16.86	0.00597918826605279\\
16.87	0.00597919056755336\\
16.88	0.00597919287001205\\
16.89	0.00597919517342934\\
16.9	0.00597919747780567\\
16.91	0.00597919978314151\\
16.92	0.00597920208943734\\
16.93	0.00597920439669359\\
16.94	0.00597920670491074\\
16.95	0.00597920901408925\\
16.96	0.00597921132422959\\
16.97	0.00597921363533221\\
16.98	0.00597921594739758\\
16.99	0.00597921826042617\\
17	0.00597922057441842\\
17.01	0.00597922288937483\\
17.02	0.00597922520529584\\
17.03	0.00597922752218192\\
17.04	0.00597922984003354\\
17.05	0.00597923215885116\\
17.06	0.00597923447863524\\
17.07	0.00597923679938626\\
17.08	0.00597923912110467\\
17.09	0.00597924144379094\\
17.1	0.00597924376744555\\
17.11	0.00597924609206896\\
17.12	0.00597924841766163\\
17.13	0.00597925074422403\\
17.14	0.00597925307175663\\
17.15	0.0059792554002599\\
17.16	0.00597925772973431\\
17.17	0.00597926006018032\\
17.18	0.0059792623915984\\
17.19	0.00597926472398902\\
17.2	0.00597926705735266\\
17.21	0.00597926939168977\\
17.22	0.00597927172700083\\
17.23	0.00597927406328632\\
17.24	0.00597927640054669\\
17.25	0.00597927873878243\\
17.26	0.00597928107799399\\
17.27	0.00597928341818186\\
17.28	0.0059792857593465\\
17.29	0.00597928810148839\\
17.3	0.005979290444608\\
17.31	0.00597929278870579\\
17.32	0.00597929513378225\\
17.33	0.00597929747983784\\
17.34	0.00597929982687303\\
17.35	0.00597930217488831\\
17.36	0.00597930452388414\\
17.37	0.005979306873861\\
17.38	0.00597930922481937\\
17.39	0.0059793115767597\\
17.4	0.00597931392968249\\
17.41	0.00597931628358821\\
17.42	0.00597931863847732\\
17.43	0.00597932099435031\\
17.44	0.00597932335120765\\
17.45	0.00597932570904982\\
17.46	0.00597932806787729\\
17.47	0.00597933042769054\\
17.48	0.00597933278849004\\
17.49	0.00597933515027628\\
17.5	0.00597933751304973\\
17.51	0.00597933987681087\\
17.52	0.00597934224156017\\
17.53	0.00597934460729812\\
17.54	0.00597934697402519\\
17.55	0.00597934934174186\\
17.56	0.00597935171044861\\
17.57	0.00597935408014591\\
17.58	0.00597935645083426\\
17.59	0.00597935882251412\\
17.6	0.00597936119518599\\
17.61	0.00597936356885032\\
17.62	0.00597936594350762\\
17.63	0.00597936831915836\\
17.64	0.00597937069580301\\
17.65	0.00597937307344207\\
17.66	0.00597937545207601\\
17.67	0.00597937783170531\\
17.68	0.00597938021233046\\
17.69	0.00597938259395194\\
17.7	0.00597938497657023\\
17.71	0.00597938736018581\\
17.72	0.00597938974479918\\
17.73	0.0059793921304108\\
17.74	0.00597939451702117\\
17.75	0.00597939690463077\\
17.76	0.00597939929324008\\
17.77	0.00597940168284959\\
17.78	0.00597940407345978\\
17.79	0.00597940646507114\\
17.8	0.00597940885768415\\
17.81	0.0059794112512993\\
17.82	0.00597941364591708\\
17.83	0.00597941604153797\\
17.84	0.00597941843816245\\
17.85	0.00597942083579101\\
17.86	0.00597942323442415\\
17.87	0.00597942563406235\\
17.88	0.00597942803470609\\
17.89	0.00597943043635587\\
17.9	0.00597943283901217\\
17.91	0.00597943524267547\\
17.92	0.00597943764734628\\
17.93	0.00597944005302507\\
17.94	0.00597944245971235\\
17.95	0.00597944486740858\\
17.96	0.00597944727611427\\
17.97	0.00597944968582992\\
17.98	0.00597945209655599\\
17.99	0.005979454508293\\
18	0.00597945692104142\\
18.01	0.00597945933480175\\
18.02	0.00597946174957448\\
18.03	0.00597946416536011\\
18.04	0.00597946658215911\\
18.05	0.005979468999972\\
18.06	0.00597947141879925\\
18.07	0.00597947383864137\\
18.08	0.00597947625949884\\
18.09	0.00597947868137215\\
18.1	0.00597948110426182\\
18.11	0.00597948352816831\\
18.12	0.00597948595309213\\
18.13	0.00597948837903379\\
18.14	0.00597949080599375\\
18.15	0.00597949323397253\\
18.16	0.00597949566297063\\
18.17	0.00597949809298853\\
18.18	0.00597950052402672\\
18.19	0.00597950295608572\\
18.2	0.00597950538916601\\
18.21	0.00597950782326809\\
18.22	0.00597951025839245\\
18.23	0.00597951269453961\\
18.24	0.00597951513171004\\
18.25	0.00597951756990426\\
18.26	0.00597952000912275\\
18.27	0.00597952244936601\\
18.28	0.00597952489063456\\
18.29	0.00597952733292887\\
18.3	0.00597952977624946\\
18.31	0.00597953222059682\\
18.32	0.00597953466597145\\
18.33	0.00597953711237386\\
18.34	0.00597953955980454\\
18.35	0.00597954200826399\\
18.36	0.00597954445775272\\
18.37	0.00597954690827122\\
18.38	0.00597954935982\\
18.39	0.00597955181239955\\
18.4	0.00597955426601039\\
18.41	0.00597955672065301\\
18.42	0.00597955917632791\\
18.43	0.00597956163303561\\
18.44	0.00597956409077659\\
18.45	0.00597956654955137\\
18.46	0.00597956900936045\\
18.47	0.00597957147020433\\
18.48	0.00597957393208352\\
18.49	0.00597957639499852\\
18.5	0.00597957885894984\\
18.51	0.00597958132393797\\
18.52	0.00597958378996344\\
18.53	0.00597958625702673\\
18.54	0.00597958872512837\\
18.55	0.00597959119426885\\
18.56	0.00597959366444868\\
18.57	0.00597959613566837\\
18.58	0.00597959860792842\\
18.59	0.00597960108122935\\
18.6	0.00597960355557166\\
18.61	0.00597960603095585\\
18.62	0.00597960850738244\\
18.63	0.00597961098485194\\
18.64	0.00597961346336485\\
18.65	0.00597961594292168\\
18.66	0.00597961842352294\\
18.67	0.00597962090516915\\
18.68	0.00597962338786081\\
18.69	0.00597962587159843\\
18.7	0.00597962835638252\\
18.71	0.0059796308422136\\
18.72	0.00597963332909216\\
18.73	0.00597963581701874\\
18.74	0.00597963830599384\\
18.75	0.00597964079601796\\
18.76	0.00597964328709162\\
18.77	0.00597964577921534\\
18.78	0.00597964827238963\\
18.79	0.00597965076661499\\
18.8	0.00597965326189195\\
18.81	0.00597965575822102\\
18.82	0.0059796582556027\\
18.83	0.00597966075403752\\
18.84	0.00597966325352599\\
18.85	0.00597966575406863\\
18.86	0.00597966825566594\\
18.87	0.00597967075831845\\
18.88	0.00597967326202668\\
18.89	0.00597967576679112\\
18.9	0.00597967827261231\\
18.91	0.00597968077949076\\
18.92	0.00597968328742699\\
18.93	0.00597968579642151\\
18.94	0.00597968830647484\\
18.95	0.00597969081758749\\
18.96	0.00597969332975999\\
18.97	0.00597969584299286\\
18.98	0.00597969835728661\\
18.99	0.00597970087264177\\
19	0.00597970338905884\\
19.01	0.00597970590653835\\
19.02	0.00597970842508083\\
19.03	0.00597971094468678\\
19.04	0.00597971346535674\\
19.05	0.00597971598709121\\
19.06	0.00597971850989073\\
19.07	0.00597972103375581\\
19.08	0.00597972355868698\\
19.09	0.00597972608468475\\
19.1	0.00597972861174965\\
19.11	0.0059797311398822\\
19.12	0.00597973366908291\\
19.13	0.00597973619935233\\
19.14	0.00597973873069097\\
19.15	0.00597974126309935\\
19.16	0.005979743796578\\
19.17	0.00597974633112743\\
19.18	0.00597974886674818\\
19.19	0.00597975140344077\\
19.2	0.00597975394120572\\
19.21	0.00597975648004356\\
19.22	0.00597975901995482\\
19.23	0.00597976156094002\\
19.24	0.00597976410299968\\
19.25	0.00597976664613433\\
19.26	0.00597976919034451\\
19.27	0.00597977173563073\\
19.28	0.00597977428199353\\
19.29	0.00597977682943343\\
19.3	0.00597977937795096\\
19.31	0.00597978192754665\\
19.32	0.00597978447822102\\
19.33	0.0059797870299746\\
19.34	0.00597978958280793\\
19.35	0.00597979213672153\\
19.36	0.00597979469171593\\
19.37	0.00597979724779167\\
19.38	0.00597979980494927\\
19.39	0.00597980236318926\\
19.4	0.00597980492251218\\
19.41	0.00597980748291855\\
19.42	0.0059798100444089\\
19.43	0.00597981260698377\\
19.44	0.0059798151706437\\
19.45	0.0059798177353892\\
19.46	0.00597982030122081\\
19.47	0.00597982286813908\\
19.48	0.00597982543614452\\
19.49	0.00597982800523768\\
19.5	0.00597983057541909\\
19.51	0.00597983314668928\\
19.52	0.00597983571904878\\
19.53	0.00597983829249813\\
19.54	0.00597984086703787\\
19.55	0.00597984344266852\\
19.56	0.00597984601939064\\
19.57	0.00597984859720474\\
19.58	0.00597985117611137\\
19.59	0.00597985375611107\\
19.6	0.00597985633720436\\
19.61	0.00597985891939179\\
19.62	0.0059798615026739\\
19.63	0.00597986408705122\\
19.64	0.00597986667252429\\
19.65	0.00597986925909365\\
19.66	0.00597987184675983\\
19.67	0.00597987443552338\\
19.68	0.00597987702538483\\
19.69	0.00597987961634473\\
19.7	0.0059798822084036\\
19.71	0.00597988480156201\\
19.72	0.00597988739582047\\
19.73	0.00597988999117954\\
19.74	0.00597989258763975\\
19.75	0.00597989518520165\\
19.76	0.00597989778386577\\
19.77	0.00597990038363266\\
19.78	0.00597990298450287\\
19.79	0.00597990558647692\\
19.8	0.00597990818955537\\
19.81	0.00597991079373876\\
19.82	0.00597991339902763\\
19.83	0.00597991600542253\\
19.84	0.00597991861292399\\
19.85	0.00597992122153257\\
19.86	0.0059799238312488\\
19.87	0.00597992644207324\\
19.88	0.00597992905400642\\
19.89	0.0059799316670489\\
19.9	0.0059799342812012\\
19.91	0.0059799368964639\\
19.92	0.00597993951283752\\
19.93	0.00597994213032262\\
19.94	0.00597994474891975\\
19.95	0.00597994736862944\\
19.96	0.00597994998945224\\
19.97	0.00597995261138872\\
19.98	0.0059799552344394\\
19.99	0.00597995785860485\\
20	0.00597996048388561\\
20.01	0.00597996311028222\\
20.02	0.00597996573779524\\
20.03	0.00597996836642522\\
20.04	0.0059799709961727\\
20.05	0.00597997362703825\\
20.06	0.0059799762590224\\
20.07	0.00597997889212571\\
20.08	0.00597998152634872\\
20.09	0.005979984161692\\
20.1	0.00597998679815608\\
20.11	0.00597998943574153\\
20.12	0.0059799920744489\\
20.13	0.00597999471427873\\
20.14	0.00597999735523158\\
20.15	0.00597999999730801\\
20.16	0.00598000264050857\\
20.17	0.0059800052848338\\
20.18	0.00598000793028427\\
20.19	0.00598001057686053\\
20.2	0.00598001322456313\\
20.21	0.00598001587339263\\
20.22	0.00598001852334959\\
20.23	0.00598002117443455\\
20.24	0.00598002382664808\\
20.25	0.00598002647999073\\
20.26	0.00598002913446306\\
20.27	0.00598003179006562\\
20.28	0.00598003444679897\\
20.29	0.00598003710466367\\
20.3	0.00598003976366028\\
20.31	0.00598004242378935\\
20.32	0.00598004508505144\\
20.33	0.00598004774744711\\
20.34	0.00598005041097693\\
20.35	0.00598005307564144\\
20.36	0.00598005574144121\\
20.37	0.00598005840837679\\
20.38	0.00598006107644876\\
20.39	0.00598006374565766\\
20.4	0.00598006641600407\\
20.41	0.00598006908748852\\
20.42	0.00598007176011161\\
20.43	0.00598007443387387\\
20.44	0.00598007710877588\\
20.45	0.0059800797848182\\
20.46	0.00598008246200138\\
20.47	0.005980085140326\\
20.48	0.00598008781979261\\
20.49	0.00598009050040178\\
20.5	0.00598009318215408\\
20.51	0.00598009586505005\\
20.52	0.00598009854909028\\
20.53	0.00598010123427533\\
20.54	0.00598010392060575\\
20.55	0.00598010660808212\\
20.56	0.005980109296705\\
20.57	0.00598011198647496\\
20.58	0.00598011467739256\\
20.59	0.00598011736945837\\
20.6	0.00598012006267295\\
20.61	0.00598012275703688\\
20.62	0.00598012545255072\\
20.63	0.00598012814921504\\
20.64	0.0059801308470304\\
20.65	0.00598013354599738\\
20.66	0.00598013624611654\\
20.67	0.00598013894738846\\
20.68	0.00598014164981369\\
20.69	0.00598014435339282\\
20.7	0.0059801470581264\\
20.71	0.00598014976401502\\
20.72	0.00598015247105924\\
20.73	0.00598015517925963\\
20.74	0.00598015788861677\\
20.75	0.00598016059913122\\
20.76	0.00598016331080355\\
20.77	0.00598016602363435\\
20.78	0.00598016873762418\\
20.79	0.00598017145277361\\
20.8	0.00598017416908321\\
20.81	0.00598017688655357\\
20.82	0.00598017960518525\\
20.83	0.00598018232497882\\
20.84	0.00598018504593487\\
20.85	0.00598018776805397\\
20.86	0.00598019049133668\\
20.87	0.00598019321578359\\
20.88	0.00598019594139527\\
20.89	0.00598019866817231\\
20.9	0.00598020139611526\\
20.91	0.00598020412522472\\
20.92	0.00598020685550124\\
20.93	0.00598020958694542\\
20.94	0.00598021231955784\\
20.95	0.00598021505333906\\
20.96	0.00598021778828967\\
20.97	0.00598022052441025\\
20.98	0.00598022326170136\\
20.99	0.0059802260001636\\
21	0.00598022873979755\\
21.01	0.00598023148060377\\
21.02	0.00598023422258285\\
21.03	0.00598023696573538\\
21.04	0.00598023971006193\\
21.05	0.00598024245556307\\
21.06	0.00598024520223941\\
21.07	0.0059802479500915\\
21.08	0.00598025069911995\\
21.09	0.00598025344932532\\
21.1	0.0059802562007082\\
21.11	0.00598025895326918\\
21.12	0.00598026170700884\\
21.13	0.00598026446192775\\
21.14	0.00598026721802651\\
21.15	0.0059802699753057\\
21.16	0.0059802727337659\\
21.17	0.0059802754934077\\
21.18	0.00598027825423167\\
21.19	0.00598028101623842\\
21.2	0.00598028377942852\\
21.21	0.00598028654380255\\
21.22	0.00598028930936111\\
21.23	0.00598029207610478\\
21.24	0.00598029484403415\\
21.25	0.00598029761314981\\
21.26	0.00598030038345233\\
21.27	0.00598030315494232\\
21.28	0.00598030592762036\\
21.29	0.00598030870148703\\
21.3	0.00598031147654293\\
21.31	0.00598031425278864\\
21.32	0.00598031703022476\\
21.33	0.00598031980885187\\
21.34	0.00598032258867057\\
21.35	0.00598032536968144\\
21.36	0.00598032815188508\\
21.37	0.00598033093528207\\
21.38	0.00598033371987301\\
21.39	0.00598033650565849\\
21.4	0.00598033929263911\\
21.41	0.00598034208081544\\
21.42	0.00598034487018809\\
21.43	0.00598034766075765\\
21.44	0.00598035045252472\\
21.45	0.00598035324548988\\
21.46	0.00598035603965373\\
21.47	0.00598035883501687\\
21.48	0.00598036163157988\\
21.49	0.00598036442934337\\
21.5	0.00598036722830793\\
21.51	0.00598037002847415\\
21.52	0.00598037282984264\\
21.53	0.00598037563241398\\
21.54	0.00598037843618878\\
21.55	0.00598038124116762\\
21.56	0.00598038404735112\\
21.57	0.00598038685473986\\
21.58	0.00598038966333444\\
21.59	0.00598039247313547\\
21.6	0.00598039528414353\\
21.61	0.00598039809635924\\
21.62	0.00598040090978319\\
21.63	0.00598040372441597\\
21.64	0.00598040654025819\\
21.65	0.00598040935731045\\
21.66	0.00598041217557334\\
21.67	0.00598041499504748\\
21.68	0.00598041781573346\\
21.69	0.00598042063763188\\
21.7	0.00598042346074335\\
21.71	0.00598042628506847\\
21.72	0.00598042911060783\\
21.73	0.00598043193736205\\
21.74	0.00598043476533172\\
21.75	0.00598043759451745\\
21.76	0.00598044042491985\\
21.77	0.00598044325653951\\
21.78	0.00598044608937705\\
21.79	0.00598044892343306\\
21.8	0.00598045175870816\\
21.81	0.00598045459520294\\
21.82	0.00598045743291802\\
21.83	0.005980460271854\\
21.84	0.00598046311201148\\
21.85	0.00598046595339109\\
21.86	0.00598046879599341\\
21.87	0.00598047163981907\\
21.88	0.00598047448486866\\
21.89	0.00598047733114279\\
21.9	0.00598048017864209\\
21.91	0.00598048302736714\\
21.92	0.00598048587731857\\
21.93	0.00598048872849699\\
21.94	0.00598049158090299\\
21.95	0.00598049443453721\\
21.96	0.00598049728940023\\
21.97	0.00598050014549268\\
21.98	0.00598050300281518\\
21.99	0.00598050586136832\\
22	0.00598050872115272\\
22.01	0.005980511582169\\
22.02	0.00598051444441776\\
22.03	0.00598051730789963\\
22.04	0.00598052017261521\\
22.05	0.00598052303856513\\
22.06	0.00598052590574998\\
22.07	0.00598052877417039\\
22.08	0.00598053164382698\\
22.09	0.00598053451472036\\
22.1	0.00598053738685114\\
22.11	0.00598054026021994\\
22.12	0.00598054313482737\\
22.13	0.00598054601067407\\
22.14	0.00598054888776064\\
22.15	0.0059805517660877\\
22.16	0.00598055464565586\\
22.17	0.00598055752646575\\
22.18	0.00598056040851799\\
22.19	0.00598056329181319\\
22.2	0.00598056617635197\\
22.21	0.00598056906213496\\
22.22	0.00598057194916277\\
22.23	0.00598057483743603\\
22.24	0.00598057772695535\\
22.25	0.00598058061772136\\
22.26	0.00598058350973468\\
22.27	0.00598058640299593\\
22.28	0.00598058929750574\\
22.29	0.00598059219326472\\
22.3	0.0059805950902735\\
22.31	0.00598059798853271\\
22.32	0.00598060088804296\\
22.33	0.00598060378880489\\
22.34	0.00598060669081912\\
22.35	0.00598060959408627\\
22.36	0.00598061249860696\\
22.37	0.00598061540438183\\
22.38	0.00598061831141151\\
22.39	0.0059806212196966\\
22.4	0.00598062412923776\\
22.41	0.0059806270400356\\
22.42	0.00598062995209075\\
22.43	0.00598063286540384\\
22.44	0.00598063577997549\\
22.45	0.00598063869580634\\
22.46	0.00598064161289702\\
22.47	0.00598064453124816\\
22.48	0.00598064745086039\\
22.49	0.00598065037173433\\
22.5	0.00598065329387062\\
22.51	0.0059806562172699\\
22.52	0.00598065914193279\\
22.53	0.00598066206785992\\
22.54	0.00598066499505193\\
22.55	0.00598066792350946\\
22.56	0.00598067085323313\\
22.57	0.00598067378422358\\
22.58	0.00598067671648144\\
22.59	0.00598067965000736\\
22.6	0.00598068258480196\\
22.61	0.00598068552086588\\
22.62	0.00598068845819976\\
22.63	0.00598069139680423\\
22.64	0.00598069433667994\\
22.65	0.00598069727782751\\
22.66	0.00598070022024759\\
22.67	0.00598070316394082\\
22.68	0.00598070610890783\\
22.69	0.00598070905514926\\
22.7	0.00598071200266576\\
22.71	0.00598071495145796\\
22.72	0.00598071790152651\\
22.73	0.00598072085287204\\
22.74	0.0059807238054952\\
22.75	0.00598072675939663\\
22.76	0.00598072971457697\\
22.77	0.00598073267103687\\
22.78	0.00598073562877696\\
22.79	0.0059807385877979\\
22.8	0.00598074154810032\\
22.81	0.00598074450968488\\
22.82	0.00598074747255222\\
22.83	0.00598075043670297\\
22.84	0.0059807534021378\\
22.85	0.00598075636885734\\
22.86	0.00598075933686225\\
22.87	0.00598076230615317\\
22.88	0.00598076527673074\\
22.89	0.00598076824859563\\
22.9	0.00598077122174847\\
22.91	0.00598077419618992\\
22.92	0.00598077717192062\\
22.93	0.00598078014894124\\
22.94	0.00598078312725241\\
22.95	0.0059807861068548\\
22.96	0.00598078908774905\\
22.97	0.00598079206993581\\
22.98	0.00598079505341575\\
22.99	0.00598079803818951\\
23	0.00598080102425775\\
23.01	0.00598080401162112\\
23.02	0.00598080700028028\\
23.03	0.00598080999023588\\
23.04	0.00598081298148859\\
23.05	0.00598081597403905\\
23.06	0.00598081896788792\\
23.07	0.00598082196303588\\
23.08	0.00598082495948356\\
23.09	0.00598082795723164\\
23.1	0.00598083095628076\\
23.11	0.0059808339566316\\
23.12	0.00598083695828481\\
23.13	0.00598083996124106\\
23.14	0.00598084296550101\\
23.15	0.00598084597106531\\
23.16	0.00598084897793463\\
23.17	0.00598085198610965\\
23.18	0.00598085499559101\\
23.19	0.00598085800637939\\
23.2	0.00598086101847545\\
23.21	0.00598086403187986\\
23.22	0.00598086704659329\\
23.23	0.00598087006261639\\
23.24	0.00598087307994986\\
23.25	0.00598087609859434\\
23.26	0.00598087911855051\\
23.27	0.00598088213981904\\
23.28	0.0059808851624006\\
23.29	0.00598088818629586\\
23.3	0.00598089121150549\\
23.31	0.00598089423803018\\
23.32	0.00598089726587057\\
23.33	0.00598090029502737\\
23.34	0.00598090332550124\\
23.35	0.00598090635729284\\
23.36	0.00598090939040287\\
23.37	0.005980912424832\\
23.38	0.0059809154605809\\
23.39	0.00598091849765025\\
23.4	0.00598092153604074\\
23.41	0.00598092457575304\\
23.42	0.00598092761678782\\
23.43	0.00598093065914579\\
23.44	0.0059809337028276\\
23.45	0.00598093674783396\\
23.46	0.00598093979416554\\
23.47	0.00598094284182302\\
23.48	0.00598094589080709\\
23.49	0.00598094894111844\\
23.5	0.00598095199275775\\
23.51	0.00598095504572571\\
23.52	0.00598095810002301\\
23.53	0.00598096115565033\\
23.54	0.00598096421260837\\
23.55	0.00598096727089782\\
23.56	0.00598097033051937\\
23.57	0.0059809733914737\\
23.58	0.00598097645376152\\
23.59	0.00598097951738351\\
23.6	0.00598098258234038\\
23.61	0.00598098564863282\\
23.62	0.00598098871626152\\
23.63	0.00598099178522717\\
23.64	0.00598099485553049\\
23.65	0.00598099792717218\\
23.66	0.00598100100015291\\
23.67	0.00598100407447341\\
23.68	0.00598100715013437\\
23.69	0.0059810102271365\\
23.7	0.00598101330548049\\
23.71	0.00598101638516706\\
23.72	0.00598101946619691\\
23.73	0.00598102254857074\\
23.74	0.00598102563228927\\
23.75	0.00598102871735321\\
23.76	0.00598103180376325\\
23.77	0.00598103489152012\\
23.78	0.00598103798062453\\
23.79	0.00598104107107718\\
23.8	0.0059810441628788\\
23.81	0.0059810472560301\\
23.82	0.00598105035053178\\
23.83	0.00598105344638459\\
23.84	0.00598105654358922\\
23.85	0.0059810596421464\\
23.86	0.00598106274205686\\
23.87	0.0059810658433213\\
23.88	0.00598106894594046\\
23.89	0.00598107204991506\\
23.9	0.00598107515524583\\
23.91	0.00598107826193349\\
23.92	0.00598108136997876\\
23.93	0.00598108447938239\\
23.94	0.0059810875901451\\
23.95	0.00598109070226761\\
23.96	0.00598109381575067\\
23.97	0.00598109693059501\\
23.98	0.00598110004680136\\
23.99	0.00598110316437046\\
24	0.00598110628330305\\
24.01	0.00598110940359987\\
24.02	0.00598111252526166\\
24.03	0.00598111564828916\\
24.04	0.00598111877268311\\
24.05	0.00598112189844426\\
24.06	0.00598112502557337\\
24.07	0.00598112815407116\\
24.08	0.0059811312839384\\
24.09	0.00598113441517584\\
24.1	0.00598113754778423\\
24.11	0.00598114068176431\\
24.12	0.00598114381711686\\
24.13	0.00598114695384262\\
24.14	0.00598115009194236\\
24.15	0.00598115323141683\\
24.16	0.0059811563722668\\
24.17	0.00598115951449303\\
24.18	0.00598116265809628\\
24.19	0.00598116580307734\\
24.2	0.00598116894943695\\
24.21	0.00598117209717589\\
24.22	0.00598117524629494\\
24.23	0.00598117839679487\\
24.24	0.00598118154867646\\
24.25	0.00598118470194048\\
24.26	0.00598118785658772\\
24.27	0.00598119101261895\\
24.28	0.00598119417003496\\
24.29	0.00598119732883653\\
24.3	0.00598120048902445\\
24.31	0.00598120365059952\\
24.32	0.00598120681356252\\
24.33	0.00598120997791425\\
24.34	0.0059812131436555\\
24.35	0.00598121631078706\\
24.36	0.00598121947930975\\
24.37	0.00598122264922436\\
24.38	0.0059812258205317\\
24.39	0.00598122899323257\\
24.4	0.00598123216732778\\
24.41	0.00598123534281814\\
24.42	0.00598123851970447\\
24.43	0.00598124169798758\\
24.44	0.00598124487766828\\
24.45	0.00598124805874741\\
24.46	0.00598125124122578\\
24.47	0.00598125442510422\\
24.48	0.00598125761038355\\
24.49	0.00598126079706461\\
24.5	0.00598126398514823\\
24.51	0.00598126717463524\\
24.52	0.00598127036552649\\
24.53	0.00598127355782281\\
24.54	0.00598127675152504\\
24.55	0.00598127994663404\\
24.56	0.00598128314315066\\
24.57	0.00598128634107574\\
24.58	0.00598128954041014\\
24.59	0.00598129274115471\\
24.6	0.00598129594331033\\
24.61	0.00598129914687785\\
24.62	0.00598130235185814\\
24.63	0.00598130555825207\\
24.64	0.00598130876606051\\
24.65	0.00598131197528434\\
24.66	0.00598131518592443\\
24.67	0.00598131839798169\\
24.68	0.00598132161145698\\
24.69	0.00598132482635119\\
24.7	0.00598132804266523\\
24.71	0.00598133126039998\\
24.72	0.00598133447955635\\
24.73	0.00598133770013524\\
24.74	0.00598134092213755\\
24.75	0.00598134414556421\\
24.76	0.00598134737041612\\
24.77	0.00598135059669419\\
24.78	0.00598135382439936\\
24.79	0.00598135705353255\\
24.8	0.00598136028409469\\
24.81	0.00598136351608672\\
24.82	0.00598136674950957\\
24.83	0.00598136998436417\\
24.84	0.00598137322065149\\
24.85	0.00598137645837247\\
24.86	0.00598137969752807\\
24.87	0.00598138293811924\\
24.88	0.00598138618014695\\
24.89	0.00598138942361217\\
24.9	0.00598139266851587\\
24.91	0.00598139591485902\\
24.92	0.00598139916264262\\
24.93	0.00598140241186764\\
24.94	0.00598140566253508\\
24.95	0.00598140891464593\\
24.96	0.00598141216820121\\
24.97	0.0059814154232019\\
24.98	0.00598141867964904\\
24.99	0.00598142193754362\\
25	0.00598142519688668\\
25.01	0.00598142845767925\\
25.02	0.00598143171992236\\
25.03	0.00598143498361703\\
25.04	0.00598143824876433\\
25.05	0.0059814415153653\\
25.06	0.005981444783421\\
25.07	0.00598144805293249\\
25.08	0.00598145132390083\\
25.09	0.0059814545963271\\
25.1	0.00598145787021239\\
25.11	0.00598146114555777\\
25.12	0.00598146442236435\\
25.13	0.0059814677006332\\
25.14	0.00598147098036546\\
25.15	0.00598147426156222\\
25.16	0.00598147754422462\\
25.17	0.00598148082835376\\
25.18	0.00598148411395079\\
25.19	0.00598148740101684\\
25.2	0.00598149068955306\\
25.21	0.00598149397956062\\
25.22	0.00598149727104065\\
25.23	0.00598150056399435\\
25.24	0.00598150385842288\\
25.25	0.00598150715432743\\
25.26	0.00598151045170919\\
25.27	0.00598151375056936\\
25.28	0.00598151705090916\\
25.29	0.00598152035272979\\
25.3	0.00598152365603248\\
25.31	0.00598152696081847\\
25.32	0.00598153026708899\\
25.33	0.0059815335748453\\
25.34	0.00598153688408866\\
25.35	0.00598154019482033\\
25.36	0.0059815435070416\\
25.37	0.00598154682075375\\
25.38	0.00598155013595807\\
25.39	0.00598155345265588\\
25.4	0.00598155677084848\\
25.41	0.00598156009053722\\
25.42	0.00598156341172341\\
25.43	0.0059815667344084\\
25.44	0.00598157005859356\\
25.45	0.00598157338428024\\
25.46	0.00598157671146983\\
25.47	0.00598158004016372\\
25.48	0.00598158337036329\\
25.49	0.00598158670206997\\
25.5	0.00598159003528517\\
25.51	0.00598159337001033\\
25.52	0.00598159670624689\\
25.53	0.0059816000439963\\
25.54	0.00598160338326004\\
25.55	0.00598160672403959\\
25.56	0.00598161006633643\\
25.57	0.00598161341015208\\
25.58	0.00598161675548805\\
25.59	0.00598162010234588\\
25.6	0.00598162345072711\\
25.61	0.00598162680063329\\
25.62	0.00598163015206601\\
25.63	0.00598163350502683\\
25.64	0.00598163685951737\\
25.65	0.00598164021553925\\
25.66	0.00598164357309408\\
25.67	0.0059816469321835\\
25.68	0.00598165029280919\\
25.69	0.0059816536549728\\
25.7	0.00598165701867603\\
25.71	0.00598166038392059\\
25.72	0.00598166375070819\\
25.73	0.00598166711904058\\
25.74	0.00598167048891949\\
25.75	0.00598167386034671\\
25.76	0.00598167723332401\\
25.77	0.0059816806078532\\
25.78	0.00598168398393611\\
25.79	0.00598168736157457\\
25.8	0.00598169074077044\\
25.81	0.00598169412152559\\
25.82	0.00598169750384191\\
25.83	0.00598170088772132\\
25.84	0.00598170427316576\\
25.85	0.00598170766017716\\
25.86	0.00598171104875751\\
25.87	0.00598171443890878\\
25.88	0.005981717830633\\
25.89	0.0059817212239322\\
25.9	0.00598172461880842\\
25.91	0.00598172801526375\\
25.92	0.00598173141330027\\
25.93	0.00598173481292011\\
25.94	0.0059817382141254\\
25.95	0.00598174161691832\\
25.96	0.00598174502130104\\
25.97	0.00598174842727577\\
25.98	0.00598175183484476\\
25.99	0.00598175524401025\\
26	0.00598175865477453\\
26.01	0.0059817620671399\\
26.02	0.0059817654811087\\
26.03	0.00598176889668329\\
26.04	0.00598177231386605\\
26.05	0.00598177573265939\\
26.06	0.00598177915306575\\
26.07	0.0059817825750876\\
26.08	0.00598178599872742\\
26.09	0.00598178942398774\\
26.1	0.00598179285087111\\
26.11	0.00598179627938012\\
26.12	0.00598179970951735\\
26.13	0.00598180314128547\\
26.14	0.00598180657468713\\
26.15	0.00598181000972504\\
26.16	0.00598181344640192\\
26.17	0.00598181688472055\\
26.18	0.00598182032468372\\
26.19	0.00598182376629426\\
26.2	0.00598182720955502\\
26.21	0.00598183065446892\\
26.22	0.00598183410103889\\
26.23	0.00598183754926787\\
26.24	0.00598184099915889\\
26.25	0.00598184445071498\\
26.26	0.00598184790393922\\
26.27	0.00598185135883471\\
26.28	0.00598185481540462\\
26.29	0.00598185827365212\\
26.3	0.00598186173358045\\
26.31	0.00598186519519288\\
26.32	0.00598186865849272\\
26.33	0.00598187212348331\\
26.34	0.00598187559016806\\
26.35	0.00598187905855038\\
26.36	0.00598188252863377\\
26.37	0.00598188600042174\\
26.38	0.00598188947391786\\
26.39	0.00598189294912573\\
26.4	0.00598189642604901\\
26.41	0.00598189990469142\\
26.42	0.00598190338505669\\
26.43	0.00598190686714862\\
26.44	0.00598191035097106\\
26.45	0.00598191383652791\\
26.46	0.00598191732382312\\
26.47	0.00598192081286067\\
26.48	0.00598192430364462\\
26.49	0.00598192779617908\\
26.5	0.00598193129046818\\
26.51	0.00598193478651616\\
26.52	0.00598193828432726\\
26.53	0.00598194178390581\\
26.54	0.0059819452852562\\
26.55	0.00598194878838284\\
26.56	0.00598195229329024\\
26.57	0.00598195579998295\\
26.58	0.00598195930846557\\
26.59	0.0059819628187428\\
26.6	0.00598196633081935\\
26.61	0.00598196984470005\\
26.62	0.00598197336038975\\
26.63	0.00598197687789338\\
26.64	0.00598198039721593\\
26.65	0.00598198391836247\\
26.66	0.00598198744133814\\
26.67	0.00598199096614812\\
26.68	0.00598199449279771\\
26.69	0.00598199802129223\\
26.7	0.00598200155163712\\
26.71	0.00598200508383784\\
26.72	0.00598200861789998\\
26.73	0.00598201215382917\\
26.74	0.00598201569163113\\
26.75	0.00598201923131166\\
26.76	0.00598202277287664\\
26.77	0.00598202631633202\\
26.78	0.00598202986168384\\
26.79	0.00598203340893822\\
26.8	0.00598203695810138\\
26.81	0.0059820405091796\\
26.82	0.00598204406217927\\
26.83	0.00598204761710686\\
26.84	0.00598205117396892\\
26.85	0.00598205473277211\\
26.86	0.00598205829352317\\
26.87	0.00598206185622895\\
26.88	0.00598206542089637\\
26.89	0.00598206898753246\\
26.9	0.00598207255614435\\
26.91	0.00598207612673927\\
26.92	0.00598207969932456\\
26.93	0.00598208327390765\\
26.94	0.00598208685049607\\
26.95	0.00598209042909747\\
26.96	0.00598209400971961\\
26.97	0.00598209759237035\\
26.98	0.00598210117705765\\
26.99	0.00598210476378961\\
27	0.00598210835257443\\
27.01	0.00598211194342043\\
27.02	0.00598211553633603\\
27.03	0.0059821191313298\\
27.04	0.00598212272841042\\
27.05	0.00598212632758669\\
27.06	0.00598212992886751\\
27.07	0.00598213353226197\\
27.08	0.00598213713777923\\
27.09	0.00598214074542861\\
27.1	0.00598214435521954\\
27.11	0.00598214796716163\\
27.12	0.00598215158126459\\
27.13	0.00598215519753827\\
27.14	0.00598215881599267\\
27.15	0.00598216243663793\\
27.16	0.00598216605948434\\
27.17	0.00598216968454234\\
27.18	0.00598217331182251\\
27.19	0.00598217694133559\\
27.2	0.00598218057309247\\
27.21	0.00598218420710419\\
27.22	0.00598218784338196\\
27.23	0.00598219148193716\\
27.24	0.00598219512278131\\
27.25	0.00598219876592611\\
27.26	0.00598220241138342\\
27.27	0.00598220605916528\\
27.28	0.00598220970928391\\
27.29	0.00598221336175168\\
27.3	0.00598221701658117\\
27.31	0.00598222067378511\\
27.32	0.00598222433337644\\
27.33	0.00598222799536827\\
27.34	0.00598223165977391\\
27.35	0.00598223532660686\\
27.36	0.00598223899588081\\
27.37	0.00598224266760964\\
27.38	0.00598224634180745\\
27.39	0.00598225001848853\\
27.4	0.00598225369766738\\
27.41	0.00598225737935872\\
27.42	0.00598226106357745\\
27.43	0.00598226475033872\\
27.44	0.00598226843965788\\
27.45	0.00598227213155051\\
27.46	0.00598227582603242\\
27.47	0.00598227952311962\\
27.48	0.00598228322282839\\
27.49	0.00598228692517522\\
27.5	0.00598229063017684\\
27.51	0.00598229433785022\\
27.52	0.00598229804821259\\
27.53	0.00598230176128142\\
27.54	0.00598230547707441\\
27.55	0.00598230919560956\\
27.56	0.00598231291690509\\
27.57	0.0059823166409795\\
27.58	0.00598232036785155\\
27.59	0.00598232409754027\\
27.6	0.00598232783006498\\
27.61	0.00598233156544526\\
27.62	0.00598233530370096\\
27.63	0.00598233904485226\\
27.64	0.00598234278891957\\
27.65	0.00598234653592365\\
27.66	0.00598235028588552\\
27.67	0.00598235403882653\\
27.68	0.00598235779476829\\
27.69	0.00598236155373278\\
27.7	0.00598236531574226\\
27.71	0.00598236908081929\\
27.72	0.00598237284898681\\
27.73	0.00598237662026805\\
27.74	0.00598238039468655\\
27.75	0.00598238417226625\\
27.76	0.00598238795303137\\
27.77	0.00598239173700652\\
27.78	0.00598239552421663\\
27.79	0.00598239931468701\\
27.8	0.00598240310844332\\
27.81	0.00598240690551157\\
27.82	0.00598241070591817\\
27.83	0.00598241450968989\\
27.84	0.00598241831685387\\
27.85	0.00598242212743766\\
27.86	0.00598242594146919\\
27.87	0.00598242975897677\\
27.88	0.00598243357998914\\
27.89	0.00598243740453543\\
27.9	0.00598244123264519\\
27.91	0.00598244506434837\\
27.92	0.00598244889967538\\
27.93	0.00598245273865701\\
27.94	0.00598245658132454\\
27.95	0.00598246042770965\\
27.96	0.00598246427784448\\
27.97	0.00598246813176162\\
27.98	0.00598247198949413\\
27.99	0.00598247585107552\\
28	0.00598247971653978\\
28.01	0.00598248358592138\\
28.02	0.00598248745925527\\
28.03	0.00598249133657688\\
28.04	0.00598249521792216\\
28.05	0.00598249910332753\\
28.06	0.00598250299282997\\
28.07	0.00598250688646692\\
28.08	0.00598251078427638\\
28.09	0.00598251468629687\\
28.1	0.00598251859256745\\
28.11	0.00598252250312772\\
28.12	0.00598252641801785\\
28.13	0.00598253033727853\\
28.14	0.00598253426095106\\
28.15	0.00598253818907728\\
28.16	0.00598254212169964\\
28.17	0.00598254605886116\\
28.18	0.00598255000060547\\
28.19	0.00598255394697679\\
28.2	0.00598255789801997\\
28.21	0.00598256185378046\\
28.22	0.00598256581430437\\
28.23	0.00598256977963842\\
28.24	0.00598257374983\\
28.25	0.00598257772492712\\
28.26	0.0059825817049785\\
28.27	0.00598258569003349\\
28.28	0.00598258968014216\\
28.29	0.00598259367535523\\
28.3	0.00598259767572416\\
28.31	0.00598260168130109\\
28.32	0.0059826056921389\\
28.33	0.00598260970829118\\
28.34	0.00598261372981228\\
28.35	0.00598261775675727\\
28.36	0.00598262178918201\\
28.37	0.00598262582714312\\
28.38	0.00598262987069798\\
28.39	0.00598263391990479\\
28.4	0.00598263797482254\\
28.41	0.00598264203551102\\
28.42	0.00598264610203086\\
28.43	0.00598265017444353\\
28.44	0.00598265425281134\\
28.45	0.00598265833719744\\
28.46	0.0059826624276659\\
28.47	0.00598266652428162\\
28.48	0.00598267062711044\\
28.49	0.00598267473621909\\
28.5	0.00598267885167522\\
28.51	0.00598268297354744\\
28.52	0.00598268710190527\\
28.53	0.00598269123681923\\
28.54	0.00598269537836082\\
28.55	0.00598269952660251\\
28.56	0.0059827036816178\\
28.57	0.0059827078434812\\
28.58	0.00598271201226827\\
28.59	0.00598271618805562\\
28.6	0.00598272037092094\\
28.61	0.00598272456094301\\
28.62	0.00598272875820171\\
28.63	0.00598273296277805\\
28.64	0.00598273717475419\\
28.65	0.00598274139421344\\
28.66	0.00598274562124029\\
28.67	0.00598274985592045\\
28.68	0.00598275409834084\\
28.69	0.00598275834858962\\
28.7	0.00598276260675621\\
28.71	0.00598276687293134\\
28.72	0.00598277114720701\\
28.73	0.00598277542967659\\
28.74	0.00598277972043478\\
28.75	0.00598278401957768\\
28.76	0.00598278832720277\\
28.77	0.005982792643409\\
28.78	0.00598279696829675\\
28.79	0.00598280130196789\\
28.8	0.00598280564452581\\
28.81	0.00598280999607546\\
28.82	0.00598281435672334\\
28.83	0.00598281872657759\\
28.84	0.00598282310574796\\
28.85	0.00598282749434588\\
28.86	0.00598283189248452\\
28.87	0.00598283630027876\\
28.88	0.00598284071784526\\
28.89	0.00598284514530255\\
28.9	0.00598284958277096\\
28.91	0.00598285403037277\\
28.92	0.00598285848823217\\
28.93	0.00598286295647537\\
28.94	0.00598286743523059\\
28.95	0.00598287192462815\\
28.96	0.00598287642480051\\
28.97	0.00598288093588228\\
28.98	0.00598288545801034\\
28.99	0.00598288999132385\\
29	0.00598289453596431\\
29.01	0.00598289909207564\\
29.02	0.00598290365980423\\
29.03	0.005982908239299\\
29.04	0.00598291283071144\\
29.05	0.00598291743419577\\
29.06	0.00598292204990889\\
29.07	0.00598292667801053\\
29.08	0.00598293131866333\\
29.09	0.00598293597203288\\
29.1	0.00598294063828783\\
29.11	0.00598294531759996\\
29.12	0.00598295001014429\\
29.13	0.00598295471609916\\
29.14	0.00598295943564633\\
29.15	0.00598296416897106\\
29.16	0.00598296891626227\\
29.17	0.00598297367771259\\
29.18	0.00598297845351851\\
29.19	0.00598298324388048\\
29.2	0.00598298804900303\\
29.21	0.00598299286909493\\
29.22	0.00598299770436928\\
29.23	0.00598300255504367\\
29.24	0.00598300742134033\\
29.25	0.00598301230348628\\
29.26	0.00598301720171345\\
29.27	0.0059830221162589\\
29.28	0.00598302704736495\\
29.29	0.00598303199527937\\
29.3	0.00598303696025556\\
29.31	0.00598304194255272\\
29.32	0.00598304694243611\\
29.33	0.00598305196017719\\
29.34	0.00598305699605386\\
29.35	0.00598306205035072\\
29.36	0.00598306712335922\\
29.37	0.00598307221537801\\
29.38	0.0059830773267131\\
29.39	0.00598308245767817\\
29.4	0.00598308760859486\\
29.41	0.00598309277979301\\
29.42	0.00598309797161099\\
29.43	0.00598310318439603\\
29.44	0.00598310841850447\\
29.45	0.00598311367430218\\
29.46	0.00598311895216487\\
29.47	0.00598312425247847\\
29.48	0.00598312957563948\\
29.49	0.00598313492205542\\
29.5	0.00598314029214519\\
29.51	0.00598314568633955\\
29.52	0.00598315110508151\\
29.53	0.00598315654882685\\
29.54	0.00598316201804461\\
29.55	0.00598316751321752\\
29.56	0.00598317303483954\\
29.57	0.00598317858341554\\
29.58	0.0059831841594616\\
29.59	0.00598318976350532\\
29.6	0.00598319539608602\\
29.61	0.00598320105775508\\
29.62	0.00598320674907619\\
29.63	0.00598321247062568\\
29.64	0.00598321822299284\\
29.65	0.00598322400678016\\
29.66	0.00598322982260373\\
29.67	0.00598323567109353\\
29.68	0.00598324155289379\\
29.69	0.00598324746866331\\
29.7	0.00598325341907583\\
29.71	0.00598325940482041\\
29.72	0.00598326542660178\\
29.73	0.00598327148514074\\
29.74	0.00598327758117454\\
29.75	0.00598328371545731\\
29.76	0.00598328988876043\\
29.77	0.00598329610187303\\
29.78	0.00598330235560235\\
29.79	0.00598330865077422\\
29.8	0.00598331498823354\\
29.81	0.00598332136884472\\
29.82	0.00598332779349218\\
29.83	0.00598333426308085\\
29.84	0.00598334077853667\\
29.85	0.00598334734080712\\
29.86	0.00598335395086175\\
29.87	0.00598336060969276\\
29.88	0.00598336731831552\\
29.89	0.00598337407776919\\
29.9	0.00598338088911731\\
29.91	0.00598338775344837\\
29.92	0.0059833946718765\\
29.93	0.0059834016455421\\
29.94	0.00598340867561245\\
29.95	0.00598341576328247\\
29.96	0.00598342290977534\\
29.97	0.00598343011634328\\
29.98	0.00598343738426826\\
29.99	0.00598344471486272\\
30	0.0059834521094704\\
30.01	0.00598345956946713\\
30.02	0.0059834670962616\\
30.03	0.00598347469129623\\
30.04	0.00598348235604806\\
30.05	0.00598349009202957\\
30.06	0.00598349790078964\\
30.07	0.00598350578391445\\
30.08	0.00598351374302845\\
30.09	0.00598352177979536\\
30.1	0.00598352989591914\\
30.11	0.00598353809314506\\
30.12	0.00598354637326074\\
30.13	0.00598355473809727\\
30.14	0.00598356318953031\\
30.15	0.00598357172948123\\
30.16	0.00598358035991833\\
30.17	0.00598358908285804\\
30.18	0.00598359790036616\\
30.19	0.00598360681455915\\
30.2	0.00598361582760543\\
30.21	0.00598362494172676\\
30.22	0.00598363415919962\\
30.23	0.00598364348235662\\
30.24	0.00598365291358797\\
30.25	0.00598366245534304\\
30.26	0.00598367211013181\\
30.27	0.00598368188052652\\
30.28	0.0059836917691633\\
30.29	0.00598370177874383\\
30.3	0.00598371191203708\\
30.31	0.00598372217188104\\
30.32	0.0059837325611846\\
30.33	0.00598374308292936\\
30.34	0.00598375374017157\\
30.35	0.0059837645360441\\
30.36	0.00598377547375846\\
30.37	0.00598378655660691\\
30.38	0.00598379778796455\\
30.39	0.00598380917129153\\
30.4	0.00598382071013533\\
30.41	0.00598383240813305\\
30.42	0.00598384426901383\\
30.43	0.00598385629660123\\
30.44	0.00598386849481583\\
30.45	0.00598388086767772\\
30.46	0.00598389341930925\\
30.47	0.00598390615393769\\
30.48	0.00598391907589807\\
30.49	0.00598393218963605\\
30.5	0.00598394549971086\\
30.51	0.00598395901079839\\
30.52	0.00598397272769429\\
30.53	0.00598398665531717\\
30.54	0.00598400079871192\\
30.55	0.00598401516305312\\
30.56	0.0059840297536485\\
30.57	0.00598404457594253\\
30.58	0.00598405963552015\\
30.59	0.00598407493811046\\
30.6	0.00598409048959074\\
30.61	0.00598410629599034\\
30.62	0.00598412236349485\\
30.63	0.0059841385742652\\
30.64	0.00598415479126208\\
30.65	0.00598417101448844\\
30.66	0.00598418724394727\\
30.67	0.00598420347964156\\
30.68	0.00598421972157427\\
30.69	0.0059842359697484\\
30.7	0.00598425222416693\\
30.71	0.00598426848483284\\
30.72	0.00598428475174914\\
30.73	0.0059843010249188\\
30.74	0.00598431730434482\\
30.75	0.00598433359003021\\
30.76	0.00598434988197796\\
30.77	0.00598436618019106\\
30.78	0.00598438248467253\\
30.79	0.00598439879542536\\
30.8	0.00598441511245256\\
30.81	0.00598443143575715\\
30.82	0.00598444776534213\\
30.83	0.00598446410121051\\
30.84	0.00598448044336532\\
30.85	0.00598449679180957\\
30.86	0.00598451314654627\\
30.87	0.00598452950757845\\
30.88	0.00598454587490914\\
30.89	0.00598456224854136\\
30.9	0.00598457862847813\\
30.91	0.00598459501472249\\
30.92	0.00598461140727747\\
30.93	0.00598462780614611\\
30.94	0.00598464421133144\\
30.95	0.0059846606228365\\
30.96	0.00598467704066433\\
30.97	0.00598469346481798\\
30.98	0.00598470989530048\\
30.99	0.00598472633211489\\
31	0.00598474277526426\\
31.01	0.00598475922475163\\
31.02	0.00598477568058007\\
31.03	0.00598479214275262\\
31.04	0.00598480861127236\\
31.05	0.00598482508614232\\
31.06	0.00598484156736559\\
31.07	0.00598485805494522\\
31.08	0.00598487454888428\\
31.09	0.00598489104918584\\
31.1	0.00598490755585297\\
31.11	0.00598492406888875\\
31.12	0.00598494058829624\\
31.13	0.00598495711407854\\
31.14	0.00598497364623871\\
31.15	0.00598499018477985\\
31.16	0.00598500672970502\\
31.17	0.00598502328101734\\
31.18	0.00598503983871986\\
31.19	0.00598505640281571\\
31.2	0.00598507297330795\\
31.21	0.00598508955019969\\
31.22	0.00598510613349404\\
31.23	0.00598512272319407\\
31.24	0.00598513931930291\\
31.25	0.00598515592182366\\
31.26	0.00598517253075941\\
31.27	0.00598518914611328\\
31.28	0.00598520576788838\\
31.29	0.00598522239608783\\
31.3	0.00598523903071473\\
31.31	0.00598525567177221\\
31.32	0.00598527231926339\\
31.33	0.00598528897319138\\
31.34	0.00598530563355932\\
31.35	0.00598532230037033\\
31.36	0.00598533897362753\\
31.37	0.00598535565333406\\
31.38	0.00598537233949306\\
31.39	0.00598538903210765\\
31.4	0.00598540573118098\\
31.41	0.00598542243671618\\
31.42	0.00598543914871641\\
31.43	0.00598545586718479\\
31.44	0.00598547259212449\\
31.45	0.00598548932353864\\
31.46	0.00598550606143041\\
31.47	0.00598552280580294\\
31.48	0.00598553955665939\\
31.49	0.00598555631400292\\
31.5	0.00598557307783668\\
31.51	0.00598558984816385\\
31.52	0.00598560662498759\\
31.53	0.00598562340831106\\
31.54	0.00598564019813743\\
31.55	0.00598565699446988\\
31.56	0.00598567379731157\\
31.57	0.0059856906066657\\
31.58	0.00598570742253543\\
31.59	0.00598572424492395\\
31.6	0.00598574107383443\\
31.61	0.00598575790927008\\
31.62	0.00598577475123406\\
31.63	0.00598579159972958\\
31.64	0.00598580845475983\\
31.65	0.00598582531632799\\
31.66	0.00598584218443728\\
31.67	0.00598585905909088\\
31.68	0.00598587594029201\\
31.69	0.00598589282804386\\
31.7	0.00598590972234964\\
31.71	0.00598592662321256\\
31.72	0.00598594353063584\\
31.73	0.00598596044462267\\
31.74	0.00598597736517629\\
31.75	0.00598599429229991\\
31.76	0.00598601122599675\\
31.77	0.00598602816627003\\
31.78	0.00598604511312298\\
31.79	0.00598606206655883\\
31.8	0.0059860790265808\\
31.81	0.00598609599319213\\
31.82	0.00598611296639606\\
31.83	0.00598612994619581\\
31.84	0.00598614693259464\\
31.85	0.00598616392559577\\
31.86	0.00598618092520246\\
31.87	0.00598619793141796\\
31.88	0.0059862149442455\\
31.89	0.00598623196368835\\
31.9	0.00598624898974975\\
31.91	0.00598626602243297\\
31.92	0.00598628306174126\\
31.93	0.00598630010767787\\
31.94	0.00598631716024609\\
31.95	0.00598633421944916\\
31.96	0.00598635128529037\\
31.97	0.00598636835777297\\
31.98	0.00598638543690024\\
31.99	0.00598640252267546\\
32	0.00598641961510191\\
32.01	0.00598643671418286\\
32.02	0.0059864538199216\\
32.03	0.00598647093232141\\
32.04	0.00598648805138558\\
32.05	0.0059865051771174\\
32.06	0.00598652230952016\\
32.07	0.00598653944859716\\
32.08	0.00598655659435168\\
32.09	0.00598657374678705\\
32.1	0.00598659090590654\\
32.11	0.00598660807171347\\
32.12	0.00598662524421115\\
32.13	0.00598664242340289\\
32.14	0.00598665960929199\\
32.15	0.00598667680188177\\
32.16	0.00598669400117554\\
32.17	0.00598671120717663\\
32.18	0.00598672841988836\\
32.19	0.00598674563931405\\
32.2	0.00598676286545703\\
32.21	0.00598678009832062\\
32.22	0.00598679733790815\\
32.23	0.00598681458422297\\
32.24	0.0059868318372684\\
32.25	0.00598684909704779\\
32.26	0.00598686636356449\\
32.27	0.00598688363682181\\
32.28	0.00598690091682313\\
32.29	0.00598691820357178\\
32.3	0.00598693549707111\\
32.31	0.00598695279732449\\
32.32	0.00598697010433526\\
32.33	0.00598698741810678\\
32.34	0.00598700473864242\\
32.35	0.00598702206594554\\
32.36	0.0059870394000195\\
32.37	0.00598705674086768\\
32.38	0.00598707408849344\\
32.39	0.00598709144290016\\
32.4	0.00598710880409122\\
32.41	0.00598712617206998\\
32.42	0.00598714354683984\\
32.43	0.00598716092840417\\
32.44	0.00598717831676637\\
32.45	0.00598719571192983\\
32.46	0.00598721311389792\\
32.47	0.00598723052267406\\
32.48	0.00598724793826163\\
32.49	0.00598726536066403\\
32.5	0.00598728278988466\\
32.51	0.00598730022592693\\
32.52	0.00598731766879425\\
32.53	0.00598733511849002\\
32.54	0.00598735257501765\\
32.55	0.00598737003838057\\
32.56	0.00598738750858218\\
32.57	0.0059874049856259\\
32.58	0.00598742246951516\\
32.59	0.00598743996025338\\
32.6	0.00598745745784399\\
32.61	0.00598747496229042\\
32.62	0.0059874924735961\\
32.63	0.00598750999176446\\
32.64	0.00598752751679894\\
32.65	0.00598754504870299\\
32.66	0.00598756258748004\\
32.67	0.00598758013313354\\
32.68	0.00598759768566694\\
32.69	0.00598761524508368\\
32.7	0.00598763281138723\\
32.71	0.00598765038458102\\
32.72	0.00598766796466854\\
32.73	0.00598768555165323\\
32.74	0.00598770314553855\\
32.75	0.00598772074632797\\
32.76	0.00598773835402497\\
32.77	0.00598775596863302\\
32.78	0.00598777359015558\\
32.79	0.00598779121859613\\
32.8	0.00598780885395815\\
32.81	0.00598782649624513\\
32.82	0.00598784414546055\\
32.83	0.00598786180160789\\
32.84	0.00598787946469065\\
32.85	0.00598789713471231\\
32.86	0.00598791481167637\\
32.87	0.00598793249558634\\
32.88	0.0059879501864457\\
32.89	0.00598796788425796\\
32.9	0.00598798558902664\\
32.91	0.00598800330075523\\
32.92	0.00598802101944725\\
32.93	0.0059880387451062\\
32.94	0.00598805647773562\\
32.95	0.005988074217339\\
32.96	0.00598809196391988\\
32.97	0.00598810971748179\\
32.98	0.00598812747802824\\
32.99	0.00598814524556277\\
33	0.00598816302008891\\
33.01	0.0059881808016102\\
33.02	0.00598819859013016\\
33.03	0.00598821638565235\\
33.04	0.0059882341881803\\
33.05	0.00598825199771757\\
33.06	0.00598826981426769\\
33.07	0.00598828763783423\\
33.08	0.00598830546842073\\
33.09	0.00598832330603075\\
33.1	0.00598834115066785\\
33.11	0.00598835900233559\\
33.12	0.00598837686103754\\
33.13	0.00598839472677727\\
33.14	0.00598841259955834\\
33.15	0.00598843047938433\\
33.16	0.00598844836625882\\
33.17	0.00598846626018537\\
33.18	0.00598848416116759\\
33.19	0.00598850206920903\\
33.2	0.00598851998431331\\
33.21	0.00598853790648399\\
33.22	0.00598855583572469\\
33.23	0.00598857377203898\\
33.24	0.00598859171543047\\
33.25	0.00598860966590276\\
33.26	0.00598862762345946\\
33.27	0.00598864558810416\\
33.28	0.00598866355984047\\
33.29	0.00598868153867202\\
33.3	0.00598869952460241\\
33.31	0.00598871751763527\\
33.32	0.0059887355177742\\
33.33	0.00598875352502283\\
33.34	0.0059887715393848\\
33.35	0.00598878956086372\\
33.36	0.00598880758946323\\
33.37	0.00598882562518697\\
33.38	0.00598884366803856\\
33.39	0.00598886171802165\\
33.4	0.00598887977513989\\
33.41	0.0059888978393969\\
33.42	0.00598891591079636\\
33.43	0.0059889339893419\\
33.44	0.00598895207503718\\
33.45	0.00598897016788586\\
33.46	0.00598898826789159\\
33.47	0.00598900637505803\\
33.48	0.00598902448938887\\
33.49	0.00598904261088775\\
33.5	0.00598906073955835\\
33.51	0.00598907887540435\\
33.52	0.00598909701842942\\
33.53	0.00598911516863724\\
33.54	0.00598913332603149\\
33.55	0.00598915149061586\\
33.56	0.00598916966239404\\
33.57	0.00598918784136972\\
33.58	0.00598920602754658\\
33.59	0.00598922422092833\\
33.6	0.00598924242151867\\
33.61	0.0059892606293213\\
33.62	0.00598927884433992\\
33.63	0.00598929706657824\\
33.64	0.00598931529603998\\
33.65	0.00598933353272884\\
33.66	0.00598935177664855\\
33.67	0.00598937002780282\\
33.68	0.00598938828619538\\
33.69	0.00598940655182995\\
33.7	0.00598942482471026\\
33.71	0.00598944310484004\\
33.72	0.00598946139222303\\
33.73	0.00598947968686296\\
33.74	0.00598949798876357\\
33.75	0.00598951629792861\\
33.76	0.00598953461436182\\
33.77	0.00598955293806696\\
33.78	0.00598957126904777\\
33.79	0.005989589607308\\
33.8	0.00598960795285143\\
33.81	0.00598962630568179\\
33.82	0.00598964466580288\\
33.83	0.00598966303321843\\
33.84	0.00598968140793224\\
33.85	0.00598969978994806\\
33.86	0.00598971817926968\\
33.87	0.00598973657590087\\
33.88	0.00598975497984542\\
33.89	0.00598977339110711\\
33.9	0.00598979180968972\\
33.91	0.00598981023559706\\
33.92	0.0059898286688329\\
33.93	0.00598984710940105\\
33.94	0.00598986555730531\\
33.95	0.00598988401254948\\
33.96	0.00598990247513737\\
33.97	0.00598992094507278\\
33.98	0.00598993942235952\\
33.99	0.00598995790700141\\
34	0.00598997639900227\\
34.01	0.00598999489836592\\
34.02	0.00599001340509618\\
34.03	0.00599003191919688\\
34.04	0.00599005044067185\\
34.05	0.00599006896952491\\
34.06	0.00599008750575992\\
34.07	0.0059901060493807\\
34.08	0.00599012460039109\\
34.09	0.00599014315879495\\
34.1	0.00599016172459612\\
34.11	0.00599018029779844\\
34.12	0.00599019887840579\\
34.13	0.005990217466422\\
34.14	0.00599023606185095\\
34.15	0.00599025466469649\\
34.16	0.0059902732749625\\
34.17	0.00599029189265283\\
34.18	0.00599031051777137\\
34.19	0.005990329150322\\
34.2	0.00599034779030857\\
34.21	0.00599036643773499\\
34.22	0.00599038509260513\\
34.23	0.00599040375492289\\
34.24	0.00599042242469215\\
34.25	0.00599044110191681\\
34.26	0.00599045978660077\\
34.27	0.00599047847874792\\
34.28	0.00599049717836217\\
34.29	0.00599051588544743\\
34.3	0.00599053460000761\\
34.31	0.00599055332204661\\
34.32	0.00599057205156836\\
34.33	0.00599059078857677\\
34.34	0.00599060953307578\\
34.35	0.00599062828506929\\
34.36	0.00599064704456125\\
34.37	0.00599066581155558\\
34.38	0.00599068458605621\\
34.39	0.0059907033680671\\
34.4	0.00599072215759218\\
34.41	0.00599074095463538\\
34.42	0.00599075975920066\\
34.43	0.00599077857129198\\
34.44	0.00599079739091328\\
34.45	0.00599081621806853\\
34.46	0.00599083505276167\\
34.47	0.00599085389499669\\
34.48	0.00599087274477753\\
34.49	0.00599089160210818\\
34.5	0.00599091046699261\\
34.51	0.00599092933943479\\
34.52	0.0059909482194387\\
34.53	0.00599096710700833\\
34.54	0.00599098600214766\\
34.55	0.00599100490486068\\
34.56	0.00599102381515139\\
34.57	0.00599104273302377\\
34.58	0.00599106165848183\\
34.59	0.00599108059152957\\
34.6	0.005991099532171\\
34.61	0.00599111848041012\\
34.62	0.00599113743625096\\
34.63	0.00599115639969751\\
34.64	0.0059911753707538\\
34.65	0.00599119434942386\\
34.66	0.0059912133357117\\
34.67	0.00599123232962135\\
34.68	0.00599125133115686\\
34.69	0.00599127034032225\\
34.7	0.00599128935712156\\
34.71	0.00599130838155883\\
34.72	0.0059913274136381\\
34.73	0.00599134645336344\\
34.74	0.00599136550073887\\
34.75	0.00599138455576847\\
34.76	0.00599140361845629\\
34.77	0.00599142268880638\\
34.78	0.00599144176682282\\
34.79	0.00599146085250967\\
34.8	0.00599147994587101\\
34.81	0.0059914990469109\\
34.82	0.00599151815563344\\
34.83	0.00599153727204268\\
34.84	0.00599155639614273\\
34.85	0.00599157552793767\\
34.86	0.00599159466743159\\
34.87	0.00599161381462858\\
34.88	0.00599163296953274\\
34.89	0.00599165213214818\\
34.9	0.005991671302479\\
34.91	0.00599169048052931\\
34.92	0.00599170966630322\\
34.93	0.00599172885980484\\
34.94	0.00599174806103828\\
34.95	0.00599176727000769\\
34.96	0.00599178648671717\\
34.97	0.00599180571117086\\
34.98	0.00599182494337288\\
34.99	0.00599184418332738\\
35	0.00599186343103849\\
35.01	0.00599188268651035\\
35.02	0.00599190194974711\\
35.03	0.00599192122075293\\
35.04	0.00599194049953194\\
35.05	0.0059919597860883\\
35.06	0.00599197908042618\\
35.07	0.00599199838254974\\
35.08	0.00599201769246313\\
35.09	0.00599203701017054\\
35.1	0.00599205633567614\\
35.11	0.00599207566898409\\
35.12	0.00599209501009859\\
35.13	0.00599211435902381\\
35.14	0.00599213371576394\\
35.15	0.00599215308032318\\
35.16	0.0059921724527057\\
35.17	0.00599219183291572\\
35.18	0.00599221122095743\\
35.19	0.00599223061683504\\
35.2	0.00599225002055275\\
35.21	0.00599226943211477\\
35.22	0.00599228885152532\\
35.23	0.00599230827878862\\
35.24	0.00599232771390889\\
35.25	0.00599234715689035\\
35.26	0.00599236660773723\\
35.27	0.00599238606645376\\
35.28	0.00599240553304419\\
35.29	0.00599242500751274\\
35.3	0.00599244448986367\\
35.31	0.00599246398010121\\
35.32	0.00599248347822961\\
35.33	0.00599250298425314\\
35.34	0.00599252249817605\\
35.35	0.00599254202000259\\
35.36	0.00599256154973703\\
35.37	0.00599258108738364\\
35.38	0.00599260063294669\\
35.39	0.00599262018643046\\
35.4	0.00599263974783922\\
35.41	0.00599265931717725\\
35.42	0.00599267889444884\\
35.43	0.00599269847965827\\
35.44	0.00599271807280985\\
35.45	0.00599273767390787\\
35.46	0.00599275728295662\\
35.47	0.00599277689996041\\
35.48	0.00599279652492355\\
35.49	0.00599281615785035\\
35.5	0.00599283579874511\\
35.51	0.00599285544761217\\
35.52	0.00599287510445583\\
35.53	0.00599289476928044\\
35.54	0.00599291444209031\\
35.55	0.00599293412288977\\
35.56	0.00599295381168317\\
35.57	0.00599297350847483\\
35.58	0.00599299321326911\\
35.59	0.00599301292607037\\
35.6	0.00599303264688292\\
35.61	0.00599305237571115\\
35.62	0.0059930721125594\\
35.63	0.00599309185743204\\
35.64	0.00599311161033343\\
35.65	0.00599313137126794\\
35.66	0.00599315114023995\\
35.67	0.00599317091725382\\
35.68	0.00599319070231395\\
35.69	0.00599321049542471\\
35.7	0.00599323029659049\\
35.71	0.00599325010581568\\
35.72	0.00599326992310468\\
35.73	0.00599328974846189\\
35.74	0.00599330958189171\\
35.75	0.00599332942339853\\
35.76	0.00599334927298679\\
35.77	0.00599336913066089\\
35.78	0.00599338899642523\\
35.79	0.00599340887028426\\
35.8	0.00599342875224239\\
35.81	0.00599344864230405\\
35.82	0.00599346854047367\\
35.83	0.00599348844675568\\
35.84	0.00599350836115454\\
35.85	0.00599352828367468\\
35.86	0.00599354821432055\\
35.87	0.00599356815309661\\
35.88	0.00599358810000729\\
35.89	0.00599360805505708\\
35.9	0.00599362801825041\\
35.91	0.00599364798959178\\
35.92	0.00599366796908564\\
35.93	0.00599368795673646\\
35.94	0.00599370795254873\\
35.95	0.00599372795652693\\
35.96	0.00599374796867555\\
35.97	0.00599376798899907\\
35.98	0.00599378801750198\\
35.99	0.00599380805418879\\
36	0.00599382809906399\\
36.01	0.00599384815213209\\
36.02	0.0059938682133976\\
36.03	0.00599388828286503\\
36.04	0.0059939083605389\\
36.05	0.00599392844642373\\
36.06	0.00599394854052403\\
36.07	0.00599396864284435\\
36.08	0.0059939887533892\\
36.09	0.00599400887216314\\
36.1	0.00599402899917069\\
36.11	0.0059940491344164\\
36.12	0.00599406927790483\\
36.13	0.0059940894296405\\
36.14	0.005994109589628\\
36.15	0.00599412975787187\\
36.16	0.00599414993437668\\
36.17	0.00599417011914699\\
36.18	0.00599419031218738\\
36.19	0.00599421051350242\\
36.2	0.00599423072309669\\
36.21	0.00599425094097477\\
36.22	0.00599427116714125\\
36.23	0.00599429140160073\\
36.24	0.00599431164435778\\
36.25	0.00599433189541703\\
36.26	0.00599435215478306\\
36.27	0.00599437242246048\\
36.28	0.00599439269845391\\
36.29	0.00599441298276796\\
36.3	0.00599443327540725\\
36.31	0.0059944535763764\\
36.32	0.00599447388568003\\
36.33	0.00599449420332279\\
36.34	0.0059945145293093\\
36.35	0.0059945348636442\\
36.36	0.00599455520633214\\
36.37	0.00599457555737776\\
36.38	0.00599459591678571\\
36.39	0.00599461628456065\\
36.4	0.00599463666070724\\
36.41	0.00599465704523014\\
36.42	0.00599467743813401\\
36.43	0.00599469783942354\\
36.44	0.00599471824910339\\
36.45	0.00599473866717824\\
36.46	0.00599475909365278\\
36.47	0.00599477952853169\\
36.48	0.00599479997181966\\
36.49	0.00599482042352141\\
36.5	0.0059948408836416\\
36.51	0.00599486135218497\\
36.52	0.00599488182915621\\
36.53	0.00599490231456003\\
36.54	0.00599492280840116\\
36.55	0.00599494331068431\\
36.56	0.0059949638214142\\
36.57	0.00599498434059557\\
36.58	0.00599500486823315\\
36.59	0.00599502540433168\\
36.6	0.00599504594889588\\
36.61	0.00599506650193052\\
36.62	0.00599508706344034\\
36.63	0.0059951076334301\\
36.64	0.00599512821190455\\
36.65	0.00599514879886845\\
36.66	0.00599516939432657\\
36.67	0.00599518999828368\\
36.68	0.00599521061074456\\
36.69	0.00599523123171398\\
36.7	0.00599525186119673\\
36.71	0.00599527249919759\\
36.72	0.00599529314572135\\
36.73	0.00599531380077282\\
36.74	0.00599533446435678\\
36.75	0.00599535513647806\\
36.76	0.00599537581714144\\
36.77	0.00599539650635175\\
36.78	0.0059954172041138\\
36.79	0.00599543791043241\\
36.8	0.00599545862531241\\
36.81	0.00599547934875863\\
36.82	0.0059955000807759\\
36.83	0.00599552082136907\\
36.84	0.00599554157054295\\
36.85	0.00599556232830243\\
36.86	0.00599558309465233\\
36.87	0.00599560386959751\\
36.88	0.00599562465314284\\
36.89	0.00599564544529318\\
36.9	0.00599566624605339\\
36.91	0.00599568705542835\\
36.92	0.00599570787342293\\
36.93	0.00599572870004203\\
36.94	0.00599574953529051\\
36.95	0.00599577037917327\\
36.96	0.00599579123169522\\
36.97	0.00599581209286123\\
36.98	0.00599583296267622\\
36.99	0.0059958538411451\\
37	0.00599587472827277\\
37.01	0.00599589562406416\\
37.02	0.00599591652852418\\
37.03	0.00599593744165775\\
37.04	0.00599595836346981\\
37.05	0.00599597929396528\\
37.06	0.00599600023314912\\
37.07	0.00599602118102625\\
37.08	0.00599604213760163\\
37.09	0.00599606310288021\\
37.1	0.00599608407686694\\
37.11	0.00599610505956678\\
37.12	0.00599612605098471\\
37.13	0.00599614705112567\\
37.14	0.00599616805999465\\
37.15	0.00599618907759663\\
37.16	0.0059962101039366\\
37.17	0.00599623113901952\\
37.18	0.00599625218285041\\
37.19	0.00599627323543424\\
37.2	0.00599629429677603\\
37.21	0.00599631536688078\\
37.22	0.00599633644575349\\
37.23	0.00599635753339918\\
37.24	0.00599637862982288\\
37.25	0.00599639973502959\\
37.26	0.00599642084902434\\
37.27	0.00599644197181218\\
37.28	0.00599646310339813\\
37.29	0.00599648424378724\\
37.3	0.00599650539298455\\
37.31	0.0059965265509951\\
37.32	0.00599654771782397\\
37.33	0.00599656889347619\\
37.34	0.00599659007795685\\
37.35	0.005996611271271\\
37.36	0.00599663247342371\\
37.37	0.00599665368442007\\
37.38	0.00599667490426516\\
37.39	0.00599669613296406\\
37.4	0.00599671737052186\\
37.41	0.00599673861694366\\
37.42	0.00599675987223456\\
37.43	0.00599678113639966\\
37.44	0.00599680240944408\\
37.45	0.00599682369137292\\
37.46	0.00599684498219131\\
37.47	0.00599686628190436\\
37.48	0.00599688759051721\\
37.49	0.00599690890803499\\
37.5	0.00599693023446283\\
37.51	0.00599695156980588\\
37.52	0.00599697291406929\\
37.53	0.00599699426725819\\
37.54	0.00599701562937776\\
37.55	0.00599703700043315\\
37.56	0.00599705838042952\\
37.57	0.00599707976937205\\
37.58	0.00599710116726591\\
37.59	0.00599712257411628\\
37.6	0.00599714398992834\\
37.61	0.00599716541470729\\
37.62	0.00599718684845831\\
37.63	0.0059972082911866\\
37.64	0.00599722974289736\\
37.65	0.00599725120359581\\
37.66	0.00599727267328716\\
37.67	0.00599729415197662\\
37.68	0.00599731563966941\\
37.69	0.00599733713637076\\
37.7	0.00599735864208591\\
37.71	0.00599738015682008\\
37.72	0.00599740168057853\\
37.73	0.00599742321336648\\
37.74	0.0059974447551892\\
37.75	0.00599746630605195\\
37.76	0.00599748786595997\\
37.77	0.00599750943491854\\
37.78	0.00599753101293292\\
37.79	0.00599755260000839\\
37.8	0.00599757419615023\\
37.81	0.00599759580136373\\
37.82	0.00599761741565416\\
37.83	0.00599763903902683\\
37.84	0.00599766067148703\\
37.85	0.00599768231304008\\
37.86	0.00599770396369126\\
37.87	0.0059977256234459\\
37.88	0.00599774729230933\\
37.89	0.00599776897028685\\
37.9	0.0059977906573838\\
37.91	0.00599781235360551\\
37.92	0.00599783405895732\\
37.93	0.00599785577344457\\
37.94	0.00599787749707261\\
37.95	0.00599789922984679\\
37.96	0.00599792097177247\\
37.97	0.00599794272285501\\
37.98	0.00599796448309978\\
37.99	0.00599798625251214\\
38	0.00599800803109748\\
38.01	0.00599802981886118\\
38.02	0.00599805161580863\\
38.03	0.00599807342194521\\
38.04	0.00599809523727633\\
38.05	0.00599811706180738\\
38.06	0.00599813889554378\\
38.07	0.00599816073849093\\
38.08	0.00599818259065426\\
38.09	0.00599820445203917\\
38.1	0.00599822632265111\\
38.11	0.0059982482024955\\
38.12	0.00599827009157777\\
38.13	0.00599829198990338\\
38.14	0.00599831389747777\\
38.15	0.00599833581430639\\
38.16	0.00599835774039469\\
38.17	0.00599837967574814\\
38.18	0.00599840162037221\\
38.19	0.00599842357427236\\
38.2	0.00599844553745408\\
38.21	0.00599846750992285\\
38.22	0.00599848949168415\\
38.23	0.00599851148274347\\
38.24	0.00599853348310633\\
38.25	0.0059985554927782\\
38.26	0.00599857751176462\\
38.27	0.00599859954007108\\
38.28	0.0059986215777031\\
38.29	0.00599864362466622\\
38.3	0.00599866568096595\\
38.31	0.00599868774660783\\
38.32	0.0059987098215974\\
38.33	0.0059987319059402\\
38.34	0.00599875399964178\\
38.35	0.00599877610270769\\
38.36	0.0059987982151435\\
38.37	0.00599882033695476\\
38.38	0.00599884246814705\\
38.39	0.00599886460872595\\
38.4	0.00599888675869702\\
38.41	0.00599890891806586\\
38.42	0.00599893108683805\\
38.43	0.0059989532650192\\
38.44	0.00599897545261489\\
38.45	0.00599899764963075\\
38.46	0.00599901985607237\\
38.47	0.00599904207194537\\
38.48	0.00599906429725538\\
38.49	0.00599908653200803\\
38.5	0.00599910877620893\\
38.51	0.00599913102986374\\
38.52	0.00599915329297809\\
38.53	0.00599917556555763\\
38.54	0.00599919784760802\\
38.55	0.00599922013913491\\
38.56	0.00599924244014396\\
38.57	0.00599926475064084\\
38.58	0.00599928707063124\\
38.59	0.00599930940012082\\
38.6	0.00599933173911528\\
38.61	0.00599935408762029\\
38.62	0.00599937644564156\\
38.63	0.00599939881318479\\
38.64	0.00599942119025569\\
38.65	0.00599944357685995\\
38.66	0.00599946597300331\\
38.67	0.00599948837869148\\
38.68	0.00599951079393019\\
38.69	0.00599953321872517\\
38.7	0.00599955565308216\\
38.71	0.00599957809700691\\
38.72	0.00599960055050515\\
38.73	0.00599962301358266\\
38.74	0.00599964548624517\\
38.75	0.00599966796849847\\
38.76	0.00599969046034832\\
38.77	0.00599971296180049\\
38.78	0.00599973547286077\\
38.79	0.00599975799353494\\
38.8	0.0059997805238288\\
38.81	0.00599980306374814\\
38.82	0.00599982561329876\\
38.83	0.00599984817248647\\
38.84	0.0059998707413171\\
38.85	0.00599989331979644\\
38.86	0.00599991590793033\\
38.87	0.00599993850572461\\
38.88	0.00599996111318509\\
38.89	0.00599998373031763\\
38.9	0.00600000635712808\\
38.91	0.00600002899362227\\
38.92	0.00600005163980608\\
38.93	0.00600007429568535\\
38.94	0.00600009696126597\\
38.95	0.0060001196365538\\
38.96	0.00600014232155472\\
38.97	0.00600016501627462\\
38.98	0.00600018772071939\\
38.99	0.00600021043489492\\
39	0.00600023315880712\\
39.01	0.00600025589246188\\
39.02	0.00600027863586514\\
39.03	0.00600030138902279\\
39.04	0.00600032415194077\\
39.05	0.006000346924625\\
39.06	0.00600036970708142\\
39.07	0.00600039249931596\\
39.08	0.00600041530133458\\
39.09	0.00600043811314323\\
39.1	0.00600046093474785\\
39.11	0.00600048376615442\\
39.12	0.0060005066073689\\
39.13	0.00600052945839726\\
39.14	0.00600055231924549\\
39.15	0.00600057518991957\\
39.16	0.00600059807042548\\
39.17	0.00600062096076923\\
39.18	0.00600064386095681\\
39.19	0.00600066677099424\\
39.2	0.00600068969088753\\
39.21	0.00600071262064268\\
39.22	0.00600073556026574\\
39.23	0.00600075850976272\\
39.24	0.00600078146913967\\
39.25	0.00600080443840262\\
39.26	0.00600082741755762\\
39.27	0.00600085040661073\\
39.28	0.00600087340556799\\
39.29	0.00600089641443548\\
39.3	0.00600091943321926\\
39.31	0.00600094246192542\\
39.32	0.00600096550056002\\
39.33	0.00600098854912915\\
39.34	0.00600101160763892\\
39.35	0.0060010346760954\\
39.36	0.00600105775450472\\
39.37	0.00600108084287298\\
39.38	0.00600110394120628\\
39.39	0.00600112704951077\\
39.4	0.00600115016779255\\
39.41	0.00600117329605776\\
39.42	0.00600119643431255\\
39.43	0.00600121958256304\\
39.44	0.0060012427408154\\
39.45	0.00600126590907577\\
39.46	0.00600128908735033\\
39.47	0.00600131227564523\\
39.48	0.00600133547396664\\
39.49	0.00600135868232075\\
39.5	0.00600138190071374\\
39.51	0.0060014051291518\\
39.52	0.00600142836764112\\
39.53	0.00600145161618791\\
39.54	0.00600147487479837\\
39.55	0.00600149814347871\\
39.56	0.00600152142223516\\
39.57	0.00600154471107393\\
39.58	0.00600156801000126\\
39.59	0.00600159131902339\\
39.6	0.00600161463814656\\
39.61	0.006001637967377\\
39.62	0.00600166130672098\\
39.63	0.00600168465618476\\
39.64	0.0060017080157746\\
39.65	0.00600173138549677\\
39.66	0.00600175476535755\\
39.67	0.00600177815536322\\
39.68	0.00600180155552008\\
39.69	0.0060018249658344\\
39.7	0.00600184838631251\\
39.71	0.0060018718169607\\
39.72	0.00600189525778527\\
39.73	0.00600191870879256\\
39.74	0.00600194216998888\\
39.75	0.00600196564138056\\
39.76	0.00600198912297395\\
39.77	0.00600201261477538\\
39.78	0.00600203611679119\\
39.79	0.00600205962902775\\
39.8	0.00600208315149141\\
39.81	0.00600210668418853\\
39.82	0.00600213022712549\\
39.83	0.00600215378030866\\
39.84	0.00600217734374443\\
39.85	0.00600220091743919\\
39.86	0.00600222450139932\\
39.87	0.00600224809563123\\
39.88	0.00600227170014134\\
39.89	0.00600229531493604\\
39.9	0.00600231894002176\\
39.91	0.00600234257540492\\
39.92	0.00600236622109195\\
39.93	0.0060023898770893\\
39.94	0.00600241354340339\\
39.95	0.00600243722004069\\
39.96	0.00600246090700764\\
39.97	0.0060024846043107\\
39.98	0.00600250831195634\\
39.99	0.00600253202995104\\
40	0.00600255575830126\\
40.01	0.00600257949701351\\
};
\addplot [color=blue,solid,forget plot]
  table[row sep=crcr]{%
40.01	0.00600257949701351\\
40.02	0.00600260324609425\\
40.03	0.00600262700555\\
40.04	0.00600265077538725\\
40.05	0.00600267455561252\\
40.06	0.00600269834623231\\
40.07	0.00600272214725314\\
40.08	0.00600274595868155\\
40.09	0.00600276978052407\\
40.1	0.00600279361278724\\
40.11	0.00600281745547758\\
40.12	0.00600284130860167\\
40.13	0.00600286517216606\\
40.14	0.00600288904617731\\
40.15	0.00600291293064198\\
40.16	0.00600293682556665\\
40.17	0.00600296073095792\\
40.18	0.00600298464682235\\
40.19	0.00600300857316654\\
40.2	0.0060030325099971\\
40.21	0.00600305645732063\\
40.22	0.00600308041514374\\
40.23	0.00600310438347305\\
40.24	0.00600312836231518\\
40.25	0.00600315235167676\\
40.26	0.00600317635156443\\
40.27	0.00600320036198484\\
40.28	0.00600322438294462\\
40.29	0.00600324841445043\\
40.3	0.00600327245650895\\
40.31	0.00600329650912682\\
40.32	0.00600332057231073\\
40.33	0.00600334464606735\\
40.34	0.00600336873040338\\
40.35	0.00600339282532549\\
40.36	0.0060034169308404\\
40.37	0.0060034410469548\\
40.38	0.0060034651736754\\
40.39	0.00600348931100893\\
40.4	0.0060035134589621\\
40.41	0.00600353761754165\\
40.42	0.0060035617867543\\
40.43	0.00600358596660681\\
40.44	0.00600361015710592\\
40.45	0.00600363435825838\\
40.46	0.00600365857007096\\
40.47	0.00600368279255042\\
40.48	0.00600370702570353\\
40.49	0.00600373126953708\\
40.5	0.00600375552405786\\
40.51	0.00600377978927265\\
40.52	0.00600380406518825\\
40.53	0.00600382835181147\\
40.54	0.00600385264914913\\
40.55	0.00600387695720803\\
40.56	0.006003901275995\\
40.57	0.00600392560551688\\
40.58	0.0060039499457805\\
40.59	0.00600397429679271\\
40.6	0.00600399865856036\\
40.61	0.0060040230310903\\
40.62	0.0060040474143894\\
40.63	0.00600407180846452\\
40.64	0.00600409621332255\\
40.65	0.00600412062897037\\
40.66	0.00600414505541486\\
40.67	0.00600416949266293\\
40.68	0.00600419394072148\\
40.69	0.00600421839959741\\
40.7	0.00600424286929764\\
40.71	0.0060042673498291\\
40.72	0.00600429184119872\\
40.73	0.00600431634341343\\
40.74	0.00600434085648017\\
40.75	0.0060043653804059\\
40.76	0.00600438991519757\\
40.77	0.00600441446086214\\
40.78	0.00600443901740659\\
40.79	0.0060044635848379\\
40.8	0.00600448816316304\\
40.81	0.006004512752389\\
40.82	0.00600453735252279\\
40.83	0.00600456196357141\\
40.84	0.00600458658554187\\
40.85	0.0060046112184412\\
40.86	0.00600463586227641\\
40.87	0.00600466051705454\\
40.88	0.00600468518278263\\
40.89	0.00600470985946772\\
40.9	0.00600473454711687\\
40.91	0.00600475924573714\\
40.92	0.00600478395533561\\
40.93	0.00600480867591933\\
40.94	0.00600483340749541\\
40.95	0.00600485815007092\\
40.96	0.00600488290365297\\
40.97	0.00600490766824866\\
40.98	0.0060049324438651\\
40.99	0.00600495723050941\\
41	0.00600498202818872\\
41.01	0.00600500683691016\\
41.02	0.00600503165668089\\
41.03	0.00600505648750803\\
41.04	0.00600508132939876\\
41.05	0.00600510618236024\\
41.06	0.00600513104639964\\
41.07	0.00600515592152414\\
41.08	0.00600518080774093\\
41.09	0.0060052057050572\\
41.1	0.00600523061348017\\
41.11	0.00600525553301704\\
41.12	0.00600528046367503\\
41.13	0.00600530540546137\\
41.14	0.00600533035838329\\
41.15	0.00600535532244804\\
41.16	0.00600538029766288\\
41.17	0.00600540528403506\\
41.18	0.00600543028157186\\
41.19	0.00600545529028054\\
41.2	0.00600548031016839\\
41.21	0.00600550534124272\\
41.22	0.00600553038351082\\
41.23	0.00600555543698\\
41.24	0.00600558050165758\\
41.25	0.0060056055775509\\
41.26	0.00600563066466728\\
41.27	0.00600565576301408\\
41.28	0.00600568087259865\\
41.29	0.00600570599342836\\
41.3	0.00600573112551057\\
41.31	0.00600575626885268\\
41.32	0.00600578142346207\\
41.33	0.00600580658934614\\
41.34	0.0060058317665123\\
41.35	0.00600585695496798\\
41.36	0.00600588215472061\\
41.37	0.00600590736577761\\
41.38	0.00600593258814644\\
41.39	0.00600595782183456\\
41.4	0.00600598306684944\\
41.41	0.00600600832319855\\
41.42	0.00600603359088938\\
41.43	0.00600605886992944\\
41.44	0.00600608416032622\\
41.45	0.00600610946208725\\
41.46	0.00600613477522005\\
41.47	0.00600616009973218\\
41.48	0.00600618543563117\\
41.49	0.00600621078292458\\
41.5	0.00600623614162\\
41.51	0.00600626151172499\\
41.52	0.00600628689324715\\
41.53	0.0060063122861941\\
41.54	0.00600633769057345\\
41.55	0.00600636310639281\\
41.56	0.00600638853365983\\
41.57	0.00600641397238217\\
41.58	0.00600643942256747\\
41.59	0.00600646488422342\\
41.6	0.0060064903573577\\
41.61	0.00600651584197801\\
41.62	0.00600654133809206\\
41.63	0.00600656684570756\\
41.64	0.00600659236483225\\
41.65	0.00600661789547388\\
41.66	0.00600664343764021\\
41.67	0.00600666899133902\\
41.68	0.00600669455657807\\
41.69	0.00600672013336517\\
41.7	0.00600674572170814\\
41.71	0.0060067713216148\\
41.72	0.00600679693309298\\
41.73	0.00600682255615053\\
41.74	0.00600684819079532\\
41.75	0.00600687383703523\\
41.76	0.00600689949487816\\
41.77	0.006006925164332\\
41.78	0.00600695084540467\\
41.79	0.00600697653810412\\
41.8	0.0060070022424383\\
41.81	0.00600702795841516\\
41.82	0.00600705368604268\\
41.83	0.00600707942532887\\
41.84	0.00600710517628172\\
41.85	0.00600713093890927\\
41.86	0.00600715671321955\\
41.87	0.00600718249922063\\
41.88	0.00600720829692056\\
41.89	0.00600723410632744\\
41.9	0.00600725992744938\\
41.91	0.00600728576029448\\
41.92	0.00600731160487088\\
41.93	0.00600733746118675\\
41.94	0.00600736332925024\\
41.95	0.00600738920906955\\
41.96	0.00600741510065287\\
41.97	0.00600744100400842\\
41.98	0.00600746691914445\\
41.99	0.0060074928460692\\
42	0.00600751878479094\\
42.01	0.00600754473531797\\
42.02	0.0060075706976586\\
42.03	0.00600759667182115\\
42.04	0.00600762265781396\\
42.05	0.00600764865564539\\
42.06	0.00600767466532383\\
42.07	0.00600770068685767\\
42.08	0.00600772672025533\\
42.09	0.00600775276552525\\
42.1	0.00600777882267587\\
42.11	0.00600780489171569\\
42.12	0.00600783097265319\\
42.13	0.00600785706549688\\
42.14	0.0060078831702553\\
42.15	0.00600790928693699\\
42.16	0.00600793541555053\\
42.17	0.00600796155610452\\
42.18	0.00600798770860756\\
42.19	0.00600801387306828\\
42.2	0.00600804004949534\\
42.21	0.00600806623789741\\
42.22	0.00600809243828318\\
42.23	0.00600811865066137\\
42.24	0.0060081448750407\\
42.25	0.00600817111142993\\
42.26	0.00600819735983785\\
42.27	0.00600822362027323\\
42.28	0.0060082498927449\\
42.29	0.00600827617726169\\
42.3	0.00600830247383246\\
42.31	0.0060083287824661\\
42.32	0.0060083551031715\\
42.33	0.00600838143595758\\
42.34	0.00600840778083327\\
42.35	0.00600843413780755\\
42.36	0.0060084605068894\\
42.37	0.00600848688808782\\
42.38	0.00600851328141182\\
42.39	0.00600853968687047\\
42.4	0.00600856610447283\\
42.41	0.00600859253422797\\
42.42	0.00600861897614502\\
42.43	0.00600864543023309\\
42.44	0.00600867189650133\\
42.45	0.00600869837495891\\
42.46	0.00600872486561503\\
42.47	0.00600875136847888\\
42.48	0.00600877788355969\\
42.49	0.00600880441086672\\
42.5	0.00600883095040922\\
42.51	0.00600885750219649\\
42.52	0.00600888406623783\\
42.53	0.00600891064254255\\
42.54	0.00600893723112001\\
42.55	0.00600896383197956\\
42.56	0.00600899044513058\\
42.57	0.00600901707058246\\
42.58	0.00600904370834461\\
42.59	0.00600907035842647\\
42.6	0.00600909702083749\\
42.61	0.00600912369558711\\
42.62	0.00600915038268482\\
42.63	0.0060091770821401\\
42.64	0.00600920379396247\\
42.65	0.00600923051816145\\
42.66	0.00600925725474657\\
42.67	0.00600928400372738\\
42.68	0.00600931076511344\\
42.69	0.00600933753891431\\
42.7	0.0060093643251396\\
42.71	0.00600939112379888\\
42.72	0.00600941793490178\\
42.73	0.00600944475845789\\
42.74	0.00600947159447685\\
42.75	0.00600949844296829\\
42.76	0.00600952530394185\\
42.77	0.00600955217740717\\
42.78	0.00600957906337391\\
42.79	0.00600960596185173\\
42.8	0.00600963287285029\\
42.81	0.00600965979637927\\
42.82	0.00600968673244832\\
42.83	0.00600971368106712\\
42.84	0.00600974064224535\\
42.85	0.00600976761599268\\
42.86	0.0060097946023188\\
42.87	0.00600982160123336\\
42.88	0.00600984861274604\\
42.89	0.00600987563686651\\
42.9	0.00600990267360444\\
42.91	0.00600992972296949\\
42.92	0.00600995678497131\\
42.93	0.00600998385961954\\
42.94	0.00601001094692384\\
42.95	0.00601003804689384\\
42.96	0.00601006515953915\\
42.97	0.0060100922848694\\
42.98	0.00601011942289419\\
42.99	0.00601014657362311\\
43	0.00601017373706574\\
43.01	0.00601020091323166\\
43.02	0.00601022810213042\\
43.03	0.00601025530377155\\
43.04	0.00601028251816459\\
43.05	0.00601030974531904\\
43.06	0.00601033698524441\\
43.07	0.00601036423795017\\
43.08	0.00601039150344579\\
43.09	0.00601041878174071\\
43.1	0.00601044607284438\\
43.11	0.00601047337676619\\
43.12	0.00601050069351556\\
43.13	0.00601052802310187\\
43.14	0.00601055536553448\\
43.15	0.00601058272082275\\
43.16	0.00601061008897601\\
43.17	0.0060106374700036\\
43.18	0.00601066486391483\\
43.19	0.00601069227071901\\
43.2	0.00601071969042541\\
43.21	0.00601074712304334\\
43.22	0.00601077456858209\\
43.23	0.00601080202705091\\
43.24	0.00601082949845911\\
43.25	0.00601085698281595\\
43.26	0.00601088448013074\\
43.27	0.00601091199041278\\
43.28	0.00601093951367136\\
43.29	0.00601096704991581\\
43.3	0.00601099459915548\\
43.31	0.0060110221613997\\
43.32	0.00601104973665783\\
43.33	0.00601107732493922\\
43.34	0.00601110492625327\\
43.35	0.00601113254060934\\
43.36	0.00601116016801685\\
43.37	0.00601118780848521\\
43.38	0.00601121546202381\\
43.39	0.00601124312864211\\
43.4	0.00601127080834954\\
43.41	0.00601129850115556\\
43.42	0.00601132620706962\\
43.43	0.0060113539261012\\
43.44	0.00601138165825979\\
43.45	0.00601140940355488\\
43.46	0.00601143716199598\\
43.47	0.0060114649335926\\
43.48	0.00601149271835428\\
43.49	0.00601152051629056\\
43.5	0.00601154832741098\\
43.51	0.00601157615172512\\
43.52	0.00601160398924253\\
43.53	0.00601163183997282\\
43.54	0.00601165970392557\\
43.55	0.00601168758111038\\
43.56	0.00601171547153688\\
43.57	0.00601174337521471\\
43.58	0.00601177129215348\\
43.59	0.00601179922236287\\
43.6	0.00601182716585253\\
43.61	0.00601185512263213\\
43.62	0.00601188309271136\\
43.63	0.00601191107609991\\
43.64	0.0060119390728075\\
43.65	0.00601196708284384\\
43.66	0.00601199510621867\\
43.67	0.00601202314294172\\
43.68	0.00601205119302274\\
43.69	0.00601207925647151\\
43.7	0.00601210733329779\\
43.71	0.00601213542351138\\
43.72	0.00601216352712207\\
43.73	0.00601219164413966\\
43.74	0.00601221977457399\\
43.75	0.00601224791843488\\
43.76	0.00601227607573219\\
43.77	0.00601230424647575\\
43.78	0.00601233243067545\\
43.79	0.00601236062834115\\
43.8	0.00601238883948276\\
43.81	0.00601241706411016\\
43.82	0.00601244530223328\\
43.83	0.00601247355386203\\
43.84	0.00601250181900635\\
43.85	0.00601253009767619\\
43.86	0.00601255838988152\\
43.87	0.00601258669563229\\
43.88	0.00601261501493849\\
43.89	0.00601264334781011\\
43.9	0.00601267169425717\\
43.91	0.00601270005428968\\
43.92	0.00601272842791765\\
43.93	0.00601275681515115\\
43.94	0.00601278521600021\\
43.95	0.00601281363047491\\
43.96	0.00601284205858531\\
43.97	0.00601287050034151\\
43.98	0.00601289895575359\\
43.99	0.00601292742483168\\
44	0.0060129559075859\\
44.01	0.00601298440402638\\
44.02	0.00601301291416326\\
44.03	0.0060130414380067\\
44.04	0.00601306997556687\\
44.05	0.00601309852685396\\
44.06	0.00601312709187815\\
44.07	0.00601315567064965\\
44.08	0.00601318426317868\\
44.09	0.00601321286947546\\
44.1	0.00601324148955025\\
44.11	0.00601327012341328\\
44.12	0.00601329877107482\\
44.13	0.00601332743254515\\
44.14	0.00601335610783455\\
44.15	0.00601338479695333\\
44.16	0.0060134134999118\\
44.17	0.00601344221672028\\
44.18	0.00601347094738911\\
44.19	0.00601349969192862\\
44.2	0.00601352845034919\\
44.21	0.00601355722266119\\
44.22	0.00601358600887499\\
44.23	0.006013614809001\\
44.24	0.00601364362304962\\
44.25	0.00601367245103127\\
44.26	0.00601370129295638\\
44.27	0.0060137301488354\\
44.28	0.00601375901867877\\
44.29	0.00601378790249698\\
44.3	0.00601381680030049\\
44.31	0.0060138457120998\\
44.32	0.00601387463790542\\
44.33	0.00601390357772785\\
44.34	0.00601393253157763\\
44.35	0.0060139614994653\\
44.36	0.00601399048140141\\
44.37	0.00601401947739652\\
44.38	0.00601404848746121\\
44.39	0.00601407751160606\\
44.4	0.00601410654984168\\
44.41	0.00601413560217868\\
44.42	0.00601416466862769\\
44.43	0.00601419374919934\\
44.44	0.00601422284390428\\
44.45	0.00601425195275317\\
44.46	0.00601428107575669\\
44.47	0.00601431021292552\\
44.48	0.00601433936427036\\
44.49	0.00601436852980191\\
44.5	0.00601439770953091\\
44.51	0.00601442690346808\\
44.52	0.00601445611162417\\
44.53	0.00601448533400994\\
44.54	0.00601451457063616\\
44.55	0.00601454382151361\\
44.56	0.00601457308665309\\
44.57	0.0060146023660654\\
44.58	0.00601463165976137\\
44.59	0.00601466096775183\\
44.6	0.00601469029004761\\
44.61	0.00601471962665958\\
44.62	0.0060147489775986\\
44.63	0.00601477834287556\\
44.64	0.00601480772250135\\
44.65	0.00601483711648686\\
44.66	0.00601486652484303\\
44.67	0.00601489594758077\\
44.68	0.00601492538471103\\
44.69	0.00601495483624476\\
44.7	0.00601498430219293\\
44.71	0.00601501378256651\\
44.72	0.0060150432773765\\
44.73	0.0060150727866339\\
44.74	0.00601510231034972\\
44.75	0.00601513184853499\\
44.76	0.00601516140120074\\
44.77	0.00601519096835803\\
44.78	0.00601522055001792\\
44.79	0.00601525014619149\\
44.8	0.00601527975688982\\
44.81	0.00601530938212401\\
44.82	0.00601533902190518\\
44.83	0.00601536867624444\\
44.84	0.00601539834515293\\
44.85	0.00601542802864181\\
44.86	0.00601545772672223\\
44.87	0.00601548743940535\\
44.88	0.00601551716670237\\
44.89	0.00601554690862448\\
44.9	0.00601557666518289\\
44.91	0.00601560643638883\\
44.92	0.00601563622225351\\
44.93	0.00601566602278818\\
44.94	0.00601569583800411\\
44.95	0.00601572566791255\\
44.96	0.0060157555125248\\
44.97	0.00601578537185213\\
44.98	0.00601581524590586\\
44.99	0.0060158451346973\\
45	0.00601587503823778\\
45.01	0.00601590495653862\\
45.02	0.0060159348896112\\
45.03	0.00601596483746687\\
45.04	0.006015994800117\\
45.05	0.00601602477757299\\
45.06	0.00601605476984622\\
45.07	0.00601608477694811\\
45.08	0.00601611479889008\\
45.09	0.00601614483568357\\
45.1	0.00601617488734002\\
45.11	0.00601620495387088\\
45.12	0.00601623503528762\\
45.13	0.00601626513160174\\
45.14	0.0060162952428247\\
45.15	0.00601632536896803\\
45.16	0.00601635551004322\\
45.17	0.00601638566606181\\
45.18	0.00601641583703534\\
45.19	0.00601644602297535\\
45.2	0.0060164762238934\\
45.21	0.00601650643980107\\
45.22	0.00601653667070994\\
45.23	0.00601656691663159\\
45.24	0.00601659717757764\\
45.25	0.00601662745355971\\
45.26	0.00601665774458941\\
45.27	0.00601668805067839\\
45.28	0.00601671837183829\\
45.29	0.00601674870808079\\
45.3	0.00601677905941754\\
45.31	0.00601680942586023\\
45.32	0.00601683980742054\\
45.33	0.0060168702041102\\
45.34	0.00601690061594091\\
45.35	0.0060169310429244\\
45.36	0.0060169614850724\\
45.37	0.00601699194239665\\
45.38	0.00601702241490893\\
45.39	0.00601705290262098\\
45.4	0.0060170834055446\\
45.41	0.00601711392369156\\
45.42	0.00601714445707368\\
45.43	0.00601717500570275\\
45.44	0.00601720556959059\\
45.45	0.00601723614874904\\
45.46	0.00601726674318994\\
45.47	0.00601729735292513\\
45.48	0.00601732797796647\\
45.49	0.00601735861832583\\
45.5	0.00601738927401509\\
45.51	0.00601741994504615\\
45.52	0.00601745063143088\\
45.53	0.00601748133318122\\
45.54	0.00601751205030905\\
45.55	0.00601754278282634\\
45.56	0.00601757353074499\\
45.57	0.00601760429407697\\
45.58	0.00601763507283421\\
45.59	0.0060176658670287\\
45.6	0.00601769667667239\\
45.61	0.00601772750177727\\
45.62	0.00601775834235533\\
45.63	0.00601778919841856\\
45.64	0.00601782006997898\\
45.65	0.0060178509570486\\
45.66	0.00601788185963944\\
45.67	0.00601791277776354\\
45.68	0.00601794371143293\\
45.69	0.00601797466065967\\
45.7	0.00601800562545581\\
45.71	0.00601803660583341\\
45.72	0.00601806760180455\\
45.73	0.00601809861338131\\
45.74	0.00601812964057577\\
45.75	0.00601816068340003\\
45.76	0.00601819174186618\\
45.77	0.00601822281598635\\
45.78	0.00601825390577263\\
45.79	0.00601828501123717\\
45.8	0.00601831613239207\\
45.81	0.00601834726924949\\
45.82	0.00601837842182156\\
45.83	0.00601840959012043\\
45.84	0.00601844077415826\\
45.85	0.0060184719739472\\
45.86	0.00601850318949942\\
45.87	0.0060185344208271\\
45.88	0.00601856566794242\\
45.89	0.00601859693085754\\
45.9	0.00601862820958467\\
45.91	0.006018659504136\\
45.92	0.00601869081452372\\
45.93	0.00601872214076004\\
45.94	0.00601875348285716\\
45.95	0.0060187848408273\\
45.96	0.00601881621468267\\
45.97	0.00601884760443549\\
45.98	0.00601887901009798\\
45.99	0.00601891043168238\\
46	0.00601894186920091\\
46.01	0.0060189733226658\\
46.02	0.00601900479208929\\
46.03	0.00601903627748363\\
46.04	0.00601906777886104\\
46.05	0.00601909929623379\\
46.06	0.0060191308296141\\
46.07	0.00601916237901423\\
46.08	0.00601919394444644\\
46.09	0.00601922552592296\\
46.1	0.00601925712345606\\
46.11	0.00601928873705799\\
46.12	0.00601932036674099\\
46.13	0.00601935201251733\\
46.14	0.00601938367439925\\
46.15	0.00601941535239901\\
46.16	0.00601944704652888\\
46.17	0.00601947875680109\\
46.18	0.0060195104832279\\
46.19	0.00601954222582156\\
46.2	0.00601957398459432\\
46.21	0.00601960575955844\\
46.22	0.00601963755072614\\
46.23	0.00601966935810969\\
46.24	0.00601970118172132\\
46.25	0.00601973302157326\\
46.26	0.00601976487767775\\
46.27	0.00601979675004702\\
46.28	0.0060198286386933\\
46.29	0.0060198605436288\\
46.3	0.00601989246486576\\
46.31	0.00601992440241637\\
46.32	0.00601995635629285\\
46.33	0.0060199883265074\\
46.34	0.00602002031307222\\
46.35	0.00602005231599948\\
46.36	0.00602008433530138\\
46.37	0.0060201163709901\\
46.38	0.00602014842307778\\
46.39	0.0060201804915766\\
46.4	0.00602021257649872\\
46.41	0.00602024467785626\\
46.42	0.00602027679566136\\
46.43	0.00602030892992615\\
46.44	0.00602034108066273\\
46.45	0.00602037324788322\\
46.46	0.00602040543159969\\
46.47	0.00602043763182423\\
46.48	0.00602046984856892\\
46.49	0.0060205020818458\\
46.5	0.00602053433166691\\
46.51	0.00602056659804429\\
46.52	0.00602059888098996\\
46.53	0.0060206311805159\\
46.54	0.00602066349663411\\
46.55	0.00602069582935657\\
46.56	0.00602072817869521\\
46.57	0.00602076054466199\\
46.58	0.00602079292726882\\
46.59	0.00602082532652761\\
46.6	0.00602085774245024\\
46.61	0.00602089017504857\\
46.62	0.00602092262433446\\
46.63	0.00602095509031973\\
46.64	0.0060209875730162\\
46.65	0.00602102007243563\\
46.66	0.00602105258858981\\
46.67	0.00602108512149046\\
46.68	0.00602111767114931\\
46.69	0.00602115023757806\\
46.7	0.00602118282078837\\
46.71	0.00602121542079189\\
46.72	0.00602124803760023\\
46.73	0.00602128067122499\\
46.74	0.00602131332167775\\
46.75	0.00602134598897002\\
46.76	0.00602137867311333\\
46.77	0.00602141137411915\\
46.78	0.00602144409199894\\
46.79	0.0060214768267641\\
46.8	0.00602150957842604\\
46.81	0.0060215423469961\\
46.82	0.00602157513248561\\
46.83	0.00602160793490584\\
46.84	0.00602164075426807\\
46.85	0.00602167359058349\\
46.86	0.0060217064438633\\
46.87	0.00602173931411864\\
46.88	0.0060217722013606\\
46.89	0.00602180510560026\\
46.9	0.00602183802684864\\
46.91	0.00602187096511673\\
46.92	0.00602190392041546\\
46.93	0.00602193689275574\\
46.94	0.00602196988214841\\
46.95	0.0060220028886043\\
46.96	0.00602203591213416\\
46.97	0.00602206895274872\\
46.98	0.00602210201045864\\
46.99	0.00602213508527454\\
47	0.00602216817720699\\
47.01	0.00602220128626651\\
47.02	0.00602223441246358\\
47.03	0.0060222675558086\\
47.04	0.00602230071631193\\
47.05	0.00602233389398389\\
47.06	0.0060223670888347\\
47.07	0.00602240030087458\\
47.08	0.00602243353011364\\
47.09	0.00602246677656196\\
47.1	0.00602250004022955\\
47.11	0.00602253332112635\\
47.12	0.00602256661926225\\
47.13	0.00602259993464706\\
47.14	0.00602263326729055\\
47.15	0.00602266661720238\\
47.16	0.00602269998439219\\
47.17	0.0060227333688695\\
47.18	0.0060227667706438\\
47.19	0.00602280018972449\\
47.2	0.00602283362612089\\
47.21	0.00602286707984224\\
47.22	0.00602290055089773\\
47.23	0.00602293403929645\\
47.24	0.00602296754504742\\
47.25	0.00602300106815955\\
47.26	0.00602303460864172\\
47.27	0.00602306816650267\\
47.28	0.00602310174175109\\
47.29	0.00602313533439557\\
47.3	0.00602316894444462\\
47.31	0.00602320257190664\\
47.32	0.00602323621678995\\
47.33	0.00602326987910278\\
47.34	0.00602330355885326\\
47.35	0.00602333725604942\\
47.36	0.00602337097069919\\
47.37	0.00602340470281041\\
47.38	0.0060234384523908\\
47.39	0.006023472219448\\
47.4	0.00602350600398952\\
47.41	0.00602353980602278\\
47.42	0.00602357362555507\\
47.43	0.0060236074625936\\
47.44	0.00602364131714543\\
47.45	0.00602367518921753\\
47.46	0.00602370907881676\\
47.47	0.00602374298594983\\
47.48	0.00602377691062336\\
47.49	0.00602381085284383\\
47.5	0.0060238448126176\\
47.51	0.00602387878995091\\
47.52	0.00602391278484986\\
47.53	0.00602394679732042\\
47.54	0.00602398082736845\\
47.55	0.00602401487499963\\
47.56	0.00602404894021957\\
47.57	0.00602408302303367\\
47.58	0.00602411712344723\\
47.59	0.00602415124146541\\
47.6	0.00602418537709321\\
47.61	0.00602421953033548\\
47.62	0.00602425370119694\\
47.63	0.00602428788968213\\
47.64	0.00602432209579547\\
47.65	0.0060243563195412\\
47.66	0.0060243905609234\\
47.67	0.00602442481994601\\
47.68	0.0060244590966128\\
47.69	0.00602449339092736\\
47.7	0.00602452770289312\\
47.71	0.00602456203251335\\
47.72	0.00602459637979115\\
47.73	0.00602463074472943\\
47.74	0.00602466512733093\\
47.75	0.00602469952759821\\
47.76	0.00602473394553367\\
47.77	0.00602476838113949\\
47.78	0.00602480283441769\\
47.79	0.0060248373053701\\
47.8	0.00602487179399834\\
47.81	0.00602490630030385\\
47.82	0.00602494082428788\\
47.83	0.00602497536595149\\
47.84	0.0060250099252955\\
47.85	0.00602504450232058\\
47.86	0.00602507909702715\\
47.87	0.00602511370941545\\
47.88	0.0060251483394855\\
47.89	0.0060251829872371\\
47.9	0.00602521765266985\\
47.91	0.00602525233578313\\
47.92	0.00602528703657609\\
47.93	0.00602532175504767\\
47.94	0.00602535649119658\\
47.95	0.00602539124502129\\
47.96	0.00602542601652007\\
47.97	0.00602546080569093\\
47.98	0.00602549561253166\\
47.99	0.00602553043703982\\
48	0.00602556527921271\\
48.01	0.00602560013904741\\
48.02	0.00602563501654074\\
48.03	0.00602566991168929\\
48.04	0.00602570482448939\\
48.05	0.00602573975493712\\
48.06	0.00602577470302832\\
48.07	0.00602580966875856\\
48.08	0.00602584465212317\\
48.09	0.0060258796531172\\
48.1	0.00602591467173545\\
48.11	0.00602594970797247\\
48.12	0.00602598476182252\\
48.13	0.00602601983327961\\
48.14	0.00602605492233748\\
48.15	0.00602609002898959\\
48.16	0.00602612515322914\\
48.17	0.00602616029504904\\
48.18	0.00602619545444194\\
48.19	0.00602623063140021\\
48.2	0.00602626582591592\\
48.21	0.00602630103798089\\
48.22	0.00602633626758665\\
48.23	0.00602637151472442\\
48.24	0.00602640677938517\\
48.25	0.00602644206155956\\
48.26	0.00602647736123797\\
48.27	0.0060265126784105\\
48.28	0.00602654801306696\\
48.29	0.00602658336519686\\
48.3	0.00602661873478941\\
48.31	0.00602665412183356\\
48.32	0.00602668952631794\\
48.33	0.0060267249482309\\
48.34	0.00602676038756049\\
48.35	0.00602679584429448\\
48.36	0.00602683131842034\\
48.37	0.00602686680992524\\
48.38	0.00602690231879605\\
48.39	0.00602693784501939\\
48.4	0.00602697338858153\\
48.41	0.0060270089494685\\
48.42	0.006027044527666\\
48.43	0.00602708012315947\\
48.44	0.00602711573593404\\
48.45	0.00602715136597457\\
48.46	0.00602718701326562\\
48.47	0.00602722267779147\\
48.48	0.00602725835953614\\
48.49	0.00602729405848333\\
48.5	0.00602732977461649\\
48.51	0.00602736550791879\\
48.52	0.00602740125837313\\
48.53	0.00602743702596213\\
48.54	0.00602747281066816\\
48.55	0.00602750861247331\\
48.56	0.00602754443135943\\
48.57	0.0060275802673081\\
48.58	0.00602761612030065\\
48.59	0.00602765199031817\\
48.6	0.00602768787734151\\
48.61	0.00602772378135129\\
48.62	0.00602775970232787\\
48.63	0.00602779564025142\\
48.64	0.00602783159510188\\
48.65	0.00602786756685896\\
48.66	0.00602790355550218\\
48.67	0.00602793956101087\\
48.68	0.00602797558336416\\
48.69	0.00602801162254098\\
48.7	0.0060280476785201\\
48.71	0.00602808375128014\\
48.72	0.00602811984079953\\
48.73	0.00602815594705656\\
48.74	0.00602819207002941\\
48.75	0.00602822820969609\\
48.76	0.00602826436603452\\
48.77	0.00602830053902251\\
48.78	0.00602833672863776\\
48.79	0.0060283729348579\\
48.8	0.00602840915766048\\
48.81	0.00602844539702301\\
48.82	0.00602848165292294\\
48.83	0.00602851792533769\\
48.84	0.00602855421424468\\
48.85	0.00602859051962131\\
48.86	0.00602862684144501\\
48.87	0.00602866317969323\\
48.88	0.00602869953434349\\
48.89	0.00602873590537334\\
48.9	0.00602877229276045\\
48.91	0.00602880869648258\\
48.92	0.00602884511651759\\
48.93	0.00602888155284352\\
48.94	0.00602891800543854\\
48.95	0.00602895447428102\\
48.96	0.00602899095934952\\
48.97	0.00602902746062285\\
48.98	0.00602906397808005\\
48.99	0.00602910051170044\\
49	0.00602913706146364\\
49.01	0.0060291736273496\\
49.02	0.00602921020933861\\
49.03	0.00602924680741133\\
49.04	0.00602928342154885\\
49.05	0.00602932005173266\\
49.06	0.00602935669794474\\
49.07	0.00602939336016754\\
49.08	0.00602943003838403\\
49.09	0.00602946673257775\\
49.1	0.00602950344273282\\
49.11	0.00602954016883395\\
49.12	0.00602957691086654\\
49.13	0.00602961366881665\\
49.14	0.00602965044267106\\
49.15	0.00602968723241731\\
49.16	0.00602972403804374\\
49.17	0.00602976085953951\\
49.18	0.00602979769689464\\
49.19	0.00602983455010009\\
49.2	0.00602987141914772\\
49.21	0.00602990830403041\\
49.22	0.00602994520474207\\
49.23	0.00602998212127766\\
49.24	0.00603001905363325\\
49.25	0.0060300560018061\\
49.26	0.00603009296579464\\
49.27	0.00603012994559857\\
49.28	0.00603016694121885\\
49.29	0.0060302039526578\\
49.3	0.00603024097991913\\
49.31	0.00603027802300797\\
49.32	0.00603031508193093\\
49.33	0.00603035215669615\\
49.34	0.00603038924731335\\
49.35	0.00603042635379388\\
49.36	0.00603046347615076\\
49.37	0.00603050061439875\\
49.38	0.00603053776855436\\
49.39	0.00603057493863597\\
49.4	0.0060306121246638\\
49.41	0.00603064932666003\\
49.42	0.00603068654464881\\
49.43	0.00603072377865632\\
49.44	0.00603076102871085\\
49.45	0.00603079829484279\\
49.46	0.00603083557708476\\
49.47	0.00603087287547158\\
49.48	0.00603091019004037\\
49.49	0.0060309475208306\\
49.5	0.00603098486788412\\
49.51	0.00603102223124521\\
49.52	0.00603105961096063\\
49.53	0.0060310970070797\\
49.54	0.00603113441965426\\
49.55	0.00603117184873882\\
49.56	0.00603120929439053\\
49.57	0.00603124675666922\\
49.58	0.00603128423563749\\
49.59	0.0060313217313607\\
49.6	0.00603135924390704\\
49.61	0.00603139677334751\\
49.62	0.006031434319756\\
49.63	0.00603147188320932\\
49.64	0.00603150946378715\\
49.65	0.00603154706157216\\
49.66	0.00603158467664995\\
49.67	0.0060316223091091\\
49.68	0.00603165995904116\\
49.69	0.00603169762654066\\
49.7	0.0060317353117051\\
49.71	0.00603177301463497\\
49.72	0.0060318107354337\\
49.73	0.00603184847420766\\
49.74	0.00603188623106612\\
49.75	0.00603192400612124\\
49.76	0.006031961799488\\
49.77	0.00603199961128419\\
49.78	0.0060320374416303\\
49.79	0.0060320752906495\\
49.8	0.00603211315846753\\
49.81	0.00603215104521264\\
49.82	0.00603218895101548\\
49.83	0.00603222687600899\\
49.84	0.00603226482032826\\
49.85	0.00603230278411044\\
49.86	0.00603234076749455\\
49.87	0.00603237877062133\\
49.88	0.00603241679363307\\
49.89	0.00603245483667341\\
49.9	0.00603249289988711\\
49.91	0.00603253098341984\\
49.92	0.00603256908741793\\
49.93	0.00603260721202809\\
49.94	0.00603264535739712\\
49.95	0.00603268352367159\\
49.96	0.00603272171099752\\
49.97	0.00603275991951999\\
49.98	0.00603279814938277\\
49.99	0.00603283640072791\\
50	0.00603287467369525\\
50.01	0.006032912968422\\
50.02	0.00603295128504218\\
50.03	0.00603298962368609\\
50.04	0.00603302798447976\\
50.05	0.00603306636754427\\
50.06	0.00603310477299516\\
50.07	0.00603314320094168\\
50.08	0.00603318165148607\\
50.09	0.00603322012472276\\
50.1	0.00603325862073754\\
50.11	0.00603329713960666\\
50.12	0.00603333568139592\\
50.13	0.00603337424615964\\
50.14	0.00603341283393959\\
50.15	0.00603345144476395\\
50.16	0.00603349007865749\\
50.17	0.00603352873564504\\
50.18	0.00603356741575143\\
50.19	0.00603360611900152\\
50.2	0.00603364484542017\\
50.21	0.00603368359503231\\
50.22	0.00603372236786284\\
50.23	0.00603376116393671\\
50.24	0.00603379998327886\\
50.25	0.00603383882591427\\
50.26	0.00603387769186792\\
50.27	0.00603391658116482\\
50.28	0.00603395549382999\\
50.29	0.00603399442988845\\
50.3	0.00603403338936526\\
50.31	0.00603407237228547\\
50.32	0.00603411137867417\\
50.33	0.00603415040855643\\
50.34	0.00603418946195736\\
50.35	0.00603422853890208\\
50.36	0.00603426763941572\\
50.37	0.00603430676352342\\
50.38	0.00603434591125035\\
50.39	0.0060343850826217\\
50.4	0.00603442427766267\\
50.41	0.00603446349639849\\
50.42	0.0060345027388544\\
50.43	0.00603454200505567\\
50.44	0.00603458129502763\\
50.45	0.00603462060879559\\
50.46	0.00603465994638494\\
50.47	0.00603469930782109\\
50.48	0.0060347386931295\\
50.49	0.00603477810233569\\
50.5	0.0060348175354652\\
50.51	0.00603485699254365\\
50.52	0.00603489647359669\\
50.53	0.00603493597865004\\
50.54	0.00603497550772944\\
50.55	0.00603501506086071\\
50.56	0.0060350546380697\\
50.57	0.00603509423938233\\
50.58	0.00603513386482455\\
50.59	0.00603517351442238\\
50.6	0.00603521318820187\\
50.61	0.00603525288618914\\
50.62	0.00603529260841037\\
50.63	0.00603533235489177\\
50.64	0.00603537212565961\\
50.65	0.00603541192074022\\
50.66	0.00603545174015998\\
50.67	0.00603549158394531\\
50.68	0.00603553145212271\\
50.69	0.00603557134471871\\
50.7	0.00603561126175991\\
50.71	0.00603565120327295\\
50.72	0.00603569116928455\\
50.73	0.00603573115982145\\
50.74	0.00603577117491048\\
50.75	0.0060358112145785\\
50.76	0.00603585127885243\\
50.77	0.00603589136775927\\
50.78	0.00603593148132605\\
50.79	0.00603597161957987\\
50.8	0.00603601178254789\\
50.81	0.0060360519702573\\
50.82	0.00603609218273539\\
50.83	0.00603613242000948\\
50.84	0.00603617268210696\\
50.85	0.00603621296905528\\
50.86	0.00603625328088193\\
50.87	0.00603629361761449\\
50.88	0.00603633397928058\\
50.89	0.00603637436590789\\
50.9	0.00603641477752416\\
50.91	0.00603645521415721\\
50.92	0.00603649567583489\\
50.93	0.00603653616258516\\
50.94	0.00603657667443598\\
50.95	0.00603661721141544\\
50.96	0.00603665777355164\\
50.97	0.00603669836087277\\
50.98	0.00603673897340708\\
50.99	0.00603677961118287\\
51	0.00603682027422854\\
51.01	0.00603686096257251\\
51.02	0.00603690167624331\\
51.03	0.0060369424152695\\
51.04	0.00603698317967972\\
51.05	0.00603702396950268\\
51.06	0.00603706478476716\\
51.07	0.006037105625502\\
51.08	0.00603714649173612\\
51.09	0.00603718738349849\\
51.1	0.00603722830081817\\
51.11	0.00603726924372427\\
51.12	0.00603731021224599\\
51.13	0.00603735120641259\\
51.14	0.0060373922262534\\
51.15	0.00603743327179783\\
51.16	0.00603747434307535\\
51.17	0.00603751544011552\\
51.18	0.00603755656294795\\
51.19	0.00603759771160236\\
51.2	0.00603763888610849\\
51.21	0.00603768008649622\\
51.22	0.00603772131279546\\
51.23	0.0060377625650362\\
51.24	0.00603780384324854\\
51.25	0.00603784514746262\\
51.26	0.00603788647770867\\
51.27	0.00603792783401701\\
51.28	0.00603796921641802\\
51.29	0.00603801062494218\\
51.3	0.00603805205962003\\
51.31	0.00603809352048222\\
51.32	0.00603813500755945\\
51.33	0.00603817652088252\\
51.34	0.00603821806048231\\
51.35	0.00603825962638979\\
51.36	0.006038301218636\\
51.37	0.00603834283725207\\
51.38	0.00603838448226924\\
51.39	0.00603842615371879\\
51.4	0.00603846785163214\\
51.41	0.00603850957604076\\
51.42	0.00603855132697621\\
51.43	0.00603859310447018\\
51.44	0.0060386349085544\\
51.45	0.00603867673926073\\
51.46	0.00603871859662109\\
51.47	0.0060387604806675\\
51.48	0.00603880239143211\\
51.49	0.00603884432894712\\
51.5	0.00603888629324486\\
51.51	0.00603892828435771\\
51.52	0.00603897030231819\\
51.53	0.00603901234715892\\
51.54	0.00603905441891258\\
51.55	0.00603909651761199\\
51.56	0.00603913864329004\\
51.57	0.00603918079597975\\
51.58	0.00603922297571422\\
51.59	0.00603926518252667\\
51.6	0.00603930741645042\\
51.61	0.00603934967751888\\
51.62	0.0060393919657656\\
51.63	0.0060394342812242\\
51.64	0.00603947662392844\\
51.65	0.00603951899391218\\
51.66	0.00603956139120937\\
51.67	0.00603960381585411\\
51.68	0.00603964626788059\\
51.69	0.00603968874732312\\
51.7	0.00603973125421612\\
51.71	0.00603977378859413\\
51.72	0.00603981635049181\\
51.73	0.00603985893994395\\
51.74	0.00603990155698543\\
51.75	0.00603994420165128\\
51.76	0.00603998687397666\\
51.77	0.00604002957399681\\
51.78	0.00604007230174714\\
51.79	0.00604011505726317\\
51.8	0.00604015784058055\\
51.81	0.00604020065173505\\
51.82	0.00604024349076258\\
51.83	0.0060402863576992\\
51.84	0.00604032925258107\\
51.85	0.0060403721754445\\
51.86	0.00604041512632595\\
51.87	0.006040458105262\\
51.88	0.00604050111228938\\
51.89	0.00604054414744496\\
51.9	0.00604058721076574\\
51.91	0.00604063030228889\\
51.92	0.00604067342205171\\
51.93	0.00604071657009164\\
51.94	0.00604075974644629\\
51.95	0.00604080295115341\\
51.96	0.00604084618425091\\
51.97	0.00604088944577683\\
51.98	0.00604093273576942\\
51.99	0.00604097605426702\\
52	0.00604101940130818\\
52.01	0.00604106277693161\\
52.02	0.00604110618117615\\
52.03	0.00604114961408084\\
52.04	0.00604119307568487\\
52.05	0.00604123656602762\\
52.06	0.00604128008514862\\
52.07	0.00604132363308757\\
52.08	0.00604136720988439\\
52.09	0.00604141081557912\\
52.1	0.00604145445021202\\
52.11	0.00604149811382351\\
52.12	0.00604154180645424\\
52.13	0.00604158552814497\\
52.14	0.00604162927893673\\
52.15	0.00604167305887069\\
52.16	0.00604171686798824\\
52.17	0.00604176070633095\\
52.18	0.0060418045739406\\
52.19	0.00604184847085917\\
52.2	0.00604189239712885\\
52.21	0.00604193635279203\\
52.22	0.00604198033789131\\
52.23	0.0060420243524695\\
52.24	0.00604206839656965\\
52.25	0.00604211247023499\\
52.26	0.006042156573509\\
52.27	0.00604220070643537\\
52.28	0.00604224486905802\\
52.29	0.00604228906142109\\
52.3	0.00604233328356899\\
52.31	0.00604237753554632\\
52.32	0.00604242181739794\\
52.33	0.00604246612916896\\
52.34	0.00604251047090472\\
52.35	0.00604255484265082\\
52.36	0.0060425992444531\\
52.37	0.00604264367635768\\
52.38	0.00604268813841091\\
52.39	0.00604273263065941\\
52.4	0.00604277715315009\\
52.41	0.0060428217059301\\
52.42	0.00604286628904687\\
52.43	0.00604291090254814\\
52.44	0.00604295554648187\\
52.45	0.00604300022089636\\
52.46	0.00604304492584017\\
52.47	0.00604308966136217\\
52.48	0.00604313442751151\\
52.49	0.00604317922433766\\
52.5	0.00604322405189037\\
52.51	0.00604326891021974\\
52.52	0.00604331379937614\\
52.53	0.00604335871941028\\
52.54	0.0060434036703732\\
52.55	0.00604344865231627\\
52.56	0.00604349366529117\\
52.57	0.00604353870934992\\
52.58	0.00604358378454492\\
52.59	0.00604362889092888\\
52.6	0.00604367402855486\\
52.61	0.00604371919747631\\
52.62	0.00604376439774701\\
52.63	0.00604380962942113\\
52.64	0.00604385489255319\\
52.65	0.00604390018719811\\
52.66	0.0060439455134112\\
52.67	0.00604399087124812\\
52.68	0.00604403626076498\\
52.69	0.00604408168201825\\
52.7	0.00604412713506483\\
52.71	0.00604417261996202\\
52.72	0.00604421813676754\\
52.73	0.00604426368553956\\
52.74	0.00604430926633664\\
52.75	0.00604435487921782\\
52.76	0.00604440052424257\\
52.77	0.0060444462014708\\
52.78	0.0060444919109629\\
52.79	0.0060445376527797\\
52.8	0.00604458342698255\\
52.81	0.00604462923363321\\
52.82	0.006044675072794\\
52.83	0.00604472094452768\\
52.84	0.00604476684889755\\
52.85	0.0060448127859674\\
52.86	0.00604485875580154\\
52.87	0.00604490475846482\\
52.88	0.0060449507940226\\
52.89	0.0060449968625408\\
52.9	0.0060450429640859\\
52.91	0.00604508909872491\\
52.92	0.00604513526652543\\
52.93	0.00604518146755564\\
52.94	0.00604522770188429\\
52.95	0.00604527396958073\\
52.96	0.00604532027071493\\
52.97	0.00604536660535746\\
52.98	0.00604541297357951\\
52.99	0.0060454593754529\\
53	0.00604550581105012\\
53.01	0.00604555228044429\\
53.02	0.0060455987837092\\
53.03	0.00604564532091931\\
53.04	0.00604569189214977\\
53.05	0.00604573849747642\\
53.06	0.00604578513697583\\
53.07	0.00604583181072526\\
53.08	0.0060458785188027\\
53.09	0.00604592526128692\\
53.1	0.00604597203825738\\
53.11	0.00604601884979437\\
53.12	0.00604606569597892\\
53.13	0.00604611257689285\\
53.14	0.0060461594926188\\
53.15	0.00604620644324021\\
53.16	0.00604625342884137\\
53.17	0.00604630044950739\\
53.18	0.00604634750532425\\
53.19	0.00604639459637878\\
53.2	0.00604644172275873\\
53.21	0.00604648888455272\\
53.22	0.00604653608185029\\
53.23	0.0060465833147419\\
53.24	0.00604663058331897\\
53.25	0.00604667788767387\\
53.26	0.00604672522789995\\
53.27	0.00604677260409154\\
53.28	0.00604682001634398\\
53.29	0.00604686746475362\\
53.3	0.00604691494941789\\
53.31	0.00604696247043524\\
53.32	0.00604701002790519\\
53.33	0.00604705762192838\\
53.34	0.00604710525260655\\
53.35	0.00604715292004255\\
53.36	0.0060472006243404\\
53.37	0.00604724836560528\\
53.38	0.00604729614394353\\
53.39	0.00604734395946275\\
53.4	0.00604739181227171\\
53.41	0.00604743970248045\\
53.42	0.00604748763020026\\
53.43	0.00604753559554375\\
53.44	0.0060475835986248\\
53.45	0.00604763163955865\\
53.46	0.00604767971846187\\
53.47	0.00604772783545242\\
53.48	0.00604777599064964\\
53.49	0.00604782418417432\\
53.5	0.00604787241614867\\
53.51	0.00604792068669637\\
53.52	0.0060479689959426\\
53.53	0.00604801734401408\\
53.54	0.00604806573103902\\
53.55	0.00604811415714726\\
53.56	0.0060481626224702\\
53.57	0.00604821112714088\\
53.58	0.00604825967129398\\
53.59	0.00604830825506586\\
53.6	0.00604835687859459\\
53.61	0.00604840554201997\\
53.62	0.00604845424548358\\
53.63	0.00604850298912878\\
53.64	0.00604855177310076\\
53.65	0.00604860059754656\\
53.66	0.00604864946261512\\
53.67	0.00604869836845729\\
53.68	0.00604874731522586\\
53.69	0.00604879630307563\\
53.7	0.0060488453321634\\
53.71	0.00604889440264803\\
53.72	0.00604894351469046\\
53.73	0.00604899266845376\\
53.74	0.00604904186410315\\
53.75	0.00604909110180605\\
53.76	0.00604914038173212\\
53.77	0.00604918970405328\\
53.78	0.00604923906894376\\
53.79	0.00604928847658013\\
53.8	0.00604933792714138\\
53.81	0.00604938742080888\\
53.82	0.00604943695776651\\
53.83	0.00604948653820065\\
53.84	0.00604953616230022\\
53.85	0.00604958583025675\\
53.86	0.00604963554226442\\
53.87	0.00604968529852009\\
53.88	0.00604973509922334\\
53.89	0.00604978494457655\\
53.9	0.00604983483478491\\
53.91	0.0060498847700565\\
53.92	0.00604993475060231\\
53.93	0.00604998477663632\\
53.94	0.00605003484837552\\
53.95	0.00605008496603997\\
53.96	0.00605013512985289\\
53.97	0.00605018534004063\\
53.98	0.00605023559683282\\
53.99	0.00605028590046237\\
54	0.00605033625116553\\
54.01	0.00605038664918195\\
54.02	0.00605043709475476\\
54.03	0.00605048758813058\\
54.04	0.00605053812955965\\
54.05	0.00605058871929582\\
54.06	0.00605063935759665\\
54.07	0.00605069004472348\\
54.08	0.00605074078094147\\
54.09	0.00605079156651968\\
54.1	0.00605084240173112\\
54.11	0.00605089328685285\\
54.12	0.00605094422216601\\
54.13	0.00605099520795593\\
54.14	0.00605104624451215\\
54.15	0.00605109733212853\\
54.16	0.00605114847110333\\
54.17	0.00605119966173925\\
54.18	0.00605125090434353\\
54.19	0.00605130219922802\\
54.2	0.00605135354670926\\
54.21	0.00605140494710856\\
54.22	0.00605145640075208\\
54.23	0.0060515079079709\\
54.24	0.00605155946910112\\
54.25	0.00605161108448396\\
54.26	0.00605166275446579\\
54.27	0.00605171447939827\\
54.28	0.00605176625963844\\
54.29	0.00605181809554877\\
54.3	0.00605186998749728\\
54.31	0.00605192193585763\\
54.32	0.00605197394100921\\
54.33	0.00605202600333725\\
54.34	0.0060520781232329\\
54.35	0.00605213030109334\\
54.36	0.00605218253732188\\
54.37	0.00605223483232807\\
54.38	0.00605228718652777\\
54.39	0.00605233960034331\\
54.4	0.00605239207420357\\
54.41	0.00605244460854406\\
54.42	0.0060524972038071\\
54.43	0.00605254986044189\\
54.44	0.0060526025789046\\
54.45	0.00605265535965853\\
54.46	0.00605270820317425\\
54.47	0.00605276110992963\\
54.48	0.00605281408041006\\
54.49	0.00605286711510851\\
54.5	0.0060529202145257\\
54.51	0.00605297337917017\\
54.52	0.0060530266095585\\
54.53	0.00605307990621534\\
54.54	0.00605313326967362\\
54.55	0.00605318670047464\\
54.56	0.00605324019916826\\
54.57	0.00605329376631298\\
54.58	0.00605334740247612\\
54.59	0.00605340110823396\\
54.6	0.00605345488417187\\
54.61	0.00605350873088448\\
54.62	0.00605356264897584\\
54.63	0.00605361663905952\\
54.64	0.00605367070175883\\
54.65	0.00605372483770696\\
54.66	0.0060537790475471\\
54.67	0.00605383333193264\\
54.68	0.00605388769152738\\
54.69	0.00605394212700557\\
54.7	0.0060539966390522\\
54.71	0.00605405122836313\\
54.72	0.00605410589564521\\
54.73	0.00605416064161652\\
54.74	0.00605421546700651\\
54.75	0.00605427037255618\\
54.76	0.00605432535901825\\
54.77	0.00605438042715737\\
54.78	0.00605443557775027\\
54.79	0.00605449081158596\\
54.8	0.00605454612946591\\
54.81	0.00605460153220425\\
54.82	0.00605465702062793\\
54.83	0.00605471259557698\\
54.84	0.0060547682579046\\
54.85	0.00605482400847747\\
54.86	0.00605487984817584\\
54.87	0.00605493577789386\\
54.88	0.00605499179853962\\
54.89	0.00605504791103551\\
54.9	0.00605510411631833\\
54.91	0.00605516041533951\\
54.92	0.00605521680906538\\
54.93	0.00605527329847729\\
54.94	0.00605532988457189\\
54.95	0.00605538656836133\\
54.96	0.00605544335087348\\
54.97	0.00605550023315211\\
54.98	0.00605555721625717\\
54.99	0.00605561430126495\\
55	0.00605567148926836\\
55.01	0.00605572878137712\\
55.02	0.00605578617871798\\
55.03	0.00605584368243497\\
55.04	0.00605590129368961\\
55.05	0.00605595901366115\\
55.06	0.00605601684354678\\
55.07	0.00605607478456188\\
55.08	0.00605613283794024\\
55.09	0.0060561910049343\\
55.1	0.00605624928681537\\
55.11	0.00605630768487387\\
55.12	0.00605636620041956\\
55.13	0.00605642483478177\\
55.14	0.00605648358930962\\
55.15	0.00605654246537231\\
55.16	0.00605660146435926\\
55.17	0.0060566605876804\\
55.18	0.00605671983676641\\
55.19	0.00605677921306891\\
55.2	0.0060568387180607\\
55.21	0.00605689835323601\\
55.22	0.0060569581201107\\
55.23	0.00605701802022249\\
55.24	0.00605707805513118\\
55.25	0.00605713822641887\\
55.26	0.00605719853569019\\
55.27	0.00605725898457249\\
55.28	0.00605731957471608\\
55.29	0.00605738030779439\\
55.3	0.00605744118550424\\
55.31	0.00605750220956597\\
55.32	0.00605756338172369\\
55.33	0.00605762470374543\\
55.34	0.00605768617742337\\
55.35	0.00605774780457397\\
55.36	0.00605780958703818\\
55.37	0.00605787152668158\\
55.38	0.00605793362539458\\
55.39	0.00605799588509253\\
55.4	0.00605805830771589\\
55.41	0.00605812089523037\\
55.42	0.00605818364962702\\
55.43	0.00605824657292243\\
55.44	0.00605830966715875\\
55.45	0.00605837293440385\\
55.46	0.00605843637675137\\
55.47	0.00605849999632084\\
55.48	0.00605856379525768\\
55.49	0.00605862777573333\\
55.5	0.00605869193994521\\
55.51	0.00605875629011679\\
55.52	0.00605882082849759\\
55.53	0.00605888555736316\\
55.54	0.00605895047901506\\
55.55	0.00605901559578081\\
55.56	0.00605908091001382\\
55.57	0.00605914642409335\\
55.58	0.00605921214042434\\
55.59	0.00605927806143735\\
55.6	0.00605934418958834\\
55.61	0.00605941052735859\\
55.62	0.0060594770772544\\
55.63	0.00605954384180694\\
55.64	0.00605961082357196\\
55.65	0.00605967802512956\\
55.66	0.00605974544908378\\
55.67	0.00605981309806237\\
55.68	0.00605988097471634\\
55.69	0.00605994908171957\\
55.7	0.00606001742176836\\
55.71	0.00606008599758097\\
55.72	0.00606015481189704\\
55.73	0.00606022386747709\\
55.74	0.00606029316710187\\
55.75	0.00606036271357169\\
55.76	0.00606043250970576\\
55.77	0.00606050255834144\\
55.78	0.00606057286233338\\
55.79	0.00606064342455274\\
55.8	0.0060607142478862\\
55.81	0.00606078533523509\\
55.82	0.00606085668951426\\
55.83	0.00606092831365104\\
55.84	0.00606100021058409\\
55.85	0.00606107238326213\\
55.86	0.00606114483464268\\
55.87	0.00606121756769069\\
55.88	0.00606129058537705\\
55.89	0.0060613638906771\\
55.9	0.00606143748656901\\
55.91	0.00606151137603205\\
55.92	0.00606158556204484\\
55.93	0.00606166004758344\\
55.94	0.00606173483561938\\
55.95	0.00606180992911754\\
55.96	0.00606188533103401\\
55.97	0.00606196104431377\\
55.98	0.00606203707188825\\
55.99	0.00606211341667283\\
56	0.00606219008156417\\
56.01	0.0060622670694374\\
56.02	0.00606234438314325\\
56.03	0.00606242202550493\\
56.04	0.006062499999315\\
56.05	0.00606257830733195\\
56.06	0.00606265695227673\\
56.07	0.00606273593682907\\
56.08	0.00606281526362368\\
56.09	0.00606289493524616\\
56.1	0.0060629749542289\\
56.11	0.00606305532304665\\
56.12	0.00606313604411197\\
56.13	0.00606321711977045\\
56.14	0.00606329855229573\\
56.15	0.00606338034388431\\
56.16	0.00606346249665011\\
56.17	0.00606354501261883\\
56.18	0.00606362789372205\\
56.19	0.00606371114179108\\
56.2	0.00606379475855055\\
56.21	0.0060638787456117\\
56.22	0.00606396310446548\\
56.23	0.00606404783647526\\
56.24	0.00606413294286928\\
56.25	0.0060642184247328\\
56.26	0.00606430428299991\\
56.27	0.0060643905207675\\
56.28	0.0060644771455524\\
56.29	0.00606456416505369\\
56.3	0.00606465158715751\\
56.31	0.00606473941994189\\
56.32	0.0060648276716818\\
56.33	0.00606491635085426\\
56.34	0.00606500546614368\\
56.35	0.00606509502644725\\
56.36	0.00606518504088055\\
56.37	0.00606527551878329\\
56.38	0.0060653664697252\\
56.39	0.00606545790351209\\
56.4	0.00606554983019208\\
56.41	0.006065642260062\\
56.42	0.00606573520367395\\
56.43	0.00606582867184208\\
56.44	0.00606592267564952\\
56.45	0.0060660172264555\\
56.46	0.00606611233590273\\
56.47	0.00606620801592488\\
56.48	0.00606630427875437\\
56.49	0.00606640113693028\\
56.5	0.00606649860330659\\
56.51	0.00606659669106056\\
56.52	0.00606669541370137\\
56.53	0.006066794785079\\
56.54	0.00606689481939336\\
56.55	0.00606699553120371\\
56.56	0.00606709693543825\\
56.57	0.0060671990474041\\
56.58	0.00606730188279741\\
56.59	0.00606740545771394\\
56.6	0.00606750978865974\\
56.61	0.00606761489256229\\
56.62	0.00606772078678183\\
56.63	0.00606782748912312\\
56.64	0.0060679350178474\\
56.65	0.00606804339168483\\
56.66	0.00606815262984716\\
56.67	0.0060682627520408\\
56.68	0.00606837377848031\\
56.69	0.00606848572990216\\
56.7	0.00606859862757898\\
56.71	0.00606871249333414\\
56.72	0.00606882734955679\\
56.73	0.00606894321921731\\
56.74	0.00606906012588315\\
56.75	0.0060691780937352\\
56.76	0.00606929714758459\\
56.77	0.00606941731288989\\
56.78	0.00606953861577494\\
56.79	0.00606966108304707\\
56.8	0.00606978474221584\\
56.81	0.00606990962151243\\
56.82	0.00607003574990941\\
56.83	0.00607016315714121\\
56.84	0.00607029187372508\\
56.85	0.00607042193098271\\
56.86	0.00607055336106242\\
56.87	0.00607068619696198\\
56.88	0.00607082047255217\\
56.89	0.00607095622260084\\
56.9	0.00607109348279781\\
56.91	0.00607123228978041\\
56.92	0.00607137268115975\\
56.93	0.00607151469554775\\
56.94	0.0060716569932977\\
56.95	0.00607179935664574\\
56.96	0.00607194178563709\\
56.97	0.00607208428031703\\
56.98	0.00607222684073088\\
56.99	0.00607236946692398\\
57	0.00607251215894176\\
57.01	0.00607265491682967\\
57.02	0.00607279774063322\\
57.03	0.00607294063039794\\
57.04	0.00607308358616944\\
57.05	0.00607322660799336\\
57.06	0.00607336969591538\\
57.07	0.00607351284998126\\
57.08	0.00607365607023677\\
57.09	0.00607379935672774\\
57.1	0.00607394270950005\\
57.11	0.00607408612859963\\
57.12	0.00607422961407245\\
57.13	0.00607437316596454\\
57.14	0.00607451678432195\\
57.15	0.00607466046919082\\
57.16	0.00607480422061729\\
57.17	0.00607494803864759\\
57.18	0.00607509192332796\\
57.19	0.00607523587470473\\
57.2	0.00607537989282424\\
57.21	0.00607552397773289\\
57.22	0.00607566812947715\\
57.23	0.0060758123481035\\
57.24	0.00607595663365849\\
57.25	0.00607610098618872\\
57.26	0.00607624540574083\\
57.27	0.00607638989236152\\
57.28	0.00607653444609752\\
57.29	0.00607667906699562\\
57.3	0.00607682375510267\\
57.31	0.00607696851046554\\
57.32	0.00607711333313117\\
57.33	0.00607725822314655\\
57.34	0.0060774031805587\\
57.35	0.00607754820541472\\
57.36	0.00607769329776171\\
57.37	0.00607783845764688\\
57.38	0.00607798368511743\\
57.39	0.00607812898022066\\
57.4	0.00607827434300389\\
57.41	0.00607841977351449\\
57.42	0.00607856527179988\\
57.43	0.00607871083790755\\
57.44	0.00607885647188501\\
57.45	0.00607900217377984\\
57.46	0.00607914794363966\\
57.47	0.00607929378151214\\
57.48	0.00607943968744501\\
57.49	0.00607958566148604\\
57.5	0.00607973170368305\\
57.51	0.00607987781408391\\
57.52	0.00608002399273653\\
57.53	0.00608017023968891\\
57.54	0.00608031655498904\\
57.55	0.00608046293868502\\
57.56	0.00608060939082496\\
57.57	0.00608075591145703\\
57.58	0.00608090250062946\\
57.59	0.00608104915839051\\
57.6	0.00608119588478852\\
57.61	0.00608134267987186\\
57.62	0.00608148954368895\\
57.63	0.00608163647628827\\
57.64	0.00608178347771834\\
57.65	0.00608193054802776\\
57.66	0.00608207768726513\\
57.67	0.00608222489547915\\
57.68	0.00608237217271853\\
57.69	0.00608251951903208\\
57.7	0.00608266693446862\\
57.71	0.00608281441907703\\
57.72	0.00608296197290624\\
57.73	0.00608310959600525\\
57.74	0.0060832572884231\\
57.75	0.00608340505020888\\
57.76	0.00608355288141172\\
57.77	0.00608370078208082\\
57.78	0.00608384875226543\\
57.79	0.00608399679201485\\
57.8	0.00608414490137842\\
57.81	0.00608429308040555\\
57.82	0.00608444132914569\\
57.83	0.00608458964764834\\
57.84	0.00608473803596307\\
57.85	0.00608488649413947\\
57.86	0.00608503502222723\\
57.87	0.00608518362027604\\
57.88	0.00608533228833568\\
57.89	0.00608548102645597\\
57.9	0.00608562983468678\\
57.91	0.00608577871307804\\
57.92	0.00608592766167972\\
57.93	0.00608607668054186\\
57.94	0.00608622576971454\\
57.95	0.00608637492924789\\
57.96	0.00608652415919212\\
57.97	0.00608667345959745\\
57.98	0.00608682283051419\\
57.99	0.00608697227199269\\
58	0.00608712178408335\\
58.01	0.00608727136683662\\
58.02	0.00608742102030303\\
58.03	0.00608757074453312\\
58.04	0.00608772053957752\\
58.05	0.0060878704054869\\
58.06	0.00608802034231199\\
58.07	0.00608817035010356\\
58.08	0.00608832042891244\\
58.09	0.00608847057878953\\
58.1	0.00608862079978577\\
58.11	0.00608877109195215\\
58.12	0.00608892145533972\\
58.13	0.00608907188999958\\
58.14	0.0060892223959829\\
58.15	0.00608937297334088\\
58.16	0.0060895236221248\\
58.17	0.00608967434238597\\
58.18	0.00608982513417577\\
58.19	0.00608997599754564\\
58.2	0.00609012693254706\\
58.21	0.00609027793923157\\
58.22	0.00609042901765076\\
58.23	0.0060905801678563\\
58.24	0.00609073138989989\\
58.25	0.00609088268383328\\
58.26	0.0060910340497083\\
58.27	0.00609118548757682\\
58.28	0.00609133699749076\\
58.29	0.00609148857950212\\
58.3	0.00609164023366294\\
58.31	0.00609179196002529\\
58.32	0.00609194375864134\\
58.33	0.0060920956295633\\
58.34	0.00609224757284343\\
58.35	0.00609239958853404\\
58.36	0.00609255167668751\\
58.37	0.00609270383735628\\
58.38	0.00609285607059282\\
58.39	0.00609300837644969\\
58.4	0.00609316075497948\\
58.41	0.00609331320623485\\
58.42	0.00609346573026852\\
58.43	0.00609361832713325\\
58.44	0.00609377099688188\\
58.45	0.00609392373956727\\
58.46	0.00609407655524238\\
58.47	0.00609422944396021\\
58.48	0.0060943824057738\\
58.49	0.00609453544073627\\
58.5	0.00609468854890078\\
58.51	0.00609484173032057\\
58.52	0.00609499498504892\\
58.53	0.00609514831313916\\
58.54	0.0060953017146447\\
58.55	0.00609545518961899\\
58.56	0.00609560873811555\\
58.57	0.00609576236018795\\
58.58	0.00609591605588982\\
58.59	0.00609606982527484\\
58.6	0.00609622366839676\\
58.61	0.00609637758530938\\
58.62	0.00609653157606657\\
58.63	0.00609668564072225\\
58.64	0.00609683977933039\\
58.65	0.00609699399194503\\
58.66	0.00609714827862027\\
58.67	0.00609730263941026\\
58.68	0.00609745707436921\\
58.69	0.00609761158355139\\
58.7	0.00609776616701114\\
58.71	0.00609792082480284\\
58.72	0.00609807555698095\\
58.73	0.00609823036359996\\
58.74	0.00609838524471445\\
58.75	0.00609854020037904\\
58.76	0.00609869523064842\\
58.77	0.00609885033557732\\
58.78	0.00609900551522056\\
58.79	0.006099160769633\\
58.8	0.00609931609886956\\
58.81	0.00609947150298523\\
58.82	0.00609962698203505\\
58.83	0.00609978253607411\\
58.84	0.00609993816515758\\
58.85	0.00610009386934069\\
58.86	0.00610024964867872\\
58.87	0.00610040550322702\\
58.88	0.00610056143304098\\
58.89	0.00610071743817607\\
58.9	0.00610087351868781\\
58.91	0.0061010296746318\\
58.92	0.00610118590606367\\
58.93	0.00610134221303913\\
58.94	0.00610149859561396\\
58.95	0.00610165505384398\\
58.96	0.00610181158778507\\
58.97	0.0061019681974932\\
58.98	0.00610212488302437\\
58.99	0.00610228164443465\\
59	0.00610243848178019\\
59.01	0.00610259539511717\\
59.02	0.00610275238450186\\
59.03	0.00610290944999057\\
59.04	0.00610306659163968\\
59.05	0.00610322380950565\\
59.06	0.00610338110364497\\
59.07	0.00610353847411421\\
59.08	0.00610369592097001\\
59.09	0.00610385344426904\\
59.1	0.00610401104406807\\
59.11	0.0061041687204239\\
59.12	0.00610432647339344\\
59.13	0.0061044843030336\\
59.14	0.00610464220940139\\
59.15	0.00610480019255389\\
59.16	0.00610495825254821\\
59.17	0.00610511638944156\\
59.18	0.00610527460329118\\
59.19	0.0061054328941544\\
59.2	0.00610559126208859\\
59.21	0.00610574970715121\\
59.22	0.00610590822939976\\
59.23	0.0061060668288918\\
59.24	0.00610622550568499\\
59.25	0.006106384259837\\
59.26	0.00610654309140562\\
59.27	0.00610670200044865\\
59.28	0.006106860987024\\
59.29	0.00610702005118962\\
59.3	0.00610717919300352\\
59.31	0.00610733841252379\\
59.32	0.00610749770980857\\
59.33	0.00610765708491608\\
59.34	0.00610781653790459\\
59.35	0.00610797606883244\\
59.36	0.00610813567775803\\
59.37	0.00610829536473984\\
59.38	0.0061084551298364\\
59.39	0.0061086149731063\\
59.4	0.00610877489460822\\
59.41	0.00610893489440087\\
59.42	0.00610909497254307\\
59.43	0.00610925512909366\\
59.44	0.00610941536411157\\
59.45	0.0061095756776558\\
59.46	0.00610973606978538\\
59.47	0.00610989654055946\\
59.48	0.00611005709003722\\
59.49	0.00611021771827792\\
59.5	0.00611037842534086\\
59.51	0.00611053921128544\\
59.52	0.0061107000761711\\
59.53	0.00611086102005738\\
59.54	0.00611102204300384\\
59.55	0.00611118314507015\\
59.56	0.00611134432631602\\
59.57	0.00611150558680123\\
59.58	0.00611166692658564\\
59.59	0.00611182834572916\\
59.6	0.00611198984429178\\
59.61	0.00611215142233355\\
59.62	0.00611231307991458\\
59.63	0.00611247481709506\\
59.64	0.00611263663393526\\
59.65	0.00611279853049549\\
59.66	0.00611296050683613\\
59.67	0.00611312256301764\\
59.68	0.00611328469910055\\
59.69	0.00611344691514544\\
59.7	0.00611360921121299\\
59.71	0.00611377158736391\\
59.72	0.006113934043659\\
59.73	0.00611409658015912\\
59.74	0.00611425919692521\\
59.75	0.00611442189401828\\
59.76	0.00611458467149937\\
59.77	0.00611474752942964\\
59.78	0.00611491046787029\\
59.79	0.0061150734868826\\
59.8	0.00611523658652791\\
59.81	0.00611539976686763\\
59.82	0.00611556302796325\\
59.83	0.00611572636987631\\
59.84	0.00611588979266844\\
59.85	0.00611605329640134\\
59.86	0.00611621688113675\\
59.87	0.00611638054693651\\
59.88	0.00611654429386252\\
59.89	0.00611670812197675\\
59.9	0.00611687203134124\\
59.91	0.00611703602201809\\
59.92	0.00611720009406949\\
59.93	0.00611736424755769\\
59.94	0.006117528482545\\
59.95	0.00611769279909381\\
59.96	0.00611785719726659\\
59.97	0.00611802167712588\\
59.98	0.00611818623873426\\
59.99	0.00611835088215443\\
60	0.00611851560744911\\
60.01	0.00611868041468113\\
60.02	0.00611884530391337\\
60.03	0.00611901027520879\\
60.04	0.00611917532863042\\
60.05	0.00611934046424137\\
60.06	0.0061195056821048\\
60.07	0.00611967098228396\\
60.08	0.00611983636484216\\
60.09	0.0061200018298428\\
60.1	0.00612016737734933\\
60.11	0.0061203330074253\\
60.12	0.00612049872013429\\
60.13	0.00612066451553999\\
60.14	0.00612083039370616\\
60.15	0.0061209963546966\\
60.16	0.00612116239857523\\
60.17	0.006121328525406\\
60.18	0.00612149473525295\\
60.19	0.00612166102818021\\
60.2	0.00612182740425196\\
60.21	0.00612199386353245\\
60.22	0.00612216040608603\\
60.23	0.00612232703197709\\
60.24	0.00612249374127012\\
60.25	0.00612266053402969\\
60.26	0.0061228274103204\\
60.27	0.00612299437020697\\
60.28	0.00612316141375418\\
60.29	0.00612332854102686\\
60.3	0.00612349575208996\\
60.31	0.00612366304700847\\
60.32	0.00612383042584745\\
60.33	0.00612399788867207\\
60.34	0.00612416543554755\\
60.35	0.00612433306653917\\
60.36	0.00612450078171231\\
60.37	0.00612466858113244\\
60.38	0.00612483646486506\\
60.39	0.00612500443297577\\
60.4	0.00612517248553025\\
60.41	0.00612534062259425\\
60.42	0.0061255088442336\\
60.43	0.00612567715051419\\
60.44	0.006125845541502\\
60.45	0.00612601401726309\\
60.46	0.00612618257786358\\
60.47	0.00612635122336968\\
60.48	0.00612651995384767\\
60.49	0.00612668876936391\\
60.5	0.00612685766998483\\
60.51	0.00612702665577695\\
60.52	0.00612719572680685\\
60.53	0.0061273648831412\\
60.54	0.00612753412484674\\
60.55	0.00612770345199029\\
60.56	0.00612787286463875\\
60.57	0.00612804236285908\\
60.58	0.00612821194671835\\
60.59	0.00612838161628367\\
60.6	0.00612855137162227\\
60.61	0.00612872121280142\\
60.62	0.00612889113988849\\
60.63	0.00612906115295091\\
60.64	0.00612923125205621\\
60.65	0.00612940143727198\\
60.66	0.0061295717086659\\
60.67	0.00612974206630573\\
60.68	0.0061299125102593\\
60.69	0.00613008304059452\\
60.7	0.00613025365737938\\
60.71	0.00613042436068195\\
60.72	0.00613059515057039\\
60.73	0.00613076602711292\\
60.74	0.00613093699037785\\
60.75	0.00613110804043357\\
60.76	0.00613127917734855\\
60.77	0.00613145040119133\\
60.78	0.00613162171203054\\
60.79	0.0061317931099349\\
60.8	0.00613196459497319\\
60.81	0.00613213616721427\\
60.82	0.0061323078267271\\
60.83	0.00613247957358071\\
60.84	0.0061326514078442\\
60.85	0.00613282332958678\\
60.86	0.0061329953388777\\
60.87	0.00613316743578633\\
60.88	0.00613333962038209\\
60.89	0.00613351189273452\\
60.9	0.0061336842529132\\
60.91	0.00613385670098781\\
60.92	0.00613402923702811\\
60.93	0.00613420186110395\\
60.94	0.00613437457328525\\
60.95	0.00613454737364202\\
60.96	0.00613472026224435\\
60.97	0.00613489323916241\\
60.98	0.00613506630446645\\
60.99	0.00613523945822682\\
61	0.00613541270051392\\
61.01	0.00613558603139828\\
61.02	0.00613575945095047\\
61.03	0.00613593295924115\\
61.04	0.00613610655634108\\
61.05	0.00613628024232112\\
61.06	0.00613645401725215\\
61.07	0.0061366278812052\\
61.08	0.00613680183425136\\
61.09	0.00613697587646178\\
61.1	0.00613715000790773\\
61.11	0.00613732422866055\\
61.12	0.00613749853879166\\
61.13	0.00613767293837258\\
61.14	0.00613784742747489\\
61.15	0.00613802200617027\\
61.16	0.00613819667453049\\
61.17	0.0061383714326274\\
61.18	0.00613854628053293\\
61.19	0.00613872121831909\\
61.2	0.00613889624605801\\
61.21	0.00613907136382185\\
61.22	0.00613924657168291\\
61.23	0.00613942186971355\\
61.24	0.00613959725798621\\
61.25	0.00613977273657343\\
61.26	0.00613994830554782\\
61.27	0.0061401239649821\\
61.28	0.00614029971494907\\
61.29	0.0061404755555216\\
61.3	0.00614065148677266\\
61.31	0.0061408275087753\\
61.32	0.00614100362160268\\
61.33	0.00614117982532802\\
61.34	0.00614135612002462\\
61.35	0.00614153250576592\\
61.36	0.00614170898262539\\
61.37	0.00614188555067662\\
61.38	0.00614206220999327\\
61.39	0.0061422389606491\\
61.4	0.00614241580271797\\
61.41	0.00614259273627379\\
61.42	0.00614276976139061\\
61.43	0.00614294687814252\\
61.44	0.00614312408660373\\
61.45	0.00614330138684853\\
61.46	0.00614347877895131\\
61.47	0.00614365626298652\\
61.48	0.00614383383902873\\
61.49	0.00614401150715258\\
61.5	0.00614418926743282\\
61.51	0.00614436711994426\\
61.52	0.00614454506476184\\
61.53	0.00614472310196056\\
61.54	0.00614490123161551\\
61.55	0.0061450794538019\\
61.56	0.00614525776859499\\
61.57	0.00614543617607017\\
61.58	0.00614561467630289\\
61.59	0.00614579326936871\\
61.6	0.00614597195534327\\
61.61	0.00614615073430232\\
61.62	0.00614632960632167\\
61.63	0.00614650857147725\\
61.64	0.00614668762984509\\
61.65	0.00614686678150127\\
61.66	0.006147046026522\\
61.67	0.00614722536498357\\
61.68	0.00614740479696236\\
61.69	0.00614758432253484\\
61.7	0.00614776394177759\\
61.71	0.00614794365476727\\
61.72	0.00614812346158063\\
61.73	0.00614830336229452\\
61.74	0.00614848335698589\\
61.75	0.00614866344573176\\
61.76	0.00614884362860928\\
61.77	0.00614902390569565\\
61.78	0.00614920427706821\\
61.79	0.00614938474280437\\
61.8	0.00614956530298164\\
61.81	0.00614974595767761\\
61.82	0.00614992670696999\\
61.83	0.00615010755093656\\
61.84	0.00615028848965522\\
61.85	0.00615046952320394\\
61.86	0.00615065065166082\\
61.87	0.00615083187510402\\
61.88	0.00615101319361181\\
61.89	0.00615119460726256\\
61.9	0.00615137611613474\\
61.91	0.0061515577203069\\
61.92	0.0061517394198577\\
61.93	0.00615192121486589\\
61.94	0.00615210310541033\\
61.95	0.00615228509156995\\
61.96	0.00615246717342382\\
61.97	0.00615264935105106\\
61.98	0.00615283162453092\\
61.99	0.00615301399394273\\
62	0.00615319645936594\\
62.01	0.00615337902088007\\
62.02	0.00615356167856475\\
62.03	0.00615374443249973\\
62.04	0.00615392728276482\\
62.05	0.00615411022943996\\
62.06	0.00615429327260517\\
62.07	0.00615447641234058\\
62.08	0.00615465964872642\\
62.09	0.00615484298184301\\
62.1	0.00615502641177078\\
62.11	0.00615520993859025\\
62.12	0.00615539356238205\\
62.13	0.0061555772832269\\
62.14	0.00615576110120563\\
62.15	0.00615594501639918\\
62.16	0.00615612902888856\\
62.17	0.0061563131387549\\
62.18	0.00615649734607944\\
62.19	0.00615668165094351\\
62.2	0.00615686605342853\\
62.21	0.00615705055361605\\
62.22	0.0061572351515877\\
62.23	0.00615741984742522\\
62.24	0.00615760464121046\\
62.25	0.00615778953302534\\
62.26	0.00615797452295193\\
62.27	0.00615815961107238\\
62.28	0.00615834479746893\\
62.29	0.00615853008222394\\
62.3	0.00615871546541986\\
62.31	0.00615890094713927\\
62.32	0.00615908652746483\\
62.33	0.00615927220647931\\
62.34	0.00615945798426558\\
62.35	0.00615964386090663\\
62.36	0.00615982983648553\\
62.37	0.00616001591108548\\
62.38	0.00616020208478977\\
62.39	0.00616038835768179\\
62.4	0.00616057472984506\\
62.41	0.00616076120136318\\
62.42	0.00616094777231987\\
62.43	0.00616113444279895\\
62.44	0.00616132121288434\\
62.45	0.00616150808266007\\
62.46	0.0061616950522103\\
62.47	0.00616188212161927\\
62.48	0.00616206929097133\\
62.49	0.00616225656035093\\
62.5	0.00616244392984266\\
62.51	0.00616263139953118\\
62.52	0.00616281896950128\\
62.53	0.00616300663983785\\
62.54	0.00616319441062588\\
62.55	0.00616338228195049\\
62.56	0.0061635702538969\\
62.57	0.00616375832655042\\
62.58	0.0061639464999965\\
62.59	0.00616413477432067\\
62.6	0.00616432314960858\\
62.61	0.006164511625946\\
62.62	0.0061647002034188\\
62.63	0.00616488888211296\\
62.64	0.00616507766211457\\
62.65	0.00616526654350983\\
62.66	0.00616545552638506\\
62.67	0.00616564461082667\\
62.68	0.00616583379692121\\
62.69	0.00616602308475531\\
62.7	0.00616621247441573\\
62.71	0.00616640196598935\\
62.72	0.00616659155956312\\
62.73	0.00616678125522417\\
62.74	0.00616697105305967\\
62.75	0.00616716095315695\\
62.76	0.00616735095560344\\
62.77	0.00616754106048668\\
62.78	0.00616773126789432\\
62.79	0.00616792157791413\\
62.8	0.006168111990634\\
62.81	0.0061683025061419\\
62.82	0.00616849312452596\\
62.83	0.00616868384587439\\
62.84	0.00616887467027554\\
62.85	0.00616906559781784\\
62.86	0.00616925662858987\\
62.87	0.00616944776268031\\
62.88	0.00616963900017795\\
62.89	0.0061698303411717\\
62.9	0.00617002178575059\\
62.91	0.00617021333400376\\
62.92	0.00617040498602046\\
62.93	0.00617059674189008\\
62.94	0.0061707886017021\\
62.95	0.00617098056554612\\
62.96	0.00617117263351187\\
62.97	0.00617136480568919\\
62.98	0.00617155708216804\\
62.99	0.00617174946303849\\
63	0.00617194194839074\\
63.01	0.00617213453831509\\
63.02	0.00617232723290196\\
63.03	0.00617252003224192\\
63.04	0.00617271293642562\\
63.05	0.00617290594554384\\
63.06	0.00617309905968749\\
63.07	0.00617329227894759\\
63.08	0.00617348560341527\\
63.09	0.0061736790331818\\
63.1	0.00617387256833855\\
63.11	0.00617406620897704\\
63.12	0.00617425995518887\\
63.13	0.00617445380706578\\
63.14	0.00617464776469963\\
63.15	0.00617484182818241\\
63.16	0.00617503599760622\\
63.17	0.00617523027306328\\
63.18	0.00617542465464593\\
63.19	0.00617561914244664\\
63.2	0.00617581373655799\\
63.21	0.0061760084370727\\
63.22	0.0061762032440836\\
63.23	0.00617639815768364\\
63.24	0.00617659317796589\\
63.25	0.00617678830502356\\
63.26	0.00617698353894997\\
63.27	0.00617717887983857\\
63.28	0.00617737432778291\\
63.29	0.0061775698828767\\
63.3	0.00617776554521374\\
63.31	0.00617796131488799\\
63.32	0.0061781571919935\\
63.33	0.00617835317662446\\
63.34	0.00617854926887519\\
63.35	0.00617874546884011\\
63.36	0.0061789417766138\\
63.37	0.00617913819229095\\
63.38	0.00617933471596636\\
63.39	0.00617953134773497\\
63.4	0.00617972808769185\\
63.41	0.00617992493593219\\
63.42	0.00618012189255131\\
63.43	0.00618031895764464\\
63.44	0.00618051613130776\\
63.45	0.00618071341363636\\
63.46	0.00618091080472626\\
63.47	0.00618110830467342\\
63.48	0.00618130591357391\\
63.49	0.00618150363152393\\
63.5	0.00618170145861982\\
63.51	0.00618189939495804\\
63.52	0.00618209744063517\\
63.53	0.00618229559574793\\
63.54	0.00618249386039315\\
63.55	0.00618269223466782\\
63.56	0.00618289071866904\\
63.57	0.00618308931249403\\
63.58	0.00618328801624015\\
63.59	0.00618348683000489\\
63.6	0.00618368575388587\\
63.61	0.00618388478798083\\
63.62	0.00618408393238764\\
63.63	0.00618428318720432\\
63.64	0.00618448255252899\\
63.65	0.00618468202845993\\
63.66	0.00618488161509552\\
63.67	0.0061850813125343\\
63.68	0.00618528112087492\\
63.69	0.00618548104021616\\
63.7	0.00618568107065695\\
63.71	0.00618588121229632\\
63.72	0.00618608146523347\\
63.73	0.00618628182956769\\
63.74	0.00618648230539844\\
63.75	0.00618668289282528\\
63.76	0.00618688359194791\\
63.77	0.00618708440286618\\
63.78	0.00618728532568006\\
63.79	0.00618748636048963\\
63.8	0.00618768750739513\\
63.81	0.00618788876649693\\
63.82	0.00618809013789552\\
63.83	0.00618829162169152\\
63.84	0.0061884932179857\\
63.85	0.00618869492687894\\
63.86	0.00618889674847229\\
63.87	0.00618909868286688\\
63.88	0.00618930073016401\\
63.89	0.0061895028904651\\
63.9	0.0061897051638717\\
63.91	0.00618990755048551\\
63.92	0.00619011005040835\\
63.93	0.00619031266374216\\
63.94	0.00619051539058904\\
63.95	0.00619071823105121\\
63.96	0.00619092118523101\\
63.97	0.00619112425323093\\
63.98	0.0061913274351536\\
63.99	0.00619153073110176\\
64	0.00619173414117829\\
64.01	0.00619193766548622\\
64.02	0.0061921413041287\\
64.03	0.00619234505720902\\
64.04	0.00619254892483058\\
64.05	0.00619275290709694\\
64.06	0.00619295700411179\\
64.07	0.00619316121597894\\
64.08	0.00619336554280233\\
64.09	0.00619356998468606\\
64.1	0.00619377454173434\\
64.11	0.00619397921405153\\
64.12	0.00619418400174208\\
64.13	0.00619438890491064\\
64.14	0.00619459392366194\\
64.15	0.00619479905810086\\
64.16	0.00619500430833241\\
64.17	0.00619520967446175\\
64.18	0.00619541515659414\\
64.19	0.006195620754835\\
64.2	0.00619582646928987\\
64.21	0.00619603230006442\\
64.22	0.00619623824726445\\
64.23	0.00619644431099591\\
64.24	0.00619665049136487\\
64.25	0.00619685678847753\\
64.26	0.0061970632024402\\
64.27	0.00619726973335937\\
64.28	0.00619747638134162\\
64.29	0.00619768314649368\\
64.3	0.0061978900289224\\
64.31	0.00619809702873477\\
64.32	0.0061983041460379\\
64.33	0.00619851138093904\\
64.34	0.00619871873354557\\
64.35	0.00619892620396498\\
64.36	0.00619913379230491\\
64.37	0.00619934149867312\\
64.38	0.0061995493231775\\
64.39	0.00619975726592608\\
64.4	0.00619996532702699\\
64.41	0.00620017350658851\\
64.42	0.00620038180471905\\
64.43	0.00620059022152713\\
64.44	0.0062007987571214\\
64.45	0.00620100741161064\\
64.46	0.00620121618510377\\
64.47	0.00620142507770982\\
64.48	0.00620163408953793\\
64.49	0.0062018432206974\\
64.5	0.00620205247129763\\
64.51	0.00620226184144814\\
64.52	0.0062024713312586\\
64.53	0.00620268094083877\\
64.54	0.00620289067029855\\
64.55	0.00620310051974797\\
64.56	0.00620331048929716\\
64.57	0.00620352057905639\\
64.58	0.00620373078913602\\
64.59	0.00620394111964657\\
64.6	0.00620415157069867\\
64.61	0.00620436214240304\\
64.62	0.00620457283487053\\
64.63	0.00620478364821213\\
64.64	0.00620499458253893\\
64.65	0.00620520563796212\\
64.66	0.00620541681459304\\
64.67	0.00620562811254311\\
64.68	0.0062058395319239\\
64.69	0.00620605107284704\\
64.7	0.00620626273542434\\
64.71	0.00620647451976766\\
64.72	0.00620668642598901\\
64.73	0.0062068984542005\\
64.74	0.00620711060451434\\
64.75	0.00620732287704286\\
64.76	0.00620753527189849\\
64.77	0.00620774778919377\\
64.78	0.00620796042904135\\
64.79	0.00620817319155397\\
64.8	0.0062083860768445\\
64.81	0.00620859908502588\\
64.82	0.00620881221621118\\
64.83	0.00620902547051355\\
64.84	0.00620923884804627\\
64.85	0.00620945234892268\\
64.86	0.00620966597325626\\
64.87	0.00620987972116054\\
64.88	0.00621009359274919\\
64.89	0.00621030758813595\\
64.9	0.00621052170743465\\
64.91	0.00621073595075924\\
64.92	0.00621095031822373\\
64.93	0.00621116480994224\\
64.94	0.00621137942602897\\
64.95	0.0062115941665982\\
64.96	0.00621180903176433\\
64.97	0.00621202402164181\\
64.98	0.00621223913634518\\
64.99	0.00621245437598907\\
65	0.00621266974068821\\
65.01	0.00621288523055736\\
65.02	0.00621310084571141\\
65.03	0.0062133165862653\\
65.04	0.00621353245233405\\
65.05	0.00621374844403276\\
65.06	0.00621396456147659\\
65.07	0.00621418080478079\\
65.08	0.00621439717406066\\
65.09	0.00621461366943158\\
65.1	0.006214830291009\\
65.11	0.00621504703890844\\
65.12	0.00621526391324544\\
65.13	0.00621548091413566\\
65.14	0.00621569804169479\\
65.15	0.00621591529603858\\
65.16	0.00621613267728286\\
65.17	0.00621635018554346\\
65.18	0.00621656782093633\\
65.19	0.00621678558357743\\
65.2	0.00621700347358278\\
65.21	0.00621722149106844\\
65.22	0.00621743963615054\\
65.23	0.00621765790894523\\
65.24	0.00621787630956871\\
65.25	0.00621809483813722\\
65.26	0.00621831349476705\\
65.27	0.00621853227957451\\
65.28	0.00621875119267595\\
65.29	0.00621897023418777\\
65.3	0.00621918940422637\\
65.31	0.00621940870290819\\
65.32	0.00621962813034972\\
65.33	0.00621984768666744\\
65.34	0.00622006737197786\\
65.35	0.00622028718639755\\
65.36	0.00622050713004304\\
65.37	0.00622072720303092\\
65.38	0.00622094740547777\\
65.39	0.00622116773750019\\
65.4	0.00622138819921479\\
65.41	0.00622160879073819\\
65.42	0.00622182951218701\\
65.43	0.00622205036367788\\
65.44	0.00622227134532743\\
65.45	0.00622249245725229\\
65.46	0.00622271369956907\\
65.47	0.0062229350723944\\
65.48	0.00622315657584489\\
65.49	0.00622337821003713\\
65.5	0.00622359997508773\\
65.51	0.00622382187111324\\
65.52	0.00622404389823024\\
65.53	0.00622426605655525\\
65.54	0.0062244883462048\\
65.55	0.00622471076729538\\
65.56	0.00622493331994346\\
65.57	0.00622515600426548\\
65.58	0.00622537882037786\\
65.59	0.00622560176839697\\
65.6	0.00622582484843916\\
65.61	0.00622604806062073\\
65.62	0.00622627140505796\\
65.63	0.00622649488186707\\
65.64	0.00622671849116425\\
65.65	0.00622694223306564\\
65.66	0.00622716610768734\\
65.67	0.00622739011514538\\
65.68	0.00622761425555577\\
65.69	0.00622783852903444\\
65.7	0.00622806293569727\\
65.71	0.0062282874756601\\
65.72	0.00622851214903869\\
65.73	0.00622873695594876\\
65.74	0.00622896189650594\\
65.75	0.00622918697082582\\
65.76	0.00622941217902391\\
65.77	0.00622963752121567\\
65.78	0.00622986299751646\\
65.79	0.0062300886080416\\
65.8	0.00623031435290633\\
65.81	0.0062305402322258\\
65.82	0.0062307662461151\\
65.83	0.00623099239468925\\
65.84	0.00623121867806318\\
65.85	0.00623144509635174\\
65.86	0.00623167164966973\\
65.87	0.00623189833813182\\
65.88	0.00623212516185264\\
65.89	0.00623235212094673\\
65.9	0.00623257921552854\\
65.91	0.00623280644571246\\
65.92	0.00623303381161275\\
65.93	0.00623326131334364\\
65.94	0.00623348895101926\\
65.95	0.00623371672475364\\
65.96	0.00623394463466075\\
65.97	0.00623417268085447\\
65.98	0.00623440086344861\\
65.99	0.00623462918255688\\
66	0.00623485763829294\\
66.01	0.00623508623077034\\
66.02	0.00623531496010257\\
66.03	0.00623554382640306\\
66.04	0.00623577282978514\\
66.05	0.00623600197036208\\
66.06	0.0062362312482471\\
66.07	0.00623646066355334\\
66.08	0.00623669021639387\\
66.09	0.0062369199068817\\
66.1	0.00623714973512982\\
66.11	0.00623737970125111\\
66.12	0.00623760980535844\\
66.13	0.00623784004756463\\
66.14	0.00623807042798246\\
66.15	0.00623830094672467\\
66.16	0.00623853160390397\\
66.17	0.00623876239963305\\
66.18	0.00623899333402458\\
66.19	0.00623922440719123\\
66.2	0.00623945561924565\\
66.21	0.00623968697030049\\
66.22	0.00623991846046843\\
66.23	0.00624015008986215\\
66.24	0.00624038185859437\\
66.25	0.00624061376677784\\
66.26	0.00624084581452535\\
66.27	0.00624107800194976\\
66.28	0.00624131032916397\\
66.29	0.006241542796281\\
66.3	0.0062417754034139\\
66.31	0.00624200815067589\\
66.32	0.00624224103818025\\
66.33	0.00624247406604042\\
66.34	0.00624270723436999\\
66.35	0.00624294054328268\\
66.36	0.00624317399289241\\
66.37	0.00624340758331332\\
66.38	0.00624364131465971\\
66.39	0.00624387518704617\\
66.4	0.00624410920058752\\
66.41	0.00624434335539885\\
66.42	0.00624457765159556\\
66.43	0.00624481208929336\\
66.44	0.00624504666860831\\
66.45	0.00624528138965686\\
66.46	0.00624551625255582\\
66.47	0.00624575125742245\\
66.48	0.00624598640437445\\
66.49	0.00624622169353001\\
66.5	0.00624645712500782\\
66.51	0.00624669269892713\\
66.52	0.00624692841540774\\
66.53	0.0062471642745701\\
66.54	0.00624740027653527\\
66.55	0.00624763642142503\\
66.56	0.00624787270936185\\
66.57	0.006248109140469\\
66.58	0.00624834571487051\\
66.59	0.0062485824326913\\
66.6	0.00624881929405717\\
66.61	0.00624905629909486\\
66.62	0.00624929344793209\\
66.63	0.00624953074069763\\
66.64	0.00624976817752136\\
66.65	0.00625000575853426\\
66.66	0.00625024348386856\\
66.67	0.00625048135365772\\
66.68	0.00625071936803653\\
66.69	0.00625095752714116\\
66.7	0.00625119583110922\\
66.71	0.00625143428007986\\
66.72	0.00625167287419377\\
66.73	0.00625191161359333\\
66.74	0.00625215049842264\\
66.75	0.00625238952882758\\
66.76	0.00625262870495595\\
66.77	0.0062528680269575\\
66.78	0.00625310749498401\\
66.79	0.00625334710918942\\
66.8	0.0062535868697299\\
66.81	0.00625382677676394\\
66.82	0.00625406683045243\\
66.83	0.0062543070309588\\
66.84	0.00625454737844908\\
66.85	0.00625478787309205\\
66.86	0.00625502851505931\\
66.87	0.00625526930452541\\
66.88	0.006255510241668\\
66.89	0.00625575132666786\\
66.9	0.00625599255970915\\
66.91	0.00625623394097942\\
66.92	0.00625647547066982\\
66.93	0.00625671714897522\\
66.94	0.00625695897609434\\
66.95	0.00625720095222989\\
66.96	0.00625744307758874\\
66.97	0.00625768535238207\\
66.98	0.00625792777682552\\
66.99	0.00625817035113938\\
67	0.00625841307554872\\
67.01	0.0062586559502836\\
67.02	0.00625889897557925\\
67.03	0.00625914215167623\\
67.04	0.00625938547882064\\
67.05	0.00625962895726434\\
67.06	0.00625987258726509\\
67.07	0.00626011636908684\\
67.08	0.00626036030299989\\
67.09	0.00626060438928112\\
67.1	0.00626084862812658\\
67.11	0.00626109301971804\\
67.12	0.00626133756423789\\
67.13	0.00626158226186913\\
67.14	0.0062618271127954\\
67.15	0.00626207211720095\\
67.16	0.00626231727527067\\
67.17	0.00626256258719013\\
67.18	0.00626280805314549\\
67.19	0.00626305367332364\\
67.2	0.00626329944791208\\
67.21	0.00626354537709901\\
67.22	0.00626379146107331\\
67.23	0.00626403770002455\\
67.24	0.00626428409414299\\
67.25	0.00626453064361959\\
67.26	0.00626477734864605\\
67.27	0.00626502420941477\\
67.28	0.00626527122611889\\
67.29	0.00626551839895227\\
67.3	0.00626576572810953\\
67.31	0.00626601321378605\\
67.32	0.00626626085617798\\
67.33	0.00626650865548222\\
67.34	0.00626675661189647\\
67.35	0.00626700472561922\\
67.36	0.00626725299684976\\
67.37	0.00626750142578819\\
67.38	0.00626775001263543\\
67.39	0.00626799875759324\\
67.4	0.0062682476608642\\
67.41	0.00626849672265178\\
67.42	0.00626874594316026\\
67.43	0.00626899532259484\\
67.44	0.00626924486116158\\
67.45	0.00626949455906744\\
67.46	0.00626974441652027\\
67.47	0.00626999443372886\\
67.48	0.00627024461090291\\
67.49	0.00627049494825307\\
67.5	0.00627074544599095\\
67.51	0.00627099610432911\\
67.52	0.00627124692348109\\
67.53	0.00627149790366142\\
67.54	0.00627174904508562\\
67.55	0.00627200034797026\\
67.56	0.00627225181253289\\
67.57	0.00627250343899214\\
67.58	0.00627275522756768\\
67.59	0.00627300717848024\\
67.6	0.00627325929195165\\
67.61	0.00627351156820482\\
67.62	0.00627376400746379\\
67.63	0.00627401660995372\\
67.64	0.0062742693759009\\
67.65	0.0062745223055328\\
67.66	0.00627477539907803\\
67.67	0.00627502865676643\\
67.68	0.006275282078829\\
67.69	0.00627553566549799\\
67.7	0.00627578941700689\\
67.71	0.00627604333359043\\
67.72	0.00627629741548463\\
67.73	0.00627655166292676\\
67.74	0.00627680607615545\\
67.75	0.00627706065541062\\
67.76	0.00627731540093354\\
67.77	0.00627757031296686\\
67.78	0.00627782539175459\\
67.79	0.00627808063754215\\
67.8	0.0062783360505764\\
67.81	0.0062785916311056\\
67.82	0.00627884737937952\\
67.83	0.00627910329564937\\
67.84	0.00627935938016789\\
67.85	0.00627961563318935\\
67.86	0.00627987205496954\\
67.87	0.00628012864576585\\
67.88	0.00628038540583725\\
67.89	0.00628064233544431\\
67.9	0.00628089943484928\\
67.91	0.00628115670431604\\
67.92	0.00628141414411017\\
67.93	0.00628167175449896\\
67.94	0.00628192953575143\\
67.95	0.00628218748813839\\
67.96	0.00628244561193242\\
67.97	0.00628270390740792\\
67.98	0.00628296237484112\\
67.99	0.00628322101451017\\
68	0.00628347982669505\\
68.01	0.00628373881167773\\
68.02	0.00628399796974211\\
68.03	0.00628425730117407\\
68.04	0.00628451680626153\\
68.05	0.00628477648529444\\
68.06	0.00628503633856484\\
68.07	0.00628529636636688\\
68.08	0.00628555656899684\\
68.09	0.0062858169467532\\
68.1	0.00628607749993663\\
68.11	0.00628633822885006\\
68.12	0.0062865991337987\\
68.13	0.00628686021509007\\
68.14	0.00628712147303404\\
68.15	0.00628738290794285\\
68.16	0.00628764452013121\\
68.17	0.00628790630991627\\
68.18	0.00628816827761767\\
68.19	0.00628843042355762\\
68.2	0.00628869274806088\\
68.21	0.00628895525145487\\
68.22	0.00628921793406966\\
68.23	0.00628948079623801\\
68.24	0.00628974383829546\\
68.25	0.00629000706058034\\
68.26	0.00629027046343382\\
68.27	0.00629053404719994\\
68.28	0.0062907978122257\\
68.29	0.00629106175886107\\
68.3	0.00629132588745904\\
68.31	0.00629159019837571\\
68.32	0.00629185469197027\\
68.33	0.00629211936860512\\
68.34	0.00629238422864589\\
68.35	0.00629264927246148\\
68.36	0.00629291450042416\\
68.37	0.00629317991290956\\
68.38	0.00629344551029679\\
68.39	0.00629371129296847\\
68.4	0.00629397726131077\\
68.41	0.00629424341571351\\
68.42	0.00629450975657017\\
68.43	0.00629477628427802\\
68.44	0.00629504299923812\\
68.45	0.0062953099018554\\
68.46	0.00629557699253876\\
68.47	0.00629584427170109\\
68.48	0.00629611173975936\\
68.49	0.00629637939713471\\
68.5	0.00629664724425248\\
68.51	0.00629691528154229\\
68.52	0.00629718350943815\\
68.53	0.00629745192837851\\
68.54	0.00629772053880632\\
68.55	0.00629798934116911\\
68.56	0.00629825833591914\\
68.57	0.00629852752351336\\
68.58	0.00629879690441359\\
68.59	0.00629906647908658\\
68.6	0.00629933624800407\\
68.61	0.00629960621164289\\
68.62	0.00629987637048507\\
68.63	0.00630014672501791\\
68.64	0.00630041727573407\\
68.65	0.00630068802313169\\
68.66	0.00630095896771445\\
68.67	0.00630123010999169\\
68.68	0.00630150145047853\\
68.69	0.00630177298969593\\
68.7	0.00630204472817081\\
68.71	0.00630231666643617\\
68.72	0.00630258880503121\\
68.73	0.00630286114450138\\
68.74	0.00630313368539857\\
68.75	0.00630340642828117\\
68.76	0.00630367937371423\\
68.77	0.00630395252226952\\
68.78	0.00630422587452573\\
68.79	0.00630449943106852\\
68.8	0.00630477319249071\\
68.81	0.00630504715939237\\
68.82	0.00630532133238097\\
68.83	0.0063055957120715\\
68.84	0.00630587029908665\\
68.85	0.00630614509405687\\
68.86	0.00630642009762062\\
68.87	0.00630669531042443\\
68.88	0.00630697073312308\\
68.89	0.00630724636637976\\
68.9	0.00630752221086624\\
68.91	0.00630779826726299\\
68.92	0.00630807453625936\\
68.93	0.00630835101855376\\
68.94	0.00630862771485383\\
68.95	0.00630890462587658\\
68.96	0.00630918175234859\\
68.97	0.00630945909500622\\
68.98	0.00630973665459574\\
68.99	0.00631001443187353\\
69	0.00631029242760633\\
69.01	0.00631057064257134\\
69.02	0.0063108490775565\\
69.03	0.00631112773336066\\
69.04	0.0063114066107938\\
69.05	0.00631168571067722\\
69.06	0.00631196503384381\\
69.07	0.00631224458113819\\
69.08	0.00631252435341704\\
69.09	0.00631280435154923\\
69.1	0.00631308457641614\\
69.11	0.00631336502891185\\
69.12	0.0063136457099434\\
69.13	0.00631392662043107\\
69.14	0.00631420776130858\\
69.15	0.00631448913352343\\
69.16	0.00631477073803709\\
69.17	0.00631505257582534\\
69.18	0.00631533464787852\\
69.19	0.00631561695520182\\
69.2	0.00631589949881557\\
69.21	0.00631618227975554\\
69.22	0.00631646529907329\\
69.23	0.00631674855783639\\
69.24	0.00631703205712885\\
69.25	0.00631731579805136\\
69.26	0.00631759978172166\\
69.27	0.0063178840092749\\
69.28	0.00631816848186393\\
69.29	0.00631845320065975\\
69.3	0.00631873816685178\\
69.31	0.00631902338164828\\
69.32	0.00631930884627674\\
69.33	0.00631959456198423\\
69.34	0.00631988053003786\\
69.35	0.00632016675172512\\
69.36	0.00632045322835434\\
69.37	0.00632073996125511\\
69.38	0.00632102695177869\\
69.39	0.0063213142012985\\
69.4	0.00632160171121053\\
69.41	0.00632188948293385\\
69.42	0.00632217751791102\\
69.43	0.00632246581760868\\
69.44	0.00632275438351793\\
69.45	0.00632304321715493\\
69.46	0.00632333232006139\\
69.47	0.00632362169380508\\
69.48	0.00632391133998042\\
69.49	0.00632420126020899\\
69.5	0.00632449145614016\\
69.51	0.0063247819294516\\
69.52	0.00632507268184994\\
69.53	0.00632536371507135\\
69.54	0.00632565503088218\\
69.55	0.00632594663107958\\
69.56	0.00632623851749217\\
69.57	0.00632653069198071\\
69.58	0.00632682315643878\\
69.59	0.00632711591279348\\
69.6	0.00632740896300613\\
69.61	0.00632770230907307\\
69.62	0.00632799595302633\\
69.63	0.00632828989693444\\
69.64	0.00632858414290321\\
69.65	0.00632887869307656\\
69.66	0.0063291735496373\\
69.67	0.00632946871480801\\
69.68	0.00632976419085187\\
69.69	0.00633005998007361\\
69.7	0.00633035608482033\\
69.71	0.0063306525074825\\
69.72	0.00633094925049486\\
69.73	0.00633124631633743\\
69.74	0.00633154370753647\\
69.75	0.00633184142666554\\
69.76	0.00633213947634651\\
69.77	0.00633243785925065\\
69.78	0.0063327365780997\\
69.79	0.00633303563566702\\
69.8	0.00633333503477873\\
69.81	0.00633363477831487\\
69.82	0.00633393486921064\\
69.83	0.00633423531045759\\
69.84	0.00633453610510491\\
69.85	0.00633483725626074\\
69.86	0.00633513876709345\\
69.87	0.00633544064083305\\
69.88	0.00633574288077256\\
69.89	0.0063360454902694\\
69.9	0.00633634847274694\\
69.91	0.00633665183169591\\
69.92	0.00633695557067598\\
69.93	0.00633725969331733\\
69.94	0.00633756420332224\\
69.95	0.00633786910446677\\
69.96	0.00633817440060242\\
69.97	0.00633848009565789\\
69.98	0.00633878619364086\\
69.99	0.00633909269863978\\
70	0.00633939961482579\\
70.01	0.00633970694645457\\
70.02	0.00634001469786835\\
70.03	0.0063403228734979\\
70.04	0.00634063147786462\\
70.05	0.00634094051558261\\
70.06	0.00634124999136085\\
70.07	0.00634155991000546\\
70.08	0.00634187027642192\\
70.09	0.00634218109561747\\
70.1	0.00634249237270343\\
70.11	0.00634280411289772\\
70.12	0.00634311632152734\\
70.13	0.00634342900403098\\
70.14	0.00634374216596165\\
70.15	0.00634405581298939\\
70.16	0.00634436995090409\\
70.17	0.00634468458561829\\
70.18	0.00634499972317017\\
70.19	0.00634531536972653\\
70.2	0.00634563153158585\\
70.21	0.00634594821518149\\
70.22	0.00634626542708494\\
70.23	0.00634658317400909\\
70.24	0.00634690146281171\\
70.25	0.00634722030049892\\
70.26	0.00634753969422879\\
70.27	0.00634785965131501\\
70.28	0.00634818017923072\\
70.29	0.00634850128561236\\
70.3	0.00634882297826364\\
70.31	0.00634914526515965\\
70.32	0.00634946815445108\\
70.33	0.00634979165446846\\
70.34	0.00635011577372663\\
70.35	0.00635044052092927\\
70.36	0.00635076590497353\\
70.37	0.00635109193495483\\
70.38	0.00635141862017175\\
70.39	0.00635174597013109\\
70.4	0.00635207399455296\\
70.41	0.0063524027033762\\
70.42	0.00635273210676373\\
70.43	0.00635306221510817\\
70.44	0.00635339303903759\\
70.45	0.00635372458942139\\
70.46	0.00635405687737633\\
70.47	0.00635438991427279\\
70.48	0.00635472371174108\\
70.49	0.00635505828167807\\
70.5	0.0063553936362538\\
70.51	0.00635572978791852\\
70.52	0.00635606674940966\\
70.53	0.00635640453375918\\
70.54	0.00635674315430102\\
70.55	0.00635708262467879\\
70.56	0.00635742295885367\\
70.57	0.00635776417111246\\
70.58	0.00635810627607597\\
70.59	0.00635844928870751\\
70.6	0.0063587932243217\\
70.61	0.00635913809859347\\
70.62	0.00635948392756734\\
70.63	0.00635983072766693\\
70.64	0.00636017851570471\\
70.65	0.0063605273088921\\
70.66	0.00636087712484974\\
70.67	0.0063612279816181\\
70.68	0.0063615798976684\\
70.69	0.00636193289191378\\
70.7	0.00636228698372078\\
70.71	0.00636264219292121\\
70.72	0.00636299853982424\\
70.73	0.00636335604522892\\
70.74	0.00636371473043697\\
70.75	0.00636407461726597\\
70.76	0.00636443572806293\\
70.77	0.00636479808571815\\
70.78	0.00636516171367959\\
70.79	0.00636552663596751\\
70.8	0.00636589287718965\\
70.81	0.00636626046255671\\
70.82	0.00636662941789833\\
70.83	0.00636699976967952\\
70.84	0.00636737154501751\\
70.85	0.00636774477169911\\
70.86	0.00636811947819851\\
70.87	0.00636849569369568\\
70.88	0.00636887344809511\\
70.89	0.00636925277204527\\
70.9	0.0063696336969585\\
70.91	0.00637001625503146\\
70.92	0.00637040047926624\\
70.93	0.00637078640349199\\
70.94	0.00637117406238717\\
70.95	0.00637156349150243\\
70.96	0.00637195472728421\\
70.97	0.00637234780709888\\
70.98	0.0063727427692577\\
70.99	0.00637313965304236\\
71	0.00637353849873135\\
71.01	0.00637393934762701\\
71.02	0.00637434224208341\\
71.03	0.00637474722553497\\
71.04	0.00637515434252595\\
71.05	0.0063755636387407\\
71.06	0.00637597516103488\\
71.07	0.00637638895746752\\
71.08	0.00637680507733394\\
71.09	0.00637722357119975\\
71.1	0.00637764449093571\\
71.11	0.00637806788975363\\
71.12	0.00637849382224333\\
71.13	0.00637892234441059\\
71.14	0.0063793535137163\\
71.15	0.00637978738911662\\
71.16	0.00638022403110442\\
71.17	0.0063806635017518\\
71.18	0.00638110586475393\\
71.19	0.00638155118547412\\
71.2	0.00638199953099018\\
71.21	0.00638245097014216\\
71.22	0.00638290557358145\\
71.23	0.00638336341382136\\
71.24	0.00638382456528903\\
71.25	0.00638428910437905\\
71.26	0.00638475710950846\\
71.27	0.00638522866117349\\
71.28	0.00638570384200784\\
71.29	0.00638618273684275\\
71.3	0.00638666543276881\\
71.31	0.00638715201919952\\
71.32	0.00638764258793675\\
71.33	0.00638813723323817\\
71.34	0.00638863605188657\\
71.35	0.00638913914326125\\
71.36	0.00638964660941148\\
71.37	0.00639015855513222\\
71.38	0.00639067508804196\\
71.39	0.00639119631866283\\
71.4	0.00639172236050321\\
71.41	0.00639225333014266\\
71.42	0.00639278934731937\\
71.43	0.00639333053502026\\
71.44	0.00639387701957365\\
71.45	0.00639442893074475\\
71.46	0.00639498640183395\\
71.47	0.00639554956977799\\
71.48	0.00639611857525419\\
71.49	0.00639669356278776\\
71.5	0.00639727468086223\\
71.51	0.00639786208203327\\
71.52	0.00639845592304583\\
71.53	0.00639905636495475\\
71.54	0.00639966357324907\\
71.55	0.00640027771797991\\
71.56	0.00640089897389225\\
71.57	0.00640152752056066\\
71.58	0.00640216302649419\\
71.59	0.00640279888908074\\
71.6	0.0064034351086195\\
71.61	0.0064040716853987\\
71.62	0.0064047086196952\\
71.63	0.00640534591177394\\
71.64	0.00640598356188743\\
71.65	0.00640662157027524\\
71.66	0.00640725993716345\\
71.67	0.00640789866276403\\
71.68	0.00640853774727432\\
71.69	0.00640917719087639\\
71.7	0.00640981699373641\\
71.71	0.00641045715600402\\
71.72	0.00641109767781164\\
71.73	0.00641173855927379\\
71.74	0.0064123798004864\\
71.75	0.00641302140152605\\
71.76	0.0064136633624492\\
71.77	0.00641430568329144\\
71.78	0.00641494836406665\\
71.79	0.00641559140476619\\
71.8	0.00641623480535802\\
71.81	0.0064168785657858\\
71.82	0.00641752268596802\\
71.83	0.00641816716579699\\
71.84	0.00641881200513788\\
71.85	0.00641945720382776\\
71.86	0.00642010276167447\\
71.87	0.00642074867845562\\
71.88	0.00642139495391743\\
71.89	0.00642204158777361\\
71.9	0.00642268857970418\\
71.91	0.0064233359293542\\
71.92	0.00642398363633258\\
71.93	0.00642463170021074\\
71.94	0.00642528012052127\\
71.95	0.00642592889675655\\
71.96	0.00642657802836732\\
71.97	0.0064272275147612\\
71.98	0.00642787735530119\\
71.99	0.00642852754930406\\
72	0.00642917809603879\\
72.01	0.00642982899472481\\
72.02	0.00643048024453038\\
72.03	0.00643113184457072\\
72.04	0.00643178379390622\\
72.05	0.00643243609154052\\
72.06	0.0064330887364186\\
72.07	0.0064337417274247\\
72.08	0.00643439506338029\\
72.09	0.00643504874304186\\
72.1	0.0064357027650988\\
72.11	0.006436357128171\\
72.12	0.00643701183080659\\
72.13	0.00643766687147941\\
72.14	0.0064383222485866\\
72.15	0.0064389779604459\\
72.16	0.00643963400529308\\
72.17	0.00644029038127908\\
72.18	0.00644094708646723\\
72.19	0.00644160411883029\\
72.2	0.00644226147624741\\
72.21	0.00644291915650102\\
72.22	0.00644357715727355\\
72.23	0.00644423547614416\\
72.24	0.00644489411058528\\
72.25	0.00644555305795904\\
72.26	0.00644621231551363\\
72.27	0.00644687188037951\\
72.28	0.00644753174956552\\
72.29	0.00644819191995485\\
72.3	0.00644885238830088\\
72.31	0.00644951315122291\\
72.32	0.00645017420520172\\
72.33	0.00645083554657504\\
72.34	0.00645149717153276\\
72.35	0.00645215907611218\\
72.36	0.00645282125619289\\
72.37	0.00645348370749165\\
72.38	0.00645414642555705\\
72.39	0.00645480940576392\\
72.4	0.00645547264330773\\
72.41	0.00645613613319867\\
72.42	0.00645679987025556\\
72.43	0.00645746384909963\\
72.44	0.00645812806414804\\
72.45	0.0064587925096072\\
72.46	0.0064594571794659\\
72.47	0.00646012206748816\\
72.48	0.00646078716720597\\
72.49	0.00646145247191163\\
72.5	0.00646211797464996\\
72.51	0.00646278366821026\\
72.52	0.00646344954511789\\
72.53	0.00646411559762577\\
72.54	0.00646478181770539\\
72.55	0.00646544819703773\\
72.56	0.00646611472700369\\
72.57	0.0064667813986744\\
72.58	0.00646744820280106\\
72.59	0.00646811512980452\\
72.6	0.00646878216976456\\
72.61	0.00646944931240868\\
72.62	0.00647011654710071\\
72.63	0.0064707838628289\\
72.64	0.00647145124819367\\
72.65	0.00647211869139498\\
72.66	0.00647278618021928\\
72.67	0.00647345370202599\\
72.68	0.00647412124373358\\
72.69	0.00647478879180518\\
72.7	0.00647545633223372\\
72.71	0.00647612385052657\\
72.72	0.00647679133168966\\
72.73	0.00647745876021116\\
72.74	0.00647812612004449\\
72.75	0.00647879339459399\\
72.76	0.00647946056671667\\
72.77	0.00648012761870557\\
72.78	0.00648079453227256\\
72.79	0.0064814612885308\\
72.8	0.00648212786797647\\
72.81	0.00648279425047009\\
72.82	0.00648346041521725\\
72.83	0.00648412634074874\\
72.84	0.00648479200490015\\
72.85	0.00648545738479075\\
72.86	0.00648612245680185\\
72.87	0.00648678719655448\\
72.88	0.00648745157888638\\
72.89	0.00648811557782831\\
72.9	0.00648877916657968\\
72.91	0.00648944231748344\\
72.92	0.00649010500200018\\
72.93	0.00649076719068153\\
72.94	0.00649142885314272\\
72.95	0.00649208995803431\\
72.96	0.00649275047301311\\
72.97	0.00649341036471221\\
72.98	0.00649406959871011\\
72.99	0.006494729046915\\
73	0.00649538897871655\\
73.01	0.0064960493968363\\
73.02	0.0064967103040254\\
73.03	0.00649737170306447\\
73.04	0.00649803359676347\\
73.05	0.0064986959879615\\
73.06	0.00649935887952661\\
73.07	0.00650002227435557\\
73.08	0.00650068617537358\\
73.09	0.00650135058553399\\
73.1	0.00650201550781791\\
73.11	0.00650268094523387\\
73.12	0.00650334690081738\\
73.13	0.00650401337763049\\
73.14	0.00650468037876121\\
73.15	0.00650534790732307\\
73.16	0.00650601596645438\\
73.17	0.0065066845593177\\
73.18	0.00650735368909904\\
73.19	0.00650802335900713\\
73.2	0.00650869357227261\\
73.21	0.00650936433214711\\
73.22	0.00651003564190234\\
73.23	0.00651070750482904\\
73.24	0.00651137992423591\\
73.25	0.00651205290344848\\
73.26	0.0065127264458078\\
73.27	0.00651340055466923\\
73.28	0.00651407523340093\\
73.29	0.00651475048538248\\
73.3	0.00651542631400323\\
73.31	0.0065161027226607\\
73.32	0.00651677971475874\\
73.33	0.00651745729370572\\
73.34	0.00651813546291254\\
73.35	0.00651881422579052\\
73.36	0.00651949358574924\\
73.37	0.00652017354619417\\
73.38	0.00652085411052426\\
73.39	0.00652153528212935\\
73.4	0.00652221706438741\\
73.41	0.00652289946066176\\
73.42	0.00652358247429801\\
73.43	0.00652426610862089\\
73.44	0.00652495036693096\\
73.45	0.00652563525250112\\
73.46	0.00652632076857289\\
73.47	0.00652700691835267\\
73.48	0.0065276937050076\\
73.49	0.00652838113166138\\
73.5	0.00652906920138987\\
73.51	0.00652975791721638\\
73.52	0.00653044728210684\\
73.53	0.00653113729896472\\
73.54	0.00653182797062567\\
73.55	0.00653251929985195\\
73.56	0.00653321128932661\\
73.57	0.00653390394164736\\
73.58	0.0065345972593202\\
73.59	0.00653529124475276\\
73.6	0.00653598590024729\\
73.61	0.00653668122799345\\
73.62	0.00653737723006062\\
73.63	0.00653807390839\\
73.64	0.00653877126478631\\
73.65	0.00653946930090915\\
73.66	0.00654016801826395\\
73.67	0.00654086741819252\\
73.68	0.00654156750186328\\
73.69	0.00654226827026096\\
73.7	0.00654296972417591\\
73.71	0.00654367186419298\\
73.72	0.00654437469067989\\
73.73	0.00654507820377511\\
73.74	0.00654578240337527\\
73.75	0.00654648728912198\\
73.76	0.00654719286038818\\
73.77	0.00654789911626385\\
73.78	0.00654860605554115\\
73.79	0.00654931367669901\\
73.8	0.00655002197788699\\
73.81	0.00655073095690853\\
73.82	0.00655144061120353\\
73.83	0.0065521509378302\\
73.84	0.00655286193344618\\
73.85	0.00655357359428887\\
73.86	0.00655428591615506\\
73.87	0.00655499889437963\\
73.88	0.00655571252381347\\
73.89	0.00655642679880048\\
73.9	0.00655714171315371\\
73.91	0.00655785726013047\\
73.92	0.00655857343240657\\
73.93	0.00655929022204942\\
73.94	0.00656000762049019\\
73.95	0.00656072561849476\\
73.96	0.00656144420613366\\
73.97	0.00656216337275074\\
73.98	0.00656288310693064\\
73.99	0.00656360339646509\\
74	0.00656432422831772\\
74.01	0.00656504558858773\\
74.02	0.00656576746247198\\
74.03	0.0065664898342257\\
74.04	0.00656721268712175\\
74.05	0.00656793600340817\\
74.06	0.00656865976426428\\
74.07	0.00656938394975496\\
74.08	0.0065701085387833\\
74.09	0.0065708335090414\\
74.1	0.00657155885873774\\
74.11	0.00657228458806359\\
74.12	0.00657301069719022\\
74.13	0.00657373718626799\\
74.14	0.00657446405542531\\
74.15	0.00657519130476771\\
74.16	0.00657591893437674\\
74.17	0.00657664694430894\\
74.18	0.00657737533459466\\
74.19	0.00657810410523693\\
74.2	0.00657883325621026\\
74.21	0.00657956278745936\\
74.22	0.00658029269889788\\
74.23	0.00658102299040704\\
74.24	0.00658175366183424\\
74.25	0.00658248471299163\\
74.26	0.00658321614365462\\
74.27	0.00658394795356028\\
74.28	0.00658468014240583\\
74.29	0.00658541270984687\\
74.3	0.00658614565549574\\
74.31	0.0065868789789197\\
74.32	0.00658761267963908\\
74.33	0.00658834675712537\\
74.34	0.00658908121079926\\
74.35	0.00658981604002853\\
74.36	0.00659055124412601\\
74.37	0.00659128682234728\\
74.38	0.00659202277388846\\
74.39	0.00659275909788385\\
74.4	0.00659349579340343\\
74.41	0.00659423285945039\\
74.42	0.00659497029495847\\
74.43	0.00659570809878927\\
74.44	0.00659644626972945\\
74.45	0.00659718480648779\\
74.46	0.00659792370769222\\
74.47	0.00659866297188672\\
74.48	0.00659940259752802\\
74.49	0.00660014258298237\\
74.5	0.00660088292652201\\
74.51	0.00660162362632165\\
74.52	0.00660236468045475\\
74.53	0.00660310608688973\\
74.54	0.00660384784348595\\
74.55	0.00660458994798971\\
74.56	0.00660533239802994\\
74.57	0.00660607519111386\\
74.58	0.00660681832462244\\
74.59	0.00660756179580571\\
74.6	0.00660830560177793\\
74.61	0.00660904973951254\\
74.62	0.00660979420583701\\
74.63	0.00661053899742745\\
74.64	0.00661128411080308\\
74.65	0.00661202954232046\\
74.66	0.00661277528816758\\
74.67	0.00661352134435772\\
74.68	0.00661426770672308\\
74.69	0.00661501437090822\\
74.7	0.00661576133236329\\
74.71	0.00661650858633698\\
74.72	0.00661725612786927\\
74.73	0.00661800395178391\\
74.74	0.00661875205268068\\
74.75	0.00661950042492734\\
74.76	0.00662024906265131\\
74.77	0.00662099795973113\\
74.78	0.00662174710978756\\
74.79	0.0066224965061744\\
74.8	0.00662324614196902\\
74.81	0.00662399605618991\\
74.82	0.00662474636015234\\
74.83	0.006625497054455\\
74.84	0.00662624813969999\\
74.85	0.00662699961649284\\
74.86	0.00662775148544255\\
74.87	0.00662850374716159\\
74.88	0.0066292564022659\\
74.89	0.00663000945137495\\
74.9	0.0066307628951117\\
74.91	0.00663151673410265\\
74.92	0.00663227096897785\\
74.93	0.00663302560037089\\
74.94	0.00663378062891896\\
74.95	0.00663453605526279\\
74.96	0.00663529188004672\\
74.97	0.0066360481039187\\
74.98	0.00663680472753027\\
74.99	0.00663756175153659\\
75	0.00663831917659645\\
75.01	0.00663907700337228\\
75.02	0.00663983523253013\\
75.03	0.00664059386473969\\
75.04	0.00664135290067431\\
75.05	0.00664211234101096\\
75.06	0.00664287218643029\\
75.07	0.00664363243761656\\
75.08	0.0066443930952577\\
75.09	0.00664515416004529\\
75.1	0.00664591563267452\\
75.11	0.00664667751384425\\
75.12	0.00664743980425692\\
75.13	0.00664820250461864\\
75.14	0.0066489656156391\\
75.15	0.00664972913803159\\
75.16	0.006650493072513\\
75.17	0.00665125741980378\\
75.18	0.00665202218062792\\
75.19	0.00665278735571298\\
75.2	0.00665355294579\\
75.21	0.00665431895159355\\
75.22	0.00665508537386164\\
75.23	0.00665585221333573\\
75.24	0.00665661947076071\\
75.25	0.00665738714688483\\
75.26	0.00665815524245971\\
75.27	0.00665892375824028\\
75.28	0.00665969269498475\\
75.29	0.00666046205345455\\
75.3	0.00666123183441432\\
75.31	0.00666200203863185\\
75.32	0.00666277266687801\\
75.33	0.00666354371992673\\
75.34	0.00666431519855492\\
75.35	0.00666508710354243\\
75.36	0.00666585943567197\\
75.37	0.00666663219572907\\
75.38	0.00666740538450199\\
75.39	0.00666817900278164\\
75.4	0.00666895305136153\\
75.41	0.00666972753103768\\
75.42	0.00667050244260853\\
75.43	0.00667127778687485\\
75.44	0.00667205356463966\\
75.45	0.00667282977670811\\
75.46	0.00667360642388743\\
75.47	0.00667438350698675\\
75.48	0.00667516102681704\\
75.49	0.00667593898419099\\
75.5	0.00667671737992287\\
75.51	0.0066774962148284\\
75.52	0.00667827548972466\\
75.53	0.0066790552054299\\
75.54	0.00667983536276342\\
75.55	0.00668061596254542\\
75.56	0.00668139700559685\\
75.57	0.00668217849273923\\
75.58	0.0066829604247945\\
75.59	0.00668374280258482\\
75.6	0.00668452562693241\\
75.61	0.00668530889865935\\
75.62	0.00668609261858737\\
75.63	0.00668687678753768\\
75.64	0.0066876614063307\\
75.65	0.0066884464757859\\
75.66	0.00668923199672152\\
75.67	0.00669001796995435\\
75.68	0.00669080439629949\\
75.69	0.00669159127657007\\
75.7	0.00669237861157702\\
75.71	0.00669316640212874\\
75.72	0.00669395464903087\\
75.73	0.00669474335308592\\
75.74	0.00669553251509306\\
75.75	0.00669632213584769\\
75.76	0.00669711221614121\\
75.77	0.00669790275676058\\
75.78	0.00669869375848807\\
75.79	0.0066994852221008\\
75.8	0.0067002771483704\\
75.81	0.00670106953806261\\
75.82	0.00670186239193688\\
75.83	0.00670265571074596\\
75.84	0.0067034494952354\\
75.85	0.00670424374614319\\
75.86	0.00670503846419921\\
75.87	0.0067058336501248\\
75.88	0.00670662930463227\\
75.89	0.00670742542842431\\
75.9	0.00670822202219355\\
75.91	0.00670901908662194\\
75.92	0.00670981662238022\\
75.93	0.00671061463012731\\
75.94	0.00671141311050972\\
75.95	0.0067122120641609\\
75.96	0.00671301149170063\\
75.97	0.0067138113937343\\
75.98	0.00671461177085226\\
75.99	0.00671541262362909\\
76	0.00671621395262288\\
76.01	0.00671701575837442\\
76.02	0.00671781804140648\\
76.03	0.00671862080222296\\
76.04	0.00671942404130809\\
76.05	0.00672022775912549\\
76.06	0.0067210319561174\\
76.07	0.00672183663270363\\
76.08	0.00672264178928074\\
76.09	0.00672344742622095\\
76.1	0.00672425354387123\\
76.11	0.00672506014255217\\
76.12	0.006725867222557\\
76.13	0.0067266747841504\\
76.14	0.00672748282756742\\
76.15	0.0067282913530123\\
76.16	0.00672910036065719\\
76.17	0.00672990985064102\\
76.18	0.00673071982306811\\
76.19	0.00673153027800688\\
76.2	0.0067323412154885\\
76.21	0.00673315263550546\\
76.22	0.00673396453801012\\
76.23	0.0067347769229132\\
76.24	0.00673558979008228\\
76.25	0.00673640313934015\\
76.26	0.00673721697046323\\
76.27	0.00673803128317985\\
76.28	0.0067388460771685\\
76.29	0.00673966135205609\\
76.3	0.00674047710741604\\
76.31	0.00674129334276642\\
76.32	0.00674211005756799\\
76.33	0.00674292725122216\\
76.34	0.00674374492306895\\
76.35	0.00674456307238481\\
76.36	0.00674538169838044\\
76.37	0.00674620080019851\\
76.38	0.00674702037691134\\
76.39	0.00674784042751847\\
76.4	0.00674866095094421\\
76.41	0.00674948194603505\\
76.42	0.00675030341155707\\
76.43	0.00675112534619322\\
76.44	0.00675194774854052\\
76.45	0.0067527706171072\\
76.46	0.00675359395030979\\
76.47	0.00675441774647001\\
76.48	0.00675524200381168\\
76.49	0.00675606672045751\\
76.5	0.00675689189442574\\
76.51	0.00675771752362679\\
76.52	0.00675854360585968\\
76.53	0.00675937013880845\\
76.54	0.00676019712003846\\
76.55	0.00676102454699249\\
76.56	0.00676185241698686\\
76.57	0.00676268072720735\\
76.58	0.00676350947470502\\
76.59	0.00676433865639192\\
76.6	0.00676516826903663\\
76.61	0.00676599830925977\\
76.62	0.00676682877352927\\
76.63	0.00676765965815555\\
76.64	0.00676849095928658\\
76.65	0.00676932267290276\\
76.66	0.00677015479481166\\
76.67	0.00677098732064263\\
76.68	0.00677182024584122\\
76.69	0.00677265356566346\\
76.7	0.00677348727517\\
76.71	0.00677432136921998\\
76.72	0.00677515584246488\\
76.73	0.00677599068934203\\
76.74	0.00677682590406806\\
76.75	0.00677766148063208\\
76.76	0.0067784974127887\\
76.77	0.00677933369405084\\
76.78	0.00678017031768233\\
76.79	0.00678100727669031\\
76.8	0.0067818445638174\\
76.81	0.00678268217153361\\
76.82	0.00678352009202812\\
76.83	0.00678435831720067\\
76.84	0.0067851968386529\\
76.85	0.00678603564767919\\
76.86	0.00678687473525749\\
76.87	0.00678771409203973\\
76.88	0.00678855370834198\\
76.89	0.00678939357413437\\
76.9	0.00679023367903067\\
76.91	0.00679107401227761\\
76.92	0.0067919145627439\\
76.93	0.00679275531890886\\
76.94	0.00679359626885079\\
76.95	0.00679443740023505\\
76.96	0.00679527870030164\\
76.97	0.00679612015585264\\
76.98	0.00679696175323906\\
76.99	0.0067978034783475\\
77	0.00679864531658634\\
77.01	0.00679948725287152\\
77.02	0.00680032927161197\\
77.03	0.0068011713566946\\
77.04	0.0068020134914688\\
77.05	0.00680285565873059\\
77.06	0.00680369784070629\\
77.07	0.00680454001903564\\
77.08	0.00680538217475455\\
77.09	0.00680622428827731\\
77.1	0.00680706633937824\\
77.11	0.00680790830717292\\
77.12	0.00680875017009877\\
77.13	0.00680959190589515\\
77.14	0.00681043349158285\\
77.15	0.00681127490344304\\
77.16	0.00681211611699552\\
77.17	0.00681295710697646\\
77.18	0.00681379784731545\\
77.19	0.00681463831111186\\
77.2	0.00681547847061055\\
77.21	0.00681631829717693\\
77.22	0.00681715776127121\\
77.23	0.00681799683242199\\
77.24	0.00681883547919906\\
77.25	0.00681967366918541\\
77.26	0.00682051136894845\\
77.27	0.0068213485440104\\
77.28	0.00682218515881784\\
77.29	0.00682302117671037\\
77.3	0.00682385655988836\\
77.31	0.00682469126938396\\
77.32	0.00682552526503408\\
77.33	0.0068263585054461\\
77.34	0.0068271909479627\\
77.35	0.00682802254862569\\
77.36	0.00682885326213877\\
77.37	0.0068296830418293\\
77.38	0.00683051183960899\\
77.39	0.00683133960593341\\
77.4	0.00683216628976048\\
77.41	0.00683299183850771\\
77.42	0.00683381619800826\\
77.43	0.00683463931246574\\
77.44	0.00683546112440781\\
77.45	0.00683628157463836\\
77.46	0.00683710060218851\\
77.47	0.00683791814426601\\
77.48	0.00683873413620344\\
77.49	0.00683954851140479\\
77.5	0.00684036120129063\\
77.51	0.00684117213524169\\
77.52	0.00684198124054086\\
77.53	0.00684278844231355\\
77.54	0.00684359366346645\\
77.55	0.00684439682462438\\
77.56	0.00684519784406561\\
77.57	0.00684599663765511\\
77.58	0.00684679311877612\\
77.59	0.00684758719825966\\
77.6	0.00684837878431211\\
77.61	0.00684916778244075\\
77.62	0.00684995425835396\\
77.63	0.00685074098203392\\
77.64	0.00685152795052877\\
77.65	0.00685231516083168\\
77.66	0.00685310260988019\\
77.67	0.00685389029455566\\
77.68	0.00685467821168268\\
77.69	0.00685546635802844\\
77.7	0.00685625473030225\\
77.71	0.00685704332515491\\
77.72	0.00685783213917821\\
77.73	0.00685862116890439\\
77.74	0.00685941041080562\\
77.75	0.00686019986129356\\
77.76	0.0068609895167188\\
77.77	0.00686177937337048\\
77.78	0.00686256942747582\\
77.79	0.0068633596751997\\
77.8	0.00686415011264432\\
77.81	0.0068649407358488\\
77.82	0.00686573154078884\\
77.83	0.00686652252337647\\
77.84	0.00686731367945973\\
77.85	0.00686810500482247\\
77.86	0.00686889649518411\\
77.87	0.00686968814619953\\
77.88	0.00687047995345892\\
77.89	0.0068712719124877\\
77.9	0.0068720640187465\\
77.91	0.00687285626763118\\
77.92	0.00687364865447292\\
77.93	0.00687444117453827\\
77.94	0.00687523382302942\\
77.95	0.00687602659508438\\
77.96	0.00687681948577732\\
77.97	0.00687761249011893\\
77.98	0.00687840560305686\\
77.99	0.00687919881947624\\
78	0.00687999213420028\\
78.01	0.00688078554199095\\
78.02	0.00688157903754973\\
78.03	0.00688237261551851\\
78.04	0.00688316627048049\\
78.05	0.00688395999696126\\
78.06	0.00688475378942997\\
78.07	0.00688554764230058\\
78.08	0.00688634154993326\\
78.09	0.0068871355066359\\
78.1	0.00688792950666574\\
78.11	0.00688872354423115\\
78.12	0.00688951761349354\\
78.13	0.00689031170856942\\
78.14	0.0068911058235326\\
78.15	0.00689189995241658\\
78.16	0.00689269408921712\\
78.17	0.00689348822789488\\
78.18	0.00689428236237841\\
78.19	0.00689507648656722\\
78.2	0.00689587059433507\\
78.21	0.00689666467953345\\
78.22	0.00689745873599542\\
78.23	0.00689825275753944\\
78.24	0.00689904673797367\\
78.25	0.00689984067110037\\
78.26	0.00690063455072065\\
78.27	0.00690142837063944\\
78.28	0.0069022221246708\\
78.29	0.00690301580664343\\
78.3	0.00690380941040663\\
78.31	0.00690460292983646\\
78.32	0.00690539635884234\\
78.33	0.00690618969137389\\
78.34	0.00690698292142827\\
78.35	0.00690777604305778\\
78.36	0.00690856905037796\\
78.37	0.00690936193757604\\
78.38	0.00691015469891985\\
78.39	0.0069109473287672\\
78.4	0.0069117398215757\\
78.41	0.0069125321719131\\
78.42	0.00691332437446809\\
78.43	0.00691411642406174\\
78.44	0.00691490831565937\\
78.45	0.00691570004438308\\
78.46	0.00691649160552485\\
78.47	0.0069172829945603\\
78.48	0.00691807420716303\\
78.49	0.00691886523921976\\
78.5	0.00691965608684601\\
78.51	0.0069204467464027\\
78.52	0.00692123721451337\\
78.53	0.00692202748808226\\
78.54	0.00692281756431323\\
78.55	0.0069236074407295\\
78.56	0.00692439711519429\\
78.57	0.00692518658593245\\
78.58	0.00692597585155299\\
78.59	0.00692676491107264\\
78.6	0.00692755376394048\\
78.61	0.00692834241006362\\
78.62	0.00692913084983403\\
78.63	0.00692991908415658\\
78.64	0.00693070711447823\\
78.65	0.00693149494281856\\
78.66	0.0069322825718016\\
78.67	0.00693307000468898\\
78.68	0.00693385724541462\\
78.69	0.00693464429862078\\
78.7	0.00693543116969574\\
78.71	0.00693621786481302\\
78.72	0.0069370043909723\\
78.73	0.00693779075604207\\
78.74	0.00693857696880402\\
78.75	0.00693936303899938\\
78.76	0.00694014897737708\\
78.77	0.00694093479574406\\
78.78	0.00694172050701755\\
78.79	0.00694250612527954\\
78.8	0.00694329166583357\\
78.81	0.00694407714526378\\
78.82	0.00694486258149639\\
78.83	0.00694564799386379\\
78.84	0.00694643340317109\\
78.85	0.00694721883176553\\
78.86	0.00694800430360865\\
78.87	0.00694878984435137\\
78.88	0.00694957548141211\\
78.89	0.00695036124405811\\
78.9	0.00695114716348997\\
78.91	0.00695193327292961\\
78.92	0.00695271960771177\\
78.93	0.00695350620537909\\
78.94	0.00695429310578112\\
78.95	0.00695508035117716\\
78.96	0.0069558679863432\\
78.97	0.00695665605868319\\
78.98	0.0069574446183446\\
78.99	0.00695823371833861\\
79	0.00695902337359642\\
79.01	0.00695981358597765\\
79.02	0.00696060435736961\\
79.03	0.00696139568968789\\
79.04	0.00696218758487702\\
79.05	0.00696298004491101\\
79.06	0.00696377307179408\\
79.07	0.0069645666675612\\
79.08	0.00696536083427885\\
79.09	0.00696615557404564\\
79.1	0.00696695088899304\\
79.11	0.00696774678128608\\
79.12	0.00696854325312407\\
79.13	0.00696934030674136\\
79.14	0.00697013794440811\\
79.15	0.00697093616843106\\
79.16	0.00697173498115435\\
79.17	0.0069725343849603\\
79.18	0.00697333438227031\\
79.19	0.00697413497554568\\
79.2	0.00697493616728847\\
79.21	0.00697573796004249\\
79.22	0.00697654035639412\\
79.23	0.00697734335897332\\
79.24	0.00697814697045459\\
79.25	0.00697895119355794\\
79.26	0.00697975603104992\\
79.27	0.00698056148574468\\
79.28	0.006981367560505\\
79.29	0.00698217425824341\\
79.3	0.00698298158192329\\
79.31	0.00698378953456001\\
79.32	0.00698459811922216\\
79.33	0.00698540733903267\\
79.34	0.00698621719717007\\
79.35	0.00698702769686978\\
79.36	0.00698783884142539\\
79.37	0.00698865063418995\\
79.38	0.00698946307857737\\
79.39	0.00699027617806375\\
79.4	0.00699108993618889\\
79.41	0.00699190435655767\\
79.42	0.00699271944284157\\
79.43	0.0069935351987802\\
79.44	0.0069943516281829\\
79.45	0.00699516873493026\\
79.46	0.00699598652297587\\
79.47	0.00699680499634792\\
79.48	0.006997624159151\\
79.49	0.00699844401556779\\
79.5	0.00699926456986097\\
79.51	0.00700008582637499\\
79.52	0.00700090778953803\\
79.53	0.00700173046386394\\
79.54	0.00700255385395423\\
79.55	0.00700337796450014\\
79.56	0.00700420280028472\\
79.57	0.007005028366185\\
79.58	0.0070058546671742\\
79.59	0.00700668170832398\\
79.6	0.00700750949480677\\
79.61	0.00700833803189816\\
79.62	0.00700916732497928\\
79.63	0.00700999737953939\\
79.64	0.00701082820117837\\
79.65	0.00701165979560935\\
79.66	0.00701249216866144\\
79.67	0.00701332532628247\\
79.68	0.0070141592745418\\
79.69	0.00701499401963324\\
79.7	0.00701582956787802\\
79.71	0.00701666592572783\\
79.72	0.00701750309976796\\
79.73	0.00701834109672047\\
79.74	0.0070191799234475\\
79.75	0.00702001958695467\\
79.76	0.00702086009439445\\
79.77	0.00702170145306977\\
79.78	0.00702254367043763\\
79.79	0.00702338675411281\\
79.8	0.0070242307118717\\
79.81	0.00702507555165621\\
79.82	0.00702592128157779\\
79.83	0.00702676790992153\\
79.84	0.00702761544515039\\
79.85	0.00702846389590955\\
79.86	0.00702931327103079\\
79.87	0.00703016357953713\\
79.88	0.0070310148306474\\
79.89	0.0070318670337811\\
79.9	0.00703272019856331\\
79.91	0.00703357433482966\\
79.92	0.00703442945263158\\
79.93	0.00703528556224154\\
79.94	0.0070361426741585\\
79.95	0.0070370007991135\\
79.96	0.00703785994807537\\
79.97	0.00703872013225658\\
79.98	0.00703958136311927\\
79.99	0.00704044365238141\\
80	0.00704130701202317\\
80.01	0.00704217145429333\\
};
\addplot [color=blue,solid]
  table[row sep=crcr]{%
80.01	0.00704217145429333\\
80.02	0.00704303699171604\\
80.03	0.00704390363709758\\
80.04	0.0070447714035334\\
80.05	0.00704564030441531\\
80.06	0.00704651035343886\\
80.07	0.0070473815646109\\
80.08	0.00704825395225734\\
80.09	0.00704912753103115\\
80.1	0.00705000231592049\\
80.11	0.00705087832225712\\
80.12	0.007051755565725\\
80.13	0.00705263406236912\\
80.14	0.00705351382860457\\
80.15	0.00705439488122577\\
80.16	0.00705527723741613\\
80.17	0.00705616091475767\\
80.18	0.00705704593124124\\
80.19	0.00705793230527665\\
80.2	0.00705882005570338\\
80.21	0.00705970920180134\\
80.22	0.00706059976330202\\
80.23	0.00706149176039996\\
80.24	0.0070623852137644\\
80.25	0.00706328014455135\\
80.26	0.00706417657441594\\
80.27	0.00706507452552506\\
80.28	0.00706597402057037\\
80.29	0.00706687508278165\\
80.3	0.00706777773594051\\
80.31	0.00706868200439438\\
80.32	0.00706958791307102\\
80.33	0.00707049548749322\\
80.34	0.00707140475379408\\
80.35	0.00707231573873257\\
80.36	0.00707322846970951\\
80.37	0.00707414297478401\\
80.38	0.00707505928269034\\
80.39	0.00707597742285523\\
80.4	0.00707689742541562\\
80.41	0.00707781932123689\\
80.42	0.00707874314193157\\
80.43	0.00707966891987857\\
80.44	0.00708059668824285\\
80.45	0.0070815264809957\\
80.46	0.0070824583329355\\
80.47	0.00708339227970904\\
80.48	0.00708432835783341\\
80.49	0.00708526660471847\\
80.5	0.00708620705868992\\
80.51	0.00708714975901298\\
80.52	0.00708809474591669\\
80.53	0.00708904206061889\\
80.54	0.00708999174535183\\
80.55	0.00709094384338842\\
80.56	0.00709189839906933\\
80.57	0.00709285545783062\\
80.58	0.00709381506623226\\
80.59	0.00709477727198731\\
80.6	0.00709574212399195\\
80.61	0.00709670967235624\\
80.62	0.00709767996843578\\
80.63	0.00709865306486414\\
80.64	0.00709962901558624\\
80.65	0.00710060787589251\\
80.66	0.0071015897024541\\
80.67	0.0071025745533589\\
80.68	0.00710356248814862\\
80.69	0.00710455356785681\\
80.7	0.00710554785504794\\
80.71	0.0071065454138575\\
80.72	0.00710754631003315\\
80.73	0.00710855061097704\\
80.74	0.00710955838578922\\
80.75	0.00711056970531221\\
80.76	0.00711158464217681\\
80.77	0.00711260327084909\\
80.78	0.00711362566767872\\
80.79	0.00711465191094846\\
80.8	0.00711568208092515\\
80.81	0.00711671625991197\\
80.82	0.0071177545323021\\
80.83	0.0071187969846339\\
80.84	0.00711984370564754\\
80.85	0.00712089478634309\\
80.86	0.00712195032004031\\
80.87	0.00712301040243994\\
80.88	0.00712407513168671\\
80.89	0.00712514460843402\\
80.9	0.0071262189359104\\
80.91	0.0071272982199877\\
80.92	0.00712838256925123\\
80.93	0.00712947209507168\\
80.94	0.00713056691167911\\
80.95	0.00713166713623883\\
80.96	0.00713277288892939\\
80.97	0.00713388429302271\\
80.98	0.00713500147496633\\
80.99	0.00713612456446794\\
81	0.00713725369458212\\
81.01	0.00713838900179958\\
81.02	0.00713953062613868\\
81.03	0.00714067871123951\\
81.04	0.00714183340446053\\
81.05	0.00714299485697778\\
81.06	0.0071441632238869\\
81.07	0.00714533866430783\\
81.08	0.00714652134149236\\
81.09	0.00714771142293468\\
81.1	0.00714890908048492\\
81.11	0.00715011449046574\\
81.12	0.00715132783379214\\
81.13	0.00715254929609456\\
81.14	0.00715377906784531\\
81.15	0.00715501734448843\\
81.16	0.00715626432657318\\
81.17	0.00715752021989111\\
81.18	0.00715878523561688\\
81.19	0.007160059590453\\
81.2	0.00716134350677849\\
81.21	0.00716263721280156\\
81.22	0.0071639409427166\\
81.23	0.00716525493686536\\
81.24	0.00716657944190262\\
81.25	0.00716791471096638\\
81.26	0.00716926100385273\\
81.27	0.00717061858719553\\
81.28	0.00717198773465103\\
81.29	0.00717336872708754\\
81.3	0.00717476185278041\\
81.31	0.00717616740761223\\
81.32	0.00717758569527868\\
81.33	0.00717901702750002\\
81.34	0.00718046172423833\\
81.35	0.00718192011392091\\
81.36	0.00718339253366964\\
81.37	0.0071848793295369\\
81.38	0.00718638085674779\\
81.39	0.00718789747994924\\
81.4	0.0071894295734659\\
81.41	0.00719097752156313\\
81.42	0.00719254171871732\\
81.43	0.00719412256989366\\
81.44	0.00719572049083166\\
81.45	0.0071973359083385\\
81.46	0.0071989692605907\\
81.47	0.00720062099744396\\
81.48	0.0072022915807518\\
81.49	0.00720398148469297\\
81.5	0.00720569119610799\\
81.51	0.00720742121484509\\
81.52	0.00720915720360016\\
81.53	0.00721089405183198\\
81.54	0.00721263176180608\\
81.55	0.00721437033579901\\
81.56	0.00721610977609802\\
81.57	0.00721785008500078\\
81.58	0.00721959126481496\\
81.59	0.00722133331785796\\
81.6	0.00722307624645643\\
81.61	0.00722482005294588\\
81.62	0.0072265647396703\\
81.63	0.00722831030898166\\
81.64	0.00723005676323942\\
81.65	0.00723180410481009\\
81.66	0.00723355233606664\\
81.67	0.00723530145938797\\
81.68	0.00723705147715835\\
81.69	0.00723880239176677\\
81.7	0.00724055420560636\\
81.71	0.00724230692107368\\
81.72	0.00724406054056803\\
81.73	0.00724581506649075\\
81.74	0.0072475705012444\\
81.75	0.00724932684723206\\
81.76	0.00725108410685639\\
81.77	0.00725284228251884\\
81.78	0.00725460137661873\\
81.79	0.00725636139155227\\
81.8	0.00725812232971163\\
81.81	0.00725988419348388\\
81.82	0.00726164698524992\\
81.83	0.00726341070738342\\
81.84	0.00726517536224956\\
81.85	0.00726694095220394\\
81.86	0.00726870747959124\\
81.87	0.00727047494674396\\
81.88	0.00727224335598106\\
81.89	0.00727401270960651\\
81.9	0.00727578300990787\\
81.91	0.00727755425915476\\
81.92	0.00727932645959724\\
81.93	0.00728109961346422\\
81.94	0.00728287372296169\\
81.95	0.007284648790271\\
81.96	0.00728642481754701\\
81.97	0.00728820180691615\\
81.98	0.00728997976047444\\
81.99	0.00729175868028551\\
82	0.00729353856837834\\
82.01	0.00729531942674517\\
82.02	0.00729710125733912\\
82.03	0.00729888406207187\\
82.04	0.00730066784281115\\
82.05	0.00730245260137824\\
82.06	0.00730423833954526\\
82.07	0.00730602505903248\\
82.08	0.00730781276150547\\
82.09	0.00730960144857214\\
82.1	0.00731139112177971\\
82.11	0.00731318178261156\\
82.12	0.00731497343248396\\
82.13	0.00731676607274268\\
82.14	0.0073185597046595\\
82.15	0.00732035432942863\\
82.16	0.00732214994816288\\
82.17	0.00732394656188987\\
82.18	0.00732574417154796\\
82.19	0.00732754277798214\\
82.2	0.00732934238193973\\
82.21	0.00733114298406592\\
82.22	0.00733294458489924\\
82.23	0.00733474718486674\\
82.24	0.00733655078427918\\
82.25	0.00733835538332585\\
82.26	0.00734016098206944\\
82.27	0.00734196758044053\\
82.28	0.00734377517823207\\
82.29	0.00734558377509353\\
82.3	0.00734739337052498\\
82.31	0.00734920396387085\\
82.32	0.00735101555431364\\
82.33	0.0073528281408672\\
82.34	0.00735464172237006\\
82.35	0.00735645629747829\\
82.36	0.0073582718646583\\
82.37	0.0073600884221793\\
82.38	0.00736190596810558\\
82.39	0.00736372450028848\\
82.4	0.00736554401635815\\
82.41	0.00736736451371503\\
82.42	0.00736918598952103\\
82.43	0.00737100844069042\\
82.44	0.00737283186388051\\
82.45	0.00737465625548189\\
82.46	0.00737648161160849\\
82.47	0.00737830792808725\\
82.48	0.00738013520044744\\
82.49	0.00738196342390972\\
82.5	0.00738379259337478\\
82.51	0.00738562270341162\\
82.52	0.00738745374824553\\
82.53	0.0073892857217456\\
82.54	0.00739111861741185\\
82.55	0.00739295242836205\\
82.56	0.00739478714731797\\
82.57	0.00739662276659132\\
82.58	0.00739845927806917\\
82.59	0.007400296673199\\
82.6	0.00740213494297315\\
82.61	0.00740397407791291\\
82.62	0.00740581406805205\\
82.63	0.00740765490291983\\
82.64	0.00740949657152353\\
82.65	0.00741133906233034\\
82.66	0.00741318236324881\\
82.67	0.00741502646160964\\
82.68	0.00741687134414588\\
82.69	0.00741871699697254\\
82.7	0.00742056340556558\\
82.71	0.00742241055474022\\
82.72	0.00742425842862855\\
82.73	0.00742610701065658\\
82.74	0.00742795628352041\\
82.75	0.00742980622916179\\
82.76	0.00743165682874292\\
82.77	0.00743350806262036\\
82.78	0.00743535991031833\\
82.79	0.00743721235050102\\
82.8	0.00743906536094416\\
82.81	0.00744091891850566\\
82.82	0.0074427729990954\\
82.83	0.00744462757764405\\
82.84	0.00744648262807101\\
82.85	0.00744833812325129\\
82.86	0.00745019403498147\\
82.87	0.00745205033394451\\
82.88	0.00745390698967366\\
82.89	0.00745576397051509\\
82.9	0.00745762124358955\\
82.91	0.00745947877475275\\
82.92	0.00746133652855462\\
82.93	0.00746319446819729\\
82.94	0.00746505255549183\\
82.95	0.0074669107508137\\
82.96	0.0074687690130568\\
82.97	0.0074706272995862\\
82.98	0.00747248556618942\\
82.99	0.00747434376702624\\
83	0.007476201854577\\
83.01	0.0074780597795894\\
83.02	0.00747991749102361\\
83.03	0.00748177493599586\\
83.04	0.00748363205972021\\
83.05	0.00748548880544866\\
83.06	0.00748734511440949\\
83.07	0.00748920092574367\\
83.08	0.00749105617643943\\
83.09	0.00749291080126491\\
83.1	0.00749476473269872\\
83.11	0.0074966179008585\\
83.12	0.00749847023342733\\
83.13	0.00750032165557793\\
83.14	0.00750217208989465\\
83.15	0.00750402145629312\\
83.16	0.00750586967193752\\
83.17	0.00750771665115538\\
83.18	0.00750956230534986\\
83.19	0.00751140654290946\\
83.2	0.00751324926911505\\
83.21	0.00751509038604407\\
83.22	0.00751692979247202\\
83.23	0.00751876738377097\\
83.24	0.00752060305180508\\
83.25	0.007522436684823\\
83.26	0.0075242681673472\\
83.27	0.00752609738005992\\
83.28	0.00752792419968583\\
83.29	0.00752974849887124\\
83.3	0.00753157014605965\\
83.31	0.00753338900536381\\
83.32	0.00753520493643386\\
83.33	0.00753701779432172\\
83.34	0.00753882742934136\\
83.35	0.00754063368692514\\
83.36	0.00754243640747577\\
83.37	0.00754423542621401\\
83.38	0.00754603057302196\\
83.39	0.00754782167228161\\
83.4	0.00754960854270881\\
83.41	0.00755139099718232\\
83.42	0.00755316884256784\\
83.43	0.00755494187953691\\
83.44	0.00755670990238059\\
83.45	0.00755847269881756\\
83.46	0.00756023004979678\\
83.47	0.00756198172929422\\
83.48	0.00756372750410383\\
83.49	0.00756546713362233\\
83.5	0.00756720036962772\\
83.51	0.00756892695605137\\
83.52	0.00757064662874353\\
83.53	0.0075723591152319\\
83.54	0.00757406413447329\\
83.55	0.00757576139659803\\
83.56	0.00757745060264696\\
83.57	0.00757913432378243\\
83.58	0.00758081871024833\\
83.59	0.00758250376257283\\
83.6	0.00758418948128547\\
83.61	0.00758587586691725\\
83.62	0.0075875629200006\\
83.63	0.00758925064106941\\
83.64	0.00759093903065902\\
83.65	0.00759262808930626\\
83.66	0.00759431781754948\\
83.67	0.00759600821592851\\
83.68	0.00759769928498473\\
83.69	0.00759939102526104\\
83.7	0.00760108343730193\\
83.71	0.00760277652165343\\
83.72	0.00760447027886319\\
83.73	0.00760616470948044\\
83.74	0.00760785981405605\\
83.75	0.00760955559314252\\
83.76	0.00761125204729401\\
83.77	0.00761294917706637\\
83.78	0.00761464698301711\\
83.79	0.00761634546570547\\
83.8	0.0076180446256924\\
83.81	0.00761974446354063\\
83.82	0.00762144497981461\\
83.83	0.00762314617508061\\
83.84	0.00762484804990668\\
83.85	0.00762655060486269\\
83.86	0.00762825384052036\\
83.87	0.00762995775745328\\
83.88	0.0076316623562369\\
83.89	0.00763336763744859\\
83.9	0.00763507360166763\\
83.91	0.00763678024947525\\
83.92	0.00763848758145466\\
83.93	0.00764019559819102\\
83.94	0.00764190430027155\\
83.95	0.00764361368828546\\
83.96	0.00764532376282403\\
83.97	0.00764703452448063\\
83.98	0.00764874597385073\\
83.99	0.0076504581115319\\
84	0.00765217093812389\\
84.01	0.00765388445422861\\
84.02	0.00765559866045019\\
84.03	0.00765731355739496\\
84.04	0.00765902914567152\\
84.05	0.00766074542589073\\
84.06	0.00766246239866578\\
84.07	0.00766418006461216\\
84.08	0.00766589842434776\\
84.09	0.00766761747849281\\
84.1	0.00766933722766999\\
84.11	0.00767105767250442\\
84.12	0.00767277881362366\\
84.13	0.00767450065165782\\
84.14	0.00767622318723951\\
84.15	0.00767794642100392\\
84.16	0.00767967035358882\\
84.17	0.00768139498563462\\
84.18	0.00768312031778437\\
84.19	0.00768484635068382\\
84.2	0.00768657308498144\\
84.21	0.00768830052132847\\
84.22	0.00769002866037891\\
84.23	0.00769175750278961\\
84.24	0.00769348704922027\\
84.25	0.00769521730033348\\
84.26	0.00769694825679477\\
84.27	0.00769867991927262\\
84.28	0.00770041228843853\\
84.29	0.00770214536496703\\
84.3	0.00770387914953575\\
84.31	0.00770561364282542\\
84.32	0.00770734884551991\\
84.33	0.00770908475830633\\
84.34	0.007710821381875\\
84.35	0.00771255871691953\\
84.36	0.00771429676413683\\
84.37	0.00771603552422722\\
84.38	0.00771777499789436\\
84.39	0.00771951518584543\\
84.4	0.00772125608879105\\
84.41	0.0077229977074454\\
84.42	0.00772474004252626\\
84.43	0.00772648309475501\\
84.44	0.00772822686485674\\
84.45	0.00772997135356025\\
84.46	0.0077317165615981\\
84.47	0.0077334624897067\\
84.48	0.00773520913862633\\
84.49	0.00773695650910117\\
84.5	0.00773870460187942\\
84.51	0.00774045341771326\\
84.52	0.00774220295735899\\
84.53	0.00774395322157703\\
84.54	0.00774570421113197\\
84.55	0.00774745592679268\\
84.56	0.00774920836933231\\
84.57	0.00775096153952838\\
84.58	0.00775271543816282\\
84.59	0.00775447006602202\\
84.6	0.00775622542389694\\
84.61	0.0077579815125831\\
84.62	0.0077597383328807\\
84.63	0.00776149588559466\\
84.64	0.00776325417153466\\
84.65	0.00776501319151524\\
84.66	0.00776677294635588\\
84.67	0.00776853343688097\\
84.68	0.00777029466392002\\
84.69	0.00777205662830761\\
84.7	0.0077738193308835\\
84.71	0.00777558277249273\\
84.72	0.00777734695398564\\
84.73	0.00777911187621798\\
84.74	0.00778087754005096\\
84.75	0.00778264394635133\\
84.76	0.00778441109599148\\
84.77	0.00778617898984946\\
84.78	0.00778794762880913\\
84.79	0.00778971701376019\\
84.8	0.00779148714559824\\
84.81	0.00779325802522495\\
84.82	0.00779502965354804\\
84.83	0.00779680203148142\\
84.84	0.00779857515994527\\
84.85	0.00780034903986612\\
84.86	0.00780212367217691\\
84.87	0.00780389905781714\\
84.88	0.00780567519773291\\
84.89	0.00780745209287702\\
84.9	0.00780922974420906\\
84.91	0.00781100815269552\\
84.92	0.00781278731930988\\
84.93	0.0078145672450327\\
84.94	0.00781634793085171\\
84.95	0.00781812937776191\\
84.96	0.0078199115867657\\
84.97	0.00782169455887296\\
84.98	0.00782347829510112\\
84.99	0.00782526279647533\\
85	0.00782704806402852\\
85.01	0.00782883409880153\\
85.02	0.00783062090184319\\
85.03	0.00783240847421048\\
85.04	0.00783419681696859\\
85.05	0.00783598593119105\\
85.06	0.00783777581795989\\
85.07	0.00783956647836567\\
85.08	0.00784135791350768\\
85.09	0.00784315012449401\\
85.1	0.00784494311244169\\
85.11	0.00784673687847683\\
85.12	0.0078485314237347\\
85.13	0.0078503267493599\\
85.14	0.00785212285650646\\
85.15	0.007853919746338\\
85.16	0.00785571742002782\\
85.17	0.00785751587875907\\
85.18	0.00785931512372487\\
85.19	0.00786111515612845\\
85.2	0.0078629159771833\\
85.21	0.00786471758811327\\
85.22	0.00786651999015277\\
85.23	0.00786832318454687\\
85.24	0.00787012717255148\\
85.25	0.00787193195543346\\
85.26	0.0078737375344708\\
85.27	0.00787554391095278\\
85.28	0.00787735108618007\\
85.29	0.00787915906146496\\
85.3	0.00788096783813146\\
85.31	0.00788277741751549\\
85.32	0.00788458780096502\\
85.33	0.00788639898984028\\
85.34	0.00788821098551386\\
85.35	0.00789002378937092\\
85.36	0.00789183740280937\\
85.37	0.00789365182724\\
85.38	0.00789546706408668\\
85.39	0.00789728311478654\\
85.4	0.00789909998079017\\
85.41	0.00790091766356173\\
85.42	0.00790273616457919\\
85.43	0.00790455548533451\\
85.44	0.00790637562733383\\
85.45	0.00790819659209763\\
85.46	0.00791001838116093\\
85.47	0.00791184099607352\\
85.48	0.0079136644384001\\
85.49	0.00791548870972055\\
85.5	0.00791731381163003\\
85.51	0.0079191397457393\\
85.52	0.00792096651367481\\
85.53	0.00792279411707899\\
85.54	0.00792462255761044\\
85.55	0.0079264518369441\\
85.56	0.00792828195677154\\
85.57	0.00793011291880108\\
85.58	0.0079319447247581\\
85.59	0.00793377737638521\\
85.6	0.00793561087544246\\
85.61	0.00793744522370761\\
85.62	0.00793928042297635\\
85.63	0.00794111647506247\\
85.64	0.00794295338179816\\
85.65	0.00794479114503423\\
85.66	0.0079466297666403\\
85.67	0.00794846924850511\\
85.68	0.00795030959253669\\
85.69	0.00795215080066265\\
85.7	0.00795399287483041\\
85.71	0.00795583581700744\\
85.72	0.00795767962918151\\
85.73	0.00795952431336096\\
85.74	0.0079613698715749\\
85.75	0.00796321630587354\\
85.76	0.0079650636183284\\
85.77	0.00796691181103255\\
85.78	0.00796876088610091\\
85.79	0.00797061084567052\\
85.8	0.00797246169190075\\
85.81	0.00797431342697362\\
85.82	0.00797616605309402\\
85.83	0.00797801957249003\\
85.84	0.00797987398741315\\
85.85	0.0079817293001386\\
85.86	0.00798358551296556\\
85.87	0.00798544262821748\\
85.88	0.00798730064824233\\
85.89	0.00798915957541292\\
85.9	0.00799101941212712\\
85.91	0.00799288016080817\\
85.92	0.00799474182390499\\
85.93	0.00799660440389238\\
85.94	0.00799846790327141\\
85.95	0.00800033232456962\\
85.96	0.00800219767034133\\
85.97	0.00800406394316796\\
85.98	0.00800593114565825\\
85.99	0.00800779928044861\\
86	0.00800966835020334\\
86.01	0.00801153835761498\\
86.02	0.00801340930540457\\
86.03	0.00801528119632193\\
86.04	0.00801715403314593\\
86.05	0.00801902781868483\\
86.06	0.0080209025557765\\
86.07	0.00802277824728876\\
86.08	0.0080246548961196\\
86.09	0.00802653250519754\\
86.1	0.00802841107748182\\
86.11	0.00803029061596276\\
86.12	0.008032171123662\\
86.13	0.00803405260363275\\
86.14	0.00803593505896011\\
86.15	0.00803781849276131\\
86.16	0.00803970290818599\\
86.17	0.00804158830841646\\
86.18	0.00804347469666795\\
86.19	0.00804536207618887\\
86.2	0.0080472504502611\\
86.21	0.00804913982220018\\
86.22	0.0080510301953556\\
86.23	0.00805292157311101\\
86.24	0.00805481395888447\\
86.25	0.00805670735612871\\
86.26	0.00805860176833127\\
86.27	0.00806049719901481\\
86.28	0.00806239365173724\\
86.29	0.008064291130092\\
86.3	0.00806618963770821\\
86.31	0.00806808917825084\\
86.32	0.00806998975542094\\
86.33	0.00807189137295579\\
86.34	0.00807379403462904\\
86.35	0.00807569774425089\\
86.36	0.00807760250566823\\
86.37	0.00807950832276475\\
86.38	0.00808141519946107\\
86.39	0.00808332313971485\\
86.4	0.00808523214752089\\
86.41	0.00808714222691117\\
86.42	0.00808905338195498\\
86.43	0.00809096561675893\\
86.44	0.00809287893546698\\
86.45	0.00809479334226049\\
86.46	0.0080967088413582\\
86.47	0.00809862543701623\\
86.48	0.00810054313352803\\
86.49	0.00810246193522434\\
86.5	0.0081043818464731\\
86.51	0.00810630287167939\\
86.52	0.00810822501528528\\
86.53	0.0081101482817697\\
86.54	0.00811207267564827\\
86.55	0.00811399820147312\\
86.56	0.00811592486383269\\
86.57	0.00811785266735144\\
86.58	0.00811978161668963\\
86.59	0.00812171171654297\\
86.6	0.00812364297164233\\
86.61	0.00812557538675337\\
86.62	0.00812750896667608\\
86.63	0.00812944371624443\\
86.64	0.00813137964032587\\
86.65	0.00813331674382079\\
86.66	0.00813525503166201\\
86.67	0.00813719450881416\\
86.68	0.00813913518027309\\
86.69	0.00814107705106516\\
86.7	0.00814302012624652\\
86.71	0.00814496441090235\\
86.72	0.00814690991014605\\
86.73	0.00814885662911832\\
86.74	0.00815080457298631\\
86.75	0.00815275374694257\\
86.76	0.00815470415620404\\
86.77	0.00815665580601095\\
86.78	0.00815860870162568\\
86.79	0.00816056284833145\\
86.8	0.00816251825143116\\
86.81	0.00816447491624588\\
86.82	0.00816643284811353\\
86.83	0.00816839205238731\\
86.84	0.00817035253443414\\
86.85	0.00817231429963293\\
86.86	0.00817427735337293\\
86.87	0.0081762417010518\\
86.88	0.00817820734807372\\
86.89	0.00818017429984736\\
86.9	0.00818214256178377\\
86.91	0.00818411213929416\\
86.92	0.00818608303778755\\
86.93	0.00818805526266838\\
86.94	0.00819002881933396\\
86.95	0.00819200371317178\\
86.96	0.00819397994955681\\
86.97	0.00819595753384852\\
86.98	0.00819793647138791\\
86.99	0.00819991676749432\\
87	0.00820189842746219\\
87.01	0.00820388145655756\\
87.02	0.00820586586001453\\
87.03	0.00820785164303152\\
87.04	0.00820983881076735\\
87.05	0.00821182736833727\\
87.06	0.00821381732080863\\
87.07	0.00821580867319661\\
87.08	0.00821780143045951\\
87.09	0.00821979559749414\\
87.1	0.00822179117913075\\
87.11	0.00822378818012796\\
87.12	0.00822578660516735\\
87.13	0.00822778645884796\\
87.14	0.00822978774568044\\
87.15	0.00823179047008106\\
87.16	0.0082337946363655\\
87.17	0.00823580024874232\\
87.18	0.00823780731130622\\
87.19	0.00823981582803108\\
87.2	0.00824182580276263\\
87.21	0.008243837239211\\
87.22	0.00824585014094279\\
87.23	0.00824786451137301\\
87.24	0.00824988035375663\\
87.25	0.00825189767117984\\
87.26	0.00825391646655098\\
87.27	0.00825593674259111\\
87.28	0.00825795850182427\\
87.29	0.00825998174656739\\
87.3	0.00826200647891974\\
87.31	0.00826403270075213\\
87.32	0.00826606041369557\\
87.33	0.00826808961912967\\
87.34	0.00827012031817045\\
87.35	0.0082721525116579\\
87.36	0.00827418620014292\\
87.37	0.00827622138387389\\
87.38	0.00827825806278274\\
87.39	0.00828029623647053\\
87.4	0.00828233590419246\\
87.41	0.00828437706484246\\
87.42	0.00828641971693711\\
87.43	0.00828846385859911\\
87.44	0.00829050948754006\\
87.45	0.00829255660104272\\
87.46	0.00829460519594262\\
87.47	0.00829665526860901\\
87.48	0.00829870681492519\\
87.49	0.00830075983026812\\
87.5	0.00830281430948732\\
87.51	0.00830487024688309\\
87.52	0.00830692763618393\\
87.53	0.00830898647052319\\
87.54	0.00831104674241496\\
87.55	0.0083131084437291\\
87.56	0.00831517156566538\\
87.57	0.00831723609872684\\
87.58	0.00831930203269214\\
87.59	0.00832136935658705\\
87.6	0.00832343805865492\\
87.61	0.00832550812632616\\
87.62	0.00832757954618674\\
87.63	0.00832965230394555\\
87.64	0.00833172638440071\\
87.65	0.00833380177140476\\
87.66	0.00833587844782867\\
87.67	0.0083379563955246\\
87.68	0.00834003559528748\\
87.69	0.0083421160268153\\
87.7	0.00834419766866804\\
87.71	0.00834628049822527\\
87.72	0.00834836449164234\\
87.73	0.00835044962380508\\
87.74	0.00835253586828306\\
87.75	0.00835462319728126\\
87.76	0.00835671158159017\\
87.77	0.00835880099053426\\
87.78	0.00836089139191867\\
87.79	0.00836298275197428\\
87.8	0.00836507503530088\\
87.81	0.0083671682048085\\
87.82	0.00836926222165684\\
87.83	0.00837135704519271\\
87.84	0.00837345263288542\\
87.85	0.00837554894026008\\
87.86	0.00837764592082871\\
87.87	0.00837974352601913\\
87.88	0.00838184170510149\\
87.89	0.00838394040511253\\
87.9	0.00838603957077724\\
87.91	0.00838813914442807\\
87.92	0.00839023906592154\\
87.93	0.00839233927255212\\
87.94	0.0083944396989633\\
87.95	0.00839654027705588\\
87.96	0.00839864093589326\\
87.97	0.00840074160160363\\
87.98	0.00840284219727913\\
87.99	0.00840494264287169\\
88	0.00840704285508552\\
88.01	0.00840914274726625\\
88.02	0.0084112422292864\\
88.03	0.00841334120742728\\
88.04	0.00841543958425706\\
88.05	0.00841753725850499\\
88.06	0.00841963412493152\\
88.07	0.00842173007419447\\
88.08	0.00842382499271069\\
88.09	0.00842591876251349\\
88.1	0.00842801126110551\\
88.11	0.00843010236130688\\
88.12	0.00843219193109856\\
88.13	0.00843427983346076\\
88.14	0.00843636592620617\\
88.15	0.00843845006180802\\
88.16	0.00844053208722252\\
88.17	0.0084426118437059\\
88.18	0.0084446891666255\\
88.19	0.008446763885265\\
88.2	0.00844883582262343\\
88.21	0.00845090479520792\\
88.22	0.00845297061281986\\
88.23	0.00845503307833434\\
88.24	0.00845709198747273\\
88.25	0.00845914712856798\\
88.26	0.00846119828232264\\
88.27	0.00846324522155927\\
88.28	0.00846528771096301\\
88.29	0.00846732550681607\\
88.3	0.00846935835672394\\
88.31	0.00847138599933299\\
88.32	0.00847340816403925\\
88.33	0.00847542457068813\\
88.34	0.00847743492926466\\
88.35	0.00847943893957416\\
88.36	0.0084814362909129\\
88.37	0.0084834266617285\\
88.38	0.00848540971926975\\
88.39	0.00848738511922551\\
88.4	0.00848935250535242\\
88.41	0.00849131150909099\\
88.42	0.00849326174916977\\
88.43	0.00849520283119726\\
88.44	0.00849713434724107\\
88.45	0.00849905587539418\\
88.46	0.00850096697932757\\
88.47	0.00850286720782918\\
88.48	0.00850475609432845\\
88.49	0.00850663315640625\\
88.5	0.0085084978952896\\
88.51	0.00851034979533088\\
88.52	0.00851218832347083\\
88.53	0.00851401292868509\\
88.54	0.00851582304141369\\
88.55	0.00851761807297285\\
88.56	0.00851939741494884\\
88.57	0.00852116043857307\\
88.58	0.00852290649407804\\
88.59	0.0085246349100335\\
88.6	0.00852634499266224\\
88.61	0.00852804741214841\\
88.62	0.00852975056275011\\
88.63	0.00853145444487682\\
88.64	0.0085331590589383\\
88.65	0.00853486440534453\\
88.66	0.00853657048450578\\
88.67	0.00853827729683258\\
88.68	0.00853998484273569\\
88.69	0.00854169312262613\\
88.7	0.0085434021369152\\
88.71	0.00854511188601443\\
88.72	0.00854682237033563\\
88.73	0.00854853359029085\\
88.74	0.0085502455462924\\
88.75	0.00855195823875285\\
88.76	0.00855367166808501\\
88.77	0.00855538583470198\\
88.78	0.00855710073901708\\
88.79	0.00855881638144391\\
88.8	0.00856053276239632\\
88.81	0.00856224988228842\\
88.82	0.00856396774153456\\
88.83	0.00856568634054937\\
88.84	0.00856740567974772\\
88.85	0.00856912575954473\\
88.86	0.00857084658035581\\
88.87	0.00857256814259658\\
88.88	0.00857429044668295\\
88.89	0.00857601349303108\\
88.9	0.00857773728205736\\
88.91	0.00857946181417848\\
88.92	0.00858118708981135\\
88.93	0.00858291310937315\\
88.94	0.00858463987328132\\
88.95	0.00858636738195355\\
88.96	0.00858809563580778\\
88.97	0.00858982463526222\\
88.98	0.00859155438073532\\
88.99	0.0085932848726458\\
89	0.00859501611141263\\
89.01	0.00859674809745502\\
89.02	0.00859848083119247\\
89.03	0.0086002143130447\\
89.04	0.00860194854343171\\
89.05	0.00860368352277375\\
89.06	0.00860541925149132\\
89.07	0.00860715573000516\\
89.08	0.00860889295873629\\
89.09	0.00861063093810598\\
89.1	0.00861236966853576\\
89.11	0.0086141091504474\\
89.12	0.00861584938426292\\
89.13	0.00861759037040461\\
89.14	0.00861933210929502\\
89.15	0.00862107460135695\\
89.16	0.00862281784701343\\
89.17	0.00862456184668779\\
89.18	0.00862630660080356\\
89.19	0.00862805210978457\\
89.2	0.00862979837405489\\
89.21	0.00863154539403883\\
89.22	0.00863329317016097\\
89.23	0.00863504170284613\\
89.24	0.00863679099251941\\
89.25	0.00863854103960614\\
89.26	0.0086402918445319\\
89.27	0.00864204340772254\\
89.28	0.00864379572960416\\
89.29	0.00864554881060311\\
89.3	0.00864730265114598\\
89.31	0.00864905725165964\\
89.32	0.0086508126125712\\
89.33	0.00865256873430801\\
89.34	0.00865432561729769\\
89.35	0.0086560832619681\\
89.36	0.00865784166874737\\
89.37	0.00865960083806387\\
89.38	0.00866136077034622\\
89.39	0.00866312146602329\\
89.4	0.00866488292552421\\
89.41	0.00866664514927836\\
89.42	0.00866840813771538\\
89.43	0.00867017189126514\\
89.44	0.00867193641035778\\
89.45	0.00867370169542367\\
89.46	0.00867546774689346\\
89.47	0.00867723456519802\\
89.48	0.00867900215076851\\
89.49	0.0086807705040363\\
89.5	0.00868253962543303\\
89.51	0.00868430951539058\\
89.52	0.0086860801743411\\
89.53	0.00868785160271697\\
89.54	0.00868962380095083\\
89.55	0.00869139676947556\\
89.56	0.0086931705087243\\
89.57	0.00869494501913042\\
89.58	0.00869672030112757\\
89.59	0.00869849635514963\\
89.6	0.00870027318163071\\
89.61	0.00870205078100521\\
89.62	0.00870382915370775\\
89.63	0.00870560830017319\\
89.64	0.00870738822083667\\
89.65	0.00870916891613354\\
89.66	0.00871095038649943\\
89.67	0.00871273263237019\\
89.68	0.00871451565418194\\
89.69	0.00871629945237103\\
89.7	0.00871808402737406\\
89.71	0.00871986937962787\\
89.72	0.00872165550956956\\
89.73	0.00872344241763647\\
89.74	0.00872523010426618\\
89.75	0.00872701856989652\\
89.76	0.00872880781496556\\
89.77	0.00873059783991162\\
89.78	0.00873238864517326\\
89.79	0.00873418023118928\\
89.8	0.00873597259839873\\
89.81	0.0087377657472409\\
89.82	0.00873955967815533\\
89.83	0.00874135439158179\\
89.84	0.0087431498879603\\
89.85	0.00874494616773112\\
89.86	0.00874674323133477\\
89.87	0.00874854107921196\\
89.88	0.0087503397118037\\
89.89	0.00875213912955121\\
89.9	0.00875393933289596\\
89.91	0.00875574032227965\\
89.92	0.00875754209814422\\
89.93	0.00875934466093186\\
89.94	0.008761148011085\\
89.95	0.00876295214904629\\
89.96	0.00876475707525864\\
89.97	0.00876656279016519\\
89.98	0.0087683692942093\\
89.99	0.0087701765878346\\
90	0.00877198467148493\\
90.01	0.00877379354560438\\
90.02	0.00877560321063727\\
90.03	0.00877741366702815\\
90.04	0.00877922491522182\\
90.05	0.0087810369556633\\
90.06	0.00878284978879786\\
90.07	0.00878466341507099\\
90.08	0.00878647783492841\\
90.09	0.00878829304881609\\
90.1	0.00879010905718021\\
90.11	0.00879192586046721\\
90.12	0.00879374345912374\\
90.13	0.00879556185359668\\
90.14	0.00879738104433315\\
90.15	0.0087992010317805\\
90.16	0.0088010218163863\\
90.17	0.00880284339859836\\
90.18	0.00880466577886472\\
90.19	0.00880648895763363\\
90.2	0.00880831293535358\\
90.21	0.0088101377124733\\
90.22	0.00881196328944171\\
90.23	0.00881378966670798\\
90.24	0.00881561684472152\\
90.25	0.00881744482393193\\
90.26	0.00881927360478906\\
90.27	0.00882110318774296\\
90.28	0.00882293357324393\\
90.29	0.00882476476174247\\
90.3	0.00882659675368931\\
90.31	0.00882842954953541\\
90.32	0.00883026314973193\\
90.33	0.00883209755473026\\
90.34	0.00883393276498201\\
90.35	0.00883576878093901\\
90.36	0.0088376056030533\\
90.37	0.00883944323177713\\
90.38	0.008841281667563\\
90.39	0.00884312091086359\\
90.4	0.0088449609621318\\
90.41	0.00884680182182076\\
90.42	0.0088486434903838\\
90.43	0.00885048596827445\\
90.44	0.00885232925594649\\
90.45	0.00885417335385387\\
90.46	0.00885601826245078\\
90.47	0.00885786398219159\\
90.48	0.00885971051353091\\
90.49	0.00886155785692353\\
90.5	0.00886340601282446\\
90.51	0.00886525498168892\\
90.52	0.00886710476397231\\
90.53	0.00886895536013028\\
90.54	0.00887080677061862\\
90.55	0.00887265899589339\\
90.56	0.00887451203641079\\
90.57	0.00887636589262726\\
90.58	0.00887822056499943\\
90.59	0.00888007605398412\\
90.6	0.00888193236003834\\
90.61	0.00888378948361933\\
90.62	0.00888564742518448\\
90.63	0.00888750618519141\\
90.64	0.00888936576409792\\
90.65	0.00889122616236199\\
90.66	0.00889308738044181\\
90.67	0.00889494941879575\\
90.68	0.00889681227788237\\
90.69	0.00889867595816042\\
90.7	0.00890054046008882\\
90.71	0.0089024057841267\\
90.72	0.00890427193073335\\
90.73	0.00890613890036827\\
90.74	0.00890800669349111\\
90.75	0.00890987531056173\\
90.76	0.00891174475204014\\
90.77	0.00891361501838654\\
90.78	0.00891548611006132\\
90.79	0.00891735802752502\\
90.8	0.00891923077123837\\
90.81	0.00892110434166227\\
90.82	0.00892297873925779\\
90.83	0.00892485396448615\\
90.84	0.00892673001780876\\
90.85	0.0089286068996872\\
90.86	0.0089304846105832\\
90.87	0.00893236315095866\\
90.88	0.00893424252127563\\
90.89	0.00893612272199634\\
90.9	0.00893800375358315\\
90.91	0.00893988561649862\\
90.92	0.00894176831120543\\
90.93	0.00894365183816641\\
90.94	0.00894553619784457\\
90.95	0.00894742139070305\\
90.96	0.00894930741720514\\
90.97	0.00895119427781428\\
90.98	0.00895308197299406\\
90.99	0.00895497050320821\\
91	0.00895685986892058\\
91.01	0.00895875007059519\\
91.02	0.00896064110869618\\
91.03	0.00896253298368783\\
91.04	0.00896442569603456\\
91.05	0.00896631924620091\\
91.06	0.00896821363465154\\
91.07	0.00897010886185126\\
91.08	0.008972004928265\\
91.09	0.0089739018343578\\
91.1	0.00897579958059483\\
91.11	0.00897769816744137\\
91.12	0.00897959759536284\\
91.13	0.00898149786482474\\
91.14	0.0089833989762927\\
91.15	0.00898530093023246\\
91.16	0.00898720372710987\\
91.17	0.00898910736739088\\
91.18	0.00899101185154153\\
91.19	0.00899291718002797\\
91.2	0.00899482335331646\\
91.21	0.00899673037187333\\
91.22	0.00899863823616501\\
91.23	0.00900054694665803\\
91.24	0.009002456503819\\
91.25	0.0090043669081146\\
91.26	0.00900627816001161\\
91.27	0.00900819025997687\\
91.28	0.00901010320847733\\
91.29	0.00901201700597996\\
91.3	0.00901393165295184\\
91.31	0.0090158471498601\\
91.32	0.00901776349717194\\
91.33	0.00901968069535461\\
91.34	0.00902159874487542\\
91.35	0.00902351764620174\\
91.36	0.00902543739980098\\
91.37	0.00902735800614061\\
91.38	0.00902927946568814\\
91.39	0.00903120177891111\\
91.4	0.0090331249462771\\
91.41	0.00903504896825373\\
91.42	0.00903697384530865\\
91.43	0.00903889957790953\\
91.44	0.00904082616652405\\
91.45	0.00904275361161994\\
91.46	0.00904468191366493\\
91.47	0.00904661107312673\\
91.48	0.00904854109047312\\
91.49	0.00905047196617182\\
91.5	0.00905240370069058\\
91.51	0.00905433629449715\\
91.52	0.00905626974805926\\
91.53	0.00905820406184461\\
91.54	0.00906013923632092\\
91.55	0.00906207527195584\\
91.56	0.00906401216921704\\
91.57	0.00906594992857213\\
91.58	0.00906788855048868\\
91.59	0.00906982803543424\\
91.6	0.00907176838387629\\
91.61	0.00907370959628228\\
91.62	0.00907565167311959\\
91.63	0.00907759461485555\\
91.64	0.00907953842195741\\
91.65	0.00908148309489236\\
91.66	0.00908342863412751\\
91.67	0.00908537504012988\\
91.68	0.00908732231336642\\
91.69	0.00908927045430398\\
91.7	0.0090912194634093\\
91.71	0.00909316934114903\\
91.72	0.0090951200879897\\
91.73	0.00909707170439773\\
91.74	0.00909902419083941\\
91.75	0.0091009775477809\\
91.76	0.00910293177568824\\
91.77	0.00910488687502733\\
91.78	0.00910684284626389\\
91.79	0.00910879968986351\\
91.8	0.00911075740629163\\
91.81	0.0091127159960135\\
91.82	0.0091146754594942\\
91.83	0.00911663579719862\\
91.84	0.0091185970095915\\
91.85	0.00912055909713732\\
91.86	0.00912252206030041\\
91.87	0.00912448589954485\\
91.88	0.00912645061533453\\
91.89	0.0091284162081331\\
91.9	0.00913038267840396\\
91.91	0.00913235002661028\\
91.92	0.00913431825321499\\
91.93	0.00913628735868072\\
91.94	0.00913825734346987\\
91.95	0.00914022820804454\\
91.96	0.00914219995286654\\
91.97	0.00914417257839739\\
91.98	0.00914614608509831\\
91.99	0.00914812047343018\\
92	0.00915009574385358\\
92.01	0.00915207189682873\\
92.02	0.00915404893281552\\
92.03	0.00915602685227347\\
92.04	0.00915800565566174\\
92.05	0.0091599853434391\\
92.06	0.00916196591606395\\
92.07	0.00916394737399427\\
92.08	0.00916592971768763\\
92.09	0.00916791294760117\\
92.1	0.00916989706419161\\
92.11	0.0091718820679152\\
92.12	0.00917386795922772\\
92.13	0.00917585473858451\\
92.14	0.00917784240644038\\
92.15	0.00917983096324966\\
92.16	0.00918182040946616\\
92.17	0.00918381074554313\\
92.18	0.00918580197193333\\
92.19	0.0091877940890889\\
92.2	0.00918978709746143\\
92.21	0.00919178099750192\\
92.22	0.00919377578966075\\
92.23	0.00919577147438767\\
92.24	0.00919776805213181\\
92.25	0.0091997655233416\\
92.26	0.00920176388846483\\
92.27	0.00920376314794858\\
92.28	0.00920576330223919\\
92.29	0.0092077643517823\\
92.3	0.00920976629702277\\
92.31	0.00921176913840468\\
92.32	0.00921377287637134\\
92.33	0.00921577751136521\\
92.34	0.00921778304382792\\
92.35	0.00921978947420024\\
92.36	0.00922179680292203\\
92.37	0.00922380503043228\\
92.38	0.00922581415716898\\
92.39	0.00922782418356922\\
92.4	0.00922983511006904\\
92.41	0.00923184693710351\\
92.42	0.00923385966510662\\
92.43	0.0092358732945113\\
92.44	0.00923788782574938\\
92.45	0.00923990325925152\\
92.46	0.00924191959544725\\
92.47	0.00924393683476488\\
92.48	0.00924595497763148\\
92.49	0.00924797402447285\\
92.5	0.0092499939757135\\
92.51	0.00925201483177656\\
92.52	0.00925403659308381\\
92.53	0.00925605926005558\\
92.54	0.00925808283311075\\
92.55	0.00926010731266668\\
92.56	0.00926213269913919\\
92.57	0.00926415899294249\\
92.58	0.00926618619448914\\
92.59	0.00926821430419001\\
92.6	0.00927024332245423\\
92.61	0.00927227324968911\\
92.62	0.00927430408630013\\
92.63	0.00927633583269085\\
92.64	0.00927836848926284\\
92.65	0.00928040205641568\\
92.66	0.00928243653454681\\
92.67	0.00928447192405156\\
92.68	0.009286508225323\\
92.69	0.00928854543875191\\
92.7	0.0092905835647267\\
92.71	0.00929262260363335\\
92.72	0.0092946625558553\\
92.73	0.00929670342177339\\
92.74	0.00929874520176577\\
92.75	0.00930078789620781\\
92.76	0.00930283150547201\\
92.77	0.00930487602992793\\
92.78	0.00930692146994204\\
92.79	0.00930896782587765\\
92.8	0.00931101509809481\\
92.81	0.00931306328695018\\
92.82	0.00931511239279693\\
92.83	0.00931716241598462\\
92.84	0.00931921335685906\\
92.85	0.0093212652157622\\
92.86	0.00932331799303198\\
92.87	0.00932537168900221\\
92.88	0.0093274263040024\\
92.89	0.00932948183835763\\
92.9	0.00933153829238837\\
92.91	0.00933359566641037\\
92.92	0.0093356539607344\\
92.93	0.00933771317566615\\
92.94	0.00933977331150602\\
92.95	0.00934183436854892\\
92.96	0.00934389634708409\\
92.97	0.00934595924739485\\
92.98	0.00934802306975844\\
92.99	0.00935008781444575\\
93	0.00935215348172111\\
93.01	0.00935422007184203\\
93.02	0.00935628758505895\\
93.03	0.00935835602161498\\
93.04	0.00936042538174561\\
93.05	0.00936249566567844\\
93.06	0.00936456687363287\\
93.07	0.00936663900581977\\
93.08	0.00936871206244118\\
93.09	0.00937078604368996\\
93.1	0.00937286094974944\\
93.11	0.00937493678079303\\
93.12	0.00937701353698386\\
93.13	0.00937909121847438\\
93.14	0.00938116982540592\\
93.15	0.00938324935790826\\
93.16	0.00938532981609921\\
93.17	0.00938741120008407\\
93.18	0.0093894935099552\\
93.19	0.00939157674579147\\
93.2	0.00939366090765776\\
93.21	0.00939574599560433\\
93.22	0.00939783200966631\\
93.23	0.00939991894986307\\
93.24	0.00940200681619755\\
93.25	0.00940409560865568\\
93.26	0.00940618532721096\\
93.27	0.00940827597182494\\
93.28	0.00941036754244687\\
93.29	0.00941246003901342\\
93.3	0.0094145534614484\\
93.31	0.00941664780966238\\
93.32	0.00941874308355246\\
93.33	0.00942083928300185\\
93.34	0.00942293640787958\\
93.35	0.00942503445804016\\
93.36	0.00942713343332319\\
93.37	0.009429233333553\\
93.38	0.00943133415853831\\
93.39	0.0094334359080718\\
93.4	0.00943553858192976\\
93.41	0.00943764217987163\\
93.42	0.00943974670163962\\
93.43	0.00944185214695828\\
93.44	0.00944395851553408\\
93.45	0.0094460658070549\\
93.46	0.00944817402118964\\
93.47	0.00945028315758769\\
93.48	0.00945239321587849\\
93.49	0.00945450419567099\\
93.5	0.00945661609655316\\
93.51	0.00945872891809147\\
93.52	0.00946084265983034\\
93.53	0.00946295732129159\\
93.54	0.00946507290197385\\
93.55	0.00946718940135202\\
93.56	0.00946930681887665\\
93.57	0.00947142515397328\\
93.58	0.0094735444060419\\
93.59	0.00947566457445622\\
93.6	0.00947778565856304\\
93.61	0.00947990765768156\\
93.62	0.00948203057110268\\
93.63	0.00948415439808829\\
93.64	0.00948627913787048\\
93.65	0.00948840478965085\\
93.66	0.00949053135259967\\
93.67	0.00949265882585511\\
93.68	0.00949478720852241\\
93.69	0.00949691649967302\\
93.7	0.00949904669834373\\
93.71	0.00950117780353581\\
93.72	0.00950330981421404\\
93.73	0.00950544272930582\\
93.74	0.00950757654770014\\
93.75	0.00950971126824664\\
93.76	0.00951184688975457\\
93.77	0.00951398341099173\\
93.78	0.00951612083068338\\
93.79	0.00951825914751115\\
93.8	0.00952039836011191\\
93.81	0.00952253846707655\\
93.82	0.00952467946694883\\
93.83	0.00952682135822413\\
93.84	0.00952896413934811\\
93.85	0.00953110780871553\\
93.86	0.00953325236466878\\
93.87	0.00953539780549657\\
93.88	0.00953754412943249\\
93.89	0.00953969133465357\\
93.9	0.00954183941927873\\
93.91	0.00954398838136728\\
93.92	0.00954613821891733\\
93.93	0.00954828892986415\\
93.94	0.00955044051207848\\
93.95	0.00955259296336482\\
93.96	0.00955474628145968\\
93.97	0.00955690046402971\\
93.98	0.00955905550866988\\
93.99	0.00956121141290149\\
94	0.00956336817417023\\
94.01	0.00956552578984417\\
94.02	0.00956768425721157\\
94.03	0.00956984357347883\\
94.04	0.0095720037357682\\
94.05	0.00957416474111552\\
94.06	0.00957632658646791\\
94.07	0.00957848926868131\\
94.08	0.00958065278451799\\
94.09	0.00958281713064407\\
94.1	0.00958498230362682\\
94.11	0.00958714829993199\\
94.12	0.00958931511592103\\
94.13	0.00959148274784822\\
94.14	0.00959365119185774\\
94.15	0.00959582044398067\\
94.16	0.0095979905001318\\
94.17	0.00960016135610651\\
94.18	0.00960233300757742\\
94.19	0.00960450545009101\\
94.2	0.00960667867906417\\
94.21	0.00960885268978054\\
94.22	0.00961102747738685\\
94.23	0.00961320303688914\\
94.24	0.00961537936314879\\
94.25	0.00961755645087854\\
94.26	0.00961973429463832\\
94.27	0.00962191288883099\\
94.28	0.00962409222769793\\
94.29	0.00962627230531455\\
94.3	0.00962845311558563\\
94.31	0.00963063465224048\\
94.32	0.0096328169088281\\
94.33	0.00963499987871202\\
94.34	0.00963718355506511\\
94.35	0.0096393679308642\\
94.36	0.00964155299888456\\
94.37	0.00964373875169415\\
94.38	0.0096459251816478\\
94.39	0.00964811228088112\\
94.4	0.00965030004130433\\
94.41	0.00965248845459579\\
94.42	0.00965467751219547\\
94.43	0.00965686720529811\\
94.44	0.00965905752484621\\
94.45	0.00966124846152291\\
94.46	0.00966344000574446\\
94.47	0.00966563214765266\\
94.48	0.00966782487710699\\
94.49	0.00967001818367644\\
94.5	0.00967221205663125\\
94.51	0.00967440648493425\\
94.52	0.00967660145723207\\
94.53	0.00967879696184596\\
94.54	0.00968099298676245\\
94.55	0.00968318951962364\\
94.56	0.00968538654771726\\
94.57	0.00968758405796641\\
94.58	0.00968978203691899\\
94.59	0.00969198047073677\\
94.6	0.00969417934518424\\
94.61	0.00969637864561698\\
94.62	0.00969857835696983\\
94.63	0.00970077846374455\\
94.64	0.00970297894999724\\
94.65	0.0097051797993253\\
94.66	0.00970738099485401\\
94.67	0.00970958251922271\\
94.68	0.00971178435457059\\
94.69	0.00971398648252196\\
94.7	0.0097161888841712\\
94.71	0.00971839154006714\\
94.72	0.00972059443019699\\
94.73	0.00972279753396986\\
94.74	0.00972500083019966\\
94.75	0.00972720429708757\\
94.76	0.00972940791220391\\
94.77	0.0097316116524695\\
94.78	0.00973381549413645\\
94.79	0.0097360194127683\\
94.8	0.00973822338321966\\
94.81	0.00974042737961512\\
94.82	0.00974263137532758\\
94.83	0.00974483534295588\\
94.84	0.00974703925430178\\
94.85	0.00974924308034618\\
94.86	0.00975144679122465\\
94.87	0.00975365035620222\\
94.88	0.00975585374364733\\
94.89	0.00975805692100504\\
94.9	0.00976025985476939\\
94.91	0.00976246251045491\\
94.92	0.00976466485256723\\
94.93	0.00976686684457283\\
94.94	0.0097690684488678\\
94.95	0.00977126962674569\\
94.96	0.00977347033836431\\
94.97	0.00977567054271154\\
94.98	0.00977787019757004\\
94.99	0.00978006925948098\\
95	0.00978226768370643\\
95.01	0.00978446542419082\\
95.02	0.00978666243352106\\
95.03	0.00978885866288544\\
95.04	0.00979105406203127\\
95.05	0.00979324857922124\\
95.06	0.00979544216119451\\
95.07	0.00979763475312522\\
95.08	0.00979982629857534\\
95.09	0.00980201673944598\\
95.1	0.0098042060159273\\
95.11	0.0098063940664468\\
95.12	0.009808580827616\\
95.13	0.00981076623417556\\
95.14	0.00981295021893856\\
95.15	0.00981513271273214\\
95.16	0.0098173136443373\\
95.17	0.00981949294042679\\
95.18	0.00982167052550111\\
95.19	0.00982384632182251\\
95.2	0.00982602024934697\\
95.21	0.00982819222565397\\
95.22	0.00983036216587423\\
95.23	0.00983252998261504\\
95.24	0.00983469558588341\\
95.25	0.00983685888300667\\
95.26	0.00983901977855079\\
95.27	0.00984117817423594\\
95.28	0.00984333396884961\\
95.29	0.0098454870581569\\
95.3	0.00984763733480804\\
95.31	0.00984978468824303\\
95.32	0.00985192900459333\\
95.33	0.00985407016658042\\
95.34	0.00985620805341122\\
95.35	0.00985834254067026\\
95.36	0.00986047350020845\\
95.37	0.00986260080002838\\
95.38	0.00986472430416601\\
95.39	0.00986684387256871\\
95.4	0.00986895936096943\\
95.41	0.00987107062075693\\
95.42	0.009873177498842\\
95.43	0.00987527983751939\\
95.44	0.00987737747432551\\
95.45	0.00987947024189163\\
95.46	0.00988155796779243\\
95.47	0.00988364047438985\\
95.48	0.00988571757867207\\
95.49	0.00988778909208732\\
95.5	0.00988985482037259\\
95.51	0.0098919145633769\\
95.52	0.00989396811487902\\
95.53	0.0098960152623995\\
95.54	0.00989805578700672\\
95.55	0.00990008946311692\\
95.56	0.00990211605828789\\
95.57	0.00990413533300623\\
95.58	0.0099061470404678\\
95.59	0.00990815092635141\\
95.6	0.00991014672858531\\
95.61	0.00991213417710632\\
95.62	0.00991411299361142\\
95.63	0.00991608289130149\\
95.64	0.00991804357461702\\
95.65	0.00991999473896542\\
95.66	0.00992193607043988\\
95.67	0.0099238672455292\\
95.68	0.00992578793081867\\
95.69	0.00992769778268137\\
95.7	0.00992959644695984\\
95.71	0.00993148355863768\\
95.72	0.00993335874150087\\
95.73	0.00993522160778832\\
95.74	0.00993707175783149\\
95.75	0.00993890877968268\\
95.76	0.0099407322487315\\
95.77	0.00994254172730939\\
95.78	0.00994433676428161\\
95.79	0.00994611689462638\\
95.8	0.00994788163900083\\
95.81	0.00994963050329322\\
95.82	0.0099513629781611\\
95.83	0.00995307853855489\\
95.84	0.00995477664322653\\
95.85	0.00995645673422267\\
95.86	0.00995811823636179\\
95.87	0.00995976055669505\\
95.88	0.00996138308395009\\
95.89	0.00996298518795734\\
95.9	0.00996456621905835\\
95.91	0.00996612550749545\\
95.92	0.00996766236278238\\
95.93	0.00996917607305495\\
95.94	0.00997066590440153\\
95.95	0.00997213110017244\\
95.96	0.0099735708802676\\
95.97	0.009974984440402\\
95.98	0.00997637095134808\\
95.99	0.00997772955815433\\
96	0.00997905937933954\\
96.01	0.00998035950606173\\
96.02	0.00998162900126117\\
96.03	0.0099828668987766\\
96.04	0.00998407220243377\\
96.05	0.00998524388510568\\
96.06	0.00998638088774336\\
96.07	0.00998748211837652\\
96.08	0.00998854645108307\\
96.09	0.00998957272492655\\
96.1	0.00999055974286051\\
96.11	0.00999150627059886\\
96.12	0.00999241103545119\\
96.13	0.00999327272512193\\
96.14	0.00999408998647227\\
96.15	0.00999486142424387\\
96.16	0.00999558559974299\\
96.17	0.00999626102948407\\
96.18	0.00999688618379141\\
96.19	0.00999745948535783\\
96.2	0.00999797930775888\\
96.21	0.00999844397392151\\
96.22	0.00999885175454561\\
96.23	0.00999920086647735\\
96.24	0.00999948947103253\\
96.25	0.00999971567226888\\
96.26	0.00999987751520559\\
96.27	0.00999997298398856\\
96.28	0.01\\
96.29	0.01\\
96.3	0.01\\
96.31	0.01\\
96.32	0.01\\
96.33	0.01\\
96.34	0.01\\
96.35	0.01\\
96.36	0.01\\
96.37	0.01\\
96.38	0.01\\
96.39	0.01\\
96.4	0.01\\
96.41	0.01\\
96.42	0.01\\
96.43	0.01\\
96.44	0.01\\
96.45	0.01\\
96.46	0.01\\
96.47	0.01\\
96.48	0.01\\
96.49	0.01\\
96.5	0.01\\
96.51	0.01\\
96.52	0.01\\
96.53	0.01\\
96.54	0.01\\
96.55	0.01\\
96.56	0.01\\
96.57	0.01\\
96.58	0.01\\
96.59	0.01\\
96.6	0.01\\
96.61	0.01\\
96.62	0.01\\
96.63	0.01\\
96.64	0.01\\
96.65	0.01\\
96.66	0.01\\
96.67	0.01\\
96.68	0.01\\
96.69	0.01\\
96.7	0.01\\
96.71	0.01\\
96.72	0.01\\
96.73	0.01\\
96.74	0.01\\
96.75	0.01\\
96.76	0.01\\
96.77	0.01\\
96.78	0.01\\
96.79	0.01\\
96.8	0.01\\
96.81	0.01\\
96.82	0.01\\
96.83	0.01\\
96.84	0.01\\
96.85	0.01\\
96.86	0.01\\
96.87	0.01\\
96.88	0.01\\
96.89	0.01\\
96.9	0.01\\
96.91	0.01\\
96.92	0.01\\
96.93	0.01\\
96.94	0.01\\
96.95	0.01\\
96.96	0.01\\
96.97	0.01\\
96.98	0.01\\
96.99	0.01\\
97	0.01\\
97.01	0.01\\
97.02	0.01\\
97.03	0.01\\
97.04	0.01\\
97.05	0.01\\
97.06	0.01\\
97.07	0.01\\
97.08	0.01\\
97.09	0.01\\
97.1	0.01\\
97.11	0.01\\
97.12	0.01\\
97.13	0.01\\
97.14	0.01\\
97.15	0.01\\
97.16	0.01\\
97.17	0.01\\
97.18	0.01\\
97.19	0.01\\
97.2	0.01\\
97.21	0.01\\
97.22	0.01\\
97.23	0.01\\
97.24	0.01\\
97.25	0.01\\
97.26	0.01\\
97.27	0.01\\
97.28	0.01\\
97.29	0.01\\
97.3	0.01\\
97.31	0.01\\
97.32	0.01\\
97.33	0.01\\
97.34	0.01\\
97.35	0.01\\
97.36	0.01\\
97.37	0.01\\
97.38	0.01\\
97.39	0.01\\
97.4	0.01\\
97.41	0.01\\
97.42	0.01\\
97.43	0.01\\
97.44	0.01\\
97.45	0.01\\
97.46	0.01\\
97.47	0.01\\
97.48	0.01\\
97.49	0.01\\
97.5	0.01\\
97.51	0.01\\
97.52	0.01\\
97.53	0.01\\
97.54	0.01\\
97.55	0.01\\
97.56	0.01\\
97.57	0.01\\
97.58	0.01\\
97.59	0.01\\
97.6	0.01\\
97.61	0.01\\
97.62	0.01\\
97.63	0.01\\
97.64	0.01\\
97.65	0.01\\
97.66	0.01\\
97.67	0.01\\
97.68	0.01\\
97.69	0.01\\
97.7	0.01\\
97.71	0.01\\
97.72	0.01\\
97.73	0.01\\
97.74	0.01\\
97.75	0.01\\
97.76	0.01\\
97.77	0.01\\
97.78	0.01\\
97.79	0.01\\
97.8	0.01\\
97.81	0.01\\
97.82	0.01\\
97.83	0.01\\
97.84	0.01\\
97.85	0.01\\
97.86	0.01\\
97.87	0.01\\
97.88	0.01\\
97.89	0.01\\
97.9	0.01\\
97.91	0.01\\
97.92	0.01\\
97.93	0.01\\
97.94	0.01\\
97.95	0.01\\
97.96	0.01\\
97.97	0.01\\
97.98	0.01\\
97.99	0.01\\
98	0.01\\
98.01	0.01\\
98.02	0.01\\
98.03	0.01\\
98.04	0.01\\
98.05	0.01\\
98.06	0.01\\
98.07	0.01\\
98.08	0.01\\
98.09	0.01\\
98.1	0.01\\
98.11	0.01\\
98.12	0.01\\
98.13	0.01\\
98.14	0.01\\
98.15	0.01\\
98.16	0.01\\
98.17	0.01\\
98.18	0.01\\
98.19	0.01\\
98.2	0.01\\
98.21	0.01\\
98.22	0.01\\
98.23	0.01\\
98.24	0.01\\
98.25	0.01\\
98.26	0.01\\
98.27	0.01\\
98.28	0.01\\
98.29	0.01\\
98.3	0.01\\
98.31	0.01\\
98.32	0.01\\
98.33	0.01\\
98.34	0.01\\
98.35	0.01\\
98.36	0.01\\
98.37	0.01\\
98.38	0.01\\
98.39	0.01\\
98.4	0.01\\
98.41	0.01\\
98.42	0.01\\
98.43	0.01\\
98.44	0.01\\
98.45	0.01\\
98.46	0.01\\
98.47	0.01\\
98.48	0.01\\
98.49	0.01\\
98.5	0.01\\
98.51	0.01\\
98.52	0.01\\
98.53	0.01\\
98.54	0.01\\
98.55	0.01\\
98.56	0.01\\
98.57	0.01\\
98.58	0.01\\
98.59	0.01\\
98.6	0.01\\
98.61	0.01\\
98.62	0.01\\
98.63	0.01\\
98.64	0.01\\
98.65	0.01\\
98.66	0.01\\
98.67	0.01\\
98.68	0.01\\
98.69	0.01\\
98.7	0.01\\
98.71	0.01\\
98.72	0.01\\
98.73	0.01\\
98.74	0.01\\
98.75	0.01\\
98.76	0.01\\
98.77	0.01\\
98.78	0.01\\
98.79	0.01\\
98.8	0.01\\
98.81	0.01\\
98.82	0.01\\
98.83	0.01\\
98.84	0.01\\
98.85	0.01\\
98.86	0.01\\
98.87	0.01\\
98.88	0.01\\
98.89	0.01\\
98.9	0.01\\
98.91	0.01\\
98.92	0.01\\
98.93	0.01\\
98.94	0.01\\
98.95	0.01\\
98.96	0.01\\
98.97	0.01\\
98.98	0.01\\
98.99	0.01\\
99	0.01\\
99.01	0.01\\
99.02	0.01\\
99.03	0.01\\
99.04	0.01\\
99.05	0.01\\
99.06	0.01\\
99.07	0.01\\
99.08	0.01\\
99.09	0.01\\
99.1	0.01\\
99.11	0.01\\
99.12	0.01\\
99.13	0.01\\
99.14	0.01\\
99.15	0.01\\
99.16	0.01\\
99.17	0.01\\
99.18	0.01\\
99.19	0.01\\
99.2	0.01\\
99.21	0.01\\
99.22	0.01\\
99.23	0.01\\
99.24	0.01\\
99.25	0.01\\
99.26	0.01\\
99.27	0.01\\
99.28	0.01\\
99.29	0.01\\
99.3	0.01\\
99.31	0.01\\
99.32	0.01\\
99.33	0.01\\
99.34	0.01\\
99.35	0.01\\
99.36	0.01\\
99.37	0.01\\
99.38	0.01\\
99.39	0.01\\
99.4	0.01\\
99.41	0.01\\
99.42	0.01\\
99.43	0.01\\
99.44	0.01\\
99.45	0.01\\
99.46	0.01\\
99.47	0.01\\
99.48	0.01\\
99.49	0.01\\
99.5	0.01\\
99.51	0.01\\
99.52	0.01\\
99.53	0.01\\
99.54	0.01\\
99.55	0.01\\
99.56	0.01\\
99.57	0.01\\
99.58	0.01\\
99.59	0.01\\
99.6	0.01\\
99.61	0.01\\
99.62	0.01\\
99.63	0.01\\
99.64	0.01\\
99.65	0.01\\
99.66	0.01\\
99.67	0.01\\
99.68	0.01\\
99.69	0.01\\
99.7	0.01\\
99.71	0.01\\
99.72	0.01\\
99.73	0.01\\
99.74	0.01\\
99.75	0.01\\
99.76	0.01\\
99.77	0.01\\
99.78	0.01\\
99.79	0.01\\
99.8	0.01\\
99.81	0.01\\
99.82	0.01\\
99.83	0.01\\
99.84	0.01\\
99.85	0.01\\
99.86	0.01\\
99.87	0.01\\
99.88	0.01\\
99.89	0.01\\
99.9	0.01\\
99.91	0.01\\
99.92	0.01\\
99.93	0.01\\
99.94	0.01\\
99.95	0.01\\
99.96	0.01\\
99.97	0.01\\
99.98	0.01\\
99.99	0.01\\
100	0.01\\
};
\addlegendentry{$q=1$};

\addplot [color=red,solid,forget plot]
  table[row sep=crcr]{%
0.01	0.00819891155222631\\
0.02	0.00819892028165661\\
0.03	0.00819892904835735\\
0.04	0.00819893785313367\\
0.05	0.00819894669681198\\
0.06	0.00819895558024053\\
0.07	0.00819896450429004\\
0.08	0.00819897346985436\\
0.09	0.00819898247785111\\
0.1	0.00819899152922234\\
0.11	0.00819900062493528\\
0.12	0.00819900976598298\\
0.13	0.00819901895338512\\
0.14	0.00819902818818874\\
0.15	0.008199037471469\\
0.16	0.00819904680433005\\
0.17	0.00819905618790579\\
0.18	0.00819906562336078\\
0.19	0.00819907511189112\\
0.2	0.00819908465472531\\
0.21	0.00819909425312523\\
0.22	0.00819910390838711\\
0.23	0.00819911362184249\\
0.24	0.00819912339485927\\
0.25	0.00819913322884277\\
0.26	0.00819914312523678\\
0.27	0.00819915308552473\\
0.28	0.00819916311123079\\
0.29	0.0081991732039211\\
0.3	0.00819918336520501\\
0.31	0.00819919359673626\\
0.32	0.00819920390021439\\
0.33	0.00819921427738599\\
0.34	0.00819922473004613\\
0.35	0.00819923526003979\\
0.36	0.00819924586926327\\
0.37	0.00819925655966577\\
0.38	0.00819926733325091\\
0.39	0.00819927819207835\\
0.4	0.00819928913826543\\
0.41	0.00819930017398889\\
0.42	0.00819931130148663\\
0.43	0.0081993225230595\\
0.44	0.00819933384107321\\
0.45	0.0081993452579602\\
0.46	0.00819935677622167\\
0.47	0.00819936839842962\\
0.48	0.00819938012722892\\
0.49	0.00819939196533953\\
0.5	0.00819940391555873\\
0.51	0.0081994159807634\\
0.52	0.00819942816391246\\
0.53	0.00819944046804924\\
0.54	0.00819945289630409\\
0.55	0.00819946545189698\\
0.56	0.00819947813814011\\
0.57	0.00819949095844079\\
0.58	0.00819950391630424\\
0.59	0.00819951701533655\\
0.6	0.0081995302592477\\
0.61	0.00819954365185475\\
0.62	0.00819955719708499\\
0.63	0.00819957089897935\\
0.64	0.00819958476169579\\
0.65	0.00819959878951284\\
0.66	0.00819961298683328\\
0.67	0.0081996273581879\\
0.68	0.00819964190823934\\
0.69	0.00819965664178616\\
0.7	0.00819967156376693\\
0.71	0.0081996866792645\\
0.72	0.00819970199351038\\
0.73	0.00819971751188928\\
0.74	0.00819973314568682\\
0.75	0.00819974878548379\\
0.76	0.00819976443128269\\
0.77	0.00819978008308601\\
0.78	0.00819979574089625\\
0.79	0.00819981140471589\\
0.8	0.00819982707454744\\
0.81	0.00819984275039339\\
0.82	0.00819985843225624\\
0.83	0.00819987412013849\\
0.84	0.00819988981404265\\
0.85	0.00819990551397121\\
0.86	0.00819992121992667\\
0.87	0.00819993693191155\\
0.88	0.00819995264992835\\
0.89	0.00819996837397957\\
0.9	0.00819998410406773\\
0.91	0.00819999984019533\\
0.92	0.00820001558236488\\
0.93	0.0082000313305789\\
0.94	0.0082000470848399\\
0.95	0.00820006284515039\\
0.96	0.00820007861151289\\
0.97	0.00820009438392992\\
0.98	0.00820011016240399\\
0.99	0.00820012594693762\\
1	0.00820014173753334\\
1.01	0.00820015753419366\\
1.02	0.00820017333692111\\
1.03	0.00820018914571821\\
1.04	0.00820020496058748\\
1.05	0.00820022078153146\\
1.06	0.00820023660855267\\
1.07	0.00820025244165365\\
1.08	0.00820026828083691\\
1.09	0.008200284126105\\
1.1	0.00820029997746043\\
1.11	0.00820031583490576\\
1.12	0.00820033169844351\\
1.13	0.00820034756807622\\
1.14	0.00820036344380643\\
1.15	0.00820037932563668\\
1.16	0.0082003952135695\\
1.17	0.00820041110760744\\
1.18	0.00820042700775304\\
1.19	0.00820044291400883\\
1.2	0.00820045882637738\\
1.21	0.00820047474486123\\
1.22	0.00820049066946291\\
1.23	0.00820050660018499\\
1.24	0.00820052253703\\
1.25	0.00820053848000051\\
1.26	0.00820055442909906\\
1.27	0.00820057038432821\\
1.28	0.00820058634569051\\
1.29	0.00820060231318851\\
1.3	0.00820061828682479\\
1.31	0.00820063426660189\\
1.32	0.00820065025252237\\
1.33	0.00820066624458879\\
1.34	0.00820068224280372\\
1.35	0.00820069824716972\\
1.36	0.00820071425768936\\
1.37	0.0082007302743652\\
1.38	0.0082007462971998\\
1.39	0.00820076232619574\\
1.4	0.00820077836135559\\
1.41	0.00820079440268192\\
1.42	0.00820081045017729\\
1.43	0.00820082650384429\\
1.44	0.00820084256368549\\
1.45	0.00820085862970345\\
1.46	0.00820087470190077\\
1.47	0.00820089078028002\\
1.48	0.00820090686484378\\
1.49	0.00820092295559463\\
1.5	0.00820093905253514\\
1.51	0.00820095515566791\\
1.52	0.00820097126499553\\
1.53	0.00820098738052057\\
1.54	0.00820100350224562\\
1.55	0.00820101963017328\\
1.56	0.00820103576430612\\
1.57	0.00820105190464675\\
1.58	0.00820106805119776\\
1.59	0.00820108420396173\\
1.6	0.00820110036294126\\
1.61	0.00820111652813896\\
1.62	0.00820113269955741\\
1.63	0.00820114887719921\\
1.64	0.00820116506106697\\
1.65	0.00820118125116329\\
1.66	0.00820119744749077\\
1.67	0.008201213650052\\
1.68	0.0082012298588496\\
1.69	0.00820124607388618\\
1.7	0.00820126229516434\\
1.71	0.00820127852268668\\
1.72	0.00820129475645583\\
1.73	0.00820131099647438\\
1.74	0.00820132724274496\\
1.75	0.00820134349527018\\
1.76	0.00820135975405265\\
1.77	0.00820137601909498\\
1.78	0.0082013922903998\\
1.79	0.00820140856796973\\
1.8	0.00820142485180738\\
1.81	0.00820144114191537\\
1.82	0.00820145743829634\\
1.83	0.0082014737409529\\
1.84	0.00820149004988768\\
1.85	0.0082015063651033\\
1.86	0.0082015226866024\\
1.87	0.0082015390143876\\
1.88	0.00820155534846153\\
1.89	0.00820157168882682\\
1.9	0.00820158803548612\\
1.91	0.00820160438844204\\
1.92	0.00820162074769723\\
1.93	0.00820163711325432\\
1.94	0.00820165348511596\\
1.95	0.00820166986328478\\
1.96	0.00820168624776342\\
1.97	0.00820170263855452\\
1.98	0.00820171903566073\\
1.99	0.0082017354390847\\
2	0.00820175184882906\\
2.01	0.00820176826489647\\
2.02	0.00820178468728957\\
2.03	0.00820180111601102\\
2.04	0.00820181755106346\\
2.05	0.00820183399244954\\
2.06	0.00820185044017193\\
2.07	0.00820186689423327\\
2.08	0.00820188335463622\\
2.09	0.00820189982138345\\
2.1	0.0082019162944776\\
2.11	0.00820193277392134\\
2.12	0.00820194925971733\\
2.13	0.00820196575186823\\
2.14	0.00820198225037671\\
2.15	0.00820199875524543\\
2.16	0.00820201526647707\\
2.17	0.00820203178407428\\
2.18	0.00820204830803973\\
2.19	0.0082020648383761\\
2.2	0.00820208137508606\\
2.21	0.00820209791817228\\
2.22	0.00820211446763744\\
2.23	0.00820213102348421\\
2.24	0.00820214758571527\\
2.25	0.0082021641543333\\
2.26	0.00820218072934097\\
2.27	0.00820219731074098\\
2.28	0.00820221389853599\\
2.29	0.0082022304927287\\
2.3	0.00820224709332179\\
2.31	0.00820226370031795\\
2.32	0.00820228031371986\\
2.33	0.00820229693353022\\
2.34	0.0082023135597517\\
2.35	0.00820233019238702\\
2.36	0.00820234683143885\\
2.37	0.0082023634769099\\
2.38	0.00820238012880285\\
2.39	0.00820239678712041\\
2.4	0.00820241345186527\\
2.41	0.00820243012304013\\
2.42	0.0082024468006477\\
2.43	0.00820246348469067\\
2.44	0.00820248017517175\\
2.45	0.00820249687209365\\
2.46	0.00820251357545907\\
2.47	0.00820253028527071\\
2.48	0.0082025470015313\\
2.49	0.00820256372424353\\
2.5	0.00820258045341012\\
2.51	0.00820259718903378\\
2.52	0.00820261393111723\\
2.53	0.00820263067966318\\
2.54	0.00820264743467435\\
2.55	0.00820266419615346\\
2.56	0.00820268096410322\\
2.57	0.00820269773852636\\
2.58	0.0082027145194256\\
2.59	0.00820273130680366\\
2.6	0.00820274810066327\\
2.61	0.00820276490100716\\
2.62	0.00820278170783805\\
2.63	0.00820279852115867\\
2.64	0.00820281534097175\\
2.65	0.00820283216728002\\
2.66	0.00820284900008622\\
2.67	0.00820286583939309\\
2.68	0.00820288268520335\\
2.69	0.00820289953751974\\
2.7	0.008202916396345\\
2.71	0.00820293326168187\\
2.72	0.0082029501335331\\
2.73	0.00820296701190142\\
2.74	0.00820298389678958\\
2.75	0.00820300078820033\\
2.76	0.00820301768613641\\
2.77	0.00820303459060056\\
2.78	0.00820305150159555\\
2.79	0.00820306841912411\\
2.8	0.008203085343189\\
2.81	0.00820310227379297\\
2.82	0.00820311921093879\\
2.83	0.0082031361546292\\
2.84	0.00820315310486695\\
2.85	0.00820317006165482\\
2.86	0.00820318702499556\\
2.87	0.00820320399489194\\
2.88	0.00820322097134671\\
2.89	0.00820323795436263\\
2.9	0.00820325494394249\\
2.91	0.00820327194008903\\
2.92	0.00820328894280504\\
2.93	0.00820330595209327\\
2.94	0.00820332296795651\\
2.95	0.00820333999039751\\
2.96	0.00820335701941907\\
2.97	0.00820337405502395\\
2.98	0.00820339109721492\\
2.99	0.00820340814599478\\
3	0.00820342520136629\\
3.01	0.00820344226333224\\
3.02	0.00820345933189541\\
3.03	0.00820347640705857\\
3.04	0.00820349348882453\\
3.05	0.00820351057719606\\
3.06	0.00820352767217595\\
3.07	0.00820354477376699\\
3.08	0.00820356188197197\\
3.09	0.00820357899679368\\
3.1	0.00820359611823492\\
3.11	0.00820361324629847\\
3.12	0.00820363038098715\\
3.13	0.00820364752230373\\
3.14	0.00820366467025103\\
3.15	0.00820368182483183\\
3.16	0.00820369898604895\\
3.17	0.00820371615390519\\
3.18	0.00820373332840334\\
3.19	0.00820375050954622\\
3.2	0.00820376769733664\\
3.21	0.00820378489177739\\
3.22	0.0082038020928713\\
3.23	0.00820381930062117\\
3.24	0.00820383651502981\\
3.25	0.00820385373610004\\
3.26	0.00820387096383467\\
3.27	0.00820388819823653\\
3.28	0.00820390543930842\\
3.29	0.00820392268705317\\
3.3	0.0082039399414736\\
3.31	0.00820395720257253\\
3.32	0.00820397447035279\\
3.33	0.0082039917448172\\
3.34	0.00820400902596858\\
3.35	0.00820402631380978\\
3.36	0.0082040436083436\\
3.37	0.00820406090957289\\
3.38	0.00820407821750048\\
3.39	0.0082040955321292\\
3.4	0.00820411285346189\\
3.41	0.00820413018150138\\
3.42	0.0082041475162505\\
3.43	0.00820416485771212\\
3.44	0.00820418220588904\\
3.45	0.00820419956078414\\
3.46	0.00820421692240024\\
3.47	0.00820423429074019\\
3.48	0.00820425166580683\\
3.49	0.00820426904760303\\
3.5	0.00820428643613161\\
3.51	0.00820430383139545\\
3.52	0.00820432123339737\\
3.53	0.00820433864214025\\
3.54	0.00820435605762694\\
3.55	0.00820437347986028\\
3.56	0.00820439090884315\\
3.57	0.00820440834457839\\
3.58	0.00820442578706888\\
3.59	0.00820444323631746\\
3.6	0.00820446069232701\\
3.61	0.00820447815510039\\
3.62	0.00820449562464046\\
3.63	0.0082045131009501\\
3.64	0.00820453058403217\\
3.65	0.00820454807388954\\
3.66	0.00820456557052509\\
3.67	0.00820458307394168\\
3.68	0.00820460058414221\\
3.69	0.00820461810112952\\
3.7	0.00820463562490652\\
3.71	0.00820465315547608\\
3.72	0.00820467069284107\\
3.73	0.00820468823700438\\
3.74	0.00820470578796889\\
3.75	0.0082047233457375\\
3.76	0.00820474091031307\\
3.77	0.00820475848169851\\
3.78	0.0082047760598967\\
3.79	0.00820479364491053\\
3.8	0.0082048112367429\\
3.81	0.00820482883539669\\
3.82	0.00820484644087481\\
3.83	0.00820486405318015\\
3.84	0.0082048816723156\\
3.85	0.00820489929828407\\
3.86	0.00820491693108846\\
3.87	0.00820493457073167\\
3.88	0.0082049522172166\\
3.89	0.00820496987054617\\
3.9	0.00820498753072327\\
3.91	0.00820500519775081\\
3.92	0.0082050228716317\\
3.93	0.00820504055236886\\
3.94	0.0082050582399652\\
3.95	0.00820507593442363\\
3.96	0.00820509363574707\\
3.97	0.00820511134393843\\
3.98	0.00820512905900063\\
3.99	0.00820514678093659\\
4	0.00820516450974924\\
4.01	0.00820518224544149\\
4.02	0.00820519998801627\\
4.03	0.00820521773747651\\
4.04	0.00820523549382513\\
4.05	0.00820525325706506\\
4.06	0.00820527102719923\\
4.07	0.00820528880423058\\
4.08	0.00820530658816204\\
4.09	0.00820532437899653\\
4.1	0.008205342176737\\
4.11	0.00820535998138638\\
4.12	0.00820537779294762\\
4.13	0.00820539561142365\\
4.14	0.00820541343681742\\
4.15	0.00820543126913186\\
4.16	0.00820544910836993\\
4.17	0.00820546695453457\\
4.18	0.00820548480762873\\
4.19	0.00820550266765535\\
4.2	0.00820552053461739\\
4.21	0.0082055384085178\\
4.22	0.00820555628935954\\
4.23	0.00820557417714556\\
4.24	0.00820559207187881\\
4.25	0.00820560997356226\\
4.26	0.00820562788219886\\
4.27	0.00820564579779157\\
4.28	0.00820566372034336\\
4.29	0.00820568164985719\\
4.3	0.00820569958633603\\
4.31	0.00820571752978285\\
4.32	0.00820573548020061\\
4.33	0.00820575343759228\\
4.34	0.00820577140196084\\
4.35	0.00820578937330925\\
4.36	0.0082058073516405\\
4.37	0.00820582533695756\\
4.38	0.0082058433292634\\
4.39	0.00820586132856101\\
4.4	0.00820587933485337\\
4.41	0.00820589734814345\\
4.42	0.00820591536843425\\
4.43	0.00820593339572875\\
4.44	0.00820595143002992\\
4.45	0.00820596947134077\\
4.46	0.00820598751966428\\
4.47	0.00820600557500345\\
4.48	0.00820602363736125\\
4.49	0.00820604170674071\\
4.5	0.00820605978314479\\
4.51	0.00820607786657651\\
4.52	0.00820609595703886\\
4.53	0.00820611405453485\\
4.54	0.00820613215906746\\
4.55	0.00820615027063972\\
4.56	0.00820616838925461\\
4.57	0.00820618651491516\\
4.58	0.00820620464762435\\
4.59	0.00820622278738522\\
4.6	0.00820624093420076\\
4.61	0.00820625908807399\\
4.62	0.00820627724900792\\
4.63	0.00820629541700557\\
4.64	0.00820631359206995\\
4.65	0.00820633177420409\\
4.66	0.008206349963411\\
4.67	0.0082063681596937\\
4.68	0.00820638636305523\\
4.69	0.0082064045734986\\
4.7	0.00820642279102683\\
4.71	0.00820644101564296\\
4.72	0.00820645924735002\\
4.73	0.00820647748615104\\
4.74	0.00820649573204904\\
4.75	0.00820651398504707\\
4.76	0.00820653224514815\\
4.77	0.00820655051235533\\
4.78	0.00820656878667164\\
4.79	0.00820658706810013\\
4.8	0.00820660535664383\\
4.81	0.00820662365230579\\
4.82	0.00820664195508905\\
4.83	0.00820666026499666\\
4.84	0.00820667858203167\\
4.85	0.00820669690619712\\
4.86	0.00820671523749607\\
4.87	0.00820673357593156\\
4.88	0.00820675192150666\\
4.89	0.00820677027422442\\
4.9	0.00820678863408789\\
4.91	0.00820680700110014\\
4.92	0.00820682537526421\\
4.93	0.00820684375658319\\
4.94	0.00820686214506012\\
4.95	0.00820688054069807\\
4.96	0.00820689894350012\\
4.97	0.00820691735346933\\
4.98	0.00820693577060876\\
4.99	0.00820695419492148\\
5	0.00820697262641058\\
5.01	0.00820699106507912\\
5.02	0.00820700951093018\\
5.03	0.00820702796396684\\
5.04	0.00820704642419218\\
5.05	0.00820706489160927\\
5.06	0.0082070833662212\\
5.07	0.00820710184803105\\
5.08	0.0082071203370419\\
5.09	0.00820713883325686\\
5.1	0.00820715733667899\\
5.11	0.0082071758473114\\
5.12	0.00820719436515716\\
5.13	0.00820721289021939\\
5.14	0.00820723142250116\\
5.15	0.00820724996200558\\
5.16	0.00820726850873574\\
5.17	0.00820728706269475\\
5.18	0.0082073056238857\\
5.19	0.00820732419231169\\
5.2	0.00820734276797584\\
5.21	0.00820736135088125\\
5.22	0.00820737994103102\\
5.23	0.00820739853842825\\
5.24	0.00820741714307607\\
5.25	0.00820743575497759\\
5.26	0.00820745437413591\\
5.27	0.00820747300055415\\
5.28	0.00820749163423544\\
5.29	0.00820751027518288\\
5.3	0.0082075289233996\\
5.31	0.00820754757888872\\
5.32	0.00820756624165336\\
5.33	0.00820758491169665\\
5.34	0.0082076035890217\\
5.35	0.00820762227363166\\
5.36	0.00820764096552965\\
5.37	0.00820765966471879\\
5.38	0.00820767837120224\\
5.39	0.00820769708498311\\
5.4	0.00820771580606454\\
5.41	0.00820773453444967\\
5.42	0.00820775327014164\\
5.43	0.0082077720131436\\
5.44	0.00820779076345867\\
5.45	0.00820780952109001\\
5.46	0.00820782828604076\\
5.47	0.00820784705831408\\
5.48	0.00820786583791309\\
5.49	0.00820788462484097\\
5.5	0.00820790341910086\\
5.51	0.00820792222069591\\
5.52	0.00820794102962928\\
5.53	0.00820795984590412\\
5.54	0.0082079786695236\\
5.55	0.00820799750049087\\
5.56	0.00820801633880909\\
5.57	0.00820803518448143\\
5.58	0.00820805403751106\\
5.59	0.00820807289790113\\
5.6	0.00820809176565482\\
5.61	0.0082081106407753\\
5.62	0.00820812952326573\\
5.63	0.0082081484131293\\
5.64	0.00820816731036917\\
5.65	0.00820818621498852\\
5.66	0.00820820512699054\\
5.67	0.00820822404637839\\
5.68	0.00820824297315526\\
5.69	0.00820826190732434\\
5.7	0.0082082808488888\\
5.71	0.00820829979785183\\
5.72	0.00820831875421663\\
5.73	0.00820833771798637\\
5.74	0.00820835668916426\\
5.75	0.00820837566775347\\
5.76	0.00820839465375722\\
5.77	0.00820841364717869\\
5.78	0.00820843264802108\\
5.79	0.00820845165628759\\
5.8	0.00820847067198142\\
5.81	0.00820848969510577\\
5.82	0.00820850872566385\\
5.83	0.00820852776365886\\
5.84	0.00820854680909401\\
5.85	0.00820856586197251\\
5.86	0.00820858492229756\\
5.87	0.00820860399007239\\
5.88	0.00820862306530021\\
5.89	0.00820864214798422\\
5.9	0.00820866123812766\\
5.91	0.00820868033573372\\
5.92	0.00820869944080565\\
5.93	0.00820871855334666\\
5.94	0.00820873767335997\\
5.95	0.00820875680084881\\
5.96	0.0082087759358164\\
5.97	0.00820879507826598\\
5.98	0.00820881422820077\\
5.99	0.00820883338562401\\
6	0.00820885255053894\\
6.01	0.00820887172294877\\
6.02	0.00820889090285676\\
6.03	0.00820891009026615\\
6.04	0.00820892928518017\\
6.05	0.00820894848760206\\
6.06	0.00820896769753507\\
6.07	0.00820898691498245\\
6.08	0.00820900613994743\\
6.09	0.00820902537243328\\
6.1	0.00820904461244324\\
6.11	0.00820906385998056\\
6.12	0.0082090831150485\\
6.13	0.00820910237765031\\
6.14	0.00820912164778925\\
6.15	0.00820914092546858\\
6.16	0.00820916021069156\\
6.17	0.00820917950346144\\
6.18	0.00820919880378151\\
6.19	0.00820921811165501\\
6.2	0.00820923742708523\\
6.21	0.00820925675007542\\
6.22	0.00820927608062886\\
6.23	0.00820929541874882\\
6.24	0.00820931476443857\\
6.25	0.00820933411770139\\
6.26	0.00820935347854056\\
6.27	0.00820937284695936\\
6.28	0.00820939222296107\\
6.29	0.00820941160654896\\
6.3	0.00820943099772633\\
6.31	0.00820945039649646\\
6.32	0.00820946980286264\\
6.33	0.00820948921682815\\
6.34	0.0082095086383963\\
6.35	0.00820952806757036\\
6.36	0.00820954750435364\\
6.37	0.00820956694874943\\
6.38	0.00820958640076104\\
6.39	0.00820960586039175\\
6.4	0.00820962532764486\\
6.41	0.0082096448025237\\
6.42	0.00820966428503155\\
6.43	0.00820968377517172\\
6.44	0.00820970327294753\\
6.45	0.00820972277836228\\
6.46	0.00820974229141929\\
6.47	0.00820976181212186\\
6.48	0.00820978134047331\\
6.49	0.00820980087647696\\
6.5	0.00820982042013613\\
6.51	0.00820983997145414\\
6.52	0.00820985953043431\\
6.53	0.00820987909707995\\
6.54	0.00820989867139441\\
6.55	0.008209918253381\\
6.56	0.00820993784304306\\
6.57	0.00820995744038391\\
6.58	0.00820997704540689\\
6.59	0.00820999665811533\\
6.6	0.00821001627851256\\
6.61	0.00821003590660192\\
6.62	0.00821005554238677\\
6.63	0.00821007518587042\\
6.64	0.00821009483705623\\
6.65	0.00821011449594754\\
6.66	0.0082101341625477\\
6.67	0.00821015383686005\\
6.68	0.00821017351888794\\
6.69	0.00821019320863474\\
6.7	0.00821021290610377\\
6.71	0.00821023261129842\\
6.72	0.00821025232422202\\
6.73	0.00821027204487794\\
6.74	0.00821029177326954\\
6.75	0.00821031150940017\\
6.76	0.00821033125327322\\
6.77	0.00821035100489202\\
6.78	0.00821037076425997\\
6.79	0.00821039053138042\\
6.8	0.00821041030625674\\
6.81	0.00821043008889232\\
6.82	0.00821044987929051\\
6.83	0.0082104696774547\\
6.84	0.00821048948338826\\
6.85	0.00821050929709458\\
6.86	0.00821052911857703\\
6.87	0.00821054894783901\\
6.88	0.00821056878488388\\
6.89	0.00821058862971504\\
6.9	0.00821060848233588\\
6.91	0.00821062834274979\\
6.92	0.00821064821096015\\
6.93	0.00821066808697037\\
6.94	0.00821068797078383\\
6.95	0.00821070786240393\\
6.96	0.00821072776183408\\
6.97	0.00821074766907768\\
6.98	0.00821076758413812\\
6.99	0.0082107875070188\\
7	0.00821080743772314\\
7.01	0.00821082737625455\\
7.02	0.00821084732261643\\
7.03	0.00821086727681219\\
7.04	0.00821088723884525\\
7.05	0.00821090720871902\\
7.06	0.00821092718643692\\
7.07	0.00821094717200236\\
7.08	0.00821096716541877\\
7.09	0.00821098716668957\\
7.1	0.00821100717581818\\
7.11	0.00821102719280802\\
7.12	0.00821104721766253\\
7.13	0.00821106725038514\\
7.14	0.00821108729097927\\
7.15	0.00821110733944835\\
7.16	0.00821112739579583\\
7.17	0.00821114746002513\\
7.18	0.0082111675321397\\
7.19	0.00821118761214298\\
7.2	0.0082112077000384\\
7.21	0.00821122779582941\\
7.22	0.00821124789951946\\
7.23	0.00821126801111199\\
7.24	0.00821128813061045\\
7.25	0.00821130825801829\\
7.26	0.00821132839333897\\
7.27	0.00821134853657594\\
7.28	0.00821136868773265\\
7.29	0.00821138884681256\\
7.3	0.00821140901381914\\
7.31	0.00821142918875584\\
7.32	0.00821144937162613\\
7.33	0.00821146956243346\\
7.34	0.00821148976118132\\
7.35	0.00821150996787316\\
7.36	0.00821153018251247\\
7.37	0.0082115504051027\\
7.38	0.00821157063564735\\
7.39	0.00821159087414986\\
7.4	0.00821161112061374\\
7.41	0.00821163137504246\\
7.42	0.0082116516374395\\
7.43	0.00821167190780834\\
7.44	0.00821169218615247\\
7.45	0.00821171247247537\\
7.46	0.00821173276678054\\
7.47	0.00821175306907146\\
7.48	0.00821177337935163\\
7.49	0.00821179369762454\\
7.5	0.00821181402389369\\
7.51	0.00821183435816258\\
7.52	0.0082118547004347\\
7.53	0.00821187505071356\\
7.54	0.00821189540900266\\
7.55	0.0082119157753055\\
7.56	0.0082119361496256\\
7.57	0.00821195653196646\\
7.58	0.00821197692233159\\
7.59	0.00821199732072451\\
7.6	0.00821201772714873\\
7.61	0.00821203814160776\\
7.62	0.00821205856410513\\
7.63	0.00821207899464436\\
7.64	0.00821209943322896\\
7.65	0.00821211987986246\\
7.66	0.00821214033454839\\
7.67	0.00821216079729027\\
7.68	0.00821218126809164\\
7.69	0.00821220174695602\\
7.7	0.00821222223388695\\
7.71	0.00821224272888796\\
7.72	0.00821226323196259\\
7.73	0.00821228374311438\\
7.74	0.00821230426234687\\
7.75	0.0082123247896636\\
7.76	0.00821234532506811\\
7.77	0.00821236586856395\\
7.78	0.00821238642015468\\
7.79	0.00821240697984383\\
7.8	0.00821242754763497\\
7.81	0.00821244812353164\\
7.82	0.0082124687075374\\
7.83	0.0082124892996558\\
7.84	0.00821250989989041\\
7.85	0.00821253050824479\\
7.86	0.0082125511247225\\
7.87	0.00821257174932711\\
7.88	0.00821259238206218\\
7.89	0.00821261302293128\\
7.9	0.00821263367193798\\
7.91	0.00821265432908585\\
7.92	0.00821267499437848\\
7.93	0.00821269566781943\\
7.94	0.00821271634941228\\
7.95	0.00821273703916061\\
7.96	0.008212757737068\\
7.97	0.00821277844313805\\
7.98	0.00821279915737433\\
7.99	0.00821281987978042\\
8	0.00821284061035993\\
8.01	0.00821286134911644\\
8.02	0.00821288209605355\\
8.03	0.00821290285117484\\
8.04	0.00821292361448392\\
8.05	0.00821294438598439\\
8.06	0.00821296516567984\\
8.07	0.00821298595357389\\
8.08	0.00821300674967012\\
8.09	0.00821302755397216\\
8.1	0.0082130483664836\\
8.11	0.00821306918720807\\
8.12	0.00821309001614916\\
8.13	0.0082131108533105\\
8.14	0.0082131316986957\\
8.15	0.00821315255230838\\
8.16	0.00821317341415217\\
8.17	0.00821319428423067\\
8.18	0.00821321516254751\\
8.19	0.00821323604910633\\
8.2	0.00821325694391075\\
8.21	0.00821327784696438\\
8.22	0.00821329875827089\\
8.23	0.00821331967783388\\
8.24	0.008213340605657\\
8.25	0.00821336154174388\\
8.26	0.00821338248609816\\
8.27	0.00821340343872349\\
8.28	0.00821342439962351\\
8.29	0.00821344536880186\\
8.3	0.00821346634626218\\
8.31	0.00821348733200814\\
8.32	0.00821350832604337\\
8.33	0.00821352932837154\\
8.34	0.00821355033899629\\
8.35	0.00821357135792128\\
8.36	0.00821359238515017\\
8.37	0.00821361342068662\\
8.38	0.00821363446453429\\
8.39	0.00821365551669685\\
8.4	0.00821367657717796\\
8.41	0.0082136976459813\\
8.42	0.00821371872311052\\
8.43	0.00821373980856931\\
8.44	0.00821376090236133\\
8.45	0.00821378200449027\\
8.46	0.00821380311495979\\
8.47	0.00821382423377359\\
8.48	0.00821384536093534\\
8.49	0.00821386649644872\\
8.5	0.00821388764031742\\
8.51	0.00821390879254514\\
8.52	0.00821392995313555\\
8.53	0.00821395112209235\\
8.54	0.00821397229941924\\
8.55	0.0082139934851199\\
8.56	0.00821401467919804\\
8.57	0.00821403588165735\\
8.58	0.00821405709250154\\
8.59	0.00821407831173431\\
8.6	0.00821409953935936\\
8.61	0.00821412077538041\\
8.62	0.00821414201980115\\
8.63	0.00821416327262531\\
8.64	0.0082141845338566\\
8.65	0.00821420580349872\\
8.66	0.0082142270815554\\
8.67	0.00821424836803036\\
8.68	0.00821426966292731\\
8.69	0.00821429096624998\\
8.7	0.0082143122780021\\
8.71	0.00821433359818739\\
8.72	0.00821435492680958\\
8.73	0.0082143762638724\\
8.74	0.00821439760937959\\
8.75	0.00821441896333487\\
8.76	0.00821444032574199\\
8.77	0.00821446169660468\\
8.78	0.00821448307592669\\
8.79	0.00821450446371175\\
8.8	0.00821452585996362\\
8.81	0.00821454726468604\\
8.82	0.00821456867788275\\
8.83	0.00821459009955751\\
8.84	0.00821461152971408\\
8.85	0.0082146329683562\\
8.86	0.00821465441548763\\
8.87	0.00821467587111213\\
8.88	0.00821469733523347\\
8.89	0.00821471880785539\\
8.9	0.00821474028898168\\
8.91	0.00821476177861609\\
8.92	0.0082147832767624\\
8.93	0.00821480478342437\\
8.94	0.00821482629860578\\
8.95	0.00821484782231039\\
8.96	0.008214869354542\\
8.97	0.00821489089530437\\
8.98	0.00821491244460129\\
8.99	0.00821493400243654\\
9	0.0082149555688139\\
9.01	0.00821497714373716\\
9.02	0.00821499872721012\\
9.03	0.00821502031923655\\
9.04	0.00821504191982026\\
9.05	0.00821506352896504\\
9.06	0.00821508514667467\\
9.07	0.00821510677295297\\
9.08	0.00821512840780374\\
9.09	0.00821515005123077\\
9.1	0.00821517170323787\\
9.11	0.00821519336382884\\
9.12	0.00821521503300751\\
9.13	0.00821523671077767\\
9.14	0.00821525839714314\\
9.15	0.00821528009210773\\
9.16	0.00821530179567527\\
9.17	0.00821532350784957\\
9.18	0.00821534522863444\\
9.19	0.00821536695803373\\
9.2	0.00821538869605123\\
9.21	0.00821541044269079\\
9.22	0.00821543219795624\\
9.23	0.0082154539618514\\
9.24	0.00821547573438011\\
9.25	0.00821549751554621\\
9.26	0.00821551930535352\\
9.27	0.00821554110380589\\
9.28	0.00821556291090716\\
9.29	0.00821558472666118\\
9.3	0.00821560655107179\\
9.31	0.00821562838414282\\
9.32	0.00821565022587815\\
9.33	0.00821567207628161\\
9.34	0.00821569393535706\\
9.35	0.00821571580310835\\
9.36	0.00821573767953935\\
9.37	0.0082157595646539\\
9.38	0.00821578145845588\\
9.39	0.00821580336094914\\
9.4	0.00821582527213756\\
9.41	0.00821584719202499\\
9.42	0.00821586912061531\\
9.43	0.00821589105791239\\
9.44	0.00821591300392011\\
9.45	0.00821593495864233\\
9.46	0.00821595692208293\\
9.47	0.00821597889424581\\
9.48	0.00821600087513483\\
9.49	0.00821602286475388\\
9.5	0.00821604486310686\\
9.51	0.00821606687019763\\
9.52	0.0082160888860301\\
9.53	0.00821611091060816\\
9.54	0.0082161329439357\\
9.55	0.00821615498601661\\
9.56	0.00821617703685481\\
9.57	0.00821619909645417\\
9.58	0.00821622116481861\\
9.59	0.00821624324195203\\
9.6	0.00821626532785835\\
9.61	0.00821628742254145\\
9.62	0.00821630952600526\\
9.63	0.00821633163825369\\
9.64	0.00821635375929065\\
9.65	0.00821637588912006\\
9.66	0.00821639802774585\\
9.67	0.00821642017517191\\
9.68	0.00821644233140218\\
9.69	0.00821646449644059\\
9.7	0.00821648667029107\\
9.71	0.00821650885295753\\
9.72	0.00821653104444391\\
9.73	0.00821655324475415\\
9.74	0.00821657545389217\\
9.75	0.00821659767186192\\
9.76	0.00821661989866733\\
9.77	0.00821664213431235\\
9.78	0.00821666437880092\\
9.79	0.00821668663213698\\
9.8	0.00821670889432447\\
9.81	0.00821673116536736\\
9.82	0.00821675344526958\\
9.83	0.0082167757340351\\
9.84	0.00821679803166786\\
9.85	0.00821682033817183\\
9.86	0.00821684265355096\\
9.87	0.00821686497780922\\
9.88	0.00821688731095057\\
9.89	0.00821690965297897\\
9.9	0.00821693200389839\\
9.91	0.0082169543637128\\
9.92	0.00821697673242618\\
9.93	0.00821699911004249\\
9.94	0.00821702149656571\\
9.95	0.00821704389199982\\
9.96	0.00821706629634879\\
9.97	0.00821708870961662\\
9.98	0.00821711113180728\\
9.99	0.00821713356292475\\
10	0.00821715600297304\\
10.01	0.00821717845195611\\
10.02	0.00821720090987798\\
10.03	0.00821722337674262\\
10.04	0.00821724585255405\\
10.05	0.00821726833731624\\
10.06	0.00821729083103322\\
10.07	0.00821731333370897\\
10.08	0.00821733584534749\\
10.09	0.0082173583659528\\
10.1	0.0082173808955289\\
10.11	0.00821740343407981\\
10.12	0.00821742598160953\\
10.13	0.00821744853812208\\
10.14	0.00821747110362148\\
10.15	0.00821749367811173\\
10.16	0.00821751626159687\\
10.17	0.0082175388540809\\
10.18	0.00821756145556787\\
10.19	0.00821758406606179\\
10.2	0.00821760668556669\\
10.21	0.0082176293140866\\
10.22	0.00821765195162555\\
10.23	0.00821767459818758\\
10.24	0.00821769725377672\\
10.25	0.008217719918397\\
10.26	0.00821774259205247\\
10.27	0.00821776527474718\\
10.28	0.00821778796648515\\
10.29	0.00821781066727045\\
10.3	0.00821783337710712\\
10.31	0.00821785609599919\\
10.32	0.00821787882395074\\
10.33	0.0082179015609658\\
10.34	0.00821792430704845\\
10.35	0.00821794706220272\\
10.36	0.00821796982643269\\
10.37	0.00821799259974241\\
10.38	0.00821801538213594\\
10.39	0.00821803817361736\\
10.4	0.00821806097419073\\
10.41	0.00821808378386011\\
10.42	0.00821810660262958\\
10.43	0.00821812943050321\\
10.44	0.00821815226748507\\
10.45	0.00821817511357924\\
10.46	0.00821819796878981\\
10.47	0.00821822083312084\\
10.48	0.00821824370657643\\
10.49	0.00821826658916065\\
10.5	0.00821828948087759\\
10.51	0.00821831238173134\\
10.52	0.00821833529172599\\
10.53	0.00821835821086563\\
10.54	0.00821838113915436\\
10.55	0.00821840407659626\\
10.56	0.00821842702319544\\
10.57	0.00821844997895599\\
10.58	0.00821847294388203\\
10.59	0.00821849591797764\\
10.6	0.00821851890124693\\
10.61	0.00821854189369401\\
10.62	0.00821856489532299\\
10.63	0.00821858790613797\\
10.64	0.00821861092614308\\
10.65	0.00821863395534241\\
10.66	0.0082186569937401\\
10.67	0.00821868004134024\\
10.68	0.00821870309814697\\
10.69	0.00821872616416441\\
10.7	0.00821874923939666\\
10.71	0.00821877232384787\\
10.72	0.00821879541752215\\
10.73	0.00821881852042364\\
10.74	0.00821884163255646\\
10.75	0.00821886475392474\\
10.76	0.00821888788453262\\
10.77	0.00821891102438422\\
10.78	0.00821893417348369\\
10.79	0.00821895733183517\\
10.8	0.00821898049944279\\
10.81	0.00821900367631069\\
10.82	0.00821902686244302\\
10.83	0.00821905005784392\\
10.84	0.00821907326251753\\
10.85	0.00821909647646801\\
10.86	0.00821911969969951\\
10.87	0.00821914293221616\\
10.88	0.00821916617402213\\
10.89	0.00821918942512157\\
10.9	0.00821921268551864\\
10.91	0.00821923595521749\\
10.92	0.00821925923422228\\
10.93	0.00821928252253717\\
10.94	0.00821930582016633\\
10.95	0.00821932912711391\\
10.96	0.00821935244338407\\
10.97	0.008219375768981\\
10.98	0.00821939910390884\\
10.99	0.00821942244817178\\
11	0.00821944580177398\\
11.01	0.00821946916471961\\
11.02	0.00821949253701286\\
11.03	0.00821951591865788\\
11.04	0.00821953930965885\\
11.05	0.00821956271001997\\
11.06	0.00821958611974539\\
11.07	0.00821960953883931\\
11.08	0.0082196329673059\\
11.09	0.00821965640514935\\
11.1	0.00821967985237384\\
11.11	0.00821970330898355\\
11.12	0.00821972677498267\\
11.13	0.00821975025037539\\
11.14	0.0082197737351659\\
11.15	0.00821979722935838\\
11.16	0.00821982073295703\\
11.17	0.00821984424596604\\
11.18	0.0082198677683896\\
11.19	0.00821989130023191\\
11.2	0.00821991484149716\\
11.21	0.00821993839218954\\
11.22	0.00821996195231327\\
11.23	0.00821998552187252\\
11.24	0.00822000910087151\\
11.25	0.00822003268931443\\
11.26	0.00822005628720548\\
11.27	0.00822007989454887\\
11.28	0.0082201035113488\\
11.29	0.00822012713760947\\
11.3	0.00822015077333509\\
11.31	0.00822017441852986\\
11.32	0.00822019807319799\\
11.33	0.00822022173734369\\
11.34	0.00822024541097115\\
11.35	0.00822026909408461\\
11.36	0.00822029278668825\\
11.37	0.00822031648878629\\
11.38	0.00822034020038294\\
11.39	0.00822036392148242\\
11.4	0.00822038765208892\\
11.41	0.00822041139220668\\
11.42	0.00822043514183988\\
11.43	0.00822045890099276\\
11.44	0.00822048266966953\\
11.45	0.00822050644787438\\
11.46	0.00822053023561155\\
11.47	0.00822055403288525\\
11.48	0.00822057783969968\\
11.49	0.00822060165605907\\
11.5	0.00822062548196763\\
11.51	0.00822064931742958\\
11.52	0.00822067316244912\\
11.53	0.00822069701703048\\
11.54	0.00822072088117786\\
11.55	0.00822074475489549\\
11.56	0.00822076863818759\\
11.57	0.00822079253105836\\
11.58	0.00822081643351202\\
11.59	0.00822084034555278\\
11.6	0.00822086426718487\\
11.61	0.0082208881984125\\
11.62	0.00822091213923987\\
11.63	0.0082209360896712\\
11.64	0.00822096004971072\\
11.65	0.00822098401936261\\
11.66	0.00822100799863112\\
11.67	0.00822103198752044\\
11.68	0.00822105598603477\\
11.69	0.00822107999417835\\
11.7	0.00822110401195537\\
11.71	0.00822112803937005\\
11.72	0.00822115207642659\\
11.73	0.00822117612312919\\
11.74	0.00822120017948208\\
11.75	0.00822122424548944\\
11.76	0.00822124832115549\\
11.77	0.00822127240648443\\
11.78	0.00822129650148046\\
11.79	0.00822132060614779\\
11.8	0.0082213447204906\\
11.81	0.00822136884451311\\
11.82	0.0082213929782195\\
11.83	0.00822141712161397\\
11.84	0.00822144127470072\\
11.85	0.00822146543748393\\
11.86	0.0082214896099678\\
11.87	0.00822151379215652\\
11.88	0.00822153798405427\\
11.89	0.00822156218566523\\
11.9	0.0082215863969936\\
11.91	0.00822161061804355\\
11.92	0.00822163484881925\\
11.93	0.00822165908932489\\
11.94	0.00822168333956463\\
11.95	0.00822170759954266\\
11.96	0.00822173186926314\\
11.97	0.00822175614873023\\
11.98	0.00822178043794809\\
11.99	0.0082218047369209\\
12	0.0082218290456528\\
12.01	0.00822185336414796\\
12.02	0.00822187769241051\\
12.03	0.00822190203044463\\
12.04	0.00822192637825443\\
12.05	0.00822195073584408\\
12.06	0.00822197510321771\\
12.07	0.00822199948037945\\
12.08	0.00822202386733343\\
12.09	0.0082220482640838\\
12.1	0.00822207267063465\\
12.11	0.00822209708699013\\
12.12	0.00822212151315434\\
12.13	0.0082221459491314\\
12.14	0.00822217039492541\\
12.15	0.00822219485054047\\
12.16	0.00822221931598069\\
12.17	0.00822224379125015\\
12.18	0.00822226827635296\\
12.19	0.00822229277129319\\
12.2	0.00822231727607491\\
12.21	0.00822234179070221\\
12.22	0.00822236631517915\\
12.23	0.00822239084950981\\
12.24	0.00822241539369822\\
12.25	0.00822243994774845\\
12.26	0.00822246451166454\\
12.27	0.00822248908545054\\
12.28	0.00822251366911047\\
12.29	0.00822253826264837\\
12.3	0.00822256286606825\\
12.31	0.00822258747937414\\
12.32	0.00822261210257003\\
12.33	0.00822263673565994\\
12.34	0.00822266137864785\\
12.35	0.00822268603153775\\
12.36	0.00822271069433363\\
12.37	0.00822273536703944\\
12.38	0.00822276004965917\\
12.39	0.00822278474219676\\
12.4	0.00822280944465617\\
12.41	0.00822283415704133\\
12.42	0.00822285887935617\\
12.43	0.00822288361160463\\
12.44	0.00822290835379061\\
12.45	0.00822293310591802\\
12.46	0.00822295786799076\\
12.47	0.00822298264001272\\
12.48	0.00822300742198777\\
12.49	0.00822303221391978\\
12.5	0.00822305701581262\\
12.51	0.00822308182767011\\
12.52	0.00822310664949612\\
12.53	0.00822313148129447\\
12.54	0.00822315632306896\\
12.55	0.00822318117482341\\
12.56	0.00822320603656161\\
12.57	0.00822323090828735\\
12.58	0.0082232557900044\\
12.59	0.00822328068171651\\
12.6	0.00822330558342744\\
12.61	0.00822333049514092\\
12.62	0.00822335541686067\\
12.63	0.0082233803485904\\
12.64	0.00822340529033382\\
12.65	0.00822343024209461\\
12.66	0.00822345520387643\\
12.67	0.00822348017568295\\
12.68	0.00822350515751779\\
12.69	0.00822353014938461\\
12.7	0.00822355515128701\\
12.71	0.00822358016322858\\
12.72	0.00822360518521292\\
12.73	0.00822363021724359\\
12.74	0.00822365525932415\\
12.75	0.00822368031145813\\
12.76	0.00822370537364906\\
12.77	0.00822373044590043\\
12.78	0.00822375552821575\\
12.79	0.00822378062059847\\
12.8	0.00822380572305207\\
12.81	0.00822383083557996\\
12.82	0.00822385595818557\\
12.83	0.0082238810908723\\
12.84	0.00822390623364353\\
12.85	0.00822393138650262\\
12.86	0.00822395654945293\\
12.87	0.00822398172249776\\
12.88	0.00822400690564044\\
12.89	0.00822403209888424\\
12.9	0.00822405730223242\\
12.91	0.00822408251568824\\
12.92	0.00822410773925491\\
12.93	0.00822413297293564\\
12.94	0.0082241582167336\\
12.95	0.00822418347065195\\
12.96	0.00822420873469382\\
12.97	0.00822423400886234\\
12.98	0.00822425929316058\\
12.99	0.00822428458759162\\
13	0.0082243098921585\\
13.01	0.00822433520686423\\
13.02	0.0082243605317118\\
13.03	0.0082243858667042\\
13.04	0.00822441121184435\\
13.05	0.00822443656713517\\
13.06	0.00822446193257957\\
13.07	0.0082244873081804\\
13.08	0.00822451269394051\\
13.09	0.0082245380898627\\
13.1	0.00822456349594975\\
13.11	0.00822458891220443\\
13.12	0.00822461433862947\\
13.13	0.00822463977522756\\
13.14	0.00822466522200137\\
13.15	0.00822469067895356\\
13.16	0.00822471614608672\\
13.17	0.00822474162340344\\
13.18	0.00822476711090627\\
13.19	0.00822479260859774\\
13.2	0.00822481811648033\\
13.21	0.00822484363455651\\
13.22	0.00822486916282869\\
13.23	0.00822489470129927\\
13.24	0.00822492024997062\\
13.25	0.00822494580884505\\
13.26	0.00822497137792488\\
13.27	0.00822499695721235\\
13.28	0.00822502254670969\\
13.29	0.00822504814641911\\
13.3	0.00822507375634275\\
13.31	0.00822509937648274\\
13.32	0.00822512500684116\\
13.33	0.00822515064742007\\
13.34	0.00822517629822149\\
13.35	0.00822520195924738\\
13.36	0.0082252276304997\\
13.37	0.00822525331198034\\
13.38	0.00822527900369118\\
13.39	0.00822530470563404\\
13.4	0.00822533041781072\\
13.41	0.00822535614022296\\
13.42	0.00822538187287248\\
13.43	0.00822540761576096\\
13.44	0.00822543336889003\\
13.45	0.00822545913226128\\
13.46	0.00822548490587629\\
13.47	0.00822551068973655\\
13.48	0.00822553648384355\\
13.49	0.00822556228819871\\
13.5	0.00822558810280345\\
13.51	0.00822561392765911\\
13.52	0.008225639762767\\
13.53	0.0082256656081284\\
13.54	0.00822569146374454\\
13.55	0.0082257173296166\\
13.56	0.00822574320574574\\
13.57	0.00822576909213305\\
13.58	0.00822579498877961\\
13.59	0.00822582089568644\\
13.6	0.00822584681285452\\
13.61	0.00822587274028477\\
13.62	0.00822589867797812\\
13.63	0.0082259246259354\\
13.64	0.00822595058415744\\
13.65	0.008225976552645\\
13.66	0.00822600253139882\\
13.67	0.00822602852041958\\
13.68	0.00822605451970794\\
13.69	0.0082260805292645\\
13.7	0.00822610654908983\\
13.71	0.00822613257918445\\
13.72	0.00822615861954885\\
13.73	0.00822618467018348\\
13.74	0.00822621073108874\\
13.75	0.00822623680226501\\
13.76	0.00822626288371261\\
13.77	0.00822628897543182\\
13.78	0.00822631507742292\\
13.79	0.00822634118968611\\
13.8	0.00822636731222158\\
13.81	0.00822639344502946\\
13.82	0.00822641958810988\\
13.83	0.0082264457414629\\
13.84	0.00822647190508858\\
13.85	0.00822649807898691\\
13.86	0.00822652426315789\\
13.87	0.00822655045760146\\
13.88	0.00822657666231754\\
13.89	0.00822660287730602\\
13.9	0.00822662910256677\\
13.91	0.00822665533809962\\
13.92	0.00822668158390439\\
13.93	0.00822670783998087\\
13.94	0.00822673410632884\\
13.95	0.00822676038294804\\
13.96	0.00822678666983819\\
13.97	0.00822681296699902\\
13.98	0.00822683927443024\\
13.99	0.00822686559213151\\
14	0.00822689192010252\\
14.01	0.00822691825834294\\
14.02	0.00822694460685243\\
14.03	0.00822697096563062\\
14.04	0.00822699733467719\\
14.05	0.00822702371399178\\
14.06	0.00822705010357405\\
14.07	0.00822707650342365\\
14.08	0.00822710291354025\\
14.09	0.00822712933392354\\
14.1	0.00822715576457318\\
14.11	0.00822718220548891\\
14.12	0.00822720865667044\\
14.13	0.00822723511811752\\
14.14	0.00822726158982993\\
14.15	0.00822728807180746\\
14.16	0.00822731456404997\\
14.17	0.00822734106655732\\
14.18	0.00822736757932943\\
14.19	0.00822739410236627\\
14.2	0.00822742063566784\\
14.21	0.0082274471792342\\
14.22	0.00822747373306548\\
14.23	0.00822750029716185\\
14.24	0.00822752687152357\\
14.25	0.00822755345615096\\
14.26	0.00822758005104441\\
14.27	0.00822760665620441\\
14.28	0.00822763327163152\\
14.29	0.00822765989732639\\
14.3	0.00822768653328978\\
14.31	0.00822771317952256\\
14.32	0.00822773983602569\\
14.33	0.00822776650280026\\
14.34	0.00822779317984745\\
14.35	0.00822781986716862\\
14.36	0.00822784656476522\\
14.37	0.00822787327263886\\
14.38	0.00822789999079129\\
14.39	0.00822792671922441\\
14.4	0.00822795345794031\\
14.41	0.0082279802069412\\
14.42	0.00822800696622952\\
14.43	0.00822803373580783\\
14.44	0.00822806051567893\\
14.45	0.0082280873058458\\
14.46	0.00822811410631163\\
14.47	0.00822814091707981\\
14.48	0.00822816773815396\\
14.49	0.00822819456953794\\
14.5	0.00822822141123584\\
14.51	0.00822824826325197\\
14.52	0.00822827512559094\\
14.53	0.0082283019982576\\
14.54	0.00822832888125705\\
14.55	0.0082283557745947\\
14.56	0.00822838267827625\\
14.57	0.00822840959230768\\
14.58	0.00822843651669527\\
14.59	0.00822846345144563\\
14.6	0.00822849039656568\\
14.61	0.00822851735206268\\
14.62	0.00822854431794423\\
14.63	0.00822857129421827\\
14.64	0.00822859828089309\\
14.65	0.00822862527797737\\
14.66	0.00822865228548013\\
14.67	0.00822867930341078\\
14.68	0.00822870633177912\\
14.69	0.00822873337059535\\
14.7	0.00822876041987007\\
14.71	0.00822878747961427\\
14.72	0.00822881454983936\\
14.73	0.00822884163055719\\
14.74	0.00822886872178001\\
14.75	0.00822889582352051\\
14.76	0.00822892293579181\\
14.77	0.00822895005860748\\
14.78	0.0082289771919815\\
14.79	0.00822900433592832\\
14.8	0.00822903149046283\\
14.81	0.00822905865560036\\
14.82	0.00822908583135669\\
14.83	0.00822911301774801\\
14.84	0.008229140214791\\
14.85	0.00822916742250275\\
14.86	0.00822919464090077\\
14.87	0.008229221870003\\
14.88	0.00822924910982782\\
14.89	0.00822927636039398\\
14.9	0.00822930362172064\\
14.91	0.00822933089382735\\
14.92	0.00822935817673402\\
14.93	0.0082293854704609\\
14.94	0.00822941277502858\\
14.95	0.00822944009045794\\
14.96	0.00822946741677015\\
14.97	0.00822949475398665\\
14.98	0.00822952210212907\\
14.99	0.00822954946121925\\
15	0.00822957683127917\\
15.01	0.00822960421233094\\
15.02	0.00822963160439672\\
15.03	0.00822965900749867\\
15.04	0.00822968642165895\\
15.05	0.00822971384689961\\
15.06	0.00822974128324252\\
15.07	0.00822976873070937\\
15.08	0.0082297961893215\\
15.09	0.00822982365909993\\
15.1	0.00822985114006517\\
15.11	0.0082298786322372\\
15.12	0.00822990613563533\\
15.13	0.00822993365027812\\
15.14	0.00822996117618327\\
15.15	0.00822998871336746\\
15.16	0.00823001626184626\\
15.17	0.00823004382163397\\
15.18	0.00823007139274346\\
15.19	0.00823009897518604\\
15.2	0.00823012656897128\\
15.21	0.0082301541741068\\
15.22	0.00823018179059811\\
15.23	0.00823020941845053\\
15.24	0.00823023705766938\\
15.25	0.00823026470825996\\
15.26	0.0082302923702276\\
15.27	0.00823032004357763\\
15.28	0.00823034772831537\\
15.29	0.00823037542444615\\
15.3	0.0082304031319753\\
15.31	0.00823043085090817\\
15.32	0.00823045858125009\\
15.33	0.00823048632300639\\
15.34	0.00823051407618243\\
15.35	0.00823054184078356\\
15.36	0.00823056961681511\\
15.37	0.00823059740428246\\
15.38	0.00823062520319094\\
15.39	0.00823065301354593\\
15.4	0.00823068083535278\\
15.41	0.00823070866861686\\
15.42	0.00823073651334353\\
15.43	0.00823076436953818\\
15.44	0.00823079223720617\\
15.45	0.00823082011635288\\
15.46	0.00823084800698368\\
15.47	0.00823087590910396\\
15.48	0.00823090382271911\\
15.49	0.00823093174783451\\
15.5	0.00823095968445557\\
15.51	0.00823098763258766\\
15.52	0.00823101559223619\\
15.53	0.00823104356340656\\
15.54	0.00823107154610417\\
15.55	0.00823109954033443\\
15.56	0.00823112754610275\\
15.57	0.00823115556341453\\
15.58	0.0082311835922752\\
15.59	0.00823121163269018\\
15.6	0.00823123968466488\\
15.61	0.00823126774820473\\
15.62	0.00823129582331515\\
15.63	0.00823132391000158\\
15.64	0.00823135200826944\\
15.65	0.00823138011812419\\
15.66	0.00823140823957124\\
15.67	0.00823143637261605\\
15.68	0.00823146451726405\\
15.69	0.00823149267352071\\
15.7	0.00823152084139146\\
15.71	0.00823154902088177\\
15.72	0.00823157721199709\\
15.73	0.00823160541474287\\
15.74	0.00823163362912458\\
15.75	0.0082316618551477\\
15.76	0.00823169009281768\\
15.77	0.00823171834213999\\
15.78	0.00823174660312013\\
15.79	0.00823177487576354\\
15.8	0.00823180316007574\\
15.81	0.00823183145606218\\
15.82	0.00823185976372837\\
15.83	0.00823188808307978\\
15.84	0.00823191641412192\\
15.85	0.00823194475686029\\
15.86	0.00823197311130036\\
15.87	0.00823200147744766\\
15.88	0.00823202985530768\\
15.89	0.00823205824488594\\
15.9	0.00823208664618794\\
15.91	0.00823211505921919\\
15.92	0.00823214348398522\\
15.93	0.00823217192049154\\
15.94	0.00823220036874368\\
15.95	0.00823222882874716\\
15.96	0.00823225730050751\\
15.97	0.00823228578403026\\
15.98	0.00823231427932095\\
15.99	0.00823234278638512\\
16	0.0082323713052283\\
16.01	0.00823239983585605\\
16.02	0.0082324283782739\\
16.03	0.0082324569324874\\
16.04	0.00823248549850212\\
16.05	0.0082325140763236\\
16.06	0.00823254266595741\\
16.07	0.00823257126740911\\
16.08	0.00823259988068426\\
16.09	0.00823262850578843\\
16.1	0.0082326571427272\\
16.11	0.00823268579150613\\
16.12	0.00823271445213081\\
16.13	0.0082327431246068\\
16.14	0.00823277180893972\\
16.15	0.00823280050513512\\
16.16	0.0082328292131986\\
16.17	0.00823285793313577\\
16.18	0.0082328866649522\\
16.19	0.00823291540865351\\
16.2	0.00823294416424529\\
16.21	0.00823297293173315\\
16.22	0.0082330017111227\\
16.23	0.00823303050241954\\
16.24	0.00823305930562929\\
16.25	0.00823308812075757\\
16.26	0.00823311694781\\
16.27	0.0082331457867922\\
16.28	0.0082331746377098\\
16.29	0.00823320350056842\\
16.3	0.00823323237537371\\
16.31	0.00823326126213129\\
16.32	0.0082332901608468\\
16.33	0.00823331907152589\\
16.34	0.00823334799417419\\
16.35	0.00823337692879737\\
16.36	0.00823340587540107\\
16.37	0.00823343483399093\\
16.38	0.00823346380457263\\
16.39	0.00823349278715181\\
16.4	0.00823352178173415\\
16.41	0.00823355078832531\\
16.42	0.00823357980693096\\
16.43	0.00823360883755678\\
16.44	0.00823363788020842\\
16.45	0.00823366693489159\\
16.46	0.00823369600161195\\
16.47	0.00823372508037519\\
16.48	0.008233754171187\\
16.49	0.00823378327405308\\
16.5	0.00823381238897911\\
16.51	0.0082338415159708\\
16.52	0.00823387065503384\\
16.53	0.00823389980617394\\
16.54	0.0082339289693968\\
16.55	0.00823395814470813\\
16.56	0.00823398733211365\\
16.57	0.00823401653161907\\
16.58	0.00823404574323012\\
16.59	0.0082340749669525\\
16.6	0.00823410420279196\\
16.61	0.00823413345075422\\
16.62	0.008234162710845\\
16.63	0.00823419198307005\\
16.64	0.0082342212674351\\
16.65	0.00823425056394589\\
16.66	0.00823427987260817\\
16.67	0.00823430919342768\\
16.68	0.00823433852641018\\
16.69	0.00823436787156142\\
16.7	0.00823439722888715\\
16.71	0.00823442659839313\\
16.72	0.00823445598008513\\
16.73	0.00823448537396892\\
16.74	0.00823451478005025\\
16.75	0.00823454419833491\\
16.76	0.00823457362882867\\
16.77	0.00823460307153731\\
16.78	0.00823463252646661\\
16.79	0.00823466199362235\\
16.8	0.00823469147301032\\
16.81	0.00823472096463632\\
16.82	0.00823475046850614\\
16.83	0.00823477998462557\\
16.84	0.00823480951300042\\
16.85	0.00823483905363649\\
16.86	0.00823486860653959\\
16.87	0.00823489817171552\\
16.88	0.0082349277491701\\
16.89	0.00823495733890915\\
16.9	0.00823498694093849\\
16.91	0.00823501655526394\\
16.92	0.00823504618189132\\
16.93	0.00823507582082647\\
16.94	0.0082351054720752\\
16.95	0.00823513513564338\\
16.96	0.00823516481153682\\
16.97	0.00823519449976138\\
16.98	0.00823522420032289\\
16.99	0.00823525391322721\\
17	0.00823528363848018\\
17.01	0.00823531337608766\\
17.02	0.00823534312605551\\
17.03	0.0082353728883896\\
17.04	0.00823540266309577\\
17.05	0.0082354324501799\\
17.06	0.00823546224964787\\
17.07	0.00823549206150554\\
17.08	0.0082355218857588\\
17.09	0.00823555172241351\\
17.1	0.00823558157147556\\
17.11	0.00823561143295085\\
17.12	0.00823564130684525\\
17.13	0.00823567119316467\\
17.14	0.00823570109191499\\
17.15	0.00823573100310213\\
17.16	0.00823576092673197\\
17.17	0.00823579086281043\\
17.18	0.00823582081134341\\
17.19	0.00823585077233683\\
17.2	0.0082358807457966\\
17.21	0.00823591073172865\\
17.22	0.00823594073013888\\
17.23	0.00823597074103322\\
17.24	0.00823600076441761\\
17.25	0.00823603080029798\\
17.26	0.00823606084868025\\
17.27	0.00823609090957037\\
17.28	0.00823612098297428\\
17.29	0.00823615106889792\\
17.3	0.00823618116734724\\
17.31	0.00823621127832819\\
17.32	0.00823624140184672\\
17.33	0.00823627153790879\\
17.34	0.00823630168652037\\
17.35	0.0082363318476874\\
17.36	0.00823636202141587\\
17.37	0.00823639220771174\\
17.38	0.00823642240658098\\
17.39	0.00823645261802956\\
17.4	0.00823648284206348\\
17.41	0.00823651307868871\\
17.42	0.00823654332791124\\
17.43	0.00823657358973705\\
17.44	0.00823660386417214\\
17.45	0.00823663415122252\\
17.46	0.00823666445089416\\
17.47	0.00823669476319308\\
17.48	0.00823672508812528\\
17.49	0.00823675542569678\\
17.5	0.00823678577591357\\
17.51	0.00823681613878168\\
17.52	0.00823684651430713\\
17.53	0.00823687690249594\\
17.54	0.00823690730335413\\
17.55	0.00823693771688774\\
17.56	0.00823696814310278\\
17.57	0.0082369985820053\\
17.58	0.00823702903360134\\
17.59	0.00823705949789694\\
17.6	0.00823708997489813\\
17.61	0.00823712046461098\\
17.62	0.00823715096704152\\
17.63	0.00823718148219582\\
17.64	0.00823721201007993\\
17.65	0.00823724255069992\\
17.66	0.00823727310406184\\
17.67	0.00823730367017177\\
17.68	0.00823733424903577\\
17.69	0.00823736484065993\\
17.7	0.0082373954450503\\
17.71	0.00823742606221299\\
17.72	0.00823745669215406\\
17.73	0.00823748733487961\\
17.74	0.00823751799039574\\
17.75	0.00823754865870852\\
17.76	0.00823757933982406\\
17.77	0.00823761003374846\\
17.78	0.00823764074048782\\
17.79	0.00823767146004825\\
17.8	0.00823770219243586\\
17.81	0.00823773293765677\\
17.82	0.00823776369571708\\
17.83	0.00823779446662293\\
17.84	0.00823782525038042\\
17.85	0.00823785604699569\\
17.86	0.00823788685647487\\
17.87	0.0082379176788241\\
17.88	0.0082379485140495\\
17.89	0.00823797936215722\\
17.9	0.0082380102231534\\
17.91	0.00823804109704419\\
17.92	0.00823807198383573\\
17.93	0.00823810288353419\\
17.94	0.00823813379614571\\
17.95	0.00823816472167646\\
17.96	0.00823819566013259\\
17.97	0.00823822661152028\\
17.98	0.0082382575758457\\
17.99	0.00823828855311501\\
18	0.0082383195433344\\
18.01	0.00823835054651003\\
18.02	0.00823838156264811\\
18.03	0.0082384125917548\\
18.04	0.00823844363383631\\
18.05	0.00823847468889882\\
18.06	0.00823850575694853\\
18.07	0.00823853683799164\\
18.08	0.00823856793203435\\
18.09	0.00823859903908287\\
18.1	0.00823863015914341\\
18.11	0.00823866129222217\\
18.12	0.00823869243832539\\
18.13	0.00823872359745927\\
18.14	0.00823875476963004\\
18.15	0.00823878595484392\\
18.16	0.00823881715310714\\
18.17	0.00823884836442595\\
18.18	0.00823887958880656\\
18.19	0.00823891082625522\\
18.2	0.00823894207677818\\
18.21	0.00823897334038168\\
18.22	0.00823900461707197\\
18.23	0.0082390359068553\\
18.24	0.00823906720973793\\
18.25	0.00823909852572612\\
18.26	0.00823912985482613\\
18.27	0.00823916119704423\\
18.28	0.00823919255238668\\
18.29	0.00823922392085977\\
18.3	0.00823925530246975\\
18.31	0.00823928669722293\\
18.32	0.00823931810512558\\
18.33	0.00823934952618397\\
18.34	0.00823938096040441\\
18.35	0.00823941240779319\\
18.36	0.0082394438683566\\
18.37	0.00823947534210094\\
18.38	0.00823950682903253\\
18.39	0.00823953832915765\\
18.4	0.00823956984248263\\
18.41	0.00823960136901377\\
18.42	0.00823963290875739\\
18.43	0.00823966446171982\\
18.44	0.00823969602790737\\
18.45	0.00823972760732638\\
18.46	0.00823975919998316\\
18.47	0.00823979080588406\\
18.48	0.00823982242503542\\
18.49	0.00823985405744357\\
18.5	0.00823988570311485\\
18.51	0.00823991736205561\\
18.52	0.00823994903427221\\
18.53	0.00823998071977099\\
18.54	0.00824001241855832\\
18.55	0.00824004413064056\\
18.56	0.00824007585602406\\
18.57	0.00824010759471519\\
18.58	0.00824013934672034\\
18.59	0.00824017111204586\\
18.6	0.00824020289069814\\
18.61	0.00824023468268356\\
18.62	0.0082402664880085\\
18.63	0.00824029830667936\\
18.64	0.00824033013870252\\
18.65	0.00824036198408436\\
18.66	0.0082403938428313\\
18.67	0.00824042571494974\\
18.68	0.00824045760044607\\
18.69	0.00824048949932672\\
18.7	0.00824052141159807\\
18.71	0.00824055333726656\\
18.72	0.0082405852763386\\
18.73	0.00824061722882062\\
18.74	0.00824064919471903\\
18.75	0.00824068117404027\\
18.76	0.00824071316679077\\
18.77	0.00824074517297695\\
18.78	0.00824077719260527\\
18.79	0.00824080922568216\\
18.8	0.00824084127221408\\
18.81	0.00824087333220746\\
18.82	0.00824090540566876\\
18.83	0.00824093749260443\\
18.84	0.00824096959302094\\
18.85	0.00824100170692475\\
18.86	0.00824103383432231\\
18.87	0.00824106597522012\\
18.88	0.00824109812962463\\
18.89	0.00824113029754231\\
18.9	0.00824116247897966\\
18.91	0.00824119467394315\\
18.92	0.00824122688243928\\
18.93	0.00824125910447452\\
18.94	0.00824129134005538\\
18.95	0.00824132358918834\\
18.96	0.00824135585187992\\
18.97	0.00824138812813661\\
18.98	0.00824142041796492\\
18.99	0.00824145272137137\\
19	0.00824148503836245\\
19.01	0.0082415173689447\\
19.02	0.00824154971312464\\
19.03	0.00824158207090878\\
19.04	0.00824161444230366\\
19.05	0.0082416468273158\\
19.06	0.00824167922595174\\
19.07	0.00824171163821802\\
19.08	0.00824174406412119\\
19.09	0.00824177650366778\\
19.1	0.00824180895686433\\
19.11	0.00824184142371742\\
19.12	0.00824187390423358\\
19.13	0.00824190639841939\\
19.14	0.00824193890628139\\
19.15	0.00824197142782616\\
19.16	0.00824200396306027\\
19.17	0.00824203651199028\\
19.18	0.00824206907462278\\
19.19	0.00824210165096434\\
19.2	0.00824213424102154\\
19.21	0.00824216684480098\\
19.22	0.00824219946230924\\
19.23	0.00824223209355291\\
19.24	0.00824226473853859\\
19.25	0.00824229739727289\\
19.26	0.0082423300697624\\
19.27	0.00824236275601373\\
19.28	0.00824239545603349\\
19.29	0.0082424281698283\\
19.3	0.00824246089740477\\
19.31	0.00824249363876953\\
19.32	0.00824252639392919\\
19.33	0.00824255916289039\\
19.34	0.00824259194565975\\
19.35	0.00824262474224392\\
19.36	0.00824265755264953\\
19.37	0.00824269037688322\\
19.38	0.00824272321495164\\
19.39	0.00824275606686143\\
19.4	0.00824278893261925\\
19.41	0.00824282181223175\\
19.42	0.0082428547057056\\
19.43	0.00824288761304745\\
19.44	0.00824292053426397\\
19.45	0.00824295346936183\\
19.46	0.0082429864183477\\
19.47	0.00824301938122826\\
19.48	0.00824305235801019\\
19.49	0.00824308534870016\\
19.5	0.00824311835330488\\
19.51	0.00824315137183103\\
19.52	0.0082431844042853\\
19.53	0.00824321745067439\\
19.54	0.008243250511005\\
19.55	0.00824328358528384\\
19.56	0.00824331667351761\\
19.57	0.00824334977571302\\
19.58	0.00824338289187679\\
19.59	0.00824341602201564\\
19.6	0.00824344916613628\\
19.61	0.00824348232424545\\
19.62	0.00824351549634987\\
19.63	0.00824354868245628\\
19.64	0.0082435818825714\\
19.65	0.00824361509670198\\
19.66	0.00824364832485477\\
19.67	0.0082436815670365\\
19.68	0.00824371482325393\\
19.69	0.00824374809351381\\
19.7	0.00824378137782289\\
19.71	0.00824381467618795\\
19.72	0.00824384798861573\\
19.73	0.00824388131511301\\
19.74	0.00824391465568656\\
19.75	0.00824394801034315\\
19.76	0.00824398137908956\\
19.77	0.00824401476193257\\
19.78	0.00824404815887897\\
19.79	0.00824408156993554\\
19.8	0.00824411499510907\\
19.81	0.00824414843440636\\
19.82	0.00824418188783422\\
19.83	0.00824421535539943\\
19.84	0.00824424883710881\\
19.85	0.00824428233296916\\
19.86	0.00824431584298731\\
19.87	0.00824434936717005\\
19.88	0.00824438290552422\\
19.89	0.00824441645805664\\
19.9	0.00824445002477413\\
19.91	0.00824448360568353\\
19.92	0.00824451720079166\\
19.93	0.00824455081010536\\
19.94	0.00824458443363148\\
19.95	0.00824461807137687\\
19.96	0.00824465172334836\\
19.97	0.0082446853895528\\
19.98	0.00824471906999706\\
19.99	0.008244752764688\\
20	0.00824478647363247\\
20.01	0.00824482019683733\\
20.02	0.00824485393430946\\
20.03	0.00824488768605574\\
20.04	0.00824492145208302\\
20.05	0.00824495523239821\\
20.06	0.00824498902700817\\
20.07	0.0082450228359198\\
20.08	0.00824505665913998\\
20.09	0.00824509049667561\\
20.1	0.00824512434853359\\
20.11	0.00824515821472081\\
20.12	0.00824519209524418\\
20.13	0.00824522599011061\\
20.14	0.00824525989932701\\
20.15	0.00824529382290029\\
20.16	0.00824532776083737\\
20.17	0.00824536171314518\\
20.18	0.00824539567983063\\
20.19	0.00824542966090066\\
20.2	0.0082454636563622\\
20.21	0.00824549766622218\\
20.22	0.00824553169048755\\
20.23	0.00824556572916525\\
20.24	0.00824559978226222\\
20.25	0.00824563384978541\\
20.26	0.00824566793174178\\
20.27	0.00824570202813829\\
20.28	0.00824573613898189\\
20.29	0.00824577026427955\\
20.3	0.00824580440403824\\
20.31	0.00824583855826492\\
20.32	0.00824587272696658\\
20.33	0.00824590691015019\\
20.34	0.00824594110782273\\
20.35	0.00824597531999119\\
20.36	0.00824600954666255\\
20.37	0.00824604378784382\\
20.38	0.00824607804354197\\
20.39	0.00824611231376403\\
20.4	0.00824614659851698\\
20.41	0.00824618089780783\\
20.42	0.0082462152116436\\
20.43	0.0082462495400313\\
20.44	0.00824628388297794\\
20.45	0.00824631824049054\\
20.46	0.00824635261257614\\
20.47	0.00824638699924175\\
20.48	0.00824642140049441\\
20.49	0.00824645581634116\\
20.5	0.00824649024678903\\
20.51	0.00824652469184506\\
20.52	0.0082465591515163\\
20.53	0.00824659362580981\\
20.54	0.00824662811473262\\
20.55	0.0082466626182918\\
20.56	0.00824669713649441\\
20.57	0.0082467316693475\\
20.58	0.00824676621685816\\
20.59	0.00824680077903344\\
20.6	0.00824683535588042\\
20.61	0.00824686994740618\\
20.62	0.0082469045536178\\
20.63	0.00824693917452237\\
20.64	0.00824697381012696\\
20.65	0.00824700846043869\\
20.66	0.00824704312546463\\
20.67	0.00824707780521189\\
20.68	0.00824711249968756\\
20.69	0.00824714720889877\\
20.7	0.00824718193285261\\
20.71	0.00824721667155619\\
20.72	0.00824725142501665\\
20.73	0.00824728619324109\\
20.74	0.00824732097623663\\
20.75	0.00824735577401042\\
20.76	0.00824739058656956\\
20.77	0.00824742541392121\\
20.78	0.0082474602560725\\
20.79	0.00824749511303056\\
20.8	0.00824752998480256\\
20.81	0.00824756487139562\\
20.82	0.00824759977281691\\
20.83	0.00824763468907358\\
20.84	0.00824766962017279\\
20.85	0.0082477045661217\\
20.86	0.00824773952692748\\
20.87	0.0082477745025973\\
20.88	0.00824780949313833\\
20.89	0.00824784449855775\\
20.9	0.00824787951886274\\
20.91	0.00824791455406048\\
20.92	0.00824794960415816\\
20.93	0.00824798466916298\\
20.94	0.00824801974908212\\
20.95	0.00824805484392278\\
20.96	0.00824808995369217\\
20.97	0.00824812507839749\\
20.98	0.00824816021804596\\
20.99	0.00824819537264478\\
21	0.00824823054220116\\
21.01	0.00824826572672234\\
21.02	0.00824830092621552\\
21.03	0.00824833614068795\\
21.04	0.00824837137014684\\
21.05	0.00824840661459944\\
21.06	0.00824844187405298\\
21.07	0.0082484771485147\\
21.08	0.00824851243799185\\
21.09	0.00824854774249167\\
21.1	0.00824858306202141\\
21.11	0.00824861839658834\\
21.12	0.0082486537461997\\
21.13	0.00824868911086277\\
21.14	0.0082487244905848\\
21.15	0.00824875988537307\\
21.16	0.00824879529523485\\
21.17	0.00824883072017741\\
21.18	0.00824886616020805\\
21.19	0.00824890161533403\\
21.2	0.00824893708556265\\
21.21	0.0082489725709012\\
21.22	0.00824900807135697\\
21.23	0.00824904358693726\\
21.24	0.00824907911764938\\
21.25	0.00824911466350061\\
21.26	0.00824915022449829\\
21.27	0.00824918580064971\\
21.28	0.0082492213919622\\
21.29	0.00824925699844307\\
21.3	0.00824929262009965\\
21.31	0.00824932825693926\\
21.32	0.00824936390896922\\
21.33	0.00824939957619689\\
21.34	0.00824943525862959\\
21.35	0.00824947095627466\\
21.36	0.00824950666913945\\
21.37	0.0082495423972313\\
21.38	0.00824957814055757\\
21.39	0.00824961389912562\\
21.4	0.00824964967294278\\
21.41	0.00824968546201645\\
21.42	0.00824972126635397\\
21.43	0.00824975708596272\\
21.44	0.00824979292085006\\
21.45	0.00824982877102338\\
21.46	0.00824986463649006\\
21.47	0.00824990051725748\\
21.48	0.00824993641333302\\
21.49	0.00824997232472408\\
21.5	0.00825000825143805\\
21.51	0.00825004419348233\\
21.52	0.00825008015086431\\
21.53	0.00825011612359141\\
21.54	0.00825015211167104\\
21.55	0.00825018811511059\\
21.56	0.0082502241339175\\
21.57	0.00825026016809918\\
21.58	0.00825029621766304\\
21.59	0.00825033228261652\\
21.6	0.00825036836296706\\
21.61	0.00825040445872207\\
21.62	0.008250440569889\\
21.63	0.00825047669647529\\
21.64	0.00825051283848838\\
21.65	0.00825054899593571\\
21.66	0.00825058516882475\\
21.67	0.00825062135716294\\
21.68	0.00825065756095775\\
21.69	0.00825069378021663\\
21.7	0.00825073001494705\\
21.71	0.00825076626515648\\
21.72	0.00825080253085239\\
21.73	0.00825083881204226\\
21.74	0.00825087510873357\\
21.75	0.00825091142093379\\
21.76	0.00825094774865042\\
21.77	0.00825098409189095\\
21.78	0.00825102045066288\\
21.79	0.00825105682497369\\
21.8	0.00825109321483089\\
21.81	0.00825112962024198\\
21.82	0.00825116604121448\\
21.83	0.00825120247775589\\
21.84	0.00825123892987372\\
21.85	0.00825127539757551\\
21.86	0.00825131188086876\\
21.87	0.00825134837976101\\
21.88	0.00825138489425979\\
21.89	0.00825142142437263\\
21.9	0.00825145797010706\\
21.91	0.00825149453147062\\
21.92	0.00825153110847086\\
21.93	0.00825156770111533\\
21.94	0.00825160430941158\\
21.95	0.00825164093336716\\
21.96	0.00825167757298962\\
21.97	0.00825171422828654\\
21.98	0.00825175089926547\\
21.99	0.00825178758593399\\
22	0.00825182428829966\\
22.01	0.00825186100637006\\
22.02	0.00825189774015278\\
22.03	0.00825193448965539\\
22.04	0.00825197125488548\\
22.05	0.00825200803585065\\
22.06	0.00825204483255847\\
22.07	0.00825208164501656\\
22.08	0.00825211847323251\\
22.09	0.00825215531721393\\
22.1	0.00825219217696843\\
22.11	0.0082522290525036\\
22.12	0.00825226594382708\\
22.13	0.00825230285094648\\
22.14	0.00825233977386942\\
22.15	0.00825237671260353\\
22.16	0.00825241366715643\\
22.17	0.00825245063753576\\
22.18	0.00825248762374916\\
22.19	0.00825252462580427\\
22.2	0.00825256164370872\\
22.21	0.00825259867747017\\
22.22	0.00825263572709626\\
22.23	0.00825267279259466\\
22.24	0.00825270987397301\\
22.25	0.00825274697123899\\
22.26	0.00825278408440024\\
22.27	0.00825282121346445\\
22.28	0.00825285835843929\\
22.29	0.00825289551933242\\
22.3	0.00825293269615154\\
22.31	0.00825296988890432\\
22.32	0.00825300709759845\\
22.33	0.00825304432224161\\
22.34	0.00825308156284152\\
22.35	0.00825311881940584\\
22.36	0.0082531560919423\\
22.37	0.00825319338045859\\
22.38	0.00825323068496243\\
22.39	0.00825326800546151\\
22.4	0.00825330534196357\\
22.41	0.00825334269447631\\
22.42	0.00825338006300745\\
22.43	0.00825341744756474\\
22.44	0.00825345484815588\\
22.45	0.00825349226478862\\
22.46	0.0082535296974707\\
22.47	0.00825356714620985\\
22.48	0.00825360461101381\\
22.49	0.00825364209189033\\
22.5	0.00825367958884717\\
22.51	0.00825371710189207\\
22.52	0.0082537546310328\\
22.53	0.00825379217627712\\
22.54	0.00825382973763278\\
22.55	0.00825386731510757\\
22.56	0.00825390490870924\\
22.57	0.00825394251844559\\
22.58	0.00825398014432439\\
22.59	0.00825401778635341\\
22.6	0.00825405544454045\\
22.61	0.00825409311889329\\
22.62	0.00825413080941974\\
22.63	0.00825416851612759\\
22.64	0.00825420623902463\\
22.65	0.00825424397811867\\
22.66	0.00825428173341753\\
22.67	0.008254319504929\\
22.68	0.00825435729266092\\
22.69	0.00825439509662109\\
22.7	0.00825443291681735\\
22.71	0.0082544707532575\\
22.72	0.00825450860594939\\
22.73	0.00825454647490085\\
22.74	0.00825458436011971\\
22.75	0.00825462226161382\\
22.76	0.00825466017939101\\
22.77	0.00825469811345914\\
22.78	0.00825473606382606\\
22.79	0.00825477403049963\\
22.8	0.00825481201348769\\
22.81	0.00825485001279812\\
22.82	0.00825488802843878\\
22.83	0.00825492606041754\\
22.84	0.00825496410874226\\
22.85	0.00825500217342084\\
22.86	0.00825504025446114\\
22.87	0.00825507835187106\\
22.88	0.00825511646565848\\
22.89	0.00825515459583128\\
22.9	0.00825519274239738\\
22.91	0.00825523090536465\\
22.92	0.00825526908474102\\
22.93	0.00825530728053438\\
22.94	0.00825534549275263\\
22.95	0.0082553837214037\\
22.96	0.00825542196649551\\
22.97	0.00825546022803596\\
22.98	0.00825549850603299\\
22.99	0.00825553680049452\\
23	0.00825557511142849\\
23.01	0.00825561343884282\\
23.02	0.00825565178274546\\
23.03	0.00825569014314436\\
23.04	0.00825572852004745\\
23.05	0.00825576691346268\\
23.06	0.00825580532339801\\
23.07	0.0082558437498614\\
23.08	0.00825588219286081\\
23.09	0.00825592065240419\\
23.1	0.00825595912849953\\
23.11	0.00825599762115479\\
23.12	0.00825603613037794\\
23.13	0.00825607465617697\\
23.14	0.00825611319855986\\
23.15	0.00825615175753459\\
23.16	0.00825619033310916\\
23.17	0.00825622892529156\\
23.18	0.00825626753408978\\
23.19	0.00825630615951183\\
23.2	0.00825634480156572\\
23.21	0.00825638346025945\\
23.22	0.00825642213560104\\
23.23	0.0082564608275985\\
23.24	0.00825649953625985\\
23.25	0.00825653826159312\\
23.26	0.00825657700360633\\
23.27	0.00825661576230752\\
23.28	0.00825665453770472\\
23.29	0.00825669332980598\\
23.3	0.00825673213861932\\
23.31	0.0082567709641528\\
23.32	0.00825680980641449\\
23.33	0.00825684866541241\\
23.34	0.00825688754115463\\
23.35	0.00825692643364923\\
23.36	0.00825696534290425\\
23.37	0.00825700426892777\\
23.38	0.00825704321172786\\
23.39	0.00825708217131261\\
23.4	0.0082571211476901\\
23.41	0.0082571601408684\\
23.42	0.00825719915085561\\
23.43	0.00825723817765981\\
23.44	0.00825727722128912\\
23.45	0.00825731628175161\\
23.46	0.00825735535905541\\
23.47	0.00825739445320863\\
23.48	0.00825743356421936\\
23.49	0.00825747269209573\\
23.5	0.00825751183684586\\
23.51	0.00825755099847787\\
23.52	0.0082575901769999\\
23.53	0.00825762937242006\\
23.54	0.00825766858474651\\
23.55	0.00825770781398737\\
23.56	0.0082577470601508\\
23.57	0.00825778632324494\\
23.58	0.00825782560327794\\
23.59	0.00825786490025796\\
23.6	0.00825790421419316\\
23.61	0.0082579435450917\\
23.62	0.00825798289296175\\
23.63	0.00825802225781149\\
23.64	0.00825806163964908\\
23.65	0.00825810103848272\\
23.66	0.00825814045432059\\
23.67	0.00825817988717086\\
23.68	0.00825821933704174\\
23.69	0.00825825880394142\\
23.7	0.00825829828787811\\
23.71	0.00825833778886001\\
23.72	0.00825837730689533\\
23.73	0.00825841684199228\\
23.74	0.00825845639415908\\
23.75	0.00825849596340395\\
23.76	0.00825853554973512\\
23.77	0.00825857515316082\\
23.78	0.00825861477368929\\
23.79	0.00825865441132876\\
23.8	0.00825869406608748\\
23.81	0.00825873373797369\\
23.82	0.00825877342699565\\
23.83	0.00825881313316161\\
23.84	0.00825885285647983\\
23.85	0.00825889259695859\\
23.86	0.00825893235460614\\
23.87	0.00825897212943077\\
23.88	0.00825901192144075\\
23.89	0.00825905173064436\\
23.9	0.0082590915570499\\
23.91	0.00825913140066564\\
23.92	0.0082591712614999\\
23.93	0.00825921113956098\\
23.94	0.00825925103485717\\
23.95	0.00825929094739678\\
23.96	0.00825933087718814\\
23.97	0.00825937082423956\\
23.98	0.00825941078855937\\
23.99	0.00825945077015589\\
24	0.00825949076903746\\
24.01	0.00825953078521241\\
24.02	0.0082595708186891\\
24.03	0.00825961086947586\\
24.04	0.00825965093758105\\
24.05	0.00825969102301303\\
24.06	0.00825973112578017\\
24.07	0.00825977124589081\\
24.08	0.00825981138335334\\
24.09	0.00825985153817614\\
24.1	0.00825989171036759\\
24.11	0.00825993189993608\\
24.12	0.00825997210688999\\
24.13	0.00826001233123772\\
24.14	0.00826005257298768\\
24.15	0.00826009283214827\\
24.16	0.00826013310872792\\
24.17	0.00826017340273502\\
24.18	0.00826021371417801\\
24.19	0.00826025404306532\\
24.2	0.00826029438940537\\
24.21	0.00826033475320661\\
24.22	0.00826037513447749\\
24.23	0.00826041553322645\\
24.24	0.00826045594946194\\
24.25	0.00826049638319242\\
24.26	0.00826053683442637\\
24.27	0.00826057730317226\\
24.28	0.00826061778943855\\
24.29	0.00826065829323374\\
24.3	0.00826069881456632\\
24.31	0.00826073935344477\\
24.32	0.00826077990987759\\
24.33	0.0082608204838733\\
24.34	0.00826086107544041\\
24.35	0.00826090168458743\\
24.36	0.00826094231132288\\
24.37	0.0082609829556553\\
24.38	0.00826102361759323\\
24.39	0.00826106429714519\\
24.4	0.00826110499431975\\
24.41	0.00826114570912545\\
24.42	0.00826118644157086\\
24.43	0.00826122719166454\\
24.44	0.00826126795941508\\
24.45	0.00826130874483103\\
24.46	0.00826134954792099\\
24.47	0.00826139036869356\\
24.48	0.00826143120715734\\
24.49	0.00826147206332093\\
24.5	0.00826151293719293\\
24.51	0.00826155382878198\\
24.52	0.0082615947380967\\
24.53	0.00826163566514572\\
24.54	0.00826167660993768\\
24.55	0.00826171757248123\\
24.56	0.00826175855278502\\
24.57	0.00826179955085772\\
24.58	0.008261840566708\\
24.59	0.00826188160034452\\
24.6	0.00826192265177598\\
24.61	0.00826196372101106\\
24.62	0.00826200480805847\\
24.63	0.00826204591292691\\
24.64	0.0082620870356251\\
24.65	0.00826212817616176\\
24.66	0.00826216933454561\\
24.67	0.0082622105107854\\
24.68	0.00826225170488986\\
24.69	0.00826229291686777\\
24.7	0.00826233414672788\\
24.71	0.00826237539447895\\
24.72	0.00826241666012978\\
24.73	0.00826245794368914\\
24.74	0.00826249924516583\\
24.75	0.00826254056456866\\
24.76	0.00826258190190645\\
24.77	0.00826262325718801\\
24.78	0.00826266463042218\\
24.79	0.0082627060216178\\
24.8	0.00826274743078372\\
24.81	0.0082627888579288\\
24.82	0.00826283030306191\\
24.83	0.00826287176619193\\
24.84	0.00826291324732774\\
24.85	0.00826295474647825\\
24.86	0.00826299626365237\\
24.87	0.00826303779885901\\
24.88	0.0082630793521071\\
24.89	0.00826312092340559\\
24.9	0.00826316251276342\\
24.91	0.00826320412018955\\
24.92	0.00826324574569296\\
24.93	0.00826328738928262\\
24.94	0.00826332905096754\\
24.95	0.00826337073075671\\
24.96	0.00826341242865914\\
24.97	0.00826345414468388\\
24.98	0.00826349587883995\\
24.99	0.00826353763113641\\
25	0.00826357940158232\\
25.01	0.00826362119018675\\
25.02	0.00826366299695879\\
25.03	0.00826370482190756\\
25.04	0.00826374666504214\\
25.05	0.00826378852637167\\
25.06	0.00826383040590529\\
25.07	0.00826387230365215\\
25.08	0.00826391421962141\\
25.09	0.00826395615382226\\
25.1	0.00826399810626388\\
25.11	0.00826404007695548\\
25.12	0.00826408206590627\\
25.13	0.00826412407312551\\
25.14	0.00826416609862242\\
25.15	0.00826420814240629\\
25.16	0.00826425020448638\\
25.17	0.00826429228487199\\
25.18	0.00826433438357243\\
25.19	0.00826437650059703\\
25.2	0.00826441863595512\\
25.21	0.00826446078965606\\
25.22	0.00826450296170923\\
25.23	0.00826454515212403\\
25.24	0.00826458736090984\\
25.25	0.0082646295880761\\
25.26	0.00826467183363225\\
25.27	0.00826471409758776\\
25.28	0.00826475637995209\\
25.29	0.00826479868073473\\
25.3	0.00826484099994522\\
25.31	0.00826488333759308\\
25.32	0.00826492569368785\\
25.33	0.00826496806823912\\
25.34	0.00826501046125648\\
25.35	0.00826505287274952\\
25.36	0.00826509530272789\\
25.37	0.00826513775120125\\
25.38	0.00826518021817924\\
25.39	0.00826522270367159\\
25.4	0.008265265207688\\
25.41	0.0082653077302382\\
25.42	0.00826535027133196\\
25.43	0.00826539283097906\\
25.44	0.00826543540918931\\
25.45	0.00826547800597254\\
25.46	0.0082655206213386\\
25.47	0.00826556325529736\\
25.48	0.00826560590785874\\
25.49	0.00826564857903266\\
25.5	0.00826569126882906\\
25.51	0.00826573397725793\\
25.52	0.00826577670432928\\
25.53	0.00826581945005312\\
25.54	0.00826586221443954\\
25.55	0.00826590499749861\\
25.56	0.00826594779924044\\
25.57	0.00826599061967517\\
25.58	0.00826603345881299\\
25.59	0.00826607631666409\\
25.6	0.0082661191932387\\
25.61	0.00826616208854709\\
25.62	0.00826620500259955\\
25.63	0.00826624793540641\\
25.64	0.00826629088697801\\
25.65	0.00826633385732476\\
25.66	0.00826637684645707\\
25.67	0.00826641985438541\\
25.68	0.00826646288112026\\
25.69	0.00826650592667214\\
25.7	0.00826654899105164\\
25.71	0.00826659207426933\\
25.72	0.00826663517633585\\
25.73	0.00826667829726189\\
25.74	0.00826672143705815\\
25.75	0.00826676459573537\\
25.76	0.00826680777330435\\
25.77	0.00826685096977593\\
25.78	0.00826689418516095\\
25.79	0.00826693741947035\\
25.8	0.00826698067271508\\
25.81	0.00826702394490612\\
25.82	0.00826706723605453\\
25.83	0.00826711054617139\\
25.84	0.00826715387526783\\
25.85	0.00826719722335504\\
25.86	0.00826724059044422\\
25.87	0.00826728397654666\\
25.88	0.00826732738167368\\
25.89	0.00826737080583665\\
25.9	0.00826741424904699\\
25.91	0.00826745771131617\\
25.92	0.00826750119265572\\
25.93	0.00826754469307723\\
25.94	0.00826758821259231\\
25.95	0.00826763175121266\\
25.96	0.00826767530895003\\
25.97	0.00826771888581621\\
25.98	0.00826776248182307\\
25.99	0.00826780609698253\\
26	0.00826784973130656\\
26.01	0.00826789338480721\\
26.02	0.00826793705749658\\
26.03	0.00826798074938684\\
26.04	0.00826802446049022\\
26.05	0.00826806819081903\\
26.06	0.00826811194038562\\
26.07	0.00826815570920244\\
26.08	0.00826819949728198\\
26.09	0.00826824330463683\\
26.1	0.00826828713127962\\
26.11	0.00826833097722309\\
26.12	0.00826837484248004\\
26.13	0.00826841872706332\\
26.14	0.00826846263098591\\
26.15	0.00826850655426082\\
26.16	0.00826855049690117\\
26.17	0.00826859445892015\\
26.18	0.00826863844033104\\
26.19	0.0082686824411472\\
26.2	0.0082687264613821\\
26.21	0.00826877050104925\\
26.22	0.0082688145601623\\
26.23	0.00826885863873495\\
26.24	0.00826890273678104\\
26.25	0.00826894685431446\\
26.26	0.00826899099134923\\
26.27	0.00826903514789945\\
26.28	0.00826907932397931\\
26.29	0.00826912351960316\\
26.3	0.00826916773478537\\
26.31	0.00826921196954049\\
26.32	0.00826925622388314\\
26.33	0.00826930049782806\\
26.34	0.00826934479139009\\
26.35	0.00826938910458421\\
26.36	0.0082694334374255\\
26.37	0.00826947778992917\\
26.38	0.00826952216211053\\
26.39	0.00826956655398504\\
26.4	0.00826961096556826\\
26.41	0.00826965539687591\\
26.42	0.00826969984792381\\
26.43	0.00826974431872793\\
26.44	0.00826978880930436\\
26.45	0.00826983331966935\\
26.46	0.00826987784983929\\
26.47	0.00826992239983069\\
26.48	0.00826996696966022\\
26.49	0.0082700115593447\\
26.5	0.0082700561689011\\
26.51	0.00827010079834655\\
26.52	0.00827014544769833\\
26.53	0.00827019011697388\\
26.54	0.00827023480619082\\
26.55	0.0082702795153669\\
26.56	0.00827032424452009\\
26.57	0.00827036899366849\\
26.58	0.0082704137628304\\
26.59	0.00827045855202428\\
26.6	0.0082705033612688\\
26.61	0.00827054819058279\\
26.62	0.00827059303998529\\
26.63	0.00827063790949552\\
26.64	0.00827068279913291\\
26.65	0.00827072770891708\\
26.66	0.00827077263886786\\
26.67	0.00827081758900529\\
26.68	0.00827086255934961\\
26.69	0.00827090754992129\\
26.7	0.00827095256074103\\
26.71	0.00827099759182973\\
26.72	0.00827104264320854\\
26.73	0.00827108771489883\\
26.74	0.00827113280692223\\
26.75	0.00827117791930057\\
26.76	0.00827122305205598\\
26.77	0.0082712682052108\\
26.78	0.00827131337878766\\
26.79	0.00827135857280941\\
26.8	0.00827140378729922\\
26.81	0.00827144902228047\\
26.82	0.00827149427777689\\
26.83	0.00827153955381242\\
26.84	0.00827158485041133\\
26.85	0.00827163016759818\\
26.86	0.00827167550539781\\
26.87	0.00827172086383538\\
26.88	0.00827176624293636\\
26.89	0.00827181164272653\\
26.9	0.00827185706323198\\
26.91	0.00827190250447916\\
26.92	0.00827194796649482\\
26.93	0.00827199344930608\\
26.94	0.00827203895294038\\
26.95	0.00827208447742553\\
26.96	0.0082721300227897\\
26.97	0.00827217558906142\\
26.98	0.0082722211762696\\
26.99	0.00827226678444353\\
27	0.00827231241361289\\
27.01	0.00827235806380775\\
27.02	0.00827240373505857\\
27.03	0.00827244942739627\\
27.04	0.00827249514085212\\
27.05	0.00827254087545788\\
27.06	0.00827258663124571\\
27.07	0.0082726324082482\\
27.08	0.00827267820649844\\
27.09	0.00827272402602993\\
27.1	0.00827276986687667\\
27.11	0.00827281572907311\\
27.12	0.00827286161265422\\
27.13	0.00827290751765543\\
27.14	0.00827295344411271\\
27.15	0.00827299939206251\\
27.16	0.00827304536154182\\
27.17	0.00827309135258818\\
27.18	0.00827313736523963\\
27.19	0.00827318339953481\\
27.2	0.0082732294555129\\
27.21	0.00827327553321365\\
27.22	0.00827332163267741\\
27.23	0.00827336775394512\\
27.24	0.00827341389705833\\
27.25	0.0082734600620592\\
27.26	0.00827350624899054\\
27.27	0.00827355245789577\\
27.28	0.008273598688819\\
27.29	0.00827364494180497\\
27.3	0.00827369121689913\\
27.31	0.00827373751414761\\
27.32	0.00827378383359724\\
27.33	0.00827383017529557\\
27.34	0.00827387653929087\\
27.35	0.00827392292563217\\
27.36	0.00827396933436925\\
27.37	0.00827401576555266\\
27.38	0.00827406221923373\\
27.39	0.0082741086954646\\
27.4	0.00827415519429823\\
27.41	0.00827420171578838\\
27.42	0.00827424825998968\\
27.43	0.00827429482695763\\
27.44	0.00827434141674857\\
27.45	0.00827438802941976\\
27.46	0.00827443466502936\\
27.47	0.00827448132363644\\
27.48	0.00827452800530103\\
27.49	0.00827457471008411\\
27.5	0.00827462143804764\\
27.51	0.00827466818925456\\
27.52	0.00827471496376884\\
27.53	0.00827476176165545\\
27.54	0.00827480858298044\\
27.55	0.0082748554278109\\
27.56	0.00827490229621502\\
27.57	0.00827494918826208\\
27.58	0.00827499610402249\\
27.59	0.00827504304356781\\
27.6	0.00827509000697076\\
27.61	0.00827513699430523\\
27.62	0.00827518400564633\\
27.63	0.0082752310410704\\
27.64	0.00827527810065501\\
27.65	0.00827532518447901\\
27.66	0.00827537229262253\\
27.67	0.00827541942516704\\
27.68	0.00827546658219531\\
27.69	0.00827551376379148\\
27.7	0.0082755609700411\\
27.71	0.00827560820103109\\
27.72	0.00827565545684983\\
27.73	0.00827570273758711\\
27.74	0.00827575004333425\\
27.75	0.00827579737418405\\
27.76	0.00827584473023082\\
27.77	0.00827589211157046\\
27.78	0.00827593951830043\\
27.79	0.0082759869505198\\
27.8	0.00827603440832927\\
27.81	0.00827608189183122\\
27.82	0.00827612940112969\\
27.83	0.00827617693633044\\
27.84	0.00827622449754099\\
27.85	0.00827627208487062\\
27.86	0.0082763196984304\\
27.87	0.00827636733833325\\
27.88	0.00827641500469394\\
27.89	0.00827646269762911\\
27.9	0.00827651041725733\\
27.91	0.00827655816369912\\
27.92	0.00827660593707698\\
27.93	0.0082766537375154\\
27.94	0.00827670156514092\\
27.95	0.00827674942008215\\
27.96	0.0082767973024698\\
27.97	0.00827684521243671\\
27.98	0.00827689315011788\\
27.99	0.00827694111565053\\
28	0.00827698910917407\\
28.01	0.00827703713083022\\
28.02	0.00827708518076295\\
28.03	0.00827713325911858\\
28.04	0.00827718136604578\\
28.05	0.00827722950169563\\
28.06	0.00827727766622162\\
28.07	0.00827732585977971\\
28.08	0.00827737408252836\\
28.09	0.00827742233462854\\
28.1	0.00827747061624379\\
28.11	0.00827751892754027\\
28.12	0.00827756726868673\\
28.13	0.00827761563985461\\
28.14	0.00827766404121804\\
28.15	0.00827771247295389\\
28.16	0.00827776093524179\\
28.17	0.00827780942826416\\
28.18	0.00827785795220626\\
28.19	0.00827790650725624\\
28.2	0.0082779550936051\\
28.21	0.00827800371144683\\
28.22	0.00827805236097834\\
28.23	0.00827810104239955\\
28.24	0.00827814975591343\\
28.25	0.00827819850172599\\
28.26	0.00827824728004634\\
28.27	0.00827829609108672\\
28.28	0.00827834493506251\\
28.29	0.00827839381219228\\
28.3	0.00827844272269782\\
28.31	0.00827849166680414\\
28.32	0.00827854064473956\\
28.33	0.00827858965673564\\
28.34	0.0082786387030273\\
28.35	0.00827868778385278\\
28.36	0.00827873689945372\\
28.37	0.00827878605007512\\
28.38	0.00827883523596541\\
28.39	0.00827888445737645\\
28.4	0.00827893371456355\\
28.41	0.00827898300778548\\
28.42	0.00827903233730451\\
28.43	0.00827908170338639\\
28.44	0.00827913110630041\\
28.45	0.00827918054631934\\
28.46	0.00827923002371952\\
28.47	0.00827927953878079\\
28.48	0.00827932909178655\\
28.49	0.00827937868302373\\
28.5	0.0082794283127828\\
28.51	0.00827947798135777\\
28.52	0.00827952768904618\\
28.53	0.00827957743614908\\
28.54	0.00827962722297102\\
28.55	0.00827967704982007\\
28.56	0.00827972691700774\\
28.57	0.00827977682484899\\
28.58	0.00827982677366223\\
28.59	0.00827987676376923\\
28.6	0.00827992679549511\\
28.61	0.00827997686916832\\
28.62	0.00828002698512058\\
28.63	0.00828007714368681\\
28.64	0.00828012734520512\\
28.65	0.0082801775900167\\
28.66	0.0082802278784658\\
28.67	0.00828027821089961\\
28.68	0.00828032858766823\\
28.69	0.00828037900912454\\
28.7	0.00828042947562416\\
28.71	0.00828047998752529\\
28.72	0.00828053054518866\\
28.73	0.00828058114897736\\
28.74	0.00828063179925676\\
28.75	0.00828068249639438\\
28.76	0.00828073324075971\\
28.77	0.00828078403272409\\
28.78	0.00828083487266055\\
28.79	0.0082808857609436\\
28.8	0.00828093669794912\\
28.81	0.00828098768405411\\
28.82	0.00828103871963648\\
28.83	0.00828108980507488\\
28.84	0.00828114094074844\\
28.85	0.00828119212703652\\
28.86	0.00828124336431845\\
28.87	0.00828129465297329\\
28.88	0.00828134599337951\\
28.89	0.0082813973859147\\
28.9	0.00828144883095524\\
28.91	0.00828150032887596\\
28.92	0.0082815518800498\\
28.93	0.00828160348484743\\
28.94	0.00828165514363682\\
28.95	0.0082817068567829\\
28.96	0.00828175862464703\\
28.97	0.00828181044758661\\
28.98	0.00828186232595457\\
28.99	0.00828191426009886\\
29	0.00828196625036193\\
29.01	0.00828201829708017\\
29.02	0.0082820704005833\\
29.03	0.00828212256119378\\
29.04	0.00828217477922618\\
29.05	0.00828222705498644\\
29.06	0.00828227938877125\\
29.07	0.00828233178086723\\
29.08	0.00828238423155021\\
29.09	0.00828243674108439\\
29.1	0.00828248930972148\\
29.11	0.00828254193769984\\
29.12	0.00828259462524352\\
29.13	0.00828264737256133\\
29.14	0.00828270017984578\\
29.15	0.00828275304727201\\
29.16	0.00828280597499673\\
29.17	0.00828285896315701\\
29.18	0.0082829120118691\\
29.19	0.00828296512122714\\
29.2	0.00828301829130184\\
29.21	0.00828307152213913\\
29.22	0.00828312481375868\\
29.23	0.00828317816615242\\
29.24	0.00828323157928295\\
29.25	0.00828328505308194\\
29.26	0.00828333858744837\\
29.27	0.00828339218224682\\
29.28	0.00828344583730552\\
29.29	0.00828349955241452\\
29.3	0.00828355332732357\\
29.31	0.00828360716174011\\
29.32	0.00828366105532701\\
29.33	0.00828371500770031\\
29.34	0.00828376901842686\\
29.35	0.00828382308702179\\
29.36	0.008283877212946\\
29.37	0.0082839313956034\\
29.38	0.00828398563433816\\
29.39	0.00828403992843177\\
29.4	0.00828409427710003\\
29.41	0.00828414867948984\\
29.42	0.00828420313467596\\
29.43	0.00828425764165757\\
29.44	0.00828431219935471\\
29.45	0.00828436680660456\\
29.46	0.0082844214621576\\
29.47	0.00828447616467358\\
29.48	0.00828453091271738\\
29.49	0.00828458570475463\\
29.5	0.00828464053914719\\
29.51	0.0082846954141485\\
29.52	0.00828475032789865\\
29.53	0.00828480527841933\\
29.54	0.00828486026360853\\
29.55	0.00828491528123504\\
29.56	0.00828497033093327\\
29.57	0.00828502541264783\\
29.58	0.00828508052631517\\
29.59	0.00828513567186332\\
29.6	0.00828519084921148\\
29.61	0.00828524605826963\\
29.62	0.00828530129893816\\
29.63	0.00828535657110746\\
29.64	0.0082854118746575\\
29.65	0.00828546720945739\\
29.66	0.00828552257536492\\
29.67	0.00828557797222612\\
29.68	0.00828563339987476\\
29.69	0.00828568885813184\\
29.7	0.00828574434680511\\
29.71	0.00828579986568847\\
29.72	0.00828585541456149\\
29.73	0.00828591099318877\\
29.74	0.00828596660131942\\
29.75	0.00828602223868635\\
29.76	0.00828607790500572\\
29.77	0.00828613359997624\\
29.78	0.00828618932327849\\
29.79	0.00828624507457424\\
29.8	0.00828630085350568\\
29.81	0.00828635665969472\\
29.82	0.00828641249274215\\
29.83	0.00828646835222689\\
29.84	0.00828652423770513\\
29.85	0.00828658014870944\\
29.86	0.00828663608474795\\
29.87	0.00828669204530334\\
29.88	0.00828674802983197\\
29.89	0.00828680403776281\\
29.9	0.00828686006849651\\
29.91	0.00828691612140427\\
29.92	0.00828697219582677\\
29.93	0.00828702829107307\\
29.94	0.00828708440641944\\
29.95	0.00828714054110808\\
29.96	0.00828719669434601\\
29.97	0.00828725286530364\\
29.98	0.00828730905311356\\
29.99	0.00828736525686907\\
30	0.0082874214756228\\
30.01	0.00828747770838528\\
30.02	0.00828753395412334\\
30.03	0.00828759021175858\\
30.04	0.00828764648016575\\
30.05	0.00828770275817109\\
30.06	0.00828775904455054\\
30.07	0.00828781533802803\\
30.08	0.00828787163727355\\
30.09	0.0082879279409013\\
30.1	0.00828798424746766\\
30.11	0.00828804055546921\\
30.12	0.00828809686334057\\
30.13	0.00828815316945224\\
30.14	0.00828820947210835\\
30.15	0.00828826576954435\\
30.16	0.00828832205992457\\
30.17	0.00828837834133977\\
30.18	0.00828843461180457\\
30.19	0.00828849086925478\\
30.2	0.00828854711154469\\
30.21	0.00828860333644423\\
30.22	0.00828865954163607\\
30.23	0.00828871572471256\\
30.24	0.00828877188317269\\
30.25	0.00828882801441878\\
30.26	0.00828888411575322\\
30.27	0.00828894018437506\\
30.28	0.00828899621737637\\
30.29	0.00828905221173869\\
30.3	0.00828910816432916\\
30.31	0.00828916407189669\\
30.32	0.00828921993106786\\
30.33	0.00828927573834282\\
30.34	0.00828933149009094\\
30.35	0.0082893871825464\\
30.36	0.00828944281180359\\
30.37	0.00828949837381238\\
30.38	0.00828955386437325\\
30.39	0.0082896092791322\\
30.4	0.00828966461357558\\
30.41	0.00828971986302467\\
30.42	0.00828977502263018\\
30.43	0.00828983008736643\\
30.44	0.00828988505202551\\
30.45	0.00828993991121111\\
30.46	0.00828999465933221\\
30.47	0.0082900492905966\\
30.48	0.0082901037990041\\
30.49	0.00829015817833961\\
30.5	0.00829021242216601\\
30.51	0.00829026652381666\\
30.52	0.00829032047638779\\
30.53	0.00829037427273063\\
30.54	0.00829042790544323\\
30.55	0.00829048136686207\\
30.56	0.00829053464905334\\
30.57	0.00829058774380402\\
30.58	0.00829064064261259\\
30.59	0.00829069333667951\\
30.6	0.00829074581689733\\
30.61	0.00829079807384052\\
30.62	0.00829085009775498\\
30.63	0.0082909020045421\\
30.64	0.00829095393125352\\
30.65	0.00829100587789864\\
30.66	0.00829105784448683\\
30.67	0.00829110983102749\\
30.68	0.00829116183753\\
30.69	0.00829121386400377\\
30.7	0.0082912659104582\\
30.71	0.0082913179769027\\
30.72	0.0082913700633467\\
30.73	0.00829142216979962\\
30.74	0.00829147429627089\\
30.75	0.00829152644276994\\
30.76	0.00829157860930624\\
30.77	0.00829163079588922\\
30.78	0.00829168300252834\\
30.79	0.00829173522923307\\
30.8	0.00829178747601288\\
30.81	0.00829183974287723\\
30.82	0.00829189202983563\\
30.83	0.00829194433689755\\
30.84	0.00829199666407249\\
30.85	0.00829204901136996\\
30.86	0.00829210137879946\\
30.87	0.00829215376637051\\
30.88	0.00829220617409262\\
30.89	0.00829225860197533\\
30.9	0.00829231105002818\\
30.91	0.0082923635182607\\
30.92	0.00829241600668243\\
30.93	0.00829246851530294\\
30.94	0.00829252104413179\\
30.95	0.00829257359317854\\
30.96	0.00829262616245276\\
30.97	0.00829267875196404\\
30.98	0.00829273136172195\\
30.99	0.00829278399173611\\
31	0.00829283664201609\\
31.01	0.00829288931257152\\
31.02	0.00829294200341199\\
31.03	0.00829299471454714\\
31.04	0.00829304744598658\\
31.05	0.00829310019773996\\
31.06	0.0082931529698169\\
31.07	0.00829320576222705\\
31.08	0.00829325857498007\\
31.09	0.00829331140808561\\
31.1	0.00829336426155334\\
31.11	0.00829341713539292\\
31.12	0.00829347002961405\\
31.13	0.0082935229442264\\
31.14	0.00829357587923966\\
31.15	0.00829362883466353\\
31.16	0.00829368181050772\\
31.17	0.00829373480678195\\
31.18	0.00829378782349591\\
31.19	0.00829384086065935\\
31.2	0.00829389391828199\\
31.21	0.00829394699637356\\
31.22	0.00829400009494383\\
31.23	0.00829405321400253\\
31.24	0.00829410635355942\\
31.25	0.00829415951362427\\
31.26	0.00829421269420685\\
31.27	0.00829426589531694\\
31.28	0.00829431911696432\\
31.29	0.00829437235915879\\
31.3	0.00829442562191014\\
31.31	0.00829447890522817\\
31.32	0.00829453220912271\\
31.33	0.00829458553360356\\
31.34	0.00829463887868056\\
31.35	0.00829469224436353\\
31.36	0.00829474563066231\\
31.37	0.00829479903758674\\
31.38	0.0082948524651467\\
31.39	0.00829490591335202\\
31.4	0.00829495938221257\\
31.41	0.00829501287173823\\
31.42	0.00829506638193888\\
31.43	0.00829511991282441\\
31.44	0.0082951734644047\\
31.45	0.00829522703668966\\
31.46	0.00829528062968918\\
31.47	0.0082953342434132\\
31.48	0.00829538787787162\\
31.49	0.00829544153307438\\
31.5	0.0082954952090314\\
31.51	0.00829554890575263\\
31.52	0.00829560262324802\\
31.53	0.00829565636152752\\
31.54	0.00829571012060109\\
31.55	0.0082957639004787\\
31.56	0.00829581770117033\\
31.57	0.00829587152268596\\
31.58	0.00829592536503557\\
31.59	0.00829597922822917\\
31.6	0.00829603311227675\\
31.61	0.00829608701718833\\
31.62	0.00829614094297392\\
31.63	0.00829619488964355\\
31.64	0.00829624885720725\\
31.65	0.00829630284567506\\
31.66	0.00829635685505701\\
31.67	0.00829641088536316\\
31.68	0.00829646493660358\\
31.69	0.00829651900878832\\
31.7	0.00829657310192746\\
31.71	0.00829662721603107\\
31.72	0.00829668135110925\\
31.73	0.00829673550717209\\
31.74	0.00829678968422969\\
31.75	0.00829684388229216\\
31.76	0.0082968981013696\\
31.77	0.00829695234147215\\
31.78	0.00829700660260993\\
31.79	0.00829706088479308\\
31.8	0.00829711518803174\\
31.81	0.00829716951233606\\
31.82	0.0082972238577162\\
31.83	0.00829727822418232\\
31.84	0.00829733261174459\\
31.85	0.00829738702041319\\
31.86	0.00829744145019831\\
31.87	0.00829749590111014\\
31.88	0.00829755037315887\\
31.89	0.00829760486635473\\
31.9	0.00829765938070791\\
31.91	0.00829771391622865\\
31.92	0.00829776847292716\\
31.93	0.00829782305081369\\
31.94	0.00829787764989848\\
31.95	0.00829793227019177\\
31.96	0.00829798691170383\\
31.97	0.00829804157444491\\
31.98	0.00829809625842529\\
31.99	0.00829815096365526\\
32	0.00829820569014508\\
32.01	0.00829826043790506\\
32.02	0.0082983152069455\\
32.03	0.0082983699972767\\
32.04	0.00829842480890899\\
32.05	0.00829847964185267\\
32.06	0.00829853449611808\\
32.07	0.00829858937171556\\
32.08	0.00829864426865545\\
32.09	0.0082986991869481\\
32.1	0.00829875412660388\\
32.11	0.00829880908763313\\
32.12	0.00829886407004624\\
32.13	0.0082989190738536\\
32.14	0.00829897409906557\\
32.15	0.00829902914569257\\
32.16	0.00829908421374499\\
32.17	0.00829913930323324\\
32.18	0.00829919441416773\\
32.19	0.0082992495465589\\
32.2	0.00829930470041718\\
32.21	0.008299359875753\\
32.22	0.0082994150725768\\
32.23	0.00829947029089906\\
32.24	0.00829952553073021\\
32.25	0.00829958079208074\\
32.26	0.00829963607496111\\
32.27	0.00829969137938182\\
32.28	0.00829974670535334\\
32.29	0.00829980205288619\\
32.3	0.00829985742199086\\
32.31	0.00829991281267787\\
32.32	0.00829996822495774\\
32.33	0.008300023658841\\
32.34	0.00830007911433817\\
32.35	0.00830013459145982\\
32.36	0.00830019009021647\\
32.37	0.0083002456106187\\
32.38	0.00830030115267707\\
32.39	0.00830035671640214\\
32.4	0.00830041230180451\\
32.41	0.00830046790889476\\
32.42	0.00830052353768348\\
32.43	0.00830057918818127\\
32.44	0.00830063486039876\\
32.45	0.00830069055434654\\
32.46	0.00830074627003526\\
32.47	0.00830080200747554\\
32.48	0.00830085776667803\\
32.49	0.00830091354765337\\
32.5	0.00830096935041221\\
32.51	0.00830102517496523\\
32.52	0.00830108102132308\\
32.53	0.00830113688949646\\
32.54	0.00830119277949605\\
32.55	0.00830124869133254\\
32.56	0.00830130462501663\\
32.57	0.00830136058055903\\
32.58	0.00830141655797046\\
32.59	0.00830147255726164\\
32.6	0.00830152857844331\\
32.61	0.00830158462152619\\
32.62	0.00830164068652106\\
32.63	0.00830169677343865\\
32.64	0.00830175288228972\\
32.65	0.00830180901308506\\
32.66	0.00830186516583544\\
32.67	0.00830192134055164\\
32.68	0.00830197753724446\\
32.69	0.0083020337559247\\
32.7	0.00830208999660316\\
32.71	0.00830214625929068\\
32.72	0.00830220254399806\\
32.73	0.00830225885073614\\
32.74	0.00830231517951577\\
32.75	0.0083023715303478\\
32.76	0.00830242790324306\\
32.77	0.00830248429821244\\
32.78	0.0083025407152668\\
32.79	0.00830259715441702\\
32.8	0.00830265361567398\\
32.81	0.0083027100990486\\
32.82	0.00830276660455175\\
32.83	0.00830282313219436\\
32.84	0.00830287968198735\\
32.85	0.00830293625394164\\
32.86	0.00830299284806816\\
32.87	0.00830304946437786\\
32.88	0.00830310610288169\\
32.89	0.0083031627635906\\
32.9	0.00830321944651557\\
32.91	0.00830327615166756\\
32.92	0.00830333287905755\\
32.93	0.00830338962869654\\
32.94	0.00830344640059552\\
32.95	0.0083035031947655\\
32.96	0.0083035600112175\\
32.97	0.00830361684996253\\
32.98	0.00830367371101162\\
32.99	0.00830373059437581\\
33	0.00830378750006614\\
33.01	0.00830384442809367\\
33.02	0.00830390137846947\\
33.03	0.00830395835120459\\
33.04	0.00830401534631012\\
33.05	0.00830407236379714\\
33.06	0.00830412940367675\\
33.07	0.00830418646596004\\
33.08	0.00830424355065813\\
33.09	0.00830430065778214\\
33.1	0.00830435778734318\\
33.11	0.00830441493935239\\
33.12	0.00830447211382092\\
33.13	0.00830452931075992\\
33.14	0.00830458653018054\\
33.15	0.00830464377209395\\
33.16	0.00830470103651132\\
33.17	0.00830475832344383\\
33.18	0.00830481563290269\\
33.19	0.00830487296489908\\
33.2	0.00830493031944421\\
33.21	0.0083049876965493\\
33.22	0.00830504509622557\\
33.23	0.00830510251848425\\
33.24	0.00830515996333658\\
33.25	0.00830521743079381\\
33.26	0.00830527492086719\\
33.27	0.008305332433568\\
33.28	0.00830538996890749\\
33.29	0.00830544752689696\\
33.3	0.00830550510754769\\
33.31	0.00830556271087098\\
33.32	0.00830562033687813\\
33.33	0.00830567798558045\\
33.34	0.00830573565698928\\
33.35	0.00830579335111594\\
33.36	0.00830585106797177\\
33.37	0.00830590880756811\\
33.38	0.00830596656991632\\
33.39	0.00830602435502777\\
33.4	0.00830608216291382\\
33.41	0.00830613999358586\\
33.42	0.00830619784705527\\
33.43	0.00830625572333346\\
33.44	0.00830631362243182\\
33.45	0.00830637154436176\\
33.46	0.00830642948913473\\
33.47	0.00830648745676213\\
33.48	0.00830654544725542\\
33.49	0.00830660346062603\\
33.5	0.00830666149688544\\
33.51	0.00830671955604508\\
33.52	0.00830677763811645\\
33.53	0.00830683574311103\\
33.54	0.00830689387104029\\
33.55	0.00830695202191574\\
33.56	0.00830701019574889\\
33.57	0.00830706839255125\\
33.58	0.00830712661233434\\
33.59	0.0083071848551097\\
33.6	0.00830724312088886\\
33.61	0.00830730140968338\\
33.62	0.00830735972150481\\
33.63	0.00830741805636471\\
33.64	0.00830747641427468\\
33.65	0.00830753479524627\\
33.66	0.0083075931992911\\
33.67	0.00830765162642075\\
33.68	0.00830771007664684\\
33.69	0.00830776854998098\\
33.7	0.0083078270464348\\
33.71	0.00830788556601993\\
33.72	0.00830794410874802\\
33.73	0.00830800267463072\\
33.74	0.00830806126367968\\
33.75	0.00830811987590659\\
33.76	0.00830817851132311\\
33.77	0.00830823716994094\\
33.78	0.00830829585177176\\
33.79	0.00830835455682728\\
33.8	0.00830841328511922\\
33.81	0.00830847203665929\\
33.82	0.00830853081145922\\
33.83	0.00830858960953077\\
33.84	0.00830864843088565\\
33.85	0.00830870727553565\\
33.86	0.00830876614349252\\
33.87	0.00830882503476803\\
33.88	0.00830888394937397\\
33.89	0.00830894288732213\\
33.9	0.0083090018486243\\
33.91	0.0083090608332923\\
33.92	0.00830911984133795\\
33.93	0.00830917887277306\\
33.94	0.00830923792760948\\
33.95	0.00830929700585906\\
33.96	0.00830935610753363\\
33.97	0.00830941523264506\\
33.98	0.00830947438120523\\
33.99	0.00830953355322602\\
34	0.0083095927487193\\
34.01	0.00830965196769699\\
34.02	0.00830971121017098\\
34.03	0.00830977047615319\\
34.04	0.00830982976565555\\
34.05	0.00830988907868998\\
34.06	0.00830994841526842\\
34.07	0.00831000777540284\\
34.08	0.00831006715910519\\
34.09	0.00831012656638743\\
34.1	0.00831018599726155\\
34.11	0.00831024545173953\\
34.12	0.00831030492983336\\
34.13	0.00831036443155506\\
34.14	0.00831042395691664\\
34.15	0.0083104835059301\\
34.16	0.0083105430786075\\
34.17	0.00831060267496087\\
34.18	0.00831066229500225\\
34.19	0.00831072193874372\\
34.2	0.00831078160619732\\
34.21	0.00831084129737516\\
34.22	0.0083109010122893\\
34.23	0.00831096075095184\\
34.24	0.0083110205133749\\
34.25	0.00831108029957057\\
34.26	0.00831114010955099\\
34.27	0.00831119994332828\\
34.28	0.00831125980091459\\
34.29	0.00831131968232207\\
34.3	0.00831137958756287\\
34.31	0.00831143951664916\\
34.32	0.00831149946959312\\
34.33	0.00831155944640694\\
34.34	0.00831161944710282\\
34.35	0.00831167947169295\\
34.36	0.00831173952018956\\
34.37	0.00831179959260486\\
34.38	0.00831185968895109\\
34.39	0.00831191980924049\\
34.4	0.00831197995348532\\
34.41	0.00831204012169783\\
34.42	0.00831210031389029\\
34.43	0.00831216053007498\\
34.44	0.0083122207702642\\
34.45	0.00831228103447023\\
34.46	0.00831234132270539\\
34.47	0.00831240163498199\\
34.48	0.00831246197131235\\
34.49	0.00831252233170883\\
34.5	0.00831258271618374\\
34.51	0.00831264312474946\\
34.52	0.00831270355741835\\
34.53	0.00831276401420277\\
34.54	0.00831282449511511\\
34.55	0.00831288500016777\\
34.56	0.00831294552937313\\
34.57	0.00831300608274362\\
34.58	0.00831306666029165\\
34.59	0.00831312726202965\\
34.6	0.00831318788797007\\
34.61	0.00831324853812535\\
34.62	0.00831330921250794\\
34.63	0.00831336991113031\\
34.64	0.00831343063400495\\
34.65	0.00831349138114434\\
34.66	0.00831355215256096\\
34.67	0.00831361294826734\\
34.68	0.00831367376827598\\
34.69	0.00831373461259941\\
34.7	0.00831379548125016\\
34.71	0.00831385637424078\\
34.72	0.00831391729158382\\
34.73	0.00831397823329184\\
34.74	0.00831403919937741\\
34.75	0.00831410018985312\\
34.76	0.00831416120473156\\
34.77	0.00831422224402533\\
34.78	0.00831428330774704\\
34.79	0.00831434439590931\\
34.8	0.00831440550852478\\
34.81	0.00831446664560607\\
34.82	0.00831452780716586\\
34.83	0.00831458899321678\\
34.84	0.00831465020377151\\
34.85	0.00831471143884274\\
34.86	0.00831477269844315\\
34.87	0.00831483398258543\\
34.88	0.00831489529128229\\
34.89	0.00831495662454646\\
34.9	0.00831501798239067\\
34.91	0.00831507936482764\\
34.92	0.00831514077187013\\
34.93	0.0083152022035309\\
34.94	0.0083152636598227\\
34.95	0.00831532514075833\\
34.96	0.00831538664635056\\
34.97	0.00831544817661219\\
34.98	0.00831550973155603\\
34.99	0.00831557131119489\\
35	0.0083156329155416\\
35.01	0.00831569454460901\\
35.02	0.00831575619840994\\
35.03	0.00831581787695727\\
35.04	0.00831587958026385\\
35.05	0.00831594130834256\\
35.06	0.00831600306120629\\
35.07	0.00831606483886793\\
35.08	0.00831612664134039\\
35.09	0.00831618846863658\\
35.1	0.00831625032076943\\
35.11	0.00831631219775189\\
35.12	0.00831637409959688\\
35.13	0.00831643602631737\\
35.14	0.00831649797792633\\
35.15	0.00831655995443672\\
35.16	0.00831662195586154\\
35.17	0.00831668398221379\\
35.18	0.00831674603350646\\
35.19	0.00831680810975258\\
35.2	0.00831687021096517\\
35.21	0.00831693233715727\\
35.22	0.00831699448834193\\
35.23	0.00831705666453219\\
35.24	0.00831711886574113\\
35.25	0.00831718109198184\\
35.26	0.00831724334326738\\
35.27	0.00831730561961087\\
35.28	0.0083173679210254\\
35.29	0.0083174302475241\\
35.3	0.0083174925991201\\
35.31	0.00831755497582653\\
35.32	0.00831761737765655\\
35.33	0.0083176798046233\\
35.34	0.00831774225673997\\
35.35	0.00831780473401973\\
35.36	0.00831786723647576\\
35.37	0.00831792976412128\\
35.38	0.00831799231696949\\
35.39	0.0083180548950336\\
35.4	0.00831811749832687\\
35.41	0.00831818012686251\\
35.42	0.0083182427806538\\
35.43	0.00831830545971398\\
35.44	0.00831836816405633\\
35.45	0.00831843089369414\\
35.46	0.00831849364864069\\
35.47	0.0083185564289093\\
35.48	0.00831861923451327\\
35.49	0.00831868206546593\\
35.5	0.00831874492178062\\
35.51	0.00831880780347068\\
35.52	0.00831887071054947\\
35.53	0.00831893364303034\\
35.54	0.00831899660092669\\
35.55	0.00831905958425191\\
35.56	0.00831912259301937\\
35.57	0.00831918562724251\\
35.58	0.00831924868693472\\
35.59	0.00831931177210946\\
35.6	0.00831937488278014\\
35.61	0.00831943801896024\\
35.62	0.0083195011806632\\
35.63	0.0083195643679025\\
35.64	0.00831962758069162\\
35.65	0.00831969081904406\\
35.66	0.00831975408297332\\
35.67	0.00831981737249291\\
35.68	0.00831988068761637\\
35.69	0.00831994402835722\\
35.7	0.00832000739472901\\
35.71	0.0083200707867453\\
35.72	0.00832013420441967\\
35.73	0.00832019764776568\\
35.74	0.00832026111679693\\
35.75	0.00832032461152701\\
35.76	0.00832038813196955\\
35.77	0.00832045167813815\\
35.78	0.00832051525004646\\
35.79	0.00832057884770812\\
35.8	0.00832064247113677\\
35.81	0.0083207061203461\\
35.82	0.00832076979534976\\
35.83	0.00832083349616146\\
35.84	0.00832089722279488\\
35.85	0.00832096097526373\\
35.86	0.00832102475358174\\
35.87	0.00832108855776264\\
35.88	0.00832115238782017\\
35.89	0.00832121624376807\\
35.9	0.00832128012562011\\
35.91	0.00832134403339006\\
35.92	0.00832140796709173\\
35.93	0.00832147192673888\\
35.94	0.00832153591234534\\
35.95	0.00832159992392492\\
35.96	0.00832166396149146\\
35.97	0.00832172802505878\\
35.98	0.00832179211464075\\
35.99	0.00832185623025122\\
36	0.00832192037190406\\
36.01	0.00832198453961317\\
36.02	0.00832204873339243\\
36.03	0.00832211295325575\\
36.04	0.00832217719921706\\
36.05	0.00832224147129027\\
36.06	0.00832230576948933\\
36.07	0.00832237009382819\\
36.08	0.00832243444432081\\
36.09	0.00832249882098117\\
36.1	0.00832256322382324\\
36.11	0.00832262765286103\\
36.12	0.00832269210810853\\
36.13	0.00832275658957978\\
36.14	0.0083228210972888\\
36.15	0.00832288563124962\\
36.16	0.0083229501914763\\
36.17	0.00832301477798291\\
36.18	0.00832307939078352\\
36.19	0.0083231440298922\\
36.2	0.00832320869532307\\
36.21	0.00832327338709022\\
36.22	0.00832333810520778\\
36.23	0.00832340284968988\\
36.24	0.00832346762055066\\
36.25	0.00832353241780427\\
36.26	0.00832359724146488\\
36.27	0.00832366209154667\\
36.28	0.00832372696806382\\
36.29	0.00832379187103052\\
36.3	0.00832385680046101\\
36.31	0.00832392175636949\\
36.32	0.00832398673877019\\
36.33	0.00832405174767737\\
36.34	0.00832411678310528\\
36.35	0.00832418184506819\\
36.36	0.00832424693358037\\
36.37	0.00832431204865612\\
36.38	0.00832437719030975\\
36.39	0.00832444235855555\\
36.4	0.00832450755340786\\
36.41	0.00832457277488103\\
36.42	0.00832463802298939\\
36.43	0.00832470329774731\\
36.44	0.00832476859916915\\
36.45	0.00832483392726931\\
36.46	0.00832489928206218\\
36.47	0.00832496466356216\\
36.48	0.00832503007178367\\
36.49	0.00832509550674115\\
36.5	0.00832516096844903\\
36.51	0.00832522645692177\\
36.52	0.00832529197217384\\
36.53	0.00832535751421971\\
36.54	0.00832542308307387\\
36.55	0.00832548867875082\\
36.56	0.00832555430126508\\
36.57	0.00832561995063116\\
36.58	0.00832568562686361\\
36.59	0.00832575132997697\\
36.6	0.00832581705998581\\
36.61	0.00832588281690468\\
36.62	0.00832594860074819\\
36.63	0.00832601441153091\\
36.64	0.00832608024926746\\
36.65	0.00832614611397247\\
36.66	0.00832621200566055\\
36.67	0.00832627792434636\\
36.68	0.00832634387004454\\
36.69	0.00832640984276977\\
36.7	0.00832647584253672\\
36.71	0.00832654186936008\\
36.72	0.00832660792325456\\
36.73	0.00832667400423487\\
36.74	0.00832674011231573\\
36.75	0.00832680624751189\\
36.76	0.0083268724098381\\
36.77	0.00832693859930912\\
36.78	0.00832700481593973\\
36.79	0.0083270710597447\\
36.8	0.00832713733073885\\
36.81	0.00832720362893697\\
36.82	0.0083272699543539\\
36.83	0.00832733630700448\\
36.84	0.00832740268690354\\
36.85	0.00832746909406596\\
36.86	0.00832753552850658\\
36.87	0.00832760199024032\\
36.88	0.00832766847928205\\
36.89	0.0083277349956467\\
36.9	0.00832780153934917\\
36.91	0.00832786811040441\\
36.92	0.00832793470882736\\
36.93	0.00832800133463297\\
36.94	0.00832806798783622\\
36.95	0.00832813466845209\\
36.96	0.00832820137649557\\
36.97	0.00832826811198167\\
36.98	0.00832833487492541\\
36.99	0.00832840166534182\\
37	0.00832846848324595\\
37.01	0.00832853532865284\\
37.02	0.00832860220157758\\
37.03	0.00832866910203523\\
37.04	0.00832873603004091\\
37.05	0.0083288029856097\\
37.06	0.00832886996875673\\
37.07	0.00832893697949714\\
37.08	0.00832900401784605\\
37.09	0.00832907108381863\\
37.1	0.00832913817743006\\
37.11	0.00832920529869551\\
37.12	0.00832927244763016\\
37.13	0.00832933962424924\\
37.14	0.00832940682856795\\
37.15	0.00832947406060153\\
37.16	0.00832954132036523\\
37.17	0.00832960860787429\\
37.18	0.00832967592314399\\
37.19	0.00832974326618962\\
37.2	0.00832981063702645\\
37.21	0.00832987803566981\\
37.22	0.00832994546213502\\
37.23	0.00833001291643739\\
37.24	0.00833008039859229\\
37.25	0.00833014790861506\\
37.26	0.00833021544652108\\
37.27	0.00833028301232574\\
37.28	0.00833035060604442\\
37.29	0.00833041822769254\\
37.3	0.00833048587728552\\
37.31	0.00833055355483879\\
37.32	0.0083306212603678\\
37.33	0.00833068899388802\\
37.34	0.0083307567554149\\
37.35	0.00833082454496396\\
37.36	0.00833089236255067\\
37.37	0.00833096020819055\\
37.38	0.00833102808189913\\
37.39	0.00833109598369194\\
37.4	0.00833116391358453\\
37.41	0.00833123187159248\\
37.42	0.00833129985773135\\
37.43	0.00833136787201674\\
37.44	0.00833143591446424\\
37.45	0.00833150398508947\\
37.46	0.00833157208390807\\
37.47	0.00833164021093567\\
37.48	0.00833170836618793\\
37.49	0.00833177654968051\\
37.5	0.0083318447614291\\
37.51	0.00833191300144939\\
37.52	0.0083319812697571\\
37.53	0.00833204956636792\\
37.54	0.00833211789129762\\
37.55	0.00833218624456192\\
37.56	0.00833225462617658\\
37.57	0.00833232303615739\\
37.58	0.00833239147452013\\
37.59	0.0083324599412806\\
37.6	0.0083325284364546\\
37.61	0.00833259696005797\\
37.62	0.00833266551210655\\
37.63	0.00833273409261618\\
37.64	0.00833280270160274\\
37.65	0.00833287133908209\\
37.66	0.00833294000507015\\
37.67	0.0083330086995828\\
37.68	0.00833307742263598\\
37.69	0.00833314617424561\\
37.7	0.00833321495442764\\
37.71	0.00833328376319802\\
37.72	0.00833335260057274\\
37.73	0.00833342146656777\\
37.74	0.00833349036119912\\
37.75	0.0083335592844828\\
37.76	0.00833362823643484\\
37.77	0.00833369721707127\\
37.78	0.00833376622640816\\
37.79	0.00833383526446156\\
37.8	0.00833390433124756\\
37.81	0.00833397342678225\\
37.82	0.00833404255108174\\
37.83	0.00833411170416215\\
37.84	0.00833418088603962\\
37.85	0.0083342500967303\\
37.86	0.00833431933625034\\
37.87	0.00833438860461592\\
37.88	0.00833445790184324\\
37.89	0.0083345272279485\\
37.9	0.00833459658294791\\
37.91	0.0083346659668577\\
37.92	0.00833473537969412\\
37.93	0.00833480482147343\\
37.94	0.00833487429221189\\
37.95	0.0083349437919258\\
37.96	0.00833501332063144\\
37.97	0.00833508287834515\\
37.98	0.00833515246508324\\
37.99	0.00833522208086205\\
38	0.00833529172569795\\
38.01	0.00833536139960729\\
38.02	0.00833543110260645\\
38.03	0.00833550083471185\\
38.04	0.00833557059593988\\
38.05	0.00833564038630697\\
38.06	0.00833571020582956\\
38.07	0.00833578005452409\\
38.08	0.00833584993240705\\
38.09	0.0083359198394949\\
38.1	0.00833598977580413\\
38.11	0.00833605974135127\\
38.12	0.00833612973615283\\
38.13	0.00833619976022534\\
38.14	0.00833626981358536\\
38.15	0.00833633989624944\\
38.16	0.00833641000823418\\
38.17	0.00833648014955615\\
38.18	0.00833655032023198\\
38.19	0.00833662052027827\\
38.2	0.00833669074971166\\
38.21	0.0083367610085488\\
38.22	0.00833683129680635\\
38.23	0.008336901614501\\
38.24	0.00833697196164942\\
38.25	0.00833704233826834\\
38.26	0.00833711274437446\\
38.27	0.00833718317998452\\
38.28	0.00833725364511527\\
38.29	0.00833732413978348\\
38.3	0.00833739466400592\\
38.31	0.00833746521779937\\
38.32	0.00833753580118065\\
38.33	0.00833760641416658\\
38.34	0.00833767705677399\\
38.35	0.00833774772901973\\
38.36	0.00833781843092066\\
38.37	0.00833788916249365\\
38.38	0.00833795992375562\\
38.39	0.00833803071472345\\
38.4	0.00833810153541406\\
38.41	0.0083381723858444\\
38.42	0.00833824326603141\\
38.43	0.00833831417599206\\
38.44	0.00833838511574333\\
38.45	0.0083384560853022\\
38.46	0.0083385270846857\\
38.47	0.00833859811391082\\
38.48	0.00833866917299463\\
38.49	0.00833874026195417\\
38.5	0.00833881138080649\\
38.51	0.00833888252956869\\
38.52	0.00833895370825786\\
38.53	0.0083390249168911\\
38.54	0.00833909615548555\\
38.55	0.00833916742405834\\
38.56	0.00833923872262663\\
38.57	0.00833931005120758\\
38.58	0.00833938140981838\\
38.59	0.00833945279847623\\
38.6	0.00833952421719833\\
38.61	0.00833959566600192\\
38.62	0.00833966714490424\\
38.63	0.00833973865392254\\
38.64	0.0083398101930741\\
38.65	0.00833988176237621\\
38.66	0.00833995336184616\\
38.67	0.00834002499150127\\
38.68	0.00834009665135887\\
38.69	0.00834016834143631\\
38.7	0.00834024006175095\\
38.71	0.00834031181232017\\
38.72	0.00834038359316135\\
38.73	0.0083404554042919\\
38.74	0.00834052724572925\\
38.75	0.00834059911749082\\
38.76	0.00834067101959407\\
38.77	0.00834074295205647\\
38.78	0.00834081491489549\\
38.79	0.00834088690812863\\
38.8	0.0083409589317734\\
38.81	0.00834103098584732\\
38.82	0.00834110307036795\\
38.83	0.00834117518535283\\
38.84	0.00834124733081952\\
38.85	0.00834131950678564\\
38.86	0.00834139171326875\\
38.87	0.0083414639502865\\
38.88	0.0083415362178565\\
38.89	0.00834160851599641\\
38.9	0.00834168084472388\\
38.91	0.00834175320405659\\
38.92	0.00834182559401225\\
38.93	0.00834189801460853\\
38.94	0.00834197046586319\\
38.95	0.00834204294779395\\
38.96	0.00834211546041856\\
38.97	0.0083421880037548\\
38.98	0.00834226057782043\\
38.99	0.00834233318263328\\
39	0.00834240581821115\\
39.01	0.00834247848457186\\
39.02	0.00834255118173327\\
39.03	0.00834262390971322\\
39.04	0.00834269666852961\\
39.05	0.00834276945820032\\
39.06	0.00834284227874325\\
39.07	0.00834291513017633\\
39.08	0.00834298801251749\\
39.09	0.00834306092578469\\
39.1	0.00834313386999589\\
39.11	0.00834320684516908\\
39.12	0.00834327985132225\\
39.13	0.00834335288847343\\
39.14	0.00834342595664063\\
39.15	0.0083434990558419\\
39.16	0.00834357218609532\\
39.17	0.00834364534741894\\
39.18	0.00834371853983087\\
39.19	0.0083437917633492\\
39.2	0.00834386501799207\\
39.21	0.00834393830377761\\
39.22	0.00834401162072398\\
39.23	0.00834408496884933\\
39.24	0.00834415834817187\\
39.25	0.00834423175870979\\
39.26	0.0083443052004813\\
39.27	0.00834437867350464\\
39.28	0.00834445217779807\\
39.29	0.00834452571337982\\
39.3	0.0083445992802682\\
39.31	0.00834467287848149\\
39.32	0.008344746508038\\
39.33	0.00834482016895606\\
39.34	0.00834489386125402\\
39.35	0.00834496758495022\\
39.36	0.00834504134006304\\
39.37	0.00834511512661088\\
39.38	0.00834518894461212\\
39.39	0.00834526279408521\\
39.4	0.00834533667504857\\
39.41	0.00834541058752064\\
39.42	0.00834548453151992\\
39.43	0.00834555850706487\\
39.44	0.00834563251417399\\
39.45	0.0083457065528658\\
39.46	0.00834578062315884\\
39.47	0.00834585472507164\\
39.48	0.00834592885862277\\
39.49	0.00834600302383082\\
39.5	0.00834607722071436\\
39.51	0.00834615144929202\\
39.52	0.00834622570958243\\
39.53	0.00834630000160422\\
39.54	0.00834637432537604\\
39.55	0.00834644868091659\\
39.56	0.00834652306824454\\
39.57	0.0083465974873786\\
39.58	0.0083466719383375\\
39.59	0.00834674642113996\\
39.6	0.00834682093580475\\
39.61	0.00834689548235064\\
39.62	0.00834697006079641\\
39.63	0.00834704467116086\\
39.64	0.00834711931346281\\
39.65	0.0083471939877211\\
39.66	0.00834726869395457\\
39.67	0.00834734343218209\\
39.68	0.00834741820242254\\
39.69	0.00834749300469482\\
39.7	0.00834756783901784\\
39.71	0.00834764270541054\\
39.72	0.00834771760389186\\
39.73	0.00834779253448076\\
39.74	0.00834786749719622\\
39.75	0.00834794249205723\\
39.76	0.00834801751908281\\
39.77	0.00834809257829197\\
39.78	0.00834816766970378\\
39.79	0.00834824279333727\\
39.8	0.00834831794921153\\
39.81	0.00834839313734564\\
39.82	0.00834846835775872\\
39.83	0.00834854361046989\\
39.84	0.00834861889549828\\
39.85	0.00834869421286305\\
39.86	0.00834876956258338\\
39.87	0.00834884494467845\\
39.88	0.00834892035916746\\
39.89	0.00834899580606964\\
39.9	0.00834907128540421\\
39.91	0.00834914679719044\\
39.92	0.00834922234144759\\
39.93	0.00834929791819494\\
39.94	0.0083493735274518\\
39.95	0.00834944916923748\\
39.96	0.00834952484357131\\
39.97	0.00834960055047265\\
39.98	0.00834967628996086\\
39.99	0.00834975206205533\\
40	0.00834982786677544\\
40.01	0.00834990370414062\\
};
\addplot [color=red,solid,forget plot]
  table[row sep=crcr]{%
40.01	0.00834990370414062\\
40.02	0.00834997957417028\\
40.03	0.00835005547688389\\
40.04	0.0083501314123009\\
40.05	0.00835020738044079\\
40.06	0.00835028338132306\\
40.07	0.00835035941496721\\
40.08	0.00835043548139278\\
40.09	0.0083505115806193\\
40.1	0.00835058771266635\\
40.11	0.00835066387755349\\
40.12	0.00835074007530031\\
40.13	0.00835081630592643\\
40.14	0.00835089256945146\\
40.15	0.00835096886589506\\
40.16	0.00835104519527687\\
40.17	0.00835112155761657\\
40.18	0.00835119795293385\\
40.19	0.00835127438124842\\
40.2	0.00835135084258\\
40.21	0.00835142733694832\\
40.22	0.00835150386437313\\
40.23	0.00835158042487422\\
40.24	0.00835165701847137\\
40.25	0.00835173364518438\\
40.26	0.00835181030503306\\
40.27	0.00835188699803726\\
40.28	0.00835196372421682\\
40.29	0.00835204048359163\\
40.3	0.00835211727618155\\
40.31	0.00835219410200648\\
40.32	0.00835227096108635\\
40.33	0.00835234785344109\\
40.34	0.00835242477909065\\
40.35	0.00835250173805499\\
40.36	0.00835257873035409\\
40.37	0.00835265575600796\\
40.38	0.00835273281503659\\
40.39	0.00835280990746003\\
40.4	0.00835288703329833\\
40.41	0.00835296419257154\\
40.42	0.00835304138529974\\
40.43	0.00835311861150302\\
40.44	0.0083531958712015\\
40.45	0.00835327316441531\\
40.46	0.00835335049116458\\
40.47	0.00835342785146948\\
40.48	0.00835350524535019\\
40.49	0.00835358267282689\\
40.5	0.00835366013391979\\
40.51	0.00835373762864912\\
40.52	0.00835381515703513\\
40.53	0.00835389271909805\\
40.54	0.00835397031485817\\
40.55	0.00835404794433579\\
40.56	0.00835412560755119\\
40.57	0.0083542033045247\\
40.58	0.00835428103527667\\
40.59	0.00835435879982745\\
40.6	0.00835443659819739\\
40.61	0.0083545144304069\\
40.62	0.00835459229647638\\
40.63	0.00835467019642623\\
40.64	0.0083547481302769\\
40.65	0.00835482609804884\\
40.66	0.00835490409976251\\
40.67	0.0083549821354384\\
40.68	0.00835506020509701\\
40.69	0.00835513830875886\\
40.7	0.00835521644644447\\
40.71	0.00835529461817438\\
40.72	0.00835537282396918\\
40.73	0.00835545106384944\\
40.74	0.00835552933783575\\
40.75	0.00835560764594873\\
40.76	0.00835568598820901\\
40.77	0.00835576436463723\\
40.78	0.00835584277525405\\
40.79	0.00835592122008016\\
40.8	0.00835599969913624\\
40.81	0.00835607821244301\\
40.82	0.00835615676002119\\
40.83	0.00835623534189152\\
40.84	0.00835631395807478\\
40.85	0.00835639260859172\\
40.86	0.00835647129346315\\
40.87	0.00835655001270987\\
40.88	0.0083566287663527\\
40.89	0.00835670755441249\\
40.9	0.00835678637691009\\
40.91	0.00835686523386639\\
40.92	0.00835694412530226\\
40.93	0.00835702305123861\\
40.94	0.00835710201169637\\
40.95	0.00835718100669648\\
40.96	0.00835726003625989\\
40.97	0.00835733910040758\\
40.98	0.00835741819916053\\
40.99	0.00835749733253976\\
41	0.00835757650056627\\
41.01	0.00835765570326112\\
41.02	0.00835773494064536\\
41.03	0.00835781421274005\\
41.04	0.0083578935195663\\
41.05	0.0083579728611452\\
41.06	0.00835805223749788\\
41.07	0.00835813164864547\\
41.08	0.00835821109460914\\
41.09	0.00835829057541005\\
41.1	0.00835837009106941\\
41.11	0.00835844964160841\\
41.12	0.00835852922704827\\
41.13	0.00835860884741026\\
41.14	0.00835868850271561\\
41.15	0.00835876819298561\\
41.16	0.00835884791824155\\
41.17	0.00835892767850474\\
41.18	0.00835900747379651\\
41.19	0.00835908730413821\\
41.2	0.00835916716955121\\
41.21	0.00835924707005687\\
41.22	0.00835932700567661\\
41.23	0.00835940697643183\\
41.24	0.00835948698234399\\
41.25	0.00835956702343452\\
41.26	0.0083596470997249\\
41.27	0.00835972721123662\\
41.28	0.00835980735799118\\
41.29	0.00835988754001013\\
41.3	0.00835996775731499\\
41.31	0.00836004800992733\\
41.32	0.00836012829786874\\
41.33	0.00836020862116083\\
41.34	0.0083602889798252\\
41.35	0.0083603693738835\\
41.36	0.00836044980335739\\
41.37	0.00836053026826854\\
41.38	0.00836061076863867\\
41.39	0.00836069130448948\\
41.4	0.00836077187584272\\
41.41	0.00836085248272014\\
41.42	0.00836093312514353\\
41.43	0.00836101380313468\\
41.44	0.00836109451671541\\
41.45	0.00836117526590757\\
41.46	0.00836125605073301\\
41.47	0.00836133687121363\\
41.48	0.00836141772737134\\
41.49	0.00836149861922804\\
41.5	0.0083615795468057\\
41.51	0.0083616605101263\\
41.52	0.00836174150921181\\
41.53	0.00836182254408427\\
41.54	0.00836190361476571\\
41.55	0.00836198472127819\\
41.56	0.00836206586364381\\
41.57	0.00836214704188468\\
41.58	0.00836222825602292\\
41.59	0.00836230950608072\\
41.6	0.00836239079208024\\
41.61	0.0083624721140437\\
41.62	0.00836255347199334\\
41.63	0.00836263486595142\\
41.64	0.00836271629594023\\
41.65	0.00836279776198209\\
41.66	0.00836287926409933\\
41.67	0.00836296080231434\\
41.68	0.00836304237664951\\
41.69	0.00836312398712727\\
41.7	0.00836320563377008\\
41.71	0.00836328731660041\\
41.72	0.0083633690356408\\
41.73	0.00836345079091379\\
41.74	0.00836353258244194\\
41.75	0.00836361441024788\\
41.76	0.00836369627435425\\
41.77	0.00836377817478371\\
41.78	0.00836386011155898\\
41.79	0.0083639420847028\\
41.8	0.00836402409423795\\
41.81	0.00836410614018723\\
41.82	0.00836418822257349\\
41.83	0.00836427034141962\\
41.84	0.00836435249674853\\
41.85	0.00836443468858319\\
41.86	0.00836451691694659\\
41.87	0.00836459918186178\\
41.88	0.00836468148335181\\
41.89	0.00836476382143982\\
41.9	0.00836484619614897\\
41.91	0.00836492860750245\\
41.92	0.00836501105552351\\
41.93	0.00836509354023545\\
41.94	0.00836517606166159\\
41.95	0.00836525861982533\\
41.96	0.00836534121475008\\
41.97	0.00836542384645933\\
41.98	0.00836550651497661\\
41.99	0.0083655892203255\\
42	0.00836567196252961\\
42.01	0.00836575474161264\\
42.02	0.00836583755759832\\
42.03	0.00836592041051044\\
42.04	0.00836600330037285\\
42.05	0.00836608622720945\\
42.06	0.00836616919104421\\
42.07	0.00836625219190114\\
42.08	0.00836633522980433\\
42.09	0.00836641830477794\\
42.1	0.00836650141684615\\
42.11	0.00836658456603327\\
42.12	0.00836666775236363\\
42.13	0.00836675097586163\\
42.14	0.00836683423655176\\
42.15	0.00836691753445857\\
42.16	0.00836700086960669\\
42.17	0.00836708424202083\\
42.18	0.00836716765172574\\
42.19	0.0083672510987463\\
42.2	0.00836733458310743\\
42.21	0.00836741810483414\\
42.22	0.00836750166395155\\
42.23	0.00836758526048482\\
42.24	0.00836766889445925\\
42.25	0.00836775256590017\\
42.26	0.00836783627483305\\
42.27	0.00836792002128343\\
42.28	0.00836800380527695\\
42.29	0.00836808762683934\\
42.3	0.00836817148599644\\
42.31	0.00836825538277417\\
42.32	0.0083683393171986\\
42.33	0.00836842328929585\\
42.34	0.00836850729909217\\
42.35	0.00836859134661393\\
42.36	0.00836867543188761\\
42.37	0.00836875955493979\\
42.38	0.00836884371579718\\
42.39	0.0083689279144866\\
42.4	0.00836901215103501\\
42.41	0.00836909642546945\\
42.42	0.00836918073781715\\
42.43	0.00836926508810542\\
42.44	0.00836934947636171\\
42.45	0.00836943390261363\\
42.46	0.00836951836688889\\
42.47	0.00836960286921537\\
42.48	0.00836968740962107\\
42.49	0.00836977198813414\\
42.5	0.00836985660478288\\
42.51	0.00836994125959575\\
42.52	0.00837002595260134\\
42.53	0.0083701106838284\\
42.54	0.00837019545330584\\
42.55	0.00837028026106274\\
42.56	0.00837036510712833\\
42.57	0.00837044999153199\\
42.58	0.0083705349143033\\
42.59	0.00837061987547199\\
42.6	0.00837070487506795\\
42.61	0.00837078991312128\\
42.62	0.00837087498966221\\
42.63	0.00837096010472118\\
42.64	0.0083710452583288\\
42.65	0.00837113045051586\\
42.66	0.00837121568131335\\
42.67	0.00837130095075243\\
42.68	0.00837138625886443\\
42.69	0.00837147160568092\\
42.7	0.00837155699123361\\
42.71	0.00837164241555444\\
42.72	0.00837172787867551\\
42.73	0.00837181338062914\\
42.74	0.00837189892144783\\
42.75	0.00837198450116428\\
42.76	0.0083720701198114\\
42.77	0.00837215577742226\\
42.78	0.00837224147403016\\
42.79	0.00837232720966859\\
42.8	0.00837241298437121\\
42.81	0.00837249879817191\\
42.82	0.00837258465110473\\
42.83	0.00837267054320393\\
42.84	0.00837275647450397\\
42.85	0.00837284244503944\\
42.86	0.00837292845484518\\
42.87	0.00837301450395616\\
42.88	0.00837310059240754\\
42.89	0.00837318672023466\\
42.9	0.008373272887473\\
42.91	0.00837335909415822\\
42.92	0.00837344534032613\\
42.93	0.00837353162601268\\
42.94	0.00837361795125396\\
42.95	0.00837370431608618\\
42.96	0.00837379072054568\\
42.97	0.0083738771646689\\
42.98	0.00837396364849238\\
42.99	0.00837405017205274\\
43	0.00837413673538668\\
43.01	0.00837422333853095\\
43.02	0.00837430998152233\\
43.03	0.00837439666439765\\
43.04	0.00837448338719372\\
43.05	0.00837457014994733\\
43.06	0.00837465695269527\\
43.07	0.00837474379547423\\
43.08	0.00837483067832083\\
43.09	0.00837491760127158\\
43.1	0.00837500456436287\\
43.11	0.00837509156763088\\
43.12	0.00837517861111162\\
43.13	0.00837526569484086\\
43.14	0.0083753528188541\\
43.15	0.00837543998318651\\
43.16	0.00837552718787292\\
43.17	0.00837561443294776\\
43.18	0.00837570171844502\\
43.19	0.0083757890443982\\
43.2	0.00837587641084023\\
43.21	0.00837596381780346\\
43.22	0.00837605126531956\\
43.23	0.00837613875341947\\
43.24	0.00837622628213336\\
43.25	0.00837631385149049\\
43.26	0.00837640146151921\\
43.27	0.00837648911224684\\
43.28	0.00837657680370064\\
43.29	0.0083766645359079\\
43.3	0.00837675230889594\\
43.31	0.0083768401226921\\
43.32	0.00837692797732375\\
43.33	0.00837701587281831\\
43.34	0.00837710380920318\\
43.35	0.00837719178650584\\
43.36	0.00837727980475376\\
43.37	0.00837736786397445\\
43.38	0.00837745596419546\\
43.39	0.00837754410544436\\
43.4	0.00837763228774872\\
43.41	0.00837772051113619\\
43.42	0.00837780877563442\\
43.43	0.00837789708127107\\
43.44	0.00837798542807387\\
43.45	0.00837807381607054\\
43.46	0.00837816224528885\\
43.47	0.00837825071575659\\
43.48	0.00837833922750158\\
43.49	0.00837842778055167\\
43.5	0.00837851637493474\\
43.51	0.00837860501067869\\
43.52	0.00837869368781146\\
43.53	0.008378782406361\\
43.54	0.00837887116635531\\
43.55	0.00837895996782241\\
43.56	0.00837904881079033\\
43.57	0.00837913769528717\\
43.58	0.00837922662134102\\
43.59	0.00837931558898001\\
43.6	0.00837940459823231\\
43.61	0.00837949364912611\\
43.62	0.00837958274168963\\
43.63	0.0083796718759511\\
43.64	0.00837976105193882\\
43.65	0.00837985026968108\\
43.66	0.00837993952920622\\
43.67	0.0083800288305426\\
43.68	0.00838011817371861\\
43.69	0.00838020755876268\\
43.7	0.00838029698570325\\
43.71	0.00838038645456879\\
43.72	0.00838047596538782\\
43.73	0.00838056551818888\\
43.74	0.00838065511300052\\
43.75	0.00838074474985135\\
43.76	0.00838083442876999\\
43.77	0.00838092414978508\\
43.78	0.00838101391292532\\
43.79	0.0083811037182194\\
43.8	0.00838119356569608\\
43.81	0.00838128345538412\\
43.82	0.00838137338731232\\
43.83	0.00838146336150952\\
43.84	0.00838155337800455\\
43.85	0.00838164343682632\\
43.86	0.00838173353800374\\
43.87	0.00838182368156575\\
43.88	0.00838191386754134\\
43.89	0.0083820040959595\\
43.9	0.00838209436684927\\
43.91	0.00838218468023971\\
43.92	0.00838227503615992\\
43.93	0.00838236543463902\\
43.94	0.00838245587570616\\
43.95	0.00838254635939053\\
43.96	0.00838263688572133\\
43.97	0.00838272745472781\\
43.98	0.00838281806643925\\
43.99	0.00838290872088493\\
44	0.0083829994180942\\
44.01	0.00838309015809641\\
44.02	0.00838318094092095\\
44.03	0.00838327176659725\\
44.04	0.00838336263515475\\
44.05	0.00838345354662295\\
44.06	0.00838354450103133\\
44.07	0.00838363549840946\\
44.08	0.00838372653878689\\
44.09	0.00838381762219323\\
44.1	0.00838390874865812\\
44.11	0.00838399991821119\\
44.12	0.00838409113088216\\
44.13	0.00838418238670074\\
44.14	0.00838427368569668\\
44.15	0.00838436502789977\\
44.16	0.00838445641333981\\
44.17	0.00838454784204664\\
44.18	0.00838463931405014\\
44.19	0.00838473082938022\\
44.2	0.0083848223880668\\
44.21	0.00838491399013984\\
44.22	0.00838500563562933\\
44.23	0.00838509732456531\\
44.24	0.00838518905697783\\
44.25	0.00838528083289697\\
44.26	0.00838537265235283\\
44.27	0.00838546451537558\\
44.28	0.00838555642199538\\
44.29	0.00838564837224244\\
44.3	0.00838574036614699\\
44.31	0.00838583240373931\\
44.32	0.00838592448504969\\
44.33	0.00838601661010845\\
44.34	0.00838610877894596\\
44.35	0.0083862009915926\\
44.36	0.00838629324807879\\
44.37	0.008386385548435\\
44.38	0.00838647789269167\\
44.39	0.00838657028087935\\
44.4	0.00838666271302856\\
44.41	0.00838675518916988\\
44.42	0.00838684770933391\\
44.43	0.00838694027355128\\
44.44	0.00838703288185266\\
44.45	0.00838712553426874\\
44.46	0.00838721823083025\\
44.47	0.00838731097156794\\
44.48	0.0083874037565126\\
44.49	0.00838749658569505\\
44.5	0.00838758945914614\\
44.51	0.00838768237689674\\
44.52	0.00838777533897775\\
44.53	0.00838786834542014\\
44.54	0.00838796139625485\\
44.55	0.0083880544915129\\
44.56	0.00838814763122532\\
44.57	0.00838824081542316\\
44.58	0.00838833404413753\\
44.59	0.00838842731739954\\
44.6	0.00838852063524035\\
44.61	0.00838861399769114\\
44.62	0.00838870740478314\\
44.63	0.00838880085654759\\
44.64	0.00838889435301575\\
44.65	0.00838898789421895\\
44.66	0.00838908148018853\\
44.67	0.00838917511095584\\
44.68	0.00838926878655228\\
44.69	0.0083893625070093\\
44.7	0.00838945627235834\\
44.71	0.00838955008263091\\
44.72	0.00838964393785852\\
44.73	0.00838973783807272\\
44.74	0.0083898317833051\\
44.75	0.00838992577358727\\
44.76	0.00839001980895087\\
44.77	0.00839011388942758\\
44.78	0.0083902080150491\\
44.79	0.00839030218584717\\
44.8	0.00839039640185355\\
44.81	0.00839049066310003\\
44.82	0.00839058496961845\\
44.83	0.00839067932144066\\
44.84	0.00839077371859854\\
44.85	0.00839086816112401\\
44.86	0.00839096264904902\\
44.87	0.00839105718240554\\
44.88	0.00839115176122559\\
44.89	0.00839124638554119\\
44.9	0.00839134105538441\\
44.91	0.00839143577078735\\
44.92	0.00839153053178215\\
44.93	0.00839162533840094\\
44.94	0.00839172019067592\\
44.95	0.0083918150886393\\
44.96	0.00839191003232334\\
44.97	0.0083920050217603\\
44.98	0.00839210005698249\\
44.99	0.00839219513802225\\
45	0.00839229026491193\\
45.01	0.00839238543768394\\
45.02	0.00839248065637068\\
45.03	0.00839257592100463\\
45.04	0.00839267123161824\\
45.05	0.00839276658824404\\
45.06	0.00839286199091457\\
45.07	0.00839295743966238\\
45.08	0.00839305293452008\\
45.09	0.0083931484755203\\
45.1	0.00839324406269568\\
45.11	0.00839333969607891\\
45.12	0.0083934353757027\\
45.13	0.00839353110159979\\
45.14	0.00839362687380295\\
45.15	0.00839372269234498\\
45.16	0.00839381855725869\\
45.17	0.00839391446857695\\
45.18	0.00839401042633262\\
45.19	0.00839410643055862\\
45.2	0.00839420248128789\\
45.21	0.00839429857855338\\
45.22	0.00839439472238808\\
45.23	0.00839449091282502\\
45.24	0.00839458714989724\\
45.25	0.00839468343363781\\
45.26	0.00839477976407983\\
45.27	0.00839487614125642\\
45.28	0.00839497256520074\\
45.29	0.00839506903594597\\
45.3	0.0083951655535253\\
45.31	0.00839526211797197\\
45.32	0.00839535872931924\\
45.33	0.00839545538760039\\
45.34	0.00839555209284873\\
45.35	0.00839564884509759\\
45.36	0.00839574564438033\\
45.37	0.00839584249073033\\
45.38	0.008395939384181\\
45.39	0.00839603632476578\\
45.4	0.00839613331251812\\
45.41	0.00839623034747151\\
45.42	0.00839632742965945\\
45.43	0.00839642455911547\\
45.44	0.00839652173587314\\
45.45	0.00839661895996601\\
45.46	0.0083967162314277\\
45.47	0.00839681355029183\\
45.48	0.00839691091659203\\
45.49	0.00839700833036199\\
45.5	0.0083971057916354\\
45.51	0.00839720330044597\\
45.52	0.00839730085682742\\
45.53	0.00839739846081352\\
45.54	0.00839749611243804\\
45.55	0.00839759381173479\\
45.56	0.00839769155873757\\
45.57	0.00839778935348023\\
45.58	0.00839788719599663\\
45.59	0.00839798508632064\\
45.6	0.00839808302448616\\
45.61	0.00839818101052711\\
45.62	0.00839827904447743\\
45.63	0.00839837712637106\\
45.64	0.00839847525624198\\
45.65	0.00839857343412417\\
45.66	0.00839867166005165\\
45.67	0.00839876993405843\\
45.68	0.00839886825617855\\
45.69	0.00839896662644607\\
45.7	0.00839906504489506\\
45.71	0.0083991635115596\\
45.72	0.0083992620264738\\
45.73	0.00839936058967177\\
45.74	0.00839945920118764\\
45.75	0.00839955786105554\\
45.76	0.00839965656930964\\
45.77	0.00839975532598411\\
45.78	0.00839985413111311\\
45.79	0.00839995298473085\\
45.8	0.00840005188687151\\
45.81	0.00840015083756933\\
45.82	0.00840024983685851\\
45.83	0.00840034888477329\\
45.84	0.00840044798134791\\
45.85	0.00840054712661662\\
45.86	0.00840064632061367\\
45.87	0.00840074556337333\\
45.88	0.00840084485492986\\
45.89	0.00840094419531754\\
45.9	0.00840104358457064\\
45.91	0.00840114302272347\\
45.92	0.00840124250981029\\
45.93	0.0084013420458654\\
45.94	0.0084014416309231\\
45.95	0.00840154126501768\\
45.96	0.00840164094818344\\
45.97	0.00840174068045467\\
45.98	0.00840184046186567\\
45.99	0.00840194029245074\\
46	0.00840204017224416\\
46.01	0.00840214010128022\\
46.02	0.00840224007959321\\
46.03	0.00840234010721741\\
46.04	0.00840244018418709\\
46.05	0.00840254031053654\\
46.06	0.00840264048629999\\
46.07	0.00840274071151171\\
46.08	0.00840284098620593\\
46.09	0.0084029413104169\\
46.1	0.00840304168417884\\
46.11	0.00840314210752595\\
46.12	0.00840324258049243\\
46.13	0.00840334310311245\\
46.14	0.0084034436754202\\
46.15	0.00840354429744981\\
46.16	0.00840364496923542\\
46.17	0.00840374569081115\\
46.18	0.00840384646221108\\
46.19	0.00840394728346928\\
46.2	0.00840404815461981\\
46.21	0.00840414907569668\\
46.22	0.00840425004673391\\
46.23	0.00840435106776546\\
46.24	0.00840445213882527\\
46.25	0.00840455325994727\\
46.26	0.00840465443116533\\
46.27	0.00840475565251332\\
46.28	0.00840485692402505\\
46.29	0.00840495824573431\\
46.3	0.00840505961767485\\
46.31	0.00840516103988036\\
46.32	0.00840526251238454\\
46.33	0.008405364035221\\
46.34	0.00840546560842334\\
46.35	0.00840556723202508\\
46.36	0.00840566890605972\\
46.37	0.00840577063056072\\
46.38	0.00840587240556146\\
46.39	0.00840597423109529\\
46.4	0.00840607610719551\\
46.41	0.00840617803389533\\
46.42	0.00840628001122794\\
46.43	0.00840638203922645\\
46.44	0.00840648411792392\\
46.45	0.00840658624735334\\
46.46	0.00840668842754762\\
46.47	0.00840679065853963\\
46.48	0.00840689294036214\\
46.49	0.00840699527304786\\
46.5	0.00840709765662943\\
46.51	0.00840720009113941\\
46.52	0.00840730257661026\\
46.53	0.00840740511307438\\
46.54	0.00840750770056409\\
46.55	0.00840761033911159\\
46.56	0.00840771302874901\\
46.57	0.0084078157695084\\
46.58	0.00840791856142168\\
46.59	0.00840802140452069\\
46.6	0.00840812429883717\\
46.61	0.00840822724440276\\
46.62	0.00840833024124896\\
46.63	0.00840843328940719\\
46.64	0.00840853638890875\\
46.65	0.00840863953978481\\
46.66	0.00840874274206643\\
46.67	0.00840884599578454\\
46.68	0.00840894930096995\\
46.69	0.00840905265765332\\
46.7	0.00840915606586519\\
46.71	0.00840925952563596\\
46.72	0.00840936303699587\\
46.73	0.00840946659997505\\
46.74	0.00840957021460344\\
46.75	0.00840967388091085\\
46.76	0.00840977759892691\\
46.77	0.0084098813686811\\
46.78	0.00840998519020275\\
46.79	0.00841008906352099\\
46.8	0.00841019298866477\\
46.81	0.0084102969656629\\
46.82	0.00841040099454395\\
46.83	0.00841050507533634\\
46.84	0.00841060920806829\\
46.85	0.00841071339276779\\
46.86	0.00841081762946266\\
46.87	0.00841092191818049\\
46.88	0.00841102625894866\\
46.89	0.00841113065179433\\
46.9	0.00841123509674444\\
46.91	0.00841133959382567\\
46.92	0.00841144414306449\\
46.93	0.00841154874448712\\
46.94	0.00841165339811952\\
46.95	0.0084117581039874\\
46.96	0.00841186286211621\\
46.97	0.00841196767253113\\
46.98	0.00841207253525705\\
46.99	0.0084121774503186\\
47	0.00841228241774011\\
47.01	0.00841238743754561\\
47.02	0.00841249250975884\\
47.03	0.0084125976344032\\
47.04	0.00841270281150182\\
47.05	0.00841280804107746\\
47.06	0.00841291332315257\\
47.07	0.00841301865774924\\
47.08	0.00841312404488924\\
47.09	0.00841322948459394\\
47.1	0.0084133349768844\\
47.11	0.00841344052178125\\
47.12	0.00841354611930477\\
47.13	0.00841365176947483\\
47.14	0.00841375747231092\\
47.15	0.00841386322783209\\
47.16	0.00841396903605699\\
47.17	0.00841407489700385\\
47.18	0.00841418081069042\\
47.19	0.00841428677713404\\
47.2	0.00841439279635156\\
47.21	0.00841449886835939\\
47.22	0.00841460499317341\\
47.23	0.00841471117080906\\
47.24	0.00841481740128125\\
47.25	0.00841492368460436\\
47.26	0.00841503002079227\\
47.27	0.0084151364098583\\
47.28	0.00841524285181523\\
47.29	0.00841534934667526\\
47.3	0.00841545589445004\\
47.31	0.00841556249515059\\
47.32	0.00841566914878737\\
47.33	0.0084157758553702\\
47.34	0.00841588261490826\\
47.35	0.00841598942741009\\
47.36	0.00841609629288359\\
47.37	0.00841620321133596\\
47.38	0.00841631018277373\\
47.39	0.00841641720720271\\
47.4	0.00841652428462799\\
47.41	0.00841663141505394\\
47.42	0.00841673859848416\\
47.43	0.00841684583492149\\
47.44	0.00841695312436798\\
47.45	0.00841706046682487\\
47.46	0.00841716786229259\\
47.47	0.00841727531077071\\
47.48	0.00841738281225798\\
47.49	0.00841749036675224\\
47.5	0.00841759797425043\\
47.51	0.00841770563474861\\
47.52	0.00841781334824186\\
47.53	0.00841792111472434\\
47.54	0.00841802893418921\\
47.55	0.00841813680662864\\
47.56	0.00841824473203379\\
47.57	0.00841835271039476\\
47.58	0.0084184607417006\\
47.59	0.00841856882593926\\
47.6	0.00841867696309758\\
47.61	0.00841878515316128\\
47.62	0.00841889339611492\\
47.63	0.00841900169194184\\
47.64	0.00841911004062421\\
47.65	0.00841921844214296\\
47.66	0.00841932689647773\\
47.67	0.0084194354036069\\
47.68	0.00841954396350753\\
47.69	0.00841965257615532\\
47.7	0.0084197612415246\\
47.71	0.00841986995958831\\
47.72	0.00841997873031795\\
47.73	0.00842008755368356\\
47.74	0.00842019642965368\\
47.75	0.00842030535819534\\
47.76	0.00842041433927398\\
47.77	0.00842052337285349\\
47.78	0.00842063245889612\\
47.79	0.00842074159736245\\
47.8	0.00842085078821139\\
47.81	0.00842096003140011\\
47.82	0.00842106932688401\\
47.83	0.0084211786746167\\
47.84	0.00842128807454996\\
47.85	0.00842139752663367\\
47.86	0.00842150703081582\\
47.87	0.00842161658704241\\
47.88	0.00842172619525749\\
47.89	0.00842183585540305\\
47.9	0.00842194556741898\\
47.91	0.00842205533124309\\
47.92	0.00842216514681099\\
47.93	0.00842227501405611\\
47.94	0.00842238493290961\\
47.95	0.00842249490330033\\
47.96	0.00842260492515481\\
47.97	0.00842271499839714\\
47.98	0.00842282512294899\\
47.99	0.00842293529872956\\
48	0.00842304552565546\\
48.01	0.00842315580364073\\
48.02	0.00842326613259677\\
48.03	0.00842337651243224\\
48.04	0.00842348694305309\\
48.05	0.00842359742436243\\
48.06	0.0084237079562605\\
48.07	0.00842381853864462\\
48.08	0.00842392917140915\\
48.09	0.00842403985444536\\
48.1	0.00842415058764146\\
48.11	0.00842426137088247\\
48.12	0.00842437220405019\\
48.13	0.00842448308702313\\
48.14	0.00842459401967645\\
48.15	0.00842470500188186\\
48.16	0.00842481603350762\\
48.17	0.0084249271144184\\
48.18	0.00842503824447527\\
48.19	0.0084251494235356\\
48.2	0.00842526065145296\\
48.21	0.00842537192807712\\
48.22	0.00842548325325391\\
48.23	0.00842559462682518\\
48.24	0.00842570604862871\\
48.25	0.00842581751849814\\
48.26	0.00842592903626288\\
48.27	0.00842604060174803\\
48.28	0.00842615221477433\\
48.29	0.00842626387515803\\
48.3	0.00842637558271083\\
48.31	0.00842648733723981\\
48.32	0.0084265991385473\\
48.33	0.00842671098643085\\
48.34	0.00842682288068308\\
48.35	0.00842693482109164\\
48.36	0.00842704680743908\\
48.37	0.0084271588395028\\
48.38	0.00842727091705489\\
48.39	0.00842738303986211\\
48.4	0.00842749520768574\\
48.41	0.00842760742028151\\
48.42	0.00842771967739946\\
48.43	0.00842783197878389\\
48.44	0.00842794432417324\\
48.45	0.00842805671329996\\
48.46	0.00842816914589045\\
48.47	0.00842828162166491\\
48.48	0.00842839414033725\\
48.49	0.00842850670161499\\
48.5	0.00842861930519915\\
48.51	0.00842873195078412\\
48.52	0.00842884463805755\\
48.53	0.00842895736670027\\
48.54	0.00842907013638611\\
48.55	0.00842918294678186\\
48.56	0.00842929579754708\\
48.57	0.00842940868833405\\
48.58	0.00842952161878758\\
48.59	0.00842963458854496\\
48.6	0.00842974759723576\\
48.61	0.00842986064448179\\
48.62	0.00842997372989691\\
48.63	0.00843008685308693\\
48.64	0.00843020001364951\\
48.65	0.00843031321117396\\
48.66	0.0084304264452412\\
48.67	0.00843053971542357\\
48.68	0.00843065302128474\\
48.69	0.00843076636237955\\
48.7	0.00843087973825388\\
48.71	0.00843099314844455\\
48.72	0.00843110659247918\\
48.73	0.00843122006987601\\
48.74	0.00843133358014386\\
48.75	0.00843144712278189\\
48.76	0.00843156069727957\\
48.77	0.00843167430311648\\
48.78	0.00843178793976221\\
48.79	0.0084319016066762\\
48.8	0.00843201530330766\\
48.81	0.00843212902909538\\
48.82	0.00843224278346767\\
48.83	0.00843235656584213\\
48.84	0.00843247037562563\\
48.85	0.00843258421221414\\
48.86	0.00843269807499257\\
48.87	0.00843281196333469\\
48.88	0.00843292587660302\\
48.89	0.00843303981414865\\
48.9	0.00843315377531118\\
48.91	0.00843326775941858\\
48.92	0.00843338176578708\\
48.93	0.00843349579372105\\
48.94	0.00843360984251293\\
48.95	0.00843372391144306\\
48.96	0.00843383799977965\\
48.97	0.00843395210677864\\
48.98	0.00843406623168363\\
48.99	0.00843418037372578\\
49	0.00843429453212372\\
49.01	0.00843440870608351\\
49.02	0.00843452289479851\\
49.03	0.00843463709744937\\
49.04	0.00843475131320395\\
49.05	0.00843486554121725\\
49.06	0.00843497978063141\\
49.07	0.0084350940305756\\
49.08	0.00843520829016608\\
49.09	0.00843532255850612\\
49.1	0.008435436834686\\
49.11	0.00843555111778301\\
49.12	0.00843566540686148\\
49.13	0.00843577970097278\\
49.14	0.00843589399915537\\
49.15	0.0084360083004348\\
49.16	0.00843612260382384\\
49.17	0.0084362369083225\\
49.18	0.00843635121291812\\
49.19	0.0084364655165855\\
49.2	0.008436579818287\\
49.21	0.00843669411697268\\
49.22	0.00843680841158044\\
49.23	0.00843692270103621\\
49.24	0.00843703698425415\\
49.25	0.00843715126013684\\
49.26	0.00843726552757552\\
49.27	0.00843737978545038\\
49.28	0.0084374940326308\\
49.29	0.00843760826797568\\
49.3	0.0084377224903338\\
49.31	0.00843783669854412\\
49.32	0.00843795089143624\\
49.33	0.0084380650678308\\
49.34	0.00843817922653992\\
49.35	0.00843829336636772\\
49.36	0.00843840748611083\\
49.37	0.00843852158455894\\
49.38	0.00843863566049547\\
49.39	0.0084387497126981\\
49.4	0.00843886373993959\\
49.41	0.0084389777409884\\
49.42	0.00843909171460953\\
49.43	0.00843920565956533\\
49.44	0.0084393195746164\\
49.45	0.00843943345852249\\
49.46	0.00843954731004351\\
49.47	0.00843966112794057\\
49.48	0.00843977491097711\\
49.49	0.00843988865792003\\
49.5	0.00844000236754099\\
49.51	0.00844011603861766\\
49.52	0.00844022966993514\\
49.53	0.00844034326028741\\
49.54	0.00844045680847887\\
49.55	0.00844057031332595\\
49.56	0.00844068377365882\\
49.57	0.00844079718832321\\
49.58	0.00844091055618229\\
49.59	0.00844102387611864\\
49.6	0.00844113714703635\\
49.61	0.00844125036786326\\
49.62	0.00844136353755321\\
49.63	0.00844147665508849\\
49.64	0.00844158971948243\\
49.65	0.00844170272978199\\
49.66	0.00844181568507065\\
49.67	0.00844192858447128\\
49.68	0.00844204142714929\\
49.69	0.00844215421231579\\
49.7	0.00844226693923104\\
49.71	0.00844237960720798\\
49.72	0.00844249221561593\\
49.73	0.00844260476388452\\
49.74	0.00844271725150768\\
49.75	0.00844282967804803\\
49.76	0.00844294204314122\\
49.77	0.00844305434650067\\
49.78	0.00844316658792242\\
49.79	0.00844327876729024\\
49.8	0.00844339088458094\\
49.81	0.00844350293987\\
49.82	0.00844361493333733\\
49.83	0.00844372686527339\\
49.84	0.00844383873608556\\
49.85	0.00844395054630473\\
49.86	0.0084440622965923\\
49.87	0.00844417398774735\\
49.88	0.00844428562071425\\
49.89	0.00844439719659051\\
49.9	0.00844450871663504\\
49.91	0.00844462018227671\\
49.92	0.00844473159512332\\
49.93	0.00844484295697096\\
49.94	0.00844495426981371\\
49.95	0.00844506553585388\\
49.96	0.00844517675751252\\
49.97	0.00844528793744054\\
49.98	0.00844539907853018\\
49.99	0.00844551018392705\\
50	0.00844562125704263\\
50.01	0.00844573230156728\\
50.02	0.00844584332148386\\
50.03	0.00844595432108189\\
50.04	0.00844606530497225\\
50.05	0.00844617627810256\\
50.06	0.00844628724577321\\
50.07	0.00844639821365394\\
50.08	0.00844650918780126\\
50.09	0.00844662017467648\\
50.1	0.0084467311811645\\
50.11	0.00844684221459339\\
50.12	0.00844695328275479\\
50.13	0.0084470643939251\\
50.14	0.00844717555688753\\
50.15	0.00844728678095511\\
50.16	0.00844739806847898\\
50.17	0.00844750941953051\\
50.18	0.00844762083418143\\
50.19	0.00844773231250382\\
50.2	0.00844784385457007\\
50.21	0.0084479554604529\\
50.22	0.0084480671302253\\
50.23	0.00844817886396051\\
50.24	0.00844829066173205\\
50.25	0.0084484025236136\\
50.26	0.00844851444967902\\
50.27	0.00844862644000236\\
50.28	0.00844873849465771\\
50.29	0.00844885061371928\\
50.3	0.00844896279726128\\
50.31	0.00844907504535789\\
50.32	0.00844918735808324\\
50.33	0.00844929973551131\\
50.34	0.00844941217771592\\
50.35	0.00844952468477064\\
50.36	0.00844963725674871\\
50.37	0.008449749893723\\
50.38	0.00844986259576593\\
50.39	0.00844997536294938\\
50.4	0.00845008819534459\\
50.41	0.00845020109302209\\
50.42	0.0084503140560516\\
50.43	0.00845042708450191\\
50.44	0.00845054017844079\\
50.45	0.00845065333793483\\
50.46	0.00845076656304939\\
50.47	0.00845087985384836\\
50.48	0.00845099321039414\\
50.49	0.00845110663274791\\
50.5	0.00845122012097097\\
50.51	0.00845133367512472\\
50.52	0.00845144729527062\\
50.53	0.00845156098147027\\
50.54	0.00845167473378534\\
50.55	0.00845178855227758\\
50.56	0.00845190243700889\\
50.57	0.00845201638804121\\
50.58	0.0084521304054366\\
50.59	0.00845224448925723\\
50.6	0.00845235863956535\\
50.61	0.0084524728564233\\
50.62	0.00845258713989355\\
50.63	0.00845270149003864\\
50.64	0.00845281590692122\\
50.65	0.00845293039060404\\
50.66	0.00845304494114995\\
50.67	0.00845315955862189\\
50.68	0.00845327424308292\\
50.69	0.00845338899459619\\
50.7	0.00845350381322496\\
50.71	0.00845361869903257\\
50.72	0.00845373365208248\\
50.73	0.00845384867243826\\
50.74	0.00845396376016357\\
50.75	0.00845407891532218\\
50.76	0.00845419413797796\\
50.77	0.00845430942819488\\
50.78	0.00845442478603704\\
50.79	0.00845454021156861\\
50.8	0.00845465570485389\\
50.81	0.00845477126595728\\
50.82	0.00845488689494329\\
50.83	0.00845500259187654\\
50.84	0.00845511835682175\\
50.85	0.00845523418984375\\
50.86	0.00845535009100749\\
50.87	0.00845546606037801\\
50.88	0.00845558209802048\\
50.89	0.00845569820400017\\
50.9	0.00845581437838246\\
50.91	0.00845593062123286\\
50.92	0.00845604693261696\\
50.93	0.0084561633126005\\
50.94	0.0084562797612493\\
50.95	0.00845639627862931\\
50.96	0.0084565128648066\\
50.97	0.00845662951984735\\
50.98	0.00845674624381785\\
50.99	0.00845686303678452\\
51	0.00845697989881388\\
51.01	0.00845709682997258\\
51.02	0.00845721383032739\\
51.03	0.00845733089994519\\
51.04	0.00845744803889298\\
51.05	0.00845756524723789\\
51.06	0.00845768252504717\\
51.07	0.00845779987238819\\
51.08	0.00845791728932842\\
51.09	0.0084580347759355\\
51.1	0.00845815233227715\\
51.11	0.00845826995842124\\
51.12	0.00845838765443575\\
51.13	0.0084585054203888\\
51.14	0.00845862325634864\\
51.15	0.00845874116238362\\
51.16	0.00845885913856224\\
51.17	0.00845897718495314\\
51.18	0.00845909530162506\\
51.19	0.00845921348864689\\
51.2	0.00845933174608767\\
51.21	0.00845945007401653\\
51.22	0.00845956847250276\\
51.23	0.00845968694161578\\
51.24	0.00845980548142514\\
51.25	0.00845992409200054\\
51.26	0.0084600427734118\\
51.27	0.0084601615257289\\
51.28	0.00846028034902192\\
51.29	0.00846039924336113\\
51.3	0.00846051820881689\\
51.31	0.00846063724545975\\
51.32	0.00846075635336036\\
51.33	0.00846087553258955\\
51.34	0.00846099478321826\\
51.35	0.00846111410531762\\
51.36	0.00846123349895886\\
51.37	0.00846135296421338\\
51.38	0.00846147250115273\\
51.39	0.00846159210984863\\
51.4	0.00846171179037289\\
51.41	0.00846183154279755\\
51.42	0.00846195136719474\\
51.43	0.00846207126363678\\
51.44	0.00846219123219614\\
51.45	0.00846231127294544\\
51.46	0.00846243138595747\\
51.47	0.00846255157130515\\
51.48	0.00846267182906161\\
51.49	0.00846279215930011\\
51.5	0.00846291256209407\\
51.51	0.00846303303751709\\
51.52	0.00846315358564293\\
51.53	0.00846327420654552\\
51.54	0.00846339490029896\\
51.55	0.00846351566697752\\
51.56	0.00846363650665564\\
51.57	0.00846375741940793\\
51.58	0.00846387840530919\\
51.59	0.00846399946443438\\
51.6	0.00846412059685865\\
51.61	0.00846424180265732\\
51.62	0.0084643630819059\\
51.63	0.00846448443468008\\
51.64	0.00846460586105573\\
51.65	0.00846472736110892\\
51.66	0.0084648489349159\\
51.67	0.0084649705825531\\
51.68	0.00846509230409716\\
51.69	0.00846521409962491\\
51.7	0.00846533596921336\\
51.71	0.00846545791293974\\
51.72	0.00846557993088146\\
51.73	0.00846570202311614\\
51.74	0.00846582418972161\\
51.75	0.00846594643077589\\
51.76	0.00846606874635723\\
51.77	0.00846619113654406\\
51.78	0.00846631360141505\\
51.79	0.00846643614104907\\
51.8	0.0084665587555252\\
51.81	0.00846668144492275\\
51.82	0.00846680420932125\\
51.83	0.00846692704880045\\
51.84	0.00846704996344032\\
51.85	0.00846717295332108\\
51.86	0.00846729601852314\\
51.87	0.00846741915912718\\
51.88	0.0084675423752141\\
51.89	0.00846766566686504\\
51.9	0.00846778903416139\\
51.91	0.00846791247718476\\
51.92	0.00846803599601703\\
51.93	0.00846815959074032\\
51.94	0.00846828326143699\\
51.95	0.00846840700818968\\
51.96	0.00846853083108127\\
51.97	0.0084686547301949\\
51.98	0.00846877870561399\\
51.99	0.0084689027574222\\
52	0.00846902688570349\\
52.01	0.00846915109054207\\
52.02	0.00846927537202244\\
52.03	0.00846939973022937\\
52.04	0.00846952416524793\\
52.05	0.00846964867716346\\
52.06	0.00846977326606159\\
52.07	0.00846989793202826\\
52.08	0.0084700226751497\\
52.09	0.00847014749551243\\
52.1	0.00847027239320329\\
52.11	0.00847039736830944\\
52.12	0.00847052242091832\\
52.13	0.00847064755111771\\
52.14	0.00847077275899571\\
52.15	0.00847089804464075\\
52.16	0.00847102340814158\\
52.17	0.00847114884958728\\
52.18	0.00847127436906729\\
52.19	0.00847139996667137\\
52.2	0.00847152564248964\\
52.21	0.00847165139661258\\
52.22	0.00847177722913101\\
52.23	0.00847190314013612\\
52.24	0.00847202912971948\\
52.25	0.00847215519797301\\
52.26	0.00847228134498903\\
52.27	0.00847240757086023\\
52.28	0.0084725338756797\\
52.29	0.0084726602595409\\
52.3	0.00847278672253771\\
52.31	0.00847291326476443\\
52.32	0.00847303988631572\\
52.33	0.00847316658728672\\
52.34	0.00847329336777294\\
52.35	0.00847342022787036\\
52.36	0.00847354716767537\\
52.37	0.00847367418728481\\
52.38	0.00847380128679597\\
52.39	0.0084739284663066\\
52.4	0.00847405572591491\\
52.41	0.00847418306571958\\
52.42	0.00847431048581974\\
52.43	0.00847443798631506\\
52.44	0.00847456556730565\\
52.45	0.00847469322889213\\
52.46	0.00847482097117565\\
52.47	0.00847494879425785\\
52.48	0.0084750766982409\\
52.49	0.00847520468322751\\
52.5	0.00847533274932091\\
52.51	0.00847546089662488\\
52.52	0.00847558912524378\\
52.53	0.0084757174352825\\
52.54	0.00847584582684653\\
52.55	0.00847597430004193\\
52.56	0.00847610285497537\\
52.57	0.00847623149175408\\
52.58	0.00847636021048595\\
52.59	0.00847648901127947\\
52.6	0.00847661789424375\\
52.61	0.00847674685948858\\
52.62	0.00847687590712435\\
52.63	0.00847700503726216\\
52.64	0.00847713425001377\\
52.65	0.0084772635454916\\
52.66	0.00847739292380881\\
52.67	0.00847752238507925\\
52.68	0.00847765192941748\\
52.69	0.0084777815569388\\
52.7	0.00847791126775927\\
52.71	0.0084780410619957\\
52.72	0.00847817093976566\\
52.73	0.00847830090118753\\
52.74	0.00847843094638044\\
52.75	0.00847856107546441\\
52.76	0.00847869128856021\\
52.77	0.00847882158578949\\
52.78	0.00847895196727474\\
52.79	0.00847908243313935\\
52.8	0.00847921298350756\\
52.81	0.00847934361850452\\
52.82	0.00847947433825631\\
52.83	0.00847960514288994\\
52.84	0.00847973603253336\\
52.85	0.00847986700731549\\
52.86	0.00847999806736625\\
52.87	0.00848012921281653\\
52.88	0.00848026044379829\\
52.89	0.00848039176044448\\
52.9	0.00848052316288913\\
52.91	0.00848065465126736\\
52.92	0.00848078622571536\\
52.93	0.00848091788637045\\
52.94	0.00848104963337108\\
52.95	0.00848118146685688\\
52.96	0.00848131338696863\\
52.97	0.00848144539384834\\
52.98	0.00848157748763922\\
52.99	0.00848170966848574\\
53	0.00848184193653363\\
53.01	0.00848197429192993\\
53.02	0.00848210673482299\\
53.03	0.00848223926536249\\
53.04	0.00848237188369949\\
53.05	0.00848250458998646\\
53.06	0.00848263738437727\\
53.07	0.00848277026702723\\
53.08	0.00848290323809314\\
53.09	0.00848303629773331\\
53.1	0.00848316944610756\\
53.11	0.00848330268337731\\
53.12	0.00848343600970551\\
53.13	0.0084835694252568\\
53.14	0.00848370293019742\\
53.15	0.00848383652469533\\
53.16	0.0084839702089202\\
53.17	0.00848410398304344\\
53.18	0.00848423784723826\\
53.19	0.00848437180167968\\
53.2	0.00848450584654459\\
53.21	0.00848463998201177\\
53.22	0.00848477420826192\\
53.23	0.00848490852547772\\
53.24	0.00848504293384385\\
53.25	0.00848517743354705\\
53.26	0.00848531202477612\\
53.27	0.00848544670772203\\
53.28	0.00848558148257789\\
53.29	0.00848571634953904\\
53.3	0.00848585130880308\\
53.31	0.0084859863605699\\
53.32	0.00848612150504176\\
53.33	0.00848625674242332\\
53.34	0.00848639207292166\\
53.35	0.0084865274967464\\
53.36	0.00848666301410967\\
53.37	0.0084867986252262\\
53.38	0.0084869343303134\\
53.39	0.00848707012959136\\
53.4	0.00848720602328295\\
53.41	0.00848734201161384\\
53.42	0.00848747809481258\\
53.43	0.00848761427311067\\
53.44	0.00848775054674257\\
53.45	0.00848788691594583\\
53.46	0.00848802338096111\\
53.47	0.00848815994203223\\
53.48	0.00848829659940629\\
53.49	0.00848843335333369\\
53.5	0.00848857020406822\\
53.51	0.00848870715186712\\
53.52	0.00848884419699116\\
53.53	0.00848898133970472\\
53.54	0.00848911858027584\\
53.55	0.00848925591897633\\
53.56	0.00848939335608181\\
53.57	0.00848953089187184\\
53.58	0.00848966852662994\\
53.59	0.00848980626064375\\
53.6	0.00848994409420503\\
53.61	0.00849008202760982\\
53.62	0.0084902200611585\\
53.63	0.00849035819515587\\
53.64	0.00849049642991127\\
53.65	0.00849063476573867\\
53.66	0.00849077320295674\\
53.67	0.00849091174188899\\
53.68	0.00849105038286384\\
53.69	0.00849118912621478\\
53.7	0.00849132797228038\\
53.71	0.00849146692140451\\
53.72	0.00849160597393636\\
53.73	0.00849174513023062\\
53.74	0.00849188439064757\\
53.75	0.0084920237555532\\
53.76	0.00849216322531935\\
53.77	0.00849230280032379\\
53.78	0.00849244248095041\\
53.79	0.00849258226758931\\
53.8	0.00849272216063697\\
53.81	0.00849286216049634\\
53.82	0.00849300226757701\\
53.83	0.00849314248229536\\
53.84	0.0084932828050747\\
53.85	0.00849342323634541\\
53.86	0.00849356377654511\\
53.87	0.00849370442611882\\
53.88	0.00849384518551909\\
53.89	0.00849398605520621\\
53.9	0.00849412703564835\\
53.91	0.00849426812732174\\
53.92	0.00849440933071086\\
53.93	0.00849455064630859\\
53.94	0.00849469207461643\\
53.95	0.00849483361614465\\
53.96	0.00849497527141255\\
53.97	0.00849511704094857\\
53.98	0.00849525892529058\\
53.99	0.008495400924986\\
54	0.00849554304059212\\
54.01	0.0084956852726762\\
54.02	0.00849582762181578\\
54.03	0.00849597008859888\\
54.04	0.00849611267362421\\
54.05	0.00849625537750144\\
54.06	0.00849639820085145\\
54.07	0.00849654114430652\\
54.08	0.00849668420851066\\
54.09	0.00849682739411983\\
54.1	0.00849697070180223\\
54.11	0.00849711413223853\\
54.12	0.0084972576861222\\
54.13	0.00849740136415978\\
54.14	0.00849754516707116\\
54.15	0.0084976890955899\\
54.16	0.00849783315046352\\
54.17	0.00849797733245381\\
54.18	0.00849812164233719\\
54.19	0.00849826608090499\\
54.2	0.00849841064896382\\
54.21	0.0084985553473359\\
54.22	0.00849870017685943\\
54.23	0.00849884513838891\\
54.24	0.00849899023279557\\
54.25	0.0084991354609677\\
54.26	0.00849928082381104\\
54.27	0.00849942632224921\\
54.28	0.0084995719572241\\
54.29	0.00849971772969626\\
54.3	0.00849986364064535\\
54.31	0.00850000969107058\\
54.32	0.00850015588199114\\
54.33	0.00850030221444666\\
54.34	0.00850044868949766\\
54.35	0.00850059530822607\\
54.36	0.00850074207173567\\
54.37	0.00850088898115264\\
54.38	0.008501036037626\\
54.39	0.00850118324232823\\
54.4	0.00850133059645572\\
54.41	0.00850147810122934\\
54.42	0.00850162575789505\\
54.43	0.00850177356772442\\
54.44	0.00850192153201522\\
54.45	0.00850206965209207\\
54.46	0.00850221792930701\\
54.47	0.00850236636504016\\
54.48	0.00850251496070033\\
54.49	0.00850266371772571\\
54.5	0.00850281263758457\\
54.51	0.00850296172177588\\
54.52	0.0085031109718301\\
54.53	0.00850326038930984\\
54.54	0.00850340997581065\\
54.55	0.00850355973296177\\
54.56	0.00850370966242687\\
54.57	0.00850385976590492\\
54.58	0.00850401004513093\\
54.59	0.00850416050187684\\
54.6	0.00850431113795234\\
54.61	0.00850446195520577\\
54.62	0.008504612955525\\
54.63	0.00850476414083836\\
54.64	0.00850491551311556\\
54.65	0.00850506707436867\\
54.66	0.0085052188266531\\
54.67	0.00850537077206859\\
54.68	0.00850552291276026\\
54.69	0.00850567525092023\\
54.7	0.00850582778878998\\
54.71	0.00850598052866161\\
54.72	0.00850613347287908\\
54.73	0.00850628662383949\\
54.74	0.00850643998399442\\
54.75	0.00850659355585124\\
54.76	0.00850674734197457\\
54.77	0.00850690134498764\\
54.78	0.00850705556757377\\
54.79	0.0085072100124779\\
54.8	0.0085073646825081\\
54.81	0.00850751958053716\\
54.82	0.00850767470950421\\
54.83	0.00850783007241637\\
54.84	0.0085079856723505\\
54.85	0.00850814151245491\\
54.86	0.00850829759595117\\
54.87	0.00850845392613597\\
54.88	0.00850861050638298\\
54.89	0.00850876734014485\\
54.9	0.00850892443095514\\
54.91	0.00850908178243041\\
54.92	0.00850923939827233\\
54.93	0.0085093972822698\\
54.94	0.00850955543830118\\
54.95	0.00850971387033658\\
54.96	0.00850987258244018\\
54.97	0.00851003157877261\\
54.98	0.00851019086359344\\
54.99	0.00851035044126368\\
55	0.00851051031624838\\
55.01	0.00851067049311928\\
55.02	0.00851083097655757\\
55.03	0.00851099177135666\\
55.04	0.00851115288242508\\
55.05	0.00851131431478941\\
55.06	0.00851147607359736\\
55.07	0.00851163816412083\\
55.08	0.00851180059175915\\
55.09	0.00851196336204235\\
55.1	0.00851212648063455\\
55.11	0.00851228995333737\\
55.12	0.00851245378609356\\
55.13	0.0085126179849906\\
55.14	0.00851278255626448\\
55.15	0.00851294750630351\\
55.16	0.00851311284165234\\
55.17	0.00851327856901597\\
55.18	0.00851344469526393\\
55.19	0.00851361122743458\\
55.2	0.00851377817273952\\
55.21	0.00851394553856806\\
55.22	0.00851411333249193\\
55.23	0.00851428156226998\\
55.24	0.00851445023585315\\
55.25	0.00851461936138944\\
55.26	0.00851478894722909\\
55.27	0.00851495900192993\\
55.28	0.00851512953426279\\
55.29	0.00851530055321709\\
55.3	0.00851547206800666\\
55.31	0.00851564408807558\\
55.32	0.0085158166231043\\
55.33	0.00851598968301586\\
55.34	0.0085161632779823\\
55.35	0.00851633741843126\\
55.36	0.00851651211505273\\
55.37	0.008516687378806\\
55.38	0.00851686322092684\\
55.39	0.0085170396529348\\
55.4	0.00851721668664076\\
55.41	0.00851739433415468\\
55.42	0.0085175726078936\\
55.43	0.00851775152058975\\
55.44	0.00851793108529904\\
55.45	0.00851811131540964\\
55.46	0.00851829222465087\\
55.47	0.00851847382710237\\
55.48	0.00851865613720344\\
55.49	0.00851883916976265\\
55.5	0.00851902293996781\\
55.51	0.00851920746339613\\
55.52	0.00851939275602467\\
55.53	0.00851957883424108\\
55.54	0.00851976571485469\\
55.55	0.00851995341510786\\
55.56	0.00852014195268765\\
55.57	0.00852033134573783\\
55.58	0.00852052161287122\\
55.59	0.00852071277318235\\
55.6	0.00852090484626057\\
55.61	0.00852109785220335\\
55.62	0.00852129181163015\\
55.63	0.00852148674569652\\
55.64	0.0085216826761087\\
55.65	0.0085218796251386\\
55.66	0.00852207761563913\\
55.67	0.0085222766710601\\
55.68	0.00852247681546441\\
55.69	0.00852267807354485\\
55.7	0.00852288047064127\\
55.71	0.00852308403275823\\
55.72	0.00852328878658321\\
55.73	0.00852349475950532\\
55.74	0.00852370197963448\\
55.75	0.00852391047582124\\
55.76	0.00852412027767705\\
55.77	0.00852433141559518\\
55.78	0.00852454392077222\\
55.79	0.00852475782523019\\
55.8	0.00852497316183922\\
55.81	0.00852518996434101\\
55.82	0.00852540826737276\\
55.83	0.00852562810649199\\
55.84	0.00852584951820189\\
55.85	0.00852607253997754\\
55.86	0.00852629721029272\\
55.87	0.00852652356864765\\
55.88	0.00852675165559738\\
55.89	0.00852698151278108\\
55.9	0.00852721318295214\\
55.91	0.00852744671000912\\
55.92	0.00852768213902761\\
55.93	0.00852791951629295\\
55.94	0.00852815888933396\\
55.95	0.00852840030695758\\
55.96	0.00852864381928454\\
55.97	0.00852888947778603\\
55.98	0.00852913733532144\\
55.99	0.00852938744617717\\
56	0.00852963986610655\\
56.01	0.00852989465237098\\
56.02	0.00853015186378211\\
56.03	0.00853041156074541\\
56.04	0.00853067380530485\\
56.05	0.00853093866118897\\
56.06	0.00853120619385824\\
56.07	0.00853147647055374\\
56.08	0.00853174956034737\\
56.09	0.00853202553419341\\
56.1	0.00853230446498159\\
56.11	0.00853258642759174\\
56.12	0.00853287149894999\\
56.13	0.00853315975808659\\
56.14	0.00853345128619546\\
56.15	0.00853374616669544\\
56.16	0.00853404448529328\\
56.17	0.00853434633004855\\
56.18	0.00853465179144035\\
56.19	0.00853496096243605\\
56.2	0.00853527393856196\\
56.21	0.00853559081797611\\
56.22	0.00853591170154313\\
56.23	0.0085362366929113\\
56.24	0.00853656589859193\\
56.25	0.00853689942804097\\
56.26	0.00853723739374306\\
56.27	0.00853757837316989\\
56.28	0.00853791948362209\\
56.29	0.00853826072010381\\
56.3	0.00853860207737868\\
56.31	0.00853894354996083\\
56.32	0.00853928513210557\\
56.33	0.00853962681779968\\
56.34	0.00853996860075147\\
56.35	0.00854031047438045\\
56.36	0.00854065243180663\\
56.37	0.00854099446583951\\
56.38	0.00854133656896664\\
56.39	0.00854167873334183\\
56.4	0.00854202095077297\\
56.41	0.0085423632127094\\
56.42	0.00854270551022885\\
56.43	0.00854304783402401\\
56.44	0.00854339017438857\\
56.45	0.00854373252120283\\
56.46	0.00854407486391883\\
56.47	0.00854441719154492\\
56.48	0.00854475949262996\\
56.49	0.00854510175524678\\
56.5	0.0085454439669753\\
56.51	0.00854578611488494\\
56.52	0.00854612818551648\\
56.53	0.00854647016486339\\
56.54	0.00854681203835244\\
56.55	0.00854715379082369\\
56.56	0.00854749540650987\\
56.57	0.00854783686901502\\
56.58	0.00854817816129245\\
56.59	0.00854851926562195\\
56.6	0.0085488601635863\\
56.61	0.00854920083604691\\
56.62	0.00854954126311877\\
56.63	0.0085498814241445\\
56.64	0.00855022129766756\\
56.65	0.00855056086140459\\
56.66	0.00855090009221686\\
56.67	0.00855123896608076\\
56.68	0.00855157745805728\\
56.69	0.00855191554226065\\
56.7	0.00855225319182572\\
56.71	0.0085525903788745\\
56.72	0.00855292707448145\\
56.73	0.00855326324863775\\
56.74	0.00855359887021433\\
56.75	0.00855393390692369\\
56.76	0.0085542683252806\\
56.77	0.00855460209056137\\
56.78	0.00855493516676187\\
56.79	0.00855526751655426\\
56.8	0.00855559910124216\\
56.81	0.0085559298807145\\
56.82	0.00855625981339784\\
56.83	0.00855658885620715\\
56.84	0.00855691696449496\\
56.85	0.00855724409199899\\
56.86	0.00855757019078793\\
56.87	0.00855789521120563\\
56.88	0.00855821910181337\\
56.89	0.00855854180933038\\
56.9	0.00855886327857238\\
56.91	0.00855918345238817\\
56.92	0.00855950227159422\\
56.93	0.00855981967490711\\
56.94	0.00856013699889897\\
56.95	0.00856045446145748\\
56.96	0.00856077206266139\\
56.97	0.00856108980258951\\
56.98	0.00856140768132069\\
56.99	0.00856172569893389\\
57	0.00856204385550809\\
57.01	0.00856236215112235\\
57.02	0.00856268058585579\\
57.03	0.00856299915978762\\
57.04	0.00856331787299706\\
57.05	0.00856363672556346\\
57.06	0.00856395571756617\\
57.07	0.00856427484908465\\
57.08	0.00856459412019841\\
57.09	0.00856491353098702\\
57.1	0.0085652330815301\\
57.11	0.00856555277190738\\
57.12	0.0085658726021986\\
57.13	0.0085661925724836\\
57.14	0.00856651268284228\\
57.15	0.00856683293335458\\
57.16	0.00856715332410054\\
57.17	0.00856747385516025\\
57.18	0.00856779452661384\\
57.19	0.00856811533854154\\
57.2	0.00856843629102363\\
57.21	0.00856875738414045\\
57.22	0.00856907861797241\\
57.23	0.00856939999259998\\
57.24	0.00856972150810371\\
57.25	0.00857004316456419\\
57.26	0.00857036496206209\\
57.27	0.00857068690067815\\
57.28	0.00857100898049315\\
57.29	0.00857133120158797\\
57.3	0.00857165356404352\\
57.31	0.0085719760679408\\
57.32	0.00857229871336085\\
57.33	0.00857262150038481\\
57.34	0.00857294442909385\\
57.35	0.00857326749956922\\
57.36	0.00857359071189223\\
57.37	0.00857391406614426\\
57.38	0.00857423756240675\\
57.39	0.00857456120076122\\
57.4	0.00857488498128922\\
57.41	0.00857520890407239\\
57.42	0.00857553296919244\\
57.43	0.00857585717673112\\
57.44	0.00857618152677028\\
57.45	0.00857650601939179\\
57.46	0.00857683065467764\\
57.47	0.00857715543270982\\
57.48	0.00857748035357045\\
57.49	0.00857780541734166\\
57.5	0.0085781306241057\\
57.51	0.00857845597394482\\
57.52	0.00857878146694141\\
57.53	0.00857910710317786\\
57.54	0.00857943288273666\\
57.55	0.00857975880570036\\
57.56	0.00858008487215157\\
57.57	0.00858041108217297\\
57.58	0.00858073743584731\\
57.59	0.00858106393325741\\
57.6	0.00858139057448612\\
57.61	0.00858171735961641\\
57.62	0.00858204428873128\\
57.63	0.00858237136191381\\
57.64	0.00858269857924714\\
57.65	0.00858302594081447\\
57.66	0.00858335344669908\\
57.67	0.00858368109698433\\
57.68	0.0085840088917536\\
57.69	0.00858433683109039\\
57.7	0.00858466491507822\\
57.71	0.00858499314380072\\
57.72	0.00858532151734155\\
57.73	0.00858565003578446\\
57.74	0.00858597869921326\\
57.75	0.00858630750771182\\
57.76	0.00858663646136409\\
57.77	0.00858696556025408\\
57.78	0.00858729480446587\\
57.79	0.0085876241940836\\
57.8	0.00858795372919149\\
57.81	0.00858828340987382\\
57.82	0.00858861323621495\\
57.83	0.00858894320829927\\
57.84	0.00858927332621128\\
57.85	0.00858960359003554\\
57.86	0.00858993399985666\\
57.87	0.00859026455575933\\
57.88	0.0085905952578283\\
57.89	0.0085909261061484\\
57.9	0.00859125710080453\\
57.91	0.00859158824188164\\
57.92	0.00859191952946477\\
57.93	0.00859225096363901\\
57.94	0.00859258254448953\\
57.95	0.00859291427210156\\
57.96	0.0085932461465604\\
57.97	0.00859357816795144\\
57.98	0.00859391033636011\\
57.99	0.00859424265187193\\
58	0.00859457511457246\\
58.01	0.00859490772454737\\
58.02	0.00859524048188236\\
58.03	0.00859557338666323\\
58.04	0.00859590643897583\\
58.05	0.00859623963890609\\
58.06	0.00859657298654\\
58.07	0.00859690648196363\\
58.08	0.00859724012526311\\
58.09	0.00859757391652465\\
58.1	0.00859790785583452\\
58.11	0.00859824194327906\\
58.12	0.00859857617894468\\
58.13	0.00859891056291789\\
58.14	0.00859924509528521\\
58.15	0.00859957977613329\\
58.16	0.0085999146055488\\
58.17	0.00860024958361853\\
58.18	0.0086005847104293\\
58.19	0.00860091998606802\\
58.2	0.00860125541062166\\
58.21	0.00860159098417728\\
58.22	0.00860192670682199\\
58.23	0.00860226257864297\\
58.24	0.0086025985997275\\
58.25	0.0086029347701629\\
58.26	0.00860327109003656\\
58.27	0.00860360755943596\\
58.28	0.00860394417844865\\
58.29	0.00860428094716224\\
58.3	0.00860461786566442\\
58.31	0.00860495493404295\\
58.32	0.00860529215238564\\
58.33	0.00860562952078042\\
58.34	0.00860596703931524\\
58.35	0.00860630470807816\\
58.36	0.00860664252715728\\
58.37	0.00860698049664081\\
58.38	0.00860731861661699\\
58.39	0.00860765688717416\\
58.4	0.00860799530840073\\
58.41	0.00860833388038518\\
58.42	0.00860867260321604\\
58.43	0.00860901147698195\\
58.44	0.00860935050177159\\
58.45	0.00860968967767374\\
58.46	0.00861002900477724\\
58.47	0.00861036848317099\\
58.48	0.00861070811294399\\
58.49	0.00861104789418528\\
58.5	0.00861138782698401\\
58.51	0.00861172791142937\\
58.52	0.00861206814761065\\
58.53	0.00861240853561718\\
58.54	0.00861274907553842\\
58.55	0.00861308976746383\\
58.56	0.008613430611483\\
58.57	0.00861377160768558\\
58.58	0.00861411275616127\\
58.59	0.00861445405699988\\
58.6	0.00861479551029127\\
58.61	0.00861513711612537\\
58.62	0.00861547887459221\\
58.63	0.00861582078578186\\
58.64	0.0086161628497845\\
58.65	0.00861650506669036\\
58.66	0.00861684743658975\\
58.67	0.00861718995957305\\
58.68	0.00861753263573073\\
58.69	0.00861787546515332\\
58.7	0.00861821844793143\\
58.71	0.00861856158415573\\
58.72	0.00861890487391701\\
58.73	0.00861924831730608\\
58.74	0.00861959191441385\\
58.75	0.00861993566533131\\
58.76	0.00862027957014953\\
58.77	0.00862062362895962\\
58.78	0.00862096784185282\\
58.79	0.00862131220892038\\
58.8	0.00862165673025369\\
58.81	0.00862200140594417\\
58.82	0.00862234623608335\\
58.83	0.0086226912207628\\
58.84	0.0086230363600742\\
58.85	0.00862338165410927\\
58.86	0.00862372710295985\\
58.87	0.00862407270671781\\
58.88	0.00862441846547513\\
58.89	0.00862476437932386\\
58.9	0.00862511044835612\\
58.91	0.0086254566726641\\
58.92	0.00862580305234008\\
58.93	0.00862614958747642\\
58.94	0.00862649627816554\\
58.95	0.00862684312449994\\
58.96	0.0086271901265722\\
58.97	0.008627537284475\\
58.98	0.00862788459830106\\
58.99	0.00862823206814319\\
59	0.0086285796940943\\
59.01	0.00862892747624734\\
59.02	0.00862927541469537\\
59.03	0.0086296235095315\\
59.04	0.00862997176084893\\
59.05	0.00863032016874096\\
59.06	0.00863066873330093\\
59.07	0.00863101745462229\\
59.08	0.00863136633279853\\
59.09	0.00863171536792326\\
59.1	0.00863206456009014\\
59.11	0.00863241390939293\\
59.12	0.00863276341592544\\
59.13	0.00863311307978159\\
59.14	0.00863346290105535\\
59.15	0.0086338128798408\\
59.16	0.00863416301623206\\
59.17	0.00863451331032336\\
59.18	0.00863486376220901\\
59.19	0.00863521437198337\\
59.2	0.00863556513974091\\
59.21	0.00863591606557615\\
59.22	0.00863626714958373\\
59.23	0.00863661839185832\\
59.24	0.00863696979249472\\
59.25	0.00863732135158776\\
59.26	0.00863767306923239\\
59.27	0.00863802494552363\\
59.28	0.00863837698055655\\
59.29	0.00863872917442635\\
59.3	0.00863908152722827\\
59.31	0.00863943403905764\\
59.32	0.00863978671000989\\
59.33	0.00864013954018051\\
59.34	0.00864049252966506\\
59.35	0.00864084567855922\\
59.36	0.00864119898695872\\
59.37	0.00864155245495936\\
59.38	0.00864190608265706\\
59.39	0.00864225987014779\\
59.4	0.00864261381752762\\
59.41	0.00864296792489268\\
59.42	0.0086433221923392\\
59.43	0.00864367661996348\\
59.44	0.0086440312078619\\
59.45	0.00864438595613095\\
59.46	0.00864474086486716\\
59.47	0.00864509593416716\\
59.48	0.00864545116412767\\
59.49	0.00864580655484549\\
59.5	0.00864616210641748\\
59.51	0.00864651781894062\\
59.52	0.00864687369251193\\
59.53	0.00864722972722855\\
59.54	0.00864758592318768\\
59.55	0.0086479422804866\\
59.56	0.0086482987992227\\
59.57	0.00864865547949342\\
59.58	0.0086490123213963\\
59.59	0.00864936932502896\\
59.6	0.0086497264904891\\
59.61	0.00865008381787451\\
59.62	0.00865044130728305\\
59.63	0.00865079895881269\\
59.64	0.00865115677256145\\
59.65	0.00865151474862745\\
59.66	0.00865187288710891\\
59.67	0.0086522311881041\\
59.68	0.00865258965171139\\
59.69	0.00865294827802924\\
59.7	0.00865330706715619\\
59.71	0.00865366601919086\\
59.72	0.00865402513423196\\
59.73	0.00865438441237828\\
59.74	0.00865474385372869\\
59.75	0.00865510345838216\\
59.76	0.00865546322643773\\
59.77	0.00865582315799453\\
59.78	0.00865618325315177\\
59.79	0.00865654351200876\\
59.8	0.00865690393466488\\
59.81	0.00865726452121959\\
59.82	0.00865762527177246\\
59.83	0.00865798618642313\\
59.84	0.00865834726527131\\
59.85	0.00865870850841683\\
59.86	0.00865906991595958\\
59.87	0.00865943148799954\\
59.88	0.00865979322463678\\
59.89	0.00866015512597146\\
59.9	0.00866051719210381\\
59.91	0.00866087942313417\\
59.92	0.00866124181916294\\
59.93	0.00866160438029064\\
59.94	0.00866196710661784\\
59.95	0.00866232999824522\\
59.96	0.00866269305527353\\
59.97	0.00866305627780363\\
59.98	0.00866341966593645\\
59.99	0.00866378321977301\\
60	0.00866414693941442\\
60.01	0.00866451082496186\\
60.02	0.00866487487651664\\
60.03	0.00866523909418011\\
60.04	0.00866560347805373\\
60.05	0.00866596802823905\\
60.06	0.00866633274483771\\
60.07	0.00866669762795141\\
60.08	0.00866706267768198\\
60.09	0.00866742789413131\\
60.1	0.00866779327740139\\
60.11	0.00866815882759429\\
60.12	0.00866852454481217\\
60.13	0.00866889042915728\\
60.14	0.00866925648073197\\
60.15	0.00866962269963866\\
60.16	0.00866998908597987\\
60.17	0.00867035563985821\\
60.18	0.00867072236137637\\
60.19	0.00867108925063713\\
60.2	0.00867145630774338\\
60.21	0.00867182353279807\\
60.22	0.00867219092590426\\
60.23	0.0086725584871651\\
60.24	0.00867292621668381\\
60.25	0.00867329411456371\\
60.26	0.00867366218090823\\
60.27	0.00867403041582086\\
60.28	0.0086743988194052\\
60.29	0.00867476739176492\\
60.3	0.00867513613300381\\
60.31	0.00867550504322573\\
60.32	0.00867587412253464\\
60.33	0.00867624337103457\\
60.34	0.00867661278882968\\
60.35	0.00867698237602418\\
60.36	0.00867735213272239\\
60.37	0.00867772205902874\\
60.38	0.00867809215504771\\
60.39	0.0086784624208839\\
60.4	0.008678832856642\\
60.41	0.00867920346242679\\
60.42	0.00867957423834313\\
60.43	0.00867994518449598\\
60.44	0.0086803163009904\\
60.45	0.00868068758793153\\
60.46	0.00868105904542461\\
60.47	0.00868143067357497\\
60.48	0.00868180247248804\\
60.49	0.00868217444226932\\
60.5	0.00868254658302442\\
60.51	0.00868291889485906\\
60.52	0.00868329137787901\\
60.53	0.00868366403219016\\
60.54	0.00868403685789851\\
60.55	0.00868440985511012\\
60.56	0.00868478302393116\\
60.57	0.00868515636446788\\
60.58	0.00868552987682665\\
60.59	0.00868590356111392\\
60.6	0.00868627741743621\\
60.61	0.00868665144590018\\
60.62	0.00868702564661255\\
60.63	0.00868740001968015\\
60.64	0.0086877745652099\\
60.65	0.00868814928330881\\
60.66	0.00868852417408399\\
60.67	0.00868889923764265\\
60.68	0.00868927447409207\\
60.69	0.00868964988353966\\
60.7	0.00869002546609292\\
60.71	0.00869040122185941\\
60.72	0.00869077715094682\\
60.73	0.00869115325346293\\
60.74	0.0086915295295156\\
60.75	0.00869190597921281\\
60.76	0.00869228260266261\\
60.77	0.00869265939997317\\
60.78	0.00869303637125275\\
60.79	0.00869341351660968\\
60.8	0.00869379083615243\\
60.81	0.00869416832998953\\
60.82	0.00869454599822963\\
60.83	0.00869492384098145\\
60.84	0.00869530185835386\\
60.85	0.00869568005045576\\
60.86	0.00869605841739619\\
60.87	0.00869643695928428\\
60.88	0.00869681567622926\\
60.89	0.00869719456834043\\
60.9	0.00869757363572723\\
60.91	0.00869795287849916\\
60.92	0.00869833229676586\\
60.93	0.00869871189063701\\
60.94	0.00869909166022245\\
60.95	0.00869947160563208\\
60.96	0.00869985172697591\\
60.97	0.00870023202436404\\
60.98	0.00870061249790668\\
60.99	0.00870099314771414\\
61	0.00870137397389682\\
61.01	0.00870175497656521\\
61.02	0.00870213615582993\\
61.03	0.00870251751180168\\
61.04	0.00870289904459125\\
61.05	0.00870328075430955\\
61.06	0.00870366264106757\\
61.07	0.00870404470497643\\
61.08	0.0087044269461473\\
61.09	0.00870480936469151\\
61.1	0.00870519196072044\\
61.11	0.00870557473434561\\
61.12	0.0087059576856786\\
61.13	0.00870634081483113\\
61.14	0.008706724121915\\
61.15	0.00870710760704211\\
61.16	0.00870749127032447\\
61.17	0.00870787511187419\\
61.18	0.00870825913180347\\
61.19	0.00870864333022462\\
61.2	0.00870902770725005\\
61.21	0.00870941226299229\\
61.22	0.00870979699756393\\
61.23	0.00871018191107771\\
61.24	0.00871056700364643\\
61.25	0.00871095227538303\\
61.26	0.00871133772640052\\
61.27	0.00871172335681204\\
61.28	0.0087121091667308\\
61.29	0.00871249515627016\\
61.3	0.00871288132554353\\
61.31	0.00871326767466447\\
61.32	0.0087136542037466\\
61.33	0.00871404091290369\\
61.34	0.00871442780224957\\
61.35	0.00871481487189819\\
61.36	0.00871520212196363\\
61.37	0.00871558955256004\\
61.38	0.00871597716380168\\
61.39	0.00871636495580292\\
61.4	0.00871675292867823\\
61.41	0.00871714108254219\\
61.42	0.0087175294175095\\
61.43	0.00871791793369493\\
61.44	0.00871830663121337\\
61.45	0.00871869551017984\\
61.46	0.00871908457070942\\
61.47	0.00871947381291733\\
61.48	0.00871986323691888\\
61.49	0.00872025284282949\\
61.5	0.0087206426307647\\
61.51	0.00872103260084012\\
61.52	0.0087214227531715\\
61.53	0.00872181308787469\\
61.54	0.00872220360506563\\
61.55	0.00872259430486038\\
61.56	0.00872298518737511\\
61.57	0.00872337625272608\\
61.58	0.00872376750102968\\
61.59	0.00872415893240239\\
61.6	0.0087245505469608\\
61.61	0.00872494234482161\\
61.62	0.00872533432610164\\
61.63	0.00872572649091779\\
61.64	0.00872611883938709\\
61.65	0.00872651137162667\\
61.66	0.00872690408775377\\
61.67	0.00872729698788574\\
61.68	0.00872769007214004\\
61.69	0.00872808334063422\\
61.7	0.00872847679348597\\
61.71	0.00872887043081307\\
61.72	0.0087292642527334\\
61.73	0.00872965825936498\\
61.74	0.0087300524508259\\
61.75	0.0087304468272344\\
61.76	0.0087308413887088\\
61.77	0.00873123613536755\\
61.78	0.00873163106732918\\
61.79	0.00873202618471236\\
61.8	0.00873242148763586\\
61.81	0.00873281697621856\\
61.82	0.00873321265057946\\
61.83	0.00873360851083765\\
61.84	0.00873400455711235\\
61.85	0.00873440078952288\\
61.86	0.00873479720818868\\
61.87	0.00873519381322929\\
61.88	0.00873559060476437\\
61.89	0.0087359875829137\\
61.9	0.00873638474779715\\
61.91	0.00873678209953471\\
61.92	0.0087371796382465\\
61.93	0.00873757736405273\\
61.94	0.00873797527707373\\
61.95	0.00873837337742994\\
61.96	0.00873877166524193\\
61.97	0.00873917014063035\\
61.98	0.00873956880371599\\
61.99	0.00873996765461976\\
62	0.00874036669346264\\
62.01	0.00874076592036578\\
62.02	0.00874116533545039\\
62.03	0.00874156493883784\\
62.04	0.00874196473064959\\
62.05	0.00874236471100721\\
62.06	0.00874276488003239\\
62.07	0.00874316523784694\\
62.08	0.00874356578457279\\
62.09	0.00874396652033196\\
62.1	0.00874436744524662\\
62.11	0.00874476855943902\\
62.12	0.00874516986303155\\
62.13	0.00874557135614671\\
62.14	0.0087459730389071\\
62.15	0.00874637491143545\\
62.16	0.00874677697385461\\
62.17	0.00874717922628754\\
62.18	0.00874758166885732\\
62.19	0.00874798430168713\\
62.2	0.00874838712490028\\
62.21	0.00874879013862021\\
62.22	0.00874919334297045\\
62.23	0.00874959673807467\\
62.24	0.00875000032405663\\
62.25	0.00875040410104024\\
62.26	0.0087508080691495\\
62.27	0.00875121222850855\\
62.28	0.00875161657924163\\
62.29	0.0087520211214731\\
62.3	0.00875242585532746\\
62.31	0.00875283078092929\\
62.32	0.00875323589840332\\
62.33	0.00875364120787439\\
62.34	0.00875404670946746\\
62.35	0.0087544524033076\\
62.36	0.00875485828951999\\
62.37	0.00875526436822997\\
62.38	0.00875567063956297\\
62.39	0.00875607710364453\\
62.4	0.00875648376060032\\
62.41	0.00875689061055615\\
62.42	0.00875729765363792\\
62.43	0.00875770488997166\\
62.44	0.00875811231968353\\
62.45	0.00875851994289981\\
62.46	0.00875892775974688\\
62.47	0.00875933577035126\\
62.48	0.00875974397483959\\
62.49	0.00876015237333862\\
62.5	0.00876056096597523\\
62.51	0.00876096975287642\\
62.52	0.00876137873416932\\
62.53	0.00876178790998115\\
62.54	0.0087621972804393\\
62.55	0.00876260684567125\\
62.56	0.00876301660580459\\
62.57	0.00876342656096707\\
62.58	0.00876383671128654\\
62.59	0.00876424705689097\\
62.6	0.00876465759790847\\
62.61	0.00876506833446725\\
62.62	0.00876547926669566\\
62.63	0.00876589039472218\\
62.64	0.00876630171867538\\
62.65	0.00876671323868399\\
62.66	0.00876712495487684\\
62.67	0.0087675368673829\\
62.68	0.00876794897633125\\
62.69	0.00876836128185111\\
62.7	0.00876877378407181\\
62.71	0.00876918648312281\\
62.72	0.00876959937913369\\
62.73	0.00877001247223417\\
62.74	0.00877042576255407\\
62.75	0.00877083925022336\\
62.76	0.00877125293537212\\
62.77	0.00877166681813055\\
62.78	0.008772080898629\\
62.79	0.00877249517699792\\
62.8	0.00877290965336791\\
62.81	0.00877332432786966\\
62.82	0.00877373920063402\\
62.83	0.00877415427179195\\
62.84	0.00877456954147455\\
62.85	0.00877498500981303\\
62.86	0.00877540067693873\\
62.87	0.00877581654298313\\
62.88	0.00877623260807781\\
62.89	0.00877664887235452\\
62.9	0.00877706533594508\\
62.91	0.0087774819989815\\
62.92	0.00877789886159586\\
62.93	0.0087783159239204\\
62.94	0.00877873318608749\\
62.95	0.00877915064822961\\
62.96	0.00877956831047937\\
62.97	0.00877998617296953\\
62.98	0.00878040423583295\\
62.99	0.00878082249920264\\
63	0.00878124096321172\\
63.01	0.00878165962799344\\
63.02	0.00878207849368121\\
63.03	0.00878249756040854\\
63.04	0.00878291682830905\\
63.05	0.00878333629751653\\
63.06	0.00878375596816489\\
63.07	0.00878417584038814\\
63.08	0.00878459591432045\\
63.09	0.00878501619009611\\
63.1	0.00878543666784954\\
63.11	0.00878585734771529\\
63.12	0.00878627822982803\\
63.13	0.00878669931432257\\
63.14	0.00878712060133385\\
63.15	0.00878754209099695\\
63.16	0.00878796378344704\\
63.17	0.00878838567881947\\
63.18	0.00878880777724969\\
63.19	0.00878923007887329\\
63.2	0.00878965258382599\\
63.21	0.00879007529224364\\
63.22	0.00879049820426222\\
63.23	0.00879092132001784\\
63.24	0.00879134463964674\\
63.25	0.00879176816328529\\
63.26	0.00879219189107\\
63.27	0.00879261582313749\\
63.28	0.00879303995962455\\
63.29	0.00879346430066806\\
63.3	0.00879388884640504\\
63.31	0.00879431359697266\\
63.32	0.00879473855250822\\
63.33	0.00879516371314912\\
63.34	0.00879558907903292\\
63.35	0.00879601465029732\\
63.36	0.00879644042708011\\
63.37	0.00879686640951925\\
63.38	0.00879729259775282\\
63.39	0.00879771899191902\\
63.4	0.00879814559215621\\
63.41	0.00879857239860285\\
63.42	0.00879899941139755\\
63.43	0.00879942663067904\\
63.44	0.0087998540565862\\
63.45	0.00880028168925802\\
63.46	0.00880070952883364\\
63.47	0.00880113757545232\\
63.48	0.00880156582925345\\
63.49	0.00880199429037657\\
63.5	0.00880242295896134\\
63.51	0.00880285183514753\\
63.52	0.00880328091907509\\
63.53	0.00880371021088405\\
63.54	0.00880413971071462\\
63.55	0.0088045694187071\\
63.56	0.00880499933500195\\
63.57	0.00880542945973975\\
63.58	0.0088058597930612\\
63.59	0.00880629033510716\\
63.6	0.00880672108601861\\
63.61	0.00880715204593664\\
63.62	0.0088075832150025\\
63.63	0.00880801459335756\\
63.64	0.00880844618114332\\
63.65	0.00880887797850142\\
63.66	0.00880930998557362\\
63.67	0.0088097422025018\\
63.68	0.00881017462942802\\
63.69	0.0088106072664944\\
63.7	0.00881104011384326\\
63.71	0.008811473171617\\
63.72	0.00881190643995818\\
63.73	0.00881233991900948\\
63.74	0.00881277360891371\\
63.75	0.0088132075098138\\
63.76	0.00881364162185284\\
63.77	0.00881407594517402\\
63.78	0.00881451047992067\\
63.79	0.00881494522623626\\
63.8	0.00881538018426438\\
63.81	0.00881581535414874\\
63.82	0.0088162507360332\\
63.83	0.00881668633006174\\
63.84	0.00881712213637847\\
63.85	0.00881755815512762\\
63.86	0.00881799438645355\\
63.87	0.00881843083050077\\
63.88	0.0088188674874139\\
63.89	0.00881930435733768\\
63.9	0.008819741440417\\
63.91	0.00882017873679686\\
63.92	0.0088206162466224\\
63.93	0.00882105397003887\\
63.94	0.00882149190719167\\
63.95	0.00882193005822631\\
63.96	0.00882236842328842\\
63.97	0.00882280700252379\\
63.98	0.0088232457960783\\
63.99	0.00882368480409798\\
64	0.00882412402672896\\
64.01	0.00882456346411751\\
64.02	0.00882500311641003\\
64.03	0.00882544298375305\\
64.04	0.0088258830662932\\
64.05	0.00882632336417725\\
64.06	0.00882676387755209\\
64.07	0.00882720460656474\\
64.08	0.00882764555136234\\
64.09	0.00882808671209213\\
64.1	0.00882852808890152\\
64.11	0.00882896968193799\\
64.12	0.00882941149134917\\
64.13	0.00882985351728282\\
64.14	0.00883029575988679\\
64.15	0.00883073821930908\\
64.16	0.0088311808956978\\
64.17	0.00883162378920116\\
64.18	0.00883206689996752\\
64.19	0.00883251022814533\\
64.2	0.0088329537738832\\
64.21	0.0088333975373298\\
64.22	0.00883384151863397\\
64.23	0.00883428571794463\\
64.24	0.00883473013541084\\
64.25	0.00883517477118176\\
64.26	0.00883561962540668\\
64.27	0.00883606469823499\\
64.28	0.0088365099898162\\
64.29	0.00883695550029995\\
64.3	0.00883740122983596\\
64.31	0.00883784717857409\\
64.32	0.0088382933466643\\
64.33	0.00883873973425667\\
64.34	0.00883918634150139\\
64.35	0.00883963316854875\\
64.36	0.00884008021554915\\
64.37	0.00884052748265312\\
64.38	0.00884097497001128\\
64.39	0.00884142267777437\\
64.4	0.00884187060609321\\
64.41	0.00884231875511876\\
64.42	0.00884276712500209\\
64.43	0.00884321571589433\\
64.44	0.00884366452794677\\
64.45	0.00884411356131076\\
64.46	0.00884456281613779\\
64.47	0.00884501229257943\\
64.48	0.00884546199078735\\
64.49	0.00884591191091333\\
64.5	0.00884636205310926\\
64.51	0.00884681241752712\\
64.52	0.00884726300431898\\
64.53	0.00884771381363704\\
64.54	0.00884816484563355\\
64.55	0.00884861610046091\\
64.56	0.00884906757827158\\
64.57	0.00884951927921813\\
64.58	0.00884997120345321\\
64.59	0.0088504233511296\\
64.6	0.00885087572240014\\
64.61	0.00885132831741777\\
64.62	0.00885178113633552\\
64.63	0.00885223417930654\\
64.64	0.00885268744648402\\
64.65	0.00885314093802129\\
64.66	0.00885359465407173\\
64.67	0.00885404859478883\\
64.68	0.00885450276032617\\
64.69	0.00885495715083739\\
64.7	0.00885541176647625\\
64.71	0.00885586660739658\\
64.72	0.00885632167375228\\
64.73	0.00885677696569735\\
64.74	0.00885723248338588\\
64.75	0.00885768822697203\\
64.76	0.00885814419661002\\
64.77	0.0088586003924542\\
64.78	0.00885905681465895\\
64.79	0.00885951346337875\\
64.8	0.00885997033876818\\
64.81	0.00886042744098185\\
64.82	0.00886088477017448\\
64.83	0.00886134232650086\\
64.84	0.00886180011011583\\
64.85	0.00886225812117435\\
64.86	0.00886271635983141\\
64.87	0.0088631748262421\\
64.88	0.00886363352056155\\
64.89	0.008864092442945\\
64.9	0.00886455159354773\\
64.91	0.00886501097252511\\
64.92	0.00886547058003256\\
64.93	0.00886593041622557\\
64.94	0.00886639048125972\\
64.95	0.00886685077529063\\
64.96	0.00886731129847399\\
64.97	0.00886777205096557\\
64.98	0.00886823303292118\\
64.99	0.00886869424449673\\
65	0.00886915568584815\\
65.01	0.00886961735713146\\
65.02	0.00887007925850274\\
65.03	0.00887054139011812\\
65.04	0.00887100375213381\\
65.05	0.00887146634470606\\
65.06	0.00887192916799119\\
65.07	0.00887239222214557\\
65.08	0.00887285550732565\\
65.09	0.00887331902368793\\
65.1	0.00887378277138895\\
65.11	0.00887424675058533\\
65.12	0.00887471096143374\\
65.13	0.00887517540409092\\
65.14	0.00887564007871364\\
65.15	0.00887610498545875\\
65.16	0.00887657012448315\\
65.17	0.0088770354959438\\
65.18	0.00887750109999772\\
65.19	0.00887796693680198\\
65.2	0.0088784330065137\\
65.21	0.00887889930929009\\
65.22	0.00887936584528838\\
65.23	0.00887983261466587\\
65.24	0.00888029961757994\\
65.25	0.00888076685418801\\
65.26	0.00888123432464754\\
65.27	0.00888170202911609\\
65.28	0.00888216996775125\\
65.29	0.00888263814071069\\
65.3	0.00888310654815214\\
65.31	0.00888357519023337\\
65.32	0.00888404406711225\\
65.33	0.00888451317894668\\
65.34	0.00888498252589465\\
65.35	0.00888545210811421\\
65.36	0.00888592192576347\\
65.37	0.00888639197900061\\
65.38	0.00888686226798391\\
65.39	0.00888733279287168\\
65.4	0.00888780355382234\\
65.41	0.00888827455099436\\
65.42	0.00888874578454631\\
65.43	0.00888921725463681\\
65.44	0.00888968896142461\\
65.45	0.0088901609050685\\
65.46	0.00889063308572738\\
65.47	0.00889110550356024\\
65.48	0.00889157815872615\\
65.49	0.00889205105138429\\
65.5	0.00889252418169392\\
65.51	0.00889299754981442\\
65.52	0.00889347115590527\\
65.53	0.00889394500012606\\
65.54	0.00889441908263648\\
65.55	0.00889489340359635\\
65.56	0.00889536796316561\\
65.57	0.00889584276150432\\
65.58	0.00889631779877267\\
65.59	0.00889679307513098\\
65.6	0.0088972685907397\\
65.61	0.00889774434575945\\
65.62	0.00889822034035099\\
65.63	0.0088986965746752\\
65.64	0.00889917304889317\\
65.65	0.00889964976316612\\
65.66	0.00890012671765546\\
65.67	0.00890060391252278\\
65.68	0.00890108134792983\\
65.69	0.00890155902403858\\
65.7	0.00890203694101119\\
65.71	0.00890251509901001\\
65.72	0.00890299349819764\\
65.73	0.00890347213873687\\
65.74	0.00890395102079073\\
65.75	0.0089044301445225\\
65.76	0.00890490951009569\\
65.77	0.00890538911767408\\
65.78	0.00890586896742171\\
65.79	0.00890634905950292\\
65.8	0.00890682939408229\\
65.81	0.00890730997132473\\
65.82	0.00890779079139547\\
65.83	0.00890827185446004\\
65.84	0.00890875316068429\\
65.85	0.00890923471023445\\
65.86	0.00890971650327707\\
65.87	0.00891019853997909\\
65.88	0.00891068082050783\\
65.89	0.008911163345031\\
65.9	0.00891164611371673\\
65.91	0.00891212912673357\\
65.92	0.0089126123842505\\
65.93	0.00891309588643696\\
65.94	0.00891357963346289\\
65.95	0.00891406362549867\\
65.96	0.00891454786271521\\
65.97	0.00891503234528395\\
65.98	0.00891551707337686\\
65.99	0.00891600204716647\\
66	0.00891648726682587\\
66.01	0.00891697273252878\\
66.02	0.00891745844444952\\
66.03	0.00891794440276305\\
66.04	0.00891843060764497\\
66.05	0.00891891705927159\\
66.06	0.00891940375781991\\
66.07	0.00891989070346765\\
66.08	0.00892037789639329\\
66.09	0.00892086533677609\\
66.1	0.00892135302479608\\
66.11	0.00892184096063414\\
66.12	0.008922329144472\\
66.13	0.00892281757649224\\
66.14	0.00892330625687839\\
66.15	0.00892379518581485\\
66.16	0.00892428436348705\\
66.17	0.00892477379008137\\
66.18	0.0089252634657852\\
66.19	0.00892575339078702\\
66.2	0.00892624356527637\\
66.21	0.00892673398944391\\
66.22	0.00892722466348144\\
66.23	0.00892771558758197\\
66.24	0.00892820676193971\\
66.25	0.00892869818675012\\
66.26	0.00892918986220996\\
66.27	0.0089296817885173\\
66.28	0.00893017396587161\\
66.29	0.00893066639447373\\
66.3	0.00893115907452595\\
66.31	0.00893165200623158\\
66.32	0.00893214518979471\\
66.33	0.00893263862542026\\
66.34	0.00893313231331395\\
66.35	0.00893362625368238\\
66.36	0.008934120446733\\
66.37	0.00893461489267416\\
66.38	0.00893510959171512\\
66.39	0.00893560454406605\\
66.4	0.00893609974993808\\
66.41	0.00893659520954331\\
66.42	0.00893709092309482\\
66.43	0.00893758689080671\\
66.44	0.00893808311289408\\
66.45	0.00893857958957309\\
66.46	0.00893907632106097\\
66.47	0.00893957330757603\\
66.48	0.00894007054933769\\
66.49	0.00894056804656647\\
66.5	0.00894106579948404\\
66.51	0.00894156380831325\\
66.52	0.00894206207327811\\
66.53	0.00894256059460381\\
66.54	0.00894305937251677\\
66.55	0.00894355840724464\\
66.56	0.0089440576990163\\
66.57	0.0089445572480619\\
66.58	0.00894505705461287\\
66.59	0.0089455571189019\\
66.6	0.00894605744116302\\
66.61	0.00894655802163154\\
66.62	0.00894705886054412\\
66.63	0.00894755995813874\\
66.64	0.00894806131465473\\
66.65	0.00894856293033279\\
66.66	0.00894906480541494\\
66.67	0.0089495669401446\\
66.68	0.00895006933476655\\
66.69	0.00895057198952693\\
66.7	0.00895107490467326\\
66.71	0.00895157808045445\\
66.72	0.00895208151712074\\
66.73	0.00895258521492377\\
66.74	0.00895308917411649\\
66.75	0.00895359339495325\\
66.76	0.00895409787768969\\
66.77	0.00895460262258279\\
66.78	0.00895510762989083\\
66.79	0.00895561289987335\\
66.8	0.00895611843279118\\
66.81	0.00895662422890635\\
66.82	0.00895713028848211\\
66.83	0.00895763661178286\\
66.84	0.00895814319907413\\
66.85	0.00895865005062256\\
66.86	0.00895915716669579\\
66.87	0.00895966454756248\\
66.88	0.0089601721934922\\
66.89	0.00896068010475539\\
66.9	0.00896118828162333\\
66.91	0.008961696724368\\
66.92	0.00896220543326205\\
66.93	0.0089627144085787\\
66.94	0.00896322365059164\\
66.95	0.00896373315957498\\
66.96	0.00896424293580309\\
66.97	0.00896475297955049\\
66.98	0.00896526329109181\\
66.99	0.00896577387070154\\
67	0.00896628471865401\\
67.01	0.00896679583522316\\
67.02	0.00896730722068244\\
67.03	0.00896781887530461\\
67.04	0.00896833079936159\\
67.05	0.00896884299312427\\
67.06	0.00896935545686231\\
67.07	0.00896986819084391\\
67.08	0.00897038119533564\\
67.09	0.00897089447060216\\
67.1	0.0089714080169085\\
67.11	0.00897192183452043\\
67.12	0.00897243592370455\\
67.13	0.00897295028472821\\
67.14	0.00897346491785957\\
67.15	0.0089739798233676\\
67.16	0.00897449500152208\\
67.17	0.00897501045259362\\
67.18	0.00897552617685365\\
67.19	0.00897604217457443\\
67.2	0.00897655844602911\\
67.21	0.00897707499149165\\
67.22	0.00897759181123691\\
67.23	0.00897810890554061\\
67.24	0.00897862627467937\\
67.25	0.00897914391893068\\
67.26	0.00897966183857297\\
67.27	0.00898018003388555\\
67.28	0.0089806985051487\\
67.29	0.00898121725264359\\
67.3	0.00898173627665236\\
67.31	0.00898225557745811\\
67.32	0.0089827751553449\\
67.33	0.00898329501059778\\
67.34	0.00898381514350276\\
67.35	0.0089843355543469\\
67.36	0.00898485624341824\\
67.37	0.00898537721100584\\
67.38	0.00898589845739983\\
67.39	0.00898641998289135\\
67.4	0.00898694178777263\\
67.41	0.00898746387233697\\
67.42	0.00898798623687874\\
67.43	0.00898850888169342\\
67.44	0.00898903180707762\\
67.45	0.00898955501332905\\
67.46	0.00899007850074658\\
67.47	0.00899060226963021\\
67.48	0.00899112632028114\\
67.49	0.00899165065300172\\
67.5	0.00899217526809553\\
67.51	0.00899270016586732\\
67.52	0.00899322534662312\\
67.53	0.00899375081067015\\
67.54	0.00899427655831692\\
67.55	0.0089948025898732\\
67.56	0.00899532890565005\\
67.57	0.00899585550595986\\
67.58	0.00899638239111629\\
67.59	0.0089969095614344\\
67.6	0.00899743701723056\\
67.61	0.00899796475882253\\
67.62	0.00899849278652947\\
67.63	0.00899902110067195\\
67.64	0.00899954970157194\\
67.65	0.0090000785895529\\
67.66	0.00900060776493972\\
67.67	0.00900113722805879\\
67.68	0.009001666979238\\
67.69	0.00900219701880677\\
67.7	0.00900272734709606\\
67.71	0.0090032579644384\\
67.72	0.00900378887116791\\
67.73	0.00900432006762029\\
67.74	0.00900485155413292\\
67.75	0.00900538333104478\\
67.76	0.00900591539869656\\
67.77	0.00900644775743062\\
67.78	0.00900698040759106\\
67.79	0.00900751334952372\\
67.8	0.00900804658357619\\
67.81	0.00900858011009787\\
67.82	0.00900911392943999\\
67.83	0.00900964804195557\\
67.84	0.00901018244799955\\
67.85	0.00901071714792875\\
67.86	0.00901125214210189\\
67.87	0.00901178743087965\\
67.88	0.00901232301462469\\
67.89	0.00901285889370167\\
67.9	0.00901339506847726\\
67.91	0.00901393153932021\\
67.92	0.00901446830660137\\
67.93	0.00901500537069368\\
67.94	0.00901554273197224\\
67.95	0.00901608039081433\\
67.96	0.00901661834759944\\
67.97	0.0090171566027093\\
67.98	0.00901769515652791\\
67.99	0.00901823400944161\\
68	0.00901877316183903\\
68.01	0.00901931261411121\\
68.02	0.00901985236665159\\
68.03	0.00902039241985606\\
68.04	0.00902093277412298\\
68.05	0.00902147342985323\\
68.06	0.00902201438745026\\
68.07	0.00902255564732008\\
68.08	0.00902309720987137\\
68.09	0.00902363907551544\\
68.1	0.00902418124466633\\
68.11	0.00902472371774083\\
68.12	0.00902526649515851\\
68.13	0.00902580957734178\\
68.14	0.00902635296471591\\
68.15	0.00902689665770909\\
68.16	0.00902744065675249\\
68.17	0.00902798496228026\\
68.18	0.0090285295747296\\
68.19	0.00902907449454082\\
68.2	0.00902961972215736\\
68.21	0.00903016525802586\\
68.22	0.00903071110259619\\
68.23	0.0090312572563215\\
68.24	0.00903180371965829\\
68.25	0.00903235049306645\\
68.26	0.00903289757700929\\
68.27	0.00903344497195362\\
68.28	0.0090339926783698\\
68.29	0.00903454069673177\\
68.3	0.00903508902751716\\
68.31	0.00903563767120726\\
68.32	0.00903618662828716\\
68.33	0.00903673589924578\\
68.34	0.00903728548457588\\
68.35	0.00903783538477421\\
68.36	0.00903838560034148\\
68.37	0.0090389361317825\\
68.38	0.00903948697960618\\
68.39	0.00904003814432564\\
68.4	0.00904058962645824\\
68.41	0.00904114142652567\\
68.42	0.00904169354505401\\
68.43	0.00904224598257376\\
68.44	0.00904279873962\\
68.45	0.00904335181673236\\
68.46	0.00904390521445516\\
68.47	0.00904445893333743\\
68.48	0.00904501297393304\\
68.49	0.00904556733680074\\
68.5	0.00904612202250421\\
68.51	0.00904667703161221\\
68.52	0.00904723236469859\\
68.53	0.00904778802234241\\
68.54	0.00904834400512797\\
68.55	0.00904890031364498\\
68.56	0.00904945694848855\\
68.57	0.00905001391025934\\
68.58	0.00905057119956361\\
68.59	0.00905112881701331\\
68.6	0.00905168676322619\\
68.61	0.00905224503882588\\
68.62	0.00905280364444194\\
68.63	0.00905336258071004\\
68.64	0.00905392184827199\\
68.65	0.00905448144777581\\
68.66	0.00905504137987592\\
68.67	0.00905560164523314\\
68.68	0.00905616224451487\\
68.69	0.00905672317839513\\
68.7	0.00905728444755468\\
68.71	0.00905784605268116\\
68.72	0.00905840799446915\\
68.73	0.00905897027362028\\
68.74	0.00905953289084339\\
68.75	0.00906009584685456\\
68.76	0.00906065914237731\\
68.77	0.00906122277814263\\
68.78	0.00906178675488914\\
68.79	0.00906235107336322\\
68.8	0.0090629157343191\\
68.81	0.00906348073851898\\
68.82	0.00906404608673316\\
68.83	0.00906461177974017\\
68.84	0.00906517781832689\\
68.85	0.00906574420328868\\
68.86	0.00906631093542947\\
68.87	0.00906687801556198\\
68.88	0.00906744544450775\\
68.89	0.00906801322309733\\
68.9	0.0090685813521704\\
68.91	0.00906914983257594\\
68.92	0.00906971866517229\\
68.93	0.00907028785082739\\
68.94	0.00907085739041883\\
68.95	0.00907142728483407\\
68.96	0.00907199753497054\\
68.97	0.00907256814173578\\
68.98	0.00907313910604763\\
68.99	0.00907371042883436\\
69	0.00907428211103483\\
69.01	0.0090748541535986\\
69.02	0.00907542655748615\\
69.03	0.00907599932366902\\
69.04	0.00907657245312994\\
69.05	0.00907714594686303\\
69.06	0.00907771980587393\\
69.07	0.00907829403117997\\
69.08	0.00907886862381038\\
69.09	0.00907944358480641\\
69.1	0.0090800189152215\\
69.11	0.00908059461612147\\
69.12	0.0090811706885847\\
69.13	0.00908174713370229\\
69.14	0.00908232395257824\\
69.15	0.00908290114632962\\
69.16	0.00908347871608675\\
69.17	0.00908405666299341\\
69.18	0.00908463498820698\\
69.19	0.00908521369289865\\
69.2	0.00908579277825359\\
69.21	0.00908637224547113\\
69.22	0.009086952095765\\
69.23	0.00908753233036344\\
69.24	0.00908811295050943\\
69.25	0.00908869395746091\\
69.26	0.00908927535249089\\
69.27	0.00908985713688774\\
69.28	0.00909043931195529\\
69.29	0.0090910218790131\\
69.3	0.00909160483939662\\
69.31	0.00909218819445736\\
69.32	0.00909277194556315\\
69.33	0.00909335609409827\\
69.34	0.0090939406414637\\
69.35	0.00909452558907728\\
69.36	0.0090951109383739\\
69.37	0.00909569669080576\\
69.38	0.00909628284784248\\
69.39	0.00909686941097135\\
69.4	0.00909745638169752\\
69.41	0.00909804376154418\\
69.42	0.00909863155205276\\
69.43	0.00909921975478311\\
69.44	0.00909980837131374\\
69.45	0.00910039740324194\\
69.46	0.00910098685218401\\
69.47	0.00910157671977547\\
69.48	0.00910216700767118\\
69.49	0.00910275771754557\\
69.5	0.00910334885109282\\
69.51	0.00910394041002702\\
69.52	0.00910453239608234\\
69.53	0.00910512481101323\\
69.54	0.00910571765659457\\
69.55	0.00910631093462182\\
69.56	0.00910690464691121\\
69.57	0.00910749879529989\\
69.58	0.00910809338164604\\
69.59	0.00910868840782909\\
69.6	0.00910928387574979\\
69.61	0.00910987978733038\\
69.62	0.00911047614451472\\
69.63	0.0091110729492684\\
69.64	0.00911167020357884\\
69.65	0.00911226790945546\\
69.66	0.00911286606892968\\
69.67	0.0091134646840551\\
69.68	0.00911406375690755\\
69.69	0.00911466328958512\\
69.7	0.00911526328420829\\
69.71	0.00911586374291995\\
69.72	0.00911646466788542\\
69.73	0.00911706606129251\\
69.74	0.00911766792535153\\
69.75	0.00911827026229525\\
69.76	0.00911887307437895\\
69.77	0.00911947636388035\\
69.78	0.00912008013309958\\
69.79	0.00912068438435912\\
69.8	0.00912128912000372\\
69.81	0.00912189434240032\\
69.82	0.0091225000539379\\
69.83	0.00912310625702739\\
69.84	0.0091237129541015\\
69.85	0.00912432014761452\\
69.86	0.00912492784004213\\
69.87	0.00912553603388123\\
69.88	0.00912614473164959\\
69.89	0.00912675393588567\\
69.9	0.00912736364914826\\
69.91	0.00912797387401615\\
69.92	0.00912858461308781\\
69.93	0.00912919586898095\\
69.94	0.0091298076443321\\
69.95	0.00913041994179617\\
69.96	0.00913103276404594\\
69.97	0.00913164611377149\\
69.98	0.0091322599936797\\
69.99	0.00913287440649354\\
70	0.00913348935495148\\
70.01	0.00913410484180677\\
70.02	0.00913472086982667\\
70.03	0.00913533744179163\\
70.04	0.0091359545604945\\
70.05	0.00913657222873958\\
70.06	0.00913719044934165\\
70.07	0.00913780922512498\\
70.08	0.00913842855892224\\
70.09	0.00913904845357335\\
70.1	0.00913966891192429\\
70.11	0.00914028993682579\\
70.12	0.00914091153113203\\
70.13	0.00914153369769919\\
70.14	0.00914215643938393\\
70.15	0.00914277975904188\\
70.16	0.00914340365952592\\
70.17	0.00914402814368445\\
70.18	0.00914465321435956\\
70.19	0.00914527887438507\\
70.2	0.00914590512658456\\
70.21	0.00914653197376915\\
70.22	0.00914715941873537\\
70.23	0.00914778746426272\\
70.24	0.00914841611311126\\
70.25	0.00914904536801901\\
70.26	0.00914967523169927\\
70.27	0.00915030570683776\\
70.28	0.0091509367960897\\
70.29	0.00915156850207664\\
70.3	0.00915220082738329\\
70.31	0.00915283377455406\\
70.32	0.00915346734608956\\
70.33	0.00915410154444286\\
70.34	0.00915473637201563\\
70.35	0.00915537183115406\\
70.36	0.00915600792414466\\
70.37	0.0091566446532098\\
70.38	0.00915728202050317\\
70.39	0.00915792002810488\\
70.4	0.00915855867801646\\
70.41	0.00915919797215566\\
70.42	0.00915983791235092\\
70.43	0.00916047850033572\\
70.44	0.00916111973774261\\
70.45	0.00916176162609699\\
70.46	0.00916240416681072\\
70.47	0.0091630473611753\\
70.48	0.00916369121035493\\
70.49	0.00916433571537919\\
70.5	0.0091649808771354\\
70.51	0.00916562669636077\\
70.52	0.00916627317363407\\
70.53	0.0091669203093671\\
70.54	0.00916756810379577\\
70.55	0.00916821655697077\\
70.56	0.00916886566874789\\
70.57	0.00916951543877803\\
70.58	0.00917016586649662\\
70.59	0.00917081695111285\\
70.6	0.00917146869159827\\
70.61	0.00917212108667506\\
70.62	0.00917277413480378\\
70.63	0.00917342783417066\\
70.64	0.00917408218267435\\
70.65	0.00917473717791224\\
70.66	0.00917539281716609\\
70.67	0.00917604909738728\\
70.68	0.00917670601518136\\
70.69	0.00917736356679201\\
70.7	0.00917802174808445\\
70.71	0.00917868055452817\\
70.72	0.00917933998117892\\
70.73	0.00918000002266016\\
70.74	0.00918066067314368\\
70.75	0.00918132192632949\\
70.76	0.00918198377542503\\
70.77	0.00918264621312346\\
70.78	0.00918330923158124\\
70.79	0.00918397282239479\\
70.8	0.00918463697657633\\
70.81	0.0091853016845287\\
70.82	0.00918596693601935\\
70.83	0.00918663272015332\\
70.84	0.00918729902534516\\
70.85	0.00918796583928984\\
70.86	0.00918863314893258\\
70.87	0.00918930094043754\\
70.88	0.00918996919915533\\
70.89	0.00919063790958938\\
70.9	0.00919130705536096\\
70.91	0.00919197661917301\\
70.92	0.00919264658277256\\
70.93	0.00919331692691182\\
70.94	0.0091939876313078\\
70.95	0.00919465867460047\\
70.96	0.00919533003430937\\
70.97	0.00919600168678863\\
70.98	0.0091966736071804\\
70.99	0.00919734576936646\\
71	0.00919801814591825\\
71.01	0.00919869070804489\\
71.02	0.00919936342553942\\
71.03	0.00920003626672311\\
71.04	0.00920070919838769\\
71.05	0.00920138218573549\\
71.06	0.0092020551923175\\
71.07	0.00920272817996909\\
71.08	0.00920340110874352\\
71.09	0.00920407393684294\\
71.1	0.00920474662054706\\
71.11	0.00920541911413912\\
71.12	0.00920609136982931\\
71.13	0.00920676333767535\\
71.14	0.00920743496550041\\
71.15	0.00920810619880785\\
71.16	0.00920877698069316\\
71.17	0.00920944725175259\\
71.18	0.00921011694998868\\
71.19	0.00921078601071227\\
71.2	0.00921145436644121\\
71.21	0.00921212194679533\\
71.22	0.00921278867838779\\
71.23	0.00921345448471254\\
71.24	0.00921411928602781\\
71.25	0.00921478299923555\\
71.26	0.00921544553775652\\
71.27	0.00921610681140103\\
71.28	0.0092167667262352\\
71.29	0.00921742518444238\\
71.3	0.00921808208417984\\
71.31	0.00921873731943035\\
71.32	0.0092193907798486\\
71.33	0.00922004235060224\\
71.34	0.00922069191220735\\
71.35	0.00922133934035815\\
71.36	0.00922198450575081\\
71.37	0.0092226272739011\\
71.38	0.00922326750495563\\
71.39	0.00922390505349664\\
71.4	0.00922453976833986\\
71.41	0.00922517149232547\\
71.42	0.00922580006210171\\
71.43	0.00922642530790104\\
71.44	0.00922704705330856\\
71.45	0.0092276651150223\\
71.46	0.00922827930260537\\
71.47	0.00922888941822942\\
71.48	0.00922949525640929\\
71.49	0.00923009660372852\\
71.5	0.00923069323855535\\
71.51	0.00923128493074893\\
71.52	0.00923187144135548\\
71.53	0.00923245252229393\\
71.54	0.00923302791603079\\
71.55	0.00923359735524382\\
71.56	0.00923416056247416\\
71.57	0.00923471724976661\\
71.58	0.00923526764281969\\
71.59	0.00923581834425938\\
71.6	0.00923636935377676\\
71.61	0.00923692067104964\\
71.62	0.00923747229574248\\
71.63	0.00923802422750636\\
71.64	0.00923857646597888\\
71.65	0.00923912901078417\\
71.66	0.00923968186153281\\
71.67	0.00924023501782188\\
71.68	0.00924078847923491\\
71.69	0.00924134224534193\\
71.7	0.00924189631569949\\
71.71	0.00924245068985068\\
71.72	0.00924300536732527\\
71.73	0.00924356034763974\\
71.74	0.00924411563029742\\
71.75	0.00924467121478857\\
71.76	0.00924522710059063\\
71.77	0.0092457832871683\\
71.78	0.00924633977397381\\
71.79	0.00924689656044713\\
71.8	0.00924745364601621\\
71.81	0.00924801103009733\\
71.82	0.00924856871209536\\
71.83	0.00924912669140415\\
71.84	0.00924968496740692\\
71.85	0.00925024353947671\\
71.86	0.00925080240697682\\
71.87	0.00925136156926132\\
71.88	0.00925192102567565\\
71.89	0.00925248077555717\\
71.9	0.00925304081823587\\
71.91	0.009253601153035\\
71.92	0.0092541617792719\\
71.93	0.00925472269625877\\
71.94	0.00925528390330353\\
71.95	0.0092558453997108\\
71.96	0.00925640718478287\\
71.97	0.00925696925782074\\
71.98	0.00925753161812532\\
71.99	0.00925809426499861\\
72	0.00925865719774497\\
72.01	0.00925922041567255\\
72.02	0.00925978391809469\\
72.03	0.00926034770433152\\
72.04	0.00926091177371159\\
72.05	0.00926147612557359\\
72.06	0.00926204075926825\\
72.07	0.00926260567416026\\
72.08	0.00926317086963034\\
72.09	0.00926373634507746\\
72.1	0.00926430209992114\\
72.11	0.00926486813360388\\
72.12	0.00926543444559378\\
72.13	0.0092660010353872\\
72.14	0.00926656790251172\\
72.15	0.00926713504652908\\
72.16	0.00926770246703845\\
72.17	0.00926827016367972\\
72.18	0.00926883813613709\\
72.19	0.00926940638414276\\
72.2	0.00926997490748089\\
72.21	0.00927054370599166\\
72.22	0.00927111277957564\\
72.23	0.00927168212819834\\
72.24	0.00927225175189497\\
72.25	0.00927282165077548\\
72.26	0.00927339182502982\\
72.27	0.00927396227493347\\
72.28	0.00927453300085327\\
72.29	0.00927510400325348\\
72.3	0.00927567528270217\\
72.31	0.00927624683987795\\
72.32	0.00927681867557694\\
72.33	0.00927739079072018\\
72.34	0.00927796318636129\\
72.35	0.00927853586369458\\
72.36	0.0092791088240635\\
72.37	0.00927968206896946\\
72.38	0.00928025560008117\\
72.39	0.00928082941924428\\
72.4	0.00928140352849156\\
72.41	0.00928197793005352\\
72.42	0.00928255262636952\\
72.43	0.00928312762009943\\
72.44	0.00928370291413572\\
72.45	0.00928427851161623\\
72.46	0.00928485441593741\\
72.47	0.00928543063076828\\
72.48	0.00928600716006485\\
72.49	0.00928658400808537\\
72.5	0.00928716117940613\\
72.51	0.00928773867893804\\
72.52	0.00928831651194389\\
72.53	0.00928889468405645\\
72.54	0.00928947320129732\\
72.55	0.00929005207009657\\
72.56	0.00929063129731336\\
72.57	0.00929121089025737\\
72.58	0.00929179085671119\\
72.59	0.00929237120495372\\
72.6	0.0092929519437845\\
72.61	0.00929353308254924\\
72.62	0.00929411463116631\\
72.63	0.00929469660015443\\
72.64	0.00929527900066161\\
72.65	0.00929586184449526\\
72.66	0.00929644514415356\\
72.67	0.00929702891285832\\
72.68	0.00929761316458905\\
72.69	0.00929819791411864\\
72.7	0.00929878317705046\\
72.71	0.0092993689698571\\
72.72	0.00929995530992066\\
72.73	0.00930054221557487\\
72.74	0.00930112970614884\\
72.75	0.00930171780094145\\
72.76	0.00930230651358761\\
72.77	0.00930289585835774\\
72.78	0.00930348585018432\\
72.79	0.00930407650468944\\
72.8	0.00930466783821347\\
72.81	0.00930525986784466\\
72.82	0.00930585261145006\\
72.83	0.0093064460877075\\
72.84	0.00930704031613882\\
72.85	0.00930763531714433\\
72.86	0.00930823111203865\\
72.87	0.00930882772308778\\
72.88	0.00930942517354772\\
72.89	0.00931002348770438\\
72.9	0.00931062269091515\\
72.91	0.00931122280965187\\
72.92	0.00931182387154558\\
72.93	0.00931242590543278\\
72.94	0.00931302894140351\\
72.95	0.00931363301085125\\
72.96	0.00931423814652455\\
72.97	0.00931484438258072\\
72.98	0.00931545175464145\\
72.99	0.00931606028722278\\
73	0.00931667000180168\\
73.01	0.00931728092044048\\
73.02	0.00931789306580369\\
73.03	0.00931850646117528\\
73.04	0.0093191211304765\\
73.05	0.00931973709828419\\
73.06	0.0093203543898497\\
73.07	0.00932097303111824\\
73.08	0.00932159304874899\\
73.09	0.00932221447013562\\
73.1	0.00932283732342754\\
73.11	0.00932346163755179\\
73.12	0.00932408744223544\\
73.13	0.00932471476802885\\
73.14	0.00932534364632948\\
73.15	0.00932597410940646\\
73.16	0.0093266061904259\\
73.17	0.00932723992347696\\
73.18	0.00932787534359865\\
73.19	0.00932851248680752\\
73.2	0.00932915139012606\\
73.21	0.00932979209161207\\
73.22	0.00933043463038884\\
73.23	0.00933107904667622\\
73.24	0.0093317253818227\\
73.25	0.00933237367833838\\
73.26	0.00933302397992894\\
73.27	0.00933367633153068\\
73.28	0.00933433077934655\\
73.29	0.00933498737088333\\
73.3	0.00933564615498984\\
73.31	0.0093363071818964\\
73.32	0.00933697050325543\\
73.33	0.00933763617218327\\
73.34	0.00933830424330327\\
73.35	0.0093389747727902\\
73.36	0.00933964781841597\\
73.37	0.00934032343959675\\
73.38	0.00934100169744157\\
73.39	0.00934168265480225\\
73.4	0.009342366376325\\
73.41	0.00934305292850351\\
73.42	0.00934374237973363\\
73.43	0.00934443480036976\\
73.44	0.00934513026278294\\
73.45	0.0093458288414207\\
73.46	0.00934653061286871\\
73.47	0.00934723565591436\\
73.48	0.0093479440516122\\
73.49	0.00934865588335147\\
73.5	0.00934937123692563\\
73.51	0.00935009020060398\\
73.52	0.00935081286520558\\
73.53	0.00935153932417531\\
73.54	0.0093522696736623\\
73.55	0.00935300401260083\\
73.56	0.00935374244279355\\
73.57	0.00935448506899743\\
73.58	0.00935523199901221\\
73.59	0.00935598334377162\\
73.6	0.00935673921743741\\
73.61	0.0093574997374962\\
73.62	0.00935826502485946\\
73.63	0.00935903520396634\\
73.64	0.00935981040288986\\
73.65	0.00936059075344633\\
73.66	0.00936137639130799\\
73.67	0.00936216745611937\\
73.68	0.00936296409161704\\
73.69	0.00936376644575315\\
73.7	0.00936457467082278\\
73.71	0.00936538892359519\\
73.72	0.00936620936544916\\
73.73	0.00936703616251252\\
73.74	0.009367869485806\\
73.75	0.00936870951139151\\
73.76	0.00936955642052507\\
73.77	0.00937041039981441\\
73.78	0.00937127164138158\\
73.79	0.00937214034303046\\
73.8	0.00937301670841966\\
73.81	0.0093739009472406\\
73.82	0.00937479327540129\\
73.83	0.00937569391521579\\
73.84	0.00937660309559953\\
73.85	0.0093775210522708\\
73.86	0.00937844802795843\\
73.87	0.00937938427261609\\
73.88	0.00938033004364318\\
73.89	0.00938128560611263\\
73.9	0.00938225123300595\\
73.91	0.00938322720545546\\
73.92	0.00938421381299429\\
73.93	0.0093852113538141\\
73.94	0.00938622013503094\\
73.95	0.00938724047295944\\
73.96	0.00938827269339564\\
73.97	0.00938931713190872\\
73.98	0.00939037413414188\\
73.99	0.00939144405612271\\
74	0.00939252726458335\\
74.01	0.00939362413729078\\
74.02	0.00939473506338749\\
74.03	0.00939586044374288\\
74.04	0.00939700069131593\\
74.05	0.00939815623152912\\
74.06	0.00939932750265439\\
74.07	0.00940051495621119\\
74.08	0.00940171905737727\\
74.09	0.0094029402854124\\
74.1	0.0094041634172478\\
74.11	0.00940538697201502\\
74.12	0.00940661094960099\\
74.13	0.00940783534988611\\
74.14	0.0094090601727443\\
74.15	0.00941028541804323\\
74.16	0.00941151108564439\\
74.17	0.00941273717540331\\
74.18	0.00941396368716972\\
74.19	0.00941519062078782\\
74.2	0.00941641797609645\\
74.21	0.00941764575292941\\
74.22	0.00941887395111572\\
74.23	0.00942010257047994\\
74.24	0.00942133161084253\\
74.25	0.00942256107202022\\
74.26	0.00942379095382639\\
74.27	0.00942502125607159\\
74.28	0.00942625197856393\\
74.29	0.00942748312110964\\
74.3	0.00942871468351361\\
74.31	0.00942994666558003\\
74.32	0.00943117906711296\\
74.33	0.00943241188791706\\
74.34	0.00943364512779832\\
74.35	0.00943487878656485\\
74.36	0.00943611286402771\\
74.37	0.00943734736000177\\
74.38	0.00943858227430675\\
74.39	0.00943981760676813\\
74.4	0.00944105335721831\\
74.41	0.0094422895254977\\
74.42	0.00944352611145598\\
74.43	0.00944476311495336\\
74.44	0.00944600053586197\\
74.45	0.00944723837406729\\
74.46	0.00944847662946969\\
74.47	0.00944971530198607\\
74.48	0.00945095439155155\\
74.49	0.00945219389812132\\
74.5	0.00945343382167255\\
74.51	0.00945467416220638\\
74.52	0.00945591491975008\\
74.53	0.00945715609435934\\
74.54	0.0094583976861206\\
74.55	0.00945963969515356\\
74.56	0.00946088212161384\\
74.57	0.00946212496569575\\
74.58	0.00946336822763523\\
74.59	0.00946461190771288\\
74.6	0.00946585600625725\\
74.61	0.00946710052364824\\
74.62	0.00946834546032065\\
74.63	0.00946959081676796\\
74.64	0.00947083659354628\\
74.65	0.00947208279127851\\
74.66	0.00947332941065867\\
74.67	0.00947457645245649\\
74.68	0.00947582391752222\\
74.69	0.00947707180679161\\
74.7	0.00947832012129128\\
74.71	0.00947956886214417\\
74.72	0.00948081803057536\\
74.73	0.0094820676279182\\
74.74	0.0094833176556206\\
74.75	0.00948456811525179\\
74.76	0.00948581900850922\\
74.77	0.00948707033722595\\
74.78	0.00948832210337828\\
74.79	0.00948957430909374\\
74.8	0.00949082695665953\\
74.81	0.00949208004788666\\
74.82	0.00949333358308714\\
74.83	0.00949458756257265\\
74.84	0.0094958419866545\\
74.85	0.00949709685564353\\
74.86	0.00949835216985011\\
74.87	0.00949960792958402\\
74.88	0.00950086413515444\\
74.89	0.00950212078686985\\
74.9	0.009503377885038\\
74.91	0.00950463542996581\\
74.92	0.00950589342195935\\
74.93	0.00950715186132368\\
74.94	0.00950841074836289\\
74.95	0.00950967008337993\\
74.96	0.0095109298666766\\
74.97	0.00951219009855341\\
74.98	0.00951345077930955\\
74.99	0.00951471190924277\\
75	0.00951597348864933\\
75.01	0.00951723551782385\\
75.02	0.00951849799705929\\
75.03	0.00951976092664679\\
75.04	0.00952102430687562\\
75.05	0.00952228813803304\\
75.06	0.00952355242040422\\
75.07	0.00952481715427215\\
75.08	0.00952608233991747\\
75.09	0.00952734797761843\\
75.1	0.00952861406765072\\
75.11	0.00952988061028737\\
75.12	0.00953114760579863\\
75.13	0.00953241505445184\\
75.14	0.00953368295651129\\
75.15	0.00953495131223811\\
75.16	0.0095362201218901\\
75.17	0.00953748938572164\\
75.18	0.00953875910398349\\
75.19	0.00954002927692264\\
75.2	0.00954129990478225\\
75.21	0.00954257098780135\\
75.22	0.00954384252621481\\
75.23	0.00954511452025308\\
75.24	0.00954638697014208\\
75.25	0.00954765987610301\\
75.26	0.00954893323835214\\
75.27	0.00955020705710067\\
75.28	0.00955148133255453\\
75.29	0.00955275606491418\\
75.3	0.00955403125437439\\
75.31	0.00955530690112409\\
75.32	0.00955658300534611\\
75.33	0.009557859567217\\
75.34	0.00955913658690681\\
75.35	0.00956041406457882\\
75.36	0.00956169200038938\\
75.37	0.00956297039448762\\
75.38	0.00956424924701523\\
75.39	0.00956552855810622\\
75.4	0.00956680832788665\\
75.41	0.00956808855647436\\
75.42	0.00956936924397874\\
75.43	0.00957065039050043\\
75.44	0.00957193199613104\\
75.45	0.00957321406095289\\
75.46	0.00957449658503866\\
75.47	0.00957577956845117\\
75.48	0.00957706301124301\\
75.49	0.00957834691345624\\
75.5	0.00957963127512208\\
75.51	0.00958091609626058\\
75.52	0.00958220137688028\\
75.53	0.00958348711697785\\
75.54	0.00958477331653776\\
75.55	0.0095860599755319\\
75.56	0.00958734709391923\\
75.57	0.0095886346716454\\
75.58	0.00958992270864232\\
75.59	0.00959121120482784\\
75.6	0.00959250016010529\\
75.61	0.00959378957436309\\
75.62	0.00959507944747433\\
75.63	0.00959636977929631\\
75.64	0.00959766056967014\\
75.65	0.00959895181842027\\
75.66	0.00960024352535403\\
75.67	0.00960153569026116\\
75.68	0.00960282831291333\\
75.69	0.00960412139306365\\
75.7	0.00960541493044619\\
75.71	0.00960670892477541\\
75.72	0.00960800337574573\\
75.73	0.0096092982830309\\
75.74	0.00961059364628352\\
75.75	0.00961188946513448\\
75.76	0.00961318573919235\\
75.77	0.00961448246804285\\
75.78	0.00961577965124825\\
75.79	0.00961707728834676\\
75.8	0.0096183753788519\\
75.81	0.00961967392225191\\
75.82	0.00962097291800908\\
75.83	0.00962227236555915\\
75.84	0.00962357226431058\\
75.85	0.00962487261364391\\
75.86	0.00962617341291109\\
75.87	0.00962747466143474\\
75.88	0.00962877635850747\\
75.89	0.00963007850339112\\
75.9	0.00963138109531604\\
75.91	0.00963268413348035\\
75.92	0.00963398761704913\\
75.93	0.00963529154515368\\
75.94	0.0096365959168907\\
75.95	0.00963790073132148\\
75.96	0.00963920598747111\\
75.97	0.00964051168432758\\
75.98	0.00964181782084098\\
75.99	0.00964312439592263\\
76	0.00964443140844419\\
76.01	0.00964573885723674\\
76.02	0.0096470467410899\\
76.03	0.00964835505875091\\
76.04	0.0096496638089237\\
76.05	0.00965097299026788\\
76.06	0.00965228260139784\\
76.07	0.00965359264088172\\
76.08	0.00965490310724046\\
76.09	0.00965621399894673\\
76.1	0.00965752531442397\\
76.11	0.00965883705204531\\
76.12	0.0096601492101325\\
76.13	0.00966146178695489\\
76.14	0.00966277478072833\\
76.15	0.00966408818961404\\
76.16	0.00966540201171755\\
76.17	0.00966671624508756\\
76.18	0.00966803088771478\\
76.19	0.00966934593753081\\
76.2	0.00967066139240695\\
76.21	0.00967197725015308\\
76.22	0.00967329350851638\\
76.23	0.00967461016518023\\
76.24	0.00967592721776291\\
76.25	0.00967724466381645\\
76.26	0.00967856250082537\\
76.27	0.0096798807262054\\
76.28	0.0096811993373023\\
76.29	0.00968251833139052\\
76.3	0.009683837705672\\
76.31	0.00968515745727482\\
76.32	0.009686477583252\\
76.33	0.00968779808058012\\
76.34	0.00968911894615806\\
76.35	0.00969044017680572\\
76.36	0.00969176176926267\\
76.37	0.00969308372018685\\
76.38	0.00969440602615328\\
76.39	0.00969572868365272\\
76.4	0.00969705168909035\\
76.41	0.00969837503878451\\
76.42	0.00969969872896529\\
76.43	0.00970102275577334\\
76.44	0.00970234711525847\\
76.45	0.00970367180337842\\
76.46	0.00970499681599753\\
76.47	0.00970632214888548\\
76.48	0.00970764779771604\\
76.49	0.0097089737580658\\
76.5	0.00971030002541291\\
76.51	0.00971162659513593\\
76.52	0.00971295346251254\\
76.53	0.00971428062271844\\
76.54	0.00971560807082616\\
76.55	0.00971693580180394\\
76.56	0.00971826381051462\\
76.57	0.0097195920917146\\
76.58	0.0097209206400528\\
76.59	0.00972224945006966\\
76.6	0.00972357851619621\\
76.61	0.00972490783275313\\
76.62	0.00972623739394993\\
76.63	0.00972756719388414\\
76.64	0.00972889722654053\\
76.65	0.00973022748579046\\
76.66	0.00973155796539124\\
76.67	0.00973288865898555\\
76.68	0.009734219560101\\
76.69	0.00973555066214972\\
76.7	0.00973688195842799\\
76.71	0.00973821344211606\\
76.72	0.00973954510627802\\
76.73	0.00974087694386169\\
76.74	0.00974220894769879\\
76.75	0.00974354111050501\\
76.76	0.00974487342488042\\
76.77	0.0097462058833098\\
76.78	0.00974753847816329\\
76.79	0.009748871201697\\
76.8	0.00975020404605394\\
76.81	0.00975153700326498\\
76.82	0.00975287006525004\\
76.83	0.00975420322381945\\
76.84	0.00975553647067543\\
76.85	0.00975686979741389\\
76.86	0.00975820319552632\\
76.87	0.00975953665640196\\
76.88	0.00976087017133012\\
76.89	0.00976220373150286\\
76.9	0.0097635373280178\\
76.91	0.00976487095188129\\
76.92	0.00976620459401174\\
76.93	0.00976753824524338\\
76.94	0.00976887189633022\\
76.95	0.00977020553795035\\
76.96	0.00977153916071058\\
76.97	0.00977287275515149\\
76.98	0.00977420631175271\\
76.99	0.00977553982093873\\
77	0.00977687327308501\\
77.01	0.00977820665852457\\
77.02	0.00977953996755498\\
77.03	0.00978087319044584\\
77.04	0.00978220631744675\\
77.05	0.00978353933879575\\
77.06	0.0097848722447283\\
77.07	0.00978620502548684\\
77.08	0.00978753767133089\\
77.09	0.0097888701725478\\
77.1	0.00979020251946405\\
77.11	0.0097915347024573\\
77.12	0.00979286671196906\\
77.13	0.00979419853851812\\
77.14	0.00979553017271469\\
77.15	0.00979686160527538\\
77.16	0.00979819282703893\\
77.17	0.00979952382898284\\
77.18	0.00980085460224085\\
77.19	0.00980218513812139\\
77.2	0.00980351542812699\\
77.21	0.00980484546397464\\
77.22	0.00980617523761727\\
77.23	0.00980750474126632\\
77.24	0.00980883396741539\\
77.25	0.00981016290886515\\
77.26	0.00981149155874948\\
77.27	0.00981281991056284\\
77.28	0.0098141479581891\\
77.29	0.00981547569593162\\
77.3	0.00981680311854496\\
77.31	0.00981813021953688\\
77.32	0.00981945699004156\\
77.33	0.00982078342151098\\
77.34	0.00982210950574259\\
77.35	0.00982343523490834\\
77.36	0.00982476060158512\\
77.37	0.00982608559878664\\
77.38	0.00982741021999688\\
77.39	0.00982873445920517\\
77.4	0.00983005831094282\\
77.41	0.00983138177032162\\
77.42	0.00983270483307404\\
77.43	0.00983402749559531\\
77.44	0.0098353497549875\\
77.45	0.00983667160910558\\
77.46	0.00983799305660562\\
77.47	0.00983931409699511\\
77.48	0.00984063473068567\\
77.49	0.00984195495904804\\
77.5	0.00984327478446954\\
77.51	0.00984459421041416\\
77.52	0.0098459132414852\\
77.53	0.00984723188349079\\
77.54	0.00984855014351221\\
77.55	0.00984986802997521\\
77.56	0.00985118555272445\\
77.57	0.00985250272310116\\
77.58	0.00985381955402416\\
77.59	0.0098551360600743\\
77.6	0.00985645225758261\\
77.61	0.00985776816472215\\
77.62	0.00985908379925234\\
77.63	0.00986039913948631\\
77.64	0.00986171416311329\\
77.65	0.00986302884718038\\
77.66	0.00986434316807359\\
77.67	0.00986565710149849\\
77.68	0.00986697062246007\\
77.69	0.00986828370524218\\
77.7	0.00986959632338625\\
77.71	0.00987090844966935\\
77.72	0.00987222005608166\\
77.73	0.00987353111380317\\
77.74	0.00987484159317973\\
77.75	0.00987615146369837\\
77.76	0.00987746069396183\\
77.77	0.00987876925166233\\
77.78	0.00988007710355458\\
77.79	0.00988138421542792\\
77.8	0.00988269055207759\\
77.81	0.00988399607727524\\
77.82	0.00988530075373837\\
77.83	0.00988660454309896\\
77.84	0.00988790740587112\\
77.85	0.00988920930141769\\
77.86	0.0098905101879159\\
77.87	0.00989181002232189\\
77.88	0.00989310876033424\\
77.89	0.00989440635635629\\
77.9	0.00989570276345735\\
77.91	0.00989699793333271\\
77.92	0.00989829181626241\\
77.93	0.00989958436106877\\
77.94	0.0099008755150726\\
77.95	0.00990216522404798\\
77.96	0.00990345343217582\\
77.97	0.00990474008199581\\
77.98	0.00990602511435701\\
77.99	0.00990730846836683\\
78	0.00990859008133851\\
78.01	0.0099098698887369\\
78.02	0.00991114782412263\\
78.03	0.00991242381909448\\
78.04	0.00991369780323006\\
78.05	0.00991496970402456\\
78.06	0.00991623944682764\\
78.07	0.00991750695477838\\
78.08	0.00991877214873819\\
78.09	0.00992003494722158\\
78.1	0.00992129526632494\\
78.11	0.00992255301965292\\
78.12	0.00992380811824262\\
78.13	0.00992506047048541\\
78.14	0.00992630998204631\\
78.15	0.00992755655578084\\
78.16	0.00992880009164932\\
78.17	0.00993004048662842\\
78.18	0.00993127763462008\\
78.19	0.0099325114263574\\
78.2	0.0099337417493078\\
78.21	0.00993496848757299\\
78.22	0.00993619152178588\\
78.23	0.00993741072900428\\
78.24	0.00993862598260124\\
78.25	0.00993983715215196\\
78.26	0.00994104410331717\\
78.27	0.00994224669772287\\
78.28	0.00994344479283625\\
78.29	0.0099446382418378\\
78.3	0.00994582689348928\\
78.31	0.0099470105919977\\
78.32	0.00994818917687491\\
78.33	0.00994936248279285\\
78.34	0.00995053033943415\\
78.35	0.00995169257133822\\
78.36	0.00995284899774228\\
78.37	0.00995399943241754\\
78.38	0.00995514368350016\\
78.39	0.00995628155331692\\
78.4	0.00995741283820537\\
78.41	0.00995853732832832\\
78.42	0.00995965480748252\\
78.43	0.0099607650529012\\
78.44	0.00996186783505056\\
78.45	0.00996296291741966\\
78.46	0.00996405005630379\\
78.47	0.00996512900058097\\
78.48	0.00996619949148137\\
78.49	0.0099672612623495\\
78.5	0.00996831403839884\\
78.51	0.00996935753645872\\
78.52	0.00997039146471322\\
78.53	0.00997141552243183\\
78.54	0.00997242939969153\\
78.55	0.00997343277709014\\
78.56	0.0099744253254506\\
78.57	0.00997540670551581\\
78.58	0.00997637656763397\\
78.59	0.00997733455143376\\
78.6	0.00997828028548945\\
78.61	0.00997921338697523\\
78.62	0.00998013346130866\\
78.63	0.00998104010178291\\
78.64	0.00998193288918719\\
78.65	0.00998281139141529\\
78.66	0.00998367516306167\\
78.67	0.00998452374500475\\
78.68	0.00998535666397702\\
78.69	0.00998617343212148\\
78.7	0.00998697354653412\\
78.71	0.00998775648879181\\
78.72	0.0099885217244653\\
78.73	0.00998926870261676\\
78.74	0.00998999685528138\\
78.75	0.00999070559693263\\
78.76	0.00999139432393051\\
78.77	0.00999206241395232\\
78.78	0.0099927092254055\\
78.79	0.00999333409682176\\
78.8	0.00999393634623212\\
78.81	0.00999451527052212\\
78.82	0.00999507014476665\\
78.83	0.00999560022154375\\
78.84	0.00999610473022666\\
78.85	0.00999658287625354\\
78.86	0.00999703384037408\\
78.87	0.0099974567778723\\
78.88	0.00999785081776483\\
78.89	0.00999821506197382\\
78.9	0.00999854858447381\\
78.91	0.00999885043041162\\
78.92	0.00999911961519857\\
78.93	0.00999935512357399\\
78.94	0.0099995559086393\\
78.95	0.00999972089086164\\
78.96	0.00999984895704617\\
78.97	0.00999993895927606\\
78.98	0.00999998971381908\\
78.99	0.01\\
79	0.01\\
79.01	0.01\\
79.02	0.01\\
79.03	0.01\\
79.04	0.01\\
79.05	0.01\\
79.06	0.01\\
79.07	0.01\\
79.08	0.01\\
79.09	0.01\\
79.1	0.01\\
79.11	0.01\\
79.12	0.01\\
79.13	0.01\\
79.14	0.01\\
79.15	0.01\\
79.16	0.01\\
79.17	0.01\\
79.18	0.01\\
79.19	0.01\\
79.2	0.01\\
79.21	0.01\\
79.22	0.01\\
79.23	0.01\\
79.24	0.01\\
79.25	0.01\\
79.26	0.01\\
79.27	0.01\\
79.28	0.01\\
79.29	0.01\\
79.3	0.01\\
79.31	0.01\\
79.32	0.01\\
79.33	0.01\\
79.34	0.01\\
79.35	0.01\\
79.36	0.01\\
79.37	0.01\\
79.38	0.01\\
79.39	0.01\\
79.4	0.01\\
79.41	0.01\\
79.42	0.01\\
79.43	0.01\\
79.44	0.01\\
79.45	0.01\\
79.46	0.01\\
79.47	0.01\\
79.48	0.01\\
79.49	0.01\\
79.5	0.01\\
79.51	0.01\\
79.52	0.01\\
79.53	0.01\\
79.54	0.01\\
79.55	0.01\\
79.56	0.01\\
79.57	0.01\\
79.58	0.01\\
79.59	0.01\\
79.6	0.01\\
79.61	0.01\\
79.62	0.01\\
79.63	0.01\\
79.64	0.01\\
79.65	0.01\\
79.66	0.01\\
79.67	0.01\\
79.68	0.01\\
79.69	0.01\\
79.7	0.01\\
79.71	0.01\\
79.72	0.01\\
79.73	0.01\\
79.74	0.01\\
79.75	0.01\\
79.76	0.01\\
79.77	0.01\\
79.78	0.01\\
79.79	0.01\\
79.8	0.01\\
79.81	0.01\\
79.82	0.01\\
79.83	0.01\\
79.84	0.01\\
79.85	0.01\\
79.86	0.01\\
79.87	0.01\\
79.88	0.01\\
79.89	0.01\\
79.9	0.01\\
79.91	0.01\\
79.92	0.01\\
79.93	0.01\\
79.94	0.01\\
79.95	0.01\\
79.96	0.01\\
79.97	0.01\\
79.98	0.01\\
79.99	0.01\\
80	0.01\\
80.01	0.01\\
};
\addplot [color=red,solid]
  table[row sep=crcr]{%
80.01	0.01\\
80.02	0.01\\
80.03	0.01\\
80.04	0.01\\
80.05	0.01\\
80.06	0.01\\
80.07	0.01\\
80.08	0.01\\
80.09	0.01\\
80.1	0.01\\
80.11	0.01\\
80.12	0.01\\
80.13	0.01\\
80.14	0.01\\
80.15	0.01\\
80.16	0.01\\
80.17	0.01\\
80.18	0.01\\
80.19	0.01\\
80.2	0.01\\
80.21	0.01\\
80.22	0.01\\
80.23	0.01\\
80.24	0.01\\
80.25	0.01\\
80.26	0.01\\
80.27	0.01\\
80.28	0.01\\
80.29	0.01\\
80.3	0.01\\
80.31	0.01\\
80.32	0.01\\
80.33	0.01\\
80.34	0.01\\
80.35	0.01\\
80.36	0.01\\
80.37	0.01\\
80.38	0.01\\
80.39	0.01\\
80.4	0.01\\
80.41	0.01\\
80.42	0.01\\
80.43	0.01\\
80.44	0.01\\
80.45	0.01\\
80.46	0.01\\
80.47	0.01\\
80.48	0.01\\
80.49	0.01\\
80.5	0.01\\
80.51	0.01\\
80.52	0.01\\
80.53	0.01\\
80.54	0.01\\
80.55	0.01\\
80.56	0.01\\
80.57	0.01\\
80.58	0.01\\
80.59	0.01\\
80.6	0.01\\
80.61	0.01\\
80.62	0.01\\
80.63	0.01\\
80.64	0.01\\
80.65	0.01\\
80.66	0.01\\
80.67	0.01\\
80.68	0.01\\
80.69	0.01\\
80.7	0.01\\
80.71	0.01\\
80.72	0.01\\
80.73	0.01\\
80.74	0.01\\
80.75	0.01\\
80.76	0.01\\
80.77	0.01\\
80.78	0.01\\
80.79	0.01\\
80.8	0.01\\
80.81	0.01\\
80.82	0.01\\
80.83	0.01\\
80.84	0.01\\
80.85	0.01\\
80.86	0.01\\
80.87	0.01\\
80.88	0.01\\
80.89	0.01\\
80.9	0.01\\
80.91	0.01\\
80.92	0.01\\
80.93	0.01\\
80.94	0.01\\
80.95	0.01\\
80.96	0.01\\
80.97	0.01\\
80.98	0.01\\
80.99	0.01\\
81	0.01\\
81.01	0.01\\
81.02	0.01\\
81.03	0.01\\
81.04	0.01\\
81.05	0.01\\
81.06	0.01\\
81.07	0.01\\
81.08	0.01\\
81.09	0.01\\
81.1	0.01\\
81.11	0.01\\
81.12	0.01\\
81.13	0.01\\
81.14	0.01\\
81.15	0.01\\
81.16	0.01\\
81.17	0.01\\
81.18	0.01\\
81.19	0.01\\
81.2	0.01\\
81.21	0.01\\
81.22	0.01\\
81.23	0.01\\
81.24	0.01\\
81.25	0.01\\
81.26	0.01\\
81.27	0.01\\
81.28	0.01\\
81.29	0.01\\
81.3	0.01\\
81.31	0.01\\
81.32	0.01\\
81.33	0.01\\
81.34	0.01\\
81.35	0.01\\
81.36	0.01\\
81.37	0.01\\
81.38	0.01\\
81.39	0.01\\
81.4	0.01\\
81.41	0.01\\
81.42	0.01\\
81.43	0.01\\
81.44	0.01\\
81.45	0.01\\
81.46	0.01\\
81.47	0.01\\
81.48	0.01\\
81.49	0.01\\
81.5	0.01\\
81.51	0.01\\
81.52	0.01\\
81.53	0.01\\
81.54	0.01\\
81.55	0.01\\
81.56	0.01\\
81.57	0.01\\
81.58	0.01\\
81.59	0.01\\
81.6	0.01\\
81.61	0.01\\
81.62	0.01\\
81.63	0.01\\
81.64	0.01\\
81.65	0.01\\
81.66	0.01\\
81.67	0.01\\
81.68	0.01\\
81.69	0.01\\
81.7	0.01\\
81.71	0.01\\
81.72	0.01\\
81.73	0.01\\
81.74	0.01\\
81.75	0.01\\
81.76	0.01\\
81.77	0.01\\
81.78	0.01\\
81.79	0.01\\
81.8	0.01\\
81.81	0.01\\
81.82	0.01\\
81.83	0.01\\
81.84	0.01\\
81.85	0.01\\
81.86	0.01\\
81.87	0.01\\
81.88	0.01\\
81.89	0.01\\
81.9	0.01\\
81.91	0.01\\
81.92	0.01\\
81.93	0.01\\
81.94	0.01\\
81.95	0.01\\
81.96	0.01\\
81.97	0.01\\
81.98	0.01\\
81.99	0.01\\
82	0.01\\
82.01	0.01\\
82.02	0.01\\
82.03	0.01\\
82.04	0.01\\
82.05	0.01\\
82.06	0.01\\
82.07	0.01\\
82.08	0.01\\
82.09	0.01\\
82.1	0.01\\
82.11	0.01\\
82.12	0.01\\
82.13	0.01\\
82.14	0.01\\
82.15	0.01\\
82.16	0.01\\
82.17	0.01\\
82.18	0.01\\
82.19	0.01\\
82.2	0.01\\
82.21	0.01\\
82.22	0.01\\
82.23	0.01\\
82.24	0.01\\
82.25	0.01\\
82.26	0.01\\
82.27	0.01\\
82.28	0.01\\
82.29	0.01\\
82.3	0.01\\
82.31	0.01\\
82.32	0.01\\
82.33	0.01\\
82.34	0.01\\
82.35	0.01\\
82.36	0.01\\
82.37	0.01\\
82.38	0.01\\
82.39	0.01\\
82.4	0.01\\
82.41	0.01\\
82.42	0.01\\
82.43	0.01\\
82.44	0.01\\
82.45	0.01\\
82.46	0.01\\
82.47	0.01\\
82.48	0.01\\
82.49	0.01\\
82.5	0.01\\
82.51	0.01\\
82.52	0.01\\
82.53	0.01\\
82.54	0.01\\
82.55	0.01\\
82.56	0.01\\
82.57	0.01\\
82.58	0.01\\
82.59	0.01\\
82.6	0.01\\
82.61	0.01\\
82.62	0.01\\
82.63	0.01\\
82.64	0.01\\
82.65	0.01\\
82.66	0.01\\
82.67	0.01\\
82.68	0.01\\
82.69	0.01\\
82.7	0.01\\
82.71	0.01\\
82.72	0.01\\
82.73	0.01\\
82.74	0.01\\
82.75	0.01\\
82.76	0.01\\
82.77	0.01\\
82.78	0.01\\
82.79	0.01\\
82.8	0.01\\
82.81	0.01\\
82.82	0.01\\
82.83	0.01\\
82.84	0.01\\
82.85	0.01\\
82.86	0.01\\
82.87	0.01\\
82.88	0.01\\
82.89	0.01\\
82.9	0.01\\
82.91	0.01\\
82.92	0.01\\
82.93	0.01\\
82.94	0.01\\
82.95	0.01\\
82.96	0.01\\
82.97	0.01\\
82.98	0.01\\
82.99	0.01\\
83	0.01\\
83.01	0.01\\
83.02	0.01\\
83.03	0.01\\
83.04	0.01\\
83.05	0.01\\
83.06	0.01\\
83.07	0.01\\
83.08	0.01\\
83.09	0.01\\
83.1	0.01\\
83.11	0.01\\
83.12	0.01\\
83.13	0.01\\
83.14	0.01\\
83.15	0.01\\
83.16	0.01\\
83.17	0.01\\
83.18	0.01\\
83.19	0.01\\
83.2	0.01\\
83.21	0.01\\
83.22	0.01\\
83.23	0.01\\
83.24	0.01\\
83.25	0.01\\
83.26	0.01\\
83.27	0.01\\
83.28	0.01\\
83.29	0.01\\
83.3	0.01\\
83.31	0.01\\
83.32	0.01\\
83.33	0.01\\
83.34	0.01\\
83.35	0.01\\
83.36	0.01\\
83.37	0.01\\
83.38	0.01\\
83.39	0.01\\
83.4	0.01\\
83.41	0.01\\
83.42	0.01\\
83.43	0.01\\
83.44	0.01\\
83.45	0.01\\
83.46	0.01\\
83.47	0.01\\
83.48	0.01\\
83.49	0.01\\
83.5	0.01\\
83.51	0.01\\
83.52	0.01\\
83.53	0.01\\
83.54	0.01\\
83.55	0.01\\
83.56	0.01\\
83.57	0.01\\
83.58	0.01\\
83.59	0.01\\
83.6	0.01\\
83.61	0.01\\
83.62	0.01\\
83.63	0.01\\
83.64	0.01\\
83.65	0.01\\
83.66	0.01\\
83.67	0.01\\
83.68	0.01\\
83.69	0.01\\
83.7	0.01\\
83.71	0.01\\
83.72	0.01\\
83.73	0.01\\
83.74	0.01\\
83.75	0.01\\
83.76	0.01\\
83.77	0.01\\
83.78	0.01\\
83.79	0.01\\
83.8	0.01\\
83.81	0.01\\
83.82	0.01\\
83.83	0.01\\
83.84	0.01\\
83.85	0.01\\
83.86	0.01\\
83.87	0.01\\
83.88	0.01\\
83.89	0.01\\
83.9	0.01\\
83.91	0.01\\
83.92	0.01\\
83.93	0.01\\
83.94	0.01\\
83.95	0.01\\
83.96	0.01\\
83.97	0.01\\
83.98	0.01\\
83.99	0.01\\
84	0.01\\
84.01	0.01\\
84.02	0.01\\
84.03	0.01\\
84.04	0.01\\
84.05	0.01\\
84.06	0.01\\
84.07	0.01\\
84.08	0.01\\
84.09	0.01\\
84.1	0.01\\
84.11	0.01\\
84.12	0.01\\
84.13	0.01\\
84.14	0.01\\
84.15	0.01\\
84.16	0.01\\
84.17	0.01\\
84.18	0.01\\
84.19	0.01\\
84.2	0.01\\
84.21	0.01\\
84.22	0.01\\
84.23	0.01\\
84.24	0.01\\
84.25	0.01\\
84.26	0.01\\
84.27	0.01\\
84.28	0.01\\
84.29	0.01\\
84.3	0.01\\
84.31	0.01\\
84.32	0.01\\
84.33	0.01\\
84.34	0.01\\
84.35	0.01\\
84.36	0.01\\
84.37	0.01\\
84.38	0.01\\
84.39	0.01\\
84.4	0.01\\
84.41	0.01\\
84.42	0.01\\
84.43	0.01\\
84.44	0.01\\
84.45	0.01\\
84.46	0.01\\
84.47	0.01\\
84.48	0.01\\
84.49	0.01\\
84.5	0.01\\
84.51	0.01\\
84.52	0.01\\
84.53	0.01\\
84.54	0.01\\
84.55	0.01\\
84.56	0.01\\
84.57	0.01\\
84.58	0.01\\
84.59	0.01\\
84.6	0.01\\
84.61	0.01\\
84.62	0.01\\
84.63	0.01\\
84.64	0.01\\
84.65	0.01\\
84.66	0.01\\
84.67	0.01\\
84.68	0.01\\
84.69	0.01\\
84.7	0.01\\
84.71	0.01\\
84.72	0.01\\
84.73	0.01\\
84.74	0.01\\
84.75	0.01\\
84.76	0.01\\
84.77	0.01\\
84.78	0.01\\
84.79	0.01\\
84.8	0.01\\
84.81	0.01\\
84.82	0.01\\
84.83	0.01\\
84.84	0.01\\
84.85	0.01\\
84.86	0.01\\
84.87	0.01\\
84.88	0.01\\
84.89	0.01\\
84.9	0.01\\
84.91	0.01\\
84.92	0.01\\
84.93	0.01\\
84.94	0.01\\
84.95	0.01\\
84.96	0.01\\
84.97	0.01\\
84.98	0.01\\
84.99	0.01\\
85	0.01\\
85.01	0.01\\
85.02	0.01\\
85.03	0.01\\
85.04	0.01\\
85.05	0.01\\
85.06	0.01\\
85.07	0.01\\
85.08	0.01\\
85.09	0.01\\
85.1	0.01\\
85.11	0.01\\
85.12	0.01\\
85.13	0.01\\
85.14	0.01\\
85.15	0.01\\
85.16	0.01\\
85.17	0.01\\
85.18	0.01\\
85.19	0.01\\
85.2	0.01\\
85.21	0.01\\
85.22	0.01\\
85.23	0.01\\
85.24	0.01\\
85.25	0.01\\
85.26	0.01\\
85.27	0.01\\
85.28	0.01\\
85.29	0.01\\
85.3	0.01\\
85.31	0.01\\
85.32	0.01\\
85.33	0.01\\
85.34	0.01\\
85.35	0.01\\
85.36	0.01\\
85.37	0.01\\
85.38	0.01\\
85.39	0.01\\
85.4	0.01\\
85.41	0.01\\
85.42	0.01\\
85.43	0.01\\
85.44	0.01\\
85.45	0.01\\
85.46	0.01\\
85.47	0.01\\
85.48	0.01\\
85.49	0.01\\
85.5	0.01\\
85.51	0.01\\
85.52	0.01\\
85.53	0.01\\
85.54	0.01\\
85.55	0.01\\
85.56	0.01\\
85.57	0.01\\
85.58	0.01\\
85.59	0.01\\
85.6	0.01\\
85.61	0.01\\
85.62	0.01\\
85.63	0.01\\
85.64	0.01\\
85.65	0.01\\
85.66	0.01\\
85.67	0.01\\
85.68	0.01\\
85.69	0.01\\
85.7	0.01\\
85.71	0.01\\
85.72	0.01\\
85.73	0.01\\
85.74	0.01\\
85.75	0.01\\
85.76	0.01\\
85.77	0.01\\
85.78	0.01\\
85.79	0.01\\
85.8	0.01\\
85.81	0.01\\
85.82	0.01\\
85.83	0.01\\
85.84	0.01\\
85.85	0.01\\
85.86	0.01\\
85.87	0.01\\
85.88	0.01\\
85.89	0.01\\
85.9	0.01\\
85.91	0.01\\
85.92	0.01\\
85.93	0.01\\
85.94	0.01\\
85.95	0.01\\
85.96	0.01\\
85.97	0.01\\
85.98	0.01\\
85.99	0.01\\
86	0.01\\
86.01	0.01\\
86.02	0.01\\
86.03	0.01\\
86.04	0.01\\
86.05	0.01\\
86.06	0.01\\
86.07	0.01\\
86.08	0.01\\
86.09	0.01\\
86.1	0.01\\
86.11	0.01\\
86.12	0.01\\
86.13	0.01\\
86.14	0.01\\
86.15	0.01\\
86.16	0.01\\
86.17	0.01\\
86.18	0.01\\
86.19	0.01\\
86.2	0.01\\
86.21	0.01\\
86.22	0.01\\
86.23	0.01\\
86.24	0.01\\
86.25	0.01\\
86.26	0.01\\
86.27	0.01\\
86.28	0.01\\
86.29	0.01\\
86.3	0.01\\
86.31	0.01\\
86.32	0.01\\
86.33	0.01\\
86.34	0.01\\
86.35	0.01\\
86.36	0.01\\
86.37	0.01\\
86.38	0.01\\
86.39	0.01\\
86.4	0.01\\
86.41	0.01\\
86.42	0.01\\
86.43	0.01\\
86.44	0.01\\
86.45	0.01\\
86.46	0.01\\
86.47	0.01\\
86.48	0.01\\
86.49	0.01\\
86.5	0.01\\
86.51	0.01\\
86.52	0.01\\
86.53	0.01\\
86.54	0.01\\
86.55	0.01\\
86.56	0.01\\
86.57	0.01\\
86.58	0.01\\
86.59	0.01\\
86.6	0.01\\
86.61	0.01\\
86.62	0.01\\
86.63	0.01\\
86.64	0.01\\
86.65	0.01\\
86.66	0.01\\
86.67	0.01\\
86.68	0.01\\
86.69	0.01\\
86.7	0.01\\
86.71	0.01\\
86.72	0.01\\
86.73	0.01\\
86.74	0.01\\
86.75	0.01\\
86.76	0.01\\
86.77	0.01\\
86.78	0.01\\
86.79	0.01\\
86.8	0.01\\
86.81	0.01\\
86.82	0.01\\
86.83	0.01\\
86.84	0.01\\
86.85	0.01\\
86.86	0.01\\
86.87	0.01\\
86.88	0.01\\
86.89	0.01\\
86.9	0.01\\
86.91	0.01\\
86.92	0.01\\
86.93	0.01\\
86.94	0.01\\
86.95	0.01\\
86.96	0.01\\
86.97	0.01\\
86.98	0.01\\
86.99	0.01\\
87	0.01\\
87.01	0.01\\
87.02	0.01\\
87.03	0.01\\
87.04	0.01\\
87.05	0.01\\
87.06	0.01\\
87.07	0.01\\
87.08	0.01\\
87.09	0.01\\
87.1	0.01\\
87.11	0.01\\
87.12	0.01\\
87.13	0.01\\
87.14	0.01\\
87.15	0.01\\
87.16	0.01\\
87.17	0.01\\
87.18	0.01\\
87.19	0.01\\
87.2	0.01\\
87.21	0.01\\
87.22	0.01\\
87.23	0.01\\
87.24	0.01\\
87.25	0.01\\
87.26	0.01\\
87.27	0.01\\
87.28	0.01\\
87.29	0.01\\
87.3	0.01\\
87.31	0.01\\
87.32	0.01\\
87.33	0.01\\
87.34	0.01\\
87.35	0.01\\
87.36	0.01\\
87.37	0.01\\
87.38	0.01\\
87.39	0.01\\
87.4	0.01\\
87.41	0.01\\
87.42	0.01\\
87.43	0.01\\
87.44	0.01\\
87.45	0.01\\
87.46	0.01\\
87.47	0.01\\
87.48	0.01\\
87.49	0.01\\
87.5	0.01\\
87.51	0.01\\
87.52	0.01\\
87.53	0.01\\
87.54	0.01\\
87.55	0.01\\
87.56	0.01\\
87.57	0.01\\
87.58	0.01\\
87.59	0.01\\
87.6	0.01\\
87.61	0.01\\
87.62	0.01\\
87.63	0.01\\
87.64	0.01\\
87.65	0.01\\
87.66	0.01\\
87.67	0.01\\
87.68	0.01\\
87.69	0.01\\
87.7	0.01\\
87.71	0.01\\
87.72	0.01\\
87.73	0.01\\
87.74	0.01\\
87.75	0.01\\
87.76	0.01\\
87.77	0.01\\
87.78	0.01\\
87.79	0.01\\
87.8	0.01\\
87.81	0.01\\
87.82	0.01\\
87.83	0.01\\
87.84	0.01\\
87.85	0.01\\
87.86	0.01\\
87.87	0.01\\
87.88	0.01\\
87.89	0.01\\
87.9	0.01\\
87.91	0.01\\
87.92	0.01\\
87.93	0.01\\
87.94	0.01\\
87.95	0.01\\
87.96	0.01\\
87.97	0.01\\
87.98	0.01\\
87.99	0.01\\
88	0.01\\
88.01	0.01\\
88.02	0.01\\
88.03	0.01\\
88.04	0.01\\
88.05	0.01\\
88.06	0.01\\
88.07	0.01\\
88.08	0.01\\
88.09	0.01\\
88.1	0.01\\
88.11	0.01\\
88.12	0.01\\
88.13	0.01\\
88.14	0.01\\
88.15	0.01\\
88.16	0.01\\
88.17	0.01\\
88.18	0.01\\
88.19	0.01\\
88.2	0.01\\
88.21	0.01\\
88.22	0.01\\
88.23	0.01\\
88.24	0.01\\
88.25	0.01\\
88.26	0.01\\
88.27	0.01\\
88.28	0.01\\
88.29	0.01\\
88.3	0.01\\
88.31	0.01\\
88.32	0.01\\
88.33	0.01\\
88.34	0.01\\
88.35	0.01\\
88.36	0.01\\
88.37	0.01\\
88.38	0.01\\
88.39	0.01\\
88.4	0.01\\
88.41	0.01\\
88.42	0.01\\
88.43	0.01\\
88.44	0.01\\
88.45	0.01\\
88.46	0.01\\
88.47	0.01\\
88.48	0.01\\
88.49	0.01\\
88.5	0.01\\
88.51	0.01\\
88.52	0.01\\
88.53	0.01\\
88.54	0.01\\
88.55	0.01\\
88.56	0.01\\
88.57	0.01\\
88.58	0.01\\
88.59	0.01\\
88.6	0.01\\
88.61	0.01\\
88.62	0.01\\
88.63	0.01\\
88.64	0.01\\
88.65	0.01\\
88.66	0.01\\
88.67	0.01\\
88.68	0.01\\
88.69	0.01\\
88.7	0.01\\
88.71	0.01\\
88.72	0.01\\
88.73	0.01\\
88.74	0.01\\
88.75	0.01\\
88.76	0.01\\
88.77	0.01\\
88.78	0.01\\
88.79	0.01\\
88.8	0.01\\
88.81	0.01\\
88.82	0.01\\
88.83	0.01\\
88.84	0.01\\
88.85	0.01\\
88.86	0.01\\
88.87	0.01\\
88.88	0.01\\
88.89	0.01\\
88.9	0.01\\
88.91	0.01\\
88.92	0.01\\
88.93	0.01\\
88.94	0.01\\
88.95	0.01\\
88.96	0.01\\
88.97	0.01\\
88.98	0.01\\
88.99	0.01\\
89	0.01\\
89.01	0.01\\
89.02	0.01\\
89.03	0.01\\
89.04	0.01\\
89.05	0.01\\
89.06	0.01\\
89.07	0.01\\
89.08	0.01\\
89.09	0.01\\
89.1	0.01\\
89.11	0.01\\
89.12	0.01\\
89.13	0.01\\
89.14	0.01\\
89.15	0.01\\
89.16	0.01\\
89.17	0.01\\
89.18	0.01\\
89.19	0.01\\
89.2	0.01\\
89.21	0.01\\
89.22	0.01\\
89.23	0.01\\
89.24	0.01\\
89.25	0.01\\
89.26	0.01\\
89.27	0.01\\
89.28	0.01\\
89.29	0.01\\
89.3	0.01\\
89.31	0.01\\
89.32	0.01\\
89.33	0.01\\
89.34	0.01\\
89.35	0.01\\
89.36	0.01\\
89.37	0.01\\
89.38	0.01\\
89.39	0.01\\
89.4	0.01\\
89.41	0.01\\
89.42	0.01\\
89.43	0.01\\
89.44	0.01\\
89.45	0.01\\
89.46	0.01\\
89.47	0.01\\
89.48	0.01\\
89.49	0.01\\
89.5	0.01\\
89.51	0.01\\
89.52	0.01\\
89.53	0.01\\
89.54	0.01\\
89.55	0.01\\
89.56	0.01\\
89.57	0.01\\
89.58	0.01\\
89.59	0.01\\
89.6	0.01\\
89.61	0.01\\
89.62	0.01\\
89.63	0.01\\
89.64	0.01\\
89.65	0.01\\
89.66	0.01\\
89.67	0.01\\
89.68	0.01\\
89.69	0.01\\
89.7	0.01\\
89.71	0.01\\
89.72	0.01\\
89.73	0.01\\
89.74	0.01\\
89.75	0.01\\
89.76	0.01\\
89.77	0.01\\
89.78	0.01\\
89.79	0.01\\
89.8	0.01\\
89.81	0.01\\
89.82	0.01\\
89.83	0.01\\
89.84	0.01\\
89.85	0.01\\
89.86	0.01\\
89.87	0.01\\
89.88	0.01\\
89.89	0.01\\
89.9	0.01\\
89.91	0.01\\
89.92	0.01\\
89.93	0.01\\
89.94	0.01\\
89.95	0.01\\
89.96	0.01\\
89.97	0.01\\
89.98	0.01\\
89.99	0.01\\
90	0.01\\
90.01	0.01\\
90.02	0.01\\
90.03	0.01\\
90.04	0.01\\
90.05	0.01\\
90.06	0.01\\
90.07	0.01\\
90.08	0.01\\
90.09	0.01\\
90.1	0.01\\
90.11	0.01\\
90.12	0.01\\
90.13	0.01\\
90.14	0.01\\
90.15	0.01\\
90.16	0.01\\
90.17	0.01\\
90.18	0.01\\
90.19	0.01\\
90.2	0.01\\
90.21	0.01\\
90.22	0.01\\
90.23	0.01\\
90.24	0.01\\
90.25	0.01\\
90.26	0.01\\
90.27	0.01\\
90.28	0.01\\
90.29	0.01\\
90.3	0.01\\
90.31	0.01\\
90.32	0.01\\
90.33	0.01\\
90.34	0.01\\
90.35	0.01\\
90.36	0.01\\
90.37	0.01\\
90.38	0.01\\
90.39	0.01\\
90.4	0.01\\
90.41	0.01\\
90.42	0.01\\
90.43	0.01\\
90.44	0.01\\
90.45	0.01\\
90.46	0.01\\
90.47	0.01\\
90.48	0.01\\
90.49	0.01\\
90.5	0.01\\
90.51	0.01\\
90.52	0.01\\
90.53	0.01\\
90.54	0.01\\
90.55	0.01\\
90.56	0.01\\
90.57	0.01\\
90.58	0.01\\
90.59	0.01\\
90.6	0.01\\
90.61	0.01\\
90.62	0.01\\
90.63	0.01\\
90.64	0.01\\
90.65	0.01\\
90.66	0.01\\
90.67	0.01\\
90.68	0.01\\
90.69	0.01\\
90.7	0.01\\
90.71	0.01\\
90.72	0.01\\
90.73	0.01\\
90.74	0.01\\
90.75	0.01\\
90.76	0.01\\
90.77	0.01\\
90.78	0.01\\
90.79	0.01\\
90.8	0.01\\
90.81	0.01\\
90.82	0.01\\
90.83	0.01\\
90.84	0.01\\
90.85	0.01\\
90.86	0.01\\
90.87	0.01\\
90.88	0.01\\
90.89	0.01\\
90.9	0.01\\
90.91	0.01\\
90.92	0.01\\
90.93	0.01\\
90.94	0.01\\
90.95	0.01\\
90.96	0.01\\
90.97	0.01\\
90.98	0.01\\
90.99	0.01\\
91	0.01\\
91.01	0.01\\
91.02	0.01\\
91.03	0.01\\
91.04	0.01\\
91.05	0.01\\
91.06	0.01\\
91.07	0.01\\
91.08	0.01\\
91.09	0.01\\
91.1	0.01\\
91.11	0.01\\
91.12	0.01\\
91.13	0.01\\
91.14	0.01\\
91.15	0.01\\
91.16	0.01\\
91.17	0.01\\
91.18	0.01\\
91.19	0.01\\
91.2	0.01\\
91.21	0.01\\
91.22	0.01\\
91.23	0.01\\
91.24	0.01\\
91.25	0.01\\
91.26	0.01\\
91.27	0.01\\
91.28	0.01\\
91.29	0.01\\
91.3	0.01\\
91.31	0.01\\
91.32	0.01\\
91.33	0.01\\
91.34	0.01\\
91.35	0.01\\
91.36	0.01\\
91.37	0.01\\
91.38	0.01\\
91.39	0.01\\
91.4	0.01\\
91.41	0.01\\
91.42	0.01\\
91.43	0.01\\
91.44	0.01\\
91.45	0.01\\
91.46	0.01\\
91.47	0.01\\
91.48	0.01\\
91.49	0.01\\
91.5	0.01\\
91.51	0.01\\
91.52	0.01\\
91.53	0.01\\
91.54	0.01\\
91.55	0.01\\
91.56	0.01\\
91.57	0.01\\
91.58	0.01\\
91.59	0.01\\
91.6	0.01\\
91.61	0.01\\
91.62	0.01\\
91.63	0.01\\
91.64	0.01\\
91.65	0.01\\
91.66	0.01\\
91.67	0.01\\
91.68	0.01\\
91.69	0.01\\
91.7	0.01\\
91.71	0.01\\
91.72	0.01\\
91.73	0.01\\
91.74	0.01\\
91.75	0.01\\
91.76	0.01\\
91.77	0.01\\
91.78	0.01\\
91.79	0.01\\
91.8	0.01\\
91.81	0.01\\
91.82	0.01\\
91.83	0.01\\
91.84	0.01\\
91.85	0.01\\
91.86	0.01\\
91.87	0.01\\
91.88	0.01\\
91.89	0.01\\
91.9	0.01\\
91.91	0.01\\
91.92	0.01\\
91.93	0.01\\
91.94	0.01\\
91.95	0.01\\
91.96	0.01\\
91.97	0.01\\
91.98	0.01\\
91.99	0.01\\
92	0.01\\
92.01	0.01\\
92.02	0.01\\
92.03	0.01\\
92.04	0.01\\
92.05	0.01\\
92.06	0.01\\
92.07	0.01\\
92.08	0.01\\
92.09	0.01\\
92.1	0.01\\
92.11	0.01\\
92.12	0.01\\
92.13	0.01\\
92.14	0.01\\
92.15	0.01\\
92.16	0.01\\
92.17	0.01\\
92.18	0.01\\
92.19	0.01\\
92.2	0.01\\
92.21	0.01\\
92.22	0.01\\
92.23	0.01\\
92.24	0.01\\
92.25	0.01\\
92.26	0.01\\
92.27	0.01\\
92.28	0.01\\
92.29	0.01\\
92.3	0.01\\
92.31	0.01\\
92.32	0.01\\
92.33	0.01\\
92.34	0.01\\
92.35	0.01\\
92.36	0.01\\
92.37	0.01\\
92.38	0.01\\
92.39	0.01\\
92.4	0.01\\
92.41	0.01\\
92.42	0.01\\
92.43	0.01\\
92.44	0.01\\
92.45	0.01\\
92.46	0.01\\
92.47	0.01\\
92.48	0.01\\
92.49	0.01\\
92.5	0.01\\
92.51	0.01\\
92.52	0.01\\
92.53	0.01\\
92.54	0.01\\
92.55	0.01\\
92.56	0.01\\
92.57	0.01\\
92.58	0.01\\
92.59	0.01\\
92.6	0.01\\
92.61	0.01\\
92.62	0.01\\
92.63	0.01\\
92.64	0.01\\
92.65	0.01\\
92.66	0.01\\
92.67	0.01\\
92.68	0.01\\
92.69	0.01\\
92.7	0.01\\
92.71	0.01\\
92.72	0.01\\
92.73	0.01\\
92.74	0.01\\
92.75	0.01\\
92.76	0.01\\
92.77	0.01\\
92.78	0.01\\
92.79	0.01\\
92.8	0.01\\
92.81	0.01\\
92.82	0.01\\
92.83	0.01\\
92.84	0.01\\
92.85	0.01\\
92.86	0.01\\
92.87	0.01\\
92.88	0.01\\
92.89	0.01\\
92.9	0.01\\
92.91	0.01\\
92.92	0.01\\
92.93	0.01\\
92.94	0.01\\
92.95	0.01\\
92.96	0.01\\
92.97	0.01\\
92.98	0.01\\
92.99	0.01\\
93	0.01\\
93.01	0.01\\
93.02	0.01\\
93.03	0.01\\
93.04	0.01\\
93.05	0.01\\
93.06	0.01\\
93.07	0.01\\
93.08	0.01\\
93.09	0.01\\
93.1	0.01\\
93.11	0.01\\
93.12	0.01\\
93.13	0.01\\
93.14	0.01\\
93.15	0.01\\
93.16	0.01\\
93.17	0.01\\
93.18	0.01\\
93.19	0.01\\
93.2	0.01\\
93.21	0.01\\
93.22	0.01\\
93.23	0.01\\
93.24	0.01\\
93.25	0.01\\
93.26	0.01\\
93.27	0.01\\
93.28	0.01\\
93.29	0.01\\
93.3	0.01\\
93.31	0.01\\
93.32	0.01\\
93.33	0.01\\
93.34	0.01\\
93.35	0.01\\
93.36	0.01\\
93.37	0.01\\
93.38	0.01\\
93.39	0.01\\
93.4	0.01\\
93.41	0.01\\
93.42	0.01\\
93.43	0.01\\
93.44	0.01\\
93.45	0.01\\
93.46	0.01\\
93.47	0.01\\
93.48	0.01\\
93.49	0.01\\
93.5	0.01\\
93.51	0.01\\
93.52	0.01\\
93.53	0.01\\
93.54	0.01\\
93.55	0.01\\
93.56	0.01\\
93.57	0.01\\
93.58	0.01\\
93.59	0.01\\
93.6	0.01\\
93.61	0.01\\
93.62	0.01\\
93.63	0.01\\
93.64	0.01\\
93.65	0.01\\
93.66	0.01\\
93.67	0.01\\
93.68	0.01\\
93.69	0.01\\
93.7	0.01\\
93.71	0.01\\
93.72	0.01\\
93.73	0.01\\
93.74	0.01\\
93.75	0.01\\
93.76	0.01\\
93.77	0.01\\
93.78	0.01\\
93.79	0.01\\
93.8	0.01\\
93.81	0.01\\
93.82	0.01\\
93.83	0.01\\
93.84	0.01\\
93.85	0.01\\
93.86	0.01\\
93.87	0.01\\
93.88	0.01\\
93.89	0.01\\
93.9	0.01\\
93.91	0.01\\
93.92	0.01\\
93.93	0.01\\
93.94	0.01\\
93.95	0.01\\
93.96	0.01\\
93.97	0.01\\
93.98	0.01\\
93.99	0.01\\
94	0.01\\
94.01	0.01\\
94.02	0.01\\
94.03	0.01\\
94.04	0.01\\
94.05	0.01\\
94.06	0.01\\
94.07	0.01\\
94.08	0.01\\
94.09	0.01\\
94.1	0.01\\
94.11	0.01\\
94.12	0.01\\
94.13	0.01\\
94.14	0.01\\
94.15	0.01\\
94.16	0.01\\
94.17	0.01\\
94.18	0.01\\
94.19	0.01\\
94.2	0.01\\
94.21	0.01\\
94.22	0.01\\
94.23	0.01\\
94.24	0.01\\
94.25	0.01\\
94.26	0.01\\
94.27	0.01\\
94.28	0.01\\
94.29	0.01\\
94.3	0.01\\
94.31	0.01\\
94.32	0.01\\
94.33	0.01\\
94.34	0.01\\
94.35	0.01\\
94.36	0.01\\
94.37	0.01\\
94.38	0.01\\
94.39	0.01\\
94.4	0.01\\
94.41	0.01\\
94.42	0.01\\
94.43	0.01\\
94.44	0.01\\
94.45	0.01\\
94.46	0.01\\
94.47	0.01\\
94.48	0.01\\
94.49	0.01\\
94.5	0.01\\
94.51	0.01\\
94.52	0.01\\
94.53	0.01\\
94.54	0.01\\
94.55	0.01\\
94.56	0.01\\
94.57	0.01\\
94.58	0.01\\
94.59	0.01\\
94.6	0.01\\
94.61	0.01\\
94.62	0.01\\
94.63	0.01\\
94.64	0.01\\
94.65	0.01\\
94.66	0.01\\
94.67	0.01\\
94.68	0.01\\
94.69	0.01\\
94.7	0.01\\
94.71	0.01\\
94.72	0.01\\
94.73	0.01\\
94.74	0.01\\
94.75	0.01\\
94.76	0.01\\
94.77	0.01\\
94.78	0.01\\
94.79	0.01\\
94.8	0.01\\
94.81	0.01\\
94.82	0.01\\
94.83	0.01\\
94.84	0.01\\
94.85	0.01\\
94.86	0.01\\
94.87	0.01\\
94.88	0.01\\
94.89	0.01\\
94.9	0.01\\
94.91	0.01\\
94.92	0.01\\
94.93	0.01\\
94.94	0.01\\
94.95	0.01\\
94.96	0.01\\
94.97	0.01\\
94.98	0.01\\
94.99	0.01\\
95	0.01\\
95.01	0.01\\
95.02	0.01\\
95.03	0.01\\
95.04	0.01\\
95.05	0.01\\
95.06	0.01\\
95.07	0.01\\
95.08	0.01\\
95.09	0.01\\
95.1	0.01\\
95.11	0.01\\
95.12	0.01\\
95.13	0.01\\
95.14	0.01\\
95.15	0.01\\
95.16	0.01\\
95.17	0.01\\
95.18	0.01\\
95.19	0.01\\
95.2	0.01\\
95.21	0.01\\
95.22	0.01\\
95.23	0.01\\
95.24	0.01\\
95.25	0.01\\
95.26	0.01\\
95.27	0.01\\
95.28	0.01\\
95.29	0.01\\
95.3	0.01\\
95.31	0.01\\
95.32	0.01\\
95.33	0.01\\
95.34	0.01\\
95.35	0.01\\
95.36	0.01\\
95.37	0.01\\
95.38	0.01\\
95.39	0.01\\
95.4	0.01\\
95.41	0.01\\
95.42	0.01\\
95.43	0.01\\
95.44	0.01\\
95.45	0.01\\
95.46	0.01\\
95.47	0.01\\
95.48	0.01\\
95.49	0.01\\
95.5	0.01\\
95.51	0.01\\
95.52	0.01\\
95.53	0.01\\
95.54	0.01\\
95.55	0.01\\
95.56	0.01\\
95.57	0.01\\
95.58	0.01\\
95.59	0.01\\
95.6	0.01\\
95.61	0.01\\
95.62	0.01\\
95.63	0.01\\
95.64	0.01\\
95.65	0.01\\
95.66	0.01\\
95.67	0.01\\
95.68	0.01\\
95.69	0.01\\
95.7	0.01\\
95.71	0.01\\
95.72	0.01\\
95.73	0.01\\
95.74	0.01\\
95.75	0.01\\
95.76	0.01\\
95.77	0.01\\
95.78	0.01\\
95.79	0.01\\
95.8	0.01\\
95.81	0.01\\
95.82	0.01\\
95.83	0.01\\
95.84	0.01\\
95.85	0.01\\
95.86	0.01\\
95.87	0.01\\
95.88	0.01\\
95.89	0.01\\
95.9	0.01\\
95.91	0.01\\
95.92	0.01\\
95.93	0.01\\
95.94	0.01\\
95.95	0.01\\
95.96	0.01\\
95.97	0.01\\
95.98	0.01\\
95.99	0.01\\
96	0.01\\
96.01	0.01\\
96.02	0.01\\
96.03	0.01\\
96.04	0.01\\
96.05	0.01\\
96.06	0.01\\
96.07	0.01\\
96.08	0.01\\
96.09	0.01\\
96.1	0.01\\
96.11	0.01\\
96.12	0.01\\
96.13	0.01\\
96.14	0.01\\
96.15	0.01\\
96.16	0.01\\
96.17	0.01\\
96.18	0.01\\
96.19	0.01\\
96.2	0.01\\
96.21	0.01\\
96.22	0.01\\
96.23	0.01\\
96.24	0.01\\
96.25	0.01\\
96.26	0.01\\
96.27	0.01\\
96.28	0.01\\
96.29	0.01\\
96.3	0.01\\
96.31	0.01\\
96.32	0.01\\
96.33	0.01\\
96.34	0.01\\
96.35	0.01\\
96.36	0.01\\
96.37	0.01\\
96.38	0.01\\
96.39	0.01\\
96.4	0.01\\
96.41	0.01\\
96.42	0.01\\
96.43	0.01\\
96.44	0.01\\
96.45	0.01\\
96.46	0.01\\
96.47	0.01\\
96.48	0.01\\
96.49	0.01\\
96.5	0.01\\
96.51	0.01\\
96.52	0.01\\
96.53	0.01\\
96.54	0.01\\
96.55	0.01\\
96.56	0.01\\
96.57	0.01\\
96.58	0.01\\
96.59	0.01\\
96.6	0.01\\
96.61	0.01\\
96.62	0.01\\
96.63	0.01\\
96.64	0.01\\
96.65	0.01\\
96.66	0.01\\
96.67	0.01\\
96.68	0.01\\
96.69	0.01\\
96.7	0.01\\
96.71	0.01\\
96.72	0.01\\
96.73	0.01\\
96.74	0.01\\
96.75	0.01\\
96.76	0.01\\
96.77	0.01\\
96.78	0.01\\
96.79	0.01\\
96.8	0.01\\
96.81	0.01\\
96.82	0.01\\
96.83	0.01\\
96.84	0.01\\
96.85	0.01\\
96.86	0.01\\
96.87	0.01\\
96.88	0.01\\
96.89	0.01\\
96.9	0.01\\
96.91	0.01\\
96.92	0.01\\
96.93	0.01\\
96.94	0.01\\
96.95	0.01\\
96.96	0.01\\
96.97	0.01\\
96.98	0.01\\
96.99	0.01\\
97	0.01\\
97.01	0.01\\
97.02	0.01\\
97.03	0.01\\
97.04	0.01\\
97.05	0.01\\
97.06	0.01\\
97.07	0.01\\
97.08	0.01\\
97.09	0.01\\
97.1	0.01\\
97.11	0.01\\
97.12	0.01\\
97.13	0.01\\
97.14	0.01\\
97.15	0.01\\
97.16	0.01\\
97.17	0.01\\
97.18	0.01\\
97.19	0.01\\
97.2	0.01\\
97.21	0.01\\
97.22	0.01\\
97.23	0.01\\
97.24	0.01\\
97.25	0.01\\
97.26	0.01\\
97.27	0.01\\
97.28	0.01\\
97.29	0.01\\
97.3	0.01\\
97.31	0.01\\
97.32	0.01\\
97.33	0.01\\
97.34	0.01\\
97.35	0.01\\
97.36	0.01\\
97.37	0.01\\
97.38	0.01\\
97.39	0.01\\
97.4	0.01\\
97.41	0.01\\
97.42	0.01\\
97.43	0.01\\
97.44	0.01\\
97.45	0.01\\
97.46	0.01\\
97.47	0.01\\
97.48	0.01\\
97.49	0.01\\
97.5	0.01\\
97.51	0.01\\
97.52	0.01\\
97.53	0.01\\
97.54	0.01\\
97.55	0.01\\
97.56	0.01\\
97.57	0.01\\
97.58	0.01\\
97.59	0.01\\
97.6	0.01\\
97.61	0.01\\
97.62	0.01\\
97.63	0.01\\
97.64	0.01\\
97.65	0.01\\
97.66	0.01\\
97.67	0.01\\
97.68	0.01\\
97.69	0.01\\
97.7	0.01\\
97.71	0.01\\
97.72	0.01\\
97.73	0.01\\
97.74	0.01\\
97.75	0.01\\
97.76	0.01\\
97.77	0.01\\
97.78	0.01\\
97.79	0.01\\
97.8	0.01\\
97.81	0.01\\
97.82	0.01\\
97.83	0.01\\
97.84	0.01\\
97.85	0.01\\
97.86	0.01\\
97.87	0.01\\
97.88	0.01\\
97.89	0.01\\
97.9	0.01\\
97.91	0.01\\
97.92	0.01\\
97.93	0.01\\
97.94	0.01\\
97.95	0.01\\
97.96	0.01\\
97.97	0.01\\
97.98	0.01\\
97.99	0.01\\
98	0.01\\
98.01	0.01\\
98.02	0.01\\
98.03	0.01\\
98.04	0.01\\
98.05	0.01\\
98.06	0.01\\
98.07	0.01\\
98.08	0.01\\
98.09	0.01\\
98.1	0.01\\
98.11	0.01\\
98.12	0.01\\
98.13	0.01\\
98.14	0.01\\
98.15	0.01\\
98.16	0.01\\
98.17	0.01\\
98.18	0.01\\
98.19	0.01\\
98.2	0.01\\
98.21	0.01\\
98.22	0.01\\
98.23	0.01\\
98.24	0.01\\
98.25	0.01\\
98.26	0.01\\
98.27	0.01\\
98.28	0.01\\
98.29	0.01\\
98.3	0.01\\
98.31	0.01\\
98.32	0.01\\
98.33	0.01\\
98.34	0.01\\
98.35	0.01\\
98.36	0.01\\
98.37	0.01\\
98.38	0.01\\
98.39	0.01\\
98.4	0.01\\
98.41	0.01\\
98.42	0.01\\
98.43	0.01\\
98.44	0.01\\
98.45	0.01\\
98.46	0.01\\
98.47	0.01\\
98.48	0.01\\
98.49	0.01\\
98.5	0.01\\
98.51	0.01\\
98.52	0.01\\
98.53	0.01\\
98.54	0.01\\
98.55	0.01\\
98.56	0.01\\
98.57	0.01\\
98.58	0.01\\
98.59	0.01\\
98.6	0.01\\
98.61	0.01\\
98.62	0.01\\
98.63	0.01\\
98.64	0.01\\
98.65	0.01\\
98.66	0.01\\
98.67	0.01\\
98.68	0.01\\
98.69	0.01\\
98.7	0.01\\
98.71	0.01\\
98.72	0.01\\
98.73	0.01\\
98.74	0.01\\
98.75	0.01\\
98.76	0.01\\
98.77	0.01\\
98.78	0.01\\
98.79	0.01\\
98.8	0.01\\
98.81	0.01\\
98.82	0.01\\
98.83	0.01\\
98.84	0.01\\
98.85	0.01\\
98.86	0.01\\
98.87	0.01\\
98.88	0.01\\
98.89	0.01\\
98.9	0.01\\
98.91	0.01\\
98.92	0.01\\
98.93	0.01\\
98.94	0.01\\
98.95	0.01\\
98.96	0.01\\
98.97	0.01\\
98.98	0.01\\
98.99	0.01\\
99	0.01\\
99.01	0.01\\
99.02	0.01\\
99.03	0.01\\
99.04	0.01\\
99.05	0.01\\
99.06	0.01\\
99.07	0.01\\
99.08	0.01\\
99.09	0.01\\
99.1	0.01\\
99.11	0.01\\
99.12	0.01\\
99.13	0.01\\
99.14	0.01\\
99.15	0.01\\
99.16	0.01\\
99.17	0.01\\
99.18	0.01\\
99.19	0.01\\
99.2	0.01\\
99.21	0.01\\
99.22	0.01\\
99.23	0.01\\
99.24	0.01\\
99.25	0.01\\
99.26	0.01\\
99.27	0.01\\
99.28	0.01\\
99.29	0.01\\
99.3	0.01\\
99.31	0.01\\
99.32	0.01\\
99.33	0.01\\
99.34	0.01\\
99.35	0.01\\
99.36	0.01\\
99.37	0.01\\
99.38	0.01\\
99.39	0.01\\
99.4	0.01\\
99.41	0.01\\
99.42	0.01\\
99.43	0.01\\
99.44	0.01\\
99.45	0.01\\
99.46	0.01\\
99.47	0.01\\
99.48	0.01\\
99.49	0.01\\
99.5	0.01\\
99.51	0.01\\
99.52	0.01\\
99.53	0.01\\
99.54	0.01\\
99.55	0.01\\
99.56	0.01\\
99.57	0.01\\
99.58	0.01\\
99.59	0.01\\
99.6	0.01\\
99.61	0.01\\
99.62	0.01\\
99.63	0.01\\
99.64	0.01\\
99.65	0.01\\
99.66	0.01\\
99.67	0.01\\
99.68	0.01\\
99.69	0.01\\
99.7	0.01\\
99.71	0.01\\
99.72	0.01\\
99.73	0.01\\
99.74	0.01\\
99.75	0.01\\
99.76	0.01\\
99.77	0.01\\
99.78	0.01\\
99.79	0.01\\
99.8	0.01\\
99.81	0.01\\
99.82	0.01\\
99.83	0.01\\
99.84	0.01\\
99.85	0.01\\
99.86	0.01\\
99.87	0.01\\
99.88	0.01\\
99.89	0.01\\
99.9	0.01\\
99.91	0.01\\
99.92	0.01\\
99.93	0.01\\
99.94	0.01\\
99.95	0.01\\
99.96	0.01\\
99.97	0.01\\
99.98	0.01\\
99.99	0.01\\
100	0.01\\
};
\addlegendentry{$q=2$};

\addplot [color=mycolor1,solid,forget plot]
  table[row sep=crcr]{%
0.01	0.00956306055999642\\
0.02	0.00956307872228478\\
0.03	0.0095630968976805\\
0.04	0.00956311508582356\\
0.05	0.00956313328633\\
0.06	0.00956315149879088\\
0.07	0.00956316972277116\\
0.08	0.00956318795780866\\
0.09	0.00956320620341279\\
0.1	0.00956322445906338\\
0.11	0.00956324272420942\\
0.12	0.00956326099826776\\
0.13	0.0095632792806217\\
0.14	0.00956329757061965\\
0.15	0.0095633158675736\\
0.16	0.00956333417075766\\
0.17	0.00956335247940644\\
0.18	0.00956337079271344\\
0.19	0.00956338910982933\\
0.2	0.00956340742986023\\
0.21	0.00956342575186585\\
0.22	0.00956344407485762\\
0.23	0.0095634623977967\\
0.24	0.00956348071959199\\
0.25	0.00956349903909798\\
0.26	0.00956351735511256\\
0.27	0.00956353566637479\\
0.28	0.00956355397156251\\
0.29	0.00956357226928993\\
0.3	0.00956359055810508\\
0.31	0.00956360883648718\\
0.32	0.00956362710284395\\
0.33	0.0095636453555088\\
0.34	0.00956366359273788\\
0.35	0.00956368181270707\\
0.36	0.00956370001350885\\
0.37	0.00956371819314906\\
0.38	0.00956373634954351\\
0.39	0.0095637544805145\\
0.4	0.00956377258378723\\
0.41	0.00956379065698601\\
0.42	0.00956380869763044\\
0.43	0.00956382670313134\\
0.44	0.0095638446707866\\
0.45	0.00956386259777691\\
0.46	0.0095638804811612\\
0.47	0.00956389831787212\\
0.48	0.00956391610471117\\
0.49	0.00956393383834377\\
0.5	0.00956395151529413\\
0.51	0.00956396913193986\\
0.52	0.00956398668450656\\
0.53	0.00956400416906202\\
0.54	0.00956402158151035\\
0.55	0.00956403891758586\\
0.56	0.00956405617284671\\
0.57	0.00956407334266835\\
0.58	0.0095640904222367\\
0.59	0.00956410740654118\\
0.6	0.00956412429036737\\
0.61	0.00956414106828947\\
0.62	0.00956415773466253\\
0.63	0.00956417428361436\\
0.64	0.00956419070903714\\
0.65	0.0095642070045788\\
0.66	0.00956422316363404\\
0.67	0.00956423917933505\\
0.68	0.00956425504454195\\
0.69	0.00956427075183281\\
0.7	0.0095642862934934\\
0.71	0.00956430166150654\\
0.72	0.0095643168475411\\
0.73	0.0095643318429406\\
0.74	0.00956434673458947\\
0.75	0.0095643616319118\\
0.76	0.00956437653491001\\
0.77	0.0095643914435865\\
0.78	0.00956440635794366\\
0.79	0.00956442127798393\\
0.8	0.00956443620370971\\
0.81	0.00956445113512341\\
0.82	0.00956446607222744\\
0.83	0.00956448101502423\\
0.84	0.00956449596351619\\
0.85	0.00956451091770574\\
0.86	0.00956452587759529\\
0.87	0.00956454084318728\\
0.88	0.00956455581448411\\
0.89	0.00956457079148822\\
0.9	0.00956458577420203\\
0.91	0.00956460076262797\\
0.92	0.00956461575676846\\
0.93	0.00956463075662593\\
0.94	0.00956464576220282\\
0.95	0.00956466077350155\\
0.96	0.00956467579052456\\
0.97	0.00956469081327427\\
0.98	0.00956470584175314\\
0.99	0.00956472087596358\\
1	0.00956473591590805\\
1.01	0.00956475096158899\\
1.02	0.00956476601300882\\
1.03	0.00956478107017\\
1.04	0.00956479613307496\\
1.05	0.00956481120172616\\
1.06	0.00956482627612604\\
1.07	0.00956484135627704\\
1.08	0.00956485644218161\\
1.09	0.00956487153384222\\
1.1	0.00956488663126129\\
1.11	0.0095649017344413\\
1.12	0.00956491684338469\\
1.13	0.00956493195809392\\
1.14	0.00956494707857144\\
1.15	0.00956496220481972\\
1.16	0.00956497733684121\\
1.17	0.00956499247463837\\
1.18	0.00956500761821367\\
1.19	0.00956502276756957\\
1.2	0.00956503792270853\\
1.21	0.00956505308363303\\
1.22	0.00956506825034552\\
1.23	0.00956508342284848\\
1.24	0.00956509860114438\\
1.25	0.00956511378523568\\
1.26	0.00956512897512487\\
1.27	0.00956514417081441\\
1.28	0.00956515937230679\\
1.29	0.00956517457960447\\
1.3	0.00956518979270995\\
1.31	0.00956520501162568\\
1.32	0.00956522023635417\\
1.33	0.00956523546689788\\
1.34	0.00956525070325931\\
1.35	0.00956526594544094\\
1.36	0.00956528119344526\\
1.37	0.00956529644727475\\
1.38	0.0095653117069319\\
1.39	0.00956532697241921\\
1.4	0.00956534224373916\\
1.41	0.00956535752089425\\
1.42	0.00956537280388698\\
1.43	0.00956538809271983\\
1.44	0.00956540338739532\\
1.45	0.00956541868791592\\
1.46	0.00956543399428416\\
1.47	0.00956544930650252\\
1.48	0.00956546462457352\\
1.49	0.00956547994849964\\
1.5	0.00956549527828341\\
1.51	0.00956551061392733\\
1.52	0.00956552595543391\\
1.53	0.00956554130280565\\
1.54	0.00956555665604507\\
1.55	0.00956557201515468\\
1.56	0.00956558738013699\\
1.57	0.00956560275099453\\
1.58	0.00956561812772979\\
1.59	0.00956563351034532\\
1.6	0.00956564889884362\\
1.61	0.00956566429322722\\
1.62	0.00956567969349863\\
1.63	0.00956569509966039\\
1.64	0.00956571051171501\\
1.65	0.00956572592966502\\
1.66	0.00956574135351296\\
1.67	0.00956575678326135\\
1.68	0.00956577221891272\\
1.69	0.00956578766046959\\
1.7	0.00956580310793452\\
1.71	0.00956581856131003\\
1.72	0.00956583402059866\\
1.73	0.00956584948580295\\
1.74	0.00956586495692542\\
1.75	0.00956588043396864\\
1.76	0.00956589591693513\\
1.77	0.00956591140582744\\
1.78	0.00956592690064813\\
1.79	0.00956594240139972\\
1.8	0.00956595790808477\\
1.81	0.00956597342070583\\
1.82	0.00956598893926545\\
1.83	0.00956600446376617\\
1.84	0.00956601999421057\\
1.85	0.00956603553060118\\
1.86	0.00956605107294057\\
1.87	0.00956606662123129\\
1.88	0.0095660821754759\\
1.89	0.00956609773567696\\
1.9	0.00956611330183704\\
1.91	0.0095661288739587\\
1.92	0.0095661444520445\\
1.93	0.009566160036097\\
1.94	0.00956617562611878\\
1.95	0.0095661912221124\\
1.96	0.00956620682408043\\
1.97	0.00956622243202546\\
1.98	0.00956623804595004\\
1.99	0.00956625366585676\\
2	0.00956626929174819\\
2.01	0.0095662849236269\\
2.02	0.00956630056149548\\
2.03	0.00956631620535651\\
2.04	0.00956633185521257\\
2.05	0.00956634751106624\\
2.06	0.00956636317292011\\
2.07	0.00956637884077676\\
2.08	0.00956639451463878\\
2.09	0.00956641019450876\\
2.1	0.0095664258803893\\
2.11	0.00956644157228298\\
2.12	0.00956645727019239\\
2.13	0.00956647297412013\\
2.14	0.0095664886840688\\
2.15	0.009566504400041\\
2.16	0.00956652012203932\\
2.17	0.00956653585006636\\
2.18	0.00956655158412473\\
2.19	0.00956656732421704\\
2.2	0.00956658307034587\\
2.21	0.00956659882251385\\
2.22	0.00956661458072358\\
2.23	0.00956663034497768\\
2.24	0.00956664611527874\\
2.25	0.00956666189162939\\
2.26	0.00956667767403223\\
2.27	0.00956669346248988\\
2.28	0.00956670925700497\\
2.29	0.0095667250575801\\
2.3	0.0095667408642179\\
2.31	0.00956675667692099\\
2.32	0.00956677249569199\\
2.33	0.00956678832053352\\
2.34	0.00956680415144822\\
2.35	0.0095668199884387\\
2.36	0.0095668358315076\\
2.37	0.00956685168065754\\
2.38	0.00956686753589117\\
2.39	0.0095668833972111\\
2.4	0.00956689926461999\\
2.41	0.00956691513812045\\
2.42	0.00956693101771514\\
2.43	0.00956694690340668\\
2.44	0.00956696279519772\\
2.45	0.00956697869309091\\
2.46	0.00956699459708887\\
2.47	0.00956701050719427\\
2.48	0.00956702642340974\\
2.49	0.00956704234573793\\
2.5	0.0095670582741815\\
2.51	0.0095670742087431\\
2.52	0.00956709014942537\\
2.53	0.00956710609623097\\
2.54	0.00956712204916256\\
2.55	0.00956713800822278\\
2.56	0.00956715397341432\\
2.57	0.00956716994473981\\
2.58	0.00956718592220192\\
2.59	0.00956720190580332\\
2.6	0.00956721789554667\\
2.61	0.00956723389143464\\
2.62	0.00956724989346989\\
2.63	0.00956726590165509\\
2.64	0.00956728191599292\\
2.65	0.00956729793648605\\
2.66	0.00956731396313714\\
2.67	0.00956732999594887\\
2.68	0.00956734603492393\\
2.69	0.00956736208006499\\
2.7	0.00956737813137472\\
2.71	0.00956739418885581\\
2.72	0.00956741025251095\\
2.73	0.00956742632234281\\
2.74	0.00956744239835409\\
2.75	0.00956745848054746\\
2.76	0.00956747456892562\\
2.77	0.00956749066349126\\
2.78	0.00956750676424708\\
2.79	0.00956752287119575\\
2.8	0.00956753898433999\\
2.81	0.00956755510368248\\
2.82	0.00956757122922592\\
2.83	0.00956758736097302\\
2.84	0.00956760349892647\\
2.85	0.00956761964308898\\
2.86	0.00956763579346325\\
2.87	0.00956765195005198\\
2.88	0.00956766811285789\\
2.89	0.00956768428188369\\
2.9	0.00956770045713207\\
2.91	0.00956771663860576\\
2.92	0.00956773282630747\\
2.93	0.00956774902023991\\
2.94	0.0095677652204058\\
2.95	0.00956778142680787\\
2.96	0.00956779763944881\\
2.97	0.00956781385833137\\
2.98	0.00956783008345826\\
2.99	0.00956784631483221\\
3	0.00956786255245593\\
3.01	0.00956787879633217\\
3.02	0.00956789504646365\\
3.03	0.0095679113028531\\
3.04	0.00956792756550324\\
3.05	0.00956794383441682\\
3.06	0.00956796010959658\\
3.07	0.00956797639104524\\
3.08	0.00956799267876554\\
3.09	0.00956800897276024\\
3.1	0.00956802527303205\\
3.11	0.00956804157958375\\
3.12	0.00956805789241805\\
3.13	0.00956807421153772\\
3.14	0.0095680905369455\\
3.15	0.00956810686864414\\
3.16	0.00956812320663639\\
3.17	0.009568139550925\\
3.18	0.00956815590151273\\
3.19	0.00956817225840233\\
3.2	0.00956818862159656\\
3.21	0.00956820499109819\\
3.22	0.00956822136690996\\
3.23	0.00956823774903464\\
3.24	0.009568254137475\\
3.25	0.0095682705322338\\
3.26	0.00956828693331382\\
3.27	0.0095683033407178\\
3.28	0.00956831975444854\\
3.29	0.0095683361745088\\
3.3	0.00956835260090134\\
3.31	0.00956836903362896\\
3.32	0.00956838547269442\\
3.33	0.0095684019181005\\
3.34	0.00956841836984998\\
3.35	0.00956843482794565\\
3.36	0.00956845129239029\\
3.37	0.00956846776318667\\
3.38	0.00956848424033759\\
3.39	0.00956850072384584\\
3.4	0.00956851721371421\\
3.41	0.00956853370994548\\
3.42	0.00956855021254244\\
3.43	0.00956856672150791\\
3.44	0.00956858323684465\\
3.45	0.00956859975855549\\
3.46	0.00956861628664321\\
3.47	0.00956863282111062\\
3.48	0.00956864936196052\\
3.49	0.00956866590919571\\
3.5	0.009568682462819\\
3.51	0.0095686990228332\\
3.52	0.00956871558924111\\
3.53	0.00956873216204555\\
3.54	0.00956874874124933\\
3.55	0.00956876532685526\\
3.56	0.00956878191886615\\
3.57	0.00956879851728483\\
3.58	0.00956881512211412\\
3.59	0.00956883173335682\\
3.6	0.00956884835101578\\
3.61	0.0095688649750938\\
3.62	0.00956888160559372\\
3.63	0.00956889824251836\\
3.64	0.00956891488587055\\
3.65	0.00956893153565313\\
3.66	0.00956894819186892\\
3.67	0.00956896485452075\\
3.68	0.00956898152361146\\
3.69	0.0095689981991439\\
3.7	0.0095690148811209\\
3.71	0.00956903156954529\\
3.72	0.00956904826441992\\
3.73	0.00956906496574763\\
3.74	0.00956908167353128\\
3.75	0.0095690983877737\\
3.76	0.00956911510847774\\
3.77	0.00956913183564626\\
3.78	0.00956914856928211\\
3.79	0.00956916530938814\\
3.8	0.0095691820559672\\
3.81	0.00956919880902216\\
3.82	0.00956921556855586\\
3.83	0.00956923233457118\\
3.84	0.00956924910707097\\
3.85	0.0095692658860581\\
3.86	0.00956928267153543\\
3.87	0.00956929946350583\\
3.88	0.00956931626197217\\
3.89	0.00956933306693732\\
3.9	0.00956934987840414\\
3.91	0.00956936669637553\\
3.92	0.00956938352085434\\
3.93	0.00956940035184346\\
3.94	0.00956941718934577\\
3.95	0.00956943403336414\\
3.96	0.00956945088390146\\
3.97	0.00956946774096062\\
3.98	0.00956948460454449\\
3.99	0.00956950147465597\\
4	0.00956951835129794\\
4.01	0.0095695352344733\\
4.02	0.00956955212418494\\
4.03	0.00956956902043575\\
4.04	0.00956958592322863\\
4.05	0.00956960283256647\\
4.06	0.00956961974845218\\
4.07	0.00956963667088866\\
4.08	0.00956965359987881\\
4.09	0.00956967053542552\\
4.1	0.00956968747753172\\
4.11	0.00956970442620031\\
4.12	0.00956972138143419\\
4.13	0.00956973834323628\\
4.14	0.00956975531160949\\
4.15	0.00956977228655674\\
4.16	0.00956978926808094\\
4.17	0.00956980625618501\\
4.18	0.00956982325087187\\
4.19	0.00956984025214445\\
4.2	0.00956985726000566\\
4.21	0.00956987427445843\\
4.22	0.0095698912955057\\
4.23	0.00956990832315037\\
4.24	0.00956992535739539\\
4.25	0.0095699423982437\\
4.26	0.00956995944569822\\
4.27	0.00956997649976188\\
4.28	0.00956999356043763\\
4.29	0.00957001062772841\\
4.3	0.00957002770163715\\
4.31	0.00957004478216681\\
4.32	0.00957006186932031\\
4.33	0.00957007896310061\\
4.34	0.00957009606351066\\
4.35	0.00957011317055341\\
4.36	0.0095701302842318\\
4.37	0.0095701474045488\\
4.38	0.00957016453150735\\
4.39	0.00957018166511041\\
4.4	0.00957019880536095\\
4.41	0.00957021595226191\\
4.42	0.00957023310581627\\
4.43	0.00957025026602699\\
4.44	0.00957026743289703\\
4.45	0.00957028460642936\\
4.46	0.00957030178662695\\
4.47	0.00957031897349276\\
4.48	0.00957033616702978\\
4.49	0.00957035336724097\\
4.5	0.00957037057412932\\
4.51	0.0095703877876978\\
4.52	0.00957040500794939\\
4.53	0.00957042223488707\\
4.54	0.00957043946851382\\
4.55	0.00957045670883264\\
4.56	0.0095704739558465\\
4.57	0.0095704912095584\\
4.58	0.00957050846997132\\
4.59	0.00957052573708826\\
4.6	0.00957054301091221\\
4.61	0.00957056029144618\\
4.62	0.00957057757869315\\
4.63	0.00957059487265612\\
4.64	0.0095706121733381\\
4.65	0.00957062948074209\\
4.66	0.00957064679487109\\
4.67	0.00957066411572811\\
4.68	0.00957068144331617\\
4.69	0.00957069877763827\\
4.7	0.00957071611869741\\
4.71	0.00957073346649662\\
4.72	0.00957075082103892\\
4.73	0.00957076818232731\\
4.74	0.00957078555036482\\
4.75	0.00957080292515446\\
4.76	0.00957082030669928\\
4.77	0.00957083769500228\\
4.78	0.00957085509006649\\
4.79	0.00957087249189495\\
4.8	0.00957088990049068\\
4.81	0.00957090731585671\\
4.82	0.00957092473799608\\
4.83	0.00957094216691183\\
4.84	0.00957095960260699\\
4.85	0.0095709770450846\\
4.86	0.00957099449434771\\
4.87	0.00957101195039935\\
4.88	0.00957102941324257\\
4.89	0.00957104688288042\\
4.9	0.00957106435931595\\
4.91	0.0095710818425522\\
4.92	0.00957109933259224\\
4.93	0.00957111682943911\\
4.94	0.00957113433309587\\
4.95	0.00957115184356557\\
4.96	0.00957116936085129\\
4.97	0.00957118688495607\\
4.98	0.00957120441588298\\
4.99	0.0095712219536351\\
5	0.00957123949821548\\
5.01	0.00957125704962719\\
5.02	0.00957127460787332\\
5.03	0.00957129217295692\\
5.04	0.00957130974488107\\
5.05	0.00957132732364885\\
5.06	0.00957134490926334\\
5.07	0.00957136250172762\\
5.08	0.00957138010104477\\
5.09	0.00957139770721788\\
5.1	0.00957141532025002\\
5.11	0.00957143294014429\\
5.12	0.00957145056690379\\
5.13	0.00957146820053159\\
5.14	0.00957148584103079\\
5.15	0.00957150348840449\\
5.16	0.00957152114265579\\
5.17	0.00957153880378779\\
5.18	0.00957155647180358\\
5.19	0.00957157414670627\\
5.2	0.00957159182849897\\
5.21	0.00957160951718477\\
5.22	0.0095716272127668\\
5.23	0.00957164491524816\\
5.24	0.00957166262463195\\
5.25	0.0095716803409213\\
5.26	0.00957169806411933\\
5.27	0.00957171579422915\\
5.28	0.00957173353125388\\
5.29	0.00957175127519664\\
5.3	0.00957176902606056\\
5.31	0.00957178678384876\\
5.32	0.00957180454856438\\
5.33	0.00957182232021053\\
5.34	0.00957184009879035\\
5.35	0.00957185788430699\\
5.36	0.00957187567676356\\
5.37	0.00957189347616321\\
5.38	0.00957191128250908\\
5.39	0.00957192909580431\\
5.4	0.00957194691605204\\
5.41	0.00957196474325541\\
5.42	0.00957198257741758\\
5.43	0.00957200041854169\\
5.44	0.0095720182666309\\
5.45	0.00957203612168835\\
5.46	0.00957205398371721\\
5.47	0.00957207185272063\\
5.48	0.00957208972870176\\
5.49	0.00957210761166376\\
5.5	0.00957212550160981\\
5.51	0.00957214339854306\\
5.52	0.00957216130246669\\
5.53	0.00957217921338385\\
5.54	0.00957219713129773\\
5.55	0.00957221505621148\\
5.56	0.00957223298812829\\
5.57	0.00957225092705134\\
5.58	0.0095722688729838\\
5.59	0.00957228682592884\\
5.6	0.00957230478588966\\
5.61	0.00957232275286944\\
5.62	0.00957234072687137\\
5.63	0.00957235870789862\\
5.64	0.0095723766959544\\
5.65	0.00957239469104189\\
5.66	0.00957241269316429\\
5.67	0.00957243070232479\\
5.68	0.0095724487185266\\
5.69	0.0095724667417729\\
5.7	0.00957248477206691\\
5.71	0.00957250280941183\\
5.72	0.00957252085381086\\
5.73	0.00957253890526721\\
5.74	0.0095725569637841\\
5.75	0.00957257502936473\\
5.76	0.00957259310201232\\
5.77	0.00957261118173008\\
5.78	0.00957262926852123\\
5.79	0.00957264736238899\\
5.8	0.00957266546333659\\
5.81	0.00957268357136724\\
5.82	0.00957270168648417\\
5.83	0.00957271980869062\\
5.84	0.00957273793798981\\
5.85	0.00957275607438497\\
5.86	0.00957277421787934\\
5.87	0.00957279236847615\\
5.88	0.00957281052617864\\
5.89	0.00957282869099005\\
5.9	0.00957284686291363\\
5.91	0.00957286504195261\\
5.92	0.00957288322811025\\
5.93	0.00957290142138979\\
5.94	0.00957291962179448\\
5.95	0.00957293782932757\\
5.96	0.00957295604399232\\
5.97	0.00957297426579199\\
5.98	0.00957299249472983\\
5.99	0.0095730107308091\\
6	0.00957302897403307\\
6.01	0.009573047224405\\
6.02	0.00957306548192816\\
6.03	0.00957308374660581\\
6.04	0.00957310201844123\\
6.05	0.00957312029743768\\
6.06	0.00957313858359845\\
6.07	0.00957315687692681\\
6.08	0.00957317517742604\\
6.09	0.00957319348509942\\
6.1	0.00957321179995024\\
6.11	0.00957323012198176\\
6.12	0.0095732484511973\\
6.13	0.00957326678760013\\
6.14	0.00957328513119354\\
6.15	0.00957330348198084\\
6.16	0.0095733218399653\\
6.17	0.00957334020515025\\
6.18	0.00957335857753895\\
6.19	0.00957337695713474\\
6.2	0.0095733953439409\\
6.21	0.00957341373796074\\
6.22	0.00957343213919758\\
6.23	0.00957345054765471\\
6.24	0.00957346896333546\\
6.25	0.00957348738624313\\
6.26	0.00957350581638105\\
6.27	0.00957352425375252\\
6.28	0.00957354269836088\\
6.29	0.00957356115020944\\
6.3	0.00957357960930153\\
6.31	0.00957359807564048\\
6.32	0.00957361654922961\\
6.33	0.00957363503007226\\
6.34	0.00957365351817175\\
6.35	0.00957367201353143\\
6.36	0.00957369051615463\\
6.37	0.00957370902604469\\
6.38	0.00957372754320495\\
6.39	0.00957374606763876\\
6.4	0.00957376459934945\\
6.41	0.00957378313834039\\
6.42	0.00957380168461492\\
6.43	0.00957382023817639\\
6.44	0.00957383879902815\\
6.45	0.00957385736717357\\
6.46	0.00957387594261599\\
6.47	0.00957389452535879\\
6.48	0.00957391311540532\\
6.49	0.00957393171275895\\
6.5	0.00957395031742305\\
6.51	0.00957396892940098\\
6.52	0.00957398754869612\\
6.53	0.00957400617531184\\
6.54	0.00957402480925152\\
6.55	0.00957404345051853\\
6.56	0.00957406209911626\\
6.57	0.00957408075504808\\
6.58	0.00957409941831738\\
6.59	0.00957411808892755\\
6.6	0.00957413676688198\\
6.61	0.00957415545218405\\
6.62	0.00957417414483716\\
6.63	0.00957419284484471\\
6.64	0.00957421155221009\\
6.65	0.0095742302669367\\
6.66	0.00957424898902794\\
6.67	0.00957426771848722\\
6.68	0.00957428645531794\\
6.69	0.00957430519952352\\
6.7	0.00957432395110736\\
6.71	0.00957434271007287\\
6.72	0.00957436147642347\\
6.73	0.00957438025016258\\
6.74	0.00957439903129361\\
6.75	0.00957441781981999\\
6.76	0.00957443661574515\\
6.77	0.00957445541907249\\
6.78	0.00957447422980546\\
6.79	0.00957449304794749\\
6.8	0.009574511873502\\
6.81	0.00957453070647244\\
6.82	0.00957454954686223\\
6.83	0.00957456839467482\\
6.84	0.00957458724991364\\
6.85	0.00957460611258216\\
6.86	0.00957462498268379\\
6.87	0.009574643860222\\
6.88	0.00957466274520024\\
6.89	0.00957468163762196\\
6.9	0.0095747005374906\\
6.91	0.00957471944480964\\
6.92	0.00957473835958252\\
6.93	0.00957475728181272\\
6.94	0.00957477621150368\\
6.95	0.00957479514865889\\
6.96	0.00957481409328179\\
6.97	0.00957483304537588\\
6.98	0.00957485200494462\\
6.99	0.00957487097199148\\
7	0.00957488994651995\\
7.01	0.00957490892853349\\
7.02	0.00957492791803559\\
7.03	0.00957494691502974\\
7.04	0.00957496591951941\\
7.05	0.00957498493150811\\
7.06	0.00957500395099932\\
7.07	0.00957502297799653\\
7.08	0.00957504201250324\\
7.09	0.00957506105452294\\
7.1	0.00957508010405914\\
7.11	0.00957509916111533\\
7.12	0.00957511822569502\\
7.13	0.00957513729780171\\
7.14	0.00957515637743893\\
7.15	0.00957517546461017\\
7.16	0.00957519455931895\\
7.17	0.00957521366156879\\
7.18	0.00957523277136319\\
7.19	0.00957525188870569\\
7.2	0.00957527101359981\\
7.21	0.00957529014604906\\
7.22	0.00957530928605698\\
7.23	0.0095753284336271\\
7.24	0.00957534758876294\\
7.25	0.00957536675146805\\
7.26	0.00957538592174596\\
7.27	0.0095754050996002\\
7.28	0.00957542428503431\\
7.29	0.00957544347805185\\
7.3	0.00957546267865635\\
7.31	0.00957548188685137\\
7.32	0.00957550110264044\\
7.33	0.00957552032602714\\
7.34	0.009575539557015\\
7.35	0.0095755587956076\\
7.36	0.00957557804180848\\
7.37	0.0095755972956212\\
7.38	0.00957561655704934\\
7.39	0.00957563582609646\\
7.4	0.00957565510276613\\
7.41	0.00957567438706192\\
7.42	0.00957569367898739\\
7.43	0.00957571297854614\\
7.44	0.00957573228574173\\
7.45	0.00957575160057774\\
7.46	0.00957577092305777\\
7.47	0.00957579025318539\\
7.48	0.00957580959096419\\
7.49	0.00957582893639776\\
7.5	0.0095758482894897\\
7.51	0.0095758676502436\\
7.52	0.00957588701866305\\
7.53	0.00957590639475166\\
7.54	0.00957592577851302\\
7.55	0.00957594516995074\\
7.56	0.00957596456906843\\
7.57	0.00957598397586969\\
7.58	0.00957600339035814\\
7.59	0.00957602281253739\\
7.6	0.00957604224241106\\
7.61	0.00957606167998276\\
7.62	0.00957608112525611\\
7.63	0.00957610057823475\\
7.64	0.00957612003892228\\
7.65	0.00957613950732236\\
7.66	0.00957615898343859\\
7.67	0.00957617846727462\\
7.68	0.00957619795883407\\
7.69	0.00957621745812059\\
7.7	0.00957623696513783\\
7.71	0.00957625647988941\\
7.72	0.00957627600237898\\
7.73	0.0095762955326102\\
7.74	0.00957631507058671\\
7.75	0.00957633461631215\\
7.76	0.0095763541697902\\
7.77	0.0095763737310245\\
7.78	0.00957639330001872\\
7.79	0.00957641287677651\\
7.8	0.00957643246130154\\
7.81	0.00957645205359747\\
7.82	0.00957647165366799\\
7.83	0.00957649126151674\\
7.84	0.00957651087714742\\
7.85	0.00957653050056369\\
7.86	0.00957655013176924\\
7.87	0.00957656977076775\\
7.88	0.00957658941756289\\
7.89	0.00957660907215836\\
7.9	0.00957662873455785\\
7.91	0.00957664840476504\\
7.92	0.00957666808278363\\
7.93	0.00957668776861731\\
7.94	0.00957670746226979\\
7.95	0.00957672716374477\\
7.96	0.00957674687304594\\
7.97	0.00957676659017701\\
7.98	0.0095767863151417\\
7.99	0.00957680604794371\\
8	0.00957682578858675\\
8.01	0.00957684553707455\\
8.02	0.00957686529341082\\
8.03	0.00957688505759927\\
8.04	0.00957690482964364\\
8.05	0.00957692460954765\\
8.06	0.00957694439731502\\
8.07	0.0095769641929495\\
8.08	0.0095769839964548\\
8.09	0.00957700380783468\\
8.1	0.00957702362709285\\
8.11	0.00957704345423308\\
8.12	0.00957706328925909\\
8.13	0.00957708313217463\\
8.14	0.00957710298298346\\
8.15	0.00957712284168932\\
8.16	0.00957714270829597\\
8.17	0.00957716258280716\\
8.18	0.00957718246522665\\
8.19	0.00957720235555821\\
8.2	0.00957722225380558\\
8.21	0.00957724215997255\\
8.22	0.00957726207406288\\
8.23	0.00957728199608034\\
8.24	0.0095773019260287\\
8.25	0.00957732186391173\\
8.26	0.00957734180973323\\
8.27	0.00957736176349696\\
8.28	0.00957738172520672\\
8.29	0.00957740169486629\\
8.3	0.00957742167247945\\
8.31	0.00957744165805\\
8.32	0.00957746165158174\\
8.33	0.00957748165307845\\
8.34	0.00957750166254394\\
8.35	0.00957752167998201\\
8.36	0.00957754170539646\\
8.37	0.0095775617387911\\
8.38	0.00957758178016974\\
8.39	0.0095776018295362\\
8.4	0.00957762188689427\\
8.41	0.00957764195224779\\
8.42	0.00957766202560057\\
8.43	0.00957768210695644\\
8.44	0.00957770219631921\\
8.45	0.00957772229369272\\
8.46	0.00957774239908079\\
8.47	0.00957776251248727\\
8.48	0.00957778263391597\\
8.49	0.00957780276337075\\
8.5	0.00957782290085543\\
8.51	0.00957784304637387\\
8.52	0.00957786319992991\\
8.53	0.0095778833615274\\
8.54	0.00957790353117018\\
8.55	0.00957792370886211\\
8.56	0.00957794389460705\\
8.57	0.00957796408840886\\
8.58	0.00957798429027139\\
8.59	0.00957800450019852\\
8.6	0.0095780247181941\\
8.61	0.009578044944262\\
8.62	0.0095780651784061\\
8.63	0.00957808542063027\\
8.64	0.0095781056709384\\
8.65	0.00957812592933434\\
8.66	0.009578146195822\\
8.67	0.00957816647040526\\
8.68	0.00957818675308799\\
8.69	0.0095782070438741\\
8.7	0.00957822734276748\\
8.71	0.00957824764977201\\
8.72	0.0095782679648916\\
8.73	0.00957828828813015\\
8.74	0.00957830861949156\\
8.75	0.00957832895897974\\
8.76	0.0095783493065986\\
8.77	0.00957836966235204\\
8.78	0.00957839002624399\\
8.79	0.00957841039827836\\
8.8	0.00957843077845906\\
8.81	0.00957845116679003\\
8.82	0.00957847156327517\\
8.83	0.00957849196791843\\
8.84	0.00957851238072374\\
8.85	0.00957853280169501\\
8.86	0.0095785532308362\\
8.87	0.00957857366815123\\
8.88	0.00957859411364406\\
8.89	0.00957861456731861\\
8.9	0.00957863502917884\\
8.91	0.0095786554992287\\
8.92	0.00957867597747214\\
8.93	0.00957869646391311\\
8.94	0.00957871695855558\\
8.95	0.00957873746140349\\
8.96	0.00957875797246081\\
8.97	0.00957877849173151\\
8.98	0.00957879901921955\\
8.99	0.00957881955492891\\
9	0.00957884009886356\\
9.01	0.00957886065102747\\
9.02	0.00957888121142462\\
9.03	0.009578901780059\\
9.04	0.00957892235693458\\
9.05	0.00957894294205536\\
9.06	0.00957896353542533\\
9.07	0.00957898413704847\\
9.08	0.00957900474692878\\
9.09	0.00957902536507026\\
9.1	0.0095790459914769\\
9.11	0.00957906662615273\\
9.12	0.00957908726910172\\
9.13	0.00957910792032791\\
9.14	0.0095791285798353\\
9.15	0.00957914924762789\\
9.16	0.00957916992370972\\
9.17	0.00957919060808479\\
9.18	0.00957921130075714\\
9.19	0.00957923200173078\\
9.2	0.00957925271100975\\
9.21	0.00957927342859807\\
9.22	0.00957929415449978\\
9.23	0.00957931488871892\\
9.24	0.00957933563125951\\
9.25	0.00957935638212562\\
9.26	0.00957937714132127\\
9.27	0.00957939790885052\\
9.28	0.00957941868471741\\
9.29	0.009579439468926\\
9.3	0.00957946026148035\\
9.31	0.00957948106238452\\
9.32	0.00957950187164255\\
9.33	0.00957952268925852\\
9.34	0.0095795435152365\\
9.35	0.00957956434958055\\
9.36	0.00957958519229475\\
9.37	0.00957960604338317\\
9.38	0.00957962690284989\\
9.39	0.00957964777069898\\
9.4	0.00957966864693454\\
9.41	0.00957968953156065\\
9.42	0.00957971042458139\\
9.43	0.00957973132600087\\
9.44	0.00957975223582317\\
9.45	0.00957977315405239\\
9.46	0.00957979408069263\\
9.47	0.00957981501574799\\
9.48	0.00957983595922258\\
9.49	0.00957985691112052\\
9.5	0.0095798778714459\\
9.51	0.00957989884020284\\
9.52	0.00957991981739546\\
9.53	0.00957994080302788\\
9.54	0.00957996179710423\\
9.55	0.00957998279962863\\
9.56	0.00958000381060519\\
9.57	0.00958002483003807\\
9.58	0.00958004585793139\\
9.59	0.00958006689428928\\
9.6	0.00958008793911588\\
9.61	0.00958010899241535\\
9.62	0.00958013005419182\\
9.63	0.00958015112444944\\
9.64	0.00958017220319236\\
9.65	0.00958019329042473\\
9.66	0.00958021438615072\\
9.67	0.00958023549037447\\
9.68	0.00958025660310016\\
9.69	0.00958027772433195\\
9.7	0.009580298854074\\
9.71	0.00958031999233048\\
9.72	0.00958034113910557\\
9.73	0.00958036229440344\\
9.74	0.00958038345822827\\
9.75	0.00958040463058425\\
9.76	0.00958042581147556\\
9.77	0.00958044700090638\\
9.78	0.0095804681988809\\
9.79	0.00958048940540332\\
9.8	0.00958051062047783\\
9.81	0.00958053184410864\\
9.82	0.00958055307629994\\
9.83	0.00958057431705593\\
9.84	0.00958059556638083\\
9.85	0.00958061682427884\\
9.86	0.00958063809075418\\
9.87	0.00958065936581106\\
9.88	0.0095806806494537\\
9.89	0.00958070194168631\\
9.9	0.00958072324251314\\
9.91	0.00958074455193839\\
9.92	0.0095807658699663\\
9.93	0.0095807871966011\\
9.94	0.00958080853184703\\
9.95	0.00958082987570833\\
9.96	0.00958085122818923\\
9.97	0.00958087258929398\\
9.98	0.00958089395902682\\
9.99	0.00958091533739201\\
10	0.00958093672439379\\
10.01	0.00958095812003643\\
10.02	0.00958097952432417\\
10.03	0.00958100093726127\\
10.04	0.00958102235885202\\
10.05	0.00958104378910065\\
10.06	0.00958106522801145\\
10.07	0.00958108667558868\\
10.08	0.00958110813183663\\
10.09	0.00958112959675956\\
10.1	0.00958115107036176\\
10.11	0.00958117255264751\\
10.12	0.00958119404362109\\
10.13	0.00958121554328679\\
10.14	0.00958123705164891\\
10.15	0.00958125856871173\\
10.16	0.00958128009447956\\
10.17	0.00958130162895668\\
10.18	0.00958132317214741\\
10.19	0.00958134472405605\\
10.2	0.0095813662846869\\
10.21	0.00958138785404427\\
10.22	0.00958140943213248\\
10.23	0.00958143101895585\\
10.24	0.00958145261451868\\
10.25	0.0095814742188253\\
10.26	0.00958149583188004\\
10.27	0.00958151745368723\\
10.28	0.00958153908425118\\
10.29	0.00958156072357623\\
10.3	0.00958158237166671\\
10.31	0.00958160402852697\\
10.32	0.00958162569416134\\
10.33	0.00958164736857417\\
10.34	0.0095816690517698\\
10.35	0.00958169074375257\\
10.36	0.00958171244452684\\
10.37	0.00958173415409696\\
10.38	0.00958175587246728\\
10.39	0.00958177759964217\\
10.4	0.00958179933562598\\
10.41	0.00958182108042309\\
10.42	0.00958184283403784\\
10.43	0.00958186459647462\\
10.44	0.00958188636773779\\
10.45	0.00958190814783173\\
10.46	0.00958192993676082\\
10.47	0.00958195173452943\\
10.48	0.00958197354114194\\
10.49	0.00958199535660275\\
10.5	0.00958201718091622\\
10.51	0.00958203901408676\\
10.52	0.00958206085611875\\
10.53	0.0095820827070166\\
10.54	0.00958210456678469\\
10.55	0.00958212643542743\\
10.56	0.00958214831294921\\
10.57	0.00958217019935445\\
10.58	0.00958219209464754\\
10.59	0.0095822139988329\\
10.6	0.00958223591191493\\
10.61	0.00958225783389806\\
10.62	0.00958227976478669\\
10.63	0.00958230170458524\\
10.64	0.00958232365329814\\
10.65	0.00958234561092981\\
10.66	0.00958236757748468\\
10.67	0.00958238955296717\\
10.68	0.0095824115373817\\
10.69	0.00958243353073272\\
10.7	0.00958245553302466\\
10.71	0.00958247754426196\\
10.72	0.00958249956444905\\
10.73	0.00958252159359038\\
10.74	0.00958254363169038\\
10.75	0.00958256567875351\\
10.76	0.00958258773478422\\
10.77	0.00958260979978694\\
10.78	0.00958263187376614\\
10.79	0.00958265395672627\\
10.8	0.00958267604867179\\
10.81	0.00958269814960714\\
10.82	0.0095827202595368\\
10.83	0.00958274237846522\\
10.84	0.00958276450639688\\
10.85	0.00958278664333623\\
10.86	0.00958280878928774\\
10.87	0.00958283094425589\\
10.88	0.00958285310824514\\
10.89	0.00958287528125997\\
10.9	0.00958289746330486\\
10.91	0.00958291965438428\\
10.92	0.00958294185450271\\
10.93	0.00958296406366463\\
10.94	0.00958298628187453\\
10.95	0.00958300850913689\\
10.96	0.00958303074545619\\
10.97	0.00958305299083692\\
10.98	0.00958307524528358\\
10.99	0.00958309750880064\\
11	0.00958311978139262\\
11.01	0.00958314206306399\\
11.02	0.00958316435381925\\
11.03	0.00958318665366291\\
11.04	0.00958320896259945\\
11.05	0.00958323128063338\\
11.06	0.00958325360776919\\
11.07	0.0095832759440114\\
11.08	0.0095832982893645\\
11.09	0.00958332064383299\\
11.1	0.00958334300742139\\
11.11	0.00958336538013419\\
11.12	0.00958338776197592\\
11.13	0.00958341015295106\\
11.14	0.00958343255306414\\
11.15	0.00958345496231966\\
11.16	0.00958347738072214\\
11.17	0.00958349980827609\\
11.18	0.00958352224498602\\
11.19	0.00958354469085643\\
11.2	0.00958356714589186\\
11.21	0.00958358961009681\\
11.22	0.0095836120834758\\
11.23	0.00958363456603334\\
11.24	0.00958365705777394\\
11.25	0.00958367955870214\\
11.26	0.00958370206882244\\
11.27	0.00958372458813936\\
11.28	0.00958374711665742\\
11.29	0.00958376965438113\\
11.3	0.00958379220131502\\
11.31	0.00958381475746359\\
11.32	0.00958383732283137\\
11.33	0.00958385989742288\\
11.34	0.00958388248124262\\
11.35	0.00958390507429512\\
11.36	0.0095839276765849\\
11.37	0.00958395028811647\\
11.38	0.00958397290889434\\
11.39	0.00958399553892304\\
11.4	0.00958401817820706\\
11.41	0.00958404082675094\\
11.42	0.00958406348455917\\
11.43	0.00958408615163627\\
11.44	0.00958410882798676\\
11.45	0.00958413151361513\\
11.46	0.00958415420852591\\
11.47	0.00958417691272358\\
11.48	0.00958419962621267\\
11.49	0.00958422234899767\\
11.5	0.00958424508108308\\
11.51	0.00958426782247342\\
11.52	0.00958429057317316\\
11.53	0.00958431333318682\\
11.54	0.00958433610251888\\
11.55	0.00958435888117385\\
11.56	0.00958438166915619\\
11.57	0.00958440446647041\\
11.58	0.00958442727312099\\
11.59	0.00958445008911242\\
11.6	0.00958447291444916\\
11.61	0.00958449574913571\\
11.62	0.00958451859317652\\
11.63	0.00958454144657607\\
11.64	0.00958456430933884\\
11.65	0.00958458718146928\\
11.66	0.00958461006297185\\
11.67	0.009584632953851\\
11.68	0.0095846558541112\\
11.69	0.00958467876375689\\
11.7	0.00958470168279251\\
11.71	0.0095847246112225\\
11.72	0.00958474754905129\\
11.73	0.00958477049628333\\
11.74	0.00958479345292303\\
11.75	0.00958481641897481\\
11.76	0.00958483939444308\\
11.77	0.00958486237933226\\
11.78	0.00958488537364675\\
11.79	0.00958490837739096\\
11.8	0.00958493139056926\\
11.81	0.00958495441318604\\
11.82	0.00958497744524569\\
11.83	0.00958500048675258\\
11.84	0.00958502353771107\\
11.85	0.00958504659812552\\
11.86	0.00958506966800029\\
11.87	0.00958509274733971\\
11.88	0.00958511583614812\\
11.89	0.00958513893442986\\
11.9	0.00958516204218923\\
11.91	0.00958518515943056\\
11.92	0.00958520828615813\\
11.93	0.00958523142237626\\
11.94	0.00958525456808921\\
11.95	0.00958527772330127\\
11.96	0.00958530088801669\\
11.97	0.00958532406223974\\
11.98	0.00958534724597466\\
11.99	0.00958537043922568\\
12	0.00958539364199702\\
12.01	0.00958541685429289\\
12.02	0.00958544007611749\\
12.03	0.00958546330747501\\
12.04	0.00958548654836962\\
12.05	0.00958550979880549\\
12.06	0.00958553305878676\\
12.07	0.00958555632831758\\
12.08	0.00958557960740205\\
12.09	0.00958560289604429\\
12.1	0.00958562619424839\\
12.11	0.00958564950201844\\
12.12	0.00958567281935848\\
12.13	0.00958569614627257\\
12.14	0.00958571948276475\\
12.15	0.00958574282883902\\
12.16	0.00958576618449938\\
12.17	0.00958578954974982\\
12.18	0.00958581292459429\\
12.19	0.00958583630903674\\
12.2	0.00958585970308109\\
12.21	0.00958588310673125\\
12.22	0.00958590651999111\\
12.23	0.00958592994286453\\
12.24	0.00958595337535535\\
12.25	0.0095859768174674\\
12.26	0.00958600026920449\\
12.27	0.00958602373057038\\
12.28	0.00958604720156884\\
12.29	0.00958607068220359\\
12.3	0.00958609417247836\\
12.31	0.00958611767239681\\
12.32	0.00958614118196261\\
12.33	0.0095861647011794\\
12.34	0.00958618823005076\\
12.35	0.00958621176858029\\
12.36	0.00958623531677154\\
12.37	0.00958625887462802\\
12.38	0.00958628244215323\\
12.39	0.00958630601935063\\
12.4	0.00958632960622366\\
12.41	0.0095863532027757\\
12.42	0.00958637680901013\\
12.43	0.00958640042493029\\
12.44	0.00958642405053946\\
12.45	0.00958644768584093\\
12.46	0.00958647133083791\\
12.47	0.00958649498553361\\
12.48	0.00958651864993117\\
12.49	0.00958654232403373\\
12.5	0.00958656600784434\\
12.51	0.00958658970136606\\
12.52	0.00958661340460188\\
12.53	0.00958663711755476\\
12.54	0.00958666084022761\\
12.55	0.0095866845726233\\
12.56	0.00958670831474466\\
12.57	0.00958673206659445\\
12.58	0.00958675582817542\\
12.59	0.00958677959949025\\
12.6	0.00958680338054157\\
12.61	0.00958682717133195\\
12.62	0.00958685097186394\\
12.63	0.00958687478214001\\
12.64	0.00958689860216259\\
12.65	0.00958692243193403\\
12.66	0.00958694627145667\\
12.67	0.00958697012073274\\
12.68	0.00958699397976444\\
12.69	0.00958701784855391\\
12.7	0.00958704172710322\\
12.71	0.00958706561541437\\
12.72	0.00958708951348931\\
12.73	0.00958711342132992\\
12.74	0.00958713733893799\\
12.75	0.00958716126631528\\
12.76	0.00958718520346344\\
12.77	0.00958720915038408\\
12.78	0.00958723310707871\\
12.79	0.00958725707354877\\
12.8	0.00958728104979564\\
12.81	0.00958730503582059\\
12.82	0.00958732903162483\\
12.83	0.00958735303720948\\
12.84	0.00958737705257558\\
12.85	0.00958740107772406\\
12.86	0.0095874251126558\\
12.87	0.00958744915737156\\
12.88	0.00958747321187202\\
12.89	0.00958749727615774\\
12.9	0.00958752135022922\\
12.91	0.00958754543408683\\
12.92	0.00958756952773086\\
12.93	0.00958759363116147\\
12.94	0.00958761774437873\\
12.95	0.0095876418673826\\
12.96	0.00958766600017294\\
12.97	0.00958769014274946\\
12.98	0.0095877142951118\\
12.99	0.00958773845725943\\
13	0.00958776262919175\\
13.01	0.009587786810908\\
13.02	0.0095878110024073\\
13.03	0.00958783520368865\\
13.04	0.00958785941475089\\
13.05	0.00958788363559278\\
13.06	0.00958790786621287\\
13.07	0.00958793210660961\\
13.08	0.00958795635678131\\
13.09	0.00958798061672611\\
13.1	0.009588004886442\\
13.11	0.00958802916592683\\
13.12	0.00958805345517828\\
13.13	0.00958807775419387\\
13.14	0.00958810206297095\\
13.15	0.00958812638150671\\
13.16	0.00958815070979817\\
13.17	0.00958817504784215\\
13.18	0.00958819939563531\\
13.19	0.00958822375317412\\
13.2	0.00958824812045488\\
13.21	0.00958827249747365\\
13.22	0.00958829688422634\\
13.23	0.00958832128070863\\
13.24	0.00958834568691602\\
13.25	0.00958837010284378\\
13.26	0.00958839452848697\\
13.27	0.00958841896384042\\
13.28	0.00958844340889877\\
13.29	0.00958846786365639\\
13.3	0.00958849232810746\\
13.31	0.00958851680224587\\
13.32	0.00958854128606531\\
13.33	0.00958856577955921\\
13.34	0.00958859028272072\\
13.35	0.00958861479554278\\
13.36	0.00958863931801801\\
13.37	0.0095886638501388\\
13.38	0.00958868839189725\\
13.39	0.00958871294328518\\
13.4	0.00958873750429411\\
13.41	0.00958876207491528\\
13.42	0.00958878665513964\\
13.43	0.00958881124495779\\
13.44	0.00958883584436006\\
13.45	0.00958886045333646\\
13.46	0.00958888507187665\\
13.47	0.00958890969996996\\
13.48	0.00958893433760541\\
13.49	0.00958895898477163\\
13.5	0.00958898364145692\\
13.51	0.00958900830764923\\
13.52	0.00958903298333613\\
13.53	0.00958905766850481\\
13.54	0.00958908236314207\\
13.55	0.00958910706723434\\
13.56	0.00958913178076764\\
13.57	0.00958915650372759\\
13.58	0.00958918123609937\\
13.59	0.00958920597786778\\
13.6	0.00958923072901714\\
13.61	0.00958925548953137\\
13.62	0.00958928025939393\\
13.63	0.0095893050385878\\
13.64	0.00958932982709552\\
13.65	0.00958935462489914\\
13.66	0.00958937943198024\\
13.67	0.00958940424831989\\
13.68	0.00958942907389867\\
13.69	0.00958945390869663\\
13.7	0.00958947875269332\\
13.71	0.00958950360586773\\
13.72	0.00958952846819833\\
13.73	0.00958955333966302\\
13.74	0.00958957822023916\\
13.75	0.00958960310990351\\
13.76	0.00958962800863227\\
13.77	0.00958965291640102\\
13.78	0.00958967783318476\\
13.79	0.00958970275895786\\
13.8	0.00958972769369405\\
13.81	0.00958975263736646\\
13.82	0.00958977758994753\\
13.83	0.00958980255140907\\
13.84	0.0095898275217222\\
13.85	0.00958985250085735\\
13.86	0.00958987748878428\\
13.87	0.00958990248547202\\
13.88	0.0095899274908889\\
13.89	0.0095899525050025\\
13.9	0.00958997752777966\\
13.91	0.00959000255918649\\
13.92	0.00959002759918831\\
13.93	0.00959005264774966\\
13.94	0.00959007770483431\\
13.95	0.00959010277040521\\
13.96	0.00959012784442449\\
13.97	0.00959015292685348\\
13.98	0.00959017801765265\\
13.99	0.00959020311678163\\
14	0.00959022822419916\\
14.01	0.00959025333986315\\
14.02	0.00959027846373059\\
14.03	0.00959030359575758\\
14.04	0.00959032873589932\\
14.05	0.00959035388411007\\
14.06	0.00959037904034316\\
14.07	0.009590404204551\\
14.08	0.00959042937668501\\
14.09	0.00959045455669567\\
14.1	0.00959047974453246\\
14.11	0.0095905049401439\\
14.12	0.00959053014347748\\
14.13	0.00959055535447971\\
14.14	0.00959058057309608\\
14.15	0.00959060579927103\\
14.16	0.00959063103294799\\
14.17	0.00959065627406934\\
14.18	0.00959068152257642\\
14.19	0.00959070677840948\\
14.2	0.00959073204150776\\
14.21	0.00959075731180936\\
14.22	0.00959078258925138\\
14.23	0.00959080787376978\\
14.24	0.00959083316529947\\
14.25	0.00959085846377426\\
14.26	0.00959088376912687\\
14.27	0.00959090908128896\\
14.28	0.00959093440019105\\
14.29	0.00959095972576262\\
14.3	0.00959098505793202\\
14.31	0.00959101039662656\\
14.32	0.00959103574177244\\
14.33	0.0095910610932948\\
14.34	0.00959108645111771\\
14.35	0.00959111181516421\\
14.36	0.00959113718535625\\
14.37	0.00959116256161479\\
14.38	0.00959118794385974\\
14.39	0.00959121333201005\\
14.4	0.00959123872598365\\
14.41	0.00959126412569752\\
14.42	0.0095912895310677\\
14.43	0.00959131494200934\\
14.44	0.00959134035843667\\
14.45	0.0095913657802631\\
14.46	0.0095913912074012\\
14.47	0.00959141663976276\\
14.48	0.00959144207725885\\
14.49	0.00959146751979981\\
14.5	0.00959149296729535\\
14.51	0.0095915184196546\\
14.52	0.00959154387678611\\
14.53	0.00959156933859796\\
14.54	0.00959159480499783\\
14.55	0.00959162027589302\\
14.56	0.00959164575119059\\
14.57	0.00959167123079737\\
14.58	0.0095916967146201\\
14.59	0.00959172220256548\\
14.6	0.00959174769454031\\
14.61	0.00959177319045156\\
14.62	0.00959179869020645\\
14.63	0.00959182419371267\\
14.64	0.00959184970087839\\
14.65	0.00959187521161247\\
14.66	0.00959190072582455\\
14.67	0.00959192624342522\\
14.68	0.00959195176432621\\
14.69	0.00959197728844048\\
14.7	0.00959200281568247\\
14.71	0.00959202834596826\\
14.72	0.00959205387921575\\
14.73	0.00959207941534492\\
14.74	0.00959210495427799\\
14.75	0.0095921304959397\\
14.76	0.00959215604025756\\
14.77	0.00959218158716205\\
14.78	0.009592207136587\\
14.79	0.00959223268846977\\
14.8	0.00959225824275165\\
14.81	0.00959228379937813\\
14.82	0.00959230935829926\\
14.83	0.00959233491947001\\
14.84	0.00959236048285063\\
14.85	0.0095923860484071\\
14.86	0.00959241161611149\\
14.87	0.00959243718594244\\
14.88	0.00959246275788561\\
14.89	0.0095924883319342\\
14.9	0.0095925139080894\\
14.91	0.009592539486361\\
14.92	0.00959256506676793\\
14.93	0.00959259064933885\\
14.94	0.00959261623411281\\
14.95	0.00959264182113987\\
14.96	0.0095926674104818\\
14.97	0.00959269300221287\\
14.98	0.00959271859642052\\
14.99	0.00959274419320623\\
15	0.00959276979268633\\
15.01	0.00959279539499291\\
15.02	0.00959282100027473\\
15.03	0.00959284660869818\\
15.04	0.00959287222044833\\
15.05	0.00959289783572998\\
15.06	0.0095929234547688\\
15.07	0.00959294907781246\\
15.08	0.00959297470513192\\
15.09	0.0095930003370227\\
15.1	0.00959302597380622\\
15.11	0.00959305161583123\\
15.12	0.00959307726347529\\
15.13	0.00959310291714635\\
15.14	0.00959312857728433\\
15.15	0.00959315424436289\\
15.16	0.00959317991889115\\
15.17	0.00959320560141562\\
15.18	0.00959323129252215\\
15.19	0.00959325699283793\\
15.2	0.0095932827030337\\
15.21	0.00959330842382596\\
15.22	0.00959333415597935\\
15.23	0.00959335989957167\\
15.24	0.00959338565460988\\
15.25	0.009593411421101\\
15.26	0.009593437199052\\
15.27	0.00959346298846989\\
15.28	0.00959348878936168\\
15.29	0.00959351460173438\\
15.3	0.00959354042559503\\
15.31	0.00959356626095064\\
15.32	0.00959359210780825\\
15.33	0.00959361796617491\\
15.34	0.00959364383605766\\
15.35	0.00959366971746355\\
15.36	0.00959369561039967\\
15.37	0.00959372151487306\\
15.38	0.00959374743089082\\
15.39	0.00959377335846002\\
15.4	0.00959379929758776\\
15.41	0.00959382524828112\\
15.42	0.00959385121054723\\
15.43	0.00959387718439318\\
15.44	0.00959390316982609\\
15.45	0.0095939291668531\\
15.46	0.00959395517548133\\
15.47	0.00959398119571792\\
15.48	0.00959400722757001\\
15.49	0.00959403327104478\\
15.5	0.00959405932614935\\
15.51	0.00959408539289092\\
15.52	0.00959411147127664\\
15.53	0.00959413756131371\\
15.54	0.00959416366300931\\
15.55	0.00959418977637064\\
15.56	0.00959421590140488\\
15.57	0.00959424203811927\\
15.58	0.00959426818652101\\
15.59	0.00959429434661732\\
15.6	0.00959432051841544\\
15.61	0.0095943467019226\\
15.62	0.00959437289714604\\
15.63	0.00959439910409302\\
15.64	0.0095944253227708\\
15.65	0.00959445155318665\\
15.66	0.00959447779534782\\
15.67	0.00959450404926162\\
15.68	0.00959453031493531\\
15.69	0.0095945565923762\\
15.7	0.00959458288159159\\
15.71	0.00959460918258879\\
15.72	0.0095946354953751\\
15.73	0.00959466181995787\\
15.74	0.00959468815634441\\
15.75	0.00959471450454206\\
15.76	0.00959474086455817\\
15.77	0.00959476723640009\\
15.78	0.00959479362007518\\
15.79	0.00959482001559081\\
15.8	0.00959484642295434\\
15.81	0.00959487284217317\\
15.82	0.00959489927325468\\
15.83	0.00959492571620626\\
15.84	0.00959495217103532\\
15.85	0.00959497863774927\\
15.86	0.00959500511635552\\
15.87	0.00959503160686151\\
15.88	0.00959505810927465\\
15.89	0.00959508462360241\\
15.9	0.00959511114985221\\
15.91	0.00959513768803151\\
15.92	0.00959516423814778\\
15.93	0.00959519080020849\\
15.94	0.00959521737422111\\
15.95	0.00959524396019313\\
15.96	0.00959527055813204\\
15.97	0.00959529716804534\\
15.98	0.00959532378994053\\
15.99	0.00959535042382513\\
16	0.00959537706970666\\
16.01	0.00959540372759264\\
16.02	0.00959543039749062\\
16.03	0.00959545707940815\\
16.04	0.00959548377335276\\
16.05	0.00959551047933203\\
16.06	0.00959553719735351\\
16.07	0.00959556392742479\\
16.08	0.00959559066955343\\
16.09	0.00959561742374704\\
16.1	0.00959564419001321\\
16.11	0.00959567096835955\\
16.12	0.00959569775879366\\
16.13	0.00959572456132317\\
16.14	0.0095957513759557\\
16.15	0.00959577820269889\\
16.16	0.00959580504156039\\
16.17	0.00959583189254784\\
16.18	0.00959585875566889\\
16.19	0.00959588563093123\\
16.2	0.00959591251834252\\
16.21	0.00959593941791043\\
16.22	0.00959596632964268\\
16.23	0.00959599325354693\\
16.24	0.00959602018963091\\
16.25	0.00959604713790232\\
16.26	0.00959607409836889\\
16.27	0.00959610107103834\\
16.28	0.0095961280559184\\
16.29	0.00959615505301683\\
16.3	0.00959618206234138\\
16.31	0.00959620908389979\\
16.32	0.00959623611769984\\
16.33	0.00959626316374931\\
16.34	0.00959629022205597\\
16.35	0.00959631729262762\\
16.36	0.00959634437547206\\
16.37	0.00959637147059708\\
16.38	0.00959639857801052\\
16.39	0.00959642569772018\\
16.4	0.0095964528297339\\
16.41	0.00959647997405952\\
16.42	0.00959650713070487\\
16.43	0.00959653429967782\\
16.44	0.00959656148098623\\
16.45	0.00959658867463797\\
16.46	0.00959661588064092\\
16.47	0.00959664309900296\\
16.48	0.00959667032973198\\
16.49	0.00959669757283589\\
16.5	0.0095967248283226\\
16.51	0.00959675209620002\\
16.52	0.00959677937647609\\
16.53	0.00959680666915873\\
16.54	0.00959683397425589\\
16.55	0.00959686129177553\\
16.56	0.00959688862172558\\
16.57	0.00959691596411404\\
16.58	0.00959694331894886\\
16.59	0.00959697068623803\\
16.6	0.00959699806598956\\
16.61	0.00959702545821142\\
16.62	0.00959705286291163\\
16.63	0.00959708028009821\\
16.64	0.00959710770977918\\
16.65	0.00959713515196257\\
16.66	0.00959716260665642\\
16.67	0.00959719007386879\\
16.68	0.00959721755360772\\
16.69	0.00959724504588129\\
16.7	0.00959727255069756\\
16.71	0.00959730006806463\\
16.72	0.00959732759799057\\
16.73	0.00959735514048349\\
16.74	0.00959738269555149\\
16.75	0.00959741026320269\\
16.76	0.00959743784344522\\
16.77	0.0095974654362872\\
16.78	0.00959749304173677\\
16.79	0.00959752065980209\\
16.8	0.00959754829049131\\
16.81	0.0095975759338126\\
16.82	0.00959760358977412\\
16.83	0.00959763125838408\\
16.84	0.00959765893965064\\
16.85	0.00959768663358201\\
16.86	0.0095977143401864\\
16.87	0.00959774205947203\\
16.88	0.00959776979144712\\
16.89	0.00959779753611991\\
16.9	0.00959782529349863\\
16.91	0.00959785306359154\\
16.92	0.00959788084640689\\
16.93	0.00959790864195296\\
16.94	0.00959793645023801\\
16.95	0.00959796427127034\\
16.96	0.00959799210505824\\
16.97	0.00959801995161\\
16.98	0.00959804781093394\\
16.99	0.00959807568303839\\
17	0.00959810356793166\\
17.01	0.00959813146562209\\
17.02	0.00959815937611803\\
17.03	0.00959818729942783\\
17.04	0.00959821523555985\\
17.05	0.00959824318452247\\
17.06	0.00959827114632407\\
17.07	0.00959829912097303\\
17.08	0.00959832710847775\\
17.09	0.00959835510884664\\
17.1	0.00959838312208812\\
17.11	0.00959841114821061\\
17.12	0.00959843918722253\\
17.13	0.00959846723913233\\
17.14	0.00959849530394847\\
17.15	0.00959852338167941\\
17.16	0.0095985514723336\\
17.17	0.00959857957591953\\
17.18	0.00959860769244568\\
17.19	0.00959863582192054\\
17.2	0.00959866396435263\\
17.21	0.00959869211975046\\
17.22	0.00959872028812254\\
17.23	0.0095987484694774\\
17.24	0.00959877666382359\\
17.25	0.00959880487116966\\
17.26	0.00959883309152416\\
17.27	0.00959886132489566\\
17.28	0.00959888957129273\\
17.29	0.00959891783072396\\
17.3	0.00959894610319794\\
17.31	0.00959897438872328\\
17.32	0.00959900268730858\\
17.33	0.00959903099896247\\
17.34	0.00959905932369359\\
17.35	0.00959908766151055\\
17.36	0.00959911601242203\\
17.37	0.00959914437643667\\
17.38	0.00959917275356313\\
17.39	0.00959920114381011\\
17.4	0.00959922954718627\\
17.41	0.00959925796370032\\
17.42	0.00959928639336095\\
17.43	0.00959931483617689\\
17.44	0.00959934329215685\\
17.45	0.00959937176130957\\
17.46	0.00959940024364377\\
17.47	0.00959942873916823\\
17.48	0.00959945724789168\\
17.49	0.0095994857698229\\
17.5	0.00959951430497068\\
17.51	0.00959954285334379\\
17.52	0.00959957141495103\\
17.53	0.00959959998980121\\
17.54	0.00959962857790314\\
17.55	0.00959965717926566\\
17.56	0.00959968579389758\\
17.57	0.00959971442180776\\
17.58	0.00959974306300504\\
17.59	0.0095997717174983\\
17.6	0.0095998003852964\\
17.61	0.00959982906640822\\
17.62	0.00959985776084266\\
17.63	0.00959988646860861\\
17.64	0.00959991518971498\\
17.65	0.00959994392417069\\
17.66	0.00959997267198468\\
17.67	0.00960000143316588\\
17.68	0.00960003020772323\\
17.69	0.0096000589956657\\
17.7	0.00960008779700226\\
17.71	0.00960011661174187\\
17.72	0.00960014543989354\\
17.73	0.00960017428146624\\
17.74	0.00960020313646899\\
17.75	0.00960023200491081\\
17.76	0.00960026088680072\\
17.77	0.00960028978214775\\
17.78	0.00960031869096094\\
17.79	0.00960034761324936\\
17.8	0.00960037654902208\\
17.81	0.00960040549828815\\
17.82	0.00960043446105666\\
17.83	0.00960046343733672\\
17.84	0.00960049242713742\\
17.85	0.00960052143046787\\
17.86	0.0096005504473372\\
17.87	0.00960057947775454\\
17.88	0.00960060852172904\\
17.89	0.00960063757926984\\
17.9	0.00960066665038612\\
17.91	0.00960069573508703\\
17.92	0.00960072483338177\\
17.93	0.00960075394527953\\
17.94	0.0096007830707895\\
17.95	0.00960081220992091\\
17.96	0.00960084136268298\\
17.97	0.00960087052908493\\
17.98	0.009600899709136\\
17.99	0.00960092890284546\\
18	0.00960095811022257\\
18.01	0.00960098733127659\\
18.02	0.00960101656601681\\
18.03	0.00960104581445252\\
18.04	0.00960107507659303\\
18.05	0.00960110435244765\\
18.06	0.0096011336420257\\
18.07	0.00960116294533651\\
18.08	0.00960119226238944\\
18.09	0.00960122159319382\\
18.1	0.00960125093775904\\
18.11	0.00960128029609445\\
18.12	0.00960130966820945\\
18.13	0.00960133905411343\\
18.14	0.00960136845381579\\
18.15	0.00960139786732596\\
18.16	0.00960142729465335\\
18.17	0.00960145673580741\\
18.18	0.00960148619079758\\
18.19	0.00960151565963331\\
18.2	0.00960154514232407\\
18.21	0.00960157463887934\\
18.22	0.00960160414930861\\
18.23	0.00960163367362138\\
18.24	0.00960166321182716\\
18.25	0.00960169276393546\\
18.26	0.00960172232995581\\
18.27	0.00960175190989776\\
18.28	0.00960178150377085\\
18.29	0.00960181111158465\\
18.3	0.00960184073334872\\
18.31	0.00960187036907266\\
18.32	0.00960190001876605\\
18.33	0.0096019296824385\\
18.34	0.00960195936009962\\
18.35	0.00960198905175903\\
18.36	0.00960201875742638\\
18.37	0.00960204847711129\\
18.38	0.00960207821082345\\
18.39	0.0096021079585725\\
18.4	0.00960213772036814\\
18.41	0.00960216749622004\\
18.42	0.00960219728613791\\
18.43	0.00960222709013145\\
18.44	0.0096022569082104\\
18.45	0.00960228674038447\\
18.46	0.00960231658666342\\
18.47	0.00960234644705699\\
18.48	0.00960237632157495\\
18.49	0.00960240621022708\\
18.5	0.00960243611302316\\
18.51	0.00960246602997298\\
18.52	0.00960249596108637\\
18.53	0.00960252590637313\\
18.54	0.00960255586584309\\
18.55	0.00960258583950609\\
18.56	0.00960261582737199\\
18.57	0.00960264582945065\\
18.58	0.00960267584575195\\
18.59	0.00960270587628576\\
18.6	0.00960273592106197\\
18.61	0.00960276598009051\\
18.62	0.00960279605338129\\
18.63	0.00960282614094423\\
18.64	0.00960285624278927\\
18.65	0.00960288635892637\\
18.66	0.0096029164893655\\
18.67	0.00960294663411661\\
18.68	0.00960297679318969\\
18.69	0.00960300696659476\\
18.7	0.00960303715434179\\
18.71	0.00960306735644083\\
18.72	0.00960309757290189\\
18.73	0.00960312780373502\\
18.74	0.00960315804895026\\
18.75	0.00960318830855769\\
18.76	0.00960321858256738\\
18.77	0.00960324887098941\\
18.78	0.00960327917383387\\
18.79	0.00960330949111089\\
18.8	0.00960333982283058\\
18.81	0.00960337016900306\\
18.82	0.00960340052963849\\
18.83	0.00960343090474701\\
18.84	0.00960346129433879\\
18.85	0.00960349169842401\\
18.86	0.00960352211701287\\
18.87	0.00960355255011555\\
18.88	0.00960358299774227\\
18.89	0.00960361345990325\\
18.9	0.00960364393660873\\
18.91	0.00960367442786896\\
18.92	0.00960370493369419\\
18.93	0.0096037354540947\\
18.94	0.00960376598908075\\
18.95	0.00960379653866266\\
18.96	0.00960382710285071\\
18.97	0.00960385768165523\\
18.98	0.00960388827508655\\
18.99	0.009603918883155\\
19	0.00960394950587094\\
19.01	0.00960398014324473\\
19.02	0.00960401079528675\\
19.03	0.00960404146200738\\
19.04	0.00960407214341702\\
19.05	0.00960410283952608\\
19.06	0.00960413355034499\\
19.07	0.00960416427588417\\
19.08	0.00960419501615408\\
19.09	0.00960422577116518\\
19.1	0.00960425654092792\\
19.11	0.0096042873254528\\
19.12	0.00960431812475031\\
19.13	0.00960434893883096\\
19.14	0.00960437976770525\\
19.15	0.00960441061138374\\
19.16	0.00960444146987694\\
19.17	0.00960447234319543\\
19.18	0.00960450323134977\\
19.19	0.00960453413435053\\
19.2	0.0096045650522083\\
19.21	0.00960459598493369\\
19.22	0.00960462693253731\\
19.23	0.00960465789502979\\
19.24	0.00960468887242177\\
19.25	0.0096047198647239\\
19.26	0.00960475087194685\\
19.27	0.00960478189410129\\
19.28	0.0096048129311979\\
19.29	0.00960484398324739\\
19.3	0.00960487505026047\\
19.31	0.00960490613224787\\
19.32	0.00960493722922033\\
19.33	0.00960496834118859\\
19.34	0.00960499946816341\\
19.35	0.00960503061015558\\
19.36	0.00960506176717588\\
19.37	0.0096050929392351\\
19.38	0.00960512412634407\\
19.39	0.0096051553285136\\
19.4	0.00960518654575454\\
19.41	0.00960521777807772\\
19.42	0.00960524902549402\\
19.43	0.00960528028801432\\
19.44	0.00960531156564948\\
19.45	0.00960534285841043\\
19.46	0.00960537416630807\\
19.47	0.00960540548935332\\
19.48	0.00960543682755713\\
19.49	0.00960546818093044\\
19.5	0.00960549954948422\\
19.51	0.00960553093322944\\
19.52	0.0096055623321771\\
19.53	0.00960559374633819\\
19.54	0.00960562517572373\\
19.55	0.00960565662034475\\
19.56	0.00960568808021229\\
19.57	0.0096057195553374\\
19.58	0.00960575104573115\\
19.59	0.00960578255140461\\
19.6	0.00960581407236888\\
19.61	0.00960584560863506\\
19.62	0.00960587716021428\\
19.63	0.00960590872711765\\
19.64	0.00960594030935633\\
19.65	0.00960597190694148\\
19.66	0.00960600351988425\\
19.67	0.00960603514819585\\
19.68	0.00960606679188745\\
19.69	0.00960609845097028\\
19.7	0.00960613012545554\\
19.71	0.00960616181535449\\
19.72	0.00960619352067836\\
19.73	0.00960622524143842\\
19.74	0.00960625697764595\\
19.75	0.00960628872931223\\
19.76	0.00960632049644857\\
19.77	0.00960635227906627\\
19.78	0.00960638407717667\\
19.79	0.00960641589079111\\
19.8	0.00960644771992095\\
19.81	0.00960647956457755\\
19.82	0.0096065114247723\\
19.83	0.00960654330051658\\
19.84	0.00960657519182181\\
19.85	0.00960660709869941\\
19.86	0.00960663902116082\\
19.87	0.00960667095921748\\
19.88	0.00960670291288086\\
19.89	0.00960673488216244\\
19.9	0.0096067668670737\\
19.91	0.00960679886762614\\
19.92	0.00960683088383129\\
19.93	0.00960686291570067\\
19.94	0.00960689496324583\\
19.95	0.00960692702647832\\
19.96	0.00960695910540973\\
19.97	0.00960699120005163\\
19.98	0.00960702331041561\\
19.99	0.00960705543651331\\
20	0.00960708757835634\\
20.01	0.00960711973595633\\
20.02	0.00960715190932495\\
20.03	0.00960718409847387\\
20.04	0.00960721630341477\\
20.05	0.00960724852415934\\
20.06	0.00960728076071929\\
20.07	0.00960731301310634\\
20.08	0.00960734528133225\\
20.09	0.00960737756540874\\
20.1	0.0096074098653476\\
20.11	0.0096074421811606\\
20.12	0.00960747451285954\\
20.13	0.00960750686045622\\
20.14	0.00960753922396246\\
20.15	0.00960757160339011\\
20.16	0.00960760399875102\\
20.17	0.00960763641005703\\
20.18	0.00960766883732005\\
20.19	0.00960770128055195\\
20.2	0.00960773373976465\\
20.21	0.00960776621497007\\
20.22	0.00960779870618015\\
20.23	0.00960783121340682\\
20.24	0.00960786373666207\\
20.25	0.00960789627595787\\
20.26	0.0096079288313062\\
20.27	0.00960796140271909\\
20.28	0.00960799399020855\\
20.29	0.00960802659378661\\
20.3	0.00960805921346534\\
20.31	0.00960809184925679\\
20.32	0.00960812450117304\\
20.33	0.0096081571692262\\
20.34	0.00960818985342836\\
20.35	0.00960822255379166\\
20.36	0.00960825527032823\\
20.37	0.00960828800305022\\
20.38	0.00960832075196981\\
20.39	0.00960835351709917\\
20.4	0.00960838629845051\\
20.41	0.00960841909603603\\
20.42	0.00960845190986797\\
20.43	0.00960848473995856\\
20.44	0.00960851758632006\\
20.45	0.00960855044896474\\
20.46	0.00960858332790489\\
20.47	0.00960861622315281\\
20.48	0.00960864913472082\\
20.49	0.00960868206262124\\
20.5	0.00960871500686643\\
20.51	0.00960874796746874\\
20.52	0.00960878094444055\\
20.53	0.00960881393779425\\
20.54	0.00960884694754225\\
20.55	0.00960887997369697\\
20.56	0.00960891301627084\\
20.57	0.00960894607527632\\
20.58	0.00960897915072588\\
20.59	0.00960901224263199\\
20.6	0.00960904535100716\\
20.61	0.00960907847586388\\
20.62	0.00960911161721471\\
20.63	0.00960914477507217\\
20.64	0.00960917794944882\\
20.65	0.00960921114035724\\
20.66	0.00960924434781002\\
20.67	0.00960927757181976\\
20.68	0.00960931081239909\\
20.69	0.00960934406956063\\
20.7	0.00960937734331704\\
20.71	0.00960941063368099\\
20.72	0.00960944394066515\\
20.73	0.00960947726428222\\
20.74	0.00960951060454493\\
20.75	0.00960954396146599\\
20.76	0.00960957733505815\\
20.77	0.00960961072533417\\
20.78	0.00960964413230683\\
20.79	0.00960967755598891\\
20.8	0.00960971099639323\\
20.81	0.00960974445353261\\
20.82	0.00960977792741988\\
20.83	0.00960981141806791\\
20.84	0.00960984492548956\\
20.85	0.00960987844969771\\
20.86	0.00960991199070528\\
20.87	0.00960994554852518\\
20.88	0.00960997912317033\\
20.89	0.00961001271465371\\
20.9	0.00961004632298826\\
20.91	0.00961007994818697\\
20.92	0.00961011359026284\\
20.93	0.00961014724922888\\
20.94	0.00961018092509813\\
20.95	0.00961021461788363\\
20.96	0.00961024832759844\\
20.97	0.00961028205425565\\
20.98	0.00961031579786834\\
20.99	0.00961034955844964\\
21	0.00961038333601265\\
21.01	0.00961041713057055\\
21.02	0.00961045094213647\\
21.03	0.0096104847707236\\
21.04	0.00961051861634513\\
21.05	0.00961055247901427\\
21.06	0.00961058635874425\\
21.07	0.00961062025554831\\
21.08	0.00961065416943971\\
21.09	0.00961068810043172\\
21.1	0.00961072204853764\\
21.11	0.00961075601377078\\
21.12	0.00961078999614445\\
21.13	0.00961082399567201\\
21.14	0.00961085801236681\\
21.15	0.00961089204624222\\
21.16	0.00961092609731165\\
21.17	0.00961096016558849\\
21.18	0.00961099425108618\\
21.19	0.00961102835381815\\
21.2	0.00961106247379787\\
21.21	0.00961109661103881\\
21.22	0.00961113076555447\\
21.23	0.00961116493735835\\
21.24	0.00961119912646398\\
21.25	0.0096112333328849\\
21.26	0.00961126755663469\\
21.27	0.00961130179772691\\
21.28	0.00961133605617516\\
21.29	0.00961137033199305\\
21.3	0.00961140462519422\\
21.31	0.00961143893579231\\
21.32	0.00961147326380097\\
21.33	0.0096115076092339\\
21.34	0.0096115419721048\\
21.35	0.00961157635242737\\
21.36	0.00961161075021535\\
21.37	0.00961164516548249\\
21.38	0.00961167959824256\\
21.39	0.00961171404850935\\
21.4	0.00961174851629665\\
21.41	0.00961178300161828\\
21.42	0.0096118175044881\\
21.43	0.00961185202491993\\
21.44	0.00961188656292767\\
21.45	0.0096119211185252\\
21.46	0.00961195569172643\\
21.47	0.00961199028254529\\
21.48	0.00961202489099571\\
21.49	0.00961205951709166\\
21.5	0.00961209416084712\\
21.51	0.00961212882227608\\
21.52	0.00961216350139256\\
21.53	0.00961219819821059\\
21.54	0.00961223291274421\\
21.55	0.00961226764500751\\
21.56	0.00961230239501456\\
21.57	0.00961233716277946\\
21.58	0.00961237194831634\\
21.59	0.00961240675163934\\
21.6	0.00961244157276261\\
21.61	0.00961247641170034\\
21.62	0.0096125112684667\\
21.63	0.00961254614307593\\
21.64	0.00961258103554224\\
21.65	0.00961261594587989\\
21.66	0.00961265087410314\\
21.67	0.00961268582022628\\
21.68	0.0096127207842636\\
21.69	0.00961275576622945\\
21.7	0.00961279076613813\\
21.71	0.00961282578400403\\
21.72	0.00961286081984152\\
21.73	0.00961289587366499\\
21.74	0.00961293094548886\\
21.75	0.00961296603532755\\
21.76	0.00961300114319553\\
21.77	0.00961303626910725\\
21.78	0.00961307141307721\\
21.79	0.00961310657511992\\
21.8	0.00961314175524989\\
21.81	0.00961317695348167\\
21.82	0.00961321216982984\\
21.83	0.00961324740430896\\
21.84	0.00961328265693364\\
21.85	0.0096133179277185\\
21.86	0.00961335321667817\\
21.87	0.00961338852382731\\
21.88	0.00961342384918061\\
21.89	0.00961345919275275\\
21.9	0.00961349455455846\\
21.91	0.00961352993461245\\
21.92	0.00961356533292949\\
21.93	0.00961360074952435\\
21.94	0.00961363618441181\\
21.95	0.00961367163760669\\
21.96	0.00961370710912383\\
21.97	0.00961374259897805\\
21.98	0.00961377810718424\\
21.99	0.00961381363375728\\
22	0.00961384917871207\\
22.01	0.00961388474206355\\
22.02	0.00961392032382666\\
22.03	0.00961395592401636\\
22.04	0.00961399154264763\\
22.05	0.00961402717973549\\
22.06	0.00961406283529496\\
22.07	0.00961409850934107\\
22.08	0.00961413420188889\\
22.09	0.00961416991295351\\
22.1	0.00961420564255001\\
22.11	0.00961424139069354\\
22.12	0.00961427715739922\\
22.13	0.00961431294268223\\
22.14	0.00961434874655773\\
22.15	0.00961438456904094\\
22.16	0.00961442041014708\\
22.17	0.00961445626989138\\
22.18	0.00961449214828911\\
22.19	0.00961452804535555\\
22.2	0.00961456396110599\\
22.21	0.00961459989555577\\
22.22	0.00961463584872022\\
22.23	0.0096146718206147\\
22.24	0.00961470781125461\\
22.25	0.00961474382065533\\
22.26	0.0096147798488323\\
22.27	0.00961481589580095\\
22.28	0.00961485196157675\\
22.29	0.00961488804617518\\
22.3	0.00961492414961176\\
22.31	0.00961496027190199\\
22.32	0.00961499641306143\\
22.33	0.00961503257310564\\
22.34	0.00961506875205021\\
22.35	0.00961510494991076\\
22.36	0.00961514116670289\\
22.37	0.00961517740244227\\
22.38	0.00961521365714457\\
22.39	0.00961524993082546\\
22.4	0.00961528622350068\\
22.41	0.00961532253518594\\
22.42	0.00961535886589699\\
22.43	0.00961539521564962\\
22.44	0.00961543158445962\\
22.45	0.0096154679723428\\
22.46	0.00961550437931501\\
22.47	0.00961554080539209\\
22.48	0.00961557725058993\\
22.49	0.00961561371492442\\
22.5	0.0096156501984115\\
22.51	0.00961568670106711\\
22.52	0.00961572322290719\\
22.53	0.00961575976394776\\
22.54	0.0096157963242048\\
22.55	0.00961583290369435\\
22.56	0.00961586950243246\\
22.57	0.00961590612043519\\
22.58	0.00961594275771865\\
22.59	0.00961597941429895\\
22.6	0.00961601609019222\\
22.61	0.00961605278541463\\
22.62	0.00961608949998234\\
22.63	0.00961612623391157\\
22.64	0.00961616298721853\\
22.65	0.00961619975991947\\
22.66	0.00961623655203067\\
22.67	0.0096162733635684\\
22.68	0.00961631019454897\\
22.69	0.00961634704498873\\
22.7	0.00961638391490402\\
22.71	0.00961642080431123\\
22.72	0.00961645771322675\\
22.73	0.009616494641667\\
22.74	0.00961653158964843\\
22.75	0.0096165685571875\\
22.76	0.00961660554430071\\
22.77	0.00961664255100456\\
22.78	0.00961667957731558\\
22.79	0.00961671662325033\\
22.8	0.00961675368882539\\
22.81	0.00961679077405737\\
22.82	0.00961682787896287\\
22.83	0.00961686500355856\\
22.84	0.00961690214786109\\
22.85	0.00961693931188716\\
22.86	0.00961697649565348\\
22.87	0.00961701369917679\\
22.88	0.00961705092247385\\
22.89	0.00961708816556145\\
22.9	0.00961712542845638\\
22.91	0.00961716271117547\\
22.92	0.00961720001373558\\
22.93	0.00961723733615359\\
22.94	0.00961727467844638\\
22.95	0.00961731204063088\\
22.96	0.00961734942272404\\
22.97	0.00961738682474283\\
22.98	0.00961742424670422\\
22.99	0.00961746168862525\\
23	0.00961749915052294\\
23.01	0.00961753663241435\\
23.02	0.00961757413431658\\
23.03	0.00961761165624673\\
23.04	0.00961764919822193\\
23.05	0.00961768676025933\\
23.06	0.00961772434237613\\
23.07	0.00961776194458951\\
23.08	0.00961779956691671\\
23.09	0.00961783720937498\\
23.1	0.00961787487198159\\
23.11	0.00961791255475384\\
23.12	0.00961795025770905\\
23.13	0.00961798798086458\\
23.14	0.00961802572423778\\
23.15	0.00961806348784606\\
23.16	0.00961810127170683\\
23.17	0.00961813907583755\\
23.18	0.00961817690025567\\
23.19	0.0096182147449787\\
23.2	0.00961825261002413\\
23.21	0.00961829049540953\\
23.22	0.00961832840115245\\
23.23	0.00961836632727048\\
23.24	0.00961840427378124\\
23.25	0.00961844224070237\\
23.26	0.00961848022805153\\
23.27	0.00961851823584642\\
23.28	0.00961855626410474\\
23.29	0.00961859431284423\\
23.3	0.00961863238208267\\
23.31	0.00961867047183783\\
23.32	0.00961870858212754\\
23.33	0.00961874671296963\\
23.34	0.00961878486438197\\
23.35	0.00961882303638244\\
23.36	0.00961886122898897\\
23.37	0.00961889944221949\\
23.38	0.00961893767609197\\
23.39	0.0096189759306244\\
23.4	0.00961901420583479\\
23.41	0.0096190525017412\\
23.42	0.00961909081836168\\
23.43	0.00961912915571435\\
23.44	0.0096191675138173\\
23.45	0.00961920589268869\\
23.46	0.00961924429234669\\
23.47	0.0096192827128095\\
23.48	0.00961932115409534\\
23.49	0.00961935961622247\\
23.5	0.00961939809920915\\
23.51	0.0096194366030737\\
23.52	0.00961947512783443\\
23.53	0.00961951367350971\\
23.54	0.0096195522401179\\
23.55	0.00961959082767744\\
23.56	0.00961962943620675\\
23.57	0.00961966806572428\\
23.58	0.00961970671624853\\
23.59	0.00961974538779801\\
23.6	0.00961978408039126\\
23.61	0.00961982279404686\\
23.62	0.00961986152878339\\
23.63	0.00961990028461948\\
23.64	0.00961993906157378\\
23.65	0.00961997785966496\\
23.66	0.00962001667891173\\
23.67	0.00962005551933283\\
23.68	0.009620094380947\\
23.69	0.00962013326377303\\
23.7	0.00962017216782974\\
23.71	0.00962021109313597\\
23.72	0.00962025003971059\\
23.73	0.00962028900757249\\
23.74	0.0096203279967406\\
23.75	0.00962036700723386\\
23.76	0.00962040603907128\\
23.77	0.00962044509227184\\
23.78	0.00962048416685458\\
23.79	0.00962052326283857\\
23.8	0.00962056238024291\\
23.81	0.00962060151908671\\
23.82	0.00962064067938912\\
23.83	0.00962067986116933\\
23.84	0.00962071906444653\\
23.85	0.00962075828923997\\
23.86	0.00962079753556891\\
23.87	0.00962083680345264\\
23.88	0.00962087609291048\\
23.89	0.00962091540396179\\
23.9	0.00962095473662595\\
23.91	0.00962099409092236\\
23.92	0.00962103346687046\\
23.93	0.00962107286448972\\
23.94	0.00962111228379965\\
23.95	0.00962115172481976\\
23.96	0.00962119118756961\\
23.97	0.00962123067206879\\
23.98	0.00962127017833692\\
23.99	0.00962130970639365\\
24	0.00962134925625864\\
24.01	0.0096213888279516\\
24.02	0.00962142842149228\\
24.03	0.00962146803690044\\
24.04	0.00962150767419587\\
24.05	0.00962154733339841\\
24.06	0.00962158701452791\\
24.07	0.00962162671760426\\
24.08	0.00962166644264739\\
24.09	0.00962170618967723\\
24.1	0.00962174595871378\\
24.11	0.00962178574977705\\
24.12	0.00962182556288708\\
24.13	0.00962186539806395\\
24.14	0.00962190525532776\\
24.15	0.00962194513469866\\
24.16	0.00962198503619681\\
24.17	0.00962202495984241\\
24.18	0.0096220649056557\\
24.19	0.00962210487365695\\
24.2	0.00962214486386645\\
24.21	0.00962218487630454\\
24.22	0.00962222491099157\\
24.23	0.00962226496794795\\
24.24	0.00962230504719409\\
24.25	0.00962234514875047\\
24.26	0.00962238527263757\\
24.27	0.00962242541887591\\
24.28	0.00962246558748607\\
24.29	0.00962250577848863\\
24.3	0.00962254599190421\\
24.31	0.00962258622775349\\
24.32	0.00962262648605715\\
24.33	0.00962266676683591\\
24.34	0.00962270707011055\\
24.35	0.00962274739590184\\
24.36	0.00962278774423063\\
24.37	0.00962282811511779\\
24.38	0.00962286850858419\\
24.39	0.00962290892465078\\
24.4	0.00962294936333852\\
24.41	0.00962298982466843\\
24.42	0.00962303030866153\\
24.43	0.0096230708153389\\
24.44	0.00962311134472164\\
24.45	0.00962315189683091\\
24.46	0.00962319247168787\\
24.47	0.00962323306931375\\
24.48	0.00962327368972979\\
24.49	0.00962331433295729\\
24.5	0.00962335499901757\\
24.51	0.00962339568793198\\
24.52	0.00962343639972194\\
24.53	0.00962347713440886\\
24.54	0.00962351789201422\\
24.55	0.00962355867255953\\
24.56	0.00962359947606634\\
24.57	0.00962364030255623\\
24.58	0.00962368115205082\\
24.59	0.00962372202457177\\
24.6	0.00962376292014078\\
24.61	0.00962380383877959\\
24.62	0.00962384478050997\\
24.63	0.00962388574535373\\
24.64	0.00962392673333273\\
24.65	0.00962396774446886\\
24.66	0.00962400877878404\\
24.67	0.00962404983630026\\
24.68	0.00962409091703952\\
24.69	0.00962413202102389\\
24.7	0.00962417314827543\\
24.71	0.00962421429881629\\
24.72	0.00962425547266865\\
24.73	0.00962429666985471\\
24.74	0.00962433789039673\\
24.75	0.00962437913431702\\
24.76	0.0096244204016379\\
24.77	0.00962446169238176\\
24.78	0.00962450300657102\\
24.79	0.00962454434422816\\
24.8	0.00962458570537567\\
24.81	0.00962462709003611\\
24.82	0.00962466849823207\\
24.83	0.0096247099299862\\
24.84	0.00962475138532118\\
24.85	0.00962479286425974\\
24.86	0.00962483436682464\\
24.87	0.00962487589303871\\
24.88	0.0096249174429248\\
24.89	0.00962495901650583\\
24.9	0.00962500061380474\\
24.91	0.00962504223484454\\
24.92	0.00962508387964827\\
24.93	0.00962512554823903\\
24.94	0.00962516724063995\\
24.95	0.00962520895687423\\
24.96	0.00962525069696509\\
24.97	0.00962529246093583\\
24.98	0.00962533424880977\\
24.99	0.00962537606061028\\
25	0.00962541789636082\\
25.01	0.00962545975608485\\
25.02	0.00962550163980589\\
25.03	0.00962554354754755\\
25.04	0.00962558547933343\\
25.05	0.00962562743518724\\
25.06	0.00962566941513269\\
25.07	0.00962571141919358\\
25.08	0.00962575344739374\\
25.09	0.00962579549975706\\
25.1	0.0096258375763075\\
25.11	0.00962587967706905\\
25.12	0.00962592180206575\\
25.13	0.00962596395132173\\
25.14	0.00962600612486114\\
25.15	0.0096260483227082\\
25.16	0.00962609054488718\\
25.17	0.00962613279142243\\
25.18	0.00962617506233832\\
25.19	0.0096262173576593\\
25.2	0.00962625967740988\\
25.21	0.00962630202161463\\
25.22	0.00962634439029817\\
25.23	0.00962638678348517\\
25.24	0.00962642920120039\\
25.25	0.00962647164346862\\
25.26	0.00962651411031475\\
25.27	0.00962655660176369\\
25.28	0.00962659911784044\\
25.29	0.00962664165857006\\
25.3	0.00962668422397766\\
25.31	0.00962672681408843\\
25.32	0.00962676942892762\\
25.33	0.00962681206852056\\
25.34	0.00962685473289261\\
25.35	0.00962689742206925\\
25.36	0.00962694013607598\\
25.37	0.0096269828749384\\
25.38	0.00962702563868216\\
25.39	0.00962706842733301\\
25.4	0.00962711124091672\\
25.41	0.0096271540794592\\
25.42	0.00962719694298638\\
25.43	0.00962723983152428\\
25.44	0.009627282745099\\
25.45	0.00962732568373672\\
25.46	0.00962736864746367\\
25.47	0.00962741163630619\\
25.48	0.00962745465029068\\
25.49	0.00962749768944362\\
25.5	0.00962754075379159\\
25.51	0.00962758384336122\\
25.52	0.00962762695817924\\
25.53	0.00962767009827246\\
25.54	0.00962771326366779\\
25.55	0.0096277564543922\\
25.56	0.00962779967047275\\
25.57	0.00962784291193661\\
25.58	0.00962788617881102\\
25.59	0.0096279294711233\\
25.6	0.00962797278890089\\
25.61	0.0096280161321713\\
25.62	0.00962805950096215\\
25.63	0.00962810289530113\\
25.64	0.00962814631521605\\
25.65	0.00962818976073481\\
25.66	0.0096282332318854\\
25.67	0.00962827672869593\\
25.68	0.00962832025119459\\
25.69	0.00962836379940968\\
25.7	0.0096284073733696\\
25.71	0.00962845097310288\\
25.72	0.00962849459863812\\
25.73	0.00962853825000407\\
25.74	0.00962858192722954\\
25.75	0.0096286256303435\\
25.76	0.009628669359375\\
25.77	0.00962871311435323\\
25.78	0.00962875689530748\\
25.79	0.00962880070226717\\
25.8	0.00962884453526183\\
25.81	0.00962888839432112\\
25.82	0.00962893227947482\\
25.83	0.00962897619075284\\
25.84	0.00962902012818521\\
25.85	0.0096290640918021\\
25.86	0.00962910808163381\\
25.87	0.00962915209771077\\
25.88	0.00962919614006355\\
25.89	0.00962924020872286\\
25.9	0.00962928430371953\\
25.91	0.00962932842508459\\
25.92	0.00962937257284913\\
25.93	0.00962941674704447\\
25.94	0.00962946094770202\\
25.95	0.00962950517485339\\
25.96	0.00962954942853029\\
25.97	0.00962959370876464\\
25.98	0.00962963801558851\\
25.99	0.0096296823490341\\
26	0.00962972670913381\\
26.01	0.00962977109592019\\
26.02	0.00962981550942598\\
26.03	0.00962985994968407\\
26.04	0.00962990441672754\\
26.05	0.00962994891058967\\
26.06	0.00962999343130388\\
26.07	0.00963003797890383\\
26.08	0.00963008255342331\\
26.09	0.00963012715489635\\
26.1	0.00963017178335717\\
26.11	0.00963021643884016\\
26.12	0.00963026112137996\\
26.13	0.00963030583101139\\
26.14	0.00963035056776947\\
26.15	0.00963039533168948\\
26.16	0.00963044012280686\\
26.17	0.00963048494115733\\
26.18	0.0096305297867768\\
26.19	0.00963057465970143\\
26.2	0.0096306195599676\\
26.21	0.00963066448761194\\
26.22	0.00963070944267135\\
26.23	0.00963075442518293\\
26.24	0.00963079943518406\\
26.25	0.00963084447271239\\
26.26	0.00963088953780583\\
26.27	0.00963093463050254\\
26.28	0.00963097975084097\\
26.29	0.00963102489885985\\
26.3	0.00963107007459818\\
26.31	0.00963111527809528\\
26.32	0.00963116050939073\\
26.33	0.00963120576852446\\
26.34	0.00963125105553665\\
26.35	0.00963129637046784\\
26.36	0.00963134171335887\\
26.37	0.00963138708425092\\
26.38	0.00963143248318549\\
26.39	0.00963147791020442\\
26.4	0.0096315233653499\\
26.41	0.00963156884866449\\
26.42	0.00963161436019109\\
26.43	0.00963165989997298\\
26.44	0.0096317054680538\\
26.45	0.00963175106447761\\
26.46	0.00963179668928882\\
26.47	0.00963184234253226\\
26.48	0.00963188802425317\\
26.49	0.00963193373449719\\
26.5	0.00963197947331041\\
26.51	0.00963202524073934\\
26.52	0.00963207103683094\\
26.53	0.00963211686163261\\
26.54	0.00963216271519221\\
26.55	0.00963220859755809\\
26.56	0.00963225450877908\\
26.57	0.00963230044890448\\
26.58	0.00963234641798412\\
26.59	0.00963239241606832\\
26.6	0.00963243844320794\\
26.61	0.00963248449945436\\
26.62	0.00963253058485953\\
26.63	0.00963257669947593\\
26.64	0.00963262284335664\\
26.65	0.00963266901655531\\
26.66	0.00963271521912618\\
26.67	0.00963276145112409\\
26.68	0.00963280771260454\\
26.69	0.00963285400362364\\
26.7	0.00963290032423814\\
26.71	0.00963294667450548\\
26.72	0.00963299305448376\\
26.73	0.00963303946423177\\
26.74	0.00963308590380904\\
26.75	0.00963313237327579\\
26.76	0.00963317887269301\\
26.77	0.00963322540212243\\
26.78	0.00963327196162655\\
26.79	0.00963331855126869\\
26.8	0.00963336517111295\\
26.81	0.00963341182122428\\
26.82	0.00963345850166846\\
26.83	0.00963350521251216\\
26.84	0.0096335519538229\\
26.85	0.00963359872566916\\
26.86	0.00963364552812029\\
26.87	0.00963369236124663\\
26.88	0.00963373922511948\\
26.89	0.00963378611981112\\
26.9	0.00963383304539485\\
26.91	0.00963388000194503\\
26.92	0.00963392698953705\\
26.93	0.00963397400824742\\
26.94	0.00963402105815373\\
26.95	0.00963406813933475\\
26.96	0.00963411525187038\\
26.97	0.00963416239584171\\
26.98	0.00963420957133109\\
26.99	0.00963425677842209\\
27	0.00963430401719956\\
27.01	0.00963435128774966\\
27.02	0.0096343985901599\\
27.03	0.00963444592451915\\
27.04	0.00963449329091769\\
27.05	0.00963454068944724\\
27.06	0.00963458812020098\\
27.07	0.00963463558327362\\
27.08	0.00963468307876138\\
27.09	0.00963473060676208\\
27.1	0.00963477816737516\\
27.11	0.00963482576070171\\
27.12	0.00963487338684452\\
27.13	0.0096349210459081\\
27.14	0.00963496873799876\\
27.15	0.00963501646322462\\
27.16	0.00963506422169568\\
27.17	0.00963511201352383\\
27.18	0.00963515983882293\\
27.19	0.00963520769770885\\
27.2	0.0096352555902995\\
27.21	0.0096353035167149\\
27.22	0.00963535147707724\\
27.23	0.0096353994715109\\
27.24	0.00963544750014253\\
27.25	0.0096354955631011\\
27.26	0.00963554366051795\\
27.27	0.00963559179252686\\
27.28	0.0096356399592641\\
27.29	0.0096356881608685\\
27.3	0.0096357363974815\\
27.31	0.00963578466924725\\
27.32	0.00963583297631261\\
27.33	0.0096358813188273\\
27.34	0.00963592969694389\\
27.35	0.00963597811081797\\
27.36	0.00963602656060809\\
27.37	0.00963607504647598\\
27.38	0.00963612356858653\\
27.39	0.00963617212710789\\
27.4	0.00963622072221159\\
27.41	0.00963626935407257\\
27.42	0.00963631802286932\\
27.43	0.00963636672878392\\
27.44	0.00963641547200218\\
27.45	0.00963646425271368\\
27.46	0.00963651307111191\\
27.47	0.00963656192739436\\
27.48	0.0096366108217626\\
27.49	0.00963665975442241\\
27.5	0.00963670872558388\\
27.51	0.00963675773546151\\
27.52	0.00963680678427433\\
27.53	0.00963685587224603\\
27.54	0.00963690499960505\\
27.55	0.00963695416658472\\
27.56	0.00963700337342341\\
27.57	0.00963705262036459\\
27.58	0.00963710190765704\\
27.59	0.00963715123555495\\
27.6	0.00963720060431805\\
27.61	0.00963725001421178\\
27.62	0.00963729946550741\\
27.63	0.00963734895848223\\
27.64	0.00963739849341966\\
27.65	0.00963744807060944\\
27.66	0.00963749769034781\\
27.67	0.00963754735293762\\
27.68	0.00963759705868857\\
27.69	0.00963764680791736\\
27.7	0.00963769660094786\\
27.71	0.00963774643811132\\
27.72	0.00963779631974656\\
27.73	0.00963784624620016\\
27.74	0.00963789621782668\\
27.75	0.00963794623498885\\
27.76	0.0096379962980578\\
27.77	0.0096380464074133\\
27.78	0.00963809656344394\\
27.79	0.00963814676654742\\
27.8	0.00963819701713075\\
27.81	0.00963824731561051\\
27.82	0.00963829766241311\\
27.83	0.00963834805797508\\
27.84	0.00963839850274326\\
27.85	0.00963844899717517\\
27.86	0.00963849954173924\\
27.87	0.00963855013691509\\
27.88	0.00963860078319389\\
27.89	0.00963865148107863\\
27.9	0.00963870223108442\\
27.91	0.00963875303373886\\
27.92	0.00963880388958237\\
27.93	0.0096388547991685\\
27.94	0.0096389057630643\\
27.95	0.00963895678185074\\
27.96	0.00963900785612298\\
27.97	0.00963905898649088\\
27.98	0.00963911017357928\\
27.99	0.00963916141802849\\
28	0.00963921272049469\\
28.01	0.00963926408165035\\
28.02	0.00963931550218467\\
28.03	0.00963936698280406\\
28.04	0.0096394185242326\\
28.05	0.00963947012721252\\
28.06	0.0096395217925047\\
28.07	0.00963957352088919\\
28.08	0.00963962531316573\\
28.09	0.00963967717015428\\
28.1	0.00963972909269559\\
28.11	0.00963978108165178\\
28.12	0.00963983313790694\\
28.13	0.00963988526236767\\
28.14	0.00963993745596379\\
28.15	0.00963998971964895\\
28.16	0.00964004205440125\\
28.17	0.00964009446122398\\
28.18	0.00964014694114631\\
28.19	0.00964019949522395\\
28.2	0.00964025212453998\\
28.21	0.00964030483020553\\
28.22	0.00964035761336063\\
28.23	0.00964041047517496\\
28.24	0.00964046341684874\\
28.25	0.00964051643961352\\
28.26	0.00964056954473311\\
28.27	0.00964062273350446\\
28.28	0.00964067600725859\\
28.29	0.00964072936736157\\
28.3	0.00964078281521548\\
28.31	0.00964083635225945\\
28.32	0.00964088997997068\\
28.33	0.00964094369986552\\
28.34	0.00964099751350061\\
28.35	0.00964105142247398\\
28.36	0.00964110542842622\\
28.37	0.00964115953304174\\
28.38	0.00964121373804994\\
28.39	0.00964126804522657\\
28.4	0.00964132245639496\\
28.41	0.00964137697342745\\
28.42	0.00964143159824675\\
28.43	0.00964148633282737\\
28.44	0.00964154117919715\\
28.45	0.0096415961394387\\
28.46	0.00964165121569107\\
28.47	0.00964170641015125\\
28.48	0.00964176172507597\\
28.49	0.00964181716278327\\
28.5	0.00964187272565438\\
28.51	0.00964192841613547\\
28.52	0.00964198423673952\\
28.53	0.00964204019004831\\
28.54	0.00964209627871433\\
28.55	0.00964215250546284\\
28.56	0.00964220887309402\\
28.57	0.00964226538448506\\
28.58	0.00964232204259247\\
28.59	0.0096423788504543\\
28.6	0.00964243581119259\\
28.61	0.00964249292801574\\
28.62	0.00964255020422105\\
28.63	0.00964260764319731\\
28.64	0.00964266524842744\\
28.65	0.00964272302349129\\
28.66	0.00964278097206838\\
28.67	0.0096428390979409\\
28.68	0.00964289740499665\\
28.69	0.00964295589723216\\
28.7	0.00964301457875588\\
28.71	0.00964307345379143\\
28.72	0.009643132526681\\
28.73	0.00964319180188882\\
28.74	0.00964325128400478\\
28.75	0.00964331097774808\\
28.76	0.00964337088797107\\
28.77	0.00964343101966317\\
28.78	0.00964349137795489\\
28.79	0.00964355196812204\\
28.8	0.00964361279558996\\
28.81	0.00964367386593802\\
28.82	0.00964373518490412\\
28.83	0.0096437967583894\\
28.84	0.00964385859246311\\
28.85	0.00964392069336755\\
28.86	0.00964398306752326\\
28.87	0.00964404572153429\\
28.88	0.00964410866219366\\
28.89	0.009644171896489\\
28.9	0.00964423543160837\\
28.91	0.00964429927494617\\
28.92	0.00964436343410939\\
28.93	0.00964442791692389\\
28.94	0.009644492731441\\
28.95	0.00964455788594422\\
28.96	0.00964462338895622\\
28.97	0.00964468924924599\\
28.98	0.00964475547583622\\
28.99	0.00964482207801095\\
29	0.00964488906532338\\
29.01	0.00964495644760401\\
29.02	0.00964502423496897\\
29.03	0.00964509243782863\\
29.04	0.00964516106689647\\
29.05	0.00964523013319823\\
29.06	0.00964529964808131\\
29.07	0.00964536962322454\\
29.08	0.00964544007064821\\
29.09	0.00964551100272432\\
29.1	0.00964558243218737\\
29.11	0.00964565437214524\\
29.12	0.00964572683609054\\
29.13	0.00964579983791235\\
29.14	0.00964587339190822\\
29.15	0.00964594751279658\\
29.16	0.00964602221572962\\
29.17	0.00964609751630642\\
29.18	0.00964617343058661\\
29.19	0.00964624997510445\\
29.2	0.00964632716688327\\
29.21	0.00964640502345041\\
29.22	0.00964648356285269\\
29.23	0.00964656280367225\\
29.24	0.00964664276504294\\
29.25	0.00964672346666727\\
29.26	0.00964680492883381\\
29.27	0.00964688717243517\\
29.28	0.0096469702189866\\
29.29	0.00964705409064506\\
29.3	0.00964713881022899\\
29.31	0.00964722440123866\\
29.32	0.00964731088787717\\
29.33	0.00964739829507212\\
29.34	0.00964748664849794\\
29.35	0.00964757597459894\\
29.36	0.00964766630061309\\
29.37	0.00964775765459655\\
29.38	0.00964785006544898\\
29.39	0.00964794356293962\\
29.4	0.00964803817773427\\
29.41	0.00964813394142303\\
29.42	0.00964823088654896\\
29.43	0.0096483290466377\\
29.44	0.00964842845622794\\
29.45	0.00964852915090294\\
29.46	0.00964863116732295\\
29.47	0.00964873454325879\\
29.48	0.00964883931762639\\
29.49	0.00964894553052248\\
29.5	0.00964905322326143\\
29.51	0.00964916243841316\\
29.52	0.00964927321984246\\
29.53	0.00964938561274933\\
29.54	0.00964949966371077\\
29.55	0.00964961542072387\\
29.56	0.0096497314568792\\
29.57	0.00964984753078794\\
29.58	0.00964996364213344\\
29.59	0.00965007979059063\\
29.6	0.00965019597582593\\
29.61	0.00965031219749714\\
29.62	0.00965042845525335\\
29.63	0.00965054474873485\\
29.64	0.00965066107757302\\
29.65	0.0096507774413903\\
29.66	0.00965089383980005\\
29.67	0.00965101027240652\\
29.68	0.00965112673880476\\
29.69	0.00965124323858059\\
29.7	0.0096513597713105\\
29.71	0.00965147633656166\\
29.72	0.00965159293389181\\
29.73	0.00965170956284932\\
29.74	0.00965182622297309\\
29.75	0.00965194291379257\\
29.76	0.00965205963482777\\
29.77	0.00965217638558923\\
29.78	0.0096522931655781\\
29.79	0.00965240997428611\\
29.8	0.00965252681119565\\
29.81	0.00965264367577984\\
29.82	0.00965276056750258\\
29.83	0.00965287748581863\\
29.84	0.00965299443017376\\
29.85	0.00965311140000484\\
29.86	0.00965322839473996\\
29.87	0.00965334541379867\\
29.88	0.00965346245659206\\
29.89	0.00965357952252304\\
29.9	0.00965369661098654\\
29.91	0.00965381372136975\\
29.92	0.00965393085305239\\
29.93	0.00965404800540706\\
29.94	0.0096541651777995\\
29.95	0.00965428236958901\\
29.96	0.00965439958012879\\
29.97	0.0096545168087664\\
29.98	0.00965463405484421\\
29.99	0.0096547513176999\\
30	0.00965486859666693\\
30.01	0.00965498589107521\\
30.02	0.00965510320025165\\
30.03	0.00965522052352082\\
30.04	0.00965533786020569\\
30.05	0.00965545520962833\\
30.06	0.00965557257111076\\
30.07	0.00965568994397577\\
30.08	0.00965580732754784\\
30.09	0.00965592472115413\\
30.1	0.00965604212412549\\
30.11	0.00965615953579757\\
30.12	0.00965627695551193\\
30.13	0.00965639438261736\\
30.14	0.00965651181647112\\
30.15	0.00965662925644033\\
30.16	0.00965674670190344\\
30.17	0.0096568641522518\\
30.18	0.00965698160689124\\
30.19	0.00965709906524384\\
30.2	0.00965721652674974\\
30.21	0.00965733399086905\\
30.22	0.00965745145708388\\
30.23	0.00965756892490049\\
30.24	0.00965768639385154\\
30.25	0.00965780386349841\\
30.26	0.00965792133343377\\
30.27	0.00965803880328415\\
30.28	0.0096581562727127\\
30.29	0.00965827374142214\\
30.3	0.00965839120915774\\
30.31	0.00965850867571057\\
30.32	0.00965862614092086\\
30.33	0.0096587436046815\\
30.34	0.00965886106694178\\
30.35	0.00965897852771125\\
30.36	0.00965909598706384\\
30.37	0.00965921344514208\\
30.38	0.00965933090216164\\
30.39	0.00965944835841597\\
30.4	0.00965956581428128\\
30.41	0.00965968327022166\\
30.42	0.00965980072679448\\
30.43	0.00965991818465607\\
30.44	0.0096600356445676\\
30.45	0.00966015310740132\\
30.46	0.00966027057414698\\
30.47	0.00966038804591866\\
30.48	0.00966050552396183\\
30.49	0.00966062300966077\\
30.5	0.0096607405045463\\
30.51	0.00966085801030392\\
30.52	0.00966097552878223\\
30.53	0.00966109306200179\\
30.54	0.00966121061216436\\
30.55	0.00966132818166253\\
30.56	0.00966144577308982\\
30.57	0.0096615633892512\\
30.58	0.00966168103317401\\
30.59	0.00966179870811953\\
30.6	0.00966191641759483\\
30.61	0.00966203416536535\\
30.62	0.00966215195546786\\
30.63	0.00966226979013222\\
30.64	0.00966238766937752\\
30.65	0.00966250559322285\\
30.66	0.00966262356168734\\
30.67	0.00966274157479011\\
30.68	0.00966285963255027\\
30.69	0.00966297773498698\\
30.7	0.00966309588211938\\
30.71	0.00966321407396663\\
30.72	0.00966333231054792\\
30.73	0.0096634505918824\\
30.74	0.00966356891798929\\
30.75	0.00966368728888777\\
30.76	0.00966380570459706\\
30.77	0.00966392416513638\\
30.78	0.00966404267052496\\
30.79	0.00966416122078203\\
30.8	0.00966427981592686\\
30.81	0.0096643984559787\\
30.82	0.00966451714095683\\
30.83	0.00966463587088052\\
30.84	0.00966475464576907\\
30.85	0.00966487346564177\\
30.86	0.00966499233051794\\
30.87	0.0096651112404169\\
30.88	0.00966523019535797\\
30.89	0.00966534919536052\\
30.9	0.00966546824044387\\
30.91	0.00966558733062741\\
30.92	0.00966570646593049\\
30.93	0.0096658256463725\\
30.94	0.00966594487197284\\
30.95	0.0096660641427509\\
30.96	0.0096661834587261\\
30.97	0.00966630281991786\\
30.98	0.00966642222634562\\
30.99	0.00966654167802881\\
31	0.00966666117498689\\
31.01	0.00966678071723932\\
31.02	0.00966690030480559\\
31.03	0.00966701993770516\\
31.04	0.00966713961595754\\
31.05	0.00966725933958223\\
31.06	0.00966737910859874\\
31.07	0.0096674989230266\\
31.08	0.00966761878288535\\
31.09	0.00966773868819453\\
31.1	0.00966785863897369\\
31.11	0.0096679786352424\\
31.12	0.00966809867702024\\
31.13	0.0096682187643268\\
31.14	0.00966833889718166\\
31.15	0.00966845907560444\\
31.16	0.00966857929961475\\
31.17	0.00966869956923223\\
31.18	0.00966881988447651\\
31.19	0.00966894024536723\\
31.2	0.00966906065192406\\
31.21	0.00966918110416666\\
31.22	0.00966930160211472\\
31.23	0.00966942214578792\\
31.24	0.00966954273520597\\
31.25	0.00966966337038856\\
31.26	0.00966978405135543\\
31.27	0.00966990477812631\\
31.28	0.00967002555072093\\
31.29	0.00967014636915905\\
31.3	0.00967026723346043\\
31.31	0.00967038814364484\\
31.32	0.00967050909973207\\
31.33	0.0096706301017419\\
31.34	0.00967075114969415\\
31.35	0.00967087224360862\\
31.36	0.00967099338350515\\
31.37	0.00967111456940356\\
31.38	0.00967123580132369\\
31.39	0.00967135707928542\\
31.4	0.0096714784033086\\
31.41	0.0096715997734131\\
31.42	0.00967172118961882\\
31.43	0.00967184265194567\\
31.44	0.00967196416041353\\
31.45	0.00967208571504233\\
31.46	0.009672207315852\\
31.47	0.00967232896286249\\
31.48	0.00967245065609373\\
31.49	0.0096725723955657\\
31.5	0.00967269418129835\\
31.51	0.00967281601331169\\
31.52	0.00967293789162569\\
31.53	0.00967305981626036\\
31.54	0.00967318178723571\\
31.55	0.00967330380457177\\
31.56	0.00967342586828857\\
31.57	0.00967354797840616\\
31.58	0.00967367013494459\\
31.59	0.00967379233792393\\
31.6	0.00967391458736426\\
31.61	0.00967403688328565\\
31.62	0.00967415922570822\\
31.63	0.00967428161465208\\
31.64	0.00967440405013733\\
31.65	0.00967452653218411\\
31.66	0.00967464906081256\\
31.67	0.00967477163604283\\
31.68	0.00967489425789508\\
31.69	0.00967501692638949\\
31.7	0.00967513964154625\\
31.71	0.00967526240338553\\
31.72	0.00967538521192756\\
31.73	0.00967550806719253\\
31.74	0.00967563096920069\\
31.75	0.00967575391797226\\
31.76	0.0096758769135275\\
31.77	0.00967599995588666\\
31.78	0.00967612304507001\\
31.79	0.00967624618109783\\
31.8	0.00967636936399041\\
31.81	0.00967649259376805\\
31.82	0.00967661587045106\\
31.83	0.00967673919405978\\
31.84	0.00967686256461452\\
31.85	0.00967698598213564\\
31.86	0.00967710944664349\\
31.87	0.00967723295815843\\
31.88	0.00967735651670085\\
31.89	0.00967748012229112\\
31.9	0.00967760377494965\\
31.91	0.00967772747469684\\
31.92	0.00967785122155312\\
31.93	0.00967797501553892\\
31.94	0.00967809885667468\\
31.95	0.00967822274498084\\
31.96	0.00967834668047788\\
31.97	0.00967847066318627\\
31.98	0.00967859469312649\\
31.99	0.00967871877031904\\
32	0.00967884289478443\\
32.01	0.00967896706654316\\
32.02	0.00967909128561578\\
32.03	0.00967921555202282\\
32.04	0.00967933986578483\\
32.05	0.00967946422692237\\
32.06	0.00967958863545602\\
32.07	0.00967971309140636\\
32.08	0.00967983759479397\\
32.09	0.00967996214563947\\
32.1	0.00968008674396348\\
32.11	0.00968021138978661\\
32.12	0.00968033608312951\\
32.13	0.00968046082401283\\
32.14	0.00968058561245722\\
32.15	0.00968071044848337\\
32.16	0.00968083533211193\\
32.17	0.00968096026336362\\
32.18	0.00968108524225914\\
32.19	0.0096812102688192\\
32.2	0.00968133534306452\\
32.21	0.00968146046501585\\
32.22	0.00968158563469393\\
32.23	0.00968171085211951\\
32.24	0.00968183611731338\\
32.25	0.00968196143029632\\
32.26	0.0096820867910891\\
32.27	0.00968221219971254\\
32.28	0.00968233765618746\\
32.29	0.00968246316053467\\
32.3	0.00968258871277502\\
32.31	0.00968271431292935\\
32.32	0.00968283996101852\\
32.33	0.0096829656570634\\
32.34	0.00968309140108487\\
32.35	0.00968321719310383\\
32.36	0.00968334303314118\\
32.37	0.00968346892121782\\
32.38	0.0096835948573547\\
32.39	0.00968372084157274\\
32.4	0.0096838468738929\\
32.41	0.00968397295433613\\
32.42	0.0096840990829234\\
32.43	0.0096842252596757\\
32.44	0.00968435148461401\\
32.45	0.00968447775775935\\
32.46	0.00968460407913273\\
32.47	0.00968473044875517\\
32.48	0.00968485686664771\\
32.49	0.0096849833328314\\
32.5	0.00968510984732731\\
32.51	0.0096852364101565\\
32.52	0.00968536302134006\\
32.53	0.00968548968089907\\
32.54	0.00968561638885465\\
32.55	0.00968574314522792\\
32.56	0.00968586995003999\\
32.57	0.00968599680331202\\
32.58	0.00968612370506514\\
32.59	0.00968625065532054\\
32.6	0.00968637765409936\\
32.61	0.00968650470142281\\
32.62	0.00968663179731208\\
32.63	0.00968675894178836\\
32.64	0.0096868861348729\\
32.65	0.00968701337658691\\
32.66	0.00968714066695163\\
32.67	0.00968726800598832\\
32.68	0.00968739539371824\\
32.69	0.00968752283016267\\
32.7	0.0096876503153429\\
32.71	0.00968777784928022\\
32.72	0.00968790543199595\\
32.73	0.0096880330635114\\
32.74	0.00968816074384791\\
32.75	0.00968828847302682\\
32.76	0.00968841625106949\\
32.77	0.00968854407799729\\
32.78	0.00968867195383159\\
32.79	0.00968879987859378\\
32.8	0.00968892785230527\\
32.81	0.00968905587498747\\
32.82	0.00968918394666181\\
32.83	0.00968931206734972\\
32.84	0.00968944023707264\\
32.85	0.00968956845585204\\
32.86	0.00968969672370939\\
32.87	0.00968982504066617\\
32.88	0.00968995340674387\\
32.89	0.009690081821964\\
32.9	0.00969021028634808\\
32.91	0.00969033879991763\\
32.92	0.0096904673626942\\
32.93	0.00969059597469934\\
32.94	0.00969072463595461\\
32.95	0.00969085334648158\\
32.96	0.00969098210630185\\
32.97	0.009691110915437\\
32.98	0.00969123977390866\\
32.99	0.00969136868173844\\
33	0.00969149763894797\\
33.01	0.0096916266455589\\
33.02	0.00969175570159289\\
33.03	0.0096918848070716\\
33.04	0.00969201396201672\\
33.05	0.00969214316644993\\
33.06	0.00969227242039293\\
33.07	0.00969240172386746\\
33.08	0.00969253107689522\\
33.09	0.00969266047949796\\
33.1	0.00969278993169743\\
33.11	0.00969291943351539\\
33.12	0.00969304898497361\\
33.13	0.00969317858609389\\
33.14	0.009693308236898\\
33.15	0.00969343793740777\\
33.16	0.00969356768764502\\
33.17	0.00969369748763157\\
33.18	0.00969382733738927\\
33.19	0.00969395723693999\\
33.2	0.00969408718630557\\
33.21	0.00969421718550791\\
33.22	0.0096943472345689\\
33.23	0.00969447733351044\\
33.24	0.00969460748235444\\
33.25	0.00969473768112282\\
33.26	0.00969486792983755\\
33.27	0.00969499822852055\\
33.28	0.00969512857719379\\
33.29	0.00969525897587926\\
33.3	0.00969538942459892\\
33.31	0.00969551992337478\\
33.32	0.00969565047222886\\
33.33	0.00969578107118317\\
33.34	0.00969591172025975\\
33.35	0.00969604241948064\\
33.36	0.0096961731688679\\
33.37	0.0096963039684436\\
33.38	0.00969643481822983\\
33.39	0.00969656571824867\\
33.4	0.00969669666852223\\
33.41	0.00969682766907263\\
33.42	0.009696958719922\\
33.43	0.00969708982109247\\
33.44	0.00969722097260622\\
33.45	0.00969735217448539\\
33.46	0.00969748342675217\\
33.47	0.00969761472942874\\
33.48	0.00969774608253731\\
33.49	0.00969787748610009\\
33.5	0.00969800894013931\\
33.51	0.0096981404446772\\
33.52	0.00969827199973602\\
33.53	0.00969840360533803\\
33.54	0.00969853526150549\\
33.55	0.0096986669682607\\
33.56	0.00969879872562596\\
33.57	0.00969893053362357\\
33.58	0.00969906239227586\\
33.59	0.00969919430160516\\
33.6	0.00969932626163382\\
33.61	0.00969945827238419\\
33.62	0.00969959033387866\\
33.63	0.0096997224461396\\
33.64	0.0096998546091894\\
33.65	0.00969998682305048\\
33.66	0.00970011908774525\\
33.67	0.00970025140329615\\
33.68	0.00970038376972562\\
33.69	0.00970051618705612\\
33.7	0.00970064865531011\\
33.71	0.00970078117451008\\
33.72	0.00970091374467851\\
33.73	0.00970104636583792\\
33.74	0.00970117903801082\\
33.75	0.00970131176121974\\
33.76	0.00970144453548723\\
33.77	0.00970157736083583\\
33.78	0.00970171023728812\\
33.79	0.00970184316486667\\
33.8	0.00970197614359407\\
33.81	0.00970210917349293\\
33.82	0.00970224225458586\\
33.83	0.0097023753868955\\
33.84	0.00970250857044448\\
33.85	0.00970264180525545\\
33.86	0.00970277509135108\\
33.87	0.00970290842875405\\
33.88	0.00970304181748705\\
33.89	0.00970317525757277\\
33.9	0.00970330874903394\\
33.91	0.00970344229189328\\
33.92	0.00970357588617354\\
33.93	0.00970370953189746\\
33.94	0.0097038432290878\\
33.95	0.00970397697776735\\
33.96	0.00970411077795888\\
33.97	0.00970424462968522\\
33.98	0.00970437853296917\\
33.99	0.00970451248783355\\
34	0.0097046464943012\\
34.01	0.00970478055239498\\
34.02	0.00970491466213775\\
34.03	0.00970504882355239\\
34.04	0.00970518303666178\\
34.05	0.00970531730148883\\
34.06	0.00970545161805645\\
34.07	0.00970558598638757\\
34.08	0.00970572040650513\\
34.09	0.00970585487843207\\
34.1	0.00970598940219137\\
34.11	0.009706123977806\\
34.12	0.00970625860529895\\
34.13	0.00970639328469322\\
34.14	0.00970652801601183\\
34.15	0.0097066627992778\\
34.16	0.00970679763451417\\
34.17	0.009706932521744\\
34.18	0.00970706746099036\\
34.19	0.0097072024522763\\
34.2	0.00970733749562495\\
34.21	0.00970747259105938\\
34.22	0.00970760773860272\\
34.23	0.0097077429382781\\
34.24	0.00970787819010865\\
34.25	0.00970801349411754\\
34.26	0.00970814885032793\\
34.27	0.00970828425876299\\
34.28	0.00970841971944592\\
34.29	0.00970855523239993\\
34.3	0.00970869079764822\\
34.31	0.00970882641521404\\
34.32	0.00970896208512063\\
34.33	0.00970909780739123\\
34.34	0.00970923358204912\\
34.35	0.00970936940911759\\
34.36	0.00970950528861991\\
34.37	0.00970964122057941\\
34.38	0.0097097772050194\\
34.39	0.00970991324196321\\
34.4	0.00971004933143419\\
34.41	0.00971018547345569\\
34.42	0.0097103216680511\\
34.43	0.00971045791524379\\
34.44	0.00971059421505715\\
34.45	0.0097107305675146\\
34.46	0.00971086697263956\\
34.47	0.00971100343045547\\
34.48	0.00971113994098577\\
34.49	0.00971127650425393\\
34.5	0.00971141312028342\\
34.51	0.00971154978909772\\
34.52	0.00971168651072034\\
34.53	0.00971182328517479\\
34.54	0.00971196011248459\\
34.55	0.00971209699267329\\
34.56	0.00971223392576443\\
34.57	0.00971237091178159\\
34.58	0.00971250795074833\\
34.59	0.00971264504268825\\
34.6	0.00971278218762495\\
34.61	0.00971291938558206\\
34.62	0.00971305663658319\\
34.63	0.009713193940652\\
34.64	0.00971333129781214\\
34.65	0.00971346870808727\\
34.66	0.00971360617150109\\
34.67	0.00971374368807728\\
34.68	0.00971388125783954\\
34.69	0.00971401888081162\\
34.7	0.00971415655701724\\
34.71	0.00971429428648014\\
34.72	0.00971443206922409\\
34.73	0.00971456990527286\\
34.74	0.00971470779465024\\
34.75	0.00971484573738002\\
34.76	0.00971498373348602\\
34.77	0.00971512178299208\\
34.78	0.00971525988592202\\
34.79	0.00971539804229971\\
34.8	0.00971553625214899\\
34.81	0.00971567451549377\\
34.82	0.00971581283235792\\
34.83	0.00971595120276535\\
34.84	0.00971608962673999\\
34.85	0.00971622810430576\\
34.86	0.00971636663548661\\
34.87	0.0097165052203065\\
34.88	0.0097166438587894\\
34.89	0.00971678255095929\\
34.9	0.00971692129684018\\
34.91	0.00971706009645607\\
34.92	0.00971719894983099\\
34.93	0.00971733785698899\\
34.94	0.0097174768179541\\
34.95	0.0097176158327504\\
34.96	0.00971775490140196\\
34.97	0.00971789402393287\\
34.98	0.00971803320036726\\
34.99	0.00971817243072921\\
35	0.00971831171504288\\
35.01	0.00971845105333241\\
35.02	0.00971859044562195\\
35.03	0.00971872989193568\\
35.04	0.00971886939229777\\
35.05	0.00971900894673245\\
35.06	0.0097191485552639\\
35.07	0.00971928821791637\\
35.08	0.00971942793471408\\
35.09	0.0097195677056813\\
35.1	0.00971970753084228\\
35.11	0.00971984741022132\\
35.12	0.00971998734384269\\
35.13	0.00972012733173072\\
35.14	0.0097202673739097\\
35.15	0.009720407470404\\
35.16	0.00972054762123793\\
35.17	0.00972068782643589\\
35.18	0.00972082808602222\\
35.19	0.00972096840002132\\
35.2	0.0097211087684576\\
35.21	0.00972124919135547\\
35.22	0.00972138966873936\\
35.23	0.00972153020063371\\
35.24	0.00972167078706297\\
35.25	0.00972181142805162\\
35.26	0.00972195212362414\\
35.27	0.00972209287380502\\
35.28	0.00972223367861878\\
35.29	0.00972237453808993\\
35.3	0.00972251545224303\\
35.31	0.00972265642110262\\
35.32	0.00972279744469326\\
35.33	0.00972293852303954\\
35.34	0.00972307965616604\\
35.35	0.00972322084409738\\
35.36	0.00972336208685817\\
35.37	0.00972350338447304\\
35.38	0.00972364473696665\\
35.39	0.00972378614436365\\
35.4	0.00972392760668873\\
35.41	0.00972406912396656\\
35.42	0.00972421069622186\\
35.43	0.00972435232347934\\
35.44	0.00972449400576372\\
35.45	0.00972463574309976\\
35.46	0.00972477753551221\\
35.47	0.00972491938302584\\
35.48	0.00972506128566544\\
35.49	0.0097252032434558\\
35.5	0.00972534525642176\\
35.51	0.00972548732458811\\
35.52	0.00972562944797972\\
35.53	0.00972577162662143\\
35.54	0.00972591386053812\\
35.55	0.00972605614975466\\
35.56	0.00972619849429595\\
35.57	0.00972634089418691\\
35.58	0.00972648334945246\\
35.59	0.00972662586011753\\
35.6	0.00972676842620709\\
35.61	0.00972691104774609\\
35.62	0.00972705372475951\\
35.63	0.00972719645727236\\
35.64	0.00972733924530964\\
35.65	0.00972748208889638\\
35.66	0.0097276249880576\\
35.67	0.00972776794281836\\
35.68	0.00972791095320373\\
35.69	0.00972805401923878\\
35.7	0.0097281971409486\\
35.71	0.00972834031835831\\
35.72	0.00972848355149302\\
35.73	0.00972862684037788\\
35.74	0.00972877018503802\\
35.75	0.00972891358549861\\
35.76	0.00972905704178483\\
35.77	0.00972920055392188\\
35.78	0.00972934412193494\\
35.79	0.00972948774584926\\
35.8	0.00972963142569006\\
35.81	0.00972977516148258\\
35.82	0.0097299189532521\\
35.83	0.00973006280102388\\
35.84	0.00973020670482323\\
35.85	0.00973035066467544\\
35.86	0.00973049468060583\\
35.87	0.00973063875263974\\
35.88	0.00973078288080251\\
35.89	0.00973092706511951\\
35.9	0.00973107130561611\\
35.91	0.00973121560231771\\
35.92	0.0097313599552497\\
35.93	0.00973150436443752\\
35.94	0.00973164882990659\\
35.95	0.00973179335168235\\
35.96	0.00973193792979028\\
35.97	0.00973208256425584\\
35.98	0.00973222725510454\\
35.99	0.00973237200236187\\
36	0.00973251680605335\\
36.01	0.00973266166620452\\
36.02	0.00973280658284093\\
36.03	0.00973295155598814\\
36.04	0.00973309658567172\\
36.05	0.00973324167191728\\
36.06	0.00973338681475041\\
36.07	0.00973353201419673\\
36.08	0.00973367727028189\\
36.09	0.00973382258303152\\
36.1	0.00973396795247129\\
36.11	0.00973411337862689\\
36.12	0.00973425886152401\\
36.13	0.00973440440118835\\
36.14	0.00973454999764563\\
36.15	0.0097346956509216\\
36.16	0.009734841361042\\
36.17	0.00973498712803259\\
36.18	0.00973513295191917\\
36.19	0.00973527883272752\\
36.2	0.00973542477048345\\
36.21	0.00973557076521279\\
36.22	0.00973571681694137\\
36.23	0.00973586292569505\\
36.24	0.0097360090914997\\
36.25	0.0097361553143812\\
36.26	0.00973630159436544\\
36.27	0.00973644793147834\\
36.28	0.00973659432574583\\
36.29	0.00973674077719384\\
36.3	0.00973688728584833\\
36.31	0.00973703385173528\\
36.32	0.00973718047488066\\
36.33	0.00973732715531047\\
36.34	0.00973747389305074\\
36.35	0.00973762068812749\\
36.36	0.00973776754056677\\
36.37	0.00973791445039463\\
36.38	0.00973806141763715\\
36.39	0.00973820844232042\\
36.4	0.00973835552447054\\
36.41	0.00973850266411362\\
36.42	0.00973864986127581\\
36.43	0.00973879711598324\\
36.44	0.00973894442826209\\
36.45	0.00973909179813852\\
36.46	0.00973923922563874\\
36.47	0.00973938671078895\\
36.48	0.00973953425361537\\
36.49	0.00973968185414423\\
36.5	0.0097398295124018\\
36.51	0.00973997722841433\\
36.52	0.00974012500220811\\
36.53	0.00974027283380944\\
36.54	0.00974042072324462\\
36.55	0.00974056867053999\\
36.56	0.00974071667572188\\
36.57	0.00974086473881665\\
36.58	0.00974101285985068\\
36.59	0.00974116103885035\\
36.6	0.00974130927584206\\
36.61	0.00974145757085222\\
36.62	0.00974160592390728\\
36.63	0.00974175433503367\\
36.64	0.00974190280425786\\
36.65	0.00974205133160633\\
36.66	0.00974219991710557\\
36.67	0.00974234856078207\\
36.68	0.00974249726266237\\
36.69	0.00974264602277301\\
36.7	0.00974279484114053\\
36.71	0.00974294371779151\\
36.72	0.00974309265275252\\
36.73	0.00974324164605016\\
36.74	0.00974339069771105\\
36.75	0.00974353980776182\\
36.76	0.0097436889762291\\
36.77	0.00974383820313956\\
36.78	0.00974398748851987\\
36.79	0.00974413683239672\\
36.8	0.00974428623479681\\
36.81	0.00974443569574686\\
36.82	0.00974458521527361\\
36.83	0.00974473479340381\\
36.84	0.00974488443016423\\
36.85	0.00974503412558163\\
36.86	0.00974518387968282\\
36.87	0.00974533369249462\\
36.88	0.00974548356404384\\
36.89	0.00974563349435733\\
36.9	0.00974578348346194\\
36.91	0.00974593353138455\\
36.92	0.00974608363815205\\
36.93	0.00974623380379134\\
36.94	0.00974638402832933\\
36.95	0.00974653431179297\\
36.96	0.0097466846542092\\
36.97	0.00974683505560498\\
36.98	0.0097469855160073\\
36.99	0.00974713603544315\\
37	0.00974728661393955\\
37.01	0.00974743725152351\\
37.02	0.00974758794822209\\
37.03	0.00974773870406234\\
37.04	0.00974788951907133\\
37.05	0.00974804039327615\\
37.06	0.00974819132670391\\
37.07	0.00974834231938172\\
37.08	0.00974849337133673\\
37.09	0.00974864448259607\\
37.1	0.00974879565318693\\
37.11	0.00974894688313647\\
37.12	0.0097490981724719\\
37.13	0.00974924952122043\\
37.14	0.0097494009294093\\
37.15	0.00974955239706573\\
37.16	0.009749703924217\\
37.17	0.00974985551089038\\
37.18	0.00975000715711316\\
37.19	0.00975015886291265\\
37.2	0.00975031062831617\\
37.21	0.00975046245335106\\
37.22	0.00975061433804467\\
37.23	0.00975076628242437\\
37.24	0.00975091828651755\\
37.25	0.00975107035035161\\
37.26	0.00975122247395397\\
37.27	0.00975137465735206\\
37.28	0.00975152690057332\\
37.29	0.00975167920364523\\
37.3	0.00975183156659526\\
37.31	0.00975198398945091\\
37.32	0.0097521364722397\\
37.33	0.00975228901498913\\
37.34	0.00975244161772677\\
37.35	0.00975259428048017\\
37.36	0.00975274700327691\\
37.37	0.00975289978614457\\
37.38	0.00975305262911076\\
37.39	0.00975320553220312\\
37.4	0.00975335849544926\\
37.41	0.00975351151887686\\
37.42	0.00975366460251357\\
37.43	0.0097538177463871\\
37.44	0.00975397095052513\\
37.45	0.00975412421495539\\
37.46	0.00975427753970561\\
37.47	0.00975443092480355\\
37.48	0.00975458437027696\\
37.49	0.00975473787615364\\
37.5	0.00975489144246137\\
37.51	0.00975504506922798\\
37.52	0.00975519875648129\\
37.53	0.00975535250424915\\
37.54	0.00975550631255943\\
37.55	0.00975566018144\\
37.56	0.00975581411091876\\
37.57	0.00975596810102361\\
37.58	0.00975612215178249\\
37.59	0.00975627626322334\\
37.6	0.00975643043537412\\
37.61	0.00975658466826279\\
37.62	0.00975673896191737\\
37.63	0.00975689331636584\\
37.64	0.00975704773163624\\
37.65	0.00975720220775661\\
37.66	0.009757356744755\\
37.67	0.00975751134265949\\
37.68	0.00975766600149815\\
37.69	0.00975782072129911\\
37.7	0.00975797550209048\\
37.71	0.00975813034390039\\
37.72	0.009758285246757\\
37.73	0.00975844021068849\\
37.74	0.00975859523572303\\
37.75	0.00975875032188884\\
37.76	0.00975890546921412\\
37.77	0.00975906067772712\\
37.78	0.00975921594745609\\
37.79	0.00975937127842928\\
37.8	0.009759526670675\\
37.81	0.00975968212422154\\
37.82	0.00975983763909721\\
37.83	0.00975999321533036\\
37.84	0.00976014885294932\\
37.85	0.00976030455198247\\
37.86	0.0097604603124582\\
37.87	0.00976061613440488\\
37.88	0.00976077201785095\\
37.89	0.00976092796282484\\
37.9	0.00976108396935499\\
37.91	0.00976124003746987\\
37.92	0.00976139616719797\\
37.93	0.00976155235856777\\
37.94	0.0097617086116078\\
37.95	0.00976186492634658\\
37.96	0.00976202130281267\\
37.97	0.00976217774103462\\
37.98	0.00976233424104103\\
37.99	0.00976249080286047\\
38	0.00976264742652158\\
38.01	0.00976280411205298\\
38.02	0.00976296085948332\\
38.03	0.00976311766884125\\
38.04	0.00976327454015547\\
38.05	0.00976343147345467\\
38.06	0.00976358846876757\\
38.07	0.00976374552612289\\
38.08	0.00976390264554937\\
38.09	0.00976405982707579\\
38.1	0.00976421707073092\\
38.11	0.00976437437654358\\
38.12	0.00976453174454255\\
38.13	0.00976468917475668\\
38.14	0.00976484666721482\\
38.15	0.00976500422194582\\
38.16	0.00976516183897858\\
38.17	0.00976531951834198\\
38.18	0.00976547726006495\\
38.19	0.0097656350641764\\
38.2	0.0097657929307053\\
38.21	0.00976595085968061\\
38.22	0.0097661088511313\\
38.23	0.00976626690508638\\
38.24	0.00976642502157486\\
38.25	0.00976658320062578\\
38.26	0.00976674144226818\\
38.27	0.00976689974653114\\
38.28	0.00976705811344373\\
38.29	0.00976721654303505\\
38.3	0.00976737503533422\\
38.31	0.00976753359037037\\
38.32	0.00976769220817266\\
38.33	0.00976785088877026\\
38.34	0.00976800963219233\\
38.35	0.00976816843846811\\
38.36	0.00976832730762678\\
38.37	0.0097684862396976\\
38.38	0.00976864523470982\\
38.39	0.0097688042926927\\
38.4	0.00976896341367554\\
38.41	0.00976912259768763\\
38.42	0.0097692818447583\\
38.43	0.00976944115491688\\
38.44	0.00976960052819274\\
38.45	0.00976975996461524\\
38.46	0.00976991946421376\\
38.47	0.00977007902701773\\
38.48	0.00977023865305655\\
38.49	0.00977039834235968\\
38.5	0.00977055809495657\\
38.51	0.00977071791087669\\
38.52	0.00977087779014954\\
38.53	0.00977103773280462\\
38.54	0.00977119773887146\\
38.55	0.00977135780837961\\
38.56	0.00977151794135862\\
38.57	0.00977167813783808\\
38.58	0.00977183839784757\\
38.59	0.00977199872141671\\
38.6	0.00977215910857514\\
38.61	0.00977231955935249\\
38.62	0.00977248007377843\\
38.63	0.00977264065188264\\
38.64	0.00977280129369482\\
38.65	0.00977296199924469\\
38.66	0.00977312276856199\\
38.67	0.00977328360167645\\
38.68	0.00977344449861785\\
38.69	0.00977360545941597\\
38.7	0.00977376648410062\\
38.71	0.00977392757270162\\
38.72	0.00977408872524879\\
38.73	0.00977424994177201\\
38.74	0.00977441122230114\\
38.75	0.00977457256686606\\
38.76	0.00977473397549669\\
38.77	0.00977489544822295\\
38.78	0.00977505698507479\\
38.79	0.00977521858608215\\
38.8	0.00977538025127501\\
38.81	0.00977554198068339\\
38.82	0.00977570377433727\\
38.83	0.00977586563226669\\
38.84	0.0097760275545017\\
38.85	0.00977618954107236\\
38.86	0.00977635159200875\\
38.87	0.00977651370734098\\
38.88	0.00977667588709915\\
38.89	0.00977683813131341\\
38.9	0.00977700044001389\\
38.91	0.00977716281323078\\
38.92	0.00977732525099425\\
38.93	0.00977748775333452\\
38.94	0.0097776503202818\\
38.95	0.00977781295186633\\
38.96	0.00977797564811837\\
38.97	0.00977813840906819\\
38.98	0.0097783012347461\\
38.99	0.00977846412518238\\
39	0.00977862708040738\\
39.01	0.00977879010045143\\
39.02	0.0097789531853449\\
39.03	0.00977911633511818\\
39.04	0.00977927954980165\\
39.05	0.00977944282942573\\
39.06	0.00977960617402085\\
39.07	0.00977976958361747\\
39.08	0.00977993305824606\\
39.09	0.0097800965979371\\
39.1	0.00978026020272109\\
39.11	0.00978042387262856\\
39.12	0.00978058760769005\\
39.13	0.0097807514079361\\
39.14	0.0097809152733973\\
39.15	0.00978107920410425\\
39.16	0.00978124320008754\\
39.17	0.00978140726137781\\
39.18	0.0097815713880057\\
39.19	0.00978173558000188\\
39.2	0.00978189983739703\\
39.21	0.00978206416022184\\
39.22	0.00978222854850704\\
39.23	0.00978239300228336\\
39.24	0.00978255752158155\\
39.25	0.00978272210643238\\
39.26	0.00978288675686664\\
39.27	0.00978305147291514\\
39.28	0.0097832162546087\\
39.29	0.00978338110197815\\
39.3	0.00978354601505438\\
39.31	0.00978371099386823\\
39.32	0.00978387603845063\\
39.33	0.00978404114883247\\
39.34	0.00978420632504469\\
39.35	0.00978437156711824\\
39.36	0.00978453687508407\\
39.37	0.0097847022489732\\
39.38	0.0097848676888166\\
39.39	0.00978503319464531\\
39.4	0.00978519876649035\\
39.41	0.0097853644043828\\
39.42	0.00978553010835371\\
39.43	0.0097856958784342\\
39.44	0.00978586171465535\\
39.45	0.00978602761704832\\
39.46	0.00978619358564423\\
39.47	0.00978635962047425\\
39.48	0.00978652572156958\\
39.49	0.0097866918889614\\
39.5	0.00978685812268094\\
39.51	0.00978702442275943\\
39.52	0.00978719078922812\\
39.53	0.0097873572221183\\
39.54	0.00978752372146124\\
39.55	0.00978769028728827\\
39.56	0.00978785691963069\\
39.57	0.00978802361851987\\
39.58	0.00978819038398716\\
39.59	0.00978835721606394\\
39.6	0.0097885241147816\\
39.61	0.00978869108017159\\
39.62	0.00978885811226531\\
39.63	0.00978902521109423\\
39.64	0.00978919237668981\\
39.65	0.00978935960908355\\
39.66	0.00978952690830696\\
39.67	0.00978969427439155\\
39.68	0.00978986170736888\\
39.69	0.00979002920727051\\
39.7	0.009790196774128\\
39.71	0.00979036440797297\\
39.72	0.00979053210883703\\
39.73	0.00979069987675181\\
39.74	0.00979086771174896\\
39.75	0.00979103561386016\\
39.76	0.0097912035831171\\
39.77	0.00979137161955148\\
39.78	0.00979153972319502\\
39.79	0.00979170789407948\\
39.8	0.00979187613223661\\
39.81	0.00979204443769818\\
39.82	0.00979221281049601\\
39.83	0.00979238125066191\\
39.84	0.0097925497582277\\
39.85	0.00979271833322525\\
39.86	0.00979288697568643\\
39.87	0.00979305568564311\\
39.88	0.00979322446312722\\
39.89	0.00979339330817068\\
39.9	0.00979356222080543\\
39.91	0.00979373120106343\\
39.92	0.00979390024897666\\
39.93	0.00979406936457713\\
39.94	0.00979423854789685\\
39.95	0.00979440779896786\\
39.96	0.00979457711782221\\
39.97	0.00979474650449196\\
39.98	0.00979491595900923\\
39.99	0.0097950854814061\\
40	0.00979525507171471\\
40.01	0.00979542472996721\\
};
\addplot [color=mycolor1,solid,forget plot]
  table[row sep=crcr]{%
40.01	0.00979542472996721\\
40.02	0.00979559445619576\\
40.03	0.00979576425043254\\
40.04	0.00979593411270975\\
40.05	0.00979610404305962\\
40.06	0.00979627404151438\\
40.07	0.00979644410810628\\
40.08	0.00979661424286761\\
40.09	0.00979678444583065\\
40.1	0.00979695471702772\\
40.11	0.00979712505649114\\
40.12	0.00979729546425327\\
40.13	0.00979746594034647\\
40.14	0.00979763648480314\\
40.15	0.00979780709765566\\
40.16	0.00979797777893647\\
40.17	0.00979814852867801\\
40.18	0.00979831934691273\\
40.19	0.00979849023367313\\
40.2	0.00979866118899168\\
40.21	0.00979883221290092\\
40.22	0.00979900330543337\\
40.23	0.00979917446662158\\
40.24	0.00979934569649814\\
40.25	0.00979951699509563\\
40.26	0.00979968836244666\\
40.27	0.00979985979858385\\
40.28	0.00980003130353986\\
40.29	0.00980020287734735\\
40.3	0.009800374520039\\
40.31	0.00980054623164753\\
40.32	0.00980071801220564\\
40.33	0.00980088986174609\\
40.34	0.00980106178030163\\
40.35	0.00980123376790504\\
40.36	0.00980140582458912\\
40.37	0.00980157795038669\\
40.38	0.00980175014533058\\
40.39	0.00980192240945364\\
40.4	0.00980209474278876\\
40.41	0.00980226714536882\\
40.42	0.00980243961722674\\
40.43	0.00980261215839544\\
40.44	0.00980278476890789\\
40.45	0.00980295744879704\\
40.46	0.0098031301980959\\
40.47	0.00980330301683746\\
40.48	0.00980347590505475\\
40.49	0.00980364886278083\\
40.5	0.00980382189004875\\
40.51	0.00980399498689161\\
40.52	0.0098041681533425\\
40.53	0.00980434138943456\\
40.54	0.00980451469520094\\
40.55	0.00980468807067478\\
40.56	0.00980486151588928\\
40.57	0.00980503503087764\\
40.58	0.00980520861567309\\
40.59	0.00980538227030887\\
40.6	0.00980555599481824\\
40.61	0.00980572978923448\\
40.62	0.0098059036535909\\
40.63	0.00980607758792082\\
40.64	0.00980625159225759\\
40.65	0.00980642566663457\\
40.66	0.00980659981108513\\
40.67	0.00980677402564269\\
40.68	0.00980694831034067\\
40.69	0.00980712266521251\\
40.7	0.00980729709029169\\
40.71	0.00980747158561168\\
40.72	0.00980764615120599\\
40.73	0.00980782078710815\\
40.74	0.0098079954933517\\
40.75	0.00980817026997023\\
40.76	0.00980834511699731\\
40.77	0.00980852003446656\\
40.78	0.00980869502241161\\
40.79	0.0098088700808661\\
40.8	0.00980904520986373\\
40.81	0.00980922040943818\\
40.82	0.00980939567962317\\
40.83	0.00980957102045245\\
40.84	0.00980974643195976\\
40.85	0.0098099219141789\\
40.86	0.00981009746714367\\
40.87	0.00981027309088789\\
40.88	0.00981044878544542\\
40.89	0.00981062455085012\\
40.9	0.0098108003871359\\
40.91	0.00981097629433667\\
40.92	0.00981115227248636\\
40.93	0.00981132832161894\\
40.94	0.00981150444176839\\
40.95	0.00981168063296872\\
40.96	0.00981185689525397\\
40.97	0.00981203322865818\\
40.98	0.00981220963321544\\
40.99	0.00981238610895984\\
41	0.00981256265592552\\
41.01	0.00981273927414662\\
41.02	0.00981291596365731\\
41.03	0.0098130927244918\\
41.04	0.00981326955668431\\
41.05	0.00981344646026909\\
41.06	0.0098136234352804\\
41.07	0.00981380048175255\\
41.08	0.00981397759971986\\
41.09	0.00981415478921668\\
41.1	0.00981433205027738\\
41.11	0.00981450938293636\\
41.12	0.00981468678722805\\
41.13	0.00981486426318691\\
41.14	0.0098150418108474\\
41.15	0.00981521943024404\\
41.16	0.00981539712141136\\
41.17	0.00981557488438392\\
41.18	0.00981575271919631\\
41.19	0.00981593062588315\\
41.2	0.00981610860447907\\
41.21	0.00981628665501876\\
41.22	0.00981646477753691\\
41.23	0.00981664297206825\\
41.24	0.00981682123864755\\
41.25	0.00981699957730958\\
41.26	0.00981717798808917\\
41.27	0.00981735647102118\\
41.28	0.00981753502614047\\
41.29	0.00981771365348195\\
41.3	0.00981789235308057\\
41.31	0.0098180711249713\\
41.32	0.00981824996918914\\
41.33	0.00981842888576913\\
41.34	0.00981860787474634\\
41.35	0.00981878693615587\\
41.36	0.00981896607003286\\
41.37	0.00981914527641246\\
41.38	0.00981932455532989\\
41.39	0.00981950390682038\\
41.4	0.0098196833309192\\
41.41	0.00981986282766166\\
41.42	0.0098200423970831\\
41.43	0.0098202220392189\\
41.44	0.00982040175410446\\
41.45	0.00982058154177526\\
41.46	0.00982076140226677\\
41.47	0.00982094133561452\\
41.48	0.00982112134185406\\
41.49	0.00982130142102103\\
41.5	0.00982148157315104\\
41.51	0.0098216617982798\\
41.52	0.00982184209644301\\
41.53	0.00982202246767644\\
41.54	0.0098222029120159\\
41.55	0.00982238342949724\\
41.56	0.00982256402015634\\
41.57	0.00982274468402915\\
41.58	0.00982292542115163\\
41.59	0.00982310623155981\\
41.6	0.00982328711528976\\
41.61	0.00982346807237757\\
41.62	0.00982364910285942\\
41.63	0.00982383020677151\\
41.64	0.00982401138415008\\
41.65	0.00982419263503143\\
41.66	0.00982437395945192\\
41.67	0.00982455535744793\\
41.68	0.00982473682905592\\
41.69	0.00982491837431238\\
41.7	0.00982509999325386\\
41.71	0.00982528168591696\\
41.72	0.00982546345233834\\
41.73	0.00982564529255469\\
41.74	0.00982582720660278\\
41.75	0.00982600919451942\\
41.76	0.00982619125634149\\
41.77	0.0098263733921059\\
41.78	0.00982655560184965\\
41.79	0.00982673788560978\\
41.8	0.00982692024342338\\
41.81	0.00982710267532762\\
41.82	0.00982728518135971\\
41.83	0.00982746776155694\\
41.84	0.00982765041595664\\
41.85	0.00982783314459622\\
41.86	0.00982801594751316\\
41.87	0.00982819882474499\\
41.88	0.00982838177632929\\
41.89	0.00982856480230374\\
41.9	0.00982874790270607\\
41.91	0.00982893107757407\\
41.92	0.00982911432694562\\
41.93	0.00982929765085864\\
41.94	0.00982948104935116\\
41.95	0.00982966452246123\\
41.96	0.00982984807022702\\
41.97	0.00983003169268674\\
41.98	0.0098302153898787\\
41.99	0.00983039916184126\\
42	0.00983058300861288\\
42.01	0.00983076693023207\\
42.02	0.00983095092673743\\
42.03	0.00983113499816766\\
42.04	0.00983131914456149\\
42.05	0.00983150336595778\\
42.06	0.00983168766239543\\
42.07	0.00983187203391346\\
42.08	0.00983205648055095\\
42.09	0.00983224100234705\\
42.1	0.00983242559934103\\
42.11	0.00983261027157222\\
42.12	0.00983279501908004\\
42.13	0.009832979841904\\
42.14	0.0098331647400837\\
42.15	0.00983334971365882\\
42.16	0.00983353476266914\\
42.17	0.00983371988715452\\
42.18	0.00983390508715492\\
42.19	0.00983409036271039\\
42.2	0.00983427571386106\\
42.21	0.00983446114064716\\
42.22	0.00983464664310903\\
42.23	0.00983483222128707\\
42.24	0.00983501787522181\\
42.25	0.00983520360495385\\
42.26	0.0098353894105239\\
42.27	0.00983557529197276\\
42.28	0.00983576124934134\\
42.29	0.00983594728267062\\
42.3	0.0098361333920017\\
42.31	0.00983631957737578\\
42.32	0.00983650583883414\\
42.33	0.00983669217641817\\
42.34	0.00983687859016937\\
42.35	0.00983706508012932\\
42.36	0.0098372516463397\\
42.37	0.00983743828884231\\
42.38	0.00983762500767903\\
42.39	0.00983781180289185\\
42.4	0.00983799867452285\\
42.41	0.00983818562261422\\
42.42	0.00983837264720823\\
42.43	0.00983855974834728\\
42.44	0.00983874692607384\\
42.45	0.00983893418043049\\
42.46	0.00983912151145991\\
42.47	0.00983930891920486\\
42.48	0.00983949640370823\\
42.49	0.00983968396501296\\
42.5	0.00983987160316212\\
42.51	0.00984005931819886\\
42.52	0.00984024711016643\\
42.53	0.00984043497910816\\
42.54	0.00984062292506748\\
42.55	0.00984081094808789\\
42.56	0.00984099904821301\\
42.57	0.00984118722548652\\
42.58	0.00984137547995219\\
42.59	0.00984156381165387\\
42.6	0.00984175222063549\\
42.61	0.00984194070694107\\
42.62	0.0098421292706147\\
42.63	0.00984231791170053\\
42.64	0.00984250663024279\\
42.65	0.00984269542628579\\
42.66	0.00984288429987389\\
42.67	0.00984307325105153\\
42.68	0.00984326227986318\\
42.69	0.00984345138635339\\
42.7	0.00984364057056676\\
42.71	0.00984382983254794\\
42.72	0.00984401917234163\\
42.73	0.00984420858999256\\
42.74	0.00984439808554551\\
42.75	0.00984458765904528\\
42.76	0.00984477731053672\\
42.77	0.00984496704006469\\
42.78	0.00984515684767408\\
42.79	0.00984534673340979\\
42.8	0.00984553669731674\\
42.81	0.00984572673943985\\
42.82	0.00984591685982405\\
42.83	0.00984610705851425\\
42.84	0.00984629733555537\\
42.85	0.00984648769099229\\
42.86	0.0098466781248699\\
42.87	0.00984686863723305\\
42.88	0.00984705922812655\\
42.89	0.00984724989759518\\
42.9	0.00984744064568367\\
42.91	0.00984763147243671\\
42.92	0.00984782237789892\\
42.93	0.00984801336211486\\
42.94	0.00984820442512902\\
42.95	0.00984839556698582\\
42.96	0.00984858678772959\\
42.97	0.00984877808740456\\
42.98	0.00984896946605489\\
42.99	0.00984916092372461\\
43	0.00984935246045765\\
43.01	0.00984954407629783\\
43.02	0.00984973577128883\\
43.03	0.00984992754547424\\
43.04	0.00985011939889745\\
43.05	0.00985031133160178\\
43.06	0.00985050334363034\\
43.07	0.00985069543502615\\
43.08	0.00985088760583201\\
43.09	0.00985107985609062\\
43.1	0.00985127218584445\\
43.11	0.00985146459513586\\
43.12	0.009851657084007\\
43.13	0.00985184965249985\\
43.14	0.00985204230065622\\
43.15	0.00985223502851774\\
43.16	0.00985242783612585\\
43.17	0.00985262072352182\\
43.18	0.00985281369074673\\
43.19	0.0098530067378415\\
43.2	0.00985319986484687\\
43.21	0.00985339307180341\\
43.22	0.00985358635875151\\
43.23	0.00985377972573144\\
43.24	0.0098539731727833\\
43.25	0.00985416669994707\\
43.26	0.0098543603072626\\
43.27	0.00985455399476962\\
43.28	0.00985474776250789\\
43.29	0.0098549416105172\\
43.3	0.00985513553883736\\
43.31	0.0098553295475082\\
43.32	0.00985552363656958\\
43.33	0.00985571780606138\\
43.34	0.00985591205602353\\
43.35	0.00985610638649594\\
43.36	0.00985630079751859\\
43.37	0.00985649528913147\\
43.38	0.00985668986137459\\
43.39	0.00985688451428799\\
43.4	0.00985707924791172\\
43.41	0.0098572740622859\\
43.42	0.00985746895745062\\
43.43	0.00985766393344603\\
43.44	0.00985785899031232\\
43.45	0.00985805412808965\\
43.46	0.00985824934681827\\
43.47	0.0098584446465384\\
43.48	0.00985864002729033\\
43.49	0.00985883548911435\\
43.5	0.00985903103205078\\
43.51	0.00985922665613999\\
43.52	0.00985942236142233\\
43.53	0.00985961814793821\\
43.54	0.00985981401572806\\
43.55	0.00986000996483232\\
43.56	0.00986020599529149\\
43.57	0.00986040210714607\\
43.58	0.00986059830043658\\
43.59	0.00986079457520358\\
43.6	0.00986099093148767\\
43.61	0.00986118736932943\\
43.62	0.00986138388876952\\
43.63	0.00986158048984859\\
43.64	0.00986177717260733\\
43.65	0.00986197393708646\\
43.66	0.0098621707833267\\
43.67	0.00986236771136883\\
43.68	0.00986256472125365\\
43.69	0.00986276181302196\\
43.7	0.00986295898671461\\
43.71	0.00986315624237248\\
43.72	0.00986335358003645\\
43.73	0.00986355099974746\\
43.74	0.00986374850154645\\
43.75	0.00986394608547439\\
43.76	0.00986414375157229\\
43.77	0.00986434149988116\\
43.78	0.00986453933044208\\
43.79	0.00986473724329612\\
43.8	0.00986493523848438\\
43.81	0.00986513331604799\\
43.82	0.00986533147602812\\
43.83	0.00986552971846594\\
43.84	0.00986572804340268\\
43.85	0.00986592645087956\\
43.86	0.00986612494093785\\
43.87	0.00986632351361885\\
43.88	0.00986652216896386\\
43.89	0.00986672090701423\\
43.9	0.00986691972781133\\
43.91	0.00986711863139655\\
43.92	0.00986731761781132\\
43.93	0.00986751668709708\\
43.94	0.0098677158392953\\
43.95	0.0098679150744475\\
43.96	0.00986811439259518\\
43.97	0.00986831379377991\\
43.98	0.00986851327804327\\
43.99	0.00986871284542686\\
44	0.00986891249597231\\
44.01	0.00986911222972129\\
44.02	0.00986931204671547\\
44.03	0.00986951194699657\\
44.04	0.00986971193060632\\
44.05	0.00986991199758648\\
44.06	0.00987011214797886\\
44.07	0.00987031238182526\\
44.08	0.00987051269916752\\
44.09	0.00987071310004751\\
44.1	0.00987091358450714\\
44.11	0.00987111415258831\\
44.12	0.00987131480433298\\
44.13	0.00987151553978313\\
44.14	0.00987171635898074\\
44.15	0.00987191726196785\\
44.16	0.00987211824878651\\
44.17	0.00987231931947879\\
44.18	0.00987252047408681\\
44.19	0.00987272171265269\\
44.2	0.00987292303521859\\
44.21	0.00987312444182669\\
44.22	0.00987332593251921\\
44.23	0.00987352750733837\\
44.24	0.00987372916632645\\
44.25	0.00987393090952572\\
44.26	0.0098741327369785\\
44.27	0.00987433464872713\\
44.28	0.00987453664481398\\
44.29	0.00987473872528143\\
44.3	0.00987494089017191\\
44.31	0.00987514313952787\\
44.32	0.00987534547339176\\
44.33	0.00987554789180608\\
44.34	0.00987575039481336\\
44.35	0.00987595298245614\\
44.36	0.009876155654777\\
44.37	0.00987635841181854\\
44.38	0.00987656125362337\\
44.39	0.00987676418023416\\
44.4	0.00987696719169356\\
44.41	0.0098771702880443\\
44.42	0.00987737346932909\\
44.43	0.00987757673559069\\
44.44	0.00987778008687188\\
44.45	0.00987798352321545\\
44.46	0.00987818704466424\\
44.47	0.00987839065126111\\
44.48	0.00987859434304893\\
44.49	0.00987879812007061\\
44.5	0.00987900198236908\\
44.51	0.00987920592998729\\
44.52	0.00987940996296823\\
44.53	0.0098796140813549\\
44.54	0.00987981828519033\\
44.55	0.00988002257451759\\
44.56	0.00988022694937974\\
44.57	0.0098804314098199\\
44.58	0.00988063595588119\\
44.59	0.00988084058760678\\
44.6	0.00988104530503984\\
44.61	0.00988125010822357\\
44.62	0.00988145499720121\\
44.63	0.00988165997201601\\
44.64	0.00988186503271124\\
44.65	0.00988207017933021\\
44.66	0.00988227541191625\\
44.67	0.00988248073051271\\
44.68	0.00988268613516295\\
44.69	0.00988289162591039\\
44.7	0.00988309720279843\\
44.71	0.00988330286587054\\
44.72	0.00988350861517017\\
44.73	0.00988371445074082\\
44.74	0.00988392037262601\\
44.75	0.00988412638086928\\
44.76	0.00988433247551419\\
44.77	0.00988453865660434\\
44.78	0.00988474492418331\\
44.79	0.00988495127829476\\
44.8	0.00988515771898234\\
44.81	0.00988536424628973\\
44.82	0.00988557086026062\\
44.83	0.00988577756093874\\
44.84	0.00988598434836783\\
44.85	0.00988619122259167\\
44.86	0.00988639818365404\\
44.87	0.00988660523159876\\
44.88	0.00988681236646965\\
44.89	0.00988701958831057\\
44.9	0.00988722689716541\\
44.91	0.00988743429307805\\
44.92	0.00988764177609242\\
44.93	0.00988784934625245\\
44.94	0.00988805700360212\\
44.95	0.00988826474818539\\
44.96	0.00988847258004626\\
44.97	0.00988868049922877\\
44.98	0.00988888850577695\\
44.99	0.00988909659973487\\
45	0.00988930478114661\\
45.01	0.00988951305005627\\
45.02	0.00988972140650796\\
45.03	0.00988992985054583\\
45.04	0.00989013838221403\\
45.05	0.00989034700155675\\
45.06	0.00989055570861818\\
45.07	0.00989076450344253\\
45.08	0.00989097338607403\\
45.09	0.00989118235655693\\
45.1	0.00989139141493549\\
45.11	0.00989160056125401\\
45.12	0.00989180979555678\\
45.13	0.00989201911788812\\
45.14	0.00989222852829236\\
45.15	0.00989243802681385\\
45.16	0.00989264761349695\\
45.17	0.00989285728838605\\
45.18	0.00989306705152553\\
45.19	0.00989327690295982\\
45.2	0.00989348684273334\\
45.21	0.00989369687089052\\
45.22	0.00989390698747581\\
45.23	0.00989411719253369\\
45.24	0.00989432748610864\\
45.25	0.00989453786824513\\
45.26	0.00989474833898769\\
45.27	0.00989495889838082\\
45.28	0.00989516954646905\\
45.29	0.00989538028329693\\
45.3	0.009895591108909\\
45.31	0.00989580202334982\\
45.32	0.00989601302666396\\
45.33	0.00989622411889601\\
45.34	0.00989643530009056\\
45.35	0.00989664657029218\\
45.36	0.00989685792954551\\
45.37	0.00989706937789514\\
45.38	0.0098972809153857\\
45.39	0.00989749254206181\\
45.4	0.00989770425796811\\
45.41	0.00989791606314924\\
45.42	0.00989812795764984\\
45.43	0.00989833994151455\\
45.44	0.00989855201478804\\
45.45	0.00989876417751494\\
45.46	0.00989897642973993\\
45.47	0.00989918877150766\\
45.48	0.00989940120286279\\
45.49	0.00989961372384999\\
45.5	0.00989982633451391\\
45.51	0.00990003903489922\\
45.52	0.00990025182505058\\
45.53	0.00990046470501264\\
45.54	0.00990067767483007\\
45.55	0.00990089073454751\\
45.56	0.00990110388420962\\
45.57	0.00990131712386102\\
45.58	0.00990153045354638\\
45.59	0.0099017438733103\\
45.6	0.00990195738319742\\
45.61	0.00990217098325236\\
45.62	0.00990238467351971\\
45.63	0.00990259845404407\\
45.64	0.00990281232487003\\
45.65	0.00990302628604217\\
45.66	0.00990324033760503\\
45.67	0.00990345447960317\\
45.68	0.00990366871208112\\
45.69	0.00990388303508339\\
45.7	0.00990409744865448\\
45.71	0.00990431195283887\\
45.72	0.00990452654768102\\
45.73	0.00990474123322538\\
45.74	0.00990495600951636\\
45.75	0.00990517087659836\\
45.76	0.00990538583451575\\
45.77	0.00990560088331288\\
45.78	0.00990581602303407\\
45.79	0.00990603125372361\\
45.8	0.00990624657542577\\
45.81	0.00990646198818478\\
45.82	0.00990667749204484\\
45.83	0.00990689308705013\\
45.84	0.00990710877324477\\
45.85	0.00990732455067286\\
45.86	0.00990754041937847\\
45.87	0.0099077563794056\\
45.88	0.00990797243079826\\
45.89	0.00990818857360035\\
45.9	0.00990840480785579\\
45.91	0.00990862113360842\\
45.92	0.00990883755090203\\
45.93	0.00990905405978037\\
45.94	0.00990927066028714\\
45.95	0.00990948735246598\\
45.96	0.00990970413636048\\
45.97	0.00990992101201417\\
45.98	0.00991013797947051\\
45.99	0.00991035503877293\\
46	0.00991057218996476\\
46.01	0.00991078943308928\\
46.02	0.00991100676818971\\
46.03	0.00991122419530918\\
46.04	0.00991144171449076\\
46.05	0.00991165932577743\\
46.06	0.00991187702921212\\
46.07	0.00991209482483766\\
46.08	0.0099123127126968\\
46.09	0.00991253069283219\\
46.1	0.00991274876528643\\
46.11	0.00991296693010198\\
46.12	0.00991318518732125\\
46.13	0.00991340353698652\\
46.14	0.00991362197913999\\
46.15	0.00991384051382375\\
46.16	0.00991405914107978\\
46.17	0.00991427786094996\\
46.18	0.00991449667347605\\
46.19	0.00991471557869971\\
46.2	0.00991493457666244\\
46.21	0.00991515366740567\\
46.22	0.00991537285097067\\
46.23	0.00991559212739859\\
46.24	0.00991581149673044\\
46.25	0.00991603095900711\\
46.26	0.00991625051426932\\
46.27	0.00991647016255768\\
46.28	0.00991668990391262\\
46.29	0.00991690973837443\\
46.3	0.00991712966598325\\
46.31	0.00991734968677904\\
46.32	0.0099175698008016\\
46.33	0.00991779000809057\\
46.34	0.00991801030868539\\
46.35	0.00991823070262533\\
46.36	0.0099184511899495\\
46.37	0.00991867177069677\\
46.38	0.00991889244490585\\
46.39	0.00991911321261524\\
46.4	0.00991933407386323\\
46.41	0.00991955502868789\\
46.42	0.00991977607712707\\
46.43	0.00991999721921843\\
46.44	0.00992021845499935\\
46.45	0.00992043978450701\\
46.46	0.00992066120777833\\
46.47	0.00992088272484998\\
46.48	0.00992110433575838\\
46.49	0.00992132604053969\\
46.5	0.00992154783922979\\
46.51	0.00992176973186429\\
46.52	0.00992199171847852\\
46.53	0.0099222137991075\\
46.54	0.00992243597378597\\
46.55	0.00992265824254835\\
46.56	0.00992288060542875\\
46.57	0.00992310306246096\\
46.58	0.00992332561367842\\
46.59	0.00992354825911426\\
46.6	0.00992377099880123\\
46.61	0.00992399383277174\\
46.62	0.00992421676105782\\
46.63	0.00992443978369112\\
46.64	0.00992466290070293\\
46.65	0.0099248861121241\\
46.66	0.0099251094179851\\
46.67	0.00992533281831598\\
46.68	0.00992555631314635\\
46.69	0.00992577990250539\\
46.7	0.00992600358642181\\
46.71	0.00992622736492388\\
46.72	0.00992645123803938\\
46.73	0.0099266752057956\\
46.74	0.00992689926821935\\
46.75	0.00992712342533689\\
46.76	0.00992734767717399\\
46.77	0.00992757202375587\\
46.78	0.00992779646510717\\
46.79	0.00992802100125201\\
46.8	0.00992824563221388\\
46.81	0.00992847035801571\\
46.82	0.00992869517867979\\
46.83	0.00992892009422779\\
46.84	0.00992914510468074\\
46.85	0.00992937021005902\\
46.86	0.0099295954103823\\
46.87	0.00992982070566958\\
46.88	0.00993004609593913\\
46.89	0.00993027158120853\\
46.9	0.00993049716149455\\
46.91	0.00993072283681322\\
46.92	0.0099309486071798\\
46.93	0.00993117447260871\\
46.94	0.00993140043311355\\
46.95	0.00993162648870707\\
46.96	0.00993185263940117\\
46.97	0.00993207888520681\\
46.98	0.00993230522613407\\
46.99	0.00993253166219207\\
47	0.00993275819338897\\
47.01	0.00993298481973196\\
47.02	0.00993321154122717\\
47.03	0.00993343835787973\\
47.04	0.00993366526969369\\
47.05	0.009933892276672\\
47.06	0.0099341193788165\\
47.07	0.00993434657612786\\
47.08	0.00993457386860558\\
47.09	0.00993480125624796\\
47.1	0.00993502873905203\\
47.11	0.00993525631701359\\
47.12	0.00993548399012708\\
47.13	0.00993571175838565\\
47.14	0.00993593962178104\\
47.15	0.00993616758030361\\
47.16	0.00993639563394225\\
47.17	0.0099366237826844\\
47.18	0.00993685202651593\\
47.19	0.00993708036542121\\
47.2	0.00993730879938297\\
47.21	0.00993753732838232\\
47.22	0.00993776595239867\\
47.23	0.00993799467140974\\
47.24	0.00993822348539144\\
47.25	0.0099384523943179\\
47.26	0.00993868139816137\\
47.27	0.00993891049689218\\
47.28	0.00993913969047875\\
47.29	0.00993936897888743\\
47.3	0.00993959836208256\\
47.31	0.00993982784002633\\
47.32	0.0099400574126788\\
47.33	0.00994028707999777\\
47.34	0.0099405168419388\\
47.35	0.00994074669845507\\
47.36	0.00994097664949739\\
47.37	0.00994120669501409\\
47.38	0.009941436834951\\
47.39	0.00994166706925133\\
47.4	0.00994189739785565\\
47.41	0.0099421278207018\\
47.42	0.00994235833772484\\
47.43	0.00994258894885692\\
47.44	0.0099428196540273\\
47.45	0.00994305045316218\\
47.46	0.00994328134618469\\
47.47	0.00994351233301475\\
47.48	0.00994374341356905\\
47.49	0.00994397458776093\\
47.5	0.00994420585550028\\
47.51	0.00994443721669348\\
47.52	0.0099446686712433\\
47.53	0.00994490021904881\\
47.54	0.00994513186000525\\
47.55	0.00994536359400397\\
47.56	0.00994559542093232\\
47.57	0.00994582734067354\\
47.58	0.00994605935310664\\
47.59	0.00994629145810633\\
47.6	0.00994652365554284\\
47.61	0.00994675594528187\\
47.62	0.00994698832718443\\
47.63	0.00994722080110676\\
47.64	0.00994745336690012\\
47.65	0.00994768602441075\\
47.66	0.00994791877347972\\
47.67	0.00994815161394273\\
47.68	0.00994838454563005\\
47.69	0.00994861756836633\\
47.7	0.00994885068197046\\
47.71	0.00994908388625543\\
47.72	0.00994931718102814\\
47.73	0.00994955056608932\\
47.74	0.00994978404123325\\
47.75	0.00995001760624768\\
47.76	0.00995025126091364\\
47.77	0.00995048500500522\\
47.78	0.00995071883828944\\
47.79	0.00995095276052601\\
47.8	0.00995118677146721\\
47.81	0.0099514208708576\\
47.82	0.00995165505843387\\
47.83	0.00995188933392465\\
47.84	0.00995212369705022\\
47.85	0.00995235814752238\\
47.86	0.00995259268504414\\
47.87	0.00995282730930955\\
47.88	0.0099530620200034\\
47.89	0.00995329681680104\\
47.9	0.00995353169936807\\
47.91	0.0099537666673601\\
47.92	0.00995400172042249\\
47.93	0.00995423685819008\\
47.94	0.00995447208028685\\
47.95	0.00995470738632572\\
47.96	0.00995494277590818\\
47.97	0.00995517824862401\\
47.98	0.00995541380405097\\
47.99	0.00995564944175445\\
48	0.00995588516128718\\
48.01	0.00995612096218882\\
48.02	0.00995635684398567\\
48.03	0.0099565928061903\\
48.04	0.00995682884830113\\
48.05	0.00995706496980211\\
48.06	0.00995730117016227\\
48.07	0.00995753744883536\\
48.08	0.00995777380525941\\
48.09	0.0099580102388563\\
48.1	0.00995824674903134\\
48.11	0.00995848333517279\\
48.12	0.00995871999665144\\
48.13	0.00995895673282007\\
48.14	0.00995919354301302\\
48.15	0.00995943042654563\\
48.16	0.00995966738271378\\
48.17	0.0099599044107933\\
48.18	0.00996014151003945\\
48.19	0.00996037867968635\\
48.2	0.00996061591894641\\
48.21	0.0099608532270097\\
48.22	0.00996109060304338\\
48.23	0.00996132804619104\\
48.24	0.00996156555557204\\
48.25	0.00996180313028088\\
48.26	0.00996204076938649\\
48.27	0.00996227847193151\\
48.28	0.0099625162369316\\
48.29	0.00996275406337465\\
48.3	0.00996299195022005\\
48.31	0.00996322989639786\\
48.32	0.00996346790080804\\
48.33	0.00996370596231957\\
48.34	0.00996394407976964\\
48.35	0.0099641822519627\\
48.36	0.00996442047766961\\
48.37	0.00996465875562666\\
48.38	0.00996489708453463\\
48.39	0.00996513546305779\\
48.4	0.00996537388982287\\
48.41	0.00996561236341805\\
48.42	0.00996585088239181\\
48.43	0.0099660894452519\\
48.44	0.00996632805046411\\
48.45	0.00996656669645118\\
48.46	0.00996680538159153\\
48.47	0.00996704410421803\\
48.48	0.00996728286261673\\
48.49	0.00996752165502551\\
48.5	0.00996776047963277\\
48.51	0.009967999334576\\
48.52	0.00996823821794035\\
48.53	0.00996847712775712\\
48.54	0.00996871606200231\\
48.55	0.009968955018595\\
48.56	0.00996919399539571\\
48.57	0.00996943299020484\\
48.58	0.00996967200076083\\
48.59	0.00996991102473854\\
48.6	0.0099701500597473\\
48.61	0.00997038910332918\\
48.62	0.00997062815295696\\
48.63	0.00997086720603223\\
48.64	0.00997110625988333\\
48.65	0.00997134531176326\\
48.66	0.00997158435884751\\
48.67	0.00997182339823188\\
48.68	0.00997206242693018\\
48.69	0.00997230144187185\\
48.7	0.0099725404398996\\
48.71	0.00997277941776687\\
48.72	0.00997301837213528\\
48.73	0.00997325729957199\\
48.74	0.00997349619654701\\
48.75	0.00997373505943035\\
48.76	0.00997397388448916\\
48.77	0.00997421266788478\\
48.78	0.00997445140566965\\
48.79	0.0099746900937842\\
48.8	0.00997492872805357\\
48.81	0.00997516730418431\\
48.82	0.00997540581776091\\
48.83	0.00997564426424229\\
48.84	0.00997588263895812\\
48.85	0.00997612093710513\\
48.86	0.00997635915374314\\
48.87	0.00997659728379118\\
48.88	0.00997683532202335\\
48.89	0.00997707326306455\\
48.9	0.00997731110138622\\
48.91	0.00997754883130177\\
48.92	0.00997778644696203\\
48.93	0.00997802394235046\\
48.94	0.00997826131127824\\
48.95	0.00997849854737927\\
48.96	0.00997873564410492\\
48.97	0.00997897259471871\\
48.98	0.0099792093922908\\
48.99	0.00997944602969226\\
49	0.00997968249958927\\
49.01	0.00997991879443703\\
49.02	0.00998015490647362\\
49.03	0.00998039082771352\\
49.04	0.00998062654994107\\
49.05	0.00998086206470364\\
49.06	0.00998109736330463\\
49.07	0.0099813324367963\\
49.08	0.00998156727597223\\
49.09	0.00998180187135978\\
49.1	0.0099820362132121\\
49.11	0.00998227029150005\\
49.12	0.00998250409590381\\
49.13	0.00998273761580418\\
49.14	0.00998297084027377\\
49.15	0.00998320375806776\\
49.16	0.00998343635761445\\
49.17	0.00998366862700555\\
49.18	0.00998390055398607\\
49.19	0.00998413212594406\\
49.2	0.00998436332989985\\
49.21	0.0099845941524951\\
49.22	0.00998482457998147\\
49.23	0.0099850545982089\\
49.24	0.00998528419261362\\
49.25	0.00998551334820569\\
49.26	0.00998574204955623\\
49.27	0.00998597028078423\\
49.28	0.00998619802554294\\
49.29	0.00998642526700585\\
49.3	0.0099866519878523\\
49.31	0.0099868781702525\\
49.32	0.00998710379585223\\
49.33	0.00998732884575699\\
49.34	0.00998755330051571\\
49.35	0.00998777714010388\\
49.36	0.00998800034390623\\
49.37	0.00998822289069883\\
49.38	0.00998844475863068\\
49.39	0.00998866592520466\\
49.4	0.00998888636725796\\
49.41	0.00998910606094183\\
49.42	0.00998932498170078\\
49.43	0.00998954310425105\\
49.44	0.00998976040255848\\
49.45	0.00998997684981565\\
49.46	0.0099901924184183\\
49.47	0.00999040707994106\\
49.48	0.00999062080511235\\
49.49	0.0099908335637886\\
49.5	0.00999104532492754\\
49.51	0.00999125605656075\\
49.52	0.0099914657257653\\
49.53	0.00999167429863454\\
49.54	0.00999188174024795\\
49.55	0.00999208801464001\\
49.56	0.00999229308476818\\
49.57	0.00999249691247979\\
49.58	0.00999269945847795\\
49.59	0.00999290068228636\\
49.6	0.00999310054221297\\
49.61	0.00999329899531263\\
49.62	0.0099934959973484\\
49.63	0.00999369150275175\\
49.64	0.00999388546458147\\
49.65	0.00999407783448128\\
49.66	0.00999426856263614\\
49.67	0.00999445759772709\\
49.68	0.00999464488688478\\
49.69	0.00999483037564144\\
49.7	0.00999501400788138\\
49.71	0.00999519572578993\\
49.72	0.00999537546980067\\
49.73	0.00999555317854114\\
49.74	0.0099957287887767\\
49.75	0.00999590223535269\\
49.76	0.0099960734511347\\
49.77	0.00999624236694697\\
49.78	0.00999640891150885\\
49.79	0.00999657301136921\\
49.8	0.00999673459083878\\
49.81	0.00999689357192033\\
49.82	0.00999704987423663\\
49.83	0.00999720341495613\\
49.84	0.00999735410871627\\
49.85	0.00999750186754432\\
49.86	0.00999764660077572\\
49.87	0.00999778821496981\\
49.88	0.00999792661382287\\
49.89	0.00999806169807839\\
49.9	0.00999819336543446\\
49.91	0.0099983215104482\\
49.92	0.00999844602443718\\
49.93	0.0099985667953776\\
49.94	0.00999868370779928\\
49.95	0.00999879664267728\\
49.96	0.00999890547732003\\
49.97	0.00999901008525389\\
49.98	0.009999110336104\\
49.99	0.00999920609547131\\
50	0.00999929722480567\\
50.01	0.00999938358127485\\
50.02	0.00999946501762936\\
50.03	0.00999954138206288\\
50.04	0.00999961251806832\\
50.05	0.00999967826428909\\
50.06	0.00999973845436575\\
50.07	0.00999979291677763\\
50.08	0.00999984147467936\\
50.09	0.00999988394573218\\
50.1	0.0099999201419298\\
50.11	0.00999994986941864\\
50.12	0.00999997292831228\\
50.13	0.00999998911249995\\
50.14	0.00999999820944884\\
50.15	0.01\\
50.16	0.01\\
50.17	0.01\\
50.18	0.01\\
50.19	0.01\\
50.2	0.01\\
50.21	0.01\\
50.22	0.01\\
50.23	0.01\\
50.24	0.01\\
50.25	0.01\\
50.26	0.01\\
50.27	0.01\\
50.28	0.01\\
50.29	0.01\\
50.3	0.01\\
50.31	0.01\\
50.32	0.01\\
50.33	0.01\\
50.34	0.01\\
50.35	0.01\\
50.36	0.01\\
50.37	0.01\\
50.38	0.01\\
50.39	0.01\\
50.4	0.01\\
50.41	0.01\\
50.42	0.01\\
50.43	0.01\\
50.44	0.01\\
50.45	0.01\\
50.46	0.01\\
50.47	0.01\\
50.48	0.01\\
50.49	0.01\\
50.5	0.01\\
50.51	0.01\\
50.52	0.01\\
50.53	0.01\\
50.54	0.01\\
50.55	0.01\\
50.56	0.01\\
50.57	0.01\\
50.58	0.01\\
50.59	0.01\\
50.6	0.01\\
50.61	0.01\\
50.62	0.01\\
50.63	0.01\\
50.64	0.01\\
50.65	0.01\\
50.66	0.01\\
50.67	0.01\\
50.68	0.01\\
50.69	0.01\\
50.7	0.01\\
50.71	0.01\\
50.72	0.01\\
50.73	0.01\\
50.74	0.01\\
50.75	0.01\\
50.76	0.01\\
50.77	0.01\\
50.78	0.01\\
50.79	0.01\\
50.8	0.01\\
50.81	0.01\\
50.82	0.01\\
50.83	0.01\\
50.84	0.01\\
50.85	0.01\\
50.86	0.01\\
50.87	0.01\\
50.88	0.01\\
50.89	0.01\\
50.9	0.01\\
50.91	0.01\\
50.92	0.01\\
50.93	0.01\\
50.94	0.01\\
50.95	0.01\\
50.96	0.01\\
50.97	0.01\\
50.98	0.01\\
50.99	0.01\\
51	0.01\\
51.01	0.01\\
51.02	0.01\\
51.03	0.01\\
51.04	0.01\\
51.05	0.01\\
51.06	0.01\\
51.07	0.01\\
51.08	0.01\\
51.09	0.01\\
51.1	0.01\\
51.11	0.01\\
51.12	0.01\\
51.13	0.01\\
51.14	0.01\\
51.15	0.01\\
51.16	0.01\\
51.17	0.01\\
51.18	0.01\\
51.19	0.01\\
51.2	0.01\\
51.21	0.01\\
51.22	0.01\\
51.23	0.01\\
51.24	0.01\\
51.25	0.01\\
51.26	0.01\\
51.27	0.01\\
51.28	0.01\\
51.29	0.01\\
51.3	0.01\\
51.31	0.01\\
51.32	0.01\\
51.33	0.01\\
51.34	0.01\\
51.35	0.01\\
51.36	0.01\\
51.37	0.01\\
51.38	0.01\\
51.39	0.01\\
51.4	0.01\\
51.41	0.01\\
51.42	0.01\\
51.43	0.01\\
51.44	0.01\\
51.45	0.01\\
51.46	0.01\\
51.47	0.01\\
51.48	0.01\\
51.49	0.01\\
51.5	0.01\\
51.51	0.01\\
51.52	0.01\\
51.53	0.01\\
51.54	0.01\\
51.55	0.01\\
51.56	0.01\\
51.57	0.01\\
51.58	0.01\\
51.59	0.01\\
51.6	0.01\\
51.61	0.01\\
51.62	0.01\\
51.63	0.01\\
51.64	0.01\\
51.65	0.01\\
51.66	0.01\\
51.67	0.01\\
51.68	0.01\\
51.69	0.01\\
51.7	0.01\\
51.71	0.01\\
51.72	0.01\\
51.73	0.01\\
51.74	0.01\\
51.75	0.01\\
51.76	0.01\\
51.77	0.01\\
51.78	0.01\\
51.79	0.01\\
51.8	0.01\\
51.81	0.01\\
51.82	0.01\\
51.83	0.01\\
51.84	0.01\\
51.85	0.01\\
51.86	0.01\\
51.87	0.01\\
51.88	0.01\\
51.89	0.01\\
51.9	0.01\\
51.91	0.01\\
51.92	0.01\\
51.93	0.01\\
51.94	0.01\\
51.95	0.01\\
51.96	0.01\\
51.97	0.01\\
51.98	0.01\\
51.99	0.01\\
52	0.01\\
52.01	0.01\\
52.02	0.01\\
52.03	0.01\\
52.04	0.01\\
52.05	0.01\\
52.06	0.01\\
52.07	0.01\\
52.08	0.01\\
52.09	0.01\\
52.1	0.01\\
52.11	0.01\\
52.12	0.01\\
52.13	0.01\\
52.14	0.01\\
52.15	0.01\\
52.16	0.01\\
52.17	0.01\\
52.18	0.01\\
52.19	0.01\\
52.2	0.01\\
52.21	0.01\\
52.22	0.01\\
52.23	0.01\\
52.24	0.01\\
52.25	0.01\\
52.26	0.01\\
52.27	0.01\\
52.28	0.01\\
52.29	0.01\\
52.3	0.01\\
52.31	0.01\\
52.32	0.01\\
52.33	0.01\\
52.34	0.01\\
52.35	0.01\\
52.36	0.01\\
52.37	0.01\\
52.38	0.01\\
52.39	0.01\\
52.4	0.01\\
52.41	0.01\\
52.42	0.01\\
52.43	0.01\\
52.44	0.01\\
52.45	0.01\\
52.46	0.01\\
52.47	0.01\\
52.48	0.01\\
52.49	0.01\\
52.5	0.01\\
52.51	0.01\\
52.52	0.01\\
52.53	0.01\\
52.54	0.01\\
52.55	0.01\\
52.56	0.01\\
52.57	0.01\\
52.58	0.01\\
52.59	0.01\\
52.6	0.01\\
52.61	0.01\\
52.62	0.01\\
52.63	0.01\\
52.64	0.01\\
52.65	0.01\\
52.66	0.01\\
52.67	0.01\\
52.68	0.01\\
52.69	0.01\\
52.7	0.01\\
52.71	0.01\\
52.72	0.01\\
52.73	0.01\\
52.74	0.01\\
52.75	0.01\\
52.76	0.01\\
52.77	0.01\\
52.78	0.01\\
52.79	0.01\\
52.8	0.01\\
52.81	0.01\\
52.82	0.01\\
52.83	0.01\\
52.84	0.01\\
52.85	0.01\\
52.86	0.01\\
52.87	0.01\\
52.88	0.01\\
52.89	0.01\\
52.9	0.01\\
52.91	0.01\\
52.92	0.01\\
52.93	0.01\\
52.94	0.01\\
52.95	0.01\\
52.96	0.01\\
52.97	0.01\\
52.98	0.01\\
52.99	0.01\\
53	0.01\\
53.01	0.01\\
53.02	0.01\\
53.03	0.01\\
53.04	0.01\\
53.05	0.01\\
53.06	0.01\\
53.07	0.01\\
53.08	0.01\\
53.09	0.01\\
53.1	0.01\\
53.11	0.01\\
53.12	0.01\\
53.13	0.01\\
53.14	0.01\\
53.15	0.01\\
53.16	0.01\\
53.17	0.01\\
53.18	0.01\\
53.19	0.01\\
53.2	0.01\\
53.21	0.01\\
53.22	0.01\\
53.23	0.01\\
53.24	0.01\\
53.25	0.01\\
53.26	0.01\\
53.27	0.01\\
53.28	0.01\\
53.29	0.01\\
53.3	0.01\\
53.31	0.01\\
53.32	0.01\\
53.33	0.01\\
53.34	0.01\\
53.35	0.01\\
53.36	0.01\\
53.37	0.01\\
53.38	0.01\\
53.39	0.01\\
53.4	0.01\\
53.41	0.01\\
53.42	0.01\\
53.43	0.01\\
53.44	0.01\\
53.45	0.01\\
53.46	0.01\\
53.47	0.01\\
53.48	0.01\\
53.49	0.01\\
53.5	0.01\\
53.51	0.01\\
53.52	0.01\\
53.53	0.01\\
53.54	0.01\\
53.55	0.01\\
53.56	0.01\\
53.57	0.01\\
53.58	0.01\\
53.59	0.01\\
53.6	0.01\\
53.61	0.01\\
53.62	0.01\\
53.63	0.01\\
53.64	0.01\\
53.65	0.01\\
53.66	0.01\\
53.67	0.01\\
53.68	0.01\\
53.69	0.01\\
53.7	0.01\\
53.71	0.01\\
53.72	0.01\\
53.73	0.01\\
53.74	0.01\\
53.75	0.01\\
53.76	0.01\\
53.77	0.01\\
53.78	0.01\\
53.79	0.01\\
53.8	0.01\\
53.81	0.01\\
53.82	0.01\\
53.83	0.01\\
53.84	0.01\\
53.85	0.01\\
53.86	0.01\\
53.87	0.01\\
53.88	0.01\\
53.89	0.01\\
53.9	0.01\\
53.91	0.01\\
53.92	0.01\\
53.93	0.01\\
53.94	0.01\\
53.95	0.01\\
53.96	0.01\\
53.97	0.01\\
53.98	0.01\\
53.99	0.01\\
54	0.01\\
54.01	0.01\\
54.02	0.01\\
54.03	0.01\\
54.04	0.01\\
54.05	0.01\\
54.06	0.01\\
54.07	0.01\\
54.08	0.01\\
54.09	0.01\\
54.1	0.01\\
54.11	0.01\\
54.12	0.01\\
54.13	0.01\\
54.14	0.01\\
54.15	0.01\\
54.16	0.01\\
54.17	0.01\\
54.18	0.01\\
54.19	0.01\\
54.2	0.01\\
54.21	0.01\\
54.22	0.01\\
54.23	0.01\\
54.24	0.01\\
54.25	0.01\\
54.26	0.01\\
54.27	0.01\\
54.28	0.01\\
54.29	0.01\\
54.3	0.01\\
54.31	0.01\\
54.32	0.01\\
54.33	0.01\\
54.34	0.01\\
54.35	0.01\\
54.36	0.01\\
54.37	0.01\\
54.38	0.01\\
54.39	0.01\\
54.4	0.01\\
54.41	0.01\\
54.42	0.01\\
54.43	0.01\\
54.44	0.01\\
54.45	0.01\\
54.46	0.01\\
54.47	0.01\\
54.48	0.01\\
54.49	0.01\\
54.5	0.01\\
54.51	0.01\\
54.52	0.01\\
54.53	0.01\\
54.54	0.01\\
54.55	0.01\\
54.56	0.01\\
54.57	0.01\\
54.58	0.01\\
54.59	0.01\\
54.6	0.01\\
54.61	0.01\\
54.62	0.01\\
54.63	0.01\\
54.64	0.01\\
54.65	0.01\\
54.66	0.01\\
54.67	0.01\\
54.68	0.01\\
54.69	0.01\\
54.7	0.01\\
54.71	0.01\\
54.72	0.01\\
54.73	0.01\\
54.74	0.01\\
54.75	0.01\\
54.76	0.01\\
54.77	0.01\\
54.78	0.01\\
54.79	0.01\\
54.8	0.01\\
54.81	0.01\\
54.82	0.01\\
54.83	0.01\\
54.84	0.01\\
54.85	0.01\\
54.86	0.01\\
54.87	0.01\\
54.88	0.01\\
54.89	0.01\\
54.9	0.01\\
54.91	0.01\\
54.92	0.01\\
54.93	0.01\\
54.94	0.01\\
54.95	0.01\\
54.96	0.01\\
54.97	0.01\\
54.98	0.01\\
54.99	0.01\\
55	0.01\\
55.01	0.01\\
55.02	0.01\\
55.03	0.01\\
55.04	0.01\\
55.05	0.01\\
55.06	0.01\\
55.07	0.01\\
55.08	0.01\\
55.09	0.01\\
55.1	0.01\\
55.11	0.01\\
55.12	0.01\\
55.13	0.01\\
55.14	0.01\\
55.15	0.01\\
55.16	0.01\\
55.17	0.01\\
55.18	0.01\\
55.19	0.01\\
55.2	0.01\\
55.21	0.01\\
55.22	0.01\\
55.23	0.01\\
55.24	0.01\\
55.25	0.01\\
55.26	0.01\\
55.27	0.01\\
55.28	0.01\\
55.29	0.01\\
55.3	0.01\\
55.31	0.01\\
55.32	0.01\\
55.33	0.01\\
55.34	0.01\\
55.35	0.01\\
55.36	0.01\\
55.37	0.01\\
55.38	0.01\\
55.39	0.01\\
55.4	0.01\\
55.41	0.01\\
55.42	0.01\\
55.43	0.01\\
55.44	0.01\\
55.45	0.01\\
55.46	0.01\\
55.47	0.01\\
55.48	0.01\\
55.49	0.01\\
55.5	0.01\\
55.51	0.01\\
55.52	0.01\\
55.53	0.01\\
55.54	0.01\\
55.55	0.01\\
55.56	0.01\\
55.57	0.01\\
55.58	0.01\\
55.59	0.01\\
55.6	0.01\\
55.61	0.01\\
55.62	0.01\\
55.63	0.01\\
55.64	0.01\\
55.65	0.01\\
55.66	0.01\\
55.67	0.01\\
55.68	0.01\\
55.69	0.01\\
55.7	0.01\\
55.71	0.01\\
55.72	0.01\\
55.73	0.01\\
55.74	0.01\\
55.75	0.01\\
55.76	0.01\\
55.77	0.01\\
55.78	0.01\\
55.79	0.01\\
55.8	0.01\\
55.81	0.01\\
55.82	0.01\\
55.83	0.01\\
55.84	0.01\\
55.85	0.01\\
55.86	0.01\\
55.87	0.01\\
55.88	0.01\\
55.89	0.01\\
55.9	0.01\\
55.91	0.01\\
55.92	0.01\\
55.93	0.01\\
55.94	0.01\\
55.95	0.01\\
55.96	0.01\\
55.97	0.01\\
55.98	0.01\\
55.99	0.01\\
56	0.01\\
56.01	0.01\\
56.02	0.01\\
56.03	0.01\\
56.04	0.01\\
56.05	0.01\\
56.06	0.01\\
56.07	0.01\\
56.08	0.01\\
56.09	0.01\\
56.1	0.01\\
56.11	0.01\\
56.12	0.01\\
56.13	0.01\\
56.14	0.01\\
56.15	0.01\\
56.16	0.01\\
56.17	0.01\\
56.18	0.01\\
56.19	0.01\\
56.2	0.01\\
56.21	0.01\\
56.22	0.01\\
56.23	0.01\\
56.24	0.01\\
56.25	0.01\\
56.26	0.01\\
56.27	0.01\\
56.28	0.01\\
56.29	0.01\\
56.3	0.01\\
56.31	0.01\\
56.32	0.01\\
56.33	0.01\\
56.34	0.01\\
56.35	0.01\\
56.36	0.01\\
56.37	0.01\\
56.38	0.01\\
56.39	0.01\\
56.4	0.01\\
56.41	0.01\\
56.42	0.01\\
56.43	0.01\\
56.44	0.01\\
56.45	0.01\\
56.46	0.01\\
56.47	0.01\\
56.48	0.01\\
56.49	0.01\\
56.5	0.01\\
56.51	0.01\\
56.52	0.01\\
56.53	0.01\\
56.54	0.01\\
56.55	0.01\\
56.56	0.01\\
56.57	0.01\\
56.58	0.01\\
56.59	0.01\\
56.6	0.01\\
56.61	0.01\\
56.62	0.01\\
56.63	0.01\\
56.64	0.01\\
56.65	0.01\\
56.66	0.01\\
56.67	0.01\\
56.68	0.01\\
56.69	0.01\\
56.7	0.01\\
56.71	0.01\\
56.72	0.01\\
56.73	0.01\\
56.74	0.01\\
56.75	0.01\\
56.76	0.01\\
56.77	0.01\\
56.78	0.01\\
56.79	0.01\\
56.8	0.01\\
56.81	0.01\\
56.82	0.01\\
56.83	0.01\\
56.84	0.01\\
56.85	0.01\\
56.86	0.01\\
56.87	0.01\\
56.88	0.01\\
56.89	0.01\\
56.9	0.01\\
56.91	0.01\\
56.92	0.01\\
56.93	0.01\\
56.94	0.01\\
56.95	0.01\\
56.96	0.01\\
56.97	0.01\\
56.98	0.01\\
56.99	0.01\\
57	0.01\\
57.01	0.01\\
57.02	0.01\\
57.03	0.01\\
57.04	0.01\\
57.05	0.01\\
57.06	0.01\\
57.07	0.01\\
57.08	0.01\\
57.09	0.01\\
57.1	0.01\\
57.11	0.01\\
57.12	0.01\\
57.13	0.01\\
57.14	0.01\\
57.15	0.01\\
57.16	0.01\\
57.17	0.01\\
57.18	0.01\\
57.19	0.01\\
57.2	0.01\\
57.21	0.01\\
57.22	0.01\\
57.23	0.01\\
57.24	0.01\\
57.25	0.01\\
57.26	0.01\\
57.27	0.01\\
57.28	0.01\\
57.29	0.01\\
57.3	0.01\\
57.31	0.01\\
57.32	0.01\\
57.33	0.01\\
57.34	0.01\\
57.35	0.01\\
57.36	0.01\\
57.37	0.01\\
57.38	0.01\\
57.39	0.01\\
57.4	0.01\\
57.41	0.01\\
57.42	0.01\\
57.43	0.01\\
57.44	0.01\\
57.45	0.01\\
57.46	0.01\\
57.47	0.01\\
57.48	0.01\\
57.49	0.01\\
57.5	0.01\\
57.51	0.01\\
57.52	0.01\\
57.53	0.01\\
57.54	0.01\\
57.55	0.01\\
57.56	0.01\\
57.57	0.01\\
57.58	0.01\\
57.59	0.01\\
57.6	0.01\\
57.61	0.01\\
57.62	0.01\\
57.63	0.01\\
57.64	0.01\\
57.65	0.01\\
57.66	0.01\\
57.67	0.01\\
57.68	0.01\\
57.69	0.01\\
57.7	0.01\\
57.71	0.01\\
57.72	0.01\\
57.73	0.01\\
57.74	0.01\\
57.75	0.01\\
57.76	0.01\\
57.77	0.01\\
57.78	0.01\\
57.79	0.01\\
57.8	0.01\\
57.81	0.01\\
57.82	0.01\\
57.83	0.01\\
57.84	0.01\\
57.85	0.01\\
57.86	0.01\\
57.87	0.01\\
57.88	0.01\\
57.89	0.01\\
57.9	0.01\\
57.91	0.01\\
57.92	0.01\\
57.93	0.01\\
57.94	0.01\\
57.95	0.01\\
57.96	0.01\\
57.97	0.01\\
57.98	0.01\\
57.99	0.01\\
58	0.01\\
58.01	0.01\\
58.02	0.01\\
58.03	0.01\\
58.04	0.01\\
58.05	0.01\\
58.06	0.01\\
58.07	0.01\\
58.08	0.01\\
58.09	0.01\\
58.1	0.01\\
58.11	0.01\\
58.12	0.01\\
58.13	0.01\\
58.14	0.01\\
58.15	0.01\\
58.16	0.01\\
58.17	0.01\\
58.18	0.01\\
58.19	0.01\\
58.2	0.01\\
58.21	0.01\\
58.22	0.01\\
58.23	0.01\\
58.24	0.01\\
58.25	0.01\\
58.26	0.01\\
58.27	0.01\\
58.28	0.01\\
58.29	0.01\\
58.3	0.01\\
58.31	0.01\\
58.32	0.01\\
58.33	0.01\\
58.34	0.01\\
58.35	0.01\\
58.36	0.01\\
58.37	0.01\\
58.38	0.01\\
58.39	0.01\\
58.4	0.01\\
58.41	0.01\\
58.42	0.01\\
58.43	0.01\\
58.44	0.01\\
58.45	0.01\\
58.46	0.01\\
58.47	0.01\\
58.48	0.01\\
58.49	0.01\\
58.5	0.01\\
58.51	0.01\\
58.52	0.01\\
58.53	0.01\\
58.54	0.01\\
58.55	0.01\\
58.56	0.01\\
58.57	0.01\\
58.58	0.01\\
58.59	0.01\\
58.6	0.01\\
58.61	0.01\\
58.62	0.01\\
58.63	0.01\\
58.64	0.01\\
58.65	0.01\\
58.66	0.01\\
58.67	0.01\\
58.68	0.01\\
58.69	0.01\\
58.7	0.01\\
58.71	0.01\\
58.72	0.01\\
58.73	0.01\\
58.74	0.01\\
58.75	0.01\\
58.76	0.01\\
58.77	0.01\\
58.78	0.01\\
58.79	0.01\\
58.8	0.01\\
58.81	0.01\\
58.82	0.01\\
58.83	0.01\\
58.84	0.01\\
58.85	0.01\\
58.86	0.01\\
58.87	0.01\\
58.88	0.01\\
58.89	0.01\\
58.9	0.01\\
58.91	0.01\\
58.92	0.01\\
58.93	0.01\\
58.94	0.01\\
58.95	0.01\\
58.96	0.01\\
58.97	0.01\\
58.98	0.01\\
58.99	0.01\\
59	0.01\\
59.01	0.01\\
59.02	0.01\\
59.03	0.01\\
59.04	0.01\\
59.05	0.01\\
59.06	0.01\\
59.07	0.01\\
59.08	0.01\\
59.09	0.01\\
59.1	0.01\\
59.11	0.01\\
59.12	0.01\\
59.13	0.01\\
59.14	0.01\\
59.15	0.01\\
59.16	0.01\\
59.17	0.01\\
59.18	0.01\\
59.19	0.01\\
59.2	0.01\\
59.21	0.01\\
59.22	0.01\\
59.23	0.01\\
59.24	0.01\\
59.25	0.01\\
59.26	0.01\\
59.27	0.01\\
59.28	0.01\\
59.29	0.01\\
59.3	0.01\\
59.31	0.01\\
59.32	0.01\\
59.33	0.01\\
59.34	0.01\\
59.35	0.01\\
59.36	0.01\\
59.37	0.01\\
59.38	0.01\\
59.39	0.01\\
59.4	0.01\\
59.41	0.01\\
59.42	0.01\\
59.43	0.01\\
59.44	0.01\\
59.45	0.01\\
59.46	0.01\\
59.47	0.01\\
59.48	0.01\\
59.49	0.01\\
59.5	0.01\\
59.51	0.01\\
59.52	0.01\\
59.53	0.01\\
59.54	0.01\\
59.55	0.01\\
59.56	0.01\\
59.57	0.01\\
59.58	0.01\\
59.59	0.01\\
59.6	0.01\\
59.61	0.01\\
59.62	0.01\\
59.63	0.01\\
59.64	0.01\\
59.65	0.01\\
59.66	0.01\\
59.67	0.01\\
59.68	0.01\\
59.69	0.01\\
59.7	0.01\\
59.71	0.01\\
59.72	0.01\\
59.73	0.01\\
59.74	0.01\\
59.75	0.01\\
59.76	0.01\\
59.77	0.01\\
59.78	0.01\\
59.79	0.01\\
59.8	0.01\\
59.81	0.01\\
59.82	0.01\\
59.83	0.01\\
59.84	0.01\\
59.85	0.01\\
59.86	0.01\\
59.87	0.01\\
59.88	0.01\\
59.89	0.01\\
59.9	0.01\\
59.91	0.01\\
59.92	0.01\\
59.93	0.01\\
59.94	0.01\\
59.95	0.01\\
59.96	0.01\\
59.97	0.01\\
59.98	0.01\\
59.99	0.01\\
60	0.01\\
60.01	0.01\\
60.02	0.01\\
60.03	0.01\\
60.04	0.01\\
60.05	0.01\\
60.06	0.01\\
60.07	0.01\\
60.08	0.01\\
60.09	0.01\\
60.1	0.01\\
60.11	0.01\\
60.12	0.01\\
60.13	0.01\\
60.14	0.01\\
60.15	0.01\\
60.16	0.01\\
60.17	0.01\\
60.18	0.01\\
60.19	0.01\\
60.2	0.01\\
60.21	0.01\\
60.22	0.01\\
60.23	0.01\\
60.24	0.01\\
60.25	0.01\\
60.26	0.01\\
60.27	0.01\\
60.28	0.01\\
60.29	0.01\\
60.3	0.01\\
60.31	0.01\\
60.32	0.01\\
60.33	0.01\\
60.34	0.01\\
60.35	0.01\\
60.36	0.01\\
60.37	0.01\\
60.38	0.01\\
60.39	0.01\\
60.4	0.01\\
60.41	0.01\\
60.42	0.01\\
60.43	0.01\\
60.44	0.01\\
60.45	0.01\\
60.46	0.01\\
60.47	0.01\\
60.48	0.01\\
60.49	0.01\\
60.5	0.01\\
60.51	0.01\\
60.52	0.01\\
60.53	0.01\\
60.54	0.01\\
60.55	0.01\\
60.56	0.01\\
60.57	0.01\\
60.58	0.01\\
60.59	0.01\\
60.6	0.01\\
60.61	0.01\\
60.62	0.01\\
60.63	0.01\\
60.64	0.01\\
60.65	0.01\\
60.66	0.01\\
60.67	0.01\\
60.68	0.01\\
60.69	0.01\\
60.7	0.01\\
60.71	0.01\\
60.72	0.01\\
60.73	0.01\\
60.74	0.01\\
60.75	0.01\\
60.76	0.01\\
60.77	0.01\\
60.78	0.01\\
60.79	0.01\\
60.8	0.01\\
60.81	0.01\\
60.82	0.01\\
60.83	0.01\\
60.84	0.01\\
60.85	0.01\\
60.86	0.01\\
60.87	0.01\\
60.88	0.01\\
60.89	0.01\\
60.9	0.01\\
60.91	0.01\\
60.92	0.01\\
60.93	0.01\\
60.94	0.01\\
60.95	0.01\\
60.96	0.01\\
60.97	0.01\\
60.98	0.01\\
60.99	0.01\\
61	0.01\\
61.01	0.01\\
61.02	0.01\\
61.03	0.01\\
61.04	0.01\\
61.05	0.01\\
61.06	0.01\\
61.07	0.01\\
61.08	0.01\\
61.09	0.01\\
61.1	0.01\\
61.11	0.01\\
61.12	0.01\\
61.13	0.01\\
61.14	0.01\\
61.15	0.01\\
61.16	0.01\\
61.17	0.01\\
61.18	0.01\\
61.19	0.01\\
61.2	0.01\\
61.21	0.01\\
61.22	0.01\\
61.23	0.01\\
61.24	0.01\\
61.25	0.01\\
61.26	0.01\\
61.27	0.01\\
61.28	0.01\\
61.29	0.01\\
61.3	0.01\\
61.31	0.01\\
61.32	0.01\\
61.33	0.01\\
61.34	0.01\\
61.35	0.01\\
61.36	0.01\\
61.37	0.01\\
61.38	0.01\\
61.39	0.01\\
61.4	0.01\\
61.41	0.01\\
61.42	0.01\\
61.43	0.01\\
61.44	0.01\\
61.45	0.01\\
61.46	0.01\\
61.47	0.01\\
61.48	0.01\\
61.49	0.01\\
61.5	0.01\\
61.51	0.01\\
61.52	0.01\\
61.53	0.01\\
61.54	0.01\\
61.55	0.01\\
61.56	0.01\\
61.57	0.01\\
61.58	0.01\\
61.59	0.01\\
61.6	0.01\\
61.61	0.01\\
61.62	0.01\\
61.63	0.01\\
61.64	0.01\\
61.65	0.01\\
61.66	0.01\\
61.67	0.01\\
61.68	0.01\\
61.69	0.01\\
61.7	0.01\\
61.71	0.01\\
61.72	0.01\\
61.73	0.01\\
61.74	0.01\\
61.75	0.01\\
61.76	0.01\\
61.77	0.01\\
61.78	0.01\\
61.79	0.01\\
61.8	0.01\\
61.81	0.01\\
61.82	0.01\\
61.83	0.01\\
61.84	0.01\\
61.85	0.01\\
61.86	0.01\\
61.87	0.01\\
61.88	0.01\\
61.89	0.01\\
61.9	0.01\\
61.91	0.01\\
61.92	0.01\\
61.93	0.01\\
61.94	0.01\\
61.95	0.01\\
61.96	0.01\\
61.97	0.01\\
61.98	0.01\\
61.99	0.01\\
62	0.01\\
62.01	0.01\\
62.02	0.01\\
62.03	0.01\\
62.04	0.01\\
62.05	0.01\\
62.06	0.01\\
62.07	0.01\\
62.08	0.01\\
62.09	0.01\\
62.1	0.01\\
62.11	0.01\\
62.12	0.01\\
62.13	0.01\\
62.14	0.01\\
62.15	0.01\\
62.16	0.01\\
62.17	0.01\\
62.18	0.01\\
62.19	0.01\\
62.2	0.01\\
62.21	0.01\\
62.22	0.01\\
62.23	0.01\\
62.24	0.01\\
62.25	0.01\\
62.26	0.01\\
62.27	0.01\\
62.28	0.01\\
62.29	0.01\\
62.3	0.01\\
62.31	0.01\\
62.32	0.01\\
62.33	0.01\\
62.34	0.01\\
62.35	0.01\\
62.36	0.01\\
62.37	0.01\\
62.38	0.01\\
62.39	0.01\\
62.4	0.01\\
62.41	0.01\\
62.42	0.01\\
62.43	0.01\\
62.44	0.01\\
62.45	0.01\\
62.46	0.01\\
62.47	0.01\\
62.48	0.01\\
62.49	0.01\\
62.5	0.01\\
62.51	0.01\\
62.52	0.01\\
62.53	0.01\\
62.54	0.01\\
62.55	0.01\\
62.56	0.01\\
62.57	0.01\\
62.58	0.01\\
62.59	0.01\\
62.6	0.01\\
62.61	0.01\\
62.62	0.01\\
62.63	0.01\\
62.64	0.01\\
62.65	0.01\\
62.66	0.01\\
62.67	0.01\\
62.68	0.01\\
62.69	0.01\\
62.7	0.01\\
62.71	0.01\\
62.72	0.01\\
62.73	0.01\\
62.74	0.01\\
62.75	0.01\\
62.76	0.01\\
62.77	0.01\\
62.78	0.01\\
62.79	0.01\\
62.8	0.01\\
62.81	0.01\\
62.82	0.01\\
62.83	0.01\\
62.84	0.01\\
62.85	0.01\\
62.86	0.01\\
62.87	0.01\\
62.88	0.01\\
62.89	0.01\\
62.9	0.01\\
62.91	0.01\\
62.92	0.01\\
62.93	0.01\\
62.94	0.01\\
62.95	0.01\\
62.96	0.01\\
62.97	0.01\\
62.98	0.01\\
62.99	0.01\\
63	0.01\\
63.01	0.01\\
63.02	0.01\\
63.03	0.01\\
63.04	0.01\\
63.05	0.01\\
63.06	0.01\\
63.07	0.01\\
63.08	0.01\\
63.09	0.01\\
63.1	0.01\\
63.11	0.01\\
63.12	0.01\\
63.13	0.01\\
63.14	0.01\\
63.15	0.01\\
63.16	0.01\\
63.17	0.01\\
63.18	0.01\\
63.19	0.01\\
63.2	0.01\\
63.21	0.01\\
63.22	0.01\\
63.23	0.01\\
63.24	0.01\\
63.25	0.01\\
63.26	0.01\\
63.27	0.01\\
63.28	0.01\\
63.29	0.01\\
63.3	0.01\\
63.31	0.01\\
63.32	0.01\\
63.33	0.01\\
63.34	0.01\\
63.35	0.01\\
63.36	0.01\\
63.37	0.01\\
63.38	0.01\\
63.39	0.01\\
63.4	0.01\\
63.41	0.01\\
63.42	0.01\\
63.43	0.01\\
63.44	0.01\\
63.45	0.01\\
63.46	0.01\\
63.47	0.01\\
63.48	0.01\\
63.49	0.01\\
63.5	0.01\\
63.51	0.01\\
63.52	0.01\\
63.53	0.01\\
63.54	0.01\\
63.55	0.01\\
63.56	0.01\\
63.57	0.01\\
63.58	0.01\\
63.59	0.01\\
63.6	0.01\\
63.61	0.01\\
63.62	0.01\\
63.63	0.01\\
63.64	0.01\\
63.65	0.01\\
63.66	0.01\\
63.67	0.01\\
63.68	0.01\\
63.69	0.01\\
63.7	0.01\\
63.71	0.01\\
63.72	0.01\\
63.73	0.01\\
63.74	0.01\\
63.75	0.01\\
63.76	0.01\\
63.77	0.01\\
63.78	0.01\\
63.79	0.01\\
63.8	0.01\\
63.81	0.01\\
63.82	0.01\\
63.83	0.01\\
63.84	0.01\\
63.85	0.01\\
63.86	0.01\\
63.87	0.01\\
63.88	0.01\\
63.89	0.01\\
63.9	0.01\\
63.91	0.01\\
63.92	0.01\\
63.93	0.01\\
63.94	0.01\\
63.95	0.01\\
63.96	0.01\\
63.97	0.01\\
63.98	0.01\\
63.99	0.01\\
64	0.01\\
64.01	0.01\\
64.02	0.01\\
64.03	0.01\\
64.04	0.01\\
64.05	0.01\\
64.06	0.01\\
64.07	0.01\\
64.08	0.01\\
64.09	0.01\\
64.1	0.01\\
64.11	0.01\\
64.12	0.01\\
64.13	0.01\\
64.14	0.01\\
64.15	0.01\\
64.16	0.01\\
64.17	0.01\\
64.18	0.01\\
64.19	0.01\\
64.2	0.01\\
64.21	0.01\\
64.22	0.01\\
64.23	0.01\\
64.24	0.01\\
64.25	0.01\\
64.26	0.01\\
64.27	0.01\\
64.28	0.01\\
64.29	0.01\\
64.3	0.01\\
64.31	0.01\\
64.32	0.01\\
64.33	0.01\\
64.34	0.01\\
64.35	0.01\\
64.36	0.01\\
64.37	0.01\\
64.38	0.01\\
64.39	0.01\\
64.4	0.01\\
64.41	0.01\\
64.42	0.01\\
64.43	0.01\\
64.44	0.01\\
64.45	0.01\\
64.46	0.01\\
64.47	0.01\\
64.48	0.01\\
64.49	0.01\\
64.5	0.01\\
64.51	0.01\\
64.52	0.01\\
64.53	0.01\\
64.54	0.01\\
64.55	0.01\\
64.56	0.01\\
64.57	0.01\\
64.58	0.01\\
64.59	0.01\\
64.6	0.01\\
64.61	0.01\\
64.62	0.01\\
64.63	0.01\\
64.64	0.01\\
64.65	0.01\\
64.66	0.01\\
64.67	0.01\\
64.68	0.01\\
64.69	0.01\\
64.7	0.01\\
64.71	0.01\\
64.72	0.01\\
64.73	0.01\\
64.74	0.01\\
64.75	0.01\\
64.76	0.01\\
64.77	0.01\\
64.78	0.01\\
64.79	0.01\\
64.8	0.01\\
64.81	0.01\\
64.82	0.01\\
64.83	0.01\\
64.84	0.01\\
64.85	0.01\\
64.86	0.01\\
64.87	0.01\\
64.88	0.01\\
64.89	0.01\\
64.9	0.01\\
64.91	0.01\\
64.92	0.01\\
64.93	0.01\\
64.94	0.01\\
64.95	0.01\\
64.96	0.01\\
64.97	0.01\\
64.98	0.01\\
64.99	0.01\\
65	0.01\\
65.01	0.01\\
65.02	0.01\\
65.03	0.01\\
65.04	0.01\\
65.05	0.01\\
65.06	0.01\\
65.07	0.01\\
65.08	0.01\\
65.09	0.01\\
65.1	0.01\\
65.11	0.01\\
65.12	0.01\\
65.13	0.01\\
65.14	0.01\\
65.15	0.01\\
65.16	0.01\\
65.17	0.01\\
65.18	0.01\\
65.19	0.01\\
65.2	0.01\\
65.21	0.01\\
65.22	0.01\\
65.23	0.01\\
65.24	0.01\\
65.25	0.01\\
65.26	0.01\\
65.27	0.01\\
65.28	0.01\\
65.29	0.01\\
65.3	0.01\\
65.31	0.01\\
65.32	0.01\\
65.33	0.01\\
65.34	0.01\\
65.35	0.01\\
65.36	0.01\\
65.37	0.01\\
65.38	0.01\\
65.39	0.01\\
65.4	0.01\\
65.41	0.01\\
65.42	0.01\\
65.43	0.01\\
65.44	0.01\\
65.45	0.01\\
65.46	0.01\\
65.47	0.01\\
65.48	0.01\\
65.49	0.01\\
65.5	0.01\\
65.51	0.01\\
65.52	0.01\\
65.53	0.01\\
65.54	0.01\\
65.55	0.01\\
65.56	0.01\\
65.57	0.01\\
65.58	0.01\\
65.59	0.01\\
65.6	0.01\\
65.61	0.01\\
65.62	0.01\\
65.63	0.01\\
65.64	0.01\\
65.65	0.01\\
65.66	0.01\\
65.67	0.01\\
65.68	0.01\\
65.69	0.01\\
65.7	0.01\\
65.71	0.01\\
65.72	0.01\\
65.73	0.01\\
65.74	0.01\\
65.75	0.01\\
65.76	0.01\\
65.77	0.01\\
65.78	0.01\\
65.79	0.01\\
65.8	0.01\\
65.81	0.01\\
65.82	0.01\\
65.83	0.01\\
65.84	0.01\\
65.85	0.01\\
65.86	0.01\\
65.87	0.01\\
65.88	0.01\\
65.89	0.01\\
65.9	0.01\\
65.91	0.01\\
65.92	0.01\\
65.93	0.01\\
65.94	0.01\\
65.95	0.01\\
65.96	0.01\\
65.97	0.01\\
65.98	0.01\\
65.99	0.01\\
66	0.01\\
66.01	0.01\\
66.02	0.01\\
66.03	0.01\\
66.04	0.01\\
66.05	0.01\\
66.06	0.01\\
66.07	0.01\\
66.08	0.01\\
66.09	0.01\\
66.1	0.01\\
66.11	0.01\\
66.12	0.01\\
66.13	0.01\\
66.14	0.01\\
66.15	0.01\\
66.16	0.01\\
66.17	0.01\\
66.18	0.01\\
66.19	0.01\\
66.2	0.01\\
66.21	0.01\\
66.22	0.01\\
66.23	0.01\\
66.24	0.01\\
66.25	0.01\\
66.26	0.01\\
66.27	0.01\\
66.28	0.01\\
66.29	0.01\\
66.3	0.01\\
66.31	0.01\\
66.32	0.01\\
66.33	0.01\\
66.34	0.01\\
66.35	0.01\\
66.36	0.01\\
66.37	0.01\\
66.38	0.01\\
66.39	0.01\\
66.4	0.01\\
66.41	0.01\\
66.42	0.01\\
66.43	0.01\\
66.44	0.01\\
66.45	0.01\\
66.46	0.01\\
66.47	0.01\\
66.48	0.01\\
66.49	0.01\\
66.5	0.01\\
66.51	0.01\\
66.52	0.01\\
66.53	0.01\\
66.54	0.01\\
66.55	0.01\\
66.56	0.01\\
66.57	0.01\\
66.58	0.01\\
66.59	0.01\\
66.6	0.01\\
66.61	0.01\\
66.62	0.01\\
66.63	0.01\\
66.64	0.01\\
66.65	0.01\\
66.66	0.01\\
66.67	0.01\\
66.68	0.01\\
66.69	0.01\\
66.7	0.01\\
66.71	0.01\\
66.72	0.01\\
66.73	0.01\\
66.74	0.01\\
66.75	0.01\\
66.76	0.01\\
66.77	0.01\\
66.78	0.01\\
66.79	0.01\\
66.8	0.01\\
66.81	0.01\\
66.82	0.01\\
66.83	0.01\\
66.84	0.01\\
66.85	0.01\\
66.86	0.01\\
66.87	0.01\\
66.88	0.01\\
66.89	0.01\\
66.9	0.01\\
66.91	0.01\\
66.92	0.01\\
66.93	0.01\\
66.94	0.01\\
66.95	0.01\\
66.96	0.01\\
66.97	0.01\\
66.98	0.01\\
66.99	0.01\\
67	0.01\\
67.01	0.01\\
67.02	0.01\\
67.03	0.01\\
67.04	0.01\\
67.05	0.01\\
67.06	0.01\\
67.07	0.01\\
67.08	0.01\\
67.09	0.01\\
67.1	0.01\\
67.11	0.01\\
67.12	0.01\\
67.13	0.01\\
67.14	0.01\\
67.15	0.01\\
67.16	0.01\\
67.17	0.01\\
67.18	0.01\\
67.19	0.01\\
67.2	0.01\\
67.21	0.01\\
67.22	0.01\\
67.23	0.01\\
67.24	0.01\\
67.25	0.01\\
67.26	0.01\\
67.27	0.01\\
67.28	0.01\\
67.29	0.01\\
67.3	0.01\\
67.31	0.01\\
67.32	0.01\\
67.33	0.01\\
67.34	0.01\\
67.35	0.01\\
67.36	0.01\\
67.37	0.01\\
67.38	0.01\\
67.39	0.01\\
67.4	0.01\\
67.41	0.01\\
67.42	0.01\\
67.43	0.01\\
67.44	0.01\\
67.45	0.01\\
67.46	0.01\\
67.47	0.01\\
67.48	0.01\\
67.49	0.01\\
67.5	0.01\\
67.51	0.01\\
67.52	0.01\\
67.53	0.01\\
67.54	0.01\\
67.55	0.01\\
67.56	0.01\\
67.57	0.01\\
67.58	0.01\\
67.59	0.01\\
67.6	0.01\\
67.61	0.01\\
67.62	0.01\\
67.63	0.01\\
67.64	0.01\\
67.65	0.01\\
67.66	0.01\\
67.67	0.01\\
67.68	0.01\\
67.69	0.01\\
67.7	0.01\\
67.71	0.01\\
67.72	0.01\\
67.73	0.01\\
67.74	0.01\\
67.75	0.01\\
67.76	0.01\\
67.77	0.01\\
67.78	0.01\\
67.79	0.01\\
67.8	0.01\\
67.81	0.01\\
67.82	0.01\\
67.83	0.01\\
67.84	0.01\\
67.85	0.01\\
67.86	0.01\\
67.87	0.01\\
67.88	0.01\\
67.89	0.01\\
67.9	0.01\\
67.91	0.01\\
67.92	0.01\\
67.93	0.01\\
67.94	0.01\\
67.95	0.01\\
67.96	0.01\\
67.97	0.01\\
67.98	0.01\\
67.99	0.01\\
68	0.01\\
68.01	0.01\\
68.02	0.01\\
68.03	0.01\\
68.04	0.01\\
68.05	0.01\\
68.06	0.01\\
68.07	0.01\\
68.08	0.01\\
68.09	0.01\\
68.1	0.01\\
68.11	0.01\\
68.12	0.01\\
68.13	0.01\\
68.14	0.01\\
68.15	0.01\\
68.16	0.01\\
68.17	0.01\\
68.18	0.01\\
68.19	0.01\\
68.2	0.01\\
68.21	0.01\\
68.22	0.01\\
68.23	0.01\\
68.24	0.01\\
68.25	0.01\\
68.26	0.01\\
68.27	0.01\\
68.28	0.01\\
68.29	0.01\\
68.3	0.01\\
68.31	0.01\\
68.32	0.01\\
68.33	0.01\\
68.34	0.01\\
68.35	0.01\\
68.36	0.01\\
68.37	0.01\\
68.38	0.01\\
68.39	0.01\\
68.4	0.01\\
68.41	0.01\\
68.42	0.01\\
68.43	0.01\\
68.44	0.01\\
68.45	0.01\\
68.46	0.01\\
68.47	0.01\\
68.48	0.01\\
68.49	0.01\\
68.5	0.01\\
68.51	0.01\\
68.52	0.01\\
68.53	0.01\\
68.54	0.01\\
68.55	0.01\\
68.56	0.01\\
68.57	0.01\\
68.58	0.01\\
68.59	0.01\\
68.6	0.01\\
68.61	0.01\\
68.62	0.01\\
68.63	0.01\\
68.64	0.01\\
68.65	0.01\\
68.66	0.01\\
68.67	0.01\\
68.68	0.01\\
68.69	0.01\\
68.7	0.01\\
68.71	0.01\\
68.72	0.01\\
68.73	0.01\\
68.74	0.01\\
68.75	0.01\\
68.76	0.01\\
68.77	0.01\\
68.78	0.01\\
68.79	0.01\\
68.8	0.01\\
68.81	0.01\\
68.82	0.01\\
68.83	0.01\\
68.84	0.01\\
68.85	0.01\\
68.86	0.01\\
68.87	0.01\\
68.88	0.01\\
68.89	0.01\\
68.9	0.01\\
68.91	0.01\\
68.92	0.01\\
68.93	0.01\\
68.94	0.01\\
68.95	0.01\\
68.96	0.01\\
68.97	0.01\\
68.98	0.01\\
68.99	0.01\\
69	0.01\\
69.01	0.01\\
69.02	0.01\\
69.03	0.01\\
69.04	0.01\\
69.05	0.01\\
69.06	0.01\\
69.07	0.01\\
69.08	0.01\\
69.09	0.01\\
69.1	0.01\\
69.11	0.01\\
69.12	0.01\\
69.13	0.01\\
69.14	0.01\\
69.15	0.01\\
69.16	0.01\\
69.17	0.01\\
69.18	0.01\\
69.19	0.01\\
69.2	0.01\\
69.21	0.01\\
69.22	0.01\\
69.23	0.01\\
69.24	0.01\\
69.25	0.01\\
69.26	0.01\\
69.27	0.01\\
69.28	0.01\\
69.29	0.01\\
69.3	0.01\\
69.31	0.01\\
69.32	0.01\\
69.33	0.01\\
69.34	0.01\\
69.35	0.01\\
69.36	0.01\\
69.37	0.01\\
69.38	0.01\\
69.39	0.01\\
69.4	0.01\\
69.41	0.01\\
69.42	0.01\\
69.43	0.01\\
69.44	0.01\\
69.45	0.01\\
69.46	0.01\\
69.47	0.01\\
69.48	0.01\\
69.49	0.01\\
69.5	0.01\\
69.51	0.01\\
69.52	0.01\\
69.53	0.01\\
69.54	0.01\\
69.55	0.01\\
69.56	0.01\\
69.57	0.01\\
69.58	0.01\\
69.59	0.01\\
69.6	0.01\\
69.61	0.01\\
69.62	0.01\\
69.63	0.01\\
69.64	0.01\\
69.65	0.01\\
69.66	0.01\\
69.67	0.01\\
69.68	0.01\\
69.69	0.01\\
69.7	0.01\\
69.71	0.01\\
69.72	0.01\\
69.73	0.01\\
69.74	0.01\\
69.75	0.01\\
69.76	0.01\\
69.77	0.01\\
69.78	0.01\\
69.79	0.01\\
69.8	0.01\\
69.81	0.01\\
69.82	0.01\\
69.83	0.01\\
69.84	0.01\\
69.85	0.01\\
69.86	0.01\\
69.87	0.01\\
69.88	0.01\\
69.89	0.01\\
69.9	0.01\\
69.91	0.01\\
69.92	0.01\\
69.93	0.01\\
69.94	0.01\\
69.95	0.01\\
69.96	0.01\\
69.97	0.01\\
69.98	0.01\\
69.99	0.01\\
70	0.01\\
70.01	0.01\\
70.02	0.01\\
70.03	0.01\\
70.04	0.01\\
70.05	0.01\\
70.06	0.01\\
70.07	0.01\\
70.08	0.01\\
70.09	0.01\\
70.1	0.01\\
70.11	0.01\\
70.12	0.01\\
70.13	0.01\\
70.14	0.01\\
70.15	0.01\\
70.16	0.01\\
70.17	0.01\\
70.18	0.01\\
70.19	0.01\\
70.2	0.01\\
70.21	0.01\\
70.22	0.01\\
70.23	0.01\\
70.24	0.01\\
70.25	0.01\\
70.26	0.01\\
70.27	0.01\\
70.28	0.01\\
70.29	0.01\\
70.3	0.01\\
70.31	0.01\\
70.32	0.01\\
70.33	0.01\\
70.34	0.01\\
70.35	0.01\\
70.36	0.01\\
70.37	0.01\\
70.38	0.01\\
70.39	0.01\\
70.4	0.01\\
70.41	0.01\\
70.42	0.01\\
70.43	0.01\\
70.44	0.01\\
70.45	0.01\\
70.46	0.01\\
70.47	0.01\\
70.48	0.01\\
70.49	0.01\\
70.5	0.01\\
70.51	0.01\\
70.52	0.01\\
70.53	0.01\\
70.54	0.01\\
70.55	0.01\\
70.56	0.01\\
70.57	0.01\\
70.58	0.01\\
70.59	0.01\\
70.6	0.01\\
70.61	0.01\\
70.62	0.01\\
70.63	0.01\\
70.64	0.01\\
70.65	0.01\\
70.66	0.01\\
70.67	0.01\\
70.68	0.01\\
70.69	0.01\\
70.7	0.01\\
70.71	0.01\\
70.72	0.01\\
70.73	0.01\\
70.74	0.01\\
70.75	0.01\\
70.76	0.01\\
70.77	0.01\\
70.78	0.01\\
70.79	0.01\\
70.8	0.01\\
70.81	0.01\\
70.82	0.01\\
70.83	0.01\\
70.84	0.01\\
70.85	0.01\\
70.86	0.01\\
70.87	0.01\\
70.88	0.01\\
70.89	0.01\\
70.9	0.01\\
70.91	0.01\\
70.92	0.01\\
70.93	0.01\\
70.94	0.01\\
70.95	0.01\\
70.96	0.01\\
70.97	0.01\\
70.98	0.01\\
70.99	0.01\\
71	0.01\\
71.01	0.01\\
71.02	0.01\\
71.03	0.01\\
71.04	0.01\\
71.05	0.01\\
71.06	0.01\\
71.07	0.01\\
71.08	0.01\\
71.09	0.01\\
71.1	0.01\\
71.11	0.01\\
71.12	0.01\\
71.13	0.01\\
71.14	0.01\\
71.15	0.01\\
71.16	0.01\\
71.17	0.01\\
71.18	0.01\\
71.19	0.01\\
71.2	0.01\\
71.21	0.01\\
71.22	0.01\\
71.23	0.01\\
71.24	0.01\\
71.25	0.01\\
71.26	0.01\\
71.27	0.01\\
71.28	0.01\\
71.29	0.01\\
71.3	0.01\\
71.31	0.01\\
71.32	0.01\\
71.33	0.01\\
71.34	0.01\\
71.35	0.01\\
71.36	0.01\\
71.37	0.01\\
71.38	0.01\\
71.39	0.01\\
71.4	0.01\\
71.41	0.01\\
71.42	0.01\\
71.43	0.01\\
71.44	0.01\\
71.45	0.01\\
71.46	0.01\\
71.47	0.01\\
71.48	0.01\\
71.49	0.01\\
71.5	0.01\\
71.51	0.01\\
71.52	0.01\\
71.53	0.01\\
71.54	0.01\\
71.55	0.01\\
71.56	0.01\\
71.57	0.01\\
71.58	0.01\\
71.59	0.01\\
71.6	0.01\\
71.61	0.01\\
71.62	0.01\\
71.63	0.01\\
71.64	0.01\\
71.65	0.01\\
71.66	0.01\\
71.67	0.01\\
71.68	0.01\\
71.69	0.01\\
71.7	0.01\\
71.71	0.01\\
71.72	0.01\\
71.73	0.01\\
71.74	0.01\\
71.75	0.01\\
71.76	0.01\\
71.77	0.01\\
71.78	0.01\\
71.79	0.01\\
71.8	0.01\\
71.81	0.01\\
71.82	0.01\\
71.83	0.01\\
71.84	0.01\\
71.85	0.01\\
71.86	0.01\\
71.87	0.01\\
71.88	0.01\\
71.89	0.01\\
71.9	0.01\\
71.91	0.01\\
71.92	0.01\\
71.93	0.01\\
71.94	0.01\\
71.95	0.01\\
71.96	0.01\\
71.97	0.01\\
71.98	0.01\\
71.99	0.01\\
72	0.01\\
72.01	0.01\\
72.02	0.01\\
72.03	0.01\\
72.04	0.01\\
72.05	0.01\\
72.06	0.01\\
72.07	0.01\\
72.08	0.01\\
72.09	0.01\\
72.1	0.01\\
72.11	0.01\\
72.12	0.01\\
72.13	0.01\\
72.14	0.01\\
72.15	0.01\\
72.16	0.01\\
72.17	0.01\\
72.18	0.01\\
72.19	0.01\\
72.2	0.01\\
72.21	0.01\\
72.22	0.01\\
72.23	0.01\\
72.24	0.01\\
72.25	0.01\\
72.26	0.01\\
72.27	0.01\\
72.28	0.01\\
72.29	0.01\\
72.3	0.01\\
72.31	0.01\\
72.32	0.01\\
72.33	0.01\\
72.34	0.01\\
72.35	0.01\\
72.36	0.01\\
72.37	0.01\\
72.38	0.01\\
72.39	0.01\\
72.4	0.01\\
72.41	0.01\\
72.42	0.01\\
72.43	0.01\\
72.44	0.01\\
72.45	0.01\\
72.46	0.01\\
72.47	0.01\\
72.48	0.01\\
72.49	0.01\\
72.5	0.01\\
72.51	0.01\\
72.52	0.01\\
72.53	0.01\\
72.54	0.01\\
72.55	0.01\\
72.56	0.01\\
72.57	0.01\\
72.58	0.01\\
72.59	0.01\\
72.6	0.01\\
72.61	0.01\\
72.62	0.01\\
72.63	0.01\\
72.64	0.01\\
72.65	0.01\\
72.66	0.01\\
72.67	0.01\\
72.68	0.01\\
72.69	0.01\\
72.7	0.01\\
72.71	0.01\\
72.72	0.01\\
72.73	0.01\\
72.74	0.01\\
72.75	0.01\\
72.76	0.01\\
72.77	0.01\\
72.78	0.01\\
72.79	0.01\\
72.8	0.01\\
72.81	0.01\\
72.82	0.01\\
72.83	0.01\\
72.84	0.01\\
72.85	0.01\\
72.86	0.01\\
72.87	0.01\\
72.88	0.01\\
72.89	0.01\\
72.9	0.01\\
72.91	0.01\\
72.92	0.01\\
72.93	0.01\\
72.94	0.01\\
72.95	0.01\\
72.96	0.01\\
72.97	0.01\\
72.98	0.01\\
72.99	0.01\\
73	0.01\\
73.01	0.01\\
73.02	0.01\\
73.03	0.01\\
73.04	0.01\\
73.05	0.01\\
73.06	0.01\\
73.07	0.01\\
73.08	0.01\\
73.09	0.01\\
73.1	0.01\\
73.11	0.01\\
73.12	0.01\\
73.13	0.01\\
73.14	0.01\\
73.15	0.01\\
73.16	0.01\\
73.17	0.01\\
73.18	0.01\\
73.19	0.01\\
73.2	0.01\\
73.21	0.01\\
73.22	0.01\\
73.23	0.01\\
73.24	0.01\\
73.25	0.01\\
73.26	0.01\\
73.27	0.01\\
73.28	0.01\\
73.29	0.01\\
73.3	0.01\\
73.31	0.01\\
73.32	0.01\\
73.33	0.01\\
73.34	0.01\\
73.35	0.01\\
73.36	0.01\\
73.37	0.01\\
73.38	0.01\\
73.39	0.01\\
73.4	0.01\\
73.41	0.01\\
73.42	0.01\\
73.43	0.01\\
73.44	0.01\\
73.45	0.01\\
73.46	0.01\\
73.47	0.01\\
73.48	0.01\\
73.49	0.01\\
73.5	0.01\\
73.51	0.01\\
73.52	0.01\\
73.53	0.01\\
73.54	0.01\\
73.55	0.01\\
73.56	0.01\\
73.57	0.01\\
73.58	0.01\\
73.59	0.01\\
73.6	0.01\\
73.61	0.01\\
73.62	0.01\\
73.63	0.01\\
73.64	0.01\\
73.65	0.01\\
73.66	0.01\\
73.67	0.01\\
73.68	0.01\\
73.69	0.01\\
73.7	0.01\\
73.71	0.01\\
73.72	0.01\\
73.73	0.01\\
73.74	0.01\\
73.75	0.01\\
73.76	0.01\\
73.77	0.01\\
73.78	0.01\\
73.79	0.01\\
73.8	0.01\\
73.81	0.01\\
73.82	0.01\\
73.83	0.01\\
73.84	0.01\\
73.85	0.01\\
73.86	0.01\\
73.87	0.01\\
73.88	0.01\\
73.89	0.01\\
73.9	0.01\\
73.91	0.01\\
73.92	0.01\\
73.93	0.01\\
73.94	0.01\\
73.95	0.01\\
73.96	0.01\\
73.97	0.01\\
73.98	0.01\\
73.99	0.01\\
74	0.01\\
74.01	0.01\\
74.02	0.01\\
74.03	0.01\\
74.04	0.01\\
74.05	0.01\\
74.06	0.01\\
74.07	0.01\\
74.08	0.01\\
74.09	0.01\\
74.1	0.01\\
74.11	0.01\\
74.12	0.01\\
74.13	0.01\\
74.14	0.01\\
74.15	0.01\\
74.16	0.01\\
74.17	0.01\\
74.18	0.01\\
74.19	0.01\\
74.2	0.01\\
74.21	0.01\\
74.22	0.01\\
74.23	0.01\\
74.24	0.01\\
74.25	0.01\\
74.26	0.01\\
74.27	0.01\\
74.28	0.01\\
74.29	0.01\\
74.3	0.01\\
74.31	0.01\\
74.32	0.01\\
74.33	0.01\\
74.34	0.01\\
74.35	0.01\\
74.36	0.01\\
74.37	0.01\\
74.38	0.01\\
74.39	0.01\\
74.4	0.01\\
74.41	0.01\\
74.42	0.01\\
74.43	0.01\\
74.44	0.01\\
74.45	0.01\\
74.46	0.01\\
74.47	0.01\\
74.48	0.01\\
74.49	0.01\\
74.5	0.01\\
74.51	0.01\\
74.52	0.01\\
74.53	0.01\\
74.54	0.01\\
74.55	0.01\\
74.56	0.01\\
74.57	0.01\\
74.58	0.01\\
74.59	0.01\\
74.6	0.01\\
74.61	0.01\\
74.62	0.01\\
74.63	0.01\\
74.64	0.01\\
74.65	0.01\\
74.66	0.01\\
74.67	0.01\\
74.68	0.01\\
74.69	0.01\\
74.7	0.01\\
74.71	0.01\\
74.72	0.01\\
74.73	0.01\\
74.74	0.01\\
74.75	0.01\\
74.76	0.01\\
74.77	0.01\\
74.78	0.01\\
74.79	0.01\\
74.8	0.01\\
74.81	0.01\\
74.82	0.01\\
74.83	0.01\\
74.84	0.01\\
74.85	0.01\\
74.86	0.01\\
74.87	0.01\\
74.88	0.01\\
74.89	0.01\\
74.9	0.01\\
74.91	0.01\\
74.92	0.01\\
74.93	0.01\\
74.94	0.01\\
74.95	0.01\\
74.96	0.01\\
74.97	0.01\\
74.98	0.01\\
74.99	0.01\\
75	0.01\\
75.01	0.01\\
75.02	0.01\\
75.03	0.01\\
75.04	0.01\\
75.05	0.01\\
75.06	0.01\\
75.07	0.01\\
75.08	0.01\\
75.09	0.01\\
75.1	0.01\\
75.11	0.01\\
75.12	0.01\\
75.13	0.01\\
75.14	0.01\\
75.15	0.01\\
75.16	0.01\\
75.17	0.01\\
75.18	0.01\\
75.19	0.01\\
75.2	0.01\\
75.21	0.01\\
75.22	0.01\\
75.23	0.01\\
75.24	0.01\\
75.25	0.01\\
75.26	0.01\\
75.27	0.01\\
75.28	0.01\\
75.29	0.01\\
75.3	0.01\\
75.31	0.01\\
75.32	0.01\\
75.33	0.01\\
75.34	0.01\\
75.35	0.01\\
75.36	0.01\\
75.37	0.01\\
75.38	0.01\\
75.39	0.01\\
75.4	0.01\\
75.41	0.01\\
75.42	0.01\\
75.43	0.01\\
75.44	0.01\\
75.45	0.01\\
75.46	0.01\\
75.47	0.01\\
75.48	0.01\\
75.49	0.01\\
75.5	0.01\\
75.51	0.01\\
75.52	0.01\\
75.53	0.01\\
75.54	0.01\\
75.55	0.01\\
75.56	0.01\\
75.57	0.01\\
75.58	0.01\\
75.59	0.01\\
75.6	0.01\\
75.61	0.01\\
75.62	0.01\\
75.63	0.01\\
75.64	0.01\\
75.65	0.01\\
75.66	0.01\\
75.67	0.01\\
75.68	0.01\\
75.69	0.01\\
75.7	0.01\\
75.71	0.01\\
75.72	0.01\\
75.73	0.01\\
75.74	0.01\\
75.75	0.01\\
75.76	0.01\\
75.77	0.01\\
75.78	0.01\\
75.79	0.01\\
75.8	0.01\\
75.81	0.01\\
75.82	0.01\\
75.83	0.01\\
75.84	0.01\\
75.85	0.01\\
75.86	0.01\\
75.87	0.01\\
75.88	0.01\\
75.89	0.01\\
75.9	0.01\\
75.91	0.01\\
75.92	0.01\\
75.93	0.01\\
75.94	0.01\\
75.95	0.01\\
75.96	0.01\\
75.97	0.01\\
75.98	0.01\\
75.99	0.01\\
76	0.01\\
76.01	0.01\\
76.02	0.01\\
76.03	0.01\\
76.04	0.01\\
76.05	0.01\\
76.06	0.01\\
76.07	0.01\\
76.08	0.01\\
76.09	0.01\\
76.1	0.01\\
76.11	0.01\\
76.12	0.01\\
76.13	0.01\\
76.14	0.01\\
76.15	0.01\\
76.16	0.01\\
76.17	0.01\\
76.18	0.01\\
76.19	0.01\\
76.2	0.01\\
76.21	0.01\\
76.22	0.01\\
76.23	0.01\\
76.24	0.01\\
76.25	0.01\\
76.26	0.01\\
76.27	0.01\\
76.28	0.01\\
76.29	0.01\\
76.3	0.01\\
76.31	0.01\\
76.32	0.01\\
76.33	0.01\\
76.34	0.01\\
76.35	0.01\\
76.36	0.01\\
76.37	0.01\\
76.38	0.01\\
76.39	0.01\\
76.4	0.01\\
76.41	0.01\\
76.42	0.01\\
76.43	0.01\\
76.44	0.01\\
76.45	0.01\\
76.46	0.01\\
76.47	0.01\\
76.48	0.01\\
76.49	0.01\\
76.5	0.01\\
76.51	0.01\\
76.52	0.01\\
76.53	0.01\\
76.54	0.01\\
76.55	0.01\\
76.56	0.01\\
76.57	0.01\\
76.58	0.01\\
76.59	0.01\\
76.6	0.01\\
76.61	0.01\\
76.62	0.01\\
76.63	0.01\\
76.64	0.01\\
76.65	0.01\\
76.66	0.01\\
76.67	0.01\\
76.68	0.01\\
76.69	0.01\\
76.7	0.01\\
76.71	0.01\\
76.72	0.01\\
76.73	0.01\\
76.74	0.01\\
76.75	0.01\\
76.76	0.01\\
76.77	0.01\\
76.78	0.01\\
76.79	0.01\\
76.8	0.01\\
76.81	0.01\\
76.82	0.01\\
76.83	0.01\\
76.84	0.01\\
76.85	0.01\\
76.86	0.01\\
76.87	0.01\\
76.88	0.01\\
76.89	0.01\\
76.9	0.01\\
76.91	0.01\\
76.92	0.01\\
76.93	0.01\\
76.94	0.01\\
76.95	0.01\\
76.96	0.01\\
76.97	0.01\\
76.98	0.01\\
76.99	0.01\\
77	0.01\\
77.01	0.01\\
77.02	0.01\\
77.03	0.01\\
77.04	0.01\\
77.05	0.01\\
77.06	0.01\\
77.07	0.01\\
77.08	0.01\\
77.09	0.01\\
77.1	0.01\\
77.11	0.01\\
77.12	0.01\\
77.13	0.01\\
77.14	0.01\\
77.15	0.01\\
77.16	0.01\\
77.17	0.01\\
77.18	0.01\\
77.19	0.01\\
77.2	0.01\\
77.21	0.01\\
77.22	0.01\\
77.23	0.01\\
77.24	0.01\\
77.25	0.01\\
77.26	0.01\\
77.27	0.01\\
77.28	0.01\\
77.29	0.01\\
77.3	0.01\\
77.31	0.01\\
77.32	0.01\\
77.33	0.01\\
77.34	0.01\\
77.35	0.01\\
77.36	0.01\\
77.37	0.01\\
77.38	0.01\\
77.39	0.01\\
77.4	0.01\\
77.41	0.01\\
77.42	0.01\\
77.43	0.01\\
77.44	0.01\\
77.45	0.01\\
77.46	0.01\\
77.47	0.01\\
77.48	0.01\\
77.49	0.01\\
77.5	0.01\\
77.51	0.01\\
77.52	0.01\\
77.53	0.01\\
77.54	0.01\\
77.55	0.01\\
77.56	0.01\\
77.57	0.01\\
77.58	0.01\\
77.59	0.01\\
77.6	0.01\\
77.61	0.01\\
77.62	0.01\\
77.63	0.01\\
77.64	0.01\\
77.65	0.01\\
77.66	0.01\\
77.67	0.01\\
77.68	0.01\\
77.69	0.01\\
77.7	0.01\\
77.71	0.01\\
77.72	0.01\\
77.73	0.01\\
77.74	0.01\\
77.75	0.01\\
77.76	0.01\\
77.77	0.01\\
77.78	0.01\\
77.79	0.01\\
77.8	0.01\\
77.81	0.01\\
77.82	0.01\\
77.83	0.01\\
77.84	0.01\\
77.85	0.01\\
77.86	0.01\\
77.87	0.01\\
77.88	0.01\\
77.89	0.01\\
77.9	0.01\\
77.91	0.01\\
77.92	0.01\\
77.93	0.01\\
77.94	0.01\\
77.95	0.01\\
77.96	0.01\\
77.97	0.01\\
77.98	0.01\\
77.99	0.01\\
78	0.01\\
78.01	0.01\\
78.02	0.01\\
78.03	0.01\\
78.04	0.01\\
78.05	0.01\\
78.06	0.01\\
78.07	0.01\\
78.08	0.01\\
78.09	0.01\\
78.1	0.01\\
78.11	0.01\\
78.12	0.01\\
78.13	0.01\\
78.14	0.01\\
78.15	0.01\\
78.16	0.01\\
78.17	0.01\\
78.18	0.01\\
78.19	0.01\\
78.2	0.01\\
78.21	0.01\\
78.22	0.01\\
78.23	0.01\\
78.24	0.01\\
78.25	0.01\\
78.26	0.01\\
78.27	0.01\\
78.28	0.01\\
78.29	0.01\\
78.3	0.01\\
78.31	0.01\\
78.32	0.01\\
78.33	0.01\\
78.34	0.01\\
78.35	0.01\\
78.36	0.01\\
78.37	0.01\\
78.38	0.01\\
78.39	0.01\\
78.4	0.01\\
78.41	0.01\\
78.42	0.01\\
78.43	0.01\\
78.44	0.01\\
78.45	0.01\\
78.46	0.01\\
78.47	0.01\\
78.48	0.01\\
78.49	0.01\\
78.5	0.01\\
78.51	0.01\\
78.52	0.01\\
78.53	0.01\\
78.54	0.01\\
78.55	0.01\\
78.56	0.01\\
78.57	0.01\\
78.58	0.01\\
78.59	0.01\\
78.6	0.01\\
78.61	0.01\\
78.62	0.01\\
78.63	0.01\\
78.64	0.01\\
78.65	0.01\\
78.66	0.01\\
78.67	0.01\\
78.68	0.01\\
78.69	0.01\\
78.7	0.01\\
78.71	0.01\\
78.72	0.01\\
78.73	0.01\\
78.74	0.01\\
78.75	0.01\\
78.76	0.01\\
78.77	0.01\\
78.78	0.01\\
78.79	0.01\\
78.8	0.01\\
78.81	0.01\\
78.82	0.01\\
78.83	0.01\\
78.84	0.01\\
78.85	0.01\\
78.86	0.01\\
78.87	0.01\\
78.88	0.01\\
78.89	0.01\\
78.9	0.01\\
78.91	0.01\\
78.92	0.01\\
78.93	0.01\\
78.94	0.01\\
78.95	0.01\\
78.96	0.01\\
78.97	0.01\\
78.98	0.01\\
78.99	0.01\\
79	0.01\\
79.01	0.01\\
79.02	0.01\\
79.03	0.01\\
79.04	0.01\\
79.05	0.01\\
79.06	0.01\\
79.07	0.01\\
79.08	0.01\\
79.09	0.01\\
79.1	0.01\\
79.11	0.01\\
79.12	0.01\\
79.13	0.01\\
79.14	0.01\\
79.15	0.01\\
79.16	0.01\\
79.17	0.01\\
79.18	0.01\\
79.19	0.01\\
79.2	0.01\\
79.21	0.01\\
79.22	0.01\\
79.23	0.01\\
79.24	0.01\\
79.25	0.01\\
79.26	0.01\\
79.27	0.01\\
79.28	0.01\\
79.29	0.01\\
79.3	0.01\\
79.31	0.01\\
79.32	0.01\\
79.33	0.01\\
79.34	0.01\\
79.35	0.01\\
79.36	0.01\\
79.37	0.01\\
79.38	0.01\\
79.39	0.01\\
79.4	0.01\\
79.41	0.01\\
79.42	0.01\\
79.43	0.01\\
79.44	0.01\\
79.45	0.01\\
79.46	0.01\\
79.47	0.01\\
79.48	0.01\\
79.49	0.01\\
79.5	0.01\\
79.51	0.01\\
79.52	0.01\\
79.53	0.01\\
79.54	0.01\\
79.55	0.01\\
79.56	0.01\\
79.57	0.01\\
79.58	0.01\\
79.59	0.01\\
79.6	0.01\\
79.61	0.01\\
79.62	0.01\\
79.63	0.01\\
79.64	0.01\\
79.65	0.01\\
79.66	0.01\\
79.67	0.01\\
79.68	0.01\\
79.69	0.01\\
79.7	0.01\\
79.71	0.01\\
79.72	0.01\\
79.73	0.01\\
79.74	0.01\\
79.75	0.01\\
79.76	0.01\\
79.77	0.01\\
79.78	0.01\\
79.79	0.01\\
79.8	0.01\\
79.81	0.01\\
79.82	0.01\\
79.83	0.01\\
79.84	0.01\\
79.85	0.01\\
79.86	0.01\\
79.87	0.01\\
79.88	0.01\\
79.89	0.01\\
79.9	0.01\\
79.91	0.01\\
79.92	0.01\\
79.93	0.01\\
79.94	0.01\\
79.95	0.01\\
79.96	0.01\\
79.97	0.01\\
79.98	0.01\\
79.99	0.01\\
80	0.01\\
80.01	0.01\\
};
\addplot [color=mycolor1,solid]
  table[row sep=crcr]{%
80.01	0.01\\
80.02	0.01\\
80.03	0.01\\
80.04	0.01\\
80.05	0.01\\
80.06	0.01\\
80.07	0.01\\
80.08	0.01\\
80.09	0.01\\
80.1	0.01\\
80.11	0.01\\
80.12	0.01\\
80.13	0.01\\
80.14	0.01\\
80.15	0.01\\
80.16	0.01\\
80.17	0.01\\
80.18	0.01\\
80.19	0.01\\
80.2	0.01\\
80.21	0.01\\
80.22	0.01\\
80.23	0.01\\
80.24	0.01\\
80.25	0.01\\
80.26	0.01\\
80.27	0.01\\
80.28	0.01\\
80.29	0.01\\
80.3	0.01\\
80.31	0.01\\
80.32	0.01\\
80.33	0.01\\
80.34	0.01\\
80.35	0.01\\
80.36	0.01\\
80.37	0.01\\
80.38	0.01\\
80.39	0.01\\
80.4	0.01\\
80.41	0.01\\
80.42	0.01\\
80.43	0.01\\
80.44	0.01\\
80.45	0.01\\
80.46	0.01\\
80.47	0.01\\
80.48	0.01\\
80.49	0.01\\
80.5	0.01\\
80.51	0.01\\
80.52	0.01\\
80.53	0.01\\
80.54	0.01\\
80.55	0.01\\
80.56	0.01\\
80.57	0.01\\
80.58	0.01\\
80.59	0.01\\
80.6	0.01\\
80.61	0.01\\
80.62	0.01\\
80.63	0.01\\
80.64	0.01\\
80.65	0.01\\
80.66	0.01\\
80.67	0.01\\
80.68	0.01\\
80.69	0.01\\
80.7	0.01\\
80.71	0.01\\
80.72	0.01\\
80.73	0.01\\
80.74	0.01\\
80.75	0.01\\
80.76	0.01\\
80.77	0.01\\
80.78	0.01\\
80.79	0.01\\
80.8	0.01\\
80.81	0.01\\
80.82	0.01\\
80.83	0.01\\
80.84	0.01\\
80.85	0.01\\
80.86	0.01\\
80.87	0.01\\
80.88	0.01\\
80.89	0.01\\
80.9	0.01\\
80.91	0.01\\
80.92	0.01\\
80.93	0.01\\
80.94	0.01\\
80.95	0.01\\
80.96	0.01\\
80.97	0.01\\
80.98	0.01\\
80.99	0.01\\
81	0.01\\
81.01	0.01\\
81.02	0.01\\
81.03	0.01\\
81.04	0.01\\
81.05	0.01\\
81.06	0.01\\
81.07	0.01\\
81.08	0.01\\
81.09	0.01\\
81.1	0.01\\
81.11	0.01\\
81.12	0.01\\
81.13	0.01\\
81.14	0.01\\
81.15	0.01\\
81.16	0.01\\
81.17	0.01\\
81.18	0.01\\
81.19	0.01\\
81.2	0.01\\
81.21	0.01\\
81.22	0.01\\
81.23	0.01\\
81.24	0.01\\
81.25	0.01\\
81.26	0.01\\
81.27	0.01\\
81.28	0.01\\
81.29	0.01\\
81.3	0.01\\
81.31	0.01\\
81.32	0.01\\
81.33	0.01\\
81.34	0.01\\
81.35	0.01\\
81.36	0.01\\
81.37	0.01\\
81.38	0.01\\
81.39	0.01\\
81.4	0.01\\
81.41	0.01\\
81.42	0.01\\
81.43	0.01\\
81.44	0.01\\
81.45	0.01\\
81.46	0.01\\
81.47	0.01\\
81.48	0.01\\
81.49	0.01\\
81.5	0.01\\
81.51	0.01\\
81.52	0.01\\
81.53	0.01\\
81.54	0.01\\
81.55	0.01\\
81.56	0.01\\
81.57	0.01\\
81.58	0.01\\
81.59	0.01\\
81.6	0.01\\
81.61	0.01\\
81.62	0.01\\
81.63	0.01\\
81.64	0.01\\
81.65	0.01\\
81.66	0.01\\
81.67	0.01\\
81.68	0.01\\
81.69	0.01\\
81.7	0.01\\
81.71	0.01\\
81.72	0.01\\
81.73	0.01\\
81.74	0.01\\
81.75	0.01\\
81.76	0.01\\
81.77	0.01\\
81.78	0.01\\
81.79	0.01\\
81.8	0.01\\
81.81	0.01\\
81.82	0.01\\
81.83	0.01\\
81.84	0.01\\
81.85	0.01\\
81.86	0.01\\
81.87	0.01\\
81.88	0.01\\
81.89	0.01\\
81.9	0.01\\
81.91	0.01\\
81.92	0.01\\
81.93	0.01\\
81.94	0.01\\
81.95	0.01\\
81.96	0.01\\
81.97	0.01\\
81.98	0.01\\
81.99	0.01\\
82	0.01\\
82.01	0.01\\
82.02	0.01\\
82.03	0.01\\
82.04	0.01\\
82.05	0.01\\
82.06	0.01\\
82.07	0.01\\
82.08	0.01\\
82.09	0.01\\
82.1	0.01\\
82.11	0.01\\
82.12	0.01\\
82.13	0.01\\
82.14	0.01\\
82.15	0.01\\
82.16	0.01\\
82.17	0.01\\
82.18	0.01\\
82.19	0.01\\
82.2	0.01\\
82.21	0.01\\
82.22	0.01\\
82.23	0.01\\
82.24	0.01\\
82.25	0.01\\
82.26	0.01\\
82.27	0.01\\
82.28	0.01\\
82.29	0.01\\
82.3	0.01\\
82.31	0.01\\
82.32	0.01\\
82.33	0.01\\
82.34	0.01\\
82.35	0.01\\
82.36	0.01\\
82.37	0.01\\
82.38	0.01\\
82.39	0.01\\
82.4	0.01\\
82.41	0.01\\
82.42	0.01\\
82.43	0.01\\
82.44	0.01\\
82.45	0.01\\
82.46	0.01\\
82.47	0.01\\
82.48	0.01\\
82.49	0.01\\
82.5	0.01\\
82.51	0.01\\
82.52	0.01\\
82.53	0.01\\
82.54	0.01\\
82.55	0.01\\
82.56	0.01\\
82.57	0.01\\
82.58	0.01\\
82.59	0.01\\
82.6	0.01\\
82.61	0.01\\
82.62	0.01\\
82.63	0.01\\
82.64	0.01\\
82.65	0.01\\
82.66	0.01\\
82.67	0.01\\
82.68	0.01\\
82.69	0.01\\
82.7	0.01\\
82.71	0.01\\
82.72	0.01\\
82.73	0.01\\
82.74	0.01\\
82.75	0.01\\
82.76	0.01\\
82.77	0.01\\
82.78	0.01\\
82.79	0.01\\
82.8	0.01\\
82.81	0.01\\
82.82	0.01\\
82.83	0.01\\
82.84	0.01\\
82.85	0.01\\
82.86	0.01\\
82.87	0.01\\
82.88	0.01\\
82.89	0.01\\
82.9	0.01\\
82.91	0.01\\
82.92	0.01\\
82.93	0.01\\
82.94	0.01\\
82.95	0.01\\
82.96	0.01\\
82.97	0.01\\
82.98	0.01\\
82.99	0.01\\
83	0.01\\
83.01	0.01\\
83.02	0.01\\
83.03	0.01\\
83.04	0.01\\
83.05	0.01\\
83.06	0.01\\
83.07	0.01\\
83.08	0.01\\
83.09	0.01\\
83.1	0.01\\
83.11	0.01\\
83.12	0.01\\
83.13	0.01\\
83.14	0.01\\
83.15	0.01\\
83.16	0.01\\
83.17	0.01\\
83.18	0.01\\
83.19	0.01\\
83.2	0.01\\
83.21	0.01\\
83.22	0.01\\
83.23	0.01\\
83.24	0.01\\
83.25	0.01\\
83.26	0.01\\
83.27	0.01\\
83.28	0.01\\
83.29	0.01\\
83.3	0.01\\
83.31	0.01\\
83.32	0.01\\
83.33	0.01\\
83.34	0.01\\
83.35	0.01\\
83.36	0.01\\
83.37	0.01\\
83.38	0.01\\
83.39	0.01\\
83.4	0.01\\
83.41	0.01\\
83.42	0.01\\
83.43	0.01\\
83.44	0.01\\
83.45	0.01\\
83.46	0.01\\
83.47	0.01\\
83.48	0.01\\
83.49	0.01\\
83.5	0.01\\
83.51	0.01\\
83.52	0.01\\
83.53	0.01\\
83.54	0.01\\
83.55	0.01\\
83.56	0.01\\
83.57	0.01\\
83.58	0.01\\
83.59	0.01\\
83.6	0.01\\
83.61	0.01\\
83.62	0.01\\
83.63	0.01\\
83.64	0.01\\
83.65	0.01\\
83.66	0.01\\
83.67	0.01\\
83.68	0.01\\
83.69	0.01\\
83.7	0.01\\
83.71	0.01\\
83.72	0.01\\
83.73	0.01\\
83.74	0.01\\
83.75	0.01\\
83.76	0.01\\
83.77	0.01\\
83.78	0.01\\
83.79	0.01\\
83.8	0.01\\
83.81	0.01\\
83.82	0.01\\
83.83	0.01\\
83.84	0.01\\
83.85	0.01\\
83.86	0.01\\
83.87	0.01\\
83.88	0.01\\
83.89	0.01\\
83.9	0.01\\
83.91	0.01\\
83.92	0.01\\
83.93	0.01\\
83.94	0.01\\
83.95	0.01\\
83.96	0.01\\
83.97	0.01\\
83.98	0.01\\
83.99	0.01\\
84	0.01\\
84.01	0.01\\
84.02	0.01\\
84.03	0.01\\
84.04	0.01\\
84.05	0.01\\
84.06	0.01\\
84.07	0.01\\
84.08	0.01\\
84.09	0.01\\
84.1	0.01\\
84.11	0.01\\
84.12	0.01\\
84.13	0.01\\
84.14	0.01\\
84.15	0.01\\
84.16	0.01\\
84.17	0.01\\
84.18	0.01\\
84.19	0.01\\
84.2	0.01\\
84.21	0.01\\
84.22	0.01\\
84.23	0.01\\
84.24	0.01\\
84.25	0.01\\
84.26	0.01\\
84.27	0.01\\
84.28	0.01\\
84.29	0.01\\
84.3	0.01\\
84.31	0.01\\
84.32	0.01\\
84.33	0.01\\
84.34	0.01\\
84.35	0.01\\
84.36	0.01\\
84.37	0.01\\
84.38	0.01\\
84.39	0.01\\
84.4	0.01\\
84.41	0.01\\
84.42	0.01\\
84.43	0.01\\
84.44	0.01\\
84.45	0.01\\
84.46	0.01\\
84.47	0.01\\
84.48	0.01\\
84.49	0.01\\
84.5	0.01\\
84.51	0.01\\
84.52	0.01\\
84.53	0.01\\
84.54	0.01\\
84.55	0.01\\
84.56	0.01\\
84.57	0.01\\
84.58	0.01\\
84.59	0.01\\
84.6	0.01\\
84.61	0.01\\
84.62	0.01\\
84.63	0.01\\
84.64	0.01\\
84.65	0.01\\
84.66	0.01\\
84.67	0.01\\
84.68	0.01\\
84.69	0.01\\
84.7	0.01\\
84.71	0.01\\
84.72	0.01\\
84.73	0.01\\
84.74	0.01\\
84.75	0.01\\
84.76	0.01\\
84.77	0.01\\
84.78	0.01\\
84.79	0.01\\
84.8	0.01\\
84.81	0.01\\
84.82	0.01\\
84.83	0.01\\
84.84	0.01\\
84.85	0.01\\
84.86	0.01\\
84.87	0.01\\
84.88	0.01\\
84.89	0.01\\
84.9	0.01\\
84.91	0.01\\
84.92	0.01\\
84.93	0.01\\
84.94	0.01\\
84.95	0.01\\
84.96	0.01\\
84.97	0.01\\
84.98	0.01\\
84.99	0.01\\
85	0.01\\
85.01	0.01\\
85.02	0.01\\
85.03	0.01\\
85.04	0.01\\
85.05	0.01\\
85.06	0.01\\
85.07	0.01\\
85.08	0.01\\
85.09	0.01\\
85.1	0.01\\
85.11	0.01\\
85.12	0.01\\
85.13	0.01\\
85.14	0.01\\
85.15	0.01\\
85.16	0.01\\
85.17	0.01\\
85.18	0.01\\
85.19	0.01\\
85.2	0.01\\
85.21	0.01\\
85.22	0.01\\
85.23	0.01\\
85.24	0.01\\
85.25	0.01\\
85.26	0.01\\
85.27	0.01\\
85.28	0.01\\
85.29	0.01\\
85.3	0.01\\
85.31	0.01\\
85.32	0.01\\
85.33	0.01\\
85.34	0.01\\
85.35	0.01\\
85.36	0.01\\
85.37	0.01\\
85.38	0.01\\
85.39	0.01\\
85.4	0.01\\
85.41	0.01\\
85.42	0.01\\
85.43	0.01\\
85.44	0.01\\
85.45	0.01\\
85.46	0.01\\
85.47	0.01\\
85.48	0.01\\
85.49	0.01\\
85.5	0.01\\
85.51	0.01\\
85.52	0.01\\
85.53	0.01\\
85.54	0.01\\
85.55	0.01\\
85.56	0.01\\
85.57	0.01\\
85.58	0.01\\
85.59	0.01\\
85.6	0.01\\
85.61	0.01\\
85.62	0.01\\
85.63	0.01\\
85.64	0.01\\
85.65	0.01\\
85.66	0.01\\
85.67	0.01\\
85.68	0.01\\
85.69	0.01\\
85.7	0.01\\
85.71	0.01\\
85.72	0.01\\
85.73	0.01\\
85.74	0.01\\
85.75	0.01\\
85.76	0.01\\
85.77	0.01\\
85.78	0.01\\
85.79	0.01\\
85.8	0.01\\
85.81	0.01\\
85.82	0.01\\
85.83	0.01\\
85.84	0.01\\
85.85	0.01\\
85.86	0.01\\
85.87	0.01\\
85.88	0.01\\
85.89	0.01\\
85.9	0.01\\
85.91	0.01\\
85.92	0.01\\
85.93	0.01\\
85.94	0.01\\
85.95	0.01\\
85.96	0.01\\
85.97	0.01\\
85.98	0.01\\
85.99	0.01\\
86	0.01\\
86.01	0.01\\
86.02	0.01\\
86.03	0.01\\
86.04	0.01\\
86.05	0.01\\
86.06	0.01\\
86.07	0.01\\
86.08	0.01\\
86.09	0.01\\
86.1	0.01\\
86.11	0.01\\
86.12	0.01\\
86.13	0.01\\
86.14	0.01\\
86.15	0.01\\
86.16	0.01\\
86.17	0.01\\
86.18	0.01\\
86.19	0.01\\
86.2	0.01\\
86.21	0.01\\
86.22	0.01\\
86.23	0.01\\
86.24	0.01\\
86.25	0.01\\
86.26	0.01\\
86.27	0.01\\
86.28	0.01\\
86.29	0.01\\
86.3	0.01\\
86.31	0.01\\
86.32	0.01\\
86.33	0.01\\
86.34	0.01\\
86.35	0.01\\
86.36	0.01\\
86.37	0.01\\
86.38	0.01\\
86.39	0.01\\
86.4	0.01\\
86.41	0.01\\
86.42	0.01\\
86.43	0.01\\
86.44	0.01\\
86.45	0.01\\
86.46	0.01\\
86.47	0.01\\
86.48	0.01\\
86.49	0.01\\
86.5	0.01\\
86.51	0.01\\
86.52	0.01\\
86.53	0.01\\
86.54	0.01\\
86.55	0.01\\
86.56	0.01\\
86.57	0.01\\
86.58	0.01\\
86.59	0.01\\
86.6	0.01\\
86.61	0.01\\
86.62	0.01\\
86.63	0.01\\
86.64	0.01\\
86.65	0.01\\
86.66	0.01\\
86.67	0.01\\
86.68	0.01\\
86.69	0.01\\
86.7	0.01\\
86.71	0.01\\
86.72	0.01\\
86.73	0.01\\
86.74	0.01\\
86.75	0.01\\
86.76	0.01\\
86.77	0.01\\
86.78	0.01\\
86.79	0.01\\
86.8	0.01\\
86.81	0.01\\
86.82	0.01\\
86.83	0.01\\
86.84	0.01\\
86.85	0.01\\
86.86	0.01\\
86.87	0.01\\
86.88	0.01\\
86.89	0.01\\
86.9	0.01\\
86.91	0.01\\
86.92	0.01\\
86.93	0.01\\
86.94	0.01\\
86.95	0.01\\
86.96	0.01\\
86.97	0.01\\
86.98	0.01\\
86.99	0.01\\
87	0.01\\
87.01	0.01\\
87.02	0.01\\
87.03	0.01\\
87.04	0.01\\
87.05	0.01\\
87.06	0.01\\
87.07	0.01\\
87.08	0.01\\
87.09	0.01\\
87.1	0.01\\
87.11	0.01\\
87.12	0.01\\
87.13	0.01\\
87.14	0.01\\
87.15	0.01\\
87.16	0.01\\
87.17	0.01\\
87.18	0.01\\
87.19	0.01\\
87.2	0.01\\
87.21	0.01\\
87.22	0.01\\
87.23	0.01\\
87.24	0.01\\
87.25	0.01\\
87.26	0.01\\
87.27	0.01\\
87.28	0.01\\
87.29	0.01\\
87.3	0.01\\
87.31	0.01\\
87.32	0.01\\
87.33	0.01\\
87.34	0.01\\
87.35	0.01\\
87.36	0.01\\
87.37	0.01\\
87.38	0.01\\
87.39	0.01\\
87.4	0.01\\
87.41	0.01\\
87.42	0.01\\
87.43	0.01\\
87.44	0.01\\
87.45	0.01\\
87.46	0.01\\
87.47	0.01\\
87.48	0.01\\
87.49	0.01\\
87.5	0.01\\
87.51	0.01\\
87.52	0.01\\
87.53	0.01\\
87.54	0.01\\
87.55	0.01\\
87.56	0.01\\
87.57	0.01\\
87.58	0.01\\
87.59	0.01\\
87.6	0.01\\
87.61	0.01\\
87.62	0.01\\
87.63	0.01\\
87.64	0.01\\
87.65	0.01\\
87.66	0.01\\
87.67	0.01\\
87.68	0.01\\
87.69	0.01\\
87.7	0.01\\
87.71	0.01\\
87.72	0.01\\
87.73	0.01\\
87.74	0.01\\
87.75	0.01\\
87.76	0.01\\
87.77	0.01\\
87.78	0.01\\
87.79	0.01\\
87.8	0.01\\
87.81	0.01\\
87.82	0.01\\
87.83	0.01\\
87.84	0.01\\
87.85	0.01\\
87.86	0.01\\
87.87	0.01\\
87.88	0.01\\
87.89	0.01\\
87.9	0.01\\
87.91	0.01\\
87.92	0.01\\
87.93	0.01\\
87.94	0.01\\
87.95	0.01\\
87.96	0.01\\
87.97	0.01\\
87.98	0.01\\
87.99	0.01\\
88	0.01\\
88.01	0.01\\
88.02	0.01\\
88.03	0.01\\
88.04	0.01\\
88.05	0.01\\
88.06	0.01\\
88.07	0.01\\
88.08	0.01\\
88.09	0.01\\
88.1	0.01\\
88.11	0.01\\
88.12	0.01\\
88.13	0.01\\
88.14	0.01\\
88.15	0.01\\
88.16	0.01\\
88.17	0.01\\
88.18	0.01\\
88.19	0.01\\
88.2	0.01\\
88.21	0.01\\
88.22	0.01\\
88.23	0.01\\
88.24	0.01\\
88.25	0.01\\
88.26	0.01\\
88.27	0.01\\
88.28	0.01\\
88.29	0.01\\
88.3	0.01\\
88.31	0.01\\
88.32	0.01\\
88.33	0.01\\
88.34	0.01\\
88.35	0.01\\
88.36	0.01\\
88.37	0.01\\
88.38	0.01\\
88.39	0.01\\
88.4	0.01\\
88.41	0.01\\
88.42	0.01\\
88.43	0.01\\
88.44	0.01\\
88.45	0.01\\
88.46	0.01\\
88.47	0.01\\
88.48	0.01\\
88.49	0.01\\
88.5	0.01\\
88.51	0.01\\
88.52	0.01\\
88.53	0.01\\
88.54	0.01\\
88.55	0.01\\
88.56	0.01\\
88.57	0.01\\
88.58	0.01\\
88.59	0.01\\
88.6	0.01\\
88.61	0.01\\
88.62	0.01\\
88.63	0.01\\
88.64	0.01\\
88.65	0.01\\
88.66	0.01\\
88.67	0.01\\
88.68	0.01\\
88.69	0.01\\
88.7	0.01\\
88.71	0.01\\
88.72	0.01\\
88.73	0.01\\
88.74	0.01\\
88.75	0.01\\
88.76	0.01\\
88.77	0.01\\
88.78	0.01\\
88.79	0.01\\
88.8	0.01\\
88.81	0.01\\
88.82	0.01\\
88.83	0.01\\
88.84	0.01\\
88.85	0.01\\
88.86	0.01\\
88.87	0.01\\
88.88	0.01\\
88.89	0.01\\
88.9	0.01\\
88.91	0.01\\
88.92	0.01\\
88.93	0.01\\
88.94	0.01\\
88.95	0.01\\
88.96	0.01\\
88.97	0.01\\
88.98	0.01\\
88.99	0.01\\
89	0.01\\
89.01	0.01\\
89.02	0.01\\
89.03	0.01\\
89.04	0.01\\
89.05	0.01\\
89.06	0.01\\
89.07	0.01\\
89.08	0.01\\
89.09	0.01\\
89.1	0.01\\
89.11	0.01\\
89.12	0.01\\
89.13	0.01\\
89.14	0.01\\
89.15	0.01\\
89.16	0.01\\
89.17	0.01\\
89.18	0.01\\
89.19	0.01\\
89.2	0.01\\
89.21	0.01\\
89.22	0.01\\
89.23	0.01\\
89.24	0.01\\
89.25	0.01\\
89.26	0.01\\
89.27	0.01\\
89.28	0.01\\
89.29	0.01\\
89.3	0.01\\
89.31	0.01\\
89.32	0.01\\
89.33	0.01\\
89.34	0.01\\
89.35	0.01\\
89.36	0.01\\
89.37	0.01\\
89.38	0.01\\
89.39	0.01\\
89.4	0.01\\
89.41	0.01\\
89.42	0.01\\
89.43	0.01\\
89.44	0.01\\
89.45	0.01\\
89.46	0.01\\
89.47	0.01\\
89.48	0.01\\
89.49	0.01\\
89.5	0.01\\
89.51	0.01\\
89.52	0.01\\
89.53	0.01\\
89.54	0.01\\
89.55	0.01\\
89.56	0.01\\
89.57	0.01\\
89.58	0.01\\
89.59	0.01\\
89.6	0.01\\
89.61	0.01\\
89.62	0.01\\
89.63	0.01\\
89.64	0.01\\
89.65	0.01\\
89.66	0.01\\
89.67	0.01\\
89.68	0.01\\
89.69	0.01\\
89.7	0.01\\
89.71	0.01\\
89.72	0.01\\
89.73	0.01\\
89.74	0.01\\
89.75	0.01\\
89.76	0.01\\
89.77	0.01\\
89.78	0.01\\
89.79	0.01\\
89.8	0.01\\
89.81	0.01\\
89.82	0.01\\
89.83	0.01\\
89.84	0.01\\
89.85	0.01\\
89.86	0.01\\
89.87	0.01\\
89.88	0.01\\
89.89	0.01\\
89.9	0.01\\
89.91	0.01\\
89.92	0.01\\
89.93	0.01\\
89.94	0.01\\
89.95	0.01\\
89.96	0.01\\
89.97	0.01\\
89.98	0.01\\
89.99	0.01\\
90	0.01\\
90.01	0.01\\
90.02	0.01\\
90.03	0.01\\
90.04	0.01\\
90.05	0.01\\
90.06	0.01\\
90.07	0.01\\
90.08	0.01\\
90.09	0.01\\
90.1	0.01\\
90.11	0.01\\
90.12	0.01\\
90.13	0.01\\
90.14	0.01\\
90.15	0.01\\
90.16	0.01\\
90.17	0.01\\
90.18	0.01\\
90.19	0.01\\
90.2	0.01\\
90.21	0.01\\
90.22	0.01\\
90.23	0.01\\
90.24	0.01\\
90.25	0.01\\
90.26	0.01\\
90.27	0.01\\
90.28	0.01\\
90.29	0.01\\
90.3	0.01\\
90.31	0.01\\
90.32	0.01\\
90.33	0.01\\
90.34	0.01\\
90.35	0.01\\
90.36	0.01\\
90.37	0.01\\
90.38	0.01\\
90.39	0.01\\
90.4	0.01\\
90.41	0.01\\
90.42	0.01\\
90.43	0.01\\
90.44	0.01\\
90.45	0.01\\
90.46	0.01\\
90.47	0.01\\
90.48	0.01\\
90.49	0.01\\
90.5	0.01\\
90.51	0.01\\
90.52	0.01\\
90.53	0.01\\
90.54	0.01\\
90.55	0.01\\
90.56	0.01\\
90.57	0.01\\
90.58	0.01\\
90.59	0.01\\
90.6	0.01\\
90.61	0.01\\
90.62	0.01\\
90.63	0.01\\
90.64	0.01\\
90.65	0.01\\
90.66	0.01\\
90.67	0.01\\
90.68	0.01\\
90.69	0.01\\
90.7	0.01\\
90.71	0.01\\
90.72	0.01\\
90.73	0.01\\
90.74	0.01\\
90.75	0.01\\
90.76	0.01\\
90.77	0.01\\
90.78	0.01\\
90.79	0.01\\
90.8	0.01\\
90.81	0.01\\
90.82	0.01\\
90.83	0.01\\
90.84	0.01\\
90.85	0.01\\
90.86	0.01\\
90.87	0.01\\
90.88	0.01\\
90.89	0.01\\
90.9	0.01\\
90.91	0.01\\
90.92	0.01\\
90.93	0.01\\
90.94	0.01\\
90.95	0.01\\
90.96	0.01\\
90.97	0.01\\
90.98	0.01\\
90.99	0.01\\
91	0.01\\
91.01	0.01\\
91.02	0.01\\
91.03	0.01\\
91.04	0.01\\
91.05	0.01\\
91.06	0.01\\
91.07	0.01\\
91.08	0.01\\
91.09	0.01\\
91.1	0.01\\
91.11	0.01\\
91.12	0.01\\
91.13	0.01\\
91.14	0.01\\
91.15	0.01\\
91.16	0.01\\
91.17	0.01\\
91.18	0.01\\
91.19	0.01\\
91.2	0.01\\
91.21	0.01\\
91.22	0.01\\
91.23	0.01\\
91.24	0.01\\
91.25	0.01\\
91.26	0.01\\
91.27	0.01\\
91.28	0.01\\
91.29	0.01\\
91.3	0.01\\
91.31	0.01\\
91.32	0.01\\
91.33	0.01\\
91.34	0.01\\
91.35	0.01\\
91.36	0.01\\
91.37	0.01\\
91.38	0.01\\
91.39	0.01\\
91.4	0.01\\
91.41	0.01\\
91.42	0.01\\
91.43	0.01\\
91.44	0.01\\
91.45	0.01\\
91.46	0.01\\
91.47	0.01\\
91.48	0.01\\
91.49	0.01\\
91.5	0.01\\
91.51	0.01\\
91.52	0.01\\
91.53	0.01\\
91.54	0.01\\
91.55	0.01\\
91.56	0.01\\
91.57	0.01\\
91.58	0.01\\
91.59	0.01\\
91.6	0.01\\
91.61	0.01\\
91.62	0.01\\
91.63	0.01\\
91.64	0.01\\
91.65	0.01\\
91.66	0.01\\
91.67	0.01\\
91.68	0.01\\
91.69	0.01\\
91.7	0.01\\
91.71	0.01\\
91.72	0.01\\
91.73	0.01\\
91.74	0.01\\
91.75	0.01\\
91.76	0.01\\
91.77	0.01\\
91.78	0.01\\
91.79	0.01\\
91.8	0.01\\
91.81	0.01\\
91.82	0.01\\
91.83	0.01\\
91.84	0.01\\
91.85	0.01\\
91.86	0.01\\
91.87	0.01\\
91.88	0.01\\
91.89	0.01\\
91.9	0.01\\
91.91	0.01\\
91.92	0.01\\
91.93	0.01\\
91.94	0.01\\
91.95	0.01\\
91.96	0.01\\
91.97	0.01\\
91.98	0.01\\
91.99	0.01\\
92	0.01\\
92.01	0.01\\
92.02	0.01\\
92.03	0.01\\
92.04	0.01\\
92.05	0.01\\
92.06	0.01\\
92.07	0.01\\
92.08	0.01\\
92.09	0.01\\
92.1	0.01\\
92.11	0.01\\
92.12	0.01\\
92.13	0.01\\
92.14	0.01\\
92.15	0.01\\
92.16	0.01\\
92.17	0.01\\
92.18	0.01\\
92.19	0.01\\
92.2	0.01\\
92.21	0.01\\
92.22	0.01\\
92.23	0.01\\
92.24	0.01\\
92.25	0.01\\
92.26	0.01\\
92.27	0.01\\
92.28	0.01\\
92.29	0.01\\
92.3	0.01\\
92.31	0.01\\
92.32	0.01\\
92.33	0.01\\
92.34	0.01\\
92.35	0.01\\
92.36	0.01\\
92.37	0.01\\
92.38	0.01\\
92.39	0.01\\
92.4	0.01\\
92.41	0.01\\
92.42	0.01\\
92.43	0.01\\
92.44	0.01\\
92.45	0.01\\
92.46	0.01\\
92.47	0.01\\
92.48	0.01\\
92.49	0.01\\
92.5	0.01\\
92.51	0.01\\
92.52	0.01\\
92.53	0.01\\
92.54	0.01\\
92.55	0.01\\
92.56	0.01\\
92.57	0.01\\
92.58	0.01\\
92.59	0.01\\
92.6	0.01\\
92.61	0.01\\
92.62	0.01\\
92.63	0.01\\
92.64	0.01\\
92.65	0.01\\
92.66	0.01\\
92.67	0.01\\
92.68	0.01\\
92.69	0.01\\
92.7	0.01\\
92.71	0.01\\
92.72	0.01\\
92.73	0.01\\
92.74	0.01\\
92.75	0.01\\
92.76	0.01\\
92.77	0.01\\
92.78	0.01\\
92.79	0.01\\
92.8	0.01\\
92.81	0.01\\
92.82	0.01\\
92.83	0.01\\
92.84	0.01\\
92.85	0.01\\
92.86	0.01\\
92.87	0.01\\
92.88	0.01\\
92.89	0.01\\
92.9	0.01\\
92.91	0.01\\
92.92	0.01\\
92.93	0.01\\
92.94	0.01\\
92.95	0.01\\
92.96	0.01\\
92.97	0.01\\
92.98	0.01\\
92.99	0.01\\
93	0.01\\
93.01	0.01\\
93.02	0.01\\
93.03	0.01\\
93.04	0.01\\
93.05	0.01\\
93.06	0.01\\
93.07	0.01\\
93.08	0.01\\
93.09	0.01\\
93.1	0.01\\
93.11	0.01\\
93.12	0.01\\
93.13	0.01\\
93.14	0.01\\
93.15	0.01\\
93.16	0.01\\
93.17	0.01\\
93.18	0.01\\
93.19	0.01\\
93.2	0.01\\
93.21	0.01\\
93.22	0.01\\
93.23	0.01\\
93.24	0.01\\
93.25	0.01\\
93.26	0.01\\
93.27	0.01\\
93.28	0.01\\
93.29	0.01\\
93.3	0.01\\
93.31	0.01\\
93.32	0.01\\
93.33	0.01\\
93.34	0.01\\
93.35	0.01\\
93.36	0.01\\
93.37	0.01\\
93.38	0.01\\
93.39	0.01\\
93.4	0.01\\
93.41	0.01\\
93.42	0.01\\
93.43	0.01\\
93.44	0.01\\
93.45	0.01\\
93.46	0.01\\
93.47	0.01\\
93.48	0.01\\
93.49	0.01\\
93.5	0.01\\
93.51	0.01\\
93.52	0.01\\
93.53	0.01\\
93.54	0.01\\
93.55	0.01\\
93.56	0.01\\
93.57	0.01\\
93.58	0.01\\
93.59	0.01\\
93.6	0.01\\
93.61	0.01\\
93.62	0.01\\
93.63	0.01\\
93.64	0.01\\
93.65	0.01\\
93.66	0.01\\
93.67	0.01\\
93.68	0.01\\
93.69	0.01\\
93.7	0.01\\
93.71	0.01\\
93.72	0.01\\
93.73	0.01\\
93.74	0.01\\
93.75	0.01\\
93.76	0.01\\
93.77	0.01\\
93.78	0.01\\
93.79	0.01\\
93.8	0.01\\
93.81	0.01\\
93.82	0.01\\
93.83	0.01\\
93.84	0.01\\
93.85	0.01\\
93.86	0.01\\
93.87	0.01\\
93.88	0.01\\
93.89	0.01\\
93.9	0.01\\
93.91	0.01\\
93.92	0.01\\
93.93	0.01\\
93.94	0.01\\
93.95	0.01\\
93.96	0.01\\
93.97	0.01\\
93.98	0.01\\
93.99	0.01\\
94	0.01\\
94.01	0.01\\
94.02	0.01\\
94.03	0.01\\
94.04	0.01\\
94.05	0.01\\
94.06	0.01\\
94.07	0.01\\
94.08	0.01\\
94.09	0.01\\
94.1	0.01\\
94.11	0.01\\
94.12	0.01\\
94.13	0.01\\
94.14	0.01\\
94.15	0.01\\
94.16	0.01\\
94.17	0.01\\
94.18	0.01\\
94.19	0.01\\
94.2	0.01\\
94.21	0.01\\
94.22	0.01\\
94.23	0.01\\
94.24	0.01\\
94.25	0.01\\
94.26	0.01\\
94.27	0.01\\
94.28	0.01\\
94.29	0.01\\
94.3	0.01\\
94.31	0.01\\
94.32	0.01\\
94.33	0.01\\
94.34	0.01\\
94.35	0.01\\
94.36	0.01\\
94.37	0.01\\
94.38	0.01\\
94.39	0.01\\
94.4	0.01\\
94.41	0.01\\
94.42	0.01\\
94.43	0.01\\
94.44	0.01\\
94.45	0.01\\
94.46	0.01\\
94.47	0.01\\
94.48	0.01\\
94.49	0.01\\
94.5	0.01\\
94.51	0.01\\
94.52	0.01\\
94.53	0.01\\
94.54	0.01\\
94.55	0.01\\
94.56	0.01\\
94.57	0.01\\
94.58	0.01\\
94.59	0.01\\
94.6	0.01\\
94.61	0.01\\
94.62	0.01\\
94.63	0.01\\
94.64	0.01\\
94.65	0.01\\
94.66	0.01\\
94.67	0.01\\
94.68	0.01\\
94.69	0.01\\
94.7	0.01\\
94.71	0.01\\
94.72	0.01\\
94.73	0.01\\
94.74	0.01\\
94.75	0.01\\
94.76	0.01\\
94.77	0.01\\
94.78	0.01\\
94.79	0.01\\
94.8	0.01\\
94.81	0.01\\
94.82	0.01\\
94.83	0.01\\
94.84	0.01\\
94.85	0.01\\
94.86	0.01\\
94.87	0.01\\
94.88	0.01\\
94.89	0.01\\
94.9	0.01\\
94.91	0.01\\
94.92	0.01\\
94.93	0.01\\
94.94	0.01\\
94.95	0.01\\
94.96	0.01\\
94.97	0.01\\
94.98	0.01\\
94.99	0.01\\
95	0.01\\
95.01	0.01\\
95.02	0.01\\
95.03	0.01\\
95.04	0.01\\
95.05	0.01\\
95.06	0.01\\
95.07	0.01\\
95.08	0.01\\
95.09	0.01\\
95.1	0.01\\
95.11	0.01\\
95.12	0.01\\
95.13	0.01\\
95.14	0.01\\
95.15	0.01\\
95.16	0.01\\
95.17	0.01\\
95.18	0.01\\
95.19	0.01\\
95.2	0.01\\
95.21	0.01\\
95.22	0.01\\
95.23	0.01\\
95.24	0.01\\
95.25	0.01\\
95.26	0.01\\
95.27	0.01\\
95.28	0.01\\
95.29	0.01\\
95.3	0.01\\
95.31	0.01\\
95.32	0.01\\
95.33	0.01\\
95.34	0.01\\
95.35	0.01\\
95.36	0.01\\
95.37	0.01\\
95.38	0.01\\
95.39	0.01\\
95.4	0.01\\
95.41	0.01\\
95.42	0.01\\
95.43	0.01\\
95.44	0.01\\
95.45	0.01\\
95.46	0.01\\
95.47	0.01\\
95.48	0.01\\
95.49	0.01\\
95.5	0.01\\
95.51	0.01\\
95.52	0.01\\
95.53	0.01\\
95.54	0.01\\
95.55	0.01\\
95.56	0.01\\
95.57	0.01\\
95.58	0.01\\
95.59	0.01\\
95.6	0.01\\
95.61	0.01\\
95.62	0.01\\
95.63	0.01\\
95.64	0.01\\
95.65	0.01\\
95.66	0.01\\
95.67	0.01\\
95.68	0.01\\
95.69	0.01\\
95.7	0.01\\
95.71	0.01\\
95.72	0.01\\
95.73	0.01\\
95.74	0.01\\
95.75	0.01\\
95.76	0.01\\
95.77	0.01\\
95.78	0.01\\
95.79	0.01\\
95.8	0.01\\
95.81	0.01\\
95.82	0.01\\
95.83	0.01\\
95.84	0.01\\
95.85	0.01\\
95.86	0.01\\
95.87	0.01\\
95.88	0.01\\
95.89	0.01\\
95.9	0.01\\
95.91	0.01\\
95.92	0.01\\
95.93	0.01\\
95.94	0.01\\
95.95	0.01\\
95.96	0.01\\
95.97	0.01\\
95.98	0.01\\
95.99	0.01\\
96	0.01\\
96.01	0.01\\
96.02	0.01\\
96.03	0.01\\
96.04	0.01\\
96.05	0.01\\
96.06	0.01\\
96.07	0.01\\
96.08	0.01\\
96.09	0.01\\
96.1	0.01\\
96.11	0.01\\
96.12	0.01\\
96.13	0.01\\
96.14	0.01\\
96.15	0.01\\
96.16	0.01\\
96.17	0.01\\
96.18	0.01\\
96.19	0.01\\
96.2	0.01\\
96.21	0.01\\
96.22	0.01\\
96.23	0.01\\
96.24	0.01\\
96.25	0.01\\
96.26	0.01\\
96.27	0.01\\
96.28	0.01\\
96.29	0.01\\
96.3	0.01\\
96.31	0.01\\
96.32	0.01\\
96.33	0.01\\
96.34	0.01\\
96.35	0.01\\
96.36	0.01\\
96.37	0.01\\
96.38	0.01\\
96.39	0.01\\
96.4	0.01\\
96.41	0.01\\
96.42	0.01\\
96.43	0.01\\
96.44	0.01\\
96.45	0.01\\
96.46	0.01\\
96.47	0.01\\
96.48	0.01\\
96.49	0.01\\
96.5	0.01\\
96.51	0.01\\
96.52	0.01\\
96.53	0.01\\
96.54	0.01\\
96.55	0.01\\
96.56	0.01\\
96.57	0.01\\
96.58	0.01\\
96.59	0.01\\
96.6	0.01\\
96.61	0.01\\
96.62	0.01\\
96.63	0.01\\
96.64	0.01\\
96.65	0.01\\
96.66	0.01\\
96.67	0.01\\
96.68	0.01\\
96.69	0.01\\
96.7	0.01\\
96.71	0.01\\
96.72	0.01\\
96.73	0.01\\
96.74	0.01\\
96.75	0.01\\
96.76	0.01\\
96.77	0.01\\
96.78	0.01\\
96.79	0.01\\
96.8	0.01\\
96.81	0.01\\
96.82	0.01\\
96.83	0.01\\
96.84	0.01\\
96.85	0.01\\
96.86	0.01\\
96.87	0.01\\
96.88	0.01\\
96.89	0.01\\
96.9	0.01\\
96.91	0.01\\
96.92	0.01\\
96.93	0.01\\
96.94	0.01\\
96.95	0.01\\
96.96	0.01\\
96.97	0.01\\
96.98	0.01\\
96.99	0.01\\
97	0.01\\
97.01	0.01\\
97.02	0.01\\
97.03	0.01\\
97.04	0.01\\
97.05	0.01\\
97.06	0.01\\
97.07	0.01\\
97.08	0.01\\
97.09	0.01\\
97.1	0.01\\
97.11	0.01\\
97.12	0.01\\
97.13	0.01\\
97.14	0.01\\
97.15	0.01\\
97.16	0.01\\
97.17	0.01\\
97.18	0.01\\
97.19	0.01\\
97.2	0.01\\
97.21	0.01\\
97.22	0.01\\
97.23	0.01\\
97.24	0.01\\
97.25	0.01\\
97.26	0.01\\
97.27	0.01\\
97.28	0.01\\
97.29	0.01\\
97.3	0.01\\
97.31	0.01\\
97.32	0.01\\
97.33	0.01\\
97.34	0.01\\
97.35	0.01\\
97.36	0.01\\
97.37	0.01\\
97.38	0.01\\
97.39	0.01\\
97.4	0.01\\
97.41	0.01\\
97.42	0.01\\
97.43	0.01\\
97.44	0.01\\
97.45	0.01\\
97.46	0.01\\
97.47	0.01\\
97.48	0.01\\
97.49	0.01\\
97.5	0.01\\
97.51	0.01\\
97.52	0.01\\
97.53	0.01\\
97.54	0.01\\
97.55	0.01\\
97.56	0.01\\
97.57	0.01\\
97.58	0.01\\
97.59	0.01\\
97.6	0.01\\
97.61	0.01\\
97.62	0.01\\
97.63	0.01\\
97.64	0.01\\
97.65	0.01\\
97.66	0.01\\
97.67	0.01\\
97.68	0.01\\
97.69	0.01\\
97.7	0.01\\
97.71	0.01\\
97.72	0.01\\
97.73	0.01\\
97.74	0.01\\
97.75	0.01\\
97.76	0.01\\
97.77	0.01\\
97.78	0.01\\
97.79	0.01\\
97.8	0.01\\
97.81	0.01\\
97.82	0.01\\
97.83	0.01\\
97.84	0.01\\
97.85	0.01\\
97.86	0.01\\
97.87	0.01\\
97.88	0.01\\
97.89	0.01\\
97.9	0.01\\
97.91	0.01\\
97.92	0.01\\
97.93	0.01\\
97.94	0.01\\
97.95	0.01\\
97.96	0.01\\
97.97	0.01\\
97.98	0.01\\
97.99	0.01\\
98	0.01\\
98.01	0.01\\
98.02	0.01\\
98.03	0.01\\
98.04	0.01\\
98.05	0.01\\
98.06	0.01\\
98.07	0.01\\
98.08	0.01\\
98.09	0.01\\
98.1	0.01\\
98.11	0.01\\
98.12	0.01\\
98.13	0.01\\
98.14	0.01\\
98.15	0.01\\
98.16	0.01\\
98.17	0.01\\
98.18	0.01\\
98.19	0.01\\
98.2	0.01\\
98.21	0.01\\
98.22	0.01\\
98.23	0.01\\
98.24	0.01\\
98.25	0.01\\
98.26	0.01\\
98.27	0.01\\
98.28	0.01\\
98.29	0.01\\
98.3	0.01\\
98.31	0.01\\
98.32	0.01\\
98.33	0.01\\
98.34	0.01\\
98.35	0.01\\
98.36	0.01\\
98.37	0.01\\
98.38	0.01\\
98.39	0.01\\
98.4	0.01\\
98.41	0.01\\
98.42	0.01\\
98.43	0.01\\
98.44	0.01\\
98.45	0.01\\
98.46	0.01\\
98.47	0.01\\
98.48	0.01\\
98.49	0.01\\
98.5	0.01\\
98.51	0.01\\
98.52	0.01\\
98.53	0.01\\
98.54	0.01\\
98.55	0.01\\
98.56	0.01\\
98.57	0.01\\
98.58	0.01\\
98.59	0.01\\
98.6	0.01\\
98.61	0.01\\
98.62	0.01\\
98.63	0.01\\
98.64	0.01\\
98.65	0.01\\
98.66	0.01\\
98.67	0.01\\
98.68	0.01\\
98.69	0.01\\
98.7	0.01\\
98.71	0.01\\
98.72	0.01\\
98.73	0.01\\
98.74	0.01\\
98.75	0.01\\
98.76	0.01\\
98.77	0.01\\
98.78	0.01\\
98.79	0.01\\
98.8	0.01\\
98.81	0.01\\
98.82	0.01\\
98.83	0.01\\
98.84	0.01\\
98.85	0.01\\
98.86	0.01\\
98.87	0.01\\
98.88	0.01\\
98.89	0.01\\
98.9	0.01\\
98.91	0.01\\
98.92	0.01\\
98.93	0.01\\
98.94	0.01\\
98.95	0.01\\
98.96	0.01\\
98.97	0.01\\
98.98	0.01\\
98.99	0.01\\
99	0.01\\
99.01	0.01\\
99.02	0.01\\
99.03	0.01\\
99.04	0.01\\
99.05	0.01\\
99.06	0.01\\
99.07	0.01\\
99.08	0.01\\
99.09	0.01\\
99.1	0.01\\
99.11	0.01\\
99.12	0.01\\
99.13	0.01\\
99.14	0.01\\
99.15	0.01\\
99.16	0.01\\
99.17	0.01\\
99.18	0.01\\
99.19	0.01\\
99.2	0.01\\
99.21	0.01\\
99.22	0.01\\
99.23	0.01\\
99.24	0.01\\
99.25	0.01\\
99.26	0.01\\
99.27	0.01\\
99.28	0.01\\
99.29	0.01\\
99.3	0.01\\
99.31	0.01\\
99.32	0.01\\
99.33	0.01\\
99.34	0.01\\
99.35	0.01\\
99.36	0.01\\
99.37	0.01\\
99.38	0.01\\
99.39	0.01\\
99.4	0.01\\
99.41	0.01\\
99.42	0.01\\
99.43	0.01\\
99.44	0.01\\
99.45	0.01\\
99.46	0.01\\
99.47	0.01\\
99.48	0.01\\
99.49	0.01\\
99.5	0.01\\
99.51	0.01\\
99.52	0.01\\
99.53	0.01\\
99.54	0.01\\
99.55	0.01\\
99.56	0.01\\
99.57	0.01\\
99.58	0.01\\
99.59	0.01\\
99.6	0.01\\
99.61	0.01\\
99.62	0.01\\
99.63	0.01\\
99.64	0.01\\
99.65	0.01\\
99.66	0.01\\
99.67	0.01\\
99.68	0.01\\
99.69	0.01\\
99.7	0.01\\
99.71	0.01\\
99.72	0.01\\
99.73	0.01\\
99.74	0.01\\
99.75	0.01\\
99.76	0.01\\
99.77	0.01\\
99.78	0.01\\
99.79	0.01\\
99.8	0.01\\
99.81	0.01\\
99.82	0.01\\
99.83	0.01\\
99.84	0.01\\
99.85	0.01\\
99.86	0.01\\
99.87	0.01\\
99.88	0.01\\
99.89	0.01\\
99.9	0.01\\
99.91	0.01\\
99.92	0.01\\
99.93	0.01\\
99.94	0.01\\
99.95	0.01\\
99.96	0.01\\
99.97	0.01\\
99.98	0.01\\
99.99	0.01\\
100	0.01\\
};
\addlegendentry{$q=3$};

\addplot [color=green,solid,forget plot]
  table[row sep=crcr]{%
0.01	0.01\\
0.02	0.01\\
0.03	0.01\\
0.04	0.01\\
0.05	0.01\\
0.06	0.01\\
0.07	0.01\\
0.08	0.01\\
0.09	0.01\\
0.1	0.01\\
0.11	0.01\\
0.12	0.01\\
0.13	0.01\\
0.14	0.01\\
0.15	0.01\\
0.16	0.01\\
0.17	0.01\\
0.18	0.01\\
0.19	0.01\\
0.2	0.01\\
0.21	0.01\\
0.22	0.01\\
0.23	0.01\\
0.24	0.01\\
0.25	0.01\\
0.26	0.01\\
0.27	0.01\\
0.28	0.01\\
0.29	0.01\\
0.3	0.01\\
0.31	0.01\\
0.32	0.01\\
0.33	0.01\\
0.34	0.01\\
0.35	0.01\\
0.36	0.01\\
0.37	0.01\\
0.38	0.01\\
0.39	0.01\\
0.4	0.01\\
0.41	0.01\\
0.42	0.01\\
0.43	0.01\\
0.44	0.01\\
0.45	0.01\\
0.46	0.01\\
0.47	0.01\\
0.48	0.01\\
0.49	0.01\\
0.5	0.01\\
0.51	0.01\\
0.52	0.01\\
0.53	0.01\\
0.54	0.01\\
0.55	0.01\\
0.56	0.01\\
0.57	0.01\\
0.58	0.01\\
0.59	0.01\\
0.6	0.01\\
0.61	0.01\\
0.62	0.01\\
0.63	0.01\\
0.64	0.01\\
0.65	0.01\\
0.66	0.01\\
0.67	0.01\\
0.68	0.01\\
0.69	0.01\\
0.7	0.01\\
0.71	0.01\\
0.72	0.01\\
0.73	0.01\\
0.74	0.01\\
0.75	0.01\\
0.76	0.01\\
0.77	0.01\\
0.78	0.01\\
0.79	0.01\\
0.8	0.01\\
0.81	0.01\\
0.82	0.01\\
0.83	0.01\\
0.84	0.01\\
0.85	0.01\\
0.86	0.01\\
0.87	0.01\\
0.88	0.01\\
0.89	0.01\\
0.9	0.01\\
0.91	0.01\\
0.92	0.01\\
0.93	0.01\\
0.94	0.01\\
0.95	0.01\\
0.96	0.01\\
0.97	0.01\\
0.98	0.01\\
0.99	0.01\\
1	0.01\\
1.01	0.01\\
1.02	0.01\\
1.03	0.01\\
1.04	0.01\\
1.05	0.01\\
1.06	0.01\\
1.07	0.01\\
1.08	0.01\\
1.09	0.01\\
1.1	0.01\\
1.11	0.01\\
1.12	0.01\\
1.13	0.01\\
1.14	0.01\\
1.15	0.01\\
1.16	0.01\\
1.17	0.01\\
1.18	0.01\\
1.19	0.01\\
1.2	0.01\\
1.21	0.01\\
1.22	0.01\\
1.23	0.01\\
1.24	0.01\\
1.25	0.01\\
1.26	0.01\\
1.27	0.01\\
1.28	0.01\\
1.29	0.01\\
1.3	0.01\\
1.31	0.01\\
1.32	0.01\\
1.33	0.01\\
1.34	0.01\\
1.35	0.01\\
1.36	0.01\\
1.37	0.01\\
1.38	0.01\\
1.39	0.01\\
1.4	0.01\\
1.41	0.01\\
1.42	0.01\\
1.43	0.01\\
1.44	0.01\\
1.45	0.01\\
1.46	0.01\\
1.47	0.01\\
1.48	0.01\\
1.49	0.01\\
1.5	0.01\\
1.51	0.01\\
1.52	0.01\\
1.53	0.01\\
1.54	0.01\\
1.55	0.01\\
1.56	0.01\\
1.57	0.01\\
1.58	0.01\\
1.59	0.01\\
1.6	0.01\\
1.61	0.01\\
1.62	0.01\\
1.63	0.01\\
1.64	0.01\\
1.65	0.01\\
1.66	0.01\\
1.67	0.01\\
1.68	0.01\\
1.69	0.01\\
1.7	0.01\\
1.71	0.01\\
1.72	0.01\\
1.73	0.01\\
1.74	0.01\\
1.75	0.01\\
1.76	0.01\\
1.77	0.01\\
1.78	0.01\\
1.79	0.01\\
1.8	0.01\\
1.81	0.01\\
1.82	0.01\\
1.83	0.01\\
1.84	0.01\\
1.85	0.01\\
1.86	0.01\\
1.87	0.01\\
1.88	0.01\\
1.89	0.01\\
1.9	0.01\\
1.91	0.01\\
1.92	0.01\\
1.93	0.01\\
1.94	0.01\\
1.95	0.01\\
1.96	0.01\\
1.97	0.01\\
1.98	0.01\\
1.99	0.01\\
2	0.01\\
2.01	0.01\\
2.02	0.01\\
2.03	0.01\\
2.04	0.01\\
2.05	0.01\\
2.06	0.01\\
2.07	0.01\\
2.08	0.01\\
2.09	0.01\\
2.1	0.01\\
2.11	0.01\\
2.12	0.01\\
2.13	0.01\\
2.14	0.01\\
2.15	0.01\\
2.16	0.01\\
2.17	0.01\\
2.18	0.01\\
2.19	0.01\\
2.2	0.01\\
2.21	0.01\\
2.22	0.01\\
2.23	0.01\\
2.24	0.01\\
2.25	0.01\\
2.26	0.01\\
2.27	0.01\\
2.28	0.01\\
2.29	0.01\\
2.3	0.01\\
2.31	0.01\\
2.32	0.01\\
2.33	0.01\\
2.34	0.01\\
2.35	0.01\\
2.36	0.01\\
2.37	0.01\\
2.38	0.01\\
2.39	0.01\\
2.4	0.01\\
2.41	0.01\\
2.42	0.01\\
2.43	0.01\\
2.44	0.01\\
2.45	0.01\\
2.46	0.01\\
2.47	0.01\\
2.48	0.01\\
2.49	0.01\\
2.5	0.01\\
2.51	0.01\\
2.52	0.01\\
2.53	0.01\\
2.54	0.01\\
2.55	0.01\\
2.56	0.01\\
2.57	0.01\\
2.58	0.01\\
2.59	0.01\\
2.6	0.01\\
2.61	0.01\\
2.62	0.01\\
2.63	0.01\\
2.64	0.01\\
2.65	0.01\\
2.66	0.01\\
2.67	0.01\\
2.68	0.01\\
2.69	0.01\\
2.7	0.01\\
2.71	0.01\\
2.72	0.01\\
2.73	0.01\\
2.74	0.01\\
2.75	0.01\\
2.76	0.01\\
2.77	0.01\\
2.78	0.01\\
2.79	0.01\\
2.8	0.01\\
2.81	0.01\\
2.82	0.01\\
2.83	0.01\\
2.84	0.01\\
2.85	0.01\\
2.86	0.01\\
2.87	0.01\\
2.88	0.01\\
2.89	0.01\\
2.9	0.01\\
2.91	0.01\\
2.92	0.01\\
2.93	0.01\\
2.94	0.01\\
2.95	0.01\\
2.96	0.01\\
2.97	0.01\\
2.98	0.01\\
2.99	0.01\\
3	0.01\\
3.01	0.01\\
3.02	0.01\\
3.03	0.01\\
3.04	0.01\\
3.05	0.01\\
3.06	0.01\\
3.07	0.01\\
3.08	0.01\\
3.09	0.01\\
3.1	0.01\\
3.11	0.01\\
3.12	0.01\\
3.13	0.01\\
3.14	0.01\\
3.15	0.01\\
3.16	0.01\\
3.17	0.01\\
3.18	0.01\\
3.19	0.01\\
3.2	0.01\\
3.21	0.01\\
3.22	0.01\\
3.23	0.01\\
3.24	0.01\\
3.25	0.01\\
3.26	0.01\\
3.27	0.01\\
3.28	0.01\\
3.29	0.01\\
3.3	0.01\\
3.31	0.01\\
3.32	0.01\\
3.33	0.01\\
3.34	0.01\\
3.35	0.01\\
3.36	0.01\\
3.37	0.01\\
3.38	0.01\\
3.39	0.01\\
3.4	0.01\\
3.41	0.01\\
3.42	0.01\\
3.43	0.01\\
3.44	0.01\\
3.45	0.01\\
3.46	0.01\\
3.47	0.01\\
3.48	0.01\\
3.49	0.01\\
3.5	0.01\\
3.51	0.01\\
3.52	0.01\\
3.53	0.01\\
3.54	0.01\\
3.55	0.01\\
3.56	0.01\\
3.57	0.01\\
3.58	0.01\\
3.59	0.01\\
3.6	0.01\\
3.61	0.01\\
3.62	0.01\\
3.63	0.01\\
3.64	0.01\\
3.65	0.01\\
3.66	0.01\\
3.67	0.01\\
3.68	0.01\\
3.69	0.01\\
3.7	0.01\\
3.71	0.01\\
3.72	0.01\\
3.73	0.01\\
3.74	0.01\\
3.75	0.01\\
3.76	0.01\\
3.77	0.01\\
3.78	0.01\\
3.79	0.01\\
3.8	0.01\\
3.81	0.01\\
3.82	0.01\\
3.83	0.01\\
3.84	0.01\\
3.85	0.01\\
3.86	0.01\\
3.87	0.01\\
3.88	0.01\\
3.89	0.01\\
3.9	0.01\\
3.91	0.01\\
3.92	0.01\\
3.93	0.01\\
3.94	0.01\\
3.95	0.01\\
3.96	0.01\\
3.97	0.01\\
3.98	0.01\\
3.99	0.01\\
4	0.01\\
4.01	0.01\\
4.02	0.01\\
4.03	0.01\\
4.04	0.01\\
4.05	0.01\\
4.06	0.01\\
4.07	0.01\\
4.08	0.01\\
4.09	0.01\\
4.1	0.01\\
4.11	0.01\\
4.12	0.01\\
4.13	0.01\\
4.14	0.01\\
4.15	0.01\\
4.16	0.01\\
4.17	0.01\\
4.18	0.01\\
4.19	0.01\\
4.2	0.01\\
4.21	0.01\\
4.22	0.01\\
4.23	0.01\\
4.24	0.01\\
4.25	0.01\\
4.26	0.01\\
4.27	0.01\\
4.28	0.01\\
4.29	0.01\\
4.3	0.01\\
4.31	0.01\\
4.32	0.01\\
4.33	0.01\\
4.34	0.01\\
4.35	0.01\\
4.36	0.01\\
4.37	0.01\\
4.38	0.01\\
4.39	0.01\\
4.4	0.01\\
4.41	0.01\\
4.42	0.01\\
4.43	0.01\\
4.44	0.01\\
4.45	0.01\\
4.46	0.01\\
4.47	0.01\\
4.48	0.01\\
4.49	0.01\\
4.5	0.01\\
4.51	0.01\\
4.52	0.01\\
4.53	0.01\\
4.54	0.01\\
4.55	0.01\\
4.56	0.01\\
4.57	0.01\\
4.58	0.01\\
4.59	0.01\\
4.6	0.01\\
4.61	0.01\\
4.62	0.01\\
4.63	0.01\\
4.64	0.01\\
4.65	0.01\\
4.66	0.01\\
4.67	0.01\\
4.68	0.01\\
4.69	0.01\\
4.7	0.01\\
4.71	0.01\\
4.72	0.01\\
4.73	0.01\\
4.74	0.01\\
4.75	0.01\\
4.76	0.01\\
4.77	0.01\\
4.78	0.01\\
4.79	0.01\\
4.8	0.01\\
4.81	0.01\\
4.82	0.01\\
4.83	0.01\\
4.84	0.01\\
4.85	0.01\\
4.86	0.01\\
4.87	0.01\\
4.88	0.01\\
4.89	0.01\\
4.9	0.01\\
4.91	0.01\\
4.92	0.01\\
4.93	0.01\\
4.94	0.01\\
4.95	0.01\\
4.96	0.01\\
4.97	0.01\\
4.98	0.01\\
4.99	0.01\\
5	0.01\\
5.01	0.01\\
5.02	0.01\\
5.03	0.01\\
5.04	0.01\\
5.05	0.01\\
5.06	0.01\\
5.07	0.01\\
5.08	0.01\\
5.09	0.01\\
5.1	0.01\\
5.11	0.01\\
5.12	0.01\\
5.13	0.01\\
5.14	0.01\\
5.15	0.01\\
5.16	0.01\\
5.17	0.01\\
5.18	0.01\\
5.19	0.01\\
5.2	0.01\\
5.21	0.01\\
5.22	0.01\\
5.23	0.01\\
5.24	0.01\\
5.25	0.01\\
5.26	0.01\\
5.27	0.01\\
5.28	0.01\\
5.29	0.01\\
5.3	0.01\\
5.31	0.01\\
5.32	0.01\\
5.33	0.01\\
5.34	0.01\\
5.35	0.01\\
5.36	0.01\\
5.37	0.01\\
5.38	0.01\\
5.39	0.01\\
5.4	0.01\\
5.41	0.01\\
5.42	0.01\\
5.43	0.01\\
5.44	0.01\\
5.45	0.01\\
5.46	0.01\\
5.47	0.01\\
5.48	0.01\\
5.49	0.01\\
5.5	0.01\\
5.51	0.01\\
5.52	0.01\\
5.53	0.01\\
5.54	0.01\\
5.55	0.01\\
5.56	0.01\\
5.57	0.01\\
5.58	0.01\\
5.59	0.01\\
5.6	0.01\\
5.61	0.01\\
5.62	0.01\\
5.63	0.01\\
5.64	0.01\\
5.65	0.01\\
5.66	0.01\\
5.67	0.01\\
5.68	0.01\\
5.69	0.01\\
5.7	0.01\\
5.71	0.01\\
5.72	0.01\\
5.73	0.01\\
5.74	0.01\\
5.75	0.01\\
5.76	0.01\\
5.77	0.01\\
5.78	0.01\\
5.79	0.01\\
5.8	0.01\\
5.81	0.01\\
5.82	0.01\\
5.83	0.01\\
5.84	0.01\\
5.85	0.01\\
5.86	0.01\\
5.87	0.01\\
5.88	0.01\\
5.89	0.01\\
5.9	0.01\\
5.91	0.01\\
5.92	0.01\\
5.93	0.01\\
5.94	0.01\\
5.95	0.01\\
5.96	0.01\\
5.97	0.01\\
5.98	0.01\\
5.99	0.01\\
6	0.01\\
6.01	0.01\\
6.02	0.01\\
6.03	0.01\\
6.04	0.01\\
6.05	0.01\\
6.06	0.01\\
6.07	0.01\\
6.08	0.01\\
6.09	0.01\\
6.1	0.01\\
6.11	0.01\\
6.12	0.01\\
6.13	0.01\\
6.14	0.01\\
6.15	0.01\\
6.16	0.01\\
6.17	0.01\\
6.18	0.01\\
6.19	0.01\\
6.2	0.01\\
6.21	0.01\\
6.22	0.01\\
6.23	0.01\\
6.24	0.01\\
6.25	0.01\\
6.26	0.01\\
6.27	0.01\\
6.28	0.01\\
6.29	0.01\\
6.3	0.01\\
6.31	0.01\\
6.32	0.01\\
6.33	0.01\\
6.34	0.01\\
6.35	0.01\\
6.36	0.01\\
6.37	0.01\\
6.38	0.01\\
6.39	0.01\\
6.4	0.01\\
6.41	0.01\\
6.42	0.01\\
6.43	0.01\\
6.44	0.01\\
6.45	0.01\\
6.46	0.01\\
6.47	0.01\\
6.48	0.01\\
6.49	0.01\\
6.5	0.01\\
6.51	0.01\\
6.52	0.01\\
6.53	0.01\\
6.54	0.01\\
6.55	0.01\\
6.56	0.01\\
6.57	0.01\\
6.58	0.01\\
6.59	0.01\\
6.6	0.01\\
6.61	0.01\\
6.62	0.01\\
6.63	0.01\\
6.64	0.01\\
6.65	0.01\\
6.66	0.01\\
6.67	0.01\\
6.68	0.01\\
6.69	0.01\\
6.7	0.01\\
6.71	0.01\\
6.72	0.01\\
6.73	0.01\\
6.74	0.01\\
6.75	0.01\\
6.76	0.01\\
6.77	0.01\\
6.78	0.01\\
6.79	0.01\\
6.8	0.01\\
6.81	0.01\\
6.82	0.01\\
6.83	0.01\\
6.84	0.01\\
6.85	0.01\\
6.86	0.01\\
6.87	0.01\\
6.88	0.01\\
6.89	0.01\\
6.9	0.01\\
6.91	0.01\\
6.92	0.01\\
6.93	0.01\\
6.94	0.01\\
6.95	0.01\\
6.96	0.01\\
6.97	0.01\\
6.98	0.01\\
6.99	0.01\\
7	0.01\\
7.01	0.01\\
7.02	0.01\\
7.03	0.01\\
7.04	0.01\\
7.05	0.01\\
7.06	0.01\\
7.07	0.01\\
7.08	0.01\\
7.09	0.01\\
7.1	0.01\\
7.11	0.01\\
7.12	0.01\\
7.13	0.01\\
7.14	0.01\\
7.15	0.01\\
7.16	0.01\\
7.17	0.01\\
7.18	0.01\\
7.19	0.01\\
7.2	0.01\\
7.21	0.01\\
7.22	0.01\\
7.23	0.01\\
7.24	0.01\\
7.25	0.01\\
7.26	0.01\\
7.27	0.01\\
7.28	0.01\\
7.29	0.01\\
7.3	0.01\\
7.31	0.01\\
7.32	0.01\\
7.33	0.01\\
7.34	0.01\\
7.35	0.01\\
7.36	0.01\\
7.37	0.01\\
7.38	0.01\\
7.39	0.01\\
7.4	0.01\\
7.41	0.01\\
7.42	0.01\\
7.43	0.01\\
7.44	0.01\\
7.45	0.01\\
7.46	0.01\\
7.47	0.01\\
7.48	0.01\\
7.49	0.01\\
7.5	0.01\\
7.51	0.01\\
7.52	0.01\\
7.53	0.01\\
7.54	0.01\\
7.55	0.01\\
7.56	0.01\\
7.57	0.01\\
7.58	0.01\\
7.59	0.01\\
7.6	0.01\\
7.61	0.01\\
7.62	0.01\\
7.63	0.01\\
7.64	0.01\\
7.65	0.01\\
7.66	0.01\\
7.67	0.01\\
7.68	0.01\\
7.69	0.01\\
7.7	0.01\\
7.71	0.01\\
7.72	0.01\\
7.73	0.01\\
7.74	0.01\\
7.75	0.01\\
7.76	0.01\\
7.77	0.01\\
7.78	0.01\\
7.79	0.01\\
7.8	0.01\\
7.81	0.01\\
7.82	0.01\\
7.83	0.01\\
7.84	0.01\\
7.85	0.01\\
7.86	0.01\\
7.87	0.01\\
7.88	0.01\\
7.89	0.01\\
7.9	0.01\\
7.91	0.01\\
7.92	0.01\\
7.93	0.01\\
7.94	0.01\\
7.95	0.01\\
7.96	0.01\\
7.97	0.01\\
7.98	0.01\\
7.99	0.01\\
8	0.01\\
8.01	0.01\\
8.02	0.01\\
8.03	0.01\\
8.04	0.01\\
8.05	0.01\\
8.06	0.01\\
8.07	0.01\\
8.08	0.01\\
8.09	0.01\\
8.1	0.01\\
8.11	0.01\\
8.12	0.01\\
8.13	0.01\\
8.14	0.01\\
8.15	0.01\\
8.16	0.01\\
8.17	0.01\\
8.18	0.01\\
8.19	0.01\\
8.2	0.01\\
8.21	0.01\\
8.22	0.01\\
8.23	0.01\\
8.24	0.01\\
8.25	0.01\\
8.26	0.01\\
8.27	0.01\\
8.28	0.01\\
8.29	0.01\\
8.3	0.01\\
8.31	0.01\\
8.32	0.01\\
8.33	0.01\\
8.34	0.01\\
8.35	0.01\\
8.36	0.01\\
8.37	0.01\\
8.38	0.01\\
8.39	0.01\\
8.4	0.01\\
8.41	0.01\\
8.42	0.01\\
8.43	0.01\\
8.44	0.01\\
8.45	0.01\\
8.46	0.01\\
8.47	0.01\\
8.48	0.01\\
8.49	0.01\\
8.5	0.01\\
8.51	0.01\\
8.52	0.01\\
8.53	0.01\\
8.54	0.01\\
8.55	0.01\\
8.56	0.01\\
8.57	0.01\\
8.58	0.01\\
8.59	0.01\\
8.6	0.01\\
8.61	0.01\\
8.62	0.01\\
8.63	0.01\\
8.64	0.01\\
8.65	0.01\\
8.66	0.01\\
8.67	0.01\\
8.68	0.01\\
8.69	0.01\\
8.7	0.01\\
8.71	0.01\\
8.72	0.01\\
8.73	0.01\\
8.74	0.01\\
8.75	0.01\\
8.76	0.01\\
8.77	0.01\\
8.78	0.01\\
8.79	0.01\\
8.8	0.01\\
8.81	0.01\\
8.82	0.01\\
8.83	0.01\\
8.84	0.01\\
8.85	0.01\\
8.86	0.01\\
8.87	0.01\\
8.88	0.01\\
8.89	0.01\\
8.9	0.01\\
8.91	0.01\\
8.92	0.01\\
8.93	0.01\\
8.94	0.01\\
8.95	0.01\\
8.96	0.01\\
8.97	0.01\\
8.98	0.01\\
8.99	0.01\\
9	0.01\\
9.01	0.01\\
9.02	0.01\\
9.03	0.01\\
9.04	0.01\\
9.05	0.01\\
9.06	0.01\\
9.07	0.01\\
9.08	0.01\\
9.09	0.01\\
9.1	0.01\\
9.11	0.01\\
9.12	0.01\\
9.13	0.01\\
9.14	0.01\\
9.15	0.01\\
9.16	0.01\\
9.17	0.01\\
9.18	0.01\\
9.19	0.01\\
9.2	0.01\\
9.21	0.01\\
9.22	0.01\\
9.23	0.01\\
9.24	0.01\\
9.25	0.01\\
9.26	0.01\\
9.27	0.01\\
9.28	0.01\\
9.29	0.01\\
9.3	0.01\\
9.31	0.01\\
9.32	0.01\\
9.33	0.01\\
9.34	0.01\\
9.35	0.01\\
9.36	0.01\\
9.37	0.01\\
9.38	0.01\\
9.39	0.01\\
9.4	0.01\\
9.41	0.01\\
9.42	0.01\\
9.43	0.01\\
9.44	0.01\\
9.45	0.01\\
9.46	0.01\\
9.47	0.01\\
9.48	0.01\\
9.49	0.01\\
9.5	0.01\\
9.51	0.01\\
9.52	0.01\\
9.53	0.01\\
9.54	0.01\\
9.55	0.01\\
9.56	0.01\\
9.57	0.01\\
9.58	0.01\\
9.59	0.01\\
9.6	0.01\\
9.61	0.01\\
9.62	0.01\\
9.63	0.01\\
9.64	0.01\\
9.65	0.01\\
9.66	0.01\\
9.67	0.01\\
9.68	0.01\\
9.69	0.01\\
9.7	0.01\\
9.71	0.01\\
9.72	0.01\\
9.73	0.01\\
9.74	0.01\\
9.75	0.01\\
9.76	0.01\\
9.77	0.01\\
9.78	0.01\\
9.79	0.01\\
9.8	0.01\\
9.81	0.01\\
9.82	0.01\\
9.83	0.01\\
9.84	0.01\\
9.85	0.01\\
9.86	0.01\\
9.87	0.01\\
9.88	0.01\\
9.89	0.01\\
9.9	0.01\\
9.91	0.01\\
9.92	0.01\\
9.93	0.01\\
9.94	0.01\\
9.95	0.01\\
9.96	0.01\\
9.97	0.01\\
9.98	0.01\\
9.99	0.01\\
10	0.01\\
10.01	0.01\\
10.02	0.01\\
10.03	0.01\\
10.04	0.01\\
10.05	0.01\\
10.06	0.01\\
10.07	0.01\\
10.08	0.01\\
10.09	0.01\\
10.1	0.01\\
10.11	0.01\\
10.12	0.01\\
10.13	0.01\\
10.14	0.01\\
10.15	0.01\\
10.16	0.01\\
10.17	0.01\\
10.18	0.01\\
10.19	0.01\\
10.2	0.01\\
10.21	0.01\\
10.22	0.01\\
10.23	0.01\\
10.24	0.01\\
10.25	0.01\\
10.26	0.01\\
10.27	0.01\\
10.28	0.01\\
10.29	0.01\\
10.3	0.01\\
10.31	0.01\\
10.32	0.01\\
10.33	0.01\\
10.34	0.01\\
10.35	0.01\\
10.36	0.01\\
10.37	0.01\\
10.38	0.01\\
10.39	0.01\\
10.4	0.01\\
10.41	0.01\\
10.42	0.01\\
10.43	0.01\\
10.44	0.01\\
10.45	0.01\\
10.46	0.01\\
10.47	0.01\\
10.48	0.01\\
10.49	0.01\\
10.5	0.01\\
10.51	0.01\\
10.52	0.01\\
10.53	0.01\\
10.54	0.01\\
10.55	0.01\\
10.56	0.01\\
10.57	0.01\\
10.58	0.01\\
10.59	0.01\\
10.6	0.01\\
10.61	0.01\\
10.62	0.01\\
10.63	0.01\\
10.64	0.01\\
10.65	0.01\\
10.66	0.01\\
10.67	0.01\\
10.68	0.01\\
10.69	0.01\\
10.7	0.01\\
10.71	0.01\\
10.72	0.01\\
10.73	0.01\\
10.74	0.01\\
10.75	0.01\\
10.76	0.01\\
10.77	0.01\\
10.78	0.01\\
10.79	0.01\\
10.8	0.01\\
10.81	0.01\\
10.82	0.01\\
10.83	0.01\\
10.84	0.01\\
10.85	0.01\\
10.86	0.01\\
10.87	0.01\\
10.88	0.01\\
10.89	0.01\\
10.9	0.01\\
10.91	0.01\\
10.92	0.01\\
10.93	0.01\\
10.94	0.01\\
10.95	0.01\\
10.96	0.01\\
10.97	0.01\\
10.98	0.01\\
10.99	0.01\\
11	0.01\\
11.01	0.01\\
11.02	0.01\\
11.03	0.01\\
11.04	0.01\\
11.05	0.01\\
11.06	0.01\\
11.07	0.01\\
11.08	0.01\\
11.09	0.01\\
11.1	0.01\\
11.11	0.01\\
11.12	0.01\\
11.13	0.01\\
11.14	0.01\\
11.15	0.01\\
11.16	0.01\\
11.17	0.01\\
11.18	0.01\\
11.19	0.01\\
11.2	0.01\\
11.21	0.01\\
11.22	0.01\\
11.23	0.01\\
11.24	0.01\\
11.25	0.01\\
11.26	0.01\\
11.27	0.01\\
11.28	0.01\\
11.29	0.01\\
11.3	0.01\\
11.31	0.01\\
11.32	0.01\\
11.33	0.01\\
11.34	0.01\\
11.35	0.01\\
11.36	0.01\\
11.37	0.01\\
11.38	0.01\\
11.39	0.01\\
11.4	0.01\\
11.41	0.01\\
11.42	0.01\\
11.43	0.01\\
11.44	0.01\\
11.45	0.01\\
11.46	0.01\\
11.47	0.01\\
11.48	0.01\\
11.49	0.01\\
11.5	0.01\\
11.51	0.01\\
11.52	0.01\\
11.53	0.01\\
11.54	0.01\\
11.55	0.01\\
11.56	0.01\\
11.57	0.01\\
11.58	0.01\\
11.59	0.01\\
11.6	0.01\\
11.61	0.01\\
11.62	0.01\\
11.63	0.01\\
11.64	0.01\\
11.65	0.01\\
11.66	0.01\\
11.67	0.01\\
11.68	0.01\\
11.69	0.01\\
11.7	0.01\\
11.71	0.01\\
11.72	0.01\\
11.73	0.01\\
11.74	0.01\\
11.75	0.01\\
11.76	0.01\\
11.77	0.01\\
11.78	0.01\\
11.79	0.01\\
11.8	0.01\\
11.81	0.01\\
11.82	0.01\\
11.83	0.01\\
11.84	0.01\\
11.85	0.01\\
11.86	0.01\\
11.87	0.01\\
11.88	0.01\\
11.89	0.01\\
11.9	0.01\\
11.91	0.01\\
11.92	0.01\\
11.93	0.01\\
11.94	0.01\\
11.95	0.01\\
11.96	0.01\\
11.97	0.01\\
11.98	0.01\\
11.99	0.01\\
12	0.01\\
12.01	0.01\\
12.02	0.01\\
12.03	0.01\\
12.04	0.01\\
12.05	0.01\\
12.06	0.01\\
12.07	0.01\\
12.08	0.01\\
12.09	0.01\\
12.1	0.01\\
12.11	0.01\\
12.12	0.01\\
12.13	0.01\\
12.14	0.01\\
12.15	0.01\\
12.16	0.01\\
12.17	0.01\\
12.18	0.01\\
12.19	0.01\\
12.2	0.01\\
12.21	0.01\\
12.22	0.01\\
12.23	0.01\\
12.24	0.01\\
12.25	0.01\\
12.26	0.01\\
12.27	0.01\\
12.28	0.01\\
12.29	0.01\\
12.3	0.01\\
12.31	0.01\\
12.32	0.01\\
12.33	0.01\\
12.34	0.01\\
12.35	0.01\\
12.36	0.01\\
12.37	0.01\\
12.38	0.01\\
12.39	0.01\\
12.4	0.01\\
12.41	0.01\\
12.42	0.01\\
12.43	0.01\\
12.44	0.01\\
12.45	0.01\\
12.46	0.01\\
12.47	0.01\\
12.48	0.01\\
12.49	0.01\\
12.5	0.01\\
12.51	0.01\\
12.52	0.01\\
12.53	0.01\\
12.54	0.01\\
12.55	0.01\\
12.56	0.01\\
12.57	0.01\\
12.58	0.01\\
12.59	0.01\\
12.6	0.01\\
12.61	0.01\\
12.62	0.01\\
12.63	0.01\\
12.64	0.01\\
12.65	0.01\\
12.66	0.01\\
12.67	0.01\\
12.68	0.01\\
12.69	0.01\\
12.7	0.01\\
12.71	0.01\\
12.72	0.01\\
12.73	0.01\\
12.74	0.01\\
12.75	0.01\\
12.76	0.01\\
12.77	0.01\\
12.78	0.01\\
12.79	0.01\\
12.8	0.01\\
12.81	0.01\\
12.82	0.01\\
12.83	0.01\\
12.84	0.01\\
12.85	0.01\\
12.86	0.01\\
12.87	0.01\\
12.88	0.01\\
12.89	0.01\\
12.9	0.01\\
12.91	0.01\\
12.92	0.01\\
12.93	0.01\\
12.94	0.01\\
12.95	0.01\\
12.96	0.01\\
12.97	0.01\\
12.98	0.01\\
12.99	0.01\\
13	0.01\\
13.01	0.01\\
13.02	0.01\\
13.03	0.01\\
13.04	0.01\\
13.05	0.01\\
13.06	0.01\\
13.07	0.01\\
13.08	0.01\\
13.09	0.01\\
13.1	0.01\\
13.11	0.01\\
13.12	0.01\\
13.13	0.01\\
13.14	0.01\\
13.15	0.01\\
13.16	0.01\\
13.17	0.01\\
13.18	0.01\\
13.19	0.01\\
13.2	0.01\\
13.21	0.01\\
13.22	0.01\\
13.23	0.01\\
13.24	0.01\\
13.25	0.01\\
13.26	0.01\\
13.27	0.01\\
13.28	0.01\\
13.29	0.01\\
13.3	0.01\\
13.31	0.01\\
13.32	0.01\\
13.33	0.01\\
13.34	0.01\\
13.35	0.01\\
13.36	0.01\\
13.37	0.01\\
13.38	0.01\\
13.39	0.01\\
13.4	0.01\\
13.41	0.01\\
13.42	0.01\\
13.43	0.01\\
13.44	0.01\\
13.45	0.01\\
13.46	0.01\\
13.47	0.01\\
13.48	0.01\\
13.49	0.01\\
13.5	0.01\\
13.51	0.01\\
13.52	0.01\\
13.53	0.01\\
13.54	0.01\\
13.55	0.01\\
13.56	0.01\\
13.57	0.01\\
13.58	0.01\\
13.59	0.01\\
13.6	0.01\\
13.61	0.01\\
13.62	0.01\\
13.63	0.01\\
13.64	0.01\\
13.65	0.01\\
13.66	0.01\\
13.67	0.01\\
13.68	0.01\\
13.69	0.01\\
13.7	0.01\\
13.71	0.01\\
13.72	0.01\\
13.73	0.01\\
13.74	0.01\\
13.75	0.01\\
13.76	0.01\\
13.77	0.01\\
13.78	0.01\\
13.79	0.01\\
13.8	0.01\\
13.81	0.01\\
13.82	0.01\\
13.83	0.01\\
13.84	0.01\\
13.85	0.01\\
13.86	0.01\\
13.87	0.01\\
13.88	0.01\\
13.89	0.01\\
13.9	0.01\\
13.91	0.01\\
13.92	0.01\\
13.93	0.01\\
13.94	0.01\\
13.95	0.01\\
13.96	0.01\\
13.97	0.01\\
13.98	0.01\\
13.99	0.01\\
14	0.01\\
14.01	0.01\\
14.02	0.01\\
14.03	0.01\\
14.04	0.01\\
14.05	0.01\\
14.06	0.01\\
14.07	0.01\\
14.08	0.01\\
14.09	0.01\\
14.1	0.01\\
14.11	0.01\\
14.12	0.01\\
14.13	0.01\\
14.14	0.01\\
14.15	0.01\\
14.16	0.01\\
14.17	0.01\\
14.18	0.01\\
14.19	0.01\\
14.2	0.01\\
14.21	0.01\\
14.22	0.01\\
14.23	0.01\\
14.24	0.01\\
14.25	0.01\\
14.26	0.01\\
14.27	0.01\\
14.28	0.01\\
14.29	0.01\\
14.3	0.01\\
14.31	0.01\\
14.32	0.01\\
14.33	0.01\\
14.34	0.01\\
14.35	0.01\\
14.36	0.01\\
14.37	0.01\\
14.38	0.01\\
14.39	0.01\\
14.4	0.01\\
14.41	0.01\\
14.42	0.01\\
14.43	0.01\\
14.44	0.01\\
14.45	0.01\\
14.46	0.01\\
14.47	0.01\\
14.48	0.01\\
14.49	0.01\\
14.5	0.01\\
14.51	0.01\\
14.52	0.01\\
14.53	0.01\\
14.54	0.01\\
14.55	0.01\\
14.56	0.01\\
14.57	0.01\\
14.58	0.01\\
14.59	0.01\\
14.6	0.01\\
14.61	0.01\\
14.62	0.01\\
14.63	0.01\\
14.64	0.01\\
14.65	0.01\\
14.66	0.01\\
14.67	0.01\\
14.68	0.01\\
14.69	0.01\\
14.7	0.01\\
14.71	0.01\\
14.72	0.01\\
14.73	0.01\\
14.74	0.01\\
14.75	0.01\\
14.76	0.01\\
14.77	0.01\\
14.78	0.01\\
14.79	0.01\\
14.8	0.01\\
14.81	0.01\\
14.82	0.01\\
14.83	0.01\\
14.84	0.01\\
14.85	0.01\\
14.86	0.01\\
14.87	0.01\\
14.88	0.01\\
14.89	0.01\\
14.9	0.01\\
14.91	0.01\\
14.92	0.01\\
14.93	0.01\\
14.94	0.01\\
14.95	0.01\\
14.96	0.01\\
14.97	0.01\\
14.98	0.01\\
14.99	0.01\\
15	0.01\\
15.01	0.01\\
15.02	0.01\\
15.03	0.01\\
15.04	0.01\\
15.05	0.01\\
15.06	0.01\\
15.07	0.01\\
15.08	0.01\\
15.09	0.01\\
15.1	0.01\\
15.11	0.01\\
15.12	0.01\\
15.13	0.01\\
15.14	0.01\\
15.15	0.01\\
15.16	0.01\\
15.17	0.01\\
15.18	0.01\\
15.19	0.01\\
15.2	0.01\\
15.21	0.01\\
15.22	0.01\\
15.23	0.01\\
15.24	0.01\\
15.25	0.01\\
15.26	0.01\\
15.27	0.01\\
15.28	0.01\\
15.29	0.01\\
15.3	0.01\\
15.31	0.01\\
15.32	0.01\\
15.33	0.01\\
15.34	0.01\\
15.35	0.01\\
15.36	0.01\\
15.37	0.01\\
15.38	0.01\\
15.39	0.01\\
15.4	0.01\\
15.41	0.01\\
15.42	0.01\\
15.43	0.01\\
15.44	0.01\\
15.45	0.01\\
15.46	0.01\\
15.47	0.01\\
15.48	0.01\\
15.49	0.01\\
15.5	0.01\\
15.51	0.01\\
15.52	0.01\\
15.53	0.01\\
15.54	0.01\\
15.55	0.01\\
15.56	0.01\\
15.57	0.01\\
15.58	0.01\\
15.59	0.01\\
15.6	0.01\\
15.61	0.01\\
15.62	0.01\\
15.63	0.01\\
15.64	0.01\\
15.65	0.01\\
15.66	0.01\\
15.67	0.01\\
15.68	0.01\\
15.69	0.01\\
15.7	0.01\\
15.71	0.01\\
15.72	0.01\\
15.73	0.01\\
15.74	0.01\\
15.75	0.01\\
15.76	0.01\\
15.77	0.01\\
15.78	0.01\\
15.79	0.01\\
15.8	0.01\\
15.81	0.01\\
15.82	0.01\\
15.83	0.01\\
15.84	0.01\\
15.85	0.01\\
15.86	0.01\\
15.87	0.01\\
15.88	0.01\\
15.89	0.01\\
15.9	0.01\\
15.91	0.01\\
15.92	0.01\\
15.93	0.01\\
15.94	0.01\\
15.95	0.01\\
15.96	0.01\\
15.97	0.01\\
15.98	0.01\\
15.99	0.01\\
16	0.01\\
16.01	0.01\\
16.02	0.01\\
16.03	0.01\\
16.04	0.01\\
16.05	0.01\\
16.06	0.01\\
16.07	0.01\\
16.08	0.01\\
16.09	0.01\\
16.1	0.01\\
16.11	0.01\\
16.12	0.01\\
16.13	0.01\\
16.14	0.01\\
16.15	0.01\\
16.16	0.01\\
16.17	0.01\\
16.18	0.01\\
16.19	0.01\\
16.2	0.01\\
16.21	0.01\\
16.22	0.01\\
16.23	0.01\\
16.24	0.01\\
16.25	0.01\\
16.26	0.01\\
16.27	0.01\\
16.28	0.01\\
16.29	0.01\\
16.3	0.01\\
16.31	0.01\\
16.32	0.01\\
16.33	0.01\\
16.34	0.01\\
16.35	0.01\\
16.36	0.01\\
16.37	0.01\\
16.38	0.01\\
16.39	0.01\\
16.4	0.01\\
16.41	0.01\\
16.42	0.01\\
16.43	0.01\\
16.44	0.01\\
16.45	0.01\\
16.46	0.01\\
16.47	0.01\\
16.48	0.01\\
16.49	0.01\\
16.5	0.01\\
16.51	0.01\\
16.52	0.01\\
16.53	0.01\\
16.54	0.01\\
16.55	0.01\\
16.56	0.01\\
16.57	0.01\\
16.58	0.01\\
16.59	0.01\\
16.6	0.01\\
16.61	0.01\\
16.62	0.01\\
16.63	0.01\\
16.64	0.01\\
16.65	0.01\\
16.66	0.01\\
16.67	0.01\\
16.68	0.01\\
16.69	0.01\\
16.7	0.01\\
16.71	0.01\\
16.72	0.01\\
16.73	0.01\\
16.74	0.01\\
16.75	0.01\\
16.76	0.01\\
16.77	0.01\\
16.78	0.01\\
16.79	0.01\\
16.8	0.01\\
16.81	0.01\\
16.82	0.01\\
16.83	0.01\\
16.84	0.01\\
16.85	0.01\\
16.86	0.01\\
16.87	0.01\\
16.88	0.01\\
16.89	0.01\\
16.9	0.01\\
16.91	0.01\\
16.92	0.01\\
16.93	0.01\\
16.94	0.01\\
16.95	0.01\\
16.96	0.01\\
16.97	0.01\\
16.98	0.01\\
16.99	0.01\\
17	0.01\\
17.01	0.01\\
17.02	0.01\\
17.03	0.01\\
17.04	0.01\\
17.05	0.01\\
17.06	0.01\\
17.07	0.01\\
17.08	0.01\\
17.09	0.01\\
17.1	0.01\\
17.11	0.01\\
17.12	0.01\\
17.13	0.01\\
17.14	0.01\\
17.15	0.01\\
17.16	0.01\\
17.17	0.01\\
17.18	0.01\\
17.19	0.01\\
17.2	0.01\\
17.21	0.01\\
17.22	0.01\\
17.23	0.01\\
17.24	0.01\\
17.25	0.01\\
17.26	0.01\\
17.27	0.01\\
17.28	0.01\\
17.29	0.01\\
17.3	0.01\\
17.31	0.01\\
17.32	0.01\\
17.33	0.01\\
17.34	0.01\\
17.35	0.01\\
17.36	0.01\\
17.37	0.01\\
17.38	0.01\\
17.39	0.01\\
17.4	0.01\\
17.41	0.01\\
17.42	0.01\\
17.43	0.01\\
17.44	0.01\\
17.45	0.01\\
17.46	0.01\\
17.47	0.01\\
17.48	0.01\\
17.49	0.01\\
17.5	0.01\\
17.51	0.01\\
17.52	0.01\\
17.53	0.01\\
17.54	0.01\\
17.55	0.01\\
17.56	0.01\\
17.57	0.01\\
17.58	0.01\\
17.59	0.01\\
17.6	0.01\\
17.61	0.01\\
17.62	0.01\\
17.63	0.01\\
17.64	0.01\\
17.65	0.01\\
17.66	0.01\\
17.67	0.01\\
17.68	0.01\\
17.69	0.01\\
17.7	0.01\\
17.71	0.01\\
17.72	0.01\\
17.73	0.01\\
17.74	0.01\\
17.75	0.01\\
17.76	0.01\\
17.77	0.01\\
17.78	0.01\\
17.79	0.01\\
17.8	0.01\\
17.81	0.01\\
17.82	0.01\\
17.83	0.01\\
17.84	0.01\\
17.85	0.01\\
17.86	0.01\\
17.87	0.01\\
17.88	0.01\\
17.89	0.01\\
17.9	0.01\\
17.91	0.01\\
17.92	0.01\\
17.93	0.01\\
17.94	0.01\\
17.95	0.01\\
17.96	0.01\\
17.97	0.01\\
17.98	0.01\\
17.99	0.01\\
18	0.01\\
18.01	0.01\\
18.02	0.01\\
18.03	0.01\\
18.04	0.01\\
18.05	0.01\\
18.06	0.01\\
18.07	0.01\\
18.08	0.01\\
18.09	0.01\\
18.1	0.01\\
18.11	0.01\\
18.12	0.01\\
18.13	0.01\\
18.14	0.01\\
18.15	0.01\\
18.16	0.01\\
18.17	0.01\\
18.18	0.01\\
18.19	0.01\\
18.2	0.01\\
18.21	0.01\\
18.22	0.01\\
18.23	0.01\\
18.24	0.01\\
18.25	0.01\\
18.26	0.01\\
18.27	0.01\\
18.28	0.01\\
18.29	0.01\\
18.3	0.01\\
18.31	0.01\\
18.32	0.01\\
18.33	0.01\\
18.34	0.01\\
18.35	0.01\\
18.36	0.01\\
18.37	0.01\\
18.38	0.01\\
18.39	0.01\\
18.4	0.01\\
18.41	0.01\\
18.42	0.01\\
18.43	0.01\\
18.44	0.01\\
18.45	0.01\\
18.46	0.01\\
18.47	0.01\\
18.48	0.01\\
18.49	0.01\\
18.5	0.01\\
18.51	0.01\\
18.52	0.01\\
18.53	0.01\\
18.54	0.01\\
18.55	0.01\\
18.56	0.01\\
18.57	0.01\\
18.58	0.01\\
18.59	0.01\\
18.6	0.01\\
18.61	0.01\\
18.62	0.01\\
18.63	0.01\\
18.64	0.01\\
18.65	0.01\\
18.66	0.01\\
18.67	0.01\\
18.68	0.01\\
18.69	0.01\\
18.7	0.01\\
18.71	0.01\\
18.72	0.01\\
18.73	0.01\\
18.74	0.01\\
18.75	0.01\\
18.76	0.01\\
18.77	0.01\\
18.78	0.01\\
18.79	0.01\\
18.8	0.01\\
18.81	0.01\\
18.82	0.01\\
18.83	0.01\\
18.84	0.01\\
18.85	0.01\\
18.86	0.01\\
18.87	0.01\\
18.88	0.01\\
18.89	0.01\\
18.9	0.01\\
18.91	0.01\\
18.92	0.01\\
18.93	0.01\\
18.94	0.01\\
18.95	0.01\\
18.96	0.01\\
18.97	0.01\\
18.98	0.01\\
18.99	0.01\\
19	0.01\\
19.01	0.01\\
19.02	0.01\\
19.03	0.01\\
19.04	0.01\\
19.05	0.01\\
19.06	0.01\\
19.07	0.01\\
19.08	0.01\\
19.09	0.01\\
19.1	0.01\\
19.11	0.01\\
19.12	0.01\\
19.13	0.01\\
19.14	0.01\\
19.15	0.01\\
19.16	0.01\\
19.17	0.01\\
19.18	0.01\\
19.19	0.01\\
19.2	0.01\\
19.21	0.01\\
19.22	0.01\\
19.23	0.01\\
19.24	0.01\\
19.25	0.01\\
19.26	0.01\\
19.27	0.01\\
19.28	0.01\\
19.29	0.01\\
19.3	0.01\\
19.31	0.01\\
19.32	0.01\\
19.33	0.01\\
19.34	0.01\\
19.35	0.01\\
19.36	0.01\\
19.37	0.01\\
19.38	0.01\\
19.39	0.01\\
19.4	0.01\\
19.41	0.01\\
19.42	0.01\\
19.43	0.01\\
19.44	0.01\\
19.45	0.01\\
19.46	0.01\\
19.47	0.01\\
19.48	0.01\\
19.49	0.01\\
19.5	0.01\\
19.51	0.01\\
19.52	0.01\\
19.53	0.01\\
19.54	0.01\\
19.55	0.01\\
19.56	0.01\\
19.57	0.01\\
19.58	0.01\\
19.59	0.01\\
19.6	0.01\\
19.61	0.01\\
19.62	0.01\\
19.63	0.01\\
19.64	0.01\\
19.65	0.01\\
19.66	0.01\\
19.67	0.01\\
19.68	0.01\\
19.69	0.01\\
19.7	0.01\\
19.71	0.01\\
19.72	0.01\\
19.73	0.01\\
19.74	0.01\\
19.75	0.01\\
19.76	0.01\\
19.77	0.01\\
19.78	0.01\\
19.79	0.01\\
19.8	0.01\\
19.81	0.01\\
19.82	0.01\\
19.83	0.01\\
19.84	0.01\\
19.85	0.01\\
19.86	0.01\\
19.87	0.01\\
19.88	0.01\\
19.89	0.01\\
19.9	0.01\\
19.91	0.01\\
19.92	0.01\\
19.93	0.01\\
19.94	0.01\\
19.95	0.01\\
19.96	0.01\\
19.97	0.01\\
19.98	0.01\\
19.99	0.01\\
20	0.01\\
20.01	0.01\\
20.02	0.01\\
20.03	0.01\\
20.04	0.01\\
20.05	0.01\\
20.06	0.01\\
20.07	0.01\\
20.08	0.01\\
20.09	0.01\\
20.1	0.01\\
20.11	0.01\\
20.12	0.01\\
20.13	0.01\\
20.14	0.01\\
20.15	0.01\\
20.16	0.01\\
20.17	0.01\\
20.18	0.01\\
20.19	0.01\\
20.2	0.01\\
20.21	0.01\\
20.22	0.01\\
20.23	0.01\\
20.24	0.01\\
20.25	0.01\\
20.26	0.01\\
20.27	0.01\\
20.28	0.01\\
20.29	0.01\\
20.3	0.01\\
20.31	0.01\\
20.32	0.01\\
20.33	0.01\\
20.34	0.01\\
20.35	0.01\\
20.36	0.01\\
20.37	0.01\\
20.38	0.01\\
20.39	0.01\\
20.4	0.01\\
20.41	0.01\\
20.42	0.01\\
20.43	0.01\\
20.44	0.01\\
20.45	0.01\\
20.46	0.01\\
20.47	0.01\\
20.48	0.01\\
20.49	0.01\\
20.5	0.01\\
20.51	0.01\\
20.52	0.01\\
20.53	0.01\\
20.54	0.01\\
20.55	0.01\\
20.56	0.01\\
20.57	0.01\\
20.58	0.01\\
20.59	0.01\\
20.6	0.01\\
20.61	0.01\\
20.62	0.01\\
20.63	0.01\\
20.64	0.01\\
20.65	0.01\\
20.66	0.01\\
20.67	0.01\\
20.68	0.01\\
20.69	0.01\\
20.7	0.01\\
20.71	0.01\\
20.72	0.01\\
20.73	0.01\\
20.74	0.01\\
20.75	0.01\\
20.76	0.01\\
20.77	0.01\\
20.78	0.01\\
20.79	0.01\\
20.8	0.01\\
20.81	0.01\\
20.82	0.01\\
20.83	0.01\\
20.84	0.01\\
20.85	0.01\\
20.86	0.01\\
20.87	0.01\\
20.88	0.01\\
20.89	0.01\\
20.9	0.01\\
20.91	0.01\\
20.92	0.01\\
20.93	0.01\\
20.94	0.01\\
20.95	0.01\\
20.96	0.01\\
20.97	0.01\\
20.98	0.01\\
20.99	0.01\\
21	0.01\\
21.01	0.01\\
21.02	0.01\\
21.03	0.01\\
21.04	0.01\\
21.05	0.01\\
21.06	0.01\\
21.07	0.01\\
21.08	0.01\\
21.09	0.01\\
21.1	0.01\\
21.11	0.01\\
21.12	0.01\\
21.13	0.01\\
21.14	0.01\\
21.15	0.01\\
21.16	0.01\\
21.17	0.01\\
21.18	0.01\\
21.19	0.01\\
21.2	0.01\\
21.21	0.01\\
21.22	0.01\\
21.23	0.01\\
21.24	0.01\\
21.25	0.01\\
21.26	0.01\\
21.27	0.01\\
21.28	0.01\\
21.29	0.01\\
21.3	0.01\\
21.31	0.01\\
21.32	0.01\\
21.33	0.01\\
21.34	0.01\\
21.35	0.01\\
21.36	0.01\\
21.37	0.01\\
21.38	0.01\\
21.39	0.01\\
21.4	0.01\\
21.41	0.01\\
21.42	0.01\\
21.43	0.01\\
21.44	0.01\\
21.45	0.01\\
21.46	0.01\\
21.47	0.01\\
21.48	0.01\\
21.49	0.01\\
21.5	0.01\\
21.51	0.01\\
21.52	0.01\\
21.53	0.01\\
21.54	0.01\\
21.55	0.01\\
21.56	0.01\\
21.57	0.01\\
21.58	0.01\\
21.59	0.01\\
21.6	0.01\\
21.61	0.01\\
21.62	0.01\\
21.63	0.01\\
21.64	0.01\\
21.65	0.01\\
21.66	0.01\\
21.67	0.01\\
21.68	0.01\\
21.69	0.01\\
21.7	0.01\\
21.71	0.01\\
21.72	0.01\\
21.73	0.01\\
21.74	0.01\\
21.75	0.01\\
21.76	0.01\\
21.77	0.01\\
21.78	0.01\\
21.79	0.01\\
21.8	0.01\\
21.81	0.01\\
21.82	0.01\\
21.83	0.01\\
21.84	0.01\\
21.85	0.01\\
21.86	0.01\\
21.87	0.01\\
21.88	0.01\\
21.89	0.01\\
21.9	0.01\\
21.91	0.01\\
21.92	0.01\\
21.93	0.01\\
21.94	0.01\\
21.95	0.01\\
21.96	0.01\\
21.97	0.01\\
21.98	0.01\\
21.99	0.01\\
22	0.01\\
22.01	0.01\\
22.02	0.01\\
22.03	0.01\\
22.04	0.01\\
22.05	0.01\\
22.06	0.01\\
22.07	0.01\\
22.08	0.01\\
22.09	0.01\\
22.1	0.01\\
22.11	0.01\\
22.12	0.01\\
22.13	0.01\\
22.14	0.01\\
22.15	0.01\\
22.16	0.01\\
22.17	0.01\\
22.18	0.01\\
22.19	0.01\\
22.2	0.01\\
22.21	0.01\\
22.22	0.01\\
22.23	0.01\\
22.24	0.01\\
22.25	0.01\\
22.26	0.01\\
22.27	0.01\\
22.28	0.01\\
22.29	0.01\\
22.3	0.01\\
22.31	0.01\\
22.32	0.01\\
22.33	0.01\\
22.34	0.01\\
22.35	0.01\\
22.36	0.01\\
22.37	0.01\\
22.38	0.01\\
22.39	0.01\\
22.4	0.01\\
22.41	0.01\\
22.42	0.01\\
22.43	0.01\\
22.44	0.01\\
22.45	0.01\\
22.46	0.01\\
22.47	0.01\\
22.48	0.01\\
22.49	0.01\\
22.5	0.01\\
22.51	0.01\\
22.52	0.01\\
22.53	0.01\\
22.54	0.01\\
22.55	0.01\\
22.56	0.01\\
22.57	0.01\\
22.58	0.01\\
22.59	0.01\\
22.6	0.01\\
22.61	0.01\\
22.62	0.01\\
22.63	0.01\\
22.64	0.01\\
22.65	0.01\\
22.66	0.01\\
22.67	0.01\\
22.68	0.01\\
22.69	0.01\\
22.7	0.01\\
22.71	0.01\\
22.72	0.01\\
22.73	0.01\\
22.74	0.01\\
22.75	0.01\\
22.76	0.01\\
22.77	0.01\\
22.78	0.01\\
22.79	0.01\\
22.8	0.01\\
22.81	0.01\\
22.82	0.01\\
22.83	0.01\\
22.84	0.01\\
22.85	0.01\\
22.86	0.01\\
22.87	0.01\\
22.88	0.01\\
22.89	0.01\\
22.9	0.01\\
22.91	0.01\\
22.92	0.01\\
22.93	0.01\\
22.94	0.01\\
22.95	0.01\\
22.96	0.01\\
22.97	0.01\\
22.98	0.01\\
22.99	0.01\\
23	0.01\\
23.01	0.01\\
23.02	0.01\\
23.03	0.01\\
23.04	0.01\\
23.05	0.01\\
23.06	0.01\\
23.07	0.01\\
23.08	0.01\\
23.09	0.01\\
23.1	0.01\\
23.11	0.01\\
23.12	0.01\\
23.13	0.01\\
23.14	0.01\\
23.15	0.01\\
23.16	0.01\\
23.17	0.01\\
23.18	0.01\\
23.19	0.01\\
23.2	0.01\\
23.21	0.01\\
23.22	0.01\\
23.23	0.01\\
23.24	0.01\\
23.25	0.01\\
23.26	0.01\\
23.27	0.01\\
23.28	0.01\\
23.29	0.01\\
23.3	0.01\\
23.31	0.01\\
23.32	0.01\\
23.33	0.01\\
23.34	0.01\\
23.35	0.01\\
23.36	0.01\\
23.37	0.01\\
23.38	0.01\\
23.39	0.01\\
23.4	0.01\\
23.41	0.01\\
23.42	0.01\\
23.43	0.01\\
23.44	0.01\\
23.45	0.01\\
23.46	0.01\\
23.47	0.01\\
23.48	0.01\\
23.49	0.01\\
23.5	0.01\\
23.51	0.01\\
23.52	0.01\\
23.53	0.01\\
23.54	0.01\\
23.55	0.01\\
23.56	0.01\\
23.57	0.01\\
23.58	0.01\\
23.59	0.01\\
23.6	0.01\\
23.61	0.01\\
23.62	0.01\\
23.63	0.01\\
23.64	0.01\\
23.65	0.01\\
23.66	0.01\\
23.67	0.01\\
23.68	0.01\\
23.69	0.01\\
23.7	0.01\\
23.71	0.01\\
23.72	0.01\\
23.73	0.01\\
23.74	0.01\\
23.75	0.01\\
23.76	0.01\\
23.77	0.01\\
23.78	0.01\\
23.79	0.01\\
23.8	0.01\\
23.81	0.01\\
23.82	0.01\\
23.83	0.01\\
23.84	0.01\\
23.85	0.01\\
23.86	0.01\\
23.87	0.01\\
23.88	0.01\\
23.89	0.01\\
23.9	0.01\\
23.91	0.01\\
23.92	0.01\\
23.93	0.01\\
23.94	0.01\\
23.95	0.01\\
23.96	0.01\\
23.97	0.01\\
23.98	0.01\\
23.99	0.01\\
24	0.01\\
24.01	0.01\\
24.02	0.01\\
24.03	0.01\\
24.04	0.01\\
24.05	0.01\\
24.06	0.01\\
24.07	0.01\\
24.08	0.01\\
24.09	0.01\\
24.1	0.01\\
24.11	0.01\\
24.12	0.01\\
24.13	0.01\\
24.14	0.01\\
24.15	0.01\\
24.16	0.01\\
24.17	0.01\\
24.18	0.01\\
24.19	0.01\\
24.2	0.01\\
24.21	0.01\\
24.22	0.01\\
24.23	0.01\\
24.24	0.01\\
24.25	0.01\\
24.26	0.01\\
24.27	0.01\\
24.28	0.01\\
24.29	0.01\\
24.3	0.01\\
24.31	0.01\\
24.32	0.01\\
24.33	0.01\\
24.34	0.01\\
24.35	0.01\\
24.36	0.01\\
24.37	0.01\\
24.38	0.01\\
24.39	0.01\\
24.4	0.01\\
24.41	0.01\\
24.42	0.01\\
24.43	0.01\\
24.44	0.01\\
24.45	0.01\\
24.46	0.01\\
24.47	0.01\\
24.48	0.01\\
24.49	0.01\\
24.5	0.01\\
24.51	0.01\\
24.52	0.01\\
24.53	0.01\\
24.54	0.01\\
24.55	0.01\\
24.56	0.01\\
24.57	0.01\\
24.58	0.01\\
24.59	0.01\\
24.6	0.01\\
24.61	0.01\\
24.62	0.01\\
24.63	0.01\\
24.64	0.01\\
24.65	0.01\\
24.66	0.01\\
24.67	0.01\\
24.68	0.01\\
24.69	0.01\\
24.7	0.01\\
24.71	0.01\\
24.72	0.01\\
24.73	0.01\\
24.74	0.01\\
24.75	0.01\\
24.76	0.01\\
24.77	0.01\\
24.78	0.01\\
24.79	0.01\\
24.8	0.01\\
24.81	0.01\\
24.82	0.01\\
24.83	0.01\\
24.84	0.01\\
24.85	0.01\\
24.86	0.01\\
24.87	0.01\\
24.88	0.01\\
24.89	0.01\\
24.9	0.01\\
24.91	0.01\\
24.92	0.01\\
24.93	0.01\\
24.94	0.01\\
24.95	0.01\\
24.96	0.01\\
24.97	0.01\\
24.98	0.01\\
24.99	0.01\\
25	0.01\\
25.01	0.01\\
25.02	0.01\\
25.03	0.01\\
25.04	0.01\\
25.05	0.01\\
25.06	0.01\\
25.07	0.01\\
25.08	0.01\\
25.09	0.01\\
25.1	0.01\\
25.11	0.01\\
25.12	0.01\\
25.13	0.01\\
25.14	0.01\\
25.15	0.01\\
25.16	0.01\\
25.17	0.01\\
25.18	0.01\\
25.19	0.01\\
25.2	0.01\\
25.21	0.01\\
25.22	0.01\\
25.23	0.01\\
25.24	0.01\\
25.25	0.01\\
25.26	0.01\\
25.27	0.01\\
25.28	0.01\\
25.29	0.01\\
25.3	0.01\\
25.31	0.01\\
25.32	0.01\\
25.33	0.01\\
25.34	0.01\\
25.35	0.01\\
25.36	0.01\\
25.37	0.01\\
25.38	0.01\\
25.39	0.01\\
25.4	0.01\\
25.41	0.01\\
25.42	0.01\\
25.43	0.01\\
25.44	0.01\\
25.45	0.01\\
25.46	0.01\\
25.47	0.01\\
25.48	0.01\\
25.49	0.01\\
25.5	0.01\\
25.51	0.01\\
25.52	0.01\\
25.53	0.01\\
25.54	0.01\\
25.55	0.01\\
25.56	0.01\\
25.57	0.01\\
25.58	0.01\\
25.59	0.01\\
25.6	0.01\\
25.61	0.01\\
25.62	0.01\\
25.63	0.01\\
25.64	0.01\\
25.65	0.01\\
25.66	0.01\\
25.67	0.01\\
25.68	0.01\\
25.69	0.01\\
25.7	0.01\\
25.71	0.01\\
25.72	0.01\\
25.73	0.01\\
25.74	0.01\\
25.75	0.01\\
25.76	0.01\\
25.77	0.01\\
25.78	0.01\\
25.79	0.01\\
25.8	0.01\\
25.81	0.01\\
25.82	0.01\\
25.83	0.01\\
25.84	0.01\\
25.85	0.01\\
25.86	0.01\\
25.87	0.01\\
25.88	0.01\\
25.89	0.01\\
25.9	0.01\\
25.91	0.01\\
25.92	0.01\\
25.93	0.01\\
25.94	0.01\\
25.95	0.01\\
25.96	0.01\\
25.97	0.01\\
25.98	0.01\\
25.99	0.01\\
26	0.01\\
26.01	0.01\\
26.02	0.01\\
26.03	0.01\\
26.04	0.01\\
26.05	0.01\\
26.06	0.01\\
26.07	0.01\\
26.08	0.01\\
26.09	0.01\\
26.1	0.01\\
26.11	0.01\\
26.12	0.01\\
26.13	0.01\\
26.14	0.01\\
26.15	0.01\\
26.16	0.01\\
26.17	0.01\\
26.18	0.01\\
26.19	0.01\\
26.2	0.01\\
26.21	0.01\\
26.22	0.01\\
26.23	0.01\\
26.24	0.01\\
26.25	0.01\\
26.26	0.01\\
26.27	0.01\\
26.28	0.01\\
26.29	0.01\\
26.3	0.01\\
26.31	0.01\\
26.32	0.01\\
26.33	0.01\\
26.34	0.01\\
26.35	0.01\\
26.36	0.01\\
26.37	0.01\\
26.38	0.01\\
26.39	0.01\\
26.4	0.01\\
26.41	0.01\\
26.42	0.01\\
26.43	0.01\\
26.44	0.01\\
26.45	0.01\\
26.46	0.01\\
26.47	0.01\\
26.48	0.01\\
26.49	0.01\\
26.5	0.01\\
26.51	0.01\\
26.52	0.01\\
26.53	0.01\\
26.54	0.01\\
26.55	0.01\\
26.56	0.01\\
26.57	0.01\\
26.58	0.01\\
26.59	0.01\\
26.6	0.01\\
26.61	0.01\\
26.62	0.01\\
26.63	0.01\\
26.64	0.01\\
26.65	0.01\\
26.66	0.01\\
26.67	0.01\\
26.68	0.01\\
26.69	0.01\\
26.7	0.01\\
26.71	0.01\\
26.72	0.01\\
26.73	0.01\\
26.74	0.01\\
26.75	0.01\\
26.76	0.01\\
26.77	0.01\\
26.78	0.01\\
26.79	0.01\\
26.8	0.01\\
26.81	0.01\\
26.82	0.01\\
26.83	0.01\\
26.84	0.01\\
26.85	0.01\\
26.86	0.01\\
26.87	0.01\\
26.88	0.01\\
26.89	0.01\\
26.9	0.01\\
26.91	0.01\\
26.92	0.01\\
26.93	0.01\\
26.94	0.01\\
26.95	0.01\\
26.96	0.01\\
26.97	0.01\\
26.98	0.01\\
26.99	0.01\\
27	0.01\\
27.01	0.01\\
27.02	0.01\\
27.03	0.01\\
27.04	0.01\\
27.05	0.01\\
27.06	0.01\\
27.07	0.01\\
27.08	0.01\\
27.09	0.01\\
27.1	0.01\\
27.11	0.01\\
27.12	0.01\\
27.13	0.01\\
27.14	0.01\\
27.15	0.01\\
27.16	0.01\\
27.17	0.01\\
27.18	0.01\\
27.19	0.01\\
27.2	0.01\\
27.21	0.01\\
27.22	0.01\\
27.23	0.01\\
27.24	0.01\\
27.25	0.01\\
27.26	0.01\\
27.27	0.01\\
27.28	0.01\\
27.29	0.01\\
27.3	0.01\\
27.31	0.01\\
27.32	0.01\\
27.33	0.01\\
27.34	0.01\\
27.35	0.01\\
27.36	0.01\\
27.37	0.01\\
27.38	0.01\\
27.39	0.01\\
27.4	0.01\\
27.41	0.01\\
27.42	0.01\\
27.43	0.01\\
27.44	0.01\\
27.45	0.01\\
27.46	0.01\\
27.47	0.01\\
27.48	0.01\\
27.49	0.01\\
27.5	0.01\\
27.51	0.01\\
27.52	0.01\\
27.53	0.01\\
27.54	0.01\\
27.55	0.01\\
27.56	0.01\\
27.57	0.01\\
27.58	0.01\\
27.59	0.01\\
27.6	0.01\\
27.61	0.01\\
27.62	0.01\\
27.63	0.01\\
27.64	0.01\\
27.65	0.01\\
27.66	0.01\\
27.67	0.01\\
27.68	0.01\\
27.69	0.01\\
27.7	0.01\\
27.71	0.01\\
27.72	0.01\\
27.73	0.01\\
27.74	0.01\\
27.75	0.01\\
27.76	0.01\\
27.77	0.01\\
27.78	0.01\\
27.79	0.01\\
27.8	0.01\\
27.81	0.01\\
27.82	0.01\\
27.83	0.01\\
27.84	0.01\\
27.85	0.01\\
27.86	0.01\\
27.87	0.01\\
27.88	0.01\\
27.89	0.01\\
27.9	0.01\\
27.91	0.01\\
27.92	0.01\\
27.93	0.01\\
27.94	0.01\\
27.95	0.01\\
27.96	0.01\\
27.97	0.01\\
27.98	0.01\\
27.99	0.01\\
28	0.01\\
28.01	0.01\\
28.02	0.01\\
28.03	0.01\\
28.04	0.01\\
28.05	0.01\\
28.06	0.01\\
28.07	0.01\\
28.08	0.01\\
28.09	0.01\\
28.1	0.01\\
28.11	0.01\\
28.12	0.01\\
28.13	0.01\\
28.14	0.01\\
28.15	0.01\\
28.16	0.01\\
28.17	0.01\\
28.18	0.01\\
28.19	0.01\\
28.2	0.01\\
28.21	0.01\\
28.22	0.01\\
28.23	0.01\\
28.24	0.01\\
28.25	0.01\\
28.26	0.01\\
28.27	0.01\\
28.28	0.01\\
28.29	0.01\\
28.3	0.01\\
28.31	0.01\\
28.32	0.01\\
28.33	0.01\\
28.34	0.01\\
28.35	0.01\\
28.36	0.01\\
28.37	0.01\\
28.38	0.01\\
28.39	0.01\\
28.4	0.01\\
28.41	0.01\\
28.42	0.01\\
28.43	0.01\\
28.44	0.01\\
28.45	0.01\\
28.46	0.01\\
28.47	0.01\\
28.48	0.01\\
28.49	0.01\\
28.5	0.01\\
28.51	0.01\\
28.52	0.01\\
28.53	0.01\\
28.54	0.01\\
28.55	0.01\\
28.56	0.01\\
28.57	0.01\\
28.58	0.01\\
28.59	0.01\\
28.6	0.01\\
28.61	0.01\\
28.62	0.01\\
28.63	0.01\\
28.64	0.01\\
28.65	0.01\\
28.66	0.01\\
28.67	0.01\\
28.68	0.01\\
28.69	0.01\\
28.7	0.01\\
28.71	0.01\\
28.72	0.01\\
28.73	0.01\\
28.74	0.01\\
28.75	0.01\\
28.76	0.01\\
28.77	0.01\\
28.78	0.01\\
28.79	0.01\\
28.8	0.01\\
28.81	0.01\\
28.82	0.01\\
28.83	0.01\\
28.84	0.01\\
28.85	0.01\\
28.86	0.01\\
28.87	0.01\\
28.88	0.01\\
28.89	0.01\\
28.9	0.01\\
28.91	0.01\\
28.92	0.01\\
28.93	0.01\\
28.94	0.01\\
28.95	0.01\\
28.96	0.01\\
28.97	0.01\\
28.98	0.01\\
28.99	0.01\\
29	0.01\\
29.01	0.01\\
29.02	0.01\\
29.03	0.01\\
29.04	0.01\\
29.05	0.01\\
29.06	0.01\\
29.07	0.01\\
29.08	0.01\\
29.09	0.01\\
29.1	0.01\\
29.11	0.01\\
29.12	0.01\\
29.13	0.01\\
29.14	0.01\\
29.15	0.01\\
29.16	0.01\\
29.17	0.01\\
29.18	0.01\\
29.19	0.01\\
29.2	0.01\\
29.21	0.01\\
29.22	0.01\\
29.23	0.01\\
29.24	0.01\\
29.25	0.01\\
29.26	0.01\\
29.27	0.01\\
29.28	0.01\\
29.29	0.01\\
29.3	0.01\\
29.31	0.01\\
29.32	0.01\\
29.33	0.01\\
29.34	0.01\\
29.35	0.01\\
29.36	0.01\\
29.37	0.01\\
29.38	0.01\\
29.39	0.01\\
29.4	0.01\\
29.41	0.01\\
29.42	0.01\\
29.43	0.01\\
29.44	0.01\\
29.45	0.01\\
29.46	0.01\\
29.47	0.01\\
29.48	0.01\\
29.49	0.01\\
29.5	0.01\\
29.51	0.01\\
29.52	0.01\\
29.53	0.01\\
29.54	0.01\\
29.55	0.01\\
29.56	0.01\\
29.57	0.01\\
29.58	0.01\\
29.59	0.01\\
29.6	0.01\\
29.61	0.01\\
29.62	0.01\\
29.63	0.01\\
29.64	0.01\\
29.65	0.01\\
29.66	0.01\\
29.67	0.01\\
29.68	0.01\\
29.69	0.01\\
29.7	0.01\\
29.71	0.01\\
29.72	0.01\\
29.73	0.01\\
29.74	0.01\\
29.75	0.01\\
29.76	0.01\\
29.77	0.01\\
29.78	0.01\\
29.79	0.01\\
29.8	0.01\\
29.81	0.01\\
29.82	0.01\\
29.83	0.01\\
29.84	0.01\\
29.85	0.01\\
29.86	0.01\\
29.87	0.01\\
29.88	0.01\\
29.89	0.01\\
29.9	0.01\\
29.91	0.01\\
29.92	0.01\\
29.93	0.01\\
29.94	0.01\\
29.95	0.01\\
29.96	0.01\\
29.97	0.01\\
29.98	0.01\\
29.99	0.01\\
30	0.01\\
30.01	0.01\\
30.02	0.01\\
30.03	0.01\\
30.04	0.01\\
30.05	0.01\\
30.06	0.01\\
30.07	0.01\\
30.08	0.01\\
30.09	0.01\\
30.1	0.01\\
30.11	0.01\\
30.12	0.01\\
30.13	0.01\\
30.14	0.01\\
30.15	0.01\\
30.16	0.01\\
30.17	0.01\\
30.18	0.01\\
30.19	0.01\\
30.2	0.01\\
30.21	0.01\\
30.22	0.01\\
30.23	0.01\\
30.24	0.01\\
30.25	0.01\\
30.26	0.01\\
30.27	0.01\\
30.28	0.01\\
30.29	0.01\\
30.3	0.01\\
30.31	0.01\\
30.32	0.01\\
30.33	0.01\\
30.34	0.01\\
30.35	0.01\\
30.36	0.01\\
30.37	0.01\\
30.38	0.01\\
30.39	0.01\\
30.4	0.01\\
30.41	0.01\\
30.42	0.01\\
30.43	0.01\\
30.44	0.01\\
30.45	0.01\\
30.46	0.01\\
30.47	0.01\\
30.48	0.01\\
30.49	0.01\\
30.5	0.01\\
30.51	0.01\\
30.52	0.01\\
30.53	0.01\\
30.54	0.01\\
30.55	0.01\\
30.56	0.01\\
30.57	0.01\\
30.58	0.01\\
30.59	0.01\\
30.6	0.01\\
30.61	0.01\\
30.62	0.01\\
30.63	0.01\\
30.64	0.01\\
30.65	0.01\\
30.66	0.01\\
30.67	0.01\\
30.68	0.01\\
30.69	0.01\\
30.7	0.01\\
30.71	0.01\\
30.72	0.01\\
30.73	0.01\\
30.74	0.01\\
30.75	0.01\\
30.76	0.01\\
30.77	0.01\\
30.78	0.01\\
30.79	0.01\\
30.8	0.01\\
30.81	0.01\\
30.82	0.01\\
30.83	0.01\\
30.84	0.01\\
30.85	0.01\\
30.86	0.01\\
30.87	0.01\\
30.88	0.01\\
30.89	0.01\\
30.9	0.01\\
30.91	0.01\\
30.92	0.01\\
30.93	0.01\\
30.94	0.01\\
30.95	0.01\\
30.96	0.01\\
30.97	0.01\\
30.98	0.01\\
30.99	0.01\\
31	0.01\\
31.01	0.01\\
31.02	0.01\\
31.03	0.01\\
31.04	0.01\\
31.05	0.01\\
31.06	0.01\\
31.07	0.01\\
31.08	0.01\\
31.09	0.01\\
31.1	0.01\\
31.11	0.01\\
31.12	0.01\\
31.13	0.01\\
31.14	0.01\\
31.15	0.01\\
31.16	0.01\\
31.17	0.01\\
31.18	0.01\\
31.19	0.01\\
31.2	0.01\\
31.21	0.01\\
31.22	0.01\\
31.23	0.01\\
31.24	0.01\\
31.25	0.01\\
31.26	0.01\\
31.27	0.01\\
31.28	0.01\\
31.29	0.01\\
31.3	0.01\\
31.31	0.01\\
31.32	0.01\\
31.33	0.01\\
31.34	0.01\\
31.35	0.01\\
31.36	0.01\\
31.37	0.01\\
31.38	0.01\\
31.39	0.01\\
31.4	0.01\\
31.41	0.01\\
31.42	0.01\\
31.43	0.01\\
31.44	0.01\\
31.45	0.01\\
31.46	0.01\\
31.47	0.01\\
31.48	0.01\\
31.49	0.01\\
31.5	0.01\\
31.51	0.01\\
31.52	0.01\\
31.53	0.01\\
31.54	0.01\\
31.55	0.01\\
31.56	0.01\\
31.57	0.01\\
31.58	0.01\\
31.59	0.01\\
31.6	0.01\\
31.61	0.01\\
31.62	0.01\\
31.63	0.01\\
31.64	0.01\\
31.65	0.01\\
31.66	0.01\\
31.67	0.01\\
31.68	0.01\\
31.69	0.01\\
31.7	0.01\\
31.71	0.01\\
31.72	0.01\\
31.73	0.01\\
31.74	0.01\\
31.75	0.01\\
31.76	0.01\\
31.77	0.01\\
31.78	0.01\\
31.79	0.01\\
31.8	0.01\\
31.81	0.01\\
31.82	0.01\\
31.83	0.01\\
31.84	0.01\\
31.85	0.01\\
31.86	0.01\\
31.87	0.01\\
31.88	0.01\\
31.89	0.01\\
31.9	0.01\\
31.91	0.01\\
31.92	0.01\\
31.93	0.01\\
31.94	0.01\\
31.95	0.01\\
31.96	0.01\\
31.97	0.01\\
31.98	0.01\\
31.99	0.01\\
32	0.01\\
32.01	0.01\\
32.02	0.01\\
32.03	0.01\\
32.04	0.01\\
32.05	0.01\\
32.06	0.01\\
32.07	0.01\\
32.08	0.01\\
32.09	0.01\\
32.1	0.01\\
32.11	0.01\\
32.12	0.01\\
32.13	0.01\\
32.14	0.01\\
32.15	0.01\\
32.16	0.01\\
32.17	0.01\\
32.18	0.01\\
32.19	0.01\\
32.2	0.01\\
32.21	0.01\\
32.22	0.01\\
32.23	0.01\\
32.24	0.01\\
32.25	0.01\\
32.26	0.01\\
32.27	0.01\\
32.28	0.01\\
32.29	0.01\\
32.3	0.01\\
32.31	0.01\\
32.32	0.01\\
32.33	0.01\\
32.34	0.01\\
32.35	0.01\\
32.36	0.01\\
32.37	0.01\\
32.38	0.01\\
32.39	0.01\\
32.4	0.01\\
32.41	0.01\\
32.42	0.01\\
32.43	0.01\\
32.44	0.01\\
32.45	0.01\\
32.46	0.01\\
32.47	0.01\\
32.48	0.01\\
32.49	0.01\\
32.5	0.01\\
32.51	0.01\\
32.52	0.01\\
32.53	0.01\\
32.54	0.01\\
32.55	0.01\\
32.56	0.01\\
32.57	0.01\\
32.58	0.01\\
32.59	0.01\\
32.6	0.01\\
32.61	0.01\\
32.62	0.01\\
32.63	0.01\\
32.64	0.01\\
32.65	0.01\\
32.66	0.01\\
32.67	0.01\\
32.68	0.01\\
32.69	0.01\\
32.7	0.01\\
32.71	0.01\\
32.72	0.01\\
32.73	0.01\\
32.74	0.01\\
32.75	0.01\\
32.76	0.01\\
32.77	0.01\\
32.78	0.01\\
32.79	0.01\\
32.8	0.01\\
32.81	0.01\\
32.82	0.01\\
32.83	0.01\\
32.84	0.01\\
32.85	0.01\\
32.86	0.01\\
32.87	0.01\\
32.88	0.01\\
32.89	0.01\\
32.9	0.01\\
32.91	0.01\\
32.92	0.01\\
32.93	0.01\\
32.94	0.01\\
32.95	0.01\\
32.96	0.01\\
32.97	0.01\\
32.98	0.01\\
32.99	0.01\\
33	0.01\\
33.01	0.01\\
33.02	0.01\\
33.03	0.01\\
33.04	0.01\\
33.05	0.01\\
33.06	0.01\\
33.07	0.01\\
33.08	0.01\\
33.09	0.01\\
33.1	0.01\\
33.11	0.01\\
33.12	0.01\\
33.13	0.01\\
33.14	0.01\\
33.15	0.01\\
33.16	0.01\\
33.17	0.01\\
33.18	0.01\\
33.19	0.01\\
33.2	0.01\\
33.21	0.01\\
33.22	0.01\\
33.23	0.01\\
33.24	0.01\\
33.25	0.01\\
33.26	0.01\\
33.27	0.01\\
33.28	0.01\\
33.29	0.01\\
33.3	0.01\\
33.31	0.01\\
33.32	0.01\\
33.33	0.01\\
33.34	0.01\\
33.35	0.01\\
33.36	0.01\\
33.37	0.01\\
33.38	0.01\\
33.39	0.01\\
33.4	0.01\\
33.41	0.01\\
33.42	0.01\\
33.43	0.01\\
33.44	0.01\\
33.45	0.01\\
33.46	0.01\\
33.47	0.01\\
33.48	0.01\\
33.49	0.01\\
33.5	0.01\\
33.51	0.01\\
33.52	0.01\\
33.53	0.01\\
33.54	0.01\\
33.55	0.01\\
33.56	0.01\\
33.57	0.01\\
33.58	0.01\\
33.59	0.01\\
33.6	0.01\\
33.61	0.01\\
33.62	0.01\\
33.63	0.01\\
33.64	0.01\\
33.65	0.01\\
33.66	0.01\\
33.67	0.01\\
33.68	0.01\\
33.69	0.01\\
33.7	0.01\\
33.71	0.01\\
33.72	0.01\\
33.73	0.01\\
33.74	0.01\\
33.75	0.01\\
33.76	0.01\\
33.77	0.01\\
33.78	0.01\\
33.79	0.01\\
33.8	0.01\\
33.81	0.01\\
33.82	0.01\\
33.83	0.01\\
33.84	0.01\\
33.85	0.01\\
33.86	0.01\\
33.87	0.01\\
33.88	0.01\\
33.89	0.01\\
33.9	0.01\\
33.91	0.01\\
33.92	0.01\\
33.93	0.01\\
33.94	0.01\\
33.95	0.01\\
33.96	0.01\\
33.97	0.01\\
33.98	0.01\\
33.99	0.01\\
34	0.01\\
34.01	0.01\\
34.02	0.01\\
34.03	0.01\\
34.04	0.01\\
34.05	0.01\\
34.06	0.01\\
34.07	0.01\\
34.08	0.01\\
34.09	0.01\\
34.1	0.01\\
34.11	0.01\\
34.12	0.01\\
34.13	0.01\\
34.14	0.01\\
34.15	0.01\\
34.16	0.01\\
34.17	0.01\\
34.18	0.01\\
34.19	0.01\\
34.2	0.01\\
34.21	0.01\\
34.22	0.01\\
34.23	0.01\\
34.24	0.01\\
34.25	0.01\\
34.26	0.01\\
34.27	0.01\\
34.28	0.01\\
34.29	0.01\\
34.3	0.01\\
34.31	0.01\\
34.32	0.01\\
34.33	0.01\\
34.34	0.01\\
34.35	0.01\\
34.36	0.01\\
34.37	0.01\\
34.38	0.01\\
34.39	0.01\\
34.4	0.01\\
34.41	0.01\\
34.42	0.01\\
34.43	0.01\\
34.44	0.01\\
34.45	0.01\\
34.46	0.01\\
34.47	0.01\\
34.48	0.01\\
34.49	0.01\\
34.5	0.01\\
34.51	0.01\\
34.52	0.01\\
34.53	0.01\\
34.54	0.01\\
34.55	0.01\\
34.56	0.01\\
34.57	0.01\\
34.58	0.01\\
34.59	0.01\\
34.6	0.01\\
34.61	0.01\\
34.62	0.01\\
34.63	0.01\\
34.64	0.01\\
34.65	0.01\\
34.66	0.01\\
34.67	0.01\\
34.68	0.01\\
34.69	0.01\\
34.7	0.01\\
34.71	0.01\\
34.72	0.01\\
34.73	0.01\\
34.74	0.01\\
34.75	0.01\\
34.76	0.01\\
34.77	0.01\\
34.78	0.01\\
34.79	0.01\\
34.8	0.01\\
34.81	0.01\\
34.82	0.01\\
34.83	0.01\\
34.84	0.01\\
34.85	0.01\\
34.86	0.01\\
34.87	0.01\\
34.88	0.01\\
34.89	0.01\\
34.9	0.01\\
34.91	0.01\\
34.92	0.01\\
34.93	0.01\\
34.94	0.01\\
34.95	0.01\\
34.96	0.01\\
34.97	0.01\\
34.98	0.01\\
34.99	0.01\\
35	0.01\\
35.01	0.01\\
35.02	0.01\\
35.03	0.01\\
35.04	0.01\\
35.05	0.01\\
35.06	0.01\\
35.07	0.01\\
35.08	0.01\\
35.09	0.01\\
35.1	0.01\\
35.11	0.01\\
35.12	0.01\\
35.13	0.01\\
35.14	0.01\\
35.15	0.01\\
35.16	0.01\\
35.17	0.01\\
35.18	0.01\\
35.19	0.01\\
35.2	0.01\\
35.21	0.01\\
35.22	0.01\\
35.23	0.01\\
35.24	0.01\\
35.25	0.01\\
35.26	0.01\\
35.27	0.01\\
35.28	0.01\\
35.29	0.01\\
35.3	0.01\\
35.31	0.01\\
35.32	0.01\\
35.33	0.01\\
35.34	0.01\\
35.35	0.01\\
35.36	0.01\\
35.37	0.01\\
35.38	0.01\\
35.39	0.01\\
35.4	0.01\\
35.41	0.01\\
35.42	0.01\\
35.43	0.01\\
35.44	0.01\\
35.45	0.01\\
35.46	0.01\\
35.47	0.01\\
35.48	0.01\\
35.49	0.01\\
35.5	0.01\\
35.51	0.01\\
35.52	0.01\\
35.53	0.01\\
35.54	0.01\\
35.55	0.01\\
35.56	0.01\\
35.57	0.01\\
35.58	0.01\\
35.59	0.01\\
35.6	0.01\\
35.61	0.01\\
35.62	0.01\\
35.63	0.01\\
35.64	0.01\\
35.65	0.01\\
35.66	0.01\\
35.67	0.01\\
35.68	0.01\\
35.69	0.01\\
35.7	0.01\\
35.71	0.01\\
35.72	0.01\\
35.73	0.01\\
35.74	0.01\\
35.75	0.01\\
35.76	0.01\\
35.77	0.01\\
35.78	0.01\\
35.79	0.01\\
35.8	0.01\\
35.81	0.01\\
35.82	0.01\\
35.83	0.01\\
35.84	0.01\\
35.85	0.01\\
35.86	0.01\\
35.87	0.01\\
35.88	0.01\\
35.89	0.01\\
35.9	0.01\\
35.91	0.01\\
35.92	0.01\\
35.93	0.01\\
35.94	0.01\\
35.95	0.01\\
35.96	0.01\\
35.97	0.01\\
35.98	0.01\\
35.99	0.01\\
36	0.01\\
36.01	0.01\\
36.02	0.01\\
36.03	0.01\\
36.04	0.01\\
36.05	0.01\\
36.06	0.01\\
36.07	0.01\\
36.08	0.01\\
36.09	0.01\\
36.1	0.01\\
36.11	0.01\\
36.12	0.01\\
36.13	0.01\\
36.14	0.01\\
36.15	0.01\\
36.16	0.01\\
36.17	0.01\\
36.18	0.01\\
36.19	0.01\\
36.2	0.01\\
36.21	0.01\\
36.22	0.01\\
36.23	0.01\\
36.24	0.01\\
36.25	0.01\\
36.26	0.01\\
36.27	0.01\\
36.28	0.01\\
36.29	0.01\\
36.3	0.01\\
36.31	0.01\\
36.32	0.01\\
36.33	0.01\\
36.34	0.01\\
36.35	0.01\\
36.36	0.01\\
36.37	0.01\\
36.38	0.01\\
36.39	0.01\\
36.4	0.01\\
36.41	0.01\\
36.42	0.01\\
36.43	0.01\\
36.44	0.01\\
36.45	0.01\\
36.46	0.01\\
36.47	0.01\\
36.48	0.01\\
36.49	0.01\\
36.5	0.01\\
36.51	0.01\\
36.52	0.01\\
36.53	0.01\\
36.54	0.01\\
36.55	0.01\\
36.56	0.01\\
36.57	0.01\\
36.58	0.01\\
36.59	0.01\\
36.6	0.01\\
36.61	0.01\\
36.62	0.01\\
36.63	0.01\\
36.64	0.01\\
36.65	0.01\\
36.66	0.01\\
36.67	0.01\\
36.68	0.01\\
36.69	0.01\\
36.7	0.01\\
36.71	0.01\\
36.72	0.01\\
36.73	0.01\\
36.74	0.01\\
36.75	0.01\\
36.76	0.01\\
36.77	0.01\\
36.78	0.01\\
36.79	0.01\\
36.8	0.01\\
36.81	0.01\\
36.82	0.01\\
36.83	0.01\\
36.84	0.01\\
36.85	0.01\\
36.86	0.01\\
36.87	0.01\\
36.88	0.01\\
36.89	0.01\\
36.9	0.01\\
36.91	0.01\\
36.92	0.01\\
36.93	0.01\\
36.94	0.01\\
36.95	0.01\\
36.96	0.01\\
36.97	0.01\\
36.98	0.01\\
36.99	0.01\\
37	0.01\\
37.01	0.01\\
37.02	0.01\\
37.03	0.01\\
37.04	0.01\\
37.05	0.01\\
37.06	0.01\\
37.07	0.01\\
37.08	0.01\\
37.09	0.01\\
37.1	0.01\\
37.11	0.01\\
37.12	0.01\\
37.13	0.01\\
37.14	0.01\\
37.15	0.01\\
37.16	0.01\\
37.17	0.01\\
37.18	0.01\\
37.19	0.01\\
37.2	0.01\\
37.21	0.01\\
37.22	0.01\\
37.23	0.01\\
37.24	0.01\\
37.25	0.01\\
37.26	0.01\\
37.27	0.01\\
37.28	0.01\\
37.29	0.01\\
37.3	0.01\\
37.31	0.01\\
37.32	0.01\\
37.33	0.01\\
37.34	0.01\\
37.35	0.01\\
37.36	0.01\\
37.37	0.01\\
37.38	0.01\\
37.39	0.01\\
37.4	0.01\\
37.41	0.01\\
37.42	0.01\\
37.43	0.01\\
37.44	0.01\\
37.45	0.01\\
37.46	0.01\\
37.47	0.01\\
37.48	0.01\\
37.49	0.01\\
37.5	0.01\\
37.51	0.01\\
37.52	0.01\\
37.53	0.01\\
37.54	0.01\\
37.55	0.01\\
37.56	0.01\\
37.57	0.01\\
37.58	0.01\\
37.59	0.01\\
37.6	0.01\\
37.61	0.01\\
37.62	0.01\\
37.63	0.01\\
37.64	0.01\\
37.65	0.01\\
37.66	0.01\\
37.67	0.01\\
37.68	0.01\\
37.69	0.01\\
37.7	0.01\\
37.71	0.01\\
37.72	0.01\\
37.73	0.01\\
37.74	0.01\\
37.75	0.01\\
37.76	0.01\\
37.77	0.01\\
37.78	0.01\\
37.79	0.01\\
37.8	0.01\\
37.81	0.01\\
37.82	0.01\\
37.83	0.01\\
37.84	0.01\\
37.85	0.01\\
37.86	0.01\\
37.87	0.01\\
37.88	0.01\\
37.89	0.01\\
37.9	0.01\\
37.91	0.01\\
37.92	0.01\\
37.93	0.01\\
37.94	0.01\\
37.95	0.01\\
37.96	0.01\\
37.97	0.01\\
37.98	0.01\\
37.99	0.01\\
38	0.01\\
38.01	0.01\\
38.02	0.01\\
38.03	0.01\\
38.04	0.01\\
38.05	0.01\\
38.06	0.01\\
38.07	0.01\\
38.08	0.01\\
38.09	0.01\\
38.1	0.01\\
38.11	0.01\\
38.12	0.01\\
38.13	0.01\\
38.14	0.01\\
38.15	0.01\\
38.16	0.01\\
38.17	0.01\\
38.18	0.01\\
38.19	0.01\\
38.2	0.01\\
38.21	0.01\\
38.22	0.01\\
38.23	0.01\\
38.24	0.01\\
38.25	0.01\\
38.26	0.01\\
38.27	0.01\\
38.28	0.01\\
38.29	0.01\\
38.3	0.01\\
38.31	0.01\\
38.32	0.01\\
38.33	0.01\\
38.34	0.01\\
38.35	0.01\\
38.36	0.01\\
38.37	0.01\\
38.38	0.01\\
38.39	0.01\\
38.4	0.01\\
38.41	0.01\\
38.42	0.01\\
38.43	0.01\\
38.44	0.01\\
38.45	0.01\\
38.46	0.01\\
38.47	0.01\\
38.48	0.01\\
38.49	0.01\\
38.5	0.01\\
38.51	0.01\\
38.52	0.01\\
38.53	0.01\\
38.54	0.01\\
38.55	0.01\\
38.56	0.01\\
38.57	0.01\\
38.58	0.01\\
38.59	0.01\\
38.6	0.01\\
38.61	0.01\\
38.62	0.01\\
38.63	0.01\\
38.64	0.01\\
38.65	0.01\\
38.66	0.01\\
38.67	0.01\\
38.68	0.01\\
38.69	0.01\\
38.7	0.01\\
38.71	0.01\\
38.72	0.01\\
38.73	0.01\\
38.74	0.01\\
38.75	0.01\\
38.76	0.01\\
38.77	0.01\\
38.78	0.01\\
38.79	0.01\\
38.8	0.01\\
38.81	0.01\\
38.82	0.01\\
38.83	0.01\\
38.84	0.01\\
38.85	0.01\\
38.86	0.01\\
38.87	0.01\\
38.88	0.01\\
38.89	0.01\\
38.9	0.01\\
38.91	0.01\\
38.92	0.01\\
38.93	0.01\\
38.94	0.01\\
38.95	0.01\\
38.96	0.01\\
38.97	0.01\\
38.98	0.01\\
38.99	0.01\\
39	0.01\\
39.01	0.01\\
39.02	0.01\\
39.03	0.01\\
39.04	0.01\\
39.05	0.01\\
39.06	0.01\\
39.07	0.01\\
39.08	0.01\\
39.09	0.01\\
39.1	0.01\\
39.11	0.01\\
39.12	0.01\\
39.13	0.01\\
39.14	0.01\\
39.15	0.01\\
39.16	0.01\\
39.17	0.01\\
39.18	0.01\\
39.19	0.01\\
39.2	0.01\\
39.21	0.01\\
39.22	0.01\\
39.23	0.01\\
39.24	0.01\\
39.25	0.01\\
39.26	0.01\\
39.27	0.01\\
39.28	0.01\\
39.29	0.01\\
39.3	0.01\\
39.31	0.01\\
39.32	0.01\\
39.33	0.01\\
39.34	0.01\\
39.35	0.01\\
39.36	0.01\\
39.37	0.01\\
39.38	0.01\\
39.39	0.01\\
39.4	0.01\\
39.41	0.01\\
39.42	0.01\\
39.43	0.01\\
39.44	0.01\\
39.45	0.01\\
39.46	0.01\\
39.47	0.01\\
39.48	0.01\\
39.49	0.01\\
39.5	0.01\\
39.51	0.01\\
39.52	0.01\\
39.53	0.01\\
39.54	0.01\\
39.55	0.01\\
39.56	0.01\\
39.57	0.01\\
39.58	0.01\\
39.59	0.01\\
39.6	0.01\\
39.61	0.01\\
39.62	0.01\\
39.63	0.01\\
39.64	0.01\\
39.65	0.01\\
39.66	0.01\\
39.67	0.01\\
39.68	0.01\\
39.69	0.01\\
39.7	0.01\\
39.71	0.01\\
39.72	0.01\\
39.73	0.01\\
39.74	0.01\\
39.75	0.01\\
39.76	0.01\\
39.77	0.01\\
39.78	0.01\\
39.79	0.01\\
39.8	0.01\\
39.81	0.01\\
39.82	0.01\\
39.83	0.01\\
39.84	0.01\\
39.85	0.01\\
39.86	0.01\\
39.87	0.01\\
39.88	0.01\\
39.89	0.01\\
39.9	0.01\\
39.91	0.01\\
39.92	0.01\\
39.93	0.01\\
39.94	0.01\\
39.95	0.01\\
39.96	0.01\\
39.97	0.01\\
39.98	0.01\\
39.99	0.01\\
40	0.01\\
40.01	0.01\\
};
\addplot [color=green,solid,forget plot]
  table[row sep=crcr]{%
40.01	0.01\\
40.02	0.01\\
40.03	0.01\\
40.04	0.01\\
40.05	0.01\\
40.06	0.01\\
40.07	0.01\\
40.08	0.01\\
40.09	0.01\\
40.1	0.01\\
40.11	0.01\\
40.12	0.01\\
40.13	0.01\\
40.14	0.01\\
40.15	0.01\\
40.16	0.01\\
40.17	0.01\\
40.18	0.01\\
40.19	0.01\\
40.2	0.01\\
40.21	0.01\\
40.22	0.01\\
40.23	0.01\\
40.24	0.01\\
40.25	0.01\\
40.26	0.01\\
40.27	0.01\\
40.28	0.01\\
40.29	0.01\\
40.3	0.01\\
40.31	0.01\\
40.32	0.01\\
40.33	0.01\\
40.34	0.01\\
40.35	0.01\\
40.36	0.01\\
40.37	0.01\\
40.38	0.01\\
40.39	0.01\\
40.4	0.01\\
40.41	0.01\\
40.42	0.01\\
40.43	0.01\\
40.44	0.01\\
40.45	0.01\\
40.46	0.01\\
40.47	0.01\\
40.48	0.01\\
40.49	0.01\\
40.5	0.01\\
40.51	0.01\\
40.52	0.01\\
40.53	0.01\\
40.54	0.01\\
40.55	0.01\\
40.56	0.01\\
40.57	0.01\\
40.58	0.01\\
40.59	0.01\\
40.6	0.01\\
40.61	0.01\\
40.62	0.01\\
40.63	0.01\\
40.64	0.01\\
40.65	0.01\\
40.66	0.01\\
40.67	0.01\\
40.68	0.01\\
40.69	0.01\\
40.7	0.01\\
40.71	0.01\\
40.72	0.01\\
40.73	0.01\\
40.74	0.01\\
40.75	0.01\\
40.76	0.01\\
40.77	0.01\\
40.78	0.01\\
40.79	0.01\\
40.8	0.01\\
40.81	0.01\\
40.82	0.01\\
40.83	0.01\\
40.84	0.01\\
40.85	0.01\\
40.86	0.01\\
40.87	0.01\\
40.88	0.01\\
40.89	0.01\\
40.9	0.01\\
40.91	0.01\\
40.92	0.01\\
40.93	0.01\\
40.94	0.01\\
40.95	0.01\\
40.96	0.01\\
40.97	0.01\\
40.98	0.01\\
40.99	0.01\\
41	0.01\\
41.01	0.01\\
41.02	0.01\\
41.03	0.01\\
41.04	0.01\\
41.05	0.01\\
41.06	0.01\\
41.07	0.01\\
41.08	0.01\\
41.09	0.01\\
41.1	0.01\\
41.11	0.01\\
41.12	0.01\\
41.13	0.01\\
41.14	0.01\\
41.15	0.01\\
41.16	0.01\\
41.17	0.01\\
41.18	0.01\\
41.19	0.01\\
41.2	0.01\\
41.21	0.01\\
41.22	0.01\\
41.23	0.01\\
41.24	0.01\\
41.25	0.01\\
41.26	0.01\\
41.27	0.01\\
41.28	0.01\\
41.29	0.01\\
41.3	0.01\\
41.31	0.01\\
41.32	0.01\\
41.33	0.01\\
41.34	0.01\\
41.35	0.01\\
41.36	0.01\\
41.37	0.01\\
41.38	0.01\\
41.39	0.01\\
41.4	0.01\\
41.41	0.01\\
41.42	0.01\\
41.43	0.01\\
41.44	0.01\\
41.45	0.01\\
41.46	0.01\\
41.47	0.01\\
41.48	0.01\\
41.49	0.01\\
41.5	0.01\\
41.51	0.01\\
41.52	0.01\\
41.53	0.01\\
41.54	0.01\\
41.55	0.01\\
41.56	0.01\\
41.57	0.01\\
41.58	0.01\\
41.59	0.01\\
41.6	0.01\\
41.61	0.01\\
41.62	0.01\\
41.63	0.01\\
41.64	0.01\\
41.65	0.01\\
41.66	0.01\\
41.67	0.01\\
41.68	0.01\\
41.69	0.01\\
41.7	0.01\\
41.71	0.01\\
41.72	0.01\\
41.73	0.01\\
41.74	0.01\\
41.75	0.01\\
41.76	0.01\\
41.77	0.01\\
41.78	0.01\\
41.79	0.01\\
41.8	0.01\\
41.81	0.01\\
41.82	0.01\\
41.83	0.01\\
41.84	0.01\\
41.85	0.01\\
41.86	0.01\\
41.87	0.01\\
41.88	0.01\\
41.89	0.01\\
41.9	0.01\\
41.91	0.01\\
41.92	0.01\\
41.93	0.01\\
41.94	0.01\\
41.95	0.01\\
41.96	0.01\\
41.97	0.01\\
41.98	0.01\\
41.99	0.01\\
42	0.01\\
42.01	0.01\\
42.02	0.01\\
42.03	0.01\\
42.04	0.01\\
42.05	0.01\\
42.06	0.01\\
42.07	0.01\\
42.08	0.01\\
42.09	0.01\\
42.1	0.01\\
42.11	0.01\\
42.12	0.01\\
42.13	0.01\\
42.14	0.01\\
42.15	0.01\\
42.16	0.01\\
42.17	0.01\\
42.18	0.01\\
42.19	0.01\\
42.2	0.01\\
42.21	0.01\\
42.22	0.01\\
42.23	0.01\\
42.24	0.01\\
42.25	0.01\\
42.26	0.01\\
42.27	0.01\\
42.28	0.01\\
42.29	0.01\\
42.3	0.01\\
42.31	0.01\\
42.32	0.01\\
42.33	0.01\\
42.34	0.01\\
42.35	0.01\\
42.36	0.01\\
42.37	0.01\\
42.38	0.01\\
42.39	0.01\\
42.4	0.01\\
42.41	0.01\\
42.42	0.01\\
42.43	0.01\\
42.44	0.01\\
42.45	0.01\\
42.46	0.01\\
42.47	0.01\\
42.48	0.01\\
42.49	0.01\\
42.5	0.01\\
42.51	0.01\\
42.52	0.01\\
42.53	0.01\\
42.54	0.01\\
42.55	0.01\\
42.56	0.01\\
42.57	0.01\\
42.58	0.01\\
42.59	0.01\\
42.6	0.01\\
42.61	0.01\\
42.62	0.01\\
42.63	0.01\\
42.64	0.01\\
42.65	0.01\\
42.66	0.01\\
42.67	0.01\\
42.68	0.01\\
42.69	0.01\\
42.7	0.01\\
42.71	0.01\\
42.72	0.01\\
42.73	0.01\\
42.74	0.01\\
42.75	0.01\\
42.76	0.01\\
42.77	0.01\\
42.78	0.01\\
42.79	0.01\\
42.8	0.01\\
42.81	0.01\\
42.82	0.01\\
42.83	0.01\\
42.84	0.01\\
42.85	0.01\\
42.86	0.01\\
42.87	0.01\\
42.88	0.01\\
42.89	0.01\\
42.9	0.01\\
42.91	0.01\\
42.92	0.01\\
42.93	0.01\\
42.94	0.01\\
42.95	0.01\\
42.96	0.01\\
42.97	0.01\\
42.98	0.01\\
42.99	0.01\\
43	0.01\\
43.01	0.01\\
43.02	0.01\\
43.03	0.01\\
43.04	0.01\\
43.05	0.01\\
43.06	0.01\\
43.07	0.01\\
43.08	0.01\\
43.09	0.01\\
43.1	0.01\\
43.11	0.01\\
43.12	0.01\\
43.13	0.01\\
43.14	0.01\\
43.15	0.01\\
43.16	0.01\\
43.17	0.01\\
43.18	0.01\\
43.19	0.01\\
43.2	0.01\\
43.21	0.01\\
43.22	0.01\\
43.23	0.01\\
43.24	0.01\\
43.25	0.01\\
43.26	0.01\\
43.27	0.01\\
43.28	0.01\\
43.29	0.01\\
43.3	0.01\\
43.31	0.01\\
43.32	0.01\\
43.33	0.01\\
43.34	0.01\\
43.35	0.01\\
43.36	0.01\\
43.37	0.01\\
43.38	0.01\\
43.39	0.01\\
43.4	0.01\\
43.41	0.01\\
43.42	0.01\\
43.43	0.01\\
43.44	0.01\\
43.45	0.01\\
43.46	0.01\\
43.47	0.01\\
43.48	0.01\\
43.49	0.01\\
43.5	0.01\\
43.51	0.01\\
43.52	0.01\\
43.53	0.01\\
43.54	0.01\\
43.55	0.01\\
43.56	0.01\\
43.57	0.01\\
43.58	0.01\\
43.59	0.01\\
43.6	0.01\\
43.61	0.01\\
43.62	0.01\\
43.63	0.01\\
43.64	0.01\\
43.65	0.01\\
43.66	0.01\\
43.67	0.01\\
43.68	0.01\\
43.69	0.01\\
43.7	0.01\\
43.71	0.01\\
43.72	0.01\\
43.73	0.01\\
43.74	0.01\\
43.75	0.01\\
43.76	0.01\\
43.77	0.01\\
43.78	0.01\\
43.79	0.01\\
43.8	0.01\\
43.81	0.01\\
43.82	0.01\\
43.83	0.01\\
43.84	0.01\\
43.85	0.01\\
43.86	0.01\\
43.87	0.01\\
43.88	0.01\\
43.89	0.01\\
43.9	0.01\\
43.91	0.01\\
43.92	0.01\\
43.93	0.01\\
43.94	0.01\\
43.95	0.01\\
43.96	0.01\\
43.97	0.01\\
43.98	0.01\\
43.99	0.01\\
44	0.01\\
44.01	0.01\\
44.02	0.01\\
44.03	0.01\\
44.04	0.01\\
44.05	0.01\\
44.06	0.01\\
44.07	0.01\\
44.08	0.01\\
44.09	0.01\\
44.1	0.01\\
44.11	0.01\\
44.12	0.01\\
44.13	0.01\\
44.14	0.01\\
44.15	0.01\\
44.16	0.01\\
44.17	0.01\\
44.18	0.01\\
44.19	0.01\\
44.2	0.01\\
44.21	0.01\\
44.22	0.01\\
44.23	0.01\\
44.24	0.01\\
44.25	0.01\\
44.26	0.01\\
44.27	0.01\\
44.28	0.01\\
44.29	0.01\\
44.3	0.01\\
44.31	0.01\\
44.32	0.01\\
44.33	0.01\\
44.34	0.01\\
44.35	0.01\\
44.36	0.01\\
44.37	0.01\\
44.38	0.01\\
44.39	0.01\\
44.4	0.01\\
44.41	0.01\\
44.42	0.01\\
44.43	0.01\\
44.44	0.01\\
44.45	0.01\\
44.46	0.01\\
44.47	0.01\\
44.48	0.01\\
44.49	0.01\\
44.5	0.01\\
44.51	0.01\\
44.52	0.01\\
44.53	0.01\\
44.54	0.01\\
44.55	0.01\\
44.56	0.01\\
44.57	0.01\\
44.58	0.01\\
44.59	0.01\\
44.6	0.01\\
44.61	0.01\\
44.62	0.01\\
44.63	0.01\\
44.64	0.01\\
44.65	0.01\\
44.66	0.01\\
44.67	0.01\\
44.68	0.01\\
44.69	0.01\\
44.7	0.01\\
44.71	0.01\\
44.72	0.01\\
44.73	0.01\\
44.74	0.01\\
44.75	0.01\\
44.76	0.01\\
44.77	0.01\\
44.78	0.01\\
44.79	0.01\\
44.8	0.01\\
44.81	0.01\\
44.82	0.01\\
44.83	0.01\\
44.84	0.01\\
44.85	0.01\\
44.86	0.01\\
44.87	0.01\\
44.88	0.01\\
44.89	0.01\\
44.9	0.01\\
44.91	0.01\\
44.92	0.01\\
44.93	0.01\\
44.94	0.01\\
44.95	0.01\\
44.96	0.01\\
44.97	0.01\\
44.98	0.01\\
44.99	0.01\\
45	0.01\\
45.01	0.01\\
45.02	0.01\\
45.03	0.01\\
45.04	0.01\\
45.05	0.01\\
45.06	0.01\\
45.07	0.01\\
45.08	0.01\\
45.09	0.01\\
45.1	0.01\\
45.11	0.01\\
45.12	0.01\\
45.13	0.01\\
45.14	0.01\\
45.15	0.01\\
45.16	0.01\\
45.17	0.01\\
45.18	0.01\\
45.19	0.01\\
45.2	0.01\\
45.21	0.01\\
45.22	0.01\\
45.23	0.01\\
45.24	0.01\\
45.25	0.01\\
45.26	0.01\\
45.27	0.01\\
45.28	0.01\\
45.29	0.01\\
45.3	0.01\\
45.31	0.01\\
45.32	0.01\\
45.33	0.01\\
45.34	0.01\\
45.35	0.01\\
45.36	0.01\\
45.37	0.01\\
45.38	0.01\\
45.39	0.01\\
45.4	0.01\\
45.41	0.01\\
45.42	0.01\\
45.43	0.01\\
45.44	0.01\\
45.45	0.01\\
45.46	0.01\\
45.47	0.01\\
45.48	0.01\\
45.49	0.01\\
45.5	0.01\\
45.51	0.01\\
45.52	0.01\\
45.53	0.01\\
45.54	0.01\\
45.55	0.01\\
45.56	0.01\\
45.57	0.01\\
45.58	0.01\\
45.59	0.01\\
45.6	0.01\\
45.61	0.01\\
45.62	0.01\\
45.63	0.01\\
45.64	0.01\\
45.65	0.01\\
45.66	0.01\\
45.67	0.01\\
45.68	0.01\\
45.69	0.01\\
45.7	0.01\\
45.71	0.01\\
45.72	0.01\\
45.73	0.01\\
45.74	0.01\\
45.75	0.01\\
45.76	0.01\\
45.77	0.01\\
45.78	0.01\\
45.79	0.01\\
45.8	0.01\\
45.81	0.01\\
45.82	0.01\\
45.83	0.01\\
45.84	0.01\\
45.85	0.01\\
45.86	0.01\\
45.87	0.01\\
45.88	0.01\\
45.89	0.01\\
45.9	0.01\\
45.91	0.01\\
45.92	0.01\\
45.93	0.01\\
45.94	0.01\\
45.95	0.01\\
45.96	0.01\\
45.97	0.01\\
45.98	0.01\\
45.99	0.01\\
46	0.01\\
46.01	0.01\\
46.02	0.01\\
46.03	0.01\\
46.04	0.01\\
46.05	0.01\\
46.06	0.01\\
46.07	0.01\\
46.08	0.01\\
46.09	0.01\\
46.1	0.01\\
46.11	0.01\\
46.12	0.01\\
46.13	0.01\\
46.14	0.01\\
46.15	0.01\\
46.16	0.01\\
46.17	0.01\\
46.18	0.01\\
46.19	0.01\\
46.2	0.01\\
46.21	0.01\\
46.22	0.01\\
46.23	0.01\\
46.24	0.01\\
46.25	0.01\\
46.26	0.01\\
46.27	0.01\\
46.28	0.01\\
46.29	0.01\\
46.3	0.01\\
46.31	0.01\\
46.32	0.01\\
46.33	0.01\\
46.34	0.01\\
46.35	0.01\\
46.36	0.01\\
46.37	0.01\\
46.38	0.01\\
46.39	0.01\\
46.4	0.01\\
46.41	0.01\\
46.42	0.01\\
46.43	0.01\\
46.44	0.01\\
46.45	0.01\\
46.46	0.01\\
46.47	0.01\\
46.48	0.01\\
46.49	0.01\\
46.5	0.01\\
46.51	0.01\\
46.52	0.01\\
46.53	0.01\\
46.54	0.01\\
46.55	0.01\\
46.56	0.01\\
46.57	0.01\\
46.58	0.01\\
46.59	0.01\\
46.6	0.01\\
46.61	0.01\\
46.62	0.01\\
46.63	0.01\\
46.64	0.01\\
46.65	0.01\\
46.66	0.01\\
46.67	0.01\\
46.68	0.01\\
46.69	0.01\\
46.7	0.01\\
46.71	0.01\\
46.72	0.01\\
46.73	0.01\\
46.74	0.01\\
46.75	0.01\\
46.76	0.01\\
46.77	0.01\\
46.78	0.01\\
46.79	0.01\\
46.8	0.01\\
46.81	0.01\\
46.82	0.01\\
46.83	0.01\\
46.84	0.01\\
46.85	0.01\\
46.86	0.01\\
46.87	0.01\\
46.88	0.01\\
46.89	0.01\\
46.9	0.01\\
46.91	0.01\\
46.92	0.01\\
46.93	0.01\\
46.94	0.01\\
46.95	0.01\\
46.96	0.01\\
46.97	0.01\\
46.98	0.01\\
46.99	0.01\\
47	0.01\\
47.01	0.01\\
47.02	0.01\\
47.03	0.01\\
47.04	0.01\\
47.05	0.01\\
47.06	0.01\\
47.07	0.01\\
47.08	0.01\\
47.09	0.01\\
47.1	0.01\\
47.11	0.01\\
47.12	0.01\\
47.13	0.01\\
47.14	0.01\\
47.15	0.01\\
47.16	0.01\\
47.17	0.01\\
47.18	0.01\\
47.19	0.01\\
47.2	0.01\\
47.21	0.01\\
47.22	0.01\\
47.23	0.01\\
47.24	0.01\\
47.25	0.01\\
47.26	0.01\\
47.27	0.01\\
47.28	0.01\\
47.29	0.01\\
47.3	0.01\\
47.31	0.01\\
47.32	0.01\\
47.33	0.01\\
47.34	0.01\\
47.35	0.01\\
47.36	0.01\\
47.37	0.01\\
47.38	0.01\\
47.39	0.01\\
47.4	0.01\\
47.41	0.01\\
47.42	0.01\\
47.43	0.01\\
47.44	0.01\\
47.45	0.01\\
47.46	0.01\\
47.47	0.01\\
47.48	0.01\\
47.49	0.01\\
47.5	0.01\\
47.51	0.01\\
47.52	0.01\\
47.53	0.01\\
47.54	0.01\\
47.55	0.01\\
47.56	0.01\\
47.57	0.01\\
47.58	0.01\\
47.59	0.01\\
47.6	0.01\\
47.61	0.01\\
47.62	0.01\\
47.63	0.01\\
47.64	0.01\\
47.65	0.01\\
47.66	0.01\\
47.67	0.01\\
47.68	0.01\\
47.69	0.01\\
47.7	0.01\\
47.71	0.01\\
47.72	0.01\\
47.73	0.01\\
47.74	0.01\\
47.75	0.01\\
47.76	0.01\\
47.77	0.01\\
47.78	0.01\\
47.79	0.01\\
47.8	0.01\\
47.81	0.01\\
47.82	0.01\\
47.83	0.01\\
47.84	0.01\\
47.85	0.01\\
47.86	0.01\\
47.87	0.01\\
47.88	0.01\\
47.89	0.01\\
47.9	0.01\\
47.91	0.01\\
47.92	0.01\\
47.93	0.01\\
47.94	0.01\\
47.95	0.01\\
47.96	0.01\\
47.97	0.01\\
47.98	0.01\\
47.99	0.01\\
48	0.01\\
48.01	0.01\\
48.02	0.01\\
48.03	0.01\\
48.04	0.01\\
48.05	0.01\\
48.06	0.01\\
48.07	0.01\\
48.08	0.01\\
48.09	0.01\\
48.1	0.01\\
48.11	0.01\\
48.12	0.01\\
48.13	0.01\\
48.14	0.01\\
48.15	0.01\\
48.16	0.01\\
48.17	0.01\\
48.18	0.01\\
48.19	0.01\\
48.2	0.01\\
48.21	0.01\\
48.22	0.01\\
48.23	0.01\\
48.24	0.01\\
48.25	0.01\\
48.26	0.01\\
48.27	0.01\\
48.28	0.01\\
48.29	0.01\\
48.3	0.01\\
48.31	0.01\\
48.32	0.01\\
48.33	0.01\\
48.34	0.01\\
48.35	0.01\\
48.36	0.01\\
48.37	0.01\\
48.38	0.01\\
48.39	0.01\\
48.4	0.01\\
48.41	0.01\\
48.42	0.01\\
48.43	0.01\\
48.44	0.01\\
48.45	0.01\\
48.46	0.01\\
48.47	0.01\\
48.48	0.01\\
48.49	0.01\\
48.5	0.01\\
48.51	0.01\\
48.52	0.01\\
48.53	0.01\\
48.54	0.01\\
48.55	0.01\\
48.56	0.01\\
48.57	0.01\\
48.58	0.01\\
48.59	0.01\\
48.6	0.01\\
48.61	0.01\\
48.62	0.01\\
48.63	0.01\\
48.64	0.01\\
48.65	0.01\\
48.66	0.01\\
48.67	0.01\\
48.68	0.01\\
48.69	0.01\\
48.7	0.01\\
48.71	0.01\\
48.72	0.01\\
48.73	0.01\\
48.74	0.01\\
48.75	0.01\\
48.76	0.01\\
48.77	0.01\\
48.78	0.01\\
48.79	0.01\\
48.8	0.01\\
48.81	0.01\\
48.82	0.01\\
48.83	0.01\\
48.84	0.01\\
48.85	0.01\\
48.86	0.01\\
48.87	0.01\\
48.88	0.01\\
48.89	0.01\\
48.9	0.01\\
48.91	0.01\\
48.92	0.01\\
48.93	0.01\\
48.94	0.01\\
48.95	0.01\\
48.96	0.01\\
48.97	0.01\\
48.98	0.01\\
48.99	0.01\\
49	0.01\\
49.01	0.01\\
49.02	0.01\\
49.03	0.01\\
49.04	0.01\\
49.05	0.01\\
49.06	0.01\\
49.07	0.01\\
49.08	0.01\\
49.09	0.01\\
49.1	0.01\\
49.11	0.01\\
49.12	0.01\\
49.13	0.01\\
49.14	0.01\\
49.15	0.01\\
49.16	0.01\\
49.17	0.01\\
49.18	0.01\\
49.19	0.01\\
49.2	0.01\\
49.21	0.01\\
49.22	0.01\\
49.23	0.01\\
49.24	0.01\\
49.25	0.01\\
49.26	0.01\\
49.27	0.01\\
49.28	0.01\\
49.29	0.01\\
49.3	0.01\\
49.31	0.01\\
49.32	0.01\\
49.33	0.01\\
49.34	0.01\\
49.35	0.01\\
49.36	0.01\\
49.37	0.01\\
49.38	0.01\\
49.39	0.01\\
49.4	0.01\\
49.41	0.01\\
49.42	0.01\\
49.43	0.01\\
49.44	0.01\\
49.45	0.01\\
49.46	0.01\\
49.47	0.01\\
49.48	0.01\\
49.49	0.01\\
49.5	0.01\\
49.51	0.01\\
49.52	0.01\\
49.53	0.01\\
49.54	0.01\\
49.55	0.01\\
49.56	0.01\\
49.57	0.01\\
49.58	0.01\\
49.59	0.01\\
49.6	0.01\\
49.61	0.01\\
49.62	0.01\\
49.63	0.01\\
49.64	0.01\\
49.65	0.01\\
49.66	0.01\\
49.67	0.01\\
49.68	0.01\\
49.69	0.01\\
49.7	0.01\\
49.71	0.01\\
49.72	0.01\\
49.73	0.01\\
49.74	0.01\\
49.75	0.01\\
49.76	0.01\\
49.77	0.01\\
49.78	0.01\\
49.79	0.01\\
49.8	0.01\\
49.81	0.01\\
49.82	0.01\\
49.83	0.01\\
49.84	0.01\\
49.85	0.01\\
49.86	0.01\\
49.87	0.01\\
49.88	0.01\\
49.89	0.01\\
49.9	0.01\\
49.91	0.01\\
49.92	0.01\\
49.93	0.01\\
49.94	0.01\\
49.95	0.01\\
49.96	0.01\\
49.97	0.01\\
49.98	0.01\\
49.99	0.01\\
50	0.01\\
50.01	0.01\\
50.02	0.01\\
50.03	0.01\\
50.04	0.01\\
50.05	0.01\\
50.06	0.01\\
50.07	0.01\\
50.08	0.01\\
50.09	0.01\\
50.1	0.01\\
50.11	0.01\\
50.12	0.01\\
50.13	0.01\\
50.14	0.01\\
50.15	0.01\\
50.16	0.01\\
50.17	0.01\\
50.18	0.01\\
50.19	0.01\\
50.2	0.01\\
50.21	0.01\\
50.22	0.01\\
50.23	0.01\\
50.24	0.01\\
50.25	0.01\\
50.26	0.01\\
50.27	0.01\\
50.28	0.01\\
50.29	0.01\\
50.3	0.01\\
50.31	0.01\\
50.32	0.01\\
50.33	0.01\\
50.34	0.01\\
50.35	0.01\\
50.36	0.01\\
50.37	0.01\\
50.38	0.01\\
50.39	0.01\\
50.4	0.01\\
50.41	0.01\\
50.42	0.01\\
50.43	0.01\\
50.44	0.01\\
50.45	0.01\\
50.46	0.01\\
50.47	0.01\\
50.48	0.01\\
50.49	0.01\\
50.5	0.01\\
50.51	0.01\\
50.52	0.01\\
50.53	0.01\\
50.54	0.01\\
50.55	0.01\\
50.56	0.01\\
50.57	0.01\\
50.58	0.01\\
50.59	0.01\\
50.6	0.01\\
50.61	0.01\\
50.62	0.01\\
50.63	0.01\\
50.64	0.01\\
50.65	0.01\\
50.66	0.01\\
50.67	0.01\\
50.68	0.01\\
50.69	0.01\\
50.7	0.01\\
50.71	0.01\\
50.72	0.01\\
50.73	0.01\\
50.74	0.01\\
50.75	0.01\\
50.76	0.01\\
50.77	0.01\\
50.78	0.01\\
50.79	0.01\\
50.8	0.01\\
50.81	0.01\\
50.82	0.01\\
50.83	0.01\\
50.84	0.01\\
50.85	0.01\\
50.86	0.01\\
50.87	0.01\\
50.88	0.01\\
50.89	0.01\\
50.9	0.01\\
50.91	0.01\\
50.92	0.01\\
50.93	0.01\\
50.94	0.01\\
50.95	0.01\\
50.96	0.01\\
50.97	0.01\\
50.98	0.01\\
50.99	0.01\\
51	0.01\\
51.01	0.01\\
51.02	0.01\\
51.03	0.01\\
51.04	0.01\\
51.05	0.01\\
51.06	0.01\\
51.07	0.01\\
51.08	0.01\\
51.09	0.01\\
51.1	0.01\\
51.11	0.01\\
51.12	0.01\\
51.13	0.01\\
51.14	0.01\\
51.15	0.01\\
51.16	0.01\\
51.17	0.01\\
51.18	0.01\\
51.19	0.01\\
51.2	0.01\\
51.21	0.01\\
51.22	0.01\\
51.23	0.01\\
51.24	0.01\\
51.25	0.01\\
51.26	0.01\\
51.27	0.01\\
51.28	0.01\\
51.29	0.01\\
51.3	0.01\\
51.31	0.01\\
51.32	0.01\\
51.33	0.01\\
51.34	0.01\\
51.35	0.01\\
51.36	0.01\\
51.37	0.01\\
51.38	0.01\\
51.39	0.01\\
51.4	0.01\\
51.41	0.01\\
51.42	0.01\\
51.43	0.01\\
51.44	0.01\\
51.45	0.01\\
51.46	0.01\\
51.47	0.01\\
51.48	0.01\\
51.49	0.01\\
51.5	0.01\\
51.51	0.01\\
51.52	0.01\\
51.53	0.01\\
51.54	0.01\\
51.55	0.01\\
51.56	0.01\\
51.57	0.01\\
51.58	0.01\\
51.59	0.01\\
51.6	0.01\\
51.61	0.01\\
51.62	0.01\\
51.63	0.01\\
51.64	0.01\\
51.65	0.01\\
51.66	0.01\\
51.67	0.01\\
51.68	0.01\\
51.69	0.01\\
51.7	0.01\\
51.71	0.01\\
51.72	0.01\\
51.73	0.01\\
51.74	0.01\\
51.75	0.01\\
51.76	0.01\\
51.77	0.01\\
51.78	0.01\\
51.79	0.01\\
51.8	0.01\\
51.81	0.01\\
51.82	0.01\\
51.83	0.01\\
51.84	0.01\\
51.85	0.01\\
51.86	0.01\\
51.87	0.01\\
51.88	0.01\\
51.89	0.01\\
51.9	0.01\\
51.91	0.01\\
51.92	0.01\\
51.93	0.01\\
51.94	0.01\\
51.95	0.01\\
51.96	0.01\\
51.97	0.01\\
51.98	0.01\\
51.99	0.01\\
52	0.01\\
52.01	0.01\\
52.02	0.01\\
52.03	0.01\\
52.04	0.01\\
52.05	0.01\\
52.06	0.01\\
52.07	0.01\\
52.08	0.01\\
52.09	0.01\\
52.1	0.01\\
52.11	0.01\\
52.12	0.01\\
52.13	0.01\\
52.14	0.01\\
52.15	0.01\\
52.16	0.01\\
52.17	0.01\\
52.18	0.01\\
52.19	0.01\\
52.2	0.01\\
52.21	0.01\\
52.22	0.01\\
52.23	0.01\\
52.24	0.01\\
52.25	0.01\\
52.26	0.01\\
52.27	0.01\\
52.28	0.01\\
52.29	0.01\\
52.3	0.01\\
52.31	0.01\\
52.32	0.01\\
52.33	0.01\\
52.34	0.01\\
52.35	0.01\\
52.36	0.01\\
52.37	0.01\\
52.38	0.01\\
52.39	0.01\\
52.4	0.01\\
52.41	0.01\\
52.42	0.01\\
52.43	0.01\\
52.44	0.01\\
52.45	0.01\\
52.46	0.01\\
52.47	0.01\\
52.48	0.01\\
52.49	0.01\\
52.5	0.01\\
52.51	0.01\\
52.52	0.01\\
52.53	0.01\\
52.54	0.01\\
52.55	0.01\\
52.56	0.01\\
52.57	0.01\\
52.58	0.01\\
52.59	0.01\\
52.6	0.01\\
52.61	0.01\\
52.62	0.01\\
52.63	0.01\\
52.64	0.01\\
52.65	0.01\\
52.66	0.01\\
52.67	0.01\\
52.68	0.01\\
52.69	0.01\\
52.7	0.01\\
52.71	0.01\\
52.72	0.01\\
52.73	0.01\\
52.74	0.01\\
52.75	0.01\\
52.76	0.01\\
52.77	0.01\\
52.78	0.01\\
52.79	0.01\\
52.8	0.01\\
52.81	0.01\\
52.82	0.01\\
52.83	0.01\\
52.84	0.01\\
52.85	0.01\\
52.86	0.01\\
52.87	0.01\\
52.88	0.01\\
52.89	0.01\\
52.9	0.01\\
52.91	0.01\\
52.92	0.01\\
52.93	0.01\\
52.94	0.01\\
52.95	0.01\\
52.96	0.01\\
52.97	0.01\\
52.98	0.01\\
52.99	0.01\\
53	0.01\\
53.01	0.01\\
53.02	0.01\\
53.03	0.01\\
53.04	0.01\\
53.05	0.01\\
53.06	0.01\\
53.07	0.01\\
53.08	0.01\\
53.09	0.01\\
53.1	0.01\\
53.11	0.01\\
53.12	0.01\\
53.13	0.01\\
53.14	0.01\\
53.15	0.01\\
53.16	0.01\\
53.17	0.01\\
53.18	0.01\\
53.19	0.01\\
53.2	0.01\\
53.21	0.01\\
53.22	0.01\\
53.23	0.01\\
53.24	0.01\\
53.25	0.01\\
53.26	0.01\\
53.27	0.01\\
53.28	0.01\\
53.29	0.01\\
53.3	0.01\\
53.31	0.01\\
53.32	0.01\\
53.33	0.01\\
53.34	0.01\\
53.35	0.01\\
53.36	0.01\\
53.37	0.01\\
53.38	0.01\\
53.39	0.01\\
53.4	0.01\\
53.41	0.01\\
53.42	0.01\\
53.43	0.01\\
53.44	0.01\\
53.45	0.01\\
53.46	0.01\\
53.47	0.01\\
53.48	0.01\\
53.49	0.01\\
53.5	0.01\\
53.51	0.01\\
53.52	0.01\\
53.53	0.01\\
53.54	0.01\\
53.55	0.01\\
53.56	0.01\\
53.57	0.01\\
53.58	0.01\\
53.59	0.01\\
53.6	0.01\\
53.61	0.01\\
53.62	0.01\\
53.63	0.01\\
53.64	0.01\\
53.65	0.01\\
53.66	0.01\\
53.67	0.01\\
53.68	0.01\\
53.69	0.01\\
53.7	0.01\\
53.71	0.01\\
53.72	0.01\\
53.73	0.01\\
53.74	0.01\\
53.75	0.01\\
53.76	0.01\\
53.77	0.01\\
53.78	0.01\\
53.79	0.01\\
53.8	0.01\\
53.81	0.01\\
53.82	0.01\\
53.83	0.01\\
53.84	0.01\\
53.85	0.01\\
53.86	0.01\\
53.87	0.01\\
53.88	0.01\\
53.89	0.01\\
53.9	0.01\\
53.91	0.01\\
53.92	0.01\\
53.93	0.01\\
53.94	0.01\\
53.95	0.01\\
53.96	0.01\\
53.97	0.01\\
53.98	0.01\\
53.99	0.01\\
54	0.01\\
54.01	0.01\\
54.02	0.01\\
54.03	0.01\\
54.04	0.01\\
54.05	0.01\\
54.06	0.01\\
54.07	0.01\\
54.08	0.01\\
54.09	0.01\\
54.1	0.01\\
54.11	0.01\\
54.12	0.01\\
54.13	0.01\\
54.14	0.01\\
54.15	0.01\\
54.16	0.01\\
54.17	0.01\\
54.18	0.01\\
54.19	0.01\\
54.2	0.01\\
54.21	0.01\\
54.22	0.01\\
54.23	0.01\\
54.24	0.01\\
54.25	0.01\\
54.26	0.01\\
54.27	0.01\\
54.28	0.01\\
54.29	0.01\\
54.3	0.01\\
54.31	0.01\\
54.32	0.01\\
54.33	0.01\\
54.34	0.01\\
54.35	0.01\\
54.36	0.01\\
54.37	0.01\\
54.38	0.01\\
54.39	0.01\\
54.4	0.01\\
54.41	0.01\\
54.42	0.01\\
54.43	0.01\\
54.44	0.01\\
54.45	0.01\\
54.46	0.01\\
54.47	0.01\\
54.48	0.01\\
54.49	0.01\\
54.5	0.01\\
54.51	0.01\\
54.52	0.01\\
54.53	0.01\\
54.54	0.01\\
54.55	0.01\\
54.56	0.01\\
54.57	0.01\\
54.58	0.01\\
54.59	0.01\\
54.6	0.01\\
54.61	0.01\\
54.62	0.01\\
54.63	0.01\\
54.64	0.01\\
54.65	0.01\\
54.66	0.01\\
54.67	0.01\\
54.68	0.01\\
54.69	0.01\\
54.7	0.01\\
54.71	0.01\\
54.72	0.01\\
54.73	0.01\\
54.74	0.01\\
54.75	0.01\\
54.76	0.01\\
54.77	0.01\\
54.78	0.01\\
54.79	0.01\\
54.8	0.01\\
54.81	0.01\\
54.82	0.01\\
54.83	0.01\\
54.84	0.01\\
54.85	0.01\\
54.86	0.01\\
54.87	0.01\\
54.88	0.01\\
54.89	0.01\\
54.9	0.01\\
54.91	0.01\\
54.92	0.01\\
54.93	0.01\\
54.94	0.01\\
54.95	0.01\\
54.96	0.01\\
54.97	0.01\\
54.98	0.01\\
54.99	0.01\\
55	0.01\\
55.01	0.01\\
55.02	0.01\\
55.03	0.01\\
55.04	0.01\\
55.05	0.01\\
55.06	0.01\\
55.07	0.01\\
55.08	0.01\\
55.09	0.01\\
55.1	0.01\\
55.11	0.01\\
55.12	0.01\\
55.13	0.01\\
55.14	0.01\\
55.15	0.01\\
55.16	0.01\\
55.17	0.01\\
55.18	0.01\\
55.19	0.01\\
55.2	0.01\\
55.21	0.01\\
55.22	0.01\\
55.23	0.01\\
55.24	0.01\\
55.25	0.01\\
55.26	0.01\\
55.27	0.01\\
55.28	0.01\\
55.29	0.01\\
55.3	0.01\\
55.31	0.01\\
55.32	0.01\\
55.33	0.01\\
55.34	0.01\\
55.35	0.01\\
55.36	0.01\\
55.37	0.01\\
55.38	0.01\\
55.39	0.01\\
55.4	0.01\\
55.41	0.01\\
55.42	0.01\\
55.43	0.01\\
55.44	0.01\\
55.45	0.01\\
55.46	0.01\\
55.47	0.01\\
55.48	0.01\\
55.49	0.01\\
55.5	0.01\\
55.51	0.01\\
55.52	0.01\\
55.53	0.01\\
55.54	0.01\\
55.55	0.01\\
55.56	0.01\\
55.57	0.01\\
55.58	0.01\\
55.59	0.01\\
55.6	0.01\\
55.61	0.01\\
55.62	0.01\\
55.63	0.01\\
55.64	0.01\\
55.65	0.01\\
55.66	0.01\\
55.67	0.01\\
55.68	0.01\\
55.69	0.01\\
55.7	0.01\\
55.71	0.01\\
55.72	0.01\\
55.73	0.01\\
55.74	0.01\\
55.75	0.01\\
55.76	0.01\\
55.77	0.01\\
55.78	0.01\\
55.79	0.01\\
55.8	0.01\\
55.81	0.01\\
55.82	0.01\\
55.83	0.01\\
55.84	0.01\\
55.85	0.01\\
55.86	0.01\\
55.87	0.01\\
55.88	0.01\\
55.89	0.01\\
55.9	0.01\\
55.91	0.01\\
55.92	0.01\\
55.93	0.01\\
55.94	0.01\\
55.95	0.01\\
55.96	0.01\\
55.97	0.01\\
55.98	0.01\\
55.99	0.01\\
56	0.01\\
56.01	0.01\\
56.02	0.01\\
56.03	0.01\\
56.04	0.01\\
56.05	0.01\\
56.06	0.01\\
56.07	0.01\\
56.08	0.01\\
56.09	0.01\\
56.1	0.01\\
56.11	0.01\\
56.12	0.01\\
56.13	0.01\\
56.14	0.01\\
56.15	0.01\\
56.16	0.01\\
56.17	0.01\\
56.18	0.01\\
56.19	0.01\\
56.2	0.01\\
56.21	0.01\\
56.22	0.01\\
56.23	0.01\\
56.24	0.01\\
56.25	0.01\\
56.26	0.01\\
56.27	0.01\\
56.28	0.01\\
56.29	0.01\\
56.3	0.01\\
56.31	0.01\\
56.32	0.01\\
56.33	0.01\\
56.34	0.01\\
56.35	0.01\\
56.36	0.01\\
56.37	0.01\\
56.38	0.01\\
56.39	0.01\\
56.4	0.01\\
56.41	0.01\\
56.42	0.01\\
56.43	0.01\\
56.44	0.01\\
56.45	0.01\\
56.46	0.01\\
56.47	0.01\\
56.48	0.01\\
56.49	0.01\\
56.5	0.01\\
56.51	0.01\\
56.52	0.01\\
56.53	0.01\\
56.54	0.01\\
56.55	0.01\\
56.56	0.01\\
56.57	0.01\\
56.58	0.01\\
56.59	0.01\\
56.6	0.01\\
56.61	0.01\\
56.62	0.01\\
56.63	0.01\\
56.64	0.01\\
56.65	0.01\\
56.66	0.01\\
56.67	0.01\\
56.68	0.01\\
56.69	0.01\\
56.7	0.01\\
56.71	0.01\\
56.72	0.01\\
56.73	0.01\\
56.74	0.01\\
56.75	0.01\\
56.76	0.01\\
56.77	0.01\\
56.78	0.01\\
56.79	0.01\\
56.8	0.01\\
56.81	0.01\\
56.82	0.01\\
56.83	0.01\\
56.84	0.01\\
56.85	0.01\\
56.86	0.01\\
56.87	0.01\\
56.88	0.01\\
56.89	0.01\\
56.9	0.01\\
56.91	0.01\\
56.92	0.01\\
56.93	0.01\\
56.94	0.01\\
56.95	0.01\\
56.96	0.01\\
56.97	0.01\\
56.98	0.01\\
56.99	0.01\\
57	0.01\\
57.01	0.01\\
57.02	0.01\\
57.03	0.01\\
57.04	0.01\\
57.05	0.01\\
57.06	0.01\\
57.07	0.01\\
57.08	0.01\\
57.09	0.01\\
57.1	0.01\\
57.11	0.01\\
57.12	0.01\\
57.13	0.01\\
57.14	0.01\\
57.15	0.01\\
57.16	0.01\\
57.17	0.01\\
57.18	0.01\\
57.19	0.01\\
57.2	0.01\\
57.21	0.01\\
57.22	0.01\\
57.23	0.01\\
57.24	0.01\\
57.25	0.01\\
57.26	0.01\\
57.27	0.01\\
57.28	0.01\\
57.29	0.01\\
57.3	0.01\\
57.31	0.01\\
57.32	0.01\\
57.33	0.01\\
57.34	0.01\\
57.35	0.01\\
57.36	0.01\\
57.37	0.01\\
57.38	0.01\\
57.39	0.01\\
57.4	0.01\\
57.41	0.01\\
57.42	0.01\\
57.43	0.01\\
57.44	0.01\\
57.45	0.01\\
57.46	0.01\\
57.47	0.01\\
57.48	0.01\\
57.49	0.01\\
57.5	0.01\\
57.51	0.01\\
57.52	0.01\\
57.53	0.01\\
57.54	0.01\\
57.55	0.01\\
57.56	0.01\\
57.57	0.01\\
57.58	0.01\\
57.59	0.01\\
57.6	0.01\\
57.61	0.01\\
57.62	0.01\\
57.63	0.01\\
57.64	0.01\\
57.65	0.01\\
57.66	0.01\\
57.67	0.01\\
57.68	0.01\\
57.69	0.01\\
57.7	0.01\\
57.71	0.01\\
57.72	0.01\\
57.73	0.01\\
57.74	0.01\\
57.75	0.01\\
57.76	0.01\\
57.77	0.01\\
57.78	0.01\\
57.79	0.01\\
57.8	0.01\\
57.81	0.01\\
57.82	0.01\\
57.83	0.01\\
57.84	0.01\\
57.85	0.01\\
57.86	0.01\\
57.87	0.01\\
57.88	0.01\\
57.89	0.01\\
57.9	0.01\\
57.91	0.01\\
57.92	0.01\\
57.93	0.01\\
57.94	0.01\\
57.95	0.01\\
57.96	0.01\\
57.97	0.01\\
57.98	0.01\\
57.99	0.01\\
58	0.01\\
58.01	0.01\\
58.02	0.01\\
58.03	0.01\\
58.04	0.01\\
58.05	0.01\\
58.06	0.01\\
58.07	0.01\\
58.08	0.01\\
58.09	0.01\\
58.1	0.01\\
58.11	0.01\\
58.12	0.01\\
58.13	0.01\\
58.14	0.01\\
58.15	0.01\\
58.16	0.01\\
58.17	0.01\\
58.18	0.01\\
58.19	0.01\\
58.2	0.01\\
58.21	0.01\\
58.22	0.01\\
58.23	0.01\\
58.24	0.01\\
58.25	0.01\\
58.26	0.01\\
58.27	0.01\\
58.28	0.01\\
58.29	0.01\\
58.3	0.01\\
58.31	0.01\\
58.32	0.01\\
58.33	0.01\\
58.34	0.01\\
58.35	0.01\\
58.36	0.01\\
58.37	0.01\\
58.38	0.01\\
58.39	0.01\\
58.4	0.01\\
58.41	0.01\\
58.42	0.01\\
58.43	0.01\\
58.44	0.01\\
58.45	0.01\\
58.46	0.01\\
58.47	0.01\\
58.48	0.01\\
58.49	0.01\\
58.5	0.01\\
58.51	0.01\\
58.52	0.01\\
58.53	0.01\\
58.54	0.01\\
58.55	0.01\\
58.56	0.01\\
58.57	0.01\\
58.58	0.01\\
58.59	0.01\\
58.6	0.01\\
58.61	0.01\\
58.62	0.01\\
58.63	0.01\\
58.64	0.01\\
58.65	0.01\\
58.66	0.01\\
58.67	0.01\\
58.68	0.01\\
58.69	0.01\\
58.7	0.01\\
58.71	0.01\\
58.72	0.01\\
58.73	0.01\\
58.74	0.01\\
58.75	0.01\\
58.76	0.01\\
58.77	0.01\\
58.78	0.01\\
58.79	0.01\\
58.8	0.01\\
58.81	0.01\\
58.82	0.01\\
58.83	0.01\\
58.84	0.01\\
58.85	0.01\\
58.86	0.01\\
58.87	0.01\\
58.88	0.01\\
58.89	0.01\\
58.9	0.01\\
58.91	0.01\\
58.92	0.01\\
58.93	0.01\\
58.94	0.01\\
58.95	0.01\\
58.96	0.01\\
58.97	0.01\\
58.98	0.01\\
58.99	0.01\\
59	0.01\\
59.01	0.01\\
59.02	0.01\\
59.03	0.01\\
59.04	0.01\\
59.05	0.01\\
59.06	0.01\\
59.07	0.01\\
59.08	0.01\\
59.09	0.01\\
59.1	0.01\\
59.11	0.01\\
59.12	0.01\\
59.13	0.01\\
59.14	0.01\\
59.15	0.01\\
59.16	0.01\\
59.17	0.01\\
59.18	0.01\\
59.19	0.01\\
59.2	0.01\\
59.21	0.01\\
59.22	0.01\\
59.23	0.01\\
59.24	0.01\\
59.25	0.01\\
59.26	0.01\\
59.27	0.01\\
59.28	0.01\\
59.29	0.01\\
59.3	0.01\\
59.31	0.01\\
59.32	0.01\\
59.33	0.01\\
59.34	0.01\\
59.35	0.01\\
59.36	0.01\\
59.37	0.01\\
59.38	0.01\\
59.39	0.01\\
59.4	0.01\\
59.41	0.01\\
59.42	0.01\\
59.43	0.01\\
59.44	0.01\\
59.45	0.01\\
59.46	0.01\\
59.47	0.01\\
59.48	0.01\\
59.49	0.01\\
59.5	0.01\\
59.51	0.01\\
59.52	0.01\\
59.53	0.01\\
59.54	0.01\\
59.55	0.01\\
59.56	0.01\\
59.57	0.01\\
59.58	0.01\\
59.59	0.01\\
59.6	0.01\\
59.61	0.01\\
59.62	0.01\\
59.63	0.01\\
59.64	0.01\\
59.65	0.01\\
59.66	0.01\\
59.67	0.01\\
59.68	0.01\\
59.69	0.01\\
59.7	0.01\\
59.71	0.01\\
59.72	0.01\\
59.73	0.01\\
59.74	0.01\\
59.75	0.01\\
59.76	0.01\\
59.77	0.01\\
59.78	0.01\\
59.79	0.01\\
59.8	0.01\\
59.81	0.01\\
59.82	0.01\\
59.83	0.01\\
59.84	0.01\\
59.85	0.01\\
59.86	0.01\\
59.87	0.01\\
59.88	0.01\\
59.89	0.01\\
59.9	0.01\\
59.91	0.01\\
59.92	0.01\\
59.93	0.01\\
59.94	0.01\\
59.95	0.01\\
59.96	0.01\\
59.97	0.01\\
59.98	0.01\\
59.99	0.01\\
60	0.01\\
60.01	0.01\\
60.02	0.01\\
60.03	0.01\\
60.04	0.01\\
60.05	0.01\\
60.06	0.01\\
60.07	0.01\\
60.08	0.01\\
60.09	0.01\\
60.1	0.01\\
60.11	0.01\\
60.12	0.01\\
60.13	0.01\\
60.14	0.01\\
60.15	0.01\\
60.16	0.01\\
60.17	0.01\\
60.18	0.01\\
60.19	0.01\\
60.2	0.01\\
60.21	0.01\\
60.22	0.01\\
60.23	0.01\\
60.24	0.01\\
60.25	0.01\\
60.26	0.01\\
60.27	0.01\\
60.28	0.01\\
60.29	0.01\\
60.3	0.01\\
60.31	0.01\\
60.32	0.01\\
60.33	0.01\\
60.34	0.01\\
60.35	0.01\\
60.36	0.01\\
60.37	0.01\\
60.38	0.01\\
60.39	0.01\\
60.4	0.01\\
60.41	0.01\\
60.42	0.01\\
60.43	0.01\\
60.44	0.01\\
60.45	0.01\\
60.46	0.01\\
60.47	0.01\\
60.48	0.01\\
60.49	0.01\\
60.5	0.01\\
60.51	0.01\\
60.52	0.01\\
60.53	0.01\\
60.54	0.01\\
60.55	0.01\\
60.56	0.01\\
60.57	0.01\\
60.58	0.01\\
60.59	0.01\\
60.6	0.01\\
60.61	0.01\\
60.62	0.01\\
60.63	0.01\\
60.64	0.01\\
60.65	0.01\\
60.66	0.01\\
60.67	0.01\\
60.68	0.01\\
60.69	0.01\\
60.7	0.01\\
60.71	0.01\\
60.72	0.01\\
60.73	0.01\\
60.74	0.01\\
60.75	0.01\\
60.76	0.01\\
60.77	0.01\\
60.78	0.01\\
60.79	0.01\\
60.8	0.01\\
60.81	0.01\\
60.82	0.01\\
60.83	0.01\\
60.84	0.01\\
60.85	0.01\\
60.86	0.01\\
60.87	0.01\\
60.88	0.01\\
60.89	0.01\\
60.9	0.01\\
60.91	0.01\\
60.92	0.01\\
60.93	0.01\\
60.94	0.01\\
60.95	0.01\\
60.96	0.01\\
60.97	0.01\\
60.98	0.01\\
60.99	0.01\\
61	0.01\\
61.01	0.01\\
61.02	0.01\\
61.03	0.01\\
61.04	0.01\\
61.05	0.01\\
61.06	0.01\\
61.07	0.01\\
61.08	0.01\\
61.09	0.01\\
61.1	0.01\\
61.11	0.01\\
61.12	0.01\\
61.13	0.01\\
61.14	0.01\\
61.15	0.01\\
61.16	0.01\\
61.17	0.01\\
61.18	0.01\\
61.19	0.01\\
61.2	0.01\\
61.21	0.01\\
61.22	0.01\\
61.23	0.01\\
61.24	0.01\\
61.25	0.01\\
61.26	0.01\\
61.27	0.01\\
61.28	0.01\\
61.29	0.01\\
61.3	0.01\\
61.31	0.01\\
61.32	0.01\\
61.33	0.01\\
61.34	0.01\\
61.35	0.01\\
61.36	0.01\\
61.37	0.01\\
61.38	0.01\\
61.39	0.01\\
61.4	0.01\\
61.41	0.01\\
61.42	0.01\\
61.43	0.01\\
61.44	0.01\\
61.45	0.01\\
61.46	0.01\\
61.47	0.01\\
61.48	0.01\\
61.49	0.01\\
61.5	0.01\\
61.51	0.01\\
61.52	0.01\\
61.53	0.01\\
61.54	0.01\\
61.55	0.01\\
61.56	0.01\\
61.57	0.01\\
61.58	0.01\\
61.59	0.01\\
61.6	0.01\\
61.61	0.01\\
61.62	0.01\\
61.63	0.01\\
61.64	0.01\\
61.65	0.01\\
61.66	0.01\\
61.67	0.01\\
61.68	0.01\\
61.69	0.01\\
61.7	0.01\\
61.71	0.01\\
61.72	0.01\\
61.73	0.01\\
61.74	0.01\\
61.75	0.01\\
61.76	0.01\\
61.77	0.01\\
61.78	0.01\\
61.79	0.01\\
61.8	0.01\\
61.81	0.01\\
61.82	0.01\\
61.83	0.01\\
61.84	0.01\\
61.85	0.01\\
61.86	0.01\\
61.87	0.01\\
61.88	0.01\\
61.89	0.01\\
61.9	0.01\\
61.91	0.01\\
61.92	0.01\\
61.93	0.01\\
61.94	0.01\\
61.95	0.01\\
61.96	0.01\\
61.97	0.01\\
61.98	0.01\\
61.99	0.01\\
62	0.01\\
62.01	0.01\\
62.02	0.01\\
62.03	0.01\\
62.04	0.01\\
62.05	0.01\\
62.06	0.01\\
62.07	0.01\\
62.08	0.01\\
62.09	0.01\\
62.1	0.01\\
62.11	0.01\\
62.12	0.01\\
62.13	0.01\\
62.14	0.01\\
62.15	0.01\\
62.16	0.01\\
62.17	0.01\\
62.18	0.01\\
62.19	0.01\\
62.2	0.01\\
62.21	0.01\\
62.22	0.01\\
62.23	0.01\\
62.24	0.01\\
62.25	0.01\\
62.26	0.01\\
62.27	0.01\\
62.28	0.01\\
62.29	0.01\\
62.3	0.01\\
62.31	0.01\\
62.32	0.01\\
62.33	0.01\\
62.34	0.01\\
62.35	0.01\\
62.36	0.01\\
62.37	0.01\\
62.38	0.01\\
62.39	0.01\\
62.4	0.01\\
62.41	0.01\\
62.42	0.01\\
62.43	0.01\\
62.44	0.01\\
62.45	0.01\\
62.46	0.01\\
62.47	0.01\\
62.48	0.01\\
62.49	0.01\\
62.5	0.01\\
62.51	0.01\\
62.52	0.01\\
62.53	0.01\\
62.54	0.01\\
62.55	0.01\\
62.56	0.01\\
62.57	0.01\\
62.58	0.01\\
62.59	0.01\\
62.6	0.01\\
62.61	0.01\\
62.62	0.01\\
62.63	0.01\\
62.64	0.01\\
62.65	0.01\\
62.66	0.01\\
62.67	0.01\\
62.68	0.01\\
62.69	0.01\\
62.7	0.01\\
62.71	0.01\\
62.72	0.01\\
62.73	0.01\\
62.74	0.01\\
62.75	0.01\\
62.76	0.01\\
62.77	0.01\\
62.78	0.01\\
62.79	0.01\\
62.8	0.01\\
62.81	0.01\\
62.82	0.01\\
62.83	0.01\\
62.84	0.01\\
62.85	0.01\\
62.86	0.01\\
62.87	0.01\\
62.88	0.01\\
62.89	0.01\\
62.9	0.01\\
62.91	0.01\\
62.92	0.01\\
62.93	0.01\\
62.94	0.01\\
62.95	0.01\\
62.96	0.01\\
62.97	0.01\\
62.98	0.01\\
62.99	0.01\\
63	0.01\\
63.01	0.01\\
63.02	0.01\\
63.03	0.01\\
63.04	0.01\\
63.05	0.01\\
63.06	0.01\\
63.07	0.01\\
63.08	0.01\\
63.09	0.01\\
63.1	0.01\\
63.11	0.01\\
63.12	0.01\\
63.13	0.01\\
63.14	0.01\\
63.15	0.01\\
63.16	0.01\\
63.17	0.01\\
63.18	0.01\\
63.19	0.01\\
63.2	0.01\\
63.21	0.01\\
63.22	0.01\\
63.23	0.01\\
63.24	0.01\\
63.25	0.01\\
63.26	0.01\\
63.27	0.01\\
63.28	0.01\\
63.29	0.01\\
63.3	0.01\\
63.31	0.01\\
63.32	0.01\\
63.33	0.01\\
63.34	0.01\\
63.35	0.01\\
63.36	0.01\\
63.37	0.01\\
63.38	0.01\\
63.39	0.01\\
63.4	0.01\\
63.41	0.01\\
63.42	0.01\\
63.43	0.01\\
63.44	0.01\\
63.45	0.01\\
63.46	0.01\\
63.47	0.01\\
63.48	0.01\\
63.49	0.01\\
63.5	0.01\\
63.51	0.01\\
63.52	0.01\\
63.53	0.01\\
63.54	0.01\\
63.55	0.01\\
63.56	0.01\\
63.57	0.01\\
63.58	0.01\\
63.59	0.01\\
63.6	0.01\\
63.61	0.01\\
63.62	0.01\\
63.63	0.01\\
63.64	0.01\\
63.65	0.01\\
63.66	0.01\\
63.67	0.01\\
63.68	0.01\\
63.69	0.01\\
63.7	0.01\\
63.71	0.01\\
63.72	0.01\\
63.73	0.01\\
63.74	0.01\\
63.75	0.01\\
63.76	0.01\\
63.77	0.01\\
63.78	0.01\\
63.79	0.01\\
63.8	0.01\\
63.81	0.01\\
63.82	0.01\\
63.83	0.01\\
63.84	0.01\\
63.85	0.01\\
63.86	0.01\\
63.87	0.01\\
63.88	0.01\\
63.89	0.01\\
63.9	0.01\\
63.91	0.01\\
63.92	0.01\\
63.93	0.01\\
63.94	0.01\\
63.95	0.01\\
63.96	0.01\\
63.97	0.01\\
63.98	0.01\\
63.99	0.01\\
64	0.01\\
64.01	0.01\\
64.02	0.01\\
64.03	0.01\\
64.04	0.01\\
64.05	0.01\\
64.06	0.01\\
64.07	0.01\\
64.08	0.01\\
64.09	0.01\\
64.1	0.01\\
64.11	0.01\\
64.12	0.01\\
64.13	0.01\\
64.14	0.01\\
64.15	0.01\\
64.16	0.01\\
64.17	0.01\\
64.18	0.01\\
64.19	0.01\\
64.2	0.01\\
64.21	0.01\\
64.22	0.01\\
64.23	0.01\\
64.24	0.01\\
64.25	0.01\\
64.26	0.01\\
64.27	0.01\\
64.28	0.01\\
64.29	0.01\\
64.3	0.01\\
64.31	0.01\\
64.32	0.01\\
64.33	0.01\\
64.34	0.01\\
64.35	0.01\\
64.36	0.01\\
64.37	0.01\\
64.38	0.01\\
64.39	0.01\\
64.4	0.01\\
64.41	0.01\\
64.42	0.01\\
64.43	0.01\\
64.44	0.01\\
64.45	0.01\\
64.46	0.01\\
64.47	0.01\\
64.48	0.01\\
64.49	0.01\\
64.5	0.01\\
64.51	0.01\\
64.52	0.01\\
64.53	0.01\\
64.54	0.01\\
64.55	0.01\\
64.56	0.01\\
64.57	0.01\\
64.58	0.01\\
64.59	0.01\\
64.6	0.01\\
64.61	0.01\\
64.62	0.01\\
64.63	0.01\\
64.64	0.01\\
64.65	0.01\\
64.66	0.01\\
64.67	0.01\\
64.68	0.01\\
64.69	0.01\\
64.7	0.01\\
64.71	0.01\\
64.72	0.01\\
64.73	0.01\\
64.74	0.01\\
64.75	0.01\\
64.76	0.01\\
64.77	0.01\\
64.78	0.01\\
64.79	0.01\\
64.8	0.01\\
64.81	0.01\\
64.82	0.01\\
64.83	0.01\\
64.84	0.01\\
64.85	0.01\\
64.86	0.01\\
64.87	0.01\\
64.88	0.01\\
64.89	0.01\\
64.9	0.01\\
64.91	0.01\\
64.92	0.01\\
64.93	0.01\\
64.94	0.01\\
64.95	0.01\\
64.96	0.01\\
64.97	0.01\\
64.98	0.01\\
64.99	0.01\\
65	0.01\\
65.01	0.01\\
65.02	0.01\\
65.03	0.01\\
65.04	0.01\\
65.05	0.01\\
65.06	0.01\\
65.07	0.01\\
65.08	0.01\\
65.09	0.01\\
65.1	0.01\\
65.11	0.01\\
65.12	0.01\\
65.13	0.01\\
65.14	0.01\\
65.15	0.01\\
65.16	0.01\\
65.17	0.01\\
65.18	0.01\\
65.19	0.01\\
65.2	0.01\\
65.21	0.01\\
65.22	0.01\\
65.23	0.01\\
65.24	0.01\\
65.25	0.01\\
65.26	0.01\\
65.27	0.01\\
65.28	0.01\\
65.29	0.01\\
65.3	0.01\\
65.31	0.01\\
65.32	0.01\\
65.33	0.01\\
65.34	0.01\\
65.35	0.01\\
65.36	0.01\\
65.37	0.01\\
65.38	0.01\\
65.39	0.01\\
65.4	0.01\\
65.41	0.01\\
65.42	0.01\\
65.43	0.01\\
65.44	0.01\\
65.45	0.01\\
65.46	0.01\\
65.47	0.01\\
65.48	0.01\\
65.49	0.01\\
65.5	0.01\\
65.51	0.01\\
65.52	0.01\\
65.53	0.01\\
65.54	0.01\\
65.55	0.01\\
65.56	0.01\\
65.57	0.01\\
65.58	0.01\\
65.59	0.01\\
65.6	0.01\\
65.61	0.01\\
65.62	0.01\\
65.63	0.01\\
65.64	0.01\\
65.65	0.01\\
65.66	0.01\\
65.67	0.01\\
65.68	0.01\\
65.69	0.01\\
65.7	0.01\\
65.71	0.01\\
65.72	0.01\\
65.73	0.01\\
65.74	0.01\\
65.75	0.01\\
65.76	0.01\\
65.77	0.01\\
65.78	0.01\\
65.79	0.01\\
65.8	0.01\\
65.81	0.01\\
65.82	0.01\\
65.83	0.01\\
65.84	0.01\\
65.85	0.01\\
65.86	0.01\\
65.87	0.01\\
65.88	0.01\\
65.89	0.01\\
65.9	0.01\\
65.91	0.01\\
65.92	0.01\\
65.93	0.01\\
65.94	0.01\\
65.95	0.01\\
65.96	0.01\\
65.97	0.01\\
65.98	0.01\\
65.99	0.01\\
66	0.01\\
66.01	0.01\\
66.02	0.01\\
66.03	0.01\\
66.04	0.01\\
66.05	0.01\\
66.06	0.01\\
66.07	0.01\\
66.08	0.01\\
66.09	0.01\\
66.1	0.01\\
66.11	0.01\\
66.12	0.01\\
66.13	0.01\\
66.14	0.01\\
66.15	0.01\\
66.16	0.01\\
66.17	0.01\\
66.18	0.01\\
66.19	0.01\\
66.2	0.01\\
66.21	0.01\\
66.22	0.01\\
66.23	0.01\\
66.24	0.01\\
66.25	0.01\\
66.26	0.01\\
66.27	0.01\\
66.28	0.01\\
66.29	0.01\\
66.3	0.01\\
66.31	0.01\\
66.32	0.01\\
66.33	0.01\\
66.34	0.01\\
66.35	0.01\\
66.36	0.01\\
66.37	0.01\\
66.38	0.01\\
66.39	0.01\\
66.4	0.01\\
66.41	0.01\\
66.42	0.01\\
66.43	0.01\\
66.44	0.01\\
66.45	0.01\\
66.46	0.01\\
66.47	0.01\\
66.48	0.01\\
66.49	0.01\\
66.5	0.01\\
66.51	0.01\\
66.52	0.01\\
66.53	0.01\\
66.54	0.01\\
66.55	0.01\\
66.56	0.01\\
66.57	0.01\\
66.58	0.01\\
66.59	0.01\\
66.6	0.01\\
66.61	0.01\\
66.62	0.01\\
66.63	0.01\\
66.64	0.01\\
66.65	0.01\\
66.66	0.01\\
66.67	0.01\\
66.68	0.01\\
66.69	0.01\\
66.7	0.01\\
66.71	0.01\\
66.72	0.01\\
66.73	0.01\\
66.74	0.01\\
66.75	0.01\\
66.76	0.01\\
66.77	0.01\\
66.78	0.01\\
66.79	0.01\\
66.8	0.01\\
66.81	0.01\\
66.82	0.01\\
66.83	0.01\\
66.84	0.01\\
66.85	0.01\\
66.86	0.01\\
66.87	0.01\\
66.88	0.01\\
66.89	0.01\\
66.9	0.01\\
66.91	0.01\\
66.92	0.01\\
66.93	0.01\\
66.94	0.01\\
66.95	0.01\\
66.96	0.01\\
66.97	0.01\\
66.98	0.01\\
66.99	0.01\\
67	0.01\\
67.01	0.01\\
67.02	0.01\\
67.03	0.01\\
67.04	0.01\\
67.05	0.01\\
67.06	0.01\\
67.07	0.01\\
67.08	0.01\\
67.09	0.01\\
67.1	0.01\\
67.11	0.01\\
67.12	0.01\\
67.13	0.01\\
67.14	0.01\\
67.15	0.01\\
67.16	0.01\\
67.17	0.01\\
67.18	0.01\\
67.19	0.01\\
67.2	0.01\\
67.21	0.01\\
67.22	0.01\\
67.23	0.01\\
67.24	0.01\\
67.25	0.01\\
67.26	0.01\\
67.27	0.01\\
67.28	0.01\\
67.29	0.01\\
67.3	0.01\\
67.31	0.01\\
67.32	0.01\\
67.33	0.01\\
67.34	0.01\\
67.35	0.01\\
67.36	0.01\\
67.37	0.01\\
67.38	0.01\\
67.39	0.01\\
67.4	0.01\\
67.41	0.01\\
67.42	0.01\\
67.43	0.01\\
67.44	0.01\\
67.45	0.01\\
67.46	0.01\\
67.47	0.01\\
67.48	0.01\\
67.49	0.01\\
67.5	0.01\\
67.51	0.01\\
67.52	0.01\\
67.53	0.01\\
67.54	0.01\\
67.55	0.01\\
67.56	0.01\\
67.57	0.01\\
67.58	0.01\\
67.59	0.01\\
67.6	0.01\\
67.61	0.01\\
67.62	0.01\\
67.63	0.01\\
67.64	0.01\\
67.65	0.01\\
67.66	0.01\\
67.67	0.01\\
67.68	0.01\\
67.69	0.01\\
67.7	0.01\\
67.71	0.01\\
67.72	0.01\\
67.73	0.01\\
67.74	0.01\\
67.75	0.01\\
67.76	0.01\\
67.77	0.01\\
67.78	0.01\\
67.79	0.01\\
67.8	0.01\\
67.81	0.01\\
67.82	0.01\\
67.83	0.01\\
67.84	0.01\\
67.85	0.01\\
67.86	0.01\\
67.87	0.01\\
67.88	0.01\\
67.89	0.01\\
67.9	0.01\\
67.91	0.01\\
67.92	0.01\\
67.93	0.01\\
67.94	0.01\\
67.95	0.01\\
67.96	0.01\\
67.97	0.01\\
67.98	0.01\\
67.99	0.01\\
68	0.01\\
68.01	0.01\\
68.02	0.01\\
68.03	0.01\\
68.04	0.01\\
68.05	0.01\\
68.06	0.01\\
68.07	0.01\\
68.08	0.01\\
68.09	0.01\\
68.1	0.01\\
68.11	0.01\\
68.12	0.01\\
68.13	0.01\\
68.14	0.01\\
68.15	0.01\\
68.16	0.01\\
68.17	0.01\\
68.18	0.01\\
68.19	0.01\\
68.2	0.01\\
68.21	0.01\\
68.22	0.01\\
68.23	0.01\\
68.24	0.01\\
68.25	0.01\\
68.26	0.01\\
68.27	0.01\\
68.28	0.01\\
68.29	0.01\\
68.3	0.01\\
68.31	0.01\\
68.32	0.01\\
68.33	0.01\\
68.34	0.01\\
68.35	0.01\\
68.36	0.01\\
68.37	0.01\\
68.38	0.01\\
68.39	0.01\\
68.4	0.01\\
68.41	0.01\\
68.42	0.01\\
68.43	0.01\\
68.44	0.01\\
68.45	0.01\\
68.46	0.01\\
68.47	0.01\\
68.48	0.01\\
68.49	0.01\\
68.5	0.01\\
68.51	0.01\\
68.52	0.01\\
68.53	0.01\\
68.54	0.01\\
68.55	0.01\\
68.56	0.01\\
68.57	0.01\\
68.58	0.01\\
68.59	0.01\\
68.6	0.01\\
68.61	0.01\\
68.62	0.01\\
68.63	0.01\\
68.64	0.01\\
68.65	0.01\\
68.66	0.01\\
68.67	0.01\\
68.68	0.01\\
68.69	0.01\\
68.7	0.01\\
68.71	0.01\\
68.72	0.01\\
68.73	0.01\\
68.74	0.01\\
68.75	0.01\\
68.76	0.01\\
68.77	0.01\\
68.78	0.01\\
68.79	0.01\\
68.8	0.01\\
68.81	0.01\\
68.82	0.01\\
68.83	0.01\\
68.84	0.01\\
68.85	0.01\\
68.86	0.01\\
68.87	0.01\\
68.88	0.01\\
68.89	0.01\\
68.9	0.01\\
68.91	0.01\\
68.92	0.01\\
68.93	0.01\\
68.94	0.01\\
68.95	0.01\\
68.96	0.01\\
68.97	0.01\\
68.98	0.01\\
68.99	0.01\\
69	0.01\\
69.01	0.01\\
69.02	0.01\\
69.03	0.01\\
69.04	0.01\\
69.05	0.01\\
69.06	0.01\\
69.07	0.01\\
69.08	0.01\\
69.09	0.01\\
69.1	0.01\\
69.11	0.01\\
69.12	0.01\\
69.13	0.01\\
69.14	0.01\\
69.15	0.01\\
69.16	0.01\\
69.17	0.01\\
69.18	0.01\\
69.19	0.01\\
69.2	0.01\\
69.21	0.01\\
69.22	0.01\\
69.23	0.01\\
69.24	0.01\\
69.25	0.01\\
69.26	0.01\\
69.27	0.01\\
69.28	0.01\\
69.29	0.01\\
69.3	0.01\\
69.31	0.01\\
69.32	0.01\\
69.33	0.01\\
69.34	0.01\\
69.35	0.01\\
69.36	0.01\\
69.37	0.01\\
69.38	0.01\\
69.39	0.01\\
69.4	0.01\\
69.41	0.01\\
69.42	0.01\\
69.43	0.01\\
69.44	0.01\\
69.45	0.01\\
69.46	0.01\\
69.47	0.01\\
69.48	0.01\\
69.49	0.01\\
69.5	0.01\\
69.51	0.01\\
69.52	0.01\\
69.53	0.01\\
69.54	0.01\\
69.55	0.01\\
69.56	0.01\\
69.57	0.01\\
69.58	0.01\\
69.59	0.01\\
69.6	0.01\\
69.61	0.01\\
69.62	0.01\\
69.63	0.01\\
69.64	0.01\\
69.65	0.01\\
69.66	0.01\\
69.67	0.01\\
69.68	0.01\\
69.69	0.01\\
69.7	0.01\\
69.71	0.01\\
69.72	0.01\\
69.73	0.01\\
69.74	0.01\\
69.75	0.01\\
69.76	0.01\\
69.77	0.01\\
69.78	0.01\\
69.79	0.01\\
69.8	0.01\\
69.81	0.01\\
69.82	0.01\\
69.83	0.01\\
69.84	0.01\\
69.85	0.01\\
69.86	0.01\\
69.87	0.01\\
69.88	0.01\\
69.89	0.01\\
69.9	0.01\\
69.91	0.01\\
69.92	0.01\\
69.93	0.01\\
69.94	0.01\\
69.95	0.01\\
69.96	0.01\\
69.97	0.01\\
69.98	0.01\\
69.99	0.01\\
70	0.01\\
70.01	0.01\\
70.02	0.01\\
70.03	0.01\\
70.04	0.01\\
70.05	0.01\\
70.06	0.01\\
70.07	0.01\\
70.08	0.01\\
70.09	0.01\\
70.1	0.01\\
70.11	0.01\\
70.12	0.01\\
70.13	0.01\\
70.14	0.01\\
70.15	0.01\\
70.16	0.01\\
70.17	0.01\\
70.18	0.01\\
70.19	0.01\\
70.2	0.01\\
70.21	0.01\\
70.22	0.01\\
70.23	0.01\\
70.24	0.01\\
70.25	0.01\\
70.26	0.01\\
70.27	0.01\\
70.28	0.01\\
70.29	0.01\\
70.3	0.01\\
70.31	0.01\\
70.32	0.01\\
70.33	0.01\\
70.34	0.01\\
70.35	0.01\\
70.36	0.01\\
70.37	0.01\\
70.38	0.01\\
70.39	0.01\\
70.4	0.01\\
70.41	0.01\\
70.42	0.01\\
70.43	0.01\\
70.44	0.01\\
70.45	0.01\\
70.46	0.01\\
70.47	0.01\\
70.48	0.01\\
70.49	0.01\\
70.5	0.01\\
70.51	0.01\\
70.52	0.01\\
70.53	0.01\\
70.54	0.01\\
70.55	0.01\\
70.56	0.01\\
70.57	0.01\\
70.58	0.01\\
70.59	0.01\\
70.6	0.01\\
70.61	0.01\\
70.62	0.01\\
70.63	0.01\\
70.64	0.01\\
70.65	0.01\\
70.66	0.01\\
70.67	0.01\\
70.68	0.01\\
70.69	0.01\\
70.7	0.01\\
70.71	0.01\\
70.72	0.01\\
70.73	0.01\\
70.74	0.01\\
70.75	0.01\\
70.76	0.01\\
70.77	0.01\\
70.78	0.01\\
70.79	0.01\\
70.8	0.01\\
70.81	0.01\\
70.82	0.01\\
70.83	0.01\\
70.84	0.01\\
70.85	0.01\\
70.86	0.01\\
70.87	0.01\\
70.88	0.01\\
70.89	0.01\\
70.9	0.01\\
70.91	0.01\\
70.92	0.01\\
70.93	0.01\\
70.94	0.01\\
70.95	0.01\\
70.96	0.01\\
70.97	0.01\\
70.98	0.01\\
70.99	0.01\\
71	0.01\\
71.01	0.01\\
71.02	0.01\\
71.03	0.01\\
71.04	0.01\\
71.05	0.01\\
71.06	0.01\\
71.07	0.01\\
71.08	0.01\\
71.09	0.01\\
71.1	0.01\\
71.11	0.01\\
71.12	0.01\\
71.13	0.01\\
71.14	0.01\\
71.15	0.01\\
71.16	0.01\\
71.17	0.01\\
71.18	0.01\\
71.19	0.01\\
71.2	0.01\\
71.21	0.01\\
71.22	0.01\\
71.23	0.01\\
71.24	0.01\\
71.25	0.01\\
71.26	0.01\\
71.27	0.01\\
71.28	0.01\\
71.29	0.01\\
71.3	0.01\\
71.31	0.01\\
71.32	0.01\\
71.33	0.01\\
71.34	0.01\\
71.35	0.01\\
71.36	0.01\\
71.37	0.01\\
71.38	0.01\\
71.39	0.01\\
71.4	0.01\\
71.41	0.01\\
71.42	0.01\\
71.43	0.01\\
71.44	0.01\\
71.45	0.01\\
71.46	0.01\\
71.47	0.01\\
71.48	0.01\\
71.49	0.01\\
71.5	0.01\\
71.51	0.01\\
71.52	0.01\\
71.53	0.01\\
71.54	0.01\\
71.55	0.01\\
71.56	0.01\\
71.57	0.01\\
71.58	0.01\\
71.59	0.01\\
71.6	0.01\\
71.61	0.01\\
71.62	0.01\\
71.63	0.01\\
71.64	0.01\\
71.65	0.01\\
71.66	0.01\\
71.67	0.01\\
71.68	0.01\\
71.69	0.01\\
71.7	0.01\\
71.71	0.01\\
71.72	0.01\\
71.73	0.01\\
71.74	0.01\\
71.75	0.01\\
71.76	0.01\\
71.77	0.01\\
71.78	0.01\\
71.79	0.01\\
71.8	0.01\\
71.81	0.01\\
71.82	0.01\\
71.83	0.01\\
71.84	0.01\\
71.85	0.01\\
71.86	0.01\\
71.87	0.01\\
71.88	0.01\\
71.89	0.01\\
71.9	0.01\\
71.91	0.01\\
71.92	0.01\\
71.93	0.01\\
71.94	0.01\\
71.95	0.01\\
71.96	0.01\\
71.97	0.01\\
71.98	0.01\\
71.99	0.01\\
72	0.01\\
72.01	0.01\\
72.02	0.01\\
72.03	0.01\\
72.04	0.01\\
72.05	0.01\\
72.06	0.01\\
72.07	0.01\\
72.08	0.01\\
72.09	0.01\\
72.1	0.01\\
72.11	0.01\\
72.12	0.01\\
72.13	0.01\\
72.14	0.01\\
72.15	0.01\\
72.16	0.01\\
72.17	0.01\\
72.18	0.01\\
72.19	0.01\\
72.2	0.01\\
72.21	0.01\\
72.22	0.01\\
72.23	0.01\\
72.24	0.01\\
72.25	0.01\\
72.26	0.01\\
72.27	0.01\\
72.28	0.01\\
72.29	0.01\\
72.3	0.01\\
72.31	0.01\\
72.32	0.01\\
72.33	0.01\\
72.34	0.01\\
72.35	0.01\\
72.36	0.01\\
72.37	0.01\\
72.38	0.01\\
72.39	0.01\\
72.4	0.01\\
72.41	0.01\\
72.42	0.01\\
72.43	0.01\\
72.44	0.01\\
72.45	0.01\\
72.46	0.01\\
72.47	0.01\\
72.48	0.01\\
72.49	0.01\\
72.5	0.01\\
72.51	0.01\\
72.52	0.01\\
72.53	0.01\\
72.54	0.01\\
72.55	0.01\\
72.56	0.01\\
72.57	0.01\\
72.58	0.01\\
72.59	0.01\\
72.6	0.01\\
72.61	0.01\\
72.62	0.01\\
72.63	0.01\\
72.64	0.01\\
72.65	0.01\\
72.66	0.01\\
72.67	0.01\\
72.68	0.01\\
72.69	0.01\\
72.7	0.01\\
72.71	0.01\\
72.72	0.01\\
72.73	0.01\\
72.74	0.01\\
72.75	0.01\\
72.76	0.01\\
72.77	0.01\\
72.78	0.01\\
72.79	0.01\\
72.8	0.01\\
72.81	0.01\\
72.82	0.01\\
72.83	0.01\\
72.84	0.01\\
72.85	0.01\\
72.86	0.01\\
72.87	0.01\\
72.88	0.01\\
72.89	0.01\\
72.9	0.01\\
72.91	0.01\\
72.92	0.01\\
72.93	0.01\\
72.94	0.01\\
72.95	0.01\\
72.96	0.01\\
72.97	0.01\\
72.98	0.01\\
72.99	0.01\\
73	0.01\\
73.01	0.01\\
73.02	0.01\\
73.03	0.01\\
73.04	0.01\\
73.05	0.01\\
73.06	0.01\\
73.07	0.01\\
73.08	0.01\\
73.09	0.01\\
73.1	0.01\\
73.11	0.01\\
73.12	0.01\\
73.13	0.01\\
73.14	0.01\\
73.15	0.01\\
73.16	0.01\\
73.17	0.01\\
73.18	0.01\\
73.19	0.01\\
73.2	0.01\\
73.21	0.01\\
73.22	0.01\\
73.23	0.01\\
73.24	0.01\\
73.25	0.01\\
73.26	0.01\\
73.27	0.01\\
73.28	0.01\\
73.29	0.01\\
73.3	0.01\\
73.31	0.01\\
73.32	0.01\\
73.33	0.01\\
73.34	0.01\\
73.35	0.01\\
73.36	0.01\\
73.37	0.01\\
73.38	0.01\\
73.39	0.01\\
73.4	0.01\\
73.41	0.01\\
73.42	0.01\\
73.43	0.01\\
73.44	0.01\\
73.45	0.01\\
73.46	0.01\\
73.47	0.01\\
73.48	0.01\\
73.49	0.01\\
73.5	0.01\\
73.51	0.01\\
73.52	0.01\\
73.53	0.01\\
73.54	0.01\\
73.55	0.01\\
73.56	0.01\\
73.57	0.01\\
73.58	0.01\\
73.59	0.01\\
73.6	0.01\\
73.61	0.01\\
73.62	0.01\\
73.63	0.01\\
73.64	0.01\\
73.65	0.01\\
73.66	0.01\\
73.67	0.01\\
73.68	0.01\\
73.69	0.01\\
73.7	0.01\\
73.71	0.01\\
73.72	0.01\\
73.73	0.01\\
73.74	0.01\\
73.75	0.01\\
73.76	0.01\\
73.77	0.01\\
73.78	0.01\\
73.79	0.01\\
73.8	0.01\\
73.81	0.01\\
73.82	0.01\\
73.83	0.01\\
73.84	0.01\\
73.85	0.01\\
73.86	0.01\\
73.87	0.01\\
73.88	0.01\\
73.89	0.01\\
73.9	0.01\\
73.91	0.01\\
73.92	0.01\\
73.93	0.01\\
73.94	0.01\\
73.95	0.01\\
73.96	0.01\\
73.97	0.01\\
73.98	0.01\\
73.99	0.01\\
74	0.01\\
74.01	0.01\\
74.02	0.01\\
74.03	0.01\\
74.04	0.01\\
74.05	0.01\\
74.06	0.01\\
74.07	0.01\\
74.08	0.01\\
74.09	0.01\\
74.1	0.01\\
74.11	0.01\\
74.12	0.01\\
74.13	0.01\\
74.14	0.01\\
74.15	0.01\\
74.16	0.01\\
74.17	0.01\\
74.18	0.01\\
74.19	0.01\\
74.2	0.01\\
74.21	0.01\\
74.22	0.01\\
74.23	0.01\\
74.24	0.01\\
74.25	0.01\\
74.26	0.01\\
74.27	0.01\\
74.28	0.01\\
74.29	0.01\\
74.3	0.01\\
74.31	0.01\\
74.32	0.01\\
74.33	0.01\\
74.34	0.01\\
74.35	0.01\\
74.36	0.01\\
74.37	0.01\\
74.38	0.01\\
74.39	0.01\\
74.4	0.01\\
74.41	0.01\\
74.42	0.01\\
74.43	0.01\\
74.44	0.01\\
74.45	0.01\\
74.46	0.01\\
74.47	0.01\\
74.48	0.01\\
74.49	0.01\\
74.5	0.01\\
74.51	0.01\\
74.52	0.01\\
74.53	0.01\\
74.54	0.01\\
74.55	0.01\\
74.56	0.01\\
74.57	0.01\\
74.58	0.01\\
74.59	0.01\\
74.6	0.01\\
74.61	0.01\\
74.62	0.01\\
74.63	0.01\\
74.64	0.01\\
74.65	0.01\\
74.66	0.01\\
74.67	0.01\\
74.68	0.01\\
74.69	0.01\\
74.7	0.01\\
74.71	0.01\\
74.72	0.01\\
74.73	0.01\\
74.74	0.01\\
74.75	0.01\\
74.76	0.01\\
74.77	0.01\\
74.78	0.01\\
74.79	0.01\\
74.8	0.01\\
74.81	0.01\\
74.82	0.01\\
74.83	0.01\\
74.84	0.01\\
74.85	0.01\\
74.86	0.01\\
74.87	0.01\\
74.88	0.01\\
74.89	0.01\\
74.9	0.01\\
74.91	0.01\\
74.92	0.01\\
74.93	0.01\\
74.94	0.01\\
74.95	0.01\\
74.96	0.01\\
74.97	0.01\\
74.98	0.01\\
74.99	0.01\\
75	0.01\\
75.01	0.01\\
75.02	0.01\\
75.03	0.01\\
75.04	0.01\\
75.05	0.01\\
75.06	0.01\\
75.07	0.01\\
75.08	0.01\\
75.09	0.01\\
75.1	0.01\\
75.11	0.01\\
75.12	0.01\\
75.13	0.01\\
75.14	0.01\\
75.15	0.01\\
75.16	0.01\\
75.17	0.01\\
75.18	0.01\\
75.19	0.01\\
75.2	0.01\\
75.21	0.01\\
75.22	0.01\\
75.23	0.01\\
75.24	0.01\\
75.25	0.01\\
75.26	0.01\\
75.27	0.01\\
75.28	0.01\\
75.29	0.01\\
75.3	0.01\\
75.31	0.01\\
75.32	0.01\\
75.33	0.01\\
75.34	0.01\\
75.35	0.01\\
75.36	0.01\\
75.37	0.01\\
75.38	0.01\\
75.39	0.01\\
75.4	0.01\\
75.41	0.01\\
75.42	0.01\\
75.43	0.01\\
75.44	0.01\\
75.45	0.01\\
75.46	0.01\\
75.47	0.01\\
75.48	0.01\\
75.49	0.01\\
75.5	0.01\\
75.51	0.01\\
75.52	0.01\\
75.53	0.01\\
75.54	0.01\\
75.55	0.01\\
75.56	0.01\\
75.57	0.01\\
75.58	0.01\\
75.59	0.01\\
75.6	0.01\\
75.61	0.01\\
75.62	0.01\\
75.63	0.01\\
75.64	0.01\\
75.65	0.01\\
75.66	0.01\\
75.67	0.01\\
75.68	0.01\\
75.69	0.01\\
75.7	0.01\\
75.71	0.01\\
75.72	0.01\\
75.73	0.01\\
75.74	0.01\\
75.75	0.01\\
75.76	0.01\\
75.77	0.01\\
75.78	0.01\\
75.79	0.01\\
75.8	0.01\\
75.81	0.01\\
75.82	0.01\\
75.83	0.01\\
75.84	0.01\\
75.85	0.01\\
75.86	0.01\\
75.87	0.01\\
75.88	0.01\\
75.89	0.01\\
75.9	0.01\\
75.91	0.01\\
75.92	0.01\\
75.93	0.01\\
75.94	0.01\\
75.95	0.01\\
75.96	0.01\\
75.97	0.01\\
75.98	0.01\\
75.99	0.01\\
76	0.01\\
76.01	0.01\\
76.02	0.01\\
76.03	0.01\\
76.04	0.01\\
76.05	0.01\\
76.06	0.01\\
76.07	0.01\\
76.08	0.01\\
76.09	0.01\\
76.1	0.01\\
76.11	0.01\\
76.12	0.01\\
76.13	0.01\\
76.14	0.01\\
76.15	0.01\\
76.16	0.01\\
76.17	0.01\\
76.18	0.01\\
76.19	0.01\\
76.2	0.01\\
76.21	0.01\\
76.22	0.01\\
76.23	0.01\\
76.24	0.01\\
76.25	0.01\\
76.26	0.01\\
76.27	0.01\\
76.28	0.01\\
76.29	0.01\\
76.3	0.01\\
76.31	0.01\\
76.32	0.01\\
76.33	0.01\\
76.34	0.01\\
76.35	0.01\\
76.36	0.01\\
76.37	0.01\\
76.38	0.01\\
76.39	0.01\\
76.4	0.01\\
76.41	0.01\\
76.42	0.01\\
76.43	0.01\\
76.44	0.01\\
76.45	0.01\\
76.46	0.01\\
76.47	0.01\\
76.48	0.01\\
76.49	0.01\\
76.5	0.01\\
76.51	0.01\\
76.52	0.01\\
76.53	0.01\\
76.54	0.01\\
76.55	0.01\\
76.56	0.01\\
76.57	0.01\\
76.58	0.01\\
76.59	0.01\\
76.6	0.01\\
76.61	0.01\\
76.62	0.01\\
76.63	0.01\\
76.64	0.01\\
76.65	0.01\\
76.66	0.01\\
76.67	0.01\\
76.68	0.01\\
76.69	0.01\\
76.7	0.01\\
76.71	0.01\\
76.72	0.01\\
76.73	0.01\\
76.74	0.01\\
76.75	0.01\\
76.76	0.01\\
76.77	0.01\\
76.78	0.01\\
76.79	0.01\\
76.8	0.01\\
76.81	0.01\\
76.82	0.01\\
76.83	0.01\\
76.84	0.01\\
76.85	0.01\\
76.86	0.01\\
76.87	0.01\\
76.88	0.01\\
76.89	0.01\\
76.9	0.01\\
76.91	0.01\\
76.92	0.01\\
76.93	0.01\\
76.94	0.01\\
76.95	0.01\\
76.96	0.01\\
76.97	0.01\\
76.98	0.01\\
76.99	0.01\\
77	0.01\\
77.01	0.01\\
77.02	0.01\\
77.03	0.01\\
77.04	0.01\\
77.05	0.01\\
77.06	0.01\\
77.07	0.01\\
77.08	0.01\\
77.09	0.01\\
77.1	0.01\\
77.11	0.01\\
77.12	0.01\\
77.13	0.01\\
77.14	0.01\\
77.15	0.01\\
77.16	0.01\\
77.17	0.01\\
77.18	0.01\\
77.19	0.01\\
77.2	0.01\\
77.21	0.01\\
77.22	0.01\\
77.23	0.01\\
77.24	0.01\\
77.25	0.01\\
77.26	0.01\\
77.27	0.01\\
77.28	0.01\\
77.29	0.01\\
77.3	0.01\\
77.31	0.01\\
77.32	0.01\\
77.33	0.01\\
77.34	0.01\\
77.35	0.01\\
77.36	0.01\\
77.37	0.01\\
77.38	0.01\\
77.39	0.01\\
77.4	0.01\\
77.41	0.01\\
77.42	0.01\\
77.43	0.01\\
77.44	0.01\\
77.45	0.01\\
77.46	0.01\\
77.47	0.01\\
77.48	0.01\\
77.49	0.01\\
77.5	0.01\\
77.51	0.01\\
77.52	0.01\\
77.53	0.01\\
77.54	0.01\\
77.55	0.01\\
77.56	0.01\\
77.57	0.01\\
77.58	0.01\\
77.59	0.01\\
77.6	0.01\\
77.61	0.01\\
77.62	0.01\\
77.63	0.01\\
77.64	0.01\\
77.65	0.01\\
77.66	0.01\\
77.67	0.01\\
77.68	0.01\\
77.69	0.01\\
77.7	0.01\\
77.71	0.01\\
77.72	0.01\\
77.73	0.01\\
77.74	0.01\\
77.75	0.01\\
77.76	0.01\\
77.77	0.01\\
77.78	0.01\\
77.79	0.01\\
77.8	0.01\\
77.81	0.01\\
77.82	0.01\\
77.83	0.01\\
77.84	0.01\\
77.85	0.01\\
77.86	0.01\\
77.87	0.01\\
77.88	0.01\\
77.89	0.01\\
77.9	0.01\\
77.91	0.01\\
77.92	0.01\\
77.93	0.01\\
77.94	0.01\\
77.95	0.01\\
77.96	0.01\\
77.97	0.01\\
77.98	0.01\\
77.99	0.01\\
78	0.01\\
78.01	0.01\\
78.02	0.01\\
78.03	0.01\\
78.04	0.01\\
78.05	0.01\\
78.06	0.01\\
78.07	0.01\\
78.08	0.01\\
78.09	0.01\\
78.1	0.01\\
78.11	0.01\\
78.12	0.01\\
78.13	0.01\\
78.14	0.01\\
78.15	0.01\\
78.16	0.01\\
78.17	0.01\\
78.18	0.01\\
78.19	0.01\\
78.2	0.01\\
78.21	0.01\\
78.22	0.01\\
78.23	0.01\\
78.24	0.01\\
78.25	0.01\\
78.26	0.01\\
78.27	0.01\\
78.28	0.01\\
78.29	0.01\\
78.3	0.01\\
78.31	0.01\\
78.32	0.01\\
78.33	0.01\\
78.34	0.01\\
78.35	0.01\\
78.36	0.01\\
78.37	0.01\\
78.38	0.01\\
78.39	0.01\\
78.4	0.01\\
78.41	0.01\\
78.42	0.01\\
78.43	0.01\\
78.44	0.01\\
78.45	0.01\\
78.46	0.01\\
78.47	0.01\\
78.48	0.01\\
78.49	0.01\\
78.5	0.01\\
78.51	0.01\\
78.52	0.01\\
78.53	0.01\\
78.54	0.01\\
78.55	0.01\\
78.56	0.01\\
78.57	0.01\\
78.58	0.01\\
78.59	0.01\\
78.6	0.01\\
78.61	0.01\\
78.62	0.01\\
78.63	0.01\\
78.64	0.01\\
78.65	0.01\\
78.66	0.01\\
78.67	0.01\\
78.68	0.01\\
78.69	0.01\\
78.7	0.01\\
78.71	0.01\\
78.72	0.01\\
78.73	0.01\\
78.74	0.01\\
78.75	0.01\\
78.76	0.01\\
78.77	0.01\\
78.78	0.01\\
78.79	0.01\\
78.8	0.01\\
78.81	0.01\\
78.82	0.01\\
78.83	0.01\\
78.84	0.01\\
78.85	0.01\\
78.86	0.01\\
78.87	0.01\\
78.88	0.01\\
78.89	0.01\\
78.9	0.01\\
78.91	0.01\\
78.92	0.01\\
78.93	0.01\\
78.94	0.01\\
78.95	0.01\\
78.96	0.01\\
78.97	0.01\\
78.98	0.01\\
78.99	0.01\\
79	0.01\\
79.01	0.01\\
79.02	0.01\\
79.03	0.01\\
79.04	0.01\\
79.05	0.01\\
79.06	0.01\\
79.07	0.01\\
79.08	0.01\\
79.09	0.01\\
79.1	0.01\\
79.11	0.01\\
79.12	0.01\\
79.13	0.01\\
79.14	0.01\\
79.15	0.01\\
79.16	0.01\\
79.17	0.01\\
79.18	0.01\\
79.19	0.01\\
79.2	0.01\\
79.21	0.01\\
79.22	0.01\\
79.23	0.01\\
79.24	0.01\\
79.25	0.01\\
79.26	0.01\\
79.27	0.01\\
79.28	0.01\\
79.29	0.01\\
79.3	0.01\\
79.31	0.01\\
79.32	0.01\\
79.33	0.01\\
79.34	0.01\\
79.35	0.01\\
79.36	0.01\\
79.37	0.01\\
79.38	0.01\\
79.39	0.01\\
79.4	0.01\\
79.41	0.01\\
79.42	0.01\\
79.43	0.01\\
79.44	0.01\\
79.45	0.01\\
79.46	0.01\\
79.47	0.01\\
79.48	0.01\\
79.49	0.01\\
79.5	0.01\\
79.51	0.01\\
79.52	0.01\\
79.53	0.01\\
79.54	0.01\\
79.55	0.01\\
79.56	0.01\\
79.57	0.01\\
79.58	0.01\\
79.59	0.01\\
79.6	0.01\\
79.61	0.01\\
79.62	0.01\\
79.63	0.01\\
79.64	0.01\\
79.65	0.01\\
79.66	0.01\\
79.67	0.01\\
79.68	0.01\\
79.69	0.01\\
79.7	0.01\\
79.71	0.01\\
79.72	0.01\\
79.73	0.01\\
79.74	0.01\\
79.75	0.01\\
79.76	0.01\\
79.77	0.01\\
79.78	0.01\\
79.79	0.01\\
79.8	0.01\\
79.81	0.01\\
79.82	0.01\\
79.83	0.01\\
79.84	0.01\\
79.85	0.01\\
79.86	0.01\\
79.87	0.01\\
79.88	0.01\\
79.89	0.01\\
79.9	0.01\\
79.91	0.01\\
79.92	0.01\\
79.93	0.01\\
79.94	0.01\\
79.95	0.01\\
79.96	0.01\\
79.97	0.01\\
79.98	0.01\\
79.99	0.01\\
80	0.01\\
80.01	0.01\\
};
\addplot [color=green,solid]
  table[row sep=crcr]{%
80.01	0.01\\
80.02	0.01\\
80.03	0.01\\
80.04	0.01\\
80.05	0.01\\
80.06	0.01\\
80.07	0.01\\
80.08	0.01\\
80.09	0.01\\
80.1	0.01\\
80.11	0.01\\
80.12	0.01\\
80.13	0.01\\
80.14	0.01\\
80.15	0.01\\
80.16	0.01\\
80.17	0.01\\
80.18	0.01\\
80.19	0.01\\
80.2	0.01\\
80.21	0.01\\
80.22	0.01\\
80.23	0.01\\
80.24	0.01\\
80.25	0.01\\
80.26	0.01\\
80.27	0.01\\
80.28	0.01\\
80.29	0.01\\
80.3	0.01\\
80.31	0.01\\
80.32	0.01\\
80.33	0.01\\
80.34	0.01\\
80.35	0.01\\
80.36	0.01\\
80.37	0.01\\
80.38	0.01\\
80.39	0.01\\
80.4	0.01\\
80.41	0.01\\
80.42	0.01\\
80.43	0.01\\
80.44	0.01\\
80.45	0.01\\
80.46	0.01\\
80.47	0.01\\
80.48	0.01\\
80.49	0.01\\
80.5	0.01\\
80.51	0.01\\
80.52	0.01\\
80.53	0.01\\
80.54	0.01\\
80.55	0.01\\
80.56	0.01\\
80.57	0.01\\
80.58	0.01\\
80.59	0.01\\
80.6	0.01\\
80.61	0.01\\
80.62	0.01\\
80.63	0.01\\
80.64	0.01\\
80.65	0.01\\
80.66	0.01\\
80.67	0.01\\
80.68	0.01\\
80.69	0.01\\
80.7	0.01\\
80.71	0.01\\
80.72	0.01\\
80.73	0.01\\
80.74	0.01\\
80.75	0.01\\
80.76	0.01\\
80.77	0.01\\
80.78	0.01\\
80.79	0.01\\
80.8	0.01\\
80.81	0.01\\
80.82	0.01\\
80.83	0.01\\
80.84	0.01\\
80.85	0.01\\
80.86	0.01\\
80.87	0.01\\
80.88	0.01\\
80.89	0.01\\
80.9	0.01\\
80.91	0.01\\
80.92	0.01\\
80.93	0.01\\
80.94	0.01\\
80.95	0.01\\
80.96	0.01\\
80.97	0.01\\
80.98	0.01\\
80.99	0.01\\
81	0.01\\
81.01	0.01\\
81.02	0.01\\
81.03	0.01\\
81.04	0.01\\
81.05	0.01\\
81.06	0.01\\
81.07	0.01\\
81.08	0.01\\
81.09	0.01\\
81.1	0.01\\
81.11	0.01\\
81.12	0.01\\
81.13	0.01\\
81.14	0.01\\
81.15	0.01\\
81.16	0.01\\
81.17	0.01\\
81.18	0.01\\
81.19	0.01\\
81.2	0.01\\
81.21	0.01\\
81.22	0.01\\
81.23	0.01\\
81.24	0.01\\
81.25	0.01\\
81.26	0.01\\
81.27	0.01\\
81.28	0.01\\
81.29	0.01\\
81.3	0.01\\
81.31	0.01\\
81.32	0.01\\
81.33	0.01\\
81.34	0.01\\
81.35	0.01\\
81.36	0.01\\
81.37	0.01\\
81.38	0.01\\
81.39	0.01\\
81.4	0.01\\
81.41	0.01\\
81.42	0.01\\
81.43	0.01\\
81.44	0.01\\
81.45	0.01\\
81.46	0.01\\
81.47	0.01\\
81.48	0.01\\
81.49	0.01\\
81.5	0.01\\
81.51	0.01\\
81.52	0.01\\
81.53	0.01\\
81.54	0.01\\
81.55	0.01\\
81.56	0.01\\
81.57	0.01\\
81.58	0.01\\
81.59	0.01\\
81.6	0.01\\
81.61	0.01\\
81.62	0.01\\
81.63	0.01\\
81.64	0.01\\
81.65	0.01\\
81.66	0.01\\
81.67	0.01\\
81.68	0.01\\
81.69	0.01\\
81.7	0.01\\
81.71	0.01\\
81.72	0.01\\
81.73	0.01\\
81.74	0.01\\
81.75	0.01\\
81.76	0.01\\
81.77	0.01\\
81.78	0.01\\
81.79	0.01\\
81.8	0.01\\
81.81	0.01\\
81.82	0.01\\
81.83	0.01\\
81.84	0.01\\
81.85	0.01\\
81.86	0.01\\
81.87	0.01\\
81.88	0.01\\
81.89	0.01\\
81.9	0.01\\
81.91	0.01\\
81.92	0.01\\
81.93	0.01\\
81.94	0.01\\
81.95	0.01\\
81.96	0.01\\
81.97	0.01\\
81.98	0.01\\
81.99	0.01\\
82	0.01\\
82.01	0.01\\
82.02	0.01\\
82.03	0.01\\
82.04	0.01\\
82.05	0.01\\
82.06	0.01\\
82.07	0.01\\
82.08	0.01\\
82.09	0.01\\
82.1	0.01\\
82.11	0.01\\
82.12	0.01\\
82.13	0.01\\
82.14	0.01\\
82.15	0.01\\
82.16	0.01\\
82.17	0.01\\
82.18	0.01\\
82.19	0.01\\
82.2	0.01\\
82.21	0.01\\
82.22	0.01\\
82.23	0.01\\
82.24	0.01\\
82.25	0.01\\
82.26	0.01\\
82.27	0.01\\
82.28	0.01\\
82.29	0.01\\
82.3	0.01\\
82.31	0.01\\
82.32	0.01\\
82.33	0.01\\
82.34	0.01\\
82.35	0.01\\
82.36	0.01\\
82.37	0.01\\
82.38	0.01\\
82.39	0.01\\
82.4	0.01\\
82.41	0.01\\
82.42	0.01\\
82.43	0.01\\
82.44	0.01\\
82.45	0.01\\
82.46	0.01\\
82.47	0.01\\
82.48	0.01\\
82.49	0.01\\
82.5	0.01\\
82.51	0.01\\
82.52	0.01\\
82.53	0.01\\
82.54	0.01\\
82.55	0.01\\
82.56	0.01\\
82.57	0.01\\
82.58	0.01\\
82.59	0.01\\
82.6	0.01\\
82.61	0.01\\
82.62	0.01\\
82.63	0.01\\
82.64	0.01\\
82.65	0.01\\
82.66	0.01\\
82.67	0.01\\
82.68	0.01\\
82.69	0.01\\
82.7	0.01\\
82.71	0.01\\
82.72	0.01\\
82.73	0.01\\
82.74	0.01\\
82.75	0.01\\
82.76	0.01\\
82.77	0.01\\
82.78	0.01\\
82.79	0.01\\
82.8	0.01\\
82.81	0.01\\
82.82	0.01\\
82.83	0.01\\
82.84	0.01\\
82.85	0.01\\
82.86	0.01\\
82.87	0.01\\
82.88	0.01\\
82.89	0.01\\
82.9	0.01\\
82.91	0.01\\
82.92	0.01\\
82.93	0.01\\
82.94	0.01\\
82.95	0.01\\
82.96	0.01\\
82.97	0.01\\
82.98	0.01\\
82.99	0.01\\
83	0.01\\
83.01	0.01\\
83.02	0.01\\
83.03	0.01\\
83.04	0.01\\
83.05	0.01\\
83.06	0.01\\
83.07	0.01\\
83.08	0.01\\
83.09	0.01\\
83.1	0.01\\
83.11	0.01\\
83.12	0.01\\
83.13	0.01\\
83.14	0.01\\
83.15	0.01\\
83.16	0.01\\
83.17	0.01\\
83.18	0.01\\
83.19	0.01\\
83.2	0.01\\
83.21	0.01\\
83.22	0.01\\
83.23	0.01\\
83.24	0.01\\
83.25	0.01\\
83.26	0.01\\
83.27	0.01\\
83.28	0.01\\
83.29	0.01\\
83.3	0.01\\
83.31	0.01\\
83.32	0.01\\
83.33	0.01\\
83.34	0.01\\
83.35	0.01\\
83.36	0.01\\
83.37	0.01\\
83.38	0.01\\
83.39	0.01\\
83.4	0.01\\
83.41	0.01\\
83.42	0.01\\
83.43	0.01\\
83.44	0.01\\
83.45	0.01\\
83.46	0.01\\
83.47	0.01\\
83.48	0.01\\
83.49	0.01\\
83.5	0.01\\
83.51	0.01\\
83.52	0.01\\
83.53	0.01\\
83.54	0.01\\
83.55	0.01\\
83.56	0.01\\
83.57	0.01\\
83.58	0.01\\
83.59	0.01\\
83.6	0.01\\
83.61	0.01\\
83.62	0.01\\
83.63	0.01\\
83.64	0.01\\
83.65	0.01\\
83.66	0.01\\
83.67	0.01\\
83.68	0.01\\
83.69	0.01\\
83.7	0.01\\
83.71	0.01\\
83.72	0.01\\
83.73	0.01\\
83.74	0.01\\
83.75	0.01\\
83.76	0.01\\
83.77	0.01\\
83.78	0.01\\
83.79	0.01\\
83.8	0.01\\
83.81	0.01\\
83.82	0.01\\
83.83	0.01\\
83.84	0.01\\
83.85	0.01\\
83.86	0.01\\
83.87	0.01\\
83.88	0.01\\
83.89	0.01\\
83.9	0.01\\
83.91	0.01\\
83.92	0.01\\
83.93	0.01\\
83.94	0.01\\
83.95	0.01\\
83.96	0.01\\
83.97	0.01\\
83.98	0.01\\
83.99	0.01\\
84	0.01\\
84.01	0.01\\
84.02	0.01\\
84.03	0.01\\
84.04	0.01\\
84.05	0.01\\
84.06	0.01\\
84.07	0.01\\
84.08	0.01\\
84.09	0.01\\
84.1	0.01\\
84.11	0.01\\
84.12	0.01\\
84.13	0.01\\
84.14	0.01\\
84.15	0.01\\
84.16	0.01\\
84.17	0.01\\
84.18	0.01\\
84.19	0.01\\
84.2	0.01\\
84.21	0.01\\
84.22	0.01\\
84.23	0.01\\
84.24	0.01\\
84.25	0.01\\
84.26	0.01\\
84.27	0.01\\
84.28	0.01\\
84.29	0.01\\
84.3	0.01\\
84.31	0.01\\
84.32	0.01\\
84.33	0.01\\
84.34	0.01\\
84.35	0.01\\
84.36	0.01\\
84.37	0.01\\
84.38	0.01\\
84.39	0.01\\
84.4	0.01\\
84.41	0.01\\
84.42	0.01\\
84.43	0.01\\
84.44	0.01\\
84.45	0.01\\
84.46	0.01\\
84.47	0.01\\
84.48	0.01\\
84.49	0.01\\
84.5	0.01\\
84.51	0.01\\
84.52	0.01\\
84.53	0.01\\
84.54	0.01\\
84.55	0.01\\
84.56	0.01\\
84.57	0.01\\
84.58	0.01\\
84.59	0.01\\
84.6	0.01\\
84.61	0.01\\
84.62	0.01\\
84.63	0.01\\
84.64	0.01\\
84.65	0.01\\
84.66	0.01\\
84.67	0.01\\
84.68	0.01\\
84.69	0.01\\
84.7	0.01\\
84.71	0.01\\
84.72	0.01\\
84.73	0.01\\
84.74	0.01\\
84.75	0.01\\
84.76	0.01\\
84.77	0.01\\
84.78	0.01\\
84.79	0.01\\
84.8	0.01\\
84.81	0.01\\
84.82	0.01\\
84.83	0.01\\
84.84	0.01\\
84.85	0.01\\
84.86	0.01\\
84.87	0.01\\
84.88	0.01\\
84.89	0.01\\
84.9	0.01\\
84.91	0.01\\
84.92	0.01\\
84.93	0.01\\
84.94	0.01\\
84.95	0.01\\
84.96	0.01\\
84.97	0.01\\
84.98	0.01\\
84.99	0.01\\
85	0.01\\
85.01	0.01\\
85.02	0.01\\
85.03	0.01\\
85.04	0.01\\
85.05	0.01\\
85.06	0.01\\
85.07	0.01\\
85.08	0.01\\
85.09	0.01\\
85.1	0.01\\
85.11	0.01\\
85.12	0.01\\
85.13	0.01\\
85.14	0.01\\
85.15	0.01\\
85.16	0.01\\
85.17	0.01\\
85.18	0.01\\
85.19	0.01\\
85.2	0.01\\
85.21	0.01\\
85.22	0.01\\
85.23	0.01\\
85.24	0.01\\
85.25	0.01\\
85.26	0.01\\
85.27	0.01\\
85.28	0.01\\
85.29	0.01\\
85.3	0.01\\
85.31	0.01\\
85.32	0.01\\
85.33	0.01\\
85.34	0.01\\
85.35	0.01\\
85.36	0.01\\
85.37	0.01\\
85.38	0.01\\
85.39	0.01\\
85.4	0.01\\
85.41	0.01\\
85.42	0.01\\
85.43	0.01\\
85.44	0.01\\
85.45	0.01\\
85.46	0.01\\
85.47	0.01\\
85.48	0.01\\
85.49	0.01\\
85.5	0.01\\
85.51	0.01\\
85.52	0.01\\
85.53	0.01\\
85.54	0.01\\
85.55	0.01\\
85.56	0.01\\
85.57	0.01\\
85.58	0.01\\
85.59	0.01\\
85.6	0.01\\
85.61	0.01\\
85.62	0.01\\
85.63	0.01\\
85.64	0.01\\
85.65	0.01\\
85.66	0.01\\
85.67	0.01\\
85.68	0.01\\
85.69	0.01\\
85.7	0.01\\
85.71	0.01\\
85.72	0.01\\
85.73	0.01\\
85.74	0.01\\
85.75	0.01\\
85.76	0.01\\
85.77	0.01\\
85.78	0.01\\
85.79	0.01\\
85.8	0.01\\
85.81	0.01\\
85.82	0.01\\
85.83	0.01\\
85.84	0.01\\
85.85	0.01\\
85.86	0.01\\
85.87	0.01\\
85.88	0.01\\
85.89	0.01\\
85.9	0.01\\
85.91	0.01\\
85.92	0.01\\
85.93	0.01\\
85.94	0.01\\
85.95	0.01\\
85.96	0.01\\
85.97	0.01\\
85.98	0.01\\
85.99	0.01\\
86	0.01\\
86.01	0.01\\
86.02	0.01\\
86.03	0.01\\
86.04	0.01\\
86.05	0.01\\
86.06	0.01\\
86.07	0.01\\
86.08	0.01\\
86.09	0.01\\
86.1	0.01\\
86.11	0.01\\
86.12	0.01\\
86.13	0.01\\
86.14	0.01\\
86.15	0.01\\
86.16	0.01\\
86.17	0.01\\
86.18	0.01\\
86.19	0.01\\
86.2	0.01\\
86.21	0.01\\
86.22	0.01\\
86.23	0.01\\
86.24	0.01\\
86.25	0.01\\
86.26	0.01\\
86.27	0.01\\
86.28	0.01\\
86.29	0.01\\
86.3	0.01\\
86.31	0.01\\
86.32	0.01\\
86.33	0.01\\
86.34	0.01\\
86.35	0.01\\
86.36	0.01\\
86.37	0.01\\
86.38	0.01\\
86.39	0.01\\
86.4	0.01\\
86.41	0.01\\
86.42	0.01\\
86.43	0.01\\
86.44	0.01\\
86.45	0.01\\
86.46	0.01\\
86.47	0.01\\
86.48	0.01\\
86.49	0.01\\
86.5	0.01\\
86.51	0.01\\
86.52	0.01\\
86.53	0.01\\
86.54	0.01\\
86.55	0.01\\
86.56	0.01\\
86.57	0.01\\
86.58	0.01\\
86.59	0.01\\
86.6	0.01\\
86.61	0.01\\
86.62	0.01\\
86.63	0.01\\
86.64	0.01\\
86.65	0.01\\
86.66	0.01\\
86.67	0.01\\
86.68	0.01\\
86.69	0.01\\
86.7	0.01\\
86.71	0.01\\
86.72	0.01\\
86.73	0.01\\
86.74	0.01\\
86.75	0.01\\
86.76	0.01\\
86.77	0.01\\
86.78	0.01\\
86.79	0.01\\
86.8	0.01\\
86.81	0.01\\
86.82	0.01\\
86.83	0.01\\
86.84	0.01\\
86.85	0.01\\
86.86	0.01\\
86.87	0.01\\
86.88	0.01\\
86.89	0.01\\
86.9	0.01\\
86.91	0.01\\
86.92	0.01\\
86.93	0.01\\
86.94	0.01\\
86.95	0.01\\
86.96	0.01\\
86.97	0.01\\
86.98	0.01\\
86.99	0.01\\
87	0.01\\
87.01	0.01\\
87.02	0.01\\
87.03	0.01\\
87.04	0.01\\
87.05	0.01\\
87.06	0.01\\
87.07	0.01\\
87.08	0.01\\
87.09	0.01\\
87.1	0.01\\
87.11	0.01\\
87.12	0.01\\
87.13	0.01\\
87.14	0.01\\
87.15	0.01\\
87.16	0.01\\
87.17	0.01\\
87.18	0.01\\
87.19	0.01\\
87.2	0.01\\
87.21	0.01\\
87.22	0.01\\
87.23	0.01\\
87.24	0.01\\
87.25	0.01\\
87.26	0.01\\
87.27	0.01\\
87.28	0.01\\
87.29	0.01\\
87.3	0.01\\
87.31	0.01\\
87.32	0.01\\
87.33	0.01\\
87.34	0.01\\
87.35	0.01\\
87.36	0.01\\
87.37	0.01\\
87.38	0.01\\
87.39	0.01\\
87.4	0.01\\
87.41	0.01\\
87.42	0.01\\
87.43	0.01\\
87.44	0.01\\
87.45	0.01\\
87.46	0.01\\
87.47	0.01\\
87.48	0.01\\
87.49	0.01\\
87.5	0.01\\
87.51	0.01\\
87.52	0.01\\
87.53	0.01\\
87.54	0.01\\
87.55	0.01\\
87.56	0.01\\
87.57	0.01\\
87.58	0.01\\
87.59	0.01\\
87.6	0.01\\
87.61	0.01\\
87.62	0.01\\
87.63	0.01\\
87.64	0.01\\
87.65	0.01\\
87.66	0.01\\
87.67	0.01\\
87.68	0.01\\
87.69	0.01\\
87.7	0.01\\
87.71	0.01\\
87.72	0.01\\
87.73	0.01\\
87.74	0.01\\
87.75	0.01\\
87.76	0.01\\
87.77	0.01\\
87.78	0.01\\
87.79	0.01\\
87.8	0.01\\
87.81	0.01\\
87.82	0.01\\
87.83	0.01\\
87.84	0.01\\
87.85	0.01\\
87.86	0.01\\
87.87	0.01\\
87.88	0.01\\
87.89	0.01\\
87.9	0.01\\
87.91	0.01\\
87.92	0.01\\
87.93	0.01\\
87.94	0.01\\
87.95	0.01\\
87.96	0.01\\
87.97	0.01\\
87.98	0.01\\
87.99	0.01\\
88	0.01\\
88.01	0.01\\
88.02	0.01\\
88.03	0.01\\
88.04	0.01\\
88.05	0.01\\
88.06	0.01\\
88.07	0.01\\
88.08	0.01\\
88.09	0.01\\
88.1	0.01\\
88.11	0.01\\
88.12	0.01\\
88.13	0.01\\
88.14	0.01\\
88.15	0.01\\
88.16	0.01\\
88.17	0.01\\
88.18	0.01\\
88.19	0.01\\
88.2	0.01\\
88.21	0.01\\
88.22	0.01\\
88.23	0.01\\
88.24	0.01\\
88.25	0.01\\
88.26	0.01\\
88.27	0.01\\
88.28	0.01\\
88.29	0.01\\
88.3	0.01\\
88.31	0.01\\
88.32	0.01\\
88.33	0.01\\
88.34	0.01\\
88.35	0.01\\
88.36	0.01\\
88.37	0.01\\
88.38	0.01\\
88.39	0.01\\
88.4	0.01\\
88.41	0.01\\
88.42	0.01\\
88.43	0.01\\
88.44	0.01\\
88.45	0.01\\
88.46	0.01\\
88.47	0.01\\
88.48	0.01\\
88.49	0.01\\
88.5	0.01\\
88.51	0.01\\
88.52	0.01\\
88.53	0.01\\
88.54	0.01\\
88.55	0.01\\
88.56	0.01\\
88.57	0.01\\
88.58	0.01\\
88.59	0.01\\
88.6	0.01\\
88.61	0.01\\
88.62	0.01\\
88.63	0.01\\
88.64	0.01\\
88.65	0.01\\
88.66	0.01\\
88.67	0.01\\
88.68	0.01\\
88.69	0.01\\
88.7	0.01\\
88.71	0.01\\
88.72	0.01\\
88.73	0.01\\
88.74	0.01\\
88.75	0.01\\
88.76	0.01\\
88.77	0.01\\
88.78	0.01\\
88.79	0.01\\
88.8	0.01\\
88.81	0.01\\
88.82	0.01\\
88.83	0.01\\
88.84	0.01\\
88.85	0.01\\
88.86	0.01\\
88.87	0.01\\
88.88	0.01\\
88.89	0.01\\
88.9	0.01\\
88.91	0.01\\
88.92	0.01\\
88.93	0.01\\
88.94	0.01\\
88.95	0.01\\
88.96	0.01\\
88.97	0.01\\
88.98	0.01\\
88.99	0.01\\
89	0.01\\
89.01	0.01\\
89.02	0.01\\
89.03	0.01\\
89.04	0.01\\
89.05	0.01\\
89.06	0.01\\
89.07	0.01\\
89.08	0.01\\
89.09	0.01\\
89.1	0.01\\
89.11	0.01\\
89.12	0.01\\
89.13	0.01\\
89.14	0.01\\
89.15	0.01\\
89.16	0.01\\
89.17	0.01\\
89.18	0.01\\
89.19	0.01\\
89.2	0.01\\
89.21	0.01\\
89.22	0.01\\
89.23	0.01\\
89.24	0.01\\
89.25	0.01\\
89.26	0.01\\
89.27	0.01\\
89.28	0.01\\
89.29	0.01\\
89.3	0.01\\
89.31	0.01\\
89.32	0.01\\
89.33	0.01\\
89.34	0.01\\
89.35	0.01\\
89.36	0.01\\
89.37	0.01\\
89.38	0.01\\
89.39	0.01\\
89.4	0.01\\
89.41	0.01\\
89.42	0.01\\
89.43	0.01\\
89.44	0.01\\
89.45	0.01\\
89.46	0.01\\
89.47	0.01\\
89.48	0.01\\
89.49	0.01\\
89.5	0.01\\
89.51	0.01\\
89.52	0.01\\
89.53	0.01\\
89.54	0.01\\
89.55	0.01\\
89.56	0.01\\
89.57	0.01\\
89.58	0.01\\
89.59	0.01\\
89.6	0.01\\
89.61	0.01\\
89.62	0.01\\
89.63	0.01\\
89.64	0.01\\
89.65	0.01\\
89.66	0.01\\
89.67	0.01\\
89.68	0.01\\
89.69	0.01\\
89.7	0.01\\
89.71	0.01\\
89.72	0.01\\
89.73	0.01\\
89.74	0.01\\
89.75	0.01\\
89.76	0.01\\
89.77	0.01\\
89.78	0.01\\
89.79	0.01\\
89.8	0.01\\
89.81	0.01\\
89.82	0.01\\
89.83	0.01\\
89.84	0.01\\
89.85	0.01\\
89.86	0.01\\
89.87	0.01\\
89.88	0.01\\
89.89	0.01\\
89.9	0.01\\
89.91	0.01\\
89.92	0.01\\
89.93	0.01\\
89.94	0.01\\
89.95	0.01\\
89.96	0.01\\
89.97	0.01\\
89.98	0.01\\
89.99	0.01\\
90	0.01\\
90.01	0.01\\
90.02	0.01\\
90.03	0.01\\
90.04	0.01\\
90.05	0.01\\
90.06	0.01\\
90.07	0.01\\
90.08	0.01\\
90.09	0.01\\
90.1	0.01\\
90.11	0.01\\
90.12	0.01\\
90.13	0.01\\
90.14	0.01\\
90.15	0.01\\
90.16	0.01\\
90.17	0.01\\
90.18	0.01\\
90.19	0.01\\
90.2	0.01\\
90.21	0.01\\
90.22	0.01\\
90.23	0.01\\
90.24	0.01\\
90.25	0.01\\
90.26	0.01\\
90.27	0.01\\
90.28	0.01\\
90.29	0.01\\
90.3	0.01\\
90.31	0.01\\
90.32	0.01\\
90.33	0.01\\
90.34	0.01\\
90.35	0.01\\
90.36	0.01\\
90.37	0.01\\
90.38	0.01\\
90.39	0.01\\
90.4	0.01\\
90.41	0.01\\
90.42	0.01\\
90.43	0.01\\
90.44	0.01\\
90.45	0.01\\
90.46	0.01\\
90.47	0.01\\
90.48	0.01\\
90.49	0.01\\
90.5	0.01\\
90.51	0.01\\
90.52	0.01\\
90.53	0.01\\
90.54	0.01\\
90.55	0.01\\
90.56	0.01\\
90.57	0.01\\
90.58	0.01\\
90.59	0.01\\
90.6	0.01\\
90.61	0.01\\
90.62	0.01\\
90.63	0.01\\
90.64	0.01\\
90.65	0.01\\
90.66	0.01\\
90.67	0.01\\
90.68	0.01\\
90.69	0.01\\
90.7	0.01\\
90.71	0.01\\
90.72	0.01\\
90.73	0.01\\
90.74	0.01\\
90.75	0.01\\
90.76	0.01\\
90.77	0.01\\
90.78	0.01\\
90.79	0.01\\
90.8	0.01\\
90.81	0.01\\
90.82	0.01\\
90.83	0.01\\
90.84	0.01\\
90.85	0.01\\
90.86	0.01\\
90.87	0.01\\
90.88	0.01\\
90.89	0.01\\
90.9	0.01\\
90.91	0.01\\
90.92	0.01\\
90.93	0.01\\
90.94	0.01\\
90.95	0.01\\
90.96	0.01\\
90.97	0.01\\
90.98	0.01\\
90.99	0.01\\
91	0.01\\
91.01	0.01\\
91.02	0.01\\
91.03	0.01\\
91.04	0.01\\
91.05	0.01\\
91.06	0.01\\
91.07	0.01\\
91.08	0.01\\
91.09	0.01\\
91.1	0.01\\
91.11	0.01\\
91.12	0.01\\
91.13	0.01\\
91.14	0.01\\
91.15	0.01\\
91.16	0.01\\
91.17	0.01\\
91.18	0.01\\
91.19	0.01\\
91.2	0.01\\
91.21	0.01\\
91.22	0.01\\
91.23	0.01\\
91.24	0.01\\
91.25	0.01\\
91.26	0.01\\
91.27	0.01\\
91.28	0.01\\
91.29	0.01\\
91.3	0.01\\
91.31	0.01\\
91.32	0.01\\
91.33	0.01\\
91.34	0.01\\
91.35	0.01\\
91.36	0.01\\
91.37	0.01\\
91.38	0.01\\
91.39	0.01\\
91.4	0.01\\
91.41	0.01\\
91.42	0.01\\
91.43	0.01\\
91.44	0.01\\
91.45	0.01\\
91.46	0.01\\
91.47	0.01\\
91.48	0.01\\
91.49	0.01\\
91.5	0.01\\
91.51	0.01\\
91.52	0.01\\
91.53	0.01\\
91.54	0.01\\
91.55	0.01\\
91.56	0.01\\
91.57	0.01\\
91.58	0.01\\
91.59	0.01\\
91.6	0.01\\
91.61	0.01\\
91.62	0.01\\
91.63	0.01\\
91.64	0.01\\
91.65	0.01\\
91.66	0.01\\
91.67	0.01\\
91.68	0.01\\
91.69	0.01\\
91.7	0.01\\
91.71	0.01\\
91.72	0.01\\
91.73	0.01\\
91.74	0.01\\
91.75	0.01\\
91.76	0.01\\
91.77	0.01\\
91.78	0.01\\
91.79	0.01\\
91.8	0.01\\
91.81	0.01\\
91.82	0.01\\
91.83	0.01\\
91.84	0.01\\
91.85	0.01\\
91.86	0.01\\
91.87	0.01\\
91.88	0.01\\
91.89	0.01\\
91.9	0.01\\
91.91	0.01\\
91.92	0.01\\
91.93	0.01\\
91.94	0.01\\
91.95	0.01\\
91.96	0.01\\
91.97	0.01\\
91.98	0.01\\
91.99	0.01\\
92	0.01\\
92.01	0.01\\
92.02	0.01\\
92.03	0.01\\
92.04	0.01\\
92.05	0.01\\
92.06	0.01\\
92.07	0.01\\
92.08	0.01\\
92.09	0.01\\
92.1	0.01\\
92.11	0.01\\
92.12	0.01\\
92.13	0.01\\
92.14	0.01\\
92.15	0.01\\
92.16	0.01\\
92.17	0.01\\
92.18	0.01\\
92.19	0.01\\
92.2	0.01\\
92.21	0.01\\
92.22	0.01\\
92.23	0.01\\
92.24	0.01\\
92.25	0.01\\
92.26	0.01\\
92.27	0.01\\
92.28	0.01\\
92.29	0.01\\
92.3	0.01\\
92.31	0.01\\
92.32	0.01\\
92.33	0.01\\
92.34	0.01\\
92.35	0.01\\
92.36	0.01\\
92.37	0.01\\
92.38	0.01\\
92.39	0.01\\
92.4	0.01\\
92.41	0.01\\
92.42	0.01\\
92.43	0.01\\
92.44	0.01\\
92.45	0.01\\
92.46	0.01\\
92.47	0.01\\
92.48	0.01\\
92.49	0.01\\
92.5	0.01\\
92.51	0.01\\
92.52	0.01\\
92.53	0.01\\
92.54	0.01\\
92.55	0.01\\
92.56	0.01\\
92.57	0.01\\
92.58	0.01\\
92.59	0.01\\
92.6	0.01\\
92.61	0.01\\
92.62	0.01\\
92.63	0.01\\
92.64	0.01\\
92.65	0.01\\
92.66	0.01\\
92.67	0.01\\
92.68	0.01\\
92.69	0.01\\
92.7	0.01\\
92.71	0.01\\
92.72	0.01\\
92.73	0.01\\
92.74	0.01\\
92.75	0.01\\
92.76	0.01\\
92.77	0.01\\
92.78	0.01\\
92.79	0.01\\
92.8	0.01\\
92.81	0.01\\
92.82	0.01\\
92.83	0.01\\
92.84	0.01\\
92.85	0.01\\
92.86	0.01\\
92.87	0.01\\
92.88	0.01\\
92.89	0.01\\
92.9	0.01\\
92.91	0.01\\
92.92	0.01\\
92.93	0.01\\
92.94	0.01\\
92.95	0.01\\
92.96	0.01\\
92.97	0.01\\
92.98	0.01\\
92.99	0.01\\
93	0.01\\
93.01	0.01\\
93.02	0.01\\
93.03	0.01\\
93.04	0.01\\
93.05	0.01\\
93.06	0.01\\
93.07	0.01\\
93.08	0.01\\
93.09	0.01\\
93.1	0.01\\
93.11	0.01\\
93.12	0.01\\
93.13	0.01\\
93.14	0.01\\
93.15	0.01\\
93.16	0.01\\
93.17	0.01\\
93.18	0.01\\
93.19	0.01\\
93.2	0.01\\
93.21	0.01\\
93.22	0.01\\
93.23	0.01\\
93.24	0.01\\
93.25	0.01\\
93.26	0.01\\
93.27	0.01\\
93.28	0.01\\
93.29	0.01\\
93.3	0.01\\
93.31	0.01\\
93.32	0.01\\
93.33	0.01\\
93.34	0.01\\
93.35	0.01\\
93.36	0.01\\
93.37	0.01\\
93.38	0.01\\
93.39	0.01\\
93.4	0.01\\
93.41	0.01\\
93.42	0.01\\
93.43	0.01\\
93.44	0.01\\
93.45	0.01\\
93.46	0.01\\
93.47	0.01\\
93.48	0.01\\
93.49	0.01\\
93.5	0.01\\
93.51	0.01\\
93.52	0.01\\
93.53	0.01\\
93.54	0.01\\
93.55	0.01\\
93.56	0.01\\
93.57	0.01\\
93.58	0.01\\
93.59	0.01\\
93.6	0.01\\
93.61	0.01\\
93.62	0.01\\
93.63	0.01\\
93.64	0.01\\
93.65	0.01\\
93.66	0.01\\
93.67	0.01\\
93.68	0.01\\
93.69	0.01\\
93.7	0.01\\
93.71	0.01\\
93.72	0.01\\
93.73	0.01\\
93.74	0.01\\
93.75	0.01\\
93.76	0.01\\
93.77	0.01\\
93.78	0.01\\
93.79	0.01\\
93.8	0.01\\
93.81	0.01\\
93.82	0.01\\
93.83	0.01\\
93.84	0.01\\
93.85	0.01\\
93.86	0.01\\
93.87	0.01\\
93.88	0.01\\
93.89	0.01\\
93.9	0.01\\
93.91	0.01\\
93.92	0.01\\
93.93	0.01\\
93.94	0.01\\
93.95	0.01\\
93.96	0.01\\
93.97	0.01\\
93.98	0.01\\
93.99	0.01\\
94	0.01\\
94.01	0.01\\
94.02	0.01\\
94.03	0.01\\
94.04	0.01\\
94.05	0.01\\
94.06	0.01\\
94.07	0.01\\
94.08	0.01\\
94.09	0.01\\
94.1	0.01\\
94.11	0.01\\
94.12	0.01\\
94.13	0.01\\
94.14	0.01\\
94.15	0.01\\
94.16	0.01\\
94.17	0.01\\
94.18	0.01\\
94.19	0.01\\
94.2	0.01\\
94.21	0.01\\
94.22	0.01\\
94.23	0.01\\
94.24	0.01\\
94.25	0.01\\
94.26	0.01\\
94.27	0.01\\
94.28	0.01\\
94.29	0.01\\
94.3	0.01\\
94.31	0.01\\
94.32	0.01\\
94.33	0.01\\
94.34	0.01\\
94.35	0.01\\
94.36	0.01\\
94.37	0.01\\
94.38	0.01\\
94.39	0.01\\
94.4	0.01\\
94.41	0.01\\
94.42	0.01\\
94.43	0.01\\
94.44	0.01\\
94.45	0.01\\
94.46	0.01\\
94.47	0.01\\
94.48	0.01\\
94.49	0.01\\
94.5	0.01\\
94.51	0.01\\
94.52	0.01\\
94.53	0.01\\
94.54	0.01\\
94.55	0.01\\
94.56	0.01\\
94.57	0.01\\
94.58	0.01\\
94.59	0.01\\
94.6	0.01\\
94.61	0.01\\
94.62	0.01\\
94.63	0.01\\
94.64	0.01\\
94.65	0.01\\
94.66	0.01\\
94.67	0.01\\
94.68	0.01\\
94.69	0.01\\
94.7	0.01\\
94.71	0.01\\
94.72	0.01\\
94.73	0.01\\
94.74	0.01\\
94.75	0.01\\
94.76	0.01\\
94.77	0.01\\
94.78	0.01\\
94.79	0.01\\
94.8	0.01\\
94.81	0.01\\
94.82	0.01\\
94.83	0.01\\
94.84	0.01\\
94.85	0.01\\
94.86	0.01\\
94.87	0.01\\
94.88	0.01\\
94.89	0.01\\
94.9	0.01\\
94.91	0.01\\
94.92	0.01\\
94.93	0.01\\
94.94	0.01\\
94.95	0.01\\
94.96	0.01\\
94.97	0.01\\
94.98	0.01\\
94.99	0.01\\
95	0.01\\
95.01	0.01\\
95.02	0.01\\
95.03	0.01\\
95.04	0.01\\
95.05	0.01\\
95.06	0.01\\
95.07	0.01\\
95.08	0.01\\
95.09	0.01\\
95.1	0.01\\
95.11	0.01\\
95.12	0.01\\
95.13	0.01\\
95.14	0.01\\
95.15	0.01\\
95.16	0.01\\
95.17	0.01\\
95.18	0.01\\
95.19	0.01\\
95.2	0.01\\
95.21	0.01\\
95.22	0.01\\
95.23	0.01\\
95.24	0.01\\
95.25	0.01\\
95.26	0.01\\
95.27	0.01\\
95.28	0.01\\
95.29	0.01\\
95.3	0.01\\
95.31	0.01\\
95.32	0.01\\
95.33	0.01\\
95.34	0.01\\
95.35	0.01\\
95.36	0.01\\
95.37	0.01\\
95.38	0.01\\
95.39	0.01\\
95.4	0.01\\
95.41	0.01\\
95.42	0.01\\
95.43	0.01\\
95.44	0.01\\
95.45	0.01\\
95.46	0.01\\
95.47	0.01\\
95.48	0.01\\
95.49	0.01\\
95.5	0.01\\
95.51	0.01\\
95.52	0.01\\
95.53	0.01\\
95.54	0.01\\
95.55	0.01\\
95.56	0.01\\
95.57	0.01\\
95.58	0.01\\
95.59	0.01\\
95.6	0.01\\
95.61	0.01\\
95.62	0.01\\
95.63	0.01\\
95.64	0.01\\
95.65	0.01\\
95.66	0.01\\
95.67	0.01\\
95.68	0.01\\
95.69	0.01\\
95.7	0.01\\
95.71	0.01\\
95.72	0.01\\
95.73	0.01\\
95.74	0.01\\
95.75	0.01\\
95.76	0.01\\
95.77	0.01\\
95.78	0.01\\
95.79	0.01\\
95.8	0.01\\
95.81	0.01\\
95.82	0.01\\
95.83	0.01\\
95.84	0.01\\
95.85	0.01\\
95.86	0.01\\
95.87	0.01\\
95.88	0.01\\
95.89	0.01\\
95.9	0.01\\
95.91	0.01\\
95.92	0.01\\
95.93	0.01\\
95.94	0.01\\
95.95	0.01\\
95.96	0.01\\
95.97	0.01\\
95.98	0.01\\
95.99	0.01\\
96	0.01\\
96.01	0.01\\
96.02	0.01\\
96.03	0.01\\
96.04	0.01\\
96.05	0.01\\
96.06	0.01\\
96.07	0.01\\
96.08	0.01\\
96.09	0.01\\
96.1	0.01\\
96.11	0.01\\
96.12	0.01\\
96.13	0.01\\
96.14	0.01\\
96.15	0.01\\
96.16	0.01\\
96.17	0.01\\
96.18	0.01\\
96.19	0.01\\
96.2	0.01\\
96.21	0.01\\
96.22	0.01\\
96.23	0.01\\
96.24	0.01\\
96.25	0.01\\
96.26	0.01\\
96.27	0.01\\
96.28	0.01\\
96.29	0.01\\
96.3	0.01\\
96.31	0.01\\
96.32	0.01\\
96.33	0.01\\
96.34	0.01\\
96.35	0.01\\
96.36	0.01\\
96.37	0.01\\
96.38	0.01\\
96.39	0.01\\
96.4	0.01\\
96.41	0.01\\
96.42	0.01\\
96.43	0.01\\
96.44	0.01\\
96.45	0.01\\
96.46	0.01\\
96.47	0.01\\
96.48	0.01\\
96.49	0.01\\
96.5	0.01\\
96.51	0.01\\
96.52	0.01\\
96.53	0.01\\
96.54	0.01\\
96.55	0.01\\
96.56	0.01\\
96.57	0.01\\
96.58	0.01\\
96.59	0.01\\
96.6	0.01\\
96.61	0.01\\
96.62	0.01\\
96.63	0.01\\
96.64	0.01\\
96.65	0.01\\
96.66	0.01\\
96.67	0.01\\
96.68	0.01\\
96.69	0.01\\
96.7	0.01\\
96.71	0.01\\
96.72	0.01\\
96.73	0.01\\
96.74	0.01\\
96.75	0.01\\
96.76	0.01\\
96.77	0.01\\
96.78	0.01\\
96.79	0.01\\
96.8	0.01\\
96.81	0.01\\
96.82	0.01\\
96.83	0.01\\
96.84	0.01\\
96.85	0.01\\
96.86	0.01\\
96.87	0.01\\
96.88	0.01\\
96.89	0.01\\
96.9	0.01\\
96.91	0.01\\
96.92	0.01\\
96.93	0.01\\
96.94	0.01\\
96.95	0.01\\
96.96	0.01\\
96.97	0.01\\
96.98	0.01\\
96.99	0.01\\
97	0.01\\
97.01	0.01\\
97.02	0.01\\
97.03	0.01\\
97.04	0.01\\
97.05	0.01\\
97.06	0.01\\
97.07	0.01\\
97.08	0.01\\
97.09	0.01\\
97.1	0.01\\
97.11	0.01\\
97.12	0.01\\
97.13	0.01\\
97.14	0.01\\
97.15	0.01\\
97.16	0.01\\
97.17	0.01\\
97.18	0.01\\
97.19	0.01\\
97.2	0.01\\
97.21	0.01\\
97.22	0.01\\
97.23	0.01\\
97.24	0.01\\
97.25	0.01\\
97.26	0.01\\
97.27	0.01\\
97.28	0.01\\
97.29	0.01\\
97.3	0.01\\
97.31	0.01\\
97.32	0.01\\
97.33	0.01\\
97.34	0.01\\
97.35	0.01\\
97.36	0.01\\
97.37	0.01\\
97.38	0.01\\
97.39	0.01\\
97.4	0.01\\
97.41	0.01\\
97.42	0.01\\
97.43	0.01\\
97.44	0.01\\
97.45	0.01\\
97.46	0.01\\
97.47	0.01\\
97.48	0.01\\
97.49	0.01\\
97.5	0.01\\
97.51	0.01\\
97.52	0.01\\
97.53	0.01\\
97.54	0.01\\
97.55	0.01\\
97.56	0.01\\
97.57	0.01\\
97.58	0.01\\
97.59	0.01\\
97.6	0.01\\
97.61	0.01\\
97.62	0.01\\
97.63	0.01\\
97.64	0.01\\
97.65	0.01\\
97.66	0.01\\
97.67	0.01\\
97.68	0.01\\
97.69	0.01\\
97.7	0.01\\
97.71	0.01\\
97.72	0.01\\
97.73	0.01\\
97.74	0.01\\
97.75	0.01\\
97.76	0.01\\
97.77	0.01\\
97.78	0.01\\
97.79	0.01\\
97.8	0.01\\
97.81	0.01\\
97.82	0.01\\
97.83	0.01\\
97.84	0.01\\
97.85	0.01\\
97.86	0.01\\
97.87	0.01\\
97.88	0.01\\
97.89	0.01\\
97.9	0.01\\
97.91	0.01\\
97.92	0.01\\
97.93	0.01\\
97.94	0.01\\
97.95	0.01\\
97.96	0.01\\
97.97	0.01\\
97.98	0.01\\
97.99	0.01\\
98	0.01\\
98.01	0.01\\
98.02	0.01\\
98.03	0.01\\
98.04	0.01\\
98.05	0.01\\
98.06	0.01\\
98.07	0.01\\
98.08	0.01\\
98.09	0.01\\
98.1	0.01\\
98.11	0.01\\
98.12	0.01\\
98.13	0.01\\
98.14	0.01\\
98.15	0.01\\
98.16	0.01\\
98.17	0.01\\
98.18	0.01\\
98.19	0.01\\
98.2	0.01\\
98.21	0.01\\
98.22	0.01\\
98.23	0.01\\
98.24	0.01\\
98.25	0.01\\
98.26	0.01\\
98.27	0.01\\
98.28	0.01\\
98.29	0.01\\
98.3	0.01\\
98.31	0.01\\
98.32	0.01\\
98.33	0.01\\
98.34	0.01\\
98.35	0.01\\
98.36	0.01\\
98.37	0.01\\
98.38	0.01\\
98.39	0.01\\
98.4	0.01\\
98.41	0.01\\
98.42	0.01\\
98.43	0.01\\
98.44	0.01\\
98.45	0.01\\
98.46	0.01\\
98.47	0.01\\
98.48	0.01\\
98.49	0.01\\
98.5	0.01\\
98.51	0.01\\
98.52	0.01\\
98.53	0.01\\
98.54	0.01\\
98.55	0.01\\
98.56	0.01\\
98.57	0.01\\
98.58	0.01\\
98.59	0.01\\
98.6	0.01\\
98.61	0.01\\
98.62	0.01\\
98.63	0.01\\
98.64	0.01\\
98.65	0.01\\
98.66	0.01\\
98.67	0.01\\
98.68	0.01\\
98.69	0.01\\
98.7	0.01\\
98.71	0.01\\
98.72	0.01\\
98.73	0.01\\
98.74	0.01\\
98.75	0.01\\
98.76	0.01\\
98.77	0.01\\
98.78	0.01\\
98.79	0.01\\
98.8	0.01\\
98.81	0.01\\
98.82	0.01\\
98.83	0.01\\
98.84	0.01\\
98.85	0.01\\
98.86	0.01\\
98.87	0.01\\
98.88	0.01\\
98.89	0.01\\
98.9	0.01\\
98.91	0.01\\
98.92	0.01\\
98.93	0.01\\
98.94	0.01\\
98.95	0.01\\
98.96	0.01\\
98.97	0.01\\
98.98	0.01\\
98.99	0.01\\
99	0.01\\
99.01	0.01\\
99.02	0.01\\
99.03	0.01\\
99.04	0.01\\
99.05	0.01\\
99.06	0.01\\
99.07	0.01\\
99.08	0.01\\
99.09	0.01\\
99.1	0.01\\
99.11	0.01\\
99.12	0.01\\
99.13	0.01\\
99.14	0.01\\
99.15	0.01\\
99.16	0.01\\
99.17	0.01\\
99.18	0.01\\
99.19	0.01\\
99.2	0.01\\
99.21	0.01\\
99.22	0.01\\
99.23	0.01\\
99.24	0.01\\
99.25	0.01\\
99.26	0.01\\
99.27	0.01\\
99.28	0.01\\
99.29	0.01\\
99.3	0.01\\
99.31	0.01\\
99.32	0.01\\
99.33	0.01\\
99.34	0.01\\
99.35	0.01\\
99.36	0.01\\
99.37	0.01\\
99.38	0.01\\
99.39	0.01\\
99.4	0.01\\
99.41	0.01\\
99.42	0.01\\
99.43	0.01\\
99.44	0.01\\
99.45	0.01\\
99.46	0.01\\
99.47	0.01\\
99.48	0.01\\
99.49	0.01\\
99.5	0.01\\
99.51	0.01\\
99.52	0.01\\
99.53	0.01\\
99.54	0.01\\
99.55	0.01\\
99.56	0.01\\
99.57	0.01\\
99.58	0.01\\
99.59	0.01\\
99.6	0.01\\
99.61	0.01\\
99.62	0.01\\
99.63	0.01\\
99.64	0.01\\
99.65	0.01\\
99.66	0.01\\
99.67	0.01\\
99.68	0.01\\
99.69	0.01\\
99.7	0.01\\
99.71	0.01\\
99.72	0.01\\
99.73	0.01\\
99.74	0.01\\
99.75	0.01\\
99.76	0.01\\
99.77	0.01\\
99.78	0.01\\
99.79	0.01\\
99.8	0.01\\
99.81	0.01\\
99.82	0.01\\
99.83	0.01\\
99.84	0.01\\
99.85	0.01\\
99.86	0.01\\
99.87	0.01\\
99.88	0.01\\
99.89	0.01\\
99.9	0.01\\
99.91	0.01\\
99.92	0.01\\
99.93	0.01\\
99.94	0.01\\
99.95	0.01\\
99.96	0.01\\
99.97	0.01\\
99.98	0.01\\
99.99	0.01\\
100	0.01\\
};
\addlegendentry{$q=4$};

\end{axis}
\end{tikzpicture}%
  \caption{Continuous Time}
\end{subfigure}%
\hfill%
\begin{subfigure}{.45\linewidth}
  \centering
  \setlength\figureheight{\linewidth} 
  \setlength\figurewidth{\linewidth}
  \tikzsetnextfilename{dp_colorbar/dp_dscr_z15}
  % This file was created by matlab2tikz.
%
%The latest updates can be retrieved from
%  http://www.mathworks.com/matlabcentral/fileexchange/22022-matlab2tikz-matlab2tikz
%where you can also make suggestions and rate matlab2tikz.
%
\definecolor{mycolor1}{rgb}{0.00000,1.00000,0.14286}%
\definecolor{mycolor2}{rgb}{0.00000,1.00000,0.28571}%
\definecolor{mycolor3}{rgb}{0.00000,1.00000,0.42857}%
\definecolor{mycolor4}{rgb}{0.00000,1.00000,0.57143}%
\definecolor{mycolor5}{rgb}{0.00000,1.00000,0.71429}%
\definecolor{mycolor6}{rgb}{0.00000,1.00000,0.85714}%
\definecolor{mycolor7}{rgb}{0.00000,1.00000,1.00000}%
\definecolor{mycolor8}{rgb}{0.00000,0.87500,1.00000}%
\definecolor{mycolor9}{rgb}{0.00000,0.62500,1.00000}%
\definecolor{mycolor10}{rgb}{0.12500,0.00000,1.00000}%
\definecolor{mycolor11}{rgb}{0.25000,0.00000,1.00000}%
\definecolor{mycolor12}{rgb}{0.37500,0.00000,1.00000}%
\definecolor{mycolor13}{rgb}{0.50000,0.00000,1.00000}%
\definecolor{mycolor14}{rgb}{0.62500,0.00000,1.00000}%
\definecolor{mycolor15}{rgb}{0.75000,0.00000,1.00000}%
\definecolor{mycolor16}{rgb}{0.87500,0.00000,1.00000}%
\definecolor{mycolor17}{rgb}{1.00000,0.00000,1.00000}%
\definecolor{mycolor18}{rgb}{1.00000,0.00000,0.87500}%
\definecolor{mycolor19}{rgb}{1.00000,0.00000,0.62500}%
\definecolor{mycolor20}{rgb}{0.85714,0.00000,0.00000}%
\definecolor{mycolor21}{rgb}{0.71429,0.00000,0.00000}%
%
\begin{tikzpicture}[trim axis left, trim axis right]

\begin{axis}[%
width=\figurewidth,
height=\figureheight,
at={(0\figurewidth,0\figureheight)},
scale only axis,
point meta min=0,
point meta max=1,
every outer x axis line/.append style={black},
every x tick label/.append style={font=\color{black}},
xmin=0,
xmax=600,
every outer y axis line/.append style={black},
every y tick label/.append style={font=\color{black}},
ymin=0,
ymax=0.014,
axis background/.style={fill=white},
axis x line*=bottom,
axis y line*=left,
]
\addplot [color=green,solid,forget plot]
  table[row sep=crcr]{%
1	0\\
2	0\\
3	0\\
4	0\\
5	0\\
6	0\\
7	0\\
8	0\\
9	0\\
10	0\\
11	0\\
12	0\\
13	0\\
14	0\\
15	0\\
16	0\\
17	0\\
18	0\\
19	0\\
20	0\\
21	0\\
22	0\\
23	0\\
24	0\\
25	0\\
26	0\\
27	0\\
28	0\\
29	0\\
30	0\\
31	0\\
32	0\\
33	0\\
34	0\\
35	0\\
36	0\\
37	0\\
38	0\\
39	0\\
40	0\\
41	0\\
42	0\\
43	0\\
44	0\\
45	0\\
46	0\\
47	0\\
48	0\\
49	0\\
50	0\\
51	0\\
52	0\\
53	0\\
54	0\\
55	0\\
56	0\\
57	0\\
58	0\\
59	0\\
60	0\\
61	0\\
62	0\\
63	0\\
64	0\\
65	0\\
66	0\\
67	0\\
68	0\\
69	0\\
70	0\\
71	0\\
72	0\\
73	0\\
74	0\\
75	0\\
76	0\\
77	0\\
78	0\\
79	0\\
80	0\\
81	0\\
82	0\\
83	0\\
84	0\\
85	0\\
86	0\\
87	0\\
88	0\\
89	0\\
90	0\\
91	0\\
92	0\\
93	0\\
94	0\\
95	0\\
96	0\\
97	0\\
98	0\\
99	0\\
100	0\\
101	0\\
102	0\\
103	0\\
104	0\\
105	0\\
106	0\\
107	0\\
108	0\\
109	0\\
110	0\\
111	0\\
112	0\\
113	0\\
114	0\\
115	0\\
116	0\\
117	0\\
118	0\\
119	0\\
120	0\\
121	0\\
122	0\\
123	0\\
124	0\\
125	0\\
126	0\\
127	0\\
128	0\\
129	0\\
130	0\\
131	0\\
132	0\\
133	0\\
134	0\\
135	0\\
136	0\\
137	0\\
138	0\\
139	0\\
140	0\\
141	0\\
142	0\\
143	0\\
144	0\\
145	0\\
146	0\\
147	0\\
148	0\\
149	0\\
150	0\\
151	0\\
152	0\\
153	0\\
154	0\\
155	0\\
156	0\\
157	0\\
158	0\\
159	0\\
160	0\\
161	0\\
162	0\\
163	0\\
164	0\\
165	0\\
166	0\\
167	0\\
168	0\\
169	0\\
170	0\\
171	0\\
172	0\\
173	0\\
174	0\\
175	0\\
176	0\\
177	0\\
178	0\\
179	0\\
180	0\\
181	0\\
182	0\\
183	0\\
184	0\\
185	0\\
186	0\\
187	0\\
188	0\\
189	0\\
190	0\\
191	0\\
192	0\\
193	0\\
194	0\\
195	0\\
196	0\\
197	0\\
198	0\\
199	0\\
200	0\\
201	0\\
202	0\\
203	0\\
204	0\\
205	0\\
206	0\\
207	0\\
208	0\\
209	0\\
210	0\\
211	0\\
212	0\\
213	0\\
214	0\\
215	0\\
216	0\\
217	0\\
218	0\\
219	0\\
220	0\\
221	0\\
222	0\\
223	0\\
224	0\\
225	0\\
226	0\\
227	0\\
228	0\\
229	0\\
230	0\\
231	0\\
232	0\\
233	0\\
234	0\\
235	0\\
236	0\\
237	0\\
238	0\\
239	0\\
240	0\\
241	0\\
242	0\\
243	0\\
244	0\\
245	0\\
246	0\\
247	0\\
248	0\\
249	0\\
250	0\\
251	0\\
252	0\\
253	0\\
254	0\\
255	0\\
256	0\\
257	0\\
258	0\\
259	0\\
260	0\\
261	0\\
262	0\\
263	0\\
264	0\\
265	0\\
266	0\\
267	0\\
268	0\\
269	0\\
270	0\\
271	0\\
272	0\\
273	0\\
274	0\\
275	0\\
276	0\\
277	0\\
278	0\\
279	0\\
280	0\\
281	0\\
282	0\\
283	0\\
284	0\\
285	0\\
286	0\\
287	0\\
288	0\\
289	0\\
290	0\\
291	0\\
292	0\\
293	0\\
294	0\\
295	0\\
296	0\\
297	0\\
298	0\\
299	0\\
300	0\\
301	0\\
302	0\\
303	0\\
304	0\\
305	0\\
306	0\\
307	0\\
308	0\\
309	0\\
310	0\\
311	0\\
312	0\\
313	0\\
314	0\\
315	0\\
316	0\\
317	0\\
318	0\\
319	0\\
320	0\\
321	0\\
322	0\\
323	0\\
324	0\\
325	0\\
326	0\\
327	0\\
328	0\\
329	0\\
330	0\\
331	0\\
332	0\\
333	0\\
334	0\\
335	0\\
336	0\\
337	0\\
338	0\\
339	0\\
340	0\\
341	0\\
342	0\\
343	0\\
344	0\\
345	0\\
346	0\\
347	0\\
348	0\\
349	0\\
350	0\\
351	0\\
352	0\\
353	0\\
354	0\\
355	0\\
356	0\\
357	0\\
358	0\\
359	0\\
360	0\\
361	0\\
362	0\\
363	0\\
364	0\\
365	0\\
366	0\\
367	0\\
368	0\\
369	0\\
370	0\\
371	0\\
372	0\\
373	0\\
374	0\\
375	0\\
376	0\\
377	0\\
378	0\\
379	0\\
380	0\\
381	0\\
382	0\\
383	0\\
384	0\\
385	0\\
386	0\\
387	0\\
388	0\\
389	0\\
390	0\\
391	0\\
392	0\\
393	0\\
394	0\\
395	0\\
396	0\\
397	0\\
398	0\\
399	0\\
400	0\\
401	0\\
402	0\\
403	0\\
404	0\\
405	0\\
406	0\\
407	0\\
408	0\\
409	0\\
410	0\\
411	0\\
412	0\\
413	0\\
414	0\\
415	0\\
416	0\\
417	0\\
418	0\\
419	0\\
420	0\\
421	0\\
422	0\\
423	0\\
424	0\\
425	0\\
426	0\\
427	0\\
428	0\\
429	0\\
430	0\\
431	0\\
432	0\\
433	0\\
434	0\\
435	0\\
436	0\\
437	0\\
438	0\\
439	0\\
440	0\\
441	0\\
442	0\\
443	0\\
444	0\\
445	0\\
446	0\\
447	0\\
448	0\\
449	0\\
450	0\\
451	0\\
452	0\\
453	0\\
454	0\\
455	0\\
456	0\\
457	0\\
458	0\\
459	0\\
460	0\\
461	0\\
462	0\\
463	0\\
464	0\\
465	0\\
466	0\\
467	0\\
468	0\\
469	0\\
470	0\\
471	0\\
472	0\\
473	0\\
474	0\\
475	0\\
476	0\\
477	0\\
478	0\\
479	0\\
480	0\\
481	0\\
482	0\\
483	0\\
484	0\\
485	0\\
486	0\\
487	0\\
488	0\\
489	0\\
490	0\\
491	0\\
492	0\\
493	0\\
494	0\\
495	0\\
496	0\\
497	0\\
498	0\\
499	0\\
500	0\\
501	0\\
502	0\\
503	0\\
504	0\\
505	0\\
506	0\\
507	0\\
508	0\\
509	0\\
510	0\\
511	0\\
512	0\\
513	0\\
514	0\\
515	0\\
516	0\\
517	0\\
518	0\\
519	0\\
520	0\\
521	0\\
522	0\\
523	0\\
524	0\\
525	0\\
526	0\\
527	0\\
528	0\\
529	0\\
530	0\\
531	0\\
532	0\\
533	0\\
534	0\\
535	0\\
536	0\\
537	0\\
538	0\\
539	0\\
540	0\\
541	0\\
542	0\\
543	0\\
544	0\\
545	0\\
546	0\\
547	0\\
548	0\\
549	0\\
550	0\\
551	0\\
552	0\\
553	0\\
554	0\\
555	0\\
556	0\\
557	0\\
558	0\\
559	0\\
560	0\\
561	0\\
562	0\\
563	0\\
564	0\\
565	0\\
566	0\\
567	0\\
568	0\\
569	0\\
570	0\\
571	0\\
572	0\\
573	0\\
574	0\\
575	0\\
576	0\\
577	0\\
578	0\\
579	0\\
580	0\\
581	0\\
582	0\\
583	0\\
584	0\\
585	0\\
586	0\\
587	0\\
588	0\\
589	0\\
590	0\\
591	0\\
592	0\\
593	0\\
594	0\\
595	0\\
596	0\\
597	0\\
598	0\\
599	0\\
600	0\\
};
\addplot [color=mycolor1,solid,forget plot]
  table[row sep=crcr]{%
1	0\\
2	0\\
3	0\\
4	0\\
5	0\\
6	0\\
7	0\\
8	0\\
9	0\\
10	0\\
11	0\\
12	0\\
13	0\\
14	0\\
15	0\\
16	0\\
17	0\\
18	0\\
19	0\\
20	0\\
21	0\\
22	0\\
23	0\\
24	0\\
25	0\\
26	0\\
27	0\\
28	0\\
29	0\\
30	0\\
31	0\\
32	0\\
33	0\\
34	0\\
35	0\\
36	0\\
37	0\\
38	0\\
39	0\\
40	0\\
41	0\\
42	0\\
43	0\\
44	0\\
45	0\\
46	0\\
47	0\\
48	0\\
49	0\\
50	0\\
51	0\\
52	0\\
53	0\\
54	0\\
55	0\\
56	0\\
57	0\\
58	0\\
59	0\\
60	0\\
61	0\\
62	0\\
63	0\\
64	0\\
65	0\\
66	0\\
67	0\\
68	0\\
69	0\\
70	0\\
71	0\\
72	0\\
73	0\\
74	0\\
75	0\\
76	0\\
77	0\\
78	0\\
79	0\\
80	0\\
81	0\\
82	0\\
83	0\\
84	0\\
85	0\\
86	0\\
87	0\\
88	0\\
89	0\\
90	0\\
91	0\\
92	0\\
93	0\\
94	0\\
95	0\\
96	0\\
97	0\\
98	0\\
99	0\\
100	0\\
101	0\\
102	0\\
103	0\\
104	0\\
105	0\\
106	0\\
107	0\\
108	0\\
109	0\\
110	0\\
111	0\\
112	0\\
113	0\\
114	0\\
115	0\\
116	0\\
117	0\\
118	0\\
119	0\\
120	0\\
121	0\\
122	0\\
123	0\\
124	0\\
125	0\\
126	0\\
127	0\\
128	0\\
129	0\\
130	0\\
131	0\\
132	0\\
133	0\\
134	0\\
135	0\\
136	0\\
137	0\\
138	0\\
139	0\\
140	0\\
141	0\\
142	0\\
143	0\\
144	0\\
145	0\\
146	0\\
147	0\\
148	0\\
149	0\\
150	0\\
151	0\\
152	0\\
153	0\\
154	0\\
155	0\\
156	0\\
157	0\\
158	0\\
159	0\\
160	0\\
161	0\\
162	0\\
163	0\\
164	0\\
165	0\\
166	0\\
167	0\\
168	0\\
169	0\\
170	0\\
171	0\\
172	0\\
173	0\\
174	0\\
175	0\\
176	0\\
177	0\\
178	0\\
179	0\\
180	0\\
181	0\\
182	0\\
183	0\\
184	0\\
185	0\\
186	0\\
187	0\\
188	0\\
189	0\\
190	0\\
191	0\\
192	0\\
193	0\\
194	0\\
195	0\\
196	0\\
197	0\\
198	0\\
199	0\\
200	0\\
201	0\\
202	0\\
203	0\\
204	0\\
205	0\\
206	0\\
207	0\\
208	0\\
209	0\\
210	0\\
211	0\\
212	0\\
213	0\\
214	0\\
215	0\\
216	0\\
217	0\\
218	0\\
219	0\\
220	0\\
221	0\\
222	0\\
223	0\\
224	0\\
225	0\\
226	0\\
227	0\\
228	0\\
229	0\\
230	0\\
231	0\\
232	0\\
233	0\\
234	0\\
235	0\\
236	0\\
237	0\\
238	0\\
239	0\\
240	0\\
241	0\\
242	0\\
243	0\\
244	0\\
245	0\\
246	0\\
247	0\\
248	0\\
249	0\\
250	0\\
251	0\\
252	0\\
253	0\\
254	0\\
255	0\\
256	0\\
257	0\\
258	0\\
259	0\\
260	0\\
261	0\\
262	0\\
263	0\\
264	0\\
265	0\\
266	0\\
267	0\\
268	0\\
269	0\\
270	0\\
271	0\\
272	0\\
273	0\\
274	0\\
275	0\\
276	0\\
277	0\\
278	0\\
279	0\\
280	0\\
281	0\\
282	0\\
283	0\\
284	0\\
285	0\\
286	0\\
287	0\\
288	0\\
289	0\\
290	0\\
291	0\\
292	0\\
293	0\\
294	0\\
295	0\\
296	0\\
297	0\\
298	0\\
299	0\\
300	0\\
301	0\\
302	0\\
303	0\\
304	0\\
305	0\\
306	0\\
307	0\\
308	0\\
309	0\\
310	0\\
311	0\\
312	0\\
313	0\\
314	0\\
315	0\\
316	0\\
317	0\\
318	0\\
319	0\\
320	0\\
321	0\\
322	0\\
323	0\\
324	0\\
325	0\\
326	0\\
327	0\\
328	0\\
329	0\\
330	0\\
331	0\\
332	0\\
333	0\\
334	0\\
335	0\\
336	0\\
337	0\\
338	0\\
339	0\\
340	0\\
341	0\\
342	0\\
343	0\\
344	0\\
345	0\\
346	0\\
347	0\\
348	0\\
349	0\\
350	0\\
351	0\\
352	0\\
353	0\\
354	0\\
355	0\\
356	0\\
357	0\\
358	0\\
359	0\\
360	0\\
361	0\\
362	0\\
363	0\\
364	0\\
365	0\\
366	0\\
367	0\\
368	0\\
369	0\\
370	0\\
371	0\\
372	0\\
373	0\\
374	0\\
375	0\\
376	0\\
377	0\\
378	0\\
379	0\\
380	0\\
381	0\\
382	0\\
383	0\\
384	0\\
385	0\\
386	0\\
387	0\\
388	0\\
389	0\\
390	0\\
391	0\\
392	0\\
393	0\\
394	0\\
395	0\\
396	0\\
397	0\\
398	0\\
399	0\\
400	0\\
401	0\\
402	0\\
403	0\\
404	0\\
405	0\\
406	0\\
407	0\\
408	0\\
409	0\\
410	0\\
411	0\\
412	0\\
413	0\\
414	0\\
415	0\\
416	0\\
417	0\\
418	0\\
419	0\\
420	0\\
421	0\\
422	0\\
423	0\\
424	0\\
425	0\\
426	0\\
427	0\\
428	0\\
429	0\\
430	0\\
431	0\\
432	0\\
433	0\\
434	0\\
435	0\\
436	0\\
437	0\\
438	0\\
439	0\\
440	0\\
441	0\\
442	0\\
443	0\\
444	0\\
445	0\\
446	0\\
447	0\\
448	0\\
449	0\\
450	0\\
451	0\\
452	0\\
453	0\\
454	0\\
455	0\\
456	0\\
457	0\\
458	0\\
459	0\\
460	0\\
461	0\\
462	0\\
463	0\\
464	0\\
465	0\\
466	0\\
467	0\\
468	0\\
469	0\\
470	0\\
471	0\\
472	0\\
473	0\\
474	0\\
475	0\\
476	0\\
477	0\\
478	0\\
479	0\\
480	0\\
481	0\\
482	0\\
483	0\\
484	0\\
485	0\\
486	0\\
487	0\\
488	0\\
489	0\\
490	0\\
491	0\\
492	0\\
493	0\\
494	0\\
495	0\\
496	0\\
497	0\\
498	0\\
499	0\\
500	0\\
501	0\\
502	0\\
503	0\\
504	0\\
505	0\\
506	0\\
507	0\\
508	0\\
509	0\\
510	0\\
511	0\\
512	0\\
513	0\\
514	0\\
515	0\\
516	0\\
517	0\\
518	0\\
519	0\\
520	0\\
521	0\\
522	0\\
523	0\\
524	0\\
525	0\\
526	0\\
527	0\\
528	0\\
529	0\\
530	0\\
531	0\\
532	0\\
533	0\\
534	0\\
535	0\\
536	0\\
537	0\\
538	0\\
539	0\\
540	0\\
541	0\\
542	0\\
543	0\\
544	0\\
545	0\\
546	0\\
547	0\\
548	0\\
549	0\\
550	0\\
551	0\\
552	0\\
553	0\\
554	0\\
555	0\\
556	0\\
557	0\\
558	0\\
559	0\\
560	0\\
561	0\\
562	0\\
563	0\\
564	0\\
565	0\\
566	0\\
567	0\\
568	0\\
569	0\\
570	0\\
571	0\\
572	0\\
573	0\\
574	0\\
575	0\\
576	0\\
577	0\\
578	0\\
579	0\\
580	0\\
581	0\\
582	0\\
583	0\\
584	0\\
585	0\\
586	0\\
587	0\\
588	0\\
589	0\\
590	0\\
591	0\\
592	0\\
593	0\\
594	0\\
595	0\\
596	0\\
597	0\\
598	0\\
599	0\\
600	0\\
};
\addplot [color=mycolor2,solid,forget plot]
  table[row sep=crcr]{%
1	0\\
2	0\\
3	0\\
4	0\\
5	0\\
6	0\\
7	0\\
8	0\\
9	0\\
10	0\\
11	0\\
12	0\\
13	0\\
14	0\\
15	0\\
16	0\\
17	0\\
18	0\\
19	0\\
20	0\\
21	0\\
22	0\\
23	0\\
24	0\\
25	0\\
26	0\\
27	0\\
28	0\\
29	0\\
30	0\\
31	0\\
32	0\\
33	0\\
34	0\\
35	0\\
36	0\\
37	0\\
38	0\\
39	0\\
40	0\\
41	0\\
42	0\\
43	0\\
44	0\\
45	0\\
46	0\\
47	0\\
48	0\\
49	0\\
50	0\\
51	0\\
52	0\\
53	0\\
54	0\\
55	0\\
56	0\\
57	0\\
58	0\\
59	0\\
60	0\\
61	0\\
62	0\\
63	0\\
64	0\\
65	0\\
66	0\\
67	0\\
68	0\\
69	0\\
70	0\\
71	0\\
72	0\\
73	0\\
74	0\\
75	0\\
76	0\\
77	0\\
78	0\\
79	0\\
80	0\\
81	0\\
82	0\\
83	0\\
84	0\\
85	0\\
86	0\\
87	0\\
88	0\\
89	0\\
90	0\\
91	0\\
92	0\\
93	0\\
94	0\\
95	0\\
96	0\\
97	0\\
98	0\\
99	0\\
100	0\\
101	0\\
102	0\\
103	0\\
104	0\\
105	0\\
106	0\\
107	0\\
108	0\\
109	0\\
110	0\\
111	0\\
112	0\\
113	0\\
114	0\\
115	0\\
116	0\\
117	0\\
118	0\\
119	0\\
120	0\\
121	0\\
122	0\\
123	0\\
124	0\\
125	0\\
126	0\\
127	0\\
128	0\\
129	0\\
130	0\\
131	0\\
132	0\\
133	0\\
134	0\\
135	0\\
136	0\\
137	0\\
138	0\\
139	0\\
140	0\\
141	0\\
142	0\\
143	0\\
144	0\\
145	0\\
146	0\\
147	0\\
148	0\\
149	0\\
150	0\\
151	0\\
152	0\\
153	0\\
154	0\\
155	0\\
156	0\\
157	0\\
158	0\\
159	0\\
160	0\\
161	0\\
162	0\\
163	0\\
164	0\\
165	0\\
166	0\\
167	0\\
168	0\\
169	0\\
170	0\\
171	0\\
172	0\\
173	0\\
174	0\\
175	0\\
176	0\\
177	0\\
178	0\\
179	0\\
180	0\\
181	0\\
182	0\\
183	0\\
184	0\\
185	0\\
186	0\\
187	0\\
188	0\\
189	0\\
190	0\\
191	0\\
192	0\\
193	0\\
194	0\\
195	0\\
196	0\\
197	0\\
198	0\\
199	0\\
200	0\\
201	0\\
202	0\\
203	0\\
204	0\\
205	0\\
206	0\\
207	0\\
208	0\\
209	0\\
210	0\\
211	0\\
212	0\\
213	0\\
214	0\\
215	0\\
216	0\\
217	0\\
218	0\\
219	0\\
220	0\\
221	0\\
222	0\\
223	0\\
224	0\\
225	0\\
226	0\\
227	0\\
228	0\\
229	0\\
230	0\\
231	0\\
232	0\\
233	0\\
234	0\\
235	0\\
236	0\\
237	0\\
238	0\\
239	0\\
240	0\\
241	0\\
242	0\\
243	0\\
244	0\\
245	0\\
246	0\\
247	0\\
248	0\\
249	0\\
250	0\\
251	0\\
252	0\\
253	0\\
254	0\\
255	0\\
256	0\\
257	0\\
258	0\\
259	0\\
260	0\\
261	0\\
262	0\\
263	0\\
264	0\\
265	0\\
266	0\\
267	0\\
268	0\\
269	0\\
270	0\\
271	0\\
272	0\\
273	0\\
274	0\\
275	0\\
276	0\\
277	0\\
278	0\\
279	0\\
280	0\\
281	0\\
282	0\\
283	0\\
284	0\\
285	0\\
286	0\\
287	0\\
288	0\\
289	0\\
290	0\\
291	0\\
292	0\\
293	0\\
294	0\\
295	0\\
296	0\\
297	0\\
298	0\\
299	0\\
300	0\\
301	0\\
302	0\\
303	0\\
304	0\\
305	0\\
306	0\\
307	0\\
308	0\\
309	0\\
310	0\\
311	0\\
312	0\\
313	0\\
314	0\\
315	0\\
316	0\\
317	0\\
318	0\\
319	0\\
320	0\\
321	0\\
322	0\\
323	0\\
324	0\\
325	0\\
326	0\\
327	0\\
328	0\\
329	0\\
330	0\\
331	0\\
332	0\\
333	0\\
334	0\\
335	0\\
336	0\\
337	0\\
338	0\\
339	0\\
340	0\\
341	0\\
342	0\\
343	0\\
344	0\\
345	0\\
346	0\\
347	0\\
348	0\\
349	0\\
350	0\\
351	0\\
352	0\\
353	0\\
354	0\\
355	0\\
356	0\\
357	0\\
358	0\\
359	0\\
360	0\\
361	0\\
362	0\\
363	0\\
364	0\\
365	0\\
366	0\\
367	0\\
368	0\\
369	0\\
370	0\\
371	0\\
372	0\\
373	0\\
374	0\\
375	0\\
376	0\\
377	0\\
378	0\\
379	0\\
380	0\\
381	0\\
382	0\\
383	0\\
384	0\\
385	0\\
386	0\\
387	0\\
388	0\\
389	0\\
390	0\\
391	0\\
392	0\\
393	0\\
394	0\\
395	0\\
396	0\\
397	0\\
398	0\\
399	0\\
400	0\\
401	0\\
402	0\\
403	0\\
404	0\\
405	0\\
406	0\\
407	0\\
408	0\\
409	0\\
410	0\\
411	0\\
412	0\\
413	0\\
414	0\\
415	0\\
416	0\\
417	0\\
418	0\\
419	0\\
420	0\\
421	0\\
422	0\\
423	0\\
424	0\\
425	0\\
426	0\\
427	0\\
428	0\\
429	0\\
430	0\\
431	0\\
432	0\\
433	0\\
434	0\\
435	0\\
436	0\\
437	0\\
438	0\\
439	0\\
440	0\\
441	0\\
442	0\\
443	0\\
444	0\\
445	0\\
446	0\\
447	0\\
448	0\\
449	0\\
450	0\\
451	0\\
452	0\\
453	0\\
454	0\\
455	0\\
456	0\\
457	0\\
458	0\\
459	0\\
460	0\\
461	0\\
462	0\\
463	0\\
464	0\\
465	0\\
466	0\\
467	0\\
468	0\\
469	0\\
470	0\\
471	0\\
472	0\\
473	0\\
474	0\\
475	0\\
476	0\\
477	0\\
478	0\\
479	0\\
480	0\\
481	0\\
482	0\\
483	0\\
484	0\\
485	0\\
486	0\\
487	0\\
488	0\\
489	0\\
490	0\\
491	0\\
492	0\\
493	0\\
494	0\\
495	0\\
496	0\\
497	0\\
498	0\\
499	0\\
500	0\\
501	0\\
502	0\\
503	0\\
504	0\\
505	0\\
506	0\\
507	0\\
508	0\\
509	0\\
510	0\\
511	0\\
512	0\\
513	0\\
514	0\\
515	0\\
516	0\\
517	0\\
518	0\\
519	0\\
520	0\\
521	0\\
522	0\\
523	0\\
524	0\\
525	0\\
526	0\\
527	0\\
528	0\\
529	0\\
530	0\\
531	0\\
532	0\\
533	0\\
534	0\\
535	0\\
536	0\\
537	0\\
538	0\\
539	0\\
540	0\\
541	0\\
542	0\\
543	0\\
544	0\\
545	0\\
546	0\\
547	0\\
548	0\\
549	0\\
550	0\\
551	0\\
552	0\\
553	0\\
554	0\\
555	0\\
556	0\\
557	0\\
558	0\\
559	0\\
560	0\\
561	0\\
562	0\\
563	0\\
564	0\\
565	0\\
566	0\\
567	0\\
568	0\\
569	0\\
570	0\\
571	0\\
572	0\\
573	0\\
574	0\\
575	0\\
576	0\\
577	0\\
578	0\\
579	0\\
580	0\\
581	0\\
582	0\\
583	0\\
584	0\\
585	0\\
586	0\\
587	0\\
588	0\\
589	0\\
590	0\\
591	0\\
592	0\\
593	0\\
594	0\\
595	0\\
596	0\\
597	0\\
598	0\\
599	0\\
600	0\\
};
\addplot [color=mycolor3,solid,forget plot]
  table[row sep=crcr]{%
1	0\\
2	0\\
3	0\\
4	0\\
5	0\\
6	0\\
7	0\\
8	0\\
9	0\\
10	0\\
11	0\\
12	0\\
13	0\\
14	0\\
15	0\\
16	0\\
17	0\\
18	0\\
19	0\\
20	0\\
21	0\\
22	0\\
23	0\\
24	0\\
25	0\\
26	0\\
27	0\\
28	0\\
29	0\\
30	0\\
31	0\\
32	0\\
33	0\\
34	0\\
35	0\\
36	0\\
37	0\\
38	0\\
39	0\\
40	0\\
41	0\\
42	0\\
43	0\\
44	0\\
45	0\\
46	0\\
47	0\\
48	0\\
49	0\\
50	0\\
51	0\\
52	0\\
53	0\\
54	0\\
55	0\\
56	0\\
57	0\\
58	0\\
59	0\\
60	0\\
61	0\\
62	0\\
63	0\\
64	0\\
65	0\\
66	0\\
67	0\\
68	0\\
69	0\\
70	0\\
71	0\\
72	0\\
73	0\\
74	0\\
75	0\\
76	0\\
77	0\\
78	0\\
79	0\\
80	0\\
81	0\\
82	0\\
83	0\\
84	0\\
85	0\\
86	0\\
87	0\\
88	0\\
89	0\\
90	0\\
91	0\\
92	0\\
93	0\\
94	0\\
95	0\\
96	0\\
97	0\\
98	0\\
99	0\\
100	0\\
101	0\\
102	0\\
103	0\\
104	0\\
105	0\\
106	0\\
107	0\\
108	0\\
109	0\\
110	0\\
111	0\\
112	0\\
113	0\\
114	0\\
115	0\\
116	0\\
117	0\\
118	0\\
119	0\\
120	0\\
121	0\\
122	0\\
123	0\\
124	0\\
125	0\\
126	0\\
127	0\\
128	0\\
129	0\\
130	0\\
131	0\\
132	0\\
133	0\\
134	0\\
135	0\\
136	0\\
137	0\\
138	0\\
139	0\\
140	0\\
141	0\\
142	0\\
143	0\\
144	0\\
145	0\\
146	0\\
147	0\\
148	0\\
149	0\\
150	0\\
151	0\\
152	0\\
153	0\\
154	0\\
155	0\\
156	0\\
157	0\\
158	0\\
159	0\\
160	0\\
161	0\\
162	0\\
163	0\\
164	0\\
165	0\\
166	0\\
167	0\\
168	0\\
169	0\\
170	0\\
171	0\\
172	0\\
173	0\\
174	0\\
175	0\\
176	0\\
177	0\\
178	0\\
179	0\\
180	0\\
181	0\\
182	0\\
183	0\\
184	0\\
185	0\\
186	0\\
187	0\\
188	0\\
189	0\\
190	0\\
191	0\\
192	0\\
193	0\\
194	0\\
195	0\\
196	0\\
197	0\\
198	0\\
199	0\\
200	0\\
201	0\\
202	0\\
203	0\\
204	0\\
205	0\\
206	0\\
207	0\\
208	0\\
209	0\\
210	0\\
211	0\\
212	0\\
213	0\\
214	0\\
215	0\\
216	0\\
217	0\\
218	0\\
219	0\\
220	0\\
221	0\\
222	0\\
223	0\\
224	0\\
225	0\\
226	0\\
227	0\\
228	0\\
229	0\\
230	0\\
231	0\\
232	0\\
233	0\\
234	0\\
235	0\\
236	0\\
237	0\\
238	0\\
239	0\\
240	0\\
241	0\\
242	0\\
243	0\\
244	0\\
245	0\\
246	0\\
247	0\\
248	0\\
249	0\\
250	0\\
251	0\\
252	0\\
253	0\\
254	0\\
255	0\\
256	0\\
257	0\\
258	0\\
259	0\\
260	0\\
261	0\\
262	0\\
263	0\\
264	0\\
265	0\\
266	0\\
267	0\\
268	0\\
269	0\\
270	0\\
271	0\\
272	0\\
273	0\\
274	0\\
275	0\\
276	0\\
277	0\\
278	0\\
279	0\\
280	0\\
281	0\\
282	0\\
283	0\\
284	0\\
285	0\\
286	0\\
287	0\\
288	0\\
289	0\\
290	0\\
291	0\\
292	0\\
293	0\\
294	0\\
295	0\\
296	0\\
297	0\\
298	0\\
299	0\\
300	0\\
301	0\\
302	0\\
303	0\\
304	0\\
305	0\\
306	0\\
307	0\\
308	0\\
309	0\\
310	0\\
311	0\\
312	0\\
313	0\\
314	0\\
315	0\\
316	0\\
317	0\\
318	0\\
319	0\\
320	0\\
321	0\\
322	0\\
323	0\\
324	0\\
325	0\\
326	0\\
327	0\\
328	0\\
329	0\\
330	0\\
331	0\\
332	0\\
333	0\\
334	0\\
335	0\\
336	0\\
337	0\\
338	0\\
339	0\\
340	0\\
341	0\\
342	0\\
343	0\\
344	0\\
345	0\\
346	0\\
347	0\\
348	0\\
349	0\\
350	0\\
351	0\\
352	0\\
353	0\\
354	0\\
355	0\\
356	0\\
357	0\\
358	0\\
359	0\\
360	0\\
361	0\\
362	0\\
363	0\\
364	0\\
365	0\\
366	0\\
367	0\\
368	0\\
369	0\\
370	0\\
371	0\\
372	0\\
373	0\\
374	0\\
375	0\\
376	0\\
377	0\\
378	0\\
379	0\\
380	0\\
381	0\\
382	0\\
383	0\\
384	0\\
385	0\\
386	0\\
387	0\\
388	0\\
389	0\\
390	0\\
391	0\\
392	0\\
393	0\\
394	0\\
395	0\\
396	0\\
397	0\\
398	0\\
399	0\\
400	0\\
401	0\\
402	0\\
403	0\\
404	0\\
405	0\\
406	0\\
407	0\\
408	0\\
409	0\\
410	0\\
411	0\\
412	0\\
413	0\\
414	0\\
415	0\\
416	0\\
417	0\\
418	0\\
419	0\\
420	0\\
421	0\\
422	0\\
423	0\\
424	0\\
425	0\\
426	0\\
427	0\\
428	0\\
429	0\\
430	0\\
431	0\\
432	0\\
433	0\\
434	0\\
435	0\\
436	0\\
437	0\\
438	0\\
439	0\\
440	0\\
441	0\\
442	0\\
443	0\\
444	0\\
445	0\\
446	0\\
447	0\\
448	0\\
449	0\\
450	0\\
451	0\\
452	0\\
453	0\\
454	0\\
455	0\\
456	0\\
457	0\\
458	0\\
459	0\\
460	0\\
461	0\\
462	0\\
463	0\\
464	0\\
465	0\\
466	0\\
467	0\\
468	0\\
469	0\\
470	0\\
471	0\\
472	0\\
473	0\\
474	0\\
475	0\\
476	0\\
477	0\\
478	0\\
479	0\\
480	0\\
481	0\\
482	0\\
483	0\\
484	0\\
485	0\\
486	0\\
487	0\\
488	0\\
489	0\\
490	0\\
491	0\\
492	0\\
493	0\\
494	0\\
495	0\\
496	0\\
497	0\\
498	0\\
499	0\\
500	0\\
501	0\\
502	0\\
503	0\\
504	0\\
505	0\\
506	0\\
507	0\\
508	0\\
509	0\\
510	0\\
511	0\\
512	0\\
513	0\\
514	0\\
515	0\\
516	0\\
517	0\\
518	0\\
519	0\\
520	0\\
521	0\\
522	0\\
523	0\\
524	0\\
525	0\\
526	0\\
527	0\\
528	0\\
529	0\\
530	0\\
531	0\\
532	0\\
533	0\\
534	0\\
535	0\\
536	0\\
537	0\\
538	0\\
539	0\\
540	0\\
541	0\\
542	0\\
543	0\\
544	0\\
545	0\\
546	0\\
547	0\\
548	0\\
549	0\\
550	0\\
551	0\\
552	0\\
553	0\\
554	0\\
555	0\\
556	0\\
557	0\\
558	0\\
559	0\\
560	0\\
561	0\\
562	0\\
563	0\\
564	0\\
565	0\\
566	0\\
567	0\\
568	0\\
569	0\\
570	0\\
571	0\\
572	0\\
573	0\\
574	0\\
575	0\\
576	0\\
577	0\\
578	0\\
579	0\\
580	0\\
581	0\\
582	0\\
583	0\\
584	0\\
585	0\\
586	0\\
587	0\\
588	0\\
589	0\\
590	0\\
591	0\\
592	0\\
593	0\\
594	0\\
595	0\\
596	0\\
597	0\\
598	0\\
599	0\\
600	0\\
};
\addplot [color=mycolor4,solid,forget plot]
  table[row sep=crcr]{%
1	0\\
2	0\\
3	0\\
4	0\\
5	0\\
6	0\\
7	0\\
8	0\\
9	0\\
10	0\\
11	0\\
12	0\\
13	0\\
14	0\\
15	0\\
16	0\\
17	0\\
18	0\\
19	0\\
20	0\\
21	0\\
22	0\\
23	0\\
24	0\\
25	0\\
26	0\\
27	0\\
28	0\\
29	0\\
30	0\\
31	0\\
32	0\\
33	0\\
34	0\\
35	0\\
36	0\\
37	0\\
38	0\\
39	0\\
40	0\\
41	0\\
42	0\\
43	0\\
44	0\\
45	0\\
46	0\\
47	0\\
48	0\\
49	0\\
50	0\\
51	0\\
52	0\\
53	0\\
54	0\\
55	0\\
56	0\\
57	0\\
58	0\\
59	0\\
60	0\\
61	0\\
62	0\\
63	0\\
64	0\\
65	0\\
66	0\\
67	0\\
68	0\\
69	0\\
70	0\\
71	0\\
72	0\\
73	0\\
74	0\\
75	0\\
76	0\\
77	0\\
78	0\\
79	0\\
80	0\\
81	0\\
82	0\\
83	0\\
84	0\\
85	0\\
86	0\\
87	0\\
88	0\\
89	0\\
90	0\\
91	0\\
92	0\\
93	0\\
94	0\\
95	0\\
96	0\\
97	0\\
98	0\\
99	0\\
100	0\\
101	0\\
102	0\\
103	0\\
104	0\\
105	0\\
106	0\\
107	0\\
108	0\\
109	0\\
110	0\\
111	0\\
112	0\\
113	0\\
114	0\\
115	0\\
116	0\\
117	0\\
118	0\\
119	0\\
120	0\\
121	0\\
122	0\\
123	0\\
124	0\\
125	0\\
126	0\\
127	0\\
128	0\\
129	0\\
130	0\\
131	0\\
132	0\\
133	0\\
134	0\\
135	0\\
136	0\\
137	0\\
138	0\\
139	0\\
140	0\\
141	0\\
142	0\\
143	0\\
144	0\\
145	0\\
146	0\\
147	0\\
148	0\\
149	0\\
150	0\\
151	0\\
152	0\\
153	0\\
154	0\\
155	0\\
156	0\\
157	0\\
158	0\\
159	0\\
160	0\\
161	0\\
162	0\\
163	0\\
164	0\\
165	0\\
166	0\\
167	0\\
168	0\\
169	0\\
170	0\\
171	0\\
172	0\\
173	0\\
174	0\\
175	0\\
176	0\\
177	0\\
178	0\\
179	0\\
180	0\\
181	0\\
182	0\\
183	0\\
184	0\\
185	0\\
186	0\\
187	0\\
188	0\\
189	0\\
190	0\\
191	0\\
192	0\\
193	0\\
194	0\\
195	0\\
196	0\\
197	0\\
198	0\\
199	0\\
200	0\\
201	0\\
202	0\\
203	0\\
204	0\\
205	0\\
206	0\\
207	0\\
208	0\\
209	0\\
210	0\\
211	0\\
212	0\\
213	0\\
214	0\\
215	0\\
216	0\\
217	0\\
218	0\\
219	0\\
220	0\\
221	0\\
222	0\\
223	0\\
224	0\\
225	0\\
226	0\\
227	0\\
228	0\\
229	0\\
230	0\\
231	0\\
232	0\\
233	0\\
234	0\\
235	0\\
236	0\\
237	0\\
238	0\\
239	0\\
240	0\\
241	0\\
242	0\\
243	0\\
244	0\\
245	0\\
246	0\\
247	0\\
248	0\\
249	0\\
250	0\\
251	0\\
252	0\\
253	0\\
254	0\\
255	0\\
256	0\\
257	0\\
258	0\\
259	0\\
260	0\\
261	0\\
262	0\\
263	0\\
264	0\\
265	0\\
266	0\\
267	0\\
268	0\\
269	0\\
270	0\\
271	0\\
272	0\\
273	0\\
274	0\\
275	0\\
276	0\\
277	0\\
278	0\\
279	0\\
280	0\\
281	0\\
282	0\\
283	0\\
284	0\\
285	0\\
286	0\\
287	0\\
288	0\\
289	0\\
290	0\\
291	0\\
292	0\\
293	0\\
294	0\\
295	0\\
296	0\\
297	0\\
298	0\\
299	0\\
300	0\\
301	0\\
302	0\\
303	0\\
304	0\\
305	0\\
306	0\\
307	0\\
308	0\\
309	0\\
310	0\\
311	0\\
312	0\\
313	0\\
314	0\\
315	0\\
316	0\\
317	0\\
318	0\\
319	0\\
320	0\\
321	0\\
322	0\\
323	0\\
324	0\\
325	0\\
326	0\\
327	0\\
328	0\\
329	0\\
330	0\\
331	0\\
332	0\\
333	0\\
334	0\\
335	0\\
336	0\\
337	0\\
338	0\\
339	0\\
340	0\\
341	0\\
342	0\\
343	0\\
344	0\\
345	0\\
346	0\\
347	0\\
348	0\\
349	0\\
350	0\\
351	0\\
352	0\\
353	0\\
354	0\\
355	0\\
356	0\\
357	0\\
358	0\\
359	0\\
360	0\\
361	0\\
362	0\\
363	0\\
364	0\\
365	0\\
366	0\\
367	0\\
368	0\\
369	0\\
370	0\\
371	0\\
372	0\\
373	0\\
374	0\\
375	0\\
376	0\\
377	0\\
378	0\\
379	0\\
380	0\\
381	0\\
382	0\\
383	0\\
384	0\\
385	0\\
386	0\\
387	0\\
388	0\\
389	0\\
390	0\\
391	0\\
392	0\\
393	0\\
394	0\\
395	0\\
396	0\\
397	0\\
398	0\\
399	0\\
400	0\\
401	0\\
402	0\\
403	0\\
404	0\\
405	0\\
406	0\\
407	0\\
408	0\\
409	0\\
410	0\\
411	0\\
412	0\\
413	0\\
414	0\\
415	0\\
416	0\\
417	0\\
418	0\\
419	0\\
420	0\\
421	0\\
422	0\\
423	0\\
424	0\\
425	0\\
426	0\\
427	0\\
428	0\\
429	0\\
430	0\\
431	0\\
432	0\\
433	0\\
434	0\\
435	0\\
436	0\\
437	0\\
438	0\\
439	0\\
440	0\\
441	0\\
442	0\\
443	0\\
444	0\\
445	0\\
446	0\\
447	0\\
448	0\\
449	0\\
450	0\\
451	0\\
452	0\\
453	0\\
454	0\\
455	0\\
456	0\\
457	0\\
458	0\\
459	0\\
460	0\\
461	0\\
462	0\\
463	0\\
464	0\\
465	0\\
466	0\\
467	0\\
468	0\\
469	0\\
470	0\\
471	0\\
472	0\\
473	0\\
474	0\\
475	0\\
476	0\\
477	0\\
478	0\\
479	0\\
480	0\\
481	0\\
482	0\\
483	0\\
484	0\\
485	0\\
486	0\\
487	0\\
488	0\\
489	0\\
490	0\\
491	0\\
492	0\\
493	0\\
494	0\\
495	0\\
496	0\\
497	0\\
498	0\\
499	0\\
500	0\\
501	0\\
502	0\\
503	0\\
504	0\\
505	0\\
506	0\\
507	0\\
508	0\\
509	0\\
510	0\\
511	0\\
512	0\\
513	0\\
514	0\\
515	0\\
516	0\\
517	0\\
518	0\\
519	0\\
520	0\\
521	0\\
522	0\\
523	0\\
524	0\\
525	0\\
526	0\\
527	0\\
528	0\\
529	0\\
530	0\\
531	0\\
532	0\\
533	0\\
534	0\\
535	0\\
536	0\\
537	0\\
538	0\\
539	0\\
540	0\\
541	0\\
542	0\\
543	0\\
544	0\\
545	0\\
546	0\\
547	0\\
548	0\\
549	0\\
550	0\\
551	0\\
552	0\\
553	0\\
554	0\\
555	0\\
556	0\\
557	0\\
558	0\\
559	0\\
560	0\\
561	0\\
562	0\\
563	0\\
564	0\\
565	0\\
566	0\\
567	0\\
568	0\\
569	0\\
570	0\\
571	0\\
572	0\\
573	0\\
574	0\\
575	0\\
576	0\\
577	0\\
578	0\\
579	0\\
580	0\\
581	0\\
582	0\\
583	0\\
584	0\\
585	0\\
586	0\\
587	0\\
588	0\\
589	0\\
590	0\\
591	0\\
592	0\\
593	0\\
594	0\\
595	0\\
596	0\\
597	0\\
598	0\\
599	0\\
600	0\\
};
\addplot [color=mycolor5,solid,forget plot]
  table[row sep=crcr]{%
1	0\\
2	0\\
3	0\\
4	0\\
5	0\\
6	0\\
7	0\\
8	0\\
9	0\\
10	0\\
11	0\\
12	0\\
13	0\\
14	0\\
15	0\\
16	0\\
17	0\\
18	0\\
19	0\\
20	0\\
21	0\\
22	0\\
23	0\\
24	0\\
25	0\\
26	0\\
27	0\\
28	0\\
29	0\\
30	0\\
31	0\\
32	0\\
33	0\\
34	0\\
35	0\\
36	0\\
37	0\\
38	0\\
39	0\\
40	0\\
41	0\\
42	0\\
43	0\\
44	0\\
45	0\\
46	0\\
47	0\\
48	0\\
49	0\\
50	0\\
51	0\\
52	0\\
53	0\\
54	0\\
55	0\\
56	0\\
57	0\\
58	0\\
59	0\\
60	0\\
61	0\\
62	0\\
63	0\\
64	0\\
65	0\\
66	0\\
67	0\\
68	0\\
69	0\\
70	0\\
71	0\\
72	0\\
73	0\\
74	0\\
75	0\\
76	0\\
77	0\\
78	0\\
79	0\\
80	0\\
81	0\\
82	0\\
83	0\\
84	0\\
85	0\\
86	0\\
87	0\\
88	0\\
89	0\\
90	0\\
91	0\\
92	0\\
93	0\\
94	0\\
95	0\\
96	0\\
97	0\\
98	0\\
99	0\\
100	0\\
101	0\\
102	0\\
103	0\\
104	0\\
105	0\\
106	0\\
107	0\\
108	0\\
109	0\\
110	0\\
111	0\\
112	0\\
113	0\\
114	0\\
115	0\\
116	0\\
117	0\\
118	0\\
119	0\\
120	0\\
121	0\\
122	0\\
123	0\\
124	0\\
125	0\\
126	0\\
127	0\\
128	0\\
129	0\\
130	0\\
131	0\\
132	0\\
133	0\\
134	0\\
135	0\\
136	0\\
137	0\\
138	0\\
139	0\\
140	0\\
141	0\\
142	0\\
143	0\\
144	0\\
145	0\\
146	0\\
147	0\\
148	0\\
149	0\\
150	0\\
151	0\\
152	0\\
153	0\\
154	0\\
155	0\\
156	0\\
157	0\\
158	0\\
159	0\\
160	0\\
161	0\\
162	0\\
163	0\\
164	0\\
165	0\\
166	0\\
167	0\\
168	0\\
169	0\\
170	0\\
171	0\\
172	0\\
173	0\\
174	0\\
175	0\\
176	0\\
177	0\\
178	0\\
179	0\\
180	0\\
181	0\\
182	0\\
183	0\\
184	0\\
185	0\\
186	0\\
187	0\\
188	0\\
189	0\\
190	0\\
191	0\\
192	0\\
193	0\\
194	0\\
195	0\\
196	0\\
197	0\\
198	0\\
199	0\\
200	0\\
201	0\\
202	0\\
203	0\\
204	0\\
205	0\\
206	0\\
207	0\\
208	0\\
209	0\\
210	0\\
211	0\\
212	0\\
213	0\\
214	0\\
215	0\\
216	0\\
217	0\\
218	0\\
219	0\\
220	0\\
221	0\\
222	0\\
223	0\\
224	0\\
225	0\\
226	0\\
227	0\\
228	0\\
229	0\\
230	0\\
231	0\\
232	0\\
233	0\\
234	0\\
235	0\\
236	0\\
237	0\\
238	0\\
239	0\\
240	0\\
241	0\\
242	0\\
243	0\\
244	0\\
245	0\\
246	0\\
247	0\\
248	0\\
249	0\\
250	0\\
251	0\\
252	0\\
253	0\\
254	0\\
255	0\\
256	0\\
257	0\\
258	0\\
259	0\\
260	0\\
261	0\\
262	0\\
263	0\\
264	0\\
265	0\\
266	0\\
267	0\\
268	0\\
269	0\\
270	0\\
271	0\\
272	0\\
273	0\\
274	0\\
275	0\\
276	0\\
277	0\\
278	0\\
279	0\\
280	0\\
281	0\\
282	0\\
283	0\\
284	0\\
285	0\\
286	0\\
287	0\\
288	0\\
289	0\\
290	0\\
291	0\\
292	0\\
293	0\\
294	0\\
295	0\\
296	0\\
297	0\\
298	0\\
299	0\\
300	0\\
301	0\\
302	0\\
303	0\\
304	0\\
305	0\\
306	0\\
307	0\\
308	0\\
309	0\\
310	0\\
311	0\\
312	0\\
313	0\\
314	0\\
315	0\\
316	0\\
317	0\\
318	0\\
319	0\\
320	0\\
321	0\\
322	0\\
323	0\\
324	0\\
325	0\\
326	0\\
327	0\\
328	0\\
329	0\\
330	0\\
331	0\\
332	0\\
333	0\\
334	0\\
335	0\\
336	0\\
337	0\\
338	0\\
339	0\\
340	0\\
341	0\\
342	0\\
343	0\\
344	0\\
345	0\\
346	0\\
347	0\\
348	0\\
349	0\\
350	0\\
351	0\\
352	0\\
353	0\\
354	0\\
355	0\\
356	0\\
357	0\\
358	0\\
359	0\\
360	0\\
361	0\\
362	0\\
363	0\\
364	0\\
365	0\\
366	0\\
367	0\\
368	0\\
369	0\\
370	0\\
371	0\\
372	0\\
373	0\\
374	0\\
375	0\\
376	0\\
377	0\\
378	0\\
379	0\\
380	0\\
381	0\\
382	0\\
383	0\\
384	0\\
385	0\\
386	0\\
387	0\\
388	0\\
389	0\\
390	0\\
391	0\\
392	0\\
393	0\\
394	0\\
395	0\\
396	0\\
397	0\\
398	0\\
399	0\\
400	0\\
401	0\\
402	0\\
403	0\\
404	0\\
405	0\\
406	0\\
407	0\\
408	0\\
409	0\\
410	0\\
411	0\\
412	0\\
413	0\\
414	0\\
415	0\\
416	0\\
417	0\\
418	0\\
419	0\\
420	0\\
421	0\\
422	0\\
423	0\\
424	0\\
425	0\\
426	0\\
427	0\\
428	0\\
429	0\\
430	0\\
431	0\\
432	0\\
433	0\\
434	0\\
435	0\\
436	0\\
437	0\\
438	0\\
439	0\\
440	0\\
441	0\\
442	0\\
443	0\\
444	0\\
445	0\\
446	0\\
447	0\\
448	0\\
449	0\\
450	0\\
451	0\\
452	0\\
453	0\\
454	0\\
455	0\\
456	0\\
457	0\\
458	0\\
459	0\\
460	0\\
461	0\\
462	0\\
463	0\\
464	0\\
465	0\\
466	0\\
467	0\\
468	0\\
469	0\\
470	0\\
471	0\\
472	0\\
473	0\\
474	0\\
475	0\\
476	0\\
477	0\\
478	0\\
479	0\\
480	0\\
481	0\\
482	0\\
483	0\\
484	0\\
485	0\\
486	0\\
487	0\\
488	0\\
489	0\\
490	0\\
491	0\\
492	0\\
493	0\\
494	0\\
495	0\\
496	0\\
497	0\\
498	0\\
499	0\\
500	0\\
501	0\\
502	0\\
503	0\\
504	0\\
505	0\\
506	0\\
507	0\\
508	0\\
509	0\\
510	0\\
511	0\\
512	0\\
513	0\\
514	0\\
515	0\\
516	0\\
517	0\\
518	0\\
519	0\\
520	0\\
521	0\\
522	0\\
523	0\\
524	0\\
525	0\\
526	0\\
527	0\\
528	0\\
529	0\\
530	0\\
531	0\\
532	0\\
533	0\\
534	0\\
535	0\\
536	0\\
537	0\\
538	0\\
539	0\\
540	0\\
541	0\\
542	0\\
543	0\\
544	0\\
545	0\\
546	0\\
547	0\\
548	0\\
549	0\\
550	0\\
551	0\\
552	0\\
553	0\\
554	0\\
555	0\\
556	0\\
557	0\\
558	0\\
559	0\\
560	0\\
561	0\\
562	0\\
563	0\\
564	0\\
565	0\\
566	0\\
567	0\\
568	0\\
569	0\\
570	0\\
571	0\\
572	0\\
573	0\\
574	0\\
575	0\\
576	0\\
577	0\\
578	0\\
579	0\\
580	0\\
581	0\\
582	0\\
583	0\\
584	0\\
585	0\\
586	0\\
587	0\\
588	0\\
589	0\\
590	0\\
591	0\\
592	0\\
593	0\\
594	0\\
595	0\\
596	0\\
597	0\\
598	0\\
599	0\\
600	0\\
};
\addplot [color=mycolor6,solid,forget plot]
  table[row sep=crcr]{%
1	0\\
2	0\\
3	0\\
4	0\\
5	0\\
6	0\\
7	0\\
8	0\\
9	0\\
10	0\\
11	0\\
12	0\\
13	0\\
14	0\\
15	0\\
16	0\\
17	0\\
18	0\\
19	0\\
20	0\\
21	0\\
22	0\\
23	0\\
24	0\\
25	0\\
26	0\\
27	0\\
28	0\\
29	0\\
30	0\\
31	0\\
32	0\\
33	0\\
34	0\\
35	0\\
36	0\\
37	0\\
38	0\\
39	0\\
40	0\\
41	0\\
42	0\\
43	0\\
44	0\\
45	0\\
46	0\\
47	0\\
48	0\\
49	0\\
50	0\\
51	0\\
52	0\\
53	0\\
54	0\\
55	0\\
56	0\\
57	0\\
58	0\\
59	0\\
60	0\\
61	0\\
62	0\\
63	0\\
64	0\\
65	0\\
66	0\\
67	0\\
68	0\\
69	0\\
70	0\\
71	0\\
72	0\\
73	0\\
74	0\\
75	0\\
76	0\\
77	0\\
78	0\\
79	0\\
80	0\\
81	0\\
82	0\\
83	0\\
84	0\\
85	0\\
86	0\\
87	0\\
88	0\\
89	0\\
90	0\\
91	0\\
92	0\\
93	0\\
94	0\\
95	0\\
96	0\\
97	0\\
98	0\\
99	0\\
100	0\\
101	0\\
102	0\\
103	0\\
104	0\\
105	0\\
106	0\\
107	0\\
108	0\\
109	0\\
110	0\\
111	0\\
112	0\\
113	0\\
114	0\\
115	0\\
116	0\\
117	0\\
118	0\\
119	0\\
120	0\\
121	0\\
122	0\\
123	0\\
124	0\\
125	0\\
126	0\\
127	0\\
128	0\\
129	0\\
130	0\\
131	0\\
132	0\\
133	0\\
134	0\\
135	0\\
136	0\\
137	0\\
138	0\\
139	0\\
140	0\\
141	0\\
142	0\\
143	0\\
144	0\\
145	0\\
146	0\\
147	0\\
148	0\\
149	0\\
150	0\\
151	0\\
152	0\\
153	0\\
154	0\\
155	0\\
156	0\\
157	0\\
158	0\\
159	0\\
160	0\\
161	0\\
162	0\\
163	0\\
164	0\\
165	0\\
166	0\\
167	0\\
168	0\\
169	0\\
170	0\\
171	0\\
172	0\\
173	0\\
174	0\\
175	0\\
176	0\\
177	0\\
178	0\\
179	0\\
180	0\\
181	0\\
182	0\\
183	0\\
184	0\\
185	0\\
186	0\\
187	0\\
188	0\\
189	0\\
190	0\\
191	0\\
192	0\\
193	0\\
194	0\\
195	0\\
196	0\\
197	0\\
198	0\\
199	0\\
200	0\\
201	0\\
202	0\\
203	0\\
204	0\\
205	0\\
206	0\\
207	0\\
208	0\\
209	0\\
210	0\\
211	0\\
212	0\\
213	0\\
214	0\\
215	0\\
216	0\\
217	0\\
218	0\\
219	0\\
220	0\\
221	0\\
222	0\\
223	0\\
224	0\\
225	0\\
226	0\\
227	0\\
228	0\\
229	0\\
230	0\\
231	0\\
232	0\\
233	0\\
234	0\\
235	0\\
236	0\\
237	0\\
238	0\\
239	0\\
240	0\\
241	0\\
242	0\\
243	0\\
244	0\\
245	0\\
246	0\\
247	0\\
248	0\\
249	0\\
250	0\\
251	0\\
252	0\\
253	0\\
254	0\\
255	0\\
256	0\\
257	0\\
258	0\\
259	0\\
260	0\\
261	0\\
262	0\\
263	0\\
264	0\\
265	0\\
266	0\\
267	0\\
268	0\\
269	0\\
270	0\\
271	0\\
272	0\\
273	0\\
274	0\\
275	0\\
276	0\\
277	0\\
278	0\\
279	0\\
280	0\\
281	0\\
282	0\\
283	0\\
284	0\\
285	0\\
286	0\\
287	0\\
288	0\\
289	0\\
290	0\\
291	0\\
292	0\\
293	0\\
294	0\\
295	0\\
296	0\\
297	0\\
298	0\\
299	0\\
300	0\\
301	0\\
302	0\\
303	0\\
304	0\\
305	0\\
306	0\\
307	0\\
308	0\\
309	0\\
310	0\\
311	0\\
312	0\\
313	0\\
314	0\\
315	0\\
316	0\\
317	0\\
318	0\\
319	0\\
320	0\\
321	0\\
322	0\\
323	0\\
324	0\\
325	0\\
326	0\\
327	0\\
328	0\\
329	0\\
330	0\\
331	0\\
332	0\\
333	0\\
334	0\\
335	0\\
336	0\\
337	0\\
338	0\\
339	0\\
340	0\\
341	0\\
342	0\\
343	0\\
344	0\\
345	0\\
346	0\\
347	0\\
348	0\\
349	0\\
350	0\\
351	0\\
352	0\\
353	0\\
354	0\\
355	0\\
356	0\\
357	0\\
358	0\\
359	0\\
360	0\\
361	0\\
362	0\\
363	0\\
364	0\\
365	0\\
366	0\\
367	0\\
368	0\\
369	0\\
370	0\\
371	0\\
372	0\\
373	0\\
374	0\\
375	0\\
376	0\\
377	0\\
378	0\\
379	0\\
380	0\\
381	0\\
382	0\\
383	0\\
384	0\\
385	0\\
386	0\\
387	0\\
388	0\\
389	0\\
390	0\\
391	0\\
392	0\\
393	0\\
394	0\\
395	0\\
396	0\\
397	0\\
398	0\\
399	0\\
400	0\\
401	0\\
402	0\\
403	0\\
404	0\\
405	0\\
406	0\\
407	0\\
408	0\\
409	0\\
410	0\\
411	0\\
412	0\\
413	0\\
414	0\\
415	0\\
416	0\\
417	0\\
418	0\\
419	0\\
420	0\\
421	0\\
422	0\\
423	0\\
424	0\\
425	0\\
426	0\\
427	0\\
428	0\\
429	0\\
430	0\\
431	0\\
432	0\\
433	0\\
434	0\\
435	0\\
436	0\\
437	0\\
438	0\\
439	0\\
440	0\\
441	0\\
442	0\\
443	0\\
444	0\\
445	0\\
446	0\\
447	0\\
448	0\\
449	0\\
450	0\\
451	0\\
452	0\\
453	0\\
454	0\\
455	0\\
456	0\\
457	0\\
458	0\\
459	0\\
460	0\\
461	0\\
462	0\\
463	0\\
464	0\\
465	0\\
466	0\\
467	0\\
468	0\\
469	0\\
470	0\\
471	0\\
472	0\\
473	0\\
474	0\\
475	0\\
476	0\\
477	0\\
478	0\\
479	0\\
480	0\\
481	0\\
482	0\\
483	0\\
484	0\\
485	0\\
486	0\\
487	0\\
488	0\\
489	0\\
490	0\\
491	0\\
492	0\\
493	0\\
494	0\\
495	0\\
496	0\\
497	0\\
498	0\\
499	0\\
500	0\\
501	0\\
502	0\\
503	0\\
504	0\\
505	0\\
506	0\\
507	0\\
508	0\\
509	0\\
510	0\\
511	0\\
512	0\\
513	0\\
514	0\\
515	0\\
516	0\\
517	0\\
518	0\\
519	0\\
520	0\\
521	0\\
522	0\\
523	0\\
524	0\\
525	0\\
526	0\\
527	0\\
528	0\\
529	0\\
530	0\\
531	0\\
532	0\\
533	0\\
534	0\\
535	0\\
536	0\\
537	0\\
538	0\\
539	0\\
540	0\\
541	0\\
542	0\\
543	0\\
544	0\\
545	0\\
546	0\\
547	0\\
548	0\\
549	0\\
550	0\\
551	0\\
552	0\\
553	0\\
554	0\\
555	0\\
556	0\\
557	0\\
558	0\\
559	0\\
560	0\\
561	0\\
562	0\\
563	0\\
564	0\\
565	0\\
566	0\\
567	0\\
568	0\\
569	0\\
570	0\\
571	0\\
572	0\\
573	0\\
574	0\\
575	0\\
576	0\\
577	0\\
578	0\\
579	0\\
580	0\\
581	0\\
582	0\\
583	0\\
584	0\\
585	0\\
586	0\\
587	0\\
588	0\\
589	0\\
590	0\\
591	0\\
592	0\\
593	0\\
594	0\\
595	0\\
596	0\\
597	0\\
598	0\\
599	0\\
600	0\\
};
\addplot [color=mycolor7,solid,forget plot]
  table[row sep=crcr]{%
1	0\\
2	0\\
3	0\\
4	0\\
5	0\\
6	0\\
7	0\\
8	0\\
9	0\\
10	0\\
11	0\\
12	0\\
13	0\\
14	0\\
15	0\\
16	0\\
17	0\\
18	0\\
19	0\\
20	0\\
21	0\\
22	0\\
23	0\\
24	0\\
25	0\\
26	0\\
27	0\\
28	0\\
29	0\\
30	0\\
31	0\\
32	0\\
33	0\\
34	0\\
35	0\\
36	0\\
37	0\\
38	0\\
39	0\\
40	0\\
41	0\\
42	0\\
43	0\\
44	0\\
45	0\\
46	0\\
47	0\\
48	0\\
49	0\\
50	0\\
51	0\\
52	0\\
53	0\\
54	0\\
55	0\\
56	0\\
57	0\\
58	0\\
59	0\\
60	0\\
61	0\\
62	0\\
63	0\\
64	0\\
65	0\\
66	0\\
67	0\\
68	0\\
69	0\\
70	0\\
71	0\\
72	0\\
73	0\\
74	0\\
75	0\\
76	0\\
77	0\\
78	0\\
79	0\\
80	0\\
81	0\\
82	0\\
83	0\\
84	0\\
85	0\\
86	0\\
87	0\\
88	0\\
89	0\\
90	0\\
91	0\\
92	0\\
93	0\\
94	0\\
95	0\\
96	0\\
97	0\\
98	0\\
99	0\\
100	0\\
101	0\\
102	0\\
103	0\\
104	0\\
105	0\\
106	0\\
107	0\\
108	0\\
109	0\\
110	0\\
111	0\\
112	0\\
113	0\\
114	0\\
115	0\\
116	0\\
117	0\\
118	0\\
119	0\\
120	0\\
121	0\\
122	0\\
123	0\\
124	0\\
125	0\\
126	0\\
127	0\\
128	0\\
129	0\\
130	0\\
131	0\\
132	0\\
133	0\\
134	0\\
135	0\\
136	0\\
137	0\\
138	0\\
139	0\\
140	0\\
141	0\\
142	0\\
143	0\\
144	0\\
145	0\\
146	0\\
147	0\\
148	0\\
149	0\\
150	0\\
151	0\\
152	0\\
153	0\\
154	0\\
155	0\\
156	0\\
157	0\\
158	0\\
159	0\\
160	0\\
161	0\\
162	0\\
163	0\\
164	0\\
165	0\\
166	0\\
167	0\\
168	0\\
169	0\\
170	0\\
171	0\\
172	0\\
173	0\\
174	0\\
175	0\\
176	0\\
177	0\\
178	0\\
179	0\\
180	0\\
181	0\\
182	0\\
183	0\\
184	0\\
185	0\\
186	0\\
187	0\\
188	0\\
189	0\\
190	0\\
191	0\\
192	0\\
193	0\\
194	0\\
195	0\\
196	0\\
197	0\\
198	0\\
199	0\\
200	0\\
201	0\\
202	0\\
203	0\\
204	0\\
205	0\\
206	0\\
207	0\\
208	0\\
209	0\\
210	0\\
211	0\\
212	0\\
213	0\\
214	0\\
215	0\\
216	0\\
217	0\\
218	0\\
219	0\\
220	0\\
221	0\\
222	0\\
223	0\\
224	0\\
225	0\\
226	0\\
227	0\\
228	0\\
229	0\\
230	0\\
231	0\\
232	0\\
233	0\\
234	0\\
235	0\\
236	0\\
237	0\\
238	0\\
239	0\\
240	0\\
241	0\\
242	0\\
243	0\\
244	0\\
245	0\\
246	0\\
247	0\\
248	0\\
249	0\\
250	0\\
251	0\\
252	0\\
253	0\\
254	0\\
255	0\\
256	0\\
257	0\\
258	0\\
259	0\\
260	0\\
261	0\\
262	0\\
263	0\\
264	0\\
265	0\\
266	0\\
267	0\\
268	0\\
269	0\\
270	0\\
271	0\\
272	0\\
273	0\\
274	0\\
275	0\\
276	0\\
277	0\\
278	0\\
279	0\\
280	0\\
281	0\\
282	0\\
283	0\\
284	0\\
285	0\\
286	0\\
287	0\\
288	0\\
289	0\\
290	0\\
291	0\\
292	0\\
293	0\\
294	0\\
295	0\\
296	0\\
297	0\\
298	0\\
299	0\\
300	0\\
301	0\\
302	0\\
303	0\\
304	0\\
305	0\\
306	0\\
307	0\\
308	0\\
309	0\\
310	0\\
311	0\\
312	0\\
313	0\\
314	0\\
315	0\\
316	0\\
317	0\\
318	0\\
319	0\\
320	0\\
321	0\\
322	0\\
323	0\\
324	0\\
325	0\\
326	0\\
327	0\\
328	0\\
329	0\\
330	0\\
331	0\\
332	0\\
333	0\\
334	0\\
335	0\\
336	0\\
337	0\\
338	0\\
339	0\\
340	0\\
341	0\\
342	0\\
343	0\\
344	0\\
345	0\\
346	0\\
347	0\\
348	0\\
349	0\\
350	0\\
351	0\\
352	0\\
353	0\\
354	0\\
355	0\\
356	0\\
357	0\\
358	0\\
359	0\\
360	0\\
361	0\\
362	0\\
363	0\\
364	0\\
365	0\\
366	0\\
367	0\\
368	0\\
369	0\\
370	0\\
371	0\\
372	0\\
373	0\\
374	0\\
375	0\\
376	0\\
377	0\\
378	0\\
379	0\\
380	0\\
381	0\\
382	0\\
383	0\\
384	0\\
385	0\\
386	0\\
387	0\\
388	0\\
389	0\\
390	0\\
391	0\\
392	0\\
393	0\\
394	0\\
395	0\\
396	0\\
397	0\\
398	0\\
399	0\\
400	0\\
401	0\\
402	0\\
403	0\\
404	0\\
405	0\\
406	0\\
407	0\\
408	0\\
409	0\\
410	0\\
411	0\\
412	0\\
413	0\\
414	0\\
415	0\\
416	0\\
417	0\\
418	0\\
419	0\\
420	0\\
421	0\\
422	0\\
423	0\\
424	0\\
425	0\\
426	0\\
427	0\\
428	0\\
429	0\\
430	0\\
431	0\\
432	0\\
433	0\\
434	0\\
435	0\\
436	0\\
437	0\\
438	0\\
439	0\\
440	0\\
441	0\\
442	0\\
443	0\\
444	0\\
445	0\\
446	0\\
447	0\\
448	0\\
449	0\\
450	0\\
451	0\\
452	0\\
453	0\\
454	0\\
455	0\\
456	0\\
457	0\\
458	0\\
459	0\\
460	0\\
461	0\\
462	0\\
463	0\\
464	0\\
465	0\\
466	0\\
467	0\\
468	0\\
469	0\\
470	0\\
471	0\\
472	0\\
473	0\\
474	0\\
475	0\\
476	0\\
477	0\\
478	0\\
479	0\\
480	0\\
481	0\\
482	0\\
483	0\\
484	0\\
485	0\\
486	0\\
487	0\\
488	0\\
489	0\\
490	0\\
491	0\\
492	0\\
493	0\\
494	0\\
495	0\\
496	0\\
497	0\\
498	0\\
499	0\\
500	0\\
501	0\\
502	0\\
503	0\\
504	0\\
505	0\\
506	0\\
507	0\\
508	0\\
509	0\\
510	0\\
511	0\\
512	0\\
513	0\\
514	0\\
515	0\\
516	0\\
517	0\\
518	0\\
519	0\\
520	0\\
521	0\\
522	0\\
523	0\\
524	0\\
525	0\\
526	0\\
527	0\\
528	0\\
529	0\\
530	0\\
531	0\\
532	0\\
533	0\\
534	0\\
535	0\\
536	0\\
537	0\\
538	0\\
539	0\\
540	0\\
541	0\\
542	0\\
543	0\\
544	0\\
545	0\\
546	0\\
547	0\\
548	0\\
549	0\\
550	0\\
551	0\\
552	0\\
553	0\\
554	0\\
555	0\\
556	0\\
557	0\\
558	0\\
559	0\\
560	0\\
561	0\\
562	0\\
563	0\\
564	0\\
565	0\\
566	0\\
567	0\\
568	0\\
569	0\\
570	0\\
571	0\\
572	0\\
573	0\\
574	0\\
575	0\\
576	0\\
577	0\\
578	0\\
579	0\\
580	0\\
581	0\\
582	0\\
583	0\\
584	0\\
585	0\\
586	0\\
587	0\\
588	0\\
589	0\\
590	0\\
591	0\\
592	0\\
593	0\\
594	0\\
595	0\\
596	0\\
597	0\\
598	0\\
599	0\\
600	0\\
};
\addplot [color=mycolor8,solid,forget plot]
  table[row sep=crcr]{%
1	0\\
2	0\\
3	0\\
4	0\\
5	0\\
6	0\\
7	0\\
8	0\\
9	0\\
10	0\\
11	0\\
12	0\\
13	0\\
14	0\\
15	0\\
16	0\\
17	0\\
18	0\\
19	0\\
20	0\\
21	0\\
22	0\\
23	0\\
24	0\\
25	0\\
26	0\\
27	0\\
28	0\\
29	0\\
30	0\\
31	0\\
32	0\\
33	0\\
34	0\\
35	0\\
36	0\\
37	0\\
38	0\\
39	0\\
40	0\\
41	0\\
42	0\\
43	0\\
44	0\\
45	0\\
46	0\\
47	0\\
48	0\\
49	0\\
50	0\\
51	0\\
52	0\\
53	0\\
54	0\\
55	0\\
56	0\\
57	0\\
58	0\\
59	0\\
60	0\\
61	0\\
62	0\\
63	0\\
64	0\\
65	0\\
66	0\\
67	0\\
68	0\\
69	0\\
70	0\\
71	0\\
72	0\\
73	0\\
74	0\\
75	0\\
76	0\\
77	0\\
78	0\\
79	0\\
80	0\\
81	0\\
82	0\\
83	0\\
84	0\\
85	0\\
86	0\\
87	0\\
88	0\\
89	0\\
90	0\\
91	0\\
92	0\\
93	0\\
94	0\\
95	0\\
96	0\\
97	0\\
98	0\\
99	0\\
100	0\\
101	0\\
102	0\\
103	0\\
104	0\\
105	0\\
106	0\\
107	0\\
108	0\\
109	0\\
110	0\\
111	0\\
112	0\\
113	0\\
114	0\\
115	0\\
116	0\\
117	0\\
118	0\\
119	0\\
120	0\\
121	0\\
122	0\\
123	0\\
124	0\\
125	0\\
126	0\\
127	0\\
128	0\\
129	0\\
130	0\\
131	0\\
132	0\\
133	0\\
134	0\\
135	0\\
136	0\\
137	0\\
138	0\\
139	0\\
140	0\\
141	0\\
142	0\\
143	0\\
144	0\\
145	0\\
146	0\\
147	0\\
148	0\\
149	0\\
150	0\\
151	0\\
152	0\\
153	0\\
154	0\\
155	0\\
156	0\\
157	0\\
158	0\\
159	0\\
160	0\\
161	0\\
162	0\\
163	0\\
164	0\\
165	0\\
166	0\\
167	0\\
168	0\\
169	0\\
170	0\\
171	0\\
172	0\\
173	0\\
174	0\\
175	0\\
176	0\\
177	0\\
178	0\\
179	0\\
180	0\\
181	0\\
182	0\\
183	0\\
184	0\\
185	0\\
186	0\\
187	0\\
188	0\\
189	0\\
190	0\\
191	0\\
192	0\\
193	0\\
194	0\\
195	0\\
196	0\\
197	0\\
198	0\\
199	0\\
200	0\\
201	0\\
202	0\\
203	0\\
204	0\\
205	0\\
206	0\\
207	0\\
208	0\\
209	0\\
210	0\\
211	0\\
212	0\\
213	0\\
214	0\\
215	0\\
216	0\\
217	0\\
218	0\\
219	0\\
220	0\\
221	0\\
222	0\\
223	0\\
224	0\\
225	0\\
226	0\\
227	0\\
228	0\\
229	0\\
230	0\\
231	0\\
232	0\\
233	0\\
234	0\\
235	0\\
236	0\\
237	0\\
238	0\\
239	0\\
240	0\\
241	0\\
242	0\\
243	0\\
244	0\\
245	0\\
246	0\\
247	0\\
248	0\\
249	0\\
250	0\\
251	0\\
252	0\\
253	0\\
254	0\\
255	0\\
256	0\\
257	0\\
258	0\\
259	0\\
260	0\\
261	0\\
262	0\\
263	0\\
264	0\\
265	0\\
266	0\\
267	0\\
268	0\\
269	0\\
270	0\\
271	0\\
272	0\\
273	0\\
274	0\\
275	0\\
276	0\\
277	0\\
278	0\\
279	0\\
280	0\\
281	0\\
282	0\\
283	0\\
284	0\\
285	0\\
286	0\\
287	0\\
288	0\\
289	0\\
290	0\\
291	0\\
292	0\\
293	0\\
294	0\\
295	0\\
296	0\\
297	0\\
298	0\\
299	0\\
300	0\\
301	0\\
302	0\\
303	0\\
304	0\\
305	0\\
306	0\\
307	0\\
308	0\\
309	0\\
310	0\\
311	0\\
312	0\\
313	0\\
314	0\\
315	0\\
316	0\\
317	0\\
318	0\\
319	0\\
320	0\\
321	0\\
322	0\\
323	0\\
324	0\\
325	0\\
326	0\\
327	0\\
328	0\\
329	0\\
330	0\\
331	0\\
332	0\\
333	0\\
334	0\\
335	0\\
336	0\\
337	0\\
338	0\\
339	0\\
340	0\\
341	0\\
342	0\\
343	0\\
344	0\\
345	0\\
346	0\\
347	0\\
348	0\\
349	0\\
350	0\\
351	0\\
352	0\\
353	0\\
354	0\\
355	0\\
356	0\\
357	0\\
358	0\\
359	0\\
360	0\\
361	0\\
362	0\\
363	0\\
364	0\\
365	0\\
366	0\\
367	0\\
368	0\\
369	0\\
370	0\\
371	0\\
372	0\\
373	0\\
374	0\\
375	0\\
376	0\\
377	0\\
378	0\\
379	0\\
380	0\\
381	0\\
382	0\\
383	0\\
384	0\\
385	0\\
386	0\\
387	0\\
388	0\\
389	0\\
390	0\\
391	0\\
392	0\\
393	0\\
394	0\\
395	0\\
396	0\\
397	0\\
398	0\\
399	0\\
400	0\\
401	0\\
402	0\\
403	0\\
404	0\\
405	0\\
406	0\\
407	0\\
408	0\\
409	0\\
410	0\\
411	0\\
412	0\\
413	0\\
414	0\\
415	0\\
416	0\\
417	0\\
418	0\\
419	0\\
420	0\\
421	0\\
422	0\\
423	0\\
424	0\\
425	0\\
426	0\\
427	0\\
428	0\\
429	0\\
430	0\\
431	0\\
432	0\\
433	0\\
434	0\\
435	0\\
436	0\\
437	0\\
438	0\\
439	0\\
440	0\\
441	0\\
442	0\\
443	0\\
444	0\\
445	0\\
446	0\\
447	0\\
448	0\\
449	0\\
450	0\\
451	0\\
452	0\\
453	0\\
454	0\\
455	0\\
456	0\\
457	0\\
458	0\\
459	0\\
460	0\\
461	0\\
462	0\\
463	0\\
464	0\\
465	0\\
466	0\\
467	0\\
468	0\\
469	0\\
470	0\\
471	0\\
472	0\\
473	0\\
474	0\\
475	0\\
476	0\\
477	0\\
478	0\\
479	0\\
480	0\\
481	0\\
482	0\\
483	0\\
484	0\\
485	0\\
486	0\\
487	0\\
488	0\\
489	0\\
490	0\\
491	0\\
492	0\\
493	0\\
494	0\\
495	0\\
496	0\\
497	0\\
498	0\\
499	0\\
500	0\\
501	0\\
502	0\\
503	0\\
504	0\\
505	0\\
506	0\\
507	0\\
508	0\\
509	0\\
510	0\\
511	0\\
512	0\\
513	0\\
514	0\\
515	0\\
516	0\\
517	0\\
518	0\\
519	0\\
520	0\\
521	0\\
522	0\\
523	0\\
524	0\\
525	0\\
526	0\\
527	0\\
528	0\\
529	0\\
530	0\\
531	0\\
532	0\\
533	0\\
534	0\\
535	0\\
536	0\\
537	0\\
538	0\\
539	0\\
540	0\\
541	0\\
542	0\\
543	0\\
544	0\\
545	0\\
546	0\\
547	0\\
548	0\\
549	0\\
550	0\\
551	0\\
552	0\\
553	0\\
554	0\\
555	0\\
556	0\\
557	0\\
558	0\\
559	0\\
560	0\\
561	0\\
562	0\\
563	0\\
564	0\\
565	0\\
566	0\\
567	0\\
568	0\\
569	0\\
570	0\\
571	0\\
572	0\\
573	0\\
574	0\\
575	0\\
576	0\\
577	0\\
578	0\\
579	0\\
580	0\\
581	0\\
582	0\\
583	0\\
584	0\\
585	0\\
586	0\\
587	0\\
588	0\\
589	0\\
590	0\\
591	0\\
592	0\\
593	0\\
594	0\\
595	0\\
596	0\\
597	0\\
598	0\\
599	0\\
600	0\\
};
\addplot [color=blue!25!mycolor7,solid,forget plot]
  table[row sep=crcr]{%
1	0\\
2	0\\
3	0\\
4	0\\
5	0\\
6	0\\
7	0\\
8	0\\
9	0\\
10	0\\
11	0\\
12	0\\
13	0\\
14	0\\
15	0\\
16	0\\
17	0\\
18	0\\
19	0\\
20	0\\
21	0\\
22	0\\
23	0\\
24	0\\
25	0\\
26	0\\
27	0\\
28	0\\
29	0\\
30	0\\
31	0\\
32	0\\
33	0\\
34	0\\
35	0\\
36	0\\
37	0\\
38	0\\
39	0\\
40	0\\
41	0\\
42	0\\
43	0\\
44	0\\
45	0\\
46	0\\
47	0\\
48	0\\
49	0\\
50	0\\
51	0\\
52	0\\
53	0\\
54	0\\
55	0\\
56	0\\
57	0\\
58	0\\
59	0\\
60	0\\
61	0\\
62	0\\
63	0\\
64	0\\
65	0\\
66	0\\
67	0\\
68	0\\
69	0\\
70	0\\
71	0\\
72	0\\
73	0\\
74	0\\
75	0\\
76	0\\
77	0\\
78	0\\
79	0\\
80	0\\
81	0\\
82	0\\
83	0\\
84	0\\
85	0\\
86	0\\
87	0\\
88	0\\
89	0\\
90	0\\
91	0\\
92	0\\
93	0\\
94	0\\
95	0\\
96	0\\
97	0\\
98	0\\
99	0\\
100	0\\
101	0\\
102	0\\
103	0\\
104	0\\
105	0\\
106	0\\
107	0\\
108	0\\
109	0\\
110	0\\
111	0\\
112	0\\
113	0\\
114	0\\
115	0\\
116	0\\
117	0\\
118	0\\
119	0\\
120	0\\
121	0\\
122	0\\
123	0\\
124	0\\
125	0\\
126	0\\
127	0\\
128	0\\
129	0\\
130	0\\
131	0\\
132	0\\
133	0\\
134	0\\
135	0\\
136	0\\
137	0\\
138	0\\
139	0\\
140	0\\
141	0\\
142	0\\
143	0\\
144	0\\
145	0\\
146	0\\
147	0\\
148	0\\
149	0\\
150	0\\
151	0\\
152	0\\
153	0\\
154	0\\
155	0\\
156	0\\
157	0\\
158	0\\
159	0\\
160	0\\
161	0\\
162	0\\
163	0\\
164	0\\
165	0\\
166	0\\
167	0\\
168	0\\
169	0\\
170	0\\
171	0\\
172	0\\
173	0\\
174	0\\
175	0\\
176	0\\
177	0\\
178	0\\
179	0\\
180	0\\
181	0\\
182	0\\
183	0\\
184	0\\
185	0\\
186	0\\
187	0\\
188	0\\
189	0\\
190	0\\
191	0\\
192	0\\
193	0\\
194	0\\
195	0\\
196	0\\
197	0\\
198	0\\
199	0\\
200	0\\
201	0\\
202	0\\
203	0\\
204	0\\
205	0\\
206	0\\
207	0\\
208	0\\
209	0\\
210	0\\
211	0\\
212	0\\
213	0\\
214	0\\
215	0\\
216	0\\
217	0\\
218	0\\
219	0\\
220	0\\
221	0\\
222	0\\
223	0\\
224	0\\
225	0\\
226	0\\
227	0\\
228	0\\
229	0\\
230	0\\
231	0\\
232	0\\
233	0\\
234	0\\
235	0\\
236	0\\
237	0\\
238	0\\
239	0\\
240	0\\
241	0\\
242	0\\
243	0\\
244	0\\
245	0\\
246	0\\
247	0\\
248	0\\
249	0\\
250	0\\
251	0\\
252	0\\
253	0\\
254	0\\
255	0\\
256	0\\
257	0\\
258	0\\
259	0\\
260	0\\
261	0\\
262	0\\
263	0\\
264	0\\
265	0\\
266	0\\
267	0\\
268	0\\
269	0\\
270	0\\
271	0\\
272	0\\
273	0\\
274	0\\
275	0\\
276	0\\
277	0\\
278	0\\
279	0\\
280	0\\
281	0\\
282	0\\
283	0\\
284	0\\
285	0\\
286	0\\
287	0\\
288	0\\
289	0\\
290	0\\
291	0\\
292	0\\
293	0\\
294	0\\
295	0\\
296	0\\
297	0\\
298	0\\
299	0\\
300	0\\
301	0\\
302	0\\
303	0\\
304	0\\
305	0\\
306	0\\
307	0\\
308	0\\
309	0\\
310	0\\
311	0\\
312	0\\
313	0\\
314	0\\
315	0\\
316	0\\
317	0\\
318	0\\
319	0\\
320	0\\
321	0\\
322	0\\
323	0\\
324	0\\
325	0\\
326	0\\
327	0\\
328	0\\
329	0\\
330	0\\
331	0\\
332	0\\
333	0\\
334	0\\
335	0\\
336	0\\
337	0\\
338	0\\
339	0\\
340	0\\
341	0\\
342	0\\
343	0\\
344	0\\
345	0\\
346	0\\
347	0\\
348	0\\
349	0\\
350	0\\
351	0\\
352	0\\
353	0\\
354	0\\
355	0\\
356	0\\
357	0\\
358	0\\
359	0\\
360	0\\
361	0\\
362	0\\
363	0\\
364	0\\
365	0\\
366	0\\
367	0\\
368	0\\
369	0\\
370	0\\
371	0\\
372	0\\
373	0\\
374	0\\
375	0\\
376	0\\
377	0\\
378	0\\
379	0\\
380	0\\
381	0\\
382	0\\
383	0\\
384	0\\
385	0\\
386	0\\
387	0\\
388	0\\
389	0\\
390	0\\
391	0\\
392	0\\
393	0\\
394	0\\
395	0\\
396	0\\
397	0\\
398	0\\
399	0\\
400	0\\
401	0\\
402	0\\
403	0\\
404	0\\
405	0\\
406	0\\
407	0\\
408	0\\
409	0\\
410	0\\
411	0\\
412	0\\
413	0\\
414	0\\
415	0\\
416	0\\
417	0\\
418	0\\
419	0\\
420	0\\
421	0\\
422	0\\
423	0\\
424	0\\
425	0\\
426	0\\
427	0\\
428	0\\
429	0\\
430	0\\
431	0\\
432	0\\
433	0\\
434	0\\
435	0\\
436	0\\
437	0\\
438	0\\
439	0\\
440	0\\
441	0\\
442	0\\
443	0\\
444	0\\
445	0\\
446	0\\
447	0\\
448	0\\
449	0\\
450	0\\
451	0\\
452	0\\
453	0\\
454	0\\
455	0\\
456	0\\
457	0\\
458	0\\
459	0\\
460	0\\
461	0\\
462	0\\
463	0\\
464	0\\
465	0\\
466	0\\
467	0\\
468	0\\
469	0\\
470	0\\
471	0\\
472	0\\
473	0\\
474	0\\
475	0\\
476	0\\
477	0\\
478	0\\
479	0\\
480	0\\
481	0\\
482	0\\
483	0\\
484	0\\
485	0\\
486	0\\
487	0\\
488	0\\
489	0\\
490	0\\
491	0\\
492	0\\
493	0\\
494	0\\
495	0\\
496	0\\
497	0\\
498	0\\
499	0\\
500	0\\
501	0\\
502	0\\
503	0\\
504	0\\
505	0\\
506	0\\
507	0\\
508	0\\
509	0\\
510	0\\
511	0\\
512	0\\
513	0\\
514	0\\
515	0\\
516	0\\
517	0\\
518	0\\
519	0\\
520	0\\
521	0\\
522	0\\
523	0\\
524	0\\
525	0\\
526	0\\
527	0\\
528	0\\
529	0\\
530	0\\
531	0\\
532	0\\
533	0\\
534	0\\
535	0\\
536	0\\
537	0\\
538	0\\
539	0\\
540	0\\
541	0\\
542	0\\
543	0\\
544	0\\
545	0\\
546	0\\
547	0\\
548	0\\
549	0\\
550	0\\
551	0\\
552	0\\
553	0\\
554	0\\
555	0\\
556	0\\
557	0\\
558	0\\
559	0\\
560	0\\
561	0\\
562	0\\
563	0\\
564	0\\
565	0\\
566	0\\
567	0\\
568	0\\
569	0\\
570	0\\
571	0\\
572	0\\
573	0\\
574	0\\
575	0\\
576	0\\
577	0\\
578	0\\
579	0\\
580	0\\
581	0\\
582	0\\
583	0\\
584	0\\
585	0\\
586	0\\
587	0\\
588	0\\
589	0\\
590	0\\
591	0\\
592	0\\
593	0\\
594	0\\
595	0\\
596	0\\
597	0\\
598	0\\
599	0\\
600	0\\
};
\addplot [color=mycolor9,solid,forget plot]
  table[row sep=crcr]{%
1	0\\
2	0\\
3	0\\
4	0\\
5	0\\
6	0\\
7	0\\
8	0\\
9	0\\
10	0\\
11	0\\
12	0\\
13	0\\
14	0\\
15	0\\
16	0\\
17	0\\
18	0\\
19	0\\
20	0\\
21	0\\
22	0\\
23	0\\
24	0\\
25	0\\
26	0\\
27	0\\
28	0\\
29	0\\
30	0\\
31	0\\
32	0\\
33	0\\
34	0\\
35	0\\
36	0\\
37	0\\
38	0\\
39	0\\
40	0\\
41	0\\
42	0\\
43	0\\
44	0\\
45	0\\
46	0\\
47	0\\
48	0\\
49	0\\
50	0\\
51	0\\
52	0\\
53	0\\
54	0\\
55	0\\
56	0\\
57	0\\
58	0\\
59	0\\
60	0\\
61	0\\
62	0\\
63	0\\
64	0\\
65	0\\
66	0\\
67	0\\
68	0\\
69	0\\
70	0\\
71	0\\
72	0\\
73	0\\
74	0\\
75	0\\
76	0\\
77	0\\
78	0\\
79	0\\
80	0\\
81	0\\
82	0\\
83	0\\
84	0\\
85	0\\
86	0\\
87	0\\
88	0\\
89	0\\
90	0\\
91	0\\
92	0\\
93	0\\
94	0\\
95	0\\
96	0\\
97	0\\
98	0\\
99	0\\
100	0\\
101	0\\
102	0\\
103	0\\
104	0\\
105	0\\
106	0\\
107	0\\
108	0\\
109	0\\
110	0\\
111	0\\
112	0\\
113	0\\
114	0\\
115	0\\
116	0\\
117	0\\
118	0\\
119	0\\
120	0\\
121	0\\
122	0\\
123	0\\
124	0\\
125	0\\
126	0\\
127	0\\
128	0\\
129	0\\
130	0\\
131	0\\
132	0\\
133	0\\
134	0\\
135	0\\
136	0\\
137	0\\
138	0\\
139	0\\
140	0\\
141	0\\
142	0\\
143	0\\
144	0\\
145	0\\
146	0\\
147	0\\
148	0\\
149	0\\
150	0\\
151	0\\
152	0\\
153	0\\
154	0\\
155	0\\
156	0\\
157	0\\
158	0\\
159	0\\
160	0\\
161	0\\
162	0\\
163	0\\
164	0\\
165	0\\
166	0\\
167	0\\
168	0\\
169	0\\
170	0\\
171	0\\
172	0\\
173	0\\
174	0\\
175	0\\
176	0\\
177	0\\
178	0\\
179	0\\
180	0\\
181	0\\
182	0\\
183	0\\
184	0\\
185	0\\
186	0\\
187	0\\
188	0\\
189	0\\
190	0\\
191	0\\
192	0\\
193	0\\
194	0\\
195	0\\
196	0\\
197	0\\
198	0\\
199	0\\
200	0\\
201	0\\
202	0\\
203	0\\
204	0\\
205	0\\
206	0\\
207	0\\
208	0\\
209	0\\
210	0\\
211	0\\
212	0\\
213	0\\
214	0\\
215	0\\
216	0\\
217	0\\
218	0\\
219	0\\
220	0\\
221	0\\
222	0\\
223	0\\
224	0\\
225	0\\
226	0\\
227	0\\
228	0\\
229	0\\
230	0\\
231	0\\
232	0\\
233	0\\
234	0\\
235	0\\
236	0\\
237	0\\
238	0\\
239	0\\
240	0\\
241	0\\
242	0\\
243	0\\
244	0\\
245	0\\
246	0\\
247	0\\
248	0\\
249	0\\
250	0\\
251	0\\
252	0\\
253	0\\
254	0\\
255	0\\
256	0\\
257	0\\
258	0\\
259	0\\
260	0\\
261	0\\
262	0\\
263	0\\
264	0\\
265	0\\
266	0\\
267	0\\
268	0\\
269	0\\
270	0\\
271	0\\
272	0\\
273	0\\
274	0\\
275	0\\
276	0\\
277	0\\
278	0\\
279	0\\
280	0\\
281	0\\
282	0\\
283	0\\
284	0\\
285	0\\
286	0\\
287	0\\
288	0\\
289	0\\
290	0\\
291	0\\
292	0\\
293	0\\
294	0\\
295	0\\
296	0\\
297	0\\
298	0\\
299	0\\
300	0\\
301	0\\
302	0\\
303	0\\
304	0\\
305	0\\
306	0\\
307	0\\
308	0\\
309	0\\
310	0\\
311	0\\
312	0\\
313	0\\
314	0\\
315	0\\
316	0\\
317	0\\
318	0\\
319	0\\
320	0\\
321	0\\
322	0\\
323	0\\
324	0\\
325	0\\
326	0\\
327	0\\
328	0\\
329	0\\
330	0\\
331	0\\
332	0\\
333	0\\
334	0\\
335	0\\
336	0\\
337	0\\
338	0\\
339	0\\
340	0\\
341	0\\
342	0\\
343	0\\
344	0\\
345	0\\
346	0\\
347	0\\
348	0\\
349	0\\
350	0\\
351	0\\
352	0\\
353	0\\
354	0\\
355	0\\
356	0\\
357	0\\
358	0\\
359	0\\
360	0\\
361	0\\
362	0\\
363	0\\
364	0\\
365	0\\
366	0\\
367	0\\
368	0\\
369	0\\
370	0\\
371	0\\
372	0\\
373	0\\
374	0\\
375	0\\
376	0\\
377	0\\
378	0\\
379	0\\
380	0\\
381	0\\
382	0\\
383	0\\
384	0\\
385	0\\
386	0\\
387	0\\
388	0\\
389	0\\
390	0\\
391	0\\
392	0\\
393	0\\
394	0\\
395	0\\
396	0\\
397	0\\
398	0\\
399	0\\
400	0\\
401	0\\
402	0\\
403	0\\
404	0\\
405	0\\
406	0\\
407	0\\
408	0\\
409	0\\
410	0\\
411	0\\
412	0\\
413	0\\
414	0\\
415	0\\
416	0\\
417	0\\
418	0\\
419	0\\
420	0\\
421	0\\
422	0\\
423	0\\
424	0\\
425	0\\
426	0\\
427	0\\
428	0\\
429	0\\
430	0\\
431	0\\
432	0\\
433	0\\
434	0\\
435	0\\
436	0\\
437	0\\
438	0\\
439	0\\
440	0\\
441	0\\
442	0\\
443	0\\
444	0\\
445	0\\
446	0\\
447	0\\
448	0\\
449	0\\
450	0\\
451	0\\
452	0\\
453	0\\
454	0\\
455	0\\
456	0\\
457	0\\
458	0\\
459	0\\
460	0\\
461	0\\
462	0\\
463	0\\
464	0\\
465	0\\
466	0\\
467	0\\
468	0\\
469	0\\
470	0\\
471	0\\
472	0\\
473	0\\
474	0\\
475	0\\
476	0\\
477	0\\
478	0\\
479	0\\
480	0\\
481	0\\
482	0\\
483	0\\
484	0\\
485	0\\
486	0\\
487	0\\
488	0\\
489	0\\
490	0\\
491	0\\
492	0\\
493	0\\
494	0\\
495	0\\
496	0\\
497	0\\
498	0\\
499	0\\
500	0\\
501	0\\
502	0\\
503	0\\
504	0\\
505	0\\
506	0\\
507	0\\
508	0\\
509	0\\
510	0\\
511	0\\
512	0\\
513	0\\
514	0\\
515	0\\
516	0\\
517	0\\
518	0\\
519	0\\
520	0\\
521	0\\
522	0\\
523	0\\
524	0\\
525	0\\
526	0\\
527	0\\
528	0\\
529	0\\
530	0\\
531	0\\
532	0\\
533	0\\
534	0\\
535	0\\
536	0\\
537	0\\
538	0\\
539	0\\
540	0\\
541	0\\
542	0\\
543	0\\
544	0\\
545	0\\
546	0\\
547	0\\
548	0\\
549	0\\
550	0\\
551	0\\
552	0\\
553	0\\
554	0\\
555	0\\
556	0\\
557	0\\
558	0\\
559	0\\
560	0\\
561	0\\
562	0\\
563	0\\
564	0\\
565	0\\
566	0\\
567	0\\
568	0\\
569	0\\
570	0\\
571	0\\
572	0\\
573	0\\
574	0\\
575	0\\
576	0\\
577	0\\
578	0\\
579	0\\
580	0\\
581	0\\
582	0\\
583	0\\
584	0\\
585	0\\
586	0\\
587	0\\
588	0\\
589	0\\
590	0\\
591	0\\
592	0\\
593	0\\
594	0\\
595	0\\
596	0\\
597	0\\
598	0\\
599	0\\
600	0\\
};
\addplot [color=blue!50!mycolor7,solid,forget plot]
  table[row sep=crcr]{%
1	0\\
2	0\\
3	0\\
4	0\\
5	0\\
6	0\\
7	0\\
8	0\\
9	0\\
10	0\\
11	0\\
12	0\\
13	0\\
14	0\\
15	0\\
16	0\\
17	0\\
18	0\\
19	0\\
20	0\\
21	0\\
22	0\\
23	0\\
24	0\\
25	0\\
26	0\\
27	0\\
28	0\\
29	0\\
30	0\\
31	0\\
32	0\\
33	0\\
34	0\\
35	0\\
36	0\\
37	0\\
38	0\\
39	0\\
40	0\\
41	0\\
42	0\\
43	0\\
44	0\\
45	0\\
46	0\\
47	0\\
48	0\\
49	0\\
50	0\\
51	0\\
52	0\\
53	0\\
54	0\\
55	0\\
56	0\\
57	0\\
58	0\\
59	0\\
60	0\\
61	0\\
62	0\\
63	0\\
64	0\\
65	0\\
66	0\\
67	0\\
68	0\\
69	0\\
70	0\\
71	0\\
72	0\\
73	0\\
74	0\\
75	0\\
76	0\\
77	0\\
78	0\\
79	0\\
80	0\\
81	0\\
82	0\\
83	0\\
84	0\\
85	0\\
86	0\\
87	0\\
88	0\\
89	0\\
90	0\\
91	0\\
92	0\\
93	0\\
94	0\\
95	0\\
96	0\\
97	0\\
98	0\\
99	0\\
100	0\\
101	0\\
102	0\\
103	0\\
104	0\\
105	0\\
106	0\\
107	0\\
108	0\\
109	0\\
110	0\\
111	0\\
112	0\\
113	0\\
114	0\\
115	0\\
116	0\\
117	0\\
118	0\\
119	0\\
120	0\\
121	0\\
122	0\\
123	0\\
124	0\\
125	0\\
126	0\\
127	0\\
128	0\\
129	0\\
130	0\\
131	0\\
132	0\\
133	0\\
134	0\\
135	0\\
136	0\\
137	0\\
138	0\\
139	0\\
140	0\\
141	0\\
142	0\\
143	0\\
144	0\\
145	0\\
146	0\\
147	0\\
148	0\\
149	0\\
150	0\\
151	0\\
152	0\\
153	0\\
154	0\\
155	0\\
156	0\\
157	0\\
158	0\\
159	0\\
160	0\\
161	0\\
162	0\\
163	0\\
164	0\\
165	0\\
166	0\\
167	0\\
168	0\\
169	0\\
170	0\\
171	0\\
172	0\\
173	0\\
174	0\\
175	0\\
176	0\\
177	0\\
178	0\\
179	0\\
180	0\\
181	0\\
182	0\\
183	0\\
184	0\\
185	0\\
186	0\\
187	0\\
188	0\\
189	0\\
190	0\\
191	0\\
192	0\\
193	0\\
194	0\\
195	0\\
196	0\\
197	0\\
198	0\\
199	0\\
200	0\\
201	0\\
202	0\\
203	0\\
204	0\\
205	0\\
206	0\\
207	0\\
208	0\\
209	0\\
210	0\\
211	0\\
212	0\\
213	0\\
214	0\\
215	0\\
216	0\\
217	0\\
218	0\\
219	0\\
220	0\\
221	0\\
222	0\\
223	0\\
224	0\\
225	0\\
226	0\\
227	0\\
228	0\\
229	0\\
230	0\\
231	0\\
232	0\\
233	0\\
234	0\\
235	0\\
236	0\\
237	0\\
238	0\\
239	0\\
240	0\\
241	0\\
242	0\\
243	0\\
244	0\\
245	0\\
246	0\\
247	0\\
248	0\\
249	0\\
250	0\\
251	0\\
252	0\\
253	0\\
254	0\\
255	0\\
256	0\\
257	0\\
258	0\\
259	0\\
260	0\\
261	0\\
262	0\\
263	0\\
264	0\\
265	0\\
266	0\\
267	0\\
268	0\\
269	0\\
270	0\\
271	0\\
272	0\\
273	0\\
274	0\\
275	0\\
276	0\\
277	0\\
278	0\\
279	0\\
280	0\\
281	0\\
282	0\\
283	0\\
284	0\\
285	0\\
286	0\\
287	0\\
288	0\\
289	0\\
290	0\\
291	0\\
292	0\\
293	0\\
294	0\\
295	0\\
296	0\\
297	0\\
298	0\\
299	0\\
300	0\\
301	0\\
302	0\\
303	0\\
304	0\\
305	0\\
306	0\\
307	0\\
308	0\\
309	0\\
310	0\\
311	0\\
312	0\\
313	0\\
314	0\\
315	0\\
316	0\\
317	0\\
318	0\\
319	0\\
320	0\\
321	0\\
322	0\\
323	0\\
324	0\\
325	0\\
326	0\\
327	0\\
328	0\\
329	0\\
330	0\\
331	0\\
332	0\\
333	0\\
334	0\\
335	0\\
336	0\\
337	0\\
338	0\\
339	0\\
340	0\\
341	0\\
342	0\\
343	0\\
344	0\\
345	0\\
346	0\\
347	0\\
348	0\\
349	0\\
350	0\\
351	0\\
352	0\\
353	0\\
354	0\\
355	0\\
356	0\\
357	0\\
358	0\\
359	0\\
360	0\\
361	0\\
362	0\\
363	0\\
364	0\\
365	0\\
366	0\\
367	0\\
368	0\\
369	0\\
370	0\\
371	0\\
372	0\\
373	0\\
374	0\\
375	0\\
376	0\\
377	0\\
378	0\\
379	0\\
380	0\\
381	0\\
382	0\\
383	0\\
384	0\\
385	0\\
386	0\\
387	0\\
388	0\\
389	0\\
390	0\\
391	0\\
392	0\\
393	0\\
394	0\\
395	0\\
396	0\\
397	0\\
398	0\\
399	0\\
400	0\\
401	0\\
402	0\\
403	0\\
404	0\\
405	0\\
406	0\\
407	0\\
408	0\\
409	0\\
410	0\\
411	0\\
412	0\\
413	0\\
414	0\\
415	0\\
416	0\\
417	0\\
418	0\\
419	0\\
420	0\\
421	0\\
422	0\\
423	0\\
424	0\\
425	0\\
426	0\\
427	0\\
428	0\\
429	0\\
430	0\\
431	0\\
432	0\\
433	0\\
434	0\\
435	0\\
436	0\\
437	0\\
438	0\\
439	0\\
440	0\\
441	0\\
442	0\\
443	0\\
444	0\\
445	0\\
446	0\\
447	0\\
448	0\\
449	0\\
450	0\\
451	0\\
452	0\\
453	0\\
454	0\\
455	0\\
456	0\\
457	0\\
458	0\\
459	0\\
460	0\\
461	0\\
462	0\\
463	0\\
464	0\\
465	0\\
466	0\\
467	0\\
468	0\\
469	0\\
470	0\\
471	0\\
472	0\\
473	0\\
474	0\\
475	0\\
476	0\\
477	0\\
478	0\\
479	0\\
480	0\\
481	0\\
482	0\\
483	0\\
484	0\\
485	0\\
486	0\\
487	0\\
488	0\\
489	0\\
490	0\\
491	0\\
492	0\\
493	0\\
494	0\\
495	0\\
496	0\\
497	0\\
498	0\\
499	0\\
500	0\\
501	0\\
502	0\\
503	0\\
504	0\\
505	0\\
506	0\\
507	0\\
508	0\\
509	0\\
510	0\\
511	0\\
512	0\\
513	0\\
514	0\\
515	0\\
516	0\\
517	0\\
518	0\\
519	0\\
520	0\\
521	0\\
522	0\\
523	0\\
524	0\\
525	0\\
526	0\\
527	0\\
528	0\\
529	0\\
530	0\\
531	0\\
532	0\\
533	0\\
534	0\\
535	0\\
536	0\\
537	0\\
538	0\\
539	0\\
540	0\\
541	0\\
542	0\\
543	0\\
544	0\\
545	0\\
546	0\\
547	0\\
548	0\\
549	0\\
550	0\\
551	0\\
552	0\\
553	0\\
554	0\\
555	0\\
556	0\\
557	0\\
558	0\\
559	0\\
560	0\\
561	0\\
562	0\\
563	0\\
564	0\\
565	0\\
566	0\\
567	0\\
568	0\\
569	0\\
570	0\\
571	0\\
572	0\\
573	0\\
574	0\\
575	0\\
576	0\\
577	0\\
578	0\\
579	0\\
580	0\\
581	0\\
582	0\\
583	0\\
584	0\\
585	0\\
586	0\\
587	0\\
588	0\\
589	0\\
590	0\\
591	0\\
592	0\\
593	0\\
594	0\\
595	0\\
596	0\\
597	0\\
598	0\\
599	0\\
600	0\\
};
\addplot [color=blue!40!mycolor9,solid,forget plot]
  table[row sep=crcr]{%
1	0.000572289361079503\\
2	0.000572279828444312\\
3	0.000572270135439682\\
4	0.000572260279371882\\
5	0.000572250257501964\\
6	0.000572240067045005\\
7	0.000572229705169346\\
8	0.00057221916899576\\
9	0.000572208455596701\\
10	0.000572197561995472\\
11	0.000572186485165404\\
12	0.000572175222028997\\
13	0.000572163769457075\\
14	0.000572152124267915\\
15	0.000572140283226384\\
16	0.000572128243043005\\
17	0.000572116000373054\\
18	0.000572103551815636\\
19	0.000572090893912736\\
20	0.000572078023148238\\
21	0.000572064935946996\\
22	0.000572051628673754\\
23	0.000572038097632211\\
24	0.000572024339063951\\
25	0.000572010349147427\\
26	0.000571996123996853\\
27	0.000571981659661137\\
28	0.000571966952122788\\
29	0.00057195199729678\\
30	0.000571936791029434\\
31	0.000571921329097228\\
32	0.000571905607205632\\
33	0.000571889620987887\\
34	0.000571873366003836\\
35	0.000571856837738625\\
36	0.000571840031601462\\
37	0.000571822942924336\\
38	0.000571805566960702\\
39	0.000571787898884128\\
40	0.00057176993378699\\
41	0.000571751666679058\\
42	0.000571733092486115\\
43	0.000571714206048501\\
44	0.000571695002119696\\
45	0.000571675475364842\\
46	0.000571655620359213\\
47	0.000571635431586721\\
48	0.00057161490343834\\
49	0.000571594030210513\\
50	0.000571572806103606\\
51	0.000571551225220185\\
52	0.000571529281563428\\
53	0.000571506969035371\\
54	0.000571484281435221\\
55	0.000571461212457569\\
56	0.00057143775569063\\
57	0.000571413904614426\\
58	0.000571389652598912\\
59	0.000571364992902124\\
60	0.000571339918668244\\
61	0.000571314422925635\\
62	0.000571288498584887\\
63	0.000571262138436795\\
64	0.000571235335150263\\
65	0.000571208081270274\\
66	0.00057118036921569\\
67	0.000571152191277152\\
68	0.000571123539614814\\
69	0.00057109440625612\\
70	0.000571064783093518\\
71	0.000571034661882132\\
72	0.000571004034237361\\
73	0.000570972891632495\\
74	0.000570941225396272\\
75	0.000570909026710325\\
76	0.000570876286606628\\
77	0.000570842995964968\\
78	0.000570809145510212\\
79	0.0005707747258097\\
80	0.00057073972727042\\
81	0.000570704140136268\\
82	0.000570667954485155\\
83	0.000570631160226152\\
84	0.000570593747096516\\
85	0.000570555704658669\\
86	0.000570517022297149\\
87	0.000570477689215458\\
88	0.000570437694432896\\
89	0.000570397026781319\\
90	0.000570355674901806\\
91	0.000570313627241323\\
92	0.000570270872049269\\
93	0.00057022739737396\\
94	0.000570183191059074\\
95	0.000570138240739997\\
96	0.000570092533840138\\
97	0.000570046057567128\\
98	0.000569998798908942\\
99	0.00056995074462999\\
100	0.000569901881267093\\
101	0.000569852195125393\\
102	0.00056980167227418\\
103	0.00056975029854262\\
104	0.000569698059515415\\
105	0.000569644940528375\\
106	0.000569590926663907\\
107	0.000569536002746355\\
108	0.000569480153337341\\
109	0.000569423362730919\\
110	0.000569365614948678\\
111	0.000569306893734748\\
112	0.000569247182550685\\
113	0.000569186464570213\\
114	0.000569124722673949\\
115	0.000569061939443922\\
116	0.000568998097158043\\
117	0.000568933177784415\\
118	0.000568867162975534\\
119	0.000568800034062384\\
120	0.000568731772048393\\
121	0.000568662357603254\\
122	0.000568591771056625\\
123	0.00056851999239169\\
124	0.000568447001238563\\
125	0.000568372776867582\\
126	0.000568297298182457\\
127	0.000568220543713245\\
128	0.000568142491609186\\
129	0.000568063119631404\\
130	0.000567982405145427\\
131	0.000567900325113538\\
132	0.000567816856087025\\
133	0.000567731974198204\\
134	0.000567645655152298\\
135	0.000567557874219155\\
136	0.000567468606224778\\
137	0.000567377825542686\\
138	0.000567285506085131\\
139	0.000567191621294093\\
140	0.000567096144132122\\
141	0.000566999047073011\\
142	0.00056690030209229\\
143	0.000566799880657502\\
144	0.000566697753718357\\
145	0.000566593891696704\\
146	0.000566488264476299\\
147	0.000566380841392455\\
148	0.000566271591221453\\
149	0.000566160482169865\\
150	0.0005660474818637\\
151	0.000565932557337377\\
152	0.000565815675022591\\
153	0.000565696800737025\\
154	0.000565575899672972\\
155	0.000565452936385781\\
156	0.000565327874782326\\
157	0.00056520067810925\\
158	0.000565071308941251\\
159	0.000564939729169241\\
160	0.000564805899988544\\
161	0.000564669781887021\\
162	0.000564531334633243\\
163	0.00056439051726466\\
164	0.000564247288075846\\
165	0.000564101604606802\\
166	0.000563953423631374\\
167	0.000563802701145759\\
168	0.000563649392357172\\
169	0.000563493451672658\\
170	0.000563334832688151\\
171	0.000563173488177561\\
172	0.000563009370082275\\
173	0.000562842429500666\\
174	0.000562672616677942\\
175	0.000562499880996109\\
176	0.000562324170964073\\
177	0.000562145434207974\\
178	0.000561963617461486\\
179	0.00056177866655611\\
180	0.000561590526411449\\
181	0.000561399141025247\\
182	0.000561204453462971\\
183	0.000561006405846999\\
184	0.000560804939345069\\
185	0.000560599994157678\\
186	0.000560391509504379\\
187	0.000560179423608616\\
188	0.000559963673680764\\
189	0.000559744195899196\\
190	0.00055952092538907\\
191	0.000559293796198689\\
192	0.000559062741273323\\
193	0.000558827692426567\\
194	0.000558588580309725\\
195	0.000558345334379785\\
196	0.000558097882867651\\
197	0.000557846152747879\\
198	0.000557590069711211\\
199	0.000557329558141205\\
200	0.000557064541090551\\
201	0.000556794940256949\\
202	0.000556520675958512\\
203	0.000556241667108771\\
204	0.000555957831191096\\
205	0.000555669084232686\\
206	0.000555375340778087\\
207	0.000555076513862098\\
208	0.00055477251498228\\
209	0.000554463254070845\\
210	0.000554148639466027\\
211	0.000553828577882921\\
212	0.00055350297438373\\
213	0.000553171732347426\\
214	0.000552834753438846\\
215	0.000552491937577186\\
216	0.000552143182903838\\
217	0.000551788385749663\\
218	0.000551427440601567\\
219	0.000551060240068446\\
220	0.000550686674846437\\
221	0.000550306633683534\\
222	0.000549920003343464\\
223	0.000549526668568835\\
224	0.000549126512043608\\
225	0.000548719414354775\\
226	0.000548305253953295\\
227	0.000547883907114225\\
228	0.0005474552478961\\
229	0.000547019148099482\\
230	0.000546575477224634\\
231	0.000546124102428374\\
232	0.000545664888480091\\
233	0.000545197697716812\\
234	0.000544722389997394\\
235	0.000544238822655763\\
236	0.000543746850453227\\
237	0.00054324632552977\\
238	0.000542737097354379\\
239	0.000542219012674355\\
240	0.000541691915463587\\
241	0.000541155646869651\\
242	0.000540610045160008\\
243	0.000540054945666896\\
244	0.000539490180731165\\
245	0.000538915579644934\\
246	0.000538330968592991\\
247	0.000537736170593055\\
248	0.000537131005434672\\
249	0.00053651528961688\\
250	0.000535888836284555\\
251	0.000535251455163334\\
252	0.000534602952493253\\
253	0.000533943130960853\\
254	0.00053327178962992\\
255	0.000532588723870659\\
256	0.000531893725287404\\
257	0.000531186581644703\\
258	0.000530467076791844\\
259	0.000529734990585686\\
260	0.000528990098811893\\
261	0.000528232173104341\\
262	0.00052746098086278\\
263	0.000526676285168801\\
264	0.000525877844699746\\
265	0.000525065413640908\\
266	0.000524238741595681\\
267	0.000523397573493714\\
268	0.000522541649497094\\
269	0.000521670704904323\\
270	0.000520784470052275\\
271	0.000519882670215849\\
272	0.000518965025505488\\
273	0.000518031250762205\\
274	0.000517081055450443\\
275	0.000516114143548415\\
276	0.000515130213435959\\
277	0.000514128957779876\\
278	0.00051311006341665\\
279	0.000512073211232512\\
280	0.00051101807604073\\
281	0.000509944326456171\\
282	0.000508851624766938\\
283	0.000507739626803091\\
284	0.000506607981802348\\
285	0.000505456332272721\\
286	0.000504284313852017\\
287	0.000503091555164066\\
288	0.000501877677671679\\
289	0.00050064229552625\\
290	0.000499385015413872\\
291	0.000498105436397899\\
292	0.000496803149758002\\
293	0.00049547773882538\\
294	0.000494128778814315\\
295	0.000492755836649867\\
296	0.000491358470791584\\
297	0.000489936231053213\\
298	0.000488488658418356\\
299	0.000487015284851931\\
300	0.00048551563310734\\
301	0.000483989216529351\\
302	0.000482435538852573\\
303	0.000480854093995371\\
304	0.000479244365849313\\
305	0.000477605828063928\\
306	0.000475937943826749\\
307	0.000474240165638641\\
308	0.000472511935084228\\
309	0.000470752682597514\\
310	0.000468961827222439\\
311	0.000467138776368593\\
312	0.000465282925561809\\
313	0.000463393658189717\\
314	0.000461470345242214\\
315	0.000459512345046848\\
316	0.000457519002999091\\
317	0.000455489651287555\\
318	0.000453423608614047\\
319	0.000451320179908691\\
320	0.000449178656040015\\
321	0.000446998313520135\\
322	0.00044477841420512\\
323	0.000442518204990639\\
324	0.000440216917503031\\
325	0.000437873767785933\\
326	0.000435487955982651\\
327	0.000433058666014573\\
328	0.000430585065255739\\
329	0.000428066304203932\\
330	0.000425501516148543\\
331	0.000422889816835637\\
332	0.000420230304130622\\
333	0.00041752205767883\\
334	0.000414764138564654\\
335	0.000411955588969695\\
336	0.000409095431830549\\
337	0.000406182670496853\\
338	0.000403216288390314\\
339	0.000400195248665453\\
340	0.000397118493873001\\
341	0.000393984945626641\\
342	0.000390793504274319\\
343	0.000387543048574891\\
344	0.00038423243538136\\
345	0.000380860499331881\\
346	0.000377426052549601\\
347	0.000373927884352812\\
348	0.000370364760976633\\
349	0.000366735425307715\\
350	0.000363038596633317\\
351	0.000359272970406324\\
352	0.000355437218027721\\
353	0.000351529986647915\\
354	0.000347549898988649\\
355	0.000343495553186723\\
356	0.000339365522661187\\
357	0.000335158356005238\\
358	0.000330872576904045\\
359	0.000326506684079658\\
360	0.000322059151263819\\
361	0.000317528427199376\\
362	0.000312912935670473\\
363	0.000308211075561671\\
364	0.000303421220945299\\
365	0.000298541721196293\\
366	0.000293570901132558\\
367	0.000288507061178865\\
368	0.000283348477551088\\
369	0.000278093402456643\\
370	0.000272740064306439\\
371	0.000267286667932114\\
372	0.000261731394801327\\
373	0.000256072403222602\\
374	0.000250307828529804\\
375	0.000244435783234991\\
376	0.000238454357136863\\
377	0.000232361617370447\\
378	0.000226155608382433\\
379	0.000219834351815218\\
380	0.000213395846281269\\
381	0.000206838067008912\\
382	0.000200158965339931\\
383	0.000193356468059979\\
384	0.000186428476544544\\
385	0.000179372865706474\\
386	0.00017218748273451\\
387	0.000164870145606742\\
388	0.000157418641337741\\
389	0.000149830723905435\\
390	0.000142104111922943\\
391	0.00013423648565384\\
392	0.000126225483485972\\
393	0.000118068697859579\\
394	0.00010976367063733\\
395	0.000101307887735736\\
396	9.26987715723901e-05\\
397	8.39336683081326e-05\\
398	7.50098166564155e-05\\
399	6.59242533909565e-05\\
400	5.66735099699828e-05\\
401	4.72526162782195e-05\\
402	3.76517861404188e-05\\
403	2.78452662667821e-05\\
404	1.77533983572104e-05\\
405	7.11205651264132e-06\\
406	0\\
407	0\\
408	0\\
409	0\\
410	0\\
411	0\\
412	0\\
413	0\\
414	0\\
415	0\\
416	0\\
417	0\\
418	0\\
419	0\\
420	0\\
421	0\\
422	0\\
423	0\\
424	0\\
425	0\\
426	0\\
427	0\\
428	0\\
429	0\\
430	0\\
431	0\\
432	0\\
433	0\\
434	0\\
435	0\\
436	0\\
437	0\\
438	0\\
439	0\\
440	0\\
441	0\\
442	0\\
443	0\\
444	0\\
445	0\\
446	0\\
447	0\\
448	0\\
449	0\\
450	0\\
451	0\\
452	0\\
453	0\\
454	0\\
455	0\\
456	0\\
457	0\\
458	0\\
459	0\\
460	0\\
461	0\\
462	0\\
463	0\\
464	0\\
465	0\\
466	0\\
467	0\\
468	0\\
469	0\\
470	0\\
471	0\\
472	0\\
473	0\\
474	0\\
475	0\\
476	0\\
477	0\\
478	0\\
479	0\\
480	0\\
481	0\\
482	0\\
483	0\\
484	0\\
485	0\\
486	0\\
487	0\\
488	0\\
489	0\\
490	0\\
491	0\\
492	0\\
493	0\\
494	0\\
495	0\\
496	0\\
497	0\\
498	0\\
499	0\\
500	0\\
501	0\\
502	0\\
503	0\\
504	0\\
505	0\\
506	0\\
507	0\\
508	0\\
509	0\\
510	0\\
511	0\\
512	0\\
513	0\\
514	0\\
515	0\\
516	0\\
517	0\\
518	0\\
519	0\\
520	0\\
521	0\\
522	0\\
523	0\\
524	0\\
525	0\\
526	0\\
527	0\\
528	0\\
529	0\\
530	0\\
531	0\\
532	0\\
533	0\\
534	0\\
535	0\\
536	0\\
537	0\\
538	0\\
539	0\\
540	0\\
541	0\\
542	0\\
543	0\\
544	0\\
545	0\\
546	0\\
547	0\\
548	0\\
549	0\\
550	0\\
551	0\\
552	0\\
553	0\\
554	0\\
555	0\\
556	0\\
557	0\\
558	0\\
559	0\\
560	0\\
561	0\\
562	0\\
563	0\\
564	0\\
565	0\\
566	0\\
567	0\\
568	0\\
569	0\\
570	0\\
571	0\\
572	0\\
573	0\\
574	0\\
575	0\\
576	0\\
577	0\\
578	0\\
579	0\\
580	0\\
581	0\\
582	0\\
583	0\\
584	0\\
585	0\\
586	0\\
587	0\\
588	0\\
589	0\\
590	0\\
591	0\\
592	0\\
593	0\\
594	0\\
595	0\\
596	0\\
597	0\\
598	0\\
599	0\\
600	0\\
};
\addplot [color=blue!75!mycolor7,solid,forget plot]
  table[row sep=crcr]{%
1	0.00162295980517879\\
2	0.00162295046252148\\
3	0.00162294096273216\\
4	0.00162293130316988\\
5	0.00162292148114924\\
6	0.0016229114939396\\
7	0.00162290133876437\\
8	0.00162289101280022\\
9	0.00162288051317626\\
10	0.00162286983697323\\
11	0.00162285898122273\\
12	0.00162284794290632\\
13	0.00162283671895473\\
14	0.00162282530624696\\
15	0.00162281370160937\\
16	0.00162280190181487\\
17	0.00162278990358195\\
18	0.00162277770357375\\
19	0.00162276529839714\\
20	0.00162275268460179\\
21	0.00162273985867911\\
22	0.00162272681706135\\
23	0.00162271355612052\\
24	0.0016227000721674\\
25	0.00162268636145049\\
26	0.00162267242015493\\
27	0.00162265824440144\\
28	0.00162264383024521\\
29	0.00162262917367475\\
30	0.00162261427061081\\
31	0.00162259911690516\\
32	0.00162258370833943\\
33	0.00162256804062393\\
34	0.0016225521093964\\
35	0.00162253591022075\\
36	0.00162251943858585\\
37	0.00162250268990415\\
38	0.00162248565951048\\
39	0.00162246834266063\\
40	0.00162245073452999\\
41	0.00162243283021221\\
42	0.00162241462471777\\
43	0.00162239611297254\\
44	0.00162237728981633\\
45	0.00162235815000139\\
46	0.0016223386881909\\
47	0.00162231889895745\\
48	0.00162229877678144\\
49	0.00162227831604951\\
50	0.00162225751105288\\
51	0.00162223635598575\\
52	0.00162221484494356\\
53	0.00162219297192129\\
54	0.00162217073081174\\
55	0.00162214811540372\\
56	0.00162212511938027\\
57	0.00162210173631675\\
58	0.00162207795967905\\
59	0.0016220537828216\\
60	0.00162202919898548\\
61	0.00162200420129641\\
62	0.00162197878276273\\
63	0.00162195293627333\\
64	0.00162192665459562\\
65	0.00162189993037326\\
66	0.00162187275612416\\
67	0.00162184512423811\\
68	0.00162181702697464\\
69	0.00162178845646065\\
70	0.00162175940468811\\
71	0.00162172986351164\\
72	0.00162169982464612\\
73	0.00162166927966421\\
74	0.00162163821999377\\
75	0.00162160663691539\\
76	0.00162157452155972\\
77	0.00162154186490478\\
78	0.00162150865777329\\
79	0.00162147489082988\\
80	0.00162144055457829\\
81	0.00162140563935846\\
82	0.00162137013534365\\
83	0.00162133403253743\\
84	0.00162129732077062\\
85	0.00162125998969823\\
86	0.00162122202879629\\
87	0.00162118342735862\\
88	0.00162114417449357\\
89	0.00162110425912068\\
90	0.00162106366996724\\
91	0.00162102239556486\\
92	0.00162098042424589\\
93	0.00162093774413989\\
94	0.00162089434316986\\
95	0.00162085020904855\\
96	0.00162080532927466\\
97	0.00162075969112888\\
98	0.00162071328167\\
99	0.00162066608773084\\
100	0.00162061809591408\\
101	0.00162056929258815\\
102	0.0016205196638829\\
103	0.00162046919568523\\
104	0.00162041787363467\\
105	0.00162036568311883\\
106	0.00162031260926882\\
107	0.00162025863695449\\
108	0.00162020375077967\\
109	0.00162014793507727\\
110	0.0016200911739043\\
111	0.00162003345103682\\
112	0.00161997474996473\\
113	0.00161991505388653\\
114	0.00161985434570394\\
115	0.00161979260801643\\
116	0.00161972982311564\\
117	0.00161966597297978\\
118	0.00161960103926773\\
119	0.00161953500331324\\
120	0.00161946784611892\\
121	0.00161939954835011\\
122	0.00161933009032872\\
123	0.00161925945202685\\
124	0.00161918761306037\\
125	0.00161911455268241\\
126	0.00161904024977661\\
127	0.00161896468285042\\
128	0.00161888783002816\\
129	0.00161880966904399\\
130	0.00161873017723482\\
131	0.00161864933153302\\
132	0.00161856710845903\\
133	0.00161848348411394\\
134	0.00161839843417181\\
135	0.00161831193387193\\
136	0.00161822395801102\\
137	0.00161813448093525\\
138	0.00161804347653212\\
139	0.00161795091822225\\
140	0.00161785677895108\\
141	0.00161776103118042\\
142	0.00161766364687985\\
143	0.00161756459751812\\
144	0.00161746385405431\\
145	0.00161736138692892\\
146	0.00161725716605491\\
147	0.00161715116080853\\
148	0.00161704334002016\\
149	0.00161693367196489\\
150	0.00161682212435318\\
151	0.00161670866432125\\
152	0.00161659325842153\\
153	0.00161647587261285\\
154	0.00161635647225068\\
155	0.0016162350220772\\
156	0.00161611148621123\\
157	0.00161598582813821\\
158	0.00161585801069996\\
159	0.0016157279960844\\
160	0.00161559574581517\\
161	0.00161546122074116\\
162	0.00161532438102591\\
163	0.00161518518613694\\
164	0.00161504359483497\\
165	0.00161489956516297\\
166	0.00161475305443516\\
167	0.00161460401922579\\
168	0.00161445241535785\\
169	0.00161429819789152\\
170	0.00161414132111245\\
171	0.00161398173851994\\
172	0.00161381940281468\\
173	0.00161365426588638\\
174	0.00161348627880102\\
175	0.00161331539178782\\
176	0.00161314155422576\\
177	0.00161296471462976\\
178	0.00161278482063631\\
179	0.00161260181898868\\
180	0.00161241565552155\\
181	0.001612226275145\\
182	0.00161203362182788\\
183	0.00161183763858056\\
184	0.00161163826743685\\
185	0.00161143544943531\\
186	0.00161122912459969\\
187	0.00161101923191878\\
188	0.00161080570932537\\
189	0.00161058849367476\\
190	0.00161036752072246\\
191	0.00161014272510164\\
192	0.00160991404030002\\
193	0.00160968139863671\\
194	0.00160944473123896\\
195	0.00160920396801896\\
196	0.0016089590376509\\
197	0.00160870986754805\\
198	0.00160845638383983\\
199	0.00160819851134835\\
200	0.00160793617356469\\
201	0.00160766929262451\\
202	0.00160739778928338\\
203	0.00160712158289149\\
204	0.00160684059136798\\
205	0.00160655473117476\\
206	0.00160626391728973\\
207	0.00160596806317967\\
208	0.00160566708077244\\
209	0.00160536088042871\\
210	0.00160504937091316\\
211	0.00160473245936512\\
212	0.0016044100512686\\
213	0.0016040820504218\\
214	0.001603748358906\\
215	0.00160340887705384\\
216	0.00160306350341704\\
217	0.00160271213473337\\
218	0.00160235466589315\\
219	0.00160199098990493\\
220	0.00160162099786064\\
221	0.0016012445789\\
222	0.00160086162017417\\
223	0.00160047200680884\\
224	0.00160007562186647\\
225	0.00159967234630785\\
226	0.00159926205895291\\
227	0.00159884463644074\\
228	0.00159841995318881\\
229	0.00159798788135147\\
230	0.00159754829077751\\
231	0.00159710104896702\\
232	0.00159664602102728\\
233	0.00159618306962786\\
234	0.00159571205495483\\
235	0.001595232834664\\
236	0.00159474526383333\\
237	0.00159424919491431\\
238	0.00159374447768245\\
239	0.00159323095918677\\
240	0.00159270848369826\\
241	0.00159217689265735\\
242	0.00159163602462031\\
243	0.00159108571520466\\
244	0.00159052579703339\\
245	0.00158995609967814\\
246	0.00158937644960128\\
247	0.00158878667009675\\
248	0.00158818658122983\\
249	0.00158757599977561\\
250	0.00158695473915633\\
251	0.00158632260937742\\
252	0.00158567941696233\\
253	0.00158502496488597\\
254	0.00158435905250693\\
255	0.00158368147549834\\
256	0.00158299202577729\\
257	0.00158229049143296\\
258	0.00158157665665321\\
259	0.00158085030164991\\
260	0.00158011120258253\\
261	0.0015793591314805\\
262	0.00157859385616387\\
263	0.00157781514016245\\
264	0.00157702274263341\\
265	0.00157621641827724\\
266	0.00157539591725204\\
267	0.00157456098508619\\
268	0.00157371136258933\\
269	0.00157284678576154\\
270	0.00157196698570083\\
271	0.00157107168850885\\
272	0.00157016061519472\\
273	0.0015692334815771\\
274	0.00156828999818434\\
275	0.00156732987015272\\
276	0.00156635279712276\\
277	0.00156535847313365\\
278	0.00156434658651553\\
279	0.00156331681977994\\
280	0.00156226884950806\\
281	0.00156120234623699\\
282	0.00156011697434384\\
283	0.00155901239192775\\
284	0.00155788825068978\\
285	0.00155674419581046\\
286	0.00155557986582534\\
287	0.00155439489249817\\
288	0.00155318890069183\\
289	0.00155196150823705\\
290	0.00155071232579872\\
291	0.00154944095673998\\
292	0.00154814699698386\\
293	0.00154683003487266\\
294	0.00154548965102486\\
295	0.00154412541818962\\
296	0.00154273690109895\\
297	0.00154132365631739\\
298	0.00153988523208923\\
299	0.00153842116818331\\
300	0.00153693099573536\\
301	0.00153541423708787\\
302	0.00153387040562747\\
303	0.00153229900561981\\
304	0.00153069953204209\\
305	0.00152907147041296\\
306	0.00152741429662004\\
307	0.00152572747674503\\
308	0.00152401046688623\\
309	0.00152226271297875\\
310	0.00152048365061228\\
311	0.00151867270484644\\
312	0.0015168292900237\\
313	0.00151495280958008\\
314	0.0015130426558534\\
315	0.00151109820988929\\
316	0.0015091188412449\\
317	0.00150710390779045\\
318	0.00150505275550851\\
319	0.00150296471829112\\
320	0.0015008391177349\\
321	0.00149867526293396\\
322	0.00149647245027083\\
323	0.00149422996320542\\
324	0.00149194707206201\\
325	0.0014896230338143\\
326	0.00148725709186868\\
327	0.00148484847584552\\
328	0.00148239640135883\\
329	0.00147990006979397\\
330	0.0014773586680837\\
331	0.00147477136848247\\
332	0.00147213732833884\\
333	0.00146945568986628\\
334	0.00146672557991202\\
335	0.00146394610972412\\
336	0.0014611163747165\\
337	0.00145823545423205\\
338	0.00145530241130345\\
339	0.0014523162924117\\
340	0.00144927612724208\\
341	0.00144618092843732\\
342	0.00144302969134765\\
343	0.00143982139377739\\
344	0.00143655499572767\\
345	0.00143322943913474\\
346	0.00142984364760332\\
347	0.00142639652613438\\
348	0.0014228869608465\\
349	0.00141931381868992\\
350	0.00141567594715247\\
351	0.00141197217395605\\
352	0.00140820130674252\\
353	0.00140436213274755\\
354	0.00140045341846087\\
355	0.00139647390927118\\
356	0.0013924223290937\\
357	0.00138829737997833\\
358	0.001384097741696\\
359	0.00137982207130068\\
360	0.00137546900266415\\
361	0.00137103714598066\\
362	0.00136652508723811\\
363	0.0013619313876523\\
364	0.00135725458306041\\
365	0.00135249318326998\\
366	0.0013476456713591\\
367	0.00134271050292344\\
368	0.00133768610526583\\
369	0.0013325708765238\\
370	0.0013273631847304\\
371	0.0013220613668038\\
372	0.00131666372746137\\
373	0.00131116853805386\\
374	0.0013055740353159\\
375	0.00129987842002959\\
376	0.00129407985559819\\
377	0.00128817646652833\\
378	0.00128216633681967\\
379	0.00127604750826262\\
380	0.00126981797864585\\
381	0.00126347569987737\\
382	0.00125701857602496\\
383	0.00125044446128388\\
384	0.00124375115788245\\
385	0.00123693641393882\\
386	0.00122999792128435\\
387	0.00122293331327162\\
388	0.00121574016258898\\
389	0.00120841597911064\\
390	0.00120095820781001\\
391	0.00119336422679553\\
392	0.00118563134552309\\
393	0.00117775680323375\\
394	0.00116973776762585\\
395	0.00116157133362488\\
396	0.00115325452171927\\
397	0.00114478427407384\\
398	0.00113615744329277\\
399	0.0011273707596541\\
400	0.00111842073867647\\
401	0.0011093034318014\\
402	0.00110001379547988\\
403	0.00109054426602079\\
404	0.00108088232894597\\
405	0.00107101049869601\\
406	0.00106092951432877\\
407	0.00105065977031644\\
408	0.00104019699284573\\
409	0.00102953677969847\\
410	0.00101867464720291\\
411	0.00100760617591885\\
412	0.000996327407526952\\
413	0.000984835854188185\\
414	0.000973132915070548\\
415	0.000961229198796052\\
416	0.000949154471044381\\
417	0.000936968437368094\\
418	0.000924731355869571\\
419	0.000912268345570118\\
420	0.000899549957971183\\
421	0.00088657119851197\\
422	0.000873327160728552\\
423	0.000859813072293951\\
424	0.000846024350965498\\
425	0.000831956672444576\\
426	0.000817606052518265\\
427	0.000802968946306856\\
428	0.000788042368116034\\
429	0.00077282403565568\\
430	0.000757312543762736\\
431	0.000741507575906905\\
432	0.00072541017123818\\
433	0.000709023104064201\\
434	0.000692351598338899\\
435	0.000675405257300434\\
436	0.00065820468577962\\
437	0.000640806376765546\\
438	0.000623398212885161\\
439	0.000606665885673703\\
440	0.000593192831486634\\
441	0.000581579396905864\\
442	0.000569713302659357\\
443	0.000557586012442185\\
444	0.000545188168945224\\
445	0.000532509458907316\\
446	0.000519538454379433\\
447	0.000506262425739107\\
448	0.000492667121011696\\
449	0.000478736504767938\\
450	0.00046445244806613\\
451	0.000449794358060503\\
452	0.000434738730400776\\
453	0.000419258594188202\\
454	0.000403322780750657\\
455	0.000386894828016613\\
456	0.000369930972609213\\
457	0.000352375618567614\\
458	0.000334148997715952\\
459	0.000315109979488416\\
460	0.000294938204456355\\
461	0.000272749625838252\\
462	0.000245909460098321\\
463	0.000218990310996033\\
464	0.000192182069022129\\
465	0.000164840854180496\\
466	0.000136951103771785\\
467	0.000108481453583217\\
468	7.9346960826296e-05\\
469	4.92801445037634e-05\\
470	1.73859184336513e-05\\
471	0\\
472	0\\
473	0\\
474	0\\
475	0\\
476	0\\
477	0\\
478	0\\
479	0\\
480	0\\
481	0\\
482	0\\
483	0\\
484	0\\
485	0\\
486	0\\
487	0\\
488	0\\
489	0\\
490	0\\
491	0\\
492	0\\
493	0\\
494	0\\
495	0\\
496	0\\
497	0\\
498	0\\
499	0\\
500	0\\
501	0\\
502	0\\
503	0\\
504	0\\
505	0\\
506	0\\
507	0\\
508	0\\
509	0\\
510	0\\
511	0\\
512	0\\
513	0\\
514	0\\
515	0\\
516	0\\
517	0\\
518	0\\
519	0\\
520	0\\
521	0\\
522	0\\
523	0\\
524	0\\
525	0\\
526	0\\
527	0\\
528	0\\
529	0\\
530	0\\
531	0\\
532	0\\
533	0\\
534	0\\
535	0\\
536	0\\
537	0\\
538	0\\
539	0\\
540	0\\
541	0\\
542	0\\
543	0\\
544	0\\
545	0\\
546	0\\
547	0\\
548	0\\
549	0\\
550	0\\
551	0\\
552	0\\
553	0\\
554	0\\
555	0\\
556	0\\
557	0\\
558	0\\
559	0\\
560	0\\
561	0\\
562	0\\
563	0\\
564	0\\
565	0\\
566	0\\
567	0\\
568	0\\
569	0\\
570	0\\
571	0\\
572	0\\
573	0\\
574	0\\
575	0\\
576	0\\
577	0\\
578	0\\
579	0\\
580	0\\
581	0\\
582	0\\
583	0\\
584	0\\
585	0\\
586	0\\
587	0\\
588	0\\
589	0\\
590	0\\
591	0\\
592	0\\
593	0\\
594	0\\
595	0\\
596	0\\
597	0\\
598	0\\
599	0\\
600	0\\
};
\addplot [color=blue!80!mycolor9,solid,forget plot]
  table[row sep=crcr]{%
1	0.0031488158044997\\
2	0.00314881010942653\\
3	0.00314880431857838\\
4	0.00314879843034492\\
5	0.00314879244308868\\
6	0.00314878635514459\\
7	0.00314878016481948\\
8	0.00314877387039164\\
9	0.0031487674701103\\
10	0.00314876096219517\\
11	0.0031487543448359\\
12	0.00314874761619161\\
13	0.00314874077439029\\
14	0.00314873381752838\\
15	0.00314872674367014\\
16	0.00314871955084715\\
17	0.00314871223705772\\
18	0.00314870480026634\\
19	0.00314869723840309\\
20	0.00314868954936306\\
21	0.00314868173100576\\
22	0.00314867378115445\\
23	0.00314866569759563\\
24	0.00314865747807829\\
25	0.00314864912031333\\
26	0.00314864062197292\\
27	0.00314863198068978\\
28	0.00314862319405655\\
29	0.00314861425962505\\
30	0.00314860517490563\\
31	0.00314859593736642\\
32	0.00314858654443263\\
33	0.00314857699348576\\
34	0.00314856728186291\\
35	0.00314855740685594\\
36	0.00314854736571077\\
37	0.0031485371556265\\
38	0.00314852677375466\\
39	0.00314851621719838\\
40	0.0031485054830115\\
41	0.00314849456819778\\
42	0.00314848346970997\\
43	0.00314847218444898\\
44	0.00314846070926293\\
45	0.00314844904094627\\
46	0.0031484371762388\\
47	0.00314842511182474\\
48	0.00314841284433181\\
49	0.00314840037033013\\
50	0.0031483876863313\\
51	0.00314837478878737\\
52	0.00314836167408975\\
53	0.0031483483385682\\
54	0.0031483347784897\\
55	0.00314832099005738\\
56	0.00314830696940938\\
57	0.00314829271261774\\
58	0.00314827821568716\\
59	0.0031482634745539\\
60	0.00314824848508452\\
61	0.00314823324307466\\
62	0.00314821774424781\\
63	0.00314820198425396\\
64	0.0031481859586684\\
65	0.00314816966299032\\
66	0.00314815309264149\\
67	0.00314813624296486\\
68	0.00314811910922321\\
69	0.00314810168659764\\
70	0.00314808397018622\\
71	0.0031480659550024\\
72	0.00314804763597357\\
73	0.00314802900793951\\
74	0.0031480100656508\\
75	0.00314799080376724\\
76	0.00314797121685624\\
77	0.00314795129939111\\
78	0.00314793104574942\\
79	0.00314791045021127\\
80	0.00314788950695751\\
81	0.00314786821006797\\
82	0.00314784655351965\\
83	0.00314782453118485\\
84	0.00314780213682928\\
85	0.00314777936411013\\
86	0.00314775620657416\\
87	0.00314773265765558\\
88	0.00314770871067414\\
89	0.00314768435883298\\
90	0.00314765959521654\\
91	0.00314763441278841\\
92	0.00314760880438911\\
93	0.00314758276273387\\
94	0.00314755628041037\\
95	0.00314752934987637\\
96	0.00314750196345738\\
97	0.00314747411334424\\
98	0.00314744579159069\\
99	0.00314741699011081\\
100	0.00314738770067656\\
101	0.00314735791491511\\
102	0.00314732762430623\\
103	0.00314729682017962\\
104	0.00314726549371214\\
105	0.00314723363592501\\
106	0.00314720123768105\\
107	0.00314716828968168\\
108	0.00314713478246407\\
109	0.00314710070639809\\
110	0.00314706605168329\\
111	0.00314703080834579\\
112	0.00314699496623511\\
113	0.00314695851502099\\
114	0.00314692144419008\\
115	0.00314688374304268\\
116	0.0031468454006893\\
117	0.00314680640604725\\
118	0.00314676674783712\\
119	0.0031467264145793\\
120	0.00314668539459026\\
121	0.00314664367597899\\
122	0.00314660124664317\\
123	0.00314655809426546\\
124	0.0031465142063096\\
125	0.00314646957001652\\
126	0.0031464241724004\\
127	0.00314637800024457\\
128	0.00314633104009748\\
129	0.0031462832782685\\
130	0.00314623470082376\\
131	0.00314618529358183\\
132	0.0031461350421094\\
133	0.00314608393171689\\
134	0.00314603194745396\\
135	0.00314597907410506\\
136	0.00314592529618477\\
137	0.00314587059793318\\
138	0.00314581496331124\\
139	0.00314575837599591\\
140	0.00314570081937539\\
141	0.00314564227654423\\
142	0.00314558273029834\\
143	0.00314552216313004\\
144	0.0031454605572229\\
145	0.0031453978944467\\
146	0.00314533415635217\\
147	0.0031452693241657\\
148	0.00314520337878409\\
149	0.00314513630076908\\
150	0.00314506807034191\\
151	0.0031449986673778\\
152	0.00314492807140031\\
153	0.0031448562615757\\
154	0.0031447832167071\\
155	0.00314470891522876\\
156	0.00314463333520007\\
157	0.00314455645429954\\
158	0.00314447824981876\\
159	0.00314439869865611\\
160	0.00314431777731053\\
161	0.00314423546187506\\
162	0.0031441517280303\\
163	0.0031440665510378\\
164	0.00314397990573318\\
165	0.0031438917665193\\
166	0.00314380210735908\\
167	0.00314371090176827\\
168	0.00314361812280806\\
169	0.00314352374307744\\
170	0.00314342773470537\\
171	0.0031433300693428\\
172	0.0031432307181544\\
173	0.00314312965181013\\
174	0.00314302684047649\\
175	0.00314292225380755\\
176	0.00314281586093575\\
177	0.0031427076304624\\
178	0.00314259753044784\\
179	0.00314248552840148\\
180	0.00314237159127135\\
181	0.00314225568543354\\
182	0.00314213777668128\\
183	0.00314201783021375\\
184	0.00314189581062462\\
185	0.00314177168189044\\
186	0.00314164540735863\\
187	0.00314151694973542\\
188	0.00314138627107351\\
189	0.00314125333275963\\
190	0.00314111809550184\\
191	0.00314098051931684\\
192	0.00314084056351704\\
193	0.00314069818669758\\
194	0.00314055334672318\\
195	0.00314040600071485\\
196	0.00314025610503636\\
197	0.00314010361528055\\
198	0.00313994848625536\\
199	0.00313979067196957\\
200	0.00313963012561829\\
201	0.0031394667995682\\
202	0.00313930064534247\\
203	0.00313913161360541\\
204	0.00313895965414691\\
205	0.00313878471586639\\
206	0.00313860674675669\\
207	0.00313842569388749\\
208	0.00313824150338844\\
209	0.00313805412043207\\
210	0.00313786348921623\\
211	0.00313766955294633\\
212	0.00313747225381716\\
213	0.00313727153299439\\
214	0.00313706733059575\\
215	0.0031368595856718\\
216	0.00313664823618643\\
217	0.00313643321899681\\
218	0.0031362144698332\\
219	0.00313599192327818\\
220	0.0031357655127456\\
221	0.00313553517045906\\
222	0.00313530082743007\\
223	0.00313506241343567\\
224	0.00313481985699579\\
225	0.00313457308535002\\
226	0.00313432202443405\\
227	0.0031340665988556\\
228	0.00313380673186999\\
229	0.00313354234535508\\
230	0.00313327335978592\\
231	0.00313299969420882\\
232	0.00313272126621495\\
233	0.00313243799191342\\
234	0.00313214978590395\\
235	0.00313185656124885\\
236	0.00313155822944468\\
237	0.00313125470039322\\
238	0.00313094588237195\\
239	0.00313063168200402\\
240	0.00313031200422758\\
241	0.00312998675226461\\
242	0.0031296558275891\\
243	0.00312931912989473\\
244	0.00312897655706183\\
245	0.00312862800512388\\
246	0.0031282733682332\\
247	0.00312791253862621\\
248	0.00312754540658795\\
249	0.00312717186041587\\
250	0.00312679178638316\\
251	0.00312640506870121\\
252	0.00312601158948155\\
253	0.00312561122869699\\
254	0.00312520386414209\\
255	0.00312478937139297\\
256	0.00312436762376635\\
257	0.00312393849227786\\
258	0.00312350184559967\\
259	0.00312305755001726\\
260	0.0031226054693856\\
261	0.00312214546508438\\
262	0.00312167739597262\\
263	0.00312120111834239\\
264	0.00312071648587183\\
265	0.00312022334957726\\
266	0.00311972155776458\\
267	0.00311921095597983\\
268	0.00311869138695886\\
269	0.00311816269057627\\
270	0.00311762470379345\\
271	0.00311707726060576\\
272	0.00311652019198895\\
273	0.00311595332584462\\
274	0.00311537648694485\\
275	0.00311478949687602\\
276	0.00311419217398167\\
277	0.00311358433330457\\
278	0.00311296578652785\\
279	0.0031123363419153\\
280	0.00311169580425075\\
281	0.00311104397477658\\
282	0.00311038065113134\\
283	0.00310970562728657\\
284	0.00310901869348254\\
285	0.00310831963616339\\
286	0.00310760823791114\\
287	0.00310688427737895\\
288	0.00310614752922352\\
289	0.00310539776403659\\
290	0.00310463474827559\\
291	0.00310385824419343\\
292	0.00310306800976752\\
293	0.00310226379862781\\
294	0.00310144535998416\\
295	0.00310061243855285\\
296	0.0030997647744822\\
297	0.00309890210327755\\
298	0.00309802415572534\\
299	0.00309713065781651\\
300	0.00309622133066907\\
301	0.00309529589045001\\
302	0.00309435404829648\\
303	0.0030933955102362\\
304	0.00309241997710728\\
305	0.00309142714447724\\
306	0.00309041670256155\\
307	0.00308938833614131\\
308	0.00308834172448047\\
309	0.00308727654124234\\
310	0.00308619245440554\\
311	0.00308508912617928\\
312	0.00308396621291814\\
313	0.00308282336503617\\
314	0.00308166022692044\\
315	0.00308047643684404\\
316	0.00307927162687835\\
317	0.00307804542280484\\
318	0.00307679744402615\\
319	0.00307552730347649\\
320	0.00307423460753143\\
321	0.00307291895591687\\
322	0.00307157994161721\\
323	0.00307021715078276\\
324	0.00306883016263607\\
325	0.00306741854937735\\
326	0.00306598187608868\\
327	0.00306451970063701\\
328	0.0030630315735757\\
329	0.00306151703804454\\
330	0.00305997562966797\\
331	0.00305840687645137\\
332	0.0030568102986751\\
333	0.00305518540878608\\
334	0.00305353171128658\\
335	0.0030518487026199\\
336	0.00305013587105251\\
337	0.00304839269655235\\
338	0.00304661865066269\\
339	0.00304481319637114\\
340	0.00304297578797323\\
341	0.00304110587092995\\
342	0.00303920288171855\\
343	0.00303726624767596\\
344	0.00303529538683399\\
345	0.00303328970774543\\
346	0.00303124860930018\\
347	0.00302917148053046\\
348	0.00302705770040394\\
349	0.00302490663760374\\
350	0.00302271765029403\\
351	0.00302049008587002\\
352	0.00301822328069084\\
353	0.00301591655979405\\
354	0.00301356923659012\\
355	0.00301118061253527\\
356	0.00300874997678115\\
357	0.0030062766057994\\
358	0.00300375976297957\\
359	0.00300119869819815\\
360	0.00299859264735711\\
361	0.00299594083188986\\
362	0.00299324245823253\\
363	0.00299049671725858\\
364	0.00298770278367464\\
365	0.00298485981537552\\
366	0.00298196695275613\\
367	0.00297902331797829\\
368	0.00297602801419028\\
369	0.00297298012469691\\
370	0.00296987871207802\\
371	0.00296672281725328\\
372	0.00296351145849092\\
373	0.00296024363035836\\
374	0.00295691830261222\\
375	0.00295353441902534\\
376	0.00295009089614794\\
377	0.00294658662200026\\
378	0.00294302045469302\\
379	0.00293939122097195\\
380	0.00293569771468192\\
381	0.00293193869514513\\
382	0.00292811288544692\\
383	0.0029242189706215\\
384	0.00292025559572781\\
385	0.00291622136380439\\
386	0.00291211483368896\\
387	0.00290793451768638\\
388	0.00290367887906473\\
389	0.00289934632935552\\
390	0.00289493522542896\\
391	0.00289044386630764\\
392	0.00288587048967122\\
393	0.00288121326799\\
394	0.00287647030419765\\
395	0.00287163962676288\\
396	0.00286671918390948\\
397	0.00286170683650244\\
398	0.00285660034861841\\
399	0.00285139737384472\\
400	0.00284609543383834\\
401	0.00284069188478618\\
402	0.00283518387313868\\
403	0.0028295683130977\\
404	0.00282384200316826\\
405	0.00281800200718434\\
406	0.0028120454234071\\
407	0.00280596865857969\\
408	0.0027997679172769\\
409	0.00279343919198299\\
410	0.00278697825900345\\
411	0.00278038068760407\\
412	0.00277364187577775\\
413	0.00276675713162891\\
414	0.00275972180882898\\
415	0.00275253142415608\\
416	0.00274518141276298\\
417	0.0027376656857939\\
418	0.00272997422962037\\
419	0.00272209965955078\\
420	0.00271403454251588\\
421	0.00270577084354978\\
422	0.00269729985161397\\
423	0.00268861209420816\\
424	0.00267969723882085\\
425	0.00267054397888468\\
426	0.00266113990141535\\
427	0.00265147133286866\\
428	0.00264152315879101\\
429	0.00263127861120611\\
430	0.00262071901415549\\
431	0.00260982346880575\\
432	0.00259856843392222\\
433	0.00258692707942594\\
434	0.00257486804188674\\
435	0.00256235239401671\\
436	0.00254932489016489\\
437	0.00253568608287143\\
438	0.00252119866904994\\
439	0.00250516269361007\\
440	0.00248526366011957\\
441	0.00246288514601484\\
442	0.00243997915080848\\
443	0.00241652916105559\\
444	0.0023925180371453\\
445	0.00236792800599649\\
446	0.00234274065856143\\
447	0.00231693695337167\\
448	0.00229049722759709\\
449	0.00226340121732881\\
450	0.00223562808895717\\
451	0.00220715648332688\\
452	0.00217796457304159\\
453	0.00214803012875291\\
454	0.00211733057709868\\
455	0.0020858429982616\\
456	0.00205354392497408\\
457	0.00202040860346689\\
458	0.00198640900300285\\
459	0.00195150951581333\\
460	0.0019156610278376\\
461	0.00187880925939435\\
462	0.00184101668332657\\
463	0.00180306060647431\\
464	0.00176695657113213\\
465	0.00174002127743041\\
466	0.00171255806413367\\
467	0.00168454871256319\\
468	0.00165596653527007\\
469	0.0016267693496688\\
470	0.00159690408596396\\
471	0.00156638900126274\\
472	0.001535271194343\\
473	0.00150354977120314\\
474	0.00147124695810853\\
475	0.0014384333574062\\
476	0.00140525426127058\\
477	0.00137184892090271\\
478	0.00133779609282825\\
479	0.00130297800531827\\
480	0.0012673700781407\\
481	0.00123094637685959\\
482	0.00119367951199304\\
483	0.00115554057384795\\
484	0.00111649904362321\\
485	0.00107652267505202\\
486	0.00103557736814537\\
487	0.00099362708868859\\
488	0.00095063371984951\\
489	0.000906556731923931\\
490	0.000861352959695491\\
491	0.00081497635688427\\
492	0.000767377725191132\\
493	0.000718504413599808\\
494	0.0006682999778587\\
495	0.000616703772697162\\
496	0.000563650397617503\\
497	0.000509068764401486\\
498	0.000452880105573953\\
499	0.000394992927758715\\
500	0.000335289074672282\\
501	0.00027358392453668\\
502	0.000209511731993176\\
503	0.000142196249937518\\
504	6.92959835515677e-05\\
505	0\\
506	0\\
507	0\\
508	0\\
509	0\\
510	0\\
511	0\\
512	0\\
513	0\\
514	0\\
515	0\\
516	0\\
517	0\\
518	0\\
519	0\\
520	0\\
521	0\\
522	0\\
523	0\\
524	0\\
525	0\\
526	0\\
527	0\\
528	0\\
529	0\\
530	0\\
531	0\\
532	0\\
533	0\\
534	0\\
535	0\\
536	0\\
537	0\\
538	0\\
539	0\\
540	0\\
541	0\\
542	0\\
543	0\\
544	0\\
545	0\\
546	0\\
547	0\\
548	0\\
549	0\\
550	0\\
551	0\\
552	0\\
553	0\\
554	0\\
555	0\\
556	0\\
557	0\\
558	0\\
559	0\\
560	0\\
561	0\\
562	0\\
563	0\\
564	0\\
565	0\\
566	0\\
567	0\\
568	0\\
569	0\\
570	0\\
571	0\\
572	0\\
573	0\\
574	0\\
575	0\\
576	0\\
577	0\\
578	0\\
579	0\\
580	0\\
581	0\\
582	0\\
583	0\\
584	0\\
585	0\\
586	0\\
587	0\\
588	0\\
589	0\\
590	0\\
591	0\\
592	0\\
593	0\\
594	0\\
595	0\\
596	0\\
597	0\\
598	0\\
599	0\\
600	0\\
};
\addplot [color=blue,solid,forget plot]
  table[row sep=crcr]{%
1	0.00391145423208649\\
2	0.00391145359632034\\
3	0.00391145294986301\\
4	0.00391145229253474\\
5	0.00391145162415274\\
6	0.00391145094453115\\
7	0.00391145025348098\\
8	0.00391144955081003\\
9	0.00391144883632287\\
10	0.00391144810982078\\
11	0.00391144737110169\\
12	0.00391144661996008\\
13	0.00391144585618702\\
14	0.00391144507957\\
15	0.00391144428989295\\
16	0.00391144348693615\\
17	0.00391144267047614\\
18	0.00391144184028571\\
19	0.0039114409961338\\
20	0.00391144013778546\\
21	0.00391143926500173\\
22	0.00391143837753964\\
23	0.00391143747515211\\
24	0.00391143655758784\\
25	0.00391143562459133\\
26	0.00391143467590271\\
27	0.00391143371125772\\
28	0.00391143273038765\\
29	0.00391143173301919\\
30	0.00391143071887444\\
31	0.00391142968767075\\
32	0.00391142863912071\\
33	0.00391142757293201\\
34	0.00391142648880739\\
35	0.00391142538644453\\
36	0.003911424265536\\
37	0.00391142312576913\\
38	0.00391142196682593\\
39	0.00391142078838303\\
40	0.00391141959011152\\
41	0.00391141837167694\\
42	0.0039114171327391\\
43	0.00391141587295205\\
44	0.00391141459196392\\
45	0.00391141328941686\\
46	0.00391141196494692\\
47	0.00391141061818395\\
48	0.00391140924875146\\
49	0.00391140785626657\\
50	0.00391140644033982\\
51	0.00391140500057516\\
52	0.00391140353656969\\
53	0.00391140204791371\\
54	0.00391140053419044\\
55	0.003911398994976\\
56	0.00391139742983927\\
57	0.00391139583834171\\
58	0.00391139422003729\\
59	0.00391139257447231\\
60	0.00391139090118532\\
61	0.00391138919970692\\
62	0.00391138746955967\\
63	0.00391138571025791\\
64	0.00391138392130764\\
65	0.00391138210220638\\
66	0.00391138025244296\\
67	0.00391137837149746\\
68	0.00391137645884097\\
69	0.00391137451393546\\
70	0.00391137253623364\\
71	0.00391137052517878\\
72	0.00391136848020449\\
73	0.00391136640073466\\
74	0.00391136428618315\\
75	0.00391136213595374\\
76	0.00391135994943986\\
77	0.00391135772602443\\
78	0.00391135546507968\\
79	0.00391135316596697\\
80	0.00391135082803654\\
81	0.00391134845062738\\
82	0.00391134603306697\\
83	0.00391134357467112\\
84	0.00391134107474372\\
85	0.00391133853257653\\
86	0.003911335947449\\
87	0.00391133331862798\\
88	0.00391133064536757\\
89	0.00391132792690882\\
90	0.00391132516247953\\
91	0.003911322351294\\
92	0.00391131949255279\\
93	0.00391131658544247\\
94	0.00391131362913536\\
95	0.00391131062278926\\
96	0.00391130756554721\\
97	0.00391130445653723\\
98	0.00391130129487198\\
99	0.00391129807964859\\
100	0.00391129480994825\\
101	0.00391129148483605\\
102	0.00391128810336056\\
103	0.00391128466455363\\
104	0.00391128116743004\\
105	0.00391127761098719\\
106	0.00391127399420479\\
107	0.00391127031604454\\
108	0.0039112665754498\\
109	0.00391126277134525\\
110	0.00391125890263657\\
111	0.00391125496821008\\
112	0.00391125096693238\\
113	0.00391124689765002\\
114	0.00391124275918911\\
115	0.00391123855035498\\
116	0.00391123426993176\\
117	0.00391122991668204\\
118	0.00391122548934648\\
119	0.00391122098664338\\
120	0.00391121640726832\\
121	0.00391121174989372\\
122	0.00391120701316845\\
123	0.00391120219571741\\
124	0.00391119729614108\\
125	0.00391119231301512\\
126	0.0039111872448899\\
127	0.00391118209029005\\
128	0.00391117684771408\\
129	0.00391117151563381\\
130	0.003911166092494\\
131	0.0039111605767118\\
132	0.00391115496667634\\
133	0.00391114926074819\\
134	0.00391114345725893\\
135	0.00391113755451056\\
136	0.00391113155077509\\
137	0.00391112544429398\\
138	0.0039111192332776\\
139	0.00391111291590479\\
140	0.00391110649032221\\
141	0.0039110999546439\\
142	0.00391109330695071\\
143	0.0039110865452897\\
144	0.00391107966767366\\
145	0.00391107267208049\\
146	0.00391106555645263\\
147	0.00391105831869653\\
148	0.00391105095668199\\
149	0.00391104346824163\\
150	0.00391103585117025\\
151	0.00391102810322426\\
152	0.00391102022212101\\
153	0.00391101220553822\\
154	0.00391100405111328\\
155	0.00391099575644269\\
156	0.00391098731908134\\
157	0.00391097873654187\\
158	0.003910970006294\\
159	0.00391096112576384\\
160	0.00391095209233318\\
161	0.00391094290333881\\
162	0.00391093355607176\\
163	0.00391092404777658\\
164	0.0039109143756506\\
165	0.00391090453684312\\
166	0.00391089452845464\\
167	0.00391088434753605\\
168	0.00391087399108782\\
169	0.0039108634560591\\
170	0.00391085273934689\\
171	0.00391084183779514\\
172	0.00391083074819378\\
173	0.00391081946727783\\
174	0.0039108079917264\\
175	0.00391079631816164\\
176	0.0039107844431478\\
177	0.00391077236319006\\
178	0.00391076007473353\\
179	0.00391074757416205\\
180	0.00391073485779707\\
181	0.00391072192189647\\
182	0.00391070876265335\\
183	0.00391069537619475\\
184	0.00391068175858045\\
185	0.00391066790580162\\
186	0.00391065381377956\\
187	0.00391063947836435\\
188	0.00391062489533349\\
189	0.00391061006039058\\
190	0.00391059496916392\\
191	0.0039105796172051\\
192	0.00391056399998768\\
193	0.00391054811290567\\
194	0.00391053195127217\\
195	0.0039105155103179\\
196	0.00391049878518971\\
197	0.00391048177094907\\
198	0.00391046446257056\\
199	0.00391044685494027\\
200	0.00391042894285424\\
201	0.00391041072101682\\
202	0.00391039218403905\\
203	0.00391037332643693\\
204	0.00391035414262974\\
205	0.00391033462693829\\
206	0.00391031477358314\\
207	0.0039102945766828\\
208	0.00391027403025186\\
209	0.00391025312819916\\
210	0.00391023186432583\\
211	0.00391021023232338\\
212	0.00391018822577169\\
213	0.003910165838137\\
214	0.00391014306276988\\
215	0.00391011989290308\\
216	0.00391009632164944\\
217	0.00391007234199972\\
218	0.00391004794682034\\
219	0.00391002312885119\\
220	0.00390999788070329\\
221	0.00390997219485646\\
222	0.00390994606365694\\
223	0.003909919479315\\
224	0.0039098924339024\\
225	0.00390986491934995\\
226	0.00390983692744492\\
227	0.00390980844982839\\
228	0.00390977947799267\\
229	0.00390975000327855\\
230	0.00390972001687256\\
231	0.00390968950980415\\
232	0.00390965847294288\\
233	0.00390962689699544\\
234	0.00390959477250276\\
235	0.00390956208983697\\
236	0.00390952883919834\\
237	0.00390949501061212\\
238	0.00390946059392543\\
239	0.00390942557880395\\
240	0.00390938995472869\\
241	0.00390935371099259\\
242	0.00390931683669712\\
243	0.00390927932074883\\
244	0.0039092411518558\\
245	0.00390920231852401\\
246	0.00390916280905373\\
247	0.00390912261153578\\
248	0.0039090817138477\\
249	0.00390904010364997\\
250	0.00390899776838202\\
251	0.00390895469525827\\
252	0.00390891087126407\\
253	0.00390886628315158\\
254	0.00390882091743554\\
255	0.00390877476038906\\
256	0.00390872779803919\\
257	0.00390868001616263\\
258	0.00390863140028114\\
259	0.00390858193565703\\
260	0.00390853160728851\\
261	0.00390848039990502\\
262	0.0039084282979624\\
263	0.00390837528563805\\
264	0.003908321346826\\
265	0.0039082664651319\\
266	0.00390821062386793\\
267	0.00390815380604762\\
268	0.0039080959943806\\
269	0.00390803717126733\\
270	0.00390797731879361\\
271	0.00390791641872519\\
272	0.00390785445250214\\
273	0.00390779140123325\\
274	0.0039077272456903\\
275	0.00390766196630227\\
276	0.00390759554314946\\
277	0.00390752795595753\\
278	0.0039074591840915\\
279	0.00390738920654961\\
280	0.00390731800195713\\
281	0.00390724554856015\\
282	0.00390717182421921\\
283	0.00390709680640285\\
284	0.00390702047218123\\
285	0.00390694279821945\\
286	0.00390686376077104\\
287	0.00390678333567118\\
288	0.00390670149832998\\
289	0.00390661822372562\\
290	0.00390653348639748\\
291	0.00390644726043916\\
292	0.00390635951949148\\
293	0.00390627023673536\\
294	0.00390617938488474\\
295	0.00390608693617933\\
296	0.00390599286237742\\
297	0.00390589713474855\\
298	0.00390579972406615\\
299	0.00390570060060022\\
300	0.00390559973410983\\
301	0.00390549709383573\\
302	0.00390539264849274\\
303	0.00390528636626233\\
304	0.00390517821478497\\
305	0.00390506816115258\\
306	0.00390495617190089\\
307	0.00390484221300179\\
308	0.00390472624985571\\
309	0.00390460824728389\\
310	0.00390448816952074\\
311	0.00390436598020609\\
312	0.00390424164237752\\
313	0.0039041151184626\\
314	0.00390398637027116\\
315	0.00390385535898758\\
316	0.00390372204516303\\
317	0.0039035863887077\\
318	0.00390344834888305\\
319	0.00390330788429402\\
320	0.00390316495288124\\
321	0.00390301951191325\\
322	0.0039028715179786\\
323	0.00390272092697804\\
324	0.00390256769411655\\
325	0.00390241177389542\\
326	0.0039022531201042\\
327	0.00390209168581261\\
328	0.00390192742336232\\
329	0.00390176028435864\\
330	0.00390159021966208\\
331	0.00390141717937973\\
332	0.00390124111285647\\
333	0.00390106196866593\\
334	0.00390087969460124\\
335	0.00390069423766547\\
336	0.0039005055440617\\
337	0.00390031355918274\\
338	0.00390011822760038\\
339	0.00389991949305422\\
340	0.00389971729843982\\
341	0.0038995115857963\\
342	0.00389930229629321\\
343	0.00389908937021664\\
344	0.00389887274695436\\
345	0.00389865236498013\\
346	0.00389842816183678\\
347	0.0038982000741182\\
348	0.00389796803745\\
349	0.00389773198646871\\
350	0.00389749185479949\\
351	0.003897247575032\\
352	0.00389699907869457\\
353	0.00389674629622624\\
354	0.00389648915694669\\
355	0.00389622758902382\\
356	0.00389596151943881\\
357	0.00389569087394843\\
358	0.0038954155770446\\
359	0.00389513555191069\\
360	0.00389485072037473\\
361	0.00389456100285906\\
362	0.00389426631832628\\
363	0.00389396658422146\\
364	0.00389366171641021\\
365	0.00389335162911248\\
366	0.00389303623483195\\
367	0.00389271544428074\\
368	0.00389238916629931\\
369	0.00389205730777134\\
370	0.0038917197735333\\
371	0.00389137646627871\\
372	0.00389102728645676\\
373	0.00389067213216507\\
374	0.0038903108990365\\
375	0.00388994348011961\\
376	0.0038895697657526\\
377	0.00388918964343038\\
378	0.00388880299766439\\
379	0.00388840970983479\\
380	0.00388800965803425\\
381	0.003887602716903\\
382	0.00388718875745404\\
383	0.00388676764688758\\
384	0.00388633924839353\\
385	0.00388590342094045\\
386	0.0038854600190492\\
387	0.00388500889254906\\
388	0.00388454988631361\\
389	0.00388408283997319\\
390	0.00388360758759974\\
391	0.00388312395735876\\
392	0.0038826317711215\\
393	0.00388213084402801\\
394	0.00388162098398701\\
395	0.00388110199109091\\
396	0.00388057365691044\\
397	0.00388003576361041\\
398	0.0038794880827994\\
399	0.00387893037401931\\
400	0.00387836238289632\\
401	0.00387778383944593\\
402	0.00387719445811215\\
403	0.00387659394204268\\
404	0.00387598198951732\\
405	0.00387535828191595\\
406	0.00387472246678478\\
407	0.00387407417172406\\
408	0.0038734130033376\\
409	0.00387273854653858\\
410	0.00387205036454529\\
411	0.00387134799997831\\
412	0.00387063097718826\\
413	0.00386989880449444\\
414	0.00386915097106735\\
415	0.00386838692695242\\
416	0.00386760603773274\\
417	0.00386680756080642\\
418	0.00386599080346426\\
419	0.00386515503338101\\
420	0.00386429945743454\\
421	0.00386342321410664\\
422	0.00386252536473786\\
423	0.00386160488343816\\
424	0.00386066064541408\\
425	0.00385969141341615\\
426	0.00385869582192358\\
427	0.00385767235852507\\
428	0.00385661934161984\\
429	0.00385553489274932\\
430	0.00385441689976431\\
431	0.00385326296140546\\
432	0.00385207028869819\\
433	0.00385083549785069\\
434	0.00384955412228839\\
435	0.00384821940056294\\
436	0.00384681926064951\\
437	0.0038453291565604\\
438	0.00384369704008515\\
439	0.00384182192617243\\
440	0.00383957929471537\\
441	0.00383713441622716\\
442	0.00383464069763525\\
443	0.00383209695164383\\
444	0.0038295019659953\\
445	0.00382685450630891\\
446	0.00382415331976878\\
447	0.00382139713981654\\
448	0.00381858469201794\\
449	0.00381571470126403\\
450	0.00381278590039141\\
451	0.00380979704004449\\
452	0.00380674689886397\\
453	0.00380363429115159\\
454	0.00380045806428968\\
455	0.00379721706609648\\
456	0.00379391003229393\\
457	0.00379053526911188\\
458	0.00378708980565845\\
459	0.00378356710029378\\
460	0.00377995040566156\\
461	0.00377619143105491\\
462	0.0037721334462718\\
463	0.0037672220531096\\
464	0.00375953648037518\\
465	0.00374259590734272\\
466	0.00372533544350868\\
467	0.00370774507665263\\
468	0.00368981399443503\\
469	0.00367153110135167\\
470	0.00365288634541003\\
471	0.00363386936568687\\
472	0.00361446838002628\\
473	0.00359467186430756\\
474	0.00357446889131582\\
475	0.00355384847238614\\
476	0.00353279567495486\\
477	0.00351128549018164\\
478	0.00348929938958904\\
479	0.00346681983209936\\
480	0.00344382812335562\\
481	0.00342030431059674\\
482	0.00339622706474022\\
483	0.00337157354405513\\
484	0.00334631923702561\\
485	0.00332043778241085\\
486	0.00329390076330926\\
487	0.0032666774663388\\
488	0.00323873460222879\\
489	0.0032100359918576\\
490	0.00318054220155812\\
491	0.00315021011715184\\
492	0.00311899244306825\\
493	0.00308683710781825\\
494	0.00305368654728657\\
495	0.00301947681520266\\
496	0.0029841364142171\\
497	0.0029475845893136\\
498	0.00290972839773051\\
499	0.00287045663723923\\
500	0.00282962510700878\\
501	0.00278701696808144\\
502	0.00274222972852771\\
503	0.00269434170656592\\
504	0.00264090605458067\\
505	0.00257465229279033\\
506	0.00249789749396208\\
507	0.00241955633568501\\
508	0.00233969171468241\\
509	0.00225848184594896\\
510	0.00217611043500788\\
511	0.00209209122396565\\
512	0.00200591458078461\\
513	0.00191746610396971\\
514	0.00182662038391383\\
515	0.00173323967234684\\
516	0.00163717243645703\\
517	0.00153825200793115\\
518	0.00143629628630456\\
519	0.00133111244935181\\
520	0.00122252232193912\\
521	0.00111046918952441\\
522	0.000995439653286779\\
523	0.000880092985072983\\
524	0.000775499111761147\\
525	0.000689867989463739\\
526	0.000601110061332961\\
527	0.000508813764893824\\
528	0.000412124598928745\\
529	0.000308800153975222\\
530	0.000192070774563915\\
531	5.80908334167806e-05\\
532	0\\
533	0\\
534	0\\
535	0\\
536	0\\
537	0\\
538	0\\
539	0\\
540	0\\
541	0\\
542	0\\
543	0\\
544	0\\
545	0\\
546	0\\
547	0\\
548	0\\
549	0\\
550	0\\
551	0\\
552	0\\
553	0\\
554	0\\
555	0\\
556	0\\
557	0\\
558	0\\
559	0\\
560	0\\
561	0\\
562	0\\
563	0\\
564	0\\
565	0\\
566	0\\
567	0\\
568	0\\
569	0\\
570	0\\
571	0\\
572	0\\
573	0\\
574	0\\
575	0\\
576	0\\
577	0\\
578	0\\
579	0\\
580	0\\
581	0\\
582	0\\
583	0\\
584	0\\
585	0\\
586	0\\
587	0\\
588	0\\
589	0\\
590	0\\
591	0\\
592	0\\
593	0\\
594	0\\
595	0\\
596	0\\
597	0\\
598	0\\
599	0\\
600	0\\
};
\addplot [color=mycolor10,solid,forget plot]
  table[row sep=crcr]{%
1	0.00400023926843176\\
2	0.00400023923938983\\
3	0.00400023920985951\\
4	0.0040002391798326\\
5	0.00400023914930074\\
6	0.00400023911825545\\
7	0.00400023908668808\\
8	0.00400023905458986\\
9	0.00400023902195187\\
10	0.00400023898876501\\
11	0.00400023895502006\\
12	0.00400023892070763\\
13	0.00400023888581818\\
14	0.00400023885034198\\
15	0.00400023881426918\\
16	0.00400023877758973\\
17	0.00400023874029343\\
18	0.00400023870236989\\
19	0.00400023866380856\\
20	0.00400023862459869\\
21	0.00400023858472939\\
22	0.00400023854418953\\
23	0.00400023850296783\\
24	0.00400023846105281\\
25	0.0040002384184328\\
26	0.00400023837509592\\
27	0.00400023833103009\\
28	0.00400023828622305\\
29	0.00400023824066229\\
30	0.00400023819433513\\
31	0.00400023814722864\\
32	0.0040002380993297\\
33	0.00400023805062494\\
34	0.00400023800110079\\
35	0.00400023795074342\\
36	0.0040002378995388\\
37	0.00400023784747262\\
38	0.00400023779453037\\
39	0.00400023774069726\\
40	0.00400023768595826\\
41	0.00400023763029808\\
42	0.0040002375737012\\
43	0.00400023751615178\\
44	0.00400023745763376\\
45	0.00400023739813077\\
46	0.00400023733762619\\
47	0.00400023727610309\\
48	0.00400023721354428\\
49	0.00400023714993225\\
50	0.0040002370852492\\
51	0.00400023701947703\\
52	0.00400023695259733\\
53	0.00400023688459137\\
54	0.0040002368154401\\
55	0.00400023674512415\\
56	0.00400023667362381\\
57	0.00400023660091903\\
58	0.00400023652698944\\
59	0.00400023645181429\\
60	0.00400023637537249\\
61	0.00400023629764258\\
62	0.00400023621860273\\
63	0.00400023613823075\\
64	0.00400023605650405\\
65	0.00400023597339967\\
66	0.00400023588889424\\
67	0.00400023580296397\\
68	0.00400023571558471\\
69	0.00400023562673184\\
70	0.00400023553638036\\
71	0.0040002354445048\\
72	0.00400023535107926\\
73	0.00400023525607742\\
74	0.00400023515947247\\
75	0.00400023506123715\\
76	0.00400023496134373\\
77	0.00400023485976398\\
78	0.00400023475646921\\
79	0.00400023465143022\\
80	0.00400023454461729\\
81	0.00400023443600019\\
82	0.00400023432554817\\
83	0.00400023421322995\\
84	0.00400023409901369\\
85	0.004000233982867\\
86	0.00400023386475693\\
87	0.00400023374464995\\
88	0.00400023362251195\\
89	0.00400023349830821\\
90	0.00400023337200343\\
91	0.00400023324356165\\
92	0.00400023311294632\\
93	0.00400023298012024\\
94	0.00400023284504553\\
95	0.00400023270768369\\
96	0.00400023256799551\\
97	0.0040002324259411\\
98	0.00400023228147987\\
99	0.00400023213457053\\
100	0.00400023198517103\\
101	0.00400023183323862\\
102	0.00400023167872976\\
103	0.00400023152160017\\
104	0.00400023136180476\\
105	0.00400023119929767\\
106	0.00400023103403222\\
107	0.00400023086596091\\
108	0.00400023069503539\\
109	0.00400023052120645\\
110	0.00400023034442404\\
111	0.00400023016463718\\
112	0.00400022998179403\\
113	0.0040002297958418\\
114	0.00400022960672678\\
115	0.0040002294143943\\
116	0.00400022921878872\\
117	0.00400022901985342\\
118	0.00400022881753078\\
119	0.00400022861176213\\
120	0.00400022840248779\\
121	0.004000228189647\\
122	0.00400022797317793\\
123	0.00400022775301766\\
124	0.00400022752910214\\
125	0.00400022730136618\\
126	0.00400022706974344\\
127	0.00400022683416641\\
128	0.00400022659456636\\
129	0.00400022635087337\\
130	0.00400022610301626\\
131	0.00400022585092258\\
132	0.00400022559451863\\
133	0.00400022533372936\\
134	0.00400022506847843\\
135	0.00400022479868814\\
136	0.00400022452427939\\
137	0.00400022424517172\\
138	0.00400022396128322\\
139	0.00400022367253056\\
140	0.00400022337882892\\
141	0.004000223080092\\
142	0.00400022277623198\\
143	0.00400022246715948\\
144	0.00400022215278359\\
145	0.00400022183301175\\
146	0.00400022150774984\\
147	0.00400022117690204\\
148	0.0040002208403709\\
149	0.00400022049805724\\
150	0.00400022014986016\\
151	0.00400021979567702\\
152	0.00400021943540338\\
153	0.00400021906893297\\
154	0.00400021869615773\\
155	0.00400021831696768\\
156	0.00400021793125095\\
157	0.00400021753889377\\
158	0.00400021713978037\\
159	0.004000216733793\\
160	0.00400021632081189\\
161	0.0040002159007152\\
162	0.00400021547337901\\
163	0.00400021503867728\\
164	0.00400021459648178\\
165	0.00400021414666211\\
166	0.00400021368908563\\
167	0.00400021322361742\\
168	0.00400021275012026\\
169	0.00400021226845458\\
170	0.00400021177847842\\
171	0.00400021128004738\\
172	0.00400021077301461\\
173	0.0040002102572307\\
174	0.0040002097325437\\
175	0.00400020919879907\\
176	0.00400020865583955\\
177	0.00400020810350523\\
178	0.00400020754163341\\
179	0.00400020697005859\\
180	0.00400020638861238\\
181	0.00400020579712349\\
182	0.00400020519541765\\
183	0.00400020458331754\\
184	0.00400020396064278\\
185	0.0040002033272098\\
186	0.00400020268283185\\
187	0.00400020202731888\\
188	0.00400020136047753\\
189	0.00400020068211103\\
190	0.00400019999201917\\
191	0.00400019928999819\\
192	0.00400019857584077\\
193	0.00400019784933593\\
194	0.00400019711026898\\
195	0.00400019635842145\\
196	0.004000195593571\\
197	0.00400019481549139\\
198	0.0040001940239524\\
199	0.0040001932187197\\
200	0.00400019239955489\\
201	0.0040001915662153\\
202	0.004000190718454\\
203	0.0040001898560197\\
204	0.00400018897865665\\
205	0.0040001880861046\\
206	0.00400018717809867\\
207	0.00400018625436929\\
208	0.00400018531464214\\
209	0.00400018435863801\\
210	0.00400018338607277\\
211	0.00400018239665722\\
212	0.00400018139009704\\
213	0.00400018036609271\\
214	0.00400017932433935\\
215	0.0040001782645267\\
216	0.00400017718633895\\
217	0.00400017608945471\\
218	0.00400017497354685\\
219	0.00400017383828242\\
220	0.00400017268332256\\
221	0.00400017150832235\\
222	0.00400017031293074\\
223	0.00400016909679041\\
224	0.00400016785953769\\
225	0.0040001666008024\\
226	0.00400016532020775\\
227	0.00400016401737025\\
228	0.00400016269189953\\
229	0.00400016134339824\\
230	0.00400015997146196\\
231	0.00400015857567902\\
232	0.00400015715563037\\
233	0.00400015571088948\\
234	0.00400015424102217\\
235	0.00400015274558649\\
236	0.00400015122413256\\
237	0.00400014967620246\\
238	0.00400014810133003\\
239	0.00400014649904076\\
240	0.00400014486885164\\
241	0.00400014321027096\\
242	0.00400014152279822\\
243	0.00400013980592388\\
244	0.0040001380591293\\
245	0.00400013628188648\\
246	0.00400013447365795\\
247	0.00400013263389657\\
248	0.00400013076204536\\
249	0.00400012885753732\\
250	0.00400012691979526\\
251	0.00400012494823157\\
252	0.00400012294224811\\
253	0.00400012090123594\\
254	0.00400011882457518\\
255	0.00400011671163477\\
256	0.0040001145617723\\
257	0.00400011237433378\\
258	0.00400011014865345\\
259	0.00400010788405356\\
260	0.00400010557984417\\
261	0.0040001032353229\\
262	0.0040001008497747\\
263	0.00400009842247171\\
264	0.00400009595267291\\
265	0.00400009343962396\\
266	0.00400009088255696\\
267	0.00400008828069019\\
268	0.00400008563322785\\
269	0.00400008293935985\\
270	0.00400008019826155\\
271	0.00400007740909347\\
272	0.00400007457100108\\
273	0.00400007168311449\\
274	0.00400006874454821\\
275	0.00400006575440089\\
276	0.004000062711755\\
277	0.00400005961567661\\
278	0.00400005646521507\\
279	0.00400005325940272\\
280	0.00400004999725463\\
281	0.00400004667776828\\
282	0.00400004329992329\\
283	0.00400003986268108\\
284	0.00400003636498461\\
285	0.00400003280575806\\
286	0.00400002918390648\\
287	0.00400002549831555\\
288	0.00400002174785119\\
289	0.00400001793135928\\
290	0.00400001404766534\\
291	0.00400001009557416\\
292	0.00400000607386954\\
293	0.00400000198131388\\
294	0.0039999978166479\\
295	0.00399999357859029\\
296	0.00399998926583733\\
297	0.00399998487706262\\
298	0.00399998041091665\\
299	0.00399997586602652\\
300	0.00399997124099554\\
301	0.00399996653440293\\
302	0.00399996174480339\\
303	0.00399995687072683\\
304	0.00399995191067793\\
305	0.00399994686313584\\
306	0.00399994172655379\\
307	0.00399993649935871\\
308	0.00399993117995091\\
309	0.00399992576670368\\
310	0.00399992025796293\\
311	0.00399991465204682\\
312	0.00399990894724539\\
313	0.00399990314182019\\
314	0.00399989723400391\\
315	0.00399989122199999\\
316	0.00399988510398228\\
317	0.00399987887809461\\
318	0.00399987254245047\\
319	0.00399986609513257\\
320	0.00399985953419253\\
321	0.00399985285765041\\
322	0.00399984606349439\\
323	0.00399983914968035\\
324	0.00399983211413148\\
325	0.00399982495473788\\
326	0.00399981766935617\\
327	0.00399981025580903\\
328	0.00399980271188486\\
329	0.00399979503533729\\
330	0.00399978722388479\\
331	0.00399977927521018\\
332	0.00399977118696023\\
333	0.00399976295674514\\
334	0.00399975458213807\\
335	0.00399974606067465\\
336	0.00399973738985242\\
337	0.00399972856713033\\
338	0.00399971958992814\\
339	0.0039997104556258\\
340	0.00399970116156288\\
341	0.00399969170503784\\
342	0.00399968208330732\\
343	0.00399967229358544\\
344	0.00399966233304298\\
345	0.00399965219880651\\
346	0.00399964188795746\\
347	0.00399963139753122\\
348	0.00399962072451601\\
349	0.00399960986585184\\
350	0.00399959881842922\\
351	0.00399958757908794\\
352	0.00399957614461565\\
353	0.00399956451174636\\
354	0.00399955267715886\\
355	0.00399954063747494\\
356	0.00399952838925759\\
357	0.003999515929009\\
358	0.00399950325316837\\
359	0.00399949035810966\\
360	0.00399947724013911\\
361	0.00399946389549262\\
362	0.0039994503203329\\
363	0.0039994365107465\\
364	0.00399942246274054\\
365	0.00399940817223933\\
366	0.00399939363508073\\
367	0.00399937884701221\\
368	0.00399936380368681\\
369	0.00399934850065873\\
370	0.00399933293337871\\
371	0.00399931709718917\\
372	0.00399930098731901\\
373	0.00399928459887818\\
374	0.00399926792685191\\
375	0.00399925096609462\\
376	0.00399923371132354\\
377	0.00399921615711194\\
378	0.003999198297882\\
379	0.00399918012789727\\
380	0.00399916164125467\\
381	0.00399914283187607\\
382	0.00399912369349926\\
383	0.00399910421966836\\
384	0.00399908440372357\\
385	0.00399906423879013\\
386	0.0039990437177664\\
387	0.00399902283331092\\
388	0.00399900157782828\\
389	0.00399897994345366\\
390	0.0039989579220355\\
391	0.00399893550511629\\
392	0.00399891268391063\\
393	0.0039988894492798\\
394	0.00399886579170178\\
395	0.00399884170123454\\
396	0.00399881716747023\\
397	0.00399879217947742\\
398	0.00399876672573156\\
399	0.00399874079404327\\
400	0.00399871437151541\\
401	0.00399868744458336\\
402	0.00399865999916131\\
403	0.00399863202070385\\
404	0.00399860349370784\\
405	0.00399857440128768\\
406	0.00399854472560509\\
407	0.00399851444782748\\
408	0.00399848354810668\\
409	0.00399845200558938\\
410	0.00399841979846344\\
411	0.00399838690401134\\
412	0.00399835329856065\\
413	0.00399831895709304\\
414	0.00399828385224411\\
415	0.0039982479529837\\
416	0.00399821122496071\\
417	0.00399817363471804\\
418	0.00399813514658468\\
419	0.00399809572187645\\
420	0.00399805531851538\\
421	0.00399801389059089\\
422	0.00399797138785207\\
423	0.00399792775511748\\
424	0.00399788293158426\\
425	0.00399783685000886\\
426	0.00399778943571111\\
427	0.00399774060530526\\
428	0.00399769026494495\\
429	0.00399763830758216\\
430	0.0039975846080481\\
431	0.0039975290131297\\
432	0.00399747132012266\\
433	0.00399741122957038\\
434	0.00399734824351354\\
435	0.00399728146088603\\
436	0.00399720922077238\\
437	0.00399712866805858\\
438	0.00399703587301835\\
439	0.00399692855921201\\
440	0.00399681364852388\\
441	0.00399669628499214\\
442	0.00399657640237339\\
443	0.00399645393267616\\
444	0.00399632880627346\\
445	0.00399620095205716\\
446	0.00399607029764142\\
447	0.00399593676962055\\
448	0.00399580029388026\\
449	0.00399566079594179\\
450	0.00399551820126536\\
451	0.00399537243530081\\
452	0.00399522342271854\\
453	0.00399507108434469\\
454	0.00399491532797951\\
455	0.00399475602314739\\
456	0.00399459293353915\\
457	0.00399442553680086\\
458	0.00399425254035359\\
459	0.0039940705701746\\
460	0.00399387062197271\\
461	0.00399362865816175\\
462	0.00399328224082247\\
463	0.00399268179826925\\
464	0.00399152034506261\\
465	0.00398942525004241\\
466	0.003987295346337\\
467	0.00398512966627529\\
468	0.00398292721000522\\
469	0.00398068697662909\\
470	0.00397840791133193\\
471	0.00397608888287425\\
472	0.00397372874284859\\
473	0.00397132632195255\\
474	0.00396888037428475\\
475	0.00396638943872895\\
476	0.00396385175245868\\
477	0.0039612657638208\\
478	0.00395862990298715\\
479	0.00395594249982952\\
480	0.00395320177500551\\
481	0.00395040582985437\\
482	0.0039475526346765\\
483	0.00394464001520579\\
484	0.00394166563706097\\
485	0.00393862698785699\\
486	0.00393552135645141\\
487	0.00393234580918657\\
488	0.00392909716312208\\
489	0.00392577195493935\\
490	0.00392236640455913\\
491	0.00391887637213836\\
492	0.00391529730638808\\
493	0.0039116241805659\\
494	0.00390785140870817\\
495	0.0039039727252085\\
496	0.00389998098680041\\
497	0.00389586779479689\\
498	0.0038916226814718\\
499	0.00388723122648509\\
500	0.00388267058066376\\
501	0.0038778989473318\\
502	0.00387283208197042\\
503	0.00386729674842656\\
504	0.00386096597286974\\
505	0.0038534149944631\\
506	0.00384507261769542\\
507	0.00383665866359003\\
508	0.00382817739036378\\
509	0.00381962802362775\\
510	0.00381099192896018\\
511	0.00380225392818428\\
512	0.00379341008661713\\
513	0.00378445601683264\\
514	0.00377538679341138\\
515	0.0037661967812744\\
516	0.0037568792082255\\
517	0.00374742495454606\\
518	0.00373781888331623\\
519	0.00372802825804615\\
520	0.00371796508866977\\
521	0.00370736053989885\\
522	0.00369533602505041\\
523	0.00367890636217919\\
524	0.0036476543580472\\
525	0.00359447189176134\\
526	0.00353963152376423\\
527	0.0034829720449033\\
528	0.00342425368577073\\
529	0.00336310709277027\\
530	0.00329908329211729\\
531	0.00323217575810164\\
532	0.00316298165281235\\
533	0.00309166511403029\\
534	0.00301814370668786\\
535	0.00294111313161606\\
536	0.00285499501479115\\
537	0.00274701482624045\\
538	0.00261820957907621\\
539	0.0024858002176776\\
540	0.00234954247109751\\
541	0.00220916502212144\\
542	0.0020643662871162\\
543	0.00191481016471741\\
544	0.00176011232967979\\
545	0.00159980419294568\\
546	0.00143325509308531\\
547	0.00125945611117335\\
548	0.00107633464582706\\
549	0.000878403440878852\\
550	0.000652863504763817\\
551	0.00041853810151925\\
552	0.00017128870103758\\
553	0\\
554	0\\
555	0\\
556	0\\
557	0\\
558	0\\
559	0\\
560	0\\
561	0\\
562	0\\
563	0\\
564	0\\
565	0\\
566	0\\
567	0\\
568	0\\
569	0\\
570	0\\
571	0\\
572	0\\
573	0\\
574	0\\
575	0\\
576	0\\
577	0\\
578	0\\
579	0\\
580	0\\
581	0\\
582	0\\
583	0\\
584	0\\
585	0\\
586	0\\
587	0\\
588	0\\
589	0\\
590	0\\
591	0\\
592	0\\
593	0\\
594	0\\
595	0\\
596	0\\
597	0\\
598	0\\
599	0\\
600	0\\
};
\addplot [color=mycolor11,solid,forget plot]
  table[row sep=crcr]{%
1	0.00406147387421855\\
2	0.00406147387294227\\
3	0.00406147387164454\\
4	0.00406147387032498\\
5	0.00406147386898323\\
6	0.00406147386761891\\
7	0.00406147386623165\\
8	0.00406147386482107\\
9	0.00406147386338676\\
10	0.00406147386192833\\
11	0.00406147386044538\\
12	0.00406147385893749\\
13	0.00406147385740424\\
14	0.0040614738558452\\
15	0.00406147385425995\\
16	0.00406147385264803\\
17	0.004061473851009\\
18	0.00406147384934242\\
19	0.0040614738476478\\
20	0.00406147384592468\\
21	0.00406147384417258\\
22	0.00406147384239101\\
23	0.00406147384057948\\
24	0.00406147383873747\\
25	0.00406147383686449\\
26	0.00406147383495999\\
27	0.00406147383302347\\
28	0.00406147383105436\\
29	0.00406147382905214\\
30	0.00406147382701623\\
31	0.00406147382494607\\
32	0.00406147382284109\\
33	0.00406147382070069\\
34	0.00406147381852427\\
35	0.00406147381631124\\
36	0.00406147381406098\\
37	0.00406147381177285\\
38	0.00406147380944621\\
39	0.00406147380708042\\
40	0.00406147380467482\\
41	0.00406147380222874\\
42	0.00406147379974148\\
43	0.00406147379721236\\
44	0.00406147379464067\\
45	0.0040614737920257\\
46	0.0040614737893667\\
47	0.00406147378666293\\
48	0.00406147378391364\\
49	0.00406147378111807\\
50	0.00406147377827542\\
51	0.0040614737753849\\
52	0.00406147377244571\\
53	0.00406147376945701\\
54	0.00406147376641797\\
55	0.00406147376332775\\
56	0.00406147376018546\\
57	0.00406147375699024\\
58	0.00406147375374118\\
59	0.00406147375043737\\
60	0.00406147374707789\\
61	0.00406147374366179\\
62	0.00406147374018812\\
63	0.00406147373665589\\
64	0.00406147373306411\\
65	0.00406147372941178\\
66	0.00406147372569786\\
67	0.00406147372192131\\
68	0.00406147371808107\\
69	0.00406147371417606\\
70	0.00406147371020517\\
71	0.00406147370616729\\
72	0.00406147370206127\\
73	0.00406147369788597\\
74	0.00406147369364019\\
75	0.00406147368932274\\
76	0.00406147368493241\\
77	0.00406147368046794\\
78	0.00406147367592809\\
79	0.00406147367131156\\
80	0.00406147366661704\\
81	0.00406147366184321\\
82	0.00406147365698872\\
83	0.00406147365205219\\
84	0.00406147364703221\\
85	0.00406147364192737\\
86	0.00406147363673621\\
87	0.00406147363145726\\
88	0.00406147362608902\\
89	0.00406147362062996\\
90	0.00406147361507853\\
91	0.00406147360943315\\
92	0.0040614736036922\\
93	0.00406147359785406\\
94	0.00406147359191705\\
95	0.00406147358587949\\
96	0.00406147357973964\\
97	0.00406147357349575\\
98	0.00406147356714604\\
99	0.00406147356068869\\
100	0.00406147355412185\\
101	0.00406147354744364\\
102	0.00406147354065214\\
103	0.00406147353374541\\
104	0.00406147352672145\\
105	0.00406147351957827\\
106	0.00406147351231378\\
107	0.00406147350492592\\
108	0.00406147349741255\\
109	0.0040614734897715\\
110	0.00406147348200058\\
111	0.00406147347409753\\
112	0.00406147346606009\\
113	0.00406147345788592\\
114	0.00406147344957267\\
115	0.00406147344111792\\
116	0.00406147343251923\\
117	0.0040614734237741\\
118	0.00406147341488001\\
119	0.00406147340583437\\
120	0.00406147339663454\\
121	0.00406147338727787\\
122	0.00406147337776163\\
123	0.00406147336808304\\
124	0.0040614733582393\\
125	0.00406147334822752\\
126	0.0040614733380448\\
127	0.00406147332768816\\
128	0.00406147331715458\\
129	0.00406147330644098\\
130	0.00406147329554423\\
131	0.00406147328446114\\
132	0.00406147327318846\\
133	0.0040614732617229\\
134	0.00406147325006109\\
135	0.00406147323819962\\
136	0.004061473226135\\
137	0.00406147321386371\\
138	0.00406147320138212\\
139	0.00406147318868658\\
140	0.00406147317577335\\
141	0.00406147316263864\\
142	0.00406147314927858\\
143	0.00406147313568924\\
144	0.00406147312186662\\
145	0.00406147310780666\\
146	0.0040614730935052\\
147	0.00406147307895804\\
148	0.00406147306416089\\
149	0.00406147304910937\\
150	0.00406147303379907\\
151	0.00406147301822546\\
152	0.00406147300238396\\
153	0.00406147298626988\\
154	0.00406147296987847\\
155	0.0040614729532049\\
156	0.00406147293624424\\
157	0.0040614729189915\\
158	0.00406147290144159\\
159	0.00406147288358932\\
160	0.00406147286542943\\
161	0.00406147284695656\\
162	0.00406147282816527\\
163	0.00406147280905001\\
164	0.00406147278960514\\
165	0.00406147276982494\\
166	0.00406147274970356\\
167	0.00406147272923508\\
168	0.00406147270841346\\
169	0.00406147268723257\\
170	0.00406147266568616\\
171	0.00406147264376789\\
172	0.0040614726214713\\
173	0.00406147259878982\\
174	0.00406147257571676\\
175	0.00406147255224534\\
176	0.00406147252836863\\
177	0.00406147250407961\\
178	0.00406147247937111\\
179	0.00406147245423586\\
180	0.00406147242866646\\
181	0.00406147240265536\\
182	0.0040614723761949\\
183	0.00406147234927729\\
184	0.00406147232189459\\
185	0.00406147229403871\\
186	0.00406147226570145\\
187	0.00406147223687445\\
188	0.00406147220754919\\
189	0.00406147217771702\\
190	0.00406147214736913\\
191	0.00406147211649655\\
192	0.00406147208509017\\
193	0.0040614720531407\\
194	0.00406147202063869\\
195	0.00406147198757453\\
196	0.00406147195393844\\
197	0.00406147191972046\\
198	0.00406147188491047\\
199	0.00406147184949814\\
200	0.00406147181347299\\
201	0.00406147177682435\\
202	0.00406147173954134\\
203	0.00406147170161291\\
204	0.00406147166302779\\
205	0.00406147162377455\\
206	0.00406147158384153\\
207	0.00406147154321685\\
208	0.00406147150188845\\
209	0.00406147145984404\\
210	0.00406147141707111\\
211	0.00406147137355695\\
212	0.00406147132928859\\
213	0.00406147128425285\\
214	0.00406147123843631\\
215	0.00406147119182533\\
216	0.00406147114440599\\
217	0.00406147109616416\\
218	0.00406147104708543\\
219	0.00406147099715515\\
220	0.00406147094635841\\
221	0.00406147089468003\\
222	0.00406147084210454\\
223	0.00406147078861623\\
224	0.00406147073419909\\
225	0.00406147067883681\\
226	0.00406147062251282\\
227	0.00406147056521024\\
228	0.00406147050691188\\
229	0.00406147044760024\\
230	0.00406147038725754\\
231	0.00406147032586563\\
232	0.00406147026340609\\
233	0.00406147019986012\\
234	0.00406147013520862\\
235	0.00406147006943212\\
236	0.00406147000251083\\
237	0.00406146993442459\\
238	0.00406146986515287\\
239	0.00406146979467477\\
240	0.00406146972296904\\
241	0.00406146965001403\\
242	0.0040614695757877\\
243	0.00406146950026761\\
244	0.00406146942343094\\
245	0.00406146934525444\\
246	0.00406146926571443\\
247	0.00406146918478683\\
248	0.00406146910244713\\
249	0.00406146901867034\\
250	0.00406146893343106\\
251	0.00406146884670341\\
252	0.00406146875846106\\
253	0.0040614686686772\\
254	0.00406146857732452\\
255	0.00406146848437525\\
256	0.0040614683898011\\
257	0.00406146829357326\\
258	0.00406146819566243\\
259	0.00406146809603876\\
260	0.00406146799467186\\
261	0.00406146789153082\\
262	0.00406146778658415\\
263	0.00406146767979979\\
264	0.00406146757114512\\
265	0.00406146746058691\\
266	0.00406146734809136\\
267	0.00406146723362405\\
268	0.00406146711714993\\
269	0.00406146699863334\\
270	0.00406146687803795\\
271	0.00406146675532681\\
272	0.00406146663046229\\
273	0.00406146650340609\\
274	0.00406146637411921\\
275	0.00406146624256198\\
276	0.00406146610869399\\
277	0.00406146597247412\\
278	0.00406146583386051\\
279	0.00406146569281057\\
280	0.00406146554928092\\
281	0.00406146540322742\\
282	0.00406146525460517\\
283	0.00406146510336842\\
284	0.00406146494947065\\
285	0.00406146479286449\\
286	0.00406146463350175\\
287	0.00406146447133337\\
288	0.00406146430630942\\
289	0.00406146413837912\\
290	0.00406146396749075\\
291	0.00406146379359172\\
292	0.00406146361662849\\
293	0.0040614634365466\\
294	0.00406146325329061\\
295	0.00406146306680416\\
296	0.00406146287702986\\
297	0.00406146268390935\\
298	0.00406146248738325\\
299	0.00406146228739114\\
300	0.00406146208387159\\
301	0.00406146187676207\\
302	0.00406146166599901\\
303	0.00406146145151774\\
304	0.00406146123325248\\
305	0.00406146101113634\\
306	0.00406146078510128\\
307	0.00406146055507813\\
308	0.00406146032099653\\
309	0.00406146008278495\\
310	0.00406145984037066\\
311	0.00406145959367972\\
312	0.00406145934263694\\
313	0.0040614590871659\\
314	0.00406145882718891\\
315	0.00406145856262702\\
316	0.00406145829339994\\
317	0.00406145801942611\\
318	0.00406145774062263\\
319	0.00406145745690525\\
320	0.00406145716818835\\
321	0.00406145687438495\\
322	0.00406145657540666\\
323	0.00406145627116369\\
324	0.00406145596156482\\
325	0.00406145564651736\\
326	0.00406145532592719\\
327	0.00406145499969867\\
328	0.00406145466773469\\
329	0.00406145432993661\\
330	0.00406145398620423\\
331	0.00406145363643581\\
332	0.00406145328052804\\
333	0.00406145291837597\\
334	0.00406145254987306\\
335	0.00406145217491112\\
336	0.00406145179338026\\
337	0.00406145140516892\\
338	0.00406145101016383\\
339	0.00406145060824992\\
340	0.00406145019931037\\
341	0.00406144978322657\\
342	0.00406144935987801\\
343	0.00406144892914235\\
344	0.00406144849089529\\
345	0.00406144804501061\\
346	0.00406144759136004\\
347	0.00406144712981331\\
348	0.00406144666023802\\
349	0.00406144618249962\\
350	0.00406144569646135\\
351	0.00406144520198418\\
352	0.00406144469892675\\
353	0.00406144418714527\\
354	0.00406144366649348\\
355	0.00406144313682254\\
356	0.00406144259798095\\
357	0.00406144204981445\\
358	0.00406144149216595\\
359	0.00406144092487536\\
360	0.00406144034777953\\
361	0.00406143976071209\\
362	0.00406143916350335\\
363	0.00406143855598009\\
364	0.00406143793796551\\
365	0.00406143730927897\\
366	0.00406143666973589\\
367	0.00406143601914752\\
368	0.00406143535732077\\
369	0.00406143468405802\\
370	0.00406143399915687\\
371	0.00406143330240993\\
372	0.00406143259360459\\
373	0.00406143187252274\\
374	0.00406143113894053\\
375	0.00406143039262806\\
376	0.00406142963334911\\
377	0.0040614288608608\\
378	0.00406142807491329\\
379	0.00406142727524937\\
380	0.00406142646160414\\
381	0.00406142563370459\\
382	0.00406142479126918\\
383	0.00406142393400737\\
384	0.00406142306161913\\
385	0.00406142217379441\\
386	0.00406142127021255\\
387	0.00406142035054169\\
388	0.00406141941443797\\
389	0.00406141846154484\\
390	0.0040614174914921\\
391	0.00406141650389494\\
392	0.00406141549835268\\
393	0.00406141447444738\\
394	0.00406141343174207\\
395	0.0040614123697786\\
396	0.00406141128807509\\
397	0.00406141018612313\\
398	0.00406140906338551\\
399	0.00406140791929555\\
400	0.00406140675325923\\
401	0.00406140556465838\\
402	0.00406140435284709\\
403	0.00406140311713441\\
404	0.00406140185677566\\
405	0.00406140057098452\\
406	0.00406139925893173\\
407	0.00406139791974453\\
408	0.00406139655250733\\
409	0.00406139515626258\\
410	0.00406139373000972\\
411	0.00406139227269652\\
412	0.00406139078319603\\
413	0.00406138926026895\\
414	0.00406138770253677\\
415	0.00406138610852019\\
416	0.00406138447673179\\
417	0.00406138280557569\\
418	0.00406138109331847\\
419	0.00406137933807158\\
420	0.00406137753777094\\
421	0.00406137569015324\\
422	0.00406137379272801\\
423	0.00406137184274438\\
424	0.00406136983715\\
425	0.00406136777253767\\
426	0.0040613656450696\\
427	0.00406136345035738\\
428	0.00406136118324853\\
429	0.00406135883741312\\
430	0.00406135640450675\\
431	0.00406135387247213\\
432	0.0040613512222096\\
433	0.00406134842152963\\
434	0.00406134541564776\\
435	0.00406134211645094\\
436	0.0040613384023065\\
437	0.00406133415886565\\
438	0.00406132939785058\\
439	0.00406132437530529\\
440	0.00406131924489778\\
441	0.00406131400369967\\
442	0.00406130864870767\\
443	0.00406130317684924\\
444	0.00406129758499008\\
445	0.00406129186994368\\
446	0.00406128602848235\\
447	0.0040612800573483\\
448	0.00406127395325978\\
449	0.00406126771289894\\
450	0.00406126133284752\\
451	0.00406125480938392\\
452	0.00406124813792591\\
453	0.00406124131158086\\
454	0.00406123431746821\\
455	0.00406122712752537\\
456	0.00406121967581232\\
457	0.00406121180344801\\
458	0.0040612031287363\\
459	0.00406119275532929\\
460	0.00406117867059865\\
461	0.00406115669857275\\
462	0.00406111933689257\\
463	0.0040610566265805\\
464	0.00406096569839961\\
465	0.00406087317548331\\
466	0.00406077901137583\\
467	0.00406068315815308\\
468	0.0040605855668307\\
469	0.00406048618571796\\
470	0.00406038496016992\\
471	0.00406028183402784\\
472	0.00406017674864634\\
473	0.00406006964010068\\
474	0.00405996043544122\\
475	0.00405984905393745\\
476	0.00405973542005483\\
477	0.00405961945579985\\
478	0.00405950107808962\\
479	0.00405938019828523\\
480	0.00405925672166008\\
481	0.00405913054678598\\
482	0.00405900156482686\\
483	0.0040588696587283\\
484	0.00405873470228564\\
485	0.0040585965590667\\
486	0.00405845508117609\\
487	0.00405831010784321\\
488	0.00405816146376941\\
489	0.00405800895716853\\
490	0.00405785237739415\\
491	0.00405769149195838\\
492	0.00405752604254285\\
493	0.00405735573912402\\
494	0.00405718025020847\\
495	0.0040569991845618\\
496	0.00405681205392423\\
497	0.00405661819352445\\
498	0.00405641659189319\\
499	0.00405620553753419\\
500	0.00405598193594154\\
501	0.00405574016564575\\
502	0.00405547073960516\\
503	0.00405516067814581\\
504	0.00405480160717274\\
505	0.00405441381211283\\
506	0.00405402099120583\\
507	0.0040536231225162\\
508	0.00405321987985991\\
509	0.00405281039522229\\
510	0.00405239398982295\\
511	0.00405197033797043\\
512	0.00405153907375548\\
513	0.00405109977356717\\
514	0.00405065191342985\\
515	0.00405019475676502\\
516	0.00404972704919735\\
517	0.00404924617980659\\
518	0.00404874588097109\\
519	0.00404820999895577\\
520	0.00404759604850658\\
521	0.00404679381227801\\
522	0.00404553046041244\\
523	0.00404319768801735\\
524	0.00403876878224668\\
525	0.0040321999423493\\
526	0.00402547244576288\\
527	0.00401856983045933\\
528	0.00401147129796997\\
529	0.0040041569952074\\
530	0.00399662529008306\\
531	0.00398888828751701\\
532	0.00398092784527301\\
533	0.00397268828426009\\
534	0.0039640176156458\\
535	0.00395455217310666\\
536	0.00394369284046989\\
537	0.00393062769130563\\
538	0.00391601606445189\\
539	0.00390129681388448\\
540	0.00388646202356258\\
541	0.00387150286326357\\
542	0.00385640915205039\\
543	0.00384116793333786\\
544	0.0038257600908453\\
545	0.00381015356436016\\
546	0.00379428777375626\\
547	0.00377803953381254\\
548	0.00376116179861637\\
549	0.00374325944500222\\
550	0.00372418624906977\\
551	0.00370500254122745\\
552	0.00368548067614745\\
553	0.00366542135653332\\
554	0.00364300836795439\\
555	0.00360768816192235\\
556	0.00352012484187683\\
557	0.0034200905470597\\
558	0.0033114881502614\\
559	0.00318586122956869\\
560	0.00301590666812544\\
561	0.00282709200229717\\
562	0.00263094421539516\\
563	0.00242651276754218\\
564	0.00221179816707534\\
565	0.00198061034947524\\
566	0.00171394633613848\\
567	0.00143955403725653\\
568	0.00115727971134879\\
569	0.000866069879143256\\
570	0.000563740415471681\\
571	0.000244432763086153\\
572	0\\
573	0\\
574	0\\
575	0\\
576	0\\
577	0\\
578	0\\
579	0\\
580	0\\
581	0\\
582	0\\
583	0\\
584	0\\
585	0\\
586	0\\
587	0\\
588	0\\
589	0\\
590	0\\
591	0\\
592	0\\
593	0\\
594	0\\
595	0\\
596	0\\
597	0\\
598	0\\
599	0\\
600	0\\
};
\addplot [color=mycolor12,solid,forget plot]
  table[row sep=crcr]{%
1	0.00510109179991326\\
2	0.00510109179985886\\
3	0.00510109179980354\\
4	0.00510109179974729\\
5	0.0051010917996901\\
6	0.00510109179963195\\
7	0.00510109179957282\\
8	0.00510109179951269\\
9	0.00510109179945155\\
10	0.00510109179938939\\
11	0.00510109179932617\\
12	0.0051010917992619\\
13	0.00510109179919654\\
14	0.00510109179913009\\
15	0.00510109179906252\\
16	0.00510109179899381\\
17	0.00510109179892395\\
18	0.00510109179885291\\
19	0.00510109179878067\\
20	0.00510109179870723\\
21	0.00510109179863254\\
22	0.0051010917985566\\
23	0.00510109179847938\\
24	0.00510109179840087\\
25	0.00510109179832103\\
26	0.00510109179823985\\
27	0.00510109179815731\\
28	0.00510109179807337\\
29	0.00510109179798803\\
30	0.00510109179790124\\
31	0.005101091797813\\
32	0.00510109179772328\\
33	0.00510109179763204\\
34	0.00510109179753927\\
35	0.00510109179744493\\
36	0.00510109179734901\\
37	0.00510109179725148\\
38	0.00510109179715231\\
39	0.00510109179705146\\
40	0.00510109179694892\\
41	0.00510109179684465\\
42	0.00510109179673863\\
43	0.00510109179663082\\
44	0.0051010917965212\\
45	0.00510109179640974\\
46	0.00510109179629639\\
47	0.00510109179618114\\
48	0.00510109179606395\\
49	0.00510109179594478\\
50	0.00510109179582361\\
51	0.00510109179570039\\
52	0.00510109179557511\\
53	0.00510109179544771\\
54	0.00510109179531816\\
55	0.00510109179518643\\
56	0.00510109179505248\\
57	0.00510109179491628\\
58	0.00510109179477778\\
59	0.00510109179463695\\
60	0.00510109179449374\\
61	0.00510109179434812\\
62	0.00510109179420005\\
63	0.00510109179404947\\
64	0.00510109179389636\\
65	0.00510109179374067\\
66	0.00510109179358235\\
67	0.00510109179342137\\
68	0.00510109179325766\\
69	0.0051010917930912\\
70	0.00510109179292192\\
71	0.00510109179274979\\
72	0.00510109179257475\\
73	0.00510109179239676\\
74	0.00510109179221577\\
75	0.00510109179203172\\
76	0.00510109179184456\\
77	0.00510109179165424\\
78	0.0051010917914607\\
79	0.0051010917912639\\
80	0.00510109179106377\\
81	0.00510109179086026\\
82	0.00510109179065331\\
83	0.00510109179044286\\
84	0.00510109179022885\\
85	0.00510109179001123\\
86	0.00510109178978992\\
87	0.00510109178956487\\
88	0.00510109178933601\\
89	0.00510109178910328\\
90	0.00510109178886661\\
91	0.00510109178862593\\
92	0.00510109178838118\\
93	0.00510109178813228\\
94	0.00510109178787917\\
95	0.00510109178762177\\
96	0.00510109178736\\
97	0.0051010917870938\\
98	0.00510109178682309\\
99	0.00510109178654778\\
100	0.00510109178626781\\
101	0.00510109178598308\\
102	0.00510109178569353\\
103	0.00510109178539906\\
104	0.00510109178509958\\
105	0.00510109178479503\\
106	0.0051010917844853\\
107	0.0051010917841703\\
108	0.00510109178384996\\
109	0.00510109178352417\\
110	0.00510109178319283\\
111	0.00510109178285587\\
112	0.00510109178251317\\
113	0.00510109178216463\\
114	0.00510109178181017\\
115	0.00510109178144967\\
116	0.00510109178108302\\
117	0.00510109178071014\\
118	0.00510109178033089\\
119	0.00510109177994519\\
120	0.0051010917795529\\
121	0.00510109177915392\\
122	0.00510109177874814\\
123	0.00510109177833543\\
124	0.00510109177791568\\
125	0.00510109177748875\\
126	0.00510109177705454\\
127	0.0051010917766129\\
128	0.00510109177616372\\
129	0.00510109177570685\\
130	0.00510109177524217\\
131	0.00510109177476954\\
132	0.00510109177428883\\
133	0.00510109177379988\\
134	0.00510109177330256\\
135	0.00510109177279672\\
136	0.00510109177228222\\
137	0.00510109177175889\\
138	0.0051010917712266\\
139	0.00510109177068517\\
140	0.00510109177013446\\
141	0.0051010917695743\\
142	0.00510109176900452\\
143	0.00510109176842496\\
144	0.00510109176783545\\
145	0.00510109176723581\\
146	0.00510109176662587\\
147	0.00510109176600544\\
148	0.00510109176537435\\
149	0.00510109176473241\\
150	0.00510109176407942\\
151	0.0051010917634152\\
152	0.00510109176273955\\
153	0.00510109176205227\\
154	0.00510109176135316\\
155	0.00510109176064201\\
156	0.00510109175991861\\
157	0.00510109175918275\\
158	0.00510109175843421\\
159	0.00510109175767277\\
160	0.00510109175689821\\
161	0.00510109175611029\\
162	0.00510109175530879\\
163	0.00510109175449347\\
164	0.00510109175366408\\
165	0.00510109175282039\\
166	0.00510109175196214\\
167	0.00510109175108909\\
168	0.00510109175020097\\
169	0.00510109174929752\\
170	0.00510109174837848\\
171	0.00510109174744358\\
172	0.00510109174649253\\
173	0.00510109174552507\\
174	0.0051010917445409\\
175	0.00510109174353974\\
176	0.00510109174252128\\
177	0.00510109174148524\\
178	0.00510109174043131\\
179	0.00510109173935916\\
180	0.0051010917382685\\
181	0.00510109173715899\\
182	0.00510109173603032\\
183	0.00510109173488214\\
184	0.00510109173371412\\
185	0.00510109173252591\\
186	0.00510109173131717\\
187	0.00510109173008753\\
188	0.00510109172883664\\
189	0.00510109172756412\\
190	0.0051010917262696\\
191	0.0051010917249527\\
192	0.00510109172361302\\
193	0.00510109172225017\\
194	0.00510109172086375\\
195	0.00510109171945334\\
196	0.00510109171801854\\
197	0.00510109171655891\\
198	0.00510109171507402\\
199	0.00510109171356343\\
200	0.00510109171202669\\
201	0.00510109171046336\\
202	0.00510109170887296\\
203	0.00510109170725502\\
204	0.00510109170560906\\
205	0.0051010917039346\\
206	0.00510109170223113\\
207	0.00510109170049815\\
208	0.00510109169873514\\
209	0.00510109169694158\\
210	0.00510109169511695\\
211	0.00510109169326067\\
212	0.00510109169137222\\
213	0.00510109168945103\\
214	0.00510109168749652\\
215	0.00510109168550811\\
216	0.00510109168348521\\
217	0.00510109168142722\\
218	0.00510109167933351\\
219	0.00510109167720346\\
220	0.00510109167503645\\
221	0.00510109167283181\\
222	0.00510109167058888\\
223	0.00510109166830701\\
224	0.0051010916659855\\
225	0.00510109166362365\\
226	0.00510109166122077\\
227	0.00510109165877613\\
228	0.00510109165628899\\
229	0.00510109165375861\\
230	0.00510109165118423\\
231	0.00510109164856507\\
232	0.00510109164590035\\
233	0.00510109164318926\\
234	0.005101091640431\\
235	0.00510109163762472\\
236	0.00510109163476959\\
237	0.00510109163186473\\
238	0.00510109162890929\\
239	0.00510109162590235\\
240	0.00510109162284302\\
241	0.00510109161973038\\
242	0.00510109161656347\\
243	0.00510109161334134\\
244	0.00510109161006302\\
245	0.00510109160672752\\
246	0.00510109160333382\\
247	0.00510109159988089\\
248	0.00510109159636769\\
249	0.00510109159279315\\
250	0.00510109158915619\\
251	0.0051010915854557\\
252	0.00510109158169055\\
253	0.00510109157785961\\
254	0.0051010915739617\\
255	0.00510109156999565\\
256	0.00510109156596023\\
257	0.00510109156185422\\
258	0.00510109155767638\\
259	0.00510109155342541\\
260	0.00510109154910004\\
261	0.00510109154469893\\
262	0.00510109154022075\\
263	0.00510109153566412\\
264	0.00510109153102764\\
265	0.00510109152630991\\
266	0.00510109152150948\\
267	0.00510109151662488\\
268	0.0051010915116546\\
269	0.00510109150659713\\
270	0.00510109150145092\\
271	0.0051010914962144\\
272	0.00510109149088594\\
273	0.00510109148546392\\
274	0.00510109147994667\\
275	0.00510109147433249\\
276	0.00510109146861967\\
277	0.00510109146280644\\
278	0.00510109145689103\\
279	0.0051010914508716\\
280	0.00510109144474632\\
281	0.00510109143851329\\
282	0.00510109143217059\\
283	0.00510109142571628\\
284	0.00510109141914836\\
285	0.00510109141246483\\
286	0.00510109140566361\\
287	0.00510109139874261\\
288	0.00510109139169971\\
289	0.00510109138453274\\
290	0.00510109137723948\\
291	0.0051010913698177\\
292	0.00510109136226511\\
293	0.00510109135457938\\
294	0.00510109134675816\\
295	0.00510109133879903\\
296	0.00510109133069954\\
297	0.00510109132245721\\
298	0.00510109131406951\\
299	0.00510109130553385\\
300	0.00510109129684761\\
301	0.00510109128800814\\
302	0.00510109127901271\\
303	0.00510109126985857\\
304	0.00510109126054291\\
305	0.00510109125106289\\
306	0.00510109124141559\\
307	0.00510109123159808\\
308	0.00510109122160734\\
309	0.00510109121144035\\
310	0.00510109120109398\\
311	0.0051010911905651\\
312	0.00510109117985049\\
313	0.0051010911689469\\
314	0.00510109115785102\\
315	0.00510109114655949\\
316	0.00510109113506889\\
317	0.00510109112337573\\
318	0.0051010911114765\\
319	0.0051010910993676\\
320	0.00510109108704538\\
321	0.00510109107450615\\
322	0.00510109106174613\\
323	0.00510109104876151\\
324	0.00510109103554841\\
325	0.00510109102210286\\
326	0.00510109100842088\\
327	0.00510109099449839\\
328	0.00510109098033125\\
329	0.00510109096591527\\
330	0.00510109095124619\\
331	0.00510109093631967\\
332	0.00510109092113132\\
333	0.00510109090567667\\
334	0.00510109088995119\\
335	0.00510109087395028\\
336	0.00510109085766926\\
337	0.00510109084110338\\
338	0.00510109082424782\\
339	0.00510109080709769\\
340	0.00510109078964801\\
341	0.00510109077189374\\
342	0.00510109075382975\\
343	0.00510109073545083\\
344	0.00510109071675169\\
345	0.00510109069772694\\
346	0.00510109067837114\\
347	0.00510109065867872\\
348	0.00510109063864405\\
349	0.00510109061826139\\
350	0.00510109059752491\\
351	0.00510109057642869\\
352	0.00510109055496668\\
353	0.00510109053313275\\
354	0.00510109051092066\\
355	0.00510109048832405\\
356	0.00510109046533644\\
357	0.00510109044195123\\
358	0.0051010904181617\\
359	0.005101090393961\\
360	0.00510109036934214\\
361	0.00510109034429798\\
362	0.00510109031882124\\
363	0.00510109029290449\\
364	0.00510109026654014\\
365	0.0051010902397204\\
366	0.00510109021243735\\
367	0.00510109018468284\\
368	0.00510109015644857\\
369	0.00510109012772599\\
370	0.00510109009850639\\
371	0.00510109006878078\\
372	0.00510109003853998\\
373	0.00510109000777454\\
374	0.00510108997647477\\
375	0.00510108994463069\\
376	0.00510108991223205\\
377	0.0051010898792683\\
378	0.00510108984572858\\
379	0.00510108981160168\\
380	0.00510108977687607\\
381	0.00510108974153983\\
382	0.00510108970558068\\
383	0.00510108966898591\\
384	0.0051010896317424\\
385	0.00510108959383656\\
386	0.00510108955525432\\
387	0.0051010895159811\\
388	0.00510108947600176\\
389	0.00510108943530057\\
390	0.00510108939386118\\
391	0.00510108935166651\\
392	0.00510108930869876\\
393	0.00510108926493928\\
394	0.00510108922036845\\
395	0.00510108917496566\\
396	0.00510108912870914\\
397	0.00510108908157593\\
398	0.00510108903354188\\
399	0.00510108898458167\\
400	0.00510108893466886\\
401	0.00510108888377556\\
402	0.00510108883187196\\
403	0.00510108877892617\\
404	0.00510108872490455\\
405	0.00510108866977168\\
406	0.00510108861349032\\
407	0.00510108855602139\\
408	0.00510108849732398\\
409	0.00510108843735507\\
410	0.00510108837606896\\
411	0.00510108831341628\\
412	0.00510108824934293\\
413	0.00510108818378981\\
414	0.00510108811669441\\
415	0.00510108804799222\\
416	0.00510108797761368\\
417	0.00510108790548305\\
418	0.00510108783151769\\
419	0.00510108775562703\\
420	0.00510108767771156\\
421	0.00510108759766138\\
422	0.00510108751535461\\
423	0.00510108743065502\\
424	0.0051010873434088\\
425	0.00510108725343919\\
426	0.00510108716053712\\
427	0.0051010870644438\\
428	0.00510108696481774\\
429	0.00510108686117289\\
430	0.00510108675276765\\
431	0.00510108663842159\\
432	0.00510108651625388\\
433	0.00510108638341271\\
434	0.00510108623605208\\
435	0.00510108607011444\\
436	0.00510108588358932\\
437	0.00510108567976443\\
438	0.00510108546715944\\
439	0.00510108524996865\\
440	0.00510108502806737\\
441	0.0051010848013279\\
442	0.00510108456961976\\
443	0.00510108433281003\\
444	0.00510108409076368\\
445	0.0051010838433438\\
446	0.00510108359041134\\
447	0.00510108333182357\\
448	0.00510108306742894\\
449	0.00510108279705347\\
450	0.00510108252046636\\
451	0.00510108223729676\\
452	0.00510108194683727\\
453	0.00510108164759044\\
454	0.00510108133624716\\
455	0.00510108100545546\\
456	0.00510108063915062\\
457	0.00510108020337091\\
458	0.00510107962988618\\
459	0.00510107879175149\\
460	0.00510107747983606\\
461	0.00510107541719667\\
462	0.00510107239218583\\
463	0.0051010685759856\\
464	0.00510106469226201\\
465	0.00510106073903992\\
466	0.00510105671427971\\
467	0.00510105261587299\\
468	0.00510104844159795\\
469	0.00510104418911902\\
470	0.00510103985600564\\
471	0.00510103543967317\\
472	0.00510103093728759\\
473	0.00510102634570732\\
474	0.00510102166160921\\
475	0.00510101688175904\\
476	0.00510101200277714\\
477	0.00510100702105975\\
478	0.00510100193275785\\
479	0.00510099673375294\\
480	0.00510099141962911\\
481	0.00510098598564094\\
482	0.0051009804266767\\
483	0.00510097473721582\\
484	0.00510096891127982\\
485	0.0051009629423756\\
486	0.00510095682342939\\
487	0.00510095054670774\\
488	0.0051009441037201\\
489	0.00510093748509278\\
490	0.00510093068039336\\
491	0.00510092367786143\\
492	0.00510091646395133\\
493	0.00510090902248744\\
494	0.00510090133302177\\
495	0.00510089336758987\\
496	0.00510088508440327\\
497	0.00510087641614243\\
498	0.00510086725001991\\
499	0.00510085739884259\\
500	0.00510084657245365\\
501	0.00510083438676113\\
502	0.00510082049579685\\
503	0.00510080494095675\\
504	0.0051007885116239\\
505	0.0051007718624104\\
506	0.00510075498557465\\
507	0.00510073786101444\\
508	0.00510072045661552\\
509	0.00510070274822352\\
510	0.00510068472006531\\
511	0.00510066635294463\\
512	0.00510064762101038\\
513	0.00510062848380644\\
514	0.00510060886667179\\
515	0.0051005886129507\\
516	0.00510056736990364\\
517	0.00510054432491002\\
518	0.00510051762369321\\
519	0.0051004831778553\\
520	0.00510043250933198\\
521	0.00510034977287883\\
522	0.00510021062121271\\
523	0.00509999263205191\\
524	0.00509971271547763\\
525	0.00509942494706406\\
526	0.00509912851553759\\
527	0.00509882253688619\\
528	0.00509850628966137\\
529	0.00509817947339278\\
530	0.00509784152438081\\
531	0.00509748999325015\\
532	0.00509711952279453\\
533	0.00509671942295103\\
534	0.00509627145328356\\
535	0.00509575324774495\\
536	0.00509515058158533\\
537	0.00509449543104951\\
538	0.00509383028532832\\
539	0.00509315442498853\\
540	0.00509246701986117\\
541	0.00509176705844467\\
542	0.00509105318802637\\
543	0.00509032339795339\\
544	0.00508957445850129\\
545	0.00508880094436249\\
546	0.00508799382521893\\
547	0.00508713939509688\\
548	0.0050862221275849\\
549	0.00508523810584875\\
550	0.00508420975301684\\
551	0.00508308849968024\\
552	0.00508175972602962\\
553	0.00507990484665965\\
554	0.00507669461183126\\
555	0.00507044038034362\\
556	0.00505922864793289\\
557	0.00504656909230222\\
558	0.00503244725364375\\
559	0.00501578264226783\\
560	0.00499545046704825\\
561	0.00497398626098588\\
562	0.00495230936879711\\
563	0.00493026676404129\\
564	0.00490753314712123\\
565	0.00488359464059053\\
566	0.0048582284635519\\
567	0.00483327164309529\\
568	0.00480872353417642\\
569	0.00478452488381562\\
570	0.00476060339303416\\
571	0.00473706064438208\\
572	0.0047147640350247\\
573	0.00469513866116233\\
574	0.00467754660897937\\
575	0.00465821856935375\\
576	0.00463326745163829\\
577	0.00459144151574509\\
578	0.00450001461999156\\
579	0.00425903716898513\\
580	0.00396782385977364\\
581	0.00364026773511628\\
582	0.00330161831566113\\
583	0.00294712380411371\\
584	0.00256295996406943\\
585	0.0021687264087653\\
586	0.00176496329681716\\
587	0.00134883630364181\\
588	0.000913380900315357\\
589	0.000438890169453977\\
590	0\\
591	0\\
592	0\\
593	0\\
594	0\\
595	0\\
596	0\\
597	0\\
598	0\\
599	0\\
600	0\\
};
\addplot [color=mycolor13,solid,forget plot]
  table[row sep=crcr]{%
1	0\\
2	0\\
3	0\\
4	0\\
5	0\\
6	0\\
7	0\\
8	0\\
9	0\\
10	0\\
11	0\\
12	0\\
13	0\\
14	0\\
15	0\\
16	0\\
17	0\\
18	0\\
19	0\\
20	0\\
21	0\\
22	0\\
23	0\\
24	0\\
25	0\\
26	0\\
27	0\\
28	0\\
29	0\\
30	0\\
31	0\\
32	0\\
33	0\\
34	0\\
35	0\\
36	0\\
37	0\\
38	0\\
39	0\\
40	0\\
41	0\\
42	0\\
43	0\\
44	0\\
45	0\\
46	0\\
47	0\\
48	0\\
49	0\\
50	0\\
51	0\\
52	0\\
53	0\\
54	0\\
55	0\\
56	0\\
57	0\\
58	0\\
59	0\\
60	0\\
61	0\\
62	0\\
63	0\\
64	0\\
65	0\\
66	0\\
67	0\\
68	0\\
69	0\\
70	0\\
71	0\\
72	0\\
73	0\\
74	0\\
75	0\\
76	0\\
77	0\\
78	0\\
79	0\\
80	0\\
81	0\\
82	0\\
83	0\\
84	0\\
85	0\\
86	0\\
87	0\\
88	0\\
89	0\\
90	0\\
91	0\\
92	0\\
93	0\\
94	0\\
95	0\\
96	0\\
97	0\\
98	0\\
99	0\\
100	0\\
101	0\\
102	0\\
103	0\\
104	0\\
105	0\\
106	0\\
107	0\\
108	0\\
109	0\\
110	0\\
111	0\\
112	0\\
113	0\\
114	0\\
115	0\\
116	0\\
117	0\\
118	0\\
119	0\\
120	0\\
121	0\\
122	0\\
123	0\\
124	0\\
125	0\\
126	0\\
127	0\\
128	0\\
129	0\\
130	0\\
131	0\\
132	0\\
133	0\\
134	0\\
135	0\\
136	0\\
137	0\\
138	0\\
139	0\\
140	0\\
141	0\\
142	0\\
143	0\\
144	0\\
145	0\\
146	0\\
147	0\\
148	0\\
149	0\\
150	0\\
151	0\\
152	0\\
153	0\\
154	0\\
155	0\\
156	0\\
157	0\\
158	0\\
159	0\\
160	0\\
161	0\\
162	0\\
163	0\\
164	0\\
165	0\\
166	0\\
167	0\\
168	0\\
169	0\\
170	0\\
171	0\\
172	0\\
173	0\\
174	0\\
175	0\\
176	0\\
177	0\\
178	0\\
179	0\\
180	0\\
181	0\\
182	0\\
183	0\\
184	0\\
185	0\\
186	0\\
187	0\\
188	0\\
189	0\\
190	0\\
191	0\\
192	0\\
193	0\\
194	0\\
195	0\\
196	0\\
197	0\\
198	0\\
199	0\\
200	0\\
201	0\\
202	0\\
203	0\\
204	0\\
205	0\\
206	0\\
207	0\\
208	0\\
209	0\\
210	0\\
211	0\\
212	0\\
213	0\\
214	0\\
215	0\\
216	0\\
217	0\\
218	0\\
219	0\\
220	0\\
221	0\\
222	0\\
223	0\\
224	0\\
225	0\\
226	0\\
227	0\\
228	0\\
229	0\\
230	0\\
231	0\\
232	0\\
233	0\\
234	0\\
235	0\\
236	0\\
237	0\\
238	0\\
239	0\\
240	0\\
241	0\\
242	0\\
243	0\\
244	0\\
245	0\\
246	0\\
247	0\\
248	0\\
249	0\\
250	0\\
251	0\\
252	0\\
253	0\\
254	0\\
255	0\\
256	0\\
257	0\\
258	0\\
259	0\\
260	0\\
261	0\\
262	0\\
263	0\\
264	0\\
265	0\\
266	0\\
267	0\\
268	0\\
269	0\\
270	0\\
271	0\\
272	0\\
273	0\\
274	0\\
275	0\\
276	0\\
277	0\\
278	0\\
279	0\\
280	0\\
281	0\\
282	0\\
283	0\\
284	0\\
285	0\\
286	0\\
287	0\\
288	0\\
289	0\\
290	0\\
291	0\\
292	0\\
293	0\\
294	0\\
295	0\\
296	0\\
297	0\\
298	0\\
299	0\\
300	0\\
301	0\\
302	0\\
303	0\\
304	0\\
305	0\\
306	0\\
307	0\\
308	0\\
309	0\\
310	0\\
311	0\\
312	0\\
313	0\\
314	0\\
315	0\\
316	0\\
317	0\\
318	0\\
319	0\\
320	0\\
321	0\\
322	0\\
323	0\\
324	0\\
325	0\\
326	0\\
327	0\\
328	0\\
329	0\\
330	0\\
331	0\\
332	0\\
333	0\\
334	0\\
335	0\\
336	0\\
337	0\\
338	0\\
339	0\\
340	0\\
341	0\\
342	0\\
343	0\\
344	0\\
345	0\\
346	0\\
347	0\\
348	0\\
349	0\\
350	0\\
351	0\\
352	0\\
353	0\\
354	0\\
355	0\\
356	0\\
357	0\\
358	0\\
359	0\\
360	0\\
361	0\\
362	0\\
363	0\\
364	0\\
365	0\\
366	0\\
367	0\\
368	0\\
369	0\\
370	0\\
371	0\\
372	0\\
373	0\\
374	0\\
375	0\\
376	0\\
377	0\\
378	0\\
379	0\\
380	0\\
381	0\\
382	0\\
383	0\\
384	0\\
385	0\\
386	0\\
387	0\\
388	0\\
389	0\\
390	0\\
391	0\\
392	0\\
393	0\\
394	0\\
395	0\\
396	0\\
397	0\\
398	0\\
399	0\\
400	0\\
401	0\\
402	0\\
403	0\\
404	0\\
405	0\\
406	0\\
407	0\\
408	0\\
409	0\\
410	0\\
411	0\\
412	0\\
413	0\\
414	0\\
415	0\\
416	0\\
417	0\\
418	0\\
419	0\\
420	0\\
421	0\\
422	0\\
423	0\\
424	0\\
425	0\\
426	0\\
427	0\\
428	0\\
429	0\\
430	0\\
431	0\\
432	0\\
433	0\\
434	0\\
435	0\\
436	0\\
437	0\\
438	0\\
439	0\\
440	0\\
441	0\\
442	0\\
443	0\\
444	0\\
445	0\\
446	0\\
447	0\\
448	0\\
449	0\\
450	0\\
451	0\\
452	0\\
453	0\\
454	0\\
455	0\\
456	0\\
457	0\\
458	0\\
459	0\\
460	0\\
461	0\\
462	0\\
463	0\\
464	0\\
465	0\\
466	0\\
467	0\\
468	0\\
469	0\\
470	0\\
471	0\\
472	0\\
473	0\\
474	0\\
475	0\\
476	0\\
477	0\\
478	0\\
479	0\\
480	0\\
481	0\\
482	0\\
483	0\\
484	0\\
485	0\\
486	0\\
487	0\\
488	0\\
489	0\\
490	0\\
491	0\\
492	0\\
493	0\\
494	0\\
495	0\\
496	0\\
497	0\\
498	0\\
499	0\\
500	0\\
501	0\\
502	0\\
503	0\\
504	0\\
505	0\\
506	0\\
507	0\\
508	0\\
509	0\\
510	0\\
511	0\\
512	0\\
513	0\\
514	0\\
515	0\\
516	0\\
517	0\\
518	0\\
519	0\\
520	0\\
521	0\\
522	0\\
523	0\\
524	0\\
525	0\\
526	0\\
527	0\\
528	0\\
529	0\\
530	0\\
531	0\\
532	0\\
533	0\\
534	0\\
535	0\\
536	0\\
537	0\\
538	0\\
539	0\\
540	0\\
541	0\\
542	0\\
543	0\\
544	0\\
545	0\\
546	0\\
547	0\\
548	0\\
549	0\\
550	0\\
551	0\\
552	0\\
553	0\\
554	0\\
555	0\\
556	0\\
557	0\\
558	0\\
559	0\\
560	0\\
561	0\\
562	0\\
563	0\\
564	0\\
565	0\\
566	0\\
567	0\\
568	0\\
569	0\\
570	0\\
571	0\\
572	0\\
573	0\\
574	0\\
575	0\\
576	0\\
577	0\\
578	0\\
579	0\\
580	0\\
581	0\\
582	0\\
583	0\\
584	0\\
585	0\\
586	0\\
587	0\\
588	0\\
589	0\\
590	0\\
591	0\\
592	0\\
593	0\\
594	9.98598075601821e-05\\
595	0.000594592516223747\\
596	0.00167287594548724\\
597	0.00320324344438694\\
598	0.00587935965236376\\
599	0\\
600	0\\
};
\addplot [color=mycolor14,solid,forget plot]
  table[row sep=crcr]{%
1	0\\
2	0\\
3	0\\
4	0\\
5	0\\
6	0\\
7	0\\
8	0\\
9	0\\
10	0\\
11	0\\
12	0\\
13	0\\
14	0\\
15	0\\
16	0\\
17	0\\
18	0\\
19	0\\
20	0\\
21	0\\
22	0\\
23	0\\
24	0\\
25	0\\
26	0\\
27	0\\
28	0\\
29	0\\
30	0\\
31	0\\
32	0\\
33	0\\
34	0\\
35	0\\
36	0\\
37	0\\
38	0\\
39	0\\
40	0\\
41	0\\
42	0\\
43	0\\
44	0\\
45	0\\
46	0\\
47	0\\
48	0\\
49	0\\
50	0\\
51	0\\
52	0\\
53	0\\
54	0\\
55	0\\
56	0\\
57	0\\
58	0\\
59	0\\
60	0\\
61	0\\
62	0\\
63	0\\
64	0\\
65	0\\
66	0\\
67	0\\
68	0\\
69	0\\
70	0\\
71	0\\
72	0\\
73	0\\
74	0\\
75	0\\
76	0\\
77	0\\
78	0\\
79	0\\
80	0\\
81	0\\
82	0\\
83	0\\
84	0\\
85	0\\
86	0\\
87	0\\
88	0\\
89	0\\
90	0\\
91	0\\
92	0\\
93	0\\
94	0\\
95	0\\
96	0\\
97	0\\
98	0\\
99	0\\
100	0\\
101	0\\
102	0\\
103	0\\
104	0\\
105	0\\
106	0\\
107	0\\
108	0\\
109	0\\
110	0\\
111	0\\
112	0\\
113	0\\
114	0\\
115	0\\
116	0\\
117	0\\
118	0\\
119	0\\
120	0\\
121	0\\
122	0\\
123	0\\
124	0\\
125	0\\
126	0\\
127	0\\
128	0\\
129	0\\
130	0\\
131	0\\
132	0\\
133	0\\
134	0\\
135	0\\
136	0\\
137	0\\
138	0\\
139	0\\
140	0\\
141	0\\
142	0\\
143	0\\
144	0\\
145	0\\
146	0\\
147	0\\
148	0\\
149	0\\
150	0\\
151	0\\
152	0\\
153	0\\
154	0\\
155	0\\
156	0\\
157	0\\
158	0\\
159	0\\
160	0\\
161	0\\
162	0\\
163	0\\
164	0\\
165	0\\
166	0\\
167	0\\
168	0\\
169	0\\
170	0\\
171	0\\
172	0\\
173	0\\
174	0\\
175	0\\
176	0\\
177	0\\
178	0\\
179	0\\
180	0\\
181	0\\
182	0\\
183	0\\
184	0\\
185	0\\
186	0\\
187	0\\
188	0\\
189	0\\
190	0\\
191	0\\
192	0\\
193	0\\
194	0\\
195	0\\
196	0\\
197	0\\
198	0\\
199	0\\
200	0\\
201	0\\
202	0\\
203	0\\
204	0\\
205	0\\
206	0\\
207	0\\
208	0\\
209	0\\
210	0\\
211	0\\
212	0\\
213	0\\
214	0\\
215	0\\
216	0\\
217	0\\
218	0\\
219	0\\
220	0\\
221	0\\
222	0\\
223	0\\
224	0\\
225	0\\
226	0\\
227	0\\
228	0\\
229	0\\
230	0\\
231	0\\
232	0\\
233	0\\
234	0\\
235	0\\
236	0\\
237	0\\
238	0\\
239	0\\
240	0\\
241	0\\
242	0\\
243	0\\
244	0\\
245	0\\
246	0\\
247	0\\
248	0\\
249	0\\
250	0\\
251	0\\
252	0\\
253	0\\
254	0\\
255	0\\
256	0\\
257	0\\
258	0\\
259	0\\
260	0\\
261	0\\
262	0\\
263	0\\
264	0\\
265	0\\
266	0\\
267	0\\
268	0\\
269	0\\
270	0\\
271	0\\
272	0\\
273	0\\
274	0\\
275	0\\
276	0\\
277	0\\
278	0\\
279	0\\
280	0\\
281	0\\
282	0\\
283	0\\
284	0\\
285	0\\
286	0\\
287	0\\
288	0\\
289	0\\
290	0\\
291	0\\
292	0\\
293	0\\
294	0\\
295	0\\
296	0\\
297	0\\
298	0\\
299	0\\
300	0\\
301	0\\
302	0\\
303	0\\
304	0\\
305	0\\
306	0\\
307	0\\
308	0\\
309	0\\
310	0\\
311	0\\
312	0\\
313	0\\
314	0\\
315	0\\
316	0\\
317	0\\
318	0\\
319	0\\
320	0\\
321	0\\
322	0\\
323	0\\
324	0\\
325	0\\
326	0\\
327	0\\
328	0\\
329	0\\
330	0\\
331	0\\
332	0\\
333	0\\
334	0\\
335	0\\
336	0\\
337	0\\
338	0\\
339	0\\
340	0\\
341	0\\
342	0\\
343	0\\
344	0\\
345	0\\
346	0\\
347	0\\
348	0\\
349	0\\
350	0\\
351	0\\
352	0\\
353	0\\
354	0\\
355	0\\
356	0\\
357	0\\
358	0\\
359	0\\
360	0\\
361	0\\
362	0\\
363	0\\
364	0\\
365	0\\
366	0\\
367	0\\
368	0\\
369	0\\
370	0\\
371	0\\
372	0\\
373	0\\
374	0\\
375	0\\
376	0\\
377	0\\
378	0\\
379	0\\
380	0\\
381	0\\
382	0\\
383	0\\
384	0\\
385	0\\
386	0\\
387	0\\
388	0\\
389	0\\
390	0\\
391	0\\
392	0\\
393	0\\
394	0\\
395	0\\
396	0\\
397	0\\
398	0\\
399	0\\
400	0\\
401	0\\
402	0\\
403	0\\
404	0\\
405	0\\
406	0\\
407	0\\
408	0\\
409	0\\
410	0\\
411	0\\
412	0\\
413	0\\
414	0\\
415	0\\
416	0\\
417	0\\
418	0\\
419	0\\
420	0\\
421	0\\
422	0\\
423	0\\
424	0\\
425	0\\
426	0\\
427	0\\
428	0\\
429	0\\
430	0\\
431	0\\
432	0\\
433	0\\
434	0\\
435	0\\
436	0\\
437	0\\
438	0\\
439	0\\
440	0\\
441	0\\
442	0\\
443	0\\
444	0\\
445	0\\
446	0\\
447	0\\
448	0\\
449	0\\
450	0\\
451	0\\
452	0\\
453	0\\
454	0\\
455	0\\
456	0\\
457	0\\
458	0\\
459	0\\
460	0\\
461	0\\
462	0\\
463	0\\
464	0\\
465	0\\
466	0\\
467	0\\
468	0\\
469	0\\
470	0\\
471	0\\
472	0\\
473	0\\
474	0\\
475	0\\
476	0\\
477	0\\
478	0\\
479	0\\
480	0\\
481	0\\
482	0\\
483	0\\
484	0\\
485	0\\
486	0\\
487	0\\
488	0\\
489	0\\
490	0\\
491	0\\
492	0\\
493	0\\
494	0\\
495	0\\
496	0\\
497	0\\
498	0\\
499	0\\
500	0\\
501	0\\
502	0\\
503	0\\
504	0\\
505	0\\
506	0\\
507	0\\
508	0\\
509	0\\
510	0\\
511	0\\
512	0\\
513	0\\
514	0\\
515	0\\
516	0\\
517	0\\
518	0\\
519	0\\
520	0\\
521	0\\
522	0\\
523	0\\
524	0\\
525	0\\
526	0\\
527	0\\
528	0\\
529	0\\
530	0\\
531	0\\
532	0\\
533	0\\
534	0\\
535	0\\
536	0\\
537	0\\
538	0\\
539	0\\
540	0\\
541	0\\
542	0\\
543	0\\
544	0\\
545	0\\
546	0\\
547	0\\
548	0\\
549	0\\
550	0\\
551	0\\
552	0\\
553	0\\
554	0\\
555	0\\
556	0\\
557	0\\
558	0\\
559	0\\
560	0\\
561	0\\
562	0\\
563	0\\
564	0\\
565	0\\
566	0\\
567	0\\
568	0\\
569	0\\
570	0\\
571	0\\
572	0\\
573	0\\
574	0\\
575	0\\
576	0\\
577	0\\
578	0\\
579	0\\
580	0\\
581	0\\
582	0\\
583	0\\
584	0\\
585	0\\
586	0.000167079632204449\\
587	0.000456071715075539\\
588	0.000759434260115618\\
589	0.000951398265291226\\
590	0.00114744334461536\\
591	0.0013504417486169\\
592	0.00161095250100881\\
593	0.00219885314480476\\
594	0.00282841440239993\\
595	0.00353909829286836\\
596	0.00399657994176672\\
597	0.0047545345923529\\
598	0.00632942537858856\\
599	0\\
600	0\\
};
\addplot [color=mycolor15,solid,forget plot]
  table[row sep=crcr]{%
1	0\\
2	0\\
3	0\\
4	0\\
5	0\\
6	0\\
7	0\\
8	0\\
9	0\\
10	0\\
11	0\\
12	0\\
13	0\\
14	0\\
15	0\\
16	0\\
17	0\\
18	0\\
19	0\\
20	0\\
21	0\\
22	0\\
23	0\\
24	0\\
25	0\\
26	0\\
27	0\\
28	0\\
29	0\\
30	0\\
31	0\\
32	0\\
33	0\\
34	0\\
35	0\\
36	0\\
37	0\\
38	0\\
39	0\\
40	0\\
41	0\\
42	0\\
43	0\\
44	0\\
45	0\\
46	0\\
47	0\\
48	0\\
49	0\\
50	0\\
51	0\\
52	0\\
53	0\\
54	0\\
55	0\\
56	0\\
57	0\\
58	0\\
59	0\\
60	0\\
61	0\\
62	0\\
63	0\\
64	0\\
65	0\\
66	0\\
67	0\\
68	0\\
69	0\\
70	0\\
71	0\\
72	0\\
73	0\\
74	0\\
75	0\\
76	0\\
77	0\\
78	0\\
79	0\\
80	0\\
81	0\\
82	0\\
83	0\\
84	0\\
85	0\\
86	0\\
87	0\\
88	0\\
89	0\\
90	0\\
91	0\\
92	0\\
93	0\\
94	0\\
95	0\\
96	0\\
97	0\\
98	0\\
99	0\\
100	0\\
101	0\\
102	0\\
103	0\\
104	0\\
105	0\\
106	0\\
107	0\\
108	0\\
109	0\\
110	0\\
111	0\\
112	0\\
113	0\\
114	0\\
115	0\\
116	0\\
117	0\\
118	0\\
119	0\\
120	0\\
121	0\\
122	0\\
123	0\\
124	0\\
125	0\\
126	0\\
127	0\\
128	0\\
129	0\\
130	0\\
131	0\\
132	0\\
133	0\\
134	0\\
135	0\\
136	0\\
137	0\\
138	0\\
139	0\\
140	0\\
141	0\\
142	0\\
143	0\\
144	0\\
145	0\\
146	0\\
147	0\\
148	0\\
149	0\\
150	0\\
151	0\\
152	0\\
153	0\\
154	0\\
155	0\\
156	0\\
157	0\\
158	0\\
159	0\\
160	0\\
161	0\\
162	0\\
163	0\\
164	0\\
165	0\\
166	0\\
167	0\\
168	0\\
169	0\\
170	0\\
171	0\\
172	0\\
173	0\\
174	0\\
175	0\\
176	0\\
177	0\\
178	0\\
179	0\\
180	0\\
181	0\\
182	0\\
183	0\\
184	0\\
185	0\\
186	0\\
187	0\\
188	0\\
189	0\\
190	0\\
191	0\\
192	0\\
193	0\\
194	0\\
195	0\\
196	0\\
197	0\\
198	0\\
199	0\\
200	0\\
201	0\\
202	0\\
203	0\\
204	0\\
205	0\\
206	0\\
207	0\\
208	0\\
209	0\\
210	0\\
211	0\\
212	0\\
213	0\\
214	0\\
215	0\\
216	0\\
217	0\\
218	0\\
219	0\\
220	0\\
221	0\\
222	0\\
223	0\\
224	0\\
225	0\\
226	0\\
227	0\\
228	0\\
229	0\\
230	0\\
231	0\\
232	0\\
233	0\\
234	0\\
235	0\\
236	0\\
237	0\\
238	0\\
239	0\\
240	0\\
241	0\\
242	0\\
243	0\\
244	0\\
245	0\\
246	0\\
247	0\\
248	0\\
249	0\\
250	0\\
251	0\\
252	0\\
253	0\\
254	0\\
255	0\\
256	0\\
257	0\\
258	0\\
259	0\\
260	0\\
261	0\\
262	0\\
263	0\\
264	0\\
265	0\\
266	0\\
267	0\\
268	0\\
269	0\\
270	0\\
271	0\\
272	0\\
273	0\\
274	0\\
275	0\\
276	0\\
277	0\\
278	0\\
279	0\\
280	0\\
281	0\\
282	0\\
283	0\\
284	0\\
285	0\\
286	0\\
287	0\\
288	0\\
289	0\\
290	0\\
291	0\\
292	0\\
293	0\\
294	0\\
295	0\\
296	0\\
297	0\\
298	0\\
299	0\\
300	0\\
301	0\\
302	0\\
303	0\\
304	0\\
305	0\\
306	0\\
307	0\\
308	0\\
309	0\\
310	0\\
311	0\\
312	0\\
313	0\\
314	0\\
315	0\\
316	0\\
317	0\\
318	0\\
319	0\\
320	0\\
321	0\\
322	0\\
323	0\\
324	0\\
325	0\\
326	0\\
327	0\\
328	0\\
329	0\\
330	0\\
331	0\\
332	0\\
333	0\\
334	0\\
335	0\\
336	0\\
337	0\\
338	0\\
339	0\\
340	0\\
341	0\\
342	0\\
343	0\\
344	0\\
345	0\\
346	0\\
347	0\\
348	0\\
349	0\\
350	0\\
351	0\\
352	0\\
353	0\\
354	0\\
355	0\\
356	0\\
357	0\\
358	0\\
359	0\\
360	0\\
361	0\\
362	0\\
363	0\\
364	0\\
365	0\\
366	0\\
367	0\\
368	0\\
369	0\\
370	0\\
371	0\\
372	0\\
373	0\\
374	0\\
375	0\\
376	0\\
377	0\\
378	0\\
379	0\\
380	0\\
381	0\\
382	0\\
383	0\\
384	0\\
385	0\\
386	0\\
387	0\\
388	0\\
389	0\\
390	0\\
391	0\\
392	0\\
393	0\\
394	0\\
395	0\\
396	0\\
397	0\\
398	0\\
399	0\\
400	0\\
401	0\\
402	0\\
403	0\\
404	0\\
405	0\\
406	0\\
407	0\\
408	0\\
409	0\\
410	0\\
411	0\\
412	0\\
413	0\\
414	0\\
415	0\\
416	0\\
417	0\\
418	0\\
419	0\\
420	0\\
421	0\\
422	0\\
423	0\\
424	0\\
425	0\\
426	0\\
427	0\\
428	0\\
429	0\\
430	0\\
431	0\\
432	0\\
433	0\\
434	0\\
435	0\\
436	0\\
437	0\\
438	0\\
439	0\\
440	0\\
441	0\\
442	0\\
443	0\\
444	0\\
445	0\\
446	0\\
447	0\\
448	0\\
449	0\\
450	0\\
451	0\\
452	0\\
453	0\\
454	0\\
455	0\\
456	0\\
457	0\\
458	0\\
459	0\\
460	0\\
461	0\\
462	0\\
463	0\\
464	0\\
465	0\\
466	0\\
467	0\\
468	0\\
469	0\\
470	0\\
471	0\\
472	0\\
473	0\\
474	0\\
475	0\\
476	0\\
477	0\\
478	0\\
479	0\\
480	0\\
481	0\\
482	0\\
483	0\\
484	0\\
485	0\\
486	0\\
487	0\\
488	0\\
489	0\\
490	0\\
491	0\\
492	0\\
493	0\\
494	0\\
495	0\\
496	0\\
497	0\\
498	0\\
499	0\\
500	0\\
501	0\\
502	0\\
503	0\\
504	0\\
505	0\\
506	0\\
507	0\\
508	0\\
509	0\\
510	0\\
511	0\\
512	0\\
513	0\\
514	0\\
515	0\\
516	0\\
517	0\\
518	0\\
519	0\\
520	0\\
521	0\\
522	0\\
523	0\\
524	0\\
525	0\\
526	0\\
527	0\\
528	0\\
529	0\\
530	0\\
531	0\\
532	0\\
533	0\\
534	0\\
535	0\\
536	0\\
537	0\\
538	0\\
539	0\\
540	0\\
541	0\\
542	0\\
543	0\\
544	0\\
545	0\\
546	0\\
547	0\\
548	0\\
549	0\\
550	0\\
551	0\\
552	0\\
553	0\\
554	0\\
555	0\\
556	0\\
557	0\\
558	0\\
559	0\\
560	0\\
561	0\\
562	0\\
563	0\\
564	0\\
565	0\\
566	0\\
567	0\\
568	0\\
569	0\\
570	0\\
571	0\\
572	0\\
573	0\\
574	0\\
575	0\\
576	0\\
577	0\\
578	0\\
579	0.000143936771195565\\
580	0.000398036759884528\\
581	0.00066092679016297\\
582	0.000819713415261157\\
583	0.00096387575301797\\
584	0.00110653927940749\\
585	0.00124960147875673\\
586	0.00139021956195936\\
587	0.00152910028465296\\
588	0.00165893564622842\\
589	0.00185748867593825\\
590	0.00235311191256397\\
591	0.00285266479926953\\
592	0.00331714209625941\\
593	0.00350129544269832\\
594	0.00369876633599398\\
595	0.00393138688805932\\
596	0.00424489192197758\\
597	0.0048700418989439\\
598	0.00632942537858856\\
599	0\\
600	0\\
};
\addplot [color=mycolor16,solid,forget plot]
  table[row sep=crcr]{%
1	0\\
2	0\\
3	0\\
4	0\\
5	0\\
6	0\\
7	0\\
8	0\\
9	0\\
10	0\\
11	0\\
12	0\\
13	0\\
14	0\\
15	0\\
16	0\\
17	0\\
18	0\\
19	0\\
20	0\\
21	0\\
22	0\\
23	0\\
24	0\\
25	0\\
26	0\\
27	0\\
28	0\\
29	0\\
30	0\\
31	0\\
32	0\\
33	0\\
34	0\\
35	0\\
36	0\\
37	0\\
38	0\\
39	0\\
40	0\\
41	0\\
42	0\\
43	0\\
44	0\\
45	0\\
46	0\\
47	0\\
48	0\\
49	0\\
50	0\\
51	0\\
52	0\\
53	0\\
54	0\\
55	0\\
56	0\\
57	0\\
58	0\\
59	0\\
60	0\\
61	0\\
62	0\\
63	0\\
64	0\\
65	0\\
66	0\\
67	0\\
68	0\\
69	0\\
70	0\\
71	0\\
72	0\\
73	0\\
74	0\\
75	0\\
76	0\\
77	0\\
78	0\\
79	0\\
80	0\\
81	0\\
82	0\\
83	0\\
84	0\\
85	0\\
86	0\\
87	0\\
88	0\\
89	0\\
90	0\\
91	0\\
92	0\\
93	0\\
94	0\\
95	0\\
96	0\\
97	0\\
98	0\\
99	0\\
100	0\\
101	0\\
102	0\\
103	0\\
104	0\\
105	0\\
106	0\\
107	0\\
108	0\\
109	0\\
110	0\\
111	0\\
112	0\\
113	0\\
114	0\\
115	0\\
116	0\\
117	0\\
118	0\\
119	0\\
120	0\\
121	0\\
122	0\\
123	0\\
124	0\\
125	0\\
126	0\\
127	0\\
128	0\\
129	0\\
130	0\\
131	0\\
132	0\\
133	0\\
134	0\\
135	0\\
136	0\\
137	0\\
138	0\\
139	0\\
140	0\\
141	0\\
142	0\\
143	0\\
144	0\\
145	0\\
146	0\\
147	0\\
148	0\\
149	0\\
150	0\\
151	0\\
152	0\\
153	0\\
154	0\\
155	0\\
156	0\\
157	0\\
158	0\\
159	0\\
160	0\\
161	0\\
162	0\\
163	0\\
164	0\\
165	0\\
166	0\\
167	0\\
168	0\\
169	0\\
170	0\\
171	0\\
172	0\\
173	0\\
174	0\\
175	0\\
176	0\\
177	0\\
178	0\\
179	0\\
180	0\\
181	0\\
182	0\\
183	0\\
184	0\\
185	0\\
186	0\\
187	0\\
188	0\\
189	0\\
190	0\\
191	0\\
192	0\\
193	0\\
194	0\\
195	0\\
196	0\\
197	0\\
198	0\\
199	0\\
200	0\\
201	0\\
202	0\\
203	0\\
204	0\\
205	0\\
206	0\\
207	0\\
208	0\\
209	0\\
210	0\\
211	0\\
212	0\\
213	0\\
214	0\\
215	0\\
216	0\\
217	0\\
218	0\\
219	0\\
220	0\\
221	0\\
222	0\\
223	0\\
224	0\\
225	0\\
226	0\\
227	0\\
228	0\\
229	0\\
230	0\\
231	0\\
232	0\\
233	0\\
234	0\\
235	0\\
236	0\\
237	0\\
238	0\\
239	0\\
240	0\\
241	0\\
242	0\\
243	0\\
244	0\\
245	0\\
246	0\\
247	0\\
248	0\\
249	0\\
250	0\\
251	0\\
252	0\\
253	0\\
254	0\\
255	0\\
256	0\\
257	0\\
258	0\\
259	0\\
260	0\\
261	0\\
262	0\\
263	0\\
264	0\\
265	0\\
266	0\\
267	0\\
268	0\\
269	0\\
270	0\\
271	0\\
272	0\\
273	0\\
274	0\\
275	0\\
276	0\\
277	0\\
278	0\\
279	0\\
280	0\\
281	0\\
282	0\\
283	0\\
284	0\\
285	0\\
286	0\\
287	0\\
288	0\\
289	0\\
290	0\\
291	0\\
292	0\\
293	0\\
294	0\\
295	0\\
296	0\\
297	0\\
298	0\\
299	0\\
300	0\\
301	0\\
302	0\\
303	0\\
304	0\\
305	0\\
306	0\\
307	0\\
308	0\\
309	0\\
310	0\\
311	0\\
312	0\\
313	0\\
314	0\\
315	0\\
316	0\\
317	0\\
318	0\\
319	0\\
320	0\\
321	0\\
322	0\\
323	0\\
324	0\\
325	0\\
326	0\\
327	0\\
328	0\\
329	0\\
330	0\\
331	0\\
332	0\\
333	0\\
334	0\\
335	0\\
336	0\\
337	0\\
338	0\\
339	0\\
340	0\\
341	0\\
342	0\\
343	0\\
344	0\\
345	0\\
346	0\\
347	0\\
348	0\\
349	0\\
350	0\\
351	0\\
352	0\\
353	0\\
354	0\\
355	0\\
356	0\\
357	0\\
358	0\\
359	0\\
360	0\\
361	0\\
362	0\\
363	0\\
364	0\\
365	0\\
366	0\\
367	0\\
368	0\\
369	0\\
370	0\\
371	0\\
372	0\\
373	0\\
374	0\\
375	0\\
376	0\\
377	0\\
378	0\\
379	0\\
380	0\\
381	0\\
382	0\\
383	0\\
384	0\\
385	0\\
386	0\\
387	0\\
388	0\\
389	0\\
390	0\\
391	0\\
392	0\\
393	0\\
394	0\\
395	0\\
396	0\\
397	0\\
398	0\\
399	0\\
400	0\\
401	0\\
402	0\\
403	0\\
404	0\\
405	0\\
406	0\\
407	0\\
408	0\\
409	0\\
410	0\\
411	0\\
412	0\\
413	0\\
414	0\\
415	0\\
416	0\\
417	0\\
418	0\\
419	0\\
420	0\\
421	0\\
422	0\\
423	0\\
424	0\\
425	0\\
426	0\\
427	0\\
428	0\\
429	0\\
430	0\\
431	0\\
432	0\\
433	0\\
434	0\\
435	0\\
436	0\\
437	0\\
438	0\\
439	0\\
440	0\\
441	0\\
442	0\\
443	0\\
444	0\\
445	0\\
446	0\\
447	0\\
448	0\\
449	0\\
450	0\\
451	0\\
452	0\\
453	0\\
454	0\\
455	0\\
456	0\\
457	0\\
458	0\\
459	0\\
460	0\\
461	0\\
462	0\\
463	0\\
464	0\\
465	0\\
466	0\\
467	0\\
468	0\\
469	0\\
470	0\\
471	0\\
472	0\\
473	0\\
474	0\\
475	0\\
476	0\\
477	0\\
478	0\\
479	0\\
480	0\\
481	0\\
482	0\\
483	0\\
484	0\\
485	0\\
486	0\\
487	0\\
488	0\\
489	0\\
490	0\\
491	0\\
492	0\\
493	0\\
494	0\\
495	0\\
496	0\\
497	0\\
498	0\\
499	0\\
500	0\\
501	0\\
502	0\\
503	0\\
504	0\\
505	0\\
506	0\\
507	0\\
508	0\\
509	0\\
510	0\\
511	0\\
512	0\\
513	0\\
514	0\\
515	0\\
516	0\\
517	0\\
518	0\\
519	0\\
520	0\\
521	0\\
522	0\\
523	0\\
524	0\\
525	0\\
526	0\\
527	0\\
528	0\\
529	0\\
530	0\\
531	0\\
532	0\\
533	0\\
534	0\\
535	0\\
536	0\\
537	0\\
538	0\\
539	0\\
540	0\\
541	0\\
542	0\\
543	0\\
544	0\\
545	0\\
546	0\\
547	0\\
548	0\\
549	0\\
550	0\\
551	0\\
552	0\\
553	0\\
554	0\\
555	0\\
556	0\\
557	0\\
558	0\\
559	0\\
560	0\\
561	0\\
562	0\\
563	0\\
564	0\\
565	0\\
566	0\\
567	0\\
568	0\\
569	0\\
570	0\\
571	0\\
572	0\\
573	0.000161696296701996\\
574	0.000393713864803922\\
575	0.000598899566224693\\
576	0.000720797071750311\\
577	0.000841506955469324\\
578	0.000959997374964595\\
579	0.00107594147230983\\
580	0.00118817755523261\\
581	0.00129194223306109\\
582	0.00138617795658753\\
583	0.00147953228162457\\
584	0.00157379580842985\\
585	0.00166431968199408\\
586	0.00175199502013113\\
587	0.00222288556602185\\
588	0.00270274779694502\\
589	0.00311897222604471\\
590	0.00326220003215725\\
591	0.00340447372974604\\
592	0.00353900409992704\\
593	0.00365273120413177\\
594	0.00378238158822432\\
595	0.0039567480325281\\
596	0.00425297862742962\\
597	0.0048700418989439\\
598	0.00632942537858856\\
599	0\\
600	0\\
};
\addplot [color=mycolor17,solid,forget plot]
  table[row sep=crcr]{%
1	0\\
2	0\\
3	0\\
4	0\\
5	0\\
6	0\\
7	0\\
8	0\\
9	0\\
10	0\\
11	0\\
12	0\\
13	0\\
14	0\\
15	0\\
16	0\\
17	0\\
18	0\\
19	0\\
20	0\\
21	0\\
22	0\\
23	0\\
24	0\\
25	0\\
26	0\\
27	0\\
28	0\\
29	0\\
30	0\\
31	0\\
32	0\\
33	0\\
34	0\\
35	0\\
36	0\\
37	0\\
38	0\\
39	0\\
40	0\\
41	0\\
42	0\\
43	0\\
44	0\\
45	0\\
46	0\\
47	0\\
48	0\\
49	0\\
50	0\\
51	0\\
52	0\\
53	0\\
54	0\\
55	0\\
56	0\\
57	0\\
58	0\\
59	0\\
60	0\\
61	0\\
62	0\\
63	0\\
64	0\\
65	0\\
66	0\\
67	0\\
68	0\\
69	0\\
70	0\\
71	0\\
72	0\\
73	0\\
74	0\\
75	0\\
76	0\\
77	0\\
78	0\\
79	0\\
80	0\\
81	0\\
82	0\\
83	0\\
84	0\\
85	0\\
86	0\\
87	0\\
88	0\\
89	0\\
90	0\\
91	0\\
92	0\\
93	0\\
94	0\\
95	0\\
96	0\\
97	0\\
98	0\\
99	0\\
100	0\\
101	0\\
102	0\\
103	0\\
104	0\\
105	0\\
106	0\\
107	0\\
108	0\\
109	0\\
110	0\\
111	0\\
112	0\\
113	0\\
114	0\\
115	0\\
116	0\\
117	0\\
118	0\\
119	0\\
120	0\\
121	0\\
122	0\\
123	0\\
124	0\\
125	0\\
126	0\\
127	0\\
128	0\\
129	0\\
130	0\\
131	0\\
132	0\\
133	0\\
134	0\\
135	0\\
136	0\\
137	0\\
138	0\\
139	0\\
140	0\\
141	0\\
142	0\\
143	0\\
144	0\\
145	0\\
146	0\\
147	0\\
148	0\\
149	0\\
150	0\\
151	0\\
152	0\\
153	0\\
154	0\\
155	0\\
156	0\\
157	0\\
158	0\\
159	0\\
160	0\\
161	0\\
162	0\\
163	0\\
164	0\\
165	0\\
166	0\\
167	0\\
168	0\\
169	0\\
170	0\\
171	0\\
172	0\\
173	0\\
174	0\\
175	0\\
176	0\\
177	0\\
178	0\\
179	0\\
180	0\\
181	0\\
182	0\\
183	0\\
184	0\\
185	0\\
186	0\\
187	0\\
188	0\\
189	0\\
190	0\\
191	0\\
192	0\\
193	0\\
194	0\\
195	0\\
196	0\\
197	0\\
198	0\\
199	0\\
200	0\\
201	0\\
202	0\\
203	0\\
204	0\\
205	0\\
206	0\\
207	0\\
208	0\\
209	0\\
210	0\\
211	0\\
212	0\\
213	0\\
214	0\\
215	0\\
216	0\\
217	0\\
218	0\\
219	0\\
220	0\\
221	0\\
222	0\\
223	0\\
224	0\\
225	0\\
226	0\\
227	0\\
228	0\\
229	0\\
230	0\\
231	0\\
232	0\\
233	0\\
234	0\\
235	0\\
236	0\\
237	0\\
238	0\\
239	0\\
240	0\\
241	0\\
242	0\\
243	0\\
244	0\\
245	0\\
246	0\\
247	0\\
248	0\\
249	0\\
250	0\\
251	0\\
252	0\\
253	0\\
254	0\\
255	0\\
256	0\\
257	0\\
258	0\\
259	0\\
260	0\\
261	0\\
262	0\\
263	0\\
264	0\\
265	0\\
266	0\\
267	0\\
268	0\\
269	0\\
270	0\\
271	0\\
272	0\\
273	0\\
274	0\\
275	0\\
276	0\\
277	0\\
278	0\\
279	0\\
280	0\\
281	0\\
282	0\\
283	0\\
284	0\\
285	0\\
286	0\\
287	0\\
288	0\\
289	0\\
290	0\\
291	0\\
292	0\\
293	0\\
294	0\\
295	0\\
296	0\\
297	0\\
298	0\\
299	0\\
300	0\\
301	0\\
302	0\\
303	0\\
304	0\\
305	0\\
306	0\\
307	0\\
308	0\\
309	0\\
310	0\\
311	0\\
312	0\\
313	0\\
314	0\\
315	0\\
316	0\\
317	0\\
318	0\\
319	0\\
320	0\\
321	0\\
322	0\\
323	0\\
324	0\\
325	0\\
326	0\\
327	0\\
328	0\\
329	0\\
330	0\\
331	0\\
332	0\\
333	0\\
334	0\\
335	0\\
336	0\\
337	0\\
338	0\\
339	0\\
340	0\\
341	0\\
342	0\\
343	0\\
344	0\\
345	0\\
346	0\\
347	0\\
348	0\\
349	0\\
350	0\\
351	0\\
352	0\\
353	0\\
354	0\\
355	0\\
356	0\\
357	0\\
358	0\\
359	0\\
360	0\\
361	0\\
362	0\\
363	0\\
364	0\\
365	0\\
366	0\\
367	0\\
368	0\\
369	0\\
370	0\\
371	0\\
372	0\\
373	0\\
374	0\\
375	0\\
376	0\\
377	0\\
378	0\\
379	0\\
380	0\\
381	0\\
382	0\\
383	0\\
384	0\\
385	0\\
386	0\\
387	0\\
388	0\\
389	0\\
390	0\\
391	0\\
392	0\\
393	0\\
394	0\\
395	0\\
396	0\\
397	0\\
398	0\\
399	0\\
400	0\\
401	0\\
402	0\\
403	0\\
404	0\\
405	0\\
406	0\\
407	0\\
408	0\\
409	0\\
410	0\\
411	0\\
412	0\\
413	0\\
414	0\\
415	0\\
416	0\\
417	0\\
418	0\\
419	0\\
420	0\\
421	0\\
422	0\\
423	0\\
424	0\\
425	0\\
426	0\\
427	0\\
428	0\\
429	0\\
430	0\\
431	0\\
432	0\\
433	0\\
434	0\\
435	0\\
436	0\\
437	0\\
438	0\\
439	0\\
440	0\\
441	0\\
442	0\\
443	0\\
444	0\\
445	0\\
446	0\\
447	0\\
448	0\\
449	0\\
450	0\\
451	0\\
452	0\\
453	0\\
454	0\\
455	0\\
456	0\\
457	0\\
458	0\\
459	0\\
460	0\\
461	0\\
462	0\\
463	0\\
464	0\\
465	0\\
466	0\\
467	0\\
468	0\\
469	0\\
470	0\\
471	0\\
472	0\\
473	0\\
474	0\\
475	0\\
476	0\\
477	0\\
478	0\\
479	0\\
480	0\\
481	0\\
482	0\\
483	0\\
484	0\\
485	0\\
486	0\\
487	0\\
488	0\\
489	0\\
490	0\\
491	0\\
492	0\\
493	0\\
494	0\\
495	0\\
496	0\\
497	0\\
498	0\\
499	0\\
500	0\\
501	0\\
502	0\\
503	0\\
504	0\\
505	0\\
506	0\\
507	0\\
508	0\\
509	0\\
510	0\\
511	0\\
512	0\\
513	0\\
514	0\\
515	0\\
516	0\\
517	0\\
518	0\\
519	0\\
520	0\\
521	0\\
522	0\\
523	0\\
524	0\\
525	0\\
526	0\\
527	0\\
528	0\\
529	0\\
530	0\\
531	0\\
532	0\\
533	0\\
534	0\\
535	0\\
536	0\\
537	0\\
538	0\\
539	0\\
540	0\\
541	0\\
542	0\\
543	0\\
544	0\\
545	0\\
546	0\\
547	0\\
548	0\\
549	0\\
550	0\\
551	0\\
552	0\\
553	0\\
554	0\\
555	0\\
556	0\\
557	0\\
558	0\\
559	0\\
560	0\\
561	0\\
562	0\\
563	0\\
564	0\\
565	0\\
566	0\\
567	5.88949037576846e-05\\
568	0.000278923522351577\\
569	0.000465215789347868\\
570	0.000573095561531768\\
571	0.000679064755436726\\
572	0.000781407100120683\\
573	0.000879327214528265\\
574	0.000971216243397552\\
575	0.00105576846491719\\
576	0.00113450507723386\\
577	0.00121198502541605\\
578	0.00128801880280719\\
579	0.00136213507180008\\
580	0.00143518170906471\\
581	0.00150992377660234\\
582	0.00158312346660404\\
583	0.00165386180226987\\
584	0.00192128218169381\\
585	0.00239891244873237\\
586	0.00288945262252655\\
587	0.00302337868377814\\
588	0.00315170319741699\\
589	0.00326939440428176\\
590	0.00336308691806496\\
591	0.00345791823244592\\
592	0.00355546598078343\\
593	0.00366084952809565\\
594	0.00378520541414672\\
595	0.00395742119285503\\
596	0.00425297862742962\\
597	0.0048700418989439\\
598	0.00632942537858856\\
599	0\\
600	0\\
};
\addplot [color=mycolor18,solid,forget plot]
  table[row sep=crcr]{%
1	0\\
2	0\\
3	0\\
4	0\\
5	0\\
6	0\\
7	0\\
8	0\\
9	0\\
10	0\\
11	0\\
12	0\\
13	0\\
14	0\\
15	0\\
16	0\\
17	0\\
18	0\\
19	0\\
20	0\\
21	0\\
22	0\\
23	0\\
24	0\\
25	0\\
26	0\\
27	0\\
28	0\\
29	0\\
30	0\\
31	0\\
32	0\\
33	0\\
34	0\\
35	0\\
36	0\\
37	0\\
38	0\\
39	0\\
40	0\\
41	0\\
42	0\\
43	0\\
44	0\\
45	0\\
46	0\\
47	0\\
48	0\\
49	0\\
50	0\\
51	0\\
52	0\\
53	0\\
54	0\\
55	0\\
56	0\\
57	0\\
58	0\\
59	0\\
60	0\\
61	0\\
62	0\\
63	0\\
64	0\\
65	0\\
66	0\\
67	0\\
68	0\\
69	0\\
70	0\\
71	0\\
72	0\\
73	0\\
74	0\\
75	0\\
76	0\\
77	0\\
78	0\\
79	0\\
80	0\\
81	0\\
82	0\\
83	0\\
84	0\\
85	0\\
86	0\\
87	0\\
88	0\\
89	0\\
90	0\\
91	0\\
92	0\\
93	0\\
94	0\\
95	0\\
96	0\\
97	0\\
98	0\\
99	0\\
100	0\\
101	0\\
102	0\\
103	0\\
104	0\\
105	0\\
106	0\\
107	0\\
108	0\\
109	0\\
110	0\\
111	0\\
112	0\\
113	0\\
114	0\\
115	0\\
116	0\\
117	0\\
118	0\\
119	0\\
120	0\\
121	0\\
122	0\\
123	0\\
124	0\\
125	0\\
126	0\\
127	0\\
128	0\\
129	0\\
130	0\\
131	0\\
132	0\\
133	0\\
134	0\\
135	0\\
136	0\\
137	0\\
138	0\\
139	0\\
140	0\\
141	0\\
142	0\\
143	0\\
144	0\\
145	0\\
146	0\\
147	0\\
148	0\\
149	0\\
150	0\\
151	0\\
152	0\\
153	0\\
154	0\\
155	0\\
156	0\\
157	0\\
158	0\\
159	0\\
160	0\\
161	0\\
162	0\\
163	0\\
164	0\\
165	0\\
166	0\\
167	0\\
168	0\\
169	0\\
170	0\\
171	0\\
172	0\\
173	0\\
174	0\\
175	0\\
176	0\\
177	0\\
178	0\\
179	0\\
180	0\\
181	0\\
182	0\\
183	0\\
184	0\\
185	0\\
186	0\\
187	0\\
188	0\\
189	0\\
190	0\\
191	0\\
192	0\\
193	0\\
194	0\\
195	0\\
196	0\\
197	0\\
198	0\\
199	0\\
200	0\\
201	0\\
202	0\\
203	0\\
204	0\\
205	0\\
206	0\\
207	0\\
208	0\\
209	0\\
210	0\\
211	0\\
212	0\\
213	0\\
214	0\\
215	0\\
216	0\\
217	0\\
218	0\\
219	0\\
220	0\\
221	0\\
222	0\\
223	0\\
224	0\\
225	0\\
226	0\\
227	0\\
228	0\\
229	0\\
230	0\\
231	0\\
232	0\\
233	0\\
234	0\\
235	0\\
236	0\\
237	0\\
238	0\\
239	0\\
240	0\\
241	0\\
242	0\\
243	0\\
244	0\\
245	0\\
246	0\\
247	0\\
248	0\\
249	0\\
250	0\\
251	0\\
252	0\\
253	0\\
254	0\\
255	0\\
256	0\\
257	0\\
258	0\\
259	0\\
260	0\\
261	0\\
262	0\\
263	0\\
264	0\\
265	0\\
266	0\\
267	0\\
268	0\\
269	0\\
270	0\\
271	0\\
272	0\\
273	0\\
274	0\\
275	0\\
276	0\\
277	0\\
278	0\\
279	0\\
280	0\\
281	0\\
282	0\\
283	0\\
284	0\\
285	0\\
286	0\\
287	0\\
288	0\\
289	0\\
290	0\\
291	0\\
292	0\\
293	0\\
294	0\\
295	0\\
296	0\\
297	0\\
298	0\\
299	0\\
300	0\\
301	0\\
302	0\\
303	0\\
304	0\\
305	0\\
306	0\\
307	0\\
308	0\\
309	0\\
310	0\\
311	0\\
312	0\\
313	0\\
314	0\\
315	0\\
316	0\\
317	0\\
318	0\\
319	0\\
320	0\\
321	0\\
322	0\\
323	0\\
324	0\\
325	0\\
326	0\\
327	0\\
328	0\\
329	0\\
330	0\\
331	0\\
332	0\\
333	0\\
334	0\\
335	0\\
336	0\\
337	0\\
338	0\\
339	0\\
340	0\\
341	0\\
342	0\\
343	0\\
344	0\\
345	0\\
346	0\\
347	0\\
348	0\\
349	0\\
350	0\\
351	0\\
352	0\\
353	0\\
354	0\\
355	0\\
356	0\\
357	0\\
358	0\\
359	0\\
360	0\\
361	0\\
362	0\\
363	0\\
364	0\\
365	0\\
366	0\\
367	0\\
368	0\\
369	0\\
370	0\\
371	0\\
372	0\\
373	0\\
374	0\\
375	0\\
376	0\\
377	0\\
378	0\\
379	0\\
380	0\\
381	0\\
382	0\\
383	0\\
384	0\\
385	0\\
386	0\\
387	0\\
388	0\\
389	0\\
390	0\\
391	0\\
392	0\\
393	0\\
394	0\\
395	0\\
396	0\\
397	0\\
398	0\\
399	0\\
400	0\\
401	0\\
402	0\\
403	0\\
404	0\\
405	0\\
406	0\\
407	0\\
408	0\\
409	0\\
410	0\\
411	0\\
412	0\\
413	0\\
414	0\\
415	0\\
416	0\\
417	0\\
418	0\\
419	0\\
420	0\\
421	0\\
422	0\\
423	0\\
424	0\\
425	0\\
426	0\\
427	0\\
428	0\\
429	0\\
430	0\\
431	0\\
432	0\\
433	0\\
434	0\\
435	0\\
436	0\\
437	0\\
438	0\\
439	0\\
440	0\\
441	0\\
442	0\\
443	0\\
444	0\\
445	0\\
446	0\\
447	0\\
448	0\\
449	0\\
450	0\\
451	0\\
452	0\\
453	0\\
454	0\\
455	0\\
456	0\\
457	0\\
458	0\\
459	0\\
460	0\\
461	0\\
462	0\\
463	0\\
464	0\\
465	0\\
466	0\\
467	0\\
468	0\\
469	0\\
470	0\\
471	0\\
472	0\\
473	0\\
474	0\\
475	0\\
476	0\\
477	0\\
478	0\\
479	0\\
480	0\\
481	0\\
482	0\\
483	0\\
484	0\\
485	0\\
486	0\\
487	0\\
488	0\\
489	0\\
490	0\\
491	0\\
492	0\\
493	0\\
494	0\\
495	0\\
496	0\\
497	0\\
498	0\\
499	0\\
500	0\\
501	0\\
502	0\\
503	0\\
504	0\\
505	0\\
506	0\\
507	0\\
508	0\\
509	0\\
510	0\\
511	0\\
512	0\\
513	0\\
514	0\\
515	0\\
516	0\\
517	0\\
518	0\\
519	0\\
520	0\\
521	0\\
522	0\\
523	0\\
524	0\\
525	0\\
526	0\\
527	0\\
528	0\\
529	0\\
530	0\\
531	0\\
532	0\\
533	0\\
534	0\\
535	0\\
536	0\\
537	0\\
538	0\\
539	0\\
540	0\\
541	0\\
542	0\\
543	0\\
544	0\\
545	0\\
546	0\\
547	0\\
548	0\\
549	0\\
550	0\\
551	0\\
552	0\\
553	0\\
554	0\\
555	0\\
556	0\\
557	0\\
558	0\\
559	0\\
560	0\\
561	0\\
562	9.90403746872915e-05\\
563	0.000304968769744545\\
564	0.000402367207070976\\
565	0.000497931023480415\\
566	0.000590494868905439\\
567	0.000678077044756074\\
568	0.000758685659673696\\
569	0.000830189121630715\\
570	0.000895650850952982\\
571	0.000961984822474034\\
572	0.0010278398051382\\
573	0.00109225696560847\\
574	0.00115548679208059\\
575	0.00121672171450071\\
576	0.00127705683954306\\
577	0.00133948023067548\\
578	0.00140501315387362\\
579	0.00146934333816017\\
580	0.00153055864356594\\
581	0.00158857538063284\\
582	0.00197419871052425\\
583	0.00246466618649465\\
584	0.00277686403795953\\
585	0.00289976434281973\\
586	0.00301736053854785\\
587	0.00310424045129586\\
588	0.00319120167814864\\
589	0.00327881165853131\\
590	0.00336825554262852\\
591	0.00346025420288071\\
592	0.00355649290755294\\
593	0.00366118457402185\\
594	0.00378527046594782\\
595	0.00395742119285503\\
596	0.00425297862742962\\
597	0.0048700418989439\\
598	0.00632942537858856\\
599	0\\
600	0\\
};
\addplot [color=red!25!mycolor17,solid,forget plot]
  table[row sep=crcr]{%
1	0\\
2	0\\
3	0\\
4	0\\
5	0\\
6	0\\
7	0\\
8	0\\
9	0\\
10	0\\
11	0\\
12	0\\
13	0\\
14	0\\
15	0\\
16	0\\
17	0\\
18	0\\
19	0\\
20	0\\
21	0\\
22	0\\
23	0\\
24	0\\
25	0\\
26	0\\
27	0\\
28	0\\
29	0\\
30	0\\
31	0\\
32	0\\
33	0\\
34	0\\
35	0\\
36	0\\
37	0\\
38	0\\
39	0\\
40	0\\
41	0\\
42	0\\
43	0\\
44	0\\
45	0\\
46	0\\
47	0\\
48	0\\
49	0\\
50	0\\
51	0\\
52	0\\
53	0\\
54	0\\
55	0\\
56	0\\
57	0\\
58	0\\
59	0\\
60	0\\
61	0\\
62	0\\
63	0\\
64	0\\
65	0\\
66	0\\
67	0\\
68	0\\
69	0\\
70	0\\
71	0\\
72	0\\
73	0\\
74	0\\
75	0\\
76	0\\
77	0\\
78	0\\
79	0\\
80	0\\
81	0\\
82	0\\
83	0\\
84	0\\
85	0\\
86	0\\
87	0\\
88	0\\
89	0\\
90	0\\
91	0\\
92	0\\
93	0\\
94	0\\
95	0\\
96	0\\
97	0\\
98	0\\
99	0\\
100	0\\
101	0\\
102	0\\
103	0\\
104	0\\
105	0\\
106	0\\
107	0\\
108	0\\
109	0\\
110	0\\
111	0\\
112	0\\
113	0\\
114	0\\
115	0\\
116	0\\
117	0\\
118	0\\
119	0\\
120	0\\
121	0\\
122	0\\
123	0\\
124	0\\
125	0\\
126	0\\
127	0\\
128	0\\
129	0\\
130	0\\
131	0\\
132	0\\
133	0\\
134	0\\
135	0\\
136	0\\
137	0\\
138	0\\
139	0\\
140	0\\
141	0\\
142	0\\
143	0\\
144	0\\
145	0\\
146	0\\
147	0\\
148	0\\
149	0\\
150	0\\
151	0\\
152	0\\
153	0\\
154	0\\
155	0\\
156	0\\
157	0\\
158	0\\
159	0\\
160	0\\
161	0\\
162	0\\
163	0\\
164	0\\
165	0\\
166	0\\
167	0\\
168	0\\
169	0\\
170	0\\
171	0\\
172	0\\
173	0\\
174	0\\
175	0\\
176	0\\
177	0\\
178	0\\
179	0\\
180	0\\
181	0\\
182	0\\
183	0\\
184	0\\
185	0\\
186	0\\
187	0\\
188	0\\
189	0\\
190	0\\
191	0\\
192	0\\
193	0\\
194	0\\
195	0\\
196	0\\
197	0\\
198	0\\
199	0\\
200	0\\
201	0\\
202	0\\
203	0\\
204	0\\
205	0\\
206	0\\
207	0\\
208	0\\
209	0\\
210	0\\
211	0\\
212	0\\
213	0\\
214	0\\
215	0\\
216	0\\
217	0\\
218	0\\
219	0\\
220	0\\
221	0\\
222	0\\
223	0\\
224	0\\
225	0\\
226	0\\
227	0\\
228	0\\
229	0\\
230	0\\
231	0\\
232	0\\
233	0\\
234	0\\
235	0\\
236	0\\
237	0\\
238	0\\
239	0\\
240	0\\
241	0\\
242	0\\
243	0\\
244	0\\
245	0\\
246	0\\
247	0\\
248	0\\
249	0\\
250	0\\
251	0\\
252	0\\
253	0\\
254	0\\
255	0\\
256	0\\
257	0\\
258	0\\
259	0\\
260	0\\
261	0\\
262	0\\
263	0\\
264	0\\
265	0\\
266	0\\
267	0\\
268	0\\
269	0\\
270	0\\
271	0\\
272	0\\
273	0\\
274	0\\
275	0\\
276	0\\
277	0\\
278	0\\
279	0\\
280	0\\
281	0\\
282	0\\
283	0\\
284	0\\
285	0\\
286	0\\
287	0\\
288	0\\
289	0\\
290	0\\
291	0\\
292	0\\
293	0\\
294	0\\
295	0\\
296	0\\
297	0\\
298	0\\
299	0\\
300	0\\
301	0\\
302	0\\
303	0\\
304	0\\
305	0\\
306	0\\
307	0\\
308	0\\
309	0\\
310	0\\
311	0\\
312	0\\
313	0\\
314	0\\
315	0\\
316	0\\
317	0\\
318	0\\
319	0\\
320	0\\
321	0\\
322	0\\
323	0\\
324	0\\
325	0\\
326	0\\
327	0\\
328	0\\
329	0\\
330	0\\
331	0\\
332	0\\
333	0\\
334	0\\
335	0\\
336	0\\
337	0\\
338	0\\
339	0\\
340	0\\
341	0\\
342	0\\
343	0\\
344	0\\
345	0\\
346	0\\
347	0\\
348	0\\
349	0\\
350	0\\
351	0\\
352	0\\
353	0\\
354	0\\
355	0\\
356	0\\
357	0\\
358	0\\
359	0\\
360	0\\
361	0\\
362	0\\
363	0\\
364	0\\
365	0\\
366	0\\
367	0\\
368	0\\
369	0\\
370	0\\
371	0\\
372	0\\
373	0\\
374	0\\
375	0\\
376	0\\
377	0\\
378	0\\
379	0\\
380	0\\
381	0\\
382	0\\
383	0\\
384	0\\
385	0\\
386	0\\
387	0\\
388	0\\
389	0\\
390	0\\
391	0\\
392	0\\
393	0\\
394	0\\
395	0\\
396	0\\
397	0\\
398	0\\
399	0\\
400	0\\
401	0\\
402	0\\
403	0\\
404	0\\
405	0\\
406	0\\
407	0\\
408	0\\
409	0\\
410	0\\
411	0\\
412	0\\
413	0\\
414	0\\
415	0\\
416	0\\
417	0\\
418	0\\
419	0\\
420	0\\
421	0\\
422	0\\
423	0\\
424	0\\
425	0\\
426	0\\
427	0\\
428	0\\
429	0\\
430	0\\
431	0\\
432	0\\
433	0\\
434	0\\
435	0\\
436	0\\
437	0\\
438	0\\
439	0\\
440	0\\
441	0\\
442	0\\
443	0\\
444	0\\
445	0\\
446	0\\
447	0\\
448	0\\
449	0\\
450	0\\
451	0\\
452	0\\
453	0\\
454	0\\
455	0\\
456	0\\
457	0\\
458	0\\
459	0\\
460	0\\
461	0\\
462	0\\
463	0\\
464	0\\
465	0\\
466	0\\
467	0\\
468	0\\
469	0\\
470	0\\
471	0\\
472	0\\
473	0\\
474	0\\
475	0\\
476	0\\
477	0\\
478	0\\
479	0\\
480	0\\
481	0\\
482	0\\
483	0\\
484	0\\
485	0\\
486	0\\
487	0\\
488	0\\
489	0\\
490	0\\
491	0\\
492	0\\
493	0\\
494	0\\
495	0\\
496	0\\
497	0\\
498	0\\
499	0\\
500	0\\
501	0\\
502	0\\
503	0\\
504	0\\
505	0\\
506	0\\
507	0\\
508	0\\
509	0\\
510	0\\
511	0\\
512	0\\
513	0\\
514	0\\
515	0\\
516	0\\
517	0\\
518	0\\
519	0\\
520	0\\
521	0\\
522	0\\
523	0\\
524	0\\
525	0\\
526	0\\
527	0\\
528	0\\
529	0\\
530	0\\
531	0\\
532	0\\
533	0\\
534	0\\
535	0\\
536	0\\
537	0\\
538	0\\
539	0\\
540	0\\
541	0\\
542	0\\
543	0\\
544	0\\
545	0\\
546	0\\
547	0\\
548	0\\
549	0\\
550	0\\
551	0\\
552	0\\
553	0\\
554	0\\
555	0\\
556	0\\
557	8.01734108211468e-05\\
558	0.000223371265806801\\
559	0.000311803468712962\\
560	0.000397274215465461\\
561	0.000478427144538355\\
562	0.000553370163734247\\
563	0.0006186795095291\\
564	0.000675713681809035\\
565	0.00073191039909959\\
566	0.000787276194635121\\
567	0.000843166680497421\\
568	0.000899372565017849\\
569	0.000954897785455491\\
570	0.00101018526904893\\
571	0.0010637562962866\\
572	0.00111682448780514\\
573	0.00117048000700388\\
574	0.00122509244478267\\
575	0.00128514626774489\\
576	0.00134430788764564\\
577	0.00140097157268809\\
578	0.00145403611431188\\
579	0.00150752687252756\\
580	0.00193194802301185\\
581	0.00243680986819744\\
582	0.00264404234620892\\
583	0.00276021838521539\\
584	0.00285857309192273\\
585	0.00294082185579374\\
586	0.00302342706147123\\
587	0.00310731224695134\\
588	0.00319261476955509\\
589	0.00327956317486525\\
590	0.00336860380042371\\
591	0.00346039193281902\\
592	0.00355653331527142\\
593	0.00366119131364763\\
594	0.00378527046594782\\
595	0.00395742119285503\\
596	0.00425297862742962\\
597	0.0048700418989439\\
598	0.00632942537858856\\
599	0\\
600	0\\
};
\addplot [color=mycolor19,solid,forget plot]
  table[row sep=crcr]{%
1	0\\
2	0\\
3	0\\
4	0\\
5	0\\
6	0\\
7	0\\
8	0\\
9	0\\
10	0\\
11	0\\
12	0\\
13	0\\
14	0\\
15	0\\
16	0\\
17	0\\
18	0\\
19	0\\
20	0\\
21	0\\
22	0\\
23	0\\
24	0\\
25	0\\
26	0\\
27	0\\
28	0\\
29	0\\
30	0\\
31	0\\
32	0\\
33	0\\
34	0\\
35	0\\
36	0\\
37	0\\
38	0\\
39	0\\
40	0\\
41	0\\
42	0\\
43	0\\
44	0\\
45	0\\
46	0\\
47	0\\
48	0\\
49	0\\
50	0\\
51	0\\
52	0\\
53	0\\
54	0\\
55	0\\
56	0\\
57	0\\
58	0\\
59	0\\
60	0\\
61	0\\
62	0\\
63	0\\
64	0\\
65	0\\
66	0\\
67	0\\
68	0\\
69	0\\
70	0\\
71	0\\
72	0\\
73	0\\
74	0\\
75	0\\
76	0\\
77	0\\
78	0\\
79	0\\
80	0\\
81	0\\
82	0\\
83	0\\
84	0\\
85	0\\
86	0\\
87	0\\
88	0\\
89	0\\
90	0\\
91	0\\
92	0\\
93	0\\
94	0\\
95	0\\
96	0\\
97	0\\
98	0\\
99	0\\
100	0\\
101	0\\
102	0\\
103	0\\
104	0\\
105	0\\
106	0\\
107	0\\
108	0\\
109	0\\
110	0\\
111	0\\
112	0\\
113	0\\
114	0\\
115	0\\
116	0\\
117	0\\
118	0\\
119	0\\
120	0\\
121	0\\
122	0\\
123	0\\
124	0\\
125	0\\
126	0\\
127	0\\
128	0\\
129	0\\
130	0\\
131	0\\
132	0\\
133	0\\
134	0\\
135	0\\
136	0\\
137	0\\
138	0\\
139	0\\
140	0\\
141	0\\
142	0\\
143	0\\
144	0\\
145	0\\
146	0\\
147	0\\
148	0\\
149	0\\
150	0\\
151	0\\
152	0\\
153	0\\
154	0\\
155	0\\
156	0\\
157	0\\
158	0\\
159	0\\
160	0\\
161	0\\
162	0\\
163	0\\
164	0\\
165	0\\
166	0\\
167	0\\
168	0\\
169	0\\
170	0\\
171	0\\
172	0\\
173	0\\
174	0\\
175	0\\
176	0\\
177	0\\
178	0\\
179	0\\
180	0\\
181	0\\
182	0\\
183	0\\
184	0\\
185	0\\
186	0\\
187	0\\
188	0\\
189	0\\
190	0\\
191	0\\
192	0\\
193	0\\
194	0\\
195	0\\
196	0\\
197	0\\
198	0\\
199	0\\
200	0\\
201	0\\
202	0\\
203	0\\
204	0\\
205	0\\
206	0\\
207	0\\
208	0\\
209	0\\
210	0\\
211	0\\
212	0\\
213	0\\
214	0\\
215	0\\
216	0\\
217	0\\
218	0\\
219	0\\
220	0\\
221	0\\
222	0\\
223	0\\
224	0\\
225	0\\
226	0\\
227	0\\
228	0\\
229	0\\
230	0\\
231	0\\
232	0\\
233	0\\
234	0\\
235	0\\
236	0\\
237	0\\
238	0\\
239	0\\
240	0\\
241	0\\
242	0\\
243	0\\
244	0\\
245	0\\
246	0\\
247	0\\
248	0\\
249	0\\
250	0\\
251	0\\
252	0\\
253	0\\
254	0\\
255	0\\
256	0\\
257	0\\
258	0\\
259	0\\
260	0\\
261	0\\
262	0\\
263	0\\
264	0\\
265	0\\
266	0\\
267	0\\
268	0\\
269	0\\
270	0\\
271	0\\
272	0\\
273	0\\
274	0\\
275	0\\
276	0\\
277	0\\
278	0\\
279	0\\
280	0\\
281	0\\
282	0\\
283	0\\
284	0\\
285	0\\
286	0\\
287	0\\
288	0\\
289	0\\
290	0\\
291	0\\
292	0\\
293	0\\
294	0\\
295	0\\
296	0\\
297	0\\
298	0\\
299	0\\
300	0\\
301	0\\
302	0\\
303	0\\
304	0\\
305	0\\
306	0\\
307	0\\
308	0\\
309	0\\
310	0\\
311	0\\
312	0\\
313	0\\
314	0\\
315	0\\
316	0\\
317	0\\
318	0\\
319	0\\
320	0\\
321	0\\
322	0\\
323	0\\
324	0\\
325	0\\
326	0\\
327	0\\
328	0\\
329	0\\
330	0\\
331	0\\
332	0\\
333	0\\
334	0\\
335	0\\
336	0\\
337	0\\
338	0\\
339	0\\
340	0\\
341	0\\
342	0\\
343	0\\
344	0\\
345	0\\
346	0\\
347	0\\
348	0\\
349	0\\
350	0\\
351	0\\
352	0\\
353	0\\
354	0\\
355	0\\
356	0\\
357	0\\
358	0\\
359	0\\
360	0\\
361	0\\
362	0\\
363	0\\
364	0\\
365	0\\
366	0\\
367	0\\
368	0\\
369	0\\
370	0\\
371	0\\
372	0\\
373	0\\
374	0\\
375	0\\
376	0\\
377	0\\
378	0\\
379	0\\
380	0\\
381	0\\
382	0\\
383	0\\
384	0\\
385	0\\
386	0\\
387	0\\
388	0\\
389	0\\
390	0\\
391	0\\
392	0\\
393	0\\
394	0\\
395	0\\
396	0\\
397	0\\
398	0\\
399	0\\
400	0\\
401	0\\
402	0\\
403	0\\
404	0\\
405	0\\
406	0\\
407	0\\
408	0\\
409	0\\
410	0\\
411	0\\
412	0\\
413	0\\
414	0\\
415	0\\
416	0\\
417	0\\
418	0\\
419	0\\
420	0\\
421	0\\
422	0\\
423	0\\
424	0\\
425	0\\
426	0\\
427	0\\
428	0\\
429	0\\
430	0\\
431	0\\
432	0\\
433	0\\
434	0\\
435	0\\
436	0\\
437	0\\
438	0\\
439	0\\
440	0\\
441	0\\
442	0\\
443	0\\
444	0\\
445	0\\
446	0\\
447	0\\
448	0\\
449	0\\
450	0\\
451	0\\
452	0\\
453	0\\
454	0\\
455	0\\
456	0\\
457	0\\
458	0\\
459	0\\
460	0\\
461	0\\
462	0\\
463	0\\
464	0\\
465	0\\
466	0\\
467	0\\
468	0\\
469	0\\
470	0\\
471	0\\
472	0\\
473	0\\
474	0\\
475	0\\
476	0\\
477	0\\
478	0\\
479	0\\
480	0\\
481	0\\
482	0\\
483	0\\
484	0\\
485	0\\
486	0\\
487	0\\
488	0\\
489	0\\
490	0\\
491	0\\
492	0\\
493	0\\
494	0\\
495	0\\
496	0\\
497	0\\
498	0\\
499	0\\
500	0\\
501	0\\
502	0\\
503	0\\
504	0\\
505	0\\
506	0\\
507	0\\
508	0\\
509	0\\
510	0\\
511	0\\
512	0\\
513	0\\
514	0\\
515	0\\
516	0\\
517	0\\
518	0\\
519	0\\
520	0\\
521	0\\
522	0\\
523	0\\
524	0\\
525	0\\
526	0\\
527	0\\
528	0\\
529	0\\
530	0\\
531	0\\
532	0\\
533	0\\
534	0\\
535	0\\
536	0\\
537	0\\
538	0\\
539	0\\
540	0\\
541	0\\
542	0\\
543	0\\
544	0\\
545	0\\
546	0\\
547	0\\
548	0\\
549	0\\
550	0\\
551	0\\
552	1.50670897009176e-05\\
553	0.000124382787751889\\
554	0.00020503601022344\\
555	0.000281906524719583\\
556	0.000353475266029824\\
557	0.000416928686445217\\
558	0.000471622760274698\\
559	0.000521843264876268\\
560	0.000570759546781793\\
561	0.000618739159396117\\
562	0.000666121145496063\\
563	0.000713768966029745\\
564	0.000763815747326926\\
565	0.0008137163277131\\
566	0.000863511816317939\\
567	0.000912037857953922\\
568	0.000959785906419871\\
569	0.00100821347014938\\
570	0.00105737215001796\\
571	0.00110730055108194\\
572	0.00116093038778452\\
573	0.00121697978948583\\
574	0.00127198613892506\\
575	0.00132150660081605\\
576	0.00137146061891735\\
577	0.00142187466526321\\
578	0.00180758151954677\\
579	0.00232459949440059\\
580	0.00250414369719064\\
581	0.00261607079037986\\
582	0.00270221271091265\\
583	0.00278105910457229\\
584	0.00286069531223358\\
585	0.00294160817599962\\
586	0.00302387110985324\\
587	0.00310754666812213\\
588	0.00319273161070454\\
589	0.00327961396220161\\
590	0.00336862207624017\\
591	0.00346039677087353\\
592	0.00355653402929194\\
593	0.00366119131364763\\
594	0.00378527046594783\\
595	0.00395742119285503\\
596	0.00425297862742962\\
597	0.0048700418989439\\
598	0.00632942537858856\\
599	0\\
600	0\\
};
\addplot [color=red!50!mycolor17,solid,forget plot]
  table[row sep=crcr]{%
1	0\\
2	0\\
3	0\\
4	0\\
5	0\\
6	0\\
7	0\\
8	0\\
9	0\\
10	0\\
11	0\\
12	0\\
13	0\\
14	0\\
15	0\\
16	0\\
17	0\\
18	0\\
19	0\\
20	0\\
21	0\\
22	0\\
23	0\\
24	0\\
25	0\\
26	0\\
27	0\\
28	0\\
29	0\\
30	0\\
31	0\\
32	0\\
33	0\\
34	0\\
35	0\\
36	0\\
37	0\\
38	0\\
39	0\\
40	0\\
41	0\\
42	0\\
43	0\\
44	0\\
45	0\\
46	0\\
47	0\\
48	0\\
49	0\\
50	0\\
51	0\\
52	0\\
53	0\\
54	0\\
55	0\\
56	0\\
57	0\\
58	0\\
59	0\\
60	0\\
61	0\\
62	0\\
63	0\\
64	0\\
65	0\\
66	0\\
67	0\\
68	0\\
69	0\\
70	0\\
71	0\\
72	0\\
73	0\\
74	0\\
75	0\\
76	0\\
77	0\\
78	0\\
79	0\\
80	0\\
81	0\\
82	0\\
83	0\\
84	0\\
85	0\\
86	0\\
87	0\\
88	0\\
89	0\\
90	0\\
91	0\\
92	0\\
93	0\\
94	0\\
95	0\\
96	0\\
97	0\\
98	0\\
99	0\\
100	0\\
101	0\\
102	0\\
103	0\\
104	0\\
105	0\\
106	0\\
107	0\\
108	0\\
109	0\\
110	0\\
111	0\\
112	0\\
113	0\\
114	0\\
115	0\\
116	0\\
117	0\\
118	0\\
119	0\\
120	0\\
121	0\\
122	0\\
123	0\\
124	0\\
125	0\\
126	0\\
127	0\\
128	0\\
129	0\\
130	0\\
131	0\\
132	0\\
133	0\\
134	0\\
135	0\\
136	0\\
137	0\\
138	0\\
139	0\\
140	0\\
141	0\\
142	0\\
143	0\\
144	0\\
145	0\\
146	0\\
147	0\\
148	0\\
149	0\\
150	0\\
151	0\\
152	0\\
153	0\\
154	0\\
155	0\\
156	0\\
157	0\\
158	0\\
159	0\\
160	0\\
161	0\\
162	0\\
163	0\\
164	0\\
165	0\\
166	0\\
167	0\\
168	0\\
169	0\\
170	0\\
171	0\\
172	0\\
173	0\\
174	0\\
175	0\\
176	0\\
177	0\\
178	0\\
179	0\\
180	0\\
181	0\\
182	0\\
183	0\\
184	0\\
185	0\\
186	0\\
187	0\\
188	0\\
189	0\\
190	0\\
191	0\\
192	0\\
193	0\\
194	0\\
195	0\\
196	0\\
197	0\\
198	0\\
199	0\\
200	0\\
201	0\\
202	0\\
203	0\\
204	0\\
205	0\\
206	0\\
207	0\\
208	0\\
209	0\\
210	0\\
211	0\\
212	0\\
213	0\\
214	0\\
215	0\\
216	0\\
217	0\\
218	0\\
219	0\\
220	0\\
221	0\\
222	0\\
223	0\\
224	0\\
225	0\\
226	0\\
227	0\\
228	0\\
229	0\\
230	0\\
231	0\\
232	0\\
233	0\\
234	0\\
235	0\\
236	0\\
237	0\\
238	0\\
239	0\\
240	0\\
241	0\\
242	0\\
243	0\\
244	0\\
245	0\\
246	0\\
247	0\\
248	0\\
249	0\\
250	0\\
251	0\\
252	0\\
253	0\\
254	0\\
255	0\\
256	0\\
257	0\\
258	0\\
259	0\\
260	0\\
261	0\\
262	0\\
263	0\\
264	0\\
265	0\\
266	0\\
267	0\\
268	0\\
269	0\\
270	0\\
271	0\\
272	0\\
273	0\\
274	0\\
275	0\\
276	0\\
277	0\\
278	0\\
279	0\\
280	0\\
281	0\\
282	0\\
283	0\\
284	0\\
285	0\\
286	0\\
287	0\\
288	0\\
289	0\\
290	0\\
291	0\\
292	0\\
293	0\\
294	0\\
295	0\\
296	0\\
297	0\\
298	0\\
299	0\\
300	0\\
301	0\\
302	0\\
303	0\\
304	0\\
305	0\\
306	0\\
307	0\\
308	0\\
309	0\\
310	0\\
311	0\\
312	0\\
313	0\\
314	0\\
315	0\\
316	0\\
317	0\\
318	0\\
319	0\\
320	0\\
321	0\\
322	0\\
323	0\\
324	0\\
325	0\\
326	0\\
327	0\\
328	0\\
329	0\\
330	0\\
331	0\\
332	0\\
333	0\\
334	0\\
335	0\\
336	0\\
337	0\\
338	0\\
339	0\\
340	0\\
341	0\\
342	0\\
343	0\\
344	0\\
345	0\\
346	0\\
347	0\\
348	0\\
349	0\\
350	0\\
351	0\\
352	0\\
353	0\\
354	0\\
355	0\\
356	0\\
357	0\\
358	0\\
359	0\\
360	0\\
361	0\\
362	0\\
363	0\\
364	0\\
365	0\\
366	0\\
367	0\\
368	0\\
369	0\\
370	0\\
371	0\\
372	0\\
373	0\\
374	0\\
375	0\\
376	0\\
377	0\\
378	0\\
379	0\\
380	0\\
381	0\\
382	0\\
383	0\\
384	0\\
385	0\\
386	0\\
387	0\\
388	0\\
389	0\\
390	0\\
391	0\\
392	0\\
393	0\\
394	0\\
395	0\\
396	0\\
397	0\\
398	0\\
399	0\\
400	0\\
401	0\\
402	0\\
403	0\\
404	0\\
405	0\\
406	0\\
407	0\\
408	0\\
409	0\\
410	0\\
411	0\\
412	0\\
413	0\\
414	0\\
415	0\\
416	0\\
417	0\\
418	0\\
419	0\\
420	0\\
421	0\\
422	0\\
423	0\\
424	0\\
425	0\\
426	0\\
427	0\\
428	0\\
429	0\\
430	0\\
431	0\\
432	0\\
433	0\\
434	0\\
435	0\\
436	0\\
437	0\\
438	0\\
439	0\\
440	0\\
441	0\\
442	0\\
443	0\\
444	0\\
445	0\\
446	0\\
447	0\\
448	0\\
449	0\\
450	0\\
451	0\\
452	0\\
453	0\\
454	0\\
455	0\\
456	0\\
457	0\\
458	0\\
459	0\\
460	0\\
461	0\\
462	0\\
463	0\\
464	0\\
465	0\\
466	0\\
467	0\\
468	0\\
469	0\\
470	0\\
471	0\\
472	0\\
473	0\\
474	0\\
475	0\\
476	0\\
477	0\\
478	0\\
479	0\\
480	0\\
481	0\\
482	0\\
483	0\\
484	0\\
485	0\\
486	0\\
487	0\\
488	0\\
489	0\\
490	0\\
491	0\\
492	0\\
493	0\\
494	0\\
495	0\\
496	0\\
497	0\\
498	0\\
499	0\\
500	0\\
501	0\\
502	0\\
503	0\\
504	0\\
505	0\\
506	0\\
507	0\\
508	0\\
509	0\\
510	0\\
511	0\\
512	0\\
513	0\\
514	0\\
515	0\\
516	0\\
517	0\\
518	0\\
519	0\\
520	0\\
521	0\\
522	0\\
523	0\\
524	0\\
525	0\\
526	0\\
527	0\\
528	0\\
529	0\\
530	0\\
531	0\\
532	0\\
533	0\\
534	0\\
535	0\\
536	0\\
537	0\\
538	0\\
539	0\\
540	0\\
541	0\\
542	0\\
543	0\\
544	0\\
545	0\\
546	0\\
547	0\\
548	1.64327239660599e-05\\
549	9.08311762122886e-05\\
550	0.000161022938856363\\
551	0.000224806955478363\\
552	0.000280485308906282\\
553	0.000327622000208862\\
554	0.000372012735657423\\
555	0.000415398793616001\\
556	0.000457922393037952\\
557	0.000499893593165752\\
558	0.000541753390275113\\
559	0.000584113654120537\\
560	0.000628623132028616\\
561	0.000674257201963313\\
562	0.000719918679057545\\
563	0.000765234608134834\\
564	0.000808741945395788\\
565	0.000852912506720194\\
566	0.00089779277058568\\
567	0.000943386792557535\\
568	0.000989813512646882\\
569	0.00103725286168404\\
570	0.00109022682293433\\
571	0.00114341698775839\\
572	0.00119302706840499\\
573	0.00124014217818203\\
574	0.00128770080075513\\
575	0.00133565715457315\\
576	0.00160918154831211\\
577	0.00213608599489938\\
578	0.00235964274414307\\
579	0.00246903551743822\\
580	0.00255013683265288\\
581	0.00262610814362027\\
582	0.00270308299221297\\
583	0.00278130768962817\\
584	0.00286082862410346\\
585	0.00294168211648289\\
586	0.00302390908465139\\
587	0.003107564358473\\
588	0.00319273875713431\\
589	0.00327961632359304\\
590	0.00336862264580875\\
591	0.00346039684658503\\
592	0.00355653402929193\\
593	0.00366119131364763\\
594	0.00378527046594782\\
595	0.00395742119285503\\
596	0.00425297862742962\\
597	0.0048700418989439\\
598	0.00632942537858856\\
599	0\\
600	0\\
};
\addplot [color=red!40!mycolor19,solid,forget plot]
  table[row sep=crcr]{%
1	0\\
2	0\\
3	0\\
4	0\\
5	0\\
6	0\\
7	0\\
8	0\\
9	0\\
10	0\\
11	0\\
12	0\\
13	0\\
14	0\\
15	0\\
16	0\\
17	0\\
18	0\\
19	0\\
20	0\\
21	0\\
22	0\\
23	0\\
24	0\\
25	0\\
26	0\\
27	0\\
28	0\\
29	0\\
30	0\\
31	0\\
32	0\\
33	0\\
34	0\\
35	0\\
36	0\\
37	0\\
38	0\\
39	0\\
40	0\\
41	0\\
42	0\\
43	0\\
44	0\\
45	0\\
46	0\\
47	0\\
48	0\\
49	0\\
50	0\\
51	0\\
52	0\\
53	0\\
54	0\\
55	0\\
56	0\\
57	0\\
58	0\\
59	0\\
60	0\\
61	0\\
62	0\\
63	0\\
64	0\\
65	0\\
66	0\\
67	0\\
68	0\\
69	0\\
70	0\\
71	0\\
72	0\\
73	0\\
74	0\\
75	0\\
76	0\\
77	0\\
78	0\\
79	0\\
80	0\\
81	0\\
82	0\\
83	0\\
84	0\\
85	0\\
86	0\\
87	0\\
88	0\\
89	0\\
90	0\\
91	0\\
92	0\\
93	0\\
94	0\\
95	0\\
96	0\\
97	0\\
98	0\\
99	0\\
100	0\\
101	0\\
102	0\\
103	0\\
104	0\\
105	0\\
106	0\\
107	0\\
108	0\\
109	0\\
110	0\\
111	0\\
112	0\\
113	0\\
114	0\\
115	0\\
116	0\\
117	0\\
118	0\\
119	0\\
120	0\\
121	0\\
122	0\\
123	0\\
124	0\\
125	0\\
126	0\\
127	0\\
128	0\\
129	0\\
130	0\\
131	0\\
132	0\\
133	0\\
134	0\\
135	0\\
136	0\\
137	0\\
138	0\\
139	0\\
140	0\\
141	0\\
142	0\\
143	0\\
144	0\\
145	0\\
146	0\\
147	0\\
148	0\\
149	0\\
150	0\\
151	0\\
152	0\\
153	0\\
154	0\\
155	0\\
156	0\\
157	0\\
158	0\\
159	0\\
160	0\\
161	0\\
162	0\\
163	0\\
164	0\\
165	0\\
166	0\\
167	0\\
168	0\\
169	0\\
170	0\\
171	0\\
172	0\\
173	0\\
174	0\\
175	0\\
176	0\\
177	0\\
178	0\\
179	0\\
180	0\\
181	0\\
182	0\\
183	0\\
184	0\\
185	0\\
186	0\\
187	0\\
188	0\\
189	0\\
190	0\\
191	0\\
192	0\\
193	0\\
194	0\\
195	0\\
196	0\\
197	0\\
198	0\\
199	0\\
200	0\\
201	0\\
202	0\\
203	0\\
204	0\\
205	0\\
206	0\\
207	0\\
208	0\\
209	0\\
210	0\\
211	0\\
212	0\\
213	0\\
214	0\\
215	0\\
216	0\\
217	0\\
218	0\\
219	0\\
220	0\\
221	0\\
222	0\\
223	0\\
224	0\\
225	0\\
226	0\\
227	0\\
228	0\\
229	0\\
230	0\\
231	0\\
232	0\\
233	0\\
234	0\\
235	0\\
236	0\\
237	0\\
238	0\\
239	0\\
240	0\\
241	0\\
242	0\\
243	0\\
244	0\\
245	0\\
246	0\\
247	0\\
248	0\\
249	0\\
250	0\\
251	0\\
252	0\\
253	0\\
254	0\\
255	0\\
256	0\\
257	0\\
258	0\\
259	0\\
260	0\\
261	0\\
262	0\\
263	0\\
264	0\\
265	0\\
266	0\\
267	0\\
268	0\\
269	0\\
270	0\\
271	0\\
272	0\\
273	0\\
274	0\\
275	0\\
276	0\\
277	0\\
278	0\\
279	0\\
280	0\\
281	0\\
282	0\\
283	0\\
284	0\\
285	0\\
286	0\\
287	0\\
288	0\\
289	0\\
290	0\\
291	0\\
292	0\\
293	0\\
294	0\\
295	0\\
296	0\\
297	0\\
298	0\\
299	0\\
300	0\\
301	0\\
302	0\\
303	0\\
304	0\\
305	0\\
306	0\\
307	0\\
308	0\\
309	0\\
310	0\\
311	0\\
312	0\\
313	0\\
314	0\\
315	0\\
316	0\\
317	0\\
318	0\\
319	0\\
320	0\\
321	0\\
322	0\\
323	0\\
324	0\\
325	0\\
326	0\\
327	0\\
328	0\\
329	0\\
330	0\\
331	0\\
332	0\\
333	0\\
334	0\\
335	0\\
336	0\\
337	0\\
338	0\\
339	0\\
340	0\\
341	0\\
342	0\\
343	0\\
344	0\\
345	0\\
346	0\\
347	0\\
348	0\\
349	0\\
350	0\\
351	0\\
352	0\\
353	0\\
354	0\\
355	0\\
356	0\\
357	0\\
358	0\\
359	0\\
360	0\\
361	0\\
362	0\\
363	0\\
364	0\\
365	0\\
366	0\\
367	0\\
368	0\\
369	0\\
370	0\\
371	0\\
372	0\\
373	0\\
374	0\\
375	0\\
376	0\\
377	0\\
378	0\\
379	0\\
380	0\\
381	0\\
382	0\\
383	0\\
384	0\\
385	0\\
386	0\\
387	0\\
388	0\\
389	0\\
390	0\\
391	0\\
392	0\\
393	0\\
394	0\\
395	0\\
396	0\\
397	0\\
398	0\\
399	0\\
400	0\\
401	0\\
402	0\\
403	0\\
404	0\\
405	0\\
406	0\\
407	0\\
408	0\\
409	0\\
410	0\\
411	0\\
412	0\\
413	0\\
414	0\\
415	0\\
416	0\\
417	0\\
418	0\\
419	0\\
420	0\\
421	0\\
422	0\\
423	0\\
424	0\\
425	0\\
426	0\\
427	0\\
428	0\\
429	0\\
430	0\\
431	0\\
432	0\\
433	0\\
434	0\\
435	0\\
436	0\\
437	0\\
438	0\\
439	0\\
440	0\\
441	0\\
442	0\\
443	0\\
444	0\\
445	0\\
446	0\\
447	0\\
448	0\\
449	0\\
450	0\\
451	0\\
452	0\\
453	0\\
454	0\\
455	0\\
456	0\\
457	0\\
458	0\\
459	0\\
460	0\\
461	0\\
462	0\\
463	0\\
464	0\\
465	0\\
466	0\\
467	0\\
468	0\\
469	0\\
470	0\\
471	0\\
472	0\\
473	0\\
474	0\\
475	0\\
476	0\\
477	0\\
478	0\\
479	0\\
480	0\\
481	0\\
482	0\\
483	0\\
484	0\\
485	0\\
486	0\\
487	0\\
488	0\\
489	0\\
490	0\\
491	0\\
492	0\\
493	0\\
494	0\\
495	0\\
496	0\\
497	0\\
498	0\\
499	0\\
500	0\\
501	0\\
502	0\\
503	0\\
504	0\\
505	0\\
506	0\\
507	0\\
508	0\\
509	0\\
510	0\\
511	0\\
512	0\\
513	0\\
514	0\\
515	0\\
516	0\\
517	0\\
518	0\\
519	0\\
520	0\\
521	0\\
522	0\\
523	0\\
524	0\\
525	0\\
526	0\\
527	0\\
528	0\\
529	0\\
530	0\\
531	0\\
532	0\\
533	0\\
534	0\\
535	0\\
536	0\\
537	0\\
538	0\\
539	0\\
540	0\\
541	0\\
542	0\\
543	0\\
544	0\\
545	3.82391076137234e-05\\
546	9.68023725063791e-05\\
547	0.000147720982546101\\
548	0.000190142250714629\\
549	0.00022979867141449\\
550	0.000268454303467937\\
551	0.000306284037234123\\
552	0.00034374680041995\\
553	0.000381433047254094\\
554	0.000419542603462078\\
555	0.000458172315969797\\
556	0.000497380378338157\\
557	0.00053955610927823\\
558	0.000581988250460856\\
559	0.000624481094669423\\
560	0.000665506456059995\\
561	0.000706089745012563\\
562	0.000747338613311647\\
563	0.000789267713198676\\
564	0.000831888557103819\\
565	0.000875412308035493\\
566	0.000919872431348538\\
567	0.000965414080806018\\
568	0.00101726491970261\\
569	0.00106823974206828\\
570	0.00111385344432899\\
571	0.00115893121983135\\
572	0.00120437493685983\\
573	0.00125042703199772\\
574	0.00134080429263664\\
575	0.00187574822298428\\
576	0.00221187192817254\\
577	0.00232010641060596\\
578	0.00240256038064571\\
579	0.00247601921057826\\
580	0.00255050068707274\\
581	0.00262619072506652\\
582	0.00270312408738541\\
583	0.00278133051979913\\
584	0.00286084071525918\\
585	0.00294168800565563\\
586	0.00302391165742733\\
587	0.00310756532765526\\
588	0.00319273905359195\\
589	0.00327961638930458\\
590	0.00336862265377153\\
591	0.00346039684658503\\
592	0.00355653402929194\\
593	0.00366119131364763\\
594	0.00378527046594783\\
595	0.00395742119285503\\
596	0.00425297862742962\\
597	0.0048700418989439\\
598	0.00632942537858856\\
599	0\\
600	0\\
};
\addplot [color=red!75!mycolor17,solid,forget plot]
  table[row sep=crcr]{%
1	0\\
2	0\\
3	0\\
4	0\\
5	0\\
6	0\\
7	0\\
8	0\\
9	0\\
10	0\\
11	0\\
12	0\\
13	0\\
14	0\\
15	0\\
16	0\\
17	0\\
18	0\\
19	0\\
20	0\\
21	0\\
22	0\\
23	0\\
24	0\\
25	0\\
26	0\\
27	0\\
28	0\\
29	0\\
30	0\\
31	0\\
32	0\\
33	0\\
34	0\\
35	0\\
36	0\\
37	0\\
38	0\\
39	0\\
40	0\\
41	0\\
42	0\\
43	0\\
44	0\\
45	0\\
46	0\\
47	0\\
48	0\\
49	0\\
50	0\\
51	0\\
52	0\\
53	0\\
54	0\\
55	0\\
56	0\\
57	0\\
58	0\\
59	0\\
60	0\\
61	0\\
62	0\\
63	0\\
64	0\\
65	0\\
66	0\\
67	0\\
68	0\\
69	0\\
70	0\\
71	0\\
72	0\\
73	0\\
74	0\\
75	0\\
76	0\\
77	0\\
78	0\\
79	0\\
80	0\\
81	0\\
82	0\\
83	0\\
84	0\\
85	0\\
86	0\\
87	0\\
88	0\\
89	0\\
90	0\\
91	0\\
92	0\\
93	0\\
94	0\\
95	0\\
96	0\\
97	0\\
98	0\\
99	0\\
100	0\\
101	0\\
102	0\\
103	0\\
104	0\\
105	0\\
106	0\\
107	0\\
108	0\\
109	0\\
110	0\\
111	0\\
112	0\\
113	0\\
114	0\\
115	0\\
116	0\\
117	0\\
118	0\\
119	0\\
120	0\\
121	0\\
122	0\\
123	0\\
124	0\\
125	0\\
126	0\\
127	0\\
128	0\\
129	0\\
130	0\\
131	0\\
132	0\\
133	0\\
134	0\\
135	0\\
136	0\\
137	0\\
138	0\\
139	0\\
140	0\\
141	0\\
142	0\\
143	0\\
144	0\\
145	0\\
146	0\\
147	0\\
148	0\\
149	0\\
150	0\\
151	0\\
152	0\\
153	0\\
154	0\\
155	0\\
156	0\\
157	0\\
158	0\\
159	0\\
160	0\\
161	0\\
162	0\\
163	0\\
164	0\\
165	0\\
166	0\\
167	0\\
168	0\\
169	0\\
170	0\\
171	0\\
172	0\\
173	0\\
174	0\\
175	0\\
176	0\\
177	0\\
178	0\\
179	0\\
180	0\\
181	0\\
182	0\\
183	0\\
184	0\\
185	0\\
186	0\\
187	0\\
188	0\\
189	0\\
190	0\\
191	0\\
192	0\\
193	0\\
194	0\\
195	0\\
196	0\\
197	0\\
198	0\\
199	0\\
200	0\\
201	0\\
202	0\\
203	0\\
204	0\\
205	0\\
206	0\\
207	0\\
208	0\\
209	0\\
210	0\\
211	0\\
212	0\\
213	0\\
214	0\\
215	0\\
216	0\\
217	0\\
218	0\\
219	0\\
220	0\\
221	0\\
222	0\\
223	0\\
224	0\\
225	0\\
226	0\\
227	0\\
228	0\\
229	0\\
230	0\\
231	0\\
232	0\\
233	0\\
234	0\\
235	0\\
236	0\\
237	0\\
238	0\\
239	0\\
240	0\\
241	0\\
242	0\\
243	0\\
244	0\\
245	0\\
246	0\\
247	0\\
248	0\\
249	0\\
250	0\\
251	0\\
252	0\\
253	0\\
254	0\\
255	0\\
256	0\\
257	0\\
258	0\\
259	0\\
260	0\\
261	0\\
262	0\\
263	0\\
264	0\\
265	0\\
266	0\\
267	0\\
268	0\\
269	0\\
270	0\\
271	0\\
272	0\\
273	0\\
274	0\\
275	0\\
276	0\\
277	0\\
278	0\\
279	0\\
280	0\\
281	0\\
282	0\\
283	0\\
284	0\\
285	0\\
286	0\\
287	0\\
288	0\\
289	0\\
290	0\\
291	0\\
292	0\\
293	0\\
294	0\\
295	0\\
296	0\\
297	0\\
298	0\\
299	0\\
300	0\\
301	0\\
302	0\\
303	0\\
304	0\\
305	0\\
306	0\\
307	0\\
308	0\\
309	0\\
310	0\\
311	0\\
312	0\\
313	0\\
314	0\\
315	0\\
316	0\\
317	0\\
318	0\\
319	0\\
320	0\\
321	0\\
322	0\\
323	0\\
324	0\\
325	0\\
326	0\\
327	0\\
328	0\\
329	0\\
330	0\\
331	0\\
332	0\\
333	0\\
334	0\\
335	0\\
336	0\\
337	0\\
338	0\\
339	0\\
340	0\\
341	0\\
342	0\\
343	0\\
344	0\\
345	0\\
346	0\\
347	0\\
348	0\\
349	0\\
350	0\\
351	0\\
352	0\\
353	0\\
354	0\\
355	0\\
356	0\\
357	0\\
358	0\\
359	0\\
360	0\\
361	0\\
362	0\\
363	0\\
364	0\\
365	0\\
366	0\\
367	0\\
368	0\\
369	0\\
370	0\\
371	0\\
372	0\\
373	0\\
374	0\\
375	0\\
376	0\\
377	0\\
378	0\\
379	0\\
380	0\\
381	0\\
382	0\\
383	0\\
384	0\\
385	0\\
386	0\\
387	0\\
388	0\\
389	0\\
390	0\\
391	0\\
392	0\\
393	0\\
394	0\\
395	0\\
396	0\\
397	0\\
398	0\\
399	0\\
400	0\\
401	0\\
402	0\\
403	0\\
404	0\\
405	0\\
406	0\\
407	0\\
408	0\\
409	0\\
410	0\\
411	0\\
412	0\\
413	0\\
414	0\\
415	0\\
416	0\\
417	0\\
418	0\\
419	0\\
420	0\\
421	0\\
422	0\\
423	0\\
424	0\\
425	0\\
426	0\\
427	0\\
428	0\\
429	0\\
430	0\\
431	0\\
432	0\\
433	0\\
434	0\\
435	0\\
436	0\\
437	0\\
438	0\\
439	0\\
440	0\\
441	0\\
442	0\\
443	0\\
444	0\\
445	0\\
446	0\\
447	0\\
448	0\\
449	0\\
450	0\\
451	0\\
452	0\\
453	0\\
454	0\\
455	0\\
456	0\\
457	0\\
458	0\\
459	0\\
460	0\\
461	0\\
462	0\\
463	0\\
464	0\\
465	0\\
466	0\\
467	0\\
468	0\\
469	0\\
470	0\\
471	0\\
472	0\\
473	0\\
474	0\\
475	0\\
476	0\\
477	0\\
478	0\\
479	0\\
480	0\\
481	0\\
482	0\\
483	0\\
484	0\\
485	0\\
486	0\\
487	0\\
488	0\\
489	0\\
490	0\\
491	0\\
492	0\\
493	0\\
494	0\\
495	0\\
496	0\\
497	0\\
498	0\\
499	0\\
500	0\\
501	0\\
502	0\\
503	0\\
504	0\\
505	0\\
506	0\\
507	0\\
508	0\\
509	0\\
510	0\\
511	0\\
512	0\\
513	0\\
514	0\\
515	0\\
516	0\\
517	0\\
518	0\\
519	0\\
520	0\\
521	0\\
522	0\\
523	0\\
524	0\\
525	0\\
526	0\\
527	0\\
528	0\\
529	0\\
530	0\\
531	0\\
532	0\\
533	0\\
534	0\\
535	0\\
536	0\\
537	0\\
538	0\\
539	0\\
540	0\\
541	0\\
542	2.04853442762617e-05\\
543	6.06220385510295e-05\\
544	9.64802235070158e-05\\
545	0.000131320163008711\\
546	0.000165361460811259\\
547	0.000199053503989163\\
548	0.000232946229837191\\
549	0.00026724254469209\\
550	0.000302038372857046\\
551	0.000337433066059789\\
552	0.000373524405103253\\
553	0.000410607915064151\\
554	0.000450645605495247\\
555	0.000490655083798201\\
556	0.000530561602653835\\
557	0.000568219493923374\\
558	0.000606305612849557\\
559	0.000645024893172913\\
560	0.000684358053097023\\
561	0.000724443518312681\\
562	0.000765389047441491\\
563	0.000807225433839359\\
564	0.000849982974356452\\
565	0.000893867927586915\\
566	0.000944065882789967\\
567	0.000993239584238119\\
568	0.00103619979679305\\
569	0.00107953810792023\\
570	0.00112318247816187\\
571	0.00116757410108689\\
572	0.00121282583955014\\
573	0.00154610456066244\\
574	0.00206173146214074\\
575	0.00217000160705326\\
576	0.00225943058102288\\
577	0.00233065932806681\\
578	0.00240277285646767\\
579	0.00247604740747313\\
580	0.00255051340445631\\
581	0.00262619770980058\\
582	0.00270312784773939\\
583	0.00278133241566772\\
584	0.00286084158979877\\
585	0.00294168836537039\\
586	0.00302391178438079\\
587	0.00310756536383239\\
588	0.00319273906101823\\
589	0.00327961639013216\\
590	0.00336862265377154\\
591	0.00346039684658504\\
592	0.00355653402929194\\
593	0.00366119131364762\\
594	0.00378527046594782\\
595	0.00395742119285503\\
596	0.00425297862742962\\
597	0.0048700418989439\\
598	0.00632942537858856\\
599	0\\
600	0\\
};
\addplot [color=red!80!mycolor19,solid,forget plot]
  table[row sep=crcr]{%
1	0\\
2	0\\
3	0\\
4	0\\
5	0\\
6	0\\
7	0\\
8	0\\
9	0\\
10	0\\
11	0\\
12	0\\
13	0\\
14	0\\
15	0\\
16	0\\
17	0\\
18	0\\
19	0\\
20	0\\
21	0\\
22	0\\
23	0\\
24	0\\
25	0\\
26	0\\
27	0\\
28	0\\
29	0\\
30	0\\
31	0\\
32	0\\
33	0\\
34	0\\
35	0\\
36	0\\
37	0\\
38	0\\
39	0\\
40	0\\
41	0\\
42	0\\
43	0\\
44	0\\
45	0\\
46	0\\
47	0\\
48	0\\
49	0\\
50	0\\
51	0\\
52	0\\
53	0\\
54	0\\
55	0\\
56	0\\
57	0\\
58	0\\
59	0\\
60	0\\
61	0\\
62	0\\
63	0\\
64	0\\
65	0\\
66	0\\
67	0\\
68	0\\
69	0\\
70	0\\
71	0\\
72	0\\
73	0\\
74	0\\
75	0\\
76	0\\
77	0\\
78	0\\
79	0\\
80	0\\
81	0\\
82	0\\
83	0\\
84	0\\
85	0\\
86	0\\
87	0\\
88	0\\
89	0\\
90	0\\
91	0\\
92	0\\
93	0\\
94	0\\
95	0\\
96	0\\
97	0\\
98	0\\
99	0\\
100	0\\
101	0\\
102	0\\
103	0\\
104	0\\
105	0\\
106	0\\
107	0\\
108	0\\
109	0\\
110	0\\
111	0\\
112	0\\
113	0\\
114	0\\
115	0\\
116	0\\
117	0\\
118	0\\
119	0\\
120	0\\
121	0\\
122	0\\
123	0\\
124	0\\
125	0\\
126	0\\
127	0\\
128	0\\
129	0\\
130	0\\
131	0\\
132	0\\
133	0\\
134	0\\
135	0\\
136	0\\
137	0\\
138	0\\
139	0\\
140	0\\
141	0\\
142	0\\
143	0\\
144	0\\
145	0\\
146	0\\
147	0\\
148	0\\
149	0\\
150	0\\
151	0\\
152	0\\
153	0\\
154	0\\
155	0\\
156	0\\
157	0\\
158	0\\
159	0\\
160	0\\
161	0\\
162	0\\
163	0\\
164	0\\
165	0\\
166	0\\
167	0\\
168	0\\
169	0\\
170	0\\
171	0\\
172	0\\
173	0\\
174	0\\
175	0\\
176	0\\
177	0\\
178	0\\
179	0\\
180	0\\
181	0\\
182	0\\
183	0\\
184	0\\
185	0\\
186	0\\
187	0\\
188	0\\
189	0\\
190	0\\
191	0\\
192	0\\
193	0\\
194	0\\
195	0\\
196	0\\
197	0\\
198	0\\
199	0\\
200	0\\
201	0\\
202	0\\
203	0\\
204	0\\
205	0\\
206	0\\
207	0\\
208	0\\
209	0\\
210	0\\
211	0\\
212	0\\
213	0\\
214	0\\
215	0\\
216	0\\
217	0\\
218	0\\
219	0\\
220	0\\
221	0\\
222	0\\
223	0\\
224	0\\
225	0\\
226	0\\
227	0\\
228	0\\
229	0\\
230	0\\
231	0\\
232	0\\
233	0\\
234	0\\
235	0\\
236	0\\
237	0\\
238	0\\
239	0\\
240	0\\
241	0\\
242	0\\
243	0\\
244	0\\
245	0\\
246	0\\
247	0\\
248	0\\
249	0\\
250	0\\
251	0\\
252	0\\
253	0\\
254	0\\
255	0\\
256	0\\
257	0\\
258	0\\
259	0\\
260	0\\
261	0\\
262	0\\
263	0\\
264	0\\
265	0\\
266	0\\
267	0\\
268	0\\
269	0\\
270	0\\
271	0\\
272	0\\
273	0\\
274	0\\
275	0\\
276	0\\
277	0\\
278	0\\
279	0\\
280	0\\
281	0\\
282	0\\
283	0\\
284	0\\
285	0\\
286	0\\
287	0\\
288	0\\
289	0\\
290	0\\
291	0\\
292	0\\
293	0\\
294	0\\
295	0\\
296	0\\
297	0\\
298	0\\
299	0\\
300	0\\
301	0\\
302	0\\
303	0\\
304	0\\
305	0\\
306	0\\
307	0\\
308	0\\
309	0\\
310	0\\
311	0\\
312	0\\
313	0\\
314	0\\
315	0\\
316	0\\
317	0\\
318	0\\
319	0\\
320	0\\
321	0\\
322	0\\
323	0\\
324	0\\
325	0\\
326	0\\
327	0\\
328	0\\
329	0\\
330	0\\
331	0\\
332	0\\
333	0\\
334	0\\
335	0\\
336	0\\
337	0\\
338	0\\
339	0\\
340	0\\
341	0\\
342	0\\
343	0\\
344	0\\
345	0\\
346	0\\
347	0\\
348	0\\
349	0\\
350	0\\
351	0\\
352	0\\
353	0\\
354	0\\
355	0\\
356	0\\
357	0\\
358	0\\
359	0\\
360	0\\
361	0\\
362	0\\
363	0\\
364	0\\
365	0\\
366	0\\
367	0\\
368	0\\
369	0\\
370	0\\
371	0\\
372	0\\
373	0\\
374	0\\
375	0\\
376	0\\
377	0\\
378	0\\
379	0\\
380	0\\
381	0\\
382	0\\
383	0\\
384	0\\
385	0\\
386	0\\
387	0\\
388	0\\
389	0\\
390	0\\
391	0\\
392	0\\
393	0\\
394	0\\
395	0\\
396	0\\
397	0\\
398	0\\
399	0\\
400	0\\
401	0\\
402	0\\
403	0\\
404	0\\
405	0\\
406	0\\
407	0\\
408	0\\
409	0\\
410	0\\
411	0\\
412	0\\
413	0\\
414	0\\
415	0\\
416	0\\
417	0\\
418	0\\
419	0\\
420	0\\
421	0\\
422	0\\
423	0\\
424	0\\
425	0\\
426	0\\
427	0\\
428	0\\
429	0\\
430	0\\
431	0\\
432	0\\
433	0\\
434	0\\
435	0\\
436	0\\
437	0\\
438	0\\
439	0\\
440	0\\
441	0\\
442	0\\
443	0\\
444	0\\
445	0\\
446	0\\
447	0\\
448	0\\
449	0\\
450	0\\
451	0\\
452	0\\
453	0\\
454	0\\
455	0\\
456	0\\
457	0\\
458	0\\
459	0\\
460	0\\
461	0\\
462	0\\
463	0\\
464	0\\
465	0\\
466	0\\
467	0\\
468	0\\
469	0\\
470	0\\
471	0\\
472	0\\
473	0\\
474	0\\
475	0\\
476	0\\
477	0\\
478	0\\
479	0\\
480	0\\
481	0\\
482	0\\
483	0\\
484	0\\
485	0\\
486	0\\
487	0\\
488	0\\
489	0\\
490	0\\
491	0\\
492	0\\
493	0\\
494	0\\
495	0\\
496	0\\
497	0\\
498	0\\
499	0\\
500	0\\
501	0\\
502	0\\
503	0\\
504	0\\
505	0\\
506	0\\
507	0\\
508	0\\
509	0\\
510	0\\
511	0\\
512	0\\
513	0\\
514	0\\
515	0\\
516	0\\
517	0\\
518	0\\
519	0\\
520	0\\
521	0\\
522	0\\
523	0\\
524	0\\
525	0\\
526	0\\
527	0\\
528	0\\
529	0\\
530	0\\
531	0\\
532	0\\
533	0\\
534	0\\
535	0\\
536	0\\
537	0\\
538	0\\
539	0\\
540	4.09915539253185e-06\\
541	3.50855341422652e-05\\
542	6.56241650374363e-05\\
543	9.62275076458695e-05\\
544	0.000127190880966168\\
545	0.000158606315656507\\
546	0.000190566750982985\\
547	0.000223147011209739\\
548	0.000256388207954706\\
549	0.000290340966566383\\
550	0.000325350059729218\\
551	0.00036325193318367\\
552	0.000401173491610415\\
553	0.000438840853802339\\
554	0.000474293154902854\\
555	0.000510160152892572\\
556	0.000546586920712631\\
557	0.000583562655249454\\
558	0.000621307465471656\\
559	0.00065986095225696\\
560	0.000699251876427312\\
561	0.0007395059311949\\
562	0.000780644872361251\\
563	0.000822720633982572\\
564	0.000870893647640451\\
565	0.000918570800930397\\
566	0.000959929760329168\\
567	0.00100162913418252\\
568	0.00104358144557724\\
569	0.00108631367623227\\
570	0.0011298652614946\\
571	0.00117427238843636\\
572	0.00171542356870662\\
573	0.00201939825841623\\
574	0.00212061739353802\\
575	0.0021898538250397\\
576	0.00225970681651282\\
577	0.00233067087660055\\
578	0.00240277681456536\\
579	0.00247604952405764\\
580	0.00255051455217948\\
581	0.00262619830282023\\
582	0.00270312813271299\\
583	0.00278133254043779\\
584	0.00286084163829612\\
585	0.00294168838147584\\
586	0.00302391178867891\\
587	0.00310756536465452\\
588	0.00319273906110311\\
589	0.00327961639013215\\
590	0.00336862265377152\\
591	0.00346039684658503\\
592	0.00355653402929193\\
593	0.00366119131364762\\
594	0.00378527046594782\\
595	0.00395742119285502\\
596	0.00425297862742962\\
597	0.0048700418989439\\
598	0.00632942537858856\\
599	0\\
600	0\\
};
\addplot [color=red,solid,forget plot]
  table[row sep=crcr]{%
1	0\\
2	0\\
3	0\\
4	0\\
5	0\\
6	0\\
7	0\\
8	0\\
9	0\\
10	0\\
11	0\\
12	0\\
13	0\\
14	0\\
15	0\\
16	0\\
17	0\\
18	0\\
19	0\\
20	0\\
21	0\\
22	0\\
23	0\\
24	0\\
25	0\\
26	0\\
27	0\\
28	0\\
29	0\\
30	0\\
31	0\\
32	0\\
33	0\\
34	0\\
35	0\\
36	0\\
37	0\\
38	0\\
39	0\\
40	0\\
41	0\\
42	0\\
43	0\\
44	0\\
45	0\\
46	0\\
47	0\\
48	0\\
49	0\\
50	0\\
51	0\\
52	0\\
53	0\\
54	0\\
55	0\\
56	0\\
57	0\\
58	0\\
59	0\\
60	0\\
61	0\\
62	0\\
63	0\\
64	0\\
65	0\\
66	0\\
67	0\\
68	0\\
69	0\\
70	0\\
71	0\\
72	0\\
73	0\\
74	0\\
75	0\\
76	0\\
77	0\\
78	0\\
79	0\\
80	0\\
81	0\\
82	0\\
83	0\\
84	0\\
85	0\\
86	0\\
87	0\\
88	0\\
89	0\\
90	0\\
91	0\\
92	0\\
93	0\\
94	0\\
95	0\\
96	0\\
97	0\\
98	0\\
99	0\\
100	0\\
101	0\\
102	0\\
103	0\\
104	0\\
105	0\\
106	0\\
107	0\\
108	0\\
109	0\\
110	0\\
111	0\\
112	0\\
113	0\\
114	0\\
115	0\\
116	0\\
117	0\\
118	0\\
119	0\\
120	0\\
121	0\\
122	0\\
123	0\\
124	0\\
125	0\\
126	0\\
127	0\\
128	0\\
129	0\\
130	0\\
131	0\\
132	0\\
133	0\\
134	0\\
135	0\\
136	0\\
137	0\\
138	0\\
139	0\\
140	0\\
141	0\\
142	0\\
143	0\\
144	0\\
145	0\\
146	0\\
147	0\\
148	0\\
149	0\\
150	0\\
151	0\\
152	0\\
153	0\\
154	0\\
155	0\\
156	0\\
157	0\\
158	0\\
159	0\\
160	0\\
161	0\\
162	0\\
163	0\\
164	0\\
165	0\\
166	0\\
167	0\\
168	0\\
169	0\\
170	0\\
171	0\\
172	0\\
173	0\\
174	0\\
175	0\\
176	0\\
177	0\\
178	0\\
179	0\\
180	0\\
181	0\\
182	0\\
183	0\\
184	0\\
185	0\\
186	0\\
187	0\\
188	0\\
189	0\\
190	0\\
191	0\\
192	0\\
193	0\\
194	0\\
195	0\\
196	0\\
197	0\\
198	0\\
199	0\\
200	0\\
201	0\\
202	0\\
203	0\\
204	0\\
205	0\\
206	0\\
207	0\\
208	0\\
209	0\\
210	0\\
211	0\\
212	0\\
213	0\\
214	0\\
215	0\\
216	0\\
217	0\\
218	0\\
219	0\\
220	0\\
221	0\\
222	0\\
223	0\\
224	0\\
225	0\\
226	0\\
227	0\\
228	0\\
229	0\\
230	0\\
231	0\\
232	0\\
233	0\\
234	0\\
235	0\\
236	0\\
237	0\\
238	0\\
239	0\\
240	0\\
241	0\\
242	0\\
243	0\\
244	0\\
245	0\\
246	0\\
247	0\\
248	0\\
249	0\\
250	0\\
251	0\\
252	0\\
253	0\\
254	0\\
255	0\\
256	0\\
257	0\\
258	0\\
259	0\\
260	0\\
261	0\\
262	0\\
263	0\\
264	0\\
265	0\\
266	0\\
267	0\\
268	0\\
269	0\\
270	0\\
271	0\\
272	0\\
273	0\\
274	0\\
275	0\\
276	0\\
277	0\\
278	0\\
279	0\\
280	0\\
281	0\\
282	0\\
283	0\\
284	0\\
285	0\\
286	0\\
287	0\\
288	0\\
289	0\\
290	0\\
291	0\\
292	0\\
293	0\\
294	0\\
295	0\\
296	0\\
297	0\\
298	0\\
299	0\\
300	0\\
301	0\\
302	0\\
303	0\\
304	0\\
305	0\\
306	0\\
307	0\\
308	0\\
309	0\\
310	0\\
311	0\\
312	0\\
313	0\\
314	0\\
315	0\\
316	0\\
317	0\\
318	0\\
319	0\\
320	0\\
321	0\\
322	0\\
323	0\\
324	0\\
325	0\\
326	0\\
327	0\\
328	0\\
329	0\\
330	0\\
331	0\\
332	0\\
333	0\\
334	0\\
335	0\\
336	0\\
337	0\\
338	0\\
339	0\\
340	0\\
341	0\\
342	0\\
343	0\\
344	0\\
345	0\\
346	0\\
347	0\\
348	0\\
349	0\\
350	0\\
351	0\\
352	0\\
353	0\\
354	0\\
355	0\\
356	0\\
357	0\\
358	0\\
359	0\\
360	0\\
361	0\\
362	0\\
363	0\\
364	0\\
365	0\\
366	0\\
367	0\\
368	0\\
369	0\\
370	0\\
371	0\\
372	0\\
373	0\\
374	0\\
375	0\\
376	0\\
377	0\\
378	0\\
379	0\\
380	0\\
381	0\\
382	0\\
383	0\\
384	0\\
385	0\\
386	0\\
387	0\\
388	0\\
389	0\\
390	0\\
391	0\\
392	0\\
393	0\\
394	0\\
395	0\\
396	0\\
397	0\\
398	0\\
399	0\\
400	0\\
401	0\\
402	0\\
403	0\\
404	0\\
405	0\\
406	0\\
407	0\\
408	0\\
409	0\\
410	0\\
411	0\\
412	0\\
413	0\\
414	0\\
415	0\\
416	0\\
417	0\\
418	0\\
419	0\\
420	0\\
421	0\\
422	0\\
423	0\\
424	0\\
425	0\\
426	0\\
427	0\\
428	0\\
429	0\\
430	0\\
431	0\\
432	0\\
433	0\\
434	0\\
435	0\\
436	0\\
437	0\\
438	0\\
439	0\\
440	0\\
441	0\\
442	0\\
443	0\\
444	0\\
445	0\\
446	0\\
447	0\\
448	0\\
449	0\\
450	0\\
451	0\\
452	0\\
453	0\\
454	0\\
455	0\\
456	0\\
457	0\\
458	0\\
459	0\\
460	0\\
461	0\\
462	0\\
463	0\\
464	0\\
465	0\\
466	0\\
467	0\\
468	0\\
469	0\\
470	0\\
471	0\\
472	0\\
473	0\\
474	0\\
475	0\\
476	0\\
477	0\\
478	0\\
479	0\\
480	0\\
481	0\\
482	0\\
483	0\\
484	0\\
485	0\\
486	0\\
487	0\\
488	0\\
489	0\\
490	0\\
491	0\\
492	0\\
493	0\\
494	0\\
495	0\\
496	0\\
497	0\\
498	0\\
499	0\\
500	0\\
501	0\\
502	0\\
503	0\\
504	0\\
505	0\\
506	0\\
507	0\\
508	0\\
509	0\\
510	0\\
511	0\\
512	0\\
513	0\\
514	0\\
515	0\\
516	0\\
517	0\\
518	0\\
519	0\\
520	0\\
521	0\\
522	0\\
523	0\\
524	0\\
525	0\\
526	0\\
527	0\\
528	0\\
529	0\\
530	0\\
531	0\\
532	0\\
533	0\\
534	0\\
535	0\\
536	0\\
537	0\\
538	0\\
539	0\\
540	2.72097330305856e-05\\
541	5.61210861900328e-05\\
542	8.5585431440554e-05\\
543	0.000115648715291517\\
544	0.000146340270906872\\
545	0.000177684940527103\\
546	0.000209720256944281\\
547	0.000242556230441567\\
548	0.000278557139405352\\
549	0.000314550976192727\\
550	0.00035020981117787\\
551	0.000383724276119911\\
552	0.000417754807534627\\
553	0.000452289907333273\\
554	0.000487256054489089\\
555	0.000522838353760378\\
556	0.000559179818522669\\
557	0.000596309200618129\\
558	0.000634247900431846\\
559	0.000673017586461841\\
560	0.00071263931167027\\
561	0.000753150047157175\\
562	0.000798519250497241\\
563	0.000845250121868294\\
564	0.000885803380539406\\
565	0.000926054546725999\\
566	0.00096653431220477\\
567	0.00100777034248639\\
568	0.00104980318090226\\
569	0.0010926538929629\\
570	0.00126482494133247\\
571	0.00184533543969108\\
572	0.0019701721867511\\
573	0.00205342365444128\\
574	0.00212111322836602\\
575	0.00218986312710426\\
576	0.0022597081208049\\
577	0.0023306715146185\\
578	0.00240277715940921\\
579	0.0024760497051662\\
580	0.00255051464193462\\
581	0.00262619834400932\\
582	0.00270312814987377\\
583	0.00278133254676196\\
584	0.0028608416402799\\
585	0.00294168838197388\\
586	0.00302391178876813\\
587	0.00310756536466312\\
588	0.00319273906110311\\
589	0.00327961639013215\\
590	0.00336862265377153\\
591	0.00346039684658503\\
592	0.00355653402929193\\
593	0.00366119131364763\\
594	0.00378527046594782\\
595	0.00395742119285503\\
596	0.00425297862742962\\
597	0.0048700418989439\\
598	0.00632942537858856\\
599	0\\
600	0\\
};
\addplot [color=mycolor20,solid,forget plot]
  table[row sep=crcr]{%
1	0\\
2	0\\
3	0\\
4	0\\
5	0\\
6	0\\
7	0\\
8	0\\
9	0\\
10	0\\
11	0\\
12	0\\
13	0\\
14	0\\
15	0\\
16	0\\
17	0\\
18	0\\
19	0\\
20	0\\
21	0\\
22	0\\
23	0\\
24	0\\
25	0\\
26	0\\
27	0\\
28	0\\
29	0\\
30	0\\
31	0\\
32	0\\
33	0\\
34	0\\
35	0\\
36	0\\
37	0\\
38	0\\
39	0\\
40	0\\
41	0\\
42	0\\
43	0\\
44	0\\
45	0\\
46	0\\
47	0\\
48	0\\
49	0\\
50	0\\
51	0\\
52	0\\
53	0\\
54	0\\
55	0\\
56	0\\
57	0\\
58	0\\
59	0\\
60	0\\
61	0\\
62	0\\
63	0\\
64	0\\
65	0\\
66	0\\
67	0\\
68	0\\
69	0\\
70	0\\
71	0\\
72	0\\
73	0\\
74	0\\
75	0\\
76	0\\
77	0\\
78	0\\
79	0\\
80	0\\
81	0\\
82	0\\
83	0\\
84	0\\
85	0\\
86	0\\
87	0\\
88	0\\
89	0\\
90	0\\
91	0\\
92	0\\
93	0\\
94	0\\
95	0\\
96	0\\
97	0\\
98	0\\
99	0\\
100	0\\
101	0\\
102	0\\
103	0\\
104	0\\
105	0\\
106	0\\
107	0\\
108	0\\
109	0\\
110	0\\
111	0\\
112	0\\
113	0\\
114	0\\
115	0\\
116	0\\
117	0\\
118	0\\
119	0\\
120	0\\
121	0\\
122	0\\
123	0\\
124	0\\
125	0\\
126	0\\
127	0\\
128	0\\
129	0\\
130	0\\
131	0\\
132	0\\
133	0\\
134	0\\
135	0\\
136	0\\
137	0\\
138	0\\
139	0\\
140	0\\
141	0\\
142	0\\
143	0\\
144	0\\
145	0\\
146	0\\
147	0\\
148	0\\
149	0\\
150	0\\
151	0\\
152	0\\
153	0\\
154	0\\
155	0\\
156	0\\
157	0\\
158	0\\
159	0\\
160	0\\
161	0\\
162	0\\
163	0\\
164	0\\
165	0\\
166	0\\
167	0\\
168	0\\
169	0\\
170	0\\
171	0\\
172	0\\
173	0\\
174	0\\
175	0\\
176	0\\
177	0\\
178	0\\
179	0\\
180	0\\
181	0\\
182	0\\
183	0\\
184	0\\
185	0\\
186	0\\
187	0\\
188	0\\
189	0\\
190	0\\
191	0\\
192	0\\
193	0\\
194	0\\
195	0\\
196	0\\
197	0\\
198	0\\
199	0\\
200	0\\
201	0\\
202	0\\
203	0\\
204	0\\
205	0\\
206	0\\
207	0\\
208	0\\
209	0\\
210	0\\
211	0\\
212	0\\
213	0\\
214	0\\
215	0\\
216	0\\
217	0\\
218	0\\
219	0\\
220	0\\
221	0\\
222	0\\
223	0\\
224	0\\
225	0\\
226	0\\
227	0\\
228	0\\
229	0\\
230	0\\
231	0\\
232	0\\
233	0\\
234	0\\
235	0\\
236	0\\
237	0\\
238	0\\
239	0\\
240	0\\
241	0\\
242	0\\
243	0\\
244	0\\
245	0\\
246	0\\
247	0\\
248	0\\
249	0\\
250	0\\
251	0\\
252	0\\
253	0\\
254	0\\
255	0\\
256	0\\
257	0\\
258	0\\
259	0\\
260	0\\
261	0\\
262	0\\
263	0\\
264	0\\
265	0\\
266	0\\
267	0\\
268	0\\
269	0\\
270	0\\
271	0\\
272	0\\
273	0\\
274	0\\
275	0\\
276	0\\
277	0\\
278	0\\
279	0\\
280	0\\
281	0\\
282	0\\
283	0\\
284	0\\
285	0\\
286	0\\
287	0\\
288	0\\
289	0\\
290	0\\
291	0\\
292	0\\
293	0\\
294	0\\
295	0\\
296	0\\
297	0\\
298	0\\
299	0\\
300	0\\
301	0\\
302	0\\
303	0\\
304	0\\
305	0\\
306	0\\
307	0\\
308	0\\
309	0\\
310	0\\
311	0\\
312	0\\
313	0\\
314	0\\
315	0\\
316	0\\
317	0\\
318	0\\
319	0\\
320	0\\
321	0\\
322	0\\
323	0\\
324	0\\
325	0\\
326	0\\
327	0\\
328	0\\
329	0\\
330	0\\
331	0\\
332	0\\
333	0\\
334	0\\
335	0\\
336	0\\
337	0\\
338	0\\
339	0\\
340	0\\
341	0\\
342	0\\
343	0\\
344	0\\
345	0\\
346	0\\
347	0\\
348	0\\
349	0\\
350	0\\
351	0\\
352	0\\
353	0\\
354	0\\
355	0\\
356	0\\
357	0\\
358	0\\
359	0\\
360	0\\
361	0\\
362	0\\
363	0\\
364	0\\
365	0\\
366	0\\
367	0\\
368	0\\
369	0\\
370	0\\
371	0\\
372	0\\
373	0\\
374	0\\
375	0\\
376	0\\
377	0\\
378	0\\
379	0\\
380	0\\
381	0\\
382	0\\
383	0\\
384	0\\
385	0\\
386	0\\
387	0\\
388	0\\
389	0\\
390	0\\
391	0\\
392	0\\
393	0\\
394	0\\
395	0\\
396	0\\
397	0\\
398	0\\
399	0\\
400	0\\
401	0\\
402	0\\
403	0\\
404	0\\
405	0\\
406	0\\
407	0\\
408	0\\
409	0\\
410	0\\
411	0\\
412	0\\
413	0\\
414	0\\
415	0\\
416	0\\
417	0\\
418	0\\
419	0\\
420	0\\
421	0\\
422	0\\
423	0\\
424	0\\
425	0\\
426	0\\
427	0\\
428	0\\
429	0\\
430	0\\
431	0\\
432	0\\
433	0\\
434	0\\
435	0\\
436	0\\
437	0\\
438	0\\
439	0\\
440	0\\
441	0\\
442	0\\
443	0\\
444	0\\
445	0\\
446	0\\
447	0\\
448	0\\
449	0\\
450	0\\
451	0\\
452	0\\
453	0\\
454	0\\
455	0\\
456	0\\
457	0\\
458	0\\
459	0\\
460	0\\
461	0\\
462	0\\
463	0\\
464	0\\
465	0\\
466	0\\
467	0\\
468	0\\
469	0\\
470	0\\
471	0\\
472	0\\
473	0\\
474	0\\
475	0\\
476	0\\
477	0\\
478	0\\
479	0\\
480	0\\
481	0\\
482	0\\
483	0\\
484	0\\
485	0\\
486	0\\
487	0\\
488	0\\
489	0\\
490	0\\
491	0\\
492	0\\
493	0\\
494	0\\
495	0\\
496	0\\
497	0\\
498	0\\
499	0\\
500	0\\
501	0\\
502	0\\
503	0\\
504	0\\
505	0\\
506	0\\
507	0\\
508	0\\
509	0\\
510	0\\
511	0\\
512	0\\
513	0\\
514	0\\
515	0\\
516	0\\
517	0\\
518	0\\
519	0\\
520	0\\
521	0\\
522	0\\
523	0\\
524	0\\
525	0\\
526	0\\
527	0\\
528	0\\
529	0\\
530	0\\
531	0\\
532	0\\
533	0\\
534	0\\
535	0\\
536	0\\
537	0\\
538	0\\
539	1.49492185966091e-05\\
540	4.3322083531691e-05\\
541	7.23064945701987e-05\\
542	0.000101919514130588\\
543	0.000132186799457533\\
544	0.000163148555289638\\
545	0.000196899412872687\\
546	0.000231163070872471\\
547	0.000265280313724257\\
548	0.000296939134999868\\
549	0.000329084648366537\\
550	0.000361719670533462\\
551	0.000394828145692352\\
552	0.000428606969412651\\
553	0.000463052122313129\\
554	0.000498111370312253\\
555	0.000533863357465193\\
556	0.00057039128795026\\
557	0.000607714878606064\\
558	0.000645854512820227\\
559	0.000684829623640465\\
560	0.000726776106591716\\
561	0.000772673465119796\\
562	0.000813534509467145\\
563	0.000852431652279271\\
564	0.000891516665074081\\
565	0.000931278020432426\\
566	0.000971798618077333\\
567	0.00101309731890282\\
568	0.00105519192697659\\
569	0.00134093035495452\\
570	0.00181655877875805\\
571	0.00191924729878529\\
572	0.00198681216416615\\
573	0.00205343742759807\\
574	0.00212111382330841\\
575	0.00218986332242957\\
576	0.00225970822322969\\
577	0.00233067156883697\\
578	0.00240277718689623\\
579	0.00247604971824475\\
580	0.00255051464767741\\
581	0.00262619834629172\\
582	0.0027031281506736\\
583	0.00278133254699977\\
584	0.00286084164033628\\
585	0.00294168838198338\\
586	0.00302391178876899\\
587	0.00310756536466312\\
588	0.00319273906110312\\
589	0.00327961639013216\\
590	0.00336862265377153\\
591	0.00346039684658503\\
592	0.00355653402929193\\
593	0.00366119131364763\\
594	0.00378527046594782\\
595	0.00395742119285503\\
596	0.00425297862742962\\
597	0.0048700418989439\\
598	0.00632942537858856\\
599	0\\
600	0\\
};
\addplot [color=mycolor21,solid,forget plot]
  table[row sep=crcr]{%
1	0\\
2	0\\
3	0\\
4	0\\
5	0\\
6	0\\
7	0\\
8	0\\
9	0\\
10	0\\
11	0\\
12	0\\
13	0\\
14	0\\
15	0\\
16	0\\
17	0\\
18	0\\
19	0\\
20	0\\
21	0\\
22	0\\
23	0\\
24	0\\
25	0\\
26	0\\
27	0\\
28	0\\
29	0\\
30	0\\
31	0\\
32	0\\
33	0\\
34	0\\
35	0\\
36	0\\
37	0\\
38	0\\
39	0\\
40	0\\
41	0\\
42	0\\
43	0\\
44	0\\
45	0\\
46	0\\
47	0\\
48	0\\
49	0\\
50	0\\
51	0\\
52	0\\
53	0\\
54	0\\
55	0\\
56	0\\
57	0\\
58	0\\
59	0\\
60	0\\
61	0\\
62	0\\
63	0\\
64	0\\
65	0\\
66	0\\
67	0\\
68	0\\
69	0\\
70	0\\
71	0\\
72	0\\
73	0\\
74	0\\
75	0\\
76	0\\
77	0\\
78	0\\
79	0\\
80	0\\
81	0\\
82	0\\
83	0\\
84	0\\
85	0\\
86	0\\
87	0\\
88	0\\
89	0\\
90	0\\
91	0\\
92	0\\
93	0\\
94	0\\
95	0\\
96	0\\
97	0\\
98	0\\
99	0\\
100	0\\
101	0\\
102	0\\
103	0\\
104	0\\
105	0\\
106	0\\
107	0\\
108	0\\
109	0\\
110	0\\
111	0\\
112	0\\
113	0\\
114	0\\
115	0\\
116	0\\
117	0\\
118	0\\
119	0\\
120	0\\
121	0\\
122	0\\
123	0\\
124	0\\
125	0\\
126	0\\
127	0\\
128	0\\
129	0\\
130	0\\
131	0\\
132	0\\
133	0\\
134	0\\
135	0\\
136	0\\
137	0\\
138	0\\
139	0\\
140	0\\
141	0\\
142	0\\
143	0\\
144	0\\
145	0\\
146	0\\
147	0\\
148	0\\
149	0\\
150	0\\
151	0\\
152	0\\
153	0\\
154	0\\
155	0\\
156	0\\
157	0\\
158	0\\
159	0\\
160	0\\
161	0\\
162	0\\
163	0\\
164	0\\
165	0\\
166	0\\
167	0\\
168	0\\
169	0\\
170	0\\
171	0\\
172	0\\
173	0\\
174	0\\
175	0\\
176	0\\
177	0\\
178	0\\
179	0\\
180	0\\
181	0\\
182	0\\
183	0\\
184	0\\
185	0\\
186	0\\
187	0\\
188	0\\
189	0\\
190	0\\
191	0\\
192	0\\
193	0\\
194	0\\
195	0\\
196	0\\
197	0\\
198	0\\
199	0\\
200	0\\
201	0\\
202	0\\
203	0\\
204	0\\
205	0\\
206	0\\
207	0\\
208	0\\
209	0\\
210	0\\
211	0\\
212	0\\
213	0\\
214	0\\
215	0\\
216	0\\
217	0\\
218	0\\
219	0\\
220	0\\
221	0\\
222	0\\
223	0\\
224	0\\
225	0\\
226	0\\
227	0\\
228	0\\
229	0\\
230	0\\
231	0\\
232	0\\
233	0\\
234	0\\
235	0\\
236	0\\
237	0\\
238	0\\
239	0\\
240	0\\
241	0\\
242	0\\
243	0\\
244	0\\
245	0\\
246	0\\
247	0\\
248	0\\
249	0\\
250	0\\
251	0\\
252	0\\
253	0\\
254	0\\
255	0\\
256	0\\
257	0\\
258	0\\
259	0\\
260	0\\
261	0\\
262	0\\
263	0\\
264	0\\
265	0\\
266	0\\
267	0\\
268	0\\
269	0\\
270	0\\
271	0\\
272	0\\
273	0\\
274	0\\
275	0\\
276	0\\
277	0\\
278	0\\
279	0\\
280	0\\
281	0\\
282	0\\
283	0\\
284	0\\
285	0\\
286	0\\
287	0\\
288	0\\
289	0\\
290	0\\
291	0\\
292	0\\
293	0\\
294	0\\
295	0\\
296	0\\
297	0\\
298	0\\
299	0\\
300	0\\
301	0\\
302	0\\
303	0\\
304	0\\
305	0\\
306	0\\
307	0\\
308	0\\
309	0\\
310	0\\
311	0\\
312	0\\
313	0\\
314	0\\
315	0\\
316	0\\
317	0\\
318	0\\
319	0\\
320	0\\
321	0\\
322	0\\
323	0\\
324	0\\
325	0\\
326	0\\
327	0\\
328	0\\
329	0\\
330	0\\
331	0\\
332	0\\
333	0\\
334	0\\
335	0\\
336	0\\
337	0\\
338	0\\
339	0\\
340	0\\
341	0\\
342	0\\
343	0\\
344	0\\
345	0\\
346	0\\
347	0\\
348	0\\
349	0\\
350	0\\
351	0\\
352	0\\
353	0\\
354	0\\
355	0\\
356	0\\
357	0\\
358	0\\
359	0\\
360	0\\
361	0\\
362	0\\
363	0\\
364	0\\
365	0\\
366	0\\
367	0\\
368	0\\
369	0\\
370	0\\
371	0\\
372	0\\
373	0\\
374	0\\
375	0\\
376	0\\
377	0\\
378	0\\
379	0\\
380	0\\
381	0\\
382	0\\
383	0\\
384	0\\
385	0\\
386	0\\
387	0\\
388	0\\
389	0\\
390	0\\
391	0\\
392	0\\
393	0\\
394	0\\
395	0\\
396	0\\
397	0\\
398	0\\
399	0\\
400	0\\
401	0\\
402	0\\
403	0\\
404	0\\
405	0\\
406	0\\
407	0\\
408	0\\
409	0\\
410	0\\
411	0\\
412	0\\
413	0\\
414	0\\
415	0\\
416	0\\
417	0\\
418	0\\
419	0\\
420	0\\
421	0\\
422	0\\
423	0\\
424	0\\
425	0\\
426	0\\
427	0\\
428	0\\
429	0\\
430	0\\
431	0\\
432	0\\
433	0\\
434	0\\
435	0\\
436	0\\
437	0\\
438	0\\
439	0\\
440	0\\
441	0\\
442	0\\
443	0\\
444	0\\
445	0\\
446	0\\
447	0\\
448	0\\
449	0\\
450	0\\
451	0\\
452	0\\
453	0\\
454	0\\
455	0\\
456	0\\
457	0\\
458	0\\
459	0\\
460	0\\
461	0\\
462	0\\
463	0\\
464	0\\
465	0\\
466	0\\
467	0\\
468	0\\
469	0\\
470	0\\
471	0\\
472	0\\
473	0\\
474	0\\
475	0\\
476	0\\
477	0\\
478	0\\
479	0\\
480	0\\
481	0\\
482	0\\
483	0\\
484	0\\
485	0\\
486	0\\
487	0\\
488	0\\
489	0\\
490	0\\
491	0\\
492	0\\
493	0\\
494	0\\
495	0\\
496	0\\
497	0\\
498	0\\
499	0\\
500	0\\
501	0\\
502	0\\
503	0\\
504	0\\
505	0\\
506	0\\
507	0\\
508	0\\
509	0\\
510	0\\
511	0\\
512	0\\
513	0\\
514	0\\
515	0\\
516	0\\
517	0\\
518	0\\
519	0\\
520	0\\
521	0\\
522	0\\
523	0\\
524	0\\
525	0\\
526	0\\
527	0\\
528	0\\
529	0\\
530	0\\
531	0\\
532	0\\
533	0\\
534	0\\
535	0\\
536	0\\
537	0\\
538	1.21694107886508e-06\\
539	2.92045651449222e-05\\
540	5.78049456237067e-05\\
541	8.70536593544758e-05\\
542	0.000118312744446052\\
543	0.000151016328109347\\
544	0.000183614512075941\\
545	0.000214055086743375\\
546	0.000244429244110815\\
547	0.000275261690455392\\
548	0.00030651277983146\\
549	0.000338404008295745\\
550	0.000370954886140724\\
551	0.00040418384758811\\
552	0.000438095852672766\\
553	0.000472675235959531\\
554	0.000507841527916458\\
555	0.000543753467495973\\
556	0.000580443856548278\\
557	0.000617931118089366\\
558	0.000656254294614929\\
559	0.000701069072514467\\
560	0.000743145599645497\\
561	0.000780787630662042\\
562	0.000818610761048887\\
563	0.000856957736854093\\
564	0.000896029558364049\\
565	0.000935845227354032\\
566	0.000976422049464571\\
567	0.00101777751913248\\
568	0.00138028712439642\\
569	0.00176338880029626\\
570	0.00185658807509324\\
571	0.00192121863126382\\
572	0.00198681271195394\\
573	0.00205343749576471\\
574	0.00212111385372883\\
575	0.00218986333840477\\
576	0.00225970823145245\\
577	0.00233067157285982\\
578	0.00240277718873636\\
579	0.00247604971901934\\
580	0.00255051464797174\\
581	0.00262619834639004\\
582	0.00270312815070138\\
583	0.00278133254700601\\
584	0.00286084164033727\\
585	0.00294168838198347\\
586	0.00302391178876899\\
587	0.00310756536466311\\
588	0.00319273906110311\\
589	0.00327961639013215\\
590	0.00336862265377153\\
591	0.00346039684658503\\
592	0.00355653402929193\\
593	0.00366119131364763\\
594	0.00378527046594783\\
595	0.00395742119285503\\
596	0.00425297862742962\\
597	0.0048700418989439\\
598	0.00632942537858856\\
599	0\\
600	0\\
};
\addplot [color=black!20!mycolor21,solid,forget plot]
  table[row sep=crcr]{%
1	0\\
2	0\\
3	0\\
4	0\\
5	0\\
6	0\\
7	0\\
8	0\\
9	0\\
10	0\\
11	0\\
12	0\\
13	0\\
14	0\\
15	0\\
16	0\\
17	0\\
18	0\\
19	0\\
20	0\\
21	0\\
22	0\\
23	0\\
24	0\\
25	0\\
26	0\\
27	0\\
28	0\\
29	0\\
30	0\\
31	0\\
32	0\\
33	0\\
34	0\\
35	0\\
36	0\\
37	0\\
38	0\\
39	0\\
40	0\\
41	0\\
42	0\\
43	0\\
44	0\\
45	0\\
46	0\\
47	0\\
48	0\\
49	0\\
50	0\\
51	0\\
52	0\\
53	0\\
54	0\\
55	0\\
56	0\\
57	0\\
58	0\\
59	0\\
60	0\\
61	0\\
62	0\\
63	0\\
64	0\\
65	0\\
66	0\\
67	0\\
68	0\\
69	0\\
70	0\\
71	0\\
72	0\\
73	0\\
74	0\\
75	0\\
76	0\\
77	0\\
78	0\\
79	0\\
80	0\\
81	0\\
82	0\\
83	0\\
84	0\\
85	0\\
86	0\\
87	0\\
88	0\\
89	0\\
90	0\\
91	0\\
92	0\\
93	0\\
94	0\\
95	0\\
96	0\\
97	0\\
98	0\\
99	0\\
100	0\\
101	0\\
102	0\\
103	0\\
104	0\\
105	0\\
106	0\\
107	0\\
108	0\\
109	0\\
110	0\\
111	0\\
112	0\\
113	0\\
114	0\\
115	0\\
116	0\\
117	0\\
118	0\\
119	0\\
120	0\\
121	0\\
122	0\\
123	0\\
124	0\\
125	0\\
126	0\\
127	0\\
128	0\\
129	0\\
130	0\\
131	0\\
132	0\\
133	0\\
134	0\\
135	0\\
136	0\\
137	0\\
138	0\\
139	0\\
140	0\\
141	0\\
142	0\\
143	0\\
144	0\\
145	0\\
146	0\\
147	0\\
148	0\\
149	0\\
150	0\\
151	0\\
152	0\\
153	0\\
154	0\\
155	0\\
156	0\\
157	0\\
158	0\\
159	0\\
160	0\\
161	0\\
162	0\\
163	0\\
164	0\\
165	0\\
166	0\\
167	0\\
168	0\\
169	0\\
170	0\\
171	0\\
172	0\\
173	0\\
174	0\\
175	0\\
176	0\\
177	0\\
178	0\\
179	0\\
180	0\\
181	0\\
182	0\\
183	0\\
184	0\\
185	0\\
186	0\\
187	0\\
188	0\\
189	0\\
190	0\\
191	0\\
192	0\\
193	0\\
194	0\\
195	0\\
196	0\\
197	0\\
198	0\\
199	0\\
200	0\\
201	0\\
202	0\\
203	0\\
204	0\\
205	0\\
206	0\\
207	0\\
208	0\\
209	0\\
210	0\\
211	0\\
212	0\\
213	0\\
214	0\\
215	0\\
216	0\\
217	0\\
218	0\\
219	0\\
220	0\\
221	0\\
222	0\\
223	0\\
224	0\\
225	0\\
226	0\\
227	0\\
228	0\\
229	0\\
230	0\\
231	0\\
232	0\\
233	0\\
234	0\\
235	0\\
236	0\\
237	0\\
238	0\\
239	0\\
240	0\\
241	0\\
242	0\\
243	0\\
244	0\\
245	0\\
246	0\\
247	0\\
248	0\\
249	0\\
250	0\\
251	0\\
252	0\\
253	0\\
254	0\\
255	0\\
256	0\\
257	0\\
258	0\\
259	0\\
260	0\\
261	0\\
262	0\\
263	0\\
264	0\\
265	0\\
266	0\\
267	0\\
268	0\\
269	0\\
270	0\\
271	0\\
272	0\\
273	0\\
274	0\\
275	0\\
276	0\\
277	0\\
278	0\\
279	0\\
280	0\\
281	0\\
282	0\\
283	0\\
284	0\\
285	0\\
286	0\\
287	0\\
288	0\\
289	0\\
290	0\\
291	0\\
292	0\\
293	0\\
294	0\\
295	0\\
296	0\\
297	0\\
298	0\\
299	0\\
300	0\\
301	0\\
302	0\\
303	0\\
304	0\\
305	0\\
306	0\\
307	0\\
308	0\\
309	0\\
310	0\\
311	0\\
312	0\\
313	0\\
314	0\\
315	0\\
316	0\\
317	0\\
318	0\\
319	0\\
320	0\\
321	0\\
322	0\\
323	0\\
324	0\\
325	0\\
326	0\\
327	0\\
328	0\\
329	0\\
330	0\\
331	0\\
332	0\\
333	0\\
334	0\\
335	0\\
336	0\\
337	0\\
338	0\\
339	0\\
340	0\\
341	0\\
342	0\\
343	0\\
344	0\\
345	0\\
346	0\\
347	0\\
348	0\\
349	0\\
350	0\\
351	0\\
352	0\\
353	0\\
354	0\\
355	0\\
356	0\\
357	0\\
358	0\\
359	0\\
360	0\\
361	0\\
362	0\\
363	0\\
364	0\\
365	0\\
366	0\\
367	0\\
368	0\\
369	0\\
370	0\\
371	0\\
372	0\\
373	0\\
374	0\\
375	0\\
376	0\\
377	0\\
378	0\\
379	0\\
380	0\\
381	0\\
382	0\\
383	0\\
384	0\\
385	0\\
386	0\\
387	0\\
388	0\\
389	0\\
390	0\\
391	0\\
392	0\\
393	0\\
394	0\\
395	0\\
396	0\\
397	0\\
398	0\\
399	0\\
400	0\\
401	0\\
402	0\\
403	0\\
404	0\\
405	0\\
406	0\\
407	0\\
408	0\\
409	0\\
410	0\\
411	0\\
412	0\\
413	0\\
414	0\\
415	0\\
416	0\\
417	0\\
418	0\\
419	0\\
420	0\\
421	0\\
422	0\\
423	0\\
424	0\\
425	0\\
426	0\\
427	0\\
428	0\\
429	0\\
430	0\\
431	0\\
432	0\\
433	0\\
434	0\\
435	0\\
436	0\\
437	0\\
438	0\\
439	0\\
440	0\\
441	0\\
442	0\\
443	0\\
444	0\\
445	0\\
446	0\\
447	0\\
448	0\\
449	0\\
450	0\\
451	0\\
452	0\\
453	0\\
454	0\\
455	0\\
456	0\\
457	0\\
458	0\\
459	0\\
460	0\\
461	0\\
462	0\\
463	0\\
464	0\\
465	0\\
466	0\\
467	0\\
468	0\\
469	0\\
470	0\\
471	0\\
472	0\\
473	0\\
474	0\\
475	0\\
476	0\\
477	0\\
478	0\\
479	0\\
480	0\\
481	0\\
482	0\\
483	0\\
484	0\\
485	0\\
486	0\\
487	0\\
488	0\\
489	0\\
490	0\\
491	0\\
492	0\\
493	0\\
494	0\\
495	0\\
496	0\\
497	0\\
498	0\\
499	0\\
500	0\\
501	0\\
502	0\\
503	0\\
504	0\\
505	0\\
506	0\\
507	0\\
508	0\\
509	0\\
510	0\\
511	0\\
512	0\\
513	0\\
514	0\\
515	0\\
516	0\\
517	0\\
518	0\\
519	0\\
520	0\\
521	0\\
522	0\\
523	0\\
524	0\\
525	0\\
526	0\\
527	0\\
528	0\\
529	0\\
530	0\\
531	0\\
532	0\\
533	0\\
534	0\\
535	0\\
536	0\\
537	0\\
538	1.426719923304e-05\\
539	4.29091327027637e-05\\
540	7.42159152757329e-05\\
541	0.00010540163623223\\
542	0.000135128139185704\\
543	0.000163860507202848\\
544	0.000193019670144231\\
545	0.000222567303902726\\
546	0.00025268187255965\\
547	0.000283418338955152\\
548	0.000314798578510525\\
549	0.000346837628639188\\
550	0.000379549362854509\\
551	0.00041294395250958\\
552	0.0004470207297917\\
553	0.000481745886966618\\
554	0.000517094416528536\\
555	0.000553212180424865\\
556	0.000590118819429489\\
557	0.000630458343480549\\
558	0.000674248409785912\\
559	0.000711217113060013\\
560	0.000747899309859749\\
561	0.0007848972137292\\
562	0.000822585793724534\\
563	0.00086098759293346\\
564	0.000900118540435697\\
565	0.000939995557047767\\
566	0.000980636019795443\\
567	0.00138345405867184\\
568	0.0017080408379249\\
569	0.00179304145117328\\
570	0.00185663516672567\\
571	0.00192121866981793\\
572	0.00198681272153112\\
573	0.00205343750043299\\
574	0.00212111385614151\\
575	0.00218986333960979\\
576	0.00225970823202167\\
577	0.00233067157311048\\
578	0.00240277718883768\\
579	0.00247604971905621\\
580	0.00255051464798352\\
581	0.00262619834639323\\
582	0.00270312815070207\\
583	0.00278133254700611\\
584	0.00286084164033729\\
585	0.00294168838198347\\
586	0.00302391178876899\\
587	0.00310756536466313\\
588	0.00319273906110312\\
589	0.00327961639013215\\
590	0.00336862265377153\\
591	0.00346039684658503\\
592	0.00355653402929193\\
593	0.00366119131364763\\
594	0.00378527046594782\\
595	0.00395742119285503\\
596	0.00425297862742962\\
597	0.0048700418989439\\
598	0.00632942537858856\\
599	0\\
600	0\\
};
\addplot [color=black!50!mycolor20,solid,forget plot]
  table[row sep=crcr]{%
1	0\\
2	0\\
3	0\\
4	0\\
5	0\\
6	0\\
7	0\\
8	0\\
9	0\\
10	0\\
11	0\\
12	0\\
13	0\\
14	0\\
15	0\\
16	0\\
17	0\\
18	0\\
19	0\\
20	0\\
21	0\\
22	0\\
23	0\\
24	0\\
25	0\\
26	0\\
27	0\\
28	0\\
29	0\\
30	0\\
31	0\\
32	0\\
33	0\\
34	0\\
35	0\\
36	0\\
37	0\\
38	0\\
39	0\\
40	0\\
41	0\\
42	0\\
43	0\\
44	0\\
45	0\\
46	0\\
47	0\\
48	0\\
49	0\\
50	0\\
51	0\\
52	0\\
53	0\\
54	0\\
55	0\\
56	0\\
57	0\\
58	0\\
59	0\\
60	0\\
61	0\\
62	0\\
63	0\\
64	0\\
65	0\\
66	0\\
67	0\\
68	0\\
69	0\\
70	0\\
71	0\\
72	0\\
73	0\\
74	0\\
75	0\\
76	0\\
77	0\\
78	0\\
79	0\\
80	0\\
81	0\\
82	0\\
83	0\\
84	0\\
85	0\\
86	0\\
87	0\\
88	0\\
89	0\\
90	0\\
91	0\\
92	0\\
93	0\\
94	0\\
95	0\\
96	0\\
97	0\\
98	0\\
99	0\\
100	0\\
101	0\\
102	0\\
103	0\\
104	0\\
105	0\\
106	0\\
107	0\\
108	0\\
109	0\\
110	0\\
111	0\\
112	0\\
113	0\\
114	0\\
115	0\\
116	0\\
117	0\\
118	0\\
119	0\\
120	0\\
121	0\\
122	0\\
123	0\\
124	0\\
125	0\\
126	0\\
127	0\\
128	0\\
129	0\\
130	0\\
131	0\\
132	0\\
133	0\\
134	0\\
135	0\\
136	0\\
137	0\\
138	0\\
139	0\\
140	0\\
141	0\\
142	0\\
143	0\\
144	0\\
145	0\\
146	0\\
147	0\\
148	0\\
149	0\\
150	0\\
151	0\\
152	0\\
153	0\\
154	0\\
155	0\\
156	0\\
157	0\\
158	0\\
159	0\\
160	0\\
161	0\\
162	0\\
163	0\\
164	0\\
165	0\\
166	0\\
167	0\\
168	0\\
169	0\\
170	0\\
171	0\\
172	0\\
173	0\\
174	0\\
175	0\\
176	0\\
177	0\\
178	0\\
179	0\\
180	0\\
181	0\\
182	0\\
183	0\\
184	0\\
185	0\\
186	0\\
187	0\\
188	0\\
189	0\\
190	0\\
191	0\\
192	0\\
193	0\\
194	0\\
195	0\\
196	0\\
197	0\\
198	0\\
199	0\\
200	0\\
201	0\\
202	0\\
203	0\\
204	0\\
205	0\\
206	0\\
207	0\\
208	0\\
209	0\\
210	0\\
211	0\\
212	0\\
213	0\\
214	0\\
215	0\\
216	0\\
217	0\\
218	0\\
219	0\\
220	0\\
221	0\\
222	0\\
223	0\\
224	0\\
225	0\\
226	0\\
227	0\\
228	0\\
229	0\\
230	0\\
231	0\\
232	0\\
233	0\\
234	0\\
235	0\\
236	0\\
237	0\\
238	0\\
239	0\\
240	0\\
241	0\\
242	0\\
243	0\\
244	0\\
245	0\\
246	0\\
247	0\\
248	0\\
249	0\\
250	0\\
251	0\\
252	0\\
253	0\\
254	0\\
255	0\\
256	0\\
257	0\\
258	0\\
259	0\\
260	0\\
261	0\\
262	0\\
263	0\\
264	0\\
265	0\\
266	0\\
267	0\\
268	0\\
269	0\\
270	0\\
271	0\\
272	0\\
273	0\\
274	0\\
275	0\\
276	0\\
277	0\\
278	0\\
279	0\\
280	0\\
281	0\\
282	0\\
283	0\\
284	0\\
285	0\\
286	0\\
287	0\\
288	0\\
289	0\\
290	0\\
291	0\\
292	0\\
293	0\\
294	0\\
295	0\\
296	0\\
297	0\\
298	0\\
299	0\\
300	0\\
301	0\\
302	0\\
303	0\\
304	0\\
305	0\\
306	0\\
307	0\\
308	0\\
309	0\\
310	0\\
311	0\\
312	0\\
313	0\\
314	0\\
315	0\\
316	0\\
317	0\\
318	0\\
319	0\\
320	0\\
321	0\\
322	0\\
323	0\\
324	0\\
325	0\\
326	0\\
327	0\\
328	0\\
329	0\\
330	0\\
331	0\\
332	0\\
333	0\\
334	0\\
335	0\\
336	0\\
337	0\\
338	0\\
339	0\\
340	0\\
341	0\\
342	0\\
343	0\\
344	0\\
345	0\\
346	0\\
347	0\\
348	0\\
349	0\\
350	0\\
351	0\\
352	0\\
353	0\\
354	0\\
355	0\\
356	0\\
357	0\\
358	0\\
359	0\\
360	0\\
361	0\\
362	0\\
363	0\\
364	0\\
365	0\\
366	0\\
367	0\\
368	0\\
369	0\\
370	0\\
371	0\\
372	0\\
373	0\\
374	0\\
375	0\\
376	0\\
377	0\\
378	0\\
379	0\\
380	0\\
381	0\\
382	0\\
383	0\\
384	0\\
385	0\\
386	0\\
387	0\\
388	0\\
389	0\\
390	0\\
391	0\\
392	0\\
393	0\\
394	0\\
395	0\\
396	0\\
397	0\\
398	0\\
399	0\\
400	0\\
401	0\\
402	0\\
403	0\\
404	0\\
405	0\\
406	0\\
407	0\\
408	0\\
409	0\\
410	0\\
411	0\\
412	0\\
413	0\\
414	0\\
415	0\\
416	0\\
417	0\\
418	0\\
419	0\\
420	0\\
421	0\\
422	0\\
423	0\\
424	0\\
425	0\\
426	0\\
427	0\\
428	0\\
429	0\\
430	0\\
431	0\\
432	0\\
433	0\\
434	0\\
435	0\\
436	0\\
437	0\\
438	0\\
439	0\\
440	0\\
441	0\\
442	0\\
443	0\\
444	0\\
445	0\\
446	0\\
447	0\\
448	0\\
449	0\\
450	0\\
451	0\\
452	0\\
453	0\\
454	0\\
455	0\\
456	0\\
457	0\\
458	0\\
459	0\\
460	0\\
461	0\\
462	0\\
463	0\\
464	0\\
465	0\\
466	0\\
467	0\\
468	0\\
469	0\\
470	0\\
471	0\\
472	0\\
473	0\\
474	0\\
475	0\\
476	0\\
477	0\\
478	0\\
479	0\\
480	0\\
481	0\\
482	0\\
483	0\\
484	0\\
485	0\\
486	0\\
487	0\\
488	0\\
489	0\\
490	0\\
491	0\\
492	0\\
493	0\\
494	0\\
495	0\\
496	0\\
497	0\\
498	0\\
499	0\\
500	0\\
501	0\\
502	0\\
503	0\\
504	0\\
505	0\\
506	0\\
507	0\\
508	0\\
509	0\\
510	0\\
511	0\\
512	0\\
513	0\\
514	0\\
515	0\\
516	0\\
517	0\\
518	0\\
519	0\\
520	0\\
521	0\\
522	0\\
523	0\\
524	0\\
525	0\\
526	0\\
527	0\\
528	0\\
529	0\\
530	0\\
531	0\\
532	0\\
533	0\\
534	0\\
535	0\\
536	0\\
537	6.49828894175203e-07\\
538	3.05652458911548e-05\\
539	5.99404394527822e-05\\
540	8.71312739401948e-05\\
541	0.000114716903641183\\
542	0.000142674508084298\\
543	0.000171100011748483\\
544	0.000200107410759274\\
545	0.000229717106180836\\
546	0.000259944556361032\\
547	0.000290804095303178\\
548	0.000322309893047164\\
549	0.000354475852884199\\
550	0.000387314440674418\\
551	0.000420833252736105\\
552	0.000455025048030861\\
553	0.00048983964048552\\
554	0.000525318728056389\\
555	0.000561574556760999\\
556	0.000603919415462835\\
557	0.00064369719785649\\
558	0.000679340140340759\\
559	0.000715080552741755\\
560	0.000751442297042198\\
561	0.000788488188292752\\
562	0.000826233343400132\\
563	0.000864693097341739\\
564	0.000903883877005878\\
565	0.000943822831240413\\
566	0.00135108213105103\\
567	0.00165100703747856\\
568	0.00173042226223131\\
569	0.0017930426874416\\
570	0.00185663517077488\\
571	0.00192121867121681\\
572	0.00198681272222949\\
573	0.00205343750078598\\
574	0.00212111385631247\\
575	0.00218986333968779\\
576	0.00225970823205476\\
577	0.00233067157312331\\
578	0.00240277718884216\\
579	0.00247604971905759\\
580	0.00255051464798387\\
581	0.0026261983463933\\
582	0.00270312815070208\\
583	0.0027813325470061\\
584	0.00286084164033727\\
585	0.00294168838198346\\
586	0.00302391178876899\\
587	0.00310756536466312\\
588	0.00319273906110312\\
589	0.00327961639013215\\
590	0.00336862265377153\\
591	0.00346039684658503\\
592	0.00355653402929193\\
593	0.00366119131364762\\
594	0.00378527046594782\\
595	0.00395742119285503\\
596	0.00425297862742961\\
597	0.0048700418989439\\
598	0.00632942537858856\\
599	0\\
600	0\\
};
\addplot [color=black!60!mycolor21,solid,forget plot]
  table[row sep=crcr]{%
1	0\\
2	0\\
3	0\\
4	0\\
5	0\\
6	0\\
7	0\\
8	0\\
9	0\\
10	0\\
11	0\\
12	0\\
13	0\\
14	0\\
15	0\\
16	0\\
17	0\\
18	0\\
19	0\\
20	0\\
21	0\\
22	0\\
23	0\\
24	0\\
25	0\\
26	0\\
27	0\\
28	0\\
29	0\\
30	0\\
31	0\\
32	0\\
33	0\\
34	0\\
35	0\\
36	0\\
37	0\\
38	0\\
39	0\\
40	0\\
41	0\\
42	0\\
43	0\\
44	0\\
45	0\\
46	0\\
47	0\\
48	0\\
49	0\\
50	0\\
51	0\\
52	0\\
53	0\\
54	0\\
55	0\\
56	0\\
57	0\\
58	0\\
59	0\\
60	0\\
61	0\\
62	0\\
63	0\\
64	0\\
65	0\\
66	0\\
67	0\\
68	0\\
69	0\\
70	0\\
71	0\\
72	0\\
73	0\\
74	0\\
75	0\\
76	0\\
77	0\\
78	0\\
79	0\\
80	0\\
81	0\\
82	0\\
83	0\\
84	0\\
85	0\\
86	0\\
87	0\\
88	0\\
89	0\\
90	0\\
91	0\\
92	0\\
93	0\\
94	0\\
95	0\\
96	0\\
97	0\\
98	0\\
99	0\\
100	0\\
101	0\\
102	0\\
103	0\\
104	0\\
105	0\\
106	0\\
107	0\\
108	0\\
109	0\\
110	0\\
111	0\\
112	0\\
113	0\\
114	0\\
115	0\\
116	0\\
117	0\\
118	0\\
119	0\\
120	0\\
121	0\\
122	0\\
123	0\\
124	0\\
125	0\\
126	0\\
127	0\\
128	0\\
129	0\\
130	0\\
131	0\\
132	0\\
133	0\\
134	0\\
135	0\\
136	0\\
137	0\\
138	0\\
139	0\\
140	0\\
141	0\\
142	0\\
143	0\\
144	0\\
145	0\\
146	0\\
147	0\\
148	0\\
149	0\\
150	0\\
151	0\\
152	0\\
153	0\\
154	0\\
155	0\\
156	0\\
157	0\\
158	0\\
159	0\\
160	0\\
161	0\\
162	0\\
163	0\\
164	0\\
165	0\\
166	0\\
167	0\\
168	0\\
169	0\\
170	0\\
171	0\\
172	0\\
173	0\\
174	0\\
175	0\\
176	0\\
177	0\\
178	0\\
179	0\\
180	0\\
181	0\\
182	0\\
183	0\\
184	0\\
185	0\\
186	0\\
187	0\\
188	0\\
189	0\\
190	0\\
191	0\\
192	0\\
193	0\\
194	0\\
195	0\\
196	0\\
197	0\\
198	0\\
199	0\\
200	0\\
201	0\\
202	0\\
203	0\\
204	0\\
205	0\\
206	0\\
207	0\\
208	0\\
209	0\\
210	0\\
211	0\\
212	0\\
213	0\\
214	0\\
215	0\\
216	0\\
217	0\\
218	0\\
219	0\\
220	0\\
221	0\\
222	0\\
223	0\\
224	0\\
225	0\\
226	0\\
227	0\\
228	0\\
229	0\\
230	0\\
231	0\\
232	0\\
233	0\\
234	0\\
235	0\\
236	0\\
237	0\\
238	0\\
239	0\\
240	0\\
241	0\\
242	0\\
243	0\\
244	0\\
245	0\\
246	0\\
247	0\\
248	0\\
249	0\\
250	0\\
251	0\\
252	0\\
253	0\\
254	0\\
255	0\\
256	0\\
257	0\\
258	0\\
259	0\\
260	0\\
261	0\\
262	0\\
263	0\\
264	0\\
265	0\\
266	0\\
267	0\\
268	0\\
269	0\\
270	0\\
271	0\\
272	0\\
273	0\\
274	0\\
275	0\\
276	0\\
277	0\\
278	0\\
279	0\\
280	0\\
281	0\\
282	0\\
283	0\\
284	0\\
285	0\\
286	0\\
287	0\\
288	0\\
289	0\\
290	0\\
291	0\\
292	0\\
293	0\\
294	0\\
295	0\\
296	0\\
297	0\\
298	0\\
299	0\\
300	0\\
301	0\\
302	0\\
303	0\\
304	0\\
305	0\\
306	0\\
307	0\\
308	0\\
309	0\\
310	0\\
311	0\\
312	0\\
313	0\\
314	0\\
315	0\\
316	0\\
317	0\\
318	0\\
319	0\\
320	0\\
321	0\\
322	0\\
323	0\\
324	0\\
325	0\\
326	0\\
327	0\\
328	0\\
329	0\\
330	0\\
331	0\\
332	0\\
333	0\\
334	0\\
335	0\\
336	0\\
337	0\\
338	0\\
339	0\\
340	0\\
341	0\\
342	0\\
343	0\\
344	0\\
345	0\\
346	0\\
347	0\\
348	0\\
349	0\\
350	0\\
351	0\\
352	0\\
353	0\\
354	0\\
355	0\\
356	0\\
357	0\\
358	0\\
359	0\\
360	0\\
361	0\\
362	0\\
363	0\\
364	0\\
365	0\\
366	0\\
367	0\\
368	0\\
369	0\\
370	0\\
371	0\\
372	0\\
373	0\\
374	0\\
375	0\\
376	0\\
377	0\\
378	0\\
379	0\\
380	0\\
381	0\\
382	0\\
383	0\\
384	0\\
385	0\\
386	0\\
387	0\\
388	0\\
389	0\\
390	0\\
391	0\\
392	0\\
393	0\\
394	0\\
395	0\\
396	0\\
397	0\\
398	0\\
399	0\\
400	0\\
401	0\\
402	0\\
403	0\\
404	0\\
405	0\\
406	0\\
407	0\\
408	0\\
409	0\\
410	0\\
411	0\\
412	0\\
413	0\\
414	0\\
415	0\\
416	0\\
417	0\\
418	0\\
419	0\\
420	0\\
421	0\\
422	0\\
423	0\\
424	0\\
425	0\\
426	0\\
427	0\\
428	0\\
429	0\\
430	0\\
431	0\\
432	0\\
433	0\\
434	0\\
435	0\\
436	0\\
437	0\\
438	0\\
439	0\\
440	0\\
441	0\\
442	0\\
443	0\\
444	0\\
445	0\\
446	0\\
447	0\\
448	0\\
449	0\\
450	0\\
451	0\\
452	0\\
453	0\\
454	0\\
455	0\\
456	0\\
457	0\\
458	0\\
459	0\\
460	0\\
461	0\\
462	0\\
463	0\\
464	0\\
465	0\\
466	0\\
467	0\\
468	0\\
469	0\\
470	0\\
471	0\\
472	0\\
473	0\\
474	0\\
475	0\\
476	0\\
477	0\\
478	0\\
479	0\\
480	0\\
481	0\\
482	0\\
483	0\\
484	0\\
485	0\\
486	0\\
487	0\\
488	0\\
489	0\\
490	0\\
491	0\\
492	0\\
493	0\\
494	0\\
495	0\\
496	0\\
497	0\\
498	0\\
499	0\\
500	0\\
501	0\\
502	0\\
503	0\\
504	0\\
505	0\\
506	0\\
507	0\\
508	0\\
509	0\\
510	0\\
511	0\\
512	0\\
513	0\\
514	0\\
515	0\\
516	0\\
517	0\\
518	0\\
519	0\\
520	0\\
521	0\\
522	0\\
523	0\\
524	0\\
525	0\\
526	0\\
527	0\\
528	0\\
529	0\\
530	0\\
531	0\\
532	0\\
533	0\\
534	0\\
535	0\\
536	0\\
537	1.41981037307091e-05\\
538	4.03194267452914e-05\\
539	6.68049003892867e-05\\
540	9.36455912896163e-05\\
541	0.000121031570767503\\
542	0.000148981487706946\\
543	0.000177511834873603\\
544	0.000206636133043701\\
545	0.00023636795016899\\
546	0.000266721132093178\\
547	0.000297709825439374\\
548	0.000329348337773177\\
549	0.000361650643511393\\
550	0.000394628923195159\\
551	0.000428289376922921\\
552	0.000462620483140471\\
553	0.000497556235318828\\
554	0.000534404821662524\\
555	0.00057682151016282\\
556	0.000612670727538693\\
557	0.000647448601386753\\
558	0.000682535085499909\\
559	0.000718276611240496\\
560	0.000754689045844615\\
561	0.000791786693129664\\
562	0.000829584407608802\\
563	0.000868098067449268\\
564	0.00090734463744055\\
565	0.00128398061588739\\
566	0.00159228887629074\\
567	0.00166875568695174\\
568	0.00173042230411937\\
569	0.00179304268794214\\
570	0.00185663517097841\\
571	0.00192121867131843\\
572	0.00198681272227957\\
573	0.0020534375008095\\
574	0.00212111385632287\\
575	0.00218986333969204\\
576	0.00225970823205636\\
577	0.00233067157312388\\
578	0.00240277718884235\\
579	0.00247604971905763\\
580	0.00255051464798387\\
581	0.00262619834639329\\
582	0.00270312815070207\\
583	0.00278133254700611\\
584	0.00286084164033729\\
585	0.00294168838198347\\
586	0.00302391178876898\\
587	0.00310756536466311\\
588	0.0031927390611031\\
589	0.00327961639013215\\
590	0.00336862265377153\\
591	0.00346039684658503\\
592	0.00355653402929193\\
593	0.00366119131364762\\
594	0.00378527046594782\\
595	0.00395742119285503\\
596	0.00425297862742963\\
597	0.0048700418989439\\
598	0.00632942537858856\\
599	0\\
600	0\\
};
\addplot [color=black!80!mycolor21,solid,forget plot]
  table[row sep=crcr]{%
1	0\\
2	0\\
3	0\\
4	0\\
5	0\\
6	0\\
7	0\\
8	0\\
9	0\\
10	0\\
11	0\\
12	0\\
13	0\\
14	0\\
15	0\\
16	0\\
17	0\\
18	0\\
19	0\\
20	0\\
21	0\\
22	0\\
23	0\\
24	0\\
25	0\\
26	0\\
27	0\\
28	0\\
29	0\\
30	0\\
31	0\\
32	0\\
33	0\\
34	0\\
35	0\\
36	0\\
37	0\\
38	0\\
39	0\\
40	0\\
41	0\\
42	0\\
43	0\\
44	0\\
45	0\\
46	0\\
47	0\\
48	0\\
49	0\\
50	0\\
51	0\\
52	0\\
53	0\\
54	0\\
55	0\\
56	0\\
57	0\\
58	0\\
59	0\\
60	0\\
61	0\\
62	0\\
63	0\\
64	0\\
65	0\\
66	0\\
67	0\\
68	0\\
69	0\\
70	0\\
71	0\\
72	0\\
73	0\\
74	0\\
75	0\\
76	0\\
77	0\\
78	0\\
79	0\\
80	0\\
81	0\\
82	0\\
83	0\\
84	0\\
85	0\\
86	0\\
87	0\\
88	0\\
89	0\\
90	0\\
91	0\\
92	0\\
93	0\\
94	0\\
95	0\\
96	0\\
97	0\\
98	0\\
99	0\\
100	0\\
101	0\\
102	0\\
103	0\\
104	0\\
105	0\\
106	0\\
107	0\\
108	0\\
109	0\\
110	0\\
111	0\\
112	0\\
113	0\\
114	0\\
115	0\\
116	0\\
117	0\\
118	0\\
119	0\\
120	0\\
121	0\\
122	0\\
123	0\\
124	0\\
125	0\\
126	0\\
127	0\\
128	0\\
129	0\\
130	0\\
131	0\\
132	0\\
133	0\\
134	0\\
135	0\\
136	0\\
137	0\\
138	0\\
139	0\\
140	0\\
141	0\\
142	0\\
143	0\\
144	0\\
145	0\\
146	0\\
147	0\\
148	0\\
149	0\\
150	0\\
151	0\\
152	0\\
153	0\\
154	0\\
155	0\\
156	0\\
157	0\\
158	0\\
159	0\\
160	0\\
161	0\\
162	0\\
163	0\\
164	0\\
165	0\\
166	0\\
167	0\\
168	0\\
169	0\\
170	0\\
171	0\\
172	0\\
173	0\\
174	0\\
175	0\\
176	0\\
177	0\\
178	0\\
179	0\\
180	0\\
181	0\\
182	0\\
183	0\\
184	0\\
185	0\\
186	0\\
187	0\\
188	0\\
189	0\\
190	0\\
191	0\\
192	0\\
193	0\\
194	0\\
195	0\\
196	0\\
197	0\\
198	0\\
199	0\\
200	0\\
201	0\\
202	0\\
203	0\\
204	0\\
205	0\\
206	0\\
207	0\\
208	0\\
209	0\\
210	0\\
211	0\\
212	0\\
213	0\\
214	0\\
215	0\\
216	0\\
217	0\\
218	0\\
219	0\\
220	0\\
221	0\\
222	0\\
223	0\\
224	0\\
225	0\\
226	0\\
227	0\\
228	0\\
229	0\\
230	0\\
231	0\\
232	0\\
233	0\\
234	0\\
235	0\\
236	0\\
237	0\\
238	0\\
239	0\\
240	0\\
241	0\\
242	0\\
243	0\\
244	0\\
245	0\\
246	0\\
247	0\\
248	0\\
249	0\\
250	0\\
251	0\\
252	0\\
253	0\\
254	0\\
255	0\\
256	0\\
257	0\\
258	0\\
259	0\\
260	0\\
261	0\\
262	0\\
263	0\\
264	0\\
265	0\\
266	0\\
267	0\\
268	0\\
269	0\\
270	0\\
271	0\\
272	0\\
273	0\\
274	0\\
275	0\\
276	0\\
277	0\\
278	0\\
279	0\\
280	0\\
281	0\\
282	0\\
283	0\\
284	0\\
285	0\\
286	0\\
287	0\\
288	0\\
289	0\\
290	0\\
291	0\\
292	0\\
293	0\\
294	0\\
295	0\\
296	0\\
297	0\\
298	0\\
299	0\\
300	0\\
301	0\\
302	0\\
303	0\\
304	0\\
305	0\\
306	0\\
307	0\\
308	0\\
309	0\\
310	0\\
311	0\\
312	0\\
313	0\\
314	0\\
315	0\\
316	0\\
317	0\\
318	0\\
319	0\\
320	0\\
321	0\\
322	0\\
323	0\\
324	0\\
325	0\\
326	0\\
327	0\\
328	0\\
329	0\\
330	0\\
331	0\\
332	0\\
333	0\\
334	0\\
335	0\\
336	0\\
337	0\\
338	0\\
339	0\\
340	0\\
341	0\\
342	0\\
343	0\\
344	0\\
345	0\\
346	0\\
347	0\\
348	0\\
349	0\\
350	0\\
351	0\\
352	0\\
353	0\\
354	0\\
355	0\\
356	0\\
357	0\\
358	0\\
359	0\\
360	0\\
361	0\\
362	0\\
363	0\\
364	0\\
365	0\\
366	0\\
367	0\\
368	0\\
369	0\\
370	0\\
371	0\\
372	0\\
373	0\\
374	0\\
375	0\\
376	0\\
377	0\\
378	0\\
379	0\\
380	0\\
381	0\\
382	0\\
383	0\\
384	0\\
385	0\\
386	0\\
387	0\\
388	0\\
389	0\\
390	0\\
391	0\\
392	0\\
393	0\\
394	0\\
395	0\\
396	0\\
397	0\\
398	0\\
399	0\\
400	0\\
401	0\\
402	0\\
403	0\\
404	0\\
405	0\\
406	0\\
407	0\\
408	0\\
409	0\\
410	0\\
411	0\\
412	0\\
413	0\\
414	0\\
415	0\\
416	0\\
417	0\\
418	0\\
419	0\\
420	0\\
421	0\\
422	0\\
423	0\\
424	0\\
425	0\\
426	0\\
427	0\\
428	0\\
429	0\\
430	0\\
431	0\\
432	0\\
433	0\\
434	0\\
435	0\\
436	0\\
437	0\\
438	0\\
439	0\\
440	0\\
441	0\\
442	0\\
443	0\\
444	0\\
445	0\\
446	0\\
447	0\\
448	0\\
449	0\\
450	0\\
451	0\\
452	0\\
453	0\\
454	0\\
455	0\\
456	0\\
457	0\\
458	0\\
459	0\\
460	0\\
461	0\\
462	0\\
463	0\\
464	0\\
465	0\\
466	0\\
467	0\\
468	0\\
469	0\\
470	0\\
471	0\\
472	0\\
473	0\\
474	0\\
475	0\\
476	0\\
477	0\\
478	0\\
479	0\\
480	0\\
481	0\\
482	0\\
483	0\\
484	0\\
485	0\\
486	0\\
487	0\\
488	0\\
489	0\\
490	0\\
491	0\\
492	0\\
493	0\\
494	0\\
495	0\\
496	0\\
497	0\\
498	0\\
499	0\\
500	0\\
501	0\\
502	0\\
503	0\\
504	0\\
505	0\\
506	0\\
507	0\\
508	0\\
509	0\\
510	0\\
511	0\\
512	0\\
513	0\\
514	0\\
515	0\\
516	0\\
517	0\\
518	0\\
519	0\\
520	0\\
521	0\\
522	0\\
523	0\\
524	0\\
525	0\\
526	0\\
527	0\\
528	0\\
529	0\\
530	0\\
531	0\\
532	0\\
533	0\\
534	0\\
535	0\\
536	0\\
537	2.00311818478477e-05\\
538	4.58947831388607e-05\\
539	7.22859642090788e-05\\
540	9.92227167136692e-05\\
541	0.000126717645174376\\
542	0.000154783396446169\\
543	0.000183432792668513\\
544	0.000212678988439243\\
545	0.000242535495908309\\
546	0.000273016183234799\\
547	0.000304135226905855\\
548	0.000335906938006349\\
549	0.000368345226021222\\
550	0.000401462007738327\\
551	0.00043526250918298\\
552	0.00046973121451101\\
553	0.000507806233402199\\
554	0.000547931958570056\\
555	0.000581827365374636\\
556	0.000615782626155859\\
557	0.000650269640106297\\
558	0.000685400368111508\\
559	0.000721188332977737\\
560	0.000757647340897565\\
561	0.00079479173610764\\
562	0.000832636728120108\\
563	0.000871198966466937\\
564	0.00118295200706542\\
565	0.00153193885094584\\
566	0.00160802507255323\\
567	0.00166875568906997\\
568	0.00173042230418516\\
569	0.0017930426879713\\
570	0.00185663517099282\\
571	0.00192121867132536\\
572	0.00198681272228273\\
573	0.00205343750081087\\
574	0.00212111385632341\\
575	0.00218986333969228\\
576	0.00225970823205644\\
577	0.0023306715731239\\
578	0.00240277718884234\\
579	0.00247604971905764\\
580	0.00255051464798387\\
581	0.00262619834639331\\
582	0.00270312815070208\\
583	0.00278133254700611\\
584	0.00286084164033728\\
585	0.00294168838198347\\
586	0.003023911788769\\
587	0.00310756536466313\\
588	0.00319273906110312\\
589	0.00327961639013215\\
590	0.00336862265377153\\
591	0.00346039684658503\\
592	0.00355653402929195\\
593	0.00366119131364764\\
594	0.00378527046594783\\
595	0.00395742119285504\\
596	0.00425297862742962\\
597	0.00487004189894391\\
598	0.00632942537858856\\
599	0\\
600	0\\
};
\addplot [color=black,solid,forget plot]
  table[row sep=crcr]{%
1	0\\
2	0\\
3	0\\
4	0\\
5	0\\
6	0\\
7	0\\
8	0\\
9	0\\
10	0\\
11	0\\
12	0\\
13	0\\
14	0\\
15	0\\
16	0\\
17	0\\
18	0\\
19	0\\
20	0\\
21	0\\
22	0\\
23	0\\
24	0\\
25	0\\
26	0\\
27	0\\
28	0\\
29	0\\
30	0\\
31	0\\
32	0\\
33	0\\
34	0\\
35	0\\
36	0\\
37	0\\
38	0\\
39	0\\
40	0\\
41	0\\
42	0\\
43	0\\
44	0\\
45	0\\
46	0\\
47	0\\
48	0\\
49	0\\
50	0\\
51	0\\
52	0\\
53	0\\
54	0\\
55	0\\
56	0\\
57	0\\
58	0\\
59	0\\
60	0\\
61	0\\
62	0\\
63	0\\
64	0\\
65	0\\
66	0\\
67	0\\
68	0\\
69	0\\
70	0\\
71	0\\
72	0\\
73	0\\
74	0\\
75	0\\
76	0\\
77	0\\
78	0\\
79	0\\
80	0\\
81	0\\
82	0\\
83	0\\
84	0\\
85	0\\
86	0\\
87	0\\
88	0\\
89	0\\
90	0\\
91	0\\
92	0\\
93	0\\
94	0\\
95	0\\
96	0\\
97	0\\
98	0\\
99	0\\
100	0\\
101	0\\
102	0\\
103	0\\
104	0\\
105	0\\
106	0\\
107	0\\
108	0\\
109	0\\
110	0\\
111	0\\
112	0\\
113	0\\
114	0\\
115	0\\
116	0\\
117	0\\
118	0\\
119	0\\
120	0\\
121	0\\
122	0\\
123	0\\
124	0\\
125	0\\
126	0\\
127	0\\
128	0\\
129	0\\
130	0\\
131	0\\
132	0\\
133	0\\
134	0\\
135	0\\
136	0\\
137	0\\
138	0\\
139	0\\
140	0\\
141	0\\
142	0\\
143	0\\
144	0\\
145	0\\
146	0\\
147	0\\
148	0\\
149	0\\
150	0\\
151	0\\
152	0\\
153	0\\
154	0\\
155	0\\
156	0\\
157	0\\
158	0\\
159	0\\
160	0\\
161	0\\
162	0\\
163	0\\
164	0\\
165	0\\
166	0\\
167	0\\
168	0\\
169	0\\
170	0\\
171	0\\
172	0\\
173	0\\
174	0\\
175	0\\
176	0\\
177	0\\
178	0\\
179	0\\
180	0\\
181	0\\
182	0\\
183	0\\
184	0\\
185	0\\
186	0\\
187	0\\
188	0\\
189	0\\
190	0\\
191	0\\
192	0\\
193	0\\
194	0\\
195	0\\
196	0\\
197	0\\
198	0\\
199	0\\
200	0\\
201	0\\
202	0\\
203	0\\
204	0\\
205	0\\
206	0\\
207	0\\
208	0\\
209	0\\
210	0\\
211	0\\
212	0\\
213	0\\
214	0\\
215	0\\
216	0\\
217	0\\
218	0\\
219	0\\
220	0\\
221	0\\
222	0\\
223	0\\
224	0\\
225	0\\
226	0\\
227	0\\
228	0\\
229	0\\
230	0\\
231	0\\
232	0\\
233	0\\
234	0\\
235	0\\
236	0\\
237	0\\
238	0\\
239	0\\
240	0\\
241	0\\
242	0\\
243	0\\
244	0\\
245	0\\
246	0\\
247	0\\
248	0\\
249	0\\
250	0\\
251	0\\
252	0\\
253	0\\
254	0\\
255	0\\
256	0\\
257	0\\
258	0\\
259	0\\
260	0\\
261	0\\
262	0\\
263	0\\
264	0\\
265	0\\
266	0\\
267	0\\
268	0\\
269	0\\
270	0\\
271	0\\
272	0\\
273	0\\
274	0\\
275	0\\
276	0\\
277	0\\
278	0\\
279	0\\
280	0\\
281	0\\
282	0\\
283	0\\
284	0\\
285	0\\
286	0\\
287	0\\
288	0\\
289	0\\
290	0\\
291	0\\
292	0\\
293	0\\
294	0\\
295	0\\
296	0\\
297	0\\
298	0\\
299	0\\
300	0\\
301	0\\
302	0\\
303	0\\
304	0\\
305	0\\
306	0\\
307	0\\
308	0\\
309	0\\
310	0\\
311	0\\
312	0\\
313	0\\
314	0\\
315	0\\
316	0\\
317	0\\
318	0\\
319	0\\
320	0\\
321	0\\
322	0\\
323	0\\
324	0\\
325	0\\
326	0\\
327	0\\
328	0\\
329	0\\
330	0\\
331	0\\
332	0\\
333	0\\
334	0\\
335	0\\
336	0\\
337	0\\
338	0\\
339	0\\
340	0\\
341	0\\
342	0\\
343	0\\
344	0\\
345	0\\
346	0\\
347	0\\
348	0\\
349	0\\
350	0\\
351	0\\
352	0\\
353	0\\
354	0\\
355	0\\
356	0\\
357	0\\
358	0\\
359	0\\
360	0\\
361	0\\
362	0\\
363	0\\
364	0\\
365	0\\
366	0\\
367	0\\
368	0\\
369	0\\
370	0\\
371	0\\
372	0\\
373	0\\
374	0\\
375	0\\
376	0\\
377	0\\
378	0\\
379	0\\
380	0\\
381	0\\
382	0\\
383	0\\
384	0\\
385	0\\
386	0\\
387	0\\
388	0\\
389	0\\
390	0\\
391	0\\
392	0\\
393	0\\
394	0\\
395	0\\
396	0\\
397	0\\
398	0\\
399	0\\
400	0\\
401	0\\
402	0\\
403	0\\
404	0\\
405	0\\
406	0\\
407	0\\
408	0\\
409	0\\
410	0\\
411	0\\
412	0\\
413	0\\
414	0\\
415	0\\
416	0\\
417	0\\
418	0\\
419	0\\
420	0\\
421	0\\
422	0\\
423	0\\
424	0\\
425	0\\
426	0\\
427	0\\
428	0\\
429	0\\
430	0\\
431	0\\
432	0\\
433	0\\
434	0\\
435	0\\
436	0\\
437	0\\
438	0\\
439	0\\
440	0\\
441	0\\
442	0\\
443	0\\
444	0\\
445	0\\
446	0\\
447	0\\
448	0\\
449	0\\
450	0\\
451	0\\
452	0\\
453	0\\
454	0\\
455	0\\
456	0\\
457	0\\
458	0\\
459	0\\
460	0\\
461	0\\
462	0\\
463	0\\
464	0\\
465	0\\
466	0\\
467	0\\
468	0\\
469	0\\
470	0\\
471	0\\
472	0\\
473	0\\
474	0\\
475	0\\
476	0\\
477	0\\
478	0\\
479	0\\
480	0\\
481	0\\
482	0\\
483	0\\
484	0\\
485	0\\
486	0\\
487	0\\
488	0\\
489	0\\
490	0\\
491	0\\
492	0\\
493	0\\
494	0\\
495	0\\
496	0\\
497	0\\
498	0\\
499	0\\
500	0\\
501	0\\
502	0\\
503	0\\
504	0\\
505	0\\
506	0\\
507	0\\
508	0\\
509	0\\
510	0\\
511	0\\
512	0\\
513	0\\
514	0\\
515	0\\
516	0\\
517	0\\
518	0\\
519	0\\
520	0\\
521	0\\
522	0\\
523	0\\
524	0\\
525	0\\
526	0\\
527	0\\
528	0\\
529	0\\
530	0\\
531	0\\
532	0\\
533	0\\
534	0\\
535	0\\
536	0\\
537	2.41449941958518e-05\\
538	5.01062475752778e-05\\
539	7.66056323086619e-05\\
540	0.000103655192469625\\
541	0.000131267283450877\\
542	0.000159454603022728\\
543	0.000188230216867344\\
544	0.0002176075760922\\
545	0.000247600530527473\\
546	0.000278223330210997\\
547	0.000309490592847296\\
548	0.000341417179393022\\
549	0.000374017835784068\\
550	0.000407306284469521\\
551	0.000441293207629602\\
552	0.000480387614922689\\
553	0.000517733822270981\\
554	0.000550900656801395\\
555	0.000584195800718177\\
556	0.000618106994392449\\
557	0.000652650864010514\\
558	0.000687840631956863\\
559	0.000723689894931791\\
560	0.000760212780169657\\
561	0.000797424370009976\\
562	0.000835341903111315\\
563	0.00104794067987499\\
564	0.0014700571270915\\
565	0.00154821321979401\\
566	0.00160802507270126\\
567	0.00166875568907878\\
568	0.00173042230418922\\
569	0.00179304268797327\\
570	0.00185663517099374\\
571	0.00192121867132575\\
572	0.00198681272228291\\
573	0.00205343750081092\\
574	0.00212111385632343\\
575	0.00218986333969225\\
576	0.00225970823205644\\
577	0.0023306715731239\\
578	0.00240277718884235\\
579	0.00247604971905765\\
580	0.00255051464798389\\
581	0.0026261983463933\\
582	0.00270312815070208\\
583	0.00278133254700612\\
584	0.00286084164033729\\
585	0.00294168838198346\\
586	0.00302391178876899\\
587	0.00310756536466312\\
588	0.00319273906110312\\
589	0.00327961639013216\\
590	0.00336862265377154\\
591	0.00346039684658502\\
592	0.00355653402929193\\
593	0.00366119131364762\\
594	0.00378527046594782\\
595	0.00395742119285502\\
596	0.00425297862742962\\
597	0.0048700418989439\\
598	0.00632942537858856\\
599	0\\
600	0\\
};
\end{axis}
\end{tikzpicture}%
 
  \caption{Discrete Time}
\end{subfigure}\\
\vspace{1cm}
\begin{subfigure}{.45\linewidth}
  \centering
  \setlength\figureheight{\linewidth} 
  \setlength\figurewidth{\linewidth}
  \tikzsetnextfilename{dp_colorbar/dp_cts_nFPC_z15}
  % This file was created by matlab2tikz.
%
%The latest updates can be retrieved from
%  http://www.mathworks.com/matlabcentral/fileexchange/22022-matlab2tikz-matlab2tikz
%where you can also make suggestions and rate matlab2tikz.
%
\definecolor{mycolor1}{rgb}{0.00000,1.00000,0.14286}%
\definecolor{mycolor2}{rgb}{0.00000,1.00000,0.28571}%
\definecolor{mycolor3}{rgb}{0.00000,1.00000,0.42857}%
\definecolor{mycolor4}{rgb}{0.00000,1.00000,0.57143}%
\definecolor{mycolor5}{rgb}{0.00000,1.00000,0.71429}%
\definecolor{mycolor6}{rgb}{0.00000,1.00000,0.85714}%
\definecolor{mycolor7}{rgb}{0.00000,1.00000,1.00000}%
\definecolor{mycolor8}{rgb}{0.00000,0.87500,1.00000}%
\definecolor{mycolor9}{rgb}{0.00000,0.62500,1.00000}%
\definecolor{mycolor10}{rgb}{0.12500,0.00000,1.00000}%
\definecolor{mycolor11}{rgb}{0.25000,0.00000,1.00000}%
\definecolor{mycolor12}{rgb}{0.37500,0.00000,1.00000}%
\definecolor{mycolor13}{rgb}{0.50000,0.00000,1.00000}%
\definecolor{mycolor14}{rgb}{0.62500,0.00000,1.00000}%
\definecolor{mycolor15}{rgb}{0.75000,0.00000,1.00000}%
\definecolor{mycolor16}{rgb}{0.87500,0.00000,1.00000}%
\definecolor{mycolor17}{rgb}{1.00000,0.00000,1.00000}%
\definecolor{mycolor18}{rgb}{1.00000,0.00000,0.87500}%
\definecolor{mycolor19}{rgb}{1.00000,0.00000,0.62500}%
\definecolor{mycolor20}{rgb}{0.85714,0.00000,0.00000}%
\definecolor{mycolor21}{rgb}{0.71429,0.00000,0.00000}%
%
\begin{tikzpicture}[trim axis left, trim axis right]

\begin{axis}[%
width=\figurewidth,
height=\figureheight,
at={(0\figurewidth,0\figureheight)},
scale only axis,
every outer x axis line/.append style={black},
every x tick label/.append style={font=\color{black}},
xmin=0,
xmax=600,
every outer y axis line/.append style={black},
every y tick label/.append style={font=\color{black}},
ymin=0,
ymax=0.014,
axis background/.style={fill=white},
axis x line*=bottom,
axis y line*=left,
yticklabel style={
        /pgf/number format/fixed,
        /pgf/number format/precision=3
},
scaled y ticks=false
]
\addplot [color=green,solid,forget plot]
  table[row sep=crcr]{%
0.01	0.01\\
1.01	0.01\\
2.01	0.01\\
3.01	0.01\\
4.01	0.01\\
5.01	0.01\\
6.01	0.01\\
7.01	0.01\\
8.01	0.01\\
9.01	0.01\\
10.01	0.01\\
11.01	0.01\\
12.01	0.01\\
13.01	0.01\\
14.01	0.01\\
15.01	0.01\\
16.01	0.01\\
17.01	0.01\\
18.01	0.01\\
19.01	0.01\\
20.01	0.01\\
21.01	0.01\\
22.01	0.01\\
23.01	0.01\\
24.01	0.01\\
25.01	0.01\\
26.01	0.01\\
27.01	0.01\\
28.01	0.01\\
29.01	0.01\\
30.01	0.01\\
31.01	0.01\\
32.01	0.01\\
33.01	0.01\\
34.01	0.01\\
35.01	0.01\\
36.01	0.01\\
37.01	0.01\\
38.01	0.01\\
39.01	0.01\\
40.01	0.01\\
41.01	0.01\\
42.01	0.01\\
43.01	0.01\\
44.01	0.01\\
45.01	0.01\\
46.01	0.01\\
47.01	0.01\\
48.01	0.01\\
49.01	0.01\\
50.01	0.01\\
51.01	0.01\\
52.01	0.01\\
53.01	0.01\\
54.01	0.01\\
55.01	0.01\\
56.01	0.01\\
57.01	0.01\\
58.01	0.01\\
59.01	0.01\\
60.01	0.01\\
61.01	0.01\\
62.01	0.01\\
63.01	0.01\\
64.01	0.01\\
65.01	0.01\\
66.01	0.01\\
67.01	0.01\\
68.01	0.01\\
69.01	0.01\\
70.01	0.01\\
71.01	0.01\\
72.01	0.01\\
73.01	0.01\\
74.01	0.01\\
75.01	0.01\\
76.01	0.01\\
77.01	0.01\\
78.01	0.01\\
79.01	0.01\\
80.01	0.01\\
81.01	0.01\\
82.01	0.01\\
83.01	0.01\\
84.01	0.01\\
85.01	0.01\\
86.01	0.01\\
87.01	0.01\\
88.01	0.01\\
89.01	0.01\\
90.01	0.01\\
91.01	0.01\\
92.01	0.01\\
93.01	0.01\\
94.01	0.01\\
95.01	0.01\\
96.01	0.01\\
97.01	0.01\\
98.01	0.01\\
99.01	0.01\\
100.01	0.01\\
101.01	0.01\\
102.01	0.01\\
103.01	0.01\\
104.01	0.01\\
105.01	0.01\\
106.01	0.01\\
107.01	0.01\\
108.01	0.01\\
109.01	0.01\\
110.01	0.01\\
111.01	0.01\\
112.01	0.01\\
113.01	0.01\\
114.01	0.01\\
115.01	0.01\\
116.01	0.01\\
117.01	0.01\\
118.01	0.01\\
119.01	0.01\\
120.01	0.01\\
121.01	0.01\\
122.01	0.01\\
123.01	0.01\\
124.01	0.01\\
125.01	0.01\\
126.01	0.01\\
127.01	0.01\\
128.01	0.01\\
129.01	0.01\\
130.01	0.01\\
131.01	0.01\\
132.01	0.01\\
133.01	0.01\\
134.01	0.01\\
135.01	0.01\\
136.01	0.01\\
137.01	0.01\\
138.01	0.01\\
139.01	0.01\\
140.01	0.01\\
141.01	0.01\\
142.01	0.01\\
143.01	0.01\\
144.01	0.01\\
145.01	0.01\\
146.01	0.01\\
147.01	0.01\\
148.01	0.01\\
149.01	0.01\\
150.01	0.01\\
151.01	0.01\\
152.01	0.01\\
153.01	0.01\\
154.01	0.01\\
155.01	0.01\\
156.01	0.01\\
157.01	0.01\\
158.01	0.01\\
159.01	0.01\\
160.01	0.01\\
161.01	0.01\\
162.01	0.01\\
163.01	0.01\\
164.01	0.01\\
165.01	0.01\\
166.01	0.01\\
167.01	0.01\\
168.01	0.01\\
169.01	0.01\\
170.01	0.01\\
171.01	0.01\\
172.01	0.01\\
173.01	0.01\\
174.01	0.01\\
175.01	0.01\\
176.01	0.01\\
177.01	0.01\\
178.01	0.01\\
179.01	0.01\\
180.01	0.01\\
181.01	0.01\\
182.01	0.01\\
183.01	0.01\\
184.01	0.01\\
185.01	0.01\\
186.01	0.01\\
187.01	0.01\\
188.01	0.01\\
189.01	0.01\\
190.01	0.01\\
191.01	0.01\\
192.01	0.01\\
193.01	0.01\\
194.01	0.01\\
195.01	0.01\\
196.01	0.01\\
197.01	0.01\\
198.01	0.01\\
199.01	0.01\\
200.01	0.01\\
201.01	0.01\\
202.01	0.01\\
203.01	0.01\\
204.01	0.01\\
205.01	0.01\\
206.01	0.01\\
207.01	0.01\\
208.01	0.01\\
209.01	0.01\\
210.01	0.01\\
211.01	0.01\\
212.01	0.01\\
213.01	0.01\\
214.01	0.01\\
215.01	0.01\\
216.01	0.01\\
217.01	0.01\\
218.01	0.01\\
219.01	0.01\\
220.01	0.01\\
221.01	0.01\\
222.01	0.01\\
223.01	0.01\\
224.01	0.01\\
225.01	0.01\\
226.01	0.01\\
227.01	0.01\\
228.01	0.01\\
229.01	0.01\\
230.01	0.01\\
231.01	0.01\\
232.01	0.01\\
233.01	0.01\\
234.01	0.01\\
235.01	0.01\\
236.01	0.01\\
237.01	0.01\\
238.01	0.01\\
239.01	0.01\\
240.01	0.01\\
241.01	0.01\\
242.01	0.01\\
243.01	0.01\\
244.01	0.01\\
245.01	0.01\\
246.01	0.01\\
247.01	0.01\\
248.01	0.01\\
249.01	0.01\\
250.01	0.01\\
251.01	0.01\\
252.01	0.01\\
253.01	0.01\\
254.01	0.01\\
255.01	0.01\\
256.01	0.01\\
257.01	0.01\\
258.01	0.01\\
259.01	0.01\\
260.01	0.01\\
261.01	0.01\\
262.01	0.01\\
263.01	0.01\\
264.01	0.01\\
265.01	0.01\\
266.01	0.01\\
267.01	0.01\\
268.01	0.01\\
269.01	0.01\\
270.01	0.01\\
271.01	0.01\\
272.01	0.01\\
273.01	0.01\\
274.01	0.01\\
275.01	0.01\\
276.01	0.01\\
277.01	0.01\\
278.01	0.01\\
279.01	0.01\\
280.01	0.01\\
281.01	0.01\\
282.01	0.01\\
283.01	0.01\\
284.01	0.01\\
285.01	0.01\\
286.01	0.01\\
287.01	0.01\\
288.01	0.01\\
289.01	0.01\\
290.01	0.01\\
291.01	0.01\\
292.01	0.01\\
293.01	0.01\\
294.01	0.01\\
295.01	0.01\\
296.01	0.01\\
297.01	0.01\\
298.01	0.01\\
299.01	0.01\\
300.01	0.01\\
301.01	0.01\\
302.01	0.01\\
303.01	0.01\\
304.01	0.01\\
305.01	0.01\\
306.01	0.01\\
307.01	0.01\\
308.01	0.01\\
309.01	0.01\\
310.01	0.01\\
311.01	0.01\\
312.01	0.01\\
313.01	0.01\\
314.01	0.01\\
315.01	0.01\\
316.01	0.01\\
317.01	0.01\\
318.01	0.01\\
319.01	0.01\\
320.01	0.01\\
321.01	0.01\\
322.01	0.01\\
323.01	0.01\\
324.01	0.01\\
325.01	0.01\\
326.01	0.01\\
327.01	0.01\\
328.01	0.01\\
329.01	0.01\\
330.01	0.01\\
331.01	0.01\\
332.01	0.01\\
333.01	0.01\\
334.01	0.01\\
335.01	0.01\\
336.01	0.01\\
337.01	0.01\\
338.01	0.01\\
339.01	0.01\\
340.01	0.01\\
341.01	0.01\\
342.01	0.01\\
343.01	0.01\\
344.01	0.01\\
345.01	0.01\\
346.01	0.01\\
347.01	0.01\\
348.01	0.01\\
349.01	0.01\\
350.01	0.01\\
351.01	0.01\\
352.01	0.01\\
353.01	0.01\\
354.01	0.01\\
355.01	0.01\\
356.01	0.01\\
357.01	0.01\\
358.01	0.01\\
359.01	0.01\\
360.01	0.01\\
361.01	0.01\\
362.01	0.01\\
363.01	0.01\\
364.01	0.01\\
365.01	0.01\\
366.01	0.01\\
367.01	0.01\\
368.01	0.01\\
369.01	0.01\\
370.01	0.01\\
371.01	0.01\\
372.01	0.01\\
373.01	0.01\\
374.01	0.01\\
375.01	0.01\\
376.01	0.01\\
377.01	0.01\\
378.01	0.01\\
379.01	0.01\\
380.01	0.01\\
381.01	0.01\\
382.01	0.01\\
383.01	0.01\\
384.01	0.01\\
385.01	0.01\\
386.01	0.01\\
387.01	0.01\\
388.01	0.01\\
389.01	0.01\\
390.01	0.01\\
391.01	0.01\\
392.01	0.01\\
393.01	0.01\\
394.01	0.01\\
395.01	0.01\\
396.01	0.01\\
397.01	0.01\\
398.01	0.01\\
399.01	0.01\\
400.01	0.01\\
401.01	0.01\\
402.01	0.01\\
403.01	0.01\\
404.01	0.01\\
405.01	0.01\\
406.01	0.01\\
407.01	0.01\\
408.01	0.01\\
409.01	0.01\\
410.01	0.01\\
411.01	0.01\\
412.01	0.01\\
413.01	0.01\\
414.01	0.01\\
415.01	0.01\\
416.01	0.01\\
417.01	0.01\\
418.01	0.01\\
419.01	0.01\\
420.01	0.01\\
421.01	0.01\\
422.01	0.01\\
423.01	0.01\\
424.01	0.01\\
425.01	0.01\\
426.01	0.01\\
427.01	0.01\\
428.01	0.01\\
429.01	0.01\\
430.01	0.01\\
431.01	0.01\\
432.01	0.01\\
433.01	0.01\\
434.01	0.01\\
435.01	0.01\\
436.01	0.01\\
437.01	0.01\\
438.01	0.01\\
439.01	0.01\\
440.01	0.01\\
441.01	0.01\\
442.01	0.01\\
443.01	0.01\\
444.01	0.01\\
445.01	0.01\\
446.01	0.01\\
447.01	0.01\\
448.01	0.01\\
449.01	0.01\\
450.01	0.01\\
451.01	0.01\\
452.01	0.01\\
453.01	0.01\\
454.01	0.01\\
455.01	0.01\\
456.01	0.01\\
457.01	0.01\\
458.01	0.01\\
459.01	0.01\\
460.01	0.01\\
461.01	0.01\\
462.01	0.01\\
463.01	0.01\\
464.01	0.01\\
465.01	0.01\\
466.01	0.01\\
467.01	0.01\\
468.01	0.01\\
469.01	0.01\\
470.01	0.01\\
471.01	0.01\\
472.01	0.01\\
473.01	0.01\\
474.01	0.01\\
475.01	0.01\\
476.01	0.01\\
477.01	0.01\\
478.01	0.01\\
479.01	0.01\\
480.01	0.01\\
481.01	0.01\\
482.01	0.01\\
483.01	0.01\\
484.01	0.01\\
485.01	0.01\\
486.01	0.01\\
487.01	0.01\\
488.01	0.01\\
489.01	0.01\\
490.01	0.01\\
491.01	0.01\\
492.01	0.01\\
493.01	0.01\\
494.01	0.01\\
495.01	0.01\\
496.01	0.01\\
497.01	0.01\\
498.01	0.01\\
499.01	0.01\\
500.01	0.01\\
501.01	0.01\\
502.01	0.01\\
503.01	0.01\\
504.01	0.01\\
505.01	0.01\\
506.01	0.01\\
507.01	0.01\\
508.01	0.01\\
509.01	0.01\\
510.01	0.01\\
511.01	0.01\\
512.01	0.01\\
513.01	0.01\\
514.01	0.01\\
515.01	0.01\\
516.01	0.01\\
517.01	0.01\\
518.01	0.01\\
519.01	0.01\\
520.01	0.01\\
521.01	0.01\\
522.01	0.01\\
523.01	0.01\\
524.01	0.01\\
525.01	0.01\\
526.01	0.01\\
527.01	0.01\\
528.01	0.01\\
529.01	0.01\\
530.01	0.01\\
531.01	0.01\\
532.01	0.01\\
533.01	0.01\\
534.01	0.01\\
535.01	0.01\\
536.01	0.01\\
537.01	0.01\\
538.01	0.01\\
539.01	0.01\\
540.01	0.01\\
541.01	0.01\\
542.01	0.01\\
543.01	0.01\\
544.01	0.01\\
545.01	0.01\\
546.01	0.01\\
547.01	0.01\\
548.01	0.01\\
549.01	0.01\\
550.01	0.01\\
551.01	0.01\\
552.01	0.01\\
553.01	0.01\\
554.01	0.01\\
555.01	0.01\\
556.01	0.01\\
557.01	0.01\\
558.01	0.01\\
559.01	0.01\\
560.01	0.01\\
561.01	0.01\\
562.01	0.01\\
563.01	0.01\\
564.01	0.01\\
565.01	0.01\\
566.01	0.01\\
567.01	0.01\\
568.01	0.01\\
569.01	0.01\\
570.01	0.01\\
571.01	0.01\\
572.01	0.01\\
573.01	0.01\\
574.01	0.01\\
575.01	0.01\\
576.01	0.01\\
577.01	0.01\\
578.01	0.01\\
579.01	0.01\\
580.01	0.01\\
581.01	0.01\\
582.01	0.01\\
583.01	0.01\\
584.01	0.01\\
585.01	0.01\\
586.01	0.01\\
587.01	0.01\\
588.01	0.01\\
589.01	0.01\\
590.01	0.01\\
591.01	0.01\\
592.01	0.01\\
593.01	0.01\\
594.01	0.01\\
595.01	0.01\\
596.01	0.01\\
597.01	0.01\\
598.01	0.00863902290697456\\
599.01	0.00623513569400516\\
599.02	0.00619747359804885\\
599.03	0.00615944475593815\\
599.04	0.00612104556201828\\
599.05	0.00608227237517915\\
599.06	0.0060431215185067\\
599.07	0.00600358927893089\\
599.08	0.00596367190687008\\
599.09	0.00592336561587203\\
599.1	0.00588266658225147\\
599.11	0.00584157094472378\\
599.12	0.00580007480403545\\
599.13	0.00575817422259057\\
599.14	0.00571586522407394\\
599.15	0.00567314379307032\\
599.16	0.00563000587467982\\
599.17	0.0055864473741298\\
599.18	0.00554246415638276\\
599.19	0.00549805204574034\\
599.2	0.00545320682544357\\
599.21	0.00540792423726908\\
599.22	0.0053621999811214\\
599.23	0.0053160297146212\\
599.24	0.00526940905268953\\
599.25	0.00522233356712795\\
599.26	0.00517479878619452\\
599.27	0.00512680019417564\\
599.28	0.00507833323095367\\
599.29	0.0050293932915703\\
599.3	0.00497997572578562\\
599.31	0.00493007583763287\\
599.32	0.00487968888496881\\
599.33	0.00482881007901965\\
599.34	0.00477743458392257\\
599.35	0.00472555751626257\\
599.36	0.00467317394460506\\
599.37	0.00462027888902352\\
599.38	0.00456686732062279\\
599.39	0.00451293416105749\\
599.4	0.00445847428204579\\
599.41	0.00440348251095323\\
599.42	0.00434795363554174\\
599.43	0.00429188239249109\\
599.44	0.00423526346689831\\
599.45	0.00417809149177217\\
599.46	0.00412036104752264\\
599.47	0.00406206666144545\\
599.48	0.0040032028072016\\
599.49	0.00394376390429165\\
599.5	0.00388374431752493\\
599.51	0.00382313835648359\\
599.52	0.00376194027498136\\
599.53	0.00370014427051698\\
599.54	0.00363774448372229\\
599.55	0.00357473499780494\\
599.56	0.00351110983798562\\
599.57	0.00344686297092976\\
599.58	0.00338198830417367\\
599.59	0.00331647968554504\\
599.6	0.00325033090257784\\
599.61	0.00318353568192134\\
599.62	0.00311608768874351\\
599.63	0.00304798052612841\\
599.64	0.00297920773446781\\
599.65	0.00290976279084677\\
599.66	0.00283963910842317\\
599.67	0.00276883003580129\\
599.68	0.00269732885639913\\
599.69	0.00262512878780952\\
599.7	0.00255222298115511\\
599.71	0.00247860452043687\\
599.72	0.00240426642187631\\
599.73	0.00232920163325121\\
599.74	0.0022534030332249\\
599.75	0.00217686343066883\\
599.76	0.00209957556397869\\
599.77	0.00202153210038368\\
599.78	0.00194272563524906\\
599.79	0.00186314869137187\\
599.8	0.00178279371826977\\
599.81	0.00170165309146284\\
599.82	0.00161971911174842\\
599.83	0.00153698400446879\\
599.84	0.00145343991877172\\
599.85	0.00136907892686372\\
599.86	0.00128389302325598\\
599.87	0.001197874124003\\
599.88	0.00111101406593363\\
599.89	0.00102330460587469\\
599.9	0.000934737419866898\\
599.91	0.00084530410237319\\
599.92	0.000754996165479168\\
599.93	0.000663805038085871\\
599.94	0.000571722065094477\\
599.95	0.000478738506583132\\
599.96	0.000384845536975641\\
599.97	0.000290034244202046\\
599.98	0.00019429562885097\\
599.99	9.76206033136556e-05\\
600	0\\
};
\addplot [color=mycolor1,solid,forget plot]
  table[row sep=crcr]{%
0.01	0.01\\
1.01	0.01\\
2.01	0.01\\
3.01	0.01\\
4.01	0.01\\
5.01	0.01\\
6.01	0.01\\
7.01	0.01\\
8.01	0.01\\
9.01	0.01\\
10.01	0.01\\
11.01	0.01\\
12.01	0.01\\
13.01	0.01\\
14.01	0.01\\
15.01	0.01\\
16.01	0.01\\
17.01	0.01\\
18.01	0.01\\
19.01	0.01\\
20.01	0.01\\
21.01	0.01\\
22.01	0.01\\
23.01	0.01\\
24.01	0.01\\
25.01	0.01\\
26.01	0.01\\
27.01	0.01\\
28.01	0.01\\
29.01	0.01\\
30.01	0.01\\
31.01	0.01\\
32.01	0.01\\
33.01	0.01\\
34.01	0.01\\
35.01	0.01\\
36.01	0.01\\
37.01	0.01\\
38.01	0.01\\
39.01	0.01\\
40.01	0.01\\
41.01	0.01\\
42.01	0.01\\
43.01	0.01\\
44.01	0.01\\
45.01	0.01\\
46.01	0.01\\
47.01	0.01\\
48.01	0.01\\
49.01	0.01\\
50.01	0.01\\
51.01	0.01\\
52.01	0.01\\
53.01	0.01\\
54.01	0.01\\
55.01	0.01\\
56.01	0.01\\
57.01	0.01\\
58.01	0.01\\
59.01	0.01\\
60.01	0.01\\
61.01	0.01\\
62.01	0.01\\
63.01	0.01\\
64.01	0.01\\
65.01	0.01\\
66.01	0.01\\
67.01	0.01\\
68.01	0.01\\
69.01	0.01\\
70.01	0.01\\
71.01	0.01\\
72.01	0.01\\
73.01	0.01\\
74.01	0.01\\
75.01	0.01\\
76.01	0.01\\
77.01	0.01\\
78.01	0.01\\
79.01	0.01\\
80.01	0.01\\
81.01	0.01\\
82.01	0.01\\
83.01	0.01\\
84.01	0.01\\
85.01	0.01\\
86.01	0.01\\
87.01	0.01\\
88.01	0.01\\
89.01	0.01\\
90.01	0.01\\
91.01	0.01\\
92.01	0.01\\
93.01	0.01\\
94.01	0.01\\
95.01	0.01\\
96.01	0.01\\
97.01	0.01\\
98.01	0.01\\
99.01	0.01\\
100.01	0.01\\
101.01	0.01\\
102.01	0.01\\
103.01	0.01\\
104.01	0.01\\
105.01	0.01\\
106.01	0.01\\
107.01	0.01\\
108.01	0.01\\
109.01	0.01\\
110.01	0.01\\
111.01	0.01\\
112.01	0.01\\
113.01	0.01\\
114.01	0.01\\
115.01	0.01\\
116.01	0.01\\
117.01	0.01\\
118.01	0.01\\
119.01	0.01\\
120.01	0.01\\
121.01	0.01\\
122.01	0.01\\
123.01	0.01\\
124.01	0.01\\
125.01	0.01\\
126.01	0.01\\
127.01	0.01\\
128.01	0.01\\
129.01	0.01\\
130.01	0.01\\
131.01	0.01\\
132.01	0.01\\
133.01	0.01\\
134.01	0.01\\
135.01	0.01\\
136.01	0.01\\
137.01	0.01\\
138.01	0.01\\
139.01	0.01\\
140.01	0.01\\
141.01	0.01\\
142.01	0.01\\
143.01	0.01\\
144.01	0.01\\
145.01	0.01\\
146.01	0.01\\
147.01	0.01\\
148.01	0.01\\
149.01	0.01\\
150.01	0.01\\
151.01	0.01\\
152.01	0.01\\
153.01	0.01\\
154.01	0.01\\
155.01	0.01\\
156.01	0.01\\
157.01	0.01\\
158.01	0.01\\
159.01	0.01\\
160.01	0.01\\
161.01	0.01\\
162.01	0.01\\
163.01	0.01\\
164.01	0.01\\
165.01	0.01\\
166.01	0.01\\
167.01	0.01\\
168.01	0.01\\
169.01	0.01\\
170.01	0.01\\
171.01	0.01\\
172.01	0.01\\
173.01	0.01\\
174.01	0.01\\
175.01	0.01\\
176.01	0.01\\
177.01	0.01\\
178.01	0.01\\
179.01	0.01\\
180.01	0.01\\
181.01	0.01\\
182.01	0.01\\
183.01	0.01\\
184.01	0.01\\
185.01	0.01\\
186.01	0.01\\
187.01	0.01\\
188.01	0.01\\
189.01	0.01\\
190.01	0.01\\
191.01	0.01\\
192.01	0.01\\
193.01	0.01\\
194.01	0.01\\
195.01	0.01\\
196.01	0.01\\
197.01	0.01\\
198.01	0.01\\
199.01	0.01\\
200.01	0.01\\
201.01	0.01\\
202.01	0.01\\
203.01	0.01\\
204.01	0.01\\
205.01	0.01\\
206.01	0.01\\
207.01	0.01\\
208.01	0.01\\
209.01	0.01\\
210.01	0.01\\
211.01	0.01\\
212.01	0.01\\
213.01	0.01\\
214.01	0.01\\
215.01	0.01\\
216.01	0.01\\
217.01	0.01\\
218.01	0.01\\
219.01	0.01\\
220.01	0.01\\
221.01	0.01\\
222.01	0.01\\
223.01	0.01\\
224.01	0.01\\
225.01	0.01\\
226.01	0.01\\
227.01	0.01\\
228.01	0.01\\
229.01	0.01\\
230.01	0.01\\
231.01	0.01\\
232.01	0.01\\
233.01	0.01\\
234.01	0.01\\
235.01	0.01\\
236.01	0.01\\
237.01	0.01\\
238.01	0.01\\
239.01	0.01\\
240.01	0.01\\
241.01	0.01\\
242.01	0.01\\
243.01	0.01\\
244.01	0.01\\
245.01	0.01\\
246.01	0.01\\
247.01	0.01\\
248.01	0.01\\
249.01	0.01\\
250.01	0.01\\
251.01	0.01\\
252.01	0.01\\
253.01	0.01\\
254.01	0.01\\
255.01	0.01\\
256.01	0.01\\
257.01	0.01\\
258.01	0.01\\
259.01	0.01\\
260.01	0.01\\
261.01	0.01\\
262.01	0.01\\
263.01	0.01\\
264.01	0.01\\
265.01	0.01\\
266.01	0.01\\
267.01	0.01\\
268.01	0.01\\
269.01	0.01\\
270.01	0.01\\
271.01	0.01\\
272.01	0.01\\
273.01	0.01\\
274.01	0.01\\
275.01	0.01\\
276.01	0.01\\
277.01	0.01\\
278.01	0.01\\
279.01	0.01\\
280.01	0.01\\
281.01	0.01\\
282.01	0.01\\
283.01	0.01\\
284.01	0.01\\
285.01	0.01\\
286.01	0.01\\
287.01	0.01\\
288.01	0.01\\
289.01	0.01\\
290.01	0.01\\
291.01	0.01\\
292.01	0.01\\
293.01	0.01\\
294.01	0.01\\
295.01	0.01\\
296.01	0.01\\
297.01	0.01\\
298.01	0.01\\
299.01	0.01\\
300.01	0.01\\
301.01	0.01\\
302.01	0.01\\
303.01	0.01\\
304.01	0.01\\
305.01	0.01\\
306.01	0.01\\
307.01	0.01\\
308.01	0.01\\
309.01	0.01\\
310.01	0.01\\
311.01	0.01\\
312.01	0.01\\
313.01	0.01\\
314.01	0.01\\
315.01	0.01\\
316.01	0.01\\
317.01	0.01\\
318.01	0.01\\
319.01	0.01\\
320.01	0.01\\
321.01	0.01\\
322.01	0.01\\
323.01	0.01\\
324.01	0.01\\
325.01	0.01\\
326.01	0.01\\
327.01	0.01\\
328.01	0.01\\
329.01	0.01\\
330.01	0.01\\
331.01	0.01\\
332.01	0.01\\
333.01	0.01\\
334.01	0.01\\
335.01	0.01\\
336.01	0.01\\
337.01	0.01\\
338.01	0.01\\
339.01	0.01\\
340.01	0.01\\
341.01	0.01\\
342.01	0.01\\
343.01	0.01\\
344.01	0.01\\
345.01	0.01\\
346.01	0.01\\
347.01	0.01\\
348.01	0.01\\
349.01	0.01\\
350.01	0.01\\
351.01	0.01\\
352.01	0.01\\
353.01	0.01\\
354.01	0.01\\
355.01	0.01\\
356.01	0.01\\
357.01	0.01\\
358.01	0.01\\
359.01	0.01\\
360.01	0.01\\
361.01	0.01\\
362.01	0.01\\
363.01	0.01\\
364.01	0.01\\
365.01	0.01\\
366.01	0.01\\
367.01	0.01\\
368.01	0.01\\
369.01	0.01\\
370.01	0.01\\
371.01	0.01\\
372.01	0.01\\
373.01	0.01\\
374.01	0.01\\
375.01	0.01\\
376.01	0.01\\
377.01	0.01\\
378.01	0.01\\
379.01	0.01\\
380.01	0.01\\
381.01	0.01\\
382.01	0.01\\
383.01	0.01\\
384.01	0.01\\
385.01	0.01\\
386.01	0.01\\
387.01	0.01\\
388.01	0.01\\
389.01	0.01\\
390.01	0.01\\
391.01	0.01\\
392.01	0.01\\
393.01	0.01\\
394.01	0.01\\
395.01	0.01\\
396.01	0.01\\
397.01	0.01\\
398.01	0.01\\
399.01	0.01\\
400.01	0.01\\
401.01	0.01\\
402.01	0.01\\
403.01	0.01\\
404.01	0.01\\
405.01	0.01\\
406.01	0.01\\
407.01	0.01\\
408.01	0.01\\
409.01	0.01\\
410.01	0.01\\
411.01	0.01\\
412.01	0.01\\
413.01	0.01\\
414.01	0.01\\
415.01	0.01\\
416.01	0.01\\
417.01	0.01\\
418.01	0.01\\
419.01	0.01\\
420.01	0.01\\
421.01	0.01\\
422.01	0.01\\
423.01	0.01\\
424.01	0.01\\
425.01	0.01\\
426.01	0.01\\
427.01	0.01\\
428.01	0.01\\
429.01	0.01\\
430.01	0.01\\
431.01	0.01\\
432.01	0.01\\
433.01	0.01\\
434.01	0.01\\
435.01	0.01\\
436.01	0.01\\
437.01	0.01\\
438.01	0.01\\
439.01	0.01\\
440.01	0.01\\
441.01	0.01\\
442.01	0.01\\
443.01	0.01\\
444.01	0.01\\
445.01	0.01\\
446.01	0.01\\
447.01	0.01\\
448.01	0.01\\
449.01	0.01\\
450.01	0.01\\
451.01	0.01\\
452.01	0.01\\
453.01	0.01\\
454.01	0.01\\
455.01	0.01\\
456.01	0.01\\
457.01	0.01\\
458.01	0.01\\
459.01	0.01\\
460.01	0.01\\
461.01	0.01\\
462.01	0.01\\
463.01	0.01\\
464.01	0.01\\
465.01	0.01\\
466.01	0.01\\
467.01	0.01\\
468.01	0.01\\
469.01	0.01\\
470.01	0.01\\
471.01	0.01\\
472.01	0.01\\
473.01	0.01\\
474.01	0.01\\
475.01	0.01\\
476.01	0.01\\
477.01	0.01\\
478.01	0.01\\
479.01	0.01\\
480.01	0.01\\
481.01	0.01\\
482.01	0.01\\
483.01	0.01\\
484.01	0.01\\
485.01	0.01\\
486.01	0.01\\
487.01	0.01\\
488.01	0.01\\
489.01	0.01\\
490.01	0.01\\
491.01	0.01\\
492.01	0.01\\
493.01	0.01\\
494.01	0.01\\
495.01	0.01\\
496.01	0.01\\
497.01	0.01\\
498.01	0.01\\
499.01	0.01\\
500.01	0.01\\
501.01	0.01\\
502.01	0.01\\
503.01	0.01\\
504.01	0.01\\
505.01	0.01\\
506.01	0.01\\
507.01	0.01\\
508.01	0.01\\
509.01	0.01\\
510.01	0.01\\
511.01	0.01\\
512.01	0.01\\
513.01	0.01\\
514.01	0.01\\
515.01	0.01\\
516.01	0.01\\
517.01	0.01\\
518.01	0.01\\
519.01	0.01\\
520.01	0.01\\
521.01	0.01\\
522.01	0.01\\
523.01	0.01\\
524.01	0.01\\
525.01	0.01\\
526.01	0.01\\
527.01	0.01\\
528.01	0.01\\
529.01	0.01\\
530.01	0.01\\
531.01	0.01\\
532.01	0.01\\
533.01	0.01\\
534.01	0.01\\
535.01	0.01\\
536.01	0.01\\
537.01	0.01\\
538.01	0.01\\
539.01	0.01\\
540.01	0.01\\
541.01	0.01\\
542.01	0.01\\
543.01	0.01\\
544.01	0.01\\
545.01	0.01\\
546.01	0.01\\
547.01	0.01\\
548.01	0.01\\
549.01	0.01\\
550.01	0.01\\
551.01	0.01\\
552.01	0.01\\
553.01	0.01\\
554.01	0.01\\
555.01	0.01\\
556.01	0.01\\
557.01	0.01\\
558.01	0.01\\
559.01	0.01\\
560.01	0.01\\
561.01	0.01\\
562.01	0.01\\
563.01	0.01\\
564.01	0.01\\
565.01	0.01\\
566.01	0.01\\
567.01	0.01\\
568.01	0.01\\
569.01	0.01\\
570.01	0.01\\
571.01	0.01\\
572.01	0.01\\
573.01	0.01\\
574.01	0.01\\
575.01	0.01\\
576.01	0.01\\
577.01	0.01\\
578.01	0.01\\
579.01	0.01\\
580.01	0.01\\
581.01	0.01\\
582.01	0.01\\
583.01	0.01\\
584.01	0.01\\
585.01	0.01\\
586.01	0.01\\
587.01	0.01\\
588.01	0.01\\
589.01	0.01\\
590.01	0.01\\
591.01	0.01\\
592.01	0.01\\
593.01	0.01\\
594.01	0.01\\
595.01	0.01\\
596.01	0.01\\
597.01	0.01\\
598.01	0.0086390371714876\\
599.01	0.00623513569400516\\
599.02	0.00619747359804889\\
599.03	0.00615944475593817\\
599.04	0.00612104556201829\\
599.05	0.00608227237517917\\
599.06	0.00604312151850673\\
599.07	0.00600358927893092\\
599.08	0.00596367190687008\\
599.09	0.00592336561587206\\
599.1	0.0058826665822515\\
599.11	0.00584157094472381\\
599.12	0.00580007480403546\\
599.13	0.00575817422259058\\
599.14	0.00571586522407398\\
599.15	0.00567314379307033\\
599.16	0.00563000587467983\\
599.17	0.00558644737412983\\
599.18	0.00554246415638277\\
599.19	0.00549805204574035\\
599.2	0.00545320682544357\\
599.21	0.00540792423726907\\
599.22	0.00536219998112139\\
599.23	0.00531602971462118\\
599.24	0.0052694090526895\\
599.25	0.00522233356712793\\
599.26	0.00517479878619452\\
599.27	0.00512680019417564\\
599.28	0.00507833323095367\\
599.29	0.0050293932915703\\
599.3	0.0049799757257856\\
599.31	0.00493007583763285\\
599.32	0.00487968888496879\\
599.33	0.00482881007901965\\
599.34	0.00477743458392255\\
599.35	0.00472555751626256\\
599.36	0.00467317394460504\\
599.37	0.00462027888902352\\
599.38	0.0045668673206228\\
599.39	0.00451293416105749\\
599.4	0.00445847428204579\\
599.41	0.00440348251095323\\
599.42	0.00434795363554174\\
599.43	0.00429188239249109\\
599.44	0.00423526346689831\\
599.45	0.00417809149177215\\
599.46	0.00412036104752261\\
599.47	0.00406206666144544\\
599.48	0.0040032028072016\\
599.49	0.00394376390429164\\
599.5	0.00388374431752493\\
599.51	0.00382313835648359\\
599.52	0.00376194027498137\\
599.53	0.00370014427051698\\
599.54	0.00363774448372229\\
599.55	0.00357473499780495\\
599.56	0.00351110983798565\\
599.57	0.00344686297092978\\
599.58	0.00338198830417368\\
599.59	0.00331647968554506\\
599.6	0.00325033090257785\\
599.61	0.00318353568192136\\
599.62	0.00311608768874352\\
599.63	0.00304798052612842\\
599.64	0.00297920773446781\\
599.65	0.00290976279084675\\
599.66	0.00283963910842315\\
599.67	0.00276883003580128\\
599.68	0.00269732885639911\\
599.69	0.0026251287878095\\
599.7	0.0025522229811551\\
599.71	0.00247860452043685\\
599.72	0.00240426642187628\\
599.73	0.00232920163325119\\
599.74	0.00225340303322486\\
599.75	0.0021768634306688\\
599.76	0.00209957556397866\\
599.77	0.00202153210038365\\
599.78	0.00194272563524903\\
599.79	0.00186314869137185\\
599.8	0.00178279371826975\\
599.81	0.00170165309146282\\
599.82	0.00161971911174841\\
599.83	0.00153698400446878\\
599.84	0.00145343991877171\\
599.85	0.00136907892686371\\
599.86	0.00128389302325597\\
599.87	0.00119787412400299\\
599.88	0.00111101406593363\\
599.89	0.00102330460587469\\
599.9	0.000934737419866901\\
599.91	0.000845304102373181\\
599.92	0.000754996165479166\\
599.93	0.000663805038085864\\
599.94	0.000571722065094473\\
599.95	0.000478738506583126\\
599.96	0.000384845536975636\\
599.97	0.000290034244202042\\
599.98	0.00019429562885097\\
599.99	9.76206033136556e-05\\
600	0\\
};
\addplot [color=mycolor2,solid,forget plot]
  table[row sep=crcr]{%
0.01	0.01\\
1.01	0.01\\
2.01	0.01\\
3.01	0.01\\
4.01	0.01\\
5.01	0.01\\
6.01	0.01\\
7.01	0.01\\
8.01	0.01\\
9.01	0.01\\
10.01	0.01\\
11.01	0.01\\
12.01	0.01\\
13.01	0.01\\
14.01	0.01\\
15.01	0.01\\
16.01	0.01\\
17.01	0.01\\
18.01	0.01\\
19.01	0.01\\
20.01	0.01\\
21.01	0.01\\
22.01	0.01\\
23.01	0.01\\
24.01	0.01\\
25.01	0.01\\
26.01	0.01\\
27.01	0.01\\
28.01	0.01\\
29.01	0.01\\
30.01	0.01\\
31.01	0.01\\
32.01	0.01\\
33.01	0.01\\
34.01	0.01\\
35.01	0.01\\
36.01	0.01\\
37.01	0.01\\
38.01	0.01\\
39.01	0.01\\
40.01	0.01\\
41.01	0.01\\
42.01	0.01\\
43.01	0.01\\
44.01	0.01\\
45.01	0.01\\
46.01	0.01\\
47.01	0.01\\
48.01	0.01\\
49.01	0.01\\
50.01	0.01\\
51.01	0.01\\
52.01	0.01\\
53.01	0.01\\
54.01	0.01\\
55.01	0.01\\
56.01	0.01\\
57.01	0.01\\
58.01	0.01\\
59.01	0.01\\
60.01	0.01\\
61.01	0.01\\
62.01	0.01\\
63.01	0.01\\
64.01	0.01\\
65.01	0.01\\
66.01	0.01\\
67.01	0.01\\
68.01	0.01\\
69.01	0.01\\
70.01	0.01\\
71.01	0.01\\
72.01	0.01\\
73.01	0.01\\
74.01	0.01\\
75.01	0.01\\
76.01	0.01\\
77.01	0.01\\
78.01	0.01\\
79.01	0.01\\
80.01	0.01\\
81.01	0.01\\
82.01	0.01\\
83.01	0.01\\
84.01	0.01\\
85.01	0.01\\
86.01	0.01\\
87.01	0.01\\
88.01	0.01\\
89.01	0.01\\
90.01	0.01\\
91.01	0.01\\
92.01	0.01\\
93.01	0.01\\
94.01	0.01\\
95.01	0.01\\
96.01	0.01\\
97.01	0.01\\
98.01	0.01\\
99.01	0.01\\
100.01	0.01\\
101.01	0.01\\
102.01	0.01\\
103.01	0.01\\
104.01	0.01\\
105.01	0.01\\
106.01	0.01\\
107.01	0.01\\
108.01	0.01\\
109.01	0.01\\
110.01	0.01\\
111.01	0.01\\
112.01	0.01\\
113.01	0.01\\
114.01	0.01\\
115.01	0.01\\
116.01	0.01\\
117.01	0.01\\
118.01	0.01\\
119.01	0.01\\
120.01	0.01\\
121.01	0.01\\
122.01	0.01\\
123.01	0.01\\
124.01	0.01\\
125.01	0.01\\
126.01	0.01\\
127.01	0.01\\
128.01	0.01\\
129.01	0.01\\
130.01	0.01\\
131.01	0.01\\
132.01	0.01\\
133.01	0.01\\
134.01	0.01\\
135.01	0.01\\
136.01	0.01\\
137.01	0.01\\
138.01	0.01\\
139.01	0.01\\
140.01	0.01\\
141.01	0.01\\
142.01	0.01\\
143.01	0.01\\
144.01	0.01\\
145.01	0.01\\
146.01	0.01\\
147.01	0.01\\
148.01	0.01\\
149.01	0.01\\
150.01	0.01\\
151.01	0.01\\
152.01	0.01\\
153.01	0.01\\
154.01	0.01\\
155.01	0.01\\
156.01	0.01\\
157.01	0.01\\
158.01	0.01\\
159.01	0.01\\
160.01	0.01\\
161.01	0.01\\
162.01	0.01\\
163.01	0.01\\
164.01	0.01\\
165.01	0.01\\
166.01	0.01\\
167.01	0.01\\
168.01	0.01\\
169.01	0.01\\
170.01	0.01\\
171.01	0.01\\
172.01	0.01\\
173.01	0.01\\
174.01	0.01\\
175.01	0.01\\
176.01	0.01\\
177.01	0.01\\
178.01	0.01\\
179.01	0.01\\
180.01	0.01\\
181.01	0.01\\
182.01	0.01\\
183.01	0.01\\
184.01	0.01\\
185.01	0.01\\
186.01	0.01\\
187.01	0.01\\
188.01	0.01\\
189.01	0.01\\
190.01	0.01\\
191.01	0.01\\
192.01	0.01\\
193.01	0.01\\
194.01	0.01\\
195.01	0.01\\
196.01	0.01\\
197.01	0.01\\
198.01	0.01\\
199.01	0.01\\
200.01	0.01\\
201.01	0.01\\
202.01	0.01\\
203.01	0.01\\
204.01	0.01\\
205.01	0.01\\
206.01	0.01\\
207.01	0.01\\
208.01	0.01\\
209.01	0.01\\
210.01	0.01\\
211.01	0.01\\
212.01	0.01\\
213.01	0.01\\
214.01	0.01\\
215.01	0.01\\
216.01	0.01\\
217.01	0.01\\
218.01	0.01\\
219.01	0.01\\
220.01	0.01\\
221.01	0.01\\
222.01	0.01\\
223.01	0.01\\
224.01	0.01\\
225.01	0.01\\
226.01	0.01\\
227.01	0.01\\
228.01	0.01\\
229.01	0.01\\
230.01	0.01\\
231.01	0.01\\
232.01	0.01\\
233.01	0.01\\
234.01	0.01\\
235.01	0.01\\
236.01	0.01\\
237.01	0.01\\
238.01	0.01\\
239.01	0.01\\
240.01	0.01\\
241.01	0.01\\
242.01	0.01\\
243.01	0.01\\
244.01	0.01\\
245.01	0.01\\
246.01	0.01\\
247.01	0.01\\
248.01	0.01\\
249.01	0.01\\
250.01	0.01\\
251.01	0.01\\
252.01	0.01\\
253.01	0.01\\
254.01	0.01\\
255.01	0.01\\
256.01	0.01\\
257.01	0.01\\
258.01	0.01\\
259.01	0.01\\
260.01	0.01\\
261.01	0.01\\
262.01	0.01\\
263.01	0.01\\
264.01	0.01\\
265.01	0.01\\
266.01	0.01\\
267.01	0.01\\
268.01	0.01\\
269.01	0.01\\
270.01	0.01\\
271.01	0.01\\
272.01	0.01\\
273.01	0.01\\
274.01	0.01\\
275.01	0.01\\
276.01	0.01\\
277.01	0.01\\
278.01	0.01\\
279.01	0.01\\
280.01	0.01\\
281.01	0.01\\
282.01	0.01\\
283.01	0.01\\
284.01	0.01\\
285.01	0.01\\
286.01	0.01\\
287.01	0.01\\
288.01	0.01\\
289.01	0.01\\
290.01	0.01\\
291.01	0.01\\
292.01	0.01\\
293.01	0.01\\
294.01	0.01\\
295.01	0.01\\
296.01	0.01\\
297.01	0.01\\
298.01	0.01\\
299.01	0.01\\
300.01	0.01\\
301.01	0.01\\
302.01	0.01\\
303.01	0.01\\
304.01	0.01\\
305.01	0.01\\
306.01	0.01\\
307.01	0.01\\
308.01	0.01\\
309.01	0.01\\
310.01	0.01\\
311.01	0.01\\
312.01	0.01\\
313.01	0.01\\
314.01	0.01\\
315.01	0.01\\
316.01	0.01\\
317.01	0.01\\
318.01	0.01\\
319.01	0.01\\
320.01	0.01\\
321.01	0.01\\
322.01	0.01\\
323.01	0.01\\
324.01	0.01\\
325.01	0.01\\
326.01	0.01\\
327.01	0.01\\
328.01	0.01\\
329.01	0.01\\
330.01	0.01\\
331.01	0.01\\
332.01	0.01\\
333.01	0.01\\
334.01	0.01\\
335.01	0.01\\
336.01	0.01\\
337.01	0.01\\
338.01	0.01\\
339.01	0.01\\
340.01	0.01\\
341.01	0.01\\
342.01	0.01\\
343.01	0.01\\
344.01	0.01\\
345.01	0.01\\
346.01	0.01\\
347.01	0.01\\
348.01	0.01\\
349.01	0.01\\
350.01	0.01\\
351.01	0.01\\
352.01	0.01\\
353.01	0.01\\
354.01	0.01\\
355.01	0.01\\
356.01	0.01\\
357.01	0.01\\
358.01	0.01\\
359.01	0.01\\
360.01	0.01\\
361.01	0.01\\
362.01	0.01\\
363.01	0.01\\
364.01	0.01\\
365.01	0.01\\
366.01	0.01\\
367.01	0.01\\
368.01	0.01\\
369.01	0.01\\
370.01	0.01\\
371.01	0.01\\
372.01	0.01\\
373.01	0.01\\
374.01	0.01\\
375.01	0.01\\
376.01	0.01\\
377.01	0.01\\
378.01	0.01\\
379.01	0.01\\
380.01	0.01\\
381.01	0.01\\
382.01	0.01\\
383.01	0.01\\
384.01	0.01\\
385.01	0.01\\
386.01	0.01\\
387.01	0.01\\
388.01	0.01\\
389.01	0.01\\
390.01	0.01\\
391.01	0.01\\
392.01	0.01\\
393.01	0.01\\
394.01	0.01\\
395.01	0.01\\
396.01	0.01\\
397.01	0.01\\
398.01	0.01\\
399.01	0.01\\
400.01	0.01\\
401.01	0.01\\
402.01	0.01\\
403.01	0.01\\
404.01	0.01\\
405.01	0.01\\
406.01	0.01\\
407.01	0.01\\
408.01	0.01\\
409.01	0.01\\
410.01	0.01\\
411.01	0.01\\
412.01	0.01\\
413.01	0.01\\
414.01	0.01\\
415.01	0.01\\
416.01	0.01\\
417.01	0.01\\
418.01	0.01\\
419.01	0.01\\
420.01	0.01\\
421.01	0.01\\
422.01	0.01\\
423.01	0.01\\
424.01	0.01\\
425.01	0.01\\
426.01	0.01\\
427.01	0.01\\
428.01	0.01\\
429.01	0.01\\
430.01	0.01\\
431.01	0.01\\
432.01	0.01\\
433.01	0.01\\
434.01	0.01\\
435.01	0.01\\
436.01	0.01\\
437.01	0.01\\
438.01	0.01\\
439.01	0.01\\
440.01	0.01\\
441.01	0.01\\
442.01	0.01\\
443.01	0.01\\
444.01	0.01\\
445.01	0.01\\
446.01	0.01\\
447.01	0.01\\
448.01	0.01\\
449.01	0.01\\
450.01	0.01\\
451.01	0.01\\
452.01	0.01\\
453.01	0.01\\
454.01	0.01\\
455.01	0.01\\
456.01	0.01\\
457.01	0.01\\
458.01	0.01\\
459.01	0.01\\
460.01	0.01\\
461.01	0.01\\
462.01	0.01\\
463.01	0.01\\
464.01	0.01\\
465.01	0.01\\
466.01	0.01\\
467.01	0.01\\
468.01	0.01\\
469.01	0.01\\
470.01	0.01\\
471.01	0.01\\
472.01	0.01\\
473.01	0.01\\
474.01	0.01\\
475.01	0.01\\
476.01	0.01\\
477.01	0.01\\
478.01	0.01\\
479.01	0.01\\
480.01	0.01\\
481.01	0.01\\
482.01	0.01\\
483.01	0.01\\
484.01	0.01\\
485.01	0.01\\
486.01	0.01\\
487.01	0.01\\
488.01	0.01\\
489.01	0.01\\
490.01	0.01\\
491.01	0.01\\
492.01	0.01\\
493.01	0.01\\
494.01	0.01\\
495.01	0.01\\
496.01	0.01\\
497.01	0.01\\
498.01	0.01\\
499.01	0.01\\
500.01	0.01\\
501.01	0.01\\
502.01	0.01\\
503.01	0.01\\
504.01	0.01\\
505.01	0.01\\
506.01	0.01\\
507.01	0.01\\
508.01	0.01\\
509.01	0.01\\
510.01	0.01\\
511.01	0.01\\
512.01	0.01\\
513.01	0.01\\
514.01	0.01\\
515.01	0.01\\
516.01	0.01\\
517.01	0.01\\
518.01	0.01\\
519.01	0.01\\
520.01	0.01\\
521.01	0.01\\
522.01	0.01\\
523.01	0.01\\
524.01	0.01\\
525.01	0.01\\
526.01	0.01\\
527.01	0.01\\
528.01	0.01\\
529.01	0.01\\
530.01	0.01\\
531.01	0.01\\
532.01	0.01\\
533.01	0.01\\
534.01	0.01\\
535.01	0.01\\
536.01	0.01\\
537.01	0.01\\
538.01	0.01\\
539.01	0.01\\
540.01	0.01\\
541.01	0.01\\
542.01	0.01\\
543.01	0.01\\
544.01	0.01\\
545.01	0.01\\
546.01	0.01\\
547.01	0.01\\
548.01	0.01\\
549.01	0.01\\
550.01	0.01\\
551.01	0.01\\
552.01	0.01\\
553.01	0.01\\
554.01	0.01\\
555.01	0.01\\
556.01	0.01\\
557.01	0.01\\
558.01	0.01\\
559.01	0.01\\
560.01	0.01\\
561.01	0.01\\
562.01	0.01\\
563.01	0.01\\
564.01	0.01\\
565.01	0.01\\
566.01	0.01\\
567.01	0.01\\
568.01	0.01\\
569.01	0.01\\
570.01	0.01\\
571.01	0.01\\
572.01	0.01\\
573.01	0.01\\
574.01	0.01\\
575.01	0.01\\
576.01	0.01\\
577.01	0.01\\
578.01	0.01\\
579.01	0.01\\
580.01	0.01\\
581.01	0.01\\
582.01	0.01\\
583.01	0.01\\
584.01	0.01\\
585.01	0.01\\
586.01	0.01\\
587.01	0.01\\
588.01	0.01\\
589.01	0.01\\
590.01	0.01\\
591.01	0.01\\
592.01	0.01\\
593.01	0.01\\
594.01	0.01\\
595.01	0.01\\
596.01	0.01\\
597.01	0.01\\
598.01	0.00863908687856317\\
599.01	0.00623513569400518\\
599.02	0.00619747359804889\\
599.03	0.00615944475593821\\
599.04	0.00612104556201832\\
599.05	0.00608227237517919\\
599.06	0.00604312151850674\\
599.07	0.00600358927893092\\
599.08	0.00596367190687009\\
599.09	0.00592336561587206\\
599.1	0.00588266658225149\\
599.11	0.0058415709447238\\
599.12	0.00580007480403545\\
599.13	0.00575817422259057\\
599.14	0.00571586522407395\\
599.15	0.00567314379307032\\
599.16	0.00563000587467982\\
599.17	0.0055864473741298\\
599.18	0.00554246415638276\\
599.19	0.00549805204574034\\
599.2	0.00545320682544357\\
599.21	0.00540792423726908\\
599.22	0.00536219998112139\\
599.23	0.00531602971462118\\
599.24	0.00526940905268952\\
599.25	0.00522233356712793\\
599.26	0.0051747987861945\\
599.27	0.00512680019417563\\
599.28	0.00507833323095366\\
599.29	0.00502939329157028\\
599.3	0.0049799757257856\\
599.31	0.00493007583763287\\
599.32	0.00487968888496881\\
599.33	0.00482881007901966\\
599.34	0.00477743458392257\\
599.35	0.00472555751626259\\
599.36	0.00467317394460509\\
599.37	0.00462027888902355\\
599.38	0.00456686732062282\\
599.39	0.00451293416105752\\
599.4	0.00445847428204582\\
599.41	0.00440348251095326\\
599.42	0.00434795363554177\\
599.43	0.00429188239249111\\
599.44	0.00423526346689834\\
599.45	0.00417809149177218\\
599.46	0.00412036104752265\\
599.47	0.00406206666144548\\
599.48	0.00400320280720164\\
599.49	0.00394376390429169\\
599.5	0.00388374431752497\\
599.51	0.00382313835648363\\
599.52	0.0037619402749814\\
599.53	0.00370014427051701\\
599.54	0.00363774448372232\\
599.55	0.00357473499780497\\
599.56	0.00351110983798565\\
599.57	0.0034468629709298\\
599.58	0.0033819883041737\\
599.59	0.00331647968554508\\
599.6	0.00325033090257788\\
599.61	0.00318353568192138\\
599.62	0.00311608768874354\\
599.63	0.00304798052612844\\
599.64	0.00297920773446784\\
599.65	0.00290976279084679\\
599.66	0.00283963910842319\\
599.67	0.00276883003580131\\
599.68	0.00269732885639914\\
599.69	0.00262512878780953\\
599.7	0.00255222298115512\\
599.71	0.00247860452043688\\
599.72	0.00240426642187632\\
599.73	0.00232920163325123\\
599.74	0.0022534030332249\\
599.75	0.00217686343066884\\
599.76	0.0020995755639787\\
599.77	0.00202153210038368\\
599.78	0.00194272563524906\\
599.79	0.00186314869137187\\
599.8	0.00178279371826977\\
599.81	0.00170165309146284\\
599.82	0.00161971911174841\\
599.83	0.00153698400446879\\
599.84	0.00145343991877172\\
599.85	0.00136907892686371\\
599.86	0.00128389302325598\\
599.87	0.001197874124003\\
599.88	0.00111101406593362\\
599.89	0.00102330460587468\\
599.9	0.000934737419866898\\
599.91	0.000845304102373184\\
599.92	0.000754996165479166\\
599.93	0.000663805038085866\\
599.94	0.000571722065094472\\
599.95	0.000478738506583129\\
599.96	0.000384845536975638\\
599.97	0.000290034244202044\\
599.98	0.00019429562885097\\
599.99	9.76206033136556e-05\\
600	0\\
};
\addplot [color=mycolor3,solid,forget plot]
  table[row sep=crcr]{%
0.01	0.01\\
1.01	0.01\\
2.01	0.01\\
3.01	0.01\\
4.01	0.01\\
5.01	0.01\\
6.01	0.01\\
7.01	0.01\\
8.01	0.01\\
9.01	0.01\\
10.01	0.01\\
11.01	0.01\\
12.01	0.01\\
13.01	0.01\\
14.01	0.01\\
15.01	0.01\\
16.01	0.01\\
17.01	0.01\\
18.01	0.01\\
19.01	0.01\\
20.01	0.01\\
21.01	0.01\\
22.01	0.01\\
23.01	0.01\\
24.01	0.01\\
25.01	0.01\\
26.01	0.01\\
27.01	0.01\\
28.01	0.01\\
29.01	0.01\\
30.01	0.01\\
31.01	0.01\\
32.01	0.01\\
33.01	0.01\\
34.01	0.01\\
35.01	0.01\\
36.01	0.01\\
37.01	0.01\\
38.01	0.01\\
39.01	0.01\\
40.01	0.01\\
41.01	0.01\\
42.01	0.01\\
43.01	0.01\\
44.01	0.01\\
45.01	0.01\\
46.01	0.01\\
47.01	0.01\\
48.01	0.01\\
49.01	0.01\\
50.01	0.01\\
51.01	0.01\\
52.01	0.01\\
53.01	0.01\\
54.01	0.01\\
55.01	0.01\\
56.01	0.01\\
57.01	0.01\\
58.01	0.01\\
59.01	0.01\\
60.01	0.01\\
61.01	0.01\\
62.01	0.01\\
63.01	0.01\\
64.01	0.01\\
65.01	0.01\\
66.01	0.01\\
67.01	0.01\\
68.01	0.01\\
69.01	0.01\\
70.01	0.01\\
71.01	0.01\\
72.01	0.01\\
73.01	0.01\\
74.01	0.01\\
75.01	0.01\\
76.01	0.01\\
77.01	0.01\\
78.01	0.01\\
79.01	0.01\\
80.01	0.01\\
81.01	0.01\\
82.01	0.01\\
83.01	0.01\\
84.01	0.01\\
85.01	0.01\\
86.01	0.01\\
87.01	0.01\\
88.01	0.01\\
89.01	0.01\\
90.01	0.01\\
91.01	0.01\\
92.01	0.01\\
93.01	0.01\\
94.01	0.01\\
95.01	0.01\\
96.01	0.01\\
97.01	0.01\\
98.01	0.01\\
99.01	0.01\\
100.01	0.01\\
101.01	0.01\\
102.01	0.01\\
103.01	0.01\\
104.01	0.01\\
105.01	0.01\\
106.01	0.01\\
107.01	0.01\\
108.01	0.01\\
109.01	0.01\\
110.01	0.01\\
111.01	0.01\\
112.01	0.01\\
113.01	0.01\\
114.01	0.01\\
115.01	0.01\\
116.01	0.01\\
117.01	0.01\\
118.01	0.01\\
119.01	0.01\\
120.01	0.01\\
121.01	0.01\\
122.01	0.01\\
123.01	0.01\\
124.01	0.01\\
125.01	0.01\\
126.01	0.01\\
127.01	0.01\\
128.01	0.01\\
129.01	0.01\\
130.01	0.01\\
131.01	0.01\\
132.01	0.01\\
133.01	0.01\\
134.01	0.01\\
135.01	0.01\\
136.01	0.01\\
137.01	0.01\\
138.01	0.01\\
139.01	0.01\\
140.01	0.01\\
141.01	0.01\\
142.01	0.01\\
143.01	0.01\\
144.01	0.01\\
145.01	0.01\\
146.01	0.01\\
147.01	0.01\\
148.01	0.01\\
149.01	0.01\\
150.01	0.01\\
151.01	0.01\\
152.01	0.01\\
153.01	0.01\\
154.01	0.01\\
155.01	0.01\\
156.01	0.01\\
157.01	0.01\\
158.01	0.01\\
159.01	0.01\\
160.01	0.01\\
161.01	0.01\\
162.01	0.01\\
163.01	0.01\\
164.01	0.01\\
165.01	0.01\\
166.01	0.01\\
167.01	0.01\\
168.01	0.01\\
169.01	0.01\\
170.01	0.01\\
171.01	0.01\\
172.01	0.01\\
173.01	0.01\\
174.01	0.01\\
175.01	0.01\\
176.01	0.01\\
177.01	0.01\\
178.01	0.01\\
179.01	0.01\\
180.01	0.01\\
181.01	0.01\\
182.01	0.01\\
183.01	0.01\\
184.01	0.01\\
185.01	0.01\\
186.01	0.01\\
187.01	0.01\\
188.01	0.01\\
189.01	0.01\\
190.01	0.01\\
191.01	0.01\\
192.01	0.01\\
193.01	0.01\\
194.01	0.01\\
195.01	0.01\\
196.01	0.01\\
197.01	0.01\\
198.01	0.01\\
199.01	0.01\\
200.01	0.01\\
201.01	0.01\\
202.01	0.01\\
203.01	0.01\\
204.01	0.01\\
205.01	0.01\\
206.01	0.01\\
207.01	0.01\\
208.01	0.01\\
209.01	0.01\\
210.01	0.01\\
211.01	0.01\\
212.01	0.01\\
213.01	0.01\\
214.01	0.01\\
215.01	0.01\\
216.01	0.01\\
217.01	0.01\\
218.01	0.01\\
219.01	0.01\\
220.01	0.01\\
221.01	0.01\\
222.01	0.01\\
223.01	0.01\\
224.01	0.01\\
225.01	0.01\\
226.01	0.01\\
227.01	0.01\\
228.01	0.01\\
229.01	0.01\\
230.01	0.01\\
231.01	0.01\\
232.01	0.01\\
233.01	0.01\\
234.01	0.01\\
235.01	0.01\\
236.01	0.01\\
237.01	0.01\\
238.01	0.01\\
239.01	0.01\\
240.01	0.01\\
241.01	0.01\\
242.01	0.01\\
243.01	0.01\\
244.01	0.01\\
245.01	0.01\\
246.01	0.01\\
247.01	0.01\\
248.01	0.01\\
249.01	0.01\\
250.01	0.01\\
251.01	0.01\\
252.01	0.01\\
253.01	0.01\\
254.01	0.01\\
255.01	0.01\\
256.01	0.01\\
257.01	0.01\\
258.01	0.01\\
259.01	0.01\\
260.01	0.01\\
261.01	0.01\\
262.01	0.01\\
263.01	0.01\\
264.01	0.01\\
265.01	0.01\\
266.01	0.01\\
267.01	0.01\\
268.01	0.01\\
269.01	0.01\\
270.01	0.01\\
271.01	0.01\\
272.01	0.01\\
273.01	0.01\\
274.01	0.01\\
275.01	0.01\\
276.01	0.01\\
277.01	0.01\\
278.01	0.01\\
279.01	0.01\\
280.01	0.01\\
281.01	0.01\\
282.01	0.01\\
283.01	0.01\\
284.01	0.01\\
285.01	0.01\\
286.01	0.01\\
287.01	0.01\\
288.01	0.01\\
289.01	0.01\\
290.01	0.01\\
291.01	0.01\\
292.01	0.01\\
293.01	0.01\\
294.01	0.01\\
295.01	0.01\\
296.01	0.01\\
297.01	0.01\\
298.01	0.01\\
299.01	0.01\\
300.01	0.01\\
301.01	0.01\\
302.01	0.01\\
303.01	0.01\\
304.01	0.01\\
305.01	0.01\\
306.01	0.01\\
307.01	0.01\\
308.01	0.01\\
309.01	0.01\\
310.01	0.01\\
311.01	0.01\\
312.01	0.01\\
313.01	0.01\\
314.01	0.01\\
315.01	0.01\\
316.01	0.01\\
317.01	0.01\\
318.01	0.01\\
319.01	0.01\\
320.01	0.01\\
321.01	0.01\\
322.01	0.01\\
323.01	0.01\\
324.01	0.01\\
325.01	0.01\\
326.01	0.01\\
327.01	0.01\\
328.01	0.01\\
329.01	0.01\\
330.01	0.01\\
331.01	0.01\\
332.01	0.01\\
333.01	0.01\\
334.01	0.01\\
335.01	0.01\\
336.01	0.01\\
337.01	0.01\\
338.01	0.01\\
339.01	0.01\\
340.01	0.01\\
341.01	0.01\\
342.01	0.01\\
343.01	0.01\\
344.01	0.01\\
345.01	0.01\\
346.01	0.01\\
347.01	0.01\\
348.01	0.01\\
349.01	0.01\\
350.01	0.01\\
351.01	0.01\\
352.01	0.01\\
353.01	0.01\\
354.01	0.01\\
355.01	0.01\\
356.01	0.01\\
357.01	0.01\\
358.01	0.01\\
359.01	0.01\\
360.01	0.01\\
361.01	0.01\\
362.01	0.01\\
363.01	0.01\\
364.01	0.01\\
365.01	0.01\\
366.01	0.01\\
367.01	0.01\\
368.01	0.01\\
369.01	0.01\\
370.01	0.01\\
371.01	0.01\\
372.01	0.01\\
373.01	0.01\\
374.01	0.01\\
375.01	0.01\\
376.01	0.01\\
377.01	0.01\\
378.01	0.01\\
379.01	0.01\\
380.01	0.01\\
381.01	0.01\\
382.01	0.01\\
383.01	0.01\\
384.01	0.01\\
385.01	0.01\\
386.01	0.01\\
387.01	0.01\\
388.01	0.01\\
389.01	0.01\\
390.01	0.01\\
391.01	0.01\\
392.01	0.01\\
393.01	0.01\\
394.01	0.01\\
395.01	0.01\\
396.01	0.01\\
397.01	0.01\\
398.01	0.01\\
399.01	0.01\\
400.01	0.01\\
401.01	0.01\\
402.01	0.01\\
403.01	0.01\\
404.01	0.01\\
405.01	0.01\\
406.01	0.01\\
407.01	0.01\\
408.01	0.01\\
409.01	0.01\\
410.01	0.01\\
411.01	0.01\\
412.01	0.01\\
413.01	0.01\\
414.01	0.01\\
415.01	0.01\\
416.01	0.01\\
417.01	0.01\\
418.01	0.01\\
419.01	0.01\\
420.01	0.01\\
421.01	0.01\\
422.01	0.01\\
423.01	0.01\\
424.01	0.01\\
425.01	0.01\\
426.01	0.01\\
427.01	0.01\\
428.01	0.01\\
429.01	0.01\\
430.01	0.01\\
431.01	0.01\\
432.01	0.01\\
433.01	0.01\\
434.01	0.01\\
435.01	0.01\\
436.01	0.01\\
437.01	0.01\\
438.01	0.01\\
439.01	0.01\\
440.01	0.01\\
441.01	0.01\\
442.01	0.01\\
443.01	0.01\\
444.01	0.01\\
445.01	0.01\\
446.01	0.01\\
447.01	0.01\\
448.01	0.01\\
449.01	0.01\\
450.01	0.01\\
451.01	0.01\\
452.01	0.01\\
453.01	0.01\\
454.01	0.01\\
455.01	0.01\\
456.01	0.01\\
457.01	0.01\\
458.01	0.01\\
459.01	0.01\\
460.01	0.01\\
461.01	0.01\\
462.01	0.01\\
463.01	0.01\\
464.01	0.01\\
465.01	0.01\\
466.01	0.01\\
467.01	0.01\\
468.01	0.01\\
469.01	0.01\\
470.01	0.01\\
471.01	0.01\\
472.01	0.01\\
473.01	0.01\\
474.01	0.01\\
475.01	0.01\\
476.01	0.01\\
477.01	0.01\\
478.01	0.01\\
479.01	0.01\\
480.01	0.01\\
481.01	0.01\\
482.01	0.01\\
483.01	0.01\\
484.01	0.01\\
485.01	0.01\\
486.01	0.01\\
487.01	0.01\\
488.01	0.01\\
489.01	0.01\\
490.01	0.01\\
491.01	0.01\\
492.01	0.01\\
493.01	0.01\\
494.01	0.01\\
495.01	0.01\\
496.01	0.01\\
497.01	0.01\\
498.01	0.01\\
499.01	0.01\\
500.01	0.01\\
501.01	0.01\\
502.01	0.01\\
503.01	0.01\\
504.01	0.01\\
505.01	0.01\\
506.01	0.01\\
507.01	0.01\\
508.01	0.01\\
509.01	0.01\\
510.01	0.01\\
511.01	0.01\\
512.01	0.01\\
513.01	0.01\\
514.01	0.01\\
515.01	0.01\\
516.01	0.01\\
517.01	0.01\\
518.01	0.01\\
519.01	0.01\\
520.01	0.01\\
521.01	0.01\\
522.01	0.01\\
523.01	0.01\\
524.01	0.01\\
525.01	0.01\\
526.01	0.01\\
527.01	0.01\\
528.01	0.01\\
529.01	0.01\\
530.01	0.01\\
531.01	0.01\\
532.01	0.01\\
533.01	0.01\\
534.01	0.01\\
535.01	0.01\\
536.01	0.01\\
537.01	0.01\\
538.01	0.01\\
539.01	0.01\\
540.01	0.01\\
541.01	0.01\\
542.01	0.01\\
543.01	0.01\\
544.01	0.01\\
545.01	0.01\\
546.01	0.01\\
547.01	0.01\\
548.01	0.01\\
549.01	0.01\\
550.01	0.01\\
551.01	0.01\\
552.01	0.01\\
553.01	0.01\\
554.01	0.01\\
555.01	0.01\\
556.01	0.01\\
557.01	0.01\\
558.01	0.01\\
559.01	0.01\\
560.01	0.01\\
561.01	0.01\\
562.01	0.01\\
563.01	0.01\\
564.01	0.01\\
565.01	0.01\\
566.01	0.01\\
567.01	0.01\\
568.01	0.01\\
569.01	0.01\\
570.01	0.01\\
571.01	0.01\\
572.01	0.01\\
573.01	0.01\\
574.01	0.01\\
575.01	0.01\\
576.01	0.01\\
577.01	0.01\\
578.01	0.01\\
579.01	0.01\\
580.01	0.01\\
581.01	0.01\\
582.01	0.01\\
583.01	0.01\\
584.01	0.01\\
585.01	0.01\\
586.01	0.01\\
587.01	0.01\\
588.01	0.01\\
589.01	0.01\\
590.01	0.01\\
591.01	0.01\\
592.01	0.01\\
593.01	0.01\\
594.01	0.01\\
595.01	0.01\\
596.01	0.01\\
597.01	0.01\\
598.01	0.00863963718816675\\
599.01	0.00623513569400517\\
599.02	0.00619747359804888\\
599.03	0.00615944475593818\\
599.04	0.00612104556201831\\
599.05	0.00608227237517917\\
599.06	0.00604312151850673\\
599.07	0.00600358927893092\\
599.08	0.00596367190687011\\
599.09	0.00592336561587207\\
599.1	0.0058826665822515\\
599.11	0.00584157094472381\\
599.12	0.00580007480403546\\
599.13	0.00575817422259058\\
599.14	0.00571586522407398\\
599.15	0.00567314379307033\\
599.16	0.00563000587467983\\
599.17	0.00558644737412983\\
599.18	0.00554246415638279\\
599.19	0.00549805204574037\\
599.2	0.00545320682544359\\
599.21	0.00540792423726908\\
599.22	0.0053621999811214\\
599.23	0.0053160297146212\\
599.24	0.00526940905268953\\
599.25	0.00522233356712796\\
599.26	0.00517479878619454\\
599.27	0.00512680019417566\\
599.28	0.0050783332309537\\
599.29	0.00502939329157033\\
599.3	0.00497997572578563\\
599.31	0.00493007583763288\\
599.32	0.00487968888496882\\
599.33	0.00482881007901966\\
599.34	0.00477743458392257\\
599.35	0.00472555751626257\\
599.36	0.00467317394460506\\
599.37	0.00462027888902354\\
599.38	0.00456686732062282\\
599.39	0.00451293416105752\\
599.4	0.00445847428204582\\
599.41	0.00440348251095325\\
599.42	0.00434795363554174\\
599.43	0.0042918823924911\\
599.44	0.00423526346689832\\
599.45	0.00417809149177218\\
599.46	0.00412036104752265\\
599.47	0.00406206666144548\\
599.48	0.00400320280720163\\
599.49	0.00394376390429167\\
599.5	0.00388374431752494\\
599.51	0.00382313835648361\\
599.52	0.00376194027498137\\
599.53	0.00370014427051699\\
599.54	0.0036377444837223\\
599.55	0.00357473499780496\\
599.56	0.00351110983798566\\
599.57	0.0034468629709298\\
599.58	0.00338198830417371\\
599.59	0.00331647968554508\\
599.6	0.00325033090257786\\
599.61	0.00318353568192138\\
599.62	0.00311608768874353\\
599.63	0.00304798052612843\\
599.64	0.00297920773446783\\
599.65	0.00290976279084677\\
599.66	0.00283963910842317\\
599.67	0.00276883003580129\\
599.68	0.00269732885639912\\
599.69	0.00262512878780952\\
599.7	0.00255222298115511\\
599.71	0.00247860452043686\\
599.72	0.0024042664218763\\
599.73	0.00232920163325121\\
599.74	0.00225340303322488\\
599.75	0.00217686343066882\\
599.76	0.00209957556397868\\
599.77	0.00202153210038367\\
599.78	0.00194272563524905\\
599.79	0.00186314869137186\\
599.8	0.00178279371826976\\
599.81	0.00170165309146283\\
599.82	0.00161971911174842\\
599.83	0.0015369840044688\\
599.84	0.00145343991877173\\
599.85	0.00136907892686372\\
599.86	0.00128389302325599\\
599.87	0.001197874124003\\
599.88	0.00111101406593363\\
599.89	0.00102330460587469\\
599.9	0.000934737419866907\\
599.91	0.00084530410237319\\
599.92	0.000754996165479171\\
599.93	0.000663805038085873\\
599.94	0.000571722065094477\\
599.95	0.000478738506583131\\
599.96	0.000384845536975641\\
599.97	0.000290034244202044\\
599.98	0.00019429562885097\\
599.99	9.76206033136574e-05\\
600	0\\
};
\addplot [color=mycolor4,solid,forget plot]
  table[row sep=crcr]{%
0.01	0.01\\
1.01	0.01\\
2.01	0.01\\
3.01	0.01\\
4.01	0.01\\
5.01	0.01\\
6.01	0.01\\
7.01	0.01\\
8.01	0.01\\
9.01	0.01\\
10.01	0.01\\
11.01	0.01\\
12.01	0.01\\
13.01	0.01\\
14.01	0.01\\
15.01	0.01\\
16.01	0.01\\
17.01	0.01\\
18.01	0.01\\
19.01	0.01\\
20.01	0.01\\
21.01	0.01\\
22.01	0.01\\
23.01	0.01\\
24.01	0.01\\
25.01	0.01\\
26.01	0.01\\
27.01	0.01\\
28.01	0.01\\
29.01	0.01\\
30.01	0.01\\
31.01	0.01\\
32.01	0.01\\
33.01	0.01\\
34.01	0.01\\
35.01	0.01\\
36.01	0.01\\
37.01	0.01\\
38.01	0.01\\
39.01	0.01\\
40.01	0.01\\
41.01	0.01\\
42.01	0.01\\
43.01	0.01\\
44.01	0.01\\
45.01	0.01\\
46.01	0.01\\
47.01	0.01\\
48.01	0.01\\
49.01	0.01\\
50.01	0.01\\
51.01	0.01\\
52.01	0.01\\
53.01	0.01\\
54.01	0.01\\
55.01	0.01\\
56.01	0.01\\
57.01	0.01\\
58.01	0.01\\
59.01	0.01\\
60.01	0.01\\
61.01	0.01\\
62.01	0.01\\
63.01	0.01\\
64.01	0.01\\
65.01	0.01\\
66.01	0.01\\
67.01	0.01\\
68.01	0.01\\
69.01	0.01\\
70.01	0.01\\
71.01	0.01\\
72.01	0.01\\
73.01	0.01\\
74.01	0.01\\
75.01	0.01\\
76.01	0.01\\
77.01	0.01\\
78.01	0.01\\
79.01	0.01\\
80.01	0.01\\
81.01	0.01\\
82.01	0.01\\
83.01	0.01\\
84.01	0.01\\
85.01	0.01\\
86.01	0.01\\
87.01	0.01\\
88.01	0.01\\
89.01	0.01\\
90.01	0.01\\
91.01	0.01\\
92.01	0.01\\
93.01	0.01\\
94.01	0.01\\
95.01	0.01\\
96.01	0.01\\
97.01	0.01\\
98.01	0.01\\
99.01	0.01\\
100.01	0.01\\
101.01	0.01\\
102.01	0.01\\
103.01	0.01\\
104.01	0.01\\
105.01	0.01\\
106.01	0.01\\
107.01	0.01\\
108.01	0.01\\
109.01	0.01\\
110.01	0.01\\
111.01	0.01\\
112.01	0.01\\
113.01	0.01\\
114.01	0.01\\
115.01	0.01\\
116.01	0.01\\
117.01	0.01\\
118.01	0.01\\
119.01	0.01\\
120.01	0.01\\
121.01	0.01\\
122.01	0.01\\
123.01	0.01\\
124.01	0.01\\
125.01	0.01\\
126.01	0.01\\
127.01	0.01\\
128.01	0.01\\
129.01	0.01\\
130.01	0.01\\
131.01	0.01\\
132.01	0.01\\
133.01	0.01\\
134.01	0.01\\
135.01	0.01\\
136.01	0.01\\
137.01	0.01\\
138.01	0.01\\
139.01	0.01\\
140.01	0.01\\
141.01	0.01\\
142.01	0.01\\
143.01	0.01\\
144.01	0.01\\
145.01	0.01\\
146.01	0.01\\
147.01	0.01\\
148.01	0.01\\
149.01	0.01\\
150.01	0.01\\
151.01	0.01\\
152.01	0.01\\
153.01	0.01\\
154.01	0.01\\
155.01	0.01\\
156.01	0.01\\
157.01	0.01\\
158.01	0.01\\
159.01	0.01\\
160.01	0.01\\
161.01	0.01\\
162.01	0.01\\
163.01	0.01\\
164.01	0.01\\
165.01	0.01\\
166.01	0.01\\
167.01	0.01\\
168.01	0.01\\
169.01	0.01\\
170.01	0.01\\
171.01	0.01\\
172.01	0.01\\
173.01	0.01\\
174.01	0.01\\
175.01	0.01\\
176.01	0.01\\
177.01	0.01\\
178.01	0.01\\
179.01	0.01\\
180.01	0.01\\
181.01	0.01\\
182.01	0.01\\
183.01	0.01\\
184.01	0.01\\
185.01	0.01\\
186.01	0.01\\
187.01	0.01\\
188.01	0.01\\
189.01	0.01\\
190.01	0.01\\
191.01	0.01\\
192.01	0.01\\
193.01	0.01\\
194.01	0.01\\
195.01	0.01\\
196.01	0.01\\
197.01	0.01\\
198.01	0.01\\
199.01	0.01\\
200.01	0.01\\
201.01	0.01\\
202.01	0.01\\
203.01	0.01\\
204.01	0.01\\
205.01	0.01\\
206.01	0.01\\
207.01	0.01\\
208.01	0.01\\
209.01	0.01\\
210.01	0.01\\
211.01	0.01\\
212.01	0.01\\
213.01	0.01\\
214.01	0.01\\
215.01	0.01\\
216.01	0.01\\
217.01	0.01\\
218.01	0.01\\
219.01	0.01\\
220.01	0.01\\
221.01	0.01\\
222.01	0.01\\
223.01	0.01\\
224.01	0.01\\
225.01	0.01\\
226.01	0.01\\
227.01	0.01\\
228.01	0.01\\
229.01	0.01\\
230.01	0.01\\
231.01	0.01\\
232.01	0.01\\
233.01	0.01\\
234.01	0.01\\
235.01	0.01\\
236.01	0.01\\
237.01	0.01\\
238.01	0.01\\
239.01	0.01\\
240.01	0.01\\
241.01	0.01\\
242.01	0.01\\
243.01	0.01\\
244.01	0.01\\
245.01	0.01\\
246.01	0.01\\
247.01	0.01\\
248.01	0.01\\
249.01	0.01\\
250.01	0.01\\
251.01	0.01\\
252.01	0.01\\
253.01	0.01\\
254.01	0.01\\
255.01	0.01\\
256.01	0.01\\
257.01	0.01\\
258.01	0.01\\
259.01	0.01\\
260.01	0.01\\
261.01	0.01\\
262.01	0.01\\
263.01	0.01\\
264.01	0.01\\
265.01	0.01\\
266.01	0.01\\
267.01	0.01\\
268.01	0.01\\
269.01	0.01\\
270.01	0.01\\
271.01	0.01\\
272.01	0.01\\
273.01	0.01\\
274.01	0.01\\
275.01	0.01\\
276.01	0.01\\
277.01	0.01\\
278.01	0.01\\
279.01	0.01\\
280.01	0.01\\
281.01	0.01\\
282.01	0.01\\
283.01	0.01\\
284.01	0.01\\
285.01	0.01\\
286.01	0.01\\
287.01	0.01\\
288.01	0.01\\
289.01	0.01\\
290.01	0.01\\
291.01	0.01\\
292.01	0.01\\
293.01	0.01\\
294.01	0.01\\
295.01	0.01\\
296.01	0.01\\
297.01	0.01\\
298.01	0.01\\
299.01	0.01\\
300.01	0.01\\
301.01	0.01\\
302.01	0.01\\
303.01	0.01\\
304.01	0.01\\
305.01	0.01\\
306.01	0.01\\
307.01	0.01\\
308.01	0.01\\
309.01	0.01\\
310.01	0.01\\
311.01	0.01\\
312.01	0.01\\
313.01	0.01\\
314.01	0.01\\
315.01	0.01\\
316.01	0.01\\
317.01	0.01\\
318.01	0.01\\
319.01	0.01\\
320.01	0.01\\
321.01	0.01\\
322.01	0.01\\
323.01	0.01\\
324.01	0.01\\
325.01	0.01\\
326.01	0.01\\
327.01	0.01\\
328.01	0.01\\
329.01	0.01\\
330.01	0.01\\
331.01	0.01\\
332.01	0.01\\
333.01	0.01\\
334.01	0.01\\
335.01	0.01\\
336.01	0.01\\
337.01	0.01\\
338.01	0.01\\
339.01	0.01\\
340.01	0.01\\
341.01	0.01\\
342.01	0.01\\
343.01	0.01\\
344.01	0.01\\
345.01	0.01\\
346.01	0.01\\
347.01	0.01\\
348.01	0.01\\
349.01	0.01\\
350.01	0.01\\
351.01	0.01\\
352.01	0.01\\
353.01	0.01\\
354.01	0.01\\
355.01	0.01\\
356.01	0.01\\
357.01	0.01\\
358.01	0.01\\
359.01	0.01\\
360.01	0.01\\
361.01	0.01\\
362.01	0.01\\
363.01	0.01\\
364.01	0.01\\
365.01	0.01\\
366.01	0.01\\
367.01	0.01\\
368.01	0.01\\
369.01	0.01\\
370.01	0.01\\
371.01	0.01\\
372.01	0.01\\
373.01	0.01\\
374.01	0.01\\
375.01	0.01\\
376.01	0.01\\
377.01	0.01\\
378.01	0.01\\
379.01	0.01\\
380.01	0.01\\
381.01	0.01\\
382.01	0.01\\
383.01	0.01\\
384.01	0.01\\
385.01	0.01\\
386.01	0.01\\
387.01	0.01\\
388.01	0.01\\
389.01	0.01\\
390.01	0.01\\
391.01	0.01\\
392.01	0.01\\
393.01	0.01\\
394.01	0.01\\
395.01	0.01\\
396.01	0.01\\
397.01	0.01\\
398.01	0.01\\
399.01	0.01\\
400.01	0.01\\
401.01	0.01\\
402.01	0.01\\
403.01	0.01\\
404.01	0.01\\
405.01	0.01\\
406.01	0.01\\
407.01	0.01\\
408.01	0.01\\
409.01	0.01\\
410.01	0.01\\
411.01	0.01\\
412.01	0.01\\
413.01	0.01\\
414.01	0.01\\
415.01	0.01\\
416.01	0.01\\
417.01	0.01\\
418.01	0.01\\
419.01	0.01\\
420.01	0.01\\
421.01	0.01\\
422.01	0.01\\
423.01	0.01\\
424.01	0.01\\
425.01	0.01\\
426.01	0.01\\
427.01	0.01\\
428.01	0.01\\
429.01	0.01\\
430.01	0.01\\
431.01	0.01\\
432.01	0.01\\
433.01	0.01\\
434.01	0.01\\
435.01	0.01\\
436.01	0.01\\
437.01	0.01\\
438.01	0.01\\
439.01	0.01\\
440.01	0.01\\
441.01	0.01\\
442.01	0.01\\
443.01	0.01\\
444.01	0.01\\
445.01	0.01\\
446.01	0.01\\
447.01	0.01\\
448.01	0.01\\
449.01	0.01\\
450.01	0.01\\
451.01	0.01\\
452.01	0.01\\
453.01	0.01\\
454.01	0.01\\
455.01	0.01\\
456.01	0.01\\
457.01	0.01\\
458.01	0.01\\
459.01	0.01\\
460.01	0.01\\
461.01	0.01\\
462.01	0.01\\
463.01	0.01\\
464.01	0.01\\
465.01	0.01\\
466.01	0.01\\
467.01	0.01\\
468.01	0.01\\
469.01	0.01\\
470.01	0.01\\
471.01	0.01\\
472.01	0.01\\
473.01	0.01\\
474.01	0.01\\
475.01	0.01\\
476.01	0.01\\
477.01	0.01\\
478.01	0.01\\
479.01	0.01\\
480.01	0.01\\
481.01	0.01\\
482.01	0.01\\
483.01	0.01\\
484.01	0.01\\
485.01	0.01\\
486.01	0.01\\
487.01	0.01\\
488.01	0.01\\
489.01	0.01\\
490.01	0.01\\
491.01	0.01\\
492.01	0.01\\
493.01	0.01\\
494.01	0.01\\
495.01	0.01\\
496.01	0.01\\
497.01	0.01\\
498.01	0.01\\
499.01	0.01\\
500.01	0.01\\
501.01	0.01\\
502.01	0.01\\
503.01	0.01\\
504.01	0.01\\
505.01	0.01\\
506.01	0.01\\
507.01	0.01\\
508.01	0.01\\
509.01	0.01\\
510.01	0.01\\
511.01	0.01\\
512.01	0.01\\
513.01	0.01\\
514.01	0.01\\
515.01	0.01\\
516.01	0.01\\
517.01	0.01\\
518.01	0.01\\
519.01	0.01\\
520.01	0.01\\
521.01	0.01\\
522.01	0.01\\
523.01	0.01\\
524.01	0.01\\
525.01	0.01\\
526.01	0.01\\
527.01	0.01\\
528.01	0.01\\
529.01	0.01\\
530.01	0.01\\
531.01	0.01\\
532.01	0.01\\
533.01	0.01\\
534.01	0.01\\
535.01	0.01\\
536.01	0.01\\
537.01	0.01\\
538.01	0.01\\
539.01	0.01\\
540.01	0.01\\
541.01	0.01\\
542.01	0.01\\
543.01	0.01\\
544.01	0.01\\
545.01	0.01\\
546.01	0.01\\
547.01	0.01\\
548.01	0.01\\
549.01	0.01\\
550.01	0.01\\
551.01	0.01\\
552.01	0.01\\
553.01	0.01\\
554.01	0.01\\
555.01	0.01\\
556.01	0.01\\
557.01	0.01\\
558.01	0.01\\
559.01	0.01\\
560.01	0.01\\
561.01	0.01\\
562.01	0.01\\
563.01	0.01\\
564.01	0.01\\
565.01	0.01\\
566.01	0.01\\
567.01	0.01\\
568.01	0.01\\
569.01	0.01\\
570.01	0.01\\
571.01	0.01\\
572.01	0.01\\
573.01	0.01\\
574.01	0.01\\
575.01	0.01\\
576.01	0.01\\
577.01	0.01\\
578.01	0.01\\
579.01	0.01\\
580.01	0.01\\
581.01	0.01\\
582.01	0.01\\
583.01	0.01\\
584.01	0.01\\
585.01	0.01\\
586.01	0.01\\
587.01	0.01\\
588.01	0.01\\
589.01	0.01\\
590.01	0.01\\
591.01	0.01\\
592.01	0.01\\
593.01	0.01\\
594.01	0.01\\
595.01	0.01\\
596.01	0.01\\
597.01	0.01\\
598.01	0.00912621121604349\\
599.01	0.00623513569400514\\
599.02	0.00619747359804885\\
599.03	0.00615944475593814\\
599.04	0.00612104556201827\\
599.05	0.00608227237517913\\
599.06	0.0060431215185067\\
599.07	0.00600358927893089\\
599.08	0.00596367190687006\\
599.09	0.00592336561587204\\
599.1	0.00588266658225147\\
599.11	0.00584157094472378\\
599.12	0.00580007480403543\\
599.13	0.00575817422259054\\
599.14	0.00571586522407392\\
599.15	0.00567314379307028\\
599.16	0.00563000587467979\\
599.17	0.00558644737412979\\
599.18	0.00554246415638275\\
599.19	0.00549805204574031\\
599.2	0.00545320682544354\\
599.21	0.00540792423726905\\
599.22	0.00536219998112137\\
599.23	0.00531602971462116\\
599.24	0.0052694090526895\\
599.25	0.00522233356712793\\
599.26	0.0051747987861945\\
599.27	0.00512680019417563\\
599.28	0.00507833323095366\\
599.29	0.00502939329157028\\
599.3	0.0049799757257856\\
599.31	0.00493007583763285\\
599.32	0.00487968888496879\\
599.33	0.00482881007901965\\
599.34	0.00477743458392255\\
599.35	0.00472555751626257\\
599.36	0.00467317394460506\\
599.37	0.00462027888902352\\
599.38	0.00456686732062279\\
599.39	0.00451293416105749\\
599.4	0.00445847428204579\\
599.41	0.00440348251095323\\
599.42	0.00434795363554174\\
599.43	0.00429188239249109\\
599.44	0.00423526346689831\\
599.45	0.00417809149177215\\
599.46	0.00412036104752262\\
599.47	0.00406206666144545\\
599.48	0.00400320280720161\\
599.49	0.00394376390429166\\
599.5	0.00388374431752493\\
599.51	0.00382313835648359\\
599.52	0.00376194027498136\\
599.53	0.00370014427051698\\
599.54	0.00363774448372228\\
599.55	0.00357473499780494\\
599.56	0.00351110983798563\\
599.57	0.00344686297092977\\
599.58	0.00338198830417368\\
599.59	0.00331647968554506\\
599.6	0.00325033090257784\\
599.61	0.00318353568192134\\
599.62	0.00311608768874351\\
599.63	0.00304798052612841\\
599.64	0.00297920773446781\\
599.65	0.00290976279084676\\
599.66	0.00283963910842317\\
599.67	0.00276883003580129\\
599.68	0.00269732885639912\\
599.69	0.00262512878780951\\
599.7	0.0025522229811551\\
599.71	0.00247860452043686\\
599.72	0.00240426642187629\\
599.73	0.0023292016332512\\
599.74	0.00225340303322488\\
599.75	0.00217686343066882\\
599.76	0.00209957556397868\\
599.77	0.00202153210038366\\
599.78	0.00194272563524904\\
599.79	0.00186314869137186\\
599.8	0.00178279371826975\\
599.81	0.00170165309146282\\
599.82	0.00161971911174841\\
599.83	0.00153698400446878\\
599.84	0.00145343991877171\\
599.85	0.00136907892686371\\
599.86	0.00128389302325597\\
599.87	0.00119787412400299\\
599.88	0.00111101406593362\\
599.89	0.00102330460587469\\
599.9	0.0009347374198669\\
599.91	0.000845304102373186\\
599.92	0.000754996165479164\\
599.93	0.000663805038085868\\
599.94	0.000571722065094472\\
599.95	0.000478738506583129\\
599.96	0.000384845536975638\\
599.97	0.000290034244202044\\
599.98	0.00019429562885097\\
599.99	9.76206033136574e-05\\
600	0\\
};
\addplot [color=mycolor5,solid,forget plot]
  table[row sep=crcr]{%
0.01	0.01\\
1.01	0.01\\
2.01	0.01\\
3.01	0.01\\
4.01	0.01\\
5.01	0.01\\
6.01	0.01\\
7.01	0.01\\
8.01	0.01\\
9.01	0.01\\
10.01	0.01\\
11.01	0.01\\
12.01	0.01\\
13.01	0.01\\
14.01	0.01\\
15.01	0.01\\
16.01	0.01\\
17.01	0.01\\
18.01	0.01\\
19.01	0.01\\
20.01	0.01\\
21.01	0.01\\
22.01	0.01\\
23.01	0.01\\
24.01	0.01\\
25.01	0.01\\
26.01	0.01\\
27.01	0.01\\
28.01	0.01\\
29.01	0.01\\
30.01	0.01\\
31.01	0.01\\
32.01	0.01\\
33.01	0.01\\
34.01	0.01\\
35.01	0.01\\
36.01	0.01\\
37.01	0.01\\
38.01	0.01\\
39.01	0.01\\
40.01	0.01\\
41.01	0.01\\
42.01	0.01\\
43.01	0.01\\
44.01	0.01\\
45.01	0.01\\
46.01	0.01\\
47.01	0.01\\
48.01	0.01\\
49.01	0.01\\
50.01	0.01\\
51.01	0.01\\
52.01	0.01\\
53.01	0.01\\
54.01	0.01\\
55.01	0.01\\
56.01	0.01\\
57.01	0.01\\
58.01	0.01\\
59.01	0.01\\
60.01	0.01\\
61.01	0.01\\
62.01	0.01\\
63.01	0.01\\
64.01	0.01\\
65.01	0.01\\
66.01	0.01\\
67.01	0.01\\
68.01	0.01\\
69.01	0.01\\
70.01	0.01\\
71.01	0.01\\
72.01	0.01\\
73.01	0.01\\
74.01	0.01\\
75.01	0.01\\
76.01	0.01\\
77.01	0.01\\
78.01	0.01\\
79.01	0.01\\
80.01	0.01\\
81.01	0.01\\
82.01	0.01\\
83.01	0.01\\
84.01	0.01\\
85.01	0.01\\
86.01	0.01\\
87.01	0.01\\
88.01	0.01\\
89.01	0.01\\
90.01	0.01\\
91.01	0.01\\
92.01	0.01\\
93.01	0.01\\
94.01	0.01\\
95.01	0.01\\
96.01	0.01\\
97.01	0.01\\
98.01	0.01\\
99.01	0.01\\
100.01	0.01\\
101.01	0.01\\
102.01	0.01\\
103.01	0.01\\
104.01	0.01\\
105.01	0.01\\
106.01	0.01\\
107.01	0.01\\
108.01	0.01\\
109.01	0.01\\
110.01	0.01\\
111.01	0.01\\
112.01	0.01\\
113.01	0.01\\
114.01	0.01\\
115.01	0.01\\
116.01	0.01\\
117.01	0.01\\
118.01	0.01\\
119.01	0.01\\
120.01	0.01\\
121.01	0.01\\
122.01	0.01\\
123.01	0.01\\
124.01	0.01\\
125.01	0.01\\
126.01	0.01\\
127.01	0.01\\
128.01	0.01\\
129.01	0.01\\
130.01	0.01\\
131.01	0.01\\
132.01	0.01\\
133.01	0.01\\
134.01	0.01\\
135.01	0.01\\
136.01	0.01\\
137.01	0.01\\
138.01	0.01\\
139.01	0.01\\
140.01	0.01\\
141.01	0.01\\
142.01	0.01\\
143.01	0.01\\
144.01	0.01\\
145.01	0.01\\
146.01	0.01\\
147.01	0.01\\
148.01	0.01\\
149.01	0.01\\
150.01	0.01\\
151.01	0.01\\
152.01	0.01\\
153.01	0.01\\
154.01	0.01\\
155.01	0.01\\
156.01	0.01\\
157.01	0.01\\
158.01	0.01\\
159.01	0.01\\
160.01	0.01\\
161.01	0.01\\
162.01	0.01\\
163.01	0.01\\
164.01	0.01\\
165.01	0.01\\
166.01	0.01\\
167.01	0.01\\
168.01	0.01\\
169.01	0.01\\
170.01	0.01\\
171.01	0.01\\
172.01	0.01\\
173.01	0.01\\
174.01	0.01\\
175.01	0.01\\
176.01	0.01\\
177.01	0.01\\
178.01	0.01\\
179.01	0.01\\
180.01	0.01\\
181.01	0.01\\
182.01	0.01\\
183.01	0.01\\
184.01	0.01\\
185.01	0.01\\
186.01	0.01\\
187.01	0.01\\
188.01	0.01\\
189.01	0.01\\
190.01	0.01\\
191.01	0.01\\
192.01	0.01\\
193.01	0.01\\
194.01	0.01\\
195.01	0.01\\
196.01	0.01\\
197.01	0.01\\
198.01	0.01\\
199.01	0.01\\
200.01	0.01\\
201.01	0.01\\
202.01	0.01\\
203.01	0.01\\
204.01	0.01\\
205.01	0.01\\
206.01	0.01\\
207.01	0.01\\
208.01	0.01\\
209.01	0.01\\
210.01	0.01\\
211.01	0.01\\
212.01	0.01\\
213.01	0.01\\
214.01	0.01\\
215.01	0.01\\
216.01	0.01\\
217.01	0.01\\
218.01	0.01\\
219.01	0.01\\
220.01	0.01\\
221.01	0.01\\
222.01	0.01\\
223.01	0.01\\
224.01	0.01\\
225.01	0.01\\
226.01	0.01\\
227.01	0.01\\
228.01	0.01\\
229.01	0.01\\
230.01	0.01\\
231.01	0.01\\
232.01	0.01\\
233.01	0.01\\
234.01	0.01\\
235.01	0.01\\
236.01	0.01\\
237.01	0.01\\
238.01	0.01\\
239.01	0.01\\
240.01	0.01\\
241.01	0.01\\
242.01	0.01\\
243.01	0.01\\
244.01	0.01\\
245.01	0.01\\
246.01	0.01\\
247.01	0.01\\
248.01	0.01\\
249.01	0.01\\
250.01	0.01\\
251.01	0.01\\
252.01	0.01\\
253.01	0.01\\
254.01	0.01\\
255.01	0.01\\
256.01	0.01\\
257.01	0.01\\
258.01	0.01\\
259.01	0.01\\
260.01	0.01\\
261.01	0.01\\
262.01	0.01\\
263.01	0.01\\
264.01	0.01\\
265.01	0.01\\
266.01	0.01\\
267.01	0.01\\
268.01	0.01\\
269.01	0.01\\
270.01	0.01\\
271.01	0.01\\
272.01	0.01\\
273.01	0.01\\
274.01	0.01\\
275.01	0.01\\
276.01	0.01\\
277.01	0.01\\
278.01	0.01\\
279.01	0.01\\
280.01	0.01\\
281.01	0.01\\
282.01	0.01\\
283.01	0.01\\
284.01	0.01\\
285.01	0.01\\
286.01	0.01\\
287.01	0.01\\
288.01	0.01\\
289.01	0.01\\
290.01	0.01\\
291.01	0.01\\
292.01	0.01\\
293.01	0.01\\
294.01	0.01\\
295.01	0.01\\
296.01	0.01\\
297.01	0.01\\
298.01	0.01\\
299.01	0.01\\
300.01	0.01\\
301.01	0.01\\
302.01	0.01\\
303.01	0.01\\
304.01	0.01\\
305.01	0.01\\
306.01	0.01\\
307.01	0.01\\
308.01	0.01\\
309.01	0.01\\
310.01	0.01\\
311.01	0.01\\
312.01	0.01\\
313.01	0.01\\
314.01	0.01\\
315.01	0.01\\
316.01	0.01\\
317.01	0.01\\
318.01	0.01\\
319.01	0.01\\
320.01	0.01\\
321.01	0.01\\
322.01	0.01\\
323.01	0.01\\
324.01	0.01\\
325.01	0.01\\
326.01	0.01\\
327.01	0.01\\
328.01	0.01\\
329.01	0.01\\
330.01	0.01\\
331.01	0.01\\
332.01	0.01\\
333.01	0.01\\
334.01	0.01\\
335.01	0.01\\
336.01	0.01\\
337.01	0.01\\
338.01	0.01\\
339.01	0.01\\
340.01	0.01\\
341.01	0.01\\
342.01	0.01\\
343.01	0.01\\
344.01	0.01\\
345.01	0.01\\
346.01	0.01\\
347.01	0.01\\
348.01	0.01\\
349.01	0.01\\
350.01	0.01\\
351.01	0.01\\
352.01	0.01\\
353.01	0.01\\
354.01	0.01\\
355.01	0.01\\
356.01	0.01\\
357.01	0.01\\
358.01	0.01\\
359.01	0.01\\
360.01	0.01\\
361.01	0.01\\
362.01	0.01\\
363.01	0.01\\
364.01	0.01\\
365.01	0.01\\
366.01	0.01\\
367.01	0.01\\
368.01	0.01\\
369.01	0.01\\
370.01	0.01\\
371.01	0.01\\
372.01	0.01\\
373.01	0.01\\
374.01	0.01\\
375.01	0.01\\
376.01	0.01\\
377.01	0.01\\
378.01	0.01\\
379.01	0.01\\
380.01	0.01\\
381.01	0.01\\
382.01	0.01\\
383.01	0.01\\
384.01	0.01\\
385.01	0.01\\
386.01	0.01\\
387.01	0.01\\
388.01	0.01\\
389.01	0.01\\
390.01	0.01\\
391.01	0.01\\
392.01	0.01\\
393.01	0.01\\
394.01	0.01\\
395.01	0.01\\
396.01	0.01\\
397.01	0.01\\
398.01	0.01\\
399.01	0.01\\
400.01	0.01\\
401.01	0.01\\
402.01	0.01\\
403.01	0.01\\
404.01	0.01\\
405.01	0.01\\
406.01	0.01\\
407.01	0.01\\
408.01	0.01\\
409.01	0.01\\
410.01	0.01\\
411.01	0.01\\
412.01	0.01\\
413.01	0.01\\
414.01	0.01\\
415.01	0.01\\
416.01	0.01\\
417.01	0.01\\
418.01	0.01\\
419.01	0.01\\
420.01	0.01\\
421.01	0.01\\
422.01	0.01\\
423.01	0.01\\
424.01	0.01\\
425.01	0.01\\
426.01	0.01\\
427.01	0.01\\
428.01	0.01\\
429.01	0.01\\
430.01	0.01\\
431.01	0.01\\
432.01	0.01\\
433.01	0.01\\
434.01	0.01\\
435.01	0.01\\
436.01	0.01\\
437.01	0.01\\
438.01	0.01\\
439.01	0.01\\
440.01	0.01\\
441.01	0.01\\
442.01	0.01\\
443.01	0.01\\
444.01	0.01\\
445.01	0.01\\
446.01	0.01\\
447.01	0.01\\
448.01	0.01\\
449.01	0.01\\
450.01	0.01\\
451.01	0.01\\
452.01	0.01\\
453.01	0.01\\
454.01	0.01\\
455.01	0.01\\
456.01	0.01\\
457.01	0.01\\
458.01	0.01\\
459.01	0.01\\
460.01	0.01\\
461.01	0.01\\
462.01	0.01\\
463.01	0.01\\
464.01	0.01\\
465.01	0.01\\
466.01	0.01\\
467.01	0.01\\
468.01	0.01\\
469.01	0.01\\
470.01	0.01\\
471.01	0.01\\
472.01	0.01\\
473.01	0.01\\
474.01	0.01\\
475.01	0.01\\
476.01	0.01\\
477.01	0.01\\
478.01	0.01\\
479.01	0.01\\
480.01	0.01\\
481.01	0.01\\
482.01	0.01\\
483.01	0.01\\
484.01	0.01\\
485.01	0.01\\
486.01	0.01\\
487.01	0.01\\
488.01	0.01\\
489.01	0.01\\
490.01	0.01\\
491.01	0.01\\
492.01	0.01\\
493.01	0.01\\
494.01	0.01\\
495.01	0.01\\
496.01	0.01\\
497.01	0.01\\
498.01	0.01\\
499.01	0.01\\
500.01	0.01\\
501.01	0.01\\
502.01	0.01\\
503.01	0.01\\
504.01	0.01\\
505.01	0.01\\
506.01	0.01\\
507.01	0.01\\
508.01	0.01\\
509.01	0.01\\
510.01	0.01\\
511.01	0.01\\
512.01	0.01\\
513.01	0.01\\
514.01	0.01\\
515.01	0.01\\
516.01	0.01\\
517.01	0.01\\
518.01	0.01\\
519.01	0.01\\
520.01	0.01\\
521.01	0.01\\
522.01	0.01\\
523.01	0.01\\
524.01	0.01\\
525.01	0.01\\
526.01	0.01\\
527.01	0.01\\
528.01	0.01\\
529.01	0.01\\
530.01	0.01\\
531.01	0.01\\
532.01	0.01\\
533.01	0.01\\
534.01	0.01\\
535.01	0.01\\
536.01	0.01\\
537.01	0.01\\
538.01	0.01\\
539.01	0.01\\
540.01	0.01\\
541.01	0.01\\
542.01	0.01\\
543.01	0.01\\
544.01	0.01\\
545.01	0.01\\
546.01	0.01\\
547.01	0.01\\
548.01	0.01\\
549.01	0.01\\
550.01	0.01\\
551.01	0.01\\
552.01	0.01\\
553.01	0.01\\
554.01	0.01\\
555.01	0.01\\
556.01	0.01\\
557.01	0.01\\
558.01	0.01\\
559.01	0.01\\
560.01	0.01\\
561.01	0.01\\
562.01	0.01\\
563.01	0.01\\
564.01	0.01\\
565.01	0.01\\
566.01	0.01\\
567.01	0.01\\
568.01	0.01\\
569.01	0.01\\
570.01	0.01\\
571.01	0.01\\
572.01	0.01\\
573.01	0.01\\
574.01	0.01\\
575.01	0.01\\
576.01	0.01\\
577.01	0.01\\
578.01	0.01\\
579.01	0.01\\
580.01	0.01\\
581.01	0.01\\
582.01	0.01\\
583.01	0.01\\
584.01	0.01\\
585.01	0.01\\
586.01	0.01\\
587.01	0.01\\
588.01	0.01\\
589.01	0.01\\
590.01	0.01\\
591.01	0.01\\
592.01	0.01\\
593.01	0.01\\
594.01	0.01\\
595.01	0.01\\
596.01	0.01\\
597.01	0.01\\
598.01	0.01\\
599.01	0.00623513569400518\\
599.02	0.00619747359804889\\
599.03	0.0061594447559382\\
599.04	0.00612104556201832\\
599.05	0.0060822723751792\\
599.06	0.00604312151850676\\
599.07	0.00600358927893093\\
599.08	0.00596367190687011\\
599.09	0.00592336561587207\\
599.1	0.0058826665822515\\
599.11	0.00584157094472381\\
599.12	0.00580007480403546\\
599.13	0.00575817422259058\\
599.14	0.00571586522407396\\
599.15	0.00567314379307033\\
599.16	0.00563000587467983\\
599.17	0.00558644737412983\\
599.18	0.00554246415638277\\
599.19	0.00549805204574035\\
599.2	0.00545320682544357\\
599.21	0.00540792423726908\\
599.22	0.00536219998112139\\
599.23	0.00531602971462119\\
599.24	0.00526940905268952\\
599.25	0.00522233356712793\\
599.26	0.0051747987861945\\
599.27	0.00512680019417563\\
599.28	0.00507833323095366\\
599.29	0.00502939329157028\\
599.3	0.0049799757257856\\
599.31	0.00493007583763285\\
599.32	0.00487968888496879\\
599.33	0.00482881007901965\\
599.34	0.00477743458392255\\
599.35	0.00472555751626256\\
599.36	0.00467317394460504\\
599.37	0.00462027888902352\\
599.38	0.0045668673206228\\
599.39	0.00451293416105749\\
599.4	0.00445847428204579\\
599.41	0.00440348251095323\\
599.42	0.00434795363554174\\
599.43	0.00429188239249109\\
599.44	0.00423526346689832\\
599.45	0.00417809149177216\\
599.46	0.00412036104752262\\
599.47	0.00406206666144545\\
599.48	0.00400320280720162\\
599.49	0.00394376390429167\\
599.5	0.00388374431752495\\
599.51	0.00382313835648361\\
599.52	0.00376194027498138\\
599.53	0.00370014427051699\\
599.54	0.00363774448372231\\
599.55	0.00357473499780496\\
599.56	0.00351110983798565\\
599.57	0.00344686297092978\\
599.58	0.0033819883041737\\
599.59	0.00331647968554507\\
599.6	0.00325033090257786\\
599.61	0.00318353568192137\\
599.62	0.00311608768874353\\
599.63	0.00304798052612843\\
599.64	0.00297920773446783\\
599.65	0.00290976279084677\\
599.66	0.00283963910842317\\
599.67	0.00276883003580129\\
599.68	0.00269732885639912\\
599.69	0.00262512878780951\\
599.7	0.0025522229811551\\
599.71	0.00247860452043686\\
599.72	0.00240426642187629\\
599.73	0.0023292016332512\\
599.74	0.00225340303322488\\
599.75	0.00217686343066881\\
599.76	0.00209957556397868\\
599.77	0.00202153210038366\\
599.78	0.00194272563524905\\
599.79	0.00186314869137186\\
599.8	0.00178279371826976\\
599.81	0.00170165309146283\\
599.82	0.00161971911174841\\
599.83	0.00153698400446879\\
599.84	0.00145343991877172\\
599.85	0.00136907892686372\\
599.86	0.00128389302325598\\
599.87	0.001197874124003\\
599.88	0.00111101406593363\\
599.89	0.00102330460587469\\
599.9	0.000934737419866901\\
599.91	0.000845304102373188\\
599.92	0.000754996165479166\\
599.93	0.000663805038085868\\
599.94	0.000571722065094475\\
599.95	0.000478738506583129\\
599.96	0.000384845536975638\\
599.97	0.000290034244202044\\
599.98	0.00019429562885097\\
599.99	9.76206033136556e-05\\
600	0\\
};
\addplot [color=mycolor6,solid,forget plot]
  table[row sep=crcr]{%
0.01	0.01\\
1.01	0.01\\
2.01	0.01\\
3.01	0.01\\
4.01	0.01\\
5.01	0.01\\
6.01	0.01\\
7.01	0.01\\
8.01	0.01\\
9.01	0.01\\
10.01	0.01\\
11.01	0.01\\
12.01	0.01\\
13.01	0.01\\
14.01	0.01\\
15.01	0.01\\
16.01	0.01\\
17.01	0.01\\
18.01	0.01\\
19.01	0.01\\
20.01	0.01\\
21.01	0.01\\
22.01	0.01\\
23.01	0.01\\
24.01	0.01\\
25.01	0.01\\
26.01	0.01\\
27.01	0.01\\
28.01	0.01\\
29.01	0.01\\
30.01	0.01\\
31.01	0.01\\
32.01	0.01\\
33.01	0.01\\
34.01	0.01\\
35.01	0.01\\
36.01	0.01\\
37.01	0.01\\
38.01	0.01\\
39.01	0.01\\
40.01	0.01\\
41.01	0.01\\
42.01	0.01\\
43.01	0.01\\
44.01	0.01\\
45.01	0.01\\
46.01	0.01\\
47.01	0.01\\
48.01	0.01\\
49.01	0.01\\
50.01	0.01\\
51.01	0.01\\
52.01	0.01\\
53.01	0.01\\
54.01	0.01\\
55.01	0.01\\
56.01	0.01\\
57.01	0.01\\
58.01	0.01\\
59.01	0.01\\
60.01	0.01\\
61.01	0.01\\
62.01	0.01\\
63.01	0.01\\
64.01	0.01\\
65.01	0.01\\
66.01	0.01\\
67.01	0.01\\
68.01	0.01\\
69.01	0.01\\
70.01	0.01\\
71.01	0.01\\
72.01	0.01\\
73.01	0.01\\
74.01	0.01\\
75.01	0.01\\
76.01	0.01\\
77.01	0.01\\
78.01	0.01\\
79.01	0.01\\
80.01	0.01\\
81.01	0.01\\
82.01	0.01\\
83.01	0.01\\
84.01	0.01\\
85.01	0.01\\
86.01	0.01\\
87.01	0.01\\
88.01	0.01\\
89.01	0.01\\
90.01	0.01\\
91.01	0.01\\
92.01	0.01\\
93.01	0.01\\
94.01	0.01\\
95.01	0.01\\
96.01	0.01\\
97.01	0.01\\
98.01	0.01\\
99.01	0.01\\
100.01	0.01\\
101.01	0.01\\
102.01	0.01\\
103.01	0.01\\
104.01	0.01\\
105.01	0.01\\
106.01	0.01\\
107.01	0.01\\
108.01	0.01\\
109.01	0.01\\
110.01	0.01\\
111.01	0.01\\
112.01	0.01\\
113.01	0.01\\
114.01	0.01\\
115.01	0.01\\
116.01	0.01\\
117.01	0.01\\
118.01	0.01\\
119.01	0.01\\
120.01	0.01\\
121.01	0.01\\
122.01	0.01\\
123.01	0.01\\
124.01	0.01\\
125.01	0.01\\
126.01	0.01\\
127.01	0.01\\
128.01	0.01\\
129.01	0.01\\
130.01	0.01\\
131.01	0.01\\
132.01	0.01\\
133.01	0.01\\
134.01	0.01\\
135.01	0.01\\
136.01	0.01\\
137.01	0.01\\
138.01	0.01\\
139.01	0.01\\
140.01	0.01\\
141.01	0.01\\
142.01	0.01\\
143.01	0.01\\
144.01	0.01\\
145.01	0.01\\
146.01	0.01\\
147.01	0.01\\
148.01	0.01\\
149.01	0.01\\
150.01	0.01\\
151.01	0.01\\
152.01	0.01\\
153.01	0.01\\
154.01	0.01\\
155.01	0.01\\
156.01	0.01\\
157.01	0.01\\
158.01	0.01\\
159.01	0.01\\
160.01	0.01\\
161.01	0.01\\
162.01	0.01\\
163.01	0.01\\
164.01	0.01\\
165.01	0.01\\
166.01	0.01\\
167.01	0.01\\
168.01	0.01\\
169.01	0.01\\
170.01	0.01\\
171.01	0.01\\
172.01	0.01\\
173.01	0.01\\
174.01	0.01\\
175.01	0.01\\
176.01	0.01\\
177.01	0.01\\
178.01	0.01\\
179.01	0.01\\
180.01	0.01\\
181.01	0.01\\
182.01	0.01\\
183.01	0.01\\
184.01	0.01\\
185.01	0.01\\
186.01	0.01\\
187.01	0.01\\
188.01	0.01\\
189.01	0.01\\
190.01	0.01\\
191.01	0.01\\
192.01	0.01\\
193.01	0.01\\
194.01	0.01\\
195.01	0.01\\
196.01	0.01\\
197.01	0.01\\
198.01	0.01\\
199.01	0.01\\
200.01	0.01\\
201.01	0.01\\
202.01	0.01\\
203.01	0.01\\
204.01	0.01\\
205.01	0.01\\
206.01	0.01\\
207.01	0.01\\
208.01	0.01\\
209.01	0.01\\
210.01	0.01\\
211.01	0.01\\
212.01	0.01\\
213.01	0.01\\
214.01	0.01\\
215.01	0.01\\
216.01	0.01\\
217.01	0.01\\
218.01	0.01\\
219.01	0.01\\
220.01	0.01\\
221.01	0.01\\
222.01	0.01\\
223.01	0.01\\
224.01	0.01\\
225.01	0.01\\
226.01	0.01\\
227.01	0.01\\
228.01	0.01\\
229.01	0.01\\
230.01	0.01\\
231.01	0.01\\
232.01	0.01\\
233.01	0.01\\
234.01	0.01\\
235.01	0.01\\
236.01	0.01\\
237.01	0.01\\
238.01	0.01\\
239.01	0.01\\
240.01	0.01\\
241.01	0.01\\
242.01	0.01\\
243.01	0.01\\
244.01	0.01\\
245.01	0.01\\
246.01	0.01\\
247.01	0.01\\
248.01	0.01\\
249.01	0.01\\
250.01	0.01\\
251.01	0.01\\
252.01	0.01\\
253.01	0.01\\
254.01	0.01\\
255.01	0.01\\
256.01	0.01\\
257.01	0.01\\
258.01	0.01\\
259.01	0.01\\
260.01	0.01\\
261.01	0.01\\
262.01	0.01\\
263.01	0.01\\
264.01	0.01\\
265.01	0.01\\
266.01	0.01\\
267.01	0.01\\
268.01	0.01\\
269.01	0.01\\
270.01	0.01\\
271.01	0.01\\
272.01	0.01\\
273.01	0.01\\
274.01	0.01\\
275.01	0.01\\
276.01	0.01\\
277.01	0.01\\
278.01	0.01\\
279.01	0.01\\
280.01	0.01\\
281.01	0.01\\
282.01	0.01\\
283.01	0.01\\
284.01	0.01\\
285.01	0.01\\
286.01	0.01\\
287.01	0.01\\
288.01	0.01\\
289.01	0.01\\
290.01	0.01\\
291.01	0.01\\
292.01	0.01\\
293.01	0.01\\
294.01	0.01\\
295.01	0.01\\
296.01	0.01\\
297.01	0.01\\
298.01	0.01\\
299.01	0.01\\
300.01	0.01\\
301.01	0.01\\
302.01	0.01\\
303.01	0.01\\
304.01	0.01\\
305.01	0.01\\
306.01	0.01\\
307.01	0.01\\
308.01	0.01\\
309.01	0.01\\
310.01	0.01\\
311.01	0.01\\
312.01	0.01\\
313.01	0.01\\
314.01	0.01\\
315.01	0.01\\
316.01	0.01\\
317.01	0.01\\
318.01	0.01\\
319.01	0.01\\
320.01	0.01\\
321.01	0.01\\
322.01	0.01\\
323.01	0.01\\
324.01	0.01\\
325.01	0.01\\
326.01	0.01\\
327.01	0.01\\
328.01	0.01\\
329.01	0.01\\
330.01	0.01\\
331.01	0.01\\
332.01	0.01\\
333.01	0.01\\
334.01	0.01\\
335.01	0.01\\
336.01	0.01\\
337.01	0.01\\
338.01	0.01\\
339.01	0.01\\
340.01	0.01\\
341.01	0.01\\
342.01	0.01\\
343.01	0.01\\
344.01	0.01\\
345.01	0.01\\
346.01	0.01\\
347.01	0.01\\
348.01	0.01\\
349.01	0.01\\
350.01	0.01\\
351.01	0.01\\
352.01	0.01\\
353.01	0.01\\
354.01	0.01\\
355.01	0.01\\
356.01	0.01\\
357.01	0.01\\
358.01	0.01\\
359.01	0.01\\
360.01	0.01\\
361.01	0.01\\
362.01	0.01\\
363.01	0.01\\
364.01	0.01\\
365.01	0.01\\
366.01	0.01\\
367.01	0.01\\
368.01	0.01\\
369.01	0.01\\
370.01	0.01\\
371.01	0.01\\
372.01	0.01\\
373.01	0.01\\
374.01	0.01\\
375.01	0.01\\
376.01	0.01\\
377.01	0.01\\
378.01	0.01\\
379.01	0.01\\
380.01	0.01\\
381.01	0.01\\
382.01	0.01\\
383.01	0.01\\
384.01	0.01\\
385.01	0.01\\
386.01	0.01\\
387.01	0.01\\
388.01	0.01\\
389.01	0.01\\
390.01	0.01\\
391.01	0.01\\
392.01	0.01\\
393.01	0.01\\
394.01	0.01\\
395.01	0.01\\
396.01	0.01\\
397.01	0.01\\
398.01	0.01\\
399.01	0.01\\
400.01	0.01\\
401.01	0.01\\
402.01	0.01\\
403.01	0.01\\
404.01	0.01\\
405.01	0.01\\
406.01	0.01\\
407.01	0.01\\
408.01	0.01\\
409.01	0.01\\
410.01	0.01\\
411.01	0.01\\
412.01	0.01\\
413.01	0.01\\
414.01	0.01\\
415.01	0.01\\
416.01	0.01\\
417.01	0.01\\
418.01	0.01\\
419.01	0.01\\
420.01	0.01\\
421.01	0.01\\
422.01	0.01\\
423.01	0.01\\
424.01	0.01\\
425.01	0.01\\
426.01	0.01\\
427.01	0.01\\
428.01	0.01\\
429.01	0.01\\
430.01	0.01\\
431.01	0.01\\
432.01	0.01\\
433.01	0.01\\
434.01	0.01\\
435.01	0.01\\
436.01	0.01\\
437.01	0.01\\
438.01	0.01\\
439.01	0.01\\
440.01	0.01\\
441.01	0.01\\
442.01	0.01\\
443.01	0.01\\
444.01	0.01\\
445.01	0.01\\
446.01	0.01\\
447.01	0.01\\
448.01	0.01\\
449.01	0.01\\
450.01	0.01\\
451.01	0.01\\
452.01	0.01\\
453.01	0.01\\
454.01	0.01\\
455.01	0.01\\
456.01	0.01\\
457.01	0.01\\
458.01	0.01\\
459.01	0.01\\
460.01	0.01\\
461.01	0.01\\
462.01	0.01\\
463.01	0.01\\
464.01	0.01\\
465.01	0.01\\
466.01	0.01\\
467.01	0.01\\
468.01	0.01\\
469.01	0.01\\
470.01	0.01\\
471.01	0.01\\
472.01	0.01\\
473.01	0.01\\
474.01	0.01\\
475.01	0.01\\
476.01	0.01\\
477.01	0.01\\
478.01	0.01\\
479.01	0.01\\
480.01	0.01\\
481.01	0.01\\
482.01	0.01\\
483.01	0.01\\
484.01	0.01\\
485.01	0.01\\
486.01	0.01\\
487.01	0.01\\
488.01	0.01\\
489.01	0.01\\
490.01	0.01\\
491.01	0.01\\
492.01	0.01\\
493.01	0.01\\
494.01	0.01\\
495.01	0.01\\
496.01	0.01\\
497.01	0.01\\
498.01	0.01\\
499.01	0.01\\
500.01	0.01\\
501.01	0.01\\
502.01	0.01\\
503.01	0.01\\
504.01	0.01\\
505.01	0.01\\
506.01	0.01\\
507.01	0.01\\
508.01	0.01\\
509.01	0.01\\
510.01	0.01\\
511.01	0.01\\
512.01	0.01\\
513.01	0.01\\
514.01	0.01\\
515.01	0.01\\
516.01	0.01\\
517.01	0.01\\
518.01	0.01\\
519.01	0.01\\
520.01	0.01\\
521.01	0.01\\
522.01	0.01\\
523.01	0.01\\
524.01	0.01\\
525.01	0.01\\
526.01	0.01\\
527.01	0.01\\
528.01	0.01\\
529.01	0.01\\
530.01	0.01\\
531.01	0.01\\
532.01	0.01\\
533.01	0.01\\
534.01	0.01\\
535.01	0.01\\
536.01	0.01\\
537.01	0.01\\
538.01	0.01\\
539.01	0.01\\
540.01	0.01\\
541.01	0.01\\
542.01	0.01\\
543.01	0.01\\
544.01	0.01\\
545.01	0.01\\
546.01	0.01\\
547.01	0.01\\
548.01	0.01\\
549.01	0.01\\
550.01	0.01\\
551.01	0.01\\
552.01	0.01\\
553.01	0.01\\
554.01	0.01\\
555.01	0.01\\
556.01	0.01\\
557.01	0.01\\
558.01	0.01\\
559.01	0.01\\
560.01	0.01\\
561.01	0.01\\
562.01	0.01\\
563.01	0.01\\
564.01	0.01\\
565.01	0.01\\
566.01	0.01\\
567.01	0.01\\
568.01	0.01\\
569.01	0.01\\
570.01	0.01\\
571.01	0.01\\
572.01	0.01\\
573.01	0.01\\
574.01	0.01\\
575.01	0.01\\
576.01	0.01\\
577.01	0.01\\
578.01	0.01\\
579.01	0.01\\
580.01	0.01\\
581.01	0.01\\
582.01	0.01\\
583.01	0.01\\
584.01	0.01\\
585.01	0.01\\
586.01	0.01\\
587.01	0.01\\
588.01	0.01\\
589.01	0.01\\
590.01	0.01\\
591.01	0.01\\
592.01	0.01\\
593.01	0.01\\
594.01	0.01\\
595.01	0.01\\
596.01	0.01\\
597.01	0.01\\
598.01	0.01\\
599.01	0.00623513569400517\\
599.02	0.00619747359804888\\
599.03	0.00615944475593818\\
599.04	0.00612104556201831\\
599.05	0.00608227237517917\\
599.06	0.00604312151850673\\
599.07	0.00600358927893092\\
599.08	0.00596367190687009\\
599.09	0.00592336561587206\\
599.1	0.00588266658225149\\
599.11	0.0058415709447238\\
599.12	0.00580007480403545\\
599.13	0.00575817422259057\\
599.14	0.00571586522407396\\
599.15	0.00567314379307032\\
599.16	0.00563000587467982\\
599.17	0.0055864473741298\\
599.18	0.00554246415638276\\
599.19	0.00549805204574034\\
599.2	0.00545320682544357\\
599.21	0.00540792423726907\\
599.22	0.00536219998112139\\
599.23	0.00531602971462118\\
599.24	0.0052694090526895\\
599.25	0.00522233356712793\\
599.26	0.00517479878619452\\
599.27	0.00512680019417564\\
599.28	0.00507833323095367\\
599.29	0.00502939329157031\\
599.3	0.00497997572578562\\
599.31	0.00493007583763287\\
599.32	0.00487968888496882\\
599.33	0.00482881007901966\\
599.34	0.00477743458392257\\
599.35	0.00472555751626259\\
599.36	0.00467317394460509\\
599.37	0.00462027888902355\\
599.38	0.00456686732062282\\
599.39	0.00451293416105752\\
599.4	0.00445847428204581\\
599.41	0.00440348251095325\\
599.42	0.00434795363554175\\
599.43	0.00429188239249109\\
599.44	0.00423526346689832\\
599.45	0.00417809149177218\\
599.46	0.00412036104752264\\
599.47	0.00406206666144547\\
599.48	0.00400320280720163\\
599.49	0.00394376390429167\\
599.5	0.00388374431752495\\
599.51	0.0038231383564836\\
599.52	0.00376194027498136\\
599.53	0.00370014427051698\\
599.54	0.0036377444837223\\
599.55	0.00357473499780495\\
599.56	0.00351110983798565\\
599.57	0.00344686297092979\\
599.58	0.0033819883041737\\
599.59	0.00331647968554508\\
599.6	0.00325033090257787\\
599.61	0.00318353568192137\\
599.62	0.00311608768874353\\
599.63	0.00304798052612843\\
599.64	0.00297920773446783\\
599.65	0.00290976279084677\\
599.66	0.00283963910842317\\
599.67	0.00276883003580129\\
599.68	0.00269732885639913\\
599.69	0.00262512878780953\\
599.7	0.00255222298115511\\
599.71	0.00247860452043686\\
599.72	0.0024042664218763\\
599.73	0.00232920163325121\\
599.74	0.00225340303322488\\
599.75	0.00217686343066882\\
599.76	0.00209957556397868\\
599.77	0.00202153210038366\\
599.78	0.00194272563524904\\
599.79	0.00186314869137186\\
599.8	0.00178279371826976\\
599.81	0.00170165309146283\\
599.82	0.00161971911174841\\
599.83	0.00153698400446879\\
599.84	0.00145343991877172\\
599.85	0.00136907892686371\\
599.86	0.00128389302325598\\
599.87	0.00119787412400299\\
599.88	0.00111101406593363\\
599.89	0.00102330460587468\\
599.9	0.000934737419866896\\
599.91	0.000845304102373183\\
599.92	0.000754996165479164\\
599.93	0.000663805038085864\\
599.94	0.000571722065094472\\
599.95	0.000478738506583127\\
599.96	0.000384845536975636\\
599.97	0.000290034244202042\\
599.98	0.000194295628850972\\
599.99	9.76206033136556e-05\\
600	0\\
};
\addplot [color=mycolor7,solid,forget plot]
  table[row sep=crcr]{%
0.01	0.00999999999999999\\
1.01	0.00999999999999999\\
2.01	0.00999999999999999\\
3.01	0.00999999999999999\\
4.01	0.00999999999999999\\
5.01	0.00999999999999999\\
6.01	0.00999999999999999\\
7.01	0.00999999999999999\\
8.01	0.00999999999999999\\
9.01	0.00999999999999999\\
10.01	0.00999999999999999\\
11.01	0.00999999999999999\\
12.01	0.00999999999999999\\
13.01	0.00999999999999999\\
14.01	0.00999999999999999\\
15.01	0.00999999999999999\\
16.01	0.00999999999999999\\
17.01	0.00999999999999999\\
18.01	0.00999999999999999\\
19.01	0.00999999999999999\\
20.01	0.00999999999999999\\
21.01	0.00999999999999999\\
22.01	0.00999999999999999\\
23.01	0.00999999999999999\\
24.01	0.00999999999999999\\
25.01	0.00999999999999999\\
26.01	0.00999999999999999\\
27.01	0.00999999999999999\\
28.01	0.00999999999999999\\
29.01	0.00999999999999999\\
30.01	0.00999999999999999\\
31.01	0.00999999999999999\\
32.01	0.00999999999999999\\
33.01	0.00999999999999999\\
34.01	0.00999999999999999\\
35.01	0.00999999999999999\\
36.01	0.00999999999999999\\
37.01	0.00999999999999999\\
38.01	0.00999999999999999\\
39.01	0.00999999999999999\\
40.01	0.00999999999999999\\
41.01	0.00999999999999999\\
42.01	0.00999999999999999\\
43.01	0.00999999999999999\\
44.01	0.00999999999999999\\
45.01	0.00999999999999999\\
46.01	0.00999999999999999\\
47.01	0.00999999999999999\\
48.01	0.00999999999999999\\
49.01	0.00999999999999999\\
50.01	0.00999999999999999\\
51.01	0.00999999999999999\\
52.01	0.00999999999999999\\
53.01	0.00999999999999999\\
54.01	0.00999999999999999\\
55.01	0.00999999999999999\\
56.01	0.00999999999999999\\
57.01	0.00999999999999999\\
58.01	0.00999999999999999\\
59.01	0.00999999999999999\\
60.01	0.00999999999999999\\
61.01	0.00999999999999999\\
62.01	0.00999999999999999\\
63.01	0.00999999999999999\\
64.01	0.00999999999999999\\
65.01	0.00999999999999999\\
66.01	0.00999999999999999\\
67.01	0.00999999999999999\\
68.01	0.00999999999999999\\
69.01	0.00999999999999999\\
70.01	0.00999999999999999\\
71.01	0.00999999999999999\\
72.01	0.00999999999999999\\
73.01	0.00999999999999999\\
74.01	0.00999999999999999\\
75.01	0.00999999999999999\\
76.01	0.00999999999999999\\
77.01	0.00999999999999999\\
78.01	0.00999999999999999\\
79.01	0.00999999999999999\\
80.01	0.00999999999999999\\
81.01	0.00999999999999999\\
82.01	0.00999999999999999\\
83.01	0.00999999999999999\\
84.01	0.00999999999999999\\
85.01	0.00999999999999999\\
86.01	0.00999999999999999\\
87.01	0.00999999999999999\\
88.01	0.00999999999999999\\
89.01	0.00999999999999999\\
90.01	0.00999999999999999\\
91.01	0.00999999999999999\\
92.01	0.00999999999999999\\
93.01	0.00999999999999999\\
94.01	0.00999999999999999\\
95.01	0.00999999999999999\\
96.01	0.00999999999999999\\
97.01	0.00999999999999999\\
98.01	0.00999999999999999\\
99.01	0.00999999999999999\\
100.01	0.00999999999999999\\
101.01	0.00999999999999999\\
102.01	0.00999999999999999\\
103.01	0.00999999999999999\\
104.01	0.00999999999999999\\
105.01	0.00999999999999999\\
106.01	0.00999999999999999\\
107.01	0.00999999999999999\\
108.01	0.00999999999999999\\
109.01	0.00999999999999999\\
110.01	0.00999999999999999\\
111.01	0.00999999999999999\\
112.01	0.00999999999999999\\
113.01	0.00999999999999999\\
114.01	0.00999999999999999\\
115.01	0.00999999999999999\\
116.01	0.00999999999999999\\
117.01	0.00999999999999999\\
118.01	0.00999999999999999\\
119.01	0.00999999999999999\\
120.01	0.00999999999999999\\
121.01	0.00999999999999999\\
122.01	0.00999999999999999\\
123.01	0.00999999999999999\\
124.01	0.00999999999999999\\
125.01	0.00999999999999999\\
126.01	0.00999999999999999\\
127.01	0.00999999999999999\\
128.01	0.00999999999999999\\
129.01	0.00999999999999999\\
130.01	0.00999999999999999\\
131.01	0.00999999999999999\\
132.01	0.00999999999999999\\
133.01	0.00999999999999999\\
134.01	0.00999999999999999\\
135.01	0.00999999999999999\\
136.01	0.00999999999999999\\
137.01	0.00999999999999999\\
138.01	0.00999999999999999\\
139.01	0.00999999999999999\\
140.01	0.00999999999999999\\
141.01	0.00999999999999999\\
142.01	0.00999999999999999\\
143.01	0.00999999999999999\\
144.01	0.00999999999999999\\
145.01	0.00999999999999999\\
146.01	0.00999999999999999\\
147.01	0.00999999999999999\\
148.01	0.00999999999999999\\
149.01	0.00999999999999999\\
150.01	0.00999999999999999\\
151.01	0.00999999999999999\\
152.01	0.00999999999999999\\
153.01	0.00999999999999999\\
154.01	0.00999999999999999\\
155.01	0.00999999999999999\\
156.01	0.00999999999999999\\
157.01	0.00999999999999999\\
158.01	0.00999999999999999\\
159.01	0.00999999999999999\\
160.01	0.00999999999999999\\
161.01	0.00999999999999999\\
162.01	0.00999999999999999\\
163.01	0.00999999999999999\\
164.01	0.00999999999999999\\
165.01	0.00999999999999999\\
166.01	0.00999999999999999\\
167.01	0.00999999999999999\\
168.01	0.00999999999999999\\
169.01	0.00999999999999999\\
170.01	0.00999999999999999\\
171.01	0.00999999999999999\\
172.01	0.00999999999999999\\
173.01	0.00999999999999999\\
174.01	0.00999999999999999\\
175.01	0.00999999999999999\\
176.01	0.00999999999999999\\
177.01	0.00999999999999999\\
178.01	0.00999999999999999\\
179.01	0.00999999999999999\\
180.01	0.00999999999999999\\
181.01	0.00999999999999999\\
182.01	0.00999999999999999\\
183.01	0.00999999999999999\\
184.01	0.00999999999999999\\
185.01	0.00999999999999999\\
186.01	0.00999999999999999\\
187.01	0.00999999999999999\\
188.01	0.00999999999999999\\
189.01	0.00999999999999999\\
190.01	0.00999999999999999\\
191.01	0.00999999999999999\\
192.01	0.00999999999999999\\
193.01	0.00999999999999999\\
194.01	0.00999999999999999\\
195.01	0.00999999999999999\\
196.01	0.00999999999999999\\
197.01	0.00999999999999999\\
198.01	0.00999999999999999\\
199.01	0.00999999999999999\\
200.01	0.00999999999999999\\
201.01	0.00999999999999999\\
202.01	0.00999999999999999\\
203.01	0.00999999999999999\\
204.01	0.00999999999999999\\
205.01	0.00999999999999999\\
206.01	0.00999999999999999\\
207.01	0.00999999999999999\\
208.01	0.00999999999999999\\
209.01	0.00999999999999999\\
210.01	0.00999999999999999\\
211.01	0.00999999999999999\\
212.01	0.00999999999999999\\
213.01	0.00999999999999999\\
214.01	0.00999999999999999\\
215.01	0.00999999999999999\\
216.01	0.00999999999999999\\
217.01	0.00999999999999999\\
218.01	0.00999999999999999\\
219.01	0.00999999999999999\\
220.01	0.00999999999999999\\
221.01	0.00999999999999999\\
222.01	0.00999999999999999\\
223.01	0.00999999999999999\\
224.01	0.00999999999999999\\
225.01	0.00999999999999999\\
226.01	0.00999999999999999\\
227.01	0.00999999999999999\\
228.01	0.00999999999999999\\
229.01	0.00999999999999999\\
230.01	0.00999999999999999\\
231.01	0.00999999999999999\\
232.01	0.00999999999999999\\
233.01	0.00999999999999999\\
234.01	0.00999999999999999\\
235.01	0.00999999999999999\\
236.01	0.00999999999999999\\
237.01	0.00999999999999999\\
238.01	0.00999999999999999\\
239.01	0.00999999999999999\\
240.01	0.00999999999999999\\
241.01	0.00999999999999999\\
242.01	0.00999999999999999\\
243.01	0.00999999999999999\\
244.01	0.00999999999999999\\
245.01	0.00999999999999999\\
246.01	0.00999999999999999\\
247.01	0.00999999999999999\\
248.01	0.00999999999999999\\
249.01	0.00999999999999999\\
250.01	0.00999999999999999\\
251.01	0.00999999999999999\\
252.01	0.00999999999999999\\
253.01	0.00999999999999999\\
254.01	0.00999999999999999\\
255.01	0.00999999999999999\\
256.01	0.00999999999999999\\
257.01	0.00999999999999999\\
258.01	0.00999999999999999\\
259.01	0.00999999999999999\\
260.01	0.00999999999999999\\
261.01	0.00999999999999999\\
262.01	0.00999999999999999\\
263.01	0.00999999999999999\\
264.01	0.00999999999999999\\
265.01	0.00999999999999999\\
266.01	0.00999999999999999\\
267.01	0.00999999999999999\\
268.01	0.00999999999999999\\
269.01	0.00999999999999999\\
270.01	0.00999999999999999\\
271.01	0.00999999999999999\\
272.01	0.00999999999999999\\
273.01	0.00999999999999999\\
274.01	0.00999999999999999\\
275.01	0.00999999999999999\\
276.01	0.00999999999999999\\
277.01	0.00999999999999999\\
278.01	0.00999999999999999\\
279.01	0.00999999999999999\\
280.01	0.00999999999999999\\
281.01	0.00999999999999999\\
282.01	0.00999999999999999\\
283.01	0.00999999999999999\\
284.01	0.00999999999999999\\
285.01	0.00999999999999999\\
286.01	0.00999999999999999\\
287.01	0.00999999999999999\\
288.01	0.00999999999999999\\
289.01	0.00999999999999999\\
290.01	0.00999999999999999\\
291.01	0.00999999999999999\\
292.01	0.00999999999999999\\
293.01	0.00999999999999999\\
294.01	0.00999999999999999\\
295.01	0.00999999999999999\\
296.01	0.00999999999999999\\
297.01	0.00999999999999999\\
298.01	0.00999999999999999\\
299.01	0.00999999999999999\\
300.01	0.00999999999999999\\
301.01	0.00999999999999999\\
302.01	0.00999999999999999\\
303.01	0.00999999999999999\\
304.01	0.00999999999999999\\
305.01	0.00999999999999999\\
306.01	0.00999999999999999\\
307.01	0.00999999999999999\\
308.01	0.00999999999999999\\
309.01	0.00999999999999999\\
310.01	0.00999999999999999\\
311.01	0.00999999999999999\\
312.01	0.00999999999999999\\
313.01	0.00999999999999999\\
314.01	0.00999999999999999\\
315.01	0.00999999999999999\\
316.01	0.00999999999999999\\
317.01	0.00999999999999999\\
318.01	0.00999999999999999\\
319.01	0.00999999999999999\\
320.01	0.00999999999999999\\
321.01	0.00999999999999999\\
322.01	0.00999999999999999\\
323.01	0.00999999999999999\\
324.01	0.00999999999999999\\
325.01	0.00999999999999999\\
326.01	0.00999999999999999\\
327.01	0.00999999999999999\\
328.01	0.00999999999999999\\
329.01	0.00999999999999999\\
330.01	0.00999999999999999\\
331.01	0.00999999999999999\\
332.01	0.00999999999999999\\
333.01	0.00999999999999999\\
334.01	0.00999999999999999\\
335.01	0.00999999999999999\\
336.01	0.00999999999999999\\
337.01	0.00999999999999999\\
338.01	0.00999999999999999\\
339.01	0.00999999999999999\\
340.01	0.00999999999999999\\
341.01	0.00999999999999999\\
342.01	0.00999999999999999\\
343.01	0.00999999999999999\\
344.01	0.00999999999999999\\
345.01	0.00999999999999999\\
346.01	0.00999999999999999\\
347.01	0.00999999999999999\\
348.01	0.00999999999999999\\
349.01	0.00999999999999999\\
350.01	0.00999999999999999\\
351.01	0.00999999999999999\\
352.01	0.00999999999999999\\
353.01	0.00999999999999999\\
354.01	0.00999999999999999\\
355.01	0.00999999999999999\\
356.01	0.00999999999999999\\
357.01	0.00999999999999999\\
358.01	0.00999999999999999\\
359.01	0.00999999999999999\\
360.01	0.00999999999999999\\
361.01	0.00999999999999999\\
362.01	0.00999999999999999\\
363.01	0.00999999999999999\\
364.01	0.00999999999999999\\
365.01	0.00999999999999999\\
366.01	0.00999999999999999\\
367.01	0.00999999999999999\\
368.01	0.00999999999999999\\
369.01	0.00999999999999999\\
370.01	0.00999999999999999\\
371.01	0.00999999999999999\\
372.01	0.00999999999999999\\
373.01	0.00999999999999999\\
374.01	0.00999999999999999\\
375.01	0.00999999999999999\\
376.01	0.00999999999999999\\
377.01	0.00999999999999999\\
378.01	0.00999999999999999\\
379.01	0.00999999999999999\\
380.01	0.00999999999999999\\
381.01	0.00999999999999999\\
382.01	0.00999999999999999\\
383.01	0.00999999999999999\\
384.01	0.00999999999999999\\
385.01	0.00999999999999999\\
386.01	0.00999999999999999\\
387.01	0.00999999999999999\\
388.01	0.00999999999999999\\
389.01	0.00999999999999999\\
390.01	0.00999999999999999\\
391.01	0.00999999999999999\\
392.01	0.00999999999999999\\
393.01	0.00999999999999999\\
394.01	0.00999999999999999\\
395.01	0.00999999999999999\\
396.01	0.00999999999999999\\
397.01	0.00999999999999999\\
398.01	0.00999999999999999\\
399.01	0.00999999999999999\\
400.01	0.00999999999999999\\
401.01	0.00999999999999999\\
402.01	0.00999999999999999\\
403.01	0.00999999999999999\\
404.01	0.00999999999999999\\
405.01	0.00999999999999999\\
406.01	0.00999999999999999\\
407.01	0.00999999999999999\\
408.01	0.00999999999999999\\
409.01	0.00999999999999999\\
410.01	0.00999999999999999\\
411.01	0.00999999999999999\\
412.01	0.00999999999999999\\
413.01	0.00999999999999999\\
414.01	0.00999999999999999\\
415.01	0.00999999999999999\\
416.01	0.00999999999999999\\
417.01	0.00999999999999999\\
418.01	0.00999999999999999\\
419.01	0.00999999999999999\\
420.01	0.00999999999999999\\
421.01	0.00999999999999999\\
422.01	0.00999999999999999\\
423.01	0.00999999999999999\\
424.01	0.00999999999999999\\
425.01	0.00999999999999999\\
426.01	0.00999999999999999\\
427.01	0.00999999999999999\\
428.01	0.00999999999999999\\
429.01	0.00999999999999999\\
430.01	0.00999999999999999\\
431.01	0.00999999999999999\\
432.01	0.00999999999999999\\
433.01	0.00999999999999999\\
434.01	0.00999999999999999\\
435.01	0.00999999999999999\\
436.01	0.00999999999999999\\
437.01	0.00999999999999999\\
438.01	0.00999999999999999\\
439.01	0.00999999999999999\\
440.01	0.00999999999999999\\
441.01	0.00999999999999999\\
442.01	0.00999999999999999\\
443.01	0.00999999999999999\\
444.01	0.00999999999999999\\
445.01	0.00999999999999999\\
446.01	0.00999999999999999\\
447.01	0.00999999999999999\\
448.01	0.00999999999999999\\
449.01	0.00999999999999999\\
450.01	0.00999999999999999\\
451.01	0.00999999999999999\\
452.01	0.00999999999999999\\
453.01	0.00999999999999999\\
454.01	0.00999999999999999\\
455.01	0.00999999999999999\\
456.01	0.00999999999999999\\
457.01	0.00999999999999999\\
458.01	0.00999999999999999\\
459.01	0.00999999999999999\\
460.01	0.00999999999999999\\
461.01	0.00999999999999999\\
462.01	0.00999999999999999\\
463.01	0.00999999999999999\\
464.01	0.00999999999999999\\
465.01	0.00999999999999999\\
466.01	0.00999999999999999\\
467.01	0.00999999999999999\\
468.01	0.00999999999999999\\
469.01	0.00999999999999999\\
470.01	0.00999999999999999\\
471.01	0.00999999999999999\\
472.01	0.00999999999999999\\
473.01	0.00999999999999999\\
474.01	0.00999999999999999\\
475.01	0.00999999999999999\\
476.01	0.00999999999999999\\
477.01	0.00999999999999999\\
478.01	0.00999999999999999\\
479.01	0.00999999999999999\\
480.01	0.00999999999999999\\
481.01	0.00999999999999999\\
482.01	0.00999999999999999\\
483.01	0.00999999999999999\\
484.01	0.00999999999999999\\
485.01	0.00999999999999999\\
486.01	0.00999999999999999\\
487.01	0.00999999999999999\\
488.01	0.00999999999999999\\
489.01	0.00999999999999999\\
490.01	0.00999999999999999\\
491.01	0.00999999999999999\\
492.01	0.00999999999999999\\
493.01	0.00999999999999999\\
494.01	0.00999999999999999\\
495.01	0.00999999999999999\\
496.01	0.00999999999999999\\
497.01	0.00999999999999999\\
498.01	0.00999999999999999\\
499.01	0.00999999999999999\\
500.01	0.00999999999999999\\
501.01	0.00999999999999999\\
502.01	0.00999999999999999\\
503.01	0.00999999999999999\\
504.01	0.00999999999999999\\
505.01	0.00999999999999999\\
506.01	0.00999999999999999\\
507.01	0.00999999999999999\\
508.01	0.00999999999999999\\
509.01	0.00999999999999999\\
510.01	0.00999999999999999\\
511.01	0.00999999999999999\\
512.01	0.00999999999999999\\
513.01	0.00999999999999999\\
514.01	0.00999999999999999\\
515.01	0.00999999999999999\\
516.01	0.00999999999999999\\
517.01	0.00999999999999999\\
518.01	0.00999999999999999\\
519.01	0.00999999999999999\\
520.01	0.00999999999999999\\
521.01	0.00999999999999999\\
522.01	0.00999999999999999\\
523.01	0.00999999999999999\\
524.01	0.00999999999999999\\
525.01	0.00999999999999999\\
526.01	0.00999999999999999\\
527.01	0.00999999999999999\\
528.01	0.00999999999999999\\
529.01	0.00999999999999999\\
530.01	0.00999999999999999\\
531.01	0.00999999999999999\\
532.01	0.00999999999999999\\
533.01	0.00999999999999999\\
534.01	0.00999999999999999\\
535.01	0.00999999999999999\\
536.01	0.00999999999999999\\
537.01	0.00999999999999999\\
538.01	0.00999999999999999\\
539.01	0.00999999999999999\\
540.01	0.00999999999999999\\
541.01	0.00999999999999999\\
542.01	0.00999999999999999\\
543.01	0.00999999999999999\\
544.01	0.00999999999999999\\
545.01	0.00999999999999999\\
546.01	0.00999999999999999\\
547.01	0.00999999999999999\\
548.01	0.00999999999999999\\
549.01	0.00999999999999999\\
550.01	0.00999999999999999\\
551.01	0.00999999999999999\\
552.01	0.00999999999999999\\
553.01	0.00999999999999999\\
554.01	0.00999999999999999\\
555.01	0.00999999999999999\\
556.01	0.00999999999999999\\
557.01	0.00999999999999999\\
558.01	0.00999999999999999\\
559.01	0.00999999999999999\\
560.01	0.00999999999999999\\
561.01	0.00999999999999999\\
562.01	0.00999999999999999\\
563.01	0.00999999999999999\\
564.01	0.00999999999999999\\
565.01	0.00999999999999999\\
566.01	0.00999999999999999\\
567.01	0.00999999999999999\\
568.01	0.00999999999999999\\
569.01	0.00999999999999999\\
570.01	0.00999999999999999\\
571.01	0.00999999999999999\\
572.01	0.00999999999999999\\
573.01	0.00999999999999999\\
574.01	0.00999999999999999\\
575.01	0.00999999999999999\\
576.01	0.00999999999999999\\
577.01	0.00999999999999999\\
578.01	0.00999999999999999\\
579.01	0.00999999999999999\\
580.01	0.00999999999999999\\
581.01	0.00999999999999999\\
582.01	0.00999999999999999\\
583.01	0.00999999999999999\\
584.01	0.00999999999999999\\
585.01	0.00999999999999999\\
586.01	0.00999999999999999\\
587.01	0.00999999999999999\\
588.01	0.00999999999999999\\
589.01	0.00999999999999999\\
590.01	0.00999999999999999\\
591.01	0.00999999999999999\\
592.01	0.00999999999999999\\
593.01	0.00999999999999999\\
594.01	0.00999999999999999\\
595.01	0.00999999999999999\\
596.01	0.00999999999999999\\
597.01	0.00999999999999999\\
598.01	0.00999999999999999\\
599.01	0.00623513569400514\\
599.02	0.00619747359804885\\
599.03	0.00615944475593815\\
599.04	0.00612104556201828\\
599.05	0.00608227237517915\\
599.06	0.00604312151850672\\
599.07	0.00600358927893089\\
599.08	0.00596367190687008\\
599.09	0.00592336561587206\\
599.1	0.00588266658225149\\
599.11	0.0058415709447238\\
599.12	0.00580007480403545\\
599.13	0.00575817422259057\\
599.14	0.00571586522407395\\
599.15	0.00567314379307032\\
599.16	0.00563000587467982\\
599.17	0.00558644737412981\\
599.18	0.00554246415638277\\
599.19	0.00549805204574035\\
599.2	0.00545320682544357\\
599.21	0.00540792423726908\\
599.22	0.00536219998112139\\
599.23	0.00531602971462118\\
599.24	0.00526940905268952\\
599.25	0.00522233356712795\\
599.26	0.00517479878619452\\
599.27	0.00512680019417563\\
599.28	0.00507833323095367\\
599.29	0.0050293932915703\\
599.3	0.00497997572578562\\
599.31	0.00493007583763287\\
599.32	0.0048796888849688\\
599.33	0.00482881007901966\\
599.34	0.00477743458392256\\
599.35	0.00472555751626257\\
599.36	0.00467317394460506\\
599.37	0.00462027888902354\\
599.38	0.00456686732062281\\
599.39	0.00451293416105751\\
599.4	0.00445847428204581\\
599.41	0.00440348251095325\\
599.42	0.00434795363554175\\
599.43	0.0042918823924911\\
599.44	0.00423526346689832\\
599.45	0.00417809149177217\\
599.46	0.00412036104752262\\
599.47	0.00406206666144545\\
599.48	0.00400320280720161\\
599.49	0.00394376390429165\\
599.5	0.00388374431752494\\
599.51	0.00382313835648361\\
599.52	0.00376194027498138\\
599.53	0.00370014427051699\\
599.54	0.0036377444837223\\
599.55	0.00357473499780495\\
599.56	0.00351110983798564\\
599.57	0.00344686297092978\\
599.58	0.00338198830417369\\
599.59	0.00331647968554506\\
599.6	0.00325033090257786\\
599.61	0.00318353568192136\\
599.62	0.00311608768874353\\
599.63	0.00304798052612843\\
599.64	0.00297920773446783\\
599.65	0.00290976279084677\\
599.66	0.00283963910842317\\
599.67	0.00276883003580129\\
599.68	0.00269732885639912\\
599.69	0.00262512878780951\\
599.7	0.0025522229811551\\
599.71	0.00247860452043686\\
599.72	0.00240426642187629\\
599.73	0.0023292016332512\\
599.74	0.00225340303322488\\
599.75	0.00217686343066881\\
599.76	0.00209957556397867\\
599.77	0.00202153210038366\\
599.78	0.00194272563524904\\
599.79	0.00186314869137185\\
599.8	0.00178279371826975\\
599.81	0.00170165309146282\\
599.82	0.0016197191117484\\
599.83	0.00153698400446878\\
599.84	0.00145343991877171\\
599.85	0.00136907892686371\\
599.86	0.00128389302325598\\
599.87	0.001197874124003\\
599.88	0.00111101406593362\\
599.89	0.00102330460587468\\
599.9	0.0009347374198669\\
599.91	0.000845304102373186\\
599.92	0.000754996165479166\\
599.93	0.000663805038085868\\
599.94	0.000571722065094475\\
599.95	0.000478738506583127\\
599.96	0.000384845536975638\\
599.97	0.000290034244202044\\
599.98	0.00019429562885097\\
599.99	9.76206033136556e-05\\
600	0\\
};
\addplot [color=mycolor8,solid,forget plot]
  table[row sep=crcr]{%
0.01	0.00999999999999999\\
1.01	0.00999999999999999\\
2.01	0.00999999999999999\\
3.01	0.00999999999999999\\
4.01	0.00999999999999999\\
5.01	0.00999999999999999\\
6.01	0.00999999999999999\\
7.01	0.00999999999999999\\
8.01	0.00999999999999999\\
9.01	0.00999999999999999\\
10.01	0.00999999999999999\\
11.01	0.00999999999999999\\
12.01	0.00999999999999999\\
13.01	0.00999999999999999\\
14.01	0.00999999999999999\\
15.01	0.00999999999999999\\
16.01	0.00999999999999999\\
17.01	0.00999999999999999\\
18.01	0.00999999999999999\\
19.01	0.00999999999999999\\
20.01	0.00999999999999999\\
21.01	0.00999999999999999\\
22.01	0.00999999999999999\\
23.01	0.00999999999999999\\
24.01	0.00999999999999999\\
25.01	0.00999999999999999\\
26.01	0.00999999999999999\\
27.01	0.00999999999999999\\
28.01	0.00999999999999999\\
29.01	0.00999999999999999\\
30.01	0.00999999999999999\\
31.01	0.00999999999999999\\
32.01	0.00999999999999999\\
33.01	0.00999999999999999\\
34.01	0.00999999999999999\\
35.01	0.00999999999999999\\
36.01	0.00999999999999999\\
37.01	0.00999999999999999\\
38.01	0.00999999999999999\\
39.01	0.00999999999999999\\
40.01	0.00999999999999999\\
41.01	0.00999999999999999\\
42.01	0.00999999999999999\\
43.01	0.00999999999999999\\
44.01	0.00999999999999999\\
45.01	0.00999999999999999\\
46.01	0.00999999999999999\\
47.01	0.00999999999999999\\
48.01	0.00999999999999999\\
49.01	0.00999999999999999\\
50.01	0.00999999999999999\\
51.01	0.00999999999999999\\
52.01	0.00999999999999999\\
53.01	0.00999999999999999\\
54.01	0.00999999999999999\\
55.01	0.00999999999999999\\
56.01	0.00999999999999999\\
57.01	0.00999999999999999\\
58.01	0.00999999999999999\\
59.01	0.00999999999999999\\
60.01	0.00999999999999999\\
61.01	0.00999999999999999\\
62.01	0.00999999999999999\\
63.01	0.00999999999999999\\
64.01	0.00999999999999999\\
65.01	0.00999999999999999\\
66.01	0.00999999999999999\\
67.01	0.00999999999999999\\
68.01	0.00999999999999999\\
69.01	0.00999999999999999\\
70.01	0.00999999999999999\\
71.01	0.00999999999999999\\
72.01	0.00999999999999999\\
73.01	0.00999999999999999\\
74.01	0.00999999999999999\\
75.01	0.00999999999999999\\
76.01	0.00999999999999999\\
77.01	0.00999999999999999\\
78.01	0.00999999999999999\\
79.01	0.00999999999999999\\
80.01	0.00999999999999999\\
81.01	0.00999999999999999\\
82.01	0.00999999999999999\\
83.01	0.00999999999999999\\
84.01	0.00999999999999999\\
85.01	0.00999999999999999\\
86.01	0.00999999999999999\\
87.01	0.00999999999999999\\
88.01	0.00999999999999999\\
89.01	0.00999999999999999\\
90.01	0.00999999999999999\\
91.01	0.00999999999999999\\
92.01	0.00999999999999999\\
93.01	0.00999999999999999\\
94.01	0.00999999999999999\\
95.01	0.00999999999999999\\
96.01	0.00999999999999999\\
97.01	0.00999999999999999\\
98.01	0.00999999999999999\\
99.01	0.00999999999999999\\
100.01	0.00999999999999999\\
101.01	0.00999999999999999\\
102.01	0.00999999999999999\\
103.01	0.00999999999999999\\
104.01	0.00999999999999999\\
105.01	0.00999999999999999\\
106.01	0.00999999999999999\\
107.01	0.00999999999999999\\
108.01	0.00999999999999999\\
109.01	0.00999999999999999\\
110.01	0.00999999999999999\\
111.01	0.00999999999999999\\
112.01	0.00999999999999999\\
113.01	0.00999999999999999\\
114.01	0.00999999999999999\\
115.01	0.00999999999999999\\
116.01	0.00999999999999999\\
117.01	0.00999999999999999\\
118.01	0.00999999999999999\\
119.01	0.00999999999999999\\
120.01	0.00999999999999999\\
121.01	0.00999999999999999\\
122.01	0.00999999999999999\\
123.01	0.00999999999999999\\
124.01	0.00999999999999999\\
125.01	0.00999999999999999\\
126.01	0.00999999999999999\\
127.01	0.00999999999999999\\
128.01	0.00999999999999999\\
129.01	0.00999999999999999\\
130.01	0.00999999999999999\\
131.01	0.00999999999999999\\
132.01	0.00999999999999999\\
133.01	0.00999999999999999\\
134.01	0.00999999999999999\\
135.01	0.00999999999999999\\
136.01	0.00999999999999999\\
137.01	0.00999999999999999\\
138.01	0.00999999999999999\\
139.01	0.00999999999999999\\
140.01	0.00999999999999999\\
141.01	0.00999999999999999\\
142.01	0.00999999999999999\\
143.01	0.00999999999999999\\
144.01	0.00999999999999999\\
145.01	0.00999999999999999\\
146.01	0.00999999999999999\\
147.01	0.00999999999999999\\
148.01	0.00999999999999999\\
149.01	0.00999999999999999\\
150.01	0.00999999999999999\\
151.01	0.00999999999999999\\
152.01	0.00999999999999999\\
153.01	0.00999999999999999\\
154.01	0.00999999999999999\\
155.01	0.00999999999999999\\
156.01	0.00999999999999999\\
157.01	0.00999999999999999\\
158.01	0.00999999999999999\\
159.01	0.00999999999999999\\
160.01	0.00999999999999999\\
161.01	0.00999999999999999\\
162.01	0.00999999999999999\\
163.01	0.00999999999999999\\
164.01	0.00999999999999999\\
165.01	0.00999999999999999\\
166.01	0.00999999999999999\\
167.01	0.00999999999999999\\
168.01	0.00999999999999999\\
169.01	0.00999999999999999\\
170.01	0.00999999999999999\\
171.01	0.00999999999999999\\
172.01	0.00999999999999999\\
173.01	0.00999999999999999\\
174.01	0.00999999999999999\\
175.01	0.00999999999999999\\
176.01	0.00999999999999999\\
177.01	0.00999999999999999\\
178.01	0.00999999999999999\\
179.01	0.00999999999999999\\
180.01	0.00999999999999999\\
181.01	0.00999999999999999\\
182.01	0.00999999999999999\\
183.01	0.00999999999999999\\
184.01	0.00999999999999999\\
185.01	0.00999999999999999\\
186.01	0.00999999999999999\\
187.01	0.00999999999999999\\
188.01	0.00999999999999999\\
189.01	0.00999999999999999\\
190.01	0.00999999999999999\\
191.01	0.00999999999999999\\
192.01	0.00999999999999999\\
193.01	0.00999999999999999\\
194.01	0.00999999999999999\\
195.01	0.00999999999999999\\
196.01	0.00999999999999999\\
197.01	0.00999999999999999\\
198.01	0.00999999999999999\\
199.01	0.00999999999999999\\
200.01	0.00999999999999999\\
201.01	0.00999999999999999\\
202.01	0.00999999999999999\\
203.01	0.00999999999999999\\
204.01	0.00999999999999999\\
205.01	0.00999999999999999\\
206.01	0.00999999999999999\\
207.01	0.00999999999999999\\
208.01	0.00999999999999999\\
209.01	0.00999999999999999\\
210.01	0.00999999999999999\\
211.01	0.00999999999999999\\
212.01	0.00999999999999999\\
213.01	0.00999999999999999\\
214.01	0.00999999999999999\\
215.01	0.00999999999999999\\
216.01	0.00999999999999999\\
217.01	0.00999999999999999\\
218.01	0.00999999999999999\\
219.01	0.00999999999999999\\
220.01	0.00999999999999999\\
221.01	0.00999999999999999\\
222.01	0.00999999999999999\\
223.01	0.00999999999999999\\
224.01	0.00999999999999999\\
225.01	0.00999999999999999\\
226.01	0.00999999999999999\\
227.01	0.00999999999999999\\
228.01	0.00999999999999999\\
229.01	0.00999999999999999\\
230.01	0.00999999999999999\\
231.01	0.00999999999999999\\
232.01	0.00999999999999999\\
233.01	0.00999999999999999\\
234.01	0.00999999999999999\\
235.01	0.00999999999999999\\
236.01	0.00999999999999999\\
237.01	0.00999999999999999\\
238.01	0.00999999999999999\\
239.01	0.00999999999999999\\
240.01	0.00999999999999999\\
241.01	0.00999999999999999\\
242.01	0.00999999999999999\\
243.01	0.00999999999999999\\
244.01	0.00999999999999999\\
245.01	0.00999999999999999\\
246.01	0.00999999999999999\\
247.01	0.00999999999999999\\
248.01	0.00999999999999999\\
249.01	0.00999999999999999\\
250.01	0.00999999999999999\\
251.01	0.00999999999999999\\
252.01	0.00999999999999999\\
253.01	0.00999999999999999\\
254.01	0.00999999999999999\\
255.01	0.00999999999999999\\
256.01	0.00999999999999999\\
257.01	0.00999999999999999\\
258.01	0.00999999999999999\\
259.01	0.00999999999999999\\
260.01	0.00999999999999999\\
261.01	0.00999999999999999\\
262.01	0.00999999999999999\\
263.01	0.00999999999999999\\
264.01	0.00999999999999999\\
265.01	0.00999999999999999\\
266.01	0.00999999999999999\\
267.01	0.00999999999999999\\
268.01	0.00999999999999999\\
269.01	0.00999999999999999\\
270.01	0.00999999999999999\\
271.01	0.00999999999999999\\
272.01	0.00999999999999999\\
273.01	0.00999999999999999\\
274.01	0.00999999999999999\\
275.01	0.00999999999999999\\
276.01	0.00999999999999999\\
277.01	0.00999999999999999\\
278.01	0.00999999999999999\\
279.01	0.00999999999999999\\
280.01	0.00999999999999999\\
281.01	0.00999999999999999\\
282.01	0.00999999999999999\\
283.01	0.00999999999999999\\
284.01	0.00999999999999999\\
285.01	0.00999999999999999\\
286.01	0.00999999999999999\\
287.01	0.00999999999999999\\
288.01	0.00999999999999999\\
289.01	0.00999999999999999\\
290.01	0.00999999999999999\\
291.01	0.00999999999999999\\
292.01	0.00999999999999999\\
293.01	0.00999999999999999\\
294.01	0.00999999999999999\\
295.01	0.00999999999999999\\
296.01	0.00999999999999999\\
297.01	0.00999999999999999\\
298.01	0.00999999999999999\\
299.01	0.00999999999999999\\
300.01	0.00999999999999999\\
301.01	0.00999999999999999\\
302.01	0.00999999999999999\\
303.01	0.00999999999999999\\
304.01	0.00999999999999999\\
305.01	0.00999999999999999\\
306.01	0.00999999999999999\\
307.01	0.00999999999999999\\
308.01	0.00999999999999999\\
309.01	0.00999999999999999\\
310.01	0.00999999999999999\\
311.01	0.00999999999999999\\
312.01	0.00999999999999999\\
313.01	0.00999999999999999\\
314.01	0.00999999999999999\\
315.01	0.00999999999999999\\
316.01	0.00999999999999999\\
317.01	0.00999999999999999\\
318.01	0.00999999999999999\\
319.01	0.00999999999999999\\
320.01	0.00999999999999999\\
321.01	0.00999999999999999\\
322.01	0.00999999999999999\\
323.01	0.00999999999999999\\
324.01	0.00999999999999999\\
325.01	0.00999999999999999\\
326.01	0.00999999999999999\\
327.01	0.00999999999999999\\
328.01	0.00999999999999999\\
329.01	0.00999999999999999\\
330.01	0.00999999999999999\\
331.01	0.00999999999999999\\
332.01	0.00999999999999999\\
333.01	0.00999999999999999\\
334.01	0.00999999999999999\\
335.01	0.00999999999999999\\
336.01	0.00999999999999999\\
337.01	0.00999999999999999\\
338.01	0.00999999999999999\\
339.01	0.00999999999999999\\
340.01	0.00999999999999999\\
341.01	0.00999999999999999\\
342.01	0.00999999999999999\\
343.01	0.00999999999999999\\
344.01	0.00999999999999999\\
345.01	0.00999999999999999\\
346.01	0.00999999999999999\\
347.01	0.00999999999999999\\
348.01	0.00999999999999999\\
349.01	0.00999999999999999\\
350.01	0.00999999999999999\\
351.01	0.00999999999999999\\
352.01	0.00999999999999999\\
353.01	0.00999999999999999\\
354.01	0.00999999999999999\\
355.01	0.00999999999999999\\
356.01	0.00999999999999999\\
357.01	0.00999999999999999\\
358.01	0.00999999999999999\\
359.01	0.00999999999999999\\
360.01	0.00999999999999999\\
361.01	0.00999999999999999\\
362.01	0.00999999999999999\\
363.01	0.00999999999999999\\
364.01	0.00999999999999999\\
365.01	0.00999999999999999\\
366.01	0.00999999999999999\\
367.01	0.00999999999999999\\
368.01	0.00999999999999999\\
369.01	0.00999999999999999\\
370.01	0.00999999999999999\\
371.01	0.00999999999999999\\
372.01	0.00999999999999999\\
373.01	0.00999999999999999\\
374.01	0.00999999999999999\\
375.01	0.00999999999999999\\
376.01	0.00999999999999999\\
377.01	0.00999999999999999\\
378.01	0.00999999999999999\\
379.01	0.00999999999999999\\
380.01	0.00999999999999999\\
381.01	0.00999999999999999\\
382.01	0.00999999999999999\\
383.01	0.00999999999999999\\
384.01	0.00999999999999999\\
385.01	0.00999999999999999\\
386.01	0.00999999999999999\\
387.01	0.00999999999999999\\
388.01	0.00999999999999999\\
389.01	0.00999999999999999\\
390.01	0.00999999999999999\\
391.01	0.00999999999999999\\
392.01	0.00999999999999999\\
393.01	0.00999999999999999\\
394.01	0.00999999999999999\\
395.01	0.00999999999999999\\
396.01	0.00999999999999999\\
397.01	0.00999999999999999\\
398.01	0.00999999999999999\\
399.01	0.00999999999999999\\
400.01	0.00999999999999999\\
401.01	0.00999999999999999\\
402.01	0.00999999999999999\\
403.01	0.00999999999999999\\
404.01	0.00999999999999999\\
405.01	0.00999999999999999\\
406.01	0.00999999999999999\\
407.01	0.00999999999999999\\
408.01	0.00999999999999999\\
409.01	0.00999999999999999\\
410.01	0.00999999999999999\\
411.01	0.00999999999999999\\
412.01	0.00999999999999999\\
413.01	0.00999999999999999\\
414.01	0.00999999999999999\\
415.01	0.00999999999999999\\
416.01	0.00999999999999999\\
417.01	0.00999999999999999\\
418.01	0.00999999999999999\\
419.01	0.00999999999999999\\
420.01	0.00999999999999999\\
421.01	0.00999999999999999\\
422.01	0.00999999999999999\\
423.01	0.00999999999999999\\
424.01	0.00999999999999999\\
425.01	0.00999999999999999\\
426.01	0.00999999999999999\\
427.01	0.00999999999999999\\
428.01	0.00999999999999999\\
429.01	0.00999999999999999\\
430.01	0.00999999999999999\\
431.01	0.00999999999999999\\
432.01	0.00999999999999999\\
433.01	0.00999999999999999\\
434.01	0.00999999999999999\\
435.01	0.00999999999999999\\
436.01	0.00999999999999999\\
437.01	0.00999999999999999\\
438.01	0.00999999999999999\\
439.01	0.00999999999999999\\
440.01	0.00999999999999999\\
441.01	0.00999999999999999\\
442.01	0.00999999999999999\\
443.01	0.00999999999999999\\
444.01	0.00999999999999999\\
445.01	0.00999999999999999\\
446.01	0.00999999999999999\\
447.01	0.00999999999999999\\
448.01	0.00999999999999999\\
449.01	0.00999999999999999\\
450.01	0.00999999999999999\\
451.01	0.00999999999999999\\
452.01	0.00999999999999999\\
453.01	0.00999999999999999\\
454.01	0.00999999999999999\\
455.01	0.00999999999999999\\
456.01	0.00999999999999999\\
457.01	0.00999999999999999\\
458.01	0.00999999999999999\\
459.01	0.00999999999999999\\
460.01	0.00999999999999999\\
461.01	0.00999999999999999\\
462.01	0.00999999999999999\\
463.01	0.00999999999999999\\
464.01	0.00999999999999999\\
465.01	0.00999999999999999\\
466.01	0.00999999999999999\\
467.01	0.00999999999999999\\
468.01	0.00999999999999999\\
469.01	0.00999999999999999\\
470.01	0.00999999999999999\\
471.01	0.00999999999999999\\
472.01	0.00999999999999999\\
473.01	0.00999999999999999\\
474.01	0.00999999999999999\\
475.01	0.00999999999999999\\
476.01	0.00999999999999999\\
477.01	0.00999999999999999\\
478.01	0.00999999999999999\\
479.01	0.00999999999999999\\
480.01	0.00999999999999999\\
481.01	0.00999999999999999\\
482.01	0.00999999999999999\\
483.01	0.00999999999999999\\
484.01	0.00999999999999999\\
485.01	0.00999999999999999\\
486.01	0.00999999999999999\\
487.01	0.00999999999999999\\
488.01	0.00999999999999999\\
489.01	0.00999999999999999\\
490.01	0.00999999999999999\\
491.01	0.00999999999999999\\
492.01	0.00999999999999999\\
493.01	0.00999999999999999\\
494.01	0.00999999999999999\\
495.01	0.00999999999999999\\
496.01	0.00999999999999999\\
497.01	0.00999999999999999\\
498.01	0.00999999999999999\\
499.01	0.00999999999999999\\
500.01	0.00999999999999999\\
501.01	0.00999999999999999\\
502.01	0.00999999999999999\\
503.01	0.00999999999999999\\
504.01	0.00999999999999999\\
505.01	0.00999999999999999\\
506.01	0.00999999999999999\\
507.01	0.00999999999999999\\
508.01	0.00999999999999999\\
509.01	0.00999999999999999\\
510.01	0.00999999999999999\\
511.01	0.00999999999999999\\
512.01	0.00999999999999999\\
513.01	0.00999999999999999\\
514.01	0.00999999999999999\\
515.01	0.00999999999999999\\
516.01	0.00999999999999999\\
517.01	0.00999999999999999\\
518.01	0.00999999999999999\\
519.01	0.00999999999999999\\
520.01	0.00999999999999999\\
521.01	0.00999999999999999\\
522.01	0.00999999999999999\\
523.01	0.00999999999999999\\
524.01	0.00999999999999999\\
525.01	0.00999999999999999\\
526.01	0.00999999999999999\\
527.01	0.00999999999999999\\
528.01	0.00999999999999999\\
529.01	0.00999999999999999\\
530.01	0.00999999999999999\\
531.01	0.00999999999999999\\
532.01	0.00999999999999999\\
533.01	0.00999999999999999\\
534.01	0.00999999999999999\\
535.01	0.00999999999999999\\
536.01	0.00999999999999999\\
537.01	0.00999999999999999\\
538.01	0.00999999999999999\\
539.01	0.00999999999999999\\
540.01	0.00999999999999999\\
541.01	0.00999999999999999\\
542.01	0.00999999999999999\\
543.01	0.00999999999999999\\
544.01	0.00999999999999999\\
545.01	0.00999999999999999\\
546.01	0.00999999999999999\\
547.01	0.00999999999999999\\
548.01	0.00999999999999999\\
549.01	0.00999999999999999\\
550.01	0.00999999999999999\\
551.01	0.00999999999999999\\
552.01	0.00999999999999999\\
553.01	0.00999999999999999\\
554.01	0.00999999999999999\\
555.01	0.00999999999999999\\
556.01	0.00999999999999999\\
557.01	0.00999999999999999\\
558.01	0.00999999999999999\\
559.01	0.00999999999999999\\
560.01	0.00999999999999999\\
561.01	0.00999999999999999\\
562.01	0.00999999999999999\\
563.01	0.00999999999999999\\
564.01	0.00999999999999999\\
565.01	0.00999999999999999\\
566.01	0.00999999999999999\\
567.01	0.00999999999999999\\
568.01	0.00999999999999999\\
569.01	0.00999999999999999\\
570.01	0.00999999999999999\\
571.01	0.00999999999999999\\
572.01	0.00999999999999999\\
573.01	0.00999999999999999\\
574.01	0.00999999999999999\\
575.01	0.00999999999999999\\
576.01	0.00999999999999999\\
577.01	0.00999999999999999\\
578.01	0.00999999999999999\\
579.01	0.00999999999999999\\
580.01	0.00999999999999999\\
581.01	0.00999999999999999\\
582.01	0.00999999999999999\\
583.01	0.00999999999999999\\
584.01	0.00999999999999999\\
585.01	0.00999999999999999\\
586.01	0.00999999999999999\\
587.01	0.00999999999999999\\
588.01	0.00999999999999999\\
589.01	0.00999999999999999\\
590.01	0.00999999999999999\\
591.01	0.00999999999999999\\
592.01	0.00999999999999999\\
593.01	0.00999999999999999\\
594.01	0.00999999999999999\\
595.01	0.00999999999999999\\
596.01	0.00999999999999999\\
597.01	0.00999999999999999\\
598.01	0.00999999999999999\\
599.01	0.00623513569400518\\
599.02	0.00619747359804889\\
599.03	0.00615944475593818\\
599.04	0.00612104556201831\\
599.05	0.00608227237517916\\
599.06	0.00604312151850672\\
599.07	0.0060035892789309\\
599.08	0.00596367190687008\\
599.09	0.00592336561587204\\
599.1	0.00588266658225147\\
599.11	0.00584157094472378\\
599.12	0.00580007480403543\\
599.13	0.00575817422259056\\
599.14	0.00571586522407395\\
599.15	0.0056731437930703\\
599.16	0.0056300058746798\\
599.17	0.0055864473741298\\
599.18	0.00554246415638276\\
599.19	0.00549805204574034\\
599.2	0.00545320682544356\\
599.21	0.00540792423726907\\
599.22	0.00536219998112139\\
599.23	0.00531602971462118\\
599.24	0.00526940905268951\\
599.25	0.00522233356712793\\
599.26	0.00517479878619451\\
599.27	0.00512680019417564\\
599.28	0.00507833323095367\\
599.29	0.00502939329157029\\
599.3	0.0049799757257856\\
599.31	0.00493007583763286\\
599.32	0.00487968888496881\\
599.33	0.00482881007901966\\
599.34	0.00477743458392257\\
599.35	0.00472555751626257\\
599.36	0.00467317394460505\\
599.37	0.00462027888902352\\
599.38	0.0045668673206228\\
599.39	0.00451293416105749\\
599.4	0.0044584742820458\\
599.41	0.00440348251095323\\
599.42	0.00434795363554173\\
599.43	0.00429188239249109\\
599.44	0.00423526346689832\\
599.45	0.00417809149177216\\
599.46	0.00412036104752264\\
599.47	0.00406206666144547\\
599.48	0.00400320280720162\\
599.49	0.00394376390429167\\
599.5	0.00388374431752494\\
599.51	0.00382313835648361\\
599.52	0.00376194027498138\\
599.53	0.00370014427051699\\
599.54	0.0036377444837223\\
599.55	0.00357473499780495\\
599.56	0.00351110983798564\\
599.57	0.00344686297092978\\
599.58	0.00338198830417369\\
599.59	0.00331647968554506\\
599.6	0.00325033090257786\\
599.61	0.00318353568192136\\
599.62	0.00311608768874352\\
599.63	0.00304798052612842\\
599.64	0.00297920773446782\\
599.65	0.00290976279084676\\
599.66	0.00283963910842317\\
599.67	0.00276883003580129\\
599.68	0.00269732885639912\\
599.69	0.00262512878780951\\
599.7	0.0025522229811551\\
599.71	0.00247860452043686\\
599.72	0.00240426642187629\\
599.73	0.0023292016332512\\
599.74	0.00225340303322488\\
599.75	0.00217686343066881\\
599.76	0.00209957556397867\\
599.77	0.00202153210038366\\
599.78	0.00194272563524905\\
599.79	0.00186314869137186\\
599.8	0.00178279371826976\\
599.81	0.00170165309146283\\
599.82	0.00161971911174841\\
599.83	0.00153698400446879\\
599.84	0.00145343991877172\\
599.85	0.00136907892686372\\
599.86	0.00128389302325598\\
599.87	0.00119787412400299\\
599.88	0.00111101406593363\\
599.89	0.00102330460587468\\
599.9	0.000934737419866901\\
599.91	0.000845304102373188\\
599.92	0.000754996165479168\\
599.93	0.000663805038085868\\
599.94	0.000571722065094473\\
599.95	0.000478738506583127\\
599.96	0.000384845536975638\\
599.97	0.000290034244202044\\
599.98	0.00019429562885097\\
599.99	9.76206033136574e-05\\
600	0\\
};
\addplot [color=blue!25!mycolor7,solid,forget plot]
  table[row sep=crcr]{%
0.01	0.00999999999999999\\
1.01	0.00999999999999999\\
2.01	0.00999999999999999\\
3.01	0.00999999999999999\\
4.01	0.00999999999999999\\
5.01	0.00999999999999999\\
6.01	0.00999999999999999\\
7.01	0.00999999999999999\\
8.01	0.00999999999999999\\
9.01	0.00999999999999999\\
10.01	0.00999999999999999\\
11.01	0.00999999999999999\\
12.01	0.00999999999999999\\
13.01	0.00999999999999999\\
14.01	0.00999999999999999\\
15.01	0.00999999999999999\\
16.01	0.00999999999999999\\
17.01	0.00999999999999999\\
18.01	0.00999999999999999\\
19.01	0.00999999999999999\\
20.01	0.00999999999999999\\
21.01	0.00999999999999999\\
22.01	0.00999999999999999\\
23.01	0.00999999999999999\\
24.01	0.00999999999999999\\
25.01	0.00999999999999999\\
26.01	0.00999999999999999\\
27.01	0.00999999999999999\\
28.01	0.00999999999999999\\
29.01	0.00999999999999999\\
30.01	0.00999999999999999\\
31.01	0.00999999999999999\\
32.01	0.00999999999999999\\
33.01	0.00999999999999999\\
34.01	0.00999999999999999\\
35.01	0.00999999999999999\\
36.01	0.00999999999999999\\
37.01	0.00999999999999999\\
38.01	0.00999999999999999\\
39.01	0.00999999999999999\\
40.01	0.00999999999999999\\
41.01	0.00999999999999999\\
42.01	0.00999999999999999\\
43.01	0.00999999999999999\\
44.01	0.00999999999999999\\
45.01	0.00999999999999999\\
46.01	0.00999999999999999\\
47.01	0.00999999999999999\\
48.01	0.00999999999999999\\
49.01	0.00999999999999999\\
50.01	0.00999999999999999\\
51.01	0.00999999999999999\\
52.01	0.00999999999999999\\
53.01	0.00999999999999999\\
54.01	0.00999999999999999\\
55.01	0.00999999999999999\\
56.01	0.00999999999999999\\
57.01	0.00999999999999999\\
58.01	0.00999999999999999\\
59.01	0.00999999999999999\\
60.01	0.00999999999999999\\
61.01	0.00999999999999999\\
62.01	0.00999999999999999\\
63.01	0.00999999999999999\\
64.01	0.00999999999999999\\
65.01	0.00999999999999999\\
66.01	0.00999999999999999\\
67.01	0.00999999999999999\\
68.01	0.00999999999999999\\
69.01	0.00999999999999999\\
70.01	0.00999999999999999\\
71.01	0.00999999999999999\\
72.01	0.00999999999999999\\
73.01	0.00999999999999999\\
74.01	0.00999999999999999\\
75.01	0.00999999999999999\\
76.01	0.00999999999999999\\
77.01	0.00999999999999999\\
78.01	0.00999999999999999\\
79.01	0.00999999999999999\\
80.01	0.00999999999999999\\
81.01	0.00999999999999999\\
82.01	0.00999999999999999\\
83.01	0.00999999999999999\\
84.01	0.00999999999999999\\
85.01	0.00999999999999999\\
86.01	0.00999999999999999\\
87.01	0.00999999999999999\\
88.01	0.00999999999999999\\
89.01	0.00999999999999999\\
90.01	0.00999999999999999\\
91.01	0.00999999999999999\\
92.01	0.00999999999999999\\
93.01	0.00999999999999999\\
94.01	0.00999999999999999\\
95.01	0.00999999999999999\\
96.01	0.00999999999999999\\
97.01	0.00999999999999999\\
98.01	0.00999999999999999\\
99.01	0.00999999999999999\\
100.01	0.00999999999999999\\
101.01	0.00999999999999999\\
102.01	0.00999999999999999\\
103.01	0.00999999999999999\\
104.01	0.00999999999999999\\
105.01	0.00999999999999999\\
106.01	0.00999999999999999\\
107.01	0.00999999999999999\\
108.01	0.00999999999999999\\
109.01	0.00999999999999999\\
110.01	0.00999999999999999\\
111.01	0.00999999999999999\\
112.01	0.00999999999999999\\
113.01	0.00999999999999999\\
114.01	0.00999999999999999\\
115.01	0.00999999999999999\\
116.01	0.00999999999999999\\
117.01	0.00999999999999999\\
118.01	0.00999999999999999\\
119.01	0.00999999999999999\\
120.01	0.00999999999999999\\
121.01	0.00999999999999999\\
122.01	0.00999999999999999\\
123.01	0.00999999999999999\\
124.01	0.00999999999999999\\
125.01	0.00999999999999999\\
126.01	0.00999999999999999\\
127.01	0.00999999999999999\\
128.01	0.00999999999999999\\
129.01	0.00999999999999999\\
130.01	0.00999999999999999\\
131.01	0.00999999999999999\\
132.01	0.00999999999999999\\
133.01	0.00999999999999999\\
134.01	0.00999999999999999\\
135.01	0.00999999999999999\\
136.01	0.00999999999999999\\
137.01	0.00999999999999999\\
138.01	0.00999999999999999\\
139.01	0.00999999999999999\\
140.01	0.00999999999999999\\
141.01	0.00999999999999999\\
142.01	0.00999999999999999\\
143.01	0.00999999999999999\\
144.01	0.00999999999999999\\
145.01	0.00999999999999999\\
146.01	0.00999999999999999\\
147.01	0.00999999999999999\\
148.01	0.00999999999999999\\
149.01	0.00999999999999999\\
150.01	0.00999999999999999\\
151.01	0.00999999999999999\\
152.01	0.00999999999999999\\
153.01	0.00999999999999999\\
154.01	0.00999999999999999\\
155.01	0.00999999999999999\\
156.01	0.00999999999999999\\
157.01	0.00999999999999999\\
158.01	0.00999999999999999\\
159.01	0.00999999999999999\\
160.01	0.00999999999999999\\
161.01	0.00999999999999999\\
162.01	0.00999999999999999\\
163.01	0.00999999999999999\\
164.01	0.00999999999999999\\
165.01	0.00999999999999999\\
166.01	0.00999999999999999\\
167.01	0.00999999999999999\\
168.01	0.00999999999999999\\
169.01	0.00999999999999999\\
170.01	0.00999999999999999\\
171.01	0.00999999999999999\\
172.01	0.00999999999999999\\
173.01	0.00999999999999999\\
174.01	0.00999999999999999\\
175.01	0.00999999999999999\\
176.01	0.00999999999999999\\
177.01	0.00999999999999999\\
178.01	0.00999999999999999\\
179.01	0.00999999999999999\\
180.01	0.00999999999999999\\
181.01	0.00999999999999999\\
182.01	0.00999999999999999\\
183.01	0.00999999999999999\\
184.01	0.00999999999999999\\
185.01	0.00999999999999999\\
186.01	0.00999999999999999\\
187.01	0.00999999999999999\\
188.01	0.00999999999999999\\
189.01	0.00999999999999999\\
190.01	0.00999999999999999\\
191.01	0.00999999999999999\\
192.01	0.00999999999999999\\
193.01	0.00999999999999999\\
194.01	0.00999999999999999\\
195.01	0.00999999999999999\\
196.01	0.00999999999999999\\
197.01	0.00999999999999999\\
198.01	0.00999999999999999\\
199.01	0.00999999999999999\\
200.01	0.00999999999999999\\
201.01	0.00999999999999999\\
202.01	0.00999999999999999\\
203.01	0.00999999999999999\\
204.01	0.00999999999999999\\
205.01	0.00999999999999999\\
206.01	0.00999999999999999\\
207.01	0.00999999999999999\\
208.01	0.00999999999999999\\
209.01	0.00999999999999999\\
210.01	0.00999999999999999\\
211.01	0.00999999999999999\\
212.01	0.00999999999999999\\
213.01	0.00999999999999999\\
214.01	0.00999999999999999\\
215.01	0.00999999999999999\\
216.01	0.00999999999999999\\
217.01	0.00999999999999999\\
218.01	0.00999999999999999\\
219.01	0.00999999999999999\\
220.01	0.00999999999999999\\
221.01	0.00999999999999999\\
222.01	0.00999999999999999\\
223.01	0.00999999999999999\\
224.01	0.00999999999999999\\
225.01	0.00999999999999999\\
226.01	0.00999999999999999\\
227.01	0.00999999999999999\\
228.01	0.00999999999999999\\
229.01	0.00999999999999999\\
230.01	0.00999999999999999\\
231.01	0.00999999999999999\\
232.01	0.00999999999999999\\
233.01	0.00999999999999999\\
234.01	0.00999999999999999\\
235.01	0.00999999999999999\\
236.01	0.00999999999999999\\
237.01	0.00999999999999999\\
238.01	0.00999999999999999\\
239.01	0.00999999999999999\\
240.01	0.00999999999999999\\
241.01	0.00999999999999999\\
242.01	0.00999999999999999\\
243.01	0.00999999999999999\\
244.01	0.00999999999999999\\
245.01	0.00999999999999999\\
246.01	0.00999999999999999\\
247.01	0.00999999999999999\\
248.01	0.00999999999999999\\
249.01	0.00999999999999999\\
250.01	0.00999999999999999\\
251.01	0.00999999999999999\\
252.01	0.00999999999999999\\
253.01	0.00999999999999999\\
254.01	0.00999999999999999\\
255.01	0.00999999999999999\\
256.01	0.00999999999999999\\
257.01	0.00999999999999999\\
258.01	0.00999999999999999\\
259.01	0.00999999999999999\\
260.01	0.00999999999999999\\
261.01	0.00999999999999999\\
262.01	0.00999999999999999\\
263.01	0.00999999999999999\\
264.01	0.00999999999999999\\
265.01	0.00999999999999999\\
266.01	0.00999999999999999\\
267.01	0.00999999999999999\\
268.01	0.00999999999999999\\
269.01	0.00999999999999999\\
270.01	0.00999999999999999\\
271.01	0.00999999999999999\\
272.01	0.00999999999999999\\
273.01	0.00999999999999999\\
274.01	0.00999999999999999\\
275.01	0.00999999999999999\\
276.01	0.00999999999999999\\
277.01	0.00999999999999999\\
278.01	0.00999999999999999\\
279.01	0.00999999999999999\\
280.01	0.00999999999999999\\
281.01	0.00999999999999999\\
282.01	0.00999999999999999\\
283.01	0.00999999999999999\\
284.01	0.00999999999999999\\
285.01	0.00999999999999999\\
286.01	0.00999999999999999\\
287.01	0.00999999999999999\\
288.01	0.00999999999999999\\
289.01	0.00999999999999999\\
290.01	0.00999999999999999\\
291.01	0.00999999999999999\\
292.01	0.00999999999999999\\
293.01	0.00999999999999999\\
294.01	0.00999999999999999\\
295.01	0.00999999999999999\\
296.01	0.00999999999999999\\
297.01	0.00999999999999999\\
298.01	0.00999999999999999\\
299.01	0.00999999999999999\\
300.01	0.00999999999999999\\
301.01	0.00999999999999999\\
302.01	0.00999999999999999\\
303.01	0.00999999999999999\\
304.01	0.00999999999999999\\
305.01	0.00999999999999999\\
306.01	0.00999999999999999\\
307.01	0.00999999999999999\\
308.01	0.00999999999999999\\
309.01	0.00999999999999999\\
310.01	0.00999999999999999\\
311.01	0.00999999999999999\\
312.01	0.00999999999999999\\
313.01	0.00999999999999999\\
314.01	0.00999999999999999\\
315.01	0.00999999999999999\\
316.01	0.00999999999999999\\
317.01	0.00999999999999999\\
318.01	0.00999999999999999\\
319.01	0.00999999999999999\\
320.01	0.00999999999999999\\
321.01	0.00999999999999999\\
322.01	0.00999999999999999\\
323.01	0.00999999999999999\\
324.01	0.00999999999999999\\
325.01	0.00999999999999999\\
326.01	0.00999999999999999\\
327.01	0.00999999999999999\\
328.01	0.00999999999999999\\
329.01	0.00999999999999999\\
330.01	0.00999999999999999\\
331.01	0.00999999999999999\\
332.01	0.00999999999999999\\
333.01	0.00999999999999999\\
334.01	0.00999999999999999\\
335.01	0.00999999999999999\\
336.01	0.00999999999999999\\
337.01	0.00999999999999999\\
338.01	0.00999999999999999\\
339.01	0.00999999999999999\\
340.01	0.00999999999999999\\
341.01	0.00999999999999999\\
342.01	0.00999999999999999\\
343.01	0.00999999999999999\\
344.01	0.00999999999999999\\
345.01	0.00999999999999999\\
346.01	0.00999999999999999\\
347.01	0.00999999999999999\\
348.01	0.00999999999999999\\
349.01	0.00999999999999999\\
350.01	0.00999999999999999\\
351.01	0.00999999999999999\\
352.01	0.00999999999999999\\
353.01	0.00999999999999999\\
354.01	0.00999999999999999\\
355.01	0.00999999999999999\\
356.01	0.00999999999999999\\
357.01	0.00999999999999999\\
358.01	0.00999999999999999\\
359.01	0.00999999999999999\\
360.01	0.00999999999999999\\
361.01	0.00999999999999999\\
362.01	0.00999999999999999\\
363.01	0.00999999999999999\\
364.01	0.00999999999999999\\
365.01	0.00999999999999999\\
366.01	0.00999999999999999\\
367.01	0.00999999999999999\\
368.01	0.00999999999999999\\
369.01	0.00999999999999999\\
370.01	0.00999999999999999\\
371.01	0.00999999999999999\\
372.01	0.00999999999999999\\
373.01	0.00999999999999999\\
374.01	0.00999999999999999\\
375.01	0.00999999999999999\\
376.01	0.00999999999999999\\
377.01	0.00999999999999999\\
378.01	0.00999999999999999\\
379.01	0.00999999999999999\\
380.01	0.00999999999999999\\
381.01	0.00999999999999999\\
382.01	0.00999999999999999\\
383.01	0.00999999999999999\\
384.01	0.00999999999999999\\
385.01	0.00999999999999999\\
386.01	0.00999999999999999\\
387.01	0.00999999999999999\\
388.01	0.00999999999999999\\
389.01	0.00999999999999999\\
390.01	0.00999999999999999\\
391.01	0.00999999999999999\\
392.01	0.00999999999999999\\
393.01	0.00999999999999999\\
394.01	0.00999999999999999\\
395.01	0.00999999999999999\\
396.01	0.00999999999999999\\
397.01	0.00999999999999999\\
398.01	0.00999999999999999\\
399.01	0.00999999999999999\\
400.01	0.00999999999999999\\
401.01	0.00999999999999999\\
402.01	0.00999999999999999\\
403.01	0.00999999999999999\\
404.01	0.00999999999999999\\
405.01	0.00999999999999999\\
406.01	0.00999999999999999\\
407.01	0.00999999999999999\\
408.01	0.00999999999999999\\
409.01	0.00999999999999999\\
410.01	0.00999999999999999\\
411.01	0.00999999999999999\\
412.01	0.00999999999999999\\
413.01	0.00999999999999999\\
414.01	0.00999999999999999\\
415.01	0.00999999999999999\\
416.01	0.00999999999999999\\
417.01	0.00999999999999999\\
418.01	0.00999999999999999\\
419.01	0.00999999999999999\\
420.01	0.00999999999999999\\
421.01	0.00999999999999999\\
422.01	0.00999999999999999\\
423.01	0.00999999999999999\\
424.01	0.00999999999999999\\
425.01	0.00999999999999999\\
426.01	0.00999999999999999\\
427.01	0.00999999999999999\\
428.01	0.00999999999999999\\
429.01	0.00999999999999999\\
430.01	0.00999999999999999\\
431.01	0.00999999999999999\\
432.01	0.00999999999999999\\
433.01	0.00999999999999999\\
434.01	0.00999999999999999\\
435.01	0.00999999999999999\\
436.01	0.00999999999999999\\
437.01	0.00999999999999999\\
438.01	0.00999999999999999\\
439.01	0.00999999999999999\\
440.01	0.00999999999999999\\
441.01	0.00999999999999999\\
442.01	0.00999999999999999\\
443.01	0.00999999999999999\\
444.01	0.00999999999999999\\
445.01	0.00999999999999999\\
446.01	0.00999999999999999\\
447.01	0.00999999999999999\\
448.01	0.00999999999999999\\
449.01	0.00999999999999999\\
450.01	0.00999999999999999\\
451.01	0.00999999999999999\\
452.01	0.00999999999999999\\
453.01	0.00999999999999999\\
454.01	0.00999999999999999\\
455.01	0.00999999999999999\\
456.01	0.00999999999999999\\
457.01	0.00999999999999999\\
458.01	0.00999999999999999\\
459.01	0.00999999999999999\\
460.01	0.00999999999999999\\
461.01	0.00999999999999999\\
462.01	0.00999999999999999\\
463.01	0.00999999999999999\\
464.01	0.00999999999999999\\
465.01	0.00999999999999999\\
466.01	0.00999999999999999\\
467.01	0.00999999999999999\\
468.01	0.00999999999999999\\
469.01	0.00999999999999999\\
470.01	0.00999999999999999\\
471.01	0.00999999999999999\\
472.01	0.00999999999999999\\
473.01	0.00999999999999999\\
474.01	0.00999999999999999\\
475.01	0.00999999999999999\\
476.01	0.00999999999999999\\
477.01	0.00999999999999999\\
478.01	0.00999999999999999\\
479.01	0.00999999999999999\\
480.01	0.00999999999999999\\
481.01	0.00999999999999999\\
482.01	0.00999999999999999\\
483.01	0.00999999999999999\\
484.01	0.00999999999999999\\
485.01	0.00999999999999999\\
486.01	0.00999999999999999\\
487.01	0.00999999999999999\\
488.01	0.00999999999999999\\
489.01	0.00999999999999999\\
490.01	0.00999999999999999\\
491.01	0.00999999999999999\\
492.01	0.00999999999999999\\
493.01	0.00999999999999999\\
494.01	0.00999999999999999\\
495.01	0.00999999999999999\\
496.01	0.00999999999999999\\
497.01	0.00999999999999999\\
498.01	0.00999999999999999\\
499.01	0.00999999999999999\\
500.01	0.00999999999999999\\
501.01	0.00999999999999999\\
502.01	0.00999999999999999\\
503.01	0.00999999999999999\\
504.01	0.00999999999999999\\
505.01	0.00999999999999999\\
506.01	0.00999999999999999\\
507.01	0.00999999999999999\\
508.01	0.00999999999999999\\
509.01	0.00999999999999999\\
510.01	0.00999999999999999\\
511.01	0.00999999999999999\\
512.01	0.00999999999999999\\
513.01	0.00999999999999999\\
514.01	0.00999999999999999\\
515.01	0.00999999999999999\\
516.01	0.00999999999999999\\
517.01	0.00999999999999999\\
518.01	0.00999999999999999\\
519.01	0.00999999999999999\\
520.01	0.00999999999999999\\
521.01	0.00999999999999999\\
522.01	0.00999999999999999\\
523.01	0.00999999999999999\\
524.01	0.00999999999999999\\
525.01	0.00999999999999999\\
526.01	0.00999999999999999\\
527.01	0.00999999999999999\\
528.01	0.00999999999999999\\
529.01	0.00999999999999999\\
530.01	0.00999999999999999\\
531.01	0.00999999999999999\\
532.01	0.00999999999999999\\
533.01	0.00999999999999999\\
534.01	0.00999999999999999\\
535.01	0.00999999999999999\\
536.01	0.00999999999999999\\
537.01	0.00999999999999999\\
538.01	0.00999999999999999\\
539.01	0.00999999999999999\\
540.01	0.00999999999999999\\
541.01	0.00999999999999999\\
542.01	0.00999999999999999\\
543.01	0.00999999999999999\\
544.01	0.00999999999999999\\
545.01	0.00999999999999999\\
546.01	0.00999999999999999\\
547.01	0.00999999999999999\\
548.01	0.00999999999999999\\
549.01	0.00999999999999999\\
550.01	0.00999999999999999\\
551.01	0.00999999999999999\\
552.01	0.00999999999999999\\
553.01	0.00999999999999999\\
554.01	0.00999999999999999\\
555.01	0.00999999999999999\\
556.01	0.00999999999999999\\
557.01	0.00999999999999999\\
558.01	0.00999999999999999\\
559.01	0.00999999999999999\\
560.01	0.00999999999999999\\
561.01	0.00999999999999999\\
562.01	0.00999999999999999\\
563.01	0.00999999999999999\\
564.01	0.00999999999999999\\
565.01	0.00999999999999999\\
566.01	0.00999999999999999\\
567.01	0.00999999999999999\\
568.01	0.00999999999999999\\
569.01	0.00999999999999999\\
570.01	0.00999999999999999\\
571.01	0.00999999999999999\\
572.01	0.00999999999999999\\
573.01	0.00999999999999999\\
574.01	0.00999999999999999\\
575.01	0.00999999999999999\\
576.01	0.00999999999999999\\
577.01	0.00999999999999999\\
578.01	0.00999999999999999\\
579.01	0.00999999999999999\\
580.01	0.00999999999999999\\
581.01	0.00999999999999999\\
582.01	0.00999999999999999\\
583.01	0.00999999999999999\\
584.01	0.00999999999999999\\
585.01	0.00999999999999999\\
586.01	0.00999999999999999\\
587.01	0.00999999999999999\\
588.01	0.00999999999999999\\
589.01	0.00999999999999999\\
590.01	0.00999999999999999\\
591.01	0.00999999999999999\\
592.01	0.00999999999999999\\
593.01	0.00999999999999999\\
594.01	0.00999999999999999\\
595.01	0.00999999999999999\\
596.01	0.00999999999999999\\
597.01	0.00999999999999999\\
598.01	0.00999999999999999\\
599.01	0.00623513569400616\\
599.02	0.00619747359804981\\
599.03	0.00615944475593906\\
599.04	0.00612104556201914\\
599.05	0.00608227237517996\\
599.06	0.00604312151850748\\
599.07	0.00600358927893161\\
599.08	0.00596367190687074\\
599.09	0.00592336561587267\\
599.1	0.00588266658225206\\
599.11	0.00584157094472434\\
599.12	0.00580007480403596\\
599.13	0.00575817422259104\\
599.14	0.00571586522407439\\
599.15	0.00567314379307072\\
599.16	0.00563000587468019\\
599.17	0.00558644737413016\\
599.18	0.00554246415638309\\
599.19	0.00549805204574064\\
599.2	0.00545320682544385\\
599.21	0.00540792423726933\\
599.22	0.00536219998112162\\
599.23	0.00531602971462141\\
599.24	0.00526940905268973\\
599.25	0.00522233356712813\\
599.26	0.00517479878619469\\
599.27	0.00512680019417579\\
599.28	0.0050783332309538\\
599.29	0.00502939329157042\\
599.3	0.00497997572578572\\
599.31	0.00493007583763297\\
599.32	0.0048796888849689\\
599.33	0.00482881007901974\\
599.34	0.00477743458392264\\
599.35	0.00472555751626265\\
599.36	0.00467317394460513\\
599.37	0.00462027888902359\\
599.38	0.00456686732062286\\
599.39	0.00451293416105756\\
599.4	0.00445847428204585\\
599.41	0.0044034825109533\\
599.42	0.0043479536355418\\
599.43	0.00429188239249114\\
599.44	0.00423526346689836\\
599.45	0.0041780914917722\\
599.46	0.00412036104752266\\
599.47	0.00406206666144549\\
599.48	0.00400320280720165\\
599.49	0.00394376390429169\\
599.5	0.00388374431752497\\
599.51	0.00382313835648363\\
599.52	0.0037619402749814\\
599.53	0.00370014427051701\\
599.54	0.00363774448372232\\
599.55	0.00357473499780498\\
599.56	0.00351110983798567\\
599.57	0.00344686297092981\\
599.58	0.00338198830417371\\
599.59	0.00331647968554508\\
599.6	0.00325033090257787\\
599.61	0.00318353568192137\\
599.62	0.00311608768874353\\
599.63	0.00304798052612843\\
599.64	0.00297920773446783\\
599.65	0.00290976279084677\\
599.66	0.00283963910842317\\
599.67	0.0027688300358013\\
599.68	0.00269732885639913\\
599.69	0.00262512878780952\\
599.7	0.00255222298115511\\
599.71	0.00247860452043687\\
599.72	0.00240426642187631\\
599.73	0.00232920163325121\\
599.74	0.00225340303322489\\
599.75	0.00217686343066882\\
599.76	0.00209957556397868\\
599.77	0.00202153210038367\\
599.78	0.00194272563524905\\
599.79	0.00186314869137186\\
599.8	0.00178279371826976\\
599.81	0.00170165309146283\\
599.82	0.00161971911174841\\
599.83	0.00153698400446879\\
599.84	0.00145343991877172\\
599.85	0.00136907892686371\\
599.86	0.00128389302325598\\
599.87	0.00119787412400299\\
599.88	0.00111101406593363\\
599.89	0.00102330460587468\\
599.9	0.000934737419866903\\
599.91	0.000845304102373186\\
599.92	0.000754996165479169\\
599.93	0.000663805038085868\\
599.94	0.000571722065094473\\
599.95	0.000478738506583129\\
599.96	0.000384845536975638\\
599.97	0.000290034244202044\\
599.98	0.00019429562885097\\
599.99	9.76206033136556e-05\\
600	0\\
};
\addplot [color=mycolor9,solid,forget plot]
  table[row sep=crcr]{%
0.01	0.00999999999999999\\
1.01	0.00999999999999999\\
2.01	0.00999999999999999\\
3.01	0.00999999999999999\\
4.01	0.00999999999999999\\
5.01	0.00999999999999999\\
6.01	0.00999999999999999\\
7.01	0.00999999999999999\\
8.01	0.00999999999999999\\
9.01	0.00999999999999999\\
10.01	0.00999999999999999\\
11.01	0.00999999999999999\\
12.01	0.00999999999999999\\
13.01	0.00999999999999999\\
14.01	0.00999999999999999\\
15.01	0.00999999999999999\\
16.01	0.00999999999999999\\
17.01	0.00999999999999999\\
18.01	0.00999999999999999\\
19.01	0.00999999999999999\\
20.01	0.00999999999999999\\
21.01	0.00999999999999999\\
22.01	0.00999999999999999\\
23.01	0.00999999999999999\\
24.01	0.00999999999999999\\
25.01	0.00999999999999999\\
26.01	0.00999999999999999\\
27.01	0.00999999999999999\\
28.01	0.00999999999999999\\
29.01	0.00999999999999999\\
30.01	0.00999999999999999\\
31.01	0.00999999999999999\\
32.01	0.00999999999999999\\
33.01	0.00999999999999999\\
34.01	0.00999999999999999\\
35.01	0.00999999999999999\\
36.01	0.00999999999999999\\
37.01	0.00999999999999999\\
38.01	0.00999999999999999\\
39.01	0.00999999999999999\\
40.01	0.00999999999999999\\
41.01	0.00999999999999999\\
42.01	0.00999999999999999\\
43.01	0.00999999999999999\\
44.01	0.00999999999999999\\
45.01	0.00999999999999999\\
46.01	0.00999999999999999\\
47.01	0.00999999999999999\\
48.01	0.00999999999999999\\
49.01	0.00999999999999999\\
50.01	0.00999999999999999\\
51.01	0.00999999999999999\\
52.01	0.00999999999999999\\
53.01	0.00999999999999999\\
54.01	0.00999999999999999\\
55.01	0.00999999999999999\\
56.01	0.00999999999999999\\
57.01	0.00999999999999999\\
58.01	0.00999999999999999\\
59.01	0.00999999999999999\\
60.01	0.00999999999999999\\
61.01	0.00999999999999999\\
62.01	0.00999999999999999\\
63.01	0.00999999999999999\\
64.01	0.00999999999999999\\
65.01	0.00999999999999999\\
66.01	0.00999999999999999\\
67.01	0.00999999999999999\\
68.01	0.00999999999999999\\
69.01	0.00999999999999999\\
70.01	0.00999999999999999\\
71.01	0.00999999999999999\\
72.01	0.00999999999999999\\
73.01	0.00999999999999999\\
74.01	0.00999999999999999\\
75.01	0.00999999999999999\\
76.01	0.00999999999999999\\
77.01	0.00999999999999999\\
78.01	0.00999999999999999\\
79.01	0.00999999999999999\\
80.01	0.00999999999999999\\
81.01	0.00999999999999999\\
82.01	0.00999999999999999\\
83.01	0.00999999999999999\\
84.01	0.00999999999999999\\
85.01	0.00999999999999999\\
86.01	0.00999999999999999\\
87.01	0.00999999999999999\\
88.01	0.00999999999999999\\
89.01	0.00999999999999999\\
90.01	0.00999999999999999\\
91.01	0.00999999999999999\\
92.01	0.00999999999999999\\
93.01	0.00999999999999999\\
94.01	0.00999999999999999\\
95.01	0.00999999999999999\\
96.01	0.00999999999999999\\
97.01	0.00999999999999999\\
98.01	0.00999999999999999\\
99.01	0.00999999999999999\\
100.01	0.00999999999999999\\
101.01	0.00999999999999999\\
102.01	0.00999999999999999\\
103.01	0.00999999999999999\\
104.01	0.00999999999999999\\
105.01	0.00999999999999999\\
106.01	0.00999999999999999\\
107.01	0.00999999999999999\\
108.01	0.00999999999999999\\
109.01	0.00999999999999999\\
110.01	0.00999999999999999\\
111.01	0.00999999999999999\\
112.01	0.00999999999999999\\
113.01	0.00999999999999999\\
114.01	0.00999999999999999\\
115.01	0.00999999999999999\\
116.01	0.00999999999999999\\
117.01	0.00999999999999999\\
118.01	0.00999999999999999\\
119.01	0.00999999999999999\\
120.01	0.00999999999999999\\
121.01	0.00999999999999999\\
122.01	0.00999999999999999\\
123.01	0.00999999999999999\\
124.01	0.00999999999999999\\
125.01	0.00999999999999999\\
126.01	0.00999999999999999\\
127.01	0.00999999999999999\\
128.01	0.00999999999999999\\
129.01	0.00999999999999999\\
130.01	0.00999999999999999\\
131.01	0.00999999999999999\\
132.01	0.00999999999999999\\
133.01	0.00999999999999999\\
134.01	0.00999999999999999\\
135.01	0.00999999999999999\\
136.01	0.00999999999999999\\
137.01	0.00999999999999999\\
138.01	0.00999999999999999\\
139.01	0.00999999999999999\\
140.01	0.00999999999999999\\
141.01	0.00999999999999999\\
142.01	0.00999999999999999\\
143.01	0.00999999999999999\\
144.01	0.00999999999999999\\
145.01	0.00999999999999999\\
146.01	0.00999999999999999\\
147.01	0.00999999999999999\\
148.01	0.00999999999999999\\
149.01	0.00999999999999999\\
150.01	0.00999999999999999\\
151.01	0.00999999999999999\\
152.01	0.00999999999999999\\
153.01	0.00999999999999999\\
154.01	0.00999999999999999\\
155.01	0.00999999999999999\\
156.01	0.00999999999999999\\
157.01	0.00999999999999999\\
158.01	0.00999999999999999\\
159.01	0.00999999999999999\\
160.01	0.00999999999999999\\
161.01	0.00999999999999999\\
162.01	0.00999999999999999\\
163.01	0.00999999999999999\\
164.01	0.00999999999999999\\
165.01	0.00999999999999999\\
166.01	0.00999999999999999\\
167.01	0.00999999999999999\\
168.01	0.00999999999999999\\
169.01	0.00999999999999999\\
170.01	0.00999999999999999\\
171.01	0.00999999999999999\\
172.01	0.00999999999999999\\
173.01	0.00999999999999999\\
174.01	0.00999999999999999\\
175.01	0.00999999999999999\\
176.01	0.00999999999999999\\
177.01	0.00999999999999999\\
178.01	0.00999999999999999\\
179.01	0.00999999999999999\\
180.01	0.00999999999999999\\
181.01	0.00999999999999999\\
182.01	0.00999999999999999\\
183.01	0.00999999999999999\\
184.01	0.00999999999999999\\
185.01	0.00999999999999999\\
186.01	0.00999999999999999\\
187.01	0.00999999999999999\\
188.01	0.00999999999999999\\
189.01	0.00999999999999999\\
190.01	0.00999999999999999\\
191.01	0.00999999999999999\\
192.01	0.00999999999999999\\
193.01	0.00999999999999999\\
194.01	0.00999999999999999\\
195.01	0.00999999999999999\\
196.01	0.00999999999999999\\
197.01	0.00999999999999999\\
198.01	0.00999999999999999\\
199.01	0.00999999999999999\\
200.01	0.00999999999999999\\
201.01	0.00999999999999999\\
202.01	0.00999999999999999\\
203.01	0.00999999999999999\\
204.01	0.00999999999999999\\
205.01	0.00999999999999999\\
206.01	0.00999999999999999\\
207.01	0.00999999999999999\\
208.01	0.00999999999999999\\
209.01	0.00999999999999999\\
210.01	0.00999999999999999\\
211.01	0.00999999999999999\\
212.01	0.00999999999999999\\
213.01	0.00999999999999999\\
214.01	0.00999999999999999\\
215.01	0.00999999999999999\\
216.01	0.00999999999999999\\
217.01	0.00999999999999999\\
218.01	0.00999999999999999\\
219.01	0.00999999999999999\\
220.01	0.00999999999999999\\
221.01	0.00999999999999999\\
222.01	0.00999999999999999\\
223.01	0.00999999999999999\\
224.01	0.00999999999999999\\
225.01	0.00999999999999999\\
226.01	0.00999999999999999\\
227.01	0.00999999999999999\\
228.01	0.00999999999999999\\
229.01	0.00999999999999999\\
230.01	0.00999999999999999\\
231.01	0.00999999999999999\\
232.01	0.00999999999999999\\
233.01	0.00999999999999999\\
234.01	0.00999999999999999\\
235.01	0.00999999999999999\\
236.01	0.00999999999999999\\
237.01	0.00999999999999999\\
238.01	0.00999999999999999\\
239.01	0.00999999999999999\\
240.01	0.00999999999999999\\
241.01	0.00999999999999999\\
242.01	0.00999999999999999\\
243.01	0.00999999999999999\\
244.01	0.00999999999999999\\
245.01	0.00999999999999999\\
246.01	0.00999999999999999\\
247.01	0.00999999999999999\\
248.01	0.00999999999999999\\
249.01	0.00999999999999999\\
250.01	0.00999999999999999\\
251.01	0.00999999999999999\\
252.01	0.00999999999999999\\
253.01	0.00999999999999999\\
254.01	0.00999999999999999\\
255.01	0.00999999999999999\\
256.01	0.00999999999999999\\
257.01	0.00999999999999999\\
258.01	0.00999999999999999\\
259.01	0.00999999999999999\\
260.01	0.00999999999999999\\
261.01	0.00999999999999999\\
262.01	0.00999999999999999\\
263.01	0.00999999999999999\\
264.01	0.00999999999999999\\
265.01	0.00999999999999999\\
266.01	0.00999999999999999\\
267.01	0.00999999999999999\\
268.01	0.00999999999999999\\
269.01	0.00999999999999999\\
270.01	0.00999999999999999\\
271.01	0.00999999999999999\\
272.01	0.00999999999999999\\
273.01	0.00999999999999999\\
274.01	0.00999999999999999\\
275.01	0.00999999999999999\\
276.01	0.00999999999999999\\
277.01	0.00999999999999999\\
278.01	0.00999999999999999\\
279.01	0.00999999999999999\\
280.01	0.00999999999999999\\
281.01	0.00999999999999999\\
282.01	0.00999999999999999\\
283.01	0.00999999999999999\\
284.01	0.00999999999999999\\
285.01	0.00999999999999999\\
286.01	0.00999999999999999\\
287.01	0.00999999999999999\\
288.01	0.00999999999999999\\
289.01	0.00999999999999999\\
290.01	0.00999999999999999\\
291.01	0.00999999999999999\\
292.01	0.00999999999999999\\
293.01	0.00999999999999999\\
294.01	0.00999999999999999\\
295.01	0.00999999999999999\\
296.01	0.00999999999999999\\
297.01	0.00999999999999999\\
298.01	0.00999999999999999\\
299.01	0.00999999999999999\\
300.01	0.00999999999999999\\
301.01	0.00999999999999999\\
302.01	0.00999999999999999\\
303.01	0.00999999999999999\\
304.01	0.00999999999999999\\
305.01	0.00999999999999999\\
306.01	0.00999999999999999\\
307.01	0.00999999999999999\\
308.01	0.00999999999999999\\
309.01	0.00999999999999999\\
310.01	0.00999999999999999\\
311.01	0.00999999999999999\\
312.01	0.00999999999999999\\
313.01	0.00999999999999999\\
314.01	0.00999999999999999\\
315.01	0.00999999999999999\\
316.01	0.00999999999999999\\
317.01	0.00999999999999999\\
318.01	0.00999999999999999\\
319.01	0.00999999999999999\\
320.01	0.00999999999999999\\
321.01	0.00999999999999999\\
322.01	0.00999999999999999\\
323.01	0.00999999999999999\\
324.01	0.00999999999999999\\
325.01	0.00999999999999999\\
326.01	0.00999999999999999\\
327.01	0.00999999999999999\\
328.01	0.00999999999999999\\
329.01	0.00999999999999999\\
330.01	0.00999999999999999\\
331.01	0.00999999999999999\\
332.01	0.00999999999999999\\
333.01	0.00999999999999999\\
334.01	0.00999999999999999\\
335.01	0.00999999999999999\\
336.01	0.00999999999999999\\
337.01	0.00999999999999999\\
338.01	0.00999999999999999\\
339.01	0.00999999999999999\\
340.01	0.00999999999999999\\
341.01	0.00999999999999999\\
342.01	0.00999999999999999\\
343.01	0.00999999999999999\\
344.01	0.00999999999999999\\
345.01	0.00999999999999999\\
346.01	0.00999999999999999\\
347.01	0.00999999999999999\\
348.01	0.00999999999999999\\
349.01	0.00999999999999999\\
350.01	0.00999999999999999\\
351.01	0.00999999999999999\\
352.01	0.00999999999999999\\
353.01	0.00999999999999999\\
354.01	0.00999999999999999\\
355.01	0.00999999999999999\\
356.01	0.00999999999999999\\
357.01	0.00999999999999999\\
358.01	0.00999999999999999\\
359.01	0.00999999999999999\\
360.01	0.00999999999999999\\
361.01	0.00999999999999999\\
362.01	0.00999999999999999\\
363.01	0.00999999999999999\\
364.01	0.00999999999999999\\
365.01	0.00999999999999999\\
366.01	0.00999999999999999\\
367.01	0.00999999999999999\\
368.01	0.00999999999999999\\
369.01	0.00999999999999999\\
370.01	0.00999999999999999\\
371.01	0.00999999999999999\\
372.01	0.00999999999999999\\
373.01	0.00999999999999999\\
374.01	0.00999999999999999\\
375.01	0.00999999999999999\\
376.01	0.00999999999999999\\
377.01	0.00999999999999999\\
378.01	0.00999999999999999\\
379.01	0.00999999999999999\\
380.01	0.00999999999999999\\
381.01	0.00999999999999999\\
382.01	0.00999999999999999\\
383.01	0.00999999999999999\\
384.01	0.00999999999999999\\
385.01	0.00999999999999999\\
386.01	0.00999999999999999\\
387.01	0.00999999999999999\\
388.01	0.00999999999999999\\
389.01	0.00999999999999999\\
390.01	0.00999999999999999\\
391.01	0.00999999999999999\\
392.01	0.00999999999999999\\
393.01	0.00999999999999999\\
394.01	0.00999999999999999\\
395.01	0.00999999999999999\\
396.01	0.00999999999999999\\
397.01	0.00999999999999999\\
398.01	0.00999999999999999\\
399.01	0.00999999999999999\\
400.01	0.00999999999999999\\
401.01	0.00999999999999999\\
402.01	0.00999999999999999\\
403.01	0.00999999999999999\\
404.01	0.00999999999999999\\
405.01	0.00999999999999999\\
406.01	0.00999999999999999\\
407.01	0.00999999999999999\\
408.01	0.00999999999999999\\
409.01	0.00999999999999999\\
410.01	0.00999999999999999\\
411.01	0.00999999999999999\\
412.01	0.00999999999999999\\
413.01	0.00999999999999999\\
414.01	0.00999999999999999\\
415.01	0.00999999999999999\\
416.01	0.00999999999999999\\
417.01	0.00999999999999999\\
418.01	0.00999999999999999\\
419.01	0.00999999999999999\\
420.01	0.00999999999999999\\
421.01	0.00999999999999999\\
422.01	0.00999999999999999\\
423.01	0.00999999999999999\\
424.01	0.00999999999999999\\
425.01	0.00999999999999999\\
426.01	0.00999999999999999\\
427.01	0.00999999999999999\\
428.01	0.00999999999999999\\
429.01	0.00999999999999999\\
430.01	0.00999999999999999\\
431.01	0.00999999999999999\\
432.01	0.00999999999999999\\
433.01	0.00999999999999999\\
434.01	0.00999999999999999\\
435.01	0.00999999999999999\\
436.01	0.00999999999999999\\
437.01	0.00999999999999999\\
438.01	0.00999999999999999\\
439.01	0.00999999999999999\\
440.01	0.00999999999999999\\
441.01	0.00999999999999999\\
442.01	0.00999999999999999\\
443.01	0.00999999999999999\\
444.01	0.00999999999999999\\
445.01	0.00999999999999999\\
446.01	0.00999999999999999\\
447.01	0.00999999999999999\\
448.01	0.00999999999999999\\
449.01	0.00999999999999999\\
450.01	0.00999999999999999\\
451.01	0.00999999999999999\\
452.01	0.00999999999999999\\
453.01	0.00999999999999999\\
454.01	0.00999999999999999\\
455.01	0.00999999999999999\\
456.01	0.00999999999999999\\
457.01	0.00999999999999999\\
458.01	0.00999999999999999\\
459.01	0.00999999999999999\\
460.01	0.00999999999999999\\
461.01	0.00999999999999999\\
462.01	0.00999999999999999\\
463.01	0.00999999999999999\\
464.01	0.00999999999999999\\
465.01	0.00999999999999999\\
466.01	0.00999999999999999\\
467.01	0.00999999999999999\\
468.01	0.00999999999999999\\
469.01	0.00999999999999999\\
470.01	0.00999999999999999\\
471.01	0.00999999999999999\\
472.01	0.00999999999999999\\
473.01	0.00999999999999999\\
474.01	0.00999999999999999\\
475.01	0.00999999999999999\\
476.01	0.00999999999999999\\
477.01	0.00999999999999999\\
478.01	0.00999999999999999\\
479.01	0.00999999999999999\\
480.01	0.00999999999999999\\
481.01	0.00999999999999999\\
482.01	0.00999999999999999\\
483.01	0.00999999999999999\\
484.01	0.00999999999999999\\
485.01	0.00999999999999999\\
486.01	0.00999999999999999\\
487.01	0.00999999999999999\\
488.01	0.00999999999999999\\
489.01	0.00999999999999999\\
490.01	0.00999999999999999\\
491.01	0.00999999999999999\\
492.01	0.00999999999999999\\
493.01	0.00999999999999999\\
494.01	0.00999999999999999\\
495.01	0.00999999999999999\\
496.01	0.00999999999999999\\
497.01	0.00999999999999999\\
498.01	0.00999999999999999\\
499.01	0.00999999999999999\\
500.01	0.00999999999999999\\
501.01	0.00999999999999999\\
502.01	0.00999999999999999\\
503.01	0.00999999999999999\\
504.01	0.00999999999999999\\
505.01	0.00999999999999999\\
506.01	0.00999999999999999\\
507.01	0.00999999999999999\\
508.01	0.00999999999999999\\
509.01	0.00999999999999999\\
510.01	0.00999999999999999\\
511.01	0.00999999999999999\\
512.01	0.00999999999999999\\
513.01	0.00999999999999999\\
514.01	0.00999999999999999\\
515.01	0.00999999999999999\\
516.01	0.00999999999999999\\
517.01	0.00999999999999999\\
518.01	0.00999999999999999\\
519.01	0.00999999999999999\\
520.01	0.00999999999999999\\
521.01	0.00999999999999999\\
522.01	0.00999999999999999\\
523.01	0.00999999999999999\\
524.01	0.00999999999999999\\
525.01	0.00999999999999999\\
526.01	0.00999999999999999\\
527.01	0.00999999999999999\\
528.01	0.00999999999999999\\
529.01	0.00999999999999999\\
530.01	0.00999999999999999\\
531.01	0.00999999999999999\\
532.01	0.00999999999999999\\
533.01	0.00999999999999999\\
534.01	0.00999999999999999\\
535.01	0.00999999999999999\\
536.01	0.00999999999999999\\
537.01	0.00999999999999999\\
538.01	0.00999999999999999\\
539.01	0.00999999999999999\\
540.01	0.00999999999999999\\
541.01	0.00999999999999999\\
542.01	0.00999999999999999\\
543.01	0.00999999999999999\\
544.01	0.00999999999999999\\
545.01	0.00999999999999999\\
546.01	0.00999999999999999\\
547.01	0.00999999999999999\\
548.01	0.00999999999999999\\
549.01	0.00999999999999999\\
550.01	0.00999999999999999\\
551.01	0.00999999999999999\\
552.01	0.00999999999999999\\
553.01	0.00999999999999999\\
554.01	0.00999999999999999\\
555.01	0.00999999999999999\\
556.01	0.00999999999999999\\
557.01	0.00999999999999999\\
558.01	0.00999999999999999\\
559.01	0.00999999999999999\\
560.01	0.00999999999999999\\
561.01	0.00999999999999999\\
562.01	0.00999999999999999\\
563.01	0.00999999999999999\\
564.01	0.00999999999999999\\
565.01	0.00999999999999999\\
566.01	0.00999999999999999\\
567.01	0.00999999999999999\\
568.01	0.00999999999999999\\
569.01	0.00999999999999999\\
570.01	0.00999999999999999\\
571.01	0.00999999999999999\\
572.01	0.00999999999999999\\
573.01	0.00999999999999999\\
574.01	0.00999999999999999\\
575.01	0.00999999999999999\\
576.01	0.00999999999999999\\
577.01	0.00999999999999999\\
578.01	0.00999999999999999\\
579.01	0.00999999999999999\\
580.01	0.00999999999999999\\
581.01	0.00999999999999999\\
582.01	0.00999999999999999\\
583.01	0.00999999999999999\\
584.01	0.00999999999999999\\
585.01	0.00999999999999999\\
586.01	0.00999999999999999\\
587.01	0.00999999999999999\\
588.01	0.00999999999999999\\
589.01	0.00999999999999999\\
590.01	0.00999999999999999\\
591.01	0.00999999999999999\\
592.01	0.00999999999999999\\
593.01	0.00999999999999999\\
594.01	0.00999999999999999\\
595.01	0.00999999999999999\\
596.01	0.00999999999999999\\
597.01	0.00999999999999999\\
598.01	0.00999999999999999\\
599.01	0.00623513569405279\\
599.02	0.0061974735980932\\
599.03	0.00615944475597968\\
599.04	0.0061210455620573\\
599.05	0.00608227237521589\\
599.06	0.00604312151854133\\
599.07	0.00600358927896352\\
599.08	0.0059636719069008\\
599.09	0.00592336561590098\\
599.1	0.00588266658227871\\
599.11	0.00584157094474941\\
599.12	0.00580007480405952\\
599.13	0.00575817422261318\\
599.14	0.00571586522409518\\
599.15	0.00567314379309022\\
599.16	0.00563000587469847\\
599.17	0.00558644737414726\\
599.18	0.00554246415639909\\
599.19	0.00549805204575559\\
599.2	0.00545320682545778\\
599.21	0.00540792423728233\\
599.22	0.00536219998113371\\
599.23	0.00531602971463264\\
599.24	0.00526940905270014\\
599.25	0.00522233356713779\\
599.26	0.00517479878620363\\
599.27	0.00512680019418406\\
599.28	0.00507833323096144\\
599.29	0.00502939329157744\\
599.3	0.00497997572579218\\
599.31	0.00493007583763889\\
599.32	0.00487968888497432\\
599.33	0.00482881007902469\\
599.34	0.00477743458392716\\
599.35	0.00472555751626675\\
599.36	0.00467317394460885\\
599.37	0.00462027888902696\\
599.38	0.0045668673206259\\
599.39	0.0045129341610603\\
599.4	0.00445847428204831\\
599.41	0.00440348251095549\\
599.42	0.00434795363554375\\
599.43	0.00429188239249288\\
599.44	0.00423526346689991\\
599.45	0.00417809149177358\\
599.46	0.00412036104752388\\
599.47	0.00406206666144656\\
599.48	0.00400320280720258\\
599.49	0.00394376390429251\\
599.5	0.00388374431752568\\
599.51	0.00382313835648424\\
599.52	0.00376194027498192\\
599.53	0.00370014427051746\\
599.54	0.0036377444837227\\
599.55	0.00357473499780529\\
599.56	0.00351110983798593\\
599.57	0.00344686297093003\\
599.58	0.00338198830417389\\
599.59	0.00331647968554523\\
599.6	0.003250330902578\\
599.61	0.00318353568192148\\
599.62	0.00311608768874362\\
599.63	0.0030479805261285\\
599.64	0.00297920773446789\\
599.65	0.00290976279084682\\
599.66	0.0028396391084232\\
599.67	0.00276883003580131\\
599.68	0.00269732885639914\\
599.69	0.00262512878780953\\
599.7	0.00255222298115511\\
599.71	0.00247860452043686\\
599.72	0.0024042664218763\\
599.73	0.0023292016332512\\
599.74	0.00225340303322488\\
599.75	0.00217686343066881\\
599.76	0.00209957556397867\\
599.77	0.00202153210038367\\
599.78	0.00194272563524904\\
599.79	0.00186314869137186\\
599.8	0.00178279371826976\\
599.81	0.00170165309146283\\
599.82	0.00161971911174841\\
599.83	0.00153698400446879\\
599.84	0.00145343991877172\\
599.85	0.00136907892686372\\
599.86	0.00128389302325598\\
599.87	0.001197874124003\\
599.88	0.00111101406593363\\
599.89	0.00102330460587468\\
599.9	0.0009347374198669\\
599.91	0.000845304102373183\\
599.92	0.000754996165479164\\
599.93	0.000663805038085866\\
599.94	0.000571722065094472\\
599.95	0.000478738506583127\\
599.96	0.000384845536975636\\
599.97	0.000290034244202044\\
599.98	0.00019429562885097\\
599.99	9.76206033136556e-05\\
600	0\\
};
\addplot [color=blue!50!mycolor7,solid,forget plot]
  table[row sep=crcr]{%
0.01	0.00999999999999999\\
1.01	0.00999999999999999\\
2.01	0.00999999999999999\\
3.01	0.00999999999999999\\
4.01	0.00999999999999999\\
5.01	0.00999999999999999\\
6.01	0.00999999999999999\\
7.01	0.00999999999999999\\
8.01	0.00999999999999999\\
9.01	0.00999999999999999\\
10.01	0.00999999999999999\\
11.01	0.00999999999999999\\
12.01	0.00999999999999999\\
13.01	0.00999999999999999\\
14.01	0.00999999999999999\\
15.01	0.00999999999999999\\
16.01	0.00999999999999999\\
17.01	0.00999999999999999\\
18.01	0.00999999999999999\\
19.01	0.00999999999999999\\
20.01	0.00999999999999999\\
21.01	0.00999999999999999\\
22.01	0.00999999999999999\\
23.01	0.00999999999999999\\
24.01	0.00999999999999999\\
25.01	0.00999999999999999\\
26.01	0.00999999999999999\\
27.01	0.00999999999999999\\
28.01	0.00999999999999999\\
29.01	0.00999999999999999\\
30.01	0.00999999999999999\\
31.01	0.00999999999999999\\
32.01	0.00999999999999999\\
33.01	0.00999999999999999\\
34.01	0.00999999999999999\\
35.01	0.00999999999999999\\
36.01	0.00999999999999999\\
37.01	0.00999999999999999\\
38.01	0.00999999999999999\\
39.01	0.00999999999999999\\
40.01	0.00999999999999999\\
41.01	0.00999999999999999\\
42.01	0.00999999999999999\\
43.01	0.00999999999999999\\
44.01	0.00999999999999999\\
45.01	0.00999999999999999\\
46.01	0.00999999999999999\\
47.01	0.00999999999999999\\
48.01	0.00999999999999999\\
49.01	0.00999999999999999\\
50.01	0.00999999999999999\\
51.01	0.00999999999999999\\
52.01	0.00999999999999999\\
53.01	0.00999999999999999\\
54.01	0.00999999999999999\\
55.01	0.00999999999999999\\
56.01	0.00999999999999999\\
57.01	0.00999999999999999\\
58.01	0.00999999999999999\\
59.01	0.00999999999999999\\
60.01	0.00999999999999999\\
61.01	0.00999999999999999\\
62.01	0.00999999999999999\\
63.01	0.00999999999999999\\
64.01	0.00999999999999999\\
65.01	0.00999999999999999\\
66.01	0.00999999999999999\\
67.01	0.00999999999999999\\
68.01	0.00999999999999999\\
69.01	0.00999999999999999\\
70.01	0.00999999999999999\\
71.01	0.00999999999999999\\
72.01	0.00999999999999999\\
73.01	0.00999999999999999\\
74.01	0.00999999999999999\\
75.01	0.00999999999999999\\
76.01	0.00999999999999999\\
77.01	0.00999999999999999\\
78.01	0.00999999999999999\\
79.01	0.00999999999999999\\
80.01	0.00999999999999999\\
81.01	0.00999999999999999\\
82.01	0.00999999999999999\\
83.01	0.00999999999999999\\
84.01	0.00999999999999999\\
85.01	0.00999999999999999\\
86.01	0.00999999999999999\\
87.01	0.00999999999999999\\
88.01	0.00999999999999999\\
89.01	0.00999999999999999\\
90.01	0.00999999999999999\\
91.01	0.00999999999999999\\
92.01	0.00999999999999999\\
93.01	0.00999999999999999\\
94.01	0.00999999999999999\\
95.01	0.00999999999999999\\
96.01	0.00999999999999999\\
97.01	0.00999999999999999\\
98.01	0.00999999999999999\\
99.01	0.00999999999999999\\
100.01	0.00999999999999999\\
101.01	0.00999999999999999\\
102.01	0.00999999999999999\\
103.01	0.00999999999999999\\
104.01	0.00999999999999999\\
105.01	0.00999999999999999\\
106.01	0.00999999999999999\\
107.01	0.00999999999999999\\
108.01	0.00999999999999999\\
109.01	0.00999999999999999\\
110.01	0.00999999999999999\\
111.01	0.00999999999999999\\
112.01	0.00999999999999999\\
113.01	0.00999999999999999\\
114.01	0.00999999999999999\\
115.01	0.00999999999999999\\
116.01	0.00999999999999999\\
117.01	0.00999999999999999\\
118.01	0.00999999999999999\\
119.01	0.00999999999999999\\
120.01	0.00999999999999999\\
121.01	0.00999999999999999\\
122.01	0.00999999999999999\\
123.01	0.00999999999999999\\
124.01	0.00999999999999999\\
125.01	0.00999999999999999\\
126.01	0.00999999999999999\\
127.01	0.00999999999999999\\
128.01	0.00999999999999999\\
129.01	0.00999999999999999\\
130.01	0.00999999999999999\\
131.01	0.00999999999999999\\
132.01	0.00999999999999999\\
133.01	0.00999999999999999\\
134.01	0.00999999999999999\\
135.01	0.00999999999999999\\
136.01	0.00999999999999999\\
137.01	0.00999999999999999\\
138.01	0.00999999999999999\\
139.01	0.00999999999999999\\
140.01	0.00999999999999999\\
141.01	0.00999999999999999\\
142.01	0.00999999999999999\\
143.01	0.00999999999999999\\
144.01	0.00999999999999999\\
145.01	0.00999999999999999\\
146.01	0.00999999999999999\\
147.01	0.00999999999999999\\
148.01	0.00999999999999999\\
149.01	0.00999999999999999\\
150.01	0.00999999999999999\\
151.01	0.00999999999999999\\
152.01	0.00999999999999999\\
153.01	0.00999999999999999\\
154.01	0.00999999999999999\\
155.01	0.00999999999999999\\
156.01	0.00999999999999999\\
157.01	0.00999999999999999\\
158.01	0.00999999999999999\\
159.01	0.00999999999999999\\
160.01	0.00999999999999999\\
161.01	0.00999999999999999\\
162.01	0.00999999999999999\\
163.01	0.00999999999999999\\
164.01	0.00999999999999999\\
165.01	0.00999999999999999\\
166.01	0.00999999999999999\\
167.01	0.00999999999999999\\
168.01	0.00999999999999999\\
169.01	0.00999999999999999\\
170.01	0.00999999999999999\\
171.01	0.00999999999999999\\
172.01	0.00999999999999999\\
173.01	0.00999999999999999\\
174.01	0.00999999999999999\\
175.01	0.00999999999999999\\
176.01	0.00999999999999999\\
177.01	0.00999999999999999\\
178.01	0.00999999999999999\\
179.01	0.00999999999999999\\
180.01	0.00999999999999999\\
181.01	0.00999999999999999\\
182.01	0.00999999999999999\\
183.01	0.00999999999999999\\
184.01	0.00999999999999999\\
185.01	0.00999999999999999\\
186.01	0.00999999999999999\\
187.01	0.00999999999999999\\
188.01	0.00999999999999999\\
189.01	0.00999999999999999\\
190.01	0.00999999999999999\\
191.01	0.00999999999999999\\
192.01	0.00999999999999999\\
193.01	0.00999999999999999\\
194.01	0.00999999999999999\\
195.01	0.00999999999999999\\
196.01	0.00999999999999999\\
197.01	0.00999999999999999\\
198.01	0.00999999999999999\\
199.01	0.00999999999999999\\
200.01	0.00999999999999999\\
201.01	0.00999999999999999\\
202.01	0.00999999999999999\\
203.01	0.00999999999999999\\
204.01	0.00999999999999999\\
205.01	0.00999999999999999\\
206.01	0.00999999999999999\\
207.01	0.00999999999999999\\
208.01	0.00999999999999999\\
209.01	0.00999999999999999\\
210.01	0.00999999999999999\\
211.01	0.00999999999999999\\
212.01	0.00999999999999999\\
213.01	0.00999999999999999\\
214.01	0.00999999999999999\\
215.01	0.00999999999999999\\
216.01	0.00999999999999999\\
217.01	0.00999999999999999\\
218.01	0.00999999999999999\\
219.01	0.00999999999999999\\
220.01	0.00999999999999999\\
221.01	0.00999999999999999\\
222.01	0.00999999999999999\\
223.01	0.00999999999999999\\
224.01	0.00999999999999999\\
225.01	0.00999999999999999\\
226.01	0.00999999999999999\\
227.01	0.00999999999999999\\
228.01	0.00999999999999999\\
229.01	0.00999999999999999\\
230.01	0.00999999999999999\\
231.01	0.00999999999999999\\
232.01	0.00999999999999999\\
233.01	0.00999999999999999\\
234.01	0.00999999999999999\\
235.01	0.00999999999999999\\
236.01	0.00999999999999999\\
237.01	0.00999999999999999\\
238.01	0.00999999999999999\\
239.01	0.00999999999999999\\
240.01	0.00999999999999999\\
241.01	0.00999999999999999\\
242.01	0.00999999999999999\\
243.01	0.00999999999999999\\
244.01	0.00999999999999999\\
245.01	0.00999999999999999\\
246.01	0.00999999999999999\\
247.01	0.00999999999999999\\
248.01	0.00999999999999999\\
249.01	0.00999999999999999\\
250.01	0.00999999999999999\\
251.01	0.00999999999999999\\
252.01	0.00999999999999999\\
253.01	0.00999999999999999\\
254.01	0.00999999999999999\\
255.01	0.00999999999999999\\
256.01	0.00999999999999999\\
257.01	0.00999999999999999\\
258.01	0.00999999999999999\\
259.01	0.00999999999999999\\
260.01	0.00999999999999999\\
261.01	0.00999999999999999\\
262.01	0.00999999999999999\\
263.01	0.00999999999999999\\
264.01	0.00999999999999999\\
265.01	0.00999999999999999\\
266.01	0.00999999999999999\\
267.01	0.00999999999999999\\
268.01	0.00999999999999999\\
269.01	0.00999999999999999\\
270.01	0.00999999999999999\\
271.01	0.00999999999999999\\
272.01	0.00999999999999999\\
273.01	0.00999999999999999\\
274.01	0.00999999999999999\\
275.01	0.00999999999999999\\
276.01	0.00999999999999999\\
277.01	0.00999999999999999\\
278.01	0.00999999999999999\\
279.01	0.00999999999999999\\
280.01	0.00999999999999999\\
281.01	0.00999999999999999\\
282.01	0.00999999999999999\\
283.01	0.00999999999999999\\
284.01	0.00999999999999999\\
285.01	0.00999999999999999\\
286.01	0.00999999999999999\\
287.01	0.00999999999999999\\
288.01	0.00999999999999999\\
289.01	0.00999999999999999\\
290.01	0.00999999999999999\\
291.01	0.00999999999999999\\
292.01	0.00999999999999999\\
293.01	0.00999999999999999\\
294.01	0.00999999999999999\\
295.01	0.00999999999999999\\
296.01	0.00999999999999999\\
297.01	0.00999999999999999\\
298.01	0.00999999999999999\\
299.01	0.00999999999999999\\
300.01	0.00999999999999999\\
301.01	0.00999999999999999\\
302.01	0.00999999999999999\\
303.01	0.00999999999999999\\
304.01	0.00999999999999999\\
305.01	0.00999999999999999\\
306.01	0.00999999999999999\\
307.01	0.00999999999999999\\
308.01	0.00999999999999999\\
309.01	0.00999999999999999\\
310.01	0.00999999999999999\\
311.01	0.00999999999999999\\
312.01	0.00999999999999999\\
313.01	0.00999999999999999\\
314.01	0.00999999999999999\\
315.01	0.00999999999999999\\
316.01	0.00999999999999999\\
317.01	0.00999999999999999\\
318.01	0.00999999999999999\\
319.01	0.00999999999999999\\
320.01	0.00999999999999999\\
321.01	0.00999999999999999\\
322.01	0.00999999999999999\\
323.01	0.00999999999999999\\
324.01	0.00999999999999999\\
325.01	0.00999999999999999\\
326.01	0.00999999999999999\\
327.01	0.00999999999999999\\
328.01	0.00999999999999999\\
329.01	0.00999999999999999\\
330.01	0.00999999999999999\\
331.01	0.00999999999999999\\
332.01	0.00999999999999999\\
333.01	0.00999999999999999\\
334.01	0.00999999999999999\\
335.01	0.00999999999999999\\
336.01	0.00999999999999999\\
337.01	0.00999999999999999\\
338.01	0.00999999999999999\\
339.01	0.00999999999999999\\
340.01	0.00999999999999999\\
341.01	0.00999999999999999\\
342.01	0.00999999999999999\\
343.01	0.00999999999999999\\
344.01	0.00999999999999999\\
345.01	0.00999999999999999\\
346.01	0.00999999999999999\\
347.01	0.00999999999999999\\
348.01	0.00999999999999999\\
349.01	0.00999999999999999\\
350.01	0.00999999999999999\\
351.01	0.00999999999999999\\
352.01	0.00999999999999999\\
353.01	0.00999999999999999\\
354.01	0.00999999999999999\\
355.01	0.00999999999999999\\
356.01	0.00999999999999999\\
357.01	0.00999999999999999\\
358.01	0.00999999999999999\\
359.01	0.00999999999999999\\
360.01	0.00999999999999999\\
361.01	0.00999999999999999\\
362.01	0.00999999999999999\\
363.01	0.00999999999999999\\
364.01	0.00999999999999999\\
365.01	0.00999999999999999\\
366.01	0.00999999999999999\\
367.01	0.00999999999999999\\
368.01	0.00999999999999999\\
369.01	0.00999999999999999\\
370.01	0.00999999999999999\\
371.01	0.00999999999999999\\
372.01	0.00999999999999999\\
373.01	0.00999999999999999\\
374.01	0.00999999999999999\\
375.01	0.00999999999999999\\
376.01	0.00999999999999999\\
377.01	0.00999999999999999\\
378.01	0.00999999999999999\\
379.01	0.00999999999999999\\
380.01	0.00999999999999999\\
381.01	0.00999999999999999\\
382.01	0.00999999999999999\\
383.01	0.00999999999999999\\
384.01	0.00999999999999999\\
385.01	0.00999999999999999\\
386.01	0.00999999999999999\\
387.01	0.00999999999999999\\
388.01	0.00999999999999999\\
389.01	0.00999999999999999\\
390.01	0.00999999999999999\\
391.01	0.00999999999999999\\
392.01	0.00999999999999999\\
393.01	0.00999999999999999\\
394.01	0.00999999999999999\\
395.01	0.00999999999999999\\
396.01	0.00999999999999999\\
397.01	0.00999999999999999\\
398.01	0.00999999999999999\\
399.01	0.00999999999999999\\
400.01	0.00999999999999999\\
401.01	0.00999999999999999\\
402.01	0.00999999999999999\\
403.01	0.00999999999999999\\
404.01	0.00999999999999999\\
405.01	0.00999999999999999\\
406.01	0.00999999999999999\\
407.01	0.00999999999999999\\
408.01	0.00999999999999999\\
409.01	0.00999999999999999\\
410.01	0.00999999999999999\\
411.01	0.00999999999999999\\
412.01	0.00999999999999999\\
413.01	0.00999999999999999\\
414.01	0.00999999999999999\\
415.01	0.00999999999999999\\
416.01	0.00999999999999999\\
417.01	0.00999999999999999\\
418.01	0.00999999999999999\\
419.01	0.00999999999999999\\
420.01	0.00999999999999999\\
421.01	0.00999999999999999\\
422.01	0.00999999999999999\\
423.01	0.00999999999999999\\
424.01	0.00999999999999999\\
425.01	0.00999999999999999\\
426.01	0.00999999999999999\\
427.01	0.00999999999999999\\
428.01	0.00999999999999999\\
429.01	0.00999999999999999\\
430.01	0.00999999999999999\\
431.01	0.00999999999999999\\
432.01	0.00999999999999999\\
433.01	0.00999999999999999\\
434.01	0.00999999999999999\\
435.01	0.00999999999999999\\
436.01	0.00999999999999999\\
437.01	0.00999999999999999\\
438.01	0.00999999999999999\\
439.01	0.00999999999999999\\
440.01	0.00999999999999999\\
441.01	0.00999999999999999\\
442.01	0.00999999999999999\\
443.01	0.00999999999999999\\
444.01	0.00999999999999999\\
445.01	0.00999999999999999\\
446.01	0.00999999999999999\\
447.01	0.00999999999999999\\
448.01	0.00999999999999999\\
449.01	0.00999999999999999\\
450.01	0.00999999999999999\\
451.01	0.00999999999999999\\
452.01	0.00999999999999999\\
453.01	0.00999999999999999\\
454.01	0.00999999999999999\\
455.01	0.00999999999999999\\
456.01	0.00999999999999999\\
457.01	0.00999999999999999\\
458.01	0.00999999999999999\\
459.01	0.00999999999999999\\
460.01	0.00999999999999999\\
461.01	0.00999999999999999\\
462.01	0.00999999999999999\\
463.01	0.00999999999999999\\
464.01	0.00999999999999999\\
465.01	0.00999999999999999\\
466.01	0.00999999999999999\\
467.01	0.00999999999999999\\
468.01	0.00999999999999999\\
469.01	0.00999999999999999\\
470.01	0.00999999999999999\\
471.01	0.00999999999999999\\
472.01	0.00999999999999999\\
473.01	0.00999999999999999\\
474.01	0.00999999999999999\\
475.01	0.00999999999999999\\
476.01	0.00999999999999999\\
477.01	0.00999999999999999\\
478.01	0.00999999999999999\\
479.01	0.00999999999999999\\
480.01	0.00999999999999999\\
481.01	0.00999999999999999\\
482.01	0.00999999999999999\\
483.01	0.00999999999999999\\
484.01	0.00999999999999999\\
485.01	0.00999999999999999\\
486.01	0.00999999999999999\\
487.01	0.00999999999999999\\
488.01	0.00999999999999999\\
489.01	0.00999999999999999\\
490.01	0.00999999999999999\\
491.01	0.00999999999999999\\
492.01	0.00999999999999999\\
493.01	0.00999999999999999\\
494.01	0.00999999999999999\\
495.01	0.00999999999999999\\
496.01	0.00999999999999999\\
497.01	0.00999999999999999\\
498.01	0.00999999999999999\\
499.01	0.00999999999999999\\
500.01	0.00999999999999999\\
501.01	0.00999999999999999\\
502.01	0.00999999999999999\\
503.01	0.00999999999999999\\
504.01	0.00999999999999999\\
505.01	0.00999999999999999\\
506.01	0.00999999999999999\\
507.01	0.00999999999999999\\
508.01	0.00999999999999999\\
509.01	0.00999999999999999\\
510.01	0.00999999999999999\\
511.01	0.00999999999999999\\
512.01	0.00999999999999999\\
513.01	0.00999999999999999\\
514.01	0.00999999999999999\\
515.01	0.00999999999999999\\
516.01	0.00999999999999999\\
517.01	0.00999999999999999\\
518.01	0.00999999999999999\\
519.01	0.00999999999999999\\
520.01	0.00999999999999999\\
521.01	0.00999999999999999\\
522.01	0.00999999999999999\\
523.01	0.00999999999999999\\
524.01	0.00999999999999999\\
525.01	0.00999999999999999\\
526.01	0.00999999999999999\\
527.01	0.00999999999999999\\
528.01	0.00999999999999999\\
529.01	0.00999999999999999\\
530.01	0.00999999999999999\\
531.01	0.00999999999999999\\
532.01	0.00999999999999999\\
533.01	0.00999999999999999\\
534.01	0.00999999999999999\\
535.01	0.00999999999999999\\
536.01	0.00999999999999999\\
537.01	0.00999999999999999\\
538.01	0.00999999999999999\\
539.01	0.00999999999999999\\
540.01	0.00999999999999999\\
541.01	0.00999999999999999\\
542.01	0.00999999999999999\\
543.01	0.00999999999999999\\
544.01	0.00999999999999999\\
545.01	0.00999999999999999\\
546.01	0.00999999999999999\\
547.01	0.00999999999999999\\
548.01	0.00999999999999999\\
549.01	0.00999999999999999\\
550.01	0.00999999999999999\\
551.01	0.00999999999999999\\
552.01	0.00999999999999999\\
553.01	0.00999999999999999\\
554.01	0.00999999999999999\\
555.01	0.00999999999999999\\
556.01	0.00999999999999999\\
557.01	0.00999999999999999\\
558.01	0.00999999999999999\\
559.01	0.00999999999999999\\
560.01	0.00999999999999999\\
561.01	0.00999999999999999\\
562.01	0.00999999999999999\\
563.01	0.00999999999999999\\
564.01	0.00999999999999999\\
565.01	0.00999999999999999\\
566.01	0.00999999999999999\\
567.01	0.00999999999999999\\
568.01	0.00999999999999999\\
569.01	0.00999999999999999\\
570.01	0.00999999999999999\\
571.01	0.00999999999999999\\
572.01	0.00999999999999999\\
573.01	0.00999999999999999\\
574.01	0.00999999999999999\\
575.01	0.00999999999999999\\
576.01	0.00999999999999999\\
577.01	0.00999999999999999\\
578.01	0.00999999999999999\\
579.01	0.00999999999999999\\
580.01	0.00999999999999999\\
581.01	0.00999999999999999\\
582.01	0.00999999999999999\\
583.01	0.00999999999999999\\
584.01	0.00999999999999999\\
585.01	0.00999999999999999\\
586.01	0.00999999999999999\\
587.01	0.00999999999999999\\
588.01	0.00999999999999999\\
589.01	0.00999999999999999\\
590.01	0.00999999999999999\\
591.01	0.00999999999999999\\
592.01	0.00999999999999999\\
593.01	0.00999999999999999\\
594.01	0.00999999999999999\\
595.01	0.00999999999999999\\
596.01	0.00999999999999999\\
597.01	0.00999999999999999\\
598.01	0.00999999999999999\\
599.01	0.00623513569652336\\
599.02	0.00619747360025308\\
599.03	0.00615944475790929\\
599.04	0.00612104556381531\\
599.05	0.0060822723768421\\
599.06	0.00604312152006039\\
599.07	0.00600358928038925\\
599.08	0.00596367190824141\\
599.09	0.00592336561716329\\
599.1	0.00588266658346843\\
599.11	0.00584157094587136\\
599.12	0.00580007480511781\\
599.13	0.00575817422361138\\
599.14	0.00571586522503648\\
599.15	0.00567314379397755\\
599.16	0.00563000587553454\\
599.17	0.00558644737493464\\
599.18	0.00554246415714017\\
599.19	0.00549805204645267\\
599.2	0.00545320682611304\\
599.21	0.0054079242378978\\
599.22	0.00536219998171136\\
599.23	0.00531602971517431\\
599.24	0.0052694090532076\\
599.25	0.0052223335676127\\
599.26	0.00517479878664759\\
599.27	0.0051268001945986\\
599.28	0.00507833323134802\\
599.29	0.00502939329193748\\
599.3	0.004979975726127\\
599.31	0.00493007583794978\\
599.32	0.00487968888526254\\
599.33	0.00482881007929144\\
599.34	0.00477743458417358\\
599.35	0.00472555751649399\\
599.36	0.00467317394481798\\
599.37	0.00462027888921903\\
599.38	0.00456686732080193\\
599.39	0.00451293416122126\\
599.4	0.00445847428219516\\
599.41	0.00440348251108914\\
599.42	0.00434795363566509\\
599.43	0.00429188239260275\\
599.44	0.00423526346699912\\
599.45	0.00417809149186292\\
599.46	0.0041203610476041\\
599.47	0.00406206666151837\\
599.48	0.00400320280726666\\
599.49	0.0039437639043495\\
599.5	0.00388374431757618\\
599.51	0.00382313835652885\\
599.52	0.00376194027502117\\
599.53	0.00370014427055186\\
599.54	0.00363774448375272\\
599.55	0.00357473499783138\\
599.56	0.0035111098380085\\
599.57	0.00344686297094946\\
599.58	0.00338198830419054\\
599.59	0.00331647968555942\\
599.6	0.00325033090259001\\
599.61	0.00318353568193159\\
599.62	0.00311608768875208\\
599.63	0.00304798052613553\\
599.64	0.00297920773447369\\
599.65	0.00290976279085158\\
599.66	0.00283963910842707\\
599.67	0.00276883003580444\\
599.68	0.00269732885640163\\
599.69	0.0026251287878115\\
599.7	0.00255222298115666\\
599.71	0.00247860452043806\\
599.72	0.00240426642187721\\
599.73	0.00232920163325189\\
599.74	0.0022534030332254\\
599.75	0.00217686343066919\\
599.76	0.00209957556397894\\
599.77	0.00202153210038386\\
599.78	0.00194272563524917\\
599.79	0.00186314869137194\\
599.8	0.00178279371826981\\
599.81	0.00170165309146286\\
599.82	0.00161971911174843\\
599.83	0.0015369840044688\\
599.84	0.00145343991877172\\
599.85	0.00136907892686372\\
599.86	0.00128389302325598\\
599.87	0.00119787412400299\\
599.88	0.00111101406593363\\
599.89	0.00102330460587468\\
599.9	0.0009347374198669\\
599.91	0.000845304102373183\\
599.92	0.000754996165479166\\
599.93	0.000663805038085868\\
599.94	0.000571722065094473\\
599.95	0.000478738506583129\\
599.96	0.000384845536975638\\
599.97	0.000290034244202042\\
599.98	0.00019429562885097\\
599.99	9.76206033136556e-05\\
600	0\\
};
\addplot [color=blue!40!mycolor9,solid,forget plot]
  table[row sep=crcr]{%
0.01	0.01\\
1.01	0.01\\
2.01	0.01\\
3.01	0.01\\
4.01	0.01\\
5.01	0.01\\
6.01	0.01\\
7.01	0.01\\
8.01	0.01\\
9.01	0.01\\
10.01	0.01\\
11.01	0.01\\
12.01	0.01\\
13.01	0.01\\
14.01	0.01\\
15.01	0.01\\
16.01	0.01\\
17.01	0.01\\
18.01	0.01\\
19.01	0.01\\
20.01	0.01\\
21.01	0.01\\
22.01	0.01\\
23.01	0.01\\
24.01	0.01\\
25.01	0.01\\
26.01	0.01\\
27.01	0.01\\
28.01	0.01\\
29.01	0.01\\
30.01	0.01\\
31.01	0.01\\
32.01	0.01\\
33.01	0.01\\
34.01	0.01\\
35.01	0.01\\
36.01	0.01\\
37.01	0.01\\
38.01	0.01\\
39.01	0.01\\
40.01	0.01\\
41.01	0.01\\
42.01	0.01\\
43.01	0.01\\
44.01	0.01\\
45.01	0.01\\
46.01	0.01\\
47.01	0.01\\
48.01	0.01\\
49.01	0.01\\
50.01	0.01\\
51.01	0.01\\
52.01	0.01\\
53.01	0.01\\
54.01	0.01\\
55.01	0.01\\
56.01	0.01\\
57.01	0.01\\
58.01	0.01\\
59.01	0.01\\
60.01	0.01\\
61.01	0.01\\
62.01	0.01\\
63.01	0.01\\
64.01	0.01\\
65.01	0.01\\
66.01	0.01\\
67.01	0.01\\
68.01	0.01\\
69.01	0.01\\
70.01	0.01\\
71.01	0.01\\
72.01	0.01\\
73.01	0.01\\
74.01	0.01\\
75.01	0.01\\
76.01	0.01\\
77.01	0.01\\
78.01	0.01\\
79.01	0.01\\
80.01	0.01\\
81.01	0.01\\
82.01	0.01\\
83.01	0.01\\
84.01	0.01\\
85.01	0.01\\
86.01	0.01\\
87.01	0.01\\
88.01	0.01\\
89.01	0.01\\
90.01	0.01\\
91.01	0.01\\
92.01	0.01\\
93.01	0.01\\
94.01	0.01\\
95.01	0.01\\
96.01	0.01\\
97.01	0.01\\
98.01	0.01\\
99.01	0.01\\
100.01	0.01\\
101.01	0.01\\
102.01	0.01\\
103.01	0.01\\
104.01	0.01\\
105.01	0.01\\
106.01	0.01\\
107.01	0.01\\
108.01	0.01\\
109.01	0.01\\
110.01	0.01\\
111.01	0.01\\
112.01	0.01\\
113.01	0.01\\
114.01	0.01\\
115.01	0.01\\
116.01	0.01\\
117.01	0.01\\
118.01	0.01\\
119.01	0.01\\
120.01	0.01\\
121.01	0.01\\
122.01	0.01\\
123.01	0.01\\
124.01	0.01\\
125.01	0.01\\
126.01	0.01\\
127.01	0.01\\
128.01	0.01\\
129.01	0.01\\
130.01	0.01\\
131.01	0.01\\
132.01	0.01\\
133.01	0.01\\
134.01	0.01\\
135.01	0.01\\
136.01	0.01\\
137.01	0.01\\
138.01	0.01\\
139.01	0.01\\
140.01	0.01\\
141.01	0.01\\
142.01	0.01\\
143.01	0.01\\
144.01	0.01\\
145.01	0.01\\
146.01	0.01\\
147.01	0.01\\
148.01	0.01\\
149.01	0.01\\
150.01	0.01\\
151.01	0.01\\
152.01	0.01\\
153.01	0.01\\
154.01	0.01\\
155.01	0.01\\
156.01	0.01\\
157.01	0.01\\
158.01	0.01\\
159.01	0.01\\
160.01	0.01\\
161.01	0.01\\
162.01	0.01\\
163.01	0.01\\
164.01	0.01\\
165.01	0.01\\
166.01	0.01\\
167.01	0.01\\
168.01	0.01\\
169.01	0.01\\
170.01	0.01\\
171.01	0.01\\
172.01	0.01\\
173.01	0.01\\
174.01	0.01\\
175.01	0.01\\
176.01	0.01\\
177.01	0.01\\
178.01	0.01\\
179.01	0.01\\
180.01	0.01\\
181.01	0.01\\
182.01	0.01\\
183.01	0.01\\
184.01	0.01\\
185.01	0.01\\
186.01	0.01\\
187.01	0.01\\
188.01	0.01\\
189.01	0.01\\
190.01	0.01\\
191.01	0.01\\
192.01	0.01\\
193.01	0.01\\
194.01	0.01\\
195.01	0.01\\
196.01	0.01\\
197.01	0.01\\
198.01	0.01\\
199.01	0.01\\
200.01	0.01\\
201.01	0.01\\
202.01	0.01\\
203.01	0.01\\
204.01	0.01\\
205.01	0.01\\
206.01	0.01\\
207.01	0.01\\
208.01	0.01\\
209.01	0.01\\
210.01	0.01\\
211.01	0.01\\
212.01	0.01\\
213.01	0.01\\
214.01	0.01\\
215.01	0.01\\
216.01	0.01\\
217.01	0.01\\
218.01	0.01\\
219.01	0.01\\
220.01	0.01\\
221.01	0.01\\
222.01	0.01\\
223.01	0.01\\
224.01	0.01\\
225.01	0.01\\
226.01	0.01\\
227.01	0.01\\
228.01	0.01\\
229.01	0.01\\
230.01	0.01\\
231.01	0.01\\
232.01	0.01\\
233.01	0.01\\
234.01	0.01\\
235.01	0.01\\
236.01	0.01\\
237.01	0.01\\
238.01	0.01\\
239.01	0.01\\
240.01	0.01\\
241.01	0.01\\
242.01	0.01\\
243.01	0.01\\
244.01	0.01\\
245.01	0.01\\
246.01	0.01\\
247.01	0.01\\
248.01	0.01\\
249.01	0.01\\
250.01	0.01\\
251.01	0.01\\
252.01	0.01\\
253.01	0.01\\
254.01	0.01\\
255.01	0.01\\
256.01	0.01\\
257.01	0.01\\
258.01	0.01\\
259.01	0.01\\
260.01	0.01\\
261.01	0.01\\
262.01	0.01\\
263.01	0.01\\
264.01	0.01\\
265.01	0.01\\
266.01	0.01\\
267.01	0.01\\
268.01	0.01\\
269.01	0.01\\
270.01	0.01\\
271.01	0.01\\
272.01	0.01\\
273.01	0.01\\
274.01	0.01\\
275.01	0.01\\
276.01	0.01\\
277.01	0.01\\
278.01	0.01\\
279.01	0.01\\
280.01	0.01\\
281.01	0.01\\
282.01	0.01\\
283.01	0.01\\
284.01	0.01\\
285.01	0.01\\
286.01	0.01\\
287.01	0.01\\
288.01	0.01\\
289.01	0.01\\
290.01	0.01\\
291.01	0.01\\
292.01	0.01\\
293.01	0.01\\
294.01	0.01\\
295.01	0.01\\
296.01	0.01\\
297.01	0.01\\
298.01	0.01\\
299.01	0.01\\
300.01	0.01\\
301.01	0.01\\
302.01	0.01\\
303.01	0.01\\
304.01	0.01\\
305.01	0.01\\
306.01	0.01\\
307.01	0.01\\
308.01	0.01\\
309.01	0.01\\
310.01	0.01\\
311.01	0.01\\
312.01	0.01\\
313.01	0.01\\
314.01	0.01\\
315.01	0.01\\
316.01	0.01\\
317.01	0.01\\
318.01	0.01\\
319.01	0.01\\
320.01	0.01\\
321.01	0.01\\
322.01	0.01\\
323.01	0.01\\
324.01	0.01\\
325.01	0.01\\
326.01	0.01\\
327.01	0.01\\
328.01	0.01\\
329.01	0.01\\
330.01	0.01\\
331.01	0.01\\
332.01	0.01\\
333.01	0.01\\
334.01	0.01\\
335.01	0.01\\
336.01	0.01\\
337.01	0.01\\
338.01	0.01\\
339.01	0.01\\
340.01	0.01\\
341.01	0.01\\
342.01	0.01\\
343.01	0.01\\
344.01	0.01\\
345.01	0.01\\
346.01	0.01\\
347.01	0.01\\
348.01	0.01\\
349.01	0.01\\
350.01	0.01\\
351.01	0.01\\
352.01	0.01\\
353.01	0.01\\
354.01	0.01\\
355.01	0.01\\
356.01	0.01\\
357.01	0.01\\
358.01	0.01\\
359.01	0.01\\
360.01	0.01\\
361.01	0.01\\
362.01	0.01\\
363.01	0.01\\
364.01	0.01\\
365.01	0.01\\
366.01	0.01\\
367.01	0.01\\
368.01	0.01\\
369.01	0.01\\
370.01	0.01\\
371.01	0.01\\
372.01	0.01\\
373.01	0.01\\
374.01	0.01\\
375.01	0.01\\
376.01	0.01\\
377.01	0.01\\
378.01	0.01\\
379.01	0.01\\
380.01	0.01\\
381.01	0.01\\
382.01	0.01\\
383.01	0.01\\
384.01	0.01\\
385.01	0.01\\
386.01	0.01\\
387.01	0.01\\
388.01	0.01\\
389.01	0.01\\
390.01	0.01\\
391.01	0.01\\
392.01	0.01\\
393.01	0.01\\
394.01	0.01\\
395.01	0.01\\
396.01	0.01\\
397.01	0.01\\
398.01	0.01\\
399.01	0.01\\
400.01	0.01\\
401.01	0.01\\
402.01	0.01\\
403.01	0.01\\
404.01	0.01\\
405.01	0.01\\
406.01	0.01\\
407.01	0.01\\
408.01	0.01\\
409.01	0.01\\
410.01	0.01\\
411.01	0.01\\
412.01	0.01\\
413.01	0.01\\
414.01	0.01\\
415.01	0.01\\
416.01	0.01\\
417.01	0.01\\
418.01	0.01\\
419.01	0.01\\
420.01	0.01\\
421.01	0.01\\
422.01	0.01\\
423.01	0.01\\
424.01	0.01\\
425.01	0.01\\
426.01	0.01\\
427.01	0.01\\
428.01	0.01\\
429.01	0.01\\
430.01	0.01\\
431.01	0.01\\
432.01	0.01\\
433.01	0.01\\
434.01	0.01\\
435.01	0.01\\
436.01	0.01\\
437.01	0.01\\
438.01	0.01\\
439.01	0.01\\
440.01	0.01\\
441.01	0.01\\
442.01	0.01\\
443.01	0.01\\
444.01	0.01\\
445.01	0.01\\
446.01	0.01\\
447.01	0.01\\
448.01	0.01\\
449.01	0.01\\
450.01	0.01\\
451.01	0.01\\
452.01	0.01\\
453.01	0.01\\
454.01	0.01\\
455.01	0.01\\
456.01	0.01\\
457.01	0.01\\
458.01	0.01\\
459.01	0.01\\
460.01	0.01\\
461.01	0.01\\
462.01	0.01\\
463.01	0.01\\
464.01	0.01\\
465.01	0.01\\
466.01	0.01\\
467.01	0.01\\
468.01	0.01\\
469.01	0.01\\
470.01	0.01\\
471.01	0.01\\
472.01	0.01\\
473.01	0.01\\
474.01	0.01\\
475.01	0.01\\
476.01	0.01\\
477.01	0.01\\
478.01	0.01\\
479.01	0.01\\
480.01	0.01\\
481.01	0.01\\
482.01	0.01\\
483.01	0.01\\
484.01	0.01\\
485.01	0.01\\
486.01	0.01\\
487.01	0.01\\
488.01	0.01\\
489.01	0.01\\
490.01	0.01\\
491.01	0.01\\
492.01	0.01\\
493.01	0.01\\
494.01	0.01\\
495.01	0.01\\
496.01	0.01\\
497.01	0.01\\
498.01	0.01\\
499.01	0.01\\
500.01	0.01\\
501.01	0.01\\
502.01	0.01\\
503.01	0.01\\
504.01	0.01\\
505.01	0.01\\
506.01	0.01\\
507.01	0.01\\
508.01	0.01\\
509.01	0.01\\
510.01	0.01\\
511.01	0.01\\
512.01	0.01\\
513.01	0.01\\
514.01	0.01\\
515.01	0.01\\
516.01	0.01\\
517.01	0.01\\
518.01	0.01\\
519.01	0.01\\
520.01	0.01\\
521.01	0.01\\
522.01	0.01\\
523.01	0.01\\
524.01	0.01\\
525.01	0.01\\
526.01	0.01\\
527.01	0.01\\
528.01	0.01\\
529.01	0.01\\
530.01	0.01\\
531.01	0.01\\
532.01	0.01\\
533.01	0.01\\
534.01	0.01\\
535.01	0.01\\
536.01	0.01\\
537.01	0.01\\
538.01	0.01\\
539.01	0.01\\
540.01	0.01\\
541.01	0.01\\
542.01	0.01\\
543.01	0.01\\
544.01	0.01\\
545.01	0.01\\
546.01	0.01\\
547.01	0.01\\
548.01	0.01\\
549.01	0.01\\
550.01	0.01\\
551.01	0.01\\
552.01	0.01\\
553.01	0.01\\
554.01	0.01\\
555.01	0.01\\
556.01	0.01\\
557.01	0.01\\
558.01	0.01\\
559.01	0.01\\
560.01	0.01\\
561.01	0.01\\
562.01	0.01\\
563.01	0.01\\
564.01	0.01\\
565.01	0.01\\
566.01	0.01\\
567.01	0.01\\
568.01	0.01\\
569.01	0.01\\
570.01	0.01\\
571.01	0.01\\
572.01	0.01\\
573.01	0.01\\
574.01	0.01\\
575.01	0.01\\
576.01	0.01\\
577.01	0.01\\
578.01	0.01\\
579.01	0.01\\
580.01	0.01\\
581.01	0.01\\
582.01	0.01\\
583.01	0.01\\
584.01	0.01\\
585.01	0.01\\
586.01	0.01\\
587.01	0.01\\
588.01	0.01\\
589.01	0.01\\
590.01	0.01\\
591.01	0.01\\
592.01	0.01\\
593.01	0.01\\
594.01	0.01\\
595.01	0.01\\
596.01	0.01\\
597.01	0.01\\
598.01	0.01\\
599.01	0.00623513588550072\\
599.02	0.00619747375148484\\
599.03	0.00615944487982368\\
599.04	0.00612104566378498\\
599.05	0.00608227246107566\\
599.06	0.00604312159348489\\
599.07	0.00600358934652321\\
599.08	0.00596367196905715\\
599.09	0.00592336567339871\\
599.1	0.00588266663573343\\
599.11	0.00584157099466158\\
599.12	0.00580007485082921\\
599.13	0.00575817426655544\\
599.14	0.00571586526545588\\
599.15	0.00567314383206202\\
599.16	0.00563000591143662\\
599.17	0.00558644740878492\\
599.18	0.00554246418905517\\
599.19	0.00549805207654024\\
599.2	0.00545320685447463\\
599.21	0.0054079242646287\\
599.22	0.00536220000690092\\
599.23	0.00531602973890626\\
599.24	0.00526940907556028\\
599.25	0.00522233358865938\\
599.26	0.00517479880645675\\
599.27	0.00512680021323422\\
599.28	0.00507833324886986\\
599.29	0.0050293933084014\\
599.3	0.0049799757415853\\
599.31	0.00493007585245144\\
599.32	0.00487968889885358\\
599.33	0.00482881009201521\\
599.34	0.00477743459607107\\
599.35	0.00472555752760411\\
599.36	0.00467317395517781\\
599.37	0.0046202788988641\\
599.38	0.00456686732976642\\
599.39	0.00451293416953819\\
599.4	0.00445847428989657\\
599.41	0.00440348251820621\\
599.42	0.00434795364222823\\
599.43	0.00429188239864167\\
599.44	0.00423526347254286\\
599.45	0.00417809149693984\\
599.46	0.00412036105224189\\
599.47	0.00406206666574399\\
599.48	0.0040032028111063\\
599.49	0.00394376390782856\\
599.5	0.00388374432071923\\
599.51	0.0038231383593596\\
599.52	0.00376194027756244\\
599.53	0.00370014427282557\\
599.54	0.00363774448577988\\
599.55	0.00357473499963204\\
599.56	0.00351110983960176\\
599.57	0.00344686297235346\\
599.58	0.00338198830542247\\
599.59	0.00331647968663548\\
599.6	0.00325033090352545\\
599.61	0.0031835356827407\\
599.62	0.0031160876894482\\
599.63	0.00304798052673106\\
599.64	0.00297920773498011\\
599.65	0.00290976279127949\\
599.66	0.0028396391087862\\
599.67	0.00276883003610366\\
599.68	0.00269732885664901\\
599.69	0.00262512878801433\\
599.7	0.00255222298132147\\
599.71	0.00247860452057071\\
599.72	0.00240426642198287\\
599.73	0.0023292016333351\\
599.74	0.00225340303329012\\
599.75	0.00217686343071887\\
599.76	0.00209957556401652\\
599.77	0.00202153210041183\\
599.78	0.00194272563526962\\
599.79	0.0018631486913866\\
599.8	0.00178279371828008\\
599.81	0.00170165309146988\\
599.82	0.00161971911175309\\
599.83	0.00153698400447179\\
599.84	0.00145343991877358\\
599.85	0.00136907892686481\\
599.86	0.0012838930232566\\
599.87	0.00119787412400332\\
599.88	0.00111101406593378\\
599.89	0.00102330460587476\\
599.9	0.000934737419866927\\
599.91	0.000845304102373193\\
599.92	0.000754996165479169\\
599.93	0.000663805038085868\\
599.94	0.000571722065094473\\
599.95	0.000478738506583127\\
599.96	0.000384845536975638\\
599.97	0.000290034244202044\\
599.98	0.00019429562885097\\
599.99	9.76206033136574e-05\\
600	0\\
};
\addplot [color=blue!75!mycolor7,solid,forget plot]
  table[row sep=crcr]{%
0.01	0.01\\
1.01	0.01\\
2.01	0.01\\
3.01	0.01\\
4.01	0.01\\
5.01	0.01\\
6.01	0.01\\
7.01	0.01\\
8.01	0.01\\
9.01	0.01\\
10.01	0.01\\
11.01	0.01\\
12.01	0.01\\
13.01	0.01\\
14.01	0.01\\
15.01	0.01\\
16.01	0.01\\
17.01	0.01\\
18.01	0.01\\
19.01	0.01\\
20.01	0.01\\
21.01	0.01\\
22.01	0.01\\
23.01	0.01\\
24.01	0.01\\
25.01	0.01\\
26.01	0.01\\
27.01	0.01\\
28.01	0.01\\
29.01	0.01\\
30.01	0.01\\
31.01	0.01\\
32.01	0.01\\
33.01	0.01\\
34.01	0.01\\
35.01	0.01\\
36.01	0.01\\
37.01	0.01\\
38.01	0.01\\
39.01	0.01\\
40.01	0.01\\
41.01	0.01\\
42.01	0.01\\
43.01	0.01\\
44.01	0.01\\
45.01	0.01\\
46.01	0.01\\
47.01	0.01\\
48.01	0.01\\
49.01	0.01\\
50.01	0.01\\
51.01	0.01\\
52.01	0.01\\
53.01	0.01\\
54.01	0.01\\
55.01	0.01\\
56.01	0.01\\
57.01	0.01\\
58.01	0.01\\
59.01	0.01\\
60.01	0.01\\
61.01	0.01\\
62.01	0.01\\
63.01	0.01\\
64.01	0.01\\
65.01	0.01\\
66.01	0.01\\
67.01	0.01\\
68.01	0.01\\
69.01	0.01\\
70.01	0.01\\
71.01	0.01\\
72.01	0.01\\
73.01	0.01\\
74.01	0.01\\
75.01	0.01\\
76.01	0.01\\
77.01	0.01\\
78.01	0.01\\
79.01	0.01\\
80.01	0.01\\
81.01	0.01\\
82.01	0.01\\
83.01	0.01\\
84.01	0.01\\
85.01	0.01\\
86.01	0.01\\
87.01	0.01\\
88.01	0.01\\
89.01	0.01\\
90.01	0.01\\
91.01	0.01\\
92.01	0.01\\
93.01	0.01\\
94.01	0.01\\
95.01	0.01\\
96.01	0.01\\
97.01	0.01\\
98.01	0.01\\
99.01	0.01\\
100.01	0.01\\
101.01	0.01\\
102.01	0.01\\
103.01	0.01\\
104.01	0.01\\
105.01	0.01\\
106.01	0.01\\
107.01	0.01\\
108.01	0.01\\
109.01	0.01\\
110.01	0.01\\
111.01	0.01\\
112.01	0.01\\
113.01	0.01\\
114.01	0.01\\
115.01	0.01\\
116.01	0.01\\
117.01	0.01\\
118.01	0.01\\
119.01	0.01\\
120.01	0.01\\
121.01	0.01\\
122.01	0.01\\
123.01	0.01\\
124.01	0.01\\
125.01	0.01\\
126.01	0.01\\
127.01	0.01\\
128.01	0.01\\
129.01	0.01\\
130.01	0.01\\
131.01	0.01\\
132.01	0.01\\
133.01	0.01\\
134.01	0.01\\
135.01	0.01\\
136.01	0.01\\
137.01	0.01\\
138.01	0.01\\
139.01	0.01\\
140.01	0.01\\
141.01	0.01\\
142.01	0.01\\
143.01	0.01\\
144.01	0.01\\
145.01	0.01\\
146.01	0.01\\
147.01	0.01\\
148.01	0.01\\
149.01	0.01\\
150.01	0.01\\
151.01	0.01\\
152.01	0.01\\
153.01	0.01\\
154.01	0.01\\
155.01	0.01\\
156.01	0.01\\
157.01	0.01\\
158.01	0.01\\
159.01	0.01\\
160.01	0.01\\
161.01	0.01\\
162.01	0.01\\
163.01	0.01\\
164.01	0.01\\
165.01	0.01\\
166.01	0.01\\
167.01	0.01\\
168.01	0.01\\
169.01	0.01\\
170.01	0.01\\
171.01	0.01\\
172.01	0.01\\
173.01	0.01\\
174.01	0.01\\
175.01	0.01\\
176.01	0.01\\
177.01	0.01\\
178.01	0.01\\
179.01	0.01\\
180.01	0.01\\
181.01	0.01\\
182.01	0.01\\
183.01	0.01\\
184.01	0.01\\
185.01	0.01\\
186.01	0.01\\
187.01	0.01\\
188.01	0.01\\
189.01	0.01\\
190.01	0.01\\
191.01	0.01\\
192.01	0.01\\
193.01	0.01\\
194.01	0.01\\
195.01	0.01\\
196.01	0.01\\
197.01	0.01\\
198.01	0.01\\
199.01	0.01\\
200.01	0.01\\
201.01	0.01\\
202.01	0.01\\
203.01	0.01\\
204.01	0.01\\
205.01	0.01\\
206.01	0.01\\
207.01	0.01\\
208.01	0.01\\
209.01	0.01\\
210.01	0.01\\
211.01	0.01\\
212.01	0.01\\
213.01	0.01\\
214.01	0.01\\
215.01	0.01\\
216.01	0.01\\
217.01	0.01\\
218.01	0.01\\
219.01	0.01\\
220.01	0.01\\
221.01	0.01\\
222.01	0.01\\
223.01	0.01\\
224.01	0.01\\
225.01	0.01\\
226.01	0.01\\
227.01	0.01\\
228.01	0.01\\
229.01	0.01\\
230.01	0.01\\
231.01	0.01\\
232.01	0.01\\
233.01	0.01\\
234.01	0.01\\
235.01	0.01\\
236.01	0.01\\
237.01	0.01\\
238.01	0.01\\
239.01	0.01\\
240.01	0.01\\
241.01	0.01\\
242.01	0.01\\
243.01	0.01\\
244.01	0.01\\
245.01	0.01\\
246.01	0.01\\
247.01	0.01\\
248.01	0.01\\
249.01	0.01\\
250.01	0.01\\
251.01	0.01\\
252.01	0.01\\
253.01	0.01\\
254.01	0.01\\
255.01	0.01\\
256.01	0.01\\
257.01	0.01\\
258.01	0.01\\
259.01	0.01\\
260.01	0.01\\
261.01	0.01\\
262.01	0.01\\
263.01	0.01\\
264.01	0.01\\
265.01	0.01\\
266.01	0.01\\
267.01	0.01\\
268.01	0.01\\
269.01	0.01\\
270.01	0.01\\
271.01	0.01\\
272.01	0.01\\
273.01	0.01\\
274.01	0.01\\
275.01	0.01\\
276.01	0.01\\
277.01	0.01\\
278.01	0.01\\
279.01	0.01\\
280.01	0.01\\
281.01	0.01\\
282.01	0.01\\
283.01	0.01\\
284.01	0.01\\
285.01	0.01\\
286.01	0.01\\
287.01	0.01\\
288.01	0.01\\
289.01	0.01\\
290.01	0.01\\
291.01	0.01\\
292.01	0.01\\
293.01	0.01\\
294.01	0.01\\
295.01	0.01\\
296.01	0.01\\
297.01	0.01\\
298.01	0.01\\
299.01	0.01\\
300.01	0.01\\
301.01	0.01\\
302.01	0.01\\
303.01	0.01\\
304.01	0.01\\
305.01	0.01\\
306.01	0.01\\
307.01	0.01\\
308.01	0.01\\
309.01	0.01\\
310.01	0.01\\
311.01	0.01\\
312.01	0.01\\
313.01	0.01\\
314.01	0.01\\
315.01	0.01\\
316.01	0.01\\
317.01	0.01\\
318.01	0.01\\
319.01	0.01\\
320.01	0.01\\
321.01	0.01\\
322.01	0.01\\
323.01	0.01\\
324.01	0.01\\
325.01	0.01\\
326.01	0.01\\
327.01	0.01\\
328.01	0.01\\
329.01	0.01\\
330.01	0.01\\
331.01	0.01\\
332.01	0.01\\
333.01	0.01\\
334.01	0.01\\
335.01	0.01\\
336.01	0.01\\
337.01	0.01\\
338.01	0.01\\
339.01	0.01\\
340.01	0.01\\
341.01	0.01\\
342.01	0.01\\
343.01	0.01\\
344.01	0.01\\
345.01	0.01\\
346.01	0.01\\
347.01	0.01\\
348.01	0.01\\
349.01	0.01\\
350.01	0.01\\
351.01	0.01\\
352.01	0.01\\
353.01	0.01\\
354.01	0.01\\
355.01	0.01\\
356.01	0.01\\
357.01	0.01\\
358.01	0.01\\
359.01	0.01\\
360.01	0.01\\
361.01	0.01\\
362.01	0.01\\
363.01	0.01\\
364.01	0.01\\
365.01	0.01\\
366.01	0.01\\
367.01	0.01\\
368.01	0.01\\
369.01	0.01\\
370.01	0.01\\
371.01	0.01\\
372.01	0.01\\
373.01	0.01\\
374.01	0.01\\
375.01	0.01\\
376.01	0.01\\
377.01	0.01\\
378.01	0.01\\
379.01	0.01\\
380.01	0.01\\
381.01	0.01\\
382.01	0.01\\
383.01	0.01\\
384.01	0.01\\
385.01	0.01\\
386.01	0.01\\
387.01	0.01\\
388.01	0.01\\
389.01	0.01\\
390.01	0.01\\
391.01	0.01\\
392.01	0.01\\
393.01	0.01\\
394.01	0.01\\
395.01	0.01\\
396.01	0.01\\
397.01	0.01\\
398.01	0.01\\
399.01	0.01\\
400.01	0.01\\
401.01	0.01\\
402.01	0.01\\
403.01	0.01\\
404.01	0.01\\
405.01	0.01\\
406.01	0.01\\
407.01	0.01\\
408.01	0.01\\
409.01	0.01\\
410.01	0.01\\
411.01	0.01\\
412.01	0.01\\
413.01	0.01\\
414.01	0.01\\
415.01	0.01\\
416.01	0.01\\
417.01	0.01\\
418.01	0.01\\
419.01	0.01\\
420.01	0.01\\
421.01	0.01\\
422.01	0.01\\
423.01	0.01\\
424.01	0.01\\
425.01	0.01\\
426.01	0.01\\
427.01	0.01\\
428.01	0.01\\
429.01	0.01\\
430.01	0.01\\
431.01	0.01\\
432.01	0.01\\
433.01	0.01\\
434.01	0.01\\
435.01	0.01\\
436.01	0.01\\
437.01	0.01\\
438.01	0.01\\
439.01	0.01\\
440.01	0.01\\
441.01	0.01\\
442.01	0.01\\
443.01	0.01\\
444.01	0.01\\
445.01	0.01\\
446.01	0.01\\
447.01	0.01\\
448.01	0.01\\
449.01	0.01\\
450.01	0.01\\
451.01	0.01\\
452.01	0.01\\
453.01	0.01\\
454.01	0.01\\
455.01	0.01\\
456.01	0.01\\
457.01	0.01\\
458.01	0.01\\
459.01	0.01\\
460.01	0.01\\
461.01	0.01\\
462.01	0.01\\
463.01	0.01\\
464.01	0.01\\
465.01	0.01\\
466.01	0.01\\
467.01	0.01\\
468.01	0.01\\
469.01	0.01\\
470.01	0.01\\
471.01	0.01\\
472.01	0.01\\
473.01	0.01\\
474.01	0.01\\
475.01	0.01\\
476.01	0.01\\
477.01	0.01\\
478.01	0.01\\
479.01	0.01\\
480.01	0.01\\
481.01	0.01\\
482.01	0.01\\
483.01	0.01\\
484.01	0.01\\
485.01	0.01\\
486.01	0.01\\
487.01	0.01\\
488.01	0.01\\
489.01	0.01\\
490.01	0.01\\
491.01	0.01\\
492.01	0.01\\
493.01	0.01\\
494.01	0.01\\
495.01	0.01\\
496.01	0.01\\
497.01	0.01\\
498.01	0.01\\
499.01	0.01\\
500.01	0.01\\
501.01	0.01\\
502.01	0.01\\
503.01	0.01\\
504.01	0.01\\
505.01	0.01\\
506.01	0.01\\
507.01	0.01\\
508.01	0.01\\
509.01	0.01\\
510.01	0.01\\
511.01	0.01\\
512.01	0.01\\
513.01	0.01\\
514.01	0.01\\
515.01	0.01\\
516.01	0.01\\
517.01	0.01\\
518.01	0.01\\
519.01	0.01\\
520.01	0.01\\
521.01	0.01\\
522.01	0.01\\
523.01	0.01\\
524.01	0.01\\
525.01	0.01\\
526.01	0.01\\
527.01	0.01\\
528.01	0.01\\
529.01	0.01\\
530.01	0.01\\
531.01	0.01\\
532.01	0.01\\
533.01	0.01\\
534.01	0.01\\
535.01	0.01\\
536.01	0.01\\
537.01	0.01\\
538.01	0.01\\
539.01	0.01\\
540.01	0.01\\
541.01	0.01\\
542.01	0.01\\
543.01	0.01\\
544.01	0.01\\
545.01	0.01\\
546.01	0.01\\
547.01	0.01\\
548.01	0.01\\
549.01	0.01\\
550.01	0.01\\
551.01	0.01\\
552.01	0.01\\
553.01	0.01\\
554.01	0.01\\
555.01	0.01\\
556.01	0.01\\
557.01	0.01\\
558.01	0.01\\
559.01	0.01\\
560.01	0.01\\
561.01	0.01\\
562.01	0.01\\
563.01	0.01\\
564.01	0.01\\
565.01	0.01\\
566.01	0.01\\
567.01	0.01\\
568.01	0.01\\
569.01	0.01\\
570.01	0.01\\
571.01	0.01\\
572.01	0.01\\
573.01	0.01\\
574.01	0.01\\
575.01	0.01\\
576.01	0.01\\
577.01	0.01\\
578.01	0.01\\
579.01	0.01\\
580.01	0.01\\
581.01	0.01\\
582.01	0.01\\
583.01	0.01\\
584.01	0.01\\
585.01	0.01\\
586.01	0.01\\
587.01	0.01\\
588.01	0.01\\
589.01	0.01\\
590.01	0.01\\
591.01	0.01\\
592.01	0.01\\
593.01	0.01\\
594.01	0.01\\
595.01	0.01\\
596.01	0.01\\
597.01	0.01\\
598.01	0.01\\
599.01	0.00623515239359194\\
599.02	0.00619748725350673\\
599.03	0.00615945574907279\\
599.04	0.00612105427627316\\
599.05	0.00608227919580345\\
599.06	0.00604312683265317\\
599.07	0.00600359347579347\\
599.08	0.00596367537790493\\
599.09	0.00592336866516034\\
599.1	0.00588266927971629\\
599.11	0.00584157335142531\\
599.12	0.00580007697221952\\
599.13	0.00575817619584629\\
599.14	0.00571586703761439\\
599.15	0.00567314547415175\\
599.16	0.00563000744317626\\
599.17	0.00558644884328414\\
599.18	0.00554246553533496\\
599.19	0.00549805334084668\\
599.2	0.00545320804184588\\
599.21	0.0054079253798647\\
599.22	0.00536220105456069\\
599.23	0.00531603072330621\\
599.24	0.005269410000774\\
599.25	0.0052223344585187\\
599.26	0.00517479962455485\\
599.27	0.00512680098293079\\
599.28	0.00507833397329517\\
599.29	0.0050293939904593\\
599.3	0.00497997638395778\\
599.31	0.00493007645760491\\
599.32	0.00487968946904662\\
599.33	0.00482881062930801\\
599.34	0.00477743510233639\\
599.35	0.00472555800453965\\
599.36	0.00467317440432004\\
599.37	0.00462027932160314\\
599.38	0.00456686772736205\\
599.39	0.00451293454313655\\
599.4	0.00445847464054731\\
599.41	0.00440348284687955\\
599.42	0.00434795394983202\\
599.43	0.00429188268603743\\
599.44	0.00423526374056028\\
599.45	0.00417809174638914\\
599.46	0.00412036128392308\\
599.47	0.00406206688045222\\
599.48	0.00400320300963207\\
599.49	0.00394376409095614\\
599.5	0.00388374448922514\\
599.51	0.00382313851401101\\
599.52	0.00376194041911571\\
599.53	0.00370014440202472\\
599.54	0.00363774460335511\\
599.55	0.00357473510629834\\
599.56	0.00351110993605749\\
599.57	0.00344686305927907\\
599.58	0.00338198838347919\\
599.59	0.00331647975646418\\
599.6	0.00325033096574548\\
599.61	0.00318353573794888\\
599.62	0.00311608773821786\\
599.63	0.00304798056961123\\
599.64	0.00297920777249475\\
599.65	0.00290976282392687\\
599.66	0.0028396391370384\\
599.67	0.00276883006040616\\
599.68	0.00269732887742047\\
599.69	0.00262512880564641\\
599.7	0.00255222299617888\\
599.71	0.00247860453299131\\
599.72	0.00240426643227798\\
599.73	0.00232920164178993\\
599.74	0.00225340304016429\\
599.75	0.00217686343624712\\
599.76	0.00209957556840957\\
599.77	0.00202153210385725\\
599.78	0.00194272563793302\\
599.79	0.00186314869341269\\
599.8	0.00178279371979404\\
599.81	0.00170165309257868\\
599.82	0.00161971911254694\\
599.83	0.00153698400502566\\
599.84	0.0014534399191487\\
599.85	0.00136907892711024\\
599.86	0.00128389302341077\\
599.87	0.00119787412409557\\
599.88	0.00111101406598581\\
599.89	0.001023304605902\\
599.9	0.000934737419879901\\
599.91	0.000845304102378633\\
599.92	0.000754996165481069\\
599.93	0.000663805038086366\\
599.94	0.00057172206509455\\
599.95	0.000478738506583127\\
599.96	0.000384845536975638\\
599.97	0.000290034244202044\\
599.98	0.00019429562885097\\
599.99	9.76206033136556e-05\\
600	0\\
};
\addplot [color=blue!80!mycolor9,solid,forget plot]
  table[row sep=crcr]{%
0.01	0.01\\
1.01	0.01\\
2.01	0.01\\
3.01	0.01\\
4.01	0.01\\
5.01	0.01\\
6.01	0.01\\
7.01	0.01\\
8.01	0.01\\
9.01	0.01\\
10.01	0.01\\
11.01	0.01\\
12.01	0.01\\
13.01	0.01\\
14.01	0.01\\
15.01	0.01\\
16.01	0.01\\
17.01	0.01\\
18.01	0.01\\
19.01	0.01\\
20.01	0.01\\
21.01	0.01\\
22.01	0.01\\
23.01	0.01\\
24.01	0.01\\
25.01	0.01\\
26.01	0.01\\
27.01	0.01\\
28.01	0.01\\
29.01	0.01\\
30.01	0.01\\
31.01	0.01\\
32.01	0.01\\
33.01	0.01\\
34.01	0.01\\
35.01	0.01\\
36.01	0.01\\
37.01	0.01\\
38.01	0.01\\
39.01	0.01\\
40.01	0.01\\
41.01	0.01\\
42.01	0.01\\
43.01	0.01\\
44.01	0.01\\
45.01	0.01\\
46.01	0.01\\
47.01	0.01\\
48.01	0.01\\
49.01	0.01\\
50.01	0.01\\
51.01	0.01\\
52.01	0.01\\
53.01	0.01\\
54.01	0.01\\
55.01	0.01\\
56.01	0.01\\
57.01	0.01\\
58.01	0.01\\
59.01	0.01\\
60.01	0.01\\
61.01	0.01\\
62.01	0.01\\
63.01	0.01\\
64.01	0.01\\
65.01	0.01\\
66.01	0.01\\
67.01	0.01\\
68.01	0.01\\
69.01	0.01\\
70.01	0.01\\
71.01	0.01\\
72.01	0.01\\
73.01	0.01\\
74.01	0.01\\
75.01	0.01\\
76.01	0.01\\
77.01	0.01\\
78.01	0.01\\
79.01	0.01\\
80.01	0.01\\
81.01	0.01\\
82.01	0.01\\
83.01	0.01\\
84.01	0.01\\
85.01	0.01\\
86.01	0.01\\
87.01	0.01\\
88.01	0.01\\
89.01	0.01\\
90.01	0.01\\
91.01	0.01\\
92.01	0.01\\
93.01	0.01\\
94.01	0.01\\
95.01	0.01\\
96.01	0.01\\
97.01	0.01\\
98.01	0.01\\
99.01	0.01\\
100.01	0.01\\
101.01	0.01\\
102.01	0.01\\
103.01	0.01\\
104.01	0.01\\
105.01	0.01\\
106.01	0.01\\
107.01	0.01\\
108.01	0.01\\
109.01	0.01\\
110.01	0.01\\
111.01	0.01\\
112.01	0.01\\
113.01	0.01\\
114.01	0.01\\
115.01	0.01\\
116.01	0.01\\
117.01	0.01\\
118.01	0.01\\
119.01	0.01\\
120.01	0.01\\
121.01	0.01\\
122.01	0.01\\
123.01	0.01\\
124.01	0.01\\
125.01	0.01\\
126.01	0.01\\
127.01	0.01\\
128.01	0.01\\
129.01	0.01\\
130.01	0.01\\
131.01	0.01\\
132.01	0.01\\
133.01	0.01\\
134.01	0.01\\
135.01	0.01\\
136.01	0.01\\
137.01	0.01\\
138.01	0.01\\
139.01	0.01\\
140.01	0.01\\
141.01	0.01\\
142.01	0.01\\
143.01	0.01\\
144.01	0.01\\
145.01	0.01\\
146.01	0.01\\
147.01	0.01\\
148.01	0.01\\
149.01	0.01\\
150.01	0.01\\
151.01	0.01\\
152.01	0.01\\
153.01	0.01\\
154.01	0.01\\
155.01	0.01\\
156.01	0.01\\
157.01	0.01\\
158.01	0.01\\
159.01	0.01\\
160.01	0.01\\
161.01	0.01\\
162.01	0.01\\
163.01	0.01\\
164.01	0.01\\
165.01	0.01\\
166.01	0.01\\
167.01	0.01\\
168.01	0.01\\
169.01	0.01\\
170.01	0.01\\
171.01	0.01\\
172.01	0.01\\
173.01	0.01\\
174.01	0.01\\
175.01	0.01\\
176.01	0.01\\
177.01	0.01\\
178.01	0.01\\
179.01	0.01\\
180.01	0.01\\
181.01	0.01\\
182.01	0.01\\
183.01	0.01\\
184.01	0.01\\
185.01	0.01\\
186.01	0.01\\
187.01	0.01\\
188.01	0.01\\
189.01	0.01\\
190.01	0.01\\
191.01	0.01\\
192.01	0.01\\
193.01	0.01\\
194.01	0.01\\
195.01	0.01\\
196.01	0.01\\
197.01	0.01\\
198.01	0.01\\
199.01	0.01\\
200.01	0.01\\
201.01	0.01\\
202.01	0.01\\
203.01	0.01\\
204.01	0.01\\
205.01	0.01\\
206.01	0.01\\
207.01	0.01\\
208.01	0.01\\
209.01	0.01\\
210.01	0.01\\
211.01	0.01\\
212.01	0.01\\
213.01	0.01\\
214.01	0.01\\
215.01	0.01\\
216.01	0.01\\
217.01	0.01\\
218.01	0.01\\
219.01	0.01\\
220.01	0.01\\
221.01	0.01\\
222.01	0.01\\
223.01	0.01\\
224.01	0.01\\
225.01	0.01\\
226.01	0.01\\
227.01	0.01\\
228.01	0.01\\
229.01	0.01\\
230.01	0.01\\
231.01	0.01\\
232.01	0.01\\
233.01	0.01\\
234.01	0.01\\
235.01	0.01\\
236.01	0.01\\
237.01	0.01\\
238.01	0.01\\
239.01	0.01\\
240.01	0.01\\
241.01	0.01\\
242.01	0.01\\
243.01	0.01\\
244.01	0.01\\
245.01	0.01\\
246.01	0.01\\
247.01	0.01\\
248.01	0.01\\
249.01	0.01\\
250.01	0.01\\
251.01	0.01\\
252.01	0.01\\
253.01	0.01\\
254.01	0.01\\
255.01	0.01\\
256.01	0.01\\
257.01	0.01\\
258.01	0.01\\
259.01	0.01\\
260.01	0.01\\
261.01	0.01\\
262.01	0.01\\
263.01	0.01\\
264.01	0.01\\
265.01	0.01\\
266.01	0.01\\
267.01	0.01\\
268.01	0.01\\
269.01	0.01\\
270.01	0.01\\
271.01	0.01\\
272.01	0.01\\
273.01	0.01\\
274.01	0.01\\
275.01	0.01\\
276.01	0.01\\
277.01	0.01\\
278.01	0.01\\
279.01	0.01\\
280.01	0.01\\
281.01	0.01\\
282.01	0.01\\
283.01	0.01\\
284.01	0.01\\
285.01	0.01\\
286.01	0.01\\
287.01	0.01\\
288.01	0.01\\
289.01	0.01\\
290.01	0.01\\
291.01	0.01\\
292.01	0.01\\
293.01	0.01\\
294.01	0.01\\
295.01	0.01\\
296.01	0.01\\
297.01	0.01\\
298.01	0.01\\
299.01	0.01\\
300.01	0.01\\
301.01	0.01\\
302.01	0.01\\
303.01	0.01\\
304.01	0.01\\
305.01	0.01\\
306.01	0.01\\
307.01	0.01\\
308.01	0.01\\
309.01	0.01\\
310.01	0.01\\
311.01	0.01\\
312.01	0.01\\
313.01	0.01\\
314.01	0.01\\
315.01	0.01\\
316.01	0.01\\
317.01	0.01\\
318.01	0.01\\
319.01	0.01\\
320.01	0.01\\
321.01	0.01\\
322.01	0.01\\
323.01	0.01\\
324.01	0.01\\
325.01	0.01\\
326.01	0.01\\
327.01	0.01\\
328.01	0.01\\
329.01	0.01\\
330.01	0.01\\
331.01	0.01\\
332.01	0.01\\
333.01	0.01\\
334.01	0.01\\
335.01	0.01\\
336.01	0.01\\
337.01	0.01\\
338.01	0.01\\
339.01	0.01\\
340.01	0.01\\
341.01	0.01\\
342.01	0.01\\
343.01	0.01\\
344.01	0.01\\
345.01	0.01\\
346.01	0.01\\
347.01	0.01\\
348.01	0.01\\
349.01	0.01\\
350.01	0.01\\
351.01	0.01\\
352.01	0.01\\
353.01	0.01\\
354.01	0.01\\
355.01	0.01\\
356.01	0.01\\
357.01	0.01\\
358.01	0.01\\
359.01	0.01\\
360.01	0.01\\
361.01	0.01\\
362.01	0.01\\
363.01	0.01\\
364.01	0.01\\
365.01	0.01\\
366.01	0.01\\
367.01	0.01\\
368.01	0.01\\
369.01	0.01\\
370.01	0.01\\
371.01	0.01\\
372.01	0.01\\
373.01	0.01\\
374.01	0.01\\
375.01	0.01\\
376.01	0.01\\
377.01	0.01\\
378.01	0.01\\
379.01	0.01\\
380.01	0.01\\
381.01	0.01\\
382.01	0.01\\
383.01	0.01\\
384.01	0.01\\
385.01	0.01\\
386.01	0.01\\
387.01	0.01\\
388.01	0.01\\
389.01	0.01\\
390.01	0.01\\
391.01	0.01\\
392.01	0.01\\
393.01	0.01\\
394.01	0.01\\
395.01	0.01\\
396.01	0.01\\
397.01	0.01\\
398.01	0.01\\
399.01	0.01\\
400.01	0.01\\
401.01	0.01\\
402.01	0.01\\
403.01	0.01\\
404.01	0.01\\
405.01	0.01\\
406.01	0.01\\
407.01	0.01\\
408.01	0.01\\
409.01	0.01\\
410.01	0.01\\
411.01	0.01\\
412.01	0.01\\
413.01	0.01\\
414.01	0.01\\
415.01	0.01\\
416.01	0.01\\
417.01	0.01\\
418.01	0.01\\
419.01	0.01\\
420.01	0.01\\
421.01	0.01\\
422.01	0.01\\
423.01	0.01\\
424.01	0.01\\
425.01	0.01\\
426.01	0.01\\
427.01	0.01\\
428.01	0.01\\
429.01	0.01\\
430.01	0.01\\
431.01	0.01\\
432.01	0.01\\
433.01	0.01\\
434.01	0.01\\
435.01	0.01\\
436.01	0.01\\
437.01	0.01\\
438.01	0.01\\
439.01	0.01\\
440.01	0.01\\
441.01	0.01\\
442.01	0.01\\
443.01	0.01\\
444.01	0.01\\
445.01	0.01\\
446.01	0.01\\
447.01	0.01\\
448.01	0.01\\
449.01	0.01\\
450.01	0.01\\
451.01	0.01\\
452.01	0.01\\
453.01	0.01\\
454.01	0.01\\
455.01	0.01\\
456.01	0.01\\
457.01	0.01\\
458.01	0.01\\
459.01	0.01\\
460.01	0.01\\
461.01	0.01\\
462.01	0.01\\
463.01	0.01\\
464.01	0.01\\
465.01	0.01\\
466.01	0.01\\
467.01	0.01\\
468.01	0.01\\
469.01	0.01\\
470.01	0.01\\
471.01	0.01\\
472.01	0.01\\
473.01	0.01\\
474.01	0.01\\
475.01	0.01\\
476.01	0.01\\
477.01	0.01\\
478.01	0.01\\
479.01	0.01\\
480.01	0.01\\
481.01	0.01\\
482.01	0.01\\
483.01	0.01\\
484.01	0.01\\
485.01	0.01\\
486.01	0.01\\
487.01	0.01\\
488.01	0.01\\
489.01	0.01\\
490.01	0.01\\
491.01	0.01\\
492.01	0.01\\
493.01	0.01\\
494.01	0.01\\
495.01	0.01\\
496.01	0.01\\
497.01	0.01\\
498.01	0.01\\
499.01	0.01\\
500.01	0.01\\
501.01	0.01\\
502.01	0.01\\
503.01	0.01\\
504.01	0.01\\
505.01	0.01\\
506.01	0.01\\
507.01	0.01\\
508.01	0.01\\
509.01	0.01\\
510.01	0.01\\
511.01	0.01\\
512.01	0.01\\
513.01	0.01\\
514.01	0.01\\
515.01	0.01\\
516.01	0.01\\
517.01	0.01\\
518.01	0.01\\
519.01	0.01\\
520.01	0.01\\
521.01	0.01\\
522.01	0.01\\
523.01	0.01\\
524.01	0.01\\
525.01	0.01\\
526.01	0.01\\
527.01	0.01\\
528.01	0.01\\
529.01	0.01\\
530.01	0.01\\
531.01	0.01\\
532.01	0.01\\
533.01	0.01\\
534.01	0.01\\
535.01	0.01\\
536.01	0.01\\
537.01	0.01\\
538.01	0.01\\
539.01	0.01\\
540.01	0.01\\
541.01	0.01\\
542.01	0.01\\
543.01	0.01\\
544.01	0.01\\
545.01	0.01\\
546.01	0.01\\
547.01	0.01\\
548.01	0.01\\
549.01	0.01\\
550.01	0.01\\
551.01	0.01\\
552.01	0.01\\
553.01	0.01\\
554.01	0.01\\
555.01	0.01\\
556.01	0.01\\
557.01	0.01\\
558.01	0.01\\
559.01	0.01\\
560.01	0.01\\
561.01	0.01\\
562.01	0.01\\
563.01	0.01\\
564.01	0.01\\
565.01	0.01\\
566.01	0.01\\
567.01	0.01\\
568.01	0.01\\
569.01	0.01\\
570.01	0.01\\
571.01	0.01\\
572.01	0.01\\
573.01	0.01\\
574.01	0.01\\
575.01	0.01\\
576.01	0.01\\
577.01	0.01\\
578.01	0.01\\
579.01	0.01\\
580.01	0.01\\
581.01	0.01\\
582.01	0.01\\
583.01	0.01\\
584.01	0.01\\
585.01	0.01\\
586.01	0.01\\
587.01	0.01\\
588.01	0.01\\
589.01	0.01\\
590.01	0.01\\
591.01	0.01\\
592.01	0.01\\
593.01	0.01\\
594.01	0.01\\
595.01	0.01\\
596.01	0.01\\
597.01	0.01\\
598.01	0.01\\
599.01	0.00623588074663459\\
599.02	0.00619813582657612\\
599.03	0.00616002993582383\\
599.04	0.00612155943362391\\
599.05	0.00608272061297158\\
599.06	0.00604350974526074\\
599.07	0.00600392304473869\\
599.08	0.00596395665477922\\
599.09	0.00592360675084337\\
599.1	0.00588286958222237\\
599.11	0.0058417410941521\\
599.12	0.00580021715091981\\
599.13	0.00575829352900724\\
599.14	0.00571596591257865\\
599.15	0.0056732298887007\\
599.16	0.00563008094227915\\
599.17	0.0055865144500541\\
599.18	0.0055425253800716\\
599.19	0.00549810886649928\\
599.2	0.00545325991747207\\
599.21	0.00540797380783162\\
599.22	0.00536224623367933\\
599.23	0.005316072848296\\
599.24	0.00526944926174205\\
599.25	0.00522237104041281\\
599.26	0.00517483370647011\\
599.27	0.00512683273741956\\
599.28	0.00507836356645391\\
599.29	0.00502942158197927\\
599.3	0.00498000212655932\\
599.31	0.00493010049649361\\
599.32	0.00487971194140611\\
599.33	0.00482883166383815\\
599.34	0.00477745481884658\\
599.35	0.00472557651360802\\
599.36	0.00467319180703039\\
599.37	0.00462029570937272\\
599.38	0.00456688318187443\\
599.39	0.00451294913639552\\
599.4	0.00445848843506895\\
599.41	0.0044034958960355\\
599.42	0.00434796629837901\\
599.43	0.00429189437083759\\
599.44	0.00423527479152567\\
599.45	0.00417810218767722\\
599.46	0.00412037113541252\\
599.47	0.00406207615953092\\
599.48	0.00400321173333205\\
599.49	0.00394377227665567\\
599.5	0.00388375215461717\\
599.51	0.00382314567706917\\
599.52	0.00376194709805824\\
599.53	0.00370015061527655\\
599.54	0.00363775036950859\\
599.55	0.00357474044407246\\
599.56	0.00351111486425591\\
599.57	0.00344686759674706\\
599.58	0.00338199254905957\\
599.59	0.00331648356895239\\
599.6	0.00325033444384401\\
599.61	0.00318353890022104\\
599.62	0.0031160906030414\\
599.63	0.00304798315513165\\
599.64	0.00297921009657883\\
599.65	0.00290976490411649\\
599.66	0.00283964099050497\\
599.67	0.00276883170390586\\
599.68	0.00269733032725062\\
599.69	0.00262513007760331\\
599.7	0.00255222410551739\\
599.71	0.00247860549438648\\
599.72	0.0024042672597891\\
599.73	0.00232920234882737\\
599.74	0.00225340363945949\\
599.75	0.00217686393982603\\
599.76	0.00209957598757\\
599.77	0.00202153244915048\\
599.78	0.00194272591915\\
599.79	0.00186314891957526\\
599.8	0.00178279389915142\\
599.81	0.00170165323260962\\
599.82	0.00161971921996775\\
599.83	0.00153698408580431\\
599.84	0.00145343997852515\\
599.85	0.00136907896962321\\
599.86	0.00128389305293063\\
599.87	0.00119787414386362\\
599.88	0.00111101407865932\\
599.89	0.00102330461360488\\
599.9	0.000934737424258189\\
599.91	0.000845304104660138\\
599.92	0.000754996166537987\\
599.93	0.00066380503849954\\
599.94	0.000571722065217627\\
599.95	0.000478738506604516\\
599.96	0.000384845536975636\\
599.97	0.000290034244202044\\
599.98	0.00019429562885097\\
599.99	9.76206033136556e-05\\
600	0\\
};
\addplot [color=blue,solid,forget plot]
  table[row sep=crcr]{%
0.01	0.01\\
1.01	0.01\\
2.01	0.01\\
3.01	0.01\\
4.01	0.01\\
5.01	0.01\\
6.01	0.01\\
7.01	0.01\\
8.01	0.01\\
9.01	0.01\\
10.01	0.01\\
11.01	0.01\\
12.01	0.01\\
13.01	0.01\\
14.01	0.01\\
15.01	0.01\\
16.01	0.01\\
17.01	0.01\\
18.01	0.01\\
19.01	0.01\\
20.01	0.01\\
21.01	0.01\\
22.01	0.01\\
23.01	0.01\\
24.01	0.01\\
25.01	0.01\\
26.01	0.01\\
27.01	0.01\\
28.01	0.01\\
29.01	0.01\\
30.01	0.01\\
31.01	0.01\\
32.01	0.01\\
33.01	0.01\\
34.01	0.01\\
35.01	0.01\\
36.01	0.01\\
37.01	0.01\\
38.01	0.01\\
39.01	0.01\\
40.01	0.01\\
41.01	0.01\\
42.01	0.01\\
43.01	0.01\\
44.01	0.01\\
45.01	0.01\\
46.01	0.01\\
47.01	0.01\\
48.01	0.01\\
49.01	0.01\\
50.01	0.01\\
51.01	0.01\\
52.01	0.01\\
53.01	0.01\\
54.01	0.01\\
55.01	0.01\\
56.01	0.01\\
57.01	0.01\\
58.01	0.01\\
59.01	0.01\\
60.01	0.01\\
61.01	0.01\\
62.01	0.01\\
63.01	0.01\\
64.01	0.01\\
65.01	0.01\\
66.01	0.01\\
67.01	0.01\\
68.01	0.01\\
69.01	0.01\\
70.01	0.01\\
71.01	0.01\\
72.01	0.01\\
73.01	0.01\\
74.01	0.01\\
75.01	0.01\\
76.01	0.01\\
77.01	0.01\\
78.01	0.01\\
79.01	0.01\\
80.01	0.01\\
81.01	0.01\\
82.01	0.01\\
83.01	0.01\\
84.01	0.01\\
85.01	0.01\\
86.01	0.01\\
87.01	0.01\\
88.01	0.01\\
89.01	0.01\\
90.01	0.01\\
91.01	0.01\\
92.01	0.01\\
93.01	0.01\\
94.01	0.01\\
95.01	0.01\\
96.01	0.01\\
97.01	0.01\\
98.01	0.01\\
99.01	0.01\\
100.01	0.01\\
101.01	0.01\\
102.01	0.01\\
103.01	0.01\\
104.01	0.01\\
105.01	0.01\\
106.01	0.01\\
107.01	0.01\\
108.01	0.01\\
109.01	0.01\\
110.01	0.01\\
111.01	0.01\\
112.01	0.01\\
113.01	0.01\\
114.01	0.01\\
115.01	0.01\\
116.01	0.01\\
117.01	0.01\\
118.01	0.01\\
119.01	0.01\\
120.01	0.01\\
121.01	0.01\\
122.01	0.01\\
123.01	0.01\\
124.01	0.01\\
125.01	0.01\\
126.01	0.01\\
127.01	0.01\\
128.01	0.01\\
129.01	0.01\\
130.01	0.01\\
131.01	0.01\\
132.01	0.01\\
133.01	0.01\\
134.01	0.01\\
135.01	0.01\\
136.01	0.01\\
137.01	0.01\\
138.01	0.01\\
139.01	0.01\\
140.01	0.01\\
141.01	0.01\\
142.01	0.01\\
143.01	0.01\\
144.01	0.01\\
145.01	0.01\\
146.01	0.01\\
147.01	0.01\\
148.01	0.01\\
149.01	0.01\\
150.01	0.01\\
151.01	0.01\\
152.01	0.01\\
153.01	0.01\\
154.01	0.01\\
155.01	0.01\\
156.01	0.01\\
157.01	0.01\\
158.01	0.01\\
159.01	0.01\\
160.01	0.01\\
161.01	0.01\\
162.01	0.01\\
163.01	0.01\\
164.01	0.01\\
165.01	0.01\\
166.01	0.01\\
167.01	0.01\\
168.01	0.01\\
169.01	0.01\\
170.01	0.01\\
171.01	0.01\\
172.01	0.01\\
173.01	0.01\\
174.01	0.01\\
175.01	0.01\\
176.01	0.01\\
177.01	0.01\\
178.01	0.01\\
179.01	0.01\\
180.01	0.01\\
181.01	0.01\\
182.01	0.01\\
183.01	0.01\\
184.01	0.01\\
185.01	0.01\\
186.01	0.01\\
187.01	0.01\\
188.01	0.01\\
189.01	0.01\\
190.01	0.01\\
191.01	0.01\\
192.01	0.01\\
193.01	0.01\\
194.01	0.01\\
195.01	0.01\\
196.01	0.01\\
197.01	0.01\\
198.01	0.01\\
199.01	0.01\\
200.01	0.01\\
201.01	0.01\\
202.01	0.01\\
203.01	0.01\\
204.01	0.01\\
205.01	0.01\\
206.01	0.01\\
207.01	0.01\\
208.01	0.01\\
209.01	0.01\\
210.01	0.01\\
211.01	0.01\\
212.01	0.01\\
213.01	0.01\\
214.01	0.01\\
215.01	0.01\\
216.01	0.01\\
217.01	0.01\\
218.01	0.01\\
219.01	0.01\\
220.01	0.01\\
221.01	0.01\\
222.01	0.01\\
223.01	0.01\\
224.01	0.01\\
225.01	0.01\\
226.01	0.01\\
227.01	0.01\\
228.01	0.01\\
229.01	0.01\\
230.01	0.01\\
231.01	0.01\\
232.01	0.01\\
233.01	0.01\\
234.01	0.01\\
235.01	0.01\\
236.01	0.01\\
237.01	0.01\\
238.01	0.01\\
239.01	0.01\\
240.01	0.01\\
241.01	0.01\\
242.01	0.01\\
243.01	0.01\\
244.01	0.01\\
245.01	0.01\\
246.01	0.01\\
247.01	0.01\\
248.01	0.01\\
249.01	0.01\\
250.01	0.01\\
251.01	0.01\\
252.01	0.01\\
253.01	0.01\\
254.01	0.01\\
255.01	0.01\\
256.01	0.01\\
257.01	0.01\\
258.01	0.01\\
259.01	0.01\\
260.01	0.01\\
261.01	0.01\\
262.01	0.01\\
263.01	0.01\\
264.01	0.01\\
265.01	0.01\\
266.01	0.01\\
267.01	0.01\\
268.01	0.01\\
269.01	0.01\\
270.01	0.01\\
271.01	0.01\\
272.01	0.01\\
273.01	0.01\\
274.01	0.01\\
275.01	0.01\\
276.01	0.01\\
277.01	0.01\\
278.01	0.01\\
279.01	0.01\\
280.01	0.01\\
281.01	0.01\\
282.01	0.01\\
283.01	0.01\\
284.01	0.01\\
285.01	0.01\\
286.01	0.01\\
287.01	0.01\\
288.01	0.01\\
289.01	0.01\\
290.01	0.01\\
291.01	0.01\\
292.01	0.01\\
293.01	0.01\\
294.01	0.01\\
295.01	0.01\\
296.01	0.01\\
297.01	0.01\\
298.01	0.01\\
299.01	0.01\\
300.01	0.01\\
301.01	0.01\\
302.01	0.01\\
303.01	0.01\\
304.01	0.01\\
305.01	0.01\\
306.01	0.01\\
307.01	0.01\\
308.01	0.01\\
309.01	0.01\\
310.01	0.01\\
311.01	0.01\\
312.01	0.01\\
313.01	0.01\\
314.01	0.01\\
315.01	0.01\\
316.01	0.01\\
317.01	0.01\\
318.01	0.01\\
319.01	0.01\\
320.01	0.01\\
321.01	0.01\\
322.01	0.01\\
323.01	0.01\\
324.01	0.01\\
325.01	0.01\\
326.01	0.01\\
327.01	0.01\\
328.01	0.01\\
329.01	0.01\\
330.01	0.01\\
331.01	0.01\\
332.01	0.01\\
333.01	0.01\\
334.01	0.01\\
335.01	0.01\\
336.01	0.01\\
337.01	0.01\\
338.01	0.01\\
339.01	0.01\\
340.01	0.01\\
341.01	0.01\\
342.01	0.01\\
343.01	0.01\\
344.01	0.01\\
345.01	0.01\\
346.01	0.01\\
347.01	0.01\\
348.01	0.01\\
349.01	0.01\\
350.01	0.01\\
351.01	0.01\\
352.01	0.01\\
353.01	0.01\\
354.01	0.01\\
355.01	0.01\\
356.01	0.01\\
357.01	0.01\\
358.01	0.01\\
359.01	0.01\\
360.01	0.01\\
361.01	0.01\\
362.01	0.01\\
363.01	0.01\\
364.01	0.01\\
365.01	0.01\\
366.01	0.01\\
367.01	0.01\\
368.01	0.01\\
369.01	0.01\\
370.01	0.01\\
371.01	0.01\\
372.01	0.01\\
373.01	0.01\\
374.01	0.01\\
375.01	0.01\\
376.01	0.01\\
377.01	0.01\\
378.01	0.01\\
379.01	0.01\\
380.01	0.01\\
381.01	0.01\\
382.01	0.01\\
383.01	0.01\\
384.01	0.01\\
385.01	0.01\\
386.01	0.01\\
387.01	0.01\\
388.01	0.01\\
389.01	0.01\\
390.01	0.01\\
391.01	0.01\\
392.01	0.01\\
393.01	0.01\\
394.01	0.01\\
395.01	0.01\\
396.01	0.01\\
397.01	0.01\\
398.01	0.01\\
399.01	0.01\\
400.01	0.01\\
401.01	0.01\\
402.01	0.01\\
403.01	0.01\\
404.01	0.01\\
405.01	0.01\\
406.01	0.01\\
407.01	0.01\\
408.01	0.01\\
409.01	0.01\\
410.01	0.01\\
411.01	0.01\\
412.01	0.01\\
413.01	0.01\\
414.01	0.01\\
415.01	0.01\\
416.01	0.01\\
417.01	0.01\\
418.01	0.01\\
419.01	0.01\\
420.01	0.01\\
421.01	0.01\\
422.01	0.01\\
423.01	0.01\\
424.01	0.01\\
425.01	0.01\\
426.01	0.01\\
427.01	0.01\\
428.01	0.01\\
429.01	0.01\\
430.01	0.01\\
431.01	0.01\\
432.01	0.01\\
433.01	0.01\\
434.01	0.01\\
435.01	0.01\\
436.01	0.01\\
437.01	0.01\\
438.01	0.01\\
439.01	0.01\\
440.01	0.01\\
441.01	0.01\\
442.01	0.01\\
443.01	0.01\\
444.01	0.01\\
445.01	0.01\\
446.01	0.01\\
447.01	0.01\\
448.01	0.01\\
449.01	0.01\\
450.01	0.01\\
451.01	0.01\\
452.01	0.01\\
453.01	0.01\\
454.01	0.01\\
455.01	0.01\\
456.01	0.01\\
457.01	0.01\\
458.01	0.01\\
459.01	0.01\\
460.01	0.01\\
461.01	0.01\\
462.01	0.01\\
463.01	0.01\\
464.01	0.01\\
465.01	0.01\\
466.01	0.01\\
467.01	0.01\\
468.01	0.01\\
469.01	0.01\\
470.01	0.01\\
471.01	0.01\\
472.01	0.01\\
473.01	0.01\\
474.01	0.01\\
475.01	0.01\\
476.01	0.01\\
477.01	0.01\\
478.01	0.01\\
479.01	0.01\\
480.01	0.01\\
481.01	0.01\\
482.01	0.01\\
483.01	0.01\\
484.01	0.01\\
485.01	0.01\\
486.01	0.01\\
487.01	0.01\\
488.01	0.01\\
489.01	0.01\\
490.01	0.01\\
491.01	0.01\\
492.01	0.01\\
493.01	0.01\\
494.01	0.01\\
495.01	0.01\\
496.01	0.01\\
497.01	0.01\\
498.01	0.01\\
499.01	0.01\\
500.01	0.01\\
501.01	0.01\\
502.01	0.01\\
503.01	0.01\\
504.01	0.01\\
505.01	0.01\\
506.01	0.01\\
507.01	0.01\\
508.01	0.01\\
509.01	0.01\\
510.01	0.01\\
511.01	0.01\\
512.01	0.01\\
513.01	0.01\\
514.01	0.01\\
515.01	0.01\\
516.01	0.01\\
517.01	0.01\\
518.01	0.01\\
519.01	0.01\\
520.01	0.01\\
521.01	0.01\\
522.01	0.01\\
523.01	0.01\\
524.01	0.01\\
525.01	0.01\\
526.01	0.01\\
527.01	0.01\\
528.01	0.01\\
529.01	0.01\\
530.01	0.01\\
531.01	0.01\\
532.01	0.01\\
533.01	0.01\\
534.01	0.01\\
535.01	0.01\\
536.01	0.01\\
537.01	0.01\\
538.01	0.01\\
539.01	0.01\\
540.01	0.01\\
541.01	0.01\\
542.01	0.01\\
543.01	0.01\\
544.01	0.01\\
545.01	0.01\\
546.01	0.01\\
547.01	0.01\\
548.01	0.01\\
549.01	0.01\\
550.01	0.01\\
551.01	0.01\\
552.01	0.01\\
553.01	0.01\\
554.01	0.01\\
555.01	0.01\\
556.01	0.01\\
557.01	0.01\\
558.01	0.01\\
559.01	0.01\\
560.01	0.01\\
561.01	0.01\\
562.01	0.01\\
563.01	0.01\\
564.01	0.01\\
565.01	0.01\\
566.01	0.01\\
567.01	0.01\\
568.01	0.01\\
569.01	0.01\\
570.01	0.01\\
571.01	0.01\\
572.01	0.01\\
573.01	0.01\\
574.01	0.01\\
575.01	0.01\\
576.01	0.01\\
577.01	0.01\\
578.01	0.01\\
579.01	0.01\\
580.01	0.01\\
581.01	0.01\\
582.01	0.01\\
583.01	0.01\\
584.01	0.01\\
585.01	0.01\\
586.01	0.01\\
587.01	0.01\\
588.01	0.01\\
589.01	0.01\\
590.01	0.01\\
591.01	0.01\\
592.01	0.01\\
593.01	0.01\\
594.01	0.01\\
595.01	0.01\\
596.01	0.01\\
597.01	0.01\\
598.01	0.01\\
599.01	0.00629021083727977\\
599.02	0.006248202568395\\
599.03	0.00620593501768914\\
599.04	0.00616341592265327\\
599.05	0.00612065360930352\\
599.06	0.00607764614660782\\
599.07	0.00603440015398376\\
599.08	0.00599092538614726\\
599.09	0.00594722899179982\\
599.1	0.00590332194821352\\
599.11	0.00585921584344904\\
599.12	0.00581492271785932\\
599.13	0.00577045600772243\\
599.14	0.00572583003446701\\
599.15	0.00568106004996\\
599.16	0.00563616228426199\\
599.17	0.00559115399666164\\
599.18	0.00554605382720621\\
599.19	0.00550088126862895\\
599.2	0.00545565701151042\\
599.21	0.00541022061590217\\
599.22	0.00536434805706764\\
599.23	0.00531803513257949\\
599.24	0.00527127758210094\\
599.25	0.00522407109808666\\
599.26	0.00517641135752969\\
599.27	0.00512829397750254\\
599.28	0.00507971437988895\\
599.29	0.00503066794367191\\
599.3	0.004981150101723\\
599.31	0.00493115621964229\\
599.32	0.00488068159105543\\
599.33	0.00482972143447222\\
599.34	0.00477827088995449\\
599.35	0.00472632501558263\\
599.36	0.00467387878370851\\
599.37	0.00462092707698275\\
599.38	0.00456746468414297\\
599.39	0.00451348629554908\\
599.4	0.00445898649845102\\
599.41	0.00440395977782773\\
599.42	0.00434840051773466\\
599.43	0.00429230298406572\\
599.44	0.0042356613181626\\
599.45	0.00417846953050769\\
599.46	0.00412072149404243\\
599.47	0.00406241093708933\\
599.48	0.00400353143585462\\
599.49	0.00394407712271724\\
599.5	0.00388404237753246\\
599.51	0.00382342152498173\\
599.52	0.00376220883389879\\
599.53	0.00370039851658248\\
599.54	0.00363798472815847\\
599.55	0.00357496156595277\\
599.56	0.0035113230688626\\
599.57	0.00344706321673412\\
599.58	0.00338217592973297\\
599.59	0.00331665506770788\\
599.6	0.00325049442954719\\
599.61	0.00318368775252859\\
599.62	0.00311622871165683\\
599.63	0.00304811091899751\\
599.64	0.00297932792300593\\
599.65	0.00290987320784969\\
599.66	0.00283974019272552\\
599.67	0.00276892223117018\\
599.68	0.00269741261036607\\
599.69	0.00262520455044173\\
599.7	0.0025522912037677\\
599.71	0.00247866565424824\\
599.72	0.00240432091660945\\
599.73	0.00232924993568433\\
599.74	0.00225344558569628\\
599.75	0.00217690066954147\\
599.76	0.00209960791807056\\
599.77	0.00202155998937109\\
599.78	0.00194274946805127\\
599.79	0.0018631688645289\\
599.8	0.0017828106143252\\
599.81	0.00170166707736891\\
599.82	0.00161973053731143\\
599.83	0.00153699320085363\\
599.84	0.00145344719708646\\
599.85	0.00136908457684828\\
599.86	0.00128389731210149\\
599.87	0.001197877295332\\
599.88	0.00111101633897473\\
599.89	0.00102330617486933\\
599.9	0.000934738453750302\\
599.91	0.000845304744776263\\
599.92	0.000754996535103638\\
599.93	0.000663805229510536\\
599.94	0.000571722150077203\\
599.95	0.000478738535930024\\
599.96	0.000384845543056832\\
599.97	0.000290034244202044\\
599.98	0.00019429562885097\\
599.99	9.76206033136574e-05\\
600	0\\
};
\addplot [color=mycolor10,solid,forget plot]
  table[row sep=crcr]{%
0.01	0.01\\
1.01	0.01\\
2.01	0.01\\
3.01	0.01\\
4.01	0.01\\
5.01	0.01\\
6.01	0.01\\
7.01	0.01\\
8.01	0.01\\
9.01	0.01\\
10.01	0.01\\
11.01	0.01\\
12.01	0.01\\
13.01	0.01\\
14.01	0.01\\
15.01	0.01\\
16.01	0.01\\
17.01	0.01\\
18.01	0.01\\
19.01	0.01\\
20.01	0.01\\
21.01	0.01\\
22.01	0.01\\
23.01	0.01\\
24.01	0.01\\
25.01	0.01\\
26.01	0.01\\
27.01	0.01\\
28.01	0.01\\
29.01	0.01\\
30.01	0.01\\
31.01	0.01\\
32.01	0.01\\
33.01	0.01\\
34.01	0.01\\
35.01	0.01\\
36.01	0.01\\
37.01	0.01\\
38.01	0.01\\
39.01	0.01\\
40.01	0.01\\
41.01	0.01\\
42.01	0.01\\
43.01	0.01\\
44.01	0.01\\
45.01	0.01\\
46.01	0.01\\
47.01	0.01\\
48.01	0.01\\
49.01	0.01\\
50.01	0.01\\
51.01	0.01\\
52.01	0.01\\
53.01	0.01\\
54.01	0.01\\
55.01	0.01\\
56.01	0.01\\
57.01	0.01\\
58.01	0.01\\
59.01	0.01\\
60.01	0.01\\
61.01	0.01\\
62.01	0.01\\
63.01	0.01\\
64.01	0.01\\
65.01	0.01\\
66.01	0.01\\
67.01	0.01\\
68.01	0.01\\
69.01	0.01\\
70.01	0.01\\
71.01	0.01\\
72.01	0.01\\
73.01	0.01\\
74.01	0.01\\
75.01	0.01\\
76.01	0.01\\
77.01	0.01\\
78.01	0.01\\
79.01	0.01\\
80.01	0.01\\
81.01	0.01\\
82.01	0.01\\
83.01	0.01\\
84.01	0.01\\
85.01	0.01\\
86.01	0.01\\
87.01	0.01\\
88.01	0.01\\
89.01	0.01\\
90.01	0.01\\
91.01	0.01\\
92.01	0.01\\
93.01	0.01\\
94.01	0.01\\
95.01	0.01\\
96.01	0.01\\
97.01	0.01\\
98.01	0.01\\
99.01	0.01\\
100.01	0.01\\
101.01	0.01\\
102.01	0.01\\
103.01	0.01\\
104.01	0.01\\
105.01	0.01\\
106.01	0.01\\
107.01	0.01\\
108.01	0.01\\
109.01	0.01\\
110.01	0.01\\
111.01	0.01\\
112.01	0.01\\
113.01	0.01\\
114.01	0.01\\
115.01	0.01\\
116.01	0.01\\
117.01	0.01\\
118.01	0.01\\
119.01	0.01\\
120.01	0.01\\
121.01	0.01\\
122.01	0.01\\
123.01	0.01\\
124.01	0.01\\
125.01	0.01\\
126.01	0.01\\
127.01	0.01\\
128.01	0.01\\
129.01	0.01\\
130.01	0.01\\
131.01	0.01\\
132.01	0.01\\
133.01	0.01\\
134.01	0.01\\
135.01	0.01\\
136.01	0.01\\
137.01	0.01\\
138.01	0.01\\
139.01	0.01\\
140.01	0.01\\
141.01	0.01\\
142.01	0.01\\
143.01	0.01\\
144.01	0.01\\
145.01	0.01\\
146.01	0.01\\
147.01	0.01\\
148.01	0.01\\
149.01	0.01\\
150.01	0.01\\
151.01	0.01\\
152.01	0.01\\
153.01	0.01\\
154.01	0.01\\
155.01	0.01\\
156.01	0.01\\
157.01	0.01\\
158.01	0.01\\
159.01	0.01\\
160.01	0.01\\
161.01	0.01\\
162.01	0.01\\
163.01	0.01\\
164.01	0.01\\
165.01	0.01\\
166.01	0.01\\
167.01	0.01\\
168.01	0.01\\
169.01	0.01\\
170.01	0.01\\
171.01	0.01\\
172.01	0.01\\
173.01	0.01\\
174.01	0.01\\
175.01	0.01\\
176.01	0.01\\
177.01	0.01\\
178.01	0.01\\
179.01	0.01\\
180.01	0.01\\
181.01	0.01\\
182.01	0.01\\
183.01	0.01\\
184.01	0.01\\
185.01	0.01\\
186.01	0.01\\
187.01	0.01\\
188.01	0.01\\
189.01	0.01\\
190.01	0.01\\
191.01	0.01\\
192.01	0.01\\
193.01	0.01\\
194.01	0.01\\
195.01	0.01\\
196.01	0.01\\
197.01	0.01\\
198.01	0.01\\
199.01	0.01\\
200.01	0.01\\
201.01	0.01\\
202.01	0.01\\
203.01	0.01\\
204.01	0.01\\
205.01	0.01\\
206.01	0.01\\
207.01	0.01\\
208.01	0.01\\
209.01	0.01\\
210.01	0.01\\
211.01	0.01\\
212.01	0.01\\
213.01	0.01\\
214.01	0.01\\
215.01	0.01\\
216.01	0.01\\
217.01	0.01\\
218.01	0.01\\
219.01	0.01\\
220.01	0.01\\
221.01	0.01\\
222.01	0.01\\
223.01	0.01\\
224.01	0.01\\
225.01	0.01\\
226.01	0.01\\
227.01	0.01\\
228.01	0.01\\
229.01	0.01\\
230.01	0.01\\
231.01	0.01\\
232.01	0.01\\
233.01	0.01\\
234.01	0.01\\
235.01	0.01\\
236.01	0.01\\
237.01	0.01\\
238.01	0.01\\
239.01	0.01\\
240.01	0.01\\
241.01	0.01\\
242.01	0.01\\
243.01	0.01\\
244.01	0.01\\
245.01	0.01\\
246.01	0.01\\
247.01	0.01\\
248.01	0.01\\
249.01	0.01\\
250.01	0.01\\
251.01	0.01\\
252.01	0.01\\
253.01	0.01\\
254.01	0.01\\
255.01	0.01\\
256.01	0.01\\
257.01	0.01\\
258.01	0.01\\
259.01	0.01\\
260.01	0.01\\
261.01	0.01\\
262.01	0.01\\
263.01	0.01\\
264.01	0.01\\
265.01	0.01\\
266.01	0.01\\
267.01	0.01\\
268.01	0.01\\
269.01	0.01\\
270.01	0.01\\
271.01	0.01\\
272.01	0.01\\
273.01	0.01\\
274.01	0.01\\
275.01	0.01\\
276.01	0.01\\
277.01	0.01\\
278.01	0.01\\
279.01	0.01\\
280.01	0.01\\
281.01	0.01\\
282.01	0.01\\
283.01	0.01\\
284.01	0.01\\
285.01	0.01\\
286.01	0.01\\
287.01	0.01\\
288.01	0.01\\
289.01	0.01\\
290.01	0.01\\
291.01	0.01\\
292.01	0.01\\
293.01	0.01\\
294.01	0.01\\
295.01	0.01\\
296.01	0.01\\
297.01	0.01\\
298.01	0.01\\
299.01	0.01\\
300.01	0.01\\
301.01	0.01\\
302.01	0.01\\
303.01	0.01\\
304.01	0.01\\
305.01	0.01\\
306.01	0.01\\
307.01	0.01\\
308.01	0.01\\
309.01	0.01\\
310.01	0.01\\
311.01	0.01\\
312.01	0.01\\
313.01	0.01\\
314.01	0.01\\
315.01	0.01\\
316.01	0.01\\
317.01	0.01\\
318.01	0.01\\
319.01	0.01\\
320.01	0.01\\
321.01	0.01\\
322.01	0.01\\
323.01	0.01\\
324.01	0.01\\
325.01	0.01\\
326.01	0.01\\
327.01	0.01\\
328.01	0.01\\
329.01	0.01\\
330.01	0.01\\
331.01	0.01\\
332.01	0.01\\
333.01	0.01\\
334.01	0.01\\
335.01	0.01\\
336.01	0.01\\
337.01	0.01\\
338.01	0.01\\
339.01	0.01\\
340.01	0.01\\
341.01	0.01\\
342.01	0.01\\
343.01	0.01\\
344.01	0.01\\
345.01	0.01\\
346.01	0.01\\
347.01	0.01\\
348.01	0.01\\
349.01	0.01\\
350.01	0.01\\
351.01	0.01\\
352.01	0.01\\
353.01	0.01\\
354.01	0.01\\
355.01	0.01\\
356.01	0.01\\
357.01	0.01\\
358.01	0.01\\
359.01	0.01\\
360.01	0.01\\
361.01	0.01\\
362.01	0.01\\
363.01	0.01\\
364.01	0.01\\
365.01	0.01\\
366.01	0.01\\
367.01	0.01\\
368.01	0.01\\
369.01	0.01\\
370.01	0.01\\
371.01	0.01\\
372.01	0.01\\
373.01	0.01\\
374.01	0.01\\
375.01	0.01\\
376.01	0.01\\
377.01	0.01\\
378.01	0.01\\
379.01	0.01\\
380.01	0.01\\
381.01	0.01\\
382.01	0.01\\
383.01	0.01\\
384.01	0.01\\
385.01	0.01\\
386.01	0.01\\
387.01	0.01\\
388.01	0.01\\
389.01	0.01\\
390.01	0.01\\
391.01	0.01\\
392.01	0.01\\
393.01	0.01\\
394.01	0.01\\
395.01	0.01\\
396.01	0.01\\
397.01	0.01\\
398.01	0.01\\
399.01	0.01\\
400.01	0.01\\
401.01	0.01\\
402.01	0.01\\
403.01	0.01\\
404.01	0.01\\
405.01	0.01\\
406.01	0.01\\
407.01	0.01\\
408.01	0.01\\
409.01	0.01\\
410.01	0.01\\
411.01	0.01\\
412.01	0.01\\
413.01	0.01\\
414.01	0.01\\
415.01	0.01\\
416.01	0.01\\
417.01	0.01\\
418.01	0.01\\
419.01	0.01\\
420.01	0.01\\
421.01	0.01\\
422.01	0.01\\
423.01	0.01\\
424.01	0.01\\
425.01	0.01\\
426.01	0.01\\
427.01	0.01\\
428.01	0.01\\
429.01	0.01\\
430.01	0.01\\
431.01	0.01\\
432.01	0.01\\
433.01	0.01\\
434.01	0.01\\
435.01	0.01\\
436.01	0.01\\
437.01	0.01\\
438.01	0.01\\
439.01	0.01\\
440.01	0.01\\
441.01	0.01\\
442.01	0.01\\
443.01	0.01\\
444.01	0.01\\
445.01	0.01\\
446.01	0.01\\
447.01	0.01\\
448.01	0.01\\
449.01	0.01\\
450.01	0.01\\
451.01	0.01\\
452.01	0.01\\
453.01	0.01\\
454.01	0.01\\
455.01	0.01\\
456.01	0.01\\
457.01	0.01\\
458.01	0.01\\
459.01	0.01\\
460.01	0.01\\
461.01	0.01\\
462.01	0.01\\
463.01	0.01\\
464.01	0.01\\
465.01	0.01\\
466.01	0.01\\
467.01	0.01\\
468.01	0.01\\
469.01	0.01\\
470.01	0.01\\
471.01	0.01\\
472.01	0.01\\
473.01	0.01\\
474.01	0.01\\
475.01	0.01\\
476.01	0.01\\
477.01	0.01\\
478.01	0.01\\
479.01	0.01\\
480.01	0.01\\
481.01	0.01\\
482.01	0.01\\
483.01	0.01\\
484.01	0.01\\
485.01	0.01\\
486.01	0.01\\
487.01	0.01\\
488.01	0.01\\
489.01	0.01\\
490.01	0.01\\
491.01	0.01\\
492.01	0.01\\
493.01	0.01\\
494.01	0.01\\
495.01	0.01\\
496.01	0.01\\
497.01	0.01\\
498.01	0.01\\
499.01	0.01\\
500.01	0.01\\
501.01	0.01\\
502.01	0.01\\
503.01	0.01\\
504.01	0.01\\
505.01	0.01\\
506.01	0.01\\
507.01	0.01\\
508.01	0.01\\
509.01	0.01\\
510.01	0.01\\
511.01	0.01\\
512.01	0.01\\
513.01	0.01\\
514.01	0.01\\
515.01	0.01\\
516.01	0.01\\
517.01	0.01\\
518.01	0.01\\
519.01	0.01\\
520.01	0.01\\
521.01	0.01\\
522.01	0.01\\
523.01	0.01\\
524.01	0.01\\
525.01	0.01\\
526.01	0.01\\
527.01	0.01\\
528.01	0.01\\
529.01	0.01\\
530.01	0.01\\
531.01	0.01\\
532.01	0.01\\
533.01	0.01\\
534.01	0.01\\
535.01	0.01\\
536.01	0.01\\
537.01	0.01\\
538.01	0.01\\
539.01	0.01\\
540.01	0.01\\
541.01	0.01\\
542.01	0.01\\
543.01	0.01\\
544.01	0.01\\
545.01	0.01\\
546.01	0.01\\
547.01	0.01\\
548.01	0.01\\
549.01	0.01\\
550.01	0.01\\
551.01	0.01\\
552.01	0.01\\
553.01	0.01\\
554.01	0.01\\
555.01	0.01\\
556.01	0.01\\
557.01	0.01\\
558.01	0.01\\
559.01	0.01\\
560.01	0.01\\
561.01	0.01\\
562.01	0.01\\
563.01	0.01\\
564.01	0.01\\
565.01	0.01\\
566.01	0.01\\
567.01	0.01\\
568.01	0.01\\
569.01	0.01\\
570.01	0.01\\
571.01	0.01\\
572.01	0.01\\
573.01	0.01\\
574.01	0.01\\
575.01	0.01\\
576.01	0.01\\
577.01	0.01\\
578.01	0.01\\
579.01	0.01\\
580.01	0.01\\
581.01	0.01\\
582.01	0.01\\
583.01	0.01\\
584.01	0.01\\
585.01	0.01\\
586.01	0.01\\
587.01	0.01\\
588.01	0.01\\
589.01	0.01\\
590.01	0.01\\
591.01	0.01\\
592.01	0.01\\
593.01	0.01\\
594.01	0.01\\
595.01	0.01\\
596.01	0.01\\
597.01	0.01\\
598.01	0.01\\
599.01	0.00987101508599242\\
599.02	0.00966435710912938\\
599.03	0.00945619745033124\\
599.04	0.00924651414318886\\
599.05	0.00903528450781953\\
599.06	0.00882249594459827\\
599.07	0.0086081271628761\\
599.08	0.0083921536080565\\
599.09	0.00817455318581161\\
599.1	0.00795529996678749\\
599.11	0.00773436739351956\\
599.12	0.00751172833030985\\
599.13	0.00728735412338303\\
599.14	0.00706121511304294\\
599.15	0.00683328058997119\\
599.16	0.00660351874927539\\
599.17	0.00637189664212814\\
599.18	0.00613838011978858\\
599.19	0.00590293378481197\\
599.2	0.00566552093448356\\
599.21	0.00550985573121098\\
599.22	0.00545922463758062\\
599.23	0.00540818945802247\\
599.24	0.00535675333215924\\
599.25	0.00530491703217995\\
599.26	0.00525267475303119\\
599.27	0.00520003038725458\\
599.28	0.00514698602738943\\
599.29	0.00509354574416965\\
599.3	0.00503971428484819\\
599.31	0.00498549690289976\\
599.32	0.00493089970370655\\
599.33	0.00487592932800012\\
599.34	0.00482059297849655\\
599.35	0.0047648984478651\\
599.36	0.00470885414810968\\
599.37	0.00465246914144833\\
599.38	0.00459575317278141\\
599.39	0.00453871670384681\\
599.4	0.00448137094916667\\
599.41	0.00442372791380806\\
599.42	0.00436580029185107\\
599.43	0.00430760172657727\\
599.44	0.00424914685408985\\
599.45	0.00419045127414728\\
599.46	0.00413153159953789\\
599.47	0.00407240550824231\\
599.48	0.00401309179857429\\
599.49	0.00395332722398739\\
599.5	0.0038929821849682\\
599.51	0.0038320512203846\\
599.52	0.00377052884641714\\
599.53	0.00370840955592181\\
599.54	0.00364568780239088\\
599.55	0.00358235799429536\\
599.56	0.00351841449275467\\
599.57	0.00345385160745471\\
599.58	0.00338866359564696\\
599.59	0.00332284466108848\\
599.6	0.00325638895291892\\
599.61	0.0031892905644707\\
599.62	0.00312154353299735\\
599.63	0.00305314183776542\\
599.64	0.00298407939834351\\
599.65	0.00291435007302961\\
599.66	0.00284394765722354\\
599.67	0.0027728658817277\\
599.68	0.00270109841094376\\
599.69	0.00262863884095882\\
599.7	0.00255548069751408\\
599.71	0.00248161743384854\\
599.72	0.00240704242840991\\
599.73	0.00233174898242438\\
599.74	0.00225573031705331\\
599.75	0.00217897957054852\\
599.76	0.00210148979528493\\
599.77	0.00202325395460674\\
599.78	0.00194426491947531\\
599.79	0.00186451546417297\\
599.8	0.00178399826230015\\
599.81	0.00170270588132737\\
599.82	0.00162063077725271\\
599.83	0.00153776528948584\\
599.84	0.00145410163537975\\
599.85	0.001369631904385\\
599.86	0.00128434805179975\\
599.87	0.00119824189208689\\
599.88	0.00111130509172728\\
599.89	0.00102352916157608\\
599.9	0.00093490544868644\\
599.91	0.000845425127562385\\
599.92	0.000755079190799706\\
599.93	0.000663858439070786\\
599.94	0.000571753470405756\\
599.95	0.000478754668719049\\
599.96	0.000384852191526369\\
599.97	0.000290035956793153\\
599.98	0.00019429562885097\\
599.99	9.76206033136556e-05\\
600	0\\
};
\addplot [color=mycolor11,solid,forget plot]
  table[row sep=crcr]{%
0.01	0.01\\
1.01	0.01\\
2.01	0.01\\
3.01	0.01\\
4.01	0.01\\
5.01	0.01\\
6.01	0.01\\
7.01	0.01\\
8.01	0.01\\
9.01	0.01\\
10.01	0.01\\
11.01	0.01\\
12.01	0.01\\
13.01	0.01\\
14.01	0.01\\
15.01	0.01\\
16.01	0.01\\
17.01	0.01\\
18.01	0.01\\
19.01	0.01\\
20.01	0.01\\
21.01	0.01\\
22.01	0.01\\
23.01	0.01\\
24.01	0.01\\
25.01	0.01\\
26.01	0.01\\
27.01	0.01\\
28.01	0.01\\
29.01	0.01\\
30.01	0.01\\
31.01	0.01\\
32.01	0.01\\
33.01	0.01\\
34.01	0.01\\
35.01	0.01\\
36.01	0.01\\
37.01	0.01\\
38.01	0.01\\
39.01	0.01\\
40.01	0.01\\
41.01	0.01\\
42.01	0.01\\
43.01	0.01\\
44.01	0.01\\
45.01	0.01\\
46.01	0.01\\
47.01	0.01\\
48.01	0.01\\
49.01	0.01\\
50.01	0.01\\
51.01	0.01\\
52.01	0.01\\
53.01	0.01\\
54.01	0.01\\
55.01	0.01\\
56.01	0.01\\
57.01	0.01\\
58.01	0.01\\
59.01	0.01\\
60.01	0.01\\
61.01	0.01\\
62.01	0.01\\
63.01	0.01\\
64.01	0.01\\
65.01	0.01\\
66.01	0.01\\
67.01	0.01\\
68.01	0.01\\
69.01	0.01\\
70.01	0.01\\
71.01	0.01\\
72.01	0.01\\
73.01	0.01\\
74.01	0.01\\
75.01	0.01\\
76.01	0.01\\
77.01	0.01\\
78.01	0.01\\
79.01	0.01\\
80.01	0.01\\
81.01	0.01\\
82.01	0.01\\
83.01	0.01\\
84.01	0.01\\
85.01	0.01\\
86.01	0.01\\
87.01	0.01\\
88.01	0.01\\
89.01	0.01\\
90.01	0.01\\
91.01	0.01\\
92.01	0.01\\
93.01	0.01\\
94.01	0.01\\
95.01	0.01\\
96.01	0.01\\
97.01	0.01\\
98.01	0.01\\
99.01	0.01\\
100.01	0.01\\
101.01	0.01\\
102.01	0.01\\
103.01	0.01\\
104.01	0.01\\
105.01	0.01\\
106.01	0.01\\
107.01	0.01\\
108.01	0.01\\
109.01	0.01\\
110.01	0.01\\
111.01	0.01\\
112.01	0.01\\
113.01	0.01\\
114.01	0.01\\
115.01	0.01\\
116.01	0.01\\
117.01	0.01\\
118.01	0.01\\
119.01	0.01\\
120.01	0.01\\
121.01	0.01\\
122.01	0.01\\
123.01	0.01\\
124.01	0.01\\
125.01	0.01\\
126.01	0.01\\
127.01	0.01\\
128.01	0.01\\
129.01	0.01\\
130.01	0.01\\
131.01	0.01\\
132.01	0.01\\
133.01	0.01\\
134.01	0.01\\
135.01	0.01\\
136.01	0.01\\
137.01	0.01\\
138.01	0.01\\
139.01	0.01\\
140.01	0.01\\
141.01	0.01\\
142.01	0.01\\
143.01	0.01\\
144.01	0.01\\
145.01	0.01\\
146.01	0.01\\
147.01	0.01\\
148.01	0.01\\
149.01	0.01\\
150.01	0.01\\
151.01	0.01\\
152.01	0.01\\
153.01	0.01\\
154.01	0.01\\
155.01	0.01\\
156.01	0.01\\
157.01	0.01\\
158.01	0.01\\
159.01	0.01\\
160.01	0.01\\
161.01	0.01\\
162.01	0.01\\
163.01	0.01\\
164.01	0.01\\
165.01	0.01\\
166.01	0.01\\
167.01	0.01\\
168.01	0.01\\
169.01	0.01\\
170.01	0.01\\
171.01	0.01\\
172.01	0.01\\
173.01	0.01\\
174.01	0.01\\
175.01	0.01\\
176.01	0.01\\
177.01	0.01\\
178.01	0.01\\
179.01	0.01\\
180.01	0.01\\
181.01	0.01\\
182.01	0.01\\
183.01	0.01\\
184.01	0.01\\
185.01	0.01\\
186.01	0.01\\
187.01	0.01\\
188.01	0.01\\
189.01	0.01\\
190.01	0.01\\
191.01	0.01\\
192.01	0.01\\
193.01	0.01\\
194.01	0.01\\
195.01	0.01\\
196.01	0.01\\
197.01	0.01\\
198.01	0.01\\
199.01	0.01\\
200.01	0.01\\
201.01	0.01\\
202.01	0.01\\
203.01	0.01\\
204.01	0.01\\
205.01	0.01\\
206.01	0.01\\
207.01	0.01\\
208.01	0.01\\
209.01	0.01\\
210.01	0.01\\
211.01	0.01\\
212.01	0.01\\
213.01	0.01\\
214.01	0.01\\
215.01	0.01\\
216.01	0.01\\
217.01	0.01\\
218.01	0.01\\
219.01	0.01\\
220.01	0.01\\
221.01	0.01\\
222.01	0.01\\
223.01	0.01\\
224.01	0.01\\
225.01	0.01\\
226.01	0.01\\
227.01	0.01\\
228.01	0.01\\
229.01	0.01\\
230.01	0.01\\
231.01	0.01\\
232.01	0.01\\
233.01	0.01\\
234.01	0.01\\
235.01	0.01\\
236.01	0.01\\
237.01	0.01\\
238.01	0.01\\
239.01	0.01\\
240.01	0.01\\
241.01	0.01\\
242.01	0.01\\
243.01	0.01\\
244.01	0.01\\
245.01	0.01\\
246.01	0.01\\
247.01	0.01\\
248.01	0.01\\
249.01	0.01\\
250.01	0.01\\
251.01	0.01\\
252.01	0.01\\
253.01	0.01\\
254.01	0.01\\
255.01	0.01\\
256.01	0.01\\
257.01	0.01\\
258.01	0.01\\
259.01	0.01\\
260.01	0.01\\
261.01	0.01\\
262.01	0.01\\
263.01	0.01\\
264.01	0.01\\
265.01	0.01\\
266.01	0.01\\
267.01	0.01\\
268.01	0.01\\
269.01	0.01\\
270.01	0.01\\
271.01	0.01\\
272.01	0.01\\
273.01	0.01\\
274.01	0.01\\
275.01	0.01\\
276.01	0.01\\
277.01	0.01\\
278.01	0.01\\
279.01	0.01\\
280.01	0.01\\
281.01	0.01\\
282.01	0.01\\
283.01	0.01\\
284.01	0.01\\
285.01	0.01\\
286.01	0.01\\
287.01	0.01\\
288.01	0.01\\
289.01	0.01\\
290.01	0.01\\
291.01	0.01\\
292.01	0.01\\
293.01	0.01\\
294.01	0.01\\
295.01	0.01\\
296.01	0.01\\
297.01	0.01\\
298.01	0.01\\
299.01	0.01\\
300.01	0.01\\
301.01	0.01\\
302.01	0.01\\
303.01	0.01\\
304.01	0.01\\
305.01	0.01\\
306.01	0.01\\
307.01	0.01\\
308.01	0.01\\
309.01	0.01\\
310.01	0.01\\
311.01	0.01\\
312.01	0.01\\
313.01	0.01\\
314.01	0.01\\
315.01	0.01\\
316.01	0.01\\
317.01	0.01\\
318.01	0.01\\
319.01	0.01\\
320.01	0.01\\
321.01	0.01\\
322.01	0.01\\
323.01	0.01\\
324.01	0.01\\
325.01	0.01\\
326.01	0.01\\
327.01	0.01\\
328.01	0.01\\
329.01	0.01\\
330.01	0.01\\
331.01	0.01\\
332.01	0.01\\
333.01	0.01\\
334.01	0.01\\
335.01	0.01\\
336.01	0.01\\
337.01	0.01\\
338.01	0.01\\
339.01	0.01\\
340.01	0.01\\
341.01	0.01\\
342.01	0.01\\
343.01	0.01\\
344.01	0.01\\
345.01	0.01\\
346.01	0.01\\
347.01	0.01\\
348.01	0.01\\
349.01	0.01\\
350.01	0.01\\
351.01	0.01\\
352.01	0.01\\
353.01	0.01\\
354.01	0.01\\
355.01	0.01\\
356.01	0.01\\
357.01	0.01\\
358.01	0.01\\
359.01	0.01\\
360.01	0.01\\
361.01	0.01\\
362.01	0.01\\
363.01	0.01\\
364.01	0.01\\
365.01	0.01\\
366.01	0.01\\
367.01	0.01\\
368.01	0.01\\
369.01	0.01\\
370.01	0.01\\
371.01	0.01\\
372.01	0.01\\
373.01	0.01\\
374.01	0.01\\
375.01	0.01\\
376.01	0.01\\
377.01	0.01\\
378.01	0.01\\
379.01	0.01\\
380.01	0.01\\
381.01	0.01\\
382.01	0.01\\
383.01	0.01\\
384.01	0.01\\
385.01	0.01\\
386.01	0.01\\
387.01	0.01\\
388.01	0.01\\
389.01	0.01\\
390.01	0.01\\
391.01	0.01\\
392.01	0.01\\
393.01	0.01\\
394.01	0.01\\
395.01	0.01\\
396.01	0.01\\
397.01	0.01\\
398.01	0.01\\
399.01	0.01\\
400.01	0.01\\
401.01	0.01\\
402.01	0.01\\
403.01	0.01\\
404.01	0.01\\
405.01	0.01\\
406.01	0.01\\
407.01	0.01\\
408.01	0.01\\
409.01	0.01\\
410.01	0.01\\
411.01	0.01\\
412.01	0.01\\
413.01	0.01\\
414.01	0.01\\
415.01	0.01\\
416.01	0.01\\
417.01	0.01\\
418.01	0.01\\
419.01	0.01\\
420.01	0.01\\
421.01	0.01\\
422.01	0.01\\
423.01	0.01\\
424.01	0.01\\
425.01	0.01\\
426.01	0.01\\
427.01	0.01\\
428.01	0.01\\
429.01	0.01\\
430.01	0.01\\
431.01	0.01\\
432.01	0.01\\
433.01	0.01\\
434.01	0.01\\
435.01	0.01\\
436.01	0.01\\
437.01	0.01\\
438.01	0.01\\
439.01	0.01\\
440.01	0.01\\
441.01	0.01\\
442.01	0.01\\
443.01	0.01\\
444.01	0.01\\
445.01	0.01\\
446.01	0.01\\
447.01	0.01\\
448.01	0.01\\
449.01	0.01\\
450.01	0.01\\
451.01	0.01\\
452.01	0.01\\
453.01	0.01\\
454.01	0.01\\
455.01	0.01\\
456.01	0.01\\
457.01	0.01\\
458.01	0.01\\
459.01	0.01\\
460.01	0.01\\
461.01	0.01\\
462.01	0.01\\
463.01	0.01\\
464.01	0.01\\
465.01	0.01\\
466.01	0.01\\
467.01	0.01\\
468.01	0.01\\
469.01	0.01\\
470.01	0.01\\
471.01	0.01\\
472.01	0.01\\
473.01	0.01\\
474.01	0.01\\
475.01	0.01\\
476.01	0.01\\
477.01	0.01\\
478.01	0.01\\
479.01	0.01\\
480.01	0.01\\
481.01	0.01\\
482.01	0.01\\
483.01	0.01\\
484.01	0.01\\
485.01	0.01\\
486.01	0.01\\
487.01	0.01\\
488.01	0.01\\
489.01	0.01\\
490.01	0.01\\
491.01	0.01\\
492.01	0.01\\
493.01	0.01\\
494.01	0.01\\
495.01	0.01\\
496.01	0.01\\
497.01	0.01\\
498.01	0.01\\
499.01	0.01\\
500.01	0.01\\
501.01	0.01\\
502.01	0.01\\
503.01	0.01\\
504.01	0.01\\
505.01	0.01\\
506.01	0.01\\
507.01	0.01\\
508.01	0.01\\
509.01	0.01\\
510.01	0.01\\
511.01	0.01\\
512.01	0.01\\
513.01	0.01\\
514.01	0.01\\
515.01	0.01\\
516.01	0.01\\
517.01	0.01\\
518.01	0.01\\
519.01	0.01\\
520.01	0.01\\
521.01	0.01\\
522.01	0.01\\
523.01	0.01\\
524.01	0.01\\
525.01	0.01\\
526.01	0.01\\
527.01	0.01\\
528.01	0.01\\
529.01	0.01\\
530.01	0.01\\
531.01	0.01\\
532.01	0.01\\
533.01	0.01\\
534.01	0.01\\
535.01	0.01\\
536.01	0.01\\
537.01	0.01\\
538.01	0.01\\
539.01	0.01\\
540.01	0.01\\
541.01	0.01\\
542.01	0.01\\
543.01	0.01\\
544.01	0.01\\
545.01	0.01\\
546.01	0.01\\
547.01	0.01\\
548.01	0.01\\
549.01	0.01\\
550.01	0.01\\
551.01	0.01\\
552.01	0.01\\
553.01	0.01\\
554.01	0.01\\
555.01	0.01\\
556.01	0.01\\
557.01	0.01\\
558.01	0.01\\
559.01	0.01\\
560.01	0.01\\
561.01	0.01\\
562.01	0.01\\
563.01	0.01\\
564.01	0.01\\
565.01	0.01\\
566.01	0.01\\
567.01	0.01\\
568.01	0.01\\
569.01	0.01\\
570.01	0.01\\
571.01	0.01\\
572.01	0.01\\
573.01	0.01\\
574.01	0.01\\
575.01	0.01\\
576.01	0.01\\
577.01	0.01\\
578.01	0.01\\
579.01	0.01\\
580.01	0.01\\
581.01	0.01\\
582.01	0.01\\
583.01	0.01\\
584.01	0.01\\
585.01	0.01\\
586.01	0.01\\
587.01	0.01\\
588.01	0.01\\
589.01	0.01\\
590.01	0.01\\
591.01	0.01\\
592.01	0.01\\
593.01	0.01\\
594.01	0.01\\
595.01	0.01\\
596.01	0.01\\
597.01	0.01\\
598.01	0.01\\
599.01	0.01\\
599.02	0.01\\
599.03	0.01\\
599.04	0.01\\
599.05	0.01\\
599.06	0.01\\
599.07	0.01\\
599.08	0.01\\
599.09	0.01\\
599.1	0.01\\
599.11	0.01\\
599.12	0.01\\
599.13	0.01\\
599.14	0.01\\
599.15	0.01\\
599.16	0.01\\
599.17	0.01\\
599.18	0.01\\
599.19	0.01\\
599.2	0.01\\
599.21	0.00991642967089183\\
599.22	0.00972620919023257\\
599.23	0.00953476336815602\\
599.24	0.00934207553102435\\
599.25	0.0091481312066784\\
599.26	0.00895292229058141\\
599.27	0.00875643082365402\\
599.28	0.00855864061331242\\
599.29	0.00835953330433368\\
599.3	0.00815908957986898\\
599.31	0.00795728944928943\\
599.32	0.00775411195876306\\
599.33	0.00754953562889951\\
599.34	0.00734353825683295\\
599.35	0.00713609688778657\\
599.36	0.0069271877853486\\
599.37	0.00671678640038438\\
599.38	0.00650486733850286\\
599.39	0.00629140432599014\\
599.4	0.00607637017411562\\
599.41	0.00585973674171082\\
599.42	0.00564147504834083\\
599.43	0.00542155500704852\\
599.44	0.00519994537848073\\
599.45	0.0049766138012967\\
599.46	0.00475152674427759\\
599.47	0.00452464945590616\\
599.48	0.00429594591124254\\
599.49	0.00417928104780683\\
599.5	0.00411405932776891\\
599.51	0.00404820617347344\\
599.52	0.00398171253686825\\
599.53	0.00391456978350743\\
599.54	0.00384677251359477\\
599.55	0.00377831638301909\\
599.56	0.00370919748084717\\
599.57	0.00363941256110831\\
599.58	0.00356895850446607\\
599.59	0.0034978323251514\\
599.6	0.00342603117816287\\
599.61	0.0033535523667451\\
599.62	0.00328039320558156\\
599.63	0.00320655126747823\\
599.64	0.00313202434399071\\
599.65	0.00305681043605939\\
599.66	0.00298090774846901\\
599.67	0.0029043146896599\\
599.68	0.00282702988180877\\
599.69	0.00274905217132293\\
599.7	0.00267038063976806\\
599.71	0.00259101461525092\\
599.72	0.00251095368427965\\
599.73	0.00243019770412575\\
599.74	0.00234874690447312\\
599.75	0.00226660184212179\\
599.76	0.00218376339726204\\
599.77	0.00210023278902947\\
599.78	0.00201601159177037\\
599.79	0.00193110198940867\\
599.8	0.0018455066110789\\
599.81	0.00175922892935269\\
599.82	0.00167227313610708\\
599.83	0.00158464398574982\\
599.84	0.00149634682616875\\
599.85	0.00140738763146645\\
599.86	0.00131777303660006\\
599.87	0.00122751037405548\\
599.88	0.0011366077126951\\
599.89	0.00104507389892841\\
599.9	0.000952918600367008\\
599.91	0.000860152352137411\\
599.92	0.000766786606038976\\
599.93	0.000672833782748917\\
599.94	0.000578307327292129\\
599.95	0.000483221768011027\\
599.96	0.000387592779289264\\
599.97	0.000291437248303513\\
599.98	0.000194773346099669\\
599.99	9.76206033136556e-05\\
600	0\\
};
\addplot [color=mycolor12,solid,forget plot]
  table[row sep=crcr]{%
0.01	0.01\\
1.01	0.01\\
2.01	0.01\\
3.01	0.01\\
4.01	0.01\\
5.01	0.01\\
6.01	0.01\\
7.01	0.01\\
8.01	0.01\\
9.01	0.01\\
10.01	0.01\\
11.01	0.01\\
12.01	0.01\\
13.01	0.01\\
14.01	0.01\\
15.01	0.01\\
16.01	0.01\\
17.01	0.01\\
18.01	0.01\\
19.01	0.01\\
20.01	0.01\\
21.01	0.01\\
22.01	0.01\\
23.01	0.01\\
24.01	0.01\\
25.01	0.01\\
26.01	0.01\\
27.01	0.01\\
28.01	0.01\\
29.01	0.01\\
30.01	0.01\\
31.01	0.01\\
32.01	0.01\\
33.01	0.01\\
34.01	0.01\\
35.01	0.01\\
36.01	0.01\\
37.01	0.01\\
38.01	0.01\\
39.01	0.01\\
40.01	0.01\\
41.01	0.01\\
42.01	0.01\\
43.01	0.01\\
44.01	0.01\\
45.01	0.01\\
46.01	0.01\\
47.01	0.01\\
48.01	0.01\\
49.01	0.01\\
50.01	0.01\\
51.01	0.01\\
52.01	0.01\\
53.01	0.01\\
54.01	0.01\\
55.01	0.01\\
56.01	0.01\\
57.01	0.01\\
58.01	0.01\\
59.01	0.01\\
60.01	0.01\\
61.01	0.01\\
62.01	0.01\\
63.01	0.01\\
64.01	0.01\\
65.01	0.01\\
66.01	0.01\\
67.01	0.01\\
68.01	0.01\\
69.01	0.01\\
70.01	0.01\\
71.01	0.01\\
72.01	0.01\\
73.01	0.01\\
74.01	0.01\\
75.01	0.01\\
76.01	0.01\\
77.01	0.01\\
78.01	0.01\\
79.01	0.01\\
80.01	0.01\\
81.01	0.01\\
82.01	0.01\\
83.01	0.01\\
84.01	0.01\\
85.01	0.01\\
86.01	0.01\\
87.01	0.01\\
88.01	0.01\\
89.01	0.01\\
90.01	0.01\\
91.01	0.01\\
92.01	0.01\\
93.01	0.01\\
94.01	0.01\\
95.01	0.01\\
96.01	0.01\\
97.01	0.01\\
98.01	0.01\\
99.01	0.01\\
100.01	0.01\\
101.01	0.01\\
102.01	0.01\\
103.01	0.01\\
104.01	0.01\\
105.01	0.01\\
106.01	0.01\\
107.01	0.01\\
108.01	0.01\\
109.01	0.01\\
110.01	0.01\\
111.01	0.01\\
112.01	0.01\\
113.01	0.01\\
114.01	0.01\\
115.01	0.01\\
116.01	0.01\\
117.01	0.01\\
118.01	0.01\\
119.01	0.01\\
120.01	0.01\\
121.01	0.01\\
122.01	0.01\\
123.01	0.01\\
124.01	0.01\\
125.01	0.01\\
126.01	0.01\\
127.01	0.01\\
128.01	0.01\\
129.01	0.01\\
130.01	0.01\\
131.01	0.01\\
132.01	0.01\\
133.01	0.01\\
134.01	0.01\\
135.01	0.01\\
136.01	0.01\\
137.01	0.01\\
138.01	0.01\\
139.01	0.01\\
140.01	0.01\\
141.01	0.01\\
142.01	0.01\\
143.01	0.01\\
144.01	0.01\\
145.01	0.01\\
146.01	0.01\\
147.01	0.01\\
148.01	0.01\\
149.01	0.01\\
150.01	0.01\\
151.01	0.01\\
152.01	0.01\\
153.01	0.01\\
154.01	0.01\\
155.01	0.01\\
156.01	0.01\\
157.01	0.01\\
158.01	0.01\\
159.01	0.01\\
160.01	0.01\\
161.01	0.01\\
162.01	0.01\\
163.01	0.01\\
164.01	0.01\\
165.01	0.01\\
166.01	0.01\\
167.01	0.01\\
168.01	0.01\\
169.01	0.01\\
170.01	0.01\\
171.01	0.01\\
172.01	0.01\\
173.01	0.01\\
174.01	0.01\\
175.01	0.01\\
176.01	0.01\\
177.01	0.01\\
178.01	0.01\\
179.01	0.01\\
180.01	0.01\\
181.01	0.01\\
182.01	0.01\\
183.01	0.01\\
184.01	0.01\\
185.01	0.01\\
186.01	0.01\\
187.01	0.01\\
188.01	0.01\\
189.01	0.01\\
190.01	0.01\\
191.01	0.01\\
192.01	0.01\\
193.01	0.01\\
194.01	0.01\\
195.01	0.01\\
196.01	0.01\\
197.01	0.01\\
198.01	0.01\\
199.01	0.01\\
200.01	0.01\\
201.01	0.01\\
202.01	0.01\\
203.01	0.01\\
204.01	0.01\\
205.01	0.01\\
206.01	0.01\\
207.01	0.01\\
208.01	0.01\\
209.01	0.01\\
210.01	0.01\\
211.01	0.01\\
212.01	0.01\\
213.01	0.01\\
214.01	0.01\\
215.01	0.01\\
216.01	0.01\\
217.01	0.01\\
218.01	0.01\\
219.01	0.01\\
220.01	0.01\\
221.01	0.01\\
222.01	0.01\\
223.01	0.01\\
224.01	0.01\\
225.01	0.01\\
226.01	0.01\\
227.01	0.01\\
228.01	0.01\\
229.01	0.01\\
230.01	0.01\\
231.01	0.01\\
232.01	0.01\\
233.01	0.01\\
234.01	0.01\\
235.01	0.01\\
236.01	0.01\\
237.01	0.01\\
238.01	0.01\\
239.01	0.01\\
240.01	0.01\\
241.01	0.01\\
242.01	0.01\\
243.01	0.01\\
244.01	0.01\\
245.01	0.01\\
246.01	0.01\\
247.01	0.01\\
248.01	0.01\\
249.01	0.01\\
250.01	0.01\\
251.01	0.01\\
252.01	0.01\\
253.01	0.01\\
254.01	0.01\\
255.01	0.01\\
256.01	0.01\\
257.01	0.01\\
258.01	0.01\\
259.01	0.01\\
260.01	0.01\\
261.01	0.01\\
262.01	0.01\\
263.01	0.01\\
264.01	0.01\\
265.01	0.01\\
266.01	0.01\\
267.01	0.01\\
268.01	0.01\\
269.01	0.01\\
270.01	0.01\\
271.01	0.01\\
272.01	0.01\\
273.01	0.01\\
274.01	0.01\\
275.01	0.01\\
276.01	0.01\\
277.01	0.01\\
278.01	0.01\\
279.01	0.01\\
280.01	0.01\\
281.01	0.01\\
282.01	0.01\\
283.01	0.01\\
284.01	0.01\\
285.01	0.01\\
286.01	0.01\\
287.01	0.01\\
288.01	0.01\\
289.01	0.01\\
290.01	0.01\\
291.01	0.01\\
292.01	0.01\\
293.01	0.01\\
294.01	0.01\\
295.01	0.01\\
296.01	0.01\\
297.01	0.01\\
298.01	0.01\\
299.01	0.01\\
300.01	0.01\\
301.01	0.01\\
302.01	0.01\\
303.01	0.01\\
304.01	0.01\\
305.01	0.01\\
306.01	0.01\\
307.01	0.01\\
308.01	0.01\\
309.01	0.01\\
310.01	0.01\\
311.01	0.01\\
312.01	0.01\\
313.01	0.01\\
314.01	0.01\\
315.01	0.01\\
316.01	0.01\\
317.01	0.01\\
318.01	0.01\\
319.01	0.01\\
320.01	0.01\\
321.01	0.01\\
322.01	0.01\\
323.01	0.01\\
324.01	0.01\\
325.01	0.01\\
326.01	0.01\\
327.01	0.01\\
328.01	0.01\\
329.01	0.01\\
330.01	0.01\\
331.01	0.01\\
332.01	0.01\\
333.01	0.01\\
334.01	0.01\\
335.01	0.01\\
336.01	0.01\\
337.01	0.01\\
338.01	0.01\\
339.01	0.01\\
340.01	0.01\\
341.01	0.01\\
342.01	0.01\\
343.01	0.01\\
344.01	0.01\\
345.01	0.01\\
346.01	0.01\\
347.01	0.01\\
348.01	0.01\\
349.01	0.01\\
350.01	0.01\\
351.01	0.01\\
352.01	0.01\\
353.01	0.01\\
354.01	0.01\\
355.01	0.01\\
356.01	0.01\\
357.01	0.01\\
358.01	0.01\\
359.01	0.01\\
360.01	0.01\\
361.01	0.01\\
362.01	0.01\\
363.01	0.01\\
364.01	0.01\\
365.01	0.01\\
366.01	0.01\\
367.01	0.01\\
368.01	0.01\\
369.01	0.01\\
370.01	0.01\\
371.01	0.01\\
372.01	0.01\\
373.01	0.01\\
374.01	0.01\\
375.01	0.01\\
376.01	0.01\\
377.01	0.01\\
378.01	0.01\\
379.01	0.01\\
380.01	0.01\\
381.01	0.01\\
382.01	0.01\\
383.01	0.01\\
384.01	0.01\\
385.01	0.01\\
386.01	0.01\\
387.01	0.01\\
388.01	0.01\\
389.01	0.01\\
390.01	0.01\\
391.01	0.01\\
392.01	0.01\\
393.01	0.01\\
394.01	0.01\\
395.01	0.01\\
396.01	0.01\\
397.01	0.01\\
398.01	0.01\\
399.01	0.01\\
400.01	0.01\\
401.01	0.01\\
402.01	0.01\\
403.01	0.01\\
404.01	0.01\\
405.01	0.01\\
406.01	0.01\\
407.01	0.01\\
408.01	0.01\\
409.01	0.01\\
410.01	0.01\\
411.01	0.01\\
412.01	0.01\\
413.01	0.01\\
414.01	0.01\\
415.01	0.01\\
416.01	0.01\\
417.01	0.01\\
418.01	0.01\\
419.01	0.01\\
420.01	0.01\\
421.01	0.01\\
422.01	0.01\\
423.01	0.01\\
424.01	0.01\\
425.01	0.01\\
426.01	0.01\\
427.01	0.01\\
428.01	0.01\\
429.01	0.01\\
430.01	0.01\\
431.01	0.01\\
432.01	0.01\\
433.01	0.01\\
434.01	0.01\\
435.01	0.01\\
436.01	0.01\\
437.01	0.01\\
438.01	0.01\\
439.01	0.01\\
440.01	0.01\\
441.01	0.01\\
442.01	0.01\\
443.01	0.01\\
444.01	0.01\\
445.01	0.01\\
446.01	0.01\\
447.01	0.01\\
448.01	0.01\\
449.01	0.01\\
450.01	0.01\\
451.01	0.01\\
452.01	0.01\\
453.01	0.01\\
454.01	0.01\\
455.01	0.01\\
456.01	0.01\\
457.01	0.01\\
458.01	0.01\\
459.01	0.01\\
460.01	0.01\\
461.01	0.01\\
462.01	0.01\\
463.01	0.01\\
464.01	0.01\\
465.01	0.01\\
466.01	0.01\\
467.01	0.01\\
468.01	0.01\\
469.01	0.01\\
470.01	0.01\\
471.01	0.01\\
472.01	0.01\\
473.01	0.01\\
474.01	0.01\\
475.01	0.01\\
476.01	0.01\\
477.01	0.01\\
478.01	0.01\\
479.01	0.01\\
480.01	0.01\\
481.01	0.01\\
482.01	0.01\\
483.01	0.01\\
484.01	0.01\\
485.01	0.01\\
486.01	0.01\\
487.01	0.01\\
488.01	0.01\\
489.01	0.01\\
490.01	0.01\\
491.01	0.01\\
492.01	0.01\\
493.01	0.01\\
494.01	0.01\\
495.01	0.01\\
496.01	0.01\\
497.01	0.01\\
498.01	0.01\\
499.01	0.01\\
500.01	0.01\\
501.01	0.01\\
502.01	0.01\\
503.01	0.01\\
504.01	0.01\\
505.01	0.01\\
506.01	0.01\\
507.01	0.01\\
508.01	0.01\\
509.01	0.01\\
510.01	0.01\\
511.01	0.01\\
512.01	0.01\\
513.01	0.01\\
514.01	0.01\\
515.01	0.01\\
516.01	0.01\\
517.01	0.01\\
518.01	0.01\\
519.01	0.01\\
520.01	0.01\\
521.01	0.01\\
522.01	0.01\\
523.01	0.01\\
524.01	0.01\\
525.01	0.01\\
526.01	0.01\\
527.01	0.01\\
528.01	0.01\\
529.01	0.01\\
530.01	0.01\\
531.01	0.01\\
532.01	0.01\\
533.01	0.01\\
534.01	0.01\\
535.01	0.01\\
536.01	0.01\\
537.01	0.01\\
538.01	0.01\\
539.01	0.01\\
540.01	0.01\\
541.01	0.01\\
542.01	0.01\\
543.01	0.01\\
544.01	0.01\\
545.01	0.01\\
546.01	0.01\\
547.01	0.01\\
548.01	0.01\\
549.01	0.01\\
550.01	0.01\\
551.01	0.01\\
552.01	0.01\\
553.01	0.01\\
554.01	0.01\\
555.01	0.01\\
556.01	0.01\\
557.01	0.01\\
558.01	0.01\\
559.01	0.01\\
560.01	0.01\\
561.01	0.01\\
562.01	0.01\\
563.01	0.01\\
564.01	0.01\\
565.01	0.01\\
566.01	0.01\\
567.01	0.01\\
568.01	0.01\\
569.01	0.01\\
570.01	0.01\\
571.01	0.01\\
572.01	0.01\\
573.01	0.01\\
574.01	0.01\\
575.01	0.01\\
576.01	0.01\\
577.01	0.01\\
578.01	0.01\\
579.01	0.01\\
580.01	0.01\\
581.01	0.01\\
582.01	0.01\\
583.01	0.01\\
584.01	0.01\\
585.01	0.01\\
586.01	0.01\\
587.01	0.01\\
588.01	0.01\\
589.01	0.01\\
590.01	0.01\\
591.01	0.01\\
592.01	0.01\\
593.01	0.01\\
594.01	0.01\\
595.01	0.01\\
596.01	0.01\\
597.01	0.01\\
598.01	0.01\\
599.01	0.01\\
599.02	0.01\\
599.03	0.01\\
599.04	0.01\\
599.05	0.01\\
599.06	0.01\\
599.07	0.01\\
599.08	0.01\\
599.09	0.01\\
599.1	0.01\\
599.11	0.01\\
599.12	0.01\\
599.13	0.01\\
599.14	0.01\\
599.15	0.01\\
599.16	0.01\\
599.17	0.01\\
599.18	0.01\\
599.19	0.01\\
599.2	0.01\\
599.21	0.01\\
599.22	0.01\\
599.23	0.01\\
599.24	0.01\\
599.25	0.01\\
599.26	0.01\\
599.27	0.01\\
599.28	0.01\\
599.29	0.01\\
599.3	0.01\\
599.31	0.01\\
599.32	0.01\\
599.33	0.01\\
599.34	0.01\\
599.35	0.01\\
599.36	0.01\\
599.37	0.01\\
599.38	0.01\\
599.39	0.01\\
599.4	0.01\\
599.41	0.01\\
599.42	0.01\\
599.43	0.01\\
599.44	0.01\\
599.45	0.01\\
599.46	0.01\\
599.47	0.01\\
599.48	0.01\\
599.49	0.00988638009963208\\
599.5	0.00971984661000477\\
599.51	0.00955246047742145\\
599.52	0.00938421686791766\\
599.53	0.00921511031874022\\
599.54	0.00904513189583664\\
599.55	0.00887427136911025\\
599.56	0.00870251783210568\\
599.57	0.00852985946949492\\
599.58	0.00835628409069571\\
599.59	0.00818177911813606\\
599.6	0.00800633157515677\\
599.61	0.00782992807353735\\
599.62	0.00765255494364272\\
599.63	0.00747419798352703\\
599.64	0.00729484249349292\\
599.65	0.00711447330785863\\
599.66	0.00693307478015587\\
599.67	0.00675063076798984\\
599.68	0.00656712461739275\\
599.69	0.00638253914664802\\
599.7	0.00619685662956123\\
599.71	0.00601005877815255\\
599.72	0.00582212672474394\\
599.73	0.00563304100341284\\
599.74	0.00544278152954363\\
599.75	0.0052513275795273\\
599.76	0.00505865776998709\\
599.77	0.00486475003623244\\
599.78	0.00466958160990404\\
599.79	0.00447312899252071\\
599.8	0.00427536793068844\\
599.81	0.0040762733851378\\
599.82	0.00387581950088586\\
599.83	0.00367397957887031\\
599.84	0.00347072604640581\\
599.85	0.00326603042640224\\
599.86	0.00305986330528188\\
599.87	0.00285219429952965\\
599.88	0.00264299202080817\\
599.89	0.00243222403956698\\
599.9	0.00221985684707294\\
599.91	0.00200585581578775\\
599.92	0.00179018515801705\\
599.93	0.00157280788275587\\
599.94	0.00135368575065601\\
599.95	0.00113277922704381\\
599.96	0.000910047432921331\\
599.97	0.000685448093891365\\
599.98	0.000458937486957849\\
599.99	0.00023047038516868\\
600	0\\
};
\addplot [color=mycolor13,solid,forget plot]
  table[row sep=crcr]{%
0.01	0\\
1.01	0\\
2.01	0\\
3.01	0\\
4.01	0\\
5.01	0\\
6.01	0\\
7.01	0\\
8.01	0\\
9.01	0\\
10.01	0\\
11.01	0\\
12.01	0\\
13.01	0\\
14.01	0\\
15.01	0\\
16.01	0\\
17.01	0\\
18.01	0\\
19.01	0\\
20.01	0\\
21.01	0\\
22.01	0\\
23.01	0\\
24.01	0\\
25.01	0\\
26.01	0\\
27.01	0\\
28.01	0\\
29.01	0\\
30.01	0\\
31.01	0\\
32.01	0\\
33.01	0\\
34.01	0\\
35.01	0\\
36.01	0\\
37.01	0\\
38.01	0\\
39.01	0\\
40.01	0\\
41.01	0\\
42.01	0\\
43.01	0\\
44.01	0\\
45.01	0\\
46.01	0\\
47.01	0\\
48.01	0\\
49.01	0\\
50.01	0\\
51.01	0\\
52.01	0\\
53.01	0\\
54.01	0\\
55.01	0\\
56.01	0\\
57.01	0\\
58.01	0\\
59.01	0\\
60.01	0\\
61.01	0\\
62.01	0\\
63.01	0\\
64.01	0\\
65.01	0\\
66.01	0\\
67.01	0\\
68.01	0\\
69.01	0\\
70.01	0\\
71.01	0\\
72.01	0\\
73.01	0\\
74.01	0\\
75.01	0\\
76.01	0\\
77.01	0\\
78.01	0\\
79.01	0\\
80.01	0\\
81.01	0\\
82.01	0\\
83.01	0\\
84.01	0\\
85.01	0\\
86.01	0\\
87.01	0\\
88.01	0\\
89.01	0\\
90.01	0\\
91.01	0\\
92.01	0\\
93.01	0\\
94.01	0\\
95.01	0\\
96.01	0\\
97.01	0\\
98.01	0\\
99.01	0\\
100.01	0\\
101.01	0\\
102.01	0\\
103.01	0\\
104.01	0\\
105.01	0\\
106.01	0\\
107.01	0\\
108.01	0\\
109.01	0\\
110.01	0\\
111.01	0\\
112.01	0\\
113.01	0\\
114.01	0\\
115.01	0\\
116.01	0\\
117.01	0\\
118.01	0\\
119.01	0\\
120.01	0\\
121.01	0\\
122.01	0\\
123.01	0\\
124.01	0\\
125.01	0\\
126.01	0\\
127.01	0\\
128.01	0\\
129.01	0\\
130.01	0\\
131.01	0\\
132.01	0\\
133.01	0\\
134.01	0\\
135.01	0\\
136.01	0\\
137.01	0\\
138.01	0\\
139.01	0\\
140.01	0\\
141.01	0\\
142.01	0\\
143.01	0\\
144.01	0\\
145.01	0\\
146.01	0\\
147.01	0\\
148.01	0\\
149.01	0\\
150.01	0\\
151.01	0\\
152.01	0\\
153.01	0\\
154.01	0\\
155.01	0\\
156.01	0\\
157.01	0\\
158.01	0\\
159.01	0\\
160.01	0\\
161.01	0\\
162.01	0\\
163.01	0\\
164.01	0\\
165.01	0\\
166.01	0\\
167.01	0\\
168.01	0\\
169.01	0\\
170.01	0\\
171.01	0\\
172.01	0\\
173.01	0\\
174.01	0\\
175.01	0\\
176.01	0\\
177.01	0\\
178.01	0\\
179.01	0\\
180.01	0\\
181.01	0\\
182.01	0\\
183.01	0\\
184.01	0\\
185.01	0\\
186.01	0\\
187.01	0\\
188.01	0\\
189.01	0\\
190.01	0\\
191.01	0\\
192.01	0\\
193.01	0\\
194.01	0\\
195.01	0\\
196.01	0\\
197.01	0\\
198.01	0\\
199.01	0\\
200.01	0\\
201.01	0\\
202.01	0\\
203.01	0\\
204.01	0\\
205.01	0\\
206.01	0\\
207.01	0\\
208.01	0\\
209.01	0\\
210.01	0\\
211.01	0\\
212.01	0\\
213.01	0\\
214.01	0\\
215.01	0\\
216.01	0\\
217.01	0\\
218.01	0\\
219.01	0\\
220.01	0\\
221.01	0\\
222.01	0\\
223.01	0\\
224.01	0\\
225.01	0\\
226.01	0\\
227.01	0\\
228.01	0\\
229.01	0\\
230.01	0\\
231.01	0\\
232.01	0\\
233.01	0\\
234.01	0\\
235.01	0\\
236.01	0\\
237.01	0\\
238.01	0\\
239.01	0\\
240.01	0\\
241.01	0\\
242.01	0\\
243.01	0\\
244.01	0\\
245.01	0\\
246.01	0\\
247.01	0\\
248.01	0\\
249.01	0\\
250.01	0\\
251.01	0\\
252.01	0\\
253.01	0\\
254.01	0\\
255.01	0\\
256.01	0\\
257.01	0\\
258.01	0\\
259.01	0\\
260.01	0\\
261.01	0\\
262.01	0\\
263.01	0\\
264.01	0\\
265.01	0\\
266.01	0\\
267.01	0\\
268.01	0\\
269.01	0\\
270.01	0\\
271.01	0\\
272.01	0\\
273.01	0\\
274.01	0\\
275.01	0\\
276.01	0\\
277.01	0\\
278.01	0\\
279.01	0\\
280.01	0\\
281.01	0\\
282.01	0\\
283.01	0\\
284.01	0\\
285.01	0\\
286.01	0\\
287.01	0\\
288.01	0\\
289.01	0\\
290.01	0\\
291.01	0\\
292.01	0\\
293.01	0\\
294.01	0\\
295.01	0\\
296.01	0\\
297.01	0\\
298.01	0\\
299.01	0\\
300.01	0\\
301.01	0\\
302.01	0\\
303.01	0\\
304.01	0\\
305.01	0\\
306.01	0\\
307.01	0\\
308.01	0\\
309.01	0\\
310.01	0\\
311.01	0\\
312.01	0\\
313.01	0\\
314.01	0\\
315.01	0\\
316.01	0\\
317.01	0\\
318.01	0\\
319.01	0\\
320.01	0\\
321.01	0\\
322.01	0\\
323.01	0\\
324.01	0\\
325.01	0\\
326.01	0\\
327.01	0\\
328.01	0\\
329.01	0\\
330.01	0\\
331.01	0\\
332.01	0\\
333.01	0\\
334.01	0\\
335.01	0\\
336.01	0\\
337.01	0\\
338.01	0\\
339.01	0\\
340.01	0\\
341.01	0\\
342.01	0\\
343.01	0\\
344.01	0\\
345.01	0\\
346.01	0\\
347.01	0\\
348.01	0\\
349.01	0\\
350.01	0\\
351.01	0\\
352.01	0\\
353.01	0\\
354.01	0\\
355.01	0\\
356.01	0\\
357.01	0\\
358.01	0\\
359.01	0\\
360.01	0\\
361.01	0\\
362.01	0\\
363.01	0\\
364.01	0\\
365.01	0\\
366.01	0\\
367.01	0\\
368.01	0\\
369.01	0\\
370.01	0\\
371.01	0\\
372.01	0\\
373.01	0\\
374.01	0\\
375.01	0\\
376.01	0\\
377.01	0\\
378.01	0\\
379.01	0\\
380.01	0\\
381.01	0\\
382.01	0\\
383.01	0\\
384.01	0\\
385.01	0\\
386.01	0\\
387.01	0\\
388.01	0\\
389.01	0\\
390.01	0\\
391.01	0\\
392.01	0\\
393.01	0\\
394.01	0\\
395.01	0\\
396.01	0\\
397.01	0\\
398.01	0\\
399.01	0\\
400.01	0\\
401.01	0\\
402.01	0\\
403.01	0\\
404.01	0\\
405.01	0\\
406.01	0\\
407.01	0\\
408.01	0\\
409.01	0\\
410.01	0\\
411.01	0\\
412.01	0\\
413.01	0\\
414.01	0\\
415.01	0\\
416.01	0\\
417.01	0\\
418.01	0\\
419.01	0\\
420.01	0\\
421.01	0\\
422.01	0\\
423.01	0\\
424.01	0\\
425.01	0\\
426.01	0\\
427.01	0\\
428.01	0\\
429.01	0\\
430.01	0\\
431.01	0\\
432.01	0\\
433.01	0\\
434.01	0\\
435.01	0\\
436.01	0\\
437.01	0\\
438.01	0\\
439.01	0\\
440.01	0\\
441.01	0\\
442.01	0\\
443.01	0\\
444.01	0\\
445.01	0\\
446.01	0\\
447.01	0\\
448.01	0\\
449.01	0\\
450.01	0\\
451.01	0\\
452.01	0\\
453.01	0\\
454.01	0\\
455.01	0\\
456.01	0\\
457.01	0\\
458.01	0\\
459.01	0\\
460.01	0\\
461.01	0\\
462.01	0\\
463.01	0\\
464.01	0\\
465.01	0\\
466.01	0\\
467.01	0\\
468.01	0\\
469.01	0\\
470.01	0\\
471.01	0\\
472.01	0\\
473.01	0\\
474.01	0\\
475.01	0\\
476.01	0\\
477.01	0\\
478.01	0\\
479.01	0\\
480.01	0\\
481.01	0\\
482.01	0\\
483.01	0\\
484.01	0\\
485.01	0\\
486.01	0\\
487.01	0\\
488.01	0\\
489.01	0\\
490.01	0\\
491.01	0\\
492.01	0\\
493.01	0\\
494.01	0\\
495.01	0\\
496.01	0\\
497.01	0\\
498.01	0\\
499.01	0\\
500.01	0\\
501.01	0\\
502.01	0\\
503.01	0\\
504.01	0\\
505.01	0\\
506.01	0\\
507.01	0\\
508.01	0\\
509.01	0\\
510.01	0\\
511.01	0\\
512.01	0\\
513.01	0\\
514.01	0\\
515.01	0\\
516.01	0\\
517.01	0\\
518.01	0\\
519.01	0\\
520.01	0\\
521.01	0\\
522.01	0\\
523.01	0\\
524.01	0\\
525.01	0\\
526.01	0\\
527.01	0\\
528.01	0\\
529.01	0\\
530.01	0\\
531.01	0\\
532.01	0\\
533.01	0\\
534.01	0\\
535.01	0\\
536.01	0\\
537.01	0\\
538.01	0\\
539.01	0\\
540.01	0\\
541.01	0\\
542.01	0\\
543.01	0\\
544.01	0\\
545.01	0\\
546.01	0\\
547.01	0\\
548.01	0\\
549.01	0\\
550.01	0\\
551.01	0\\
552.01	0\\
553.01	0\\
554.01	0\\
555.01	0\\
556.01	0\\
557.01	0\\
558.01	0\\
559.01	0\\
560.01	0\\
561.01	0\\
562.01	0\\
563.01	0\\
564.01	0\\
565.01	0\\
566.01	0\\
567.01	0\\
568.01	0\\
569.01	0\\
570.01	0\\
571.01	0\\
572.01	0\\
573.01	0\\
574.01	0\\
575.01	0\\
576.01	0\\
577.01	0\\
578.01	0\\
579.01	0\\
580.01	0\\
581.01	0\\
582.01	0\\
583.01	0\\
584.01	0\\
585.01	0\\
586.01	0.000142170676920603\\
587.01	0.000355128567196075\\
588.01	0.000573748988783096\\
589.01	0.00079851512560985\\
590.01	0.00102998128508491\\
591.01	0.00126877990469422\\
592.01	0.00151563757222356\\
593.01	0.00177139624661609\\
594.01	0.00203704594793394\\
595.01	0.00231380797226837\\
596.01	0.00260357560208499\\
597.01	0.0029123341470069\\
598.01	0.00328029355590324\\
599.01	0.00407681843730328\\
599.02	0.00409226336446341\\
599.03	0.00410800018922777\\
599.04	0.00412403637215258\\
599.05	0.00414037958426491\\
599.06	0.00415703771382205\\
599.07	0.0041740188733256\\
599.08	0.00419133140680166\\
599.09	0.00420898389735902\\
599.1	0.00422698517503799\\
599.11	0.00424534432496329\\
599.12	0.00426407069581482\\
599.13	0.00428317390863123\\
599.14	0.0043026638659619\\
599.15	0.0043225507613836\\
599.16	0.00434284508939945\\
599.17	0.00436355765573831\\
599.18	0.00438469958807417\\
599.19	0.00440628234718589\\
599.2	0.00442831773857909\\
599.21	0.00445081792654077\\
599.22	0.00447379544795006\\
599.23	0.00449726322260612\\
599.24	0.00452123456695878\\
599.25	0.00454572320843149\\
599.26	0.00457074330036714\\
599.27	0.00459630944765488\\
599.28	0.00462243672849619\\
599.29	0.0046491406975709\\
599.3	0.00467643740391242\\
599.31	0.00470434342570141\\
599.32	0.00473287589009266\\
599.33	0.00476205247943\\
599.34	0.00479189145277651\\
599.35	0.00482241171590761\\
599.36	0.00485363281887117\\
599.37	0.00488557496399592\\
599.38	0.00491825903212873\\
599.39	0.0049517066101656\\
599.4	0.00498594001995394\\
599.41	0.00502098234864931\\
599.42	0.00505685748061574\\
599.43	0.00509359013096528\\
599.44	0.00513120588083937\\
599.45	0.00516973121454266\\
599.46	0.00520919355864768\\
599.47	0.00524962132319844\\
599.48	0.00529104394515042\\
599.49	0.00533348993758731\\
599.5	0.00537698981313269\\
599.51	0.00542157628050479\\
599.52	0.00546728327083941\\
599.53	0.00551414153003291\\
599.54	0.00556217430889\\
599.55	0.00561141758723835\\
599.56	0.00566190175434073\\
599.57	0.00571365261530301\\
599.58	0.00576670966414243\\
599.59	0.00582107484520414\\
599.6	0.00587678781064857\\
599.61	0.00593389135595841\\
599.62	0.00599242992818502\\
599.63	0.00605244970697346\\
599.64	0.00611399869066074\\
599.65	0.0061771267878398\\
599.66	0.0062418859148182\\
599.67	0.00630833009953979\\
599.68	0.00637651559233562\\
599.69	0.00644650098408254\\
599.7	0.00651834733242259\\
599.71	0.0065921182967225\\
599.72	0.00666788028252302\\
599.73	0.00674570259633623\\
599.74	0.00682565761164589\\
599.75	0.00690782094712556\\
599.76	0.00699227165820456\\
599.77	0.00707909244321486\\
599.78	0.00716836986547855\\
599.79	0.00726019459286861\\
599.8	0.00735466165654207\\
599.81	0.0074518707307418\\
599.82	0.00755192643578738\\
599.83	0.00765493866663039\\
599.84	0.00776102294964022\\
599.85	0.0078703008306186\\
599.86	0.0079829002974221\\
599.87	0.00809895624100861\\
599.88	0.00821861095922719\\
599.89	0.00834201470825166\\
599.9	0.00846932630723019\\
599.91	0.00860071380250349\\
599.92	0.00873635519865198\\
599.93	0.00887643926469208\\
599.94	0.00902116642498313\\
599.95	0.00917074974586409\\
599.96	0.00932541603075827\\
599.97	0.00948540703851739\\
599.98	0.00965098084219061\\
599.99	0.00982241334828153\\
600	0.01\\
};
\addplot [color=mycolor14,solid,forget plot]
  table[row sep=crcr]{%
0.01	0.00371232059378437\\
1.01	0.00371232059383302\\
2.01	0.00371232059388272\\
3.01	0.00371232059393349\\
4.01	0.00371232059398535\\
5.01	0.00371232059403833\\
6.01	0.00371232059409247\\
7.01	0.00371232059414777\\
8.01	0.00371232059420427\\
9.01	0.003712320594262\\
10.01	0.00371232059432096\\
11.01	0.0037123205943812\\
12.01	0.00371232059444275\\
13.01	0.00371232059450564\\
14.01	0.00371232059456988\\
15.01	0.00371232059463551\\
16.01	0.00371232059470256\\
17.01	0.00371232059477106\\
18.01	0.00371232059484106\\
19.01	0.00371232059491256\\
20.01	0.00371232059498561\\
21.01	0.00371232059506025\\
22.01	0.00371232059513651\\
23.01	0.00371232059521442\\
24.01	0.00371232059529402\\
25.01	0.00371232059537534\\
26.01	0.00371232059545843\\
27.01	0.00371232059554332\\
28.01	0.00371232059563005\\
29.01	0.00371232059571867\\
30.01	0.00371232059580921\\
31.01	0.00371232059590171\\
32.01	0.00371232059599623\\
33.01	0.0037123205960928\\
34.01	0.00371232059619147\\
35.01	0.00371232059629228\\
36.01	0.00371232059639528\\
37.01	0.00371232059650052\\
38.01	0.00371232059660805\\
39.01	0.00371232059671792\\
40.01	0.00371232059683017\\
41.01	0.00371232059694488\\
42.01	0.00371232059706207\\
43.01	0.00371232059718182\\
44.01	0.00371232059730417\\
45.01	0.00371232059742919\\
46.01	0.00371232059755694\\
47.01	0.00371232059768746\\
48.01	0.00371232059782083\\
49.01	0.00371232059795711\\
50.01	0.00371232059809636\\
51.01	0.00371232059823864\\
52.01	0.00371232059838403\\
53.01	0.00371232059853259\\
54.01	0.00371232059868438\\
55.01	0.0037123205988395\\
56.01	0.003712320598998\\
57.01	0.00371232059915996\\
58.01	0.00371232059932546\\
59.01	0.00371232059949457\\
60.01	0.00371232059966738\\
61.01	0.00371232059984397\\
62.01	0.00371232060002442\\
63.01	0.00371232060020881\\
64.01	0.00371232060039724\\
65.01	0.00371232060058978\\
66.01	0.00371232060078655\\
67.01	0.00371232060098762\\
68.01	0.00371232060119309\\
69.01	0.00371232060140306\\
70.01	0.00371232060161764\\
71.01	0.00371232060183691\\
72.01	0.00371232060206099\\
73.01	0.00371232060228998\\
74.01	0.00371232060252399\\
75.01	0.00371232060276314\\
76.01	0.00371232060300754\\
77.01	0.00371232060325729\\
78.01	0.00371232060351253\\
79.01	0.00371232060377339\\
80.01	0.00371232060403996\\
81.01	0.00371232060431239\\
82.01	0.00371232060459081\\
83.01	0.00371232060487535\\
84.01	0.00371232060516614\\
85.01	0.00371232060546334\\
86.01	0.00371232060576707\\
87.01	0.00371232060607748\\
88.01	0.00371232060639473\\
89.01	0.00371232060671898\\
90.01	0.00371232060705036\\
91.01	0.00371232060738904\\
92.01	0.00371232060773519\\
93.01	0.00371232060808896\\
94.01	0.00371232060845054\\
95.01	0.0037123206088201\\
96.01	0.00371232060919782\\
97.01	0.00371232060958387\\
98.01	0.00371232060997845\\
99.01	0.00371232061038174\\
100.01	0.00371232061079394\\
101.01	0.00371232061121525\\
102.01	0.00371232061164588\\
103.01	0.00371232061208604\\
104.01	0.00371232061253593\\
105.01	0.00371232061299578\\
106.01	0.00371232061346581\\
107.01	0.00371232061394625\\
108.01	0.00371232061443734\\
109.01	0.0037123206149393\\
110.01	0.00371232061545239\\
111.01	0.00371232061597687\\
112.01	0.00371232061651298\\
113.01	0.00371232061706098\\
114.01	0.00371232061762115\\
115.01	0.00371232061819376\\
116.01	0.00371232061877909\\
117.01	0.00371232061937743\\
118.01	0.00371232061998907\\
119.01	0.00371232062061432\\
120.01	0.00371232062125347\\
121.01	0.00371232062190684\\
122.01	0.00371232062257477\\
123.01	0.00371232062325757\\
124.01	0.00371232062395558\\
125.01	0.00371232062466915\\
126.01	0.00371232062539863\\
127.01	0.00371232062614438\\
128.01	0.00371232062690677\\
129.01	0.00371232062768619\\
130.01	0.003712320628483\\
131.01	0.00371232062929762\\
132.01	0.00371232063013044\\
133.01	0.00371232063098188\\
134.01	0.00371232063185236\\
135.01	0.00371232063274232\\
136.01	0.00371232063365221\\
137.01	0.00371232063458247\\
138.01	0.00371232063553356\\
139.01	0.00371232063650597\\
140.01	0.00371232063750019\\
141.01	0.0037123206385167\\
142.01	0.00371232063955604\\
143.01	0.00371232064061869\\
144.01	0.00371232064170521\\
145.01	0.00371232064281615\\
146.01	0.00371232064395205\\
147.01	0.00371232064511351\\
148.01	0.00371232064630108\\
149.01	0.00371232064751539\\
150.01	0.00371232064875705\\
151.01	0.00371232065002667\\
152.01	0.0037123206513249\\
153.01	0.00371232065265241\\
154.01	0.00371232065400985\\
155.01	0.00371232065539794\\
156.01	0.00371232065681736\\
157.01	0.00371232065826884\\
158.01	0.00371232065975312\\
159.01	0.00371232066127098\\
160.01	0.00371232066282315\\
161.01	0.00371232066441046\\
162.01	0.00371232066603369\\
163.01	0.00371232066769369\\
164.01	0.00371232066939131\\
165.01	0.00371232067112742\\
166.01	0.00371232067290291\\
167.01	0.0037123206747187\\
168.01	0.00371232067657572\\
169.01	0.00371232067847492\\
170.01	0.00371232068041728\\
171.01	0.00371232068240382\\
172.01	0.00371232068443554\\
173.01	0.00371232068651351\\
174.01	0.0037123206886388\\
175.01	0.00371232069081252\\
176.01	0.00371232069303579\\
177.01	0.00371232069530977\\
178.01	0.00371232069763563\\
179.01	0.00371232070001459\\
180.01	0.0037123207024479\\
181.01	0.00371232070493682\\
182.01	0.00371232070748265\\
183.01	0.00371232071008673\\
184.01	0.00371232071275042\\
185.01	0.0037123207154751\\
186.01	0.00371232071826223\\
187.01	0.00371232072111325\\
188.01	0.00371232072402968\\
189.01	0.00371232072701302\\
190.01	0.00371232073006491\\
191.01	0.0037123207331869\\
192.01	0.00371232073638066\\
193.01	0.00371232073964787\\
194.01	0.00371232074299028\\
195.01	0.00371232074640965\\
196.01	0.0037123207499078\\
197.01	0.00371232075348659\\
198.01	0.00371232075714793\\
199.01	0.00371232076089375\\
200.01	0.00371232076472607\\
201.01	0.00371232076864693\\
202.01	0.00371232077265842\\
203.01	0.0037123207767627\\
204.01	0.00371232078096196\\
205.01	0.00371232078525845\\
206.01	0.00371232078965449\\
207.01	0.00371232079415246\\
208.01	0.00371232079875472\\
209.01	0.00371232080346381\\
210.01	0.00371232080828226\\
211.01	0.00371232081321267\\
212.01	0.0037123208182577\\
213.01	0.00371232082342009\\
214.01	0.00371232082870263\\
215.01	0.00371232083410821\\
216.01	0.00371232083963975\\
217.01	0.00371232084530027\\
218.01	0.00371232085109285\\
219.01	0.00371232085702064\\
220.01	0.00371232086308692\\
221.01	0.00371232086929496\\
222.01	0.00371232087564818\\
223.01	0.00371232088215005\\
224.01	0.00371232088880417\\
225.01	0.00371232089561415\\
226.01	0.00371232090258377\\
227.01	0.00371232090971685\\
228.01	0.00371232091701734\\
229.01	0.00371232092448927\\
230.01	0.00371232093213675\\
231.01	0.00371232093996402\\
232.01	0.00371232094797543\\
233.01	0.00371232095617542\\
234.01	0.00371232096456853\\
235.01	0.00371232097315944\\
236.01	0.00371232098195293\\
237.01	0.00371232099095387\\
238.01	0.00371232100016732\\
239.01	0.00371232100959841\\
240.01	0.00371232101925242\\
241.01	0.00371232102913474\\
242.01	0.00371232103925093\\
243.01	0.00371232104960664\\
244.01	0.0037123210602077\\
245.01	0.00371232107106008\\
246.01	0.00371232108216986\\
247.01	0.00371232109354332\\
248.01	0.00371232110518687\\
249.01	0.00371232111710712\\
250.01	0.00371232112931075\\
251.01	0.00371232114180471\\
252.01	0.00371232115459606\\
253.01	0.00371232116769208\\
254.01	0.00371232118110014\\
255.01	0.00371232119482793\\
256.01	0.00371232120888322\\
257.01	0.00371232122327404\\
258.01	0.00371232123800859\\
259.01	0.00371232125309527\\
260.01	0.0037123212685427\\
261.01	0.00371232128435971\\
262.01	0.00371232130055537\\
263.01	0.00371232131713894\\
264.01	0.00371232133411994\\
265.01	0.00371232135150813\\
266.01	0.00371232136931347\\
267.01	0.00371232138754624\\
268.01	0.00371232140621689\\
269.01	0.00371232142533625\\
270.01	0.00371232144491527\\
271.01	0.0037123214649653\\
272.01	0.00371232148549791\\
273.01	0.00371232150652497\\
274.01	0.00371232152805866\\
275.01	0.00371232155011144\\
276.01	0.00371232157269612\\
277.01	0.0037123215958258\\
278.01	0.0037123216195139\\
279.01	0.00371232164377418\\
280.01	0.00371232166862078\\
281.01	0.00371232169406816\\
282.01	0.00371232172013114\\
283.01	0.00371232174682492\\
284.01	0.00371232177416507\\
285.01	0.00371232180216756\\
286.01	0.00371232183084877\\
287.01	0.00371232186022544\\
288.01	0.00371232189031481\\
289.01	0.00371232192113446\\
290.01	0.00371232195270249\\
291.01	0.00371232198503738\\
292.01	0.00371232201815815\\
293.01	0.00371232205208421\\
294.01	0.00371232208683553\\
295.01	0.00371232212243253\\
296.01	0.00371232215889617\\
297.01	0.00371232219624792\\
298.01	0.00371232223450978\\
299.01	0.00371232227370431\\
300.01	0.00371232231385464\\
301.01	0.00371232235498447\\
302.01	0.00371232239711811\\
303.01	0.00371232244028043\\
304.01	0.00371232248449698\\
305.01	0.0037123225297939\\
306.01	0.00371232257619805\\
307.01	0.0037123226237369\\
308.01	0.00371232267243861\\
309.01	0.00371232272233209\\
310.01	0.00371232277344695\\
311.01	0.00371232282581353\\
312.01	0.00371232287946294\\
313.01	0.0037123229344271\\
314.01	0.00371232299073868\\
315.01	0.00371232304843119\\
316.01	0.00371232310753902\\
317.01	0.00371232316809736\\
318.01	0.00371232323014234\\
319.01	0.00371232329371096\\
320.01	0.00371232335884119\\
321.01	0.0037123234255719\\
322.01	0.00371232349394301\\
323.01	0.00371232356399543\\
324.01	0.00371232363577105\\
325.01	0.00371232370931289\\
326.01	0.00371232378466501\\
327.01	0.00371232386187263\\
328.01	0.00371232394098209\\
329.01	0.00371232402204089\\
330.01	0.0037123241050978\\
331.01	0.00371232419020279\\
332.01	0.00371232427740711\\
333.01	0.00371232436676332\\
334.01	0.00371232445832535\\
335.01	0.00371232455214848\\
336.01	0.00371232464828945\\
337.01	0.00371232474680644\\
338.01	0.00371232484775911\\
339.01	0.00371232495120873\\
340.01	0.00371232505721808\\
341.01	0.00371232516585164\\
342.01	0.00371232527717554\\
343.01	0.00371232539125763\\
344.01	0.00371232550816756\\
345.01	0.00371232562797679\\
346.01	0.0037123257507587\\
347.01	0.00371232587658854\\
348.01	0.00371232600554365\\
349.01	0.00371232613770333\\
350.01	0.00371232627314907\\
351.01	0.00371232641196453\\
352.01	0.00371232655423558\\
353.01	0.00371232670005044\\
354.01	0.00371232684949972\\
355.01	0.00371232700267647\\
356.01	0.00371232715967632\\
357.01	0.00371232732059748\\
358.01	0.00371232748554088\\
359.01	0.00371232765461025\\
360.01	0.00371232782791217\\
361.01	0.00371232800555623\\
362.01	0.00371232818765508\\
363.01	0.00371232837432455\\
364.01	0.00371232856568371\\
365.01	0.00371232876185505\\
366.01	0.00371232896296457\\
367.01	0.00371232916914186\\
368.01	0.00371232938052026\\
369.01	0.00371232959723702\\
370.01	0.00371232981943338\\
371.01	0.00371233004725469\\
372.01	0.00371233028085067\\
373.01	0.00371233052037542\\
374.01	0.00371233076598773\\
375.01	0.00371233101785108\\
376.01	0.00371233127613399\\
377.01	0.00371233154101006\\
378.01	0.00371233181265826\\
379.01	0.00371233209126305\\
380.01	0.00371233237701473\\
381.01	0.00371233267010947\\
382.01	0.00371233297074968\\
383.01	0.00371233327914428\\
384.01	0.0037123335955088\\
385.01	0.00371233392006582\\
386.01	0.00371233425304517\\
387.01	0.00371233459468429\\
388.01	0.00371233494522846\\
389.01	0.00371233530493123\\
390.01	0.00371233567405473\\
391.01	0.00371233605287007\\
392.01	0.00371233644165774\\
393.01	0.00371233684070799\\
394.01	0.00371233725032134\\
395.01	0.00371233767080906\\
396.01	0.00371233810249358\\
397.01	0.00371233854570914\\
398.01	0.00371233900080227\\
399.01	0.00371233946813244\\
400.01	0.0037123399480727\\
401.01	0.00371234044101025\\
402.01	0.0037123409473473\\
403.01	0.00371234146750167\\
404.01	0.00371234200190762\\
405.01	0.00371234255101666\\
406.01	0.00371234311529841\\
407.01	0.0037123436952414\\
408.01	0.003712344291354\\
409.01	0.00371234490416529\\
410.01	0.00371234553422596\\
411.01	0.00371234618210913\\
412.01	0.00371234684841086\\
413.01	0.00371234753374937\\
414.01	0.00371234823876087\\
415.01	0.00371234896410251\\
416.01	0.00371234971051709\\
417.01	0.00371235047897498\\
418.01	0.00371235127189158\\
419.01	0.00371235210202903\\
420.01	0.00371235305360244\\
421.01	0.00371235463003312\\
422.01	0.00371235776114621\\
423.01	0.00371236120668805\\
424.01	0.00371236475388728\\
425.01	0.00371236840621751\\
426.01	0.00371237216728881\\
427.01	0.00371237604085278\\
428.01	0.00371238003080758\\
429.01	0.00371238414120308\\
430.01	0.0037123883762459\\
431.01	0.0037123927403043\\
432.01	0.00371239723791314\\
433.01	0.00371240187377847\\
434.01	0.00371240665278212\\
435.01	0.00371241157998598\\
436.01	0.00371241666063594\\
437.01	0.00371242190016565\\
438.01	0.00371242730419993\\
439.01	0.00371243287855778\\
440.01	0.00371243862925526\\
441.01	0.00371244456250806\\
442.01	0.00371245068473388\\
443.01	0.00371245700255509\\
444.01	0.00371246352280165\\
445.01	0.00371247025251468\\
446.01	0.00371247719895162\\
447.01	0.00371248436959319\\
448.01	0.00371249177215348\\
449.01	0.00371249941459339\\
450.01	0.00371250730513867\\
451.01	0.00371251545231206\\
452.01	0.00371252386500198\\
453.01	0.00371253255250386\\
454.01	0.00371254152459322\\
455.01	0.00371255079175736\\
456.01	0.00371256036607533\\
457.01	0.00371257026634798\\
458.01	0.00371258054933302\\
459.01	0.00371259147149347\\
460.01	0.003712603890558\\
461.01	0.0037126178219913\\
462.01	0.00371263219023903\\
463.01	0.0037126469724045\\
464.01	0.0037126621818956\\
465.01	0.00371267783273444\\
466.01	0.00371269393958189\\
467.01	0.00371271051771323\\
468.01	0.00371272758290139\\
469.01	0.00371274515163655\\
470.01	0.003712763242706\\
471.01	0.00371278187972719\\
472.01	0.00371280110797478\\
473.01	0.00371282105340721\\
474.01	0.00371284190039475\\
475.01	0.0037128634861283\\
476.01	0.00371288572413619\\
477.01	0.00371290864348864\\
478.01	0.00371293227573453\\
479.01	0.00371295665495277\\
480.01	0.00371298181803624\\
481.01	0.00371300780501231\\
482.01	0.00371303465940402\\
483.01	0.00371306242863533\\
484.01	0.00371309116447902\\
485.01	0.00371312092355296\\
486.01	0.00371315176798094\\
487.01	0.00371318376624932\\
488.01	0.00371321699369153\\
489.01	0.00371325153336621\\
490.01	0.00371328747728075\\
491.01	0.00371332492859875\\
492.01	0.00371336401075783\\
493.01	0.00371340492535683\\
494.01	0.00371344835289228\\
495.01	0.00371349807380068\\
496.01	0.00371357169769694\\
497.01	0.00371366827917016\\
498.01	0.00371376798644745\\
499.01	0.00371387090780026\\
500.01	0.00371397718638241\\
501.01	0.00371408697533663\\
502.01	0.00371420043877106\\
503.01	0.00371431775287138\\
504.01	0.00371443910796197\\
505.01	0.00371456471072978\\
506.01	0.00371469478465389\\
507.01	0.00371482957266362\\
508.01	0.00371496934003496\\
509.01	0.00371511437754475\\
510.01	0.00371526500522995\\
511.01	0.00371542157705445\\
512.01	0.00371558448791376\\
513.01	0.00371575419213165\\
514.01	0.00371593129393625\\
515.01	0.00371611707591379\\
516.01	0.00371631609988022\\
517.01	0.00371654061402795\\
518.01	0.0037167859127879\\
519.01	0.00371704263620946\\
520.01	0.00371731463846147\\
521.01	0.00371762255867969\\
522.01	0.00371805578781655\\
523.01	0.0037185876132974\\
524.01	0.00371913708452109\\
525.01	0.00371970512564631\\
526.01	0.00372029274048168\\
527.01	0.00372090101722667\\
528.01	0.00372153113909118\\
529.01	0.00372218439768938\\
530.01	0.00372286221314574\\
531.01	0.00372356620137852\\
532.01	0.00372429859800109\\
533.01	0.00372506559641567\\
534.01	0.00372590708245823\\
535.01	0.0037271657759248\\
536.01	0.00372880840188161\\
537.01	0.00373053250235488\\
538.01	0.00373234484324824\\
539.01	0.00373425920144489\\
540.01	0.00373632476310768\\
541.01	0.00373881151959034\\
542.01	0.00374264026881439\\
543.01	0.00374815218611865\\
544.01	0.0037538952752454\\
545.01	0.00375987008897414\\
546.01	0.00376610063725892\\
547.01	0.00377261378869254\\
548.01	0.00377945235751856\\
549.01	0.00378677440762343\\
550.01	0.00379555663285403\\
551.01	0.00380905636103867\\
552.01	0.00382361115705503\\
553.01	0.00383876210916985\\
554.01	0.00385455756218182\\
555.01	0.00387105201588933\\
556.01	0.00388830767127321\\
557.01	0.00390640016670082\\
558.01	0.00392545874195979\\
559.01	0.0039460192508848\\
560.01	0.00397268468044839\\
561.01	0.00401480471382403\\
562.01	0.00406056486634537\\
563.01	0.00411002754511294\\
564.01	0.0041612123977801\\
565.01	0.00421413501518444\\
566.01	0.00426897077941056\\
567.01	0.00432600352534416\\
568.01	0.00438542430483328\\
569.01	0.00444744552069263\\
570.01	0.00451229689357401\\
571.01	0.00458025456503107\\
572.01	0.00465204500330351\\
573.01	0.00473271924694896\\
574.01	0.00486279822256183\\
575.01	0.00501940891484108\\
576.01	0.00517995413927412\\
577.01	0.00534458503903426\\
578.01	0.00551327045537046\\
579.01	0.00568626589840416\\
580.01	0.00586390647491116\\
581.01	0.00604657367145632\\
582.01	0.00623470318704096\\
583.01	0.00642878031955761\\
584.01	0.00662933650587675\\
585.01	0.00684825160576811\\
586.01	0.00703475413857012\\
587.01	0.00717428404336578\\
588.01	0.00731761703393303\\
589.01	0.00746489083537659\\
590.01	0.00761623766918582\\
591.01	0.00777178314411316\\
592.01	0.00793164049198838\\
593.01	0.00809590288331832\\
594.01	0.00826463697850347\\
595.01	0.00843791101233693\\
596.01	0.00861615191248504\\
597.01	0.00880342261792005\\
598.01	0.00903592442409936\\
599.01	0.00957365461413272\\
599.02	0.00958277041975926\\
599.03	0.00959197264387863\\
599.04	0.00960125988934567\\
599.05	0.00961063057421246\\
599.06	0.00962008292013701\\
599.07	0.00962961494015077\\
599.08	0.00963922442574839\\
599.09	0.00964890893326474\\
599.1	0.00965866576950219\\
599.11	0.00966849197656794\\
599.12	0.00967838431587352\\
599.13	0.00968833925125193\\
599.14	0.0096983529311477\\
599.15	0.00970842116982741\\
599.16	0.00971853942755756\\
599.17	0.009728702789693\\
599.18	0.00973890594461645\\
599.19	0.00974914316046814\\
599.2	0.00975940826059864\\
599.21	0.00976969321646363\\
599.22	0.00977998868251903\\
599.23	0.00979028683966976\\
599.24	0.00980057930540565\\
599.25	0.00981085710148202\\
599.26	0.0098211106198357\\
599.27	0.00983132243721219\\
599.28	0.0098414706723303\\
599.29	0.00985154283965676\\
599.3	0.00986152563527719\\
599.31	0.00987139336310963\\
599.32	0.00988111946709059\\
599.33	0.00989068690280136\\
599.34	0.0099000775652852\\
599.35	0.0099092381529734\\
599.36	0.00991813369390734\\
599.37	0.00992674039223167\\
599.38	0.00993503304638513\\
599.39	0.00994298497352877\\
599.4	0.00995056793008674\\
599.41	0.00995775202821507\\
599.42	0.00996450564800631\\
599.43	0.00997079534523213\\
599.44	0.00997658575441854\\
599.45	0.00998183948704209\\
599.46	0.00998651702462867\\
599.47	0.0099905766065302\\
599.48	0.0099939741121483\\
599.49	0.0099966649971801\\
599.5	0.00999860126363529\\
599.51	0.00999973114745285\\
599.52	0.01\\
599.53	0.01\\
599.54	0.01\\
599.55	0.01\\
599.56	0.01\\
599.57	0.01\\
599.58	0.01\\
599.59	0.01\\
599.6	0.01\\
599.61	0.01\\
599.62	0.01\\
599.63	0.01\\
599.64	0.01\\
599.65	0.01\\
599.66	0.01\\
599.67	0.01\\
599.68	0.01\\
599.69	0.01\\
599.7	0.01\\
599.71	0.01\\
599.72	0.01\\
599.73	0.01\\
599.74	0.01\\
599.75	0.01\\
599.76	0.01\\
599.77	0.01\\
599.78	0.01\\
599.79	0.01\\
599.8	0.01\\
599.81	0.01\\
599.82	0.01\\
599.83	0.01\\
599.84	0.01\\
599.85	0.01\\
599.86	0.01\\
599.87	0.01\\
599.88	0.01\\
599.89	0.01\\
599.9	0.01\\
599.91	0.01\\
599.92	0.01\\
599.93	0.01\\
599.94	0.01\\
599.95	0.01\\
599.96	0.01\\
599.97	0.01\\
599.98	0.01\\
599.99	0.01\\
600	0.01\\
};
\addplot [color=mycolor15,solid,forget plot]
  table[row sep=crcr]{%
0.01	0.00622155120829262\\
1.01	0.00622155120887005\\
2.01	0.00622155120945996\\
3.01	0.00622155121006258\\
4.01	0.00622155121067824\\
5.01	0.00622155121130722\\
6.01	0.00622155121194978\\
7.01	0.00622155121260623\\
8.01	0.00622155121327687\\
9.01	0.00622155121396201\\
10.01	0.00622155121466198\\
11.01	0.0062215512153771\\
12.01	0.00622155121610768\\
13.01	0.00622155121685406\\
14.01	0.00622155121761659\\
15.01	0.00622155121839564\\
16.01	0.00622155121919155\\
17.01	0.0062215512200047\\
18.01	0.00622155122083545\\
19.01	0.00622155122168422\\
20.01	0.00622155122255136\\
21.01	0.0062215512234373\\
22.01	0.00622155122434243\\
23.01	0.00622155122526719\\
24.01	0.00622155122621201\\
25.01	0.0062215512271773\\
26.01	0.00622155122816355\\
27.01	0.00622155122917119\\
28.01	0.00622155123020068\\
29.01	0.00622155123125252\\
30.01	0.00622155123232719\\
31.01	0.00622155123342521\\
32.01	0.00622155123454706\\
33.01	0.00622155123569328\\
34.01	0.0062215512368644\\
35.01	0.00622155123806097\\
36.01	0.00622155123928354\\
37.01	0.0062215512405327\\
38.01	0.006221551241809\\
39.01	0.00622155124311307\\
40.01	0.0062215512444455\\
41.01	0.00622155124580691\\
42.01	0.00622155124719796\\
43.01	0.00622155124861928\\
44.01	0.00622155125007153\\
45.01	0.00622155125155546\\
46.01	0.00622155125307165\\
47.01	0.00622155125462087\\
48.01	0.00622155125620387\\
49.01	0.00622155125782135\\
50.01	0.00622155125947409\\
51.01	0.00622155126116289\\
52.01	0.00622155126288849\\
53.01	0.00622155126465175\\
54.01	0.00622155126645348\\
55.01	0.00622155126829452\\
56.01	0.00622155127017576\\
57.01	0.00622155127209806\\
58.01	0.00622155127406237\\
59.01	0.00622155127606956\\
60.01	0.00622155127812062\\
61.01	0.00622155128021649\\
62.01	0.00622155128235819\\
63.01	0.00622155128454673\\
64.01	0.00622155128678311\\
65.01	0.00622155128906842\\
66.01	0.00622155129140374\\
67.01	0.00622155129379017\\
68.01	0.00622155129622886\\
69.01	0.00622155129872094\\
70.01	0.00622155130126761\\
71.01	0.00622155130387008\\
72.01	0.00622155130652957\\
73.01	0.00622155130924737\\
74.01	0.00622155131202476\\
75.01	0.00622155131486306\\
76.01	0.00622155131776365\\
77.01	0.00622155132072787\\
78.01	0.00622155132375717\\
79.01	0.00622155132685297\\
80.01	0.00622155133001677\\
81.01	0.00622155133325006\\
82.01	0.00622155133655442\\
83.01	0.0062215513399314\\
84.01	0.00622155134338264\\
85.01	0.00622155134690977\\
86.01	0.00622155135051451\\
87.01	0.00622155135419857\\
88.01	0.00622155135796372\\
89.01	0.00622155136181178\\
90.01	0.0062215513657446\\
91.01	0.00622155136976405\\
92.01	0.0062215513738721\\
93.01	0.00622155137807071\\
94.01	0.00622155138236189\\
95.01	0.00622155138674776\\
96.01	0.00622155139123038\\
97.01	0.00622155139581195\\
98.01	0.00622155140049468\\
99.01	0.00622155140528083\\
100.01	0.00622155141017272\\
101.01	0.00622155141517272\\
102.01	0.00622155142028325\\
103.01	0.00622155142550681\\
104.01	0.00622155143084591\\
105.01	0.00622155143630316\\
106.01	0.00622155144188121\\
107.01	0.00622155144758276\\
108.01	0.00622155145341064\\
109.01	0.00622155145936764\\
110.01	0.0062215514654567\\
111.01	0.00622155147168074\\
112.01	0.00622155147804284\\
113.01	0.00622155148454611\\
114.01	0.00622155149119374\\
115.01	0.00622155149798896\\
116.01	0.00622155150493512\\
117.01	0.00622155151203561\\
118.01	0.00622155151929394\\
119.01	0.00622155152671365\\
120.01	0.00622155153429843\\
121.01	0.00622155154205196\\
122.01	0.00622155154997809\\
123.01	0.00622155155808072\\
124.01	0.00622155156636384\\
125.01	0.00622155157483155\\
126.01	0.00622155158348804\\
127.01	0.00622155159233758\\
128.01	0.00622155160138456\\
129.01	0.00622155161063344\\
130.01	0.00622155162008883\\
131.01	0.00622155162975541\\
132.01	0.00622155163963799\\
133.01	0.00622155164974149\\
134.01	0.00622155166007092\\
135.01	0.00622155167063143\\
136.01	0.0062215516814283\\
137.01	0.00622155169246688\\
138.01	0.00622155170375274\\
139.01	0.00622155171529148\\
140.01	0.00622155172708886\\
141.01	0.00622155173915083\\
142.01	0.00622155175148341\\
143.01	0.00622155176409279\\
144.01	0.0062215517769853\\
145.01	0.00622155179016744\\
146.01	0.00622155180364584\\
147.01	0.00622155181742724\\
148.01	0.00622155183151864\\
149.01	0.00622155184592715\\
150.01	0.00622155186066002\\
151.01	0.00622155187572472\\
152.01	0.00622155189112885\\
153.01	0.00622155190688023\\
154.01	0.00622155192298685\\
155.01	0.00622155193945687\\
156.01	0.00622155195629866\\
157.01	0.00622155197352082\\
158.01	0.00622155199113207\\
159.01	0.00622155200914142\\
160.01	0.00622155202755804\\
161.01	0.00622155204639131\\
162.01	0.00622155206565091\\
163.01	0.00622155208534666\\
164.01	0.00622155210548866\\
165.01	0.00622155212608725\\
166.01	0.00622155214715297\\
167.01	0.00622155216869668\\
168.01	0.00622155219072945\\
169.01	0.00622155221326262\\
170.01	0.00622155223630781\\
171.01	0.0062215522598769\\
172.01	0.00622155228398209\\
173.01	0.00622155230863583\\
174.01	0.00622155233385086\\
175.01	0.00622155235964032\\
176.01	0.00622155238601752\\
177.01	0.00622155241299619\\
178.01	0.00622155244059035\\
179.01	0.00622155246881438\\
180.01	0.00622155249768297\\
181.01	0.0062215525272112\\
182.01	0.00622155255741447\\
183.01	0.00622155258830858\\
184.01	0.00622155261990972\\
185.01	0.00622155265223444\\
186.01	0.00622155268529968\\
187.01	0.00622155271912284\\
188.01	0.00622155275372167\\
189.01	0.00622155278911444\\
190.01	0.00622155282531972\\
191.01	0.00622155286235669\\
192.01	0.00622155290024489\\
193.01	0.00622155293900436\\
194.01	0.00622155297865559\\
195.01	0.00622155301921964\\
196.01	0.00622155306071802\\
197.01	0.00622155310317279\\
198.01	0.00622155314660652\\
199.01	0.00622155319104236\\
200.01	0.00622155323650397\\
201.01	0.00622155328301566\\
202.01	0.00622155333060226\\
203.01	0.00622155337928923\\
204.01	0.00622155342910266\\
205.01	0.00622155348006928\\
206.01	0.00622155353221643\\
207.01	0.00622155358557215\\
208.01	0.0062215536401652\\
209.01	0.00622155369602495\\
210.01	0.00622155375318155\\
211.01	0.00622155381166588\\
212.01	0.00622155387150955\\
213.01	0.006221553932745\\
214.01	0.00622155399540538\\
215.01	0.00622155405952473\\
216.01	0.00622155412513788\\
217.01	0.00622155419228052\\
218.01	0.00622155426098922\\
219.01	0.00622155433130148\\
220.01	0.00622155440325561\\
221.01	0.006221554476891\\
222.01	0.00622155455224797\\
223.01	0.00622155462936774\\
224.01	0.00622155470829266\\
225.01	0.00622155478906609\\
226.01	0.00622155487173239\\
227.01	0.0062215549563371\\
228.01	0.00622155504292686\\
229.01	0.00622155513154942\\
230.01	0.00622155522225376\\
231.01	0.00622155531509003\\
232.01	0.00622155541010964\\
233.01	0.00622155550736529\\
234.01	0.00622155560691094\\
235.01	0.00622155570880189\\
236.01	0.00622155581309484\\
237.01	0.00622155591984788\\
238.01	0.00622155602912051\\
239.01	0.00622155614097376\\
240.01	0.00622155625547011\\
241.01	0.00622155637267364\\
242.01	0.00622155649264999\\
243.01	0.00622155661546641\\
244.01	0.00622155674119189\\
245.01	0.00622155686989707\\
246.01	0.00622155700165436\\
247.01	0.00622155713653796\\
248.01	0.00622155727462399\\
249.01	0.00622155741599027\\
250.01	0.00622155756071679\\
251.01	0.0062215577088854\\
252.01	0.00622155786058\\
253.01	0.00622155801588656\\
254.01	0.00622155817489329\\
255.01	0.00622155833769045\\
256.01	0.00622155850437065\\
257.01	0.00622155867502878\\
258.01	0.00622155884976208\\
259.01	0.00622155902867026\\
260.01	0.00622155921185544\\
261.01	0.00622155939942235\\
262.01	0.00622155959147829\\
263.01	0.00622155978813329\\
264.01	0.00622155998950005\\
265.01	0.00622156019569416\\
266.01	0.00622156040683405\\
267.01	0.00622156062304108\\
268.01	0.00622156084443972\\
269.01	0.00622156107115744\\
270.01	0.00622156130332501\\
271.01	0.00622156154107637\\
272.01	0.00622156178454885\\
273.01	0.00622156203388321\\
274.01	0.0062215622892237\\
275.01	0.00622156255071821\\
276.01	0.00622156281851827\\
277.01	0.00622156309277926\\
278.01	0.0062215633736604\\
279.01	0.0062215636613249\\
280.01	0.00622156395594006\\
281.01	0.00622156425767731\\
282.01	0.00622156456671241\\
283.01	0.00622156488322552\\
284.01	0.00622156520740129\\
285.01	0.00622156553942895\\
286.01	0.0062215658795025\\
287.01	0.00622156622782078\\
288.01	0.00622156658458758\\
289.01	0.00622156695001182\\
290.01	0.0062215673243076\\
291.01	0.0062215677076944\\
292.01	0.00622156810039717\\
293.01	0.0062215685026465\\
294.01	0.0062215689146787\\
295.01	0.00622156933673608\\
296.01	0.00622156976906689\\
297.01	0.00622157021192568\\
298.01	0.00622157066557331\\
299.01	0.0062215711302772\\
300.01	0.00622157160631143\\
301.01	0.00622157209395695\\
302.01	0.00622157259350171\\
303.01	0.00622157310524091\\
304.01	0.00622157362947712\\
305.01	0.00622157416652048\\
306.01	0.00622157471668884\\
307.01	0.00622157528030811\\
308.01	0.00622157585771229\\
309.01	0.00622157644924378\\
310.01	0.00622157705525353\\
311.01	0.0062215776761013\\
312.01	0.00622157831215591\\
313.01	0.00622157896379535\\
314.01	0.00622157963140718\\
315.01	0.00622158031538865\\
316.01	0.00622158101614698\\
317.01	0.00622158173409967\\
318.01	0.00622158246967468\\
319.01	0.00622158322331077\\
320.01	0.00622158399545772\\
321.01	0.00622158478657667\\
322.01	0.0062215855971404\\
323.01	0.0062215864276336\\
324.01	0.00622158727855324\\
325.01	0.00622158815040886\\
326.01	0.00622158904372287\\
327.01	0.00622158995903096\\
328.01	0.00622159089688241\\
329.01	0.00622159185784045\\
330.01	0.00622159284248266\\
331.01	0.00622159385140135\\
332.01	0.00622159488520392\\
333.01	0.00622159594451332\\
334.01	0.00622159702996845\\
335.01	0.0062215981422246\\
336.01	0.0062215992819539\\
337.01	0.00622160044984581\\
338.01	0.00622160164660756\\
339.01	0.00622160287296467\\
340.01	0.00622160412966148\\
341.01	0.00622160541746165\\
342.01	0.00622160673714875\\
343.01	0.0062216080895268\\
344.01	0.00622160947542085\\
345.01	0.00622161089567762\\
346.01	0.00622161235116614\\
347.01	0.00622161384277835\\
348.01	0.00622161537142986\\
349.01	0.00622161693806054\\
350.01	0.0062216185436354\\
351.01	0.00622162018914516\\
352.01	0.00622162187560722\\
353.01	0.00622162360406636\\
354.01	0.00622162537559557\\
355.01	0.00622162719129703\\
356.01	0.0062216290523029\\
357.01	0.00622163095977639\\
358.01	0.00622163291491256\\
359.01	0.00622163491893956\\
360.01	0.00622163697311954\\
361.01	0.00622163907874975\\
362.01	0.00622164123716379\\
363.01	0.00622164344973269\\
364.01	0.00622164571786625\\
365.01	0.00622164804301427\\
366.01	0.00622165042666785\\
367.01	0.00622165287036091\\
368.01	0.0062216553756716\\
369.01	0.00622165794422379\\
370.01	0.00622166057768865\\
371.01	0.00622166327778641\\
372.01	0.00622166604628798\\
373.01	0.00622166888501681\\
374.01	0.00622167179585077\\
375.01	0.00622167478072413\\
376.01	0.0062216778416296\\
377.01	0.00622168098062057\\
378.01	0.00622168419981326\\
379.01	0.00622168750138925\\
380.01	0.00622169088759773\\
381.01	0.00622169436075843\\
382.01	0.00622169792326414\\
383.01	0.00622170157758361\\
384.01	0.00622170532626473\\
385.01	0.0062217091719376\\
386.01	0.00622171311731796\\
387.01	0.00622171716521068\\
388.01	0.00622172131851364\\
389.01	0.00622172558022155\\
390.01	0.00622172995343021\\
391.01	0.00622173444134093\\
392.01	0.00622173904726521\\
393.01	0.00622174377462969\\
394.01	0.00622174862698143\\
395.01	0.0062217536079934\\
396.01	0.00622175872147051\\
397.01	0.00622176397135567\\
398.01	0.00622176936173661\\
399.01	0.00622177489685269\\
400.01	0.00622178058110238\\
401.01	0.00622178641905114\\
402.01	0.00622179241543954\\
403.01	0.00622179857519203\\
404.01	0.00622180490342598\\
405.01	0.00622181140546122\\
406.01	0.00622181808682995\\
407.01	0.00622182495328689\\
408.01	0.00622183201081992\\
409.01	0.00622183926566072\\
410.01	0.00622184672429543\\
411.01	0.00622185439347518\\
412.01	0.00622186228022559\\
413.01	0.00622187039185207\\
414.01	0.00622187873592565\\
415.01	0.00622188732023299\\
416.01	0.0062218961530539\\
417.01	0.00622190524358895\\
418.01	0.00622191460404523\\
419.01	0.0062219242689194\\
420.01	0.00622193444799481\\
421.01	0.00622194712215251\\
422.01	0.006221979253888\\
423.01	0.00622202009866039\\
424.01	0.00622206214640011\\
425.01	0.0062221054381339\\
426.01	0.00622215001650164\\
427.01	0.00622219592581707\\
428.01	0.00622224321212885\\
429.01	0.00622229192328177\\
430.01	0.00622234210897748\\
431.01	0.00622239382083473\\
432.01	0.00622244711244828\\
433.01	0.00622250203944632\\
434.01	0.00622255865954566\\
435.01	0.0062226170326043\\
436.01	0.00622267722067105\\
437.01	0.00622273928803165\\
438.01	0.00622280330125103\\
439.01	0.00622286932921189\\
440.01	0.00622293744314948\\
441.01	0.00622300771668329\\
442.01	0.00622308022584675\\
443.01	0.00622315504911658\\
444.01	0.00622323226744481\\
445.01	0.00622331196429756\\
446.01	0.00622339422570616\\
447.01	0.0062234791403391\\
448.01	0.00622356679960483\\
449.01	0.00622365729779789\\
450.01	0.00622375073229723\\
451.01	0.00622384720384408\\
452.01	0.0062239468171431\\
453.01	0.00622404968153397\\
454.01	0.00622415591147117\\
455.01	0.00622426562807719\\
456.01	0.0062243789623651\\
457.01	0.00622449606955524\\
458.01	0.00622461722507646\\
459.01	0.00622474356308274\\
460.01	0.00622488233570331\\
461.01	0.00622504597766834\\
462.01	0.00622521630290586\\
463.01	0.00622539153089623\\
464.01	0.00622557181976172\\
465.01	0.00622575733485051\\
466.01	0.00622594824915053\\
467.01	0.00622614474340794\\
468.01	0.00622634700494363\\
469.01	0.00622655522548624\\
470.01	0.00622676961479236\\
471.01	0.00622699041375242\\
472.01	0.00622721795204145\\
473.01	0.00622745313423117\\
474.01	0.00622769881616883\\
475.01	0.00622795462605029\\
476.01	0.00622821814989739\\
477.01	0.0062284897063457\\
478.01	0.00622876966278281\\
479.01	0.00622905841569682\\
480.01	0.00622935639392405\\
481.01	0.00622966406231198\\
482.01	0.00622998192585361\\
483.01	0.00623031053434989\\
484.01	0.00623065048763709\\
485.01	0.0062310024413455\\
486.01	0.00623136711357164\\
487.01	0.00623174529462344\\
488.01	0.00623213785435975\\
489.01	0.00623254575046085\\
490.01	0.00623297004129636\\
491.01	0.00623341190095683\\
492.01	0.00623387264973774\\
493.01	0.00623435388759447\\
494.01	0.00623485840305036\\
495.01	0.00623539795663178\\
496.01	0.00623609367330455\\
497.01	0.00623723115650128\\
498.01	0.00623841312880773\\
499.01	0.00623963317120134\\
500.01	0.00624089297069207\\
501.01	0.00624219433087263\\
502.01	0.00624353918395286\\
503.01	0.00624492960313452\\
504.01	0.00624636782012887\\
505.01	0.00624785625643472\\
506.01	0.0062493975316999\\
507.01	0.00625099448791722\\
508.01	0.0062526502240204\\
509.01	0.00625436813154553\\
510.01	0.00625615193677643\\
511.01	0.00625800575114142\\
512.01	0.00625993413429125\\
513.01	0.00626194219242883\\
514.01	0.00626403587083316\\
515.01	0.00626622368312657\\
516.01	0.00626853002351242\\
517.01	0.00627108363368029\\
518.01	0.00627398335748705\\
519.01	0.00627701725843666\\
520.01	0.00628019768657443\\
521.01	0.00628359192946967\\
522.01	0.00628791812806813\\
523.01	0.00629425286810341\\
524.01	0.00630080273615997\\
525.01	0.00630757700598057\\
526.01	0.00631458789434046\\
527.01	0.00632184865783699\\
528.01	0.0063293737043835\\
529.01	0.00633717875027882\\
530.01	0.00634528099472029\\
531.01	0.00635369939877915\\
532.01	0.00636245555927423\\
533.01	0.00637157828048203\\
534.01	0.00638112238410973\\
535.01	0.00639099780968543\\
536.01	0.00640113448444666\\
537.01	0.00641176212081205\\
538.01	0.00642288681870064\\
539.01	0.00643456213066824\\
540.01	0.00644692533589876\\
541.01	0.00646090346237331\\
542.01	0.00648361374822135\\
543.01	0.00650939644223189\\
544.01	0.00653605287866877\\
545.01	0.00656357799123173\\
546.01	0.00659203582323319\\
547.01	0.00662155071739224\\
548.01	0.00665221862704111\\
549.01	0.00668441019219837\\
550.01	0.00671956868728407\\
551.01	0.00675457158607755\\
552.01	0.00679036743480562\\
553.01	0.00682746902595869\\
554.01	0.00686597622574857\\
555.01	0.00690600127058039\\
556.01	0.00694767174634376\\
557.01	0.00699114032408532\\
558.01	0.0070366419278237\\
559.01	0.00708482349212101\\
560.01	0.00713626857861373\\
561.01	0.0071829610180714\\
562.01	0.00724162846701401\\
563.01	0.00734597812578352\\
564.01	0.00745479814104576\\
565.01	0.00756487131968776\\
566.01	0.00767486113237245\\
567.01	0.00778744862957332\\
568.01	0.00790292696768786\\
569.01	0.00802148922935688\\
570.01	0.00814336003385193\\
571.01	0.00826886548535915\\
572.01	0.00839882785219053\\
573.01	0.0085370466661109\\
574.01	0.00866729652998845\\
575.01	0.00878465492073466\\
576.01	0.00890431818522714\\
577.01	0.00902628977288373\\
578.01	0.00915073393145681\\
579.01	0.00927770547942364\\
580.01	0.00940721524314349\\
581.01	0.00953925685629601\\
582.01	0.00967379261945761\\
583.01	0.00981066165926008\\
584.01	0.0099465256436832\\
585.01	0.01\\
586.01	0.01\\
587.01	0.01\\
588.01	0.01\\
589.01	0.01\\
590.01	0.01\\
591.01	0.01\\
592.01	0.01\\
593.01	0.01\\
594.01	0.01\\
595.01	0.01\\
596.01	0.01\\
597.01	0.01\\
598.01	0.01\\
599.01	0.01\\
599.02	0.01\\
599.03	0.01\\
599.04	0.01\\
599.05	0.01\\
599.06	0.01\\
599.07	0.01\\
599.08	0.01\\
599.09	0.01\\
599.1	0.01\\
599.11	0.01\\
599.12	0.01\\
599.13	0.01\\
599.14	0.01\\
599.15	0.01\\
599.16	0.01\\
599.17	0.01\\
599.18	0.01\\
599.19	0.01\\
599.2	0.01\\
599.21	0.01\\
599.22	0.01\\
599.23	0.01\\
599.24	0.01\\
599.25	0.01\\
599.26	0.01\\
599.27	0.01\\
599.28	0.01\\
599.29	0.01\\
599.3	0.01\\
599.31	0.01\\
599.32	0.01\\
599.33	0.01\\
599.34	0.01\\
599.35	0.01\\
599.36	0.01\\
599.37	0.01\\
599.38	0.01\\
599.39	0.01\\
599.4	0.01\\
599.41	0.01\\
599.42	0.01\\
599.43	0.01\\
599.44	0.01\\
599.45	0.01\\
599.46	0.01\\
599.47	0.01\\
599.48	0.01\\
599.49	0.01\\
599.5	0.01\\
599.51	0.01\\
599.52	0.01\\
599.53	0.01\\
599.54	0.01\\
599.55	0.01\\
599.56	0.01\\
599.57	0.01\\
599.58	0.01\\
599.59	0.01\\
599.6	0.01\\
599.61	0.01\\
599.62	0.01\\
599.63	0.01\\
599.64	0.01\\
599.65	0.01\\
599.66	0.01\\
599.67	0.01\\
599.68	0.01\\
599.69	0.01\\
599.7	0.01\\
599.71	0.01\\
599.72	0.01\\
599.73	0.01\\
599.74	0.01\\
599.75	0.01\\
599.76	0.01\\
599.77	0.01\\
599.78	0.01\\
599.79	0.01\\
599.8	0.01\\
599.81	0.01\\
599.82	0.01\\
599.83	0.01\\
599.84	0.01\\
599.85	0.01\\
599.86	0.01\\
599.87	0.01\\
599.88	0.01\\
599.89	0.01\\
599.9	0.01\\
599.91	0.01\\
599.92	0.01\\
599.93	0.01\\
599.94	0.01\\
599.95	0.01\\
599.96	0.01\\
599.97	0.01\\
599.98	0.01\\
599.99	0.01\\
600	0.01\\
};
\addplot [color=mycolor16,solid,forget plot]
  table[row sep=crcr]{%
0.01	0.0075997228867832\\
1.01	0.00759972289627518\\
2.01	0.00759972290597215\\
3.01	0.00759972291587861\\
4.01	0.00759972292599908\\
5.01	0.00759972293633824\\
6.01	0.00759972294690085\\
7.01	0.00759972295769177\\
8.01	0.00759972296871599\\
9.01	0.00759972297997857\\
10.01	0.00759972299148474\\
11.01	0.00759972300323978\\
12.01	0.00759972301524913\\
13.01	0.00759972302751832\\
14.01	0.00759972304005305\\
15.01	0.00759972305285909\\
16.01	0.00759972306594238\\
17.01	0.00759972307930895\\
18.01	0.00759972309296501\\
19.01	0.00759972310691687\\
20.01	0.00759972312117099\\
21.01	0.00759972313573398\\
22.01	0.00759972315061258\\
23.01	0.00759972316581372\\
24.01	0.00759972318134442\\
25.01	0.00759972319721191\\
26.01	0.00759972321342354\\
27.01	0.00759972322998687\\
28.01	0.00759972324690956\\
29.01	0.00759972326419952\\
30.01	0.00759972328186475\\
31.01	0.00759972329991352\\
32.01	0.00759972331835418\\
33.01	0.00759972333719536\\
34.01	0.00759972335644583\\
35.01	0.00759972337611456\\
36.01	0.00759972339621072\\
37.01	0.0075997234167437\\
38.01	0.00759972343772309\\
39.01	0.00759972345915866\\
40.01	0.00759972348106046\\
41.01	0.00759972350343872\\
42.01	0.00759972352630391\\
43.01	0.00759972354966672\\
44.01	0.00759972357353811\\
45.01	0.00759972359792922\\
46.01	0.00759972362285154\\
47.01	0.00759972364831672\\
48.01	0.00759972367433672\\
49.01	0.00759972370092373\\
50.01	0.00759972372809028\\
51.01	0.00759972375584908\\
52.01	0.00759972378421323\\
53.01	0.00759972381319604\\
54.01	0.00759972384281115\\
55.01	0.00759972387307254\\
56.01	0.00759972390399441\\
57.01	0.00759972393559138\\
58.01	0.00759972396787833\\
59.01	0.00759972400087051\\
60.01	0.00759972403458348\\
61.01	0.00759972406903316\\
62.01	0.00759972410423585\\
63.01	0.0075997241402082\\
64.01	0.00759972417696721\\
65.01	0.00759972421453028\\
66.01	0.00759972425291524\\
67.01	0.00759972429214024\\
68.01	0.00759972433222391\\
69.01	0.00759972437318526\\
70.01	0.00759972441504373\\
71.01	0.00759972445781921\\
72.01	0.00759972450153205\\
73.01	0.00759972454620303\\
74.01	0.0075997245918534\\
75.01	0.0075997246385049\\
76.01	0.00759972468617976\\
77.01	0.00759972473490073\\
78.01	0.00759972478469102\\
79.01	0.00759972483557442\\
80.01	0.00759972488757523\\
81.01	0.0075997249407183\\
82.01	0.00759972499502906\\
83.01	0.0075997250505335\\
84.01	0.00759972510725817\\
85.01	0.00759972516523029\\
86.01	0.00759972522447765\\
87.01	0.00759972528502864\\
88.01	0.00759972534691238\\
89.01	0.00759972541015857\\
90.01	0.00759972547479763\\
91.01	0.00759972554086066\\
92.01	0.00759972560837946\\
93.01	0.00759972567738655\\
94.01	0.00759972574791521\\
95.01	0.00759972581999944\\
96.01	0.00759972589367407\\
97.01	0.00759972596897467\\
98.01	0.00759972604593764\\
99.01	0.00759972612460022\\
100.01	0.00759972620500049\\
101.01	0.00759972628717741\\
102.01	0.00759972637117079\\
103.01	0.0075997264570214\\
104.01	0.00759972654477091\\
105.01	0.00759972663446199\\
106.01	0.00759972672613821\\
107.01	0.00759972681984419\\
108.01	0.00759972691562557\\
109.01	0.00759972701352902\\
110.01	0.00759972711360229\\
111.01	0.00759972721589423\\
112.01	0.0075997273204548\\
113.01	0.0075997274273351\\
114.01	0.00759972753658743\\
115.01	0.00759972764826528\\
116.01	0.00759972776242337\\
117.01	0.00759972787911767\\
118.01	0.00759972799840547\\
119.01	0.00759972812034534\\
120.01	0.00759972824499724\\
121.01	0.00759972837242248\\
122.01	0.00759972850268381\\
123.01	0.00759972863584542\\
124.01	0.00759972877197297\\
125.01	0.00759972891113368\\
126.01	0.00759972905339626\\
127.01	0.00759972919883109\\
128.01	0.0075997293475101\\
129.01	0.00759972949950697\\
130.01	0.00759972965489703\\
131.01	0.00759972981375736\\
132.01	0.00759972997616687\\
133.01	0.00759973014220627\\
134.01	0.00759973031195815\\
135.01	0.00759973048550703\\
136.01	0.00759973066293941\\
137.01	0.00759973084434378\\
138.01	0.00759973102981073\\
139.01	0.00759973121943288\\
140.01	0.00759973141330514\\
141.01	0.00759973161152451\\
142.01	0.00759973181419033\\
143.01	0.00759973202140427\\
144.01	0.00759973223327033\\
145.01	0.00759973244989497\\
146.01	0.00759973267138716\\
147.01	0.00759973289785838\\
148.01	0.00759973312942279\\
149.01	0.00759973336619716\\
150.01	0.00759973360830106\\
151.01	0.00759973385585684\\
152.01	0.00759973410898973\\
153.01	0.0075997343678279\\
154.01	0.00759973463250259\\
155.01	0.00759973490314805\\
156.01	0.00759973517990174\\
157.01	0.00759973546290439\\
158.01	0.00759973575229999\\
159.01	0.00759973604823598\\
160.01	0.00759973635086327\\
161.01	0.00759973666033635\\
162.01	0.00759973697681332\\
163.01	0.00759973730045613\\
164.01	0.00759973763143042\\
165.01	0.0075997379699059\\
166.01	0.00759973831605624\\
167.01	0.00759973867005919\\
168.01	0.00759973903209684\\
169.01	0.00759973940235551\\
170.01	0.007599739781026\\
171.01	0.00759974016830364\\
172.01	0.00759974056438843\\
173.01	0.0075997409694851\\
174.01	0.00759974138380334\\
175.01	0.00759974180755772\\
176.01	0.00759974224096808\\
177.01	0.00759974268425942\\
178.01	0.00759974313766217\\
179.01	0.00759974360141224\\
180.01	0.00759974407575121\\
181.01	0.00759974456092645\\
182.01	0.00759974505719126\\
183.01	0.00759974556480504\\
184.01	0.00759974608403338\\
185.01	0.00759974661514828\\
186.01	0.00759974715842828\\
187.01	0.00759974771415862\\
188.01	0.0075997482826314\\
189.01	0.00759974886414575\\
190.01	0.00759974945900806\\
191.01	0.00759975006753205\\
192.01	0.00759975069003905\\
193.01	0.00759975132685816\\
194.01	0.0075997519783264\\
195.01	0.00759975264478898\\
196.01	0.00759975332659944\\
197.01	0.00759975402411987\\
198.01	0.00759975473772119\\
199.01	0.00759975546778324\\
200.01	0.00759975621469514\\
201.01	0.00759975697885542\\
202.01	0.00759975776067233\\
203.01	0.00759975856056398\\
204.01	0.00759975937895871\\
205.01	0.00759976021629528\\
206.01	0.00759976107302309\\
207.01	0.00759976194960251\\
208.01	0.0075997628465052\\
209.01	0.00759976376421423\\
210.01	0.0075997647032245\\
211.01	0.00759976566404304\\
212.01	0.0075997666471892\\
213.01	0.00759976765319507\\
214.01	0.00759976868260577\\
215.01	0.00759976973597974\\
216.01	0.0075997708138891\\
217.01	0.00759977191692\\
218.01	0.00759977304567301\\
219.01	0.00759977420076332\\
220.01	0.00759977538282132\\
221.01	0.00759977659249287\\
222.01	0.00759977783043965\\
223.01	0.00759977909733968\\
224.01	0.00759978039388761\\
225.01	0.00759978172079519\\
226.01	0.00759978307879174\\
227.01	0.00759978446862448\\
228.01	0.00759978589105911\\
229.01	0.00759978734688018\\
230.01	0.0075997888368916\\
231.01	0.00759979036191713\\
232.01	0.00759979192280089\\
233.01	0.00759979352040779\\
234.01	0.00759979515562419\\
235.01	0.00759979682935834\\
236.01	0.00759979854254096\\
237.01	0.00759980029612582\\
238.01	0.00759980209109026\\
239.01	0.00759980392843586\\
240.01	0.00759980580918903\\
241.01	0.0075998077344016\\
242.01	0.00759980970515147\\
243.01	0.00759981172254335\\
244.01	0.00759981378770926\\
245.01	0.00759981590180939\\
246.01	0.00759981806603273\\
247.01	0.0075998202815978\\
248.01	0.00759982254975339\\
249.01	0.00759982487177936\\
250.01	0.00759982724898734\\
251.01	0.00759982968272164\\
252.01	0.00759983217436\\
253.01	0.0075998347253144\\
254.01	0.00759983733703203\\
255.01	0.00759984001099609\\
256.01	0.0075998427487267\\
257.01	0.00759984555178185\\
258.01	0.00759984842175841\\
259.01	0.00759985136029292\\
260.01	0.00759985436906287\\
261.01	0.00759985744978746\\
262.01	0.00759986060422879\\
263.01	0.00759986383419293\\
264.01	0.00759986714153099\\
265.01	0.00759987052814024\\
266.01	0.00759987399596534\\
267.01	0.00759987754699946\\
268.01	0.0075998811832855\\
269.01	0.0075998849069174\\
270.01	0.00759988872004135\\
271.01	0.00759989262485712\\
272.01	0.00759989662361944\\
273.01	0.00759990071863936\\
274.01	0.00759990491228559\\
275.01	0.00759990920698607\\
276.01	0.00759991360522938\\
277.01	0.00759991810956623\\
278.01	0.00759992272261109\\
279.01	0.00759992744704378\\
280.01	0.00759993228561101\\
281.01	0.00759993724112818\\
282.01	0.00759994231648104\\
283.01	0.00759994751462741\\
284.01	0.00759995283859906\\
285.01	0.00759995829150353\\
286.01	0.00759996387652602\\
287.01	0.00759996959693133\\
288.01	0.00759997545606586\\
289.01	0.00759998145735966\\
290.01	0.00759998760432855\\
291.01	0.00759999390057616\\
292.01	0.00760000034979624\\
293.01	0.00760000695577486\\
294.01	0.0076000137223928\\
295.01	0.00760002065362776\\
296.01	0.00760002775355698\\
297.01	0.00760003502635958\\
298.01	0.00760004247631926\\
299.01	0.00760005010782679\\
300.01	0.00760005792538275\\
301.01	0.00760006593360032\\
302.01	0.00760007413720808\\
303.01	0.00760008254105289\\
304.01	0.00760009115010298\\
305.01	0.00760009996945083\\
306.01	0.00760010900431646\\
307.01	0.00760011826005058\\
308.01	0.0076001277421379\\
309.01	0.00760013745620057\\
310.01	0.0076001474080016\\
311.01	0.00760015760344849\\
312.01	0.00760016804859684\\
313.01	0.00760017874965424\\
314.01	0.00760018971298399\\
315.01	0.00760020094510929\\
316.01	0.00760021245271712\\
317.01	0.00760022424266258\\
318.01	0.00760023632197322\\
319.01	0.00760024869785342\\
320.01	0.00760026137768905\\
321.01	0.00760027436905213\\
322.01	0.00760028767970563\\
323.01	0.0076003013176086\\
324.01	0.00760031529092114\\
325.01	0.00760032960800982\\
326.01	0.00760034427745301\\
327.01	0.00760035930804658\\
328.01	0.0076003747088096\\
329.01	0.00760039048899038\\
330.01	0.00760040665807249\\
331.01	0.00760042322578122\\
332.01	0.00760044020208998\\
333.01	0.00760045759722715\\
334.01	0.00760047542168291\\
335.01	0.00760049368621652\\
336.01	0.00760051240186368\\
337.01	0.00760053157994412\\
338.01	0.00760055123206961\\
339.01	0.00760057137015203\\
340.01	0.00760059200641189\\
341.01	0.00760061315338696\\
342.01	0.00760063482394134\\
343.01	0.00760065703127484\\
344.01	0.00760067978893257\\
345.01	0.00760070311081498\\
346.01	0.00760072701118814\\
347.01	0.00760075150469457\\
348.01	0.00760077660636424\\
349.01	0.0076008023316262\\
350.01	0.00760082869632034\\
351.01	0.00760085571670991\\
352.01	0.00760088340949429\\
353.01	0.00760091179182226\\
354.01	0.00760094088130588\\
355.01	0.00760097069603476\\
356.01	0.00760100125459091\\
357.01	0.00760103257606429\\
358.01	0.00760106468006884\\
359.01	0.00760109758675905\\
360.01	0.00760113131684741\\
361.01	0.00760116589162238\\
362.01	0.0076012013329672\\
363.01	0.00760123766337925\\
364.01	0.00760127490599049\\
365.01	0.00760131308458853\\
366.01	0.00760135222363861\\
367.01	0.00760139234830653\\
368.01	0.00760143348448254\\
369.01	0.00760147565880621\\
370.01	0.00760151889869246\\
371.01	0.00760156323235858\\
372.01	0.00760160868885252\\
373.01	0.00760165529808248\\
374.01	0.0076017030908477\\
375.01	0.00760175209887078\\
376.01	0.00760180235483138\\
377.01	0.00760185389240156\\
378.01	0.00760190674628281\\
379.01	0.00760196095224471\\
380.01	0.00760201654716569\\
381.01	0.00760207356907561\\
382.01	0.00760213205720063\\
383.01	0.00760219205201022\\
384.01	0.00760225359526671\\
385.01	0.00760231673007743\\
386.01	0.00760238150094952\\
387.01	0.00760244795384794\\
388.01	0.00760251613625645\\
389.01	0.00760258609724204\\
390.01	0.00760265788752318\\
391.01	0.00760273155954169\\
392.01	0.007602807167539\\
393.01	0.00760288476763681\\
394.01	0.00760296441792241\\
395.01	0.00760304617853924\\
396.01	0.0076031301117826\\
397.01	0.00760321628220133\\
398.01	0.00760330475670521\\
399.01	0.00760339560467894\\
400.01	0.00760348889810256\\
401.01	0.00760358471167878\\
402.01	0.00760368312296734\\
403.01	0.00760378421252653\\
404.01	0.00760388806406186\\
405.01	0.0076039947645817\\
406.01	0.00760410440455963\\
407.01	0.00760421707810279\\
408.01	0.00760433288312531\\
409.01	0.00760445192152548\\
410.01	0.00760457429936466\\
411.01	0.00760470012704498\\
412.01	0.0076048295194813\\
413.01	0.00760496259625344\\
414.01	0.00760509948166441\\
415.01	0.00760524030430592\\
416.01	0.00760538519787622\\
417.01	0.00760553430562554\\
418.01	0.00760568777766635\\
419.01	0.00760584580684255\\
420.01	0.00760600885696256\\
421.01	0.00760617863052147\\
422.01	0.00760634640848477\\
423.01	0.00760651319648749\\
424.01	0.00760668478246389\\
425.01	0.00760686132953937\\
426.01	0.0076070430081998\\
427.01	0.00760722999677242\\
428.01	0.00760742248195161\\
429.01	0.00760762065937462\\
430.01	0.00760782473425401\\
431.01	0.00760803492207397\\
432.01	0.00760825144935882\\
433.01	0.00760847455452393\\
434.01	0.00760870448882025\\
435.01	0.00760894151738591\\
436.01	0.00760918592042077\\
437.01	0.00760943799450191\\
438.01	0.00760969805406225\\
439.01	0.00760996643305728\\
440.01	0.00761024348685034\\
441.01	0.00761052959435185\\
442.01	0.00761082516045514\\
443.01	0.00761113061881869\\
444.01	0.00761144643505497\\
445.01	0.00761177311039714\\
446.01	0.00761211118592914\\
447.01	0.00761246124748107\\
448.01	0.00761282393131137\\
449.01	0.00761319993071357\\
450.01	0.00761359000366071\\
451.01	0.0076139949814758\\
452.01	0.00761441578020722\\
453.01	0.0076148534162388\\
454.01	0.00761530901516673\\
455.01	0.00761578383195522\\
456.01	0.00761627927634888\\
457.01	0.0076167969664957\\
458.01	0.00761733901033187\\
459.01	0.00761791060538318\\
460.01	0.00761855555398422\\
461.01	0.00761960975108633\\
462.01	0.00762079844158388\\
463.01	0.00762202335305392\\
464.01	0.0076232856601801\\
465.01	0.00762458656691337\\
466.01	0.00762592730344935\\
467.01	0.00762730912025099\\
468.01	0.00762873326755406\\
469.01	0.00763020092171577\\
470.01	0.00763171325348148\\
471.01	0.00763327176100925\\
472.01	0.0076348781059127\\
473.01	0.00763653736701755\\
474.01	0.00763828430449948\\
475.01	0.00764015497130059\\
476.01	0.00764208033378298\\
477.01	0.00764405979089676\\
478.01	0.00764609515843145\\
479.01	0.0076481883331306\\
480.01	0.00765034129745081\\
481.01	0.00765255612462537\\
482.01	0.00765483498403869\\
483.01	0.00765718014689514\\
484.01	0.00765959399202028\\
485.01	0.00766207901078184\\
486.01	0.00766463780923741\\
487.01	0.00766727313250799\\
488.01	0.00766998787905202\\
489.01	0.0076727850659321\\
490.01	0.00767566785836698\\
491.01	0.00767863958285805\\
492.01	0.00768170376233876\\
493.01	0.00768486432687156\\
494.01	0.00768812708722715\\
495.01	0.00769150920317194\\
496.01	0.00769502995574532\\
497.01	0.00769841915877319\\
498.01	0.0077019095592544\\
499.01	0.00770551370753597\\
500.01	0.00770923696302975\\
501.01	0.00771308510211642\\
502.01	0.00771706436780387\\
503.01	0.00772118150695649\\
504.01	0.00772544380366489\\
505.01	0.00772985930962392\\
506.01	0.00773443685937805\\
507.01	0.00773918603597756\\
508.01	0.0077441173773363\\
509.01	0.00774924251481989\\
510.01	0.00775457433445633\\
511.01	0.00776012716816022\\
512.01	0.00776591702640784\\
513.01	0.0077719619228586\\
514.01	0.00777828267036745\\
515.01	0.00778490770485502\\
516.01	0.00779192274698233\\
517.01	0.0078002441517052\\
518.01	0.00781419974703577\\
519.01	0.00782910294009469\\
520.01	0.0078445061763311\\
521.01	0.00786050236092747\\
522.01	0.00787705367427117\\
523.01	0.00789294633197331\\
524.01	0.00790933261609763\\
525.01	0.00792623894042811\\
526.01	0.00794368808082389\\
527.01	0.00796170438911548\\
528.01	0.00798031352540036\\
529.01	0.00799954282651978\\
530.01	0.00801942148739208\\
531.01	0.00803998086201145\\
532.01	0.00806125546453226\\
533.01	0.00808328691513915\\
534.01	0.00810612538960956\\
535.01	0.00812974042920438\\
536.01	0.0081546802971531\\
537.01	0.00818147614643572\\
538.01	0.0082092359172629\\
539.01	0.00823802012322006\\
540.01	0.00826803462013862\\
541.01	0.00830004988378167\\
542.01	0.00833035837788714\\
543.01	0.00835914708340572\\
544.01	0.00838899167252555\\
545.01	0.00841909986528914\\
546.01	0.00844855488491541\\
547.01	0.00847937083965617\\
548.01	0.00851179792589284\\
549.01	0.0085468865663905\\
550.01	0.0086022412879655\\
551.01	0.00868415334816426\\
552.01	0.00876776352755252\\
553.01	0.00885322269770616\\
554.01	0.0089406222800461\\
555.01	0.00903006284670626\\
556.01	0.00912165662638719\\
557.01	0.00921553778926058\\
558.01	0.00931191955770558\\
559.01	0.00941127044642449\\
560.01	0.00951264538348107\\
561.01	0.00961458353049139\\
562.01	0.0097243347526615\\
563.01	0.00981553266355713\\
564.01	0.0099057566752182\\
565.01	0.00998914127892605\\
566.01	0.01\\
567.01	0.01\\
568.01	0.01\\
569.01	0.01\\
570.01	0.01\\
571.01	0.01\\
572.01	0.01\\
573.01	0.01\\
574.01	0.01\\
575.01	0.01\\
576.01	0.01\\
577.01	0.01\\
578.01	0.01\\
579.01	0.01\\
580.01	0.01\\
581.01	0.01\\
582.01	0.01\\
583.01	0.01\\
584.01	0.01\\
585.01	0.01\\
586.01	0.01\\
587.01	0.01\\
588.01	0.01\\
589.01	0.01\\
590.01	0.01\\
591.01	0.01\\
592.01	0.01\\
593.01	0.01\\
594.01	0.01\\
595.01	0.01\\
596.01	0.01\\
597.01	0.01\\
598.01	0.01\\
599.01	0.01\\
599.02	0.01\\
599.03	0.01\\
599.04	0.01\\
599.05	0.01\\
599.06	0.01\\
599.07	0.01\\
599.08	0.01\\
599.09	0.01\\
599.1	0.01\\
599.11	0.01\\
599.12	0.01\\
599.13	0.01\\
599.14	0.01\\
599.15	0.01\\
599.16	0.01\\
599.17	0.01\\
599.18	0.01\\
599.19	0.01\\
599.2	0.01\\
599.21	0.01\\
599.22	0.01\\
599.23	0.01\\
599.24	0.01\\
599.25	0.01\\
599.26	0.01\\
599.27	0.01\\
599.28	0.01\\
599.29	0.01\\
599.3	0.01\\
599.31	0.01\\
599.32	0.01\\
599.33	0.01\\
599.34	0.01\\
599.35	0.01\\
599.36	0.01\\
599.37	0.01\\
599.38	0.01\\
599.39	0.01\\
599.4	0.01\\
599.41	0.01\\
599.42	0.01\\
599.43	0.01\\
599.44	0.01\\
599.45	0.01\\
599.46	0.01\\
599.47	0.01\\
599.48	0.01\\
599.49	0.01\\
599.5	0.01\\
599.51	0.01\\
599.52	0.01\\
599.53	0.01\\
599.54	0.01\\
599.55	0.01\\
599.56	0.01\\
599.57	0.01\\
599.58	0.01\\
599.59	0.01\\
599.6	0.01\\
599.61	0.01\\
599.62	0.01\\
599.63	0.01\\
599.64	0.01\\
599.65	0.01\\
599.66	0.01\\
599.67	0.01\\
599.68	0.01\\
599.69	0.01\\
599.7	0.01\\
599.71	0.01\\
599.72	0.01\\
599.73	0.01\\
599.74	0.01\\
599.75	0.01\\
599.76	0.01\\
599.77	0.01\\
599.78	0.01\\
599.79	0.01\\
599.8	0.01\\
599.81	0.01\\
599.82	0.01\\
599.83	0.01\\
599.84	0.01\\
599.85	0.01\\
599.86	0.01\\
599.87	0.01\\
599.88	0.01\\
599.89	0.01\\
599.9	0.01\\
599.91	0.01\\
599.92	0.01\\
599.93	0.01\\
599.94	0.01\\
599.95	0.01\\
599.96	0.01\\
599.97	0.01\\
599.98	0.01\\
599.99	0.01\\
600	0.01\\
};
\addplot [color=mycolor17,solid,forget plot]
  table[row sep=crcr]{%
0.01	0.00851903386541942\\
1.01	0.00851903389681088\\
2.01	0.00851903392887927\\
3.01	0.00851903396163929\\
4.01	0.00851903399510599\\
5.01	0.0085190340292947\\
6.01	0.00851903406422111\\
7.01	0.00851903409990128\\
8.01	0.00851903413635157\\
9.01	0.00851903417358875\\
10.01	0.00851903421162992\\
11.01	0.00851903425049259\\
12.01	0.0085190342901946\\
13.01	0.00851903433075426\\
14.01	0.00851903437219022\\
15.01	0.00851903441452157\\
16.01	0.0085190344577678\\
17.01	0.00851903450194884\\
18.01	0.00851903454708506\\
19.01	0.00851903459319728\\
20.01	0.00851903464030676\\
21.01	0.00851903468843526\\
22.01	0.008519034737605\\
23.01	0.0085190347878387\\
24.01	0.00851903483915954\\
25.01	0.00851903489159127\\
26.01	0.00851903494515813\\
27.01	0.0085190349998849\\
28.01	0.00851903505579692\\
29.01	0.00851903511292007\\
30.01	0.00851903517128081\\
31.01	0.00851903523090618\\
32.01	0.00851903529182384\\
33.01	0.00851903535406203\\
34.01	0.00851903541764963\\
35.01	0.00851903548261616\\
36.01	0.00851903554899178\\
37.01	0.00851903561680733\\
38.01	0.00851903568609434\\
39.01	0.00851903575688501\\
40.01	0.0085190358292123\\
41.01	0.00851903590310986\\
42.01	0.00851903597861208\\
43.01	0.00851903605575415\\
44.01	0.00851903613457203\\
45.01	0.00851903621510241\\
46.01	0.00851903629738294\\
47.01	0.00851903638145197\\
48.01	0.00851903646734875\\
49.01	0.00851903655511342\\
50.01	0.008519036644787\\
51.01	0.0085190367364114\\
52.01	0.00851903683002949\\
53.01	0.00851903692568508\\
54.01	0.00851903702342297\\
55.01	0.00851903712328895\\
56.01	0.0085190372253298\\
57.01	0.00851903732959339\\
58.01	0.00851903743612863\\
59.01	0.00851903754498555\\
60.01	0.00851903765621525\\
61.01	0.00851903776987001\\
62.01	0.00851903788600325\\
63.01	0.00851903800466959\\
64.01	0.00851903812592489\\
65.01	0.00851903824982625\\
66.01	0.00851903837643202\\
67.01	0.0085190385058019\\
68.01	0.00851903863799688\\
69.01	0.00851903877307935\\
70.01	0.00851903891111307\\
71.01	0.00851903905216327\\
72.01	0.00851903919629661\\
73.01	0.00851903934358124\\
74.01	0.00851903949408684\\
75.01	0.0085190396478847\\
76.01	0.00851903980504764\\
77.01	0.00851903996565017\\
78.01	0.00851904012976845\\
79.01	0.00851904029748035\\
80.01	0.0085190404688655\\
81.01	0.00851904064400534\\
82.01	0.00851904082298309\\
83.01	0.0085190410058839\\
84.01	0.00851904119279483\\
85.01	0.00851904138380485\\
86.01	0.00851904157900499\\
87.01	0.00851904177848834\\
88.01	0.00851904198235004\\
89.01	0.00851904219068744\\
90.01	0.00851904240360003\\
91.01	0.00851904262118959\\
92.01	0.00851904284356017\\
93.01	0.00851904307081822\\
94.01	0.00851904330307255\\
95.01	0.00851904354043446\\
96.01	0.00851904378301777\\
97.01	0.00851904403093887\\
98.01	0.0085190442843168\\
99.01	0.00851904454327332\\
100.01	0.0085190448079329\\
101.01	0.00851904507842285\\
102.01	0.00851904535487344\\
103.01	0.00851904563741782\\
104.01	0.00851904592619221\\
105.01	0.00851904622133591\\
106.01	0.00851904652299143\\
107.01	0.00851904683130446\\
108.01	0.00851904714642407\\
109.01	0.00851904746850268\\
110.01	0.00851904779769626\\
111.01	0.00851904813416426\\
112.01	0.00851904847806982\\
113.01	0.0085190488295798\\
114.01	0.00851904918886484\\
115.01	0.00851904955609954\\
116.01	0.00851904993146245\\
117.01	0.00851905031513625\\
118.01	0.00851905070730774\\
119.01	0.00851905110816807\\
120.01	0.00851905151791277\\
121.01	0.00851905193674178\\
122.01	0.00851905236485974\\
123.01	0.00851905280247592\\
124.01	0.0085190532498044\\
125.01	0.00851905370706421\\
126.01	0.00851905417447942\\
127.01	0.00851905465227925\\
128.01	0.0085190551406982\\
129.01	0.00851905563997617\\
130.01	0.00851905615035861\\
131.01	0.00851905667209663\\
132.01	0.00851905720544713\\
133.01	0.00851905775067294\\
134.01	0.00851905830804299\\
135.01	0.00851905887783243\\
136.01	0.00851905946032274\\
137.01	0.00851906005580197\\
138.01	0.00851906066456478\\
139.01	0.00851906128691276\\
140.01	0.0085190619231544\\
141.01	0.0085190625736054\\
142.01	0.00851906323858882\\
143.01	0.00851906391843514\\
144.01	0.00851906461348262\\
145.01	0.00851906532407734\\
146.01	0.00851906605057345\\
147.01	0.00851906679333339\\
148.01	0.00851906755272798\\
149.01	0.00851906832913673\\
150.01	0.00851906912294802\\
151.01	0.00851906993455927\\
152.01	0.00851907076437724\\
153.01	0.00851907161281816\\
154.01	0.00851907248030799\\
155.01	0.00851907336728273\\
156.01	0.00851907427418853\\
157.01	0.00851907520148204\\
158.01	0.00851907614963062\\
159.01	0.00851907711911257\\
160.01	0.00851907811041749\\
161.01	0.00851907912404642\\
162.01	0.00851908016051225\\
163.01	0.00851908122033991\\
164.01	0.00851908230406669\\
165.01	0.00851908341224257\\
166.01	0.00851908454543047\\
167.01	0.00851908570420663\\
168.01	0.00851908688916085\\
169.01	0.00851908810089696\\
170.01	0.00851908934003295\\
171.01	0.00851909060720155\\
172.01	0.00851909190305039\\
173.01	0.00851909322824248\\
174.01	0.00851909458345653\\
175.01	0.00851909596938742\\
176.01	0.00851909738674645\\
177.01	0.00851909883626185\\
178.01	0.00851910031867916\\
179.01	0.00851910183476169\\
180.01	0.00851910338529087\\
181.01	0.00851910497106679\\
182.01	0.00851910659290859\\
183.01	0.00851910825165496\\
184.01	0.00851910994816461\\
185.01	0.00851911168331674\\
186.01	0.00851911345801163\\
187.01	0.00851911527317101\\
188.01	0.00851911712973874\\
189.01	0.00851911902868126\\
190.01	0.00851912097098817\\
191.01	0.00851912295767282\\
192.01	0.00851912498977289\\
193.01	0.00851912706835093\\
194.01	0.00851912919449509\\
195.01	0.00851913136931964\\
196.01	0.00851913359396569\\
197.01	0.00851913586960185\\
198.01	0.00851913819742483\\
199.01	0.00851914057866026\\
200.01	0.00851914301456329\\
201.01	0.00851914550641944\\
202.01	0.00851914805554521\\
203.01	0.008519150663289\\
204.01	0.00851915333103179\\
205.01	0.00851915606018802\\
206.01	0.00851915885220637\\
207.01	0.00851916170857064\\
208.01	0.00851916463080064\\
209.01	0.00851916762045307\\
210.01	0.00851917067912239\\
211.01	0.00851917380844189\\
212.01	0.00851917701008451\\
213.01	0.00851918028576393\\
214.01	0.00851918363723557\\
215.01	0.0085191870662976\\
216.01	0.00851919057479201\\
217.01	0.00851919416460578\\
218.01	0.00851919783767192\\
219.01	0.00851920159597066\\
220.01	0.00851920544153062\\
221.01	0.00851920937643004\\
222.01	0.00851921340279799\\
223.01	0.00851921752281573\\
224.01	0.00851922173871789\\
225.01	0.00851922605279391\\
226.01	0.0085192304673894\\
227.01	0.0085192349849075\\
228.01	0.00851923960781038\\
229.01	0.00851924433862071\\
230.01	0.00851924917992318\\
231.01	0.00851925413436603\\
232.01	0.00851925920466273\\
233.01	0.00851926439359349\\
234.01	0.00851926970400708\\
235.01	0.00851927513882249\\
236.01	0.00851928070103067\\
237.01	0.00851928639369638\\
238.01	0.00851929221996008\\
239.01	0.00851929818303978\\
240.01	0.00851930428623303\\
241.01	0.00851931053291889\\
242.01	0.00851931692656006\\
243.01	0.00851932347070488\\
244.01	0.0085193301689896\\
245.01	0.00851933702514052\\
246.01	0.00851934404297633\\
247.01	0.00851935122641035\\
248.01	0.00851935857945302\\
249.01	0.00851936610621427\\
250.01	0.00851937381090609\\
251.01	0.00851938169784507\\
252.01	0.00851938977145506\\
253.01	0.00851939803626987\\
254.01	0.00851940649693605\\
255.01	0.00851941515821571\\
256.01	0.0085194240249895\\
257.01	0.00851943310225955\\
258.01	0.00851944239515253\\
259.01	0.00851945190892285\\
260.01	0.00851946164895584\\
261.01	0.00851947162077104\\
262.01	0.00851948183002565\\
263.01	0.00851949228251794\\
264.01	0.00851950298419083\\
265.01	0.0085195139411356\\
266.01	0.00851952515959549\\
267.01	0.00851953664596972\\
268.01	0.00851954840681722\\
269.01	0.00851956044886084\\
270.01	0.00851957277899136\\
271.01	0.0085195854042718\\
272.01	0.0085195983319417\\
273.01	0.00851961156942161\\
274.01	0.00851962512431759\\
275.01	0.00851963900442599\\
276.01	0.00851965321773815\\
277.01	0.00851966777244529\\
278.01	0.0085196826769436\\
279.01	0.00851969793983942\\
280.01	0.00851971356995443\\
281.01	0.00851972957633114\\
282.01	0.00851974596823842\\
283.01	0.00851976275517718\\
284.01	0.00851977994688619\\
285.01	0.00851979755334807\\
286.01	0.00851981558479545\\
287.01	0.00851983405171714\\
288.01	0.0085198529648647\\
289.01	0.00851987233525888\\
290.01	0.00851989217419651\\
291.01	0.00851991249325733\\
292.01	0.00851993330431114\\
293.01	0.008519954619525\\
294.01	0.00851997645137079\\
295.01	0.00851999881263271\\
296.01	0.00852002171641526\\
297.01	0.00852004517615114\\
298.01	0.00852006920560947\\
299.01	0.00852009381890434\\
300.01	0.00852011903050333\\
301.01	0.00852014485523636\\
302.01	0.00852017130830491\\
303.01	0.00852019840529111\\
304.01	0.00852022616216744\\
305.01	0.00852025459530644\\
306.01	0.00852028372149079\\
307.01	0.00852031355792348\\
308.01	0.00852034412223848\\
309.01	0.00852037543251143\\
310.01	0.00852040750727082\\
311.01	0.00852044036550928\\
312.01	0.00852047402669528\\
313.01	0.00852050851078508\\
314.01	0.00852054383823503\\
315.01	0.00852058003001403\\
316.01	0.00852061710761662\\
317.01	0.00852065509307611\\
318.01	0.0085206940089782\\
319.01	0.00852073387847495\\
320.01	0.00852077472529914\\
321.01	0.00852081657377892\\
322.01	0.00852085944885293\\
323.01	0.00852090337608586\\
324.01	0.00852094838168435\\
325.01	0.00852099449251327\\
326.01	0.00852104173611275\\
327.01	0.00852109014071521\\
328.01	0.00852113973526331\\
329.01	0.00852119054942814\\
330.01	0.00852124261362796\\
331.01	0.00852129595904753\\
332.01	0.00852135061765802\\
333.01	0.00852140662223739\\
334.01	0.00852146400639137\\
335.01	0.00852152280457522\\
336.01	0.00852158305211584\\
337.01	0.00852164478523488\\
338.01	0.00852170804107222\\
339.01	0.00852177285771041\\
340.01	0.00852183927419964\\
341.01	0.00852190733058369\\
342.01	0.00852197706792645\\
343.01	0.0085220485283395\\
344.01	0.00852212175501035\\
345.01	0.0085221967922317\\
346.01	0.0085222736854316\\
347.01	0.00852235248120448\\
348.01	0.00852243322734335\\
349.01	0.00852251597287295\\
350.01	0.0085226007680839\\
351.01	0.00852268766456818\\
352.01	0.00852277671525556\\
353.01	0.00852286797445135\\
354.01	0.00852296149787546\\
355.01	0.00852305734270264\\
356.01	0.00852315556760422\\
357.01	0.00852325623279104\\
358.01	0.00852335940005817\\
359.01	0.00852346513283078\\
360.01	0.00852357349621188\\
361.01	0.00852368455703154\\
362.01	0.00852379838389776\\
363.01	0.00852391504724926\\
364.01	0.00852403461940987\\
365.01	0.00852415717464492\\
366.01	0.00852428278921963\\
367.01	0.0085244115414592\\
368.01	0.00852454351181135\\
369.01	0.00852467878291075\\
370.01	0.00852481743964565\\
371.01	0.00852495956922692\\
372.01	0.00852510526125928\\
373.01	0.00852525460781502\\
374.01	0.00852540770351018\\
375.01	0.00852556464558327\\
376.01	0.00852572553397657\\
377.01	0.00852589047142027\\
378.01	0.00852605956351916\\
379.01	0.00852623291884242\\
380.01	0.0085264106490161\\
381.01	0.00852659286881875\\
382.01	0.00852677969628006\\
383.01	0.00852697125278272\\
384.01	0.00852716766316754\\
385.01	0.00852736905584179\\
386.01	0.00852757556289121\\
387.01	0.00852778732019534\\
388.01	0.00852800446754665\\
389.01	0.00852822714877336\\
390.01	0.00852845551186626\\
391.01	0.00852868970910944\\
392.01	0.00852892989721528\\
393.01	0.00852917623746391\\
394.01	0.00852942889584704\\
395.01	0.00852968804321676\\
396.01	0.00852995385543926\\
397.01	0.00853022651355389\\
398.01	0.00853050620393787\\
399.01	0.0085307931184771\\
400.01	0.00853108745474314\\
401.01	0.00853138941617745\\
402.01	0.00853169921228288\\
403.01	0.00853201705882334\\
404.01	0.0085323431780325\\
405.01	0.00853267779883209\\
406.01	0.00853302115706107\\
407.01	0.00853337349571685\\
408.01	0.00853373506520972\\
409.01	0.00853410612363215\\
410.01	0.00853448693704495\\
411.01	0.0085348777797816\\
412.01	0.00853527893477052\\
413.01	0.00853569069385219\\
414.01	0.00853611335785201\\
415.01	0.00853654723387496\\
416.01	0.00853699262679034\\
417.01	0.00853744990332334\\
418.01	0.0085379194266223\\
419.01	0.00853840161436998\\
420.01	0.00853889716461353\\
421.01	0.00853940663745721\\
422.01	0.00853992386244635\\
423.01	0.00854045345562234\\
424.01	0.00854099745077567\\
425.01	0.00854155627968325\\
426.01	0.00854213038955616\\
427.01	0.00854272024378953\\
428.01	0.00854332632276554\\
429.01	0.00854394912471401\\
430.01	0.00854458916663657\\
431.01	0.00854524698530027\\
432.01	0.00854592313830782\\
433.01	0.008546618205252\\
434.01	0.00854733278896304\\
435.01	0.00854806751685886\\
436.01	0.00854882304240884\\
437.01	0.00854960004672395\\
438.01	0.00855039924028687\\
439.01	0.00855122136483846\\
440.01	0.00855206719543803\\
441.01	0.00855293754271855\\
442.01	0.00855383325535988\\
443.01	0.00855475522280683\\
444.01	0.00855570437826286\\
445.01	0.008556681701995\\
446.01	0.00855768822499032\\
447.01	0.00855872503301097\\
448.01	0.00855979327109931\\
449.01	0.00856089414856185\\
450.01	0.00856202894417724\\
451.01	0.0085631990090945\\
452.01	0.00856440576781709\\
453.01	0.00856565078392652\\
454.01	0.00856693569839355\\
455.01	0.0085682622418768\\
456.01	0.00856963227505992\\
457.01	0.00857104786178624\\
458.01	0.00857251180814016\\
459.01	0.00857403225298102\\
460.01	0.00857564600632979\\
461.01	0.00857716839994526\\
462.01	0.00857866471785247\\
463.01	0.00858021372895401\\
464.01	0.00858181839781352\\
465.01	0.00858348198743255\\
466.01	0.0085852081030211\\
467.01	0.00858700073482906\\
468.01	0.00858886420859308\\
469.01	0.00859080228702808\\
470.01	0.00859281669008256\\
471.01	0.00859491732257235\\
472.01	0.00859711429223264\\
473.01	0.00859943173771808\\
474.01	0.00860215295954599\\
475.01	0.00860684526039444\\
476.01	0.00861187566236694\\
477.01	0.00861704203429633\\
478.01	0.00862234839691171\\
479.01	0.00862779890889316\\
480.01	0.00863339787294322\\
481.01	0.0086391497422098\\
482.01	0.00864505912707618\\
483.01	0.00865113080227834\\
484.01	0.00865736971376215\\
485.01	0.00866378097929141\\
486.01	0.00867036984112\\
487.01	0.00867714171497274\\
488.01	0.00868410266418304\\
489.01	0.00869125864186352\\
490.01	0.00869861575874439\\
491.01	0.00870618036815419\\
492.01	0.00871395912455498\\
493.01	0.00872195930390418\\
494.01	0.00873019064284751\\
495.01	0.00873867081320213\\
496.01	0.00874736519847467\\
497.01	0.00875619829089197\\
498.01	0.0087652904495427\\
499.01	0.00877465138881615\\
500.01	0.00878429083743185\\
501.01	0.00879421906688389\\
502.01	0.00880444703401725\\
503.01	0.00881498630465298\\
504.01	0.00882584852487436\\
505.01	0.00883704667524924\\
506.01	0.008848596387323\\
507.01	0.00886051230198366\\
508.01	0.00887280983688171\\
509.01	0.00888550535659654\\
510.01	0.00889861626069576\\
511.01	0.00891216108505397\\
512.01	0.00892615962815026\\
513.01	0.0089406331862638\\
514.01	0.00895560564261965\\
515.01	0.00897111254928596\\
516.01	0.00898728333367069\\
517.01	0.00900462258209694\\
518.01	0.00902005533817399\\
519.01	0.00903556989142077\\
520.01	0.00905179543968464\\
521.01	0.00906929777513815\\
522.01	0.00908690158381172\\
523.01	0.00910325478368778\\
524.01	0.00911996382008945\\
525.01	0.00913726484255246\\
526.01	0.00915519484973122\\
527.01	0.00917379853846171\\
528.01	0.00919312249530783\\
529.01	0.00921321828829849\\
530.01	0.00923414366959399\\
531.01	0.00925596373087513\\
532.01	0.00927875287109759\\
533.01	0.00930259734269146\\
534.01	0.0093275873085074\\
535.01	0.00935394603339217\\
536.01	0.00938619944115896\\
537.01	0.00944802730378025\\
538.01	0.00951341341243095\\
539.01	0.00958015524452214\\
540.01	0.00964844390480317\\
541.01	0.00971857849326479\\
542.01	0.00978779473566883\\
543.01	0.00985784525650115\\
544.01	0.00992932943037659\\
545.01	0.00999424628944576\\
546.01	0.01\\
547.01	0.01\\
548.01	0.01\\
549.01	0.01\\
550.01	0.01\\
551.01	0.01\\
552.01	0.01\\
553.01	0.01\\
554.01	0.01\\
555.01	0.01\\
556.01	0.01\\
557.01	0.01\\
558.01	0.01\\
559.01	0.01\\
560.01	0.01\\
561.01	0.01\\
562.01	0.01\\
563.01	0.01\\
564.01	0.01\\
565.01	0.01\\
566.01	0.01\\
567.01	0.01\\
568.01	0.01\\
569.01	0.01\\
570.01	0.01\\
571.01	0.01\\
572.01	0.01\\
573.01	0.01\\
574.01	0.01\\
575.01	0.01\\
576.01	0.01\\
577.01	0.01\\
578.01	0.01\\
579.01	0.01\\
580.01	0.01\\
581.01	0.01\\
582.01	0.01\\
583.01	0.01\\
584.01	0.01\\
585.01	0.01\\
586.01	0.01\\
587.01	0.01\\
588.01	0.01\\
589.01	0.01\\
590.01	0.01\\
591.01	0.01\\
592.01	0.01\\
593.01	0.01\\
594.01	0.01\\
595.01	0.01\\
596.01	0.01\\
597.01	0.01\\
598.01	0.01\\
599.01	0.01\\
599.02	0.01\\
599.03	0.01\\
599.04	0.01\\
599.05	0.01\\
599.06	0.01\\
599.07	0.01\\
599.08	0.01\\
599.09	0.01\\
599.1	0.01\\
599.11	0.01\\
599.12	0.01\\
599.13	0.01\\
599.14	0.01\\
599.15	0.01\\
599.16	0.01\\
599.17	0.01\\
599.18	0.01\\
599.19	0.01\\
599.2	0.01\\
599.21	0.01\\
599.22	0.01\\
599.23	0.01\\
599.24	0.01\\
599.25	0.01\\
599.26	0.01\\
599.27	0.01\\
599.28	0.01\\
599.29	0.01\\
599.3	0.01\\
599.31	0.01\\
599.32	0.01\\
599.33	0.01\\
599.34	0.01\\
599.35	0.01\\
599.36	0.01\\
599.37	0.01\\
599.38	0.01\\
599.39	0.01\\
599.4	0.01\\
599.41	0.01\\
599.42	0.01\\
599.43	0.01\\
599.44	0.01\\
599.45	0.01\\
599.46	0.01\\
599.47	0.01\\
599.48	0.01\\
599.49	0.01\\
599.5	0.01\\
599.51	0.01\\
599.52	0.01\\
599.53	0.01\\
599.54	0.01\\
599.55	0.01\\
599.56	0.01\\
599.57	0.01\\
599.58	0.01\\
599.59	0.01\\
599.6	0.01\\
599.61	0.01\\
599.62	0.01\\
599.63	0.01\\
599.64	0.01\\
599.65	0.01\\
599.66	0.01\\
599.67	0.01\\
599.68	0.01\\
599.69	0.01\\
599.7	0.01\\
599.71	0.01\\
599.72	0.01\\
599.73	0.01\\
599.74	0.01\\
599.75	0.01\\
599.76	0.01\\
599.77	0.01\\
599.78	0.01\\
599.79	0.01\\
599.8	0.01\\
599.81	0.01\\
599.82	0.01\\
599.83	0.01\\
599.84	0.01\\
599.85	0.01\\
599.86	0.01\\
599.87	0.01\\
599.88	0.01\\
599.89	0.01\\
599.9	0.01\\
599.91	0.01\\
599.92	0.01\\
599.93	0.01\\
599.94	0.01\\
599.95	0.01\\
599.96	0.01\\
599.97	0.01\\
599.98	0.01\\
599.99	0.01\\
600	0.01\\
};
\addplot [color=mycolor18,solid,forget plot]
  table[row sep=crcr]{%
0.01	0.00921890721345464\\
1.01	0.00921890730374509\\
2.01	0.00921890739597887\\
3.01	0.00921890749019809\\
4.01	0.00921890758644576\\
5.01	0.00921890768476586\\
6.01	0.00921890778520332\\
7.01	0.00921890788780404\\
8.01	0.00921890799261493\\
9.01	0.00921890809968396\\
10.01	0.00921890820906009\\
11.01	0.00921890832079337\\
12.01	0.00921890843493498\\
13.01	0.0092189085515372\\
14.01	0.00921890867065344\\
15.01	0.00921890879233829\\
16.01	0.00921890891664755\\
17.01	0.00921890904363823\\
18.01	0.00921890917336859\\
19.01	0.00921890930589814\\
20.01	0.00921890944128779\\
21.01	0.00921890957959967\\
22.01	0.00921890972089736\\
23.01	0.00921890986524575\\
24.01	0.00921891001271128\\
25.01	0.00921891016336171\\
26.01	0.00921891031726639\\
27.01	0.00921891047449615\\
28.01	0.00921891063512338\\
29.01	0.00921891079922208\\
30.01	0.00921891096686785\\
31.01	0.00921891113813799\\
32.01	0.00921891131311146\\
33.01	0.009218911491869\\
34.01	0.00921891167449308\\
35.01	0.00921891186106804\\
36.01	0.00921891205168002\\
37.01	0.00921891224641713\\
38.01	0.00921891244536936\\
39.01	0.0092189126486287\\
40.01	0.0092189128562892\\
41.01	0.00921891306844694\\
42.01	0.00921891328520016\\
43.01	0.00921891350664924\\
44.01	0.00921891373289679\\
45.01	0.00921891396404769\\
46.01	0.00921891420020914\\
47.01	0.0092189144414907\\
48.01	0.00921891468800435\\
49.01	0.00921891493986457\\
50.01	0.00921891519718833\\
51.01	0.00921891546009524\\
52.01	0.00921891572870752\\
53.01	0.00921891600315015\\
54.01	0.00921891628355082\\
55.01	0.00921891657004004\\
56.01	0.00921891686275127\\
57.01	0.00921891716182092\\
58.01	0.00921891746738841\\
59.01	0.00921891777959622\\
60.01	0.00921891809859007\\
61.01	0.00921891842451887\\
62.01	0.00921891875753484\\
63.01	0.00921891909779355\\
64.01	0.00921891944545415\\
65.01	0.0092189198006792\\
66.01	0.0092189201636349\\
67.01	0.0092189205344912\\
68.01	0.00921892091342182\\
69.01	0.00921892130060432\\
70.01	0.00921892169622022\\
71.01	0.00921892210045515\\
72.01	0.00921892251349874\\
73.01	0.00921892293554501\\
74.01	0.00921892336679221\\
75.01	0.00921892380744307\\
76.01	0.00921892425770478\\
77.01	0.00921892471778922\\
78.01	0.00921892518791298\\
79.01	0.0092189256682975\\
80.01	0.00921892615916919\\
81.01	0.00921892666075947\\
82.01	0.009218927173305\\
83.01	0.00921892769704771\\
84.01	0.00921892823223495\\
85.01	0.00921892877911963\\
86.01	0.00921892933796032\\
87.01	0.00921892990902138\\
88.01	0.00921893049257313\\
89.01	0.00921893108889191\\
90.01	0.00921893169826033\\
91.01	0.0092189323209673\\
92.01	0.00921893295730824\\
93.01	0.00921893360758522\\
94.01	0.00921893427210712\\
95.01	0.0092189349511897\\
96.01	0.00921893564515591\\
97.01	0.00921893635433596\\
98.01	0.00921893707906747\\
99.01	0.00921893781969569\\
100.01	0.00921893857657367\\
101.01	0.00921893935006242\\
102.01	0.00921894014053106\\
103.01	0.00921894094835713\\
104.01	0.00921894177392664\\
105.01	0.00921894261763431\\
106.01	0.00921894347988386\\
107.01	0.00921894436108809\\
108.01	0.00921894526166918\\
109.01	0.00921894618205884\\
110.01	0.00921894712269858\\
111.01	0.00921894808403991\\
112.01	0.00921894906654464\\
113.01	0.00921895007068496\\
114.01	0.00921895109694385\\
115.01	0.00921895214581526\\
116.01	0.00921895321780434\\
117.01	0.00921895431342775\\
118.01	0.0092189554332139\\
119.01	0.0092189565777032\\
120.01	0.00921895774744843\\
121.01	0.0092189589430149\\
122.01	0.00921896016498084\\
123.01	0.00921896141393768\\
124.01	0.00921896269049034\\
125.01	0.00921896399525753\\
126.01	0.00921896532887211\\
127.01	0.0092189666919814\\
128.01	0.00921896808524758\\
129.01	0.00921896950934787\\
130.01	0.00921897096497513\\
131.01	0.00921897245283797\\
132.01	0.00921897397366134\\
133.01	0.00921897552818671\\
134.01	0.00921897711717267\\
135.01	0.00921897874139517\\
136.01	0.00921898040164799\\
137.01	0.00921898209874318\\
138.01	0.0092189838335114\\
139.01	0.00921898560680249\\
140.01	0.00921898741948585\\
141.01	0.00921898927245085\\
142.01	0.00921899116660746\\
143.01	0.00921899310288653\\
144.01	0.00921899508224044\\
145.01	0.00921899710564359\\
146.01	0.00921899917409285\\
147.01	0.00921900128860814\\
148.01	0.00921900345023298\\
149.01	0.00921900566003505\\
150.01	0.00921900791910675\\
151.01	0.0092190102285658\\
152.01	0.00921901258955581\\
153.01	0.00921901500324697\\
154.01	0.00921901747083657\\
155.01	0.0092190199935498\\
156.01	0.00921902257264027\\
157.01	0.00921902520939073\\
158.01	0.00921902790511385\\
159.01	0.0092190306611528\\
160.01	0.0092190334788821\\
161.01	0.00921903635970831\\
162.01	0.00921903930507079\\
163.01	0.00921904231644249\\
164.01	0.00921904539533084\\
165.01	0.00921904854327842\\
166.01	0.00921905176186396\\
167.01	0.00921905505270312\\
168.01	0.00921905841744938\\
169.01	0.00921906185779502\\
170.01	0.00921906537547199\\
171.01	0.00921906897225286\\
172.01	0.00921907264995184\\
173.01	0.0092190764104258\\
174.01	0.00921908025557525\\
175.01	0.00921908418734543\\
176.01	0.00921908820772733\\
177.01	0.00921909231875895\\
178.01	0.00921909652252625\\
179.01	0.00921910082116448\\
180.01	0.00921910521685924\\
181.01	0.00921910971184779\\
182.01	0.0092191143084203\\
183.01	0.00921911900892108\\
184.01	0.00921912381574998\\
185.01	0.0092191287313637\\
186.01	0.00921913375827713\\
187.01	0.00921913889906488\\
188.01	0.00921914415636263\\
189.01	0.00921914953286872\\
190.01	0.00921915503134557\\
191.01	0.00921916065462139\\
192.01	0.00921916640559164\\
193.01	0.00921917228722081\\
194.01	0.00921917830254402\\
195.01	0.00921918445466883\\
196.01	0.00921919074677693\\
197.01	0.00921919718212605\\
198.01	0.00921920376405177\\
199.01	0.00921921049596948\\
200.01	0.00921921738137632\\
201.01	0.0092192244238532\\
202.01	0.00921923162706686\\
203.01	0.00921923899477202\\
204.01	0.00921924653081352\\
205.01	0.00921925423912855\\
206.01	0.00921926212374895\\
207.01	0.00921927018880356\\
208.01	0.0092192784385206\\
209.01	0.00921928687723014\\
210.01	0.00921929550936663\\
211.01	0.0092193043394715\\
212.01	0.00921931337219581\\
213.01	0.00921932261230295\\
214.01	0.00921933206467149\\
215.01	0.00921934173429797\\
216.01	0.00921935162629992\\
217.01	0.00921936174591882\\
218.01	0.00921937209852319\\
219.01	0.0092193826896118\\
220.01	0.00921939352481688\\
221.01	0.00921940460990744\\
222.01	0.0092194159507928\\
223.01	0.00921942755352593\\
224.01	0.00921943942430716\\
225.01	0.00921945156948784\\
226.01	0.00921946399557413\\
227.01	0.00921947670923086\\
228.01	0.0092194897172855\\
229.01	0.00921950302673227\\
230.01	0.00921951664473636\\
231.01	0.00921953057863811\\
232.01	0.0092195448359575\\
233.01	0.00921955942439871\\
234.01	0.00921957435185456\\
235.01	0.00921958962641151\\
236.01	0.00921960525635432\\
237.01	0.00921962125017119\\
238.01	0.0092196376165588\\
239.01	0.00921965436442761\\
240.01	0.00921967150290721\\
241.01	0.00921968904135194\\
242.01	0.00921970698934639\\
243.01	0.00921972535671141\\
244.01	0.00921974415350995\\
245.01	0.00921976339005324\\
246.01	0.00921978307690698\\
247.01	0.0092198032248979\\
248.01	0.00921982384512024\\
249.01	0.00921984494894264\\
250.01	0.00921986654801494\\
251.01	0.00921988865427544\\
252.01	0.00921991127995808\\
253.01	0.00921993443760005\\
254.01	0.00921995814004934\\
255.01	0.00921998240047273\\
256.01	0.00922000723236382\\
257.01	0.00922003264955129\\
258.01	0.00922005866620747\\
259.01	0.00922008529685696\\
260.01	0.00922011255638563\\
261.01	0.00922014046004974\\
262.01	0.00922016902348541\\
263.01	0.00922019826271813\\
264.01	0.0092202281941728\\
265.01	0.00922025883468371\\
266.01	0.00922029020150506\\
267.01	0.00922032231232145\\
268.01	0.00922035518525899\\
269.01	0.0092203888388964\\
270.01	0.00922042329227649\\
271.01	0.00922045856491797\\
272.01	0.00922049467682749\\
273.01	0.00922053164851207\\
274.01	0.00922056950099175\\
275.01	0.00922060825581258\\
276.01	0.00922064793506005\\
277.01	0.00922068856137268\\
278.01	0.00922073015795612\\
279.01	0.0092207727485974\\
280.01	0.00922081635767986\\
281.01	0.00922086101019805\\
282.01	0.00922090673177337\\
283.01	0.00922095354866987\\
284.01	0.00922100148781057\\
285.01	0.0092210505767941\\
286.01	0.00922110084391179\\
287.01	0.00922115231816529\\
288.01	0.00922120502928439\\
289.01	0.00922125900774555\\
290.01	0.00922131428479074\\
291.01	0.00922137089244673\\
292.01	0.009221428863545\\
293.01	0.009221488231742\\
294.01	0.00922154903153996\\
295.01	0.00922161129830828\\
296.01	0.00922167506830539\\
297.01	0.00922174037870109\\
298.01	0.00922180726759964\\
299.01	0.00922187577406314\\
300.01	0.00922194593813585\\
301.01	0.00922201780086869\\
302.01	0.00922209140434469\\
303.01	0.00922216679170498\\
304.01	0.00922224400717528\\
305.01	0.00922232309609317\\
306.01	0.00922240410493612\\
307.01	0.00922248708134997\\
308.01	0.00922257207417836\\
309.01	0.00922265913349269\\
310.01	0.00922274831062301\\
311.01	0.00922283965818949\\
312.01	0.00922293323013474\\
313.01	0.00922302908175709\\
314.01	0.00922312726974437\\
315.01	0.0092232278522088\\
316.01	0.00922333088872266\\
317.01	0.00922343644035481\\
318.01	0.00922354456970816\\
319.01	0.0092236553409581\\
320.01	0.00922376881989177\\
321.01	0.00922388507394854\\
322.01	0.00922400417226123\\
323.01	0.0092241261856986\\
324.01	0.00922425118690877\\
325.01	0.00922437925036381\\
326.01	0.00922451045240545\\
327.01	0.00922464487129194\\
328.01	0.00922478258724609\\
329.01	0.00922492368250462\\
330.01	0.00922506824136872\\
331.01	0.00922521635025588\\
332.01	0.00922536809775322\\
333.01	0.00922552357467207\\
334.01	0.00922568287410402\\
335.01	0.00922584609147853\\
336.01	0.00922601332462195\\
337.01	0.00922618467381824\\
338.01	0.00922636024187123\\
339.01	0.00922654013416866\\
340.01	0.00922672445874789\\
341.01	0.00922691332636346\\
342.01	0.00922710685055648\\
343.01	0.00922730514772609\\
344.01	0.00922750833720266\\
345.01	0.00922771654132343\\
346.01	0.00922792988550996\\
347.01	0.00922814849834811\\
348.01	0.00922837251167\\
349.01	0.00922860206063883\\
350.01	0.00922883728383571\\
351.01	0.00922907832334947\\
352.01	0.00922932532486894\\
353.01	0.00922957843777813\\
354.01	0.00922983781525418\\
355.01	0.00923010361436858\\
356.01	0.00923037599619122\\
357.01	0.00923065512589792\\
358.01	0.00923094117288126\\
359.01	0.00923123431086496\\
360.01	0.00923153471802176\\
361.01	0.00923184257709542\\
362.01	0.00923215807552635\\
363.01	0.0092324814055815\\
364.01	0.0092328127644885\\
365.01	0.00923315235457411\\
366.01	0.00923350038340743\\
367.01	0.00923385706394779\\
368.01	0.00923422261469753\\
369.01	0.00923459725986001\\
370.01	0.00923498122950288\\
371.01	0.00923537475972682\\
372.01	0.00923577809284016\\
373.01	0.00923619147753918\\
374.01	0.00923661516909463\\
375.01	0.00923704942954462\\
376.01	0.00923749452789394\\
377.01	0.00923795074032027\\
378.01	0.00923841835038712\\
379.01	0.00923889764926422\\
380.01	0.00923938893595511\\
381.01	0.00923989251753244\\
382.01	0.00924040870938114\\
383.01	0.0092409378354496\\
384.01	0.00924148022850912\\
385.01	0.00924203623042203\\
386.01	0.00924260619241836\\
387.01	0.00924319047538165\\
388.01	0.00924378945014376\\
389.01	0.00924440349778941\\
390.01	0.00924503300997007\\
391.01	0.00924567838922794\\
392.01	0.00924634004933008\\
393.01	0.0092470184156128\\
394.01	0.00924771392533688\\
395.01	0.00924842702805363\\
396.01	0.00924915818598221\\
397.01	0.00924990787439864\\
398.01	0.00925067658203652\\
399.01	0.00925146481150018\\
400.01	0.00925227307969074\\
401.01	0.00925310191824505\\
402.01	0.00925395187398881\\
403.01	0.00925482350940373\\
404.01	0.00925571740311017\\
405.01	0.00925663415036531\\
406.01	0.00925757436357807\\
407.01	0.00925853867284173\\
408.01	0.00925952772648465\\
409.01	0.00926054219164027\\
410.01	0.00926158275483667\\
411.01	0.00926265012260504\\
412.01	0.00926374502210001\\
413.01	0.00926486820166126\\
414.01	0.00926602043076913\\
415.01	0.00926720249699666\\
416.01	0.00926841521199671\\
417.01	0.00926965943905911\\
418.01	0.0092709360566141\\
419.01	0.00927224602791092\\
420.01	0.00927359050517867\\
421.01	0.00927496955357544\\
422.01	0.00927638087076898\\
423.01	0.00927782874595184\\
424.01	0.00927931460337114\\
425.01	0.00928083949176558\\
426.01	0.00928240449039553\\
427.01	0.00928401071001225\\
428.01	0.00928565929386367\\
429.01	0.00928735141873935\\
430.01	0.00928908829605571\\
431.01	0.00929087117298365\\
432.01	0.00929270133362052\\
433.01	0.00929458010020767\\
434.01	0.0092965088343959\\
435.01	0.00929848893855982\\
436.01	0.00930052185716275\\
437.01	0.00930260907817301\\
438.01	0.00930475213453213\\
439.01	0.00930695260567474\\
440.01	0.00930921211909932\\
441.01	0.00931153235198747\\
442.01	0.00931391503286795\\
443.01	0.0093163619433199\\
444.01	0.00931887491970683\\
445.01	0.00932145585492944\\
446.01	0.00932410670018146\\
447.01	0.0093268294666857\\
448.01	0.00932962622737258\\
449.01	0.00933249911836562\\
450.01	0.00933545033908872\\
451.01	0.00933848213706191\\
452.01	0.00934159667112291\\
453.01	0.00934479659703667\\
454.01	0.00934808476999769\\
455.01	0.00935146371890439\\
456.01	0.0093549360494896\\
457.01	0.00935850454685896\\
458.01	0.00936217300499512\\
459.01	0.0093659511498217\\
460.01	0.00936984638181615\\
461.01	0.00937373246944184\\
462.01	0.00937770933007804\\
463.01	0.00938180059729186\\
464.01	0.00938601016855558\\
465.01	0.00939034213804762\\
466.01	0.00939480081308246\\
467.01	0.00939939069515051\\
468.01	0.00940411586563104\\
469.01	0.00940897021722652\\
470.01	0.00941386170534561\\
471.01	0.00941882801766466\\
472.01	0.00942395102325723\\
473.01	0.00942926726025694\\
474.01	0.00943497500895296\\
475.01	0.00944004627798229\\
476.01	0.00944508056108339\\
477.01	0.0094502528079978\\
478.01	0.00945556729011318\\
479.01	0.00946102843227558\\
480.01	0.00946664081958607\\
481.01	0.00947240920462607\\
482.01	0.00947833851514803\\
483.01	0.00948443386215957\\
484.01	0.00949070054692894\\
485.01	0.00949714404642335\\
486.01	0.00950376969226967\\
487.01	0.00951058130173159\\
488.01	0.00951758896062574\\
489.01	0.00952480075565114\\
490.01	0.00953222337148154\\
491.01	0.00953986370343647\\
492.01	0.00954772890825923\\
493.01	0.00955582676231791\\
494.01	0.00956416733201233\\
495.01	0.00957276200482003\\
496.01	0.00958156826414436\\
497.01	0.00959058878391121\\
498.01	0.00959987903732562\\
499.01	0.00960944673872136\\
500.01	0.00961929935139057\\
501.01	0.00962944415214239\\
502.01	0.00963989047896042\\
503.01	0.00965065096520565\\
504.01	0.00966173522705664\\
505.01	0.00967314736790401\\
506.01	0.00968493137212471\\
507.01	0.00969711067637659\\
508.01	0.00970970714788098\\
509.01	0.00972274467702539\\
510.01	0.00973624944209958\\
511.01	0.009750250218431\\
512.01	0.00976477875412207\\
513.01	0.00977987034961125\\
514.01	0.00979556583031401\\
515.01	0.0098119252359359\\
516.01	0.0098291106000916\\
517.01	0.0098470939149584\\
518.01	0.00986437059233572\\
519.01	0.00988253806599425\\
520.01	0.00990313228910655\\
521.01	0.00994326831934782\\
522.01	0.00998766257926984\\
523.01	0.01\\
524.01	0.01\\
525.01	0.01\\
526.01	0.01\\
527.01	0.01\\
528.01	0.01\\
529.01	0.01\\
530.01	0.01\\
531.01	0.01\\
532.01	0.01\\
533.01	0.01\\
534.01	0.01\\
535.01	0.01\\
536.01	0.01\\
537.01	0.01\\
538.01	0.01\\
539.01	0.01\\
540.01	0.01\\
541.01	0.01\\
542.01	0.01\\
543.01	0.01\\
544.01	0.01\\
545.01	0.01\\
546.01	0.01\\
547.01	0.01\\
548.01	0.01\\
549.01	0.01\\
550.01	0.01\\
551.01	0.01\\
552.01	0.01\\
553.01	0.01\\
554.01	0.01\\
555.01	0.01\\
556.01	0.01\\
557.01	0.01\\
558.01	0.01\\
559.01	0.01\\
560.01	0.01\\
561.01	0.01\\
562.01	0.01\\
563.01	0.01\\
564.01	0.01\\
565.01	0.01\\
566.01	0.01\\
567.01	0.01\\
568.01	0.01\\
569.01	0.01\\
570.01	0.01\\
571.01	0.01\\
572.01	0.01\\
573.01	0.01\\
574.01	0.01\\
575.01	0.01\\
576.01	0.01\\
577.01	0.01\\
578.01	0.01\\
579.01	0.01\\
580.01	0.01\\
581.01	0.01\\
582.01	0.01\\
583.01	0.01\\
584.01	0.01\\
585.01	0.01\\
586.01	0.01\\
587.01	0.01\\
588.01	0.01\\
589.01	0.01\\
590.01	0.01\\
591.01	0.01\\
592.01	0.01\\
593.01	0.01\\
594.01	0.01\\
595.01	0.01\\
596.01	0.01\\
597.01	0.01\\
598.01	0.01\\
599.01	0.01\\
599.02	0.01\\
599.03	0.01\\
599.04	0.01\\
599.05	0.01\\
599.06	0.01\\
599.07	0.01\\
599.08	0.01\\
599.09	0.01\\
599.1	0.01\\
599.11	0.01\\
599.12	0.01\\
599.13	0.01\\
599.14	0.01\\
599.15	0.01\\
599.16	0.01\\
599.17	0.01\\
599.18	0.01\\
599.19	0.01\\
599.2	0.01\\
599.21	0.01\\
599.22	0.01\\
599.23	0.01\\
599.24	0.01\\
599.25	0.01\\
599.26	0.01\\
599.27	0.01\\
599.28	0.01\\
599.29	0.01\\
599.3	0.01\\
599.31	0.01\\
599.32	0.01\\
599.33	0.01\\
599.34	0.01\\
599.35	0.01\\
599.36	0.01\\
599.37	0.01\\
599.38	0.01\\
599.39	0.01\\
599.4	0.01\\
599.41	0.01\\
599.42	0.01\\
599.43	0.01\\
599.44	0.01\\
599.45	0.01\\
599.46	0.01\\
599.47	0.01\\
599.48	0.01\\
599.49	0.01\\
599.5	0.01\\
599.51	0.01\\
599.52	0.01\\
599.53	0.01\\
599.54	0.01\\
599.55	0.01\\
599.56	0.01\\
599.57	0.01\\
599.58	0.01\\
599.59	0.01\\
599.6	0.01\\
599.61	0.01\\
599.62	0.01\\
599.63	0.01\\
599.64	0.01\\
599.65	0.01\\
599.66	0.01\\
599.67	0.01\\
599.68	0.01\\
599.69	0.01\\
599.7	0.01\\
599.71	0.01\\
599.72	0.01\\
599.73	0.01\\
599.74	0.01\\
599.75	0.01\\
599.76	0.01\\
599.77	0.01\\
599.78	0.01\\
599.79	0.01\\
599.8	0.01\\
599.81	0.01\\
599.82	0.01\\
599.83	0.01\\
599.84	0.01\\
599.85	0.01\\
599.86	0.01\\
599.87	0.01\\
599.88	0.01\\
599.89	0.01\\
599.9	0.01\\
599.91	0.01\\
599.92	0.01\\
599.93	0.01\\
599.94	0.01\\
599.95	0.01\\
599.96	0.01\\
599.97	0.01\\
599.98	0.01\\
599.99	0.01\\
600	0.01\\
};
\addplot [color=red!25!mycolor17,solid,forget plot]
  table[row sep=crcr]{%
0.01	0.00980203451190364\\
1.01	0.00980203461505372\\
2.01	0.00980203472041514\\
3.01	0.00980203482803558\\
4.01	0.00980203493796371\\
5.01	0.00980203505024933\\
6.01	0.00980203516494325\\
7.01	0.00980203528209744\\
8.01	0.00980203540176496\\
9.01	0.00980203552400002\\
10.01	0.00980203564885804\\
11.01	0.00980203577639563\\
12.01	0.00980203590667063\\
13.01	0.00980203603974211\\
14.01	0.00980203617567048\\
15.01	0.00980203631451742\\
16.01	0.00980203645634597\\
17.01	0.00980203660122053\\
18.01	0.0098020367492069\\
19.01	0.00980203690037236\\
20.01	0.00980203705478558\\
21.01	0.00980203721251679\\
22.01	0.00980203737363769\\
23.01	0.00980203753822158\\
24.01	0.00980203770634337\\
25.01	0.00980203787807956\\
26.01	0.00980203805350837\\
27.01	0.00980203823270968\\
28.01	0.00980203841576513\\
29.01	0.00980203860275812\\
30.01	0.00980203879377392\\
31.01	0.00980203898889962\\
32.01	0.00980203918822421\\
33.01	0.00980203939183867\\
34.01	0.00980203959983588\\
35.01	0.00980203981231083\\
36.01	0.00980204002936054\\
37.01	0.00980204025108421\\
38.01	0.0098020404775831\\
39.01	0.0098020407089608\\
40.01	0.00980204094532308\\
41.01	0.00980204118677806\\
42.01	0.00980204143343625\\
43.01	0.00980204168541053\\
44.01	0.00980204194281626\\
45.01	0.00980204220577137\\
46.01	0.00980204247439631\\
47.01	0.00980204274881421\\
48.01	0.00980204302915088\\
49.01	0.00980204331553492\\
50.01	0.00980204360809769\\
51.01	0.0098020439069735\\
52.01	0.00980204421229954\\
53.01	0.00980204452421604\\
54.01	0.00980204484286633\\
55.01	0.00980204516839683\\
56.01	0.00980204550095724\\
57.01	0.0098020458407005\\
58.01	0.00980204618778292\\
59.01	0.00980204654236427\\
60.01	0.00980204690460779\\
61.01	0.00980204727468032\\
62.01	0.00980204765275242\\
63.01	0.00980204803899836\\
64.01	0.00980204843359621\\
65.01	0.00980204883672801\\
66.01	0.0098020492485798\\
67.01	0.00980204966934171\\
68.01	0.00980205009920807\\
69.01	0.00980205053837746\\
70.01	0.00980205098705287\\
71.01	0.00980205144544175\\
72.01	0.00980205191375614\\
73.01	0.00980205239221273\\
74.01	0.00980205288103301\\
75.01	0.00980205338044336\\
76.01	0.00980205389067516\\
77.01	0.00980205441196491\\
78.01	0.00980205494455431\\
79.01	0.00980205548869043\\
80.01	0.00980205604462574\\
81.01	0.00980205661261838\\
82.01	0.00980205719293218\\
83.01	0.00980205778583676\\
84.01	0.00980205839160772\\
85.01	0.00980205901052681\\
86.01	0.009802059642882\\
87.01	0.0098020602889676\\
88.01	0.0098020609490845\\
89.01	0.00980206162354023\\
90.01	0.00980206231264915\\
91.01	0.00980206301673259\\
92.01	0.00980206373611903\\
93.01	0.0098020644711442\\
94.01	0.00980206522215133\\
95.01	0.00980206598949126\\
96.01	0.00980206677352265\\
97.01	0.00980206757461208\\
98.01	0.00980206839313434\\
99.01	0.00980206922947251\\
100.01	0.00980207008401826\\
101.01	0.00980207095717189\\
102.01	0.00980207184934274\\
103.01	0.00980207276094914\\
104.01	0.00980207369241881\\
105.01	0.00980207464418905\\
106.01	0.0098020756167068\\
107.01	0.00980207661042909\\
108.01	0.00980207762582308\\
109.01	0.00980207866336639\\
110.01	0.00980207972354731\\
111.01	0.00980208080686505\\
112.01	0.00980208191382994\\
113.01	0.00980208304496378\\
114.01	0.00980208420080002\\
115.01	0.00980208538188406\\
116.01	0.00980208658877349\\
117.01	0.00980208782203841\\
118.01	0.00980208908226172\\
119.01	0.00980209037003934\\
120.01	0.00980209168598059\\
121.01	0.00980209303070846\\
122.01	0.00980209440485993\\
123.01	0.00980209580908628\\
124.01	0.0098020972440534\\
125.01	0.00980209871044219\\
126.01	0.00980210020894885\\
127.01	0.00980210174028523\\
128.01	0.0098021033051792\\
129.01	0.00980210490437506\\
130.01	0.00980210653863381\\
131.01	0.00980210820873368\\
132.01	0.00980210991547038\\
133.01	0.00980211165965762\\
134.01	0.00980211344212744\\
135.01	0.00980211526373069\\
136.01	0.00980211712533743\\
137.01	0.00980211902783738\\
138.01	0.00980212097214039\\
139.01	0.0098021229591769\\
140.01	0.00980212498989838\\
141.01	0.00980212706527788\\
142.01	0.00980212918631045\\
143.01	0.00980213135401375\\
144.01	0.0098021335694285\\
145.01	0.00980213583361897\\
146.01	0.00980213814767369\\
147.01	0.00980214051270584\\
148.01	0.00980214292985395\\
149.01	0.00980214540028236\\
150.01	0.00980214792518196\\
151.01	0.00980215050577073\\
152.01	0.00980215314329436\\
153.01	0.00980215583902694\\
154.01	0.00980215859427159\\
155.01	0.00980216141036116\\
156.01	0.0098021642886589\\
157.01	0.00980216723055921\\
158.01	0.00980217023748829\\
159.01	0.00980217331090501\\
160.01	0.00980217645230155\\
161.01	0.00980217966320422\\
162.01	0.00980218294517434\\
163.01	0.00980218629980894\\
164.01	0.00980218972874166\\
165.01	0.00980219323364362\\
166.01	0.00980219681622425\\
167.01	0.00980220047823228\\
168.01	0.00980220422145658\\
169.01	0.00980220804772713\\
170.01	0.00980221195891603\\
171.01	0.00980221595693844\\
172.01	0.00980222004375361\\
173.01	0.00980222422136595\\
174.01	0.00980222849182608\\
175.01	0.0098022328572319\\
176.01	0.0098022373197298\\
177.01	0.00980224188151567\\
178.01	0.00980224654483618\\
179.01	0.00980225131198997\\
180.01	0.00980225618532883\\
181.01	0.00980226116725906\\
182.01	0.00980226626024265\\
183.01	0.00980227146679872\\
184.01	0.0098022767895048\\
185.01	0.00980228223099827\\
186.01	0.00980228779397779\\
187.01	0.00980229348120469\\
188.01	0.0098022992955046\\
189.01	0.00980230523976888\\
190.01	0.00980231131695622\\
191.01	0.00980231753009427\\
192.01	0.00980232388228131\\
193.01	0.00980233037668795\\
194.01	0.00980233701655877\\
195.01	0.00980234380521423\\
196.01	0.00980235074605243\\
197.01	0.00980235784255098\\
198.01	0.00980236509826898\\
199.01	0.00980237251684886\\
200.01	0.00980238010201851\\
201.01	0.0098023878575933\\
202.01	0.00980239578747825\\
203.01	0.00980240389567004\\
204.01	0.00980241218625947\\
205.01	0.00980242066343355\\
206.01	0.00980242933147791\\
207.01	0.00980243819477919\\
208.01	0.00980244725782755\\
209.01	0.00980245652521907\\
210.01	0.00980246600165843\\
211.01	0.00980247569196153\\
212.01	0.00980248560105821\\
213.01	0.00980249573399499\\
214.01	0.00980250609593801\\
215.01	0.00980251669217586\\
216.01	0.00980252752812266\\
217.01	0.00980253860932105\\
218.01	0.00980254994144547\\
219.01	0.00980256153030528\\
220.01	0.00980257338184813\\
221.01	0.00980258550216335\\
222.01	0.00980259789748542\\
223.01	0.00980261057419759\\
224.01	0.00980262353883554\\
225.01	0.00980263679809111\\
226.01	0.00980265035881616\\
227.01	0.00980266422802658\\
228.01	0.00980267841290636\\
229.01	0.0098026929208116\\
230.01	0.00980270775927498\\
231.01	0.00980272293601005\\
232.01	0.0098027384589157\\
233.01	0.0098027543360808\\
234.01	0.00980277057578893\\
235.01	0.00980278718652324\\
236.01	0.0098028041769714\\
237.01	0.00980282155603073\\
238.01	0.00980283933281337\\
239.01	0.00980285751665176\\
240.01	0.00980287611710407\\
241.01	0.00980289514395988\\
242.01	0.00980291460724596\\
243.01	0.00980293451723226\\
244.01	0.00980295488443798\\
245.01	0.00980297571963784\\
246.01	0.00980299703386851\\
247.01	0.00980301883843521\\
248.01	0.00980304114491846\\
249.01	0.00980306396518106\\
250.01	0.00980308731137515\\
251.01	0.00980311119594953\\
252.01	0.00980313563165728\\
253.01	0.00980316063156323\\
254.01	0.00980318620905208\\
255.01	0.00980321237783634\\
256.01	0.00980323915196476\\
257.01	0.00980326654583077\\
258.01	0.00980329457418129\\
259.01	0.00980332325212569\\
260.01	0.00980335259514508\\
261.01	0.00980338261910163\\
262.01	0.00980341334024839\\
263.01	0.00980344477523923\\
264.01	0.00980347694113912\\
265.01	0.00980350985543446\\
266.01	0.00980354353604404\\
267.01	0.00980357800132998\\
268.01	0.0098036132701091\\
269.01	0.00980364936166458\\
270.01	0.00980368629575788\\
271.01	0.00980372409264102\\
272.01	0.00980376277306918\\
273.01	0.00980380235831351\\
274.01	0.00980384287017456\\
275.01	0.00980388433099574\\
276.01	0.0098039267636773\\
277.01	0.00980397019169068\\
278.01	0.00980401463909328\\
279.01	0.00980406013054341\\
280.01	0.00980410669131584\\
281.01	0.00980415434731772\\
282.01	0.00980420312510484\\
283.01	0.00980425305189839\\
284.01	0.00980430415560211\\
285.01	0.00980435646481999\\
286.01	0.0098044100088742\\
287.01	0.00980446481782378\\
288.01	0.00980452092248366\\
289.01	0.00980457835444419\\
290.01	0.00980463714609115\\
291.01	0.00980469733062639\\
292.01	0.0098047589420889\\
293.01	0.00980482201537648\\
294.01	0.00980488658626797\\
295.01	0.00980495269144602\\
296.01	0.00980502036852058\\
297.01	0.00980508965605276\\
298.01	0.00980516059357952\\
299.01	0.00980523322163897\\
300.01	0.00980530758179619\\
301.01	0.00980538371666995\\
302.01	0.00980546166995975\\
303.01	0.00980554148647398\\
304.01	0.00980562321215851\\
305.01	0.00980570689412618\\
306.01	0.00980579258068682\\
307.01	0.00980588032137833\\
308.01	0.00980597016699834\\
309.01	0.00980606216963671\\
310.01	0.00980615638270898\\
311.01	0.00980625286099045\\
312.01	0.00980635166065133\\
313.01	0.00980645283929258\\
314.01	0.00980655645598292\\
315.01	0.00980666257129639\\
316.01	0.00980677124735119\\
317.01	0.00980688254784933\\
318.01	0.00980699653811726\\
319.01	0.00980711328514754\\
320.01	0.00980723285764154\\
321.01	0.00980735532605316\\
322.01	0.00980748076263367\\
323.01	0.00980760924147757\\
324.01	0.00980774083856965\\
325.01	0.00980787563183319\\
326.01	0.00980801370117916\\
327.01	0.00980815512855689\\
328.01	0.00980829999800574\\
329.01	0.00980844839570805\\
330.01	0.00980860041004337\\
331.01	0.00980875613164409\\
332.01	0.00980891565345213\\
333.01	0.00980907907077717\\
334.01	0.00980924648135629\\
335.01	0.00980941798541476\\
336.01	0.00980959368572856\\
337.01	0.00980977368768805\\
338.01	0.00980995809936338\\
339.01	0.00981014703157126\\
340.01	0.0098103405979433\\
341.01	0.00981053891499589\\
342.01	0.00981074210220188\\
343.01	0.00981095028206352\\
344.01	0.00981116358018744\\
345.01	0.00981138212536114\\
346.01	0.00981160604963131\\
347.01	0.00981183548838374\\
348.01	0.00981207058042535\\
349.01	0.00981231146806785\\
350.01	0.00981255829721333\\
351.01	0.00981281121744186\\
352.01	0.00981307038210104\\
353.01	0.00981333594839757\\
354.01	0.00981360807749095\\
355.01	0.00981388693458924\\
356.01	0.00981417268904712\\
357.01	0.0098144655144661\\
358.01	0.00981476558879709\\
359.01	0.00981507309444537\\
360.01	0.00981538821837796\\
361.01	0.00981571115223354\\
362.01	0.00981604209243492\\
363.01	0.00981638124030427\\
364.01	0.00981672880218104\\
365.01	0.00981708498954277\\
366.01	0.00981745001912881\\
367.01	0.00981782411306719\\
368.01	0.00981820749900443\\
369.01	0.00981860041023897\\
370.01	0.00981900308585763\\
371.01	0.00981941577087583\\
372.01	0.00981983871638133\\
373.01	0.00982027217968185\\
374.01	0.00982071642445649\\
375.01	0.00982117172091147\\
376.01	0.00982163834593977\\
377.01	0.00982211658328549\\
378.01	0.00982260672371261\\
379.01	0.00982310906517857\\
380.01	0.00982362391301274\\
381.01	0.0098241515801001\\
382.01	0.00982469238707007\\
383.01	0.00982524666249113\\
384.01	0.00982581474307093\\
385.01	0.00982639697386255\\
386.01	0.00982699370847676\\
387.01	0.00982760530930099\\
388.01	0.00982823214772481\\
389.01	0.00982887460437256\\
390.01	0.0098295330693433\\
391.01	0.00983020794245857\\
392.01	0.00983089963351855\\
393.01	0.0098316085625669\\
394.01	0.00983233516016546\\
395.01	0.00983307986767956\\
396.01	0.00983384313757525\\
397.01	0.00983462543373026\\
398.01	0.00983542723176098\\
399.01	0.00983624901936843\\
400.01	0.00983709129670747\\
401.01	0.0098379545767848\\
402.01	0.00983883938589321\\
403.01	0.00983974626409236\\
404.01	0.00984067576574953\\
405.01	0.00984162846015902\\
406.01	0.00984260493226483\\
407.01	0.00984360578352005\\
408.01	0.00984463163292778\\
409.01	0.00984568311832402\\
410.01	0.00984676089798293\\
411.01	0.00984786565265236\\
412.01	0.00984899808815074\\
413.01	0.00985015893850606\\
414.01	0.00985134896684873\\
415.01	0.00985256895867454\\
416.01	0.00985381972994977\\
417.01	0.0098551021324202\\
418.01	0.00985641704423509\\
419.01	0.00985776542681464\\
420.01	0.00985914827999014\\
421.01	0.00986056534351576\\
422.01	0.00986201655058395\\
423.01	0.00986350478886627\\
424.01	0.0098650311722287\\
425.01	0.00986659675550315\\
426.01	0.00986820262515298\\
427.01	0.00986984990029115\\
428.01	0.00987153973373128\\
429.01	0.00987327331307184\\
430.01	0.0098750518618138\\
431.01	0.00987687664051176\\
432.01	0.00987874894795836\\
433.01	0.00988067012240117\\
434.01	0.00988264154279109\\
435.01	0.00988466463006031\\
436.01	0.00988674084842737\\
437.01	0.0098888717067257\\
438.01	0.00989105875975091\\
439.01	0.00989330360962027\\
440.01	0.00989560790713621\\
441.01	0.0098979733531429\\
442.01	0.00990040169986228\\
443.01	0.00990289475219151\\
444.01	0.00990545436893974\\
445.01	0.00990808246397572\\
446.01	0.0099107810072504\\
447.01	0.00991355202564792\\
448.01	0.00991639760359253\\
449.01	0.00991931988315817\\
450.01	0.0099223210615491\\
451.01	0.00992540336719763\\
452.01	0.00992856901146673\\
453.01	0.00993182063564788\\
454.01	0.00993516080375945\\
455.01	0.00993859197332626\\
456.01	0.00994211666275934\\
457.01	0.00994573759471677\\
458.01	0.00994945870445162\\
459.01	0.00995328829824902\\
460.01	0.00995721174437341\\
461.01	0.00996117126676836\\
462.01	0.00996523561568764\\
463.01	0.00996941206370552\\
464.01	0.00997370348413337\\
465.01	0.00997811270845559\\
466.01	0.00998264248507081\\
467.01	0.00998729532035657\\
468.01	0.0099920711809774\\
469.01	0.00999691239801213\\
470.01	0.01\\
471.01	0.01\\
472.01	0.01\\
473.01	0.01\\
474.01	0.01\\
475.01	0.01\\
476.01	0.01\\
477.01	0.01\\
478.01	0.01\\
479.01	0.01\\
480.01	0.01\\
481.01	0.01\\
482.01	0.01\\
483.01	0.01\\
484.01	0.01\\
485.01	0.01\\
486.01	0.01\\
487.01	0.01\\
488.01	0.01\\
489.01	0.01\\
490.01	0.01\\
491.01	0.01\\
492.01	0.01\\
493.01	0.01\\
494.01	0.01\\
495.01	0.01\\
496.01	0.01\\
497.01	0.01\\
498.01	0.01\\
499.01	0.01\\
500.01	0.01\\
501.01	0.01\\
502.01	0.01\\
503.01	0.01\\
504.01	0.01\\
505.01	0.01\\
506.01	0.01\\
507.01	0.01\\
508.01	0.01\\
509.01	0.01\\
510.01	0.01\\
511.01	0.01\\
512.01	0.01\\
513.01	0.01\\
514.01	0.01\\
515.01	0.01\\
516.01	0.01\\
517.01	0.01\\
518.01	0.01\\
519.01	0.01\\
520.01	0.01\\
521.01	0.01\\
522.01	0.01\\
523.01	0.01\\
524.01	0.01\\
525.01	0.01\\
526.01	0.01\\
527.01	0.01\\
528.01	0.01\\
529.01	0.01\\
530.01	0.01\\
531.01	0.01\\
532.01	0.01\\
533.01	0.01\\
534.01	0.01\\
535.01	0.01\\
536.01	0.01\\
537.01	0.01\\
538.01	0.01\\
539.01	0.01\\
540.01	0.01\\
541.01	0.01\\
542.01	0.01\\
543.01	0.01\\
544.01	0.01\\
545.01	0.01\\
546.01	0.01\\
547.01	0.01\\
548.01	0.01\\
549.01	0.01\\
550.01	0.01\\
551.01	0.01\\
552.01	0.01\\
553.01	0.01\\
554.01	0.01\\
555.01	0.01\\
556.01	0.01\\
557.01	0.01\\
558.01	0.01\\
559.01	0.01\\
560.01	0.01\\
561.01	0.01\\
562.01	0.01\\
563.01	0.01\\
564.01	0.01\\
565.01	0.01\\
566.01	0.01\\
567.01	0.01\\
568.01	0.01\\
569.01	0.01\\
570.01	0.01\\
571.01	0.01\\
572.01	0.01\\
573.01	0.01\\
574.01	0.01\\
575.01	0.01\\
576.01	0.01\\
577.01	0.01\\
578.01	0.01\\
579.01	0.01\\
580.01	0.01\\
581.01	0.01\\
582.01	0.01\\
583.01	0.01\\
584.01	0.01\\
585.01	0.01\\
586.01	0.01\\
587.01	0.01\\
588.01	0.01\\
589.01	0.01\\
590.01	0.01\\
591.01	0.01\\
592.01	0.01\\
593.01	0.01\\
594.01	0.01\\
595.01	0.01\\
596.01	0.01\\
597.01	0.01\\
598.01	0.01\\
599.01	0.01\\
599.02	0.01\\
599.03	0.01\\
599.04	0.01\\
599.05	0.01\\
599.06	0.01\\
599.07	0.01\\
599.08	0.01\\
599.09	0.01\\
599.1	0.01\\
599.11	0.01\\
599.12	0.01\\
599.13	0.01\\
599.14	0.01\\
599.15	0.01\\
599.16	0.01\\
599.17	0.01\\
599.18	0.01\\
599.19	0.01\\
599.2	0.01\\
599.21	0.01\\
599.22	0.01\\
599.23	0.01\\
599.24	0.01\\
599.25	0.01\\
599.26	0.01\\
599.27	0.01\\
599.28	0.01\\
599.29	0.01\\
599.3	0.01\\
599.31	0.01\\
599.32	0.01\\
599.33	0.01\\
599.34	0.01\\
599.35	0.01\\
599.36	0.01\\
599.37	0.01\\
599.38	0.01\\
599.39	0.01\\
599.4	0.01\\
599.41	0.01\\
599.42	0.01\\
599.43	0.01\\
599.44	0.01\\
599.45	0.01\\
599.46	0.01\\
599.47	0.01\\
599.48	0.01\\
599.49	0.01\\
599.5	0.01\\
599.51	0.01\\
599.52	0.01\\
599.53	0.01\\
599.54	0.01\\
599.55	0.01\\
599.56	0.01\\
599.57	0.01\\
599.58	0.01\\
599.59	0.01\\
599.6	0.01\\
599.61	0.01\\
599.62	0.01\\
599.63	0.01\\
599.64	0.01\\
599.65	0.01\\
599.66	0.01\\
599.67	0.01\\
599.68	0.01\\
599.69	0.01\\
599.7	0.01\\
599.71	0.01\\
599.72	0.01\\
599.73	0.01\\
599.74	0.01\\
599.75	0.01\\
599.76	0.01\\
599.77	0.01\\
599.78	0.01\\
599.79	0.01\\
599.8	0.01\\
599.81	0.01\\
599.82	0.01\\
599.83	0.01\\
599.84	0.01\\
599.85	0.01\\
599.86	0.01\\
599.87	0.01\\
599.88	0.01\\
599.89	0.01\\
599.9	0.01\\
599.91	0.01\\
599.92	0.01\\
599.93	0.01\\
599.94	0.01\\
599.95	0.01\\
599.96	0.01\\
599.97	0.01\\
599.98	0.01\\
599.99	0.01\\
600	0.01\\
};
\addplot [color=mycolor19,solid,forget plot]
  table[row sep=crcr]{%
0.01	0.01\\
1.01	0.01\\
2.01	0.01\\
3.01	0.01\\
4.01	0.01\\
5.01	0.01\\
6.01	0.01\\
7.01	0.01\\
8.01	0.01\\
9.01	0.01\\
10.01	0.01\\
11.01	0.01\\
12.01	0.01\\
13.01	0.01\\
14.01	0.01\\
15.01	0.01\\
16.01	0.01\\
17.01	0.01\\
18.01	0.01\\
19.01	0.01\\
20.01	0.01\\
21.01	0.01\\
22.01	0.01\\
23.01	0.01\\
24.01	0.01\\
25.01	0.01\\
26.01	0.01\\
27.01	0.01\\
28.01	0.01\\
29.01	0.01\\
30.01	0.01\\
31.01	0.01\\
32.01	0.01\\
33.01	0.01\\
34.01	0.01\\
35.01	0.01\\
36.01	0.01\\
37.01	0.01\\
38.01	0.01\\
39.01	0.01\\
40.01	0.01\\
41.01	0.01\\
42.01	0.01\\
43.01	0.01\\
44.01	0.01\\
45.01	0.01\\
46.01	0.01\\
47.01	0.01\\
48.01	0.01\\
49.01	0.01\\
50.01	0.01\\
51.01	0.01\\
52.01	0.01\\
53.01	0.01\\
54.01	0.01\\
55.01	0.01\\
56.01	0.01\\
57.01	0.01\\
58.01	0.01\\
59.01	0.01\\
60.01	0.01\\
61.01	0.01\\
62.01	0.01\\
63.01	0.01\\
64.01	0.01\\
65.01	0.01\\
66.01	0.01\\
67.01	0.01\\
68.01	0.01\\
69.01	0.01\\
70.01	0.01\\
71.01	0.01\\
72.01	0.01\\
73.01	0.01\\
74.01	0.01\\
75.01	0.01\\
76.01	0.01\\
77.01	0.01\\
78.01	0.01\\
79.01	0.01\\
80.01	0.01\\
81.01	0.01\\
82.01	0.01\\
83.01	0.01\\
84.01	0.01\\
85.01	0.01\\
86.01	0.01\\
87.01	0.01\\
88.01	0.01\\
89.01	0.01\\
90.01	0.01\\
91.01	0.01\\
92.01	0.01\\
93.01	0.01\\
94.01	0.01\\
95.01	0.01\\
96.01	0.01\\
97.01	0.01\\
98.01	0.01\\
99.01	0.01\\
100.01	0.01\\
101.01	0.01\\
102.01	0.01\\
103.01	0.01\\
104.01	0.01\\
105.01	0.01\\
106.01	0.01\\
107.01	0.01\\
108.01	0.01\\
109.01	0.01\\
110.01	0.01\\
111.01	0.01\\
112.01	0.01\\
113.01	0.01\\
114.01	0.01\\
115.01	0.01\\
116.01	0.01\\
117.01	0.01\\
118.01	0.01\\
119.01	0.01\\
120.01	0.01\\
121.01	0.01\\
122.01	0.01\\
123.01	0.01\\
124.01	0.01\\
125.01	0.01\\
126.01	0.01\\
127.01	0.01\\
128.01	0.01\\
129.01	0.01\\
130.01	0.01\\
131.01	0.01\\
132.01	0.01\\
133.01	0.01\\
134.01	0.01\\
135.01	0.01\\
136.01	0.01\\
137.01	0.01\\
138.01	0.01\\
139.01	0.01\\
140.01	0.01\\
141.01	0.01\\
142.01	0.01\\
143.01	0.01\\
144.01	0.01\\
145.01	0.01\\
146.01	0.01\\
147.01	0.01\\
148.01	0.01\\
149.01	0.01\\
150.01	0.01\\
151.01	0.01\\
152.01	0.01\\
153.01	0.01\\
154.01	0.01\\
155.01	0.01\\
156.01	0.01\\
157.01	0.01\\
158.01	0.01\\
159.01	0.01\\
160.01	0.01\\
161.01	0.01\\
162.01	0.01\\
163.01	0.01\\
164.01	0.01\\
165.01	0.01\\
166.01	0.01\\
167.01	0.01\\
168.01	0.01\\
169.01	0.01\\
170.01	0.01\\
171.01	0.01\\
172.01	0.01\\
173.01	0.01\\
174.01	0.01\\
175.01	0.01\\
176.01	0.01\\
177.01	0.01\\
178.01	0.01\\
179.01	0.01\\
180.01	0.01\\
181.01	0.01\\
182.01	0.01\\
183.01	0.01\\
184.01	0.01\\
185.01	0.01\\
186.01	0.01\\
187.01	0.01\\
188.01	0.01\\
189.01	0.01\\
190.01	0.01\\
191.01	0.01\\
192.01	0.01\\
193.01	0.01\\
194.01	0.01\\
195.01	0.01\\
196.01	0.01\\
197.01	0.01\\
198.01	0.01\\
199.01	0.01\\
200.01	0.01\\
201.01	0.01\\
202.01	0.01\\
203.01	0.01\\
204.01	0.01\\
205.01	0.01\\
206.01	0.01\\
207.01	0.01\\
208.01	0.01\\
209.01	0.01\\
210.01	0.01\\
211.01	0.01\\
212.01	0.01\\
213.01	0.01\\
214.01	0.01\\
215.01	0.01\\
216.01	0.01\\
217.01	0.01\\
218.01	0.01\\
219.01	0.01\\
220.01	0.01\\
221.01	0.01\\
222.01	0.01\\
223.01	0.01\\
224.01	0.01\\
225.01	0.01\\
226.01	0.01\\
227.01	0.01\\
228.01	0.01\\
229.01	0.01\\
230.01	0.01\\
231.01	0.01\\
232.01	0.01\\
233.01	0.01\\
234.01	0.01\\
235.01	0.01\\
236.01	0.01\\
237.01	0.01\\
238.01	0.01\\
239.01	0.01\\
240.01	0.01\\
241.01	0.01\\
242.01	0.01\\
243.01	0.01\\
244.01	0.01\\
245.01	0.01\\
246.01	0.01\\
247.01	0.01\\
248.01	0.01\\
249.01	0.01\\
250.01	0.01\\
251.01	0.01\\
252.01	0.01\\
253.01	0.01\\
254.01	0.01\\
255.01	0.01\\
256.01	0.01\\
257.01	0.01\\
258.01	0.01\\
259.01	0.01\\
260.01	0.01\\
261.01	0.01\\
262.01	0.01\\
263.01	0.01\\
264.01	0.01\\
265.01	0.01\\
266.01	0.01\\
267.01	0.01\\
268.01	0.01\\
269.01	0.01\\
270.01	0.01\\
271.01	0.01\\
272.01	0.01\\
273.01	0.01\\
274.01	0.01\\
275.01	0.01\\
276.01	0.01\\
277.01	0.01\\
278.01	0.01\\
279.01	0.01\\
280.01	0.01\\
281.01	0.01\\
282.01	0.01\\
283.01	0.01\\
284.01	0.01\\
285.01	0.01\\
286.01	0.01\\
287.01	0.01\\
288.01	0.01\\
289.01	0.01\\
290.01	0.01\\
291.01	0.01\\
292.01	0.01\\
293.01	0.01\\
294.01	0.01\\
295.01	0.01\\
296.01	0.01\\
297.01	0.01\\
298.01	0.01\\
299.01	0.01\\
300.01	0.01\\
301.01	0.01\\
302.01	0.01\\
303.01	0.01\\
304.01	0.01\\
305.01	0.01\\
306.01	0.01\\
307.01	0.01\\
308.01	0.01\\
309.01	0.01\\
310.01	0.01\\
311.01	0.01\\
312.01	0.01\\
313.01	0.01\\
314.01	0.01\\
315.01	0.01\\
316.01	0.01\\
317.01	0.01\\
318.01	0.01\\
319.01	0.01\\
320.01	0.01\\
321.01	0.01\\
322.01	0.01\\
323.01	0.01\\
324.01	0.01\\
325.01	0.01\\
326.01	0.01\\
327.01	0.01\\
328.01	0.01\\
329.01	0.01\\
330.01	0.01\\
331.01	0.01\\
332.01	0.01\\
333.01	0.01\\
334.01	0.01\\
335.01	0.01\\
336.01	0.01\\
337.01	0.01\\
338.01	0.01\\
339.01	0.01\\
340.01	0.01\\
341.01	0.01\\
342.01	0.01\\
343.01	0.01\\
344.01	0.01\\
345.01	0.01\\
346.01	0.01\\
347.01	0.01\\
348.01	0.01\\
349.01	0.01\\
350.01	0.01\\
351.01	0.01\\
352.01	0.01\\
353.01	0.01\\
354.01	0.01\\
355.01	0.01\\
356.01	0.01\\
357.01	0.01\\
358.01	0.01\\
359.01	0.01\\
360.01	0.01\\
361.01	0.01\\
362.01	0.01\\
363.01	0.01\\
364.01	0.01\\
365.01	0.01\\
366.01	0.01\\
367.01	0.01\\
368.01	0.01\\
369.01	0.01\\
370.01	0.01\\
371.01	0.01\\
372.01	0.01\\
373.01	0.01\\
374.01	0.01\\
375.01	0.01\\
376.01	0.01\\
377.01	0.01\\
378.01	0.01\\
379.01	0.01\\
380.01	0.01\\
381.01	0.01\\
382.01	0.01\\
383.01	0.01\\
384.01	0.01\\
385.01	0.01\\
386.01	0.01\\
387.01	0.01\\
388.01	0.01\\
389.01	0.01\\
390.01	0.01\\
391.01	0.01\\
392.01	0.01\\
393.01	0.01\\
394.01	0.01\\
395.01	0.01\\
396.01	0.01\\
397.01	0.01\\
398.01	0.01\\
399.01	0.01\\
400.01	0.01\\
401.01	0.01\\
402.01	0.01\\
403.01	0.01\\
404.01	0.01\\
405.01	0.01\\
406.01	0.01\\
407.01	0.01\\
408.01	0.01\\
409.01	0.01\\
410.01	0.01\\
411.01	0.01\\
412.01	0.01\\
413.01	0.01\\
414.01	0.01\\
415.01	0.01\\
416.01	0.01\\
417.01	0.01\\
418.01	0.01\\
419.01	0.01\\
420.01	0.01\\
421.01	0.01\\
422.01	0.01\\
423.01	0.01\\
424.01	0.01\\
425.01	0.01\\
426.01	0.01\\
427.01	0.01\\
428.01	0.01\\
429.01	0.01\\
430.01	0.01\\
431.01	0.01\\
432.01	0.01\\
433.01	0.01\\
434.01	0.01\\
435.01	0.01\\
436.01	0.01\\
437.01	0.01\\
438.01	0.01\\
439.01	0.01\\
440.01	0.01\\
441.01	0.01\\
442.01	0.01\\
443.01	0.01\\
444.01	0.01\\
445.01	0.01\\
446.01	0.01\\
447.01	0.01\\
448.01	0.01\\
449.01	0.01\\
450.01	0.01\\
451.01	0.01\\
452.01	0.01\\
453.01	0.01\\
454.01	0.01\\
455.01	0.01\\
456.01	0.01\\
457.01	0.01\\
458.01	0.01\\
459.01	0.01\\
460.01	0.01\\
461.01	0.01\\
462.01	0.01\\
463.01	0.01\\
464.01	0.01\\
465.01	0.01\\
466.01	0.01\\
467.01	0.01\\
468.01	0.01\\
469.01	0.01\\
470.01	0.01\\
471.01	0.01\\
472.01	0.01\\
473.01	0.01\\
474.01	0.01\\
475.01	0.01\\
476.01	0.01\\
477.01	0.01\\
478.01	0.01\\
479.01	0.01\\
480.01	0.01\\
481.01	0.01\\
482.01	0.01\\
483.01	0.01\\
484.01	0.01\\
485.01	0.01\\
486.01	0.01\\
487.01	0.01\\
488.01	0.01\\
489.01	0.01\\
490.01	0.01\\
491.01	0.01\\
492.01	0.01\\
493.01	0.01\\
494.01	0.01\\
495.01	0.01\\
496.01	0.01\\
497.01	0.01\\
498.01	0.01\\
499.01	0.01\\
500.01	0.01\\
501.01	0.01\\
502.01	0.01\\
503.01	0.01\\
504.01	0.01\\
505.01	0.01\\
506.01	0.01\\
507.01	0.01\\
508.01	0.01\\
509.01	0.01\\
510.01	0.01\\
511.01	0.01\\
512.01	0.01\\
513.01	0.01\\
514.01	0.01\\
515.01	0.01\\
516.01	0.01\\
517.01	0.01\\
518.01	0.01\\
519.01	0.01\\
520.01	0.01\\
521.01	0.01\\
522.01	0.01\\
523.01	0.01\\
524.01	0.01\\
525.01	0.01\\
526.01	0.01\\
527.01	0.01\\
528.01	0.01\\
529.01	0.01\\
530.01	0.01\\
531.01	0.01\\
532.01	0.01\\
533.01	0.01\\
534.01	0.01\\
535.01	0.01\\
536.01	0.01\\
537.01	0.01\\
538.01	0.01\\
539.01	0.01\\
540.01	0.01\\
541.01	0.01\\
542.01	0.01\\
543.01	0.01\\
544.01	0.01\\
545.01	0.01\\
546.01	0.01\\
547.01	0.01\\
548.01	0.01\\
549.01	0.01\\
550.01	0.01\\
551.01	0.01\\
552.01	0.01\\
553.01	0.01\\
554.01	0.01\\
555.01	0.01\\
556.01	0.01\\
557.01	0.01\\
558.01	0.01\\
559.01	0.01\\
560.01	0.01\\
561.01	0.01\\
562.01	0.01\\
563.01	0.01\\
564.01	0.01\\
565.01	0.01\\
566.01	0.01\\
567.01	0.01\\
568.01	0.01\\
569.01	0.01\\
570.01	0.01\\
571.01	0.01\\
572.01	0.01\\
573.01	0.01\\
574.01	0.01\\
575.01	0.01\\
576.01	0.01\\
577.01	0.01\\
578.01	0.01\\
579.01	0.01\\
580.01	0.01\\
581.01	0.01\\
582.01	0.01\\
583.01	0.01\\
584.01	0.01\\
585.01	0.01\\
586.01	0.01\\
587.01	0.01\\
588.01	0.01\\
589.01	0.01\\
590.01	0.01\\
591.01	0.01\\
592.01	0.01\\
593.01	0.01\\
594.01	0.01\\
595.01	0.01\\
596.01	0.01\\
597.01	0.01\\
598.01	0.01\\
599.01	0.01\\
599.02	0.01\\
599.03	0.01\\
599.04	0.01\\
599.05	0.01\\
599.06	0.01\\
599.07	0.01\\
599.08	0.01\\
599.09	0.01\\
599.1	0.01\\
599.11	0.01\\
599.12	0.01\\
599.13	0.01\\
599.14	0.01\\
599.15	0.01\\
599.16	0.01\\
599.17	0.01\\
599.18	0.01\\
599.19	0.01\\
599.2	0.01\\
599.21	0.01\\
599.22	0.01\\
599.23	0.01\\
599.24	0.01\\
599.25	0.01\\
599.26	0.01\\
599.27	0.01\\
599.28	0.01\\
599.29	0.01\\
599.3	0.01\\
599.31	0.01\\
599.32	0.01\\
599.33	0.01\\
599.34	0.01\\
599.35	0.01\\
599.36	0.01\\
599.37	0.01\\
599.38	0.01\\
599.39	0.01\\
599.4	0.01\\
599.41	0.01\\
599.42	0.01\\
599.43	0.01\\
599.44	0.01\\
599.45	0.01\\
599.46	0.01\\
599.47	0.01\\
599.48	0.01\\
599.49	0.01\\
599.5	0.01\\
599.51	0.01\\
599.52	0.01\\
599.53	0.01\\
599.54	0.01\\
599.55	0.01\\
599.56	0.01\\
599.57	0.01\\
599.58	0.01\\
599.59	0.01\\
599.6	0.01\\
599.61	0.01\\
599.62	0.01\\
599.63	0.01\\
599.64	0.01\\
599.65	0.01\\
599.66	0.01\\
599.67	0.01\\
599.68	0.01\\
599.69	0.01\\
599.7	0.01\\
599.71	0.01\\
599.72	0.01\\
599.73	0.01\\
599.74	0.01\\
599.75	0.01\\
599.76	0.01\\
599.77	0.01\\
599.78	0.01\\
599.79	0.01\\
599.8	0.01\\
599.81	0.01\\
599.82	0.01\\
599.83	0.01\\
599.84	0.01\\
599.85	0.01\\
599.86	0.01\\
599.87	0.01\\
599.88	0.01\\
599.89	0.01\\
599.9	0.01\\
599.91	0.01\\
599.92	0.01\\
599.93	0.01\\
599.94	0.01\\
599.95	0.01\\
599.96	0.01\\
599.97	0.01\\
599.98	0.01\\
599.99	0.01\\
600	0.01\\
};
\addplot [color=red!50!mycolor17,solid,forget plot]
  table[row sep=crcr]{%
0.01	0.01\\
1.01	0.01\\
2.01	0.01\\
3.01	0.01\\
4.01	0.01\\
5.01	0.01\\
6.01	0.01\\
7.01	0.01\\
8.01	0.01\\
9.01	0.01\\
10.01	0.01\\
11.01	0.01\\
12.01	0.01\\
13.01	0.01\\
14.01	0.01\\
15.01	0.01\\
16.01	0.01\\
17.01	0.01\\
18.01	0.01\\
19.01	0.01\\
20.01	0.01\\
21.01	0.01\\
22.01	0.01\\
23.01	0.01\\
24.01	0.01\\
25.01	0.01\\
26.01	0.01\\
27.01	0.01\\
28.01	0.01\\
29.01	0.01\\
30.01	0.01\\
31.01	0.01\\
32.01	0.01\\
33.01	0.01\\
34.01	0.01\\
35.01	0.01\\
36.01	0.01\\
37.01	0.01\\
38.01	0.01\\
39.01	0.01\\
40.01	0.01\\
41.01	0.01\\
42.01	0.01\\
43.01	0.01\\
44.01	0.01\\
45.01	0.01\\
46.01	0.01\\
47.01	0.01\\
48.01	0.01\\
49.01	0.01\\
50.01	0.01\\
51.01	0.01\\
52.01	0.01\\
53.01	0.01\\
54.01	0.01\\
55.01	0.01\\
56.01	0.01\\
57.01	0.01\\
58.01	0.01\\
59.01	0.01\\
60.01	0.01\\
61.01	0.01\\
62.01	0.01\\
63.01	0.01\\
64.01	0.01\\
65.01	0.01\\
66.01	0.01\\
67.01	0.01\\
68.01	0.01\\
69.01	0.01\\
70.01	0.01\\
71.01	0.01\\
72.01	0.01\\
73.01	0.01\\
74.01	0.01\\
75.01	0.01\\
76.01	0.01\\
77.01	0.01\\
78.01	0.01\\
79.01	0.01\\
80.01	0.01\\
81.01	0.01\\
82.01	0.01\\
83.01	0.01\\
84.01	0.01\\
85.01	0.01\\
86.01	0.01\\
87.01	0.01\\
88.01	0.01\\
89.01	0.01\\
90.01	0.01\\
91.01	0.01\\
92.01	0.01\\
93.01	0.01\\
94.01	0.01\\
95.01	0.01\\
96.01	0.01\\
97.01	0.01\\
98.01	0.01\\
99.01	0.01\\
100.01	0.01\\
101.01	0.01\\
102.01	0.01\\
103.01	0.01\\
104.01	0.01\\
105.01	0.01\\
106.01	0.01\\
107.01	0.01\\
108.01	0.01\\
109.01	0.01\\
110.01	0.01\\
111.01	0.01\\
112.01	0.01\\
113.01	0.01\\
114.01	0.01\\
115.01	0.01\\
116.01	0.01\\
117.01	0.01\\
118.01	0.01\\
119.01	0.01\\
120.01	0.01\\
121.01	0.01\\
122.01	0.01\\
123.01	0.01\\
124.01	0.01\\
125.01	0.01\\
126.01	0.01\\
127.01	0.01\\
128.01	0.01\\
129.01	0.01\\
130.01	0.01\\
131.01	0.01\\
132.01	0.01\\
133.01	0.01\\
134.01	0.01\\
135.01	0.01\\
136.01	0.01\\
137.01	0.01\\
138.01	0.01\\
139.01	0.01\\
140.01	0.01\\
141.01	0.01\\
142.01	0.01\\
143.01	0.01\\
144.01	0.01\\
145.01	0.01\\
146.01	0.01\\
147.01	0.01\\
148.01	0.01\\
149.01	0.01\\
150.01	0.01\\
151.01	0.01\\
152.01	0.01\\
153.01	0.01\\
154.01	0.01\\
155.01	0.01\\
156.01	0.01\\
157.01	0.01\\
158.01	0.01\\
159.01	0.01\\
160.01	0.01\\
161.01	0.01\\
162.01	0.01\\
163.01	0.01\\
164.01	0.01\\
165.01	0.01\\
166.01	0.01\\
167.01	0.01\\
168.01	0.01\\
169.01	0.01\\
170.01	0.01\\
171.01	0.01\\
172.01	0.01\\
173.01	0.01\\
174.01	0.01\\
175.01	0.01\\
176.01	0.01\\
177.01	0.01\\
178.01	0.01\\
179.01	0.01\\
180.01	0.01\\
181.01	0.01\\
182.01	0.01\\
183.01	0.01\\
184.01	0.01\\
185.01	0.01\\
186.01	0.01\\
187.01	0.01\\
188.01	0.01\\
189.01	0.01\\
190.01	0.01\\
191.01	0.01\\
192.01	0.01\\
193.01	0.01\\
194.01	0.01\\
195.01	0.01\\
196.01	0.01\\
197.01	0.01\\
198.01	0.01\\
199.01	0.01\\
200.01	0.01\\
201.01	0.01\\
202.01	0.01\\
203.01	0.01\\
204.01	0.01\\
205.01	0.01\\
206.01	0.01\\
207.01	0.01\\
208.01	0.01\\
209.01	0.01\\
210.01	0.01\\
211.01	0.01\\
212.01	0.01\\
213.01	0.01\\
214.01	0.01\\
215.01	0.01\\
216.01	0.01\\
217.01	0.01\\
218.01	0.01\\
219.01	0.01\\
220.01	0.01\\
221.01	0.01\\
222.01	0.01\\
223.01	0.01\\
224.01	0.01\\
225.01	0.01\\
226.01	0.01\\
227.01	0.01\\
228.01	0.01\\
229.01	0.01\\
230.01	0.01\\
231.01	0.01\\
232.01	0.01\\
233.01	0.01\\
234.01	0.01\\
235.01	0.01\\
236.01	0.01\\
237.01	0.01\\
238.01	0.01\\
239.01	0.01\\
240.01	0.01\\
241.01	0.01\\
242.01	0.01\\
243.01	0.01\\
244.01	0.01\\
245.01	0.01\\
246.01	0.01\\
247.01	0.01\\
248.01	0.01\\
249.01	0.01\\
250.01	0.01\\
251.01	0.01\\
252.01	0.01\\
253.01	0.01\\
254.01	0.01\\
255.01	0.01\\
256.01	0.01\\
257.01	0.01\\
258.01	0.01\\
259.01	0.01\\
260.01	0.01\\
261.01	0.01\\
262.01	0.01\\
263.01	0.01\\
264.01	0.01\\
265.01	0.01\\
266.01	0.01\\
267.01	0.01\\
268.01	0.01\\
269.01	0.01\\
270.01	0.01\\
271.01	0.01\\
272.01	0.01\\
273.01	0.01\\
274.01	0.01\\
275.01	0.01\\
276.01	0.01\\
277.01	0.01\\
278.01	0.01\\
279.01	0.01\\
280.01	0.01\\
281.01	0.01\\
282.01	0.01\\
283.01	0.01\\
284.01	0.01\\
285.01	0.01\\
286.01	0.01\\
287.01	0.01\\
288.01	0.01\\
289.01	0.01\\
290.01	0.01\\
291.01	0.01\\
292.01	0.01\\
293.01	0.01\\
294.01	0.01\\
295.01	0.01\\
296.01	0.01\\
297.01	0.01\\
298.01	0.01\\
299.01	0.01\\
300.01	0.01\\
301.01	0.01\\
302.01	0.01\\
303.01	0.01\\
304.01	0.01\\
305.01	0.01\\
306.01	0.01\\
307.01	0.01\\
308.01	0.01\\
309.01	0.01\\
310.01	0.01\\
311.01	0.01\\
312.01	0.01\\
313.01	0.01\\
314.01	0.01\\
315.01	0.01\\
316.01	0.01\\
317.01	0.01\\
318.01	0.01\\
319.01	0.01\\
320.01	0.01\\
321.01	0.01\\
322.01	0.01\\
323.01	0.01\\
324.01	0.01\\
325.01	0.01\\
326.01	0.01\\
327.01	0.01\\
328.01	0.01\\
329.01	0.01\\
330.01	0.01\\
331.01	0.01\\
332.01	0.01\\
333.01	0.01\\
334.01	0.01\\
335.01	0.01\\
336.01	0.01\\
337.01	0.01\\
338.01	0.01\\
339.01	0.01\\
340.01	0.01\\
341.01	0.01\\
342.01	0.01\\
343.01	0.01\\
344.01	0.01\\
345.01	0.01\\
346.01	0.01\\
347.01	0.01\\
348.01	0.01\\
349.01	0.01\\
350.01	0.01\\
351.01	0.01\\
352.01	0.01\\
353.01	0.01\\
354.01	0.01\\
355.01	0.01\\
356.01	0.01\\
357.01	0.01\\
358.01	0.01\\
359.01	0.01\\
360.01	0.01\\
361.01	0.01\\
362.01	0.01\\
363.01	0.01\\
364.01	0.01\\
365.01	0.01\\
366.01	0.01\\
367.01	0.01\\
368.01	0.01\\
369.01	0.01\\
370.01	0.01\\
371.01	0.01\\
372.01	0.01\\
373.01	0.01\\
374.01	0.01\\
375.01	0.01\\
376.01	0.01\\
377.01	0.01\\
378.01	0.01\\
379.01	0.01\\
380.01	0.01\\
381.01	0.01\\
382.01	0.01\\
383.01	0.01\\
384.01	0.01\\
385.01	0.01\\
386.01	0.01\\
387.01	0.01\\
388.01	0.01\\
389.01	0.01\\
390.01	0.01\\
391.01	0.01\\
392.01	0.01\\
393.01	0.01\\
394.01	0.01\\
395.01	0.01\\
396.01	0.01\\
397.01	0.01\\
398.01	0.01\\
399.01	0.01\\
400.01	0.01\\
401.01	0.01\\
402.01	0.01\\
403.01	0.01\\
404.01	0.01\\
405.01	0.01\\
406.01	0.01\\
407.01	0.01\\
408.01	0.01\\
409.01	0.01\\
410.01	0.01\\
411.01	0.01\\
412.01	0.01\\
413.01	0.01\\
414.01	0.01\\
415.01	0.01\\
416.01	0.01\\
417.01	0.01\\
418.01	0.01\\
419.01	0.01\\
420.01	0.01\\
421.01	0.01\\
422.01	0.01\\
423.01	0.01\\
424.01	0.01\\
425.01	0.01\\
426.01	0.01\\
427.01	0.01\\
428.01	0.01\\
429.01	0.01\\
430.01	0.01\\
431.01	0.01\\
432.01	0.01\\
433.01	0.01\\
434.01	0.01\\
435.01	0.01\\
436.01	0.01\\
437.01	0.01\\
438.01	0.01\\
439.01	0.01\\
440.01	0.01\\
441.01	0.01\\
442.01	0.01\\
443.01	0.01\\
444.01	0.01\\
445.01	0.01\\
446.01	0.01\\
447.01	0.01\\
448.01	0.01\\
449.01	0.01\\
450.01	0.01\\
451.01	0.01\\
452.01	0.01\\
453.01	0.01\\
454.01	0.01\\
455.01	0.01\\
456.01	0.01\\
457.01	0.01\\
458.01	0.01\\
459.01	0.01\\
460.01	0.01\\
461.01	0.01\\
462.01	0.01\\
463.01	0.01\\
464.01	0.01\\
465.01	0.01\\
466.01	0.01\\
467.01	0.01\\
468.01	0.01\\
469.01	0.01\\
470.01	0.01\\
471.01	0.01\\
472.01	0.01\\
473.01	0.01\\
474.01	0.01\\
475.01	0.01\\
476.01	0.01\\
477.01	0.01\\
478.01	0.01\\
479.01	0.01\\
480.01	0.01\\
481.01	0.01\\
482.01	0.01\\
483.01	0.01\\
484.01	0.01\\
485.01	0.01\\
486.01	0.01\\
487.01	0.01\\
488.01	0.01\\
489.01	0.01\\
490.01	0.01\\
491.01	0.01\\
492.01	0.01\\
493.01	0.01\\
494.01	0.01\\
495.01	0.01\\
496.01	0.01\\
497.01	0.01\\
498.01	0.01\\
499.01	0.01\\
500.01	0.01\\
501.01	0.01\\
502.01	0.01\\
503.01	0.01\\
504.01	0.01\\
505.01	0.01\\
506.01	0.01\\
507.01	0.01\\
508.01	0.01\\
509.01	0.01\\
510.01	0.01\\
511.01	0.01\\
512.01	0.01\\
513.01	0.01\\
514.01	0.01\\
515.01	0.01\\
516.01	0.01\\
517.01	0.01\\
518.01	0.01\\
519.01	0.01\\
520.01	0.01\\
521.01	0.01\\
522.01	0.01\\
523.01	0.01\\
524.01	0.01\\
525.01	0.01\\
526.01	0.01\\
527.01	0.01\\
528.01	0.01\\
529.01	0.01\\
530.01	0.01\\
531.01	0.01\\
532.01	0.01\\
533.01	0.01\\
534.01	0.01\\
535.01	0.01\\
536.01	0.01\\
537.01	0.01\\
538.01	0.01\\
539.01	0.01\\
540.01	0.01\\
541.01	0.01\\
542.01	0.01\\
543.01	0.01\\
544.01	0.01\\
545.01	0.01\\
546.01	0.01\\
547.01	0.01\\
548.01	0.01\\
549.01	0.01\\
550.01	0.01\\
551.01	0.01\\
552.01	0.01\\
553.01	0.01\\
554.01	0.01\\
555.01	0.01\\
556.01	0.01\\
557.01	0.01\\
558.01	0.01\\
559.01	0.01\\
560.01	0.01\\
561.01	0.01\\
562.01	0.01\\
563.01	0.01\\
564.01	0.01\\
565.01	0.01\\
566.01	0.01\\
567.01	0.01\\
568.01	0.01\\
569.01	0.01\\
570.01	0.01\\
571.01	0.01\\
572.01	0.01\\
573.01	0.01\\
574.01	0.01\\
575.01	0.01\\
576.01	0.01\\
577.01	0.01\\
578.01	0.01\\
579.01	0.01\\
580.01	0.01\\
581.01	0.01\\
582.01	0.01\\
583.01	0.01\\
584.01	0.01\\
585.01	0.01\\
586.01	0.01\\
587.01	0.01\\
588.01	0.01\\
589.01	0.01\\
590.01	0.01\\
591.01	0.01\\
592.01	0.01\\
593.01	0.01\\
594.01	0.01\\
595.01	0.01\\
596.01	0.01\\
597.01	0.01\\
598.01	0.01\\
599.01	0.01\\
599.02	0.01\\
599.03	0.01\\
599.04	0.01\\
599.05	0.01\\
599.06	0.01\\
599.07	0.01\\
599.08	0.01\\
599.09	0.01\\
599.1	0.01\\
599.11	0.01\\
599.12	0.01\\
599.13	0.01\\
599.14	0.01\\
599.15	0.01\\
599.16	0.01\\
599.17	0.01\\
599.18	0.01\\
599.19	0.01\\
599.2	0.01\\
599.21	0.01\\
599.22	0.01\\
599.23	0.01\\
599.24	0.01\\
599.25	0.01\\
599.26	0.01\\
599.27	0.01\\
599.28	0.01\\
599.29	0.01\\
599.3	0.01\\
599.31	0.01\\
599.32	0.01\\
599.33	0.01\\
599.34	0.01\\
599.35	0.01\\
599.36	0.01\\
599.37	0.01\\
599.38	0.01\\
599.39	0.01\\
599.4	0.01\\
599.41	0.01\\
599.42	0.01\\
599.43	0.01\\
599.44	0.01\\
599.45	0.01\\
599.46	0.01\\
599.47	0.01\\
599.48	0.01\\
599.49	0.01\\
599.5	0.01\\
599.51	0.01\\
599.52	0.01\\
599.53	0.01\\
599.54	0.01\\
599.55	0.01\\
599.56	0.01\\
599.57	0.01\\
599.58	0.01\\
599.59	0.01\\
599.6	0.01\\
599.61	0.01\\
599.62	0.01\\
599.63	0.01\\
599.64	0.01\\
599.65	0.01\\
599.66	0.01\\
599.67	0.01\\
599.68	0.01\\
599.69	0.01\\
599.7	0.01\\
599.71	0.01\\
599.72	0.01\\
599.73	0.01\\
599.74	0.01\\
599.75	0.01\\
599.76	0.01\\
599.77	0.01\\
599.78	0.01\\
599.79	0.01\\
599.8	0.01\\
599.81	0.01\\
599.82	0.01\\
599.83	0.01\\
599.84	0.01\\
599.85	0.01\\
599.86	0.01\\
599.87	0.01\\
599.88	0.01\\
599.89	0.01\\
599.9	0.01\\
599.91	0.01\\
599.92	0.01\\
599.93	0.01\\
599.94	0.01\\
599.95	0.01\\
599.96	0.01\\
599.97	0.01\\
599.98	0.01\\
599.99	0.01\\
600	0.01\\
};
\addplot [color=red!40!mycolor19,solid,forget plot]
  table[row sep=crcr]{%
0.01	0.01\\
1.01	0.01\\
2.01	0.01\\
3.01	0.01\\
4.01	0.01\\
5.01	0.01\\
6.01	0.01\\
7.01	0.01\\
8.01	0.01\\
9.01	0.01\\
10.01	0.01\\
11.01	0.01\\
12.01	0.01\\
13.01	0.01\\
14.01	0.01\\
15.01	0.01\\
16.01	0.01\\
17.01	0.01\\
18.01	0.01\\
19.01	0.01\\
20.01	0.01\\
21.01	0.01\\
22.01	0.01\\
23.01	0.01\\
24.01	0.01\\
25.01	0.01\\
26.01	0.01\\
27.01	0.01\\
28.01	0.01\\
29.01	0.01\\
30.01	0.01\\
31.01	0.01\\
32.01	0.01\\
33.01	0.01\\
34.01	0.01\\
35.01	0.01\\
36.01	0.01\\
37.01	0.01\\
38.01	0.01\\
39.01	0.01\\
40.01	0.01\\
41.01	0.01\\
42.01	0.01\\
43.01	0.01\\
44.01	0.01\\
45.01	0.01\\
46.01	0.01\\
47.01	0.01\\
48.01	0.01\\
49.01	0.01\\
50.01	0.01\\
51.01	0.01\\
52.01	0.01\\
53.01	0.01\\
54.01	0.01\\
55.01	0.01\\
56.01	0.01\\
57.01	0.01\\
58.01	0.01\\
59.01	0.01\\
60.01	0.01\\
61.01	0.01\\
62.01	0.01\\
63.01	0.01\\
64.01	0.01\\
65.01	0.01\\
66.01	0.01\\
67.01	0.01\\
68.01	0.01\\
69.01	0.01\\
70.01	0.01\\
71.01	0.01\\
72.01	0.01\\
73.01	0.01\\
74.01	0.01\\
75.01	0.01\\
76.01	0.01\\
77.01	0.01\\
78.01	0.01\\
79.01	0.01\\
80.01	0.01\\
81.01	0.01\\
82.01	0.01\\
83.01	0.01\\
84.01	0.01\\
85.01	0.01\\
86.01	0.01\\
87.01	0.01\\
88.01	0.01\\
89.01	0.01\\
90.01	0.01\\
91.01	0.01\\
92.01	0.01\\
93.01	0.01\\
94.01	0.01\\
95.01	0.01\\
96.01	0.01\\
97.01	0.01\\
98.01	0.01\\
99.01	0.01\\
100.01	0.01\\
101.01	0.01\\
102.01	0.01\\
103.01	0.01\\
104.01	0.01\\
105.01	0.01\\
106.01	0.01\\
107.01	0.01\\
108.01	0.01\\
109.01	0.01\\
110.01	0.01\\
111.01	0.01\\
112.01	0.01\\
113.01	0.01\\
114.01	0.01\\
115.01	0.01\\
116.01	0.01\\
117.01	0.01\\
118.01	0.01\\
119.01	0.01\\
120.01	0.01\\
121.01	0.01\\
122.01	0.01\\
123.01	0.01\\
124.01	0.01\\
125.01	0.01\\
126.01	0.01\\
127.01	0.01\\
128.01	0.01\\
129.01	0.01\\
130.01	0.01\\
131.01	0.01\\
132.01	0.01\\
133.01	0.01\\
134.01	0.01\\
135.01	0.01\\
136.01	0.01\\
137.01	0.01\\
138.01	0.01\\
139.01	0.01\\
140.01	0.01\\
141.01	0.01\\
142.01	0.01\\
143.01	0.01\\
144.01	0.01\\
145.01	0.01\\
146.01	0.01\\
147.01	0.01\\
148.01	0.01\\
149.01	0.01\\
150.01	0.01\\
151.01	0.01\\
152.01	0.01\\
153.01	0.01\\
154.01	0.01\\
155.01	0.01\\
156.01	0.01\\
157.01	0.01\\
158.01	0.01\\
159.01	0.01\\
160.01	0.01\\
161.01	0.01\\
162.01	0.01\\
163.01	0.01\\
164.01	0.01\\
165.01	0.01\\
166.01	0.01\\
167.01	0.01\\
168.01	0.01\\
169.01	0.01\\
170.01	0.01\\
171.01	0.01\\
172.01	0.01\\
173.01	0.01\\
174.01	0.01\\
175.01	0.01\\
176.01	0.01\\
177.01	0.01\\
178.01	0.01\\
179.01	0.01\\
180.01	0.01\\
181.01	0.01\\
182.01	0.01\\
183.01	0.01\\
184.01	0.01\\
185.01	0.01\\
186.01	0.01\\
187.01	0.01\\
188.01	0.01\\
189.01	0.01\\
190.01	0.01\\
191.01	0.01\\
192.01	0.01\\
193.01	0.01\\
194.01	0.01\\
195.01	0.01\\
196.01	0.01\\
197.01	0.01\\
198.01	0.01\\
199.01	0.01\\
200.01	0.01\\
201.01	0.01\\
202.01	0.01\\
203.01	0.01\\
204.01	0.01\\
205.01	0.01\\
206.01	0.01\\
207.01	0.01\\
208.01	0.01\\
209.01	0.01\\
210.01	0.01\\
211.01	0.01\\
212.01	0.01\\
213.01	0.01\\
214.01	0.01\\
215.01	0.01\\
216.01	0.01\\
217.01	0.01\\
218.01	0.01\\
219.01	0.01\\
220.01	0.01\\
221.01	0.01\\
222.01	0.01\\
223.01	0.01\\
224.01	0.01\\
225.01	0.01\\
226.01	0.01\\
227.01	0.01\\
228.01	0.01\\
229.01	0.01\\
230.01	0.01\\
231.01	0.01\\
232.01	0.01\\
233.01	0.01\\
234.01	0.01\\
235.01	0.01\\
236.01	0.01\\
237.01	0.01\\
238.01	0.01\\
239.01	0.01\\
240.01	0.01\\
241.01	0.01\\
242.01	0.01\\
243.01	0.01\\
244.01	0.01\\
245.01	0.01\\
246.01	0.01\\
247.01	0.01\\
248.01	0.01\\
249.01	0.01\\
250.01	0.01\\
251.01	0.01\\
252.01	0.01\\
253.01	0.01\\
254.01	0.01\\
255.01	0.01\\
256.01	0.01\\
257.01	0.01\\
258.01	0.01\\
259.01	0.01\\
260.01	0.01\\
261.01	0.01\\
262.01	0.01\\
263.01	0.01\\
264.01	0.01\\
265.01	0.01\\
266.01	0.01\\
267.01	0.01\\
268.01	0.01\\
269.01	0.01\\
270.01	0.01\\
271.01	0.01\\
272.01	0.01\\
273.01	0.01\\
274.01	0.01\\
275.01	0.01\\
276.01	0.01\\
277.01	0.01\\
278.01	0.01\\
279.01	0.01\\
280.01	0.01\\
281.01	0.01\\
282.01	0.01\\
283.01	0.01\\
284.01	0.01\\
285.01	0.01\\
286.01	0.01\\
287.01	0.01\\
288.01	0.01\\
289.01	0.01\\
290.01	0.01\\
291.01	0.01\\
292.01	0.01\\
293.01	0.01\\
294.01	0.01\\
295.01	0.01\\
296.01	0.01\\
297.01	0.01\\
298.01	0.01\\
299.01	0.01\\
300.01	0.01\\
301.01	0.01\\
302.01	0.01\\
303.01	0.01\\
304.01	0.01\\
305.01	0.01\\
306.01	0.01\\
307.01	0.01\\
308.01	0.01\\
309.01	0.01\\
310.01	0.01\\
311.01	0.01\\
312.01	0.01\\
313.01	0.01\\
314.01	0.01\\
315.01	0.01\\
316.01	0.01\\
317.01	0.01\\
318.01	0.01\\
319.01	0.01\\
320.01	0.01\\
321.01	0.01\\
322.01	0.01\\
323.01	0.01\\
324.01	0.01\\
325.01	0.01\\
326.01	0.01\\
327.01	0.01\\
328.01	0.01\\
329.01	0.01\\
330.01	0.01\\
331.01	0.01\\
332.01	0.01\\
333.01	0.01\\
334.01	0.01\\
335.01	0.01\\
336.01	0.01\\
337.01	0.01\\
338.01	0.01\\
339.01	0.01\\
340.01	0.01\\
341.01	0.01\\
342.01	0.01\\
343.01	0.01\\
344.01	0.01\\
345.01	0.01\\
346.01	0.01\\
347.01	0.01\\
348.01	0.01\\
349.01	0.01\\
350.01	0.01\\
351.01	0.01\\
352.01	0.01\\
353.01	0.01\\
354.01	0.01\\
355.01	0.01\\
356.01	0.01\\
357.01	0.01\\
358.01	0.01\\
359.01	0.01\\
360.01	0.01\\
361.01	0.01\\
362.01	0.01\\
363.01	0.01\\
364.01	0.01\\
365.01	0.01\\
366.01	0.01\\
367.01	0.01\\
368.01	0.01\\
369.01	0.01\\
370.01	0.01\\
371.01	0.01\\
372.01	0.01\\
373.01	0.01\\
374.01	0.01\\
375.01	0.01\\
376.01	0.01\\
377.01	0.01\\
378.01	0.01\\
379.01	0.01\\
380.01	0.01\\
381.01	0.01\\
382.01	0.01\\
383.01	0.01\\
384.01	0.01\\
385.01	0.01\\
386.01	0.01\\
387.01	0.01\\
388.01	0.01\\
389.01	0.01\\
390.01	0.01\\
391.01	0.01\\
392.01	0.01\\
393.01	0.01\\
394.01	0.01\\
395.01	0.01\\
396.01	0.01\\
397.01	0.01\\
398.01	0.01\\
399.01	0.01\\
400.01	0.01\\
401.01	0.01\\
402.01	0.01\\
403.01	0.01\\
404.01	0.01\\
405.01	0.01\\
406.01	0.01\\
407.01	0.01\\
408.01	0.01\\
409.01	0.01\\
410.01	0.01\\
411.01	0.01\\
412.01	0.01\\
413.01	0.01\\
414.01	0.01\\
415.01	0.01\\
416.01	0.01\\
417.01	0.01\\
418.01	0.01\\
419.01	0.01\\
420.01	0.01\\
421.01	0.01\\
422.01	0.01\\
423.01	0.01\\
424.01	0.01\\
425.01	0.01\\
426.01	0.01\\
427.01	0.01\\
428.01	0.01\\
429.01	0.01\\
430.01	0.01\\
431.01	0.01\\
432.01	0.01\\
433.01	0.01\\
434.01	0.01\\
435.01	0.01\\
436.01	0.01\\
437.01	0.01\\
438.01	0.01\\
439.01	0.01\\
440.01	0.01\\
441.01	0.01\\
442.01	0.01\\
443.01	0.01\\
444.01	0.01\\
445.01	0.01\\
446.01	0.01\\
447.01	0.01\\
448.01	0.01\\
449.01	0.01\\
450.01	0.01\\
451.01	0.01\\
452.01	0.01\\
453.01	0.01\\
454.01	0.01\\
455.01	0.01\\
456.01	0.01\\
457.01	0.01\\
458.01	0.01\\
459.01	0.01\\
460.01	0.01\\
461.01	0.01\\
462.01	0.01\\
463.01	0.01\\
464.01	0.01\\
465.01	0.01\\
466.01	0.01\\
467.01	0.01\\
468.01	0.01\\
469.01	0.01\\
470.01	0.01\\
471.01	0.01\\
472.01	0.01\\
473.01	0.01\\
474.01	0.01\\
475.01	0.01\\
476.01	0.01\\
477.01	0.01\\
478.01	0.01\\
479.01	0.01\\
480.01	0.01\\
481.01	0.01\\
482.01	0.01\\
483.01	0.01\\
484.01	0.01\\
485.01	0.01\\
486.01	0.01\\
487.01	0.01\\
488.01	0.01\\
489.01	0.01\\
490.01	0.01\\
491.01	0.01\\
492.01	0.01\\
493.01	0.01\\
494.01	0.01\\
495.01	0.01\\
496.01	0.01\\
497.01	0.01\\
498.01	0.01\\
499.01	0.01\\
500.01	0.01\\
501.01	0.01\\
502.01	0.01\\
503.01	0.01\\
504.01	0.01\\
505.01	0.01\\
506.01	0.01\\
507.01	0.01\\
508.01	0.01\\
509.01	0.01\\
510.01	0.01\\
511.01	0.01\\
512.01	0.01\\
513.01	0.01\\
514.01	0.01\\
515.01	0.01\\
516.01	0.01\\
517.01	0.01\\
518.01	0.01\\
519.01	0.01\\
520.01	0.01\\
521.01	0.01\\
522.01	0.01\\
523.01	0.01\\
524.01	0.01\\
525.01	0.01\\
526.01	0.01\\
527.01	0.01\\
528.01	0.01\\
529.01	0.01\\
530.01	0.01\\
531.01	0.01\\
532.01	0.01\\
533.01	0.01\\
534.01	0.01\\
535.01	0.01\\
536.01	0.01\\
537.01	0.01\\
538.01	0.01\\
539.01	0.01\\
540.01	0.01\\
541.01	0.01\\
542.01	0.01\\
543.01	0.01\\
544.01	0.01\\
545.01	0.01\\
546.01	0.01\\
547.01	0.01\\
548.01	0.01\\
549.01	0.01\\
550.01	0.01\\
551.01	0.01\\
552.01	0.01\\
553.01	0.01\\
554.01	0.01\\
555.01	0.01\\
556.01	0.01\\
557.01	0.01\\
558.01	0.01\\
559.01	0.01\\
560.01	0.01\\
561.01	0.01\\
562.01	0.01\\
563.01	0.01\\
564.01	0.01\\
565.01	0.01\\
566.01	0.01\\
567.01	0.01\\
568.01	0.01\\
569.01	0.01\\
570.01	0.01\\
571.01	0.01\\
572.01	0.01\\
573.01	0.01\\
574.01	0.01\\
575.01	0.01\\
576.01	0.01\\
577.01	0.01\\
578.01	0.01\\
579.01	0.01\\
580.01	0.01\\
581.01	0.01\\
582.01	0.01\\
583.01	0.01\\
584.01	0.01\\
585.01	0.01\\
586.01	0.01\\
587.01	0.01\\
588.01	0.01\\
589.01	0.01\\
590.01	0.01\\
591.01	0.01\\
592.01	0.01\\
593.01	0.01\\
594.01	0.01\\
595.01	0.01\\
596.01	0.01\\
597.01	0.01\\
598.01	0.01\\
599.01	0.01\\
599.02	0.01\\
599.03	0.01\\
599.04	0.01\\
599.05	0.01\\
599.06	0.01\\
599.07	0.01\\
599.08	0.01\\
599.09	0.01\\
599.1	0.01\\
599.11	0.01\\
599.12	0.01\\
599.13	0.01\\
599.14	0.01\\
599.15	0.01\\
599.16	0.01\\
599.17	0.01\\
599.18	0.01\\
599.19	0.01\\
599.2	0.01\\
599.21	0.01\\
599.22	0.01\\
599.23	0.01\\
599.24	0.01\\
599.25	0.01\\
599.26	0.01\\
599.27	0.01\\
599.28	0.01\\
599.29	0.01\\
599.3	0.01\\
599.31	0.01\\
599.32	0.01\\
599.33	0.01\\
599.34	0.01\\
599.35	0.01\\
599.36	0.01\\
599.37	0.01\\
599.38	0.01\\
599.39	0.01\\
599.4	0.01\\
599.41	0.01\\
599.42	0.01\\
599.43	0.01\\
599.44	0.01\\
599.45	0.01\\
599.46	0.01\\
599.47	0.01\\
599.48	0.01\\
599.49	0.01\\
599.5	0.01\\
599.51	0.01\\
599.52	0.01\\
599.53	0.01\\
599.54	0.01\\
599.55	0.01\\
599.56	0.01\\
599.57	0.01\\
599.58	0.01\\
599.59	0.01\\
599.6	0.01\\
599.61	0.01\\
599.62	0.01\\
599.63	0.01\\
599.64	0.01\\
599.65	0.01\\
599.66	0.01\\
599.67	0.01\\
599.68	0.01\\
599.69	0.01\\
599.7	0.01\\
599.71	0.01\\
599.72	0.01\\
599.73	0.01\\
599.74	0.01\\
599.75	0.01\\
599.76	0.01\\
599.77	0.01\\
599.78	0.01\\
599.79	0.01\\
599.8	0.01\\
599.81	0.01\\
599.82	0.01\\
599.83	0.01\\
599.84	0.01\\
599.85	0.01\\
599.86	0.01\\
599.87	0.01\\
599.88	0.01\\
599.89	0.01\\
599.9	0.01\\
599.91	0.01\\
599.92	0.01\\
599.93	0.01\\
599.94	0.01\\
599.95	0.01\\
599.96	0.01\\
599.97	0.01\\
599.98	0.01\\
599.99	0.01\\
600	0.01\\
};
\addplot [color=red!75!mycolor17,solid,forget plot]
  table[row sep=crcr]{%
0.01	0.01\\
1.01	0.01\\
2.01	0.01\\
3.01	0.01\\
4.01	0.01\\
5.01	0.01\\
6.01	0.01\\
7.01	0.01\\
8.01	0.01\\
9.01	0.01\\
10.01	0.01\\
11.01	0.01\\
12.01	0.01\\
13.01	0.01\\
14.01	0.01\\
15.01	0.01\\
16.01	0.01\\
17.01	0.01\\
18.01	0.01\\
19.01	0.01\\
20.01	0.01\\
21.01	0.01\\
22.01	0.01\\
23.01	0.01\\
24.01	0.01\\
25.01	0.01\\
26.01	0.01\\
27.01	0.01\\
28.01	0.01\\
29.01	0.01\\
30.01	0.01\\
31.01	0.01\\
32.01	0.01\\
33.01	0.01\\
34.01	0.01\\
35.01	0.01\\
36.01	0.01\\
37.01	0.01\\
38.01	0.01\\
39.01	0.01\\
40.01	0.01\\
41.01	0.01\\
42.01	0.01\\
43.01	0.01\\
44.01	0.01\\
45.01	0.01\\
46.01	0.01\\
47.01	0.01\\
48.01	0.01\\
49.01	0.01\\
50.01	0.01\\
51.01	0.01\\
52.01	0.01\\
53.01	0.01\\
54.01	0.01\\
55.01	0.01\\
56.01	0.01\\
57.01	0.01\\
58.01	0.01\\
59.01	0.01\\
60.01	0.01\\
61.01	0.01\\
62.01	0.01\\
63.01	0.01\\
64.01	0.01\\
65.01	0.01\\
66.01	0.01\\
67.01	0.01\\
68.01	0.01\\
69.01	0.01\\
70.01	0.01\\
71.01	0.01\\
72.01	0.01\\
73.01	0.01\\
74.01	0.01\\
75.01	0.01\\
76.01	0.01\\
77.01	0.01\\
78.01	0.01\\
79.01	0.01\\
80.01	0.01\\
81.01	0.01\\
82.01	0.01\\
83.01	0.01\\
84.01	0.01\\
85.01	0.01\\
86.01	0.01\\
87.01	0.01\\
88.01	0.01\\
89.01	0.01\\
90.01	0.01\\
91.01	0.01\\
92.01	0.01\\
93.01	0.01\\
94.01	0.01\\
95.01	0.01\\
96.01	0.01\\
97.01	0.01\\
98.01	0.01\\
99.01	0.01\\
100.01	0.01\\
101.01	0.01\\
102.01	0.01\\
103.01	0.01\\
104.01	0.01\\
105.01	0.01\\
106.01	0.01\\
107.01	0.01\\
108.01	0.01\\
109.01	0.01\\
110.01	0.01\\
111.01	0.01\\
112.01	0.01\\
113.01	0.01\\
114.01	0.01\\
115.01	0.01\\
116.01	0.01\\
117.01	0.01\\
118.01	0.01\\
119.01	0.01\\
120.01	0.01\\
121.01	0.01\\
122.01	0.01\\
123.01	0.01\\
124.01	0.01\\
125.01	0.01\\
126.01	0.01\\
127.01	0.01\\
128.01	0.01\\
129.01	0.01\\
130.01	0.01\\
131.01	0.01\\
132.01	0.01\\
133.01	0.01\\
134.01	0.01\\
135.01	0.01\\
136.01	0.01\\
137.01	0.01\\
138.01	0.01\\
139.01	0.01\\
140.01	0.01\\
141.01	0.01\\
142.01	0.01\\
143.01	0.01\\
144.01	0.01\\
145.01	0.01\\
146.01	0.01\\
147.01	0.01\\
148.01	0.01\\
149.01	0.01\\
150.01	0.01\\
151.01	0.01\\
152.01	0.01\\
153.01	0.01\\
154.01	0.01\\
155.01	0.01\\
156.01	0.01\\
157.01	0.01\\
158.01	0.01\\
159.01	0.01\\
160.01	0.01\\
161.01	0.01\\
162.01	0.01\\
163.01	0.01\\
164.01	0.01\\
165.01	0.01\\
166.01	0.01\\
167.01	0.01\\
168.01	0.01\\
169.01	0.01\\
170.01	0.01\\
171.01	0.01\\
172.01	0.01\\
173.01	0.01\\
174.01	0.01\\
175.01	0.01\\
176.01	0.01\\
177.01	0.01\\
178.01	0.01\\
179.01	0.01\\
180.01	0.01\\
181.01	0.01\\
182.01	0.01\\
183.01	0.01\\
184.01	0.01\\
185.01	0.01\\
186.01	0.01\\
187.01	0.01\\
188.01	0.01\\
189.01	0.01\\
190.01	0.01\\
191.01	0.01\\
192.01	0.01\\
193.01	0.01\\
194.01	0.01\\
195.01	0.01\\
196.01	0.01\\
197.01	0.01\\
198.01	0.01\\
199.01	0.01\\
200.01	0.01\\
201.01	0.01\\
202.01	0.01\\
203.01	0.01\\
204.01	0.01\\
205.01	0.01\\
206.01	0.01\\
207.01	0.01\\
208.01	0.01\\
209.01	0.01\\
210.01	0.01\\
211.01	0.01\\
212.01	0.01\\
213.01	0.01\\
214.01	0.01\\
215.01	0.01\\
216.01	0.01\\
217.01	0.01\\
218.01	0.01\\
219.01	0.01\\
220.01	0.01\\
221.01	0.01\\
222.01	0.01\\
223.01	0.01\\
224.01	0.01\\
225.01	0.01\\
226.01	0.01\\
227.01	0.01\\
228.01	0.01\\
229.01	0.01\\
230.01	0.01\\
231.01	0.01\\
232.01	0.01\\
233.01	0.01\\
234.01	0.01\\
235.01	0.01\\
236.01	0.01\\
237.01	0.01\\
238.01	0.01\\
239.01	0.01\\
240.01	0.01\\
241.01	0.01\\
242.01	0.01\\
243.01	0.01\\
244.01	0.01\\
245.01	0.01\\
246.01	0.01\\
247.01	0.01\\
248.01	0.01\\
249.01	0.01\\
250.01	0.01\\
251.01	0.01\\
252.01	0.01\\
253.01	0.01\\
254.01	0.01\\
255.01	0.01\\
256.01	0.01\\
257.01	0.01\\
258.01	0.01\\
259.01	0.01\\
260.01	0.01\\
261.01	0.01\\
262.01	0.01\\
263.01	0.01\\
264.01	0.01\\
265.01	0.01\\
266.01	0.01\\
267.01	0.01\\
268.01	0.01\\
269.01	0.01\\
270.01	0.01\\
271.01	0.01\\
272.01	0.01\\
273.01	0.01\\
274.01	0.01\\
275.01	0.01\\
276.01	0.01\\
277.01	0.01\\
278.01	0.01\\
279.01	0.01\\
280.01	0.01\\
281.01	0.01\\
282.01	0.01\\
283.01	0.01\\
284.01	0.01\\
285.01	0.01\\
286.01	0.01\\
287.01	0.01\\
288.01	0.01\\
289.01	0.01\\
290.01	0.01\\
291.01	0.01\\
292.01	0.01\\
293.01	0.01\\
294.01	0.01\\
295.01	0.01\\
296.01	0.01\\
297.01	0.01\\
298.01	0.01\\
299.01	0.01\\
300.01	0.01\\
301.01	0.01\\
302.01	0.01\\
303.01	0.01\\
304.01	0.01\\
305.01	0.01\\
306.01	0.01\\
307.01	0.01\\
308.01	0.01\\
309.01	0.01\\
310.01	0.01\\
311.01	0.01\\
312.01	0.01\\
313.01	0.01\\
314.01	0.01\\
315.01	0.01\\
316.01	0.01\\
317.01	0.01\\
318.01	0.01\\
319.01	0.01\\
320.01	0.01\\
321.01	0.01\\
322.01	0.01\\
323.01	0.01\\
324.01	0.01\\
325.01	0.01\\
326.01	0.01\\
327.01	0.01\\
328.01	0.01\\
329.01	0.01\\
330.01	0.01\\
331.01	0.01\\
332.01	0.01\\
333.01	0.01\\
334.01	0.01\\
335.01	0.01\\
336.01	0.01\\
337.01	0.01\\
338.01	0.01\\
339.01	0.01\\
340.01	0.01\\
341.01	0.01\\
342.01	0.01\\
343.01	0.01\\
344.01	0.01\\
345.01	0.01\\
346.01	0.01\\
347.01	0.01\\
348.01	0.01\\
349.01	0.01\\
350.01	0.01\\
351.01	0.01\\
352.01	0.01\\
353.01	0.01\\
354.01	0.01\\
355.01	0.01\\
356.01	0.01\\
357.01	0.01\\
358.01	0.01\\
359.01	0.01\\
360.01	0.01\\
361.01	0.01\\
362.01	0.01\\
363.01	0.01\\
364.01	0.01\\
365.01	0.01\\
366.01	0.01\\
367.01	0.01\\
368.01	0.01\\
369.01	0.01\\
370.01	0.01\\
371.01	0.01\\
372.01	0.01\\
373.01	0.01\\
374.01	0.01\\
375.01	0.01\\
376.01	0.01\\
377.01	0.01\\
378.01	0.01\\
379.01	0.01\\
380.01	0.01\\
381.01	0.01\\
382.01	0.01\\
383.01	0.01\\
384.01	0.01\\
385.01	0.01\\
386.01	0.01\\
387.01	0.01\\
388.01	0.01\\
389.01	0.01\\
390.01	0.01\\
391.01	0.01\\
392.01	0.01\\
393.01	0.01\\
394.01	0.01\\
395.01	0.01\\
396.01	0.01\\
397.01	0.01\\
398.01	0.01\\
399.01	0.01\\
400.01	0.01\\
401.01	0.01\\
402.01	0.01\\
403.01	0.01\\
404.01	0.01\\
405.01	0.01\\
406.01	0.01\\
407.01	0.01\\
408.01	0.01\\
409.01	0.01\\
410.01	0.01\\
411.01	0.01\\
412.01	0.01\\
413.01	0.01\\
414.01	0.01\\
415.01	0.01\\
416.01	0.01\\
417.01	0.01\\
418.01	0.01\\
419.01	0.01\\
420.01	0.01\\
421.01	0.01\\
422.01	0.01\\
423.01	0.01\\
424.01	0.01\\
425.01	0.01\\
426.01	0.01\\
427.01	0.01\\
428.01	0.01\\
429.01	0.01\\
430.01	0.01\\
431.01	0.01\\
432.01	0.01\\
433.01	0.01\\
434.01	0.01\\
435.01	0.01\\
436.01	0.01\\
437.01	0.01\\
438.01	0.01\\
439.01	0.01\\
440.01	0.01\\
441.01	0.01\\
442.01	0.01\\
443.01	0.01\\
444.01	0.01\\
445.01	0.01\\
446.01	0.01\\
447.01	0.01\\
448.01	0.01\\
449.01	0.01\\
450.01	0.01\\
451.01	0.01\\
452.01	0.01\\
453.01	0.01\\
454.01	0.01\\
455.01	0.01\\
456.01	0.01\\
457.01	0.01\\
458.01	0.01\\
459.01	0.01\\
460.01	0.01\\
461.01	0.01\\
462.01	0.01\\
463.01	0.01\\
464.01	0.01\\
465.01	0.01\\
466.01	0.01\\
467.01	0.01\\
468.01	0.01\\
469.01	0.01\\
470.01	0.01\\
471.01	0.01\\
472.01	0.01\\
473.01	0.01\\
474.01	0.01\\
475.01	0.01\\
476.01	0.01\\
477.01	0.01\\
478.01	0.01\\
479.01	0.01\\
480.01	0.01\\
481.01	0.01\\
482.01	0.01\\
483.01	0.01\\
484.01	0.01\\
485.01	0.01\\
486.01	0.01\\
487.01	0.01\\
488.01	0.01\\
489.01	0.01\\
490.01	0.01\\
491.01	0.01\\
492.01	0.01\\
493.01	0.01\\
494.01	0.01\\
495.01	0.01\\
496.01	0.01\\
497.01	0.01\\
498.01	0.01\\
499.01	0.01\\
500.01	0.01\\
501.01	0.01\\
502.01	0.01\\
503.01	0.01\\
504.01	0.01\\
505.01	0.01\\
506.01	0.01\\
507.01	0.01\\
508.01	0.01\\
509.01	0.01\\
510.01	0.01\\
511.01	0.01\\
512.01	0.01\\
513.01	0.01\\
514.01	0.01\\
515.01	0.01\\
516.01	0.01\\
517.01	0.01\\
518.01	0.01\\
519.01	0.01\\
520.01	0.01\\
521.01	0.01\\
522.01	0.01\\
523.01	0.01\\
524.01	0.01\\
525.01	0.01\\
526.01	0.01\\
527.01	0.01\\
528.01	0.01\\
529.01	0.01\\
530.01	0.01\\
531.01	0.01\\
532.01	0.01\\
533.01	0.01\\
534.01	0.01\\
535.01	0.01\\
536.01	0.01\\
537.01	0.01\\
538.01	0.01\\
539.01	0.01\\
540.01	0.01\\
541.01	0.01\\
542.01	0.01\\
543.01	0.01\\
544.01	0.01\\
545.01	0.01\\
546.01	0.01\\
547.01	0.01\\
548.01	0.01\\
549.01	0.01\\
550.01	0.01\\
551.01	0.01\\
552.01	0.01\\
553.01	0.01\\
554.01	0.01\\
555.01	0.01\\
556.01	0.01\\
557.01	0.01\\
558.01	0.01\\
559.01	0.01\\
560.01	0.01\\
561.01	0.01\\
562.01	0.01\\
563.01	0.01\\
564.01	0.01\\
565.01	0.01\\
566.01	0.01\\
567.01	0.01\\
568.01	0.01\\
569.01	0.01\\
570.01	0.01\\
571.01	0.01\\
572.01	0.01\\
573.01	0.01\\
574.01	0.01\\
575.01	0.01\\
576.01	0.01\\
577.01	0.01\\
578.01	0.01\\
579.01	0.01\\
580.01	0.01\\
581.01	0.01\\
582.01	0.01\\
583.01	0.01\\
584.01	0.01\\
585.01	0.01\\
586.01	0.01\\
587.01	0.01\\
588.01	0.01\\
589.01	0.01\\
590.01	0.01\\
591.01	0.01\\
592.01	0.01\\
593.01	0.01\\
594.01	0.01\\
595.01	0.01\\
596.01	0.01\\
597.01	0.01\\
598.01	0.01\\
599.01	0.01\\
599.02	0.01\\
599.03	0.01\\
599.04	0.01\\
599.05	0.01\\
599.06	0.01\\
599.07	0.01\\
599.08	0.01\\
599.09	0.01\\
599.1	0.01\\
599.11	0.01\\
599.12	0.01\\
599.13	0.01\\
599.14	0.01\\
599.15	0.01\\
599.16	0.01\\
599.17	0.01\\
599.18	0.01\\
599.19	0.01\\
599.2	0.01\\
599.21	0.01\\
599.22	0.01\\
599.23	0.01\\
599.24	0.01\\
599.25	0.01\\
599.26	0.01\\
599.27	0.01\\
599.28	0.01\\
599.29	0.01\\
599.3	0.01\\
599.31	0.01\\
599.32	0.01\\
599.33	0.01\\
599.34	0.01\\
599.35	0.01\\
599.36	0.01\\
599.37	0.01\\
599.38	0.01\\
599.39	0.01\\
599.4	0.01\\
599.41	0.01\\
599.42	0.01\\
599.43	0.01\\
599.44	0.01\\
599.45	0.01\\
599.46	0.01\\
599.47	0.01\\
599.48	0.01\\
599.49	0.01\\
599.5	0.01\\
599.51	0.01\\
599.52	0.01\\
599.53	0.01\\
599.54	0.01\\
599.55	0.01\\
599.56	0.01\\
599.57	0.01\\
599.58	0.01\\
599.59	0.01\\
599.6	0.01\\
599.61	0.01\\
599.62	0.01\\
599.63	0.01\\
599.64	0.01\\
599.65	0.01\\
599.66	0.01\\
599.67	0.01\\
599.68	0.01\\
599.69	0.01\\
599.7	0.01\\
599.71	0.01\\
599.72	0.01\\
599.73	0.01\\
599.74	0.01\\
599.75	0.01\\
599.76	0.01\\
599.77	0.01\\
599.78	0.01\\
599.79	0.01\\
599.8	0.01\\
599.81	0.01\\
599.82	0.01\\
599.83	0.01\\
599.84	0.01\\
599.85	0.01\\
599.86	0.01\\
599.87	0.01\\
599.88	0.01\\
599.89	0.01\\
599.9	0.01\\
599.91	0.01\\
599.92	0.01\\
599.93	0.01\\
599.94	0.01\\
599.95	0.01\\
599.96	0.01\\
599.97	0.01\\
599.98	0.01\\
599.99	0.01\\
600	0.01\\
};
\addplot [color=red!80!mycolor19,solid,forget plot]
  table[row sep=crcr]{%
0.01	0.01\\
1.01	0.01\\
2.01	0.01\\
3.01	0.01\\
4.01	0.01\\
5.01	0.01\\
6.01	0.01\\
7.01	0.01\\
8.01	0.01\\
9.01	0.01\\
10.01	0.01\\
11.01	0.01\\
12.01	0.01\\
13.01	0.01\\
14.01	0.01\\
15.01	0.01\\
16.01	0.01\\
17.01	0.01\\
18.01	0.01\\
19.01	0.01\\
20.01	0.01\\
21.01	0.01\\
22.01	0.01\\
23.01	0.01\\
24.01	0.01\\
25.01	0.01\\
26.01	0.01\\
27.01	0.01\\
28.01	0.01\\
29.01	0.01\\
30.01	0.01\\
31.01	0.01\\
32.01	0.01\\
33.01	0.01\\
34.01	0.01\\
35.01	0.01\\
36.01	0.01\\
37.01	0.01\\
38.01	0.01\\
39.01	0.01\\
40.01	0.01\\
41.01	0.01\\
42.01	0.01\\
43.01	0.01\\
44.01	0.01\\
45.01	0.01\\
46.01	0.01\\
47.01	0.01\\
48.01	0.01\\
49.01	0.01\\
50.01	0.01\\
51.01	0.01\\
52.01	0.01\\
53.01	0.01\\
54.01	0.01\\
55.01	0.01\\
56.01	0.01\\
57.01	0.01\\
58.01	0.01\\
59.01	0.01\\
60.01	0.01\\
61.01	0.01\\
62.01	0.01\\
63.01	0.01\\
64.01	0.01\\
65.01	0.01\\
66.01	0.01\\
67.01	0.01\\
68.01	0.01\\
69.01	0.01\\
70.01	0.01\\
71.01	0.01\\
72.01	0.01\\
73.01	0.01\\
74.01	0.01\\
75.01	0.01\\
76.01	0.01\\
77.01	0.01\\
78.01	0.01\\
79.01	0.01\\
80.01	0.01\\
81.01	0.01\\
82.01	0.01\\
83.01	0.01\\
84.01	0.01\\
85.01	0.01\\
86.01	0.01\\
87.01	0.01\\
88.01	0.01\\
89.01	0.01\\
90.01	0.01\\
91.01	0.01\\
92.01	0.01\\
93.01	0.01\\
94.01	0.01\\
95.01	0.01\\
96.01	0.01\\
97.01	0.01\\
98.01	0.01\\
99.01	0.01\\
100.01	0.01\\
101.01	0.01\\
102.01	0.01\\
103.01	0.01\\
104.01	0.01\\
105.01	0.01\\
106.01	0.01\\
107.01	0.01\\
108.01	0.01\\
109.01	0.01\\
110.01	0.01\\
111.01	0.01\\
112.01	0.01\\
113.01	0.01\\
114.01	0.01\\
115.01	0.01\\
116.01	0.01\\
117.01	0.01\\
118.01	0.01\\
119.01	0.01\\
120.01	0.01\\
121.01	0.01\\
122.01	0.01\\
123.01	0.01\\
124.01	0.01\\
125.01	0.01\\
126.01	0.01\\
127.01	0.01\\
128.01	0.01\\
129.01	0.01\\
130.01	0.01\\
131.01	0.01\\
132.01	0.01\\
133.01	0.01\\
134.01	0.01\\
135.01	0.01\\
136.01	0.01\\
137.01	0.01\\
138.01	0.01\\
139.01	0.01\\
140.01	0.01\\
141.01	0.01\\
142.01	0.01\\
143.01	0.01\\
144.01	0.01\\
145.01	0.01\\
146.01	0.01\\
147.01	0.01\\
148.01	0.01\\
149.01	0.01\\
150.01	0.01\\
151.01	0.01\\
152.01	0.01\\
153.01	0.01\\
154.01	0.01\\
155.01	0.01\\
156.01	0.01\\
157.01	0.01\\
158.01	0.01\\
159.01	0.01\\
160.01	0.01\\
161.01	0.01\\
162.01	0.01\\
163.01	0.01\\
164.01	0.01\\
165.01	0.01\\
166.01	0.01\\
167.01	0.01\\
168.01	0.01\\
169.01	0.01\\
170.01	0.01\\
171.01	0.01\\
172.01	0.01\\
173.01	0.01\\
174.01	0.01\\
175.01	0.01\\
176.01	0.01\\
177.01	0.01\\
178.01	0.01\\
179.01	0.01\\
180.01	0.01\\
181.01	0.01\\
182.01	0.01\\
183.01	0.01\\
184.01	0.01\\
185.01	0.01\\
186.01	0.01\\
187.01	0.01\\
188.01	0.01\\
189.01	0.01\\
190.01	0.01\\
191.01	0.01\\
192.01	0.01\\
193.01	0.01\\
194.01	0.01\\
195.01	0.01\\
196.01	0.01\\
197.01	0.01\\
198.01	0.01\\
199.01	0.01\\
200.01	0.01\\
201.01	0.01\\
202.01	0.01\\
203.01	0.01\\
204.01	0.01\\
205.01	0.01\\
206.01	0.01\\
207.01	0.01\\
208.01	0.01\\
209.01	0.01\\
210.01	0.01\\
211.01	0.01\\
212.01	0.01\\
213.01	0.01\\
214.01	0.01\\
215.01	0.01\\
216.01	0.01\\
217.01	0.01\\
218.01	0.01\\
219.01	0.01\\
220.01	0.01\\
221.01	0.01\\
222.01	0.01\\
223.01	0.01\\
224.01	0.01\\
225.01	0.01\\
226.01	0.01\\
227.01	0.01\\
228.01	0.01\\
229.01	0.01\\
230.01	0.01\\
231.01	0.01\\
232.01	0.01\\
233.01	0.01\\
234.01	0.01\\
235.01	0.01\\
236.01	0.01\\
237.01	0.01\\
238.01	0.01\\
239.01	0.01\\
240.01	0.01\\
241.01	0.01\\
242.01	0.01\\
243.01	0.01\\
244.01	0.01\\
245.01	0.01\\
246.01	0.01\\
247.01	0.01\\
248.01	0.01\\
249.01	0.01\\
250.01	0.01\\
251.01	0.01\\
252.01	0.01\\
253.01	0.01\\
254.01	0.01\\
255.01	0.01\\
256.01	0.01\\
257.01	0.01\\
258.01	0.01\\
259.01	0.01\\
260.01	0.01\\
261.01	0.01\\
262.01	0.01\\
263.01	0.01\\
264.01	0.01\\
265.01	0.01\\
266.01	0.01\\
267.01	0.01\\
268.01	0.01\\
269.01	0.01\\
270.01	0.01\\
271.01	0.01\\
272.01	0.01\\
273.01	0.01\\
274.01	0.01\\
275.01	0.01\\
276.01	0.01\\
277.01	0.01\\
278.01	0.01\\
279.01	0.01\\
280.01	0.01\\
281.01	0.01\\
282.01	0.01\\
283.01	0.01\\
284.01	0.01\\
285.01	0.01\\
286.01	0.01\\
287.01	0.01\\
288.01	0.01\\
289.01	0.01\\
290.01	0.01\\
291.01	0.01\\
292.01	0.01\\
293.01	0.01\\
294.01	0.01\\
295.01	0.01\\
296.01	0.01\\
297.01	0.01\\
298.01	0.01\\
299.01	0.01\\
300.01	0.01\\
301.01	0.01\\
302.01	0.01\\
303.01	0.01\\
304.01	0.01\\
305.01	0.01\\
306.01	0.01\\
307.01	0.01\\
308.01	0.01\\
309.01	0.01\\
310.01	0.01\\
311.01	0.01\\
312.01	0.01\\
313.01	0.01\\
314.01	0.01\\
315.01	0.01\\
316.01	0.01\\
317.01	0.01\\
318.01	0.01\\
319.01	0.01\\
320.01	0.01\\
321.01	0.01\\
322.01	0.01\\
323.01	0.01\\
324.01	0.01\\
325.01	0.01\\
326.01	0.01\\
327.01	0.01\\
328.01	0.01\\
329.01	0.01\\
330.01	0.01\\
331.01	0.01\\
332.01	0.01\\
333.01	0.01\\
334.01	0.01\\
335.01	0.01\\
336.01	0.01\\
337.01	0.01\\
338.01	0.01\\
339.01	0.01\\
340.01	0.01\\
341.01	0.01\\
342.01	0.01\\
343.01	0.01\\
344.01	0.01\\
345.01	0.01\\
346.01	0.01\\
347.01	0.01\\
348.01	0.01\\
349.01	0.01\\
350.01	0.01\\
351.01	0.01\\
352.01	0.01\\
353.01	0.01\\
354.01	0.01\\
355.01	0.01\\
356.01	0.01\\
357.01	0.01\\
358.01	0.01\\
359.01	0.01\\
360.01	0.01\\
361.01	0.01\\
362.01	0.01\\
363.01	0.01\\
364.01	0.01\\
365.01	0.01\\
366.01	0.01\\
367.01	0.01\\
368.01	0.01\\
369.01	0.01\\
370.01	0.01\\
371.01	0.01\\
372.01	0.01\\
373.01	0.01\\
374.01	0.01\\
375.01	0.01\\
376.01	0.01\\
377.01	0.01\\
378.01	0.01\\
379.01	0.01\\
380.01	0.01\\
381.01	0.01\\
382.01	0.01\\
383.01	0.01\\
384.01	0.01\\
385.01	0.01\\
386.01	0.01\\
387.01	0.01\\
388.01	0.01\\
389.01	0.01\\
390.01	0.01\\
391.01	0.01\\
392.01	0.01\\
393.01	0.01\\
394.01	0.01\\
395.01	0.01\\
396.01	0.01\\
397.01	0.01\\
398.01	0.01\\
399.01	0.01\\
400.01	0.01\\
401.01	0.01\\
402.01	0.01\\
403.01	0.01\\
404.01	0.01\\
405.01	0.01\\
406.01	0.01\\
407.01	0.01\\
408.01	0.01\\
409.01	0.01\\
410.01	0.01\\
411.01	0.01\\
412.01	0.01\\
413.01	0.01\\
414.01	0.01\\
415.01	0.01\\
416.01	0.01\\
417.01	0.01\\
418.01	0.01\\
419.01	0.01\\
420.01	0.01\\
421.01	0.01\\
422.01	0.01\\
423.01	0.01\\
424.01	0.01\\
425.01	0.01\\
426.01	0.01\\
427.01	0.01\\
428.01	0.01\\
429.01	0.01\\
430.01	0.01\\
431.01	0.01\\
432.01	0.01\\
433.01	0.01\\
434.01	0.01\\
435.01	0.01\\
436.01	0.01\\
437.01	0.01\\
438.01	0.01\\
439.01	0.01\\
440.01	0.01\\
441.01	0.01\\
442.01	0.01\\
443.01	0.01\\
444.01	0.01\\
445.01	0.01\\
446.01	0.01\\
447.01	0.01\\
448.01	0.01\\
449.01	0.01\\
450.01	0.01\\
451.01	0.01\\
452.01	0.01\\
453.01	0.01\\
454.01	0.01\\
455.01	0.01\\
456.01	0.01\\
457.01	0.01\\
458.01	0.01\\
459.01	0.01\\
460.01	0.01\\
461.01	0.01\\
462.01	0.01\\
463.01	0.01\\
464.01	0.01\\
465.01	0.01\\
466.01	0.01\\
467.01	0.01\\
468.01	0.01\\
469.01	0.01\\
470.01	0.01\\
471.01	0.01\\
472.01	0.01\\
473.01	0.01\\
474.01	0.01\\
475.01	0.01\\
476.01	0.01\\
477.01	0.01\\
478.01	0.01\\
479.01	0.01\\
480.01	0.01\\
481.01	0.01\\
482.01	0.01\\
483.01	0.01\\
484.01	0.01\\
485.01	0.01\\
486.01	0.01\\
487.01	0.01\\
488.01	0.01\\
489.01	0.01\\
490.01	0.01\\
491.01	0.01\\
492.01	0.01\\
493.01	0.01\\
494.01	0.01\\
495.01	0.01\\
496.01	0.01\\
497.01	0.01\\
498.01	0.01\\
499.01	0.01\\
500.01	0.01\\
501.01	0.01\\
502.01	0.01\\
503.01	0.01\\
504.01	0.01\\
505.01	0.01\\
506.01	0.01\\
507.01	0.01\\
508.01	0.01\\
509.01	0.01\\
510.01	0.01\\
511.01	0.01\\
512.01	0.01\\
513.01	0.01\\
514.01	0.01\\
515.01	0.01\\
516.01	0.01\\
517.01	0.01\\
518.01	0.01\\
519.01	0.01\\
520.01	0.01\\
521.01	0.01\\
522.01	0.01\\
523.01	0.01\\
524.01	0.01\\
525.01	0.01\\
526.01	0.01\\
527.01	0.01\\
528.01	0.01\\
529.01	0.01\\
530.01	0.01\\
531.01	0.01\\
532.01	0.01\\
533.01	0.01\\
534.01	0.01\\
535.01	0.01\\
536.01	0.01\\
537.01	0.01\\
538.01	0.01\\
539.01	0.01\\
540.01	0.01\\
541.01	0.01\\
542.01	0.01\\
543.01	0.01\\
544.01	0.01\\
545.01	0.01\\
546.01	0.01\\
547.01	0.01\\
548.01	0.01\\
549.01	0.01\\
550.01	0.01\\
551.01	0.01\\
552.01	0.01\\
553.01	0.01\\
554.01	0.01\\
555.01	0.01\\
556.01	0.01\\
557.01	0.01\\
558.01	0.01\\
559.01	0.01\\
560.01	0.01\\
561.01	0.01\\
562.01	0.01\\
563.01	0.01\\
564.01	0.01\\
565.01	0.01\\
566.01	0.01\\
567.01	0.01\\
568.01	0.01\\
569.01	0.01\\
570.01	0.01\\
571.01	0.01\\
572.01	0.01\\
573.01	0.01\\
574.01	0.01\\
575.01	0.01\\
576.01	0.01\\
577.01	0.01\\
578.01	0.01\\
579.01	0.01\\
580.01	0.01\\
581.01	0.01\\
582.01	0.01\\
583.01	0.01\\
584.01	0.01\\
585.01	0.01\\
586.01	0.01\\
587.01	0.01\\
588.01	0.01\\
589.01	0.01\\
590.01	0.01\\
591.01	0.01\\
592.01	0.01\\
593.01	0.01\\
594.01	0.01\\
595.01	0.01\\
596.01	0.01\\
597.01	0.01\\
598.01	0.01\\
599.01	0.01\\
599.02	0.01\\
599.03	0.01\\
599.04	0.01\\
599.05	0.01\\
599.06	0.01\\
599.07	0.01\\
599.08	0.01\\
599.09	0.01\\
599.1	0.01\\
599.11	0.01\\
599.12	0.01\\
599.13	0.01\\
599.14	0.01\\
599.15	0.01\\
599.16	0.01\\
599.17	0.01\\
599.18	0.01\\
599.19	0.01\\
599.2	0.01\\
599.21	0.01\\
599.22	0.01\\
599.23	0.01\\
599.24	0.01\\
599.25	0.01\\
599.26	0.01\\
599.27	0.01\\
599.28	0.01\\
599.29	0.01\\
599.3	0.01\\
599.31	0.01\\
599.32	0.01\\
599.33	0.01\\
599.34	0.01\\
599.35	0.01\\
599.36	0.01\\
599.37	0.01\\
599.38	0.01\\
599.39	0.01\\
599.4	0.01\\
599.41	0.01\\
599.42	0.01\\
599.43	0.01\\
599.44	0.01\\
599.45	0.01\\
599.46	0.01\\
599.47	0.01\\
599.48	0.01\\
599.49	0.01\\
599.5	0.01\\
599.51	0.01\\
599.52	0.01\\
599.53	0.01\\
599.54	0.01\\
599.55	0.01\\
599.56	0.01\\
599.57	0.01\\
599.58	0.01\\
599.59	0.01\\
599.6	0.01\\
599.61	0.01\\
599.62	0.01\\
599.63	0.01\\
599.64	0.01\\
599.65	0.01\\
599.66	0.01\\
599.67	0.01\\
599.68	0.01\\
599.69	0.01\\
599.7	0.01\\
599.71	0.01\\
599.72	0.01\\
599.73	0.01\\
599.74	0.01\\
599.75	0.01\\
599.76	0.01\\
599.77	0.01\\
599.78	0.01\\
599.79	0.01\\
599.8	0.01\\
599.81	0.01\\
599.82	0.01\\
599.83	0.01\\
599.84	0.01\\
599.85	0.01\\
599.86	0.01\\
599.87	0.01\\
599.88	0.01\\
599.89	0.01\\
599.9	0.01\\
599.91	0.01\\
599.92	0.01\\
599.93	0.01\\
599.94	0.01\\
599.95	0.01\\
599.96	0.01\\
599.97	0.01\\
599.98	0.01\\
599.99	0.01\\
600	0.01\\
};
\addplot [color=red,solid,forget plot]
  table[row sep=crcr]{%
0.01	0.01\\
1.01	0.01\\
2.01	0.01\\
3.01	0.01\\
4.01	0.01\\
5.01	0.01\\
6.01	0.01\\
7.01	0.01\\
8.01	0.01\\
9.01	0.01\\
10.01	0.01\\
11.01	0.01\\
12.01	0.01\\
13.01	0.01\\
14.01	0.01\\
15.01	0.01\\
16.01	0.01\\
17.01	0.01\\
18.01	0.01\\
19.01	0.01\\
20.01	0.01\\
21.01	0.01\\
22.01	0.01\\
23.01	0.01\\
24.01	0.01\\
25.01	0.01\\
26.01	0.01\\
27.01	0.01\\
28.01	0.01\\
29.01	0.01\\
30.01	0.01\\
31.01	0.01\\
32.01	0.01\\
33.01	0.01\\
34.01	0.01\\
35.01	0.01\\
36.01	0.01\\
37.01	0.01\\
38.01	0.01\\
39.01	0.01\\
40.01	0.01\\
41.01	0.01\\
42.01	0.01\\
43.01	0.01\\
44.01	0.01\\
45.01	0.01\\
46.01	0.01\\
47.01	0.01\\
48.01	0.01\\
49.01	0.01\\
50.01	0.01\\
51.01	0.01\\
52.01	0.01\\
53.01	0.01\\
54.01	0.01\\
55.01	0.01\\
56.01	0.01\\
57.01	0.01\\
58.01	0.01\\
59.01	0.01\\
60.01	0.01\\
61.01	0.01\\
62.01	0.01\\
63.01	0.01\\
64.01	0.01\\
65.01	0.01\\
66.01	0.01\\
67.01	0.01\\
68.01	0.01\\
69.01	0.01\\
70.01	0.01\\
71.01	0.01\\
72.01	0.01\\
73.01	0.01\\
74.01	0.01\\
75.01	0.01\\
76.01	0.01\\
77.01	0.01\\
78.01	0.01\\
79.01	0.01\\
80.01	0.01\\
81.01	0.01\\
82.01	0.01\\
83.01	0.01\\
84.01	0.01\\
85.01	0.01\\
86.01	0.01\\
87.01	0.01\\
88.01	0.01\\
89.01	0.01\\
90.01	0.01\\
91.01	0.01\\
92.01	0.01\\
93.01	0.01\\
94.01	0.01\\
95.01	0.01\\
96.01	0.01\\
97.01	0.01\\
98.01	0.01\\
99.01	0.01\\
100.01	0.01\\
101.01	0.01\\
102.01	0.01\\
103.01	0.01\\
104.01	0.01\\
105.01	0.01\\
106.01	0.01\\
107.01	0.01\\
108.01	0.01\\
109.01	0.01\\
110.01	0.01\\
111.01	0.01\\
112.01	0.01\\
113.01	0.01\\
114.01	0.01\\
115.01	0.01\\
116.01	0.01\\
117.01	0.01\\
118.01	0.01\\
119.01	0.01\\
120.01	0.01\\
121.01	0.01\\
122.01	0.01\\
123.01	0.01\\
124.01	0.01\\
125.01	0.01\\
126.01	0.01\\
127.01	0.01\\
128.01	0.01\\
129.01	0.01\\
130.01	0.01\\
131.01	0.01\\
132.01	0.01\\
133.01	0.01\\
134.01	0.01\\
135.01	0.01\\
136.01	0.01\\
137.01	0.01\\
138.01	0.01\\
139.01	0.01\\
140.01	0.01\\
141.01	0.01\\
142.01	0.01\\
143.01	0.01\\
144.01	0.01\\
145.01	0.01\\
146.01	0.01\\
147.01	0.01\\
148.01	0.01\\
149.01	0.01\\
150.01	0.01\\
151.01	0.01\\
152.01	0.01\\
153.01	0.01\\
154.01	0.01\\
155.01	0.01\\
156.01	0.01\\
157.01	0.01\\
158.01	0.01\\
159.01	0.01\\
160.01	0.01\\
161.01	0.01\\
162.01	0.01\\
163.01	0.01\\
164.01	0.01\\
165.01	0.01\\
166.01	0.01\\
167.01	0.01\\
168.01	0.01\\
169.01	0.01\\
170.01	0.01\\
171.01	0.01\\
172.01	0.01\\
173.01	0.01\\
174.01	0.01\\
175.01	0.01\\
176.01	0.01\\
177.01	0.01\\
178.01	0.01\\
179.01	0.01\\
180.01	0.01\\
181.01	0.01\\
182.01	0.01\\
183.01	0.01\\
184.01	0.01\\
185.01	0.01\\
186.01	0.01\\
187.01	0.01\\
188.01	0.01\\
189.01	0.01\\
190.01	0.01\\
191.01	0.01\\
192.01	0.01\\
193.01	0.01\\
194.01	0.01\\
195.01	0.01\\
196.01	0.01\\
197.01	0.01\\
198.01	0.01\\
199.01	0.01\\
200.01	0.01\\
201.01	0.01\\
202.01	0.01\\
203.01	0.01\\
204.01	0.01\\
205.01	0.01\\
206.01	0.01\\
207.01	0.01\\
208.01	0.01\\
209.01	0.01\\
210.01	0.01\\
211.01	0.01\\
212.01	0.01\\
213.01	0.01\\
214.01	0.01\\
215.01	0.01\\
216.01	0.01\\
217.01	0.01\\
218.01	0.01\\
219.01	0.01\\
220.01	0.01\\
221.01	0.01\\
222.01	0.01\\
223.01	0.01\\
224.01	0.01\\
225.01	0.01\\
226.01	0.01\\
227.01	0.01\\
228.01	0.01\\
229.01	0.01\\
230.01	0.01\\
231.01	0.01\\
232.01	0.01\\
233.01	0.01\\
234.01	0.01\\
235.01	0.01\\
236.01	0.01\\
237.01	0.01\\
238.01	0.01\\
239.01	0.01\\
240.01	0.01\\
241.01	0.01\\
242.01	0.01\\
243.01	0.01\\
244.01	0.01\\
245.01	0.01\\
246.01	0.01\\
247.01	0.01\\
248.01	0.01\\
249.01	0.01\\
250.01	0.01\\
251.01	0.01\\
252.01	0.01\\
253.01	0.01\\
254.01	0.01\\
255.01	0.01\\
256.01	0.01\\
257.01	0.01\\
258.01	0.01\\
259.01	0.01\\
260.01	0.01\\
261.01	0.01\\
262.01	0.01\\
263.01	0.01\\
264.01	0.01\\
265.01	0.01\\
266.01	0.01\\
267.01	0.01\\
268.01	0.01\\
269.01	0.01\\
270.01	0.01\\
271.01	0.01\\
272.01	0.01\\
273.01	0.01\\
274.01	0.01\\
275.01	0.01\\
276.01	0.01\\
277.01	0.01\\
278.01	0.01\\
279.01	0.01\\
280.01	0.01\\
281.01	0.01\\
282.01	0.01\\
283.01	0.01\\
284.01	0.01\\
285.01	0.01\\
286.01	0.01\\
287.01	0.01\\
288.01	0.01\\
289.01	0.01\\
290.01	0.01\\
291.01	0.01\\
292.01	0.01\\
293.01	0.01\\
294.01	0.01\\
295.01	0.01\\
296.01	0.01\\
297.01	0.01\\
298.01	0.01\\
299.01	0.01\\
300.01	0.01\\
301.01	0.01\\
302.01	0.01\\
303.01	0.01\\
304.01	0.01\\
305.01	0.01\\
306.01	0.01\\
307.01	0.01\\
308.01	0.01\\
309.01	0.01\\
310.01	0.01\\
311.01	0.01\\
312.01	0.01\\
313.01	0.01\\
314.01	0.01\\
315.01	0.01\\
316.01	0.01\\
317.01	0.01\\
318.01	0.01\\
319.01	0.01\\
320.01	0.01\\
321.01	0.01\\
322.01	0.01\\
323.01	0.01\\
324.01	0.01\\
325.01	0.01\\
326.01	0.01\\
327.01	0.01\\
328.01	0.01\\
329.01	0.01\\
330.01	0.01\\
331.01	0.01\\
332.01	0.01\\
333.01	0.01\\
334.01	0.01\\
335.01	0.01\\
336.01	0.01\\
337.01	0.01\\
338.01	0.01\\
339.01	0.01\\
340.01	0.01\\
341.01	0.01\\
342.01	0.01\\
343.01	0.01\\
344.01	0.01\\
345.01	0.01\\
346.01	0.01\\
347.01	0.01\\
348.01	0.01\\
349.01	0.01\\
350.01	0.01\\
351.01	0.01\\
352.01	0.01\\
353.01	0.01\\
354.01	0.01\\
355.01	0.01\\
356.01	0.01\\
357.01	0.01\\
358.01	0.01\\
359.01	0.01\\
360.01	0.01\\
361.01	0.01\\
362.01	0.01\\
363.01	0.01\\
364.01	0.01\\
365.01	0.01\\
366.01	0.01\\
367.01	0.01\\
368.01	0.01\\
369.01	0.01\\
370.01	0.01\\
371.01	0.01\\
372.01	0.01\\
373.01	0.01\\
374.01	0.01\\
375.01	0.01\\
376.01	0.01\\
377.01	0.01\\
378.01	0.01\\
379.01	0.01\\
380.01	0.01\\
381.01	0.01\\
382.01	0.01\\
383.01	0.01\\
384.01	0.01\\
385.01	0.01\\
386.01	0.01\\
387.01	0.01\\
388.01	0.01\\
389.01	0.01\\
390.01	0.01\\
391.01	0.01\\
392.01	0.01\\
393.01	0.01\\
394.01	0.01\\
395.01	0.01\\
396.01	0.01\\
397.01	0.01\\
398.01	0.01\\
399.01	0.01\\
400.01	0.01\\
401.01	0.01\\
402.01	0.01\\
403.01	0.01\\
404.01	0.01\\
405.01	0.01\\
406.01	0.01\\
407.01	0.01\\
408.01	0.01\\
409.01	0.01\\
410.01	0.01\\
411.01	0.01\\
412.01	0.01\\
413.01	0.01\\
414.01	0.01\\
415.01	0.01\\
416.01	0.01\\
417.01	0.01\\
418.01	0.01\\
419.01	0.01\\
420.01	0.01\\
421.01	0.01\\
422.01	0.01\\
423.01	0.01\\
424.01	0.01\\
425.01	0.01\\
426.01	0.01\\
427.01	0.01\\
428.01	0.01\\
429.01	0.01\\
430.01	0.01\\
431.01	0.01\\
432.01	0.01\\
433.01	0.01\\
434.01	0.01\\
435.01	0.01\\
436.01	0.01\\
437.01	0.01\\
438.01	0.01\\
439.01	0.01\\
440.01	0.01\\
441.01	0.01\\
442.01	0.01\\
443.01	0.01\\
444.01	0.01\\
445.01	0.01\\
446.01	0.01\\
447.01	0.01\\
448.01	0.01\\
449.01	0.01\\
450.01	0.01\\
451.01	0.01\\
452.01	0.01\\
453.01	0.01\\
454.01	0.01\\
455.01	0.01\\
456.01	0.01\\
457.01	0.01\\
458.01	0.01\\
459.01	0.01\\
460.01	0.01\\
461.01	0.01\\
462.01	0.01\\
463.01	0.01\\
464.01	0.01\\
465.01	0.01\\
466.01	0.01\\
467.01	0.01\\
468.01	0.01\\
469.01	0.01\\
470.01	0.01\\
471.01	0.01\\
472.01	0.01\\
473.01	0.01\\
474.01	0.01\\
475.01	0.01\\
476.01	0.01\\
477.01	0.01\\
478.01	0.01\\
479.01	0.01\\
480.01	0.01\\
481.01	0.01\\
482.01	0.01\\
483.01	0.01\\
484.01	0.01\\
485.01	0.01\\
486.01	0.01\\
487.01	0.01\\
488.01	0.01\\
489.01	0.01\\
490.01	0.01\\
491.01	0.01\\
492.01	0.01\\
493.01	0.01\\
494.01	0.01\\
495.01	0.01\\
496.01	0.01\\
497.01	0.01\\
498.01	0.01\\
499.01	0.01\\
500.01	0.01\\
501.01	0.01\\
502.01	0.01\\
503.01	0.01\\
504.01	0.01\\
505.01	0.01\\
506.01	0.01\\
507.01	0.01\\
508.01	0.01\\
509.01	0.01\\
510.01	0.01\\
511.01	0.01\\
512.01	0.01\\
513.01	0.01\\
514.01	0.01\\
515.01	0.01\\
516.01	0.01\\
517.01	0.01\\
518.01	0.01\\
519.01	0.01\\
520.01	0.01\\
521.01	0.01\\
522.01	0.01\\
523.01	0.01\\
524.01	0.01\\
525.01	0.01\\
526.01	0.01\\
527.01	0.01\\
528.01	0.01\\
529.01	0.01\\
530.01	0.01\\
531.01	0.01\\
532.01	0.01\\
533.01	0.01\\
534.01	0.01\\
535.01	0.01\\
536.01	0.01\\
537.01	0.01\\
538.01	0.01\\
539.01	0.01\\
540.01	0.01\\
541.01	0.01\\
542.01	0.01\\
543.01	0.01\\
544.01	0.01\\
545.01	0.01\\
546.01	0.01\\
547.01	0.01\\
548.01	0.01\\
549.01	0.01\\
550.01	0.01\\
551.01	0.01\\
552.01	0.01\\
553.01	0.01\\
554.01	0.01\\
555.01	0.01\\
556.01	0.01\\
557.01	0.01\\
558.01	0.01\\
559.01	0.01\\
560.01	0.01\\
561.01	0.01\\
562.01	0.01\\
563.01	0.01\\
564.01	0.01\\
565.01	0.01\\
566.01	0.01\\
567.01	0.01\\
568.01	0.01\\
569.01	0.01\\
570.01	0.01\\
571.01	0.01\\
572.01	0.01\\
573.01	0.01\\
574.01	0.01\\
575.01	0.01\\
576.01	0.01\\
577.01	0.01\\
578.01	0.01\\
579.01	0.01\\
580.01	0.01\\
581.01	0.01\\
582.01	0.01\\
583.01	0.01\\
584.01	0.01\\
585.01	0.01\\
586.01	0.01\\
587.01	0.01\\
588.01	0.01\\
589.01	0.01\\
590.01	0.01\\
591.01	0.01\\
592.01	0.01\\
593.01	0.01\\
594.01	0.01\\
595.01	0.01\\
596.01	0.01\\
597.01	0.01\\
598.01	0.01\\
599.01	0.01\\
599.02	0.01\\
599.03	0.01\\
599.04	0.01\\
599.05	0.01\\
599.06	0.01\\
599.07	0.01\\
599.08	0.01\\
599.09	0.01\\
599.1	0.01\\
599.11	0.01\\
599.12	0.01\\
599.13	0.01\\
599.14	0.01\\
599.15	0.01\\
599.16	0.01\\
599.17	0.01\\
599.18	0.01\\
599.19	0.01\\
599.2	0.01\\
599.21	0.01\\
599.22	0.01\\
599.23	0.01\\
599.24	0.01\\
599.25	0.01\\
599.26	0.01\\
599.27	0.01\\
599.28	0.01\\
599.29	0.01\\
599.3	0.01\\
599.31	0.01\\
599.32	0.01\\
599.33	0.01\\
599.34	0.01\\
599.35	0.01\\
599.36	0.01\\
599.37	0.01\\
599.38	0.01\\
599.39	0.01\\
599.4	0.01\\
599.41	0.01\\
599.42	0.01\\
599.43	0.01\\
599.44	0.01\\
599.45	0.01\\
599.46	0.01\\
599.47	0.01\\
599.48	0.01\\
599.49	0.01\\
599.5	0.01\\
599.51	0.01\\
599.52	0.01\\
599.53	0.01\\
599.54	0.01\\
599.55	0.01\\
599.56	0.01\\
599.57	0.01\\
599.58	0.01\\
599.59	0.01\\
599.6	0.01\\
599.61	0.01\\
599.62	0.01\\
599.63	0.01\\
599.64	0.01\\
599.65	0.01\\
599.66	0.01\\
599.67	0.01\\
599.68	0.01\\
599.69	0.01\\
599.7	0.01\\
599.71	0.01\\
599.72	0.01\\
599.73	0.01\\
599.74	0.01\\
599.75	0.01\\
599.76	0.01\\
599.77	0.01\\
599.78	0.01\\
599.79	0.01\\
599.8	0.01\\
599.81	0.01\\
599.82	0.01\\
599.83	0.01\\
599.84	0.01\\
599.85	0.01\\
599.86	0.01\\
599.87	0.01\\
599.88	0.01\\
599.89	0.01\\
599.9	0.01\\
599.91	0.01\\
599.92	0.01\\
599.93	0.01\\
599.94	0.01\\
599.95	0.01\\
599.96	0.01\\
599.97	0.01\\
599.98	0.01\\
599.99	0.01\\
600	0.01\\
};
\addplot [color=mycolor20,solid,forget plot]
  table[row sep=crcr]{%
0.01	0.01\\
1.01	0.01\\
2.01	0.01\\
3.01	0.01\\
4.01	0.01\\
5.01	0.01\\
6.01	0.01\\
7.01	0.01\\
8.01	0.01\\
9.01	0.01\\
10.01	0.01\\
11.01	0.01\\
12.01	0.01\\
13.01	0.01\\
14.01	0.01\\
15.01	0.01\\
16.01	0.01\\
17.01	0.01\\
18.01	0.01\\
19.01	0.01\\
20.01	0.01\\
21.01	0.01\\
22.01	0.01\\
23.01	0.01\\
24.01	0.01\\
25.01	0.01\\
26.01	0.01\\
27.01	0.01\\
28.01	0.01\\
29.01	0.01\\
30.01	0.01\\
31.01	0.01\\
32.01	0.01\\
33.01	0.01\\
34.01	0.01\\
35.01	0.01\\
36.01	0.01\\
37.01	0.01\\
38.01	0.01\\
39.01	0.01\\
40.01	0.01\\
41.01	0.01\\
42.01	0.01\\
43.01	0.01\\
44.01	0.01\\
45.01	0.01\\
46.01	0.01\\
47.01	0.01\\
48.01	0.01\\
49.01	0.01\\
50.01	0.01\\
51.01	0.01\\
52.01	0.01\\
53.01	0.01\\
54.01	0.01\\
55.01	0.01\\
56.01	0.01\\
57.01	0.01\\
58.01	0.01\\
59.01	0.01\\
60.01	0.01\\
61.01	0.01\\
62.01	0.01\\
63.01	0.01\\
64.01	0.01\\
65.01	0.01\\
66.01	0.01\\
67.01	0.01\\
68.01	0.01\\
69.01	0.01\\
70.01	0.01\\
71.01	0.01\\
72.01	0.01\\
73.01	0.01\\
74.01	0.01\\
75.01	0.01\\
76.01	0.01\\
77.01	0.01\\
78.01	0.01\\
79.01	0.01\\
80.01	0.01\\
81.01	0.01\\
82.01	0.01\\
83.01	0.01\\
84.01	0.01\\
85.01	0.01\\
86.01	0.01\\
87.01	0.01\\
88.01	0.01\\
89.01	0.01\\
90.01	0.01\\
91.01	0.01\\
92.01	0.01\\
93.01	0.01\\
94.01	0.01\\
95.01	0.01\\
96.01	0.01\\
97.01	0.01\\
98.01	0.01\\
99.01	0.01\\
100.01	0.01\\
101.01	0.01\\
102.01	0.01\\
103.01	0.01\\
104.01	0.01\\
105.01	0.01\\
106.01	0.01\\
107.01	0.01\\
108.01	0.01\\
109.01	0.01\\
110.01	0.01\\
111.01	0.01\\
112.01	0.01\\
113.01	0.01\\
114.01	0.01\\
115.01	0.01\\
116.01	0.01\\
117.01	0.01\\
118.01	0.01\\
119.01	0.01\\
120.01	0.01\\
121.01	0.01\\
122.01	0.01\\
123.01	0.01\\
124.01	0.01\\
125.01	0.01\\
126.01	0.01\\
127.01	0.01\\
128.01	0.01\\
129.01	0.01\\
130.01	0.01\\
131.01	0.01\\
132.01	0.01\\
133.01	0.01\\
134.01	0.01\\
135.01	0.01\\
136.01	0.01\\
137.01	0.01\\
138.01	0.01\\
139.01	0.01\\
140.01	0.01\\
141.01	0.01\\
142.01	0.01\\
143.01	0.01\\
144.01	0.01\\
145.01	0.01\\
146.01	0.01\\
147.01	0.01\\
148.01	0.01\\
149.01	0.01\\
150.01	0.01\\
151.01	0.01\\
152.01	0.01\\
153.01	0.01\\
154.01	0.01\\
155.01	0.01\\
156.01	0.01\\
157.01	0.01\\
158.01	0.01\\
159.01	0.01\\
160.01	0.01\\
161.01	0.01\\
162.01	0.01\\
163.01	0.01\\
164.01	0.01\\
165.01	0.01\\
166.01	0.01\\
167.01	0.01\\
168.01	0.01\\
169.01	0.01\\
170.01	0.01\\
171.01	0.01\\
172.01	0.01\\
173.01	0.01\\
174.01	0.01\\
175.01	0.01\\
176.01	0.01\\
177.01	0.01\\
178.01	0.01\\
179.01	0.01\\
180.01	0.01\\
181.01	0.01\\
182.01	0.01\\
183.01	0.01\\
184.01	0.01\\
185.01	0.01\\
186.01	0.01\\
187.01	0.01\\
188.01	0.01\\
189.01	0.01\\
190.01	0.01\\
191.01	0.01\\
192.01	0.01\\
193.01	0.01\\
194.01	0.01\\
195.01	0.01\\
196.01	0.01\\
197.01	0.01\\
198.01	0.01\\
199.01	0.01\\
200.01	0.01\\
201.01	0.01\\
202.01	0.01\\
203.01	0.01\\
204.01	0.01\\
205.01	0.01\\
206.01	0.01\\
207.01	0.01\\
208.01	0.01\\
209.01	0.01\\
210.01	0.01\\
211.01	0.01\\
212.01	0.01\\
213.01	0.01\\
214.01	0.01\\
215.01	0.01\\
216.01	0.01\\
217.01	0.01\\
218.01	0.01\\
219.01	0.01\\
220.01	0.01\\
221.01	0.01\\
222.01	0.01\\
223.01	0.01\\
224.01	0.01\\
225.01	0.01\\
226.01	0.01\\
227.01	0.01\\
228.01	0.01\\
229.01	0.01\\
230.01	0.01\\
231.01	0.01\\
232.01	0.01\\
233.01	0.01\\
234.01	0.01\\
235.01	0.01\\
236.01	0.01\\
237.01	0.01\\
238.01	0.01\\
239.01	0.01\\
240.01	0.01\\
241.01	0.01\\
242.01	0.01\\
243.01	0.01\\
244.01	0.01\\
245.01	0.01\\
246.01	0.01\\
247.01	0.01\\
248.01	0.01\\
249.01	0.01\\
250.01	0.01\\
251.01	0.01\\
252.01	0.01\\
253.01	0.01\\
254.01	0.01\\
255.01	0.01\\
256.01	0.01\\
257.01	0.01\\
258.01	0.01\\
259.01	0.01\\
260.01	0.01\\
261.01	0.01\\
262.01	0.01\\
263.01	0.01\\
264.01	0.01\\
265.01	0.01\\
266.01	0.01\\
267.01	0.01\\
268.01	0.01\\
269.01	0.01\\
270.01	0.01\\
271.01	0.01\\
272.01	0.01\\
273.01	0.01\\
274.01	0.01\\
275.01	0.01\\
276.01	0.01\\
277.01	0.01\\
278.01	0.01\\
279.01	0.01\\
280.01	0.01\\
281.01	0.01\\
282.01	0.01\\
283.01	0.01\\
284.01	0.01\\
285.01	0.01\\
286.01	0.01\\
287.01	0.01\\
288.01	0.01\\
289.01	0.01\\
290.01	0.01\\
291.01	0.01\\
292.01	0.01\\
293.01	0.01\\
294.01	0.01\\
295.01	0.01\\
296.01	0.01\\
297.01	0.01\\
298.01	0.01\\
299.01	0.01\\
300.01	0.01\\
301.01	0.01\\
302.01	0.01\\
303.01	0.01\\
304.01	0.01\\
305.01	0.01\\
306.01	0.01\\
307.01	0.01\\
308.01	0.01\\
309.01	0.01\\
310.01	0.01\\
311.01	0.01\\
312.01	0.01\\
313.01	0.01\\
314.01	0.01\\
315.01	0.01\\
316.01	0.01\\
317.01	0.01\\
318.01	0.01\\
319.01	0.01\\
320.01	0.01\\
321.01	0.01\\
322.01	0.01\\
323.01	0.01\\
324.01	0.01\\
325.01	0.01\\
326.01	0.01\\
327.01	0.01\\
328.01	0.01\\
329.01	0.01\\
330.01	0.01\\
331.01	0.01\\
332.01	0.01\\
333.01	0.01\\
334.01	0.01\\
335.01	0.01\\
336.01	0.01\\
337.01	0.01\\
338.01	0.01\\
339.01	0.01\\
340.01	0.01\\
341.01	0.01\\
342.01	0.01\\
343.01	0.01\\
344.01	0.01\\
345.01	0.01\\
346.01	0.01\\
347.01	0.01\\
348.01	0.01\\
349.01	0.01\\
350.01	0.01\\
351.01	0.01\\
352.01	0.01\\
353.01	0.01\\
354.01	0.01\\
355.01	0.01\\
356.01	0.01\\
357.01	0.01\\
358.01	0.01\\
359.01	0.01\\
360.01	0.01\\
361.01	0.01\\
362.01	0.01\\
363.01	0.01\\
364.01	0.01\\
365.01	0.01\\
366.01	0.01\\
367.01	0.01\\
368.01	0.01\\
369.01	0.01\\
370.01	0.01\\
371.01	0.01\\
372.01	0.01\\
373.01	0.01\\
374.01	0.01\\
375.01	0.01\\
376.01	0.01\\
377.01	0.01\\
378.01	0.01\\
379.01	0.01\\
380.01	0.01\\
381.01	0.01\\
382.01	0.01\\
383.01	0.01\\
384.01	0.01\\
385.01	0.01\\
386.01	0.01\\
387.01	0.01\\
388.01	0.01\\
389.01	0.01\\
390.01	0.01\\
391.01	0.01\\
392.01	0.01\\
393.01	0.01\\
394.01	0.01\\
395.01	0.01\\
396.01	0.01\\
397.01	0.01\\
398.01	0.01\\
399.01	0.01\\
400.01	0.01\\
401.01	0.01\\
402.01	0.01\\
403.01	0.01\\
404.01	0.01\\
405.01	0.01\\
406.01	0.01\\
407.01	0.01\\
408.01	0.01\\
409.01	0.01\\
410.01	0.01\\
411.01	0.01\\
412.01	0.01\\
413.01	0.01\\
414.01	0.01\\
415.01	0.01\\
416.01	0.01\\
417.01	0.01\\
418.01	0.01\\
419.01	0.01\\
420.01	0.01\\
421.01	0.01\\
422.01	0.01\\
423.01	0.01\\
424.01	0.01\\
425.01	0.01\\
426.01	0.01\\
427.01	0.01\\
428.01	0.01\\
429.01	0.01\\
430.01	0.01\\
431.01	0.01\\
432.01	0.01\\
433.01	0.01\\
434.01	0.01\\
435.01	0.01\\
436.01	0.01\\
437.01	0.01\\
438.01	0.01\\
439.01	0.01\\
440.01	0.01\\
441.01	0.01\\
442.01	0.01\\
443.01	0.01\\
444.01	0.01\\
445.01	0.01\\
446.01	0.01\\
447.01	0.01\\
448.01	0.01\\
449.01	0.01\\
450.01	0.01\\
451.01	0.01\\
452.01	0.01\\
453.01	0.01\\
454.01	0.01\\
455.01	0.01\\
456.01	0.01\\
457.01	0.01\\
458.01	0.01\\
459.01	0.01\\
460.01	0.01\\
461.01	0.01\\
462.01	0.01\\
463.01	0.01\\
464.01	0.01\\
465.01	0.01\\
466.01	0.01\\
467.01	0.01\\
468.01	0.01\\
469.01	0.01\\
470.01	0.01\\
471.01	0.01\\
472.01	0.01\\
473.01	0.01\\
474.01	0.01\\
475.01	0.01\\
476.01	0.01\\
477.01	0.01\\
478.01	0.01\\
479.01	0.01\\
480.01	0.01\\
481.01	0.01\\
482.01	0.01\\
483.01	0.01\\
484.01	0.01\\
485.01	0.01\\
486.01	0.01\\
487.01	0.01\\
488.01	0.01\\
489.01	0.01\\
490.01	0.01\\
491.01	0.01\\
492.01	0.01\\
493.01	0.01\\
494.01	0.01\\
495.01	0.01\\
496.01	0.01\\
497.01	0.01\\
498.01	0.01\\
499.01	0.01\\
500.01	0.01\\
501.01	0.01\\
502.01	0.01\\
503.01	0.01\\
504.01	0.01\\
505.01	0.01\\
506.01	0.01\\
507.01	0.01\\
508.01	0.01\\
509.01	0.01\\
510.01	0.01\\
511.01	0.01\\
512.01	0.01\\
513.01	0.01\\
514.01	0.01\\
515.01	0.01\\
516.01	0.01\\
517.01	0.01\\
518.01	0.01\\
519.01	0.01\\
520.01	0.01\\
521.01	0.01\\
522.01	0.01\\
523.01	0.01\\
524.01	0.01\\
525.01	0.01\\
526.01	0.01\\
527.01	0.01\\
528.01	0.01\\
529.01	0.01\\
530.01	0.01\\
531.01	0.01\\
532.01	0.01\\
533.01	0.01\\
534.01	0.01\\
535.01	0.01\\
536.01	0.01\\
537.01	0.01\\
538.01	0.01\\
539.01	0.01\\
540.01	0.01\\
541.01	0.01\\
542.01	0.01\\
543.01	0.01\\
544.01	0.01\\
545.01	0.01\\
546.01	0.01\\
547.01	0.01\\
548.01	0.01\\
549.01	0.01\\
550.01	0.01\\
551.01	0.01\\
552.01	0.01\\
553.01	0.01\\
554.01	0.01\\
555.01	0.01\\
556.01	0.01\\
557.01	0.01\\
558.01	0.01\\
559.01	0.01\\
560.01	0.01\\
561.01	0.01\\
562.01	0.01\\
563.01	0.01\\
564.01	0.01\\
565.01	0.01\\
566.01	0.01\\
567.01	0.01\\
568.01	0.01\\
569.01	0.01\\
570.01	0.01\\
571.01	0.01\\
572.01	0.01\\
573.01	0.01\\
574.01	0.01\\
575.01	0.01\\
576.01	0.01\\
577.01	0.01\\
578.01	0.01\\
579.01	0.01\\
580.01	0.01\\
581.01	0.01\\
582.01	0.01\\
583.01	0.01\\
584.01	0.01\\
585.01	0.01\\
586.01	0.01\\
587.01	0.01\\
588.01	0.01\\
589.01	0.01\\
590.01	0.01\\
591.01	0.01\\
592.01	0.01\\
593.01	0.01\\
594.01	0.01\\
595.01	0.01\\
596.01	0.01\\
597.01	0.01\\
598.01	0.01\\
599.01	0.01\\
599.02	0.01\\
599.03	0.01\\
599.04	0.01\\
599.05	0.01\\
599.06	0.01\\
599.07	0.01\\
599.08	0.01\\
599.09	0.01\\
599.1	0.01\\
599.11	0.01\\
599.12	0.01\\
599.13	0.01\\
599.14	0.01\\
599.15	0.01\\
599.16	0.01\\
599.17	0.01\\
599.18	0.01\\
599.19	0.01\\
599.2	0.01\\
599.21	0.01\\
599.22	0.01\\
599.23	0.01\\
599.24	0.01\\
599.25	0.01\\
599.26	0.01\\
599.27	0.01\\
599.28	0.01\\
599.29	0.01\\
599.3	0.01\\
599.31	0.01\\
599.32	0.01\\
599.33	0.01\\
599.34	0.01\\
599.35	0.01\\
599.36	0.01\\
599.37	0.01\\
599.38	0.01\\
599.39	0.01\\
599.4	0.01\\
599.41	0.01\\
599.42	0.01\\
599.43	0.01\\
599.44	0.01\\
599.45	0.01\\
599.46	0.01\\
599.47	0.01\\
599.48	0.01\\
599.49	0.01\\
599.5	0.01\\
599.51	0.01\\
599.52	0.01\\
599.53	0.01\\
599.54	0.01\\
599.55	0.01\\
599.56	0.01\\
599.57	0.01\\
599.58	0.01\\
599.59	0.01\\
599.6	0.01\\
599.61	0.01\\
599.62	0.01\\
599.63	0.01\\
599.64	0.01\\
599.65	0.01\\
599.66	0.01\\
599.67	0.01\\
599.68	0.01\\
599.69	0.01\\
599.7	0.01\\
599.71	0.01\\
599.72	0.01\\
599.73	0.01\\
599.74	0.01\\
599.75	0.01\\
599.76	0.01\\
599.77	0.01\\
599.78	0.01\\
599.79	0.01\\
599.8	0.01\\
599.81	0.01\\
599.82	0.01\\
599.83	0.01\\
599.84	0.01\\
599.85	0.01\\
599.86	0.01\\
599.87	0.01\\
599.88	0.01\\
599.89	0.01\\
599.9	0.01\\
599.91	0.01\\
599.92	0.01\\
599.93	0.01\\
599.94	0.01\\
599.95	0.01\\
599.96	0.01\\
599.97	0.01\\
599.98	0.01\\
599.99	0.01\\
600	0.01\\
};
\addplot [color=mycolor21,solid,forget plot]
  table[row sep=crcr]{%
0.01	0.01\\
1.01	0.01\\
2.01	0.01\\
3.01	0.01\\
4.01	0.01\\
5.01	0.01\\
6.01	0.01\\
7.01	0.01\\
8.01	0.01\\
9.01	0.01\\
10.01	0.01\\
11.01	0.01\\
12.01	0.01\\
13.01	0.01\\
14.01	0.01\\
15.01	0.01\\
16.01	0.01\\
17.01	0.01\\
18.01	0.01\\
19.01	0.01\\
20.01	0.01\\
21.01	0.01\\
22.01	0.01\\
23.01	0.01\\
24.01	0.01\\
25.01	0.01\\
26.01	0.01\\
27.01	0.01\\
28.01	0.01\\
29.01	0.01\\
30.01	0.01\\
31.01	0.01\\
32.01	0.01\\
33.01	0.01\\
34.01	0.01\\
35.01	0.01\\
36.01	0.01\\
37.01	0.01\\
38.01	0.01\\
39.01	0.01\\
40.01	0.01\\
41.01	0.01\\
42.01	0.01\\
43.01	0.01\\
44.01	0.01\\
45.01	0.01\\
46.01	0.01\\
47.01	0.01\\
48.01	0.01\\
49.01	0.01\\
50.01	0.01\\
51.01	0.01\\
52.01	0.01\\
53.01	0.01\\
54.01	0.01\\
55.01	0.01\\
56.01	0.01\\
57.01	0.01\\
58.01	0.01\\
59.01	0.01\\
60.01	0.01\\
61.01	0.01\\
62.01	0.01\\
63.01	0.01\\
64.01	0.01\\
65.01	0.01\\
66.01	0.01\\
67.01	0.01\\
68.01	0.01\\
69.01	0.01\\
70.01	0.01\\
71.01	0.01\\
72.01	0.01\\
73.01	0.01\\
74.01	0.01\\
75.01	0.01\\
76.01	0.01\\
77.01	0.01\\
78.01	0.01\\
79.01	0.01\\
80.01	0.01\\
81.01	0.01\\
82.01	0.01\\
83.01	0.01\\
84.01	0.01\\
85.01	0.01\\
86.01	0.01\\
87.01	0.01\\
88.01	0.01\\
89.01	0.01\\
90.01	0.01\\
91.01	0.01\\
92.01	0.01\\
93.01	0.01\\
94.01	0.01\\
95.01	0.01\\
96.01	0.01\\
97.01	0.01\\
98.01	0.01\\
99.01	0.01\\
100.01	0.01\\
101.01	0.01\\
102.01	0.01\\
103.01	0.01\\
104.01	0.01\\
105.01	0.01\\
106.01	0.01\\
107.01	0.01\\
108.01	0.01\\
109.01	0.01\\
110.01	0.01\\
111.01	0.01\\
112.01	0.01\\
113.01	0.01\\
114.01	0.01\\
115.01	0.01\\
116.01	0.01\\
117.01	0.01\\
118.01	0.01\\
119.01	0.01\\
120.01	0.01\\
121.01	0.01\\
122.01	0.01\\
123.01	0.01\\
124.01	0.01\\
125.01	0.01\\
126.01	0.01\\
127.01	0.01\\
128.01	0.01\\
129.01	0.01\\
130.01	0.01\\
131.01	0.01\\
132.01	0.01\\
133.01	0.01\\
134.01	0.01\\
135.01	0.01\\
136.01	0.01\\
137.01	0.01\\
138.01	0.01\\
139.01	0.01\\
140.01	0.01\\
141.01	0.01\\
142.01	0.01\\
143.01	0.01\\
144.01	0.01\\
145.01	0.01\\
146.01	0.01\\
147.01	0.01\\
148.01	0.01\\
149.01	0.01\\
150.01	0.01\\
151.01	0.01\\
152.01	0.01\\
153.01	0.01\\
154.01	0.01\\
155.01	0.01\\
156.01	0.01\\
157.01	0.01\\
158.01	0.01\\
159.01	0.01\\
160.01	0.01\\
161.01	0.01\\
162.01	0.01\\
163.01	0.01\\
164.01	0.01\\
165.01	0.01\\
166.01	0.01\\
167.01	0.01\\
168.01	0.01\\
169.01	0.01\\
170.01	0.01\\
171.01	0.01\\
172.01	0.01\\
173.01	0.01\\
174.01	0.01\\
175.01	0.01\\
176.01	0.01\\
177.01	0.01\\
178.01	0.01\\
179.01	0.01\\
180.01	0.01\\
181.01	0.01\\
182.01	0.01\\
183.01	0.01\\
184.01	0.01\\
185.01	0.01\\
186.01	0.01\\
187.01	0.01\\
188.01	0.01\\
189.01	0.01\\
190.01	0.01\\
191.01	0.01\\
192.01	0.01\\
193.01	0.01\\
194.01	0.01\\
195.01	0.01\\
196.01	0.01\\
197.01	0.01\\
198.01	0.01\\
199.01	0.01\\
200.01	0.01\\
201.01	0.01\\
202.01	0.01\\
203.01	0.01\\
204.01	0.01\\
205.01	0.01\\
206.01	0.01\\
207.01	0.01\\
208.01	0.01\\
209.01	0.01\\
210.01	0.01\\
211.01	0.01\\
212.01	0.01\\
213.01	0.01\\
214.01	0.01\\
215.01	0.01\\
216.01	0.01\\
217.01	0.01\\
218.01	0.01\\
219.01	0.01\\
220.01	0.01\\
221.01	0.01\\
222.01	0.01\\
223.01	0.01\\
224.01	0.01\\
225.01	0.01\\
226.01	0.01\\
227.01	0.01\\
228.01	0.01\\
229.01	0.01\\
230.01	0.01\\
231.01	0.01\\
232.01	0.01\\
233.01	0.01\\
234.01	0.01\\
235.01	0.01\\
236.01	0.01\\
237.01	0.01\\
238.01	0.01\\
239.01	0.01\\
240.01	0.01\\
241.01	0.01\\
242.01	0.01\\
243.01	0.01\\
244.01	0.01\\
245.01	0.01\\
246.01	0.01\\
247.01	0.01\\
248.01	0.01\\
249.01	0.01\\
250.01	0.01\\
251.01	0.01\\
252.01	0.01\\
253.01	0.01\\
254.01	0.01\\
255.01	0.01\\
256.01	0.01\\
257.01	0.01\\
258.01	0.01\\
259.01	0.01\\
260.01	0.01\\
261.01	0.01\\
262.01	0.01\\
263.01	0.01\\
264.01	0.01\\
265.01	0.01\\
266.01	0.01\\
267.01	0.01\\
268.01	0.01\\
269.01	0.01\\
270.01	0.01\\
271.01	0.01\\
272.01	0.01\\
273.01	0.01\\
274.01	0.01\\
275.01	0.01\\
276.01	0.01\\
277.01	0.01\\
278.01	0.01\\
279.01	0.01\\
280.01	0.01\\
281.01	0.01\\
282.01	0.01\\
283.01	0.01\\
284.01	0.01\\
285.01	0.01\\
286.01	0.01\\
287.01	0.01\\
288.01	0.01\\
289.01	0.01\\
290.01	0.01\\
291.01	0.01\\
292.01	0.01\\
293.01	0.01\\
294.01	0.01\\
295.01	0.01\\
296.01	0.01\\
297.01	0.01\\
298.01	0.01\\
299.01	0.01\\
300.01	0.01\\
301.01	0.01\\
302.01	0.01\\
303.01	0.01\\
304.01	0.01\\
305.01	0.01\\
306.01	0.01\\
307.01	0.01\\
308.01	0.01\\
309.01	0.01\\
310.01	0.01\\
311.01	0.01\\
312.01	0.01\\
313.01	0.01\\
314.01	0.01\\
315.01	0.01\\
316.01	0.01\\
317.01	0.01\\
318.01	0.01\\
319.01	0.01\\
320.01	0.01\\
321.01	0.01\\
322.01	0.01\\
323.01	0.01\\
324.01	0.01\\
325.01	0.01\\
326.01	0.01\\
327.01	0.01\\
328.01	0.01\\
329.01	0.01\\
330.01	0.01\\
331.01	0.01\\
332.01	0.01\\
333.01	0.01\\
334.01	0.01\\
335.01	0.01\\
336.01	0.01\\
337.01	0.01\\
338.01	0.01\\
339.01	0.01\\
340.01	0.01\\
341.01	0.01\\
342.01	0.01\\
343.01	0.01\\
344.01	0.01\\
345.01	0.01\\
346.01	0.01\\
347.01	0.01\\
348.01	0.01\\
349.01	0.01\\
350.01	0.01\\
351.01	0.01\\
352.01	0.01\\
353.01	0.01\\
354.01	0.01\\
355.01	0.01\\
356.01	0.01\\
357.01	0.01\\
358.01	0.01\\
359.01	0.01\\
360.01	0.01\\
361.01	0.01\\
362.01	0.01\\
363.01	0.01\\
364.01	0.01\\
365.01	0.01\\
366.01	0.01\\
367.01	0.01\\
368.01	0.01\\
369.01	0.01\\
370.01	0.01\\
371.01	0.01\\
372.01	0.01\\
373.01	0.01\\
374.01	0.01\\
375.01	0.01\\
376.01	0.01\\
377.01	0.01\\
378.01	0.01\\
379.01	0.01\\
380.01	0.01\\
381.01	0.01\\
382.01	0.01\\
383.01	0.01\\
384.01	0.01\\
385.01	0.01\\
386.01	0.01\\
387.01	0.01\\
388.01	0.01\\
389.01	0.01\\
390.01	0.01\\
391.01	0.01\\
392.01	0.01\\
393.01	0.01\\
394.01	0.01\\
395.01	0.01\\
396.01	0.01\\
397.01	0.01\\
398.01	0.01\\
399.01	0.01\\
400.01	0.01\\
401.01	0.01\\
402.01	0.01\\
403.01	0.01\\
404.01	0.01\\
405.01	0.01\\
406.01	0.01\\
407.01	0.01\\
408.01	0.01\\
409.01	0.01\\
410.01	0.01\\
411.01	0.01\\
412.01	0.01\\
413.01	0.01\\
414.01	0.01\\
415.01	0.01\\
416.01	0.01\\
417.01	0.01\\
418.01	0.01\\
419.01	0.01\\
420.01	0.01\\
421.01	0.01\\
422.01	0.01\\
423.01	0.01\\
424.01	0.01\\
425.01	0.01\\
426.01	0.01\\
427.01	0.01\\
428.01	0.01\\
429.01	0.01\\
430.01	0.01\\
431.01	0.01\\
432.01	0.01\\
433.01	0.01\\
434.01	0.01\\
435.01	0.01\\
436.01	0.01\\
437.01	0.01\\
438.01	0.01\\
439.01	0.01\\
440.01	0.01\\
441.01	0.01\\
442.01	0.01\\
443.01	0.01\\
444.01	0.01\\
445.01	0.01\\
446.01	0.01\\
447.01	0.01\\
448.01	0.01\\
449.01	0.01\\
450.01	0.01\\
451.01	0.01\\
452.01	0.01\\
453.01	0.01\\
454.01	0.01\\
455.01	0.01\\
456.01	0.01\\
457.01	0.01\\
458.01	0.01\\
459.01	0.01\\
460.01	0.01\\
461.01	0.01\\
462.01	0.01\\
463.01	0.01\\
464.01	0.01\\
465.01	0.01\\
466.01	0.01\\
467.01	0.01\\
468.01	0.01\\
469.01	0.01\\
470.01	0.01\\
471.01	0.01\\
472.01	0.01\\
473.01	0.01\\
474.01	0.01\\
475.01	0.01\\
476.01	0.01\\
477.01	0.01\\
478.01	0.01\\
479.01	0.01\\
480.01	0.01\\
481.01	0.01\\
482.01	0.01\\
483.01	0.01\\
484.01	0.01\\
485.01	0.01\\
486.01	0.01\\
487.01	0.01\\
488.01	0.01\\
489.01	0.01\\
490.01	0.01\\
491.01	0.01\\
492.01	0.01\\
493.01	0.01\\
494.01	0.01\\
495.01	0.01\\
496.01	0.01\\
497.01	0.01\\
498.01	0.01\\
499.01	0.01\\
500.01	0.01\\
501.01	0.01\\
502.01	0.01\\
503.01	0.01\\
504.01	0.01\\
505.01	0.01\\
506.01	0.01\\
507.01	0.01\\
508.01	0.01\\
509.01	0.01\\
510.01	0.01\\
511.01	0.01\\
512.01	0.01\\
513.01	0.01\\
514.01	0.01\\
515.01	0.01\\
516.01	0.01\\
517.01	0.01\\
518.01	0.01\\
519.01	0.01\\
520.01	0.01\\
521.01	0.01\\
522.01	0.01\\
523.01	0.01\\
524.01	0.01\\
525.01	0.01\\
526.01	0.01\\
527.01	0.01\\
528.01	0.01\\
529.01	0.01\\
530.01	0.01\\
531.01	0.01\\
532.01	0.01\\
533.01	0.01\\
534.01	0.01\\
535.01	0.01\\
536.01	0.01\\
537.01	0.01\\
538.01	0.01\\
539.01	0.01\\
540.01	0.01\\
541.01	0.01\\
542.01	0.01\\
543.01	0.01\\
544.01	0.01\\
545.01	0.01\\
546.01	0.01\\
547.01	0.01\\
548.01	0.01\\
549.01	0.01\\
550.01	0.01\\
551.01	0.01\\
552.01	0.01\\
553.01	0.01\\
554.01	0.01\\
555.01	0.01\\
556.01	0.01\\
557.01	0.01\\
558.01	0.01\\
559.01	0.01\\
560.01	0.01\\
561.01	0.01\\
562.01	0.01\\
563.01	0.01\\
564.01	0.01\\
565.01	0.01\\
566.01	0.01\\
567.01	0.01\\
568.01	0.01\\
569.01	0.01\\
570.01	0.01\\
571.01	0.01\\
572.01	0.01\\
573.01	0.01\\
574.01	0.01\\
575.01	0.01\\
576.01	0.01\\
577.01	0.01\\
578.01	0.01\\
579.01	0.01\\
580.01	0.01\\
581.01	0.01\\
582.01	0.01\\
583.01	0.01\\
584.01	0.01\\
585.01	0.01\\
586.01	0.01\\
587.01	0.01\\
588.01	0.01\\
589.01	0.01\\
590.01	0.01\\
591.01	0.01\\
592.01	0.01\\
593.01	0.01\\
594.01	0.01\\
595.01	0.01\\
596.01	0.01\\
597.01	0.01\\
598.01	0.01\\
599.01	0.01\\
599.02	0.01\\
599.03	0.01\\
599.04	0.01\\
599.05	0.01\\
599.06	0.01\\
599.07	0.01\\
599.08	0.01\\
599.09	0.01\\
599.1	0.01\\
599.11	0.01\\
599.12	0.01\\
599.13	0.01\\
599.14	0.01\\
599.15	0.01\\
599.16	0.01\\
599.17	0.01\\
599.18	0.01\\
599.19	0.01\\
599.2	0.01\\
599.21	0.01\\
599.22	0.01\\
599.23	0.01\\
599.24	0.01\\
599.25	0.01\\
599.26	0.01\\
599.27	0.01\\
599.28	0.01\\
599.29	0.01\\
599.3	0.01\\
599.31	0.01\\
599.32	0.01\\
599.33	0.01\\
599.34	0.01\\
599.35	0.01\\
599.36	0.01\\
599.37	0.01\\
599.38	0.01\\
599.39	0.01\\
599.4	0.01\\
599.41	0.01\\
599.42	0.01\\
599.43	0.01\\
599.44	0.01\\
599.45	0.01\\
599.46	0.01\\
599.47	0.01\\
599.48	0.01\\
599.49	0.01\\
599.5	0.01\\
599.51	0.01\\
599.52	0.01\\
599.53	0.01\\
599.54	0.01\\
599.55	0.01\\
599.56	0.01\\
599.57	0.01\\
599.58	0.01\\
599.59	0.01\\
599.6	0.01\\
599.61	0.01\\
599.62	0.01\\
599.63	0.01\\
599.64	0.01\\
599.65	0.01\\
599.66	0.01\\
599.67	0.01\\
599.68	0.01\\
599.69	0.01\\
599.7	0.01\\
599.71	0.01\\
599.72	0.01\\
599.73	0.01\\
599.74	0.01\\
599.75	0.01\\
599.76	0.01\\
599.77	0.01\\
599.78	0.01\\
599.79	0.01\\
599.8	0.01\\
599.81	0.01\\
599.82	0.01\\
599.83	0.01\\
599.84	0.01\\
599.85	0.01\\
599.86	0.01\\
599.87	0.01\\
599.88	0.01\\
599.89	0.01\\
599.9	0.01\\
599.91	0.01\\
599.92	0.01\\
599.93	0.01\\
599.94	0.01\\
599.95	0.01\\
599.96	0.01\\
599.97	0.01\\
599.98	0.01\\
599.99	0.01\\
600	0.01\\
};
\addplot [color=black!20!mycolor21,solid,forget plot]
  table[row sep=crcr]{%
0.01	0.01\\
1.01	0.01\\
2.01	0.01\\
3.01	0.01\\
4.01	0.01\\
5.01	0.01\\
6.01	0.01\\
7.01	0.01\\
8.01	0.01\\
9.01	0.01\\
10.01	0.01\\
11.01	0.01\\
12.01	0.01\\
13.01	0.01\\
14.01	0.01\\
15.01	0.01\\
16.01	0.01\\
17.01	0.01\\
18.01	0.01\\
19.01	0.01\\
20.01	0.01\\
21.01	0.01\\
22.01	0.01\\
23.01	0.01\\
24.01	0.01\\
25.01	0.01\\
26.01	0.01\\
27.01	0.01\\
28.01	0.01\\
29.01	0.01\\
30.01	0.01\\
31.01	0.01\\
32.01	0.01\\
33.01	0.01\\
34.01	0.01\\
35.01	0.01\\
36.01	0.01\\
37.01	0.01\\
38.01	0.01\\
39.01	0.01\\
40.01	0.01\\
41.01	0.01\\
42.01	0.01\\
43.01	0.01\\
44.01	0.01\\
45.01	0.01\\
46.01	0.01\\
47.01	0.01\\
48.01	0.01\\
49.01	0.01\\
50.01	0.01\\
51.01	0.01\\
52.01	0.01\\
53.01	0.01\\
54.01	0.01\\
55.01	0.01\\
56.01	0.01\\
57.01	0.01\\
58.01	0.01\\
59.01	0.01\\
60.01	0.01\\
61.01	0.01\\
62.01	0.01\\
63.01	0.01\\
64.01	0.01\\
65.01	0.01\\
66.01	0.01\\
67.01	0.01\\
68.01	0.01\\
69.01	0.01\\
70.01	0.01\\
71.01	0.01\\
72.01	0.01\\
73.01	0.01\\
74.01	0.01\\
75.01	0.01\\
76.01	0.01\\
77.01	0.01\\
78.01	0.01\\
79.01	0.01\\
80.01	0.01\\
81.01	0.01\\
82.01	0.01\\
83.01	0.01\\
84.01	0.01\\
85.01	0.01\\
86.01	0.01\\
87.01	0.01\\
88.01	0.01\\
89.01	0.01\\
90.01	0.01\\
91.01	0.01\\
92.01	0.01\\
93.01	0.01\\
94.01	0.01\\
95.01	0.01\\
96.01	0.01\\
97.01	0.01\\
98.01	0.01\\
99.01	0.01\\
100.01	0.01\\
101.01	0.01\\
102.01	0.01\\
103.01	0.01\\
104.01	0.01\\
105.01	0.01\\
106.01	0.01\\
107.01	0.01\\
108.01	0.01\\
109.01	0.01\\
110.01	0.01\\
111.01	0.01\\
112.01	0.01\\
113.01	0.01\\
114.01	0.01\\
115.01	0.01\\
116.01	0.01\\
117.01	0.01\\
118.01	0.01\\
119.01	0.01\\
120.01	0.01\\
121.01	0.01\\
122.01	0.01\\
123.01	0.01\\
124.01	0.01\\
125.01	0.01\\
126.01	0.01\\
127.01	0.01\\
128.01	0.01\\
129.01	0.01\\
130.01	0.01\\
131.01	0.01\\
132.01	0.01\\
133.01	0.01\\
134.01	0.01\\
135.01	0.01\\
136.01	0.01\\
137.01	0.01\\
138.01	0.01\\
139.01	0.01\\
140.01	0.01\\
141.01	0.01\\
142.01	0.01\\
143.01	0.01\\
144.01	0.01\\
145.01	0.01\\
146.01	0.01\\
147.01	0.01\\
148.01	0.01\\
149.01	0.01\\
150.01	0.01\\
151.01	0.01\\
152.01	0.01\\
153.01	0.01\\
154.01	0.01\\
155.01	0.01\\
156.01	0.01\\
157.01	0.01\\
158.01	0.01\\
159.01	0.01\\
160.01	0.01\\
161.01	0.01\\
162.01	0.01\\
163.01	0.01\\
164.01	0.01\\
165.01	0.01\\
166.01	0.01\\
167.01	0.01\\
168.01	0.01\\
169.01	0.01\\
170.01	0.01\\
171.01	0.01\\
172.01	0.01\\
173.01	0.01\\
174.01	0.01\\
175.01	0.01\\
176.01	0.01\\
177.01	0.01\\
178.01	0.01\\
179.01	0.01\\
180.01	0.01\\
181.01	0.01\\
182.01	0.01\\
183.01	0.01\\
184.01	0.01\\
185.01	0.01\\
186.01	0.01\\
187.01	0.01\\
188.01	0.01\\
189.01	0.01\\
190.01	0.01\\
191.01	0.01\\
192.01	0.01\\
193.01	0.01\\
194.01	0.01\\
195.01	0.01\\
196.01	0.01\\
197.01	0.01\\
198.01	0.01\\
199.01	0.01\\
200.01	0.01\\
201.01	0.01\\
202.01	0.01\\
203.01	0.01\\
204.01	0.01\\
205.01	0.01\\
206.01	0.01\\
207.01	0.01\\
208.01	0.01\\
209.01	0.01\\
210.01	0.01\\
211.01	0.01\\
212.01	0.01\\
213.01	0.01\\
214.01	0.01\\
215.01	0.01\\
216.01	0.01\\
217.01	0.01\\
218.01	0.01\\
219.01	0.01\\
220.01	0.01\\
221.01	0.01\\
222.01	0.01\\
223.01	0.01\\
224.01	0.01\\
225.01	0.01\\
226.01	0.01\\
227.01	0.01\\
228.01	0.01\\
229.01	0.01\\
230.01	0.01\\
231.01	0.01\\
232.01	0.01\\
233.01	0.01\\
234.01	0.01\\
235.01	0.01\\
236.01	0.01\\
237.01	0.01\\
238.01	0.01\\
239.01	0.01\\
240.01	0.01\\
241.01	0.01\\
242.01	0.01\\
243.01	0.01\\
244.01	0.01\\
245.01	0.01\\
246.01	0.01\\
247.01	0.01\\
248.01	0.01\\
249.01	0.01\\
250.01	0.01\\
251.01	0.01\\
252.01	0.01\\
253.01	0.01\\
254.01	0.01\\
255.01	0.01\\
256.01	0.01\\
257.01	0.01\\
258.01	0.01\\
259.01	0.01\\
260.01	0.01\\
261.01	0.01\\
262.01	0.01\\
263.01	0.01\\
264.01	0.01\\
265.01	0.01\\
266.01	0.01\\
267.01	0.01\\
268.01	0.01\\
269.01	0.01\\
270.01	0.01\\
271.01	0.01\\
272.01	0.01\\
273.01	0.01\\
274.01	0.01\\
275.01	0.01\\
276.01	0.01\\
277.01	0.01\\
278.01	0.01\\
279.01	0.01\\
280.01	0.01\\
281.01	0.01\\
282.01	0.01\\
283.01	0.01\\
284.01	0.01\\
285.01	0.01\\
286.01	0.01\\
287.01	0.01\\
288.01	0.01\\
289.01	0.01\\
290.01	0.01\\
291.01	0.01\\
292.01	0.01\\
293.01	0.01\\
294.01	0.01\\
295.01	0.01\\
296.01	0.01\\
297.01	0.01\\
298.01	0.01\\
299.01	0.01\\
300.01	0.01\\
301.01	0.01\\
302.01	0.01\\
303.01	0.01\\
304.01	0.01\\
305.01	0.01\\
306.01	0.01\\
307.01	0.01\\
308.01	0.01\\
309.01	0.01\\
310.01	0.01\\
311.01	0.01\\
312.01	0.01\\
313.01	0.01\\
314.01	0.01\\
315.01	0.01\\
316.01	0.01\\
317.01	0.01\\
318.01	0.01\\
319.01	0.01\\
320.01	0.01\\
321.01	0.01\\
322.01	0.01\\
323.01	0.01\\
324.01	0.01\\
325.01	0.01\\
326.01	0.01\\
327.01	0.01\\
328.01	0.01\\
329.01	0.01\\
330.01	0.01\\
331.01	0.01\\
332.01	0.01\\
333.01	0.01\\
334.01	0.01\\
335.01	0.01\\
336.01	0.01\\
337.01	0.01\\
338.01	0.01\\
339.01	0.01\\
340.01	0.01\\
341.01	0.01\\
342.01	0.01\\
343.01	0.01\\
344.01	0.01\\
345.01	0.01\\
346.01	0.01\\
347.01	0.01\\
348.01	0.01\\
349.01	0.01\\
350.01	0.01\\
351.01	0.01\\
352.01	0.01\\
353.01	0.01\\
354.01	0.01\\
355.01	0.01\\
356.01	0.01\\
357.01	0.01\\
358.01	0.01\\
359.01	0.01\\
360.01	0.01\\
361.01	0.01\\
362.01	0.01\\
363.01	0.01\\
364.01	0.01\\
365.01	0.01\\
366.01	0.01\\
367.01	0.01\\
368.01	0.01\\
369.01	0.01\\
370.01	0.01\\
371.01	0.01\\
372.01	0.01\\
373.01	0.01\\
374.01	0.01\\
375.01	0.01\\
376.01	0.01\\
377.01	0.01\\
378.01	0.01\\
379.01	0.01\\
380.01	0.01\\
381.01	0.01\\
382.01	0.01\\
383.01	0.01\\
384.01	0.01\\
385.01	0.01\\
386.01	0.01\\
387.01	0.01\\
388.01	0.01\\
389.01	0.01\\
390.01	0.01\\
391.01	0.01\\
392.01	0.01\\
393.01	0.01\\
394.01	0.01\\
395.01	0.01\\
396.01	0.01\\
397.01	0.01\\
398.01	0.01\\
399.01	0.01\\
400.01	0.01\\
401.01	0.01\\
402.01	0.01\\
403.01	0.01\\
404.01	0.01\\
405.01	0.01\\
406.01	0.01\\
407.01	0.01\\
408.01	0.01\\
409.01	0.01\\
410.01	0.01\\
411.01	0.01\\
412.01	0.01\\
413.01	0.01\\
414.01	0.01\\
415.01	0.01\\
416.01	0.01\\
417.01	0.01\\
418.01	0.01\\
419.01	0.01\\
420.01	0.01\\
421.01	0.01\\
422.01	0.01\\
423.01	0.01\\
424.01	0.01\\
425.01	0.01\\
426.01	0.01\\
427.01	0.01\\
428.01	0.01\\
429.01	0.01\\
430.01	0.01\\
431.01	0.01\\
432.01	0.01\\
433.01	0.01\\
434.01	0.01\\
435.01	0.01\\
436.01	0.01\\
437.01	0.01\\
438.01	0.01\\
439.01	0.01\\
440.01	0.01\\
441.01	0.01\\
442.01	0.01\\
443.01	0.01\\
444.01	0.01\\
445.01	0.01\\
446.01	0.01\\
447.01	0.01\\
448.01	0.01\\
449.01	0.01\\
450.01	0.01\\
451.01	0.01\\
452.01	0.01\\
453.01	0.01\\
454.01	0.01\\
455.01	0.01\\
456.01	0.01\\
457.01	0.01\\
458.01	0.01\\
459.01	0.01\\
460.01	0.01\\
461.01	0.01\\
462.01	0.01\\
463.01	0.01\\
464.01	0.01\\
465.01	0.01\\
466.01	0.01\\
467.01	0.01\\
468.01	0.01\\
469.01	0.01\\
470.01	0.01\\
471.01	0.01\\
472.01	0.01\\
473.01	0.01\\
474.01	0.01\\
475.01	0.01\\
476.01	0.01\\
477.01	0.01\\
478.01	0.01\\
479.01	0.01\\
480.01	0.01\\
481.01	0.01\\
482.01	0.01\\
483.01	0.01\\
484.01	0.01\\
485.01	0.01\\
486.01	0.01\\
487.01	0.01\\
488.01	0.01\\
489.01	0.01\\
490.01	0.01\\
491.01	0.01\\
492.01	0.01\\
493.01	0.01\\
494.01	0.01\\
495.01	0.01\\
496.01	0.01\\
497.01	0.01\\
498.01	0.01\\
499.01	0.01\\
500.01	0.01\\
501.01	0.01\\
502.01	0.01\\
503.01	0.01\\
504.01	0.01\\
505.01	0.01\\
506.01	0.01\\
507.01	0.01\\
508.01	0.01\\
509.01	0.01\\
510.01	0.01\\
511.01	0.01\\
512.01	0.01\\
513.01	0.01\\
514.01	0.01\\
515.01	0.01\\
516.01	0.01\\
517.01	0.01\\
518.01	0.01\\
519.01	0.01\\
520.01	0.01\\
521.01	0.01\\
522.01	0.01\\
523.01	0.01\\
524.01	0.01\\
525.01	0.01\\
526.01	0.01\\
527.01	0.01\\
528.01	0.01\\
529.01	0.01\\
530.01	0.01\\
531.01	0.01\\
532.01	0.01\\
533.01	0.01\\
534.01	0.01\\
535.01	0.01\\
536.01	0.01\\
537.01	0.01\\
538.01	0.01\\
539.01	0.01\\
540.01	0.01\\
541.01	0.01\\
542.01	0.01\\
543.01	0.01\\
544.01	0.01\\
545.01	0.01\\
546.01	0.01\\
547.01	0.01\\
548.01	0.01\\
549.01	0.01\\
550.01	0.01\\
551.01	0.01\\
552.01	0.01\\
553.01	0.01\\
554.01	0.01\\
555.01	0.01\\
556.01	0.01\\
557.01	0.01\\
558.01	0.01\\
559.01	0.01\\
560.01	0.01\\
561.01	0.01\\
562.01	0.01\\
563.01	0.01\\
564.01	0.01\\
565.01	0.01\\
566.01	0.01\\
567.01	0.01\\
568.01	0.01\\
569.01	0.01\\
570.01	0.01\\
571.01	0.01\\
572.01	0.01\\
573.01	0.01\\
574.01	0.01\\
575.01	0.01\\
576.01	0.01\\
577.01	0.01\\
578.01	0.01\\
579.01	0.01\\
580.01	0.01\\
581.01	0.01\\
582.01	0.01\\
583.01	0.01\\
584.01	0.01\\
585.01	0.01\\
586.01	0.01\\
587.01	0.01\\
588.01	0.01\\
589.01	0.01\\
590.01	0.01\\
591.01	0.01\\
592.01	0.01\\
593.01	0.01\\
594.01	0.01\\
595.01	0.01\\
596.01	0.01\\
597.01	0.01\\
598.01	0.01\\
599.01	0.01\\
599.02	0.01\\
599.03	0.01\\
599.04	0.01\\
599.05	0.01\\
599.06	0.01\\
599.07	0.01\\
599.08	0.01\\
599.09	0.01\\
599.1	0.01\\
599.11	0.01\\
599.12	0.01\\
599.13	0.01\\
599.14	0.01\\
599.15	0.01\\
599.16	0.01\\
599.17	0.01\\
599.18	0.01\\
599.19	0.01\\
599.2	0.01\\
599.21	0.01\\
599.22	0.01\\
599.23	0.01\\
599.24	0.01\\
599.25	0.01\\
599.26	0.01\\
599.27	0.01\\
599.28	0.01\\
599.29	0.01\\
599.3	0.01\\
599.31	0.01\\
599.32	0.01\\
599.33	0.01\\
599.34	0.01\\
599.35	0.01\\
599.36	0.01\\
599.37	0.01\\
599.38	0.01\\
599.39	0.01\\
599.4	0.01\\
599.41	0.01\\
599.42	0.01\\
599.43	0.01\\
599.44	0.01\\
599.45	0.01\\
599.46	0.01\\
599.47	0.01\\
599.48	0.01\\
599.49	0.01\\
599.5	0.01\\
599.51	0.01\\
599.52	0.01\\
599.53	0.01\\
599.54	0.01\\
599.55	0.01\\
599.56	0.01\\
599.57	0.01\\
599.58	0.01\\
599.59	0.01\\
599.6	0.01\\
599.61	0.01\\
599.62	0.01\\
599.63	0.01\\
599.64	0.01\\
599.65	0.01\\
599.66	0.01\\
599.67	0.01\\
599.68	0.01\\
599.69	0.01\\
599.7	0.01\\
599.71	0.01\\
599.72	0.01\\
599.73	0.01\\
599.74	0.01\\
599.75	0.01\\
599.76	0.01\\
599.77	0.01\\
599.78	0.01\\
599.79	0.01\\
599.8	0.01\\
599.81	0.01\\
599.82	0.01\\
599.83	0.01\\
599.84	0.01\\
599.85	0.01\\
599.86	0.01\\
599.87	0.01\\
599.88	0.01\\
599.89	0.01\\
599.9	0.01\\
599.91	0.01\\
599.92	0.01\\
599.93	0.01\\
599.94	0.01\\
599.95	0.01\\
599.96	0.01\\
599.97	0.01\\
599.98	0.01\\
599.99	0.01\\
600	0.01\\
};
\addplot [color=black!50!mycolor20,solid,forget plot]
  table[row sep=crcr]{%
0.01	0.01\\
1.01	0.01\\
2.01	0.01\\
3.01	0.01\\
4.01	0.01\\
5.01	0.01\\
6.01	0.01\\
7.01	0.01\\
8.01	0.01\\
9.01	0.01\\
10.01	0.01\\
11.01	0.01\\
12.01	0.01\\
13.01	0.01\\
14.01	0.01\\
15.01	0.01\\
16.01	0.01\\
17.01	0.01\\
18.01	0.01\\
19.01	0.01\\
20.01	0.01\\
21.01	0.01\\
22.01	0.01\\
23.01	0.01\\
24.01	0.01\\
25.01	0.01\\
26.01	0.01\\
27.01	0.01\\
28.01	0.01\\
29.01	0.01\\
30.01	0.01\\
31.01	0.01\\
32.01	0.01\\
33.01	0.01\\
34.01	0.01\\
35.01	0.01\\
36.01	0.01\\
37.01	0.01\\
38.01	0.01\\
39.01	0.01\\
40.01	0.01\\
41.01	0.01\\
42.01	0.01\\
43.01	0.01\\
44.01	0.01\\
45.01	0.01\\
46.01	0.01\\
47.01	0.01\\
48.01	0.01\\
49.01	0.01\\
50.01	0.01\\
51.01	0.01\\
52.01	0.01\\
53.01	0.01\\
54.01	0.01\\
55.01	0.01\\
56.01	0.01\\
57.01	0.01\\
58.01	0.01\\
59.01	0.01\\
60.01	0.01\\
61.01	0.01\\
62.01	0.01\\
63.01	0.01\\
64.01	0.01\\
65.01	0.01\\
66.01	0.01\\
67.01	0.01\\
68.01	0.01\\
69.01	0.01\\
70.01	0.01\\
71.01	0.01\\
72.01	0.01\\
73.01	0.01\\
74.01	0.01\\
75.01	0.01\\
76.01	0.01\\
77.01	0.01\\
78.01	0.01\\
79.01	0.01\\
80.01	0.01\\
81.01	0.01\\
82.01	0.01\\
83.01	0.01\\
84.01	0.01\\
85.01	0.01\\
86.01	0.01\\
87.01	0.01\\
88.01	0.01\\
89.01	0.01\\
90.01	0.01\\
91.01	0.01\\
92.01	0.01\\
93.01	0.01\\
94.01	0.01\\
95.01	0.01\\
96.01	0.01\\
97.01	0.01\\
98.01	0.01\\
99.01	0.01\\
100.01	0.01\\
101.01	0.01\\
102.01	0.01\\
103.01	0.01\\
104.01	0.01\\
105.01	0.01\\
106.01	0.01\\
107.01	0.01\\
108.01	0.01\\
109.01	0.01\\
110.01	0.01\\
111.01	0.01\\
112.01	0.01\\
113.01	0.01\\
114.01	0.01\\
115.01	0.01\\
116.01	0.01\\
117.01	0.01\\
118.01	0.01\\
119.01	0.01\\
120.01	0.01\\
121.01	0.01\\
122.01	0.01\\
123.01	0.01\\
124.01	0.01\\
125.01	0.01\\
126.01	0.01\\
127.01	0.01\\
128.01	0.01\\
129.01	0.01\\
130.01	0.01\\
131.01	0.01\\
132.01	0.01\\
133.01	0.01\\
134.01	0.01\\
135.01	0.01\\
136.01	0.01\\
137.01	0.01\\
138.01	0.01\\
139.01	0.01\\
140.01	0.01\\
141.01	0.01\\
142.01	0.01\\
143.01	0.01\\
144.01	0.01\\
145.01	0.01\\
146.01	0.01\\
147.01	0.01\\
148.01	0.01\\
149.01	0.01\\
150.01	0.01\\
151.01	0.01\\
152.01	0.01\\
153.01	0.01\\
154.01	0.01\\
155.01	0.01\\
156.01	0.01\\
157.01	0.01\\
158.01	0.01\\
159.01	0.01\\
160.01	0.01\\
161.01	0.01\\
162.01	0.01\\
163.01	0.01\\
164.01	0.01\\
165.01	0.01\\
166.01	0.01\\
167.01	0.01\\
168.01	0.01\\
169.01	0.01\\
170.01	0.01\\
171.01	0.01\\
172.01	0.01\\
173.01	0.01\\
174.01	0.01\\
175.01	0.01\\
176.01	0.01\\
177.01	0.01\\
178.01	0.01\\
179.01	0.01\\
180.01	0.01\\
181.01	0.01\\
182.01	0.01\\
183.01	0.01\\
184.01	0.01\\
185.01	0.01\\
186.01	0.01\\
187.01	0.01\\
188.01	0.01\\
189.01	0.01\\
190.01	0.01\\
191.01	0.01\\
192.01	0.01\\
193.01	0.01\\
194.01	0.01\\
195.01	0.01\\
196.01	0.01\\
197.01	0.01\\
198.01	0.01\\
199.01	0.01\\
200.01	0.01\\
201.01	0.01\\
202.01	0.01\\
203.01	0.01\\
204.01	0.01\\
205.01	0.01\\
206.01	0.01\\
207.01	0.01\\
208.01	0.01\\
209.01	0.01\\
210.01	0.01\\
211.01	0.01\\
212.01	0.01\\
213.01	0.01\\
214.01	0.01\\
215.01	0.01\\
216.01	0.01\\
217.01	0.01\\
218.01	0.01\\
219.01	0.01\\
220.01	0.01\\
221.01	0.01\\
222.01	0.01\\
223.01	0.01\\
224.01	0.01\\
225.01	0.01\\
226.01	0.01\\
227.01	0.01\\
228.01	0.01\\
229.01	0.01\\
230.01	0.01\\
231.01	0.01\\
232.01	0.01\\
233.01	0.01\\
234.01	0.01\\
235.01	0.01\\
236.01	0.01\\
237.01	0.01\\
238.01	0.01\\
239.01	0.01\\
240.01	0.01\\
241.01	0.01\\
242.01	0.01\\
243.01	0.01\\
244.01	0.01\\
245.01	0.01\\
246.01	0.01\\
247.01	0.01\\
248.01	0.01\\
249.01	0.01\\
250.01	0.01\\
251.01	0.01\\
252.01	0.01\\
253.01	0.01\\
254.01	0.01\\
255.01	0.01\\
256.01	0.01\\
257.01	0.01\\
258.01	0.01\\
259.01	0.01\\
260.01	0.01\\
261.01	0.01\\
262.01	0.01\\
263.01	0.01\\
264.01	0.01\\
265.01	0.01\\
266.01	0.01\\
267.01	0.01\\
268.01	0.01\\
269.01	0.01\\
270.01	0.01\\
271.01	0.01\\
272.01	0.01\\
273.01	0.01\\
274.01	0.01\\
275.01	0.01\\
276.01	0.01\\
277.01	0.01\\
278.01	0.01\\
279.01	0.01\\
280.01	0.01\\
281.01	0.01\\
282.01	0.01\\
283.01	0.01\\
284.01	0.01\\
285.01	0.01\\
286.01	0.01\\
287.01	0.01\\
288.01	0.01\\
289.01	0.01\\
290.01	0.01\\
291.01	0.01\\
292.01	0.01\\
293.01	0.01\\
294.01	0.01\\
295.01	0.01\\
296.01	0.01\\
297.01	0.01\\
298.01	0.01\\
299.01	0.01\\
300.01	0.01\\
301.01	0.01\\
302.01	0.01\\
303.01	0.01\\
304.01	0.01\\
305.01	0.01\\
306.01	0.01\\
307.01	0.01\\
308.01	0.01\\
309.01	0.01\\
310.01	0.01\\
311.01	0.01\\
312.01	0.01\\
313.01	0.01\\
314.01	0.01\\
315.01	0.01\\
316.01	0.01\\
317.01	0.01\\
318.01	0.01\\
319.01	0.01\\
320.01	0.01\\
321.01	0.01\\
322.01	0.01\\
323.01	0.01\\
324.01	0.01\\
325.01	0.01\\
326.01	0.01\\
327.01	0.01\\
328.01	0.01\\
329.01	0.01\\
330.01	0.01\\
331.01	0.01\\
332.01	0.01\\
333.01	0.01\\
334.01	0.01\\
335.01	0.01\\
336.01	0.01\\
337.01	0.01\\
338.01	0.01\\
339.01	0.01\\
340.01	0.01\\
341.01	0.01\\
342.01	0.01\\
343.01	0.01\\
344.01	0.01\\
345.01	0.01\\
346.01	0.01\\
347.01	0.01\\
348.01	0.01\\
349.01	0.01\\
350.01	0.01\\
351.01	0.01\\
352.01	0.01\\
353.01	0.01\\
354.01	0.01\\
355.01	0.01\\
356.01	0.01\\
357.01	0.01\\
358.01	0.01\\
359.01	0.01\\
360.01	0.01\\
361.01	0.01\\
362.01	0.01\\
363.01	0.01\\
364.01	0.01\\
365.01	0.01\\
366.01	0.01\\
367.01	0.01\\
368.01	0.01\\
369.01	0.01\\
370.01	0.01\\
371.01	0.01\\
372.01	0.01\\
373.01	0.01\\
374.01	0.01\\
375.01	0.01\\
376.01	0.01\\
377.01	0.01\\
378.01	0.01\\
379.01	0.01\\
380.01	0.01\\
381.01	0.01\\
382.01	0.01\\
383.01	0.01\\
384.01	0.01\\
385.01	0.01\\
386.01	0.01\\
387.01	0.01\\
388.01	0.01\\
389.01	0.01\\
390.01	0.01\\
391.01	0.01\\
392.01	0.01\\
393.01	0.01\\
394.01	0.01\\
395.01	0.01\\
396.01	0.01\\
397.01	0.01\\
398.01	0.01\\
399.01	0.01\\
400.01	0.01\\
401.01	0.01\\
402.01	0.01\\
403.01	0.01\\
404.01	0.01\\
405.01	0.01\\
406.01	0.01\\
407.01	0.01\\
408.01	0.01\\
409.01	0.01\\
410.01	0.01\\
411.01	0.01\\
412.01	0.01\\
413.01	0.01\\
414.01	0.01\\
415.01	0.01\\
416.01	0.01\\
417.01	0.01\\
418.01	0.01\\
419.01	0.01\\
420.01	0.01\\
421.01	0.01\\
422.01	0.01\\
423.01	0.01\\
424.01	0.01\\
425.01	0.01\\
426.01	0.01\\
427.01	0.01\\
428.01	0.01\\
429.01	0.01\\
430.01	0.01\\
431.01	0.01\\
432.01	0.01\\
433.01	0.01\\
434.01	0.01\\
435.01	0.01\\
436.01	0.01\\
437.01	0.01\\
438.01	0.01\\
439.01	0.01\\
440.01	0.01\\
441.01	0.01\\
442.01	0.01\\
443.01	0.01\\
444.01	0.01\\
445.01	0.01\\
446.01	0.01\\
447.01	0.01\\
448.01	0.01\\
449.01	0.01\\
450.01	0.01\\
451.01	0.01\\
452.01	0.01\\
453.01	0.01\\
454.01	0.01\\
455.01	0.01\\
456.01	0.01\\
457.01	0.01\\
458.01	0.01\\
459.01	0.01\\
460.01	0.01\\
461.01	0.01\\
462.01	0.01\\
463.01	0.01\\
464.01	0.01\\
465.01	0.01\\
466.01	0.01\\
467.01	0.01\\
468.01	0.01\\
469.01	0.01\\
470.01	0.01\\
471.01	0.01\\
472.01	0.01\\
473.01	0.01\\
474.01	0.01\\
475.01	0.01\\
476.01	0.01\\
477.01	0.01\\
478.01	0.01\\
479.01	0.01\\
480.01	0.01\\
481.01	0.01\\
482.01	0.01\\
483.01	0.01\\
484.01	0.01\\
485.01	0.01\\
486.01	0.01\\
487.01	0.01\\
488.01	0.01\\
489.01	0.01\\
490.01	0.01\\
491.01	0.01\\
492.01	0.01\\
493.01	0.01\\
494.01	0.01\\
495.01	0.01\\
496.01	0.01\\
497.01	0.01\\
498.01	0.01\\
499.01	0.01\\
500.01	0.01\\
501.01	0.01\\
502.01	0.01\\
503.01	0.01\\
504.01	0.01\\
505.01	0.01\\
506.01	0.01\\
507.01	0.01\\
508.01	0.01\\
509.01	0.01\\
510.01	0.01\\
511.01	0.01\\
512.01	0.01\\
513.01	0.01\\
514.01	0.01\\
515.01	0.01\\
516.01	0.01\\
517.01	0.01\\
518.01	0.01\\
519.01	0.01\\
520.01	0.01\\
521.01	0.01\\
522.01	0.01\\
523.01	0.01\\
524.01	0.01\\
525.01	0.01\\
526.01	0.01\\
527.01	0.01\\
528.01	0.01\\
529.01	0.01\\
530.01	0.01\\
531.01	0.01\\
532.01	0.01\\
533.01	0.01\\
534.01	0.01\\
535.01	0.01\\
536.01	0.01\\
537.01	0.01\\
538.01	0.01\\
539.01	0.01\\
540.01	0.01\\
541.01	0.01\\
542.01	0.01\\
543.01	0.01\\
544.01	0.01\\
545.01	0.01\\
546.01	0.01\\
547.01	0.01\\
548.01	0.01\\
549.01	0.01\\
550.01	0.01\\
551.01	0.01\\
552.01	0.01\\
553.01	0.01\\
554.01	0.01\\
555.01	0.01\\
556.01	0.01\\
557.01	0.01\\
558.01	0.01\\
559.01	0.01\\
560.01	0.01\\
561.01	0.01\\
562.01	0.01\\
563.01	0.01\\
564.01	0.01\\
565.01	0.01\\
566.01	0.01\\
567.01	0.01\\
568.01	0.01\\
569.01	0.01\\
570.01	0.01\\
571.01	0.01\\
572.01	0.01\\
573.01	0.01\\
574.01	0.01\\
575.01	0.01\\
576.01	0.01\\
577.01	0.01\\
578.01	0.01\\
579.01	0.01\\
580.01	0.01\\
581.01	0.01\\
582.01	0.01\\
583.01	0.01\\
584.01	0.01\\
585.01	0.01\\
586.01	0.01\\
587.01	0.01\\
588.01	0.01\\
589.01	0.01\\
590.01	0.01\\
591.01	0.01\\
592.01	0.01\\
593.01	0.01\\
594.01	0.01\\
595.01	0.01\\
596.01	0.01\\
597.01	0.01\\
598.01	0.01\\
599.01	0.01\\
599.02	0.01\\
599.03	0.01\\
599.04	0.01\\
599.05	0.01\\
599.06	0.01\\
599.07	0.01\\
599.08	0.01\\
599.09	0.01\\
599.1	0.01\\
599.11	0.01\\
599.12	0.01\\
599.13	0.01\\
599.14	0.01\\
599.15	0.01\\
599.16	0.01\\
599.17	0.01\\
599.18	0.01\\
599.19	0.01\\
599.2	0.01\\
599.21	0.01\\
599.22	0.01\\
599.23	0.01\\
599.24	0.01\\
599.25	0.01\\
599.26	0.01\\
599.27	0.01\\
599.28	0.01\\
599.29	0.01\\
599.3	0.01\\
599.31	0.01\\
599.32	0.01\\
599.33	0.01\\
599.34	0.01\\
599.35	0.01\\
599.36	0.01\\
599.37	0.01\\
599.38	0.01\\
599.39	0.01\\
599.4	0.01\\
599.41	0.01\\
599.42	0.01\\
599.43	0.01\\
599.44	0.01\\
599.45	0.01\\
599.46	0.01\\
599.47	0.01\\
599.48	0.01\\
599.49	0.01\\
599.5	0.01\\
599.51	0.01\\
599.52	0.01\\
599.53	0.01\\
599.54	0.01\\
599.55	0.01\\
599.56	0.01\\
599.57	0.01\\
599.58	0.01\\
599.59	0.01\\
599.6	0.01\\
599.61	0.01\\
599.62	0.01\\
599.63	0.01\\
599.64	0.01\\
599.65	0.01\\
599.66	0.01\\
599.67	0.01\\
599.68	0.01\\
599.69	0.01\\
599.7	0.01\\
599.71	0.01\\
599.72	0.01\\
599.73	0.01\\
599.74	0.01\\
599.75	0.01\\
599.76	0.01\\
599.77	0.01\\
599.78	0.01\\
599.79	0.01\\
599.8	0.01\\
599.81	0.01\\
599.82	0.01\\
599.83	0.01\\
599.84	0.01\\
599.85	0.01\\
599.86	0.01\\
599.87	0.01\\
599.88	0.01\\
599.89	0.01\\
599.9	0.01\\
599.91	0.01\\
599.92	0.01\\
599.93	0.01\\
599.94	0.01\\
599.95	0.01\\
599.96	0.01\\
599.97	0.01\\
599.98	0.01\\
599.99	0.01\\
600	0.01\\
};
\addplot [color=black!60!mycolor21,solid,forget plot]
  table[row sep=crcr]{%
0.01	0.01\\
1.01	0.01\\
2.01	0.01\\
3.01	0.01\\
4.01	0.01\\
5.01	0.01\\
6.01	0.01\\
7.01	0.01\\
8.01	0.01\\
9.01	0.01\\
10.01	0.01\\
11.01	0.01\\
12.01	0.01\\
13.01	0.01\\
14.01	0.01\\
15.01	0.01\\
16.01	0.01\\
17.01	0.01\\
18.01	0.01\\
19.01	0.01\\
20.01	0.01\\
21.01	0.01\\
22.01	0.01\\
23.01	0.01\\
24.01	0.01\\
25.01	0.01\\
26.01	0.01\\
27.01	0.01\\
28.01	0.01\\
29.01	0.01\\
30.01	0.01\\
31.01	0.01\\
32.01	0.01\\
33.01	0.01\\
34.01	0.01\\
35.01	0.01\\
36.01	0.01\\
37.01	0.01\\
38.01	0.01\\
39.01	0.01\\
40.01	0.01\\
41.01	0.01\\
42.01	0.01\\
43.01	0.01\\
44.01	0.01\\
45.01	0.01\\
46.01	0.01\\
47.01	0.01\\
48.01	0.01\\
49.01	0.01\\
50.01	0.01\\
51.01	0.01\\
52.01	0.01\\
53.01	0.01\\
54.01	0.01\\
55.01	0.01\\
56.01	0.01\\
57.01	0.01\\
58.01	0.01\\
59.01	0.01\\
60.01	0.01\\
61.01	0.01\\
62.01	0.01\\
63.01	0.01\\
64.01	0.01\\
65.01	0.01\\
66.01	0.01\\
67.01	0.01\\
68.01	0.01\\
69.01	0.01\\
70.01	0.01\\
71.01	0.01\\
72.01	0.01\\
73.01	0.01\\
74.01	0.01\\
75.01	0.01\\
76.01	0.01\\
77.01	0.01\\
78.01	0.01\\
79.01	0.01\\
80.01	0.01\\
81.01	0.01\\
82.01	0.01\\
83.01	0.01\\
84.01	0.01\\
85.01	0.01\\
86.01	0.01\\
87.01	0.01\\
88.01	0.01\\
89.01	0.01\\
90.01	0.01\\
91.01	0.01\\
92.01	0.01\\
93.01	0.01\\
94.01	0.01\\
95.01	0.01\\
96.01	0.01\\
97.01	0.01\\
98.01	0.01\\
99.01	0.01\\
100.01	0.01\\
101.01	0.01\\
102.01	0.01\\
103.01	0.01\\
104.01	0.01\\
105.01	0.01\\
106.01	0.01\\
107.01	0.01\\
108.01	0.01\\
109.01	0.01\\
110.01	0.01\\
111.01	0.01\\
112.01	0.01\\
113.01	0.01\\
114.01	0.01\\
115.01	0.01\\
116.01	0.01\\
117.01	0.01\\
118.01	0.01\\
119.01	0.01\\
120.01	0.01\\
121.01	0.01\\
122.01	0.01\\
123.01	0.01\\
124.01	0.01\\
125.01	0.01\\
126.01	0.01\\
127.01	0.01\\
128.01	0.01\\
129.01	0.01\\
130.01	0.01\\
131.01	0.01\\
132.01	0.01\\
133.01	0.01\\
134.01	0.01\\
135.01	0.01\\
136.01	0.01\\
137.01	0.01\\
138.01	0.01\\
139.01	0.01\\
140.01	0.01\\
141.01	0.01\\
142.01	0.01\\
143.01	0.01\\
144.01	0.01\\
145.01	0.01\\
146.01	0.01\\
147.01	0.01\\
148.01	0.01\\
149.01	0.01\\
150.01	0.01\\
151.01	0.01\\
152.01	0.01\\
153.01	0.01\\
154.01	0.01\\
155.01	0.01\\
156.01	0.01\\
157.01	0.01\\
158.01	0.01\\
159.01	0.01\\
160.01	0.01\\
161.01	0.01\\
162.01	0.01\\
163.01	0.01\\
164.01	0.01\\
165.01	0.01\\
166.01	0.01\\
167.01	0.01\\
168.01	0.01\\
169.01	0.01\\
170.01	0.01\\
171.01	0.01\\
172.01	0.01\\
173.01	0.01\\
174.01	0.01\\
175.01	0.01\\
176.01	0.01\\
177.01	0.01\\
178.01	0.01\\
179.01	0.01\\
180.01	0.01\\
181.01	0.01\\
182.01	0.01\\
183.01	0.01\\
184.01	0.01\\
185.01	0.01\\
186.01	0.01\\
187.01	0.01\\
188.01	0.01\\
189.01	0.01\\
190.01	0.01\\
191.01	0.01\\
192.01	0.01\\
193.01	0.01\\
194.01	0.01\\
195.01	0.01\\
196.01	0.01\\
197.01	0.01\\
198.01	0.01\\
199.01	0.01\\
200.01	0.01\\
201.01	0.01\\
202.01	0.01\\
203.01	0.01\\
204.01	0.01\\
205.01	0.01\\
206.01	0.01\\
207.01	0.01\\
208.01	0.01\\
209.01	0.01\\
210.01	0.01\\
211.01	0.01\\
212.01	0.01\\
213.01	0.01\\
214.01	0.01\\
215.01	0.01\\
216.01	0.01\\
217.01	0.01\\
218.01	0.01\\
219.01	0.01\\
220.01	0.01\\
221.01	0.01\\
222.01	0.01\\
223.01	0.01\\
224.01	0.01\\
225.01	0.01\\
226.01	0.01\\
227.01	0.01\\
228.01	0.01\\
229.01	0.01\\
230.01	0.01\\
231.01	0.01\\
232.01	0.01\\
233.01	0.01\\
234.01	0.01\\
235.01	0.01\\
236.01	0.01\\
237.01	0.01\\
238.01	0.01\\
239.01	0.01\\
240.01	0.01\\
241.01	0.01\\
242.01	0.01\\
243.01	0.01\\
244.01	0.01\\
245.01	0.01\\
246.01	0.01\\
247.01	0.01\\
248.01	0.01\\
249.01	0.01\\
250.01	0.01\\
251.01	0.01\\
252.01	0.01\\
253.01	0.01\\
254.01	0.01\\
255.01	0.01\\
256.01	0.01\\
257.01	0.01\\
258.01	0.01\\
259.01	0.01\\
260.01	0.01\\
261.01	0.01\\
262.01	0.01\\
263.01	0.01\\
264.01	0.01\\
265.01	0.01\\
266.01	0.01\\
267.01	0.01\\
268.01	0.01\\
269.01	0.01\\
270.01	0.01\\
271.01	0.01\\
272.01	0.01\\
273.01	0.01\\
274.01	0.01\\
275.01	0.01\\
276.01	0.01\\
277.01	0.01\\
278.01	0.01\\
279.01	0.01\\
280.01	0.01\\
281.01	0.01\\
282.01	0.01\\
283.01	0.01\\
284.01	0.01\\
285.01	0.01\\
286.01	0.01\\
287.01	0.01\\
288.01	0.01\\
289.01	0.01\\
290.01	0.01\\
291.01	0.01\\
292.01	0.01\\
293.01	0.01\\
294.01	0.01\\
295.01	0.01\\
296.01	0.01\\
297.01	0.01\\
298.01	0.01\\
299.01	0.01\\
300.01	0.01\\
301.01	0.01\\
302.01	0.01\\
303.01	0.01\\
304.01	0.01\\
305.01	0.01\\
306.01	0.01\\
307.01	0.01\\
308.01	0.01\\
309.01	0.01\\
310.01	0.01\\
311.01	0.01\\
312.01	0.01\\
313.01	0.01\\
314.01	0.01\\
315.01	0.01\\
316.01	0.01\\
317.01	0.01\\
318.01	0.01\\
319.01	0.01\\
320.01	0.01\\
321.01	0.01\\
322.01	0.01\\
323.01	0.01\\
324.01	0.01\\
325.01	0.01\\
326.01	0.01\\
327.01	0.01\\
328.01	0.01\\
329.01	0.01\\
330.01	0.01\\
331.01	0.01\\
332.01	0.01\\
333.01	0.01\\
334.01	0.01\\
335.01	0.01\\
336.01	0.01\\
337.01	0.01\\
338.01	0.01\\
339.01	0.01\\
340.01	0.01\\
341.01	0.01\\
342.01	0.01\\
343.01	0.01\\
344.01	0.01\\
345.01	0.01\\
346.01	0.01\\
347.01	0.01\\
348.01	0.01\\
349.01	0.01\\
350.01	0.01\\
351.01	0.01\\
352.01	0.01\\
353.01	0.01\\
354.01	0.01\\
355.01	0.01\\
356.01	0.01\\
357.01	0.01\\
358.01	0.01\\
359.01	0.01\\
360.01	0.01\\
361.01	0.01\\
362.01	0.01\\
363.01	0.01\\
364.01	0.01\\
365.01	0.01\\
366.01	0.01\\
367.01	0.01\\
368.01	0.01\\
369.01	0.01\\
370.01	0.01\\
371.01	0.01\\
372.01	0.01\\
373.01	0.01\\
374.01	0.01\\
375.01	0.01\\
376.01	0.01\\
377.01	0.01\\
378.01	0.01\\
379.01	0.01\\
380.01	0.01\\
381.01	0.01\\
382.01	0.01\\
383.01	0.01\\
384.01	0.01\\
385.01	0.01\\
386.01	0.01\\
387.01	0.01\\
388.01	0.01\\
389.01	0.01\\
390.01	0.01\\
391.01	0.01\\
392.01	0.01\\
393.01	0.01\\
394.01	0.01\\
395.01	0.01\\
396.01	0.01\\
397.01	0.01\\
398.01	0.01\\
399.01	0.01\\
400.01	0.01\\
401.01	0.01\\
402.01	0.01\\
403.01	0.01\\
404.01	0.01\\
405.01	0.01\\
406.01	0.01\\
407.01	0.01\\
408.01	0.01\\
409.01	0.01\\
410.01	0.01\\
411.01	0.01\\
412.01	0.01\\
413.01	0.01\\
414.01	0.01\\
415.01	0.01\\
416.01	0.01\\
417.01	0.01\\
418.01	0.01\\
419.01	0.01\\
420.01	0.01\\
421.01	0.01\\
422.01	0.01\\
423.01	0.01\\
424.01	0.01\\
425.01	0.01\\
426.01	0.01\\
427.01	0.01\\
428.01	0.01\\
429.01	0.01\\
430.01	0.01\\
431.01	0.01\\
432.01	0.01\\
433.01	0.01\\
434.01	0.01\\
435.01	0.01\\
436.01	0.01\\
437.01	0.01\\
438.01	0.01\\
439.01	0.01\\
440.01	0.01\\
441.01	0.01\\
442.01	0.01\\
443.01	0.01\\
444.01	0.01\\
445.01	0.01\\
446.01	0.01\\
447.01	0.01\\
448.01	0.01\\
449.01	0.01\\
450.01	0.01\\
451.01	0.01\\
452.01	0.01\\
453.01	0.01\\
454.01	0.01\\
455.01	0.01\\
456.01	0.01\\
457.01	0.01\\
458.01	0.01\\
459.01	0.01\\
460.01	0.01\\
461.01	0.01\\
462.01	0.01\\
463.01	0.01\\
464.01	0.01\\
465.01	0.01\\
466.01	0.01\\
467.01	0.01\\
468.01	0.01\\
469.01	0.01\\
470.01	0.01\\
471.01	0.01\\
472.01	0.01\\
473.01	0.01\\
474.01	0.01\\
475.01	0.01\\
476.01	0.01\\
477.01	0.01\\
478.01	0.01\\
479.01	0.01\\
480.01	0.01\\
481.01	0.01\\
482.01	0.01\\
483.01	0.01\\
484.01	0.01\\
485.01	0.01\\
486.01	0.01\\
487.01	0.01\\
488.01	0.01\\
489.01	0.01\\
490.01	0.01\\
491.01	0.01\\
492.01	0.01\\
493.01	0.01\\
494.01	0.01\\
495.01	0.01\\
496.01	0.01\\
497.01	0.01\\
498.01	0.01\\
499.01	0.01\\
500.01	0.01\\
501.01	0.01\\
502.01	0.01\\
503.01	0.01\\
504.01	0.01\\
505.01	0.01\\
506.01	0.01\\
507.01	0.01\\
508.01	0.01\\
509.01	0.01\\
510.01	0.01\\
511.01	0.01\\
512.01	0.01\\
513.01	0.01\\
514.01	0.01\\
515.01	0.01\\
516.01	0.01\\
517.01	0.01\\
518.01	0.01\\
519.01	0.01\\
520.01	0.01\\
521.01	0.01\\
522.01	0.01\\
523.01	0.01\\
524.01	0.01\\
525.01	0.01\\
526.01	0.01\\
527.01	0.01\\
528.01	0.01\\
529.01	0.01\\
530.01	0.01\\
531.01	0.01\\
532.01	0.01\\
533.01	0.01\\
534.01	0.01\\
535.01	0.01\\
536.01	0.01\\
537.01	0.01\\
538.01	0.01\\
539.01	0.01\\
540.01	0.01\\
541.01	0.01\\
542.01	0.01\\
543.01	0.01\\
544.01	0.01\\
545.01	0.01\\
546.01	0.01\\
547.01	0.01\\
548.01	0.01\\
549.01	0.01\\
550.01	0.01\\
551.01	0.01\\
552.01	0.01\\
553.01	0.01\\
554.01	0.01\\
555.01	0.01\\
556.01	0.01\\
557.01	0.01\\
558.01	0.01\\
559.01	0.01\\
560.01	0.01\\
561.01	0.01\\
562.01	0.01\\
563.01	0.01\\
564.01	0.01\\
565.01	0.01\\
566.01	0.01\\
567.01	0.01\\
568.01	0.01\\
569.01	0.01\\
570.01	0.01\\
571.01	0.01\\
572.01	0.01\\
573.01	0.01\\
574.01	0.01\\
575.01	0.01\\
576.01	0.01\\
577.01	0.01\\
578.01	0.01\\
579.01	0.01\\
580.01	0.01\\
581.01	0.01\\
582.01	0.01\\
583.01	0.01\\
584.01	0.01\\
585.01	0.01\\
586.01	0.01\\
587.01	0.01\\
588.01	0.01\\
589.01	0.01\\
590.01	0.01\\
591.01	0.01\\
592.01	0.01\\
593.01	0.01\\
594.01	0.01\\
595.01	0.01\\
596.01	0.01\\
597.01	0.01\\
598.01	0.01\\
599.01	0.01\\
599.02	0.01\\
599.03	0.01\\
599.04	0.01\\
599.05	0.01\\
599.06	0.01\\
599.07	0.01\\
599.08	0.01\\
599.09	0.01\\
599.1	0.01\\
599.11	0.01\\
599.12	0.01\\
599.13	0.01\\
599.14	0.01\\
599.15	0.01\\
599.16	0.01\\
599.17	0.01\\
599.18	0.01\\
599.19	0.01\\
599.2	0.01\\
599.21	0.01\\
599.22	0.01\\
599.23	0.01\\
599.24	0.01\\
599.25	0.01\\
599.26	0.01\\
599.27	0.01\\
599.28	0.01\\
599.29	0.01\\
599.3	0.01\\
599.31	0.01\\
599.32	0.01\\
599.33	0.01\\
599.34	0.01\\
599.35	0.01\\
599.36	0.01\\
599.37	0.01\\
599.38	0.01\\
599.39	0.01\\
599.4	0.01\\
599.41	0.01\\
599.42	0.01\\
599.43	0.01\\
599.44	0.01\\
599.45	0.01\\
599.46	0.01\\
599.47	0.01\\
599.48	0.01\\
599.49	0.01\\
599.5	0.01\\
599.51	0.01\\
599.52	0.01\\
599.53	0.01\\
599.54	0.01\\
599.55	0.01\\
599.56	0.01\\
599.57	0.01\\
599.58	0.01\\
599.59	0.01\\
599.6	0.01\\
599.61	0.01\\
599.62	0.01\\
599.63	0.01\\
599.64	0.01\\
599.65	0.01\\
599.66	0.01\\
599.67	0.01\\
599.68	0.01\\
599.69	0.01\\
599.7	0.01\\
599.71	0.01\\
599.72	0.01\\
599.73	0.01\\
599.74	0.01\\
599.75	0.01\\
599.76	0.01\\
599.77	0.01\\
599.78	0.01\\
599.79	0.01\\
599.8	0.01\\
599.81	0.01\\
599.82	0.01\\
599.83	0.01\\
599.84	0.01\\
599.85	0.01\\
599.86	0.01\\
599.87	0.01\\
599.88	0.01\\
599.89	0.01\\
599.9	0.01\\
599.91	0.01\\
599.92	0.01\\
599.93	0.01\\
599.94	0.01\\
599.95	0.01\\
599.96	0.01\\
599.97	0.01\\
599.98	0.01\\
599.99	0.01\\
600	0.01\\
};
\addplot [color=black!80!mycolor21,solid,forget plot]
  table[row sep=crcr]{%
0.01	0.01\\
1.01	0.01\\
2.01	0.01\\
3.01	0.01\\
4.01	0.01\\
5.01	0.01\\
6.01	0.01\\
7.01	0.01\\
8.01	0.01\\
9.01	0.01\\
10.01	0.01\\
11.01	0.01\\
12.01	0.01\\
13.01	0.01\\
14.01	0.01\\
15.01	0.01\\
16.01	0.01\\
17.01	0.01\\
18.01	0.01\\
19.01	0.01\\
20.01	0.01\\
21.01	0.01\\
22.01	0.01\\
23.01	0.01\\
24.01	0.01\\
25.01	0.01\\
26.01	0.01\\
27.01	0.01\\
28.01	0.01\\
29.01	0.01\\
30.01	0.01\\
31.01	0.01\\
32.01	0.01\\
33.01	0.01\\
34.01	0.01\\
35.01	0.01\\
36.01	0.01\\
37.01	0.01\\
38.01	0.01\\
39.01	0.01\\
40.01	0.01\\
41.01	0.01\\
42.01	0.01\\
43.01	0.01\\
44.01	0.01\\
45.01	0.01\\
46.01	0.01\\
47.01	0.01\\
48.01	0.01\\
49.01	0.01\\
50.01	0.01\\
51.01	0.01\\
52.01	0.01\\
53.01	0.01\\
54.01	0.01\\
55.01	0.01\\
56.01	0.01\\
57.01	0.01\\
58.01	0.01\\
59.01	0.01\\
60.01	0.01\\
61.01	0.01\\
62.01	0.01\\
63.01	0.01\\
64.01	0.01\\
65.01	0.01\\
66.01	0.01\\
67.01	0.01\\
68.01	0.01\\
69.01	0.01\\
70.01	0.01\\
71.01	0.01\\
72.01	0.01\\
73.01	0.01\\
74.01	0.01\\
75.01	0.01\\
76.01	0.01\\
77.01	0.01\\
78.01	0.01\\
79.01	0.01\\
80.01	0.01\\
81.01	0.01\\
82.01	0.01\\
83.01	0.01\\
84.01	0.01\\
85.01	0.01\\
86.01	0.01\\
87.01	0.01\\
88.01	0.01\\
89.01	0.01\\
90.01	0.01\\
91.01	0.01\\
92.01	0.01\\
93.01	0.01\\
94.01	0.01\\
95.01	0.01\\
96.01	0.01\\
97.01	0.01\\
98.01	0.01\\
99.01	0.01\\
100.01	0.01\\
101.01	0.01\\
102.01	0.01\\
103.01	0.01\\
104.01	0.01\\
105.01	0.01\\
106.01	0.01\\
107.01	0.01\\
108.01	0.01\\
109.01	0.01\\
110.01	0.01\\
111.01	0.01\\
112.01	0.01\\
113.01	0.01\\
114.01	0.01\\
115.01	0.01\\
116.01	0.01\\
117.01	0.01\\
118.01	0.01\\
119.01	0.01\\
120.01	0.01\\
121.01	0.01\\
122.01	0.01\\
123.01	0.01\\
124.01	0.01\\
125.01	0.01\\
126.01	0.01\\
127.01	0.01\\
128.01	0.01\\
129.01	0.01\\
130.01	0.01\\
131.01	0.01\\
132.01	0.01\\
133.01	0.01\\
134.01	0.01\\
135.01	0.01\\
136.01	0.01\\
137.01	0.01\\
138.01	0.01\\
139.01	0.01\\
140.01	0.01\\
141.01	0.01\\
142.01	0.01\\
143.01	0.01\\
144.01	0.01\\
145.01	0.01\\
146.01	0.01\\
147.01	0.01\\
148.01	0.01\\
149.01	0.01\\
150.01	0.01\\
151.01	0.01\\
152.01	0.01\\
153.01	0.01\\
154.01	0.01\\
155.01	0.01\\
156.01	0.01\\
157.01	0.01\\
158.01	0.01\\
159.01	0.01\\
160.01	0.01\\
161.01	0.01\\
162.01	0.01\\
163.01	0.01\\
164.01	0.01\\
165.01	0.01\\
166.01	0.01\\
167.01	0.01\\
168.01	0.01\\
169.01	0.01\\
170.01	0.01\\
171.01	0.01\\
172.01	0.01\\
173.01	0.01\\
174.01	0.01\\
175.01	0.01\\
176.01	0.01\\
177.01	0.01\\
178.01	0.01\\
179.01	0.01\\
180.01	0.01\\
181.01	0.01\\
182.01	0.01\\
183.01	0.01\\
184.01	0.01\\
185.01	0.01\\
186.01	0.01\\
187.01	0.01\\
188.01	0.01\\
189.01	0.01\\
190.01	0.01\\
191.01	0.01\\
192.01	0.01\\
193.01	0.01\\
194.01	0.01\\
195.01	0.01\\
196.01	0.01\\
197.01	0.01\\
198.01	0.01\\
199.01	0.01\\
200.01	0.01\\
201.01	0.01\\
202.01	0.01\\
203.01	0.01\\
204.01	0.01\\
205.01	0.01\\
206.01	0.01\\
207.01	0.01\\
208.01	0.01\\
209.01	0.01\\
210.01	0.01\\
211.01	0.01\\
212.01	0.01\\
213.01	0.01\\
214.01	0.01\\
215.01	0.01\\
216.01	0.01\\
217.01	0.01\\
218.01	0.01\\
219.01	0.01\\
220.01	0.01\\
221.01	0.01\\
222.01	0.01\\
223.01	0.01\\
224.01	0.01\\
225.01	0.01\\
226.01	0.01\\
227.01	0.01\\
228.01	0.01\\
229.01	0.01\\
230.01	0.01\\
231.01	0.01\\
232.01	0.01\\
233.01	0.01\\
234.01	0.01\\
235.01	0.01\\
236.01	0.01\\
237.01	0.01\\
238.01	0.01\\
239.01	0.01\\
240.01	0.01\\
241.01	0.01\\
242.01	0.01\\
243.01	0.01\\
244.01	0.01\\
245.01	0.01\\
246.01	0.01\\
247.01	0.01\\
248.01	0.01\\
249.01	0.01\\
250.01	0.01\\
251.01	0.01\\
252.01	0.01\\
253.01	0.01\\
254.01	0.01\\
255.01	0.01\\
256.01	0.01\\
257.01	0.01\\
258.01	0.01\\
259.01	0.01\\
260.01	0.01\\
261.01	0.01\\
262.01	0.01\\
263.01	0.01\\
264.01	0.01\\
265.01	0.01\\
266.01	0.01\\
267.01	0.01\\
268.01	0.01\\
269.01	0.01\\
270.01	0.01\\
271.01	0.01\\
272.01	0.01\\
273.01	0.01\\
274.01	0.01\\
275.01	0.01\\
276.01	0.01\\
277.01	0.01\\
278.01	0.01\\
279.01	0.01\\
280.01	0.01\\
281.01	0.01\\
282.01	0.01\\
283.01	0.01\\
284.01	0.01\\
285.01	0.01\\
286.01	0.01\\
287.01	0.01\\
288.01	0.01\\
289.01	0.01\\
290.01	0.01\\
291.01	0.01\\
292.01	0.01\\
293.01	0.01\\
294.01	0.01\\
295.01	0.01\\
296.01	0.01\\
297.01	0.01\\
298.01	0.01\\
299.01	0.01\\
300.01	0.01\\
301.01	0.01\\
302.01	0.01\\
303.01	0.01\\
304.01	0.01\\
305.01	0.01\\
306.01	0.01\\
307.01	0.01\\
308.01	0.01\\
309.01	0.01\\
310.01	0.01\\
311.01	0.01\\
312.01	0.01\\
313.01	0.01\\
314.01	0.01\\
315.01	0.01\\
316.01	0.01\\
317.01	0.01\\
318.01	0.01\\
319.01	0.01\\
320.01	0.01\\
321.01	0.01\\
322.01	0.01\\
323.01	0.01\\
324.01	0.01\\
325.01	0.01\\
326.01	0.01\\
327.01	0.01\\
328.01	0.01\\
329.01	0.01\\
330.01	0.01\\
331.01	0.01\\
332.01	0.01\\
333.01	0.01\\
334.01	0.01\\
335.01	0.01\\
336.01	0.01\\
337.01	0.01\\
338.01	0.01\\
339.01	0.01\\
340.01	0.01\\
341.01	0.01\\
342.01	0.01\\
343.01	0.01\\
344.01	0.01\\
345.01	0.01\\
346.01	0.01\\
347.01	0.01\\
348.01	0.01\\
349.01	0.01\\
350.01	0.01\\
351.01	0.01\\
352.01	0.01\\
353.01	0.01\\
354.01	0.01\\
355.01	0.01\\
356.01	0.01\\
357.01	0.01\\
358.01	0.01\\
359.01	0.01\\
360.01	0.01\\
361.01	0.01\\
362.01	0.01\\
363.01	0.01\\
364.01	0.01\\
365.01	0.01\\
366.01	0.01\\
367.01	0.01\\
368.01	0.01\\
369.01	0.01\\
370.01	0.01\\
371.01	0.01\\
372.01	0.01\\
373.01	0.01\\
374.01	0.01\\
375.01	0.01\\
376.01	0.01\\
377.01	0.01\\
378.01	0.01\\
379.01	0.01\\
380.01	0.01\\
381.01	0.01\\
382.01	0.01\\
383.01	0.01\\
384.01	0.01\\
385.01	0.01\\
386.01	0.01\\
387.01	0.01\\
388.01	0.01\\
389.01	0.01\\
390.01	0.01\\
391.01	0.01\\
392.01	0.01\\
393.01	0.01\\
394.01	0.01\\
395.01	0.01\\
396.01	0.01\\
397.01	0.01\\
398.01	0.01\\
399.01	0.01\\
400.01	0.01\\
401.01	0.01\\
402.01	0.01\\
403.01	0.01\\
404.01	0.01\\
405.01	0.01\\
406.01	0.01\\
407.01	0.01\\
408.01	0.01\\
409.01	0.01\\
410.01	0.01\\
411.01	0.01\\
412.01	0.01\\
413.01	0.01\\
414.01	0.01\\
415.01	0.01\\
416.01	0.01\\
417.01	0.01\\
418.01	0.01\\
419.01	0.01\\
420.01	0.01\\
421.01	0.01\\
422.01	0.01\\
423.01	0.01\\
424.01	0.01\\
425.01	0.01\\
426.01	0.01\\
427.01	0.01\\
428.01	0.01\\
429.01	0.01\\
430.01	0.01\\
431.01	0.01\\
432.01	0.01\\
433.01	0.01\\
434.01	0.01\\
435.01	0.01\\
436.01	0.01\\
437.01	0.01\\
438.01	0.01\\
439.01	0.01\\
440.01	0.01\\
441.01	0.01\\
442.01	0.01\\
443.01	0.01\\
444.01	0.01\\
445.01	0.01\\
446.01	0.01\\
447.01	0.01\\
448.01	0.01\\
449.01	0.01\\
450.01	0.01\\
451.01	0.01\\
452.01	0.01\\
453.01	0.01\\
454.01	0.01\\
455.01	0.01\\
456.01	0.01\\
457.01	0.01\\
458.01	0.01\\
459.01	0.01\\
460.01	0.01\\
461.01	0.01\\
462.01	0.01\\
463.01	0.01\\
464.01	0.01\\
465.01	0.01\\
466.01	0.01\\
467.01	0.01\\
468.01	0.01\\
469.01	0.01\\
470.01	0.01\\
471.01	0.01\\
472.01	0.01\\
473.01	0.01\\
474.01	0.01\\
475.01	0.01\\
476.01	0.01\\
477.01	0.01\\
478.01	0.01\\
479.01	0.01\\
480.01	0.01\\
481.01	0.01\\
482.01	0.01\\
483.01	0.01\\
484.01	0.01\\
485.01	0.01\\
486.01	0.01\\
487.01	0.01\\
488.01	0.01\\
489.01	0.01\\
490.01	0.01\\
491.01	0.01\\
492.01	0.01\\
493.01	0.01\\
494.01	0.01\\
495.01	0.01\\
496.01	0.01\\
497.01	0.01\\
498.01	0.01\\
499.01	0.01\\
500.01	0.01\\
501.01	0.01\\
502.01	0.01\\
503.01	0.01\\
504.01	0.01\\
505.01	0.01\\
506.01	0.01\\
507.01	0.01\\
508.01	0.01\\
509.01	0.01\\
510.01	0.01\\
511.01	0.01\\
512.01	0.01\\
513.01	0.01\\
514.01	0.01\\
515.01	0.01\\
516.01	0.01\\
517.01	0.01\\
518.01	0.01\\
519.01	0.01\\
520.01	0.01\\
521.01	0.01\\
522.01	0.01\\
523.01	0.01\\
524.01	0.01\\
525.01	0.01\\
526.01	0.01\\
527.01	0.01\\
528.01	0.01\\
529.01	0.01\\
530.01	0.01\\
531.01	0.01\\
532.01	0.01\\
533.01	0.01\\
534.01	0.01\\
535.01	0.01\\
536.01	0.01\\
537.01	0.01\\
538.01	0.01\\
539.01	0.01\\
540.01	0.01\\
541.01	0.01\\
542.01	0.01\\
543.01	0.01\\
544.01	0.01\\
545.01	0.01\\
546.01	0.01\\
547.01	0.01\\
548.01	0.01\\
549.01	0.01\\
550.01	0.01\\
551.01	0.01\\
552.01	0.01\\
553.01	0.01\\
554.01	0.01\\
555.01	0.01\\
556.01	0.01\\
557.01	0.01\\
558.01	0.01\\
559.01	0.01\\
560.01	0.01\\
561.01	0.01\\
562.01	0.01\\
563.01	0.01\\
564.01	0.01\\
565.01	0.01\\
566.01	0.01\\
567.01	0.01\\
568.01	0.01\\
569.01	0.01\\
570.01	0.01\\
571.01	0.01\\
572.01	0.01\\
573.01	0.01\\
574.01	0.01\\
575.01	0.01\\
576.01	0.01\\
577.01	0.01\\
578.01	0.01\\
579.01	0.01\\
580.01	0.01\\
581.01	0.01\\
582.01	0.01\\
583.01	0.01\\
584.01	0.01\\
585.01	0.01\\
586.01	0.01\\
587.01	0.01\\
588.01	0.01\\
589.01	0.01\\
590.01	0.01\\
591.01	0.01\\
592.01	0.01\\
593.01	0.01\\
594.01	0.01\\
595.01	0.01\\
596.01	0.01\\
597.01	0.01\\
598.01	0.01\\
599.01	0.01\\
599.02	0.01\\
599.03	0.01\\
599.04	0.01\\
599.05	0.01\\
599.06	0.01\\
599.07	0.01\\
599.08	0.01\\
599.09	0.01\\
599.1	0.01\\
599.11	0.01\\
599.12	0.01\\
599.13	0.01\\
599.14	0.01\\
599.15	0.01\\
599.16	0.01\\
599.17	0.01\\
599.18	0.01\\
599.19	0.01\\
599.2	0.01\\
599.21	0.01\\
599.22	0.01\\
599.23	0.01\\
599.24	0.01\\
599.25	0.01\\
599.26	0.01\\
599.27	0.01\\
599.28	0.01\\
599.29	0.01\\
599.3	0.01\\
599.31	0.01\\
599.32	0.01\\
599.33	0.01\\
599.34	0.01\\
599.35	0.01\\
599.36	0.01\\
599.37	0.01\\
599.38	0.01\\
599.39	0.01\\
599.4	0.01\\
599.41	0.01\\
599.42	0.01\\
599.43	0.01\\
599.44	0.01\\
599.45	0.01\\
599.46	0.01\\
599.47	0.01\\
599.48	0.01\\
599.49	0.01\\
599.5	0.01\\
599.51	0.01\\
599.52	0.01\\
599.53	0.01\\
599.54	0.01\\
599.55	0.01\\
599.56	0.01\\
599.57	0.01\\
599.58	0.01\\
599.59	0.01\\
599.6	0.01\\
599.61	0.01\\
599.62	0.01\\
599.63	0.01\\
599.64	0.01\\
599.65	0.01\\
599.66	0.01\\
599.67	0.01\\
599.68	0.01\\
599.69	0.01\\
599.7	0.01\\
599.71	0.01\\
599.72	0.01\\
599.73	0.01\\
599.74	0.01\\
599.75	0.01\\
599.76	0.01\\
599.77	0.01\\
599.78	0.01\\
599.79	0.01\\
599.8	0.01\\
599.81	0.01\\
599.82	0.01\\
599.83	0.01\\
599.84	0.01\\
599.85	0.01\\
599.86	0.01\\
599.87	0.01\\
599.88	0.01\\
599.89	0.01\\
599.9	0.01\\
599.91	0.01\\
599.92	0.01\\
599.93	0.01\\
599.94	0.01\\
599.95	0.01\\
599.96	0.01\\
599.97	0.01\\
599.98	0.01\\
599.99	0.01\\
600	0.01\\
};
\addplot [color=black,solid,forget plot]
  table[row sep=crcr]{%
0.01	0.01\\
1.01	0.01\\
2.01	0.01\\
3.01	0.01\\
4.01	0.01\\
5.01	0.01\\
6.01	0.01\\
7.01	0.01\\
8.01	0.01\\
9.01	0.01\\
10.01	0.01\\
11.01	0.01\\
12.01	0.01\\
13.01	0.01\\
14.01	0.01\\
15.01	0.01\\
16.01	0.01\\
17.01	0.01\\
18.01	0.01\\
19.01	0.01\\
20.01	0.01\\
21.01	0.01\\
22.01	0.01\\
23.01	0.01\\
24.01	0.01\\
25.01	0.01\\
26.01	0.01\\
27.01	0.01\\
28.01	0.01\\
29.01	0.01\\
30.01	0.01\\
31.01	0.01\\
32.01	0.01\\
33.01	0.01\\
34.01	0.01\\
35.01	0.01\\
36.01	0.01\\
37.01	0.01\\
38.01	0.01\\
39.01	0.01\\
40.01	0.01\\
41.01	0.01\\
42.01	0.01\\
43.01	0.01\\
44.01	0.01\\
45.01	0.01\\
46.01	0.01\\
47.01	0.01\\
48.01	0.01\\
49.01	0.01\\
50.01	0.01\\
51.01	0.01\\
52.01	0.01\\
53.01	0.01\\
54.01	0.01\\
55.01	0.01\\
56.01	0.01\\
57.01	0.01\\
58.01	0.01\\
59.01	0.01\\
60.01	0.01\\
61.01	0.01\\
62.01	0.01\\
63.01	0.01\\
64.01	0.01\\
65.01	0.01\\
66.01	0.01\\
67.01	0.01\\
68.01	0.01\\
69.01	0.01\\
70.01	0.01\\
71.01	0.01\\
72.01	0.01\\
73.01	0.01\\
74.01	0.01\\
75.01	0.01\\
76.01	0.01\\
77.01	0.01\\
78.01	0.01\\
79.01	0.01\\
80.01	0.01\\
81.01	0.01\\
82.01	0.01\\
83.01	0.01\\
84.01	0.01\\
85.01	0.01\\
86.01	0.01\\
87.01	0.01\\
88.01	0.01\\
89.01	0.01\\
90.01	0.01\\
91.01	0.01\\
92.01	0.01\\
93.01	0.01\\
94.01	0.01\\
95.01	0.01\\
96.01	0.01\\
97.01	0.01\\
98.01	0.01\\
99.01	0.01\\
100.01	0.01\\
101.01	0.01\\
102.01	0.01\\
103.01	0.01\\
104.01	0.01\\
105.01	0.01\\
106.01	0.01\\
107.01	0.01\\
108.01	0.01\\
109.01	0.01\\
110.01	0.01\\
111.01	0.01\\
112.01	0.01\\
113.01	0.01\\
114.01	0.01\\
115.01	0.01\\
116.01	0.01\\
117.01	0.01\\
118.01	0.01\\
119.01	0.01\\
120.01	0.01\\
121.01	0.01\\
122.01	0.01\\
123.01	0.01\\
124.01	0.01\\
125.01	0.01\\
126.01	0.01\\
127.01	0.01\\
128.01	0.01\\
129.01	0.01\\
130.01	0.01\\
131.01	0.01\\
132.01	0.01\\
133.01	0.01\\
134.01	0.01\\
135.01	0.01\\
136.01	0.01\\
137.01	0.01\\
138.01	0.01\\
139.01	0.01\\
140.01	0.01\\
141.01	0.01\\
142.01	0.01\\
143.01	0.01\\
144.01	0.01\\
145.01	0.01\\
146.01	0.01\\
147.01	0.01\\
148.01	0.01\\
149.01	0.01\\
150.01	0.01\\
151.01	0.01\\
152.01	0.01\\
153.01	0.01\\
154.01	0.01\\
155.01	0.01\\
156.01	0.01\\
157.01	0.01\\
158.01	0.01\\
159.01	0.01\\
160.01	0.01\\
161.01	0.01\\
162.01	0.01\\
163.01	0.01\\
164.01	0.01\\
165.01	0.01\\
166.01	0.01\\
167.01	0.01\\
168.01	0.01\\
169.01	0.01\\
170.01	0.01\\
171.01	0.01\\
172.01	0.01\\
173.01	0.01\\
174.01	0.01\\
175.01	0.01\\
176.01	0.01\\
177.01	0.01\\
178.01	0.01\\
179.01	0.01\\
180.01	0.01\\
181.01	0.01\\
182.01	0.01\\
183.01	0.01\\
184.01	0.01\\
185.01	0.01\\
186.01	0.01\\
187.01	0.01\\
188.01	0.01\\
189.01	0.01\\
190.01	0.01\\
191.01	0.01\\
192.01	0.01\\
193.01	0.01\\
194.01	0.01\\
195.01	0.01\\
196.01	0.01\\
197.01	0.01\\
198.01	0.01\\
199.01	0.01\\
200.01	0.01\\
201.01	0.01\\
202.01	0.01\\
203.01	0.01\\
204.01	0.01\\
205.01	0.01\\
206.01	0.01\\
207.01	0.01\\
208.01	0.01\\
209.01	0.01\\
210.01	0.01\\
211.01	0.01\\
212.01	0.01\\
213.01	0.01\\
214.01	0.01\\
215.01	0.01\\
216.01	0.01\\
217.01	0.01\\
218.01	0.01\\
219.01	0.01\\
220.01	0.01\\
221.01	0.01\\
222.01	0.01\\
223.01	0.01\\
224.01	0.01\\
225.01	0.01\\
226.01	0.01\\
227.01	0.01\\
228.01	0.01\\
229.01	0.01\\
230.01	0.01\\
231.01	0.01\\
232.01	0.01\\
233.01	0.01\\
234.01	0.01\\
235.01	0.01\\
236.01	0.01\\
237.01	0.01\\
238.01	0.01\\
239.01	0.01\\
240.01	0.01\\
241.01	0.01\\
242.01	0.01\\
243.01	0.01\\
244.01	0.01\\
245.01	0.01\\
246.01	0.01\\
247.01	0.01\\
248.01	0.01\\
249.01	0.01\\
250.01	0.01\\
251.01	0.01\\
252.01	0.01\\
253.01	0.01\\
254.01	0.01\\
255.01	0.01\\
256.01	0.01\\
257.01	0.01\\
258.01	0.01\\
259.01	0.01\\
260.01	0.01\\
261.01	0.01\\
262.01	0.01\\
263.01	0.01\\
264.01	0.01\\
265.01	0.01\\
266.01	0.01\\
267.01	0.01\\
268.01	0.01\\
269.01	0.01\\
270.01	0.01\\
271.01	0.01\\
272.01	0.01\\
273.01	0.01\\
274.01	0.01\\
275.01	0.01\\
276.01	0.01\\
277.01	0.01\\
278.01	0.01\\
279.01	0.01\\
280.01	0.01\\
281.01	0.01\\
282.01	0.01\\
283.01	0.01\\
284.01	0.01\\
285.01	0.01\\
286.01	0.01\\
287.01	0.01\\
288.01	0.01\\
289.01	0.01\\
290.01	0.01\\
291.01	0.01\\
292.01	0.01\\
293.01	0.01\\
294.01	0.01\\
295.01	0.01\\
296.01	0.01\\
297.01	0.01\\
298.01	0.01\\
299.01	0.01\\
300.01	0.01\\
301.01	0.01\\
302.01	0.01\\
303.01	0.01\\
304.01	0.01\\
305.01	0.01\\
306.01	0.01\\
307.01	0.01\\
308.01	0.01\\
309.01	0.01\\
310.01	0.01\\
311.01	0.01\\
312.01	0.01\\
313.01	0.01\\
314.01	0.01\\
315.01	0.01\\
316.01	0.01\\
317.01	0.01\\
318.01	0.01\\
319.01	0.01\\
320.01	0.01\\
321.01	0.01\\
322.01	0.01\\
323.01	0.01\\
324.01	0.01\\
325.01	0.01\\
326.01	0.01\\
327.01	0.01\\
328.01	0.01\\
329.01	0.01\\
330.01	0.01\\
331.01	0.01\\
332.01	0.01\\
333.01	0.01\\
334.01	0.01\\
335.01	0.01\\
336.01	0.01\\
337.01	0.01\\
338.01	0.01\\
339.01	0.01\\
340.01	0.01\\
341.01	0.01\\
342.01	0.01\\
343.01	0.01\\
344.01	0.01\\
345.01	0.01\\
346.01	0.01\\
347.01	0.01\\
348.01	0.01\\
349.01	0.01\\
350.01	0.01\\
351.01	0.01\\
352.01	0.01\\
353.01	0.01\\
354.01	0.01\\
355.01	0.01\\
356.01	0.01\\
357.01	0.01\\
358.01	0.01\\
359.01	0.01\\
360.01	0.01\\
361.01	0.01\\
362.01	0.01\\
363.01	0.01\\
364.01	0.01\\
365.01	0.01\\
366.01	0.01\\
367.01	0.01\\
368.01	0.01\\
369.01	0.01\\
370.01	0.01\\
371.01	0.01\\
372.01	0.01\\
373.01	0.01\\
374.01	0.01\\
375.01	0.01\\
376.01	0.01\\
377.01	0.01\\
378.01	0.01\\
379.01	0.01\\
380.01	0.01\\
381.01	0.01\\
382.01	0.01\\
383.01	0.01\\
384.01	0.01\\
385.01	0.01\\
386.01	0.01\\
387.01	0.01\\
388.01	0.01\\
389.01	0.01\\
390.01	0.01\\
391.01	0.01\\
392.01	0.01\\
393.01	0.01\\
394.01	0.01\\
395.01	0.01\\
396.01	0.01\\
397.01	0.01\\
398.01	0.01\\
399.01	0.01\\
400.01	0.01\\
401.01	0.01\\
402.01	0.01\\
403.01	0.01\\
404.01	0.01\\
405.01	0.01\\
406.01	0.01\\
407.01	0.01\\
408.01	0.01\\
409.01	0.01\\
410.01	0.01\\
411.01	0.01\\
412.01	0.01\\
413.01	0.01\\
414.01	0.01\\
415.01	0.01\\
416.01	0.01\\
417.01	0.01\\
418.01	0.01\\
419.01	0.01\\
420.01	0.01\\
421.01	0.01\\
422.01	0.01\\
423.01	0.01\\
424.01	0.01\\
425.01	0.01\\
426.01	0.01\\
427.01	0.01\\
428.01	0.01\\
429.01	0.01\\
430.01	0.01\\
431.01	0.01\\
432.01	0.01\\
433.01	0.01\\
434.01	0.01\\
435.01	0.01\\
436.01	0.01\\
437.01	0.01\\
438.01	0.01\\
439.01	0.01\\
440.01	0.01\\
441.01	0.01\\
442.01	0.01\\
443.01	0.01\\
444.01	0.01\\
445.01	0.01\\
446.01	0.01\\
447.01	0.01\\
448.01	0.01\\
449.01	0.01\\
450.01	0.01\\
451.01	0.01\\
452.01	0.01\\
453.01	0.01\\
454.01	0.01\\
455.01	0.01\\
456.01	0.01\\
457.01	0.01\\
458.01	0.01\\
459.01	0.01\\
460.01	0.01\\
461.01	0.01\\
462.01	0.01\\
463.01	0.01\\
464.01	0.01\\
465.01	0.01\\
466.01	0.01\\
467.01	0.01\\
468.01	0.01\\
469.01	0.01\\
470.01	0.01\\
471.01	0.01\\
472.01	0.01\\
473.01	0.01\\
474.01	0.01\\
475.01	0.01\\
476.01	0.01\\
477.01	0.01\\
478.01	0.01\\
479.01	0.01\\
480.01	0.01\\
481.01	0.01\\
482.01	0.01\\
483.01	0.01\\
484.01	0.01\\
485.01	0.01\\
486.01	0.01\\
487.01	0.01\\
488.01	0.01\\
489.01	0.01\\
490.01	0.01\\
491.01	0.01\\
492.01	0.01\\
493.01	0.01\\
494.01	0.01\\
495.01	0.01\\
496.01	0.01\\
497.01	0.01\\
498.01	0.01\\
499.01	0.01\\
500.01	0.01\\
501.01	0.01\\
502.01	0.01\\
503.01	0.01\\
504.01	0.01\\
505.01	0.01\\
506.01	0.01\\
507.01	0.01\\
508.01	0.01\\
509.01	0.01\\
510.01	0.01\\
511.01	0.01\\
512.01	0.01\\
513.01	0.01\\
514.01	0.01\\
515.01	0.01\\
516.01	0.01\\
517.01	0.01\\
518.01	0.01\\
519.01	0.01\\
520.01	0.01\\
521.01	0.01\\
522.01	0.01\\
523.01	0.01\\
524.01	0.01\\
525.01	0.01\\
526.01	0.01\\
527.01	0.01\\
528.01	0.01\\
529.01	0.01\\
530.01	0.01\\
531.01	0.01\\
532.01	0.01\\
533.01	0.01\\
534.01	0.01\\
535.01	0.01\\
536.01	0.01\\
537.01	0.01\\
538.01	0.01\\
539.01	0.01\\
540.01	0.01\\
541.01	0.01\\
542.01	0.01\\
543.01	0.01\\
544.01	0.01\\
545.01	0.01\\
546.01	0.01\\
547.01	0.01\\
548.01	0.01\\
549.01	0.01\\
550.01	0.01\\
551.01	0.01\\
552.01	0.01\\
553.01	0.01\\
554.01	0.01\\
555.01	0.01\\
556.01	0.01\\
557.01	0.01\\
558.01	0.01\\
559.01	0.01\\
560.01	0.01\\
561.01	0.01\\
562.01	0.01\\
563.01	0.01\\
564.01	0.01\\
565.01	0.01\\
566.01	0.01\\
567.01	0.01\\
568.01	0.01\\
569.01	0.01\\
570.01	0.01\\
571.01	0.01\\
572.01	0.01\\
573.01	0.01\\
574.01	0.01\\
575.01	0.01\\
576.01	0.01\\
577.01	0.01\\
578.01	0.01\\
579.01	0.01\\
580.01	0.01\\
581.01	0.01\\
582.01	0.01\\
583.01	0.01\\
584.01	0.01\\
585.01	0.01\\
586.01	0.01\\
587.01	0.01\\
588.01	0.01\\
589.01	0.01\\
590.01	0.01\\
591.01	0.01\\
592.01	0.01\\
593.01	0.01\\
594.01	0.01\\
595.01	0.01\\
596.01	0.01\\
597.01	0.01\\
598.01	0.01\\
599.01	0.01\\
599.02	0.01\\
599.03	0.01\\
599.04	0.01\\
599.05	0.01\\
599.06	0.01\\
599.07	0.01\\
599.08	0.01\\
599.09	0.01\\
599.1	0.01\\
599.11	0.01\\
599.12	0.01\\
599.13	0.01\\
599.14	0.01\\
599.15	0.01\\
599.16	0.01\\
599.17	0.01\\
599.18	0.01\\
599.19	0.01\\
599.2	0.01\\
599.21	0.01\\
599.22	0.01\\
599.23	0.01\\
599.24	0.01\\
599.25	0.01\\
599.26	0.01\\
599.27	0.01\\
599.28	0.01\\
599.29	0.01\\
599.3	0.01\\
599.31	0.01\\
599.32	0.01\\
599.33	0.01\\
599.34	0.01\\
599.35	0.01\\
599.36	0.01\\
599.37	0.01\\
599.38	0.01\\
599.39	0.01\\
599.4	0.01\\
599.41	0.01\\
599.42	0.01\\
599.43	0.01\\
599.44	0.01\\
599.45	0.01\\
599.46	0.01\\
599.47	0.01\\
599.48	0.01\\
599.49	0.01\\
599.5	0.01\\
599.51	0.01\\
599.52	0.01\\
599.53	0.01\\
599.54	0.01\\
599.55	0.01\\
599.56	0.01\\
599.57	0.01\\
599.58	0.01\\
599.59	0.01\\
599.6	0.01\\
599.61	0.01\\
599.62	0.01\\
599.63	0.01\\
599.64	0.01\\
599.65	0.01\\
599.66	0.01\\
599.67	0.01\\
599.68	0.01\\
599.69	0.01\\
599.7	0.01\\
599.71	0.01\\
599.72	0.01\\
599.73	0.01\\
599.74	0.01\\
599.75	0.01\\
599.76	0.01\\
599.77	0.01\\
599.78	0.01\\
599.79	0.01\\
599.8	0.01\\
599.81	0.01\\
599.82	0.01\\
599.83	0.01\\
599.84	0.01\\
599.85	0.01\\
599.86	0.01\\
599.87	0.01\\
599.88	0.01\\
599.89	0.01\\
599.9	0.01\\
599.91	0.01\\
599.92	0.01\\
599.93	0.01\\
599.94	0.01\\
599.95	0.01\\
599.96	0.01\\
599.97	0.01\\
599.98	0.01\\
599.99	0.01\\
600	0.01\\
};
\end{axis}
\end{tikzpicture}%

  \caption{Continuous Time}
\end{subfigure}%
\hfill%
\begin{subfigure}{.45\linewidth}
  \centering
  \setlength\figureheight{\linewidth} 
  \setlength\figurewidth{\linewidth}
  \tikzsetnextfilename{dp_colorbar/dp_dscr_nFPC_z15}
  % This file was created by matlab2tikz.
%
%The latest updates can be retrieved from
%  http://www.mathworks.com/matlabcentral/fileexchange/22022-matlab2tikz-matlab2tikz
%where you can also make suggestions and rate matlab2tikz.
%
\definecolor{mycolor1}{rgb}{0.00000,1.00000,0.14286}%
\definecolor{mycolor2}{rgb}{0.00000,1.00000,0.28571}%
\definecolor{mycolor3}{rgb}{0.00000,1.00000,0.42857}%
\definecolor{mycolor4}{rgb}{0.00000,1.00000,0.57143}%
\definecolor{mycolor5}{rgb}{0.00000,1.00000,0.71429}%
\definecolor{mycolor6}{rgb}{0.00000,1.00000,0.85714}%
\definecolor{mycolor7}{rgb}{0.00000,1.00000,1.00000}%
\definecolor{mycolor8}{rgb}{0.00000,0.87500,1.00000}%
\definecolor{mycolor9}{rgb}{0.00000,0.62500,1.00000}%
\definecolor{mycolor10}{rgb}{0.12500,0.00000,1.00000}%
\definecolor{mycolor11}{rgb}{0.25000,0.00000,1.00000}%
\definecolor{mycolor12}{rgb}{0.37500,0.00000,1.00000}%
\definecolor{mycolor13}{rgb}{0.50000,0.00000,1.00000}%
\definecolor{mycolor14}{rgb}{0.62500,0.00000,1.00000}%
\definecolor{mycolor15}{rgb}{0.75000,0.00000,1.00000}%
\definecolor{mycolor16}{rgb}{0.87500,0.00000,1.00000}%
\definecolor{mycolor17}{rgb}{1.00000,0.00000,1.00000}%
\definecolor{mycolor18}{rgb}{1.00000,0.00000,0.87500}%
\definecolor{mycolor19}{rgb}{1.00000,0.00000,0.62500}%
\definecolor{mycolor20}{rgb}{0.85714,0.00000,0.00000}%
\definecolor{mycolor21}{rgb}{0.71429,0.00000,0.00000}%
%
\begin{tikzpicture}[trim axis left, trim axis right]

\begin{axis}[%
width=\figurewidth,
height=\figureheight,
at={(0\figurewidth,0\figureheight)},
scale only axis,
every outer x axis line/.append style={black},
every x tick label/.append style={font=\color{black}},
xmin=0,
xmax=600,
every outer y axis line/.append style={black},
every y tick label/.append style={font=\color{black}},
ymin=0,
ymax=0.014,
axis background/.style={fill=white},
axis x line*=bottom,
axis y line*=left,
yticklabel style={
        /pgf/number format/fixed,
        /pgf/number format/precision=3
},
scaled y ticks=false
]
\addplot [color=green,solid,forget plot]
  table[row sep=crcr]{%
1	0\\
2	0\\
3	0\\
4	0\\
5	0\\
6	0\\
7	0\\
8	0\\
9	0\\
10	0\\
11	0\\
12	0\\
13	0\\
14	0\\
15	0\\
16	0\\
17	0\\
18	0\\
19	0\\
20	0\\
21	0\\
22	0\\
23	0\\
24	0\\
25	0\\
26	0\\
27	0\\
28	0\\
29	0\\
30	0\\
31	0\\
32	0\\
33	0\\
34	0\\
35	0\\
36	0\\
37	0\\
38	0\\
39	0\\
40	0\\
41	0\\
42	0\\
43	0\\
44	0\\
45	0\\
46	0\\
47	0\\
48	0\\
49	0\\
50	0\\
51	0\\
52	0\\
53	0\\
54	0\\
55	0\\
56	0\\
57	0\\
58	0\\
59	0\\
60	0\\
61	0\\
62	0\\
63	0\\
64	0\\
65	0\\
66	0\\
67	0\\
68	0\\
69	0\\
70	0\\
71	0\\
72	0\\
73	0\\
74	0\\
75	0\\
76	0\\
77	0\\
78	0\\
79	0\\
80	0\\
81	0\\
82	0\\
83	0\\
84	0\\
85	0\\
86	0\\
87	0\\
88	0\\
89	0\\
90	0\\
91	0\\
92	0\\
93	0\\
94	0\\
95	0\\
96	0\\
97	0\\
98	0\\
99	0\\
100	0\\
101	0\\
102	0\\
103	0\\
104	0\\
105	0\\
106	0\\
107	0\\
108	0\\
109	0\\
110	0\\
111	0\\
112	0\\
113	0\\
114	0\\
115	0\\
116	0\\
117	0\\
118	0\\
119	0\\
120	0\\
121	0\\
122	0\\
123	0\\
124	0\\
125	0\\
126	0\\
127	0\\
128	0\\
129	0\\
130	0\\
131	0\\
132	0\\
133	0\\
134	0\\
135	0\\
136	0\\
137	0\\
138	0\\
139	0\\
140	0\\
141	0\\
142	0\\
143	0\\
144	0\\
145	0\\
146	0\\
147	0\\
148	0\\
149	0\\
150	0\\
151	0\\
152	0\\
153	0\\
154	0\\
155	0\\
156	0\\
157	0\\
158	0\\
159	0\\
160	0\\
161	0\\
162	0\\
163	0\\
164	0\\
165	0\\
166	0\\
167	0\\
168	0\\
169	0\\
170	0\\
171	0\\
172	0\\
173	0\\
174	0\\
175	0\\
176	0\\
177	0\\
178	0\\
179	0\\
180	0\\
181	0\\
182	0\\
183	0\\
184	0\\
185	0\\
186	0\\
187	0\\
188	0\\
189	0\\
190	0\\
191	0\\
192	0\\
193	0\\
194	0\\
195	0\\
196	0\\
197	0\\
198	0\\
199	0\\
200	0\\
201	0\\
202	0\\
203	0\\
204	0\\
205	0\\
206	0\\
207	0\\
208	0\\
209	0\\
210	0\\
211	0\\
212	0\\
213	0\\
214	0\\
215	0\\
216	0\\
217	0\\
218	0\\
219	0\\
220	0\\
221	0\\
222	0\\
223	0\\
224	0\\
225	0\\
226	0\\
227	0\\
228	0\\
229	0\\
230	0\\
231	0\\
232	0\\
233	0\\
234	0\\
235	0\\
236	0\\
237	0\\
238	0\\
239	0\\
240	0\\
241	0\\
242	0\\
243	0\\
244	0\\
245	0\\
246	0\\
247	0\\
248	0\\
249	0\\
250	0\\
251	0\\
252	0\\
253	0\\
254	0\\
255	0\\
256	0\\
257	0\\
258	0\\
259	0\\
260	0\\
261	0\\
262	0\\
263	0\\
264	0\\
265	0\\
266	0\\
267	0\\
268	0\\
269	0\\
270	0\\
271	0\\
272	0\\
273	0\\
274	0\\
275	0\\
276	0\\
277	0\\
278	0\\
279	0\\
280	0\\
281	0\\
282	0\\
283	0\\
284	0\\
285	0\\
286	0\\
287	0\\
288	0\\
289	0\\
290	0\\
291	0\\
292	0\\
293	0\\
294	0\\
295	0\\
296	0\\
297	0\\
298	0\\
299	0\\
300	0\\
301	0\\
302	0\\
303	0\\
304	0\\
305	0\\
306	0\\
307	0\\
308	0\\
309	0\\
310	0\\
311	0\\
312	0\\
313	0\\
314	0\\
315	0\\
316	0\\
317	0\\
318	0\\
319	0\\
320	0\\
321	0\\
322	0\\
323	0\\
324	0\\
325	0\\
326	0\\
327	0\\
328	0\\
329	0\\
330	0\\
331	0\\
332	0\\
333	0\\
334	0\\
335	0\\
336	0\\
337	0\\
338	0\\
339	0\\
340	0\\
341	0\\
342	0\\
343	0\\
344	0\\
345	0\\
346	0\\
347	0\\
348	0\\
349	0\\
350	0\\
351	0\\
352	0\\
353	0\\
354	0\\
355	0\\
356	0\\
357	0\\
358	0\\
359	0\\
360	0\\
361	0\\
362	0\\
363	0\\
364	0\\
365	0\\
366	0\\
367	0\\
368	0\\
369	0\\
370	0\\
371	0\\
372	0\\
373	0\\
374	0\\
375	0\\
376	0\\
377	0\\
378	0\\
379	0\\
380	0\\
381	0\\
382	0\\
383	0\\
384	0\\
385	0\\
386	0\\
387	0\\
388	0\\
389	0\\
390	0\\
391	0\\
392	0\\
393	0\\
394	0\\
395	0\\
396	0\\
397	0\\
398	0\\
399	0\\
400	0\\
401	0\\
402	0\\
403	0\\
404	0\\
405	0\\
406	0\\
407	0\\
408	0\\
409	0\\
410	0\\
411	0\\
412	0\\
413	0\\
414	0\\
415	0\\
416	0\\
417	0\\
418	0\\
419	0\\
420	0\\
421	0\\
422	0\\
423	0\\
424	0\\
425	0\\
426	0\\
427	0\\
428	0\\
429	0\\
430	0\\
431	0\\
432	0\\
433	0\\
434	0\\
435	0\\
436	0\\
437	0\\
438	0\\
439	0\\
440	0\\
441	0\\
442	0\\
443	0\\
444	0\\
445	0\\
446	0\\
447	0\\
448	0\\
449	0\\
450	0\\
451	0\\
452	0\\
453	0\\
454	0\\
455	0\\
456	0\\
457	0\\
458	0\\
459	0\\
460	0\\
461	0\\
462	0\\
463	0\\
464	0\\
465	0\\
466	0\\
467	0\\
468	0\\
469	0\\
470	0\\
471	0\\
472	0\\
473	0\\
474	0\\
475	0\\
476	0\\
477	0\\
478	0\\
479	0\\
480	0\\
481	0\\
482	0\\
483	0\\
484	0\\
485	0\\
486	0\\
487	0\\
488	0\\
489	0\\
490	0\\
491	0\\
492	0\\
493	0\\
494	0\\
495	0\\
496	0\\
497	0\\
498	0\\
499	0\\
500	0\\
501	0\\
502	0\\
503	0\\
504	0\\
505	0\\
506	0\\
507	0\\
508	0\\
509	0\\
510	0\\
511	0\\
512	0\\
513	0\\
514	0\\
515	0\\
516	0\\
517	0\\
518	0\\
519	0\\
520	0\\
521	0\\
522	0\\
523	0\\
524	0\\
525	0\\
526	0\\
527	0\\
528	0\\
529	0\\
530	0\\
531	0\\
532	0\\
533	0\\
534	0\\
535	0\\
536	0\\
537	0\\
538	0\\
539	0\\
540	0\\
541	0\\
542	0\\
543	0\\
544	0\\
545	0\\
546	0\\
547	0\\
548	0\\
549	0\\
550	0\\
551	0\\
552	0\\
553	0\\
554	0\\
555	0\\
556	0\\
557	0\\
558	0\\
559	0\\
560	0\\
561	0\\
562	0\\
563	0\\
564	0\\
565	0\\
566	0\\
567	0\\
568	0\\
569	0\\
570	0\\
571	0\\
572	0\\
573	0\\
574	0\\
575	0\\
576	0\\
577	0\\
578	0\\
579	0\\
580	0\\
581	0\\
582	0\\
583	0\\
584	0\\
585	0\\
586	0\\
587	0\\
588	0\\
589	0\\
590	0\\
591	0\\
592	0\\
593	0\\
594	0\\
595	0\\
596	0\\
597	0\\
598	0\\
599	0\\
600	0\\
};
\addplot [color=mycolor1,solid,forget plot]
  table[row sep=crcr]{%
1	0\\
2	0\\
3	0\\
4	0\\
5	0\\
6	0\\
7	0\\
8	0\\
9	0\\
10	0\\
11	0\\
12	0\\
13	0\\
14	0\\
15	0\\
16	0\\
17	0\\
18	0\\
19	0\\
20	0\\
21	0\\
22	0\\
23	0\\
24	0\\
25	0\\
26	0\\
27	0\\
28	0\\
29	0\\
30	0\\
31	0\\
32	0\\
33	0\\
34	0\\
35	0\\
36	0\\
37	0\\
38	0\\
39	0\\
40	0\\
41	0\\
42	0\\
43	0\\
44	0\\
45	0\\
46	0\\
47	0\\
48	0\\
49	0\\
50	0\\
51	0\\
52	0\\
53	0\\
54	0\\
55	0\\
56	0\\
57	0\\
58	0\\
59	0\\
60	0\\
61	0\\
62	0\\
63	0\\
64	0\\
65	0\\
66	0\\
67	0\\
68	0\\
69	0\\
70	0\\
71	0\\
72	0\\
73	0\\
74	0\\
75	0\\
76	0\\
77	0\\
78	0\\
79	0\\
80	0\\
81	0\\
82	0\\
83	0\\
84	0\\
85	0\\
86	0\\
87	0\\
88	0\\
89	0\\
90	0\\
91	0\\
92	0\\
93	0\\
94	0\\
95	0\\
96	0\\
97	0\\
98	0\\
99	0\\
100	0\\
101	0\\
102	0\\
103	0\\
104	0\\
105	0\\
106	0\\
107	0\\
108	0\\
109	0\\
110	0\\
111	0\\
112	0\\
113	0\\
114	0\\
115	0\\
116	0\\
117	0\\
118	0\\
119	0\\
120	0\\
121	0\\
122	0\\
123	0\\
124	0\\
125	0\\
126	0\\
127	0\\
128	0\\
129	0\\
130	0\\
131	0\\
132	0\\
133	0\\
134	0\\
135	0\\
136	0\\
137	0\\
138	0\\
139	0\\
140	0\\
141	0\\
142	0\\
143	0\\
144	0\\
145	0\\
146	0\\
147	0\\
148	0\\
149	0\\
150	0\\
151	0\\
152	0\\
153	0\\
154	0\\
155	0\\
156	0\\
157	0\\
158	0\\
159	0\\
160	0\\
161	0\\
162	0\\
163	0\\
164	0\\
165	0\\
166	0\\
167	0\\
168	0\\
169	0\\
170	0\\
171	0\\
172	0\\
173	0\\
174	0\\
175	0\\
176	0\\
177	0\\
178	0\\
179	0\\
180	0\\
181	0\\
182	0\\
183	0\\
184	0\\
185	0\\
186	0\\
187	0\\
188	0\\
189	0\\
190	0\\
191	0\\
192	0\\
193	0\\
194	0\\
195	0\\
196	0\\
197	0\\
198	0\\
199	0\\
200	0\\
201	0\\
202	0\\
203	0\\
204	0\\
205	0\\
206	0\\
207	0\\
208	0\\
209	0\\
210	0\\
211	0\\
212	0\\
213	0\\
214	0\\
215	0\\
216	0\\
217	0\\
218	0\\
219	0\\
220	0\\
221	0\\
222	0\\
223	0\\
224	0\\
225	0\\
226	0\\
227	0\\
228	0\\
229	0\\
230	0\\
231	0\\
232	0\\
233	0\\
234	0\\
235	0\\
236	0\\
237	0\\
238	0\\
239	0\\
240	0\\
241	0\\
242	0\\
243	0\\
244	0\\
245	0\\
246	0\\
247	0\\
248	0\\
249	0\\
250	0\\
251	0\\
252	0\\
253	0\\
254	0\\
255	0\\
256	0\\
257	0\\
258	0\\
259	0\\
260	0\\
261	0\\
262	0\\
263	0\\
264	0\\
265	0\\
266	0\\
267	0\\
268	0\\
269	0\\
270	0\\
271	0\\
272	0\\
273	0\\
274	0\\
275	0\\
276	0\\
277	0\\
278	0\\
279	0\\
280	0\\
281	0\\
282	0\\
283	0\\
284	0\\
285	0\\
286	0\\
287	0\\
288	0\\
289	0\\
290	0\\
291	0\\
292	0\\
293	0\\
294	0\\
295	0\\
296	0\\
297	0\\
298	0\\
299	0\\
300	0\\
301	0\\
302	0\\
303	0\\
304	0\\
305	0\\
306	0\\
307	0\\
308	0\\
309	0\\
310	0\\
311	0\\
312	0\\
313	0\\
314	0\\
315	0\\
316	0\\
317	0\\
318	0\\
319	0\\
320	0\\
321	0\\
322	0\\
323	0\\
324	0\\
325	0\\
326	0\\
327	0\\
328	0\\
329	0\\
330	0\\
331	0\\
332	0\\
333	0\\
334	0\\
335	0\\
336	0\\
337	0\\
338	0\\
339	0\\
340	0\\
341	0\\
342	0\\
343	0\\
344	0\\
345	0\\
346	0\\
347	0\\
348	0\\
349	0\\
350	0\\
351	0\\
352	0\\
353	0\\
354	0\\
355	0\\
356	0\\
357	0\\
358	0\\
359	0\\
360	0\\
361	0\\
362	0\\
363	0\\
364	0\\
365	0\\
366	0\\
367	0\\
368	0\\
369	0\\
370	0\\
371	0\\
372	0\\
373	0\\
374	0\\
375	0\\
376	0\\
377	0\\
378	0\\
379	0\\
380	0\\
381	0\\
382	0\\
383	0\\
384	0\\
385	0\\
386	0\\
387	0\\
388	0\\
389	0\\
390	0\\
391	0\\
392	0\\
393	0\\
394	0\\
395	0\\
396	0\\
397	0\\
398	0\\
399	0\\
400	0\\
401	0\\
402	0\\
403	0\\
404	0\\
405	0\\
406	0\\
407	0\\
408	0\\
409	0\\
410	0\\
411	0\\
412	0\\
413	0\\
414	0\\
415	0\\
416	0\\
417	0\\
418	0\\
419	0\\
420	0\\
421	0\\
422	0\\
423	0\\
424	0\\
425	0\\
426	0\\
427	0\\
428	0\\
429	0\\
430	0\\
431	0\\
432	0\\
433	0\\
434	0\\
435	0\\
436	0\\
437	0\\
438	0\\
439	0\\
440	0\\
441	0\\
442	0\\
443	0\\
444	0\\
445	0\\
446	0\\
447	0\\
448	0\\
449	0\\
450	0\\
451	0\\
452	0\\
453	0\\
454	0\\
455	0\\
456	0\\
457	0\\
458	0\\
459	0\\
460	0\\
461	0\\
462	0\\
463	0\\
464	0\\
465	0\\
466	0\\
467	0\\
468	0\\
469	0\\
470	0\\
471	0\\
472	0\\
473	0\\
474	0\\
475	0\\
476	0\\
477	0\\
478	0\\
479	0\\
480	0\\
481	0\\
482	0\\
483	0\\
484	0\\
485	0\\
486	0\\
487	0\\
488	0\\
489	0\\
490	0\\
491	0\\
492	0\\
493	0\\
494	0\\
495	0\\
496	0\\
497	0\\
498	0\\
499	0\\
500	0\\
501	0\\
502	0\\
503	0\\
504	0\\
505	0\\
506	0\\
507	0\\
508	0\\
509	0\\
510	0\\
511	0\\
512	0\\
513	0\\
514	0\\
515	0\\
516	0\\
517	0\\
518	0\\
519	0\\
520	0\\
521	0\\
522	0\\
523	0\\
524	0\\
525	0\\
526	0\\
527	0\\
528	0\\
529	0\\
530	0\\
531	0\\
532	0\\
533	0\\
534	0\\
535	0\\
536	0\\
537	0\\
538	0\\
539	0\\
540	0\\
541	0\\
542	0\\
543	0\\
544	0\\
545	0\\
546	0\\
547	0\\
548	0\\
549	0\\
550	0\\
551	0\\
552	0\\
553	0\\
554	0\\
555	0\\
556	0\\
557	0\\
558	0\\
559	0\\
560	0\\
561	0\\
562	0\\
563	0\\
564	0\\
565	0\\
566	0\\
567	0\\
568	0\\
569	0\\
570	0\\
571	0\\
572	0\\
573	0\\
574	0\\
575	0\\
576	0\\
577	0\\
578	0\\
579	0\\
580	0\\
581	0\\
582	0\\
583	0\\
584	0\\
585	0\\
586	0\\
587	0\\
588	0\\
589	0\\
590	0\\
591	0\\
592	0\\
593	0\\
594	0\\
595	0\\
596	0\\
597	0\\
598	0\\
599	0\\
600	0\\
};
\addplot [color=mycolor2,solid,forget plot]
  table[row sep=crcr]{%
1	0\\
2	0\\
3	0\\
4	0\\
5	0\\
6	0\\
7	0\\
8	0\\
9	0\\
10	0\\
11	0\\
12	0\\
13	0\\
14	0\\
15	0\\
16	0\\
17	0\\
18	0\\
19	0\\
20	0\\
21	0\\
22	0\\
23	0\\
24	0\\
25	0\\
26	0\\
27	0\\
28	0\\
29	0\\
30	0\\
31	0\\
32	0\\
33	0\\
34	0\\
35	0\\
36	0\\
37	0\\
38	0\\
39	0\\
40	0\\
41	0\\
42	0\\
43	0\\
44	0\\
45	0\\
46	0\\
47	0\\
48	0\\
49	0\\
50	0\\
51	0\\
52	0\\
53	0\\
54	0\\
55	0\\
56	0\\
57	0\\
58	0\\
59	0\\
60	0\\
61	0\\
62	0\\
63	0\\
64	0\\
65	0\\
66	0\\
67	0\\
68	0\\
69	0\\
70	0\\
71	0\\
72	0\\
73	0\\
74	0\\
75	0\\
76	0\\
77	0\\
78	0\\
79	0\\
80	0\\
81	0\\
82	0\\
83	0\\
84	0\\
85	0\\
86	0\\
87	0\\
88	0\\
89	0\\
90	0\\
91	0\\
92	0\\
93	0\\
94	0\\
95	0\\
96	0\\
97	0\\
98	0\\
99	0\\
100	0\\
101	0\\
102	0\\
103	0\\
104	0\\
105	0\\
106	0\\
107	0\\
108	0\\
109	0\\
110	0\\
111	0\\
112	0\\
113	0\\
114	0\\
115	0\\
116	0\\
117	0\\
118	0\\
119	0\\
120	0\\
121	0\\
122	0\\
123	0\\
124	0\\
125	0\\
126	0\\
127	0\\
128	0\\
129	0\\
130	0\\
131	0\\
132	0\\
133	0\\
134	0\\
135	0\\
136	0\\
137	0\\
138	0\\
139	0\\
140	0\\
141	0\\
142	0\\
143	0\\
144	0\\
145	0\\
146	0\\
147	0\\
148	0\\
149	0\\
150	0\\
151	0\\
152	0\\
153	0\\
154	0\\
155	0\\
156	0\\
157	0\\
158	0\\
159	0\\
160	0\\
161	0\\
162	0\\
163	0\\
164	0\\
165	0\\
166	0\\
167	0\\
168	0\\
169	0\\
170	0\\
171	0\\
172	0\\
173	0\\
174	0\\
175	0\\
176	0\\
177	0\\
178	0\\
179	0\\
180	0\\
181	0\\
182	0\\
183	0\\
184	0\\
185	0\\
186	0\\
187	0\\
188	0\\
189	0\\
190	0\\
191	0\\
192	0\\
193	0\\
194	0\\
195	0\\
196	0\\
197	0\\
198	0\\
199	0\\
200	0\\
201	0\\
202	0\\
203	0\\
204	0\\
205	0\\
206	0\\
207	0\\
208	0\\
209	0\\
210	0\\
211	0\\
212	0\\
213	0\\
214	0\\
215	0\\
216	0\\
217	0\\
218	0\\
219	0\\
220	0\\
221	0\\
222	0\\
223	0\\
224	0\\
225	0\\
226	0\\
227	0\\
228	0\\
229	0\\
230	0\\
231	0\\
232	0\\
233	0\\
234	0\\
235	0\\
236	0\\
237	0\\
238	0\\
239	0\\
240	0\\
241	0\\
242	0\\
243	0\\
244	0\\
245	0\\
246	0\\
247	0\\
248	0\\
249	0\\
250	0\\
251	0\\
252	0\\
253	0\\
254	0\\
255	0\\
256	0\\
257	0\\
258	0\\
259	0\\
260	0\\
261	0\\
262	0\\
263	0\\
264	0\\
265	0\\
266	0\\
267	0\\
268	0\\
269	0\\
270	0\\
271	0\\
272	0\\
273	0\\
274	0\\
275	0\\
276	0\\
277	0\\
278	0\\
279	0\\
280	0\\
281	0\\
282	0\\
283	0\\
284	0\\
285	0\\
286	0\\
287	0\\
288	0\\
289	0\\
290	0\\
291	0\\
292	0\\
293	0\\
294	0\\
295	0\\
296	0\\
297	0\\
298	0\\
299	0\\
300	0\\
301	0\\
302	0\\
303	0\\
304	0\\
305	0\\
306	0\\
307	0\\
308	0\\
309	0\\
310	0\\
311	0\\
312	0\\
313	0\\
314	0\\
315	0\\
316	0\\
317	0\\
318	0\\
319	0\\
320	0\\
321	0\\
322	0\\
323	0\\
324	0\\
325	0\\
326	0\\
327	0\\
328	0\\
329	0\\
330	0\\
331	0\\
332	0\\
333	0\\
334	0\\
335	0\\
336	0\\
337	0\\
338	0\\
339	0\\
340	0\\
341	0\\
342	0\\
343	0\\
344	0\\
345	0\\
346	0\\
347	0\\
348	0\\
349	0\\
350	0\\
351	0\\
352	0\\
353	0\\
354	0\\
355	0\\
356	0\\
357	0\\
358	0\\
359	0\\
360	0\\
361	0\\
362	0\\
363	0\\
364	0\\
365	0\\
366	0\\
367	0\\
368	0\\
369	0\\
370	0\\
371	0\\
372	0\\
373	0\\
374	0\\
375	0\\
376	0\\
377	0\\
378	0\\
379	0\\
380	0\\
381	0\\
382	0\\
383	0\\
384	0\\
385	0\\
386	0\\
387	0\\
388	0\\
389	0\\
390	0\\
391	0\\
392	0\\
393	0\\
394	0\\
395	0\\
396	0\\
397	0\\
398	0\\
399	0\\
400	0\\
401	0\\
402	0\\
403	0\\
404	0\\
405	0\\
406	0\\
407	0\\
408	0\\
409	0\\
410	0\\
411	0\\
412	0\\
413	0\\
414	0\\
415	0\\
416	0\\
417	0\\
418	0\\
419	0\\
420	0\\
421	0\\
422	0\\
423	0\\
424	0\\
425	0\\
426	0\\
427	0\\
428	0\\
429	0\\
430	0\\
431	0\\
432	0\\
433	0\\
434	0\\
435	0\\
436	0\\
437	0\\
438	0\\
439	0\\
440	0\\
441	0\\
442	0\\
443	0\\
444	0\\
445	0\\
446	0\\
447	0\\
448	0\\
449	0\\
450	0\\
451	0\\
452	0\\
453	0\\
454	0\\
455	0\\
456	0\\
457	0\\
458	0\\
459	0\\
460	0\\
461	0\\
462	0\\
463	0\\
464	0\\
465	0\\
466	0\\
467	0\\
468	0\\
469	0\\
470	0\\
471	0\\
472	0\\
473	0\\
474	0\\
475	0\\
476	0\\
477	0\\
478	0\\
479	0\\
480	0\\
481	0\\
482	0\\
483	0\\
484	0\\
485	0\\
486	0\\
487	0\\
488	0\\
489	0\\
490	0\\
491	0\\
492	0\\
493	0\\
494	0\\
495	0\\
496	0\\
497	0\\
498	0\\
499	0\\
500	0\\
501	0\\
502	0\\
503	0\\
504	0\\
505	0\\
506	0\\
507	0\\
508	0\\
509	0\\
510	0\\
511	0\\
512	0\\
513	0\\
514	0\\
515	0\\
516	0\\
517	0\\
518	0\\
519	0\\
520	0\\
521	0\\
522	0\\
523	0\\
524	0\\
525	0\\
526	0\\
527	0\\
528	0\\
529	0\\
530	0\\
531	0\\
532	0\\
533	0\\
534	0\\
535	0\\
536	0\\
537	0\\
538	0\\
539	0\\
540	0\\
541	0\\
542	0\\
543	0\\
544	0\\
545	0\\
546	0\\
547	0\\
548	0\\
549	0\\
550	0\\
551	0\\
552	0\\
553	0\\
554	0\\
555	0\\
556	0\\
557	0\\
558	0\\
559	0\\
560	0\\
561	0\\
562	0\\
563	0\\
564	0\\
565	0\\
566	0\\
567	0\\
568	0\\
569	0\\
570	0\\
571	0\\
572	0\\
573	0\\
574	0\\
575	0\\
576	0\\
577	0\\
578	0\\
579	0\\
580	0\\
581	0\\
582	0\\
583	0\\
584	0\\
585	0\\
586	0\\
587	0\\
588	0\\
589	0\\
590	0\\
591	0\\
592	0\\
593	0\\
594	0\\
595	0\\
596	0\\
597	0\\
598	0\\
599	0\\
600	0\\
};
\addplot [color=mycolor3,solid,forget plot]
  table[row sep=crcr]{%
1	0\\
2	0\\
3	0\\
4	0\\
5	0\\
6	0\\
7	0\\
8	0\\
9	0\\
10	0\\
11	0\\
12	0\\
13	0\\
14	0\\
15	0\\
16	0\\
17	0\\
18	0\\
19	0\\
20	0\\
21	0\\
22	0\\
23	0\\
24	0\\
25	0\\
26	0\\
27	0\\
28	0\\
29	0\\
30	0\\
31	0\\
32	0\\
33	0\\
34	0\\
35	0\\
36	0\\
37	0\\
38	0\\
39	0\\
40	0\\
41	0\\
42	0\\
43	0\\
44	0\\
45	0\\
46	0\\
47	0\\
48	0\\
49	0\\
50	0\\
51	0\\
52	0\\
53	0\\
54	0\\
55	0\\
56	0\\
57	0\\
58	0\\
59	0\\
60	0\\
61	0\\
62	0\\
63	0\\
64	0\\
65	0\\
66	0\\
67	0\\
68	0\\
69	0\\
70	0\\
71	0\\
72	0\\
73	0\\
74	0\\
75	0\\
76	0\\
77	0\\
78	0\\
79	0\\
80	0\\
81	0\\
82	0\\
83	0\\
84	0\\
85	0\\
86	0\\
87	0\\
88	0\\
89	0\\
90	0\\
91	0\\
92	0\\
93	0\\
94	0\\
95	0\\
96	0\\
97	0\\
98	0\\
99	0\\
100	0\\
101	0\\
102	0\\
103	0\\
104	0\\
105	0\\
106	0\\
107	0\\
108	0\\
109	0\\
110	0\\
111	0\\
112	0\\
113	0\\
114	0\\
115	0\\
116	0\\
117	0\\
118	0\\
119	0\\
120	0\\
121	0\\
122	0\\
123	0\\
124	0\\
125	0\\
126	0\\
127	0\\
128	0\\
129	0\\
130	0\\
131	0\\
132	0\\
133	0\\
134	0\\
135	0\\
136	0\\
137	0\\
138	0\\
139	0\\
140	0\\
141	0\\
142	0\\
143	0\\
144	0\\
145	0\\
146	0\\
147	0\\
148	0\\
149	0\\
150	0\\
151	0\\
152	0\\
153	0\\
154	0\\
155	0\\
156	0\\
157	0\\
158	0\\
159	0\\
160	0\\
161	0\\
162	0\\
163	0\\
164	0\\
165	0\\
166	0\\
167	0\\
168	0\\
169	0\\
170	0\\
171	0\\
172	0\\
173	0\\
174	0\\
175	0\\
176	0\\
177	0\\
178	0\\
179	0\\
180	0\\
181	0\\
182	0\\
183	0\\
184	0\\
185	0\\
186	0\\
187	0\\
188	0\\
189	0\\
190	0\\
191	0\\
192	0\\
193	0\\
194	0\\
195	0\\
196	0\\
197	0\\
198	0\\
199	0\\
200	0\\
201	0\\
202	0\\
203	0\\
204	0\\
205	0\\
206	0\\
207	0\\
208	0\\
209	0\\
210	0\\
211	0\\
212	0\\
213	0\\
214	0\\
215	0\\
216	0\\
217	0\\
218	0\\
219	0\\
220	0\\
221	0\\
222	0\\
223	0\\
224	0\\
225	0\\
226	0\\
227	0\\
228	0\\
229	0\\
230	0\\
231	0\\
232	0\\
233	0\\
234	0\\
235	0\\
236	0\\
237	0\\
238	0\\
239	0\\
240	0\\
241	0\\
242	0\\
243	0\\
244	0\\
245	0\\
246	0\\
247	0\\
248	0\\
249	0\\
250	0\\
251	0\\
252	0\\
253	0\\
254	0\\
255	0\\
256	0\\
257	0\\
258	0\\
259	0\\
260	0\\
261	0\\
262	0\\
263	0\\
264	0\\
265	0\\
266	0\\
267	0\\
268	0\\
269	0\\
270	0\\
271	0\\
272	0\\
273	0\\
274	0\\
275	0\\
276	0\\
277	0\\
278	0\\
279	0\\
280	0\\
281	0\\
282	0\\
283	0\\
284	0\\
285	0\\
286	0\\
287	0\\
288	0\\
289	0\\
290	0\\
291	0\\
292	0\\
293	0\\
294	0\\
295	0\\
296	0\\
297	0\\
298	0\\
299	0\\
300	0\\
301	0\\
302	0\\
303	0\\
304	0\\
305	0\\
306	0\\
307	0\\
308	0\\
309	0\\
310	0\\
311	0\\
312	0\\
313	0\\
314	0\\
315	0\\
316	0\\
317	0\\
318	0\\
319	0\\
320	0\\
321	0\\
322	0\\
323	0\\
324	0\\
325	0\\
326	0\\
327	0\\
328	0\\
329	0\\
330	0\\
331	0\\
332	0\\
333	0\\
334	0\\
335	0\\
336	0\\
337	0\\
338	0\\
339	0\\
340	0\\
341	0\\
342	0\\
343	0\\
344	0\\
345	0\\
346	0\\
347	0\\
348	0\\
349	0\\
350	0\\
351	0\\
352	0\\
353	0\\
354	0\\
355	0\\
356	0\\
357	0\\
358	0\\
359	0\\
360	0\\
361	0\\
362	0\\
363	0\\
364	0\\
365	0\\
366	0\\
367	0\\
368	0\\
369	0\\
370	0\\
371	0\\
372	0\\
373	0\\
374	0\\
375	0\\
376	0\\
377	0\\
378	0\\
379	0\\
380	0\\
381	0\\
382	0\\
383	0\\
384	0\\
385	0\\
386	0\\
387	0\\
388	0\\
389	0\\
390	0\\
391	0\\
392	0\\
393	0\\
394	0\\
395	0\\
396	0\\
397	0\\
398	0\\
399	0\\
400	0\\
401	0\\
402	0\\
403	0\\
404	0\\
405	0\\
406	0\\
407	0\\
408	0\\
409	0\\
410	0\\
411	0\\
412	0\\
413	0\\
414	0\\
415	0\\
416	0\\
417	0\\
418	0\\
419	0\\
420	0\\
421	0\\
422	0\\
423	0\\
424	0\\
425	0\\
426	0\\
427	0\\
428	0\\
429	0\\
430	0\\
431	0\\
432	0\\
433	0\\
434	0\\
435	0\\
436	0\\
437	0\\
438	0\\
439	0\\
440	0\\
441	0\\
442	0\\
443	0\\
444	0\\
445	0\\
446	0\\
447	0\\
448	0\\
449	0\\
450	0\\
451	0\\
452	0\\
453	0\\
454	0\\
455	0\\
456	0\\
457	0\\
458	0\\
459	0\\
460	0\\
461	0\\
462	0\\
463	0\\
464	0\\
465	0\\
466	0\\
467	0\\
468	0\\
469	0\\
470	0\\
471	0\\
472	0\\
473	0\\
474	0\\
475	0\\
476	0\\
477	0\\
478	0\\
479	0\\
480	0\\
481	0\\
482	0\\
483	0\\
484	0\\
485	0\\
486	0\\
487	0\\
488	0\\
489	0\\
490	0\\
491	0\\
492	0\\
493	0\\
494	0\\
495	0\\
496	0\\
497	0\\
498	0\\
499	0\\
500	0\\
501	0\\
502	0\\
503	0\\
504	0\\
505	0\\
506	0\\
507	0\\
508	0\\
509	0\\
510	0\\
511	0\\
512	0\\
513	0\\
514	0\\
515	0\\
516	0\\
517	0\\
518	0\\
519	0\\
520	0\\
521	0\\
522	0\\
523	0\\
524	0\\
525	0\\
526	0\\
527	0\\
528	0\\
529	0\\
530	0\\
531	0\\
532	0\\
533	0\\
534	0\\
535	0\\
536	0\\
537	0\\
538	0\\
539	0\\
540	0\\
541	0\\
542	0\\
543	0\\
544	0\\
545	0\\
546	0\\
547	0\\
548	0\\
549	0\\
550	0\\
551	0\\
552	0\\
553	0\\
554	0\\
555	0\\
556	0\\
557	0\\
558	0\\
559	0\\
560	0\\
561	0\\
562	0\\
563	0\\
564	0\\
565	0\\
566	0\\
567	0\\
568	0\\
569	0\\
570	0\\
571	0\\
572	0\\
573	0\\
574	0\\
575	0\\
576	0\\
577	0\\
578	0\\
579	0\\
580	0\\
581	0\\
582	0\\
583	0\\
584	0\\
585	0\\
586	0\\
587	0\\
588	0\\
589	0\\
590	0\\
591	0\\
592	0\\
593	0\\
594	0\\
595	0\\
596	0\\
597	0\\
598	0\\
599	0\\
600	0\\
};
\addplot [color=mycolor4,solid,forget plot]
  table[row sep=crcr]{%
1	0\\
2	0\\
3	0\\
4	0\\
5	0\\
6	0\\
7	0\\
8	0\\
9	0\\
10	0\\
11	0\\
12	0\\
13	0\\
14	0\\
15	0\\
16	0\\
17	0\\
18	0\\
19	0\\
20	0\\
21	0\\
22	0\\
23	0\\
24	0\\
25	0\\
26	0\\
27	0\\
28	0\\
29	0\\
30	0\\
31	0\\
32	0\\
33	0\\
34	0\\
35	0\\
36	0\\
37	0\\
38	0\\
39	0\\
40	0\\
41	0\\
42	0\\
43	0\\
44	0\\
45	0\\
46	0\\
47	0\\
48	0\\
49	0\\
50	0\\
51	0\\
52	0\\
53	0\\
54	0\\
55	0\\
56	0\\
57	0\\
58	0\\
59	0\\
60	0\\
61	0\\
62	0\\
63	0\\
64	0\\
65	0\\
66	0\\
67	0\\
68	0\\
69	0\\
70	0\\
71	0\\
72	0\\
73	0\\
74	0\\
75	0\\
76	0\\
77	0\\
78	0\\
79	0\\
80	0\\
81	0\\
82	0\\
83	0\\
84	0\\
85	0\\
86	0\\
87	0\\
88	0\\
89	0\\
90	0\\
91	0\\
92	0\\
93	0\\
94	0\\
95	0\\
96	0\\
97	0\\
98	0\\
99	0\\
100	0\\
101	0\\
102	0\\
103	0\\
104	0\\
105	0\\
106	0\\
107	0\\
108	0\\
109	0\\
110	0\\
111	0\\
112	0\\
113	0\\
114	0\\
115	0\\
116	0\\
117	0\\
118	0\\
119	0\\
120	0\\
121	0\\
122	0\\
123	0\\
124	0\\
125	0\\
126	0\\
127	0\\
128	0\\
129	0\\
130	0\\
131	0\\
132	0\\
133	0\\
134	0\\
135	0\\
136	0\\
137	0\\
138	0\\
139	0\\
140	0\\
141	0\\
142	0\\
143	0\\
144	0\\
145	0\\
146	0\\
147	0\\
148	0\\
149	0\\
150	0\\
151	0\\
152	0\\
153	0\\
154	0\\
155	0\\
156	0\\
157	0\\
158	0\\
159	0\\
160	0\\
161	0\\
162	0\\
163	0\\
164	0\\
165	0\\
166	0\\
167	0\\
168	0\\
169	0\\
170	0\\
171	0\\
172	0\\
173	0\\
174	0\\
175	0\\
176	0\\
177	0\\
178	0\\
179	0\\
180	0\\
181	0\\
182	0\\
183	0\\
184	0\\
185	0\\
186	0\\
187	0\\
188	0\\
189	0\\
190	0\\
191	0\\
192	0\\
193	0\\
194	0\\
195	0\\
196	0\\
197	0\\
198	0\\
199	0\\
200	0\\
201	0\\
202	0\\
203	0\\
204	0\\
205	0\\
206	0\\
207	0\\
208	0\\
209	0\\
210	0\\
211	0\\
212	0\\
213	0\\
214	0\\
215	0\\
216	0\\
217	0\\
218	0\\
219	0\\
220	0\\
221	0\\
222	0\\
223	0\\
224	0\\
225	0\\
226	0\\
227	0\\
228	0\\
229	0\\
230	0\\
231	0\\
232	0\\
233	0\\
234	0\\
235	0\\
236	0\\
237	0\\
238	0\\
239	0\\
240	0\\
241	0\\
242	0\\
243	0\\
244	0\\
245	0\\
246	0\\
247	0\\
248	0\\
249	0\\
250	0\\
251	0\\
252	0\\
253	0\\
254	0\\
255	0\\
256	0\\
257	0\\
258	0\\
259	0\\
260	0\\
261	0\\
262	0\\
263	0\\
264	0\\
265	0\\
266	0\\
267	0\\
268	0\\
269	0\\
270	0\\
271	0\\
272	0\\
273	0\\
274	0\\
275	0\\
276	0\\
277	0\\
278	0\\
279	0\\
280	0\\
281	0\\
282	0\\
283	0\\
284	0\\
285	0\\
286	0\\
287	0\\
288	0\\
289	0\\
290	0\\
291	0\\
292	0\\
293	0\\
294	0\\
295	0\\
296	0\\
297	0\\
298	0\\
299	0\\
300	0\\
301	0\\
302	0\\
303	0\\
304	0\\
305	0\\
306	0\\
307	0\\
308	0\\
309	0\\
310	0\\
311	0\\
312	0\\
313	0\\
314	0\\
315	0\\
316	0\\
317	0\\
318	0\\
319	0\\
320	0\\
321	0\\
322	0\\
323	0\\
324	0\\
325	0\\
326	0\\
327	0\\
328	0\\
329	0\\
330	0\\
331	0\\
332	0\\
333	0\\
334	0\\
335	0\\
336	0\\
337	0\\
338	0\\
339	0\\
340	0\\
341	0\\
342	0\\
343	0\\
344	0\\
345	0\\
346	0\\
347	0\\
348	0\\
349	0\\
350	0\\
351	0\\
352	0\\
353	0\\
354	0\\
355	0\\
356	0\\
357	0\\
358	0\\
359	0\\
360	0\\
361	0\\
362	0\\
363	0\\
364	0\\
365	0\\
366	0\\
367	0\\
368	0\\
369	0\\
370	0\\
371	0\\
372	0\\
373	0\\
374	0\\
375	0\\
376	0\\
377	0\\
378	0\\
379	0\\
380	0\\
381	0\\
382	0\\
383	0\\
384	0\\
385	0\\
386	0\\
387	0\\
388	0\\
389	0\\
390	0\\
391	0\\
392	0\\
393	0\\
394	0\\
395	0\\
396	0\\
397	0\\
398	0\\
399	0\\
400	0\\
401	0\\
402	0\\
403	0\\
404	0\\
405	0\\
406	0\\
407	0\\
408	0\\
409	0\\
410	0\\
411	0\\
412	0\\
413	0\\
414	0\\
415	0\\
416	0\\
417	0\\
418	0\\
419	0\\
420	0\\
421	0\\
422	0\\
423	0\\
424	0\\
425	0\\
426	0\\
427	0\\
428	0\\
429	0\\
430	0\\
431	0\\
432	0\\
433	0\\
434	0\\
435	0\\
436	0\\
437	0\\
438	0\\
439	0\\
440	0\\
441	0\\
442	0\\
443	0\\
444	0\\
445	0\\
446	0\\
447	0\\
448	0\\
449	0\\
450	0\\
451	0\\
452	0\\
453	0\\
454	0\\
455	0\\
456	0\\
457	0\\
458	0\\
459	0\\
460	0\\
461	0\\
462	0\\
463	0\\
464	0\\
465	0\\
466	0\\
467	0\\
468	0\\
469	0\\
470	0\\
471	0\\
472	0\\
473	0\\
474	0\\
475	0\\
476	0\\
477	0\\
478	0\\
479	0\\
480	0\\
481	0\\
482	0\\
483	0\\
484	0\\
485	0\\
486	0\\
487	0\\
488	0\\
489	0\\
490	0\\
491	0\\
492	0\\
493	0\\
494	0\\
495	0\\
496	0\\
497	0\\
498	0\\
499	0\\
500	0\\
501	0\\
502	0\\
503	0\\
504	0\\
505	0\\
506	0\\
507	0\\
508	0\\
509	0\\
510	0\\
511	0\\
512	0\\
513	0\\
514	0\\
515	0\\
516	0\\
517	0\\
518	0\\
519	0\\
520	0\\
521	0\\
522	0\\
523	0\\
524	0\\
525	0\\
526	0\\
527	0\\
528	0\\
529	0\\
530	0\\
531	0\\
532	0\\
533	0\\
534	0\\
535	0\\
536	0\\
537	0\\
538	0\\
539	0\\
540	0\\
541	0\\
542	0\\
543	0\\
544	0\\
545	0\\
546	0\\
547	0\\
548	0\\
549	0\\
550	0\\
551	0\\
552	0\\
553	0\\
554	0\\
555	0\\
556	0\\
557	0\\
558	0\\
559	0\\
560	0\\
561	0\\
562	0\\
563	0\\
564	0\\
565	0\\
566	0\\
567	0\\
568	0\\
569	0\\
570	0\\
571	0\\
572	0\\
573	0\\
574	0\\
575	0\\
576	0\\
577	0\\
578	0\\
579	0\\
580	0\\
581	0\\
582	0\\
583	0\\
584	0\\
585	0\\
586	0\\
587	0\\
588	0\\
589	0\\
590	0\\
591	0\\
592	0\\
593	0\\
594	0\\
595	0\\
596	0\\
597	0\\
598	0\\
599	0\\
600	0\\
};
\addplot [color=mycolor5,solid,forget plot]
  table[row sep=crcr]{%
1	0\\
2	0\\
3	0\\
4	0\\
5	0\\
6	0\\
7	0\\
8	0\\
9	0\\
10	0\\
11	0\\
12	0\\
13	0\\
14	0\\
15	0\\
16	0\\
17	0\\
18	0\\
19	0\\
20	0\\
21	0\\
22	0\\
23	0\\
24	0\\
25	0\\
26	0\\
27	0\\
28	0\\
29	0\\
30	0\\
31	0\\
32	0\\
33	0\\
34	0\\
35	0\\
36	0\\
37	0\\
38	0\\
39	0\\
40	0\\
41	0\\
42	0\\
43	0\\
44	0\\
45	0\\
46	0\\
47	0\\
48	0\\
49	0\\
50	0\\
51	0\\
52	0\\
53	0\\
54	0\\
55	0\\
56	0\\
57	0\\
58	0\\
59	0\\
60	0\\
61	0\\
62	0\\
63	0\\
64	0\\
65	0\\
66	0\\
67	0\\
68	0\\
69	0\\
70	0\\
71	0\\
72	0\\
73	0\\
74	0\\
75	0\\
76	0\\
77	0\\
78	0\\
79	0\\
80	0\\
81	0\\
82	0\\
83	0\\
84	0\\
85	0\\
86	0\\
87	0\\
88	0\\
89	0\\
90	0\\
91	0\\
92	0\\
93	0\\
94	0\\
95	0\\
96	0\\
97	0\\
98	0\\
99	0\\
100	0\\
101	0\\
102	0\\
103	0\\
104	0\\
105	0\\
106	0\\
107	0\\
108	0\\
109	0\\
110	0\\
111	0\\
112	0\\
113	0\\
114	0\\
115	0\\
116	0\\
117	0\\
118	0\\
119	0\\
120	0\\
121	0\\
122	0\\
123	0\\
124	0\\
125	0\\
126	0\\
127	0\\
128	0\\
129	0\\
130	0\\
131	0\\
132	0\\
133	0\\
134	0\\
135	0\\
136	0\\
137	0\\
138	0\\
139	0\\
140	0\\
141	0\\
142	0\\
143	0\\
144	0\\
145	0\\
146	0\\
147	0\\
148	0\\
149	0\\
150	0\\
151	0\\
152	0\\
153	0\\
154	0\\
155	0\\
156	0\\
157	0\\
158	0\\
159	0\\
160	0\\
161	0\\
162	0\\
163	0\\
164	0\\
165	0\\
166	0\\
167	0\\
168	0\\
169	0\\
170	0\\
171	0\\
172	0\\
173	0\\
174	0\\
175	0\\
176	0\\
177	0\\
178	0\\
179	0\\
180	0\\
181	0\\
182	0\\
183	0\\
184	0\\
185	0\\
186	0\\
187	0\\
188	0\\
189	0\\
190	0\\
191	0\\
192	0\\
193	0\\
194	0\\
195	0\\
196	0\\
197	0\\
198	0\\
199	0\\
200	0\\
201	0\\
202	0\\
203	0\\
204	0\\
205	0\\
206	0\\
207	0\\
208	0\\
209	0\\
210	0\\
211	0\\
212	0\\
213	0\\
214	0\\
215	0\\
216	0\\
217	0\\
218	0\\
219	0\\
220	0\\
221	0\\
222	0\\
223	0\\
224	0\\
225	0\\
226	0\\
227	0\\
228	0\\
229	0\\
230	0\\
231	0\\
232	0\\
233	0\\
234	0\\
235	0\\
236	0\\
237	0\\
238	0\\
239	0\\
240	0\\
241	0\\
242	0\\
243	0\\
244	0\\
245	0\\
246	0\\
247	0\\
248	0\\
249	0\\
250	0\\
251	0\\
252	0\\
253	0\\
254	0\\
255	0\\
256	0\\
257	0\\
258	0\\
259	0\\
260	0\\
261	0\\
262	0\\
263	0\\
264	0\\
265	0\\
266	0\\
267	0\\
268	0\\
269	0\\
270	0\\
271	0\\
272	0\\
273	0\\
274	0\\
275	0\\
276	0\\
277	0\\
278	0\\
279	0\\
280	0\\
281	0\\
282	0\\
283	0\\
284	0\\
285	0\\
286	0\\
287	0\\
288	0\\
289	0\\
290	0\\
291	0\\
292	0\\
293	0\\
294	0\\
295	0\\
296	0\\
297	0\\
298	0\\
299	0\\
300	0\\
301	0\\
302	0\\
303	0\\
304	0\\
305	0\\
306	0\\
307	0\\
308	0\\
309	0\\
310	0\\
311	0\\
312	0\\
313	0\\
314	0\\
315	0\\
316	0\\
317	0\\
318	0\\
319	0\\
320	0\\
321	0\\
322	0\\
323	0\\
324	0\\
325	0\\
326	0\\
327	0\\
328	0\\
329	0\\
330	0\\
331	0\\
332	0\\
333	0\\
334	0\\
335	0\\
336	0\\
337	0\\
338	0\\
339	0\\
340	0\\
341	0\\
342	0\\
343	0\\
344	0\\
345	0\\
346	0\\
347	0\\
348	0\\
349	0\\
350	0\\
351	0\\
352	0\\
353	0\\
354	0\\
355	0\\
356	0\\
357	0\\
358	0\\
359	0\\
360	0\\
361	0\\
362	0\\
363	0\\
364	0\\
365	0\\
366	0\\
367	0\\
368	0\\
369	0\\
370	0\\
371	0\\
372	0\\
373	0\\
374	0\\
375	0\\
376	0\\
377	0\\
378	0\\
379	0\\
380	0\\
381	0\\
382	0\\
383	0\\
384	0\\
385	0\\
386	0\\
387	0\\
388	0\\
389	0\\
390	0\\
391	0\\
392	0\\
393	0\\
394	0\\
395	0\\
396	0\\
397	0\\
398	0\\
399	0\\
400	0\\
401	0\\
402	0\\
403	0\\
404	0\\
405	0\\
406	0\\
407	0\\
408	0\\
409	0\\
410	0\\
411	0\\
412	0\\
413	0\\
414	0\\
415	0\\
416	0\\
417	0\\
418	0\\
419	0\\
420	0\\
421	0\\
422	0\\
423	0\\
424	0\\
425	0\\
426	0\\
427	0\\
428	0\\
429	0\\
430	0\\
431	0\\
432	0\\
433	0\\
434	0\\
435	0\\
436	0\\
437	0\\
438	0\\
439	0\\
440	0\\
441	0\\
442	0\\
443	0\\
444	0\\
445	0\\
446	0\\
447	0\\
448	0\\
449	0\\
450	0\\
451	0\\
452	0\\
453	0\\
454	0\\
455	0\\
456	0\\
457	0\\
458	0\\
459	0\\
460	0\\
461	0\\
462	0\\
463	0\\
464	0\\
465	0\\
466	0\\
467	0\\
468	0\\
469	0\\
470	0\\
471	0\\
472	0\\
473	0\\
474	0\\
475	0\\
476	0\\
477	0\\
478	0\\
479	0\\
480	0\\
481	0\\
482	0\\
483	0\\
484	0\\
485	0\\
486	0\\
487	0\\
488	0\\
489	0\\
490	0\\
491	0\\
492	0\\
493	0\\
494	0\\
495	0\\
496	0\\
497	0\\
498	0\\
499	0\\
500	0\\
501	0\\
502	0\\
503	0\\
504	0\\
505	0\\
506	0\\
507	0\\
508	0\\
509	0\\
510	0\\
511	0\\
512	0\\
513	0\\
514	0\\
515	0\\
516	0\\
517	0\\
518	0\\
519	0\\
520	0\\
521	0\\
522	0\\
523	0\\
524	0\\
525	0\\
526	0\\
527	0\\
528	0\\
529	0\\
530	0\\
531	0\\
532	0\\
533	0\\
534	0\\
535	0\\
536	0\\
537	0\\
538	0\\
539	0\\
540	0\\
541	0\\
542	0\\
543	0\\
544	0\\
545	0\\
546	0\\
547	0\\
548	0\\
549	0\\
550	0\\
551	0\\
552	0\\
553	0\\
554	0\\
555	0\\
556	0\\
557	0\\
558	0\\
559	0\\
560	0\\
561	0\\
562	0\\
563	0\\
564	0\\
565	0\\
566	0\\
567	0\\
568	0\\
569	0\\
570	0\\
571	0\\
572	0\\
573	0\\
574	0\\
575	0\\
576	0\\
577	0\\
578	0\\
579	0\\
580	0\\
581	0\\
582	0\\
583	0\\
584	0\\
585	0\\
586	0\\
587	0\\
588	0\\
589	0\\
590	0\\
591	0\\
592	0\\
593	0\\
594	0\\
595	0\\
596	0\\
597	0\\
598	0\\
599	0\\
600	0\\
};
\addplot [color=mycolor6,solid,forget plot]
  table[row sep=crcr]{%
1	0\\
2	0\\
3	0\\
4	0\\
5	0\\
6	0\\
7	0\\
8	0\\
9	0\\
10	0\\
11	0\\
12	0\\
13	0\\
14	0\\
15	0\\
16	0\\
17	0\\
18	0\\
19	0\\
20	0\\
21	0\\
22	0\\
23	0\\
24	0\\
25	0\\
26	0\\
27	0\\
28	0\\
29	0\\
30	0\\
31	0\\
32	0\\
33	0\\
34	0\\
35	0\\
36	0\\
37	0\\
38	0\\
39	0\\
40	0\\
41	0\\
42	0\\
43	0\\
44	0\\
45	0\\
46	0\\
47	0\\
48	0\\
49	0\\
50	0\\
51	0\\
52	0\\
53	0\\
54	0\\
55	0\\
56	0\\
57	0\\
58	0\\
59	0\\
60	0\\
61	0\\
62	0\\
63	0\\
64	0\\
65	0\\
66	0\\
67	0\\
68	0\\
69	0\\
70	0\\
71	0\\
72	0\\
73	0\\
74	0\\
75	0\\
76	0\\
77	0\\
78	0\\
79	0\\
80	0\\
81	0\\
82	0\\
83	0\\
84	0\\
85	0\\
86	0\\
87	0\\
88	0\\
89	0\\
90	0\\
91	0\\
92	0\\
93	0\\
94	0\\
95	0\\
96	0\\
97	0\\
98	0\\
99	0\\
100	0\\
101	0\\
102	0\\
103	0\\
104	0\\
105	0\\
106	0\\
107	0\\
108	0\\
109	0\\
110	0\\
111	0\\
112	0\\
113	0\\
114	0\\
115	0\\
116	0\\
117	0\\
118	0\\
119	0\\
120	0\\
121	0\\
122	0\\
123	0\\
124	0\\
125	0\\
126	0\\
127	0\\
128	0\\
129	0\\
130	0\\
131	0\\
132	0\\
133	0\\
134	0\\
135	0\\
136	0\\
137	0\\
138	0\\
139	0\\
140	0\\
141	0\\
142	0\\
143	0\\
144	0\\
145	0\\
146	0\\
147	0\\
148	0\\
149	0\\
150	0\\
151	0\\
152	0\\
153	0\\
154	0\\
155	0\\
156	0\\
157	0\\
158	0\\
159	0\\
160	0\\
161	0\\
162	0\\
163	0\\
164	0\\
165	0\\
166	0\\
167	0\\
168	0\\
169	0\\
170	0\\
171	0\\
172	0\\
173	0\\
174	0\\
175	0\\
176	0\\
177	0\\
178	0\\
179	0\\
180	0\\
181	0\\
182	0\\
183	0\\
184	0\\
185	0\\
186	0\\
187	0\\
188	0\\
189	0\\
190	0\\
191	0\\
192	0\\
193	0\\
194	0\\
195	0\\
196	0\\
197	0\\
198	0\\
199	0\\
200	0\\
201	0\\
202	0\\
203	0\\
204	0\\
205	0\\
206	0\\
207	0\\
208	0\\
209	0\\
210	0\\
211	0\\
212	0\\
213	0\\
214	0\\
215	0\\
216	0\\
217	0\\
218	0\\
219	0\\
220	0\\
221	0\\
222	0\\
223	0\\
224	0\\
225	0\\
226	0\\
227	0\\
228	0\\
229	0\\
230	0\\
231	0\\
232	0\\
233	0\\
234	0\\
235	0\\
236	0\\
237	0\\
238	0\\
239	0\\
240	0\\
241	0\\
242	0\\
243	0\\
244	0\\
245	0\\
246	0\\
247	0\\
248	0\\
249	0\\
250	0\\
251	0\\
252	0\\
253	0\\
254	0\\
255	0\\
256	0\\
257	0\\
258	0\\
259	0\\
260	0\\
261	0\\
262	0\\
263	0\\
264	0\\
265	0\\
266	0\\
267	0\\
268	0\\
269	0\\
270	0\\
271	0\\
272	0\\
273	0\\
274	0\\
275	0\\
276	0\\
277	0\\
278	0\\
279	0\\
280	0\\
281	0\\
282	0\\
283	0\\
284	0\\
285	0\\
286	0\\
287	0\\
288	0\\
289	0\\
290	0\\
291	0\\
292	0\\
293	0\\
294	0\\
295	0\\
296	0\\
297	0\\
298	0\\
299	0\\
300	0\\
301	0\\
302	0\\
303	0\\
304	0\\
305	0\\
306	0\\
307	0\\
308	0\\
309	0\\
310	0\\
311	0\\
312	0\\
313	0\\
314	0\\
315	0\\
316	0\\
317	0\\
318	0\\
319	0\\
320	0\\
321	0\\
322	0\\
323	0\\
324	0\\
325	0\\
326	0\\
327	0\\
328	0\\
329	0\\
330	0\\
331	0\\
332	0\\
333	0\\
334	0\\
335	0\\
336	0\\
337	0\\
338	0\\
339	0\\
340	0\\
341	0\\
342	0\\
343	0\\
344	0\\
345	0\\
346	0\\
347	0\\
348	0\\
349	0\\
350	0\\
351	0\\
352	0\\
353	0\\
354	0\\
355	0\\
356	0\\
357	0\\
358	0\\
359	0\\
360	0\\
361	0\\
362	0\\
363	0\\
364	0\\
365	0\\
366	0\\
367	0\\
368	0\\
369	0\\
370	0\\
371	0\\
372	0\\
373	0\\
374	0\\
375	0\\
376	0\\
377	0\\
378	0\\
379	0\\
380	0\\
381	0\\
382	0\\
383	0\\
384	0\\
385	0\\
386	0\\
387	0\\
388	0\\
389	0\\
390	0\\
391	0\\
392	0\\
393	0\\
394	0\\
395	0\\
396	0\\
397	0\\
398	0\\
399	0\\
400	0\\
401	0\\
402	0\\
403	0\\
404	0\\
405	0\\
406	0\\
407	0\\
408	0\\
409	0\\
410	0\\
411	0\\
412	0\\
413	0\\
414	0\\
415	0\\
416	0\\
417	0\\
418	0\\
419	0\\
420	0\\
421	0\\
422	0\\
423	0\\
424	0\\
425	0\\
426	0\\
427	0\\
428	0\\
429	0\\
430	0\\
431	0\\
432	0\\
433	0\\
434	0\\
435	0\\
436	0\\
437	0\\
438	0\\
439	0\\
440	0\\
441	0\\
442	0\\
443	0\\
444	0\\
445	0\\
446	0\\
447	0\\
448	0\\
449	0\\
450	0\\
451	0\\
452	0\\
453	0\\
454	0\\
455	0\\
456	0\\
457	0\\
458	0\\
459	0\\
460	0\\
461	0\\
462	0\\
463	0\\
464	0\\
465	0\\
466	0\\
467	0\\
468	0\\
469	0\\
470	0\\
471	0\\
472	0\\
473	0\\
474	0\\
475	0\\
476	0\\
477	0\\
478	0\\
479	0\\
480	0\\
481	0\\
482	0\\
483	0\\
484	0\\
485	0\\
486	0\\
487	0\\
488	0\\
489	0\\
490	0\\
491	0\\
492	0\\
493	0\\
494	0\\
495	0\\
496	0\\
497	0\\
498	0\\
499	0\\
500	0\\
501	0\\
502	0\\
503	0\\
504	0\\
505	0\\
506	0\\
507	0\\
508	0\\
509	0\\
510	0\\
511	0\\
512	0\\
513	0\\
514	0\\
515	0\\
516	0\\
517	0\\
518	0\\
519	0\\
520	0\\
521	0\\
522	0\\
523	0\\
524	0\\
525	0\\
526	0\\
527	0\\
528	0\\
529	0\\
530	0\\
531	0\\
532	0\\
533	0\\
534	0\\
535	0\\
536	0\\
537	0\\
538	0\\
539	0\\
540	0\\
541	0\\
542	0\\
543	0\\
544	0\\
545	0\\
546	0\\
547	0\\
548	0\\
549	0\\
550	0\\
551	0\\
552	0\\
553	0\\
554	0\\
555	0\\
556	0\\
557	0\\
558	0\\
559	0\\
560	0\\
561	0\\
562	0\\
563	0\\
564	0\\
565	0\\
566	0\\
567	0\\
568	0\\
569	0\\
570	0\\
571	0\\
572	0\\
573	0\\
574	0\\
575	0\\
576	0\\
577	0\\
578	0\\
579	0\\
580	0\\
581	0\\
582	0\\
583	0\\
584	0\\
585	0\\
586	0\\
587	0\\
588	0\\
589	0\\
590	0\\
591	0\\
592	0\\
593	0\\
594	0\\
595	0\\
596	0\\
597	0\\
598	0\\
599	0\\
600	0\\
};
\addplot [color=mycolor7,solid,forget plot]
  table[row sep=crcr]{%
1	0\\
2	0\\
3	0\\
4	0\\
5	0\\
6	0\\
7	0\\
8	0\\
9	0\\
10	0\\
11	0\\
12	0\\
13	0\\
14	0\\
15	0\\
16	0\\
17	0\\
18	0\\
19	0\\
20	0\\
21	0\\
22	0\\
23	0\\
24	0\\
25	0\\
26	0\\
27	0\\
28	0\\
29	0\\
30	0\\
31	0\\
32	0\\
33	0\\
34	0\\
35	0\\
36	0\\
37	0\\
38	0\\
39	0\\
40	0\\
41	0\\
42	0\\
43	0\\
44	0\\
45	0\\
46	0\\
47	0\\
48	0\\
49	0\\
50	0\\
51	0\\
52	0\\
53	0\\
54	0\\
55	0\\
56	0\\
57	0\\
58	0\\
59	0\\
60	0\\
61	0\\
62	0\\
63	0\\
64	0\\
65	0\\
66	0\\
67	0\\
68	0\\
69	0\\
70	0\\
71	0\\
72	0\\
73	0\\
74	0\\
75	0\\
76	0\\
77	0\\
78	0\\
79	0\\
80	0\\
81	0\\
82	0\\
83	0\\
84	0\\
85	0\\
86	0\\
87	0\\
88	0\\
89	0\\
90	0\\
91	0\\
92	0\\
93	0\\
94	0\\
95	0\\
96	0\\
97	0\\
98	0\\
99	0\\
100	0\\
101	0\\
102	0\\
103	0\\
104	0\\
105	0\\
106	0\\
107	0\\
108	0\\
109	0\\
110	0\\
111	0\\
112	0\\
113	0\\
114	0\\
115	0\\
116	0\\
117	0\\
118	0\\
119	0\\
120	0\\
121	0\\
122	0\\
123	0\\
124	0\\
125	0\\
126	0\\
127	0\\
128	0\\
129	0\\
130	0\\
131	0\\
132	0\\
133	0\\
134	0\\
135	0\\
136	0\\
137	0\\
138	0\\
139	0\\
140	0\\
141	0\\
142	0\\
143	0\\
144	0\\
145	0\\
146	0\\
147	0\\
148	0\\
149	0\\
150	0\\
151	0\\
152	0\\
153	0\\
154	0\\
155	0\\
156	0\\
157	0\\
158	0\\
159	0\\
160	0\\
161	0\\
162	0\\
163	0\\
164	0\\
165	0\\
166	0\\
167	0\\
168	0\\
169	0\\
170	0\\
171	0\\
172	0\\
173	0\\
174	0\\
175	0\\
176	0\\
177	0\\
178	0\\
179	0\\
180	0\\
181	0\\
182	0\\
183	0\\
184	0\\
185	0\\
186	0\\
187	0\\
188	0\\
189	0\\
190	0\\
191	0\\
192	0\\
193	0\\
194	0\\
195	0\\
196	0\\
197	0\\
198	0\\
199	0\\
200	0\\
201	0\\
202	0\\
203	0\\
204	0\\
205	0\\
206	0\\
207	0\\
208	0\\
209	0\\
210	0\\
211	0\\
212	0\\
213	0\\
214	0\\
215	0\\
216	0\\
217	0\\
218	0\\
219	0\\
220	0\\
221	0\\
222	0\\
223	0\\
224	0\\
225	0\\
226	0\\
227	0\\
228	0\\
229	0\\
230	0\\
231	0\\
232	0\\
233	0\\
234	0\\
235	0\\
236	0\\
237	0\\
238	0\\
239	0\\
240	0\\
241	0\\
242	0\\
243	0\\
244	0\\
245	0\\
246	0\\
247	0\\
248	0\\
249	0\\
250	0\\
251	0\\
252	0\\
253	0\\
254	0\\
255	0\\
256	0\\
257	0\\
258	0\\
259	0\\
260	0\\
261	0\\
262	0\\
263	0\\
264	0\\
265	0\\
266	0\\
267	0\\
268	0\\
269	0\\
270	0\\
271	0\\
272	0\\
273	0\\
274	0\\
275	0\\
276	0\\
277	0\\
278	0\\
279	0\\
280	0\\
281	0\\
282	0\\
283	0\\
284	0\\
285	0\\
286	0\\
287	0\\
288	0\\
289	0\\
290	0\\
291	0\\
292	0\\
293	0\\
294	0\\
295	0\\
296	0\\
297	0\\
298	0\\
299	0\\
300	0\\
301	0\\
302	0\\
303	0\\
304	0\\
305	0\\
306	0\\
307	0\\
308	0\\
309	0\\
310	0\\
311	0\\
312	0\\
313	0\\
314	0\\
315	0\\
316	0\\
317	0\\
318	0\\
319	0\\
320	0\\
321	0\\
322	0\\
323	0\\
324	0\\
325	0\\
326	0\\
327	0\\
328	0\\
329	0\\
330	0\\
331	0\\
332	0\\
333	0\\
334	0\\
335	0\\
336	0\\
337	0\\
338	0\\
339	0\\
340	0\\
341	0\\
342	0\\
343	0\\
344	0\\
345	0\\
346	0\\
347	0\\
348	0\\
349	0\\
350	0\\
351	0\\
352	0\\
353	0\\
354	0\\
355	0\\
356	0\\
357	0\\
358	0\\
359	0\\
360	0\\
361	0\\
362	0\\
363	0\\
364	0\\
365	0\\
366	0\\
367	0\\
368	0\\
369	0\\
370	0\\
371	0\\
372	0\\
373	0\\
374	0\\
375	0\\
376	0\\
377	0\\
378	0\\
379	0\\
380	0\\
381	0\\
382	0\\
383	0\\
384	0\\
385	0\\
386	0\\
387	0\\
388	0\\
389	0\\
390	0\\
391	0\\
392	0\\
393	0\\
394	0\\
395	0\\
396	0\\
397	0\\
398	0\\
399	0\\
400	0\\
401	0\\
402	0\\
403	0\\
404	0\\
405	0\\
406	0\\
407	0\\
408	0\\
409	0\\
410	0\\
411	0\\
412	0\\
413	0\\
414	0\\
415	0\\
416	0\\
417	0\\
418	0\\
419	0\\
420	0\\
421	0\\
422	0\\
423	0\\
424	0\\
425	0\\
426	0\\
427	0\\
428	0\\
429	0\\
430	0\\
431	0\\
432	0\\
433	0\\
434	0\\
435	0\\
436	0\\
437	0\\
438	0\\
439	0\\
440	0\\
441	0\\
442	0\\
443	0\\
444	0\\
445	0\\
446	0\\
447	0\\
448	0\\
449	0\\
450	0\\
451	0\\
452	0\\
453	0\\
454	0\\
455	0\\
456	0\\
457	0\\
458	0\\
459	0\\
460	0\\
461	0\\
462	0\\
463	0\\
464	0\\
465	0\\
466	0\\
467	0\\
468	0\\
469	0\\
470	0\\
471	0\\
472	0\\
473	0\\
474	0\\
475	0\\
476	0\\
477	0\\
478	0\\
479	0\\
480	0\\
481	0\\
482	0\\
483	0\\
484	0\\
485	0\\
486	0\\
487	0\\
488	0\\
489	0\\
490	0\\
491	0\\
492	0\\
493	0\\
494	0\\
495	0\\
496	0\\
497	0\\
498	0\\
499	0\\
500	0\\
501	0\\
502	0\\
503	0\\
504	0\\
505	0\\
506	0\\
507	0\\
508	0\\
509	0\\
510	0\\
511	0\\
512	0\\
513	0\\
514	0\\
515	0\\
516	0\\
517	0\\
518	0\\
519	0\\
520	0\\
521	0\\
522	0\\
523	0\\
524	0\\
525	0\\
526	0\\
527	0\\
528	0\\
529	0\\
530	0\\
531	0\\
532	0\\
533	0\\
534	0\\
535	0\\
536	0\\
537	0\\
538	0\\
539	0\\
540	0\\
541	0\\
542	0\\
543	0\\
544	0\\
545	0\\
546	0\\
547	0\\
548	0\\
549	0\\
550	0\\
551	0\\
552	0\\
553	0\\
554	0\\
555	0\\
556	0\\
557	0\\
558	0\\
559	0\\
560	0\\
561	0\\
562	0\\
563	0\\
564	0\\
565	0\\
566	0\\
567	0\\
568	0\\
569	0\\
570	0\\
571	0\\
572	0\\
573	0\\
574	0\\
575	0\\
576	0\\
577	0\\
578	0\\
579	0\\
580	0\\
581	0\\
582	0\\
583	0\\
584	0\\
585	0\\
586	0\\
587	0\\
588	0\\
589	0\\
590	0\\
591	0\\
592	0\\
593	0\\
594	0\\
595	0\\
596	0\\
597	0\\
598	0\\
599	0\\
600	0\\
};
\addplot [color=mycolor8,solid,forget plot]
  table[row sep=crcr]{%
1	0\\
2	0\\
3	0\\
4	0\\
5	0\\
6	0\\
7	0\\
8	0\\
9	0\\
10	0\\
11	0\\
12	0\\
13	0\\
14	0\\
15	0\\
16	0\\
17	0\\
18	0\\
19	0\\
20	0\\
21	0\\
22	0\\
23	0\\
24	0\\
25	0\\
26	0\\
27	0\\
28	0\\
29	0\\
30	0\\
31	0\\
32	0\\
33	0\\
34	0\\
35	0\\
36	0\\
37	0\\
38	0\\
39	0\\
40	0\\
41	0\\
42	0\\
43	0\\
44	0\\
45	0\\
46	0\\
47	0\\
48	0\\
49	0\\
50	0\\
51	0\\
52	0\\
53	0\\
54	0\\
55	0\\
56	0\\
57	0\\
58	0\\
59	0\\
60	0\\
61	0\\
62	0\\
63	0\\
64	0\\
65	0\\
66	0\\
67	0\\
68	0\\
69	0\\
70	0\\
71	0\\
72	0\\
73	0\\
74	0\\
75	0\\
76	0\\
77	0\\
78	0\\
79	0\\
80	0\\
81	0\\
82	0\\
83	0\\
84	0\\
85	0\\
86	0\\
87	0\\
88	0\\
89	0\\
90	0\\
91	0\\
92	0\\
93	0\\
94	0\\
95	0\\
96	0\\
97	0\\
98	0\\
99	0\\
100	0\\
101	0\\
102	0\\
103	0\\
104	0\\
105	0\\
106	0\\
107	0\\
108	0\\
109	0\\
110	0\\
111	0\\
112	0\\
113	0\\
114	0\\
115	0\\
116	0\\
117	0\\
118	0\\
119	0\\
120	0\\
121	0\\
122	0\\
123	0\\
124	0\\
125	0\\
126	0\\
127	0\\
128	0\\
129	0\\
130	0\\
131	0\\
132	0\\
133	0\\
134	0\\
135	0\\
136	0\\
137	0\\
138	0\\
139	0\\
140	0\\
141	0\\
142	0\\
143	0\\
144	0\\
145	0\\
146	0\\
147	0\\
148	0\\
149	0\\
150	0\\
151	0\\
152	0\\
153	0\\
154	0\\
155	0\\
156	0\\
157	0\\
158	0\\
159	0\\
160	0\\
161	0\\
162	0\\
163	0\\
164	0\\
165	0\\
166	0\\
167	0\\
168	0\\
169	0\\
170	0\\
171	0\\
172	0\\
173	0\\
174	0\\
175	0\\
176	0\\
177	0\\
178	0\\
179	0\\
180	0\\
181	0\\
182	0\\
183	0\\
184	0\\
185	0\\
186	0\\
187	0\\
188	0\\
189	0\\
190	0\\
191	0\\
192	0\\
193	0\\
194	0\\
195	0\\
196	0\\
197	0\\
198	0\\
199	0\\
200	0\\
201	0\\
202	0\\
203	0\\
204	0\\
205	0\\
206	0\\
207	0\\
208	0\\
209	0\\
210	0\\
211	0\\
212	0\\
213	0\\
214	0\\
215	0\\
216	0\\
217	0\\
218	0\\
219	0\\
220	0\\
221	0\\
222	0\\
223	0\\
224	0\\
225	0\\
226	0\\
227	0\\
228	0\\
229	0\\
230	0\\
231	0\\
232	0\\
233	0\\
234	0\\
235	0\\
236	0\\
237	0\\
238	0\\
239	0\\
240	0\\
241	0\\
242	0\\
243	0\\
244	0\\
245	0\\
246	0\\
247	0\\
248	0\\
249	0\\
250	0\\
251	0\\
252	0\\
253	0\\
254	0\\
255	0\\
256	0\\
257	0\\
258	0\\
259	0\\
260	0\\
261	0\\
262	0\\
263	0\\
264	0\\
265	0\\
266	0\\
267	0\\
268	0\\
269	0\\
270	0\\
271	0\\
272	0\\
273	0\\
274	0\\
275	0\\
276	0\\
277	0\\
278	0\\
279	0\\
280	0\\
281	0\\
282	0\\
283	0\\
284	0\\
285	0\\
286	0\\
287	0\\
288	0\\
289	0\\
290	0\\
291	0\\
292	0\\
293	0\\
294	0\\
295	0\\
296	0\\
297	0\\
298	0\\
299	0\\
300	0\\
301	0\\
302	0\\
303	0\\
304	0\\
305	0\\
306	0\\
307	0\\
308	0\\
309	0\\
310	0\\
311	0\\
312	0\\
313	0\\
314	0\\
315	0\\
316	0\\
317	0\\
318	0\\
319	0\\
320	0\\
321	0\\
322	0\\
323	0\\
324	0\\
325	0\\
326	0\\
327	0\\
328	0\\
329	0\\
330	0\\
331	0\\
332	0\\
333	0\\
334	0\\
335	0\\
336	0\\
337	0\\
338	0\\
339	0\\
340	0\\
341	0\\
342	0\\
343	0\\
344	0\\
345	0\\
346	0\\
347	0\\
348	0\\
349	0\\
350	0\\
351	0\\
352	0\\
353	0\\
354	0\\
355	0\\
356	0\\
357	0\\
358	0\\
359	0\\
360	0\\
361	0\\
362	0\\
363	0\\
364	0\\
365	0\\
366	0\\
367	0\\
368	0\\
369	0\\
370	0\\
371	0\\
372	0\\
373	0\\
374	0\\
375	0\\
376	0\\
377	0\\
378	0\\
379	0\\
380	0\\
381	0\\
382	0\\
383	0\\
384	0\\
385	0\\
386	0\\
387	0\\
388	0\\
389	0\\
390	0\\
391	0\\
392	0\\
393	0\\
394	0\\
395	0\\
396	0\\
397	0\\
398	0\\
399	0\\
400	0\\
401	0\\
402	0\\
403	0\\
404	0\\
405	0\\
406	0\\
407	0\\
408	0\\
409	0\\
410	0\\
411	0\\
412	0\\
413	0\\
414	0\\
415	0\\
416	0\\
417	0\\
418	0\\
419	0\\
420	0\\
421	0\\
422	0\\
423	0\\
424	0\\
425	0\\
426	0\\
427	0\\
428	0\\
429	0\\
430	0\\
431	0\\
432	0\\
433	0\\
434	0\\
435	0\\
436	0\\
437	0\\
438	0\\
439	0\\
440	0\\
441	0\\
442	0\\
443	0\\
444	0\\
445	0\\
446	0\\
447	0\\
448	0\\
449	0\\
450	0\\
451	0\\
452	0\\
453	0\\
454	0\\
455	0\\
456	0\\
457	0\\
458	0\\
459	0\\
460	0\\
461	0\\
462	0\\
463	0\\
464	0\\
465	0\\
466	0\\
467	0\\
468	0\\
469	0\\
470	0\\
471	0\\
472	0\\
473	0\\
474	0\\
475	0\\
476	0\\
477	0\\
478	0\\
479	0\\
480	0\\
481	0\\
482	0\\
483	0\\
484	0\\
485	0\\
486	0\\
487	0\\
488	0\\
489	0\\
490	0\\
491	0\\
492	0\\
493	0\\
494	0\\
495	0\\
496	0\\
497	0\\
498	0\\
499	0\\
500	0\\
501	0\\
502	0\\
503	0\\
504	0\\
505	0\\
506	0\\
507	0\\
508	0\\
509	0\\
510	0\\
511	0\\
512	0\\
513	0\\
514	0\\
515	0\\
516	0\\
517	0\\
518	0\\
519	0\\
520	0\\
521	0\\
522	0\\
523	0\\
524	0\\
525	0\\
526	0\\
527	0\\
528	0\\
529	0\\
530	0\\
531	0\\
532	0\\
533	0\\
534	0\\
535	0\\
536	0\\
537	0\\
538	0\\
539	0\\
540	0\\
541	0\\
542	0\\
543	0\\
544	0\\
545	0\\
546	0\\
547	0\\
548	0\\
549	0\\
550	0\\
551	0\\
552	0\\
553	0\\
554	0\\
555	0\\
556	0\\
557	0\\
558	0\\
559	0\\
560	0\\
561	0\\
562	0\\
563	0\\
564	0\\
565	0\\
566	0\\
567	0\\
568	0\\
569	0\\
570	0\\
571	0\\
572	0\\
573	0\\
574	0\\
575	0\\
576	0\\
577	0\\
578	0\\
579	0\\
580	0\\
581	0\\
582	0\\
583	0\\
584	0\\
585	0\\
586	0\\
587	0\\
588	0\\
589	0\\
590	0\\
591	0\\
592	0\\
593	0\\
594	0\\
595	0\\
596	0\\
597	0\\
598	0\\
599	0\\
600	0\\
};
\addplot [color=blue!25!mycolor7,solid,forget plot]
  table[row sep=crcr]{%
1	0\\
2	0\\
3	0\\
4	0\\
5	0\\
6	0\\
7	0\\
8	0\\
9	0\\
10	0\\
11	0\\
12	0\\
13	0\\
14	0\\
15	0\\
16	0\\
17	0\\
18	0\\
19	0\\
20	0\\
21	0\\
22	0\\
23	0\\
24	0\\
25	0\\
26	0\\
27	0\\
28	0\\
29	0\\
30	0\\
31	0\\
32	0\\
33	0\\
34	0\\
35	0\\
36	0\\
37	0\\
38	0\\
39	0\\
40	0\\
41	0\\
42	0\\
43	0\\
44	0\\
45	0\\
46	0\\
47	0\\
48	0\\
49	0\\
50	0\\
51	0\\
52	0\\
53	0\\
54	0\\
55	0\\
56	0\\
57	0\\
58	0\\
59	0\\
60	0\\
61	0\\
62	0\\
63	0\\
64	0\\
65	0\\
66	0\\
67	0\\
68	0\\
69	0\\
70	0\\
71	0\\
72	0\\
73	0\\
74	0\\
75	0\\
76	0\\
77	0\\
78	0\\
79	0\\
80	0\\
81	0\\
82	0\\
83	0\\
84	0\\
85	0\\
86	0\\
87	0\\
88	0\\
89	0\\
90	0\\
91	0\\
92	0\\
93	0\\
94	0\\
95	0\\
96	0\\
97	0\\
98	0\\
99	0\\
100	0\\
101	0\\
102	0\\
103	0\\
104	0\\
105	0\\
106	0\\
107	0\\
108	0\\
109	0\\
110	0\\
111	0\\
112	0\\
113	0\\
114	0\\
115	0\\
116	0\\
117	0\\
118	0\\
119	0\\
120	0\\
121	0\\
122	0\\
123	0\\
124	0\\
125	0\\
126	0\\
127	0\\
128	0\\
129	0\\
130	0\\
131	0\\
132	0\\
133	0\\
134	0\\
135	0\\
136	0\\
137	0\\
138	0\\
139	0\\
140	0\\
141	0\\
142	0\\
143	0\\
144	0\\
145	0\\
146	0\\
147	0\\
148	0\\
149	0\\
150	0\\
151	0\\
152	0\\
153	0\\
154	0\\
155	0\\
156	0\\
157	0\\
158	0\\
159	0\\
160	0\\
161	0\\
162	0\\
163	0\\
164	0\\
165	0\\
166	0\\
167	0\\
168	0\\
169	0\\
170	0\\
171	0\\
172	0\\
173	0\\
174	0\\
175	0\\
176	0\\
177	0\\
178	0\\
179	0\\
180	0\\
181	0\\
182	0\\
183	0\\
184	0\\
185	0\\
186	0\\
187	0\\
188	0\\
189	0\\
190	0\\
191	0\\
192	0\\
193	0\\
194	0\\
195	0\\
196	0\\
197	0\\
198	0\\
199	0\\
200	0\\
201	0\\
202	0\\
203	0\\
204	0\\
205	0\\
206	0\\
207	0\\
208	0\\
209	0\\
210	0\\
211	0\\
212	0\\
213	0\\
214	0\\
215	0\\
216	0\\
217	0\\
218	0\\
219	0\\
220	0\\
221	0\\
222	0\\
223	0\\
224	0\\
225	0\\
226	0\\
227	0\\
228	0\\
229	0\\
230	0\\
231	0\\
232	0\\
233	0\\
234	0\\
235	0\\
236	0\\
237	0\\
238	0\\
239	0\\
240	0\\
241	0\\
242	0\\
243	0\\
244	0\\
245	0\\
246	0\\
247	0\\
248	0\\
249	0\\
250	0\\
251	0\\
252	0\\
253	0\\
254	0\\
255	0\\
256	0\\
257	0\\
258	0\\
259	0\\
260	0\\
261	0\\
262	0\\
263	0\\
264	0\\
265	0\\
266	0\\
267	0\\
268	0\\
269	0\\
270	0\\
271	0\\
272	0\\
273	0\\
274	0\\
275	0\\
276	0\\
277	0\\
278	0\\
279	0\\
280	0\\
281	0\\
282	0\\
283	0\\
284	0\\
285	0\\
286	0\\
287	0\\
288	0\\
289	0\\
290	0\\
291	0\\
292	0\\
293	0\\
294	0\\
295	0\\
296	0\\
297	0\\
298	0\\
299	0\\
300	0\\
301	0\\
302	0\\
303	0\\
304	0\\
305	0\\
306	0\\
307	0\\
308	0\\
309	0\\
310	0\\
311	0\\
312	0\\
313	0\\
314	0\\
315	0\\
316	0\\
317	0\\
318	0\\
319	0\\
320	0\\
321	0\\
322	0\\
323	0\\
324	0\\
325	0\\
326	0\\
327	0\\
328	0\\
329	0\\
330	0\\
331	0\\
332	0\\
333	0\\
334	0\\
335	0\\
336	0\\
337	0\\
338	0\\
339	0\\
340	0\\
341	0\\
342	0\\
343	0\\
344	0\\
345	0\\
346	0\\
347	0\\
348	0\\
349	0\\
350	0\\
351	0\\
352	0\\
353	0\\
354	0\\
355	0\\
356	0\\
357	0\\
358	0\\
359	0\\
360	0\\
361	0\\
362	0\\
363	0\\
364	0\\
365	0\\
366	0\\
367	0\\
368	0\\
369	0\\
370	0\\
371	0\\
372	0\\
373	0\\
374	0\\
375	0\\
376	0\\
377	0\\
378	0\\
379	0\\
380	0\\
381	0\\
382	0\\
383	0\\
384	0\\
385	0\\
386	0\\
387	0\\
388	0\\
389	0\\
390	0\\
391	0\\
392	0\\
393	0\\
394	0\\
395	0\\
396	0\\
397	0\\
398	0\\
399	0\\
400	0\\
401	0\\
402	0\\
403	0\\
404	0\\
405	0\\
406	0\\
407	0\\
408	0\\
409	0\\
410	0\\
411	0\\
412	0\\
413	0\\
414	0\\
415	0\\
416	0\\
417	0\\
418	0\\
419	0\\
420	0\\
421	0\\
422	0\\
423	0\\
424	0\\
425	0\\
426	0\\
427	0\\
428	0\\
429	0\\
430	0\\
431	0\\
432	0\\
433	0\\
434	0\\
435	0\\
436	0\\
437	0\\
438	0\\
439	0\\
440	0\\
441	0\\
442	0\\
443	0\\
444	0\\
445	0\\
446	0\\
447	0\\
448	0\\
449	0\\
450	0\\
451	0\\
452	0\\
453	0\\
454	0\\
455	0\\
456	0\\
457	0\\
458	0\\
459	0\\
460	0\\
461	0\\
462	0\\
463	0\\
464	0\\
465	0\\
466	0\\
467	0\\
468	0\\
469	0\\
470	0\\
471	0\\
472	0\\
473	0\\
474	0\\
475	0\\
476	0\\
477	0\\
478	0\\
479	0\\
480	0\\
481	0\\
482	0\\
483	0\\
484	0\\
485	0\\
486	0\\
487	0\\
488	0\\
489	0\\
490	0\\
491	0\\
492	0\\
493	0\\
494	0\\
495	0\\
496	0\\
497	0\\
498	0\\
499	0\\
500	0\\
501	0\\
502	0\\
503	0\\
504	0\\
505	0\\
506	0\\
507	0\\
508	0\\
509	0\\
510	0\\
511	0\\
512	0\\
513	0\\
514	0\\
515	0\\
516	0\\
517	0\\
518	0\\
519	0\\
520	0\\
521	0\\
522	0\\
523	0\\
524	0\\
525	0\\
526	0\\
527	0\\
528	0\\
529	0\\
530	0\\
531	0\\
532	0\\
533	0\\
534	0\\
535	0\\
536	0\\
537	0\\
538	0\\
539	0\\
540	0\\
541	0\\
542	0\\
543	0\\
544	0\\
545	0\\
546	0\\
547	0\\
548	0\\
549	0\\
550	0\\
551	0\\
552	0\\
553	0\\
554	0\\
555	0\\
556	0\\
557	0\\
558	0\\
559	0\\
560	0\\
561	0\\
562	0\\
563	0\\
564	0\\
565	0\\
566	0\\
567	0\\
568	0\\
569	0\\
570	0\\
571	0\\
572	0\\
573	0\\
574	0\\
575	0\\
576	0\\
577	0\\
578	0\\
579	0\\
580	0\\
581	0\\
582	0\\
583	0\\
584	0\\
585	0\\
586	0\\
587	0\\
588	0\\
589	0\\
590	0\\
591	0\\
592	0\\
593	0\\
594	0\\
595	0\\
596	0\\
597	0\\
598	0\\
599	0\\
600	0\\
};
\addplot [color=mycolor9,solid,forget plot]
  table[row sep=crcr]{%
1	0\\
2	0\\
3	0\\
4	0\\
5	0\\
6	0\\
7	0\\
8	0\\
9	0\\
10	0\\
11	0\\
12	0\\
13	0\\
14	0\\
15	0\\
16	0\\
17	0\\
18	0\\
19	0\\
20	0\\
21	0\\
22	0\\
23	0\\
24	0\\
25	0\\
26	0\\
27	0\\
28	0\\
29	0\\
30	0\\
31	0\\
32	0\\
33	0\\
34	0\\
35	0\\
36	0\\
37	0\\
38	0\\
39	0\\
40	0\\
41	0\\
42	0\\
43	0\\
44	0\\
45	0\\
46	0\\
47	0\\
48	0\\
49	0\\
50	0\\
51	0\\
52	0\\
53	0\\
54	0\\
55	0\\
56	0\\
57	0\\
58	0\\
59	0\\
60	0\\
61	0\\
62	0\\
63	0\\
64	0\\
65	0\\
66	0\\
67	0\\
68	0\\
69	0\\
70	0\\
71	0\\
72	0\\
73	0\\
74	0\\
75	0\\
76	0\\
77	0\\
78	0\\
79	0\\
80	0\\
81	0\\
82	0\\
83	0\\
84	0\\
85	0\\
86	0\\
87	0\\
88	0\\
89	0\\
90	0\\
91	0\\
92	0\\
93	0\\
94	0\\
95	0\\
96	0\\
97	0\\
98	0\\
99	0\\
100	0\\
101	0\\
102	0\\
103	0\\
104	0\\
105	0\\
106	0\\
107	0\\
108	0\\
109	0\\
110	0\\
111	0\\
112	0\\
113	0\\
114	0\\
115	0\\
116	0\\
117	0\\
118	0\\
119	0\\
120	0\\
121	0\\
122	0\\
123	0\\
124	0\\
125	0\\
126	0\\
127	0\\
128	0\\
129	0\\
130	0\\
131	0\\
132	0\\
133	0\\
134	0\\
135	0\\
136	0\\
137	0\\
138	0\\
139	0\\
140	0\\
141	0\\
142	0\\
143	0\\
144	0\\
145	0\\
146	0\\
147	0\\
148	0\\
149	0\\
150	0\\
151	0\\
152	0\\
153	0\\
154	0\\
155	0\\
156	0\\
157	0\\
158	0\\
159	0\\
160	0\\
161	0\\
162	0\\
163	0\\
164	0\\
165	0\\
166	0\\
167	0\\
168	0\\
169	0\\
170	0\\
171	0\\
172	0\\
173	0\\
174	0\\
175	0\\
176	0\\
177	0\\
178	0\\
179	0\\
180	0\\
181	0\\
182	0\\
183	0\\
184	0\\
185	0\\
186	0\\
187	0\\
188	0\\
189	0\\
190	0\\
191	0\\
192	0\\
193	0\\
194	0\\
195	0\\
196	0\\
197	0\\
198	0\\
199	0\\
200	0\\
201	0\\
202	0\\
203	0\\
204	0\\
205	0\\
206	0\\
207	0\\
208	0\\
209	0\\
210	0\\
211	0\\
212	0\\
213	0\\
214	0\\
215	0\\
216	0\\
217	0\\
218	0\\
219	0\\
220	0\\
221	0\\
222	0\\
223	0\\
224	0\\
225	0\\
226	0\\
227	0\\
228	0\\
229	0\\
230	0\\
231	0\\
232	0\\
233	0\\
234	0\\
235	0\\
236	0\\
237	0\\
238	0\\
239	0\\
240	0\\
241	0\\
242	0\\
243	0\\
244	0\\
245	0\\
246	0\\
247	0\\
248	0\\
249	0\\
250	0\\
251	0\\
252	0\\
253	0\\
254	0\\
255	0\\
256	0\\
257	0\\
258	0\\
259	0\\
260	0\\
261	0\\
262	0\\
263	0\\
264	0\\
265	0\\
266	0\\
267	0\\
268	0\\
269	0\\
270	0\\
271	0\\
272	0\\
273	0\\
274	0\\
275	0\\
276	0\\
277	0\\
278	0\\
279	0\\
280	0\\
281	0\\
282	0\\
283	0\\
284	0\\
285	0\\
286	0\\
287	0\\
288	0\\
289	0\\
290	0\\
291	0\\
292	0\\
293	0\\
294	0\\
295	0\\
296	0\\
297	0\\
298	0\\
299	0\\
300	0\\
301	0\\
302	0\\
303	0\\
304	0\\
305	0\\
306	0\\
307	0\\
308	0\\
309	0\\
310	0\\
311	0\\
312	0\\
313	0\\
314	0\\
315	0\\
316	0\\
317	0\\
318	0\\
319	0\\
320	0\\
321	0\\
322	0\\
323	0\\
324	0\\
325	0\\
326	0\\
327	0\\
328	0\\
329	0\\
330	0\\
331	0\\
332	0\\
333	0\\
334	0\\
335	0\\
336	0\\
337	0\\
338	0\\
339	0\\
340	0\\
341	0\\
342	0\\
343	0\\
344	0\\
345	0\\
346	0\\
347	0\\
348	0\\
349	0\\
350	0\\
351	0\\
352	0\\
353	0\\
354	0\\
355	0\\
356	0\\
357	0\\
358	0\\
359	0\\
360	0\\
361	0\\
362	0\\
363	0\\
364	0\\
365	0\\
366	0\\
367	0\\
368	0\\
369	0\\
370	0\\
371	0\\
372	0\\
373	0\\
374	0\\
375	0\\
376	0\\
377	0\\
378	0\\
379	0\\
380	0\\
381	0\\
382	0\\
383	0\\
384	0\\
385	0\\
386	0\\
387	0\\
388	0\\
389	0\\
390	0\\
391	0\\
392	0\\
393	0\\
394	0\\
395	0\\
396	0\\
397	0\\
398	0\\
399	0\\
400	0\\
401	0\\
402	0\\
403	0\\
404	0\\
405	0\\
406	0\\
407	0\\
408	0\\
409	0\\
410	0\\
411	0\\
412	0\\
413	0\\
414	0\\
415	0\\
416	0\\
417	0\\
418	0\\
419	0\\
420	0\\
421	0\\
422	0\\
423	0\\
424	0\\
425	0\\
426	0\\
427	0\\
428	0\\
429	0\\
430	0\\
431	0\\
432	0\\
433	0\\
434	0\\
435	0\\
436	0\\
437	0\\
438	0\\
439	0\\
440	0\\
441	0\\
442	0\\
443	0\\
444	0\\
445	0\\
446	0\\
447	0\\
448	0\\
449	0\\
450	0\\
451	0\\
452	0\\
453	0\\
454	0\\
455	0\\
456	0\\
457	0\\
458	0\\
459	0\\
460	0\\
461	0\\
462	0\\
463	0\\
464	0\\
465	0\\
466	0\\
467	0\\
468	0\\
469	0\\
470	0\\
471	0\\
472	0\\
473	0\\
474	0\\
475	0\\
476	0\\
477	0\\
478	0\\
479	0\\
480	0\\
481	0\\
482	0\\
483	0\\
484	0\\
485	0\\
486	0\\
487	0\\
488	0\\
489	0\\
490	0\\
491	0\\
492	0\\
493	0\\
494	0\\
495	0\\
496	0\\
497	0\\
498	0\\
499	0\\
500	0\\
501	0\\
502	0\\
503	0\\
504	0\\
505	0\\
506	0\\
507	0\\
508	0\\
509	0\\
510	0\\
511	0\\
512	0\\
513	0\\
514	0\\
515	0\\
516	0\\
517	0\\
518	0\\
519	0\\
520	0\\
521	0\\
522	0\\
523	0\\
524	0\\
525	0\\
526	0\\
527	0\\
528	0\\
529	0\\
530	0\\
531	0\\
532	0\\
533	0\\
534	0\\
535	0\\
536	0\\
537	0\\
538	0\\
539	0\\
540	0\\
541	0\\
542	0\\
543	0\\
544	0\\
545	0\\
546	0\\
547	0\\
548	0\\
549	0\\
550	0\\
551	0\\
552	0\\
553	0\\
554	0\\
555	0\\
556	0\\
557	0\\
558	0\\
559	0\\
560	0\\
561	0\\
562	0\\
563	0\\
564	0\\
565	0\\
566	0\\
567	0\\
568	0\\
569	0\\
570	0\\
571	0\\
572	0\\
573	0\\
574	0\\
575	0\\
576	0\\
577	0\\
578	0\\
579	0\\
580	0\\
581	0\\
582	0\\
583	0\\
584	0\\
585	0\\
586	0\\
587	0\\
588	0\\
589	0\\
590	0\\
591	0\\
592	0\\
593	0\\
594	0\\
595	0\\
596	0\\
597	0\\
598	0\\
599	0\\
600	0\\
};
\addplot [color=blue!50!mycolor7,solid,forget plot]
  table[row sep=crcr]{%
1	0\\
2	0\\
3	0\\
4	0\\
5	0\\
6	0\\
7	0\\
8	0\\
9	0\\
10	0\\
11	0\\
12	0\\
13	0\\
14	0\\
15	0\\
16	0\\
17	0\\
18	0\\
19	0\\
20	0\\
21	0\\
22	0\\
23	0\\
24	0\\
25	0\\
26	0\\
27	0\\
28	0\\
29	0\\
30	0\\
31	0\\
32	0\\
33	0\\
34	0\\
35	0\\
36	0\\
37	0\\
38	0\\
39	0\\
40	0\\
41	0\\
42	0\\
43	0\\
44	0\\
45	0\\
46	0\\
47	0\\
48	0\\
49	0\\
50	0\\
51	0\\
52	0\\
53	0\\
54	0\\
55	0\\
56	0\\
57	0\\
58	0\\
59	0\\
60	0\\
61	0\\
62	0\\
63	0\\
64	0\\
65	0\\
66	0\\
67	0\\
68	0\\
69	0\\
70	0\\
71	0\\
72	0\\
73	0\\
74	0\\
75	0\\
76	0\\
77	0\\
78	0\\
79	0\\
80	0\\
81	0\\
82	0\\
83	0\\
84	0\\
85	0\\
86	0\\
87	0\\
88	0\\
89	0\\
90	0\\
91	0\\
92	0\\
93	0\\
94	0\\
95	0\\
96	0\\
97	0\\
98	0\\
99	0\\
100	0\\
101	0\\
102	0\\
103	0\\
104	0\\
105	0\\
106	0\\
107	0\\
108	0\\
109	0\\
110	0\\
111	0\\
112	0\\
113	0\\
114	0\\
115	0\\
116	0\\
117	0\\
118	0\\
119	0\\
120	0\\
121	0\\
122	0\\
123	0\\
124	0\\
125	0\\
126	0\\
127	0\\
128	0\\
129	0\\
130	0\\
131	0\\
132	0\\
133	0\\
134	0\\
135	0\\
136	0\\
137	0\\
138	0\\
139	0\\
140	0\\
141	0\\
142	0\\
143	0\\
144	0\\
145	0\\
146	0\\
147	0\\
148	0\\
149	0\\
150	0\\
151	0\\
152	0\\
153	0\\
154	0\\
155	0\\
156	0\\
157	0\\
158	0\\
159	0\\
160	0\\
161	0\\
162	0\\
163	0\\
164	0\\
165	0\\
166	0\\
167	0\\
168	0\\
169	0\\
170	0\\
171	0\\
172	0\\
173	0\\
174	0\\
175	0\\
176	0\\
177	0\\
178	0\\
179	0\\
180	0\\
181	0\\
182	0\\
183	0\\
184	0\\
185	0\\
186	0\\
187	0\\
188	0\\
189	0\\
190	0\\
191	0\\
192	0\\
193	0\\
194	0\\
195	0\\
196	0\\
197	0\\
198	0\\
199	0\\
200	0\\
201	0\\
202	0\\
203	0\\
204	0\\
205	0\\
206	0\\
207	0\\
208	0\\
209	0\\
210	0\\
211	0\\
212	0\\
213	0\\
214	0\\
215	0\\
216	0\\
217	0\\
218	0\\
219	0\\
220	0\\
221	0\\
222	0\\
223	0\\
224	0\\
225	0\\
226	0\\
227	0\\
228	0\\
229	0\\
230	0\\
231	0\\
232	0\\
233	0\\
234	0\\
235	0\\
236	0\\
237	0\\
238	0\\
239	0\\
240	0\\
241	0\\
242	0\\
243	0\\
244	0\\
245	0\\
246	0\\
247	0\\
248	0\\
249	0\\
250	0\\
251	0\\
252	0\\
253	0\\
254	0\\
255	0\\
256	0\\
257	0\\
258	0\\
259	0\\
260	0\\
261	0\\
262	0\\
263	0\\
264	0\\
265	0\\
266	0\\
267	0\\
268	0\\
269	0\\
270	0\\
271	0\\
272	0\\
273	0\\
274	0\\
275	0\\
276	0\\
277	0\\
278	0\\
279	0\\
280	0\\
281	0\\
282	0\\
283	0\\
284	0\\
285	0\\
286	0\\
287	0\\
288	0\\
289	0\\
290	0\\
291	0\\
292	0\\
293	0\\
294	0\\
295	0\\
296	0\\
297	0\\
298	0\\
299	0\\
300	0\\
301	0\\
302	0\\
303	0\\
304	0\\
305	0\\
306	0\\
307	0\\
308	0\\
309	0\\
310	0\\
311	0\\
312	0\\
313	0\\
314	0\\
315	0\\
316	0\\
317	0\\
318	0\\
319	0\\
320	0\\
321	0\\
322	0\\
323	0\\
324	0\\
325	0\\
326	0\\
327	0\\
328	0\\
329	0\\
330	0\\
331	0\\
332	0\\
333	0\\
334	0\\
335	0\\
336	0\\
337	0\\
338	0\\
339	0\\
340	0\\
341	0\\
342	0\\
343	0\\
344	0\\
345	0\\
346	0\\
347	0\\
348	0\\
349	0\\
350	0\\
351	0\\
352	0\\
353	0\\
354	0\\
355	0\\
356	0\\
357	0\\
358	0\\
359	0\\
360	0\\
361	0\\
362	0\\
363	0\\
364	0\\
365	0\\
366	0\\
367	0\\
368	0\\
369	0\\
370	0\\
371	0\\
372	0\\
373	0\\
374	0\\
375	0\\
376	0\\
377	0\\
378	0\\
379	0\\
380	0\\
381	0\\
382	0\\
383	0\\
384	0\\
385	0\\
386	0\\
387	0\\
388	0\\
389	0\\
390	0\\
391	0\\
392	0\\
393	0\\
394	0\\
395	0\\
396	0\\
397	0\\
398	0\\
399	0\\
400	0\\
401	0\\
402	0\\
403	0\\
404	0\\
405	0\\
406	0\\
407	0\\
408	0\\
409	0\\
410	0\\
411	0\\
412	0\\
413	0\\
414	0\\
415	0\\
416	0\\
417	0\\
418	0\\
419	0\\
420	0\\
421	0\\
422	0\\
423	0\\
424	0\\
425	0\\
426	0\\
427	0\\
428	0\\
429	0\\
430	0\\
431	0\\
432	0\\
433	0\\
434	0\\
435	0\\
436	0\\
437	0\\
438	0\\
439	0\\
440	0\\
441	0\\
442	0\\
443	0\\
444	0\\
445	0\\
446	0\\
447	0\\
448	0\\
449	0\\
450	0\\
451	0\\
452	0\\
453	0\\
454	0\\
455	0\\
456	0\\
457	0\\
458	0\\
459	0\\
460	0\\
461	0\\
462	0\\
463	0\\
464	0\\
465	0\\
466	0\\
467	0\\
468	0\\
469	0\\
470	0\\
471	0\\
472	0\\
473	0\\
474	0\\
475	0\\
476	0\\
477	0\\
478	0\\
479	0\\
480	0\\
481	0\\
482	0\\
483	0\\
484	0\\
485	0\\
486	0\\
487	0\\
488	0\\
489	0\\
490	0\\
491	0\\
492	0\\
493	0\\
494	0\\
495	0\\
496	0\\
497	0\\
498	0\\
499	0\\
500	0\\
501	0\\
502	0\\
503	0\\
504	0\\
505	0\\
506	0\\
507	0\\
508	0\\
509	0\\
510	0\\
511	0\\
512	0\\
513	0\\
514	0\\
515	0\\
516	0\\
517	0\\
518	0\\
519	0\\
520	0\\
521	0\\
522	0\\
523	0\\
524	0\\
525	0\\
526	0\\
527	0\\
528	0\\
529	0\\
530	0\\
531	0\\
532	0\\
533	0\\
534	0\\
535	0\\
536	0\\
537	0\\
538	0\\
539	0\\
540	0\\
541	0\\
542	0\\
543	0\\
544	0\\
545	0\\
546	0\\
547	0\\
548	0\\
549	0\\
550	0\\
551	0\\
552	0\\
553	0\\
554	0\\
555	0\\
556	0\\
557	0\\
558	0\\
559	0\\
560	0\\
561	0\\
562	0\\
563	0\\
564	0\\
565	0\\
566	0\\
567	0\\
568	0\\
569	0\\
570	0\\
571	0\\
572	0\\
573	0\\
574	0\\
575	0\\
576	0\\
577	0\\
578	0\\
579	0\\
580	0\\
581	0\\
582	0\\
583	0\\
584	0\\
585	0\\
586	0\\
587	0\\
588	0\\
589	0\\
590	0\\
591	0\\
592	0\\
593	0\\
594	0\\
595	0\\
596	0\\
597	0\\
598	0\\
599	0\\
600	0\\
};
\addplot [color=blue!40!mycolor9,solid,forget plot]
  table[row sep=crcr]{%
1	0\\
2	0\\
3	0\\
4	0\\
5	0\\
6	0\\
7	0\\
8	0\\
9	0\\
10	0\\
11	0\\
12	0\\
13	0\\
14	0\\
15	0\\
16	0\\
17	0\\
18	0\\
19	0\\
20	0\\
21	0\\
22	0\\
23	0\\
24	0\\
25	0\\
26	0\\
27	0\\
28	0\\
29	0\\
30	0\\
31	0\\
32	0\\
33	0\\
34	0\\
35	0\\
36	0\\
37	0\\
38	0\\
39	0\\
40	0\\
41	0\\
42	0\\
43	0\\
44	0\\
45	0\\
46	0\\
47	0\\
48	0\\
49	0\\
50	0\\
51	0\\
52	0\\
53	0\\
54	0\\
55	0\\
56	0\\
57	0\\
58	0\\
59	0\\
60	0\\
61	0\\
62	0\\
63	0\\
64	0\\
65	0\\
66	0\\
67	0\\
68	0\\
69	0\\
70	0\\
71	0\\
72	0\\
73	0\\
74	0\\
75	0\\
76	0\\
77	0\\
78	0\\
79	0\\
80	0\\
81	0\\
82	0\\
83	0\\
84	0\\
85	0\\
86	0\\
87	0\\
88	0\\
89	0\\
90	0\\
91	0\\
92	0\\
93	0\\
94	0\\
95	0\\
96	0\\
97	0\\
98	0\\
99	0\\
100	0\\
101	0\\
102	0\\
103	0\\
104	0\\
105	0\\
106	0\\
107	0\\
108	0\\
109	0\\
110	0\\
111	0\\
112	0\\
113	0\\
114	0\\
115	0\\
116	0\\
117	0\\
118	0\\
119	0\\
120	0\\
121	0\\
122	0\\
123	0\\
124	0\\
125	0\\
126	0\\
127	0\\
128	0\\
129	0\\
130	0\\
131	0\\
132	0\\
133	0\\
134	0\\
135	0\\
136	0\\
137	0\\
138	0\\
139	0\\
140	0\\
141	0\\
142	0\\
143	0\\
144	0\\
145	0\\
146	0\\
147	0\\
148	0\\
149	0\\
150	0\\
151	0\\
152	0\\
153	0\\
154	0\\
155	0\\
156	0\\
157	0\\
158	0\\
159	0\\
160	0\\
161	0\\
162	0\\
163	0\\
164	0\\
165	0\\
166	0\\
167	0\\
168	0\\
169	0\\
170	0\\
171	0\\
172	0\\
173	0\\
174	0\\
175	0\\
176	0\\
177	0\\
178	0\\
179	0\\
180	0\\
181	0\\
182	0\\
183	0\\
184	0\\
185	0\\
186	0\\
187	0\\
188	0\\
189	0\\
190	0\\
191	0\\
192	0\\
193	0\\
194	0\\
195	0\\
196	0\\
197	0\\
198	0\\
199	0\\
200	0\\
201	0\\
202	0\\
203	0\\
204	0\\
205	0\\
206	0\\
207	0\\
208	0\\
209	0\\
210	0\\
211	0\\
212	0\\
213	0\\
214	0\\
215	0\\
216	0\\
217	0\\
218	0\\
219	0\\
220	0\\
221	0\\
222	0\\
223	0\\
224	0\\
225	0\\
226	0\\
227	0\\
228	0\\
229	0\\
230	0\\
231	0\\
232	0\\
233	0\\
234	0\\
235	0\\
236	0\\
237	0\\
238	0\\
239	0\\
240	0\\
241	0\\
242	0\\
243	0\\
244	0\\
245	0\\
246	0\\
247	0\\
248	0\\
249	0\\
250	0\\
251	0\\
252	0\\
253	0\\
254	0\\
255	0\\
256	0\\
257	0\\
258	0\\
259	0\\
260	0\\
261	0\\
262	0\\
263	0\\
264	0\\
265	0\\
266	0\\
267	0\\
268	0\\
269	0\\
270	0\\
271	0\\
272	0\\
273	0\\
274	0\\
275	0\\
276	0\\
277	0\\
278	0\\
279	0\\
280	0\\
281	0\\
282	0\\
283	0\\
284	0\\
285	0\\
286	0\\
287	0\\
288	0\\
289	0\\
290	0\\
291	0\\
292	0\\
293	0\\
294	0\\
295	0\\
296	0\\
297	0\\
298	0\\
299	0\\
300	0\\
301	0\\
302	0\\
303	0\\
304	0\\
305	0\\
306	0\\
307	0\\
308	0\\
309	0\\
310	0\\
311	0\\
312	0\\
313	0\\
314	0\\
315	0\\
316	0\\
317	0\\
318	0\\
319	0\\
320	0\\
321	0\\
322	0\\
323	0\\
324	0\\
325	0\\
326	0\\
327	0\\
328	0\\
329	0\\
330	0\\
331	0\\
332	0\\
333	0\\
334	0\\
335	0\\
336	0\\
337	0\\
338	0\\
339	0\\
340	0\\
341	0\\
342	0\\
343	0\\
344	0\\
345	0\\
346	0\\
347	0\\
348	0\\
349	0\\
350	0\\
351	0\\
352	0\\
353	0\\
354	0\\
355	0\\
356	0\\
357	0\\
358	0\\
359	0\\
360	0\\
361	0\\
362	0\\
363	0\\
364	0\\
365	0\\
366	0\\
367	0\\
368	0\\
369	0\\
370	0\\
371	0\\
372	0\\
373	0\\
374	0\\
375	0\\
376	0\\
377	0\\
378	0\\
379	0\\
380	0\\
381	0\\
382	0\\
383	0\\
384	0\\
385	0\\
386	0\\
387	0\\
388	0\\
389	0\\
390	0\\
391	0\\
392	0\\
393	0\\
394	0\\
395	0\\
396	0\\
397	0\\
398	0\\
399	0\\
400	0\\
401	0\\
402	0\\
403	0\\
404	0\\
405	0\\
406	0\\
407	0\\
408	0\\
409	0\\
410	0\\
411	0\\
412	0\\
413	0\\
414	0\\
415	0\\
416	0\\
417	0\\
418	0\\
419	0\\
420	0\\
421	0\\
422	0\\
423	0\\
424	0\\
425	0\\
426	0\\
427	0\\
428	0\\
429	0\\
430	0\\
431	0\\
432	0\\
433	0\\
434	0\\
435	0\\
436	0\\
437	0\\
438	0\\
439	0\\
440	0\\
441	0\\
442	0\\
443	0\\
444	0\\
445	0\\
446	0\\
447	0\\
448	0\\
449	0\\
450	0\\
451	0\\
452	0\\
453	0\\
454	0\\
455	0\\
456	0\\
457	0\\
458	0\\
459	0\\
460	0\\
461	0\\
462	0\\
463	0\\
464	0\\
465	0\\
466	0\\
467	0\\
468	0\\
469	0\\
470	0\\
471	0\\
472	0\\
473	0\\
474	0\\
475	0\\
476	0\\
477	0\\
478	0\\
479	0\\
480	0\\
481	0\\
482	0\\
483	0\\
484	0\\
485	0\\
486	0\\
487	0\\
488	0\\
489	0\\
490	0\\
491	0\\
492	0\\
493	0\\
494	0\\
495	0\\
496	0\\
497	0\\
498	0\\
499	0\\
500	0\\
501	0\\
502	0\\
503	0\\
504	0\\
505	0\\
506	0\\
507	0\\
508	0\\
509	0\\
510	0\\
511	0\\
512	0\\
513	0\\
514	0\\
515	0\\
516	0\\
517	0\\
518	0\\
519	0\\
520	0\\
521	0\\
522	0\\
523	0\\
524	0\\
525	0\\
526	0\\
527	0\\
528	0\\
529	0\\
530	0\\
531	0\\
532	0\\
533	0\\
534	0\\
535	0\\
536	0\\
537	0\\
538	0\\
539	0\\
540	0\\
541	0\\
542	0\\
543	0\\
544	0\\
545	0\\
546	0\\
547	0\\
548	0\\
549	0\\
550	0\\
551	0\\
552	0\\
553	0\\
554	0\\
555	0\\
556	0\\
557	0\\
558	0\\
559	0\\
560	0\\
561	0\\
562	0\\
563	0\\
564	0\\
565	0\\
566	0\\
567	0\\
568	0\\
569	0\\
570	0\\
571	0\\
572	0\\
573	0\\
574	0\\
575	0\\
576	0\\
577	0\\
578	0\\
579	0\\
580	0\\
581	0\\
582	0\\
583	0\\
584	0\\
585	0\\
586	0\\
587	0\\
588	0\\
589	0\\
590	0\\
591	0\\
592	0\\
593	0\\
594	0\\
595	0\\
596	0\\
597	0\\
598	0\\
599	0\\
600	0\\
};
\addplot [color=blue!75!mycolor7,solid,forget plot]
  table[row sep=crcr]{%
1	0\\
2	0\\
3	0\\
4	0\\
5	0\\
6	0\\
7	0\\
8	0\\
9	0\\
10	0\\
11	0\\
12	0\\
13	0\\
14	0\\
15	0\\
16	0\\
17	0\\
18	0\\
19	0\\
20	0\\
21	0\\
22	0\\
23	0\\
24	0\\
25	0\\
26	0\\
27	0\\
28	0\\
29	0\\
30	0\\
31	0\\
32	0\\
33	0\\
34	0\\
35	0\\
36	0\\
37	0\\
38	0\\
39	0\\
40	0\\
41	0\\
42	0\\
43	0\\
44	0\\
45	0\\
46	0\\
47	0\\
48	0\\
49	0\\
50	0\\
51	0\\
52	0\\
53	0\\
54	0\\
55	0\\
56	0\\
57	0\\
58	0\\
59	0\\
60	0\\
61	0\\
62	0\\
63	0\\
64	0\\
65	0\\
66	0\\
67	0\\
68	0\\
69	0\\
70	0\\
71	0\\
72	0\\
73	0\\
74	0\\
75	0\\
76	0\\
77	0\\
78	0\\
79	0\\
80	0\\
81	0\\
82	0\\
83	0\\
84	0\\
85	0\\
86	0\\
87	0\\
88	0\\
89	0\\
90	0\\
91	0\\
92	0\\
93	0\\
94	0\\
95	0\\
96	0\\
97	0\\
98	0\\
99	0\\
100	0\\
101	0\\
102	0\\
103	0\\
104	0\\
105	0\\
106	0\\
107	0\\
108	0\\
109	0\\
110	0\\
111	0\\
112	0\\
113	0\\
114	0\\
115	0\\
116	0\\
117	0\\
118	0\\
119	0\\
120	0\\
121	0\\
122	0\\
123	0\\
124	0\\
125	0\\
126	0\\
127	0\\
128	0\\
129	0\\
130	0\\
131	0\\
132	0\\
133	0\\
134	0\\
135	0\\
136	0\\
137	0\\
138	0\\
139	0\\
140	0\\
141	0\\
142	0\\
143	0\\
144	0\\
145	0\\
146	0\\
147	0\\
148	0\\
149	0\\
150	0\\
151	0\\
152	0\\
153	0\\
154	0\\
155	0\\
156	0\\
157	0\\
158	0\\
159	0\\
160	0\\
161	0\\
162	0\\
163	0\\
164	0\\
165	0\\
166	0\\
167	0\\
168	0\\
169	0\\
170	0\\
171	0\\
172	0\\
173	0\\
174	0\\
175	0\\
176	0\\
177	0\\
178	0\\
179	0\\
180	0\\
181	0\\
182	0\\
183	0\\
184	0\\
185	0\\
186	0\\
187	0\\
188	0\\
189	0\\
190	0\\
191	0\\
192	0\\
193	0\\
194	0\\
195	0\\
196	0\\
197	0\\
198	0\\
199	0\\
200	0\\
201	0\\
202	0\\
203	0\\
204	0\\
205	0\\
206	0\\
207	0\\
208	0\\
209	0\\
210	0\\
211	0\\
212	0\\
213	0\\
214	0\\
215	0\\
216	0\\
217	0\\
218	0\\
219	0\\
220	0\\
221	0\\
222	0\\
223	0\\
224	0\\
225	0\\
226	0\\
227	0\\
228	0\\
229	0\\
230	0\\
231	0\\
232	0\\
233	0\\
234	0\\
235	0\\
236	0\\
237	0\\
238	0\\
239	0\\
240	0\\
241	0\\
242	0\\
243	0\\
244	0\\
245	0\\
246	0\\
247	0\\
248	0\\
249	0\\
250	0\\
251	0\\
252	0\\
253	0\\
254	0\\
255	0\\
256	0\\
257	0\\
258	0\\
259	0\\
260	0\\
261	0\\
262	0\\
263	0\\
264	0\\
265	0\\
266	0\\
267	0\\
268	0\\
269	0\\
270	0\\
271	0\\
272	0\\
273	0\\
274	0\\
275	0\\
276	0\\
277	0\\
278	0\\
279	0\\
280	0\\
281	0\\
282	0\\
283	0\\
284	0\\
285	0\\
286	0\\
287	0\\
288	0\\
289	0\\
290	0\\
291	0\\
292	0\\
293	0\\
294	0\\
295	0\\
296	0\\
297	0\\
298	0\\
299	0\\
300	0\\
301	0\\
302	0\\
303	0\\
304	0\\
305	0\\
306	0\\
307	0\\
308	0\\
309	0\\
310	0\\
311	0\\
312	0\\
313	0\\
314	0\\
315	0\\
316	0\\
317	0\\
318	0\\
319	0\\
320	0\\
321	0\\
322	0\\
323	0\\
324	0\\
325	0\\
326	0\\
327	0\\
328	0\\
329	0\\
330	0\\
331	0\\
332	0\\
333	0\\
334	0\\
335	0\\
336	0\\
337	0\\
338	0\\
339	0\\
340	0\\
341	0\\
342	0\\
343	0\\
344	0\\
345	0\\
346	0\\
347	0\\
348	0\\
349	0\\
350	0\\
351	0\\
352	0\\
353	0\\
354	0\\
355	0\\
356	0\\
357	0\\
358	0\\
359	0\\
360	0\\
361	0\\
362	0\\
363	0\\
364	0\\
365	0\\
366	0\\
367	0\\
368	0\\
369	0\\
370	0\\
371	0\\
372	0\\
373	0\\
374	0\\
375	0\\
376	0\\
377	0\\
378	0\\
379	0\\
380	0\\
381	0\\
382	0\\
383	0\\
384	0\\
385	0\\
386	0\\
387	0\\
388	0\\
389	0\\
390	0\\
391	0\\
392	0\\
393	0\\
394	0\\
395	0\\
396	0\\
397	0\\
398	0\\
399	0\\
400	0\\
401	0\\
402	0\\
403	0\\
404	0\\
405	0\\
406	0\\
407	0\\
408	0\\
409	0\\
410	0\\
411	0\\
412	0\\
413	0\\
414	0\\
415	0\\
416	0\\
417	0\\
418	0\\
419	0\\
420	0\\
421	0\\
422	0\\
423	0\\
424	0\\
425	0\\
426	0\\
427	0\\
428	0\\
429	0\\
430	0\\
431	0\\
432	0\\
433	0\\
434	0\\
435	0\\
436	0\\
437	0\\
438	0\\
439	0\\
440	0\\
441	0\\
442	0\\
443	0\\
444	0\\
445	0\\
446	0\\
447	0\\
448	0\\
449	0\\
450	0\\
451	0\\
452	0\\
453	0\\
454	0\\
455	0\\
456	0\\
457	0\\
458	0\\
459	0\\
460	0\\
461	0\\
462	0\\
463	0\\
464	0\\
465	0\\
466	0\\
467	0\\
468	0\\
469	0\\
470	0\\
471	0\\
472	0\\
473	0\\
474	0\\
475	0\\
476	0\\
477	0\\
478	0\\
479	0\\
480	0\\
481	0\\
482	0\\
483	0\\
484	0\\
485	0\\
486	0\\
487	0\\
488	0\\
489	0\\
490	0\\
491	0\\
492	0\\
493	0\\
494	0\\
495	0\\
496	0\\
497	0\\
498	0\\
499	0\\
500	0\\
501	0\\
502	0\\
503	0\\
504	0\\
505	0\\
506	0\\
507	0\\
508	0\\
509	0\\
510	0\\
511	0\\
512	0\\
513	0\\
514	0\\
515	0\\
516	0\\
517	0\\
518	0\\
519	0\\
520	0\\
521	0\\
522	0\\
523	0\\
524	0\\
525	0\\
526	0\\
527	0\\
528	0\\
529	0\\
530	0\\
531	0\\
532	0\\
533	0\\
534	0\\
535	0\\
536	0\\
537	0\\
538	0\\
539	0\\
540	0\\
541	0\\
542	0\\
543	0\\
544	0\\
545	0\\
546	0\\
547	0\\
548	0\\
549	0\\
550	0\\
551	0\\
552	0\\
553	0\\
554	0\\
555	0\\
556	0\\
557	0\\
558	0\\
559	0\\
560	0\\
561	0\\
562	0\\
563	0\\
564	0\\
565	0\\
566	0\\
567	0\\
568	0\\
569	0\\
570	0\\
571	0\\
572	0\\
573	0\\
574	0\\
575	0\\
576	0\\
577	0\\
578	0\\
579	0\\
580	0\\
581	0\\
582	0\\
583	0\\
584	0\\
585	0\\
586	0\\
587	0\\
588	0\\
589	0\\
590	0\\
591	0\\
592	0\\
593	0\\
594	0\\
595	0\\
596	0\\
597	0\\
598	0\\
599	0\\
600	0\\
};
\addplot [color=blue!80!mycolor9,solid,forget plot]
  table[row sep=crcr]{%
1	0.000420463240150569\\
2	0.00042044912030799\\
3	0.000420434759003375\\
4	0.000420420152089871\\
5	0.00042040529534938\\
6	0.000420390184491366\\
7	0.00042037481515161\\
8	0.00042035918289092\\
9	0.000420343283193896\\
10	0.000420327111467586\\
11	0.000420310663040172\\
12	0.000420293933159629\\
13	0.000420276916992343\\
14	0.000420259609621713\\
15	0.000420242006046732\\
16	0.00042022410118055\\
17	0.000420205889849007\\
18	0.000420187366789113\\
19	0.000420168526647563\\
20	0.000420149363979176\\
21	0.000420129873245317\\
22	0.000420110048812314\\
23	0.000420089884949808\\
24	0.000420069375829147\\
25	0.00042004851552164\\
26	0.00042002729799689\\
27	0.000420005717121061\\
28	0.000419983766655072\\
29	0.000419961440252826\\
30	0.000419938731459381\\
31	0.000419915633709058\\
32	0.0004198921403236\\
33	0.000419868244510194\\
34	0.000419843939359556\\
35	0.000419819217843911\\
36	0.000419794072814985\\
37	0.000419768497001938\\
38	0.000419742483009261\\
39	0.000419716023314677\\
40	0.000419689110266941\\
41	0.000419661736083651\\
42	0.000419633892849017\\
43	0.000419605572511561\\
44	0.00041957676688182\\
45	0.000419547467629971\\
46	0.000419517666283459\\
47	0.00041948735422453\\
48	0.000419456522687782\\
49	0.000419425162757618\\
50	0.000419393265365691\\
51	0.000419360821288323\\
52	0.000419327821143804\\
53	0.000419294255389739\\
54	0.000419260114320285\\
55	0.000419225388063368\\
56	0.000419190066577843\\
57	0.00041915413965062\\
58	0.000419117596893731\\
59	0.00041908042774133\\
60	0.000419042621446685\\
61	0.000419004167079077\\
62	0.000418965053520673\\
63	0.000418925269463327\\
64	0.000418884803405359\\
65	0.000418843643648231\\
66	0.000418801778293208\\
67	0.000418759195237954\\
68	0.000418715882173041\\
69	0.000418671826578452\\
70	0.000418627015719973\\
71	0.000418581436645575\\
72	0.000418535076181664\\
73	0.000418487920929352\\
74	0.000418439957260604\\
75	0.000418391171314359\\
76	0.000418341548992521\\
77	0.000418291075956001\\
78	0.000418239737620543\\
79	0.000418187519152627\\
80	0.000418134405465184\\
81	0.000418080381213302\\
82	0.000418025430789865\\
83	0.000417969538321074\\
84	0.000417912687661938\\
85	0.000417854862391662\\
86	0.000417796045808958\\
87	0.000417736220927311\\
88	0.000417675370470104\\
89	0.000417613476865731\\
90	0.00041755052224259\\
91	0.000417486488423966\\
92	0.000417421356922902\\
93	0.000417355108936909\\
94	0.000417287725342635\\
95	0.000417219186690436\\
96	0.000417149473198828\\
97	0.000417078564748892\\
98	0.000417006440878541\\
99	0.000416933080776717\\
100	0.000416858463277496\\
101	0.00041678256685407\\
102	0.000416705369612641\\
103	0.000416626849286235\\
104	0.000416546983228358\\
105	0.000416465748406587\\
106	0.000416383121396063\\
107	0.000416299078372831\\
108	0.000416213595107108\\
109	0.000416126646956417\\
110	0.000416038208858606\\
111	0.000415948255324737\\
112	0.000415856760431929\\
113	0.00041576369781595\\
114	0.000415669040663792\\
115	0.000415572761706077\\
116	0.000415474833209363\\
117	0.000415375226968243\\
118	0.000415273914297402\\
119	0.000415170866023496\\
120	0.000415066052476872\\
121	0.000414959443483184\\
122	0.000414851008354854\\
123	0.000414740715882372\\
124	0.000414628534325447\\
125	0.000414514431404022\\
126	0.000414398374289128\\
127	0.00041428032959356\\
128	0.000414160263362431\\
129	0.000414038141063515\\
130	0.000413913927577448\\
131	0.000413787587187775\\
132	0.000413659083570779\\
133	0.000413528379785166\\
134	0.000413395438261569\\
135	0.000413260220791854\\
136	0.000413122688518262\\
137	0.000412982801922331\\
138	0.000412840520813672\\
139	0.000412695804318507\\
140	0.000412548610868033\\
141	0.000412398898186595\\
142	0.000412246623279623\\
143	0.000412091742421269\\
144	0.000411934211141827\\
145	0.000411773984214908\\
146	0.00041161101564502\\
147	0.000411445258655372\\
148	0.000411276665673553\\
149	0.000411105188317988\\
150	0.000410930777384189\\
151	0.000410753382830757\\
152	0.000410572953765137\\
153	0.000410389438429166\\
154	0.00041020278418429\\
155	0.000410012937496649\\
156	0.00040981984392177\\
157	0.000409623448089111\\
158	0.000409423693686277\\
159	0.000409220523442977\\
160	0.000409013879114696\\
161	0.000408803701466113\\
162	0.000408589930254217\\
163	0.00040837250421111\\
164	0.000408151361026537\\
165	0.00040792643733012\\
166	0.00040769766867327\\
167	0.000407464989510785\\
168	0.000407228333182126\\
169	0.000406987631892374\\
170	0.000406742816692882\\
171	0.000406493817461535\\
172	0.0004062405628827\\
173	0.000405982980426843\\
174	0.000405720996329757\\
175	0.000405454535571414\\
176	0.000405183521854547\\
177	0.000404907877582701\\
178	0.000404627523838044\\
179	0.000404342380358698\\
180	0.000404052365515714\\
181	0.000403757396289626\\
182	0.000403457388246604\\
183	0.000403152255514178\\
184	0.000402841910756524\\
185	0.000402526265149358\\
186	0.000402205228354325\\
187	0.000401878708492977\\
188	0.000401546612120275\\
189	0.000401208844197626\\
190	0.00040086530806543\\
191	0.000400515905415176\\
192	0.000400160536260948\\
193	0.000399799098910555\\
194	0.000399431489936029\\
195	0.000399057604143654\\
196	0.000398677334543464\\
197	0.000398290572318125\\
198	0.000397897206791365\\
199	0.000397497125395721\\
200	0.000397090213639784\\
201	0.000396676355074823\\
202	0.000396255431260803\\
203	0.00039582732173178\\
204	0.000395391903960692\\
205	0.00039494905332346\\
206	0.000394498643062507\\
207	0.000394040544249511\\
208	0.000393574625747573\\
209	0.000393100754172621\\
210	0.000392618793854148\\
211	0.000392128606795174\\
212	0.000391630052631514\\
213	0.00039112298859028\\
214	0.000390607269447579\\
215	0.000390082747485471\\
216	0.000389549272448078\\
217	0.000389006691496911\\
218	0.000388454849165321\\
219	0.000387893587312119\\
220	0.000387322745074294\\
221	0.000386742158818885\\
222	0.000386151662093866\\
223	0.000385551085578178\\
224	0.000384940257030721\\
225	0.000384319001238465\\
226	0.000383687139963456\\
227	0.000383044491888938\\
228	0.00038239087256427\\
229	0.000381726094348952\\
230	0.000381049966355412\\
231	0.000380362294390771\\
232	0.00037966288089746\\
233	0.000378951524892641\\
234	0.000378228021906455\\
235	0.000377492163919073\\
236	0.000376743739296484\\
237	0.000375982532725003\\
238	0.000375208325144516\\
239	0.000374420893680352\\
240	0.000373620011573821\\
241	0.000372805448111359\\
242	0.000371976968552242\\
243	0.00037113433405487\\
244	0.000370277301601518\\
245	0.000369405623921625\\
246	0.00036851904941349\\
247	0.000367617322064392\\
248	0.000366700181369074\\
249	0.000365767362246593\\
250	0.000364818594955471\\
251	0.000363853605007071\\
252	0.000362872113077254\\
253	0.000361873834916212\\
254	0.000360858481256441\\
255	0.000359825757718859\\
256	0.000358775364716957\\
257	0.000357706997359041\\
258	0.000356620345348432\\
259	0.000355515092881627\\
260	0.000354390918544418\\
261	0.000353247495205839\\
262	0.000352084489909975\\
263	0.000350901563765589\\
264	0.000349698371833468\\
265	0.000348474563011527\\
266	0.000347229779917538\\
267	0.000345963658769561\\
268	0.000344675829263897\\
269	0.000343365914450595\\
270	0.000342033530606426\\
271	0.000340678287105166\\
272	0.000339299786285226\\
273	0.000337897623314623\\
274	0.000336471386054643\\
275	0.000335020654925319\\
276	0.000333545002775104\\
277	0.000332043994732079\\
278	0.000330517187994447\\
279	0.000328964131722927\\
280	0.000327384366888339\\
281	0.000325777426113161\\
282	0.000324142833509996\\
283	0.00032248010451688\\
284	0.000320788745729423\\
285	0.000319068254729721\\
286	0.000317318119912109\\
287	0.000315537820305653\\
288	0.000313726825393466\\
289	0.00031188459492877\\
290	0.000310010578747858\\
291	0.000308104216579848\\
292	0.00030616493785344\\
293	0.000304192161500579\\
294	0.000302185295757328\\
295	0.000300143737961838\\
296	0.000298066874349785\\
297	0.000295954079847361\\
298	0.000293804717861986\\
299	0.000291618140071164\\
300	0.00028939368620967\\
301	0.00028713068385552\\
302	0.000284828448215171\\
303	0.00028248628190848\\
304	0.000280103474754065\\
305	0.00027767930355581\\
306	0.000275213031891497\\
307	0.000272703909904661\\
308	0.000270151174100987\\
309	0.000267554047150566\\
310	0.000264911737695903\\
311	0.000262223440161519\\
312	0.000259488334553529\\
313	0.000256705586246752\\
314	0.00025387434585603\\
315	0.00025099374935309\\
316	0.000248062917857218\\
317	0.000245080957487294\\
318	0.00024204695931954\\
319	0.000238959999380692\\
320	0.00023581913868224\\
321	0.000232623423301923\\
322	0.00022937188451976\\
323	0.000226063539016568\\
324	0.000222697389144192\\
325	0.000219272423277816\\
326	0.000215787616262068\\
327	0.000212241929964235\\
328	0.000208634313949584\\
329	0.000204963706295905\\
330	0.000201229034566454\\
331	0.000197429216963232\\
332	0.000193563163685222\\
333	0.000189629778519623\\
334	0.000185627960697774\\
335	0.000181556607051661\\
336	0.000177414614511688\\
337	0.000173200882991794\\
338	0.000168914318714217\\
339	0.000164553838033065\\
340	0.000160118371823967\\
341	0.000155606870515935\\
342	0.000151018309852072\\
343	0.000146351697477284\\
344	0.000141606080464596\\
345	0.000136780553906929\\
346	0.000131874270718485\\
347	0.000126886452809903\\
348	0.000121816403823762\\
349	0.000116663523643309\\
350	0.000111427324916498\\
351	0.000106107451871358\\
352	0.000100703701736782\\
353	9.52160491260074e-05\\
354	8.96446737914473e-05\\
355	8.39899922321503e-05\\
356	7.82526937707919e-05\\
357	7.24337819585433e-05\\
358	6.65346219905049e-05\\
359	6.05569928351092e-05\\
360	5.45031451081293e-05\\
361	4.83758688722205e-05\\
362	4.21785594234741e-05\\
363	3.59152575781306e-05\\
364	2.95905654047414e-05\\
365	2.3209056409204e-05\\
366	1.67726731719349e-05\\
367	1.026963175326e-05\\
368	3.62063219975855e-06\\
369	0\\
370	0\\
371	0\\
372	0\\
373	0\\
374	0\\
375	0\\
376	0\\
377	0\\
378	0\\
379	0\\
380	0\\
381	0\\
382	0\\
383	0\\
384	0\\
385	0\\
386	0\\
387	0\\
388	0\\
389	0\\
390	0\\
391	0\\
392	0\\
393	0\\
394	0\\
395	0\\
396	0\\
397	0\\
398	0\\
399	0\\
400	0\\
401	0\\
402	0\\
403	0\\
404	0\\
405	0\\
406	0\\
407	0\\
408	0\\
409	0\\
410	0\\
411	0\\
412	0\\
413	0\\
414	0\\
415	0\\
416	0\\
417	0\\
418	0\\
419	0\\
420	0\\
421	0\\
422	0\\
423	0\\
424	0\\
425	0\\
426	0\\
427	0\\
428	0\\
429	0\\
430	0\\
431	0\\
432	0\\
433	0\\
434	0\\
435	0\\
436	0\\
437	0\\
438	0\\
439	0\\
440	0\\
441	0\\
442	0\\
443	0\\
444	0\\
445	0\\
446	0\\
447	0\\
448	0\\
449	0\\
450	0\\
451	0\\
452	0\\
453	0\\
454	0\\
455	0\\
456	0\\
457	0\\
458	0\\
459	0\\
460	0\\
461	0\\
462	0\\
463	0\\
464	0\\
465	0\\
466	0\\
467	0\\
468	0\\
469	0\\
470	0\\
471	0\\
472	0\\
473	0\\
474	0\\
475	0\\
476	0\\
477	0\\
478	0\\
479	0\\
480	0\\
481	0\\
482	0\\
483	0\\
484	0\\
485	0\\
486	0\\
487	0\\
488	0\\
489	0\\
490	0\\
491	0\\
492	0\\
493	0\\
494	0\\
495	0\\
496	0\\
497	0\\
498	0\\
499	0\\
500	0\\
501	0\\
502	0\\
503	0\\
504	0\\
505	0\\
506	0\\
507	0\\
508	0\\
509	0\\
510	0\\
511	0\\
512	0\\
513	0\\
514	0\\
515	0\\
516	0\\
517	0\\
518	0\\
519	0\\
520	0\\
521	0\\
522	0\\
523	0\\
524	0\\
525	0\\
526	0\\
527	0\\
528	0\\
529	0\\
530	0\\
531	0\\
532	0\\
533	0\\
534	0\\
535	0\\
536	0\\
537	0\\
538	0\\
539	0\\
540	0\\
541	0\\
542	0\\
543	0\\
544	0\\
545	0\\
546	0\\
547	0\\
548	0\\
549	0\\
550	0\\
551	0\\
552	0\\
553	0\\
554	0\\
555	0\\
556	0\\
557	0\\
558	0\\
559	0\\
560	0\\
561	0\\
562	0\\
563	0\\
564	0\\
565	0\\
566	0\\
567	0\\
568	0\\
569	0\\
570	0\\
571	0\\
572	0\\
573	0\\
574	0\\
575	0\\
576	0\\
577	0\\
578	0\\
579	0\\
580	0\\
581	0\\
582	0\\
583	0\\
584	0\\
585	0\\
586	0\\
587	0\\
588	0\\
589	0\\
590	0\\
591	0\\
592	0\\
593	0\\
594	0\\
595	0\\
596	0\\
597	0\\
598	0\\
599	0\\
600	0\\
};
\addplot [color=blue,solid,forget plot]
  table[row sep=crcr]{%
1	0.00225547679368458\\
2	0.00225546436267677\\
3	0.0022554517189349\\
4	0.00225543885880797\\
5	0.00225542577858241\\
6	0.00225541247448096\\
7	0.00225539894266163\\
8	0.00225538517921661\\
9	0.00225537118017108\\
10	0.00225535694148212\\
11	0.00225534245903754\\
12	0.00225532772865468\\
13	0.00225531274607924\\
14	0.00225529750698401\\
15	0.00225528200696767\\
16	0.00225526624155352\\
17	0.00225525020618817\\
18	0.00225523389624026\\
19	0.00225521730699912\\
20	0.00225520043367341\\
21	0.00225518327138976\\
22	0.00225516581519137\\
23	0.00225514806003657\\
24	0.0022551300007974\\
25	0.00225511163225812\\
26	0.00225509294911375\\
27	0.00225507394596848\\
28	0.00225505461733422\\
29	0.00225503495762892\\
30	0.00225501496117508\\
31	0.00225499462219804\\
32	0.00225497393482437\\
33	0.00225495289308018\\
34	0.00225493149088941\\
35	0.00225490972207209\\
36	0.00225488758034257\\
37	0.00225486505930775\\
38	0.00225484215246521\\
39	0.00225481885320136\\
40	0.00225479515478961\\
41	0.00225477105038835\\
42	0.00225474653303909\\
43	0.00225472159566441\\
44	0.00225469623106597\\
45	0.00225467043192244\\
46	0.00225464419078744\\
47	0.00225461750008738\\
48	0.00225459035211933\\
49	0.00225456273904881\\
50	0.00225453465290756\\
51	0.00225450608559128\\
52	0.0022544770288573\\
53	0.00225444747432227\\
54	0.00225441741345974\\
55	0.00225438683759773\\
56	0.00225435573791633\\
57	0.00225432410544509\\
58	0.00225429193106057\\
59	0.00225425920548367\\
60	0.00225422591927705\\
61	0.0022541920628424\\
62	0.00225415762641777\\
63	0.00225412260007477\\
64	0.00225408697371574\\
65	0.00225405073707093\\
66	0.00225401387969554\\
67	0.0022539763909668\\
68	0.00225393826008092\\
69	0.0022538994760501\\
70	0.00225386002769933\\
71	0.0022538199036633\\
72	0.00225377909238315\\
73	0.00225373758210322\\
74	0.0022536953608677\\
75	0.00225365241651727\\
76	0.00225360873668569\\
77	0.00225356430879626\\
78	0.0022535191200583\\
79	0.00225347315746352\\
80	0.0022534264077824\\
81	0.00225337885756039\\
82	0.00225333049311418\\
83	0.00225328130052779\\
84	0.0022532312656487\\
85	0.00225318037408382\\
86	0.00225312861119549\\
87	0.00225307596209731\\
88	0.00225302241164999\\
89	0.00225296794445707\\
90	0.0022529125448606\\
91	0.00225285619693674\\
92	0.0022527988844913\\
93	0.00225274059105517\\
94	0.0022526812998797\\
95	0.00225262099393201\\
96	0.00225255965589022\\
97	0.00225249726813854\\
98	0.00225243381276239\\
99	0.00225236927154336\\
100	0.00225230362595409\\
101	0.00225223685715307\\
102	0.00225216894597939\\
103	0.00225209987294737\\
104	0.00225202961824105\\
105	0.00225195816170873\\
106	0.00225188548285725\\
107	0.00225181156084631\\
108	0.00225173637448258\\
109	0.00225165990221383\\
110	0.00225158212212285\\
111	0.00225150301192136\\
112	0.00225142254894375\\
113	0.00225134071014072\\
114	0.00225125747207289\\
115	0.00225117281090421\\
116	0.00225108670239529\\
117	0.00225099912189661\\
118	0.0022509100443417\\
119	0.00225081944424\\
120	0.00225072729566989\\
121	0.00225063357227131\\
122	0.00225053824723845\\
123	0.00225044129331226\\
124	0.0022503426827728\\
125	0.0022502423874315\\
126	0.0022501403786233\\
127	0.00225003662719858\\
128	0.00224993110351509\\
129	0.00224982377742956\\
130	0.00224971461828938\\
131	0.00224960359492395\\
132	0.00224949067563599\\
133	0.00224937582819269\\
134	0.00224925901981669\\
135	0.00224914021717691\\
136	0.00224901938637924\\
137	0.00224889649295707\\
138	0.00224877150186167\\
139	0.00224864437745237\\
140	0.00224851508348662\\
141	0.00224838358310987\\
142	0.00224824983884521\\
143	0.00224811381258292\\
144	0.00224797546556979\\
145	0.00224783475839845\\
146	0.00224769165099652\\
147	0.00224754610261519\\
148	0.0022473980718179\\
149	0.00224724751646874\\
150	0.00224709439372072\\
151	0.00224693866000382\\
152	0.00224678027101282\\
153	0.00224661918169497\\
154	0.0022464553462374\\
155	0.00224628871805434\\
156	0.00224611924977416\\
157	0.00224594689322612\\
158	0.00224577159942695\\
159	0.00224559331856717\\
160	0.00224541199999721\\
161	0.00224522759221328\\
162	0.00224504004284297\\
163	0.00224484929863065\\
164	0.00224465530542263\\
165	0.002244458008152\\
166	0.00224425735082327\\
167	0.00224405327649673\\
168	0.00224384572727259\\
169	0.00224363464427471\\
170	0.00224341996763425\\
171	0.00224320163647287\\
172	0.00224297958888574\\
173	0.0022427537619242\\
174	0.00224252409157822\\
175	0.0022422905127584\\
176	0.00224205295927782\\
177	0.00224181136383346\\
178	0.0022415656579874\\
179	0.00224131577214762\\
180	0.00224106163554846\\
181	0.00224080317623084\\
182	0.00224054032102202\\
183	0.00224027299551509\\
184	0.00224000112404807\\
185	0.00223972462968265\\
186	0.00223944343418256\\
187	0.00223915745799157\\
188	0.00223886662021111\\
189	0.00223857083857746\\
190	0.0022382700294386\\
191	0.00223796410773064\\
192	0.00223765298695376\\
193	0.00223733657914782\\
194	0.00223701479486753\\
195	0.00223668754315712\\
196	0.00223635473152461\\
197	0.00223601626591562\\
198	0.0022356720506867\\
199	0.00223532198857818\\
200	0.00223496598068657\\
201	0.00223460392643642\\
202	0.00223423572355172\\
203	0.00223386126802676\\
204	0.00223348045409645\\
205	0.00223309317420614\\
206	0.00223269931898089\\
207	0.00223229877719413\\
208	0.00223189143573586\\
209	0.00223147717958013\\
210	0.00223105589175204\\
211	0.00223062745329407\\
212	0.00223019174323182\\
213	0.00222974863853913\\
214	0.00222929801410253\\
215	0.002228839742685\\
216	0.00222837369488917\\
217	0.00222789973911968\\
218	0.00222741774154498\\
219	0.00222692756605833\\
220	0.00222642907423807\\
221	0.00222592212530717\\
222	0.00222540657609198\\
223	0.00222488228098028\\
224	0.00222434909187839\\
225	0.00222380685816758\\
226	0.0022232554266596\\
227	0.00222269464155133\\
228	0.0022221243443786\\
229	0.00222154437396911\\
230	0.00222095456639435\\
231	0.00222035475492071\\
232	0.00221974476995955\\
233	0.00221912443901628\\
234	0.00221849358663846\\
235	0.00221785203436292\\
236	0.00221719960066168\\
237	0.00221653610088695\\
238	0.00221586134721493\\
239	0.00221517514858848\\
240	0.00221447731065868\\
241	0.00221376763572513\\
242	0.0022130459226751\\
243	0.00221231196692134\\
244	0.00221156556033872\\
245	0.00221080649119944\\
246	0.00221003454410697\\
247	0.00220924949992857\\
248	0.00220845113572638\\
249	0.00220763922468709\\
250	0.00220681353605007\\
251	0.00220597383503397\\
252	0.00220511988276175\\
253	0.00220425143618409\\
254	0.00220336824800114\\
255	0.00220247006658248\\
256	0.00220155663588542\\
257	0.00220062769537139\\
258	0.00219968297992052\\
259	0.00219872221974431\\
260	0.0021977451402962\\
261	0.00219675146218023\\
262	0.00219574090105754\\
263	0.00219471316755064\\
264	0.00219366796714554\\
265	0.00219260500009143\\
266	0.00219152396129811\\
267	0.00219042454023081\\
268	0.00218930642080246\\
269	0.00218816928126336\\
270	0.00218701279408796\\
271	0.00218583662585887\\
272	0.00218464043714798\\
273	0.00218342388239507\\
274	0.00218218660978429\\
275	0.00218092826111787\\
276	0.00217964847168101\\
277	0.00217834687009569\\
278	0.00217702307819071\\
279	0.00217567671085904\\
280	0.00217430737591045\\
281	0.00217291467392013\\
282	0.00217149819807274\\
283	0.00217005753400202\\
284	0.00216859225962569\\
285	0.00216710194497527\\
286	0.00216558615202082\\
287	0.00216404443449025\\
288	0.00216247633768293\\
289	0.00216088139827743\\
290	0.00215925914413306\\
291	0.00215760909408485\\
292	0.00215593075773177\\
293	0.00215422363521778\\
294	0.00215248721700542\\
295	0.00215072098364137\\
296	0.00214892440551389\\
297	0.00214709694260139\\
298	0.00214523804421187\\
299	0.00214334714871261\\
300	0.00214142368324968\\
301	0.00213946706345662\\
302	0.00213747669315173\\
303	0.00213545196402326\\
304	0.00213339225530193\\
305	0.00213129693341986\\
306	0.00212916535165536\\
307	0.00212699684976246\\
308	0.00212479075358427\\
309	0.00212254637464819\\
310	0.00212026300974022\\
311	0.0021179399404549\\
312	0.00211557643272357\\
313	0.0021131717363424\\
314	0.00211072508450726\\
315	0.00210823569325357\\
316	0.00210570276087912\\
317	0.00210312546735227\\
318	0.0021005029736881\\
319	0.00209783442128978\\
320	0.00209511893125242\\
321	0.00209235560362633\\
322	0.00208954351663622\\
323	0.00208668172585272\\
324	0.00208376926331188\\
325	0.00208080513657825\\
326	0.00207778832774644\\
327	0.00207471779237544\\
328	0.00207159245834967\\
329	0.0020684112246598\\
330	0.00206517296009569\\
331	0.00206187650184302\\
332	0.00205852065397409\\
333	0.00205510418582252\\
334	0.00205162583022995\\
335	0.00204808428165185\\
336	0.00204447819410802\\
337	0.00204080617896145\\
338	0.00203706680250769\\
339	0.00203325858335442\\
340	0.00202937998956875\\
341	0.00202542943556714\\
342	0.00202140527871949\\
343	0.00201730581563618\\
344	0.00201312927810233\\
345	0.00200887382861977\\
346	0.00200453755551226\\
347	0.00200011846754379\\
348	0.00199561448799404\\
349	0.00199102344812764\\
350	0.00198634307998626\\
351	0.00198157100842312\\
352	0.00197670474228982\\
353	0.00197174166467503\\
354	0.0019666790220888\\
355	0.00196151391248782\\
356	0.0019562432720273\\
357	0.00195086386027465\\
358	0.00194537224329756\\
359	0.00193976477445752\\
360	0.00193403757236674\\
361	0.00192818649089298\\
362	0.00192220706749014\\
363	0.00191609439936248\\
364	0.00190984276577908\\
365	0.001903444344481\\
366	0.00189688471607222\\
367	0.00189012753770206\\
368	0.00188307204594991\\
369	0.00187532607576948\\
370	0.00186741687321954\\
371	0.00185937371942756\\
372	0.00185119429012358\\
373	0.00184287622349694\\
374	0.00183441711805204\\
375	0.0018258145280678\\
376	0.00181706595713902\\
377	0.00180816886457825\\
378	0.0017991207509713\\
379	0.00178991932478652\\
380	0.00178056209059273\\
381	0.00177104636415697\\
382	0.0017613694237474\\
383	0.00175152851070232\\
384	0.00174152083010462\\
385	0.00173134355156662\\
386	0.00172099381012973\\
387	0.00171046870728252\\
388	0.00169976531209971\\
389	0.00168888066250364\\
390	0.00167781176664776\\
391	0.00166655560442005\\
392	0.00165510912906219\\
393	0.00164346926889762\\
394	0.00163163292916047\\
395	0.00161959699391801\\
396	0.00160735832808923\\
397	0.00159491377959592\\
398	0.00158226018178169\\
399	0.00156939435649901\\
400	0.00155631311892567\\
401	0.00154301328669679\\
402	0.00152949169902099\\
403	0.00151574525607928\\
404	0.00150177098980964\\
405	0.0014875661528874\\
406	0.00147312828649604\\
407	0.00145845627135828\\
408	0.00144355769227254\\
409	0.00142846262402851\\
410	0.00141326501158894\\
411	0.00139822731853432\\
412	0.00138388740253358\\
413	0.00137011704235746\\
414	0.00135607976325594\\
415	0.00134177068139713\\
416	0.0013271838316562\\
417	0.00131231249661213\\
418	0.0012971497300087\\
419	0.00128168834618063\\
420	0.00126592090865607\\
421	0.00124983971672581\\
422	0.00123343678767319\\
423	0.00121670383904977\\
424	0.00119963228876237\\
425	0.00118221324775915\\
426	0.00116443749003025\\
427	0.00114629543543994\\
428	0.0011277771315331\\
429	0.00110887223425105\\
430	0.00108956998748904\\
431	0.00106985920142357\\
432	0.00104972822953443\\
433	0.00102916494424355\\
434	0.00100815671108913\\
435	0.000986690361352274\\
436	0.000964752163052489\\
437	0.000942327790230635\\
438	0.000919402290441712\\
439	0.000895960050373461\\
440	0.000871984759420209\\
441	0.000847459370560392\\
442	0.000822366055769701\\
443	0.000796686145148563\\
444	0.000770400012820854\\
445	0.0007434868148554\\
446	0.000715924066459259\\
447	0.000687689175228605\\
448	0.000658765806217179\\
449	0.000629129746906497\\
450	0.000598755261626512\\
451	0.00056761530781351\\
452	0.00053568151467907\\
453	0.000502924087514885\\
454	0.000469311590360925\\
455	0.000434810187659242\\
456	0.000399380515904592\\
457	0.000362964325713837\\
458	0.000325427915679923\\
459	0.000286121859631025\\
460	0.00024577460918375\\
461	0.000204529627511481\\
462	0.000162346680408577\\
463	0.000119118848819356\\
464	7.45041232746955e-05\\
465	2.6781960106478e-05\\
466	0\\
467	0\\
468	0\\
469	0\\
470	0\\
471	0\\
472	0\\
473	0\\
474	0\\
475	0\\
476	0\\
477	0\\
478	0\\
479	0\\
480	0\\
481	0\\
482	0\\
483	0\\
484	0\\
485	0\\
486	0\\
487	0\\
488	0\\
489	0\\
490	0\\
491	0\\
492	0\\
493	0\\
494	0\\
495	0\\
496	0\\
497	0\\
498	0\\
499	0\\
500	0\\
501	0\\
502	0\\
503	0\\
504	0\\
505	0\\
506	0\\
507	0\\
508	0\\
509	0\\
510	0\\
511	0\\
512	0\\
513	0\\
514	0\\
515	0\\
516	0\\
517	0\\
518	0\\
519	0\\
520	0\\
521	0\\
522	0\\
523	0\\
524	0\\
525	0\\
526	0\\
527	0\\
528	0\\
529	0\\
530	0\\
531	0\\
532	0\\
533	0\\
534	0\\
535	0\\
536	0\\
537	0\\
538	0\\
539	0\\
540	0\\
541	0\\
542	0\\
543	0\\
544	0\\
545	0\\
546	0\\
547	0\\
548	0\\
549	0\\
550	0\\
551	0\\
552	0\\
553	0\\
554	0\\
555	0\\
556	0\\
557	0\\
558	0\\
559	0\\
560	0\\
561	0\\
562	0\\
563	0\\
564	0\\
565	0\\
566	0\\
567	0\\
568	0\\
569	0\\
570	0\\
571	0\\
572	0\\
573	0\\
574	0\\
575	0\\
576	0\\
577	0\\
578	0\\
579	0\\
580	0\\
581	0\\
582	0\\
583	0\\
584	0\\
585	0\\
586	0\\
587	0\\
588	0\\
589	0\\
590	0\\
591	0\\
592	0\\
593	0\\
594	0\\
595	0\\
596	0\\
597	0\\
598	0\\
599	0\\
600	0\\
};
\addplot [color=mycolor10,solid,forget plot]
  table[row sep=crcr]{%
1	0.00367136005952458\\
2	0.0036713542087014\\
3	0.00367134825773797\\
4	0.00367134220491635\\
5	0.00367133604848915\\
6	0.00367132978667907\\
7	0.00367132341767836\\
8	0.00367131693964833\\
9	0.00367131035071882\\
10	0.00367130364898763\\
11	0.00367129683252001\\
12	0.0036712898993481\\
13	0.00367128284747037\\
14	0.003671275674851\\
15	0.00367126837941938\\
16	0.00367126095906944\\
17	0.00367125341165909\\
18	0.00367124573500959\\
19	0.00367123792690494\\
20	0.00367122998509123\\
21	0.00367122190727602\\
22	0.00367121369112766\\
23	0.00367120533427462\\
24	0.00367119683430483\\
25	0.003671188188765\\
26	0.0036711793951599\\
27	0.00367117045095163\\
28	0.00367116135355894\\
29	0.00367115210035647\\
30	0.00367114268867399\\
31	0.00367113311579566\\
32	0.00367112337895925\\
33	0.00367111347535534\\
34	0.00367110340212652\\
35	0.00367109315636661\\
36	0.00367108273511978\\
37	0.00367107213537974\\
38	0.00367106135408891\\
39	0.00367105038813747\\
40	0.00367103923436256\\
41	0.00367102788954732\\
42	0.00367101635042003\\
43	0.00367100461365313\\
44	0.0036709926758623\\
45	0.00367098053360549\\
46	0.00367096818338196\\
47	0.00367095562163127\\
48	0.00367094284473225\\
49	0.00367092984900204\\
50	0.00367091663069497\\
51	0.00367090318600154\\
52	0.00367088951104733\\
53	0.00367087560189192\\
54	0.00367086145452774\\
55	0.00367084706487896\\
56	0.00367083242880035\\
57	0.00367081754207607\\
58	0.0036708024004185\\
59	0.00367078699946705\\
60	0.00367077133478688\\
61	0.00367075540186769\\
62	0.00367073919612242\\
63	0.00367072271288598\\
64	0.00367070594741391\\
65	0.00367068889488107\\
66	0.00367067155038026\\
67	0.00367065390892083\\
68	0.0036706359654273\\
69	0.00367061771473792\\
70	0.00367059915160324\\
71	0.00367058027068457\\
72	0.00367056106655255\\
73	0.0036705415336856\\
74	0.00367052166646835\\
75	0.0036705014591901\\
76	0.00367048090604318\\
77	0.00367046000112136\\
78	0.00367043873841816\\
79	0.00367041711182519\\
80	0.00367039511513042\\
81	0.00367037274201647\\
82	0.0036703499860588\\
83	0.00367032684072394\\
84	0.00367030329936769\\
85	0.00367027935523319\\
86	0.0036702550014491\\
87	0.00367023023102764\\
88	0.00367020503686266\\
89	0.00367017941172767\\
90	0.00367015334827377\\
91	0.00367012683902765\\
92	0.0036700998763895\\
93	0.00367007245263087\\
94	0.00367004455989254\\
95	0.00367001619018233\\
96	0.00366998733537283\\
97	0.00366995798719921\\
98	0.00366992813725689\\
99	0.00366989777699916\\
100	0.00366986689773488\\
101	0.003669835490626\\
102	0.00366980354668515\\
103	0.00366977105677311\\
104	0.00366973801159628\\
105	0.0036697044017041\\
106	0.00366967021748645\\
107	0.00366963544917095\\
108	0.00366960008682027\\
109	0.00366956412032934\\
110	0.00366952753942259\\
111	0.00366949033365109\\
112	0.00366945249238962\\
113	0.00366941400483377\\
114	0.00366937485999693\\
115	0.00366933504670721\\
116	0.00366929455360442\\
117	0.00366925336913683\\
118	0.00366921148155806\\
119	0.00366916887892377\\
120	0.00366912554908839\\
121	0.0036690814797017\\
122	0.00366903665820549\\
123	0.00366899107183002\\
124	0.00366894470759055\\
125	0.00366889755228367\\
126	0.00366884959248373\\
127	0.00366880081453906\\
128	0.00366875120456826\\
129	0.0036687007484563\\
130	0.00366864943185067\\
131	0.0036685972401574\\
132	0.00366854415853704\\
133	0.00366849017190053\\
134	0.00366843526490509\\
135	0.00366837942194993\\
136	0.00366832262717202\\
137	0.00366826486444162\\
138	0.00366820611735795\\
139	0.00366814636924458\\
140	0.00366808560314488\\
141	0.00366802380181734\\
142	0.00366796094773078\\
143	0.00366789702305959\\
144	0.0036678320096788\\
145	0.00366776588915917\\
146	0.00366769864276207\\
147	0.00366763025143433\\
148	0.00366756069580298\\
149	0.00366748995616998\\
150	0.00366741801250679\\
151	0.00366734484444885\\
152	0.00366727043129002\\
153	0.0036671947519769\\
154	0.00366711778510305\\
155	0.0036670395089031\\
156	0.00366695990124682\\
157	0.00366687893963302\\
158	0.0036667966011834\\
159	0.00366671286263627\\
160	0.0036666277003402\\
161	0.00366654109024748\\
162	0.00366645300790763\\
163	0.0036663634284606\\
164	0.00366627232663002\\
165	0.00366617967671628\\
166	0.00366608545258947\\
167	0.00366598962768222\\
168	0.00366589217498246\\
169	0.00366579306702598\\
170	0.00366569227588894\\
171	0.00366558977318025\\
172	0.0036654855300337\\
173	0.0036653795171002\\
174	0.00366527170453961\\
175	0.00366516206201264\\
176	0.00366505055867255\\
177	0.00366493716315665\\
178	0.00366482184357777\\
179	0.00366470456751549\\
180	0.00366458530200726\\
181	0.00366446401353939\\
182	0.00366434066803786\\
183	0.00366421523085895\\
184	0.0036640876667798\\
185	0.00366395793998869\\
186	0.00366382601407527\\
187	0.00366369185202057\\
188	0.00366355541618677\\
189	0.00366341666830697\\
190	0.00366327556947463\\
191	0.00366313208013285\\
192	0.00366298616006357\\
193	0.00366283776837647\\
194	0.00366268686349774\\
195	0.00366253340315862\\
196	0.00366237734438379\\
197	0.0036622186434795\\
198	0.00366205725602156\\
199	0.00366189313684307\\
200	0.00366172624002197\\
201	0.00366155651886834\\
202	0.00366138392591151\\
203	0.00366120841288693\\
204	0.00366102993072281\\
205	0.00366084842952655\\
206	0.00366066385857087\\
207	0.00366047616627979\\
208	0.00366028530021424\\
209	0.00366009120705756\\
210	0.00365989383260061\\
211	0.00365969312172667\\
212	0.00365948901839613\\
213	0.00365928146563077\\
214	0.0036590704054979\\
215	0.0036588557790941\\
216	0.00365863752652877\\
217	0.00365841558690728\\
218	0.00365818989831388\\
219	0.0036579603977943\\
220	0.00365772702133799\\
221	0.00365748970386007\\
222	0.00365724837918293\\
223	0.00365700298001751\\
224	0.00365675343794421\\
225	0.00365649968339345\\
226	0.0036562416456259\\
227	0.00365597925271227\\
228	0.00365571243151281\\
229	0.00365544110765633\\
230	0.00365516520551892\\
231	0.00365488464820216\\
232	0.00365459935751102\\
233	0.00365430925393122\\
234	0.0036540142566063\\
235	0.00365371428331407\\
236	0.00365340925044277\\
237	0.00365309907296666\\
238	0.00365278366442114\\
239	0.00365246293687742\\
240	0.00365213680091666\\
241	0.00365180516560358\\
242	0.00365146793845959\\
243	0.00365112502543529\\
244	0.00365077633088253\\
245	0.00365042175752576\\
246	0.00365006120643292\\
247	0.00364969457698562\\
248	0.00364932176684879\\
249	0.0036489426719396\\
250	0.00364855718639577\\
251	0.00364816520254322\\
252	0.00364776661086298\\
253	0.0036473612999574\\
254	0.0036469491565156\\
255	0.00364653006527817\\
256	0.0036461039090011\\
257	0.00364567056841883\\
258	0.0036452299222065\\
259	0.00364478184694136\\
260	0.00364432621706316\\
261	0.00364386290483372\\
262	0.00364339178029548\\
263	0.00364291271122899\\
264	0.00364242556310947\\
265	0.00364193019906216\\
266	0.00364142647981658\\
267	0.00364091426365965\\
268	0.00364039340638746\\
269	0.00363986376125587\\
270	0.00363932517892966\\
271	0.00363877750743043\\
272	0.00363822059208316\\
273	0.00363765427546155\\
274	0.00363707839733184\\
275	0.0036364927945936\\
276	0.00363589730121682\\
277	0.00363529174818207\\
278	0.00363467596341668\\
279	0.00363404977172866\\
280	0.00363341299473854\\
281	0.00363276545080885\\
282	0.00363210695497132\\
283	0.00363143731885151\\
284	0.0036307563505908\\
285	0.00363006385476552\\
286	0.00362935963230314\\
287	0.00362864348039527\\
288	0.00362791519240729\\
289	0.0036271745577844\\
290	0.00362642136195385\\
291	0.00362565538622313\\
292	0.00362487640767375\\
293	0.00362408419905044\\
294	0.00362327852864536\\
295	0.00362245916017698\\
296	0.0036216258526632\\
297	0.00362077836028839\\
298	0.00361991643226378\\
299	0.00361903981268069\\
300	0.00361814824035613\\
301	0.00361724144867005\\
302	0.00361631916539358\\
303	0.00361538111250761\\
304	0.00361442700601071\\
305	0.00361345655571573\\
306	0.00361246946503382\\
307	0.00361146543074481\\
308	0.00361044414275246\\
309	0.00360940528382247\\
310	0.0036083485293015\\
311	0.00360727354681671\\
312	0.00360617999595942\\
313	0.00360506752795359\\
314	0.00360393578528118\\
315	0.00360278440128144\\
316	0.00360161299972272\\
317	0.00360042119433957\\
318	0.00359920858833189\\
319	0.00359797477382177\\
320	0.0035967193312638\\
321	0.00359544182880375\\
322	0.00359414182157997\\
323	0.00359281885096147\\
324	0.00359147244371551\\
325	0.00359010211109698\\
326	0.00358870734785085\\
327	0.00358728763111789\\
328	0.00358584241923271\\
329	0.00358437115040184\\
330	0.00358287324124808\\
331	0.00358134808520573\\
332	0.00357979505074929\\
333	0.00357821347943625\\
334	0.00357660268374211\\
335	0.00357496194466306\\
336	0.00357329050905873\\
337	0.00357158758670406\\
338	0.00356985234701524\\
339	0.00356808391541051\\
340	0.00356628136926162\\
341	0.00356444373338597\\
342	0.0035625699750233\\
343	0.00356065899823356\\
344	0.00355870963764444\\
345	0.0035567206514678\\
346	0.00355469071369377\\
347	0.00355261840535946\\
348	0.00355050220477556\\
349	0.00354834047657885\\
350	0.0035461314594607\\
351	0.00354387325240232\\
352	0.00354156379922452\\
353	0.00353920087123593\\
354	0.00353678204773568\\
355	0.00353430469408663\\
356	0.00353176593699083\\
357	0.00352916263644676\\
358	0.00352649135382048\\
359	0.00352374831505654\\
360	0.00352092936611005\\
361	0.00351802991340023\\
362	0.00351504482719273\\
363	0.00351196823786181\\
364	0.00350879299930346\\
365	0.00350550909462353\\
366	0.00350209870266089\\
367	0.00349852084308253\\
368	0.00349465963001152\\
369	0.00349010686104009\\
370	0.00348544033389064\\
371	0.00348070343857037\\
372	0.00347589524788633\\
373	0.0034710148278648\\
374	0.00346606123691451\\
375	0.00346103352543611\\
376	0.00345593074002277\\
377	0.00345075194222006\\
378	0.00344549623478454\\
379	0.0034401626745596\\
380	0.00343475028031016\\
381	0.0034292580655807\\
382	0.00342368503853128\\
383	0.00341803020166984\\
384	0.00341229255146185\\
385	0.00340647107779616\\
386	0.00340056476328271\\
387	0.00339457258235557\\
388	0.00338849350015083\\
389	0.00338232647112589\\
390	0.00337607043738275\\
391	0.00336972432665376\\
392	0.0033632870499045\\
393	0.00335675749850444\\
394	0.00335013454091347\\
395	0.00334341701883367\\
396	0.00333660374278532\\
397	0.00332969348709849\\
398	0.00332268498439105\\
399	0.00331557691978132\\
400	0.00330836792542846\\
401	0.00330105657652901\\
402	0.00329364139035451\\
403	0.00328612082939462\\
404	0.00327849330934434\\
405	0.00327075724168134\\
406	0.00326291136982867\\
407	0.00325495636404665\\
408	0.00324689763874729\\
409	0.00323875231388637\\
410	0.00323056097037767\\
411	0.00322238312421398\\
412	0.00321417988858275\\
413	0.0032058389817077\\
414	0.00319735795331092\\
415	0.00318873408097052\\
416	0.00317996444183571\\
417	0.00317104601964941\\
418	0.0031619757002138\\
419	0.00315275026644325\\
420	0.00314336639261583\\
421	0.00313382063738896\\
422	0.00312410943678926\\
423	0.00311422910202385\\
424	0.00310417581523831\\
425	0.00309394561880077\\
426	0.00308353440793545\\
427	0.00307293792291095\\
428	0.00306215174075572\\
429	0.00305117126647352\\
430	0.00303999172372835\\
431	0.00302860814496947\\
432	0.0030170153609647\\
433	0.00300520798970998\\
434	0.00299318042468135\\
435	0.00298092682239276\\
436	0.00296844108921502\\
437	0.00295571686738979\\
438	0.00294274752011353\\
439	0.00292952611540876\\
440	0.00291604540808881\\
441	0.00290229781811762\\
442	0.00288827540156554\\
443	0.00287396980789372\\
444	0.00285937222539627\\
445	0.00284447339254048\\
446	0.00282926403681843\\
447	0.0028137358241746\\
448	0.00279787831489232\\
449	0.00278168049455409\\
450	0.00276513081731052\\
451	0.00274821718821128\\
452	0.00273092691441245\\
453	0.00271324658327365\\
454	0.00269516167107363\\
455	0.00267665513762852\\
456	0.0026577021184723\\
457	0.00263824930352147\\
458	0.00261813291909802\\
459	0.00259690828776238\\
460	0.00257525010215384\\
461	0.00255329429443533\\
462	0.00253101105647798\\
463	0.00250831014118649\\
464	0.00248489435582048\\
465	0.00246008673831945\\
466	0.00243435385994019\\
467	0.00240879072737593\\
468	0.00237947399244218\\
469	0.0023482221547078\\
470	0.00231642921410384\\
471	0.0022841442400962\\
472	0.00225135420759003\\
473	0.00221804954900144\\
474	0.00218423389328816\\
475	0.00214995437136689\\
476	0.00211539324347675\\
477	0.00208111881827806\\
478	0.00204850191368291\\
479	0.00201860096918738\\
480	0.0019879012972538\\
481	0.00195636036166827\\
482	0.00192393154007199\\
483	0.00189056366105782\\
484	0.00185619990513972\\
485	0.00182077534695337\\
486	0.00178420965799812\\
487	0.00174637801333911\\
488	0.00170696088632293\\
489	0.00166344164530458\\
490	0.00161757795728889\\
491	0.00157061342368385\\
492	0.00152251019137918\\
493	0.00147322862401854\\
494	0.00142272720797365\\
495	0.00137096245850899\\
496	0.00131788886726747\\
497	0.00126345902010643\\
498	0.00120762420237788\\
499	0.00115033542055204\\
500	0.00109153966695643\\
501	0.00103116247127905\\
502	0.000969113373693511\\
503	0.000905317299843979\\
504	0.000839604316548774\\
505	0.000771463659200474\\
506	0.000701477745376201\\
507	0.000629526791820348\\
508	0.000554952970295424\\
509	0.000476998662898304\\
510	0.000397384018534237\\
511	0.000316032786533761\\
512	0.00023278160760913\\
513	0.000147012106123333\\
514	5.53656290951778e-05\\
515	0\\
516	0\\
517	0\\
518	0\\
519	0\\
520	0\\
521	0\\
522	0\\
523	0\\
524	0\\
525	0\\
526	0\\
527	0\\
528	0\\
529	0\\
530	0\\
531	0\\
532	0\\
533	0\\
534	0\\
535	0\\
536	0\\
537	0\\
538	0\\
539	0\\
540	0\\
541	0\\
542	0\\
543	0\\
544	0\\
545	0\\
546	0\\
547	0\\
548	0\\
549	0\\
550	0\\
551	0\\
552	0\\
553	0\\
554	0\\
555	0\\
556	0\\
557	0\\
558	0\\
559	0\\
560	0\\
561	0\\
562	0\\
563	0\\
564	0\\
565	0\\
566	0\\
567	0\\
568	0\\
569	0\\
570	0\\
571	0\\
572	0\\
573	0\\
574	0\\
575	0\\
576	0\\
577	0\\
578	0\\
579	0\\
580	0\\
581	0\\
582	0\\
583	0\\
584	0\\
585	0\\
586	0\\
587	0\\
588	0\\
589	0\\
590	0\\
591	0\\
592	0\\
593	0\\
594	0\\
595	0\\
596	0\\
597	0\\
598	0\\
599	0\\
600	0\\
};
\addplot [color=mycolor11,solid,forget plot]
  table[row sep=crcr]{%
1	0.00516053505583464\\
2	0.00516053396220163\\
3	0.00516053284984937\\
4	0.00516053171845674\\
5	0.0051605305676971\\
6	0.00516052939723823\\
7	0.00516052820674222\\
8	0.00516052699586537\\
9	0.00516052576425811\\
10	0.00516052451156487\\
11	0.005160523237424\\
12	0.00516052194146766\\
13	0.00516052062332171\\
14	0.00516051928260563\\
15	0.00516051791893236\\
16	0.00516051653190824\\
17	0.00516051512113286\\
18	0.00516051368619896\\
19	0.00516051222669234\\
20	0.00516051074219169\\
21	0.00516050923226851\\
22	0.00516050769648695\\
23	0.00516050613440375\\
24	0.00516050454556803\\
25	0.00516050292952122\\
26	0.00516050128579693\\
27	0.00516049961392076\\
28	0.00516049791341025\\
29	0.00516049618377466\\
30	0.00516049442451487\\
31	0.00516049263512326\\
32	0.00516049081508352\\
33	0.0051604889638705\\
34	0.00516048708095013\\
35	0.00516048516577919\\
36	0.00516048321780519\\
37	0.00516048123646621\\
38	0.00516047922119073\\
39	0.0051604771713975\\
40	0.00516047508649533\\
41	0.00516047296588294\\
42	0.0051604708089488\\
43	0.00516046861507095\\
44	0.00516046638361682\\
45	0.00516046411394304\\
46	0.0051604618053953\\
47	0.00516045945730809\\
48	0.00516045706900458\\
49	0.00516045463979641\\
50	0.00516045216898348\\
51	0.00516044965585375\\
52	0.00516044709968305\\
53	0.00516044449973488\\
54	0.0051604418552602\\
55	0.0051604391654972\\
56	0.0051604364296711\\
57	0.00516043364699393\\
58	0.00516043081666431\\
59	0.00516042793786721\\
60	0.00516042500977374\\
61	0.00516042203154088\\
62	0.00516041900231132\\
63	0.0051604159212131\\
64	0.00516041278735947\\
65	0.0051604095998486\\
66	0.00516040635776331\\
67	0.00516040306017083\\
68	0.00516039970612255\\
69	0.00516039629465373\\
70	0.00516039282478322\\
71	0.00516038929551324\\
72	0.00516038570582903\\
73	0.00516038205469859\\
74	0.00516037834107243\\
75	0.00516037456388322\\
76	0.00516037072204551\\
77	0.00516036681445543\\
78	0.00516036283999039\\
79	0.00516035879750876\\
80	0.00516035468584951\\
81	0.00516035050383195\\
82	0.00516034625025537\\
83	0.00516034192389869\\
84	0.00516033752352013\\
85	0.00516033304785689\\
86	0.00516032849562474\\
87	0.00516032386551772\\
88	0.00516031915620774\\
89	0.00516031436634421\\
90	0.00516030949455368\\
91	0.00516030453943946\\
92	0.0051602994995812\\
93	0.00516029437353453\\
94	0.00516028915983063\\
95	0.00516028385697584\\
96	0.00516027846345123\\
97	0.00516027297771219\\
98	0.00516026739818799\\
99	0.00516026172328136\\
100	0.00516025595136801\\
101	0.00516025008079623\\
102	0.00516024410988637\\
103	0.00516023803693045\\
104	0.00516023186019159\\
105	0.00516022557790363\\
106	0.00516021918827055\\
107	0.00516021268946602\\
108	0.00516020607963291\\
109	0.0051601993568827\\
110	0.00516019251929506\\
111	0.00516018556491722\\
112	0.00516017849176349\\
113	0.00516017129781467\\
114	0.00516016398101754\\
115	0.00516015653928424\\
116	0.00516014897049172\\
117	0.00516014127248113\\
118	0.00516013344305727\\
119	0.00516012547998791\\
120	0.00516011738100323\\
121	0.00516010914379517\\
122	0.00516010076601679\\
123	0.00516009224528163\\
124	0.00516008357916303\\
125	0.00516007476519348\\
126	0.00516006580086392\\
127	0.00516005668362308\\
128	0.00516004741087672\\
129	0.00516003797998697\\
130	0.00516002838827154\\
131	0.00516001863300304\\
132	0.0051600087114082\\
133	0.0051599986206671\\
134	0.00515998835791238\\
135	0.00515997792022849\\
136	0.00515996730465084\\
137	0.00515995650816502\\
138	0.00515994552770596\\
139	0.00515993436015705\\
140	0.00515992300234932\\
141	0.00515991145106056\\
142	0.00515989970301442\\
143	0.00515988775487952\\
144	0.00515987560326854\\
145	0.00515986324473728\\
146	0.00515985067578372\\
147	0.00515983789284705\\
148	0.00515982489230669\\
149	0.00515981167048129\\
150	0.00515979822362773\\
151	0.00515978454794009\\
152	0.00515977063954862\\
153	0.00515975649451863\\
154	0.00515974210884946\\
155	0.00515972747847336\\
156	0.00515971259925437\\
157	0.00515969746698721\\
158	0.00515968207739608\\
159	0.00515966642613356\\
160	0.00515965050877932\\
161	0.00515963432083899\\
162	0.0051596178577429\\
163	0.00515960111484483\\
164	0.00515958408742071\\
165	0.00515956677066737\\
166	0.00515954915970118\\
167	0.00515953124955675\\
168	0.00515951303518555\\
169	0.0051594945114545\\
170	0.00515947567314463\\
171	0.00515945651494957\\
172	0.00515943703147415\\
173	0.00515941721723292\\
174	0.00515939706664861\\
175	0.00515937657405062\\
176	0.00515935573367348\\
177	0.00515933453965525\\
178	0.00515931298603592\\
179	0.00515929106675577\\
180	0.00515926877565369\\
181	0.00515924610646555\\
182	0.00515922305282241\\
183	0.00515919960824879\\
184	0.00515917576616094\\
185	0.00515915151986495\\
186	0.00515912686255499\\
187	0.00515910178731138\\
188	0.00515907628709871\\
189	0.00515905035476392\\
190	0.0051590239830343\\
191	0.0051589971645155\\
192	0.00515896989168953\\
193	0.0051589421569126\\
194	0.00515891395241311\\
195	0.00515888527028945\\
196	0.00515885610250783\\
197	0.00515882644090004\\
198	0.00515879627716127\\
199	0.00515876560284769\\
200	0.00515873440937424\\
201	0.00515870268801215\\
202	0.00515867042988658\\
203	0.00515863762597415\\
204	0.00515860426710041\\
205	0.00515857034393732\\
206	0.00515853584700062\\
207	0.00515850076664722\\
208	0.00515846509307253\\
209	0.00515842881630766\\
210	0.00515839192621668\\
211	0.0051583544124938\\
212	0.00515831626466043\\
213	0.00515827747206232\\
214	0.00515823802386648\\
215	0.0051581979090582\\
216	0.00515815711643796\\
217	0.00515811563461819\\
218	0.00515807345202016\\
219	0.00515803055687064\\
220	0.00515798693719857\\
221	0.0051579425808317\\
222	0.00515789747539312\\
223	0.00515785160829771\\
224	0.00515780496674857\\
225	0.00515775753773339\\
226	0.00515770930802069\\
227	0.00515766026415605\\
228	0.00515761039245821\\
229	0.00515755967901517\\
230	0.00515750810968016\\
231	0.00515745567006754\\
232	0.00515740234554864\\
233	0.0051573481212475\\
234	0.00515729298203654\\
235	0.00515723691253213\\
236	0.0051571798970901\\
237	0.00515712191980113\\
238	0.00515706296448607\\
239	0.00515700301469116\\
240	0.00515694205368316\\
241	0.00515688006444434\\
242	0.00515681702966746\\
243	0.00515675293175053\\
244	0.00515668775279156\\
245	0.00515662147458317\\
246	0.00515655407860704\\
247	0.00515648554602832\\
248	0.00515641585768987\\
249	0.00515634499410639\\
250	0.00515627293545845\\
251	0.00515619966158632\\
252	0.00515612515198376\\
253	0.00515604938579157\\
254	0.00515597234179111\\
255	0.00515589399839756\\
256	0.0051558143336531\\
257	0.00515573332521991\\
258	0.00515565095037301\\
259	0.00515556718599293\\
260	0.00515548200855819\\
261	0.00515539539413762\\
262	0.00515530731838246\\
263	0.00515521775651835\\
264	0.00515512668333697\\
265	0.00515503407318759\\
266	0.00515493989996835\\
267	0.0051548441371173\\
268	0.00515474675760322\\
269	0.0051546477339162\\
270	0.0051545470380579\\
271	0.00515444464153171\\
272	0.00515434051533257\\
273	0.00515423462993652\\
274	0.00515412695528961\\
275	0.00515401746079621\\
276	0.00515390611530788\\
277	0.00515379288711148\\
278	0.00515367774391692\\
279	0.00515356065284438\\
280	0.00515344158041124\\
281	0.00515332049251849\\
282	0.00515319735443667\\
283	0.00515307213079131\\
284	0.00515294478554787\\
285	0.00515281528199611\\
286	0.00515268358273379\\
287	0.00515254964964986\\
288	0.00515241344390688\\
289	0.00515227492592275\\
290	0.00515213405535171\\
291	0.00515199079106441\\
292	0.00515184509112727\\
293	0.00515169691278076\\
294	0.00515154621241674\\
295	0.00515139294555468\\
296	0.00515123706681681\\
297	0.0051510785299019\\
298	0.00515091728755775\\
299	0.00515075329155227\\
300	0.0051505864926429\\
301	0.0051504168405444\\
302	0.00515024428389477\\
303	0.00515006877021921\\
304	0.00514989024589186\\
305	0.00514970865609519\\
306	0.00514952394477684\\
307	0.00514933605460346\\
308	0.00514914492691124\\
309	0.00514895050165295\\
310	0.00514875271734128\\
311	0.00514855151098949\\
312	0.00514834681804893\\
313	0.00514813857233824\\
314	0.0051479267059676\\
315	0.00514771114925784\\
316	0.00514749183065296\\
317	0.00514726867662538\\
318	0.00514704161157321\\
319	0.00514681055770862\\
320	0.00514657543493639\\
321	0.00514633616072161\\
322	0.00514609264994523\\
323	0.00514584481474628\\
324	0.0051455925643491\\
325	0.00514533580487412\\
326	0.0051450744391301\\
327	0.00514480836638589\\
328	0.00514453748211933\\
329	0.0051442616777406\\
330	0.00514398084028718\\
331	0.005143694852087\\
332	0.00514340359038611\\
333	0.00514310692693669\\
334	0.00514280472754071\\
335	0.00514249685154394\\
336	0.00514218315127439\\
337	0.00514186347141843\\
338	0.00514153764832721\\
339	0.00514120550924475\\
340	0.00514086687144821\\
341	0.00514052154128956\\
342	0.00514016931312651\\
343	0.00513980996812899\\
344	0.00513944327294557\\
345	0.00513906897821246\\
346	0.00513868681688523\\
347	0.00513829650237064\\
348	0.00513789772643345\\
349	0.00513749015684897\\
350	0.0051370734347688\\
351	0.00513664717176195\\
352	0.00513621094648797\\
353	0.00513576430094946\\
354	0.00513530673625371\\
355	0.00513483770777226\\
356	0.00513435661949786\\
357	0.00513386281723007\\
358	0.00513335557972848\\
359	0.00513283410557918\\
360	0.00513229749024203\\
361	0.00513174467925832\\
362	0.00513117436263682\\
363	0.00513058472537829\\
364	0.00512997285665973\\
365	0.00512933339553511\\
366	0.0051286556461704\\
367	0.00512791809116632\\
368	0.00512707556718617\\
369	0.00512621365677636\\
370	0.00512533867006539\\
371	0.00512445043152696\\
372	0.00512354876402859\\
373	0.00512263348865814\\
374	0.00512170442482359\\
375	0.00512076139150791\\
376	0.00511980421108008\\
377	0.00511883271336735\\
378	0.00511784671898175\\
379	0.00511684604105501\\
380	0.00511583049149508\\
381	0.00511479988093934\\
382	0.00511375401868709\\
383	0.00511269271260762\\
384	0.00511161576901979\\
385	0.00511052299253852\\
386	0.00510941418588275\\
387	0.00510828914963941\\
388	0.00510714768197668\\
389	0.00510598957829947\\
390	0.00510481463083948\\
391	0.00510362262817123\\
392	0.00510241335464608\\
393	0.00510118658973624\\
394	0.00509994210728471\\
395	0.0050986796746647\\
396	0.00509739905187089\\
397	0.00509609999060445\\
398	0.0050947822334919\\
399	0.0050934455137228\\
400	0.00509208955564237\\
401	0.00509071407725901\\
402	0.00508931879649426\\
403	0.00508790344576597\\
404	0.00508646781104311\\
405	0.00508501185636\\
406	0.00508353610539178\\
407	0.00508204219128767\\
408	0.00508053380441878\\
409	0.00507901748525117\\
410	0.00507749997480398\\
411	0.00507597297051831\\
412	0.00507441994867542\\
413	0.00507284042967778\\
414	0.00507123388203909\\
415	0.00506959973481857\\
416	0.00506793739819613\\
417	0.00506624626253051\\
418	0.0050645256973056\\
419	0.00506277504987666\\
420	0.00506099364396145\\
421	0.00505918077819892\\
422	0.00505733572583228\\
423	0.00505545773387667\\
424	0.00505354602078519\\
425	0.00505159977490034\\
426	0.00504961815280787\\
427	0.00504760027758672\\
428	0.00504554523694768\\
429	0.00504345208125428\\
430	0.00504131982141767\\
431	0.00503914742665731\\
432	0.00503693382211835\\
433	0.00503467788633475\\
434	0.00503237844852357\\
435	0.00503003428568796\\
436	0.00502764411948796\\
437	0.00502520661279766\\
438	0.00502272036578203\\
439	0.00502018391115925\\
440	0.00501759570803928\\
441	0.00501495413345738\\
442	0.00501225747122418\\
443	0.00500950390204384\\
444	0.00500669151594826\\
445	0.00500381841065281\\
446	0.00500088292228883\\
447	0.0049978829043709\\
448	0.00499481608057586\\
449	0.00499168004997979\\
450	0.00498847227099568\\
451	0.00498519002024984\\
452	0.00498183029229815\\
453	0.00497838954526938\\
454	0.00497486305463673\\
455	0.00497124330908564\\
456	0.00496751623004757\\
457	0.00496365317341908\\
458	0.00495959803985892\\
459	0.00495545847701009\\
460	0.00495125238269577\\
461	0.0049469672450374\\
462	0.00494257469975514\\
463	0.00493801023687081\\
464	0.00493315076640688\\
465	0.00492798717113069\\
466	0.00492242062516231\\
467	0.00491552816158838\\
468	0.00490322237550712\\
469	0.00488875730748457\\
470	0.0048740630976783\\
471	0.0048591286169599\\
472	0.00484394361654972\\
473	0.00482850186063932\\
474	0.00481280854466351\\
475	0.00479689621963633\\
476	0.00478085290830506\\
477	0.00476484288526039\\
478	0.00474893442827621\\
479	0.0047325699405543\\
480	0.00471571948336329\\
481	0.00469834975067911\\
482	0.00468042346135682\\
483	0.00466189768238898\\
484	0.00464271982189616\\
485	0.00462281735643028\\
486	0.00460207060107256\\
487	0.00458023940772211\\
488	0.00455676418504533\\
489	0.00452948477304095\\
490	0.00450046664968492\\
491	0.00447100333840286\\
492	0.00444108528019643\\
493	0.0044107025642607\\
494	0.00437984492848087\\
495	0.00434850177656219\\
496	0.00431666220770141\\
497	0.00428431500858952\\
498	0.00425144835853428\\
499	0.00421804861420509\\
500	0.00418409693491697\\
501	0.00414957075012113\\
502	0.00411444927293261\\
503	0.00407869978160293\\
504	0.00404226856299079\\
505	0.00400519790882657\\
506	0.00396741722630471\\
507	0.0039287588917257\\
508	0.00388876341634874\\
509	0.00384719310549665\\
510	0.00380535624004568\\
511	0.00376318093942821\\
512	0.0037204691751195\\
513	0.00367670479506395\\
514	0.00363073524723451\\
515	0.00358294946602486\\
516	0.00353505653663763\\
517	0.00348709399632349\\
518	0.00343914842590566\\
519	0.00339142939890124\\
520	0.00334444598783759\\
521	0.00329937957587658\\
522	0.00325854851736096\\
523	0.00322419649826356\\
524	0.00318922244626424\\
525	0.00315326103247786\\
526	0.00311514836296493\\
527	0.00307123341315276\\
528	0.0030098509516261\\
529	0.00294020748052116\\
530	0.00286758526591559\\
531	0.00279104864096957\\
532	0.00270989433435107\\
533	0.00262004142303426\\
534	0.00252218784358832\\
535	0.00242178491691014\\
536	0.00231871896930725\\
537	0.00221285935169346\\
538	0.00210403327127977\\
539	0.00199194507799155\\
540	0.00187591619531658\\
541	0.00175411070440876\\
542	0.00162813783804569\\
543	0.00149958178120082\\
544	0.00136835829867665\\
545	0.00123439189756769\\
546	0.00109760179491459\\
547	0.000957899619460617\\
548	0.000815177798917114\\
549	0.000669268858551629\\
550	0.000519806197145862\\
551	0.000365747564450615\\
552	0.000203770911622493\\
553	2.3893647376284e-05\\
554	0\\
555	0\\
556	0\\
557	0\\
558	0\\
559	0\\
560	0\\
561	0\\
562	0\\
563	0\\
564	0\\
565	0\\
566	0\\
567	0\\
568	0\\
569	0\\
570	0\\
571	0\\
572	0\\
573	0\\
574	0\\
575	0\\
576	0\\
577	0\\
578	0\\
579	0\\
580	0\\
581	0\\
582	0\\
583	0\\
584	0\\
585	0\\
586	0\\
587	0\\
588	0\\
589	0\\
590	0\\
591	0\\
592	0\\
593	0\\
594	0\\
595	0\\
596	0\\
597	0\\
598	0\\
599	0\\
600	0\\
};
\addplot [color=mycolor12,solid,forget plot]
  table[row sep=crcr]{%
1	0.00638811260735049\\
2	0.00638811227827753\\
3	0.00638811194357173\\
4	0.00638811160313645\\
5	0.00638811125687342\\
6	0.00638811090468267\\
7	0.00638811054646253\\
8	0.00638811018210957\\
9	0.00638810981151861\\
10	0.00638810943458267\\
11	0.00638810905119292\\
12	0.00638810866123869\\
13	0.00638810826460739\\
14	0.00638810786118453\\
15	0.00638810745085365\\
16	0.0063881070334963\\
17	0.006388106608992\\
18	0.0063881061772182\\
19	0.00638810573805029\\
20	0.00638810529136148\\
21	0.00638810483702286\\
22	0.00638810437490328\\
23	0.00638810390486936\\
24	0.00638810342678545\\
25	0.00638810294051357\\
26	0.00638810244591339\\
27	0.00638810194284216\\
28	0.00638810143115473\\
29	0.00638810091070344\\
30	0.00638810038133811\\
31	0.00638809984290601\\
32	0.00638809929525179\\
33	0.00638809873821745\\
34	0.00638809817164229\\
35	0.00638809759536288\\
36	0.00638809700921297\\
37	0.00638809641302351\\
38	0.00638809580662254\\
39	0.00638809518983516\\
40	0.00638809456248351\\
41	0.00638809392438669\\
42	0.00638809327536069\\
43	0.00638809261521839\\
44	0.00638809194376946\\
45	0.00638809126082035\\
46	0.00638809056617417\\
47	0.00638808985963071\\
48	0.00638808914098634\\
49	0.00638808841003393\\
50	0.00638808766656286\\
51	0.00638808691035891\\
52	0.0063880861412042\\
53	0.00638808535887714\\
54	0.00638808456315237\\
55	0.00638808375380071\\
56	0.00638808293058904\\
57	0.00638808209328031\\
58	0.00638808124163341\\
59	0.00638808037540313\\
60	0.0063880794943401\\
61	0.00638807859819071\\
62	0.00638807768669701\\
63	0.00638807675959669\\
64	0.00638807581662296\\
65	0.00638807485750451\\
66	0.00638807388196541\\
67	0.00638807288972504\\
68	0.00638807188049802\\
69	0.0063880708539941\\
70	0.00638806980991812\\
71	0.0063880687479699\\
72	0.00638806766784416\\
73	0.00638806656923044\\
74	0.00638806545181301\\
75	0.00638806431527078\\
76	0.0063880631592772\\
77	0.00638806198350021\\
78	0.00638806078760207\\
79	0.00638805957123936\\
80	0.00638805833406281\\
81	0.00638805707571723\\
82	0.00638805579584141\\
83	0.00638805449406802\\
84	0.0063880531700235\\
85	0.00638805182332797\\
86	0.00638805045359511\\
87	0.00638804906043206\\
88	0.00638804764343931\\
89	0.00638804620221056\\
90	0.00638804473633268\\
91	0.0063880432453855\\
92	0.00638804172894177\\
93	0.00638804018656701\\
94	0.00638803861781938\\
95	0.00638803702224957\\
96	0.00638803539940067\\
97	0.00638803374880806\\
98	0.00638803206999925\\
99	0.00638803036249378\\
100	0.00638802862580305\\
101	0.00638802685943024\\
102	0.00638802506287009\\
103	0.00638802323560886\\
104	0.00638802137712409\\
105	0.00638801948688452\\
106	0.00638801756434994\\
107	0.00638801560897098\\
108	0.00638801362018904\\
109	0.00638801159743608\\
110	0.00638800954013446\\
111	0.00638800744769684\\
112	0.00638800531952593\\
113	0.00638800315501439\\
114	0.00638800095354464\\
115	0.00638799871448869\\
116	0.00638799643720797\\
117	0.00638799412105314\\
118	0.00638799176536393\\
119	0.00638798936946893\\
120	0.00638798693268543\\
121	0.00638798445431922\\
122	0.00638798193366442\\
123	0.00638797937000323\\
124	0.00638797676260578\\
125	0.00638797411072992\\
126	0.00638797141362101\\
127	0.0063879686705117\\
128	0.00638796588062174\\
129	0.00638796304315773\\
130	0.00638796015731296\\
131	0.00638795722226712\\
132	0.00638795423718612\\
133	0.00638795120122183\\
134	0.00638794811351189\\
135	0.00638794497317941\\
136	0.00638794177933278\\
137	0.0063879385310654\\
138	0.00638793522745544\\
139	0.00638793186756558\\
140	0.00638792845044275\\
141	0.00638792497511786\\
142	0.00638792144060558\\
143	0.00638791784590399\\
144	0.00638791418999439\\
145	0.00638791047184096\\
146	0.0063879066903905\\
147	0.00638790284457213\\
148	0.006387898933297\\
149	0.006387894955458\\
150	0.00638789090992943\\
151	0.00638788679556675\\
152	0.00638788261120617\\
153	0.00638787835566442\\
154	0.0063878740277384\\
155	0.00638786962620481\\
156	0.00638786514981987\\
157	0.00638786059731894\\
158	0.0063878559674162\\
159	0.00638785125880428\\
160	0.0063878464701539\\
161	0.00638784160011354\\
162	0.00638783664730901\\
163	0.00638783161034311\\
164	0.00638782648779528\\
165	0.00638782127822112\\
166	0.00638781598015209\\
167	0.00638781059209505\\
168	0.00638780511253186\\
169	0.00638779953991899\\
170	0.00638779387268706\\
171	0.00638778810924043\\
172	0.00638778224795679\\
173	0.00638777628718663\\
174	0.00638777022525291\\
175	0.00638776406045047\\
176	0.00638775779104569\\
177	0.00638775141527589\\
178	0.00638774493134897\\
179	0.00638773833744281\\
180	0.00638773163170484\\
181	0.00638772481225151\\
182	0.00638771787716775\\
183	0.00638771082450651\\
184	0.00638770365228814\\
185	0.00638769635849991\\
186	0.00638768894109543\\
187	0.00638768139799409\\
188	0.00638767372708052\\
189	0.00638766592620393\\
190	0.00638765799317761\\
191	0.00638764992577827\\
192	0.00638764172174544\\
193	0.00638763337878085\\
194	0.00638762489454782\\
195	0.00638761626667057\\
196	0.00638760749273358\\
197	0.00638759857028096\\
198	0.00638758949681571\\
199	0.00638758026979909\\
200	0.00638757088664987\\
201	0.00638756134474363\\
202	0.00638755164141204\\
203	0.00638754177394213\\
204	0.0063875317395755\\
205	0.00638752153550759\\
206	0.00638751115888687\\
207	0.00638750060681408\\
208	0.0063874898763414\\
209	0.00638747896447161\\
210	0.0063874678681573\\
211	0.00638745658429997\\
212	0.00638744510974919\\
213	0.00638743344130171\\
214	0.00638742157570056\\
215	0.00638740950963414\\
216	0.00638739723973526\\
217	0.00638738476258024\\
218	0.0063873720746879\\
219	0.00638735917251859\\
220	0.00638734605247319\\
221	0.00638733271089208\\
222	0.00638731914405413\\
223	0.00638730534817558\\
224	0.00638729131940902\\
225	0.00638727705384226\\
226	0.00638726254749722\\
227	0.00638724779632878\\
228	0.00638723279622361\\
229	0.00638721754299901\\
230	0.00638720203240168\\
231	0.00638718626010648\\
232	0.0063871702217152\\
233	0.00638715391275527\\
234	0.00638713732867844\\
235	0.00638712046485946\\
236	0.00638710331659474\\
237	0.00638708587910093\\
238	0.00638706814751352\\
239	0.00638705011688543\\
240	0.00638703178218551\\
241	0.00638701313829702\\
242	0.00638699418001616\\
243	0.00638697490205046\\
244	0.00638695529901721\\
245	0.0063869353654418\\
246	0.00638691509575612\\
247	0.00638689448429679\\
248	0.00638687352530348\\
249	0.00638685221291712\\
250	0.00638683054117809\\
251	0.00638680850402438\\
252	0.00638678609528968\\
253	0.00638676330870148\\
254	0.00638674013787907\\
255	0.00638671657633152\\
256	0.00638669261745562\\
257	0.00638666825453377\\
258	0.00638664348073181\\
259	0.00638661828909679\\
260	0.0063865926725547\\
261	0.00638656662390817\\
262	0.00638654013583404\\
263	0.00638651320088097\\
264	0.00638648581146687\\
265	0.00638645795987635\\
266	0.0063864296382581\\
267	0.00638640083862214\\
268	0.00638637155283704\\
269	0.00638634177262707\\
270	0.00638631148956926\\
271	0.00638628069509043\\
272	0.00638624938046401\\
273	0.00638621753680686\\
274	0.00638618515507574\\
275	0.00638615222606409\\
276	0.00638611874039846\\
277	0.00638608468853491\\
278	0.00638605006075521\\
279	0.00638601484716301\\
280	0.00638597903767974\\
281	0.00638594262204055\\
282	0.00638590558978992\\
283	0.00638586793027723\\
284	0.00638582963265214\\
285	0.00638579068585975\\
286	0.00638575107863561\\
287	0.00638571079950053\\
288	0.00638566983675517\\
289	0.00638562817847436\\
290	0.00638558581250126\\
291	0.00638554272644116\\
292	0.00638549890765511\\
293	0.00638545434325314\\
294	0.00638540902008723\\
295	0.00638536292474393\\
296	0.00638531604353653\\
297	0.00638526836249695\\
298	0.00638521986736709\\
299	0.00638517054358977\\
300	0.00638512037629914\\
301	0.00638506935031051\\
302	0.00638501745010968\\
303	0.00638496465984148\\
304	0.00638491096329772\\
305	0.00638485634390431\\
306	0.00638480078470742\\
307	0.00638474426835884\\
308	0.00638468677710006\\
309	0.00638462829274554\\
310	0.00638456879666494\\
311	0.00638450826976433\\
312	0.00638444669246502\\
313	0.00638438404468094\\
314	0.00638432030579448\\
315	0.00638425545463028\\
316	0.00638418946942683\\
317	0.00638412232780574\\
318	0.00638405400673823\\
319	0.00638398448250878\\
320	0.00638391373067538\\
321	0.00638384172602621\\
322	0.00638376844253228\\
323	0.00638369385329557\\
324	0.00638361793049211\\
325	0.00638354064530964\\
326	0.00638346196787895\\
327	0.00638338186719835\\
328	0.00638330031105052\\
329	0.00638321726591067\\
330	0.00638313269684517\\
331	0.00638304656739947\\
332	0.00638295883947404\\
333	0.00638286947318688\\
334	0.0063827784267211\\
335	0.0063826856561557\\
336	0.00638259111527748\\
337	0.00638249475537197\\
338	0.00638239652499067\\
339	0.00638229636969169\\
340	0.00638219423175074\\
341	0.00638209004983853\\
342	0.00638198375866058\\
343	0.00638187528855478\\
344	0.00638176456504129\\
345	0.00638165150831882\\
346	0.00638153603270058\\
347	0.00638141804598205\\
348	0.00638129744873196\\
349	0.00638117413349642\\
350	0.00638104798390462\\
351	0.00638091887366286\\
352	0.00638078666541956\\
353	0.00638065120947726\\
354	0.00638051234231107\\
355	0.00638036988481699\\
356	0.00638022364013296\\
357	0.0063800733906508\\
358	0.00637991889324608\\
359	0.00637975987032702\\
360	0.0063795959907522\\
361	0.00637942682626216\\
362	0.00637925175035374\\
363	0.00637906970942778\\
364	0.00637887874041027\\
365	0.00637867510162692\\
366	0.00637845230977798\\
367	0.00637820317999826\\
368	0.0063779486827292\\
369	0.00637769033540312\\
370	0.00637742808668752\\
371	0.00637716188478615\\
372	0.00637689167741668\\
373	0.00637661741192001\\
374	0.00637633903573714\\
375	0.00637605649738284\\
376	0.00637576974673577\\
377	0.00637547873112774\\
378	0.00637518339603238\\
379	0.00637488368657652\\
380	0.00637457954752541\\
381	0.00637427092326166\\
382	0.00637395775775675\\
383	0.00637363999453385\\
384	0.00637331757662061\\
385	0.00637299044649023\\
386	0.0063726585459893\\
387	0.00637232181625021\\
388	0.00637198019758615\\
389	0.00637163362936644\\
390	0.00637128204986959\\
391	0.00637092539611196\\
392	0.00637056360364985\\
393	0.00637019660635467\\
394	0.00636982433616361\\
395	0.00636944672281565\\
396	0.00636906369359698\\
397	0.00636867517314809\\
398	0.00636828108343642\\
399	0.00636788134409195\\
400	0.00636747587349228\\
401	0.00636706459147168\\
402	0.00636664742611355\\
403	0.00636622433229439\\
404	0.00636579534338305\\
405	0.00636536069391904\\
406	0.00636492099227602\\
407	0.00636447744632584\\
408	0.00636403184281951\\
409	0.00636358533742221\\
410	0.0063631352141959\\
411	0.00636267744808165\\
412	0.0063622118967298\\
413	0.00636173840340145\\
414	0.00636125680084869\\
415	0.00636076691629268\\
416	0.00636026857114308\\
417	0.00635976158068075\\
418	0.00635924575368311\\
419	0.00635872089199831\\
420	0.00635818679017809\\
421	0.00635764323535161\\
422	0.00635709000691684\\
423	0.00635652687590073\\
424	0.00635595360450518\\
425	0.00635536994562446\\
426	0.00635477564233225\\
427	0.00635417042733614\\
428	0.0063535540223977\\
429	0.00635292613771579\\
430	0.00635228647127053\\
431	0.00635163470812541\\
432	0.0063509705196839\\
433	0.0063502935628958\\
434	0.00634960347940539\\
435	0.00634889989462626\\
436	0.00634818241671315\\
437	0.00634745063537106\\
438	0.00634670412039079\\
439	0.00634594241973513\\
440	0.00634516505701116\\
441	0.00634437152861232\\
442	0.00634356130271134\\
443	0.00634273382722985\\
444	0.00634188855859463\\
445	0.0063410249902627\\
446	0.00634014249251388\\
447	0.00633924039810001\\
448	0.00633831800238986\\
449	0.00633737455658749\\
450	0.0063364092506127\\
451	0.00633542117093243\\
452	0.00633440919541388\\
453	0.00633337173729664\\
454	0.00633230615981661\\
455	0.00633120761774576\\
456	0.00633006763021697\\
457	0.00632887711133775\\
458	0.00632766282066212\\
459	0.00632642830815137\\
460	0.00632516878913098\\
461	0.00632387383674367\\
462	0.00632252232735194\\
463	0.00632108164693185\\
464	0.00631953543083207\\
465	0.00631779877024216\\
466	0.00631550642290275\\
467	0.00631181128253835\\
468	0.00630758251061971\\
469	0.0063032886687178\\
470	0.00629892670499869\\
471	0.00629449405557355\\
472	0.00628998981684965\\
473	0.00628541731536539\\
474	0.00628078894352176\\
475	0.00627613232540686\\
476	0.00627148627539327\\
477	0.00626684319508133\\
478	0.00626204858967383\\
479	0.00625711260959464\\
480	0.00625202665348447\\
481	0.00624677980986374\\
482	0.00624135926103719\\
483	0.00623574865960632\\
484	0.00622992393734656\\
485	0.00622384233034361\\
486	0.00621741401359264\\
487	0.00621043315997757\\
488	0.00620244420629681\\
489	0.00619400965776899\\
490	0.0061854524589268\\
491	0.00617676989141488\\
492	0.00616795912551411\\
493	0.0061590172263189\\
494	0.00614994117299156\\
495	0.00614072789386952\\
496	0.00613137429602339\\
497	0.00612187716581177\\
498	0.00611223259079458\\
499	0.00610243450128191\\
500	0.00609247530938651\\
501	0.00608234825131482\\
502	0.0060720425385402\\
503	0.00606153892911889\\
504	0.00605084367335183\\
505	0.00603994004356179\\
506	0.00602877154022212\\
507	0.00601721376170444\\
508	0.00600523421996853\\
509	0.00599317503690893\\
510	0.00598101128349881\\
511	0.00596867831489254\\
512	0.00595601896616613\\
513	0.00594273840786949\\
514	0.00592896458149047\\
515	0.00591514726359842\\
516	0.00590129512228386\\
517	0.00588743147298232\\
518	0.00587361595475236\\
519	0.00585999549864396\\
520	0.00584690599328163\\
521	0.00583500592475455\\
522	0.00582490792822357\\
523	0.0058146042098332\\
524	0.00580398987612161\\
525	0.00579273376978727\\
526	0.00577980378322902\\
527	0.00576202624003748\\
528	0.00572377841157942\\
529	0.00567699475830189\\
530	0.00562869022943325\\
531	0.0055786912893133\\
532	0.00552598034901514\\
533	0.00546614625759935\\
534	0.00539998269028576\\
535	0.00533380742098718\\
536	0.00526762319228078\\
537	0.00520142728789096\\
538	0.00513519630576685\\
539	0.00506883791671606\\
540	0.00500203947876236\\
541	0.00493320527137919\\
542	0.00486390184218159\\
543	0.00479565705149332\\
544	0.00472849590663815\\
545	0.00466244415217564\\
546	0.00459752828388643\\
547	0.0045337742286058\\
548	0.00447120240486943\\
549	0.00440981133989045\\
550	0.00434952421512032\\
551	0.00429001456373431\\
552	0.00423008609656117\\
553	0.0041662083985624\\
554	0.00410400778309653\\
555	0.00404418208447621\\
556	0.00398683441474163\\
557	0.00393197442332481\\
558	0.00387962678014643\\
559	0.00382986338433296\\
560	0.00378275613597533\\
561	0.00373847095787726\\
562	0.00369740504797728\\
563	0.00366052991955191\\
564	0.00363023942217948\\
565	0.00360500619487998\\
566	0.00352973838064425\\
567	0.00344272230220717\\
568	0.00334614191579287\\
569	0.00324691932347805\\
570	0.00314497980923681\\
571	0.0030401664970947\\
572	0.00293180965279257\\
573	0.00281238234143975\\
574	0.00264080565626875\\
575	0.00246652723666886\\
576	0.00228946683670179\\
577	0.00210943841006402\\
578	0.00192677268954282\\
579	0.00174132574584865\\
580	0.00155260207180459\\
581	0.00136033065535838\\
582	0.00116605697170137\\
583	0.000969668956010576\\
584	0.000770929493206139\\
585	0.000569036622337176\\
586	0.000361344392078551\\
587	0.000140168163487911\\
588	0\\
589	0\\
590	0\\
591	0\\
592	0\\
593	0\\
594	0\\
595	0\\
596	0\\
597	0\\
598	0\\
599	0\\
600	0\\
};
\addplot [color=mycolor13,solid,forget plot]
  table[row sep=crcr]{%
1	0.00142341042040566\\
2	0.00142341042040566\\
3	0.00142341042040566\\
4	0.00142341042040566\\
5	0.00142341042040566\\
6	0.00142341042040566\\
7	0.00142341042040566\\
8	0.00142341042040566\\
9	0.00142341042040566\\
10	0.00142341042040566\\
11	0.00142341042040566\\
12	0.00142341042040566\\
13	0.00142341042040566\\
14	0.00142341042040566\\
15	0.00142341042040566\\
16	0.00142341042040566\\
17	0.00142341042040566\\
18	0.00142341042040566\\
19	0.00142341042040566\\
20	0.00142341042040566\\
21	0.00142341042040566\\
22	0.00142341042040566\\
23	0.00142341042040566\\
24	0.00142341042040566\\
25	0.00142341042040566\\
26	0.00142341042040566\\
27	0.00142341042040566\\
28	0.00142341042040566\\
29	0.00142341042040566\\
30	0.00142341042040566\\
31	0.00142341042040566\\
32	0.00142341042040566\\
33	0.00142341042040566\\
34	0.00142341042040566\\
35	0.00142341042040566\\
36	0.00142341042040566\\
37	0.00142341042040566\\
38	0.00142341042040566\\
39	0.00142341042040566\\
40	0.00142341042040566\\
41	0.00142341042040566\\
42	0.00142341042040566\\
43	0.00142341042040566\\
44	0.00142341042040566\\
45	0.00142341042040566\\
46	0.00142341042040566\\
47	0.00142341042040566\\
48	0.00142341042040566\\
49	0.00142341042040566\\
50	0.00142341042040566\\
51	0.00142341042040566\\
52	0.00142341042040566\\
53	0.00142341042040566\\
54	0.00142341042040566\\
55	0.00142341042040566\\
56	0.00142341042040566\\
57	0.00142341042040566\\
58	0.00142341042040566\\
59	0.00142341042040566\\
60	0.00142341042040566\\
61	0.00142341042040566\\
62	0.00142341042040566\\
63	0.00142341042040566\\
64	0.00142341042040566\\
65	0.00142341042040566\\
66	0.00142341042040566\\
67	0.00142341042040566\\
68	0.00142341042040566\\
69	0.00142341042040566\\
70	0.00142341042040566\\
71	0.00142341042040566\\
72	0.00142341042040566\\
73	0.00142341042040566\\
74	0.00142341042040566\\
75	0.00142341042040566\\
76	0.00142341042040566\\
77	0.00142341042040566\\
78	0.00142341042040566\\
79	0.00142341042040566\\
80	0.00142341042040566\\
81	0.00142341042040566\\
82	0.00142341042040566\\
83	0.00142341042040566\\
84	0.00142341042040566\\
85	0.00142341042040566\\
86	0.00142341042040566\\
87	0.00142341042040566\\
88	0.00142341042040566\\
89	0.00142341042040566\\
90	0.00142341042040566\\
91	0.00142341042040566\\
92	0.00142341042040566\\
93	0.00142341042040566\\
94	0.00142341042040566\\
95	0.00142341042040566\\
96	0.00142341042040566\\
97	0.00142341042040566\\
98	0.00142341042040566\\
99	0.00142341042040566\\
100	0.00142341042040566\\
101	0.00142341042040566\\
102	0.00142341042040566\\
103	0.00142341042040566\\
104	0.00142341042040566\\
105	0.00142341042040566\\
106	0.00142341042040566\\
107	0.00142341042040566\\
108	0.00142341042040566\\
109	0.00142341042040566\\
110	0.00142341042040566\\
111	0.00142341042040566\\
112	0.00142341042040566\\
113	0.00142341042040566\\
114	0.00142341042040566\\
115	0.00142341042040566\\
116	0.00142341042040566\\
117	0.00142341042040566\\
118	0.00142341042040566\\
119	0.00142341042040566\\
120	0.00142341042040566\\
121	0.00142341042040566\\
122	0.00142341042040566\\
123	0.00142341042040566\\
124	0.00142341042040566\\
125	0.00142341042040566\\
126	0.00142341042040566\\
127	0.00142341042040566\\
128	0.00142341042040566\\
129	0.00142341042040566\\
130	0.00142341042040566\\
131	0.00142341042040566\\
132	0.00142341042040566\\
133	0.00142341042040566\\
134	0.00142341042040566\\
135	0.00142341042040566\\
136	0.00142341042040566\\
137	0.00142341042040566\\
138	0.00142341042040566\\
139	0.00142341042040566\\
140	0.00142341042040566\\
141	0.00142341042040566\\
142	0.00142341042040566\\
143	0.00142341042040566\\
144	0.00142341042040566\\
145	0.00142341042040566\\
146	0.00142341042040566\\
147	0.00142341042040566\\
148	0.00142341042040566\\
149	0.00142341042040566\\
150	0.00142341042040566\\
151	0.00142341042040566\\
152	0.00142341042040566\\
153	0.00142341042040566\\
154	0.00142341042040566\\
155	0.00142341042040566\\
156	0.00142341042040566\\
157	0.00142341042040566\\
158	0.00142341042040566\\
159	0.00142341042040566\\
160	0.00142341042040566\\
161	0.00142341042040566\\
162	0.00142341042040566\\
163	0.00142341042040566\\
164	0.00142341042040566\\
165	0.00142341042040566\\
166	0.00142341042040566\\
167	0.00142341042040566\\
168	0.00142341042040566\\
169	0.00142341042040566\\
170	0.00142341042040566\\
171	0.00142341042040566\\
172	0.00142341042040566\\
173	0.00142341042040566\\
174	0.00142341042040566\\
175	0.00142341042040566\\
176	0.00142341042040566\\
177	0.00142341042040566\\
178	0.00142341042040566\\
179	0.00142341042040566\\
180	0.00142341042040566\\
181	0.00142341042040566\\
182	0.00142341042040566\\
183	0.00142341042040566\\
184	0.00142341042040566\\
185	0.00142341042040566\\
186	0.00142341042040566\\
187	0.00142341042040566\\
188	0.00142341042040566\\
189	0.00142341042040566\\
190	0.00142341042040566\\
191	0.00142341042040566\\
192	0.00142341042040566\\
193	0.00142341042040566\\
194	0.00142341042040566\\
195	0.00142341042040566\\
196	0.00142341042040566\\
197	0.00142341042040566\\
198	0.00142341042040566\\
199	0.00142341042040566\\
200	0.00142341042040566\\
201	0.00142341042040566\\
202	0.00142341042040566\\
203	0.00142341042040566\\
204	0.00142341042040566\\
205	0.00142341042040566\\
206	0.00142341042040566\\
207	0.00142341042040566\\
208	0.00142341042040566\\
209	0.00142341042040566\\
210	0.00142341042040566\\
211	0.00142341042040566\\
212	0.00142341042040566\\
213	0.00142341042040566\\
214	0.00142341042040566\\
215	0.00142341042040566\\
216	0.00142341042040566\\
217	0.00142341042040566\\
218	0.00142341042040566\\
219	0.00142341042040566\\
220	0.00142341042040566\\
221	0.00142341042040566\\
222	0.00142341042040566\\
223	0.00142341042040566\\
224	0.00142341042040566\\
225	0.00142341042040566\\
226	0.00142341042040566\\
227	0.00142341042040566\\
228	0.00142341042040566\\
229	0.00142341042040566\\
230	0.00142341042040566\\
231	0.00142341042040566\\
232	0.00142341042040566\\
233	0.00142341042040566\\
234	0.00142341042040566\\
235	0.00142341042040566\\
236	0.00142341042040566\\
237	0.00142341042040566\\
238	0.00142341042040566\\
239	0.00142341042040566\\
240	0.00142341042040566\\
241	0.00142341042040566\\
242	0.00142341042040566\\
243	0.00142341042040566\\
244	0.00142341042040566\\
245	0.00142341042040566\\
246	0.00142341042040566\\
247	0.00142341042040566\\
248	0.00142341042040566\\
249	0.00142341042040566\\
250	0.00142341042040566\\
251	0.00142341042040566\\
252	0.00142341042040566\\
253	0.00142341042040566\\
254	0.00142341042040566\\
255	0.00142341042040566\\
256	0.00142341042040566\\
257	0.00142341042040566\\
258	0.00142341042040566\\
259	0.00142341042040566\\
260	0.00142341042040566\\
261	0.00142341042040566\\
262	0.00142341042040566\\
263	0.00142341042040566\\
264	0.00142341042040566\\
265	0.00142341042040566\\
266	0.00142341042040566\\
267	0.00142341042040566\\
268	0.00142341042040566\\
269	0.00142341042040566\\
270	0.00142341042040566\\
271	0.00142341042040566\\
272	0.00142341042040566\\
273	0.00142341042040566\\
274	0.00142341042040566\\
275	0.00142341042040566\\
276	0.00142341042040566\\
277	0.00142341042040566\\
278	0.00142341042040566\\
279	0.00142341042040566\\
280	0.00142341042040566\\
281	0.00142341042040566\\
282	0.00142341042040566\\
283	0.00142341042040566\\
284	0.00142341042040566\\
285	0.00142341042040566\\
286	0.00142341042040566\\
287	0.00142341042040566\\
288	0.00142341042040566\\
289	0.00142341042040566\\
290	0.00142341042040566\\
291	0.00142341042040566\\
292	0.00142341042040566\\
293	0.00142341042040566\\
294	0.00142341042040566\\
295	0.00142341042040566\\
296	0.00142341042040566\\
297	0.00142341042040566\\
298	0.00142341042040566\\
299	0.00142341042040566\\
300	0.00142341042040566\\
301	0.00142341042040566\\
302	0.00142341042040566\\
303	0.00142341042040566\\
304	0.00142341042040566\\
305	0.00142341042040566\\
306	0.00142341042040566\\
307	0.00142341042040566\\
308	0.00142341042040566\\
309	0.00142341042040566\\
310	0.00142341042040566\\
311	0.00142341042040566\\
312	0.00142341042040566\\
313	0.00142341042040566\\
314	0.00142341042040566\\
315	0.00142341042040566\\
316	0.00142341042040566\\
317	0.00142341042040566\\
318	0.00142341042040566\\
319	0.00142341042040566\\
320	0.00142341042040566\\
321	0.00142341042040566\\
322	0.00142341042040566\\
323	0.00142341042040566\\
324	0.00142341042040566\\
325	0.00142341042040566\\
326	0.00142341042040566\\
327	0.00142341042040566\\
328	0.00142341042040566\\
329	0.00142341042040566\\
330	0.00142341042040566\\
331	0.00142341042040566\\
332	0.00142341042040566\\
333	0.00142341042040566\\
334	0.00142341042040566\\
335	0.00142341042040566\\
336	0.00142341042040566\\
337	0.00142341042040566\\
338	0.00142341042040566\\
339	0.00142341042040566\\
340	0.00142341042040566\\
341	0.00142341042040566\\
342	0.00142341042040566\\
343	0.00142341042040566\\
344	0.00142341042040566\\
345	0.00142341042040566\\
346	0.00142341042040566\\
347	0.00142341042040566\\
348	0.00142341042040566\\
349	0.00142341042040566\\
350	0.00142341042040566\\
351	0.00142341042040566\\
352	0.00142341042040566\\
353	0.00142341042040566\\
354	0.00142341042040566\\
355	0.00142341042040566\\
356	0.00142341042040566\\
357	0.00142341042040566\\
358	0.00142341042040566\\
359	0.00142341042040566\\
360	0.00142341042040566\\
361	0.00142341042040566\\
362	0.00142341042040566\\
363	0.00142341042040566\\
364	0.00142341042040566\\
365	0.00142341042040566\\
366	0.00142341042040566\\
367	0.00142341042040566\\
368	0.00142341042040566\\
369	0.00142341042040566\\
370	0.00142341042040566\\
371	0.00142341042040566\\
372	0.00142341042040566\\
373	0.00142341042040566\\
374	0.00142341042040566\\
375	0.00142341042040566\\
376	0.00142341042040566\\
377	0.00142341042040566\\
378	0.00142341042040566\\
379	0.00142341042040566\\
380	0.00142341042040566\\
381	0.00142341042040566\\
382	0.00142341042040566\\
383	0.00142341042040566\\
384	0.00142341042040566\\
385	0.00142341042040566\\
386	0.00142341042040566\\
387	0.00142341042040566\\
388	0.00142341042040566\\
389	0.00142341042040566\\
390	0.00142341042040566\\
391	0.00142341042040566\\
392	0.00142341042040566\\
393	0.00142341042040566\\
394	0.00142341042040566\\
395	0.00142341042040566\\
396	0.00142341042040566\\
397	0.00142341042040566\\
398	0.00142341042040566\\
399	0.00142341042040566\\
400	0.00142341042040566\\
401	0.00142341042040566\\
402	0.00142341042040566\\
403	0.00142341042040566\\
404	0.00142341042040566\\
405	0.00142341042040566\\
406	0.00142341042040566\\
407	0.00142341042040566\\
408	0.00142341042040566\\
409	0.00142341042040566\\
410	0.00142341042040566\\
411	0.00142341042040566\\
412	0.00142341042040566\\
413	0.00142341042040566\\
414	0.00142341042040566\\
415	0.00142341042040566\\
416	0.00142341042040566\\
417	0.00142341042040566\\
418	0.00142341042040566\\
419	0.00142341042040566\\
420	0.00142341042040566\\
421	0.00142341042040566\\
422	0.00142341042040566\\
423	0.00142341042040566\\
424	0.00142341042040566\\
425	0.00142341042040566\\
426	0.00142341042040566\\
427	0.00142341042040566\\
428	0.00142341042040566\\
429	0.00142341042040566\\
430	0.00142341042040566\\
431	0.00142341042040566\\
432	0.00142341042040566\\
433	0.00142341042040566\\
434	0.00142341042040566\\
435	0.00142341042040566\\
436	0.00142341042040566\\
437	0.00142341042040566\\
438	0.00142341042040566\\
439	0.00142341042040566\\
440	0.00142341042040566\\
441	0.00142341042040566\\
442	0.00142341042040566\\
443	0.00142341042040566\\
444	0.00142341042040566\\
445	0.00142341042040566\\
446	0.00142341042040566\\
447	0.00142341042040566\\
448	0.00142341042040566\\
449	0.00142341042040566\\
450	0.00142341042040566\\
451	0.00142341042040566\\
452	0.00142341042040566\\
453	0.00142341042040566\\
454	0.00142341042040566\\
455	0.00142341042040566\\
456	0.00142341042040566\\
457	0.00142341042040566\\
458	0.00142341042040566\\
459	0.00142341042040566\\
460	0.00142341042040566\\
461	0.00142341042040566\\
462	0.00142341042040566\\
463	0.00142341042040566\\
464	0.00142341042040566\\
465	0.00142341042040566\\
466	0.00142341042040566\\
467	0.00142341042040566\\
468	0.00142341042040566\\
469	0.00142341042040566\\
470	0.00142341042040566\\
471	0.00142341042040566\\
472	0.00142341042040566\\
473	0.00142341042040566\\
474	0.00142341042040566\\
475	0.00142341042040566\\
476	0.00142341042040566\\
477	0.00142341042040566\\
478	0.00142341042040566\\
479	0.00142341042040566\\
480	0.00142341042040566\\
481	0.00142341042040566\\
482	0.00142341042040566\\
483	0.00142341042040566\\
484	0.00142341042040566\\
485	0.00142341042040566\\
486	0.00142341042040566\\
487	0.00142341042040566\\
488	0.00142341042040566\\
489	0.00142341042040566\\
490	0.00142341042040566\\
491	0.00142341042040566\\
492	0.00142341042040566\\
493	0.00142341042040566\\
494	0.00142341042040566\\
495	0.00142341042040566\\
496	0.00142341042040566\\
497	0.00142341042040566\\
498	0.00142341042040566\\
499	0.00142341042040566\\
500	0.00142341042040566\\
501	0.00142341042040566\\
502	0.00142341042040566\\
503	0.00142341042040566\\
504	0.00142341042040566\\
505	0.00142341042040566\\
506	0.00142341042040566\\
507	0.00142341042040566\\
508	0.00142341042040566\\
509	0.00142341042040566\\
510	0.00142341042040566\\
511	0.00142341042040566\\
512	0.00142341042040566\\
513	0.00142341042040566\\
514	0.00142341042040566\\
515	0.00142341042040566\\
516	0.00142341042040566\\
517	0.00142341042040566\\
518	0.00142341042040566\\
519	0.00142341042040566\\
520	0.00142341042040566\\
521	0.00142341042040566\\
522	0.00142341042040566\\
523	0.00142341042040566\\
524	0.00142341042040566\\
525	0.00142341042040566\\
526	0.00142341042040566\\
527	0.00142341042040566\\
528	0.00142341042040566\\
529	0.00142341042040566\\
530	0.00142341042040566\\
531	0.00142341042040566\\
532	0.00142341042040566\\
533	0.00142341042040566\\
534	0.00142341042040566\\
535	0.00142341042040566\\
536	0.00142341042040566\\
537	0.00142341042040566\\
538	0.00142341042040566\\
539	0.00142341042040566\\
540	0.00142341042040566\\
541	0.00142341042040566\\
542	0.00142341042040566\\
543	0.00142341042040566\\
544	0.00142341042040566\\
545	0.00142341042040566\\
546	0.00142341042040566\\
547	0.00142341042040566\\
548	0.00142341042040566\\
549	0.00142341042040566\\
550	0.00142341042040566\\
551	0.00142341042040566\\
552	0.00142341042040566\\
553	0.00142341042040566\\
554	0.00142341042040566\\
555	0.00142341042040566\\
556	0.00142341042040566\\
557	0.00142341042040566\\
558	0.00142341042040566\\
559	0.00142341042040566\\
560	0.00142341042040566\\
561	0.00142341042040566\\
562	0.00142341042040566\\
563	0.00142341042040566\\
564	0.00142341042040566\\
565	0.00141808680651778\\
566	0.00132573838914067\\
567	0.00121575937955735\\
568	0.00110234926539761\\
569	0.000991487125274636\\
570	0.000881523074320698\\
571	0.000772562041893994\\
572	0.000667166343829018\\
573	0.000574706660558508\\
574	0.000427188374771164\\
575	0.000287780647258003\\
576	0.000156402139129492\\
577	3.29294685952495e-05\\
578	0\\
579	0\\
580	0\\
581	0\\
582	0\\
583	0\\
584	0\\
585	0\\
586	0\\
587	0\\
588	0\\
589	0\\
590	0\\
591	0\\
592	5.56832960656924e-05\\
593	0.000164162311403634\\
594	0.00028999157753712\\
595	0.000432641173258269\\
596	0.000588326666482334\\
597	0.000749148955320047\\
598	0.00347642786857269\\
599	0\\
600	0\\
};
\addplot [color=mycolor14,solid,forget plot]
  table[row sep=crcr]{%
1	0\\
2	0\\
3	0\\
4	0\\
5	0\\
6	0\\
7	0\\
8	0\\
9	0\\
10	0\\
11	0\\
12	0\\
13	0\\
14	0\\
15	0\\
16	0\\
17	0\\
18	0\\
19	0\\
20	0\\
21	0\\
22	0\\
23	0\\
24	0\\
25	0\\
26	0\\
27	0\\
28	0\\
29	0\\
30	0\\
31	0\\
32	0\\
33	0\\
34	0\\
35	0\\
36	0\\
37	0\\
38	0\\
39	0\\
40	0\\
41	0\\
42	0\\
43	0\\
44	0\\
45	0\\
46	0\\
47	0\\
48	0\\
49	0\\
50	0\\
51	0\\
52	0\\
53	0\\
54	0\\
55	0\\
56	0\\
57	0\\
58	0\\
59	0\\
60	0\\
61	0\\
62	0\\
63	0\\
64	0\\
65	0\\
66	0\\
67	0\\
68	0\\
69	0\\
70	0\\
71	0\\
72	0\\
73	0\\
74	0\\
75	0\\
76	0\\
77	0\\
78	0\\
79	0\\
80	0\\
81	0\\
82	0\\
83	0\\
84	0\\
85	0\\
86	0\\
87	0\\
88	0\\
89	0\\
90	0\\
91	0\\
92	0\\
93	0\\
94	0\\
95	0\\
96	0\\
97	0\\
98	0\\
99	0\\
100	0\\
101	0\\
102	0\\
103	0\\
104	0\\
105	0\\
106	0\\
107	0\\
108	0\\
109	0\\
110	0\\
111	0\\
112	0\\
113	0\\
114	0\\
115	0\\
116	0\\
117	0\\
118	0\\
119	0\\
120	0\\
121	0\\
122	0\\
123	0\\
124	0\\
125	0\\
126	0\\
127	0\\
128	0\\
129	0\\
130	0\\
131	0\\
132	0\\
133	0\\
134	0\\
135	0\\
136	0\\
137	0\\
138	0\\
139	0\\
140	0\\
141	0\\
142	0\\
143	0\\
144	0\\
145	0\\
146	0\\
147	0\\
148	0\\
149	0\\
150	0\\
151	0\\
152	0\\
153	0\\
154	0\\
155	0\\
156	0\\
157	0\\
158	0\\
159	0\\
160	0\\
161	0\\
162	0\\
163	0\\
164	0\\
165	0\\
166	0\\
167	0\\
168	0\\
169	0\\
170	0\\
171	0\\
172	0\\
173	0\\
174	0\\
175	0\\
176	0\\
177	0\\
178	0\\
179	0\\
180	0\\
181	0\\
182	0\\
183	0\\
184	0\\
185	0\\
186	0\\
187	0\\
188	0\\
189	0\\
190	0\\
191	0\\
192	0\\
193	0\\
194	0\\
195	0\\
196	0\\
197	0\\
198	0\\
199	0\\
200	0\\
201	0\\
202	0\\
203	0\\
204	0\\
205	0\\
206	0\\
207	0\\
208	0\\
209	0\\
210	0\\
211	0\\
212	0\\
213	0\\
214	0\\
215	0\\
216	0\\
217	0\\
218	0\\
219	0\\
220	0\\
221	0\\
222	0\\
223	0\\
224	0\\
225	0\\
226	0\\
227	0\\
228	0\\
229	0\\
230	0\\
231	0\\
232	0\\
233	0\\
234	0\\
235	0\\
236	0\\
237	0\\
238	0\\
239	0\\
240	0\\
241	0\\
242	0\\
243	0\\
244	0\\
245	0\\
246	0\\
247	0\\
248	0\\
249	0\\
250	0\\
251	0\\
252	0\\
253	0\\
254	0\\
255	0\\
256	0\\
257	0\\
258	0\\
259	0\\
260	0\\
261	0\\
262	0\\
263	0\\
264	0\\
265	0\\
266	0\\
267	0\\
268	0\\
269	0\\
270	0\\
271	0\\
272	0\\
273	0\\
274	0\\
275	0\\
276	0\\
277	0\\
278	0\\
279	0\\
280	0\\
281	0\\
282	0\\
283	0\\
284	0\\
285	0\\
286	0\\
287	0\\
288	0\\
289	0\\
290	0\\
291	0\\
292	0\\
293	0\\
294	0\\
295	0\\
296	0\\
297	0\\
298	0\\
299	0\\
300	0\\
301	0\\
302	0\\
303	0\\
304	0\\
305	0\\
306	0\\
307	0\\
308	0\\
309	0\\
310	0\\
311	0\\
312	0\\
313	0\\
314	0\\
315	0\\
316	0\\
317	0\\
318	0\\
319	0\\
320	0\\
321	0\\
322	0\\
323	0\\
324	0\\
325	0\\
326	0\\
327	0\\
328	0\\
329	0\\
330	0\\
331	0\\
332	0\\
333	0\\
334	0\\
335	0\\
336	0\\
337	0\\
338	0\\
339	0\\
340	0\\
341	0\\
342	0\\
343	0\\
344	0\\
345	0\\
346	0\\
347	0\\
348	0\\
349	0\\
350	0\\
351	0\\
352	0\\
353	0\\
354	0\\
355	0\\
356	0\\
357	0\\
358	0\\
359	0\\
360	0\\
361	0\\
362	0\\
363	0\\
364	0\\
365	0\\
366	0\\
367	0\\
368	0\\
369	0\\
370	0\\
371	0\\
372	0\\
373	0\\
374	0\\
375	0\\
376	0\\
377	0\\
378	0\\
379	0\\
380	0\\
381	0\\
382	0\\
383	0\\
384	0\\
385	0\\
386	0\\
387	0\\
388	0\\
389	0\\
390	0\\
391	0\\
392	0\\
393	0\\
394	0\\
395	0\\
396	0\\
397	0\\
398	0\\
399	0\\
400	0\\
401	0\\
402	0\\
403	0\\
404	0\\
405	0\\
406	0\\
407	0\\
408	0\\
409	0\\
410	0\\
411	0\\
412	0\\
413	0\\
414	0\\
415	0\\
416	0\\
417	0\\
418	0\\
419	0\\
420	0\\
421	0\\
422	0\\
423	0\\
424	0\\
425	0\\
426	0\\
427	0\\
428	0\\
429	0\\
430	0\\
431	0\\
432	0\\
433	0\\
434	0\\
435	0\\
436	0\\
437	0\\
438	0\\
439	0\\
440	0\\
441	0\\
442	0\\
443	0\\
444	0\\
445	0\\
446	0\\
447	0\\
448	0\\
449	0\\
450	0\\
451	0\\
452	0\\
453	0\\
454	0\\
455	0\\
456	0\\
457	0\\
458	0\\
459	0\\
460	0\\
461	0\\
462	0\\
463	0\\
464	0\\
465	0\\
466	0\\
467	0\\
468	0\\
469	0\\
470	0\\
471	0\\
472	0\\
473	0\\
474	0\\
475	0\\
476	0\\
477	0\\
478	0\\
479	0\\
480	0\\
481	0\\
482	0\\
483	0\\
484	0\\
485	0\\
486	0\\
487	0\\
488	0\\
489	0\\
490	0\\
491	0\\
492	0\\
493	0\\
494	0\\
495	0\\
496	0\\
497	0\\
498	0\\
499	0\\
500	0\\
501	0\\
502	0\\
503	0\\
504	0\\
505	0\\
506	0\\
507	0\\
508	0\\
509	0\\
510	0\\
511	0\\
512	0\\
513	0\\
514	0\\
515	0\\
516	0\\
517	0\\
518	0\\
519	0\\
520	0\\
521	0\\
522	0\\
523	0\\
524	0\\
525	0\\
526	0\\
527	0\\
528	0\\
529	0\\
530	0\\
531	0\\
532	0\\
533	0\\
534	0\\
535	0\\
536	0\\
537	0\\
538	0\\
539	0\\
540	0\\
541	0\\
542	0\\
543	0\\
544	0\\
545	0\\
546	0\\
547	0\\
548	0\\
549	0\\
550	0\\
551	0\\
552	0\\
553	0\\
554	0\\
555	0\\
556	0\\
557	0\\
558	0\\
559	0\\
560	0\\
561	0\\
562	0\\
563	0\\
564	0\\
565	0\\
566	0\\
567	0\\
568	0\\
569	0\\
570	0\\
571	0\\
572	0\\
573	0\\
574	0\\
575	0\\
576	0\\
577	0\\
578	0\\
579	0\\
580	0\\
581	0\\
582	0\\
583	0\\
584	0\\
585	0.000134515837093855\\
586	0.000281838017575936\\
587	0.000435050286072654\\
588	0.000594542785312444\\
589	0.0007604561023499\\
590	0.000933270892818283\\
591	0.00111362236991578\\
592	0.00130217091110698\\
593	0.00149974464812233\\
594	0.00170715331224199\\
595	0.00192583625726673\\
596	0.00215816651678185\\
597	0.00340423165914315\\
598	0.00644286460810295\\
599	0\\
600	0\\
};
\addplot [color=mycolor15,solid,forget plot]
  table[row sep=crcr]{%
1	0\\
2	0\\
3	0\\
4	0\\
5	0\\
6	0\\
7	0\\
8	0\\
9	0\\
10	0\\
11	0\\
12	0\\
13	0\\
14	0\\
15	0\\
16	0\\
17	0\\
18	0\\
19	0\\
20	0\\
21	0\\
22	0\\
23	0\\
24	0\\
25	0\\
26	0\\
27	0\\
28	0\\
29	0\\
30	0\\
31	0\\
32	0\\
33	0\\
34	0\\
35	0\\
36	0\\
37	0\\
38	0\\
39	0\\
40	0\\
41	0\\
42	0\\
43	0\\
44	0\\
45	0\\
46	0\\
47	0\\
48	0\\
49	0\\
50	0\\
51	0\\
52	0\\
53	0\\
54	0\\
55	0\\
56	0\\
57	0\\
58	0\\
59	0\\
60	0\\
61	0\\
62	0\\
63	0\\
64	0\\
65	0\\
66	0\\
67	0\\
68	0\\
69	0\\
70	0\\
71	0\\
72	0\\
73	0\\
74	0\\
75	0\\
76	0\\
77	0\\
78	0\\
79	0\\
80	0\\
81	0\\
82	0\\
83	0\\
84	0\\
85	0\\
86	0\\
87	0\\
88	0\\
89	0\\
90	0\\
91	0\\
92	0\\
93	0\\
94	0\\
95	0\\
96	0\\
97	0\\
98	0\\
99	0\\
100	0\\
101	0\\
102	0\\
103	0\\
104	0\\
105	0\\
106	0\\
107	0\\
108	0\\
109	0\\
110	0\\
111	0\\
112	0\\
113	0\\
114	0\\
115	0\\
116	0\\
117	0\\
118	0\\
119	0\\
120	0\\
121	0\\
122	0\\
123	0\\
124	0\\
125	0\\
126	0\\
127	0\\
128	0\\
129	0\\
130	0\\
131	0\\
132	0\\
133	0\\
134	0\\
135	0\\
136	0\\
137	0\\
138	0\\
139	0\\
140	0\\
141	0\\
142	0\\
143	0\\
144	0\\
145	0\\
146	0\\
147	0\\
148	0\\
149	0\\
150	0\\
151	0\\
152	0\\
153	0\\
154	0\\
155	0\\
156	0\\
157	0\\
158	0\\
159	0\\
160	0\\
161	0\\
162	0\\
163	0\\
164	0\\
165	0\\
166	0\\
167	0\\
168	0\\
169	0\\
170	0\\
171	0\\
172	0\\
173	0\\
174	0\\
175	0\\
176	0\\
177	0\\
178	0\\
179	0\\
180	0\\
181	0\\
182	0\\
183	0\\
184	0\\
185	0\\
186	0\\
187	0\\
188	0\\
189	0\\
190	0\\
191	0\\
192	0\\
193	0\\
194	0\\
195	0\\
196	0\\
197	0\\
198	0\\
199	0\\
200	0\\
201	0\\
202	0\\
203	0\\
204	0\\
205	0\\
206	0\\
207	0\\
208	0\\
209	0\\
210	0\\
211	0\\
212	0\\
213	0\\
214	0\\
215	0\\
216	0\\
217	0\\
218	0\\
219	0\\
220	0\\
221	0\\
222	0\\
223	0\\
224	0\\
225	0\\
226	0\\
227	0\\
228	0\\
229	0\\
230	0\\
231	0\\
232	0\\
233	0\\
234	0\\
235	0\\
236	0\\
237	0\\
238	0\\
239	0\\
240	0\\
241	0\\
242	0\\
243	0\\
244	0\\
245	0\\
246	0\\
247	0\\
248	0\\
249	0\\
250	0\\
251	0\\
252	0\\
253	0\\
254	0\\
255	0\\
256	0\\
257	0\\
258	0\\
259	0\\
260	0\\
261	0\\
262	0\\
263	0\\
264	0\\
265	0\\
266	0\\
267	0\\
268	0\\
269	0\\
270	0\\
271	0\\
272	0\\
273	0\\
274	0\\
275	0\\
276	0\\
277	0\\
278	0\\
279	0\\
280	0\\
281	0\\
282	0\\
283	0\\
284	0\\
285	0\\
286	0\\
287	0\\
288	0\\
289	0\\
290	0\\
291	0\\
292	0\\
293	0\\
294	0\\
295	0\\
296	0\\
297	0\\
298	0\\
299	0\\
300	0\\
301	0\\
302	0\\
303	0\\
304	0\\
305	0\\
306	0\\
307	0\\
308	0\\
309	0\\
310	0\\
311	0\\
312	0\\
313	0\\
314	0\\
315	0\\
316	0\\
317	0\\
318	0\\
319	0\\
320	0\\
321	0\\
322	0\\
323	0\\
324	0\\
325	0\\
326	0\\
327	0\\
328	0\\
329	0\\
330	0\\
331	0\\
332	0\\
333	0\\
334	0\\
335	0\\
336	0\\
337	0\\
338	0\\
339	0\\
340	0\\
341	0\\
342	0\\
343	0\\
344	0\\
345	0\\
346	0\\
347	0\\
348	0\\
349	0\\
350	0\\
351	0\\
352	0\\
353	0\\
354	0\\
355	0\\
356	0\\
357	0\\
358	0\\
359	0\\
360	0\\
361	0\\
362	0\\
363	0\\
364	0\\
365	0\\
366	0\\
367	0\\
368	0\\
369	0\\
370	0\\
371	0\\
372	0\\
373	0\\
374	0\\
375	0\\
376	0\\
377	0\\
378	0\\
379	0\\
380	0\\
381	0\\
382	0\\
383	0\\
384	0\\
385	0\\
386	0\\
387	0\\
388	0\\
389	0\\
390	0\\
391	0\\
392	0\\
393	0\\
394	0\\
395	0\\
396	0\\
397	0\\
398	0\\
399	0\\
400	0\\
401	0\\
402	0\\
403	0\\
404	0\\
405	0\\
406	0\\
407	0\\
408	0\\
409	0\\
410	0\\
411	0\\
412	0\\
413	0\\
414	0\\
415	0\\
416	0\\
417	0\\
418	0\\
419	0\\
420	0\\
421	0\\
422	0\\
423	0\\
424	0\\
425	0\\
426	0\\
427	0\\
428	0\\
429	0\\
430	0\\
431	0\\
432	0\\
433	0\\
434	0\\
435	0\\
436	0\\
437	0\\
438	0\\
439	0\\
440	0\\
441	0\\
442	0\\
443	0\\
444	0\\
445	0\\
446	0\\
447	0\\
448	0\\
449	0\\
450	0\\
451	0\\
452	0\\
453	0\\
454	0\\
455	0\\
456	0\\
457	0\\
458	0\\
459	0\\
460	0\\
461	0\\
462	0\\
463	0\\
464	0\\
465	0\\
466	0\\
467	0\\
468	0\\
469	0\\
470	0\\
471	0\\
472	0\\
473	0\\
474	0\\
475	0\\
476	0\\
477	0\\
478	0\\
479	0\\
480	0\\
481	0\\
482	0\\
483	0\\
484	0\\
485	0\\
486	0\\
487	0\\
488	0\\
489	0\\
490	0\\
491	0\\
492	0\\
493	0\\
494	0\\
495	0\\
496	0\\
497	0\\
498	0\\
499	0\\
500	0\\
501	0\\
502	0\\
503	0\\
504	0\\
505	0\\
506	0\\
507	0\\
508	0\\
509	0\\
510	0\\
511	0\\
512	0\\
513	0\\
514	0\\
515	0\\
516	0\\
517	0\\
518	0\\
519	0\\
520	0\\
521	0\\
522	0\\
523	0\\
524	0\\
525	0\\
526	0\\
527	0\\
528	0\\
529	0\\
530	0\\
531	0\\
532	0\\
533	0\\
534	0\\
535	0\\
536	0\\
537	0\\
538	0\\
539	0\\
540	0\\
541	0\\
542	0\\
543	0\\
544	0\\
545	0\\
546	0\\
547	0\\
548	0\\
549	0\\
550	0\\
551	0\\
552	0\\
553	0\\
554	0\\
555	0\\
556	0\\
557	0\\
558	0\\
559	0\\
560	0\\
561	0\\
562	0\\
563	0\\
564	0\\
565	0\\
566	0\\
567	0\\
568	0\\
569	0\\
570	0\\
571	0\\
572	0\\
573	2.84550322669103e-05\\
574	0.000136105490885286\\
575	0.000246360904768022\\
576	0.00035920980371322\\
577	0.000474576836408107\\
578	0.000592251873280648\\
579	0.000712460866895543\\
580	0.000835098336304564\\
581	0.000931987463067577\\
582	0.00102435477630977\\
583	0.00111851701928916\\
584	0.00121497734033712\\
585	0.00131373271689993\\
586	0.00141477870147017\\
587	0.00151807891510166\\
588	0.00162355054108618\\
589	0.00173106951191172\\
590	0.00184045717402978\\
591	0.00195146461619112\\
592	0.00206375309347527\\
593	0.00217686668353044\\
594	0.00229019015194498\\
595	0.00264340599857734\\
596	0.00335667832726079\\
597	0.00469756192662931\\
598	0.00644286460810295\\
599	0\\
600	0\\
};
\addplot [color=mycolor16,solid,forget plot]
  table[row sep=crcr]{%
1	0\\
2	0\\
3	0\\
4	0\\
5	0\\
6	0\\
7	0\\
8	0\\
9	0\\
10	0\\
11	0\\
12	0\\
13	0\\
14	0\\
15	0\\
16	0\\
17	0\\
18	0\\
19	0\\
20	0\\
21	0\\
22	0\\
23	0\\
24	0\\
25	0\\
26	0\\
27	0\\
28	0\\
29	0\\
30	0\\
31	0\\
32	0\\
33	0\\
34	0\\
35	0\\
36	0\\
37	0\\
38	0\\
39	0\\
40	0\\
41	0\\
42	0\\
43	0\\
44	0\\
45	0\\
46	0\\
47	0\\
48	0\\
49	0\\
50	0\\
51	0\\
52	0\\
53	0\\
54	0\\
55	0\\
56	0\\
57	0\\
58	0\\
59	0\\
60	0\\
61	0\\
62	0\\
63	0\\
64	0\\
65	0\\
66	0\\
67	0\\
68	0\\
69	0\\
70	0\\
71	0\\
72	0\\
73	0\\
74	0\\
75	0\\
76	0\\
77	0\\
78	0\\
79	0\\
80	0\\
81	0\\
82	0\\
83	0\\
84	0\\
85	0\\
86	0\\
87	0\\
88	0\\
89	0\\
90	0\\
91	0\\
92	0\\
93	0\\
94	0\\
95	0\\
96	0\\
97	0\\
98	0\\
99	0\\
100	0\\
101	0\\
102	0\\
103	0\\
104	0\\
105	0\\
106	0\\
107	0\\
108	0\\
109	0\\
110	0\\
111	0\\
112	0\\
113	0\\
114	0\\
115	0\\
116	0\\
117	0\\
118	0\\
119	0\\
120	0\\
121	0\\
122	0\\
123	0\\
124	0\\
125	0\\
126	0\\
127	0\\
128	0\\
129	0\\
130	0\\
131	0\\
132	0\\
133	0\\
134	0\\
135	0\\
136	0\\
137	0\\
138	0\\
139	0\\
140	0\\
141	0\\
142	0\\
143	0\\
144	0\\
145	0\\
146	0\\
147	0\\
148	0\\
149	0\\
150	0\\
151	0\\
152	0\\
153	0\\
154	0\\
155	0\\
156	0\\
157	0\\
158	0\\
159	0\\
160	0\\
161	0\\
162	0\\
163	0\\
164	0\\
165	0\\
166	0\\
167	0\\
168	0\\
169	0\\
170	0\\
171	0\\
172	0\\
173	0\\
174	0\\
175	0\\
176	0\\
177	0\\
178	0\\
179	0\\
180	0\\
181	0\\
182	0\\
183	0\\
184	0\\
185	0\\
186	0\\
187	0\\
188	0\\
189	0\\
190	0\\
191	0\\
192	0\\
193	0\\
194	0\\
195	0\\
196	0\\
197	0\\
198	0\\
199	0\\
200	0\\
201	0\\
202	0\\
203	0\\
204	0\\
205	0\\
206	0\\
207	0\\
208	0\\
209	0\\
210	0\\
211	0\\
212	0\\
213	0\\
214	0\\
215	0\\
216	0\\
217	0\\
218	0\\
219	0\\
220	0\\
221	0\\
222	0\\
223	0\\
224	0\\
225	0\\
226	0\\
227	0\\
228	0\\
229	0\\
230	0\\
231	0\\
232	0\\
233	0\\
234	0\\
235	0\\
236	0\\
237	0\\
238	0\\
239	0\\
240	0\\
241	0\\
242	0\\
243	0\\
244	0\\
245	0\\
246	0\\
247	0\\
248	0\\
249	0\\
250	0\\
251	0\\
252	0\\
253	0\\
254	0\\
255	0\\
256	0\\
257	0\\
258	0\\
259	0\\
260	0\\
261	0\\
262	0\\
263	0\\
264	0\\
265	0\\
266	0\\
267	0\\
268	0\\
269	0\\
270	0\\
271	0\\
272	0\\
273	0\\
274	0\\
275	0\\
276	0\\
277	0\\
278	0\\
279	0\\
280	0\\
281	0\\
282	0\\
283	0\\
284	0\\
285	0\\
286	0\\
287	0\\
288	0\\
289	0\\
290	0\\
291	0\\
292	0\\
293	0\\
294	0\\
295	0\\
296	0\\
297	0\\
298	0\\
299	0\\
300	0\\
301	0\\
302	0\\
303	0\\
304	0\\
305	0\\
306	0\\
307	0\\
308	0\\
309	0\\
310	0\\
311	0\\
312	0\\
313	0\\
314	0\\
315	0\\
316	0\\
317	0\\
318	0\\
319	0\\
320	0\\
321	0\\
322	0\\
323	0\\
324	0\\
325	0\\
326	0\\
327	0\\
328	0\\
329	0\\
330	0\\
331	0\\
332	0\\
333	0\\
334	0\\
335	0\\
336	0\\
337	0\\
338	0\\
339	0\\
340	0\\
341	0\\
342	0\\
343	0\\
344	0\\
345	0\\
346	0\\
347	0\\
348	0\\
349	0\\
350	0\\
351	0\\
352	0\\
353	0\\
354	0\\
355	0\\
356	0\\
357	0\\
358	0\\
359	0\\
360	0\\
361	0\\
362	0\\
363	0\\
364	0\\
365	0\\
366	0\\
367	0\\
368	0\\
369	0\\
370	0\\
371	0\\
372	0\\
373	0\\
374	0\\
375	0\\
376	0\\
377	0\\
378	0\\
379	0\\
380	0\\
381	0\\
382	0\\
383	0\\
384	0\\
385	0\\
386	0\\
387	0\\
388	0\\
389	0\\
390	0\\
391	0\\
392	0\\
393	0\\
394	0\\
395	0\\
396	0\\
397	0\\
398	0\\
399	0\\
400	0\\
401	0\\
402	0\\
403	0\\
404	0\\
405	0\\
406	0\\
407	0\\
408	0\\
409	0\\
410	0\\
411	0\\
412	0\\
413	0\\
414	0\\
415	0\\
416	0\\
417	0\\
418	0\\
419	0\\
420	0\\
421	0\\
422	0\\
423	0\\
424	0\\
425	0\\
426	0\\
427	0\\
428	0\\
429	0\\
430	0\\
431	0\\
432	0\\
433	0\\
434	0\\
435	0\\
436	0\\
437	0\\
438	0\\
439	0\\
440	0\\
441	0\\
442	0\\
443	0\\
444	0\\
445	0\\
446	0\\
447	0\\
448	0\\
449	0\\
450	0\\
451	0\\
452	0\\
453	0\\
454	0\\
455	0\\
456	0\\
457	0\\
458	0\\
459	0\\
460	0\\
461	0\\
462	0\\
463	0\\
464	0\\
465	0\\
466	0\\
467	0\\
468	0\\
469	0\\
470	0\\
471	0\\
472	0\\
473	0\\
474	0\\
475	0\\
476	0\\
477	0\\
478	0\\
479	0\\
480	0\\
481	0\\
482	0\\
483	0\\
484	0\\
485	0\\
486	0\\
487	0\\
488	0\\
489	0\\
490	0\\
491	0\\
492	0\\
493	0\\
494	0\\
495	0\\
496	0\\
497	0\\
498	0\\
499	0\\
500	0\\
501	0\\
502	0\\
503	0\\
504	0\\
505	0\\
506	0\\
507	0\\
508	0\\
509	0\\
510	0\\
511	0\\
512	0\\
513	0\\
514	0\\
515	0\\
516	0\\
517	0\\
518	0\\
519	0\\
520	0\\
521	0\\
522	0\\
523	0\\
524	0\\
525	0\\
526	0\\
527	0\\
528	0\\
529	0\\
530	0\\
531	0\\
532	0\\
533	0\\
534	0\\
535	0\\
536	0\\
537	0\\
538	0\\
539	0\\
540	0\\
541	0\\
542	0\\
543	0\\
544	0\\
545	0\\
546	0\\
547	0\\
548	0\\
549	0\\
550	0\\
551	0\\
552	0\\
553	0\\
554	0\\
555	0\\
556	0\\
557	0\\
558	0\\
559	0\\
560	0\\
561	0\\
562	0\\
563	0\\
564	6.11241591688388e-05\\
565	0.000153833609333673\\
566	0.000247621371611974\\
567	0.000342275882838338\\
568	0.000426454386023444\\
569	0.000490146502349677\\
570	0.000554714051789001\\
571	0.000620092626673308\\
572	0.000686241977351121\\
573	0.000753338581932168\\
574	0.000821311602791528\\
575	0.000890073230408475\\
576	0.000959515814844145\\
577	0.0010295081722191\\
578	0.0010998907600857\\
579	0.00117046937600367\\
580	0.00124101035188772\\
581	0.00131279084750798\\
582	0.00138613243267675\\
583	0.00146105292021269\\
584	0.00153756685550637\\
585	0.00161568839009441\\
586	0.00169543074241313\\
587	0.00177681010814605\\
588	0.00185987394737956\\
589	0.00194472985818468\\
590	0.00203176867243585\\
591	0.00218539535278842\\
592	0.00246148836585481\\
593	0.00274934460856801\\
594	0.00309920221510837\\
595	0.0035236770965596\\
596	0.00425769487060698\\
597	0.00503983077166121\\
598	0.00644286460810295\\
599	0\\
600	0\\
};
\addplot [color=mycolor17,solid,forget plot]
  table[row sep=crcr]{%
1	0\\
2	0\\
3	0\\
4	0\\
5	0\\
6	0\\
7	0\\
8	0\\
9	0\\
10	0\\
11	0\\
12	0\\
13	0\\
14	0\\
15	0\\
16	0\\
17	0\\
18	0\\
19	0\\
20	0\\
21	0\\
22	0\\
23	0\\
24	0\\
25	0\\
26	0\\
27	0\\
28	0\\
29	0\\
30	0\\
31	0\\
32	0\\
33	0\\
34	0\\
35	0\\
36	0\\
37	0\\
38	0\\
39	0\\
40	0\\
41	0\\
42	0\\
43	0\\
44	0\\
45	0\\
46	0\\
47	0\\
48	0\\
49	0\\
50	0\\
51	0\\
52	0\\
53	0\\
54	0\\
55	0\\
56	0\\
57	0\\
58	0\\
59	0\\
60	0\\
61	0\\
62	0\\
63	0\\
64	0\\
65	0\\
66	0\\
67	0\\
68	0\\
69	0\\
70	0\\
71	0\\
72	0\\
73	0\\
74	0\\
75	0\\
76	0\\
77	0\\
78	0\\
79	0\\
80	0\\
81	0\\
82	0\\
83	0\\
84	0\\
85	0\\
86	0\\
87	0\\
88	0\\
89	0\\
90	0\\
91	0\\
92	0\\
93	0\\
94	0\\
95	0\\
96	0\\
97	0\\
98	0\\
99	0\\
100	0\\
101	0\\
102	0\\
103	0\\
104	0\\
105	0\\
106	0\\
107	0\\
108	0\\
109	0\\
110	0\\
111	0\\
112	0\\
113	0\\
114	0\\
115	0\\
116	0\\
117	0\\
118	0\\
119	0\\
120	0\\
121	0\\
122	0\\
123	0\\
124	0\\
125	0\\
126	0\\
127	0\\
128	0\\
129	0\\
130	0\\
131	0\\
132	0\\
133	0\\
134	0\\
135	0\\
136	0\\
137	0\\
138	0\\
139	0\\
140	0\\
141	0\\
142	0\\
143	0\\
144	0\\
145	0\\
146	0\\
147	0\\
148	0\\
149	0\\
150	0\\
151	0\\
152	0\\
153	0\\
154	0\\
155	0\\
156	0\\
157	0\\
158	0\\
159	0\\
160	0\\
161	0\\
162	0\\
163	0\\
164	0\\
165	0\\
166	0\\
167	0\\
168	0\\
169	0\\
170	0\\
171	0\\
172	0\\
173	0\\
174	0\\
175	0\\
176	0\\
177	0\\
178	0\\
179	0\\
180	0\\
181	0\\
182	0\\
183	0\\
184	0\\
185	0\\
186	0\\
187	0\\
188	0\\
189	0\\
190	0\\
191	0\\
192	0\\
193	0\\
194	0\\
195	0\\
196	0\\
197	0\\
198	0\\
199	0\\
200	0\\
201	0\\
202	0\\
203	0\\
204	0\\
205	0\\
206	0\\
207	0\\
208	0\\
209	0\\
210	0\\
211	0\\
212	0\\
213	0\\
214	0\\
215	0\\
216	0\\
217	0\\
218	0\\
219	0\\
220	0\\
221	0\\
222	0\\
223	0\\
224	0\\
225	0\\
226	0\\
227	0\\
228	0\\
229	0\\
230	0\\
231	0\\
232	0\\
233	0\\
234	0\\
235	0\\
236	0\\
237	0\\
238	0\\
239	0\\
240	0\\
241	0\\
242	0\\
243	0\\
244	0\\
245	0\\
246	0\\
247	0\\
248	0\\
249	0\\
250	0\\
251	0\\
252	0\\
253	0\\
254	0\\
255	0\\
256	0\\
257	0\\
258	0\\
259	0\\
260	0\\
261	0\\
262	0\\
263	0\\
264	0\\
265	0\\
266	0\\
267	0\\
268	0\\
269	0\\
270	0\\
271	0\\
272	0\\
273	0\\
274	0\\
275	0\\
276	0\\
277	0\\
278	0\\
279	0\\
280	0\\
281	0\\
282	0\\
283	0\\
284	0\\
285	0\\
286	0\\
287	0\\
288	0\\
289	0\\
290	0\\
291	0\\
292	0\\
293	0\\
294	0\\
295	0\\
296	0\\
297	0\\
298	0\\
299	0\\
300	0\\
301	0\\
302	0\\
303	0\\
304	0\\
305	0\\
306	0\\
307	0\\
308	0\\
309	0\\
310	0\\
311	0\\
312	0\\
313	0\\
314	0\\
315	0\\
316	0\\
317	0\\
318	0\\
319	0\\
320	0\\
321	0\\
322	0\\
323	0\\
324	0\\
325	0\\
326	0\\
327	0\\
328	0\\
329	0\\
330	0\\
331	0\\
332	0\\
333	0\\
334	0\\
335	0\\
336	0\\
337	0\\
338	0\\
339	0\\
340	0\\
341	0\\
342	0\\
343	0\\
344	0\\
345	0\\
346	0\\
347	0\\
348	0\\
349	0\\
350	0\\
351	0\\
352	0\\
353	0\\
354	0\\
355	0\\
356	0\\
357	0\\
358	0\\
359	0\\
360	0\\
361	0\\
362	0\\
363	0\\
364	0\\
365	0\\
366	0\\
367	0\\
368	0\\
369	0\\
370	0\\
371	0\\
372	0\\
373	0\\
374	0\\
375	0\\
376	0\\
377	0\\
378	0\\
379	0\\
380	0\\
381	0\\
382	0\\
383	0\\
384	0\\
385	0\\
386	0\\
387	0\\
388	0\\
389	0\\
390	0\\
391	0\\
392	0\\
393	0\\
394	0\\
395	0\\
396	0\\
397	0\\
398	0\\
399	0\\
400	0\\
401	0\\
402	0\\
403	0\\
404	0\\
405	0\\
406	0\\
407	0\\
408	0\\
409	0\\
410	0\\
411	0\\
412	0\\
413	0\\
414	0\\
415	0\\
416	0\\
417	0\\
418	0\\
419	0\\
420	0\\
421	0\\
422	0\\
423	0\\
424	0\\
425	0\\
426	0\\
427	0\\
428	0\\
429	0\\
430	0\\
431	0\\
432	0\\
433	0\\
434	0\\
435	0\\
436	0\\
437	0\\
438	0\\
439	0\\
440	0\\
441	0\\
442	0\\
443	0\\
444	0\\
445	0\\
446	0\\
447	0\\
448	0\\
449	0\\
450	0\\
451	0\\
452	0\\
453	0\\
454	0\\
455	0\\
456	0\\
457	0\\
458	0\\
459	0\\
460	0\\
461	0\\
462	0\\
463	0\\
464	0\\
465	0\\
466	0\\
467	0\\
468	0\\
469	0\\
470	0\\
471	0\\
472	0\\
473	0\\
474	0\\
475	0\\
476	0\\
477	0\\
478	0\\
479	0\\
480	0\\
481	0\\
482	0\\
483	0\\
484	0\\
485	0\\
486	0\\
487	0\\
488	0\\
489	0\\
490	0\\
491	0\\
492	0\\
493	0\\
494	0\\
495	0\\
496	0\\
497	0\\
498	0\\
499	0\\
500	0\\
501	0\\
502	0\\
503	0\\
504	0\\
505	0\\
506	0\\
507	0\\
508	0\\
509	0\\
510	0\\
511	0\\
512	0\\
513	0\\
514	0\\
515	0\\
516	0\\
517	0\\
518	0\\
519	0\\
520	0\\
521	0\\
522	0\\
523	0\\
524	0\\
525	0\\
526	0\\
527	0\\
528	0\\
529	0\\
530	0\\
531	0\\
532	0\\
533	0\\
534	0\\
535	0\\
536	0\\
537	0\\
538	0\\
539	0\\
540	0\\
541	0\\
542	0\\
543	0\\
544	0\\
545	0\\
546	0\\
547	0\\
548	0\\
549	0\\
550	0\\
551	0\\
552	0\\
553	0\\
554	0\\
555	2.26142481073285e-05\\
556	9.40356632703977e-05\\
557	0.000142834796507731\\
558	0.000192136179755935\\
559	0.000241890470607289\\
560	0.000292049373855764\\
561	0.000342545162924321\\
562	0.00039329687625306\\
563	0.000444211080034401\\
564	0.000495180373312776\\
565	0.000546081753293092\\
566	0.000596774292680829\\
567	0.000647098131380762\\
568	0.00069742347279142\\
569	0.000748696607560934\\
570	0.000800939380062976\\
571	0.000854177294682413\\
572	0.000908439254781951\\
573	0.000963750984062382\\
574	0.00102014366327751\\
575	0.00107765573610446\\
576	0.00113633523369278\\
577	0.00119624324802379\\
578	0.00125745746076917\\
579	0.00132007724518431\\
580	0.00138423112120714\\
581	0.0014499990846374\\
582	0.00151746183138254\\
583	0.001586633971824\\
584	0.00165775679116301\\
585	0.00173067211608291\\
586	0.00180616539338125\\
587	0.0018831539132818\\
588	0.00209862842761386\\
589	0.00237026081777734\\
590	0.00265207182564143\\
591	0.00293299955993958\\
592	0.0031940225228887\\
593	0.00348473277442626\\
594	0.00383471000680442\\
595	0.00412245907862188\\
596	0.00444436306303141\\
597	0.00503983077166121\\
598	0.00644286460810295\\
599	0\\
600	0\\
};
\addplot [color=mycolor18,solid,forget plot]
  table[row sep=crcr]{%
1	0\\
2	0\\
3	0\\
4	0\\
5	0\\
6	0\\
7	0\\
8	0\\
9	0\\
10	0\\
11	0\\
12	0\\
13	0\\
14	0\\
15	0\\
16	0\\
17	0\\
18	0\\
19	0\\
20	0\\
21	0\\
22	0\\
23	0\\
24	0\\
25	0\\
26	0\\
27	0\\
28	0\\
29	0\\
30	0\\
31	0\\
32	0\\
33	0\\
34	0\\
35	0\\
36	0\\
37	0\\
38	0\\
39	0\\
40	0\\
41	0\\
42	0\\
43	0\\
44	0\\
45	0\\
46	0\\
47	0\\
48	0\\
49	0\\
50	0\\
51	0\\
52	0\\
53	0\\
54	0\\
55	0\\
56	0\\
57	0\\
58	0\\
59	0\\
60	0\\
61	0\\
62	0\\
63	0\\
64	0\\
65	0\\
66	0\\
67	0\\
68	0\\
69	0\\
70	0\\
71	0\\
72	0\\
73	0\\
74	0\\
75	0\\
76	0\\
77	0\\
78	0\\
79	0\\
80	0\\
81	0\\
82	0\\
83	0\\
84	0\\
85	0\\
86	0\\
87	0\\
88	0\\
89	0\\
90	0\\
91	0\\
92	0\\
93	0\\
94	0\\
95	0\\
96	0\\
97	0\\
98	0\\
99	0\\
100	0\\
101	0\\
102	0\\
103	0\\
104	0\\
105	0\\
106	0\\
107	0\\
108	0\\
109	0\\
110	0\\
111	0\\
112	0\\
113	0\\
114	0\\
115	0\\
116	0\\
117	0\\
118	0\\
119	0\\
120	0\\
121	0\\
122	0\\
123	0\\
124	0\\
125	0\\
126	0\\
127	0\\
128	0\\
129	0\\
130	0\\
131	0\\
132	0\\
133	0\\
134	0\\
135	0\\
136	0\\
137	0\\
138	0\\
139	0\\
140	0\\
141	0\\
142	0\\
143	0\\
144	0\\
145	0\\
146	0\\
147	0\\
148	0\\
149	0\\
150	0\\
151	0\\
152	0\\
153	0\\
154	0\\
155	0\\
156	0\\
157	0\\
158	0\\
159	0\\
160	0\\
161	0\\
162	0\\
163	0\\
164	0\\
165	0\\
166	0\\
167	0\\
168	0\\
169	0\\
170	0\\
171	0\\
172	0\\
173	0\\
174	0\\
175	0\\
176	0\\
177	0\\
178	0\\
179	0\\
180	0\\
181	0\\
182	0\\
183	0\\
184	0\\
185	0\\
186	0\\
187	0\\
188	0\\
189	0\\
190	0\\
191	0\\
192	0\\
193	0\\
194	0\\
195	0\\
196	0\\
197	0\\
198	0\\
199	0\\
200	0\\
201	0\\
202	0\\
203	0\\
204	0\\
205	0\\
206	0\\
207	0\\
208	0\\
209	0\\
210	0\\
211	0\\
212	0\\
213	0\\
214	0\\
215	0\\
216	0\\
217	0\\
218	0\\
219	0\\
220	0\\
221	0\\
222	0\\
223	0\\
224	0\\
225	0\\
226	0\\
227	0\\
228	0\\
229	0\\
230	0\\
231	0\\
232	0\\
233	0\\
234	0\\
235	0\\
236	0\\
237	0\\
238	0\\
239	0\\
240	0\\
241	0\\
242	0\\
243	0\\
244	0\\
245	0\\
246	0\\
247	0\\
248	0\\
249	0\\
250	0\\
251	0\\
252	0\\
253	0\\
254	0\\
255	0\\
256	0\\
257	0\\
258	0\\
259	0\\
260	0\\
261	0\\
262	0\\
263	0\\
264	0\\
265	0\\
266	0\\
267	0\\
268	0\\
269	0\\
270	0\\
271	0\\
272	0\\
273	0\\
274	0\\
275	0\\
276	0\\
277	0\\
278	0\\
279	0\\
280	0\\
281	0\\
282	0\\
283	0\\
284	0\\
285	0\\
286	0\\
287	0\\
288	0\\
289	0\\
290	0\\
291	0\\
292	0\\
293	0\\
294	0\\
295	0\\
296	0\\
297	0\\
298	0\\
299	0\\
300	0\\
301	0\\
302	0\\
303	0\\
304	0\\
305	0\\
306	0\\
307	0\\
308	0\\
309	0\\
310	0\\
311	0\\
312	0\\
313	0\\
314	0\\
315	0\\
316	0\\
317	0\\
318	0\\
319	0\\
320	0\\
321	0\\
322	0\\
323	0\\
324	0\\
325	0\\
326	0\\
327	0\\
328	0\\
329	0\\
330	0\\
331	0\\
332	0\\
333	0\\
334	0\\
335	0\\
336	0\\
337	0\\
338	0\\
339	0\\
340	0\\
341	0\\
342	0\\
343	0\\
344	0\\
345	0\\
346	0\\
347	0\\
348	0\\
349	0\\
350	0\\
351	0\\
352	0\\
353	0\\
354	0\\
355	0\\
356	0\\
357	0\\
358	0\\
359	0\\
360	0\\
361	0\\
362	0\\
363	0\\
364	0\\
365	0\\
366	0\\
367	0\\
368	0\\
369	0\\
370	0\\
371	0\\
372	0\\
373	0\\
374	0\\
375	0\\
376	0\\
377	0\\
378	0\\
379	0\\
380	0\\
381	0\\
382	0\\
383	0\\
384	0\\
385	0\\
386	0\\
387	0\\
388	0\\
389	0\\
390	0\\
391	0\\
392	0\\
393	0\\
394	0\\
395	0\\
396	0\\
397	0\\
398	0\\
399	0\\
400	0\\
401	0\\
402	0\\
403	0\\
404	0\\
405	0\\
406	0\\
407	0\\
408	0\\
409	0\\
410	0\\
411	0\\
412	0\\
413	0\\
414	0\\
415	0\\
416	0\\
417	0\\
418	0\\
419	0\\
420	0\\
421	0\\
422	0\\
423	0\\
424	0\\
425	0\\
426	0\\
427	0\\
428	0\\
429	0\\
430	0\\
431	0\\
432	0\\
433	0\\
434	0\\
435	0\\
436	0\\
437	0\\
438	0\\
439	0\\
440	0\\
441	0\\
442	0\\
443	0\\
444	0\\
445	0\\
446	0\\
447	0\\
448	0\\
449	0\\
450	0\\
451	0\\
452	0\\
453	0\\
454	0\\
455	0\\
456	0\\
457	0\\
458	0\\
459	0\\
460	0\\
461	0\\
462	0\\
463	0\\
464	0\\
465	0\\
466	0\\
467	0\\
468	0\\
469	0\\
470	0\\
471	0\\
472	0\\
473	0\\
474	0\\
475	0\\
476	0\\
477	0\\
478	0\\
479	0\\
480	0\\
481	0\\
482	0\\
483	0\\
484	0\\
485	0\\
486	0\\
487	0\\
488	0\\
489	0\\
490	0\\
491	0\\
492	0\\
493	0\\
494	0\\
495	0\\
496	0\\
497	0\\
498	0\\
499	0\\
500	0\\
501	0\\
502	0\\
503	0\\
504	0\\
505	0\\
506	0\\
507	0\\
508	0\\
509	0\\
510	0\\
511	0\\
512	0\\
513	0\\
514	0\\
515	0\\
516	0\\
517	0\\
518	0\\
519	0\\
520	0\\
521	0\\
522	0\\
523	0\\
524	0\\
525	0\\
526	0\\
527	0\\
528	0\\
529	0\\
530	0\\
531	0\\
532	0\\
533	0\\
534	0\\
535	0\\
536	0\\
537	0\\
538	0\\
539	0\\
540	0\\
541	0\\
542	0\\
543	0\\
544	0\\
545	0\\
546	0\\
547	0\\
548	0\\
549	2.32520210821688e-05\\
550	6.29822389686342e-05\\
551	0.000102594284985584\\
552	0.000142000070962094\\
553	0.000181109221367284\\
554	0.000219804609413708\\
555	0.000257961564445722\\
556	0.000295780758493718\\
557	0.000334217678983119\\
558	0.000373283831162645\\
559	0.000412993211787373\\
560	0.000453363060135725\\
561	0.000494414867686697\\
562	0.000536175585096235\\
563	0.000578679087463031\\
564	0.000621967988206539\\
565	0.000666095878568849\\
566	0.000711130094990775\\
567	0.000757155082147057\\
568	0.000804247376185817\\
569	0.000852443417848277\\
570	0.00090178212884265\\
571	0.000952305207691893\\
572	0.00100405760152108\\
573	0.00105708760588472\\
574	0.00111144706147627\\
575	0.0011671914887648\\
576	0.00122442232407532\\
577	0.00128313096863521\\
578	0.0013433810070614\\
579	0.00140538131268943\\
580	0.00146902614413746\\
581	0.00153430958574611\\
582	0.00160209512119075\\
583	0.00167132716120464\\
584	0.0017419265568923\\
585	0.0019582562819987\\
586	0.00223293796086997\\
587	0.00252513250926167\\
588	0.00278502806553789\\
589	0.00303946052792341\\
590	0.0033058313870396\\
591	0.00353307097397887\\
592	0.00367241590802784\\
593	0.00382251134333667\\
594	0.00397190834964223\\
595	0.00415280575412962\\
596	0.00444436306303141\\
597	0.00503983077166121\\
598	0.00644286460810295\\
599	0\\
600	0\\
};
\addplot [color=red!25!mycolor17,solid,forget plot]
  table[row sep=crcr]{%
1	0\\
2	0\\
3	0\\
4	0\\
5	0\\
6	0\\
7	0\\
8	0\\
9	0\\
10	0\\
11	0\\
12	0\\
13	0\\
14	0\\
15	0\\
16	0\\
17	0\\
18	0\\
19	0\\
20	0\\
21	0\\
22	0\\
23	0\\
24	0\\
25	0\\
26	0\\
27	0\\
28	0\\
29	0\\
30	0\\
31	0\\
32	0\\
33	0\\
34	0\\
35	0\\
36	0\\
37	0\\
38	0\\
39	0\\
40	0\\
41	0\\
42	0\\
43	0\\
44	0\\
45	0\\
46	0\\
47	0\\
48	0\\
49	0\\
50	0\\
51	0\\
52	0\\
53	0\\
54	0\\
55	0\\
56	0\\
57	0\\
58	0\\
59	0\\
60	0\\
61	0\\
62	0\\
63	0\\
64	0\\
65	0\\
66	0\\
67	0\\
68	0\\
69	0\\
70	0\\
71	0\\
72	0\\
73	0\\
74	0\\
75	0\\
76	0\\
77	0\\
78	0\\
79	0\\
80	0\\
81	0\\
82	0\\
83	0\\
84	0\\
85	0\\
86	0\\
87	0\\
88	0\\
89	0\\
90	0\\
91	0\\
92	0\\
93	0\\
94	0\\
95	0\\
96	0\\
97	0\\
98	0\\
99	0\\
100	0\\
101	0\\
102	0\\
103	0\\
104	0\\
105	0\\
106	0\\
107	0\\
108	0\\
109	0\\
110	0\\
111	0\\
112	0\\
113	0\\
114	0\\
115	0\\
116	0\\
117	0\\
118	0\\
119	0\\
120	0\\
121	0\\
122	0\\
123	0\\
124	0\\
125	0\\
126	0\\
127	0\\
128	0\\
129	0\\
130	0\\
131	0\\
132	0\\
133	0\\
134	0\\
135	0\\
136	0\\
137	0\\
138	0\\
139	0\\
140	0\\
141	0\\
142	0\\
143	0\\
144	0\\
145	0\\
146	0\\
147	0\\
148	0\\
149	0\\
150	0\\
151	0\\
152	0\\
153	0\\
154	0\\
155	0\\
156	0\\
157	0\\
158	0\\
159	0\\
160	0\\
161	0\\
162	0\\
163	0\\
164	0\\
165	0\\
166	0\\
167	0\\
168	0\\
169	0\\
170	0\\
171	0\\
172	0\\
173	0\\
174	0\\
175	0\\
176	0\\
177	0\\
178	0\\
179	0\\
180	0\\
181	0\\
182	0\\
183	0\\
184	0\\
185	0\\
186	0\\
187	0\\
188	0\\
189	0\\
190	0\\
191	0\\
192	0\\
193	0\\
194	0\\
195	0\\
196	0\\
197	0\\
198	0\\
199	0\\
200	0\\
201	0\\
202	0\\
203	0\\
204	0\\
205	0\\
206	0\\
207	0\\
208	0\\
209	0\\
210	0\\
211	0\\
212	0\\
213	0\\
214	0\\
215	0\\
216	0\\
217	0\\
218	0\\
219	0\\
220	0\\
221	0\\
222	0\\
223	0\\
224	0\\
225	0\\
226	0\\
227	0\\
228	0\\
229	0\\
230	0\\
231	0\\
232	0\\
233	0\\
234	0\\
235	0\\
236	0\\
237	0\\
238	0\\
239	0\\
240	0\\
241	0\\
242	0\\
243	0\\
244	0\\
245	0\\
246	0\\
247	0\\
248	0\\
249	0\\
250	0\\
251	0\\
252	0\\
253	0\\
254	0\\
255	0\\
256	0\\
257	0\\
258	0\\
259	0\\
260	0\\
261	0\\
262	0\\
263	0\\
264	0\\
265	0\\
266	0\\
267	0\\
268	0\\
269	0\\
270	0\\
271	0\\
272	0\\
273	0\\
274	0\\
275	0\\
276	0\\
277	0\\
278	0\\
279	0\\
280	0\\
281	0\\
282	0\\
283	0\\
284	0\\
285	0\\
286	0\\
287	0\\
288	0\\
289	0\\
290	0\\
291	0\\
292	0\\
293	0\\
294	0\\
295	0\\
296	0\\
297	0\\
298	0\\
299	0\\
300	0\\
301	0\\
302	0\\
303	0\\
304	0\\
305	0\\
306	0\\
307	0\\
308	0\\
309	0\\
310	0\\
311	0\\
312	0\\
313	0\\
314	0\\
315	0\\
316	0\\
317	0\\
318	0\\
319	0\\
320	0\\
321	0\\
322	0\\
323	0\\
324	0\\
325	0\\
326	0\\
327	0\\
328	0\\
329	0\\
330	0\\
331	0\\
332	0\\
333	0\\
334	0\\
335	0\\
336	0\\
337	0\\
338	0\\
339	0\\
340	0\\
341	0\\
342	0\\
343	0\\
344	0\\
345	0\\
346	0\\
347	0\\
348	0\\
349	0\\
350	0\\
351	0\\
352	0\\
353	0\\
354	0\\
355	0\\
356	0\\
357	0\\
358	0\\
359	0\\
360	0\\
361	0\\
362	0\\
363	0\\
364	0\\
365	0\\
366	0\\
367	0\\
368	0\\
369	0\\
370	0\\
371	0\\
372	0\\
373	0\\
374	0\\
375	0\\
376	0\\
377	0\\
378	0\\
379	0\\
380	0\\
381	0\\
382	0\\
383	0\\
384	0\\
385	0\\
386	0\\
387	0\\
388	0\\
389	0\\
390	0\\
391	0\\
392	0\\
393	0\\
394	0\\
395	0\\
396	0\\
397	0\\
398	0\\
399	0\\
400	0\\
401	0\\
402	0\\
403	0\\
404	0\\
405	0\\
406	0\\
407	0\\
408	0\\
409	0\\
410	0\\
411	0\\
412	0\\
413	0\\
414	0\\
415	0\\
416	0\\
417	0\\
418	0\\
419	0\\
420	0\\
421	0\\
422	0\\
423	0\\
424	0\\
425	0\\
426	0\\
427	0\\
428	0\\
429	0\\
430	0\\
431	0\\
432	0\\
433	0\\
434	0\\
435	0\\
436	0\\
437	0\\
438	0\\
439	0\\
440	0\\
441	0\\
442	0\\
443	0\\
444	0\\
445	0\\
446	0\\
447	0\\
448	0\\
449	0\\
450	0\\
451	0\\
452	0\\
453	0\\
454	0\\
455	0\\
456	0\\
457	0\\
458	0\\
459	0\\
460	0\\
461	0\\
462	0\\
463	0\\
464	0\\
465	0\\
466	0\\
467	0\\
468	0\\
469	0\\
470	0\\
471	0\\
472	0\\
473	0\\
474	0\\
475	0\\
476	0\\
477	0\\
478	0\\
479	0\\
480	0\\
481	0\\
482	0\\
483	0\\
484	0\\
485	0\\
486	0\\
487	0\\
488	0\\
489	0\\
490	0\\
491	0\\
492	0\\
493	0\\
494	0\\
495	0\\
496	0\\
497	0\\
498	0\\
499	0\\
500	0\\
501	0\\
502	0\\
503	0\\
504	0\\
505	0\\
506	0\\
507	0\\
508	0\\
509	0\\
510	0\\
511	0\\
512	0\\
513	0\\
514	0\\
515	0\\
516	0\\
517	0\\
518	0\\
519	0\\
520	0\\
521	0\\
522	0\\
523	0\\
524	0\\
525	0\\
526	0\\
527	0\\
528	0\\
529	0\\
530	0\\
531	0\\
532	0\\
533	0\\
534	0\\
535	0\\
536	0\\
537	0\\
538	0\\
539	0\\
540	0\\
541	0\\
542	0\\
543	0\\
544	0\\
545	1.60914883874299e-05\\
546	4.61802426780253e-05\\
547	7.66629135509208e-05\\
548	0.000107555208654854\\
549	0.000138876541337362\\
550	0.000170650593952384\\
551	0.000202914517604868\\
552	0.000235708465118579\\
553	0.000269081896836238\\
554	0.000303087169673518\\
555	0.000337789238210994\\
556	0.000373251698431509\\
557	0.000409499257060565\\
558	0.000446558227820018\\
559	0.000484456617435164\\
560	0.000523224188690424\\
561	0.000562892484763894\\
562	0.000603494791845105\\
563	0.000645066023735202\\
564	0.000687642494400291\\
565	0.000731261531316776\\
566	0.000775960935897401\\
567	0.000821778458148387\\
568	0.000868751612779525\\
569	0.000916919738931042\\
570	0.000966334419799042\\
571	0.00101705468567766\\
572	0.00106906417782553\\
573	0.00112241341855564\\
574	0.00117715592173479\\
575	0.00123354169018176\\
576	0.00129128803526078\\
577	0.00135048220327041\\
578	0.00141156374788578\\
579	0.00147451749231703\\
580	0.00153861733116412\\
581	0.00160404882434682\\
582	0.0017589725145217\\
583	0.00203755571568777\\
584	0.00233345386073883\\
585	0.00259266855288805\\
586	0.00284513355551299\\
587	0.00310713312521459\\
588	0.0032721022788783\\
589	0.00339546551785458\\
590	0.00351237211383029\\
591	0.00361868673458143\\
592	0.0037291949721508\\
593	0.00384481075529708\\
594	0.00397672044113206\\
595	0.00415280575412961\\
596	0.00444436306303141\\
597	0.00503983077166121\\
598	0.00644286460810295\\
599	0\\
600	0\\
};
\addplot [color=mycolor19,solid,forget plot]
  table[row sep=crcr]{%
1	0\\
2	0\\
3	0\\
4	0\\
5	0\\
6	0\\
7	0\\
8	0\\
9	0\\
10	0\\
11	0\\
12	0\\
13	0\\
14	0\\
15	0\\
16	0\\
17	0\\
18	0\\
19	0\\
20	0\\
21	0\\
22	0\\
23	0\\
24	0\\
25	0\\
26	0\\
27	0\\
28	0\\
29	0\\
30	0\\
31	0\\
32	0\\
33	0\\
34	0\\
35	0\\
36	0\\
37	0\\
38	0\\
39	0\\
40	0\\
41	0\\
42	0\\
43	0\\
44	0\\
45	0\\
46	0\\
47	0\\
48	0\\
49	0\\
50	0\\
51	0\\
52	0\\
53	0\\
54	0\\
55	0\\
56	0\\
57	0\\
58	0\\
59	0\\
60	0\\
61	0\\
62	0\\
63	0\\
64	0\\
65	0\\
66	0\\
67	0\\
68	0\\
69	0\\
70	0\\
71	0\\
72	0\\
73	0\\
74	0\\
75	0\\
76	0\\
77	0\\
78	0\\
79	0\\
80	0\\
81	0\\
82	0\\
83	0\\
84	0\\
85	0\\
86	0\\
87	0\\
88	0\\
89	0\\
90	0\\
91	0\\
92	0\\
93	0\\
94	0\\
95	0\\
96	0\\
97	0\\
98	0\\
99	0\\
100	0\\
101	0\\
102	0\\
103	0\\
104	0\\
105	0\\
106	0\\
107	0\\
108	0\\
109	0\\
110	0\\
111	0\\
112	0\\
113	0\\
114	0\\
115	0\\
116	0\\
117	0\\
118	0\\
119	0\\
120	0\\
121	0\\
122	0\\
123	0\\
124	0\\
125	0\\
126	0\\
127	0\\
128	0\\
129	0\\
130	0\\
131	0\\
132	0\\
133	0\\
134	0\\
135	0\\
136	0\\
137	0\\
138	0\\
139	0\\
140	0\\
141	0\\
142	0\\
143	0\\
144	0\\
145	0\\
146	0\\
147	0\\
148	0\\
149	0\\
150	0\\
151	0\\
152	0\\
153	0\\
154	0\\
155	0\\
156	0\\
157	0\\
158	0\\
159	0\\
160	0\\
161	0\\
162	0\\
163	0\\
164	0\\
165	0\\
166	0\\
167	0\\
168	0\\
169	0\\
170	0\\
171	0\\
172	0\\
173	0\\
174	0\\
175	0\\
176	0\\
177	0\\
178	0\\
179	0\\
180	0\\
181	0\\
182	0\\
183	0\\
184	0\\
185	0\\
186	0\\
187	0\\
188	0\\
189	0\\
190	0\\
191	0\\
192	0\\
193	0\\
194	0\\
195	0\\
196	0\\
197	0\\
198	0\\
199	0\\
200	0\\
201	0\\
202	0\\
203	0\\
204	0\\
205	0\\
206	0\\
207	0\\
208	0\\
209	0\\
210	0\\
211	0\\
212	0\\
213	0\\
214	0\\
215	0\\
216	0\\
217	0\\
218	0\\
219	0\\
220	0\\
221	0\\
222	0\\
223	0\\
224	0\\
225	0\\
226	0\\
227	0\\
228	0\\
229	0\\
230	0\\
231	0\\
232	0\\
233	0\\
234	0\\
235	0\\
236	0\\
237	0\\
238	0\\
239	0\\
240	0\\
241	0\\
242	0\\
243	0\\
244	0\\
245	0\\
246	0\\
247	0\\
248	0\\
249	0\\
250	0\\
251	0\\
252	0\\
253	0\\
254	0\\
255	0\\
256	0\\
257	0\\
258	0\\
259	0\\
260	0\\
261	0\\
262	0\\
263	0\\
264	0\\
265	0\\
266	0\\
267	0\\
268	0\\
269	0\\
270	0\\
271	0\\
272	0\\
273	0\\
274	0\\
275	0\\
276	0\\
277	0\\
278	0\\
279	0\\
280	0\\
281	0\\
282	0\\
283	0\\
284	0\\
285	0\\
286	0\\
287	0\\
288	0\\
289	0\\
290	0\\
291	0\\
292	0\\
293	0\\
294	0\\
295	0\\
296	0\\
297	0\\
298	0\\
299	0\\
300	0\\
301	0\\
302	0\\
303	0\\
304	0\\
305	0\\
306	0\\
307	0\\
308	0\\
309	0\\
310	0\\
311	0\\
312	0\\
313	0\\
314	0\\
315	0\\
316	0\\
317	0\\
318	0\\
319	0\\
320	0\\
321	0\\
322	0\\
323	0\\
324	0\\
325	0\\
326	0\\
327	0\\
328	0\\
329	0\\
330	0\\
331	0\\
332	0\\
333	0\\
334	0\\
335	0\\
336	0\\
337	0\\
338	0\\
339	0\\
340	0\\
341	0\\
342	0\\
343	0\\
344	0\\
345	0\\
346	0\\
347	0\\
348	0\\
349	0\\
350	0\\
351	0\\
352	0\\
353	0\\
354	0\\
355	0\\
356	0\\
357	0\\
358	0\\
359	0\\
360	0\\
361	0\\
362	0\\
363	0\\
364	0\\
365	0\\
366	0\\
367	0\\
368	0\\
369	0\\
370	0\\
371	0\\
372	0\\
373	0\\
374	0\\
375	0\\
376	0\\
377	0\\
378	0\\
379	0\\
380	0\\
381	0\\
382	0\\
383	0\\
384	0\\
385	0\\
386	0\\
387	0\\
388	0\\
389	0\\
390	0\\
391	0\\
392	0\\
393	0\\
394	0\\
395	0\\
396	0\\
397	0\\
398	0\\
399	0\\
400	0\\
401	0\\
402	0\\
403	0\\
404	0\\
405	0\\
406	0\\
407	0\\
408	0\\
409	0\\
410	0\\
411	0\\
412	0\\
413	0\\
414	0\\
415	0\\
416	0\\
417	0\\
418	0\\
419	0\\
420	0\\
421	0\\
422	0\\
423	0\\
424	0\\
425	0\\
426	0\\
427	0\\
428	0\\
429	0\\
430	0\\
431	0\\
432	0\\
433	0\\
434	0\\
435	0\\
436	0\\
437	0\\
438	0\\
439	0\\
440	0\\
441	0\\
442	0\\
443	0\\
444	0\\
445	0\\
446	0\\
447	0\\
448	0\\
449	0\\
450	0\\
451	0\\
452	0\\
453	0\\
454	0\\
455	0\\
456	0\\
457	0\\
458	0\\
459	0\\
460	0\\
461	0\\
462	0\\
463	0\\
464	0\\
465	0\\
466	0\\
467	0\\
468	0\\
469	0\\
470	0\\
471	0\\
472	0\\
473	0\\
474	0\\
475	0\\
476	0\\
477	0\\
478	0\\
479	0\\
480	0\\
481	0\\
482	0\\
483	0\\
484	0\\
485	0\\
486	0\\
487	0\\
488	0\\
489	0\\
490	0\\
491	0\\
492	0\\
493	0\\
494	0\\
495	0\\
496	0\\
497	0\\
498	0\\
499	0\\
500	0\\
501	0\\
502	0\\
503	0\\
504	0\\
505	0\\
506	0\\
507	0\\
508	0\\
509	0\\
510	0\\
511	0\\
512	0\\
513	0\\
514	0\\
515	0\\
516	0\\
517	0\\
518	0\\
519	0\\
520	0\\
521	0\\
522	0\\
523	0\\
524	0\\
525	0\\
526	0\\
527	0\\
528	0\\
529	0\\
530	0\\
531	0\\
532	0\\
533	0\\
534	0\\
535	0\\
536	0\\
537	0\\
538	0\\
539	0\\
540	0\\
541	0\\
542	0\\
543	1.6281523779886e-05\\
544	4.35246637006791e-05\\
545	7.13087122634737e-05\\
546	9.96552821353406e-05\\
547	0.000128587210985315\\
548	0.000158128564717491\\
549	0.000188304258646555\\
550	0.000219151430388676\\
551	0.000250701151250487\\
552	0.000282981360877781\\
553	0.000316017445589515\\
554	0.000349834654432245\\
555	0.000384457427831735\\
556	0.000419909302190068\\
557	0.000456214671566793\\
558	0.000493398798486647\\
559	0.000531487829578367\\
560	0.000570508810018441\\
561	0.000610489697648098\\
562	0.000651459746879398\\
563	0.000693449329616425\\
564	0.000736489807219065\\
565	0.000780619589166497\\
566	0.000825901473907119\\
567	0.000872296831358665\\
568	0.000919843835008248\\
569	0.000968584807559033\\
570	0.00101855905585974\\
571	0.00106996928087748\\
572	0.0011226278711555\\
573	0.00117653489629999\\
574	0.00123176683108536\\
575	0.00128877167129672\\
576	0.00134736204101222\\
577	0.00140700004607742\\
578	0.00146757979920448\\
579	0.00152899875679746\\
580	0.00178174393908874\\
581	0.00207611800840707\\
582	0.00235517143267944\\
583	0.0026041334094422\\
584	0.00286156611296512\\
585	0.00301924574469048\\
586	0.00313397161576881\\
587	0.00323536006699526\\
588	0.00333221291604738\\
589	0.00342954842160734\\
590	0.0035273364337839\\
591	0.00362803120808679\\
592	0.00373275376052986\\
593	0.00384555904940006\\
594	0.00397672044113206\\
595	0.00415280575412961\\
596	0.00444436306303141\\
597	0.00503983077166121\\
598	0.00644286460810295\\
599	0\\
600	0\\
};
\addplot [color=red!50!mycolor17,solid,forget plot]
  table[row sep=crcr]{%
1	0\\
2	0\\
3	0\\
4	0\\
5	0\\
6	0\\
7	0\\
8	0\\
9	0\\
10	0\\
11	0\\
12	0\\
13	0\\
14	0\\
15	0\\
16	0\\
17	0\\
18	0\\
19	0\\
20	0\\
21	0\\
22	0\\
23	0\\
24	0\\
25	0\\
26	0\\
27	0\\
28	0\\
29	0\\
30	0\\
31	0\\
32	0\\
33	0\\
34	0\\
35	0\\
36	0\\
37	0\\
38	0\\
39	0\\
40	0\\
41	0\\
42	0\\
43	0\\
44	0\\
45	0\\
46	0\\
47	0\\
48	0\\
49	0\\
50	0\\
51	0\\
52	0\\
53	0\\
54	0\\
55	0\\
56	0\\
57	0\\
58	0\\
59	0\\
60	0\\
61	0\\
62	0\\
63	0\\
64	0\\
65	0\\
66	0\\
67	0\\
68	0\\
69	0\\
70	0\\
71	0\\
72	0\\
73	0\\
74	0\\
75	0\\
76	0\\
77	0\\
78	0\\
79	0\\
80	0\\
81	0\\
82	0\\
83	0\\
84	0\\
85	0\\
86	0\\
87	0\\
88	0\\
89	0\\
90	0\\
91	0\\
92	0\\
93	0\\
94	0\\
95	0\\
96	0\\
97	0\\
98	0\\
99	0\\
100	0\\
101	0\\
102	0\\
103	0\\
104	0\\
105	0\\
106	0\\
107	0\\
108	0\\
109	0\\
110	0\\
111	0\\
112	0\\
113	0\\
114	0\\
115	0\\
116	0\\
117	0\\
118	0\\
119	0\\
120	0\\
121	0\\
122	0\\
123	0\\
124	0\\
125	0\\
126	0\\
127	0\\
128	0\\
129	0\\
130	0\\
131	0\\
132	0\\
133	0\\
134	0\\
135	0\\
136	0\\
137	0\\
138	0\\
139	0\\
140	0\\
141	0\\
142	0\\
143	0\\
144	0\\
145	0\\
146	0\\
147	0\\
148	0\\
149	0\\
150	0\\
151	0\\
152	0\\
153	0\\
154	0\\
155	0\\
156	0\\
157	0\\
158	0\\
159	0\\
160	0\\
161	0\\
162	0\\
163	0\\
164	0\\
165	0\\
166	0\\
167	0\\
168	0\\
169	0\\
170	0\\
171	0\\
172	0\\
173	0\\
174	0\\
175	0\\
176	0\\
177	0\\
178	0\\
179	0\\
180	0\\
181	0\\
182	0\\
183	0\\
184	0\\
185	0\\
186	0\\
187	0\\
188	0\\
189	0\\
190	0\\
191	0\\
192	0\\
193	0\\
194	0\\
195	0\\
196	0\\
197	0\\
198	0\\
199	0\\
200	0\\
201	0\\
202	0\\
203	0\\
204	0\\
205	0\\
206	0\\
207	0\\
208	0\\
209	0\\
210	0\\
211	0\\
212	0\\
213	0\\
214	0\\
215	0\\
216	0\\
217	0\\
218	0\\
219	0\\
220	0\\
221	0\\
222	0\\
223	0\\
224	0\\
225	0\\
226	0\\
227	0\\
228	0\\
229	0\\
230	0\\
231	0\\
232	0\\
233	0\\
234	0\\
235	0\\
236	0\\
237	0\\
238	0\\
239	0\\
240	0\\
241	0\\
242	0\\
243	0\\
244	0\\
245	0\\
246	0\\
247	0\\
248	0\\
249	0\\
250	0\\
251	0\\
252	0\\
253	0\\
254	0\\
255	0\\
256	0\\
257	0\\
258	0\\
259	0\\
260	0\\
261	0\\
262	0\\
263	0\\
264	0\\
265	0\\
266	0\\
267	0\\
268	0\\
269	0\\
270	0\\
271	0\\
272	0\\
273	0\\
274	0\\
275	0\\
276	0\\
277	0\\
278	0\\
279	0\\
280	0\\
281	0\\
282	0\\
283	0\\
284	0\\
285	0\\
286	0\\
287	0\\
288	0\\
289	0\\
290	0\\
291	0\\
292	0\\
293	0\\
294	0\\
295	0\\
296	0\\
297	0\\
298	0\\
299	0\\
300	0\\
301	0\\
302	0\\
303	0\\
304	0\\
305	0\\
306	0\\
307	0\\
308	0\\
309	0\\
310	0\\
311	0\\
312	0\\
313	0\\
314	0\\
315	0\\
316	0\\
317	0\\
318	0\\
319	0\\
320	0\\
321	0\\
322	0\\
323	0\\
324	0\\
325	0\\
326	0\\
327	0\\
328	0\\
329	0\\
330	0\\
331	0\\
332	0\\
333	0\\
334	0\\
335	0\\
336	0\\
337	0\\
338	0\\
339	0\\
340	0\\
341	0\\
342	0\\
343	0\\
344	0\\
345	0\\
346	0\\
347	0\\
348	0\\
349	0\\
350	0\\
351	0\\
352	0\\
353	0\\
354	0\\
355	0\\
356	0\\
357	0\\
358	0\\
359	0\\
360	0\\
361	0\\
362	0\\
363	0\\
364	0\\
365	0\\
366	0\\
367	0\\
368	0\\
369	0\\
370	0\\
371	0\\
372	0\\
373	0\\
374	0\\
375	0\\
376	0\\
377	0\\
378	0\\
379	0\\
380	0\\
381	0\\
382	0\\
383	0\\
384	0\\
385	0\\
386	0\\
387	0\\
388	0\\
389	0\\
390	0\\
391	0\\
392	0\\
393	0\\
394	0\\
395	0\\
396	0\\
397	0\\
398	0\\
399	0\\
400	0\\
401	0\\
402	0\\
403	0\\
404	0\\
405	0\\
406	0\\
407	0\\
408	0\\
409	0\\
410	0\\
411	0\\
412	0\\
413	0\\
414	0\\
415	0\\
416	0\\
417	0\\
418	0\\
419	0\\
420	0\\
421	0\\
422	0\\
423	0\\
424	0\\
425	0\\
426	0\\
427	0\\
428	0\\
429	0\\
430	0\\
431	0\\
432	0\\
433	0\\
434	0\\
435	0\\
436	0\\
437	0\\
438	0\\
439	0\\
440	0\\
441	0\\
442	0\\
443	0\\
444	0\\
445	0\\
446	0\\
447	0\\
448	0\\
449	0\\
450	0\\
451	0\\
452	0\\
453	0\\
454	0\\
455	0\\
456	0\\
457	0\\
458	0\\
459	0\\
460	0\\
461	0\\
462	0\\
463	0\\
464	0\\
465	0\\
466	0\\
467	0\\
468	0\\
469	0\\
470	0\\
471	0\\
472	0\\
473	0\\
474	0\\
475	0\\
476	0\\
477	0\\
478	0\\
479	0\\
480	0\\
481	0\\
482	0\\
483	0\\
484	0\\
485	0\\
486	0\\
487	0\\
488	0\\
489	0\\
490	0\\
491	0\\
492	0\\
493	0\\
494	0\\
495	0\\
496	0\\
497	0\\
498	0\\
499	0\\
500	0\\
501	0\\
502	0\\
503	0\\
504	0\\
505	0\\
506	0\\
507	0\\
508	0\\
509	0\\
510	0\\
511	0\\
512	0\\
513	0\\
514	0\\
515	0\\
516	0\\
517	0\\
518	0\\
519	0\\
520	0\\
521	0\\
522	0\\
523	0\\
524	0\\
525	0\\
526	0\\
527	0\\
528	0\\
529	0\\
530	0\\
531	0\\
532	0\\
533	0\\
534	0\\
535	0\\
536	0\\
537	0\\
538	0\\
539	0\\
540	0\\
541	0\\
542	2.31819356943729e-05\\
543	4.97808753607325e-05\\
544	7.69628614176778e-05\\
545	0.000104747319125633\\
546	0.000133154249522804\\
547	0.000162204210497156\\
548	0.000191918039953681\\
549	0.000222329556741014\\
550	0.000253464773718292\\
551	0.000285344204403849\\
552	0.00031798791545124\\
553	0.000351416424199199\\
554	0.000385650737194353\\
555	0.00042071243627783\\
556	0.000456623783107471\\
557	0.000493407751197362\\
558	0.000531088593263046\\
559	0.00056969136713803\\
560	0.000609241967672466\\
561	0.000649799792050662\\
562	0.000691347682418427\\
563	0.000733888225033008\\
564	0.000777453378418064\\
565	0.00082207615087447\\
566	0.000867771153143799\\
567	0.000914688584196929\\
568	0.000962856697167533\\
569	0.00101209789593015\\
570	0.00106246397516016\\
571	0.00111390186467904\\
572	0.00116698002710393\\
573	0.00122180007812752\\
574	0.00127748242167784\\
575	0.00133371885625811\\
576	0.00139081764574271\\
577	0.00146406439474163\\
578	0.00175490587469454\\
579	0.00206663004967047\\
580	0.00232104643036537\\
581	0.00257412158610118\\
582	0.00277111122536755\\
583	0.00287961411679738\\
584	0.00297466536060553\\
585	0.00306490423400819\\
586	0.00315466992901367\\
587	0.00324537223361411\\
588	0.0033377976514749\\
589	0.00343211111647921\\
590	0.00352885108616147\\
591	0.00362859107064115\\
592	0.00373286830485375\\
593	0.00384555904940007\\
594	0.00397672044113206\\
595	0.00415280575412962\\
596	0.00444436306303141\\
597	0.00503983077166121\\
598	0.00644286460810295\\
599	0\\
600	0\\
};
\addplot [color=red!40!mycolor19,solid,forget plot]
  table[row sep=crcr]{%
1	0\\
2	0\\
3	0\\
4	0\\
5	0\\
6	0\\
7	0\\
8	0\\
9	0\\
10	0\\
11	0\\
12	0\\
13	0\\
14	0\\
15	0\\
16	0\\
17	0\\
18	0\\
19	0\\
20	0\\
21	0\\
22	0\\
23	0\\
24	0\\
25	0\\
26	0\\
27	0\\
28	0\\
29	0\\
30	0\\
31	0\\
32	0\\
33	0\\
34	0\\
35	0\\
36	0\\
37	0\\
38	0\\
39	0\\
40	0\\
41	0\\
42	0\\
43	0\\
44	0\\
45	0\\
46	0\\
47	0\\
48	0\\
49	0\\
50	0\\
51	0\\
52	0\\
53	0\\
54	0\\
55	0\\
56	0\\
57	0\\
58	0\\
59	0\\
60	0\\
61	0\\
62	0\\
63	0\\
64	0\\
65	0\\
66	0\\
67	0\\
68	0\\
69	0\\
70	0\\
71	0\\
72	0\\
73	0\\
74	0\\
75	0\\
76	0\\
77	0\\
78	0\\
79	0\\
80	0\\
81	0\\
82	0\\
83	0\\
84	0\\
85	0\\
86	0\\
87	0\\
88	0\\
89	0\\
90	0\\
91	0\\
92	0\\
93	0\\
94	0\\
95	0\\
96	0\\
97	0\\
98	0\\
99	0\\
100	0\\
101	0\\
102	0\\
103	0\\
104	0\\
105	0\\
106	0\\
107	0\\
108	0\\
109	0\\
110	0\\
111	0\\
112	0\\
113	0\\
114	0\\
115	0\\
116	0\\
117	0\\
118	0\\
119	0\\
120	0\\
121	0\\
122	0\\
123	0\\
124	0\\
125	0\\
126	0\\
127	0\\
128	0\\
129	0\\
130	0\\
131	0\\
132	0\\
133	0\\
134	0\\
135	0\\
136	0\\
137	0\\
138	0\\
139	0\\
140	0\\
141	0\\
142	0\\
143	0\\
144	0\\
145	0\\
146	0\\
147	0\\
148	0\\
149	0\\
150	0\\
151	0\\
152	0\\
153	0\\
154	0\\
155	0\\
156	0\\
157	0\\
158	0\\
159	0\\
160	0\\
161	0\\
162	0\\
163	0\\
164	0\\
165	0\\
166	0\\
167	0\\
168	0\\
169	0\\
170	0\\
171	0\\
172	0\\
173	0\\
174	0\\
175	0\\
176	0\\
177	0\\
178	0\\
179	0\\
180	0\\
181	0\\
182	0\\
183	0\\
184	0\\
185	0\\
186	0\\
187	0\\
188	0\\
189	0\\
190	0\\
191	0\\
192	0\\
193	0\\
194	0\\
195	0\\
196	0\\
197	0\\
198	0\\
199	0\\
200	0\\
201	0\\
202	0\\
203	0\\
204	0\\
205	0\\
206	0\\
207	0\\
208	0\\
209	0\\
210	0\\
211	0\\
212	0\\
213	0\\
214	0\\
215	0\\
216	0\\
217	0\\
218	0\\
219	0\\
220	0\\
221	0\\
222	0\\
223	0\\
224	0\\
225	0\\
226	0\\
227	0\\
228	0\\
229	0\\
230	0\\
231	0\\
232	0\\
233	0\\
234	0\\
235	0\\
236	0\\
237	0\\
238	0\\
239	0\\
240	0\\
241	0\\
242	0\\
243	0\\
244	0\\
245	0\\
246	0\\
247	0\\
248	0\\
249	0\\
250	0\\
251	0\\
252	0\\
253	0\\
254	0\\
255	0\\
256	0\\
257	0\\
258	0\\
259	0\\
260	0\\
261	0\\
262	0\\
263	0\\
264	0\\
265	0\\
266	0\\
267	0\\
268	0\\
269	0\\
270	0\\
271	0\\
272	0\\
273	0\\
274	0\\
275	0\\
276	0\\
277	0\\
278	0\\
279	0\\
280	0\\
281	0\\
282	0\\
283	0\\
284	0\\
285	0\\
286	0\\
287	0\\
288	0\\
289	0\\
290	0\\
291	0\\
292	0\\
293	0\\
294	0\\
295	0\\
296	0\\
297	0\\
298	0\\
299	0\\
300	0\\
301	0\\
302	0\\
303	0\\
304	0\\
305	0\\
306	0\\
307	0\\
308	0\\
309	0\\
310	0\\
311	0\\
312	0\\
313	0\\
314	0\\
315	0\\
316	0\\
317	0\\
318	0\\
319	0\\
320	0\\
321	0\\
322	0\\
323	0\\
324	0\\
325	0\\
326	0\\
327	0\\
328	0\\
329	0\\
330	0\\
331	0\\
332	0\\
333	0\\
334	0\\
335	0\\
336	0\\
337	0\\
338	0\\
339	0\\
340	0\\
341	0\\
342	0\\
343	0\\
344	0\\
345	0\\
346	0\\
347	0\\
348	0\\
349	0\\
350	0\\
351	0\\
352	0\\
353	0\\
354	0\\
355	0\\
356	0\\
357	0\\
358	0\\
359	0\\
360	0\\
361	0\\
362	0\\
363	0\\
364	0\\
365	0\\
366	0\\
367	0\\
368	0\\
369	0\\
370	0\\
371	0\\
372	0\\
373	0\\
374	0\\
375	0\\
376	0\\
377	0\\
378	0\\
379	0\\
380	0\\
381	0\\
382	0\\
383	0\\
384	0\\
385	0\\
386	0\\
387	0\\
388	0\\
389	0\\
390	0\\
391	0\\
392	0\\
393	0\\
394	0\\
395	0\\
396	0\\
397	0\\
398	0\\
399	0\\
400	0\\
401	0\\
402	0\\
403	0\\
404	0\\
405	0\\
406	0\\
407	0\\
408	0\\
409	0\\
410	0\\
411	0\\
412	0\\
413	0\\
414	0\\
415	0\\
416	0\\
417	0\\
418	0\\
419	0\\
420	0\\
421	0\\
422	0\\
423	0\\
424	0\\
425	0\\
426	0\\
427	0\\
428	0\\
429	0\\
430	0\\
431	0\\
432	0\\
433	0\\
434	0\\
435	0\\
436	0\\
437	0\\
438	0\\
439	0\\
440	0\\
441	0\\
442	0\\
443	0\\
444	0\\
445	0\\
446	0\\
447	0\\
448	0\\
449	0\\
450	0\\
451	0\\
452	0\\
453	0\\
454	0\\
455	0\\
456	0\\
457	0\\
458	0\\
459	0\\
460	0\\
461	0\\
462	0\\
463	0\\
464	0\\
465	0\\
466	0\\
467	0\\
468	0\\
469	0\\
470	0\\
471	0\\
472	0\\
473	0\\
474	0\\
475	0\\
476	0\\
477	0\\
478	0\\
479	0\\
480	0\\
481	0\\
482	0\\
483	0\\
484	0\\
485	0\\
486	0\\
487	0\\
488	0\\
489	0\\
490	0\\
491	0\\
492	0\\
493	0\\
494	0\\
495	0\\
496	0\\
497	0\\
498	0\\
499	0\\
500	0\\
501	0\\
502	0\\
503	0\\
504	0\\
505	0\\
506	0\\
507	0\\
508	0\\
509	0\\
510	0\\
511	0\\
512	0\\
513	0\\
514	0\\
515	0\\
516	0\\
517	0\\
518	0\\
519	0\\
520	0\\
521	0\\
522	0\\
523	0\\
524	0\\
525	0\\
526	0\\
527	0\\
528	0\\
529	0\\
530	0\\
531	0\\
532	0\\
533	0\\
534	0\\
535	0\\
536	0\\
537	0\\
538	0\\
539	0\\
540	0\\
541	2.25145755298818e-05\\
542	4.88271536578352e-05\\
543	7.57244088362082e-05\\
544	0.000103224216248849\\
545	0.000131344948204077\\
546	0.000160105482906932\\
547	0.000189525285251365\\
548	0.000219636184945385\\
549	0.00025046323928381\\
550	0.000282024413180058\\
551	0.000314338152631727\\
552	0.000347423417862681\\
553	0.000381299724883977\\
554	0.000415987720097952\\
555	0.000451508665813917\\
556	0.000487884480583933\\
557	0.000525184470262948\\
558	0.000563348240308322\\
559	0.000602398540839097\\
560	0.00064236432155272\\
561	0.00068325304476226\\
562	0.000725110760189988\\
563	0.000767980392286332\\
564	0.000812039238838757\\
565	0.000857143872510442\\
566	0.000903200365809114\\
567	0.000950252013351359\\
568	0.00099832220385339\\
569	0.0010475654496909\\
570	0.00109902998654938\\
571	0.00115110526262422\\
572	0.00120382222347614\\
573	0.00125713529321454\\
574	0.0013114218600033\\
575	0.00138416572299911\\
576	0.00168362914241537\\
577	0.00200121253368579\\
578	0.00225018105728763\\
579	0.00250573441948143\\
580	0.00263216434825482\\
581	0.00272775687042408\\
582	0.00281259168714533\\
583	0.00289684044792754\\
584	0.00298182678347091\\
585	0.00306822904379038\\
586	0.00315632033977421\\
587	0.00324628153905063\\
588	0.00333822811831753\\
589	0.00343235311897413\\
590	0.00352893796606579\\
591	0.00362860835901882\\
592	0.00373286830485375\\
593	0.00384555904940006\\
594	0.00397672044113206\\
595	0.00415280575412962\\
596	0.00444436306303141\\
597	0.00503983077166121\\
598	0.00644286460810295\\
599	0\\
600	0\\
};
\addplot [color=red!75!mycolor17,solid,forget plot]
  table[row sep=crcr]{%
1	0\\
2	0\\
3	0\\
4	0\\
5	0\\
6	0\\
7	0\\
8	0\\
9	0\\
10	0\\
11	0\\
12	0\\
13	0\\
14	0\\
15	0\\
16	0\\
17	0\\
18	0\\
19	0\\
20	0\\
21	0\\
22	0\\
23	0\\
24	0\\
25	0\\
26	0\\
27	0\\
28	0\\
29	0\\
30	0\\
31	0\\
32	0\\
33	0\\
34	0\\
35	0\\
36	0\\
37	0\\
38	0\\
39	0\\
40	0\\
41	0\\
42	0\\
43	0\\
44	0\\
45	0\\
46	0\\
47	0\\
48	0\\
49	0\\
50	0\\
51	0\\
52	0\\
53	0\\
54	0\\
55	0\\
56	0\\
57	0\\
58	0\\
59	0\\
60	0\\
61	0\\
62	0\\
63	0\\
64	0\\
65	0\\
66	0\\
67	0\\
68	0\\
69	0\\
70	0\\
71	0\\
72	0\\
73	0\\
74	0\\
75	0\\
76	0\\
77	0\\
78	0\\
79	0\\
80	0\\
81	0\\
82	0\\
83	0\\
84	0\\
85	0\\
86	0\\
87	0\\
88	0\\
89	0\\
90	0\\
91	0\\
92	0\\
93	0\\
94	0\\
95	0\\
96	0\\
97	0\\
98	0\\
99	0\\
100	0\\
101	0\\
102	0\\
103	0\\
104	0\\
105	0\\
106	0\\
107	0\\
108	0\\
109	0\\
110	0\\
111	0\\
112	0\\
113	0\\
114	0\\
115	0\\
116	0\\
117	0\\
118	0\\
119	0\\
120	0\\
121	0\\
122	0\\
123	0\\
124	0\\
125	0\\
126	0\\
127	0\\
128	0\\
129	0\\
130	0\\
131	0\\
132	0\\
133	0\\
134	0\\
135	0\\
136	0\\
137	0\\
138	0\\
139	0\\
140	0\\
141	0\\
142	0\\
143	0\\
144	0\\
145	0\\
146	0\\
147	0\\
148	0\\
149	0\\
150	0\\
151	0\\
152	0\\
153	0\\
154	0\\
155	0\\
156	0\\
157	0\\
158	0\\
159	0\\
160	0\\
161	0\\
162	0\\
163	0\\
164	0\\
165	0\\
166	0\\
167	0\\
168	0\\
169	0\\
170	0\\
171	0\\
172	0\\
173	0\\
174	0\\
175	0\\
176	0\\
177	0\\
178	0\\
179	0\\
180	0\\
181	0\\
182	0\\
183	0\\
184	0\\
185	0\\
186	0\\
187	0\\
188	0\\
189	0\\
190	0\\
191	0\\
192	0\\
193	0\\
194	0\\
195	0\\
196	0\\
197	0\\
198	0\\
199	0\\
200	0\\
201	0\\
202	0\\
203	0\\
204	0\\
205	0\\
206	0\\
207	0\\
208	0\\
209	0\\
210	0\\
211	0\\
212	0\\
213	0\\
214	0\\
215	0\\
216	0\\
217	0\\
218	0\\
219	0\\
220	0\\
221	0\\
222	0\\
223	0\\
224	0\\
225	0\\
226	0\\
227	0\\
228	0\\
229	0\\
230	0\\
231	0\\
232	0\\
233	0\\
234	0\\
235	0\\
236	0\\
237	0\\
238	0\\
239	0\\
240	0\\
241	0\\
242	0\\
243	0\\
244	0\\
245	0\\
246	0\\
247	0\\
248	0\\
249	0\\
250	0\\
251	0\\
252	0\\
253	0\\
254	0\\
255	0\\
256	0\\
257	0\\
258	0\\
259	0\\
260	0\\
261	0\\
262	0\\
263	0\\
264	0\\
265	0\\
266	0\\
267	0\\
268	0\\
269	0\\
270	0\\
271	0\\
272	0\\
273	0\\
274	0\\
275	0\\
276	0\\
277	0\\
278	0\\
279	0\\
280	0\\
281	0\\
282	0\\
283	0\\
284	0\\
285	0\\
286	0\\
287	0\\
288	0\\
289	0\\
290	0\\
291	0\\
292	0\\
293	0\\
294	0\\
295	0\\
296	0\\
297	0\\
298	0\\
299	0\\
300	0\\
301	0\\
302	0\\
303	0\\
304	0\\
305	0\\
306	0\\
307	0\\
308	0\\
309	0\\
310	0\\
311	0\\
312	0\\
313	0\\
314	0\\
315	0\\
316	0\\
317	0\\
318	0\\
319	0\\
320	0\\
321	0\\
322	0\\
323	0\\
324	0\\
325	0\\
326	0\\
327	0\\
328	0\\
329	0\\
330	0\\
331	0\\
332	0\\
333	0\\
334	0\\
335	0\\
336	0\\
337	0\\
338	0\\
339	0\\
340	0\\
341	0\\
342	0\\
343	0\\
344	0\\
345	0\\
346	0\\
347	0\\
348	0\\
349	0\\
350	0\\
351	0\\
352	0\\
353	0\\
354	0\\
355	0\\
356	0\\
357	0\\
358	0\\
359	0\\
360	0\\
361	0\\
362	0\\
363	0\\
364	0\\
365	0\\
366	0\\
367	0\\
368	0\\
369	0\\
370	0\\
371	0\\
372	0\\
373	0\\
374	0\\
375	0\\
376	0\\
377	0\\
378	0\\
379	0\\
380	0\\
381	0\\
382	0\\
383	0\\
384	0\\
385	0\\
386	0\\
387	0\\
388	0\\
389	0\\
390	0\\
391	0\\
392	0\\
393	0\\
394	0\\
395	0\\
396	0\\
397	0\\
398	0\\
399	0\\
400	0\\
401	0\\
402	0\\
403	0\\
404	0\\
405	0\\
406	0\\
407	0\\
408	0\\
409	0\\
410	0\\
411	0\\
412	0\\
413	0\\
414	0\\
415	0\\
416	0\\
417	0\\
418	0\\
419	0\\
420	0\\
421	0\\
422	0\\
423	0\\
424	0\\
425	0\\
426	0\\
427	0\\
428	0\\
429	0\\
430	0\\
431	0\\
432	0\\
433	0\\
434	0\\
435	0\\
436	0\\
437	0\\
438	0\\
439	0\\
440	0\\
441	0\\
442	0\\
443	0\\
444	0\\
445	0\\
446	0\\
447	0\\
448	0\\
449	0\\
450	0\\
451	0\\
452	0\\
453	0\\
454	0\\
455	0\\
456	0\\
457	0\\
458	0\\
459	0\\
460	0\\
461	0\\
462	0\\
463	0\\
464	0\\
465	0\\
466	0\\
467	0\\
468	0\\
469	0\\
470	0\\
471	0\\
472	0\\
473	0\\
474	0\\
475	0\\
476	0\\
477	0\\
478	0\\
479	0\\
480	0\\
481	0\\
482	0\\
483	0\\
484	0\\
485	0\\
486	0\\
487	0\\
488	0\\
489	0\\
490	0\\
491	0\\
492	0\\
493	0\\
494	0\\
495	0\\
496	0\\
497	0\\
498	0\\
499	0\\
500	0\\
501	0\\
502	0\\
503	0\\
504	0\\
505	0\\
506	0\\
507	0\\
508	0\\
509	0\\
510	0\\
511	0\\
512	0\\
513	0\\
514	0\\
515	0\\
516	0\\
517	0\\
518	0\\
519	0\\
520	0\\
521	0\\
522	0\\
523	0\\
524	0\\
525	0\\
526	0\\
527	0\\
528	0\\
529	0\\
530	0\\
531	0\\
532	0\\
533	0\\
534	0\\
535	0\\
536	0\\
537	0\\
538	0\\
539	0\\
540	1.80572611580464e-05\\
541	4.4118006200525e-05\\
542	7.07542032362198e-05\\
543	9.79828118109392e-05\\
544	0.000125821262619649\\
545	0.000154287471883814\\
546	0.000183399616108164\\
547	0.000213189224881146\\
548	0.000243680143284574\\
549	0.00027488924859386\\
550	0.000306834372255217\\
551	0.000339533902369266\\
552	0.00037300762948773\\
553	0.000407321246047958\\
554	0.000442407509995026\\
555	0.000478288527401575\\
556	0.000514990430784664\\
557	0.000552506247906185\\
558	0.000590897395376276\\
559	0.000630191158809039\\
560	0.000670413411890405\\
561	0.000711735510261396\\
562	0.00075401250105808\\
563	0.000797191808578931\\
564	0.000841224772035708\\
565	0.000886213838323151\\
566	0.000932258739848622\\
567	0.000979764676566103\\
568	0.00102876658617279\\
569	0.00107851644739995\\
570	0.00112845627858277\\
571	0.00117925389621404\\
572	0.00123071645413494\\
573	0.00128292972718838\\
574	0.00156765047686712\\
575	0.00189256888265377\\
576	0.00214409180946553\\
577	0.00239107643868339\\
578	0.00249057912042086\\
579	0.00257377822812136\\
580	0.00265393568636939\\
581	0.00273394370368479\\
582	0.00281520413723953\\
583	0.00289795905152283\\
584	0.00298235807749845\\
585	0.00306849821799598\\
586	0.00315646723333688\\
587	0.003246352526171\\
588	0.00333826624313612\\
589	0.00343236642273867\\
590	0.00352894054086423\\
591	0.00362860835901881\\
592	0.00373286830485375\\
593	0.00384555904940006\\
594	0.00397672044113205\\
595	0.00415280575412962\\
596	0.00444436306303141\\
597	0.00503983077166121\\
598	0.00644286460810295\\
599	0\\
600	0\\
};
\addplot [color=red!80!mycolor19,solid,forget plot]
  table[row sep=crcr]{%
1	0\\
2	0\\
3	0\\
4	0\\
5	0\\
6	0\\
7	0\\
8	0\\
9	0\\
10	0\\
11	0\\
12	0\\
13	0\\
14	0\\
15	0\\
16	0\\
17	0\\
18	0\\
19	0\\
20	0\\
21	0\\
22	0\\
23	0\\
24	0\\
25	0\\
26	0\\
27	0\\
28	0\\
29	0\\
30	0\\
31	0\\
32	0\\
33	0\\
34	0\\
35	0\\
36	0\\
37	0\\
38	0\\
39	0\\
40	0\\
41	0\\
42	0\\
43	0\\
44	0\\
45	0\\
46	0\\
47	0\\
48	0\\
49	0\\
50	0\\
51	0\\
52	0\\
53	0\\
54	0\\
55	0\\
56	0\\
57	0\\
58	0\\
59	0\\
60	0\\
61	0\\
62	0\\
63	0\\
64	0\\
65	0\\
66	0\\
67	0\\
68	0\\
69	0\\
70	0\\
71	0\\
72	0\\
73	0\\
74	0\\
75	0\\
76	0\\
77	0\\
78	0\\
79	0\\
80	0\\
81	0\\
82	0\\
83	0\\
84	0\\
85	0\\
86	0\\
87	0\\
88	0\\
89	0\\
90	0\\
91	0\\
92	0\\
93	0\\
94	0\\
95	0\\
96	0\\
97	0\\
98	0\\
99	0\\
100	0\\
101	0\\
102	0\\
103	0\\
104	0\\
105	0\\
106	0\\
107	0\\
108	0\\
109	0\\
110	0\\
111	0\\
112	0\\
113	0\\
114	0\\
115	0\\
116	0\\
117	0\\
118	0\\
119	0\\
120	0\\
121	0\\
122	0\\
123	0\\
124	0\\
125	0\\
126	0\\
127	0\\
128	0\\
129	0\\
130	0\\
131	0\\
132	0\\
133	0\\
134	0\\
135	0\\
136	0\\
137	0\\
138	0\\
139	0\\
140	0\\
141	0\\
142	0\\
143	0\\
144	0\\
145	0\\
146	0\\
147	0\\
148	0\\
149	0\\
150	0\\
151	0\\
152	0\\
153	0\\
154	0\\
155	0\\
156	0\\
157	0\\
158	0\\
159	0\\
160	0\\
161	0\\
162	0\\
163	0\\
164	0\\
165	0\\
166	0\\
167	0\\
168	0\\
169	0\\
170	0\\
171	0\\
172	0\\
173	0\\
174	0\\
175	0\\
176	0\\
177	0\\
178	0\\
179	0\\
180	0\\
181	0\\
182	0\\
183	0\\
184	0\\
185	0\\
186	0\\
187	0\\
188	0\\
189	0\\
190	0\\
191	0\\
192	0\\
193	0\\
194	0\\
195	0\\
196	0\\
197	0\\
198	0\\
199	0\\
200	0\\
201	0\\
202	0\\
203	0\\
204	0\\
205	0\\
206	0\\
207	0\\
208	0\\
209	0\\
210	0\\
211	0\\
212	0\\
213	0\\
214	0\\
215	0\\
216	0\\
217	0\\
218	0\\
219	0\\
220	0\\
221	0\\
222	0\\
223	0\\
224	0\\
225	0\\
226	0\\
227	0\\
228	0\\
229	0\\
230	0\\
231	0\\
232	0\\
233	0\\
234	0\\
235	0\\
236	0\\
237	0\\
238	0\\
239	0\\
240	0\\
241	0\\
242	0\\
243	0\\
244	0\\
245	0\\
246	0\\
247	0\\
248	0\\
249	0\\
250	0\\
251	0\\
252	0\\
253	0\\
254	0\\
255	0\\
256	0\\
257	0\\
258	0\\
259	0\\
260	0\\
261	0\\
262	0\\
263	0\\
264	0\\
265	0\\
266	0\\
267	0\\
268	0\\
269	0\\
270	0\\
271	0\\
272	0\\
273	0\\
274	0\\
275	0\\
276	0\\
277	0\\
278	0\\
279	0\\
280	0\\
281	0\\
282	0\\
283	0\\
284	0\\
285	0\\
286	0\\
287	0\\
288	0\\
289	0\\
290	0\\
291	0\\
292	0\\
293	0\\
294	0\\
295	0\\
296	0\\
297	0\\
298	0\\
299	0\\
300	0\\
301	0\\
302	0\\
303	0\\
304	0\\
305	0\\
306	0\\
307	0\\
308	0\\
309	0\\
310	0\\
311	0\\
312	0\\
313	0\\
314	0\\
315	0\\
316	0\\
317	0\\
318	0\\
319	0\\
320	0\\
321	0\\
322	0\\
323	0\\
324	0\\
325	0\\
326	0\\
327	0\\
328	0\\
329	0\\
330	0\\
331	0\\
332	0\\
333	0\\
334	0\\
335	0\\
336	0\\
337	0\\
338	0\\
339	0\\
340	0\\
341	0\\
342	0\\
343	0\\
344	0\\
345	0\\
346	0\\
347	0\\
348	0\\
349	0\\
350	0\\
351	0\\
352	0\\
353	0\\
354	0\\
355	0\\
356	0\\
357	0\\
358	0\\
359	0\\
360	0\\
361	0\\
362	0\\
363	0\\
364	0\\
365	0\\
366	0\\
367	0\\
368	0\\
369	0\\
370	0\\
371	0\\
372	0\\
373	0\\
374	0\\
375	0\\
376	0\\
377	0\\
378	0\\
379	0\\
380	0\\
381	0\\
382	0\\
383	0\\
384	0\\
385	0\\
386	0\\
387	0\\
388	0\\
389	0\\
390	0\\
391	0\\
392	0\\
393	0\\
394	0\\
395	0\\
396	0\\
397	0\\
398	0\\
399	0\\
400	0\\
401	0\\
402	0\\
403	0\\
404	0\\
405	0\\
406	0\\
407	0\\
408	0\\
409	0\\
410	0\\
411	0\\
412	0\\
413	0\\
414	0\\
415	0\\
416	0\\
417	0\\
418	0\\
419	0\\
420	0\\
421	0\\
422	0\\
423	0\\
424	0\\
425	0\\
426	0\\
427	0\\
428	0\\
429	0\\
430	0\\
431	0\\
432	0\\
433	0\\
434	0\\
435	0\\
436	0\\
437	0\\
438	0\\
439	0\\
440	0\\
441	0\\
442	0\\
443	0\\
444	0\\
445	0\\
446	0\\
447	0\\
448	0\\
449	0\\
450	0\\
451	0\\
452	0\\
453	0\\
454	0\\
455	0\\
456	0\\
457	0\\
458	0\\
459	0\\
460	0\\
461	0\\
462	0\\
463	0\\
464	0\\
465	0\\
466	0\\
467	0\\
468	0\\
469	0\\
470	0\\
471	0\\
472	0\\
473	0\\
474	0\\
475	0\\
476	0\\
477	0\\
478	0\\
479	0\\
480	0\\
481	0\\
482	0\\
483	0\\
484	0\\
485	0\\
486	0\\
487	0\\
488	0\\
489	0\\
490	0\\
491	0\\
492	0\\
493	0\\
494	0\\
495	0\\
496	0\\
497	0\\
498	0\\
499	0\\
500	0\\
501	0\\
502	0\\
503	0\\
504	0\\
505	0\\
506	0\\
507	0\\
508	0\\
509	0\\
510	0\\
511	0\\
512	0\\
513	0\\
514	0\\
515	0\\
516	0\\
517	0\\
518	0\\
519	0\\
520	0\\
521	0\\
522	0\\
523	0\\
524	0\\
525	0\\
526	0\\
527	0\\
528	0\\
529	0\\
530	0\\
531	0\\
532	0\\
533	0\\
534	0\\
535	0\\
536	0\\
537	0\\
538	0\\
539	1.09971123564833e-05\\
540	3.67777687958609e-05\\
541	6.31213625684237e-05\\
542	9.00441133591266e-05\\
543	0.00011756267562623\\
544	0.000145694153413752\\
545	0.000174455739705744\\
546	0.000203878155683555\\
547	0.00023398579118046\\
548	0.000264795179737243\\
549	0.000296366118124326\\
550	0.000328644365716624\\
551	0.000361637574244406\\
552	0.000395368372265394\\
553	0.000429826296342454\\
554	0.000465072355643675\\
555	0.00050112828782099\\
556	0.000538014176942687\\
557	0.00057573329173453\\
558	0.000614484468182088\\
559	0.000654187719273962\\
560	0.000694706083181759\\
561	0.00073598111257611\\
562	0.00077813831988304\\
563	0.000821236596718493\\
564	0.000865236156171182\\
565	0.000910969534659842\\
566	0.000957920024337713\\
567	0.00100529101218676\\
568	0.0010530258921727\\
569	0.00110148835858687\\
570	0.00115017727882363\\
571	0.00120010163547213\\
572	0.00140776483562387\\
573	0.00174519063704481\\
574	0.00200611363934863\\
575	0.00225233627854361\\
576	0.00234714808882544\\
577	0.00242410810580428\\
578	0.00250007651960148\\
579	0.00257684337440823\\
580	0.00265485410432229\\
581	0.00273433854836808\\
582	0.00281537804625545\\
583	0.0028980434054312\\
584	0.00298240150095583\\
585	0.00306852173201496\\
586	0.00315647873661191\\
587	0.00324635845014596\\
588	0.00333826825403972\\
589	0.00343236680134408\\
590	0.00352894054086423\\
591	0.00362860835901881\\
592	0.00373286830485375\\
593	0.00384555904940006\\
594	0.00397672044113206\\
595	0.00415280575412962\\
596	0.00444436306303141\\
597	0.00503983077166121\\
598	0.00644286460810295\\
599	0\\
600	0\\
};
\addplot [color=red,solid,forget plot]
  table[row sep=crcr]{%
1	0\\
2	0\\
3	0\\
4	0\\
5	0\\
6	0\\
7	0\\
8	0\\
9	0\\
10	0\\
11	0\\
12	0\\
13	0\\
14	0\\
15	0\\
16	0\\
17	0\\
18	0\\
19	0\\
20	0\\
21	0\\
22	0\\
23	0\\
24	0\\
25	0\\
26	0\\
27	0\\
28	0\\
29	0\\
30	0\\
31	0\\
32	0\\
33	0\\
34	0\\
35	0\\
36	0\\
37	0\\
38	0\\
39	0\\
40	0\\
41	0\\
42	0\\
43	0\\
44	0\\
45	0\\
46	0\\
47	0\\
48	0\\
49	0\\
50	0\\
51	0\\
52	0\\
53	0\\
54	0\\
55	0\\
56	0\\
57	0\\
58	0\\
59	0\\
60	0\\
61	0\\
62	0\\
63	0\\
64	0\\
65	0\\
66	0\\
67	0\\
68	0\\
69	0\\
70	0\\
71	0\\
72	0\\
73	0\\
74	0\\
75	0\\
76	0\\
77	0\\
78	0\\
79	0\\
80	0\\
81	0\\
82	0\\
83	0\\
84	0\\
85	0\\
86	0\\
87	0\\
88	0\\
89	0\\
90	0\\
91	0\\
92	0\\
93	0\\
94	0\\
95	0\\
96	0\\
97	0\\
98	0\\
99	0\\
100	0\\
101	0\\
102	0\\
103	0\\
104	0\\
105	0\\
106	0\\
107	0\\
108	0\\
109	0\\
110	0\\
111	0\\
112	0\\
113	0\\
114	0\\
115	0\\
116	0\\
117	0\\
118	0\\
119	0\\
120	0\\
121	0\\
122	0\\
123	0\\
124	0\\
125	0\\
126	0\\
127	0\\
128	0\\
129	0\\
130	0\\
131	0\\
132	0\\
133	0\\
134	0\\
135	0\\
136	0\\
137	0\\
138	0\\
139	0\\
140	0\\
141	0\\
142	0\\
143	0\\
144	0\\
145	0\\
146	0\\
147	0\\
148	0\\
149	0\\
150	0\\
151	0\\
152	0\\
153	0\\
154	0\\
155	0\\
156	0\\
157	0\\
158	0\\
159	0\\
160	0\\
161	0\\
162	0\\
163	0\\
164	0\\
165	0\\
166	0\\
167	0\\
168	0\\
169	0\\
170	0\\
171	0\\
172	0\\
173	0\\
174	0\\
175	0\\
176	0\\
177	0\\
178	0\\
179	0\\
180	0\\
181	0\\
182	0\\
183	0\\
184	0\\
185	0\\
186	0\\
187	0\\
188	0\\
189	0\\
190	0\\
191	0\\
192	0\\
193	0\\
194	0\\
195	0\\
196	0\\
197	0\\
198	0\\
199	0\\
200	0\\
201	0\\
202	0\\
203	0\\
204	0\\
205	0\\
206	0\\
207	0\\
208	0\\
209	0\\
210	0\\
211	0\\
212	0\\
213	0\\
214	0\\
215	0\\
216	0\\
217	0\\
218	0\\
219	0\\
220	0\\
221	0\\
222	0\\
223	0\\
224	0\\
225	0\\
226	0\\
227	0\\
228	0\\
229	0\\
230	0\\
231	0\\
232	0\\
233	0\\
234	0\\
235	0\\
236	0\\
237	0\\
238	0\\
239	0\\
240	0\\
241	0\\
242	0\\
243	0\\
244	0\\
245	0\\
246	0\\
247	0\\
248	0\\
249	0\\
250	0\\
251	0\\
252	0\\
253	0\\
254	0\\
255	0\\
256	0\\
257	0\\
258	0\\
259	0\\
260	0\\
261	0\\
262	0\\
263	0\\
264	0\\
265	0\\
266	0\\
267	0\\
268	0\\
269	0\\
270	0\\
271	0\\
272	0\\
273	0\\
274	0\\
275	0\\
276	0\\
277	0\\
278	0\\
279	0\\
280	0\\
281	0\\
282	0\\
283	0\\
284	0\\
285	0\\
286	0\\
287	0\\
288	0\\
289	0\\
290	0\\
291	0\\
292	0\\
293	0\\
294	0\\
295	0\\
296	0\\
297	0\\
298	0\\
299	0\\
300	0\\
301	0\\
302	0\\
303	0\\
304	0\\
305	0\\
306	0\\
307	0\\
308	0\\
309	0\\
310	0\\
311	0\\
312	0\\
313	0\\
314	0\\
315	0\\
316	0\\
317	0\\
318	0\\
319	0\\
320	0\\
321	0\\
322	0\\
323	0\\
324	0\\
325	0\\
326	0\\
327	0\\
328	0\\
329	0\\
330	0\\
331	0\\
332	0\\
333	0\\
334	0\\
335	0\\
336	0\\
337	0\\
338	0\\
339	0\\
340	0\\
341	0\\
342	0\\
343	0\\
344	0\\
345	0\\
346	0\\
347	0\\
348	0\\
349	0\\
350	0\\
351	0\\
352	0\\
353	0\\
354	0\\
355	0\\
356	0\\
357	0\\
358	0\\
359	0\\
360	0\\
361	0\\
362	0\\
363	0\\
364	0\\
365	0\\
366	0\\
367	0\\
368	0\\
369	0\\
370	0\\
371	0\\
372	0\\
373	0\\
374	0\\
375	0\\
376	0\\
377	0\\
378	0\\
379	0\\
380	0\\
381	0\\
382	0\\
383	0\\
384	0\\
385	0\\
386	0\\
387	0\\
388	0\\
389	0\\
390	0\\
391	0\\
392	0\\
393	0\\
394	0\\
395	0\\
396	0\\
397	0\\
398	0\\
399	0\\
400	0\\
401	0\\
402	0\\
403	0\\
404	0\\
405	0\\
406	0\\
407	0\\
408	0\\
409	0\\
410	0\\
411	0\\
412	0\\
413	0\\
414	0\\
415	0\\
416	0\\
417	0\\
418	0\\
419	0\\
420	0\\
421	0\\
422	0\\
423	0\\
424	0\\
425	0\\
426	0\\
427	0\\
428	0\\
429	0\\
430	0\\
431	0\\
432	0\\
433	0\\
434	0\\
435	0\\
436	0\\
437	0\\
438	0\\
439	0\\
440	0\\
441	0\\
442	0\\
443	0\\
444	0\\
445	0\\
446	0\\
447	0\\
448	0\\
449	0\\
450	0\\
451	0\\
452	0\\
453	0\\
454	0\\
455	0\\
456	0\\
457	0\\
458	0\\
459	0\\
460	0\\
461	0\\
462	0\\
463	0\\
464	0\\
465	0\\
466	0\\
467	0\\
468	0\\
469	0\\
470	0\\
471	0\\
472	0\\
473	0\\
474	0\\
475	0\\
476	0\\
477	0\\
478	0\\
479	0\\
480	0\\
481	0\\
482	0\\
483	0\\
484	0\\
485	0\\
486	0\\
487	0\\
488	0\\
489	0\\
490	0\\
491	0\\
492	0\\
493	0\\
494	0\\
495	0\\
496	0\\
497	0\\
498	0\\
499	0\\
500	0\\
501	0\\
502	0\\
503	0\\
504	0\\
505	0\\
506	0\\
507	0\\
508	0\\
509	0\\
510	0\\
511	0\\
512	0\\
513	0\\
514	0\\
515	0\\
516	0\\
517	0\\
518	0\\
519	0\\
520	0\\
521	0\\
522	0\\
523	0\\
524	0\\
525	0\\
526	0\\
527	0\\
528	0\\
529	0\\
530	0\\
531	0\\
532	0\\
533	0\\
534	0\\
535	0\\
536	0\\
537	0\\
538	1.87556343843139e-06\\
539	2.73381310324152e-05\\
540	5.33495288270524e-05\\
541	7.99252542932924e-05\\
542	0.000107081251424983\\
543	0.000134834453286783\\
544	0.000163201833838333\\
545	0.000192240295758037\\
546	0.000221947411190644\\
547	0.000252298954717094\\
548	0.000283315360389555\\
549	0.000314986488951994\\
550	0.000347362970500003\\
551	0.000380469167030157\\
552	0.000414321565112513\\
553	0.000448919563346855\\
554	0.00048433294388368\\
555	0.000520634170094015\\
556	0.000557948080283569\\
557	0.000596000691589419\\
558	0.000634749598647108\\
559	0.000674269636362017\\
560	0.000714637530630376\\
561	0.000755805415251623\\
562	0.000797923053763769\\
563	0.000841878133127252\\
564	0.000886913723009165\\
565	0.00093214151302975\\
566	0.000977757767295516\\
567	0.00102380313004\\
568	0.00107036709861644\\
569	0.00111805503001417\\
570	0.00120939113958143\\
571	0.00154663998556902\\
572	0.00183959615121693\\
573	0.00209756278158636\\
574	0.00220493769069792\\
575	0.00227885628634348\\
576	0.00235163153275207\\
577	0.00242542435592874\\
578	0.00250050669328025\\
579	0.00257697908889624\\
580	0.00265491355281054\\
581	0.00273436544093214\\
582	0.00281539134096267\\
583	0.00289805033361497\\
584	0.00298240522740585\\
585	0.00306852356540592\\
586	0.00315647964482259\\
587	0.0032463587502771\\
588	0.0033382683090296\\
589	0.00343236680134408\\
590	0.00352894054086422\\
591	0.00362860835901881\\
592	0.00373286830485375\\
593	0.00384555904940006\\
594	0.00397672044113206\\
595	0.00415280575412962\\
596	0.00444436306303141\\
597	0.00503983077166121\\
598	0.00644286460810295\\
599	0\\
600	0\\
};
\addplot [color=mycolor20,solid,forget plot]
  table[row sep=crcr]{%
1	0\\
2	0\\
3	0\\
4	0\\
5	0\\
6	0\\
7	0\\
8	0\\
9	0\\
10	0\\
11	0\\
12	0\\
13	0\\
14	0\\
15	0\\
16	0\\
17	0\\
18	0\\
19	0\\
20	0\\
21	0\\
22	0\\
23	0\\
24	0\\
25	0\\
26	0\\
27	0\\
28	0\\
29	0\\
30	0\\
31	0\\
32	0\\
33	0\\
34	0\\
35	0\\
36	0\\
37	0\\
38	0\\
39	0\\
40	0\\
41	0\\
42	0\\
43	0\\
44	0\\
45	0\\
46	0\\
47	0\\
48	0\\
49	0\\
50	0\\
51	0\\
52	0\\
53	0\\
54	0\\
55	0\\
56	0\\
57	0\\
58	0\\
59	0\\
60	0\\
61	0\\
62	0\\
63	0\\
64	0\\
65	0\\
66	0\\
67	0\\
68	0\\
69	0\\
70	0\\
71	0\\
72	0\\
73	0\\
74	0\\
75	0\\
76	0\\
77	0\\
78	0\\
79	0\\
80	0\\
81	0\\
82	0\\
83	0\\
84	0\\
85	0\\
86	0\\
87	0\\
88	0\\
89	0\\
90	0\\
91	0\\
92	0\\
93	0\\
94	0\\
95	0\\
96	0\\
97	0\\
98	0\\
99	0\\
100	0\\
101	0\\
102	0\\
103	0\\
104	0\\
105	0\\
106	0\\
107	0\\
108	0\\
109	0\\
110	0\\
111	0\\
112	0\\
113	0\\
114	0\\
115	0\\
116	0\\
117	0\\
118	0\\
119	0\\
120	0\\
121	0\\
122	0\\
123	0\\
124	0\\
125	0\\
126	0\\
127	0\\
128	0\\
129	0\\
130	0\\
131	0\\
132	0\\
133	0\\
134	0\\
135	0\\
136	0\\
137	0\\
138	0\\
139	0\\
140	0\\
141	0\\
142	0\\
143	0\\
144	0\\
145	0\\
146	0\\
147	0\\
148	0\\
149	0\\
150	0\\
151	0\\
152	0\\
153	0\\
154	0\\
155	0\\
156	0\\
157	0\\
158	0\\
159	0\\
160	0\\
161	0\\
162	0\\
163	0\\
164	0\\
165	0\\
166	0\\
167	0\\
168	0\\
169	0\\
170	0\\
171	0\\
172	0\\
173	0\\
174	0\\
175	0\\
176	0\\
177	0\\
178	0\\
179	0\\
180	0\\
181	0\\
182	0\\
183	0\\
184	0\\
185	0\\
186	0\\
187	0\\
188	0\\
189	0\\
190	0\\
191	0\\
192	0\\
193	0\\
194	0\\
195	0\\
196	0\\
197	0\\
198	0\\
199	0\\
200	0\\
201	0\\
202	0\\
203	0\\
204	0\\
205	0\\
206	0\\
207	0\\
208	0\\
209	0\\
210	0\\
211	0\\
212	0\\
213	0\\
214	0\\
215	0\\
216	0\\
217	0\\
218	0\\
219	0\\
220	0\\
221	0\\
222	0\\
223	0\\
224	0\\
225	0\\
226	0\\
227	0\\
228	0\\
229	0\\
230	0\\
231	0\\
232	0\\
233	0\\
234	0\\
235	0\\
236	0\\
237	0\\
238	0\\
239	0\\
240	0\\
241	0\\
242	0\\
243	0\\
244	0\\
245	0\\
246	0\\
247	0\\
248	0\\
249	0\\
250	0\\
251	0\\
252	0\\
253	0\\
254	0\\
255	0\\
256	0\\
257	0\\
258	0\\
259	0\\
260	0\\
261	0\\
262	0\\
263	0\\
264	0\\
265	0\\
266	0\\
267	0\\
268	0\\
269	0\\
270	0\\
271	0\\
272	0\\
273	0\\
274	0\\
275	0\\
276	0\\
277	0\\
278	0\\
279	0\\
280	0\\
281	0\\
282	0\\
283	0\\
284	0\\
285	0\\
286	0\\
287	0\\
288	0\\
289	0\\
290	0\\
291	0\\
292	0\\
293	0\\
294	0\\
295	0\\
296	0\\
297	0\\
298	0\\
299	0\\
300	0\\
301	0\\
302	0\\
303	0\\
304	0\\
305	0\\
306	0\\
307	0\\
308	0\\
309	0\\
310	0\\
311	0\\
312	0\\
313	0\\
314	0\\
315	0\\
316	0\\
317	0\\
318	0\\
319	0\\
320	0\\
321	0\\
322	0\\
323	0\\
324	0\\
325	0\\
326	0\\
327	0\\
328	0\\
329	0\\
330	0\\
331	0\\
332	0\\
333	0\\
334	0\\
335	0\\
336	0\\
337	0\\
338	0\\
339	0\\
340	0\\
341	0\\
342	0\\
343	0\\
344	0\\
345	0\\
346	0\\
347	0\\
348	0\\
349	0\\
350	0\\
351	0\\
352	0\\
353	0\\
354	0\\
355	0\\
356	0\\
357	0\\
358	0\\
359	0\\
360	0\\
361	0\\
362	0\\
363	0\\
364	0\\
365	0\\
366	0\\
367	0\\
368	0\\
369	0\\
370	0\\
371	0\\
372	0\\
373	0\\
374	0\\
375	0\\
376	0\\
377	0\\
378	0\\
379	0\\
380	0\\
381	0\\
382	0\\
383	0\\
384	0\\
385	0\\
386	0\\
387	0\\
388	0\\
389	0\\
390	0\\
391	0\\
392	0\\
393	0\\
394	0\\
395	0\\
396	0\\
397	0\\
398	0\\
399	0\\
400	0\\
401	0\\
402	0\\
403	0\\
404	0\\
405	0\\
406	0\\
407	0\\
408	0\\
409	0\\
410	0\\
411	0\\
412	0\\
413	0\\
414	0\\
415	0\\
416	0\\
417	0\\
418	0\\
419	0\\
420	0\\
421	0\\
422	0\\
423	0\\
424	0\\
425	0\\
426	0\\
427	0\\
428	0\\
429	0\\
430	0\\
431	0\\
432	0\\
433	0\\
434	0\\
435	0\\
436	0\\
437	0\\
438	0\\
439	0\\
440	0\\
441	0\\
442	0\\
443	0\\
444	0\\
445	0\\
446	0\\
447	0\\
448	0\\
449	0\\
450	0\\
451	0\\
452	0\\
453	0\\
454	0\\
455	0\\
456	0\\
457	0\\
458	0\\
459	0\\
460	0\\
461	0\\
462	0\\
463	0\\
464	0\\
465	0\\
466	0\\
467	0\\
468	0\\
469	0\\
470	0\\
471	0\\
472	0\\
473	0\\
474	0\\
475	0\\
476	0\\
477	0\\
478	0\\
479	0\\
480	0\\
481	0\\
482	0\\
483	0\\
484	0\\
485	0\\
486	0\\
487	0\\
488	0\\
489	0\\
490	0\\
491	0\\
492	0\\
493	0\\
494	0\\
495	0\\
496	0\\
497	0\\
498	0\\
499	0\\
500	0\\
501	0\\
502	0\\
503	0\\
504	0\\
505	0\\
506	0\\
507	0\\
508	0\\
509	0\\
510	0\\
511	0\\
512	0\\
513	0\\
514	0\\
515	0\\
516	0\\
517	0\\
518	0\\
519	0\\
520	0\\
521	0\\
522	0\\
523	0\\
524	0\\
525	0\\
526	0\\
527	0\\
528	0\\
529	0\\
530	0\\
531	0\\
532	0\\
533	0\\
534	0\\
535	0\\
536	0\\
537	0\\
538	1.6243643999888e-05\\
539	4.18918417134826e-05\\
540	6.80899130433921e-05\\
541	9.48564491413589e-05\\
542	0.000122248418487188\\
543	0.000150198053260629\\
544	0.000178739275558913\\
545	0.000207878949442831\\
546	0.000237642445840086\\
547	0.000268066066433872\\
548	0.000299164768053091\\
549	0.000330937123302737\\
550	0.000363440569240543\\
551	0.000396699448537722\\
552	0.000430733142917223\\
553	0.000465742743696337\\
554	0.000501529298393566\\
555	0.000537988092118647\\
556	0.000575079712924593\\
557	0.000612943047238721\\
558	0.000651541350800113\\
559	0.000690963850311138\\
560	0.000731349768291028\\
561	0.000773468762908814\\
562	0.000816756500267696\\
563	0.000860107264034101\\
564	0.000903803885929959\\
565	0.000947716400417519\\
566	0.000992275947024247\\
567	0.00103793439844082\\
568	0.00108481900650592\\
569	0.00130432755340724\\
570	0.00164623612034606\\
571	0.00190410168763289\\
572	0.00206499561851727\\
573	0.00213820104703062\\
574	0.00220835825523849\\
575	0.00227946097953584\\
576	0.00235181287951247\\
577	0.00242548455611651\\
578	0.0025005266540219\\
579	0.00257698800197174\\
580	0.00265491768689218\\
581	0.0027343675196084\\
582	0.00281539243421504\\
583	0.00289805091793385\\
584	0.00298240551503779\\
585	0.00306852370283625\\
586	0.00315647968906845\\
587	0.00324635875816935\\
588	0.00333826830902959\\
589	0.00343236680134408\\
590	0.00352894054086422\\
591	0.00362860835901881\\
592	0.00373286830485375\\
593	0.00384555904940006\\
594	0.00397672044113206\\
595	0.00415280575412961\\
596	0.00444436306303141\\
597	0.00503983077166121\\
598	0.00644286460810295\\
599	0\\
600	0\\
};
\addplot [color=mycolor21,solid,forget plot]
  table[row sep=crcr]{%
1	0\\
2	0\\
3	0\\
4	0\\
5	0\\
6	0\\
7	0\\
8	0\\
9	0\\
10	0\\
11	0\\
12	0\\
13	0\\
14	0\\
15	0\\
16	0\\
17	0\\
18	0\\
19	0\\
20	0\\
21	0\\
22	0\\
23	0\\
24	0\\
25	0\\
26	0\\
27	0\\
28	0\\
29	0\\
30	0\\
31	0\\
32	0\\
33	0\\
34	0\\
35	0\\
36	0\\
37	0\\
38	0\\
39	0\\
40	0\\
41	0\\
42	0\\
43	0\\
44	0\\
45	0\\
46	0\\
47	0\\
48	0\\
49	0\\
50	0\\
51	0\\
52	0\\
53	0\\
54	0\\
55	0\\
56	0\\
57	0\\
58	0\\
59	0\\
60	0\\
61	0\\
62	0\\
63	0\\
64	0\\
65	0\\
66	0\\
67	0\\
68	0\\
69	0\\
70	0\\
71	0\\
72	0\\
73	0\\
74	0\\
75	0\\
76	0\\
77	0\\
78	0\\
79	0\\
80	0\\
81	0\\
82	0\\
83	0\\
84	0\\
85	0\\
86	0\\
87	0\\
88	0\\
89	0\\
90	0\\
91	0\\
92	0\\
93	0\\
94	0\\
95	0\\
96	0\\
97	0\\
98	0\\
99	0\\
100	0\\
101	0\\
102	0\\
103	0\\
104	0\\
105	0\\
106	0\\
107	0\\
108	0\\
109	0\\
110	0\\
111	0\\
112	0\\
113	0\\
114	0\\
115	0\\
116	0\\
117	0\\
118	0\\
119	0\\
120	0\\
121	0\\
122	0\\
123	0\\
124	0\\
125	0\\
126	0\\
127	0\\
128	0\\
129	0\\
130	0\\
131	0\\
132	0\\
133	0\\
134	0\\
135	0\\
136	0\\
137	0\\
138	0\\
139	0\\
140	0\\
141	0\\
142	0\\
143	0\\
144	0\\
145	0\\
146	0\\
147	0\\
148	0\\
149	0\\
150	0\\
151	0\\
152	0\\
153	0\\
154	0\\
155	0\\
156	0\\
157	0\\
158	0\\
159	0\\
160	0\\
161	0\\
162	0\\
163	0\\
164	0\\
165	0\\
166	0\\
167	0\\
168	0\\
169	0\\
170	0\\
171	0\\
172	0\\
173	0\\
174	0\\
175	0\\
176	0\\
177	0\\
178	0\\
179	0\\
180	0\\
181	0\\
182	0\\
183	0\\
184	0\\
185	0\\
186	0\\
187	0\\
188	0\\
189	0\\
190	0\\
191	0\\
192	0\\
193	0\\
194	0\\
195	0\\
196	0\\
197	0\\
198	0\\
199	0\\
200	0\\
201	0\\
202	0\\
203	0\\
204	0\\
205	0\\
206	0\\
207	0\\
208	0\\
209	0\\
210	0\\
211	0\\
212	0\\
213	0\\
214	0\\
215	0\\
216	0\\
217	0\\
218	0\\
219	0\\
220	0\\
221	0\\
222	0\\
223	0\\
224	0\\
225	0\\
226	0\\
227	0\\
228	0\\
229	0\\
230	0\\
231	0\\
232	0\\
233	0\\
234	0\\
235	0\\
236	0\\
237	0\\
238	0\\
239	0\\
240	0\\
241	0\\
242	0\\
243	0\\
244	0\\
245	0\\
246	0\\
247	0\\
248	0\\
249	0\\
250	0\\
251	0\\
252	0\\
253	0\\
254	0\\
255	0\\
256	0\\
257	0\\
258	0\\
259	0\\
260	0\\
261	0\\
262	0\\
263	0\\
264	0\\
265	0\\
266	0\\
267	0\\
268	0\\
269	0\\
270	0\\
271	0\\
272	0\\
273	0\\
274	0\\
275	0\\
276	0\\
277	0\\
278	0\\
279	0\\
280	0\\
281	0\\
282	0\\
283	0\\
284	0\\
285	0\\
286	0\\
287	0\\
288	0\\
289	0\\
290	0\\
291	0\\
292	0\\
293	0\\
294	0\\
295	0\\
296	0\\
297	0\\
298	0\\
299	0\\
300	0\\
301	0\\
302	0\\
303	0\\
304	0\\
305	0\\
306	0\\
307	0\\
308	0\\
309	0\\
310	0\\
311	0\\
312	0\\
313	0\\
314	0\\
315	0\\
316	0\\
317	0\\
318	0\\
319	0\\
320	0\\
321	0\\
322	0\\
323	0\\
324	0\\
325	0\\
326	0\\
327	0\\
328	0\\
329	0\\
330	0\\
331	0\\
332	0\\
333	0\\
334	0\\
335	0\\
336	0\\
337	0\\
338	0\\
339	0\\
340	0\\
341	0\\
342	0\\
343	0\\
344	0\\
345	0\\
346	0\\
347	0\\
348	0\\
349	0\\
350	0\\
351	0\\
352	0\\
353	0\\
354	0\\
355	0\\
356	0\\
357	0\\
358	0\\
359	0\\
360	0\\
361	0\\
362	0\\
363	0\\
364	0\\
365	0\\
366	0\\
367	0\\
368	0\\
369	0\\
370	0\\
371	0\\
372	0\\
373	0\\
374	0\\
375	0\\
376	0\\
377	0\\
378	0\\
379	0\\
380	0\\
381	0\\
382	0\\
383	0\\
384	0\\
385	0\\
386	0\\
387	0\\
388	0\\
389	0\\
390	0\\
391	0\\
392	0\\
393	0\\
394	0\\
395	0\\
396	0\\
397	0\\
398	0\\
399	0\\
400	0\\
401	0\\
402	0\\
403	0\\
404	0\\
405	0\\
406	0\\
407	0\\
408	0\\
409	0\\
410	0\\
411	0\\
412	0\\
413	0\\
414	0\\
415	0\\
416	0\\
417	0\\
418	0\\
419	0\\
420	0\\
421	0\\
422	0\\
423	0\\
424	0\\
425	0\\
426	0\\
427	0\\
428	0\\
429	0\\
430	0\\
431	0\\
432	0\\
433	0\\
434	0\\
435	0\\
436	0\\
437	0\\
438	0\\
439	0\\
440	0\\
441	0\\
442	0\\
443	0\\
444	0\\
445	0\\
446	0\\
447	0\\
448	0\\
449	0\\
450	0\\
451	0\\
452	0\\
453	0\\
454	0\\
455	0\\
456	0\\
457	0\\
458	0\\
459	0\\
460	0\\
461	0\\
462	0\\
463	0\\
464	0\\
465	0\\
466	0\\
467	0\\
468	0\\
469	0\\
470	0\\
471	0\\
472	0\\
473	0\\
474	0\\
475	0\\
476	0\\
477	0\\
478	0\\
479	0\\
480	0\\
481	0\\
482	0\\
483	0\\
484	0\\
485	0\\
486	0\\
487	0\\
488	0\\
489	0\\
490	0\\
491	0\\
492	0\\
493	0\\
494	0\\
495	0\\
496	0\\
497	0\\
498	0\\
499	0\\
500	0\\
501	0\\
502	0\\
503	0\\
504	0\\
505	0\\
506	0\\
507	0\\
508	0\\
509	0\\
510	0\\
511	0\\
512	0\\
513	0\\
514	0\\
515	0\\
516	0\\
517	0\\
518	0\\
519	0\\
520	0\\
521	0\\
522	0\\
523	0\\
524	0\\
525	0\\
526	0\\
527	0\\
528	0\\
529	0\\
530	0\\
531	0\\
532	0\\
533	0\\
534	0\\
535	0\\
536	0\\
537	3.82550301366919e-06\\
538	2.9106131297474e-05\\
539	5.49338526053371e-05\\
540	8.12709214509474e-05\\
541	0.000108136282402065\\
542	0.000135519390228108\\
543	0.000163492059064573\\
544	0.000192077027657505\\
545	0.000221277232734906\\
546	0.000251125248797619\\
547	0.00028165907332523\\
548	0.000312894077851403\\
549	0.000344852457097526\\
550	0.000377575892313951\\
551	0.000411272352801193\\
552	0.000445600411375425\\
553	0.000480493797716272\\
554	0.000516041013830302\\
555	0.000552284502233579\\
556	0.000589200581950451\\
557	0.000626998320133113\\
558	0.000665716783444826\\
559	0.000705887875314932\\
560	0.000747561296246264\\
561	0.000789266191669505\\
562	0.000831174369496882\\
563	0.000873178095902131\\
564	0.000915807659414534\\
565	0.000959488231720514\\
566	0.00100432096429517\\
567	0.0010503705610188\\
568	0.00138555014159372\\
569	0.00168867149361451\\
570	0.00192455116100545\\
571	0.00200205571737336\\
572	0.00207001668793317\\
573	0.00213863941240985\\
574	0.00220843933480784\\
575	0.00227948582459503\\
576	0.00235182128121044\\
577	0.00242548747819655\\
578	0.00250052798438547\\
579	0.00257698863349794\\
580	0.0026549180091809\\
581	0.0027343676902359\\
582	0.0028153925248352\\
583	0.00289805096238546\\
584	0.00298240553557085\\
585	0.00306852370928107\\
586	0.00315647969018814\\
587	0.00324635875816936\\
588	0.0033382683090296\\
589	0.00343236680134408\\
590	0.00352894054086422\\
591	0.00362860835901882\\
592	0.00373286830485375\\
593	0.00384555904940006\\
594	0.00397672044113206\\
595	0.00415280575412961\\
596	0.00444436306303141\\
597	0.00503983077166121\\
598	0.00644286460810295\\
599	0\\
600	0\\
};
\addplot [color=black!20!mycolor21,solid,forget plot]
  table[row sep=crcr]{%
1	0\\
2	0\\
3	0\\
4	0\\
5	0\\
6	0\\
7	0\\
8	0\\
9	0\\
10	0\\
11	0\\
12	0\\
13	0\\
14	0\\
15	0\\
16	0\\
17	0\\
18	0\\
19	0\\
20	0\\
21	0\\
22	0\\
23	0\\
24	0\\
25	0\\
26	0\\
27	0\\
28	0\\
29	0\\
30	0\\
31	0\\
32	0\\
33	0\\
34	0\\
35	0\\
36	0\\
37	0\\
38	0\\
39	0\\
40	0\\
41	0\\
42	0\\
43	0\\
44	0\\
45	0\\
46	0\\
47	0\\
48	0\\
49	0\\
50	0\\
51	0\\
52	0\\
53	0\\
54	0\\
55	0\\
56	0\\
57	0\\
58	0\\
59	0\\
60	0\\
61	0\\
62	0\\
63	0\\
64	0\\
65	0\\
66	0\\
67	0\\
68	0\\
69	0\\
70	0\\
71	0\\
72	0\\
73	0\\
74	0\\
75	0\\
76	0\\
77	0\\
78	0\\
79	0\\
80	0\\
81	0\\
82	0\\
83	0\\
84	0\\
85	0\\
86	0\\
87	0\\
88	0\\
89	0\\
90	0\\
91	0\\
92	0\\
93	0\\
94	0\\
95	0\\
96	0\\
97	0\\
98	0\\
99	0\\
100	0\\
101	0\\
102	0\\
103	0\\
104	0\\
105	0\\
106	0\\
107	0\\
108	0\\
109	0\\
110	0\\
111	0\\
112	0\\
113	0\\
114	0\\
115	0\\
116	0\\
117	0\\
118	0\\
119	0\\
120	0\\
121	0\\
122	0\\
123	0\\
124	0\\
125	0\\
126	0\\
127	0\\
128	0\\
129	0\\
130	0\\
131	0\\
132	0\\
133	0\\
134	0\\
135	0\\
136	0\\
137	0\\
138	0\\
139	0\\
140	0\\
141	0\\
142	0\\
143	0\\
144	0\\
145	0\\
146	0\\
147	0\\
148	0\\
149	0\\
150	0\\
151	0\\
152	0\\
153	0\\
154	0\\
155	0\\
156	0\\
157	0\\
158	0\\
159	0\\
160	0\\
161	0\\
162	0\\
163	0\\
164	0\\
165	0\\
166	0\\
167	0\\
168	0\\
169	0\\
170	0\\
171	0\\
172	0\\
173	0\\
174	0\\
175	0\\
176	0\\
177	0\\
178	0\\
179	0\\
180	0\\
181	0\\
182	0\\
183	0\\
184	0\\
185	0\\
186	0\\
187	0\\
188	0\\
189	0\\
190	0\\
191	0\\
192	0\\
193	0\\
194	0\\
195	0\\
196	0\\
197	0\\
198	0\\
199	0\\
200	0\\
201	0\\
202	0\\
203	0\\
204	0\\
205	0\\
206	0\\
207	0\\
208	0\\
209	0\\
210	0\\
211	0\\
212	0\\
213	0\\
214	0\\
215	0\\
216	0\\
217	0\\
218	0\\
219	0\\
220	0\\
221	0\\
222	0\\
223	0\\
224	0\\
225	0\\
226	0\\
227	0\\
228	0\\
229	0\\
230	0\\
231	0\\
232	0\\
233	0\\
234	0\\
235	0\\
236	0\\
237	0\\
238	0\\
239	0\\
240	0\\
241	0\\
242	0\\
243	0\\
244	0\\
245	0\\
246	0\\
247	0\\
248	0\\
249	0\\
250	0\\
251	0\\
252	0\\
253	0\\
254	0\\
255	0\\
256	0\\
257	0\\
258	0\\
259	0\\
260	0\\
261	0\\
262	0\\
263	0\\
264	0\\
265	0\\
266	0\\
267	0\\
268	0\\
269	0\\
270	0\\
271	0\\
272	0\\
273	0\\
274	0\\
275	0\\
276	0\\
277	0\\
278	0\\
279	0\\
280	0\\
281	0\\
282	0\\
283	0\\
284	0\\
285	0\\
286	0\\
287	0\\
288	0\\
289	0\\
290	0\\
291	0\\
292	0\\
293	0\\
294	0\\
295	0\\
296	0\\
297	0\\
298	0\\
299	0\\
300	0\\
301	0\\
302	0\\
303	0\\
304	0\\
305	0\\
306	0\\
307	0\\
308	0\\
309	0\\
310	0\\
311	0\\
312	0\\
313	0\\
314	0\\
315	0\\
316	0\\
317	0\\
318	0\\
319	0\\
320	0\\
321	0\\
322	0\\
323	0\\
324	0\\
325	0\\
326	0\\
327	0\\
328	0\\
329	0\\
330	0\\
331	0\\
332	0\\
333	0\\
334	0\\
335	0\\
336	0\\
337	0\\
338	0\\
339	0\\
340	0\\
341	0\\
342	0\\
343	0\\
344	0\\
345	0\\
346	0\\
347	0\\
348	0\\
349	0\\
350	0\\
351	0\\
352	0\\
353	0\\
354	0\\
355	0\\
356	0\\
357	0\\
358	0\\
359	0\\
360	0\\
361	0\\
362	0\\
363	0\\
364	0\\
365	0\\
366	0\\
367	0\\
368	0\\
369	0\\
370	0\\
371	0\\
372	0\\
373	0\\
374	0\\
375	0\\
376	0\\
377	0\\
378	0\\
379	0\\
380	0\\
381	0\\
382	0\\
383	0\\
384	0\\
385	0\\
386	0\\
387	0\\
388	0\\
389	0\\
390	0\\
391	0\\
392	0\\
393	0\\
394	0\\
395	0\\
396	0\\
397	0\\
398	0\\
399	0\\
400	0\\
401	0\\
402	0\\
403	0\\
404	0\\
405	0\\
406	0\\
407	0\\
408	0\\
409	0\\
410	0\\
411	0\\
412	0\\
413	0\\
414	0\\
415	0\\
416	0\\
417	0\\
418	0\\
419	0\\
420	0\\
421	0\\
422	0\\
423	0\\
424	0\\
425	0\\
426	0\\
427	0\\
428	0\\
429	0\\
430	0\\
431	0\\
432	0\\
433	0\\
434	0\\
435	0\\
436	0\\
437	0\\
438	0\\
439	0\\
440	0\\
441	0\\
442	0\\
443	0\\
444	0\\
445	0\\
446	0\\
447	0\\
448	0\\
449	0\\
450	0\\
451	0\\
452	0\\
453	0\\
454	0\\
455	0\\
456	0\\
457	0\\
458	0\\
459	0\\
460	0\\
461	0\\
462	0\\
463	0\\
464	0\\
465	0\\
466	0\\
467	0\\
468	0\\
469	0\\
470	0\\
471	0\\
472	0\\
473	0\\
474	0\\
475	0\\
476	0\\
477	0\\
478	0\\
479	0\\
480	0\\
481	0\\
482	0\\
483	0\\
484	0\\
485	0\\
486	0\\
487	0\\
488	0\\
489	0\\
490	0\\
491	0\\
492	0\\
493	0\\
494	0\\
495	0\\
496	0\\
497	0\\
498	0\\
499	0\\
500	0\\
501	0\\
502	0\\
503	0\\
504	0\\
505	0\\
506	0\\
507	0\\
508	0\\
509	0\\
510	0\\
511	0\\
512	0\\
513	0\\
514	0\\
515	0\\
516	0\\
517	0\\
518	0\\
519	0\\
520	0\\
521	0\\
522	0\\
523	0\\
524	0\\
525	0\\
526	0\\
527	0\\
528	0\\
529	0\\
530	0\\
531	0\\
532	0\\
533	0\\
534	0\\
535	0\\
536	0\\
537	1.5257515615078e-05\\
538	4.05712495559513e-05\\
539	6.63714757039172e-05\\
540	9.27016742245395e-05\\
541	0.000119575710927847\\
542	0.000147000013995511\\
543	0.00017504869257464\\
544	0.000203735987885211\\
545	0.000233079425442273\\
546	0.000263096693183489\\
547	0.000293803947281356\\
548	0.000325302371137699\\
549	0.000357642385253218\\
550	0.000390582931632927\\
551	0.00042399701337393\\
552	0.000458068777525668\\
553	0.000492727463226109\\
554	0.000528135608302436\\
555	0.000564380253208309\\
556	0.000601502109540624\\
557	0.000639665961938304\\
558	0.000679881186959353\\
559	0.000720197887011426\\
560	0.000760489350549704\\
561	0.00080084928102869\\
562	0.000841683425308702\\
563	0.000883510817814942\\
564	0.000926424297840563\\
565	0.000970477398949334\\
566	0.00106843923578318\\
567	0.00144259009169947\\
568	0.00171200842408046\\
569	0.0018698844738892\\
570	0.00193635519907031\\
571	0.00200267110891418\\
572	0.00207007254415696\\
573	0.00213865022157796\\
574	0.00220844272080478\\
575	0.00227948699417363\\
576	0.0023518217070068\\
577	0.0024254876758363\\
578	0.00250052808022166\\
579	0.0025769886830341\\
580	0.00265491803552198\\
581	0.00273436770413291\\
582	0.00281539253160681\\
583	0.0028980509654155\\
584	0.00298240553649866\\
585	0.00306852370943816\\
586	0.00315647969018814\\
587	0.00324635875816936\\
588	0.0033382683090296\\
589	0.00343236680134408\\
590	0.00352894054086423\\
591	0.00362860835901881\\
592	0.00373286830485375\\
593	0.00384555904940006\\
594	0.00397672044113205\\
595	0.00415280575412962\\
596	0.00444436306303141\\
597	0.00503983077166121\\
598	0.00644286460810295\\
599	0\\
600	0\\
};
\addplot [color=black!50!mycolor20,solid,forget plot]
  table[row sep=crcr]{%
1	0\\
2	0\\
3	0\\
4	0\\
5	0\\
6	0\\
7	0\\
8	0\\
9	0\\
10	0\\
11	0\\
12	0\\
13	0\\
14	0\\
15	0\\
16	0\\
17	0\\
18	0\\
19	0\\
20	0\\
21	0\\
22	0\\
23	0\\
24	0\\
25	0\\
26	0\\
27	0\\
28	0\\
29	0\\
30	0\\
31	0\\
32	0\\
33	0\\
34	0\\
35	0\\
36	0\\
37	0\\
38	0\\
39	0\\
40	0\\
41	0\\
42	0\\
43	0\\
44	0\\
45	0\\
46	0\\
47	0\\
48	0\\
49	0\\
50	0\\
51	0\\
52	0\\
53	0\\
54	0\\
55	0\\
56	0\\
57	0\\
58	0\\
59	0\\
60	0\\
61	0\\
62	0\\
63	0\\
64	0\\
65	0\\
66	0\\
67	0\\
68	0\\
69	0\\
70	0\\
71	0\\
72	0\\
73	0\\
74	0\\
75	0\\
76	0\\
77	0\\
78	0\\
79	0\\
80	0\\
81	0\\
82	0\\
83	0\\
84	0\\
85	0\\
86	0\\
87	0\\
88	0\\
89	0\\
90	0\\
91	0\\
92	0\\
93	0\\
94	0\\
95	0\\
96	0\\
97	0\\
98	0\\
99	0\\
100	0\\
101	0\\
102	0\\
103	0\\
104	0\\
105	0\\
106	0\\
107	0\\
108	0\\
109	0\\
110	0\\
111	0\\
112	0\\
113	0\\
114	0\\
115	0\\
116	0\\
117	0\\
118	0\\
119	0\\
120	0\\
121	0\\
122	0\\
123	0\\
124	0\\
125	0\\
126	0\\
127	0\\
128	0\\
129	0\\
130	0\\
131	0\\
132	0\\
133	0\\
134	0\\
135	0\\
136	0\\
137	0\\
138	0\\
139	0\\
140	0\\
141	0\\
142	0\\
143	0\\
144	0\\
145	0\\
146	0\\
147	0\\
148	0\\
149	0\\
150	0\\
151	0\\
152	0\\
153	0\\
154	0\\
155	0\\
156	0\\
157	0\\
158	0\\
159	0\\
160	0\\
161	0\\
162	0\\
163	0\\
164	0\\
165	0\\
166	0\\
167	0\\
168	0\\
169	0\\
170	0\\
171	0\\
172	0\\
173	0\\
174	0\\
175	0\\
176	0\\
177	0\\
178	0\\
179	0\\
180	0\\
181	0\\
182	0\\
183	0\\
184	0\\
185	0\\
186	0\\
187	0\\
188	0\\
189	0\\
190	0\\
191	0\\
192	0\\
193	0\\
194	0\\
195	0\\
196	0\\
197	0\\
198	0\\
199	0\\
200	0\\
201	0\\
202	0\\
203	0\\
204	0\\
205	0\\
206	0\\
207	0\\
208	0\\
209	0\\
210	0\\
211	0\\
212	0\\
213	0\\
214	0\\
215	0\\
216	0\\
217	0\\
218	0\\
219	0\\
220	0\\
221	0\\
222	0\\
223	0\\
224	0\\
225	0\\
226	0\\
227	0\\
228	0\\
229	0\\
230	0\\
231	0\\
232	0\\
233	0\\
234	0\\
235	0\\
236	0\\
237	0\\
238	0\\
239	0\\
240	0\\
241	0\\
242	0\\
243	0\\
244	0\\
245	0\\
246	0\\
247	0\\
248	0\\
249	0\\
250	0\\
251	0\\
252	0\\
253	0\\
254	0\\
255	0\\
256	0\\
257	0\\
258	0\\
259	0\\
260	0\\
261	0\\
262	0\\
263	0\\
264	0\\
265	0\\
266	0\\
267	0\\
268	0\\
269	0\\
270	0\\
271	0\\
272	0\\
273	0\\
274	0\\
275	0\\
276	0\\
277	0\\
278	0\\
279	0\\
280	0\\
281	0\\
282	0\\
283	0\\
284	0\\
285	0\\
286	0\\
287	0\\
288	0\\
289	0\\
290	0\\
291	0\\
292	0\\
293	0\\
294	0\\
295	0\\
296	0\\
297	0\\
298	0\\
299	0\\
300	0\\
301	0\\
302	0\\
303	0\\
304	0\\
305	0\\
306	0\\
307	0\\
308	0\\
309	0\\
310	0\\
311	0\\
312	0\\
313	0\\
314	0\\
315	0\\
316	0\\
317	0\\
318	0\\
319	0\\
320	0\\
321	0\\
322	0\\
323	0\\
324	0\\
325	0\\
326	0\\
327	0\\
328	0\\
329	0\\
330	0\\
331	0\\
332	0\\
333	0\\
334	0\\
335	0\\
336	0\\
337	0\\
338	0\\
339	0\\
340	0\\
341	0\\
342	0\\
343	0\\
344	0\\
345	0\\
346	0\\
347	0\\
348	0\\
349	0\\
350	0\\
351	0\\
352	0\\
353	0\\
354	0\\
355	0\\
356	0\\
357	0\\
358	0\\
359	0\\
360	0\\
361	0\\
362	0\\
363	0\\
364	0\\
365	0\\
366	0\\
367	0\\
368	0\\
369	0\\
370	0\\
371	0\\
372	0\\
373	0\\
374	0\\
375	0\\
376	0\\
377	0\\
378	0\\
379	0\\
380	0\\
381	0\\
382	0\\
383	0\\
384	0\\
385	0\\
386	0\\
387	0\\
388	0\\
389	0\\
390	0\\
391	0\\
392	0\\
393	0\\
394	0\\
395	0\\
396	0\\
397	0\\
398	0\\
399	0\\
400	0\\
401	0\\
402	0\\
403	0\\
404	0\\
405	0\\
406	0\\
407	0\\
408	0\\
409	0\\
410	0\\
411	0\\
412	0\\
413	0\\
414	0\\
415	0\\
416	0\\
417	0\\
418	0\\
419	0\\
420	0\\
421	0\\
422	0\\
423	0\\
424	0\\
425	0\\
426	0\\
427	0\\
428	0\\
429	0\\
430	0\\
431	0\\
432	0\\
433	0\\
434	0\\
435	0\\
436	0\\
437	0\\
438	0\\
439	0\\
440	0\\
441	0\\
442	0\\
443	0\\
444	0\\
445	0\\
446	0\\
447	0\\
448	0\\
449	0\\
450	0\\
451	0\\
452	0\\
453	0\\
454	0\\
455	0\\
456	0\\
457	0\\
458	0\\
459	0\\
460	0\\
461	0\\
462	0\\
463	0\\
464	0\\
465	0\\
466	0\\
467	0\\
468	0\\
469	0\\
470	0\\
471	0\\
472	0\\
473	0\\
474	0\\
475	0\\
476	0\\
477	0\\
478	0\\
479	0\\
480	0\\
481	0\\
482	0\\
483	0\\
484	0\\
485	0\\
486	0\\
487	0\\
488	0\\
489	0\\
490	0\\
491	0\\
492	0\\
493	0\\
494	0\\
495	0\\
496	0\\
497	0\\
498	0\\
499	0\\
500	0\\
501	0\\
502	0\\
503	0\\
504	0\\
505	0\\
506	0\\
507	0\\
508	0\\
509	0\\
510	0\\
511	0\\
512	0\\
513	0\\
514	0\\
515	0\\
516	0\\
517	0\\
518	0\\
519	0\\
520	0\\
521	0\\
522	0\\
523	0\\
524	0\\
525	0\\
526	0\\
527	0\\
528	0\\
529	0\\
530	0\\
531	0\\
532	0\\
533	0\\
534	0\\
535	0\\
536	3.78359368673223e-07\\
537	2.51996931176172e-05\\
538	5.05127313110625e-05\\
539	7.6341234946441e-05\\
540	0.000102729583218711\\
541	0.000129693335033606\\
542	0.000157267192673928\\
543	0.000185470106681217\\
544	0.000214316398288709\\
545	0.000243821524482409\\
546	0.000274119909113173\\
547	0.000305191906140328\\
548	0.00033677287195165\\
549	0.000368831234562262\\
550	0.000401484922657279\\
551	0.000434681869006069\\
552	0.00046864575352553\\
553	0.000503409758869739\\
554	0.000539018952998663\\
555	0.000575501755643469\\
556	0.000613762678061962\\
557	0.000652925523529455\\
558	0.000691742840127115\\
559	0.000730684352716771\\
560	0.000769821722848036\\
561	0.000809880072409532\\
562	0.000850959223737115\\
563	0.000893107882836268\\
564	0.00093637240331507\\
565	0.00109010374729446\\
566	0.00145489706217203\\
567	0.0017205761280254\\
568	0.00180709087821166\\
569	0.00187129168921907\\
570	0.0019364300875453\\
571	0.00200267818666582\\
572	0.00207007397726512\\
573	0.00213865068083015\\
574	0.00220844288323674\\
575	0.00227948705593812\\
576	0.00235182173622467\\
577	0.00242548769027979\\
578	0.00250052808776776\\
579	0.00257698868705665\\
580	0.00265491803762909\\
581	0.00273436770515037\\
582	0.00281539253204859\\
583	0.00289805096554754\\
584	0.00298240553652045\\
585	0.00306852370943815\\
586	0.00315647969018813\\
587	0.00324635875816935\\
588	0.0033382683090296\\
589	0.00343236680134408\\
590	0.00352894054086423\\
591	0.00362860835901881\\
592	0.00373286830485375\\
593	0.00384555904940006\\
594	0.00397672044113205\\
595	0.00415280575412962\\
596	0.00444436306303141\\
597	0.00503983077166121\\
598	0.00644286460810295\\
599	0\\
600	0\\
};
\addplot [color=black!60!mycolor21,solid,forget plot]
  table[row sep=crcr]{%
1	0\\
2	0\\
3	0\\
4	0\\
5	0\\
6	0\\
7	0\\
8	0\\
9	0\\
10	0\\
11	0\\
12	0\\
13	0\\
14	0\\
15	0\\
16	0\\
17	0\\
18	0\\
19	0\\
20	0\\
21	0\\
22	0\\
23	0\\
24	0\\
25	0\\
26	0\\
27	0\\
28	0\\
29	0\\
30	0\\
31	0\\
32	0\\
33	0\\
34	0\\
35	0\\
36	0\\
37	0\\
38	0\\
39	0\\
40	0\\
41	0\\
42	0\\
43	0\\
44	0\\
45	0\\
46	0\\
47	0\\
48	0\\
49	0\\
50	0\\
51	0\\
52	0\\
53	0\\
54	0\\
55	0\\
56	0\\
57	0\\
58	0\\
59	0\\
60	0\\
61	0\\
62	0\\
63	0\\
64	0\\
65	0\\
66	0\\
67	0\\
68	0\\
69	0\\
70	0\\
71	0\\
72	0\\
73	0\\
74	0\\
75	0\\
76	0\\
77	0\\
78	0\\
79	0\\
80	0\\
81	0\\
82	0\\
83	0\\
84	0\\
85	0\\
86	0\\
87	0\\
88	0\\
89	0\\
90	0\\
91	0\\
92	0\\
93	0\\
94	0\\
95	0\\
96	0\\
97	0\\
98	0\\
99	0\\
100	0\\
101	0\\
102	0\\
103	0\\
104	0\\
105	0\\
106	0\\
107	0\\
108	0\\
109	0\\
110	0\\
111	0\\
112	0\\
113	0\\
114	0\\
115	0\\
116	0\\
117	0\\
118	0\\
119	0\\
120	0\\
121	0\\
122	0\\
123	0\\
124	0\\
125	0\\
126	0\\
127	0\\
128	0\\
129	0\\
130	0\\
131	0\\
132	0\\
133	0\\
134	0\\
135	0\\
136	0\\
137	0\\
138	0\\
139	0\\
140	0\\
141	0\\
142	0\\
143	0\\
144	0\\
145	0\\
146	0\\
147	0\\
148	0\\
149	0\\
150	0\\
151	0\\
152	0\\
153	0\\
154	0\\
155	0\\
156	0\\
157	0\\
158	0\\
159	0\\
160	0\\
161	0\\
162	0\\
163	0\\
164	0\\
165	0\\
166	0\\
167	0\\
168	0\\
169	0\\
170	0\\
171	0\\
172	0\\
173	0\\
174	0\\
175	0\\
176	0\\
177	0\\
178	0\\
179	0\\
180	0\\
181	0\\
182	0\\
183	0\\
184	0\\
185	0\\
186	0\\
187	0\\
188	0\\
189	0\\
190	0\\
191	0\\
192	0\\
193	0\\
194	0\\
195	0\\
196	0\\
197	0\\
198	0\\
199	0\\
200	0\\
201	0\\
202	0\\
203	0\\
204	0\\
205	0\\
206	0\\
207	0\\
208	0\\
209	0\\
210	0\\
211	0\\
212	0\\
213	0\\
214	0\\
215	0\\
216	0\\
217	0\\
218	0\\
219	0\\
220	0\\
221	0\\
222	0\\
223	0\\
224	0\\
225	0\\
226	0\\
227	0\\
228	0\\
229	0\\
230	0\\
231	0\\
232	0\\
233	0\\
234	0\\
235	0\\
236	0\\
237	0\\
238	0\\
239	0\\
240	0\\
241	0\\
242	0\\
243	0\\
244	0\\
245	0\\
246	0\\
247	0\\
248	0\\
249	0\\
250	0\\
251	0\\
252	0\\
253	0\\
254	0\\
255	0\\
256	0\\
257	0\\
258	0\\
259	0\\
260	0\\
261	0\\
262	0\\
263	0\\
264	0\\
265	0\\
266	0\\
267	0\\
268	0\\
269	0\\
270	0\\
271	0\\
272	0\\
273	0\\
274	0\\
275	0\\
276	0\\
277	0\\
278	0\\
279	0\\
280	0\\
281	0\\
282	0\\
283	0\\
284	0\\
285	0\\
286	0\\
287	0\\
288	0\\
289	0\\
290	0\\
291	0\\
292	0\\
293	0\\
294	0\\
295	0\\
296	0\\
297	0\\
298	0\\
299	0\\
300	0\\
301	0\\
302	0\\
303	0\\
304	0\\
305	0\\
306	0\\
307	0\\
308	0\\
309	0\\
310	0\\
311	0\\
312	0\\
313	0\\
314	0\\
315	0\\
316	0\\
317	0\\
318	0\\
319	0\\
320	0\\
321	0\\
322	0\\
323	0\\
324	0\\
325	0\\
326	0\\
327	0\\
328	0\\
329	0\\
330	0\\
331	0\\
332	0\\
333	0\\
334	0\\
335	0\\
336	0\\
337	0\\
338	0\\
339	0\\
340	0\\
341	0\\
342	0\\
343	0\\
344	0\\
345	0\\
346	0\\
347	0\\
348	0\\
349	0\\
350	0\\
351	0\\
352	0\\
353	0\\
354	0\\
355	0\\
356	0\\
357	0\\
358	0\\
359	0\\
360	0\\
361	0\\
362	0\\
363	0\\
364	0\\
365	0\\
366	0\\
367	0\\
368	0\\
369	0\\
370	0\\
371	0\\
372	0\\
373	0\\
374	0\\
375	0\\
376	0\\
377	0\\
378	0\\
379	0\\
380	0\\
381	0\\
382	0\\
383	0\\
384	0\\
385	0\\
386	0\\
387	0\\
388	0\\
389	0\\
390	0\\
391	0\\
392	0\\
393	0\\
394	0\\
395	0\\
396	0\\
397	0\\
398	0\\
399	0\\
400	0\\
401	0\\
402	0\\
403	0\\
404	0\\
405	0\\
406	0\\
407	0\\
408	0\\
409	0\\
410	0\\
411	0\\
412	0\\
413	0\\
414	0\\
415	0\\
416	0\\
417	0\\
418	0\\
419	0\\
420	0\\
421	0\\
422	0\\
423	0\\
424	0\\
425	0\\
426	0\\
427	0\\
428	0\\
429	0\\
430	0\\
431	0\\
432	0\\
433	0\\
434	0\\
435	0\\
436	0\\
437	0\\
438	0\\
439	0\\
440	0\\
441	0\\
442	0\\
443	0\\
444	0\\
445	0\\
446	0\\
447	0\\
448	0\\
449	0\\
450	0\\
451	0\\
452	0\\
453	0\\
454	0\\
455	0\\
456	0\\
457	0\\
458	0\\
459	0\\
460	0\\
461	0\\
462	0\\
463	0\\
464	0\\
465	0\\
466	0\\
467	0\\
468	0\\
469	0\\
470	0\\
471	0\\
472	0\\
473	0\\
474	0\\
475	0\\
476	0\\
477	0\\
478	0\\
479	0\\
480	0\\
481	0\\
482	0\\
483	0\\
484	0\\
485	0\\
486	0\\
487	0\\
488	0\\
489	0\\
490	0\\
491	0\\
492	0\\
493	0\\
494	0\\
495	0\\
496	0\\
497	0\\
498	0\\
499	0\\
500	0\\
501	0\\
502	0\\
503	0\\
504	0\\
505	0\\
506	0\\
507	0\\
508	0\\
509	0\\
510	0\\
511	0\\
512	0\\
513	0\\
514	0\\
515	0\\
516	0\\
517	0\\
518	0\\
519	0\\
520	0\\
521	0\\
522	0\\
523	0\\
524	0\\
525	0\\
526	0\\
527	0\\
528	0\\
529	0\\
530	0\\
531	0\\
532	0\\
533	0\\
534	0\\
535	0\\
536	9.02145954336361e-06\\
537	3.38870820114369e-05\\
538	5.92843281107733e-05\\
539	8.52316144653158e-05\\
540	0.000111744299307854\\
541	0.000138841481860631\\
542	0.000166552064353744\\
543	0.00019489091642814\\
544	0.0002240061624105\\
545	0.000253856225130063\\
546	0.000284149213149703\\
547	0.000314894675820076\\
548	0.000346177518049505\\
549	0.000378018067102527\\
550	0.000410577800625992\\
551	0.000443915441746857\\
552	0.00047805631403732\\
553	0.000513026094184202\\
554	0.00054907441425197\\
555	0.000587044911282405\\
556	0.000624783568171093\\
557	0.000662504517028127\\
558	0.000700022606374216\\
559	0.000738375853450751\\
560	0.000777685135972991\\
561	0.000817995895139835\\
562	0.000859350044913817\\
563	0.000901791512692584\\
564	0.00109093692626027\\
565	0.00144213576563695\\
566	0.00167960457611589\\
567	0.00174425603612933\\
568	0.00180725733542332\\
569	0.00187130073891484\\
570	0.00193643097966661\\
571	0.00200267837569042\\
572	0.00207007403928857\\
573	0.00213865070333988\\
574	0.00220844289215589\\
575	0.00227948706023545\\
576	0.00235182173838593\\
577	0.00242548769141891\\
578	0.00250052808837546\\
579	0.00257698868737254\\
580	0.00265491803777995\\
581	0.00273436770521399\\
582	0.00281539253206718\\
583	0.00289805096555054\\
584	0.00298240553652045\\
585	0.00306852370943814\\
586	0.00315647969018814\\
587	0.00324635875816936\\
588	0.0033382683090296\\
589	0.00343236680134407\\
590	0.00352894054086422\\
591	0.00362860835901882\\
592	0.00373286830485375\\
593	0.00384555904940006\\
594	0.00397672044113206\\
595	0.00415280575412962\\
596	0.00444436306303141\\
597	0.00503983077166121\\
598	0.00644286460810295\\
599	0\\
600	0\\
};
\addplot [color=black!80!mycolor21,solid,forget plot]
  table[row sep=crcr]{%
1	0\\
2	0\\
3	0\\
4	0\\
5	0\\
6	0\\
7	0\\
8	0\\
9	0\\
10	0\\
11	0\\
12	0\\
13	0\\
14	0\\
15	0\\
16	0\\
17	0\\
18	0\\
19	0\\
20	0\\
21	0\\
22	0\\
23	0\\
24	0\\
25	0\\
26	0\\
27	0\\
28	0\\
29	0\\
30	0\\
31	0\\
32	0\\
33	0\\
34	0\\
35	0\\
36	0\\
37	0\\
38	0\\
39	0\\
40	0\\
41	0\\
42	0\\
43	0\\
44	0\\
45	0\\
46	0\\
47	0\\
48	0\\
49	0\\
50	0\\
51	0\\
52	0\\
53	0\\
54	0\\
55	0\\
56	0\\
57	0\\
58	0\\
59	0\\
60	0\\
61	0\\
62	0\\
63	0\\
64	0\\
65	0\\
66	0\\
67	0\\
68	0\\
69	0\\
70	0\\
71	0\\
72	0\\
73	0\\
74	0\\
75	0\\
76	0\\
77	0\\
78	0\\
79	0\\
80	0\\
81	0\\
82	0\\
83	0\\
84	0\\
85	0\\
86	0\\
87	0\\
88	0\\
89	0\\
90	0\\
91	0\\
92	0\\
93	0\\
94	0\\
95	0\\
96	0\\
97	0\\
98	0\\
99	0\\
100	0\\
101	0\\
102	0\\
103	0\\
104	0\\
105	0\\
106	0\\
107	0\\
108	0\\
109	0\\
110	0\\
111	0\\
112	0\\
113	0\\
114	0\\
115	0\\
116	0\\
117	0\\
118	0\\
119	0\\
120	0\\
121	0\\
122	0\\
123	0\\
124	0\\
125	0\\
126	0\\
127	0\\
128	0\\
129	0\\
130	0\\
131	0\\
132	0\\
133	0\\
134	0\\
135	0\\
136	0\\
137	0\\
138	0\\
139	0\\
140	0\\
141	0\\
142	0\\
143	0\\
144	0\\
145	0\\
146	0\\
147	0\\
148	0\\
149	0\\
150	0\\
151	0\\
152	0\\
153	0\\
154	0\\
155	0\\
156	0\\
157	0\\
158	0\\
159	0\\
160	0\\
161	0\\
162	0\\
163	0\\
164	0\\
165	0\\
166	0\\
167	0\\
168	0\\
169	0\\
170	0\\
171	0\\
172	0\\
173	0\\
174	0\\
175	0\\
176	0\\
177	0\\
178	0\\
179	0\\
180	0\\
181	0\\
182	0\\
183	0\\
184	0\\
185	0\\
186	0\\
187	0\\
188	0\\
189	0\\
190	0\\
191	0\\
192	0\\
193	0\\
194	0\\
195	0\\
196	0\\
197	0\\
198	0\\
199	0\\
200	0\\
201	0\\
202	0\\
203	0\\
204	0\\
205	0\\
206	0\\
207	0\\
208	0\\
209	0\\
210	0\\
211	0\\
212	0\\
213	0\\
214	0\\
215	0\\
216	0\\
217	0\\
218	0\\
219	0\\
220	0\\
221	0\\
222	0\\
223	0\\
224	0\\
225	0\\
226	0\\
227	0\\
228	0\\
229	0\\
230	0\\
231	0\\
232	0\\
233	0\\
234	0\\
235	0\\
236	0\\
237	0\\
238	0\\
239	0\\
240	0\\
241	0\\
242	0\\
243	0\\
244	0\\
245	0\\
246	0\\
247	0\\
248	0\\
249	0\\
250	0\\
251	0\\
252	0\\
253	0\\
254	0\\
255	0\\
256	0\\
257	0\\
258	0\\
259	0\\
260	0\\
261	0\\
262	0\\
263	0\\
264	0\\
265	0\\
266	0\\
267	0\\
268	0\\
269	0\\
270	0\\
271	0\\
272	0\\
273	0\\
274	0\\
275	0\\
276	0\\
277	0\\
278	0\\
279	0\\
280	0\\
281	0\\
282	0\\
283	0\\
284	0\\
285	0\\
286	0\\
287	0\\
288	0\\
289	0\\
290	0\\
291	0\\
292	0\\
293	0\\
294	0\\
295	0\\
296	0\\
297	0\\
298	0\\
299	0\\
300	0\\
301	0\\
302	0\\
303	0\\
304	0\\
305	0\\
306	0\\
307	0\\
308	0\\
309	0\\
310	0\\
311	0\\
312	0\\
313	0\\
314	0\\
315	0\\
316	0\\
317	0\\
318	0\\
319	0\\
320	0\\
321	0\\
322	0\\
323	0\\
324	0\\
325	0\\
326	0\\
327	0\\
328	0\\
329	0\\
330	0\\
331	0\\
332	0\\
333	0\\
334	0\\
335	0\\
336	0\\
337	0\\
338	0\\
339	0\\
340	0\\
341	0\\
342	0\\
343	0\\
344	0\\
345	0\\
346	0\\
347	0\\
348	0\\
349	0\\
350	0\\
351	0\\
352	0\\
353	0\\
354	0\\
355	0\\
356	0\\
357	0\\
358	0\\
359	0\\
360	0\\
361	0\\
362	0\\
363	0\\
364	0\\
365	0\\
366	0\\
367	0\\
368	0\\
369	0\\
370	0\\
371	0\\
372	0\\
373	0\\
374	0\\
375	0\\
376	0\\
377	0\\
378	0\\
379	0\\
380	0\\
381	0\\
382	0\\
383	0\\
384	0\\
385	0\\
386	0\\
387	0\\
388	0\\
389	0\\
390	0\\
391	0\\
392	0\\
393	0\\
394	0\\
395	0\\
396	0\\
397	0\\
398	0\\
399	0\\
400	0\\
401	0\\
402	0\\
403	0\\
404	0\\
405	0\\
406	0\\
407	0\\
408	0\\
409	0\\
410	0\\
411	0\\
412	0\\
413	0\\
414	0\\
415	0\\
416	0\\
417	0\\
418	0\\
419	0\\
420	0\\
421	0\\
422	0\\
423	0\\
424	0\\
425	0\\
426	0\\
427	0\\
428	0\\
429	0\\
430	0\\
431	0\\
432	0\\
433	0\\
434	0\\
435	0\\
436	0\\
437	0\\
438	0\\
439	0\\
440	0\\
441	0\\
442	0\\
443	0\\
444	0\\
445	0\\
446	0\\
447	0\\
448	0\\
449	0\\
450	0\\
451	0\\
452	0\\
453	0\\
454	0\\
455	0\\
456	0\\
457	0\\
458	0\\
459	0\\
460	0\\
461	0\\
462	0\\
463	0\\
464	0\\
465	0\\
466	0\\
467	0\\
468	0\\
469	0\\
470	0\\
471	0\\
472	0\\
473	0\\
474	0\\
475	0\\
476	0\\
477	0\\
478	0\\
479	0\\
480	0\\
481	0\\
482	0\\
483	0\\
484	0\\
485	0\\
486	0\\
487	0\\
488	0\\
489	0\\
490	0\\
491	0\\
492	0\\
493	0\\
494	0\\
495	0\\
496	0\\
497	0\\
498	0\\
499	0\\
500	0\\
501	0\\
502	0\\
503	0\\
504	0\\
505	0\\
506	0\\
507	0\\
508	0\\
509	0\\
510	0\\
511	0\\
512	0\\
513	0\\
514	0\\
515	0\\
516	0\\
517	0\\
518	0\\
519	0\\
520	0\\
521	0\\
522	0\\
523	0\\
524	0\\
525	0\\
526	0\\
527	0\\
528	0\\
529	0\\
530	0\\
531	0\\
532	0\\
533	0\\
534	0\\
535	0\\
536	1.67358631738589e-05\\
537	4.17113787144746e-05\\
538	6.72225789001668e-05\\
539	9.32840951840519e-05\\
540	0.000119910708273509\\
541	0.000147126549934032\\
542	0.000175083363931519\\
543	0.000203765392134202\\
544	0.000232847505447229\\
545	0.000262347629977608\\
546	0.000292337815156544\\
547	0.000322869907415254\\
548	0.000354099010332315\\
549	0.000386067917324203\\
550	0.000418801906923749\\
551	0.000452323301093407\\
552	0.000486654809378515\\
553	0.000522613247243088\\
554	0.000559557726971212\\
555	0.000596060905993073\\
556	0.000632410263998094\\
557	0.000669141593433294\\
558	0.000706767938764675\\
559	0.000745333893433888\\
560	0.000784877380144605\\
561	0.000825438279498713\\
562	0.000867058997353549\\
563	0.00106897995898566\\
564	0.00141439733785219\\
565	0.00162107369184165\\
566	0.00168232368728045\\
567	0.00174427557656458\\
568	0.00180725842153865\\
569	0.00187130085080805\\
570	0.00193643100447991\\
571	0.00200267838403535\\
572	0.00207007404240156\\
573	0.00213865070462208\\
574	0.00220844289278456\\
575	0.00227948706055647\\
576	0.00235182173855629\\
577	0.00242548769150975\\
578	0.00250052808842228\\
579	0.00257698868739462\\
580	0.00265491803778901\\
581	0.00273436770521657\\
582	0.00281539253206758\\
583	0.00289805096555055\\
584	0.00298240553652045\\
585	0.00306852370943816\\
586	0.00315647969018813\\
587	0.00324635875816936\\
588	0.0033382683090296\\
589	0.00343236680134409\\
590	0.00352894054086423\\
591	0.00362860835901882\\
592	0.00373286830485376\\
593	0.00384555904940007\\
594	0.00397672044113206\\
595	0.00415280575412962\\
596	0.00444436306303141\\
597	0.00503983077166122\\
598	0.00644286460810295\\
599	0\\
600	0\\
};
\addplot [color=black,solid,forget plot]
  table[row sep=crcr]{%
1	0\\
2	0\\
3	0\\
4	0\\
5	0\\
6	0\\
7	0\\
8	0\\
9	0\\
10	0\\
11	0\\
12	0\\
13	0\\
14	0\\
15	0\\
16	0\\
17	0\\
18	0\\
19	0\\
20	0\\
21	0\\
22	0\\
23	0\\
24	0\\
25	0\\
26	0\\
27	0\\
28	0\\
29	0\\
30	0\\
31	0\\
32	0\\
33	0\\
34	0\\
35	0\\
36	0\\
37	0\\
38	0\\
39	0\\
40	0\\
41	0\\
42	0\\
43	0\\
44	0\\
45	0\\
46	0\\
47	0\\
48	0\\
49	0\\
50	0\\
51	0\\
52	0\\
53	0\\
54	0\\
55	0\\
56	0\\
57	0\\
58	0\\
59	0\\
60	0\\
61	0\\
62	0\\
63	0\\
64	0\\
65	0\\
66	0\\
67	0\\
68	0\\
69	0\\
70	0\\
71	0\\
72	0\\
73	0\\
74	0\\
75	0\\
76	0\\
77	0\\
78	0\\
79	0\\
80	0\\
81	0\\
82	0\\
83	0\\
84	0\\
85	0\\
86	0\\
87	0\\
88	0\\
89	0\\
90	0\\
91	0\\
92	0\\
93	0\\
94	0\\
95	0\\
96	0\\
97	0\\
98	0\\
99	0\\
100	0\\
101	0\\
102	0\\
103	0\\
104	0\\
105	0\\
106	0\\
107	0\\
108	0\\
109	0\\
110	0\\
111	0\\
112	0\\
113	0\\
114	0\\
115	0\\
116	0\\
117	0\\
118	0\\
119	0\\
120	0\\
121	0\\
122	0\\
123	0\\
124	0\\
125	0\\
126	0\\
127	0\\
128	0\\
129	0\\
130	0\\
131	0\\
132	0\\
133	0\\
134	0\\
135	0\\
136	0\\
137	0\\
138	0\\
139	0\\
140	0\\
141	0\\
142	0\\
143	0\\
144	0\\
145	0\\
146	0\\
147	0\\
148	0\\
149	0\\
150	0\\
151	0\\
152	0\\
153	0\\
154	0\\
155	0\\
156	0\\
157	0\\
158	0\\
159	0\\
160	0\\
161	0\\
162	0\\
163	0\\
164	0\\
165	0\\
166	0\\
167	0\\
168	0\\
169	0\\
170	0\\
171	0\\
172	0\\
173	0\\
174	0\\
175	0\\
176	0\\
177	0\\
178	0\\
179	0\\
180	0\\
181	0\\
182	0\\
183	0\\
184	0\\
185	0\\
186	0\\
187	0\\
188	0\\
189	0\\
190	0\\
191	0\\
192	0\\
193	0\\
194	0\\
195	0\\
196	0\\
197	0\\
198	0\\
199	0\\
200	0\\
201	0\\
202	0\\
203	0\\
204	0\\
205	0\\
206	0\\
207	0\\
208	0\\
209	0\\
210	0\\
211	0\\
212	0\\
213	0\\
214	0\\
215	0\\
216	0\\
217	0\\
218	0\\
219	0\\
220	0\\
221	0\\
222	0\\
223	0\\
224	0\\
225	0\\
226	0\\
227	0\\
228	0\\
229	0\\
230	0\\
231	0\\
232	0\\
233	0\\
234	0\\
235	0\\
236	0\\
237	0\\
238	0\\
239	0\\
240	0\\
241	0\\
242	0\\
243	0\\
244	0\\
245	0\\
246	0\\
247	0\\
248	0\\
249	0\\
250	0\\
251	0\\
252	0\\
253	0\\
254	0\\
255	0\\
256	0\\
257	0\\
258	0\\
259	0\\
260	0\\
261	0\\
262	0\\
263	0\\
264	0\\
265	0\\
266	0\\
267	0\\
268	0\\
269	0\\
270	0\\
271	0\\
272	0\\
273	0\\
274	0\\
275	0\\
276	0\\
277	0\\
278	0\\
279	0\\
280	0\\
281	0\\
282	0\\
283	0\\
284	0\\
285	0\\
286	0\\
287	0\\
288	0\\
289	0\\
290	0\\
291	0\\
292	0\\
293	0\\
294	0\\
295	0\\
296	0\\
297	0\\
298	0\\
299	0\\
300	0\\
301	0\\
302	0\\
303	0\\
304	0\\
305	0\\
306	0\\
307	0\\
308	0\\
309	0\\
310	0\\
311	0\\
312	0\\
313	0\\
314	0\\
315	0\\
316	0\\
317	0\\
318	0\\
319	0\\
320	0\\
321	0\\
322	0\\
323	0\\
324	0\\
325	0\\
326	0\\
327	0\\
328	0\\
329	0\\
330	0\\
331	0\\
332	0\\
333	0\\
334	0\\
335	0\\
336	0\\
337	0\\
338	0\\
339	0\\
340	0\\
341	0\\
342	0\\
343	0\\
344	0\\
345	0\\
346	0\\
347	0\\
348	0\\
349	0\\
350	0\\
351	0\\
352	0\\
353	0\\
354	0\\
355	0\\
356	0\\
357	0\\
358	0\\
359	0\\
360	0\\
361	0\\
362	0\\
363	0\\
364	0\\
365	0\\
366	0\\
367	0\\
368	0\\
369	0\\
370	0\\
371	0\\
372	0\\
373	0\\
374	0\\
375	0\\
376	0\\
377	0\\
378	0\\
379	0\\
380	0\\
381	0\\
382	0\\
383	0\\
384	0\\
385	0\\
386	0\\
387	0\\
388	0\\
389	0\\
390	0\\
391	0\\
392	0\\
393	0\\
394	0\\
395	0\\
396	0\\
397	0\\
398	0\\
399	0\\
400	0\\
401	0\\
402	0\\
403	0\\
404	0\\
405	0\\
406	0\\
407	0\\
408	0\\
409	0\\
410	0\\
411	0\\
412	0\\
413	0\\
414	0\\
415	0\\
416	0\\
417	0\\
418	0\\
419	0\\
420	0\\
421	0\\
422	0\\
423	0\\
424	0\\
425	0\\
426	0\\
427	0\\
428	0\\
429	0\\
430	0\\
431	0\\
432	0\\
433	0\\
434	0\\
435	0\\
436	0\\
437	0\\
438	0\\
439	0\\
440	0\\
441	0\\
442	0\\
443	0\\
444	0\\
445	0\\
446	0\\
447	0\\
448	0\\
449	0\\
450	0\\
451	0\\
452	0\\
453	0\\
454	0\\
455	0\\
456	0\\
457	0\\
458	0\\
459	0\\
460	0\\
461	0\\
462	0\\
463	0\\
464	0\\
465	0\\
466	0\\
467	0\\
468	0\\
469	0\\
470	0\\
471	0\\
472	0\\
473	0\\
474	0\\
475	0\\
476	0\\
477	0\\
478	0\\
479	0\\
480	0\\
481	0\\
482	0\\
483	0\\
484	0\\
485	0\\
486	0\\
487	0\\
488	0\\
489	0\\
490	0\\
491	0\\
492	0\\
493	0\\
494	0\\
495	0\\
496	0\\
497	0\\
498	0\\
499	0\\
500	0\\
501	0\\
502	0\\
503	0\\
504	0\\
505	0\\
506	0\\
507	0\\
508	0\\
509	0\\
510	0\\
511	0\\
512	0\\
513	0\\
514	0\\
515	0\\
516	0\\
517	0\\
518	0\\
519	0\\
520	0\\
521	0\\
522	0\\
523	0\\
524	0\\
525	0\\
526	0\\
527	0\\
528	0\\
529	0\\
530	0\\
531	0\\
532	0\\
533	0\\
534	0\\
535	0\\
536	2.3273899000436e-05\\
537	4.83574677493769e-05\\
538	7.39764419503986e-05\\
539	0.000100145056725748\\
540	0.000126927776089947\\
541	0.000154551208357287\\
542	0.000182537594263675\\
543	0.000210877875154721\\
544	0.000239654581585972\\
545	0.000268942646031741\\
546	0.000298901101391198\\
547	0.000329565590919966\\
548	0.000360958489926797\\
549	0.000393100761277305\\
550	0.0004260137847175\\
551	0.000459720061076944\\
552	0.000495493917640628\\
553	0.000531330574309899\\
554	0.000566818867744161\\
555	0.000602062179114877\\
556	0.000638100502800796\\
557	0.000675022523672534\\
558	0.000712862997966947\\
559	0.00075165794188291\\
560	0.000791445930933432\\
561	0.000832268907521252\\
562	0.00101617204181345\\
563	0.00137303696588922\\
564	0.0015613898810912\\
565	0.00162138355504735\\
566	0.00168232596368277\\
567	0.0017442757060301\\
568	0.00180725843549943\\
569	0.00187130085404748\\
570	0.00193643100559733\\
571	0.00200267838446417\\
572	0.00207007404258462\\
573	0.0021386507047133\\
574	0.00220844289283176\\
575	0.00227948706058165\\
576	0.00235182173856971\\
577	0.0024254876915166\\
578	0.00250052808842547\\
579	0.00257698868739587\\
580	0.00265491803778937\\
581	0.00273436770521664\\
582	0.00281539253206758\\
583	0.00289805096555053\\
584	0.00298240553652045\\
585	0.00306852370943815\\
586	0.00315647969018813\\
587	0.00324635875816936\\
588	0.00333826830902961\\
589	0.00343236680134409\\
590	0.00352894054086423\\
591	0.00362860835901881\\
592	0.00373286830485375\\
593	0.00384555904940006\\
594	0.00397672044113206\\
595	0.00415280575412962\\
596	0.00444436306303141\\
597	0.00503983077166122\\
598	0.00644286460810295\\
599	0\\
600	0\\
};
\end{axis}
\end{tikzpicture}%
 
  \caption{Discrete Time}
\end{subfigure}\\

\leavevmode\smash{\makebox[0pt]{\hspace{-7em}% HORIZONTAL POSITION           
  \rotatebox[origin=l]{90}{\hspace{20em}% VERTICAL POSITION
    Depth $\delta^+$}%
}}\hspace{0pt plus 1filll}\null

Time (s)

\vspace{1cm}
\begin{subfigure}{\linewidth}
  \centering
  \tikzsetnextfilename{altdeltalegend}
  \definecolor{mycolor1}{rgb}{0.00000,1.00000,0.14286}%
\definecolor{mycolor2}{rgb}{0.00000,1.00000,0.28571}%
\definecolor{mycolor3}{rgb}{0.00000,1.00000,0.42857}%
\definecolor{mycolor4}{rgb}{0.00000,1.00000,0.57143}%
\definecolor{mycolor5}{rgb}{0.00000,1.00000,0.71429}%
\definecolor{mycolor6}{rgb}{0.00000,1.00000,0.85714}%
\definecolor{mycolor7}{rgb}{0.00000,1.00000,1.00000}%
\definecolor{mycolor8}{rgb}{0.00000,0.87500,1.00000}%
\definecolor{mycolor9}{rgb}{0.00000,0.62500,1.00000}%
\definecolor{mycolor10}{rgb}{0.12500,0.00000,1.00000}%
\definecolor{mycolor11}{rgb}{0.25000,0.00000,1.00000}%
\definecolor{mycolor12}{rgb}{0.37500,0.00000,1.00000}%
\definecolor{mycolor13}{rgb}{0.50000,0.00000,1.00000}%
\definecolor{mycolor14}{rgb}{0.62500,0.00000,1.00000}%
\definecolor{mycolor15}{rgb}{0.75000,0.00000,1.00000}%
\definecolor{mycolor16}{rgb}{0.87500,0.00000,1.00000}%
\definecolor{mycolor17}{rgb}{1.00000,0.00000,1.00000}%
\definecolor{mycolor18}{rgb}{1.00000,0.00000,0.87500}%
\definecolor{mycolor19}{rgb}{1.00000,0.00000,0.62500}%
\definecolor{mycolor20}{rgb}{0.85714,0.00000,0.00000}%
\definecolor{mycolor21}{rgb}{0.71429,0.00000,0.00000}%
%[trim axis left, trim axis right]
\begin{tikzpicture}
\begin{axis}[%
    hide axis,
    scale only axis,
    height=0pt,
    width=0pt,
    point meta min=-19,
    point meta max=19,
    colormap={mymap}{[1pt] rgb(0pt)=(0,1,0); rgb(7pt)=(0,1,1); rgb(15pt)=(0,0,1); rgb(23pt)=(1,0,1); rgb(31pt)=(1,0,0); rgb(38pt)=(0,0,0)},
    colorbar horizontal,
    colorbar style={width=15cm,xtick={{-15},{-10},{-5},{0},{5},{10},{15}}}
    %colorbar style={separate axis lines,every outer x axis line/.append style={black},every x tick label/.append style={font=\color{black}},every outer y axis line/.append style={black},every y tick label/.append style={font=\color{black}},yticklabels={{-19},{-17},{-15},{-13},{-11},{-9},{-7},{-5},{-3},{-1},{1},{3},{5},{7},{9},{11},{13},{15},{17},{19}}}
]%
    \addplot [draw=none] coordinates {(0,0)};
\end{axis}
\end{tikzpicture}
 
\end{subfigure}%
  \caption{Optimal buy depths $\delta^{+}$ for Markov state $Z=(\rho = +1, \Delta S = +1)$, implying heavy imbalance in favor of buy pressure, and having previously seen an upward price change. We expect the midprice to rise.}
  \label{fig:comp_dp_z15}
\end{figure}
The first notable conclusion we can make is the symmetry that has emerged between $\delta^+$ and $\delta^-$ in `opposite' Markov states. This is evident when comparing $\delta^+$ in $Z=(-1,-1)$ (\autoref{fig:comp_dp_z1}) with $\delta^-$ in $Z=(+1,+1)$ (\autoref{fig:comp_dm_z15}), $\delta^+$ in $Z=(0,0)$ (\autoref{fig:comp_dp_z8}) with $\delta^-$ in $Z=(0,0)$ (\autoref{fig:comp_dm_z8}), and $\delta^+$ in $Z=(+1,+1)$ (\autoref{fig:comp_dp_z15}) with $\delta^-$ in $Z=(-1,-1)$ (\autoref{fig:comp_dm_z1}). Thus, we focus the discussion here on the behavior of $\delta^+$.

In this calibration we have taken $\kappa=100$ and $\xi = 0.005$. From \eqref{eq:ctsBuycondition}, we thus know that a necessary condition for a buy market order to be executed is ${\delta^+}^* =  \frac{1}{\kappa} - 2 \xi = 0$.

{\bf Markov State $Z=(-1,-1)$ (\autoref{fig:comp_dp_z1})} \\
{\bf Cts versus Cts w nFPC:} For $q \geq 0$, both strategies post aggressive bid depths that suggest an inclination to buy. The nFPC strategy is less aggressive for $q<0$, posting closer to zero depth only for larger short positions.  \\
{\bf Dscr versus Dscr w nFPC:} The behaviour of these two calibrations is very similar. Both models show a discontinuity in dynamics at $q=0$, where at $q=-1$ it is posting at maximal depth, at $q=0$ it jumps to approximately \$0.006, and at $q=1$ again jumps lower. Otherwise, the nFPC strategy posts slightly more aggressively only when $q=1$ or 2. \\
{\bf Cts vs Dscr:} These models produce behaviours that are worlds apart. Whereas the Cts model seems to be saying that it wants to take this opportunity to go long, perhaps stocking up inventory while prices are low, the Dscr strategy suggests it's pulling out and avoiding purchasing. 

{\bf Markov State $Z=(0,0)$ (\autoref{fig:comp_dp_z8})}\\
Here we see near identical model behaviour. Similarities or correlations in backtesting performance can likely be attributed to this behaviour, as the majority of the day is spent in a Markov state for which $\Delta S = 0$, as seen here. (Recall that $\rho$, by contrast, is computed via evenly spaced percentiles symmetric around zero, so that time spent in each imbalance state is evenly distributed.)

{\bf Markov State $Z=(+1,+1)$ (\autoref{fig:comp_dp_z15})}\\
{\bf Cts versus Cts w nFPC:} We see a nearly symmetric behaviour compared with the opposite Markov state. Here for $q \leq 0$, both strategies post maximal bid depths, suggesting a disinclination toward buying. The nFPC strategy is more aggressive for $q>0$ and small inventory positions.\\
{\bf Dscr versus Dscr w nFPC:} Again these calibrations yield very similar behaviours, and as in the Cts case, it is near opposite to what was seen in the opposite Markov state.\\
{\bf Cts vs Dscr:} As before, these two models display near opposite behaviours between each other. The Dscr model is posting aggressive depths near zero in an attempt to purchase, while the Cts model is posting deeper into the book to avoid purchasing.

Finally, we note that we see relative stability in the posting depths at a time horizon of 600 seconds. This is consistent with the findings in \autoref{tbl:pvalues}, where we saw that the transition probability matrix $\mat{P}(t)$ converged for $\texttt{INTC}$ to an error threshold of $10^{-10}$ within 771 timesteps of 1s each.

%\begin{figure}
%\centering
%\begin{subfigure}{.45\linewidth}
%  \centering
%  \setlength\figureheight{\linewidth} 
%  \setlength\figurewidth{\linewidth}
%  \tikzsetnextfilename{dm_cts_z1}
%  % This file was created by matlab2tikz.
%
%The latest updates can be retrieved from
%  http://www.mathworks.com/matlabcentral/fileexchange/22022-matlab2tikz-matlab2tikz
%where you can also make suggestions and rate matlab2tikz.
%
\definecolor{mycolor1}{rgb}{1.00000,0.00000,1.00000}%
%
\begin{tikzpicture}[trim axis left, trim axis right]

\begin{axis}[%
width=\figurewidth,
height=\figureheight,
at={(0\figurewidth,0\figureheight)},
scale only axis,
every outer x axis line/.append style={black},
every x tick label/.append style={font=\color{black}},
xmin=0,
xmax=100,
%xlabel={Time},
every outer y axis line/.append style={black},
every y tick label/.append style={font=\color{black}},
ymin=0,
ymax=0.015,
%ylabel={Depth $\delta^-$},
axis background/.style={fill=white},
axis x line*=bottom,
axis y line*=left,
yticklabel style={
        /pgf/number format/fixed,
        /pgf/number format/precision=3
},
scaled y ticks=false,
legend style={legend cell align=left,align=left,draw=black,font=\footnotesize, at={(0.98,0.02)},anchor=south east},
every axis legend/.code={\renewcommand\addlegendentry[2][]{}}  %ignore legend locally
]
\addplot [color=green,dashed]
  table[row sep=crcr]{%
0.01	0.00986194196855773\\
1.01	0.00986341623340505\\
2.01	0.00986496661174401\\
3.01	0.00986659669123327\\
4.01	0.00986831003816703\\
5.01	0.00987011011999358\\
6.01	0.00987200019916482\\
7.01	0.00987398319805029\\
8.01	0.00987606169403044\\
9.01	0.00987823934078589\\
10.01	0.009880521604036\\
11.01	0.00988291449512232\\
12.01	0.00988542448547511\\
13.01	0.00988805855563265\\
14.01	0.0098908242511548\\
15.01	0.00989372974661671\\
16.01	0.00989678391910397\\
17.01	0.00989999643292112\\
18.01	0.00990337783757954\\
19.01	0.009906939681569\\
20.01	0.0099106946449578\\
21.01	0.00991465669453728\\
22.01	0.00991884126606046\\
23.01	0.00992326547916659\\
24.01	0.00992794839188696\\
25.01	0.00993291130326705\\
26.01	0.00993817811470555\\
27.01	0.0099437757632201\\
28.01	0.00994973474311438\\
29.01	0.00995608973624679\\
30.01	0.00996288037145739\\
31.01	0.00997015208471616\\
32.01	0.00997795635927097\\
33.01	0.00998633940302696\\
34.01	0.0099951270420302\\
35.01	0.01\\
36.01	0.01\\
37.01	0.01\\
38.01	0.01\\
39.01	0.01\\
40.01	0.01\\
41.01	0.01\\
42.01	0.01\\
43.01	0.01\\
44.01	0.01\\
45.01	0.01\\
46.01	0.01\\
47.01	0.01\\
48.01	0.01\\
49.01	0.01\\
50.01	0.01\\
51.01	0.01\\
52.01	0.01\\
53.01	0.01\\
54.01	0.01\\
55.01	0.01\\
56.01	0.01\\
57.01	0.01\\
58.01	0.01\\
59.01	0.01\\
60.01	0.01\\
61.01	0.01\\
62.01	0.01\\
63.01	0.01\\
64.01	0.01\\
65.01	0.01\\
66.01	0.01\\
67.01	0.01\\
68.01	0.01\\
69.01	0.01\\
70.01	0.01\\
71.01	0.01\\
72.01	0.01\\
73.01	0.01\\
74.01	0.01\\
75.01	0.01\\
76.01	0.01\\
77.01	0.01\\
78.01	0.01\\
79.01	0.01\\
80.01	0.01\\
81.01	0.01\\
82.01	0.01\\
83.01	0.01\\
84.01	0.01\\
85.01	0.01\\
86.01	0.01\\
87.01	0.01\\
88.01	0.01\\
89.01	0.01\\
90.01	0.01\\
91.01	0.01\\
92.01	0.01\\
93.01	0.01\\
94.01	0.01\\
95.01	0.01\\
96.01	0.01\\
97.01	0.01\\
98.01	0.01\\
99.01	0.01\\
99.02	0.01\\
99.03	0.01\\
99.04	0.01\\
99.05	0.01\\
99.06	0.01\\
99.07	0.01\\
99.08	0.01\\
99.09	0.01\\
99.1	0.01\\
99.11	0.01\\
99.12	0.01\\
99.13	0.01\\
99.14	0.01\\
99.15	0.01\\
99.16	0.01\\
99.17	0.01\\
99.18	0.01\\
99.19	0.01\\
99.2	0.01\\
99.21	0.01\\
99.22	0.01\\
99.23	0.01\\
99.24	0.01\\
99.25	0.01\\
99.26	0.01\\
99.27	0.01\\
99.28	0.01\\
99.29	0.01\\
99.3	0.01\\
99.31	0.01\\
99.32	0.01\\
99.33	0.01\\
99.34	0.01\\
99.35	0.01\\
99.36	0.01\\
99.37	0.01\\
99.38	0.01\\
99.39	0.01\\
99.4	0.01\\
99.41	0.01\\
99.42	0.01\\
99.43	0.01\\
99.44	0.01\\
99.45	0.01\\
99.46	0.01\\
99.47	0.01\\
99.48	0.01\\
99.49	0.01\\
99.5	0.01\\
99.51	0.01\\
99.52	0.01\\
99.53	0.01\\
99.54	0.01\\
99.55	0.01\\
99.56	0.01\\
99.57	0.01\\
99.58	0.01\\
99.59	0.01\\
99.6	0.01\\
99.61	0.01\\
99.62	0.01\\
99.63	0.01\\
99.64	0.01\\
99.65	0.01\\
99.66	0.01\\
99.67	0.01\\
99.68	0.01\\
99.69	0.01\\
99.7	0.01\\
99.71	0.01\\
99.72	0.01\\
99.73	0.01\\
99.74	0.01\\
99.75	0.01\\
99.76	0.01\\
99.77	0.01\\
99.78	0.01\\
99.79	0.01\\
99.8	0.01\\
99.81	0.01\\
99.82	0.01\\
99.83	0.01\\
99.84	0.01\\
99.85	0.01\\
99.86	0.01\\
99.87	0.01\\
99.88	0.01\\
99.89	0.01\\
99.9	0.01\\
99.91	0.01\\
99.92	0.01\\
99.93	0.01\\
99.94	0.01\\
99.95	0.01\\
99.96	0.01\\
99.97	0.01\\
99.98	0.01\\
99.99	0.01\\
100	0.01\\
};
\addlegendentry{$q=-4$};

\addplot [color=mycolor1,dashed]
  table[row sep=crcr]{%
0.01	0.00899553026425094\\
1.01	0.00899681616575985\\
2.01	0.00899816538225813\\
3.01	0.00899958107874248\\
4.01	0.00900106658403744\\
5.01	0.00900262540233493\\
6.01	0.00900426122739801\\
7.01	0.00900597796249557\\
8.01	0.00900777976385147\\
9.01	0.00900967112417916\\
10.01	0.00901165685622353\\
11.01	0.00901374206170057\\
12.01	0.00901593214839169\\
13.01	0.00901823285092495\\
14.01	0.00902065025312495\\
15.01	0.00902319081204287\\
16.01	0.00902586138377884\\
17.01	0.00902866925120362\\
18.01	0.00903162215367876\\
19.01	0.00903472831885562\\
20.01	0.00903799649660647\\
21.01	0.00904143599509583\\
22.01	0.00904505671893642\\
23.01	0.00904886920927959\\
24.01	0.00905288468555789\\
25.01	0.00905711508841163\\
26.01	0.00906157312307532\\
27.01	0.0090662723021453\\
28.01	0.00907122698614346\\
29.01	0.00907645241939374\\
30.01	0.00908196475559562\\
31.01	0.00908778104450535\\
32.01	0.0090939189241983\\
33.01	0.00910039335109809\\
34.01	0.00910718464044096\\
35.01	0.00911411520958012\\
36.01	0.00912130597047572\\
37.01	0.00912884125803057\\
38.01	0.00913678267528008\\
39.01	0.0091451622034404\\
40.01	0.00915401322426913\\
41.01	0.00916337289989351\\
42.01	0.00917328265695761\\
43.01	0.00918378818678301\\
44.01	0.00919493637913159\\
45.01	0.00920676382532887\\
46.01	0.00921932995469721\\
47.01	0.00923275864703257\\
48.01	0.00924714537333336\\
49.01	0.00926259594936277\\
50.01	0.00927924146141869\\
51.01	0.0092974388850447\\
52.01	0.00932168772055321\\
53.01	0.00937977735553907\\
54.01	0.00944582638911571\\
55.01	0.0095144883310851\\
56.01	0.00958588953326436\\
57.01	0.00966019282817128\\
58.01	0.00973764887814367\\
59.01	0.00981796544005801\\
60.01	0.00989919159200994\\
61.01	0.00997816795322349\\
62.01	0.01\\
63.01	0.01\\
64.01	0.01\\
65.01	0.01\\
66.01	0.01\\
67.01	0.01\\
68.01	0.01\\
69.01	0.01\\
70.01	0.01\\
71.01	0.01\\
72.01	0.01\\
73.01	0.01\\
74.01	0.01\\
75.01	0.01\\
76.01	0.01\\
77.01	0.01\\
78.01	0.01\\
79.01	0.01\\
80.01	0.01\\
81.01	0.01\\
82.01	0.01\\
83.01	0.01\\
84.01	0.01\\
85.01	0.01\\
86.01	0.01\\
87.01	0.01\\
88.01	0.01\\
89.01	0.01\\
90.01	0.01\\
91.01	0.01\\
92.01	0.01\\
93.01	0.01\\
94.01	0.01\\
95.01	0.01\\
96.01	0.01\\
97.01	0.01\\
98.01	0.01\\
99.01	0.01\\
99.02	0.01\\
99.03	0.01\\
99.04	0.01\\
99.05	0.01\\
99.06	0.01\\
99.07	0.01\\
99.08	0.01\\
99.09	0.01\\
99.1	0.01\\
99.11	0.01\\
99.12	0.01\\
99.13	0.01\\
99.14	0.01\\
99.15	0.01\\
99.16	0.01\\
99.17	0.01\\
99.18	0.01\\
99.19	0.01\\
99.2	0.01\\
99.21	0.01\\
99.22	0.01\\
99.23	0.01\\
99.24	0.01\\
99.25	0.01\\
99.26	0.01\\
99.27	0.01\\
99.28	0.01\\
99.29	0.01\\
99.3	0.01\\
99.31	0.01\\
99.32	0.01\\
99.33	0.01\\
99.34	0.01\\
99.35	0.01\\
99.36	0.01\\
99.37	0.01\\
99.38	0.01\\
99.39	0.01\\
99.4	0.01\\
99.41	0.01\\
99.42	0.01\\
99.43	0.01\\
99.44	0.01\\
99.45	0.01\\
99.46	0.01\\
99.47	0.01\\
99.48	0.01\\
99.49	0.01\\
99.5	0.01\\
99.51	0.01\\
99.52	0.01\\
99.53	0.01\\
99.54	0.01\\
99.55	0.01\\
99.56	0.01\\
99.57	0.01\\
99.58	0.01\\
99.59	0.01\\
99.6	0.01\\
99.61	0.01\\
99.62	0.01\\
99.63	0.01\\
99.64	0.01\\
99.65	0.01\\
99.66	0.01\\
99.67	0.01\\
99.68	0.01\\
99.69	0.01\\
99.7	0.01\\
99.71	0.01\\
99.72	0.01\\
99.73	0.01\\
99.74	0.01\\
99.75	0.01\\
99.76	0.01\\
99.77	0.01\\
99.78	0.01\\
99.79	0.01\\
99.8	0.01\\
99.81	0.01\\
99.82	0.01\\
99.83	0.01\\
99.84	0.01\\
99.85	0.01\\
99.86	0.01\\
99.87	0.01\\
99.88	0.01\\
99.89	0.01\\
99.9	0.01\\
99.91	0.01\\
99.92	0.01\\
99.93	0.01\\
99.94	0.01\\
99.95	0.01\\
99.96	0.01\\
99.97	0.01\\
99.98	0.01\\
99.99	0.01\\
100	0.01\\
};
\addlegendentry{$q=-3$};

\addplot [color=red,dashed]
  table[row sep=crcr]{%
0.01	0.00755870723885662\\
1.01	0.00755995005918438\\
2.01	0.00756125346743988\\
3.01	0.00756262048487351\\
4.01	0.00756405429013468\\
5.01	0.00756555822817135\\
6.01	0.00756713581970831\\
7.01	0.00756879077156534\\
8.01	0.00757052698867975\\
9.01	0.0075723485862971\\
10.01	0.00757425989849911\\
11.01	0.00757626548904896\\
12.01	0.00757837016438062\\
13.01	0.00758057898743496\\
14.01	0.00758289729236785\\
15.01	0.0075853307001909\\
16.01	0.00758788513541049\\
17.01	0.00759056684373631\\
18.01	0.00759338241093571\\
19.01	0.00759633878291826\\
20.01	0.00759944328714147\\
21.01	0.00760270365543958\\
22.01	0.00760612804838858\\
23.01	0.00760972508133611\\
24.01	0.00761350385224449\\
25.01	0.00761747397152153\\
26.01	0.00762164559404763\\
27.01	0.00762602945365452\\
28.01	0.00763063690036136\\
29.01	0.00763547994066308\\
30.01	0.00764057128052756\\
31.01	0.00764592436513198\\
32.01	0.00765155336632105\\
33.01	0.0076574727872834\\
34.01	0.00766369546365032\\
35.01	0.00767023623343137\\
36.01	0.00767712760083648\\
37.01	0.00768429575999159\\
38.01	0.00769182629889674\\
39.01	0.00769975577025217\\
40.01	0.00770810772349439\\
41.01	0.00771690722720528\\
42.01	0.00772618087361956\\
43.01	0.00773595632075487\\
44.01	0.00774625890726877\\
45.01	0.00775709205126277\\
46.01	0.00776843215176224\\
47.01	0.00778040704901406\\
48.01	0.00779307824137373\\
49.01	0.00780648966293305\\
50.01	0.00782068547000673\\
51.01	0.0078357159842505\\
52.01	0.00785196613522768\\
53.01	0.00787030265439305\\
54.01	0.00788976411747893\\
55.01	0.00791037679807219\\
56.01	0.00793224877608778\\
57.01	0.00795553195994027\\
58.01	0.00798054100001887\\
59.01	0.00800772956786817\\
60.01	0.00803269001771853\\
61.01	0.00805761703777974\\
62.01	0.00808196757295361\\
63.01	0.00810849652382792\\
64.01	0.00814444089726804\\
65.01	0.00822870901319658\\
66.01	0.00832471528271918\\
67.01	0.00842433066189096\\
68.01	0.00852774614992076\\
69.01	0.00863518389616826\\
70.01	0.00874690286396195\\
71.01	0.00886295416043401\\
72.01	0.00898271641094435\\
73.01	0.00910780635697492\\
74.01	0.00924150988703335\\
75.01	0.00937239286505567\\
76.01	0.00948105615150061\\
77.01	0.00959301356417407\\
78.01	0.00970838445792531\\
79.01	0.00982473370683794\\
80.01	0.00993766075272426\\
81.01	0.01\\
82.01	0.01\\
83.01	0.01\\
84.01	0.01\\
85.01	0.01\\
86.01	0.01\\
87.01	0.01\\
88.01	0.01\\
89.01	0.01\\
90.01	0.01\\
91.01	0.01\\
92.01	0.01\\
93.01	0.01\\
94.01	0.01\\
95.01	0.01\\
96.01	0.01\\
97.01	0.01\\
98.01	0.01\\
99.01	0.01\\
99.02	0.01\\
99.03	0.01\\
99.04	0.01\\
99.05	0.01\\
99.06	0.01\\
99.07	0.01\\
99.08	0.01\\
99.09	0.01\\
99.1	0.01\\
99.11	0.01\\
99.12	0.01\\
99.13	0.01\\
99.14	0.01\\
99.15	0.01\\
99.16	0.01\\
99.17	0.01\\
99.18	0.01\\
99.19	0.01\\
99.2	0.01\\
99.21	0.01\\
99.22	0.01\\
99.23	0.01\\
99.24	0.01\\
99.25	0.01\\
99.26	0.01\\
99.27	0.01\\
99.28	0.01\\
99.29	0.01\\
99.3	0.01\\
99.31	0.01\\
99.32	0.01\\
99.33	0.01\\
99.34	0.01\\
99.35	0.01\\
99.36	0.01\\
99.37	0.01\\
99.38	0.01\\
99.39	0.01\\
99.4	0.01\\
99.41	0.01\\
99.42	0.01\\
99.43	0.01\\
99.44	0.01\\
99.45	0.01\\
99.46	0.01\\
99.47	0.01\\
99.48	0.01\\
99.49	0.01\\
99.5	0.01\\
99.51	0.01\\
99.52	0.01\\
99.53	0.01\\
99.54	0.01\\
99.55	0.01\\
99.56	0.01\\
99.57	0.01\\
99.58	0.01\\
99.59	0.01\\
99.6	0.01\\
99.61	0.01\\
99.62	0.01\\
99.63	0.01\\
99.64	0.01\\
99.65	0.01\\
99.66	0.01\\
99.67	0.01\\
99.68	0.01\\
99.69	0.01\\
99.7	0.01\\
99.71	0.01\\
99.72	0.01\\
99.73	0.01\\
99.74	0.01\\
99.75	0.01\\
99.76	0.01\\
99.77	0.01\\
99.78	0.01\\
99.79	0.01\\
99.8	0.01\\
99.81	0.01\\
99.82	0.01\\
99.83	0.01\\
99.84	0.01\\
99.85	0.01\\
99.86	0.01\\
99.87	0.01\\
99.88	0.01\\
99.89	0.01\\
99.9	0.01\\
99.91	0.01\\
99.92	0.01\\
99.93	0.01\\
99.94	0.01\\
99.95	0.01\\
99.96	0.01\\
99.97	0.01\\
99.98	0.01\\
99.99	0.01\\
100	0.01\\
};
\addlegendentry{$q=-2$};

\addplot [color=blue,dashed]
  table[row sep=crcr]{%
0.01	0.0052336283583432\\
1.01	0.00523405952759689\\
2.01	0.00523451144426746\\
3.01	0.00523498512482119\\
4.01	0.00523548163769127\\
5.01	0.00523600210621855\\
6.01	0.00523654771180334\\
7.01	0.00523711969731229\\
8.01	0.00523771937077291\\
9.01	0.00523834810918792\\
10.01	0.00523900736248806\\
11.01	0.00523969865792891\\
12.01	0.00524042360486926\\
13.01	0.00524118389992261\\
14.01	0.00524198133251751\\
15.01	0.00524281779090816\\
16.01	0.00524369526868089\\
17.01	0.00524461587180792\\
18.01	0.00524558182630641\\
19.01	0.00524659548656777\\
20.01	0.00524765934443075\\
21.01	0.0052487760390819\\
22.01	0.00524994836787769\\
23.01	0.00525117929819646\\
24.01	0.00525247198044371\\
25.01	0.00525382976235229\\
26.01	0.00525525620474145\\
27.01	0.00525675509892372\\
28.01	0.0052583304859752\\
29.01	0.00525998667808767\\
30.01	0.00526172828204119\\
31.01	0.00526356022361065\\
32.01	0.00526548776646861\\
33.01	0.00526751651945328\\
34.01	0.00526965270930324\\
35.01	0.00527190665371612\\
36.01	0.00527432045133193\\
37.01	0.00527675072416128\\
38.01	0.00527921584008379\\
39.01	0.00528181769128329\\
40.01	0.00528456547657343\\
41.01	0.00528746920701615\\
42.01	0.0052905397557492\\
43.01	0.00529378874812398\\
44.01	0.00529722778568257\\
45.01	0.00530086665234674\\
46.01	0.00530472070108109\\
47.01	0.0053088157934421\\
48.01	0.00531317227190738\\
49.01	0.0053178116668415\\
50.01	0.00532275671269768\\
51.01	0.0053280330424496\\
52.01	0.00533370329093398\\
53.01	0.00533981801011714\\
54.01	0.00534637957896053\\
55.01	0.00535343653292726\\
56.01	0.0053610504256392\\
57.01	0.00536931687208603\\
58.01	0.00537849450791244\\
59.01	0.0053900998177927\\
60.01	0.00540984323441987\\
61.01	0.00543174270743712\\
62.01	0.00545487888937094\\
63.01	0.00547962068593745\\
64.01	0.00550706020210262\\
65.01	0.00553812969480962\\
66.01	0.00557050594163365\\
67.01	0.00560465959825306\\
68.01	0.00564079150733262\\
69.01	0.00567912382286856\\
70.01	0.00571997627079662\\
71.01	0.00576368439127989\\
72.01	0.00580831664102341\\
73.01	0.00585543815447048\\
74.01	0.00590998169975969\\
75.01	0.00600682255684936\\
76.01	0.00614685440466075\\
77.01	0.00629203145580564\\
78.01	0.00644315948253136\\
79.01	0.00659897529478316\\
80.01	0.00674910200436107\\
81.01	0.00690115119851525\\
82.01	0.00705774555742922\\
83.01	0.0072216010520313\\
84.01	0.00739598435369107\\
85.01	0.0075606539147745\\
86.01	0.00769813899918943\\
87.01	0.00784011427768724\\
88.01	0.00798655784400265\\
89.01	0.00813709028378619\\
90.01	0.00829175391790736\\
91.01	0.00845175835788679\\
92.01	0.00861728743823078\\
93.01	0.00878838632839203\\
94.01	0.00896528894135422\\
95.01	0.00914814562033832\\
96.01	0.00933768382380203\\
97.01	0.00953901623464867\\
98.01	0.00978114964483068\\
99.01	0.01\\
99.02	0.01\\
99.03	0.01\\
99.04	0.01\\
99.05	0.01\\
99.06	0.01\\
99.07	0.01\\
99.08	0.01\\
99.09	0.01\\
99.1	0.01\\
99.11	0.01\\
99.12	0.01\\
99.13	0.01\\
99.14	0.01\\
99.15	0.01\\
99.16	0.01\\
99.17	0.01\\
99.18	0.01\\
99.19	0.01\\
99.2	0.01\\
99.21	0.01\\
99.22	0.01\\
99.23	0.01\\
99.24	0.01\\
99.25	0.01\\
99.26	0.01\\
99.27	0.01\\
99.28	0.01\\
99.29	0.01\\
99.3	0.01\\
99.31	0.01\\
99.32	0.01\\
99.33	0.01\\
99.34	0.01\\
99.35	0.01\\
99.36	0.01\\
99.37	0.01\\
99.38	0.01\\
99.39	0.01\\
99.4	0.01\\
99.41	0.01\\
99.42	0.01\\
99.43	0.01\\
99.44	0.01\\
99.45	0.01\\
99.46	0.01\\
99.47	0.01\\
99.48	0.01\\
99.49	0.01\\
99.5	0.01\\
99.51	0.01\\
99.52	0.01\\
99.53	0.01\\
99.54	0.01\\
99.55	0.01\\
99.56	0.01\\
99.57	0.01\\
99.58	0.01\\
99.59	0.01\\
99.6	0.01\\
99.61	0.01\\
99.62	0.01\\
99.63	0.01\\
99.64	0.01\\
99.65	0.01\\
99.66	0.01\\
99.67	0.01\\
99.68	0.01\\
99.69	0.01\\
99.7	0.01\\
99.71	0.01\\
99.72	0.01\\
99.73	0.01\\
99.74	0.01\\
99.75	0.01\\
99.76	0.01\\
99.77	0.01\\
99.78	0.01\\
99.79	0.01\\
99.8	0.01\\
99.81	0.01\\
99.82	0.01\\
99.83	0.01\\
99.84	0.01\\
99.85	0.01\\
99.86	0.01\\
99.87	0.01\\
99.88	0.01\\
99.89	0.01\\
99.9	0.01\\
99.91	0.01\\
99.92	0.01\\
99.93	0.01\\
99.94	0.01\\
99.95	0.01\\
99.96	0.01\\
99.97	0.01\\
99.98	0.01\\
99.99	0.01\\
100	0.01\\
};
\addlegendentry{$q=-1$};

\addplot [color=black,solid]
  table[row sep=crcr]{%
0.01	0\\
1.01	0\\
2.01	0\\
3.01	0\\
4.01	0\\
5.01	0\\
6.01	0\\
7.01	0\\
8.01	0\\
9.01	0\\
10.01	0\\
11.01	0\\
12.01	0\\
13.01	0\\
14.01	0\\
15.01	0\\
16.01	0\\
17.01	0\\
18.01	0\\
19.01	0\\
20.01	0\\
21.01	0\\
22.01	0\\
23.01	0\\
24.01	0\\
25.01	0\\
26.01	0\\
27.01	0\\
28.01	0\\
29.01	0\\
30.01	0\\
31.01	0\\
32.01	0\\
33.01	0\\
34.01	0\\
35.01	0\\
36.01	0\\
37.01	2.7098314418425e-07\\
38.01	6.72601697098568e-07\\
39.01	1.09603785213019e-06\\
40.01	1.54272042145881e-06\\
41.01	2.01419497010934e-06\\
42.01	2.51210975899668e-06\\
43.01	3.03814885859767e-06\\
44.01	3.59391466852634e-06\\
45.01	4.18118877907291e-06\\
46.01	4.80313631412779e-06\\
47.01	5.46318199914041e-06\\
48.01	6.1643496488701e-06\\
49.01	6.90976409224893e-06\\
50.01	7.70261142136672e-06\\
51.01	8.54754165984829e-06\\
52.01	9.45360081083813e-06\\
53.01	1.04256579901065e-05\\
54.01	1.14689184704305e-05\\
55.01	1.25924347604209e-05\\
56.01	1.38107675656928e-05\\
57.01	1.5158325066213e-05\\
58.01	1.67443761224864e-05\\
59.01	1.88896805408314e-05\\
60.01	2.16804617222208e-05\\
61.01	2.46604273995161e-05\\
62.01	2.78498511532919e-05\\
63.01	3.13412241101391e-05\\
64.01	3.53395605633931e-05\\
65.01	4.01477618857358e-05\\
66.01	4.60670770527482e-05\\
67.01	5.2389342955924e-05\\
68.01	5.91313983273386e-05\\
69.01	6.63797301328692e-05\\
70.01	7.43735053918473e-05\\
71.01	8.40659909861242e-05\\
72.01	9.87569411535407e-05\\
73.01	0.000116219008803816\\
74.01	0.00013658469870431\\
75.01	0.00016268964669495\\
76.01	0.000191701514614633\\
77.01	0.000222542667619325\\
78.01	0.000256658377935573\\
79.01	0.000301126961267025\\
80.01	0.000364415228369714\\
81.01	0.00043130172413214\\
82.01	0.000502298045484141\\
83.01	0.000579330785094683\\
84.01	0.000671081018125567\\
85.01	0.000824801827111416\\
86.01	0.00103084607679419\\
87.01	0.00124341172800786\\
88.01	0.00146287110651892\\
89.01	0.0016896428967384\\
90.01	0.0019243436918427\\
91.01	0.00216766740583869\\
92.01	0.00242027370067842\\
93.01	0.00268293612335087\\
94.01	0.00295656462490684\\
95.01	0.00324228884436571\\
96.01	0.00354208727028447\\
97.01	0.00386305506827559\\
98.01	0.00425019667752173\\
99.01	0.00507764158572622\\
99.02	0.00509333154654971\\
99.03	0.00510930350582921\\
99.04	0.00512556415351287\\
99.05	0.00514212035096883\\
99.06	0.00515897913589884\\
99.07	0.00517614772741901\\
99.08	0.00519363353131232\\
99.09	0.00521144414546006\\
99.1	0.00522958736545719\\
99.11	0.00524807119042307\\
99.12	0.00526690382901496\\
99.13	0.0052860937056522\\
99.14	0.00530564946695966\\
99.15	0.00532557998843928\\
99.16	0.00534589438138126\\
99.17	0.00536660200002445\\
99.18	0.00538771244897647\\
99.19	0.00540923559090341\\
99.2	0.00543118155450547\\
99.21	0.00545356074278902\\
99.22	0.00547638384164783\\
99.23	0.00549966182876664\\
99.24	0.00552340598286617\\
99.25	0.005547627893303\\
99.26	0.00557233947004092\\
99.27	0.00559755295401146\\
99.28	0.0056232809278821\\
99.29	0.005649536327252\\
99.3	0.00567633245229586\\
99.31	0.00570368297987814\\
99.32	0.00573160197616093\\
99.33	0.00576010390973003\\
99.34	0.00578920366526573\\
99.35	0.00581891655778816\\
99.36	0.00584925834750219\\
99.37	0.00588024525895556\\
99.38	0.00591189402111133\\
99.39	0.00594422185780725\\
99.4	0.00597717154226581\\
99.41	0.00601075621332471\\
99.42	0.00604499237217719\\
99.43	0.00607989701668596\\
99.44	0.00611548765934936\\
99.45	0.006151768920206\\
99.46	0.00618875089111782\\
99.47	0.00622645214145022\\
99.48	0.00626487592887922\\
99.49	0.00630402223164896\\
99.5	0.00634391039886314\\
99.51	0.00638456038915832\\
99.52	0.00642599279407218\\
99.53	0.0064682288625287\\
99.54	0.00651129052650586\\
99.55	0.00655520042795615\\
99.56	0.00659998194705504\\
99.57	0.00664565923185813\\
99.58	0.00669225722945325\\
99.59	0.00673980171870044\\
99.6	0.00678831934465935\\
99.61	0.00683783765481112\\
99.62	0.00688838513718996\\
99.63	0.00693999126054839\\
99.64	0.00699268651668975\\
99.65	0.00704650246511194\\
99.66	0.00710147178011797\\
99.67	0.00715762830056103\\
99.68	0.00721500708240582\\
99.69	0.00727364445430238\\
99.7	0.00733357807638529\\
99.71	0.00739484700252884\\
99.72	0.00745749174630883\\
99.73	0.00752155435094301\\
99.74	0.00758707846350445\\
99.75	0.00765410941373053\\
99.76	0.00772269429777857\\
99.77	0.00779288206731036\\
99.78	0.00786472362432376\\
99.79	0.00793827192218827\\
99.8	0.00801358207338469\\
99.81	0.00809071146449738\\
99.82	0.00816971987906073\\
99.83	0.00825066962892118\\
99.84	0.00833362569484242\\
99.85	0.00841865587715554\\
99.86	0.00850583095733869\\
99.87	0.00859522487150397\\
99.88	0.00868691489687313\\
99.89	0.00878098185244132\\
99.9	0.00887751031515977\\
99.91	0.00897658884981403\\
99.92	0.00907831025960909\\
99.93	0.00918277186280922\\
99.94	0.00929007578758202\\
99.95	0.00940032929168484\\
99.96	0.00951364510957589\\
99.97	0.00963014182985264\\
99.98	0.00974994430628741\\
99.99	0.00987318410615103\\
100	0.01\\
};
\addlegendentry{$q=0$};

\addplot [color=blue,solid]
  table[row sep=crcr]{%
0.01	0.01\\
1.01	0.01\\
2.01	0.01\\
3.01	0.01\\
4.01	0.01\\
5.01	0.01\\
6.01	0.01\\
7.01	0.01\\
8.01	0.01\\
9.01	0.01\\
10.01	0.01\\
11.01	0.01\\
12.01	0.01\\
13.01	0.01\\
14.01	0.01\\
15.01	0.01\\
16.01	0.01\\
17.01	0.01\\
18.01	0.01\\
19.01	0.01\\
20.01	0.01\\
21.01	0.01\\
22.01	0.01\\
23.01	0.01\\
24.01	0.01\\
25.01	0.01\\
26.01	0.01\\
27.01	0.01\\
28.01	0.01\\
29.01	0.01\\
30.01	0.01\\
31.01	0.01\\
32.01	0.01\\
33.01	0.01\\
34.01	0.01\\
35.01	0.01\\
36.01	0.01\\
37.01	0.01\\
38.01	0.01\\
39.01	0.01\\
40.01	0.01\\
41.01	0.01\\
42.01	0.01\\
43.01	0.01\\
44.01	0.01\\
45.01	0.01\\
46.01	0.01\\
47.01	0.01\\
48.01	0.01\\
49.01	0.01\\
50.01	0.01\\
51.01	0.01\\
52.01	0.01\\
53.01	0.01\\
54.01	0.01\\
55.01	0.01\\
56.01	0.01\\
57.01	0.01\\
58.01	0.01\\
59.01	0.01\\
60.01	0.01\\
61.01	0.01\\
62.01	0.01\\
63.01	0.01\\
64.01	0.01\\
65.01	0.01\\
66.01	0.01\\
67.01	0.01\\
68.01	0.01\\
69.01	0.01\\
70.01	0.01\\
71.01	0.01\\
72.01	0.01\\
73.01	0.01\\
74.01	0.01\\
75.01	0.01\\
76.01	0.01\\
77.01	0.01\\
78.01	0.01\\
79.01	0.01\\
80.01	0.01\\
81.01	0.01\\
82.01	0.01\\
83.01	0.01\\
84.01	0.01\\
85.01	0.01\\
86.01	0.01\\
87.01	0.01\\
88.01	0.01\\
89.01	0.01\\
90.01	0.01\\
91.01	0.01\\
92.01	0.01\\
93.01	0.01\\
94.01	0.01\\
95.01	0.01\\
96.01	0.01\\
97.01	0.01\\
98.01	0.01\\
99.01	0.01\\
99.02	0.01\\
99.03	0.01\\
99.04	0.01\\
99.05	0.01\\
99.06	0.01\\
99.07	0.01\\
99.08	0.01\\
99.09	0.01\\
99.1	0.01\\
99.11	0.01\\
99.12	0.01\\
99.13	0.01\\
99.14	0.01\\
99.15	0.01\\
99.16	0.01\\
99.17	0.01\\
99.18	0.01\\
99.19	0.01\\
99.2	0.01\\
99.21	0.01\\
99.22	0.01\\
99.23	0.01\\
99.24	0.01\\
99.25	0.01\\
99.26	0.01\\
99.27	0.01\\
99.28	0.01\\
99.29	0.01\\
99.3	0.01\\
99.31	0.01\\
99.32	0.01\\
99.33	0.01\\
99.34	0.01\\
99.35	0.01\\
99.36	0.01\\
99.37	0.01\\
99.38	0.01\\
99.39	0.01\\
99.4	0.01\\
99.41	0.01\\
99.42	0.00987202571330398\\
99.43	0.00972725105747827\\
99.44	0.00958178279195355\\
99.45	0.00943561855610114\\
99.46	0.00928875095460674\\
99.47	0.00914117239480965\\
99.48	0.00899287508119766\\
99.49	0.00884385100975308\\
99.5	0.00869409196214463\\
99.51	0.00854358949975994\\
99.52	0.00839233495757306\\
99.53	0.00824031943784096\\
99.54	0.00808753380362287\\
99.55	0.00793396867211586\\
99.56	0.00777961440779982\\
99.57	0.00762446111538449\\
99.58	0.007468498632551\\
99.59	0.00731171652247971\\
99.6	0.0071541040661559\\
99.61	0.00699565025444443\\
99.62	0.00683634377992369\\
99.63	0.006676173028469\\
99.64	0.00651512607057483\\
99.65	0.00635319065240467\\
99.66	0.00619035418656071\\
99.67	0.00602660374255689\\
99.68	0.00586192603698214\\
99.69	0.00569630742301583\\
99.7	0.00552973387981608\\
99.71	0.00536219100166203\\
99.72	0.00519366398672537\\
99.73	0.00502413762545392\\
99.74	0.00485359628854843\\
99.75	0.00468202391451301\\
99.76	0.0045094039967583\\
99.77	0.00433571957023531\\
99.78	0.00416095319757633\\
99.79	0.00398508695471812\\
99.8	0.00380810241598083\\
99.81	0.00362998063857475\\
99.82	0.00345070214650499\\
99.83	0.0032702469138424\\
99.84	0.00308859434732733\\
99.85	0.00290572326827031\\
99.86	0.00272161189371181\\
99.87	0.0025362378168008\\
99.88	0.00234957798634912\\
99.89	0.00216160868551618\\
99.9	0.0019723055095755\\
99.91	0.00178164334271157\\
99.92	0.00158959633379209\\
99.93	0.00139613787105757\\
99.94	0.00120124055566599\\
99.95	0.00100487617402678\\
99.96	0.000807015668854001\\
99.97	0.000607629108864105\\
99.98	0.000406685657039194\\
99.99	0.00020415353737153\\
100	0\\
};
\addlegendentry{$q=1$};

\addplot [color=red,solid]
  table[row sep=crcr]{%
0.01	0.01\\
1.01	0.01\\
2.01	0.01\\
3.01	0.01\\
4.01	0.01\\
5.01	0.01\\
6.01	0.01\\
7.01	0.01\\
8.01	0.01\\
9.01	0.01\\
10.01	0.01\\
11.01	0.01\\
12.01	0.01\\
13.01	0.01\\
14.01	0.01\\
15.01	0.01\\
16.01	0.01\\
17.01	0.01\\
18.01	0.01\\
19.01	0.01\\
20.01	0.01\\
21.01	0.01\\
22.01	0.01\\
23.01	0.01\\
24.01	0.01\\
25.01	0.01\\
26.01	0.01\\
27.01	0.01\\
28.01	0.01\\
29.01	0.01\\
30.01	0.01\\
31.01	0.01\\
32.01	0.01\\
33.01	0.01\\
34.01	0.01\\
35.01	0.01\\
36.01	0.01\\
37.01	0.01\\
38.01	0.01\\
39.01	0.01\\
40.01	0.01\\
41.01	0.01\\
42.01	0.01\\
43.01	0.01\\
44.01	0.01\\
45.01	0.01\\
46.01	0.01\\
47.01	0.01\\
48.01	0.01\\
49.01	0.01\\
50.01	0.01\\
51.01	0.01\\
52.01	0.01\\
53.01	0.01\\
54.01	0.01\\
55.01	0.01\\
56.01	0.01\\
57.01	0.01\\
58.01	0.01\\
59.01	0.01\\
60.01	0.01\\
61.01	0.01\\
62.01	0.01\\
63.01	0.01\\
64.01	0.01\\
65.01	0.01\\
66.01	0.01\\
67.01	0.01\\
68.01	0.01\\
69.01	0.01\\
70.01	0.01\\
71.01	0.01\\
72.01	0.01\\
73.01	0.01\\
74.01	0.01\\
75.01	0.01\\
76.01	0.01\\
77.01	0.01\\
78.01	0.01\\
79.01	0.01\\
80.01	0.01\\
81.01	0.01\\
82.01	0.01\\
83.01	0.01\\
84.01	0.01\\
85.01	0.01\\
86.01	0.01\\
87.01	0.01\\
88.01	0.01\\
89.01	0.01\\
90.01	0.01\\
91.01	0.01\\
92.01	0.01\\
93.01	0.01\\
94.01	0.01\\
95.01	0.01\\
96.01	0.01\\
97.01	0.01\\
98.01	0.01\\
99.01	0.01\\
99.02	0.01\\
99.03	0.01\\
99.04	0.01\\
99.05	0.01\\
99.06	0.01\\
99.07	0.01\\
99.08	0.01\\
99.09	0.01\\
99.1	0.01\\
99.11	0.01\\
99.12	0.00996032578161788\\
99.13	0.00979197844179356\\
99.14	0.00962254325655415\\
99.15	0.00945200595910701\\
99.16	0.0092803538332374\\
99.17	0.00910759370775365\\
99.18	0.00893376227252234\\
99.19	0.00875884596546429\\
99.2	0.00858283085638335\\
99.21	0.00840570263481935\\
99.22	0.00822744659744947\\
99.23	0.00804804763501606\\
99.24	0.00786749021875827\\
99.25	0.00768575838632298\\
99.26	0.0075028358535259\\
99.27	0.00731870601091347\\
99.28	0.00713335177468211\\
99.29	0.00694675557035369\\
99.3	0.00675889931578866\\
99.31	0.00656976436843338\\
99.32	0.006379331514453\\
99.33	0.00618758097392389\\
99.34	0.00599449238068321\\
99.35	0.00580004476130922\\
99.36	0.00560421651318306\\
99.37	0.00540698538157952\\
99.38	0.00520832843223798\\
99.39	0.00500822202787693\\
99.4	0.00480664180388455\\
99.41	0.00460356264092079\\
99.42	0.00452719674959113\\
99.43	0.00446639176629747\\
99.44	0.00440503023173172\\
99.45	0.00434310362475017\\
99.46	0.00428060830868315\\
99.47	0.00421754069180432\\
99.48	0.00415389723063298\\
99.49	0.00408967443337529\\
99.5	0.00402486886351078\\
99.51	0.00395947714353091\\
99.52	0.00389349595883679\\
99.53	0.00382692206180365\\
99.54	0.00375975227601999\\
99.55	0.00369198350070989\\
99.56	0.00362361271534733\\
99.57	0.00355463698447212\\
99.58	0.00348505346271721\\
99.59	0.00341485940005822\\
99.6	0.00334405214729615\\
99.61	0.00327262916178542\\
99.62	0.00320058801341965\\
99.63	0.00312792639088859\\
99.64	0.00305464210822042\\
99.65	0.00298073311162465\\
99.66	0.0029061974862751\\
99.67	0.00283103346379373\\
99.68	0.00275523943008846\\
99.69	0.00267881396475747\\
99.7	0.00260175583002313\\
99.71	0.00252406396949578\\
99.72	0.00244573751764144\\
99.73	0.00236677580970948\\
99.74	0.00228717839214652\\
99.75	0.00220694503352383\\
99.76	0.00212607573600818\\
99.77	0.00204457074740711\\
99.78	0.00196243057382237\\
99.79	0.00187965599294699\\
99.8	0.00179624806804411\\
99.81	0.00171220816264801\\
99.82	0.00162753795603073\\
99.83	0.00154223945948034\\
99.84	0.00145631503344026\\
99.85	0.00136976740556229\\
99.86	0.00128259968972964\\
99.87	0.00119481540611014\\
99.88	0.00110641850230417\\
99.89	0.00101741337565592\\
99.9	0.000927804896802106\\
99.91	0.000837598434536804\\
99.92	0.000746799882077327\\
99.93	0.000655415684821711\\
99.94	0.000563452869695232\\
99.95	0.000470919076190374\\
99.96	0.000377822589212344\\
99.97	0.000284172373850622\\
99.98	0.000189978112205905\\
99.99	9.5250242411667e-05\\
100	0\\
};
\addlegendentry{$q=2$};

\addplot [color=mycolor1,solid]
  table[row sep=crcr]{%
0.01	0.01\\
1.01	0.01\\
2.01	0.01\\
3.01	0.01\\
4.01	0.01\\
5.01	0.01\\
6.01	0.01\\
7.01	0.01\\
8.01	0.01\\
9.01	0.01\\
10.01	0.01\\
11.01	0.01\\
12.01	0.01\\
13.01	0.01\\
14.01	0.01\\
15.01	0.01\\
16.01	0.01\\
17.01	0.01\\
18.01	0.01\\
19.01	0.01\\
20.01	0.01\\
21.01	0.01\\
22.01	0.01\\
23.01	0.01\\
24.01	0.01\\
25.01	0.01\\
26.01	0.01\\
27.01	0.01\\
28.01	0.01\\
29.01	0.01\\
30.01	0.01\\
31.01	0.01\\
32.01	0.01\\
33.01	0.01\\
34.01	0.01\\
35.01	0.01\\
36.01	0.01\\
37.01	0.01\\
38.01	0.01\\
39.01	0.01\\
40.01	0.01\\
41.01	0.01\\
42.01	0.01\\
43.01	0.01\\
44.01	0.01\\
45.01	0.01\\
46.01	0.01\\
47.01	0.01\\
48.01	0.01\\
49.01	0.01\\
50.01	0.01\\
51.01	0.01\\
52.01	0.01\\
53.01	0.01\\
54.01	0.01\\
55.01	0.01\\
56.01	0.01\\
57.01	0.01\\
58.01	0.01\\
59.01	0.01\\
60.01	0.01\\
61.01	0.01\\
62.01	0.01\\
63.01	0.01\\
64.01	0.01\\
65.01	0.01\\
66.01	0.01\\
67.01	0.01\\
68.01	0.01\\
69.01	0.01\\
70.01	0.01\\
71.01	0.01\\
72.01	0.01\\
73.01	0.01\\
74.01	0.01\\
75.01	0.01\\
76.01	0.01\\
77.01	0.01\\
78.01	0.01\\
79.01	0.01\\
80.01	0.01\\
81.01	0.01\\
82.01	0.01\\
83.01	0.01\\
84.01	0.01\\
85.01	0.01\\
86.01	0.01\\
87.01	0.01\\
88.01	0.01\\
89.01	0.01\\
90.01	0.01\\
91.01	0.01\\
92.01	0.01\\
93.01	0.01\\
94.01	0.01\\
95.01	0.01\\
96.01	0.01\\
97.01	0.01\\
98.01	0.01\\
99.01	0.00812091055252546\\
99.02	0.00792185855338365\\
99.03	0.00772136441716776\\
99.04	0.00751940674024627\\
99.05	0.00731596371466714\\
99.06	0.0071110129187867\\
99.07	0.0069045312524993\\
99.08	0.00669649491102331\\
99.09	0.00648687935746255\\
99.1	0.00627565929135879\\
99.11	0.00606280862076723\\
99.12	0.00588804584143462\\
99.13	0.00584057973645197\\
99.14	0.00579281442628695\\
99.15	0.00574475348166593\\
99.16	0.00569639881934934\\
99.17	0.00564773260275226\\
99.18	0.00559870678018064\\
99.19	0.00554932343408073\\
99.2	0.00549958489634066\\
99.21	0.00544949375960291\\
99.22	0.00539905288906679\\
99.23	0.00534826543480551\\
99.24	0.00529713484462432\\
99.25	0.00524566487748732\\
99.26	0.00519385961735011\\
99.27	0.00514172348758126\\
99.28	0.00508926126623092\\
99.29	0.00503647810199398\\
99.3	0.00498337953090549\\
99.31	0.00492997151127088\\
99.32	0.00487626043875326\\
99.33	0.00482225315370379\\
99.34	0.00476795696134481\\
99.35	0.00471337965291966\\
99.36	0.00465852952786457\\
99.37	0.00460341541706176\\
99.38	0.00454804670721336\\
99.39	0.0044924333664363\\
99.4	0.00443658597115357\\
99.41	0.00438051573434752\\
99.42	0.00432397001312195\\
99.43	0.00426691221325709\\
99.44	0.00420933777840974\\
99.45	0.00415124214002621\\
99.46	0.00409262069100927\\
99.47	0.00403346878523613\\
99.48	0.00397378173705597\\
99.49	0.00391355482076605\\
99.5	0.00385278327006468\\
99.51	0.00379146227748001\\
99.52	0.00372958699377298\\
99.53	0.00366715252731277\\
99.54	0.0036041539434234\\
99.55	0.00354058626369925\\
99.56	0.00347644446528808\\
99.57	0.00341172348013916\\
99.58	0.00334641819421466\\
99.59	0.0032805234466619\\
99.6	0.0032140340289441\\
99.61	0.00314694468392712\\
99.62	0.00307925010491939\\
99.63	0.00301094493466227\\
99.64	0.00294202376426769\\
99.65	0.00287248113208192\\
99.66	0.00280231155789073\\
99.67	0.00273150950733312\\
99.68	0.00266006939035517\\
99.69	0.00258798555977133\\
99.7	0.0025152523097948\\
99.71	0.00244186387451185\\
99.72	0.00236781442627067\\
99.73	0.00229309807397977\\
99.74	0.00221770886130978\\
99.75	0.0021416407647929\\
99.76	0.00206488769181328\\
99.77	0.00198744347848148\\
99.78	0.00190930188738566\\
99.79	0.00183045660521148\\
99.8	0.00175090124022237\\
99.81	0.00167062931959128\\
99.82	0.00158963428657405\\
99.83	0.00150790949751441\\
99.84	0.00142544821866952\\
99.85	0.00134224362284443\\
99.86	0.00125828878582284\\
99.87	0.00117357668258095\\
99.88	0.00108810018327\\
99.89	0.0010018520489523\\
99.9	0.000914824927074323\\
99.91	0.000827011346659487\\
99.92	0.000738403713201803\\
99.93	0.000648994303240492\\
99.94	0.000558775258594022\\
99.95	0.000467738580230706\\
99.96	0.000375876121751237\\
99.97	0.000283179582456843\\
99.98	0.000189640499974889\\
99.99	9.52502424116652e-05\\
100	0\\
};
\addlegendentry{$q=3$};

\addplot [color=green,solid]
  table[row sep=crcr]{%
0.01	0.01\\
1.01	0.01\\
2.01	0.01\\
3.01	0.01\\
4.01	0.01\\
5.01	0.01\\
6.01	0.01\\
7.01	0.01\\
8.01	0.01\\
9.01	0.01\\
10.01	0.01\\
11.01	0.01\\
12.01	0.01\\
13.01	0.01\\
14.01	0.01\\
15.01	0.01\\
16.01	0.01\\
17.01	0.01\\
18.01	0.01\\
19.01	0.01\\
20.01	0.01\\
21.01	0.01\\
22.01	0.01\\
23.01	0.01\\
24.01	0.01\\
25.01	0.01\\
26.01	0.01\\
27.01	0.01\\
28.01	0.01\\
29.01	0.01\\
30.01	0.01\\
31.01	0.01\\
32.01	0.01\\
33.01	0.01\\
34.01	0.01\\
35.01	0.01\\
36.01	0.01\\
37.01	0.01\\
38.01	0.01\\
39.01	0.01\\
40.01	0.01\\
41.01	0.01\\
42.01	0.01\\
43.01	0.01\\
44.01	0.01\\
45.01	0.01\\
46.01	0.01\\
47.01	0.01\\
48.01	0.01\\
49.01	0.01\\
50.01	0.01\\
51.01	0.01\\
52.01	0.01\\
53.01	0.01\\
54.01	0.01\\
55.01	0.01\\
56.01	0.01\\
57.01	0.01\\
58.01	0.01\\
59.01	0.01\\
60.01	0.01\\
61.01	0.01\\
62.01	0.01\\
63.01	0.01\\
64.01	0.01\\
65.01	0.01\\
66.01	0.01\\
67.01	0.01\\
68.01	0.01\\
69.01	0.01\\
70.01	0.01\\
71.01	0.01\\
72.01	0.01\\
73.01	0.01\\
74.01	0.01\\
75.01	0.01\\
76.01	0.01\\
77.01	0.01\\
78.01	0.01\\
79.01	0.01\\
80.01	0.01\\
81.01	0.01\\
82.01	0.01\\
83.01	0.01\\
84.01	0.01\\
85.01	0.01\\
86.01	0.01\\
87.01	0.01\\
88.01	0.01\\
89.01	0.01\\
90.01	0.01\\
91.01	0.01\\
92.01	0.01\\
93.01	0.01\\
94.01	0.01\\
95.01	0.01\\
96.01	0.01\\
97.01	0.01\\
98.01	0.01\\
99.01	0.00626944196033862\\
99.02	0.00622788811379813\\
99.03	0.00618614667768265\\
99.04	0.00614422623325547\\
99.05	0.00610213589073552\\
99.06	0.0060598853136017\\
99.07	0.00601748474414148\\
99.08	0.00597494503025625\\
99.09	0.00593227765359743\\
99.1	0.00588949475907492\\
99.11	0.00584660918588716\\
99.12	0.00580356314443589\\
99.13	0.00576013259711799\\
99.14	0.00571631383922182\\
99.15	0.00567210310583831\\
99.16	0.00562749657899848\\
99.17	0.00558249047377842\\
99.18	0.00553708115557056\\
99.19	0.00549126494616604\\
99.2	0.0054450381224421\\
99.21	0.00539839691498269\\
99.22	0.00535133750662894\\
99.23	0.00530385603095577\\
99.24	0.00525594857067084\\
99.25	0.00520761115593174\\
99.26	0.00515883976257758\\
99.27	0.00510963031027058\\
99.28	0.00505997866054249\\
99.29	0.00500988061474126\\
99.3	0.00495933191187222\\
99.31	0.0049083282262621\\
99.32	0.00485686516511586\\
99.33	0.00480493826599399\\
99.34	0.00475254299416004\\
99.35	0.00469967473979126\\
99.36	0.00464632881504553\\
99.37	0.00459250045097687\\
99.38	0.00453818479429186\\
99.39	0.00448337690393861\\
99.4	0.00442807174751941\\
99.41	0.00437226419751779\\
99.42	0.00431594958059887\\
99.43	0.00425912324609415\\
99.44	0.0042017805006923\\
99.45	0.00414391660792148\\
99.46	0.00408552678773567\\
99.47	0.00402660621609648\\
99.48	0.00396715002455066\\
99.49	0.00390715329980313\\
99.5	0.00384661108328551\\
99.51	0.00378551837072029\\
99.52	0.00372387011168045\\
99.53	0.00366166120914458\\
99.54	0.00359888651904769\\
99.55	0.00353554084982734\\
99.56	0.00347161896196557\\
99.57	0.00340711556752629\\
99.58	0.00334202532968828\\
99.59	0.00327634286227403\\
99.6	0.0032100627292742\\
99.61	0.0031431794443679\\
99.62	0.00307568747043893\\
99.63	0.00300758121908806\\
99.64	0.00293885505014128\\
99.65	0.00286950327283309\\
99.66	0.00279952017735991\\
99.67	0.00272890000131781\\
99.68	0.00265763692920609\\
99.69	0.00258572509192821\\
99.7	0.00251315856629011\\
99.71	0.0024399313744963\\
99.72	0.00236603748364389\\
99.73	0.002291470805215\\
99.74	0.00221622519456793\\
99.75	0.00214029445042761\\
99.76	0.0020636723143756\\
99.77	0.00198635247034051\\
99.78	0.00190832854408906\\
99.79	0.00182959410271886\\
99.8	0.00175014265415328\\
99.81	0.00166996764663939\\
99.82	0.00158906246824988\\
99.83	0.00150742044638987\\
99.84	0.00142503484730966\\
99.85	0.00134189887562472\\
99.86	0.00125800567384411\\
99.87	0.00117334832190876\\
99.88	0.00108791983674124\\
99.89	0.0010017131718087\\
99.9	0.000914721216700781\\
99.91	0.000826936796724752\\
99.92	0.000738352672519866\\
99.93	0.000648961539693646\\
99.94	0.00055875602848268\\
99.95	0.000467728703440914\\
99.96	0.000375872063158707\\
99.97	0.0002831785400162\\
99.98	0.000189640499974889\\
99.99	9.52502424116652e-05\\
100	0\\
};
\addlegendentry{$q=4$};

\end{axis}
\end{tikzpicture}%

%  \caption{Continuous Time}
%\end{subfigure}%
%\hfill%
%\begin{subfigure}{.45\linewidth}
%  \centering
%  \setlength\figureheight{\linewidth} 
%  \setlength\figurewidth{\linewidth}
%  \tikzsetnextfilename{dm_dscr_z1}
%  % This file was created by matlab2tikz.
%
%The latest updates can be retrieved from
%  http://www.mathworks.com/matlabcentral/fileexchange/22022-matlab2tikz-matlab2tikz
%where you can also make suggestions and rate matlab2tikz.
%
\definecolor{mycolor1}{rgb}{0.00000,1.00000,0.14286}%
\definecolor{mycolor2}{rgb}{0.00000,1.00000,0.28571}%
\definecolor{mycolor3}{rgb}{0.00000,1.00000,0.42857}%
\definecolor{mycolor4}{rgb}{0.00000,1.00000,0.57143}%
\definecolor{mycolor5}{rgb}{0.00000,1.00000,0.71429}%
\definecolor{mycolor6}{rgb}{0.00000,1.00000,0.85714}%
\definecolor{mycolor7}{rgb}{0.00000,1.00000,1.00000}%
\definecolor{mycolor8}{rgb}{0.00000,0.87500,1.00000}%
\definecolor{mycolor9}{rgb}{0.00000,0.62500,1.00000}%
\definecolor{mycolor10}{rgb}{0.12500,0.00000,1.00000}%
\definecolor{mycolor11}{rgb}{0.25000,0.00000,1.00000}%
\definecolor{mycolor12}{rgb}{0.37500,0.00000,1.00000}%
\definecolor{mycolor13}{rgb}{0.50000,0.00000,1.00000}%
\definecolor{mycolor14}{rgb}{0.62500,0.00000,1.00000}%
\definecolor{mycolor15}{rgb}{0.75000,0.00000,1.00000}%
\definecolor{mycolor16}{rgb}{0.87500,0.00000,1.00000}%
\definecolor{mycolor17}{rgb}{1.00000,0.00000,1.00000}%
\definecolor{mycolor18}{rgb}{1.00000,0.00000,0.87500}%
\definecolor{mycolor19}{rgb}{1.00000,0.00000,0.62500}%
\definecolor{mycolor20}{rgb}{0.85714,0.00000,0.00000}%
\definecolor{mycolor21}{rgb}{0.71429,0.00000,0.00000}%
%
\begin{tikzpicture}

\begin{axis}[%
width=4.1in,
height=3.803in,
at={(0.809in,0.513in)},
scale only axis,
point meta min=0,
point meta max=1,
every outer x axis line/.append style={black},
every x tick label/.append style={font=\color{black}},
xmin=0,
xmax=600,
every outer y axis line/.append style={black},
every y tick label/.append style={font=\color{black}},
ymin=0,
ymax=0.007,
axis background/.style={fill=white},
axis x line*=bottom,
axis y line*=left,
colormap={mymap}{[1pt] rgb(0pt)=(0,1,0); rgb(7pt)=(0,1,1); rgb(15pt)=(0,0,1); rgb(23pt)=(1,0,1); rgb(31pt)=(1,0,0); rgb(38pt)=(0,0,0)},
colorbar,
colorbar style={separate axis lines,every outer x axis line/.append style={black},every x tick label/.append style={font=\color{black}},every outer y axis line/.append style={black},every y tick label/.append style={font=\color{black}},yticklabels={{-19},{-17},{-15},{-13},{-11},{-9},{-7},{-5},{-3},{-1},{1},{3},{5},{7},{9},{11},{13},{15},{17},{19}}}
]
\addplot [color=green,solid,forget plot]
  table[row sep=crcr]{%
1	0\\
2	0\\
3	0\\
4	0\\
5	0\\
6	0\\
7	0\\
8	0\\
9	0\\
10	0\\
11	0\\
12	0\\
13	0\\
14	0\\
15	0\\
16	0\\
17	0\\
18	0\\
19	0\\
20	0\\
21	0\\
22	0\\
23	0\\
24	0\\
25	0\\
26	0\\
27	0\\
28	0\\
29	0\\
30	0\\
31	0\\
32	0\\
33	0\\
34	0\\
35	0\\
36	0\\
37	0\\
38	0\\
39	0\\
40	0\\
41	0\\
42	0\\
43	0\\
44	0\\
45	0\\
46	0\\
47	0\\
48	0\\
49	0\\
50	0\\
51	0\\
52	0\\
53	0\\
54	0\\
55	0\\
56	0\\
57	0\\
58	0\\
59	0\\
60	0\\
61	0\\
62	0\\
63	0\\
64	0\\
65	0\\
66	0\\
67	0\\
68	0\\
69	0\\
70	0\\
71	0\\
72	0\\
73	0\\
74	0\\
75	0\\
76	0\\
77	0\\
78	0\\
79	0\\
80	0\\
81	0\\
82	0\\
83	0\\
84	0\\
85	0\\
86	0\\
87	0\\
88	0\\
89	0\\
90	0\\
91	0\\
92	0\\
93	0\\
94	0\\
95	0\\
96	0\\
97	0\\
98	0\\
99	0\\
100	0\\
101	0\\
102	0\\
103	0\\
104	0\\
105	0\\
106	0\\
107	0\\
108	0\\
109	0\\
110	0\\
111	0\\
112	0\\
113	0\\
114	0\\
115	0\\
116	0\\
117	0\\
118	0\\
119	0\\
120	0\\
121	0\\
122	0\\
123	0\\
124	0\\
125	0\\
126	0\\
127	0\\
128	0\\
129	0\\
130	0\\
131	0\\
132	0\\
133	0\\
134	0\\
135	0\\
136	0\\
137	0\\
138	0\\
139	0\\
140	0\\
141	0\\
142	0\\
143	0\\
144	0\\
145	0\\
146	0\\
147	0\\
148	0\\
149	0\\
150	0\\
151	0\\
152	0\\
153	0\\
154	0\\
155	0\\
156	0\\
157	0\\
158	0\\
159	0\\
160	0\\
161	0\\
162	0\\
163	0\\
164	0\\
165	0\\
166	0\\
167	0\\
168	0\\
169	0\\
170	0\\
171	0\\
172	0\\
173	0\\
174	0\\
175	0\\
176	0\\
177	0\\
178	0\\
179	0\\
180	0\\
181	0\\
182	0\\
183	0\\
184	0\\
185	0\\
186	0\\
187	0\\
188	0\\
189	0\\
190	0\\
191	0\\
192	0\\
193	0\\
194	0\\
195	0\\
196	0\\
197	0\\
198	0\\
199	0\\
200	0\\
201	0\\
202	0\\
203	0\\
204	0\\
205	0\\
206	0\\
207	0\\
208	0\\
209	0\\
210	0\\
211	0\\
212	0\\
213	0\\
214	0\\
215	0\\
216	0\\
217	0\\
218	0\\
219	0\\
220	0\\
221	0\\
222	0\\
223	0\\
224	0\\
225	0\\
226	0\\
227	0\\
228	0\\
229	0\\
230	0\\
231	0\\
232	0\\
233	0\\
234	0\\
235	0\\
236	0\\
237	0\\
238	0\\
239	0\\
240	0\\
241	0\\
242	0\\
243	0\\
244	0\\
245	0\\
246	0\\
247	0\\
248	0\\
249	0\\
250	0\\
251	0\\
252	0\\
253	0\\
254	0\\
255	0\\
256	0\\
257	0\\
258	0\\
259	0\\
260	0\\
261	0\\
262	0\\
263	0\\
264	0\\
265	0\\
266	0\\
267	0\\
268	0\\
269	0\\
270	0\\
271	0\\
272	0\\
273	0\\
274	0\\
275	0\\
276	0\\
277	0\\
278	0\\
279	0\\
280	0\\
281	0\\
282	0\\
283	0\\
284	0\\
285	0\\
286	0\\
287	0\\
288	0\\
289	0\\
290	0\\
291	0\\
292	0\\
293	0\\
294	0\\
295	0\\
296	0\\
297	0\\
298	0\\
299	0\\
300	0\\
301	0\\
302	0\\
303	0\\
304	0\\
305	0\\
306	0\\
307	0\\
308	0\\
309	0\\
310	0\\
311	0\\
312	0\\
313	0\\
314	0\\
315	0\\
316	0\\
317	0\\
318	0\\
319	0\\
320	0\\
321	0\\
322	0\\
323	0\\
324	0\\
325	0\\
326	0\\
327	0\\
328	0\\
329	0\\
330	0\\
331	0\\
332	0\\
333	0\\
334	0\\
335	0\\
336	0\\
337	0\\
338	0\\
339	0\\
340	0\\
341	0\\
342	0\\
343	0\\
344	0\\
345	0\\
346	0\\
347	0\\
348	0\\
349	0\\
350	0\\
351	0\\
352	0\\
353	0\\
354	0\\
355	0\\
356	0\\
357	0\\
358	0\\
359	0\\
360	0\\
361	0\\
362	0\\
363	0\\
364	0\\
365	0\\
366	0\\
367	0\\
368	0\\
369	0\\
370	0\\
371	0\\
372	0\\
373	0\\
374	0\\
375	0\\
376	0\\
377	0\\
378	0\\
379	0\\
380	0\\
381	0\\
382	0\\
383	0\\
384	0\\
385	0\\
386	0\\
387	0\\
388	0\\
389	0\\
390	0\\
391	0\\
392	0\\
393	0\\
394	0\\
395	0\\
396	0\\
397	0\\
398	0\\
399	0\\
400	0\\
401	0\\
402	0\\
403	0\\
404	0\\
405	0\\
406	0\\
407	0\\
408	0\\
409	0\\
410	0\\
411	0\\
412	0\\
413	0\\
414	0\\
415	0\\
416	0\\
417	0\\
418	0\\
419	0\\
420	0\\
421	0\\
422	0\\
423	0\\
424	0\\
425	0\\
426	0\\
427	0\\
428	0\\
429	0\\
430	0\\
431	0\\
432	0\\
433	0\\
434	0\\
435	0\\
436	0\\
437	0\\
438	0\\
439	0\\
440	0\\
441	0\\
442	0\\
443	0\\
444	0\\
445	0\\
446	0\\
447	0\\
448	0\\
449	0\\
450	0\\
451	0\\
452	0\\
453	0\\
454	0\\
455	0\\
456	0\\
457	0\\
458	0\\
459	0\\
460	0\\
461	0\\
462	0\\
463	0\\
464	0\\
465	0\\
466	0\\
467	0\\
468	0\\
469	0\\
470	0\\
471	0\\
472	0\\
473	0\\
474	0\\
475	0\\
476	0\\
477	0\\
478	0\\
479	0\\
480	0\\
481	0\\
482	0\\
483	0\\
484	0\\
485	0\\
486	0\\
487	0\\
488	0\\
489	0\\
490	0\\
491	0\\
492	0\\
493	0\\
494	0\\
495	0\\
496	0\\
497	0\\
498	0\\
499	0\\
500	0\\
501	0\\
502	0\\
503	0\\
504	0\\
505	0\\
506	0\\
507	0\\
508	0\\
509	0\\
510	0\\
511	0\\
512	0\\
513	0\\
514	0\\
515	0\\
516	0\\
517	0\\
518	0\\
519	0\\
520	0\\
521	0\\
522	0\\
523	0\\
524	0\\
525	0\\
526	0\\
527	0\\
528	0\\
529	0\\
530	0\\
531	0\\
532	0\\
533	0\\
534	0\\
535	0\\
536	0\\
537	0\\
538	0\\
539	0\\
540	2.05661671419882e-05\\
541	4.80458379165604e-05\\
542	7.61310814796848e-05\\
543	0.000104836069177093\\
544	0.000134173974218699\\
545	0.000164154212091505\\
546	0.000194774568566477\\
547	0.000225999060540562\\
548	0.000257871056294527\\
549	0.000290476767183044\\
550	0.00032384026529897\\
551	0.00036317906641787\\
552	0.000398454290577665\\
553	0.000431034512338747\\
554	0.000463811877737574\\
555	0.000497238740956424\\
556	0.000531333042093673\\
557	0.000566110431097209\\
558	0.000601587038550907\\
559	0.000637779508959063\\
560	0.000674705040954899\\
561	0.000712381474657641\\
562	0.000750827512580226\\
563	0.000790063315042241\\
564	0.000830112145911742\\
565	0.00139458170127532\\
566	0.00148687528982974\\
567	0.00154917163770828\\
568	0.00161250300491599\\
569	0.00167689171273661\\
570	0.00174236090733987\\
571	0.00180893460520923\\
572	0.00187663774080877\\
573	0.00194549621643102\\
574	0.00201553695411736\\
575	0.00208678794948849\\
576	0.0021592783272771\\
577	0.00223303839836233\\
578	0.00230809971827688\\
579	0.00238449514780737\\
580	0.00246225891824811\\
581	0.00254142670923474\\
582	0.00262203576145702\\
583	0.00270412508445197\\
584	0.00278773591889263\\
585	0.00287291287145548\\
586	0.00295970681313861\\
587	0.00304818237891642\\
588	0.00313843743693858\\
589	0.00323065362889276\\
590	0.00332522743498846\\
591	0.00342286644802628\\
592	0.00352518337500009\\
593	0.00363635235366165\\
594	0.00376728283869333\\
595	0.0039465900585172\\
596	0.00424941998802837\\
597	0.00487316246495455\\
598	0.00633625614039688\\
599	0\\
600	0\\
};
\addplot [color=mycolor1,solid,forget plot]
  table[row sep=crcr]{%
1	0\\
2	0\\
3	0\\
4	0\\
5	0\\
6	0\\
7	0\\
8	0\\
9	0\\
10	0\\
11	0\\
12	0\\
13	0\\
14	0\\
15	0\\
16	0\\
17	0\\
18	0\\
19	0\\
20	0\\
21	0\\
22	0\\
23	0\\
24	0\\
25	0\\
26	0\\
27	0\\
28	0\\
29	0\\
30	0\\
31	0\\
32	0\\
33	0\\
34	0\\
35	0\\
36	0\\
37	0\\
38	0\\
39	0\\
40	0\\
41	0\\
42	0\\
43	0\\
44	0\\
45	0\\
46	0\\
47	0\\
48	0\\
49	0\\
50	0\\
51	0\\
52	0\\
53	0\\
54	0\\
55	0\\
56	0\\
57	0\\
58	0\\
59	0\\
60	0\\
61	0\\
62	0\\
63	0\\
64	0\\
65	0\\
66	0\\
67	0\\
68	0\\
69	0\\
70	0\\
71	0\\
72	0\\
73	0\\
74	0\\
75	0\\
76	0\\
77	0\\
78	0\\
79	0\\
80	0\\
81	0\\
82	0\\
83	0\\
84	0\\
85	0\\
86	0\\
87	0\\
88	0\\
89	0\\
90	0\\
91	0\\
92	0\\
93	0\\
94	0\\
95	0\\
96	0\\
97	0\\
98	0\\
99	0\\
100	0\\
101	0\\
102	0\\
103	0\\
104	0\\
105	0\\
106	0\\
107	0\\
108	0\\
109	0\\
110	0\\
111	0\\
112	0\\
113	0\\
114	0\\
115	0\\
116	0\\
117	0\\
118	0\\
119	0\\
120	0\\
121	0\\
122	0\\
123	0\\
124	0\\
125	0\\
126	0\\
127	0\\
128	0\\
129	0\\
130	0\\
131	0\\
132	0\\
133	0\\
134	0\\
135	0\\
136	0\\
137	0\\
138	0\\
139	0\\
140	0\\
141	0\\
142	0\\
143	0\\
144	0\\
145	0\\
146	0\\
147	0\\
148	0\\
149	0\\
150	0\\
151	0\\
152	0\\
153	0\\
154	0\\
155	0\\
156	0\\
157	0\\
158	0\\
159	0\\
160	0\\
161	0\\
162	0\\
163	0\\
164	0\\
165	0\\
166	0\\
167	0\\
168	0\\
169	0\\
170	0\\
171	0\\
172	0\\
173	0\\
174	0\\
175	0\\
176	0\\
177	0\\
178	0\\
179	0\\
180	0\\
181	0\\
182	0\\
183	0\\
184	0\\
185	0\\
186	0\\
187	0\\
188	0\\
189	0\\
190	0\\
191	0\\
192	0\\
193	0\\
194	0\\
195	0\\
196	0\\
197	0\\
198	0\\
199	0\\
200	0\\
201	0\\
202	0\\
203	0\\
204	0\\
205	0\\
206	0\\
207	0\\
208	0\\
209	0\\
210	0\\
211	0\\
212	0\\
213	0\\
214	0\\
215	0\\
216	0\\
217	0\\
218	0\\
219	0\\
220	0\\
221	0\\
222	0\\
223	0\\
224	0\\
225	0\\
226	0\\
227	0\\
228	0\\
229	0\\
230	0\\
231	0\\
232	0\\
233	0\\
234	0\\
235	0\\
236	0\\
237	0\\
238	0\\
239	0\\
240	0\\
241	0\\
242	0\\
243	0\\
244	0\\
245	0\\
246	0\\
247	0\\
248	0\\
249	0\\
250	0\\
251	0\\
252	0\\
253	0\\
254	0\\
255	0\\
256	0\\
257	0\\
258	0\\
259	0\\
260	0\\
261	0\\
262	0\\
263	0\\
264	0\\
265	0\\
266	0\\
267	0\\
268	0\\
269	0\\
270	0\\
271	0\\
272	0\\
273	0\\
274	0\\
275	0\\
276	0\\
277	0\\
278	0\\
279	0\\
280	0\\
281	0\\
282	0\\
283	0\\
284	0\\
285	0\\
286	0\\
287	0\\
288	0\\
289	0\\
290	0\\
291	0\\
292	0\\
293	0\\
294	0\\
295	0\\
296	0\\
297	0\\
298	0\\
299	0\\
300	0\\
301	0\\
302	0\\
303	0\\
304	0\\
305	0\\
306	0\\
307	0\\
308	0\\
309	0\\
310	0\\
311	0\\
312	0\\
313	0\\
314	0\\
315	0\\
316	0\\
317	0\\
318	0\\
319	0\\
320	0\\
321	0\\
322	0\\
323	0\\
324	0\\
325	0\\
326	0\\
327	0\\
328	0\\
329	0\\
330	0\\
331	0\\
332	0\\
333	0\\
334	0\\
335	0\\
336	0\\
337	0\\
338	0\\
339	0\\
340	0\\
341	0\\
342	0\\
343	0\\
344	0\\
345	0\\
346	0\\
347	0\\
348	0\\
349	0\\
350	0\\
351	0\\
352	0\\
353	0\\
354	0\\
355	0\\
356	0\\
357	0\\
358	0\\
359	0\\
360	0\\
361	0\\
362	0\\
363	0\\
364	0\\
365	0\\
366	0\\
367	0\\
368	0\\
369	0\\
370	0\\
371	0\\
372	0\\
373	0\\
374	0\\
375	0\\
376	0\\
377	0\\
378	0\\
379	0\\
380	0\\
381	0\\
382	0\\
383	0\\
384	0\\
385	0\\
386	0\\
387	0\\
388	0\\
389	0\\
390	0\\
391	0\\
392	0\\
393	0\\
394	0\\
395	0\\
396	0\\
397	0\\
398	0\\
399	0\\
400	0\\
401	0\\
402	0\\
403	0\\
404	0\\
405	0\\
406	0\\
407	0\\
408	0\\
409	0\\
410	0\\
411	0\\
412	0\\
413	0\\
414	0\\
415	0\\
416	0\\
417	0\\
418	0\\
419	0\\
420	0\\
421	0\\
422	0\\
423	0\\
424	0\\
425	0\\
426	0\\
427	0\\
428	0\\
429	0\\
430	0\\
431	0\\
432	0\\
433	0\\
434	0\\
435	0\\
436	0\\
437	0\\
438	0\\
439	0\\
440	0\\
441	0\\
442	0\\
443	0\\
444	0\\
445	0\\
446	0\\
447	0\\
448	0\\
449	0\\
450	0\\
451	0\\
452	0\\
453	0\\
454	0\\
455	0\\
456	0\\
457	0\\
458	0\\
459	0\\
460	0\\
461	0\\
462	0\\
463	0\\
464	0\\
465	0\\
466	0\\
467	0\\
468	0\\
469	0\\
470	0\\
471	0\\
472	0\\
473	0\\
474	0\\
475	0\\
476	0\\
477	0\\
478	0\\
479	0\\
480	0\\
481	0\\
482	0\\
483	0\\
484	0\\
485	0\\
486	0\\
487	0\\
488	0\\
489	0\\
490	0\\
491	0\\
492	0\\
493	0\\
494	0\\
495	0\\
496	0\\
497	0\\
498	0\\
499	0\\
500	0\\
501	0\\
502	0\\
503	0\\
504	0\\
505	0\\
506	0\\
507	0\\
508	0\\
509	0\\
510	0\\
511	0\\
512	0\\
513	0\\
514	0\\
515	0\\
516	0\\
517	0\\
518	0\\
519	0\\
520	0\\
521	0\\
522	0\\
523	0\\
524	0\\
525	0\\
526	0\\
527	0\\
528	0\\
529	0\\
530	0\\
531	0\\
532	0\\
533	0\\
534	0\\
535	0\\
536	0\\
537	0\\
538	0\\
539	0\\
540	1.6102870266306e-05\\
541	4.34537338603329e-05\\
542	7.14046837864829e-05\\
543	9.99696745254745e-05\\
544	0.000129161874177511\\
545	0.000158991438199705\\
546	0.000189458996257847\\
547	0.000220536555820973\\
548	0.000252227608453156\\
549	0.000284640600573077\\
550	0.00031779515745435\\
551	0.000351711615590877\\
552	0.000390037246453371\\
553	0.00042835512582672\\
554	0.00046174624242412\\
555	0.000495191349166611\\
556	0.000529232833119001\\
557	0.000563955199341248\\
558	0.000599374606030668\\
559	0.000635507545177786\\
560	0.000672371027359062\\
561	0.000709982635300644\\
562	0.000748360637990346\\
563	0.000787524257000521\\
564	0.000827494274159606\\
565	0.00100020560599384\\
566	0.00146187228657998\\
567	0.00154917163766174\\
568	0.0016125030049125\\
569	0.00167689171273503\\
570	0.00174236090733913\\
571	0.00180893460520911\\
572	0.00187663774080867\\
573	0.00194549621643094\\
574	0.00201553695411732\\
575	0.00208678794948844\\
576	0.0021592783272771\\
577	0.00223303839836235\\
578	0.00230809971827686\\
579	0.00238449514780734\\
580	0.00246225891824808\\
581	0.00254142670923472\\
582	0.002622035761457\\
583	0.00270412508445196\\
584	0.00278773591889263\\
585	0.00287291287145546\\
586	0.0029597068131386\\
587	0.00304818237891641\\
588	0.00313843743693856\\
589	0.00323065362889272\\
590	0.00332522743498843\\
591	0.00342286644802624\\
592	0.00352518337500009\\
593	0.00363635235366164\\
594	0.00376728283869331\\
595	0.00394659005851719\\
596	0.00424941998802836\\
597	0.00487316246495454\\
598	0.00633625614039688\\
599	0\\
600	0\\
};
\addplot [color=mycolor2,solid,forget plot]
  table[row sep=crcr]{%
1	0\\
2	0\\
3	0\\
4	0\\
5	0\\
6	0\\
7	0\\
8	0\\
9	0\\
10	0\\
11	0\\
12	0\\
13	0\\
14	0\\
15	0\\
16	0\\
17	0\\
18	0\\
19	0\\
20	0\\
21	0\\
22	0\\
23	0\\
24	0\\
25	0\\
26	0\\
27	0\\
28	0\\
29	0\\
30	0\\
31	0\\
32	0\\
33	0\\
34	0\\
35	0\\
36	0\\
37	0\\
38	0\\
39	0\\
40	0\\
41	0\\
42	0\\
43	0\\
44	0\\
45	0\\
46	0\\
47	0\\
48	0\\
49	0\\
50	0\\
51	0\\
52	0\\
53	0\\
54	0\\
55	0\\
56	0\\
57	0\\
58	0\\
59	0\\
60	0\\
61	0\\
62	0\\
63	0\\
64	0\\
65	0\\
66	0\\
67	0\\
68	0\\
69	0\\
70	0\\
71	0\\
72	0\\
73	0\\
74	0\\
75	0\\
76	0\\
77	0\\
78	0\\
79	0\\
80	0\\
81	0\\
82	0\\
83	0\\
84	0\\
85	0\\
86	0\\
87	0\\
88	0\\
89	0\\
90	0\\
91	0\\
92	0\\
93	0\\
94	0\\
95	0\\
96	0\\
97	0\\
98	0\\
99	0\\
100	0\\
101	0\\
102	0\\
103	0\\
104	0\\
105	0\\
106	0\\
107	0\\
108	0\\
109	0\\
110	0\\
111	0\\
112	0\\
113	0\\
114	0\\
115	0\\
116	0\\
117	0\\
118	0\\
119	0\\
120	0\\
121	0\\
122	0\\
123	0\\
124	0\\
125	0\\
126	0\\
127	0\\
128	0\\
129	0\\
130	0\\
131	0\\
132	0\\
133	0\\
134	0\\
135	0\\
136	0\\
137	0\\
138	0\\
139	0\\
140	0\\
141	0\\
142	0\\
143	0\\
144	0\\
145	0\\
146	0\\
147	0\\
148	0\\
149	0\\
150	0\\
151	0\\
152	0\\
153	0\\
154	0\\
155	0\\
156	0\\
157	0\\
158	0\\
159	0\\
160	0\\
161	0\\
162	0\\
163	0\\
164	0\\
165	0\\
166	0\\
167	0\\
168	0\\
169	0\\
170	0\\
171	0\\
172	0\\
173	0\\
174	0\\
175	0\\
176	0\\
177	0\\
178	0\\
179	0\\
180	0\\
181	0\\
182	0\\
183	0\\
184	0\\
185	0\\
186	0\\
187	0\\
188	0\\
189	0\\
190	0\\
191	0\\
192	0\\
193	0\\
194	0\\
195	0\\
196	0\\
197	0\\
198	0\\
199	0\\
200	0\\
201	0\\
202	0\\
203	0\\
204	0\\
205	0\\
206	0\\
207	0\\
208	0\\
209	0\\
210	0\\
211	0\\
212	0\\
213	0\\
214	0\\
215	0\\
216	0\\
217	0\\
218	0\\
219	0\\
220	0\\
221	0\\
222	0\\
223	0\\
224	0\\
225	0\\
226	0\\
227	0\\
228	0\\
229	0\\
230	0\\
231	0\\
232	0\\
233	0\\
234	0\\
235	0\\
236	0\\
237	0\\
238	0\\
239	0\\
240	0\\
241	0\\
242	0\\
243	0\\
244	0\\
245	0\\
246	0\\
247	0\\
248	0\\
249	0\\
250	0\\
251	0\\
252	0\\
253	0\\
254	0\\
255	0\\
256	0\\
257	0\\
258	0\\
259	0\\
260	0\\
261	0\\
262	0\\
263	0\\
264	0\\
265	0\\
266	0\\
267	0\\
268	0\\
269	0\\
270	0\\
271	0\\
272	0\\
273	0\\
274	0\\
275	0\\
276	0\\
277	0\\
278	0\\
279	0\\
280	0\\
281	0\\
282	0\\
283	0\\
284	0\\
285	0\\
286	0\\
287	0\\
288	0\\
289	0\\
290	0\\
291	0\\
292	0\\
293	0\\
294	0\\
295	0\\
296	0\\
297	0\\
298	0\\
299	0\\
300	0\\
301	0\\
302	0\\
303	0\\
304	0\\
305	0\\
306	0\\
307	0\\
308	0\\
309	0\\
310	0\\
311	0\\
312	0\\
313	0\\
314	0\\
315	0\\
316	0\\
317	0\\
318	0\\
319	0\\
320	0\\
321	0\\
322	0\\
323	0\\
324	0\\
325	0\\
326	0\\
327	0\\
328	0\\
329	0\\
330	0\\
331	0\\
332	0\\
333	0\\
334	0\\
335	0\\
336	0\\
337	0\\
338	0\\
339	0\\
340	0\\
341	0\\
342	0\\
343	0\\
344	0\\
345	0\\
346	0\\
347	0\\
348	0\\
349	0\\
350	0\\
351	0\\
352	0\\
353	0\\
354	0\\
355	0\\
356	0\\
357	0\\
358	0\\
359	0\\
360	0\\
361	0\\
362	0\\
363	0\\
364	0\\
365	0\\
366	0\\
367	0\\
368	0\\
369	0\\
370	0\\
371	0\\
372	0\\
373	0\\
374	0\\
375	0\\
376	0\\
377	0\\
378	0\\
379	0\\
380	0\\
381	0\\
382	0\\
383	0\\
384	0\\
385	0\\
386	0\\
387	0\\
388	0\\
389	0\\
390	0\\
391	0\\
392	0\\
393	0\\
394	0\\
395	0\\
396	0\\
397	0\\
398	0\\
399	0\\
400	0\\
401	0\\
402	0\\
403	0\\
404	0\\
405	0\\
406	0\\
407	0\\
408	0\\
409	0\\
410	0\\
411	0\\
412	0\\
413	0\\
414	0\\
415	0\\
416	0\\
417	0\\
418	0\\
419	0\\
420	0\\
421	0\\
422	0\\
423	0\\
424	0\\
425	0\\
426	0\\
427	0\\
428	0\\
429	0\\
430	0\\
431	0\\
432	0\\
433	0\\
434	0\\
435	0\\
436	0\\
437	0\\
438	0\\
439	0\\
440	0\\
441	0\\
442	0\\
443	0\\
444	0\\
445	0\\
446	0\\
447	0\\
448	0\\
449	0\\
450	0\\
451	0\\
452	0\\
453	0\\
454	0\\
455	0\\
456	0\\
457	0\\
458	0\\
459	0\\
460	0\\
461	0\\
462	0\\
463	0\\
464	0\\
465	0\\
466	0\\
467	0\\
468	0\\
469	0\\
470	0\\
471	0\\
472	0\\
473	0\\
474	0\\
475	0\\
476	0\\
477	0\\
478	0\\
479	0\\
480	0\\
481	0\\
482	0\\
483	0\\
484	0\\
485	0\\
486	0\\
487	0\\
488	0\\
489	0\\
490	0\\
491	0\\
492	0\\
493	0\\
494	0\\
495	0\\
496	0\\
497	0\\
498	0\\
499	0\\
500	0\\
501	0\\
502	0\\
503	0\\
504	0\\
505	0\\
506	0\\
507	0\\
508	0\\
509	0\\
510	0\\
511	0\\
512	0\\
513	0\\
514	0\\
515	0\\
516	0\\
517	0\\
518	0\\
519	0\\
520	0\\
521	0\\
522	0\\
523	0\\
524	0\\
525	0\\
526	0\\
527	0\\
528	0\\
529	0\\
530	0\\
531	0\\
532	0\\
533	0\\
534	0\\
535	0\\
536	0\\
537	0\\
538	0\\
539	0\\
540	1.043575262142e-05\\
541	3.76730705218496e-05\\
542	6.54984399976953e-05\\
543	9.39344174334583e-05\\
544	0.000122995254933542\\
545	0.000152692851212342\\
546	0.000183032131229788\\
547	0.000213998059291118\\
548	0.000245546201215325\\
549	0.00027781218108877\\
550	0.000310816922285056\\
551	0.000344581191504035\\
552	0.00037912545903052\\
553	0.000416165495955848\\
554	0.000457424449126694\\
555	0.000492300281589155\\
556	0.000526643713568248\\
557	0.000561320305433605\\
558	0.000596690131270986\\
559	0.000632772800261207\\
560	0.000669585437892132\\
561	0.000707145693984105\\
562	0.000745471835422497\\
563	0.000784582869067791\\
564	0.000824498821323294\\
565	0.000865241434196189\\
566	0.00113827341020856\\
567	0.0015274863410712\\
568	0.00161250300428452\\
569	0.00167689171270879\\
570	0.00174236090732779\\
571	0.00180893460520347\\
572	0.00187663774080614\\
573	0.00194549621642988\\
574	0.00201553695411676\\
575	0.00208678794948843\\
576	0.00215927832727709\\
577	0.00223303839836236\\
578	0.00230809971827687\\
579	0.00238449514780738\\
580	0.00246225891824811\\
581	0.00254142670923472\\
582	0.002622035761457\\
583	0.00270412508445197\\
584	0.00278773591889263\\
585	0.00287291287145548\\
586	0.0029597068131386\\
587	0.00304818237891639\\
588	0.00313843743693856\\
589	0.00323065362889273\\
590	0.00332522743498844\\
591	0.00342286644802626\\
592	0.00352518337500008\\
593	0.00363635235366164\\
594	0.00376728283869331\\
595	0.00394659005851718\\
596	0.00424941998802835\\
597	0.00487316246495454\\
598	0.00633625614039688\\
599	0\\
600	0\\
};
\addplot [color=mycolor3,solid,forget plot]
  table[row sep=crcr]{%
1	0\\
2	0\\
3	0\\
4	0\\
5	0\\
6	0\\
7	0\\
8	0\\
9	0\\
10	0\\
11	0\\
12	0\\
13	0\\
14	0\\
15	0\\
16	0\\
17	0\\
18	0\\
19	0\\
20	0\\
21	0\\
22	0\\
23	0\\
24	0\\
25	0\\
26	0\\
27	0\\
28	0\\
29	0\\
30	0\\
31	0\\
32	0\\
33	0\\
34	0\\
35	0\\
36	0\\
37	0\\
38	0\\
39	0\\
40	0\\
41	0\\
42	0\\
43	0\\
44	0\\
45	0\\
46	0\\
47	0\\
48	0\\
49	0\\
50	0\\
51	0\\
52	0\\
53	0\\
54	0\\
55	0\\
56	0\\
57	0\\
58	0\\
59	0\\
60	0\\
61	0\\
62	0\\
63	0\\
64	0\\
65	0\\
66	0\\
67	0\\
68	0\\
69	0\\
70	0\\
71	0\\
72	0\\
73	0\\
74	0\\
75	0\\
76	0\\
77	0\\
78	0\\
79	0\\
80	0\\
81	0\\
82	0\\
83	0\\
84	0\\
85	0\\
86	0\\
87	0\\
88	0\\
89	0\\
90	0\\
91	0\\
92	0\\
93	0\\
94	0\\
95	0\\
96	0\\
97	0\\
98	0\\
99	0\\
100	0\\
101	0\\
102	0\\
103	0\\
104	0\\
105	0\\
106	0\\
107	0\\
108	0\\
109	0\\
110	0\\
111	0\\
112	0\\
113	0\\
114	0\\
115	0\\
116	0\\
117	0\\
118	0\\
119	0\\
120	0\\
121	0\\
122	0\\
123	0\\
124	0\\
125	0\\
126	0\\
127	0\\
128	0\\
129	0\\
130	0\\
131	0\\
132	0\\
133	0\\
134	0\\
135	0\\
136	0\\
137	0\\
138	0\\
139	0\\
140	0\\
141	0\\
142	0\\
143	0\\
144	0\\
145	0\\
146	0\\
147	0\\
148	0\\
149	0\\
150	0\\
151	0\\
152	0\\
153	0\\
154	0\\
155	0\\
156	0\\
157	0\\
158	0\\
159	0\\
160	0\\
161	0\\
162	0\\
163	0\\
164	0\\
165	0\\
166	0\\
167	0\\
168	0\\
169	0\\
170	0\\
171	0\\
172	0\\
173	0\\
174	0\\
175	0\\
176	0\\
177	0\\
178	0\\
179	0\\
180	0\\
181	0\\
182	0\\
183	0\\
184	0\\
185	0\\
186	0\\
187	0\\
188	0\\
189	0\\
190	0\\
191	0\\
192	0\\
193	0\\
194	0\\
195	0\\
196	0\\
197	0\\
198	0\\
199	0\\
200	0\\
201	0\\
202	0\\
203	0\\
204	0\\
205	0\\
206	0\\
207	0\\
208	0\\
209	0\\
210	0\\
211	0\\
212	0\\
213	0\\
214	0\\
215	0\\
216	0\\
217	0\\
218	0\\
219	0\\
220	0\\
221	0\\
222	0\\
223	0\\
224	0\\
225	0\\
226	0\\
227	0\\
228	0\\
229	0\\
230	0\\
231	0\\
232	0\\
233	0\\
234	0\\
235	0\\
236	0\\
237	0\\
238	0\\
239	0\\
240	0\\
241	0\\
242	0\\
243	0\\
244	0\\
245	0\\
246	0\\
247	0\\
248	0\\
249	0\\
250	0\\
251	0\\
252	0\\
253	0\\
254	0\\
255	0\\
256	0\\
257	0\\
258	0\\
259	0\\
260	0\\
261	0\\
262	0\\
263	0\\
264	0\\
265	0\\
266	0\\
267	0\\
268	0\\
269	0\\
270	0\\
271	0\\
272	0\\
273	0\\
274	0\\
275	0\\
276	0\\
277	0\\
278	0\\
279	0\\
280	0\\
281	0\\
282	0\\
283	0\\
284	0\\
285	0\\
286	0\\
287	0\\
288	0\\
289	0\\
290	0\\
291	0\\
292	0\\
293	0\\
294	0\\
295	0\\
296	0\\
297	0\\
298	0\\
299	0\\
300	0\\
301	0\\
302	0\\
303	0\\
304	0\\
305	0\\
306	0\\
307	0\\
308	0\\
309	0\\
310	0\\
311	0\\
312	0\\
313	0\\
314	0\\
315	0\\
316	0\\
317	0\\
318	0\\
319	0\\
320	0\\
321	0\\
322	0\\
323	0\\
324	0\\
325	0\\
326	0\\
327	0\\
328	0\\
329	0\\
330	0\\
331	0\\
332	0\\
333	0\\
334	0\\
335	0\\
336	0\\
337	0\\
338	0\\
339	0\\
340	0\\
341	0\\
342	0\\
343	0\\
344	0\\
345	0\\
346	0\\
347	0\\
348	0\\
349	0\\
350	0\\
351	0\\
352	0\\
353	0\\
354	0\\
355	0\\
356	0\\
357	0\\
358	0\\
359	0\\
360	0\\
361	0\\
362	0\\
363	0\\
364	0\\
365	0\\
366	0\\
367	0\\
368	0\\
369	0\\
370	0\\
371	0\\
372	0\\
373	0\\
374	0\\
375	0\\
376	0\\
377	0\\
378	0\\
379	0\\
380	0\\
381	0\\
382	0\\
383	0\\
384	0\\
385	0\\
386	0\\
387	0\\
388	0\\
389	0\\
390	0\\
391	0\\
392	0\\
393	0\\
394	0\\
395	0\\
396	0\\
397	0\\
398	0\\
399	0\\
400	0\\
401	0\\
402	0\\
403	0\\
404	0\\
405	0\\
406	0\\
407	0\\
408	0\\
409	0\\
410	0\\
411	0\\
412	0\\
413	0\\
414	0\\
415	0\\
416	0\\
417	0\\
418	0\\
419	0\\
420	0\\
421	0\\
422	0\\
423	0\\
424	0\\
425	0\\
426	0\\
427	0\\
428	0\\
429	0\\
430	0\\
431	0\\
432	0\\
433	0\\
434	0\\
435	0\\
436	0\\
437	0\\
438	0\\
439	0\\
440	0\\
441	0\\
442	0\\
443	0\\
444	0\\
445	0\\
446	0\\
447	0\\
448	0\\
449	0\\
450	0\\
451	0\\
452	0\\
453	0\\
454	0\\
455	0\\
456	0\\
457	0\\
458	0\\
459	0\\
460	0\\
461	0\\
462	0\\
463	0\\
464	0\\
465	0\\
466	0\\
467	0\\
468	0\\
469	0\\
470	0\\
471	0\\
472	0\\
473	0\\
474	0\\
475	0\\
476	0\\
477	0\\
478	0\\
479	0\\
480	0\\
481	0\\
482	0\\
483	0\\
484	0\\
485	0\\
486	0\\
487	0\\
488	0\\
489	0\\
490	0\\
491	0\\
492	0\\
493	0\\
494	0\\
495	0\\
496	0\\
497	0\\
498	0\\
499	0\\
500	0\\
501	0\\
502	0\\
503	0\\
504	0\\
505	0\\
506	0\\
507	0\\
508	0\\
509	0\\
510	0\\
511	0\\
512	0\\
513	0\\
514	0\\
515	0\\
516	0\\
517	0\\
518	0\\
519	0\\
520	0\\
521	0\\
522	0\\
523	0\\
524	0\\
525	0\\
526	0\\
527	0\\
528	0\\
529	0\\
530	0\\
531	0\\
532	0\\
533	0\\
534	0\\
535	0\\
536	0\\
537	0\\
538	0\\
539	0\\
540	3.48797025866081e-06\\
541	3.10361975630429e-05\\
542	5.89987522129768e-05\\
543	8.73188240418785e-05\\
544	0.000116246767166137\\
545	0.000145801218191624\\
546	0.0001759952136604\\
547	0.000206821116582927\\
548	0.000238226463510221\\
549	0.000270319567015469\\
550	0.000303143125697196\\
551	0.000336716195874316\\
552	0.000371058848398407\\
553	0.00040619113801879\\
554	0.000442149506879859\\
555	0.000484062239463527\\
556	0.000523027009035641\\
557	0.000558307319211228\\
558	0.000593697063588013\\
559	0.000629724921195129\\
560	0.000666480983474951\\
561	0.000703983518743129\\
562	0.000742250808159385\\
563	0.000781301765229793\\
564	0.000821156099512044\\
565	0.00086183464309347\\
566	0.000903359467058362\\
567	0.00124194869524459\\
568	0.0015912465992359\\
569	0.00167689170233861\\
570	0.00174236090712023\\
571	0.00180893460512034\\
572	0.00187663774076567\\
573	0.00194549621641082\\
574	0.00201553695410846\\
575	0.00208678794948483\\
576	0.00215927832727575\\
577	0.00223303839836177\\
578	0.00230809971827685\\
579	0.00238449514780736\\
580	0.00246225891824812\\
581	0.00254142670923474\\
582	0.00262203576145702\\
583	0.00270412508445197\\
584	0.00278773591889264\\
585	0.00287291287145547\\
586	0.00295970681313861\\
587	0.00304818237891642\\
588	0.00313843743693858\\
589	0.00323065362889276\\
590	0.00332522743498845\\
591	0.00342286644802627\\
592	0.00352518337500009\\
593	0.00363635235366166\\
594	0.00376728283869333\\
595	0.0039465900585172\\
596	0.00424941998802837\\
597	0.00487316246495455\\
598	0.00633625614039688\\
599	0\\
600	0\\
};
\addplot [color=mycolor4,solid,forget plot]
  table[row sep=crcr]{%
1	0\\
2	0\\
3	0\\
4	0\\
5	0\\
6	0\\
7	0\\
8	0\\
9	0\\
10	0\\
11	0\\
12	0\\
13	0\\
14	0\\
15	0\\
16	0\\
17	0\\
18	0\\
19	0\\
20	0\\
21	0\\
22	0\\
23	0\\
24	0\\
25	0\\
26	0\\
27	0\\
28	0\\
29	0\\
30	0\\
31	0\\
32	0\\
33	0\\
34	0\\
35	0\\
36	0\\
37	0\\
38	0\\
39	0\\
40	0\\
41	0\\
42	0\\
43	0\\
44	0\\
45	0\\
46	0\\
47	0\\
48	0\\
49	0\\
50	0\\
51	0\\
52	0\\
53	0\\
54	0\\
55	0\\
56	0\\
57	0\\
58	0\\
59	0\\
60	0\\
61	0\\
62	0\\
63	0\\
64	0\\
65	0\\
66	0\\
67	0\\
68	0\\
69	0\\
70	0\\
71	0\\
72	0\\
73	0\\
74	0\\
75	0\\
76	0\\
77	0\\
78	0\\
79	0\\
80	0\\
81	0\\
82	0\\
83	0\\
84	0\\
85	0\\
86	0\\
87	0\\
88	0\\
89	0\\
90	0\\
91	0\\
92	0\\
93	0\\
94	0\\
95	0\\
96	0\\
97	0\\
98	0\\
99	0\\
100	0\\
101	0\\
102	0\\
103	0\\
104	0\\
105	0\\
106	0\\
107	0\\
108	0\\
109	0\\
110	0\\
111	0\\
112	0\\
113	0\\
114	0\\
115	0\\
116	0\\
117	0\\
118	0\\
119	0\\
120	0\\
121	0\\
122	0\\
123	0\\
124	0\\
125	0\\
126	0\\
127	0\\
128	0\\
129	0\\
130	0\\
131	0\\
132	0\\
133	0\\
134	0\\
135	0\\
136	0\\
137	0\\
138	0\\
139	0\\
140	0\\
141	0\\
142	0\\
143	0\\
144	0\\
145	0\\
146	0\\
147	0\\
148	0\\
149	0\\
150	0\\
151	0\\
152	0\\
153	0\\
154	0\\
155	0\\
156	0\\
157	0\\
158	0\\
159	0\\
160	0\\
161	0\\
162	0\\
163	0\\
164	0\\
165	0\\
166	0\\
167	0\\
168	0\\
169	0\\
170	0\\
171	0\\
172	0\\
173	0\\
174	0\\
175	0\\
176	0\\
177	0\\
178	0\\
179	0\\
180	0\\
181	0\\
182	0\\
183	0\\
184	0\\
185	0\\
186	0\\
187	0\\
188	0\\
189	0\\
190	0\\
191	0\\
192	0\\
193	0\\
194	0\\
195	0\\
196	0\\
197	0\\
198	0\\
199	0\\
200	0\\
201	0\\
202	0\\
203	0\\
204	0\\
205	0\\
206	0\\
207	0\\
208	0\\
209	0\\
210	0\\
211	0\\
212	0\\
213	0\\
214	0\\
215	0\\
216	0\\
217	0\\
218	0\\
219	0\\
220	0\\
221	0\\
222	0\\
223	0\\
224	0\\
225	0\\
226	0\\
227	0\\
228	0\\
229	0\\
230	0\\
231	0\\
232	0\\
233	0\\
234	0\\
235	0\\
236	0\\
237	0\\
238	0\\
239	0\\
240	0\\
241	0\\
242	0\\
243	0\\
244	0\\
245	0\\
246	0\\
247	0\\
248	0\\
249	0\\
250	0\\
251	0\\
252	0\\
253	0\\
254	0\\
255	0\\
256	0\\
257	0\\
258	0\\
259	0\\
260	0\\
261	0\\
262	0\\
263	0\\
264	0\\
265	0\\
266	0\\
267	0\\
268	0\\
269	0\\
270	0\\
271	0\\
272	0\\
273	0\\
274	0\\
275	0\\
276	0\\
277	0\\
278	0\\
279	0\\
280	0\\
281	0\\
282	0\\
283	0\\
284	0\\
285	0\\
286	0\\
287	0\\
288	0\\
289	0\\
290	0\\
291	0\\
292	0\\
293	0\\
294	0\\
295	0\\
296	0\\
297	0\\
298	0\\
299	0\\
300	0\\
301	0\\
302	0\\
303	0\\
304	0\\
305	0\\
306	0\\
307	0\\
308	0\\
309	0\\
310	0\\
311	0\\
312	0\\
313	0\\
314	0\\
315	0\\
316	0\\
317	0\\
318	0\\
319	0\\
320	0\\
321	0\\
322	0\\
323	0\\
324	0\\
325	0\\
326	0\\
327	0\\
328	0\\
329	0\\
330	0\\
331	0\\
332	0\\
333	0\\
334	0\\
335	0\\
336	0\\
337	0\\
338	0\\
339	0\\
340	0\\
341	0\\
342	0\\
343	0\\
344	0\\
345	0\\
346	0\\
347	0\\
348	0\\
349	0\\
350	0\\
351	0\\
352	0\\
353	0\\
354	0\\
355	0\\
356	0\\
357	0\\
358	0\\
359	0\\
360	0\\
361	0\\
362	0\\
363	0\\
364	0\\
365	0\\
366	0\\
367	0\\
368	0\\
369	0\\
370	0\\
371	0\\
372	0\\
373	0\\
374	0\\
375	0\\
376	0\\
377	0\\
378	0\\
379	0\\
380	0\\
381	0\\
382	0\\
383	0\\
384	0\\
385	0\\
386	0\\
387	0\\
388	0\\
389	0\\
390	0\\
391	0\\
392	0\\
393	0\\
394	0\\
395	0\\
396	0\\
397	0\\
398	0\\
399	0\\
400	0\\
401	0\\
402	0\\
403	0\\
404	0\\
405	0\\
406	0\\
407	0\\
408	0\\
409	0\\
410	0\\
411	0\\
412	0\\
413	0\\
414	0\\
415	0\\
416	0\\
417	0\\
418	0\\
419	0\\
420	0\\
421	0\\
422	0\\
423	0\\
424	0\\
425	0\\
426	0\\
427	0\\
428	0\\
429	0\\
430	0\\
431	0\\
432	0\\
433	0\\
434	0\\
435	0\\
436	0\\
437	0\\
438	0\\
439	0\\
440	0\\
441	0\\
442	0\\
443	0\\
444	0\\
445	0\\
446	0\\
447	0\\
448	0\\
449	0\\
450	0\\
451	0\\
452	0\\
453	0\\
454	0\\
455	0\\
456	0\\
457	0\\
458	0\\
459	0\\
460	0\\
461	0\\
462	0\\
463	0\\
464	0\\
465	0\\
466	0\\
467	0\\
468	0\\
469	0\\
470	0\\
471	0\\
472	0\\
473	0\\
474	0\\
475	0\\
476	0\\
477	0\\
478	0\\
479	0\\
480	0\\
481	0\\
482	0\\
483	0\\
484	0\\
485	0\\
486	0\\
487	0\\
488	0\\
489	0\\
490	0\\
491	0\\
492	0\\
493	0\\
494	0\\
495	0\\
496	0\\
497	0\\
498	0\\
499	0\\
500	0\\
501	0\\
502	0\\
503	0\\
504	0\\
505	0\\
506	0\\
507	0\\
508	0\\
509	0\\
510	0\\
511	0\\
512	0\\
513	0\\
514	0\\
515	0\\
516	0\\
517	0\\
518	0\\
519	0\\
520	0\\
521	0\\
522	0\\
523	0\\
524	0\\
525	0\\
526	0\\
527	0\\
528	0\\
529	0\\
530	0\\
531	0\\
532	0\\
533	0\\
534	0\\
535	0\\
536	0\\
537	0\\
538	0\\
539	0\\
540	0\\
541	1.96468212964802e-05\\
542	5.09270591881137e-05\\
543	7.96678779271732e-05\\
544	0.000108857158611573\\
545	0.000138469423037538\\
546	0.000168547998253239\\
547	0.000199251293832798\\
548	0.000230546998687823\\
549	0.000262493736100757\\
550	0.000295169201149189\\
551	0.000328592928239676\\
552	0.000362785127969851\\
553	0.000397767066658495\\
554	0.000433561181336885\\
555	0.000470189197593691\\
556	0.000510120726160936\\
557	0.000553077463148308\\
558	0.00058994417211552\\
559	0.000626359984799716\\
560	0.000663065711723231\\
561	0.000700504825484549\\
562	0.00073870671292666\\
563	0.00077769066007184\\
564	0.000817476258749015\\
565	0.000858083918159465\\
566	0.000899535269842526\\
567	0.000941852816480089\\
568	0.00131032407389147\\
569	0.00165300818929824\\
570	0.00174236067392589\\
571	0.00180893460337395\\
572	0.00187663774017175\\
573	0.00194549621611769\\
574	0.00201553695396648\\
575	0.0020867879494197\\
576	0.0021592783272479\\
577	0.00223303839835099\\
578	0.00230809971827289\\
579	0.00238449514780614\\
580	0.00246225891824759\\
581	0.00254142670923473\\
582	0.00262203576145702\\
583	0.00270412508445197\\
584	0.00278773591889265\\
585	0.00287291287145549\\
586	0.00295970681313861\\
587	0.00304818237891641\\
588	0.00313843743693858\\
589	0.00323065362889274\\
590	0.00332522743498846\\
591	0.00342286644802626\\
592	0.0035251833750001\\
593	0.00363635235366166\\
594	0.00376728283869333\\
595	0.0039465900585172\\
596	0.00424941998802838\\
597	0.00487316246495455\\
598	0.00633625614039688\\
599	0\\
600	0\\
};
\addplot [color=mycolor5,solid,forget plot]
  table[row sep=crcr]{%
1	0\\
2	0\\
3	0\\
4	0\\
5	0\\
6	0\\
7	0\\
8	0\\
9	0\\
10	0\\
11	0\\
12	0\\
13	0\\
14	0\\
15	0\\
16	0\\
17	0\\
18	0\\
19	0\\
20	0\\
21	0\\
22	0\\
23	0\\
24	0\\
25	0\\
26	0\\
27	0\\
28	0\\
29	0\\
30	0\\
31	0\\
32	0\\
33	0\\
34	0\\
35	0\\
36	0\\
37	0\\
38	0\\
39	0\\
40	0\\
41	0\\
42	0\\
43	0\\
44	0\\
45	0\\
46	0\\
47	0\\
48	0\\
49	0\\
50	0\\
51	0\\
52	0\\
53	0\\
54	0\\
55	0\\
56	0\\
57	0\\
58	0\\
59	0\\
60	0\\
61	0\\
62	0\\
63	0\\
64	0\\
65	0\\
66	0\\
67	0\\
68	0\\
69	0\\
70	0\\
71	0\\
72	0\\
73	0\\
74	0\\
75	0\\
76	0\\
77	0\\
78	0\\
79	0\\
80	0\\
81	0\\
82	0\\
83	0\\
84	0\\
85	0\\
86	0\\
87	0\\
88	0\\
89	0\\
90	0\\
91	0\\
92	0\\
93	0\\
94	0\\
95	0\\
96	0\\
97	0\\
98	0\\
99	0\\
100	0\\
101	0\\
102	0\\
103	0\\
104	0\\
105	0\\
106	0\\
107	0\\
108	0\\
109	0\\
110	0\\
111	0\\
112	0\\
113	0\\
114	0\\
115	0\\
116	0\\
117	0\\
118	0\\
119	0\\
120	0\\
121	0\\
122	0\\
123	0\\
124	0\\
125	0\\
126	0\\
127	0\\
128	0\\
129	0\\
130	0\\
131	0\\
132	0\\
133	0\\
134	0\\
135	0\\
136	0\\
137	0\\
138	0\\
139	0\\
140	0\\
141	0\\
142	0\\
143	0\\
144	0\\
145	0\\
146	0\\
147	0\\
148	0\\
149	0\\
150	0\\
151	0\\
152	0\\
153	0\\
154	0\\
155	0\\
156	0\\
157	0\\
158	0\\
159	0\\
160	0\\
161	0\\
162	0\\
163	0\\
164	0\\
165	0\\
166	0\\
167	0\\
168	0\\
169	0\\
170	0\\
171	0\\
172	0\\
173	0\\
174	0\\
175	0\\
176	0\\
177	0\\
178	0\\
179	0\\
180	0\\
181	0\\
182	0\\
183	0\\
184	0\\
185	0\\
186	0\\
187	0\\
188	0\\
189	0\\
190	0\\
191	0\\
192	0\\
193	0\\
194	0\\
195	0\\
196	0\\
197	0\\
198	0\\
199	0\\
200	0\\
201	0\\
202	0\\
203	0\\
204	0\\
205	0\\
206	0\\
207	0\\
208	0\\
209	0\\
210	0\\
211	0\\
212	0\\
213	0\\
214	0\\
215	0\\
216	0\\
217	0\\
218	0\\
219	0\\
220	0\\
221	0\\
222	0\\
223	0\\
224	0\\
225	0\\
226	0\\
227	0\\
228	0\\
229	0\\
230	0\\
231	0\\
232	0\\
233	0\\
234	0\\
235	0\\
236	0\\
237	0\\
238	0\\
239	0\\
240	0\\
241	0\\
242	0\\
243	0\\
244	0\\
245	0\\
246	0\\
247	0\\
248	0\\
249	0\\
250	0\\
251	0\\
252	0\\
253	0\\
254	0\\
255	0\\
256	0\\
257	0\\
258	0\\
259	0\\
260	0\\
261	0\\
262	0\\
263	0\\
264	0\\
265	0\\
266	0\\
267	0\\
268	0\\
269	0\\
270	0\\
271	0\\
272	0\\
273	0\\
274	0\\
275	0\\
276	0\\
277	0\\
278	0\\
279	0\\
280	0\\
281	0\\
282	0\\
283	0\\
284	0\\
285	0\\
286	0\\
287	0\\
288	0\\
289	0\\
290	0\\
291	0\\
292	0\\
293	0\\
294	0\\
295	0\\
296	0\\
297	0\\
298	0\\
299	0\\
300	0\\
301	0\\
302	0\\
303	0\\
304	0\\
305	0\\
306	0\\
307	0\\
308	0\\
309	0\\
310	0\\
311	0\\
312	0\\
313	0\\
314	0\\
315	0\\
316	0\\
317	0\\
318	0\\
319	0\\
320	0\\
321	0\\
322	0\\
323	0\\
324	0\\
325	0\\
326	0\\
327	0\\
328	0\\
329	0\\
330	0\\
331	0\\
332	0\\
333	0\\
334	0\\
335	0\\
336	0\\
337	0\\
338	0\\
339	0\\
340	0\\
341	0\\
342	0\\
343	0\\
344	0\\
345	0\\
346	0\\
347	0\\
348	0\\
349	0\\
350	0\\
351	0\\
352	0\\
353	0\\
354	0\\
355	0\\
356	0\\
357	0\\
358	0\\
359	0\\
360	0\\
361	0\\
362	0\\
363	0\\
364	0\\
365	0\\
366	0\\
367	0\\
368	0\\
369	0\\
370	0\\
371	0\\
372	0\\
373	0\\
374	0\\
375	0\\
376	0\\
377	0\\
378	0\\
379	0\\
380	0\\
381	0\\
382	0\\
383	0\\
384	0\\
385	0\\
386	0\\
387	0\\
388	0\\
389	0\\
390	0\\
391	0\\
392	0\\
393	0\\
394	0\\
395	0\\
396	0\\
397	0\\
398	0\\
399	0\\
400	0\\
401	0\\
402	0\\
403	0\\
404	0\\
405	0\\
406	0\\
407	0\\
408	0\\
409	0\\
410	0\\
411	0\\
412	0\\
413	0\\
414	0\\
415	0\\
416	0\\
417	0\\
418	0\\
419	0\\
420	0\\
421	0\\
422	0\\
423	0\\
424	0\\
425	0\\
426	0\\
427	0\\
428	0\\
429	0\\
430	0\\
431	0\\
432	0\\
433	0\\
434	0\\
435	0\\
436	0\\
437	0\\
438	0\\
439	0\\
440	0\\
441	0\\
442	0\\
443	0\\
444	0\\
445	0\\
446	0\\
447	0\\
448	0\\
449	0\\
450	0\\
451	0\\
452	0\\
453	0\\
454	0\\
455	0\\
456	0\\
457	0\\
458	0\\
459	0\\
460	0\\
461	0\\
462	0\\
463	0\\
464	0\\
465	0\\
466	0\\
467	0\\
468	0\\
469	0\\
470	0\\
471	0\\
472	0\\
473	0\\
474	0\\
475	0\\
476	0\\
477	0\\
478	0\\
479	0\\
480	0\\
481	0\\
482	0\\
483	0\\
484	0\\
485	0\\
486	0\\
487	0\\
488	0\\
489	0\\
490	0\\
491	0\\
492	0\\
493	0\\
494	0\\
495	0\\
496	0\\
497	0\\
498	0\\
499	0\\
500	0\\
501	0\\
502	0\\
503	0\\
504	0\\
505	0\\
506	0\\
507	0\\
508	0\\
509	0\\
510	0\\
511	0\\
512	0\\
513	0\\
514	0\\
515	0\\
516	0\\
517	0\\
518	0\\
519	0\\
520	0\\
521	0\\
522	0\\
523	0\\
524	0\\
525	0\\
526	0\\
527	0\\
528	0\\
529	0\\
530	0\\
531	0\\
532	0\\
533	0\\
534	0\\
535	0\\
536	0\\
537	0\\
538	0\\
539	0\\
540	0\\
541	1.60435556964414e-06\\
542	3.14302111237603e-05\\
543	6.44583727469957e-05\\
544	9.74389278759319e-05\\
545	0.000129414111585044\\
546	0.000159861388438848\\
547	0.000190773274010441\\
548	0.000222067607061106\\
549	0.000253847438959652\\
550	0.000286336225299031\\
551	0.000319561371325126\\
552	0.00035354558130088\\
553	0.000388309508804606\\
554	0.000423874256441935\\
555	0.000460261953488801\\
556	0.000497494185660627\\
557	0.000535616231024054\\
558	0.000579553452092161\\
559	0.000621594147079938\\
560	0.000659031655417829\\
561	0.000696686984170281\\
562	0.000734828044981065\\
563	0.000773741639529971\\
564	0.000813454544992937\\
565	0.000853987307148437\\
566	0.000895361001406296\\
567	0.000937597950788636\\
568	0.000980721094541498\\
569	0.00134222542701616\\
570	0.00171269757187565\\
571	0.00180892766513185\\
572	0.00187663772269433\\
573	0.00194549621194972\\
574	0.00201553695190381\\
575	0.0020867879483933\\
576	0.00215927832676105\\
577	0.00223303839813513\\
578	0.0023080997181849\\
579	0.00238449514777378\\
580	0.00246225891823735\\
581	0.00254142670923183\\
582	0.00262203576145638\\
583	0.00270412508445196\\
584	0.00278773591889265\\
585	0.00287291287145547\\
586	0.00295970681313861\\
587	0.00304818237891641\\
588	0.00313843743693857\\
589	0.00323065362889274\\
590	0.00332522743498845\\
591	0.00342286644802626\\
592	0.00352518337500009\\
593	0.00363635235366165\\
594	0.00376728283869332\\
595	0.00394659005851719\\
596	0.00424941998802836\\
597	0.00487316246495455\\
598	0.00633625614039688\\
599	0\\
600	0\\
};
\addplot [color=mycolor6,solid,forget plot]
  table[row sep=crcr]{%
1	0\\
2	0\\
3	0\\
4	0\\
5	0\\
6	0\\
7	0\\
8	0\\
9	0\\
10	0\\
11	0\\
12	0\\
13	0\\
14	0\\
15	0\\
16	0\\
17	0\\
18	0\\
19	0\\
20	0\\
21	0\\
22	0\\
23	0\\
24	0\\
25	0\\
26	0\\
27	0\\
28	0\\
29	0\\
30	0\\
31	0\\
32	0\\
33	0\\
34	0\\
35	0\\
36	0\\
37	0\\
38	0\\
39	0\\
40	0\\
41	0\\
42	0\\
43	0\\
44	0\\
45	0\\
46	0\\
47	0\\
48	0\\
49	0\\
50	0\\
51	0\\
52	0\\
53	0\\
54	0\\
55	0\\
56	0\\
57	0\\
58	0\\
59	0\\
60	0\\
61	0\\
62	0\\
63	0\\
64	0\\
65	0\\
66	0\\
67	0\\
68	0\\
69	0\\
70	0\\
71	0\\
72	0\\
73	0\\
74	0\\
75	0\\
76	0\\
77	0\\
78	0\\
79	0\\
80	0\\
81	0\\
82	0\\
83	0\\
84	0\\
85	0\\
86	0\\
87	0\\
88	0\\
89	0\\
90	0\\
91	0\\
92	0\\
93	0\\
94	0\\
95	0\\
96	0\\
97	0\\
98	0\\
99	0\\
100	0\\
101	0\\
102	0\\
103	0\\
104	0\\
105	0\\
106	0\\
107	0\\
108	0\\
109	0\\
110	0\\
111	0\\
112	0\\
113	0\\
114	0\\
115	0\\
116	0\\
117	0\\
118	0\\
119	0\\
120	0\\
121	0\\
122	0\\
123	0\\
124	0\\
125	0\\
126	0\\
127	0\\
128	0\\
129	0\\
130	0\\
131	0\\
132	0\\
133	0\\
134	0\\
135	0\\
136	0\\
137	0\\
138	0\\
139	0\\
140	0\\
141	0\\
142	0\\
143	0\\
144	0\\
145	0\\
146	0\\
147	0\\
148	0\\
149	0\\
150	0\\
151	0\\
152	0\\
153	0\\
154	0\\
155	0\\
156	0\\
157	0\\
158	0\\
159	0\\
160	0\\
161	0\\
162	0\\
163	0\\
164	0\\
165	0\\
166	0\\
167	0\\
168	0\\
169	0\\
170	0\\
171	0\\
172	0\\
173	0\\
174	0\\
175	0\\
176	0\\
177	0\\
178	0\\
179	0\\
180	0\\
181	0\\
182	0\\
183	0\\
184	0\\
185	0\\
186	0\\
187	0\\
188	0\\
189	0\\
190	0\\
191	0\\
192	0\\
193	0\\
194	0\\
195	0\\
196	0\\
197	0\\
198	0\\
199	0\\
200	0\\
201	0\\
202	0\\
203	0\\
204	0\\
205	0\\
206	0\\
207	0\\
208	0\\
209	0\\
210	0\\
211	0\\
212	0\\
213	0\\
214	0\\
215	0\\
216	0\\
217	0\\
218	0\\
219	0\\
220	0\\
221	0\\
222	0\\
223	0\\
224	0\\
225	0\\
226	0\\
227	0\\
228	0\\
229	0\\
230	0\\
231	0\\
232	0\\
233	0\\
234	0\\
235	0\\
236	0\\
237	0\\
238	0\\
239	0\\
240	0\\
241	0\\
242	0\\
243	0\\
244	0\\
245	0\\
246	0\\
247	0\\
248	0\\
249	0\\
250	0\\
251	0\\
252	0\\
253	0\\
254	0\\
255	0\\
256	0\\
257	0\\
258	0\\
259	0\\
260	0\\
261	0\\
262	0\\
263	0\\
264	0\\
265	0\\
266	0\\
267	0\\
268	0\\
269	0\\
270	0\\
271	0\\
272	0\\
273	0\\
274	0\\
275	0\\
276	0\\
277	0\\
278	0\\
279	0\\
280	0\\
281	0\\
282	0\\
283	0\\
284	0\\
285	0\\
286	0\\
287	0\\
288	0\\
289	0\\
290	0\\
291	0\\
292	0\\
293	0\\
294	0\\
295	0\\
296	0\\
297	0\\
298	0\\
299	0\\
300	0\\
301	0\\
302	0\\
303	0\\
304	0\\
305	0\\
306	0\\
307	0\\
308	0\\
309	0\\
310	0\\
311	0\\
312	0\\
313	0\\
314	0\\
315	0\\
316	0\\
317	0\\
318	0\\
319	0\\
320	0\\
321	0\\
322	0\\
323	0\\
324	0\\
325	0\\
326	0\\
327	0\\
328	0\\
329	0\\
330	0\\
331	0\\
332	0\\
333	0\\
334	0\\
335	0\\
336	0\\
337	0\\
338	0\\
339	0\\
340	0\\
341	0\\
342	0\\
343	0\\
344	0\\
345	0\\
346	0\\
347	0\\
348	0\\
349	0\\
350	0\\
351	0\\
352	0\\
353	0\\
354	0\\
355	0\\
356	0\\
357	0\\
358	0\\
359	0\\
360	0\\
361	0\\
362	0\\
363	0\\
364	0\\
365	0\\
366	0\\
367	0\\
368	0\\
369	0\\
370	0\\
371	0\\
372	0\\
373	0\\
374	0\\
375	0\\
376	0\\
377	0\\
378	0\\
379	0\\
380	0\\
381	0\\
382	0\\
383	0\\
384	0\\
385	0\\
386	0\\
387	0\\
388	0\\
389	0\\
390	0\\
391	0\\
392	0\\
393	0\\
394	0\\
395	0\\
396	0\\
397	0\\
398	0\\
399	0\\
400	0\\
401	0\\
402	0\\
403	0\\
404	0\\
405	0\\
406	0\\
407	0\\
408	0\\
409	0\\
410	0\\
411	0\\
412	0\\
413	0\\
414	0\\
415	0\\
416	0\\
417	0\\
418	0\\
419	0\\
420	0\\
421	0\\
422	0\\
423	0\\
424	0\\
425	0\\
426	0\\
427	0\\
428	0\\
429	0\\
430	0\\
431	0\\
432	0\\
433	0\\
434	0\\
435	0\\
436	0\\
437	0\\
438	0\\
439	0\\
440	0\\
441	0\\
442	0\\
443	0\\
444	0\\
445	0\\
446	0\\
447	0\\
448	0\\
449	0\\
450	0\\
451	0\\
452	0\\
453	0\\
454	0\\
455	0\\
456	0\\
457	0\\
458	0\\
459	0\\
460	0\\
461	0\\
462	0\\
463	0\\
464	0\\
465	0\\
466	0\\
467	0\\
468	0\\
469	0\\
470	0\\
471	0\\
472	0\\
473	0\\
474	0\\
475	0\\
476	0\\
477	0\\
478	0\\
479	0\\
480	0\\
481	0\\
482	0\\
483	0\\
484	0\\
485	0\\
486	0\\
487	0\\
488	0\\
489	0\\
490	0\\
491	0\\
492	0\\
493	0\\
494	0\\
495	0\\
496	0\\
497	0\\
498	0\\
499	0\\
500	0\\
501	0\\
502	0\\
503	0\\
504	0\\
505	0\\
506	0\\
507	0\\
508	0\\
509	0\\
510	0\\
511	0\\
512	0\\
513	0\\
514	0\\
515	0\\
516	0\\
517	0\\
518	0\\
519	0\\
520	0\\
521	0\\
522	0\\
523	0\\
524	0\\
525	0\\
526	0\\
527	0\\
528	0\\
529	0\\
530	0\\
531	0\\
532	0\\
533	0\\
534	0\\
535	0\\
536	0\\
537	0\\
538	0\\
539	0\\
540	0\\
541	0\\
542	1.63690351718187e-05\\
543	4.64318723006276e-05\\
544	7.72064183891249e-05\\
545	0.000109681172498384\\
546	0.000144326319315444\\
547	0.000178940796521601\\
548	0.000211991301117044\\
549	0.00024412624604269\\
550	0.000276785482440831\\
551	0.000309938580594202\\
552	0.000343753073709177\\
553	0.000378330150709172\\
554	0.000413698699966639\\
555	0.000449883738139326\\
556	0.000486908539075665\\
557	0.000524796936544112\\
558	0.000563571508554637\\
559	0.00060474487388057\\
560	0.000650150252615914\\
561	0.000691569997482022\\
562	0.000730363569551948\\
563	0.000769387350818147\\
564	0.000809034584368206\\
565	0.000849493623747197\\
566	0.000890791349062815\\
567	0.000932949900077076\\
568	0.00097599221169323\\
569	0.00101994155476744\\
570	0.00133660911702074\\
571	0.00177027877590743\\
572	0.00187639737504145\\
573	0.00194549595575903\\
574	0.00201553692275095\\
575	0.0020867879342917\\
576	0.00215927831957446\\
577	0.00223303839460999\\
578	0.00230809971656243\\
579	0.00238449514708515\\
580	0.00246225891797304\\
581	0.00254142670914249\\
582	0.0026220357614308\\
583	0.00270412508444604\\
584	0.00278773591889168\\
585	0.00287291287145548\\
586	0.00295970681313861\\
587	0.0030481823789164\\
588	0.00313843743693857\\
589	0.00323065362889274\\
590	0.00332522743498845\\
591	0.00342286644802626\\
592	0.00352518337500009\\
593	0.00363635235366165\\
594	0.00376728283869332\\
595	0.00394659005851719\\
596	0.00424941998802836\\
597	0.00487316246495455\\
598	0.00633625614039688\\
599	0\\
600	0\\
};
\addplot [color=mycolor7,solid,forget plot]
  table[row sep=crcr]{%
1	0\\
2	0\\
3	0\\
4	0\\
5	0\\
6	0\\
7	0\\
8	0\\
9	0\\
10	0\\
11	0\\
12	0\\
13	0\\
14	0\\
15	0\\
16	0\\
17	0\\
18	0\\
19	0\\
20	0\\
21	0\\
22	0\\
23	0\\
24	0\\
25	0\\
26	0\\
27	0\\
28	0\\
29	0\\
30	0\\
31	0\\
32	0\\
33	0\\
34	0\\
35	0\\
36	0\\
37	0\\
38	0\\
39	0\\
40	0\\
41	0\\
42	0\\
43	0\\
44	0\\
45	0\\
46	0\\
47	0\\
48	0\\
49	0\\
50	0\\
51	0\\
52	0\\
53	0\\
54	0\\
55	0\\
56	0\\
57	0\\
58	0\\
59	0\\
60	0\\
61	0\\
62	0\\
63	0\\
64	0\\
65	0\\
66	0\\
67	0\\
68	0\\
69	0\\
70	0\\
71	0\\
72	0\\
73	0\\
74	0\\
75	0\\
76	0\\
77	0\\
78	0\\
79	0\\
80	0\\
81	0\\
82	0\\
83	0\\
84	0\\
85	0\\
86	0\\
87	0\\
88	0\\
89	0\\
90	0\\
91	0\\
92	0\\
93	0\\
94	0\\
95	0\\
96	0\\
97	0\\
98	0\\
99	0\\
100	0\\
101	0\\
102	0\\
103	0\\
104	0\\
105	0\\
106	0\\
107	0\\
108	0\\
109	0\\
110	0\\
111	0\\
112	0\\
113	0\\
114	0\\
115	0\\
116	0\\
117	0\\
118	0\\
119	0\\
120	0\\
121	0\\
122	0\\
123	0\\
124	0\\
125	0\\
126	0\\
127	0\\
128	0\\
129	0\\
130	0\\
131	0\\
132	0\\
133	0\\
134	0\\
135	0\\
136	0\\
137	0\\
138	0\\
139	0\\
140	0\\
141	0\\
142	0\\
143	0\\
144	0\\
145	0\\
146	0\\
147	0\\
148	0\\
149	0\\
150	0\\
151	0\\
152	0\\
153	0\\
154	0\\
155	0\\
156	0\\
157	0\\
158	0\\
159	0\\
160	0\\
161	0\\
162	0\\
163	0\\
164	0\\
165	0\\
166	0\\
167	0\\
168	0\\
169	0\\
170	0\\
171	0\\
172	0\\
173	0\\
174	0\\
175	0\\
176	0\\
177	0\\
178	0\\
179	0\\
180	0\\
181	0\\
182	0\\
183	0\\
184	0\\
185	0\\
186	0\\
187	0\\
188	0\\
189	0\\
190	0\\
191	0\\
192	0\\
193	0\\
194	0\\
195	0\\
196	0\\
197	0\\
198	0\\
199	0\\
200	0\\
201	0\\
202	0\\
203	0\\
204	0\\
205	0\\
206	0\\
207	0\\
208	0\\
209	0\\
210	0\\
211	0\\
212	0\\
213	0\\
214	0\\
215	0\\
216	0\\
217	0\\
218	0\\
219	0\\
220	0\\
221	0\\
222	0\\
223	0\\
224	0\\
225	0\\
226	0\\
227	0\\
228	0\\
229	0\\
230	0\\
231	0\\
232	0\\
233	0\\
234	0\\
235	0\\
236	0\\
237	0\\
238	0\\
239	0\\
240	0\\
241	0\\
242	0\\
243	0\\
244	0\\
245	0\\
246	0\\
247	0\\
248	0\\
249	0\\
250	0\\
251	0\\
252	0\\
253	0\\
254	0\\
255	0\\
256	0\\
257	0\\
258	0\\
259	0\\
260	0\\
261	0\\
262	0\\
263	0\\
264	0\\
265	0\\
266	0\\
267	0\\
268	0\\
269	0\\
270	0\\
271	0\\
272	0\\
273	0\\
274	0\\
275	0\\
276	0\\
277	0\\
278	0\\
279	0\\
280	0\\
281	0\\
282	0\\
283	0\\
284	0\\
285	0\\
286	0\\
287	0\\
288	0\\
289	0\\
290	0\\
291	0\\
292	0\\
293	0\\
294	0\\
295	0\\
296	0\\
297	0\\
298	0\\
299	0\\
300	0\\
301	0\\
302	0\\
303	0\\
304	0\\
305	0\\
306	0\\
307	0\\
308	0\\
309	0\\
310	0\\
311	0\\
312	0\\
313	0\\
314	0\\
315	0\\
316	0\\
317	0\\
318	0\\
319	0\\
320	0\\
321	0\\
322	0\\
323	0\\
324	0\\
325	0\\
326	0\\
327	0\\
328	0\\
329	0\\
330	0\\
331	0\\
332	0\\
333	0\\
334	0\\
335	0\\
336	0\\
337	0\\
338	0\\
339	0\\
340	0\\
341	0\\
342	0\\
343	0\\
344	0\\
345	0\\
346	0\\
347	0\\
348	0\\
349	0\\
350	0\\
351	0\\
352	0\\
353	0\\
354	0\\
355	0\\
356	0\\
357	0\\
358	0\\
359	0\\
360	0\\
361	0\\
362	0\\
363	0\\
364	0\\
365	0\\
366	0\\
367	0\\
368	0\\
369	0\\
370	0\\
371	0\\
372	0\\
373	0\\
374	0\\
375	0\\
376	0\\
377	0\\
378	0\\
379	0\\
380	0\\
381	0\\
382	0\\
383	0\\
384	0\\
385	0\\
386	0\\
387	0\\
388	0\\
389	0\\
390	0\\
391	0\\
392	0\\
393	0\\
394	0\\
395	0\\
396	0\\
397	0\\
398	0\\
399	0\\
400	0\\
401	0\\
402	0\\
403	0\\
404	0\\
405	0\\
406	0\\
407	0\\
408	0\\
409	0\\
410	0\\
411	0\\
412	0\\
413	0\\
414	0\\
415	0\\
416	0\\
417	0\\
418	0\\
419	0\\
420	0\\
421	0\\
422	0\\
423	0\\
424	0\\
425	0\\
426	0\\
427	0\\
428	0\\
429	0\\
430	0\\
431	0\\
432	0\\
433	0\\
434	0\\
435	0\\
436	0\\
437	0\\
438	0\\
439	0\\
440	0\\
441	0\\
442	0\\
443	0\\
444	0\\
445	0\\
446	0\\
447	0\\
448	0\\
449	0\\
450	0\\
451	0\\
452	0\\
453	0\\
454	0\\
455	0\\
456	0\\
457	0\\
458	0\\
459	0\\
460	0\\
461	0\\
462	0\\
463	0\\
464	0\\
465	0\\
466	0\\
467	0\\
468	0\\
469	0\\
470	0\\
471	0\\
472	0\\
473	0\\
474	0\\
475	0\\
476	0\\
477	0\\
478	0\\
479	0\\
480	0\\
481	0\\
482	0\\
483	0\\
484	0\\
485	0\\
486	0\\
487	0\\
488	0\\
489	0\\
490	0\\
491	0\\
492	0\\
493	0\\
494	0\\
495	0\\
496	0\\
497	0\\
498	0\\
499	0\\
500	0\\
501	0\\
502	0\\
503	0\\
504	0\\
505	0\\
506	0\\
507	0\\
508	0\\
509	0\\
510	0\\
511	0\\
512	0\\
513	0\\
514	0\\
515	0\\
516	0\\
517	0\\
518	0\\
519	0\\
520	0\\
521	0\\
522	0\\
523	0\\
524	0\\
525	0\\
526	0\\
527	0\\
528	0\\
529	0\\
530	0\\
531	0\\
532	0\\
533	0\\
534	0\\
535	0\\
536	0\\
537	0\\
538	0\\
539	0\\
540	0\\
541	0\\
542	3.01637999570976e-08\\
543	2.99658860932134e-05\\
544	6.05018750287523e-05\\
545	9.17088661460931e-05\\
546	0.000123619841455608\\
547	0.000156278193318321\\
548	0.000191232011937836\\
549	0.000227577015837924\\
550	0.000263857115249925\\
551	0.000298369543038874\\
552	0.000332509480255741\\
553	0.000367239183237181\\
554	0.00040251235853391\\
555	0.0004385007442285\\
556	0.00047530803845914\\
557	0.000512965239185088\\
558	0.000551501490389918\\
559	0.000590941911369099\\
560	0.000631324814085933\\
561	0.000675582160272487\\
562	0.000722005927583519\\
563	0.000763647302898438\\
564	0.000803913197393734\\
565	0.00084444157933309\\
566	0.000885669182482565\\
567	0.000927749067611719\\
568	0.000970711974220374\\
569	0.00101458305214579\\
570	0.00105938727463191\\
571	0.00129287225666447\\
572	0.00182349119818199\\
573	0.00193672148976217\\
574	0.00201553086403398\\
575	0.00208678772032082\\
576	0.00215927822556618\\
577	0.00223303834594697\\
578	0.00230809969188834\\
579	0.0023844951352833\\
580	0.00246225891275016\\
581	0.00254142670704607\\
582	0.0026220357606876\\
583	0.00270412508422211\\
584	0.00278773591883777\\
585	0.00287291287144614\\
586	0.00295970681313776\\
587	0.00304818237891641\\
588	0.00313843743693857\\
589	0.00323065362889274\\
590	0.00332522743498844\\
591	0.00342286644802627\\
592	0.00352518337500009\\
593	0.00363635235366165\\
594	0.00376728283869332\\
595	0.00394659005851719\\
596	0.00424941998802836\\
597	0.00487316246495454\\
598	0.00633625614039688\\
599	0\\
600	0\\
};
\addplot [color=mycolor8,solid,forget plot]
  table[row sep=crcr]{%
1	0\\
2	0\\
3	0\\
4	0\\
5	0\\
6	0\\
7	0\\
8	0\\
9	0\\
10	0\\
11	0\\
12	0\\
13	0\\
14	0\\
15	0\\
16	0\\
17	0\\
18	0\\
19	0\\
20	0\\
21	0\\
22	0\\
23	0\\
24	0\\
25	0\\
26	0\\
27	0\\
28	0\\
29	0\\
30	0\\
31	0\\
32	0\\
33	0\\
34	0\\
35	0\\
36	0\\
37	0\\
38	0\\
39	0\\
40	0\\
41	0\\
42	0\\
43	0\\
44	0\\
45	0\\
46	0\\
47	0\\
48	0\\
49	0\\
50	0\\
51	0\\
52	0\\
53	0\\
54	0\\
55	0\\
56	0\\
57	0\\
58	0\\
59	0\\
60	0\\
61	0\\
62	0\\
63	0\\
64	0\\
65	0\\
66	0\\
67	0\\
68	0\\
69	0\\
70	0\\
71	0\\
72	0\\
73	0\\
74	0\\
75	0\\
76	0\\
77	0\\
78	0\\
79	0\\
80	0\\
81	0\\
82	0\\
83	0\\
84	0\\
85	0\\
86	0\\
87	0\\
88	0\\
89	0\\
90	0\\
91	0\\
92	0\\
93	0\\
94	0\\
95	0\\
96	0\\
97	0\\
98	0\\
99	0\\
100	0\\
101	0\\
102	0\\
103	0\\
104	0\\
105	0\\
106	0\\
107	0\\
108	0\\
109	0\\
110	0\\
111	0\\
112	0\\
113	0\\
114	0\\
115	0\\
116	0\\
117	0\\
118	0\\
119	0\\
120	0\\
121	0\\
122	0\\
123	0\\
124	0\\
125	0\\
126	0\\
127	0\\
128	0\\
129	0\\
130	0\\
131	0\\
132	0\\
133	0\\
134	0\\
135	0\\
136	0\\
137	0\\
138	0\\
139	0\\
140	0\\
141	0\\
142	0\\
143	0\\
144	0\\
145	0\\
146	0\\
147	0\\
148	0\\
149	0\\
150	0\\
151	0\\
152	0\\
153	0\\
154	0\\
155	0\\
156	0\\
157	0\\
158	0\\
159	0\\
160	0\\
161	0\\
162	0\\
163	0\\
164	0\\
165	0\\
166	0\\
167	0\\
168	0\\
169	0\\
170	0\\
171	0\\
172	0\\
173	0\\
174	0\\
175	0\\
176	0\\
177	0\\
178	0\\
179	0\\
180	0\\
181	0\\
182	0\\
183	0\\
184	0\\
185	0\\
186	0\\
187	0\\
188	0\\
189	0\\
190	0\\
191	0\\
192	0\\
193	0\\
194	0\\
195	0\\
196	0\\
197	0\\
198	0\\
199	0\\
200	0\\
201	0\\
202	0\\
203	0\\
204	0\\
205	0\\
206	0\\
207	0\\
208	0\\
209	0\\
210	0\\
211	0\\
212	0\\
213	0\\
214	0\\
215	0\\
216	0\\
217	0\\
218	0\\
219	0\\
220	0\\
221	0\\
222	0\\
223	0\\
224	0\\
225	0\\
226	0\\
227	0\\
228	0\\
229	0\\
230	0\\
231	0\\
232	0\\
233	0\\
234	0\\
235	0\\
236	0\\
237	0\\
238	0\\
239	0\\
240	0\\
241	0\\
242	0\\
243	0\\
244	0\\
245	0\\
246	0\\
247	0\\
248	0\\
249	0\\
250	0\\
251	0\\
252	0\\
253	0\\
254	0\\
255	0\\
256	0\\
257	0\\
258	0\\
259	0\\
260	0\\
261	0\\
262	0\\
263	0\\
264	0\\
265	0\\
266	0\\
267	0\\
268	0\\
269	0\\
270	0\\
271	0\\
272	0\\
273	0\\
274	0\\
275	0\\
276	0\\
277	0\\
278	0\\
279	0\\
280	0\\
281	0\\
282	0\\
283	0\\
284	0\\
285	0\\
286	0\\
287	0\\
288	0\\
289	0\\
290	0\\
291	0\\
292	0\\
293	0\\
294	0\\
295	0\\
296	0\\
297	0\\
298	0\\
299	0\\
300	0\\
301	0\\
302	0\\
303	0\\
304	0\\
305	0\\
306	0\\
307	0\\
308	0\\
309	0\\
310	0\\
311	0\\
312	0\\
313	0\\
314	0\\
315	0\\
316	0\\
317	0\\
318	0\\
319	0\\
320	0\\
321	0\\
322	0\\
323	0\\
324	0\\
325	0\\
326	0\\
327	0\\
328	0\\
329	0\\
330	0\\
331	0\\
332	0\\
333	0\\
334	0\\
335	0\\
336	0\\
337	0\\
338	0\\
339	0\\
340	0\\
341	0\\
342	0\\
343	0\\
344	0\\
345	0\\
346	0\\
347	0\\
348	0\\
349	0\\
350	0\\
351	0\\
352	0\\
353	0\\
354	0\\
355	0\\
356	0\\
357	0\\
358	0\\
359	0\\
360	0\\
361	0\\
362	0\\
363	0\\
364	0\\
365	0\\
366	0\\
367	0\\
368	0\\
369	0\\
370	0\\
371	0\\
372	0\\
373	0\\
374	0\\
375	0\\
376	0\\
377	0\\
378	0\\
379	0\\
380	0\\
381	0\\
382	0\\
383	0\\
384	0\\
385	0\\
386	0\\
387	0\\
388	0\\
389	0\\
390	0\\
391	0\\
392	0\\
393	0\\
394	0\\
395	0\\
396	0\\
397	0\\
398	0\\
399	0\\
400	0\\
401	0\\
402	0\\
403	0\\
404	0\\
405	0\\
406	0\\
407	0\\
408	0\\
409	0\\
410	0\\
411	0\\
412	0\\
413	0\\
414	0\\
415	0\\
416	0\\
417	0\\
418	0\\
419	0\\
420	0\\
421	0\\
422	0\\
423	0\\
424	0\\
425	0\\
426	0\\
427	0\\
428	0\\
429	0\\
430	0\\
431	0\\
432	0\\
433	0\\
434	0\\
435	0\\
436	0\\
437	0\\
438	0\\
439	0\\
440	0\\
441	0\\
442	0\\
443	0\\
444	0\\
445	0\\
446	0\\
447	0\\
448	0\\
449	0\\
450	0\\
451	0\\
452	0\\
453	0\\
454	0\\
455	0\\
456	0\\
457	0\\
458	0\\
459	0\\
460	0\\
461	0\\
462	0\\
463	0\\
464	0\\
465	0\\
466	0\\
467	0\\
468	0\\
469	0\\
470	0\\
471	0\\
472	0\\
473	0\\
474	0\\
475	0\\
476	0\\
477	0\\
478	0\\
479	0\\
480	0\\
481	0\\
482	0\\
483	0\\
484	0\\
485	0\\
486	0\\
487	0\\
488	0\\
489	0\\
490	0\\
491	0\\
492	0\\
493	0\\
494	0\\
495	0\\
496	0\\
497	0\\
498	0\\
499	0\\
500	0\\
501	0\\
502	0\\
503	0\\
504	0\\
505	0\\
506	0\\
507	0\\
508	0\\
509	0\\
510	0\\
511	0\\
512	0\\
513	0\\
514	0\\
515	0\\
516	0\\
517	0\\
518	0\\
519	0\\
520	0\\
521	0\\
522	0\\
523	0\\
524	0\\
525	0\\
526	0\\
527	0\\
528	0\\
529	0\\
530	0\\
531	0\\
532	0\\
533	0\\
534	0\\
535	0\\
536	0\\
537	0\\
538	0\\
539	0\\
540	0\\
541	0\\
542	0\\
543	9.36511750306959e-06\\
544	4.0595496929887e-05\\
545	7.22025169697008e-05\\
546	0.000104277738136623\\
547	0.000136901546371204\\
548	0.000170100838277375\\
549	0.000203976958092589\\
550	0.000238526871729759\\
551	0.000275685308335015\\
552	0.000313969737058781\\
553	0.000352328408951315\\
554	0.000388967976204953\\
555	0.000425291340411316\\
556	0.000462255096958164\\
557	0.000499826147251082\\
558	0.000538135586387126\\
559	0.000577322551403436\\
560	0.000617419953092465\\
561	0.000658460371560621\\
562	0.000700497322431724\\
563	0.000747313738427327\\
564	0.000795025248370413\\
565	0.000837765651433914\\
566	0.000879617402495584\\
567	0.000921780386830207\\
568	0.000964645213591637\\
569	0.00100839774047892\\
570	0.0010530754965657\\
571	0.00109870524560156\\
572	0.00121220604724951\\
573	0.00178738153659761\\
574	0.00198895788961421\\
575	0.00208659558592422\\
576	0.00215927626069132\\
577	0.0022330377236014\\
578	0.0023080993732893\\
579	0.0023844949687054\\
580	0.00246225882988089\\
581	0.0025414266687457\\
582	0.00262203574457408\\
583	0.00270412507821453\\
584	0.00278773591692754\\
585	0.00287291287095885\\
586	0.00295970681304879\\
587	0.00304818237890766\\
588	0.00313843743693858\\
589	0.00323065362889274\\
590	0.00332522743498845\\
591	0.00342286644802627\\
592	0.0035251833750001\\
593	0.00363635235366165\\
594	0.00376728283869332\\
595	0.0039465900585172\\
596	0.00424941998802837\\
597	0.00487316246495455\\
598	0.00633625614039688\\
599	0\\
600	0\\
};
\addplot [color=blue!25!mycolor7,solid,forget plot]
  table[row sep=crcr]{%
1	0\\
2	0\\
3	0\\
4	0\\
5	0\\
6	0\\
7	0\\
8	0\\
9	0\\
10	0\\
11	0\\
12	0\\
13	0\\
14	0\\
15	0\\
16	0\\
17	0\\
18	0\\
19	0\\
20	0\\
21	0\\
22	0\\
23	0\\
24	0\\
25	0\\
26	0\\
27	0\\
28	0\\
29	0\\
30	0\\
31	0\\
32	0\\
33	0\\
34	0\\
35	0\\
36	0\\
37	0\\
38	0\\
39	0\\
40	0\\
41	0\\
42	0\\
43	0\\
44	0\\
45	0\\
46	0\\
47	0\\
48	0\\
49	0\\
50	0\\
51	0\\
52	0\\
53	0\\
54	0\\
55	0\\
56	0\\
57	0\\
58	0\\
59	0\\
60	0\\
61	0\\
62	0\\
63	0\\
64	0\\
65	0\\
66	0\\
67	0\\
68	0\\
69	0\\
70	0\\
71	0\\
72	0\\
73	0\\
74	0\\
75	0\\
76	0\\
77	0\\
78	0\\
79	0\\
80	0\\
81	0\\
82	0\\
83	0\\
84	0\\
85	0\\
86	0\\
87	0\\
88	0\\
89	0\\
90	0\\
91	0\\
92	0\\
93	0\\
94	0\\
95	0\\
96	0\\
97	0\\
98	0\\
99	0\\
100	0\\
101	0\\
102	0\\
103	0\\
104	0\\
105	0\\
106	0\\
107	0\\
108	0\\
109	0\\
110	0\\
111	0\\
112	0\\
113	0\\
114	0\\
115	0\\
116	0\\
117	0\\
118	0\\
119	0\\
120	0\\
121	0\\
122	0\\
123	0\\
124	0\\
125	0\\
126	0\\
127	0\\
128	0\\
129	0\\
130	0\\
131	0\\
132	0\\
133	0\\
134	0\\
135	0\\
136	0\\
137	0\\
138	0\\
139	0\\
140	0\\
141	0\\
142	0\\
143	0\\
144	0\\
145	0\\
146	0\\
147	0\\
148	0\\
149	0\\
150	0\\
151	0\\
152	0\\
153	0\\
154	0\\
155	0\\
156	0\\
157	0\\
158	0\\
159	0\\
160	0\\
161	0\\
162	0\\
163	0\\
164	0\\
165	0\\
166	0\\
167	0\\
168	0\\
169	0\\
170	0\\
171	0\\
172	0\\
173	0\\
174	0\\
175	0\\
176	0\\
177	0\\
178	0\\
179	0\\
180	0\\
181	0\\
182	0\\
183	0\\
184	0\\
185	0\\
186	0\\
187	0\\
188	0\\
189	0\\
190	0\\
191	0\\
192	0\\
193	0\\
194	0\\
195	0\\
196	0\\
197	0\\
198	0\\
199	0\\
200	0\\
201	0\\
202	0\\
203	0\\
204	0\\
205	0\\
206	0\\
207	0\\
208	0\\
209	0\\
210	0\\
211	0\\
212	0\\
213	0\\
214	0\\
215	0\\
216	0\\
217	0\\
218	0\\
219	0\\
220	0\\
221	0\\
222	0\\
223	0\\
224	0\\
225	0\\
226	0\\
227	0\\
228	0\\
229	0\\
230	0\\
231	0\\
232	0\\
233	0\\
234	0\\
235	0\\
236	0\\
237	0\\
238	0\\
239	0\\
240	0\\
241	0\\
242	0\\
243	0\\
244	0\\
245	0\\
246	0\\
247	0\\
248	0\\
249	0\\
250	0\\
251	0\\
252	0\\
253	0\\
254	0\\
255	0\\
256	0\\
257	0\\
258	0\\
259	0\\
260	0\\
261	0\\
262	0\\
263	0\\
264	0\\
265	0\\
266	0\\
267	0\\
268	0\\
269	0\\
270	0\\
271	0\\
272	0\\
273	0\\
274	0\\
275	0\\
276	0\\
277	0\\
278	0\\
279	0\\
280	0\\
281	0\\
282	0\\
283	0\\
284	0\\
285	0\\
286	0\\
287	0\\
288	0\\
289	0\\
290	0\\
291	0\\
292	0\\
293	0\\
294	0\\
295	0\\
296	0\\
297	0\\
298	0\\
299	0\\
300	0\\
301	0\\
302	0\\
303	0\\
304	0\\
305	0\\
306	0\\
307	0\\
308	0\\
309	0\\
310	0\\
311	0\\
312	0\\
313	0\\
314	0\\
315	0\\
316	0\\
317	0\\
318	0\\
319	0\\
320	0\\
321	0\\
322	0\\
323	0\\
324	0\\
325	0\\
326	0\\
327	0\\
328	0\\
329	0\\
330	0\\
331	0\\
332	0\\
333	0\\
334	0\\
335	0\\
336	0\\
337	0\\
338	0\\
339	0\\
340	0\\
341	0\\
342	0\\
343	0\\
344	0\\
345	0\\
346	0\\
347	0\\
348	0\\
349	0\\
350	0\\
351	0\\
352	0\\
353	0\\
354	0\\
355	0\\
356	0\\
357	0\\
358	0\\
359	0\\
360	0\\
361	0\\
362	0\\
363	0\\
364	0\\
365	0\\
366	0\\
367	0\\
368	0\\
369	0\\
370	0\\
371	0\\
372	0\\
373	0\\
374	0\\
375	0\\
376	0\\
377	0\\
378	0\\
379	0\\
380	0\\
381	0\\
382	0\\
383	0\\
384	0\\
385	0\\
386	0\\
387	0\\
388	0\\
389	0\\
390	0\\
391	0\\
392	0\\
393	0\\
394	0\\
395	0\\
396	0\\
397	0\\
398	0\\
399	0\\
400	0\\
401	0\\
402	0\\
403	0\\
404	0\\
405	0\\
406	0\\
407	0\\
408	0\\
409	0\\
410	0\\
411	0\\
412	0\\
413	0\\
414	0\\
415	0\\
416	0\\
417	0\\
418	0\\
419	0\\
420	0\\
421	0\\
422	0\\
423	0\\
424	0\\
425	0\\
426	0\\
427	0\\
428	0\\
429	0\\
430	0\\
431	0\\
432	0\\
433	0\\
434	0\\
435	0\\
436	0\\
437	0\\
438	0\\
439	0\\
440	0\\
441	0\\
442	0\\
443	0\\
444	0\\
445	0\\
446	0\\
447	0\\
448	0\\
449	0\\
450	0\\
451	0\\
452	0\\
453	0\\
454	0\\
455	0\\
456	0\\
457	0\\
458	0\\
459	0\\
460	0\\
461	0\\
462	0\\
463	0\\
464	0\\
465	0\\
466	0\\
467	0\\
468	0\\
469	0\\
470	0\\
471	0\\
472	0\\
473	0\\
474	0\\
475	0\\
476	0\\
477	0\\
478	0\\
479	0\\
480	0\\
481	0\\
482	0\\
483	0\\
484	0\\
485	0\\
486	0\\
487	0\\
488	0\\
489	0\\
490	0\\
491	0\\
492	0\\
493	0\\
494	0\\
495	0\\
496	0\\
497	0\\
498	0\\
499	0\\
500	0\\
501	0\\
502	0\\
503	0\\
504	0\\
505	0\\
506	0\\
507	0\\
508	0\\
509	0\\
510	0\\
511	0\\
512	0\\
513	0\\
514	0\\
515	0\\
516	0\\
517	0\\
518	0\\
519	0\\
520	0\\
521	0\\
522	0\\
523	0\\
524	0\\
525	0\\
526	0\\
527	0\\
528	0\\
529	0\\
530	0\\
531	0\\
532	0\\
533	0\\
534	0\\
535	0\\
536	0\\
537	0\\
538	0\\
539	0\\
540	0\\
541	0\\
542	0\\
543	0\\
544	7.49517579551879e-07\\
545	3.85447340164759e-05\\
546	7.52662412083253e-05\\
547	0.000111131029203797\\
548	0.000145643679797062\\
549	0.000180296982227337\\
550	0.00021535568667807\\
551	0.00025080449961763\\
552	0.000286874977847823\\
553	0.000323643662070736\\
554	0.000363096748878308\\
555	0.000403697300813226\\
556	0.000444411306028165\\
557	0.000483782483864593\\
558	0.000522495298114908\\
559	0.000561899913432818\\
560	0.000601990613295173\\
561	0.00064280724862691\\
562	0.000684562043987083\\
563	0.000727288108897307\\
564	0.000771057340961328\\
565	0.000819898537954155\\
566	0.000869220907437458\\
567	0.000914032068548085\\
568	0.000957627672776297\\
569	0.00100160271504627\\
570	0.00104620716841636\\
571	0.00109173284949489\\
572	0.00113822717707793\\
573	0.00118572633563006\\
574	0.0016486833706938\\
575	0.00203747445819607\\
576	0.00215253303533154\\
577	0.00223300678105388\\
578	0.00230809492244212\\
579	0.00238449294006508\\
580	0.00246225774710443\\
581	0.00254142610849541\\
582	0.00262203547379105\\
583	0.00270412495853163\\
584	0.00278773586986386\\
585	0.00287291285511267\\
586	0.00295970680874964\\
587	0.00304818237806893\\
588	0.00313843743684985\\
589	0.00323065362889274\\
590	0.00332522743498845\\
591	0.00342286644802627\\
592	0.00352518337500008\\
593	0.00363635235366165\\
594	0.00376728283869331\\
595	0.00394659005851719\\
596	0.00424941998802836\\
597	0.00487316246495454\\
598	0.00633625614039688\\
599	0\\
600	0\\
};
\addplot [color=mycolor9,solid,forget plot]
  table[row sep=crcr]{%
1	0\\
2	0\\
3	0\\
4	0\\
5	0\\
6	0\\
7	0\\
8	0\\
9	0\\
10	0\\
11	0\\
12	0\\
13	0\\
14	0\\
15	0\\
16	0\\
17	0\\
18	0\\
19	0\\
20	0\\
21	0\\
22	0\\
23	0\\
24	0\\
25	0\\
26	0\\
27	0\\
28	0\\
29	0\\
30	0\\
31	0\\
32	0\\
33	0\\
34	0\\
35	0\\
36	0\\
37	0\\
38	0\\
39	0\\
40	0\\
41	0\\
42	0\\
43	0\\
44	0\\
45	0\\
46	0\\
47	0\\
48	0\\
49	0\\
50	0\\
51	0\\
52	0\\
53	0\\
54	0\\
55	0\\
56	0\\
57	0\\
58	0\\
59	0\\
60	0\\
61	0\\
62	0\\
63	0\\
64	0\\
65	0\\
66	0\\
67	0\\
68	0\\
69	0\\
70	0\\
71	0\\
72	0\\
73	0\\
74	0\\
75	0\\
76	0\\
77	0\\
78	0\\
79	0\\
80	0\\
81	0\\
82	0\\
83	0\\
84	0\\
85	0\\
86	0\\
87	0\\
88	0\\
89	0\\
90	0\\
91	0\\
92	0\\
93	0\\
94	0\\
95	0\\
96	0\\
97	0\\
98	0\\
99	0\\
100	0\\
101	0\\
102	0\\
103	0\\
104	0\\
105	0\\
106	0\\
107	0\\
108	0\\
109	0\\
110	0\\
111	0\\
112	0\\
113	0\\
114	0\\
115	0\\
116	0\\
117	0\\
118	0\\
119	0\\
120	0\\
121	0\\
122	0\\
123	0\\
124	0\\
125	0\\
126	0\\
127	0\\
128	0\\
129	0\\
130	0\\
131	0\\
132	0\\
133	0\\
134	0\\
135	0\\
136	0\\
137	0\\
138	0\\
139	0\\
140	0\\
141	0\\
142	0\\
143	0\\
144	0\\
145	0\\
146	0\\
147	0\\
148	0\\
149	0\\
150	0\\
151	0\\
152	0\\
153	0\\
154	0\\
155	0\\
156	0\\
157	0\\
158	0\\
159	0\\
160	0\\
161	0\\
162	0\\
163	0\\
164	0\\
165	0\\
166	0\\
167	0\\
168	0\\
169	0\\
170	0\\
171	0\\
172	0\\
173	0\\
174	0\\
175	0\\
176	0\\
177	0\\
178	0\\
179	0\\
180	0\\
181	0\\
182	0\\
183	0\\
184	0\\
185	0\\
186	0\\
187	0\\
188	0\\
189	0\\
190	0\\
191	0\\
192	0\\
193	0\\
194	0\\
195	0\\
196	0\\
197	0\\
198	0\\
199	0\\
200	0\\
201	0\\
202	0\\
203	0\\
204	0\\
205	0\\
206	0\\
207	0\\
208	0\\
209	0\\
210	0\\
211	0\\
212	0\\
213	0\\
214	0\\
215	0\\
216	0\\
217	0\\
218	0\\
219	0\\
220	0\\
221	0\\
222	0\\
223	0\\
224	0\\
225	0\\
226	0\\
227	0\\
228	0\\
229	0\\
230	0\\
231	0\\
232	0\\
233	0\\
234	0\\
235	0\\
236	0\\
237	0\\
238	0\\
239	0\\
240	0\\
241	0\\
242	0\\
243	0\\
244	0\\
245	0\\
246	0\\
247	0\\
248	0\\
249	0\\
250	0\\
251	0\\
252	0\\
253	0\\
254	0\\
255	0\\
256	0\\
257	0\\
258	0\\
259	0\\
260	0\\
261	0\\
262	0\\
263	0\\
264	0\\
265	0\\
266	0\\
267	0\\
268	0\\
269	0\\
270	0\\
271	0\\
272	0\\
273	0\\
274	0\\
275	0\\
276	0\\
277	0\\
278	0\\
279	0\\
280	0\\
281	0\\
282	0\\
283	0\\
284	0\\
285	0\\
286	0\\
287	0\\
288	0\\
289	0\\
290	0\\
291	0\\
292	0\\
293	0\\
294	0\\
295	0\\
296	0\\
297	0\\
298	0\\
299	0\\
300	0\\
301	0\\
302	0\\
303	0\\
304	0\\
305	0\\
306	0\\
307	0\\
308	0\\
309	0\\
310	0\\
311	0\\
312	0\\
313	0\\
314	0\\
315	0\\
316	0\\
317	0\\
318	0\\
319	0\\
320	0\\
321	0\\
322	0\\
323	0\\
324	0\\
325	0\\
326	0\\
327	0\\
328	0\\
329	0\\
330	0\\
331	0\\
332	0\\
333	0\\
334	0\\
335	0\\
336	0\\
337	0\\
338	0\\
339	0\\
340	0\\
341	0\\
342	0\\
343	0\\
344	0\\
345	0\\
346	0\\
347	0\\
348	0\\
349	0\\
350	0\\
351	0\\
352	0\\
353	0\\
354	0\\
355	0\\
356	0\\
357	0\\
358	0\\
359	0\\
360	0\\
361	0\\
362	0\\
363	0\\
364	0\\
365	0\\
366	0\\
367	0\\
368	0\\
369	0\\
370	0\\
371	0\\
372	0\\
373	0\\
374	0\\
375	0\\
376	0\\
377	0\\
378	0\\
379	0\\
380	0\\
381	0\\
382	0\\
383	0\\
384	0\\
385	0\\
386	0\\
387	0\\
388	0\\
389	0\\
390	0\\
391	0\\
392	0\\
393	0\\
394	0\\
395	0\\
396	0\\
397	0\\
398	0\\
399	0\\
400	0\\
401	0\\
402	0\\
403	0\\
404	0\\
405	0\\
406	0\\
407	0\\
408	0\\
409	0\\
410	0\\
411	0\\
412	0\\
413	0\\
414	0\\
415	0\\
416	0\\
417	0\\
418	0\\
419	0\\
420	0\\
421	0\\
422	0\\
423	0\\
424	0\\
425	0\\
426	0\\
427	0\\
428	0\\
429	0\\
430	0\\
431	0\\
432	0\\
433	0\\
434	0\\
435	0\\
436	0\\
437	0\\
438	0\\
439	0\\
440	0\\
441	0\\
442	0\\
443	0\\
444	0\\
445	0\\
446	0\\
447	0\\
448	0\\
449	0\\
450	0\\
451	0\\
452	0\\
453	0\\
454	0\\
455	0\\
456	0\\
457	0\\
458	0\\
459	0\\
460	0\\
461	0\\
462	0\\
463	0\\
464	0\\
465	0\\
466	0\\
467	0\\
468	0\\
469	0\\
470	0\\
471	0\\
472	0\\
473	0\\
474	0\\
475	0\\
476	0\\
477	0\\
478	0\\
479	0\\
480	0\\
481	0\\
482	0\\
483	0\\
484	0\\
485	0\\
486	0\\
487	0\\
488	0\\
489	0\\
490	0\\
491	0\\
492	0\\
493	0\\
494	0\\
495	0\\
496	0\\
497	0\\
498	0\\
499	0\\
500	0\\
501	0\\
502	0\\
503	0\\
504	0\\
505	0\\
506	0\\
507	0\\
508	0\\
509	0\\
510	0\\
511	0\\
512	0\\
513	0\\
514	0\\
515	0\\
516	0\\
517	0\\
518	0\\
519	0\\
520	0\\
521	0\\
522	0\\
523	0\\
524	0\\
525	0\\
526	0\\
527	0\\
528	0\\
529	0\\
530	0\\
531	0\\
532	0\\
533	0\\
534	0\\
535	0\\
536	0\\
537	0\\
538	0\\
539	0\\
540	0\\
541	0\\
542	0\\
543	0\\
544	0\\
545	0\\
546	0\\
547	4.00690256151494e-05\\
548	9.18803600518231e-05\\
549	0.000134564232292414\\
550	0.000176112314276617\\
551	0.000216585928094907\\
552	0.000256053948070476\\
553	0.000294072813221761\\
554	0.000332498645173348\\
555	0.00037143411280242\\
556	0.000411017553589854\\
557	0.000452836567132838\\
558	0.000496149912580918\\
559	0.000539598870420913\\
560	0.000582438546231535\\
561	0.00062382315388814\\
562	0.000665941388989703\\
563	0.000708824054090575\\
564	0.000752423920588811\\
565	0.000796984050205826\\
566	0.000842614329043036\\
567	0.000892957740392164\\
568	0.000944269325554044\\
569	0.000992198948404602\\
570	0.00103774091284746\\
571	0.00108375253963842\\
572	0.0011302426310335\\
573	0.00117769251944014\\
574	0.00122615490603659\\
575	0.00146936455367717\\
576	0.00201911956940423\\
577	0.00219874416083084\\
578	0.0023072684395088\\
579	0.0023844472193452\\
580	0.00246224470739401\\
581	0.00254141932241292\\
582	0.00262203183295566\\
583	0.00270412311835047\\
584	0.00278773501347819\\
585	0.00287291249876727\\
586	0.00295970668121992\\
587	0.00304818234110382\\
588	0.00313843742910297\\
589	0.00323065362800676\\
590	0.00332522743498845\\
591	0.00342286644802625\\
592	0.00352518337500009\\
593	0.00363635235366165\\
594	0.00376728283869332\\
595	0.0039465900585172\\
596	0.00424941998802836\\
597	0.00487316246495455\\
598	0.00633625614039688\\
599	0\\
600	0\\
};
\addplot [color=blue!50!mycolor7,solid,forget plot]
  table[row sep=crcr]{%
1	0\\
2	0\\
3	0\\
4	0\\
5	0\\
6	0\\
7	0\\
8	0\\
9	0\\
10	0\\
11	0\\
12	0\\
13	0\\
14	0\\
15	0\\
16	0\\
17	0\\
18	0\\
19	0\\
20	0\\
21	0\\
22	0\\
23	0\\
24	0\\
25	0\\
26	0\\
27	0\\
28	0\\
29	0\\
30	0\\
31	0\\
32	0\\
33	0\\
34	0\\
35	0\\
36	0\\
37	0\\
38	0\\
39	0\\
40	0\\
41	0\\
42	0\\
43	0\\
44	0\\
45	0\\
46	0\\
47	0\\
48	0\\
49	0\\
50	0\\
51	0\\
52	0\\
53	0\\
54	0\\
55	0\\
56	0\\
57	0\\
58	0\\
59	0\\
60	0\\
61	0\\
62	0\\
63	0\\
64	0\\
65	0\\
66	0\\
67	0\\
68	0\\
69	0\\
70	0\\
71	0\\
72	0\\
73	0\\
74	0\\
75	0\\
76	0\\
77	0\\
78	0\\
79	0\\
80	0\\
81	0\\
82	0\\
83	0\\
84	0\\
85	0\\
86	0\\
87	0\\
88	0\\
89	0\\
90	0\\
91	0\\
92	0\\
93	0\\
94	0\\
95	0\\
96	0\\
97	0\\
98	0\\
99	0\\
100	0\\
101	0\\
102	0\\
103	0\\
104	0\\
105	0\\
106	0\\
107	0\\
108	0\\
109	0\\
110	0\\
111	0\\
112	0\\
113	0\\
114	0\\
115	0\\
116	0\\
117	0\\
118	0\\
119	0\\
120	0\\
121	0\\
122	0\\
123	0\\
124	0\\
125	0\\
126	0\\
127	0\\
128	0\\
129	0\\
130	0\\
131	0\\
132	0\\
133	0\\
134	0\\
135	0\\
136	0\\
137	0\\
138	0\\
139	0\\
140	0\\
141	0\\
142	0\\
143	0\\
144	0\\
145	0\\
146	0\\
147	0\\
148	0\\
149	0\\
150	0\\
151	0\\
152	0\\
153	0\\
154	0\\
155	0\\
156	0\\
157	0\\
158	0\\
159	0\\
160	0\\
161	0\\
162	0\\
163	0\\
164	0\\
165	0\\
166	0\\
167	0\\
168	0\\
169	0\\
170	0\\
171	0\\
172	0\\
173	0\\
174	0\\
175	0\\
176	0\\
177	0\\
178	0\\
179	0\\
180	0\\
181	0\\
182	0\\
183	0\\
184	0\\
185	0\\
186	0\\
187	0\\
188	0\\
189	0\\
190	0\\
191	0\\
192	0\\
193	0\\
194	0\\
195	0\\
196	0\\
197	0\\
198	0\\
199	0\\
200	0\\
201	0\\
202	0\\
203	0\\
204	0\\
205	0\\
206	0\\
207	0\\
208	0\\
209	0\\
210	0\\
211	0\\
212	0\\
213	0\\
214	0\\
215	0\\
216	0\\
217	0\\
218	0\\
219	0\\
220	0\\
221	0\\
222	0\\
223	0\\
224	0\\
225	0\\
226	0\\
227	0\\
228	0\\
229	0\\
230	0\\
231	0\\
232	0\\
233	0\\
234	0\\
235	0\\
236	0\\
237	0\\
238	0\\
239	0\\
240	0\\
241	0\\
242	0\\
243	0\\
244	0\\
245	0\\
246	0\\
247	0\\
248	0\\
249	0\\
250	0\\
251	0\\
252	0\\
253	0\\
254	0\\
255	0\\
256	0\\
257	0\\
258	0\\
259	0\\
260	0\\
261	0\\
262	0\\
263	0\\
264	0\\
265	0\\
266	0\\
267	0\\
268	0\\
269	0\\
270	0\\
271	0\\
272	0\\
273	0\\
274	0\\
275	0\\
276	0\\
277	0\\
278	0\\
279	0\\
280	0\\
281	0\\
282	0\\
283	0\\
284	0\\
285	0\\
286	0\\
287	0\\
288	0\\
289	0\\
290	0\\
291	0\\
292	0\\
293	0\\
294	0\\
295	0\\
296	0\\
297	0\\
298	0\\
299	0\\
300	0\\
301	0\\
302	0\\
303	0\\
304	0\\
305	0\\
306	0\\
307	0\\
308	0\\
309	0\\
310	0\\
311	0\\
312	0\\
313	0\\
314	0\\
315	0\\
316	0\\
317	0\\
318	0\\
319	0\\
320	0\\
321	0\\
322	0\\
323	0\\
324	0\\
325	0\\
326	0\\
327	0\\
328	0\\
329	0\\
330	0\\
331	0\\
332	0\\
333	0\\
334	0\\
335	0\\
336	0\\
337	0\\
338	0\\
339	0\\
340	0\\
341	0\\
342	0\\
343	0\\
344	0\\
345	0\\
346	0\\
347	0\\
348	0\\
349	0\\
350	0\\
351	0\\
352	0\\
353	0\\
354	0\\
355	0\\
356	0\\
357	0\\
358	0\\
359	0\\
360	0\\
361	0\\
362	0\\
363	0\\
364	0\\
365	0\\
366	0\\
367	0\\
368	0\\
369	0\\
370	0\\
371	0\\
372	0\\
373	0\\
374	0\\
375	0\\
376	0\\
377	0\\
378	0\\
379	0\\
380	0\\
381	0\\
382	0\\
383	0\\
384	0\\
385	0\\
386	0\\
387	0\\
388	0\\
389	0\\
390	0\\
391	0\\
392	0\\
393	0\\
394	0\\
395	0\\
396	0\\
397	0\\
398	0\\
399	0\\
400	0\\
401	0\\
402	0\\
403	0\\
404	0\\
405	0\\
406	0\\
407	0\\
408	0\\
409	0\\
410	0\\
411	0\\
412	0\\
413	0\\
414	0\\
415	0\\
416	0\\
417	0\\
418	0\\
419	0\\
420	0\\
421	0\\
422	0\\
423	0\\
424	0\\
425	0\\
426	0\\
427	0\\
428	0\\
429	0\\
430	0\\
431	0\\
432	0\\
433	0\\
434	0\\
435	0\\
436	0\\
437	0\\
438	0\\
439	0\\
440	0\\
441	0\\
442	0\\
443	0\\
444	0\\
445	0\\
446	0\\
447	0\\
448	0\\
449	0\\
450	0\\
451	0\\
452	0\\
453	0\\
454	0\\
455	0\\
456	0\\
457	0\\
458	0\\
459	0\\
460	0\\
461	0\\
462	0\\
463	0\\
464	0\\
465	0\\
466	0\\
467	0\\
468	0\\
469	0\\
470	0\\
471	0\\
472	0\\
473	0\\
474	0\\
475	0\\
476	0\\
477	0\\
478	0\\
479	0\\
480	0\\
481	0\\
482	0\\
483	0\\
484	0\\
485	0\\
486	0\\
487	0\\
488	0\\
489	0\\
490	0\\
491	0\\
492	0\\
493	0\\
494	0\\
495	0\\
496	0\\
497	0\\
498	0\\
499	0\\
500	0\\
501	0\\
502	0\\
503	0\\
504	0\\
505	0\\
506	0\\
507	0\\
508	0\\
509	0\\
510	0\\
511	0\\
512	0\\
513	0\\
514	0\\
515	0\\
516	0\\
517	0\\
518	0\\
519	0\\
520	0\\
521	0\\
522	0\\
523	0\\
524	0\\
525	0\\
526	0\\
527	0\\
528	0\\
529	0\\
530	0\\
531	0\\
532	0\\
533	0\\
534	0\\
535	0\\
536	0\\
537	0\\
538	0\\
539	0\\
540	0\\
541	0\\
542	0\\
543	0\\
544	0\\
545	0\\
546	0\\
547	0\\
548	0\\
549	0\\
550	4.20820499820094e-05\\
551	0.000111412008282726\\
552	0.000174637757067829\\
553	0.000230566648731088\\
554	0.000277788239069468\\
555	0.000323717941772093\\
556	0.000368646415826842\\
557	0.000412551862509468\\
558	0.000455197630167723\\
559	0.000498339909108939\\
560	0.000542855731364164\\
561	0.000589683232461812\\
562	0.000636530039514592\\
563	0.000683424746743966\\
564	0.000728326965189056\\
565	0.00077355236992366\\
566	0.000819571520899858\\
567	0.000866401369931615\\
568	0.000914114247341885\\
569	0.000965487206123931\\
570	0.00101923516298803\\
571	0.00107141704545175\\
572	0.00111913073277248\\
573	0.0011673538424836\\
574	0.00121604908991694\\
575	0.0012655229210271\\
576	0.00131604936497319\\
577	0.00177799418687613\\
578	0.0022361256288932\\
579	0.00235742919584368\\
580	0.00246133490814186\\
581	0.0025413184666642\\
582	0.00262199004833523\\
583	0.0027041002675783\\
584	0.00278772301988261\\
585	0.00287290661496934\\
586	0.00295970408133531\\
587	0.00304818134793304\\
588	0.00313843711990243\\
589	0.0032306535579562\\
590	0.00332522742625831\\
591	0.00342286644802626\\
592	0.00352518337500009\\
593	0.00363635235366165\\
594	0.00376728283869331\\
595	0.00394659005851718\\
596	0.00424941998802836\\
597	0.00487316246495455\\
598	0.00633625614039688\\
599	0\\
600	0\\
};
\addplot [color=blue!40!mycolor9,solid,forget plot]
  table[row sep=crcr]{%
1	0\\
2	0\\
3	0\\
4	0\\
5	0\\
6	0\\
7	0\\
8	0\\
9	0\\
10	0\\
11	0\\
12	0\\
13	0\\
14	0\\
15	0\\
16	0\\
17	0\\
18	0\\
19	0\\
20	0\\
21	0\\
22	0\\
23	0\\
24	0\\
25	0\\
26	0\\
27	0\\
28	0\\
29	0\\
30	0\\
31	0\\
32	0\\
33	0\\
34	0\\
35	0\\
36	0\\
37	0\\
38	0\\
39	0\\
40	0\\
41	0\\
42	0\\
43	0\\
44	0\\
45	0\\
46	0\\
47	0\\
48	0\\
49	0\\
50	0\\
51	0\\
52	0\\
53	0\\
54	0\\
55	0\\
56	0\\
57	0\\
58	0\\
59	0\\
60	0\\
61	0\\
62	0\\
63	0\\
64	0\\
65	0\\
66	0\\
67	0\\
68	0\\
69	0\\
70	0\\
71	0\\
72	0\\
73	0\\
74	0\\
75	0\\
76	0\\
77	0\\
78	0\\
79	0\\
80	0\\
81	0\\
82	0\\
83	0\\
84	0\\
85	0\\
86	0\\
87	0\\
88	0\\
89	0\\
90	0\\
91	0\\
92	0\\
93	0\\
94	0\\
95	0\\
96	0\\
97	0\\
98	0\\
99	0\\
100	0\\
101	0\\
102	0\\
103	0\\
104	0\\
105	0\\
106	0\\
107	0\\
108	0\\
109	0\\
110	0\\
111	0\\
112	0\\
113	0\\
114	0\\
115	0\\
116	0\\
117	0\\
118	0\\
119	0\\
120	0\\
121	0\\
122	0\\
123	0\\
124	0\\
125	0\\
126	0\\
127	0\\
128	0\\
129	0\\
130	0\\
131	0\\
132	0\\
133	0\\
134	0\\
135	0\\
136	0\\
137	0\\
138	0\\
139	0\\
140	0\\
141	0\\
142	0\\
143	0\\
144	0\\
145	0\\
146	0\\
147	0\\
148	0\\
149	0\\
150	0\\
151	0\\
152	0\\
153	0\\
154	0\\
155	0\\
156	0\\
157	0\\
158	0\\
159	0\\
160	0\\
161	0\\
162	0\\
163	0\\
164	0\\
165	0\\
166	0\\
167	0\\
168	0\\
169	0\\
170	0\\
171	0\\
172	0\\
173	0\\
174	0\\
175	0\\
176	0\\
177	0\\
178	0\\
179	0\\
180	0\\
181	0\\
182	0\\
183	0\\
184	0\\
185	0\\
186	0\\
187	0\\
188	0\\
189	0\\
190	0\\
191	0\\
192	0\\
193	0\\
194	0\\
195	0\\
196	0\\
197	0\\
198	0\\
199	0\\
200	0\\
201	0\\
202	0\\
203	0\\
204	0\\
205	0\\
206	0\\
207	0\\
208	0\\
209	0\\
210	0\\
211	0\\
212	0\\
213	0\\
214	0\\
215	0\\
216	0\\
217	0\\
218	0\\
219	0\\
220	0\\
221	0\\
222	0\\
223	0\\
224	0\\
225	0\\
226	0\\
227	0\\
228	0\\
229	0\\
230	0\\
231	0\\
232	0\\
233	0\\
234	0\\
235	0\\
236	0\\
237	0\\
238	0\\
239	0\\
240	0\\
241	0\\
242	0\\
243	0\\
244	0\\
245	0\\
246	0\\
247	0\\
248	0\\
249	0\\
250	0\\
251	0\\
252	0\\
253	0\\
254	0\\
255	0\\
256	0\\
257	0\\
258	0\\
259	0\\
260	0\\
261	0\\
262	0\\
263	0\\
264	0\\
265	0\\
266	0\\
267	0\\
268	0\\
269	0\\
270	0\\
271	0\\
272	0\\
273	0\\
274	0\\
275	0\\
276	0\\
277	0\\
278	0\\
279	0\\
280	0\\
281	0\\
282	0\\
283	0\\
284	0\\
285	0\\
286	0\\
287	0\\
288	0\\
289	0\\
290	0\\
291	0\\
292	0\\
293	0\\
294	0\\
295	0\\
296	0\\
297	0\\
298	0\\
299	0\\
300	0\\
301	0\\
302	0\\
303	0\\
304	0\\
305	0\\
306	0\\
307	0\\
308	0\\
309	0\\
310	0\\
311	0\\
312	0\\
313	0\\
314	0\\
315	0\\
316	0\\
317	0\\
318	0\\
319	0\\
320	0\\
321	0\\
322	0\\
323	0\\
324	0\\
325	0\\
326	0\\
327	0\\
328	0\\
329	0\\
330	0\\
331	0\\
332	0\\
333	0\\
334	0\\
335	0\\
336	0\\
337	0\\
338	0\\
339	0\\
340	0\\
341	0\\
342	0\\
343	0\\
344	0\\
345	0\\
346	0\\
347	0\\
348	0\\
349	0\\
350	0\\
351	0\\
352	0\\
353	0\\
354	0\\
355	0\\
356	0\\
357	0\\
358	0\\
359	0\\
360	0\\
361	0\\
362	0\\
363	0\\
364	0\\
365	0\\
366	0\\
367	0\\
368	0\\
369	0\\
370	0\\
371	0\\
372	0\\
373	0\\
374	0\\
375	0\\
376	0\\
377	0\\
378	0\\
379	0\\
380	0\\
381	0\\
382	0\\
383	0\\
384	0\\
385	0\\
386	0\\
387	0\\
388	0\\
389	0\\
390	0\\
391	0\\
392	0\\
393	0\\
394	0\\
395	0\\
396	0\\
397	0\\
398	0\\
399	0\\
400	0\\
401	0\\
402	0\\
403	0\\
404	0\\
405	0\\
406	0\\
407	0\\
408	0\\
409	0\\
410	0\\
411	0\\
412	0\\
413	0\\
414	0\\
415	0\\
416	0\\
417	0\\
418	0\\
419	0\\
420	0\\
421	0\\
422	0\\
423	0\\
424	0\\
425	0\\
426	0\\
427	0\\
428	0\\
429	0\\
430	0\\
431	0\\
432	0\\
433	0\\
434	0\\
435	0\\
436	0\\
437	0\\
438	0\\
439	0\\
440	0\\
441	0\\
442	0\\
443	0\\
444	0\\
445	0\\
446	0\\
447	0\\
448	0\\
449	0\\
450	0\\
451	0\\
452	0\\
453	0\\
454	0\\
455	0\\
456	0\\
457	0\\
458	0\\
459	0\\
460	0\\
461	0\\
462	0\\
463	0\\
464	0\\
465	0\\
466	0\\
467	0\\
468	0\\
469	0\\
470	0\\
471	0\\
472	0\\
473	0\\
474	0\\
475	0\\
476	0\\
477	0\\
478	0\\
479	0\\
480	0\\
481	0\\
482	0\\
483	0\\
484	0\\
485	0\\
486	0\\
487	0\\
488	0\\
489	0\\
490	0\\
491	0\\
492	0\\
493	0\\
494	0\\
495	0\\
496	0\\
497	0\\
498	0\\
499	0\\
500	0\\
501	0\\
502	0\\
503	0\\
504	0\\
505	0\\
506	0\\
507	0\\
508	0\\
509	0\\
510	0\\
511	0\\
512	0\\
513	0\\
514	0\\
515	0\\
516	0\\
517	0\\
518	0\\
519	0\\
520	0\\
521	0\\
522	0\\
523	0\\
524	0\\
525	0\\
526	0\\
527	0\\
528	0\\
529	0\\
530	0\\
531	0\\
532	0\\
533	0\\
534	0\\
535	0\\
536	0\\
537	0\\
538	0\\
539	0\\
540	0\\
541	0\\
542	0\\
543	0\\
544	0\\
545	0\\
546	0\\
547	0\\
548	0\\
549	0\\
550	0\\
551	0\\
552	0\\
553	5.16889552756766e-06\\
554	8.77826221569004e-05\\
555	0.000167631433400357\\
556	0.000243418760937493\\
557	0.000313414182083021\\
558	0.000375902069304456\\
559	0.00042874744998465\\
560	0.000480599081949562\\
561	0.000531451904771833\\
562	0.000581189982271827\\
563	0.000629635906988094\\
564	0.000680653412810273\\
565	0.00073201619529141\\
566	0.000783309644171608\\
567	0.000833969941358776\\
568	0.000882957822718271\\
569	0.000932718658926889\\
570	0.000983330016376749\\
571	0.00103559259416003\\
572	0.00109244755817989\\
573	0.0011485579817111\\
574	0.00120114230828459\\
575	0.00125198809055414\\
576	0.00130338498565377\\
577	0.00135527218410646\\
578	0.00149765035674765\\
579	0.00201460319803586\\
580	0.00239071852558201\\
581	0.00251424382398314\\
582	0.00262058365852551\\
583	0.00270381978473586\\
584	0.00278757879201644\\
585	0.00287283135283087\\
586	0.00295966537365807\\
587	0.00304816311789294\\
588	0.00313842965532306\\
589	0.00323065104580206\\
590	0.00332522680618373\\
591	0.00342286636301155\\
592	0.00352518337500009\\
593	0.00363635235366165\\
594	0.00376728283869333\\
595	0.0039465900585172\\
596	0.00424941998802837\\
597	0.00487316246495455\\
598	0.00633625614039688\\
599	0\\
600	0\\
};
\addplot [color=blue!75!mycolor7,solid,forget plot]
  table[row sep=crcr]{%
1	0\\
2	0\\
3	0\\
4	0\\
5	0\\
6	0\\
7	0\\
8	0\\
9	0\\
10	0\\
11	0\\
12	0\\
13	0\\
14	0\\
15	0\\
16	0\\
17	0\\
18	0\\
19	0\\
20	0\\
21	0\\
22	0\\
23	0\\
24	0\\
25	0\\
26	0\\
27	0\\
28	0\\
29	0\\
30	0\\
31	0\\
32	0\\
33	0\\
34	0\\
35	0\\
36	0\\
37	0\\
38	0\\
39	0\\
40	0\\
41	0\\
42	0\\
43	0\\
44	0\\
45	0\\
46	0\\
47	0\\
48	0\\
49	0\\
50	0\\
51	0\\
52	0\\
53	0\\
54	0\\
55	0\\
56	0\\
57	0\\
58	0\\
59	0\\
60	0\\
61	0\\
62	0\\
63	0\\
64	0\\
65	0\\
66	0\\
67	0\\
68	0\\
69	0\\
70	0\\
71	0\\
72	0\\
73	0\\
74	0\\
75	0\\
76	0\\
77	0\\
78	0\\
79	0\\
80	0\\
81	0\\
82	0\\
83	0\\
84	0\\
85	0\\
86	0\\
87	0\\
88	0\\
89	0\\
90	0\\
91	0\\
92	0\\
93	0\\
94	0\\
95	0\\
96	0\\
97	0\\
98	0\\
99	0\\
100	0\\
101	0\\
102	0\\
103	0\\
104	0\\
105	0\\
106	0\\
107	0\\
108	0\\
109	0\\
110	0\\
111	0\\
112	0\\
113	0\\
114	0\\
115	0\\
116	0\\
117	0\\
118	0\\
119	0\\
120	0\\
121	0\\
122	0\\
123	0\\
124	0\\
125	0\\
126	0\\
127	0\\
128	0\\
129	0\\
130	0\\
131	0\\
132	0\\
133	0\\
134	0\\
135	0\\
136	0\\
137	0\\
138	0\\
139	0\\
140	0\\
141	0\\
142	0\\
143	0\\
144	0\\
145	0\\
146	0\\
147	0\\
148	0\\
149	0\\
150	0\\
151	0\\
152	0\\
153	0\\
154	0\\
155	0\\
156	0\\
157	0\\
158	0\\
159	0\\
160	0\\
161	0\\
162	0\\
163	0\\
164	0\\
165	0\\
166	0\\
167	0\\
168	0\\
169	0\\
170	0\\
171	0\\
172	0\\
173	0\\
174	0\\
175	0\\
176	0\\
177	0\\
178	0\\
179	0\\
180	0\\
181	0\\
182	0\\
183	0\\
184	0\\
185	0\\
186	0\\
187	0\\
188	0\\
189	0\\
190	0\\
191	0\\
192	0\\
193	0\\
194	0\\
195	0\\
196	0\\
197	0\\
198	0\\
199	0\\
200	0\\
201	0\\
202	0\\
203	0\\
204	0\\
205	0\\
206	0\\
207	0\\
208	0\\
209	0\\
210	0\\
211	0\\
212	0\\
213	0\\
214	0\\
215	0\\
216	0\\
217	0\\
218	0\\
219	0\\
220	0\\
221	0\\
222	0\\
223	0\\
224	0\\
225	0\\
226	0\\
227	0\\
228	0\\
229	0\\
230	0\\
231	0\\
232	0\\
233	0\\
234	0\\
235	0\\
236	0\\
237	0\\
238	0\\
239	0\\
240	0\\
241	0\\
242	0\\
243	0\\
244	0\\
245	0\\
246	0\\
247	0\\
248	0\\
249	0\\
250	0\\
251	0\\
252	0\\
253	0\\
254	0\\
255	0\\
256	0\\
257	0\\
258	0\\
259	0\\
260	0\\
261	0\\
262	0\\
263	0\\
264	0\\
265	0\\
266	0\\
267	0\\
268	0\\
269	0\\
270	0\\
271	0\\
272	0\\
273	0\\
274	0\\
275	0\\
276	0\\
277	0\\
278	0\\
279	0\\
280	0\\
281	0\\
282	0\\
283	0\\
284	0\\
285	0\\
286	0\\
287	0\\
288	0\\
289	0\\
290	0\\
291	0\\
292	0\\
293	0\\
294	0\\
295	0\\
296	0\\
297	0\\
298	0\\
299	0\\
300	0\\
301	0\\
302	0\\
303	0\\
304	0\\
305	0\\
306	0\\
307	0\\
308	0\\
309	0\\
310	0\\
311	0\\
312	0\\
313	0\\
314	0\\
315	0\\
316	0\\
317	0\\
318	0\\
319	0\\
320	0\\
321	0\\
322	0\\
323	0\\
324	0\\
325	0\\
326	0\\
327	0\\
328	0\\
329	0\\
330	0\\
331	0\\
332	0\\
333	0\\
334	0\\
335	0\\
336	0\\
337	0\\
338	0\\
339	0\\
340	0\\
341	0\\
342	0\\
343	0\\
344	0\\
345	0\\
346	0\\
347	0\\
348	0\\
349	0\\
350	0\\
351	0\\
352	0\\
353	0\\
354	0\\
355	0\\
356	0\\
357	0\\
358	0\\
359	0\\
360	0\\
361	0\\
362	0\\
363	0\\
364	0\\
365	0\\
366	0\\
367	0\\
368	0\\
369	0\\
370	0\\
371	0\\
372	0\\
373	0\\
374	0\\
375	0\\
376	0\\
377	0\\
378	0\\
379	0\\
380	0\\
381	0\\
382	0\\
383	0\\
384	0\\
385	0\\
386	0\\
387	0\\
388	0\\
389	0\\
390	0\\
391	0\\
392	0\\
393	0\\
394	0\\
395	0\\
396	0\\
397	0\\
398	0\\
399	0\\
400	0\\
401	0\\
402	0\\
403	0\\
404	0\\
405	0\\
406	0\\
407	0\\
408	0\\
409	0\\
410	0\\
411	0\\
412	0\\
413	0\\
414	0\\
415	0\\
416	0\\
417	0\\
418	0\\
419	0\\
420	0\\
421	0\\
422	0\\
423	0\\
424	0\\
425	0\\
426	0\\
427	0\\
428	0\\
429	0\\
430	0\\
431	0\\
432	0\\
433	0\\
434	0\\
435	0\\
436	0\\
437	0\\
438	0\\
439	0\\
440	0\\
441	0\\
442	0\\
443	0\\
444	0\\
445	0\\
446	0\\
447	0\\
448	0\\
449	0\\
450	0\\
451	0\\
452	0\\
453	0\\
454	0\\
455	0\\
456	0\\
457	0\\
458	0\\
459	0\\
460	0\\
461	0\\
462	0\\
463	0\\
464	0\\
465	0\\
466	0\\
467	0\\
468	0\\
469	0\\
470	0\\
471	0\\
472	0\\
473	0\\
474	0\\
475	0\\
476	0\\
477	0\\
478	0\\
479	0\\
480	0\\
481	0\\
482	0\\
483	0\\
484	0\\
485	0\\
486	0\\
487	0\\
488	0\\
489	0\\
490	0\\
491	0\\
492	0\\
493	0\\
494	0\\
495	0\\
496	0\\
497	0\\
498	0\\
499	0\\
500	0\\
501	0\\
502	0\\
503	0\\
504	0\\
505	0\\
506	0\\
507	0\\
508	0\\
509	0\\
510	0\\
511	0\\
512	0\\
513	0\\
514	0\\
515	0\\
516	0\\
517	0\\
518	0\\
519	0\\
520	0\\
521	0\\
522	0\\
523	0\\
524	0\\
525	0\\
526	0\\
527	0\\
528	0\\
529	0\\
530	0\\
531	0\\
532	0\\
533	0\\
534	0\\
535	0\\
536	0\\
537	0\\
538	0\\
539	0\\
540	0\\
541	0\\
542	0\\
543	0\\
544	0\\
545	0\\
546	0\\
547	0\\
548	0\\
549	0\\
550	0\\
551	0\\
552	0\\
553	0\\
554	0\\
555	0\\
556	0\\
557	0\\
558	0.000110242963623734\\
559	0.00019992901139366\\
560	0.000287549546867388\\
561	0.000371614328687814\\
562	0.000450389472560112\\
563	0.000523067401835359\\
564	0.000586315516345723\\
565	0.000645586600008865\\
566	0.000703925763667472\\
567	0.000761855676457282\\
568	0.000819911089405896\\
569	0.000877192507218882\\
570	0.000934343502707047\\
571	0.000990382564388671\\
572	0.00104522036747947\\
573	0.00110068131805173\\
574	0.00115976594531527\\
575	0.0012203982545827\\
576	0.00128005598982308\\
577	0.00133517164128012\\
578	0.00138993979421614\\
579	0.00144525306241601\\
580	0.0016674712958016\\
581	0.00217281737767825\\
582	0.0025404360521811\\
583	0.00266811515796659\\
584	0.00278469419261029\\
585	0.00287171693370239\\
586	0.00295919258138545\\
587	0.00304791879573238\\
588	0.00313830670140735\\
589	0.00323059700557465\\
590	0.00332520698609584\\
591	0.00342286098677809\\
592	0.00352518255333602\\
593	0.00363635235366164\\
594	0.00376728283869332\\
595	0.00394659005851719\\
596	0.00424941998802836\\
597	0.00487316246495455\\
598	0.00633625614039688\\
599	0\\
600	0\\
};
\addplot [color=blue!80!mycolor9,solid,forget plot]
  table[row sep=crcr]{%
1	0\\
2	0\\
3	0\\
4	0\\
5	0\\
6	0\\
7	0\\
8	0\\
9	0\\
10	0\\
11	0\\
12	0\\
13	0\\
14	0\\
15	0\\
16	0\\
17	0\\
18	0\\
19	0\\
20	0\\
21	0\\
22	0\\
23	0\\
24	0\\
25	0\\
26	0\\
27	0\\
28	0\\
29	0\\
30	0\\
31	0\\
32	0\\
33	0\\
34	0\\
35	0\\
36	0\\
37	0\\
38	0\\
39	0\\
40	0\\
41	0\\
42	0\\
43	0\\
44	0\\
45	0\\
46	0\\
47	0\\
48	0\\
49	0\\
50	0\\
51	0\\
52	0\\
53	0\\
54	0\\
55	0\\
56	0\\
57	0\\
58	0\\
59	0\\
60	0\\
61	0\\
62	0\\
63	0\\
64	0\\
65	0\\
66	0\\
67	0\\
68	0\\
69	0\\
70	0\\
71	0\\
72	0\\
73	0\\
74	0\\
75	0\\
76	0\\
77	0\\
78	0\\
79	0\\
80	0\\
81	0\\
82	0\\
83	0\\
84	0\\
85	0\\
86	0\\
87	0\\
88	0\\
89	0\\
90	0\\
91	0\\
92	0\\
93	0\\
94	0\\
95	0\\
96	0\\
97	0\\
98	0\\
99	0\\
100	0\\
101	0\\
102	0\\
103	0\\
104	0\\
105	0\\
106	0\\
107	0\\
108	0\\
109	0\\
110	0\\
111	0\\
112	0\\
113	0\\
114	0\\
115	0\\
116	0\\
117	0\\
118	0\\
119	0\\
120	0\\
121	0\\
122	0\\
123	0\\
124	0\\
125	0\\
126	0\\
127	0\\
128	0\\
129	0\\
130	0\\
131	0\\
132	0\\
133	0\\
134	0\\
135	0\\
136	0\\
137	0\\
138	0\\
139	0\\
140	0\\
141	0\\
142	0\\
143	0\\
144	0\\
145	0\\
146	0\\
147	0\\
148	0\\
149	0\\
150	0\\
151	0\\
152	0\\
153	0\\
154	0\\
155	0\\
156	0\\
157	0\\
158	0\\
159	0\\
160	0\\
161	0\\
162	0\\
163	0\\
164	0\\
165	0\\
166	0\\
167	0\\
168	0\\
169	0\\
170	0\\
171	0\\
172	0\\
173	0\\
174	0\\
175	0\\
176	0\\
177	0\\
178	0\\
179	0\\
180	0\\
181	0\\
182	0\\
183	0\\
184	0\\
185	0\\
186	0\\
187	0\\
188	0\\
189	0\\
190	0\\
191	0\\
192	0\\
193	0\\
194	0\\
195	0\\
196	0\\
197	0\\
198	0\\
199	0\\
200	0\\
201	0\\
202	0\\
203	0\\
204	0\\
205	0\\
206	0\\
207	0\\
208	0\\
209	0\\
210	0\\
211	0\\
212	0\\
213	0\\
214	0\\
215	0\\
216	0\\
217	0\\
218	0\\
219	0\\
220	0\\
221	0\\
222	0\\
223	0\\
224	0\\
225	0\\
226	0\\
227	0\\
228	0\\
229	0\\
230	0\\
231	0\\
232	0\\
233	0\\
234	0\\
235	0\\
236	0\\
237	0\\
238	0\\
239	0\\
240	0\\
241	0\\
242	0\\
243	0\\
244	0\\
245	0\\
246	0\\
247	0\\
248	0\\
249	0\\
250	0\\
251	0\\
252	0\\
253	0\\
254	0\\
255	0\\
256	0\\
257	0\\
258	0\\
259	0\\
260	0\\
261	0\\
262	0\\
263	0\\
264	0\\
265	0\\
266	0\\
267	0\\
268	0\\
269	0\\
270	0\\
271	0\\
272	0\\
273	0\\
274	0\\
275	0\\
276	0\\
277	0\\
278	0\\
279	0\\
280	0\\
281	0\\
282	0\\
283	0\\
284	0\\
285	0\\
286	0\\
287	0\\
288	0\\
289	0\\
290	0\\
291	0\\
292	0\\
293	0\\
294	0\\
295	0\\
296	0\\
297	0\\
298	0\\
299	0\\
300	0\\
301	0\\
302	0\\
303	0\\
304	0\\
305	0\\
306	0\\
307	0\\
308	0\\
309	0\\
310	0\\
311	0\\
312	0\\
313	0\\
314	0\\
315	0\\
316	0\\
317	0\\
318	0\\
319	0\\
320	0\\
321	0\\
322	0\\
323	0\\
324	0\\
325	0\\
326	0\\
327	0\\
328	0\\
329	0\\
330	0\\
331	0\\
332	0\\
333	0\\
334	0\\
335	0\\
336	0\\
337	0\\
338	0\\
339	0\\
340	0\\
341	0\\
342	0\\
343	0\\
344	0\\
345	0\\
346	0\\
347	0\\
348	0\\
349	0\\
350	0\\
351	0\\
352	0\\
353	0\\
354	0\\
355	0\\
356	0\\
357	0\\
358	0\\
359	0\\
360	0\\
361	0\\
362	0\\
363	0\\
364	0\\
365	0\\
366	0\\
367	0\\
368	0\\
369	0\\
370	0\\
371	0\\
372	0\\
373	0\\
374	0\\
375	0\\
376	0\\
377	0\\
378	0\\
379	0\\
380	0\\
381	0\\
382	0\\
383	0\\
384	0\\
385	0\\
386	0\\
387	0\\
388	0\\
389	0\\
390	0\\
391	0\\
392	0\\
393	0\\
394	0\\
395	0\\
396	0\\
397	0\\
398	0\\
399	0\\
400	0\\
401	0\\
402	0\\
403	0\\
404	0\\
405	0\\
406	0\\
407	0\\
408	0\\
409	0\\
410	0\\
411	0\\
412	0\\
413	0\\
414	0\\
415	0\\
416	0\\
417	0\\
418	0\\
419	0\\
420	0\\
421	0\\
422	0\\
423	0\\
424	0\\
425	0\\
426	0\\
427	0\\
428	0\\
429	0\\
430	0\\
431	0\\
432	0\\
433	0\\
434	0\\
435	0\\
436	0\\
437	0\\
438	0\\
439	0\\
440	0\\
441	0\\
442	0\\
443	0\\
444	0\\
445	0\\
446	0\\
447	0\\
448	0\\
449	0\\
450	0\\
451	0\\
452	0\\
453	0\\
454	0\\
455	0\\
456	0\\
457	0\\
458	0\\
459	0\\
460	0\\
461	0\\
462	0\\
463	0\\
464	0\\
465	0\\
466	0\\
467	0\\
468	0\\
469	0\\
470	0\\
471	0\\
472	0\\
473	0\\
474	0\\
475	0\\
476	0\\
477	0\\
478	0\\
479	0\\
480	0\\
481	0\\
482	0\\
483	0\\
484	0\\
485	0\\
486	0\\
487	0\\
488	0\\
489	0\\
490	0\\
491	0\\
492	0\\
493	0\\
494	0\\
495	0\\
496	0\\
497	0\\
498	0\\
499	0\\
500	0\\
501	0\\
502	0\\
503	0\\
504	0\\
505	0\\
506	0\\
507	0\\
508	0\\
509	0\\
510	0\\
511	0\\
512	0\\
513	0\\
514	0\\
515	0\\
516	0\\
517	0\\
518	0\\
519	0\\
520	0\\
521	0\\
522	0\\
523	0\\
524	0\\
525	0\\
526	0\\
527	0\\
528	0\\
529	0\\
530	0\\
531	0\\
532	0\\
533	0\\
534	0\\
535	0\\
536	0\\
537	0\\
538	0\\
539	0\\
540	0\\
541	0\\
542	0\\
543	0\\
544	0\\
545	0\\
546	0\\
547	0\\
548	0\\
549	0\\
550	0\\
551	0\\
552	0\\
553	0\\
554	0\\
555	0\\
556	0\\
557	0\\
558	0\\
559	0\\
560	0\\
561	0\\
562	0\\
563	0.000159515699674089\\
564	0.000298612314061124\\
565	0.000395677995910523\\
566	0.000490056265484305\\
567	0.00058073290802797\\
568	0.000667074587023057\\
569	0.000745398135148641\\
570	0.000813989669797523\\
571	0.000882939921427012\\
572	0.000952276485464555\\
573	0.00101907934158752\\
574	0.00108458248882874\\
575	0.00114854952533438\\
576	0.00121114417608812\\
577	0.00127807087113535\\
578	0.00134469341072086\\
579	0.00141007143413669\\
580	0.00147057610994654\\
581	0.0015307007770952\\
582	0.00175408882179572\\
583	0.00224638378967943\\
584	0.00268185823056266\\
585	0.00281454711196381\\
586	0.00294329313072073\\
587	0.0030443100960092\\
588	0.00313678597680714\\
589	0.00322978330798573\\
590	0.00332482940415523\\
591	0.00342270866919969\\
592	0.00352513673636771\\
593	0.00363634439388895\\
594	0.00376728283869332\\
595	0.00394659005851719\\
596	0.00424941998802837\\
597	0.00487316246495455\\
598	0.00633625614039688\\
599	0\\
600	0\\
};
\addplot [color=blue,solid,forget plot]
  table[row sep=crcr]{%
1	0\\
2	0\\
3	0\\
4	0\\
5	0\\
6	0\\
7	0\\
8	0\\
9	0\\
10	0\\
11	0\\
12	0\\
13	0\\
14	0\\
15	0\\
16	0\\
17	0\\
18	0\\
19	0\\
20	0\\
21	0\\
22	0\\
23	0\\
24	0\\
25	0\\
26	0\\
27	0\\
28	0\\
29	0\\
30	0\\
31	0\\
32	0\\
33	0\\
34	0\\
35	0\\
36	0\\
37	0\\
38	0\\
39	0\\
40	0\\
41	0\\
42	0\\
43	0\\
44	0\\
45	0\\
46	0\\
47	0\\
48	0\\
49	0\\
50	0\\
51	0\\
52	0\\
53	0\\
54	0\\
55	0\\
56	0\\
57	0\\
58	0\\
59	0\\
60	0\\
61	0\\
62	0\\
63	0\\
64	0\\
65	0\\
66	0\\
67	0\\
68	0\\
69	0\\
70	0\\
71	0\\
72	0\\
73	0\\
74	0\\
75	0\\
76	0\\
77	0\\
78	0\\
79	0\\
80	0\\
81	0\\
82	0\\
83	0\\
84	0\\
85	0\\
86	0\\
87	0\\
88	0\\
89	0\\
90	0\\
91	0\\
92	0\\
93	0\\
94	0\\
95	0\\
96	0\\
97	0\\
98	0\\
99	0\\
100	0\\
101	0\\
102	0\\
103	0\\
104	0\\
105	0\\
106	0\\
107	0\\
108	0\\
109	0\\
110	0\\
111	0\\
112	0\\
113	0\\
114	0\\
115	0\\
116	0\\
117	0\\
118	0\\
119	0\\
120	0\\
121	0\\
122	0\\
123	0\\
124	0\\
125	0\\
126	0\\
127	0\\
128	0\\
129	0\\
130	0\\
131	0\\
132	0\\
133	0\\
134	0\\
135	0\\
136	0\\
137	0\\
138	0\\
139	0\\
140	0\\
141	0\\
142	0\\
143	0\\
144	0\\
145	0\\
146	0\\
147	0\\
148	0\\
149	0\\
150	0\\
151	0\\
152	0\\
153	0\\
154	0\\
155	0\\
156	0\\
157	0\\
158	0\\
159	0\\
160	0\\
161	0\\
162	0\\
163	0\\
164	0\\
165	0\\
166	0\\
167	0\\
168	0\\
169	0\\
170	0\\
171	0\\
172	0\\
173	0\\
174	0\\
175	0\\
176	0\\
177	0\\
178	0\\
179	0\\
180	0\\
181	0\\
182	0\\
183	0\\
184	0\\
185	0\\
186	0\\
187	0\\
188	0\\
189	0\\
190	0\\
191	0\\
192	0\\
193	0\\
194	0\\
195	0\\
196	0\\
197	0\\
198	0\\
199	0\\
200	0\\
201	0\\
202	0\\
203	0\\
204	0\\
205	0\\
206	0\\
207	0\\
208	0\\
209	0\\
210	0\\
211	0\\
212	0\\
213	0\\
214	0\\
215	0\\
216	0\\
217	0\\
218	0\\
219	0\\
220	0\\
221	0\\
222	0\\
223	0\\
224	0\\
225	0\\
226	0\\
227	0\\
228	0\\
229	0\\
230	0\\
231	0\\
232	0\\
233	0\\
234	0\\
235	0\\
236	0\\
237	0\\
238	0\\
239	0\\
240	0\\
241	0\\
242	0\\
243	0\\
244	0\\
245	0\\
246	0\\
247	0\\
248	0\\
249	0\\
250	0\\
251	0\\
252	0\\
253	0\\
254	0\\
255	0\\
256	0\\
257	0\\
258	0\\
259	0\\
260	0\\
261	0\\
262	0\\
263	0\\
264	0\\
265	0\\
266	0\\
267	0\\
268	0\\
269	0\\
270	0\\
271	0\\
272	0\\
273	0\\
274	0\\
275	0\\
276	0\\
277	0\\
278	0\\
279	0\\
280	0\\
281	0\\
282	0\\
283	0\\
284	0\\
285	0\\
286	0\\
287	0\\
288	0\\
289	0\\
290	0\\
291	0\\
292	0\\
293	0\\
294	0\\
295	0\\
296	0\\
297	0\\
298	0\\
299	0\\
300	0\\
301	0\\
302	0\\
303	0\\
304	0\\
305	0\\
306	0\\
307	0\\
308	0\\
309	0\\
310	0\\
311	0\\
312	0\\
313	0\\
314	0\\
315	0\\
316	0\\
317	0\\
318	0\\
319	0\\
320	0\\
321	0\\
322	0\\
323	0\\
324	0\\
325	0\\
326	0\\
327	0\\
328	0\\
329	0\\
330	0\\
331	0\\
332	0\\
333	0\\
334	0\\
335	0\\
336	0\\
337	0\\
338	0\\
339	0\\
340	0\\
341	0\\
342	0\\
343	0\\
344	0\\
345	0\\
346	0\\
347	0\\
348	0\\
349	0\\
350	0\\
351	0\\
352	0\\
353	0\\
354	0\\
355	0\\
356	0\\
357	0\\
358	0\\
359	0\\
360	0\\
361	0\\
362	0\\
363	0\\
364	0\\
365	0\\
366	0\\
367	0\\
368	0\\
369	0\\
370	0\\
371	0\\
372	0\\
373	0\\
374	0\\
375	0\\
376	0\\
377	0\\
378	0\\
379	0\\
380	0\\
381	0\\
382	0\\
383	0\\
384	0\\
385	0\\
386	0\\
387	0\\
388	0\\
389	0\\
390	0\\
391	0\\
392	0\\
393	0\\
394	0\\
395	0\\
396	0\\
397	0\\
398	0\\
399	0\\
400	0\\
401	0\\
402	0\\
403	0\\
404	0\\
405	0\\
406	0\\
407	0\\
408	0\\
409	0\\
410	0\\
411	0\\
412	0\\
413	0\\
414	0\\
415	0\\
416	0\\
417	0\\
418	0\\
419	0\\
420	0\\
421	0\\
422	0\\
423	0\\
424	0\\
425	0\\
426	0\\
427	0\\
428	0\\
429	0\\
430	0\\
431	0\\
432	0\\
433	0\\
434	0\\
435	0\\
436	0\\
437	0\\
438	0\\
439	0\\
440	0\\
441	0\\
442	0\\
443	0\\
444	0\\
445	0\\
446	0\\
447	0\\
448	0\\
449	0\\
450	0\\
451	0\\
452	0\\
453	0\\
454	0\\
455	0\\
456	0\\
457	0\\
458	0\\
459	0\\
460	0\\
461	0\\
462	0\\
463	0\\
464	0\\
465	0\\
466	0\\
467	0\\
468	0\\
469	0\\
470	0\\
471	0\\
472	0\\
473	0\\
474	0\\
475	0\\
476	0\\
477	0\\
478	0\\
479	0\\
480	0\\
481	0\\
482	0\\
483	0\\
484	0\\
485	0\\
486	0\\
487	0\\
488	0\\
489	0\\
490	0\\
491	0\\
492	0\\
493	0\\
494	0\\
495	0\\
496	0\\
497	0\\
498	0\\
499	0\\
500	0\\
501	0\\
502	0\\
503	0\\
504	0\\
505	0\\
506	0\\
507	0\\
508	0\\
509	0\\
510	0\\
511	0\\
512	0\\
513	0\\
514	0\\
515	0\\
516	0\\
517	0\\
518	0\\
519	0\\
520	0\\
521	0\\
522	0\\
523	0\\
524	0\\
525	0\\
526	0\\
527	0\\
528	0\\
529	0\\
530	0\\
531	0\\
532	0\\
533	0\\
534	0\\
535	0\\
536	0\\
537	0\\
538	0\\
539	0\\
540	0\\
541	0\\
542	0\\
543	0\\
544	0\\
545	0\\
546	0\\
547	0\\
548	0\\
549	0\\
550	0\\
551	0\\
552	0\\
553	0\\
554	0\\
555	0\\
556	0\\
557	0\\
558	0\\
559	0\\
560	0\\
561	0\\
562	0\\
563	0\\
564	0\\
565	0\\
566	0\\
567	0\\
568	0.000141273876810025\\
569	0.000370678299245233\\
570	0.000478798156568623\\
571	0.000585374459440977\\
572	0.000689800406590731\\
573	0.000792165578853567\\
574	0.000889405598368621\\
575	0.000979616328357942\\
576	0.00106208975066892\\
577	0.00114339711980862\\
578	0.00122278879802299\\
579	0.00130020813270453\\
580	0.00137791300105533\\
581	0.0014543721388116\\
582	0.00152928169195912\\
583	0.00160026462104487\\
584	0.00174508295417719\\
585	0.00222608820292352\\
586	0.00271976975471917\\
587	0.00294688553227509\\
588	0.00308669213329453\\
589	0.003219110126369\\
590	0.0033189542078428\\
591	0.00342012153578412\\
592	0.00352397614616107\\
593	0.00363595699515989\\
594	0.00376720386728385\\
595	0.00394659005851719\\
596	0.00424941998802836\\
597	0.00487316246495455\\
598	0.00633625614039688\\
599	0\\
600	0\\
};
\addplot [color=mycolor10,solid,forget plot]
  table[row sep=crcr]{%
1	0\\
2	0\\
3	0\\
4	0\\
5	0\\
6	0\\
7	0\\
8	0\\
9	0\\
10	0\\
11	0\\
12	0\\
13	0\\
14	0\\
15	0\\
16	0\\
17	0\\
18	0\\
19	0\\
20	0\\
21	0\\
22	0\\
23	0\\
24	0\\
25	0\\
26	0\\
27	0\\
28	0\\
29	0\\
30	0\\
31	0\\
32	0\\
33	0\\
34	0\\
35	0\\
36	0\\
37	0\\
38	0\\
39	0\\
40	0\\
41	0\\
42	0\\
43	0\\
44	0\\
45	0\\
46	0\\
47	0\\
48	0\\
49	0\\
50	0\\
51	0\\
52	0\\
53	0\\
54	0\\
55	0\\
56	0\\
57	0\\
58	0\\
59	0\\
60	0\\
61	0\\
62	0\\
63	0\\
64	0\\
65	0\\
66	0\\
67	0\\
68	0\\
69	0\\
70	0\\
71	0\\
72	0\\
73	0\\
74	0\\
75	0\\
76	0\\
77	0\\
78	0\\
79	0\\
80	0\\
81	0\\
82	0\\
83	0\\
84	0\\
85	0\\
86	0\\
87	0\\
88	0\\
89	0\\
90	0\\
91	0\\
92	0\\
93	0\\
94	0\\
95	0\\
96	0\\
97	0\\
98	0\\
99	0\\
100	0\\
101	0\\
102	0\\
103	0\\
104	0\\
105	0\\
106	0\\
107	0\\
108	0\\
109	0\\
110	0\\
111	0\\
112	0\\
113	0\\
114	0\\
115	0\\
116	0\\
117	0\\
118	0\\
119	0\\
120	0\\
121	0\\
122	0\\
123	0\\
124	0\\
125	0\\
126	0\\
127	0\\
128	0\\
129	0\\
130	0\\
131	0\\
132	0\\
133	0\\
134	0\\
135	0\\
136	0\\
137	0\\
138	0\\
139	0\\
140	0\\
141	0\\
142	0\\
143	0\\
144	0\\
145	0\\
146	0\\
147	0\\
148	0\\
149	0\\
150	0\\
151	0\\
152	0\\
153	0\\
154	0\\
155	0\\
156	0\\
157	0\\
158	0\\
159	0\\
160	0\\
161	0\\
162	0\\
163	0\\
164	0\\
165	0\\
166	0\\
167	0\\
168	0\\
169	0\\
170	0\\
171	0\\
172	0\\
173	0\\
174	0\\
175	0\\
176	0\\
177	0\\
178	0\\
179	0\\
180	0\\
181	0\\
182	0\\
183	0\\
184	0\\
185	0\\
186	0\\
187	0\\
188	0\\
189	0\\
190	0\\
191	0\\
192	0\\
193	0\\
194	0\\
195	0\\
196	0\\
197	0\\
198	0\\
199	0\\
200	0\\
201	0\\
202	0\\
203	0\\
204	0\\
205	0\\
206	0\\
207	0\\
208	0\\
209	0\\
210	0\\
211	0\\
212	0\\
213	0\\
214	0\\
215	0\\
216	0\\
217	0\\
218	0\\
219	0\\
220	0\\
221	0\\
222	0\\
223	0\\
224	0\\
225	0\\
226	0\\
227	0\\
228	0\\
229	0\\
230	0\\
231	0\\
232	0\\
233	0\\
234	0\\
235	0\\
236	0\\
237	0\\
238	0\\
239	0\\
240	0\\
241	0\\
242	0\\
243	0\\
244	0\\
245	0\\
246	0\\
247	0\\
248	0\\
249	0\\
250	0\\
251	0\\
252	0\\
253	0\\
254	0\\
255	0\\
256	0\\
257	0\\
258	0\\
259	0\\
260	0\\
261	0\\
262	0\\
263	0\\
264	0\\
265	0\\
266	0\\
267	0\\
268	0\\
269	0\\
270	0\\
271	0\\
272	0\\
273	0\\
274	0\\
275	0\\
276	0\\
277	0\\
278	0\\
279	0\\
280	0\\
281	0\\
282	0\\
283	0\\
284	0\\
285	0\\
286	0\\
287	0\\
288	0\\
289	0\\
290	0\\
291	0\\
292	0\\
293	0\\
294	0\\
295	0\\
296	0\\
297	0\\
298	0\\
299	0\\
300	0\\
301	0\\
302	0\\
303	0\\
304	0\\
305	0\\
306	0\\
307	0\\
308	0\\
309	0\\
310	0\\
311	0\\
312	0\\
313	0\\
314	0\\
315	0\\
316	0\\
317	0\\
318	0\\
319	0\\
320	0\\
321	0\\
322	0\\
323	0\\
324	0\\
325	0\\
326	0\\
327	0\\
328	0\\
329	0\\
330	0\\
331	0\\
332	0\\
333	0\\
334	0\\
335	0\\
336	0\\
337	0\\
338	0\\
339	0\\
340	0\\
341	0\\
342	0\\
343	0\\
344	0\\
345	0\\
346	0\\
347	0\\
348	0\\
349	0\\
350	0\\
351	0\\
352	0\\
353	0\\
354	0\\
355	0\\
356	0\\
357	0\\
358	0\\
359	0\\
360	0\\
361	0\\
362	0\\
363	0\\
364	0\\
365	0\\
366	0\\
367	0\\
368	0\\
369	0\\
370	0\\
371	0\\
372	0\\
373	0\\
374	0\\
375	0\\
376	0\\
377	0\\
378	0\\
379	0\\
380	0\\
381	0\\
382	0\\
383	0\\
384	0\\
385	0\\
386	0\\
387	0\\
388	0\\
389	0\\
390	0\\
391	0\\
392	0\\
393	0\\
394	0\\
395	0\\
396	0\\
397	0\\
398	0\\
399	0\\
400	0\\
401	0\\
402	0\\
403	0\\
404	0\\
405	0\\
406	0\\
407	0\\
408	0\\
409	0\\
410	0\\
411	0\\
412	0\\
413	0\\
414	0\\
415	0\\
416	0\\
417	0\\
418	0\\
419	0\\
420	0\\
421	0\\
422	0\\
423	0\\
424	0\\
425	0\\
426	0\\
427	0\\
428	0\\
429	0\\
430	0\\
431	0\\
432	0\\
433	0\\
434	0\\
435	0\\
436	0\\
437	0\\
438	0\\
439	0\\
440	0\\
441	0\\
442	0\\
443	0\\
444	0\\
445	0\\
446	0\\
447	0\\
448	0\\
449	0\\
450	0\\
451	0\\
452	0\\
453	0\\
454	0\\
455	0\\
456	0\\
457	0\\
458	0\\
459	0\\
460	0\\
461	0\\
462	0\\
463	0\\
464	0\\
465	0\\
466	0\\
467	0\\
468	0\\
469	0\\
470	0\\
471	0\\
472	0\\
473	0\\
474	0\\
475	0\\
476	0\\
477	0\\
478	0\\
479	0\\
480	0\\
481	0\\
482	0\\
483	0\\
484	0\\
485	0\\
486	0\\
487	0\\
488	0\\
489	0\\
490	0\\
491	0\\
492	0\\
493	0\\
494	0\\
495	0\\
496	0\\
497	0\\
498	0\\
499	0\\
500	0\\
501	0\\
502	0\\
503	0\\
504	0\\
505	0\\
506	0\\
507	0\\
508	0\\
509	0\\
510	0\\
511	0\\
512	0\\
513	0\\
514	0\\
515	0\\
516	0\\
517	0\\
518	0\\
519	0\\
520	0\\
521	0\\
522	0\\
523	0\\
524	0\\
525	0\\
526	0\\
527	0\\
528	0\\
529	0\\
530	0\\
531	0\\
532	0\\
533	0\\
534	0\\
535	0\\
536	0\\
537	0\\
538	0\\
539	0\\
540	0\\
541	0\\
542	0\\
543	0\\
544	0\\
545	0\\
546	0\\
547	0\\
548	0\\
549	0\\
550	0\\
551	0\\
552	0\\
553	0\\
554	0\\
555	0\\
556	0\\
557	0\\
558	0\\
559	0\\
560	0\\
561	0\\
562	0\\
563	0\\
564	0\\
565	0\\
566	0\\
567	0\\
568	0\\
569	0\\
570	0\\
571	0\\
572	0\\
573	3.91930466837155e-05\\
574	0.000276356766258129\\
575	0.000514435338481855\\
576	0.00063669964778276\\
577	0.000758573950088116\\
578	0.000879483741525221\\
579	0.000998840951933042\\
580	0.00111633816343543\\
581	0.00122629809196345\\
582	0.00132652874047018\\
583	0.00142603296361657\\
584	0.00152331143743122\\
585	0.00161638870950575\\
586	0.00170662226027931\\
587	0.00208852306255227\\
588	0.00257153432767004\\
589	0.00304912288981168\\
590	0.0032031038023609\\
591	0.00335726982309828\\
592	0.0035053300163716\\
593	0.00362658088892234\\
594	0.00376388624932705\\
595	0.00394575354532189\\
596	0.00424941998802836\\
597	0.00487316246495455\\
598	0.00633625614039688\\
599	0\\
600	0\\
};
\addplot [color=mycolor11,solid,forget plot]
  table[row sep=crcr]{%
1	0\\
2	0\\
3	0\\
4	0\\
5	0\\
6	0\\
7	0\\
8	0\\
9	0\\
10	0\\
11	0\\
12	0\\
13	0\\
14	0\\
15	0\\
16	0\\
17	0\\
18	0\\
19	0\\
20	0\\
21	0\\
22	0\\
23	0\\
24	0\\
25	0\\
26	0\\
27	0\\
28	0\\
29	0\\
30	0\\
31	0\\
32	0\\
33	0\\
34	0\\
35	0\\
36	0\\
37	0\\
38	0\\
39	0\\
40	0\\
41	0\\
42	0\\
43	0\\
44	0\\
45	0\\
46	0\\
47	0\\
48	0\\
49	0\\
50	0\\
51	0\\
52	0\\
53	0\\
54	0\\
55	0\\
56	0\\
57	0\\
58	0\\
59	0\\
60	0\\
61	0\\
62	0\\
63	0\\
64	0\\
65	0\\
66	0\\
67	0\\
68	0\\
69	0\\
70	0\\
71	0\\
72	0\\
73	0\\
74	0\\
75	0\\
76	0\\
77	0\\
78	0\\
79	0\\
80	0\\
81	0\\
82	0\\
83	0\\
84	0\\
85	0\\
86	0\\
87	0\\
88	0\\
89	0\\
90	0\\
91	0\\
92	0\\
93	0\\
94	0\\
95	0\\
96	0\\
97	0\\
98	0\\
99	0\\
100	0\\
101	0\\
102	0\\
103	0\\
104	0\\
105	0\\
106	0\\
107	0\\
108	0\\
109	0\\
110	0\\
111	0\\
112	0\\
113	0\\
114	0\\
115	0\\
116	0\\
117	0\\
118	0\\
119	0\\
120	0\\
121	0\\
122	0\\
123	0\\
124	0\\
125	0\\
126	0\\
127	0\\
128	0\\
129	0\\
130	0\\
131	0\\
132	0\\
133	0\\
134	0\\
135	0\\
136	0\\
137	0\\
138	0\\
139	0\\
140	0\\
141	0\\
142	0\\
143	0\\
144	0\\
145	0\\
146	0\\
147	0\\
148	0\\
149	0\\
150	0\\
151	0\\
152	0\\
153	0\\
154	0\\
155	0\\
156	0\\
157	0\\
158	0\\
159	0\\
160	0\\
161	0\\
162	0\\
163	0\\
164	0\\
165	0\\
166	0\\
167	0\\
168	0\\
169	0\\
170	0\\
171	0\\
172	0\\
173	0\\
174	0\\
175	0\\
176	0\\
177	0\\
178	0\\
179	0\\
180	0\\
181	0\\
182	0\\
183	0\\
184	0\\
185	0\\
186	0\\
187	0\\
188	0\\
189	0\\
190	0\\
191	0\\
192	0\\
193	0\\
194	0\\
195	0\\
196	0\\
197	0\\
198	0\\
199	0\\
200	0\\
201	0\\
202	0\\
203	0\\
204	0\\
205	0\\
206	0\\
207	0\\
208	0\\
209	0\\
210	0\\
211	0\\
212	0\\
213	0\\
214	0\\
215	0\\
216	0\\
217	0\\
218	0\\
219	0\\
220	0\\
221	0\\
222	0\\
223	0\\
224	0\\
225	0\\
226	0\\
227	0\\
228	0\\
229	0\\
230	0\\
231	0\\
232	0\\
233	0\\
234	0\\
235	0\\
236	0\\
237	0\\
238	0\\
239	0\\
240	0\\
241	0\\
242	0\\
243	0\\
244	0\\
245	0\\
246	0\\
247	0\\
248	0\\
249	0\\
250	0\\
251	0\\
252	0\\
253	0\\
254	0\\
255	0\\
256	0\\
257	0\\
258	0\\
259	0\\
260	0\\
261	0\\
262	0\\
263	0\\
264	0\\
265	0\\
266	0\\
267	0\\
268	0\\
269	0\\
270	0\\
271	0\\
272	0\\
273	0\\
274	0\\
275	0\\
276	0\\
277	0\\
278	0\\
279	0\\
280	0\\
281	0\\
282	0\\
283	0\\
284	0\\
285	0\\
286	0\\
287	0\\
288	0\\
289	0\\
290	0\\
291	0\\
292	0\\
293	0\\
294	0\\
295	0\\
296	0\\
297	0\\
298	0\\
299	0\\
300	0\\
301	0\\
302	0\\
303	0\\
304	0\\
305	0\\
306	0\\
307	0\\
308	0\\
309	0\\
310	0\\
311	0\\
312	0\\
313	0\\
314	0\\
315	0\\
316	0\\
317	0\\
318	0\\
319	0\\
320	0\\
321	0\\
322	0\\
323	0\\
324	0\\
325	0\\
326	0\\
327	0\\
328	0\\
329	0\\
330	0\\
331	0\\
332	0\\
333	0\\
334	0\\
335	0\\
336	0\\
337	0\\
338	0\\
339	0\\
340	0\\
341	0\\
342	0\\
343	0\\
344	0\\
345	0\\
346	0\\
347	0\\
348	0\\
349	0\\
350	0\\
351	0\\
352	0\\
353	0\\
354	0\\
355	0\\
356	0\\
357	0\\
358	0\\
359	0\\
360	0\\
361	0\\
362	0\\
363	0\\
364	0\\
365	0\\
366	0\\
367	0\\
368	0\\
369	0\\
370	0\\
371	0\\
372	0\\
373	0\\
374	0\\
375	0\\
376	0\\
377	0\\
378	0\\
379	0\\
380	0\\
381	0\\
382	0\\
383	0\\
384	0\\
385	0\\
386	0\\
387	0\\
388	0\\
389	0\\
390	0\\
391	0\\
392	0\\
393	0\\
394	0\\
395	0\\
396	0\\
397	0\\
398	0\\
399	0\\
400	0\\
401	0\\
402	0\\
403	0\\
404	0\\
405	0\\
406	0\\
407	0\\
408	0\\
409	0\\
410	0\\
411	0\\
412	0\\
413	0\\
414	0\\
415	0\\
416	0\\
417	0\\
418	0\\
419	0\\
420	0\\
421	0\\
422	0\\
423	0\\
424	0\\
425	0\\
426	0\\
427	0\\
428	0\\
429	0\\
430	0\\
431	0\\
432	0\\
433	0\\
434	0\\
435	0\\
436	0\\
437	0\\
438	0\\
439	0\\
440	0\\
441	0\\
442	0\\
443	0\\
444	0\\
445	0\\
446	0\\
447	0\\
448	0\\
449	0\\
450	0\\
451	0\\
452	0\\
453	0\\
454	0\\
455	0\\
456	0\\
457	0\\
458	0\\
459	0\\
460	0\\
461	0\\
462	0\\
463	0\\
464	0\\
465	0\\
466	0\\
467	0\\
468	0\\
469	0\\
470	0\\
471	0\\
472	0\\
473	0\\
474	0\\
475	0\\
476	0\\
477	0\\
478	0\\
479	0\\
480	0\\
481	0\\
482	0\\
483	0\\
484	0\\
485	0\\
486	0\\
487	0\\
488	0\\
489	0\\
490	0\\
491	0\\
492	0\\
493	0\\
494	0\\
495	0\\
496	0\\
497	0\\
498	0\\
499	0\\
500	0\\
501	0\\
502	0\\
503	0\\
504	0\\
505	0\\
506	0\\
507	0\\
508	0\\
509	0\\
510	0\\
511	0\\
512	0\\
513	0\\
514	0\\
515	0\\
516	0\\
517	0\\
518	0\\
519	0\\
520	0\\
521	0\\
522	0\\
523	0\\
524	0\\
525	0\\
526	0\\
527	0\\
528	0\\
529	0\\
530	0\\
531	0\\
532	0\\
533	0\\
534	0\\
535	0\\
536	0\\
537	0\\
538	0\\
539	0\\
540	0\\
541	0\\
542	0\\
543	0\\
544	0\\
545	0\\
546	0\\
547	0\\
548	0\\
549	0\\
550	0\\
551	0\\
552	0\\
553	0\\
554	0\\
555	0\\
556	0\\
557	0\\
558	0\\
559	0\\
560	0\\
561	0\\
562	0\\
563	0\\
564	0\\
565	0\\
566	0\\
567	0\\
568	0\\
569	0\\
570	0\\
571	0\\
572	0\\
573	0\\
574	0\\
575	0\\
576	0\\
577	0\\
578	0\\
579	3.6352204945255e-05\\
580	0.000296758473508131\\
581	0.000566134643332366\\
582	0.000744728105586424\\
583	0.000891594524082745\\
584	0.0010379859245772\\
585	0.00118567476607632\\
586	0.00133151060223953\\
587	0.00147653786463307\\
588	0.00161254310062852\\
589	0.00175780084121125\\
590	0.00225707969882223\\
591	0.00276065377804058\\
592	0.00326067925967635\\
593	0.00345740326951338\\
594	0.00366867553292786\\
595	0.00391598560027283\\
596	0.00423940434301512\\
597	0.00487316246495455\\
598	0.00633625614039688\\
599	0\\
600	0\\
};
\addplot [color=mycolor12,solid,forget plot]
  table[row sep=crcr]{%
1	0\\
2	0\\
3	0\\
4	0\\
5	0\\
6	0\\
7	0\\
8	0\\
9	0\\
10	0\\
11	0\\
12	0\\
13	0\\
14	0\\
15	0\\
16	0\\
17	0\\
18	0\\
19	0\\
20	0\\
21	0\\
22	0\\
23	0\\
24	0\\
25	0\\
26	0\\
27	0\\
28	0\\
29	0\\
30	0\\
31	0\\
32	0\\
33	0\\
34	0\\
35	0\\
36	0\\
37	0\\
38	0\\
39	0\\
40	0\\
41	0\\
42	0\\
43	0\\
44	0\\
45	0\\
46	0\\
47	0\\
48	0\\
49	0\\
50	0\\
51	0\\
52	0\\
53	0\\
54	0\\
55	0\\
56	0\\
57	0\\
58	0\\
59	0\\
60	0\\
61	0\\
62	0\\
63	0\\
64	0\\
65	0\\
66	0\\
67	0\\
68	0\\
69	0\\
70	0\\
71	0\\
72	0\\
73	0\\
74	0\\
75	0\\
76	0\\
77	0\\
78	0\\
79	0\\
80	0\\
81	0\\
82	0\\
83	0\\
84	0\\
85	0\\
86	0\\
87	0\\
88	0\\
89	0\\
90	0\\
91	0\\
92	0\\
93	0\\
94	0\\
95	0\\
96	0\\
97	0\\
98	0\\
99	0\\
100	0\\
101	0\\
102	0\\
103	0\\
104	0\\
105	0\\
106	0\\
107	0\\
108	0\\
109	0\\
110	0\\
111	0\\
112	0\\
113	0\\
114	0\\
115	0\\
116	0\\
117	0\\
118	0\\
119	0\\
120	0\\
121	0\\
122	0\\
123	0\\
124	0\\
125	0\\
126	0\\
127	0\\
128	0\\
129	0\\
130	0\\
131	0\\
132	0\\
133	0\\
134	0\\
135	0\\
136	0\\
137	0\\
138	0\\
139	0\\
140	0\\
141	0\\
142	0\\
143	0\\
144	0\\
145	0\\
146	0\\
147	0\\
148	0\\
149	0\\
150	0\\
151	0\\
152	0\\
153	0\\
154	0\\
155	0\\
156	0\\
157	0\\
158	0\\
159	0\\
160	0\\
161	0\\
162	0\\
163	0\\
164	0\\
165	0\\
166	0\\
167	0\\
168	0\\
169	0\\
170	0\\
171	0\\
172	0\\
173	0\\
174	0\\
175	0\\
176	0\\
177	0\\
178	0\\
179	0\\
180	0\\
181	0\\
182	0\\
183	0\\
184	0\\
185	0\\
186	0\\
187	0\\
188	0\\
189	0\\
190	0\\
191	0\\
192	0\\
193	0\\
194	0\\
195	0\\
196	0\\
197	0\\
198	0\\
199	0\\
200	0\\
201	0\\
202	0\\
203	0\\
204	0\\
205	0\\
206	0\\
207	0\\
208	0\\
209	0\\
210	0\\
211	0\\
212	0\\
213	0\\
214	0\\
215	0\\
216	0\\
217	0\\
218	0\\
219	0\\
220	0\\
221	0\\
222	0\\
223	0\\
224	0\\
225	0\\
226	0\\
227	0\\
228	0\\
229	0\\
230	0\\
231	0\\
232	0\\
233	0\\
234	0\\
235	0\\
236	0\\
237	0\\
238	0\\
239	0\\
240	0\\
241	0\\
242	0\\
243	0\\
244	0\\
245	0\\
246	0\\
247	0\\
248	0\\
249	0\\
250	0\\
251	0\\
252	0\\
253	0\\
254	0\\
255	0\\
256	0\\
257	0\\
258	0\\
259	0\\
260	0\\
261	0\\
262	0\\
263	0\\
264	0\\
265	0\\
266	0\\
267	0\\
268	0\\
269	0\\
270	0\\
271	0\\
272	0\\
273	0\\
274	0\\
275	0\\
276	0\\
277	0\\
278	0\\
279	0\\
280	0\\
281	0\\
282	0\\
283	0\\
284	0\\
285	0\\
286	0\\
287	0\\
288	0\\
289	0\\
290	0\\
291	0\\
292	0\\
293	0\\
294	0\\
295	0\\
296	0\\
297	0\\
298	0\\
299	0\\
300	0\\
301	0\\
302	0\\
303	0\\
304	0\\
305	0\\
306	0\\
307	0\\
308	0\\
309	0\\
310	0\\
311	0\\
312	0\\
313	0\\
314	0\\
315	0\\
316	0\\
317	0\\
318	0\\
319	0\\
320	0\\
321	0\\
322	0\\
323	0\\
324	0\\
325	0\\
326	0\\
327	0\\
328	0\\
329	0\\
330	0\\
331	0\\
332	0\\
333	0\\
334	0\\
335	0\\
336	0\\
337	0\\
338	0\\
339	0\\
340	0\\
341	0\\
342	0\\
343	0\\
344	0\\
345	0\\
346	0\\
347	0\\
348	0\\
349	0\\
350	0\\
351	0\\
352	0\\
353	0\\
354	0\\
355	0\\
356	0\\
357	0\\
358	0\\
359	0\\
360	0\\
361	0\\
362	0\\
363	0\\
364	0\\
365	0\\
366	0\\
367	0\\
368	0\\
369	0\\
370	0\\
371	0\\
372	0\\
373	0\\
374	0\\
375	0\\
376	0\\
377	0\\
378	0\\
379	0\\
380	0\\
381	0\\
382	0\\
383	0\\
384	0\\
385	0\\
386	0\\
387	0\\
388	0\\
389	0\\
390	0\\
391	0\\
392	0\\
393	0\\
394	0\\
395	0\\
396	0\\
397	0\\
398	0\\
399	0\\
400	0\\
401	0\\
402	0\\
403	0\\
404	0\\
405	0\\
406	0\\
407	0\\
408	0\\
409	0\\
410	0\\
411	0\\
412	0\\
413	0\\
414	0\\
415	0\\
416	0\\
417	0\\
418	0\\
419	0\\
420	0\\
421	0\\
422	0\\
423	0\\
424	0\\
425	0\\
426	0\\
427	0\\
428	0\\
429	0\\
430	0\\
431	0\\
432	0\\
433	0\\
434	0\\
435	0\\
436	0\\
437	0\\
438	0\\
439	0\\
440	0\\
441	0\\
442	0\\
443	0\\
444	0\\
445	0\\
446	0\\
447	0\\
448	0\\
449	0\\
450	0\\
451	0\\
452	0\\
453	0\\
454	0\\
455	0\\
456	0\\
457	0\\
458	0\\
459	0\\
460	0\\
461	0\\
462	0\\
463	0\\
464	0\\
465	0\\
466	0\\
467	0\\
468	0\\
469	0\\
470	0\\
471	0\\
472	0\\
473	0\\
474	0\\
475	0\\
476	0\\
477	0\\
478	0\\
479	0\\
480	0\\
481	0\\
482	0\\
483	0\\
484	0\\
485	0\\
486	0\\
487	0\\
488	0\\
489	0\\
490	0\\
491	0\\
492	0\\
493	0\\
494	0\\
495	0\\
496	0\\
497	0\\
498	0\\
499	0\\
500	0\\
501	0\\
502	0\\
503	0\\
504	0\\
505	0\\
506	0\\
507	0\\
508	0\\
509	0\\
510	0\\
511	0\\
512	0\\
513	0\\
514	0\\
515	0\\
516	0\\
517	0\\
518	0\\
519	0\\
520	0\\
521	0\\
522	0\\
523	0\\
524	0\\
525	0\\
526	0\\
527	0\\
528	0\\
529	0\\
530	0\\
531	0\\
532	0\\
533	0\\
534	0\\
535	0\\
536	0\\
537	0\\
538	0\\
539	0\\
540	0\\
541	0\\
542	0\\
543	0\\
544	0\\
545	0\\
546	0\\
547	0\\
548	0\\
549	0\\
550	0\\
551	0\\
552	0\\
553	0\\
554	0\\
555	0\\
556	0\\
557	0\\
558	0\\
559	0\\
560	0\\
561	0\\
562	0\\
563	0\\
564	0\\
565	0\\
566	0\\
567	0\\
568	0\\
569	0\\
570	0\\
571	0\\
572	0\\
573	0\\
574	0\\
575	0\\
576	0\\
577	0\\
578	0\\
579	0\\
580	0\\
581	0\\
582	0\\
583	0\\
584	0\\
585	0\\
586	8.12200422505001e-05\\
587	0.000377265340553168\\
588	0.000688475577631225\\
589	0.00088764595963235\\
590	0.00108832691824831\\
591	0.00129672894429491\\
592	0.00153220976797035\\
593	0.00212558857692427\\
594	0.00276084955258031\\
595	0.00348703611492466\\
596	0.00396130651742771\\
597	0.00473923450745123\\
598	0.00633625614039688\\
599	0\\
600	0\\
};
\addplot [color=mycolor13,solid,forget plot]
  table[row sep=crcr]{%
1	0\\
2	0\\
3	0\\
4	0\\
5	0\\
6	0\\
7	0\\
8	0\\
9	0\\
10	0\\
11	0\\
12	0\\
13	0\\
14	0\\
15	0\\
16	0\\
17	0\\
18	0\\
19	0\\
20	0\\
21	0\\
22	0\\
23	0\\
24	0\\
25	0\\
26	0\\
27	0\\
28	0\\
29	0\\
30	0\\
31	0\\
32	0\\
33	0\\
34	0\\
35	0\\
36	0\\
37	0\\
38	0\\
39	0\\
40	0\\
41	0\\
42	0\\
43	0\\
44	0\\
45	0\\
46	0\\
47	0\\
48	0\\
49	0\\
50	0\\
51	0\\
52	0\\
53	0\\
54	0\\
55	0\\
56	0\\
57	0\\
58	0\\
59	0\\
60	0\\
61	0\\
62	0\\
63	0\\
64	0\\
65	0\\
66	0\\
67	0\\
68	0\\
69	0\\
70	0\\
71	0\\
72	0\\
73	0\\
74	0\\
75	0\\
76	0\\
77	0\\
78	0\\
79	0\\
80	0\\
81	0\\
82	0\\
83	0\\
84	0\\
85	0\\
86	0\\
87	0\\
88	0\\
89	0\\
90	0\\
91	0\\
92	0\\
93	0\\
94	0\\
95	0\\
96	0\\
97	0\\
98	0\\
99	0\\
100	0\\
101	0\\
102	0\\
103	0\\
104	0\\
105	0\\
106	0\\
107	0\\
108	0\\
109	0\\
110	0\\
111	0\\
112	0\\
113	0\\
114	0\\
115	0\\
116	0\\
117	0\\
118	0\\
119	0\\
120	0\\
121	0\\
122	0\\
123	0\\
124	0\\
125	0\\
126	0\\
127	0\\
128	0\\
129	0\\
130	0\\
131	0\\
132	0\\
133	0\\
134	0\\
135	0\\
136	0\\
137	0\\
138	0\\
139	0\\
140	0\\
141	0\\
142	0\\
143	0\\
144	0\\
145	0\\
146	0\\
147	0\\
148	0\\
149	0\\
150	0\\
151	0\\
152	0\\
153	0\\
154	0\\
155	0\\
156	0\\
157	0\\
158	0\\
159	0\\
160	0\\
161	0\\
162	0\\
163	0\\
164	0\\
165	0\\
166	0\\
167	0\\
168	0\\
169	0\\
170	0\\
171	0\\
172	0\\
173	0\\
174	0\\
175	0\\
176	0\\
177	0\\
178	0\\
179	0\\
180	0\\
181	0\\
182	0\\
183	0\\
184	0\\
185	0\\
186	0\\
187	0\\
188	0\\
189	0\\
190	0\\
191	0\\
192	0\\
193	0\\
194	0\\
195	0\\
196	0\\
197	0\\
198	0\\
199	0\\
200	0\\
201	0\\
202	0\\
203	0\\
204	0\\
205	0\\
206	0\\
207	0\\
208	0\\
209	0\\
210	0\\
211	0\\
212	0\\
213	0\\
214	0\\
215	0\\
216	0\\
217	0\\
218	0\\
219	0\\
220	0\\
221	0\\
222	0\\
223	0\\
224	0\\
225	0\\
226	0\\
227	0\\
228	0\\
229	0\\
230	0\\
231	0\\
232	0\\
233	0\\
234	0\\
235	0\\
236	0\\
237	0\\
238	0\\
239	0\\
240	0\\
241	0\\
242	0\\
243	0\\
244	0\\
245	0\\
246	0\\
247	0\\
248	0\\
249	0\\
250	0\\
251	0\\
252	0\\
253	0\\
254	0\\
255	0\\
256	0\\
257	0\\
258	0\\
259	0\\
260	0\\
261	0\\
262	0\\
263	0\\
264	0\\
265	0\\
266	0\\
267	0\\
268	0\\
269	0\\
270	0\\
271	0\\
272	0\\
273	0\\
274	0\\
275	0\\
276	0\\
277	0\\
278	0\\
279	0\\
280	0\\
281	0\\
282	0\\
283	0\\
284	0\\
285	0\\
286	0\\
287	0\\
288	0\\
289	0\\
290	0\\
291	0\\
292	0\\
293	0\\
294	0\\
295	0\\
296	0\\
297	0\\
298	0\\
299	0\\
300	0\\
301	0\\
302	0\\
303	0\\
304	0\\
305	0\\
306	0\\
307	0\\
308	0\\
309	0\\
310	0\\
311	0\\
312	0\\
313	0\\
314	0\\
315	0\\
316	0\\
317	0\\
318	0\\
319	0\\
320	0\\
321	0\\
322	0\\
323	0\\
324	0\\
325	0\\
326	0\\
327	0\\
328	0\\
329	0\\
330	0\\
331	0\\
332	0\\
333	0\\
334	0\\
335	0\\
336	0\\
337	0\\
338	0\\
339	0\\
340	0\\
341	0\\
342	0\\
343	0\\
344	0\\
345	0\\
346	0\\
347	0\\
348	0\\
349	0\\
350	0\\
351	0\\
352	0\\
353	0\\
354	0\\
355	0\\
356	0\\
357	0\\
358	0\\
359	0\\
360	0\\
361	0\\
362	0\\
363	0\\
364	0\\
365	0\\
366	0\\
367	0\\
368	0\\
369	0\\
370	0\\
371	0\\
372	0\\
373	0\\
374	0\\
375	0\\
376	0\\
377	0\\
378	0\\
379	0\\
380	0\\
381	0\\
382	0\\
383	0\\
384	0\\
385	0\\
386	0\\
387	0\\
388	0\\
389	0\\
390	0\\
391	0\\
392	0\\
393	0\\
394	0\\
395	0\\
396	0\\
397	0\\
398	0\\
399	0\\
400	0\\
401	0\\
402	0\\
403	0\\
404	0\\
405	0\\
406	0\\
407	0\\
408	0\\
409	0\\
410	0\\
411	0\\
412	0\\
413	0\\
414	0\\
415	0\\
416	0\\
417	0\\
418	0\\
419	0\\
420	0\\
421	0\\
422	0\\
423	0\\
424	0\\
425	0\\
426	0\\
427	0\\
428	0\\
429	0\\
430	0\\
431	0\\
432	0\\
433	0\\
434	0\\
435	0\\
436	0\\
437	0\\
438	0\\
439	0\\
440	0\\
441	0\\
442	0\\
443	0\\
444	0\\
445	0\\
446	0\\
447	0\\
448	0\\
449	0\\
450	0\\
451	0\\
452	0\\
453	0\\
454	0\\
455	0\\
456	0\\
457	0\\
458	0\\
459	0\\
460	0\\
461	0\\
462	0\\
463	0\\
464	0\\
465	0\\
466	0\\
467	0\\
468	0\\
469	0\\
470	0\\
471	0\\
472	0\\
473	0\\
474	0\\
475	0\\
476	0\\
477	0\\
478	0\\
479	0\\
480	0\\
481	0\\
482	0\\
483	0\\
484	0\\
485	0\\
486	0\\
487	0\\
488	0\\
489	0\\
490	0\\
491	0\\
492	0\\
493	0\\
494	0\\
495	0\\
496	0\\
497	0\\
498	0\\
499	0\\
500	0\\
501	0\\
502	0\\
503	0\\
504	0\\
505	0\\
506	0\\
507	0\\
508	0\\
509	0\\
510	0\\
511	0\\
512	0\\
513	0\\
514	0\\
515	0\\
516	0\\
517	0\\
518	0\\
519	0\\
520	0\\
521	0\\
522	0\\
523	0\\
524	0\\
525	0\\
526	0\\
527	0\\
528	0\\
529	0\\
530	0\\
531	0\\
532	0\\
533	0\\
534	0\\
535	0\\
536	0\\
537	0\\
538	0\\
539	0\\
540	0\\
541	0\\
542	0\\
543	0\\
544	0\\
545	0\\
546	0\\
547	0\\
548	0\\
549	0\\
550	0\\
551	0\\
552	0\\
553	0\\
554	0\\
555	0\\
556	0\\
557	0\\
558	0\\
559	0\\
560	0\\
561	0\\
562	0\\
563	0\\
564	0\\
565	0\\
566	0\\
567	0\\
568	0\\
569	0\\
570	0\\
571	0\\
572	0\\
573	0\\
574	0\\
575	0\\
576	0\\
577	0\\
578	0\\
579	0\\
580	0\\
581	0\\
582	0\\
583	0\\
584	0\\
585	0\\
586	0\\
587	0\\
588	0\\
589	0\\
590	0\\
591	0\\
592	0\\
593	0\\
594	5.46220493361368e-05\\
595	0.000539858466370388\\
596	0.00161128070222677\\
597	0.00313428048457131\\
598	0.00582886603382478\\
599	0\\
600	0\\
};
\addplot [color=mycolor14,solid,forget plot]
  table[row sep=crcr]{%
1	0.00510935392454428\\
2	0.00510935392448638\\
3	0.00510935392442745\\
4	0.00510935392436747\\
5	0.00510935392430641\\
6	0.00510935392424427\\
7	0.00510935392418101\\
8	0.00510935392411662\\
9	0.00510935392405108\\
10	0.00510935392398437\\
11	0.00510935392391646\\
12	0.00510935392384734\\
13	0.00510935392377699\\
14	0.00510935392370538\\
15	0.00510935392363249\\
16	0.0051093539235583\\
17	0.00510935392348278\\
18	0.00510935392340591\\
19	0.00510935392332767\\
20	0.00510935392324802\\
21	0.00510935392316696\\
22	0.00510935392308445\\
23	0.00510935392300046\\
24	0.00510935392291497\\
25	0.00510935392282796\\
26	0.00510935392273939\\
27	0.00510935392264923\\
28	0.00510935392255747\\
29	0.00510935392246406\\
30	0.00510935392236899\\
31	0.00510935392227222\\
32	0.00510935392217372\\
33	0.00510935392207346\\
34	0.00510935392197141\\
35	0.00510935392186754\\
36	0.00510935392176181\\
37	0.00510935392165419\\
38	0.00510935392154465\\
39	0.00510935392143315\\
40	0.00510935392131966\\
41	0.00510935392120415\\
42	0.00510935392108657\\
43	0.00510935392096689\\
44	0.00510935392084508\\
45	0.00510935392072108\\
46	0.00510935392059488\\
47	0.00510935392046642\\
48	0.00510935392033566\\
49	0.00510935392020258\\
50	0.00510935392006711\\
51	0.00510935391992922\\
52	0.00510935391978888\\
53	0.00510935391964603\\
54	0.00510935391950062\\
55	0.00510935391935262\\
56	0.00510935391920198\\
57	0.00510935391904865\\
58	0.00510935391889258\\
59	0.00510935391873372\\
60	0.00510935391857203\\
61	0.00510935391840745\\
62	0.00510935391823994\\
63	0.00510935391806943\\
64	0.00510935391789588\\
65	0.00510935391771923\\
66	0.00510935391753943\\
67	0.00510935391735641\\
68	0.00510935391717013\\
69	0.00510935391698053\\
70	0.00510935391678754\\
71	0.00510935391659111\\
72	0.00510935391639117\\
73	0.00510935391618766\\
74	0.00510935391598053\\
75	0.00510935391576969\\
76	0.00510935391555509\\
77	0.00510935391533666\\
78	0.00510935391511433\\
79	0.00510935391488804\\
80	0.0051093539146577\\
81	0.00510935391442326\\
82	0.00510935391418463\\
83	0.00510935391394175\\
84	0.00510935391369453\\
85	0.0051093539134429\\
86	0.00510935391318678\\
87	0.00510935391292609\\
88	0.00510935391266074\\
89	0.00510935391239066\\
90	0.00510935391211577\\
91	0.00510935391183597\\
92	0.00510935391155117\\
93	0.0051093539112613\\
94	0.00510935391096625\\
95	0.00510935391066594\\
96	0.00510935391036027\\
97	0.00510935391004915\\
98	0.00510935390973248\\
99	0.00510935390941015\\
100	0.00510935390908208\\
101	0.00510935390874815\\
102	0.00510935390840827\\
103	0.00510935390806232\\
104	0.0051093539077102\\
105	0.0051093539073518\\
106	0.00510935390698701\\
107	0.00510935390661571\\
108	0.00510935390623778\\
109	0.00510935390585311\\
110	0.00510935390546158\\
111	0.00510935390506306\\
112	0.00510935390465744\\
113	0.00510935390424458\\
114	0.00510935390382436\\
115	0.00510935390339664\\
116	0.00510935390296128\\
117	0.00510935390251816\\
118	0.00510935390206714\\
119	0.00510935390160807\\
120	0.00510935390114081\\
121	0.00510935390066522\\
122	0.00510935390018114\\
123	0.00510935389968842\\
124	0.00510935389918692\\
125	0.00510935389867647\\
126	0.00510935389815691\\
127	0.00510935389762808\\
128	0.00510935389708982\\
129	0.00510935389654195\\
130	0.00510935389598431\\
131	0.00510935389541673\\
132	0.00510935389483901\\
133	0.00510935389425099\\
134	0.00510935389365248\\
135	0.00510935389304328\\
136	0.00510935389242322\\
137	0.00510935389179209\\
138	0.0051093538911497\\
139	0.00510935389049585\\
140	0.00510935388983034\\
141	0.00510935388915294\\
142	0.00510935388846346\\
143	0.00510935388776167\\
144	0.00510935388704736\\
145	0.0051093538863203\\
146	0.00510935388558027\\
147	0.00510935388482702\\
148	0.00510935388406033\\
149	0.00510935388327996\\
150	0.00510935388248565\\
151	0.00510935388167717\\
152	0.00510935388085426\\
153	0.00510935388001665\\
154	0.00510935387916409\\
155	0.0051093538782963\\
156	0.00510935387741303\\
157	0.00510935387651398\\
158	0.00510935387559888\\
159	0.00510935387466743\\
160	0.00510935387371936\\
161	0.00510935387275434\\
162	0.0051093538717721\\
163	0.0051093538707723\\
164	0.00510935386975465\\
165	0.00510935386871882\\
166	0.00510935386766448\\
167	0.00510935386659131\\
168	0.00510935386549896\\
169	0.00510935386438708\\
170	0.00510935386325535\\
171	0.00510935386210338\\
172	0.00510935386093082\\
173	0.0051093538597373\\
174	0.00510935385852245\\
175	0.00510935385728588\\
176	0.0051093538560272\\
177	0.005109353854746\\
178	0.0051093538534419\\
179	0.00510935385211448\\
180	0.00510935385076331\\
181	0.00510935384938797\\
182	0.00510935384798802\\
183	0.00510935384656303\\
184	0.00510935384511253\\
185	0.00510935384363608\\
186	0.0051093538421332\\
187	0.00510935384060341\\
188	0.00510935383904624\\
189	0.00510935383746119\\
190	0.00510935383584775\\
191	0.00510935383420542\\
192	0.00510935383253366\\
193	0.00510935383083196\\
194	0.00510935382909976\\
195	0.00510935382733653\\
196	0.00510935382554169\\
197	0.00510935382371468\\
198	0.00510935382185491\\
199	0.0051093538199618\\
200	0.00510935381803473\\
201	0.0051093538160731\\
202	0.00510935381407627\\
203	0.00510935381204362\\
204	0.00510935380997449\\
205	0.00510935380786822\\
206	0.00510935380572413\\
207	0.00510935380354154\\
208	0.00510935380131975\\
209	0.00510935379905806\\
210	0.00510935379675572\\
211	0.00510935379441202\\
212	0.00510935379202618\\
213	0.00510935378959746\\
214	0.00510935378712506\\
215	0.00510935378460819\\
216	0.00510935378204604\\
217	0.00510935377943779\\
218	0.00510935377678259\\
219	0.0051093537740796\\
220	0.00510935377132793\\
221	0.0051093537685267\\
222	0.00510935376567501\\
223	0.00510935376277193\\
224	0.00510935375981652\\
225	0.00510935375680782\\
226	0.00510935375374486\\
227	0.00510935375062666\\
228	0.00510935374745218\\
229	0.0051093537442204\\
230	0.00510935374093028\\
231	0.00510935373758074\\
232	0.00510935373417069\\
233	0.00510935373069901\\
234	0.00510935372716459\\
235	0.00510935372356625\\
236	0.00510935371990283\\
237	0.00510935371617313\\
238	0.00510935371237593\\
239	0.00510935370850997\\
240	0.00510935370457401\\
241	0.00510935370056673\\
242	0.00510935369648683\\
243	0.00510935369233297\\
244	0.00510935368810377\\
245	0.00510935368379785\\
246	0.00510935367941378\\
247	0.00510935367495012\\
248	0.00510935367040539\\
249	0.00510935366577809\\
250	0.00510935366106669\\
251	0.00510935365626963\\
252	0.0051093536513853\\
253	0.00510935364641211\\
254	0.00510935364134838\\
255	0.00510935363619244\\
256	0.00510935363094257\\
257	0.00510935362559702\\
258	0.005109353620154\\
259	0.00510935361461171\\
260	0.00510935360896827\\
261	0.00510935360322182\\
262	0.00510935359737041\\
263	0.0051093535914121\\
264	0.00510935358534488\\
265	0.0051093535791667\\
266	0.0051093535728755\\
267	0.00510935356646916\\
268	0.00510935355994551\\
269	0.00510935355330235\\
270	0.00510935354653745\\
271	0.00510935353964852\\
272	0.00510935353263322\\
273	0.00510935352548919\\
274	0.00510935351821399\\
275	0.00510935351080517\\
276	0.0051093535032602\\
277	0.00510935349557652\\
278	0.00510935348775153\\
279	0.00510935347978255\\
280	0.00510935347166687\\
281	0.00510935346340172\\
282	0.00510935345498429\\
283	0.00510935344641169\\
284	0.005109353437681\\
285	0.00510935342878924\\
286	0.00510935341973335\\
287	0.00510935341051024\\
288	0.00510935340111674\\
289	0.00510935339154964\\
290	0.00510935338180565\\
291	0.00510935337188143\\
292	0.00510935336177357\\
293	0.00510935335147859\\
294	0.00510935334099296\\
295	0.00510935333031307\\
296	0.00510935331943524\\
297	0.00510935330835574\\
298	0.00510935329707074\\
299	0.00510935328557636\\
300	0.00510935327386865\\
301	0.00510935326194356\\
302	0.005109353249797\\
303	0.00510935323742479\\
304	0.00510935322482265\\
305	0.00510935321198625\\
306	0.00510935319891117\\
307	0.0051093531855929\\
308	0.00510935317202687\\
309	0.00510935315820841\\
310	0.00510935314413276\\
311	0.00510935312979509\\
312	0.00510935311519047\\
313	0.00510935310031389\\
314	0.00510935308516024\\
315	0.00510935306972432\\
316	0.00510935305400085\\
317	0.00510935303798444\\
318	0.00510935302166961\\
319	0.0051093530050508\\
320	0.00510935298812232\\
321	0.0051093529708784\\
322	0.00510935295331317\\
323	0.00510935293542065\\
324	0.00510935291719477\\
325	0.00510935289862933\\
326	0.00510935287971805\\
327	0.00510935286045451\\
328	0.00510935284083221\\
329	0.00510935282084453\\
330	0.00510935280048472\\
331	0.00510935277974593\\
332	0.00510935275862119\\
333	0.00510935273710341\\
334	0.00510935271518538\\
335	0.00510935269285976\\
336	0.00510935267011908\\
337	0.00510935264695576\\
338	0.00510935262336208\\
339	0.00510935259933018\\
340	0.00510935257485208\\
341	0.00510935254991964\\
342	0.00510935252452461\\
343	0.00510935249865856\\
344	0.00510935247231295\\
345	0.00510935244547907\\
346	0.00510935241814805\\
347	0.00510935239031089\\
348	0.00510935236195841\\
349	0.00510935233308126\\
350	0.00510935230366995\\
351	0.00510935227371478\\
352	0.00510935224320592\\
353	0.0051093522121333\\
354	0.00510935218048672\\
355	0.00510935214825574\\
356	0.00510935211542975\\
357	0.00510935208199791\\
358	0.00510935204794918\\
359	0.00510935201327231\\
360	0.00510935197795578\\
361	0.00510935194198785\\
362	0.00510935190535655\\
363	0.00510935186804961\\
364	0.00510935183005451\\
365	0.00510935179135843\\
366	0.00510935175194826\\
367	0.00510935171181056\\
368	0.00510935167093157\\
369	0.00510935162929717\\
370	0.00510935158689288\\
371	0.00510935154370381\\
372	0.00510935149971467\\
373	0.00510935145490971\\
374	0.00510935140927275\\
375	0.00510935136278707\\
376	0.00510935131543545\\
377	0.00510935126720007\\
378	0.00510935121806255\\
379	0.00510935116800382\\
380	0.00510935111700412\\
381	0.00510935106504293\\
382	0.00510935101209893\\
383	0.00510935095814987\\
384	0.00510935090317253\\
385	0.00510935084714257\\
386	0.00510935079003445\\
387	0.00510935073182125\\
388	0.00510935067247457\\
389	0.0051093506119645\\
390	0.00510935055025962\\
391	0.00510935048732717\\
392	0.005109350423133\\
393	0.00510935035764106\\
394	0.00510935029081268\\
395	0.0051093502226066\\
396	0.00510935015297935\\
397	0.0051093500818852\\
398	0.00510935000927617\\
399	0.00510934993510209\\
400	0.00510934985931063\\
401	0.00510934978184713\\
402	0.00510934970265412\\
403	0.00510934962167004\\
404	0.00510934953882765\\
405	0.00510934945405288\\
406	0.00510934936726596\\
407	0.00510934927838459\\
408	0.0051093491873237\\
409	0.00510934909399121\\
410	0.00510934899828718\\
411	0.00510934890010274\\
412	0.0051093487993189\\
413	0.00510934869580513\\
414	0.00510934858941767\\
415	0.00510934847999748\\
416	0.00510934836736759\\
417	0.00510934825132947\\
418	0.00510934813165768\\
419	0.00510934800809096\\
420	0.00510934788031657\\
421	0.00510934774794105\\
422	0.00510934761043526\\
423	0.00510934746703316\\
424	0.00510934731655571\\
425	0.00510934715713394\\
426	0.00510934698585053\\
427	0.00510934679846615\\
428	0.00510934658971189\\
429	0.00510934635501702\\
430	0.00510934609428173\\
431	0.00510934581553934\\
432	0.00510934553045902\\
433	0.00510934523883727\\
434	0.00510934494045925\\
435	0.00510934463509702\\
436	0.00510934432250662\\
437	0.0051093440024216\\
438	0.00510934367453844\\
439	0.00510934333848197\\
440	0.00510934299372291\\
441	0.00510934263938256\\
442	0.00510934227377758\\
443	0.00510934189338404\\
444	0.00510934149055524\\
445	0.00510934104871201\\
446	0.0051093405328209\\
447	0.00510933987224999\\
448	0.00510933893456566\\
449	0.00510933749814763\\
450	0.00510933525909056\\
451	0.00510933195691012\\
452	0.0051093277027712\\
453	0.00510932322326447\\
454	0.00510931866061475\\
455	0.00510931401393708\\
456	0.00510930928223714\\
457	0.00510930446437545\\
458	0.00510929955904657\\
459	0.00510929456478831\\
460	0.00510928948001827\\
461	0.00510928430303812\\
462	0.00510927903189066\\
463	0.0051092736641144\\
464	0.00510926819697369\\
465	0.00510926262766088\\
466	0.00510925695330791\\
467	0.00510925117092717\\
468	0.00510924527735716\\
469	0.0051092392693063\\
470	0.00510923314323429\\
471	0.00510922689517286\\
472	0.00510922052064365\\
473	0.00510921401495943\\
474	0.00510920737376227\\
475	0.00510920059241034\\
476	0.00510919366595188\\
477	0.00510918658909689\\
478	0.00510917935618547\\
479	0.00510917196115216\\
480	0.0051091643974856\\
481	0.00510915665818281\\
482	0.00510914873569687\\
483	0.00510914062187698\\
484	0.00510913230789939\\
485	0.00510912378418719\\
486	0.00510911504031678\\
487	0.0051091060649076\\
488	0.00510909684549009\\
489	0.00510908736834342\\
490	0.00510907761828662\\
491	0.00510906757838935\\
492	0.00510905722953093\\
493	0.00510904654965426\\
494	0.00510903551238932\\
495	0.00510902408437497\\
496	0.00510901221997077\\
497	0.00510899985100423\\
498	0.00510898686788099\\
499	0.00510897308793375\\
500	0.00510895821124027\\
501	0.00510894178309003\\
502	0.00510892323212426\\
503	0.00510890213499336\\
504	0.00510887884367902\\
505	0.00510885487124891\\
506	0.00510883054776192\\
507	0.00510880582712782\\
508	0.00510878066473975\\
509	0.00510875503919166\\
510	0.00510872892463499\\
511	0.00510870228658319\\
512	0.00510867507159252\\
513	0.0051086471819434\\
514	0.00510861841406499\\
515	0.0051085883118057\\
516	0.00510855582784442\\
517	0.00510851857885204\\
518	0.00510847132449955\\
519	0.00510840323682491\\
520	0.00510829419554075\\
521	0.00510811366057947\\
522	0.00510783477615486\\
523	0.00510748286176653\\
524	0.00510712175014415\\
525	0.00510675072678062\\
526	0.00510636890987652\\
527	0.00510597524346\\
528	0.00510556861103784\\
529	0.00510514817682335\\
530	0.00510471377207103\\
531	0.00510426496847091\\
532	0.00510379755925117\\
533	0.0051033024711498\\
534	0.00510276429339391\\
535	0.00510216082069459\\
536	0.00510146360797495\\
537	0.00510065909327794\\
538	0.00509980153479282\\
539	0.00509893015267351\\
540	0.00509804393789524\\
541	0.00509714171435844\\
542	0.00509622203850178\\
543	0.00509528298072277\\
544	0.00509432166950426\\
545	0.00509333344427241\\
546	0.00509231047983481\\
547	0.0050912399784705\\
548	0.00509010318509986\\
549	0.00508887989661606\\
550	0.00508756425762667\\
551	0.00508615254028593\\
552	0.00508451716960274\\
553	0.00508228135524955\\
554	0.00507845919342308\\
555	0.00507114661782098\\
556	0.00505821458816492\\
557	0.00504408871369172\\
558	0.00502877945922603\\
559	0.00501144345438311\\
560	0.00499075507820406\\
561	0.0049661841610934\\
562	0.00494140742442814\\
563	0.00491629961178414\\
564	0.00489058312765606\\
565	0.00486379574942121\\
566	0.00483556723733559\\
567	0.00480693086998232\\
568	0.00477879006842645\\
569	0.004751115067223\\
570	0.00472391025364932\\
571	0.00469746530010506\\
572	0.00467293971960066\\
573	0.0046515084899547\\
574	0.00463167812481128\\
575	0.00461032365983135\\
576	0.0045839928512043\\
577	0.00454273192026105\\
578	0.00445789096708661\\
579	0.00424277237319832\\
580	0.00395007387994891\\
581	0.00361600322881859\\
582	0.00327003545696772\\
583	0.00290671034981918\\
584	0.002516381243237\\
585	0.00211762748756851\\
586	0.00170903265059097\\
587	0.00128770391395649\\
588	0.000846590459924926\\
589	0.000367058431095583\\
590	0\\
591	0\\
592	0\\
593	0\\
594	0\\
595	0\\
596	0\\
597	0\\
598	0\\
599	0\\
600	0\\
};
\addplot [color=mycolor15,solid,forget plot]
  table[row sep=crcr]{%
1	0.00405783128187049\\
2	0.00405783128068545\\
3	0.00405783127947922\\
4	0.00405783127825142\\
5	0.00405783127700167\\
6	0.00405783127572958\\
7	0.00405783127443475\\
8	0.00405783127311677\\
9	0.00405783127177523\\
10	0.0040578312704097\\
11	0.00405783126901977\\
12	0.00405783126760499\\
13	0.00405783126616492\\
14	0.00405783126469911\\
15	0.0040578312632071\\
16	0.00405783126168842\\
17	0.0040578312601426\\
18	0.00405783125856915\\
19	0.00405783125696759\\
20	0.00405783125533739\\
21	0.00405783125367807\\
22	0.00405783125198909\\
23	0.00405783125026993\\
24	0.00405783124852004\\
25	0.00405783124673889\\
26	0.00405783124492591\\
27	0.00405783124308054\\
28	0.00405783124120219\\
29	0.00405783123929028\\
30	0.00405783123734421\\
31	0.00405783123536337\\
32	0.00405783123334714\\
33	0.00405783123129489\\
34	0.00405783122920598\\
35	0.00405783122707974\\
36	0.00405783122491553\\
37	0.00405783122271265\\
38	0.00405783122047042\\
39	0.00405783121818814\\
40	0.00405783121586509\\
41	0.00405783121350055\\
42	0.00405783121109377\\
43	0.004057831208644\\
44	0.00405783120615048\\
45	0.00405783120361243\\
46	0.00405783120102904\\
47	0.00405783119839952\\
48	0.00405783119572304\\
49	0.00405783119299876\\
50	0.00405783119022583\\
51	0.00405783118740338\\
52	0.00405783118453052\\
53	0.00405783118160638\\
54	0.00405783117863001\\
55	0.0040578311756005\\
56	0.0040578311725169\\
57	0.00405783116937823\\
58	0.00405783116618353\\
59	0.00405783116293179\\
60	0.004057831159622\\
61	0.00405783115625311\\
62	0.00405783115282408\\
63	0.00405783114933383\\
64	0.00405783114578127\\
65	0.00405783114216529\\
66	0.00405783113848477\\
67	0.00405783113473854\\
68	0.00405783113092544\\
69	0.00405783112704428\\
70	0.00405783112309383\\
71	0.00405783111907288\\
72	0.00405783111498015\\
73	0.00405783111081437\\
74	0.00405783110657424\\
75	0.00405783110225844\\
76	0.0040578310978656\\
77	0.00405783109339437\\
78	0.00405783108884334\\
79	0.00405783108421109\\
80	0.00405783107949617\\
81	0.00405783107469711\\
82	0.0040578310698124\\
83	0.00405783106484053\\
84	0.00405783105977993\\
85	0.00405783105462903\\
86	0.00405783104938622\\
87	0.00405783104404985\\
88	0.00405783103861826\\
89	0.00405783103308975\\
90	0.0040578310274626\\
91	0.00405783102173504\\
92	0.00405783101590529\\
93	0.00405783100997152\\
94	0.00405783100393188\\
95	0.00405783099778448\\
96	0.0040578309915274\\
97	0.00405783098515868\\
98	0.00405783097867634\\
99	0.00405783097207834\\
100	0.00405783096536263\\
101	0.00405783095852711\\
102	0.00405783095156964\\
103	0.00405783094448805\\
104	0.00405783093728011\\
105	0.00405783092994359\\
106	0.00405783092247619\\
107	0.00405783091487556\\
108	0.00405783090713935\\
109	0.00405783089926512\\
110	0.00405783089125043\\
111	0.00405783088309275\\
112	0.00405783087478954\\
113	0.00405783086633822\\
114	0.00405783085773613\\
115	0.00405783084898058\\
116	0.00405783084006884\\
117	0.00405783083099812\\
118	0.00405783082176559\\
119	0.00405783081236836\\
120	0.00405783080280348\\
121	0.00405783079306798\\
122	0.00405783078315879\\
123	0.00405783077307284\\
124	0.00405783076280695\\
125	0.00405783075235793\\
126	0.00405783074172249\\
127	0.00405783073089732\\
128	0.00405783071987903\\
129	0.00405783070866416\\
130	0.00405783069724922\\
131	0.00405783068563062\\
132	0.00405783067380473\\
133	0.00405783066176786\\
134	0.00405783064951623\\
135	0.00405783063704601\\
136	0.00405783062435329\\
137	0.0040578306114341\\
138	0.00405783059828439\\
139	0.00405783058490005\\
140	0.00405783057127688\\
141	0.00405783055741062\\
142	0.00405783054329693\\
143	0.00405783052893137\\
144	0.00405783051430945\\
145	0.00405783049942658\\
146	0.00405783048427811\\
147	0.00405783046885928\\
148	0.00405783045316526\\
149	0.00405783043719112\\
150	0.00405783042093186\\
151	0.00405783040438238\\
152	0.00405783038753748\\
153	0.00405783037039187\\
154	0.00405783035294019\\
155	0.00405783033517694\\
156	0.00405783031709656\\
157	0.00405783029869336\\
158	0.00405783027996156\\
159	0.00405783026089528\\
160	0.00405783024148854\\
161	0.00405783022173522\\
162	0.00405783020162912\\
163	0.00405783018116392\\
164	0.00405783016033319\\
165	0.00405783013913036\\
166	0.00405783011754878\\
167	0.00405783009558164\\
168	0.00405783007322204\\
169	0.00405783005046293\\
170	0.00405783002729714\\
171	0.00405783000371738\\
172	0.00405782997971622\\
173	0.00405782995528609\\
174	0.00405782993041928\\
175	0.00405782990510796\\
176	0.00405782987934414\\
177	0.00405782985311969\\
178	0.00405782982642632\\
179	0.00405782979925562\\
180	0.00405782977159899\\
181	0.00405782974344771\\
182	0.00405782971479285\\
183	0.00405782968562538\\
184	0.00405782965593606\\
185	0.00405782962571548\\
186	0.0040578295949541\\
187	0.00405782956364216\\
188	0.00405782953176975\\
189	0.00405782949932677\\
190	0.00405782946630291\\
191	0.00405782943268772\\
192	0.00405782939847052\\
193	0.00405782936364045\\
194	0.00405782932818645\\
195	0.00405782929209725\\
196	0.00405782925536137\\
197	0.00405782921796714\\
198	0.00405782917990266\\
199	0.0040578291411558\\
200	0.00405782910171422\\
201	0.00405782906156536\\
202	0.00405782902069641\\
203	0.00405782897909433\\
204	0.00405782893674585\\
205	0.00405782889363745\\
206	0.00405782884975534\\
207	0.0040578288050855\\
208	0.00405782875961363\\
209	0.00405782871332519\\
210	0.00405782866620535\\
211	0.00405782861823901\\
212	0.00405782856941079\\
213	0.00405782851970501\\
214	0.00405782846910573\\
215	0.00405782841759669\\
216	0.00405782836516133\\
217	0.00405782831178278\\
218	0.00405782825744386\\
219	0.00405782820212706\\
220	0.00405782814581456\\
221	0.00405782808848819\\
222	0.00405782803012945\\
223	0.00405782797071948\\
224	0.00405782791023909\\
225	0.0040578278486687\\
226	0.00405782778598838\\
227	0.00405782772217783\\
228	0.00405782765721635\\
229	0.00405782759108288\\
230	0.00405782752375592\\
231	0.0040578274552136\\
232	0.00405782738543363\\
233	0.00405782731439329\\
234	0.00405782724206943\\
235	0.00405782716843847\\
236	0.00405782709347639\\
237	0.00405782701715869\\
238	0.00405782693946044\\
239	0.00405782686035619\\
240	0.00405782677982006\\
241	0.00405782669782565\\
242	0.00405782661434604\\
243	0.00405782652935384\\
244	0.0040578264428211\\
245	0.00405782635471936\\
246	0.00405782626501961\\
247	0.00405782617369228\\
248	0.00405782608070726\\
249	0.00405782598603382\\
250	0.00405782588964068\\
251	0.00405782579149595\\
252	0.00405782569156712\\
253	0.00405782558982108\\
254	0.00405782548622405\\
255	0.00405782538074163\\
256	0.00405782527333876\\
257	0.00405782516397968\\
258	0.00405782505262796\\
259	0.00405782493924648\\
260	0.00405782482379739\\
261	0.00405782470624211\\
262	0.00405782458654133\\
263	0.00405782446465498\\
264	0.00405782434054219\\
265	0.00405782421416135\\
266	0.00405782408547001\\
267	0.00405782395442492\\
268	0.00405782382098198\\
269	0.00405782368509626\\
270	0.00405782354672193\\
271	0.00405782340581233\\
272	0.00405782326231983\\
273	0.00405782311619594\\
274	0.0040578229673912\\
275	0.00405782281585521\\
276	0.0040578226615366\\
277	0.00405782250438299\\
278	0.00405782234434101\\
279	0.00405782218135626\\
280	0.00405782201537327\\
281	0.00405782184633554\\
282	0.00405782167418543\\
283	0.00405782149886425\\
284	0.00405782132031214\\
285	0.0040578211384681\\
286	0.00405782095326998\\
287	0.0040578207646544\\
288	0.0040578205725568\\
289	0.00405782037691137\\
290	0.00405782017765105\\
291	0.00405781997470748\\
292	0.00405781976801102\\
293	0.0040578195574907\\
294	0.0040578193430742\\
295	0.00405781912468782\\
296	0.00405781890225648\\
297	0.00405781867570366\\
298	0.00405781844495141\\
299	0.00405781820992032\\
300	0.00405781797052948\\
301	0.00405781772669645\\
302	0.00405781747833727\\
303	0.0040578172253664\\
304	0.00405781696769673\\
305	0.00405781670523949\\
306	0.00405781643790431\\
307	0.00405781616559914\\
308	0.00405781588823022\\
309	0.0040578156057021\\
310	0.00405781531791755\\
311	0.0040578150247776\\
312	0.00405781472618145\\
313	0.0040578144220265\\
314	0.00405781411220828\\
315	0.00405781379662045\\
316	0.00405781347515476\\
317	0.00405781314770103\\
318	0.00405781281414711\\
319	0.00405781247437886\\
320	0.00405781212828012\\
321	0.00405781177573269\\
322	0.00405781141661628\\
323	0.00405781105080849\\
324	0.0040578106781848\\
325	0.0040578102986185\\
326	0.00405780991198069\\
327	0.00405780951814023\\
328	0.00405780911696372\\
329	0.00405780870831544\\
330	0.00405780829205735\\
331	0.00405780786804904\\
332	0.00405780743614766\\
333	0.00405780699620794\\
334	0.0040578065480821\\
335	0.00405780609161985\\
336	0.0040578056266683\\
337	0.00405780515307194\\
338	0.00405780467067261\\
339	0.00405780417930944\\
340	0.00405780367881878\\
341	0.00405780316903416\\
342	0.00405780264978626\\
343	0.00405780212090283\\
344	0.00405780158220862\\
345	0.00405780103352538\\
346	0.0040578004746717\\
347	0.00405779990546304\\
348	0.00405779932571159\\
349	0.00405779873522624\\
350	0.00405779813381247\\
351	0.00405779752127228\\
352	0.00405779689740411\\
353	0.0040577962620027\\
354	0.00405779561485905\\
355	0.00405779495576026\\
356	0.00405779428448941\\
357	0.00405779360082548\\
358	0.00405779290454315\\
359	0.00405779219541268\\
360	0.00405779147319974\\
361	0.00405779073766525\\
362	0.00405778998856516\\
363	0.00405778922565027\\
364	0.004057788448666\\
365	0.00405778765735214\\
366	0.00405778685144261\\
367	0.00405778603066515\\
368	0.00405778519474103\\
369	0.00405778434338473\\
370	0.00405778347630352\\
371	0.00405778259319717\\
372	0.00405778169375742\\
373	0.00405778077766757\\
374	0.00405777984460199\\
375	0.00405777889422551\\
376	0.00405777792619289\\
377	0.00405777694014812\\
378	0.00405777593572373\\
379	0.00405777491253998\\
380	0.004057773870204\\
381	0.00405777280830881\\
382	0.00405777172643225\\
383	0.00405777062413573\\
384	0.00405776950096275\\
385	0.00405776835643724\\
386	0.00405776719006136\\
387	0.00405776600131297\\
388	0.00405776478964258\\
389	0.0040577635544702\\
390	0.00405776229518312\\
391	0.00405776101113627\\
392	0.00405775970165634\\
393	0.0040577583660461\\
394	0.00405775700357734\\
395	0.00405775561346582\\
396	0.00405775419486886\\
397	0.00405775274689764\\
398	0.00405775126861576\\
399	0.0040577497590391\\
400	0.00405774821713706\\
401	0.00405774664183517\\
402	0.00405774503201691\\
403	0.00405774338651914\\
404	0.00405774170411221\\
405	0.0040577399834581\\
406	0.004057738223062\\
407	0.00405773642127506\\
408	0.00405773457639635\\
409	0.00405773268671545\\
410	0.00405773075038478\\
411	0.00405772876540242\\
412	0.00405772672959232\\
413	0.00405772464058141\\
414	0.00405772249577291\\
415	0.00405772029231515\\
416	0.00405771802706464\\
417	0.00405771569654162\\
418	0.00405771329687435\\
419	0.00405771082372536\\
420	0.00405770827218407\\
421	0.00405770563659196\\
422	0.00405770291022529\\
423	0.00405770008467385\\
424	0.00405769714858433\\
425	0.00405769408513857\\
426	0.0040576908672278\\
427	0.00405768744906833\\
428	0.00405768375418857\\
429	0.00405767966545186\\
430	0.00405767503814818\\
431	0.00405766977979874\\
432	0.00405766401724685\\
433	0.00405765812442432\\
434	0.00405765209720578\\
435	0.00405764593124648\\
436	0.00405763962196297\\
437	0.0040576331645073\\
438	0.00405762655372528\\
439	0.00405761978407451\\
440	0.00405761284943778\\
441	0.00405760574266237\\
442	0.00405759845438355\\
443	0.00405759096999643\\
444	0.00405758326190385\\
445	0.00405757526995869\\
446	0.0040575668532547\\
447	0.00405755767532177\\
448	0.00405754694458593\\
449	0.00405753287461\\
450	0.00405751171979886\\
451	0.00405747658620918\\
452	0.00405741782337858\\
453	0.00405733094206556\\
454	0.00405723774177815\\
455	0.00405714280824585\\
456	0.00405704612435729\\
457	0.00405694767112828\\
458	0.00405684742678483\\
459	0.00405674536597795\\
460	0.00405664145963472\\
461	0.00405653567601179\\
462	0.00405642798236996\\
463	0.00405631834316966\\
464	0.00405620670991157\\
465	0.004056093026219\\
466	0.00405597723410947\\
467	0.00405585927502846\\
468	0.00405573908853998\\
469	0.0040556166100909\\
470	0.00405549177375176\\
471	0.00405536451037369\\
472	0.00405523474269021\\
473	0.00405510237930375\\
474	0.00405496731849525\\
475	0.00405482947240714\\
476	0.00405468874754487\\
477	0.00405454504425679\\
478	0.00405439825618186\\
479	0.0040542482696327\\
480	0.00405409496290346\\
481	0.00405393820548961\\
482	0.00405377785720437\\
483	0.0040536137671737\\
484	0.00405344577268737\\
485	0.00405327369787945\\
486	0.00405309735220562\\
487	0.00405291652867797\\
488	0.00405273100180775\\
489	0.0040525405251902\\
490	0.00405234482864563\\
491	0.00405214361476636\\
492	0.00405193655460589\\
493	0.00405172328194951\\
494	0.00405150338494586\\
495	0.00405127639226724\\
496	0.00405104174713352\\
497	0.00405079875400372\\
498	0.00405054646426666\\
499	0.00405028343048887\\
500	0.00405000719573569\\
501	0.00404971331113423\\
502	0.00404939372089459\\
503	0.00404903503027239\\
504	0.0040486199641645\\
505	0.00404814310693598\\
506	0.00404764866385833\\
507	0.00404714756740775\\
508	0.00404663883761316\\
509	0.00404612128916025\\
510	0.00404559453378168\\
511	0.00404505813822849\\
512	0.00404451160523361\\
513	0.00404395432618424\\
514	0.00404338545841346\\
515	0.0040428035917532\\
516	0.00404220582736261\\
517	0.00404158523602025\\
518	0.00404092393042789\\
519	0.00404017466478783\\
520	0.00403921428067007\\
521	0.00403773678695009\\
522	0.00403506010340366\\
523	0.00403005195149346\\
524	0.00402283760207031\\
525	0.00401545845225671\\
526	0.00400790316818706\\
527	0.00400015730478272\\
528	0.00399220196177449\\
529	0.00398401382847418\\
530	0.00397557198005718\\
531	0.00396687959477154\\
532	0.00395797017314288\\
533	0.00394881320521243\\
534	0.003939296200743\\
535	0.00392918736081158\\
536	0.00391811384245127\\
537	0.00390536367213929\\
538	0.00389006191836176\\
539	0.00387347099840221\\
540	0.00385674672730471\\
541	0.00383987972589676\\
542	0.00382285939851747\\
543	0.0038056733707417\\
544	0.00378830593945472\\
545	0.00377073386590955\\
546	0.00375291663947729\\
547	0.00373477595823748\\
548	0.00371615447430384\\
549	0.00369675261602705\\
550	0.00367616734655516\\
551	0.00365447847782785\\
552	0.00363268508514219\\
553	0.003610797550057\\
554	0.00358686376537867\\
555	0.00354932816061378\\
556	0.003458258037869\\
557	0.00335761249498769\\
558	0.00325023830819355\\
559	0.00313186109142807\\
560	0.00298967747584707\\
561	0.00279974352474002\\
562	0.00260239693723076\\
563	0.00239685833453858\\
564	0.00218168025772699\\
565	0.00195248625972547\\
566	0.00169649043125886\\
567	0.00141986044564115\\
568	0.00113508540913231\\
569	0.000841186746185829\\
570	0.000536281815051808\\
571	0.000215644234799086\\
572	0\\
573	0\\
574	0\\
575	0\\
576	0\\
577	0\\
578	0\\
579	0\\
580	0\\
581	0\\
582	0\\
583	0\\
584	0\\
585	0\\
586	0\\
587	0\\
588	0\\
589	0\\
590	0\\
591	0\\
592	0\\
593	0\\
594	0\\
595	0\\
596	0\\
597	0\\
598	0\\
599	0\\
600	0\\
};
\addplot [color=mycolor16,solid,forget plot]
  table[row sep=crcr]{%
1	0.00399857970101667\\
2	0.00399857967758191\\
3	0.00399857965372814\\
4	0.00399857962944786\\
5	0.00399857960473347\\
6	0.00399857957957721\\
7	0.0039985795539712\\
8	0.00399857952790741\\
9	0.00399857950137767\\
10	0.00399857947437365\\
11	0.00399857944688689\\
12	0.00399857941890878\\
13	0.00399857939043054\\
14	0.00399857936144325\\
15	0.00399857933193782\\
16	0.003998579301905\\
17	0.00399857927133538\\
18	0.00399857924021939\\
19	0.00399857920854726\\
20	0.00399857917630907\\
21	0.00399857914349473\\
22	0.00399857911009395\\
23	0.00399857907609627\\
24	0.00399857904149102\\
25	0.00399857900626738\\
26	0.00399857897041431\\
27	0.00399857893392057\\
28	0.00399857889677473\\
29	0.00399857885896517\\
30	0.00399857882048004\\
31	0.00399857878130728\\
32	0.00399857874143463\\
33	0.0039985787008496\\
34	0.00399857865953948\\
35	0.00399857861749134\\
36	0.00399857857469201\\
37	0.00399857853112809\\
38	0.00399857848678594\\
39	0.00399857844165167\\
40	0.00399857839571117\\
41	0.00399857834895004\\
42	0.00399857830135366\\
43	0.00399857825290712\\
44	0.00399857820359525\\
45	0.00399857815340263\\
46	0.00399857810231355\\
47	0.00399857805031201\\
48	0.00399857799738176\\
49	0.00399857794350621\\
50	0.00399857788866852\\
51	0.00399857783285152\\
52	0.00399857777603775\\
53	0.00399857771820944\\
54	0.00399857765934849\\
55	0.00399857759943649\\
56	0.00399857753845469\\
57	0.00399857747638403\\
58	0.00399857741320508\\
59	0.00399857734889808\\
60	0.00399857728344292\\
61	0.00399857721681913\\
62	0.00399857714900586\\
63	0.00399857707998192\\
64	0.00399857700972571\\
65	0.00399857693821527\\
66	0.00399857686542822\\
67	0.00399857679134181\\
68	0.00399857671593288\\
69	0.00399857663917784\\
70	0.0039985765610527\\
71	0.00399857648153303\\
72	0.00399857640059396\\
73	0.0039985763182102\\
74	0.00399857623435598\\
75	0.0039985761490051\\
76	0.00399857606213086\\
77	0.00399857597370611\\
78	0.00399857588370321\\
79	0.00399857579209403\\
80	0.00399857569884992\\
81	0.00399857560394175\\
82	0.00399857550733984\\
83	0.00399857540901401\\
84	0.00399857530893351\\
85	0.00399857520706708\\
86	0.00399857510338286\\
87	0.00399857499784846\\
88	0.00399857489043089\\
89	0.00399857478109657\\
90	0.00399857466981134\\
91	0.00399857455654041\\
92	0.00399857444124839\\
93	0.00399857432389925\\
94	0.0039985742044563\\
95	0.00399857408288223\\
96	0.00399857395913902\\
97	0.00399857383318803\\
98	0.00399857370498987\\
99	0.00399857357450448\\
100	0.0039985734416911\\
101	0.0039985733065082\\
102	0.00399857316891353\\
103	0.0039985730288641\\
104	0.00399857288631614\\
105	0.00399857274122509\\
106	0.00399857259354561\\
107	0.00399857244323154\\
108	0.00399857229023589\\
109	0.00399857213451086\\
110	0.00399857197600775\\
111	0.00399857181467703\\
112	0.00399857165046827\\
113	0.00399857148333015\\
114	0.0039985713132104\\
115	0.00399857114005584\\
116	0.00399857096381236\\
117	0.00399857078442485\\
118	0.00399857060183721\\
119	0.00399857041599237\\
120	0.00399857022683221\\
121	0.00399857003429759\\
122	0.0039985698383283\\
123	0.00399856963886306\\
124	0.00399856943583947\\
125	0.00399856922919405\\
126	0.00399856901886217\\
127	0.00399856880477803\\
128	0.00399856858687465\\
129	0.00399856836508388\\
130	0.00399856813933632\\
131	0.00399856790956134\\
132	0.00399856767568703\\
133	0.00399856743764022\\
134	0.0039985671953464\\
135	0.00399856694872972\\
136	0.00399856669771301\\
137	0.00399856644221767\\
138	0.00399856618216371\\
139	0.0039985659174697\\
140	0.00399856564805277\\
141	0.00399856537382852\\
142	0.00399856509471106\\
143	0.00399856481061297\\
144	0.00399856452144522\\
145	0.00399856422711723\\
146	0.00399856392753674\\
147	0.00399856362260986\\
148	0.00399856331224102\\
149	0.0039985629963329\\
150	0.00399856267478646\\
151	0.00399856234750085\\
152	0.00399856201437341\\
153	0.00399856167529966\\
154	0.00399856133017319\\
155	0.00399856097888571\\
156	0.00399856062132695\\
157	0.00399856025738468\\
158	0.00399855988694461\\
159	0.00399855950989042\\
160	0.00399855912610368\\
161	0.00399855873546381\\
162	0.00399855833784805\\
163	0.00399855793313144\\
164	0.00399855752118674\\
165	0.00399855710188444\\
166	0.00399855667509264\\
167	0.00399855624067709\\
168	0.0039985557985011\\
169	0.00399855534842549\\
170	0.00399855489030857\\
171	0.00399855442400608\\
172	0.00399855394937115\\
173	0.00399855346625422\\
174	0.00399855297450304\\
175	0.00399855247396258\\
176	0.00399855196447501\\
177	0.00399855144587961\\
178	0.00399855091801275\\
179	0.00399855038070782\\
180	0.00399854983379517\\
181	0.00399854927710207\\
182	0.00399854871045264\\
183	0.00399854813366778\\
184	0.00399854754656514\\
185	0.00399854694895904\\
186	0.00399854634066039\\
187	0.00399854572147668\\
188	0.00399854509121185\\
189	0.00399854444966626\\
190	0.00399854379663663\\
191	0.00399854313191594\\
192	0.00399854245529339\\
193	0.00399854176655431\\
194	0.00399854106548009\\
195	0.00399854035184811\\
196	0.00399853962543166\\
197	0.00399853888599988\\
198	0.00399853813331763\\
199	0.00399853736714547\\
200	0.00399853658723955\\
201	0.00399853579335153\\
202	0.00399853498522846\\
203	0.00399853416261277\\
204	0.00399853332524209\\
205	0.00399853247284926\\
206	0.00399853160516211\\
207	0.0039985307219035\\
208	0.00399852982279113\\
209	0.00399852890753747\\
210	0.00399852797584966\\
211	0.00399852702742942\\
212	0.00399852606197292\\
213	0.00399852507917069\\
214	0.00399852407870751\\
215	0.00399852306026229\\
216	0.00399852202350798\\
217	0.0039985209681114\\
218	0.0039985198937332\\
219	0.00399851880002766\\
220	0.00399851768664264\\
221	0.00399851655321938\\
222	0.00399851539939245\\
223	0.00399851422478955\\
224	0.00399851302903142\\
225	0.00399851181173169\\
226	0.00399851057249673\\
227	0.00399850931092554\\
228	0.00399850802660956\\
229	0.00399850671913257\\
230	0.00399850538807049\\
231	0.00399850403299129\\
232	0.00399850265345476\\
233	0.0039985012490124\\
234	0.00399849981920724\\
235	0.00399849836357368\\
236	0.00399849688163731\\
237	0.00399849537291475\\
238	0.00399849383691345\\
239	0.00399849227313153\\
240	0.00399849068105759\\
241	0.00399848906017053\\
242	0.00399848740993934\\
243	0.00399848572982292\\
244	0.00399848401926985\\
245	0.00399848227771825\\
246	0.00399848050459551\\
247	0.00399847869931808\\
248	0.00399847686129133\\
249	0.00399847498990921\\
250	0.00399847308455413\\
251	0.00399847114459668\\
252	0.0039984691693954\\
253	0.00399846715829655\\
254	0.00399846511063386\\
255	0.00399846302572829\\
256	0.00399846090288778\\
257	0.00399845874140696\\
258	0.00399845654056695\\
259	0.00399845429963502\\
260	0.00399845201786438\\
261	0.00399844969449387\\
262	0.00399844732874765\\
263	0.00399844491983499\\
264	0.00399844246694992\\
265	0.00399843996927091\\
266	0.00399843742596062\\
267	0.00399843483616557\\
268	0.00399843219901583\\
269	0.00399842951362465\\
270	0.00399842677908822\\
271	0.00399842399448528\\
272	0.00399842115887676\\
273	0.00399841827130551\\
274	0.00399841533079588\\
275	0.00399841233635343\\
276	0.00399840928696449\\
277	0.00399840618159587\\
278	0.00399840301919443\\
279	0.00399839979868675\\
280	0.00399839651897868\\
281	0.00399839317895503\\
282	0.00399838977747912\\
283	0.00399838631339237\\
284	0.00399838278551395\\
285	0.00399837919264029\\
286	0.00399837553354475\\
287	0.0039983718069771\\
288	0.00399836801166317\\
289	0.00399836414630436\\
290	0.00399836020957722\\
291	0.00399835620013303\\
292	0.0039983521165973\\
293	0.00399834795756934\\
294	0.0039983437216218\\
295	0.00399833940730019\\
296	0.00399833501312243\\
297	0.00399833053757836\\
298	0.00399832597912925\\
299	0.00399832133620734\\
300	0.00399831660721534\\
301	0.00399831179052593\\
302	0.00399830688448126\\
303	0.00399830188739248\\
304	0.00399829679753921\\
305	0.00399829161316902\\
306	0.00399828633249696\\
307	0.00399828095370499\\
308	0.00399827547494151\\
309	0.00399826989432083\\
310	0.00399826420992262\\
311	0.00399825841979139\\
312	0.00399825252193598\\
313	0.003998246514329\\
314	0.00399824039490631\\
315	0.00399823416156646\\
316	0.00399822781217013\\
317	0.00399822134453963\\
318	0.0039982147564583\\
319	0.00399820804566994\\
320	0.00399820120987828\\
321	0.00399819424674638\\
322	0.00399818715389604\\
323	0.00399817992890722\\
324	0.00399817256931745\\
325	0.0039981650726212\\
326	0.00399815743626926\\
327	0.00399814965766809\\
328	0.00399814173417922\\
329	0.00399813366311852\\
330	0.00399812544175554\\
331	0.0039981170673128\\
332	0.00399810853696509\\
333	0.00399809984783868\\
334	0.00399809099701057\\
335	0.00399808198150768\\
336	0.00399807279830603\\
337	0.00399806344432988\\
338	0.00399805391645086\\
339	0.00399804421148699\\
340	0.00399803432620183\\
341	0.00399802425730339\\
342	0.00399801400144318\\
343	0.00399800355521513\\
344	0.00399799291515446\\
345	0.00399798207773657\\
346	0.00399797103937582\\
347	0.00399795979642428\\
348	0.00399794834517043\\
349	0.00399793668183774\\
350	0.00399792480258323\\
351	0.00399791270349591\\
352	0.00399790038059512\\
353	0.00399788782982874\\
354	0.00399787504707133\\
355	0.00399786202812206\\
356	0.00399784876870255\\
357	0.00399783526445451\\
358	0.00399782151093725\\
359	0.00399780750362489\\
360	0.00399779323790351\\
361	0.00399777870906792\\
362	0.00399776391231822\\
363	0.00399774884275618\\
364	0.00399773349538113\\
365	0.00399771786508571\\
366	0.00399770194665113\\
367	0.00399768573474207\\
368	0.00399766922390119\\
369	0.00399765240854318\\
370	0.00399763528294823\\
371	0.00399761784125501\\
372	0.00399760007745305\\
373	0.00399758198537448\\
374	0.00399756355868495\\
375	0.00399754479087392\\
376	0.00399752567524393\\
377	0.00399750620489901\\
378	0.003997486372732\\
379	0.00399746617141064\\
380	0.00399744559336238\\
381	0.0039974246307576\\
382	0.00399740327549112\\
383	0.00399738151916151\\
384	0.00399735935304781\\
385	0.00399733676808274\\
386	0.00399731375482103\\
387	0.00399729030340091\\
388	0.00399726640349559\\
389	0.00399724204425232\\
390	0.00399721721422017\\
391	0.00399719190128107\\
392	0.00399716609262436\\
393	0.00399713977482936\\
394	0.00399711293405988\\
395	0.00399708555608252\\
396	0.00399705762554789\\
397	0.00399702912580888\\
398	0.00399700003929598\\
399	0.00399697034747901\\
400	0.00399694003084921\\
401	0.00399690906893527\\
402	0.00399687744036482\\
403	0.00399684512295967\\
404	0.00399681209378542\\
405	0.00399677832894323\\
406	0.00399674380277119\\
407	0.0039967084863855\\
408	0.00399667234683618\\
409	0.00399663534974364\\
410	0.00399659746172118\\
411	0.0039965586467608\\
412	0.00399651886591\\
413	0.00399647807689943\\
414	0.00399643623371287\\
415	0.00399639328608915\\
416	0.00399634917894311\\
417	0.00399630385168904\\
418	0.00399625723744344\\
419	0.00399620926207135\\
420	0.00399615984301195\\
421	0.00399610888775217\\
422	0.00399605629165382\\
423	0.00399600193444172\\
424	0.00399594567370153\\
425	0.00399588733149965\\
426	0.00399582666528058\\
427	0.00399576330413516\\
428	0.00399569661432779\\
429	0.00399562544011629\\
430	0.0039955476934609\\
431	0.003995460009989\\
432	0.00399535855802043\\
433	0.00399524364579588\\
434	0.00399512615294169\\
435	0.00399500599877776\\
436	0.00399488309840421\\
437	0.00399475736237948\\
438	0.00399462869635417\\
439	0.00399449700062248\\
440	0.00399436216948627\\
441	0.00399422409013686\\
442	0.00399408264022094\\
443	0.00399393768172687\\
444	0.0039937890444934\\
445	0.00399363648040612\\
446	0.00399347953505816\\
447	0.00399331718877714\\
448	0.00399314686202122\\
449	0.00399296171003658\\
450	0.00399274350038222\\
451	0.00399244494390756\\
452	0.00399195091084209\\
453	0.00399101810193024\\
454	0.00398931856600917\\
455	0.00398745069141274\\
456	0.00398554914584857\\
457	0.00398361367123451\\
458	0.00398164398276477\\
459	0.00397963974954667\\
460	0.0039776005732863\\
461	0.00397552597425255\\
462	0.00397341540561641\\
463	0.00397126831564775\\
464	0.00396908418410526\\
465	0.00396686219389411\\
466	0.00396460132303283\\
467	0.00396230051428018\\
468	0.00395995871721609\\
469	0.00395757486966028\\
470	0.00395514780957969\\
471	0.00395267637158739\\
472	0.00395015937667282\\
473	0.00394759554206912\\
474	0.00394498327324883\\
475	0.00394232057158767\\
476	0.00393960589832815\\
477	0.00393683761784221\\
478	0.00393401398855618\\
479	0.00393113315363734\\
480	0.00392819313062419\\
481	0.00392519179982424\\
482	0.00392212689127023\\
483	0.00391899596998522\\
484	0.00391579641925903\\
485	0.00391252542157723\\
486	0.00390917993676815\\
487	0.00390575667684612\\
488	0.00390225207693731\\
489	0.00389866226152793\\
490	0.00389498300505269\\
491	0.00389120968580481\\
492	0.00388733723143133\\
493	0.00388336005374112\\
494	0.00387927196828374\\
495	0.00387506609003099\\
496	0.00387073468499104\\
497	0.00386626892556339\\
498	0.00386165842002803\\
499	0.00385689018870264\\
500	0.00385194627216657\\
501	0.00384679801097461\\
502	0.00384139254926805\\
503	0.00383562256041093\\
504	0.00382926552272995\\
505	0.00382189839320506\\
506	0.0038130775740306\\
507	0.00380388731055353\\
508	0.0037946208062746\\
509	0.00378526264768088\\
510	0.00377578725115241\\
511	0.00376619051702243\\
512	0.00375646791087849\\
513	0.00374661438458215\\
514	0.00373662419741183\\
515	0.00372649051374451\\
516	0.00371620429625266\\
517	0.0037057509206061\\
518	0.00369509933494614\\
519	0.00368416622119525\\
520	0.0036726943705344\\
521	0.00365983031567574\\
522	0.00364262745854996\\
523	0.0036106483834146\\
524	0.00355903794502149\\
525	0.00350587983412955\\
526	0.0034510683455673\\
527	0.00339447509023707\\
528	0.00333593083808788\\
529	0.0032751899804821\\
530	0.00321188216533384\\
531	0.00314556287288249\\
532	0.00307628479627494\\
533	0.0030052031131649\\
534	0.00293255211022299\\
535	0.00285732301245893\\
536	0.00277659728885553\\
537	0.00268580699715906\\
538	0.00257049800872363\\
539	0.0024390095813954\\
540	0.00230364665895996\\
541	0.00216414020967414\\
542	0.0020201905312924\\
543	0.00187146238020834\\
544	0.00171757856361426\\
545	0.00155810744068666\\
546	0.00139252543776588\\
547	0.00122011107609609\\
548	0.00103965922116657\\
549	0.000848609447595465\\
550	0.000640075640293778\\
551	0.000405854935500931\\
552	0.000159970413471418\\
553	0\\
554	0\\
555	0\\
556	0\\
557	0\\
558	0\\
559	0\\
560	0\\
561	0\\
562	0\\
563	0\\
564	0\\
565	0\\
566	0\\
567	0\\
568	0\\
569	0\\
570	0\\
571	0\\
572	0\\
573	0\\
574	0\\
575	0\\
576	0\\
577	0\\
578	0\\
579	0\\
580	0\\
581	0\\
582	0\\
583	0\\
584	0\\
585	0\\
586	0\\
587	0\\
588	0\\
589	0\\
590	0\\
591	0\\
592	0\\
593	0\\
594	0\\
595	0\\
596	0\\
597	0\\
598	0\\
599	0\\
600	0\\
};
\addplot [color=mycolor17,solid,forget plot]
  table[row sep=crcr]{%
1	0.00386320061721537\\
2	0.00386320017775786\\
3	0.00386319973044255\\
4	0.00386319927512914\\
5	0.00386319881167482\\
6	0.00386319833993427\\
7	0.00386319785975954\\
8	0.00386319737100007\\
9	0.0038631968735026\\
10	0.00386319636711114\\
11	0.00386319585166693\\
12	0.00386319532700837\\
13	0.00386319479297098\\
14	0.00386319424938736\\
15	0.00386319369608712\\
16	0.00386319313289682\\
17	0.00386319255963996\\
18	0.00386319197613686\\
19	0.00386319138220467\\
20	0.00386319077765725\\
21	0.00386319016230516\\
22	0.0038631895359556\\
23	0.0038631888984123\\
24	0.00386318824947553\\
25	0.00386318758894196\\
26	0.00386318691660467\\
27	0.00386318623225304\\
28	0.0038631855356727\\
29	0.00386318482664547\\
30	0.00386318410494927\\
31	0.00386318337035805\\
32	0.00386318262264177\\
33	0.00386318186156625\\
34	0.00386318108689318\\
35	0.00386318029837998\\
36	0.00386317949577974\\
37	0.00386317867884117\\
38	0.00386317784730851\\
39	0.0038631770009214\\
40	0.0038631761394149\\
41	0.00386317526251932\\
42	0.00386317436996016\\
43	0.00386317346145804\\
44	0.0038631725367286\\
45	0.00386317159548243\\
46	0.00386317063742494\\
47	0.00386316966225631\\
48	0.00386316866967137\\
49	0.00386316765935952\\
50	0.00386316663100463\\
51	0.00386316558428493\\
52	0.00386316451887291\\
53	0.00386316343443525\\
54	0.00386316233063269\\
55	0.0038631612071199\\
56	0.00386316006354544\\
57	0.00386315889955156\\
58	0.00386315771477418\\
59	0.00386315650884271\\
60	0.00386315528137998\\
61	0.00386315403200208\\
62	0.00386315276031827\\
63	0.00386315146593086\\
64	0.00386315014843505\\
65	0.00386314880741885\\
66	0.00386314744246293\\
67	0.00386314605314048\\
68	0.00386314463901709\\
69	0.00386314319965062\\
70	0.00386314173459104\\
71	0.00386314024338031\\
72	0.00386313872555222\\
73	0.00386313718063228\\
74	0.00386313560813751\\
75	0.00386313400757635\\
76	0.00386313237844846\\
77	0.00386313072024462\\
78	0.0038631290324465\\
79	0.00386312731452656\\
80	0.00386312556594783\\
81	0.0038631237861638\\
82	0.00386312197461821\\
83	0.00386312013074489\\
84	0.00386311825396756\\
85	0.00386311634369969\\
86	0.00386311439934429\\
87	0.00386311242029373\\
88	0.00386311040592954\\
89	0.00386310835562224\\
90	0.00386310626873112\\
91	0.00386310414460403\\
92	0.00386310198257722\\
93	0.00386309978197509\\
94	0.00386309754211\\
95	0.00386309526228204\\
96	0.00386309294177882\\
97	0.00386309057987525\\
98	0.0038630881758333\\
99	0.00386308572890178\\
100	0.0038630832383161\\
101	0.00386308070329802\\
102	0.00386307812305543\\
103	0.00386307549678208\\
104	0.00386307282365733\\
105	0.00386307010284588\\
106	0.00386306733349756\\
107	0.00386306451474699\\
108	0.00386306164571337\\
109	0.00386305872550014\\
110	0.00386305575319476\\
111	0.00386305272786839\\
112	0.0038630496485756\\
113	0.00386304651435407\\
114	0.00386304332422429\\
115	0.00386304007718928\\
116	0.0038630367722342\\
117	0.00386303340832611\\
118	0.00386302998441359\\
119	0.00386302649942646\\
120	0.00386302295227537\\
121	0.00386301934185151\\
122	0.00386301566702626\\
123	0.00386301192665079\\
124	0.00386300811955575\\
125	0.00386300424455084\\
126	0.00386300030042449\\
127	0.00386299628594344\\
128	0.00386299219985236\\
129	0.00386298804087345\\
130	0.00386298380770604\\
131	0.00386297949902615\\
132	0.0038629751134861\\
133	0.00386297064971409\\
134	0.0038629661063137\\
135	0.00386296148186353\\
136	0.00386295677491667\\
137	0.0038629519840003\\
138	0.00386294710761518\\
139	0.00386294214423517\\
140	0.00386293709230677\\
141	0.0038629319502486\\
142	0.00386292671645092\\
143	0.00386292138927506\\
144	0.00386291596705297\\
145	0.00386291044808661\\
146	0.00386290483064746\\
147	0.00386289911297593\\
148	0.00386289329328082\\
149	0.00386288736973871\\
150	0.00386288134049342\\
151	0.00386287520365537\\
152	0.00386286895730098\\
153	0.00386286259947206\\
154	0.00386285612817516\\
155	0.00386284954138096\\
156	0.00386284283702356\\
157	0.00386283601299985\\
158	0.00386282906716882\\
159	0.00386282199735084\\
160	0.003862814801327\\
161	0.00386280747683834\\
162	0.00386280002158514\\
163	0.00386279243322616\\
164	0.00386278470937789\\
165	0.00386277684761375\\
166	0.0038627688454633\\
167	0.00386276070041146\\
168	0.00386275240989764\\
169	0.00386274397131495\\
170	0.0038627353820093\\
171	0.00386272663927853\\
172	0.00386271774037157\\
173	0.00386270868248748\\
174	0.00386269946277452\\
175	0.00386269007832928\\
176	0.00386268052619563\\
177	0.00386267080336379\\
178	0.00386266090676933\\
179	0.00386265083329214\\
180	0.00386264057975542\\
181	0.00386263014292456\\
182	0.00386261951950614\\
183	0.00386260870614676\\
184	0.00386259769943198\\
185	0.00386258649588514\\
186	0.00386257509196619\\
187	0.00386256348407054\\
188	0.00386255166852781\\
189	0.00386253964160061\\
190	0.00386252739948329\\
191	0.00386251493830064\\
192	0.00386250225410657\\
193	0.00386248934288281\\
194	0.00386247620053752\\
195	0.00386246282290392\\
196	0.00386244920573884\\
197	0.00386243534472132\\
198	0.0038624212354511\\
199	0.00386240687344714\\
200	0.00386239225414605\\
201	0.00386237737290059\\
202	0.00386236222497802\\
203	0.0038623468055585\\
204	0.00386233110973342\\
205	0.0038623151325037\\
206	0.00386229886877809\\
207	0.00386228231337139\\
208	0.00386226546100264\\
209	0.00386224830629334\\
210	0.00386223084376549\\
211	0.00386221306783978\\
212	0.0038621949728336\\
213	0.00386217655295902\\
214	0.00386215780232084\\
215	0.00386213871491445\\
216	0.00386211928462378\\
217	0.00386209950521912\\
218	0.00386207937035493\\
219	0.00386205887356759\\
220	0.00386203800827314\\
221	0.00386201676776494\\
222	0.00386199514521127\\
223	0.00386197313365295\\
224	0.00386195072600083\\
225	0.00386192791503325\\
226	0.00386190469339351\\
227	0.00386188105358721\\
228	0.00386185698797957\\
229	0.0038618324887927\\
230	0.00386180754810279\\
231	0.00386178215783728\\
232	0.00386175630977196\\
233	0.00386172999552797\\
234	0.00386170320656881\\
235	0.00386167593419721\\
236	0.00386164816955202\\
237	0.00386161990360499\\
238	0.00386159112715746\\
239	0.00386156183083705\\
240	0.00386153200509424\\
241	0.00386150164019887\\
242	0.00386147072623659\\
243	0.00386143925310527\\
244	0.00386140721051128\\
245	0.00386137458796571\\
246	0.00386134137478055\\
247	0.00386130756006476\\
248	0.00386127313272027\\
249	0.00386123808143789\\
250	0.0038612023946932\\
251	0.00386116606074224\\
252	0.00386112906761724\\
253	0.0038610914031222\\
254	0.00386105305482836\\
255	0.00386101401006971\\
256	0.00386097425593821\\
257	0.00386093377927913\\
258	0.00386089256668612\\
259	0.00386085060449634\\
260	0.00386080787878537\\
261	0.00386076437536212\\
262	0.00386072007976358\\
263	0.0038606749772495\\
264	0.00386062905279701\\
265	0.00386058229109504\\
266	0.00386053467653874\\
267	0.00386048619322377\\
268	0.00386043682494045\\
269	0.00386038655516783\\
270	0.00386033536706773\\
271	0.0038602832434785\\
272	0.00386023016690889\\
273	0.00386017611953164\\
274	0.00386012108317706\\
275	0.00386006503932647\\
276	0.00386000796910552\\
277	0.00385994985327745\\
278	0.00385989067223624\\
279	0.00385983040599954\\
280	0.00385976903420166\\
281	0.00385970653608637\\
282	0.00385964289049955\\
283	0.00385957807588184\\
284	0.00385951207026108\\
285	0.00385944485124474\\
286	0.00385937639601218\\
287	0.00385930668130682\\
288	0.00385923568342824\\
289	0.00385916337822416\\
290	0.0038590897410823\\
291	0.00385901474692218\\
292	0.00385893837018685\\
293	0.00385886058483442\\
294	0.00385878136432962\\
295	0.00385870068163524\\
296	0.00385861850920341\\
297	0.00385853481896692\\
298	0.00385844958233035\\
299	0.00385836277016122\\
300	0.00385827435278097\\
301	0.00385818429995595\\
302	0.0038580925808883\\
303	0.00385799916420676\\
304	0.00385790401795746\\
305	0.00385780710959461\\
306	0.00385770840597112\\
307	0.00385760787332925\\
308	0.0038575054772911\\
309	0.00385740118284909\\
310	0.00385729495435646\\
311	0.00385718675551761\\
312	0.00385707654937844\\
313	0.00385696429831669\\
314	0.00385684996403215\\
315	0.00385673350753685\\
316	0.00385661488914523\\
317	0.0038564940684642\\
318	0.00385637100438315\\
319	0.00385624565506394\\
320	0.00385611797793069\\
321	0.00385598792965965\\
322	0.0038558554661688\\
323	0.00385572054260746\\
324	0.0038555831133457\\
325	0.00385544313196363\\
326	0.00385530055124052\\
327	0.00385515532314371\\
328	0.00385500739881735\\
329	0.00385485672857082\\
330	0.00385470326186702\\
331	0.00385454694731019\\
332	0.00385438773263353\\
333	0.00385422556468642\\
334	0.00385406038942125\\
335	0.00385389215187984\\
336	0.00385372079617938\\
337	0.00385354626549792\\
338	0.00385336850205936\\
339	0.0038531874471179\\
340	0.00385300304094193\\
341	0.00385281522279737\\
342	0.0038526239309304\\
343	0.00385242910254955\\
344	0.00385223067380719\\
345	0.0038520285797802\\
346	0.00385182275444996\\
347	0.00385161313068141\\
348	0.00385139964020125\\
349	0.00385118221357495\\
350	0.0038509607801827\\
351	0.00385073526819387\\
352	0.00385050560453995\\
353	0.00385027171488567\\
354	0.00385003352359821\\
355	0.00384979095371409\\
356	0.0038495439269038\\
357	0.00384929236343364\\
358	0.00384903618212468\\
359	0.00384877530030865\\
360	0.00384850963378012\\
361	0.00384823909674503\\
362	0.00384796360176499\\
363	0.00384768305969697\\
364	0.00384739737962805\\
365	0.00384710646880478\\
366	0.00384681023255653\\
367	0.00384650857421251\\
368	0.00384620139501168\\
369	0.00384588859400512\\
370	0.00384557006795003\\
371	0.00384524571119469\\
372	0.00384491541555363\\
373	0.00384457907017194\\
374	0.00384423656137798\\
375	0.00384388777252315\\
376	0.00384353258380758\\
377	0.00384317087209046\\
378	0.00384280251068319\\
379	0.0038424273691237\\
380	0.00384204531292975\\
381	0.00384165620332866\\
382	0.00384125989696056\\
383	0.00384085624555107\\
384	0.00384044509554857\\
385	0.00384002628771845\\
386	0.0038395996566827\\
387	0.00383916503038492\\
388	0.00383872222944601\\
389	0.00383827106635165\\
390	0.00383781134438536\\
391	0.00383734285622546\\
392	0.0038368653822783\\
393	0.00383637868937519\\
394	0.00383588253162161\\
395	0.00383537665575838\\
396	0.00383486080689695\\
397	0.00383433471050525\\
398	0.00383379806491724\\
399	0.00383325055168658\\
400	0.0038326918347335\\
401	0.00383212155977938\\
402	0.00383153935435902\\
403	0.00383094482882212\\
404	0.0038303375786635\\
405	0.00382971718767689\\
406	0.00382908322864968\\
407	0.00382843525276533\\
408	0.00382777275585792\\
409	0.00382709513659304\\
410	0.00382640175099086\\
411	0.00382569200259897\\
412	0.00382496524903276\\
413	0.00382422079636952\\
414	0.00382345789269941\\
415	0.00382267572069011\\
416	0.00382187338899307\\
417	0.00382104992228452\\
418	0.00382020424968888\\
419	0.00381933519126889\\
420	0.00381844144216556\\
421	0.00381752155378253\\
422	0.00381657391099624\\
423	0.00381559670336089\\
424	0.00381458788562918\\
425	0.00381354511578655\\
426	0.0038124656395306\\
427	0.00381134603853017\\
428	0.00381018162513228\\
429	0.00380896493197543\\
430	0.00380768199599962\\
431	0.0038063038431679\\
432	0.00380477037323438\\
433	0.0038029775999075\\
434	0.00380086129815737\\
435	0.00379869980355397\\
436	0.00379649180033432\\
437	0.00379423590688405\\
438	0.0037919306709257\\
439	0.00378957456430551\\
440	0.00378716597728736\\
441	0.0037847032121264\\
442	0.00378218447528161\\
443	0.00377960786636622\\
444	0.003776971358026\\
445	0.00377427274861871\\
446	0.00377150953004803\\
447	0.00376867848383706\\
448	0.0037657743871938\\
449	0.0037627857423846\\
450	0.00375968034386319\\
451	0.00375635545412733\\
452	0.00375246236568108\\
453	0.00374677720045931\\
454	0.00373490643622551\\
455	0.00372160884622855\\
456	0.00370803741193669\\
457	0.00369418850366611\\
458	0.00368005842123568\\
459	0.00366564328300176\\
460	0.00365093884673678\\
461	0.00363594024617064\\
462	0.00362064172233534\\
463	0.00360503677434\\
464	0.00358911983299576\\
465	0.00357288930539948\\
466	0.00355633899536265\\
467	0.00353945868158071\\
468	0.00352223773711009\\
469	0.00350466642955815\\
470	0.00348673600253414\\
471	0.00346843579184816\\
472	0.00344975571601982\\
473	0.00343068666761139\\
474	0.00341121947784049\\
475	0.00339133947212614\\
476	0.00337101863885799\\
477	0.00335024169153026\\
478	0.00332899237954051\\
479	0.00330725338479783\\
480	0.00328500622541673\\
481	0.00326223114833906\\
482	0.00323890700911283\\
483	0.00321501113671962\\
484	0.00319051918094141\\
485	0.00316540493926736\\
486	0.00313964015970263\\
487	0.00311319431499142\\
488	0.00308603434280269\\
489	0.00305812434561894\\
490	0.00302942524228893\\
491	0.00299989435999793\\
492	0.00296948495800572\\
493	0.00293814566607575\\
494	0.0029058198204287\\
495	0.00287244466145977\\
496	0.00283795034066952\\
497	0.00280225863393555\\
498	0.00276528106212388\\
499	0.00272691564219204\\
500	0.00268704007475841\\
501	0.0026454950243867\\
502	0.00260203892819086\\
503	0.00255621963891283\\
504	0.0025070010893715\\
505	0.0024515571816792\\
506	0.0023804071624726\\
507	0.00230400224055517\\
508	0.00222627062227436\\
509	0.00214744806552538\\
510	0.00206730837867354\\
511	0.00198508682477377\\
512	0.00190067678391583\\
513	0.00181396187354246\\
514	0.00172481490632849\\
515	0.00163309649818443\\
516	0.00153865394445642\\
517	0.00144132161519882\\
518	0.00134092725442683\\
519	0.00123732123756069\\
520	0.00113049282112899\\
521	0.0010210130159867\\
522	0.000911700043803545\\
523	0.000813861606725609\\
524	0.00073313170074981\\
525	0.000649680795420767\\
526	0.000563311348217927\\
527	0.000473772576124755\\
528	0.00038068974940592\\
529	0.000283321274931334\\
530	0.000179753860783708\\
531	6.42317041917577e-05\\
532	0\\
533	0\\
534	0\\
535	0\\
536	0\\
537	0\\
538	0\\
539	0\\
540	0\\
541	0\\
542	0\\
543	0\\
544	0\\
545	0\\
546	0\\
547	0\\
548	0\\
549	0\\
550	0\\
551	0\\
552	0\\
553	0\\
554	0\\
555	0\\
556	0\\
557	0\\
558	0\\
559	0\\
560	0\\
561	0\\
562	0\\
563	0\\
564	0\\
565	0\\
566	0\\
567	0\\
568	0\\
569	0\\
570	0\\
571	0\\
572	0\\
573	0\\
574	0\\
575	0\\
576	0\\
577	0\\
578	0\\
579	0\\
580	0\\
581	0\\
582	0\\
583	0\\
584	0\\
585	0\\
586	0\\
587	0\\
588	0\\
589	0\\
590	0\\
591	0\\
592	0\\
593	0\\
594	0\\
595	0\\
596	0\\
597	0\\
598	0\\
599	0\\
600	0\\
};
\addplot [color=mycolor18,solid,forget plot]
  table[row sep=crcr]{%
1	0.00283984636363328\\
2	0.00283984271850578\\
3	0.00283983900819595\\
4	0.00283983523153989\\
5	0.00283983138735291\\
6	0.0028398274744292\\
7	0.00283982349154148\\
8	0.00283981943744058\\
9	0.00283981531085504\\
10	0.00283981111049076\\
11	0.00283980683503055\\
12	0.00283980248313376\\
13	0.00283979805343584\\
14	0.00283979354454789\\
15	0.00283978895505627\\
16	0.00283978428352215\\
17	0.00283977952848104\\
18	0.00283977468844236\\
19	0.00283976976188895\\
20	0.00283976474727663\\
21	0.00283975964303366\\
22	0.00283975444756032\\
23	0.00283974915922837\\
24	0.00283974377638055\\
25	0.00283973829733007\\
26	0.00283973272036007\\
27	0.0028397270437231\\
28	0.00283972126564055\\
29	0.00283971538430215\\
30	0.00283970939786535\\
31	0.00283970330445476\\
32	0.00283969710216157\\
33	0.00283969078904297\\
34	0.00283968436312153\\
35	0.00283967782238456\\
36	0.00283967116478354\\
37	0.00283966438823341\\
38	0.00283965749061199\\
39	0.00283965046975926\\
40	0.00283964332347674\\
41	0.00283963604952675\\
42	0.00283962864563174\\
43	0.0028396211094736\\
44	0.0028396134386929\\
45	0.00283960563088817\\
46	0.00283959768361516\\
47	0.00283958959438605\\
48	0.00283958136066871\\
49	0.00283957297988588\\
50	0.0028395644494144\\
51	0.00283955576658435\\
52	0.00283954692867825\\
53	0.00283953793293018\\
54	0.00283952877652497\\
55	0.00283951945659726\\
56	0.00283950997023065\\
57	0.00283950031445677\\
58	0.00283949048625435\\
59	0.0028394804825483\\
60	0.00283947030020873\\
61	0.00283945993604997\\
62	0.0028394493868296\\
63	0.00283943864924741\\
64	0.00283942771994438\\
65	0.00283941659550164\\
66	0.00283940527243939\\
67	0.00283939374721582\\
68	0.00283938201622599\\
69	0.00283937007580074\\
70	0.0028393579222055\\
71	0.00283934555163913\\
72	0.00283933296023278\\
73	0.0028393201440486\\
74	0.0028393070990786\\
75	0.00283929382124331\\
76	0.00283928030639056\\
77	0.00283926655029417\\
78	0.00283925254865262\\
79	0.00283923829708771\\
80	0.00283922379114319\\
81	0.00283920902628336\\
82	0.00283919399789164\\
83	0.00283917870126916\\
84	0.00283916313163326\\
85	0.002839147284116\\
86	0.00283913115376263\\
87	0.00283911473553003\\
88	0.00283909802428516\\
89	0.00283908101480343\\
90	0.00283906370176706\\
91	0.00283904607976341\\
92	0.00283902814328329\\
93	0.00283900988671924\\
94	0.00283899130436373\\
95	0.00283897239040744\\
96	0.00283895313893736\\
97	0.00283893354393498\\
98	0.00283891359927438\\
99	0.00283889329872031\\
100	0.00283887263592623\\
101	0.00283885160443232\\
102	0.00283883019766343\\
103	0.00283880840892703\\
104	0.00283878623141109\\
105	0.00283876365818194\\
106	0.00283874068218209\\
107	0.002838717296228\\
108	0.00283869349300782\\
109	0.00283866926507907\\
110	0.00283864460486633\\
111	0.0028386195046588\\
112	0.00283859395660789\\
113	0.00283856795272475\\
114	0.00283854148487772\\
115	0.0028385145447898\\
116	0.00283848712403599\\
117	0.00283845921404063\\
118	0.00283843080607472\\
119	0.00283840189125311\\
120	0.00283837246053174\\
121	0.00283834250470471\\
122	0.00283831201440138\\
123	0.00283828098008345\\
124	0.00283824939204186\\
125	0.00283821724039373\\
126	0.00283818451507922\\
127	0.00283815120585835\\
128	0.00283811730230773\\
129	0.00283808279381722\\
130	0.00283804766958658\\
131	0.00283801191862198\\
132	0.00283797552973258\\
133	0.00283793849152687\\
134	0.00283790079240907\\
135	0.00283786242057544\\
136	0.00283782336401046\\
137	0.00283778361048304\\
138	0.00283774314754256\\
139	0.00283770196251489\\
140	0.00283766004249836\\
141	0.00283761737435955\\
142	0.00283757394472915\\
143	0.00283752973999759\\
144	0.00283748474631073\\
145	0.00283743894956534\\
146	0.00283739233540461\\
147	0.00283734488921348\\
148	0.00283729659611395\\
149	0.00283724744096026\\
150	0.00283719740833397\\
151	0.00283714648253905\\
152	0.00283709464759671\\
153	0.00283704188724026\\
154	0.00283698818490984\\
155	0.00283693352374703\\
156	0.00283687788658937\\
157	0.00283682125596478\\
158	0.00283676361408586\\
159	0.00283670494284409\\
160	0.00283664522380397\\
161	0.00283658443819692\\
162	0.00283652256691517\\
163	0.00283645959050554\\
164	0.00283639548916303\\
165	0.00283633024272432\\
166	0.00283626383066115\\
167	0.00283619623207359\\
168	0.00283612742568315\\
169	0.00283605738982573\\
170	0.00283598610244452\\
171	0.00283591354108269\\
172	0.00283583968287596\\
173	0.00283576450454504\\
174	0.00283568798238789\\
175	0.00283561009227186\\
176	0.00283553080962565\\
177	0.00283545010943117\\
178	0.00283536796621515\\
179	0.00283528435404067\\
180	0.00283519924649847\\
181	0.00283511261669812\\
182	0.002835024437259\\
183	0.00283493468030112\\
184	0.00283484331743571\\
185	0.00283475031975571\\
186	0.002834655657826\\
187	0.00283455930167346\\
188	0.00283446122077682\\
189	0.00283436138405635\\
190	0.00283425975986331\\
191	0.00283415631596916\\
192	0.00283405101955467\\
193	0.00283394383719864\\
194	0.00283383473486656\\
195	0.00283372367789893\\
196	0.00283361063099944\\
197	0.00283349555822279\\
198	0.00283337842296239\\
199	0.00283325918793775\\
200	0.0028331378151816\\
201	0.00283301426602681\\
202	0.00283288850109298\\
203	0.00283276048027282\\
204	0.00283263016271819\\
205	0.00283249750682592\\
206	0.0028323624702233\\
207	0.00283222500975327\\
208	0.00283208508145939\\
209	0.00283194264057037\\
210	0.00283179764148436\\
211	0.00283165003775298\\
212	0.00283149978206489\\
213	0.00283134682622912\\
214	0.00283119112115802\\
215	0.00283103261684988\\
216	0.00283087126237117\\
217	0.00283070700583846\\
218	0.00283053979439989\\
219	0.00283036957421633\\
220	0.00283019629044213\\
221	0.00283001988720545\\
222	0.00282984030758825\\
223	0.00282965749360577\\
224	0.00282947138618563\\
225	0.00282928192514657\\
226	0.00282908904917662\\
227	0.0028288926958109\\
228	0.00282869280140898\\
229	0.00282848930113166\\
230	0.00282828212891742\\
231	0.00282807121745825\\
232	0.00282785649817504\\
233	0.0028276379011925\\
234	0.00282741535531345\\
235	0.00282718878799268\\
236	0.00282695812531021\\
237	0.00282672329194402\\
238	0.00282648421114222\\
239	0.00282624080469461\\
240	0.0028259929929037\\
241	0.00282574069455508\\
242	0.00282548382688723\\
243	0.00282522230556064\\
244	0.00282495604462639\\
245	0.00282468495649398\\
246	0.00282440895189856\\
247	0.00282412793986747\\
248	0.00282384182768609\\
249	0.00282355052086301\\
250	0.00282325392309446\\
251	0.00282295193622805\\
252	0.00282264446022572\\
253	0.00282233139312598\\
254	0.0028220126310054\\
255	0.00282168806793927\\
256	0.0028213575959615\\
257	0.00282102110502374\\
258	0.00282067848295363\\
259	0.00282032961541223\\
260	0.00281997438585066\\
261	0.00281961267546582\\
262	0.00281924436315529\\
263	0.00281886932547126\\
264	0.0028184874365737\\
265	0.00281809856818253\\
266	0.00281770258952883\\
267	0.00281729936730527\\
268	0.00281688876561541\\
269	0.00281647064592223\\
270	0.0028160448669955\\
271	0.00281561128485836\\
272	0.00281516975273278\\
273	0.00281472012098409\\
274	0.00281426223706446\\
275	0.0028137959454554\\
276	0.00281332108760923\\
277	0.00281283750188951\\
278	0.00281234502351041\\
279	0.00281184348447511\\
280	0.00281133271351306\\
281	0.00281081253601624\\
282	0.00281028277397436\\
283	0.002809743245909\\
284	0.00280919376680665\\
285	0.00280863414805074\\
286	0.00280806419735257\\
287	0.00280748371868121\\
288	0.00280689251219233\\
289	0.00280629037415594\\
290	0.00280567709688316\\
291	0.00280505246865183\\
292	0.00280441627363122\\
293	0.00280376829180558\\
294	0.00280310829889677\\
295	0.0028024360662858\\
296	0.00280175136093345\\
297	0.00280105394529983\\
298	0.00280034357726302\\
299	0.00279962001003669\\
300	0.00279888299208685\\
301	0.00279813226704765\\
302	0.00279736757363616\\
303	0.00279658864556642\\
304	0.00279579521146249\\
305	0.00279498699477061\\
306	0.00279416371367063\\
307	0.00279332508098644\\
308	0.00279247080409567\\
309	0.00279160058483853\\
310	0.00279071411942579\\
311	0.002789811098346\\
312	0.00278889120627186\\
313	0.00278795412196575\\
314	0.00278699951818442\\
315	0.00278602706158284\\
316	0.00278503641261714\\
317	0.00278402722544659\\
318	0.00278299914783472\\
319	0.00278195182104928\\
320	0.00278088487976116\\
321	0.00277979795194219\\
322	0.00277869065876157\\
323	0.00277756261448098\\
324	0.00277641342634813\\
325	0.00277524269448873\\
326	0.00277405001179652\\
327	0.00277283496382143\\
328	0.00277159712865538\\
329	0.00277033607681578\\
330	0.00276905137112628\\
331	0.00276774256659457\\
332	0.00276640921028704\\
333	0.00276505084119987\\
334	0.00276366699012633\\
335	0.00276225717952003\\
336	0.0027608209233537\\
337	0.00275935772697327\\
338	0.002757867086947\\
339	0.00275634849090932\\
340	0.00275480141739916\\
341	0.00275322533569259\\
342	0.00275161970562948\\
343	0.00274998397743406\\
344	0.00274831759152909\\
345	0.0027466199783432\\
346	0.00274489055811124\\
347	0.00274312874066685\\
348	0.00274133392522688\\
349	0.00273950550016684\\
350	0.0027376428427863\\
351	0.00273574531906328\\
352	0.0027338122833961\\
353	0.00273184307833093\\
354	0.00272983703427334\\
355	0.00272779346918177\\
356	0.00272571168824128\\
357	0.00272359098351554\\
358	0.0027214306335751\\
359	0.00271922990309963\\
360	0.00271698804245171\\
361	0.00271470428721956\\
362	0.00271237785772575\\
363	0.00271000795849902\\
364	0.00270759377770548\\
365	0.00270513448653603\\
366	0.00270262923854566\\
367	0.00270007716894068\\
368	0.00269747739380911\\
369	0.00269482900928932\\
370	0.00269213109067148\\
371	0.00268938269142588\\
372	0.00268658284215176\\
373	0.00268373054943951\\
374	0.00268082479463852\\
375	0.0026778645325221\\
376	0.00267484868983998\\
377	0.00267177616374766\\
378	0.00266864582010079\\
379	0.00266545649160092\\
380	0.00266220697577716\\
381	0.00265889603278586\\
382	0.0026555223830072\\
383	0.00265208470441345\\
384	0.00264858162967697\\
385	0.00264501174297499\\
386	0.002641373576426\\
387	0.00263766560604577\\
388	0.00263388624700499\\
389	0.002630033847735\\
390	0.00262610668192722\\
391	0.00262210293652132\\
392	0.00261802069239686\\
393	0.00261385789407911\\
394	0.00260961231188414\\
395	0.00260528153371685\\
396	0.00260086310772563\\
397	0.00259635492022998\\
398	0.00259175470647134\\
399	0.00258705971132202\\
400	0.00258226702724363\\
401	0.00257737358682149\\
402	0.00257237615923375\\
403	0.00256727135586422\\
404	0.00256205565503259\\
405	0.00255672546191674\\
406	0.00255127721934318\\
407	0.00254570754700848\\
408	0.00254001322769217\\
409	0.00253419048557407\\
410	0.002528233073708\\
411	0.00252213320223588\\
412	0.00251588545339777\\
413	0.0025094840052289\\
414	0.00250292258335434\\
415	0.00249619440555662\\
416	0.00248929211789261\\
417	0.00248220772088192\\
418	0.00247493248399702\\
419	0.00246745684632346\\
420	0.00245977030079602\\
421	0.00245186125879021\\
422	0.00244371689089691\\
423	0.00243532293799798\\
424	0.00242666348297729\\
425	0.00241772066349657\\
426	0.00240847427715346\\
427	0.00239890113913909\\
428	0.00238897375490957\\
429	0.00237865687037138\\
430	0.0023678970214388\\
431	0.00235658812589125\\
432	0.00234445301092195\\
433	0.00233062431083223\\
434	0.00231321595823006\\
435	0.00229538887872112\\
436	0.00227712911169738\\
437	0.00225842195679285\\
438	0.00223925191838529\\
439	0.00221960264533866\\
440	0.00219945686561942\\
441	0.00217879631548368\\
442	0.00215760166320238\\
443	0.00213585242824216\\
444	0.0021135269000313\\
445	0.00209060207126689\\
446	0.00206705363748975\\
447	0.00204285624100025\\
448	0.00201798457705556\\
449	0.0019924175329847\\
450	0.00196615308900204\\
451	0.0019392618814366\\
452	0.00191208150788773\\
453	0.00188592980328271\\
454	0.00186575499415459\\
455	0.00184634549182879\\
456	0.0018264179726812\\
457	0.00180595847354217\\
458	0.001784953701425\\
459	0.00176339107088575\\
460	0.00174125826616331\\
461	0.00171854154734407\\
462	0.00169522118118219\\
463	0.00167126166615424\\
464	0.00164659860986812\\
465	0.00162116075735014\\
466	0.0015950962240734\\
467	0.00156848250343061\\
468	0.00154128321357145\\
469	0.00151344825891532\\
470	0.00148496540590409\\
471	0.00145590469527504\\
472	0.00142626966023149\\
473	0.0013960911929633\\
474	0.00136545656688742\\
475	0.00133453732621733\\
476	0.00130348109058821\\
477	0.00127171877652943\\
478	0.00123922891148697\\
479	0.00120598968326226\\
480	0.00117197819339074\\
481	0.00113717038199236\\
482	0.00110154094938564\\
483	0.0010650632747847\\
484	0.00102770933251127\\
485	0.000989449606398602\\
486	0.000950253003461431\\
487	0.000910086768033306\\
488	0.000868916396119481\\
489	0.000826705547804407\\
490	0.000783415967386659\\
491	0.000739007424426463\\
492	0.000693437597504418\\
493	0.000646661989758019\\
494	0.000598633811270603\\
495	0.00054930392098585\\
496	0.000498620679010176\\
497	0.000446529504226808\\
498	0.000392972284849068\\
499	0.000337885764413177\\
500	0.000281196759940849\\
501	0.000222807664419292\\
502	0.000162552183188757\\
503	0.000100059625682899\\
504	3.43368831611899e-05\\
505	0\\
506	0\\
507	0\\
508	0\\
509	0\\
510	0\\
511	0\\
512	0\\
513	0\\
514	0\\
515	0\\
516	0\\
517	0\\
518	0\\
519	0\\
520	0\\
521	0\\
522	0\\
523	0\\
524	0\\
525	0\\
526	0\\
527	0\\
528	0\\
529	0\\
530	0\\
531	0\\
532	0\\
533	0\\
534	0\\
535	0\\
536	0\\
537	0\\
538	0\\
539	0\\
540	0\\
541	0\\
542	0\\
543	0\\
544	0\\
545	0\\
546	0\\
547	0\\
548	0\\
549	0\\
550	0\\
551	0\\
552	0\\
553	0\\
554	0\\
555	0\\
556	0\\
557	0\\
558	0\\
559	0\\
560	0\\
561	0\\
562	0\\
563	0\\
564	0\\
565	0\\
566	0\\
567	0\\
568	0\\
569	0\\
570	0\\
571	0\\
572	0\\
573	0\\
574	0\\
575	0\\
576	0\\
577	0\\
578	0\\
579	0\\
580	0\\
581	0\\
582	0\\
583	0\\
584	0\\
585	0\\
586	0\\
587	0\\
588	0\\
589	0\\
590	0\\
591	0\\
592	0\\
593	0\\
594	0\\
595	0\\
596	0\\
597	0\\
598	0\\
599	0\\
600	0\\
};
\addplot [color=red!25!mycolor17,solid,forget plot]
  table[row sep=crcr]{%
1	0.00135496690887878\\
2	0.00135496089430922\\
3	0.0013549547721517\\
4	0.00135494854048434\\
5	0.00135494219735098\\
6	0.00135493574076059\\
7	0.00135492916868658\\
8	0.00135492247906624\\
9	0.0013549156698001\\
10	0.00135490873875119\\
11	0.00135490168374449\\
12	0.00135489450256614\\
13	0.00135488719296282\\
14	0.00135487975264104\\
15	0.00135487217926638\\
16	0.00135486447046282\\
17	0.00135485662381197\\
18	0.00135484863685228\\
19	0.00135484050707835\\
20	0.00135483223194009\\
21	0.00135482380884195\\
22	0.00135481523514208\\
23	0.00135480650815155\\
24	0.00135479762513347\\
25	0.00135478858330216\\
26	0.00135477937982228\\
27	0.00135477001180794\\
28	0.0013547604763218\\
29	0.00135475077037414\\
30	0.00135474089092195\\
31	0.00135473083486798\\
32	0.00135472059905976\\
33	0.00135471018028861\\
34	0.00135469957528865\\
35	0.00135468878073582\\
36	0.00135467779324675\\
37	0.00135466660937781\\
38	0.00135465522562394\\
39	0.00135464363841763\\
40	0.00135463184412777\\
41	0.00135461983905852\\
42	0.00135460761944819\\
43	0.00135459518146799\\
44	0.00135458252122093\\
45	0.00135456963474054\\
46	0.00135455651798963\\
47	0.0013545431668591\\
48	0.00135452957716658\\
49	0.00135451574465513\\
50	0.00135450166499198\\
51	0.00135448733376711\\
52	0.00135447274649191\\
53	0.00135445789859774\\
54	0.00135444278543457\\
55	0.00135442740226945\\
56	0.00135441174428509\\
57	0.00135439580657834\\
58	0.00135437958415866\\
59	0.00135436307194654\\
60	0.00135434626477197\\
61	0.00135432915737274\\
62	0.0013543117443929\\
63	0.001354294020381\\
64	0.00135427597978841\\
65	0.00135425761696763\\
66	0.00135423892617049\\
67	0.00135421990154635\\
68	0.00135420053714027\\
69	0.00135418082689117\\
70	0.00135416076462995\\
71	0.00135414034407751\\
72	0.00135411955884283\\
73	0.00135409840242096\\
74	0.00135407686819098\\
75	0.00135405494941393\\
76	0.00135403263923073\\
77	0.00135400993066\\
78	0.0013539868165959\\
79	0.00135396328980589\\
80	0.00135393934292847\\
81	0.0013539149684709\\
82	0.00135389015880683\\
83	0.00135386490617392\\
84	0.00135383920267142\\
85	0.00135381304025768\\
86	0.00135378641074764\\
87	0.00135375930581028\\
88	0.00135373171696598\\
89	0.00135370363558388\\
90	0.0013536750528792\\
91	0.00135364595991044\\
92	0.00135361634757661\\
93	0.00135358620661437\\
94	0.00135355552759512\\
95	0.00135352430092204\\
96	0.00135349251682708\\
97	0.00135346016536791\\
98	0.00135342723642478\\
99	0.00135339371969735\\
100	0.00135335960470146\\
101	0.00135332488076582\\
102	0.00135328953702868\\
103	0.00135325356243441\\
104	0.00135321694573\\
105	0.00135317967546154\\
106	0.00135314173997065\\
107	0.00135310312739074\\
108	0.00135306382564332\\
109	0.00135302382243422\\
110	0.00135298310524965\\
111	0.0013529416613523\\
112	0.00135289947777735\\
113	0.00135285654132832\\
114	0.00135281283857295\\
115	0.00135276835583895\\
116	0.0013527230792097\\
117	0.00135267699451981\\
118	0.0013526300873507\\
119	0.001352582343026\\
120	0.00135253374660697\\
121	0.00135248428288769\\
122	0.0013524339363903\\
123	0.00135238269136011\\
124	0.00135233053176057\\
125	0.00135227744126824\\
126	0.00135222340326755\\
127	0.0013521684008456\\
128	0.00135211241678672\\
129	0.00135205543356708\\
130	0.00135199743334905\\
131	0.00135193839797559\\
132	0.00135187830896445\\
133	0.00135181714750229\\
134	0.00135175489443867\\
135	0.00135169153028\\
136	0.00135162703518328\\
137	0.00135156138894978\\
138	0.00135149457101863\\
139	0.00135142656046019\\
140	0.00135135733596943\\
141	0.00135128687585906\\
142	0.00135121515805263\\
143	0.00135114216007744\\
144	0.00135106785905737\\
145	0.00135099223170547\\
146	0.0013509152543166\\
147	0.0013508369027597\\
148	0.00135075715247013\\
149	0.00135067597844172\\
150	0.00135059335521874\\
151	0.00135050925688768\\
152	0.00135042365706891\\
153	0.00135033652890818\\
154	0.00135024784506793\\
155	0.00135015757771846\\
156	0.00135006569852893\\
157	0.00134997217865818\\
158	0.00134987698874539\\
159	0.00134978009890056\\
160	0.00134968147869478\\
161	0.00134958109715037\\
162	0.00134947892273082\\
163	0.00134937492333046\\
164	0.00134926906626402\\
165	0.00134916131825601\\
166	0.00134905164542979\\
167	0.0013489400132965\\
168	0.00134882638674382\\
169	0.0013487107300244\\
170	0.00134859300674417\\
171	0.00134847317985039\\
172	0.00134835121161942\\
173	0.00134822706364436\\
174	0.00134810069682235\\
175	0.00134797207134167\\
176	0.00134784114666862\\
177	0.00134770788153404\\
178	0.00134757223391971\\
179	0.00134743416104437\\
180	0.00134729361934952\\
181	0.00134715056448493\\
182	0.00134700495129389\\
183	0.00134685673379811\\
184	0.00134670586518244\\
185	0.00134655229777907\\
186	0.00134639598305176\\
187	0.00134623687157937\\
188	0.00134607491303939\\
189	0.00134591005619096\\
190	0.00134574224885759\\
191	0.00134557143790959\\
192	0.00134539756924606\\
193	0.00134522058777666\\
194	0.00134504043740283\\
195	0.00134485706099881\\
196	0.00134467040039217\\
197	0.00134448039634399\\
198	0.00134428698852867\\
199	0.00134409011551324\\
200	0.00134388971473637\\
201	0.00134368572248689\\
202	0.00134347807388187\\
203	0.00134326670284429\\
204	0.00134305154208025\\
205	0.00134283252305568\\
206	0.00134260957597265\\
207	0.0013423826297451\\
208	0.00134215161197416\\
209	0.00134191644892294\\
210	0.00134167706549077\\
211	0.00134143338518696\\
212	0.00134118533010397\\
213	0.00134093282089009\\
214	0.00134067577672151\\
215	0.0013404141152738\\
216	0.00134014775269288\\
217	0.00133987660356527\\
218	0.00133960058088785\\
219	0.00133931959603687\\
220	0.00133903355873639\\
221	0.00133874237702607\\
222	0.00133844595722824\\
223	0.00133814420391427\\
224	0.00133783701987031\\
225	0.00133752430606225\\
226	0.00133720596159994\\
227	0.0013368818837007\\
228	0.00133655196765203\\
229	0.00133621610677351\\
230	0.00133587419237795\\
231	0.00133552611373166\\
232	0.00133517175801394\\
233	0.0013348110102756\\
234	0.00133444375339676\\
235	0.00133406986804358\\
236	0.00133368923262419\\
237	0.00133330172324359\\
238	0.00133290721365768\\
239	0.00133250557522622\\
240	0.00133209667686478\\
241	0.00133168038499576\\
242	0.00133125656349823\\
243	0.00133082507365676\\
244	0.00133038577410916\\
245	0.00132993852079302\\
246	0.00132948316689118\\
247	0.00132901956277597\\
248	0.00132854755595224\\
249	0.00132806699099922\\
250	0.00132757770951103\\
251	0.00132707955003596\\
252	0.00132657234801445\\
253	0.00132605593571566\\
254	0.00132553014217272\\
255	0.00132499479311653\\
256	0.00132444971090822\\
257	0.00132389471446996\\
258	0.00132332961921451\\
259	0.001322754236973\\
260	0.00132216837592137\\
261	0.001321571840505\\
262	0.0013209644313619\\
263	0.00132034594524407\\
264	0.00131971617493732\\
265	0.00131907490917922\\
266	0.00131842193257536\\
267	0.00131775702551378\\
268	0.00131707996407755\\
269	0.00131639051995551\\
270	0.00131568846035101\\
271	0.00131497354788884\\
272	0.00131424554052002\\
273	0.00131350419142467\\
274	0.00131274924891278\\
275	0.00131198045632283\\
276	0.0013111975519184\\
277	0.00131040026878244\\
278	0.0013095883347095\\
279	0.00130876147209553\\
280	0.00130791939782557\\
281	0.00130706182315896\\
282	0.00130618845361234\\
283	0.00130529898884009\\
284	0.0013043931225125\\
285	0.00130347054219139\\
286	0.00130253092920325\\
287	0.00130157395850985\\
288	0.00130059929857632\\
289	0.00129960661123658\\
290	0.0012985955515563\\
291	0.00129756576769301\\
292	0.00129651690075369\\
293	0.0012954485846497\\
294	0.00129436044594889\\
295	0.00129325210372504\\
296	0.00129212316940462\\
297	0.00129097324661072\\
298	0.00128980193100424\\
299	0.00128860881012234\\
300	0.00128739346321414\\
301	0.0012861554610736\\
302	0.00128489436586965\\
303	0.00128360973097366\\
304	0.00128230110078401\\
305	0.00128096801054811\\
306	0.00127960998618158\\
307	0.00127822654408486\\
308	0.00127681719095712\\
309	0.00127538142360755\\
310	0.00127391872876414\\
311	0.00127242858287987\\
312	0.00127091045193644\\
313	0.00126936379124556\\
314	0.0012677880452479\\
315	0.00126618264730961\\
316	0.0012645470195167\\
317	0.00126288057246705\\
318	0.00126118270506038\\
319	0.00125945280428601\\
320	0.00125769024500862\\
321	0.00125589438975189\\
322	0.00125406458848017\\
323	0.00125220017837815\\
324	0.00125030048362847\\
325	0.00124836481518733\\
326	0.00124639247055803\\
327	0.0012443827335622\\
328	0.00124233487410888\\
329	0.00124024814796093\\
330	0.0012381217964989\\
331	0.00123595504648178\\
332	0.00123374710980439\\
333	0.00123149718325107\\
334	0.00122920444824498\\
335	0.00122686807059255\\
336	0.0012244872002224\\
337	0.00122206097091792\\
338	0.00121958850004269\\
339	0.00121706888825804\\
340	0.00121450121923165\\
341	0.00121188455933671\\
342	0.00120921795734093\\
343	0.00120650044408543\\
344	0.00120373103215371\\
345	0.00120090871553188\\
346	0.00119803246926022\\
347	0.00119510124907725\\
348	0.00119211399105691\\
349	0.00118906961124057\\
350	0.00118596700526513\\
351	0.00118280504798926\\
352	0.00117958259311903\\
353	0.00117629847283388\\
354	0.0011729514974118\\
355	0.00116954045485176\\
356	0.00116606411048735\\
357	0.00116252120659077\\
358	0.00115891046196604\\
359	0.00115523057153034\\
360	0.00115148020588229\\
361	0.00114765801085599\\
362	0.00114376260705909\\
363	0.00113979258939345\\
364	0.00113574652655665\\
365	0.00113162296052233\\
366	0.0011274204059974\\
367	0.00112313734985399\\
368	0.00111877225053367\\
369	0.00111432353742172\\
370	0.00110978961018889\\
371	0.0011051688380981\\
372	0.00110045955927375\\
373	0.00109566007993108\\
374	0.0010907686735636\\
375	0.00108578358008659\\
376	0.00108070300493549\\
377	0.00107552511811846\\
378	0.0010702480532234\\
379	0.00106486990638113\\
380	0.0010593887351879\\
381	0.00105380255759245\\
382	0.00104810935075489\\
383	0.00104230704988652\\
384	0.0010363935470779\\
385	0.00103036669011159\\
386	0.0010242242812134\\
387	0.00101796407556321\\
388	0.00101158377899479\\
389	0.0010050810431937\\
390	0.000998453453580481\\
391	0.000991698496614763\\
392	0.000984813471315704\\
393	0.000977795256818242\\
394	0.000970639737619394\\
395	0.000963340544719313\\
396	0.00095588706417462\\
397	0.000948265363656983\\
398	0.000940482377897597\\
399	0.000932549926432309\\
400	0.000924464723434965\\
401	0.000916223368217055\\
402	0.000907822366721184\\
403	0.000899258207615171\\
404	0.000890527579727133\\
405	0.000881627940607092\\
406	0.000872558913773815\\
407	0.000863325483724827\\
408	0.000853944449055149\\
409	0.000844453882714544\\
410	0.000834909614338404\\
411	0.000825286993102631\\
412	0.000815461872882012\\
413	0.000805430106370623\\
414	0.000795187588521162\\
415	0.000784730288121531\\
416	0.000774054285943402\\
417	0.000763155821140604\\
418	0.00075203134761258\\
419	0.000740677602422308\\
420	0.000729091688817819\\
421	0.000717271176972546\\
422	0.00070521422628937\\
423	0.000692919734044887\\
424	0.000680387516983011\\
425	0.00066761853716504\\
426	0.000654615200771179\\
427	0.000641381823586381\\
428	0.000627925606596906\\
429	0.000614259419503956\\
430	0.000600411310885943\\
431	0.000586459318011014\\
432	0.000572661427200527\\
433	0.000559942695223632\\
434	0.000550463121390471\\
435	0.000540793076180043\\
436	0.000530929798269977\\
437	0.000520870642980117\\
438	0.000510613104098667\\
439	0.00050015483806598\\
440	0.000489493690757346\\
441	0.000478627727235138\\
442	0.000467555265246462\\
443	0.00045627491450659\\
444	0.000444785627672718\\
445	0.000433086780486784\\
446	0.000421178332435776\\
447	0.000409061215383379\\
448	0.000396738357956686\\
449	0.00038421740466746\\
450	0.00037151756828031\\
451	0.00035868472472185\\
452	0.000345813460923171\\
453	0.000333015268270055\\
454	0.000319915237785431\\
455	0.000306408211838269\\
456	0.000292461935805177\\
457	0.000278039159059683\\
458	0.000263096498968872\\
459	0.00024758257884974\\
460	0.000231434215845809\\
461	0.000214566844856925\\
462	0.000196847069257896\\
463	0.000178008364435826\\
464	0.000157382959220837\\
465	0.000133019685236701\\
466	0.000106644759587974\\
467	7.9655848393468e-05\\
468	5.18777476358028e-05\\
469	2.2741055242951e-05\\
470	0\\
471	0\\
472	0\\
473	0\\
474	0\\
475	0\\
476	0\\
477	0\\
478	0\\
479	0\\
480	0\\
481	0\\
482	0\\
483	0\\
484	0\\
485	0\\
486	0\\
487	0\\
488	0\\
489	0\\
490	0\\
491	0\\
492	0\\
493	0\\
494	0\\
495	0\\
496	0\\
497	0\\
498	0\\
499	0\\
500	0\\
501	0\\
502	0\\
503	0\\
504	0\\
505	0\\
506	0\\
507	0\\
508	0\\
509	0\\
510	0\\
511	0\\
512	0\\
513	0\\
514	0\\
515	0\\
516	0\\
517	0\\
518	0\\
519	0\\
520	0\\
521	0\\
522	0\\
523	0\\
524	0\\
525	0\\
526	0\\
527	0\\
528	0\\
529	0\\
530	0\\
531	0\\
532	0\\
533	0\\
534	0\\
535	0\\
536	0\\
537	0\\
538	0\\
539	0\\
540	0\\
541	0\\
542	0\\
543	0\\
544	0\\
545	0\\
546	0\\
547	0\\
548	0\\
549	0\\
550	0\\
551	0\\
552	0\\
553	0\\
554	0\\
555	0\\
556	0\\
557	0\\
558	0\\
559	0\\
560	0\\
561	0\\
562	0\\
563	0\\
564	0\\
565	0\\
566	0\\
567	0\\
568	0\\
569	0\\
570	0\\
571	0\\
572	0\\
573	0\\
574	0\\
575	0\\
576	0\\
577	0\\
578	0\\
579	0\\
580	0\\
581	0\\
582	0\\
583	0\\
584	0\\
585	0\\
586	0\\
587	0\\
588	0\\
589	0\\
590	0\\
591	0\\
592	0\\
593	0\\
594	0\\
595	0\\
596	0\\
597	0\\
598	0\\
599	0\\
600	0\\
};
\addplot [color=mycolor19,solid,forget plot]
  table[row sep=crcr]{%
1	0.000420210774699427\\
2	0.00042020462908718\\
3	0.000420198373475441\\
4	0.000420192005897872\\
5	0.00042018552435305\\
6	0.000420178926803814\\
7	0.000420172211176661\\
8	0.000420165375361069\\
9	0.000420158417208842\\
10	0.000420151334533464\\
11	0.000420144125109368\\
12	0.000420136786671287\\
13	0.000420129316913536\\
14	0.000420121713489273\\
15	0.000420113974009767\\
16	0.000420106096043688\\
17	0.00042009807711629\\
18	0.000420089914708694\\
19	0.000420081606257064\\
20	0.00042007314915182\\
21	0.000420064540736825\\
22	0.000420055778308539\\
23	0.000420046859115208\\
24	0.000420037780355971\\
25	0.000420028539180003\\
26	0.000420019132685619\\
27	0.000420009557919356\\
28	0.000419999811875083\\
29	0.000419989891493038\\
30	0.000419979793658882\\
31	0.000419969515202706\\
32	0.00041995905289807\\
33	0.000419948403460986\\
34	0.000419937563548894\\
35	0.000419926529759596\\
36	0.000419915298630228\\
37	0.000419903866636155\\
38	0.000419892230189896\\
39	0.000419880385639962\\
40	0.00041986832926977\\
41	0.000419856057296442\\
42	0.000419843565869626\\
43	0.000419830851070321\\
44	0.000419817908909638\\
45	0.000419804735327548\\
46	0.000419791326191633\\
47	0.000419777677295765\\
48	0.000419763784358834\\
49	0.000419749643023389\\
50	0.000419735248854272\\
51	0.000419720597337248\\
52	0.000419705683877593\\
53	0.000419690503798645\\
54	0.000419675052340375\\
55	0.000419659324657884\\
56	0.000419643315819889\\
57	0.000419627020807169\\
58	0.00041961043451104\\
59	0.00041959355173171\\
60	0.000419576367176722\\
61	0.000419558875459217\\
62	0.00041954107109633\\
63	0.000419522948507448\\
64	0.000419504502012458\\
65	0.000419485725829991\\
66	0.000419466614075598\\
67	0.000419447160759941\\
68	0.0004194273597869\\
69	0.000419407204951699\\
70	0.000419386689938925\\
71	0.000419365808320598\\
72	0.000419344553554127\\
73	0.000419322918980322\\
74	0.000419300897821263\\
75	0.000419278483178203\\
76	0.000419255668029415\\
77	0.000419232445228003\\
78	0.000419208807499666\\
79	0.00041918474744043\\
80	0.000419160257514329\\
81	0.000419135330051058\\
82	0.000419109957243569\\
83	0.000419084131145653\\
84	0.000419057843669439\\
85	0.000419031086582873\\
86	0.000419003851507143\\
87	0.000418976129914079\\
88	0.000418947913123461\\
89	0.000418919192300326\\
90	0.000418889958452197\\
91	0.000418860202426298\\
92	0.000418829914906654\\
93	0.000418799086411201\\
94	0.000418767707288828\\
95	0.00041873576771634\\
96	0.000418703257695417\\
97	0.000418670167049447\\
98	0.00041863648542038\\
99	0.000418602202265472\\
100	0.000418567306853973\\
101	0.000418531788263796\\
102	0.0004184956353781\\
103	0.000418458836881776\\
104	0.000418421381257942\\
105	0.000418383256784331\\
106	0.000418344451529609\\
107	0.000418304953349629\\
108	0.000418264749883661\\
109	0.000418223828550504\\
110	0.000418182176544531\\
111	0.000418139780831689\\
112	0.000418096628145412\\
113	0.000418052704982474\\
114	0.000418007997598726\\
115	0.000417962492004816\\
116	0.000417916173961774\\
117	0.000417869028976577\\
118	0.000417821042297561\\
119	0.000417772198909828\\
120	0.0004177224835305\\
121	0.000417671880603947\\
122	0.000417620374296881\\
123	0.000417567948493395\\
124	0.000417514586789881\\
125	0.000417460272489879\\
126	0.000417404988598841\\
127	0.00041734871781878\\
128	0.000417291442542798\\
129	0.000417233144849582\\
130	0.000417173806497729\\
131	0.00041711340892004\\
132	0.00041705193321764\\
133	0.00041698936015401\\
134	0.000416925670148956\\
135	0.000416860843272422\\
136	0.000416794859238187\\
137	0.000416727697397464\\
138	0.000416659336732411\\
139	0.000416589755849435\\
140	0.000416518932972488\\
141	0.000416446845936148\\
142	0.000416373472178619\\
143	0.000416298788734591\\
144	0.000416222772227961\\
145	0.00041614539886445\\
146	0.000416066644424035\\
147	0.000415986484253294\\
148	0.000415904893257578\\
149	0.000415821845893031\\
150	0.000415737316158512\\
151	0.000415651277587309\\
152	0.000415563703238727\\
153	0.000415474565689533\\
154	0.00041538383702522\\
155	0.000415291488831118\\
156	0.000415197492183353\\
157	0.000415101817639604\\
158	0.000415004435229723\\
159	0.000414905314446171\\
160	0.000414804424234272\\
161	0.000414701732982281\\
162	0.000414597208511289\\
163	0.000414490818064882\\
164	0.0004143825282987\\
165	0.000414272305269728\\
166	0.00041416011442539\\
167	0.000414045920592484\\
168	0.000413929687965865\\
169	0.000413811380096935\\
170	0.000413690959881913\\
171	0.000413568389549891\\
172	0.00041344363065066\\
173	0.000413316644042281\\
174	0.000413187389878481\\
175	0.000413055827595743\\
176	0.00041292191590022\\
177	0.000412785612754324\\
178	0.00041264687536313\\
179	0.000412505660160458\\
180	0.000412361922794757\\
181	0.000412215618114693\\
182	0.000412066700154405\\
183	0.000411915122118589\\
184	0.000411760836367182\\
185	0.000411603794399866\\
186	0.000411443946840139\\
187	0.000411281243419234\\
188	0.000411115632959598\\
189	0.000410947063358121\\
190	0.000410775481569032\\
191	0.000410600833586431\\
192	0.000410423064426531\\
193	0.000410242118109514\\
194	0.000410057937641047\\
195	0.000409870464993467\\
196	0.000409679641086545\\
197	0.000409485405767922\\
198	0.000409287697793106\\
199	0.000409086454805152\\
200	0.000408881613313874\\
201	0.000408673108674678\\
202	0.000408460875066989\\
203	0.00040824484547221\\
204	0.000408024951651287\\
205	0.000407801124121793\\
206	0.000407573292134596\\
207	0.000407341383650007\\
208	0.000407105325313521\\
209	0.000406865042430991\\
210	0.00040662045894337\\
211	0.000406371497400927\\
212	0.000406118078936918\\
213	0.000405860123240738\\
214	0.000405597548530553\\
215	0.000405330271525351\\
216	0.000405058207416416\\
217	0.000404781269838246\\
218	0.000404499370838857\\
219	0.000404212420849496\\
220	0.000403920328653705\\
221	0.000403623001355766\\
222	0.000403320344348494\\
223	0.000403012261280353\\
224	0.000402698654021902\\
225	0.000402379422631531\\
226	0.000402054465320472\\
227	0.000401723678417106\\
228	0.000401386956330495\\
229	0.000401044191513181\\
230	0.00040069527442313\\
231	0.000400340093484939\\
232	0.000399978535050196\\
233	0.000399610483356988\\
234	0.000399235820488535\\
235	0.00039885442633097\\
236	0.000398466178530193\\
237	0.000398070952447773\\
238	0.00039766862111594\\
239	0.000397259055191547\\
240	0.000396842122909098\\
241	0.000396417690032664\\
242	0.000395985619806851\\
243	0.00039554577290657\\
244	0.000395098007385799\\
245	0.00039464217862516\\
246	0.000394178139278344\\
247	0.000393705739217331\\
248	0.000393224825476414\\
249	0.000392735242194934\\
250	0.000392236830558735\\
251	0.00039172942874031\\
252	0.000391212871837579\\
253	0.000390686991811258\\
254	0.000390151617420819\\
255	0.000389606574158995\\
256	0.000389051684184712\\
257	0.000388486766254549\\
258	0.000387911635652571\\
259	0.000387326104118527\\
260	0.000386729979774372\\
261	0.000386123067049086\\
262	0.000385505166601701\\
263	0.000384876075242515\\
264	0.000384235585852435\\
265	0.000383583487300444\\
266	0.000382919564359009\\
267	0.000382243597617537\\
268	0.000381555363393725\\
269	0.000380854633642751\\
270	0.000380141175864319\\
271	0.000379414753007368\\
272	0.000378675123372514\\
273	0.000377922040512071\\
274	0.000377155253127655\\
275	0.000376374504965213\\
276	0.000375579534707496\\
277	0.000374770075863866\\
278	0.000373945856657327\\
279	0.000373106599908788\\
280	0.000372252022918408\\
281	0.000371381837343941\\
282	0.000370495749076078\\
283	0.000369593458110635\\
284	0.000368674658417494\\
285	0.000367739037806256\\
286	0.000366786277788477\\
287	0.000365816053436458\\
288	0.000364828033238396\\
289	0.000363821878949912\\
290	0.000362797245441791\\
291	0.000361753780543842\\
292	0.000360691124884887\\
293	0.000359608911728576\\
294	0.000358506766805149\\
295	0.000357384308138916\\
296	0.000356241145871404\\
297	0.000355076882080079\\
298	0.000353891110592532\\
299	0.000352683416796083\\
300	0.000351453377442623\\
301	0.000350200560448717\\
302	0.00034892452469083\\
303	0.000347624819795581\\
304	0.000346300985925042\\
305	0.000344952553556883\\
306	0.000343579043259457\\
307	0.00034217996546167\\
308	0.000340754820217642\\
309	0.000339303096966149\\
310	0.000337824274284814\\
311	0.000336317819639054\\
312	0.000334783189125854\\
313	0.00033321982721231\\
314	0.000331627166469109\\
315	0.000330004627299008\\
316	0.0003283516176604\\
317	0.000326667532786106\\
318	0.000324951754897629\\
319	0.000323203652915015\\
320	0.000321422582162619\\
321	0.000319607884071005\\
322	0.000317758885875354\\
323	0.000315874900310746\\
324	0.000313955225304691\\
325	0.00031199914366742\\
326	0.000310005922780365\\
327	0.000307974814283448\\
328	0.000305905053761702\\
329	0.000303795860431847\\
330	0.000301646436829404\\
331	0.000299455968497052\\
332	0.000297223623674725\\
333	0.000294948552992024\\
334	0.000292629889163423\\
335	0.000290266746686432\\
336	0.000287858221542931\\
337	0.000285403390903285\\
338	0.000282901312832588\\
339	0.00028035102599768\\
340	0.000277751549372744\\
341	0.000275101881939878\\
342	0.000272401002379806\\
343	0.000269647868747095\\
344	0.000266841418124055\\
345	0.000263980566248142\\
346	0.000261064207119875\\
347	0.000258091212589031\\
348	0.000255060431917373\\
349	0.000251970691317588\\
350	0.000248820793470033\\
351	0.000245609517022436\\
352	0.000242335616083149\\
353	0.000238997819725432\\
354	0.000235594831523735\\
355	0.000232125329139045\\
356	0.000228587963984056\\
357	0.00022498136090938\\
358	0.000221304117908507\\
359	0.00021755480584361\\
360	0.000213731968194677\\
361	0.000209834120834403\\
362	0.0002058597518315\\
363	0.000201807321285005\\
364	0.000197675261192476\\
365	0.000193461975354694\\
366	0.000189165839319934\\
367	0.000184785200370424\\
368	0.000180318377553957\\
369	0.000175763661763264\\
370	0.000171119315865822\\
371	0.00016638357488653\\
372	0.00016155464624531\\
373	0.000156630710051668\\
374	0.000151609919457554\\
375	0.000146490401069541\\
376	0.000141270255420753\\
377	0.000135947557502195\\
378	0.000130520357352411\\
379	0.000124986680703435\\
380	0.000119344529679791\\
381	0.000113591883545952\\
382	0.000107726699495406\\
383	0.000101746913470269\\
384	9.56504409906719e-05\\
385	8.94351779450992e-05\\
386	8.30990012107413e-05\\
387	7.66397687185278e-05\\
388	7.00553177826701e-05\\
389	6.33434579958076e-05\\
390	5.65019469337396e-05\\
391	4.95284108677339e-05\\
392	4.24200875911155e-05\\
393	3.51729873734826e-05\\
394	2.77791287649591e-05\\
395	2.02173300938862e-05\\
396	1.24221703714644e-05\\
397	4.17810039301993e-06\\
398	0\\
399	0\\
400	0\\
401	0\\
402	0\\
403	0\\
404	0\\
405	0\\
406	0\\
407	0\\
408	0\\
409	0\\
410	0\\
411	0\\
412	0\\
413	0\\
414	0\\
415	0\\
416	0\\
417	0\\
418	0\\
419	0\\
420	0\\
421	0\\
422	0\\
423	0\\
424	0\\
425	0\\
426	0\\
427	0\\
428	0\\
429	0\\
430	0\\
431	0\\
432	0\\
433	0\\
434	0\\
435	0\\
436	0\\
437	0\\
438	0\\
439	0\\
440	0\\
441	0\\
442	0\\
443	0\\
444	0\\
445	0\\
446	0\\
447	0\\
448	0\\
449	0\\
450	0\\
451	0\\
452	0\\
453	0\\
454	0\\
455	0\\
456	0\\
457	0\\
458	0\\
459	0\\
460	0\\
461	0\\
462	0\\
463	0\\
464	0\\
465	0\\
466	0\\
467	0\\
468	0\\
469	0\\
470	0\\
471	0\\
472	0\\
473	0\\
474	0\\
475	0\\
476	0\\
477	0\\
478	0\\
479	0\\
480	0\\
481	0\\
482	0\\
483	0\\
484	0\\
485	0\\
486	0\\
487	0\\
488	0\\
489	0\\
490	0\\
491	0\\
492	0\\
493	0\\
494	0\\
495	0\\
496	0\\
497	0\\
498	0\\
499	0\\
500	0\\
501	0\\
502	0\\
503	0\\
504	0\\
505	0\\
506	0\\
507	0\\
508	0\\
509	0\\
510	0\\
511	0\\
512	0\\
513	0\\
514	0\\
515	0\\
516	0\\
517	0\\
518	0\\
519	0\\
520	0\\
521	0\\
522	0\\
523	0\\
524	0\\
525	0\\
526	0\\
527	0\\
528	0\\
529	0\\
530	0\\
531	0\\
532	0\\
533	0\\
534	0\\
535	0\\
536	0\\
537	0\\
538	0\\
539	0\\
540	0\\
541	0\\
542	0\\
543	0\\
544	0\\
545	0\\
546	0\\
547	0\\
548	0\\
549	0\\
550	0\\
551	0\\
552	0\\
553	0\\
554	0\\
555	0\\
556	0\\
557	0\\
558	0\\
559	0\\
560	0\\
561	0\\
562	0\\
563	0\\
564	0\\
565	0\\
566	0\\
567	0\\
568	0\\
569	0\\
570	0\\
571	0\\
572	0\\
573	0\\
574	0\\
575	0\\
576	0\\
577	0\\
578	0\\
579	0\\
580	0\\
581	0\\
582	0\\
583	0\\
584	0\\
585	0\\
586	0\\
587	0\\
588	0\\
589	0\\
590	0\\
591	0\\
592	0\\
593	0\\
594	0\\
595	0\\
596	0\\
597	0\\
598	0\\
599	0\\
600	0\\
};
\addplot [color=red!50!mycolor17,solid,forget plot]
  table[row sep=crcr]{%
1	0\\
2	0\\
3	0\\
4	0\\
5	0\\
6	0\\
7	0\\
8	0\\
9	0\\
10	0\\
11	0\\
12	0\\
13	0\\
14	0\\
15	0\\
16	0\\
17	0\\
18	0\\
19	0\\
20	0\\
21	0\\
22	0\\
23	0\\
24	0\\
25	0\\
26	0\\
27	0\\
28	0\\
29	0\\
30	0\\
31	0\\
32	0\\
33	0\\
34	0\\
35	0\\
36	0\\
37	0\\
38	0\\
39	0\\
40	0\\
41	0\\
42	0\\
43	0\\
44	0\\
45	0\\
46	0\\
47	0\\
48	0\\
49	0\\
50	0\\
51	0\\
52	0\\
53	0\\
54	0\\
55	0\\
56	0\\
57	0\\
58	0\\
59	0\\
60	0\\
61	0\\
62	0\\
63	0\\
64	0\\
65	0\\
66	0\\
67	0\\
68	0\\
69	0\\
70	0\\
71	0\\
72	0\\
73	0\\
74	0\\
75	0\\
76	0\\
77	0\\
78	0\\
79	0\\
80	0\\
81	0\\
82	0\\
83	0\\
84	0\\
85	0\\
86	0\\
87	0\\
88	0\\
89	0\\
90	0\\
91	0\\
92	0\\
93	0\\
94	0\\
95	0\\
96	0\\
97	0\\
98	0\\
99	0\\
100	0\\
101	0\\
102	0\\
103	0\\
104	0\\
105	0\\
106	0\\
107	0\\
108	0\\
109	0\\
110	0\\
111	0\\
112	0\\
113	0\\
114	0\\
115	0\\
116	0\\
117	0\\
118	0\\
119	0\\
120	0\\
121	0\\
122	0\\
123	0\\
124	0\\
125	0\\
126	0\\
127	0\\
128	0\\
129	0\\
130	0\\
131	0\\
132	0\\
133	0\\
134	0\\
135	0\\
136	0\\
137	0\\
138	0\\
139	0\\
140	0\\
141	0\\
142	0\\
143	0\\
144	0\\
145	0\\
146	0\\
147	0\\
148	0\\
149	0\\
150	0\\
151	0\\
152	0\\
153	0\\
154	0\\
155	0\\
156	0\\
157	0\\
158	0\\
159	0\\
160	0\\
161	0\\
162	0\\
163	0\\
164	0\\
165	0\\
166	0\\
167	0\\
168	0\\
169	0\\
170	0\\
171	0\\
172	0\\
173	0\\
174	0\\
175	0\\
176	0\\
177	0\\
178	0\\
179	0\\
180	0\\
181	0\\
182	0\\
183	0\\
184	0\\
185	0\\
186	0\\
187	0\\
188	0\\
189	0\\
190	0\\
191	0\\
192	0\\
193	0\\
194	0\\
195	0\\
196	0\\
197	0\\
198	0\\
199	0\\
200	0\\
201	0\\
202	0\\
203	0\\
204	0\\
205	0\\
206	0\\
207	0\\
208	0\\
209	0\\
210	0\\
211	0\\
212	0\\
213	0\\
214	0\\
215	0\\
216	0\\
217	0\\
218	0\\
219	0\\
220	0\\
221	0\\
222	0\\
223	0\\
224	0\\
225	0\\
226	0\\
227	0\\
228	0\\
229	0\\
230	0\\
231	0\\
232	0\\
233	0\\
234	0\\
235	0\\
236	0\\
237	0\\
238	0\\
239	0\\
240	0\\
241	0\\
242	0\\
243	0\\
244	0\\
245	0\\
246	0\\
247	0\\
248	0\\
249	0\\
250	0\\
251	0\\
252	0\\
253	0\\
254	0\\
255	0\\
256	0\\
257	0\\
258	0\\
259	0\\
260	0\\
261	0\\
262	0\\
263	0\\
264	0\\
265	0\\
266	0\\
267	0\\
268	0\\
269	0\\
270	0\\
271	0\\
272	0\\
273	0\\
274	0\\
275	0\\
276	0\\
277	0\\
278	0\\
279	0\\
280	0\\
281	0\\
282	0\\
283	0\\
284	0\\
285	0\\
286	0\\
287	0\\
288	0\\
289	0\\
290	0\\
291	0\\
292	0\\
293	0\\
294	0\\
295	0\\
296	0\\
297	0\\
298	0\\
299	0\\
300	0\\
301	0\\
302	0\\
303	0\\
304	0\\
305	0\\
306	0\\
307	0\\
308	0\\
309	0\\
310	0\\
311	0\\
312	0\\
313	0\\
314	0\\
315	0\\
316	0\\
317	0\\
318	0\\
319	0\\
320	0\\
321	0\\
322	0\\
323	0\\
324	0\\
325	0\\
326	0\\
327	0\\
328	0\\
329	0\\
330	0\\
331	0\\
332	0\\
333	0\\
334	0\\
335	0\\
336	0\\
337	0\\
338	0\\
339	0\\
340	0\\
341	0\\
342	0\\
343	0\\
344	0\\
345	0\\
346	0\\
347	0\\
348	0\\
349	0\\
350	0\\
351	0\\
352	0\\
353	0\\
354	0\\
355	0\\
356	0\\
357	0\\
358	0\\
359	0\\
360	0\\
361	0\\
362	0\\
363	0\\
364	0\\
365	0\\
366	0\\
367	0\\
368	0\\
369	0\\
370	0\\
371	0\\
372	0\\
373	0\\
374	0\\
375	0\\
376	0\\
377	0\\
378	0\\
379	0\\
380	0\\
381	0\\
382	0\\
383	0\\
384	0\\
385	0\\
386	0\\
387	0\\
388	0\\
389	0\\
390	0\\
391	0\\
392	0\\
393	0\\
394	0\\
395	0\\
396	0\\
397	0\\
398	0\\
399	0\\
400	0\\
401	0\\
402	0\\
403	0\\
404	0\\
405	0\\
406	0\\
407	0\\
408	0\\
409	0\\
410	0\\
411	0\\
412	0\\
413	0\\
414	0\\
415	0\\
416	0\\
417	0\\
418	0\\
419	0\\
420	0\\
421	0\\
422	0\\
423	0\\
424	0\\
425	0\\
426	0\\
427	0\\
428	0\\
429	0\\
430	0\\
431	0\\
432	0\\
433	0\\
434	0\\
435	0\\
436	0\\
437	0\\
438	0\\
439	0\\
440	0\\
441	0\\
442	0\\
443	0\\
444	0\\
445	0\\
446	0\\
447	0\\
448	0\\
449	0\\
450	0\\
451	0\\
452	0\\
453	0\\
454	0\\
455	0\\
456	0\\
457	0\\
458	0\\
459	0\\
460	0\\
461	0\\
462	0\\
463	0\\
464	0\\
465	0\\
466	0\\
467	0\\
468	0\\
469	0\\
470	0\\
471	0\\
472	0\\
473	0\\
474	0\\
475	0\\
476	0\\
477	0\\
478	0\\
479	0\\
480	0\\
481	0\\
482	0\\
483	0\\
484	0\\
485	0\\
486	0\\
487	0\\
488	0\\
489	0\\
490	0\\
491	0\\
492	0\\
493	0\\
494	0\\
495	0\\
496	0\\
497	0\\
498	0\\
499	0\\
500	0\\
501	0\\
502	0\\
503	0\\
504	0\\
505	0\\
506	0\\
507	0\\
508	0\\
509	0\\
510	0\\
511	0\\
512	0\\
513	0\\
514	0\\
515	0\\
516	0\\
517	0\\
518	0\\
519	0\\
520	0\\
521	0\\
522	0\\
523	0\\
524	0\\
525	0\\
526	0\\
527	0\\
528	0\\
529	0\\
530	0\\
531	0\\
532	0\\
533	0\\
534	0\\
535	0\\
536	0\\
537	0\\
538	0\\
539	0\\
540	0\\
541	0\\
542	0\\
543	0\\
544	0\\
545	0\\
546	0\\
547	0\\
548	0\\
549	0\\
550	0\\
551	0\\
552	0\\
553	0\\
554	0\\
555	0\\
556	0\\
557	0\\
558	0\\
559	0\\
560	0\\
561	0\\
562	0\\
563	0\\
564	0\\
565	0\\
566	0\\
567	0\\
568	0\\
569	0\\
570	0\\
571	0\\
572	0\\
573	0\\
574	0\\
575	0\\
576	0\\
577	0\\
578	0\\
579	0\\
580	0\\
581	0\\
582	0\\
583	0\\
584	0\\
585	0\\
586	0\\
587	0\\
588	0\\
589	0\\
590	0\\
591	0\\
592	0\\
593	0\\
594	0\\
595	0\\
596	0\\
597	0\\
598	0\\
599	0\\
600	0\\
};
\addplot [color=red!40!mycolor19,solid,forget plot]
  table[row sep=crcr]{%
1	0\\
2	0\\
3	0\\
4	0\\
5	0\\
6	0\\
7	0\\
8	0\\
9	0\\
10	0\\
11	0\\
12	0\\
13	0\\
14	0\\
15	0\\
16	0\\
17	0\\
18	0\\
19	0\\
20	0\\
21	0\\
22	0\\
23	0\\
24	0\\
25	0\\
26	0\\
27	0\\
28	0\\
29	0\\
30	0\\
31	0\\
32	0\\
33	0\\
34	0\\
35	0\\
36	0\\
37	0\\
38	0\\
39	0\\
40	0\\
41	0\\
42	0\\
43	0\\
44	0\\
45	0\\
46	0\\
47	0\\
48	0\\
49	0\\
50	0\\
51	0\\
52	0\\
53	0\\
54	0\\
55	0\\
56	0\\
57	0\\
58	0\\
59	0\\
60	0\\
61	0\\
62	0\\
63	0\\
64	0\\
65	0\\
66	0\\
67	0\\
68	0\\
69	0\\
70	0\\
71	0\\
72	0\\
73	0\\
74	0\\
75	0\\
76	0\\
77	0\\
78	0\\
79	0\\
80	0\\
81	0\\
82	0\\
83	0\\
84	0\\
85	0\\
86	0\\
87	0\\
88	0\\
89	0\\
90	0\\
91	0\\
92	0\\
93	0\\
94	0\\
95	0\\
96	0\\
97	0\\
98	0\\
99	0\\
100	0\\
101	0\\
102	0\\
103	0\\
104	0\\
105	0\\
106	0\\
107	0\\
108	0\\
109	0\\
110	0\\
111	0\\
112	0\\
113	0\\
114	0\\
115	0\\
116	0\\
117	0\\
118	0\\
119	0\\
120	0\\
121	0\\
122	0\\
123	0\\
124	0\\
125	0\\
126	0\\
127	0\\
128	0\\
129	0\\
130	0\\
131	0\\
132	0\\
133	0\\
134	0\\
135	0\\
136	0\\
137	0\\
138	0\\
139	0\\
140	0\\
141	0\\
142	0\\
143	0\\
144	0\\
145	0\\
146	0\\
147	0\\
148	0\\
149	0\\
150	0\\
151	0\\
152	0\\
153	0\\
154	0\\
155	0\\
156	0\\
157	0\\
158	0\\
159	0\\
160	0\\
161	0\\
162	0\\
163	0\\
164	0\\
165	0\\
166	0\\
167	0\\
168	0\\
169	0\\
170	0\\
171	0\\
172	0\\
173	0\\
174	0\\
175	0\\
176	0\\
177	0\\
178	0\\
179	0\\
180	0\\
181	0\\
182	0\\
183	0\\
184	0\\
185	0\\
186	0\\
187	0\\
188	0\\
189	0\\
190	0\\
191	0\\
192	0\\
193	0\\
194	0\\
195	0\\
196	0\\
197	0\\
198	0\\
199	0\\
200	0\\
201	0\\
202	0\\
203	0\\
204	0\\
205	0\\
206	0\\
207	0\\
208	0\\
209	0\\
210	0\\
211	0\\
212	0\\
213	0\\
214	0\\
215	0\\
216	0\\
217	0\\
218	0\\
219	0\\
220	0\\
221	0\\
222	0\\
223	0\\
224	0\\
225	0\\
226	0\\
227	0\\
228	0\\
229	0\\
230	0\\
231	0\\
232	0\\
233	0\\
234	0\\
235	0\\
236	0\\
237	0\\
238	0\\
239	0\\
240	0\\
241	0\\
242	0\\
243	0\\
244	0\\
245	0\\
246	0\\
247	0\\
248	0\\
249	0\\
250	0\\
251	0\\
252	0\\
253	0\\
254	0\\
255	0\\
256	0\\
257	0\\
258	0\\
259	0\\
260	0\\
261	0\\
262	0\\
263	0\\
264	0\\
265	0\\
266	0\\
267	0\\
268	0\\
269	0\\
270	0\\
271	0\\
272	0\\
273	0\\
274	0\\
275	0\\
276	0\\
277	0\\
278	0\\
279	0\\
280	0\\
281	0\\
282	0\\
283	0\\
284	0\\
285	0\\
286	0\\
287	0\\
288	0\\
289	0\\
290	0\\
291	0\\
292	0\\
293	0\\
294	0\\
295	0\\
296	0\\
297	0\\
298	0\\
299	0\\
300	0\\
301	0\\
302	0\\
303	0\\
304	0\\
305	0\\
306	0\\
307	0\\
308	0\\
309	0\\
310	0\\
311	0\\
312	0\\
313	0\\
314	0\\
315	0\\
316	0\\
317	0\\
318	0\\
319	0\\
320	0\\
321	0\\
322	0\\
323	0\\
324	0\\
325	0\\
326	0\\
327	0\\
328	0\\
329	0\\
330	0\\
331	0\\
332	0\\
333	0\\
334	0\\
335	0\\
336	0\\
337	0\\
338	0\\
339	0\\
340	0\\
341	0\\
342	0\\
343	0\\
344	0\\
345	0\\
346	0\\
347	0\\
348	0\\
349	0\\
350	0\\
351	0\\
352	0\\
353	0\\
354	0\\
355	0\\
356	0\\
357	0\\
358	0\\
359	0\\
360	0\\
361	0\\
362	0\\
363	0\\
364	0\\
365	0\\
366	0\\
367	0\\
368	0\\
369	0\\
370	0\\
371	0\\
372	0\\
373	0\\
374	0\\
375	0\\
376	0\\
377	0\\
378	0\\
379	0\\
380	0\\
381	0\\
382	0\\
383	0\\
384	0\\
385	0\\
386	0\\
387	0\\
388	0\\
389	0\\
390	0\\
391	0\\
392	0\\
393	0\\
394	0\\
395	0\\
396	0\\
397	0\\
398	0\\
399	0\\
400	0\\
401	0\\
402	0\\
403	0\\
404	0\\
405	0\\
406	0\\
407	0\\
408	0\\
409	0\\
410	0\\
411	0\\
412	0\\
413	0\\
414	0\\
415	0\\
416	0\\
417	0\\
418	0\\
419	0\\
420	0\\
421	0\\
422	0\\
423	0\\
424	0\\
425	0\\
426	0\\
427	0\\
428	0\\
429	0\\
430	0\\
431	0\\
432	0\\
433	0\\
434	0\\
435	0\\
436	0\\
437	0\\
438	0\\
439	0\\
440	0\\
441	0\\
442	0\\
443	0\\
444	0\\
445	0\\
446	0\\
447	0\\
448	0\\
449	0\\
450	0\\
451	0\\
452	0\\
453	0\\
454	0\\
455	0\\
456	0\\
457	0\\
458	0\\
459	0\\
460	0\\
461	0\\
462	0\\
463	0\\
464	0\\
465	0\\
466	0\\
467	0\\
468	0\\
469	0\\
470	0\\
471	0\\
472	0\\
473	0\\
474	0\\
475	0\\
476	0\\
477	0\\
478	0\\
479	0\\
480	0\\
481	0\\
482	0\\
483	0\\
484	0\\
485	0\\
486	0\\
487	0\\
488	0\\
489	0\\
490	0\\
491	0\\
492	0\\
493	0\\
494	0\\
495	0\\
496	0\\
497	0\\
498	0\\
499	0\\
500	0\\
501	0\\
502	0\\
503	0\\
504	0\\
505	0\\
506	0\\
507	0\\
508	0\\
509	0\\
510	0\\
511	0\\
512	0\\
513	0\\
514	0\\
515	0\\
516	0\\
517	0\\
518	0\\
519	0\\
520	0\\
521	0\\
522	0\\
523	0\\
524	0\\
525	0\\
526	0\\
527	0\\
528	0\\
529	0\\
530	0\\
531	0\\
532	0\\
533	0\\
534	0\\
535	0\\
536	0\\
537	0\\
538	0\\
539	0\\
540	0\\
541	0\\
542	0\\
543	0\\
544	0\\
545	0\\
546	0\\
547	0\\
548	0\\
549	0\\
550	0\\
551	0\\
552	0\\
553	0\\
554	0\\
555	0\\
556	0\\
557	0\\
558	0\\
559	0\\
560	0\\
561	0\\
562	0\\
563	0\\
564	0\\
565	0\\
566	0\\
567	0\\
568	0\\
569	0\\
570	0\\
571	0\\
572	0\\
573	0\\
574	0\\
575	0\\
576	0\\
577	0\\
578	0\\
579	0\\
580	0\\
581	0\\
582	0\\
583	0\\
584	0\\
585	0\\
586	0\\
587	0\\
588	0\\
589	0\\
590	0\\
591	0\\
592	0\\
593	0\\
594	0\\
595	0\\
596	0\\
597	0\\
598	0\\
599	0\\
600	0\\
};
\addplot [color=red!75!mycolor17,solid,forget plot]
  table[row sep=crcr]{%
1	0\\
2	0\\
3	0\\
4	0\\
5	0\\
6	0\\
7	0\\
8	0\\
9	0\\
10	0\\
11	0\\
12	0\\
13	0\\
14	0\\
15	0\\
16	0\\
17	0\\
18	0\\
19	0\\
20	0\\
21	0\\
22	0\\
23	0\\
24	0\\
25	0\\
26	0\\
27	0\\
28	0\\
29	0\\
30	0\\
31	0\\
32	0\\
33	0\\
34	0\\
35	0\\
36	0\\
37	0\\
38	0\\
39	0\\
40	0\\
41	0\\
42	0\\
43	0\\
44	0\\
45	0\\
46	0\\
47	0\\
48	0\\
49	0\\
50	0\\
51	0\\
52	0\\
53	0\\
54	0\\
55	0\\
56	0\\
57	0\\
58	0\\
59	0\\
60	0\\
61	0\\
62	0\\
63	0\\
64	0\\
65	0\\
66	0\\
67	0\\
68	0\\
69	0\\
70	0\\
71	0\\
72	0\\
73	0\\
74	0\\
75	0\\
76	0\\
77	0\\
78	0\\
79	0\\
80	0\\
81	0\\
82	0\\
83	0\\
84	0\\
85	0\\
86	0\\
87	0\\
88	0\\
89	0\\
90	0\\
91	0\\
92	0\\
93	0\\
94	0\\
95	0\\
96	0\\
97	0\\
98	0\\
99	0\\
100	0\\
101	0\\
102	0\\
103	0\\
104	0\\
105	0\\
106	0\\
107	0\\
108	0\\
109	0\\
110	0\\
111	0\\
112	0\\
113	0\\
114	0\\
115	0\\
116	0\\
117	0\\
118	0\\
119	0\\
120	0\\
121	0\\
122	0\\
123	0\\
124	0\\
125	0\\
126	0\\
127	0\\
128	0\\
129	0\\
130	0\\
131	0\\
132	0\\
133	0\\
134	0\\
135	0\\
136	0\\
137	0\\
138	0\\
139	0\\
140	0\\
141	0\\
142	0\\
143	0\\
144	0\\
145	0\\
146	0\\
147	0\\
148	0\\
149	0\\
150	0\\
151	0\\
152	0\\
153	0\\
154	0\\
155	0\\
156	0\\
157	0\\
158	0\\
159	0\\
160	0\\
161	0\\
162	0\\
163	0\\
164	0\\
165	0\\
166	0\\
167	0\\
168	0\\
169	0\\
170	0\\
171	0\\
172	0\\
173	0\\
174	0\\
175	0\\
176	0\\
177	0\\
178	0\\
179	0\\
180	0\\
181	0\\
182	0\\
183	0\\
184	0\\
185	0\\
186	0\\
187	0\\
188	0\\
189	0\\
190	0\\
191	0\\
192	0\\
193	0\\
194	0\\
195	0\\
196	0\\
197	0\\
198	0\\
199	0\\
200	0\\
201	0\\
202	0\\
203	0\\
204	0\\
205	0\\
206	0\\
207	0\\
208	0\\
209	0\\
210	0\\
211	0\\
212	0\\
213	0\\
214	0\\
215	0\\
216	0\\
217	0\\
218	0\\
219	0\\
220	0\\
221	0\\
222	0\\
223	0\\
224	0\\
225	0\\
226	0\\
227	0\\
228	0\\
229	0\\
230	0\\
231	0\\
232	0\\
233	0\\
234	0\\
235	0\\
236	0\\
237	0\\
238	0\\
239	0\\
240	0\\
241	0\\
242	0\\
243	0\\
244	0\\
245	0\\
246	0\\
247	0\\
248	0\\
249	0\\
250	0\\
251	0\\
252	0\\
253	0\\
254	0\\
255	0\\
256	0\\
257	0\\
258	0\\
259	0\\
260	0\\
261	0\\
262	0\\
263	0\\
264	0\\
265	0\\
266	0\\
267	0\\
268	0\\
269	0\\
270	0\\
271	0\\
272	0\\
273	0\\
274	0\\
275	0\\
276	0\\
277	0\\
278	0\\
279	0\\
280	0\\
281	0\\
282	0\\
283	0\\
284	0\\
285	0\\
286	0\\
287	0\\
288	0\\
289	0\\
290	0\\
291	0\\
292	0\\
293	0\\
294	0\\
295	0\\
296	0\\
297	0\\
298	0\\
299	0\\
300	0\\
301	0\\
302	0\\
303	0\\
304	0\\
305	0\\
306	0\\
307	0\\
308	0\\
309	0\\
310	0\\
311	0\\
312	0\\
313	0\\
314	0\\
315	0\\
316	0\\
317	0\\
318	0\\
319	0\\
320	0\\
321	0\\
322	0\\
323	0\\
324	0\\
325	0\\
326	0\\
327	0\\
328	0\\
329	0\\
330	0\\
331	0\\
332	0\\
333	0\\
334	0\\
335	0\\
336	0\\
337	0\\
338	0\\
339	0\\
340	0\\
341	0\\
342	0\\
343	0\\
344	0\\
345	0\\
346	0\\
347	0\\
348	0\\
349	0\\
350	0\\
351	0\\
352	0\\
353	0\\
354	0\\
355	0\\
356	0\\
357	0\\
358	0\\
359	0\\
360	0\\
361	0\\
362	0\\
363	0\\
364	0\\
365	0\\
366	0\\
367	0\\
368	0\\
369	0\\
370	0\\
371	0\\
372	0\\
373	0\\
374	0\\
375	0\\
376	0\\
377	0\\
378	0\\
379	0\\
380	0\\
381	0\\
382	0\\
383	0\\
384	0\\
385	0\\
386	0\\
387	0\\
388	0\\
389	0\\
390	0\\
391	0\\
392	0\\
393	0\\
394	0\\
395	0\\
396	0\\
397	0\\
398	0\\
399	0\\
400	0\\
401	0\\
402	0\\
403	0\\
404	0\\
405	0\\
406	0\\
407	0\\
408	0\\
409	0\\
410	0\\
411	0\\
412	0\\
413	0\\
414	0\\
415	0\\
416	0\\
417	0\\
418	0\\
419	0\\
420	0\\
421	0\\
422	0\\
423	0\\
424	0\\
425	0\\
426	0\\
427	0\\
428	0\\
429	0\\
430	0\\
431	0\\
432	0\\
433	0\\
434	0\\
435	0\\
436	0\\
437	0\\
438	0\\
439	0\\
440	0\\
441	0\\
442	0\\
443	0\\
444	0\\
445	0\\
446	0\\
447	0\\
448	0\\
449	0\\
450	0\\
451	0\\
452	0\\
453	0\\
454	0\\
455	0\\
456	0\\
457	0\\
458	0\\
459	0\\
460	0\\
461	0\\
462	0\\
463	0\\
464	0\\
465	0\\
466	0\\
467	0\\
468	0\\
469	0\\
470	0\\
471	0\\
472	0\\
473	0\\
474	0\\
475	0\\
476	0\\
477	0\\
478	0\\
479	0\\
480	0\\
481	0\\
482	0\\
483	0\\
484	0\\
485	0\\
486	0\\
487	0\\
488	0\\
489	0\\
490	0\\
491	0\\
492	0\\
493	0\\
494	0\\
495	0\\
496	0\\
497	0\\
498	0\\
499	0\\
500	0\\
501	0\\
502	0\\
503	0\\
504	0\\
505	0\\
506	0\\
507	0\\
508	0\\
509	0\\
510	0\\
511	0\\
512	0\\
513	0\\
514	0\\
515	0\\
516	0\\
517	0\\
518	0\\
519	0\\
520	0\\
521	0\\
522	0\\
523	0\\
524	0\\
525	0\\
526	0\\
527	0\\
528	0\\
529	0\\
530	0\\
531	0\\
532	0\\
533	0\\
534	0\\
535	0\\
536	0\\
537	0\\
538	0\\
539	0\\
540	0\\
541	0\\
542	0\\
543	0\\
544	0\\
545	0\\
546	0\\
547	0\\
548	0\\
549	0\\
550	0\\
551	0\\
552	0\\
553	0\\
554	0\\
555	0\\
556	0\\
557	0\\
558	0\\
559	0\\
560	0\\
561	0\\
562	0\\
563	0\\
564	0\\
565	0\\
566	0\\
567	0\\
568	0\\
569	0\\
570	0\\
571	0\\
572	0\\
573	0\\
574	0\\
575	0\\
576	0\\
577	0\\
578	0\\
579	0\\
580	0\\
581	0\\
582	0\\
583	0\\
584	0\\
585	0\\
586	0\\
587	0\\
588	0\\
589	0\\
590	0\\
591	0\\
592	0\\
593	0\\
594	0\\
595	0\\
596	0\\
597	0\\
598	0\\
599	0\\
600	0\\
};
\addplot [color=red!80!mycolor19,solid,forget plot]
  table[row sep=crcr]{%
1	0\\
2	0\\
3	0\\
4	0\\
5	0\\
6	0\\
7	0\\
8	0\\
9	0\\
10	0\\
11	0\\
12	0\\
13	0\\
14	0\\
15	0\\
16	0\\
17	0\\
18	0\\
19	0\\
20	0\\
21	0\\
22	0\\
23	0\\
24	0\\
25	0\\
26	0\\
27	0\\
28	0\\
29	0\\
30	0\\
31	0\\
32	0\\
33	0\\
34	0\\
35	0\\
36	0\\
37	0\\
38	0\\
39	0\\
40	0\\
41	0\\
42	0\\
43	0\\
44	0\\
45	0\\
46	0\\
47	0\\
48	0\\
49	0\\
50	0\\
51	0\\
52	0\\
53	0\\
54	0\\
55	0\\
56	0\\
57	0\\
58	0\\
59	0\\
60	0\\
61	0\\
62	0\\
63	0\\
64	0\\
65	0\\
66	0\\
67	0\\
68	0\\
69	0\\
70	0\\
71	0\\
72	0\\
73	0\\
74	0\\
75	0\\
76	0\\
77	0\\
78	0\\
79	0\\
80	0\\
81	0\\
82	0\\
83	0\\
84	0\\
85	0\\
86	0\\
87	0\\
88	0\\
89	0\\
90	0\\
91	0\\
92	0\\
93	0\\
94	0\\
95	0\\
96	0\\
97	0\\
98	0\\
99	0\\
100	0\\
101	0\\
102	0\\
103	0\\
104	0\\
105	0\\
106	0\\
107	0\\
108	0\\
109	0\\
110	0\\
111	0\\
112	0\\
113	0\\
114	0\\
115	0\\
116	0\\
117	0\\
118	0\\
119	0\\
120	0\\
121	0\\
122	0\\
123	0\\
124	0\\
125	0\\
126	0\\
127	0\\
128	0\\
129	0\\
130	0\\
131	0\\
132	0\\
133	0\\
134	0\\
135	0\\
136	0\\
137	0\\
138	0\\
139	0\\
140	0\\
141	0\\
142	0\\
143	0\\
144	0\\
145	0\\
146	0\\
147	0\\
148	0\\
149	0\\
150	0\\
151	0\\
152	0\\
153	0\\
154	0\\
155	0\\
156	0\\
157	0\\
158	0\\
159	0\\
160	0\\
161	0\\
162	0\\
163	0\\
164	0\\
165	0\\
166	0\\
167	0\\
168	0\\
169	0\\
170	0\\
171	0\\
172	0\\
173	0\\
174	0\\
175	0\\
176	0\\
177	0\\
178	0\\
179	0\\
180	0\\
181	0\\
182	0\\
183	0\\
184	0\\
185	0\\
186	0\\
187	0\\
188	0\\
189	0\\
190	0\\
191	0\\
192	0\\
193	0\\
194	0\\
195	0\\
196	0\\
197	0\\
198	0\\
199	0\\
200	0\\
201	0\\
202	0\\
203	0\\
204	0\\
205	0\\
206	0\\
207	0\\
208	0\\
209	0\\
210	0\\
211	0\\
212	0\\
213	0\\
214	0\\
215	0\\
216	0\\
217	0\\
218	0\\
219	0\\
220	0\\
221	0\\
222	0\\
223	0\\
224	0\\
225	0\\
226	0\\
227	0\\
228	0\\
229	0\\
230	0\\
231	0\\
232	0\\
233	0\\
234	0\\
235	0\\
236	0\\
237	0\\
238	0\\
239	0\\
240	0\\
241	0\\
242	0\\
243	0\\
244	0\\
245	0\\
246	0\\
247	0\\
248	0\\
249	0\\
250	0\\
251	0\\
252	0\\
253	0\\
254	0\\
255	0\\
256	0\\
257	0\\
258	0\\
259	0\\
260	0\\
261	0\\
262	0\\
263	0\\
264	0\\
265	0\\
266	0\\
267	0\\
268	0\\
269	0\\
270	0\\
271	0\\
272	0\\
273	0\\
274	0\\
275	0\\
276	0\\
277	0\\
278	0\\
279	0\\
280	0\\
281	0\\
282	0\\
283	0\\
284	0\\
285	0\\
286	0\\
287	0\\
288	0\\
289	0\\
290	0\\
291	0\\
292	0\\
293	0\\
294	0\\
295	0\\
296	0\\
297	0\\
298	0\\
299	0\\
300	0\\
301	0\\
302	0\\
303	0\\
304	0\\
305	0\\
306	0\\
307	0\\
308	0\\
309	0\\
310	0\\
311	0\\
312	0\\
313	0\\
314	0\\
315	0\\
316	0\\
317	0\\
318	0\\
319	0\\
320	0\\
321	0\\
322	0\\
323	0\\
324	0\\
325	0\\
326	0\\
327	0\\
328	0\\
329	0\\
330	0\\
331	0\\
332	0\\
333	0\\
334	0\\
335	0\\
336	0\\
337	0\\
338	0\\
339	0\\
340	0\\
341	0\\
342	0\\
343	0\\
344	0\\
345	0\\
346	0\\
347	0\\
348	0\\
349	0\\
350	0\\
351	0\\
352	0\\
353	0\\
354	0\\
355	0\\
356	0\\
357	0\\
358	0\\
359	0\\
360	0\\
361	0\\
362	0\\
363	0\\
364	0\\
365	0\\
366	0\\
367	0\\
368	0\\
369	0\\
370	0\\
371	0\\
372	0\\
373	0\\
374	0\\
375	0\\
376	0\\
377	0\\
378	0\\
379	0\\
380	0\\
381	0\\
382	0\\
383	0\\
384	0\\
385	0\\
386	0\\
387	0\\
388	0\\
389	0\\
390	0\\
391	0\\
392	0\\
393	0\\
394	0\\
395	0\\
396	0\\
397	0\\
398	0\\
399	0\\
400	0\\
401	0\\
402	0\\
403	0\\
404	0\\
405	0\\
406	0\\
407	0\\
408	0\\
409	0\\
410	0\\
411	0\\
412	0\\
413	0\\
414	0\\
415	0\\
416	0\\
417	0\\
418	0\\
419	0\\
420	0\\
421	0\\
422	0\\
423	0\\
424	0\\
425	0\\
426	0\\
427	0\\
428	0\\
429	0\\
430	0\\
431	0\\
432	0\\
433	0\\
434	0\\
435	0\\
436	0\\
437	0\\
438	0\\
439	0\\
440	0\\
441	0\\
442	0\\
443	0\\
444	0\\
445	0\\
446	0\\
447	0\\
448	0\\
449	0\\
450	0\\
451	0\\
452	0\\
453	0\\
454	0\\
455	0\\
456	0\\
457	0\\
458	0\\
459	0\\
460	0\\
461	0\\
462	0\\
463	0\\
464	0\\
465	0\\
466	0\\
467	0\\
468	0\\
469	0\\
470	0\\
471	0\\
472	0\\
473	0\\
474	0\\
475	0\\
476	0\\
477	0\\
478	0\\
479	0\\
480	0\\
481	0\\
482	0\\
483	0\\
484	0\\
485	0\\
486	0\\
487	0\\
488	0\\
489	0\\
490	0\\
491	0\\
492	0\\
493	0\\
494	0\\
495	0\\
496	0\\
497	0\\
498	0\\
499	0\\
500	0\\
501	0\\
502	0\\
503	0\\
504	0\\
505	0\\
506	0\\
507	0\\
508	0\\
509	0\\
510	0\\
511	0\\
512	0\\
513	0\\
514	0\\
515	0\\
516	0\\
517	0\\
518	0\\
519	0\\
520	0\\
521	0\\
522	0\\
523	0\\
524	0\\
525	0\\
526	0\\
527	0\\
528	0\\
529	0\\
530	0\\
531	0\\
532	0\\
533	0\\
534	0\\
535	0\\
536	0\\
537	0\\
538	0\\
539	0\\
540	0\\
541	0\\
542	0\\
543	0\\
544	0\\
545	0\\
546	0\\
547	0\\
548	0\\
549	0\\
550	0\\
551	0\\
552	0\\
553	0\\
554	0\\
555	0\\
556	0\\
557	0\\
558	0\\
559	0\\
560	0\\
561	0\\
562	0\\
563	0\\
564	0\\
565	0\\
566	0\\
567	0\\
568	0\\
569	0\\
570	0\\
571	0\\
572	0\\
573	0\\
574	0\\
575	0\\
576	0\\
577	0\\
578	0\\
579	0\\
580	0\\
581	0\\
582	0\\
583	0\\
584	0\\
585	0\\
586	0\\
587	0\\
588	0\\
589	0\\
590	0\\
591	0\\
592	0\\
593	0\\
594	0\\
595	0\\
596	0\\
597	0\\
598	0\\
599	0\\
600	0\\
};
\addplot [color=red,solid,forget plot]
  table[row sep=crcr]{%
1	0\\
2	0\\
3	0\\
4	0\\
5	0\\
6	0\\
7	0\\
8	0\\
9	0\\
10	0\\
11	0\\
12	0\\
13	0\\
14	0\\
15	0\\
16	0\\
17	0\\
18	0\\
19	0\\
20	0\\
21	0\\
22	0\\
23	0\\
24	0\\
25	0\\
26	0\\
27	0\\
28	0\\
29	0\\
30	0\\
31	0\\
32	0\\
33	0\\
34	0\\
35	0\\
36	0\\
37	0\\
38	0\\
39	0\\
40	0\\
41	0\\
42	0\\
43	0\\
44	0\\
45	0\\
46	0\\
47	0\\
48	0\\
49	0\\
50	0\\
51	0\\
52	0\\
53	0\\
54	0\\
55	0\\
56	0\\
57	0\\
58	0\\
59	0\\
60	0\\
61	0\\
62	0\\
63	0\\
64	0\\
65	0\\
66	0\\
67	0\\
68	0\\
69	0\\
70	0\\
71	0\\
72	0\\
73	0\\
74	0\\
75	0\\
76	0\\
77	0\\
78	0\\
79	0\\
80	0\\
81	0\\
82	0\\
83	0\\
84	0\\
85	0\\
86	0\\
87	0\\
88	0\\
89	0\\
90	0\\
91	0\\
92	0\\
93	0\\
94	0\\
95	0\\
96	0\\
97	0\\
98	0\\
99	0\\
100	0\\
101	0\\
102	0\\
103	0\\
104	0\\
105	0\\
106	0\\
107	0\\
108	0\\
109	0\\
110	0\\
111	0\\
112	0\\
113	0\\
114	0\\
115	0\\
116	0\\
117	0\\
118	0\\
119	0\\
120	0\\
121	0\\
122	0\\
123	0\\
124	0\\
125	0\\
126	0\\
127	0\\
128	0\\
129	0\\
130	0\\
131	0\\
132	0\\
133	0\\
134	0\\
135	0\\
136	0\\
137	0\\
138	0\\
139	0\\
140	0\\
141	0\\
142	0\\
143	0\\
144	0\\
145	0\\
146	0\\
147	0\\
148	0\\
149	0\\
150	0\\
151	0\\
152	0\\
153	0\\
154	0\\
155	0\\
156	0\\
157	0\\
158	0\\
159	0\\
160	0\\
161	0\\
162	0\\
163	0\\
164	0\\
165	0\\
166	0\\
167	0\\
168	0\\
169	0\\
170	0\\
171	0\\
172	0\\
173	0\\
174	0\\
175	0\\
176	0\\
177	0\\
178	0\\
179	0\\
180	0\\
181	0\\
182	0\\
183	0\\
184	0\\
185	0\\
186	0\\
187	0\\
188	0\\
189	0\\
190	0\\
191	0\\
192	0\\
193	0\\
194	0\\
195	0\\
196	0\\
197	0\\
198	0\\
199	0\\
200	0\\
201	0\\
202	0\\
203	0\\
204	0\\
205	0\\
206	0\\
207	0\\
208	0\\
209	0\\
210	0\\
211	0\\
212	0\\
213	0\\
214	0\\
215	0\\
216	0\\
217	0\\
218	0\\
219	0\\
220	0\\
221	0\\
222	0\\
223	0\\
224	0\\
225	0\\
226	0\\
227	0\\
228	0\\
229	0\\
230	0\\
231	0\\
232	0\\
233	0\\
234	0\\
235	0\\
236	0\\
237	0\\
238	0\\
239	0\\
240	0\\
241	0\\
242	0\\
243	0\\
244	0\\
245	0\\
246	0\\
247	0\\
248	0\\
249	0\\
250	0\\
251	0\\
252	0\\
253	0\\
254	0\\
255	0\\
256	0\\
257	0\\
258	0\\
259	0\\
260	0\\
261	0\\
262	0\\
263	0\\
264	0\\
265	0\\
266	0\\
267	0\\
268	0\\
269	0\\
270	0\\
271	0\\
272	0\\
273	0\\
274	0\\
275	0\\
276	0\\
277	0\\
278	0\\
279	0\\
280	0\\
281	0\\
282	0\\
283	0\\
284	0\\
285	0\\
286	0\\
287	0\\
288	0\\
289	0\\
290	0\\
291	0\\
292	0\\
293	0\\
294	0\\
295	0\\
296	0\\
297	0\\
298	0\\
299	0\\
300	0\\
301	0\\
302	0\\
303	0\\
304	0\\
305	0\\
306	0\\
307	0\\
308	0\\
309	0\\
310	0\\
311	0\\
312	0\\
313	0\\
314	0\\
315	0\\
316	0\\
317	0\\
318	0\\
319	0\\
320	0\\
321	0\\
322	0\\
323	0\\
324	0\\
325	0\\
326	0\\
327	0\\
328	0\\
329	0\\
330	0\\
331	0\\
332	0\\
333	0\\
334	0\\
335	0\\
336	0\\
337	0\\
338	0\\
339	0\\
340	0\\
341	0\\
342	0\\
343	0\\
344	0\\
345	0\\
346	0\\
347	0\\
348	0\\
349	0\\
350	0\\
351	0\\
352	0\\
353	0\\
354	0\\
355	0\\
356	0\\
357	0\\
358	0\\
359	0\\
360	0\\
361	0\\
362	0\\
363	0\\
364	0\\
365	0\\
366	0\\
367	0\\
368	0\\
369	0\\
370	0\\
371	0\\
372	0\\
373	0\\
374	0\\
375	0\\
376	0\\
377	0\\
378	0\\
379	0\\
380	0\\
381	0\\
382	0\\
383	0\\
384	0\\
385	0\\
386	0\\
387	0\\
388	0\\
389	0\\
390	0\\
391	0\\
392	0\\
393	0\\
394	0\\
395	0\\
396	0\\
397	0\\
398	0\\
399	0\\
400	0\\
401	0\\
402	0\\
403	0\\
404	0\\
405	0\\
406	0\\
407	0\\
408	0\\
409	0\\
410	0\\
411	0\\
412	0\\
413	0\\
414	0\\
415	0\\
416	0\\
417	0\\
418	0\\
419	0\\
420	0\\
421	0\\
422	0\\
423	0\\
424	0\\
425	0\\
426	0\\
427	0\\
428	0\\
429	0\\
430	0\\
431	0\\
432	0\\
433	0\\
434	0\\
435	0\\
436	0\\
437	0\\
438	0\\
439	0\\
440	0\\
441	0\\
442	0\\
443	0\\
444	0\\
445	0\\
446	0\\
447	0\\
448	0\\
449	0\\
450	0\\
451	0\\
452	0\\
453	0\\
454	0\\
455	0\\
456	0\\
457	0\\
458	0\\
459	0\\
460	0\\
461	0\\
462	0\\
463	0\\
464	0\\
465	0\\
466	0\\
467	0\\
468	0\\
469	0\\
470	0\\
471	0\\
472	0\\
473	0\\
474	0\\
475	0\\
476	0\\
477	0\\
478	0\\
479	0\\
480	0\\
481	0\\
482	0\\
483	0\\
484	0\\
485	0\\
486	0\\
487	0\\
488	0\\
489	0\\
490	0\\
491	0\\
492	0\\
493	0\\
494	0\\
495	0\\
496	0\\
497	0\\
498	0\\
499	0\\
500	0\\
501	0\\
502	0\\
503	0\\
504	0\\
505	0\\
506	0\\
507	0\\
508	0\\
509	0\\
510	0\\
511	0\\
512	0\\
513	0\\
514	0\\
515	0\\
516	0\\
517	0\\
518	0\\
519	0\\
520	0\\
521	0\\
522	0\\
523	0\\
524	0\\
525	0\\
526	0\\
527	0\\
528	0\\
529	0\\
530	0\\
531	0\\
532	0\\
533	0\\
534	0\\
535	0\\
536	0\\
537	0\\
538	0\\
539	0\\
540	0\\
541	0\\
542	0\\
543	0\\
544	0\\
545	0\\
546	0\\
547	0\\
548	0\\
549	0\\
550	0\\
551	0\\
552	0\\
553	0\\
554	0\\
555	0\\
556	0\\
557	0\\
558	0\\
559	0\\
560	0\\
561	0\\
562	0\\
563	0\\
564	0\\
565	0\\
566	0\\
567	0\\
568	0\\
569	0\\
570	0\\
571	0\\
572	0\\
573	0\\
574	0\\
575	0\\
576	0\\
577	0\\
578	0\\
579	0\\
580	0\\
581	0\\
582	0\\
583	0\\
584	0\\
585	0\\
586	0\\
587	0\\
588	0\\
589	0\\
590	0\\
591	0\\
592	0\\
593	0\\
594	0\\
595	0\\
596	0\\
597	0\\
598	0\\
599	0\\
600	0\\
};
\addplot [color=mycolor20,solid,forget plot]
  table[row sep=crcr]{%
1	0\\
2	0\\
3	0\\
4	0\\
5	0\\
6	0\\
7	0\\
8	0\\
9	0\\
10	0\\
11	0\\
12	0\\
13	0\\
14	0\\
15	0\\
16	0\\
17	0\\
18	0\\
19	0\\
20	0\\
21	0\\
22	0\\
23	0\\
24	0\\
25	0\\
26	0\\
27	0\\
28	0\\
29	0\\
30	0\\
31	0\\
32	0\\
33	0\\
34	0\\
35	0\\
36	0\\
37	0\\
38	0\\
39	0\\
40	0\\
41	0\\
42	0\\
43	0\\
44	0\\
45	0\\
46	0\\
47	0\\
48	0\\
49	0\\
50	0\\
51	0\\
52	0\\
53	0\\
54	0\\
55	0\\
56	0\\
57	0\\
58	0\\
59	0\\
60	0\\
61	0\\
62	0\\
63	0\\
64	0\\
65	0\\
66	0\\
67	0\\
68	0\\
69	0\\
70	0\\
71	0\\
72	0\\
73	0\\
74	0\\
75	0\\
76	0\\
77	0\\
78	0\\
79	0\\
80	0\\
81	0\\
82	0\\
83	0\\
84	0\\
85	0\\
86	0\\
87	0\\
88	0\\
89	0\\
90	0\\
91	0\\
92	0\\
93	0\\
94	0\\
95	0\\
96	0\\
97	0\\
98	0\\
99	0\\
100	0\\
101	0\\
102	0\\
103	0\\
104	0\\
105	0\\
106	0\\
107	0\\
108	0\\
109	0\\
110	0\\
111	0\\
112	0\\
113	0\\
114	0\\
115	0\\
116	0\\
117	0\\
118	0\\
119	0\\
120	0\\
121	0\\
122	0\\
123	0\\
124	0\\
125	0\\
126	0\\
127	0\\
128	0\\
129	0\\
130	0\\
131	0\\
132	0\\
133	0\\
134	0\\
135	0\\
136	0\\
137	0\\
138	0\\
139	0\\
140	0\\
141	0\\
142	0\\
143	0\\
144	0\\
145	0\\
146	0\\
147	0\\
148	0\\
149	0\\
150	0\\
151	0\\
152	0\\
153	0\\
154	0\\
155	0\\
156	0\\
157	0\\
158	0\\
159	0\\
160	0\\
161	0\\
162	0\\
163	0\\
164	0\\
165	0\\
166	0\\
167	0\\
168	0\\
169	0\\
170	0\\
171	0\\
172	0\\
173	0\\
174	0\\
175	0\\
176	0\\
177	0\\
178	0\\
179	0\\
180	0\\
181	0\\
182	0\\
183	0\\
184	0\\
185	0\\
186	0\\
187	0\\
188	0\\
189	0\\
190	0\\
191	0\\
192	0\\
193	0\\
194	0\\
195	0\\
196	0\\
197	0\\
198	0\\
199	0\\
200	0\\
201	0\\
202	0\\
203	0\\
204	0\\
205	0\\
206	0\\
207	0\\
208	0\\
209	0\\
210	0\\
211	0\\
212	0\\
213	0\\
214	0\\
215	0\\
216	0\\
217	0\\
218	0\\
219	0\\
220	0\\
221	0\\
222	0\\
223	0\\
224	0\\
225	0\\
226	0\\
227	0\\
228	0\\
229	0\\
230	0\\
231	0\\
232	0\\
233	0\\
234	0\\
235	0\\
236	0\\
237	0\\
238	0\\
239	0\\
240	0\\
241	0\\
242	0\\
243	0\\
244	0\\
245	0\\
246	0\\
247	0\\
248	0\\
249	0\\
250	0\\
251	0\\
252	0\\
253	0\\
254	0\\
255	0\\
256	0\\
257	0\\
258	0\\
259	0\\
260	0\\
261	0\\
262	0\\
263	0\\
264	0\\
265	0\\
266	0\\
267	0\\
268	0\\
269	0\\
270	0\\
271	0\\
272	0\\
273	0\\
274	0\\
275	0\\
276	0\\
277	0\\
278	0\\
279	0\\
280	0\\
281	0\\
282	0\\
283	0\\
284	0\\
285	0\\
286	0\\
287	0\\
288	0\\
289	0\\
290	0\\
291	0\\
292	0\\
293	0\\
294	0\\
295	0\\
296	0\\
297	0\\
298	0\\
299	0\\
300	0\\
301	0\\
302	0\\
303	0\\
304	0\\
305	0\\
306	0\\
307	0\\
308	0\\
309	0\\
310	0\\
311	0\\
312	0\\
313	0\\
314	0\\
315	0\\
316	0\\
317	0\\
318	0\\
319	0\\
320	0\\
321	0\\
322	0\\
323	0\\
324	0\\
325	0\\
326	0\\
327	0\\
328	0\\
329	0\\
330	0\\
331	0\\
332	0\\
333	0\\
334	0\\
335	0\\
336	0\\
337	0\\
338	0\\
339	0\\
340	0\\
341	0\\
342	0\\
343	0\\
344	0\\
345	0\\
346	0\\
347	0\\
348	0\\
349	0\\
350	0\\
351	0\\
352	0\\
353	0\\
354	0\\
355	0\\
356	0\\
357	0\\
358	0\\
359	0\\
360	0\\
361	0\\
362	0\\
363	0\\
364	0\\
365	0\\
366	0\\
367	0\\
368	0\\
369	0\\
370	0\\
371	0\\
372	0\\
373	0\\
374	0\\
375	0\\
376	0\\
377	0\\
378	0\\
379	0\\
380	0\\
381	0\\
382	0\\
383	0\\
384	0\\
385	0\\
386	0\\
387	0\\
388	0\\
389	0\\
390	0\\
391	0\\
392	0\\
393	0\\
394	0\\
395	0\\
396	0\\
397	0\\
398	0\\
399	0\\
400	0\\
401	0\\
402	0\\
403	0\\
404	0\\
405	0\\
406	0\\
407	0\\
408	0\\
409	0\\
410	0\\
411	0\\
412	0\\
413	0\\
414	0\\
415	0\\
416	0\\
417	0\\
418	0\\
419	0\\
420	0\\
421	0\\
422	0\\
423	0\\
424	0\\
425	0\\
426	0\\
427	0\\
428	0\\
429	0\\
430	0\\
431	0\\
432	0\\
433	0\\
434	0\\
435	0\\
436	0\\
437	0\\
438	0\\
439	0\\
440	0\\
441	0\\
442	0\\
443	0\\
444	0\\
445	0\\
446	0\\
447	0\\
448	0\\
449	0\\
450	0\\
451	0\\
452	0\\
453	0\\
454	0\\
455	0\\
456	0\\
457	0\\
458	0\\
459	0\\
460	0\\
461	0\\
462	0\\
463	0\\
464	0\\
465	0\\
466	0\\
467	0\\
468	0\\
469	0\\
470	0\\
471	0\\
472	0\\
473	0\\
474	0\\
475	0\\
476	0\\
477	0\\
478	0\\
479	0\\
480	0\\
481	0\\
482	0\\
483	0\\
484	0\\
485	0\\
486	0\\
487	0\\
488	0\\
489	0\\
490	0\\
491	0\\
492	0\\
493	0\\
494	0\\
495	0\\
496	0\\
497	0\\
498	0\\
499	0\\
500	0\\
501	0\\
502	0\\
503	0\\
504	0\\
505	0\\
506	0\\
507	0\\
508	0\\
509	0\\
510	0\\
511	0\\
512	0\\
513	0\\
514	0\\
515	0\\
516	0\\
517	0\\
518	0\\
519	0\\
520	0\\
521	0\\
522	0\\
523	0\\
524	0\\
525	0\\
526	0\\
527	0\\
528	0\\
529	0\\
530	0\\
531	0\\
532	0\\
533	0\\
534	0\\
535	0\\
536	0\\
537	0\\
538	0\\
539	0\\
540	0\\
541	0\\
542	0\\
543	0\\
544	0\\
545	0\\
546	0\\
547	0\\
548	0\\
549	0\\
550	0\\
551	0\\
552	0\\
553	0\\
554	0\\
555	0\\
556	0\\
557	0\\
558	0\\
559	0\\
560	0\\
561	0\\
562	0\\
563	0\\
564	0\\
565	0\\
566	0\\
567	0\\
568	0\\
569	0\\
570	0\\
571	0\\
572	0\\
573	0\\
574	0\\
575	0\\
576	0\\
577	0\\
578	0\\
579	0\\
580	0\\
581	0\\
582	0\\
583	0\\
584	0\\
585	0\\
586	0\\
587	0\\
588	0\\
589	0\\
590	0\\
591	0\\
592	0\\
593	0\\
594	0\\
595	0\\
596	0\\
597	0\\
598	0\\
599	0\\
600	0\\
};
\addplot [color=mycolor21,solid,forget plot]
  table[row sep=crcr]{%
1	0\\
2	0\\
3	0\\
4	0\\
5	0\\
6	0\\
7	0\\
8	0\\
9	0\\
10	0\\
11	0\\
12	0\\
13	0\\
14	0\\
15	0\\
16	0\\
17	0\\
18	0\\
19	0\\
20	0\\
21	0\\
22	0\\
23	0\\
24	0\\
25	0\\
26	0\\
27	0\\
28	0\\
29	0\\
30	0\\
31	0\\
32	0\\
33	0\\
34	0\\
35	0\\
36	0\\
37	0\\
38	0\\
39	0\\
40	0\\
41	0\\
42	0\\
43	0\\
44	0\\
45	0\\
46	0\\
47	0\\
48	0\\
49	0\\
50	0\\
51	0\\
52	0\\
53	0\\
54	0\\
55	0\\
56	0\\
57	0\\
58	0\\
59	0\\
60	0\\
61	0\\
62	0\\
63	0\\
64	0\\
65	0\\
66	0\\
67	0\\
68	0\\
69	0\\
70	0\\
71	0\\
72	0\\
73	0\\
74	0\\
75	0\\
76	0\\
77	0\\
78	0\\
79	0\\
80	0\\
81	0\\
82	0\\
83	0\\
84	0\\
85	0\\
86	0\\
87	0\\
88	0\\
89	0\\
90	0\\
91	0\\
92	0\\
93	0\\
94	0\\
95	0\\
96	0\\
97	0\\
98	0\\
99	0\\
100	0\\
101	0\\
102	0\\
103	0\\
104	0\\
105	0\\
106	0\\
107	0\\
108	0\\
109	0\\
110	0\\
111	0\\
112	0\\
113	0\\
114	0\\
115	0\\
116	0\\
117	0\\
118	0\\
119	0\\
120	0\\
121	0\\
122	0\\
123	0\\
124	0\\
125	0\\
126	0\\
127	0\\
128	0\\
129	0\\
130	0\\
131	0\\
132	0\\
133	0\\
134	0\\
135	0\\
136	0\\
137	0\\
138	0\\
139	0\\
140	0\\
141	0\\
142	0\\
143	0\\
144	0\\
145	0\\
146	0\\
147	0\\
148	0\\
149	0\\
150	0\\
151	0\\
152	0\\
153	0\\
154	0\\
155	0\\
156	0\\
157	0\\
158	0\\
159	0\\
160	0\\
161	0\\
162	0\\
163	0\\
164	0\\
165	0\\
166	0\\
167	0\\
168	0\\
169	0\\
170	0\\
171	0\\
172	0\\
173	0\\
174	0\\
175	0\\
176	0\\
177	0\\
178	0\\
179	0\\
180	0\\
181	0\\
182	0\\
183	0\\
184	0\\
185	0\\
186	0\\
187	0\\
188	0\\
189	0\\
190	0\\
191	0\\
192	0\\
193	0\\
194	0\\
195	0\\
196	0\\
197	0\\
198	0\\
199	0\\
200	0\\
201	0\\
202	0\\
203	0\\
204	0\\
205	0\\
206	0\\
207	0\\
208	0\\
209	0\\
210	0\\
211	0\\
212	0\\
213	0\\
214	0\\
215	0\\
216	0\\
217	0\\
218	0\\
219	0\\
220	0\\
221	0\\
222	0\\
223	0\\
224	0\\
225	0\\
226	0\\
227	0\\
228	0\\
229	0\\
230	0\\
231	0\\
232	0\\
233	0\\
234	0\\
235	0\\
236	0\\
237	0\\
238	0\\
239	0\\
240	0\\
241	0\\
242	0\\
243	0\\
244	0\\
245	0\\
246	0\\
247	0\\
248	0\\
249	0\\
250	0\\
251	0\\
252	0\\
253	0\\
254	0\\
255	0\\
256	0\\
257	0\\
258	0\\
259	0\\
260	0\\
261	0\\
262	0\\
263	0\\
264	0\\
265	0\\
266	0\\
267	0\\
268	0\\
269	0\\
270	0\\
271	0\\
272	0\\
273	0\\
274	0\\
275	0\\
276	0\\
277	0\\
278	0\\
279	0\\
280	0\\
281	0\\
282	0\\
283	0\\
284	0\\
285	0\\
286	0\\
287	0\\
288	0\\
289	0\\
290	0\\
291	0\\
292	0\\
293	0\\
294	0\\
295	0\\
296	0\\
297	0\\
298	0\\
299	0\\
300	0\\
301	0\\
302	0\\
303	0\\
304	0\\
305	0\\
306	0\\
307	0\\
308	0\\
309	0\\
310	0\\
311	0\\
312	0\\
313	0\\
314	0\\
315	0\\
316	0\\
317	0\\
318	0\\
319	0\\
320	0\\
321	0\\
322	0\\
323	0\\
324	0\\
325	0\\
326	0\\
327	0\\
328	0\\
329	0\\
330	0\\
331	0\\
332	0\\
333	0\\
334	0\\
335	0\\
336	0\\
337	0\\
338	0\\
339	0\\
340	0\\
341	0\\
342	0\\
343	0\\
344	0\\
345	0\\
346	0\\
347	0\\
348	0\\
349	0\\
350	0\\
351	0\\
352	0\\
353	0\\
354	0\\
355	0\\
356	0\\
357	0\\
358	0\\
359	0\\
360	0\\
361	0\\
362	0\\
363	0\\
364	0\\
365	0\\
366	0\\
367	0\\
368	0\\
369	0\\
370	0\\
371	0\\
372	0\\
373	0\\
374	0\\
375	0\\
376	0\\
377	0\\
378	0\\
379	0\\
380	0\\
381	0\\
382	0\\
383	0\\
384	0\\
385	0\\
386	0\\
387	0\\
388	0\\
389	0\\
390	0\\
391	0\\
392	0\\
393	0\\
394	0\\
395	0\\
396	0\\
397	0\\
398	0\\
399	0\\
400	0\\
401	0\\
402	0\\
403	0\\
404	0\\
405	0\\
406	0\\
407	0\\
408	0\\
409	0\\
410	0\\
411	0\\
412	0\\
413	0\\
414	0\\
415	0\\
416	0\\
417	0\\
418	0\\
419	0\\
420	0\\
421	0\\
422	0\\
423	0\\
424	0\\
425	0\\
426	0\\
427	0\\
428	0\\
429	0\\
430	0\\
431	0\\
432	0\\
433	0\\
434	0\\
435	0\\
436	0\\
437	0\\
438	0\\
439	0\\
440	0\\
441	0\\
442	0\\
443	0\\
444	0\\
445	0\\
446	0\\
447	0\\
448	0\\
449	0\\
450	0\\
451	0\\
452	0\\
453	0\\
454	0\\
455	0\\
456	0\\
457	0\\
458	0\\
459	0\\
460	0\\
461	0\\
462	0\\
463	0\\
464	0\\
465	0\\
466	0\\
467	0\\
468	0\\
469	0\\
470	0\\
471	0\\
472	0\\
473	0\\
474	0\\
475	0\\
476	0\\
477	0\\
478	0\\
479	0\\
480	0\\
481	0\\
482	0\\
483	0\\
484	0\\
485	0\\
486	0\\
487	0\\
488	0\\
489	0\\
490	0\\
491	0\\
492	0\\
493	0\\
494	0\\
495	0\\
496	0\\
497	0\\
498	0\\
499	0\\
500	0\\
501	0\\
502	0\\
503	0\\
504	0\\
505	0\\
506	0\\
507	0\\
508	0\\
509	0\\
510	0\\
511	0\\
512	0\\
513	0\\
514	0\\
515	0\\
516	0\\
517	0\\
518	0\\
519	0\\
520	0\\
521	0\\
522	0\\
523	0\\
524	0\\
525	0\\
526	0\\
527	0\\
528	0\\
529	0\\
530	0\\
531	0\\
532	0\\
533	0\\
534	0\\
535	0\\
536	0\\
537	0\\
538	0\\
539	0\\
540	0\\
541	0\\
542	0\\
543	0\\
544	0\\
545	0\\
546	0\\
547	0\\
548	0\\
549	0\\
550	0\\
551	0\\
552	0\\
553	0\\
554	0\\
555	0\\
556	0\\
557	0\\
558	0\\
559	0\\
560	0\\
561	0\\
562	0\\
563	0\\
564	0\\
565	0\\
566	0\\
567	0\\
568	0\\
569	0\\
570	0\\
571	0\\
572	0\\
573	0\\
574	0\\
575	0\\
576	0\\
577	0\\
578	0\\
579	0\\
580	0\\
581	0\\
582	0\\
583	0\\
584	0\\
585	0\\
586	0\\
587	0\\
588	0\\
589	0\\
590	0\\
591	0\\
592	0\\
593	0\\
594	0\\
595	0\\
596	0\\
597	0\\
598	0\\
599	0\\
600	0\\
};
\addplot [color=black!20!mycolor21,solid,forget plot]
  table[row sep=crcr]{%
1	0\\
2	0\\
3	0\\
4	0\\
5	0\\
6	0\\
7	0\\
8	0\\
9	0\\
10	0\\
11	0\\
12	0\\
13	0\\
14	0\\
15	0\\
16	0\\
17	0\\
18	0\\
19	0\\
20	0\\
21	0\\
22	0\\
23	0\\
24	0\\
25	0\\
26	0\\
27	0\\
28	0\\
29	0\\
30	0\\
31	0\\
32	0\\
33	0\\
34	0\\
35	0\\
36	0\\
37	0\\
38	0\\
39	0\\
40	0\\
41	0\\
42	0\\
43	0\\
44	0\\
45	0\\
46	0\\
47	0\\
48	0\\
49	0\\
50	0\\
51	0\\
52	0\\
53	0\\
54	0\\
55	0\\
56	0\\
57	0\\
58	0\\
59	0\\
60	0\\
61	0\\
62	0\\
63	0\\
64	0\\
65	0\\
66	0\\
67	0\\
68	0\\
69	0\\
70	0\\
71	0\\
72	0\\
73	0\\
74	0\\
75	0\\
76	0\\
77	0\\
78	0\\
79	0\\
80	0\\
81	0\\
82	0\\
83	0\\
84	0\\
85	0\\
86	0\\
87	0\\
88	0\\
89	0\\
90	0\\
91	0\\
92	0\\
93	0\\
94	0\\
95	0\\
96	0\\
97	0\\
98	0\\
99	0\\
100	0\\
101	0\\
102	0\\
103	0\\
104	0\\
105	0\\
106	0\\
107	0\\
108	0\\
109	0\\
110	0\\
111	0\\
112	0\\
113	0\\
114	0\\
115	0\\
116	0\\
117	0\\
118	0\\
119	0\\
120	0\\
121	0\\
122	0\\
123	0\\
124	0\\
125	0\\
126	0\\
127	0\\
128	0\\
129	0\\
130	0\\
131	0\\
132	0\\
133	0\\
134	0\\
135	0\\
136	0\\
137	0\\
138	0\\
139	0\\
140	0\\
141	0\\
142	0\\
143	0\\
144	0\\
145	0\\
146	0\\
147	0\\
148	0\\
149	0\\
150	0\\
151	0\\
152	0\\
153	0\\
154	0\\
155	0\\
156	0\\
157	0\\
158	0\\
159	0\\
160	0\\
161	0\\
162	0\\
163	0\\
164	0\\
165	0\\
166	0\\
167	0\\
168	0\\
169	0\\
170	0\\
171	0\\
172	0\\
173	0\\
174	0\\
175	0\\
176	0\\
177	0\\
178	0\\
179	0\\
180	0\\
181	0\\
182	0\\
183	0\\
184	0\\
185	0\\
186	0\\
187	0\\
188	0\\
189	0\\
190	0\\
191	0\\
192	0\\
193	0\\
194	0\\
195	0\\
196	0\\
197	0\\
198	0\\
199	0\\
200	0\\
201	0\\
202	0\\
203	0\\
204	0\\
205	0\\
206	0\\
207	0\\
208	0\\
209	0\\
210	0\\
211	0\\
212	0\\
213	0\\
214	0\\
215	0\\
216	0\\
217	0\\
218	0\\
219	0\\
220	0\\
221	0\\
222	0\\
223	0\\
224	0\\
225	0\\
226	0\\
227	0\\
228	0\\
229	0\\
230	0\\
231	0\\
232	0\\
233	0\\
234	0\\
235	0\\
236	0\\
237	0\\
238	0\\
239	0\\
240	0\\
241	0\\
242	0\\
243	0\\
244	0\\
245	0\\
246	0\\
247	0\\
248	0\\
249	0\\
250	0\\
251	0\\
252	0\\
253	0\\
254	0\\
255	0\\
256	0\\
257	0\\
258	0\\
259	0\\
260	0\\
261	0\\
262	0\\
263	0\\
264	0\\
265	0\\
266	0\\
267	0\\
268	0\\
269	0\\
270	0\\
271	0\\
272	0\\
273	0\\
274	0\\
275	0\\
276	0\\
277	0\\
278	0\\
279	0\\
280	0\\
281	0\\
282	0\\
283	0\\
284	0\\
285	0\\
286	0\\
287	0\\
288	0\\
289	0\\
290	0\\
291	0\\
292	0\\
293	0\\
294	0\\
295	0\\
296	0\\
297	0\\
298	0\\
299	0\\
300	0\\
301	0\\
302	0\\
303	0\\
304	0\\
305	0\\
306	0\\
307	0\\
308	0\\
309	0\\
310	0\\
311	0\\
312	0\\
313	0\\
314	0\\
315	0\\
316	0\\
317	0\\
318	0\\
319	0\\
320	0\\
321	0\\
322	0\\
323	0\\
324	0\\
325	0\\
326	0\\
327	0\\
328	0\\
329	0\\
330	0\\
331	0\\
332	0\\
333	0\\
334	0\\
335	0\\
336	0\\
337	0\\
338	0\\
339	0\\
340	0\\
341	0\\
342	0\\
343	0\\
344	0\\
345	0\\
346	0\\
347	0\\
348	0\\
349	0\\
350	0\\
351	0\\
352	0\\
353	0\\
354	0\\
355	0\\
356	0\\
357	0\\
358	0\\
359	0\\
360	0\\
361	0\\
362	0\\
363	0\\
364	0\\
365	0\\
366	0\\
367	0\\
368	0\\
369	0\\
370	0\\
371	0\\
372	0\\
373	0\\
374	0\\
375	0\\
376	0\\
377	0\\
378	0\\
379	0\\
380	0\\
381	0\\
382	0\\
383	0\\
384	0\\
385	0\\
386	0\\
387	0\\
388	0\\
389	0\\
390	0\\
391	0\\
392	0\\
393	0\\
394	0\\
395	0\\
396	0\\
397	0\\
398	0\\
399	0\\
400	0\\
401	0\\
402	0\\
403	0\\
404	0\\
405	0\\
406	0\\
407	0\\
408	0\\
409	0\\
410	0\\
411	0\\
412	0\\
413	0\\
414	0\\
415	0\\
416	0\\
417	0\\
418	0\\
419	0\\
420	0\\
421	0\\
422	0\\
423	0\\
424	0\\
425	0\\
426	0\\
427	0\\
428	0\\
429	0\\
430	0\\
431	0\\
432	0\\
433	0\\
434	0\\
435	0\\
436	0\\
437	0\\
438	0\\
439	0\\
440	0\\
441	0\\
442	0\\
443	0\\
444	0\\
445	0\\
446	0\\
447	0\\
448	0\\
449	0\\
450	0\\
451	0\\
452	0\\
453	0\\
454	0\\
455	0\\
456	0\\
457	0\\
458	0\\
459	0\\
460	0\\
461	0\\
462	0\\
463	0\\
464	0\\
465	0\\
466	0\\
467	0\\
468	0\\
469	0\\
470	0\\
471	0\\
472	0\\
473	0\\
474	0\\
475	0\\
476	0\\
477	0\\
478	0\\
479	0\\
480	0\\
481	0\\
482	0\\
483	0\\
484	0\\
485	0\\
486	0\\
487	0\\
488	0\\
489	0\\
490	0\\
491	0\\
492	0\\
493	0\\
494	0\\
495	0\\
496	0\\
497	0\\
498	0\\
499	0\\
500	0\\
501	0\\
502	0\\
503	0\\
504	0\\
505	0\\
506	0\\
507	0\\
508	0\\
509	0\\
510	0\\
511	0\\
512	0\\
513	0\\
514	0\\
515	0\\
516	0\\
517	0\\
518	0\\
519	0\\
520	0\\
521	0\\
522	0\\
523	0\\
524	0\\
525	0\\
526	0\\
527	0\\
528	0\\
529	0\\
530	0\\
531	0\\
532	0\\
533	0\\
534	0\\
535	0\\
536	0\\
537	0\\
538	0\\
539	0\\
540	0\\
541	0\\
542	0\\
543	0\\
544	0\\
545	0\\
546	0\\
547	0\\
548	0\\
549	0\\
550	0\\
551	0\\
552	0\\
553	0\\
554	0\\
555	0\\
556	0\\
557	0\\
558	0\\
559	0\\
560	0\\
561	0\\
562	0\\
563	0\\
564	0\\
565	0\\
566	0\\
567	0\\
568	0\\
569	0\\
570	0\\
571	0\\
572	0\\
573	0\\
574	0\\
575	0\\
576	0\\
577	0\\
578	0\\
579	0\\
580	0\\
581	0\\
582	0\\
583	0\\
584	0\\
585	0\\
586	0\\
587	0\\
588	0\\
589	0\\
590	0\\
591	0\\
592	0\\
593	0\\
594	0\\
595	0\\
596	0\\
597	0\\
598	0\\
599	0\\
600	0\\
};
\addplot [color=black!50!mycolor20,solid,forget plot]
  table[row sep=crcr]{%
1	0\\
2	0\\
3	0\\
4	0\\
5	0\\
6	0\\
7	0\\
8	0\\
9	0\\
10	0\\
11	0\\
12	0\\
13	0\\
14	0\\
15	0\\
16	0\\
17	0\\
18	0\\
19	0\\
20	0\\
21	0\\
22	0\\
23	0\\
24	0\\
25	0\\
26	0\\
27	0\\
28	0\\
29	0\\
30	0\\
31	0\\
32	0\\
33	0\\
34	0\\
35	0\\
36	0\\
37	0\\
38	0\\
39	0\\
40	0\\
41	0\\
42	0\\
43	0\\
44	0\\
45	0\\
46	0\\
47	0\\
48	0\\
49	0\\
50	0\\
51	0\\
52	0\\
53	0\\
54	0\\
55	0\\
56	0\\
57	0\\
58	0\\
59	0\\
60	0\\
61	0\\
62	0\\
63	0\\
64	0\\
65	0\\
66	0\\
67	0\\
68	0\\
69	0\\
70	0\\
71	0\\
72	0\\
73	0\\
74	0\\
75	0\\
76	0\\
77	0\\
78	0\\
79	0\\
80	0\\
81	0\\
82	0\\
83	0\\
84	0\\
85	0\\
86	0\\
87	0\\
88	0\\
89	0\\
90	0\\
91	0\\
92	0\\
93	0\\
94	0\\
95	0\\
96	0\\
97	0\\
98	0\\
99	0\\
100	0\\
101	0\\
102	0\\
103	0\\
104	0\\
105	0\\
106	0\\
107	0\\
108	0\\
109	0\\
110	0\\
111	0\\
112	0\\
113	0\\
114	0\\
115	0\\
116	0\\
117	0\\
118	0\\
119	0\\
120	0\\
121	0\\
122	0\\
123	0\\
124	0\\
125	0\\
126	0\\
127	0\\
128	0\\
129	0\\
130	0\\
131	0\\
132	0\\
133	0\\
134	0\\
135	0\\
136	0\\
137	0\\
138	0\\
139	0\\
140	0\\
141	0\\
142	0\\
143	0\\
144	0\\
145	0\\
146	0\\
147	0\\
148	0\\
149	0\\
150	0\\
151	0\\
152	0\\
153	0\\
154	0\\
155	0\\
156	0\\
157	0\\
158	0\\
159	0\\
160	0\\
161	0\\
162	0\\
163	0\\
164	0\\
165	0\\
166	0\\
167	0\\
168	0\\
169	0\\
170	0\\
171	0\\
172	0\\
173	0\\
174	0\\
175	0\\
176	0\\
177	0\\
178	0\\
179	0\\
180	0\\
181	0\\
182	0\\
183	0\\
184	0\\
185	0\\
186	0\\
187	0\\
188	0\\
189	0\\
190	0\\
191	0\\
192	0\\
193	0\\
194	0\\
195	0\\
196	0\\
197	0\\
198	0\\
199	0\\
200	0\\
201	0\\
202	0\\
203	0\\
204	0\\
205	0\\
206	0\\
207	0\\
208	0\\
209	0\\
210	0\\
211	0\\
212	0\\
213	0\\
214	0\\
215	0\\
216	0\\
217	0\\
218	0\\
219	0\\
220	0\\
221	0\\
222	0\\
223	0\\
224	0\\
225	0\\
226	0\\
227	0\\
228	0\\
229	0\\
230	0\\
231	0\\
232	0\\
233	0\\
234	0\\
235	0\\
236	0\\
237	0\\
238	0\\
239	0\\
240	0\\
241	0\\
242	0\\
243	0\\
244	0\\
245	0\\
246	0\\
247	0\\
248	0\\
249	0\\
250	0\\
251	0\\
252	0\\
253	0\\
254	0\\
255	0\\
256	0\\
257	0\\
258	0\\
259	0\\
260	0\\
261	0\\
262	0\\
263	0\\
264	0\\
265	0\\
266	0\\
267	0\\
268	0\\
269	0\\
270	0\\
271	0\\
272	0\\
273	0\\
274	0\\
275	0\\
276	0\\
277	0\\
278	0\\
279	0\\
280	0\\
281	0\\
282	0\\
283	0\\
284	0\\
285	0\\
286	0\\
287	0\\
288	0\\
289	0\\
290	0\\
291	0\\
292	0\\
293	0\\
294	0\\
295	0\\
296	0\\
297	0\\
298	0\\
299	0\\
300	0\\
301	0\\
302	0\\
303	0\\
304	0\\
305	0\\
306	0\\
307	0\\
308	0\\
309	0\\
310	0\\
311	0\\
312	0\\
313	0\\
314	0\\
315	0\\
316	0\\
317	0\\
318	0\\
319	0\\
320	0\\
321	0\\
322	0\\
323	0\\
324	0\\
325	0\\
326	0\\
327	0\\
328	0\\
329	0\\
330	0\\
331	0\\
332	0\\
333	0\\
334	0\\
335	0\\
336	0\\
337	0\\
338	0\\
339	0\\
340	0\\
341	0\\
342	0\\
343	0\\
344	0\\
345	0\\
346	0\\
347	0\\
348	0\\
349	0\\
350	0\\
351	0\\
352	0\\
353	0\\
354	0\\
355	0\\
356	0\\
357	0\\
358	0\\
359	0\\
360	0\\
361	0\\
362	0\\
363	0\\
364	0\\
365	0\\
366	0\\
367	0\\
368	0\\
369	0\\
370	0\\
371	0\\
372	0\\
373	0\\
374	0\\
375	0\\
376	0\\
377	0\\
378	0\\
379	0\\
380	0\\
381	0\\
382	0\\
383	0\\
384	0\\
385	0\\
386	0\\
387	0\\
388	0\\
389	0\\
390	0\\
391	0\\
392	0\\
393	0\\
394	0\\
395	0\\
396	0\\
397	0\\
398	0\\
399	0\\
400	0\\
401	0\\
402	0\\
403	0\\
404	0\\
405	0\\
406	0\\
407	0\\
408	0\\
409	0\\
410	0\\
411	0\\
412	0\\
413	0\\
414	0\\
415	0\\
416	0\\
417	0\\
418	0\\
419	0\\
420	0\\
421	0\\
422	0\\
423	0\\
424	0\\
425	0\\
426	0\\
427	0\\
428	0\\
429	0\\
430	0\\
431	0\\
432	0\\
433	0\\
434	0\\
435	0\\
436	0\\
437	0\\
438	0\\
439	0\\
440	0\\
441	0\\
442	0\\
443	0\\
444	0\\
445	0\\
446	0\\
447	0\\
448	0\\
449	0\\
450	0\\
451	0\\
452	0\\
453	0\\
454	0\\
455	0\\
456	0\\
457	0\\
458	0\\
459	0\\
460	0\\
461	0\\
462	0\\
463	0\\
464	0\\
465	0\\
466	0\\
467	0\\
468	0\\
469	0\\
470	0\\
471	0\\
472	0\\
473	0\\
474	0\\
475	0\\
476	0\\
477	0\\
478	0\\
479	0\\
480	0\\
481	0\\
482	0\\
483	0\\
484	0\\
485	0\\
486	0\\
487	0\\
488	0\\
489	0\\
490	0\\
491	0\\
492	0\\
493	0\\
494	0\\
495	0\\
496	0\\
497	0\\
498	0\\
499	0\\
500	0\\
501	0\\
502	0\\
503	0\\
504	0\\
505	0\\
506	0\\
507	0\\
508	0\\
509	0\\
510	0\\
511	0\\
512	0\\
513	0\\
514	0\\
515	0\\
516	0\\
517	0\\
518	0\\
519	0\\
520	0\\
521	0\\
522	0\\
523	0\\
524	0\\
525	0\\
526	0\\
527	0\\
528	0\\
529	0\\
530	0\\
531	0\\
532	0\\
533	0\\
534	0\\
535	0\\
536	0\\
537	0\\
538	0\\
539	0\\
540	0\\
541	0\\
542	0\\
543	0\\
544	0\\
545	0\\
546	0\\
547	0\\
548	0\\
549	0\\
550	0\\
551	0\\
552	0\\
553	0\\
554	0\\
555	0\\
556	0\\
557	0\\
558	0\\
559	0\\
560	0\\
561	0\\
562	0\\
563	0\\
564	0\\
565	0\\
566	0\\
567	0\\
568	0\\
569	0\\
570	0\\
571	0\\
572	0\\
573	0\\
574	0\\
575	0\\
576	0\\
577	0\\
578	0\\
579	0\\
580	0\\
581	0\\
582	0\\
583	0\\
584	0\\
585	0\\
586	0\\
587	0\\
588	0\\
589	0\\
590	0\\
591	0\\
592	0\\
593	0\\
594	0\\
595	0\\
596	0\\
597	0\\
598	0\\
599	0\\
600	0\\
};
\addplot [color=black!60!mycolor21,solid,forget plot]
  table[row sep=crcr]{%
1	0\\
2	0\\
3	0\\
4	0\\
5	0\\
6	0\\
7	0\\
8	0\\
9	0\\
10	0\\
11	0\\
12	0\\
13	0\\
14	0\\
15	0\\
16	0\\
17	0\\
18	0\\
19	0\\
20	0\\
21	0\\
22	0\\
23	0\\
24	0\\
25	0\\
26	0\\
27	0\\
28	0\\
29	0\\
30	0\\
31	0\\
32	0\\
33	0\\
34	0\\
35	0\\
36	0\\
37	0\\
38	0\\
39	0\\
40	0\\
41	0\\
42	0\\
43	0\\
44	0\\
45	0\\
46	0\\
47	0\\
48	0\\
49	0\\
50	0\\
51	0\\
52	0\\
53	0\\
54	0\\
55	0\\
56	0\\
57	0\\
58	0\\
59	0\\
60	0\\
61	0\\
62	0\\
63	0\\
64	0\\
65	0\\
66	0\\
67	0\\
68	0\\
69	0\\
70	0\\
71	0\\
72	0\\
73	0\\
74	0\\
75	0\\
76	0\\
77	0\\
78	0\\
79	0\\
80	0\\
81	0\\
82	0\\
83	0\\
84	0\\
85	0\\
86	0\\
87	0\\
88	0\\
89	0\\
90	0\\
91	0\\
92	0\\
93	0\\
94	0\\
95	0\\
96	0\\
97	0\\
98	0\\
99	0\\
100	0\\
101	0\\
102	0\\
103	0\\
104	0\\
105	0\\
106	0\\
107	0\\
108	0\\
109	0\\
110	0\\
111	0\\
112	0\\
113	0\\
114	0\\
115	0\\
116	0\\
117	0\\
118	0\\
119	0\\
120	0\\
121	0\\
122	0\\
123	0\\
124	0\\
125	0\\
126	0\\
127	0\\
128	0\\
129	0\\
130	0\\
131	0\\
132	0\\
133	0\\
134	0\\
135	0\\
136	0\\
137	0\\
138	0\\
139	0\\
140	0\\
141	0\\
142	0\\
143	0\\
144	0\\
145	0\\
146	0\\
147	0\\
148	0\\
149	0\\
150	0\\
151	0\\
152	0\\
153	0\\
154	0\\
155	0\\
156	0\\
157	0\\
158	0\\
159	0\\
160	0\\
161	0\\
162	0\\
163	0\\
164	0\\
165	0\\
166	0\\
167	0\\
168	0\\
169	0\\
170	0\\
171	0\\
172	0\\
173	0\\
174	0\\
175	0\\
176	0\\
177	0\\
178	0\\
179	0\\
180	0\\
181	0\\
182	0\\
183	0\\
184	0\\
185	0\\
186	0\\
187	0\\
188	0\\
189	0\\
190	0\\
191	0\\
192	0\\
193	0\\
194	0\\
195	0\\
196	0\\
197	0\\
198	0\\
199	0\\
200	0\\
201	0\\
202	0\\
203	0\\
204	0\\
205	0\\
206	0\\
207	0\\
208	0\\
209	0\\
210	0\\
211	0\\
212	0\\
213	0\\
214	0\\
215	0\\
216	0\\
217	0\\
218	0\\
219	0\\
220	0\\
221	0\\
222	0\\
223	0\\
224	0\\
225	0\\
226	0\\
227	0\\
228	0\\
229	0\\
230	0\\
231	0\\
232	0\\
233	0\\
234	0\\
235	0\\
236	0\\
237	0\\
238	0\\
239	0\\
240	0\\
241	0\\
242	0\\
243	0\\
244	0\\
245	0\\
246	0\\
247	0\\
248	0\\
249	0\\
250	0\\
251	0\\
252	0\\
253	0\\
254	0\\
255	0\\
256	0\\
257	0\\
258	0\\
259	0\\
260	0\\
261	0\\
262	0\\
263	0\\
264	0\\
265	0\\
266	0\\
267	0\\
268	0\\
269	0\\
270	0\\
271	0\\
272	0\\
273	0\\
274	0\\
275	0\\
276	0\\
277	0\\
278	0\\
279	0\\
280	0\\
281	0\\
282	0\\
283	0\\
284	0\\
285	0\\
286	0\\
287	0\\
288	0\\
289	0\\
290	0\\
291	0\\
292	0\\
293	0\\
294	0\\
295	0\\
296	0\\
297	0\\
298	0\\
299	0\\
300	0\\
301	0\\
302	0\\
303	0\\
304	0\\
305	0\\
306	0\\
307	0\\
308	0\\
309	0\\
310	0\\
311	0\\
312	0\\
313	0\\
314	0\\
315	0\\
316	0\\
317	0\\
318	0\\
319	0\\
320	0\\
321	0\\
322	0\\
323	0\\
324	0\\
325	0\\
326	0\\
327	0\\
328	0\\
329	0\\
330	0\\
331	0\\
332	0\\
333	0\\
334	0\\
335	0\\
336	0\\
337	0\\
338	0\\
339	0\\
340	0\\
341	0\\
342	0\\
343	0\\
344	0\\
345	0\\
346	0\\
347	0\\
348	0\\
349	0\\
350	0\\
351	0\\
352	0\\
353	0\\
354	0\\
355	0\\
356	0\\
357	0\\
358	0\\
359	0\\
360	0\\
361	0\\
362	0\\
363	0\\
364	0\\
365	0\\
366	0\\
367	0\\
368	0\\
369	0\\
370	0\\
371	0\\
372	0\\
373	0\\
374	0\\
375	0\\
376	0\\
377	0\\
378	0\\
379	0\\
380	0\\
381	0\\
382	0\\
383	0\\
384	0\\
385	0\\
386	0\\
387	0\\
388	0\\
389	0\\
390	0\\
391	0\\
392	0\\
393	0\\
394	0\\
395	0\\
396	0\\
397	0\\
398	0\\
399	0\\
400	0\\
401	0\\
402	0\\
403	0\\
404	0\\
405	0\\
406	0\\
407	0\\
408	0\\
409	0\\
410	0\\
411	0\\
412	0\\
413	0\\
414	0\\
415	0\\
416	0\\
417	0\\
418	0\\
419	0\\
420	0\\
421	0\\
422	0\\
423	0\\
424	0\\
425	0\\
426	0\\
427	0\\
428	0\\
429	0\\
430	0\\
431	0\\
432	0\\
433	0\\
434	0\\
435	0\\
436	0\\
437	0\\
438	0\\
439	0\\
440	0\\
441	0\\
442	0\\
443	0\\
444	0\\
445	0\\
446	0\\
447	0\\
448	0\\
449	0\\
450	0\\
451	0\\
452	0\\
453	0\\
454	0\\
455	0\\
456	0\\
457	0\\
458	0\\
459	0\\
460	0\\
461	0\\
462	0\\
463	0\\
464	0\\
465	0\\
466	0\\
467	0\\
468	0\\
469	0\\
470	0\\
471	0\\
472	0\\
473	0\\
474	0\\
475	0\\
476	0\\
477	0\\
478	0\\
479	0\\
480	0\\
481	0\\
482	0\\
483	0\\
484	0\\
485	0\\
486	0\\
487	0\\
488	0\\
489	0\\
490	0\\
491	0\\
492	0\\
493	0\\
494	0\\
495	0\\
496	0\\
497	0\\
498	0\\
499	0\\
500	0\\
501	0\\
502	0\\
503	0\\
504	0\\
505	0\\
506	0\\
507	0\\
508	0\\
509	0\\
510	0\\
511	0\\
512	0\\
513	0\\
514	0\\
515	0\\
516	0\\
517	0\\
518	0\\
519	0\\
520	0\\
521	0\\
522	0\\
523	0\\
524	0\\
525	0\\
526	0\\
527	0\\
528	0\\
529	0\\
530	0\\
531	0\\
532	0\\
533	0\\
534	0\\
535	0\\
536	0\\
537	0\\
538	0\\
539	0\\
540	0\\
541	0\\
542	0\\
543	0\\
544	0\\
545	0\\
546	0\\
547	0\\
548	0\\
549	0\\
550	0\\
551	0\\
552	0\\
553	0\\
554	0\\
555	0\\
556	0\\
557	0\\
558	0\\
559	0\\
560	0\\
561	0\\
562	0\\
563	0\\
564	0\\
565	0\\
566	0\\
567	0\\
568	0\\
569	0\\
570	0\\
571	0\\
572	0\\
573	0\\
574	0\\
575	0\\
576	0\\
577	0\\
578	0\\
579	0\\
580	0\\
581	0\\
582	0\\
583	0\\
584	0\\
585	0\\
586	0\\
587	0\\
588	0\\
589	0\\
590	0\\
591	0\\
592	0\\
593	0\\
594	0\\
595	0\\
596	0\\
597	0\\
598	0\\
599	0\\
600	0\\
};
\addplot [color=black!80!mycolor21,solid,forget plot]
  table[row sep=crcr]{%
1	0\\
2	0\\
3	0\\
4	0\\
5	0\\
6	0\\
7	0\\
8	0\\
9	0\\
10	0\\
11	0\\
12	0\\
13	0\\
14	0\\
15	0\\
16	0\\
17	0\\
18	0\\
19	0\\
20	0\\
21	0\\
22	0\\
23	0\\
24	0\\
25	0\\
26	0\\
27	0\\
28	0\\
29	0\\
30	0\\
31	0\\
32	0\\
33	0\\
34	0\\
35	0\\
36	0\\
37	0\\
38	0\\
39	0\\
40	0\\
41	0\\
42	0\\
43	0\\
44	0\\
45	0\\
46	0\\
47	0\\
48	0\\
49	0\\
50	0\\
51	0\\
52	0\\
53	0\\
54	0\\
55	0\\
56	0\\
57	0\\
58	0\\
59	0\\
60	0\\
61	0\\
62	0\\
63	0\\
64	0\\
65	0\\
66	0\\
67	0\\
68	0\\
69	0\\
70	0\\
71	0\\
72	0\\
73	0\\
74	0\\
75	0\\
76	0\\
77	0\\
78	0\\
79	0\\
80	0\\
81	0\\
82	0\\
83	0\\
84	0\\
85	0\\
86	0\\
87	0\\
88	0\\
89	0\\
90	0\\
91	0\\
92	0\\
93	0\\
94	0\\
95	0\\
96	0\\
97	0\\
98	0\\
99	0\\
100	0\\
101	0\\
102	0\\
103	0\\
104	0\\
105	0\\
106	0\\
107	0\\
108	0\\
109	0\\
110	0\\
111	0\\
112	0\\
113	0\\
114	0\\
115	0\\
116	0\\
117	0\\
118	0\\
119	0\\
120	0\\
121	0\\
122	0\\
123	0\\
124	0\\
125	0\\
126	0\\
127	0\\
128	0\\
129	0\\
130	0\\
131	0\\
132	0\\
133	0\\
134	0\\
135	0\\
136	0\\
137	0\\
138	0\\
139	0\\
140	0\\
141	0\\
142	0\\
143	0\\
144	0\\
145	0\\
146	0\\
147	0\\
148	0\\
149	0\\
150	0\\
151	0\\
152	0\\
153	0\\
154	0\\
155	0\\
156	0\\
157	0\\
158	0\\
159	0\\
160	0\\
161	0\\
162	0\\
163	0\\
164	0\\
165	0\\
166	0\\
167	0\\
168	0\\
169	0\\
170	0\\
171	0\\
172	0\\
173	0\\
174	0\\
175	0\\
176	0\\
177	0\\
178	0\\
179	0\\
180	0\\
181	0\\
182	0\\
183	0\\
184	0\\
185	0\\
186	0\\
187	0\\
188	0\\
189	0\\
190	0\\
191	0\\
192	0\\
193	0\\
194	0\\
195	0\\
196	0\\
197	0\\
198	0\\
199	0\\
200	0\\
201	0\\
202	0\\
203	0\\
204	0\\
205	0\\
206	0\\
207	0\\
208	0\\
209	0\\
210	0\\
211	0\\
212	0\\
213	0\\
214	0\\
215	0\\
216	0\\
217	0\\
218	0\\
219	0\\
220	0\\
221	0\\
222	0\\
223	0\\
224	0\\
225	0\\
226	0\\
227	0\\
228	0\\
229	0\\
230	0\\
231	0\\
232	0\\
233	0\\
234	0\\
235	0\\
236	0\\
237	0\\
238	0\\
239	0\\
240	0\\
241	0\\
242	0\\
243	0\\
244	0\\
245	0\\
246	0\\
247	0\\
248	0\\
249	0\\
250	0\\
251	0\\
252	0\\
253	0\\
254	0\\
255	0\\
256	0\\
257	0\\
258	0\\
259	0\\
260	0\\
261	0\\
262	0\\
263	0\\
264	0\\
265	0\\
266	0\\
267	0\\
268	0\\
269	0\\
270	0\\
271	0\\
272	0\\
273	0\\
274	0\\
275	0\\
276	0\\
277	0\\
278	0\\
279	0\\
280	0\\
281	0\\
282	0\\
283	0\\
284	0\\
285	0\\
286	0\\
287	0\\
288	0\\
289	0\\
290	0\\
291	0\\
292	0\\
293	0\\
294	0\\
295	0\\
296	0\\
297	0\\
298	0\\
299	0\\
300	0\\
301	0\\
302	0\\
303	0\\
304	0\\
305	0\\
306	0\\
307	0\\
308	0\\
309	0\\
310	0\\
311	0\\
312	0\\
313	0\\
314	0\\
315	0\\
316	0\\
317	0\\
318	0\\
319	0\\
320	0\\
321	0\\
322	0\\
323	0\\
324	0\\
325	0\\
326	0\\
327	0\\
328	0\\
329	0\\
330	0\\
331	0\\
332	0\\
333	0\\
334	0\\
335	0\\
336	0\\
337	0\\
338	0\\
339	0\\
340	0\\
341	0\\
342	0\\
343	0\\
344	0\\
345	0\\
346	0\\
347	0\\
348	0\\
349	0\\
350	0\\
351	0\\
352	0\\
353	0\\
354	0\\
355	0\\
356	0\\
357	0\\
358	0\\
359	0\\
360	0\\
361	0\\
362	0\\
363	0\\
364	0\\
365	0\\
366	0\\
367	0\\
368	0\\
369	0\\
370	0\\
371	0\\
372	0\\
373	0\\
374	0\\
375	0\\
376	0\\
377	0\\
378	0\\
379	0\\
380	0\\
381	0\\
382	0\\
383	0\\
384	0\\
385	0\\
386	0\\
387	0\\
388	0\\
389	0\\
390	0\\
391	0\\
392	0\\
393	0\\
394	0\\
395	0\\
396	0\\
397	0\\
398	0\\
399	0\\
400	0\\
401	0\\
402	0\\
403	0\\
404	0\\
405	0\\
406	0\\
407	0\\
408	0\\
409	0\\
410	0\\
411	0\\
412	0\\
413	0\\
414	0\\
415	0\\
416	0\\
417	0\\
418	0\\
419	0\\
420	0\\
421	0\\
422	0\\
423	0\\
424	0\\
425	0\\
426	0\\
427	0\\
428	0\\
429	0\\
430	0\\
431	0\\
432	0\\
433	0\\
434	0\\
435	0\\
436	0\\
437	0\\
438	0\\
439	0\\
440	0\\
441	0\\
442	0\\
443	0\\
444	0\\
445	0\\
446	0\\
447	0\\
448	0\\
449	0\\
450	0\\
451	0\\
452	0\\
453	0\\
454	0\\
455	0\\
456	0\\
457	0\\
458	0\\
459	0\\
460	0\\
461	0\\
462	0\\
463	0\\
464	0\\
465	0\\
466	0\\
467	0\\
468	0\\
469	0\\
470	0\\
471	0\\
472	0\\
473	0\\
474	0\\
475	0\\
476	0\\
477	0\\
478	0\\
479	0\\
480	0\\
481	0\\
482	0\\
483	0\\
484	0\\
485	0\\
486	0\\
487	0\\
488	0\\
489	0\\
490	0\\
491	0\\
492	0\\
493	0\\
494	0\\
495	0\\
496	0\\
497	0\\
498	0\\
499	0\\
500	0\\
501	0\\
502	0\\
503	0\\
504	0\\
505	0\\
506	0\\
507	0\\
508	0\\
509	0\\
510	0\\
511	0\\
512	0\\
513	0\\
514	0\\
515	0\\
516	0\\
517	0\\
518	0\\
519	0\\
520	0\\
521	0\\
522	0\\
523	0\\
524	0\\
525	0\\
526	0\\
527	0\\
528	0\\
529	0\\
530	0\\
531	0\\
532	0\\
533	0\\
534	0\\
535	0\\
536	0\\
537	0\\
538	0\\
539	0\\
540	0\\
541	0\\
542	0\\
543	0\\
544	0\\
545	0\\
546	0\\
547	0\\
548	0\\
549	0\\
550	0\\
551	0\\
552	0\\
553	0\\
554	0\\
555	0\\
556	0\\
557	0\\
558	0\\
559	0\\
560	0\\
561	0\\
562	0\\
563	0\\
564	0\\
565	0\\
566	0\\
567	0\\
568	0\\
569	0\\
570	0\\
571	0\\
572	0\\
573	0\\
574	0\\
575	0\\
576	0\\
577	0\\
578	0\\
579	0\\
580	0\\
581	0\\
582	0\\
583	0\\
584	0\\
585	0\\
586	0\\
587	0\\
588	0\\
589	0\\
590	0\\
591	0\\
592	0\\
593	0\\
594	0\\
595	0\\
596	0\\
597	0\\
598	0\\
599	0\\
600	0\\
};
\addplot [color=black,solid,forget plot]
  table[row sep=crcr]{%
1	0\\
2	0\\
3	0\\
4	0\\
5	0\\
6	0\\
7	0\\
8	0\\
9	0\\
10	0\\
11	0\\
12	0\\
13	0\\
14	0\\
15	0\\
16	0\\
17	0\\
18	0\\
19	0\\
20	0\\
21	0\\
22	0\\
23	0\\
24	0\\
25	0\\
26	0\\
27	0\\
28	0\\
29	0\\
30	0\\
31	0\\
32	0\\
33	0\\
34	0\\
35	0\\
36	0\\
37	0\\
38	0\\
39	0\\
40	0\\
41	0\\
42	0\\
43	0\\
44	0\\
45	0\\
46	0\\
47	0\\
48	0\\
49	0\\
50	0\\
51	0\\
52	0\\
53	0\\
54	0\\
55	0\\
56	0\\
57	0\\
58	0\\
59	0\\
60	0\\
61	0\\
62	0\\
63	0\\
64	0\\
65	0\\
66	0\\
67	0\\
68	0\\
69	0\\
70	0\\
71	0\\
72	0\\
73	0\\
74	0\\
75	0\\
76	0\\
77	0\\
78	0\\
79	0\\
80	0\\
81	0\\
82	0\\
83	0\\
84	0\\
85	0\\
86	0\\
87	0\\
88	0\\
89	0\\
90	0\\
91	0\\
92	0\\
93	0\\
94	0\\
95	0\\
96	0\\
97	0\\
98	0\\
99	0\\
100	0\\
101	0\\
102	0\\
103	0\\
104	0\\
105	0\\
106	0\\
107	0\\
108	0\\
109	0\\
110	0\\
111	0\\
112	0\\
113	0\\
114	0\\
115	0\\
116	0\\
117	0\\
118	0\\
119	0\\
120	0\\
121	0\\
122	0\\
123	0\\
124	0\\
125	0\\
126	0\\
127	0\\
128	0\\
129	0\\
130	0\\
131	0\\
132	0\\
133	0\\
134	0\\
135	0\\
136	0\\
137	0\\
138	0\\
139	0\\
140	0\\
141	0\\
142	0\\
143	0\\
144	0\\
145	0\\
146	0\\
147	0\\
148	0\\
149	0\\
150	0\\
151	0\\
152	0\\
153	0\\
154	0\\
155	0\\
156	0\\
157	0\\
158	0\\
159	0\\
160	0\\
161	0\\
162	0\\
163	0\\
164	0\\
165	0\\
166	0\\
167	0\\
168	0\\
169	0\\
170	0\\
171	0\\
172	0\\
173	0\\
174	0\\
175	0\\
176	0\\
177	0\\
178	0\\
179	0\\
180	0\\
181	0\\
182	0\\
183	0\\
184	0\\
185	0\\
186	0\\
187	0\\
188	0\\
189	0\\
190	0\\
191	0\\
192	0\\
193	0\\
194	0\\
195	0\\
196	0\\
197	0\\
198	0\\
199	0\\
200	0\\
201	0\\
202	0\\
203	0\\
204	0\\
205	0\\
206	0\\
207	0\\
208	0\\
209	0\\
210	0\\
211	0\\
212	0\\
213	0\\
214	0\\
215	0\\
216	0\\
217	0\\
218	0\\
219	0\\
220	0\\
221	0\\
222	0\\
223	0\\
224	0\\
225	0\\
226	0\\
227	0\\
228	0\\
229	0\\
230	0\\
231	0\\
232	0\\
233	0\\
234	0\\
235	0\\
236	0\\
237	0\\
238	0\\
239	0\\
240	0\\
241	0\\
242	0\\
243	0\\
244	0\\
245	0\\
246	0\\
247	0\\
248	0\\
249	0\\
250	0\\
251	0\\
252	0\\
253	0\\
254	0\\
255	0\\
256	0\\
257	0\\
258	0\\
259	0\\
260	0\\
261	0\\
262	0\\
263	0\\
264	0\\
265	0\\
266	0\\
267	0\\
268	0\\
269	0\\
270	0\\
271	0\\
272	0\\
273	0\\
274	0\\
275	0\\
276	0\\
277	0\\
278	0\\
279	0\\
280	0\\
281	0\\
282	0\\
283	0\\
284	0\\
285	0\\
286	0\\
287	0\\
288	0\\
289	0\\
290	0\\
291	0\\
292	0\\
293	0\\
294	0\\
295	0\\
296	0\\
297	0\\
298	0\\
299	0\\
300	0\\
301	0\\
302	0\\
303	0\\
304	0\\
305	0\\
306	0\\
307	0\\
308	0\\
309	0\\
310	0\\
311	0\\
312	0\\
313	0\\
314	0\\
315	0\\
316	0\\
317	0\\
318	0\\
319	0\\
320	0\\
321	0\\
322	0\\
323	0\\
324	0\\
325	0\\
326	0\\
327	0\\
328	0\\
329	0\\
330	0\\
331	0\\
332	0\\
333	0\\
334	0\\
335	0\\
336	0\\
337	0\\
338	0\\
339	0\\
340	0\\
341	0\\
342	0\\
343	0\\
344	0\\
345	0\\
346	0\\
347	0\\
348	0\\
349	0\\
350	0\\
351	0\\
352	0\\
353	0\\
354	0\\
355	0\\
356	0\\
357	0\\
358	0\\
359	0\\
360	0\\
361	0\\
362	0\\
363	0\\
364	0\\
365	0\\
366	0\\
367	0\\
368	0\\
369	0\\
370	0\\
371	0\\
372	0\\
373	0\\
374	0\\
375	0\\
376	0\\
377	0\\
378	0\\
379	0\\
380	0\\
381	0\\
382	0\\
383	0\\
384	0\\
385	0\\
386	0\\
387	0\\
388	0\\
389	0\\
390	0\\
391	0\\
392	0\\
393	0\\
394	0\\
395	0\\
396	0\\
397	0\\
398	0\\
399	0\\
400	0\\
401	0\\
402	0\\
403	0\\
404	0\\
405	0\\
406	0\\
407	0\\
408	0\\
409	0\\
410	0\\
411	0\\
412	0\\
413	0\\
414	0\\
415	0\\
416	0\\
417	0\\
418	0\\
419	0\\
420	0\\
421	0\\
422	0\\
423	0\\
424	0\\
425	0\\
426	0\\
427	0\\
428	0\\
429	0\\
430	0\\
431	0\\
432	0\\
433	0\\
434	0\\
435	0\\
436	0\\
437	0\\
438	0\\
439	0\\
440	0\\
441	0\\
442	0\\
443	0\\
444	0\\
445	0\\
446	0\\
447	0\\
448	0\\
449	0\\
450	0\\
451	0\\
452	0\\
453	0\\
454	0\\
455	0\\
456	0\\
457	0\\
458	0\\
459	0\\
460	0\\
461	0\\
462	0\\
463	0\\
464	0\\
465	0\\
466	0\\
467	0\\
468	0\\
469	0\\
470	0\\
471	0\\
472	0\\
473	0\\
474	0\\
475	0\\
476	0\\
477	0\\
478	0\\
479	0\\
480	0\\
481	0\\
482	0\\
483	0\\
484	0\\
485	0\\
486	0\\
487	0\\
488	0\\
489	0\\
490	0\\
491	0\\
492	0\\
493	0\\
494	0\\
495	0\\
496	0\\
497	0\\
498	0\\
499	0\\
500	0\\
501	0\\
502	0\\
503	0\\
504	0\\
505	0\\
506	0\\
507	0\\
508	0\\
509	0\\
510	0\\
511	0\\
512	0\\
513	0\\
514	0\\
515	0\\
516	0\\
517	0\\
518	0\\
519	0\\
520	0\\
521	0\\
522	0\\
523	0\\
524	0\\
525	0\\
526	0\\
527	0\\
528	0\\
529	0\\
530	0\\
531	0\\
532	0\\
533	0\\
534	0\\
535	0\\
536	0\\
537	0\\
538	0\\
539	0\\
540	0\\
541	0\\
542	0\\
543	0\\
544	0\\
545	0\\
546	0\\
547	0\\
548	0\\
549	0\\
550	0\\
551	0\\
552	0\\
553	0\\
554	0\\
555	0\\
556	0\\
557	0\\
558	0\\
559	0\\
560	0\\
561	0\\
562	0\\
563	0\\
564	0\\
565	0\\
566	0\\
567	0\\
568	0\\
569	0\\
570	0\\
571	0\\
572	0\\
573	0\\
574	0\\
575	0\\
576	0\\
577	0\\
578	0\\
579	0\\
580	0\\
581	0\\
582	0\\
583	0\\
584	0\\
585	0\\
586	0\\
587	0\\
588	0\\
589	0\\
590	0\\
591	0\\
592	0\\
593	0\\
594	0\\
595	0\\
596	0\\
597	0\\
598	0\\
599	0\\
600	0\\
};
\end{axis}
\end{tikzpicture}% 
%  \caption{Discrete Time}
%\end{subfigure}\\
%\vspace{1cm}
%\begin{subfigure}{.45\linewidth}
%  \centering
%  \setlength\figureheight{\linewidth} 
%  \setlength\figurewidth{\linewidth}
%  \tikzsetnextfilename{dm_cts_nFPC_z1}
%  % This file was created by matlab2tikz.
%
%The latest updates can be retrieved from
%  http://www.mathworks.com/matlabcentral/fileexchange/22022-matlab2tikz-matlab2tikz
%where you can also make suggestions and rate matlab2tikz.
%
\definecolor{mycolor1}{rgb}{0.00000,1.00000,0.14286}%
\definecolor{mycolor2}{rgb}{0.00000,1.00000,0.28571}%
\definecolor{mycolor3}{rgb}{0.00000,1.00000,0.42857}%
\definecolor{mycolor4}{rgb}{0.00000,1.00000,0.57143}%
\definecolor{mycolor5}{rgb}{0.00000,1.00000,0.71429}%
\definecolor{mycolor6}{rgb}{0.00000,1.00000,0.85714}%
\definecolor{mycolor7}{rgb}{0.00000,1.00000,1.00000}%
\definecolor{mycolor8}{rgb}{0.00000,0.87500,1.00000}%
\definecolor{mycolor9}{rgb}{0.00000,0.62500,1.00000}%
\definecolor{mycolor10}{rgb}{0.12500,0.00000,1.00000}%
\definecolor{mycolor11}{rgb}{0.25000,0.00000,1.00000}%
\definecolor{mycolor12}{rgb}{0.37500,0.00000,1.00000}%
\definecolor{mycolor13}{rgb}{0.50000,0.00000,1.00000}%
\definecolor{mycolor14}{rgb}{0.62500,0.00000,1.00000}%
\definecolor{mycolor15}{rgb}{0.75000,0.00000,1.00000}%
\definecolor{mycolor16}{rgb}{0.87500,0.00000,1.00000}%
\definecolor{mycolor17}{rgb}{1.00000,0.00000,1.00000}%
\definecolor{mycolor18}{rgb}{1.00000,0.00000,0.87500}%
\definecolor{mycolor19}{rgb}{1.00000,0.00000,0.62500}%
\definecolor{mycolor20}{rgb}{0.85714,0.00000,0.00000}%
\definecolor{mycolor21}{rgb}{0.71429,0.00000,0.00000}%
%
\begin{tikzpicture}

\begin{axis}[%
width=4.1in,
height=3.803in,
at={(0.809in,0.513in)},
scale only axis,
point meta min=0,
point meta max=1,
every outer x axis line/.append style={black},
every x tick label/.append style={font=\color{black}},
xmin=0,
xmax=600,
every outer y axis line/.append style={black},
every y tick label/.append style={font=\color{black}},
ymin=0,
ymax=0.012,
axis background/.style={fill=white},
axis x line*=bottom,
axis y line*=left,
colormap={mymap}{[1pt] rgb(0pt)=(0,1,0); rgb(7pt)=(0,1,1); rgb(15pt)=(0,0,1); rgb(23pt)=(1,0,1); rgb(31pt)=(1,0,0); rgb(38pt)=(0,0,0)},
colorbar,
colorbar style={separate axis lines,every outer x axis line/.append style={black},every x tick label/.append style={font=\color{black}},every outer y axis line/.append style={black},every y tick label/.append style={font=\color{black}},yticklabels={{-19},{-17},{-15},{-13},{-11},{-9},{-7},{-5},{-3},{-1},{1},{3},{5},{7},{9},{11},{13},{15},{17},{19}}}
]
\addplot [color=green,solid,forget plot]
  table[row sep=crcr]{%
0.01	0.01\\
1.01	0.01\\
2.01	0.01\\
3.01	0.01\\
4.01	0.01\\
5.01	0.01\\
6.01	0.01\\
7.01	0.01\\
8.01	0.01\\
9.01	0.01\\
10.01	0.01\\
11.01	0.01\\
12.01	0.01\\
13.01	0.01\\
14.01	0.01\\
15.01	0.01\\
16.01	0.01\\
17.01	0.01\\
18.01	0.01\\
19.01	0.01\\
20.01	0.01\\
21.01	0.01\\
22.01	0.01\\
23.01	0.01\\
24.01	0.01\\
25.01	0.01\\
26.01	0.01\\
27.01	0.01\\
28.01	0.01\\
29.01	0.01\\
30.01	0.01\\
31.01	0.01\\
32.01	0.01\\
33.01	0.01\\
34.01	0.01\\
35.01	0.01\\
36.01	0.01\\
37.01	0.01\\
38.01	0.01\\
39.01	0.01\\
40.01	0.01\\
41.01	0.01\\
42.01	0.01\\
43.01	0.01\\
44.01	0.01\\
45.01	0.01\\
46.01	0.01\\
47.01	0.01\\
48.01	0.01\\
49.01	0.01\\
50.01	0.01\\
51.01	0.01\\
52.01	0.01\\
53.01	0.01\\
54.01	0.01\\
55.01	0.01\\
56.01	0.01\\
57.01	0.01\\
58.01	0.01\\
59.01	0.01\\
60.01	0.01\\
61.01	0.01\\
62.01	0.01\\
63.01	0.01\\
64.01	0.01\\
65.01	0.01\\
66.01	0.01\\
67.01	0.01\\
68.01	0.01\\
69.01	0.01\\
70.01	0.01\\
71.01	0.01\\
72.01	0.01\\
73.01	0.01\\
74.01	0.01\\
75.01	0.01\\
76.01	0.01\\
77.01	0.01\\
78.01	0.01\\
79.01	0.01\\
80.01	0.01\\
81.01	0.01\\
82.01	0.01\\
83.01	0.01\\
84.01	0.01\\
85.01	0.01\\
86.01	0.01\\
87.01	0.01\\
88.01	0.01\\
89.01	0.01\\
90.01	0.01\\
91.01	0.01\\
92.01	0.01\\
93.01	0.01\\
94.01	0.01\\
95.01	0.01\\
96.01	0.01\\
97.01	0.01\\
98.01	0.01\\
99.01	0.01\\
100.01	0.01\\
101.01	0.01\\
102.01	0.01\\
103.01	0.01\\
104.01	0.01\\
105.01	0.01\\
106.01	0.01\\
107.01	0.01\\
108.01	0.01\\
109.01	0.01\\
110.01	0.01\\
111.01	0.01\\
112.01	0.01\\
113.01	0.01\\
114.01	0.01\\
115.01	0.01\\
116.01	0.01\\
117.01	0.01\\
118.01	0.01\\
119.01	0.01\\
120.01	0.01\\
121.01	0.01\\
122.01	0.01\\
123.01	0.01\\
124.01	0.01\\
125.01	0.01\\
126.01	0.01\\
127.01	0.01\\
128.01	0.01\\
129.01	0.01\\
130.01	0.01\\
131.01	0.01\\
132.01	0.01\\
133.01	0.01\\
134.01	0.01\\
135.01	0.01\\
136.01	0.01\\
137.01	0.01\\
138.01	0.01\\
139.01	0.01\\
140.01	0.01\\
141.01	0.01\\
142.01	0.01\\
143.01	0.01\\
144.01	0.01\\
145.01	0.01\\
146.01	0.01\\
147.01	0.01\\
148.01	0.01\\
149.01	0.01\\
150.01	0.01\\
151.01	0.01\\
152.01	0.01\\
153.01	0.01\\
154.01	0.01\\
155.01	0.01\\
156.01	0.01\\
157.01	0.01\\
158.01	0.01\\
159.01	0.01\\
160.01	0.01\\
161.01	0.01\\
162.01	0.01\\
163.01	0.01\\
164.01	0.01\\
165.01	0.01\\
166.01	0.01\\
167.01	0.01\\
168.01	0.01\\
169.01	0.01\\
170.01	0.01\\
171.01	0.01\\
172.01	0.01\\
173.01	0.01\\
174.01	0.01\\
175.01	0.01\\
176.01	0.01\\
177.01	0.01\\
178.01	0.01\\
179.01	0.01\\
180.01	0.01\\
181.01	0.01\\
182.01	0.01\\
183.01	0.01\\
184.01	0.01\\
185.01	0.01\\
186.01	0.01\\
187.01	0.01\\
188.01	0.01\\
189.01	0.01\\
190.01	0.01\\
191.01	0.01\\
192.01	0.01\\
193.01	0.01\\
194.01	0.01\\
195.01	0.01\\
196.01	0.01\\
197.01	0.01\\
198.01	0.01\\
199.01	0.01\\
200.01	0.01\\
201.01	0.01\\
202.01	0.01\\
203.01	0.01\\
204.01	0.01\\
205.01	0.01\\
206.01	0.01\\
207.01	0.01\\
208.01	0.01\\
209.01	0.01\\
210.01	0.01\\
211.01	0.01\\
212.01	0.01\\
213.01	0.01\\
214.01	0.01\\
215.01	0.01\\
216.01	0.01\\
217.01	0.01\\
218.01	0.01\\
219.01	0.01\\
220.01	0.01\\
221.01	0.01\\
222.01	0.01\\
223.01	0.01\\
224.01	0.01\\
225.01	0.01\\
226.01	0.01\\
227.01	0.01\\
228.01	0.01\\
229.01	0.01\\
230.01	0.01\\
231.01	0.01\\
232.01	0.01\\
233.01	0.01\\
234.01	0.01\\
235.01	0.01\\
236.01	0.01\\
237.01	0.01\\
238.01	0.01\\
239.01	0.01\\
240.01	0.01\\
241.01	0.01\\
242.01	0.01\\
243.01	0.01\\
244.01	0.01\\
245.01	0.01\\
246.01	0.01\\
247.01	0.01\\
248.01	0.01\\
249.01	0.01\\
250.01	0.01\\
251.01	0.01\\
252.01	0.01\\
253.01	0.01\\
254.01	0.01\\
255.01	0.01\\
256.01	0.01\\
257.01	0.01\\
258.01	0.01\\
259.01	0.01\\
260.01	0.01\\
261.01	0.01\\
262.01	0.01\\
263.01	0.01\\
264.01	0.01\\
265.01	0.01\\
266.01	0.01\\
267.01	0.01\\
268.01	0.01\\
269.01	0.01\\
270.01	0.01\\
271.01	0.01\\
272.01	0.01\\
273.01	0.01\\
274.01	0.01\\
275.01	0.01\\
276.01	0.01\\
277.01	0.01\\
278.01	0.01\\
279.01	0.01\\
280.01	0.01\\
281.01	0.01\\
282.01	0.01\\
283.01	0.01\\
284.01	0.01\\
285.01	0.01\\
286.01	0.01\\
287.01	0.01\\
288.01	0.01\\
289.01	0.01\\
290.01	0.01\\
291.01	0.01\\
292.01	0.01\\
293.01	0.01\\
294.01	0.01\\
295.01	0.01\\
296.01	0.01\\
297.01	0.01\\
298.01	0.01\\
299.01	0.01\\
300.01	0.01\\
301.01	0.01\\
302.01	0.01\\
303.01	0.01\\
304.01	0.01\\
305.01	0.01\\
306.01	0.01\\
307.01	0.01\\
308.01	0.01\\
309.01	0.01\\
310.01	0.01\\
311.01	0.01\\
312.01	0.01\\
313.01	0.01\\
314.01	0.01\\
315.01	0.01\\
316.01	0.01\\
317.01	0.01\\
318.01	0.01\\
319.01	0.01\\
320.01	0.01\\
321.01	0.01\\
322.01	0.01\\
323.01	0.01\\
324.01	0.01\\
325.01	0.01\\
326.01	0.01\\
327.01	0.01\\
328.01	0.01\\
329.01	0.01\\
330.01	0.01\\
331.01	0.01\\
332.01	0.01\\
333.01	0.01\\
334.01	0.01\\
335.01	0.01\\
336.01	0.01\\
337.01	0.01\\
338.01	0.01\\
339.01	0.01\\
340.01	0.01\\
341.01	0.01\\
342.01	0.01\\
343.01	0.01\\
344.01	0.01\\
345.01	0.01\\
346.01	0.01\\
347.01	0.01\\
348.01	0.01\\
349.01	0.01\\
350.01	0.01\\
351.01	0.01\\
352.01	0.01\\
353.01	0.01\\
354.01	0.01\\
355.01	0.01\\
356.01	0.01\\
357.01	0.01\\
358.01	0.01\\
359.01	0.01\\
360.01	0.01\\
361.01	0.01\\
362.01	0.01\\
363.01	0.01\\
364.01	0.01\\
365.01	0.01\\
366.01	0.01\\
367.01	0.01\\
368.01	0.01\\
369.01	0.01\\
370.01	0.01\\
371.01	0.01\\
372.01	0.01\\
373.01	0.01\\
374.01	0.01\\
375.01	0.01\\
376.01	0.01\\
377.01	0.01\\
378.01	0.01\\
379.01	0.01\\
380.01	0.01\\
381.01	0.01\\
382.01	0.01\\
383.01	0.01\\
384.01	0.01\\
385.01	0.01\\
386.01	0.01\\
387.01	0.01\\
388.01	0.01\\
389.01	0.01\\
390.01	0.01\\
391.01	0.01\\
392.01	0.01\\
393.01	0.01\\
394.01	0.01\\
395.01	0.01\\
396.01	0.01\\
397.01	0.01\\
398.01	0.01\\
399.01	0.01\\
400.01	0.01\\
401.01	0.01\\
402.01	0.01\\
403.01	0.01\\
404.01	0.01\\
405.01	0.01\\
406.01	0.01\\
407.01	0.01\\
408.01	0.01\\
409.01	0.01\\
410.01	0.01\\
411.01	0.01\\
412.01	0.01\\
413.01	0.01\\
414.01	0.01\\
415.01	0.01\\
416.01	0.01\\
417.01	0.01\\
418.01	0.01\\
419.01	0.01\\
420.01	0.01\\
421.01	0.01\\
422.01	0.01\\
423.01	0.01\\
424.01	0.01\\
425.01	0.01\\
426.01	0.01\\
427.01	0.01\\
428.01	0.01\\
429.01	0.01\\
430.01	0.01\\
431.01	0.01\\
432.01	0.01\\
433.01	0.01\\
434.01	0.01\\
435.01	0.01\\
436.01	0.01\\
437.01	0.01\\
438.01	0.01\\
439.01	0.01\\
440.01	0.01\\
441.01	0.01\\
442.01	0.01\\
443.01	0.01\\
444.01	0.01\\
445.01	0.01\\
446.01	0.01\\
447.01	0.01\\
448.01	0.01\\
449.01	0.01\\
450.01	0.01\\
451.01	0.01\\
452.01	0.01\\
453.01	0.01\\
454.01	0.01\\
455.01	0.01\\
456.01	0.01\\
457.01	0.01\\
458.01	0.01\\
459.01	0.01\\
460.01	0.01\\
461.01	0.01\\
462.01	0.01\\
463.01	0.01\\
464.01	0.01\\
465.01	0.01\\
466.01	0.01\\
467.01	0.01\\
468.01	0.01\\
469.01	0.01\\
470.01	0.01\\
471.01	0.01\\
472.01	0.01\\
473.01	0.01\\
474.01	0.01\\
475.01	0.01\\
476.01	0.01\\
477.01	0.01\\
478.01	0.01\\
479.01	0.01\\
480.01	0.01\\
481.01	0.01\\
482.01	0.01\\
483.01	0.01\\
484.01	0.01\\
485.01	0.01\\
486.01	0.01\\
487.01	0.01\\
488.01	0.01\\
489.01	0.01\\
490.01	0.01\\
491.01	0.01\\
492.01	0.01\\
493.01	0.01\\
494.01	0.01\\
495.01	0.01\\
496.01	0.01\\
497.01	0.01\\
498.01	0.01\\
499.01	0.01\\
500.01	0.01\\
501.01	0.01\\
502.01	0.01\\
503.01	0.01\\
504.01	0.01\\
505.01	0.01\\
506.01	0.01\\
507.01	0.01\\
508.01	0.01\\
509.01	0.01\\
510.01	0.01\\
511.01	0.01\\
512.01	0.01\\
513.01	0.01\\
514.01	0.01\\
515.01	0.01\\
516.01	0.01\\
517.01	0.01\\
518.01	0.01\\
519.01	0.01\\
520.01	0.01\\
521.01	0.01\\
522.01	0.01\\
523.01	0.01\\
524.01	0.01\\
525.01	0.01\\
526.01	0.01\\
527.01	0.01\\
528.01	0.01\\
529.01	0.01\\
530.01	0.01\\
531.01	0.01\\
532.01	0.01\\
533.01	0.01\\
534.01	0.01\\
535.01	0.01\\
536.01	0.01\\
537.01	0.01\\
538.01	0.01\\
539.01	0.01\\
540.01	0.01\\
541.01	0.01\\
542.01	0.01\\
543.01	0.01\\
544.01	0.01\\
545.01	0.01\\
546.01	0.01\\
547.01	0.01\\
548.01	0.01\\
549.01	0.01\\
550.01	0.01\\
551.01	0.01\\
552.01	0.01\\
553.01	0.01\\
554.01	0.01\\
555.01	0.01\\
556.01	0.01\\
557.01	0.01\\
558.01	0.01\\
559.01	0.01\\
560.01	0.01\\
561.01	0.01\\
562.01	0.01\\
563.01	0.01\\
564.01	0.01\\
565.01	0.01\\
566.01	0.01\\
567.01	0.01\\
568.01	0.01\\
569.01	0.01\\
570.01	0.01\\
571.01	0.01\\
572.01	0.01\\
573.01	0.01\\
574.01	0.01\\
575.01	0.01\\
576.01	0.01\\
577.01	0.01\\
578.01	0.01\\
579.01	0.01\\
580.01	0.01\\
581.01	0.01\\
582.01	0.01\\
583.01	0.01\\
584.01	0.01\\
585.01	0.01\\
586.01	0.01\\
587.01	0.01\\
588.01	0.01\\
589.01	0.01\\
590.01	0.01\\
591.01	0.01\\
592.01	0.01\\
593.01	0.01\\
594.01	0.01\\
595.01	0.01\\
596.01	0.01\\
597.01	0.01\\
598.01	0.01\\
599.01	0.01\\
599.02	0.01\\
599.03	0.01\\
599.04	0.01\\
599.05	0.01\\
599.06	0.01\\
599.07	0.01\\
599.08	0.01\\
599.09	0.01\\
599.1	0.01\\
599.11	0.01\\
599.12	0.01\\
599.13	0.01\\
599.14	0.01\\
599.15	0.01\\
599.16	0.01\\
599.17	0.01\\
599.18	0.01\\
599.19	0.01\\
599.2	0.01\\
599.21	0.01\\
599.22	0.01\\
599.23	0.01\\
599.24	0.01\\
599.25	0.01\\
599.26	0.01\\
599.27	0.01\\
599.28	0.01\\
599.29	0.01\\
599.3	0.01\\
599.31	0.01\\
599.32	0.01\\
599.33	0.01\\
599.34	0.01\\
599.35	0.01\\
599.36	0.01\\
599.37	0.01\\
599.38	0.01\\
599.39	0.01\\
599.4	0.01\\
599.41	0.01\\
599.42	0.01\\
599.43	0.01\\
599.44	0.01\\
599.45	0.01\\
599.46	0.01\\
599.47	0.01\\
599.48	0.01\\
599.49	0.01\\
599.5	0.01\\
599.51	0.01\\
599.52	0.01\\
599.53	0.01\\
599.54	0.01\\
599.55	0.01\\
599.56	0.01\\
599.57	0.01\\
599.58	0.01\\
599.59	0.01\\
599.6	0.01\\
599.61	0.01\\
599.62	0.01\\
599.63	0.01\\
599.64	0.01\\
599.65	0.01\\
599.66	0.01\\
599.67	0.01\\
599.68	0.01\\
599.69	0.01\\
599.7	0.01\\
599.71	0.01\\
599.72	0.01\\
599.73	0.01\\
599.74	0.01\\
599.75	0.01\\
599.76	0.01\\
599.77	0.01\\
599.78	0.01\\
599.79	0.01\\
599.8	0.01\\
599.81	0.01\\
599.82	0.01\\
599.83	0.01\\
599.84	0.01\\
599.85	0.01\\
599.86	0.01\\
599.87	0.01\\
599.88	0.01\\
599.89	0.01\\
599.9	0.01\\
599.91	0.01\\
599.92	0.01\\
599.93	0.01\\
599.94	0.01\\
599.95	0.01\\
599.96	0.01\\
599.97	0.01\\
599.98	0.01\\
599.99	0.01\\
600	0.01\\
};
\addplot [color=mycolor1,solid,forget plot]
  table[row sep=crcr]{%
0.01	0.01\\
1.01	0.01\\
2.01	0.01\\
3.01	0.01\\
4.01	0.01\\
5.01	0.01\\
6.01	0.01\\
7.01	0.01\\
8.01	0.01\\
9.01	0.01\\
10.01	0.01\\
11.01	0.01\\
12.01	0.01\\
13.01	0.01\\
14.01	0.01\\
15.01	0.01\\
16.01	0.01\\
17.01	0.01\\
18.01	0.01\\
19.01	0.01\\
20.01	0.01\\
21.01	0.01\\
22.01	0.01\\
23.01	0.01\\
24.01	0.01\\
25.01	0.01\\
26.01	0.01\\
27.01	0.01\\
28.01	0.01\\
29.01	0.01\\
30.01	0.01\\
31.01	0.01\\
32.01	0.01\\
33.01	0.01\\
34.01	0.01\\
35.01	0.01\\
36.01	0.01\\
37.01	0.01\\
38.01	0.01\\
39.01	0.01\\
40.01	0.01\\
41.01	0.01\\
42.01	0.01\\
43.01	0.01\\
44.01	0.01\\
45.01	0.01\\
46.01	0.01\\
47.01	0.01\\
48.01	0.01\\
49.01	0.01\\
50.01	0.01\\
51.01	0.01\\
52.01	0.01\\
53.01	0.01\\
54.01	0.01\\
55.01	0.01\\
56.01	0.01\\
57.01	0.01\\
58.01	0.01\\
59.01	0.01\\
60.01	0.01\\
61.01	0.01\\
62.01	0.01\\
63.01	0.01\\
64.01	0.01\\
65.01	0.01\\
66.01	0.01\\
67.01	0.01\\
68.01	0.01\\
69.01	0.01\\
70.01	0.01\\
71.01	0.01\\
72.01	0.01\\
73.01	0.01\\
74.01	0.01\\
75.01	0.01\\
76.01	0.01\\
77.01	0.01\\
78.01	0.01\\
79.01	0.01\\
80.01	0.01\\
81.01	0.01\\
82.01	0.01\\
83.01	0.01\\
84.01	0.01\\
85.01	0.01\\
86.01	0.01\\
87.01	0.01\\
88.01	0.01\\
89.01	0.01\\
90.01	0.01\\
91.01	0.01\\
92.01	0.01\\
93.01	0.01\\
94.01	0.01\\
95.01	0.01\\
96.01	0.01\\
97.01	0.01\\
98.01	0.01\\
99.01	0.01\\
100.01	0.01\\
101.01	0.01\\
102.01	0.01\\
103.01	0.01\\
104.01	0.01\\
105.01	0.01\\
106.01	0.01\\
107.01	0.01\\
108.01	0.01\\
109.01	0.01\\
110.01	0.01\\
111.01	0.01\\
112.01	0.01\\
113.01	0.01\\
114.01	0.01\\
115.01	0.01\\
116.01	0.01\\
117.01	0.01\\
118.01	0.01\\
119.01	0.01\\
120.01	0.01\\
121.01	0.01\\
122.01	0.01\\
123.01	0.01\\
124.01	0.01\\
125.01	0.01\\
126.01	0.01\\
127.01	0.01\\
128.01	0.01\\
129.01	0.01\\
130.01	0.01\\
131.01	0.01\\
132.01	0.01\\
133.01	0.01\\
134.01	0.01\\
135.01	0.01\\
136.01	0.01\\
137.01	0.01\\
138.01	0.01\\
139.01	0.01\\
140.01	0.01\\
141.01	0.01\\
142.01	0.01\\
143.01	0.01\\
144.01	0.01\\
145.01	0.01\\
146.01	0.01\\
147.01	0.01\\
148.01	0.01\\
149.01	0.01\\
150.01	0.01\\
151.01	0.01\\
152.01	0.01\\
153.01	0.01\\
154.01	0.01\\
155.01	0.01\\
156.01	0.01\\
157.01	0.01\\
158.01	0.01\\
159.01	0.01\\
160.01	0.01\\
161.01	0.01\\
162.01	0.01\\
163.01	0.01\\
164.01	0.01\\
165.01	0.01\\
166.01	0.01\\
167.01	0.01\\
168.01	0.01\\
169.01	0.01\\
170.01	0.01\\
171.01	0.01\\
172.01	0.01\\
173.01	0.01\\
174.01	0.01\\
175.01	0.01\\
176.01	0.01\\
177.01	0.01\\
178.01	0.01\\
179.01	0.01\\
180.01	0.01\\
181.01	0.01\\
182.01	0.01\\
183.01	0.01\\
184.01	0.01\\
185.01	0.01\\
186.01	0.01\\
187.01	0.01\\
188.01	0.01\\
189.01	0.01\\
190.01	0.01\\
191.01	0.01\\
192.01	0.01\\
193.01	0.01\\
194.01	0.01\\
195.01	0.01\\
196.01	0.01\\
197.01	0.01\\
198.01	0.01\\
199.01	0.01\\
200.01	0.01\\
201.01	0.01\\
202.01	0.01\\
203.01	0.01\\
204.01	0.01\\
205.01	0.01\\
206.01	0.01\\
207.01	0.01\\
208.01	0.01\\
209.01	0.01\\
210.01	0.01\\
211.01	0.01\\
212.01	0.01\\
213.01	0.01\\
214.01	0.01\\
215.01	0.01\\
216.01	0.01\\
217.01	0.01\\
218.01	0.01\\
219.01	0.01\\
220.01	0.01\\
221.01	0.01\\
222.01	0.01\\
223.01	0.01\\
224.01	0.01\\
225.01	0.01\\
226.01	0.01\\
227.01	0.01\\
228.01	0.01\\
229.01	0.01\\
230.01	0.01\\
231.01	0.01\\
232.01	0.01\\
233.01	0.01\\
234.01	0.01\\
235.01	0.01\\
236.01	0.01\\
237.01	0.01\\
238.01	0.01\\
239.01	0.01\\
240.01	0.01\\
241.01	0.01\\
242.01	0.01\\
243.01	0.01\\
244.01	0.01\\
245.01	0.01\\
246.01	0.01\\
247.01	0.01\\
248.01	0.01\\
249.01	0.01\\
250.01	0.01\\
251.01	0.01\\
252.01	0.01\\
253.01	0.01\\
254.01	0.01\\
255.01	0.01\\
256.01	0.01\\
257.01	0.01\\
258.01	0.01\\
259.01	0.01\\
260.01	0.01\\
261.01	0.01\\
262.01	0.01\\
263.01	0.01\\
264.01	0.01\\
265.01	0.01\\
266.01	0.01\\
267.01	0.01\\
268.01	0.01\\
269.01	0.01\\
270.01	0.01\\
271.01	0.01\\
272.01	0.01\\
273.01	0.01\\
274.01	0.01\\
275.01	0.01\\
276.01	0.01\\
277.01	0.01\\
278.01	0.01\\
279.01	0.01\\
280.01	0.01\\
281.01	0.01\\
282.01	0.01\\
283.01	0.01\\
284.01	0.01\\
285.01	0.01\\
286.01	0.01\\
287.01	0.01\\
288.01	0.01\\
289.01	0.01\\
290.01	0.01\\
291.01	0.01\\
292.01	0.01\\
293.01	0.01\\
294.01	0.01\\
295.01	0.01\\
296.01	0.01\\
297.01	0.01\\
298.01	0.01\\
299.01	0.01\\
300.01	0.01\\
301.01	0.01\\
302.01	0.01\\
303.01	0.01\\
304.01	0.01\\
305.01	0.01\\
306.01	0.01\\
307.01	0.01\\
308.01	0.01\\
309.01	0.01\\
310.01	0.01\\
311.01	0.01\\
312.01	0.01\\
313.01	0.01\\
314.01	0.01\\
315.01	0.01\\
316.01	0.01\\
317.01	0.01\\
318.01	0.01\\
319.01	0.01\\
320.01	0.01\\
321.01	0.01\\
322.01	0.01\\
323.01	0.01\\
324.01	0.01\\
325.01	0.01\\
326.01	0.01\\
327.01	0.01\\
328.01	0.01\\
329.01	0.01\\
330.01	0.01\\
331.01	0.01\\
332.01	0.01\\
333.01	0.01\\
334.01	0.01\\
335.01	0.01\\
336.01	0.01\\
337.01	0.01\\
338.01	0.01\\
339.01	0.01\\
340.01	0.01\\
341.01	0.01\\
342.01	0.01\\
343.01	0.01\\
344.01	0.01\\
345.01	0.01\\
346.01	0.01\\
347.01	0.01\\
348.01	0.01\\
349.01	0.01\\
350.01	0.01\\
351.01	0.01\\
352.01	0.01\\
353.01	0.01\\
354.01	0.01\\
355.01	0.01\\
356.01	0.01\\
357.01	0.01\\
358.01	0.01\\
359.01	0.01\\
360.01	0.01\\
361.01	0.01\\
362.01	0.01\\
363.01	0.01\\
364.01	0.01\\
365.01	0.01\\
366.01	0.01\\
367.01	0.01\\
368.01	0.01\\
369.01	0.01\\
370.01	0.01\\
371.01	0.01\\
372.01	0.01\\
373.01	0.01\\
374.01	0.01\\
375.01	0.01\\
376.01	0.01\\
377.01	0.01\\
378.01	0.01\\
379.01	0.01\\
380.01	0.01\\
381.01	0.01\\
382.01	0.01\\
383.01	0.01\\
384.01	0.01\\
385.01	0.01\\
386.01	0.01\\
387.01	0.01\\
388.01	0.01\\
389.01	0.01\\
390.01	0.01\\
391.01	0.01\\
392.01	0.01\\
393.01	0.01\\
394.01	0.01\\
395.01	0.01\\
396.01	0.01\\
397.01	0.01\\
398.01	0.01\\
399.01	0.01\\
400.01	0.01\\
401.01	0.01\\
402.01	0.01\\
403.01	0.01\\
404.01	0.01\\
405.01	0.01\\
406.01	0.01\\
407.01	0.01\\
408.01	0.01\\
409.01	0.01\\
410.01	0.01\\
411.01	0.01\\
412.01	0.01\\
413.01	0.01\\
414.01	0.01\\
415.01	0.01\\
416.01	0.01\\
417.01	0.01\\
418.01	0.01\\
419.01	0.01\\
420.01	0.01\\
421.01	0.01\\
422.01	0.01\\
423.01	0.01\\
424.01	0.01\\
425.01	0.01\\
426.01	0.01\\
427.01	0.01\\
428.01	0.01\\
429.01	0.01\\
430.01	0.01\\
431.01	0.01\\
432.01	0.01\\
433.01	0.01\\
434.01	0.01\\
435.01	0.01\\
436.01	0.01\\
437.01	0.01\\
438.01	0.01\\
439.01	0.01\\
440.01	0.01\\
441.01	0.01\\
442.01	0.01\\
443.01	0.01\\
444.01	0.01\\
445.01	0.01\\
446.01	0.01\\
447.01	0.01\\
448.01	0.01\\
449.01	0.01\\
450.01	0.01\\
451.01	0.01\\
452.01	0.01\\
453.01	0.01\\
454.01	0.01\\
455.01	0.01\\
456.01	0.01\\
457.01	0.01\\
458.01	0.01\\
459.01	0.01\\
460.01	0.01\\
461.01	0.01\\
462.01	0.01\\
463.01	0.01\\
464.01	0.01\\
465.01	0.01\\
466.01	0.01\\
467.01	0.01\\
468.01	0.01\\
469.01	0.01\\
470.01	0.01\\
471.01	0.01\\
472.01	0.01\\
473.01	0.01\\
474.01	0.01\\
475.01	0.01\\
476.01	0.01\\
477.01	0.01\\
478.01	0.01\\
479.01	0.01\\
480.01	0.01\\
481.01	0.01\\
482.01	0.01\\
483.01	0.01\\
484.01	0.01\\
485.01	0.01\\
486.01	0.01\\
487.01	0.01\\
488.01	0.01\\
489.01	0.01\\
490.01	0.01\\
491.01	0.01\\
492.01	0.01\\
493.01	0.01\\
494.01	0.01\\
495.01	0.01\\
496.01	0.01\\
497.01	0.01\\
498.01	0.01\\
499.01	0.01\\
500.01	0.01\\
501.01	0.01\\
502.01	0.01\\
503.01	0.01\\
504.01	0.01\\
505.01	0.01\\
506.01	0.01\\
507.01	0.01\\
508.01	0.01\\
509.01	0.01\\
510.01	0.01\\
511.01	0.01\\
512.01	0.01\\
513.01	0.01\\
514.01	0.01\\
515.01	0.01\\
516.01	0.01\\
517.01	0.01\\
518.01	0.01\\
519.01	0.01\\
520.01	0.01\\
521.01	0.01\\
522.01	0.01\\
523.01	0.01\\
524.01	0.01\\
525.01	0.01\\
526.01	0.01\\
527.01	0.01\\
528.01	0.01\\
529.01	0.01\\
530.01	0.01\\
531.01	0.01\\
532.01	0.01\\
533.01	0.01\\
534.01	0.01\\
535.01	0.01\\
536.01	0.01\\
537.01	0.01\\
538.01	0.01\\
539.01	0.01\\
540.01	0.01\\
541.01	0.01\\
542.01	0.01\\
543.01	0.01\\
544.01	0.01\\
545.01	0.01\\
546.01	0.01\\
547.01	0.01\\
548.01	0.01\\
549.01	0.01\\
550.01	0.01\\
551.01	0.01\\
552.01	0.01\\
553.01	0.01\\
554.01	0.01\\
555.01	0.01\\
556.01	0.01\\
557.01	0.01\\
558.01	0.01\\
559.01	0.01\\
560.01	0.01\\
561.01	0.01\\
562.01	0.01\\
563.01	0.01\\
564.01	0.01\\
565.01	0.01\\
566.01	0.01\\
567.01	0.01\\
568.01	0.01\\
569.01	0.01\\
570.01	0.01\\
571.01	0.01\\
572.01	0.01\\
573.01	0.01\\
574.01	0.01\\
575.01	0.01\\
576.01	0.01\\
577.01	0.01\\
578.01	0.01\\
579.01	0.01\\
580.01	0.01\\
581.01	0.01\\
582.01	0.01\\
583.01	0.01\\
584.01	0.01\\
585.01	0.01\\
586.01	0.01\\
587.01	0.01\\
588.01	0.01\\
589.01	0.01\\
590.01	0.01\\
591.01	0.01\\
592.01	0.01\\
593.01	0.01\\
594.01	0.01\\
595.01	0.01\\
596.01	0.01\\
597.01	0.01\\
598.01	0.01\\
599.01	0.01\\
599.02	0.01\\
599.03	0.01\\
599.04	0.01\\
599.05	0.01\\
599.06	0.01\\
599.07	0.01\\
599.08	0.01\\
599.09	0.01\\
599.1	0.01\\
599.11	0.01\\
599.12	0.01\\
599.13	0.01\\
599.14	0.01\\
599.15	0.01\\
599.16	0.01\\
599.17	0.01\\
599.18	0.01\\
599.19	0.01\\
599.2	0.01\\
599.21	0.01\\
599.22	0.01\\
599.23	0.01\\
599.24	0.01\\
599.25	0.01\\
599.26	0.01\\
599.27	0.01\\
599.28	0.01\\
599.29	0.01\\
599.3	0.01\\
599.31	0.01\\
599.32	0.01\\
599.33	0.01\\
599.34	0.01\\
599.35	0.01\\
599.36	0.01\\
599.37	0.01\\
599.38	0.01\\
599.39	0.01\\
599.4	0.01\\
599.41	0.01\\
599.42	0.01\\
599.43	0.01\\
599.44	0.01\\
599.45	0.01\\
599.46	0.01\\
599.47	0.01\\
599.48	0.01\\
599.49	0.01\\
599.5	0.01\\
599.51	0.01\\
599.52	0.01\\
599.53	0.01\\
599.54	0.01\\
599.55	0.01\\
599.56	0.01\\
599.57	0.01\\
599.58	0.01\\
599.59	0.01\\
599.6	0.01\\
599.61	0.01\\
599.62	0.01\\
599.63	0.01\\
599.64	0.01\\
599.65	0.01\\
599.66	0.01\\
599.67	0.01\\
599.68	0.01\\
599.69	0.01\\
599.7	0.01\\
599.71	0.01\\
599.72	0.01\\
599.73	0.01\\
599.74	0.01\\
599.75	0.01\\
599.76	0.01\\
599.77	0.01\\
599.78	0.01\\
599.79	0.01\\
599.8	0.01\\
599.81	0.01\\
599.82	0.01\\
599.83	0.01\\
599.84	0.01\\
599.85	0.01\\
599.86	0.01\\
599.87	0.01\\
599.88	0.01\\
599.89	0.01\\
599.9	0.01\\
599.91	0.01\\
599.92	0.01\\
599.93	0.01\\
599.94	0.01\\
599.95	0.01\\
599.96	0.01\\
599.97	0.01\\
599.98	0.01\\
599.99	0.01\\
600	0.01\\
};
\addplot [color=mycolor2,solid,forget plot]
  table[row sep=crcr]{%
0.01	0.01\\
1.01	0.01\\
2.01	0.01\\
3.01	0.01\\
4.01	0.01\\
5.01	0.01\\
6.01	0.01\\
7.01	0.01\\
8.01	0.01\\
9.01	0.01\\
10.01	0.01\\
11.01	0.01\\
12.01	0.01\\
13.01	0.01\\
14.01	0.01\\
15.01	0.01\\
16.01	0.01\\
17.01	0.01\\
18.01	0.01\\
19.01	0.01\\
20.01	0.01\\
21.01	0.01\\
22.01	0.01\\
23.01	0.01\\
24.01	0.01\\
25.01	0.01\\
26.01	0.01\\
27.01	0.01\\
28.01	0.01\\
29.01	0.01\\
30.01	0.01\\
31.01	0.01\\
32.01	0.01\\
33.01	0.01\\
34.01	0.01\\
35.01	0.01\\
36.01	0.01\\
37.01	0.01\\
38.01	0.01\\
39.01	0.01\\
40.01	0.01\\
41.01	0.01\\
42.01	0.01\\
43.01	0.01\\
44.01	0.01\\
45.01	0.01\\
46.01	0.01\\
47.01	0.01\\
48.01	0.01\\
49.01	0.01\\
50.01	0.01\\
51.01	0.01\\
52.01	0.01\\
53.01	0.01\\
54.01	0.01\\
55.01	0.01\\
56.01	0.01\\
57.01	0.01\\
58.01	0.01\\
59.01	0.01\\
60.01	0.01\\
61.01	0.01\\
62.01	0.01\\
63.01	0.01\\
64.01	0.01\\
65.01	0.01\\
66.01	0.01\\
67.01	0.01\\
68.01	0.01\\
69.01	0.01\\
70.01	0.01\\
71.01	0.01\\
72.01	0.01\\
73.01	0.01\\
74.01	0.01\\
75.01	0.01\\
76.01	0.01\\
77.01	0.01\\
78.01	0.01\\
79.01	0.01\\
80.01	0.01\\
81.01	0.01\\
82.01	0.01\\
83.01	0.01\\
84.01	0.01\\
85.01	0.01\\
86.01	0.01\\
87.01	0.01\\
88.01	0.01\\
89.01	0.01\\
90.01	0.01\\
91.01	0.01\\
92.01	0.01\\
93.01	0.01\\
94.01	0.01\\
95.01	0.01\\
96.01	0.01\\
97.01	0.01\\
98.01	0.01\\
99.01	0.01\\
100.01	0.01\\
101.01	0.01\\
102.01	0.01\\
103.01	0.01\\
104.01	0.01\\
105.01	0.01\\
106.01	0.01\\
107.01	0.01\\
108.01	0.01\\
109.01	0.01\\
110.01	0.01\\
111.01	0.01\\
112.01	0.01\\
113.01	0.01\\
114.01	0.01\\
115.01	0.01\\
116.01	0.01\\
117.01	0.01\\
118.01	0.01\\
119.01	0.01\\
120.01	0.01\\
121.01	0.01\\
122.01	0.01\\
123.01	0.01\\
124.01	0.01\\
125.01	0.01\\
126.01	0.01\\
127.01	0.01\\
128.01	0.01\\
129.01	0.01\\
130.01	0.01\\
131.01	0.01\\
132.01	0.01\\
133.01	0.01\\
134.01	0.01\\
135.01	0.01\\
136.01	0.01\\
137.01	0.01\\
138.01	0.01\\
139.01	0.01\\
140.01	0.01\\
141.01	0.01\\
142.01	0.01\\
143.01	0.01\\
144.01	0.01\\
145.01	0.01\\
146.01	0.01\\
147.01	0.01\\
148.01	0.01\\
149.01	0.01\\
150.01	0.01\\
151.01	0.01\\
152.01	0.01\\
153.01	0.01\\
154.01	0.01\\
155.01	0.01\\
156.01	0.01\\
157.01	0.01\\
158.01	0.01\\
159.01	0.01\\
160.01	0.01\\
161.01	0.01\\
162.01	0.01\\
163.01	0.01\\
164.01	0.01\\
165.01	0.01\\
166.01	0.01\\
167.01	0.01\\
168.01	0.01\\
169.01	0.01\\
170.01	0.01\\
171.01	0.01\\
172.01	0.01\\
173.01	0.01\\
174.01	0.01\\
175.01	0.01\\
176.01	0.01\\
177.01	0.01\\
178.01	0.01\\
179.01	0.01\\
180.01	0.01\\
181.01	0.01\\
182.01	0.01\\
183.01	0.01\\
184.01	0.01\\
185.01	0.01\\
186.01	0.01\\
187.01	0.01\\
188.01	0.01\\
189.01	0.01\\
190.01	0.01\\
191.01	0.01\\
192.01	0.01\\
193.01	0.01\\
194.01	0.01\\
195.01	0.01\\
196.01	0.01\\
197.01	0.01\\
198.01	0.01\\
199.01	0.01\\
200.01	0.01\\
201.01	0.01\\
202.01	0.01\\
203.01	0.01\\
204.01	0.01\\
205.01	0.01\\
206.01	0.01\\
207.01	0.01\\
208.01	0.01\\
209.01	0.01\\
210.01	0.01\\
211.01	0.01\\
212.01	0.01\\
213.01	0.01\\
214.01	0.01\\
215.01	0.01\\
216.01	0.01\\
217.01	0.01\\
218.01	0.01\\
219.01	0.01\\
220.01	0.01\\
221.01	0.01\\
222.01	0.01\\
223.01	0.01\\
224.01	0.01\\
225.01	0.01\\
226.01	0.01\\
227.01	0.01\\
228.01	0.01\\
229.01	0.01\\
230.01	0.01\\
231.01	0.01\\
232.01	0.01\\
233.01	0.01\\
234.01	0.01\\
235.01	0.01\\
236.01	0.01\\
237.01	0.01\\
238.01	0.01\\
239.01	0.01\\
240.01	0.01\\
241.01	0.01\\
242.01	0.01\\
243.01	0.01\\
244.01	0.01\\
245.01	0.01\\
246.01	0.01\\
247.01	0.01\\
248.01	0.01\\
249.01	0.01\\
250.01	0.01\\
251.01	0.01\\
252.01	0.01\\
253.01	0.01\\
254.01	0.01\\
255.01	0.01\\
256.01	0.01\\
257.01	0.01\\
258.01	0.01\\
259.01	0.01\\
260.01	0.01\\
261.01	0.01\\
262.01	0.01\\
263.01	0.01\\
264.01	0.01\\
265.01	0.01\\
266.01	0.01\\
267.01	0.01\\
268.01	0.01\\
269.01	0.01\\
270.01	0.01\\
271.01	0.01\\
272.01	0.01\\
273.01	0.01\\
274.01	0.01\\
275.01	0.01\\
276.01	0.01\\
277.01	0.01\\
278.01	0.01\\
279.01	0.01\\
280.01	0.01\\
281.01	0.01\\
282.01	0.01\\
283.01	0.01\\
284.01	0.01\\
285.01	0.01\\
286.01	0.01\\
287.01	0.01\\
288.01	0.01\\
289.01	0.01\\
290.01	0.01\\
291.01	0.01\\
292.01	0.01\\
293.01	0.01\\
294.01	0.01\\
295.01	0.01\\
296.01	0.01\\
297.01	0.01\\
298.01	0.01\\
299.01	0.01\\
300.01	0.01\\
301.01	0.01\\
302.01	0.01\\
303.01	0.01\\
304.01	0.01\\
305.01	0.01\\
306.01	0.01\\
307.01	0.01\\
308.01	0.01\\
309.01	0.01\\
310.01	0.01\\
311.01	0.01\\
312.01	0.01\\
313.01	0.01\\
314.01	0.01\\
315.01	0.01\\
316.01	0.01\\
317.01	0.01\\
318.01	0.01\\
319.01	0.01\\
320.01	0.01\\
321.01	0.01\\
322.01	0.01\\
323.01	0.01\\
324.01	0.01\\
325.01	0.01\\
326.01	0.01\\
327.01	0.01\\
328.01	0.01\\
329.01	0.01\\
330.01	0.01\\
331.01	0.01\\
332.01	0.01\\
333.01	0.01\\
334.01	0.01\\
335.01	0.01\\
336.01	0.01\\
337.01	0.01\\
338.01	0.01\\
339.01	0.01\\
340.01	0.01\\
341.01	0.01\\
342.01	0.01\\
343.01	0.01\\
344.01	0.01\\
345.01	0.01\\
346.01	0.01\\
347.01	0.01\\
348.01	0.01\\
349.01	0.01\\
350.01	0.01\\
351.01	0.01\\
352.01	0.01\\
353.01	0.01\\
354.01	0.01\\
355.01	0.01\\
356.01	0.01\\
357.01	0.01\\
358.01	0.01\\
359.01	0.01\\
360.01	0.01\\
361.01	0.01\\
362.01	0.01\\
363.01	0.01\\
364.01	0.01\\
365.01	0.01\\
366.01	0.01\\
367.01	0.01\\
368.01	0.01\\
369.01	0.01\\
370.01	0.01\\
371.01	0.01\\
372.01	0.01\\
373.01	0.01\\
374.01	0.01\\
375.01	0.01\\
376.01	0.01\\
377.01	0.01\\
378.01	0.01\\
379.01	0.01\\
380.01	0.01\\
381.01	0.01\\
382.01	0.01\\
383.01	0.01\\
384.01	0.01\\
385.01	0.01\\
386.01	0.01\\
387.01	0.01\\
388.01	0.01\\
389.01	0.01\\
390.01	0.01\\
391.01	0.01\\
392.01	0.01\\
393.01	0.01\\
394.01	0.01\\
395.01	0.01\\
396.01	0.01\\
397.01	0.01\\
398.01	0.01\\
399.01	0.01\\
400.01	0.01\\
401.01	0.01\\
402.01	0.01\\
403.01	0.01\\
404.01	0.01\\
405.01	0.01\\
406.01	0.01\\
407.01	0.01\\
408.01	0.01\\
409.01	0.01\\
410.01	0.01\\
411.01	0.01\\
412.01	0.01\\
413.01	0.01\\
414.01	0.01\\
415.01	0.01\\
416.01	0.01\\
417.01	0.01\\
418.01	0.01\\
419.01	0.01\\
420.01	0.01\\
421.01	0.01\\
422.01	0.01\\
423.01	0.01\\
424.01	0.01\\
425.01	0.01\\
426.01	0.01\\
427.01	0.01\\
428.01	0.01\\
429.01	0.01\\
430.01	0.01\\
431.01	0.01\\
432.01	0.01\\
433.01	0.01\\
434.01	0.01\\
435.01	0.01\\
436.01	0.01\\
437.01	0.01\\
438.01	0.01\\
439.01	0.01\\
440.01	0.01\\
441.01	0.01\\
442.01	0.01\\
443.01	0.01\\
444.01	0.01\\
445.01	0.01\\
446.01	0.01\\
447.01	0.01\\
448.01	0.01\\
449.01	0.01\\
450.01	0.01\\
451.01	0.01\\
452.01	0.01\\
453.01	0.01\\
454.01	0.01\\
455.01	0.01\\
456.01	0.01\\
457.01	0.01\\
458.01	0.01\\
459.01	0.01\\
460.01	0.01\\
461.01	0.01\\
462.01	0.01\\
463.01	0.01\\
464.01	0.01\\
465.01	0.01\\
466.01	0.01\\
467.01	0.01\\
468.01	0.01\\
469.01	0.01\\
470.01	0.01\\
471.01	0.01\\
472.01	0.01\\
473.01	0.01\\
474.01	0.01\\
475.01	0.01\\
476.01	0.01\\
477.01	0.01\\
478.01	0.01\\
479.01	0.01\\
480.01	0.01\\
481.01	0.01\\
482.01	0.01\\
483.01	0.01\\
484.01	0.01\\
485.01	0.01\\
486.01	0.01\\
487.01	0.01\\
488.01	0.01\\
489.01	0.01\\
490.01	0.01\\
491.01	0.01\\
492.01	0.01\\
493.01	0.01\\
494.01	0.01\\
495.01	0.01\\
496.01	0.01\\
497.01	0.01\\
498.01	0.01\\
499.01	0.01\\
500.01	0.01\\
501.01	0.01\\
502.01	0.01\\
503.01	0.01\\
504.01	0.01\\
505.01	0.01\\
506.01	0.01\\
507.01	0.01\\
508.01	0.01\\
509.01	0.01\\
510.01	0.01\\
511.01	0.01\\
512.01	0.01\\
513.01	0.01\\
514.01	0.01\\
515.01	0.01\\
516.01	0.01\\
517.01	0.01\\
518.01	0.01\\
519.01	0.01\\
520.01	0.01\\
521.01	0.01\\
522.01	0.01\\
523.01	0.01\\
524.01	0.01\\
525.01	0.01\\
526.01	0.01\\
527.01	0.01\\
528.01	0.01\\
529.01	0.01\\
530.01	0.01\\
531.01	0.01\\
532.01	0.01\\
533.01	0.01\\
534.01	0.01\\
535.01	0.01\\
536.01	0.01\\
537.01	0.01\\
538.01	0.01\\
539.01	0.01\\
540.01	0.01\\
541.01	0.01\\
542.01	0.01\\
543.01	0.01\\
544.01	0.01\\
545.01	0.01\\
546.01	0.01\\
547.01	0.01\\
548.01	0.01\\
549.01	0.01\\
550.01	0.01\\
551.01	0.01\\
552.01	0.01\\
553.01	0.01\\
554.01	0.01\\
555.01	0.01\\
556.01	0.01\\
557.01	0.01\\
558.01	0.01\\
559.01	0.01\\
560.01	0.01\\
561.01	0.01\\
562.01	0.01\\
563.01	0.01\\
564.01	0.01\\
565.01	0.01\\
566.01	0.01\\
567.01	0.01\\
568.01	0.01\\
569.01	0.01\\
570.01	0.01\\
571.01	0.01\\
572.01	0.01\\
573.01	0.01\\
574.01	0.01\\
575.01	0.01\\
576.01	0.01\\
577.01	0.01\\
578.01	0.01\\
579.01	0.01\\
580.01	0.01\\
581.01	0.01\\
582.01	0.01\\
583.01	0.01\\
584.01	0.01\\
585.01	0.01\\
586.01	0.01\\
587.01	0.01\\
588.01	0.01\\
589.01	0.01\\
590.01	0.01\\
591.01	0.01\\
592.01	0.01\\
593.01	0.01\\
594.01	0.01\\
595.01	0.01\\
596.01	0.01\\
597.01	0.01\\
598.01	0.01\\
599.01	0.01\\
599.02	0.01\\
599.03	0.01\\
599.04	0.01\\
599.05	0.01\\
599.06	0.01\\
599.07	0.01\\
599.08	0.01\\
599.09	0.01\\
599.1	0.01\\
599.11	0.01\\
599.12	0.01\\
599.13	0.01\\
599.14	0.01\\
599.15	0.01\\
599.16	0.01\\
599.17	0.01\\
599.18	0.01\\
599.19	0.01\\
599.2	0.01\\
599.21	0.01\\
599.22	0.01\\
599.23	0.01\\
599.24	0.01\\
599.25	0.01\\
599.26	0.01\\
599.27	0.01\\
599.28	0.01\\
599.29	0.01\\
599.3	0.01\\
599.31	0.01\\
599.32	0.01\\
599.33	0.01\\
599.34	0.01\\
599.35	0.01\\
599.36	0.01\\
599.37	0.01\\
599.38	0.01\\
599.39	0.01\\
599.4	0.01\\
599.41	0.01\\
599.42	0.01\\
599.43	0.01\\
599.44	0.01\\
599.45	0.01\\
599.46	0.01\\
599.47	0.01\\
599.48	0.01\\
599.49	0.01\\
599.5	0.01\\
599.51	0.01\\
599.52	0.01\\
599.53	0.01\\
599.54	0.01\\
599.55	0.01\\
599.56	0.01\\
599.57	0.01\\
599.58	0.01\\
599.59	0.01\\
599.6	0.01\\
599.61	0.01\\
599.62	0.01\\
599.63	0.01\\
599.64	0.01\\
599.65	0.01\\
599.66	0.01\\
599.67	0.01\\
599.68	0.01\\
599.69	0.01\\
599.7	0.01\\
599.71	0.01\\
599.72	0.01\\
599.73	0.01\\
599.74	0.01\\
599.75	0.01\\
599.76	0.01\\
599.77	0.01\\
599.78	0.01\\
599.79	0.01\\
599.8	0.01\\
599.81	0.01\\
599.82	0.01\\
599.83	0.01\\
599.84	0.01\\
599.85	0.01\\
599.86	0.01\\
599.87	0.01\\
599.88	0.01\\
599.89	0.01\\
599.9	0.01\\
599.91	0.01\\
599.92	0.01\\
599.93	0.01\\
599.94	0.01\\
599.95	0.01\\
599.96	0.01\\
599.97	0.01\\
599.98	0.01\\
599.99	0.01\\
600	0.01\\
};
\addplot [color=mycolor3,solid,forget plot]
  table[row sep=crcr]{%
0.01	0.01\\
1.01	0.01\\
2.01	0.01\\
3.01	0.01\\
4.01	0.01\\
5.01	0.01\\
6.01	0.01\\
7.01	0.01\\
8.01	0.01\\
9.01	0.01\\
10.01	0.01\\
11.01	0.01\\
12.01	0.01\\
13.01	0.01\\
14.01	0.01\\
15.01	0.01\\
16.01	0.01\\
17.01	0.01\\
18.01	0.01\\
19.01	0.01\\
20.01	0.01\\
21.01	0.01\\
22.01	0.01\\
23.01	0.01\\
24.01	0.01\\
25.01	0.01\\
26.01	0.01\\
27.01	0.01\\
28.01	0.01\\
29.01	0.01\\
30.01	0.01\\
31.01	0.01\\
32.01	0.01\\
33.01	0.01\\
34.01	0.01\\
35.01	0.01\\
36.01	0.01\\
37.01	0.01\\
38.01	0.01\\
39.01	0.01\\
40.01	0.01\\
41.01	0.01\\
42.01	0.01\\
43.01	0.01\\
44.01	0.01\\
45.01	0.01\\
46.01	0.01\\
47.01	0.01\\
48.01	0.01\\
49.01	0.01\\
50.01	0.01\\
51.01	0.01\\
52.01	0.01\\
53.01	0.01\\
54.01	0.01\\
55.01	0.01\\
56.01	0.01\\
57.01	0.01\\
58.01	0.01\\
59.01	0.01\\
60.01	0.01\\
61.01	0.01\\
62.01	0.01\\
63.01	0.01\\
64.01	0.01\\
65.01	0.01\\
66.01	0.01\\
67.01	0.01\\
68.01	0.01\\
69.01	0.01\\
70.01	0.01\\
71.01	0.01\\
72.01	0.01\\
73.01	0.01\\
74.01	0.01\\
75.01	0.01\\
76.01	0.01\\
77.01	0.01\\
78.01	0.01\\
79.01	0.01\\
80.01	0.01\\
81.01	0.01\\
82.01	0.01\\
83.01	0.01\\
84.01	0.01\\
85.01	0.01\\
86.01	0.01\\
87.01	0.01\\
88.01	0.01\\
89.01	0.01\\
90.01	0.01\\
91.01	0.01\\
92.01	0.01\\
93.01	0.01\\
94.01	0.01\\
95.01	0.01\\
96.01	0.01\\
97.01	0.01\\
98.01	0.01\\
99.01	0.01\\
100.01	0.01\\
101.01	0.01\\
102.01	0.01\\
103.01	0.01\\
104.01	0.01\\
105.01	0.01\\
106.01	0.01\\
107.01	0.01\\
108.01	0.01\\
109.01	0.01\\
110.01	0.01\\
111.01	0.01\\
112.01	0.01\\
113.01	0.01\\
114.01	0.01\\
115.01	0.01\\
116.01	0.01\\
117.01	0.01\\
118.01	0.01\\
119.01	0.01\\
120.01	0.01\\
121.01	0.01\\
122.01	0.01\\
123.01	0.01\\
124.01	0.01\\
125.01	0.01\\
126.01	0.01\\
127.01	0.01\\
128.01	0.01\\
129.01	0.01\\
130.01	0.01\\
131.01	0.01\\
132.01	0.01\\
133.01	0.01\\
134.01	0.01\\
135.01	0.01\\
136.01	0.01\\
137.01	0.01\\
138.01	0.01\\
139.01	0.01\\
140.01	0.01\\
141.01	0.01\\
142.01	0.01\\
143.01	0.01\\
144.01	0.01\\
145.01	0.01\\
146.01	0.01\\
147.01	0.01\\
148.01	0.01\\
149.01	0.01\\
150.01	0.01\\
151.01	0.01\\
152.01	0.01\\
153.01	0.01\\
154.01	0.01\\
155.01	0.01\\
156.01	0.01\\
157.01	0.01\\
158.01	0.01\\
159.01	0.01\\
160.01	0.01\\
161.01	0.01\\
162.01	0.01\\
163.01	0.01\\
164.01	0.01\\
165.01	0.01\\
166.01	0.01\\
167.01	0.01\\
168.01	0.01\\
169.01	0.01\\
170.01	0.01\\
171.01	0.01\\
172.01	0.01\\
173.01	0.01\\
174.01	0.01\\
175.01	0.01\\
176.01	0.01\\
177.01	0.01\\
178.01	0.01\\
179.01	0.01\\
180.01	0.01\\
181.01	0.01\\
182.01	0.01\\
183.01	0.01\\
184.01	0.01\\
185.01	0.01\\
186.01	0.01\\
187.01	0.01\\
188.01	0.01\\
189.01	0.01\\
190.01	0.01\\
191.01	0.01\\
192.01	0.01\\
193.01	0.01\\
194.01	0.01\\
195.01	0.01\\
196.01	0.01\\
197.01	0.01\\
198.01	0.01\\
199.01	0.01\\
200.01	0.01\\
201.01	0.01\\
202.01	0.01\\
203.01	0.01\\
204.01	0.01\\
205.01	0.01\\
206.01	0.01\\
207.01	0.01\\
208.01	0.01\\
209.01	0.01\\
210.01	0.01\\
211.01	0.01\\
212.01	0.01\\
213.01	0.01\\
214.01	0.01\\
215.01	0.01\\
216.01	0.01\\
217.01	0.01\\
218.01	0.01\\
219.01	0.01\\
220.01	0.01\\
221.01	0.01\\
222.01	0.01\\
223.01	0.01\\
224.01	0.01\\
225.01	0.01\\
226.01	0.01\\
227.01	0.01\\
228.01	0.01\\
229.01	0.01\\
230.01	0.01\\
231.01	0.01\\
232.01	0.01\\
233.01	0.01\\
234.01	0.01\\
235.01	0.01\\
236.01	0.01\\
237.01	0.01\\
238.01	0.01\\
239.01	0.01\\
240.01	0.01\\
241.01	0.01\\
242.01	0.01\\
243.01	0.01\\
244.01	0.01\\
245.01	0.01\\
246.01	0.01\\
247.01	0.01\\
248.01	0.01\\
249.01	0.01\\
250.01	0.01\\
251.01	0.01\\
252.01	0.01\\
253.01	0.01\\
254.01	0.01\\
255.01	0.01\\
256.01	0.01\\
257.01	0.01\\
258.01	0.01\\
259.01	0.01\\
260.01	0.01\\
261.01	0.01\\
262.01	0.01\\
263.01	0.01\\
264.01	0.01\\
265.01	0.01\\
266.01	0.01\\
267.01	0.01\\
268.01	0.01\\
269.01	0.01\\
270.01	0.01\\
271.01	0.01\\
272.01	0.01\\
273.01	0.01\\
274.01	0.01\\
275.01	0.01\\
276.01	0.01\\
277.01	0.01\\
278.01	0.01\\
279.01	0.01\\
280.01	0.01\\
281.01	0.01\\
282.01	0.01\\
283.01	0.01\\
284.01	0.01\\
285.01	0.01\\
286.01	0.01\\
287.01	0.01\\
288.01	0.01\\
289.01	0.01\\
290.01	0.01\\
291.01	0.01\\
292.01	0.01\\
293.01	0.01\\
294.01	0.01\\
295.01	0.01\\
296.01	0.01\\
297.01	0.01\\
298.01	0.01\\
299.01	0.01\\
300.01	0.01\\
301.01	0.01\\
302.01	0.01\\
303.01	0.01\\
304.01	0.01\\
305.01	0.01\\
306.01	0.01\\
307.01	0.01\\
308.01	0.01\\
309.01	0.01\\
310.01	0.01\\
311.01	0.01\\
312.01	0.01\\
313.01	0.01\\
314.01	0.01\\
315.01	0.01\\
316.01	0.01\\
317.01	0.01\\
318.01	0.01\\
319.01	0.01\\
320.01	0.01\\
321.01	0.01\\
322.01	0.01\\
323.01	0.01\\
324.01	0.01\\
325.01	0.01\\
326.01	0.01\\
327.01	0.01\\
328.01	0.01\\
329.01	0.01\\
330.01	0.01\\
331.01	0.01\\
332.01	0.01\\
333.01	0.01\\
334.01	0.01\\
335.01	0.01\\
336.01	0.01\\
337.01	0.01\\
338.01	0.01\\
339.01	0.01\\
340.01	0.01\\
341.01	0.01\\
342.01	0.01\\
343.01	0.01\\
344.01	0.01\\
345.01	0.01\\
346.01	0.01\\
347.01	0.01\\
348.01	0.01\\
349.01	0.01\\
350.01	0.01\\
351.01	0.01\\
352.01	0.01\\
353.01	0.01\\
354.01	0.01\\
355.01	0.01\\
356.01	0.01\\
357.01	0.01\\
358.01	0.01\\
359.01	0.01\\
360.01	0.01\\
361.01	0.01\\
362.01	0.01\\
363.01	0.01\\
364.01	0.01\\
365.01	0.01\\
366.01	0.01\\
367.01	0.01\\
368.01	0.01\\
369.01	0.01\\
370.01	0.01\\
371.01	0.01\\
372.01	0.01\\
373.01	0.01\\
374.01	0.01\\
375.01	0.01\\
376.01	0.01\\
377.01	0.01\\
378.01	0.01\\
379.01	0.01\\
380.01	0.01\\
381.01	0.01\\
382.01	0.01\\
383.01	0.01\\
384.01	0.01\\
385.01	0.01\\
386.01	0.01\\
387.01	0.01\\
388.01	0.01\\
389.01	0.01\\
390.01	0.01\\
391.01	0.01\\
392.01	0.01\\
393.01	0.01\\
394.01	0.01\\
395.01	0.01\\
396.01	0.01\\
397.01	0.01\\
398.01	0.01\\
399.01	0.01\\
400.01	0.01\\
401.01	0.01\\
402.01	0.01\\
403.01	0.01\\
404.01	0.01\\
405.01	0.01\\
406.01	0.01\\
407.01	0.01\\
408.01	0.01\\
409.01	0.01\\
410.01	0.01\\
411.01	0.01\\
412.01	0.01\\
413.01	0.01\\
414.01	0.01\\
415.01	0.01\\
416.01	0.01\\
417.01	0.01\\
418.01	0.01\\
419.01	0.01\\
420.01	0.01\\
421.01	0.01\\
422.01	0.01\\
423.01	0.01\\
424.01	0.01\\
425.01	0.01\\
426.01	0.01\\
427.01	0.01\\
428.01	0.01\\
429.01	0.01\\
430.01	0.01\\
431.01	0.01\\
432.01	0.01\\
433.01	0.01\\
434.01	0.01\\
435.01	0.01\\
436.01	0.01\\
437.01	0.01\\
438.01	0.01\\
439.01	0.01\\
440.01	0.01\\
441.01	0.01\\
442.01	0.01\\
443.01	0.01\\
444.01	0.01\\
445.01	0.01\\
446.01	0.01\\
447.01	0.01\\
448.01	0.01\\
449.01	0.01\\
450.01	0.01\\
451.01	0.01\\
452.01	0.01\\
453.01	0.01\\
454.01	0.01\\
455.01	0.01\\
456.01	0.01\\
457.01	0.01\\
458.01	0.01\\
459.01	0.01\\
460.01	0.01\\
461.01	0.01\\
462.01	0.01\\
463.01	0.01\\
464.01	0.01\\
465.01	0.01\\
466.01	0.01\\
467.01	0.01\\
468.01	0.01\\
469.01	0.01\\
470.01	0.01\\
471.01	0.01\\
472.01	0.01\\
473.01	0.01\\
474.01	0.01\\
475.01	0.01\\
476.01	0.01\\
477.01	0.01\\
478.01	0.01\\
479.01	0.01\\
480.01	0.01\\
481.01	0.01\\
482.01	0.01\\
483.01	0.01\\
484.01	0.01\\
485.01	0.01\\
486.01	0.01\\
487.01	0.01\\
488.01	0.01\\
489.01	0.01\\
490.01	0.01\\
491.01	0.01\\
492.01	0.01\\
493.01	0.01\\
494.01	0.01\\
495.01	0.01\\
496.01	0.01\\
497.01	0.01\\
498.01	0.01\\
499.01	0.01\\
500.01	0.01\\
501.01	0.01\\
502.01	0.01\\
503.01	0.01\\
504.01	0.01\\
505.01	0.01\\
506.01	0.01\\
507.01	0.01\\
508.01	0.01\\
509.01	0.01\\
510.01	0.01\\
511.01	0.01\\
512.01	0.01\\
513.01	0.01\\
514.01	0.01\\
515.01	0.01\\
516.01	0.01\\
517.01	0.01\\
518.01	0.01\\
519.01	0.01\\
520.01	0.01\\
521.01	0.01\\
522.01	0.01\\
523.01	0.01\\
524.01	0.01\\
525.01	0.01\\
526.01	0.01\\
527.01	0.01\\
528.01	0.01\\
529.01	0.01\\
530.01	0.01\\
531.01	0.01\\
532.01	0.01\\
533.01	0.01\\
534.01	0.01\\
535.01	0.01\\
536.01	0.01\\
537.01	0.01\\
538.01	0.01\\
539.01	0.01\\
540.01	0.01\\
541.01	0.01\\
542.01	0.01\\
543.01	0.01\\
544.01	0.01\\
545.01	0.01\\
546.01	0.01\\
547.01	0.01\\
548.01	0.01\\
549.01	0.01\\
550.01	0.01\\
551.01	0.01\\
552.01	0.01\\
553.01	0.01\\
554.01	0.01\\
555.01	0.01\\
556.01	0.01\\
557.01	0.01\\
558.01	0.01\\
559.01	0.01\\
560.01	0.01\\
561.01	0.01\\
562.01	0.01\\
563.01	0.01\\
564.01	0.01\\
565.01	0.01\\
566.01	0.01\\
567.01	0.01\\
568.01	0.01\\
569.01	0.01\\
570.01	0.01\\
571.01	0.01\\
572.01	0.01\\
573.01	0.01\\
574.01	0.01\\
575.01	0.01\\
576.01	0.01\\
577.01	0.01\\
578.01	0.01\\
579.01	0.01\\
580.01	0.01\\
581.01	0.01\\
582.01	0.01\\
583.01	0.01\\
584.01	0.01\\
585.01	0.01\\
586.01	0.01\\
587.01	0.01\\
588.01	0.01\\
589.01	0.01\\
590.01	0.01\\
591.01	0.01\\
592.01	0.01\\
593.01	0.01\\
594.01	0.01\\
595.01	0.01\\
596.01	0.01\\
597.01	0.01\\
598.01	0.01\\
599.01	0.01\\
599.02	0.01\\
599.03	0.01\\
599.04	0.01\\
599.05	0.01\\
599.06	0.01\\
599.07	0.01\\
599.08	0.01\\
599.09	0.01\\
599.1	0.01\\
599.11	0.01\\
599.12	0.01\\
599.13	0.01\\
599.14	0.01\\
599.15	0.01\\
599.16	0.01\\
599.17	0.01\\
599.18	0.01\\
599.19	0.01\\
599.2	0.01\\
599.21	0.01\\
599.22	0.01\\
599.23	0.01\\
599.24	0.01\\
599.25	0.01\\
599.26	0.01\\
599.27	0.01\\
599.28	0.01\\
599.29	0.01\\
599.3	0.01\\
599.31	0.01\\
599.32	0.01\\
599.33	0.01\\
599.34	0.01\\
599.35	0.01\\
599.36	0.01\\
599.37	0.01\\
599.38	0.01\\
599.39	0.01\\
599.4	0.01\\
599.41	0.01\\
599.42	0.01\\
599.43	0.01\\
599.44	0.01\\
599.45	0.01\\
599.46	0.01\\
599.47	0.01\\
599.48	0.01\\
599.49	0.01\\
599.5	0.01\\
599.51	0.01\\
599.52	0.01\\
599.53	0.01\\
599.54	0.01\\
599.55	0.01\\
599.56	0.01\\
599.57	0.01\\
599.58	0.01\\
599.59	0.01\\
599.6	0.01\\
599.61	0.01\\
599.62	0.01\\
599.63	0.01\\
599.64	0.01\\
599.65	0.01\\
599.66	0.01\\
599.67	0.01\\
599.68	0.01\\
599.69	0.01\\
599.7	0.01\\
599.71	0.01\\
599.72	0.01\\
599.73	0.01\\
599.74	0.01\\
599.75	0.01\\
599.76	0.01\\
599.77	0.01\\
599.78	0.01\\
599.79	0.01\\
599.8	0.01\\
599.81	0.01\\
599.82	0.01\\
599.83	0.01\\
599.84	0.01\\
599.85	0.01\\
599.86	0.01\\
599.87	0.01\\
599.88	0.01\\
599.89	0.01\\
599.9	0.01\\
599.91	0.01\\
599.92	0.01\\
599.93	0.01\\
599.94	0.01\\
599.95	0.01\\
599.96	0.01\\
599.97	0.01\\
599.98	0.01\\
599.99	0.01\\
600	0.01\\
};
\addplot [color=mycolor4,solid,forget plot]
  table[row sep=crcr]{%
0.01	0.01\\
1.01	0.01\\
2.01	0.01\\
3.01	0.01\\
4.01	0.01\\
5.01	0.01\\
6.01	0.01\\
7.01	0.01\\
8.01	0.01\\
9.01	0.01\\
10.01	0.01\\
11.01	0.01\\
12.01	0.01\\
13.01	0.01\\
14.01	0.01\\
15.01	0.01\\
16.01	0.01\\
17.01	0.01\\
18.01	0.01\\
19.01	0.01\\
20.01	0.01\\
21.01	0.01\\
22.01	0.01\\
23.01	0.01\\
24.01	0.01\\
25.01	0.01\\
26.01	0.01\\
27.01	0.01\\
28.01	0.01\\
29.01	0.01\\
30.01	0.01\\
31.01	0.01\\
32.01	0.01\\
33.01	0.01\\
34.01	0.01\\
35.01	0.01\\
36.01	0.01\\
37.01	0.01\\
38.01	0.01\\
39.01	0.01\\
40.01	0.01\\
41.01	0.01\\
42.01	0.01\\
43.01	0.01\\
44.01	0.01\\
45.01	0.01\\
46.01	0.01\\
47.01	0.01\\
48.01	0.01\\
49.01	0.01\\
50.01	0.01\\
51.01	0.01\\
52.01	0.01\\
53.01	0.01\\
54.01	0.01\\
55.01	0.01\\
56.01	0.01\\
57.01	0.01\\
58.01	0.01\\
59.01	0.01\\
60.01	0.01\\
61.01	0.01\\
62.01	0.01\\
63.01	0.01\\
64.01	0.01\\
65.01	0.01\\
66.01	0.01\\
67.01	0.01\\
68.01	0.01\\
69.01	0.01\\
70.01	0.01\\
71.01	0.01\\
72.01	0.01\\
73.01	0.01\\
74.01	0.01\\
75.01	0.01\\
76.01	0.01\\
77.01	0.01\\
78.01	0.01\\
79.01	0.01\\
80.01	0.01\\
81.01	0.01\\
82.01	0.01\\
83.01	0.01\\
84.01	0.01\\
85.01	0.01\\
86.01	0.01\\
87.01	0.01\\
88.01	0.01\\
89.01	0.01\\
90.01	0.01\\
91.01	0.01\\
92.01	0.01\\
93.01	0.01\\
94.01	0.01\\
95.01	0.01\\
96.01	0.01\\
97.01	0.01\\
98.01	0.01\\
99.01	0.01\\
100.01	0.01\\
101.01	0.01\\
102.01	0.01\\
103.01	0.01\\
104.01	0.01\\
105.01	0.01\\
106.01	0.01\\
107.01	0.01\\
108.01	0.01\\
109.01	0.01\\
110.01	0.01\\
111.01	0.01\\
112.01	0.01\\
113.01	0.01\\
114.01	0.01\\
115.01	0.01\\
116.01	0.01\\
117.01	0.01\\
118.01	0.01\\
119.01	0.01\\
120.01	0.01\\
121.01	0.01\\
122.01	0.01\\
123.01	0.01\\
124.01	0.01\\
125.01	0.01\\
126.01	0.01\\
127.01	0.01\\
128.01	0.01\\
129.01	0.01\\
130.01	0.01\\
131.01	0.01\\
132.01	0.01\\
133.01	0.01\\
134.01	0.01\\
135.01	0.01\\
136.01	0.01\\
137.01	0.01\\
138.01	0.01\\
139.01	0.01\\
140.01	0.01\\
141.01	0.01\\
142.01	0.01\\
143.01	0.01\\
144.01	0.01\\
145.01	0.01\\
146.01	0.01\\
147.01	0.01\\
148.01	0.01\\
149.01	0.01\\
150.01	0.01\\
151.01	0.01\\
152.01	0.01\\
153.01	0.01\\
154.01	0.01\\
155.01	0.01\\
156.01	0.01\\
157.01	0.01\\
158.01	0.01\\
159.01	0.01\\
160.01	0.01\\
161.01	0.01\\
162.01	0.01\\
163.01	0.01\\
164.01	0.01\\
165.01	0.01\\
166.01	0.01\\
167.01	0.01\\
168.01	0.01\\
169.01	0.01\\
170.01	0.01\\
171.01	0.01\\
172.01	0.01\\
173.01	0.01\\
174.01	0.01\\
175.01	0.01\\
176.01	0.01\\
177.01	0.01\\
178.01	0.01\\
179.01	0.01\\
180.01	0.01\\
181.01	0.01\\
182.01	0.01\\
183.01	0.01\\
184.01	0.01\\
185.01	0.01\\
186.01	0.01\\
187.01	0.01\\
188.01	0.01\\
189.01	0.01\\
190.01	0.01\\
191.01	0.01\\
192.01	0.01\\
193.01	0.01\\
194.01	0.01\\
195.01	0.01\\
196.01	0.01\\
197.01	0.01\\
198.01	0.01\\
199.01	0.01\\
200.01	0.01\\
201.01	0.01\\
202.01	0.01\\
203.01	0.01\\
204.01	0.01\\
205.01	0.01\\
206.01	0.01\\
207.01	0.01\\
208.01	0.01\\
209.01	0.01\\
210.01	0.01\\
211.01	0.01\\
212.01	0.01\\
213.01	0.01\\
214.01	0.01\\
215.01	0.01\\
216.01	0.01\\
217.01	0.01\\
218.01	0.01\\
219.01	0.01\\
220.01	0.01\\
221.01	0.01\\
222.01	0.01\\
223.01	0.01\\
224.01	0.01\\
225.01	0.01\\
226.01	0.01\\
227.01	0.01\\
228.01	0.01\\
229.01	0.01\\
230.01	0.01\\
231.01	0.01\\
232.01	0.01\\
233.01	0.01\\
234.01	0.01\\
235.01	0.01\\
236.01	0.01\\
237.01	0.01\\
238.01	0.01\\
239.01	0.01\\
240.01	0.01\\
241.01	0.01\\
242.01	0.01\\
243.01	0.01\\
244.01	0.01\\
245.01	0.01\\
246.01	0.01\\
247.01	0.01\\
248.01	0.01\\
249.01	0.01\\
250.01	0.01\\
251.01	0.01\\
252.01	0.01\\
253.01	0.01\\
254.01	0.01\\
255.01	0.01\\
256.01	0.01\\
257.01	0.01\\
258.01	0.01\\
259.01	0.01\\
260.01	0.01\\
261.01	0.01\\
262.01	0.01\\
263.01	0.01\\
264.01	0.01\\
265.01	0.01\\
266.01	0.01\\
267.01	0.01\\
268.01	0.01\\
269.01	0.01\\
270.01	0.01\\
271.01	0.01\\
272.01	0.01\\
273.01	0.01\\
274.01	0.01\\
275.01	0.01\\
276.01	0.01\\
277.01	0.01\\
278.01	0.01\\
279.01	0.01\\
280.01	0.01\\
281.01	0.01\\
282.01	0.01\\
283.01	0.01\\
284.01	0.01\\
285.01	0.01\\
286.01	0.01\\
287.01	0.01\\
288.01	0.01\\
289.01	0.01\\
290.01	0.01\\
291.01	0.01\\
292.01	0.01\\
293.01	0.01\\
294.01	0.01\\
295.01	0.01\\
296.01	0.01\\
297.01	0.01\\
298.01	0.01\\
299.01	0.01\\
300.01	0.01\\
301.01	0.01\\
302.01	0.01\\
303.01	0.01\\
304.01	0.01\\
305.01	0.01\\
306.01	0.01\\
307.01	0.01\\
308.01	0.01\\
309.01	0.01\\
310.01	0.01\\
311.01	0.01\\
312.01	0.01\\
313.01	0.01\\
314.01	0.01\\
315.01	0.01\\
316.01	0.01\\
317.01	0.01\\
318.01	0.01\\
319.01	0.01\\
320.01	0.01\\
321.01	0.01\\
322.01	0.01\\
323.01	0.01\\
324.01	0.01\\
325.01	0.01\\
326.01	0.01\\
327.01	0.01\\
328.01	0.01\\
329.01	0.01\\
330.01	0.01\\
331.01	0.01\\
332.01	0.01\\
333.01	0.01\\
334.01	0.01\\
335.01	0.01\\
336.01	0.01\\
337.01	0.01\\
338.01	0.01\\
339.01	0.01\\
340.01	0.01\\
341.01	0.01\\
342.01	0.01\\
343.01	0.01\\
344.01	0.01\\
345.01	0.01\\
346.01	0.01\\
347.01	0.01\\
348.01	0.01\\
349.01	0.01\\
350.01	0.01\\
351.01	0.01\\
352.01	0.01\\
353.01	0.01\\
354.01	0.01\\
355.01	0.01\\
356.01	0.01\\
357.01	0.01\\
358.01	0.01\\
359.01	0.01\\
360.01	0.01\\
361.01	0.01\\
362.01	0.01\\
363.01	0.01\\
364.01	0.01\\
365.01	0.01\\
366.01	0.01\\
367.01	0.01\\
368.01	0.01\\
369.01	0.01\\
370.01	0.01\\
371.01	0.01\\
372.01	0.01\\
373.01	0.01\\
374.01	0.01\\
375.01	0.01\\
376.01	0.01\\
377.01	0.01\\
378.01	0.01\\
379.01	0.01\\
380.01	0.01\\
381.01	0.01\\
382.01	0.01\\
383.01	0.01\\
384.01	0.01\\
385.01	0.01\\
386.01	0.01\\
387.01	0.01\\
388.01	0.01\\
389.01	0.01\\
390.01	0.01\\
391.01	0.01\\
392.01	0.01\\
393.01	0.01\\
394.01	0.01\\
395.01	0.01\\
396.01	0.01\\
397.01	0.01\\
398.01	0.01\\
399.01	0.01\\
400.01	0.01\\
401.01	0.01\\
402.01	0.01\\
403.01	0.01\\
404.01	0.01\\
405.01	0.01\\
406.01	0.01\\
407.01	0.01\\
408.01	0.01\\
409.01	0.01\\
410.01	0.01\\
411.01	0.01\\
412.01	0.01\\
413.01	0.01\\
414.01	0.01\\
415.01	0.01\\
416.01	0.01\\
417.01	0.01\\
418.01	0.01\\
419.01	0.01\\
420.01	0.01\\
421.01	0.01\\
422.01	0.01\\
423.01	0.01\\
424.01	0.01\\
425.01	0.01\\
426.01	0.01\\
427.01	0.01\\
428.01	0.01\\
429.01	0.01\\
430.01	0.01\\
431.01	0.01\\
432.01	0.01\\
433.01	0.01\\
434.01	0.01\\
435.01	0.01\\
436.01	0.01\\
437.01	0.01\\
438.01	0.01\\
439.01	0.01\\
440.01	0.01\\
441.01	0.01\\
442.01	0.01\\
443.01	0.01\\
444.01	0.01\\
445.01	0.01\\
446.01	0.01\\
447.01	0.01\\
448.01	0.01\\
449.01	0.01\\
450.01	0.01\\
451.01	0.01\\
452.01	0.01\\
453.01	0.01\\
454.01	0.01\\
455.01	0.01\\
456.01	0.01\\
457.01	0.01\\
458.01	0.01\\
459.01	0.01\\
460.01	0.01\\
461.01	0.01\\
462.01	0.01\\
463.01	0.01\\
464.01	0.01\\
465.01	0.01\\
466.01	0.01\\
467.01	0.01\\
468.01	0.01\\
469.01	0.01\\
470.01	0.01\\
471.01	0.01\\
472.01	0.01\\
473.01	0.01\\
474.01	0.01\\
475.01	0.01\\
476.01	0.01\\
477.01	0.01\\
478.01	0.01\\
479.01	0.01\\
480.01	0.01\\
481.01	0.01\\
482.01	0.01\\
483.01	0.01\\
484.01	0.01\\
485.01	0.01\\
486.01	0.01\\
487.01	0.01\\
488.01	0.01\\
489.01	0.01\\
490.01	0.01\\
491.01	0.01\\
492.01	0.01\\
493.01	0.01\\
494.01	0.01\\
495.01	0.01\\
496.01	0.01\\
497.01	0.01\\
498.01	0.01\\
499.01	0.01\\
500.01	0.01\\
501.01	0.01\\
502.01	0.01\\
503.01	0.01\\
504.01	0.01\\
505.01	0.01\\
506.01	0.01\\
507.01	0.01\\
508.01	0.01\\
509.01	0.01\\
510.01	0.01\\
511.01	0.01\\
512.01	0.01\\
513.01	0.01\\
514.01	0.01\\
515.01	0.01\\
516.01	0.01\\
517.01	0.01\\
518.01	0.01\\
519.01	0.01\\
520.01	0.01\\
521.01	0.01\\
522.01	0.01\\
523.01	0.01\\
524.01	0.01\\
525.01	0.01\\
526.01	0.01\\
527.01	0.01\\
528.01	0.01\\
529.01	0.01\\
530.01	0.01\\
531.01	0.01\\
532.01	0.01\\
533.01	0.01\\
534.01	0.01\\
535.01	0.01\\
536.01	0.01\\
537.01	0.01\\
538.01	0.01\\
539.01	0.01\\
540.01	0.01\\
541.01	0.01\\
542.01	0.01\\
543.01	0.01\\
544.01	0.01\\
545.01	0.01\\
546.01	0.01\\
547.01	0.01\\
548.01	0.01\\
549.01	0.01\\
550.01	0.01\\
551.01	0.01\\
552.01	0.01\\
553.01	0.01\\
554.01	0.01\\
555.01	0.01\\
556.01	0.01\\
557.01	0.01\\
558.01	0.01\\
559.01	0.01\\
560.01	0.01\\
561.01	0.01\\
562.01	0.01\\
563.01	0.01\\
564.01	0.01\\
565.01	0.01\\
566.01	0.01\\
567.01	0.01\\
568.01	0.01\\
569.01	0.01\\
570.01	0.01\\
571.01	0.01\\
572.01	0.01\\
573.01	0.01\\
574.01	0.01\\
575.01	0.01\\
576.01	0.01\\
577.01	0.01\\
578.01	0.01\\
579.01	0.01\\
580.01	0.01\\
581.01	0.01\\
582.01	0.01\\
583.01	0.01\\
584.01	0.01\\
585.01	0.01\\
586.01	0.01\\
587.01	0.01\\
588.01	0.01\\
589.01	0.01\\
590.01	0.01\\
591.01	0.01\\
592.01	0.01\\
593.01	0.01\\
594.01	0.01\\
595.01	0.01\\
596.01	0.01\\
597.01	0.01\\
598.01	0.01\\
599.01	0.01\\
599.02	0.01\\
599.03	0.01\\
599.04	0.01\\
599.05	0.01\\
599.06	0.01\\
599.07	0.01\\
599.08	0.01\\
599.09	0.01\\
599.1	0.01\\
599.11	0.01\\
599.12	0.01\\
599.13	0.01\\
599.14	0.01\\
599.15	0.01\\
599.16	0.01\\
599.17	0.01\\
599.18	0.01\\
599.19	0.01\\
599.2	0.01\\
599.21	0.01\\
599.22	0.01\\
599.23	0.01\\
599.24	0.01\\
599.25	0.01\\
599.26	0.01\\
599.27	0.01\\
599.28	0.01\\
599.29	0.01\\
599.3	0.01\\
599.31	0.01\\
599.32	0.01\\
599.33	0.01\\
599.34	0.01\\
599.35	0.01\\
599.36	0.01\\
599.37	0.01\\
599.38	0.01\\
599.39	0.01\\
599.4	0.01\\
599.41	0.01\\
599.42	0.01\\
599.43	0.01\\
599.44	0.01\\
599.45	0.01\\
599.46	0.01\\
599.47	0.01\\
599.48	0.01\\
599.49	0.01\\
599.5	0.01\\
599.51	0.01\\
599.52	0.01\\
599.53	0.01\\
599.54	0.01\\
599.55	0.01\\
599.56	0.01\\
599.57	0.01\\
599.58	0.01\\
599.59	0.01\\
599.6	0.01\\
599.61	0.01\\
599.62	0.01\\
599.63	0.01\\
599.64	0.01\\
599.65	0.01\\
599.66	0.01\\
599.67	0.01\\
599.68	0.01\\
599.69	0.01\\
599.7	0.01\\
599.71	0.01\\
599.72	0.01\\
599.73	0.01\\
599.74	0.01\\
599.75	0.01\\
599.76	0.01\\
599.77	0.01\\
599.78	0.01\\
599.79	0.01\\
599.8	0.01\\
599.81	0.01\\
599.82	0.01\\
599.83	0.01\\
599.84	0.01\\
599.85	0.01\\
599.86	0.01\\
599.87	0.01\\
599.88	0.01\\
599.89	0.01\\
599.9	0.01\\
599.91	0.01\\
599.92	0.01\\
599.93	0.01\\
599.94	0.01\\
599.95	0.01\\
599.96	0.01\\
599.97	0.01\\
599.98	0.01\\
599.99	0.01\\
600	0.01\\
};
\addplot [color=mycolor5,solid,forget plot]
  table[row sep=crcr]{%
0.01	0.01\\
1.01	0.01\\
2.01	0.01\\
3.01	0.01\\
4.01	0.01\\
5.01	0.01\\
6.01	0.01\\
7.01	0.01\\
8.01	0.01\\
9.01	0.01\\
10.01	0.01\\
11.01	0.01\\
12.01	0.01\\
13.01	0.01\\
14.01	0.01\\
15.01	0.01\\
16.01	0.01\\
17.01	0.01\\
18.01	0.01\\
19.01	0.01\\
20.01	0.01\\
21.01	0.01\\
22.01	0.01\\
23.01	0.01\\
24.01	0.01\\
25.01	0.01\\
26.01	0.01\\
27.01	0.01\\
28.01	0.01\\
29.01	0.01\\
30.01	0.01\\
31.01	0.01\\
32.01	0.01\\
33.01	0.01\\
34.01	0.01\\
35.01	0.01\\
36.01	0.01\\
37.01	0.01\\
38.01	0.01\\
39.01	0.01\\
40.01	0.01\\
41.01	0.01\\
42.01	0.01\\
43.01	0.01\\
44.01	0.01\\
45.01	0.01\\
46.01	0.01\\
47.01	0.01\\
48.01	0.01\\
49.01	0.01\\
50.01	0.01\\
51.01	0.01\\
52.01	0.01\\
53.01	0.01\\
54.01	0.01\\
55.01	0.01\\
56.01	0.01\\
57.01	0.01\\
58.01	0.01\\
59.01	0.01\\
60.01	0.01\\
61.01	0.01\\
62.01	0.01\\
63.01	0.01\\
64.01	0.01\\
65.01	0.01\\
66.01	0.01\\
67.01	0.01\\
68.01	0.01\\
69.01	0.01\\
70.01	0.01\\
71.01	0.01\\
72.01	0.01\\
73.01	0.01\\
74.01	0.01\\
75.01	0.01\\
76.01	0.01\\
77.01	0.01\\
78.01	0.01\\
79.01	0.01\\
80.01	0.01\\
81.01	0.01\\
82.01	0.01\\
83.01	0.01\\
84.01	0.01\\
85.01	0.01\\
86.01	0.01\\
87.01	0.01\\
88.01	0.01\\
89.01	0.01\\
90.01	0.01\\
91.01	0.01\\
92.01	0.01\\
93.01	0.01\\
94.01	0.01\\
95.01	0.01\\
96.01	0.01\\
97.01	0.01\\
98.01	0.01\\
99.01	0.01\\
100.01	0.01\\
101.01	0.01\\
102.01	0.01\\
103.01	0.01\\
104.01	0.01\\
105.01	0.01\\
106.01	0.01\\
107.01	0.01\\
108.01	0.01\\
109.01	0.01\\
110.01	0.01\\
111.01	0.01\\
112.01	0.01\\
113.01	0.01\\
114.01	0.01\\
115.01	0.01\\
116.01	0.01\\
117.01	0.01\\
118.01	0.01\\
119.01	0.01\\
120.01	0.01\\
121.01	0.01\\
122.01	0.01\\
123.01	0.01\\
124.01	0.01\\
125.01	0.01\\
126.01	0.01\\
127.01	0.01\\
128.01	0.01\\
129.01	0.01\\
130.01	0.01\\
131.01	0.01\\
132.01	0.01\\
133.01	0.01\\
134.01	0.01\\
135.01	0.01\\
136.01	0.01\\
137.01	0.01\\
138.01	0.01\\
139.01	0.01\\
140.01	0.01\\
141.01	0.01\\
142.01	0.01\\
143.01	0.01\\
144.01	0.01\\
145.01	0.01\\
146.01	0.01\\
147.01	0.01\\
148.01	0.01\\
149.01	0.01\\
150.01	0.01\\
151.01	0.01\\
152.01	0.01\\
153.01	0.01\\
154.01	0.01\\
155.01	0.01\\
156.01	0.01\\
157.01	0.01\\
158.01	0.01\\
159.01	0.01\\
160.01	0.01\\
161.01	0.01\\
162.01	0.01\\
163.01	0.01\\
164.01	0.01\\
165.01	0.01\\
166.01	0.01\\
167.01	0.01\\
168.01	0.01\\
169.01	0.01\\
170.01	0.01\\
171.01	0.01\\
172.01	0.01\\
173.01	0.01\\
174.01	0.01\\
175.01	0.01\\
176.01	0.01\\
177.01	0.01\\
178.01	0.01\\
179.01	0.01\\
180.01	0.01\\
181.01	0.01\\
182.01	0.01\\
183.01	0.01\\
184.01	0.01\\
185.01	0.01\\
186.01	0.01\\
187.01	0.01\\
188.01	0.01\\
189.01	0.01\\
190.01	0.01\\
191.01	0.01\\
192.01	0.01\\
193.01	0.01\\
194.01	0.01\\
195.01	0.01\\
196.01	0.01\\
197.01	0.01\\
198.01	0.01\\
199.01	0.01\\
200.01	0.01\\
201.01	0.01\\
202.01	0.01\\
203.01	0.01\\
204.01	0.01\\
205.01	0.01\\
206.01	0.01\\
207.01	0.01\\
208.01	0.01\\
209.01	0.01\\
210.01	0.01\\
211.01	0.01\\
212.01	0.01\\
213.01	0.01\\
214.01	0.01\\
215.01	0.01\\
216.01	0.01\\
217.01	0.01\\
218.01	0.01\\
219.01	0.01\\
220.01	0.01\\
221.01	0.01\\
222.01	0.01\\
223.01	0.01\\
224.01	0.01\\
225.01	0.01\\
226.01	0.01\\
227.01	0.01\\
228.01	0.01\\
229.01	0.01\\
230.01	0.01\\
231.01	0.01\\
232.01	0.01\\
233.01	0.01\\
234.01	0.01\\
235.01	0.01\\
236.01	0.01\\
237.01	0.01\\
238.01	0.01\\
239.01	0.01\\
240.01	0.01\\
241.01	0.01\\
242.01	0.01\\
243.01	0.01\\
244.01	0.01\\
245.01	0.01\\
246.01	0.01\\
247.01	0.01\\
248.01	0.01\\
249.01	0.01\\
250.01	0.01\\
251.01	0.01\\
252.01	0.01\\
253.01	0.01\\
254.01	0.01\\
255.01	0.01\\
256.01	0.01\\
257.01	0.01\\
258.01	0.01\\
259.01	0.01\\
260.01	0.01\\
261.01	0.01\\
262.01	0.01\\
263.01	0.01\\
264.01	0.01\\
265.01	0.01\\
266.01	0.01\\
267.01	0.01\\
268.01	0.01\\
269.01	0.01\\
270.01	0.01\\
271.01	0.01\\
272.01	0.01\\
273.01	0.01\\
274.01	0.01\\
275.01	0.01\\
276.01	0.01\\
277.01	0.01\\
278.01	0.01\\
279.01	0.01\\
280.01	0.01\\
281.01	0.01\\
282.01	0.01\\
283.01	0.01\\
284.01	0.01\\
285.01	0.01\\
286.01	0.01\\
287.01	0.01\\
288.01	0.01\\
289.01	0.01\\
290.01	0.01\\
291.01	0.01\\
292.01	0.01\\
293.01	0.01\\
294.01	0.01\\
295.01	0.01\\
296.01	0.01\\
297.01	0.01\\
298.01	0.01\\
299.01	0.01\\
300.01	0.01\\
301.01	0.01\\
302.01	0.01\\
303.01	0.01\\
304.01	0.01\\
305.01	0.01\\
306.01	0.01\\
307.01	0.01\\
308.01	0.01\\
309.01	0.01\\
310.01	0.01\\
311.01	0.01\\
312.01	0.01\\
313.01	0.01\\
314.01	0.01\\
315.01	0.01\\
316.01	0.01\\
317.01	0.01\\
318.01	0.01\\
319.01	0.01\\
320.01	0.01\\
321.01	0.01\\
322.01	0.01\\
323.01	0.01\\
324.01	0.01\\
325.01	0.01\\
326.01	0.01\\
327.01	0.01\\
328.01	0.01\\
329.01	0.01\\
330.01	0.01\\
331.01	0.01\\
332.01	0.01\\
333.01	0.01\\
334.01	0.01\\
335.01	0.01\\
336.01	0.01\\
337.01	0.01\\
338.01	0.01\\
339.01	0.01\\
340.01	0.01\\
341.01	0.01\\
342.01	0.01\\
343.01	0.01\\
344.01	0.01\\
345.01	0.01\\
346.01	0.01\\
347.01	0.01\\
348.01	0.01\\
349.01	0.01\\
350.01	0.01\\
351.01	0.01\\
352.01	0.01\\
353.01	0.01\\
354.01	0.01\\
355.01	0.01\\
356.01	0.01\\
357.01	0.01\\
358.01	0.01\\
359.01	0.01\\
360.01	0.01\\
361.01	0.01\\
362.01	0.01\\
363.01	0.01\\
364.01	0.01\\
365.01	0.01\\
366.01	0.01\\
367.01	0.01\\
368.01	0.01\\
369.01	0.01\\
370.01	0.01\\
371.01	0.01\\
372.01	0.01\\
373.01	0.01\\
374.01	0.01\\
375.01	0.01\\
376.01	0.01\\
377.01	0.01\\
378.01	0.01\\
379.01	0.01\\
380.01	0.01\\
381.01	0.01\\
382.01	0.01\\
383.01	0.01\\
384.01	0.01\\
385.01	0.01\\
386.01	0.01\\
387.01	0.01\\
388.01	0.01\\
389.01	0.01\\
390.01	0.01\\
391.01	0.01\\
392.01	0.01\\
393.01	0.01\\
394.01	0.01\\
395.01	0.01\\
396.01	0.01\\
397.01	0.01\\
398.01	0.01\\
399.01	0.01\\
400.01	0.01\\
401.01	0.01\\
402.01	0.01\\
403.01	0.01\\
404.01	0.01\\
405.01	0.01\\
406.01	0.01\\
407.01	0.01\\
408.01	0.01\\
409.01	0.01\\
410.01	0.01\\
411.01	0.01\\
412.01	0.01\\
413.01	0.01\\
414.01	0.01\\
415.01	0.01\\
416.01	0.01\\
417.01	0.01\\
418.01	0.01\\
419.01	0.01\\
420.01	0.01\\
421.01	0.01\\
422.01	0.01\\
423.01	0.01\\
424.01	0.01\\
425.01	0.01\\
426.01	0.01\\
427.01	0.01\\
428.01	0.01\\
429.01	0.01\\
430.01	0.01\\
431.01	0.01\\
432.01	0.01\\
433.01	0.01\\
434.01	0.01\\
435.01	0.01\\
436.01	0.01\\
437.01	0.01\\
438.01	0.01\\
439.01	0.01\\
440.01	0.01\\
441.01	0.01\\
442.01	0.01\\
443.01	0.01\\
444.01	0.01\\
445.01	0.01\\
446.01	0.01\\
447.01	0.01\\
448.01	0.01\\
449.01	0.01\\
450.01	0.01\\
451.01	0.01\\
452.01	0.01\\
453.01	0.01\\
454.01	0.01\\
455.01	0.01\\
456.01	0.01\\
457.01	0.01\\
458.01	0.01\\
459.01	0.01\\
460.01	0.01\\
461.01	0.01\\
462.01	0.01\\
463.01	0.01\\
464.01	0.01\\
465.01	0.01\\
466.01	0.01\\
467.01	0.01\\
468.01	0.01\\
469.01	0.01\\
470.01	0.01\\
471.01	0.01\\
472.01	0.01\\
473.01	0.01\\
474.01	0.01\\
475.01	0.01\\
476.01	0.01\\
477.01	0.01\\
478.01	0.01\\
479.01	0.01\\
480.01	0.01\\
481.01	0.01\\
482.01	0.01\\
483.01	0.01\\
484.01	0.01\\
485.01	0.01\\
486.01	0.01\\
487.01	0.01\\
488.01	0.01\\
489.01	0.01\\
490.01	0.01\\
491.01	0.01\\
492.01	0.01\\
493.01	0.01\\
494.01	0.01\\
495.01	0.01\\
496.01	0.01\\
497.01	0.01\\
498.01	0.01\\
499.01	0.01\\
500.01	0.01\\
501.01	0.01\\
502.01	0.01\\
503.01	0.01\\
504.01	0.01\\
505.01	0.01\\
506.01	0.01\\
507.01	0.01\\
508.01	0.01\\
509.01	0.01\\
510.01	0.01\\
511.01	0.01\\
512.01	0.01\\
513.01	0.01\\
514.01	0.01\\
515.01	0.01\\
516.01	0.01\\
517.01	0.01\\
518.01	0.01\\
519.01	0.01\\
520.01	0.01\\
521.01	0.01\\
522.01	0.01\\
523.01	0.01\\
524.01	0.01\\
525.01	0.01\\
526.01	0.01\\
527.01	0.01\\
528.01	0.01\\
529.01	0.01\\
530.01	0.01\\
531.01	0.01\\
532.01	0.01\\
533.01	0.01\\
534.01	0.01\\
535.01	0.01\\
536.01	0.01\\
537.01	0.01\\
538.01	0.01\\
539.01	0.01\\
540.01	0.01\\
541.01	0.01\\
542.01	0.01\\
543.01	0.01\\
544.01	0.01\\
545.01	0.01\\
546.01	0.01\\
547.01	0.01\\
548.01	0.01\\
549.01	0.01\\
550.01	0.01\\
551.01	0.01\\
552.01	0.01\\
553.01	0.01\\
554.01	0.01\\
555.01	0.01\\
556.01	0.01\\
557.01	0.01\\
558.01	0.01\\
559.01	0.01\\
560.01	0.01\\
561.01	0.01\\
562.01	0.01\\
563.01	0.01\\
564.01	0.01\\
565.01	0.01\\
566.01	0.01\\
567.01	0.01\\
568.01	0.01\\
569.01	0.01\\
570.01	0.01\\
571.01	0.01\\
572.01	0.01\\
573.01	0.01\\
574.01	0.01\\
575.01	0.01\\
576.01	0.01\\
577.01	0.01\\
578.01	0.01\\
579.01	0.01\\
580.01	0.01\\
581.01	0.01\\
582.01	0.01\\
583.01	0.01\\
584.01	0.01\\
585.01	0.01\\
586.01	0.01\\
587.01	0.01\\
588.01	0.01\\
589.01	0.01\\
590.01	0.01\\
591.01	0.01\\
592.01	0.01\\
593.01	0.01\\
594.01	0.01\\
595.01	0.01\\
596.01	0.01\\
597.01	0.01\\
598.01	0.01\\
599.01	0.01\\
599.02	0.01\\
599.03	0.01\\
599.04	0.01\\
599.05	0.01\\
599.06	0.01\\
599.07	0.01\\
599.08	0.01\\
599.09	0.01\\
599.1	0.01\\
599.11	0.01\\
599.12	0.01\\
599.13	0.01\\
599.14	0.01\\
599.15	0.01\\
599.16	0.01\\
599.17	0.01\\
599.18	0.01\\
599.19	0.01\\
599.2	0.01\\
599.21	0.01\\
599.22	0.01\\
599.23	0.01\\
599.24	0.01\\
599.25	0.01\\
599.26	0.01\\
599.27	0.01\\
599.28	0.01\\
599.29	0.01\\
599.3	0.01\\
599.31	0.01\\
599.32	0.01\\
599.33	0.01\\
599.34	0.01\\
599.35	0.01\\
599.36	0.01\\
599.37	0.01\\
599.38	0.01\\
599.39	0.01\\
599.4	0.01\\
599.41	0.01\\
599.42	0.01\\
599.43	0.01\\
599.44	0.01\\
599.45	0.01\\
599.46	0.01\\
599.47	0.01\\
599.48	0.01\\
599.49	0.01\\
599.5	0.01\\
599.51	0.01\\
599.52	0.01\\
599.53	0.01\\
599.54	0.01\\
599.55	0.01\\
599.56	0.01\\
599.57	0.01\\
599.58	0.01\\
599.59	0.01\\
599.6	0.01\\
599.61	0.01\\
599.62	0.01\\
599.63	0.01\\
599.64	0.01\\
599.65	0.01\\
599.66	0.01\\
599.67	0.01\\
599.68	0.01\\
599.69	0.01\\
599.7	0.01\\
599.71	0.01\\
599.72	0.01\\
599.73	0.01\\
599.74	0.01\\
599.75	0.01\\
599.76	0.01\\
599.77	0.01\\
599.78	0.01\\
599.79	0.01\\
599.8	0.01\\
599.81	0.01\\
599.82	0.01\\
599.83	0.01\\
599.84	0.01\\
599.85	0.01\\
599.86	0.01\\
599.87	0.01\\
599.88	0.01\\
599.89	0.01\\
599.9	0.01\\
599.91	0.01\\
599.92	0.01\\
599.93	0.01\\
599.94	0.01\\
599.95	0.01\\
599.96	0.01\\
599.97	0.01\\
599.98	0.01\\
599.99	0.01\\
600	0.01\\
};
\addplot [color=mycolor6,solid,forget plot]
  table[row sep=crcr]{%
0.01	0.01\\
1.01	0.01\\
2.01	0.01\\
3.01	0.01\\
4.01	0.01\\
5.01	0.01\\
6.01	0.01\\
7.01	0.01\\
8.01	0.01\\
9.01	0.01\\
10.01	0.01\\
11.01	0.01\\
12.01	0.01\\
13.01	0.01\\
14.01	0.01\\
15.01	0.01\\
16.01	0.01\\
17.01	0.01\\
18.01	0.01\\
19.01	0.01\\
20.01	0.01\\
21.01	0.01\\
22.01	0.01\\
23.01	0.01\\
24.01	0.01\\
25.01	0.01\\
26.01	0.01\\
27.01	0.01\\
28.01	0.01\\
29.01	0.01\\
30.01	0.01\\
31.01	0.01\\
32.01	0.01\\
33.01	0.01\\
34.01	0.01\\
35.01	0.01\\
36.01	0.01\\
37.01	0.01\\
38.01	0.01\\
39.01	0.01\\
40.01	0.01\\
41.01	0.01\\
42.01	0.01\\
43.01	0.01\\
44.01	0.01\\
45.01	0.01\\
46.01	0.01\\
47.01	0.01\\
48.01	0.01\\
49.01	0.01\\
50.01	0.01\\
51.01	0.01\\
52.01	0.01\\
53.01	0.01\\
54.01	0.01\\
55.01	0.01\\
56.01	0.01\\
57.01	0.01\\
58.01	0.01\\
59.01	0.01\\
60.01	0.01\\
61.01	0.01\\
62.01	0.01\\
63.01	0.01\\
64.01	0.01\\
65.01	0.01\\
66.01	0.01\\
67.01	0.01\\
68.01	0.01\\
69.01	0.01\\
70.01	0.01\\
71.01	0.01\\
72.01	0.01\\
73.01	0.01\\
74.01	0.01\\
75.01	0.01\\
76.01	0.01\\
77.01	0.01\\
78.01	0.01\\
79.01	0.01\\
80.01	0.01\\
81.01	0.01\\
82.01	0.01\\
83.01	0.01\\
84.01	0.01\\
85.01	0.01\\
86.01	0.01\\
87.01	0.01\\
88.01	0.01\\
89.01	0.01\\
90.01	0.01\\
91.01	0.01\\
92.01	0.01\\
93.01	0.01\\
94.01	0.01\\
95.01	0.01\\
96.01	0.01\\
97.01	0.01\\
98.01	0.01\\
99.01	0.01\\
100.01	0.01\\
101.01	0.01\\
102.01	0.01\\
103.01	0.01\\
104.01	0.01\\
105.01	0.01\\
106.01	0.01\\
107.01	0.01\\
108.01	0.01\\
109.01	0.01\\
110.01	0.01\\
111.01	0.01\\
112.01	0.01\\
113.01	0.01\\
114.01	0.01\\
115.01	0.01\\
116.01	0.01\\
117.01	0.01\\
118.01	0.01\\
119.01	0.01\\
120.01	0.01\\
121.01	0.01\\
122.01	0.01\\
123.01	0.01\\
124.01	0.01\\
125.01	0.01\\
126.01	0.01\\
127.01	0.01\\
128.01	0.01\\
129.01	0.01\\
130.01	0.01\\
131.01	0.01\\
132.01	0.01\\
133.01	0.01\\
134.01	0.01\\
135.01	0.01\\
136.01	0.01\\
137.01	0.01\\
138.01	0.01\\
139.01	0.01\\
140.01	0.01\\
141.01	0.01\\
142.01	0.01\\
143.01	0.01\\
144.01	0.01\\
145.01	0.01\\
146.01	0.01\\
147.01	0.01\\
148.01	0.01\\
149.01	0.01\\
150.01	0.01\\
151.01	0.01\\
152.01	0.01\\
153.01	0.01\\
154.01	0.01\\
155.01	0.01\\
156.01	0.01\\
157.01	0.01\\
158.01	0.01\\
159.01	0.01\\
160.01	0.01\\
161.01	0.01\\
162.01	0.01\\
163.01	0.01\\
164.01	0.01\\
165.01	0.01\\
166.01	0.01\\
167.01	0.01\\
168.01	0.01\\
169.01	0.01\\
170.01	0.01\\
171.01	0.01\\
172.01	0.01\\
173.01	0.01\\
174.01	0.01\\
175.01	0.01\\
176.01	0.01\\
177.01	0.01\\
178.01	0.01\\
179.01	0.01\\
180.01	0.01\\
181.01	0.01\\
182.01	0.01\\
183.01	0.01\\
184.01	0.01\\
185.01	0.01\\
186.01	0.01\\
187.01	0.01\\
188.01	0.01\\
189.01	0.01\\
190.01	0.01\\
191.01	0.01\\
192.01	0.01\\
193.01	0.01\\
194.01	0.01\\
195.01	0.01\\
196.01	0.01\\
197.01	0.01\\
198.01	0.01\\
199.01	0.01\\
200.01	0.01\\
201.01	0.01\\
202.01	0.01\\
203.01	0.01\\
204.01	0.01\\
205.01	0.01\\
206.01	0.01\\
207.01	0.01\\
208.01	0.01\\
209.01	0.01\\
210.01	0.01\\
211.01	0.01\\
212.01	0.01\\
213.01	0.01\\
214.01	0.01\\
215.01	0.01\\
216.01	0.01\\
217.01	0.01\\
218.01	0.01\\
219.01	0.01\\
220.01	0.01\\
221.01	0.01\\
222.01	0.01\\
223.01	0.01\\
224.01	0.01\\
225.01	0.01\\
226.01	0.01\\
227.01	0.01\\
228.01	0.01\\
229.01	0.01\\
230.01	0.01\\
231.01	0.01\\
232.01	0.01\\
233.01	0.01\\
234.01	0.01\\
235.01	0.01\\
236.01	0.01\\
237.01	0.01\\
238.01	0.01\\
239.01	0.01\\
240.01	0.01\\
241.01	0.01\\
242.01	0.01\\
243.01	0.01\\
244.01	0.01\\
245.01	0.01\\
246.01	0.01\\
247.01	0.01\\
248.01	0.01\\
249.01	0.01\\
250.01	0.01\\
251.01	0.01\\
252.01	0.01\\
253.01	0.01\\
254.01	0.01\\
255.01	0.01\\
256.01	0.01\\
257.01	0.01\\
258.01	0.01\\
259.01	0.01\\
260.01	0.01\\
261.01	0.01\\
262.01	0.01\\
263.01	0.01\\
264.01	0.01\\
265.01	0.01\\
266.01	0.01\\
267.01	0.01\\
268.01	0.01\\
269.01	0.01\\
270.01	0.01\\
271.01	0.01\\
272.01	0.01\\
273.01	0.01\\
274.01	0.01\\
275.01	0.01\\
276.01	0.01\\
277.01	0.01\\
278.01	0.01\\
279.01	0.01\\
280.01	0.01\\
281.01	0.01\\
282.01	0.01\\
283.01	0.01\\
284.01	0.01\\
285.01	0.01\\
286.01	0.01\\
287.01	0.01\\
288.01	0.01\\
289.01	0.01\\
290.01	0.01\\
291.01	0.01\\
292.01	0.01\\
293.01	0.01\\
294.01	0.01\\
295.01	0.01\\
296.01	0.01\\
297.01	0.01\\
298.01	0.01\\
299.01	0.01\\
300.01	0.01\\
301.01	0.01\\
302.01	0.01\\
303.01	0.01\\
304.01	0.01\\
305.01	0.01\\
306.01	0.01\\
307.01	0.01\\
308.01	0.01\\
309.01	0.01\\
310.01	0.01\\
311.01	0.01\\
312.01	0.01\\
313.01	0.01\\
314.01	0.01\\
315.01	0.01\\
316.01	0.01\\
317.01	0.01\\
318.01	0.01\\
319.01	0.01\\
320.01	0.01\\
321.01	0.01\\
322.01	0.01\\
323.01	0.01\\
324.01	0.01\\
325.01	0.01\\
326.01	0.01\\
327.01	0.01\\
328.01	0.01\\
329.01	0.01\\
330.01	0.01\\
331.01	0.01\\
332.01	0.01\\
333.01	0.01\\
334.01	0.01\\
335.01	0.01\\
336.01	0.01\\
337.01	0.01\\
338.01	0.01\\
339.01	0.01\\
340.01	0.01\\
341.01	0.01\\
342.01	0.01\\
343.01	0.01\\
344.01	0.01\\
345.01	0.01\\
346.01	0.01\\
347.01	0.01\\
348.01	0.01\\
349.01	0.01\\
350.01	0.01\\
351.01	0.01\\
352.01	0.01\\
353.01	0.01\\
354.01	0.01\\
355.01	0.01\\
356.01	0.01\\
357.01	0.01\\
358.01	0.01\\
359.01	0.01\\
360.01	0.01\\
361.01	0.01\\
362.01	0.01\\
363.01	0.01\\
364.01	0.01\\
365.01	0.01\\
366.01	0.01\\
367.01	0.01\\
368.01	0.01\\
369.01	0.01\\
370.01	0.01\\
371.01	0.01\\
372.01	0.01\\
373.01	0.01\\
374.01	0.01\\
375.01	0.01\\
376.01	0.01\\
377.01	0.01\\
378.01	0.01\\
379.01	0.01\\
380.01	0.01\\
381.01	0.01\\
382.01	0.01\\
383.01	0.01\\
384.01	0.01\\
385.01	0.01\\
386.01	0.01\\
387.01	0.01\\
388.01	0.01\\
389.01	0.01\\
390.01	0.01\\
391.01	0.01\\
392.01	0.01\\
393.01	0.01\\
394.01	0.01\\
395.01	0.01\\
396.01	0.01\\
397.01	0.01\\
398.01	0.01\\
399.01	0.01\\
400.01	0.01\\
401.01	0.01\\
402.01	0.01\\
403.01	0.01\\
404.01	0.01\\
405.01	0.01\\
406.01	0.01\\
407.01	0.01\\
408.01	0.01\\
409.01	0.01\\
410.01	0.01\\
411.01	0.01\\
412.01	0.01\\
413.01	0.01\\
414.01	0.01\\
415.01	0.01\\
416.01	0.01\\
417.01	0.01\\
418.01	0.01\\
419.01	0.01\\
420.01	0.01\\
421.01	0.01\\
422.01	0.01\\
423.01	0.01\\
424.01	0.01\\
425.01	0.01\\
426.01	0.01\\
427.01	0.01\\
428.01	0.01\\
429.01	0.01\\
430.01	0.01\\
431.01	0.01\\
432.01	0.01\\
433.01	0.01\\
434.01	0.01\\
435.01	0.01\\
436.01	0.01\\
437.01	0.01\\
438.01	0.01\\
439.01	0.01\\
440.01	0.01\\
441.01	0.01\\
442.01	0.01\\
443.01	0.01\\
444.01	0.01\\
445.01	0.01\\
446.01	0.01\\
447.01	0.01\\
448.01	0.01\\
449.01	0.01\\
450.01	0.01\\
451.01	0.01\\
452.01	0.01\\
453.01	0.01\\
454.01	0.01\\
455.01	0.01\\
456.01	0.01\\
457.01	0.01\\
458.01	0.01\\
459.01	0.01\\
460.01	0.01\\
461.01	0.01\\
462.01	0.01\\
463.01	0.01\\
464.01	0.01\\
465.01	0.01\\
466.01	0.01\\
467.01	0.01\\
468.01	0.01\\
469.01	0.01\\
470.01	0.01\\
471.01	0.01\\
472.01	0.01\\
473.01	0.01\\
474.01	0.01\\
475.01	0.01\\
476.01	0.01\\
477.01	0.01\\
478.01	0.01\\
479.01	0.01\\
480.01	0.01\\
481.01	0.01\\
482.01	0.01\\
483.01	0.01\\
484.01	0.01\\
485.01	0.01\\
486.01	0.01\\
487.01	0.01\\
488.01	0.01\\
489.01	0.01\\
490.01	0.01\\
491.01	0.01\\
492.01	0.01\\
493.01	0.01\\
494.01	0.01\\
495.01	0.01\\
496.01	0.01\\
497.01	0.01\\
498.01	0.01\\
499.01	0.01\\
500.01	0.01\\
501.01	0.01\\
502.01	0.01\\
503.01	0.01\\
504.01	0.01\\
505.01	0.01\\
506.01	0.01\\
507.01	0.01\\
508.01	0.01\\
509.01	0.01\\
510.01	0.01\\
511.01	0.01\\
512.01	0.01\\
513.01	0.01\\
514.01	0.01\\
515.01	0.01\\
516.01	0.01\\
517.01	0.01\\
518.01	0.01\\
519.01	0.01\\
520.01	0.01\\
521.01	0.01\\
522.01	0.01\\
523.01	0.01\\
524.01	0.01\\
525.01	0.01\\
526.01	0.01\\
527.01	0.01\\
528.01	0.01\\
529.01	0.01\\
530.01	0.01\\
531.01	0.01\\
532.01	0.01\\
533.01	0.01\\
534.01	0.01\\
535.01	0.01\\
536.01	0.01\\
537.01	0.01\\
538.01	0.01\\
539.01	0.01\\
540.01	0.01\\
541.01	0.01\\
542.01	0.01\\
543.01	0.01\\
544.01	0.01\\
545.01	0.01\\
546.01	0.01\\
547.01	0.01\\
548.01	0.01\\
549.01	0.01\\
550.01	0.01\\
551.01	0.01\\
552.01	0.01\\
553.01	0.01\\
554.01	0.01\\
555.01	0.01\\
556.01	0.01\\
557.01	0.01\\
558.01	0.01\\
559.01	0.01\\
560.01	0.01\\
561.01	0.01\\
562.01	0.01\\
563.01	0.01\\
564.01	0.01\\
565.01	0.01\\
566.01	0.01\\
567.01	0.01\\
568.01	0.01\\
569.01	0.01\\
570.01	0.01\\
571.01	0.01\\
572.01	0.01\\
573.01	0.01\\
574.01	0.01\\
575.01	0.01\\
576.01	0.01\\
577.01	0.01\\
578.01	0.01\\
579.01	0.01\\
580.01	0.01\\
581.01	0.01\\
582.01	0.01\\
583.01	0.01\\
584.01	0.01\\
585.01	0.01\\
586.01	0.01\\
587.01	0.01\\
588.01	0.01\\
589.01	0.01\\
590.01	0.01\\
591.01	0.01\\
592.01	0.01\\
593.01	0.01\\
594.01	0.01\\
595.01	0.01\\
596.01	0.01\\
597.01	0.01\\
598.01	0.01\\
599.01	0.01\\
599.02	0.01\\
599.03	0.01\\
599.04	0.01\\
599.05	0.01\\
599.06	0.01\\
599.07	0.01\\
599.08	0.01\\
599.09	0.01\\
599.1	0.01\\
599.11	0.01\\
599.12	0.01\\
599.13	0.01\\
599.14	0.01\\
599.15	0.01\\
599.16	0.01\\
599.17	0.01\\
599.18	0.01\\
599.19	0.01\\
599.2	0.01\\
599.21	0.01\\
599.22	0.01\\
599.23	0.01\\
599.24	0.01\\
599.25	0.01\\
599.26	0.01\\
599.27	0.01\\
599.28	0.01\\
599.29	0.01\\
599.3	0.01\\
599.31	0.01\\
599.32	0.01\\
599.33	0.01\\
599.34	0.01\\
599.35	0.01\\
599.36	0.01\\
599.37	0.01\\
599.38	0.01\\
599.39	0.01\\
599.4	0.01\\
599.41	0.01\\
599.42	0.01\\
599.43	0.01\\
599.44	0.01\\
599.45	0.01\\
599.46	0.01\\
599.47	0.01\\
599.48	0.01\\
599.49	0.01\\
599.5	0.01\\
599.51	0.01\\
599.52	0.01\\
599.53	0.01\\
599.54	0.01\\
599.55	0.01\\
599.56	0.01\\
599.57	0.01\\
599.58	0.01\\
599.59	0.01\\
599.6	0.01\\
599.61	0.01\\
599.62	0.01\\
599.63	0.01\\
599.64	0.01\\
599.65	0.01\\
599.66	0.01\\
599.67	0.01\\
599.68	0.01\\
599.69	0.01\\
599.7	0.01\\
599.71	0.01\\
599.72	0.01\\
599.73	0.01\\
599.74	0.01\\
599.75	0.01\\
599.76	0.01\\
599.77	0.01\\
599.78	0.01\\
599.79	0.01\\
599.8	0.01\\
599.81	0.01\\
599.82	0.01\\
599.83	0.01\\
599.84	0.01\\
599.85	0.01\\
599.86	0.01\\
599.87	0.01\\
599.88	0.01\\
599.89	0.01\\
599.9	0.01\\
599.91	0.01\\
599.92	0.01\\
599.93	0.01\\
599.94	0.01\\
599.95	0.01\\
599.96	0.01\\
599.97	0.01\\
599.98	0.01\\
599.99	0.01\\
600	0.01\\
};
\addplot [color=mycolor7,solid,forget plot]
  table[row sep=crcr]{%
0.01	0.01\\
1.01	0.01\\
2.01	0.01\\
3.01	0.01\\
4.01	0.01\\
5.01	0.01\\
6.01	0.01\\
7.01	0.01\\
8.01	0.01\\
9.01	0.01\\
10.01	0.01\\
11.01	0.01\\
12.01	0.01\\
13.01	0.01\\
14.01	0.01\\
15.01	0.01\\
16.01	0.01\\
17.01	0.01\\
18.01	0.01\\
19.01	0.01\\
20.01	0.01\\
21.01	0.01\\
22.01	0.01\\
23.01	0.01\\
24.01	0.01\\
25.01	0.01\\
26.01	0.01\\
27.01	0.01\\
28.01	0.01\\
29.01	0.01\\
30.01	0.01\\
31.01	0.01\\
32.01	0.01\\
33.01	0.01\\
34.01	0.01\\
35.01	0.01\\
36.01	0.01\\
37.01	0.01\\
38.01	0.01\\
39.01	0.01\\
40.01	0.01\\
41.01	0.01\\
42.01	0.01\\
43.01	0.01\\
44.01	0.01\\
45.01	0.01\\
46.01	0.01\\
47.01	0.01\\
48.01	0.01\\
49.01	0.01\\
50.01	0.01\\
51.01	0.01\\
52.01	0.01\\
53.01	0.01\\
54.01	0.01\\
55.01	0.01\\
56.01	0.01\\
57.01	0.01\\
58.01	0.01\\
59.01	0.01\\
60.01	0.01\\
61.01	0.01\\
62.01	0.01\\
63.01	0.01\\
64.01	0.01\\
65.01	0.01\\
66.01	0.01\\
67.01	0.01\\
68.01	0.01\\
69.01	0.01\\
70.01	0.01\\
71.01	0.01\\
72.01	0.01\\
73.01	0.01\\
74.01	0.01\\
75.01	0.01\\
76.01	0.01\\
77.01	0.01\\
78.01	0.01\\
79.01	0.01\\
80.01	0.01\\
81.01	0.01\\
82.01	0.01\\
83.01	0.01\\
84.01	0.01\\
85.01	0.01\\
86.01	0.01\\
87.01	0.01\\
88.01	0.01\\
89.01	0.01\\
90.01	0.01\\
91.01	0.01\\
92.01	0.01\\
93.01	0.01\\
94.01	0.01\\
95.01	0.01\\
96.01	0.01\\
97.01	0.01\\
98.01	0.01\\
99.01	0.01\\
100.01	0.01\\
101.01	0.01\\
102.01	0.01\\
103.01	0.01\\
104.01	0.01\\
105.01	0.01\\
106.01	0.01\\
107.01	0.01\\
108.01	0.01\\
109.01	0.01\\
110.01	0.01\\
111.01	0.01\\
112.01	0.01\\
113.01	0.01\\
114.01	0.01\\
115.01	0.01\\
116.01	0.01\\
117.01	0.01\\
118.01	0.01\\
119.01	0.01\\
120.01	0.01\\
121.01	0.01\\
122.01	0.01\\
123.01	0.01\\
124.01	0.01\\
125.01	0.01\\
126.01	0.01\\
127.01	0.01\\
128.01	0.01\\
129.01	0.01\\
130.01	0.01\\
131.01	0.01\\
132.01	0.01\\
133.01	0.01\\
134.01	0.01\\
135.01	0.01\\
136.01	0.01\\
137.01	0.01\\
138.01	0.01\\
139.01	0.01\\
140.01	0.01\\
141.01	0.01\\
142.01	0.01\\
143.01	0.01\\
144.01	0.01\\
145.01	0.01\\
146.01	0.01\\
147.01	0.01\\
148.01	0.01\\
149.01	0.01\\
150.01	0.01\\
151.01	0.01\\
152.01	0.01\\
153.01	0.01\\
154.01	0.01\\
155.01	0.01\\
156.01	0.01\\
157.01	0.01\\
158.01	0.01\\
159.01	0.01\\
160.01	0.01\\
161.01	0.01\\
162.01	0.01\\
163.01	0.01\\
164.01	0.01\\
165.01	0.01\\
166.01	0.01\\
167.01	0.01\\
168.01	0.01\\
169.01	0.01\\
170.01	0.01\\
171.01	0.01\\
172.01	0.01\\
173.01	0.01\\
174.01	0.01\\
175.01	0.01\\
176.01	0.01\\
177.01	0.01\\
178.01	0.01\\
179.01	0.01\\
180.01	0.01\\
181.01	0.01\\
182.01	0.01\\
183.01	0.01\\
184.01	0.01\\
185.01	0.01\\
186.01	0.01\\
187.01	0.01\\
188.01	0.01\\
189.01	0.01\\
190.01	0.01\\
191.01	0.01\\
192.01	0.01\\
193.01	0.01\\
194.01	0.01\\
195.01	0.01\\
196.01	0.01\\
197.01	0.01\\
198.01	0.01\\
199.01	0.01\\
200.01	0.01\\
201.01	0.01\\
202.01	0.01\\
203.01	0.01\\
204.01	0.01\\
205.01	0.01\\
206.01	0.01\\
207.01	0.01\\
208.01	0.01\\
209.01	0.01\\
210.01	0.01\\
211.01	0.01\\
212.01	0.01\\
213.01	0.01\\
214.01	0.01\\
215.01	0.01\\
216.01	0.01\\
217.01	0.01\\
218.01	0.01\\
219.01	0.01\\
220.01	0.01\\
221.01	0.01\\
222.01	0.01\\
223.01	0.01\\
224.01	0.01\\
225.01	0.01\\
226.01	0.01\\
227.01	0.01\\
228.01	0.01\\
229.01	0.01\\
230.01	0.01\\
231.01	0.01\\
232.01	0.01\\
233.01	0.01\\
234.01	0.01\\
235.01	0.01\\
236.01	0.01\\
237.01	0.01\\
238.01	0.01\\
239.01	0.01\\
240.01	0.01\\
241.01	0.01\\
242.01	0.01\\
243.01	0.01\\
244.01	0.01\\
245.01	0.01\\
246.01	0.01\\
247.01	0.01\\
248.01	0.01\\
249.01	0.01\\
250.01	0.01\\
251.01	0.01\\
252.01	0.01\\
253.01	0.01\\
254.01	0.01\\
255.01	0.01\\
256.01	0.01\\
257.01	0.01\\
258.01	0.01\\
259.01	0.01\\
260.01	0.01\\
261.01	0.01\\
262.01	0.01\\
263.01	0.01\\
264.01	0.01\\
265.01	0.01\\
266.01	0.01\\
267.01	0.01\\
268.01	0.01\\
269.01	0.01\\
270.01	0.01\\
271.01	0.01\\
272.01	0.01\\
273.01	0.01\\
274.01	0.01\\
275.01	0.01\\
276.01	0.01\\
277.01	0.01\\
278.01	0.01\\
279.01	0.01\\
280.01	0.01\\
281.01	0.01\\
282.01	0.01\\
283.01	0.01\\
284.01	0.01\\
285.01	0.01\\
286.01	0.01\\
287.01	0.01\\
288.01	0.01\\
289.01	0.01\\
290.01	0.01\\
291.01	0.01\\
292.01	0.01\\
293.01	0.01\\
294.01	0.01\\
295.01	0.01\\
296.01	0.01\\
297.01	0.01\\
298.01	0.01\\
299.01	0.01\\
300.01	0.01\\
301.01	0.01\\
302.01	0.01\\
303.01	0.01\\
304.01	0.01\\
305.01	0.01\\
306.01	0.01\\
307.01	0.01\\
308.01	0.01\\
309.01	0.01\\
310.01	0.01\\
311.01	0.01\\
312.01	0.01\\
313.01	0.01\\
314.01	0.01\\
315.01	0.01\\
316.01	0.01\\
317.01	0.01\\
318.01	0.01\\
319.01	0.01\\
320.01	0.01\\
321.01	0.01\\
322.01	0.01\\
323.01	0.01\\
324.01	0.01\\
325.01	0.01\\
326.01	0.01\\
327.01	0.01\\
328.01	0.01\\
329.01	0.01\\
330.01	0.01\\
331.01	0.01\\
332.01	0.01\\
333.01	0.01\\
334.01	0.01\\
335.01	0.01\\
336.01	0.01\\
337.01	0.01\\
338.01	0.01\\
339.01	0.01\\
340.01	0.01\\
341.01	0.01\\
342.01	0.01\\
343.01	0.01\\
344.01	0.01\\
345.01	0.01\\
346.01	0.01\\
347.01	0.01\\
348.01	0.01\\
349.01	0.01\\
350.01	0.01\\
351.01	0.01\\
352.01	0.01\\
353.01	0.01\\
354.01	0.01\\
355.01	0.01\\
356.01	0.01\\
357.01	0.01\\
358.01	0.01\\
359.01	0.01\\
360.01	0.01\\
361.01	0.01\\
362.01	0.01\\
363.01	0.01\\
364.01	0.01\\
365.01	0.01\\
366.01	0.01\\
367.01	0.01\\
368.01	0.01\\
369.01	0.01\\
370.01	0.01\\
371.01	0.01\\
372.01	0.01\\
373.01	0.01\\
374.01	0.01\\
375.01	0.01\\
376.01	0.01\\
377.01	0.01\\
378.01	0.01\\
379.01	0.01\\
380.01	0.01\\
381.01	0.01\\
382.01	0.01\\
383.01	0.01\\
384.01	0.01\\
385.01	0.01\\
386.01	0.01\\
387.01	0.01\\
388.01	0.01\\
389.01	0.01\\
390.01	0.01\\
391.01	0.01\\
392.01	0.01\\
393.01	0.01\\
394.01	0.01\\
395.01	0.01\\
396.01	0.01\\
397.01	0.01\\
398.01	0.01\\
399.01	0.01\\
400.01	0.01\\
401.01	0.01\\
402.01	0.01\\
403.01	0.01\\
404.01	0.01\\
405.01	0.01\\
406.01	0.01\\
407.01	0.01\\
408.01	0.01\\
409.01	0.01\\
410.01	0.01\\
411.01	0.01\\
412.01	0.01\\
413.01	0.01\\
414.01	0.01\\
415.01	0.01\\
416.01	0.01\\
417.01	0.01\\
418.01	0.01\\
419.01	0.01\\
420.01	0.01\\
421.01	0.01\\
422.01	0.01\\
423.01	0.01\\
424.01	0.01\\
425.01	0.01\\
426.01	0.01\\
427.01	0.01\\
428.01	0.01\\
429.01	0.01\\
430.01	0.01\\
431.01	0.01\\
432.01	0.01\\
433.01	0.01\\
434.01	0.01\\
435.01	0.01\\
436.01	0.01\\
437.01	0.01\\
438.01	0.01\\
439.01	0.01\\
440.01	0.01\\
441.01	0.01\\
442.01	0.01\\
443.01	0.01\\
444.01	0.01\\
445.01	0.01\\
446.01	0.01\\
447.01	0.01\\
448.01	0.01\\
449.01	0.01\\
450.01	0.01\\
451.01	0.01\\
452.01	0.01\\
453.01	0.01\\
454.01	0.01\\
455.01	0.01\\
456.01	0.01\\
457.01	0.01\\
458.01	0.01\\
459.01	0.01\\
460.01	0.01\\
461.01	0.01\\
462.01	0.01\\
463.01	0.01\\
464.01	0.01\\
465.01	0.01\\
466.01	0.01\\
467.01	0.01\\
468.01	0.01\\
469.01	0.01\\
470.01	0.01\\
471.01	0.01\\
472.01	0.01\\
473.01	0.01\\
474.01	0.01\\
475.01	0.01\\
476.01	0.01\\
477.01	0.01\\
478.01	0.01\\
479.01	0.01\\
480.01	0.01\\
481.01	0.01\\
482.01	0.01\\
483.01	0.01\\
484.01	0.01\\
485.01	0.01\\
486.01	0.01\\
487.01	0.01\\
488.01	0.01\\
489.01	0.01\\
490.01	0.01\\
491.01	0.01\\
492.01	0.01\\
493.01	0.01\\
494.01	0.01\\
495.01	0.01\\
496.01	0.01\\
497.01	0.01\\
498.01	0.01\\
499.01	0.01\\
500.01	0.01\\
501.01	0.01\\
502.01	0.01\\
503.01	0.01\\
504.01	0.01\\
505.01	0.01\\
506.01	0.01\\
507.01	0.01\\
508.01	0.01\\
509.01	0.01\\
510.01	0.01\\
511.01	0.01\\
512.01	0.01\\
513.01	0.01\\
514.01	0.01\\
515.01	0.01\\
516.01	0.01\\
517.01	0.01\\
518.01	0.01\\
519.01	0.01\\
520.01	0.01\\
521.01	0.01\\
522.01	0.01\\
523.01	0.01\\
524.01	0.01\\
525.01	0.01\\
526.01	0.01\\
527.01	0.01\\
528.01	0.01\\
529.01	0.01\\
530.01	0.01\\
531.01	0.01\\
532.01	0.01\\
533.01	0.01\\
534.01	0.01\\
535.01	0.01\\
536.01	0.01\\
537.01	0.01\\
538.01	0.01\\
539.01	0.01\\
540.01	0.01\\
541.01	0.01\\
542.01	0.01\\
543.01	0.01\\
544.01	0.01\\
545.01	0.01\\
546.01	0.01\\
547.01	0.01\\
548.01	0.01\\
549.01	0.01\\
550.01	0.01\\
551.01	0.01\\
552.01	0.01\\
553.01	0.01\\
554.01	0.01\\
555.01	0.01\\
556.01	0.01\\
557.01	0.01\\
558.01	0.01\\
559.01	0.01\\
560.01	0.01\\
561.01	0.01\\
562.01	0.01\\
563.01	0.01\\
564.01	0.01\\
565.01	0.01\\
566.01	0.01\\
567.01	0.01\\
568.01	0.01\\
569.01	0.01\\
570.01	0.01\\
571.01	0.01\\
572.01	0.01\\
573.01	0.01\\
574.01	0.01\\
575.01	0.01\\
576.01	0.01\\
577.01	0.01\\
578.01	0.01\\
579.01	0.01\\
580.01	0.01\\
581.01	0.01\\
582.01	0.01\\
583.01	0.01\\
584.01	0.01\\
585.01	0.01\\
586.01	0.01\\
587.01	0.01\\
588.01	0.01\\
589.01	0.01\\
590.01	0.01\\
591.01	0.01\\
592.01	0.01\\
593.01	0.01\\
594.01	0.01\\
595.01	0.01\\
596.01	0.01\\
597.01	0.01\\
598.01	0.01\\
599.01	0.01\\
599.02	0.01\\
599.03	0.01\\
599.04	0.01\\
599.05	0.01\\
599.06	0.01\\
599.07	0.01\\
599.08	0.01\\
599.09	0.01\\
599.1	0.01\\
599.11	0.01\\
599.12	0.01\\
599.13	0.01\\
599.14	0.01\\
599.15	0.01\\
599.16	0.01\\
599.17	0.01\\
599.18	0.01\\
599.19	0.01\\
599.2	0.01\\
599.21	0.01\\
599.22	0.01\\
599.23	0.01\\
599.24	0.01\\
599.25	0.01\\
599.26	0.01\\
599.27	0.01\\
599.28	0.01\\
599.29	0.01\\
599.3	0.01\\
599.31	0.01\\
599.32	0.01\\
599.33	0.01\\
599.34	0.01\\
599.35	0.01\\
599.36	0.01\\
599.37	0.01\\
599.38	0.01\\
599.39	0.01\\
599.4	0.01\\
599.41	0.01\\
599.42	0.01\\
599.43	0.01\\
599.44	0.01\\
599.45	0.01\\
599.46	0.01\\
599.47	0.01\\
599.48	0.01\\
599.49	0.01\\
599.5	0.01\\
599.51	0.01\\
599.52	0.01\\
599.53	0.01\\
599.54	0.01\\
599.55	0.01\\
599.56	0.01\\
599.57	0.01\\
599.58	0.01\\
599.59	0.01\\
599.6	0.01\\
599.61	0.01\\
599.62	0.01\\
599.63	0.01\\
599.64	0.01\\
599.65	0.01\\
599.66	0.01\\
599.67	0.01\\
599.68	0.01\\
599.69	0.01\\
599.7	0.01\\
599.71	0.01\\
599.72	0.01\\
599.73	0.01\\
599.74	0.01\\
599.75	0.01\\
599.76	0.01\\
599.77	0.01\\
599.78	0.01\\
599.79	0.01\\
599.8	0.01\\
599.81	0.01\\
599.82	0.01\\
599.83	0.01\\
599.84	0.01\\
599.85	0.01\\
599.86	0.01\\
599.87	0.01\\
599.88	0.01\\
599.89	0.01\\
599.9	0.01\\
599.91	0.01\\
599.92	0.01\\
599.93	0.01\\
599.94	0.01\\
599.95	0.01\\
599.96	0.01\\
599.97	0.01\\
599.98	0.01\\
599.99	0.01\\
600	0.01\\
};
\addplot [color=mycolor8,solid,forget plot]
  table[row sep=crcr]{%
0.01	0.01\\
1.01	0.01\\
2.01	0.01\\
3.01	0.01\\
4.01	0.01\\
5.01	0.01\\
6.01	0.01\\
7.01	0.01\\
8.01	0.01\\
9.01	0.01\\
10.01	0.01\\
11.01	0.01\\
12.01	0.01\\
13.01	0.01\\
14.01	0.01\\
15.01	0.01\\
16.01	0.01\\
17.01	0.01\\
18.01	0.01\\
19.01	0.01\\
20.01	0.01\\
21.01	0.01\\
22.01	0.01\\
23.01	0.01\\
24.01	0.01\\
25.01	0.01\\
26.01	0.01\\
27.01	0.01\\
28.01	0.01\\
29.01	0.01\\
30.01	0.01\\
31.01	0.01\\
32.01	0.01\\
33.01	0.01\\
34.01	0.01\\
35.01	0.01\\
36.01	0.01\\
37.01	0.01\\
38.01	0.01\\
39.01	0.01\\
40.01	0.01\\
41.01	0.01\\
42.01	0.01\\
43.01	0.01\\
44.01	0.01\\
45.01	0.01\\
46.01	0.01\\
47.01	0.01\\
48.01	0.01\\
49.01	0.01\\
50.01	0.01\\
51.01	0.01\\
52.01	0.01\\
53.01	0.01\\
54.01	0.01\\
55.01	0.01\\
56.01	0.01\\
57.01	0.01\\
58.01	0.01\\
59.01	0.01\\
60.01	0.01\\
61.01	0.01\\
62.01	0.01\\
63.01	0.01\\
64.01	0.01\\
65.01	0.01\\
66.01	0.01\\
67.01	0.01\\
68.01	0.01\\
69.01	0.01\\
70.01	0.01\\
71.01	0.01\\
72.01	0.01\\
73.01	0.01\\
74.01	0.01\\
75.01	0.01\\
76.01	0.01\\
77.01	0.01\\
78.01	0.01\\
79.01	0.01\\
80.01	0.01\\
81.01	0.01\\
82.01	0.01\\
83.01	0.01\\
84.01	0.01\\
85.01	0.01\\
86.01	0.01\\
87.01	0.01\\
88.01	0.01\\
89.01	0.01\\
90.01	0.01\\
91.01	0.01\\
92.01	0.01\\
93.01	0.01\\
94.01	0.01\\
95.01	0.01\\
96.01	0.01\\
97.01	0.01\\
98.01	0.01\\
99.01	0.01\\
100.01	0.01\\
101.01	0.01\\
102.01	0.01\\
103.01	0.01\\
104.01	0.01\\
105.01	0.01\\
106.01	0.01\\
107.01	0.01\\
108.01	0.01\\
109.01	0.01\\
110.01	0.01\\
111.01	0.01\\
112.01	0.01\\
113.01	0.01\\
114.01	0.01\\
115.01	0.01\\
116.01	0.01\\
117.01	0.01\\
118.01	0.01\\
119.01	0.01\\
120.01	0.01\\
121.01	0.01\\
122.01	0.01\\
123.01	0.01\\
124.01	0.01\\
125.01	0.01\\
126.01	0.01\\
127.01	0.01\\
128.01	0.01\\
129.01	0.01\\
130.01	0.01\\
131.01	0.01\\
132.01	0.01\\
133.01	0.01\\
134.01	0.01\\
135.01	0.01\\
136.01	0.01\\
137.01	0.01\\
138.01	0.01\\
139.01	0.01\\
140.01	0.01\\
141.01	0.01\\
142.01	0.01\\
143.01	0.01\\
144.01	0.01\\
145.01	0.01\\
146.01	0.01\\
147.01	0.01\\
148.01	0.01\\
149.01	0.01\\
150.01	0.01\\
151.01	0.01\\
152.01	0.01\\
153.01	0.01\\
154.01	0.01\\
155.01	0.01\\
156.01	0.01\\
157.01	0.01\\
158.01	0.01\\
159.01	0.01\\
160.01	0.01\\
161.01	0.01\\
162.01	0.01\\
163.01	0.01\\
164.01	0.01\\
165.01	0.01\\
166.01	0.01\\
167.01	0.01\\
168.01	0.01\\
169.01	0.01\\
170.01	0.01\\
171.01	0.01\\
172.01	0.01\\
173.01	0.01\\
174.01	0.01\\
175.01	0.01\\
176.01	0.01\\
177.01	0.01\\
178.01	0.01\\
179.01	0.01\\
180.01	0.01\\
181.01	0.01\\
182.01	0.01\\
183.01	0.01\\
184.01	0.01\\
185.01	0.01\\
186.01	0.01\\
187.01	0.01\\
188.01	0.01\\
189.01	0.01\\
190.01	0.01\\
191.01	0.01\\
192.01	0.01\\
193.01	0.01\\
194.01	0.01\\
195.01	0.01\\
196.01	0.01\\
197.01	0.01\\
198.01	0.01\\
199.01	0.01\\
200.01	0.01\\
201.01	0.01\\
202.01	0.01\\
203.01	0.01\\
204.01	0.01\\
205.01	0.01\\
206.01	0.01\\
207.01	0.01\\
208.01	0.01\\
209.01	0.01\\
210.01	0.01\\
211.01	0.01\\
212.01	0.01\\
213.01	0.01\\
214.01	0.01\\
215.01	0.01\\
216.01	0.01\\
217.01	0.01\\
218.01	0.01\\
219.01	0.01\\
220.01	0.01\\
221.01	0.01\\
222.01	0.01\\
223.01	0.01\\
224.01	0.01\\
225.01	0.01\\
226.01	0.01\\
227.01	0.01\\
228.01	0.01\\
229.01	0.01\\
230.01	0.01\\
231.01	0.01\\
232.01	0.01\\
233.01	0.01\\
234.01	0.01\\
235.01	0.01\\
236.01	0.01\\
237.01	0.01\\
238.01	0.01\\
239.01	0.01\\
240.01	0.01\\
241.01	0.01\\
242.01	0.01\\
243.01	0.01\\
244.01	0.01\\
245.01	0.01\\
246.01	0.01\\
247.01	0.01\\
248.01	0.01\\
249.01	0.01\\
250.01	0.01\\
251.01	0.01\\
252.01	0.01\\
253.01	0.01\\
254.01	0.01\\
255.01	0.01\\
256.01	0.01\\
257.01	0.01\\
258.01	0.01\\
259.01	0.01\\
260.01	0.01\\
261.01	0.01\\
262.01	0.01\\
263.01	0.01\\
264.01	0.01\\
265.01	0.01\\
266.01	0.01\\
267.01	0.01\\
268.01	0.01\\
269.01	0.01\\
270.01	0.01\\
271.01	0.01\\
272.01	0.01\\
273.01	0.01\\
274.01	0.01\\
275.01	0.01\\
276.01	0.01\\
277.01	0.01\\
278.01	0.01\\
279.01	0.01\\
280.01	0.01\\
281.01	0.01\\
282.01	0.01\\
283.01	0.01\\
284.01	0.01\\
285.01	0.01\\
286.01	0.01\\
287.01	0.01\\
288.01	0.01\\
289.01	0.01\\
290.01	0.01\\
291.01	0.01\\
292.01	0.01\\
293.01	0.01\\
294.01	0.01\\
295.01	0.01\\
296.01	0.01\\
297.01	0.01\\
298.01	0.01\\
299.01	0.01\\
300.01	0.01\\
301.01	0.01\\
302.01	0.01\\
303.01	0.01\\
304.01	0.01\\
305.01	0.01\\
306.01	0.01\\
307.01	0.01\\
308.01	0.01\\
309.01	0.01\\
310.01	0.01\\
311.01	0.01\\
312.01	0.01\\
313.01	0.01\\
314.01	0.01\\
315.01	0.01\\
316.01	0.01\\
317.01	0.01\\
318.01	0.01\\
319.01	0.01\\
320.01	0.01\\
321.01	0.01\\
322.01	0.01\\
323.01	0.01\\
324.01	0.01\\
325.01	0.01\\
326.01	0.01\\
327.01	0.01\\
328.01	0.01\\
329.01	0.01\\
330.01	0.01\\
331.01	0.01\\
332.01	0.01\\
333.01	0.01\\
334.01	0.01\\
335.01	0.01\\
336.01	0.01\\
337.01	0.01\\
338.01	0.01\\
339.01	0.01\\
340.01	0.01\\
341.01	0.01\\
342.01	0.01\\
343.01	0.01\\
344.01	0.01\\
345.01	0.01\\
346.01	0.01\\
347.01	0.01\\
348.01	0.01\\
349.01	0.01\\
350.01	0.01\\
351.01	0.01\\
352.01	0.01\\
353.01	0.01\\
354.01	0.01\\
355.01	0.01\\
356.01	0.01\\
357.01	0.01\\
358.01	0.01\\
359.01	0.01\\
360.01	0.01\\
361.01	0.01\\
362.01	0.01\\
363.01	0.01\\
364.01	0.01\\
365.01	0.01\\
366.01	0.01\\
367.01	0.01\\
368.01	0.01\\
369.01	0.01\\
370.01	0.01\\
371.01	0.01\\
372.01	0.01\\
373.01	0.01\\
374.01	0.01\\
375.01	0.01\\
376.01	0.01\\
377.01	0.01\\
378.01	0.01\\
379.01	0.01\\
380.01	0.01\\
381.01	0.01\\
382.01	0.01\\
383.01	0.01\\
384.01	0.01\\
385.01	0.01\\
386.01	0.01\\
387.01	0.01\\
388.01	0.01\\
389.01	0.01\\
390.01	0.01\\
391.01	0.01\\
392.01	0.01\\
393.01	0.01\\
394.01	0.01\\
395.01	0.01\\
396.01	0.01\\
397.01	0.01\\
398.01	0.01\\
399.01	0.01\\
400.01	0.01\\
401.01	0.01\\
402.01	0.01\\
403.01	0.01\\
404.01	0.01\\
405.01	0.01\\
406.01	0.01\\
407.01	0.01\\
408.01	0.01\\
409.01	0.01\\
410.01	0.01\\
411.01	0.01\\
412.01	0.01\\
413.01	0.01\\
414.01	0.01\\
415.01	0.01\\
416.01	0.01\\
417.01	0.01\\
418.01	0.01\\
419.01	0.01\\
420.01	0.01\\
421.01	0.01\\
422.01	0.01\\
423.01	0.01\\
424.01	0.01\\
425.01	0.01\\
426.01	0.01\\
427.01	0.01\\
428.01	0.01\\
429.01	0.01\\
430.01	0.01\\
431.01	0.01\\
432.01	0.01\\
433.01	0.01\\
434.01	0.01\\
435.01	0.01\\
436.01	0.01\\
437.01	0.01\\
438.01	0.01\\
439.01	0.01\\
440.01	0.01\\
441.01	0.01\\
442.01	0.01\\
443.01	0.01\\
444.01	0.01\\
445.01	0.01\\
446.01	0.01\\
447.01	0.01\\
448.01	0.01\\
449.01	0.01\\
450.01	0.01\\
451.01	0.01\\
452.01	0.01\\
453.01	0.01\\
454.01	0.01\\
455.01	0.01\\
456.01	0.01\\
457.01	0.01\\
458.01	0.01\\
459.01	0.01\\
460.01	0.01\\
461.01	0.01\\
462.01	0.01\\
463.01	0.01\\
464.01	0.01\\
465.01	0.01\\
466.01	0.01\\
467.01	0.01\\
468.01	0.01\\
469.01	0.01\\
470.01	0.01\\
471.01	0.01\\
472.01	0.01\\
473.01	0.01\\
474.01	0.01\\
475.01	0.01\\
476.01	0.01\\
477.01	0.01\\
478.01	0.01\\
479.01	0.01\\
480.01	0.01\\
481.01	0.01\\
482.01	0.01\\
483.01	0.01\\
484.01	0.01\\
485.01	0.01\\
486.01	0.01\\
487.01	0.01\\
488.01	0.01\\
489.01	0.01\\
490.01	0.01\\
491.01	0.01\\
492.01	0.01\\
493.01	0.01\\
494.01	0.01\\
495.01	0.01\\
496.01	0.01\\
497.01	0.01\\
498.01	0.01\\
499.01	0.01\\
500.01	0.01\\
501.01	0.01\\
502.01	0.01\\
503.01	0.01\\
504.01	0.01\\
505.01	0.01\\
506.01	0.01\\
507.01	0.01\\
508.01	0.01\\
509.01	0.01\\
510.01	0.01\\
511.01	0.01\\
512.01	0.01\\
513.01	0.01\\
514.01	0.01\\
515.01	0.01\\
516.01	0.01\\
517.01	0.01\\
518.01	0.01\\
519.01	0.01\\
520.01	0.01\\
521.01	0.01\\
522.01	0.01\\
523.01	0.01\\
524.01	0.01\\
525.01	0.01\\
526.01	0.01\\
527.01	0.01\\
528.01	0.01\\
529.01	0.01\\
530.01	0.01\\
531.01	0.01\\
532.01	0.01\\
533.01	0.01\\
534.01	0.01\\
535.01	0.01\\
536.01	0.01\\
537.01	0.01\\
538.01	0.01\\
539.01	0.01\\
540.01	0.01\\
541.01	0.01\\
542.01	0.01\\
543.01	0.01\\
544.01	0.01\\
545.01	0.01\\
546.01	0.01\\
547.01	0.01\\
548.01	0.01\\
549.01	0.01\\
550.01	0.01\\
551.01	0.01\\
552.01	0.01\\
553.01	0.01\\
554.01	0.01\\
555.01	0.01\\
556.01	0.01\\
557.01	0.01\\
558.01	0.01\\
559.01	0.01\\
560.01	0.01\\
561.01	0.01\\
562.01	0.01\\
563.01	0.01\\
564.01	0.01\\
565.01	0.01\\
566.01	0.01\\
567.01	0.01\\
568.01	0.01\\
569.01	0.01\\
570.01	0.01\\
571.01	0.01\\
572.01	0.01\\
573.01	0.01\\
574.01	0.01\\
575.01	0.01\\
576.01	0.01\\
577.01	0.01\\
578.01	0.01\\
579.01	0.01\\
580.01	0.01\\
581.01	0.01\\
582.01	0.01\\
583.01	0.01\\
584.01	0.01\\
585.01	0.01\\
586.01	0.01\\
587.01	0.01\\
588.01	0.01\\
589.01	0.01\\
590.01	0.01\\
591.01	0.01\\
592.01	0.01\\
593.01	0.01\\
594.01	0.01\\
595.01	0.01\\
596.01	0.01\\
597.01	0.01\\
598.01	0.01\\
599.01	0.01\\
599.02	0.01\\
599.03	0.01\\
599.04	0.01\\
599.05	0.01\\
599.06	0.01\\
599.07	0.01\\
599.08	0.01\\
599.09	0.01\\
599.1	0.01\\
599.11	0.01\\
599.12	0.01\\
599.13	0.01\\
599.14	0.01\\
599.15	0.01\\
599.16	0.01\\
599.17	0.01\\
599.18	0.01\\
599.19	0.01\\
599.2	0.01\\
599.21	0.01\\
599.22	0.01\\
599.23	0.01\\
599.24	0.01\\
599.25	0.01\\
599.26	0.01\\
599.27	0.01\\
599.28	0.01\\
599.29	0.01\\
599.3	0.01\\
599.31	0.01\\
599.32	0.01\\
599.33	0.01\\
599.34	0.01\\
599.35	0.01\\
599.36	0.01\\
599.37	0.01\\
599.38	0.01\\
599.39	0.01\\
599.4	0.01\\
599.41	0.01\\
599.42	0.01\\
599.43	0.01\\
599.44	0.01\\
599.45	0.01\\
599.46	0.01\\
599.47	0.01\\
599.48	0.01\\
599.49	0.01\\
599.5	0.01\\
599.51	0.01\\
599.52	0.01\\
599.53	0.01\\
599.54	0.01\\
599.55	0.01\\
599.56	0.01\\
599.57	0.01\\
599.58	0.01\\
599.59	0.01\\
599.6	0.01\\
599.61	0.01\\
599.62	0.01\\
599.63	0.01\\
599.64	0.01\\
599.65	0.01\\
599.66	0.01\\
599.67	0.01\\
599.68	0.01\\
599.69	0.01\\
599.7	0.01\\
599.71	0.01\\
599.72	0.01\\
599.73	0.01\\
599.74	0.01\\
599.75	0.01\\
599.76	0.01\\
599.77	0.01\\
599.78	0.01\\
599.79	0.01\\
599.8	0.01\\
599.81	0.01\\
599.82	0.01\\
599.83	0.01\\
599.84	0.01\\
599.85	0.01\\
599.86	0.01\\
599.87	0.01\\
599.88	0.01\\
599.89	0.01\\
599.9	0.01\\
599.91	0.01\\
599.92	0.01\\
599.93	0.01\\
599.94	0.01\\
599.95	0.01\\
599.96	0.01\\
599.97	0.01\\
599.98	0.01\\
599.99	0.01\\
600	0.01\\
};
\addplot [color=blue!25!mycolor7,solid,forget plot]
  table[row sep=crcr]{%
0.01	0.01\\
1.01	0.01\\
2.01	0.01\\
3.01	0.01\\
4.01	0.01\\
5.01	0.01\\
6.01	0.01\\
7.01	0.01\\
8.01	0.01\\
9.01	0.01\\
10.01	0.01\\
11.01	0.01\\
12.01	0.01\\
13.01	0.01\\
14.01	0.01\\
15.01	0.01\\
16.01	0.01\\
17.01	0.01\\
18.01	0.01\\
19.01	0.01\\
20.01	0.01\\
21.01	0.01\\
22.01	0.01\\
23.01	0.01\\
24.01	0.01\\
25.01	0.01\\
26.01	0.01\\
27.01	0.01\\
28.01	0.01\\
29.01	0.01\\
30.01	0.01\\
31.01	0.01\\
32.01	0.01\\
33.01	0.01\\
34.01	0.01\\
35.01	0.01\\
36.01	0.01\\
37.01	0.01\\
38.01	0.01\\
39.01	0.01\\
40.01	0.01\\
41.01	0.01\\
42.01	0.01\\
43.01	0.01\\
44.01	0.01\\
45.01	0.01\\
46.01	0.01\\
47.01	0.01\\
48.01	0.01\\
49.01	0.01\\
50.01	0.01\\
51.01	0.01\\
52.01	0.01\\
53.01	0.01\\
54.01	0.01\\
55.01	0.01\\
56.01	0.01\\
57.01	0.01\\
58.01	0.01\\
59.01	0.01\\
60.01	0.01\\
61.01	0.01\\
62.01	0.01\\
63.01	0.01\\
64.01	0.01\\
65.01	0.01\\
66.01	0.01\\
67.01	0.01\\
68.01	0.01\\
69.01	0.01\\
70.01	0.01\\
71.01	0.01\\
72.01	0.01\\
73.01	0.01\\
74.01	0.01\\
75.01	0.01\\
76.01	0.01\\
77.01	0.01\\
78.01	0.01\\
79.01	0.01\\
80.01	0.01\\
81.01	0.01\\
82.01	0.01\\
83.01	0.01\\
84.01	0.01\\
85.01	0.01\\
86.01	0.01\\
87.01	0.01\\
88.01	0.01\\
89.01	0.01\\
90.01	0.01\\
91.01	0.01\\
92.01	0.01\\
93.01	0.01\\
94.01	0.01\\
95.01	0.01\\
96.01	0.01\\
97.01	0.01\\
98.01	0.01\\
99.01	0.01\\
100.01	0.01\\
101.01	0.01\\
102.01	0.01\\
103.01	0.01\\
104.01	0.01\\
105.01	0.01\\
106.01	0.01\\
107.01	0.01\\
108.01	0.01\\
109.01	0.01\\
110.01	0.01\\
111.01	0.01\\
112.01	0.01\\
113.01	0.01\\
114.01	0.01\\
115.01	0.01\\
116.01	0.01\\
117.01	0.01\\
118.01	0.01\\
119.01	0.01\\
120.01	0.01\\
121.01	0.01\\
122.01	0.01\\
123.01	0.01\\
124.01	0.01\\
125.01	0.01\\
126.01	0.01\\
127.01	0.01\\
128.01	0.01\\
129.01	0.01\\
130.01	0.01\\
131.01	0.01\\
132.01	0.01\\
133.01	0.01\\
134.01	0.01\\
135.01	0.01\\
136.01	0.01\\
137.01	0.01\\
138.01	0.01\\
139.01	0.01\\
140.01	0.01\\
141.01	0.01\\
142.01	0.01\\
143.01	0.01\\
144.01	0.01\\
145.01	0.01\\
146.01	0.01\\
147.01	0.01\\
148.01	0.01\\
149.01	0.01\\
150.01	0.01\\
151.01	0.01\\
152.01	0.01\\
153.01	0.01\\
154.01	0.01\\
155.01	0.01\\
156.01	0.01\\
157.01	0.01\\
158.01	0.01\\
159.01	0.01\\
160.01	0.01\\
161.01	0.01\\
162.01	0.01\\
163.01	0.01\\
164.01	0.01\\
165.01	0.01\\
166.01	0.01\\
167.01	0.01\\
168.01	0.01\\
169.01	0.01\\
170.01	0.01\\
171.01	0.01\\
172.01	0.01\\
173.01	0.01\\
174.01	0.01\\
175.01	0.01\\
176.01	0.01\\
177.01	0.01\\
178.01	0.01\\
179.01	0.01\\
180.01	0.01\\
181.01	0.01\\
182.01	0.01\\
183.01	0.01\\
184.01	0.01\\
185.01	0.01\\
186.01	0.01\\
187.01	0.01\\
188.01	0.01\\
189.01	0.01\\
190.01	0.01\\
191.01	0.01\\
192.01	0.01\\
193.01	0.01\\
194.01	0.01\\
195.01	0.01\\
196.01	0.01\\
197.01	0.01\\
198.01	0.01\\
199.01	0.01\\
200.01	0.01\\
201.01	0.01\\
202.01	0.01\\
203.01	0.01\\
204.01	0.01\\
205.01	0.01\\
206.01	0.01\\
207.01	0.01\\
208.01	0.01\\
209.01	0.01\\
210.01	0.01\\
211.01	0.01\\
212.01	0.01\\
213.01	0.01\\
214.01	0.01\\
215.01	0.01\\
216.01	0.01\\
217.01	0.01\\
218.01	0.01\\
219.01	0.01\\
220.01	0.01\\
221.01	0.01\\
222.01	0.01\\
223.01	0.01\\
224.01	0.01\\
225.01	0.01\\
226.01	0.01\\
227.01	0.01\\
228.01	0.01\\
229.01	0.01\\
230.01	0.01\\
231.01	0.01\\
232.01	0.01\\
233.01	0.01\\
234.01	0.01\\
235.01	0.01\\
236.01	0.01\\
237.01	0.01\\
238.01	0.01\\
239.01	0.01\\
240.01	0.01\\
241.01	0.01\\
242.01	0.01\\
243.01	0.01\\
244.01	0.01\\
245.01	0.01\\
246.01	0.01\\
247.01	0.01\\
248.01	0.01\\
249.01	0.01\\
250.01	0.01\\
251.01	0.01\\
252.01	0.01\\
253.01	0.01\\
254.01	0.01\\
255.01	0.01\\
256.01	0.01\\
257.01	0.01\\
258.01	0.01\\
259.01	0.01\\
260.01	0.01\\
261.01	0.01\\
262.01	0.01\\
263.01	0.01\\
264.01	0.01\\
265.01	0.01\\
266.01	0.01\\
267.01	0.01\\
268.01	0.01\\
269.01	0.01\\
270.01	0.01\\
271.01	0.01\\
272.01	0.01\\
273.01	0.01\\
274.01	0.01\\
275.01	0.01\\
276.01	0.01\\
277.01	0.01\\
278.01	0.01\\
279.01	0.01\\
280.01	0.01\\
281.01	0.01\\
282.01	0.01\\
283.01	0.01\\
284.01	0.01\\
285.01	0.01\\
286.01	0.01\\
287.01	0.01\\
288.01	0.01\\
289.01	0.01\\
290.01	0.01\\
291.01	0.01\\
292.01	0.01\\
293.01	0.01\\
294.01	0.01\\
295.01	0.01\\
296.01	0.01\\
297.01	0.01\\
298.01	0.01\\
299.01	0.01\\
300.01	0.01\\
301.01	0.01\\
302.01	0.01\\
303.01	0.01\\
304.01	0.01\\
305.01	0.01\\
306.01	0.01\\
307.01	0.01\\
308.01	0.01\\
309.01	0.01\\
310.01	0.01\\
311.01	0.01\\
312.01	0.01\\
313.01	0.01\\
314.01	0.01\\
315.01	0.01\\
316.01	0.01\\
317.01	0.01\\
318.01	0.01\\
319.01	0.01\\
320.01	0.01\\
321.01	0.01\\
322.01	0.01\\
323.01	0.01\\
324.01	0.01\\
325.01	0.01\\
326.01	0.01\\
327.01	0.01\\
328.01	0.01\\
329.01	0.01\\
330.01	0.01\\
331.01	0.01\\
332.01	0.01\\
333.01	0.01\\
334.01	0.01\\
335.01	0.01\\
336.01	0.01\\
337.01	0.01\\
338.01	0.01\\
339.01	0.01\\
340.01	0.01\\
341.01	0.01\\
342.01	0.01\\
343.01	0.01\\
344.01	0.01\\
345.01	0.01\\
346.01	0.01\\
347.01	0.01\\
348.01	0.01\\
349.01	0.01\\
350.01	0.01\\
351.01	0.01\\
352.01	0.01\\
353.01	0.01\\
354.01	0.01\\
355.01	0.01\\
356.01	0.01\\
357.01	0.01\\
358.01	0.01\\
359.01	0.01\\
360.01	0.01\\
361.01	0.01\\
362.01	0.01\\
363.01	0.01\\
364.01	0.01\\
365.01	0.01\\
366.01	0.01\\
367.01	0.01\\
368.01	0.01\\
369.01	0.01\\
370.01	0.01\\
371.01	0.01\\
372.01	0.01\\
373.01	0.01\\
374.01	0.01\\
375.01	0.01\\
376.01	0.01\\
377.01	0.01\\
378.01	0.01\\
379.01	0.01\\
380.01	0.01\\
381.01	0.01\\
382.01	0.01\\
383.01	0.01\\
384.01	0.01\\
385.01	0.01\\
386.01	0.01\\
387.01	0.01\\
388.01	0.01\\
389.01	0.01\\
390.01	0.01\\
391.01	0.01\\
392.01	0.01\\
393.01	0.01\\
394.01	0.01\\
395.01	0.01\\
396.01	0.01\\
397.01	0.01\\
398.01	0.01\\
399.01	0.01\\
400.01	0.01\\
401.01	0.01\\
402.01	0.01\\
403.01	0.01\\
404.01	0.01\\
405.01	0.01\\
406.01	0.01\\
407.01	0.01\\
408.01	0.01\\
409.01	0.01\\
410.01	0.01\\
411.01	0.01\\
412.01	0.01\\
413.01	0.01\\
414.01	0.01\\
415.01	0.01\\
416.01	0.01\\
417.01	0.01\\
418.01	0.01\\
419.01	0.01\\
420.01	0.01\\
421.01	0.01\\
422.01	0.01\\
423.01	0.01\\
424.01	0.01\\
425.01	0.01\\
426.01	0.01\\
427.01	0.01\\
428.01	0.01\\
429.01	0.01\\
430.01	0.01\\
431.01	0.01\\
432.01	0.01\\
433.01	0.01\\
434.01	0.01\\
435.01	0.01\\
436.01	0.01\\
437.01	0.01\\
438.01	0.01\\
439.01	0.01\\
440.01	0.01\\
441.01	0.01\\
442.01	0.01\\
443.01	0.01\\
444.01	0.01\\
445.01	0.01\\
446.01	0.01\\
447.01	0.01\\
448.01	0.01\\
449.01	0.01\\
450.01	0.01\\
451.01	0.01\\
452.01	0.01\\
453.01	0.01\\
454.01	0.01\\
455.01	0.01\\
456.01	0.01\\
457.01	0.01\\
458.01	0.01\\
459.01	0.01\\
460.01	0.01\\
461.01	0.01\\
462.01	0.01\\
463.01	0.01\\
464.01	0.01\\
465.01	0.01\\
466.01	0.01\\
467.01	0.01\\
468.01	0.01\\
469.01	0.01\\
470.01	0.01\\
471.01	0.01\\
472.01	0.01\\
473.01	0.01\\
474.01	0.01\\
475.01	0.01\\
476.01	0.01\\
477.01	0.01\\
478.01	0.01\\
479.01	0.01\\
480.01	0.01\\
481.01	0.01\\
482.01	0.01\\
483.01	0.01\\
484.01	0.01\\
485.01	0.01\\
486.01	0.01\\
487.01	0.01\\
488.01	0.01\\
489.01	0.01\\
490.01	0.01\\
491.01	0.01\\
492.01	0.01\\
493.01	0.01\\
494.01	0.01\\
495.01	0.01\\
496.01	0.01\\
497.01	0.01\\
498.01	0.01\\
499.01	0.01\\
500.01	0.01\\
501.01	0.01\\
502.01	0.01\\
503.01	0.01\\
504.01	0.01\\
505.01	0.01\\
506.01	0.01\\
507.01	0.01\\
508.01	0.01\\
509.01	0.01\\
510.01	0.01\\
511.01	0.01\\
512.01	0.01\\
513.01	0.01\\
514.01	0.01\\
515.01	0.01\\
516.01	0.01\\
517.01	0.01\\
518.01	0.01\\
519.01	0.01\\
520.01	0.01\\
521.01	0.01\\
522.01	0.01\\
523.01	0.01\\
524.01	0.01\\
525.01	0.01\\
526.01	0.01\\
527.01	0.01\\
528.01	0.01\\
529.01	0.01\\
530.01	0.01\\
531.01	0.01\\
532.01	0.01\\
533.01	0.01\\
534.01	0.01\\
535.01	0.01\\
536.01	0.01\\
537.01	0.01\\
538.01	0.01\\
539.01	0.01\\
540.01	0.01\\
541.01	0.01\\
542.01	0.01\\
543.01	0.01\\
544.01	0.01\\
545.01	0.01\\
546.01	0.01\\
547.01	0.01\\
548.01	0.01\\
549.01	0.01\\
550.01	0.01\\
551.01	0.01\\
552.01	0.01\\
553.01	0.01\\
554.01	0.01\\
555.01	0.01\\
556.01	0.01\\
557.01	0.01\\
558.01	0.01\\
559.01	0.01\\
560.01	0.01\\
561.01	0.01\\
562.01	0.01\\
563.01	0.01\\
564.01	0.01\\
565.01	0.01\\
566.01	0.01\\
567.01	0.01\\
568.01	0.01\\
569.01	0.01\\
570.01	0.01\\
571.01	0.01\\
572.01	0.01\\
573.01	0.01\\
574.01	0.01\\
575.01	0.01\\
576.01	0.01\\
577.01	0.01\\
578.01	0.01\\
579.01	0.01\\
580.01	0.01\\
581.01	0.01\\
582.01	0.01\\
583.01	0.01\\
584.01	0.01\\
585.01	0.01\\
586.01	0.01\\
587.01	0.01\\
588.01	0.01\\
589.01	0.01\\
590.01	0.01\\
591.01	0.01\\
592.01	0.01\\
593.01	0.01\\
594.01	0.01\\
595.01	0.01\\
596.01	0.01\\
597.01	0.01\\
598.01	0.01\\
599.01	0.01\\
599.02	0.01\\
599.03	0.01\\
599.04	0.01\\
599.05	0.01\\
599.06	0.01\\
599.07	0.01\\
599.08	0.01\\
599.09	0.01\\
599.1	0.01\\
599.11	0.01\\
599.12	0.01\\
599.13	0.01\\
599.14	0.01\\
599.15	0.01\\
599.16	0.01\\
599.17	0.01\\
599.18	0.01\\
599.19	0.01\\
599.2	0.01\\
599.21	0.01\\
599.22	0.01\\
599.23	0.01\\
599.24	0.01\\
599.25	0.01\\
599.26	0.01\\
599.27	0.01\\
599.28	0.01\\
599.29	0.01\\
599.3	0.01\\
599.31	0.01\\
599.32	0.01\\
599.33	0.01\\
599.34	0.01\\
599.35	0.01\\
599.36	0.01\\
599.37	0.01\\
599.38	0.01\\
599.39	0.01\\
599.4	0.01\\
599.41	0.01\\
599.42	0.01\\
599.43	0.01\\
599.44	0.01\\
599.45	0.01\\
599.46	0.01\\
599.47	0.01\\
599.48	0.01\\
599.49	0.01\\
599.5	0.01\\
599.51	0.01\\
599.52	0.01\\
599.53	0.01\\
599.54	0.01\\
599.55	0.01\\
599.56	0.01\\
599.57	0.01\\
599.58	0.01\\
599.59	0.01\\
599.6	0.01\\
599.61	0.01\\
599.62	0.01\\
599.63	0.01\\
599.64	0.01\\
599.65	0.01\\
599.66	0.01\\
599.67	0.01\\
599.68	0.01\\
599.69	0.01\\
599.7	0.01\\
599.71	0.01\\
599.72	0.01\\
599.73	0.01\\
599.74	0.01\\
599.75	0.01\\
599.76	0.01\\
599.77	0.01\\
599.78	0.01\\
599.79	0.01\\
599.8	0.01\\
599.81	0.01\\
599.82	0.01\\
599.83	0.01\\
599.84	0.01\\
599.85	0.01\\
599.86	0.01\\
599.87	0.01\\
599.88	0.01\\
599.89	0.01\\
599.9	0.01\\
599.91	0.01\\
599.92	0.01\\
599.93	0.01\\
599.94	0.01\\
599.95	0.01\\
599.96	0.01\\
599.97	0.01\\
599.98	0.01\\
599.99	0.01\\
600	0.01\\
};
\addplot [color=mycolor9,solid,forget plot]
  table[row sep=crcr]{%
0.01	0.01\\
1.01	0.01\\
2.01	0.01\\
3.01	0.01\\
4.01	0.01\\
5.01	0.01\\
6.01	0.01\\
7.01	0.01\\
8.01	0.01\\
9.01	0.01\\
10.01	0.01\\
11.01	0.01\\
12.01	0.01\\
13.01	0.01\\
14.01	0.01\\
15.01	0.01\\
16.01	0.01\\
17.01	0.01\\
18.01	0.01\\
19.01	0.01\\
20.01	0.01\\
21.01	0.01\\
22.01	0.01\\
23.01	0.01\\
24.01	0.01\\
25.01	0.01\\
26.01	0.01\\
27.01	0.01\\
28.01	0.01\\
29.01	0.01\\
30.01	0.01\\
31.01	0.01\\
32.01	0.01\\
33.01	0.01\\
34.01	0.01\\
35.01	0.01\\
36.01	0.01\\
37.01	0.01\\
38.01	0.01\\
39.01	0.01\\
40.01	0.01\\
41.01	0.01\\
42.01	0.01\\
43.01	0.01\\
44.01	0.01\\
45.01	0.01\\
46.01	0.01\\
47.01	0.01\\
48.01	0.01\\
49.01	0.01\\
50.01	0.01\\
51.01	0.01\\
52.01	0.01\\
53.01	0.01\\
54.01	0.01\\
55.01	0.01\\
56.01	0.01\\
57.01	0.01\\
58.01	0.01\\
59.01	0.01\\
60.01	0.01\\
61.01	0.01\\
62.01	0.01\\
63.01	0.01\\
64.01	0.01\\
65.01	0.01\\
66.01	0.01\\
67.01	0.01\\
68.01	0.01\\
69.01	0.01\\
70.01	0.01\\
71.01	0.01\\
72.01	0.01\\
73.01	0.01\\
74.01	0.01\\
75.01	0.01\\
76.01	0.01\\
77.01	0.01\\
78.01	0.01\\
79.01	0.01\\
80.01	0.01\\
81.01	0.01\\
82.01	0.01\\
83.01	0.01\\
84.01	0.01\\
85.01	0.01\\
86.01	0.01\\
87.01	0.01\\
88.01	0.01\\
89.01	0.01\\
90.01	0.01\\
91.01	0.01\\
92.01	0.01\\
93.01	0.01\\
94.01	0.01\\
95.01	0.01\\
96.01	0.01\\
97.01	0.01\\
98.01	0.01\\
99.01	0.01\\
100.01	0.01\\
101.01	0.01\\
102.01	0.01\\
103.01	0.01\\
104.01	0.01\\
105.01	0.01\\
106.01	0.01\\
107.01	0.01\\
108.01	0.01\\
109.01	0.01\\
110.01	0.01\\
111.01	0.01\\
112.01	0.01\\
113.01	0.01\\
114.01	0.01\\
115.01	0.01\\
116.01	0.01\\
117.01	0.01\\
118.01	0.01\\
119.01	0.01\\
120.01	0.01\\
121.01	0.01\\
122.01	0.01\\
123.01	0.01\\
124.01	0.01\\
125.01	0.01\\
126.01	0.01\\
127.01	0.01\\
128.01	0.01\\
129.01	0.01\\
130.01	0.01\\
131.01	0.01\\
132.01	0.01\\
133.01	0.01\\
134.01	0.01\\
135.01	0.01\\
136.01	0.01\\
137.01	0.01\\
138.01	0.01\\
139.01	0.01\\
140.01	0.01\\
141.01	0.01\\
142.01	0.01\\
143.01	0.01\\
144.01	0.01\\
145.01	0.01\\
146.01	0.01\\
147.01	0.01\\
148.01	0.01\\
149.01	0.01\\
150.01	0.01\\
151.01	0.01\\
152.01	0.01\\
153.01	0.01\\
154.01	0.01\\
155.01	0.01\\
156.01	0.01\\
157.01	0.01\\
158.01	0.01\\
159.01	0.01\\
160.01	0.01\\
161.01	0.01\\
162.01	0.01\\
163.01	0.01\\
164.01	0.01\\
165.01	0.01\\
166.01	0.01\\
167.01	0.01\\
168.01	0.01\\
169.01	0.01\\
170.01	0.01\\
171.01	0.01\\
172.01	0.01\\
173.01	0.01\\
174.01	0.01\\
175.01	0.01\\
176.01	0.01\\
177.01	0.01\\
178.01	0.01\\
179.01	0.01\\
180.01	0.01\\
181.01	0.01\\
182.01	0.01\\
183.01	0.01\\
184.01	0.01\\
185.01	0.01\\
186.01	0.01\\
187.01	0.01\\
188.01	0.01\\
189.01	0.01\\
190.01	0.01\\
191.01	0.01\\
192.01	0.01\\
193.01	0.01\\
194.01	0.01\\
195.01	0.01\\
196.01	0.01\\
197.01	0.01\\
198.01	0.01\\
199.01	0.01\\
200.01	0.01\\
201.01	0.01\\
202.01	0.01\\
203.01	0.01\\
204.01	0.01\\
205.01	0.01\\
206.01	0.01\\
207.01	0.01\\
208.01	0.01\\
209.01	0.01\\
210.01	0.01\\
211.01	0.01\\
212.01	0.01\\
213.01	0.01\\
214.01	0.01\\
215.01	0.01\\
216.01	0.01\\
217.01	0.01\\
218.01	0.01\\
219.01	0.01\\
220.01	0.01\\
221.01	0.01\\
222.01	0.01\\
223.01	0.01\\
224.01	0.01\\
225.01	0.01\\
226.01	0.01\\
227.01	0.01\\
228.01	0.01\\
229.01	0.01\\
230.01	0.01\\
231.01	0.01\\
232.01	0.01\\
233.01	0.01\\
234.01	0.01\\
235.01	0.01\\
236.01	0.01\\
237.01	0.01\\
238.01	0.01\\
239.01	0.01\\
240.01	0.01\\
241.01	0.01\\
242.01	0.01\\
243.01	0.01\\
244.01	0.01\\
245.01	0.01\\
246.01	0.01\\
247.01	0.01\\
248.01	0.01\\
249.01	0.01\\
250.01	0.01\\
251.01	0.01\\
252.01	0.01\\
253.01	0.01\\
254.01	0.01\\
255.01	0.01\\
256.01	0.01\\
257.01	0.01\\
258.01	0.01\\
259.01	0.01\\
260.01	0.01\\
261.01	0.01\\
262.01	0.01\\
263.01	0.01\\
264.01	0.01\\
265.01	0.01\\
266.01	0.01\\
267.01	0.01\\
268.01	0.01\\
269.01	0.01\\
270.01	0.01\\
271.01	0.01\\
272.01	0.01\\
273.01	0.01\\
274.01	0.01\\
275.01	0.01\\
276.01	0.01\\
277.01	0.01\\
278.01	0.01\\
279.01	0.01\\
280.01	0.01\\
281.01	0.01\\
282.01	0.01\\
283.01	0.01\\
284.01	0.01\\
285.01	0.01\\
286.01	0.01\\
287.01	0.01\\
288.01	0.01\\
289.01	0.01\\
290.01	0.01\\
291.01	0.01\\
292.01	0.01\\
293.01	0.01\\
294.01	0.01\\
295.01	0.01\\
296.01	0.01\\
297.01	0.01\\
298.01	0.01\\
299.01	0.01\\
300.01	0.01\\
301.01	0.01\\
302.01	0.01\\
303.01	0.01\\
304.01	0.01\\
305.01	0.01\\
306.01	0.01\\
307.01	0.01\\
308.01	0.01\\
309.01	0.01\\
310.01	0.01\\
311.01	0.01\\
312.01	0.01\\
313.01	0.01\\
314.01	0.01\\
315.01	0.01\\
316.01	0.01\\
317.01	0.01\\
318.01	0.01\\
319.01	0.01\\
320.01	0.01\\
321.01	0.01\\
322.01	0.01\\
323.01	0.01\\
324.01	0.01\\
325.01	0.01\\
326.01	0.01\\
327.01	0.01\\
328.01	0.01\\
329.01	0.01\\
330.01	0.01\\
331.01	0.01\\
332.01	0.01\\
333.01	0.01\\
334.01	0.01\\
335.01	0.01\\
336.01	0.01\\
337.01	0.01\\
338.01	0.01\\
339.01	0.01\\
340.01	0.01\\
341.01	0.01\\
342.01	0.01\\
343.01	0.01\\
344.01	0.01\\
345.01	0.01\\
346.01	0.01\\
347.01	0.01\\
348.01	0.01\\
349.01	0.01\\
350.01	0.01\\
351.01	0.01\\
352.01	0.01\\
353.01	0.01\\
354.01	0.01\\
355.01	0.01\\
356.01	0.01\\
357.01	0.01\\
358.01	0.01\\
359.01	0.01\\
360.01	0.01\\
361.01	0.01\\
362.01	0.01\\
363.01	0.01\\
364.01	0.01\\
365.01	0.01\\
366.01	0.01\\
367.01	0.01\\
368.01	0.01\\
369.01	0.01\\
370.01	0.01\\
371.01	0.01\\
372.01	0.01\\
373.01	0.01\\
374.01	0.01\\
375.01	0.01\\
376.01	0.01\\
377.01	0.01\\
378.01	0.01\\
379.01	0.01\\
380.01	0.01\\
381.01	0.01\\
382.01	0.01\\
383.01	0.01\\
384.01	0.01\\
385.01	0.01\\
386.01	0.01\\
387.01	0.01\\
388.01	0.01\\
389.01	0.01\\
390.01	0.01\\
391.01	0.01\\
392.01	0.01\\
393.01	0.01\\
394.01	0.01\\
395.01	0.01\\
396.01	0.01\\
397.01	0.01\\
398.01	0.01\\
399.01	0.01\\
400.01	0.01\\
401.01	0.01\\
402.01	0.01\\
403.01	0.01\\
404.01	0.01\\
405.01	0.01\\
406.01	0.01\\
407.01	0.01\\
408.01	0.01\\
409.01	0.01\\
410.01	0.01\\
411.01	0.01\\
412.01	0.01\\
413.01	0.01\\
414.01	0.01\\
415.01	0.01\\
416.01	0.01\\
417.01	0.01\\
418.01	0.01\\
419.01	0.01\\
420.01	0.01\\
421.01	0.01\\
422.01	0.01\\
423.01	0.01\\
424.01	0.01\\
425.01	0.01\\
426.01	0.01\\
427.01	0.01\\
428.01	0.01\\
429.01	0.01\\
430.01	0.01\\
431.01	0.01\\
432.01	0.01\\
433.01	0.01\\
434.01	0.01\\
435.01	0.01\\
436.01	0.01\\
437.01	0.01\\
438.01	0.01\\
439.01	0.01\\
440.01	0.01\\
441.01	0.01\\
442.01	0.01\\
443.01	0.01\\
444.01	0.01\\
445.01	0.01\\
446.01	0.01\\
447.01	0.01\\
448.01	0.01\\
449.01	0.01\\
450.01	0.01\\
451.01	0.01\\
452.01	0.01\\
453.01	0.01\\
454.01	0.01\\
455.01	0.01\\
456.01	0.01\\
457.01	0.01\\
458.01	0.01\\
459.01	0.01\\
460.01	0.01\\
461.01	0.01\\
462.01	0.01\\
463.01	0.01\\
464.01	0.01\\
465.01	0.01\\
466.01	0.01\\
467.01	0.01\\
468.01	0.01\\
469.01	0.01\\
470.01	0.01\\
471.01	0.01\\
472.01	0.01\\
473.01	0.01\\
474.01	0.01\\
475.01	0.01\\
476.01	0.01\\
477.01	0.01\\
478.01	0.01\\
479.01	0.01\\
480.01	0.01\\
481.01	0.01\\
482.01	0.01\\
483.01	0.01\\
484.01	0.01\\
485.01	0.01\\
486.01	0.01\\
487.01	0.01\\
488.01	0.01\\
489.01	0.01\\
490.01	0.01\\
491.01	0.01\\
492.01	0.01\\
493.01	0.01\\
494.01	0.01\\
495.01	0.01\\
496.01	0.01\\
497.01	0.01\\
498.01	0.01\\
499.01	0.01\\
500.01	0.01\\
501.01	0.01\\
502.01	0.01\\
503.01	0.01\\
504.01	0.01\\
505.01	0.01\\
506.01	0.01\\
507.01	0.01\\
508.01	0.01\\
509.01	0.01\\
510.01	0.01\\
511.01	0.01\\
512.01	0.01\\
513.01	0.01\\
514.01	0.01\\
515.01	0.01\\
516.01	0.01\\
517.01	0.01\\
518.01	0.01\\
519.01	0.01\\
520.01	0.01\\
521.01	0.01\\
522.01	0.01\\
523.01	0.01\\
524.01	0.01\\
525.01	0.01\\
526.01	0.01\\
527.01	0.01\\
528.01	0.01\\
529.01	0.01\\
530.01	0.01\\
531.01	0.01\\
532.01	0.01\\
533.01	0.01\\
534.01	0.01\\
535.01	0.01\\
536.01	0.01\\
537.01	0.01\\
538.01	0.01\\
539.01	0.01\\
540.01	0.01\\
541.01	0.01\\
542.01	0.01\\
543.01	0.01\\
544.01	0.01\\
545.01	0.01\\
546.01	0.01\\
547.01	0.01\\
548.01	0.01\\
549.01	0.01\\
550.01	0.01\\
551.01	0.01\\
552.01	0.01\\
553.01	0.01\\
554.01	0.01\\
555.01	0.01\\
556.01	0.01\\
557.01	0.01\\
558.01	0.01\\
559.01	0.01\\
560.01	0.01\\
561.01	0.01\\
562.01	0.01\\
563.01	0.01\\
564.01	0.01\\
565.01	0.01\\
566.01	0.01\\
567.01	0.01\\
568.01	0.01\\
569.01	0.01\\
570.01	0.01\\
571.01	0.01\\
572.01	0.01\\
573.01	0.01\\
574.01	0.01\\
575.01	0.01\\
576.01	0.01\\
577.01	0.01\\
578.01	0.01\\
579.01	0.01\\
580.01	0.01\\
581.01	0.01\\
582.01	0.01\\
583.01	0.01\\
584.01	0.01\\
585.01	0.01\\
586.01	0.01\\
587.01	0.01\\
588.01	0.01\\
589.01	0.01\\
590.01	0.01\\
591.01	0.01\\
592.01	0.01\\
593.01	0.01\\
594.01	0.01\\
595.01	0.01\\
596.01	0.01\\
597.01	0.01\\
598.01	0.01\\
599.01	0.01\\
599.02	0.01\\
599.03	0.01\\
599.04	0.01\\
599.05	0.01\\
599.06	0.01\\
599.07	0.01\\
599.08	0.01\\
599.09	0.01\\
599.1	0.01\\
599.11	0.01\\
599.12	0.01\\
599.13	0.01\\
599.14	0.01\\
599.15	0.01\\
599.16	0.01\\
599.17	0.01\\
599.18	0.01\\
599.19	0.01\\
599.2	0.01\\
599.21	0.01\\
599.22	0.01\\
599.23	0.01\\
599.24	0.01\\
599.25	0.01\\
599.26	0.01\\
599.27	0.01\\
599.28	0.01\\
599.29	0.01\\
599.3	0.01\\
599.31	0.01\\
599.32	0.01\\
599.33	0.01\\
599.34	0.01\\
599.35	0.01\\
599.36	0.01\\
599.37	0.01\\
599.38	0.01\\
599.39	0.01\\
599.4	0.01\\
599.41	0.01\\
599.42	0.01\\
599.43	0.01\\
599.44	0.01\\
599.45	0.01\\
599.46	0.01\\
599.47	0.01\\
599.48	0.01\\
599.49	0.01\\
599.5	0.01\\
599.51	0.01\\
599.52	0.01\\
599.53	0.01\\
599.54	0.01\\
599.55	0.01\\
599.56	0.01\\
599.57	0.01\\
599.58	0.01\\
599.59	0.01\\
599.6	0.01\\
599.61	0.01\\
599.62	0.01\\
599.63	0.01\\
599.64	0.01\\
599.65	0.01\\
599.66	0.01\\
599.67	0.01\\
599.68	0.01\\
599.69	0.01\\
599.7	0.01\\
599.71	0.01\\
599.72	0.01\\
599.73	0.01\\
599.74	0.01\\
599.75	0.01\\
599.76	0.01\\
599.77	0.01\\
599.78	0.01\\
599.79	0.01\\
599.8	0.01\\
599.81	0.01\\
599.82	0.01\\
599.83	0.01\\
599.84	0.01\\
599.85	0.01\\
599.86	0.01\\
599.87	0.01\\
599.88	0.01\\
599.89	0.01\\
599.9	0.01\\
599.91	0.01\\
599.92	0.01\\
599.93	0.01\\
599.94	0.01\\
599.95	0.01\\
599.96	0.01\\
599.97	0.01\\
599.98	0.01\\
599.99	0.01\\
600	0.01\\
};
\addplot [color=blue!50!mycolor7,solid,forget plot]
  table[row sep=crcr]{%
0.01	0.01\\
1.01	0.01\\
2.01	0.01\\
3.01	0.01\\
4.01	0.01\\
5.01	0.01\\
6.01	0.01\\
7.01	0.01\\
8.01	0.01\\
9.01	0.01\\
10.01	0.01\\
11.01	0.01\\
12.01	0.01\\
13.01	0.01\\
14.01	0.01\\
15.01	0.01\\
16.01	0.01\\
17.01	0.01\\
18.01	0.01\\
19.01	0.01\\
20.01	0.01\\
21.01	0.01\\
22.01	0.01\\
23.01	0.01\\
24.01	0.01\\
25.01	0.01\\
26.01	0.01\\
27.01	0.01\\
28.01	0.01\\
29.01	0.01\\
30.01	0.01\\
31.01	0.01\\
32.01	0.01\\
33.01	0.01\\
34.01	0.01\\
35.01	0.01\\
36.01	0.01\\
37.01	0.01\\
38.01	0.01\\
39.01	0.01\\
40.01	0.01\\
41.01	0.01\\
42.01	0.01\\
43.01	0.01\\
44.01	0.01\\
45.01	0.01\\
46.01	0.01\\
47.01	0.01\\
48.01	0.01\\
49.01	0.01\\
50.01	0.01\\
51.01	0.01\\
52.01	0.01\\
53.01	0.01\\
54.01	0.01\\
55.01	0.01\\
56.01	0.01\\
57.01	0.01\\
58.01	0.01\\
59.01	0.01\\
60.01	0.01\\
61.01	0.01\\
62.01	0.01\\
63.01	0.01\\
64.01	0.01\\
65.01	0.01\\
66.01	0.01\\
67.01	0.01\\
68.01	0.01\\
69.01	0.01\\
70.01	0.01\\
71.01	0.01\\
72.01	0.01\\
73.01	0.01\\
74.01	0.01\\
75.01	0.01\\
76.01	0.01\\
77.01	0.01\\
78.01	0.01\\
79.01	0.01\\
80.01	0.01\\
81.01	0.01\\
82.01	0.01\\
83.01	0.01\\
84.01	0.01\\
85.01	0.01\\
86.01	0.01\\
87.01	0.01\\
88.01	0.01\\
89.01	0.01\\
90.01	0.01\\
91.01	0.01\\
92.01	0.01\\
93.01	0.01\\
94.01	0.01\\
95.01	0.01\\
96.01	0.01\\
97.01	0.01\\
98.01	0.01\\
99.01	0.01\\
100.01	0.01\\
101.01	0.01\\
102.01	0.01\\
103.01	0.01\\
104.01	0.01\\
105.01	0.01\\
106.01	0.01\\
107.01	0.01\\
108.01	0.01\\
109.01	0.01\\
110.01	0.01\\
111.01	0.01\\
112.01	0.01\\
113.01	0.01\\
114.01	0.01\\
115.01	0.01\\
116.01	0.01\\
117.01	0.01\\
118.01	0.01\\
119.01	0.01\\
120.01	0.01\\
121.01	0.01\\
122.01	0.01\\
123.01	0.01\\
124.01	0.01\\
125.01	0.01\\
126.01	0.01\\
127.01	0.01\\
128.01	0.01\\
129.01	0.01\\
130.01	0.01\\
131.01	0.01\\
132.01	0.01\\
133.01	0.01\\
134.01	0.01\\
135.01	0.01\\
136.01	0.01\\
137.01	0.01\\
138.01	0.01\\
139.01	0.01\\
140.01	0.01\\
141.01	0.01\\
142.01	0.01\\
143.01	0.01\\
144.01	0.01\\
145.01	0.01\\
146.01	0.01\\
147.01	0.01\\
148.01	0.01\\
149.01	0.01\\
150.01	0.01\\
151.01	0.01\\
152.01	0.01\\
153.01	0.01\\
154.01	0.01\\
155.01	0.01\\
156.01	0.01\\
157.01	0.01\\
158.01	0.01\\
159.01	0.01\\
160.01	0.01\\
161.01	0.01\\
162.01	0.01\\
163.01	0.01\\
164.01	0.01\\
165.01	0.01\\
166.01	0.01\\
167.01	0.01\\
168.01	0.01\\
169.01	0.01\\
170.01	0.01\\
171.01	0.01\\
172.01	0.01\\
173.01	0.01\\
174.01	0.01\\
175.01	0.01\\
176.01	0.01\\
177.01	0.01\\
178.01	0.01\\
179.01	0.01\\
180.01	0.01\\
181.01	0.01\\
182.01	0.01\\
183.01	0.01\\
184.01	0.01\\
185.01	0.01\\
186.01	0.01\\
187.01	0.01\\
188.01	0.01\\
189.01	0.01\\
190.01	0.01\\
191.01	0.01\\
192.01	0.01\\
193.01	0.01\\
194.01	0.01\\
195.01	0.01\\
196.01	0.01\\
197.01	0.01\\
198.01	0.01\\
199.01	0.01\\
200.01	0.01\\
201.01	0.01\\
202.01	0.01\\
203.01	0.01\\
204.01	0.01\\
205.01	0.01\\
206.01	0.01\\
207.01	0.01\\
208.01	0.01\\
209.01	0.01\\
210.01	0.01\\
211.01	0.01\\
212.01	0.01\\
213.01	0.01\\
214.01	0.01\\
215.01	0.01\\
216.01	0.01\\
217.01	0.01\\
218.01	0.01\\
219.01	0.01\\
220.01	0.01\\
221.01	0.01\\
222.01	0.01\\
223.01	0.01\\
224.01	0.01\\
225.01	0.01\\
226.01	0.01\\
227.01	0.01\\
228.01	0.01\\
229.01	0.01\\
230.01	0.01\\
231.01	0.01\\
232.01	0.01\\
233.01	0.01\\
234.01	0.01\\
235.01	0.01\\
236.01	0.01\\
237.01	0.01\\
238.01	0.01\\
239.01	0.01\\
240.01	0.01\\
241.01	0.01\\
242.01	0.01\\
243.01	0.01\\
244.01	0.01\\
245.01	0.01\\
246.01	0.01\\
247.01	0.01\\
248.01	0.01\\
249.01	0.01\\
250.01	0.01\\
251.01	0.01\\
252.01	0.01\\
253.01	0.01\\
254.01	0.01\\
255.01	0.01\\
256.01	0.01\\
257.01	0.01\\
258.01	0.01\\
259.01	0.01\\
260.01	0.01\\
261.01	0.01\\
262.01	0.01\\
263.01	0.01\\
264.01	0.01\\
265.01	0.01\\
266.01	0.01\\
267.01	0.01\\
268.01	0.01\\
269.01	0.01\\
270.01	0.01\\
271.01	0.01\\
272.01	0.01\\
273.01	0.01\\
274.01	0.01\\
275.01	0.01\\
276.01	0.01\\
277.01	0.01\\
278.01	0.01\\
279.01	0.01\\
280.01	0.01\\
281.01	0.01\\
282.01	0.01\\
283.01	0.01\\
284.01	0.01\\
285.01	0.01\\
286.01	0.01\\
287.01	0.01\\
288.01	0.01\\
289.01	0.01\\
290.01	0.01\\
291.01	0.01\\
292.01	0.01\\
293.01	0.01\\
294.01	0.01\\
295.01	0.01\\
296.01	0.01\\
297.01	0.01\\
298.01	0.01\\
299.01	0.01\\
300.01	0.01\\
301.01	0.01\\
302.01	0.01\\
303.01	0.01\\
304.01	0.01\\
305.01	0.01\\
306.01	0.01\\
307.01	0.01\\
308.01	0.01\\
309.01	0.01\\
310.01	0.01\\
311.01	0.01\\
312.01	0.01\\
313.01	0.01\\
314.01	0.01\\
315.01	0.01\\
316.01	0.01\\
317.01	0.01\\
318.01	0.01\\
319.01	0.01\\
320.01	0.01\\
321.01	0.01\\
322.01	0.01\\
323.01	0.01\\
324.01	0.01\\
325.01	0.01\\
326.01	0.01\\
327.01	0.01\\
328.01	0.01\\
329.01	0.01\\
330.01	0.01\\
331.01	0.01\\
332.01	0.01\\
333.01	0.01\\
334.01	0.01\\
335.01	0.01\\
336.01	0.01\\
337.01	0.01\\
338.01	0.01\\
339.01	0.01\\
340.01	0.01\\
341.01	0.01\\
342.01	0.01\\
343.01	0.01\\
344.01	0.01\\
345.01	0.01\\
346.01	0.01\\
347.01	0.01\\
348.01	0.01\\
349.01	0.01\\
350.01	0.01\\
351.01	0.01\\
352.01	0.01\\
353.01	0.01\\
354.01	0.01\\
355.01	0.01\\
356.01	0.01\\
357.01	0.01\\
358.01	0.01\\
359.01	0.01\\
360.01	0.01\\
361.01	0.01\\
362.01	0.01\\
363.01	0.01\\
364.01	0.01\\
365.01	0.01\\
366.01	0.01\\
367.01	0.01\\
368.01	0.01\\
369.01	0.01\\
370.01	0.01\\
371.01	0.01\\
372.01	0.01\\
373.01	0.01\\
374.01	0.01\\
375.01	0.01\\
376.01	0.01\\
377.01	0.01\\
378.01	0.01\\
379.01	0.01\\
380.01	0.01\\
381.01	0.01\\
382.01	0.01\\
383.01	0.01\\
384.01	0.01\\
385.01	0.01\\
386.01	0.01\\
387.01	0.01\\
388.01	0.01\\
389.01	0.01\\
390.01	0.01\\
391.01	0.01\\
392.01	0.01\\
393.01	0.01\\
394.01	0.01\\
395.01	0.01\\
396.01	0.01\\
397.01	0.01\\
398.01	0.01\\
399.01	0.01\\
400.01	0.01\\
401.01	0.01\\
402.01	0.01\\
403.01	0.01\\
404.01	0.01\\
405.01	0.01\\
406.01	0.01\\
407.01	0.01\\
408.01	0.01\\
409.01	0.01\\
410.01	0.01\\
411.01	0.01\\
412.01	0.01\\
413.01	0.01\\
414.01	0.01\\
415.01	0.01\\
416.01	0.01\\
417.01	0.01\\
418.01	0.01\\
419.01	0.01\\
420.01	0.01\\
421.01	0.01\\
422.01	0.01\\
423.01	0.01\\
424.01	0.01\\
425.01	0.01\\
426.01	0.01\\
427.01	0.01\\
428.01	0.01\\
429.01	0.01\\
430.01	0.01\\
431.01	0.01\\
432.01	0.01\\
433.01	0.01\\
434.01	0.01\\
435.01	0.01\\
436.01	0.01\\
437.01	0.01\\
438.01	0.01\\
439.01	0.01\\
440.01	0.01\\
441.01	0.01\\
442.01	0.01\\
443.01	0.01\\
444.01	0.01\\
445.01	0.01\\
446.01	0.01\\
447.01	0.01\\
448.01	0.01\\
449.01	0.01\\
450.01	0.01\\
451.01	0.01\\
452.01	0.01\\
453.01	0.01\\
454.01	0.01\\
455.01	0.01\\
456.01	0.01\\
457.01	0.01\\
458.01	0.01\\
459.01	0.01\\
460.01	0.01\\
461.01	0.01\\
462.01	0.01\\
463.01	0.01\\
464.01	0.01\\
465.01	0.01\\
466.01	0.01\\
467.01	0.01\\
468.01	0.01\\
469.01	0.01\\
470.01	0.01\\
471.01	0.01\\
472.01	0.01\\
473.01	0.01\\
474.01	0.01\\
475.01	0.01\\
476.01	0.01\\
477.01	0.01\\
478.01	0.01\\
479.01	0.01\\
480.01	0.01\\
481.01	0.01\\
482.01	0.01\\
483.01	0.01\\
484.01	0.01\\
485.01	0.01\\
486.01	0.01\\
487.01	0.01\\
488.01	0.01\\
489.01	0.01\\
490.01	0.01\\
491.01	0.01\\
492.01	0.01\\
493.01	0.01\\
494.01	0.01\\
495.01	0.01\\
496.01	0.01\\
497.01	0.01\\
498.01	0.01\\
499.01	0.01\\
500.01	0.01\\
501.01	0.01\\
502.01	0.01\\
503.01	0.01\\
504.01	0.01\\
505.01	0.01\\
506.01	0.01\\
507.01	0.01\\
508.01	0.01\\
509.01	0.01\\
510.01	0.01\\
511.01	0.01\\
512.01	0.01\\
513.01	0.01\\
514.01	0.01\\
515.01	0.01\\
516.01	0.01\\
517.01	0.01\\
518.01	0.01\\
519.01	0.01\\
520.01	0.01\\
521.01	0.01\\
522.01	0.01\\
523.01	0.01\\
524.01	0.01\\
525.01	0.01\\
526.01	0.01\\
527.01	0.01\\
528.01	0.01\\
529.01	0.01\\
530.01	0.01\\
531.01	0.01\\
532.01	0.01\\
533.01	0.01\\
534.01	0.01\\
535.01	0.01\\
536.01	0.01\\
537.01	0.01\\
538.01	0.01\\
539.01	0.01\\
540.01	0.01\\
541.01	0.01\\
542.01	0.01\\
543.01	0.01\\
544.01	0.01\\
545.01	0.01\\
546.01	0.01\\
547.01	0.01\\
548.01	0.01\\
549.01	0.01\\
550.01	0.01\\
551.01	0.01\\
552.01	0.01\\
553.01	0.01\\
554.01	0.01\\
555.01	0.01\\
556.01	0.01\\
557.01	0.01\\
558.01	0.01\\
559.01	0.01\\
560.01	0.01\\
561.01	0.01\\
562.01	0.01\\
563.01	0.01\\
564.01	0.01\\
565.01	0.01\\
566.01	0.01\\
567.01	0.01\\
568.01	0.01\\
569.01	0.01\\
570.01	0.01\\
571.01	0.01\\
572.01	0.01\\
573.01	0.01\\
574.01	0.01\\
575.01	0.01\\
576.01	0.01\\
577.01	0.01\\
578.01	0.01\\
579.01	0.01\\
580.01	0.01\\
581.01	0.01\\
582.01	0.01\\
583.01	0.01\\
584.01	0.01\\
585.01	0.01\\
586.01	0.01\\
587.01	0.01\\
588.01	0.01\\
589.01	0.01\\
590.01	0.01\\
591.01	0.01\\
592.01	0.01\\
593.01	0.01\\
594.01	0.01\\
595.01	0.01\\
596.01	0.01\\
597.01	0.01\\
598.01	0.01\\
599.01	0.01\\
599.02	0.01\\
599.03	0.01\\
599.04	0.01\\
599.05	0.01\\
599.06	0.01\\
599.07	0.01\\
599.08	0.01\\
599.09	0.01\\
599.1	0.01\\
599.11	0.01\\
599.12	0.01\\
599.13	0.01\\
599.14	0.01\\
599.15	0.01\\
599.16	0.01\\
599.17	0.01\\
599.18	0.01\\
599.19	0.01\\
599.2	0.01\\
599.21	0.01\\
599.22	0.01\\
599.23	0.01\\
599.24	0.01\\
599.25	0.01\\
599.26	0.01\\
599.27	0.01\\
599.28	0.01\\
599.29	0.01\\
599.3	0.01\\
599.31	0.01\\
599.32	0.01\\
599.33	0.01\\
599.34	0.01\\
599.35	0.01\\
599.36	0.01\\
599.37	0.01\\
599.38	0.01\\
599.39	0.01\\
599.4	0.01\\
599.41	0.01\\
599.42	0.01\\
599.43	0.01\\
599.44	0.01\\
599.45	0.01\\
599.46	0.01\\
599.47	0.01\\
599.48	0.01\\
599.49	0.01\\
599.5	0.01\\
599.51	0.01\\
599.52	0.01\\
599.53	0.01\\
599.54	0.01\\
599.55	0.01\\
599.56	0.01\\
599.57	0.01\\
599.58	0.01\\
599.59	0.01\\
599.6	0.01\\
599.61	0.01\\
599.62	0.01\\
599.63	0.01\\
599.64	0.01\\
599.65	0.01\\
599.66	0.01\\
599.67	0.01\\
599.68	0.01\\
599.69	0.01\\
599.7	0.01\\
599.71	0.01\\
599.72	0.01\\
599.73	0.01\\
599.74	0.01\\
599.75	0.01\\
599.76	0.01\\
599.77	0.01\\
599.78	0.01\\
599.79	0.01\\
599.8	0.01\\
599.81	0.01\\
599.82	0.01\\
599.83	0.01\\
599.84	0.01\\
599.85	0.01\\
599.86	0.01\\
599.87	0.01\\
599.88	0.01\\
599.89	0.01\\
599.9	0.01\\
599.91	0.01\\
599.92	0.01\\
599.93	0.01\\
599.94	0.01\\
599.95	0.01\\
599.96	0.01\\
599.97	0.01\\
599.98	0.01\\
599.99	0.01\\
600	0.01\\
};
\addplot [color=blue!40!mycolor9,solid,forget plot]
  table[row sep=crcr]{%
0.01	0.01\\
1.01	0.01\\
2.01	0.01\\
3.01	0.01\\
4.01	0.01\\
5.01	0.01\\
6.01	0.01\\
7.01	0.01\\
8.01	0.01\\
9.01	0.01\\
10.01	0.01\\
11.01	0.01\\
12.01	0.01\\
13.01	0.01\\
14.01	0.01\\
15.01	0.01\\
16.01	0.01\\
17.01	0.01\\
18.01	0.01\\
19.01	0.01\\
20.01	0.01\\
21.01	0.01\\
22.01	0.01\\
23.01	0.01\\
24.01	0.01\\
25.01	0.01\\
26.01	0.01\\
27.01	0.01\\
28.01	0.01\\
29.01	0.01\\
30.01	0.01\\
31.01	0.01\\
32.01	0.01\\
33.01	0.01\\
34.01	0.01\\
35.01	0.01\\
36.01	0.01\\
37.01	0.01\\
38.01	0.01\\
39.01	0.01\\
40.01	0.01\\
41.01	0.01\\
42.01	0.01\\
43.01	0.01\\
44.01	0.01\\
45.01	0.01\\
46.01	0.01\\
47.01	0.01\\
48.01	0.01\\
49.01	0.01\\
50.01	0.01\\
51.01	0.01\\
52.01	0.01\\
53.01	0.01\\
54.01	0.01\\
55.01	0.01\\
56.01	0.01\\
57.01	0.01\\
58.01	0.01\\
59.01	0.01\\
60.01	0.01\\
61.01	0.01\\
62.01	0.01\\
63.01	0.01\\
64.01	0.01\\
65.01	0.01\\
66.01	0.01\\
67.01	0.01\\
68.01	0.01\\
69.01	0.01\\
70.01	0.01\\
71.01	0.01\\
72.01	0.01\\
73.01	0.01\\
74.01	0.01\\
75.01	0.01\\
76.01	0.01\\
77.01	0.01\\
78.01	0.01\\
79.01	0.01\\
80.01	0.01\\
81.01	0.01\\
82.01	0.01\\
83.01	0.01\\
84.01	0.01\\
85.01	0.01\\
86.01	0.01\\
87.01	0.01\\
88.01	0.01\\
89.01	0.01\\
90.01	0.01\\
91.01	0.01\\
92.01	0.01\\
93.01	0.01\\
94.01	0.01\\
95.01	0.01\\
96.01	0.01\\
97.01	0.01\\
98.01	0.01\\
99.01	0.01\\
100.01	0.01\\
101.01	0.01\\
102.01	0.01\\
103.01	0.01\\
104.01	0.01\\
105.01	0.01\\
106.01	0.01\\
107.01	0.01\\
108.01	0.01\\
109.01	0.01\\
110.01	0.01\\
111.01	0.01\\
112.01	0.01\\
113.01	0.01\\
114.01	0.01\\
115.01	0.01\\
116.01	0.01\\
117.01	0.01\\
118.01	0.01\\
119.01	0.01\\
120.01	0.01\\
121.01	0.01\\
122.01	0.01\\
123.01	0.01\\
124.01	0.01\\
125.01	0.01\\
126.01	0.01\\
127.01	0.01\\
128.01	0.01\\
129.01	0.01\\
130.01	0.01\\
131.01	0.01\\
132.01	0.01\\
133.01	0.01\\
134.01	0.01\\
135.01	0.01\\
136.01	0.01\\
137.01	0.01\\
138.01	0.01\\
139.01	0.01\\
140.01	0.01\\
141.01	0.01\\
142.01	0.01\\
143.01	0.01\\
144.01	0.01\\
145.01	0.01\\
146.01	0.01\\
147.01	0.01\\
148.01	0.01\\
149.01	0.01\\
150.01	0.01\\
151.01	0.01\\
152.01	0.01\\
153.01	0.01\\
154.01	0.01\\
155.01	0.01\\
156.01	0.01\\
157.01	0.01\\
158.01	0.01\\
159.01	0.01\\
160.01	0.01\\
161.01	0.01\\
162.01	0.01\\
163.01	0.01\\
164.01	0.01\\
165.01	0.01\\
166.01	0.01\\
167.01	0.01\\
168.01	0.01\\
169.01	0.01\\
170.01	0.01\\
171.01	0.01\\
172.01	0.01\\
173.01	0.01\\
174.01	0.01\\
175.01	0.01\\
176.01	0.01\\
177.01	0.01\\
178.01	0.01\\
179.01	0.01\\
180.01	0.01\\
181.01	0.01\\
182.01	0.01\\
183.01	0.01\\
184.01	0.01\\
185.01	0.01\\
186.01	0.01\\
187.01	0.01\\
188.01	0.01\\
189.01	0.01\\
190.01	0.01\\
191.01	0.01\\
192.01	0.01\\
193.01	0.01\\
194.01	0.01\\
195.01	0.01\\
196.01	0.01\\
197.01	0.01\\
198.01	0.01\\
199.01	0.01\\
200.01	0.01\\
201.01	0.01\\
202.01	0.01\\
203.01	0.01\\
204.01	0.01\\
205.01	0.01\\
206.01	0.01\\
207.01	0.01\\
208.01	0.01\\
209.01	0.01\\
210.01	0.01\\
211.01	0.01\\
212.01	0.01\\
213.01	0.01\\
214.01	0.01\\
215.01	0.01\\
216.01	0.01\\
217.01	0.01\\
218.01	0.01\\
219.01	0.01\\
220.01	0.01\\
221.01	0.01\\
222.01	0.01\\
223.01	0.01\\
224.01	0.01\\
225.01	0.01\\
226.01	0.01\\
227.01	0.01\\
228.01	0.01\\
229.01	0.01\\
230.01	0.01\\
231.01	0.01\\
232.01	0.01\\
233.01	0.01\\
234.01	0.01\\
235.01	0.01\\
236.01	0.01\\
237.01	0.01\\
238.01	0.01\\
239.01	0.01\\
240.01	0.01\\
241.01	0.01\\
242.01	0.01\\
243.01	0.01\\
244.01	0.01\\
245.01	0.01\\
246.01	0.01\\
247.01	0.01\\
248.01	0.01\\
249.01	0.01\\
250.01	0.01\\
251.01	0.01\\
252.01	0.01\\
253.01	0.01\\
254.01	0.01\\
255.01	0.01\\
256.01	0.01\\
257.01	0.01\\
258.01	0.01\\
259.01	0.01\\
260.01	0.01\\
261.01	0.01\\
262.01	0.01\\
263.01	0.01\\
264.01	0.01\\
265.01	0.01\\
266.01	0.01\\
267.01	0.01\\
268.01	0.01\\
269.01	0.01\\
270.01	0.01\\
271.01	0.01\\
272.01	0.01\\
273.01	0.01\\
274.01	0.01\\
275.01	0.01\\
276.01	0.01\\
277.01	0.01\\
278.01	0.01\\
279.01	0.01\\
280.01	0.01\\
281.01	0.01\\
282.01	0.01\\
283.01	0.01\\
284.01	0.01\\
285.01	0.01\\
286.01	0.01\\
287.01	0.01\\
288.01	0.01\\
289.01	0.01\\
290.01	0.01\\
291.01	0.01\\
292.01	0.01\\
293.01	0.01\\
294.01	0.01\\
295.01	0.01\\
296.01	0.01\\
297.01	0.01\\
298.01	0.01\\
299.01	0.01\\
300.01	0.01\\
301.01	0.01\\
302.01	0.01\\
303.01	0.01\\
304.01	0.01\\
305.01	0.01\\
306.01	0.01\\
307.01	0.01\\
308.01	0.01\\
309.01	0.01\\
310.01	0.01\\
311.01	0.01\\
312.01	0.01\\
313.01	0.01\\
314.01	0.01\\
315.01	0.01\\
316.01	0.01\\
317.01	0.01\\
318.01	0.01\\
319.01	0.01\\
320.01	0.01\\
321.01	0.01\\
322.01	0.01\\
323.01	0.01\\
324.01	0.01\\
325.01	0.01\\
326.01	0.01\\
327.01	0.01\\
328.01	0.01\\
329.01	0.01\\
330.01	0.01\\
331.01	0.01\\
332.01	0.01\\
333.01	0.01\\
334.01	0.01\\
335.01	0.01\\
336.01	0.01\\
337.01	0.01\\
338.01	0.01\\
339.01	0.01\\
340.01	0.01\\
341.01	0.01\\
342.01	0.01\\
343.01	0.01\\
344.01	0.01\\
345.01	0.01\\
346.01	0.01\\
347.01	0.01\\
348.01	0.01\\
349.01	0.01\\
350.01	0.01\\
351.01	0.01\\
352.01	0.01\\
353.01	0.01\\
354.01	0.01\\
355.01	0.01\\
356.01	0.01\\
357.01	0.01\\
358.01	0.01\\
359.01	0.01\\
360.01	0.01\\
361.01	0.01\\
362.01	0.01\\
363.01	0.01\\
364.01	0.01\\
365.01	0.01\\
366.01	0.01\\
367.01	0.01\\
368.01	0.01\\
369.01	0.01\\
370.01	0.01\\
371.01	0.01\\
372.01	0.01\\
373.01	0.01\\
374.01	0.01\\
375.01	0.01\\
376.01	0.01\\
377.01	0.01\\
378.01	0.01\\
379.01	0.01\\
380.01	0.01\\
381.01	0.01\\
382.01	0.01\\
383.01	0.01\\
384.01	0.01\\
385.01	0.01\\
386.01	0.01\\
387.01	0.01\\
388.01	0.01\\
389.01	0.01\\
390.01	0.01\\
391.01	0.01\\
392.01	0.01\\
393.01	0.01\\
394.01	0.01\\
395.01	0.01\\
396.01	0.01\\
397.01	0.01\\
398.01	0.01\\
399.01	0.01\\
400.01	0.01\\
401.01	0.01\\
402.01	0.01\\
403.01	0.01\\
404.01	0.01\\
405.01	0.01\\
406.01	0.01\\
407.01	0.01\\
408.01	0.01\\
409.01	0.01\\
410.01	0.01\\
411.01	0.01\\
412.01	0.01\\
413.01	0.01\\
414.01	0.01\\
415.01	0.01\\
416.01	0.01\\
417.01	0.01\\
418.01	0.01\\
419.01	0.01\\
420.01	0.01\\
421.01	0.01\\
422.01	0.01\\
423.01	0.01\\
424.01	0.01\\
425.01	0.01\\
426.01	0.01\\
427.01	0.01\\
428.01	0.01\\
429.01	0.01\\
430.01	0.01\\
431.01	0.01\\
432.01	0.01\\
433.01	0.01\\
434.01	0.01\\
435.01	0.01\\
436.01	0.01\\
437.01	0.01\\
438.01	0.01\\
439.01	0.01\\
440.01	0.01\\
441.01	0.01\\
442.01	0.01\\
443.01	0.01\\
444.01	0.01\\
445.01	0.01\\
446.01	0.01\\
447.01	0.01\\
448.01	0.01\\
449.01	0.01\\
450.01	0.01\\
451.01	0.01\\
452.01	0.01\\
453.01	0.01\\
454.01	0.01\\
455.01	0.01\\
456.01	0.01\\
457.01	0.01\\
458.01	0.01\\
459.01	0.01\\
460.01	0.01\\
461.01	0.01\\
462.01	0.01\\
463.01	0.01\\
464.01	0.01\\
465.01	0.01\\
466.01	0.01\\
467.01	0.01\\
468.01	0.01\\
469.01	0.01\\
470.01	0.01\\
471.01	0.01\\
472.01	0.01\\
473.01	0.01\\
474.01	0.01\\
475.01	0.01\\
476.01	0.01\\
477.01	0.01\\
478.01	0.01\\
479.01	0.01\\
480.01	0.01\\
481.01	0.01\\
482.01	0.01\\
483.01	0.01\\
484.01	0.01\\
485.01	0.01\\
486.01	0.01\\
487.01	0.01\\
488.01	0.01\\
489.01	0.01\\
490.01	0.01\\
491.01	0.01\\
492.01	0.01\\
493.01	0.01\\
494.01	0.01\\
495.01	0.01\\
496.01	0.01\\
497.01	0.01\\
498.01	0.01\\
499.01	0.01\\
500.01	0.01\\
501.01	0.01\\
502.01	0.01\\
503.01	0.01\\
504.01	0.01\\
505.01	0.01\\
506.01	0.01\\
507.01	0.01\\
508.01	0.01\\
509.01	0.01\\
510.01	0.01\\
511.01	0.01\\
512.01	0.01\\
513.01	0.01\\
514.01	0.01\\
515.01	0.01\\
516.01	0.01\\
517.01	0.01\\
518.01	0.01\\
519.01	0.01\\
520.01	0.01\\
521.01	0.01\\
522.01	0.01\\
523.01	0.01\\
524.01	0.01\\
525.01	0.01\\
526.01	0.01\\
527.01	0.01\\
528.01	0.01\\
529.01	0.01\\
530.01	0.01\\
531.01	0.01\\
532.01	0.01\\
533.01	0.01\\
534.01	0.01\\
535.01	0.01\\
536.01	0.01\\
537.01	0.01\\
538.01	0.01\\
539.01	0.01\\
540.01	0.01\\
541.01	0.01\\
542.01	0.01\\
543.01	0.01\\
544.01	0.01\\
545.01	0.01\\
546.01	0.01\\
547.01	0.01\\
548.01	0.01\\
549.01	0.01\\
550.01	0.01\\
551.01	0.01\\
552.01	0.01\\
553.01	0.01\\
554.01	0.01\\
555.01	0.01\\
556.01	0.01\\
557.01	0.01\\
558.01	0.01\\
559.01	0.01\\
560.01	0.01\\
561.01	0.01\\
562.01	0.01\\
563.01	0.01\\
564.01	0.01\\
565.01	0.01\\
566.01	0.01\\
567.01	0.01\\
568.01	0.01\\
569.01	0.01\\
570.01	0.01\\
571.01	0.01\\
572.01	0.01\\
573.01	0.01\\
574.01	0.01\\
575.01	0.01\\
576.01	0.01\\
577.01	0.01\\
578.01	0.01\\
579.01	0.01\\
580.01	0.01\\
581.01	0.01\\
582.01	0.01\\
583.01	0.01\\
584.01	0.01\\
585.01	0.01\\
586.01	0.01\\
587.01	0.01\\
588.01	0.01\\
589.01	0.01\\
590.01	0.01\\
591.01	0.01\\
592.01	0.01\\
593.01	0.01\\
594.01	0.01\\
595.01	0.01\\
596.01	0.01\\
597.01	0.01\\
598.01	0.01\\
599.01	0.01\\
599.02	0.01\\
599.03	0.01\\
599.04	0.01\\
599.05	0.01\\
599.06	0.01\\
599.07	0.01\\
599.08	0.01\\
599.09	0.01\\
599.1	0.01\\
599.11	0.01\\
599.12	0.01\\
599.13	0.01\\
599.14	0.01\\
599.15	0.01\\
599.16	0.01\\
599.17	0.01\\
599.18	0.01\\
599.19	0.01\\
599.2	0.01\\
599.21	0.01\\
599.22	0.01\\
599.23	0.01\\
599.24	0.01\\
599.25	0.01\\
599.26	0.01\\
599.27	0.01\\
599.28	0.01\\
599.29	0.01\\
599.3	0.01\\
599.31	0.01\\
599.32	0.01\\
599.33	0.01\\
599.34	0.01\\
599.35	0.01\\
599.36	0.01\\
599.37	0.01\\
599.38	0.01\\
599.39	0.01\\
599.4	0.01\\
599.41	0.01\\
599.42	0.01\\
599.43	0.01\\
599.44	0.01\\
599.45	0.01\\
599.46	0.01\\
599.47	0.01\\
599.48	0.01\\
599.49	0.01\\
599.5	0.01\\
599.51	0.01\\
599.52	0.01\\
599.53	0.01\\
599.54	0.01\\
599.55	0.01\\
599.56	0.01\\
599.57	0.01\\
599.58	0.01\\
599.59	0.01\\
599.6	0.01\\
599.61	0.01\\
599.62	0.01\\
599.63	0.01\\
599.64	0.01\\
599.65	0.01\\
599.66	0.01\\
599.67	0.01\\
599.68	0.01\\
599.69	0.01\\
599.7	0.01\\
599.71	0.01\\
599.72	0.01\\
599.73	0.01\\
599.74	0.01\\
599.75	0.01\\
599.76	0.01\\
599.77	0.01\\
599.78	0.01\\
599.79	0.01\\
599.8	0.01\\
599.81	0.01\\
599.82	0.01\\
599.83	0.01\\
599.84	0.01\\
599.85	0.01\\
599.86	0.01\\
599.87	0.01\\
599.88	0.01\\
599.89	0.01\\
599.9	0.01\\
599.91	0.01\\
599.92	0.01\\
599.93	0.01\\
599.94	0.01\\
599.95	0.01\\
599.96	0.01\\
599.97	0.01\\
599.98	0.01\\
599.99	0.01\\
600	0.01\\
};
\addplot [color=blue!75!mycolor7,solid,forget plot]
  table[row sep=crcr]{%
0.01	0.00980832450319652\\
1.01	0.0098083246691072\\
2.01	0.00980832483848332\\
3.01	0.00980832501139764\\
4.01	0.00980832518792445\\
5.01	0.0098083253681396\\
6.01	0.00980832555212057\\
7.01	0.00980832573994645\\
8.01	0.00980832593169802\\
9.01	0.00980832612745776\\
10.01	0.00980832632730991\\
11.01	0.00980832653134047\\
12.01	0.00980832673963727\\
13.01	0.00980832695229001\\
14.01	0.00980832716939028\\
15.01	0.0098083273910316\\
16.01	0.00980832761730947\\
17.01	0.00980832784832146\\
18.01	0.00980832808416714\\
19.01	0.00980832832494821\\
20.01	0.00980832857076858\\
21.01	0.00980832882173431\\
22.01	0.00980832907795365\\
23.01	0.00980832933953731\\
24.01	0.00980832960659818\\
25.01	0.00980832987925165\\
26.01	0.00980833015761553\\
27.01	0.00980833044181012\\
28.01	0.00980833073195828\\
29.01	0.00980833102818547\\
30.01	0.00980833133061986\\
31.01	0.00980833163939228\\
32.01	0.00980833195463639\\
33.01	0.00980833227648865\\
34.01	0.00980833260508845\\
35.01	0.00980833294057813\\
36.01	0.00980833328310306\\
37.01	0.00980833363281172\\
38.01	0.00980833398985571\\
39.01	0.0098083343543899\\
40.01	0.00980833472657244\\
41.01	0.00980833510656483\\
42.01	0.00980833549453204\\
43.01	0.00980833589064253\\
44.01	0.00980833629506836\\
45.01	0.00980833670798528\\
46.01	0.00980833712957271\\
47.01	0.00980833756001398\\
48.01	0.0098083379994963\\
49.01	0.00980833844821089\\
50.01	0.00980833890635301\\
51.01	0.00980833937412208\\
52.01	0.00980833985172188\\
53.01	0.0098083403393604\\
54.01	0.0098083408372502\\
55.01	0.0098083413456083\\
56.01	0.00980834186465638\\
57.01	0.00980834239462087\\
58.01	0.00980834293573302\\
59.01	0.00980834348822905\\
60.01	0.00980834405235023\\
61.01	0.00980834462834299\\
62.01	0.009808345216459\\
63.01	0.00980834581695539\\
64.01	0.00980834643009474\\
65.01	0.00980834705614526\\
66.01	0.00980834769538095\\
67.01	0.00980834834808164\\
68.01	0.00980834901453318\\
69.01	0.00980834969502753\\
70.01	0.00980835038986295\\
71.01	0.00980835109934406\\
72.01	0.00980835182378205\\
73.01	0.00980835256349478\\
74.01	0.00980835331880693\\
75.01	0.00980835409005017\\
76.01	0.0098083548775633\\
77.01	0.00980835568169239\\
78.01	0.00980835650279094\\
79.01	0.00980835734122012\\
80.01	0.00980835819734878\\
81.01	0.00980835907155379\\
82.01	0.00980835996422004\\
83.01	0.00980836087574083\\
84.01	0.00980836180651786\\
85.01	0.00980836275696146\\
86.01	0.00980836372749088\\
87.01	0.00980836471853437\\
88.01	0.00980836573052946\\
89.01	0.00980836676392303\\
90.01	0.00980836781917171\\
91.01	0.00980836889674193\\
92.01	0.0098083699971102\\
93.01	0.00980837112076337\\
94.01	0.00980837226819877\\
95.01	0.00980837343992449\\
96.01	0.00980837463645962\\
97.01	0.0098083758583345\\
98.01	0.00980837710609093\\
99.01	0.00980837838028242\\
100.01	0.00980837968147452\\
101.01	0.009808381010245\\
102.01	0.00980838236718419\\
103.01	0.00980838375289516\\
104.01	0.00980838516799412\\
105.01	0.00980838661311064\\
106.01	0.00980838808888794\\
107.01	0.00980838959598325\\
108.01	0.00980839113506801\\
109.01	0.00980839270682833\\
110.01	0.00980839431196517\\
111.01	0.00980839595119481\\
112.01	0.00980839762524905\\
113.01	0.00980839933487567\\
114.01	0.00980840108083868\\
115.01	0.0098084028639188\\
116.01	0.00980840468491372\\
117.01	0.00980840654463853\\
118.01	0.0098084084439261\\
119.01	0.0098084103836275\\
120.01	0.00980841236461236\\
121.01	0.00980841438776927\\
122.01	0.00980841645400629\\
123.01	0.0098084185642513\\
124.01	0.00980842071945245\\
125.01	0.00980842292057869\\
126.01	0.00980842516862014\\
127.01	0.00980842746458859\\
128.01	0.00980842980951798\\
129.01	0.00980843220446498\\
130.01	0.00980843465050935\\
131.01	0.00980843714875457\\
132.01	0.00980843970032834\\
133.01	0.00980844230638307\\
134.01	0.00980844496809657\\
135.01	0.00980844768667243\\
136.01	0.00980845046334078\\
137.01	0.00980845329935875\\
138.01	0.00980845619601118\\
139.01	0.00980845915461115\\
140.01	0.00980846217650069\\
141.01	0.0098084652630514\\
142.01	0.00980846841566509\\
143.01	0.00980847163577453\\
144.01	0.0098084749248441\\
145.01	0.0098084782843705\\
146.01	0.0098084817158835\\
147.01	0.00980848522094664\\
148.01	0.00980848880115809\\
149.01	0.00980849245815135\\
150.01	0.00980849619359605\\
151.01	0.00980850000919885\\
152.01	0.00980850390670418\\
153.01	0.00980850788789515\\
154.01	0.00980851195459441\\
155.01	0.00980851610866507\\
156.01	0.00980852035201162\\
157.01	0.00980852468658077\\
158.01	0.00980852911436264\\
159.01	0.00980853363739147\\
160.01	0.00980853825774685\\
161.01	0.00980854297755463\\
162.01	0.00980854779898799\\
163.01	0.00980855272426859\\
164.01	0.00980855775566759\\
165.01	0.00980856289550684\\
166.01	0.00980856814616001\\
167.01	0.00980857351005378\\
168.01	0.0098085789896691\\
169.01	0.00980858458754231\\
170.01	0.00980859030626658\\
171.01	0.00980859614849313\\
172.01	0.00980860211693249\\
173.01	0.009808608214356\\
174.01	0.00980861444359713\\
175.01	0.00980862080755291\\
176.01	0.00980862730918543\\
177.01	0.00980863395152332\\
178.01	0.00980864073766329\\
179.01	0.00980864767077166\\
180.01	0.00980865475408604\\
181.01	0.00980866199091696\\
182.01	0.00980866938464949\\
183.01	0.00980867693874505\\
184.01	0.00980868465674313\\
185.01	0.00980869254226314\\
186.01	0.00980870059900621\\
187.01	0.00980870883075711\\
188.01	0.00980871724138624\\
189.01	0.00980872583485153\\
190.01	0.00980873461520058\\
191.01	0.00980874358657264\\
192.01	0.00980875275320085\\
193.01	0.00980876211941433\\
194.01	0.0098087716896405\\
195.01	0.00980878146840735\\
196.01	0.00980879146034575\\
197.01	0.0098088016701919\\
198.01	0.00980881210278984\\
199.01	0.0098088227630938\\
200.01	0.00980883365617103\\
201.01	0.00980884478720425\\
202.01	0.00980885616149441\\
203.01	0.00980886778446354\\
204.01	0.00980887966165752\\
205.01	0.009808891798749\\
206.01	0.00980890420154039\\
207.01	0.00980891687596697\\
208.01	0.00980892982809992\\
209.01	0.00980894306414961\\
210.01	0.00980895659046883\\
211.01	0.00980897041355615\\
212.01	0.00980898454005947\\
213.01	0.00980899897677934\\
214.01	0.00980901373067284\\
215.01	0.00980902880885704\\
216.01	0.0098090442186129\\
217.01	0.00980905996738919\\
218.01	0.00980907606280643\\
219.01	0.00980909251266091\\
220.01	0.00980910932492897\\
221.01	0.00980912650777118\\
222.01	0.00980914406953678\\
223.01	0.00980916201876819\\
224.01	0.00980918036420551\\
225.01	0.00980919911479132\\
226.01	0.00980921827967547\\
227.01	0.00980923786822002\\
228.01	0.00980925789000434\\
229.01	0.00980927835483018\\
230.01	0.00980929927272718\\
231.01	0.0098093206539581\\
232.01	0.00980934250902452\\
233.01	0.00980936484867253\\
234.01	0.00980938768389859\\
235.01	0.00980941102595547\\
236.01	0.00980943488635843\\
237.01	0.00980945927689151\\
238.01	0.00980948420961396\\
239.01	0.00980950969686683\\
240.01	0.00980953575127974\\
241.01	0.00980956238577781\\
242.01	0.00980958961358871\\
243.01	0.00980961744825\\
244.01	0.00980964590361649\\
245.01	0.00980967499386792\\
246.01	0.0098097047335167\\
247.01	0.009809735137416\\
248.01	0.00980976622076781\\
249.01	0.00980979799913146\\
250.01	0.00980983048843212\\
251.01	0.00980986370496962\\
252.01	0.00980989766542744\\
253.01	0.00980993238688199\\
254.01	0.00980996788681199\\
255.01	0.00981000418310818\\
256.01	0.00981004129408321\\
257.01	0.00981007923848179\\
258.01	0.00981011803549108\\
259.01	0.00981015770475127\\
260.01	0.00981019826636658\\
261.01	0.00981023974091627\\
262.01	0.00981028214946614\\
263.01	0.00981032551358015\\
264.01	0.00981036985533241\\
265.01	0.00981041519731943\\
266.01	0.0098104615626726\\
267.01	0.00981050897507098\\
268.01	0.00981055745875454\\
269.01	0.00981060703853742\\
270.01	0.00981065773982179\\
271.01	0.00981070958861183\\
272.01	0.00981076261152814\\
273.01	0.00981081683582245\\
274.01	0.00981087228939261\\
275.01	0.00981092900079803\\
276.01	0.00981098699927543\\
277.01	0.00981104631475486\\
278.01	0.00981110697787623\\
279.01	0.00981116902000615\\
280.01	0.00981123247325503\\
281.01	0.00981129737049476\\
282.01	0.0098113637453767\\
283.01	0.00981143163234997\\
284.01	0.00981150106668032\\
285.01	0.00981157208446925\\
286.01	0.00981164472267366\\
287.01	0.00981171901912586\\
288.01	0.00981179501255404\\
289.01	0.00981187274260315\\
290.01	0.00981195224985628\\
291.01	0.00981203357585641\\
292.01	0.00981211676312868\\
293.01	0.0098122018552031\\
294.01	0.00981228889663777\\
295.01	0.00981237793304244\\
296.01	0.00981246901110275\\
297.01	0.0098125621786049\\
298.01	0.00981265748446061\\
299.01	0.00981275497873291\\
300.01	0.00981285471266224\\
301.01	0.00981295673869308\\
302.01	0.00981306111050112\\
303.01	0.00981316788302106\\
304.01	0.00981327711247472\\
305.01	0.00981338885639989\\
306.01	0.00981350317367973\\
307.01	0.00981362012457249\\
308.01	0.00981373977074208\\
309.01	0.00981386217528906\\
310.01	0.00981398740278219\\
311.01	0.0098141155192906\\
312.01	0.0098142465924166\\
313.01	0.00981438069132897\\
314.01	0.00981451788679696\\
315.01	0.00981465825122476\\
316.01	0.00981480185868685\\
317.01	0.00981494878496367\\
318.01	0.00981509910757821\\
319.01	0.00981525290583298\\
320.01	0.00981541026084789\\
321.01	0.00981557125559867\\
322.01	0.009815735974956\\
323.01	0.00981590450572534\\
324.01	0.00981607693668747\\
325.01	0.00981625335863986\\
326.01	0.0098164338644386\\
327.01	0.00981661854904127\\
328.01	0.00981680750955056\\
329.01	0.0098170008452586\\
330.01	0.00981719865769233\\
331.01	0.00981740105065954\\
332.01	0.00981760813029585\\
333.01	0.00981782000511266\\
334.01	0.00981803678604609\\
335.01	0.00981825858650674\\
336.01	0.00981848552243061\\
337.01	0.00981871771233105\\
338.01	0.0098189552773517\\
339.01	0.00981919834132075\\
340.01	0.00981944703080621\\
341.01	0.00981970147517243\\
342.01	0.00981996180663798\\
343.01	0.00982022816033466\\
344.01	0.00982050067436802\\
345.01	0.00982077948987909\\
346.01	0.00982106475110773\\
347.01	0.00982135660545728\\
348.01	0.00982165520356078\\
349.01	0.00982196069934876\\
350.01	0.00982227325011843\\
351.01	0.00982259301660471\\
352.01	0.00982292016305264\\
353.01	0.00982325485729153\\
354.01	0.00982359727081075\\
355.01	0.0098239475788372\\
356.01	0.00982430596041443\\
357.01	0.00982467259848359\\
358.01	0.00982504767996608\\
359.01	0.00982543139584827\\
360.01	0.00982582394126826\\
361.01	0.00982622551560493\\
362.01	0.00982663632257002\\
363.01	0.00982705657030352\\
364.01	0.00982748647147349\\
365.01	0.00982792624338166\\
366.01	0.00982837610807682\\
367.01	0.00982883629247861\\
368.01	0.00982930702851594\\
369.01	0.00982978855328536\\
370.01	0.00983028110923704\\
371.01	0.00983078494439915\\
372.01	0.009831300312655\\
373.01	0.00983182747409331\\
374.01	0.0098323666954582\\
375.01	0.0098329182507305\\
376.01	0.0098334824218243\\
377.01	0.00983405949880053\\
378.01	0.00983464977742447\\
379.01	0.00983525357750229\\
380.01	0.0098358712362743\\
381.01	0.00983650309111121\\
382.01	0.0098371494883478\\
383.01	0.00983781078344905\\
384.01	0.00983848734127284\\
385.01	0.00983917953634092\\
386.01	0.00983988775311827\\
387.01	0.00984061238630143\\
388.01	0.00984135384111574\\
389.01	0.00984211253362219\\
390.01	0.00984288889103382\\
391.01	0.0098436833520423\\
392.01	0.00984449636715484\\
393.01	0.00984532839904187\\
394.01	0.00984617992289603\\
395.01	0.00984705142680236\\
396.01	0.00984794341212078\\
397.01	0.00984885639388064\\
398.01	0.00984979090118814\\
399.01	0.0098507474776471\\
400.01	0.00985172668179315\\
401.01	0.00985272908754247\\
402.01	0.00985375528465476\\
403.01	0.00985480587921179\\
404.01	0.00985588149411155\\
405.01	0.00985698276957858\\
406.01	0.00985811036369141\\
407.01	0.00985926495292725\\
408.01	0.009860447232725\\
409.01	0.00986165791806676\\
410.01	0.00986289774407896\\
411.01	0.00986416746665347\\
412.01	0.00986546786308951\\
413.01	0.00986679973275707\\
414.01	0.00986816389778261\\
415.01	0.00986956120375771\\
416.01	0.00987099252047151\\
417.01	0.00987245874266765\\
418.01	0.00987396079082651\\
419.01	0.00987549961197355\\
420.01	0.0098770761805143\\
421.01	0.00987869149909706\\
422.01	0.00988034659950356\\
423.01	0.00988204254356856\\
424.01	0.00988378042412868\\
425.01	0.00988556136600094\\
426.01	0.00988738652699127\\
427.01	0.00988925709893294\\
428.01	0.00989117430875495\\
429.01	0.0098931394195795\\
430.01	0.00989515373184786\\
431.01	0.00989721858447311\\
432.01	0.00989933535601752\\
433.01	0.00990150546589172\\
434.01	0.00990373037557172\\
435.01	0.00990601158982828\\
436.01	0.00990835065796195\\
437.01	0.00991074917503498\\
438.01	0.0099132087830883\\
439.01	0.00991573117232946\\
440.01	0.00991831808227295\\
441.01	0.00992097130280954\\
442.01	0.00992369267517551\\
443.01	0.00992648409278455\\
444.01	0.00992934750187626\\
445.01	0.00993228490192267\\
446.01	0.00993529834571985\\
447.01	0.00993838993907268\\
448.01	0.00994156183995766\\
449.01	0.00994481625701917\\
450.01	0.00994815544721714\\
451.01	0.00995158171239833\\
452.01	0.00995509739450326\\
453.01	0.00995870486904767\\
454.01	0.0099624065364222\\
455.01	0.00996620481043573\\
456.01	0.00997010210337537\\
457.01	0.00997410080665274\\
458.01	0.00997820326570706\\
459.01	0.00998241174530182\\
460.01	0.00998672834523047\\
461.01	0.00999115415772442\\
462.01	0.00999567150184493\\
463.01	0.00999971224540899\\
464.01	0.01\\
465.01	0.01\\
466.01	0.01\\
467.01	0.01\\
468.01	0.01\\
469.01	0.01\\
470.01	0.01\\
471.01	0.01\\
472.01	0.01\\
473.01	0.01\\
474.01	0.01\\
475.01	0.01\\
476.01	0.01\\
477.01	0.01\\
478.01	0.01\\
479.01	0.01\\
480.01	0.01\\
481.01	0.01\\
482.01	0.01\\
483.01	0.01\\
484.01	0.01\\
485.01	0.01\\
486.01	0.01\\
487.01	0.01\\
488.01	0.01\\
489.01	0.01\\
490.01	0.01\\
491.01	0.01\\
492.01	0.01\\
493.01	0.01\\
494.01	0.01\\
495.01	0.01\\
496.01	0.01\\
497.01	0.01\\
498.01	0.01\\
499.01	0.01\\
500.01	0.01\\
501.01	0.01\\
502.01	0.01\\
503.01	0.01\\
504.01	0.01\\
505.01	0.01\\
506.01	0.01\\
507.01	0.01\\
508.01	0.01\\
509.01	0.01\\
510.01	0.01\\
511.01	0.01\\
512.01	0.01\\
513.01	0.01\\
514.01	0.01\\
515.01	0.01\\
516.01	0.01\\
517.01	0.01\\
518.01	0.01\\
519.01	0.01\\
520.01	0.01\\
521.01	0.01\\
522.01	0.01\\
523.01	0.01\\
524.01	0.01\\
525.01	0.01\\
526.01	0.01\\
527.01	0.01\\
528.01	0.01\\
529.01	0.01\\
530.01	0.01\\
531.01	0.01\\
532.01	0.01\\
533.01	0.01\\
534.01	0.01\\
535.01	0.01\\
536.01	0.01\\
537.01	0.01\\
538.01	0.01\\
539.01	0.01\\
540.01	0.01\\
541.01	0.01\\
542.01	0.01\\
543.01	0.01\\
544.01	0.01\\
545.01	0.01\\
546.01	0.01\\
547.01	0.01\\
548.01	0.01\\
549.01	0.01\\
550.01	0.01\\
551.01	0.01\\
552.01	0.01\\
553.01	0.01\\
554.01	0.01\\
555.01	0.01\\
556.01	0.01\\
557.01	0.01\\
558.01	0.01\\
559.01	0.01\\
560.01	0.01\\
561.01	0.01\\
562.01	0.01\\
563.01	0.01\\
564.01	0.01\\
565.01	0.01\\
566.01	0.01\\
567.01	0.01\\
568.01	0.01\\
569.01	0.01\\
570.01	0.01\\
571.01	0.01\\
572.01	0.01\\
573.01	0.01\\
574.01	0.01\\
575.01	0.01\\
576.01	0.01\\
577.01	0.01\\
578.01	0.01\\
579.01	0.01\\
580.01	0.01\\
581.01	0.01\\
582.01	0.01\\
583.01	0.01\\
584.01	0.01\\
585.01	0.01\\
586.01	0.01\\
587.01	0.01\\
588.01	0.01\\
589.01	0.01\\
590.01	0.01\\
591.01	0.01\\
592.01	0.01\\
593.01	0.01\\
594.01	0.01\\
595.01	0.01\\
596.01	0.01\\
597.01	0.01\\
598.01	0.01\\
599.01	0.01\\
599.02	0.01\\
599.03	0.01\\
599.04	0.01\\
599.05	0.01\\
599.06	0.01\\
599.07	0.01\\
599.08	0.01\\
599.09	0.01\\
599.1	0.01\\
599.11	0.01\\
599.12	0.01\\
599.13	0.01\\
599.14	0.01\\
599.15	0.01\\
599.16	0.01\\
599.17	0.01\\
599.18	0.01\\
599.19	0.01\\
599.2	0.01\\
599.21	0.01\\
599.22	0.01\\
599.23	0.01\\
599.24	0.01\\
599.25	0.01\\
599.26	0.01\\
599.27	0.01\\
599.28	0.01\\
599.29	0.01\\
599.3	0.01\\
599.31	0.01\\
599.32	0.01\\
599.33	0.01\\
599.34	0.01\\
599.35	0.01\\
599.36	0.01\\
599.37	0.01\\
599.38	0.01\\
599.39	0.01\\
599.4	0.01\\
599.41	0.01\\
599.42	0.01\\
599.43	0.01\\
599.44	0.01\\
599.45	0.01\\
599.46	0.01\\
599.47	0.01\\
599.48	0.01\\
599.49	0.01\\
599.5	0.01\\
599.51	0.01\\
599.52	0.01\\
599.53	0.01\\
599.54	0.01\\
599.55	0.01\\
599.56	0.01\\
599.57	0.01\\
599.58	0.01\\
599.59	0.01\\
599.6	0.01\\
599.61	0.01\\
599.62	0.01\\
599.63	0.01\\
599.64	0.01\\
599.65	0.01\\
599.66	0.01\\
599.67	0.01\\
599.68	0.01\\
599.69	0.01\\
599.7	0.01\\
599.71	0.01\\
599.72	0.01\\
599.73	0.01\\
599.74	0.01\\
599.75	0.01\\
599.76	0.01\\
599.77	0.01\\
599.78	0.01\\
599.79	0.01\\
599.8	0.01\\
599.81	0.01\\
599.82	0.01\\
599.83	0.01\\
599.84	0.01\\
599.85	0.01\\
599.86	0.01\\
599.87	0.01\\
599.88	0.01\\
599.89	0.01\\
599.9	0.01\\
599.91	0.01\\
599.92	0.01\\
599.93	0.01\\
599.94	0.01\\
599.95	0.01\\
599.96	0.01\\
599.97	0.01\\
599.98	0.01\\
599.99	0.01\\
600	0.01\\
};
\addplot [color=blue!80!mycolor9,solid,forget plot]
  table[row sep=crcr]{%
0.01	0.00921108894345117\\
1.01	0.00921108909757844\\
2.01	0.00921108925493703\\
3.01	0.00921108941559513\\
4.01	0.00921108957962236\\
5.01	0.0092110897470898\\
6.01	0.00921108991807012\\
7.01	0.00921109009263744\\
8.01	0.00921109027086754\\
9.01	0.00921109045283774\\
10.01	0.00921109063862709\\
11.01	0.00921109082831624\\
12.01	0.00921109102198763\\
13.01	0.00921109121972539\\
14.01	0.00921109142161552\\
15.01	0.00921109162774578\\
16.01	0.00921109183820586\\
17.01	0.00921109205308733\\
18.01	0.00921109227248374\\
19.01	0.0092110924964906\\
20.01	0.00921109272520549\\
21.01	0.00921109295872806\\
22.01	0.00921109319716011\\
23.01	0.0092110934406056\\
24.01	0.00921109368917069\\
25.01	0.00921109394296387\\
26.01	0.00921109420209587\\
27.01	0.00921109446667987\\
28.01	0.00921109473683143\\
29.01	0.00921109501266855\\
30.01	0.00921109529431183\\
31.01	0.00921109558188442\\
32.01	0.00921109587551208\\
33.01	0.00921109617532331\\
34.01	0.00921109648144934\\
35.01	0.00921109679402422\\
36.01	0.00921109711318487\\
37.01	0.00921109743907115\\
38.01	0.00921109777182594\\
39.01	0.00921109811159515\\
40.01	0.00921109845852783\\
41.01	0.00921109881277628\\
42.01	0.00921109917449598\\
43.01	0.00921109954384587\\
44.01	0.0092110999209882\\
45.01	0.00921110030608877\\
46.01	0.00921110069931692\\
47.01	0.00921110110084566\\
48.01	0.0092111015108517\\
49.01	0.00921110192951552\\
50.01	0.00921110235702158\\
51.01	0.00921110279355824\\
52.01	0.00921110323931794\\
53.01	0.00921110369449727\\
54.01	0.00921110415929704\\
55.01	0.00921110463392246\\
56.01	0.00921110511858308\\
57.01	0.009211105613493\\
58.01	0.00921110611887097\\
59.01	0.00921110663494045\\
60.01	0.00921110716192973\\
61.01	0.009211107700072\\
62.01	0.00921110824960557\\
63.01	0.00921110881077383\\
64.01	0.00921110938382545\\
65.01	0.00921110996901451\\
66.01	0.00921111056660058\\
67.01	0.00921111117684884\\
68.01	0.00921111180003021\\
69.01	0.0092111124364215\\
70.01	0.00921111308630552\\
71.01	0.00921111374997123\\
72.01	0.00921111442771379\\
73.01	0.00921111511983486\\
74.01	0.00921111582664257\\
75.01	0.00921111654845179\\
76.01	0.00921111728558423\\
77.01	0.00921111803836857\\
78.01	0.00921111880714062\\
79.01	0.00921111959224359\\
80.01	0.00921112039402802\\
81.01	0.00921112121285218\\
82.01	0.00921112204908213\\
83.01	0.00921112290309188\\
84.01	0.00921112377526357\\
85.01	0.00921112466598775\\
86.01	0.00921112557566342\\
87.01	0.00921112650469828\\
88.01	0.00921112745350899\\
89.01	0.00921112842252129\\
90.01	0.00921112941217015\\
91.01	0.00921113042290014\\
92.01	0.0092111314551655\\
93.01	0.00921113250943041\\
94.01	0.00921113358616923\\
95.01	0.00921113468586669\\
96.01	0.00921113580901812\\
97.01	0.0092111369561297\\
98.01	0.00921113812771875\\
99.01	0.0092111393243139\\
100.01	0.00921114054645536\\
101.01	0.00921114179469524\\
102.01	0.00921114306959774\\
103.01	0.00921114437173944\\
104.01	0.00921114570170964\\
105.01	0.00921114706011054\\
106.01	0.00921114844755762\\
107.01	0.0092111498646799\\
108.01	0.00921115131212024\\
109.01	0.00921115279053564\\
110.01	0.00921115430059762\\
111.01	0.00921115584299246\\
112.01	0.00921115741842157\\
113.01	0.00921115902760188\\
114.01	0.00921116067126613\\
115.01	0.0092111623501632\\
116.01	0.00921116406505858\\
117.01	0.00921116581673461\\
118.01	0.009211167605991\\
119.01	0.00921116943364509\\
120.01	0.00921117130053235\\
121.01	0.00921117320750677\\
122.01	0.0092111751554412\\
123.01	0.00921117714522787\\
124.01	0.00921117917777883\\
125.01	0.00921118125402629\\
126.01	0.00921118337492323\\
127.01	0.00921118554144374\\
128.01	0.00921118775458359\\
129.01	0.00921119001536067\\
130.01	0.00921119232481549\\
131.01	0.00921119468401175\\
132.01	0.0092111970940368\\
133.01	0.00921119955600224\\
134.01	0.00921120207104442\\
135.01	0.00921120464032502\\
136.01	0.00921120726503166\\
137.01	0.00921120994637844\\
138.01	0.00921121268560657\\
139.01	0.00921121548398506\\
140.01	0.00921121834281115\\
141.01	0.0092112212634112\\
142.01	0.00921122424714123\\
143.01	0.00921122729538759\\
144.01	0.00921123040956767\\
145.01	0.00921123359113067\\
146.01	0.00921123684155824\\
147.01	0.00921124016236528\\
148.01	0.00921124355510069\\
149.01	0.00921124702134817\\
150.01	0.00921125056272696\\
151.01	0.00921125418089275\\
152.01	0.0092112578775384\\
153.01	0.00921126165439493\\
154.01	0.00921126551323228\\
155.01	0.0092112694558603\\
156.01	0.00921127348412959\\
157.01	0.00921127759993251\\
158.01	0.00921128180520409\\
159.01	0.00921128610192305\\
160.01	0.0092112904921128\\
161.01	0.00921129497784244\\
162.01	0.00921129956122791\\
163.01	0.00921130424443293\\
164.01	0.00921130902967027\\
165.01	0.00921131391920274\\
166.01	0.00921131891534446\\
167.01	0.00921132402046197\\
168.01	0.00921132923697549\\
169.01	0.00921133456736017\\
170.01	0.00921134001414731\\
171.01	0.00921134557992575\\
172.01	0.00921135126734318\\
173.01	0.00921135707910742\\
174.01	0.00921136301798795\\
175.01	0.00921136908681724\\
176.01	0.00921137528849233\\
177.01	0.0092113816259762\\
178.01	0.00921138810229944\\
179.01	0.00921139472056167\\
180.01	0.00921140148393334\\
181.01	0.00921140839565719\\
182.01	0.00921141545905011\\
183.01	0.00921142267750477\\
184.01	0.00921143005449139\\
185.01	0.00921143759355959\\
186.01	0.00921144529834025\\
187.01	0.00921145317254738\\
188.01	0.00921146121998007\\
189.01	0.0092114694445245\\
190.01	0.00921147785015599\\
191.01	0.00921148644094104\\
192.01	0.00921149522103949\\
193.01	0.00921150419470673\\
194.01	0.00921151336629594\\
195.01	0.00921152274026031\\
196.01	0.00921153232115551\\
197.01	0.00921154211364198\\
198.01	0.00921155212248745\\
199.01	0.00921156235256943\\
200.01	0.00921157280887784\\
201.01	0.00921158349651757\\
202.01	0.00921159442071124\\
203.01	0.00921160558680197\\
204.01	0.00921161700025613\\
205.01	0.0092116286666664\\
206.01	0.00921164059175454\\
207.01	0.00921165278137459\\
208.01	0.00921166524151585\\
209.01	0.00921167797830618\\
210.01	0.00921169099801517\\
211.01	0.00921170430705754\\
212.01	0.00921171791199648\\
213.01	0.00921173181954719\\
214.01	0.00921174603658049\\
215.01	0.00921176057012637\\
216.01	0.00921177542737787\\
217.01	0.00921179061569482\\
218.01	0.00921180614260777\\
219.01	0.00921182201582208\\
220.01	0.00921183824322191\\
221.01	0.00921185483287457\\
222.01	0.00921187179303469\\
223.01	0.00921188913214872\\
224.01	0.00921190685885937\\
225.01	0.00921192498201029\\
226.01	0.00921194351065076\\
227.01	0.00921196245404046\\
228.01	0.00921198182165454\\
229.01	0.00921200162318855\\
230.01	0.00921202186856373\\
231.01	0.00921204256793219\\
232.01	0.00921206373168246\\
233.01	0.00921208537044485\\
234.01	0.00921210749509733\\
235.01	0.00921213011677117\\
236.01	0.00921215324685694\\
237.01	0.0092121768970106\\
238.01	0.00921220107915963\\
239.01	0.00921222580550948\\
240.01	0.00921225108855001\\
241.01	0.00921227694106214\\
242.01	0.00921230337612464\\
243.01	0.00921233040712109\\
244.01	0.00921235804774701\\
245.01	0.00921238631201711\\
246.01	0.00921241521427272\\
247.01	0.00921244476918937\\
248.01	0.00921247499178461\\
249.01	0.00921250589742597\\
250.01	0.00921253750183899\\
251.01	0.00921256982111563\\
252.01	0.00921260287172274\\
253.01	0.0092126366705107\\
254.01	0.00921267123472234\\
255.01	0.00921270658200198\\
256.01	0.00921274273040478\\
257.01	0.00921277969840615\\
258.01	0.0092128175049114\\
259.01	0.00921285616926582\\
260.01	0.00921289571126458\\
261.01	0.00921293615116327\\
262.01	0.00921297750968832\\
263.01	0.00921301980804791\\
264.01	0.00921306306794294\\
265.01	0.00921310731157842\\
266.01	0.0092131525616748\\
267.01	0.00921319884147994\\
268.01	0.00921324617478105\\
269.01	0.00921329458591702\\
270.01	0.00921334409979092\\
271.01	0.0092133947418829\\
272.01	0.00921344653826326\\
273.01	0.00921349951560584\\
274.01	0.00921355370120173\\
275.01	0.00921360912297319\\
276.01	0.00921366580948799\\
277.01	0.00921372378997394\\
278.01	0.00921378309433382\\
279.01	0.00921384375316053\\
280.01	0.00921390579775274\\
281.01	0.00921396926013065\\
282.01	0.00921403417305232\\
283.01	0.00921410057003013\\
284.01	0.00921416848534777\\
285.01	0.00921423795407755\\
286.01	0.00921430901209799\\
287.01	0.00921438169611191\\
288.01	0.00921445604366491\\
289.01	0.00921453209316414\\
290.01	0.00921460988389755\\
291.01	0.00921468945605368\\
292.01	0.00921477085074166\\
293.01	0.00921485411001177\\
294.01	0.00921493927687652\\
295.01	0.00921502639533209\\
296.01	0.00921511551038031\\
297.01	0.00921520666805107\\
298.01	0.00921529991542541\\
299.01	0.00921539530065896\\
300.01	0.00921549287300591\\
301.01	0.0092155926828438\\
302.01	0.00921569478169855\\
303.01	0.00921579922227029\\
304.01	0.00921590605845982\\
305.01	0.00921601534539558\\
306.01	0.00921612713946142\\
307.01	0.00921624149832495\\
308.01	0.00921635848096661\\
309.01	0.00921647814770963\\
310.01	0.00921660056025051\\
311.01	0.00921672578169044\\
312.01	0.00921685387656756\\
313.01	0.00921698491089001\\
314.01	0.0092171189521699\\
315.01	0.00921725606945824\\
316.01	0.00921739633338067\\
317.01	0.00921753981617449\\
318.01	0.00921768659172643\\
319.01	0.0092178367356117\\
320.01	0.00921799032513416\\
321.01	0.00921814743936751\\
322.01	0.00921830815919799\\
323.01	0.00921847256736807\\
324.01	0.00921864074852167\\
325.01	0.00921881278925081\\
326.01	0.00921898877814351\\
327.01	0.00921916880583342\\
328.01	0.00921935296505091\\
329.01	0.00921954135067588\\
330.01	0.00921973405979224\\
331.01	0.00921993119174421\\
332.01	0.00922013284819449\\
333.01	0.00922033913318436\\
334.01	0.00922055015319579\\
335.01	0.00922076601721564\\
336.01	0.00922098683680215\\
337.01	0.00922121272615347\\
338.01	0.00922144380217881\\
339.01	0.00922168018457177\\
340.01	0.00922192199588644\\
341.01	0.00922216936161587\\
342.01	0.00922242241027346\\
343.01	0.00922268127347702\\
344.01	0.00922294608603565\\
345.01	0.00922321698603976\\
346.01	0.00922349411495412\\
347.01	0.00922377761771381\\
348.01	0.00922406764282382\\
349.01	0.00922436434246158\\
350.01	0.00922466787258323\\
351.01	0.00922497839303309\\
352.01	0.00922529606765697\\
353.01	0.00922562106441894\\
354.01	0.009225953555522\\
355.01	0.00922629371753241\\
356.01	0.00922664173150812\\
357.01	0.00922699778313105\\
358.01	0.00922736206284342\\
359.01	0.00922773476598845\\
360.01	0.00922811609295501\\
361.01	0.00922850624932687\\
362.01	0.00922890544603633\\
363.01	0.00922931389952239\\
364.01	0.00922973183189374\\
365.01	0.00923015947109653\\
366.01	0.00923059705108736\\
367.01	0.00923104481201135\\
368.01	0.00923150300038603\\
369.01	0.00923197186929067\\
370.01	0.00923245167856213\\
371.01	0.00923294269499689\\
372.01	0.00923344519256014\\
373.01	0.00923395945260188\\
374.01	0.00923448576408023\\
375.01	0.00923502442378896\\
376.01	0.00923557573655237\\
377.01	0.00923614001494795\\
378.01	0.00923671757396008\\
379.01	0.00923730874275894\\
380.01	0.00923791388387595\\
381.01	0.00923853334486768\\
382.01	0.00923916748183644\\
383.01	0.00923981666015873\\
384.01	0.00924048125473469\\
385.01	0.00924116165024438\\
386.01	0.0092418582414112\\
387.01	0.00924257143327242\\
388.01	0.00924330164145737\\
389.01	0.00924404929247327\\
390.01	0.00924481482399892\\
391.01	0.00924559868518673\\
392.01	0.00924640133697308\\
393.01	0.00924722325239746\\
394.01	0.00924806491693053\\
395.01	0.00924892682881168\\
396.01	0.00924980949939585\\
397.01	0.00925071345351055\\
398.01	0.00925163922982314\\
399.01	0.00925258738121843\\
400.01	0.00925355847518767\\
401.01	0.00925455309422862\\
402.01	0.00925557183625763\\
403.01	0.00925661531503378\\
404.01	0.00925768416059588\\
405.01	0.00925877901971252\\
406.01	0.00925990055634604\\
407.01	0.00926104945213043\\
408.01	0.00926222640686443\\
409.01	0.00926343213901974\\
410.01	0.00926466738626559\\
411.01	0.0092659329060099\\
412.01	0.00926722947595804\\
413.01	0.00926855789468987\\
414.01	0.00926991898225584\\
415.01	0.00927131358079306\\
416.01	0.00927274255516225\\
417.01	0.00927420679360662\\
418.01	0.00927570720843356\\
419.01	0.00927724473672032\\
420.01	0.00927882034104493\\
421.01	0.00928043501024345\\
422.01	0.0092820897601948\\
423.01	0.00928378563463494\\
424.01	0.00928552370600124\\
425.01	0.00928730507630909\\
426.01	0.00928913087806211\\
427.01	0.00929100227519775\\
428.01	0.0092929204640701\\
429.01	0.00929488667447159\\
430.01	0.00929690217069581\\
431.01	0.00929896825264339\\
432.01	0.00930108625697287\\
433.01	0.00930325755829929\\
434.01	0.00930548357044225\\
435.01	0.00930776574772598\\
436.01	0.00931010558633428\\
437.01	0.00931250462572198\\
438.01	0.00931496445008634\\
439.01	0.00931748668990021\\
440.01	0.00932007302351014\\
441.01	0.00932272517880156\\
442.01	0.00932544493493375\\
443.01	0.0093282341241473\\
444.01	0.00933109463364612\\
445.01	0.00933402840755682\\
446.01	0.00933703744896772\\
447.01	0.0093401238220498\\
448.01	0.00934328965426202\\
449.01	0.00934653713864381\\
450.01	0.00934986853619746\\
451.01	0.00935328617836443\\
452.01	0.0093567924695999\\
453.01	0.00936038989005259\\
454.01	0.00936408099835907\\
455.01	0.00936786843456627\\
456.01	0.00937175492320142\\
457.01	0.00937574327651184\\
458.01	0.00937983639783687\\
459.01	0.00938403728414399\\
460.01	0.00938834901341874\\
461.01	0.0093927744991172\\
462.01	0.00939731261417764\\
463.01	0.00940191165965253\\
464.01	0.00940651584863829\\
465.01	0.00941124578153715\\
466.01	0.00941612535678167\\
467.01	0.00942107866392531\\
468.01	0.00942604576696724\\
469.01	0.00943115651871189\\
470.01	0.00943640782485806\\
471.01	0.00944179890261241\\
472.01	0.00944735264085425\\
473.01	0.00945310306573079\\
474.01	0.00945925966041799\\
475.01	0.00946483081489598\\
476.01	0.00947024052540716\\
477.01	0.0094757919546411\\
478.01	0.00948148927711075\\
479.01	0.00948733679760845\\
480.01	0.0094933389529326\\
481.01	0.00949950031282027\\
482.01	0.00950582557982904\\
483.01	0.00951231958784148\\
484.01	0.0095189872987783\\
485.01	0.00952583379698963\\
486.01	0.00953286428057719\\
487.01	0.00954008404782149\\
488.01	0.00954749846484649\\
489.01	0.00955511273202057\\
490.01	0.00956292983508929\\
491.01	0.00957095547030021\\
492.01	0.00957919070781336\\
493.01	0.00958765658063882\\
494.01	0.00959636509544663\\
495.01	0.00960532279064326\\
496.01	0.00961453590847657\\
497.01	0.00962401039666919\\
498.01	0.00963375403002067\\
499.01	0.00964377876930783\\
500.01	0.00965410040047059\\
501.01	0.00966473935232702\\
502.01	0.0096756552260708\\
503.01	0.00968682625103773\\
504.01	0.00969835078995916\\
505.01	0.00971024869980506\\
506.01	0.00972254102155007\\
507.01	0.00973525087114764\\
508.01	0.00974840373137549\\
509.01	0.00976202779420764\\
510.01	0.00977615436467055\\
511.01	0.00979081834669956\\
512.01	0.00980605889893121\\
513.01	0.00982192094670078\\
514.01	0.00983846360566679\\
515.01	0.00985581757267285\\
516.01	0.00987421662991688\\
517.01	0.0098922816907823\\
518.01	0.00991636879998524\\
519.01	0.00996005438860847\\
520.01	0.00999814685474962\\
521.01	0.01\\
522.01	0.01\\
523.01	0.01\\
524.01	0.01\\
525.01	0.01\\
526.01	0.01\\
527.01	0.01\\
528.01	0.01\\
529.01	0.01\\
530.01	0.01\\
531.01	0.01\\
532.01	0.01\\
533.01	0.01\\
534.01	0.01\\
535.01	0.01\\
536.01	0.01\\
537.01	0.01\\
538.01	0.01\\
539.01	0.01\\
540.01	0.01\\
541.01	0.01\\
542.01	0.01\\
543.01	0.01\\
544.01	0.01\\
545.01	0.01\\
546.01	0.01\\
547.01	0.01\\
548.01	0.01\\
549.01	0.01\\
550.01	0.01\\
551.01	0.01\\
552.01	0.01\\
553.01	0.01\\
554.01	0.01\\
555.01	0.01\\
556.01	0.01\\
557.01	0.01\\
558.01	0.01\\
559.01	0.01\\
560.01	0.01\\
561.01	0.01\\
562.01	0.01\\
563.01	0.01\\
564.01	0.01\\
565.01	0.01\\
566.01	0.01\\
567.01	0.01\\
568.01	0.01\\
569.01	0.01\\
570.01	0.01\\
571.01	0.01\\
572.01	0.01\\
573.01	0.01\\
574.01	0.01\\
575.01	0.01\\
576.01	0.01\\
577.01	0.01\\
578.01	0.01\\
579.01	0.01\\
580.01	0.01\\
581.01	0.01\\
582.01	0.01\\
583.01	0.01\\
584.01	0.01\\
585.01	0.01\\
586.01	0.01\\
587.01	0.01\\
588.01	0.01\\
589.01	0.01\\
590.01	0.01\\
591.01	0.01\\
592.01	0.01\\
593.01	0.01\\
594.01	0.01\\
595.01	0.01\\
596.01	0.01\\
597.01	0.01\\
598.01	0.01\\
599.01	0.01\\
599.02	0.01\\
599.03	0.01\\
599.04	0.01\\
599.05	0.01\\
599.06	0.01\\
599.07	0.01\\
599.08	0.01\\
599.09	0.01\\
599.1	0.01\\
599.11	0.01\\
599.12	0.01\\
599.13	0.01\\
599.14	0.01\\
599.15	0.01\\
599.16	0.01\\
599.17	0.01\\
599.18	0.01\\
599.19	0.01\\
599.2	0.01\\
599.21	0.01\\
599.22	0.01\\
599.23	0.01\\
599.24	0.01\\
599.25	0.01\\
599.26	0.01\\
599.27	0.01\\
599.28	0.01\\
599.29	0.01\\
599.3	0.01\\
599.31	0.01\\
599.32	0.01\\
599.33	0.01\\
599.34	0.01\\
599.35	0.01\\
599.36	0.01\\
599.37	0.01\\
599.38	0.01\\
599.39	0.01\\
599.4	0.01\\
599.41	0.01\\
599.42	0.01\\
599.43	0.01\\
599.44	0.01\\
599.45	0.01\\
599.46	0.01\\
599.47	0.01\\
599.48	0.01\\
599.49	0.01\\
599.5	0.01\\
599.51	0.01\\
599.52	0.01\\
599.53	0.01\\
599.54	0.01\\
599.55	0.01\\
599.56	0.01\\
599.57	0.01\\
599.58	0.01\\
599.59	0.01\\
599.6	0.01\\
599.61	0.01\\
599.62	0.01\\
599.63	0.01\\
599.64	0.01\\
599.65	0.01\\
599.66	0.01\\
599.67	0.01\\
599.68	0.01\\
599.69	0.01\\
599.7	0.01\\
599.71	0.01\\
599.72	0.01\\
599.73	0.01\\
599.74	0.01\\
599.75	0.01\\
599.76	0.01\\
599.77	0.01\\
599.78	0.01\\
599.79	0.01\\
599.8	0.01\\
599.81	0.01\\
599.82	0.01\\
599.83	0.01\\
599.84	0.01\\
599.85	0.01\\
599.86	0.01\\
599.87	0.01\\
599.88	0.01\\
599.89	0.01\\
599.9	0.01\\
599.91	0.01\\
599.92	0.01\\
599.93	0.01\\
599.94	0.01\\
599.95	0.01\\
599.96	0.01\\
599.97	0.01\\
599.98	0.01\\
599.99	0.01\\
600	0.01\\
};
\addplot [color=blue,solid,forget plot]
  table[row sep=crcr]{%
0.01	0.00851332874111991\\
1.01	0.00851332879671018\\
2.01	0.00851332885346802\\
3.01	0.00851332891141812\\
4.01	0.00851332897058573\\
5.01	0.00851332903099658\\
6.01	0.00851332909267698\\
7.01	0.0085133291556538\\
8.01	0.00851332921995449\\
9.01	0.00851332928560708\\
10.01	0.0085133293526402\\
11.01	0.00851332942108309\\
12.01	0.00851332949096559\\
13.01	0.00851332956231827\\
14.01	0.00851332963517223\\
15.01	0.00851332970955932\\
16.01	0.00851332978551205\\
17.01	0.00851332986306359\\
18.01	0.0085133299422479\\
19.01	0.00851333002309961\\
20.01	0.00851333010565408\\
21.01	0.00851333018994745\\
22.01	0.00851333027601668\\
23.01	0.00851333036389944\\
24.01	0.00851333045363426\\
25.01	0.00851333054526046\\
26.01	0.00851333063881827\\
27.01	0.00851333073434872\\
28.01	0.00851333083189374\\
29.01	0.00851333093149618\\
30.01	0.00851333103319978\\
31.01	0.00851333113704924\\
32.01	0.00851333124309023\\
33.01	0.00851333135136937\\
34.01	0.00851333146193432\\
35.01	0.00851333157483374\\
36.01	0.00851333169011734\\
37.01	0.00851333180783592\\
38.01	0.00851333192804136\\
39.01	0.00851333205078666\\
40.01	0.00851333217612595\\
41.01	0.00851333230411454\\
42.01	0.00851333243480898\\
43.01	0.00851333256826693\\
44.01	0.0085133327045474\\
45.01	0.00851333284371061\\
46.01	0.00851333298581812\\
47.01	0.00851333313093279\\
48.01	0.00851333327911885\\
49.01	0.00851333343044196\\
50.01	0.00851333358496911\\
51.01	0.00851333374276883\\
52.01	0.00851333390391108\\
53.01	0.00851333406846735\\
54.01	0.00851333423651067\\
55.01	0.00851333440811568\\
56.01	0.00851333458335859\\
57.01	0.00851333476231732\\
58.01	0.00851333494507145\\
59.01	0.00851333513170226\\
60.01	0.00851333532229285\\
61.01	0.00851333551692808\\
62.01	0.00851333571569468\\
63.01	0.00851333591868126\\
64.01	0.00851333612597834\\
65.01	0.00851333633767842\\
66.01	0.00851333655387602\\
67.01	0.0085133367746677\\
68.01	0.00851333700015216\\
69.01	0.00851333723043017\\
70.01	0.00851333746560478\\
71.01	0.00851333770578126\\
72.01	0.00851333795106715\\
73.01	0.00851333820157235\\
74.01	0.00851333845740918\\
75.01	0.00851333871869239\\
76.01	0.00851333898553923\\
77.01	0.00851333925806952\\
78.01	0.00851333953640572\\
79.01	0.00851333982067295\\
80.01	0.00851334011099902\\
81.01	0.00851334040751464\\
82.01	0.00851334071035329\\
83.01	0.00851334101965143\\
84.01	0.00851334133554846\\
85.01	0.00851334165818688\\
86.01	0.00851334198771229\\
87.01	0.0085133423242735\\
88.01	0.00851334266802257\\
89.01	0.0085133430191149\\
90.01	0.00851334337770934\\
91.01	0.00851334374396817\\
92.01	0.00851334411805727\\
93.01	0.00851334450014617\\
94.01	0.00851334489040814\\
95.01	0.00851334528902023\\
96.01	0.00851334569616342\\
97.01	0.0085133461120227\\
98.01	0.00851334653678704\\
99.01	0.00851334697064969\\
100.01	0.00851334741380811\\
101.01	0.00851334786646411\\
102.01	0.00851334832882398\\
103.01	0.00851334880109856\\
104.01	0.00851334928350334\\
105.01	0.00851334977625859\\
106.01	0.00851335027958945\\
107.01	0.00851335079372603\\
108.01	0.00851335131890355\\
109.01	0.00851335185536242\\
110.01	0.00851335240334842\\
111.01	0.00851335296311274\\
112.01	0.00851335353491215\\
113.01	0.00851335411900915\\
114.01	0.008513354715672\\
115.01	0.00851335532517501\\
116.01	0.0085133559477985\\
117.01	0.0085133565838291\\
118.01	0.00851335723355974\\
119.01	0.00851335789728996\\
120.01	0.00851335857532587\\
121.01	0.00851335926798048\\
122.01	0.00851335997557375\\
123.01	0.00851336069843277\\
124.01	0.00851336143689191\\
125.01	0.00851336219129305\\
126.01	0.00851336296198568\\
127.01	0.00851336374932708\\
128.01	0.00851336455368256\\
129.01	0.00851336537542558\\
130.01	0.00851336621493795\\
131.01	0.00851336707261001\\
132.01	0.0085133679488409\\
133.01	0.00851336884403867\\
134.01	0.00851336975862051\\
135.01	0.00851337069301295\\
136.01	0.0085133716476522\\
137.01	0.00851337262298412\\
138.01	0.00851337361946472\\
139.01	0.00851337463756013\\
140.01	0.00851337567774708\\
141.01	0.00851337674051297\\
142.01	0.00851337782635615\\
143.01	0.00851337893578624\\
144.01	0.00851338006932434\\
145.01	0.00851338122750325\\
146.01	0.00851338241086785\\
147.01	0.00851338361997531\\
148.01	0.00851338485539531\\
149.01	0.00851338611771049\\
150.01	0.00851338740751661\\
151.01	0.0085133887254229\\
152.01	0.00851339007205238\\
153.01	0.00851339144804219\\
154.01	0.00851339285404388\\
155.01	0.00851339429072378\\
156.01	0.00851339575876329\\
157.01	0.00851339725885929\\
158.01	0.00851339879172449\\
159.01	0.00851340035808774\\
160.01	0.00851340195869447\\
161.01	0.00851340359430704\\
162.01	0.00851340526570516\\
163.01	0.00851340697368625\\
164.01	0.00851340871906588\\
165.01	0.00851341050267821\\
166.01	0.0085134123253764\\
167.01	0.00851341418803301\\
168.01	0.00851341609154058\\
169.01	0.0085134180368119\\
170.01	0.00851342002478066\\
171.01	0.00851342205640187\\
172.01	0.00851342413265237\\
173.01	0.00851342625453122\\
174.01	0.00851342842306048\\
175.01	0.00851343063928546\\
176.01	0.00851343290427545\\
177.01	0.00851343521912422\\
178.01	0.00851343758495054\\
179.01	0.00851344000289891\\
180.01	0.00851344247413997\\
181.01	0.00851344499987128\\
182.01	0.00851344758131784\\
183.01	0.00851345021973277\\
184.01	0.00851345291639799\\
185.01	0.00851345567262482\\
186.01	0.00851345848975477\\
187.01	0.00851346136916016\\
188.01	0.00851346431224488\\
189.01	0.00851346732044513\\
190.01	0.00851347039523016\\
191.01	0.00851347353810305\\
192.01	0.00851347675060151\\
193.01	0.00851348003429867\\
194.01	0.00851348339080395\\
195.01	0.00851348682176383\\
196.01	0.00851349032886285\\
197.01	0.00851349391382435\\
198.01	0.00851349757841146\\
199.01	0.00851350132442806\\
200.01	0.00851350515371966\\
201.01	0.00851350906817441\\
202.01	0.00851351306972413\\
203.01	0.00851351716034526\\
204.01	0.00851352134205997\\
205.01	0.00851352561693717\\
206.01	0.00851352998709372\\
207.01	0.00851353445469537\\
208.01	0.00851353902195804\\
209.01	0.008513543691149\\
210.01	0.00851354846458801\\
211.01	0.00851355334464848\\
212.01	0.00851355833375897\\
213.01	0.00851356343440416\\
214.01	0.00851356864912643\\
215.01	0.00851357398052706\\
216.01	0.00851357943126762\\
217.01	0.00851358500407145\\
218.01	0.00851359070172502\\
219.01	0.00851359652707942\\
220.01	0.00851360248305194\\
221.01	0.0085136085726275\\
222.01	0.00851361479886034\\
223.01	0.00851362116487549\\
224.01	0.00851362767387058\\
225.01	0.00851363432911742\\
226.01	0.00851364113396375\\
227.01	0.00851364809183506\\
228.01	0.00851365520623631\\
229.01	0.00851366248075382\\
230.01	0.00851366991905715\\
231.01	0.00851367752490108\\
232.01	0.00851368530212747\\
233.01	0.0085136932546674\\
234.01	0.00851370138654315\\
235.01	0.00851370970187036\\
236.01	0.00851371820486018\\
237.01	0.00851372689982144\\
238.01	0.00851373579116297\\
239.01	0.00851374488339583\\
240.01	0.00851375418113573\\
241.01	0.00851376368910539\\
242.01	0.00851377341213708\\
243.01	0.00851378335517503\\
244.01	0.0085137935232781\\
245.01	0.00851380392162236\\
246.01	0.00851381455550379\\
247.01	0.00851382543034103\\
248.01	0.00851383655167819\\
249.01	0.00851384792518772\\
250.01	0.00851385955667332\\
251.01	0.00851387145207302\\
252.01	0.00851388361746212\\
253.01	0.00851389605905641\\
254.01	0.00851390878321534\\
255.01	0.00851392179644532\\
256.01	0.00851393510540302\\
257.01	0.00851394871689877\\
258.01	0.00851396263790011\\
259.01	0.00851397687553531\\
260.01	0.00851399143709704\\
261.01	0.00851400633004601\\
262.01	0.00851402156201491\\
263.01	0.0085140371408122\\
264.01	0.00851405307442606\\
265.01	0.00851406937102851\\
266.01	0.00851408603897953\\
267.01	0.00851410308683128\\
268.01	0.00851412052333247\\
269.01	0.00851413835743269\\
270.01	0.00851415659828704\\
271.01	0.00851417525526069\\
272.01	0.00851419433793356\\
273.01	0.00851421385610522\\
274.01	0.00851423381979979\\
275.01	0.00851425423927092\\
276.01	0.00851427512500704\\
277.01	0.00851429648773654\\
278.01	0.00851431833843317\\
279.01	0.00851434068832159\\
280.01	0.00851436354888293\\
281.01	0.00851438693186056\\
282.01	0.00851441084926599\\
283.01	0.00851443531338489\\
284.01	0.00851446033678318\\
285.01	0.00851448593231346\\
286.01	0.00851451211312125\\
287.01	0.00851453889265184\\
288.01	0.0085145662846568\\
289.01	0.00851459430320103\\
290.01	0.00851462296266978\\
291.01	0.00851465227777591\\
292.01	0.0085146822635673\\
293.01	0.0085147129354345\\
294.01	0.00851474430911842\\
295.01	0.0085147764007184\\
296.01	0.00851480922670039\\
297.01	0.0085148428039053\\
298.01	0.00851487714955763\\
299.01	0.00851491228127432\\
300.01	0.00851494821707382\\
301.01	0.00851498497538532\\
302.01	0.00851502257505842\\
303.01	0.00851506103537288\\
304.01	0.00851510037604871\\
305.01	0.00851514061725652\\
306.01	0.00851518177962822\\
307.01	0.00851522388426793\\
308.01	0.00851526695276327\\
309.01	0.00851531100719694\\
310.01	0.00851535607015871\\
311.01	0.00851540216475766\\
312.01	0.00851544931463482\\
313.01	0.00851549754397626\\
314.01	0.00851554687752648\\
315.01	0.00851559734060232\\
316.01	0.00851564895910721\\
317.01	0.00851570175954583\\
318.01	0.00851575576903948\\
319.01	0.00851581101534154\\
320.01	0.00851586752685382\\
321.01	0.00851592533264323\\
322.01	0.00851598446245895\\
323.01	0.00851604494675034\\
324.01	0.00851610681668532\\
325.01	0.00851617010416926\\
326.01	0.00851623484186484\\
327.01	0.00851630106321214\\
328.01	0.00851636880244967\\
329.01	0.00851643809463606\\
330.01	0.00851650897567243\\
331.01	0.00851658148232552\\
332.01	0.00851665565225163\\
333.01	0.00851673152402126\\
334.01	0.00851680913714471\\
335.01	0.00851688853209844\\
336.01	0.00851696975035224\\
337.01	0.00851705283439752\\
338.01	0.00851713782777622\\
339.01	0.00851722477511091\\
340.01	0.00851731372213571\\
341.01	0.00851740471572835\\
342.01	0.00851749780394305\\
343.01	0.00851759303604459\\
344.01	0.00851769046254337\\
345.01	0.0085177901352316\\
346.01	0.00851789210722057\\
347.01	0.00851799643297901\\
348.01	0.00851810316837259\\
349.01	0.00851821237070473\\
350.01	0.00851832409875833\\
351.01	0.00851843841283893\\
352.01	0.00851855537481896\\
353.01	0.0085186750481833\\
354.01	0.00851879749807593\\
355.01	0.00851892279134808\\
356.01	0.00851905099660741\\
357.01	0.0085191821842687\\
358.01	0.00851931642660561\\
359.01	0.00851945379780399\\
360.01	0.00851959437401624\\
361.01	0.00851973823341729\\
362.01	0.00851988545626163\\
363.01	0.00852003612494187\\
364.01	0.00852019032404855\\
365.01	0.00852034814043119\\
366.01	0.00852050966326084\\
367.01	0.00852067498409377\\
368.01	0.00852084419693646\\
369.01	0.00852101739831183\\
370.01	0.0085211946873268\\
371.01	0.00852137616574072\\
372.01	0.0085215619380352\\
373.01	0.00852175211148479\\
374.01	0.00852194679622858\\
375.01	0.00852214610534118\\
376.01	0.00852235015489174\\
377.01	0.0085225590638877\\
378.01	0.00852277295377686\\
379.01	0.0085229919508181\\
380.01	0.00852321618480231\\
381.01	0.00852344578722066\\
382.01	0.00852368089303297\\
383.01	0.00852392164077436\\
384.01	0.00852416817265735\\
385.01	0.00852442063467709\\
386.01	0.00852467917672049\\
387.01	0.00852494395267876\\
388.01	0.00852521512056409\\
389.01	0.00852549284263029\\
390.01	0.00852577728549788\\
391.01	0.00852606862028348\\
392.01	0.00852636702273421\\
393.01	0.00852667267336677\\
394.01	0.00852698575761212\\
395.01	0.00852730646596539\\
396.01	0.00852763499414178\\
397.01	0.00852797154323861\\
398.01	0.0085283163199038\\
399.01	0.00852866953651131\\
400.01	0.00852903141134382\\
401.01	0.008529402168783\\
402.01	0.00852978203950815\\
403.01	0.00853017126070327\\
404.01	0.00853057007627347\\
405.01	0.00853097873707103\\
406.01	0.0085313975011319\\
407.01	0.00853182663392318\\
408.01	0.00853226640860232\\
409.01	0.00853271710628899\\
410.01	0.00853317901635003\\
411.01	0.00853365243669897\\
412.01	0.00853413767411058\\
413.01	0.00853463504455167\\
414.01	0.00853514487352953\\
415.01	0.00853566749645883\\
416.01	0.00853620325904867\\
417.01	0.00853675251771109\\
418.01	0.00853731563999265\\
419.01	0.00853789300503082\\
420.01	0.00853848500403716\\
421.01	0.00853909204080928\\
422.01	0.00853971453227379\\
423.01	0.00854035290906294\\
424.01	0.00854100761612752\\
425.01	0.008541679113389\\
426.01	0.00854236787643436\\
427.01	0.00854307439725726\\
428.01	0.00854379918504937\\
429.01	0.00854454276704671\\
430.01	0.00854530568943574\\
431.01	0.0085460885183248\\
432.01	0.00854689184078718\\
433.01	0.00854771626598274\\
434.01	0.00854856242636574\\
435.01	0.00854943097898797\\
436.01	0.00855032260690665\\
437.01	0.00855123802070853\\
438.01	0.00855217796016293\\
439.01	0.00855314319601784\\
440.01	0.00855413453195583\\
441.01	0.00855515280672833\\
442.01	0.00855619889648994\\
443.01	0.00855727371735749\\
444.01	0.00855837822822269\\
445.01	0.00855951343385113\\
446.01	0.00856068038830647\\
447.01	0.00856188019874435\\
448.01	0.00856311402962869\\
449.01	0.00856438310743165\\
450.01	0.00856568872588968\\
451.01	0.00856703225190094\\
452.01	0.00856841513216562\\
453.01	0.00856983890068926\\
454.01	0.00857130518729276\\
455.01	0.00857281572730137\\
456.01	0.00857437237261919\\
457.01	0.00857597710443567\\
458.01	0.00857763204783812\\
459.01	0.0085793394883786\\
460.01	0.00858110188738293\\
461.01	0.00858292185510583\\
462.01	0.00858480169043087\\
463.01	0.00858674128328256\\
464.01	0.00858874488580694\\
465.01	0.00859082224758488\\
466.01	0.00859299916213625\\
467.01	0.00859526614543668\\
468.01	0.00859733328976934\\
469.01	0.00859948299348634\\
470.01	0.00860171592593705\\
471.01	0.00860401063681875\\
472.01	0.00860640443248413\\
473.01	0.00860892106162595\\
474.01	0.00861180202011184\\
475.01	0.00861681143869944\\
476.01	0.00862234308761062\\
477.01	0.00862801603513246\\
478.01	0.00863383427611849\\
479.01	0.00863980194144616\\
480.01	0.00864592330437171\\
481.01	0.00865220278732148\\
482.01	0.00865864496916522\\
483.01	0.00866525459302367\\
484.01	0.00867203657467244\\
485.01	0.00867899601161019\\
486.01	0.00868613819283572\\
487.01	0.00869346860901762\\
488.01	0.00870099295893498\\
489.01	0.00870871711602004\\
490.01	0.00871664701723107\\
491.01	0.00872478910241052\\
492.01	0.00873314989142529\\
493.01	0.00874173766525284\\
494.01	0.0087505598241744\\
495.01	0.00875962387666015\\
496.01	0.00876893767204288\\
497.01	0.0087785094618282\\
498.01	0.00878834808788105\\
499.01	0.00879846332558306\\
500.01	0.00880886789421755\\
501.01	0.00881958508416636\\
502.01	0.0088305930665103\\
503.01	0.00884174477446774\\
504.01	0.00885321989685707\\
505.01	0.00886503719565277\\
506.01	0.00887721048991623\\
507.01	0.00888975442974024\\
508.01	0.00890268457076784\\
509.01	0.00891601745821034\\
510.01	0.00892977072267643\\
511.01	0.00894396319539225\\
512.01	0.00895861509225591\\
513.01	0.00897374869033576\\
514.01	0.00898939349871925\\
515.01	0.00900563383206323\\
516.01	0.00902291526562311\\
517.01	0.00903930867369387\\
518.01	0.00905529963267023\\
519.01	0.00907220801133853\\
520.01	0.00908897671067582\\
521.01	0.00910522407637488\\
522.01	0.00912157679396798\\
523.01	0.00913814928186252\\
524.01	0.00915529289708311\\
525.01	0.00917304475334421\\
526.01	0.00919144616047542\\
527.01	0.00921054326624265\\
528.01	0.00923038762439255\\
529.01	0.00925103644318186\\
530.01	0.00927255873337311\\
531.01	0.00929503074496483\\
532.01	0.00931854097522513\\
533.01	0.00934323123125701\\
534.01	0.009370135044619\\
535.01	0.00941740712267001\\
536.01	0.00948007000197791\\
537.01	0.00954397864109383\\
538.01	0.00960916764714868\\
539.01	0.00967580163460858\\
540.01	0.00974417307012064\\
541.01	0.0098134628691532\\
542.01	0.00988194439853754\\
543.01	0.00995110711589977\\
544.01	0.01\\
545.01	0.01\\
546.01	0.01\\
547.01	0.01\\
548.01	0.01\\
549.01	0.01\\
550.01	0.01\\
551.01	0.01\\
552.01	0.01\\
553.01	0.01\\
554.01	0.01\\
555.01	0.01\\
556.01	0.01\\
557.01	0.01\\
558.01	0.01\\
559.01	0.01\\
560.01	0.01\\
561.01	0.01\\
562.01	0.01\\
563.01	0.01\\
564.01	0.01\\
565.01	0.01\\
566.01	0.01\\
567.01	0.01\\
568.01	0.01\\
569.01	0.01\\
570.01	0.01\\
571.01	0.01\\
572.01	0.01\\
573.01	0.01\\
574.01	0.01\\
575.01	0.01\\
576.01	0.01\\
577.01	0.01\\
578.01	0.01\\
579.01	0.01\\
580.01	0.01\\
581.01	0.01\\
582.01	0.01\\
583.01	0.01\\
584.01	0.01\\
585.01	0.01\\
586.01	0.01\\
587.01	0.01\\
588.01	0.01\\
589.01	0.01\\
590.01	0.01\\
591.01	0.01\\
592.01	0.01\\
593.01	0.01\\
594.01	0.01\\
595.01	0.01\\
596.01	0.01\\
597.01	0.01\\
598.01	0.01\\
599.01	0.01\\
599.02	0.01\\
599.03	0.01\\
599.04	0.01\\
599.05	0.01\\
599.06	0.01\\
599.07	0.01\\
599.08	0.01\\
599.09	0.01\\
599.1	0.01\\
599.11	0.01\\
599.12	0.01\\
599.13	0.01\\
599.14	0.01\\
599.15	0.01\\
599.16	0.01\\
599.17	0.01\\
599.18	0.01\\
599.19	0.01\\
599.2	0.01\\
599.21	0.01\\
599.22	0.01\\
599.23	0.01\\
599.24	0.01\\
599.25	0.01\\
599.26	0.01\\
599.27	0.01\\
599.28	0.01\\
599.29	0.01\\
599.3	0.01\\
599.31	0.01\\
599.32	0.01\\
599.33	0.01\\
599.34	0.01\\
599.35	0.01\\
599.36	0.01\\
599.37	0.01\\
599.38	0.01\\
599.39	0.01\\
599.4	0.01\\
599.41	0.01\\
599.42	0.01\\
599.43	0.01\\
599.44	0.01\\
599.45	0.01\\
599.46	0.01\\
599.47	0.01\\
599.48	0.01\\
599.49	0.01\\
599.5	0.01\\
599.51	0.01\\
599.52	0.01\\
599.53	0.01\\
599.54	0.01\\
599.55	0.01\\
599.56	0.01\\
599.57	0.01\\
599.58	0.01\\
599.59	0.01\\
599.6	0.01\\
599.61	0.01\\
599.62	0.01\\
599.63	0.01\\
599.64	0.01\\
599.65	0.01\\
599.66	0.01\\
599.67	0.01\\
599.68	0.01\\
599.69	0.01\\
599.7	0.01\\
599.71	0.01\\
599.72	0.01\\
599.73	0.01\\
599.74	0.01\\
599.75	0.01\\
599.76	0.01\\
599.77	0.01\\
599.78	0.01\\
599.79	0.01\\
599.8	0.01\\
599.81	0.01\\
599.82	0.01\\
599.83	0.01\\
599.84	0.01\\
599.85	0.01\\
599.86	0.01\\
599.87	0.01\\
599.88	0.01\\
599.89	0.01\\
599.9	0.01\\
599.91	0.01\\
599.92	0.01\\
599.93	0.01\\
599.94	0.01\\
599.95	0.01\\
599.96	0.01\\
599.97	0.01\\
599.98	0.01\\
599.99	0.01\\
600	0.01\\
};
\addplot [color=mycolor10,solid,forget plot]
  table[row sep=crcr]{%
0.01	0.00759711428563548\\
1.01	0.00759711430341606\\
2.01	0.00759711432157064\\
3.01	0.00759711434010715\\
4.01	0.00759711435903364\\
5.01	0.00759711437835844\\
6.01	0.00759711439808997\\
7.01	0.00759711441823685\\
8.01	0.00759711443880789\\
9.01	0.00759711445981209\\
10.01	0.00759711448125864\\
11.01	0.00759711450315694\\
12.01	0.00759711452551657\\
13.01	0.00759711454834734\\
14.01	0.00759711457165924\\
15.01	0.0075971145954625\\
16.01	0.00759711461976756\\
17.01	0.00759711464458503\\
18.01	0.00759711466992591\\
19.01	0.00759711469580127\\
20.01	0.00759711472222247\\
21.01	0.00759711474920114\\
22.01	0.00759711477674912\\
23.01	0.00759711480487854\\
24.01	0.00759711483360177\\
25.01	0.00759711486293148\\
26.01	0.00759711489288056\\
27.01	0.00759711492346218\\
28.01	0.00759711495468987\\
29.01	0.00759711498657738\\
30.01	0.00759711501913875\\
31.01	0.00759711505238838\\
32.01	0.00759711508634094\\
33.01	0.00759711512101142\\
34.01	0.00759711515641514\\
35.01	0.00759711519256775\\
36.01	0.00759711522948524\\
37.01	0.00759711526718395\\
38.01	0.00759711530568055\\
39.01	0.00759711534499211\\
40.01	0.00759711538513603\\
41.01	0.00759711542613009\\
42.01	0.00759711546799251\\
43.01	0.00759711551074182\\
44.01	0.00759711555439702\\
45.01	0.00759711559897748\\
46.01	0.00759711564450302\\
47.01	0.00759711569099386\\
48.01	0.0075971157384707\\
49.01	0.00759711578695462\\
50.01	0.00759711583646721\\
51.01	0.00759711588703054\\
52.01	0.00759711593866713\\
53.01	0.00759711599140001\\
54.01	0.00759711604525268\\
55.01	0.00759711610024916\\
56.01	0.00759711615641406\\
57.01	0.00759711621377239\\
58.01	0.00759711627234983\\
59.01	0.00759711633217256\\
60.01	0.00759711639326734\\
61.01	0.00759711645566151\\
62.01	0.00759711651938302\\
63.01	0.00759711658446041\\
64.01	0.00759711665092289\\
65.01	0.00759711671880021\\
66.01	0.00759711678812288\\
67.01	0.00759711685892201\\
68.01	0.0075971169312294\\
69.01	0.00759711700507756\\
70.01	0.00759711708049972\\
71.01	0.00759711715752978\\
72.01	0.00759711723620247\\
73.01	0.00759711731655321\\
74.01	0.00759711739861821\\
75.01	0.0075971174824345\\
76.01	0.00759711756803991\\
77.01	0.00759711765547308\\
78.01	0.00759711774477351\\
79.01	0.00759711783598156\\
80.01	0.00759711792913853\\
81.01	0.00759711802428656\\
82.01	0.0075971181214687\\
83.01	0.00759711822072907\\
84.01	0.00759711832211264\\
85.01	0.00759711842566537\\
86.01	0.0075971185314343\\
87.01	0.00759711863946749\\
88.01	0.00759711874981403\\
89.01	0.0075971188625241\\
90.01	0.007597118977649\\
91.01	0.00759711909524118\\
92.01	0.00759711921535421\\
93.01	0.00759711933804283\\
94.01	0.00759711946336305\\
95.01	0.00759711959137209\\
96.01	0.0075971197221284\\
97.01	0.00759711985569178\\
98.01	0.00759711999212334\\
99.01	0.00759712013148551\\
100.01	0.00759712027384211\\
101.01	0.00759712041925845\\
102.01	0.00759712056780117\\
103.01	0.00759712071953847\\
104.01	0.00759712087454004\\
105.01	0.00759712103287714\\
106.01	0.00759712119462256\\
107.01	0.00759712135985078\\
108.01	0.00759712152863786\\
109.01	0.00759712170106162\\
110.01	0.00759712187720156\\
111.01	0.007597122057139\\
112.01	0.00759712224095703\\
113.01	0.00759712242874062\\
114.01	0.00759712262057662\\
115.01	0.0075971228165538\\
116.01	0.00759712301676297\\
117.01	0.00759712322129691\\
118.01	0.00759712343025052\\
119.01	0.00759712364372077\\
120.01	0.00759712386180685\\
121.01	0.00759712408461014\\
122.01	0.00759712431223429\\
123.01	0.00759712454478528\\
124.01	0.00759712478237151\\
125.01	0.0075971250251037\\
126.01	0.00759712527309516\\
127.01	0.00759712552646171\\
128.01	0.00759712578532174\\
129.01	0.00759712604979632\\
130.01	0.00759712632000929\\
131.01	0.00759712659608719\\
132.01	0.00759712687815947\\
133.01	0.00759712716635846\\
134.01	0.00759712746081951\\
135.01	0.00759712776168096\\
136.01	0.00759712806908432\\
137.01	0.00759712838317429\\
138.01	0.0075971287040988\\
139.01	0.00759712903200914\\
140.01	0.00759712936706004\\
141.01	0.00759712970940968\\
142.01	0.00759713005921985\\
143.01	0.00759713041665599\\
144.01	0.00759713078188731\\
145.01	0.00759713115508681\\
146.01	0.0075971315364314\\
147.01	0.00759713192610207\\
148.01	0.00759713232428385\\
149.01	0.007597132731166\\
150.01	0.00759713314694205\\
151.01	0.00759713357180995\\
152.01	0.00759713400597214\\
153.01	0.00759713444963565\\
154.01	0.00759713490301221\\
155.01	0.00759713536631841\\
156.01	0.00759713583977574\\
157.01	0.00759713632361072\\
158.01	0.00759713681805505\\
159.01	0.00759713732334571\\
160.01	0.0075971378397251\\
161.01	0.00759713836744111\\
162.01	0.0075971389067473\\
163.01	0.00759713945790305\\
164.01	0.00759714002117363\\
165.01	0.00759714059683038\\
166.01	0.00759714118515087\\
167.01	0.00759714178641895\\
168.01	0.00759714240092506\\
169.01	0.00759714302896624\\
170.01	0.0075971436708463\\
171.01	0.00759714432687604\\
172.01	0.00759714499737342\\
173.01	0.00759714568266363\\
174.01	0.00759714638307934\\
175.01	0.00759714709896089\\
176.01	0.00759714783065634\\
177.01	0.00759714857852186\\
178.01	0.00759714934292172\\
179.01	0.00759715012422858\\
180.01	0.00759715092282368\\
181.01	0.00759715173909701\\
182.01	0.00759715257344754\\
183.01	0.00759715342628343\\
184.01	0.00759715429802221\\
185.01	0.00759715518909107\\
186.01	0.00759715609992698\\
187.01	0.00759715703097702\\
188.01	0.00759715798269857\\
189.01	0.00759715895555955\\
190.01	0.00759715995003865\\
191.01	0.00759716096662565\\
192.01	0.0075971620058216\\
193.01	0.00759716306813912\\
194.01	0.00759716415410263\\
195.01	0.00759716526424875\\
196.01	0.0075971663991264\\
197.01	0.00759716755929726\\
198.01	0.00759716874533591\\
199.01	0.00759716995783028\\
200.01	0.00759717119738185\\
201.01	0.00759717246460604\\
202.01	0.00759717376013248\\
203.01	0.00759717508460537\\
204.01	0.00759717643868381\\
205.01	0.00759717782304216\\
206.01	0.00759717923837037\\
207.01	0.00759718068537439\\
208.01	0.00759718216477649\\
209.01	0.00759718367731565\\
210.01	0.00759718522374796\\
211.01	0.00759718680484706\\
212.01	0.00759718842140444\\
213.01	0.00759719007422999\\
214.01	0.00759719176415231\\
215.01	0.00759719349201915\\
216.01	0.00759719525869804\\
217.01	0.00759719706507643\\
218.01	0.00759719891206241\\
219.01	0.00759720080058515\\
220.01	0.00759720273159523\\
221.01	0.00759720470606533\\
222.01	0.00759720672499062\\
223.01	0.00759720878938931\\
224.01	0.00759721090030323\\
225.01	0.0075972130587983\\
226.01	0.00759721526596511\\
227.01	0.00759721752291954\\
228.01	0.00759721983080323\\
229.01	0.00759722219078431\\
230.01	0.0075972246040579\\
231.01	0.00759722707184678\\
232.01	0.00759722959540205\\
233.01	0.00759723217600367\\
234.01	0.00759723481496128\\
235.01	0.00759723751361473\\
236.01	0.0075972402733349\\
237.01	0.00759724309552432\\
238.01	0.0075972459816179\\
239.01	0.00759724893308371\\
240.01	0.00759725195142374\\
241.01	0.00759725503817462\\
242.01	0.00759725819490847\\
243.01	0.00759726142323365\\
244.01	0.00759726472479561\\
245.01	0.00759726810127777\\
246.01	0.00759727155440234\\
247.01	0.00759727508593118\\
248.01	0.00759727869766676\\
249.01	0.00759728239145301\\
250.01	0.00759728616917633\\
251.01	0.00759729003276654\\
252.01	0.00759729398419776\\
253.01	0.00759729802548958\\
254.01	0.00759730215870792\\
255.01	0.00759730638596622\\
256.01	0.00759731070942638\\
257.01	0.00759731513129995\\
258.01	0.00759731965384923\\
259.01	0.00759732427938837\\
260.01	0.00759732901028455\\
261.01	0.00759733384895924\\
262.01	0.00759733879788932\\
263.01	0.00759734385960838\\
264.01	0.00759734903670801\\
265.01	0.00759735433183906\\
266.01	0.00759735974771301\\
267.01	0.0075973652871033\\
268.01	0.00759737095284672\\
269.01	0.00759737674784489\\
270.01	0.00759738267506566\\
271.01	0.00759738873754458\\
272.01	0.00759739493838654\\
273.01	0.00759740128076716\\
274.01	0.00759740776793452\\
275.01	0.00759741440321073\\
276.01	0.00759742118999361\\
277.01	0.00759742813175841\\
278.01	0.00759743523205952\\
279.01	0.00759744249453233\\
280.01	0.00759744992289498\\
281.01	0.00759745752095033\\
282.01	0.00759746529258779\\
283.01	0.00759747324178531\\
284.01	0.0075974813726115\\
285.01	0.00759748968922757\\
286.01	0.00759749819588948\\
287.01	0.00759750689695017\\
288.01	0.00759751579686174\\
289.01	0.00759752490017776\\
290.01	0.00759753421155559\\
291.01	0.0075975437357588\\
292.01	0.00759755347765966\\
293.01	0.00759756344224168\\
294.01	0.00759757363460212\\
295.01	0.00759758405995484\\
296.01	0.00759759472363292\\
297.01	0.00759760563109147\\
298.01	0.00759761678791065\\
299.01	0.00759762819979853\\
300.01	0.0075976398725943\\
301.01	0.00759765181227124\\
302.01	0.00759766402494021\\
303.01	0.00759767651685277\\
304.01	0.00759768929440477\\
305.01	0.00759770236413984\\
306.01	0.00759771573275306\\
307.01	0.00759772940709473\\
308.01	0.00759774339417425\\
309.01	0.00759775770116406\\
310.01	0.00759777233540389\\
311.01	0.00759778730440486\\
312.01	0.00759780261585395\\
313.01	0.00759781827761858\\
314.01	0.00759783429775108\\
315.01	0.00759785068449371\\
316.01	0.00759786744628348\\
317.01	0.00759788459175736\\
318.01	0.00759790212975755\\
319.01	0.00759792006933688\\
320.01	0.0075979384197645\\
321.01	0.00759795719053168\\
322.01	0.00759797639135782\\
323.01	0.00759799603219666\\
324.01	0.0075980161232426\\
325.01	0.00759803667493738\\
326.01	0.00759805769797686\\
327.01	0.00759807920331804\\
328.01	0.00759810120218639\\
329.01	0.00759812370608319\\
330.01	0.00759814672679338\\
331.01	0.0075981702763935\\
332.01	0.00759819436725983\\
333.01	0.00759821901207697\\
334.01	0.00759824422384649\\
335.01	0.00759827001589595\\
336.01	0.00759829640188821\\
337.01	0.00759832339583089\\
338.01	0.00759835101208636\\
339.01	0.0075983792653817\\
340.01	0.00759840817081923\\
341.01	0.00759843774388726\\
342.01	0.00759846800047101\\
343.01	0.0075984989568641\\
344.01	0.0075985306297802\\
345.01	0.00759856303636495\\
346.01	0.00759859619420834\\
347.01	0.00759863012135745\\
348.01	0.00759866483632943\\
349.01	0.00759870035812481\\
350.01	0.00759873670624131\\
351.01	0.00759877390068791\\
352.01	0.0075988119619993\\
353.01	0.00759885091125074\\
354.01	0.00759889077007328\\
355.01	0.00759893156066941\\
356.01	0.00759897330582899\\
357.01	0.00759901602894575\\
358.01	0.00759905975403411\\
359.01	0.00759910450574636\\
360.01	0.00759915030939039\\
361.01	0.00759919719094773\\
362.01	0.00759924517709206\\
363.01	0.00759929429520817\\
364.01	0.00759934457341124\\
365.01	0.00759939604056666\\
366.01	0.00759944872631018\\
367.01	0.00759950266106848\\
368.01	0.00759955787608007\\
369.01	0.00759961440341674\\
370.01	0.00759967227600502\\
371.01	0.00759973152764839\\
372.01	0.00759979219304943\\
373.01	0.00759985430783259\\
374.01	0.00759991790856695\\
375.01	0.00759998303278911\\
376.01	0.0076000497190231\\
377.01	0.00760011800678446\\
378.01	0.00760018793659381\\
379.01	0.00760025955021629\\
380.01	0.00760033289043611\\
381.01	0.0076004080011099\\
382.01	0.00760048492729109\\
383.01	0.00760056371526734\\
384.01	0.00760064441259949\\
385.01	0.00760072706816217\\
386.01	0.00760081173218604\\
387.01	0.00760089845630183\\
388.01	0.00760098729358639\\
389.01	0.00760107829861057\\
390.01	0.0076011715274894\\
391.01	0.00760126703793447\\
392.01	0.00760136488930877\\
393.01	0.00760146514268394\\
394.01	0.00760156786090044\\
395.01	0.00760167310863041\\
396.01	0.00760178095244376\\
397.01	0.00760189146087739\\
398.01	0.00760200470450799\\
399.01	0.00760212075602848\\
400.01	0.00760223969032841\\
401.01	0.00760236158457857\\
402.01	0.00760248651832017\\
403.01	0.00760261457355861\\
404.01	0.00760274583486264\\
405.01	0.0076028803894688\\
406.01	0.00760301832739173\\
407.01	0.00760315974154093\\
408.01	0.00760330472784394\\
409.01	0.00760345338537697\\
410.01	0.0076036058165031\\
411.01	0.00760376212701876\\
412.01	0.00760392242630911\\
413.01	0.007604086827513\\
414.01	0.00760425544769812\\
415.01	0.00760442840804718\\
416.01	0.00760460583405607\\
417.01	0.00760478785574485\\
418.01	0.00760497460788239\\
419.01	0.0076051662302263\\
420.01	0.00760536286777884\\
421.01	0.00760556467106035\\
422.01	0.00760577179640201\\
423.01	0.00760598440625889\\
424.01	0.00760620266954593\\
425.01	0.0076064267619982\\
426.01	0.00760665686655819\\
427.01	0.0076068931737924\\
428.01	0.00760713588234016\\
429.01	0.00760738519939784\\
430.01	0.00760764134124227\\
431.01	0.00760790453379695\\
432.01	0.00760817501324634\\
433.01	0.00760845302670286\\
434.01	0.00760873883293303\\
435.01	0.00760903270314941\\
436.01	0.00760933492187631\\
437.01	0.00760964578789821\\
438.01	0.00760996561530154\\
439.01	0.00761029473462189\\
440.01	0.00761063349411057\\
441.01	0.00761098226113716\\
442.01	0.00761134142374703\\
443.01	0.00761171139239634\\
444.01	0.00761209260189073\\
445.01	0.00761248551355876\\
446.01	0.00761289061769659\\
447.01	0.00761330843632716\\
448.01	0.00761373952632497\\
449.01	0.00761418448296747\\
450.01	0.00761464394398552\\
451.01	0.00761511859419933\\
452.01	0.00761560917084363\\
453.01	0.00761611646970582\\
454.01	0.00761664135222632\\
455.01	0.00761718475374046\\
456.01	0.00761774769307792\\
457.01	0.00761833128377989\\
458.01	0.00761893674724141\\
459.01	0.00761956542807913\\
460.01	0.00762021881145276\\
461.01	0.00762089853711053\\
462.01	0.0076216063823837\\
463.01	0.00762234426991841\\
464.01	0.00762311478426318\\
465.01	0.00762392195354493\\
466.01	0.00762478319873316\\
467.01	0.00762593636630352\\
468.01	0.0076276003578341\\
469.01	0.00762931595635995\\
470.01	0.00763109162793824\\
471.01	0.007632864466257\\
472.01	0.0076346798115011\\
473.01	0.00763655067009834\\
474.01	0.00763851016701323\\
475.01	0.00764061064999437\\
476.01	0.00764277196042097\\
477.01	0.0076449900776459\\
478.01	0.00764726681690006\\
479.01	0.0076496040747383\\
480.01	0.00765200383421257\\
481.01	0.00765446817046258\\
482.01	0.00765699925676254\\
483.01	0.0076595993710684\\
484.01	0.00766227090311199\\
485.01	0.00766501636209198\\
486.01	0.00766783838500193\\
487.01	0.00767073974552987\\
488.01	0.0076737233627159\\
489.01	0.00767679230576994\\
490.01	0.00767994980575148\\
491.01	0.00768319928314602\\
492.01	0.00768654437221981\\
493.01	0.00768998898310824\\
494.01	0.00769353717777799\\
495.01	0.00769719325761285\\
496.01	0.00770096178927012\\
497.01	0.00770484763228593\\
498.01	0.00770885599703586\\
499.01	0.00771299269106102\\
500.01	0.00771726576956622\\
501.01	0.00772169639195388\\
502.01	0.00772633130952657\\
503.01	0.00773080003085001\\
504.01	0.00773537642440651\\
505.01	0.00774011049938724\\
506.01	0.00774501125892881\\
507.01	0.00775008859874314\\
508.01	0.00775535343463854\\
509.01	0.00776081785262583\\
510.01	0.00776649528661062\\
511.01	0.00777240073250528\\
512.01	0.00777855103115577\\
513.01	0.00778496544275355\\
514.01	0.00779166849852973\\
515.01	0.0077987164246254\\
516.01	0.0078065580776282\\
517.01	0.00781948919250816\\
518.01	0.00783469478529009\\
519.01	0.00785036114826689\\
520.01	0.0078664740808604\\
521.01	0.00788317313806947\\
522.01	0.00790009837544658\\
523.01	0.00791665118189264\\
524.01	0.00793370157798339\\
525.01	0.0079512706726707\\
526.01	0.00796938087882422\\
527.01	0.00798805601483179\\
528.01	0.00800732138455739\\
529.01	0.00802720396433126\\
530.01	0.00804773282009358\\
531.01	0.0080689388081184\\
532.01	0.00809085599779049\\
533.01	0.00811353415102963\\
534.01	0.00813718324366485\\
535.01	0.0081626618212072\\
536.01	0.00818931436024871\\
537.01	0.00821683851746443\\
538.01	0.00824516929339367\\
539.01	0.00827456816479533\\
540.01	0.00830541471736344\\
541.01	0.0083380882861966\\
542.01	0.00836680870452482\\
543.01	0.00839602967388001\\
544.01	0.00842467257447497\\
545.01	0.00845327850708192\\
546.01	0.00848323081843706\\
547.01	0.00851471674367666\\
548.01	0.00854877081933403\\
549.01	0.00860354157848882\\
550.01	0.00868426795184223\\
551.01	0.00876598105733376\\
552.01	0.00884943829693079\\
553.01	0.00893472459669881\\
554.01	0.00902193300020245\\
555.01	0.00911116602884714\\
556.01	0.00920253879627661\\
557.01	0.00929619350666944\\
558.01	0.00939237967236016\\
559.01	0.00949163304911767\\
560.01	0.00959232145567597\\
561.01	0.00969689070041613\\
562.01	0.009793005055321\\
563.01	0.00988051041000003\\
564.01	0.00996671501144665\\
565.01	0.01\\
566.01	0.01\\
567.01	0.01\\
568.01	0.01\\
569.01	0.01\\
570.01	0.01\\
571.01	0.01\\
572.01	0.01\\
573.01	0.01\\
574.01	0.01\\
575.01	0.01\\
576.01	0.01\\
577.01	0.01\\
578.01	0.01\\
579.01	0.01\\
580.01	0.01\\
581.01	0.01\\
582.01	0.01\\
583.01	0.01\\
584.01	0.01\\
585.01	0.01\\
586.01	0.01\\
587.01	0.01\\
588.01	0.01\\
589.01	0.01\\
590.01	0.01\\
591.01	0.01\\
592.01	0.01\\
593.01	0.01\\
594.01	0.01\\
595.01	0.01\\
596.01	0.01\\
597.01	0.01\\
598.01	0.01\\
599.01	0.01\\
599.02	0.01\\
599.03	0.01\\
599.04	0.01\\
599.05	0.01\\
599.06	0.01\\
599.07	0.01\\
599.08	0.01\\
599.09	0.01\\
599.1	0.01\\
599.11	0.01\\
599.12	0.01\\
599.13	0.01\\
599.14	0.01\\
599.15	0.01\\
599.16	0.01\\
599.17	0.01\\
599.18	0.01\\
599.19	0.01\\
599.2	0.01\\
599.21	0.01\\
599.22	0.01\\
599.23	0.01\\
599.24	0.01\\
599.25	0.01\\
599.26	0.01\\
599.27	0.01\\
599.28	0.01\\
599.29	0.01\\
599.3	0.01\\
599.31	0.01\\
599.32	0.01\\
599.33	0.01\\
599.34	0.01\\
599.35	0.01\\
599.36	0.01\\
599.37	0.01\\
599.38	0.01\\
599.39	0.01\\
599.4	0.01\\
599.41	0.01\\
599.42	0.01\\
599.43	0.01\\
599.44	0.01\\
599.45	0.01\\
599.46	0.01\\
599.47	0.01\\
599.48	0.01\\
599.49	0.01\\
599.5	0.01\\
599.51	0.01\\
599.52	0.01\\
599.53	0.01\\
599.54	0.01\\
599.55	0.01\\
599.56	0.01\\
599.57	0.01\\
599.58	0.01\\
599.59	0.01\\
599.6	0.01\\
599.61	0.01\\
599.62	0.01\\
599.63	0.01\\
599.64	0.01\\
599.65	0.01\\
599.66	0.01\\
599.67	0.01\\
599.68	0.01\\
599.69	0.01\\
599.7	0.01\\
599.71	0.01\\
599.72	0.01\\
599.73	0.01\\
599.74	0.01\\
599.75	0.01\\
599.76	0.01\\
599.77	0.01\\
599.78	0.01\\
599.79	0.01\\
599.8	0.01\\
599.81	0.01\\
599.82	0.01\\
599.83	0.01\\
599.84	0.01\\
599.85	0.01\\
599.86	0.01\\
599.87	0.01\\
599.88	0.01\\
599.89	0.01\\
599.9	0.01\\
599.91	0.01\\
599.92	0.01\\
599.93	0.01\\
599.94	0.01\\
599.95	0.01\\
599.96	0.01\\
599.97	0.01\\
599.98	0.01\\
599.99	0.01\\
600	0.01\\
};
\addplot [color=mycolor11,solid,forget plot]
  table[row sep=crcr]{%
0.01	0.00622874336967211\\
1.01	0.00622874337076519\\
2.01	0.00622874337188126\\
3.01	0.00622874337302081\\
4.01	0.00622874337418433\\
5.01	0.00622874337537234\\
6.01	0.00622874337658536\\
7.01	0.0062287433778239\\
8.01	0.00622874337908854\\
9.01	0.00622874338037979\\
10.01	0.00622874338169825\\
11.01	0.00622874338304448\\
12.01	0.00622874338441908\\
13.01	0.00622874338582262\\
14.01	0.00622874338725576\\
15.01	0.0062287433887191\\
16.01	0.00622874339021331\\
17.01	0.00622874339173904\\
18.01	0.00622874339329691\\
19.01	0.00622874339488765\\
20.01	0.00622874339651194\\
21.01	0.00622874339817051\\
22.01	0.00622874339986408\\
23.01	0.00622874340159342\\
24.01	0.00622874340335925\\
25.01	0.00622874340516236\\
26.01	0.00622874340700356\\
27.01	0.00622874340888365\\
28.01	0.00622874341080346\\
29.01	0.00622874341276384\\
30.01	0.00622874341476564\\
31.01	0.00622874341680977\\
32.01	0.0062287434188971\\
33.01	0.00622874342102859\\
34.01	0.00622874342320516\\
35.01	0.00622874342542776\\
36.01	0.0062287434276974\\
37.01	0.00622874343001506\\
38.01	0.00622874343238178\\
39.01	0.0062287434347986\\
40.01	0.00622874343726661\\
41.01	0.00622874343978689\\
42.01	0.00622874344236055\\
43.01	0.00622874344498875\\
44.01	0.00622874344767263\\
45.01	0.00622874345041341\\
46.01	0.00622874345321229\\
47.01	0.00622874345607054\\
48.01	0.00622874345898939\\
49.01	0.00622874346197016\\
50.01	0.00622874346501422\\
51.01	0.00622874346812285\\
52.01	0.00622874347129746\\
53.01	0.00622874347453949\\
54.01	0.00622874347785037\\
55.01	0.00622874348123158\\
56.01	0.00622874348468462\\
57.01	0.00622874348821105\\
58.01	0.00622874349181243\\
59.01	0.00622874349549038\\
60.01	0.00622874349924654\\
61.01	0.00622874350308261\\
62.01	0.00622874350700028\\
63.01	0.00622874351100132\\
64.01	0.00622874351508752\\
65.01	0.00622874351926072\\
66.01	0.00622874352352279\\
67.01	0.00622874352787564\\
68.01	0.00622874353232123\\
69.01	0.00622874353686158\\
70.01	0.00622874354149868\\
71.01	0.00622874354623466\\
72.01	0.00622874355107163\\
73.01	0.00622874355601179\\
74.01	0.00622874356105736\\
75.01	0.0062287435662106\\
76.01	0.00622874357147386\\
77.01	0.00622874357684951\\
78.01	0.00622874358233997\\
79.01	0.00622874358794772\\
80.01	0.00622874359367532\\
81.01	0.00622874359952534\\
82.01	0.00622874360550048\\
83.01	0.00622874361160334\\
84.01	0.00622874361783679\\
85.01	0.00622874362420362\\
86.01	0.00622874363070671\\
87.01	0.00622874363734905\\
88.01	0.00622874364413363\\
89.01	0.00622874365106355\\
90.01	0.00622874365814195\\
91.01	0.00622874366537207\\
92.01	0.0062287436727572\\
93.01	0.00622874368030072\\
94.01	0.00622874368800604\\
95.01	0.00622874369587671\\
96.01	0.00622874370391631\\
97.01	0.00622874371212854\\
98.01	0.00622874372051712\\
99.01	0.00622874372908591\\
100.01	0.00622874373783884\\
101.01	0.00622874374677993\\
102.01	0.00622874375591327\\
103.01	0.00622874376524304\\
104.01	0.00622874377477355\\
105.01	0.00622874378450918\\
106.01	0.0062287437944544\\
107.01	0.00622874380461378\\
108.01	0.00622874381499201\\
109.01	0.00622874382559389\\
110.01	0.00622874383642428\\
111.01	0.0062287438474882\\
112.01	0.00622874385879076\\
113.01	0.00622874387033717\\
114.01	0.0062287438821328\\
115.01	0.00622874389418309\\
116.01	0.00622874390649363\\
117.01	0.00622874391907011\\
118.01	0.0062287439319184\\
119.01	0.00622874394504444\\
120.01	0.00622874395845434\\
121.01	0.00622874397215433\\
122.01	0.00622874398615079\\
123.01	0.00622874400045023\\
124.01	0.00622874401505932\\
125.01	0.0062287440299849\\
126.01	0.00622874404523388\\
127.01	0.00622874406081344\\
128.01	0.00622874407673082\\
129.01	0.00622874409299351\\
130.01	0.00622874410960908\\
131.01	0.00622874412658535\\
132.01	0.00622874414393026\\
133.01	0.00622874416165197\\
134.01	0.00622874417975879\\
135.01	0.0062287441982592\\
136.01	0.00622874421716195\\
137.01	0.00622874423647596\\
138.01	0.00622874425621028\\
139.01	0.00622874427637422\\
140.01	0.00622874429697733\\
141.01	0.0062287443180293\\
142.01	0.00622874433954011\\
143.01	0.00622874436151992\\
144.01	0.00622874438397915\\
145.01	0.00622874440692843\\
146.01	0.00622874443037867\\
147.01	0.00622874445434096\\
148.01	0.00622874447882671\\
149.01	0.00622874450384755\\
150.01	0.00622874452941541\\
151.01	0.00622874455554244\\
152.01	0.00622874458224109\\
153.01	0.00622874460952411\\
154.01	0.00622874463740453\\
155.01	0.00622874466589564\\
156.01	0.0062287446950111\\
157.01	0.00622874472476483\\
158.01	0.00622874475517108\\
159.01	0.00622874478624445\\
160.01	0.0062287448179998\\
161.01	0.00622874485045245\\
162.01	0.00622874488361795\\
163.01	0.00622874491751226\\
164.01	0.0062287449521517\\
165.01	0.00622874498755297\\
166.01	0.00622874502373314\\
167.01	0.00622874506070968\\
168.01	0.00622874509850044\\
169.01	0.00622874513712371\\
170.01	0.00622874517659817\\
171.01	0.00622874521694295\\
172.01	0.00622874525817759\\
173.01	0.00622874530032211\\
174.01	0.00622874534339697\\
175.01	0.00622874538742312\\
176.01	0.00622874543242195\\
177.01	0.00622874547841538\\
178.01	0.00622874552542584\\
179.01	0.00622874557347622\\
180.01	0.00622874562258999\\
181.01	0.00622874567279114\\
182.01	0.00622874572410424\\
183.01	0.00622874577655436\\
184.01	0.0062287458301672\\
185.01	0.00622874588496906\\
186.01	0.00622874594098676\\
187.01	0.00622874599824788\\
188.01	0.00622874605678054\\
189.01	0.0062287461166135\\
190.01	0.00622874617777622\\
191.01	0.00622874624029885\\
192.01	0.00622874630421219\\
193.01	0.00622874636954777\\
194.01	0.0062287464363379\\
195.01	0.00622874650461555\\
196.01	0.0062287465744145\\
197.01	0.00622874664576929\\
198.01	0.00622874671871534\\
199.01	0.00622874679328871\\
200.01	0.00622874686952648\\
201.01	0.00622874694746649\\
202.01	0.00622874702714749\\
203.01	0.00622874710860908\\
204.01	0.00622874719189184\\
205.01	0.00622874727703724\\
206.01	0.00622874736408774\\
207.01	0.00622874745308676\\
208.01	0.00622874754407874\\
209.01	0.00622874763710914\\
210.01	0.00622874773222447\\
211.01	0.00622874782947234\\
212.01	0.00622874792890143\\
213.01	0.0062287480305616\\
214.01	0.00622874813450377\\
215.01	0.00622874824078015\\
216.01	0.00622874834944407\\
217.01	0.00622874846055013\\
218.01	0.00622874857415424\\
219.01	0.00622874869031352\\
220.01	0.00622874880908646\\
221.01	0.00622874893053291\\
222.01	0.0062287490547141\\
223.01	0.00622874918169265\\
224.01	0.00622874931153266\\
225.01	0.00622874944429972\\
226.01	0.00622874958006093\\
227.01	0.0062287497188849\\
228.01	0.00622874986084189\\
229.01	0.00622875000600378\\
230.01	0.00622875015444407\\
231.01	0.006228750306238\\
232.01	0.00622875046146251\\
233.01	0.00622875062019639\\
234.01	0.00622875078252015\\
235.01	0.00622875094851629\\
236.01	0.00622875111826908\\
237.01	0.00622875129186482\\
238.01	0.0062287514693918\\
239.01	0.00622875165094032\\
240.01	0.00622875183660278\\
241.01	0.00622875202647372\\
242.01	0.00622875222064985\\
243.01	0.00622875241923009\\
244.01	0.00622875262231571\\
245.01	0.00622875283001024\\
246.01	0.00622875304241962\\
247.01	0.00622875325965224\\
248.01	0.00622875348181899\\
249.01	0.00622875370903329\\
250.01	0.0062287539414112\\
251.01	0.00622875417907139\\
252.01	0.00622875442213535\\
253.01	0.0062287546707273\\
254.01	0.00622875492497431\\
255.01	0.00622875518500639\\
256.01	0.00622875545095655\\
257.01	0.00622875572296079\\
258.01	0.0062287560011583\\
259.01	0.00622875628569144\\
260.01	0.00622875657670578\\
261.01	0.00622875687435032\\
262.01	0.00622875717877736\\
263.01	0.00622875749014281\\
264.01	0.00622875780860603\\
265.01	0.00622875813433009\\
266.01	0.0062287584674818\\
267.01	0.00622875880823172\\
268.01	0.00622875915675435\\
269.01	0.00622875951322819\\
270.01	0.00622875987783575\\
271.01	0.0062287602507638\\
272.01	0.00622876063220325\\
273.01	0.00622876102234947\\
274.01	0.00622876142140221\\
275.01	0.00622876182956587\\
276.01	0.00622876224704936\\
277.01	0.0062287626740665\\
278.01	0.00622876311083586\\
279.01	0.00622876355758107\\
280.01	0.00622876401453081\\
281.01	0.00622876448191896\\
282.01	0.00622876495998476\\
283.01	0.0062287654489729\\
284.01	0.00622876594913358\\
285.01	0.00622876646072274\\
286.01	0.0062287669840022\\
287.01	0.00622876751923962\\
288.01	0.00622876806670885\\
289.01	0.00622876862668995\\
290.01	0.00622876919946939\\
291.01	0.00622876978534014\\
292.01	0.00622877038460187\\
293.01	0.00622877099756108\\
294.01	0.0062287716245313\\
295.01	0.00622877226583321\\
296.01	0.00622877292179477\\
297.01	0.00622877359275155\\
298.01	0.00622877427904675\\
299.01	0.00622877498103145\\
300.01	0.00622877569906479\\
301.01	0.00622877643351417\\
302.01	0.00622877718475544\\
303.01	0.00622877795317313\\
304.01	0.00622877873916063\\
305.01	0.00622877954312041\\
306.01	0.00622878036546429\\
307.01	0.00622878120661364\\
308.01	0.00622878206699957\\
309.01	0.00622878294706329\\
310.01	0.00622878384725626\\
311.01	0.00622878476804051\\
312.01	0.00622878570988889\\
313.01	0.00622878667328535\\
314.01	0.00622878765872523\\
315.01	0.00622878866671555\\
316.01	0.00622878969777532\\
317.01	0.00622879075243587\\
318.01	0.00622879183124113\\
319.01	0.00622879293474808\\
320.01	0.00622879406352692\\
321.01	0.00622879521816158\\
322.01	0.00622879639925002\\
323.01	0.00622879760740464\\
324.01	0.00622879884325266\\
325.01	0.00622880010743651\\
326.01	0.00622880140061428\\
327.01	0.00622880272346016\\
328.01	0.00622880407666484\\
329.01	0.00622880546093602\\
330.01	0.00622880687699887\\
331.01	0.00622880832559649\\
332.01	0.00622880980749047\\
333.01	0.00622881132346137\\
334.01	0.00622881287430931\\
335.01	0.00622881446085444\\
336.01	0.00622881608393754\\
337.01	0.00622881774442072\\
338.01	0.00622881944318785\\
339.01	0.0062288211811453\\
340.01	0.00622882295922254\\
341.01	0.0062288247783728\\
342.01	0.00622882663957379\\
343.01	0.00622882854382834\\
344.01	0.00622883049216513\\
345.01	0.00622883248563947\\
346.01	0.00622883452533406\\
347.01	0.00622883661235968\\
348.01	0.00622883874785609\\
349.01	0.00622884093299282\\
350.01	0.00622884316896999\\
351.01	0.0062288454570192\\
352.01	0.00622884779840442\\
353.01	0.0062288501944229\\
354.01	0.00622885264640609\\
355.01	0.00622885515572062\\
356.01	0.00622885772376928\\
357.01	0.00622886035199203\\
358.01	0.00622886304186699\\
359.01	0.00622886579491157\\
360.01	0.00622886861268352\\
361.01	0.00622887149678202\\
362.01	0.00622887444884886\\
363.01	0.00622887747056955\\
364.01	0.00622888056367452\\
365.01	0.0062288837299404\\
366.01	0.00622888697119115\\
367.01	0.00622889028929936\\
368.01	0.00622889368618758\\
369.01	0.00622889716382952\\
370.01	0.00622890072425154\\
371.01	0.00622890436953377\\
372.01	0.00622890810181167\\
373.01	0.00622891192327731\\
374.01	0.00622891583618078\\
375.01	0.00622891984283135\\
376.01	0.00622892394559809\\
377.01	0.00622892814690864\\
378.01	0.00622893244925577\\
379.01	0.00622893685521343\\
380.01	0.0062289413674131\\
381.01	0.00622894598855356\\
382.01	0.00622895072140726\\
383.01	0.00622895556882267\\
384.01	0.00622896053372661\\
385.01	0.00622896561912678\\
386.01	0.00622897082811439\\
387.01	0.00622897616386689\\
388.01	0.00622898162965073\\
389.01	0.00622898722882441\\
390.01	0.0062289929648415\\
391.01	0.006228998841254\\
392.01	0.00622900486171554\\
393.01	0.00622901102998509\\
394.01	0.00622901734993058\\
395.01	0.00622902382553277\\
396.01	0.00622903046088948\\
397.01	0.00622903726021968\\
398.01	0.00622904422786811\\
399.01	0.00622905136831001\\
400.01	0.006229058686156\\
401.01	0.00622906618615744\\
402.01	0.00622907387321184\\
403.01	0.00622908175236879\\
404.01	0.00622908982883595\\
405.01	0.00622909810798566\\
406.01	0.00622910659536166\\
407.01	0.00622911529668641\\
408.01	0.00622912421786871\\
409.01	0.00622913336501174\\
410.01	0.00622914274442171\\
411.01	0.00622915236261689\\
412.01	0.00622916222633727\\
413.01	0.00622917234255481\\
414.01	0.0062291827184843\\
415.01	0.00622919336159492\\
416.01	0.00622920427962254\\
417.01	0.00622921548058287\\
418.01	0.00622922697278535\\
419.01	0.00622923876484813\\
420.01	0.00622925086571396\\
421.01	0.0062292632846672\\
422.01	0.006229276031352\\
423.01	0.00622928911579191\\
424.01	0.0062293025484106\\
425.01	0.00622931634005455\\
426.01	0.00622933050201698\\
427.01	0.00622934504606398\\
428.01	0.00622935998446241\\
429.01	0.00622937533001021\\
430.01	0.00622939109606901\\
431.01	0.00622940729659966\\
432.01	0.00622942394620052\\
433.01	0.00622944106014928\\
434.01	0.00622945865444845\\
435.01	0.00622947674587496\\
436.01	0.00622949535203458\\
437.01	0.00622951449142147\\
438.01	0.0062295341834836\\
439.01	0.00622955444869505\\
440.01	0.00622957530863579\\
441.01	0.00622959678608016\\
442.01	0.00622961890509521\\
443.01	0.00622964169115047\\
444.01	0.00622966517124064\\
445.01	0.00622968937402342\\
446.01	0.00622971432997473\\
447.01	0.0062297400715641\\
448.01	0.00622976663345353\\
449.01	0.00622979405272399\\
450.01	0.00622982236913393\\
451.01	0.00622985162541575\\
452.01	0.00622988186761656\\
453.01	0.00622991314549177\\
454.01	0.00622994551296099\\
455.01	0.00622997902863769\\
456.01	0.00623001375644706\\
457.01	0.0062300497663487\\
458.01	0.00623008713518149\\
459.01	0.0062301259476367\\
460.01	0.00623016629726389\\
461.01	0.00623020828696017\\
462.01	0.00623025202854534\\
463.01	0.00623029765571453\\
464.01	0.00623034538591046\\
465.01	0.00623039585549172\\
466.01	0.00623045288955951\\
467.01	0.00623053630881283\\
468.01	0.00623063889289436\\
469.01	0.00623074647477932\\
470.01	0.00623087880683824\\
471.01	0.00623111863693082\\
472.01	0.00623137485948774\\
473.01	0.00623163880954006\\
474.01	0.00623191347255253\\
475.01	0.00623219912330979\\
476.01	0.00623249250456114\\
477.01	0.00623279382710131\\
478.01	0.00623310339606975\\
479.01	0.00623342153728351\\
480.01	0.0062337485993486\\
481.01	0.0062340849560285\\
482.01	0.00623443100890397\\
483.01	0.00623478719036257\\
484.01	0.00623515396696156\\
485.01	0.00623553184321373\\
486.01	0.00623592136584611\\
487.01	0.00623632312856245\\
488.01	0.00623673777723619\\
489.01	0.00623716601546134\\
490.01	0.00623760861225348\\
491.01	0.00623806641025245\\
492.01	0.00623854033874851\\
493.01	0.00623903142030608\\
494.01	0.00623954077702533\\
495.01	0.00624006964774182\\
496.01	0.00624061940331291\\
497.01	0.00624119156604566\\
498.01	0.00624178784668994\\
499.01	0.00624241029588527\\
500.01	0.00624306230901286\\
501.01	0.00624375695163478\\
502.01	0.00624461716589386\\
503.01	0.00624611619553652\\
504.01	0.00624771319754122\\
505.01	0.00624936308412463\\
506.01	0.00625106855843081\\
507.01	0.00625283256118287\\
508.01	0.00625465830308046\\
509.01	0.00625654930303459\\
510.01	0.00625850943368729\\
511.01	0.00626054297740323\\
512.01	0.00626265470710641\\
513.01	0.00626485008889032\\
514.01	0.00626713634837834\\
515.01	0.0062695304191542\\
516.01	0.00627211819892404\\
517.01	0.00627509256745018\\
518.01	0.00627823160609473\\
519.01	0.00628150390118513\\
520.01	0.00628492990926515\\
521.01	0.00628862902389006\\
522.01	0.00629387313213118\\
523.01	0.00630060424757553\\
524.01	0.00630754953469232\\
525.01	0.00631472012940907\\
526.01	0.00632212810859392\\
527.01	0.00632978659833269\\
528.01	0.00633770989679532\\
529.01	0.00634591363563067\\
530.01	0.00635441494768246\\
531.01	0.00636323266963891\\
532.01	0.00637238791548698\\
533.01	0.00638190690078944\\
534.01	0.00639183545566685\\
535.01	0.00640223102191707\\
536.01	0.0064130976871169\\
537.01	0.00642448974005286\\
538.01	0.00643606523913887\\
539.01	0.00644817761602978\\
540.01	0.0064611421623985\\
541.01	0.00647792950210211\\
542.01	0.00650441513295767\\
543.01	0.00653149862206026\\
544.01	0.00655873357067719\\
545.01	0.00658686779613701\\
546.01	0.00661598895446187\\
547.01	0.0066461769705681\\
548.01	0.00667773629947955\\
549.01	0.00671207064339874\\
550.01	0.00674787100487959\\
551.01	0.00678289469210539\\
552.01	0.00681912610093263\\
553.01	0.00685665739171707\\
554.01	0.00689559221571758\\
555.01	0.00693604783846996\\
556.01	0.00697815890035221\\
557.01	0.00702209126971914\\
558.01	0.00706812629165005\\
559.01	0.00711711053467372\\
560.01	0.00716820906240298\\
561.01	0.0072185056339763\\
562.01	0.00730871849609079\\
563.01	0.00741524301621832\\
564.01	0.00752344920588999\\
565.01	0.0076313687639263\\
566.01	0.00774115467975347\\
567.01	0.00785363528701536\\
568.01	0.00796898385122909\\
569.01	0.00808739615598924\\
570.01	0.00820910690812936\\
571.01	0.00833453511991283\\
572.01	0.00846526876419162\\
573.01	0.00860519158723727\\
574.01	0.0087215661705536\\
575.01	0.00883762282490467\\
576.01	0.00895594286371285\\
577.01	0.00907652599155817\\
578.01	0.00919953250302053\\
579.01	0.00932501892119195\\
580.01	0.00945299567345381\\
581.01	0.00958345369111326\\
582.01	0.00971634636051373\\
583.01	0.00985132789365488\\
584.01	0.00997897979146759\\
585.01	0.01\\
586.01	0.01\\
587.01	0.01\\
588.01	0.01\\
589.01	0.01\\
590.01	0.01\\
591.01	0.01\\
592.01	0.01\\
593.01	0.01\\
594.01	0.01\\
595.01	0.01\\
596.01	0.01\\
597.01	0.01\\
598.01	0.01\\
599.01	0.01\\
599.02	0.01\\
599.03	0.01\\
599.04	0.01\\
599.05	0.01\\
599.06	0.01\\
599.07	0.01\\
599.08	0.01\\
599.09	0.01\\
599.1	0.01\\
599.11	0.01\\
599.12	0.01\\
599.13	0.01\\
599.14	0.01\\
599.15	0.01\\
599.16	0.01\\
599.17	0.01\\
599.18	0.01\\
599.19	0.01\\
599.2	0.01\\
599.21	0.01\\
599.22	0.01\\
599.23	0.01\\
599.24	0.01\\
599.25	0.01\\
599.26	0.01\\
599.27	0.01\\
599.28	0.01\\
599.29	0.01\\
599.3	0.01\\
599.31	0.01\\
599.32	0.01\\
599.33	0.01\\
599.34	0.01\\
599.35	0.01\\
599.36	0.01\\
599.37	0.01\\
599.38	0.01\\
599.39	0.01\\
599.4	0.01\\
599.41	0.01\\
599.42	0.01\\
599.43	0.01\\
599.44	0.01\\
599.45	0.01\\
599.46	0.01\\
599.47	0.01\\
599.48	0.01\\
599.49	0.01\\
599.5	0.01\\
599.51	0.01\\
599.52	0.01\\
599.53	0.01\\
599.54	0.01\\
599.55	0.01\\
599.56	0.01\\
599.57	0.01\\
599.58	0.01\\
599.59	0.01\\
599.6	0.01\\
599.61	0.01\\
599.62	0.01\\
599.63	0.01\\
599.64	0.01\\
599.65	0.01\\
599.66	0.01\\
599.67	0.01\\
599.68	0.01\\
599.69	0.01\\
599.7	0.01\\
599.71	0.01\\
599.72	0.01\\
599.73	0.01\\
599.74	0.01\\
599.75	0.01\\
599.76	0.01\\
599.77	0.01\\
599.78	0.01\\
599.79	0.01\\
599.8	0.01\\
599.81	0.01\\
599.82	0.01\\
599.83	0.01\\
599.84	0.01\\
599.85	0.01\\
599.86	0.01\\
599.87	0.01\\
599.88	0.01\\
599.89	0.01\\
599.9	0.01\\
599.91	0.01\\
599.92	0.01\\
599.93	0.01\\
599.94	0.01\\
599.95	0.01\\
599.96	0.01\\
599.97	0.01\\
599.98	0.01\\
599.99	0.01\\
600	0.01\\
};
\addplot [color=mycolor12,solid,forget plot]
  table[row sep=crcr]{%
0.01	0.00371473620820593\\
1.01	0.00371473620829894\\
2.01	0.00371473620839391\\
3.01	0.00371473620849088\\
4.01	0.00371473620858988\\
5.01	0.00371473620869097\\
6.01	0.0037147362087942\\
7.01	0.00371473620889959\\
8.01	0.0037147362090072\\
9.01	0.00371473620911708\\
10.01	0.00371473620922927\\
11.01	0.00371473620934383\\
12.01	0.0037147362094608\\
13.01	0.00371473620958023\\
14.01	0.00371473620970218\\
15.01	0.0037147362098267\\
16.01	0.00371473620995385\\
17.01	0.00371473621008368\\
18.01	0.00371473621021625\\
19.01	0.00371473621035161\\
20.01	0.00371473621048983\\
21.01	0.00371473621063097\\
22.01	0.00371473621077508\\
23.01	0.00371473621092224\\
24.01	0.0037147362110725\\
25.01	0.00371473621122594\\
26.01	0.00371473621138262\\
27.01	0.00371473621154261\\
28.01	0.00371473621170598\\
29.01	0.0037147362118728\\
30.01	0.00371473621204314\\
31.01	0.00371473621221709\\
32.01	0.00371473621239472\\
33.01	0.0037147362125761\\
34.01	0.00371473621276132\\
35.01	0.00371473621295046\\
36.01	0.00371473621314359\\
37.01	0.00371473621334082\\
38.01	0.00371473621354222\\
39.01	0.00371473621374789\\
40.01	0.00371473621395791\\
41.01	0.00371473621417238\\
42.01	0.00371473621439139\\
43.01	0.00371473621461505\\
44.01	0.00371473621484344\\
45.01	0.00371473621507667\\
46.01	0.00371473621531486\\
47.01	0.00371473621555809\\
48.01	0.00371473621580648\\
49.01	0.00371473621606015\\
50.01	0.00371473621631919\\
51.01	0.00371473621658373\\
52.01	0.0037147362168539\\
53.01	0.00371473621712979\\
54.01	0.00371473621741155\\
55.01	0.00371473621769929\\
56.01	0.00371473621799314\\
57.01	0.00371473621829324\\
58.01	0.00371473621859972\\
59.01	0.00371473621891272\\
60.01	0.00371473621923237\\
61.01	0.00371473621955883\\
62.01	0.00371473621989223\\
63.01	0.00371473622023272\\
64.01	0.00371473622058046\\
65.01	0.00371473622093561\\
66.01	0.00371473622129832\\
67.01	0.00371473622166876\\
68.01	0.00371473622204709\\
69.01	0.00371473622243348\\
70.01	0.00371473622282811\\
71.01	0.00371473622323116\\
72.01	0.00371473622364281\\
73.01	0.00371473622406322\\
74.01	0.00371473622449262\\
75.01	0.00371473622493118\\
76.01	0.00371473622537911\\
77.01	0.00371473622583659\\
78.01	0.00371473622630385\\
79.01	0.0037147362267811\\
80.01	0.00371473622726855\\
81.01	0.00371473622776641\\
82.01	0.00371473622827493\\
83.01	0.00371473622879431\\
84.01	0.00371473622932481\\
85.01	0.00371473622986666\\
86.01	0.00371473623042012\\
87.01	0.00371473623098542\\
88.01	0.00371473623156284\\
89.01	0.00371473623215262\\
90.01	0.00371473623275504\\
91.01	0.00371473623337037\\
92.01	0.0037147362339989\\
93.01	0.0037147362346409\\
94.01	0.00371473623529669\\
95.01	0.00371473623596655\\
96.01	0.00371473623665079\\
97.01	0.00371473623734971\\
98.01	0.00371473623806365\\
99.01	0.00371473623879293\\
100.01	0.00371473623953789\\
101.01	0.00371473624029886\\
102.01	0.00371473624107619\\
103.01	0.00371473624187024\\
104.01	0.0037147362426814\\
105.01	0.00371473624350999\\
106.01	0.00371473624435643\\
107.01	0.0037147362452211\\
108.01	0.0037147362461044\\
109.01	0.00371473624700674\\
110.01	0.00371473624792853\\
111.01	0.0037147362488702\\
112.01	0.00371473624983218\\
113.01	0.00371473625081492\\
114.01	0.00371473625181887\\
115.01	0.00371473625284451\\
116.01	0.00371473625389229\\
117.01	0.00371473625496273\\
118.01	0.00371473625605628\\
119.01	0.00371473625717348\\
120.01	0.00371473625831486\\
121.01	0.00371473625948092\\
122.01	0.00371473626067222\\
123.01	0.00371473626188931\\
124.01	0.00371473626313277\\
125.01	0.00371473626440315\\
126.01	0.00371473626570108\\
127.01	0.00371473626702715\\
128.01	0.00371473626838197\\
129.01	0.00371473626976619\\
130.01	0.00371473627118046\\
131.01	0.00371473627262542\\
132.01	0.00371473627410177\\
133.01	0.00371473627561019\\
134.01	0.0037147362771514\\
135.01	0.00371473627872616\\
136.01	0.00371473628033513\\
137.01	0.00371473628197909\\
138.01	0.00371473628365884\\
139.01	0.00371473628537519\\
140.01	0.00371473628712891\\
141.01	0.00371473628892085\\
142.01	0.00371473629075184\\
143.01	0.00371473629262277\\
144.01	0.00371473629453451\\
145.01	0.00371473629648798\\
146.01	0.00371473629848409\\
147.01	0.0037147363005238\\
148.01	0.00371473630260807\\
149.01	0.0037147363047379\\
150.01	0.00371473630691429\\
151.01	0.00371473630913829\\
152.01	0.00371473631141097\\
153.01	0.00371473631373339\\
154.01	0.00371473631610667\\
155.01	0.00371473631853194\\
156.01	0.00371473632101037\\
157.01	0.00371473632354315\\
158.01	0.00371473632613148\\
159.01	0.00371473632877661\\
160.01	0.0037147363314798\\
161.01	0.00371473633424236\\
162.01	0.0037147363370656\\
163.01	0.00371473633995091\\
164.01	0.00371473634289965\\
165.01	0.00371473634591325\\
166.01	0.00371473634899318\\
167.01	0.00371473635214093\\
168.01	0.00371473635535798\\
169.01	0.0037147363586459\\
170.01	0.00371473636200632\\
171.01	0.00371473636544083\\
172.01	0.00371473636895111\\
173.01	0.00371473637253886\\
174.01	0.00371473637620583\\
175.01	0.00371473637995379\\
176.01	0.00371473638378457\\
177.01	0.00371473638770004\\
178.01	0.0037147363917021\\
179.01	0.00371473639579271\\
180.01	0.00371473639997387\\
181.01	0.00371473640424761\\
182.01	0.00371473640861603\\
183.01	0.00371473641308127\\
184.01	0.0037147364176455\\
185.01	0.00371473642231097\\
186.01	0.00371473642708001\\
187.01	0.0037147364319549\\
188.01	0.00371473643693806\\
189.01	0.00371473644203194\\
190.01	0.00371473644723905\\
191.01	0.00371473645256196\\
192.01	0.00371473645800328\\
193.01	0.00371473646356573\\
194.01	0.00371473646925201\\
195.01	0.00371473647506497\\
196.01	0.00371473648100747\\
197.01	0.00371473648708245\\
198.01	0.00371473649329293\\
199.01	0.003714736499642\\
200.01	0.00371473650613279\\
201.01	0.00371473651276853\\
202.01	0.00371473651955252\\
203.01	0.00371473652648813\\
204.01	0.00371473653357883\\
205.01	0.00371473654082815\\
206.01	0.00371473654823969\\
207.01	0.00371473655581715\\
208.01	0.00371473656356433\\
209.01	0.0037147365714851\\
210.01	0.00371473657958342\\
211.01	0.00371473658786332\\
212.01	0.00371473659632899\\
213.01	0.00371473660498463\\
214.01	0.00371473661383461\\
215.01	0.00371473662288336\\
216.01	0.00371473663213543\\
217.01	0.00371473664159549\\
218.01	0.00371473665126825\\
219.01	0.00371473666115862\\
220.01	0.00371473667127157\\
221.01	0.00371473668161218\\
222.01	0.00371473669218568\\
223.01	0.00371473670299742\\
224.01	0.00371473671405283\\
225.01	0.0037147367253575\\
226.01	0.00371473673691716\\
227.01	0.00371473674873764\\
228.01	0.00371473676082493\\
229.01	0.00371473677318516\\
230.01	0.00371473678582457\\
231.01	0.00371473679874958\\
232.01	0.00371473681196674\\
233.01	0.00371473682548277\\
234.01	0.00371473683930452\\
235.01	0.003714736853439\\
236.01	0.00371473686789341\\
237.01	0.00371473688267509\\
238.01	0.00371473689779156\\
239.01	0.00371473691325051\\
240.01	0.00371473692905982\\
241.01	0.00371473694522752\\
242.01	0.00371473696176186\\
243.01	0.00371473697867128\\
244.01	0.00371473699596437\\
245.01	0.00371473701364998\\
246.01	0.00371473703173709\\
247.01	0.00371473705023498\\
248.01	0.00371473706915306\\
249.01	0.003714737088501\\
250.01	0.00371473710828869\\
251.01	0.00371473712852623\\
252.01	0.00371473714922397\\
253.01	0.00371473717039247\\
254.01	0.00371473719204259\\
255.01	0.00371473721418537\\
256.01	0.00371473723683213\\
257.01	0.00371473725999448\\
258.01	0.00371473728368426\\
259.01	0.00371473730791359\\
260.01	0.00371473733269488\\
261.01	0.0037147373580408\\
262.01	0.00371473738396434\\
263.01	0.00371473741047875\\
264.01	0.00371473743759763\\
265.01	0.00371473746533485\\
266.01	0.00371473749370461\\
267.01	0.00371473752272144\\
268.01	0.00371473755240022\\
269.01	0.00371473758275611\\
270.01	0.00371473761380469\\
271.01	0.00371473764556183\\
272.01	0.00371473767804383\\
273.01	0.00371473771126729\\
274.01	0.00371473774524923\\
275.01	0.00371473778000706\\
276.01	0.00371473781555857\\
277.01	0.00371473785192195\\
278.01	0.00371473788911584\\
279.01	0.00371473792715927\\
280.01	0.00371473796607169\\
281.01	0.00371473800587306\\
282.01	0.00371473804658372\\
283.01	0.00371473808822451\\
284.01	0.00371473813081676\\
285.01	0.00371473817438223\\
286.01	0.00371473821894323\\
287.01	0.00371473826452256\\
288.01	0.00371473831114352\\
289.01	0.00371473835882998\\
290.01	0.00371473840760632\\
291.01	0.00371473845749749\\
292.01	0.00371473850852902\\
293.01	0.00371473856072698\\
294.01	0.00371473861411811\\
295.01	0.00371473866872968\\
296.01	0.00371473872458965\\
297.01	0.00371473878172657\\
298.01	0.00371473884016969\\
299.01	0.00371473889994889\\
300.01	0.00371473896109475\\
301.01	0.00371473902363859\\
302.01	0.00371473908761238\\
303.01	0.0037147391530489\\
304.01	0.00371473921998162\\
305.01	0.00371473928844485\\
306.01	0.00371473935847365\\
307.01	0.0037147394301039\\
308.01	0.00371473950337232\\
309.01	0.0037147395783165\\
310.01	0.00371473965497488\\
311.01	0.00371473973338682\\
312.01	0.0037147398135926\\
313.01	0.00371473989563342\\
314.01	0.00371473997955152\\
315.01	0.00371474006539003\\
316.01	0.00371474015319321\\
317.01	0.0037147402430063\\
318.01	0.00371474033487565\\
319.01	0.0037147404288487\\
320.01	0.00371474052497403\\
321.01	0.00371474062330138\\
322.01	0.00371474072388172\\
323.01	0.00371474082676721\\
324.01	0.00371474093201128\\
325.01	0.00371474103966871\\
326.01	0.00371474114979555\\
327.01	0.00371474126244925\\
328.01	0.00371474137768867\\
329.01	0.00371474149557411\\
330.01	0.00371474161616739\\
331.01	0.00371474173953182\\
332.01	0.00371474186573233\\
333.01	0.00371474199483544\\
334.01	0.00371474212690934\\
335.01	0.00371474226202393\\
336.01	0.00371474240025091\\
337.01	0.00371474254166373\\
338.01	0.00371474268633775\\
339.01	0.00371474283435023\\
340.01	0.00371474298578043\\
341.01	0.00371474314070958\\
342.01	0.00371474329922107\\
343.01	0.00371474346140036\\
344.01	0.00371474362733519\\
345.01	0.00371474379711553\\
346.01	0.00371474397083368\\
347.01	0.00371474414858437\\
348.01	0.00371474433046478\\
349.01	0.00371474451657463\\
350.01	0.00371474470701625\\
351.01	0.00371474490189468\\
352.01	0.0037147451013177\\
353.01	0.00371474530539594\\
354.01	0.00371474551424294\\
355.01	0.00371474572797523\\
356.01	0.00371474594671247\\
357.01	0.00371474617057745\\
358.01	0.00371474639969623\\
359.01	0.00371474663419821\\
360.01	0.00371474687421626\\
361.01	0.00371474711988675\\
362.01	0.00371474737134971\\
363.01	0.00371474762874888\\
364.01	0.00371474789223186\\
365.01	0.00371474816195014\\
366.01	0.00371474843805931\\
367.01	0.00371474872071906\\
368.01	0.00371474901009336\\
369.01	0.00371474930635054\\
370.01	0.00371474960966338\\
371.01	0.0037147499202093\\
372.01	0.00371475023817042\\
373.01	0.00371475056373365\\
374.01	0.00371475089709086\\
375.01	0.00371475123843888\\
376.01	0.00371475158797938\\
377.01	0.00371475194591887\\
378.01	0.00371475231247041\\
379.01	0.0037147526878539\\
380.01	0.00371475307229368\\
381.01	0.00371475346601998\\
382.01	0.00371475386926931\\
383.01	0.00371475428228464\\
384.01	0.00371475470531573\\
385.01	0.00371475513861914\\
386.01	0.00371475558245867\\
387.01	0.00371475603710542\\
388.01	0.00371475650283813\\
389.01	0.00371475697994341\\
390.01	0.00371475746871603\\
391.01	0.0037147579694591\\
392.01	0.00371475848248448\\
393.01	0.00371475900811306\\
394.01	0.00371475954667503\\
395.01	0.00371476009851026\\
396.01	0.00371476066396858\\
397.01	0.00371476124341027\\
398.01	0.00371476183720637\\
399.01	0.00371476244573904\\
400.01	0.00371476306940211\\
401.01	0.00371476370860144\\
402.01	0.00371476436375545\\
403.01	0.00371476503529553\\
404.01	0.0037147657236667\\
405.01	0.00371476642932807\\
406.01	0.00371476715275349\\
407.01	0.00371476789443212\\
408.01	0.00371476865486911\\
409.01	0.00371476943458633\\
410.01	0.0037147702341231\\
411.01	0.00371477105403691\\
412.01	0.00371477189490436\\
413.01	0.00371477275732197\\
414.01	0.00371477364190714\\
415.01	0.00371477454929916\\
416.01	0.0037147754801603\\
417.01	0.00371477643517684\\
418.01	0.00371477741506044\\
419.01	0.00371477842054929\\
420.01	0.0037147794524096\\
421.01	0.00371478051143698\\
422.01	0.0037147815984581\\
423.01	0.00371478271433233\\
424.01	0.00371478385995362\\
425.01	0.00371478503625234\\
426.01	0.00371478624419755\\
427.01	0.00371478748479906\\
428.01	0.00371478875911005\\
429.01	0.00371479006822958\\
430.01	0.00371479141330557\\
431.01	0.00371479279553772\\
432.01	0.00371479421618102\\
433.01	0.00371479567654933\\
434.01	0.00371479717801937\\
435.01	0.0037147987220351\\
436.01	0.00371480031011245\\
437.01	0.00371480194384455\\
438.01	0.00371480362490755\\
439.01	0.00371480535506691\\
440.01	0.00371480713618448\\
441.01	0.00371480897022626\\
442.01	0.00371481085927117\\
443.01	0.00371481280552069\\
444.01	0.0037148148113098\\
445.01	0.00371481687911917\\
446.01	0.00371481901158901\\
447.01	0.00371482121153461\\
448.01	0.00371482348196422\\
449.01	0.00371482582609918\\
450.01	0.00371482824739706\\
451.01	0.0037148307495783\\
452.01	0.00371483333665677\\
453.01	0.00371483601297517\\
454.01	0.00371483878324612\\
455.01	0.0037148416525999\\
456.01	0.00371484462664034\\
457.01	0.00371484771150997\\
458.01	0.00371485091396572\\
459.01	0.00371485424146246\\
460.01	0.00371485770222783\\
461.01	0.00371486130531297\\
462.01	0.0037148650610467\\
463.01	0.00371486898466294\\
464.01	0.0037148731137831\\
465.01	0.00371487760858888\\
466.01	0.00371488317942011\\
467.01	0.00371489108398733\\
468.01	0.00371489995627214\\
469.01	0.00371490988535324\\
470.01	0.00371492452648405\\
471.01	0.00371494563636893\\
472.01	0.00371496748996078\\
473.01	0.00371499007825305\\
474.01	0.00371501361808457\\
475.01	0.00371503794990053\\
476.01	0.0037150629365636\\
477.01	0.00371508860183391\\
478.01	0.00371511497205824\\
479.01	0.00371514207539233\\
480.01	0.00371516994198696\\
481.01	0.00371519860419688\\
482.01	0.00371522809681539\\
483.01	0.00371525845733818\\
484.01	0.00371528972626019\\
485.01	0.00371532194740921\\
486.01	0.00371535516831954\\
487.01	0.00371538944064158\\
488.01	0.00371542482057514\\
489.01	0.00371546136937689\\
490.01	0.00371549915411554\\
491.01	0.00371553824844404\\
492.01	0.00371557873381042\\
493.01	0.00371562069954055\\
494.01	0.00371566424386073\\
495.01	0.00371570947533162\\
496.01	0.00371575651433542\\
497.01	0.00371580549579813\\
498.01	0.00371585657969426\\
499.01	0.00371591001637779\\
500.01	0.00371596659823481\\
501.01	0.00371603065683166\\
502.01	0.00371612405325414\\
503.01	0.00371625525780549\\
504.01	0.00371639138932336\\
505.01	0.00371653203759087\\
506.01	0.00371667743572924\\
507.01	0.0037168278377098\\
508.01	0.00371698352124755\\
509.01	0.00371714479123643\\
510.01	0.00371731198394848\\
511.01	0.00371748547293173\\
512.01	0.00371766568240781\\
513.01	0.00371785314674762\\
514.01	0.00371804885784427\\
515.01	0.00371825612386256\\
516.01	0.00371848598834987\\
517.01	0.00371874213320585\\
518.01	0.00371900987173145\\
519.01	0.00371928968602028\\
520.01	0.0037195875617622\\
521.01	0.00371993718974971\\
522.01	0.0037204418181354\\
523.01	0.00372101226720697\\
524.01	0.00372160058198296\\
525.01	0.00372220768405245\\
526.01	0.00372283457293392\\
527.01	0.00372348233490491\\
528.01	0.00372415215365289\\
529.01	0.00372484532507435\\
530.01	0.00372556327076363\\
531.01	0.00372630756874893\\
532.01	0.00372708007265731\\
533.01	0.00372788341169996\\
534.01	0.00372872200576432\\
535.01	0.00372959997164683\\
536.01	0.0037305335956814\\
537.01	0.00373168245468962\\
538.01	0.00373350854393371\\
539.01	0.00373548258913944\\
540.01	0.0037376578577701\\
541.01	0.00374045005175643\\
542.01	0.00374392631121988\\
543.01	0.00374882512298112\\
544.01	0.00375488044447884\\
545.01	0.00376117865772368\\
546.01	0.0037677422988858\\
547.01	0.00377460456468436\\
548.01	0.00378185627663242\\
549.01	0.00378982490344149\\
550.01	0.00380080429530036\\
551.01	0.00381524220095096\\
552.01	0.00383022462797278\\
553.01	0.00384579394090635\\
554.01	0.00386199774815606\\
555.01	0.00387888985441291\\
556.01	0.00389653211926337\\
557.01	0.00391500268978535\\
558.01	0.00393445765712839\\
559.01	0.00395568688045589\\
560.01	0.00398598281189649\\
561.01	0.00402999950325096\\
562.01	0.00407816099412708\\
563.01	0.00412824187571079\\
564.01	0.00417993576366653\\
565.01	0.00423331976729216\\
566.01	0.0042886911189915\\
567.01	0.00434623726267961\\
568.01	0.00440615157857568\\
569.01	0.00446865264722467\\
570.01	0.00453398872948427\\
571.01	0.00460251852537614\\
572.01	0.00467574118070142\\
573.01	0.00476776327846847\\
574.01	0.00491794626528909\\
575.01	0.00507467824241099\\
576.01	0.00523528193918925\\
577.01	0.005399916221415\\
578.01	0.00556854809725973\\
579.01	0.00574142821143175\\
580.01	0.00591889087949194\\
581.01	0.00610131589583468\\
582.01	0.00628913280226145\\
583.01	0.00648277347046916\\
584.01	0.00668248777362501\\
585.01	0.00691013716772029\\
586.01	0.00707283502695298\\
587.01	0.00721101595543212\\
588.01	0.00735296580304815\\
589.01	0.00749881702190324\\
590.01	0.00764870538441459\\
591.01	0.0078027577931949\\
592.01	0.00796109156594893\\
593.01	0.0081238057137617\\
594.01	0.00829097637799254\\
595.01	0.00846268644848632\\
596.01	0.00863938267011223\\
597.01	0.00882513249561383\\
598.01	0.00905588119071764\\
599.01	0.00958729251118108\\
599.02	0.00959623629274237\\
599.03	0.00960526138104788\\
599.04	0.00961436620037015\\
599.05	0.00962354898352098\\
599.06	0.00963280776000137\\
599.07	0.00964214034350163\\
599.08	0.00965154431871596\\
599.09	0.00966101702743224\\
599.1	0.00967055555386263\\
599.11	0.00968015670917047\\
599.12	0.00968981701514841\\
599.13	0.00969953268700591\\
599.14	0.00970929961521607\\
599.15	0.0097191133463711\\
599.16	0.00972896906299274\\
599.17	0.00973886156224116\\
599.18	0.00974878523346277\\
599.19	0.0097587340345143\\
599.2	0.00976870146679697\\
599.21	0.00977867955130989\\
599.22	0.00978865949017655\\
599.23	0.00979863316102171\\
599.24	0.00980859186698365\\
599.25	0.00981852514805581\\
599.26	0.0098284189602883\\
599.27	0.00983826246151365\\
599.28	0.00984803656229261\\
599.29	0.00985772626247611\\
599.3	0.00986730605691319\\
599.31	0.00987675003622819\\
599.32	0.00988604182000195\\
599.33	0.00989516400617934\\
599.34	0.00990409812080403\\
599.35	0.00991282455939869\\
599.36	0.00992132252525996\\
599.37	0.00992956996451327\\
599.38	0.00993754349776543\\
599.39	0.00994521834818632\\
599.4	0.00995256826584351\\
599.41	0.00995956544810689\\
599.42	0.00996614822627669\\
599.43	0.00997226774932304\\
599.44	0.00997788884364375\\
599.45	0.00998297432653716\\
599.46	0.00998748490019112\\
599.47	0.00999137904040232\\
599.48	0.00999461287979794\\
599.49	0.00999714147250411\\
599.5	0.00999891661491674\\
599.51	0.00999988677912612\\
599.52	0.01\\
599.53	0.01\\
599.54	0.01\\
599.55	0.01\\
599.56	0.01\\
599.57	0.01\\
599.58	0.01\\
599.59	0.01\\
599.6	0.01\\
599.61	0.01\\
599.62	0.01\\
599.63	0.01\\
599.64	0.01\\
599.65	0.01\\
599.66	0.01\\
599.67	0.01\\
599.68	0.01\\
599.69	0.01\\
599.7	0.01\\
599.71	0.01\\
599.72	0.01\\
599.73	0.01\\
599.74	0.01\\
599.75	0.01\\
599.76	0.01\\
599.77	0.01\\
599.78	0.01\\
599.79	0.01\\
599.8	0.01\\
599.81	0.01\\
599.82	0.01\\
599.83	0.01\\
599.84	0.01\\
599.85	0.01\\
599.86	0.01\\
599.87	0.01\\
599.88	0.01\\
599.89	0.01\\
599.9	0.01\\
599.91	0.01\\
599.92	0.01\\
599.93	0.01\\
599.94	0.01\\
599.95	0.01\\
599.96	0.01\\
599.97	0.01\\
599.98	0.01\\
599.99	0.01\\
600	0.01\\
};
\addplot [color=mycolor13,solid,forget plot]
  table[row sep=crcr]{%
0.01	0\\
1.01	0\\
2.01	0\\
3.01	0\\
4.01	0\\
5.01	0\\
6.01	0\\
7.01	0\\
8.01	0\\
9.01	0\\
10.01	0\\
11.01	0\\
12.01	0\\
13.01	0\\
14.01	0\\
15.01	0\\
16.01	0\\
17.01	0\\
18.01	0\\
19.01	0\\
20.01	0\\
21.01	0\\
22.01	0\\
23.01	0\\
24.01	0\\
25.01	0\\
26.01	0\\
27.01	0\\
28.01	0\\
29.01	0\\
30.01	0\\
31.01	0\\
32.01	0\\
33.01	0\\
34.01	0\\
35.01	0\\
36.01	0\\
37.01	0\\
38.01	0\\
39.01	0\\
40.01	0\\
41.01	0\\
42.01	0\\
43.01	0\\
44.01	0\\
45.01	0\\
46.01	0\\
47.01	0\\
48.01	0\\
49.01	0\\
50.01	0\\
51.01	0\\
52.01	0\\
53.01	0\\
54.01	0\\
55.01	0\\
56.01	0\\
57.01	0\\
58.01	0\\
59.01	0\\
60.01	0\\
61.01	0\\
62.01	0\\
63.01	0\\
64.01	0\\
65.01	0\\
66.01	0\\
67.01	0\\
68.01	0\\
69.01	0\\
70.01	0\\
71.01	0\\
72.01	0\\
73.01	0\\
74.01	0\\
75.01	0\\
76.01	0\\
77.01	0\\
78.01	0\\
79.01	0\\
80.01	0\\
81.01	0\\
82.01	0\\
83.01	0\\
84.01	0\\
85.01	0\\
86.01	0\\
87.01	0\\
88.01	0\\
89.01	0\\
90.01	0\\
91.01	0\\
92.01	0\\
93.01	0\\
94.01	0\\
95.01	0\\
96.01	0\\
97.01	0\\
98.01	0\\
99.01	0\\
100.01	0\\
101.01	0\\
102.01	0\\
103.01	0\\
104.01	0\\
105.01	0\\
106.01	0\\
107.01	0\\
108.01	0\\
109.01	0\\
110.01	0\\
111.01	0\\
112.01	0\\
113.01	0\\
114.01	0\\
115.01	0\\
116.01	0\\
117.01	0\\
118.01	0\\
119.01	0\\
120.01	0\\
121.01	0\\
122.01	0\\
123.01	0\\
124.01	0\\
125.01	0\\
126.01	0\\
127.01	0\\
128.01	0\\
129.01	0\\
130.01	0\\
131.01	0\\
132.01	0\\
133.01	0\\
134.01	0\\
135.01	0\\
136.01	0\\
137.01	0\\
138.01	0\\
139.01	0\\
140.01	0\\
141.01	0\\
142.01	0\\
143.01	0\\
144.01	0\\
145.01	0\\
146.01	0\\
147.01	0\\
148.01	0\\
149.01	0\\
150.01	0\\
151.01	0\\
152.01	0\\
153.01	0\\
154.01	0\\
155.01	0\\
156.01	0\\
157.01	0\\
158.01	0\\
159.01	0\\
160.01	0\\
161.01	0\\
162.01	0\\
163.01	0\\
164.01	0\\
165.01	0\\
166.01	0\\
167.01	0\\
168.01	0\\
169.01	0\\
170.01	0\\
171.01	0\\
172.01	0\\
173.01	0\\
174.01	0\\
175.01	0\\
176.01	0\\
177.01	0\\
178.01	0\\
179.01	0\\
180.01	0\\
181.01	0\\
182.01	0\\
183.01	0\\
184.01	0\\
185.01	0\\
186.01	0\\
187.01	0\\
188.01	0\\
189.01	0\\
190.01	0\\
191.01	0\\
192.01	0\\
193.01	0\\
194.01	0\\
195.01	0\\
196.01	0\\
197.01	0\\
198.01	0\\
199.01	0\\
200.01	0\\
201.01	0\\
202.01	0\\
203.01	0\\
204.01	0\\
205.01	0\\
206.01	0\\
207.01	0\\
208.01	0\\
209.01	0\\
210.01	0\\
211.01	0\\
212.01	0\\
213.01	0\\
214.01	0\\
215.01	0\\
216.01	0\\
217.01	0\\
218.01	0\\
219.01	0\\
220.01	0\\
221.01	0\\
222.01	0\\
223.01	0\\
224.01	0\\
225.01	0\\
226.01	0\\
227.01	0\\
228.01	0\\
229.01	0\\
230.01	0\\
231.01	0\\
232.01	0\\
233.01	0\\
234.01	0\\
235.01	0\\
236.01	0\\
237.01	0\\
238.01	0\\
239.01	0\\
240.01	0\\
241.01	0\\
242.01	0\\
243.01	0\\
244.01	0\\
245.01	0\\
246.01	0\\
247.01	0\\
248.01	0\\
249.01	0\\
250.01	0\\
251.01	0\\
252.01	0\\
253.01	0\\
254.01	0\\
255.01	0\\
256.01	0\\
257.01	0\\
258.01	0\\
259.01	0\\
260.01	0\\
261.01	0\\
262.01	0\\
263.01	0\\
264.01	0\\
265.01	0\\
266.01	0\\
267.01	0\\
268.01	0\\
269.01	0\\
270.01	0\\
271.01	0\\
272.01	0\\
273.01	0\\
274.01	0\\
275.01	0\\
276.01	0\\
277.01	0\\
278.01	0\\
279.01	0\\
280.01	0\\
281.01	0\\
282.01	0\\
283.01	0\\
284.01	0\\
285.01	0\\
286.01	0\\
287.01	0\\
288.01	0\\
289.01	0\\
290.01	0\\
291.01	0\\
292.01	0\\
293.01	0\\
294.01	0\\
295.01	0\\
296.01	0\\
297.01	0\\
298.01	0\\
299.01	0\\
300.01	0\\
301.01	0\\
302.01	0\\
303.01	0\\
304.01	0\\
305.01	0\\
306.01	0\\
307.01	0\\
308.01	0\\
309.01	0\\
310.01	0\\
311.01	0\\
312.01	0\\
313.01	0\\
314.01	0\\
315.01	0\\
316.01	0\\
317.01	0\\
318.01	0\\
319.01	0\\
320.01	0\\
321.01	0\\
322.01	0\\
323.01	0\\
324.01	0\\
325.01	0\\
326.01	0\\
327.01	0\\
328.01	0\\
329.01	0\\
330.01	0\\
331.01	0\\
332.01	0\\
333.01	0\\
334.01	0\\
335.01	0\\
336.01	0\\
337.01	0\\
338.01	0\\
339.01	0\\
340.01	0\\
341.01	0\\
342.01	0\\
343.01	0\\
344.01	0\\
345.01	0\\
346.01	0\\
347.01	0\\
348.01	0\\
349.01	0\\
350.01	0\\
351.01	0\\
352.01	0\\
353.01	0\\
354.01	0\\
355.01	0\\
356.01	0\\
357.01	0\\
358.01	0\\
359.01	0\\
360.01	0\\
361.01	0\\
362.01	0\\
363.01	0\\
364.01	0\\
365.01	0\\
366.01	0\\
367.01	0\\
368.01	0\\
369.01	0\\
370.01	0\\
371.01	0\\
372.01	0\\
373.01	0\\
374.01	0\\
375.01	0\\
376.01	0\\
377.01	0\\
378.01	0\\
379.01	0\\
380.01	0\\
381.01	0\\
382.01	0\\
383.01	0\\
384.01	0\\
385.01	0\\
386.01	0\\
387.01	0\\
388.01	0\\
389.01	0\\
390.01	0\\
391.01	0\\
392.01	0\\
393.01	0\\
394.01	0\\
395.01	0\\
396.01	0\\
397.01	0\\
398.01	0\\
399.01	0\\
400.01	0\\
401.01	0\\
402.01	0\\
403.01	0\\
404.01	0\\
405.01	0\\
406.01	0\\
407.01	0\\
408.01	0\\
409.01	0\\
410.01	0\\
411.01	0\\
412.01	0\\
413.01	0\\
414.01	0\\
415.01	0\\
416.01	0\\
417.01	0\\
418.01	0\\
419.01	0\\
420.01	0\\
421.01	0\\
422.01	0\\
423.01	0\\
424.01	0\\
425.01	0\\
426.01	0\\
427.01	0\\
428.01	0\\
429.01	0\\
430.01	0\\
431.01	0\\
432.01	0\\
433.01	0\\
434.01	0\\
435.01	0\\
436.01	0\\
437.01	0\\
438.01	0\\
439.01	0\\
440.01	0\\
441.01	0\\
442.01	0\\
443.01	0\\
444.01	0\\
445.01	0\\
446.01	0\\
447.01	0\\
448.01	0\\
449.01	0\\
450.01	0\\
451.01	0\\
452.01	0\\
453.01	0\\
454.01	0\\
455.01	0\\
456.01	0\\
457.01	0\\
458.01	0\\
459.01	0\\
460.01	0\\
461.01	0\\
462.01	0\\
463.01	0\\
464.01	0\\
465.01	0\\
466.01	0\\
467.01	0\\
468.01	0\\
469.01	0\\
470.01	0\\
471.01	0\\
472.01	0\\
473.01	0\\
474.01	0\\
475.01	0\\
476.01	0\\
477.01	0\\
478.01	0\\
479.01	0\\
480.01	0\\
481.01	0\\
482.01	0\\
483.01	0\\
484.01	0\\
485.01	0\\
486.01	0\\
487.01	0\\
488.01	0\\
489.01	0\\
490.01	0\\
491.01	0\\
492.01	0\\
493.01	0\\
494.01	0\\
495.01	0\\
496.01	0\\
497.01	0\\
498.01	0\\
499.01	0\\
500.01	0\\
501.01	0\\
502.01	0\\
503.01	0\\
504.01	0\\
505.01	0\\
506.01	0\\
507.01	0\\
508.01	0\\
509.01	0\\
510.01	0\\
511.01	0\\
512.01	0\\
513.01	0\\
514.01	0\\
515.01	0\\
516.01	0\\
517.01	0\\
518.01	0\\
519.01	0\\
520.01	0\\
521.01	0\\
522.01	0\\
523.01	0\\
524.01	0\\
525.01	0\\
526.01	0\\
527.01	0\\
528.01	0\\
529.01	0\\
530.01	0\\
531.01	0\\
532.01	0\\
533.01	0\\
534.01	0\\
535.01	0\\
536.01	0\\
537.01	0\\
538.01	0\\
539.01	0\\
540.01	0\\
541.01	0\\
542.01	0\\
543.01	0\\
544.01	0\\
545.01	0\\
546.01	0\\
547.01	0\\
548.01	0\\
549.01	0\\
550.01	0\\
551.01	0\\
552.01	0\\
553.01	0\\
554.01	0\\
555.01	0\\
556.01	0\\
557.01	0\\
558.01	0\\
559.01	0\\
560.01	0\\
561.01	0\\
562.01	0\\
563.01	0\\
564.01	0\\
565.01	0\\
566.01	0\\
567.01	0\\
568.01	0\\
569.01	0\\
570.01	0\\
571.01	0\\
572.01	0\\
573.01	0\\
574.01	0\\
575.01	0\\
576.01	0\\
577.01	0\\
578.01	0\\
579.01	0\\
580.01	0\\
581.01	0\\
582.01	0\\
583.01	0\\
584.01	0\\
585.01	0\\
586.01	0.000175800090626798\\
587.01	0.000388923417000744\\
588.01	0.000607581801831777\\
589.01	0.000832227384632539\\
590.01	0.00106339227312889\\
591.01	0.00130168499733363\\
592.01	0.00154780821626582\\
593.01	0.00180257534930202\\
594.01	0.00206693981388568\\
595.01	0.00234207488850763\\
596.01	0.00262980748550698\\
597.01	0.00293601233450368\\
598.01	0.00330057550686641\\
599.01	0.00409089741288224\\
599.02	0.00410624158337643\\
599.03	0.00412187629047293\\
599.04	0.0041378089664602\\
599.05	0.00415404725346514\\
599.06	0.0041705990101973\\
599.07	0.00418747231894723\\
599.08	0.00420467549285039\\
599.09	0.00422221708342868\\
599.1	0.00424010588842206\\
599.11	0.0042583509599236\\
599.12	0.00427696161283197\\
599.13	0.00429594743363603\\
599.14	0.0043153182895472\\
599.15	0.00433508433799597\\
599.16	0.00435525603650987\\
599.17	0.00437584415299123\\
599.18	0.00439685977641415\\
599.19	0.00441831432796106\\
599.2	0.00444021957262047\\
599.21	0.00446258763268925\\
599.22	0.00448543100083088\\
599.23	0.00450876255151259\\
599.24	0.00453259555470182\\
599.25	0.00455694369179364\\
599.26	0.00458182107472543\\
599.27	0.00460724225617071\\
599.28	0.00463322225645973\\
599.29	0.00465977657394944\\
599.3	0.00468692121658179\\
599.31	0.0047146727202229\\
599.32	0.00474304815378429\\
599.33	0.0047720651397426\\
599.34	0.00480174187561087\\
599.35	0.00483209715642481\\
599.36	0.00486315039830252\\
599.37	0.00489492166313947\\
599.38	0.0049274316845055\\
599.39	0.00496070189481463\\
599.4	0.00499475445384406\\
599.41	0.00502961227868348\\
599.42	0.00506529912031843\\
599.43	0.00510183957745583\\
599.44	0.00513925911030819\\
599.45	0.00517758407750282\\
599.46	0.00521684177494776\\
599.47	0.00525706047678138\\
599.48	0.00529826947854203\\
599.49	0.00534049779916455\\
599.5	0.00538377625915345\\
599.51	0.00542813744455676\\
599.52	0.00547361280337367\\
599.53	0.0055202340564185\\
599.54	0.00556803008723294\\
599.55	0.00561703167916341\\
599.56	0.00566725869375703\\
599.57	0.00571874907924545\\
599.58	0.00577154219388792\\
599.59	0.00582567887207813\\
599.6	0.00588120149439699\\
599.61	0.0059381540619025\\
599.62	0.00599655232226517\\
599.63	0.00605643298747278\\
599.64	0.0061178440982782\\
599.65	0.00618083560576154\\
599.66	0.00624545946877155\\
599.67	0.00631176975769696\\
599.68	0.00637982276511079\\
599.69	0.00644967712385027\\
599.7	0.00652139393315005\\
599.71	0.00659503689350761\\
599.72	0.00667067245102825\\
599.73	0.00674836995207315\\
599.74	0.00682820180916178\\
599.75	0.00691024367906377\\
599.76	0.00699457465421154\\
599.77	0.00708127746867573\\
599.78	0.00717043872007276\\
599.79	0.00726214910894212\\
599.8	0.00735650369726444\\
599.81	0.00745360218802786\\
599.82	0.00755354922795706\\
599.83	0.00765645473577685\\
599.84	0.00776243425867489\\
599.85	0.0078716093599577\\
599.86	0.00798410804127475\\
599.87	0.00810006520322131\\
599.88	0.00821962314863375\\
599.89	0.00834293213347036\\
599.9	0.00847015097084234\\
599.91	0.00860144769453793\\
599.92	0.00873700028928946\\
599.93	0.00887699749609073\\
599.94	0.00902163970211132\\
599.95	0.00917113992620961\\
599.96	0.00932572491276224\\
599.97	0.00948563634855679\\
599.98	0.00965113221990432\\
599.99	0.00982248833000031\\
600	0.01\\
};
\addplot [color=mycolor14,solid,forget plot]
  table[row sep=crcr]{%
0.01	0.01\\
1.01	0.01\\
2.01	0.01\\
3.01	0.01\\
4.01	0.01\\
5.01	0.01\\
6.01	0.01\\
7.01	0.01\\
8.01	0.01\\
9.01	0.01\\
10.01	0.01\\
11.01	0.01\\
12.01	0.01\\
13.01	0.01\\
14.01	0.01\\
15.01	0.01\\
16.01	0.01\\
17.01	0.01\\
18.01	0.01\\
19.01	0.01\\
20.01	0.01\\
21.01	0.01\\
22.01	0.01\\
23.01	0.01\\
24.01	0.01\\
25.01	0.01\\
26.01	0.01\\
27.01	0.01\\
28.01	0.01\\
29.01	0.01\\
30.01	0.01\\
31.01	0.01\\
32.01	0.01\\
33.01	0.01\\
34.01	0.01\\
35.01	0.01\\
36.01	0.01\\
37.01	0.01\\
38.01	0.01\\
39.01	0.01\\
40.01	0.01\\
41.01	0.01\\
42.01	0.01\\
43.01	0.01\\
44.01	0.01\\
45.01	0.01\\
46.01	0.01\\
47.01	0.01\\
48.01	0.01\\
49.01	0.01\\
50.01	0.01\\
51.01	0.01\\
52.01	0.01\\
53.01	0.01\\
54.01	0.01\\
55.01	0.01\\
56.01	0.01\\
57.01	0.01\\
58.01	0.01\\
59.01	0.01\\
60.01	0.01\\
61.01	0.01\\
62.01	0.01\\
63.01	0.01\\
64.01	0.01\\
65.01	0.01\\
66.01	0.01\\
67.01	0.01\\
68.01	0.01\\
69.01	0.01\\
70.01	0.01\\
71.01	0.01\\
72.01	0.01\\
73.01	0.01\\
74.01	0.01\\
75.01	0.01\\
76.01	0.01\\
77.01	0.01\\
78.01	0.01\\
79.01	0.01\\
80.01	0.01\\
81.01	0.01\\
82.01	0.01\\
83.01	0.01\\
84.01	0.01\\
85.01	0.01\\
86.01	0.01\\
87.01	0.01\\
88.01	0.01\\
89.01	0.01\\
90.01	0.01\\
91.01	0.01\\
92.01	0.01\\
93.01	0.01\\
94.01	0.01\\
95.01	0.01\\
96.01	0.01\\
97.01	0.01\\
98.01	0.01\\
99.01	0.01\\
100.01	0.01\\
101.01	0.01\\
102.01	0.01\\
103.01	0.01\\
104.01	0.01\\
105.01	0.01\\
106.01	0.01\\
107.01	0.01\\
108.01	0.01\\
109.01	0.01\\
110.01	0.01\\
111.01	0.01\\
112.01	0.01\\
113.01	0.01\\
114.01	0.01\\
115.01	0.01\\
116.01	0.01\\
117.01	0.01\\
118.01	0.01\\
119.01	0.01\\
120.01	0.01\\
121.01	0.01\\
122.01	0.01\\
123.01	0.01\\
124.01	0.01\\
125.01	0.01\\
126.01	0.01\\
127.01	0.01\\
128.01	0.01\\
129.01	0.01\\
130.01	0.01\\
131.01	0.01\\
132.01	0.01\\
133.01	0.01\\
134.01	0.01\\
135.01	0.01\\
136.01	0.01\\
137.01	0.01\\
138.01	0.01\\
139.01	0.01\\
140.01	0.01\\
141.01	0.01\\
142.01	0.01\\
143.01	0.01\\
144.01	0.01\\
145.01	0.01\\
146.01	0.01\\
147.01	0.01\\
148.01	0.01\\
149.01	0.01\\
150.01	0.01\\
151.01	0.01\\
152.01	0.01\\
153.01	0.01\\
154.01	0.01\\
155.01	0.01\\
156.01	0.01\\
157.01	0.01\\
158.01	0.01\\
159.01	0.01\\
160.01	0.01\\
161.01	0.01\\
162.01	0.01\\
163.01	0.01\\
164.01	0.01\\
165.01	0.01\\
166.01	0.01\\
167.01	0.01\\
168.01	0.01\\
169.01	0.01\\
170.01	0.01\\
171.01	0.01\\
172.01	0.01\\
173.01	0.01\\
174.01	0.01\\
175.01	0.01\\
176.01	0.01\\
177.01	0.01\\
178.01	0.01\\
179.01	0.01\\
180.01	0.01\\
181.01	0.01\\
182.01	0.01\\
183.01	0.01\\
184.01	0.01\\
185.01	0.01\\
186.01	0.01\\
187.01	0.01\\
188.01	0.01\\
189.01	0.01\\
190.01	0.01\\
191.01	0.01\\
192.01	0.01\\
193.01	0.01\\
194.01	0.01\\
195.01	0.01\\
196.01	0.01\\
197.01	0.01\\
198.01	0.01\\
199.01	0.01\\
200.01	0.01\\
201.01	0.01\\
202.01	0.01\\
203.01	0.01\\
204.01	0.01\\
205.01	0.01\\
206.01	0.01\\
207.01	0.01\\
208.01	0.01\\
209.01	0.01\\
210.01	0.01\\
211.01	0.01\\
212.01	0.01\\
213.01	0.01\\
214.01	0.01\\
215.01	0.01\\
216.01	0.01\\
217.01	0.01\\
218.01	0.01\\
219.01	0.01\\
220.01	0.01\\
221.01	0.01\\
222.01	0.01\\
223.01	0.01\\
224.01	0.01\\
225.01	0.01\\
226.01	0.01\\
227.01	0.01\\
228.01	0.01\\
229.01	0.01\\
230.01	0.01\\
231.01	0.01\\
232.01	0.01\\
233.01	0.01\\
234.01	0.01\\
235.01	0.01\\
236.01	0.01\\
237.01	0.01\\
238.01	0.01\\
239.01	0.01\\
240.01	0.01\\
241.01	0.01\\
242.01	0.01\\
243.01	0.01\\
244.01	0.01\\
245.01	0.01\\
246.01	0.01\\
247.01	0.01\\
248.01	0.01\\
249.01	0.01\\
250.01	0.01\\
251.01	0.01\\
252.01	0.01\\
253.01	0.01\\
254.01	0.01\\
255.01	0.01\\
256.01	0.01\\
257.01	0.01\\
258.01	0.01\\
259.01	0.01\\
260.01	0.01\\
261.01	0.01\\
262.01	0.01\\
263.01	0.01\\
264.01	0.01\\
265.01	0.01\\
266.01	0.01\\
267.01	0.01\\
268.01	0.01\\
269.01	0.01\\
270.01	0.01\\
271.01	0.01\\
272.01	0.01\\
273.01	0.01\\
274.01	0.01\\
275.01	0.01\\
276.01	0.01\\
277.01	0.01\\
278.01	0.01\\
279.01	0.01\\
280.01	0.01\\
281.01	0.01\\
282.01	0.01\\
283.01	0.01\\
284.01	0.01\\
285.01	0.01\\
286.01	0.01\\
287.01	0.01\\
288.01	0.01\\
289.01	0.01\\
290.01	0.01\\
291.01	0.01\\
292.01	0.01\\
293.01	0.01\\
294.01	0.01\\
295.01	0.01\\
296.01	0.01\\
297.01	0.01\\
298.01	0.01\\
299.01	0.01\\
300.01	0.01\\
301.01	0.01\\
302.01	0.01\\
303.01	0.01\\
304.01	0.01\\
305.01	0.01\\
306.01	0.01\\
307.01	0.01\\
308.01	0.01\\
309.01	0.01\\
310.01	0.01\\
311.01	0.01\\
312.01	0.01\\
313.01	0.01\\
314.01	0.01\\
315.01	0.01\\
316.01	0.01\\
317.01	0.01\\
318.01	0.01\\
319.01	0.01\\
320.01	0.01\\
321.01	0.01\\
322.01	0.01\\
323.01	0.01\\
324.01	0.01\\
325.01	0.01\\
326.01	0.01\\
327.01	0.01\\
328.01	0.01\\
329.01	0.01\\
330.01	0.01\\
331.01	0.01\\
332.01	0.01\\
333.01	0.01\\
334.01	0.01\\
335.01	0.01\\
336.01	0.01\\
337.01	0.01\\
338.01	0.01\\
339.01	0.01\\
340.01	0.01\\
341.01	0.01\\
342.01	0.01\\
343.01	0.01\\
344.01	0.01\\
345.01	0.01\\
346.01	0.01\\
347.01	0.01\\
348.01	0.01\\
349.01	0.01\\
350.01	0.01\\
351.01	0.01\\
352.01	0.01\\
353.01	0.01\\
354.01	0.01\\
355.01	0.01\\
356.01	0.01\\
357.01	0.01\\
358.01	0.01\\
359.01	0.01\\
360.01	0.01\\
361.01	0.01\\
362.01	0.01\\
363.01	0.01\\
364.01	0.01\\
365.01	0.01\\
366.01	0.01\\
367.01	0.01\\
368.01	0.01\\
369.01	0.01\\
370.01	0.01\\
371.01	0.01\\
372.01	0.01\\
373.01	0.01\\
374.01	0.01\\
375.01	0.01\\
376.01	0.01\\
377.01	0.01\\
378.01	0.01\\
379.01	0.01\\
380.01	0.01\\
381.01	0.01\\
382.01	0.01\\
383.01	0.01\\
384.01	0.01\\
385.01	0.01\\
386.01	0.01\\
387.01	0.01\\
388.01	0.01\\
389.01	0.01\\
390.01	0.01\\
391.01	0.01\\
392.01	0.01\\
393.01	0.01\\
394.01	0.01\\
395.01	0.01\\
396.01	0.01\\
397.01	0.01\\
398.01	0.01\\
399.01	0.01\\
400.01	0.01\\
401.01	0.01\\
402.01	0.01\\
403.01	0.01\\
404.01	0.01\\
405.01	0.01\\
406.01	0.01\\
407.01	0.01\\
408.01	0.01\\
409.01	0.01\\
410.01	0.01\\
411.01	0.01\\
412.01	0.01\\
413.01	0.01\\
414.01	0.01\\
415.01	0.01\\
416.01	0.01\\
417.01	0.01\\
418.01	0.01\\
419.01	0.01\\
420.01	0.01\\
421.01	0.01\\
422.01	0.01\\
423.01	0.01\\
424.01	0.01\\
425.01	0.01\\
426.01	0.01\\
427.01	0.01\\
428.01	0.01\\
429.01	0.01\\
430.01	0.01\\
431.01	0.01\\
432.01	0.01\\
433.01	0.01\\
434.01	0.01\\
435.01	0.01\\
436.01	0.01\\
437.01	0.01\\
438.01	0.01\\
439.01	0.01\\
440.01	0.01\\
441.01	0.01\\
442.01	0.01\\
443.01	0.01\\
444.01	0.01\\
445.01	0.01\\
446.01	0.01\\
447.01	0.01\\
448.01	0.01\\
449.01	0.01\\
450.01	0.01\\
451.01	0.01\\
452.01	0.01\\
453.01	0.01\\
454.01	0.01\\
455.01	0.01\\
456.01	0.01\\
457.01	0.01\\
458.01	0.01\\
459.01	0.01\\
460.01	0.01\\
461.01	0.01\\
462.01	0.01\\
463.01	0.01\\
464.01	0.01\\
465.01	0.01\\
466.01	0.01\\
467.01	0.01\\
468.01	0.01\\
469.01	0.01\\
470.01	0.01\\
471.01	0.01\\
472.01	0.01\\
473.01	0.01\\
474.01	0.01\\
475.01	0.01\\
476.01	0.01\\
477.01	0.01\\
478.01	0.01\\
479.01	0.01\\
480.01	0.01\\
481.01	0.01\\
482.01	0.01\\
483.01	0.01\\
484.01	0.01\\
485.01	0.01\\
486.01	0.01\\
487.01	0.01\\
488.01	0.01\\
489.01	0.01\\
490.01	0.01\\
491.01	0.01\\
492.01	0.01\\
493.01	0.01\\
494.01	0.01\\
495.01	0.01\\
496.01	0.01\\
497.01	0.01\\
498.01	0.01\\
499.01	0.01\\
500.01	0.01\\
501.01	0.01\\
502.01	0.01\\
503.01	0.01\\
504.01	0.01\\
505.01	0.01\\
506.01	0.01\\
507.01	0.01\\
508.01	0.01\\
509.01	0.01\\
510.01	0.01\\
511.01	0.01\\
512.01	0.01\\
513.01	0.01\\
514.01	0.01\\
515.01	0.01\\
516.01	0.01\\
517.01	0.01\\
518.01	0.01\\
519.01	0.01\\
520.01	0.01\\
521.01	0.01\\
522.01	0.01\\
523.01	0.01\\
524.01	0.01\\
525.01	0.01\\
526.01	0.01\\
527.01	0.01\\
528.01	0.01\\
529.01	0.01\\
530.01	0.01\\
531.01	0.01\\
532.01	0.01\\
533.01	0.01\\
534.01	0.01\\
535.01	0.01\\
536.01	0.01\\
537.01	0.01\\
538.01	0.01\\
539.01	0.01\\
540.01	0.01\\
541.01	0.01\\
542.01	0.01\\
543.01	0.01\\
544.01	0.01\\
545.01	0.01\\
546.01	0.01\\
547.01	0.01\\
548.01	0.01\\
549.01	0.01\\
550.01	0.01\\
551.01	0.01\\
552.01	0.01\\
553.01	0.01\\
554.01	0.01\\
555.01	0.01\\
556.01	0.01\\
557.01	0.01\\
558.01	0.01\\
559.01	0.01\\
560.01	0.01\\
561.01	0.01\\
562.01	0.01\\
563.01	0.01\\
564.01	0.01\\
565.01	0.01\\
566.01	0.01\\
567.01	0.01\\
568.01	0.01\\
569.01	0.01\\
570.01	0.01\\
571.01	0.01\\
572.01	0.01\\
573.01	0.01\\
574.01	0.01\\
575.01	0.01\\
576.01	0.01\\
577.01	0.01\\
578.01	0.01\\
579.01	0.01\\
580.01	0.01\\
581.01	0.01\\
582.01	0.01\\
583.01	0.01\\
584.01	0.01\\
585.01	0.01\\
586.01	0.01\\
587.01	0.01\\
588.01	0.01\\
589.01	0.01\\
590.01	0.01\\
591.01	0.01\\
592.01	0.01\\
593.01	0.01\\
594.01	0.01\\
595.01	0.01\\
596.01	0.01\\
597.01	0.01\\
598.01	0.01\\
599.01	0.01\\
599.02	0.01\\
599.03	0.01\\
599.04	0.01\\
599.05	0.01\\
599.06	0.01\\
599.07	0.01\\
599.08	0.01\\
599.09	0.01\\
599.1	0.01\\
599.11	0.01\\
599.12	0.01\\
599.13	0.01\\
599.14	0.01\\
599.15	0.01\\
599.16	0.01\\
599.17	0.01\\
599.18	0.01\\
599.19	0.01\\
599.2	0.01\\
599.21	0.01\\
599.22	0.01\\
599.23	0.01\\
599.24	0.01\\
599.25	0.01\\
599.26	0.01\\
599.27	0.01\\
599.28	0.01\\
599.29	0.01\\
599.3	0.01\\
599.31	0.01\\
599.32	0.01\\
599.33	0.01\\
599.34	0.01\\
599.35	0.01\\
599.36	0.01\\
599.37	0.01\\
599.38	0.01\\
599.39	0.01\\
599.4	0.01\\
599.41	0.01\\
599.42	0.01\\
599.43	0.01\\
599.44	0.01\\
599.45	0.01\\
599.46	0.01\\
599.47	0.01\\
599.48	0.01\\
599.49	0.00989153793223401\\
599.5	0.00972504357100398\\
599.51	0.00955769558107442\\
599.52	0.00938948342288605\\
599.53	0.0092204021322092\\
599.54	0.00905044766612316\\
599.55	0.00887960877732616\\
599.56	0.00870787409334322\\
599.57	0.00853523207011476\\
599.58	0.0083616703641326\\
599.59	0.00818717640695089\\
599.6	0.00801173706697357\\
599.61	0.00783533878810403\\
599.62	0.0076579675834383\\
599.63	0.00747960902168377\\
599.64	0.00730024821313421\\
599.65	0.00711986979518157\\
599.66	0.00693845791734417\\
599.67	0.00675599622540386\\
599.68	0.00657246784521872\\
599.69	0.00638785536611911\\
599.7	0.0062021408237225\\
599.71	0.0060153056821398\\
599.72	0.00582733081357053\\
599.73	0.00563819647474965\\
599.74	0.00544788228619198\\
599.75	0.00525636721096807\\
599.76	0.00506362953138433\\
599.77	0.00486964682392524\\
599.78	0.00467439593512126\\
599.79	0.0044778529564409\\
599.8	0.00427999319815484\\
599.81	0.00408079116211723\\
599.82	0.00388022051340578\\
599.83	0.00367825405075887\\
599.84	0.00347486367574443\\
599.85	0.00327002036059148\\
599.86	0.00306369411461145\\
599.87	0.00285585394913248\\
599.88	0.00264646784086591\\
599.89	0.00243550269362013\\
599.9	0.00222292429827314\\
599.91	0.00200869729091123\\
599.92	0.00179278510903777\\
599.93	0.00157514994575309\\
599.94	0.00135575270180365\\
599.95	0.00113455293539763\\
599.96	0.000911508809683114\\
599.97	0.000686577037787083\\
599.98	0.000459712825316519\\
599.99	0.000230869810230147\\
600	0\\
};
\addplot [color=mycolor15,solid,forget plot]
  table[row sep=crcr]{%
0.01	0.01\\
1.01	0.01\\
2.01	0.01\\
3.01	0.01\\
4.01	0.01\\
5.01	0.01\\
6.01	0.01\\
7.01	0.01\\
8.01	0.01\\
9.01	0.01\\
10.01	0.01\\
11.01	0.01\\
12.01	0.01\\
13.01	0.01\\
14.01	0.01\\
15.01	0.01\\
16.01	0.01\\
17.01	0.01\\
18.01	0.01\\
19.01	0.01\\
20.01	0.01\\
21.01	0.01\\
22.01	0.01\\
23.01	0.01\\
24.01	0.01\\
25.01	0.01\\
26.01	0.01\\
27.01	0.01\\
28.01	0.01\\
29.01	0.01\\
30.01	0.01\\
31.01	0.01\\
32.01	0.01\\
33.01	0.01\\
34.01	0.01\\
35.01	0.01\\
36.01	0.01\\
37.01	0.01\\
38.01	0.01\\
39.01	0.01\\
40.01	0.01\\
41.01	0.01\\
42.01	0.01\\
43.01	0.01\\
44.01	0.01\\
45.01	0.01\\
46.01	0.01\\
47.01	0.01\\
48.01	0.01\\
49.01	0.01\\
50.01	0.01\\
51.01	0.01\\
52.01	0.01\\
53.01	0.01\\
54.01	0.01\\
55.01	0.01\\
56.01	0.01\\
57.01	0.01\\
58.01	0.01\\
59.01	0.01\\
60.01	0.01\\
61.01	0.01\\
62.01	0.01\\
63.01	0.01\\
64.01	0.01\\
65.01	0.01\\
66.01	0.01\\
67.01	0.01\\
68.01	0.01\\
69.01	0.01\\
70.01	0.01\\
71.01	0.01\\
72.01	0.01\\
73.01	0.01\\
74.01	0.01\\
75.01	0.01\\
76.01	0.01\\
77.01	0.01\\
78.01	0.01\\
79.01	0.01\\
80.01	0.01\\
81.01	0.01\\
82.01	0.01\\
83.01	0.01\\
84.01	0.01\\
85.01	0.01\\
86.01	0.01\\
87.01	0.01\\
88.01	0.01\\
89.01	0.01\\
90.01	0.01\\
91.01	0.01\\
92.01	0.01\\
93.01	0.01\\
94.01	0.01\\
95.01	0.01\\
96.01	0.01\\
97.01	0.01\\
98.01	0.01\\
99.01	0.01\\
100.01	0.01\\
101.01	0.01\\
102.01	0.01\\
103.01	0.01\\
104.01	0.01\\
105.01	0.01\\
106.01	0.01\\
107.01	0.01\\
108.01	0.01\\
109.01	0.01\\
110.01	0.01\\
111.01	0.01\\
112.01	0.01\\
113.01	0.01\\
114.01	0.01\\
115.01	0.01\\
116.01	0.01\\
117.01	0.01\\
118.01	0.01\\
119.01	0.01\\
120.01	0.01\\
121.01	0.01\\
122.01	0.01\\
123.01	0.01\\
124.01	0.01\\
125.01	0.01\\
126.01	0.01\\
127.01	0.01\\
128.01	0.01\\
129.01	0.01\\
130.01	0.01\\
131.01	0.01\\
132.01	0.01\\
133.01	0.01\\
134.01	0.01\\
135.01	0.01\\
136.01	0.01\\
137.01	0.01\\
138.01	0.01\\
139.01	0.01\\
140.01	0.01\\
141.01	0.01\\
142.01	0.01\\
143.01	0.01\\
144.01	0.01\\
145.01	0.01\\
146.01	0.01\\
147.01	0.01\\
148.01	0.01\\
149.01	0.01\\
150.01	0.01\\
151.01	0.01\\
152.01	0.01\\
153.01	0.01\\
154.01	0.01\\
155.01	0.01\\
156.01	0.01\\
157.01	0.01\\
158.01	0.01\\
159.01	0.01\\
160.01	0.01\\
161.01	0.01\\
162.01	0.01\\
163.01	0.01\\
164.01	0.01\\
165.01	0.01\\
166.01	0.01\\
167.01	0.01\\
168.01	0.01\\
169.01	0.01\\
170.01	0.01\\
171.01	0.01\\
172.01	0.01\\
173.01	0.01\\
174.01	0.01\\
175.01	0.01\\
176.01	0.01\\
177.01	0.01\\
178.01	0.01\\
179.01	0.01\\
180.01	0.01\\
181.01	0.01\\
182.01	0.01\\
183.01	0.01\\
184.01	0.01\\
185.01	0.01\\
186.01	0.01\\
187.01	0.01\\
188.01	0.01\\
189.01	0.01\\
190.01	0.01\\
191.01	0.01\\
192.01	0.01\\
193.01	0.01\\
194.01	0.01\\
195.01	0.01\\
196.01	0.01\\
197.01	0.01\\
198.01	0.01\\
199.01	0.01\\
200.01	0.01\\
201.01	0.01\\
202.01	0.01\\
203.01	0.01\\
204.01	0.01\\
205.01	0.01\\
206.01	0.01\\
207.01	0.01\\
208.01	0.01\\
209.01	0.01\\
210.01	0.01\\
211.01	0.01\\
212.01	0.01\\
213.01	0.01\\
214.01	0.01\\
215.01	0.01\\
216.01	0.01\\
217.01	0.01\\
218.01	0.01\\
219.01	0.01\\
220.01	0.01\\
221.01	0.01\\
222.01	0.01\\
223.01	0.01\\
224.01	0.01\\
225.01	0.01\\
226.01	0.01\\
227.01	0.01\\
228.01	0.01\\
229.01	0.01\\
230.01	0.01\\
231.01	0.01\\
232.01	0.01\\
233.01	0.01\\
234.01	0.01\\
235.01	0.01\\
236.01	0.01\\
237.01	0.01\\
238.01	0.01\\
239.01	0.01\\
240.01	0.01\\
241.01	0.01\\
242.01	0.01\\
243.01	0.01\\
244.01	0.01\\
245.01	0.01\\
246.01	0.01\\
247.01	0.01\\
248.01	0.01\\
249.01	0.01\\
250.01	0.01\\
251.01	0.01\\
252.01	0.01\\
253.01	0.01\\
254.01	0.01\\
255.01	0.01\\
256.01	0.01\\
257.01	0.01\\
258.01	0.01\\
259.01	0.01\\
260.01	0.01\\
261.01	0.01\\
262.01	0.01\\
263.01	0.01\\
264.01	0.01\\
265.01	0.01\\
266.01	0.01\\
267.01	0.01\\
268.01	0.01\\
269.01	0.01\\
270.01	0.01\\
271.01	0.01\\
272.01	0.01\\
273.01	0.01\\
274.01	0.01\\
275.01	0.01\\
276.01	0.01\\
277.01	0.01\\
278.01	0.01\\
279.01	0.01\\
280.01	0.01\\
281.01	0.01\\
282.01	0.01\\
283.01	0.01\\
284.01	0.01\\
285.01	0.01\\
286.01	0.01\\
287.01	0.01\\
288.01	0.01\\
289.01	0.01\\
290.01	0.01\\
291.01	0.01\\
292.01	0.01\\
293.01	0.01\\
294.01	0.01\\
295.01	0.01\\
296.01	0.01\\
297.01	0.01\\
298.01	0.01\\
299.01	0.01\\
300.01	0.01\\
301.01	0.01\\
302.01	0.01\\
303.01	0.01\\
304.01	0.01\\
305.01	0.01\\
306.01	0.01\\
307.01	0.01\\
308.01	0.01\\
309.01	0.01\\
310.01	0.01\\
311.01	0.01\\
312.01	0.01\\
313.01	0.01\\
314.01	0.01\\
315.01	0.01\\
316.01	0.01\\
317.01	0.01\\
318.01	0.01\\
319.01	0.01\\
320.01	0.01\\
321.01	0.01\\
322.01	0.01\\
323.01	0.01\\
324.01	0.01\\
325.01	0.01\\
326.01	0.01\\
327.01	0.01\\
328.01	0.01\\
329.01	0.01\\
330.01	0.01\\
331.01	0.01\\
332.01	0.01\\
333.01	0.01\\
334.01	0.01\\
335.01	0.01\\
336.01	0.01\\
337.01	0.01\\
338.01	0.01\\
339.01	0.01\\
340.01	0.01\\
341.01	0.01\\
342.01	0.01\\
343.01	0.01\\
344.01	0.01\\
345.01	0.01\\
346.01	0.01\\
347.01	0.01\\
348.01	0.01\\
349.01	0.01\\
350.01	0.01\\
351.01	0.01\\
352.01	0.01\\
353.01	0.01\\
354.01	0.01\\
355.01	0.01\\
356.01	0.01\\
357.01	0.01\\
358.01	0.01\\
359.01	0.01\\
360.01	0.01\\
361.01	0.01\\
362.01	0.01\\
363.01	0.01\\
364.01	0.01\\
365.01	0.01\\
366.01	0.01\\
367.01	0.01\\
368.01	0.01\\
369.01	0.01\\
370.01	0.01\\
371.01	0.01\\
372.01	0.01\\
373.01	0.01\\
374.01	0.01\\
375.01	0.01\\
376.01	0.01\\
377.01	0.01\\
378.01	0.01\\
379.01	0.01\\
380.01	0.01\\
381.01	0.01\\
382.01	0.01\\
383.01	0.01\\
384.01	0.01\\
385.01	0.01\\
386.01	0.01\\
387.01	0.01\\
388.01	0.01\\
389.01	0.01\\
390.01	0.01\\
391.01	0.01\\
392.01	0.01\\
393.01	0.01\\
394.01	0.01\\
395.01	0.01\\
396.01	0.01\\
397.01	0.01\\
398.01	0.01\\
399.01	0.01\\
400.01	0.01\\
401.01	0.01\\
402.01	0.01\\
403.01	0.01\\
404.01	0.01\\
405.01	0.01\\
406.01	0.01\\
407.01	0.01\\
408.01	0.01\\
409.01	0.01\\
410.01	0.01\\
411.01	0.01\\
412.01	0.01\\
413.01	0.01\\
414.01	0.01\\
415.01	0.01\\
416.01	0.01\\
417.01	0.01\\
418.01	0.01\\
419.01	0.01\\
420.01	0.01\\
421.01	0.01\\
422.01	0.01\\
423.01	0.01\\
424.01	0.01\\
425.01	0.01\\
426.01	0.01\\
427.01	0.01\\
428.01	0.01\\
429.01	0.01\\
430.01	0.01\\
431.01	0.01\\
432.01	0.01\\
433.01	0.01\\
434.01	0.01\\
435.01	0.01\\
436.01	0.01\\
437.01	0.01\\
438.01	0.01\\
439.01	0.01\\
440.01	0.01\\
441.01	0.01\\
442.01	0.01\\
443.01	0.01\\
444.01	0.01\\
445.01	0.01\\
446.01	0.01\\
447.01	0.01\\
448.01	0.01\\
449.01	0.01\\
450.01	0.01\\
451.01	0.01\\
452.01	0.01\\
453.01	0.01\\
454.01	0.01\\
455.01	0.01\\
456.01	0.01\\
457.01	0.01\\
458.01	0.01\\
459.01	0.01\\
460.01	0.01\\
461.01	0.01\\
462.01	0.01\\
463.01	0.01\\
464.01	0.01\\
465.01	0.01\\
466.01	0.01\\
467.01	0.01\\
468.01	0.01\\
469.01	0.01\\
470.01	0.01\\
471.01	0.01\\
472.01	0.01\\
473.01	0.01\\
474.01	0.01\\
475.01	0.01\\
476.01	0.01\\
477.01	0.01\\
478.01	0.01\\
479.01	0.01\\
480.01	0.01\\
481.01	0.01\\
482.01	0.01\\
483.01	0.01\\
484.01	0.01\\
485.01	0.01\\
486.01	0.01\\
487.01	0.01\\
488.01	0.01\\
489.01	0.01\\
490.01	0.01\\
491.01	0.01\\
492.01	0.01\\
493.01	0.01\\
494.01	0.01\\
495.01	0.01\\
496.01	0.01\\
497.01	0.01\\
498.01	0.01\\
499.01	0.01\\
500.01	0.01\\
501.01	0.01\\
502.01	0.01\\
503.01	0.01\\
504.01	0.01\\
505.01	0.01\\
506.01	0.01\\
507.01	0.01\\
508.01	0.01\\
509.01	0.01\\
510.01	0.01\\
511.01	0.01\\
512.01	0.01\\
513.01	0.01\\
514.01	0.01\\
515.01	0.01\\
516.01	0.01\\
517.01	0.01\\
518.01	0.01\\
519.01	0.01\\
520.01	0.01\\
521.01	0.01\\
522.01	0.01\\
523.01	0.01\\
524.01	0.01\\
525.01	0.01\\
526.01	0.01\\
527.01	0.01\\
528.01	0.01\\
529.01	0.01\\
530.01	0.01\\
531.01	0.01\\
532.01	0.01\\
533.01	0.01\\
534.01	0.01\\
535.01	0.01\\
536.01	0.01\\
537.01	0.01\\
538.01	0.01\\
539.01	0.01\\
540.01	0.01\\
541.01	0.01\\
542.01	0.01\\
543.01	0.01\\
544.01	0.01\\
545.01	0.01\\
546.01	0.01\\
547.01	0.01\\
548.01	0.01\\
549.01	0.01\\
550.01	0.01\\
551.01	0.01\\
552.01	0.01\\
553.01	0.01\\
554.01	0.01\\
555.01	0.01\\
556.01	0.01\\
557.01	0.01\\
558.01	0.01\\
559.01	0.01\\
560.01	0.01\\
561.01	0.01\\
562.01	0.01\\
563.01	0.01\\
564.01	0.01\\
565.01	0.01\\
566.01	0.01\\
567.01	0.01\\
568.01	0.01\\
569.01	0.01\\
570.01	0.01\\
571.01	0.01\\
572.01	0.01\\
573.01	0.01\\
574.01	0.01\\
575.01	0.01\\
576.01	0.01\\
577.01	0.01\\
578.01	0.01\\
579.01	0.01\\
580.01	0.01\\
581.01	0.01\\
582.01	0.01\\
583.01	0.01\\
584.01	0.01\\
585.01	0.01\\
586.01	0.01\\
587.01	0.01\\
588.01	0.01\\
589.01	0.01\\
590.01	0.01\\
591.01	0.01\\
592.01	0.01\\
593.01	0.01\\
594.01	0.01\\
595.01	0.01\\
596.01	0.01\\
597.01	0.01\\
598.01	0.01\\
599.01	0.01\\
599.02	0.01\\
599.03	0.01\\
599.04	0.01\\
599.05	0.01\\
599.06	0.01\\
599.07	0.01\\
599.08	0.01\\
599.09	0.01\\
599.1	0.01\\
599.11	0.01\\
599.12	0.01\\
599.13	0.01\\
599.14	0.01\\
599.15	0.01\\
599.16	0.01\\
599.17	0.01\\
599.18	0.01\\
599.19	0.01\\
599.2	0.01\\
599.21	0.00992567596961163\\
599.22	0.00973552951769219\\
599.23	0.0095441586337754\\
599.24	0.00935154653278383\\
599.25	0.00915767589482951\\
599.26	0.00896252884506647\\
599.27	0.00876609188285174\\
599.28	0.00856835603632599\\
599.29	0.0083693020586219\\
599.3	0.00816891006986826\\
599.31	0.00796715991787855\\
599.32	0.00776403075730489\\
599.33	0.00755950089682049\\
599.34	0.00735354792802892\\
599.35	0.00714614882670781\\
599.36	0.00693727969062358\\
599.37	0.00672691574027605\\
599.38	0.0065150313405578\\
599.39	0.00630159996621509\\
599.4	0.00608659416552601\\
599.41	0.00586998552217157\\
599.42	0.00565174461541249\\
599.43	0.00543184097827676\\
599.44	0.00521024305362824\\
599.45	0.00498691814797608\\
599.46	0.00476183238287344\\
599.47	0.00453495064374151\\
599.48	0.00430623652594129\\
599.49	0.0041843866201124\\
599.5	0.00411909531665794\\
599.51	0.00405317013116703\\
599.52	0.00398660758353459\\
599.53	0.00391939838601633\\
599.54	0.00385153203685256\\
599.55	0.00378300499736274\\
599.56	0.0037138136063696\\
599.57	0.00364395412032243\\
599.58	0.00357342333912867\\
599.59	0.00350221802491313\\
599.6	0.00343033523666813\\
599.61	0.00335777218644165\\
599.62	0.00328452624030823\\
599.63	0.00321059492650609\\
599.64	0.00313597594390693\\
599.65	0.00306066717083488\\
599.66	0.00298466667425192\\
599.67	0.00290797274578559\\
599.68	0.00283058389820869\\
599.69	0.00275249886666979\\
599.7	0.00267371661980331\\
599.71	0.00259423637132924\\
599.72	0.00251405772609498\\
599.73	0.00243318086727999\\
599.74	0.00235160633425846\\
599.75	0.00226933501860845\\
599.76	0.0021863682742743\\
599.77	0.00210270800264456\\
599.78	0.00201835654384337\\
599.79	0.00193331670079489\\
599.8	0.00184759176462326\\
599.81	0.00176118554147482\\
599.82	0.00167410238085579\\
599.83	0.00158634720558587\\
599.84	0.00149792554347528\\
599.85	0.00140884356084117\\
599.86	0.00131910809798803\\
599.87	0.00122872670678631\\
599.88	0.00113770769049358\\
599.89	0.00104606014597406\\
599.9	0.000953794008484178\\
599.91	0.000860920099205182\\
599.92	0.000767450175717955\\
599.93	0.000673396985630919\\
599.94	0.000578774323588566\\
599.95	0.000483597091906617\\
599.96	0.000387881365099733\\
599.97	0.000291644458589374\\
599.98	0.000194905001903006\\
599.99	9.76830167015649e-05\\
600	0\\
};
\addplot [color=mycolor16,solid,forget plot]
  table[row sep=crcr]{%
0.01	0.01\\
1.01	0.01\\
2.01	0.01\\
3.01	0.01\\
4.01	0.01\\
5.01	0.01\\
6.01	0.01\\
7.01	0.01\\
8.01	0.01\\
9.01	0.01\\
10.01	0.01\\
11.01	0.01\\
12.01	0.01\\
13.01	0.01\\
14.01	0.01\\
15.01	0.01\\
16.01	0.01\\
17.01	0.01\\
18.01	0.01\\
19.01	0.01\\
20.01	0.01\\
21.01	0.01\\
22.01	0.01\\
23.01	0.01\\
24.01	0.01\\
25.01	0.01\\
26.01	0.01\\
27.01	0.01\\
28.01	0.01\\
29.01	0.01\\
30.01	0.01\\
31.01	0.01\\
32.01	0.01\\
33.01	0.01\\
34.01	0.01\\
35.01	0.01\\
36.01	0.01\\
37.01	0.01\\
38.01	0.01\\
39.01	0.01\\
40.01	0.01\\
41.01	0.01\\
42.01	0.01\\
43.01	0.01\\
44.01	0.01\\
45.01	0.01\\
46.01	0.01\\
47.01	0.01\\
48.01	0.01\\
49.01	0.01\\
50.01	0.01\\
51.01	0.01\\
52.01	0.01\\
53.01	0.01\\
54.01	0.01\\
55.01	0.01\\
56.01	0.01\\
57.01	0.01\\
58.01	0.01\\
59.01	0.01\\
60.01	0.01\\
61.01	0.01\\
62.01	0.01\\
63.01	0.01\\
64.01	0.01\\
65.01	0.01\\
66.01	0.01\\
67.01	0.01\\
68.01	0.01\\
69.01	0.01\\
70.01	0.01\\
71.01	0.01\\
72.01	0.01\\
73.01	0.01\\
74.01	0.01\\
75.01	0.01\\
76.01	0.01\\
77.01	0.01\\
78.01	0.01\\
79.01	0.01\\
80.01	0.01\\
81.01	0.01\\
82.01	0.01\\
83.01	0.01\\
84.01	0.01\\
85.01	0.01\\
86.01	0.01\\
87.01	0.01\\
88.01	0.01\\
89.01	0.01\\
90.01	0.01\\
91.01	0.01\\
92.01	0.01\\
93.01	0.01\\
94.01	0.01\\
95.01	0.01\\
96.01	0.01\\
97.01	0.01\\
98.01	0.01\\
99.01	0.01\\
100.01	0.01\\
101.01	0.01\\
102.01	0.01\\
103.01	0.01\\
104.01	0.01\\
105.01	0.01\\
106.01	0.01\\
107.01	0.01\\
108.01	0.01\\
109.01	0.01\\
110.01	0.01\\
111.01	0.01\\
112.01	0.01\\
113.01	0.01\\
114.01	0.01\\
115.01	0.01\\
116.01	0.01\\
117.01	0.01\\
118.01	0.01\\
119.01	0.01\\
120.01	0.01\\
121.01	0.01\\
122.01	0.01\\
123.01	0.01\\
124.01	0.01\\
125.01	0.01\\
126.01	0.01\\
127.01	0.01\\
128.01	0.01\\
129.01	0.01\\
130.01	0.01\\
131.01	0.01\\
132.01	0.01\\
133.01	0.01\\
134.01	0.01\\
135.01	0.01\\
136.01	0.01\\
137.01	0.01\\
138.01	0.01\\
139.01	0.01\\
140.01	0.01\\
141.01	0.01\\
142.01	0.01\\
143.01	0.01\\
144.01	0.01\\
145.01	0.01\\
146.01	0.01\\
147.01	0.01\\
148.01	0.01\\
149.01	0.01\\
150.01	0.01\\
151.01	0.01\\
152.01	0.01\\
153.01	0.01\\
154.01	0.01\\
155.01	0.01\\
156.01	0.01\\
157.01	0.01\\
158.01	0.01\\
159.01	0.01\\
160.01	0.01\\
161.01	0.01\\
162.01	0.01\\
163.01	0.01\\
164.01	0.01\\
165.01	0.01\\
166.01	0.01\\
167.01	0.01\\
168.01	0.01\\
169.01	0.01\\
170.01	0.01\\
171.01	0.01\\
172.01	0.01\\
173.01	0.01\\
174.01	0.01\\
175.01	0.01\\
176.01	0.01\\
177.01	0.01\\
178.01	0.01\\
179.01	0.01\\
180.01	0.01\\
181.01	0.01\\
182.01	0.01\\
183.01	0.01\\
184.01	0.01\\
185.01	0.01\\
186.01	0.01\\
187.01	0.01\\
188.01	0.01\\
189.01	0.01\\
190.01	0.01\\
191.01	0.01\\
192.01	0.01\\
193.01	0.01\\
194.01	0.01\\
195.01	0.01\\
196.01	0.01\\
197.01	0.01\\
198.01	0.01\\
199.01	0.01\\
200.01	0.01\\
201.01	0.01\\
202.01	0.01\\
203.01	0.01\\
204.01	0.01\\
205.01	0.01\\
206.01	0.01\\
207.01	0.01\\
208.01	0.01\\
209.01	0.01\\
210.01	0.01\\
211.01	0.01\\
212.01	0.01\\
213.01	0.01\\
214.01	0.01\\
215.01	0.01\\
216.01	0.01\\
217.01	0.01\\
218.01	0.01\\
219.01	0.01\\
220.01	0.01\\
221.01	0.01\\
222.01	0.01\\
223.01	0.01\\
224.01	0.01\\
225.01	0.01\\
226.01	0.01\\
227.01	0.01\\
228.01	0.01\\
229.01	0.01\\
230.01	0.01\\
231.01	0.01\\
232.01	0.01\\
233.01	0.01\\
234.01	0.01\\
235.01	0.01\\
236.01	0.01\\
237.01	0.01\\
238.01	0.01\\
239.01	0.01\\
240.01	0.01\\
241.01	0.01\\
242.01	0.01\\
243.01	0.01\\
244.01	0.01\\
245.01	0.01\\
246.01	0.01\\
247.01	0.01\\
248.01	0.01\\
249.01	0.01\\
250.01	0.01\\
251.01	0.01\\
252.01	0.01\\
253.01	0.01\\
254.01	0.01\\
255.01	0.01\\
256.01	0.01\\
257.01	0.01\\
258.01	0.01\\
259.01	0.01\\
260.01	0.01\\
261.01	0.01\\
262.01	0.01\\
263.01	0.01\\
264.01	0.01\\
265.01	0.01\\
266.01	0.01\\
267.01	0.01\\
268.01	0.01\\
269.01	0.01\\
270.01	0.01\\
271.01	0.01\\
272.01	0.01\\
273.01	0.01\\
274.01	0.01\\
275.01	0.01\\
276.01	0.01\\
277.01	0.01\\
278.01	0.01\\
279.01	0.01\\
280.01	0.01\\
281.01	0.01\\
282.01	0.01\\
283.01	0.01\\
284.01	0.01\\
285.01	0.01\\
286.01	0.01\\
287.01	0.01\\
288.01	0.01\\
289.01	0.01\\
290.01	0.01\\
291.01	0.01\\
292.01	0.01\\
293.01	0.01\\
294.01	0.01\\
295.01	0.01\\
296.01	0.01\\
297.01	0.01\\
298.01	0.01\\
299.01	0.01\\
300.01	0.01\\
301.01	0.01\\
302.01	0.01\\
303.01	0.01\\
304.01	0.01\\
305.01	0.01\\
306.01	0.01\\
307.01	0.01\\
308.01	0.01\\
309.01	0.01\\
310.01	0.01\\
311.01	0.01\\
312.01	0.01\\
313.01	0.01\\
314.01	0.01\\
315.01	0.01\\
316.01	0.01\\
317.01	0.01\\
318.01	0.01\\
319.01	0.01\\
320.01	0.01\\
321.01	0.01\\
322.01	0.01\\
323.01	0.01\\
324.01	0.01\\
325.01	0.01\\
326.01	0.01\\
327.01	0.01\\
328.01	0.01\\
329.01	0.01\\
330.01	0.01\\
331.01	0.01\\
332.01	0.01\\
333.01	0.01\\
334.01	0.01\\
335.01	0.01\\
336.01	0.01\\
337.01	0.01\\
338.01	0.01\\
339.01	0.01\\
340.01	0.01\\
341.01	0.01\\
342.01	0.01\\
343.01	0.01\\
344.01	0.01\\
345.01	0.01\\
346.01	0.01\\
347.01	0.01\\
348.01	0.01\\
349.01	0.01\\
350.01	0.01\\
351.01	0.01\\
352.01	0.01\\
353.01	0.01\\
354.01	0.01\\
355.01	0.01\\
356.01	0.01\\
357.01	0.01\\
358.01	0.01\\
359.01	0.01\\
360.01	0.01\\
361.01	0.01\\
362.01	0.01\\
363.01	0.01\\
364.01	0.01\\
365.01	0.01\\
366.01	0.01\\
367.01	0.01\\
368.01	0.01\\
369.01	0.01\\
370.01	0.01\\
371.01	0.01\\
372.01	0.01\\
373.01	0.01\\
374.01	0.01\\
375.01	0.01\\
376.01	0.01\\
377.01	0.01\\
378.01	0.01\\
379.01	0.01\\
380.01	0.01\\
381.01	0.01\\
382.01	0.01\\
383.01	0.01\\
384.01	0.01\\
385.01	0.01\\
386.01	0.01\\
387.01	0.01\\
388.01	0.01\\
389.01	0.01\\
390.01	0.01\\
391.01	0.01\\
392.01	0.01\\
393.01	0.01\\
394.01	0.01\\
395.01	0.01\\
396.01	0.01\\
397.01	0.01\\
398.01	0.01\\
399.01	0.01\\
400.01	0.01\\
401.01	0.01\\
402.01	0.01\\
403.01	0.01\\
404.01	0.01\\
405.01	0.01\\
406.01	0.01\\
407.01	0.01\\
408.01	0.01\\
409.01	0.01\\
410.01	0.01\\
411.01	0.01\\
412.01	0.01\\
413.01	0.01\\
414.01	0.01\\
415.01	0.01\\
416.01	0.01\\
417.01	0.01\\
418.01	0.01\\
419.01	0.01\\
420.01	0.01\\
421.01	0.01\\
422.01	0.01\\
423.01	0.01\\
424.01	0.01\\
425.01	0.01\\
426.01	0.01\\
427.01	0.01\\
428.01	0.01\\
429.01	0.01\\
430.01	0.01\\
431.01	0.01\\
432.01	0.01\\
433.01	0.01\\
434.01	0.01\\
435.01	0.01\\
436.01	0.01\\
437.01	0.01\\
438.01	0.01\\
439.01	0.01\\
440.01	0.01\\
441.01	0.01\\
442.01	0.01\\
443.01	0.01\\
444.01	0.01\\
445.01	0.01\\
446.01	0.01\\
447.01	0.01\\
448.01	0.01\\
449.01	0.01\\
450.01	0.01\\
451.01	0.01\\
452.01	0.01\\
453.01	0.01\\
454.01	0.01\\
455.01	0.01\\
456.01	0.01\\
457.01	0.01\\
458.01	0.01\\
459.01	0.01\\
460.01	0.01\\
461.01	0.01\\
462.01	0.01\\
463.01	0.01\\
464.01	0.01\\
465.01	0.01\\
466.01	0.01\\
467.01	0.01\\
468.01	0.01\\
469.01	0.01\\
470.01	0.01\\
471.01	0.01\\
472.01	0.01\\
473.01	0.01\\
474.01	0.01\\
475.01	0.01\\
476.01	0.01\\
477.01	0.01\\
478.01	0.01\\
479.01	0.01\\
480.01	0.01\\
481.01	0.01\\
482.01	0.01\\
483.01	0.01\\
484.01	0.01\\
485.01	0.01\\
486.01	0.01\\
487.01	0.01\\
488.01	0.01\\
489.01	0.01\\
490.01	0.01\\
491.01	0.01\\
492.01	0.01\\
493.01	0.01\\
494.01	0.01\\
495.01	0.01\\
496.01	0.01\\
497.01	0.01\\
498.01	0.01\\
499.01	0.01\\
500.01	0.01\\
501.01	0.01\\
502.01	0.01\\
503.01	0.01\\
504.01	0.01\\
505.01	0.01\\
506.01	0.01\\
507.01	0.01\\
508.01	0.01\\
509.01	0.01\\
510.01	0.01\\
511.01	0.01\\
512.01	0.01\\
513.01	0.01\\
514.01	0.01\\
515.01	0.01\\
516.01	0.01\\
517.01	0.01\\
518.01	0.01\\
519.01	0.01\\
520.01	0.01\\
521.01	0.01\\
522.01	0.01\\
523.01	0.01\\
524.01	0.01\\
525.01	0.01\\
526.01	0.01\\
527.01	0.01\\
528.01	0.01\\
529.01	0.01\\
530.01	0.01\\
531.01	0.01\\
532.01	0.01\\
533.01	0.01\\
534.01	0.01\\
535.01	0.01\\
536.01	0.01\\
537.01	0.01\\
538.01	0.01\\
539.01	0.01\\
540.01	0.01\\
541.01	0.01\\
542.01	0.01\\
543.01	0.01\\
544.01	0.01\\
545.01	0.01\\
546.01	0.01\\
547.01	0.01\\
548.01	0.01\\
549.01	0.01\\
550.01	0.01\\
551.01	0.01\\
552.01	0.01\\
553.01	0.01\\
554.01	0.01\\
555.01	0.01\\
556.01	0.01\\
557.01	0.01\\
558.01	0.01\\
559.01	0.01\\
560.01	0.01\\
561.01	0.01\\
562.01	0.01\\
563.01	0.01\\
564.01	0.01\\
565.01	0.01\\
566.01	0.01\\
567.01	0.01\\
568.01	0.01\\
569.01	0.01\\
570.01	0.01\\
571.01	0.01\\
572.01	0.01\\
573.01	0.01\\
574.01	0.01\\
575.01	0.01\\
576.01	0.01\\
577.01	0.01\\
578.01	0.01\\
579.01	0.01\\
580.01	0.01\\
581.01	0.01\\
582.01	0.01\\
583.01	0.01\\
584.01	0.01\\
585.01	0.01\\
586.01	0.01\\
587.01	0.01\\
588.01	0.01\\
589.01	0.01\\
590.01	0.01\\
591.01	0.01\\
592.01	0.01\\
593.01	0.01\\
594.01	0.01\\
595.01	0.01\\
596.01	0.01\\
597.01	0.01\\
598.01	0.01\\
599.01	0.00988739368402132\\
599.02	0.00968074340707133\\
599.03	0.00947259332618321\\
599.04	0.00926292139085469\\
599.05	0.00905170483194615\\
599.06	0.00883892013221553\\
599.07	0.00862454299548213\\
599.08	0.00840855188409677\\
599.09	0.00819093287807366\\
599.1	0.00797166046229913\\
599.11	0.00775070766774859\\
599.12	0.00752804673107546\\
599.13	0.00730364946987773\\
599.14	0.00707748601739125\\
599.15	0.00684952544651139\\
599.16	0.00661973572330175\\
599.17	0.00638808372306731\\
599.18	0.00615453520876707\\
599.19	0.00591905452880597\\
599.2	0.00568160471301551\\
599.21	0.00551663477868118\\
599.22	0.00546587110992685\\
599.23	0.00541470224664812\\
599.24	0.00536313135198442\\
599.25	0.00531116197147127\\
599.26	0.00525879805221435\\
599.27	0.00520603898014746\\
599.28	0.00515287938995751\\
599.29	0.00509932402820601\\
599.3	0.00504537811029469\\
599.31	0.00499104695446077\\
599.32	0.00493633640166715\\
599.33	0.0048812531933729\\
599.34	0.00482580463929975\\
599.35	0.00476999849829594\\
599.36	0.00471384324138586\\
599.37	0.00465734805159418\\
599.38	0.00460052280219023\\
599.39	0.00454337809122961\\
599.4	0.00448592528632182\\
599.41	0.00442817656636471\\
599.42	0.00437014495065976\\
599.43	0.00431184434152103\\
599.44	0.00425328956922003\\
599.45	0.00419449643942142\\
599.46	0.00413548178327611\\
599.47	0.00407626351035228\\
599.48	0.0040168606645986\\
599.49	0.00395702039573175\\
599.5	0.00389659954551457\\
599.51	0.00383559266882257\\
599.52	0.00377399426818631\\
599.53	0.00371179882538318\\
599.54	0.00364900081011568\\
599.55	0.00358559464277508\\
599.56	0.00352157469499213\\
599.57	0.00345693529004798\\
599.58	0.00339167069943865\\
599.59	0.00332577514300587\\
599.6	0.00325924278682438\\
599.61	0.00319206774199901\\
599.62	0.0031242440634364\\
599.63	0.00305576574854174\\
599.64	0.00298662673583587\\
599.65	0.00291682090348755\\
599.66	0.00284634206775538\\
599.67	0.00277518398125874\\
599.68	0.00270334033118015\\
599.69	0.00263080473738273\\
599.7	0.00255757075041003\\
599.71	0.00248363184936088\\
599.72	0.0024089814392363\\
599.73	0.00233361284760038\\
599.74	0.00225751932173361\\
599.75	0.00218069402566163\\
599.76	0.00210313003670668\\
599.77	0.0020248203416071\\
599.78	0.00194575783277725\\
599.79	0.00186593530430994\\
599.8	0.00178534544770319\\
599.81	0.00170398084729193\\
599.82	0.0016218339753637\\
599.83	0.00153889718693601\\
599.84	0.00145516271417133\\
599.85	0.00137062266040393\\
599.86	0.00128526899375086\\
599.87	0.00119909354027743\\
599.88	0.00111208797668507\\
599.89	0.00102424382248756\\
599.9	0.000935552431638535\\
599.91	0.000846004983570773\\
599.92	0.000755592473604684\\
599.93	0.000664305702680239\\
599.94	0.000572135266363093\\
599.95	0.000479071543072036\\
599.96	0.000385104681470808\\
599.97	0.000290224586963043\\
599.98	0.000194420907224489\\
599.99	9.76830167015649e-05\\
600	0\\
};
\addplot [color=mycolor17,solid,forget plot]
  table[row sep=crcr]{%
0.01	0.01\\
1.01	0.01\\
2.01	0.01\\
3.01	0.01\\
4.01	0.01\\
5.01	0.01\\
6.01	0.01\\
7.01	0.01\\
8.01	0.01\\
9.01	0.01\\
10.01	0.01\\
11.01	0.01\\
12.01	0.01\\
13.01	0.01\\
14.01	0.01\\
15.01	0.01\\
16.01	0.01\\
17.01	0.01\\
18.01	0.01\\
19.01	0.01\\
20.01	0.01\\
21.01	0.01\\
22.01	0.01\\
23.01	0.01\\
24.01	0.01\\
25.01	0.01\\
26.01	0.01\\
27.01	0.01\\
28.01	0.01\\
29.01	0.01\\
30.01	0.01\\
31.01	0.01\\
32.01	0.01\\
33.01	0.01\\
34.01	0.01\\
35.01	0.01\\
36.01	0.01\\
37.01	0.01\\
38.01	0.01\\
39.01	0.01\\
40.01	0.01\\
41.01	0.01\\
42.01	0.01\\
43.01	0.01\\
44.01	0.01\\
45.01	0.01\\
46.01	0.01\\
47.01	0.01\\
48.01	0.01\\
49.01	0.01\\
50.01	0.01\\
51.01	0.01\\
52.01	0.01\\
53.01	0.01\\
54.01	0.01\\
55.01	0.01\\
56.01	0.01\\
57.01	0.01\\
58.01	0.01\\
59.01	0.01\\
60.01	0.01\\
61.01	0.01\\
62.01	0.01\\
63.01	0.01\\
64.01	0.01\\
65.01	0.01\\
66.01	0.01\\
67.01	0.01\\
68.01	0.01\\
69.01	0.01\\
70.01	0.01\\
71.01	0.01\\
72.01	0.01\\
73.01	0.01\\
74.01	0.01\\
75.01	0.01\\
76.01	0.01\\
77.01	0.01\\
78.01	0.01\\
79.01	0.01\\
80.01	0.01\\
81.01	0.01\\
82.01	0.01\\
83.01	0.01\\
84.01	0.01\\
85.01	0.01\\
86.01	0.01\\
87.01	0.01\\
88.01	0.01\\
89.01	0.01\\
90.01	0.01\\
91.01	0.01\\
92.01	0.01\\
93.01	0.01\\
94.01	0.01\\
95.01	0.01\\
96.01	0.01\\
97.01	0.01\\
98.01	0.01\\
99.01	0.01\\
100.01	0.01\\
101.01	0.01\\
102.01	0.01\\
103.01	0.01\\
104.01	0.01\\
105.01	0.01\\
106.01	0.01\\
107.01	0.01\\
108.01	0.01\\
109.01	0.01\\
110.01	0.01\\
111.01	0.01\\
112.01	0.01\\
113.01	0.01\\
114.01	0.01\\
115.01	0.01\\
116.01	0.01\\
117.01	0.01\\
118.01	0.01\\
119.01	0.01\\
120.01	0.01\\
121.01	0.01\\
122.01	0.01\\
123.01	0.01\\
124.01	0.01\\
125.01	0.01\\
126.01	0.01\\
127.01	0.01\\
128.01	0.01\\
129.01	0.01\\
130.01	0.01\\
131.01	0.01\\
132.01	0.01\\
133.01	0.01\\
134.01	0.01\\
135.01	0.01\\
136.01	0.01\\
137.01	0.01\\
138.01	0.01\\
139.01	0.01\\
140.01	0.01\\
141.01	0.01\\
142.01	0.01\\
143.01	0.01\\
144.01	0.01\\
145.01	0.01\\
146.01	0.01\\
147.01	0.01\\
148.01	0.01\\
149.01	0.01\\
150.01	0.01\\
151.01	0.01\\
152.01	0.01\\
153.01	0.01\\
154.01	0.01\\
155.01	0.01\\
156.01	0.01\\
157.01	0.01\\
158.01	0.01\\
159.01	0.01\\
160.01	0.01\\
161.01	0.01\\
162.01	0.01\\
163.01	0.01\\
164.01	0.01\\
165.01	0.01\\
166.01	0.01\\
167.01	0.01\\
168.01	0.01\\
169.01	0.01\\
170.01	0.01\\
171.01	0.01\\
172.01	0.01\\
173.01	0.01\\
174.01	0.01\\
175.01	0.01\\
176.01	0.01\\
177.01	0.01\\
178.01	0.01\\
179.01	0.01\\
180.01	0.01\\
181.01	0.01\\
182.01	0.01\\
183.01	0.01\\
184.01	0.01\\
185.01	0.01\\
186.01	0.01\\
187.01	0.01\\
188.01	0.01\\
189.01	0.01\\
190.01	0.01\\
191.01	0.01\\
192.01	0.01\\
193.01	0.01\\
194.01	0.01\\
195.01	0.01\\
196.01	0.01\\
197.01	0.01\\
198.01	0.01\\
199.01	0.01\\
200.01	0.01\\
201.01	0.01\\
202.01	0.01\\
203.01	0.01\\
204.01	0.01\\
205.01	0.01\\
206.01	0.01\\
207.01	0.01\\
208.01	0.01\\
209.01	0.01\\
210.01	0.01\\
211.01	0.01\\
212.01	0.01\\
213.01	0.01\\
214.01	0.01\\
215.01	0.01\\
216.01	0.01\\
217.01	0.01\\
218.01	0.01\\
219.01	0.01\\
220.01	0.01\\
221.01	0.01\\
222.01	0.01\\
223.01	0.01\\
224.01	0.01\\
225.01	0.01\\
226.01	0.01\\
227.01	0.01\\
228.01	0.01\\
229.01	0.01\\
230.01	0.01\\
231.01	0.01\\
232.01	0.01\\
233.01	0.01\\
234.01	0.01\\
235.01	0.01\\
236.01	0.01\\
237.01	0.01\\
238.01	0.01\\
239.01	0.01\\
240.01	0.01\\
241.01	0.01\\
242.01	0.01\\
243.01	0.01\\
244.01	0.01\\
245.01	0.01\\
246.01	0.01\\
247.01	0.01\\
248.01	0.01\\
249.01	0.01\\
250.01	0.01\\
251.01	0.01\\
252.01	0.01\\
253.01	0.01\\
254.01	0.01\\
255.01	0.01\\
256.01	0.01\\
257.01	0.01\\
258.01	0.01\\
259.01	0.01\\
260.01	0.01\\
261.01	0.01\\
262.01	0.01\\
263.01	0.01\\
264.01	0.01\\
265.01	0.01\\
266.01	0.01\\
267.01	0.01\\
268.01	0.01\\
269.01	0.01\\
270.01	0.01\\
271.01	0.01\\
272.01	0.01\\
273.01	0.01\\
274.01	0.01\\
275.01	0.01\\
276.01	0.01\\
277.01	0.01\\
278.01	0.01\\
279.01	0.01\\
280.01	0.01\\
281.01	0.01\\
282.01	0.01\\
283.01	0.01\\
284.01	0.01\\
285.01	0.01\\
286.01	0.01\\
287.01	0.01\\
288.01	0.01\\
289.01	0.01\\
290.01	0.01\\
291.01	0.01\\
292.01	0.01\\
293.01	0.01\\
294.01	0.01\\
295.01	0.01\\
296.01	0.01\\
297.01	0.01\\
298.01	0.01\\
299.01	0.01\\
300.01	0.01\\
301.01	0.01\\
302.01	0.01\\
303.01	0.01\\
304.01	0.01\\
305.01	0.01\\
306.01	0.01\\
307.01	0.01\\
308.01	0.01\\
309.01	0.01\\
310.01	0.01\\
311.01	0.01\\
312.01	0.01\\
313.01	0.01\\
314.01	0.01\\
315.01	0.01\\
316.01	0.01\\
317.01	0.01\\
318.01	0.01\\
319.01	0.01\\
320.01	0.01\\
321.01	0.01\\
322.01	0.01\\
323.01	0.01\\
324.01	0.01\\
325.01	0.01\\
326.01	0.01\\
327.01	0.01\\
328.01	0.01\\
329.01	0.01\\
330.01	0.01\\
331.01	0.01\\
332.01	0.01\\
333.01	0.01\\
334.01	0.01\\
335.01	0.01\\
336.01	0.01\\
337.01	0.01\\
338.01	0.01\\
339.01	0.01\\
340.01	0.01\\
341.01	0.01\\
342.01	0.01\\
343.01	0.01\\
344.01	0.01\\
345.01	0.01\\
346.01	0.01\\
347.01	0.01\\
348.01	0.01\\
349.01	0.01\\
350.01	0.01\\
351.01	0.01\\
352.01	0.01\\
353.01	0.01\\
354.01	0.01\\
355.01	0.01\\
356.01	0.01\\
357.01	0.01\\
358.01	0.01\\
359.01	0.01\\
360.01	0.01\\
361.01	0.01\\
362.01	0.01\\
363.01	0.01\\
364.01	0.01\\
365.01	0.01\\
366.01	0.01\\
367.01	0.01\\
368.01	0.01\\
369.01	0.01\\
370.01	0.01\\
371.01	0.01\\
372.01	0.01\\
373.01	0.01\\
374.01	0.01\\
375.01	0.01\\
376.01	0.01\\
377.01	0.01\\
378.01	0.01\\
379.01	0.01\\
380.01	0.01\\
381.01	0.01\\
382.01	0.01\\
383.01	0.01\\
384.01	0.01\\
385.01	0.01\\
386.01	0.01\\
387.01	0.01\\
388.01	0.01\\
389.01	0.01\\
390.01	0.01\\
391.01	0.01\\
392.01	0.01\\
393.01	0.01\\
394.01	0.01\\
395.01	0.01\\
396.01	0.01\\
397.01	0.01\\
398.01	0.01\\
399.01	0.01\\
400.01	0.01\\
401.01	0.01\\
402.01	0.01\\
403.01	0.01\\
404.01	0.01\\
405.01	0.01\\
406.01	0.01\\
407.01	0.01\\
408.01	0.01\\
409.01	0.01\\
410.01	0.01\\
411.01	0.01\\
412.01	0.01\\
413.01	0.01\\
414.01	0.01\\
415.01	0.01\\
416.01	0.01\\
417.01	0.01\\
418.01	0.01\\
419.01	0.01\\
420.01	0.01\\
421.01	0.01\\
422.01	0.01\\
423.01	0.01\\
424.01	0.01\\
425.01	0.01\\
426.01	0.01\\
427.01	0.01\\
428.01	0.01\\
429.01	0.01\\
430.01	0.01\\
431.01	0.01\\
432.01	0.01\\
433.01	0.01\\
434.01	0.01\\
435.01	0.01\\
436.01	0.01\\
437.01	0.01\\
438.01	0.01\\
439.01	0.01\\
440.01	0.01\\
441.01	0.01\\
442.01	0.01\\
443.01	0.01\\
444.01	0.01\\
445.01	0.01\\
446.01	0.01\\
447.01	0.01\\
448.01	0.01\\
449.01	0.01\\
450.01	0.01\\
451.01	0.01\\
452.01	0.01\\
453.01	0.01\\
454.01	0.01\\
455.01	0.01\\
456.01	0.01\\
457.01	0.01\\
458.01	0.01\\
459.01	0.01\\
460.01	0.01\\
461.01	0.01\\
462.01	0.01\\
463.01	0.01\\
464.01	0.01\\
465.01	0.01\\
466.01	0.01\\
467.01	0.01\\
468.01	0.01\\
469.01	0.01\\
470.01	0.01\\
471.01	0.01\\
472.01	0.01\\
473.01	0.01\\
474.01	0.01\\
475.01	0.01\\
476.01	0.01\\
477.01	0.01\\
478.01	0.01\\
479.01	0.01\\
480.01	0.01\\
481.01	0.01\\
482.01	0.01\\
483.01	0.01\\
484.01	0.01\\
485.01	0.01\\
486.01	0.01\\
487.01	0.01\\
488.01	0.01\\
489.01	0.01\\
490.01	0.01\\
491.01	0.01\\
492.01	0.01\\
493.01	0.01\\
494.01	0.01\\
495.01	0.01\\
496.01	0.01\\
497.01	0.01\\
498.01	0.01\\
499.01	0.01\\
500.01	0.01\\
501.01	0.01\\
502.01	0.01\\
503.01	0.01\\
504.01	0.01\\
505.01	0.01\\
506.01	0.01\\
507.01	0.01\\
508.01	0.01\\
509.01	0.01\\
510.01	0.01\\
511.01	0.01\\
512.01	0.01\\
513.01	0.01\\
514.01	0.01\\
515.01	0.01\\
516.01	0.01\\
517.01	0.01\\
518.01	0.01\\
519.01	0.01\\
520.01	0.01\\
521.01	0.01\\
522.01	0.01\\
523.01	0.01\\
524.01	0.01\\
525.01	0.01\\
526.01	0.01\\
527.01	0.01\\
528.01	0.01\\
529.01	0.01\\
530.01	0.01\\
531.01	0.01\\
532.01	0.01\\
533.01	0.01\\
534.01	0.01\\
535.01	0.01\\
536.01	0.01\\
537.01	0.01\\
538.01	0.01\\
539.01	0.01\\
540.01	0.01\\
541.01	0.01\\
542.01	0.01\\
543.01	0.01\\
544.01	0.01\\
545.01	0.01\\
546.01	0.01\\
547.01	0.01\\
548.01	0.01\\
549.01	0.01\\
550.01	0.01\\
551.01	0.01\\
552.01	0.01\\
553.01	0.01\\
554.01	0.01\\
555.01	0.01\\
556.01	0.01\\
557.01	0.01\\
558.01	0.01\\
559.01	0.01\\
560.01	0.01\\
561.01	0.01\\
562.01	0.01\\
563.01	0.01\\
564.01	0.01\\
565.01	0.01\\
566.01	0.01\\
567.01	0.01\\
568.01	0.01\\
569.01	0.01\\
570.01	0.01\\
571.01	0.01\\
572.01	0.01\\
573.01	0.01\\
574.01	0.01\\
575.01	0.01\\
576.01	0.01\\
577.01	0.01\\
578.01	0.01\\
579.01	0.01\\
580.01	0.01\\
581.01	0.01\\
582.01	0.01\\
583.01	0.01\\
584.01	0.01\\
585.01	0.01\\
586.01	0.01\\
587.01	0.01\\
588.01	0.01\\
589.01	0.01\\
590.01	0.01\\
591.01	0.01\\
592.01	0.01\\
593.01	0.01\\
594.01	0.01\\
595.01	0.01\\
596.01	0.01\\
597.01	0.01\\
598.01	0.01\\
599.01	0.00629791790588287\\
599.02	0.00625576917955417\\
599.03	0.00621335984443891\\
599.04	0.00617069785691429\\
599.05	0.00612779157677909\\
599.06	0.00608464998542295\\
599.07	0.0060412827172068\\
599.08	0.0059976965053873\\
599.09	0.00595389027958764\\
599.1	0.00590987443069723\\
599.11	0.00586566091945395\\
599.12	0.00582126238082717\\
599.13	0.00577669174281891\\
599.14	0.00573196349637267\\
599.15	0.00568709307297772\\
599.16	0.00564209689300802\\
599.17	0.00559699235136999\\
599.18	0.00555179784086047\\
599.19	0.00550653305777904\\
599.2	0.00546121890881882\\
599.21	0.00541571379251483\\
599.22	0.00536977225595453\\
599.23	0.00532339010397267\\
599.24	0.00527656308367495\\
599.25	0.00522928688212764\\
599.26	0.00518155712390623\\
599.27	0.00513336939271845\\
599.28	0.00508471925558006\\
599.29	0.00503560221536904\\
599.3	0.00498601370813244\\
599.31	0.00493594910233537\\
599.32	0.00488540369593917\\
599.33	0.00483437271134595\\
599.34	0.00478285129200968\\
599.35	0.00473083449973311\\
599.36	0.00467831731039358\\
599.37	0.00462529460959125\\
599.38	0.00457176118836954\\
599.39	0.00451771174234299\\
599.4	0.00446314087797853\\
599.41	0.00440804309309745\\
599.42	0.00435241277147536\\
599.43	0.0042962441771187\\
599.44	0.00423953144819901\\
599.45	0.00418226859062502\\
599.46	0.00412444947123163\\
599.47	0.00406606781056341\\
599.48	0.00400711717522909\\
599.49	0.00394759166761367\\
599.5	0.00388748567419746\\
599.51	0.00382679352639224\\
599.52	0.00376550949999173\\
599.53	0.00370362781448029\\
599.54	0.00364114263228638\\
599.55	0.00357804805817696\\
599.56	0.00351433813863804\\
599.57	0.00345000686124116\\
599.58	0.00338504815401388\\
599.59	0.00331945588479675\\
599.6	0.00325322386059824\\
599.61	0.00318634582694277\\
599.62	0.00311881546721176\\
599.63	0.00305062640197774\\
599.64	0.00298177218833173\\
599.65	0.00291224631920415\\
599.66	0.00284204222267936\\
599.67	0.00277115326130438\\
599.68	0.002699572731392\\
599.69	0.0026272938623186\\
599.7	0.00255430981581701\\
599.71	0.00248061368526496\\
599.72	0.00240619849497075\\
599.73	0.00233105719945854\\
599.74	0.00225518268275211\\
599.75	0.00217856775765777\\
599.76	0.00210120516504843\\
599.77	0.00202308757315053\\
599.78	0.00194420757683406\\
599.79	0.00186455769690723\\
599.8	0.0017841303794173\\
599.81	0.00170291799495944\\
599.82	0.00162091283799561\\
599.83	0.00153810712618569\\
599.84	0.00145449299973336\\
599.85	0.00137006252074928\\
599.86	0.00128480767263492\\
599.87	0.00119872035948991\\
599.88	0.00111179240554713\\
599.89	0.00102401555463903\\
599.9	0.000935381469700161\\
599.91	0.000845881732310544\\
599.92	0.000755507842285438\\
599.93	0.000664251217317542\\
599.94	0.000572103192678209\\
599.95	0.000479055020985011\\
599.96	0.000385097872043671\\
599.97	0.000290222832773275\\
599.98	0.000194420907224489\\
599.99	9.76830167015649e-05\\
600	0\\
};
\addplot [color=mycolor18,solid,forget plot]
  table[row sep=crcr]{%
0.01	0.01\\
1.01	0.01\\
2.01	0.01\\
3.01	0.01\\
4.01	0.01\\
5.01	0.01\\
6.01	0.01\\
7.01	0.01\\
8.01	0.01\\
9.01	0.01\\
10.01	0.01\\
11.01	0.01\\
12.01	0.01\\
13.01	0.01\\
14.01	0.01\\
15.01	0.01\\
16.01	0.01\\
17.01	0.01\\
18.01	0.01\\
19.01	0.01\\
20.01	0.01\\
21.01	0.01\\
22.01	0.01\\
23.01	0.01\\
24.01	0.01\\
25.01	0.01\\
26.01	0.01\\
27.01	0.01\\
28.01	0.01\\
29.01	0.01\\
30.01	0.01\\
31.01	0.01\\
32.01	0.01\\
33.01	0.01\\
34.01	0.01\\
35.01	0.01\\
36.01	0.01\\
37.01	0.01\\
38.01	0.01\\
39.01	0.01\\
40.01	0.01\\
41.01	0.01\\
42.01	0.01\\
43.01	0.01\\
44.01	0.01\\
45.01	0.01\\
46.01	0.01\\
47.01	0.01\\
48.01	0.01\\
49.01	0.01\\
50.01	0.01\\
51.01	0.01\\
52.01	0.01\\
53.01	0.01\\
54.01	0.01\\
55.01	0.01\\
56.01	0.01\\
57.01	0.01\\
58.01	0.01\\
59.01	0.01\\
60.01	0.01\\
61.01	0.01\\
62.01	0.01\\
63.01	0.01\\
64.01	0.01\\
65.01	0.01\\
66.01	0.01\\
67.01	0.01\\
68.01	0.01\\
69.01	0.01\\
70.01	0.01\\
71.01	0.01\\
72.01	0.01\\
73.01	0.01\\
74.01	0.01\\
75.01	0.01\\
76.01	0.01\\
77.01	0.01\\
78.01	0.01\\
79.01	0.01\\
80.01	0.01\\
81.01	0.01\\
82.01	0.01\\
83.01	0.01\\
84.01	0.01\\
85.01	0.01\\
86.01	0.01\\
87.01	0.01\\
88.01	0.01\\
89.01	0.01\\
90.01	0.01\\
91.01	0.01\\
92.01	0.01\\
93.01	0.01\\
94.01	0.01\\
95.01	0.01\\
96.01	0.01\\
97.01	0.01\\
98.01	0.01\\
99.01	0.01\\
100.01	0.01\\
101.01	0.01\\
102.01	0.01\\
103.01	0.01\\
104.01	0.01\\
105.01	0.01\\
106.01	0.01\\
107.01	0.01\\
108.01	0.01\\
109.01	0.01\\
110.01	0.01\\
111.01	0.01\\
112.01	0.01\\
113.01	0.01\\
114.01	0.01\\
115.01	0.01\\
116.01	0.01\\
117.01	0.01\\
118.01	0.01\\
119.01	0.01\\
120.01	0.01\\
121.01	0.01\\
122.01	0.01\\
123.01	0.01\\
124.01	0.01\\
125.01	0.01\\
126.01	0.01\\
127.01	0.01\\
128.01	0.01\\
129.01	0.01\\
130.01	0.01\\
131.01	0.01\\
132.01	0.01\\
133.01	0.01\\
134.01	0.01\\
135.01	0.01\\
136.01	0.01\\
137.01	0.01\\
138.01	0.01\\
139.01	0.01\\
140.01	0.01\\
141.01	0.01\\
142.01	0.01\\
143.01	0.01\\
144.01	0.01\\
145.01	0.01\\
146.01	0.01\\
147.01	0.01\\
148.01	0.01\\
149.01	0.01\\
150.01	0.01\\
151.01	0.01\\
152.01	0.01\\
153.01	0.01\\
154.01	0.01\\
155.01	0.01\\
156.01	0.01\\
157.01	0.01\\
158.01	0.01\\
159.01	0.01\\
160.01	0.01\\
161.01	0.01\\
162.01	0.01\\
163.01	0.01\\
164.01	0.01\\
165.01	0.01\\
166.01	0.01\\
167.01	0.01\\
168.01	0.01\\
169.01	0.01\\
170.01	0.01\\
171.01	0.01\\
172.01	0.01\\
173.01	0.01\\
174.01	0.01\\
175.01	0.01\\
176.01	0.01\\
177.01	0.01\\
178.01	0.01\\
179.01	0.01\\
180.01	0.01\\
181.01	0.01\\
182.01	0.01\\
183.01	0.01\\
184.01	0.01\\
185.01	0.01\\
186.01	0.01\\
187.01	0.01\\
188.01	0.01\\
189.01	0.01\\
190.01	0.01\\
191.01	0.01\\
192.01	0.01\\
193.01	0.01\\
194.01	0.01\\
195.01	0.01\\
196.01	0.01\\
197.01	0.01\\
198.01	0.01\\
199.01	0.01\\
200.01	0.01\\
201.01	0.01\\
202.01	0.01\\
203.01	0.01\\
204.01	0.01\\
205.01	0.01\\
206.01	0.01\\
207.01	0.01\\
208.01	0.01\\
209.01	0.01\\
210.01	0.01\\
211.01	0.01\\
212.01	0.01\\
213.01	0.01\\
214.01	0.01\\
215.01	0.01\\
216.01	0.01\\
217.01	0.01\\
218.01	0.01\\
219.01	0.01\\
220.01	0.01\\
221.01	0.01\\
222.01	0.01\\
223.01	0.01\\
224.01	0.01\\
225.01	0.01\\
226.01	0.01\\
227.01	0.01\\
228.01	0.01\\
229.01	0.01\\
230.01	0.01\\
231.01	0.01\\
232.01	0.01\\
233.01	0.01\\
234.01	0.01\\
235.01	0.01\\
236.01	0.01\\
237.01	0.01\\
238.01	0.01\\
239.01	0.01\\
240.01	0.01\\
241.01	0.01\\
242.01	0.01\\
243.01	0.01\\
244.01	0.01\\
245.01	0.01\\
246.01	0.01\\
247.01	0.01\\
248.01	0.01\\
249.01	0.01\\
250.01	0.01\\
251.01	0.01\\
252.01	0.01\\
253.01	0.01\\
254.01	0.01\\
255.01	0.01\\
256.01	0.01\\
257.01	0.01\\
258.01	0.01\\
259.01	0.01\\
260.01	0.01\\
261.01	0.01\\
262.01	0.01\\
263.01	0.01\\
264.01	0.01\\
265.01	0.01\\
266.01	0.01\\
267.01	0.01\\
268.01	0.01\\
269.01	0.01\\
270.01	0.01\\
271.01	0.01\\
272.01	0.01\\
273.01	0.01\\
274.01	0.01\\
275.01	0.01\\
276.01	0.01\\
277.01	0.01\\
278.01	0.01\\
279.01	0.01\\
280.01	0.01\\
281.01	0.01\\
282.01	0.01\\
283.01	0.01\\
284.01	0.01\\
285.01	0.01\\
286.01	0.01\\
287.01	0.01\\
288.01	0.01\\
289.01	0.01\\
290.01	0.01\\
291.01	0.01\\
292.01	0.01\\
293.01	0.01\\
294.01	0.01\\
295.01	0.01\\
296.01	0.01\\
297.01	0.01\\
298.01	0.01\\
299.01	0.01\\
300.01	0.01\\
301.01	0.01\\
302.01	0.01\\
303.01	0.01\\
304.01	0.01\\
305.01	0.01\\
306.01	0.01\\
307.01	0.01\\
308.01	0.01\\
309.01	0.01\\
310.01	0.01\\
311.01	0.01\\
312.01	0.01\\
313.01	0.01\\
314.01	0.01\\
315.01	0.01\\
316.01	0.01\\
317.01	0.01\\
318.01	0.01\\
319.01	0.01\\
320.01	0.01\\
321.01	0.01\\
322.01	0.01\\
323.01	0.01\\
324.01	0.01\\
325.01	0.01\\
326.01	0.01\\
327.01	0.01\\
328.01	0.01\\
329.01	0.01\\
330.01	0.01\\
331.01	0.01\\
332.01	0.01\\
333.01	0.01\\
334.01	0.01\\
335.01	0.01\\
336.01	0.01\\
337.01	0.01\\
338.01	0.01\\
339.01	0.01\\
340.01	0.01\\
341.01	0.01\\
342.01	0.01\\
343.01	0.01\\
344.01	0.01\\
345.01	0.01\\
346.01	0.01\\
347.01	0.01\\
348.01	0.01\\
349.01	0.01\\
350.01	0.01\\
351.01	0.01\\
352.01	0.01\\
353.01	0.01\\
354.01	0.01\\
355.01	0.01\\
356.01	0.01\\
357.01	0.01\\
358.01	0.01\\
359.01	0.01\\
360.01	0.01\\
361.01	0.01\\
362.01	0.01\\
363.01	0.01\\
364.01	0.01\\
365.01	0.01\\
366.01	0.01\\
367.01	0.01\\
368.01	0.01\\
369.01	0.01\\
370.01	0.01\\
371.01	0.01\\
372.01	0.01\\
373.01	0.01\\
374.01	0.01\\
375.01	0.01\\
376.01	0.01\\
377.01	0.01\\
378.01	0.01\\
379.01	0.01\\
380.01	0.01\\
381.01	0.01\\
382.01	0.01\\
383.01	0.01\\
384.01	0.01\\
385.01	0.01\\
386.01	0.01\\
387.01	0.01\\
388.01	0.01\\
389.01	0.01\\
390.01	0.01\\
391.01	0.01\\
392.01	0.01\\
393.01	0.01\\
394.01	0.01\\
395.01	0.01\\
396.01	0.01\\
397.01	0.01\\
398.01	0.01\\
399.01	0.01\\
400.01	0.01\\
401.01	0.01\\
402.01	0.01\\
403.01	0.01\\
404.01	0.01\\
405.01	0.01\\
406.01	0.01\\
407.01	0.01\\
408.01	0.01\\
409.01	0.01\\
410.01	0.01\\
411.01	0.01\\
412.01	0.01\\
413.01	0.01\\
414.01	0.01\\
415.01	0.01\\
416.01	0.01\\
417.01	0.01\\
418.01	0.01\\
419.01	0.01\\
420.01	0.01\\
421.01	0.01\\
422.01	0.01\\
423.01	0.01\\
424.01	0.01\\
425.01	0.01\\
426.01	0.01\\
427.01	0.01\\
428.01	0.01\\
429.01	0.01\\
430.01	0.01\\
431.01	0.01\\
432.01	0.01\\
433.01	0.01\\
434.01	0.01\\
435.01	0.01\\
436.01	0.01\\
437.01	0.01\\
438.01	0.01\\
439.01	0.01\\
440.01	0.01\\
441.01	0.01\\
442.01	0.01\\
443.01	0.01\\
444.01	0.01\\
445.01	0.01\\
446.01	0.01\\
447.01	0.01\\
448.01	0.01\\
449.01	0.01\\
450.01	0.01\\
451.01	0.01\\
452.01	0.01\\
453.01	0.01\\
454.01	0.01\\
455.01	0.01\\
456.01	0.01\\
457.01	0.01\\
458.01	0.01\\
459.01	0.01\\
460.01	0.01\\
461.01	0.01\\
462.01	0.01\\
463.01	0.01\\
464.01	0.01\\
465.01	0.01\\
466.01	0.01\\
467.01	0.01\\
468.01	0.01\\
469.01	0.01\\
470.01	0.01\\
471.01	0.01\\
472.01	0.01\\
473.01	0.01\\
474.01	0.01\\
475.01	0.01\\
476.01	0.01\\
477.01	0.01\\
478.01	0.01\\
479.01	0.01\\
480.01	0.01\\
481.01	0.01\\
482.01	0.01\\
483.01	0.01\\
484.01	0.01\\
485.01	0.01\\
486.01	0.01\\
487.01	0.01\\
488.01	0.01\\
489.01	0.01\\
490.01	0.01\\
491.01	0.01\\
492.01	0.01\\
493.01	0.01\\
494.01	0.01\\
495.01	0.01\\
496.01	0.01\\
497.01	0.01\\
498.01	0.01\\
499.01	0.01\\
500.01	0.01\\
501.01	0.01\\
502.01	0.01\\
503.01	0.01\\
504.01	0.01\\
505.01	0.01\\
506.01	0.01\\
507.01	0.01\\
508.01	0.01\\
509.01	0.01\\
510.01	0.01\\
511.01	0.01\\
512.01	0.01\\
513.01	0.01\\
514.01	0.01\\
515.01	0.01\\
516.01	0.01\\
517.01	0.01\\
518.01	0.01\\
519.01	0.01\\
520.01	0.01\\
521.01	0.01\\
522.01	0.01\\
523.01	0.01\\
524.01	0.01\\
525.01	0.01\\
526.01	0.01\\
527.01	0.01\\
528.01	0.01\\
529.01	0.01\\
530.01	0.01\\
531.01	0.01\\
532.01	0.01\\
533.01	0.01\\
534.01	0.01\\
535.01	0.01\\
536.01	0.01\\
537.01	0.01\\
538.01	0.01\\
539.01	0.01\\
540.01	0.01\\
541.01	0.01\\
542.01	0.01\\
543.01	0.01\\
544.01	0.01\\
545.01	0.01\\
546.01	0.01\\
547.01	0.01\\
548.01	0.01\\
549.01	0.01\\
550.01	0.01\\
551.01	0.01\\
552.01	0.01\\
553.01	0.01\\
554.01	0.01\\
555.01	0.01\\
556.01	0.01\\
557.01	0.01\\
558.01	0.01\\
559.01	0.01\\
560.01	0.01\\
561.01	0.01\\
562.01	0.01\\
563.01	0.01\\
564.01	0.01\\
565.01	0.01\\
566.01	0.01\\
567.01	0.01\\
568.01	0.01\\
569.01	0.01\\
570.01	0.01\\
571.01	0.01\\
572.01	0.01\\
573.01	0.01\\
574.01	0.01\\
575.01	0.01\\
576.01	0.01\\
577.01	0.01\\
578.01	0.01\\
579.01	0.01\\
580.01	0.01\\
581.01	0.01\\
582.01	0.01\\
583.01	0.01\\
584.01	0.01\\
585.01	0.01\\
586.01	0.01\\
587.01	0.01\\
588.01	0.01\\
589.01	0.01\\
590.01	0.01\\
591.01	0.01\\
592.01	0.01\\
593.01	0.01\\
594.01	0.01\\
595.01	0.01\\
596.01	0.01\\
597.01	0.01\\
598.01	0.01\\
599.01	0.0062426129370936\\
599.02	0.00620480443639245\\
599.03	0.00616663519666627\\
599.04	0.00612810141013912\\
599.05	0.00608919939176771\\
599.06	0.00604992538771588\\
599.07	0.00601027557193321\\
599.08	0.00597024605900537\\
599.09	0.00592983293920137\\
599.1	0.00588903222927545\\
599.11	0.00584783986513739\\
599.12	0.0058062516984085\\
599.13	0.00576426349491306\\
599.14	0.00572187092688378\\
599.15	0.00567906956807693\\
599.16	0.00563585488859709\\
599.17	0.00559222224977897\\
599.18	0.00554816689894846\\
599.19	0.00550368396247038\\
599.2	0.00545876843925004\\
599.21	0.00541341555627284\\
599.22	0.00536762101229125\\
599.23	0.00532138046333556\\
599.24	0.00527468952231843\\
599.25	0.00522754375864171\\
599.26	0.00517993869780592\\
599.27	0.00513186982093397\\
599.28	0.00508333256419477\\
599.29	0.00503432231838664\\
599.3	0.00498483442852312\\
599.31	0.00493486419341345\\
599.32	0.00488440686524595\\
599.33	0.00483345764918343\\
599.34	0.00478201170296413\\
599.35	0.00473006413650502\\
599.36	0.00467761001151456\\
599.37	0.00462464434111552\\
599.38	0.00457116208947773\\
599.39	0.00451715817542884\\
599.4	0.00446262748254593\\
599.41	0.00440756484478676\\
599.42	0.00435196504618212\\
599.43	0.00429582282054604\\
599.44	0.0042391328512059\\
599.45	0.00418188977075464\\
599.46	0.00412408816082742\\
599.47	0.00406572255190509\\
599.48	0.0040067874231475\\
599.49	0.00394727720047447\\
599.5	0.00388718625520624\\
599.51	0.00382650890352704\\
599.52	0.00376523940594317\\
599.53	0.00370337196673631\\
599.54	0.003640900733412\\
599.55	0.00357781979614291\\
599.56	0.00351412318720674\\
599.57	0.00344980488041878\\
599.58	0.00338485879055916\\
599.59	0.00331927877279455\\
599.6	0.00325305862209448\\
599.61	0.00318619207264212\\
599.62	0.00311867279723941\\
599.63	0.00305049440670675\\
599.64	0.00298165044927692\\
599.65	0.00291213440998337\\
599.66	0.00284193971004277\\
599.67	0.00277105970623188\\
599.68	0.00269948769025851\\
599.69	0.00262721688812665\\
599.7	0.00255424045949581\\
599.71	0.00248055149703432\\
599.72	0.00240614302576671\\
599.73	0.00233100800241499\\
599.74	0.00225513931473393\\
599.75	0.00217852978084007\\
599.76	0.0021011721485346\\
599.77	0.0020230590946199\\
599.78	0.00194418322420978\\
599.79	0.00186453707003317\\
599.8	0.00178411309173136\\
599.81	0.00170290367514858\\
599.82	0.00162090113161581\\
599.83	0.00153809769722781\\
599.84	0.001454485532113\\
599.85	0.00137005671969636\\
599.86	0.00128480326595493\\
599.87	0.00119871709866576\\
599.88	0.00111179006664627\\
599.89	0.00102401393898652\\
599.9	0.000935380404273357\\
599.91	0.000845881069805998\\
599.92	0.000755507460802737\\
599.93	0.000664251019598418\\
599.94	0.000572103104832229\\
599.95	0.000479054990625295\\
599.96	0.000385097865747556\\
599.97	0.000290222832773275\\
599.98	0.000194420907224489\\
599.99	9.76830167015649e-05\\
600	0\\
};
\addplot [color=red!25!mycolor17,solid,forget plot]
  table[row sep=crcr]{%
0.01	0.01\\
1.01	0.01\\
2.01	0.01\\
3.01	0.01\\
4.01	0.01\\
5.01	0.01\\
6.01	0.01\\
7.01	0.01\\
8.01	0.01\\
9.01	0.01\\
10.01	0.01\\
11.01	0.01\\
12.01	0.01\\
13.01	0.01\\
14.01	0.01\\
15.01	0.01\\
16.01	0.01\\
17.01	0.01\\
18.01	0.01\\
19.01	0.01\\
20.01	0.01\\
21.01	0.01\\
22.01	0.01\\
23.01	0.01\\
24.01	0.01\\
25.01	0.01\\
26.01	0.01\\
27.01	0.01\\
28.01	0.01\\
29.01	0.01\\
30.01	0.01\\
31.01	0.01\\
32.01	0.01\\
33.01	0.01\\
34.01	0.01\\
35.01	0.01\\
36.01	0.01\\
37.01	0.01\\
38.01	0.01\\
39.01	0.01\\
40.01	0.01\\
41.01	0.01\\
42.01	0.01\\
43.01	0.01\\
44.01	0.01\\
45.01	0.01\\
46.01	0.01\\
47.01	0.01\\
48.01	0.01\\
49.01	0.01\\
50.01	0.01\\
51.01	0.01\\
52.01	0.01\\
53.01	0.01\\
54.01	0.01\\
55.01	0.01\\
56.01	0.01\\
57.01	0.01\\
58.01	0.01\\
59.01	0.01\\
60.01	0.01\\
61.01	0.01\\
62.01	0.01\\
63.01	0.01\\
64.01	0.01\\
65.01	0.01\\
66.01	0.01\\
67.01	0.01\\
68.01	0.01\\
69.01	0.01\\
70.01	0.01\\
71.01	0.01\\
72.01	0.01\\
73.01	0.01\\
74.01	0.01\\
75.01	0.01\\
76.01	0.01\\
77.01	0.01\\
78.01	0.01\\
79.01	0.01\\
80.01	0.01\\
81.01	0.01\\
82.01	0.01\\
83.01	0.01\\
84.01	0.01\\
85.01	0.01\\
86.01	0.01\\
87.01	0.01\\
88.01	0.01\\
89.01	0.01\\
90.01	0.01\\
91.01	0.01\\
92.01	0.01\\
93.01	0.01\\
94.01	0.01\\
95.01	0.01\\
96.01	0.01\\
97.01	0.01\\
98.01	0.01\\
99.01	0.01\\
100.01	0.01\\
101.01	0.01\\
102.01	0.01\\
103.01	0.01\\
104.01	0.01\\
105.01	0.01\\
106.01	0.01\\
107.01	0.01\\
108.01	0.01\\
109.01	0.01\\
110.01	0.01\\
111.01	0.01\\
112.01	0.01\\
113.01	0.01\\
114.01	0.01\\
115.01	0.01\\
116.01	0.01\\
117.01	0.01\\
118.01	0.01\\
119.01	0.01\\
120.01	0.01\\
121.01	0.01\\
122.01	0.01\\
123.01	0.01\\
124.01	0.01\\
125.01	0.01\\
126.01	0.01\\
127.01	0.01\\
128.01	0.01\\
129.01	0.01\\
130.01	0.01\\
131.01	0.01\\
132.01	0.01\\
133.01	0.01\\
134.01	0.01\\
135.01	0.01\\
136.01	0.01\\
137.01	0.01\\
138.01	0.01\\
139.01	0.01\\
140.01	0.01\\
141.01	0.01\\
142.01	0.01\\
143.01	0.01\\
144.01	0.01\\
145.01	0.01\\
146.01	0.01\\
147.01	0.01\\
148.01	0.01\\
149.01	0.01\\
150.01	0.01\\
151.01	0.01\\
152.01	0.01\\
153.01	0.01\\
154.01	0.01\\
155.01	0.01\\
156.01	0.01\\
157.01	0.01\\
158.01	0.01\\
159.01	0.01\\
160.01	0.01\\
161.01	0.01\\
162.01	0.01\\
163.01	0.01\\
164.01	0.01\\
165.01	0.01\\
166.01	0.01\\
167.01	0.01\\
168.01	0.01\\
169.01	0.01\\
170.01	0.01\\
171.01	0.01\\
172.01	0.01\\
173.01	0.01\\
174.01	0.01\\
175.01	0.01\\
176.01	0.01\\
177.01	0.01\\
178.01	0.01\\
179.01	0.01\\
180.01	0.01\\
181.01	0.01\\
182.01	0.01\\
183.01	0.01\\
184.01	0.01\\
185.01	0.01\\
186.01	0.01\\
187.01	0.01\\
188.01	0.01\\
189.01	0.01\\
190.01	0.01\\
191.01	0.01\\
192.01	0.01\\
193.01	0.01\\
194.01	0.01\\
195.01	0.01\\
196.01	0.01\\
197.01	0.01\\
198.01	0.01\\
199.01	0.01\\
200.01	0.01\\
201.01	0.01\\
202.01	0.01\\
203.01	0.01\\
204.01	0.01\\
205.01	0.01\\
206.01	0.01\\
207.01	0.01\\
208.01	0.01\\
209.01	0.01\\
210.01	0.01\\
211.01	0.01\\
212.01	0.01\\
213.01	0.01\\
214.01	0.01\\
215.01	0.01\\
216.01	0.01\\
217.01	0.01\\
218.01	0.01\\
219.01	0.01\\
220.01	0.01\\
221.01	0.01\\
222.01	0.01\\
223.01	0.01\\
224.01	0.01\\
225.01	0.01\\
226.01	0.01\\
227.01	0.01\\
228.01	0.01\\
229.01	0.01\\
230.01	0.01\\
231.01	0.01\\
232.01	0.01\\
233.01	0.01\\
234.01	0.01\\
235.01	0.01\\
236.01	0.01\\
237.01	0.01\\
238.01	0.01\\
239.01	0.01\\
240.01	0.01\\
241.01	0.01\\
242.01	0.01\\
243.01	0.01\\
244.01	0.01\\
245.01	0.01\\
246.01	0.01\\
247.01	0.01\\
248.01	0.01\\
249.01	0.01\\
250.01	0.01\\
251.01	0.01\\
252.01	0.01\\
253.01	0.01\\
254.01	0.01\\
255.01	0.01\\
256.01	0.01\\
257.01	0.01\\
258.01	0.01\\
259.01	0.01\\
260.01	0.01\\
261.01	0.01\\
262.01	0.01\\
263.01	0.01\\
264.01	0.01\\
265.01	0.01\\
266.01	0.01\\
267.01	0.01\\
268.01	0.01\\
269.01	0.01\\
270.01	0.01\\
271.01	0.01\\
272.01	0.01\\
273.01	0.01\\
274.01	0.01\\
275.01	0.01\\
276.01	0.01\\
277.01	0.01\\
278.01	0.01\\
279.01	0.01\\
280.01	0.01\\
281.01	0.01\\
282.01	0.01\\
283.01	0.01\\
284.01	0.01\\
285.01	0.01\\
286.01	0.01\\
287.01	0.01\\
288.01	0.01\\
289.01	0.01\\
290.01	0.01\\
291.01	0.01\\
292.01	0.01\\
293.01	0.01\\
294.01	0.01\\
295.01	0.01\\
296.01	0.01\\
297.01	0.01\\
298.01	0.01\\
299.01	0.01\\
300.01	0.01\\
301.01	0.01\\
302.01	0.01\\
303.01	0.01\\
304.01	0.01\\
305.01	0.01\\
306.01	0.01\\
307.01	0.01\\
308.01	0.01\\
309.01	0.01\\
310.01	0.01\\
311.01	0.01\\
312.01	0.01\\
313.01	0.01\\
314.01	0.01\\
315.01	0.01\\
316.01	0.01\\
317.01	0.01\\
318.01	0.01\\
319.01	0.01\\
320.01	0.01\\
321.01	0.01\\
322.01	0.01\\
323.01	0.01\\
324.01	0.01\\
325.01	0.01\\
326.01	0.01\\
327.01	0.01\\
328.01	0.01\\
329.01	0.01\\
330.01	0.01\\
331.01	0.01\\
332.01	0.01\\
333.01	0.01\\
334.01	0.01\\
335.01	0.01\\
336.01	0.01\\
337.01	0.01\\
338.01	0.01\\
339.01	0.01\\
340.01	0.01\\
341.01	0.01\\
342.01	0.01\\
343.01	0.01\\
344.01	0.01\\
345.01	0.01\\
346.01	0.01\\
347.01	0.01\\
348.01	0.01\\
349.01	0.01\\
350.01	0.01\\
351.01	0.01\\
352.01	0.01\\
353.01	0.01\\
354.01	0.01\\
355.01	0.01\\
356.01	0.01\\
357.01	0.01\\
358.01	0.01\\
359.01	0.01\\
360.01	0.01\\
361.01	0.01\\
362.01	0.01\\
363.01	0.01\\
364.01	0.01\\
365.01	0.01\\
366.01	0.01\\
367.01	0.01\\
368.01	0.01\\
369.01	0.01\\
370.01	0.01\\
371.01	0.01\\
372.01	0.01\\
373.01	0.01\\
374.01	0.01\\
375.01	0.01\\
376.01	0.01\\
377.01	0.01\\
378.01	0.01\\
379.01	0.01\\
380.01	0.01\\
381.01	0.01\\
382.01	0.01\\
383.01	0.01\\
384.01	0.01\\
385.01	0.01\\
386.01	0.01\\
387.01	0.01\\
388.01	0.01\\
389.01	0.01\\
390.01	0.01\\
391.01	0.01\\
392.01	0.01\\
393.01	0.01\\
394.01	0.01\\
395.01	0.01\\
396.01	0.01\\
397.01	0.01\\
398.01	0.01\\
399.01	0.01\\
400.01	0.01\\
401.01	0.01\\
402.01	0.01\\
403.01	0.01\\
404.01	0.01\\
405.01	0.01\\
406.01	0.01\\
407.01	0.01\\
408.01	0.01\\
409.01	0.01\\
410.01	0.01\\
411.01	0.01\\
412.01	0.01\\
413.01	0.01\\
414.01	0.01\\
415.01	0.01\\
416.01	0.01\\
417.01	0.01\\
418.01	0.01\\
419.01	0.01\\
420.01	0.01\\
421.01	0.01\\
422.01	0.01\\
423.01	0.01\\
424.01	0.01\\
425.01	0.01\\
426.01	0.01\\
427.01	0.01\\
428.01	0.01\\
429.01	0.01\\
430.01	0.01\\
431.01	0.01\\
432.01	0.01\\
433.01	0.01\\
434.01	0.01\\
435.01	0.01\\
436.01	0.01\\
437.01	0.01\\
438.01	0.01\\
439.01	0.01\\
440.01	0.01\\
441.01	0.01\\
442.01	0.01\\
443.01	0.01\\
444.01	0.01\\
445.01	0.01\\
446.01	0.01\\
447.01	0.01\\
448.01	0.01\\
449.01	0.01\\
450.01	0.01\\
451.01	0.01\\
452.01	0.01\\
453.01	0.01\\
454.01	0.01\\
455.01	0.01\\
456.01	0.01\\
457.01	0.01\\
458.01	0.01\\
459.01	0.01\\
460.01	0.01\\
461.01	0.01\\
462.01	0.01\\
463.01	0.01\\
464.01	0.01\\
465.01	0.01\\
466.01	0.01\\
467.01	0.01\\
468.01	0.01\\
469.01	0.01\\
470.01	0.01\\
471.01	0.01\\
472.01	0.01\\
473.01	0.01\\
474.01	0.01\\
475.01	0.01\\
476.01	0.01\\
477.01	0.01\\
478.01	0.01\\
479.01	0.01\\
480.01	0.01\\
481.01	0.01\\
482.01	0.01\\
483.01	0.01\\
484.01	0.01\\
485.01	0.01\\
486.01	0.01\\
487.01	0.01\\
488.01	0.01\\
489.01	0.01\\
490.01	0.01\\
491.01	0.01\\
492.01	0.01\\
493.01	0.01\\
494.01	0.01\\
495.01	0.01\\
496.01	0.01\\
497.01	0.01\\
498.01	0.01\\
499.01	0.01\\
500.01	0.01\\
501.01	0.01\\
502.01	0.01\\
503.01	0.01\\
504.01	0.01\\
505.01	0.01\\
506.01	0.01\\
507.01	0.01\\
508.01	0.01\\
509.01	0.01\\
510.01	0.01\\
511.01	0.01\\
512.01	0.01\\
513.01	0.01\\
514.01	0.01\\
515.01	0.01\\
516.01	0.01\\
517.01	0.01\\
518.01	0.01\\
519.01	0.01\\
520.01	0.01\\
521.01	0.01\\
522.01	0.01\\
523.01	0.01\\
524.01	0.01\\
525.01	0.01\\
526.01	0.01\\
527.01	0.01\\
528.01	0.01\\
529.01	0.01\\
530.01	0.01\\
531.01	0.01\\
532.01	0.01\\
533.01	0.01\\
534.01	0.01\\
535.01	0.01\\
536.01	0.01\\
537.01	0.01\\
538.01	0.01\\
539.01	0.01\\
540.01	0.01\\
541.01	0.01\\
542.01	0.01\\
543.01	0.01\\
544.01	0.01\\
545.01	0.01\\
546.01	0.01\\
547.01	0.01\\
548.01	0.01\\
549.01	0.01\\
550.01	0.01\\
551.01	0.01\\
552.01	0.01\\
553.01	0.01\\
554.01	0.01\\
555.01	0.01\\
556.01	0.01\\
557.01	0.01\\
558.01	0.01\\
559.01	0.01\\
560.01	0.01\\
561.01	0.01\\
562.01	0.01\\
563.01	0.01\\
564.01	0.01\\
565.01	0.01\\
566.01	0.01\\
567.01	0.01\\
568.01	0.01\\
569.01	0.01\\
570.01	0.01\\
571.01	0.01\\
572.01	0.01\\
573.01	0.01\\
574.01	0.01\\
575.01	0.01\\
576.01	0.01\\
577.01	0.01\\
578.01	0.01\\
579.01	0.01\\
580.01	0.01\\
581.01	0.01\\
582.01	0.01\\
583.01	0.01\\
584.01	0.01\\
585.01	0.01\\
586.01	0.01\\
587.01	0.01\\
588.01	0.01\\
589.01	0.01\\
590.01	0.01\\
591.01	0.01\\
592.01	0.01\\
593.01	0.01\\
594.01	0.01\\
595.01	0.01\\
596.01	0.01\\
597.01	0.01\\
598.01	0.01\\
599.01	0.00624187794319273\\
599.02	0.00620415217318548\\
599.03	0.00616605955071492\\
599.04	0.00612759646350277\\
599.05	0.00608875926364123\\
599.06	0.00604954426731568\\
599.07	0.00600994775453456\\
599.08	0.00596996596881047\\
599.09	0.0059295951167174\\
599.1	0.00588883136760121\\
599.11	0.0058476708533095\\
599.12	0.00580610966792835\\
599.13	0.00576414386751836\\
599.14	0.00572176946987378\\
599.15	0.00567898245429428\\
599.16	0.00563577876137102\\
599.17	0.00559215429278665\\
599.18	0.0055481049111301\\
599.19	0.00550362643973364\\
599.2	0.00545871466253047\\
599.21	0.00541336532313308\\
599.22	0.00536757412333666\\
599.23	0.00532133672270926\\
599.24	0.00527464873817792\\
599.25	0.00522750574361058\\
599.26	0.00517990326939395\\
599.27	0.00513183680200728\\
599.28	0.00508330178359264\\
599.29	0.00503429361152088\\
599.3	0.00498480763795339\\
599.31	0.00493483916939956\\
599.32	0.00488438346626986\\
599.33	0.00483343574242454\\
599.34	0.00478199116471782\\
599.35	0.00473004485253745\\
599.36	0.0046775918773397\\
599.37	0.00462462726217956\\
599.38	0.00457114598123612\\
599.39	0.00451714296330743\\
599.4	0.00446261310174465\\
599.41	0.00440755123993145\\
599.42	0.00435195217079425\\
599.43	0.00429581063630734\\
599.44	0.00423912132699231\\
599.45	0.00418187888141208\\
599.46	0.00412407788565892\\
599.47	0.00406571287283668\\
599.48	0.00400677832253673\\
599.49	0.00394726866031217\\
599.5	0.00388717825714887\\
599.51	0.00382650142893142\\
599.52	0.00376523243590372\\
599.53	0.00370336548212449\\
599.54	0.00364089471491734\\
599.55	0.0035778142243154\\
599.56	0.00351411804250074\\
599.57	0.00344980014323806\\
599.58	0.00338485444130302\\
599.59	0.00331927479190491\\
599.6	0.00325305499010369\\
599.61	0.00318618877022133\\
599.62	0.00311866980524735\\
599.63	0.00305049170623868\\
599.64	0.00298164802171346\\
599.65	0.00291213223703912\\
599.66	0.00284193777381426\\
599.67	0.0027710579892447\\
599.68	0.00269948617551322\\
599.69	0.00262721555914333\\
599.7	0.00255423930035664\\
599.71	0.00248055049242407\\
599.72	0.00240614216101062\\
599.73	0.0023310072635138\\
599.74	0.00225513868839553\\
599.75	0.0021785292545075\\
599.76	0.00210117171040993\\
599.77	0.0020230587336837\\
599.78	0.00194418293023563\\
599.79	0.00186453683359705\\
599.8	0.00178411290421546\\
599.81	0.00170290352873919\\
599.82	0.00162090101929515\\
599.83	0.00153809761275938\\
599.84	0.00145448547002052\\
599.85	0.00137005667523597\\
599.86	0.00128480323508086\\
599.87	0.00119871707798955\\
599.88	0.00111179005338966\\
599.89	0.00102401393092871\\
599.9	0.000935380399693021\\
599.91	0.000845881067419046\\
599.92	0.000755507459696892\\
599.93	0.00066425101916609\\
599.94	0.000572103104703438\\
599.95	0.000479054990602912\\
599.96	0.000385097865747556\\
599.97	0.000290222832773275\\
599.98	0.000194420907224489\\
599.99	9.76830167015649e-05\\
600	0\\
};
\addplot [color=mycolor19,solid,forget plot]
  table[row sep=crcr]{%
0.01	0.01\\
1.01	0.01\\
2.01	0.01\\
3.01	0.01\\
4.01	0.01\\
5.01	0.01\\
6.01	0.01\\
7.01	0.01\\
8.01	0.01\\
9.01	0.01\\
10.01	0.01\\
11.01	0.01\\
12.01	0.01\\
13.01	0.01\\
14.01	0.01\\
15.01	0.01\\
16.01	0.01\\
17.01	0.01\\
18.01	0.01\\
19.01	0.01\\
20.01	0.01\\
21.01	0.01\\
22.01	0.01\\
23.01	0.01\\
24.01	0.01\\
25.01	0.01\\
26.01	0.01\\
27.01	0.01\\
28.01	0.01\\
29.01	0.01\\
30.01	0.01\\
31.01	0.01\\
32.01	0.01\\
33.01	0.01\\
34.01	0.01\\
35.01	0.01\\
36.01	0.01\\
37.01	0.01\\
38.01	0.01\\
39.01	0.01\\
40.01	0.01\\
41.01	0.01\\
42.01	0.01\\
43.01	0.01\\
44.01	0.01\\
45.01	0.01\\
46.01	0.01\\
47.01	0.01\\
48.01	0.01\\
49.01	0.01\\
50.01	0.01\\
51.01	0.01\\
52.01	0.01\\
53.01	0.01\\
54.01	0.01\\
55.01	0.01\\
56.01	0.01\\
57.01	0.01\\
58.01	0.01\\
59.01	0.01\\
60.01	0.01\\
61.01	0.01\\
62.01	0.01\\
63.01	0.01\\
64.01	0.01\\
65.01	0.01\\
66.01	0.01\\
67.01	0.01\\
68.01	0.01\\
69.01	0.01\\
70.01	0.01\\
71.01	0.01\\
72.01	0.01\\
73.01	0.01\\
74.01	0.01\\
75.01	0.01\\
76.01	0.01\\
77.01	0.01\\
78.01	0.01\\
79.01	0.01\\
80.01	0.01\\
81.01	0.01\\
82.01	0.01\\
83.01	0.01\\
84.01	0.01\\
85.01	0.01\\
86.01	0.01\\
87.01	0.01\\
88.01	0.01\\
89.01	0.01\\
90.01	0.01\\
91.01	0.01\\
92.01	0.01\\
93.01	0.01\\
94.01	0.01\\
95.01	0.01\\
96.01	0.01\\
97.01	0.01\\
98.01	0.01\\
99.01	0.01\\
100.01	0.01\\
101.01	0.01\\
102.01	0.01\\
103.01	0.01\\
104.01	0.01\\
105.01	0.01\\
106.01	0.01\\
107.01	0.01\\
108.01	0.01\\
109.01	0.01\\
110.01	0.01\\
111.01	0.01\\
112.01	0.01\\
113.01	0.01\\
114.01	0.01\\
115.01	0.01\\
116.01	0.01\\
117.01	0.01\\
118.01	0.01\\
119.01	0.01\\
120.01	0.01\\
121.01	0.01\\
122.01	0.01\\
123.01	0.01\\
124.01	0.01\\
125.01	0.01\\
126.01	0.01\\
127.01	0.01\\
128.01	0.01\\
129.01	0.01\\
130.01	0.01\\
131.01	0.01\\
132.01	0.01\\
133.01	0.01\\
134.01	0.01\\
135.01	0.01\\
136.01	0.01\\
137.01	0.01\\
138.01	0.01\\
139.01	0.01\\
140.01	0.01\\
141.01	0.01\\
142.01	0.01\\
143.01	0.01\\
144.01	0.01\\
145.01	0.01\\
146.01	0.01\\
147.01	0.01\\
148.01	0.01\\
149.01	0.01\\
150.01	0.01\\
151.01	0.01\\
152.01	0.01\\
153.01	0.01\\
154.01	0.01\\
155.01	0.01\\
156.01	0.01\\
157.01	0.01\\
158.01	0.01\\
159.01	0.01\\
160.01	0.01\\
161.01	0.01\\
162.01	0.01\\
163.01	0.01\\
164.01	0.01\\
165.01	0.01\\
166.01	0.01\\
167.01	0.01\\
168.01	0.01\\
169.01	0.01\\
170.01	0.01\\
171.01	0.01\\
172.01	0.01\\
173.01	0.01\\
174.01	0.01\\
175.01	0.01\\
176.01	0.01\\
177.01	0.01\\
178.01	0.01\\
179.01	0.01\\
180.01	0.01\\
181.01	0.01\\
182.01	0.01\\
183.01	0.01\\
184.01	0.01\\
185.01	0.01\\
186.01	0.01\\
187.01	0.01\\
188.01	0.01\\
189.01	0.01\\
190.01	0.01\\
191.01	0.01\\
192.01	0.01\\
193.01	0.01\\
194.01	0.01\\
195.01	0.01\\
196.01	0.01\\
197.01	0.01\\
198.01	0.01\\
199.01	0.01\\
200.01	0.01\\
201.01	0.01\\
202.01	0.01\\
203.01	0.01\\
204.01	0.01\\
205.01	0.01\\
206.01	0.01\\
207.01	0.01\\
208.01	0.01\\
209.01	0.01\\
210.01	0.01\\
211.01	0.01\\
212.01	0.01\\
213.01	0.01\\
214.01	0.01\\
215.01	0.01\\
216.01	0.01\\
217.01	0.01\\
218.01	0.01\\
219.01	0.01\\
220.01	0.01\\
221.01	0.01\\
222.01	0.01\\
223.01	0.01\\
224.01	0.01\\
225.01	0.01\\
226.01	0.01\\
227.01	0.01\\
228.01	0.01\\
229.01	0.01\\
230.01	0.01\\
231.01	0.01\\
232.01	0.01\\
233.01	0.01\\
234.01	0.01\\
235.01	0.01\\
236.01	0.01\\
237.01	0.01\\
238.01	0.01\\
239.01	0.01\\
240.01	0.01\\
241.01	0.01\\
242.01	0.01\\
243.01	0.01\\
244.01	0.01\\
245.01	0.01\\
246.01	0.01\\
247.01	0.01\\
248.01	0.01\\
249.01	0.01\\
250.01	0.01\\
251.01	0.01\\
252.01	0.01\\
253.01	0.01\\
254.01	0.01\\
255.01	0.01\\
256.01	0.01\\
257.01	0.01\\
258.01	0.01\\
259.01	0.01\\
260.01	0.01\\
261.01	0.01\\
262.01	0.01\\
263.01	0.01\\
264.01	0.01\\
265.01	0.01\\
266.01	0.01\\
267.01	0.01\\
268.01	0.01\\
269.01	0.01\\
270.01	0.01\\
271.01	0.01\\
272.01	0.01\\
273.01	0.01\\
274.01	0.01\\
275.01	0.01\\
276.01	0.01\\
277.01	0.01\\
278.01	0.01\\
279.01	0.01\\
280.01	0.01\\
281.01	0.01\\
282.01	0.01\\
283.01	0.01\\
284.01	0.01\\
285.01	0.01\\
286.01	0.01\\
287.01	0.01\\
288.01	0.01\\
289.01	0.01\\
290.01	0.01\\
291.01	0.01\\
292.01	0.01\\
293.01	0.01\\
294.01	0.01\\
295.01	0.01\\
296.01	0.01\\
297.01	0.01\\
298.01	0.01\\
299.01	0.01\\
300.01	0.01\\
301.01	0.01\\
302.01	0.01\\
303.01	0.01\\
304.01	0.01\\
305.01	0.01\\
306.01	0.01\\
307.01	0.01\\
308.01	0.01\\
309.01	0.01\\
310.01	0.01\\
311.01	0.01\\
312.01	0.01\\
313.01	0.01\\
314.01	0.01\\
315.01	0.01\\
316.01	0.01\\
317.01	0.01\\
318.01	0.01\\
319.01	0.01\\
320.01	0.01\\
321.01	0.01\\
322.01	0.01\\
323.01	0.01\\
324.01	0.01\\
325.01	0.01\\
326.01	0.01\\
327.01	0.01\\
328.01	0.01\\
329.01	0.01\\
330.01	0.01\\
331.01	0.01\\
332.01	0.01\\
333.01	0.01\\
334.01	0.01\\
335.01	0.01\\
336.01	0.01\\
337.01	0.01\\
338.01	0.01\\
339.01	0.01\\
340.01	0.01\\
341.01	0.01\\
342.01	0.01\\
343.01	0.01\\
344.01	0.01\\
345.01	0.01\\
346.01	0.01\\
347.01	0.01\\
348.01	0.01\\
349.01	0.01\\
350.01	0.01\\
351.01	0.01\\
352.01	0.01\\
353.01	0.01\\
354.01	0.01\\
355.01	0.01\\
356.01	0.01\\
357.01	0.01\\
358.01	0.01\\
359.01	0.01\\
360.01	0.01\\
361.01	0.01\\
362.01	0.01\\
363.01	0.01\\
364.01	0.01\\
365.01	0.01\\
366.01	0.01\\
367.01	0.01\\
368.01	0.01\\
369.01	0.01\\
370.01	0.01\\
371.01	0.01\\
372.01	0.01\\
373.01	0.01\\
374.01	0.01\\
375.01	0.01\\
376.01	0.01\\
377.01	0.01\\
378.01	0.01\\
379.01	0.01\\
380.01	0.01\\
381.01	0.01\\
382.01	0.01\\
383.01	0.01\\
384.01	0.01\\
385.01	0.01\\
386.01	0.01\\
387.01	0.01\\
388.01	0.01\\
389.01	0.01\\
390.01	0.01\\
391.01	0.01\\
392.01	0.01\\
393.01	0.01\\
394.01	0.01\\
395.01	0.01\\
396.01	0.01\\
397.01	0.01\\
398.01	0.01\\
399.01	0.01\\
400.01	0.01\\
401.01	0.01\\
402.01	0.01\\
403.01	0.01\\
404.01	0.01\\
405.01	0.01\\
406.01	0.01\\
407.01	0.01\\
408.01	0.01\\
409.01	0.01\\
410.01	0.01\\
411.01	0.01\\
412.01	0.01\\
413.01	0.01\\
414.01	0.01\\
415.01	0.01\\
416.01	0.01\\
417.01	0.01\\
418.01	0.01\\
419.01	0.01\\
420.01	0.01\\
421.01	0.01\\
422.01	0.01\\
423.01	0.01\\
424.01	0.01\\
425.01	0.01\\
426.01	0.01\\
427.01	0.01\\
428.01	0.01\\
429.01	0.01\\
430.01	0.01\\
431.01	0.01\\
432.01	0.01\\
433.01	0.01\\
434.01	0.01\\
435.01	0.01\\
436.01	0.01\\
437.01	0.01\\
438.01	0.01\\
439.01	0.01\\
440.01	0.01\\
441.01	0.01\\
442.01	0.01\\
443.01	0.01\\
444.01	0.01\\
445.01	0.01\\
446.01	0.01\\
447.01	0.01\\
448.01	0.01\\
449.01	0.01\\
450.01	0.01\\
451.01	0.01\\
452.01	0.01\\
453.01	0.01\\
454.01	0.01\\
455.01	0.01\\
456.01	0.01\\
457.01	0.01\\
458.01	0.01\\
459.01	0.01\\
460.01	0.01\\
461.01	0.01\\
462.01	0.01\\
463.01	0.01\\
464.01	0.01\\
465.01	0.01\\
466.01	0.01\\
467.01	0.01\\
468.01	0.01\\
469.01	0.01\\
470.01	0.01\\
471.01	0.01\\
472.01	0.01\\
473.01	0.01\\
474.01	0.01\\
475.01	0.01\\
476.01	0.01\\
477.01	0.01\\
478.01	0.01\\
479.01	0.01\\
480.01	0.01\\
481.01	0.01\\
482.01	0.01\\
483.01	0.01\\
484.01	0.01\\
485.01	0.01\\
486.01	0.01\\
487.01	0.01\\
488.01	0.01\\
489.01	0.01\\
490.01	0.01\\
491.01	0.01\\
492.01	0.01\\
493.01	0.01\\
494.01	0.01\\
495.01	0.01\\
496.01	0.01\\
497.01	0.01\\
498.01	0.01\\
499.01	0.01\\
500.01	0.01\\
501.01	0.01\\
502.01	0.01\\
503.01	0.01\\
504.01	0.01\\
505.01	0.01\\
506.01	0.01\\
507.01	0.01\\
508.01	0.01\\
509.01	0.01\\
510.01	0.01\\
511.01	0.01\\
512.01	0.01\\
513.01	0.01\\
514.01	0.01\\
515.01	0.01\\
516.01	0.01\\
517.01	0.01\\
518.01	0.01\\
519.01	0.01\\
520.01	0.01\\
521.01	0.01\\
522.01	0.01\\
523.01	0.01\\
524.01	0.01\\
525.01	0.01\\
526.01	0.01\\
527.01	0.01\\
528.01	0.01\\
529.01	0.01\\
530.01	0.01\\
531.01	0.01\\
532.01	0.01\\
533.01	0.01\\
534.01	0.01\\
535.01	0.01\\
536.01	0.01\\
537.01	0.01\\
538.01	0.01\\
539.01	0.01\\
540.01	0.01\\
541.01	0.01\\
542.01	0.01\\
543.01	0.01\\
544.01	0.01\\
545.01	0.01\\
546.01	0.01\\
547.01	0.01\\
548.01	0.01\\
549.01	0.01\\
550.01	0.01\\
551.01	0.01\\
552.01	0.01\\
553.01	0.01\\
554.01	0.01\\
555.01	0.01\\
556.01	0.01\\
557.01	0.01\\
558.01	0.01\\
559.01	0.01\\
560.01	0.01\\
561.01	0.01\\
562.01	0.01\\
563.01	0.01\\
564.01	0.01\\
565.01	0.01\\
566.01	0.01\\
567.01	0.01\\
568.01	0.01\\
569.01	0.01\\
570.01	0.01\\
571.01	0.01\\
572.01	0.01\\
573.01	0.01\\
574.01	0.01\\
575.01	0.01\\
576.01	0.01\\
577.01	0.01\\
578.01	0.01\\
579.01	0.01\\
580.01	0.01\\
581.01	0.01\\
582.01	0.01\\
583.01	0.01\\
584.01	0.01\\
585.01	0.01\\
586.01	0.01\\
587.01	0.01\\
588.01	0.01\\
589.01	0.01\\
590.01	0.01\\
591.01	0.01\\
592.01	0.01\\
593.01	0.01\\
594.01	0.01\\
595.01	0.01\\
596.01	0.01\\
597.01	0.01\\
598.01	0.01\\
599.01	0.00624186921776351\\
599.02	0.00620414452392774\\
599.03	0.00616605285708784\\
599.04	0.0061275906126133\\
599.05	0.00608875415047225\\
599.06	0.00604953979488413\\
599.07	0.00600994383396897\\
599.08	0.00596996251939311\\
599.09	0.00592959206601176\\
599.1	0.00588882865150792\\
599.11	0.00584766841602762\\
599.12	0.00580610746181134\\
599.13	0.00576414185282154\\
599.14	0.00572176761436633\\
599.15	0.00567898073271903\\
599.16	0.00563577715473362\\
599.17	0.00559215278745594\\
599.18	0.00554810349773059\\
599.19	0.00550362511180335\\
599.2	0.00545871341491899\\
599.21	0.00541336415091619\\
599.22	0.00536757302182078\\
599.23	0.005321335687435\\
599.24	0.00527464776492264\\
599.25	0.00522750482839026\\
599.26	0.00517990240846415\\
599.27	0.0051318359918632\\
599.28	0.00508330102096758\\
599.29	0.00503429289338311\\
599.3	0.00498480696150149\\
599.31	0.004934838532056\\
599.32	0.00488438286567301\\
599.33	0.00483343517641901\\
599.34	0.00478199063134318\\
599.35	0.00473004435001546\\
599.36	0.0046775914040601\\
599.37	0.00462462681668464\\
599.38	0.00457114556220417\\
599.39	0.00451714256953565\\
599.4	0.00446261273213111\\
599.41	0.0044075508934571\\
599.42	0.0043519518465054\\
599.43	0.00429581033329887\\
599.44	0.00423912104439258\\
599.45	0.00418187861836991\\
599.46	0.00412407764133385\\
599.47	0.00406571264639331\\
599.48	0.00400677811314435\\
599.49	0.00394726846714635\\
599.5	0.00388717807939316\\
599.51	0.00382650126577892\\
599.52	0.00376523228655877\\
599.53	0.00370336534580421\\
599.54	0.00364089459085318\\
599.55	0.00357781411175471\\
599.56	0.00351411794070817\\
599.57	0.00344980005149701\\
599.58	0.00338485435891695\\
599.59	0.00331927471819867\\
599.6	0.00325305492442463\\
599.61	0.00318618871194044\\
599.62	0.00311866975376029\\
599.63	0.00305049166096664\\
599.64	0.00298164798210398\\
599.65	0.00291213220256672\\
599.66	0.00284193774398107\\
599.67	0.00277105796358078\\
599.68	0.00269948615357693\\
599.69	0.00262721554052142\\
599.7	0.00255423928466429\\
599.71	0.00248055047930473\\
599.72	0.00240614215013571\\
599.73	0.00233100725458232\\
599.74	0.0022551386811334\\
599.75	0.0021785292486669\\
599.76	0.00210117170576842\\
599.77	0.00202305873004318\\
599.78	0.00194418292742126\\
599.79	0.00186453683145598\\
599.8	0.00178411290261548\\
599.81	0.00170290352756733\\
599.82	0.00162090101845609\\
599.83	0.00153809761217394\\
599.84	0.00145448546962399\\
599.85	0.00137005667497651\\
599.86	0.00128480323491787\\
599.87	0.00119871707789201\\
599.88	0.00111179005333465\\
599.89	0.0010240139308999\\
599.9	0.000935380399679301\\
599.91	0.000845881067413297\\
599.92	0.000755507459694884\\
599.93	0.000664251019165563\\
599.94	0.000572103104703356\\
599.95	0.000479054990602912\\
599.96	0.000385097865747556\\
599.97	0.000290222832773275\\
599.98	0.000194420907224489\\
599.99	9.76830167015649e-05\\
600	0\\
};
\addplot [color=red!50!mycolor17,solid,forget plot]
  table[row sep=crcr]{%
0.01	0.00999999999999999\\
1.01	0.00999999999999999\\
2.01	0.00999999999999999\\
3.01	0.00999999999999999\\
4.01	0.00999999999999999\\
5.01	0.00999999999999999\\
6.01	0.00999999999999999\\
7.01	0.00999999999999999\\
8.01	0.00999999999999999\\
9.01	0.00999999999999999\\
10.01	0.00999999999999999\\
11.01	0.00999999999999999\\
12.01	0.00999999999999999\\
13.01	0.00999999999999999\\
14.01	0.00999999999999999\\
15.01	0.00999999999999999\\
16.01	0.00999999999999999\\
17.01	0.00999999999999999\\
18.01	0.00999999999999999\\
19.01	0.00999999999999999\\
20.01	0.00999999999999999\\
21.01	0.00999999999999999\\
22.01	0.00999999999999999\\
23.01	0.00999999999999999\\
24.01	0.00999999999999999\\
25.01	0.00999999999999999\\
26.01	0.00999999999999999\\
27.01	0.00999999999999999\\
28.01	0.00999999999999999\\
29.01	0.00999999999999999\\
30.01	0.00999999999999999\\
31.01	0.00999999999999999\\
32.01	0.00999999999999999\\
33.01	0.00999999999999999\\
34.01	0.00999999999999999\\
35.01	0.00999999999999999\\
36.01	0.00999999999999999\\
37.01	0.00999999999999999\\
38.01	0.00999999999999999\\
39.01	0.00999999999999999\\
40.01	0.00999999999999999\\
41.01	0.00999999999999999\\
42.01	0.00999999999999999\\
43.01	0.00999999999999999\\
44.01	0.00999999999999999\\
45.01	0.00999999999999999\\
46.01	0.00999999999999999\\
47.01	0.00999999999999999\\
48.01	0.00999999999999999\\
49.01	0.00999999999999999\\
50.01	0.00999999999999999\\
51.01	0.00999999999999999\\
52.01	0.00999999999999999\\
53.01	0.00999999999999999\\
54.01	0.00999999999999999\\
55.01	0.00999999999999999\\
56.01	0.00999999999999999\\
57.01	0.00999999999999999\\
58.01	0.00999999999999999\\
59.01	0.00999999999999999\\
60.01	0.00999999999999999\\
61.01	0.00999999999999999\\
62.01	0.00999999999999999\\
63.01	0.00999999999999999\\
64.01	0.00999999999999999\\
65.01	0.00999999999999999\\
66.01	0.00999999999999999\\
67.01	0.00999999999999999\\
68.01	0.00999999999999999\\
69.01	0.00999999999999999\\
70.01	0.00999999999999999\\
71.01	0.00999999999999999\\
72.01	0.00999999999999999\\
73.01	0.00999999999999999\\
74.01	0.00999999999999999\\
75.01	0.00999999999999999\\
76.01	0.00999999999999999\\
77.01	0.00999999999999999\\
78.01	0.00999999999999999\\
79.01	0.00999999999999999\\
80.01	0.00999999999999999\\
81.01	0.00999999999999999\\
82.01	0.00999999999999999\\
83.01	0.00999999999999999\\
84.01	0.00999999999999999\\
85.01	0.00999999999999999\\
86.01	0.00999999999999999\\
87.01	0.00999999999999999\\
88.01	0.00999999999999999\\
89.01	0.00999999999999999\\
90.01	0.00999999999999999\\
91.01	0.00999999999999999\\
92.01	0.00999999999999999\\
93.01	0.00999999999999999\\
94.01	0.00999999999999999\\
95.01	0.00999999999999999\\
96.01	0.00999999999999999\\
97.01	0.00999999999999999\\
98.01	0.00999999999999999\\
99.01	0.00999999999999999\\
100.01	0.00999999999999999\\
101.01	0.00999999999999999\\
102.01	0.00999999999999999\\
103.01	0.00999999999999999\\
104.01	0.00999999999999999\\
105.01	0.00999999999999999\\
106.01	0.00999999999999999\\
107.01	0.00999999999999999\\
108.01	0.00999999999999999\\
109.01	0.00999999999999999\\
110.01	0.00999999999999999\\
111.01	0.00999999999999999\\
112.01	0.00999999999999999\\
113.01	0.00999999999999999\\
114.01	0.00999999999999999\\
115.01	0.00999999999999999\\
116.01	0.00999999999999999\\
117.01	0.00999999999999999\\
118.01	0.00999999999999999\\
119.01	0.00999999999999999\\
120.01	0.00999999999999999\\
121.01	0.00999999999999999\\
122.01	0.00999999999999999\\
123.01	0.00999999999999999\\
124.01	0.00999999999999999\\
125.01	0.00999999999999999\\
126.01	0.00999999999999999\\
127.01	0.00999999999999999\\
128.01	0.00999999999999999\\
129.01	0.00999999999999999\\
130.01	0.00999999999999999\\
131.01	0.00999999999999999\\
132.01	0.00999999999999999\\
133.01	0.00999999999999999\\
134.01	0.00999999999999999\\
135.01	0.00999999999999999\\
136.01	0.00999999999999999\\
137.01	0.00999999999999999\\
138.01	0.00999999999999999\\
139.01	0.00999999999999999\\
140.01	0.00999999999999999\\
141.01	0.00999999999999999\\
142.01	0.00999999999999999\\
143.01	0.00999999999999999\\
144.01	0.00999999999999999\\
145.01	0.00999999999999999\\
146.01	0.00999999999999999\\
147.01	0.00999999999999999\\
148.01	0.00999999999999999\\
149.01	0.00999999999999999\\
150.01	0.00999999999999999\\
151.01	0.00999999999999999\\
152.01	0.00999999999999999\\
153.01	0.00999999999999999\\
154.01	0.00999999999999999\\
155.01	0.00999999999999999\\
156.01	0.00999999999999999\\
157.01	0.00999999999999999\\
158.01	0.00999999999999999\\
159.01	0.00999999999999999\\
160.01	0.00999999999999999\\
161.01	0.00999999999999999\\
162.01	0.00999999999999999\\
163.01	0.00999999999999999\\
164.01	0.00999999999999999\\
165.01	0.00999999999999999\\
166.01	0.00999999999999999\\
167.01	0.00999999999999999\\
168.01	0.00999999999999999\\
169.01	0.00999999999999999\\
170.01	0.00999999999999999\\
171.01	0.00999999999999999\\
172.01	0.00999999999999999\\
173.01	0.00999999999999999\\
174.01	0.00999999999999999\\
175.01	0.00999999999999999\\
176.01	0.00999999999999999\\
177.01	0.00999999999999999\\
178.01	0.00999999999999999\\
179.01	0.00999999999999999\\
180.01	0.00999999999999999\\
181.01	0.00999999999999999\\
182.01	0.00999999999999999\\
183.01	0.00999999999999999\\
184.01	0.00999999999999999\\
185.01	0.00999999999999999\\
186.01	0.00999999999999999\\
187.01	0.00999999999999999\\
188.01	0.00999999999999999\\
189.01	0.00999999999999999\\
190.01	0.00999999999999999\\
191.01	0.00999999999999999\\
192.01	0.00999999999999999\\
193.01	0.00999999999999999\\
194.01	0.00999999999999999\\
195.01	0.00999999999999999\\
196.01	0.00999999999999999\\
197.01	0.00999999999999999\\
198.01	0.00999999999999999\\
199.01	0.00999999999999999\\
200.01	0.00999999999999999\\
201.01	0.00999999999999999\\
202.01	0.00999999999999999\\
203.01	0.00999999999999999\\
204.01	0.00999999999999999\\
205.01	0.00999999999999999\\
206.01	0.00999999999999999\\
207.01	0.00999999999999999\\
208.01	0.00999999999999999\\
209.01	0.00999999999999999\\
210.01	0.00999999999999999\\
211.01	0.00999999999999999\\
212.01	0.00999999999999999\\
213.01	0.00999999999999999\\
214.01	0.00999999999999999\\
215.01	0.00999999999999999\\
216.01	0.00999999999999999\\
217.01	0.00999999999999999\\
218.01	0.00999999999999999\\
219.01	0.00999999999999999\\
220.01	0.00999999999999999\\
221.01	0.00999999999999999\\
222.01	0.00999999999999999\\
223.01	0.00999999999999999\\
224.01	0.00999999999999999\\
225.01	0.00999999999999999\\
226.01	0.00999999999999999\\
227.01	0.00999999999999999\\
228.01	0.00999999999999999\\
229.01	0.00999999999999999\\
230.01	0.00999999999999999\\
231.01	0.00999999999999999\\
232.01	0.00999999999999999\\
233.01	0.00999999999999999\\
234.01	0.00999999999999999\\
235.01	0.00999999999999999\\
236.01	0.00999999999999999\\
237.01	0.00999999999999999\\
238.01	0.00999999999999999\\
239.01	0.00999999999999999\\
240.01	0.00999999999999999\\
241.01	0.00999999999999999\\
242.01	0.00999999999999999\\
243.01	0.00999999999999999\\
244.01	0.00999999999999999\\
245.01	0.00999999999999999\\
246.01	0.00999999999999999\\
247.01	0.00999999999999999\\
248.01	0.00999999999999999\\
249.01	0.00999999999999999\\
250.01	0.00999999999999999\\
251.01	0.00999999999999999\\
252.01	0.00999999999999999\\
253.01	0.00999999999999999\\
254.01	0.00999999999999999\\
255.01	0.00999999999999999\\
256.01	0.00999999999999999\\
257.01	0.00999999999999999\\
258.01	0.00999999999999999\\
259.01	0.00999999999999999\\
260.01	0.00999999999999999\\
261.01	0.00999999999999999\\
262.01	0.00999999999999999\\
263.01	0.00999999999999999\\
264.01	0.00999999999999999\\
265.01	0.00999999999999999\\
266.01	0.00999999999999999\\
267.01	0.00999999999999999\\
268.01	0.00999999999999999\\
269.01	0.00999999999999999\\
270.01	0.00999999999999999\\
271.01	0.00999999999999999\\
272.01	0.00999999999999999\\
273.01	0.00999999999999999\\
274.01	0.00999999999999999\\
275.01	0.00999999999999999\\
276.01	0.00999999999999999\\
277.01	0.00999999999999999\\
278.01	0.00999999999999999\\
279.01	0.00999999999999999\\
280.01	0.00999999999999999\\
281.01	0.00999999999999999\\
282.01	0.00999999999999999\\
283.01	0.00999999999999999\\
284.01	0.00999999999999999\\
285.01	0.00999999999999999\\
286.01	0.00999999999999999\\
287.01	0.00999999999999999\\
288.01	0.00999999999999999\\
289.01	0.00999999999999999\\
290.01	0.00999999999999999\\
291.01	0.00999999999999999\\
292.01	0.00999999999999999\\
293.01	0.00999999999999999\\
294.01	0.00999999999999999\\
295.01	0.00999999999999999\\
296.01	0.00999999999999999\\
297.01	0.00999999999999999\\
298.01	0.00999999999999999\\
299.01	0.00999999999999999\\
300.01	0.00999999999999999\\
301.01	0.00999999999999999\\
302.01	0.00999999999999999\\
303.01	0.00999999999999999\\
304.01	0.00999999999999999\\
305.01	0.00999999999999999\\
306.01	0.00999999999999999\\
307.01	0.00999999999999999\\
308.01	0.00999999999999999\\
309.01	0.00999999999999999\\
310.01	0.00999999999999999\\
311.01	0.00999999999999999\\
312.01	0.00999999999999999\\
313.01	0.00999999999999999\\
314.01	0.00999999999999999\\
315.01	0.00999999999999999\\
316.01	0.00999999999999999\\
317.01	0.00999999999999999\\
318.01	0.00999999999999999\\
319.01	0.00999999999999999\\
320.01	0.00999999999999999\\
321.01	0.00999999999999999\\
322.01	0.00999999999999999\\
323.01	0.00999999999999999\\
324.01	0.00999999999999999\\
325.01	0.00999999999999999\\
326.01	0.00999999999999999\\
327.01	0.00999999999999999\\
328.01	0.00999999999999999\\
329.01	0.00999999999999999\\
330.01	0.00999999999999999\\
331.01	0.00999999999999999\\
332.01	0.00999999999999999\\
333.01	0.00999999999999999\\
334.01	0.00999999999999999\\
335.01	0.00999999999999999\\
336.01	0.00999999999999999\\
337.01	0.00999999999999999\\
338.01	0.00999999999999999\\
339.01	0.00999999999999999\\
340.01	0.00999999999999999\\
341.01	0.00999999999999999\\
342.01	0.00999999999999999\\
343.01	0.00999999999999999\\
344.01	0.00999999999999999\\
345.01	0.00999999999999999\\
346.01	0.00999999999999999\\
347.01	0.00999999999999999\\
348.01	0.00999999999999999\\
349.01	0.00999999999999999\\
350.01	0.00999999999999999\\
351.01	0.00999999999999999\\
352.01	0.00999999999999999\\
353.01	0.00999999999999999\\
354.01	0.00999999999999999\\
355.01	0.00999999999999999\\
356.01	0.00999999999999999\\
357.01	0.00999999999999999\\
358.01	0.00999999999999999\\
359.01	0.00999999999999999\\
360.01	0.00999999999999999\\
361.01	0.00999999999999999\\
362.01	0.00999999999999999\\
363.01	0.00999999999999999\\
364.01	0.00999999999999999\\
365.01	0.00999999999999999\\
366.01	0.00999999999999999\\
367.01	0.00999999999999999\\
368.01	0.00999999999999999\\
369.01	0.00999999999999999\\
370.01	0.00999999999999999\\
371.01	0.00999999999999999\\
372.01	0.00999999999999999\\
373.01	0.00999999999999999\\
374.01	0.00999999999999999\\
375.01	0.00999999999999999\\
376.01	0.00999999999999999\\
377.01	0.00999999999999999\\
378.01	0.00999999999999999\\
379.01	0.00999999999999999\\
380.01	0.00999999999999999\\
381.01	0.00999999999999999\\
382.01	0.00999999999999999\\
383.01	0.00999999999999999\\
384.01	0.00999999999999999\\
385.01	0.00999999999999999\\
386.01	0.00999999999999999\\
387.01	0.00999999999999999\\
388.01	0.00999999999999999\\
389.01	0.00999999999999999\\
390.01	0.00999999999999999\\
391.01	0.00999999999999999\\
392.01	0.00999999999999999\\
393.01	0.00999999999999999\\
394.01	0.00999999999999999\\
395.01	0.00999999999999999\\
396.01	0.00999999999999999\\
397.01	0.00999999999999999\\
398.01	0.00999999999999999\\
399.01	0.00999999999999999\\
400.01	0.00999999999999999\\
401.01	0.00999999999999999\\
402.01	0.00999999999999999\\
403.01	0.00999999999999999\\
404.01	0.00999999999999999\\
405.01	0.00999999999999999\\
406.01	0.00999999999999999\\
407.01	0.00999999999999999\\
408.01	0.00999999999999999\\
409.01	0.00999999999999999\\
410.01	0.00999999999999999\\
411.01	0.00999999999999999\\
412.01	0.00999999999999999\\
413.01	0.00999999999999999\\
414.01	0.00999999999999999\\
415.01	0.00999999999999999\\
416.01	0.00999999999999999\\
417.01	0.00999999999999999\\
418.01	0.00999999999999999\\
419.01	0.00999999999999999\\
420.01	0.00999999999999999\\
421.01	0.00999999999999999\\
422.01	0.00999999999999999\\
423.01	0.00999999999999999\\
424.01	0.00999999999999999\\
425.01	0.00999999999999999\\
426.01	0.00999999999999999\\
427.01	0.00999999999999999\\
428.01	0.00999999999999999\\
429.01	0.00999999999999999\\
430.01	0.00999999999999999\\
431.01	0.00999999999999999\\
432.01	0.00999999999999999\\
433.01	0.00999999999999999\\
434.01	0.00999999999999999\\
435.01	0.00999999999999999\\
436.01	0.00999999999999999\\
437.01	0.00999999999999999\\
438.01	0.00999999999999999\\
439.01	0.00999999999999999\\
440.01	0.00999999999999999\\
441.01	0.00999999999999999\\
442.01	0.00999999999999999\\
443.01	0.00999999999999999\\
444.01	0.00999999999999999\\
445.01	0.00999999999999999\\
446.01	0.00999999999999999\\
447.01	0.00999999999999999\\
448.01	0.00999999999999999\\
449.01	0.00999999999999999\\
450.01	0.00999999999999999\\
451.01	0.00999999999999999\\
452.01	0.00999999999999999\\
453.01	0.00999999999999999\\
454.01	0.00999999999999999\\
455.01	0.00999999999999999\\
456.01	0.00999999999999999\\
457.01	0.00999999999999999\\
458.01	0.00999999999999999\\
459.01	0.00999999999999999\\
460.01	0.00999999999999999\\
461.01	0.00999999999999999\\
462.01	0.00999999999999999\\
463.01	0.00999999999999999\\
464.01	0.00999999999999999\\
465.01	0.00999999999999999\\
466.01	0.00999999999999999\\
467.01	0.00999999999999999\\
468.01	0.00999999999999999\\
469.01	0.00999999999999999\\
470.01	0.00999999999999999\\
471.01	0.00999999999999999\\
472.01	0.00999999999999999\\
473.01	0.00999999999999999\\
474.01	0.00999999999999999\\
475.01	0.00999999999999999\\
476.01	0.00999999999999999\\
477.01	0.00999999999999999\\
478.01	0.00999999999999999\\
479.01	0.00999999999999999\\
480.01	0.00999999999999999\\
481.01	0.00999999999999999\\
482.01	0.00999999999999999\\
483.01	0.00999999999999999\\
484.01	0.00999999999999999\\
485.01	0.00999999999999999\\
486.01	0.00999999999999999\\
487.01	0.00999999999999999\\
488.01	0.00999999999999999\\
489.01	0.00999999999999999\\
490.01	0.00999999999999999\\
491.01	0.00999999999999999\\
492.01	0.00999999999999999\\
493.01	0.00999999999999999\\
494.01	0.00999999999999999\\
495.01	0.00999999999999999\\
496.01	0.00999999999999999\\
497.01	0.00999999999999999\\
498.01	0.00999999999999999\\
499.01	0.00999999999999999\\
500.01	0.00999999999999999\\
501.01	0.00999999999999999\\
502.01	0.00999999999999999\\
503.01	0.00999999999999999\\
504.01	0.00999999999999999\\
505.01	0.00999999999999999\\
506.01	0.00999999999999999\\
507.01	0.00999999999999999\\
508.01	0.00999999999999999\\
509.01	0.00999999999999999\\
510.01	0.00999999999999999\\
511.01	0.00999999999999999\\
512.01	0.00999999999999999\\
513.01	0.00999999999999999\\
514.01	0.00999999999999999\\
515.01	0.00999999999999999\\
516.01	0.00999999999999999\\
517.01	0.00999999999999999\\
518.01	0.00999999999999999\\
519.01	0.00999999999999999\\
520.01	0.00999999999999999\\
521.01	0.00999999999999999\\
522.01	0.00999999999999999\\
523.01	0.00999999999999999\\
524.01	0.00999999999999999\\
525.01	0.00999999999999999\\
526.01	0.00999999999999999\\
527.01	0.00999999999999999\\
528.01	0.00999999999999999\\
529.01	0.00999999999999999\\
530.01	0.00999999999999999\\
531.01	0.00999999999999999\\
532.01	0.00999999999999999\\
533.01	0.00999999999999999\\
534.01	0.00999999999999999\\
535.01	0.00999999999999999\\
536.01	0.00999999999999999\\
537.01	0.00999999999999999\\
538.01	0.00999999999999999\\
539.01	0.00999999999999999\\
540.01	0.00999999999999999\\
541.01	0.00999999999999999\\
542.01	0.00999999999999999\\
543.01	0.00999999999999999\\
544.01	0.00999999999999999\\
545.01	0.00999999999999999\\
546.01	0.00999999999999999\\
547.01	0.00999999999999999\\
548.01	0.00999999999999999\\
549.01	0.00999999999999999\\
550.01	0.00999999999999999\\
551.01	0.00999999999999999\\
552.01	0.00999999999999999\\
553.01	0.00999999999999999\\
554.01	0.00999999999999999\\
555.01	0.00999999999999999\\
556.01	0.00999999999999999\\
557.01	0.00999999999999999\\
558.01	0.00999999999999999\\
559.01	0.00999999999999999\\
560.01	0.00999999999999999\\
561.01	0.00999999999999999\\
562.01	0.00999999999999999\\
563.01	0.00999999999999999\\
564.01	0.00999999999999999\\
565.01	0.00999999999999999\\
566.01	0.00999999999999999\\
567.01	0.00999999999999999\\
568.01	0.00999999999999999\\
569.01	0.00999999999999999\\
570.01	0.00999999999999999\\
571.01	0.00999999999999999\\
572.01	0.00999999999999999\\
573.01	0.00999999999999999\\
574.01	0.00999999999999999\\
575.01	0.00999999999999999\\
576.01	0.00999999999999999\\
577.01	0.00999999999999999\\
578.01	0.00999999999999999\\
579.01	0.00999999999999999\\
580.01	0.00999999999999999\\
581.01	0.00999999999999999\\
582.01	0.00999999999999999\\
583.01	0.00999999999999999\\
584.01	0.00999999999999999\\
585.01	0.00999999999999999\\
586.01	0.00999999999999999\\
587.01	0.00999999999999999\\
588.01	0.00999999999999999\\
589.01	0.00999999999999999\\
590.01	0.00999999999999999\\
591.01	0.00999999999999999\\
592.01	0.00999999999999999\\
593.01	0.00999999999999999\\
594.01	0.00999999999999999\\
595.01	0.00999999999999999\\
596.01	0.00999999999999999\\
597.01	0.00999999999999999\\
598.01	0.00999999999999999\\
599.01	0.00624186909636788\\
599.02	0.00620414441457609\\
599.03	0.00616605275823867\\
599.04	0.00612759052289781\\
599.05	0.00608875406868525\\
599.06	0.00604953971997439\\
599.07	0.00600994376502898\\
599.08	0.00596996245564835\\
599.09	0.0059295920068092\\
599.1	0.00588882859630377\\
599.11	0.00584766836437461\\
599.12	0.00580610741334552\\
599.13	0.00576414180724917\\
599.14	0.00572176757145079\\
599.15	0.00567898069226832\\
599.16	0.0056357771165888\\
599.17	0.00559215275148091\\
599.18	0.00554810346380382\\
599.19	0.00550362507981218\\
599.2	0.00545871338475718\\
599.21	0.0054133641224837\\
599.22	0.00536757299502359\\
599.23	0.00532133566218485\\
599.24	0.00527464774113687\\
599.25	0.00522750480599146\\
599.26	0.00517990238737996\\
599.27	0.00513183597202595\\
599.28	0.00508330100231403\\
599.29	0.00503429287585414\\
599.3	0.00498480694504173\\
599.31	0.00493483851661353\\
599.32	0.00488438285119905\\
599.33	0.00483343516286759\\
599.34	0.00478199061867081\\
599.35	0.00473004433818085\\
599.36	0.00467759139302389\\
599.37	0.00462462680640909\\
599.38	0.004571145552653\\
599.39	0.00451714256067378\\
599.4	0.0044626127239245\\
599.41	0.00440755088587263\\
599.42	0.00435195183951073\\
599.43	0.00429581032686245\\
599.44	0.00423912103848351\\
599.45	0.00418187861295803\\
599.46	0.00412407763638975\\
599.47	0.00406571264188831\\
599.48	0.00400677810905057\\
599.49	0.00394726846343678\\
599.5	0.00388717807604165\\
599.51	0.00382650126276023\\
599.52	0.0037652322838486\\
599.53	0.00370336534337923\\
599.54	0.00364089458869102\\
599.55	0.00357781410983402\\
599.56	0.00351411793900859\\
599.57	0.00344980004999922\\
599.58	0.00338485435760266\\
599.59	0.00331927471705059\\
599.6	0.00325305492342653\\
599.61	0.00318618871107708\\
599.62	0.00311866975301747\\
599.63	0.00305049166033112\\
599.64	0.00298164798156351\\
599.65	0.00291213220211002\\
599.66	0.00284193774359775\\
599.67	0.00277105796326139\\
599.68	0.00269948615331285\\
599.69	0.00262721554030489\\
599.7	0.00255423928448833\\
599.71	0.0024805504791631\\
599.72	0.0024061421500229\\
599.73	0.00233100725449347\\
599.74	0.00225513868106428\\
599.75	0.00217852924861384\\
599.76	0.00210117170572828\\
599.77	0.00202305873001331\\
599.78	0.00194418292739942\\
599.79	0.00186453683144033\\
599.8	0.00178411290260452\\
599.81	0.00170290352755984\\
599.82	0.00162090101845112\\
599.83	0.00153809761217074\\
599.84	0.00145448546962201\\
599.85	0.00137005667497535\\
599.86	0.00128480323491721\\
599.87	0.00119871707789166\\
599.88	0.00111179005333448\\
599.89	0.00102401393089982\\
599.9	0.000935380399679274\\
599.91	0.000845881067413288\\
599.92	0.00075550745969488\\
599.93	0.000664251019165561\\
599.94	0.000572103104703356\\
599.95	0.00047905499060291\\
599.96	0.000385097865747556\\
599.97	0.000290222832773275\\
599.98	0.000194420907224489\\
599.99	9.76830167015649e-05\\
600	0\\
};
\addplot [color=red!40!mycolor19,solid,forget plot]
  table[row sep=crcr]{%
0.01	0.00999999999999999\\
1.01	0.00999999999999999\\
2.01	0.00999999999999999\\
3.01	0.00999999999999999\\
4.01	0.00999999999999999\\
5.01	0.00999999999999999\\
6.01	0.00999999999999999\\
7.01	0.00999999999999999\\
8.01	0.00999999999999999\\
9.01	0.00999999999999999\\
10.01	0.00999999999999999\\
11.01	0.00999999999999999\\
12.01	0.00999999999999999\\
13.01	0.00999999999999999\\
14.01	0.00999999999999999\\
15.01	0.00999999999999999\\
16.01	0.00999999999999999\\
17.01	0.00999999999999999\\
18.01	0.00999999999999999\\
19.01	0.00999999999999999\\
20.01	0.00999999999999999\\
21.01	0.00999999999999999\\
22.01	0.00999999999999999\\
23.01	0.00999999999999999\\
24.01	0.00999999999999999\\
25.01	0.00999999999999999\\
26.01	0.00999999999999999\\
27.01	0.00999999999999999\\
28.01	0.00999999999999999\\
29.01	0.00999999999999999\\
30.01	0.00999999999999999\\
31.01	0.00999999999999999\\
32.01	0.00999999999999999\\
33.01	0.00999999999999999\\
34.01	0.00999999999999999\\
35.01	0.00999999999999999\\
36.01	0.00999999999999999\\
37.01	0.00999999999999999\\
38.01	0.00999999999999999\\
39.01	0.00999999999999999\\
40.01	0.00999999999999999\\
41.01	0.00999999999999999\\
42.01	0.00999999999999999\\
43.01	0.00999999999999999\\
44.01	0.00999999999999999\\
45.01	0.00999999999999999\\
46.01	0.00999999999999999\\
47.01	0.00999999999999999\\
48.01	0.00999999999999999\\
49.01	0.00999999999999999\\
50.01	0.00999999999999999\\
51.01	0.00999999999999999\\
52.01	0.00999999999999999\\
53.01	0.00999999999999999\\
54.01	0.00999999999999999\\
55.01	0.00999999999999999\\
56.01	0.00999999999999999\\
57.01	0.00999999999999999\\
58.01	0.00999999999999999\\
59.01	0.00999999999999999\\
60.01	0.00999999999999999\\
61.01	0.00999999999999999\\
62.01	0.00999999999999999\\
63.01	0.00999999999999999\\
64.01	0.00999999999999999\\
65.01	0.00999999999999999\\
66.01	0.00999999999999999\\
67.01	0.00999999999999999\\
68.01	0.00999999999999999\\
69.01	0.00999999999999999\\
70.01	0.00999999999999999\\
71.01	0.00999999999999999\\
72.01	0.00999999999999999\\
73.01	0.00999999999999999\\
74.01	0.00999999999999999\\
75.01	0.00999999999999999\\
76.01	0.00999999999999999\\
77.01	0.00999999999999999\\
78.01	0.00999999999999999\\
79.01	0.00999999999999999\\
80.01	0.00999999999999999\\
81.01	0.00999999999999999\\
82.01	0.00999999999999999\\
83.01	0.00999999999999999\\
84.01	0.00999999999999999\\
85.01	0.00999999999999999\\
86.01	0.00999999999999999\\
87.01	0.00999999999999999\\
88.01	0.00999999999999999\\
89.01	0.00999999999999999\\
90.01	0.00999999999999999\\
91.01	0.00999999999999999\\
92.01	0.00999999999999999\\
93.01	0.00999999999999999\\
94.01	0.00999999999999999\\
95.01	0.00999999999999999\\
96.01	0.00999999999999999\\
97.01	0.00999999999999999\\
98.01	0.00999999999999999\\
99.01	0.00999999999999999\\
100.01	0.00999999999999999\\
101.01	0.00999999999999999\\
102.01	0.00999999999999999\\
103.01	0.00999999999999999\\
104.01	0.00999999999999999\\
105.01	0.00999999999999999\\
106.01	0.00999999999999999\\
107.01	0.00999999999999999\\
108.01	0.00999999999999999\\
109.01	0.00999999999999999\\
110.01	0.00999999999999999\\
111.01	0.00999999999999999\\
112.01	0.00999999999999999\\
113.01	0.00999999999999999\\
114.01	0.00999999999999999\\
115.01	0.00999999999999999\\
116.01	0.00999999999999999\\
117.01	0.00999999999999999\\
118.01	0.00999999999999999\\
119.01	0.00999999999999999\\
120.01	0.00999999999999999\\
121.01	0.00999999999999999\\
122.01	0.00999999999999999\\
123.01	0.00999999999999999\\
124.01	0.00999999999999999\\
125.01	0.00999999999999999\\
126.01	0.00999999999999999\\
127.01	0.00999999999999999\\
128.01	0.00999999999999999\\
129.01	0.00999999999999999\\
130.01	0.00999999999999999\\
131.01	0.00999999999999999\\
132.01	0.00999999999999999\\
133.01	0.00999999999999999\\
134.01	0.00999999999999999\\
135.01	0.00999999999999999\\
136.01	0.00999999999999999\\
137.01	0.00999999999999999\\
138.01	0.00999999999999999\\
139.01	0.00999999999999999\\
140.01	0.00999999999999999\\
141.01	0.00999999999999999\\
142.01	0.00999999999999999\\
143.01	0.00999999999999999\\
144.01	0.00999999999999999\\
145.01	0.00999999999999999\\
146.01	0.00999999999999999\\
147.01	0.00999999999999999\\
148.01	0.00999999999999999\\
149.01	0.00999999999999999\\
150.01	0.00999999999999999\\
151.01	0.00999999999999999\\
152.01	0.00999999999999999\\
153.01	0.00999999999999999\\
154.01	0.00999999999999999\\
155.01	0.00999999999999999\\
156.01	0.00999999999999999\\
157.01	0.00999999999999999\\
158.01	0.00999999999999999\\
159.01	0.00999999999999999\\
160.01	0.00999999999999999\\
161.01	0.00999999999999999\\
162.01	0.00999999999999999\\
163.01	0.00999999999999999\\
164.01	0.00999999999999999\\
165.01	0.00999999999999999\\
166.01	0.00999999999999999\\
167.01	0.00999999999999999\\
168.01	0.00999999999999999\\
169.01	0.00999999999999999\\
170.01	0.00999999999999999\\
171.01	0.00999999999999999\\
172.01	0.00999999999999999\\
173.01	0.00999999999999999\\
174.01	0.00999999999999999\\
175.01	0.00999999999999999\\
176.01	0.00999999999999999\\
177.01	0.00999999999999999\\
178.01	0.00999999999999999\\
179.01	0.00999999999999999\\
180.01	0.00999999999999999\\
181.01	0.00999999999999999\\
182.01	0.00999999999999999\\
183.01	0.00999999999999999\\
184.01	0.00999999999999999\\
185.01	0.00999999999999999\\
186.01	0.00999999999999999\\
187.01	0.00999999999999999\\
188.01	0.00999999999999999\\
189.01	0.00999999999999999\\
190.01	0.00999999999999999\\
191.01	0.00999999999999999\\
192.01	0.00999999999999999\\
193.01	0.00999999999999999\\
194.01	0.00999999999999999\\
195.01	0.00999999999999999\\
196.01	0.00999999999999999\\
197.01	0.00999999999999999\\
198.01	0.00999999999999999\\
199.01	0.00999999999999999\\
200.01	0.00999999999999999\\
201.01	0.00999999999999999\\
202.01	0.00999999999999999\\
203.01	0.00999999999999999\\
204.01	0.00999999999999999\\
205.01	0.00999999999999999\\
206.01	0.00999999999999999\\
207.01	0.00999999999999999\\
208.01	0.00999999999999999\\
209.01	0.00999999999999999\\
210.01	0.00999999999999999\\
211.01	0.00999999999999999\\
212.01	0.00999999999999999\\
213.01	0.00999999999999999\\
214.01	0.00999999999999999\\
215.01	0.00999999999999999\\
216.01	0.00999999999999999\\
217.01	0.00999999999999999\\
218.01	0.00999999999999999\\
219.01	0.00999999999999999\\
220.01	0.00999999999999999\\
221.01	0.00999999999999999\\
222.01	0.00999999999999999\\
223.01	0.00999999999999999\\
224.01	0.00999999999999999\\
225.01	0.00999999999999999\\
226.01	0.00999999999999999\\
227.01	0.00999999999999999\\
228.01	0.00999999999999999\\
229.01	0.00999999999999999\\
230.01	0.00999999999999999\\
231.01	0.00999999999999999\\
232.01	0.00999999999999999\\
233.01	0.00999999999999999\\
234.01	0.00999999999999999\\
235.01	0.00999999999999999\\
236.01	0.00999999999999999\\
237.01	0.00999999999999999\\
238.01	0.00999999999999999\\
239.01	0.00999999999999999\\
240.01	0.00999999999999999\\
241.01	0.00999999999999999\\
242.01	0.00999999999999999\\
243.01	0.00999999999999999\\
244.01	0.00999999999999999\\
245.01	0.00999999999999999\\
246.01	0.00999999999999999\\
247.01	0.00999999999999999\\
248.01	0.00999999999999999\\
249.01	0.00999999999999999\\
250.01	0.00999999999999999\\
251.01	0.00999999999999999\\
252.01	0.00999999999999999\\
253.01	0.00999999999999999\\
254.01	0.00999999999999999\\
255.01	0.00999999999999999\\
256.01	0.00999999999999999\\
257.01	0.00999999999999999\\
258.01	0.00999999999999999\\
259.01	0.00999999999999999\\
260.01	0.00999999999999999\\
261.01	0.00999999999999999\\
262.01	0.00999999999999999\\
263.01	0.00999999999999999\\
264.01	0.00999999999999999\\
265.01	0.00999999999999999\\
266.01	0.00999999999999999\\
267.01	0.00999999999999999\\
268.01	0.00999999999999999\\
269.01	0.00999999999999999\\
270.01	0.00999999999999999\\
271.01	0.00999999999999999\\
272.01	0.00999999999999999\\
273.01	0.00999999999999999\\
274.01	0.00999999999999999\\
275.01	0.00999999999999999\\
276.01	0.00999999999999999\\
277.01	0.00999999999999999\\
278.01	0.00999999999999999\\
279.01	0.00999999999999999\\
280.01	0.00999999999999999\\
281.01	0.00999999999999999\\
282.01	0.00999999999999999\\
283.01	0.00999999999999999\\
284.01	0.00999999999999999\\
285.01	0.00999999999999999\\
286.01	0.00999999999999999\\
287.01	0.00999999999999999\\
288.01	0.00999999999999999\\
289.01	0.00999999999999999\\
290.01	0.00999999999999999\\
291.01	0.00999999999999999\\
292.01	0.00999999999999999\\
293.01	0.00999999999999999\\
294.01	0.00999999999999999\\
295.01	0.00999999999999999\\
296.01	0.00999999999999999\\
297.01	0.00999999999999999\\
298.01	0.00999999999999999\\
299.01	0.00999999999999999\\
300.01	0.00999999999999999\\
301.01	0.00999999999999999\\
302.01	0.00999999999999999\\
303.01	0.00999999999999999\\
304.01	0.00999999999999999\\
305.01	0.00999999999999999\\
306.01	0.00999999999999999\\
307.01	0.00999999999999999\\
308.01	0.00999999999999999\\
309.01	0.00999999999999999\\
310.01	0.00999999999999999\\
311.01	0.00999999999999999\\
312.01	0.00999999999999999\\
313.01	0.00999999999999999\\
314.01	0.00999999999999999\\
315.01	0.00999999999999999\\
316.01	0.00999999999999999\\
317.01	0.00999999999999999\\
318.01	0.00999999999999999\\
319.01	0.00999999999999999\\
320.01	0.00999999999999999\\
321.01	0.00999999999999999\\
322.01	0.00999999999999999\\
323.01	0.00999999999999999\\
324.01	0.00999999999999999\\
325.01	0.00999999999999999\\
326.01	0.00999999999999999\\
327.01	0.00999999999999999\\
328.01	0.00999999999999999\\
329.01	0.00999999999999999\\
330.01	0.00999999999999999\\
331.01	0.00999999999999999\\
332.01	0.00999999999999999\\
333.01	0.00999999999999999\\
334.01	0.00999999999999999\\
335.01	0.00999999999999999\\
336.01	0.00999999999999999\\
337.01	0.00999999999999999\\
338.01	0.00999999999999999\\
339.01	0.00999999999999999\\
340.01	0.00999999999999999\\
341.01	0.00999999999999999\\
342.01	0.00999999999999999\\
343.01	0.00999999999999999\\
344.01	0.00999999999999999\\
345.01	0.00999999999999999\\
346.01	0.00999999999999999\\
347.01	0.00999999999999999\\
348.01	0.00999999999999999\\
349.01	0.00999999999999999\\
350.01	0.00999999999999999\\
351.01	0.00999999999999999\\
352.01	0.00999999999999999\\
353.01	0.00999999999999999\\
354.01	0.00999999999999999\\
355.01	0.00999999999999999\\
356.01	0.00999999999999999\\
357.01	0.00999999999999999\\
358.01	0.00999999999999999\\
359.01	0.00999999999999999\\
360.01	0.00999999999999999\\
361.01	0.00999999999999999\\
362.01	0.00999999999999999\\
363.01	0.00999999999999999\\
364.01	0.00999999999999999\\
365.01	0.00999999999999999\\
366.01	0.00999999999999999\\
367.01	0.00999999999999999\\
368.01	0.00999999999999999\\
369.01	0.00999999999999999\\
370.01	0.00999999999999999\\
371.01	0.00999999999999999\\
372.01	0.00999999999999999\\
373.01	0.00999999999999999\\
374.01	0.00999999999999999\\
375.01	0.00999999999999999\\
376.01	0.00999999999999999\\
377.01	0.00999999999999999\\
378.01	0.00999999999999999\\
379.01	0.00999999999999999\\
380.01	0.00999999999999999\\
381.01	0.00999999999999999\\
382.01	0.00999999999999999\\
383.01	0.00999999999999999\\
384.01	0.00999999999999999\\
385.01	0.00999999999999999\\
386.01	0.00999999999999999\\
387.01	0.00999999999999999\\
388.01	0.00999999999999999\\
389.01	0.00999999999999999\\
390.01	0.00999999999999999\\
391.01	0.00999999999999999\\
392.01	0.00999999999999999\\
393.01	0.00999999999999999\\
394.01	0.00999999999999999\\
395.01	0.00999999999999999\\
396.01	0.00999999999999999\\
397.01	0.00999999999999999\\
398.01	0.00999999999999999\\
399.01	0.00999999999999999\\
400.01	0.00999999999999999\\
401.01	0.00999999999999999\\
402.01	0.00999999999999999\\
403.01	0.00999999999999999\\
404.01	0.00999999999999999\\
405.01	0.00999999999999999\\
406.01	0.00999999999999999\\
407.01	0.00999999999999999\\
408.01	0.00999999999999999\\
409.01	0.00999999999999999\\
410.01	0.00999999999999999\\
411.01	0.00999999999999999\\
412.01	0.00999999999999999\\
413.01	0.00999999999999999\\
414.01	0.00999999999999999\\
415.01	0.00999999999999999\\
416.01	0.00999999999999999\\
417.01	0.00999999999999999\\
418.01	0.00999999999999999\\
419.01	0.00999999999999999\\
420.01	0.00999999999999999\\
421.01	0.00999999999999999\\
422.01	0.00999999999999999\\
423.01	0.00999999999999999\\
424.01	0.00999999999999999\\
425.01	0.00999999999999999\\
426.01	0.00999999999999999\\
427.01	0.00999999999999999\\
428.01	0.00999999999999999\\
429.01	0.00999999999999999\\
430.01	0.00999999999999999\\
431.01	0.00999999999999999\\
432.01	0.00999999999999999\\
433.01	0.00999999999999999\\
434.01	0.00999999999999999\\
435.01	0.00999999999999999\\
436.01	0.00999999999999999\\
437.01	0.00999999999999999\\
438.01	0.00999999999999999\\
439.01	0.00999999999999999\\
440.01	0.00999999999999999\\
441.01	0.00999999999999999\\
442.01	0.00999999999999999\\
443.01	0.00999999999999999\\
444.01	0.00999999999999999\\
445.01	0.00999999999999999\\
446.01	0.00999999999999999\\
447.01	0.00999999999999999\\
448.01	0.00999999999999999\\
449.01	0.00999999999999999\\
450.01	0.00999999999999999\\
451.01	0.00999999999999999\\
452.01	0.00999999999999999\\
453.01	0.00999999999999999\\
454.01	0.00999999999999999\\
455.01	0.00999999999999999\\
456.01	0.00999999999999999\\
457.01	0.00999999999999999\\
458.01	0.00999999999999999\\
459.01	0.00999999999999999\\
460.01	0.00999999999999999\\
461.01	0.00999999999999999\\
462.01	0.00999999999999999\\
463.01	0.00999999999999999\\
464.01	0.00999999999999999\\
465.01	0.00999999999999999\\
466.01	0.00999999999999999\\
467.01	0.00999999999999999\\
468.01	0.00999999999999999\\
469.01	0.00999999999999999\\
470.01	0.00999999999999999\\
471.01	0.00999999999999999\\
472.01	0.00999999999999999\\
473.01	0.00999999999999999\\
474.01	0.00999999999999999\\
475.01	0.00999999999999999\\
476.01	0.00999999999999999\\
477.01	0.00999999999999999\\
478.01	0.00999999999999999\\
479.01	0.00999999999999999\\
480.01	0.00999999999999999\\
481.01	0.00999999999999999\\
482.01	0.00999999999999999\\
483.01	0.00999999999999999\\
484.01	0.00999999999999999\\
485.01	0.00999999999999999\\
486.01	0.00999999999999999\\
487.01	0.00999999999999999\\
488.01	0.00999999999999999\\
489.01	0.00999999999999999\\
490.01	0.00999999999999999\\
491.01	0.00999999999999999\\
492.01	0.00999999999999999\\
493.01	0.00999999999999999\\
494.01	0.00999999999999999\\
495.01	0.00999999999999999\\
496.01	0.00999999999999999\\
497.01	0.00999999999999999\\
498.01	0.00999999999999999\\
499.01	0.00999999999999999\\
500.01	0.00999999999999999\\
501.01	0.00999999999999999\\
502.01	0.00999999999999999\\
503.01	0.00999999999999999\\
504.01	0.00999999999999999\\
505.01	0.00999999999999999\\
506.01	0.00999999999999999\\
507.01	0.00999999999999999\\
508.01	0.00999999999999999\\
509.01	0.00999999999999999\\
510.01	0.00999999999999999\\
511.01	0.00999999999999999\\
512.01	0.00999999999999999\\
513.01	0.00999999999999999\\
514.01	0.00999999999999999\\
515.01	0.00999999999999999\\
516.01	0.00999999999999999\\
517.01	0.00999999999999999\\
518.01	0.00999999999999999\\
519.01	0.00999999999999999\\
520.01	0.00999999999999999\\
521.01	0.00999999999999999\\
522.01	0.00999999999999999\\
523.01	0.00999999999999999\\
524.01	0.00999999999999999\\
525.01	0.00999999999999999\\
526.01	0.00999999999999999\\
527.01	0.00999999999999999\\
528.01	0.00999999999999999\\
529.01	0.00999999999999999\\
530.01	0.00999999999999999\\
531.01	0.00999999999999999\\
532.01	0.00999999999999999\\
533.01	0.00999999999999999\\
534.01	0.00999999999999999\\
535.01	0.00999999999999999\\
536.01	0.00999999999999999\\
537.01	0.00999999999999999\\
538.01	0.00999999999999999\\
539.01	0.00999999999999999\\
540.01	0.00999999999999999\\
541.01	0.00999999999999999\\
542.01	0.00999999999999999\\
543.01	0.00999999999999999\\
544.01	0.00999999999999999\\
545.01	0.00999999999999999\\
546.01	0.00999999999999999\\
547.01	0.00999999999999999\\
548.01	0.00999999999999999\\
549.01	0.00999999999999999\\
550.01	0.00999999999999999\\
551.01	0.00999999999999999\\
552.01	0.00999999999999999\\
553.01	0.00999999999999999\\
554.01	0.00999999999999999\\
555.01	0.00999999999999999\\
556.01	0.00999999999999999\\
557.01	0.00999999999999999\\
558.01	0.00999999999999999\\
559.01	0.00999999999999999\\
560.01	0.00999999999999999\\
561.01	0.00999999999999999\\
562.01	0.00999999999999999\\
563.01	0.00999999999999999\\
564.01	0.00999999999999999\\
565.01	0.00999999999999999\\
566.01	0.00999999999999999\\
567.01	0.00999999999999999\\
568.01	0.00999999999999999\\
569.01	0.00999999999999999\\
570.01	0.00999999999999999\\
571.01	0.00999999999999999\\
572.01	0.00999999999999999\\
573.01	0.00999999999999999\\
574.01	0.00999999999999999\\
575.01	0.00999999999999999\\
576.01	0.00999999999999999\\
577.01	0.00999999999999999\\
578.01	0.00999999999999999\\
579.01	0.00999999999999999\\
580.01	0.00999999999999999\\
581.01	0.00999999999999999\\
582.01	0.00999999999999999\\
583.01	0.00999999999999999\\
584.01	0.00999999999999999\\
585.01	0.00999999999999999\\
586.01	0.00999999999999999\\
587.01	0.00999999999999999\\
588.01	0.00999999999999999\\
589.01	0.00999999999999999\\
590.01	0.00999999999999999\\
591.01	0.00999999999999999\\
592.01	0.00999999999999999\\
593.01	0.00999999999999999\\
594.01	0.00999999999999999\\
595.01	0.00999999999999999\\
596.01	0.00999999999999999\\
597.01	0.00999999999999999\\
598.01	0.00999999999999999\\
599.01	0.00624186909410445\\
599.02	0.00620414441246906\\
599.03	0.00616605275627208\\
599.04	0.00612759052105786\\
599.05	0.00608875406696011\\
599.06	0.00604953971835391\\
599.07	0.0060099437635045\\
599.08	0.0059699624542125\\
599.09	0.00592959200545564\\
599.1	0.00588882859502706\\
599.11	0.00584766836317001\\
599.12	0.00580610741220891\\
599.13	0.00576414180617682\\
599.14	0.00572176757043935\\
599.15	0.00567898069131471\\
599.16	0.00563577711569011\\
599.17	0.00559215275063443\\
599.18	0.005548103463007\\
599.19	0.00550362507906258\\
599.2	0.00545871338405247\\
599.21	0.0054133641218217\\
599.22	0.00536757299440222\\
599.23	0.00532133566160213\\
599.24	0.00527464774059092\\
599.25	0.00522750480548048\\
599.26	0.00517990238690222\\
599.27	0.00513183597157985\\
599.28	0.00508330100189798\\
599.29	0.00503429287546663\\
599.3	0.00498480694468132\\
599.31	0.00493483851627886\\
599.32	0.00488438285088877\\
599.33	0.0048334351625804\\
599.34	0.00478199061840547\\
599.35	0.00473004433793616\\
599.36	0.00467759139279867\\
599.37	0.00462462680620223\\
599.38	0.0045711455524634\\
599.39	0.00451714256050039\\
599.4	0.0044626127237663\\
599.41	0.00440755088572863\\
599.42	0.00435195183938\\
599.43	0.00429581032674405\\
599.44	0.00423912103837659\\
599.45	0.00418187861286175\\
599.46	0.00412407763630329\\
599.47	0.0040657126418109\\
599.48	0.00400677810898149\\
599.49	0.00394726846337534\\
599.5	0.00388717807598718\\
599.51	0.00382650126271212\\
599.52	0.00376523228380627\\
599.53	0.00370336534334213\\
599.54	0.00364089458865865\\
599.55	0.00357781410980589\\
599.56	0.00351411793898426\\
599.57	0.00344980004997828\\
599.58	0.00338485435758473\\
599.59	0.00331927471703531\\
599.6	0.00325305492341358\\
599.61	0.00318618871106618\\
599.62	0.00311866975300834\\
599.63	0.00305049166032353\\
599.64	0.00298164798155725\\
599.65	0.00291213220210488\\
599.66	0.00284193774359358\\
599.67	0.00277105796325802\\
599.68	0.00269948615331017\\
599.69	0.00262721554030278\\
599.7	0.00255423928448668\\
599.71	0.00248055047916183\\
599.72	0.00240614215002194\\
599.73	0.00233100725449274\\
599.74	0.00225513868106374\\
599.75	0.00217852924861345\\
599.76	0.00210117170572801\\
599.77	0.00202305873001312\\
599.78	0.00194418292739929\\
599.79	0.00186453683144024\\
599.8	0.00178411290260446\\
599.81	0.0017029035275598\\
599.82	0.0016209010184511\\
599.83	0.00153809761217073\\
599.84	0.001454485469622\\
599.85	0.00137005667497534\\
599.86	0.00128480323491721\\
599.87	0.00119871707789166\\
599.88	0.00111179005333448\\
599.89	0.00102401393089983\\
599.9	0.000935380399679275\\
599.91	0.000845881067413288\\
599.92	0.00075550745969488\\
599.93	0.000664251019165561\\
599.94	0.000572103104703356\\
599.95	0.000479054990602912\\
599.96	0.000385097865747556\\
599.97	0.000290222832773277\\
599.98	0.000194420907224489\\
599.99	9.76830167015632e-05\\
600	0\\
};
\addplot [color=red!75!mycolor17,solid,forget plot]
  table[row sep=crcr]{%
0.01	0.00999999999999999\\
1.01	0.00999999999999999\\
2.01	0.00999999999999999\\
3.01	0.00999999999999999\\
4.01	0.00999999999999999\\
5.01	0.00999999999999999\\
6.01	0.00999999999999999\\
7.01	0.00999999999999999\\
8.01	0.00999999999999999\\
9.01	0.00999999999999999\\
10.01	0.00999999999999999\\
11.01	0.00999999999999999\\
12.01	0.00999999999999999\\
13.01	0.00999999999999999\\
14.01	0.00999999999999999\\
15.01	0.00999999999999999\\
16.01	0.00999999999999999\\
17.01	0.00999999999999999\\
18.01	0.00999999999999999\\
19.01	0.00999999999999999\\
20.01	0.00999999999999999\\
21.01	0.00999999999999999\\
22.01	0.00999999999999999\\
23.01	0.00999999999999999\\
24.01	0.00999999999999999\\
25.01	0.00999999999999999\\
26.01	0.00999999999999999\\
27.01	0.00999999999999999\\
28.01	0.00999999999999999\\
29.01	0.00999999999999999\\
30.01	0.00999999999999999\\
31.01	0.00999999999999999\\
32.01	0.00999999999999999\\
33.01	0.00999999999999999\\
34.01	0.00999999999999999\\
35.01	0.00999999999999999\\
36.01	0.00999999999999999\\
37.01	0.00999999999999999\\
38.01	0.00999999999999999\\
39.01	0.00999999999999999\\
40.01	0.00999999999999999\\
41.01	0.00999999999999999\\
42.01	0.00999999999999999\\
43.01	0.00999999999999999\\
44.01	0.00999999999999999\\
45.01	0.00999999999999999\\
46.01	0.00999999999999999\\
47.01	0.00999999999999999\\
48.01	0.00999999999999999\\
49.01	0.00999999999999999\\
50.01	0.00999999999999999\\
51.01	0.00999999999999999\\
52.01	0.00999999999999999\\
53.01	0.00999999999999999\\
54.01	0.00999999999999999\\
55.01	0.00999999999999999\\
56.01	0.00999999999999999\\
57.01	0.00999999999999999\\
58.01	0.00999999999999999\\
59.01	0.00999999999999999\\
60.01	0.00999999999999999\\
61.01	0.00999999999999999\\
62.01	0.00999999999999999\\
63.01	0.00999999999999999\\
64.01	0.00999999999999999\\
65.01	0.00999999999999999\\
66.01	0.00999999999999999\\
67.01	0.00999999999999999\\
68.01	0.00999999999999999\\
69.01	0.00999999999999999\\
70.01	0.00999999999999999\\
71.01	0.00999999999999999\\
72.01	0.00999999999999999\\
73.01	0.00999999999999999\\
74.01	0.00999999999999999\\
75.01	0.00999999999999999\\
76.01	0.00999999999999999\\
77.01	0.00999999999999999\\
78.01	0.00999999999999999\\
79.01	0.00999999999999999\\
80.01	0.00999999999999999\\
81.01	0.00999999999999999\\
82.01	0.00999999999999999\\
83.01	0.00999999999999999\\
84.01	0.00999999999999999\\
85.01	0.00999999999999999\\
86.01	0.00999999999999999\\
87.01	0.00999999999999999\\
88.01	0.00999999999999999\\
89.01	0.00999999999999999\\
90.01	0.00999999999999999\\
91.01	0.00999999999999999\\
92.01	0.00999999999999999\\
93.01	0.00999999999999999\\
94.01	0.00999999999999999\\
95.01	0.00999999999999999\\
96.01	0.00999999999999999\\
97.01	0.00999999999999999\\
98.01	0.00999999999999999\\
99.01	0.00999999999999999\\
100.01	0.00999999999999999\\
101.01	0.00999999999999999\\
102.01	0.00999999999999999\\
103.01	0.00999999999999999\\
104.01	0.00999999999999999\\
105.01	0.00999999999999999\\
106.01	0.00999999999999999\\
107.01	0.00999999999999999\\
108.01	0.00999999999999999\\
109.01	0.00999999999999999\\
110.01	0.00999999999999999\\
111.01	0.00999999999999999\\
112.01	0.00999999999999999\\
113.01	0.00999999999999999\\
114.01	0.00999999999999999\\
115.01	0.00999999999999999\\
116.01	0.00999999999999999\\
117.01	0.00999999999999999\\
118.01	0.00999999999999999\\
119.01	0.00999999999999999\\
120.01	0.00999999999999999\\
121.01	0.00999999999999999\\
122.01	0.00999999999999999\\
123.01	0.00999999999999999\\
124.01	0.00999999999999999\\
125.01	0.00999999999999999\\
126.01	0.00999999999999999\\
127.01	0.00999999999999999\\
128.01	0.00999999999999999\\
129.01	0.00999999999999999\\
130.01	0.00999999999999999\\
131.01	0.00999999999999999\\
132.01	0.00999999999999999\\
133.01	0.00999999999999999\\
134.01	0.00999999999999999\\
135.01	0.00999999999999999\\
136.01	0.00999999999999999\\
137.01	0.00999999999999999\\
138.01	0.00999999999999999\\
139.01	0.00999999999999999\\
140.01	0.00999999999999999\\
141.01	0.00999999999999999\\
142.01	0.00999999999999999\\
143.01	0.00999999999999999\\
144.01	0.00999999999999999\\
145.01	0.00999999999999999\\
146.01	0.00999999999999999\\
147.01	0.00999999999999999\\
148.01	0.00999999999999999\\
149.01	0.00999999999999999\\
150.01	0.00999999999999999\\
151.01	0.00999999999999999\\
152.01	0.00999999999999999\\
153.01	0.00999999999999999\\
154.01	0.00999999999999999\\
155.01	0.00999999999999999\\
156.01	0.00999999999999999\\
157.01	0.00999999999999999\\
158.01	0.00999999999999999\\
159.01	0.00999999999999999\\
160.01	0.00999999999999999\\
161.01	0.00999999999999999\\
162.01	0.00999999999999999\\
163.01	0.00999999999999999\\
164.01	0.00999999999999999\\
165.01	0.00999999999999999\\
166.01	0.00999999999999999\\
167.01	0.00999999999999999\\
168.01	0.00999999999999999\\
169.01	0.00999999999999999\\
170.01	0.00999999999999999\\
171.01	0.00999999999999999\\
172.01	0.00999999999999999\\
173.01	0.00999999999999999\\
174.01	0.00999999999999999\\
175.01	0.00999999999999999\\
176.01	0.00999999999999999\\
177.01	0.00999999999999999\\
178.01	0.00999999999999999\\
179.01	0.00999999999999999\\
180.01	0.00999999999999999\\
181.01	0.00999999999999999\\
182.01	0.00999999999999999\\
183.01	0.00999999999999999\\
184.01	0.00999999999999999\\
185.01	0.00999999999999999\\
186.01	0.00999999999999999\\
187.01	0.00999999999999999\\
188.01	0.00999999999999999\\
189.01	0.00999999999999999\\
190.01	0.00999999999999999\\
191.01	0.00999999999999999\\
192.01	0.00999999999999999\\
193.01	0.00999999999999999\\
194.01	0.00999999999999999\\
195.01	0.00999999999999999\\
196.01	0.00999999999999999\\
197.01	0.00999999999999999\\
198.01	0.00999999999999999\\
199.01	0.00999999999999999\\
200.01	0.00999999999999999\\
201.01	0.00999999999999999\\
202.01	0.00999999999999999\\
203.01	0.00999999999999999\\
204.01	0.00999999999999999\\
205.01	0.00999999999999999\\
206.01	0.00999999999999999\\
207.01	0.00999999999999999\\
208.01	0.00999999999999999\\
209.01	0.00999999999999999\\
210.01	0.00999999999999999\\
211.01	0.00999999999999999\\
212.01	0.00999999999999999\\
213.01	0.00999999999999999\\
214.01	0.00999999999999999\\
215.01	0.00999999999999999\\
216.01	0.00999999999999999\\
217.01	0.00999999999999999\\
218.01	0.00999999999999999\\
219.01	0.00999999999999999\\
220.01	0.00999999999999999\\
221.01	0.00999999999999999\\
222.01	0.00999999999999999\\
223.01	0.00999999999999999\\
224.01	0.00999999999999999\\
225.01	0.00999999999999999\\
226.01	0.00999999999999999\\
227.01	0.00999999999999999\\
228.01	0.00999999999999999\\
229.01	0.00999999999999999\\
230.01	0.00999999999999999\\
231.01	0.00999999999999999\\
232.01	0.00999999999999999\\
233.01	0.00999999999999999\\
234.01	0.00999999999999999\\
235.01	0.00999999999999999\\
236.01	0.00999999999999999\\
237.01	0.00999999999999999\\
238.01	0.00999999999999999\\
239.01	0.00999999999999999\\
240.01	0.00999999999999999\\
241.01	0.00999999999999999\\
242.01	0.00999999999999999\\
243.01	0.00999999999999999\\
244.01	0.00999999999999999\\
245.01	0.00999999999999999\\
246.01	0.00999999999999999\\
247.01	0.00999999999999999\\
248.01	0.00999999999999999\\
249.01	0.00999999999999999\\
250.01	0.00999999999999999\\
251.01	0.00999999999999999\\
252.01	0.00999999999999999\\
253.01	0.00999999999999999\\
254.01	0.00999999999999999\\
255.01	0.00999999999999999\\
256.01	0.00999999999999999\\
257.01	0.00999999999999999\\
258.01	0.00999999999999999\\
259.01	0.00999999999999999\\
260.01	0.00999999999999999\\
261.01	0.00999999999999999\\
262.01	0.00999999999999999\\
263.01	0.00999999999999999\\
264.01	0.00999999999999999\\
265.01	0.00999999999999999\\
266.01	0.00999999999999999\\
267.01	0.00999999999999999\\
268.01	0.00999999999999999\\
269.01	0.00999999999999999\\
270.01	0.00999999999999999\\
271.01	0.00999999999999999\\
272.01	0.00999999999999999\\
273.01	0.00999999999999999\\
274.01	0.00999999999999999\\
275.01	0.00999999999999999\\
276.01	0.00999999999999999\\
277.01	0.00999999999999999\\
278.01	0.00999999999999999\\
279.01	0.00999999999999999\\
280.01	0.00999999999999999\\
281.01	0.00999999999999999\\
282.01	0.00999999999999999\\
283.01	0.00999999999999999\\
284.01	0.00999999999999999\\
285.01	0.00999999999999999\\
286.01	0.00999999999999999\\
287.01	0.00999999999999999\\
288.01	0.00999999999999999\\
289.01	0.00999999999999999\\
290.01	0.00999999999999999\\
291.01	0.00999999999999999\\
292.01	0.00999999999999999\\
293.01	0.00999999999999999\\
294.01	0.00999999999999999\\
295.01	0.00999999999999999\\
296.01	0.00999999999999999\\
297.01	0.00999999999999999\\
298.01	0.00999999999999999\\
299.01	0.00999999999999999\\
300.01	0.00999999999999999\\
301.01	0.00999999999999999\\
302.01	0.00999999999999999\\
303.01	0.00999999999999999\\
304.01	0.00999999999999999\\
305.01	0.00999999999999999\\
306.01	0.00999999999999999\\
307.01	0.00999999999999999\\
308.01	0.00999999999999999\\
309.01	0.00999999999999999\\
310.01	0.00999999999999999\\
311.01	0.00999999999999999\\
312.01	0.00999999999999999\\
313.01	0.00999999999999999\\
314.01	0.00999999999999999\\
315.01	0.00999999999999999\\
316.01	0.00999999999999999\\
317.01	0.00999999999999999\\
318.01	0.00999999999999999\\
319.01	0.00999999999999999\\
320.01	0.00999999999999999\\
321.01	0.00999999999999999\\
322.01	0.00999999999999999\\
323.01	0.00999999999999999\\
324.01	0.00999999999999999\\
325.01	0.00999999999999999\\
326.01	0.00999999999999999\\
327.01	0.00999999999999999\\
328.01	0.00999999999999999\\
329.01	0.00999999999999999\\
330.01	0.00999999999999999\\
331.01	0.00999999999999999\\
332.01	0.00999999999999999\\
333.01	0.00999999999999999\\
334.01	0.00999999999999999\\
335.01	0.00999999999999999\\
336.01	0.00999999999999999\\
337.01	0.00999999999999999\\
338.01	0.00999999999999999\\
339.01	0.00999999999999999\\
340.01	0.00999999999999999\\
341.01	0.00999999999999999\\
342.01	0.00999999999999999\\
343.01	0.00999999999999999\\
344.01	0.00999999999999999\\
345.01	0.00999999999999999\\
346.01	0.00999999999999999\\
347.01	0.00999999999999999\\
348.01	0.00999999999999999\\
349.01	0.00999999999999999\\
350.01	0.00999999999999999\\
351.01	0.00999999999999999\\
352.01	0.00999999999999999\\
353.01	0.00999999999999999\\
354.01	0.00999999999999999\\
355.01	0.00999999999999999\\
356.01	0.00999999999999999\\
357.01	0.00999999999999999\\
358.01	0.00999999999999999\\
359.01	0.00999999999999999\\
360.01	0.00999999999999999\\
361.01	0.00999999999999999\\
362.01	0.00999999999999999\\
363.01	0.00999999999999999\\
364.01	0.00999999999999999\\
365.01	0.00999999999999999\\
366.01	0.00999999999999999\\
367.01	0.00999999999999999\\
368.01	0.00999999999999999\\
369.01	0.00999999999999999\\
370.01	0.00999999999999999\\
371.01	0.00999999999999999\\
372.01	0.00999999999999999\\
373.01	0.00999999999999999\\
374.01	0.00999999999999999\\
375.01	0.00999999999999999\\
376.01	0.00999999999999999\\
377.01	0.00999999999999999\\
378.01	0.00999999999999999\\
379.01	0.00999999999999999\\
380.01	0.00999999999999999\\
381.01	0.00999999999999999\\
382.01	0.00999999999999999\\
383.01	0.00999999999999999\\
384.01	0.00999999999999999\\
385.01	0.00999999999999999\\
386.01	0.00999999999999999\\
387.01	0.00999999999999999\\
388.01	0.00999999999999999\\
389.01	0.00999999999999999\\
390.01	0.00999999999999999\\
391.01	0.00999999999999999\\
392.01	0.00999999999999999\\
393.01	0.00999999999999999\\
394.01	0.00999999999999999\\
395.01	0.00999999999999999\\
396.01	0.00999999999999999\\
397.01	0.00999999999999999\\
398.01	0.00999999999999999\\
399.01	0.00999999999999999\\
400.01	0.00999999999999999\\
401.01	0.00999999999999999\\
402.01	0.00999999999999999\\
403.01	0.00999999999999999\\
404.01	0.00999999999999999\\
405.01	0.00999999999999999\\
406.01	0.00999999999999999\\
407.01	0.00999999999999999\\
408.01	0.00999999999999999\\
409.01	0.00999999999999999\\
410.01	0.00999999999999999\\
411.01	0.00999999999999999\\
412.01	0.00999999999999999\\
413.01	0.00999999999999999\\
414.01	0.00999999999999999\\
415.01	0.00999999999999999\\
416.01	0.00999999999999999\\
417.01	0.00999999999999999\\
418.01	0.00999999999999999\\
419.01	0.00999999999999999\\
420.01	0.00999999999999999\\
421.01	0.00999999999999999\\
422.01	0.00999999999999999\\
423.01	0.00999999999999999\\
424.01	0.00999999999999999\\
425.01	0.00999999999999999\\
426.01	0.00999999999999999\\
427.01	0.00999999999999999\\
428.01	0.00999999999999999\\
429.01	0.00999999999999999\\
430.01	0.00999999999999999\\
431.01	0.00999999999999999\\
432.01	0.00999999999999999\\
433.01	0.00999999999999999\\
434.01	0.00999999999999999\\
435.01	0.00999999999999999\\
436.01	0.00999999999999999\\
437.01	0.00999999999999999\\
438.01	0.00999999999999999\\
439.01	0.00999999999999999\\
440.01	0.00999999999999999\\
441.01	0.00999999999999999\\
442.01	0.00999999999999999\\
443.01	0.00999999999999999\\
444.01	0.00999999999999999\\
445.01	0.00999999999999999\\
446.01	0.00999999999999999\\
447.01	0.00999999999999999\\
448.01	0.00999999999999999\\
449.01	0.00999999999999999\\
450.01	0.00999999999999999\\
451.01	0.00999999999999999\\
452.01	0.00999999999999999\\
453.01	0.00999999999999999\\
454.01	0.00999999999999999\\
455.01	0.00999999999999999\\
456.01	0.00999999999999999\\
457.01	0.00999999999999999\\
458.01	0.00999999999999999\\
459.01	0.00999999999999999\\
460.01	0.00999999999999999\\
461.01	0.00999999999999999\\
462.01	0.00999999999999999\\
463.01	0.00999999999999999\\
464.01	0.00999999999999999\\
465.01	0.00999999999999999\\
466.01	0.00999999999999999\\
467.01	0.00999999999999999\\
468.01	0.00999999999999999\\
469.01	0.00999999999999999\\
470.01	0.00999999999999999\\
471.01	0.00999999999999999\\
472.01	0.00999999999999999\\
473.01	0.00999999999999999\\
474.01	0.00999999999999999\\
475.01	0.00999999999999999\\
476.01	0.00999999999999999\\
477.01	0.00999999999999999\\
478.01	0.00999999999999999\\
479.01	0.00999999999999999\\
480.01	0.00999999999999999\\
481.01	0.00999999999999999\\
482.01	0.00999999999999999\\
483.01	0.00999999999999999\\
484.01	0.00999999999999999\\
485.01	0.00999999999999999\\
486.01	0.00999999999999999\\
487.01	0.00999999999999999\\
488.01	0.00999999999999999\\
489.01	0.00999999999999999\\
490.01	0.00999999999999999\\
491.01	0.00999999999999999\\
492.01	0.00999999999999999\\
493.01	0.00999999999999999\\
494.01	0.00999999999999999\\
495.01	0.00999999999999999\\
496.01	0.00999999999999999\\
497.01	0.00999999999999999\\
498.01	0.00999999999999999\\
499.01	0.00999999999999999\\
500.01	0.00999999999999999\\
501.01	0.00999999999999999\\
502.01	0.00999999999999999\\
503.01	0.00999999999999999\\
504.01	0.00999999999999999\\
505.01	0.00999999999999999\\
506.01	0.00999999999999999\\
507.01	0.00999999999999999\\
508.01	0.00999999999999999\\
509.01	0.00999999999999999\\
510.01	0.00999999999999999\\
511.01	0.00999999999999999\\
512.01	0.00999999999999999\\
513.01	0.00999999999999999\\
514.01	0.00999999999999999\\
515.01	0.00999999999999999\\
516.01	0.00999999999999999\\
517.01	0.00999999999999999\\
518.01	0.00999999999999999\\
519.01	0.00999999999999999\\
520.01	0.00999999999999999\\
521.01	0.00999999999999999\\
522.01	0.00999999999999999\\
523.01	0.00999999999999999\\
524.01	0.00999999999999999\\
525.01	0.00999999999999999\\
526.01	0.00999999999999999\\
527.01	0.00999999999999999\\
528.01	0.00999999999999999\\
529.01	0.00999999999999999\\
530.01	0.00999999999999999\\
531.01	0.00999999999999999\\
532.01	0.00999999999999999\\
533.01	0.00999999999999999\\
534.01	0.00999999999999999\\
535.01	0.00999999999999999\\
536.01	0.00999999999999999\\
537.01	0.00999999999999999\\
538.01	0.00999999999999999\\
539.01	0.00999999999999999\\
540.01	0.00999999999999999\\
541.01	0.00999999999999999\\
542.01	0.00999999999999999\\
543.01	0.00999999999999999\\
544.01	0.00999999999999999\\
545.01	0.00999999999999999\\
546.01	0.00999999999999999\\
547.01	0.00999999999999999\\
548.01	0.00999999999999999\\
549.01	0.00999999999999999\\
550.01	0.00999999999999999\\
551.01	0.00999999999999999\\
552.01	0.00999999999999999\\
553.01	0.00999999999999999\\
554.01	0.00999999999999999\\
555.01	0.00999999999999999\\
556.01	0.00999999999999999\\
557.01	0.00999999999999999\\
558.01	0.00999999999999999\\
559.01	0.00999999999999999\\
560.01	0.00999999999999999\\
561.01	0.00999999999999999\\
562.01	0.00999999999999999\\
563.01	0.00999999999999999\\
564.01	0.00999999999999999\\
565.01	0.00999999999999999\\
566.01	0.00999999999999999\\
567.01	0.00999999999999999\\
568.01	0.00999999999999999\\
569.01	0.00999999999999999\\
570.01	0.00999999999999999\\
571.01	0.00999999999999999\\
572.01	0.00999999999999999\\
573.01	0.00999999999999999\\
574.01	0.00999999999999999\\
575.01	0.00999999999999999\\
576.01	0.00999999999999999\\
577.01	0.00999999999999999\\
578.01	0.00999999999999999\\
579.01	0.00999999999999999\\
580.01	0.00999999999999999\\
581.01	0.00999999999999999\\
582.01	0.00999999999999999\\
583.01	0.00999999999999999\\
584.01	0.00999999999999999\\
585.01	0.00999999999999999\\
586.01	0.00999999999999999\\
587.01	0.00999999999999999\\
588.01	0.00999999999999999\\
589.01	0.00999999999999999\\
590.01	0.00999999999999999\\
591.01	0.00999999999999999\\
592.01	0.00999999999999999\\
593.01	0.00999999999999999\\
594.01	0.00999999999999999\\
595.01	0.00999999999999999\\
596.01	0.00999999999999999\\
597.01	0.00999999999999999\\
598.01	0.00999999999999999\\
599.01	0.00624186909405523\\
599.02	0.0062041444124227\\
599.03	0.00616605275622837\\
599.04	0.00612759052101666\\
599.05	0.00608875406692124\\
599.06	0.00604953971831725\\
599.07	0.00600994376346995\\
599.08	0.00596996245417992\\
599.09	0.00592959200542494\\
599.1	0.00588882859499817\\
599.11	0.00584766836314282\\
599.12	0.00580610741218332\\
599.13	0.00576414180615277\\
599.14	0.00572176757041677\\
599.15	0.00567898069129352\\
599.16	0.00563577711567027\\
599.17	0.00559215275061587\\
599.18	0.00554810346298963\\
599.19	0.00550362507904635\\
599.2	0.00545871338403733\\
599.21	0.00541336412180758\\
599.22	0.00536757299438907\\
599.23	0.00532133566158991\\
599.24	0.00527464774057957\\
599.25	0.00522750480546996\\
599.26	0.0051799023868925\\
599.27	0.00513183597157084\\
599.28	0.00508330100188967\\
599.29	0.00503429287545899\\
599.3	0.0049848069446743\\
599.31	0.0049348385162724\\
599.32	0.00488438285088287\\
599.33	0.00483343516257501\\
599.34	0.00478199061840055\\
599.35	0.00473004433793169\\
599.36	0.00467759139279461\\
599.37	0.00462462680619856\\
599.38	0.00457114555246008\\
599.39	0.0045171425604974\\
599.4	0.00446261272376361\\
599.41	0.00440755088572623\\
599.42	0.00435195183937785\\
599.43	0.00429581032674213\\
599.44	0.00423912103837488\\
599.45	0.00418187861286023\\
599.46	0.00412407763630195\\
599.47	0.00406571264180972\\
599.48	0.00400677810898045\\
599.49	0.00394726846337443\\
599.5	0.00388717807598639\\
599.51	0.00382650126271144\\
599.52	0.00376523228380568\\
599.53	0.00370336534334162\\
599.54	0.00364089458865821\\
599.55	0.00357781410980551\\
599.56	0.00351411793898395\\
599.57	0.00344980004997801\\
599.58	0.0033848543575845\\
599.59	0.00331927471703512\\
599.6	0.00325305492341342\\
599.61	0.00318618871106605\\
599.62	0.00311866975300824\\
599.63	0.00305049166032345\\
599.64	0.00298164798155718\\
599.65	0.00291213220210484\\
599.66	0.00284193774359354\\
599.67	0.00277105796325799\\
599.68	0.00269948615331015\\
599.69	0.00262721554030275\\
599.7	0.00255423928448666\\
599.71	0.00248055047916181\\
599.72	0.00240614215002192\\
599.73	0.00233100725449273\\
599.74	0.00225513868106373\\
599.75	0.00217852924861344\\
599.76	0.002101171705728\\
599.77	0.0020230587300131\\
599.78	0.00194418292739928\\
599.79	0.00186453683144023\\
599.8	0.00178411290260446\\
599.81	0.00170290352755979\\
599.82	0.00162090101845109\\
599.83	0.00153809761217072\\
599.84	0.001454485469622\\
599.85	0.00137005667497534\\
599.86	0.00128480323491721\\
599.87	0.00119871707789167\\
599.88	0.00111179005333448\\
599.89	0.00102401393089983\\
599.9	0.000935380399679274\\
599.91	0.000845881067413286\\
599.92	0.00075550745969488\\
599.93	0.000664251019165564\\
599.94	0.000572103104703356\\
599.95	0.000479054990602912\\
599.96	0.000385097865747556\\
599.97	0.000290222832773275\\
599.98	0.000194420907224489\\
599.99	9.76830167015649e-05\\
600	0\\
};
\addplot [color=red!80!mycolor19,solid,forget plot]
  table[row sep=crcr]{%
0.01	0.00999999999999999\\
1.01	0.00999999999999999\\
2.01	0.00999999999999999\\
3.01	0.00999999999999999\\
4.01	0.00999999999999999\\
5.01	0.00999999999999999\\
6.01	0.00999999999999999\\
7.01	0.00999999999999999\\
8.01	0.00999999999999999\\
9.01	0.00999999999999999\\
10.01	0.00999999999999999\\
11.01	0.00999999999999999\\
12.01	0.00999999999999999\\
13.01	0.00999999999999999\\
14.01	0.00999999999999999\\
15.01	0.00999999999999999\\
16.01	0.00999999999999999\\
17.01	0.00999999999999999\\
18.01	0.00999999999999999\\
19.01	0.00999999999999999\\
20.01	0.00999999999999999\\
21.01	0.00999999999999999\\
22.01	0.00999999999999999\\
23.01	0.00999999999999999\\
24.01	0.00999999999999999\\
25.01	0.00999999999999999\\
26.01	0.00999999999999999\\
27.01	0.00999999999999999\\
28.01	0.00999999999999999\\
29.01	0.00999999999999999\\
30.01	0.00999999999999999\\
31.01	0.00999999999999999\\
32.01	0.00999999999999999\\
33.01	0.00999999999999999\\
34.01	0.00999999999999999\\
35.01	0.00999999999999999\\
36.01	0.00999999999999999\\
37.01	0.00999999999999999\\
38.01	0.00999999999999999\\
39.01	0.00999999999999999\\
40.01	0.00999999999999999\\
41.01	0.00999999999999999\\
42.01	0.00999999999999999\\
43.01	0.00999999999999999\\
44.01	0.00999999999999999\\
45.01	0.00999999999999999\\
46.01	0.00999999999999999\\
47.01	0.00999999999999999\\
48.01	0.00999999999999999\\
49.01	0.00999999999999999\\
50.01	0.00999999999999999\\
51.01	0.00999999999999999\\
52.01	0.00999999999999999\\
53.01	0.00999999999999999\\
54.01	0.00999999999999999\\
55.01	0.00999999999999999\\
56.01	0.00999999999999999\\
57.01	0.00999999999999999\\
58.01	0.00999999999999999\\
59.01	0.00999999999999999\\
60.01	0.00999999999999999\\
61.01	0.00999999999999999\\
62.01	0.00999999999999999\\
63.01	0.00999999999999999\\
64.01	0.00999999999999999\\
65.01	0.00999999999999999\\
66.01	0.00999999999999999\\
67.01	0.00999999999999999\\
68.01	0.00999999999999999\\
69.01	0.00999999999999999\\
70.01	0.00999999999999999\\
71.01	0.00999999999999999\\
72.01	0.00999999999999999\\
73.01	0.00999999999999999\\
74.01	0.00999999999999999\\
75.01	0.00999999999999999\\
76.01	0.00999999999999999\\
77.01	0.00999999999999999\\
78.01	0.00999999999999999\\
79.01	0.00999999999999999\\
80.01	0.00999999999999999\\
81.01	0.00999999999999999\\
82.01	0.00999999999999999\\
83.01	0.00999999999999999\\
84.01	0.00999999999999999\\
85.01	0.00999999999999999\\
86.01	0.00999999999999999\\
87.01	0.00999999999999999\\
88.01	0.00999999999999999\\
89.01	0.00999999999999999\\
90.01	0.00999999999999999\\
91.01	0.00999999999999999\\
92.01	0.00999999999999999\\
93.01	0.00999999999999999\\
94.01	0.00999999999999999\\
95.01	0.00999999999999999\\
96.01	0.00999999999999999\\
97.01	0.00999999999999999\\
98.01	0.00999999999999999\\
99.01	0.00999999999999999\\
100.01	0.00999999999999999\\
101.01	0.00999999999999999\\
102.01	0.00999999999999999\\
103.01	0.00999999999999999\\
104.01	0.00999999999999999\\
105.01	0.00999999999999999\\
106.01	0.00999999999999999\\
107.01	0.00999999999999999\\
108.01	0.00999999999999999\\
109.01	0.00999999999999999\\
110.01	0.00999999999999999\\
111.01	0.00999999999999999\\
112.01	0.00999999999999999\\
113.01	0.00999999999999999\\
114.01	0.00999999999999999\\
115.01	0.00999999999999999\\
116.01	0.00999999999999999\\
117.01	0.00999999999999999\\
118.01	0.00999999999999999\\
119.01	0.00999999999999999\\
120.01	0.00999999999999999\\
121.01	0.00999999999999999\\
122.01	0.00999999999999999\\
123.01	0.00999999999999999\\
124.01	0.00999999999999999\\
125.01	0.00999999999999999\\
126.01	0.00999999999999999\\
127.01	0.00999999999999999\\
128.01	0.00999999999999999\\
129.01	0.00999999999999999\\
130.01	0.00999999999999999\\
131.01	0.00999999999999999\\
132.01	0.00999999999999999\\
133.01	0.00999999999999999\\
134.01	0.00999999999999999\\
135.01	0.00999999999999999\\
136.01	0.00999999999999999\\
137.01	0.00999999999999999\\
138.01	0.00999999999999999\\
139.01	0.00999999999999999\\
140.01	0.00999999999999999\\
141.01	0.00999999999999999\\
142.01	0.00999999999999999\\
143.01	0.00999999999999999\\
144.01	0.00999999999999999\\
145.01	0.00999999999999999\\
146.01	0.00999999999999999\\
147.01	0.00999999999999999\\
148.01	0.00999999999999999\\
149.01	0.00999999999999999\\
150.01	0.00999999999999999\\
151.01	0.00999999999999999\\
152.01	0.00999999999999999\\
153.01	0.00999999999999999\\
154.01	0.00999999999999999\\
155.01	0.00999999999999999\\
156.01	0.00999999999999999\\
157.01	0.00999999999999999\\
158.01	0.00999999999999999\\
159.01	0.00999999999999999\\
160.01	0.00999999999999999\\
161.01	0.00999999999999999\\
162.01	0.00999999999999999\\
163.01	0.00999999999999999\\
164.01	0.00999999999999999\\
165.01	0.00999999999999999\\
166.01	0.00999999999999999\\
167.01	0.00999999999999999\\
168.01	0.00999999999999999\\
169.01	0.00999999999999999\\
170.01	0.00999999999999999\\
171.01	0.00999999999999999\\
172.01	0.00999999999999999\\
173.01	0.00999999999999999\\
174.01	0.00999999999999999\\
175.01	0.00999999999999999\\
176.01	0.00999999999999999\\
177.01	0.00999999999999999\\
178.01	0.00999999999999999\\
179.01	0.00999999999999999\\
180.01	0.00999999999999999\\
181.01	0.00999999999999999\\
182.01	0.00999999999999999\\
183.01	0.00999999999999999\\
184.01	0.00999999999999999\\
185.01	0.00999999999999999\\
186.01	0.00999999999999999\\
187.01	0.00999999999999999\\
188.01	0.00999999999999999\\
189.01	0.00999999999999999\\
190.01	0.00999999999999999\\
191.01	0.00999999999999999\\
192.01	0.00999999999999999\\
193.01	0.00999999999999999\\
194.01	0.00999999999999999\\
195.01	0.00999999999999999\\
196.01	0.00999999999999999\\
197.01	0.00999999999999999\\
198.01	0.00999999999999999\\
199.01	0.00999999999999999\\
200.01	0.00999999999999999\\
201.01	0.00999999999999999\\
202.01	0.00999999999999999\\
203.01	0.00999999999999999\\
204.01	0.00999999999999999\\
205.01	0.00999999999999999\\
206.01	0.00999999999999999\\
207.01	0.00999999999999999\\
208.01	0.00999999999999999\\
209.01	0.00999999999999999\\
210.01	0.00999999999999999\\
211.01	0.00999999999999999\\
212.01	0.00999999999999999\\
213.01	0.00999999999999999\\
214.01	0.00999999999999999\\
215.01	0.00999999999999999\\
216.01	0.00999999999999999\\
217.01	0.00999999999999999\\
218.01	0.00999999999999999\\
219.01	0.00999999999999999\\
220.01	0.00999999999999999\\
221.01	0.00999999999999999\\
222.01	0.00999999999999999\\
223.01	0.00999999999999999\\
224.01	0.00999999999999999\\
225.01	0.00999999999999999\\
226.01	0.00999999999999999\\
227.01	0.00999999999999999\\
228.01	0.00999999999999999\\
229.01	0.00999999999999999\\
230.01	0.00999999999999999\\
231.01	0.00999999999999999\\
232.01	0.00999999999999999\\
233.01	0.00999999999999999\\
234.01	0.00999999999999999\\
235.01	0.00999999999999999\\
236.01	0.00999999999999999\\
237.01	0.00999999999999999\\
238.01	0.00999999999999999\\
239.01	0.00999999999999999\\
240.01	0.00999999999999999\\
241.01	0.00999999999999999\\
242.01	0.00999999999999999\\
243.01	0.00999999999999999\\
244.01	0.00999999999999999\\
245.01	0.00999999999999999\\
246.01	0.00999999999999999\\
247.01	0.00999999999999999\\
248.01	0.00999999999999999\\
249.01	0.00999999999999999\\
250.01	0.00999999999999999\\
251.01	0.00999999999999999\\
252.01	0.00999999999999999\\
253.01	0.00999999999999999\\
254.01	0.00999999999999999\\
255.01	0.00999999999999999\\
256.01	0.00999999999999999\\
257.01	0.00999999999999999\\
258.01	0.00999999999999999\\
259.01	0.00999999999999999\\
260.01	0.00999999999999999\\
261.01	0.00999999999999999\\
262.01	0.00999999999999999\\
263.01	0.00999999999999999\\
264.01	0.00999999999999999\\
265.01	0.00999999999999999\\
266.01	0.00999999999999999\\
267.01	0.00999999999999999\\
268.01	0.00999999999999999\\
269.01	0.00999999999999999\\
270.01	0.00999999999999999\\
271.01	0.00999999999999999\\
272.01	0.00999999999999999\\
273.01	0.00999999999999999\\
274.01	0.00999999999999999\\
275.01	0.00999999999999999\\
276.01	0.00999999999999999\\
277.01	0.00999999999999999\\
278.01	0.00999999999999999\\
279.01	0.00999999999999999\\
280.01	0.00999999999999999\\
281.01	0.00999999999999999\\
282.01	0.00999999999999999\\
283.01	0.00999999999999999\\
284.01	0.00999999999999999\\
285.01	0.00999999999999999\\
286.01	0.00999999999999999\\
287.01	0.00999999999999999\\
288.01	0.00999999999999999\\
289.01	0.00999999999999999\\
290.01	0.00999999999999999\\
291.01	0.00999999999999999\\
292.01	0.00999999999999999\\
293.01	0.00999999999999999\\
294.01	0.00999999999999999\\
295.01	0.00999999999999999\\
296.01	0.00999999999999999\\
297.01	0.00999999999999999\\
298.01	0.00999999999999999\\
299.01	0.00999999999999999\\
300.01	0.00999999999999999\\
301.01	0.00999999999999999\\
302.01	0.00999999999999999\\
303.01	0.00999999999999999\\
304.01	0.00999999999999999\\
305.01	0.00999999999999999\\
306.01	0.00999999999999999\\
307.01	0.00999999999999999\\
308.01	0.00999999999999999\\
309.01	0.00999999999999999\\
310.01	0.00999999999999999\\
311.01	0.00999999999999999\\
312.01	0.00999999999999999\\
313.01	0.00999999999999999\\
314.01	0.00999999999999999\\
315.01	0.00999999999999999\\
316.01	0.00999999999999999\\
317.01	0.00999999999999999\\
318.01	0.00999999999999999\\
319.01	0.00999999999999999\\
320.01	0.00999999999999999\\
321.01	0.00999999999999999\\
322.01	0.00999999999999999\\
323.01	0.00999999999999999\\
324.01	0.00999999999999999\\
325.01	0.00999999999999999\\
326.01	0.00999999999999999\\
327.01	0.00999999999999999\\
328.01	0.00999999999999999\\
329.01	0.00999999999999999\\
330.01	0.00999999999999999\\
331.01	0.00999999999999999\\
332.01	0.00999999999999999\\
333.01	0.00999999999999999\\
334.01	0.00999999999999999\\
335.01	0.00999999999999999\\
336.01	0.00999999999999999\\
337.01	0.00999999999999999\\
338.01	0.00999999999999999\\
339.01	0.00999999999999999\\
340.01	0.00999999999999999\\
341.01	0.00999999999999999\\
342.01	0.00999999999999999\\
343.01	0.00999999999999999\\
344.01	0.00999999999999999\\
345.01	0.00999999999999999\\
346.01	0.00999999999999999\\
347.01	0.00999999999999999\\
348.01	0.00999999999999999\\
349.01	0.00999999999999999\\
350.01	0.00999999999999999\\
351.01	0.00999999999999999\\
352.01	0.00999999999999999\\
353.01	0.00999999999999999\\
354.01	0.00999999999999999\\
355.01	0.00999999999999999\\
356.01	0.00999999999999999\\
357.01	0.00999999999999999\\
358.01	0.00999999999999999\\
359.01	0.00999999999999999\\
360.01	0.00999999999999999\\
361.01	0.00999999999999999\\
362.01	0.00999999999999999\\
363.01	0.00999999999999999\\
364.01	0.00999999999999999\\
365.01	0.00999999999999999\\
366.01	0.00999999999999999\\
367.01	0.00999999999999999\\
368.01	0.00999999999999999\\
369.01	0.00999999999999999\\
370.01	0.00999999999999999\\
371.01	0.00999999999999999\\
372.01	0.00999999999999999\\
373.01	0.00999999999999999\\
374.01	0.00999999999999999\\
375.01	0.00999999999999999\\
376.01	0.00999999999999999\\
377.01	0.00999999999999999\\
378.01	0.00999999999999999\\
379.01	0.00999999999999999\\
380.01	0.00999999999999999\\
381.01	0.00999999999999999\\
382.01	0.00999999999999999\\
383.01	0.00999999999999999\\
384.01	0.00999999999999999\\
385.01	0.00999999999999999\\
386.01	0.00999999999999999\\
387.01	0.00999999999999999\\
388.01	0.00999999999999999\\
389.01	0.00999999999999999\\
390.01	0.00999999999999999\\
391.01	0.00999999999999999\\
392.01	0.00999999999999999\\
393.01	0.00999999999999999\\
394.01	0.00999999999999999\\
395.01	0.00999999999999999\\
396.01	0.00999999999999999\\
397.01	0.00999999999999999\\
398.01	0.00999999999999999\\
399.01	0.00999999999999999\\
400.01	0.00999999999999999\\
401.01	0.00999999999999999\\
402.01	0.00999999999999999\\
403.01	0.00999999999999999\\
404.01	0.00999999999999999\\
405.01	0.00999999999999999\\
406.01	0.00999999999999999\\
407.01	0.00999999999999999\\
408.01	0.00999999999999999\\
409.01	0.00999999999999999\\
410.01	0.00999999999999999\\
411.01	0.00999999999999999\\
412.01	0.00999999999999999\\
413.01	0.00999999999999999\\
414.01	0.00999999999999999\\
415.01	0.00999999999999999\\
416.01	0.00999999999999999\\
417.01	0.00999999999999999\\
418.01	0.00999999999999999\\
419.01	0.00999999999999999\\
420.01	0.00999999999999999\\
421.01	0.00999999999999999\\
422.01	0.00999999999999999\\
423.01	0.00999999999999999\\
424.01	0.00999999999999999\\
425.01	0.00999999999999999\\
426.01	0.00999999999999999\\
427.01	0.00999999999999999\\
428.01	0.00999999999999999\\
429.01	0.00999999999999999\\
430.01	0.00999999999999999\\
431.01	0.00999999999999999\\
432.01	0.00999999999999999\\
433.01	0.00999999999999999\\
434.01	0.00999999999999999\\
435.01	0.00999999999999999\\
436.01	0.00999999999999999\\
437.01	0.00999999999999999\\
438.01	0.00999999999999999\\
439.01	0.00999999999999999\\
440.01	0.00999999999999999\\
441.01	0.00999999999999999\\
442.01	0.00999999999999999\\
443.01	0.00999999999999999\\
444.01	0.00999999999999999\\
445.01	0.00999999999999999\\
446.01	0.00999999999999999\\
447.01	0.00999999999999999\\
448.01	0.00999999999999999\\
449.01	0.00999999999999999\\
450.01	0.00999999999999999\\
451.01	0.00999999999999999\\
452.01	0.00999999999999999\\
453.01	0.00999999999999999\\
454.01	0.00999999999999999\\
455.01	0.00999999999999999\\
456.01	0.00999999999999999\\
457.01	0.00999999999999999\\
458.01	0.00999999999999999\\
459.01	0.00999999999999999\\
460.01	0.00999999999999999\\
461.01	0.00999999999999999\\
462.01	0.00999999999999999\\
463.01	0.00999999999999999\\
464.01	0.00999999999999999\\
465.01	0.00999999999999999\\
466.01	0.00999999999999999\\
467.01	0.00999999999999999\\
468.01	0.00999999999999999\\
469.01	0.00999999999999999\\
470.01	0.00999999999999999\\
471.01	0.00999999999999999\\
472.01	0.00999999999999999\\
473.01	0.00999999999999999\\
474.01	0.00999999999999999\\
475.01	0.00999999999999999\\
476.01	0.00999999999999999\\
477.01	0.00999999999999999\\
478.01	0.00999999999999999\\
479.01	0.00999999999999999\\
480.01	0.00999999999999999\\
481.01	0.00999999999999999\\
482.01	0.00999999999999999\\
483.01	0.00999999999999999\\
484.01	0.00999999999999999\\
485.01	0.00999999999999999\\
486.01	0.00999999999999999\\
487.01	0.00999999999999999\\
488.01	0.00999999999999999\\
489.01	0.00999999999999999\\
490.01	0.00999999999999999\\
491.01	0.00999999999999999\\
492.01	0.00999999999999999\\
493.01	0.00999999999999999\\
494.01	0.00999999999999999\\
495.01	0.00999999999999999\\
496.01	0.00999999999999999\\
497.01	0.00999999999999999\\
498.01	0.00999999999999999\\
499.01	0.00999999999999999\\
500.01	0.00999999999999999\\
501.01	0.00999999999999999\\
502.01	0.00999999999999999\\
503.01	0.00999999999999999\\
504.01	0.00999999999999999\\
505.01	0.00999999999999999\\
506.01	0.00999999999999999\\
507.01	0.00999999999999999\\
508.01	0.00999999999999999\\
509.01	0.00999999999999999\\
510.01	0.00999999999999999\\
511.01	0.00999999999999999\\
512.01	0.00999999999999999\\
513.01	0.00999999999999999\\
514.01	0.00999999999999999\\
515.01	0.00999999999999999\\
516.01	0.00999999999999999\\
517.01	0.00999999999999999\\
518.01	0.00999999999999999\\
519.01	0.00999999999999999\\
520.01	0.00999999999999999\\
521.01	0.00999999999999999\\
522.01	0.00999999999999999\\
523.01	0.00999999999999999\\
524.01	0.00999999999999999\\
525.01	0.00999999999999999\\
526.01	0.00999999999999999\\
527.01	0.00999999999999999\\
528.01	0.00999999999999999\\
529.01	0.00999999999999999\\
530.01	0.00999999999999999\\
531.01	0.00999999999999999\\
532.01	0.00999999999999999\\
533.01	0.00999999999999999\\
534.01	0.00999999999999999\\
535.01	0.00999999999999999\\
536.01	0.00999999999999999\\
537.01	0.00999999999999999\\
538.01	0.00999999999999999\\
539.01	0.00999999999999999\\
540.01	0.00999999999999999\\
541.01	0.00999999999999999\\
542.01	0.00999999999999999\\
543.01	0.00999999999999999\\
544.01	0.00999999999999999\\
545.01	0.00999999999999999\\
546.01	0.00999999999999999\\
547.01	0.00999999999999999\\
548.01	0.00999999999999999\\
549.01	0.00999999999999999\\
550.01	0.00999999999999999\\
551.01	0.00999999999999999\\
552.01	0.00999999999999999\\
553.01	0.00999999999999999\\
554.01	0.00999999999999999\\
555.01	0.00999999999999999\\
556.01	0.00999999999999999\\
557.01	0.00999999999999999\\
558.01	0.00999999999999999\\
559.01	0.00999999999999999\\
560.01	0.00999999999999999\\
561.01	0.00999999999999999\\
562.01	0.00999999999999999\\
563.01	0.00999999999999999\\
564.01	0.00999999999999999\\
565.01	0.00999999999999999\\
566.01	0.00999999999999999\\
567.01	0.00999999999999999\\
568.01	0.00999999999999999\\
569.01	0.00999999999999999\\
570.01	0.00999999999999999\\
571.01	0.00999999999999999\\
572.01	0.00999999999999999\\
573.01	0.00999999999999999\\
574.01	0.00999999999999999\\
575.01	0.00999999999999999\\
576.01	0.00999999999999999\\
577.01	0.00999999999999999\\
578.01	0.00999999999999999\\
579.01	0.00999999999999999\\
580.01	0.00999999999999999\\
581.01	0.00999999999999999\\
582.01	0.00999999999999999\\
583.01	0.00999999999999999\\
584.01	0.00999999999999999\\
585.01	0.00999999999999999\\
586.01	0.00999999999999999\\
587.01	0.00999999999999999\\
588.01	0.00999999999999999\\
589.01	0.00999999999999999\\
590.01	0.00999999999999999\\
591.01	0.00999999999999999\\
592.01	0.00999999999999999\\
593.01	0.00999999999999999\\
594.01	0.00999999999999999\\
595.01	0.00999999999999999\\
596.01	0.00999999999999999\\
597.01	0.00999999999999999\\
598.01	0.00999999999999999\\
599.01	0.00624186909405409\\
599.02	0.00620414441242163\\
599.03	0.00616605275622736\\
599.04	0.0061275905210157\\
599.05	0.00608875406692034\\
599.06	0.00604953971831642\\
599.07	0.00600994376346916\\
599.08	0.0059699624541792\\
599.09	0.00592959200542426\\
599.1	0.00588882859499752\\
599.11	0.00584766836314221\\
599.12	0.00580610741218275\\
599.13	0.00576414180615225\\
599.14	0.00572176757041629\\
599.15	0.00567898069129309\\
599.16	0.00563577711566985\\
599.17	0.00559215275061547\\
599.18	0.00554810346298928\\
599.19	0.00550362507904604\\
599.2	0.00545871338403705\\
599.21	0.00541336412180734\\
599.22	0.00536757299438886\\
599.23	0.00532133566158972\\
599.24	0.00527464774057939\\
599.25	0.0052275048054698\\
599.26	0.00517990238689235\\
599.27	0.00513183597157072\\
599.28	0.00508330100188957\\
599.29	0.00503429287545889\\
599.3	0.00498480694467421\\
599.31	0.00493483851627234\\
599.32	0.00488438285088279\\
599.33	0.00483343516257495\\
599.34	0.00478199061840051\\
599.35	0.00473004433793165\\
599.36	0.00467759139279459\\
599.37	0.00462462680619853\\
599.38	0.00457114555246006\\
599.39	0.00451714256049738\\
599.4	0.00446261272376361\\
599.41	0.00440755088572623\\
599.42	0.00435195183937785\\
599.43	0.00429581032674214\\
599.44	0.00423912103837489\\
599.45	0.00418187861286024\\
599.46	0.00412407763630196\\
599.47	0.00406571264180974\\
599.48	0.00400677810898047\\
599.49	0.00394726846337445\\
599.5	0.00388717807598641\\
599.51	0.00382650126271146\\
599.52	0.0037652322838057\\
599.53	0.00370336534334164\\
599.54	0.00364089458865823\\
599.55	0.00357781410980553\\
599.56	0.00351411793898395\\
599.57	0.00344980004997801\\
599.58	0.00338485435758451\\
599.59	0.00331927471703513\\
599.6	0.00325305492341343\\
599.61	0.00318618871106605\\
599.62	0.00311866975300824\\
599.63	0.00305049166032345\\
599.64	0.00298164798155718\\
599.65	0.00291213220210483\\
599.66	0.00284193774359354\\
599.67	0.002771057963258\\
599.68	0.00269948615331015\\
599.69	0.00262721554030276\\
599.7	0.00255423928448666\\
599.71	0.00248055047916181\\
599.72	0.00240614215002192\\
599.73	0.00233100725449272\\
599.74	0.00225513868106373\\
599.75	0.00217852924861344\\
599.76	0.00210117170572799\\
599.77	0.0020230587300131\\
599.78	0.00194418292739928\\
599.79	0.00186453683144023\\
599.8	0.00178411290260445\\
599.81	0.00170290352755979\\
599.82	0.00162090101845109\\
599.83	0.00153809761217072\\
599.84	0.001454485469622\\
599.85	0.00137005667497534\\
599.86	0.00128480323491721\\
599.87	0.00119871707789166\\
599.88	0.00111179005333448\\
599.89	0.00102401393089982\\
599.9	0.000935380399679275\\
599.91	0.000845881067413283\\
599.92	0.00075550745969488\\
599.93	0.000664251019165561\\
599.94	0.000572103104703356\\
599.95	0.00047905499060291\\
599.96	0.000385097865747556\\
599.97	0.000290222832773275\\
599.98	0.000194420907224489\\
599.99	9.76830167015649e-05\\
600	0\\
};
\addplot [color=red,solid,forget plot]
  table[row sep=crcr]{%
0.01	0.00999999999999999\\
1.01	0.00999999999999999\\
2.01	0.00999999999999999\\
3.01	0.00999999999999999\\
4.01	0.00999999999999999\\
5.01	0.00999999999999999\\
6.01	0.00999999999999999\\
7.01	0.00999999999999999\\
8.01	0.00999999999999999\\
9.01	0.00999999999999999\\
10.01	0.00999999999999999\\
11.01	0.00999999999999999\\
12.01	0.00999999999999999\\
13.01	0.00999999999999999\\
14.01	0.00999999999999999\\
15.01	0.00999999999999999\\
16.01	0.00999999999999999\\
17.01	0.00999999999999999\\
18.01	0.00999999999999999\\
19.01	0.00999999999999999\\
20.01	0.00999999999999999\\
21.01	0.00999999999999999\\
22.01	0.00999999999999999\\
23.01	0.00999999999999999\\
24.01	0.00999999999999999\\
25.01	0.00999999999999999\\
26.01	0.00999999999999999\\
27.01	0.00999999999999999\\
28.01	0.00999999999999999\\
29.01	0.00999999999999999\\
30.01	0.00999999999999999\\
31.01	0.00999999999999999\\
32.01	0.00999999999999999\\
33.01	0.00999999999999999\\
34.01	0.00999999999999999\\
35.01	0.00999999999999999\\
36.01	0.00999999999999999\\
37.01	0.00999999999999999\\
38.01	0.00999999999999999\\
39.01	0.00999999999999999\\
40.01	0.00999999999999999\\
41.01	0.00999999999999999\\
42.01	0.00999999999999999\\
43.01	0.00999999999999999\\
44.01	0.00999999999999999\\
45.01	0.00999999999999999\\
46.01	0.00999999999999999\\
47.01	0.00999999999999999\\
48.01	0.00999999999999999\\
49.01	0.00999999999999999\\
50.01	0.00999999999999999\\
51.01	0.00999999999999999\\
52.01	0.00999999999999999\\
53.01	0.00999999999999999\\
54.01	0.00999999999999999\\
55.01	0.00999999999999999\\
56.01	0.00999999999999999\\
57.01	0.00999999999999999\\
58.01	0.00999999999999999\\
59.01	0.00999999999999999\\
60.01	0.00999999999999999\\
61.01	0.00999999999999999\\
62.01	0.00999999999999999\\
63.01	0.00999999999999999\\
64.01	0.00999999999999999\\
65.01	0.00999999999999999\\
66.01	0.00999999999999999\\
67.01	0.00999999999999999\\
68.01	0.00999999999999999\\
69.01	0.00999999999999999\\
70.01	0.00999999999999999\\
71.01	0.00999999999999999\\
72.01	0.00999999999999999\\
73.01	0.00999999999999999\\
74.01	0.00999999999999999\\
75.01	0.00999999999999999\\
76.01	0.00999999999999999\\
77.01	0.00999999999999999\\
78.01	0.00999999999999999\\
79.01	0.00999999999999999\\
80.01	0.00999999999999999\\
81.01	0.00999999999999999\\
82.01	0.00999999999999999\\
83.01	0.00999999999999999\\
84.01	0.00999999999999999\\
85.01	0.00999999999999999\\
86.01	0.00999999999999999\\
87.01	0.00999999999999999\\
88.01	0.00999999999999999\\
89.01	0.00999999999999999\\
90.01	0.00999999999999999\\
91.01	0.00999999999999999\\
92.01	0.00999999999999999\\
93.01	0.00999999999999999\\
94.01	0.00999999999999999\\
95.01	0.00999999999999999\\
96.01	0.00999999999999999\\
97.01	0.00999999999999999\\
98.01	0.00999999999999999\\
99.01	0.00999999999999999\\
100.01	0.00999999999999999\\
101.01	0.00999999999999999\\
102.01	0.00999999999999999\\
103.01	0.00999999999999999\\
104.01	0.00999999999999999\\
105.01	0.00999999999999999\\
106.01	0.00999999999999999\\
107.01	0.00999999999999999\\
108.01	0.00999999999999999\\
109.01	0.00999999999999999\\
110.01	0.00999999999999999\\
111.01	0.00999999999999999\\
112.01	0.00999999999999999\\
113.01	0.00999999999999999\\
114.01	0.00999999999999999\\
115.01	0.00999999999999999\\
116.01	0.00999999999999999\\
117.01	0.00999999999999999\\
118.01	0.00999999999999999\\
119.01	0.00999999999999999\\
120.01	0.00999999999999999\\
121.01	0.00999999999999999\\
122.01	0.00999999999999999\\
123.01	0.00999999999999999\\
124.01	0.00999999999999999\\
125.01	0.00999999999999999\\
126.01	0.00999999999999999\\
127.01	0.00999999999999999\\
128.01	0.00999999999999999\\
129.01	0.00999999999999999\\
130.01	0.00999999999999999\\
131.01	0.00999999999999999\\
132.01	0.00999999999999999\\
133.01	0.00999999999999999\\
134.01	0.00999999999999999\\
135.01	0.00999999999999999\\
136.01	0.00999999999999999\\
137.01	0.00999999999999999\\
138.01	0.00999999999999999\\
139.01	0.00999999999999999\\
140.01	0.00999999999999999\\
141.01	0.00999999999999999\\
142.01	0.00999999999999999\\
143.01	0.00999999999999999\\
144.01	0.00999999999999999\\
145.01	0.00999999999999999\\
146.01	0.00999999999999999\\
147.01	0.00999999999999999\\
148.01	0.00999999999999999\\
149.01	0.00999999999999999\\
150.01	0.00999999999999999\\
151.01	0.00999999999999999\\
152.01	0.00999999999999999\\
153.01	0.00999999999999999\\
154.01	0.00999999999999999\\
155.01	0.00999999999999999\\
156.01	0.00999999999999999\\
157.01	0.00999999999999999\\
158.01	0.00999999999999999\\
159.01	0.00999999999999999\\
160.01	0.00999999999999999\\
161.01	0.00999999999999999\\
162.01	0.00999999999999999\\
163.01	0.00999999999999999\\
164.01	0.00999999999999999\\
165.01	0.00999999999999999\\
166.01	0.00999999999999999\\
167.01	0.00999999999999999\\
168.01	0.00999999999999999\\
169.01	0.00999999999999999\\
170.01	0.00999999999999999\\
171.01	0.00999999999999999\\
172.01	0.00999999999999999\\
173.01	0.00999999999999999\\
174.01	0.00999999999999999\\
175.01	0.00999999999999999\\
176.01	0.00999999999999999\\
177.01	0.00999999999999999\\
178.01	0.00999999999999999\\
179.01	0.00999999999999999\\
180.01	0.00999999999999999\\
181.01	0.00999999999999999\\
182.01	0.00999999999999999\\
183.01	0.00999999999999999\\
184.01	0.00999999999999999\\
185.01	0.00999999999999999\\
186.01	0.00999999999999999\\
187.01	0.00999999999999999\\
188.01	0.00999999999999999\\
189.01	0.00999999999999999\\
190.01	0.00999999999999999\\
191.01	0.00999999999999999\\
192.01	0.00999999999999999\\
193.01	0.00999999999999999\\
194.01	0.00999999999999999\\
195.01	0.00999999999999999\\
196.01	0.00999999999999999\\
197.01	0.00999999999999999\\
198.01	0.00999999999999999\\
199.01	0.00999999999999999\\
200.01	0.00999999999999999\\
201.01	0.00999999999999999\\
202.01	0.00999999999999999\\
203.01	0.00999999999999999\\
204.01	0.00999999999999999\\
205.01	0.00999999999999999\\
206.01	0.00999999999999999\\
207.01	0.00999999999999999\\
208.01	0.00999999999999999\\
209.01	0.00999999999999999\\
210.01	0.00999999999999999\\
211.01	0.00999999999999999\\
212.01	0.00999999999999999\\
213.01	0.00999999999999999\\
214.01	0.00999999999999999\\
215.01	0.00999999999999999\\
216.01	0.00999999999999999\\
217.01	0.00999999999999999\\
218.01	0.00999999999999999\\
219.01	0.00999999999999999\\
220.01	0.00999999999999999\\
221.01	0.00999999999999999\\
222.01	0.00999999999999999\\
223.01	0.00999999999999999\\
224.01	0.00999999999999999\\
225.01	0.00999999999999999\\
226.01	0.00999999999999999\\
227.01	0.00999999999999999\\
228.01	0.00999999999999999\\
229.01	0.00999999999999999\\
230.01	0.00999999999999999\\
231.01	0.00999999999999999\\
232.01	0.00999999999999999\\
233.01	0.00999999999999999\\
234.01	0.00999999999999999\\
235.01	0.00999999999999999\\
236.01	0.00999999999999999\\
237.01	0.00999999999999999\\
238.01	0.00999999999999999\\
239.01	0.00999999999999999\\
240.01	0.00999999999999999\\
241.01	0.00999999999999999\\
242.01	0.00999999999999999\\
243.01	0.00999999999999999\\
244.01	0.00999999999999999\\
245.01	0.00999999999999999\\
246.01	0.00999999999999999\\
247.01	0.00999999999999999\\
248.01	0.00999999999999999\\
249.01	0.00999999999999999\\
250.01	0.00999999999999999\\
251.01	0.00999999999999999\\
252.01	0.00999999999999999\\
253.01	0.00999999999999999\\
254.01	0.00999999999999999\\
255.01	0.00999999999999999\\
256.01	0.00999999999999999\\
257.01	0.00999999999999999\\
258.01	0.00999999999999999\\
259.01	0.00999999999999999\\
260.01	0.00999999999999999\\
261.01	0.00999999999999999\\
262.01	0.00999999999999999\\
263.01	0.00999999999999999\\
264.01	0.00999999999999999\\
265.01	0.00999999999999999\\
266.01	0.00999999999999999\\
267.01	0.00999999999999999\\
268.01	0.00999999999999999\\
269.01	0.00999999999999999\\
270.01	0.00999999999999999\\
271.01	0.00999999999999999\\
272.01	0.00999999999999999\\
273.01	0.00999999999999999\\
274.01	0.00999999999999999\\
275.01	0.00999999999999999\\
276.01	0.00999999999999999\\
277.01	0.00999999999999999\\
278.01	0.00999999999999999\\
279.01	0.00999999999999999\\
280.01	0.00999999999999999\\
281.01	0.00999999999999999\\
282.01	0.00999999999999999\\
283.01	0.00999999999999999\\
284.01	0.00999999999999999\\
285.01	0.00999999999999999\\
286.01	0.00999999999999999\\
287.01	0.00999999999999999\\
288.01	0.00999999999999999\\
289.01	0.00999999999999999\\
290.01	0.00999999999999999\\
291.01	0.00999999999999999\\
292.01	0.00999999999999999\\
293.01	0.00999999999999999\\
294.01	0.00999999999999999\\
295.01	0.00999999999999999\\
296.01	0.00999999999999999\\
297.01	0.00999999999999999\\
298.01	0.00999999999999999\\
299.01	0.00999999999999999\\
300.01	0.00999999999999999\\
301.01	0.00999999999999999\\
302.01	0.00999999999999999\\
303.01	0.00999999999999999\\
304.01	0.00999999999999999\\
305.01	0.00999999999999999\\
306.01	0.00999999999999999\\
307.01	0.00999999999999999\\
308.01	0.00999999999999999\\
309.01	0.00999999999999999\\
310.01	0.00999999999999999\\
311.01	0.00999999999999999\\
312.01	0.00999999999999999\\
313.01	0.00999999999999999\\
314.01	0.00999999999999999\\
315.01	0.00999999999999999\\
316.01	0.00999999999999999\\
317.01	0.00999999999999999\\
318.01	0.00999999999999999\\
319.01	0.00999999999999999\\
320.01	0.00999999999999999\\
321.01	0.00999999999999999\\
322.01	0.00999999999999999\\
323.01	0.00999999999999999\\
324.01	0.00999999999999999\\
325.01	0.00999999999999999\\
326.01	0.00999999999999999\\
327.01	0.00999999999999999\\
328.01	0.00999999999999999\\
329.01	0.00999999999999999\\
330.01	0.00999999999999999\\
331.01	0.00999999999999999\\
332.01	0.00999999999999999\\
333.01	0.00999999999999999\\
334.01	0.00999999999999999\\
335.01	0.00999999999999999\\
336.01	0.00999999999999999\\
337.01	0.00999999999999999\\
338.01	0.00999999999999999\\
339.01	0.00999999999999999\\
340.01	0.00999999999999999\\
341.01	0.00999999999999999\\
342.01	0.00999999999999999\\
343.01	0.00999999999999999\\
344.01	0.00999999999999999\\
345.01	0.00999999999999999\\
346.01	0.00999999999999999\\
347.01	0.00999999999999999\\
348.01	0.00999999999999999\\
349.01	0.00999999999999999\\
350.01	0.00999999999999999\\
351.01	0.00999999999999999\\
352.01	0.00999999999999999\\
353.01	0.00999999999999999\\
354.01	0.00999999999999999\\
355.01	0.00999999999999999\\
356.01	0.00999999999999999\\
357.01	0.00999999999999999\\
358.01	0.00999999999999999\\
359.01	0.00999999999999999\\
360.01	0.00999999999999999\\
361.01	0.00999999999999999\\
362.01	0.00999999999999999\\
363.01	0.00999999999999999\\
364.01	0.00999999999999999\\
365.01	0.00999999999999999\\
366.01	0.00999999999999999\\
367.01	0.00999999999999999\\
368.01	0.00999999999999999\\
369.01	0.00999999999999999\\
370.01	0.00999999999999999\\
371.01	0.00999999999999999\\
372.01	0.00999999999999999\\
373.01	0.00999999999999999\\
374.01	0.00999999999999999\\
375.01	0.00999999999999999\\
376.01	0.00999999999999999\\
377.01	0.00999999999999999\\
378.01	0.00999999999999999\\
379.01	0.00999999999999999\\
380.01	0.00999999999999999\\
381.01	0.00999999999999999\\
382.01	0.00999999999999999\\
383.01	0.00999999999999999\\
384.01	0.00999999999999999\\
385.01	0.00999999999999999\\
386.01	0.00999999999999999\\
387.01	0.00999999999999999\\
388.01	0.00999999999999999\\
389.01	0.00999999999999999\\
390.01	0.00999999999999999\\
391.01	0.00999999999999999\\
392.01	0.00999999999999999\\
393.01	0.00999999999999999\\
394.01	0.00999999999999999\\
395.01	0.00999999999999999\\
396.01	0.00999999999999999\\
397.01	0.00999999999999999\\
398.01	0.00999999999999999\\
399.01	0.00999999999999999\\
400.01	0.00999999999999999\\
401.01	0.00999999999999999\\
402.01	0.00999999999999999\\
403.01	0.00999999999999999\\
404.01	0.00999999999999999\\
405.01	0.00999999999999999\\
406.01	0.00999999999999999\\
407.01	0.00999999999999999\\
408.01	0.00999999999999999\\
409.01	0.00999999999999999\\
410.01	0.00999999999999999\\
411.01	0.00999999999999999\\
412.01	0.00999999999999999\\
413.01	0.00999999999999999\\
414.01	0.00999999999999999\\
415.01	0.00999999999999999\\
416.01	0.00999999999999999\\
417.01	0.00999999999999999\\
418.01	0.00999999999999999\\
419.01	0.00999999999999999\\
420.01	0.00999999999999999\\
421.01	0.00999999999999999\\
422.01	0.00999999999999999\\
423.01	0.00999999999999999\\
424.01	0.00999999999999999\\
425.01	0.00999999999999999\\
426.01	0.00999999999999999\\
427.01	0.00999999999999999\\
428.01	0.00999999999999999\\
429.01	0.00999999999999999\\
430.01	0.00999999999999999\\
431.01	0.00999999999999999\\
432.01	0.00999999999999999\\
433.01	0.00999999999999999\\
434.01	0.00999999999999999\\
435.01	0.00999999999999999\\
436.01	0.00999999999999999\\
437.01	0.00999999999999999\\
438.01	0.00999999999999999\\
439.01	0.00999999999999999\\
440.01	0.00999999999999999\\
441.01	0.00999999999999999\\
442.01	0.00999999999999999\\
443.01	0.00999999999999999\\
444.01	0.00999999999999999\\
445.01	0.00999999999999999\\
446.01	0.00999999999999999\\
447.01	0.00999999999999999\\
448.01	0.00999999999999999\\
449.01	0.00999999999999999\\
450.01	0.00999999999999999\\
451.01	0.00999999999999999\\
452.01	0.00999999999999999\\
453.01	0.00999999999999999\\
454.01	0.00999999999999999\\
455.01	0.00999999999999999\\
456.01	0.00999999999999999\\
457.01	0.00999999999999999\\
458.01	0.00999999999999999\\
459.01	0.00999999999999999\\
460.01	0.00999999999999999\\
461.01	0.00999999999999999\\
462.01	0.00999999999999999\\
463.01	0.00999999999999999\\
464.01	0.00999999999999999\\
465.01	0.00999999999999999\\
466.01	0.00999999999999999\\
467.01	0.00999999999999999\\
468.01	0.00999999999999999\\
469.01	0.00999999999999999\\
470.01	0.00999999999999999\\
471.01	0.00999999999999999\\
472.01	0.00999999999999999\\
473.01	0.00999999999999999\\
474.01	0.00999999999999999\\
475.01	0.00999999999999999\\
476.01	0.00999999999999999\\
477.01	0.00999999999999999\\
478.01	0.00999999999999999\\
479.01	0.00999999999999999\\
480.01	0.00999999999999999\\
481.01	0.00999999999999999\\
482.01	0.00999999999999999\\
483.01	0.00999999999999999\\
484.01	0.00999999999999999\\
485.01	0.00999999999999999\\
486.01	0.00999999999999999\\
487.01	0.00999999999999999\\
488.01	0.00999999999999999\\
489.01	0.00999999999999999\\
490.01	0.00999999999999999\\
491.01	0.00999999999999999\\
492.01	0.00999999999999999\\
493.01	0.00999999999999999\\
494.01	0.00999999999999999\\
495.01	0.00999999999999999\\
496.01	0.00999999999999999\\
497.01	0.00999999999999999\\
498.01	0.00999999999999999\\
499.01	0.00999999999999999\\
500.01	0.00999999999999999\\
501.01	0.00999999999999999\\
502.01	0.00999999999999999\\
503.01	0.00999999999999999\\
504.01	0.00999999999999999\\
505.01	0.00999999999999999\\
506.01	0.00999999999999999\\
507.01	0.00999999999999999\\
508.01	0.00999999999999999\\
509.01	0.00999999999999999\\
510.01	0.00999999999999999\\
511.01	0.00999999999999999\\
512.01	0.00999999999999999\\
513.01	0.00999999999999999\\
514.01	0.00999999999999999\\
515.01	0.00999999999999999\\
516.01	0.00999999999999999\\
517.01	0.00999999999999999\\
518.01	0.00999999999999999\\
519.01	0.00999999999999999\\
520.01	0.00999999999999999\\
521.01	0.00999999999999999\\
522.01	0.00999999999999999\\
523.01	0.00999999999999999\\
524.01	0.00999999999999999\\
525.01	0.00999999999999999\\
526.01	0.00999999999999999\\
527.01	0.00999999999999999\\
528.01	0.00999999999999999\\
529.01	0.00999999999999999\\
530.01	0.00999999999999999\\
531.01	0.00999999999999999\\
532.01	0.00999999999999999\\
533.01	0.00999999999999999\\
534.01	0.00999999999999999\\
535.01	0.00999999999999999\\
536.01	0.00999999999999999\\
537.01	0.00999999999999999\\
538.01	0.00999999999999999\\
539.01	0.00999999999999999\\
540.01	0.00999999999999999\\
541.01	0.00999999999999999\\
542.01	0.00999999999999999\\
543.01	0.00999999999999999\\
544.01	0.00999999999999999\\
545.01	0.00999999999999999\\
546.01	0.00999999999999999\\
547.01	0.00999999999999999\\
548.01	0.00999999999999999\\
549.01	0.00999999999999999\\
550.01	0.00999999999999999\\
551.01	0.00999999999999999\\
552.01	0.00999999999999999\\
553.01	0.00999999999999999\\
554.01	0.00999999999999999\\
555.01	0.00999999999999999\\
556.01	0.00999999999999999\\
557.01	0.00999999999999999\\
558.01	0.00999999999999999\\
559.01	0.00999999999999999\\
560.01	0.00999999999999999\\
561.01	0.00999999999999999\\
562.01	0.00999999999999999\\
563.01	0.00999999999999999\\
564.01	0.00999999999999999\\
565.01	0.00999999999999999\\
566.01	0.00999999999999999\\
567.01	0.00999999999999999\\
568.01	0.00999999999999999\\
569.01	0.00999999999999999\\
570.01	0.00999999999999999\\
571.01	0.00999999999999999\\
572.01	0.00999999999999999\\
573.01	0.00999999999999999\\
574.01	0.00999999999999999\\
575.01	0.00999999999999999\\
576.01	0.00999999999999999\\
577.01	0.00999999999999999\\
578.01	0.00999999999999999\\
579.01	0.00999999999999999\\
580.01	0.00999999999999999\\
581.01	0.00999999999999999\\
582.01	0.00999999999999999\\
583.01	0.00999999999999999\\
584.01	0.00999999999999999\\
585.01	0.00999999999999999\\
586.01	0.00999999999999999\\
587.01	0.00999999999999999\\
588.01	0.00999999999999999\\
589.01	0.00999999999999999\\
590.01	0.00999999999999999\\
591.01	0.00999999999999999\\
592.01	0.00999999999999999\\
593.01	0.00999999999999999\\
594.01	0.00999999999999999\\
595.01	0.00999999999999999\\
596.01	0.00999999999999999\\
597.01	0.00999999999999999\\
598.01	0.00999999999999999\\
599.01	0.00624186909405409\\
599.02	0.00620414441242161\\
599.03	0.00616605275622734\\
599.04	0.00612759052101569\\
599.05	0.00608875406692034\\
599.06	0.00604953971831641\\
599.07	0.00600994376346915\\
599.08	0.00596996245417917\\
599.09	0.00592959200542424\\
599.1	0.0058888285949975\\
599.11	0.00584766836314219\\
599.12	0.00580610741218273\\
599.13	0.00576414180615222\\
599.14	0.00572176757041625\\
599.15	0.00567898069129305\\
599.16	0.00563577711566982\\
599.17	0.00559215275061544\\
599.18	0.00554810346298924\\
599.19	0.00550362507904598\\
599.2	0.005458713384037\\
599.21	0.00541336412180727\\
599.22	0.00536757299438878\\
599.23	0.00532133566158964\\
599.24	0.00527464774057931\\
599.25	0.00522750480546972\\
599.26	0.00517990238689226\\
599.27	0.00513183597157064\\
599.28	0.0050833010018895\\
599.29	0.00503429287545883\\
599.3	0.00498480694467416\\
599.31	0.00493483851627227\\
599.32	0.00488438285088274\\
599.33	0.0048334351625749\\
599.34	0.00478199061840046\\
599.35	0.0047300443379316\\
599.36	0.00467759139279454\\
599.37	0.00462462680619849\\
599.38	0.00457114555246003\\
599.39	0.00451714256049735\\
599.4	0.00446261272376357\\
599.41	0.00440755088572618\\
599.42	0.00435195183937782\\
599.43	0.00429581032674211\\
599.44	0.00423912103837487\\
599.45	0.00418187861286022\\
599.46	0.00412407763630193\\
599.47	0.00406571264180971\\
599.48	0.00400677810898045\\
599.49	0.00394726846337441\\
599.5	0.00388717807598638\\
599.51	0.00382650126271143\\
599.52	0.00376523228380568\\
599.53	0.00370336534334161\\
599.54	0.00364089458865821\\
599.55	0.00357781410980551\\
599.56	0.00351411793898393\\
599.57	0.00344980004997799\\
599.58	0.00338485435758448\\
599.59	0.0033192747170351\\
599.6	0.00325305492341341\\
599.61	0.00318618871106604\\
599.62	0.00311866975300822\\
599.63	0.00305049166032343\\
599.64	0.00298164798155717\\
599.65	0.00291213220210483\\
599.66	0.00284193774359353\\
599.67	0.00277105796325798\\
599.68	0.00269948615331014\\
599.69	0.00262721554030275\\
599.7	0.00255423928448666\\
599.71	0.00248055047916181\\
599.72	0.00240614215002192\\
599.73	0.00233100725449273\\
599.74	0.00225513868106373\\
599.75	0.00217852924861344\\
599.76	0.00210117170572799\\
599.77	0.0020230587300131\\
599.78	0.00194418292739928\\
599.79	0.00186453683144023\\
599.8	0.00178411290260446\\
599.81	0.00170290352755979\\
599.82	0.00162090101845109\\
599.83	0.00153809761217072\\
599.84	0.001454485469622\\
599.85	0.00137005667497534\\
599.86	0.00128480323491721\\
599.87	0.00119871707789166\\
599.88	0.00111179005333448\\
599.89	0.00102401393089982\\
599.9	0.000935380399679274\\
599.91	0.000845881067413288\\
599.92	0.000755507459694882\\
599.93	0.000664251019165561\\
599.94	0.000572103104703355\\
599.95	0.00047905499060291\\
599.96	0.000385097865747554\\
599.97	0.000290222832773275\\
599.98	0.000194420907224489\\
599.99	9.76830167015649e-05\\
600	0\\
};
\addplot [color=mycolor20,solid,forget plot]
  table[row sep=crcr]{%
0.01	0.01\\
1.01	0.01\\
2.01	0.01\\
3.01	0.01\\
4.01	0.01\\
5.01	0.01\\
6.01	0.01\\
7.01	0.01\\
8.01	0.01\\
9.01	0.01\\
10.01	0.01\\
11.01	0.01\\
12.01	0.01\\
13.01	0.01\\
14.01	0.01\\
15.01	0.01\\
16.01	0.01\\
17.01	0.01\\
18.01	0.01\\
19.01	0.01\\
20.01	0.01\\
21.01	0.01\\
22.01	0.01\\
23.01	0.01\\
24.01	0.01\\
25.01	0.01\\
26.01	0.01\\
27.01	0.01\\
28.01	0.01\\
29.01	0.01\\
30.01	0.01\\
31.01	0.01\\
32.01	0.01\\
33.01	0.01\\
34.01	0.01\\
35.01	0.01\\
36.01	0.01\\
37.01	0.01\\
38.01	0.01\\
39.01	0.01\\
40.01	0.01\\
41.01	0.01\\
42.01	0.01\\
43.01	0.01\\
44.01	0.01\\
45.01	0.01\\
46.01	0.01\\
47.01	0.01\\
48.01	0.01\\
49.01	0.01\\
50.01	0.01\\
51.01	0.01\\
52.01	0.01\\
53.01	0.01\\
54.01	0.01\\
55.01	0.01\\
56.01	0.01\\
57.01	0.01\\
58.01	0.01\\
59.01	0.01\\
60.01	0.01\\
61.01	0.01\\
62.01	0.01\\
63.01	0.01\\
64.01	0.01\\
65.01	0.01\\
66.01	0.01\\
67.01	0.01\\
68.01	0.01\\
69.01	0.01\\
70.01	0.01\\
71.01	0.01\\
72.01	0.01\\
73.01	0.01\\
74.01	0.01\\
75.01	0.01\\
76.01	0.01\\
77.01	0.01\\
78.01	0.01\\
79.01	0.01\\
80.01	0.01\\
81.01	0.01\\
82.01	0.01\\
83.01	0.01\\
84.01	0.01\\
85.01	0.01\\
86.01	0.01\\
87.01	0.01\\
88.01	0.01\\
89.01	0.01\\
90.01	0.01\\
91.01	0.01\\
92.01	0.01\\
93.01	0.01\\
94.01	0.01\\
95.01	0.01\\
96.01	0.01\\
97.01	0.01\\
98.01	0.01\\
99.01	0.01\\
100.01	0.01\\
101.01	0.01\\
102.01	0.01\\
103.01	0.01\\
104.01	0.01\\
105.01	0.01\\
106.01	0.01\\
107.01	0.01\\
108.01	0.01\\
109.01	0.01\\
110.01	0.01\\
111.01	0.01\\
112.01	0.01\\
113.01	0.01\\
114.01	0.01\\
115.01	0.01\\
116.01	0.01\\
117.01	0.01\\
118.01	0.01\\
119.01	0.01\\
120.01	0.01\\
121.01	0.01\\
122.01	0.01\\
123.01	0.01\\
124.01	0.01\\
125.01	0.01\\
126.01	0.01\\
127.01	0.01\\
128.01	0.01\\
129.01	0.01\\
130.01	0.01\\
131.01	0.01\\
132.01	0.01\\
133.01	0.01\\
134.01	0.01\\
135.01	0.01\\
136.01	0.01\\
137.01	0.01\\
138.01	0.01\\
139.01	0.01\\
140.01	0.01\\
141.01	0.01\\
142.01	0.01\\
143.01	0.01\\
144.01	0.01\\
145.01	0.01\\
146.01	0.01\\
147.01	0.01\\
148.01	0.01\\
149.01	0.01\\
150.01	0.01\\
151.01	0.01\\
152.01	0.01\\
153.01	0.01\\
154.01	0.01\\
155.01	0.01\\
156.01	0.01\\
157.01	0.01\\
158.01	0.01\\
159.01	0.01\\
160.01	0.01\\
161.01	0.01\\
162.01	0.01\\
163.01	0.01\\
164.01	0.01\\
165.01	0.01\\
166.01	0.01\\
167.01	0.01\\
168.01	0.01\\
169.01	0.01\\
170.01	0.01\\
171.01	0.01\\
172.01	0.01\\
173.01	0.01\\
174.01	0.01\\
175.01	0.01\\
176.01	0.01\\
177.01	0.01\\
178.01	0.01\\
179.01	0.01\\
180.01	0.01\\
181.01	0.01\\
182.01	0.01\\
183.01	0.01\\
184.01	0.01\\
185.01	0.01\\
186.01	0.01\\
187.01	0.01\\
188.01	0.01\\
189.01	0.01\\
190.01	0.01\\
191.01	0.01\\
192.01	0.01\\
193.01	0.01\\
194.01	0.01\\
195.01	0.01\\
196.01	0.01\\
197.01	0.01\\
198.01	0.01\\
199.01	0.01\\
200.01	0.01\\
201.01	0.01\\
202.01	0.01\\
203.01	0.01\\
204.01	0.01\\
205.01	0.01\\
206.01	0.01\\
207.01	0.01\\
208.01	0.01\\
209.01	0.01\\
210.01	0.01\\
211.01	0.01\\
212.01	0.01\\
213.01	0.01\\
214.01	0.01\\
215.01	0.01\\
216.01	0.01\\
217.01	0.01\\
218.01	0.01\\
219.01	0.01\\
220.01	0.01\\
221.01	0.01\\
222.01	0.01\\
223.01	0.01\\
224.01	0.01\\
225.01	0.01\\
226.01	0.01\\
227.01	0.01\\
228.01	0.01\\
229.01	0.01\\
230.01	0.01\\
231.01	0.01\\
232.01	0.01\\
233.01	0.01\\
234.01	0.01\\
235.01	0.01\\
236.01	0.01\\
237.01	0.01\\
238.01	0.01\\
239.01	0.01\\
240.01	0.01\\
241.01	0.01\\
242.01	0.01\\
243.01	0.01\\
244.01	0.01\\
245.01	0.01\\
246.01	0.01\\
247.01	0.01\\
248.01	0.01\\
249.01	0.01\\
250.01	0.01\\
251.01	0.01\\
252.01	0.01\\
253.01	0.01\\
254.01	0.01\\
255.01	0.01\\
256.01	0.01\\
257.01	0.01\\
258.01	0.01\\
259.01	0.01\\
260.01	0.01\\
261.01	0.01\\
262.01	0.01\\
263.01	0.01\\
264.01	0.01\\
265.01	0.01\\
266.01	0.01\\
267.01	0.01\\
268.01	0.01\\
269.01	0.01\\
270.01	0.01\\
271.01	0.01\\
272.01	0.01\\
273.01	0.01\\
274.01	0.01\\
275.01	0.01\\
276.01	0.01\\
277.01	0.01\\
278.01	0.01\\
279.01	0.01\\
280.01	0.01\\
281.01	0.01\\
282.01	0.01\\
283.01	0.01\\
284.01	0.01\\
285.01	0.01\\
286.01	0.01\\
287.01	0.01\\
288.01	0.01\\
289.01	0.01\\
290.01	0.01\\
291.01	0.01\\
292.01	0.01\\
293.01	0.01\\
294.01	0.01\\
295.01	0.01\\
296.01	0.01\\
297.01	0.01\\
298.01	0.01\\
299.01	0.01\\
300.01	0.01\\
301.01	0.01\\
302.01	0.01\\
303.01	0.01\\
304.01	0.01\\
305.01	0.01\\
306.01	0.01\\
307.01	0.01\\
308.01	0.01\\
309.01	0.01\\
310.01	0.01\\
311.01	0.01\\
312.01	0.01\\
313.01	0.01\\
314.01	0.01\\
315.01	0.01\\
316.01	0.01\\
317.01	0.01\\
318.01	0.01\\
319.01	0.01\\
320.01	0.01\\
321.01	0.01\\
322.01	0.01\\
323.01	0.01\\
324.01	0.01\\
325.01	0.01\\
326.01	0.01\\
327.01	0.01\\
328.01	0.01\\
329.01	0.01\\
330.01	0.01\\
331.01	0.01\\
332.01	0.01\\
333.01	0.01\\
334.01	0.01\\
335.01	0.01\\
336.01	0.01\\
337.01	0.01\\
338.01	0.01\\
339.01	0.01\\
340.01	0.01\\
341.01	0.01\\
342.01	0.01\\
343.01	0.01\\
344.01	0.01\\
345.01	0.01\\
346.01	0.01\\
347.01	0.01\\
348.01	0.01\\
349.01	0.01\\
350.01	0.01\\
351.01	0.01\\
352.01	0.01\\
353.01	0.01\\
354.01	0.01\\
355.01	0.01\\
356.01	0.01\\
357.01	0.01\\
358.01	0.01\\
359.01	0.01\\
360.01	0.01\\
361.01	0.01\\
362.01	0.01\\
363.01	0.01\\
364.01	0.01\\
365.01	0.01\\
366.01	0.01\\
367.01	0.01\\
368.01	0.01\\
369.01	0.01\\
370.01	0.01\\
371.01	0.01\\
372.01	0.01\\
373.01	0.01\\
374.01	0.01\\
375.01	0.01\\
376.01	0.01\\
377.01	0.01\\
378.01	0.01\\
379.01	0.01\\
380.01	0.01\\
381.01	0.01\\
382.01	0.01\\
383.01	0.01\\
384.01	0.01\\
385.01	0.01\\
386.01	0.01\\
387.01	0.01\\
388.01	0.01\\
389.01	0.01\\
390.01	0.01\\
391.01	0.01\\
392.01	0.01\\
393.01	0.01\\
394.01	0.01\\
395.01	0.01\\
396.01	0.01\\
397.01	0.01\\
398.01	0.01\\
399.01	0.01\\
400.01	0.01\\
401.01	0.01\\
402.01	0.01\\
403.01	0.01\\
404.01	0.01\\
405.01	0.01\\
406.01	0.01\\
407.01	0.01\\
408.01	0.01\\
409.01	0.01\\
410.01	0.01\\
411.01	0.01\\
412.01	0.01\\
413.01	0.01\\
414.01	0.01\\
415.01	0.01\\
416.01	0.01\\
417.01	0.01\\
418.01	0.01\\
419.01	0.01\\
420.01	0.01\\
421.01	0.01\\
422.01	0.01\\
423.01	0.01\\
424.01	0.01\\
425.01	0.01\\
426.01	0.01\\
427.01	0.01\\
428.01	0.01\\
429.01	0.01\\
430.01	0.01\\
431.01	0.01\\
432.01	0.01\\
433.01	0.01\\
434.01	0.01\\
435.01	0.01\\
436.01	0.01\\
437.01	0.01\\
438.01	0.01\\
439.01	0.01\\
440.01	0.01\\
441.01	0.01\\
442.01	0.01\\
443.01	0.01\\
444.01	0.01\\
445.01	0.01\\
446.01	0.01\\
447.01	0.01\\
448.01	0.01\\
449.01	0.01\\
450.01	0.01\\
451.01	0.01\\
452.01	0.01\\
453.01	0.01\\
454.01	0.01\\
455.01	0.01\\
456.01	0.01\\
457.01	0.01\\
458.01	0.01\\
459.01	0.01\\
460.01	0.01\\
461.01	0.01\\
462.01	0.01\\
463.01	0.01\\
464.01	0.01\\
465.01	0.01\\
466.01	0.01\\
467.01	0.01\\
468.01	0.01\\
469.01	0.01\\
470.01	0.01\\
471.01	0.01\\
472.01	0.01\\
473.01	0.01\\
474.01	0.01\\
475.01	0.01\\
476.01	0.01\\
477.01	0.01\\
478.01	0.01\\
479.01	0.01\\
480.01	0.01\\
481.01	0.01\\
482.01	0.01\\
483.01	0.01\\
484.01	0.01\\
485.01	0.01\\
486.01	0.01\\
487.01	0.01\\
488.01	0.01\\
489.01	0.01\\
490.01	0.01\\
491.01	0.01\\
492.01	0.01\\
493.01	0.01\\
494.01	0.01\\
495.01	0.01\\
496.01	0.01\\
497.01	0.01\\
498.01	0.01\\
499.01	0.01\\
500.01	0.01\\
501.01	0.01\\
502.01	0.01\\
503.01	0.01\\
504.01	0.01\\
505.01	0.01\\
506.01	0.01\\
507.01	0.01\\
508.01	0.01\\
509.01	0.01\\
510.01	0.01\\
511.01	0.01\\
512.01	0.01\\
513.01	0.01\\
514.01	0.01\\
515.01	0.01\\
516.01	0.01\\
517.01	0.01\\
518.01	0.01\\
519.01	0.01\\
520.01	0.01\\
521.01	0.01\\
522.01	0.01\\
523.01	0.01\\
524.01	0.01\\
525.01	0.01\\
526.01	0.01\\
527.01	0.01\\
528.01	0.01\\
529.01	0.01\\
530.01	0.01\\
531.01	0.01\\
532.01	0.01\\
533.01	0.01\\
534.01	0.01\\
535.01	0.01\\
536.01	0.01\\
537.01	0.01\\
538.01	0.01\\
539.01	0.01\\
540.01	0.01\\
541.01	0.01\\
542.01	0.01\\
543.01	0.01\\
544.01	0.01\\
545.01	0.01\\
546.01	0.01\\
547.01	0.01\\
548.01	0.01\\
549.01	0.01\\
550.01	0.01\\
551.01	0.01\\
552.01	0.01\\
553.01	0.01\\
554.01	0.01\\
555.01	0.01\\
556.01	0.01\\
557.01	0.01\\
558.01	0.01\\
559.01	0.01\\
560.01	0.01\\
561.01	0.01\\
562.01	0.01\\
563.01	0.01\\
564.01	0.01\\
565.01	0.01\\
566.01	0.01\\
567.01	0.01\\
568.01	0.01\\
569.01	0.01\\
570.01	0.01\\
571.01	0.01\\
572.01	0.01\\
573.01	0.01\\
574.01	0.01\\
575.01	0.01\\
576.01	0.01\\
577.01	0.01\\
578.01	0.01\\
579.01	0.01\\
580.01	0.01\\
581.01	0.01\\
582.01	0.01\\
583.01	0.01\\
584.01	0.01\\
585.01	0.01\\
586.01	0.01\\
587.01	0.01\\
588.01	0.01\\
589.01	0.01\\
590.01	0.01\\
591.01	0.01\\
592.01	0.01\\
593.01	0.01\\
594.01	0.01\\
595.01	0.01\\
596.01	0.01\\
597.01	0.01\\
598.01	0.01\\
599.01	0.00624186909405406\\
599.02	0.00620414441242159\\
599.03	0.00616605275622732\\
599.04	0.00612759052101566\\
599.05	0.0060887540669203\\
599.06	0.00604953971831638\\
599.07	0.00600994376346912\\
599.08	0.00596996245417914\\
599.09	0.00592959200542422\\
599.1	0.00588882859499749\\
599.11	0.00584766836314218\\
599.12	0.00580610741218272\\
599.13	0.00576414180615221\\
599.14	0.00572176757041625\\
599.15	0.00567898069129305\\
599.16	0.00563577711566981\\
599.17	0.00559215275061543\\
599.18	0.00554810346298924\\
599.19	0.00550362507904598\\
599.2	0.005458713384037\\
599.21	0.00541336412180729\\
599.22	0.0053675729943888\\
599.23	0.00532133566158968\\
599.24	0.00527464774057935\\
599.25	0.00522750480546976\\
599.26	0.0051799023868923\\
599.27	0.00513183597157067\\
599.28	0.00508330100188951\\
599.29	0.00503429287545883\\
599.3	0.00498480694467416\\
599.31	0.00493483851627228\\
599.32	0.00488438285088274\\
599.33	0.00483343516257489\\
599.34	0.00478199061840044\\
599.35	0.00473004433793157\\
599.36	0.00467759139279451\\
599.37	0.00462462680619847\\
599.38	0.00457114555246001\\
599.39	0.00451714256049733\\
599.4	0.00446261272376355\\
599.41	0.00440755088572616\\
599.42	0.0043519518393778\\
599.43	0.00429581032674209\\
599.44	0.00423912103837485\\
599.45	0.00418187861286021\\
599.46	0.00412407763630192\\
599.47	0.0040657126418097\\
599.48	0.00400677810898044\\
599.49	0.00394726846337442\\
599.5	0.00388717807598638\\
599.51	0.00382650126271142\\
599.52	0.00376523228380568\\
599.53	0.00370336534334162\\
599.54	0.0036408945886582\\
599.55	0.00357781410980551\\
599.56	0.00351411793898395\\
599.57	0.00344980004997801\\
599.58	0.0033848543575845\\
599.59	0.00331927471703512\\
599.6	0.00325305492341343\\
599.61	0.00318618871106605\\
599.62	0.00311866975300824\\
599.63	0.00305049166032345\\
599.64	0.00298164798155719\\
599.65	0.00291213220210484\\
599.66	0.00284193774359355\\
599.67	0.002771057963258\\
599.68	0.00269948615331015\\
599.69	0.00262721554030276\\
599.7	0.00255423928448666\\
599.71	0.00248055047916181\\
599.72	0.00240614215002192\\
599.73	0.00233100725449273\\
599.74	0.00225513868106373\\
599.75	0.00217852924861344\\
599.76	0.00210117170572799\\
599.77	0.0020230587300131\\
599.78	0.00194418292739928\\
599.79	0.00186453683144024\\
599.8	0.00178411290260446\\
599.81	0.00170290352755979\\
599.82	0.00162090101845109\\
599.83	0.00153809761217073\\
599.84	0.001454485469622\\
599.85	0.00137005667497534\\
599.86	0.00128480323491721\\
599.87	0.00119871707789166\\
599.88	0.00111179005333449\\
599.89	0.00102401393089983\\
599.9	0.000935380399679279\\
599.91	0.000845881067413288\\
599.92	0.000755507459694884\\
599.93	0.000664251019165563\\
599.94	0.00057210310470336\\
599.95	0.00047905499060291\\
599.96	0.000385097865747556\\
599.97	0.000290222832773275\\
599.98	0.000194420907224487\\
599.99	9.76830167015649e-05\\
600	0\\
};
\addplot [color=mycolor21,solid,forget plot]
  table[row sep=crcr]{%
0.01	0.01\\
1.01	0.01\\
2.01	0.01\\
3.01	0.01\\
4.01	0.01\\
5.01	0.01\\
6.01	0.01\\
7.01	0.01\\
8.01	0.01\\
9.01	0.01\\
10.01	0.01\\
11.01	0.01\\
12.01	0.01\\
13.01	0.01\\
14.01	0.01\\
15.01	0.01\\
16.01	0.01\\
17.01	0.01\\
18.01	0.01\\
19.01	0.01\\
20.01	0.01\\
21.01	0.01\\
22.01	0.01\\
23.01	0.01\\
24.01	0.01\\
25.01	0.01\\
26.01	0.01\\
27.01	0.01\\
28.01	0.01\\
29.01	0.01\\
30.01	0.01\\
31.01	0.01\\
32.01	0.01\\
33.01	0.01\\
34.01	0.01\\
35.01	0.01\\
36.01	0.01\\
37.01	0.01\\
38.01	0.01\\
39.01	0.01\\
40.01	0.01\\
41.01	0.01\\
42.01	0.01\\
43.01	0.01\\
44.01	0.01\\
45.01	0.01\\
46.01	0.01\\
47.01	0.01\\
48.01	0.01\\
49.01	0.01\\
50.01	0.01\\
51.01	0.01\\
52.01	0.01\\
53.01	0.01\\
54.01	0.01\\
55.01	0.01\\
56.01	0.01\\
57.01	0.01\\
58.01	0.01\\
59.01	0.01\\
60.01	0.01\\
61.01	0.01\\
62.01	0.01\\
63.01	0.01\\
64.01	0.01\\
65.01	0.01\\
66.01	0.01\\
67.01	0.01\\
68.01	0.01\\
69.01	0.01\\
70.01	0.01\\
71.01	0.01\\
72.01	0.01\\
73.01	0.01\\
74.01	0.01\\
75.01	0.01\\
76.01	0.01\\
77.01	0.01\\
78.01	0.01\\
79.01	0.01\\
80.01	0.01\\
81.01	0.01\\
82.01	0.01\\
83.01	0.01\\
84.01	0.01\\
85.01	0.01\\
86.01	0.01\\
87.01	0.01\\
88.01	0.01\\
89.01	0.01\\
90.01	0.01\\
91.01	0.01\\
92.01	0.01\\
93.01	0.01\\
94.01	0.01\\
95.01	0.01\\
96.01	0.01\\
97.01	0.01\\
98.01	0.01\\
99.01	0.01\\
100.01	0.01\\
101.01	0.01\\
102.01	0.01\\
103.01	0.01\\
104.01	0.01\\
105.01	0.01\\
106.01	0.01\\
107.01	0.01\\
108.01	0.01\\
109.01	0.01\\
110.01	0.01\\
111.01	0.01\\
112.01	0.01\\
113.01	0.01\\
114.01	0.01\\
115.01	0.01\\
116.01	0.01\\
117.01	0.01\\
118.01	0.01\\
119.01	0.01\\
120.01	0.01\\
121.01	0.01\\
122.01	0.01\\
123.01	0.01\\
124.01	0.01\\
125.01	0.01\\
126.01	0.01\\
127.01	0.01\\
128.01	0.01\\
129.01	0.01\\
130.01	0.01\\
131.01	0.01\\
132.01	0.01\\
133.01	0.01\\
134.01	0.01\\
135.01	0.01\\
136.01	0.01\\
137.01	0.01\\
138.01	0.01\\
139.01	0.01\\
140.01	0.01\\
141.01	0.01\\
142.01	0.01\\
143.01	0.01\\
144.01	0.01\\
145.01	0.01\\
146.01	0.01\\
147.01	0.01\\
148.01	0.01\\
149.01	0.01\\
150.01	0.01\\
151.01	0.01\\
152.01	0.01\\
153.01	0.01\\
154.01	0.01\\
155.01	0.01\\
156.01	0.01\\
157.01	0.01\\
158.01	0.01\\
159.01	0.01\\
160.01	0.01\\
161.01	0.01\\
162.01	0.01\\
163.01	0.01\\
164.01	0.01\\
165.01	0.01\\
166.01	0.01\\
167.01	0.01\\
168.01	0.01\\
169.01	0.01\\
170.01	0.01\\
171.01	0.01\\
172.01	0.01\\
173.01	0.01\\
174.01	0.01\\
175.01	0.01\\
176.01	0.01\\
177.01	0.01\\
178.01	0.01\\
179.01	0.01\\
180.01	0.01\\
181.01	0.01\\
182.01	0.01\\
183.01	0.01\\
184.01	0.01\\
185.01	0.01\\
186.01	0.01\\
187.01	0.01\\
188.01	0.01\\
189.01	0.01\\
190.01	0.01\\
191.01	0.01\\
192.01	0.01\\
193.01	0.01\\
194.01	0.01\\
195.01	0.01\\
196.01	0.01\\
197.01	0.01\\
198.01	0.01\\
199.01	0.01\\
200.01	0.01\\
201.01	0.01\\
202.01	0.01\\
203.01	0.01\\
204.01	0.01\\
205.01	0.01\\
206.01	0.01\\
207.01	0.01\\
208.01	0.01\\
209.01	0.01\\
210.01	0.01\\
211.01	0.01\\
212.01	0.01\\
213.01	0.01\\
214.01	0.01\\
215.01	0.01\\
216.01	0.01\\
217.01	0.01\\
218.01	0.01\\
219.01	0.01\\
220.01	0.01\\
221.01	0.01\\
222.01	0.01\\
223.01	0.01\\
224.01	0.01\\
225.01	0.01\\
226.01	0.01\\
227.01	0.01\\
228.01	0.01\\
229.01	0.01\\
230.01	0.01\\
231.01	0.01\\
232.01	0.01\\
233.01	0.01\\
234.01	0.01\\
235.01	0.01\\
236.01	0.01\\
237.01	0.01\\
238.01	0.01\\
239.01	0.01\\
240.01	0.01\\
241.01	0.01\\
242.01	0.01\\
243.01	0.01\\
244.01	0.01\\
245.01	0.01\\
246.01	0.01\\
247.01	0.01\\
248.01	0.01\\
249.01	0.01\\
250.01	0.01\\
251.01	0.01\\
252.01	0.01\\
253.01	0.01\\
254.01	0.01\\
255.01	0.01\\
256.01	0.01\\
257.01	0.01\\
258.01	0.01\\
259.01	0.01\\
260.01	0.01\\
261.01	0.01\\
262.01	0.01\\
263.01	0.01\\
264.01	0.01\\
265.01	0.01\\
266.01	0.01\\
267.01	0.01\\
268.01	0.01\\
269.01	0.01\\
270.01	0.01\\
271.01	0.01\\
272.01	0.01\\
273.01	0.01\\
274.01	0.01\\
275.01	0.01\\
276.01	0.01\\
277.01	0.01\\
278.01	0.01\\
279.01	0.01\\
280.01	0.01\\
281.01	0.01\\
282.01	0.01\\
283.01	0.01\\
284.01	0.01\\
285.01	0.01\\
286.01	0.01\\
287.01	0.01\\
288.01	0.01\\
289.01	0.01\\
290.01	0.01\\
291.01	0.01\\
292.01	0.01\\
293.01	0.01\\
294.01	0.01\\
295.01	0.01\\
296.01	0.01\\
297.01	0.01\\
298.01	0.01\\
299.01	0.01\\
300.01	0.01\\
301.01	0.01\\
302.01	0.01\\
303.01	0.01\\
304.01	0.01\\
305.01	0.01\\
306.01	0.01\\
307.01	0.01\\
308.01	0.01\\
309.01	0.01\\
310.01	0.01\\
311.01	0.01\\
312.01	0.01\\
313.01	0.01\\
314.01	0.01\\
315.01	0.01\\
316.01	0.01\\
317.01	0.01\\
318.01	0.01\\
319.01	0.01\\
320.01	0.01\\
321.01	0.01\\
322.01	0.01\\
323.01	0.01\\
324.01	0.01\\
325.01	0.01\\
326.01	0.01\\
327.01	0.01\\
328.01	0.01\\
329.01	0.01\\
330.01	0.01\\
331.01	0.01\\
332.01	0.01\\
333.01	0.01\\
334.01	0.01\\
335.01	0.01\\
336.01	0.01\\
337.01	0.01\\
338.01	0.01\\
339.01	0.01\\
340.01	0.01\\
341.01	0.01\\
342.01	0.01\\
343.01	0.01\\
344.01	0.01\\
345.01	0.01\\
346.01	0.01\\
347.01	0.01\\
348.01	0.01\\
349.01	0.01\\
350.01	0.01\\
351.01	0.01\\
352.01	0.01\\
353.01	0.01\\
354.01	0.01\\
355.01	0.01\\
356.01	0.01\\
357.01	0.01\\
358.01	0.01\\
359.01	0.01\\
360.01	0.01\\
361.01	0.01\\
362.01	0.01\\
363.01	0.01\\
364.01	0.01\\
365.01	0.01\\
366.01	0.01\\
367.01	0.01\\
368.01	0.01\\
369.01	0.01\\
370.01	0.01\\
371.01	0.01\\
372.01	0.01\\
373.01	0.01\\
374.01	0.01\\
375.01	0.01\\
376.01	0.01\\
377.01	0.01\\
378.01	0.01\\
379.01	0.01\\
380.01	0.01\\
381.01	0.01\\
382.01	0.01\\
383.01	0.01\\
384.01	0.01\\
385.01	0.01\\
386.01	0.01\\
387.01	0.01\\
388.01	0.01\\
389.01	0.01\\
390.01	0.01\\
391.01	0.01\\
392.01	0.01\\
393.01	0.01\\
394.01	0.01\\
395.01	0.01\\
396.01	0.01\\
397.01	0.01\\
398.01	0.01\\
399.01	0.01\\
400.01	0.01\\
401.01	0.01\\
402.01	0.01\\
403.01	0.01\\
404.01	0.01\\
405.01	0.01\\
406.01	0.01\\
407.01	0.01\\
408.01	0.01\\
409.01	0.01\\
410.01	0.01\\
411.01	0.01\\
412.01	0.01\\
413.01	0.01\\
414.01	0.01\\
415.01	0.01\\
416.01	0.01\\
417.01	0.01\\
418.01	0.01\\
419.01	0.01\\
420.01	0.01\\
421.01	0.01\\
422.01	0.01\\
423.01	0.01\\
424.01	0.01\\
425.01	0.01\\
426.01	0.01\\
427.01	0.01\\
428.01	0.01\\
429.01	0.01\\
430.01	0.01\\
431.01	0.01\\
432.01	0.01\\
433.01	0.01\\
434.01	0.01\\
435.01	0.01\\
436.01	0.01\\
437.01	0.01\\
438.01	0.01\\
439.01	0.01\\
440.01	0.01\\
441.01	0.01\\
442.01	0.01\\
443.01	0.01\\
444.01	0.01\\
445.01	0.01\\
446.01	0.01\\
447.01	0.01\\
448.01	0.01\\
449.01	0.01\\
450.01	0.01\\
451.01	0.01\\
452.01	0.01\\
453.01	0.01\\
454.01	0.01\\
455.01	0.01\\
456.01	0.01\\
457.01	0.01\\
458.01	0.01\\
459.01	0.01\\
460.01	0.01\\
461.01	0.01\\
462.01	0.01\\
463.01	0.01\\
464.01	0.01\\
465.01	0.01\\
466.01	0.01\\
467.01	0.01\\
468.01	0.01\\
469.01	0.01\\
470.01	0.01\\
471.01	0.01\\
472.01	0.01\\
473.01	0.01\\
474.01	0.01\\
475.01	0.01\\
476.01	0.01\\
477.01	0.01\\
478.01	0.01\\
479.01	0.01\\
480.01	0.01\\
481.01	0.01\\
482.01	0.01\\
483.01	0.01\\
484.01	0.01\\
485.01	0.01\\
486.01	0.01\\
487.01	0.01\\
488.01	0.01\\
489.01	0.01\\
490.01	0.01\\
491.01	0.01\\
492.01	0.01\\
493.01	0.01\\
494.01	0.01\\
495.01	0.01\\
496.01	0.01\\
497.01	0.01\\
498.01	0.01\\
499.01	0.01\\
500.01	0.01\\
501.01	0.01\\
502.01	0.01\\
503.01	0.01\\
504.01	0.01\\
505.01	0.01\\
506.01	0.01\\
507.01	0.01\\
508.01	0.01\\
509.01	0.01\\
510.01	0.01\\
511.01	0.01\\
512.01	0.01\\
513.01	0.01\\
514.01	0.01\\
515.01	0.01\\
516.01	0.01\\
517.01	0.01\\
518.01	0.01\\
519.01	0.01\\
520.01	0.01\\
521.01	0.01\\
522.01	0.01\\
523.01	0.01\\
524.01	0.01\\
525.01	0.01\\
526.01	0.01\\
527.01	0.01\\
528.01	0.01\\
529.01	0.01\\
530.01	0.01\\
531.01	0.01\\
532.01	0.01\\
533.01	0.01\\
534.01	0.01\\
535.01	0.01\\
536.01	0.01\\
537.01	0.01\\
538.01	0.01\\
539.01	0.01\\
540.01	0.01\\
541.01	0.01\\
542.01	0.01\\
543.01	0.01\\
544.01	0.01\\
545.01	0.01\\
546.01	0.01\\
547.01	0.01\\
548.01	0.01\\
549.01	0.01\\
550.01	0.01\\
551.01	0.01\\
552.01	0.01\\
553.01	0.01\\
554.01	0.01\\
555.01	0.01\\
556.01	0.01\\
557.01	0.01\\
558.01	0.01\\
559.01	0.01\\
560.01	0.01\\
561.01	0.01\\
562.01	0.01\\
563.01	0.01\\
564.01	0.01\\
565.01	0.01\\
566.01	0.01\\
567.01	0.01\\
568.01	0.01\\
569.01	0.01\\
570.01	0.01\\
571.01	0.01\\
572.01	0.01\\
573.01	0.01\\
574.01	0.01\\
575.01	0.01\\
576.01	0.01\\
577.01	0.01\\
578.01	0.01\\
579.01	0.01\\
580.01	0.01\\
581.01	0.01\\
582.01	0.01\\
583.01	0.01\\
584.01	0.01\\
585.01	0.01\\
586.01	0.01\\
587.01	0.01\\
588.01	0.01\\
589.01	0.01\\
590.01	0.01\\
591.01	0.01\\
592.01	0.01\\
593.01	0.01\\
594.01	0.01\\
595.01	0.01\\
596.01	0.01\\
597.01	0.01\\
598.01	0.01\\
599.01	0.00624186909405409\\
599.02	0.00620414441242163\\
599.03	0.00616605275622737\\
599.04	0.00612759052101571\\
599.05	0.00608875406692037\\
599.06	0.00604953971831644\\
599.07	0.00600994376346918\\
599.08	0.0059699624541792\\
599.09	0.00592959200542426\\
599.1	0.00588882859499753\\
599.11	0.00584766836314222\\
599.12	0.00580610741218277\\
599.13	0.00576414180615227\\
599.14	0.00572176757041629\\
599.15	0.00567898069129309\\
599.16	0.00563577711566987\\
599.17	0.00559215275061548\\
599.18	0.00554810346298928\\
599.19	0.00550362507904603\\
599.2	0.00545871338403702\\
599.21	0.0054133641218073\\
599.22	0.00536757299438882\\
599.23	0.00532133566158968\\
599.24	0.00527464774057935\\
599.25	0.00522750480546976\\
599.26	0.0051799023868923\\
599.27	0.00513183597157068\\
599.28	0.00508330100188952\\
599.29	0.00503429287545886\\
599.3	0.00498480694467418\\
599.31	0.00493483851627229\\
599.32	0.00488438285088277\\
599.33	0.00483343516257492\\
599.34	0.00478199061840048\\
599.35	0.00473004433793162\\
599.36	0.00467759139279456\\
599.37	0.00462462680619851\\
599.38	0.00457114555246004\\
599.39	0.00451714256049737\\
599.4	0.00446261272376359\\
599.41	0.00440755088572621\\
599.42	0.00435195183937783\\
599.43	0.00429581032674213\\
599.44	0.00423912103837488\\
599.45	0.00418187861286022\\
599.46	0.00412407763630194\\
599.47	0.00406571264180972\\
599.48	0.00400677810898045\\
599.49	0.00394726846337443\\
599.5	0.00388717807598639\\
599.51	0.00382650126271144\\
599.52	0.00376523228380568\\
599.53	0.00370336534334162\\
599.54	0.00364089458865822\\
599.55	0.00357781410980552\\
599.56	0.00351411793898394\\
599.57	0.00344980004997801\\
599.58	0.00338485435758449\\
599.59	0.00331927471703512\\
599.6	0.00325305492341342\\
599.61	0.00318618871106605\\
599.62	0.00311866975300824\\
599.63	0.00305049166032345\\
599.64	0.00298164798155718\\
599.65	0.00291213220210483\\
599.66	0.00284193774359354\\
599.67	0.002771057963258\\
599.68	0.00269948615331015\\
599.69	0.00262721554030276\\
599.7	0.00255423928448666\\
599.71	0.00248055047916181\\
599.72	0.00240614215002193\\
599.73	0.00233100725449273\\
599.74	0.00225513868106374\\
599.75	0.00217852924861345\\
599.76	0.002101171705728\\
599.77	0.00202305873001311\\
599.78	0.00194418292739928\\
599.79	0.00186453683144024\\
599.8	0.00178411290260446\\
599.81	0.0017029035275598\\
599.82	0.00162090101845109\\
599.83	0.00153809761217073\\
599.84	0.001454485469622\\
599.85	0.00137005667497534\\
599.86	0.00128480323491721\\
599.87	0.00119871707789167\\
599.88	0.00111179005333449\\
599.89	0.00102401393089982\\
599.9	0.000935380399679274\\
599.91	0.000845881067413286\\
599.92	0.000755507459694882\\
599.93	0.000664251019165561\\
599.94	0.000572103104703356\\
599.95	0.00047905499060291\\
599.96	0.000385097865747558\\
599.97	0.000290222832773275\\
599.98	0.000194420907224489\\
599.99	9.76830167015649e-05\\
600	0\\
};
\addplot [color=black!20!mycolor21,solid,forget plot]
  table[row sep=crcr]{%
0.01	0.01\\
1.01	0.01\\
2.01	0.01\\
3.01	0.01\\
4.01	0.01\\
5.01	0.01\\
6.01	0.01\\
7.01	0.01\\
8.01	0.01\\
9.01	0.01\\
10.01	0.01\\
11.01	0.01\\
12.01	0.01\\
13.01	0.01\\
14.01	0.01\\
15.01	0.01\\
16.01	0.01\\
17.01	0.01\\
18.01	0.01\\
19.01	0.01\\
20.01	0.01\\
21.01	0.01\\
22.01	0.01\\
23.01	0.01\\
24.01	0.01\\
25.01	0.01\\
26.01	0.01\\
27.01	0.01\\
28.01	0.01\\
29.01	0.01\\
30.01	0.01\\
31.01	0.01\\
32.01	0.01\\
33.01	0.01\\
34.01	0.01\\
35.01	0.01\\
36.01	0.01\\
37.01	0.01\\
38.01	0.01\\
39.01	0.01\\
40.01	0.01\\
41.01	0.01\\
42.01	0.01\\
43.01	0.01\\
44.01	0.01\\
45.01	0.01\\
46.01	0.01\\
47.01	0.01\\
48.01	0.01\\
49.01	0.01\\
50.01	0.01\\
51.01	0.01\\
52.01	0.01\\
53.01	0.01\\
54.01	0.01\\
55.01	0.01\\
56.01	0.01\\
57.01	0.01\\
58.01	0.01\\
59.01	0.01\\
60.01	0.01\\
61.01	0.01\\
62.01	0.01\\
63.01	0.01\\
64.01	0.01\\
65.01	0.01\\
66.01	0.01\\
67.01	0.01\\
68.01	0.01\\
69.01	0.01\\
70.01	0.01\\
71.01	0.01\\
72.01	0.01\\
73.01	0.01\\
74.01	0.01\\
75.01	0.01\\
76.01	0.01\\
77.01	0.01\\
78.01	0.01\\
79.01	0.01\\
80.01	0.01\\
81.01	0.01\\
82.01	0.01\\
83.01	0.01\\
84.01	0.01\\
85.01	0.01\\
86.01	0.01\\
87.01	0.01\\
88.01	0.01\\
89.01	0.01\\
90.01	0.01\\
91.01	0.01\\
92.01	0.01\\
93.01	0.01\\
94.01	0.01\\
95.01	0.01\\
96.01	0.01\\
97.01	0.01\\
98.01	0.01\\
99.01	0.01\\
100.01	0.01\\
101.01	0.01\\
102.01	0.01\\
103.01	0.01\\
104.01	0.01\\
105.01	0.01\\
106.01	0.01\\
107.01	0.01\\
108.01	0.01\\
109.01	0.01\\
110.01	0.01\\
111.01	0.01\\
112.01	0.01\\
113.01	0.01\\
114.01	0.01\\
115.01	0.01\\
116.01	0.01\\
117.01	0.01\\
118.01	0.01\\
119.01	0.01\\
120.01	0.01\\
121.01	0.01\\
122.01	0.01\\
123.01	0.01\\
124.01	0.01\\
125.01	0.01\\
126.01	0.01\\
127.01	0.01\\
128.01	0.01\\
129.01	0.01\\
130.01	0.01\\
131.01	0.01\\
132.01	0.01\\
133.01	0.01\\
134.01	0.01\\
135.01	0.01\\
136.01	0.01\\
137.01	0.01\\
138.01	0.01\\
139.01	0.01\\
140.01	0.01\\
141.01	0.01\\
142.01	0.01\\
143.01	0.01\\
144.01	0.01\\
145.01	0.01\\
146.01	0.01\\
147.01	0.01\\
148.01	0.01\\
149.01	0.01\\
150.01	0.01\\
151.01	0.01\\
152.01	0.01\\
153.01	0.01\\
154.01	0.01\\
155.01	0.01\\
156.01	0.01\\
157.01	0.01\\
158.01	0.01\\
159.01	0.01\\
160.01	0.01\\
161.01	0.01\\
162.01	0.01\\
163.01	0.01\\
164.01	0.01\\
165.01	0.01\\
166.01	0.01\\
167.01	0.01\\
168.01	0.01\\
169.01	0.01\\
170.01	0.01\\
171.01	0.01\\
172.01	0.01\\
173.01	0.01\\
174.01	0.01\\
175.01	0.01\\
176.01	0.01\\
177.01	0.01\\
178.01	0.01\\
179.01	0.01\\
180.01	0.01\\
181.01	0.01\\
182.01	0.01\\
183.01	0.01\\
184.01	0.01\\
185.01	0.01\\
186.01	0.01\\
187.01	0.01\\
188.01	0.01\\
189.01	0.01\\
190.01	0.01\\
191.01	0.01\\
192.01	0.01\\
193.01	0.01\\
194.01	0.01\\
195.01	0.01\\
196.01	0.01\\
197.01	0.01\\
198.01	0.01\\
199.01	0.01\\
200.01	0.01\\
201.01	0.01\\
202.01	0.01\\
203.01	0.01\\
204.01	0.01\\
205.01	0.01\\
206.01	0.01\\
207.01	0.01\\
208.01	0.01\\
209.01	0.01\\
210.01	0.01\\
211.01	0.01\\
212.01	0.01\\
213.01	0.01\\
214.01	0.01\\
215.01	0.01\\
216.01	0.01\\
217.01	0.01\\
218.01	0.01\\
219.01	0.01\\
220.01	0.01\\
221.01	0.01\\
222.01	0.01\\
223.01	0.01\\
224.01	0.01\\
225.01	0.01\\
226.01	0.01\\
227.01	0.01\\
228.01	0.01\\
229.01	0.01\\
230.01	0.01\\
231.01	0.01\\
232.01	0.01\\
233.01	0.01\\
234.01	0.01\\
235.01	0.01\\
236.01	0.01\\
237.01	0.01\\
238.01	0.01\\
239.01	0.01\\
240.01	0.01\\
241.01	0.01\\
242.01	0.01\\
243.01	0.01\\
244.01	0.01\\
245.01	0.01\\
246.01	0.01\\
247.01	0.01\\
248.01	0.01\\
249.01	0.01\\
250.01	0.01\\
251.01	0.01\\
252.01	0.01\\
253.01	0.01\\
254.01	0.01\\
255.01	0.01\\
256.01	0.01\\
257.01	0.01\\
258.01	0.01\\
259.01	0.01\\
260.01	0.01\\
261.01	0.01\\
262.01	0.01\\
263.01	0.01\\
264.01	0.01\\
265.01	0.01\\
266.01	0.01\\
267.01	0.01\\
268.01	0.01\\
269.01	0.01\\
270.01	0.01\\
271.01	0.01\\
272.01	0.01\\
273.01	0.01\\
274.01	0.01\\
275.01	0.01\\
276.01	0.01\\
277.01	0.01\\
278.01	0.01\\
279.01	0.01\\
280.01	0.01\\
281.01	0.01\\
282.01	0.01\\
283.01	0.01\\
284.01	0.01\\
285.01	0.01\\
286.01	0.01\\
287.01	0.01\\
288.01	0.01\\
289.01	0.01\\
290.01	0.01\\
291.01	0.01\\
292.01	0.01\\
293.01	0.01\\
294.01	0.01\\
295.01	0.01\\
296.01	0.01\\
297.01	0.01\\
298.01	0.01\\
299.01	0.01\\
300.01	0.01\\
301.01	0.01\\
302.01	0.01\\
303.01	0.01\\
304.01	0.01\\
305.01	0.01\\
306.01	0.01\\
307.01	0.01\\
308.01	0.01\\
309.01	0.01\\
310.01	0.01\\
311.01	0.01\\
312.01	0.01\\
313.01	0.01\\
314.01	0.01\\
315.01	0.01\\
316.01	0.01\\
317.01	0.01\\
318.01	0.01\\
319.01	0.01\\
320.01	0.01\\
321.01	0.01\\
322.01	0.01\\
323.01	0.01\\
324.01	0.01\\
325.01	0.01\\
326.01	0.01\\
327.01	0.01\\
328.01	0.01\\
329.01	0.01\\
330.01	0.01\\
331.01	0.01\\
332.01	0.01\\
333.01	0.01\\
334.01	0.01\\
335.01	0.01\\
336.01	0.01\\
337.01	0.01\\
338.01	0.01\\
339.01	0.01\\
340.01	0.01\\
341.01	0.01\\
342.01	0.01\\
343.01	0.01\\
344.01	0.01\\
345.01	0.01\\
346.01	0.01\\
347.01	0.01\\
348.01	0.01\\
349.01	0.01\\
350.01	0.01\\
351.01	0.01\\
352.01	0.01\\
353.01	0.01\\
354.01	0.01\\
355.01	0.01\\
356.01	0.01\\
357.01	0.01\\
358.01	0.01\\
359.01	0.01\\
360.01	0.01\\
361.01	0.01\\
362.01	0.01\\
363.01	0.01\\
364.01	0.01\\
365.01	0.01\\
366.01	0.01\\
367.01	0.01\\
368.01	0.01\\
369.01	0.01\\
370.01	0.01\\
371.01	0.01\\
372.01	0.01\\
373.01	0.01\\
374.01	0.01\\
375.01	0.01\\
376.01	0.01\\
377.01	0.01\\
378.01	0.01\\
379.01	0.01\\
380.01	0.01\\
381.01	0.01\\
382.01	0.01\\
383.01	0.01\\
384.01	0.01\\
385.01	0.01\\
386.01	0.01\\
387.01	0.01\\
388.01	0.01\\
389.01	0.01\\
390.01	0.01\\
391.01	0.01\\
392.01	0.01\\
393.01	0.01\\
394.01	0.01\\
395.01	0.01\\
396.01	0.01\\
397.01	0.01\\
398.01	0.01\\
399.01	0.01\\
400.01	0.01\\
401.01	0.01\\
402.01	0.01\\
403.01	0.01\\
404.01	0.01\\
405.01	0.01\\
406.01	0.01\\
407.01	0.01\\
408.01	0.01\\
409.01	0.01\\
410.01	0.01\\
411.01	0.01\\
412.01	0.01\\
413.01	0.01\\
414.01	0.01\\
415.01	0.01\\
416.01	0.01\\
417.01	0.01\\
418.01	0.01\\
419.01	0.01\\
420.01	0.01\\
421.01	0.01\\
422.01	0.01\\
423.01	0.01\\
424.01	0.01\\
425.01	0.01\\
426.01	0.01\\
427.01	0.01\\
428.01	0.01\\
429.01	0.01\\
430.01	0.01\\
431.01	0.01\\
432.01	0.01\\
433.01	0.01\\
434.01	0.01\\
435.01	0.01\\
436.01	0.01\\
437.01	0.01\\
438.01	0.01\\
439.01	0.01\\
440.01	0.01\\
441.01	0.01\\
442.01	0.01\\
443.01	0.01\\
444.01	0.01\\
445.01	0.01\\
446.01	0.01\\
447.01	0.01\\
448.01	0.01\\
449.01	0.01\\
450.01	0.01\\
451.01	0.01\\
452.01	0.01\\
453.01	0.01\\
454.01	0.01\\
455.01	0.01\\
456.01	0.01\\
457.01	0.01\\
458.01	0.01\\
459.01	0.01\\
460.01	0.01\\
461.01	0.01\\
462.01	0.01\\
463.01	0.01\\
464.01	0.01\\
465.01	0.01\\
466.01	0.01\\
467.01	0.01\\
468.01	0.01\\
469.01	0.01\\
470.01	0.01\\
471.01	0.01\\
472.01	0.01\\
473.01	0.01\\
474.01	0.01\\
475.01	0.01\\
476.01	0.01\\
477.01	0.01\\
478.01	0.01\\
479.01	0.01\\
480.01	0.01\\
481.01	0.01\\
482.01	0.01\\
483.01	0.01\\
484.01	0.01\\
485.01	0.01\\
486.01	0.01\\
487.01	0.01\\
488.01	0.01\\
489.01	0.01\\
490.01	0.01\\
491.01	0.01\\
492.01	0.01\\
493.01	0.01\\
494.01	0.01\\
495.01	0.01\\
496.01	0.01\\
497.01	0.01\\
498.01	0.01\\
499.01	0.01\\
500.01	0.01\\
501.01	0.01\\
502.01	0.01\\
503.01	0.01\\
504.01	0.01\\
505.01	0.01\\
506.01	0.01\\
507.01	0.01\\
508.01	0.01\\
509.01	0.01\\
510.01	0.01\\
511.01	0.01\\
512.01	0.01\\
513.01	0.01\\
514.01	0.01\\
515.01	0.01\\
516.01	0.01\\
517.01	0.01\\
518.01	0.01\\
519.01	0.01\\
520.01	0.01\\
521.01	0.01\\
522.01	0.01\\
523.01	0.01\\
524.01	0.01\\
525.01	0.01\\
526.01	0.01\\
527.01	0.01\\
528.01	0.01\\
529.01	0.01\\
530.01	0.01\\
531.01	0.01\\
532.01	0.01\\
533.01	0.01\\
534.01	0.01\\
535.01	0.01\\
536.01	0.01\\
537.01	0.01\\
538.01	0.01\\
539.01	0.01\\
540.01	0.01\\
541.01	0.01\\
542.01	0.01\\
543.01	0.01\\
544.01	0.01\\
545.01	0.01\\
546.01	0.01\\
547.01	0.01\\
548.01	0.01\\
549.01	0.01\\
550.01	0.01\\
551.01	0.01\\
552.01	0.01\\
553.01	0.01\\
554.01	0.01\\
555.01	0.01\\
556.01	0.01\\
557.01	0.01\\
558.01	0.01\\
559.01	0.01\\
560.01	0.01\\
561.01	0.01\\
562.01	0.01\\
563.01	0.01\\
564.01	0.01\\
565.01	0.01\\
566.01	0.01\\
567.01	0.01\\
568.01	0.01\\
569.01	0.01\\
570.01	0.01\\
571.01	0.01\\
572.01	0.01\\
573.01	0.01\\
574.01	0.01\\
575.01	0.01\\
576.01	0.01\\
577.01	0.01\\
578.01	0.01\\
579.01	0.01\\
580.01	0.01\\
581.01	0.01\\
582.01	0.01\\
583.01	0.01\\
584.01	0.01\\
585.01	0.01\\
586.01	0.01\\
587.01	0.01\\
588.01	0.01\\
589.01	0.01\\
590.01	0.01\\
591.01	0.01\\
592.01	0.01\\
593.01	0.01\\
594.01	0.01\\
595.01	0.01\\
596.01	0.01\\
597.01	0.01\\
598.01	0.00929975107060225\\
599.01	0.00624186909405412\\
599.02	0.00620414441242164\\
599.03	0.00616605275622737\\
599.04	0.00612759052101571\\
599.05	0.00608875406692035\\
599.06	0.00604953971831644\\
599.07	0.00600994376346918\\
599.08	0.0059699624541792\\
599.09	0.00592959200542426\\
599.1	0.00588882859499752\\
599.11	0.00584766836314222\\
599.12	0.00580610741218276\\
599.13	0.00576414180615225\\
599.14	0.00572176757041629\\
599.15	0.00567898069129307\\
599.16	0.00563577711566984\\
599.17	0.00559215275061547\\
599.18	0.00554810346298928\\
599.19	0.00550362507904604\\
599.2	0.00545871338403705\\
599.21	0.00541336412180733\\
599.22	0.00536757299438884\\
599.23	0.0053213356615897\\
599.24	0.00527464774057938\\
599.25	0.00522750480546979\\
599.26	0.00517990238689233\\
599.27	0.00513183597157071\\
599.28	0.00508330100188955\\
599.29	0.00503429287545888\\
599.3	0.0049848069446742\\
599.31	0.00493483851627232\\
599.32	0.00488438285088278\\
599.33	0.00483343516257495\\
599.34	0.00478199061840049\\
599.35	0.00473004433793162\\
599.36	0.00467759139279456\\
599.37	0.00462462680619851\\
599.38	0.00457114555246004\\
599.39	0.00451714256049737\\
599.4	0.00446261272376359\\
599.41	0.00440755088572621\\
599.42	0.00435195183937783\\
599.43	0.00429581032674213\\
599.44	0.00423912103837487\\
599.45	0.00418187861286023\\
599.46	0.00412407763630194\\
599.47	0.00406571264180971\\
599.48	0.00400677810898045\\
599.49	0.00394726846337442\\
599.5	0.00388717807598638\\
599.51	0.00382650126271144\\
599.52	0.00376523228380569\\
599.53	0.00370336534334163\\
599.54	0.00364089458865821\\
599.55	0.00357781410980551\\
599.56	0.00351411793898394\\
599.57	0.003449800049978\\
599.58	0.00338485435758449\\
599.59	0.00331927471703512\\
599.6	0.00325305492341342\\
599.61	0.00318618871106605\\
599.62	0.00311866975300823\\
599.63	0.00305049166032344\\
599.64	0.00298164798155717\\
599.65	0.00291213220210483\\
599.66	0.00284193774359354\\
599.67	0.00277105796325799\\
599.68	0.00269948615331015\\
599.69	0.00262721554030275\\
599.7	0.00255423928448665\\
599.71	0.0024805504791618\\
599.72	0.00240614215002191\\
599.73	0.00233100725449272\\
599.74	0.00225513868106372\\
599.75	0.00217852924861344\\
599.76	0.00210117170572799\\
599.77	0.0020230587300131\\
599.78	0.00194418292739928\\
599.79	0.00186453683144023\\
599.8	0.00178411290260445\\
599.81	0.00170290352755979\\
599.82	0.00162090101845109\\
599.83	0.00153809761217072\\
599.84	0.001454485469622\\
599.85	0.00137005667497534\\
599.86	0.00128480323491721\\
599.87	0.00119871707789166\\
599.88	0.00111179005333448\\
599.89	0.00102401393089982\\
599.9	0.000935380399679275\\
599.91	0.000845881067413288\\
599.92	0.000755507459694884\\
599.93	0.000664251019165561\\
599.94	0.00057210310470336\\
599.95	0.00047905499060291\\
599.96	0.000385097865747556\\
599.97	0.000290222832773275\\
599.98	0.000194420907224489\\
599.99	9.76830167015649e-05\\
600	0\\
};
\addplot [color=black!50!mycolor20,solid,forget plot]
  table[row sep=crcr]{%
0.01	0.01\\
1.01	0.01\\
2.01	0.01\\
3.01	0.01\\
4.01	0.01\\
5.01	0.01\\
6.01	0.01\\
7.01	0.01\\
8.01	0.01\\
9.01	0.01\\
10.01	0.01\\
11.01	0.01\\
12.01	0.01\\
13.01	0.01\\
14.01	0.01\\
15.01	0.01\\
16.01	0.01\\
17.01	0.01\\
18.01	0.01\\
19.01	0.01\\
20.01	0.01\\
21.01	0.01\\
22.01	0.01\\
23.01	0.01\\
24.01	0.01\\
25.01	0.01\\
26.01	0.01\\
27.01	0.01\\
28.01	0.01\\
29.01	0.01\\
30.01	0.01\\
31.01	0.01\\
32.01	0.01\\
33.01	0.01\\
34.01	0.01\\
35.01	0.01\\
36.01	0.01\\
37.01	0.01\\
38.01	0.01\\
39.01	0.01\\
40.01	0.01\\
41.01	0.01\\
42.01	0.01\\
43.01	0.01\\
44.01	0.01\\
45.01	0.01\\
46.01	0.01\\
47.01	0.01\\
48.01	0.01\\
49.01	0.01\\
50.01	0.01\\
51.01	0.01\\
52.01	0.01\\
53.01	0.01\\
54.01	0.01\\
55.01	0.01\\
56.01	0.01\\
57.01	0.01\\
58.01	0.01\\
59.01	0.01\\
60.01	0.01\\
61.01	0.01\\
62.01	0.01\\
63.01	0.01\\
64.01	0.01\\
65.01	0.01\\
66.01	0.01\\
67.01	0.01\\
68.01	0.01\\
69.01	0.01\\
70.01	0.01\\
71.01	0.01\\
72.01	0.01\\
73.01	0.01\\
74.01	0.01\\
75.01	0.01\\
76.01	0.01\\
77.01	0.01\\
78.01	0.01\\
79.01	0.01\\
80.01	0.01\\
81.01	0.01\\
82.01	0.01\\
83.01	0.01\\
84.01	0.01\\
85.01	0.01\\
86.01	0.01\\
87.01	0.01\\
88.01	0.01\\
89.01	0.01\\
90.01	0.01\\
91.01	0.01\\
92.01	0.01\\
93.01	0.01\\
94.01	0.01\\
95.01	0.01\\
96.01	0.01\\
97.01	0.01\\
98.01	0.01\\
99.01	0.01\\
100.01	0.01\\
101.01	0.01\\
102.01	0.01\\
103.01	0.01\\
104.01	0.01\\
105.01	0.01\\
106.01	0.01\\
107.01	0.01\\
108.01	0.01\\
109.01	0.01\\
110.01	0.01\\
111.01	0.01\\
112.01	0.01\\
113.01	0.01\\
114.01	0.01\\
115.01	0.01\\
116.01	0.01\\
117.01	0.01\\
118.01	0.01\\
119.01	0.01\\
120.01	0.01\\
121.01	0.01\\
122.01	0.01\\
123.01	0.01\\
124.01	0.01\\
125.01	0.01\\
126.01	0.01\\
127.01	0.01\\
128.01	0.01\\
129.01	0.01\\
130.01	0.01\\
131.01	0.01\\
132.01	0.01\\
133.01	0.01\\
134.01	0.01\\
135.01	0.01\\
136.01	0.01\\
137.01	0.01\\
138.01	0.01\\
139.01	0.01\\
140.01	0.01\\
141.01	0.01\\
142.01	0.01\\
143.01	0.01\\
144.01	0.01\\
145.01	0.01\\
146.01	0.01\\
147.01	0.01\\
148.01	0.01\\
149.01	0.01\\
150.01	0.01\\
151.01	0.01\\
152.01	0.01\\
153.01	0.01\\
154.01	0.01\\
155.01	0.01\\
156.01	0.01\\
157.01	0.01\\
158.01	0.01\\
159.01	0.01\\
160.01	0.01\\
161.01	0.01\\
162.01	0.01\\
163.01	0.01\\
164.01	0.01\\
165.01	0.01\\
166.01	0.01\\
167.01	0.01\\
168.01	0.01\\
169.01	0.01\\
170.01	0.01\\
171.01	0.01\\
172.01	0.01\\
173.01	0.01\\
174.01	0.01\\
175.01	0.01\\
176.01	0.01\\
177.01	0.01\\
178.01	0.01\\
179.01	0.01\\
180.01	0.01\\
181.01	0.01\\
182.01	0.01\\
183.01	0.01\\
184.01	0.01\\
185.01	0.01\\
186.01	0.01\\
187.01	0.01\\
188.01	0.01\\
189.01	0.01\\
190.01	0.01\\
191.01	0.01\\
192.01	0.01\\
193.01	0.01\\
194.01	0.01\\
195.01	0.01\\
196.01	0.01\\
197.01	0.01\\
198.01	0.01\\
199.01	0.01\\
200.01	0.01\\
201.01	0.01\\
202.01	0.01\\
203.01	0.01\\
204.01	0.01\\
205.01	0.01\\
206.01	0.01\\
207.01	0.01\\
208.01	0.01\\
209.01	0.01\\
210.01	0.01\\
211.01	0.01\\
212.01	0.01\\
213.01	0.01\\
214.01	0.01\\
215.01	0.01\\
216.01	0.01\\
217.01	0.01\\
218.01	0.01\\
219.01	0.01\\
220.01	0.01\\
221.01	0.01\\
222.01	0.01\\
223.01	0.01\\
224.01	0.01\\
225.01	0.01\\
226.01	0.01\\
227.01	0.01\\
228.01	0.01\\
229.01	0.01\\
230.01	0.01\\
231.01	0.01\\
232.01	0.01\\
233.01	0.01\\
234.01	0.01\\
235.01	0.01\\
236.01	0.01\\
237.01	0.01\\
238.01	0.01\\
239.01	0.01\\
240.01	0.01\\
241.01	0.01\\
242.01	0.01\\
243.01	0.01\\
244.01	0.01\\
245.01	0.01\\
246.01	0.01\\
247.01	0.01\\
248.01	0.01\\
249.01	0.01\\
250.01	0.01\\
251.01	0.01\\
252.01	0.01\\
253.01	0.01\\
254.01	0.01\\
255.01	0.01\\
256.01	0.01\\
257.01	0.01\\
258.01	0.01\\
259.01	0.01\\
260.01	0.01\\
261.01	0.01\\
262.01	0.01\\
263.01	0.01\\
264.01	0.01\\
265.01	0.01\\
266.01	0.01\\
267.01	0.01\\
268.01	0.01\\
269.01	0.01\\
270.01	0.01\\
271.01	0.01\\
272.01	0.01\\
273.01	0.01\\
274.01	0.01\\
275.01	0.01\\
276.01	0.01\\
277.01	0.01\\
278.01	0.01\\
279.01	0.01\\
280.01	0.01\\
281.01	0.01\\
282.01	0.01\\
283.01	0.01\\
284.01	0.01\\
285.01	0.01\\
286.01	0.01\\
287.01	0.01\\
288.01	0.01\\
289.01	0.01\\
290.01	0.01\\
291.01	0.01\\
292.01	0.01\\
293.01	0.01\\
294.01	0.01\\
295.01	0.01\\
296.01	0.01\\
297.01	0.01\\
298.01	0.01\\
299.01	0.01\\
300.01	0.01\\
301.01	0.01\\
302.01	0.01\\
303.01	0.01\\
304.01	0.01\\
305.01	0.01\\
306.01	0.01\\
307.01	0.01\\
308.01	0.01\\
309.01	0.01\\
310.01	0.01\\
311.01	0.01\\
312.01	0.01\\
313.01	0.01\\
314.01	0.01\\
315.01	0.01\\
316.01	0.01\\
317.01	0.01\\
318.01	0.01\\
319.01	0.01\\
320.01	0.01\\
321.01	0.01\\
322.01	0.01\\
323.01	0.01\\
324.01	0.01\\
325.01	0.01\\
326.01	0.01\\
327.01	0.01\\
328.01	0.01\\
329.01	0.01\\
330.01	0.01\\
331.01	0.01\\
332.01	0.01\\
333.01	0.01\\
334.01	0.01\\
335.01	0.01\\
336.01	0.01\\
337.01	0.01\\
338.01	0.01\\
339.01	0.01\\
340.01	0.01\\
341.01	0.01\\
342.01	0.01\\
343.01	0.01\\
344.01	0.01\\
345.01	0.01\\
346.01	0.01\\
347.01	0.01\\
348.01	0.01\\
349.01	0.01\\
350.01	0.01\\
351.01	0.01\\
352.01	0.01\\
353.01	0.01\\
354.01	0.01\\
355.01	0.01\\
356.01	0.01\\
357.01	0.01\\
358.01	0.01\\
359.01	0.01\\
360.01	0.01\\
361.01	0.01\\
362.01	0.01\\
363.01	0.01\\
364.01	0.01\\
365.01	0.01\\
366.01	0.01\\
367.01	0.01\\
368.01	0.01\\
369.01	0.01\\
370.01	0.01\\
371.01	0.01\\
372.01	0.01\\
373.01	0.01\\
374.01	0.01\\
375.01	0.01\\
376.01	0.01\\
377.01	0.01\\
378.01	0.01\\
379.01	0.01\\
380.01	0.01\\
381.01	0.01\\
382.01	0.01\\
383.01	0.01\\
384.01	0.01\\
385.01	0.01\\
386.01	0.01\\
387.01	0.01\\
388.01	0.01\\
389.01	0.01\\
390.01	0.01\\
391.01	0.01\\
392.01	0.01\\
393.01	0.01\\
394.01	0.01\\
395.01	0.01\\
396.01	0.01\\
397.01	0.01\\
398.01	0.01\\
399.01	0.01\\
400.01	0.01\\
401.01	0.01\\
402.01	0.01\\
403.01	0.01\\
404.01	0.01\\
405.01	0.01\\
406.01	0.01\\
407.01	0.01\\
408.01	0.01\\
409.01	0.01\\
410.01	0.01\\
411.01	0.01\\
412.01	0.01\\
413.01	0.01\\
414.01	0.01\\
415.01	0.01\\
416.01	0.01\\
417.01	0.01\\
418.01	0.01\\
419.01	0.01\\
420.01	0.01\\
421.01	0.01\\
422.01	0.01\\
423.01	0.01\\
424.01	0.01\\
425.01	0.01\\
426.01	0.01\\
427.01	0.01\\
428.01	0.01\\
429.01	0.01\\
430.01	0.01\\
431.01	0.01\\
432.01	0.01\\
433.01	0.01\\
434.01	0.01\\
435.01	0.01\\
436.01	0.01\\
437.01	0.01\\
438.01	0.01\\
439.01	0.01\\
440.01	0.01\\
441.01	0.01\\
442.01	0.01\\
443.01	0.01\\
444.01	0.01\\
445.01	0.01\\
446.01	0.01\\
447.01	0.01\\
448.01	0.01\\
449.01	0.01\\
450.01	0.01\\
451.01	0.01\\
452.01	0.01\\
453.01	0.01\\
454.01	0.01\\
455.01	0.01\\
456.01	0.01\\
457.01	0.01\\
458.01	0.01\\
459.01	0.01\\
460.01	0.01\\
461.01	0.01\\
462.01	0.01\\
463.01	0.01\\
464.01	0.01\\
465.01	0.01\\
466.01	0.01\\
467.01	0.01\\
468.01	0.01\\
469.01	0.01\\
470.01	0.01\\
471.01	0.01\\
472.01	0.01\\
473.01	0.01\\
474.01	0.01\\
475.01	0.01\\
476.01	0.01\\
477.01	0.01\\
478.01	0.01\\
479.01	0.01\\
480.01	0.01\\
481.01	0.01\\
482.01	0.01\\
483.01	0.01\\
484.01	0.01\\
485.01	0.01\\
486.01	0.01\\
487.01	0.01\\
488.01	0.01\\
489.01	0.01\\
490.01	0.01\\
491.01	0.01\\
492.01	0.01\\
493.01	0.01\\
494.01	0.01\\
495.01	0.01\\
496.01	0.01\\
497.01	0.01\\
498.01	0.01\\
499.01	0.01\\
500.01	0.01\\
501.01	0.01\\
502.01	0.01\\
503.01	0.01\\
504.01	0.01\\
505.01	0.01\\
506.01	0.01\\
507.01	0.01\\
508.01	0.01\\
509.01	0.01\\
510.01	0.01\\
511.01	0.01\\
512.01	0.01\\
513.01	0.01\\
514.01	0.01\\
515.01	0.01\\
516.01	0.01\\
517.01	0.01\\
518.01	0.01\\
519.01	0.01\\
520.01	0.01\\
521.01	0.01\\
522.01	0.01\\
523.01	0.01\\
524.01	0.01\\
525.01	0.01\\
526.01	0.01\\
527.01	0.01\\
528.01	0.01\\
529.01	0.01\\
530.01	0.01\\
531.01	0.01\\
532.01	0.01\\
533.01	0.01\\
534.01	0.01\\
535.01	0.01\\
536.01	0.01\\
537.01	0.01\\
538.01	0.01\\
539.01	0.01\\
540.01	0.01\\
541.01	0.01\\
542.01	0.01\\
543.01	0.01\\
544.01	0.01\\
545.01	0.01\\
546.01	0.01\\
547.01	0.01\\
548.01	0.01\\
549.01	0.01\\
550.01	0.01\\
551.01	0.01\\
552.01	0.01\\
553.01	0.01\\
554.01	0.01\\
555.01	0.01\\
556.01	0.01\\
557.01	0.01\\
558.01	0.01\\
559.01	0.01\\
560.01	0.01\\
561.01	0.01\\
562.01	0.01\\
563.01	0.01\\
564.01	0.01\\
565.01	0.01\\
566.01	0.01\\
567.01	0.01\\
568.01	0.01\\
569.01	0.01\\
570.01	0.01\\
571.01	0.01\\
572.01	0.01\\
573.01	0.01\\
574.01	0.01\\
575.01	0.01\\
576.01	0.01\\
577.01	0.01\\
578.01	0.01\\
579.01	0.01\\
580.01	0.01\\
581.01	0.01\\
582.01	0.01\\
583.01	0.01\\
584.01	0.01\\
585.01	0.01\\
586.01	0.01\\
587.01	0.01\\
588.01	0.01\\
589.01	0.01\\
590.01	0.01\\
591.01	0.01\\
592.01	0.01\\
593.01	0.01\\
594.01	0.01\\
595.01	0.01\\
596.01	0.01\\
597.01	0.01\\
598.01	0.00865254016279302\\
599.01	0.00624186909405409\\
599.02	0.00620414441242163\\
599.03	0.00616605275622736\\
599.04	0.00612759052101569\\
599.05	0.00608875406692034\\
599.06	0.00604953971831641\\
599.07	0.00600994376346915\\
599.08	0.00596996245417918\\
599.09	0.00592959200542425\\
599.1	0.00588882859499752\\
599.11	0.00584766836314221\\
599.12	0.00580610741218275\\
599.13	0.00576414180615225\\
599.14	0.00572176757041628\\
599.15	0.00567898069129307\\
599.16	0.00563577711566984\\
599.17	0.00559215275061546\\
599.18	0.00554810346298925\\
599.19	0.005503625079046\\
599.2	0.00545871338403701\\
599.21	0.0054133641218073\\
599.22	0.0053675729943888\\
599.23	0.00532133566158966\\
599.24	0.00527464774057935\\
599.25	0.00522750480546976\\
599.26	0.0051799023868923\\
599.27	0.00513183597157067\\
599.28	0.00508330100188951\\
599.29	0.00503429287545885\\
599.3	0.00498480694467418\\
599.31	0.00493483851627228\\
599.32	0.00488438285088275\\
599.33	0.0048334351625749\\
599.34	0.00478199061840046\\
599.35	0.00473004433793161\\
599.36	0.00467759139279455\\
599.37	0.0046246268061985\\
599.38	0.00457114555246003\\
599.39	0.00451714256049735\\
599.4	0.00446261272376356\\
599.41	0.00440755088572619\\
599.42	0.00435195183937782\\
599.43	0.00429581032674211\\
599.44	0.00423912103837487\\
599.45	0.00418187861286022\\
599.46	0.00412407763630194\\
599.47	0.00406571264180972\\
599.48	0.00400677810898046\\
599.49	0.00394726846337443\\
599.5	0.0038871780759864\\
599.51	0.00382650126271144\\
599.52	0.00376523228380568\\
599.53	0.00370336534334161\\
599.54	0.00364089458865821\\
599.55	0.00357781410980551\\
599.56	0.00351411793898394\\
599.57	0.00344980004997799\\
599.58	0.00338485435758449\\
599.59	0.0033192747170351\\
599.6	0.00325305492341342\\
599.61	0.00318618871106605\\
599.62	0.00311866975300822\\
599.63	0.00305049166032344\\
599.64	0.00298164798155718\\
599.65	0.00291213220210483\\
599.66	0.00284193774359354\\
599.67	0.00277105796325799\\
599.68	0.00269948615331015\\
599.69	0.00262721554030276\\
599.7	0.00255423928448666\\
599.71	0.00248055047916181\\
599.72	0.00240614215002193\\
599.73	0.00233100725449273\\
599.74	0.00225513868106373\\
599.75	0.00217852924861345\\
599.76	0.002101171705728\\
599.77	0.00202305873001312\\
599.78	0.00194418292739929\\
599.79	0.00186453683144024\\
599.8	0.00178411290260446\\
599.81	0.00170290352755979\\
599.82	0.0016209010184511\\
599.83	0.00153809761217073\\
599.84	0.001454485469622\\
599.85	0.00137005667497534\\
599.86	0.00128480323491721\\
599.87	0.00119871707789166\\
599.88	0.00111179005333448\\
599.89	0.00102401393089982\\
599.9	0.000935380399679272\\
599.91	0.000845881067413286\\
599.92	0.000755507459694884\\
599.93	0.000664251019165561\\
599.94	0.000572103104703356\\
599.95	0.00047905499060291\\
599.96	0.000385097865747554\\
599.97	0.000290222832773275\\
599.98	0.000194420907224489\\
599.99	9.76830167015649e-05\\
600	0\\
};
\addplot [color=black!60!mycolor21,solid,forget plot]
  table[row sep=crcr]{%
0.01	0.01\\
1.01	0.01\\
2.01	0.01\\
3.01	0.01\\
4.01	0.01\\
5.01	0.01\\
6.01	0.01\\
7.01	0.01\\
8.01	0.01\\
9.01	0.01\\
10.01	0.01\\
11.01	0.01\\
12.01	0.01\\
13.01	0.01\\
14.01	0.01\\
15.01	0.01\\
16.01	0.01\\
17.01	0.01\\
18.01	0.01\\
19.01	0.01\\
20.01	0.01\\
21.01	0.01\\
22.01	0.01\\
23.01	0.01\\
24.01	0.01\\
25.01	0.01\\
26.01	0.01\\
27.01	0.01\\
28.01	0.01\\
29.01	0.01\\
30.01	0.01\\
31.01	0.01\\
32.01	0.01\\
33.01	0.01\\
34.01	0.01\\
35.01	0.01\\
36.01	0.01\\
37.01	0.01\\
38.01	0.01\\
39.01	0.01\\
40.01	0.01\\
41.01	0.01\\
42.01	0.01\\
43.01	0.01\\
44.01	0.01\\
45.01	0.01\\
46.01	0.01\\
47.01	0.01\\
48.01	0.01\\
49.01	0.01\\
50.01	0.01\\
51.01	0.01\\
52.01	0.01\\
53.01	0.01\\
54.01	0.01\\
55.01	0.01\\
56.01	0.01\\
57.01	0.01\\
58.01	0.01\\
59.01	0.01\\
60.01	0.01\\
61.01	0.01\\
62.01	0.01\\
63.01	0.01\\
64.01	0.01\\
65.01	0.01\\
66.01	0.01\\
67.01	0.01\\
68.01	0.01\\
69.01	0.01\\
70.01	0.01\\
71.01	0.01\\
72.01	0.01\\
73.01	0.01\\
74.01	0.01\\
75.01	0.01\\
76.01	0.01\\
77.01	0.01\\
78.01	0.01\\
79.01	0.01\\
80.01	0.01\\
81.01	0.01\\
82.01	0.01\\
83.01	0.01\\
84.01	0.01\\
85.01	0.01\\
86.01	0.01\\
87.01	0.01\\
88.01	0.01\\
89.01	0.01\\
90.01	0.01\\
91.01	0.01\\
92.01	0.01\\
93.01	0.01\\
94.01	0.01\\
95.01	0.01\\
96.01	0.01\\
97.01	0.01\\
98.01	0.01\\
99.01	0.01\\
100.01	0.01\\
101.01	0.01\\
102.01	0.01\\
103.01	0.01\\
104.01	0.01\\
105.01	0.01\\
106.01	0.01\\
107.01	0.01\\
108.01	0.01\\
109.01	0.01\\
110.01	0.01\\
111.01	0.01\\
112.01	0.01\\
113.01	0.01\\
114.01	0.01\\
115.01	0.01\\
116.01	0.01\\
117.01	0.01\\
118.01	0.01\\
119.01	0.01\\
120.01	0.01\\
121.01	0.01\\
122.01	0.01\\
123.01	0.01\\
124.01	0.01\\
125.01	0.01\\
126.01	0.01\\
127.01	0.01\\
128.01	0.01\\
129.01	0.01\\
130.01	0.01\\
131.01	0.01\\
132.01	0.01\\
133.01	0.01\\
134.01	0.01\\
135.01	0.01\\
136.01	0.01\\
137.01	0.01\\
138.01	0.01\\
139.01	0.01\\
140.01	0.01\\
141.01	0.01\\
142.01	0.01\\
143.01	0.01\\
144.01	0.01\\
145.01	0.01\\
146.01	0.01\\
147.01	0.01\\
148.01	0.01\\
149.01	0.01\\
150.01	0.01\\
151.01	0.01\\
152.01	0.01\\
153.01	0.01\\
154.01	0.01\\
155.01	0.01\\
156.01	0.01\\
157.01	0.01\\
158.01	0.01\\
159.01	0.01\\
160.01	0.01\\
161.01	0.01\\
162.01	0.01\\
163.01	0.01\\
164.01	0.01\\
165.01	0.01\\
166.01	0.01\\
167.01	0.01\\
168.01	0.01\\
169.01	0.01\\
170.01	0.01\\
171.01	0.01\\
172.01	0.01\\
173.01	0.01\\
174.01	0.01\\
175.01	0.01\\
176.01	0.01\\
177.01	0.01\\
178.01	0.01\\
179.01	0.01\\
180.01	0.01\\
181.01	0.01\\
182.01	0.01\\
183.01	0.01\\
184.01	0.01\\
185.01	0.01\\
186.01	0.01\\
187.01	0.01\\
188.01	0.01\\
189.01	0.01\\
190.01	0.01\\
191.01	0.01\\
192.01	0.01\\
193.01	0.01\\
194.01	0.01\\
195.01	0.01\\
196.01	0.01\\
197.01	0.01\\
198.01	0.01\\
199.01	0.01\\
200.01	0.01\\
201.01	0.01\\
202.01	0.01\\
203.01	0.01\\
204.01	0.01\\
205.01	0.01\\
206.01	0.01\\
207.01	0.01\\
208.01	0.01\\
209.01	0.01\\
210.01	0.01\\
211.01	0.01\\
212.01	0.01\\
213.01	0.01\\
214.01	0.01\\
215.01	0.01\\
216.01	0.01\\
217.01	0.01\\
218.01	0.01\\
219.01	0.01\\
220.01	0.01\\
221.01	0.01\\
222.01	0.01\\
223.01	0.01\\
224.01	0.01\\
225.01	0.01\\
226.01	0.01\\
227.01	0.01\\
228.01	0.01\\
229.01	0.01\\
230.01	0.01\\
231.01	0.01\\
232.01	0.01\\
233.01	0.01\\
234.01	0.01\\
235.01	0.01\\
236.01	0.01\\
237.01	0.01\\
238.01	0.01\\
239.01	0.01\\
240.01	0.01\\
241.01	0.01\\
242.01	0.01\\
243.01	0.01\\
244.01	0.01\\
245.01	0.01\\
246.01	0.01\\
247.01	0.01\\
248.01	0.01\\
249.01	0.01\\
250.01	0.01\\
251.01	0.01\\
252.01	0.01\\
253.01	0.01\\
254.01	0.01\\
255.01	0.01\\
256.01	0.01\\
257.01	0.01\\
258.01	0.01\\
259.01	0.01\\
260.01	0.01\\
261.01	0.01\\
262.01	0.01\\
263.01	0.01\\
264.01	0.01\\
265.01	0.01\\
266.01	0.01\\
267.01	0.01\\
268.01	0.01\\
269.01	0.01\\
270.01	0.01\\
271.01	0.01\\
272.01	0.01\\
273.01	0.01\\
274.01	0.01\\
275.01	0.01\\
276.01	0.01\\
277.01	0.01\\
278.01	0.01\\
279.01	0.01\\
280.01	0.01\\
281.01	0.01\\
282.01	0.01\\
283.01	0.01\\
284.01	0.01\\
285.01	0.01\\
286.01	0.01\\
287.01	0.01\\
288.01	0.01\\
289.01	0.01\\
290.01	0.01\\
291.01	0.01\\
292.01	0.01\\
293.01	0.01\\
294.01	0.01\\
295.01	0.01\\
296.01	0.01\\
297.01	0.01\\
298.01	0.01\\
299.01	0.01\\
300.01	0.01\\
301.01	0.01\\
302.01	0.01\\
303.01	0.01\\
304.01	0.01\\
305.01	0.01\\
306.01	0.01\\
307.01	0.01\\
308.01	0.01\\
309.01	0.01\\
310.01	0.01\\
311.01	0.01\\
312.01	0.01\\
313.01	0.01\\
314.01	0.01\\
315.01	0.01\\
316.01	0.01\\
317.01	0.01\\
318.01	0.01\\
319.01	0.01\\
320.01	0.01\\
321.01	0.01\\
322.01	0.01\\
323.01	0.01\\
324.01	0.01\\
325.01	0.01\\
326.01	0.01\\
327.01	0.01\\
328.01	0.01\\
329.01	0.01\\
330.01	0.01\\
331.01	0.01\\
332.01	0.01\\
333.01	0.01\\
334.01	0.01\\
335.01	0.01\\
336.01	0.01\\
337.01	0.01\\
338.01	0.01\\
339.01	0.01\\
340.01	0.01\\
341.01	0.01\\
342.01	0.01\\
343.01	0.01\\
344.01	0.01\\
345.01	0.01\\
346.01	0.01\\
347.01	0.01\\
348.01	0.01\\
349.01	0.01\\
350.01	0.01\\
351.01	0.01\\
352.01	0.01\\
353.01	0.01\\
354.01	0.01\\
355.01	0.01\\
356.01	0.01\\
357.01	0.01\\
358.01	0.01\\
359.01	0.01\\
360.01	0.01\\
361.01	0.01\\
362.01	0.01\\
363.01	0.01\\
364.01	0.01\\
365.01	0.01\\
366.01	0.01\\
367.01	0.01\\
368.01	0.01\\
369.01	0.01\\
370.01	0.01\\
371.01	0.01\\
372.01	0.01\\
373.01	0.01\\
374.01	0.01\\
375.01	0.01\\
376.01	0.01\\
377.01	0.01\\
378.01	0.01\\
379.01	0.01\\
380.01	0.01\\
381.01	0.01\\
382.01	0.01\\
383.01	0.01\\
384.01	0.01\\
385.01	0.01\\
386.01	0.01\\
387.01	0.01\\
388.01	0.01\\
389.01	0.01\\
390.01	0.01\\
391.01	0.01\\
392.01	0.01\\
393.01	0.01\\
394.01	0.01\\
395.01	0.01\\
396.01	0.01\\
397.01	0.01\\
398.01	0.01\\
399.01	0.01\\
400.01	0.01\\
401.01	0.01\\
402.01	0.01\\
403.01	0.01\\
404.01	0.01\\
405.01	0.01\\
406.01	0.01\\
407.01	0.01\\
408.01	0.01\\
409.01	0.01\\
410.01	0.01\\
411.01	0.01\\
412.01	0.01\\
413.01	0.01\\
414.01	0.01\\
415.01	0.01\\
416.01	0.01\\
417.01	0.01\\
418.01	0.01\\
419.01	0.01\\
420.01	0.01\\
421.01	0.01\\
422.01	0.01\\
423.01	0.01\\
424.01	0.01\\
425.01	0.01\\
426.01	0.01\\
427.01	0.01\\
428.01	0.01\\
429.01	0.01\\
430.01	0.01\\
431.01	0.01\\
432.01	0.01\\
433.01	0.01\\
434.01	0.01\\
435.01	0.01\\
436.01	0.01\\
437.01	0.01\\
438.01	0.01\\
439.01	0.01\\
440.01	0.01\\
441.01	0.01\\
442.01	0.01\\
443.01	0.01\\
444.01	0.01\\
445.01	0.01\\
446.01	0.01\\
447.01	0.01\\
448.01	0.01\\
449.01	0.01\\
450.01	0.01\\
451.01	0.01\\
452.01	0.01\\
453.01	0.01\\
454.01	0.01\\
455.01	0.01\\
456.01	0.01\\
457.01	0.01\\
458.01	0.01\\
459.01	0.01\\
460.01	0.01\\
461.01	0.01\\
462.01	0.01\\
463.01	0.01\\
464.01	0.01\\
465.01	0.01\\
466.01	0.01\\
467.01	0.01\\
468.01	0.01\\
469.01	0.01\\
470.01	0.01\\
471.01	0.01\\
472.01	0.01\\
473.01	0.01\\
474.01	0.01\\
475.01	0.01\\
476.01	0.01\\
477.01	0.01\\
478.01	0.01\\
479.01	0.01\\
480.01	0.01\\
481.01	0.01\\
482.01	0.01\\
483.01	0.01\\
484.01	0.01\\
485.01	0.01\\
486.01	0.01\\
487.01	0.01\\
488.01	0.01\\
489.01	0.01\\
490.01	0.01\\
491.01	0.01\\
492.01	0.01\\
493.01	0.01\\
494.01	0.01\\
495.01	0.01\\
496.01	0.01\\
497.01	0.01\\
498.01	0.01\\
499.01	0.01\\
500.01	0.01\\
501.01	0.01\\
502.01	0.01\\
503.01	0.01\\
504.01	0.01\\
505.01	0.01\\
506.01	0.01\\
507.01	0.01\\
508.01	0.01\\
509.01	0.01\\
510.01	0.01\\
511.01	0.01\\
512.01	0.01\\
513.01	0.01\\
514.01	0.01\\
515.01	0.01\\
516.01	0.01\\
517.01	0.01\\
518.01	0.01\\
519.01	0.01\\
520.01	0.01\\
521.01	0.01\\
522.01	0.01\\
523.01	0.01\\
524.01	0.01\\
525.01	0.01\\
526.01	0.01\\
527.01	0.01\\
528.01	0.01\\
529.01	0.01\\
530.01	0.01\\
531.01	0.01\\
532.01	0.01\\
533.01	0.01\\
534.01	0.01\\
535.01	0.01\\
536.01	0.01\\
537.01	0.01\\
538.01	0.01\\
539.01	0.01\\
540.01	0.01\\
541.01	0.01\\
542.01	0.01\\
543.01	0.01\\
544.01	0.01\\
545.01	0.01\\
546.01	0.01\\
547.01	0.01\\
548.01	0.01\\
549.01	0.01\\
550.01	0.01\\
551.01	0.01\\
552.01	0.01\\
553.01	0.01\\
554.01	0.01\\
555.01	0.01\\
556.01	0.01\\
557.01	0.01\\
558.01	0.01\\
559.01	0.01\\
560.01	0.01\\
561.01	0.01\\
562.01	0.01\\
563.01	0.01\\
564.01	0.01\\
565.01	0.01\\
566.01	0.01\\
567.01	0.01\\
568.01	0.01\\
569.01	0.01\\
570.01	0.01\\
571.01	0.01\\
572.01	0.01\\
573.01	0.01\\
574.01	0.01\\
575.01	0.01\\
576.01	0.01\\
577.01	0.01\\
578.01	0.01\\
579.01	0.01\\
580.01	0.01\\
581.01	0.01\\
582.01	0.01\\
583.01	0.01\\
584.01	0.01\\
585.01	0.01\\
586.01	0.01\\
587.01	0.01\\
588.01	0.01\\
589.01	0.01\\
590.01	0.01\\
591.01	0.01\\
592.01	0.01\\
593.01	0.01\\
594.01	0.01\\
595.01	0.01\\
596.01	0.01\\
597.01	0.01\\
598.01	0.00865150542787716\\
599.01	0.00624186909405409\\
599.02	0.00620414441242163\\
599.03	0.00616605275622734\\
599.04	0.00612759052101569\\
599.05	0.00608875406692033\\
599.06	0.00604953971831639\\
599.07	0.00600994376346915\\
599.08	0.00596996245417917\\
599.09	0.00592959200542425\\
599.1	0.00588882859499752\\
599.11	0.00584766836314221\\
599.12	0.00580610741218276\\
599.13	0.00576414180615225\\
599.14	0.00572176757041629\\
599.15	0.00567898069129309\\
599.16	0.00563577711566987\\
599.17	0.00559215275061548\\
599.18	0.00554810346298928\\
599.19	0.00550362507904603\\
599.2	0.00545871338403702\\
599.21	0.0054133641218073\\
599.22	0.00536757299438882\\
599.23	0.00532133566158968\\
599.24	0.00527464774057935\\
599.25	0.00522750480546976\\
599.26	0.0051799023868923\\
599.27	0.00513183597157068\\
599.28	0.00508330100188954\\
599.29	0.00503429287545886\\
599.3	0.00498480694467418\\
599.31	0.00493483851627229\\
599.32	0.00488438285088275\\
599.33	0.0048334351625749\\
599.34	0.00478199061840045\\
599.35	0.00473004433793158\\
599.36	0.00467759139279451\\
599.37	0.00462462680619847\\
599.38	0.00457114555246\\
599.39	0.00451714256049731\\
599.4	0.00446261272376353\\
599.41	0.00440755088572614\\
599.42	0.00435195183937776\\
599.43	0.00429581032674206\\
599.44	0.00423912103837481\\
599.45	0.00418187861286016\\
599.46	0.00412407763630187\\
599.47	0.00406571264180965\\
599.48	0.00400677810898039\\
599.49	0.00394726846337437\\
599.5	0.00388717807598633\\
599.51	0.00382650126271137\\
599.52	0.00376523228380562\\
599.53	0.00370336534334157\\
599.54	0.00364089458865816\\
599.55	0.00357781410980548\\
599.56	0.00351411793898392\\
599.57	0.00344980004997798\\
599.58	0.00338485435758447\\
599.59	0.0033192747170351\\
599.6	0.00325305492341341\\
599.61	0.00318618871106603\\
599.62	0.00311866975300821\\
599.63	0.00305049166032342\\
599.64	0.00298164798155715\\
599.65	0.0029121322021048\\
599.66	0.00284193774359352\\
599.67	0.00277105796325798\\
599.68	0.00269948615331014\\
599.69	0.00262721554030274\\
599.7	0.00255423928448665\\
599.71	0.0024805504791618\\
599.72	0.00240614215002191\\
599.73	0.00233100725449272\\
599.74	0.00225513868106372\\
599.75	0.00217852924861344\\
599.76	0.00210117170572799\\
599.77	0.0020230587300131\\
599.78	0.00194418292739927\\
599.79	0.00186453683144023\\
599.8	0.00178411290260445\\
599.81	0.00170290352755979\\
599.82	0.00162090101845109\\
599.83	0.00153809761217072\\
599.84	0.001454485469622\\
599.85	0.00137005667497534\\
599.86	0.00128480323491721\\
599.87	0.00119871707789166\\
599.88	0.00111179005333448\\
599.89	0.00102401393089982\\
599.9	0.000935380399679274\\
599.91	0.000845881067413283\\
599.92	0.000755507459694877\\
599.93	0.000664251019165561\\
599.94	0.000572103104703355\\
599.95	0.000479054990602912\\
599.96	0.000385097865747556\\
599.97	0.000290222832773275\\
599.98	0.000194420907224489\\
599.99	9.76830167015632e-05\\
600	0\\
};
\addplot [color=black!80!mycolor21,solid,forget plot]
  table[row sep=crcr]{%
0.01	0.01\\
1.01	0.01\\
2.01	0.01\\
3.01	0.01\\
4.01	0.01\\
5.01	0.01\\
6.01	0.01\\
7.01	0.01\\
8.01	0.01\\
9.01	0.01\\
10.01	0.01\\
11.01	0.01\\
12.01	0.01\\
13.01	0.01\\
14.01	0.01\\
15.01	0.01\\
16.01	0.01\\
17.01	0.01\\
18.01	0.01\\
19.01	0.01\\
20.01	0.01\\
21.01	0.01\\
22.01	0.01\\
23.01	0.01\\
24.01	0.01\\
25.01	0.01\\
26.01	0.01\\
27.01	0.01\\
28.01	0.01\\
29.01	0.01\\
30.01	0.01\\
31.01	0.01\\
32.01	0.01\\
33.01	0.01\\
34.01	0.01\\
35.01	0.01\\
36.01	0.01\\
37.01	0.01\\
38.01	0.01\\
39.01	0.01\\
40.01	0.01\\
41.01	0.01\\
42.01	0.01\\
43.01	0.01\\
44.01	0.01\\
45.01	0.01\\
46.01	0.01\\
47.01	0.01\\
48.01	0.01\\
49.01	0.01\\
50.01	0.01\\
51.01	0.01\\
52.01	0.01\\
53.01	0.01\\
54.01	0.01\\
55.01	0.01\\
56.01	0.01\\
57.01	0.01\\
58.01	0.01\\
59.01	0.01\\
60.01	0.01\\
61.01	0.01\\
62.01	0.01\\
63.01	0.01\\
64.01	0.01\\
65.01	0.01\\
66.01	0.01\\
67.01	0.01\\
68.01	0.01\\
69.01	0.01\\
70.01	0.01\\
71.01	0.01\\
72.01	0.01\\
73.01	0.01\\
74.01	0.01\\
75.01	0.01\\
76.01	0.01\\
77.01	0.01\\
78.01	0.01\\
79.01	0.01\\
80.01	0.01\\
81.01	0.01\\
82.01	0.01\\
83.01	0.01\\
84.01	0.01\\
85.01	0.01\\
86.01	0.01\\
87.01	0.01\\
88.01	0.01\\
89.01	0.01\\
90.01	0.01\\
91.01	0.01\\
92.01	0.01\\
93.01	0.01\\
94.01	0.01\\
95.01	0.01\\
96.01	0.01\\
97.01	0.01\\
98.01	0.01\\
99.01	0.01\\
100.01	0.01\\
101.01	0.01\\
102.01	0.01\\
103.01	0.01\\
104.01	0.01\\
105.01	0.01\\
106.01	0.01\\
107.01	0.01\\
108.01	0.01\\
109.01	0.01\\
110.01	0.01\\
111.01	0.01\\
112.01	0.01\\
113.01	0.01\\
114.01	0.01\\
115.01	0.01\\
116.01	0.01\\
117.01	0.01\\
118.01	0.01\\
119.01	0.01\\
120.01	0.01\\
121.01	0.01\\
122.01	0.01\\
123.01	0.01\\
124.01	0.01\\
125.01	0.01\\
126.01	0.01\\
127.01	0.01\\
128.01	0.01\\
129.01	0.01\\
130.01	0.01\\
131.01	0.01\\
132.01	0.01\\
133.01	0.01\\
134.01	0.01\\
135.01	0.01\\
136.01	0.01\\
137.01	0.01\\
138.01	0.01\\
139.01	0.01\\
140.01	0.01\\
141.01	0.01\\
142.01	0.01\\
143.01	0.01\\
144.01	0.01\\
145.01	0.01\\
146.01	0.01\\
147.01	0.01\\
148.01	0.01\\
149.01	0.01\\
150.01	0.01\\
151.01	0.01\\
152.01	0.01\\
153.01	0.01\\
154.01	0.01\\
155.01	0.01\\
156.01	0.01\\
157.01	0.01\\
158.01	0.01\\
159.01	0.01\\
160.01	0.01\\
161.01	0.01\\
162.01	0.01\\
163.01	0.01\\
164.01	0.01\\
165.01	0.01\\
166.01	0.01\\
167.01	0.01\\
168.01	0.01\\
169.01	0.01\\
170.01	0.01\\
171.01	0.01\\
172.01	0.01\\
173.01	0.01\\
174.01	0.01\\
175.01	0.01\\
176.01	0.01\\
177.01	0.01\\
178.01	0.01\\
179.01	0.01\\
180.01	0.01\\
181.01	0.01\\
182.01	0.01\\
183.01	0.01\\
184.01	0.01\\
185.01	0.01\\
186.01	0.01\\
187.01	0.01\\
188.01	0.01\\
189.01	0.01\\
190.01	0.01\\
191.01	0.01\\
192.01	0.01\\
193.01	0.01\\
194.01	0.01\\
195.01	0.01\\
196.01	0.01\\
197.01	0.01\\
198.01	0.01\\
199.01	0.01\\
200.01	0.01\\
201.01	0.01\\
202.01	0.01\\
203.01	0.01\\
204.01	0.01\\
205.01	0.01\\
206.01	0.01\\
207.01	0.01\\
208.01	0.01\\
209.01	0.01\\
210.01	0.01\\
211.01	0.01\\
212.01	0.01\\
213.01	0.01\\
214.01	0.01\\
215.01	0.01\\
216.01	0.01\\
217.01	0.01\\
218.01	0.01\\
219.01	0.01\\
220.01	0.01\\
221.01	0.01\\
222.01	0.01\\
223.01	0.01\\
224.01	0.01\\
225.01	0.01\\
226.01	0.01\\
227.01	0.01\\
228.01	0.01\\
229.01	0.01\\
230.01	0.01\\
231.01	0.01\\
232.01	0.01\\
233.01	0.01\\
234.01	0.01\\
235.01	0.01\\
236.01	0.01\\
237.01	0.01\\
238.01	0.01\\
239.01	0.01\\
240.01	0.01\\
241.01	0.01\\
242.01	0.01\\
243.01	0.01\\
244.01	0.01\\
245.01	0.01\\
246.01	0.01\\
247.01	0.01\\
248.01	0.01\\
249.01	0.01\\
250.01	0.01\\
251.01	0.01\\
252.01	0.01\\
253.01	0.01\\
254.01	0.01\\
255.01	0.01\\
256.01	0.01\\
257.01	0.01\\
258.01	0.01\\
259.01	0.01\\
260.01	0.01\\
261.01	0.01\\
262.01	0.01\\
263.01	0.01\\
264.01	0.01\\
265.01	0.01\\
266.01	0.01\\
267.01	0.01\\
268.01	0.01\\
269.01	0.01\\
270.01	0.01\\
271.01	0.01\\
272.01	0.01\\
273.01	0.01\\
274.01	0.01\\
275.01	0.01\\
276.01	0.01\\
277.01	0.01\\
278.01	0.01\\
279.01	0.01\\
280.01	0.01\\
281.01	0.01\\
282.01	0.01\\
283.01	0.01\\
284.01	0.01\\
285.01	0.01\\
286.01	0.01\\
287.01	0.01\\
288.01	0.01\\
289.01	0.01\\
290.01	0.01\\
291.01	0.01\\
292.01	0.01\\
293.01	0.01\\
294.01	0.01\\
295.01	0.01\\
296.01	0.01\\
297.01	0.01\\
298.01	0.01\\
299.01	0.01\\
300.01	0.01\\
301.01	0.01\\
302.01	0.01\\
303.01	0.01\\
304.01	0.01\\
305.01	0.01\\
306.01	0.01\\
307.01	0.01\\
308.01	0.01\\
309.01	0.01\\
310.01	0.01\\
311.01	0.01\\
312.01	0.01\\
313.01	0.01\\
314.01	0.01\\
315.01	0.01\\
316.01	0.01\\
317.01	0.01\\
318.01	0.01\\
319.01	0.01\\
320.01	0.01\\
321.01	0.01\\
322.01	0.01\\
323.01	0.01\\
324.01	0.01\\
325.01	0.01\\
326.01	0.01\\
327.01	0.01\\
328.01	0.01\\
329.01	0.01\\
330.01	0.01\\
331.01	0.01\\
332.01	0.01\\
333.01	0.01\\
334.01	0.01\\
335.01	0.01\\
336.01	0.01\\
337.01	0.01\\
338.01	0.01\\
339.01	0.01\\
340.01	0.01\\
341.01	0.01\\
342.01	0.01\\
343.01	0.01\\
344.01	0.01\\
345.01	0.01\\
346.01	0.01\\
347.01	0.01\\
348.01	0.01\\
349.01	0.01\\
350.01	0.01\\
351.01	0.01\\
352.01	0.01\\
353.01	0.01\\
354.01	0.01\\
355.01	0.01\\
356.01	0.01\\
357.01	0.01\\
358.01	0.01\\
359.01	0.01\\
360.01	0.01\\
361.01	0.01\\
362.01	0.01\\
363.01	0.01\\
364.01	0.01\\
365.01	0.01\\
366.01	0.01\\
367.01	0.01\\
368.01	0.01\\
369.01	0.01\\
370.01	0.01\\
371.01	0.01\\
372.01	0.01\\
373.01	0.01\\
374.01	0.01\\
375.01	0.01\\
376.01	0.01\\
377.01	0.01\\
378.01	0.01\\
379.01	0.01\\
380.01	0.01\\
381.01	0.01\\
382.01	0.01\\
383.01	0.01\\
384.01	0.01\\
385.01	0.01\\
386.01	0.01\\
387.01	0.01\\
388.01	0.01\\
389.01	0.01\\
390.01	0.01\\
391.01	0.01\\
392.01	0.01\\
393.01	0.01\\
394.01	0.01\\
395.01	0.01\\
396.01	0.01\\
397.01	0.01\\
398.01	0.01\\
399.01	0.01\\
400.01	0.01\\
401.01	0.01\\
402.01	0.01\\
403.01	0.01\\
404.01	0.01\\
405.01	0.01\\
406.01	0.01\\
407.01	0.01\\
408.01	0.01\\
409.01	0.01\\
410.01	0.01\\
411.01	0.01\\
412.01	0.01\\
413.01	0.01\\
414.01	0.01\\
415.01	0.01\\
416.01	0.01\\
417.01	0.01\\
418.01	0.01\\
419.01	0.01\\
420.01	0.01\\
421.01	0.01\\
422.01	0.01\\
423.01	0.01\\
424.01	0.01\\
425.01	0.01\\
426.01	0.01\\
427.01	0.01\\
428.01	0.01\\
429.01	0.01\\
430.01	0.01\\
431.01	0.01\\
432.01	0.01\\
433.01	0.01\\
434.01	0.01\\
435.01	0.01\\
436.01	0.01\\
437.01	0.01\\
438.01	0.01\\
439.01	0.01\\
440.01	0.01\\
441.01	0.01\\
442.01	0.01\\
443.01	0.01\\
444.01	0.01\\
445.01	0.01\\
446.01	0.01\\
447.01	0.01\\
448.01	0.01\\
449.01	0.01\\
450.01	0.01\\
451.01	0.01\\
452.01	0.01\\
453.01	0.01\\
454.01	0.01\\
455.01	0.01\\
456.01	0.01\\
457.01	0.01\\
458.01	0.01\\
459.01	0.01\\
460.01	0.01\\
461.01	0.01\\
462.01	0.01\\
463.01	0.01\\
464.01	0.01\\
465.01	0.01\\
466.01	0.01\\
467.01	0.01\\
468.01	0.01\\
469.01	0.01\\
470.01	0.01\\
471.01	0.01\\
472.01	0.01\\
473.01	0.01\\
474.01	0.01\\
475.01	0.01\\
476.01	0.01\\
477.01	0.01\\
478.01	0.01\\
479.01	0.01\\
480.01	0.01\\
481.01	0.01\\
482.01	0.01\\
483.01	0.01\\
484.01	0.01\\
485.01	0.01\\
486.01	0.01\\
487.01	0.01\\
488.01	0.01\\
489.01	0.01\\
490.01	0.01\\
491.01	0.01\\
492.01	0.01\\
493.01	0.01\\
494.01	0.01\\
495.01	0.01\\
496.01	0.01\\
497.01	0.01\\
498.01	0.01\\
499.01	0.01\\
500.01	0.01\\
501.01	0.01\\
502.01	0.01\\
503.01	0.01\\
504.01	0.01\\
505.01	0.01\\
506.01	0.01\\
507.01	0.01\\
508.01	0.01\\
509.01	0.01\\
510.01	0.01\\
511.01	0.01\\
512.01	0.01\\
513.01	0.01\\
514.01	0.01\\
515.01	0.01\\
516.01	0.01\\
517.01	0.01\\
518.01	0.01\\
519.01	0.01\\
520.01	0.01\\
521.01	0.01\\
522.01	0.01\\
523.01	0.01\\
524.01	0.01\\
525.01	0.01\\
526.01	0.01\\
527.01	0.01\\
528.01	0.01\\
529.01	0.01\\
530.01	0.01\\
531.01	0.01\\
532.01	0.01\\
533.01	0.01\\
534.01	0.01\\
535.01	0.01\\
536.01	0.01\\
537.01	0.01\\
538.01	0.01\\
539.01	0.01\\
540.01	0.01\\
541.01	0.01\\
542.01	0.01\\
543.01	0.01\\
544.01	0.01\\
545.01	0.01\\
546.01	0.01\\
547.01	0.01\\
548.01	0.01\\
549.01	0.01\\
550.01	0.01\\
551.01	0.01\\
552.01	0.01\\
553.01	0.01\\
554.01	0.01\\
555.01	0.01\\
556.01	0.01\\
557.01	0.01\\
558.01	0.01\\
559.01	0.01\\
560.01	0.01\\
561.01	0.01\\
562.01	0.01\\
563.01	0.01\\
564.01	0.01\\
565.01	0.01\\
566.01	0.01\\
567.01	0.01\\
568.01	0.01\\
569.01	0.01\\
570.01	0.01\\
571.01	0.01\\
572.01	0.01\\
573.01	0.01\\
574.01	0.01\\
575.01	0.01\\
576.01	0.01\\
577.01	0.01\\
578.01	0.01\\
579.01	0.01\\
580.01	0.01\\
581.01	0.01\\
582.01	0.01\\
583.01	0.01\\
584.01	0.01\\
585.01	0.01\\
586.01	0.01\\
587.01	0.01\\
588.01	0.01\\
589.01	0.01\\
590.01	0.01\\
591.01	0.01\\
592.01	0.01\\
593.01	0.01\\
594.01	0.01\\
595.01	0.01\\
596.01	0.01\\
597.01	0.01\\
598.01	0.00865127650878676\\
599.01	0.0062418690940541\\
599.02	0.00620414441242163\\
599.03	0.00616605275622739\\
599.04	0.00612759052101571\\
599.05	0.00608875406692037\\
599.06	0.00604953971831644\\
599.07	0.00600994376346918\\
599.08	0.0059699624541792\\
599.09	0.00592959200542428\\
599.1	0.00588882859499753\\
599.11	0.00584766836314222\\
599.12	0.00580610741218276\\
599.13	0.00576414180615225\\
599.14	0.00572176757041628\\
599.15	0.00567898069129307\\
599.16	0.00563577711566984\\
599.17	0.00559215275061546\\
599.18	0.00554810346298927\\
599.19	0.00550362507904601\\
599.2	0.00545871338403702\\
599.21	0.00541336412180731\\
599.22	0.00536757299438883\\
599.23	0.00532133566158969\\
599.24	0.00527464774057936\\
599.25	0.00522750480546979\\
599.26	0.00517990238689235\\
599.27	0.00513183597157071\\
599.28	0.00508330100188955\\
599.29	0.00503429287545888\\
599.3	0.0049848069446742\\
599.31	0.00493483851627232\\
599.32	0.00488438285088278\\
599.33	0.00483343516257493\\
599.34	0.00478199061840049\\
599.35	0.00473004433793164\\
599.36	0.00467759139279458\\
599.37	0.00462462680619853\\
599.38	0.00457114555246006\\
599.39	0.00451714256049738\\
599.4	0.0044626127237636\\
599.41	0.00440755088572623\\
599.42	0.00435195183937786\\
599.43	0.00429581032674215\\
599.44	0.00423912103837491\\
599.45	0.00418187861286026\\
599.46	0.00412407763630198\\
599.47	0.00406571264180975\\
599.48	0.00400677810898049\\
599.49	0.00394726846337446\\
599.5	0.00388717807598643\\
599.51	0.00382650126271148\\
599.52	0.00376523228380572\\
599.53	0.00370336534334166\\
599.54	0.00364089458865823\\
599.55	0.00357781410980553\\
599.56	0.00351411793898395\\
599.57	0.00344980004997801\\
599.58	0.00338485435758451\\
599.59	0.00331927471703512\\
599.6	0.00325305492341342\\
599.61	0.00318618871106606\\
599.62	0.00311866975300824\\
599.63	0.00305049166032345\\
599.64	0.00298164798155718\\
599.65	0.00291213220210484\\
599.66	0.00284193774359354\\
599.67	0.002771057963258\\
599.68	0.00269948615331015\\
599.69	0.00262721554030276\\
599.7	0.00255423928448666\\
599.71	0.00248055047916181\\
599.72	0.00240614215002192\\
599.73	0.00233100725449273\\
599.74	0.00225513868106373\\
599.75	0.00217852924861344\\
599.76	0.002101171705728\\
599.77	0.00202305873001311\\
599.78	0.00194418292739929\\
599.79	0.00186453683144024\\
599.8	0.00178411290260446\\
599.81	0.0017029035275598\\
599.82	0.0016209010184511\\
599.83	0.00153809761217073\\
599.84	0.001454485469622\\
599.85	0.00137005667497534\\
599.86	0.00128480323491721\\
599.87	0.00119871707789167\\
599.88	0.00111179005333449\\
599.89	0.00102401393089983\\
599.9	0.000935380399679277\\
599.91	0.00084588106741329\\
599.92	0.00075550745969488\\
599.93	0.000664251019165563\\
599.94	0.00057210310470336\\
599.95	0.000479054990602914\\
599.96	0.000385097865747556\\
599.97	0.000290222832773275\\
599.98	0.000194420907224489\\
599.99	9.76830167015632e-05\\
600	0\\
};
\addplot [color=black,solid,forget plot]
  table[row sep=crcr]{%
0.01	0.01\\
1.01	0.01\\
2.01	0.01\\
3.01	0.01\\
4.01	0.01\\
5.01	0.01\\
6.01	0.01\\
7.01	0.01\\
8.01	0.01\\
9.01	0.01\\
10.01	0.01\\
11.01	0.01\\
12.01	0.01\\
13.01	0.01\\
14.01	0.01\\
15.01	0.01\\
16.01	0.01\\
17.01	0.01\\
18.01	0.01\\
19.01	0.01\\
20.01	0.01\\
21.01	0.01\\
22.01	0.01\\
23.01	0.01\\
24.01	0.01\\
25.01	0.01\\
26.01	0.01\\
27.01	0.01\\
28.01	0.01\\
29.01	0.01\\
30.01	0.01\\
31.01	0.01\\
32.01	0.01\\
33.01	0.01\\
34.01	0.01\\
35.01	0.01\\
36.01	0.01\\
37.01	0.01\\
38.01	0.01\\
39.01	0.01\\
40.01	0.01\\
41.01	0.01\\
42.01	0.01\\
43.01	0.01\\
44.01	0.01\\
45.01	0.01\\
46.01	0.01\\
47.01	0.01\\
48.01	0.01\\
49.01	0.01\\
50.01	0.01\\
51.01	0.01\\
52.01	0.01\\
53.01	0.01\\
54.01	0.01\\
55.01	0.01\\
56.01	0.01\\
57.01	0.01\\
58.01	0.01\\
59.01	0.01\\
60.01	0.01\\
61.01	0.01\\
62.01	0.01\\
63.01	0.01\\
64.01	0.01\\
65.01	0.01\\
66.01	0.01\\
67.01	0.01\\
68.01	0.01\\
69.01	0.01\\
70.01	0.01\\
71.01	0.01\\
72.01	0.01\\
73.01	0.01\\
74.01	0.01\\
75.01	0.01\\
76.01	0.01\\
77.01	0.01\\
78.01	0.01\\
79.01	0.01\\
80.01	0.01\\
81.01	0.01\\
82.01	0.01\\
83.01	0.01\\
84.01	0.01\\
85.01	0.01\\
86.01	0.01\\
87.01	0.01\\
88.01	0.01\\
89.01	0.01\\
90.01	0.01\\
91.01	0.01\\
92.01	0.01\\
93.01	0.01\\
94.01	0.01\\
95.01	0.01\\
96.01	0.01\\
97.01	0.01\\
98.01	0.01\\
99.01	0.01\\
100.01	0.01\\
101.01	0.01\\
102.01	0.01\\
103.01	0.01\\
104.01	0.01\\
105.01	0.01\\
106.01	0.01\\
107.01	0.01\\
108.01	0.01\\
109.01	0.01\\
110.01	0.01\\
111.01	0.01\\
112.01	0.01\\
113.01	0.01\\
114.01	0.01\\
115.01	0.01\\
116.01	0.01\\
117.01	0.01\\
118.01	0.01\\
119.01	0.01\\
120.01	0.01\\
121.01	0.01\\
122.01	0.01\\
123.01	0.01\\
124.01	0.01\\
125.01	0.01\\
126.01	0.01\\
127.01	0.01\\
128.01	0.01\\
129.01	0.01\\
130.01	0.01\\
131.01	0.01\\
132.01	0.01\\
133.01	0.01\\
134.01	0.01\\
135.01	0.01\\
136.01	0.01\\
137.01	0.01\\
138.01	0.01\\
139.01	0.01\\
140.01	0.01\\
141.01	0.01\\
142.01	0.01\\
143.01	0.01\\
144.01	0.01\\
145.01	0.01\\
146.01	0.01\\
147.01	0.01\\
148.01	0.01\\
149.01	0.01\\
150.01	0.01\\
151.01	0.01\\
152.01	0.01\\
153.01	0.01\\
154.01	0.01\\
155.01	0.01\\
156.01	0.01\\
157.01	0.01\\
158.01	0.01\\
159.01	0.01\\
160.01	0.01\\
161.01	0.01\\
162.01	0.01\\
163.01	0.01\\
164.01	0.01\\
165.01	0.01\\
166.01	0.01\\
167.01	0.01\\
168.01	0.01\\
169.01	0.01\\
170.01	0.01\\
171.01	0.01\\
172.01	0.01\\
173.01	0.01\\
174.01	0.01\\
175.01	0.01\\
176.01	0.01\\
177.01	0.01\\
178.01	0.01\\
179.01	0.01\\
180.01	0.01\\
181.01	0.01\\
182.01	0.01\\
183.01	0.01\\
184.01	0.01\\
185.01	0.01\\
186.01	0.01\\
187.01	0.01\\
188.01	0.01\\
189.01	0.01\\
190.01	0.01\\
191.01	0.01\\
192.01	0.01\\
193.01	0.01\\
194.01	0.01\\
195.01	0.01\\
196.01	0.01\\
197.01	0.01\\
198.01	0.01\\
199.01	0.01\\
200.01	0.01\\
201.01	0.01\\
202.01	0.01\\
203.01	0.01\\
204.01	0.01\\
205.01	0.01\\
206.01	0.01\\
207.01	0.01\\
208.01	0.01\\
209.01	0.01\\
210.01	0.01\\
211.01	0.01\\
212.01	0.01\\
213.01	0.01\\
214.01	0.01\\
215.01	0.01\\
216.01	0.01\\
217.01	0.01\\
218.01	0.01\\
219.01	0.01\\
220.01	0.01\\
221.01	0.01\\
222.01	0.01\\
223.01	0.01\\
224.01	0.01\\
225.01	0.01\\
226.01	0.01\\
227.01	0.01\\
228.01	0.01\\
229.01	0.01\\
230.01	0.01\\
231.01	0.01\\
232.01	0.01\\
233.01	0.01\\
234.01	0.01\\
235.01	0.01\\
236.01	0.01\\
237.01	0.01\\
238.01	0.01\\
239.01	0.01\\
240.01	0.01\\
241.01	0.01\\
242.01	0.01\\
243.01	0.01\\
244.01	0.01\\
245.01	0.01\\
246.01	0.01\\
247.01	0.01\\
248.01	0.01\\
249.01	0.01\\
250.01	0.01\\
251.01	0.01\\
252.01	0.01\\
253.01	0.01\\
254.01	0.01\\
255.01	0.01\\
256.01	0.01\\
257.01	0.01\\
258.01	0.01\\
259.01	0.01\\
260.01	0.01\\
261.01	0.01\\
262.01	0.01\\
263.01	0.01\\
264.01	0.01\\
265.01	0.01\\
266.01	0.01\\
267.01	0.01\\
268.01	0.01\\
269.01	0.01\\
270.01	0.01\\
271.01	0.01\\
272.01	0.01\\
273.01	0.01\\
274.01	0.01\\
275.01	0.01\\
276.01	0.01\\
277.01	0.01\\
278.01	0.01\\
279.01	0.01\\
280.01	0.01\\
281.01	0.01\\
282.01	0.01\\
283.01	0.01\\
284.01	0.01\\
285.01	0.01\\
286.01	0.01\\
287.01	0.01\\
288.01	0.01\\
289.01	0.01\\
290.01	0.01\\
291.01	0.01\\
292.01	0.01\\
293.01	0.01\\
294.01	0.01\\
295.01	0.01\\
296.01	0.01\\
297.01	0.01\\
298.01	0.01\\
299.01	0.01\\
300.01	0.01\\
301.01	0.01\\
302.01	0.01\\
303.01	0.01\\
304.01	0.01\\
305.01	0.01\\
306.01	0.01\\
307.01	0.01\\
308.01	0.01\\
309.01	0.01\\
310.01	0.01\\
311.01	0.01\\
312.01	0.01\\
313.01	0.01\\
314.01	0.01\\
315.01	0.01\\
316.01	0.01\\
317.01	0.01\\
318.01	0.01\\
319.01	0.01\\
320.01	0.01\\
321.01	0.01\\
322.01	0.01\\
323.01	0.01\\
324.01	0.01\\
325.01	0.01\\
326.01	0.01\\
327.01	0.01\\
328.01	0.01\\
329.01	0.01\\
330.01	0.01\\
331.01	0.01\\
332.01	0.01\\
333.01	0.01\\
334.01	0.01\\
335.01	0.01\\
336.01	0.01\\
337.01	0.01\\
338.01	0.01\\
339.01	0.01\\
340.01	0.01\\
341.01	0.01\\
342.01	0.01\\
343.01	0.01\\
344.01	0.01\\
345.01	0.01\\
346.01	0.01\\
347.01	0.01\\
348.01	0.01\\
349.01	0.01\\
350.01	0.01\\
351.01	0.01\\
352.01	0.01\\
353.01	0.01\\
354.01	0.01\\
355.01	0.01\\
356.01	0.01\\
357.01	0.01\\
358.01	0.01\\
359.01	0.01\\
360.01	0.01\\
361.01	0.01\\
362.01	0.01\\
363.01	0.01\\
364.01	0.01\\
365.01	0.01\\
366.01	0.01\\
367.01	0.01\\
368.01	0.01\\
369.01	0.01\\
370.01	0.01\\
371.01	0.01\\
372.01	0.01\\
373.01	0.01\\
374.01	0.01\\
375.01	0.01\\
376.01	0.01\\
377.01	0.01\\
378.01	0.01\\
379.01	0.01\\
380.01	0.01\\
381.01	0.01\\
382.01	0.01\\
383.01	0.01\\
384.01	0.01\\
385.01	0.01\\
386.01	0.01\\
387.01	0.01\\
388.01	0.01\\
389.01	0.01\\
390.01	0.01\\
391.01	0.01\\
392.01	0.01\\
393.01	0.01\\
394.01	0.01\\
395.01	0.01\\
396.01	0.01\\
397.01	0.01\\
398.01	0.01\\
399.01	0.01\\
400.01	0.01\\
401.01	0.01\\
402.01	0.01\\
403.01	0.01\\
404.01	0.01\\
405.01	0.01\\
406.01	0.01\\
407.01	0.01\\
408.01	0.01\\
409.01	0.01\\
410.01	0.01\\
411.01	0.01\\
412.01	0.01\\
413.01	0.01\\
414.01	0.01\\
415.01	0.01\\
416.01	0.01\\
417.01	0.01\\
418.01	0.01\\
419.01	0.01\\
420.01	0.01\\
421.01	0.01\\
422.01	0.01\\
423.01	0.01\\
424.01	0.01\\
425.01	0.01\\
426.01	0.01\\
427.01	0.01\\
428.01	0.01\\
429.01	0.01\\
430.01	0.01\\
431.01	0.01\\
432.01	0.01\\
433.01	0.01\\
434.01	0.01\\
435.01	0.01\\
436.01	0.01\\
437.01	0.01\\
438.01	0.01\\
439.01	0.01\\
440.01	0.01\\
441.01	0.01\\
442.01	0.01\\
443.01	0.01\\
444.01	0.01\\
445.01	0.01\\
446.01	0.01\\
447.01	0.01\\
448.01	0.01\\
449.01	0.01\\
450.01	0.01\\
451.01	0.01\\
452.01	0.01\\
453.01	0.01\\
454.01	0.01\\
455.01	0.01\\
456.01	0.01\\
457.01	0.01\\
458.01	0.01\\
459.01	0.01\\
460.01	0.01\\
461.01	0.01\\
462.01	0.01\\
463.01	0.01\\
464.01	0.01\\
465.01	0.01\\
466.01	0.01\\
467.01	0.01\\
468.01	0.01\\
469.01	0.01\\
470.01	0.01\\
471.01	0.01\\
472.01	0.01\\
473.01	0.01\\
474.01	0.01\\
475.01	0.01\\
476.01	0.01\\
477.01	0.01\\
478.01	0.01\\
479.01	0.01\\
480.01	0.01\\
481.01	0.01\\
482.01	0.01\\
483.01	0.01\\
484.01	0.01\\
485.01	0.01\\
486.01	0.01\\
487.01	0.01\\
488.01	0.01\\
489.01	0.01\\
490.01	0.01\\
491.01	0.01\\
492.01	0.01\\
493.01	0.01\\
494.01	0.01\\
495.01	0.01\\
496.01	0.01\\
497.01	0.01\\
498.01	0.01\\
499.01	0.01\\
500.01	0.01\\
501.01	0.01\\
502.01	0.01\\
503.01	0.01\\
504.01	0.01\\
505.01	0.01\\
506.01	0.01\\
507.01	0.01\\
508.01	0.01\\
509.01	0.01\\
510.01	0.01\\
511.01	0.01\\
512.01	0.01\\
513.01	0.01\\
514.01	0.01\\
515.01	0.01\\
516.01	0.01\\
517.01	0.01\\
518.01	0.01\\
519.01	0.01\\
520.01	0.01\\
521.01	0.01\\
522.01	0.01\\
523.01	0.01\\
524.01	0.01\\
525.01	0.01\\
526.01	0.01\\
527.01	0.01\\
528.01	0.01\\
529.01	0.01\\
530.01	0.01\\
531.01	0.01\\
532.01	0.01\\
533.01	0.01\\
534.01	0.01\\
535.01	0.01\\
536.01	0.01\\
537.01	0.01\\
538.01	0.01\\
539.01	0.01\\
540.01	0.01\\
541.01	0.01\\
542.01	0.01\\
543.01	0.01\\
544.01	0.01\\
545.01	0.01\\
546.01	0.01\\
547.01	0.01\\
548.01	0.01\\
549.01	0.01\\
550.01	0.01\\
551.01	0.01\\
552.01	0.01\\
553.01	0.01\\
554.01	0.01\\
555.01	0.01\\
556.01	0.01\\
557.01	0.01\\
558.01	0.01\\
559.01	0.01\\
560.01	0.01\\
561.01	0.01\\
562.01	0.01\\
563.01	0.01\\
564.01	0.01\\
565.01	0.01\\
566.01	0.01\\
567.01	0.01\\
568.01	0.01\\
569.01	0.01\\
570.01	0.01\\
571.01	0.01\\
572.01	0.01\\
573.01	0.01\\
574.01	0.01\\
575.01	0.01\\
576.01	0.01\\
577.01	0.01\\
578.01	0.01\\
579.01	0.01\\
580.01	0.01\\
581.01	0.01\\
582.01	0.01\\
583.01	0.01\\
584.01	0.01\\
585.01	0.01\\
586.01	0.01\\
587.01	0.01\\
588.01	0.01\\
589.01	0.01\\
590.01	0.01\\
591.01	0.01\\
592.01	0.01\\
593.01	0.01\\
594.01	0.01\\
595.01	0.01\\
596.01	0.01\\
597.01	0.01\\
598.01	0.00865106781478654\\
599.01	0.00624186909405412\\
599.02	0.00620414441242164\\
599.03	0.00616605275622737\\
599.04	0.00612759052101571\\
599.05	0.00608875406692037\\
599.06	0.00604953971831645\\
599.07	0.00600994376346916\\
599.08	0.00596996245417918\\
599.09	0.00592959200542426\\
599.1	0.0058888285949975\\
599.11	0.00584766836314221\\
599.12	0.00580610741218275\\
599.13	0.00576414180615224\\
599.14	0.00572176757041627\\
599.15	0.00567898069129306\\
599.16	0.00563577711566984\\
599.17	0.00559215275061547\\
599.18	0.00554810346298925\\
599.19	0.00550362507904601\\
599.2	0.00545871338403702\\
599.21	0.0054133641218073\\
599.22	0.0053675729943888\\
599.23	0.00532133566158966\\
599.24	0.00527464774057934\\
599.25	0.00522750480546975\\
599.26	0.00517990238689229\\
599.27	0.00513183597157067\\
599.28	0.00508330100188951\\
599.29	0.00503429287545885\\
599.3	0.00498480694467416\\
599.31	0.00493483851627228\\
599.32	0.00488438285088275\\
599.33	0.00483343516257492\\
599.34	0.00478199061840046\\
599.35	0.0047300443379316\\
599.36	0.00467759139279454\\
599.37	0.00462462680619848\\
599.38	0.00457114555246002\\
599.39	0.00451714256049734\\
599.4	0.00446261272376355\\
599.41	0.00440755088572616\\
599.42	0.00435195183937779\\
599.43	0.00429581032674208\\
599.44	0.00423912103837484\\
599.45	0.00418187861286019\\
599.46	0.00412407763630191\\
599.47	0.0040657126418097\\
599.48	0.00400677810898042\\
599.49	0.00394726846337441\\
599.5	0.00388717807598638\\
599.51	0.00382650126271143\\
599.52	0.00376523228380568\\
599.53	0.00370336534334161\\
599.54	0.00364089458865821\\
599.55	0.00357781410980551\\
599.56	0.00351411793898394\\
599.57	0.00344980004997801\\
599.58	0.00338485435758449\\
599.59	0.00331927471703512\\
599.6	0.00325305492341342\\
599.61	0.00318618871106605\\
599.62	0.00311866975300823\\
599.63	0.00305049166032344\\
599.64	0.00298164798155719\\
599.65	0.00291213220210483\\
599.66	0.00284193774359355\\
599.67	0.002771057963258\\
599.68	0.00269948615331015\\
599.69	0.00262721554030275\\
599.7	0.00255423928448666\\
599.71	0.00248055047916181\\
599.72	0.00240614215002193\\
599.73	0.00233100725449273\\
599.74	0.00225513868106373\\
599.75	0.00217852924861344\\
599.76	0.002101171705728\\
599.77	0.00202305873001311\\
599.78	0.00194418292739928\\
599.79	0.00186453683144024\\
599.8	0.00178411290260446\\
599.81	0.00170290352755979\\
599.82	0.00162090101845109\\
599.83	0.00153809761217073\\
599.84	0.001454485469622\\
599.85	0.00137005667497534\\
599.86	0.00128480323491721\\
599.87	0.00119871707789166\\
599.88	0.00111179005333448\\
599.89	0.00102401393089982\\
599.9	0.000935380399679277\\
599.91	0.000845881067413285\\
599.92	0.00075550745969488\\
599.93	0.000664251019165563\\
599.94	0.000572103104703356\\
599.95	0.000479054990602914\\
599.96	0.000385097865747554\\
599.97	0.000290222832773275\\
599.98	0.000194420907224489\\
599.99	9.76830167015632e-05\\
600	0\\
};
\end{axis}
\end{tikzpicture}% 
%  \caption{Continuous Time w/ nFPC}
%\end{subfigure}%
%\hfill%
%\begin{subfigure}{.45\linewidth}
%  \centering
%  \setlength\figureheight{\linewidth} 
%  \setlength\figurewidth{\linewidth}
%  \tikzsetnextfilename{dm_dscr_nFPC_z1}
%  % This file was created by matlab2tikz.
%
%The latest updates can be retrieved from
%  http://www.mathworks.com/matlabcentral/fileexchange/22022-matlab2tikz-matlab2tikz
%where you can also make suggestions and rate matlab2tikz.
%
\definecolor{mycolor1}{rgb}{0.00000,1.00000,0.14286}%
\definecolor{mycolor2}{rgb}{0.00000,1.00000,0.28571}%
\definecolor{mycolor3}{rgb}{0.00000,1.00000,0.42857}%
\definecolor{mycolor4}{rgb}{0.00000,1.00000,0.57143}%
\definecolor{mycolor5}{rgb}{0.00000,1.00000,0.71429}%
\definecolor{mycolor6}{rgb}{0.00000,1.00000,0.85714}%
\definecolor{mycolor7}{rgb}{0.00000,1.00000,1.00000}%
\definecolor{mycolor8}{rgb}{0.00000,0.87500,1.00000}%
\definecolor{mycolor9}{rgb}{0.00000,0.62500,1.00000}%
\definecolor{mycolor10}{rgb}{0.12500,0.00000,1.00000}%
\definecolor{mycolor11}{rgb}{0.25000,0.00000,1.00000}%
\definecolor{mycolor12}{rgb}{0.37500,0.00000,1.00000}%
\definecolor{mycolor13}{rgb}{0.50000,0.00000,1.00000}%
\definecolor{mycolor14}{rgb}{0.62500,0.00000,1.00000}%
\definecolor{mycolor15}{rgb}{0.75000,0.00000,1.00000}%
\definecolor{mycolor16}{rgb}{0.87500,0.00000,1.00000}%
\definecolor{mycolor17}{rgb}{1.00000,0.00000,1.00000}%
\definecolor{mycolor18}{rgb}{1.00000,0.00000,0.87500}%
\definecolor{mycolor19}{rgb}{1.00000,0.00000,0.62500}%
\definecolor{mycolor20}{rgb}{0.85714,0.00000,0.00000}%
\definecolor{mycolor21}{rgb}{0.71429,0.00000,0.00000}%
%
\begin{tikzpicture}

\begin{axis}[%
width=4.1in,
height=3.803in,
at={(0.809in,0.513in)},
scale only axis,
point meta min=0,
point meta max=1,
every outer x axis line/.append style={black},
every x tick label/.append style={font=\color{black}},
xmin=0,
xmax=600,
every outer y axis line/.append style={black},
every y tick label/.append style={font=\color{black}},
ymin=0,
ymax=0.007,
axis background/.style={fill=white},
axis x line*=bottom,
axis y line*=left,
colormap={mymap}{[1pt] rgb(0pt)=(0,1,0); rgb(7pt)=(0,1,1); rgb(15pt)=(0,0,1); rgb(23pt)=(1,0,1); rgb(31pt)=(1,0,0); rgb(38pt)=(0,0,0)},
colorbar,
colorbar style={separate axis lines,every outer x axis line/.append style={black},every x tick label/.append style={font=\color{black}},every outer y axis line/.append style={black},every y tick label/.append style={font=\color{black}},yticklabels={{-19},{-17},{-15},{-13},{-11},{-9},{-7},{-5},{-3},{-1},{1},{3},{5},{7},{9},{11},{13},{15},{17},{19}}}
]
\addplot [color=green,solid,forget plot]
  table[row sep=crcr]{%
1	0\\
2	0\\
3	0\\
4	0\\
5	0\\
6	0\\
7	0\\
8	0\\
9	0\\
10	0\\
11	0\\
12	0\\
13	0\\
14	0\\
15	0\\
16	0\\
17	0\\
18	0\\
19	0\\
20	0\\
21	0\\
22	0\\
23	0\\
24	0\\
25	0\\
26	0\\
27	0\\
28	0\\
29	0\\
30	0\\
31	0\\
32	0\\
33	0\\
34	0\\
35	0\\
36	0\\
37	0\\
38	0\\
39	0\\
40	0\\
41	0\\
42	0\\
43	0\\
44	0\\
45	0\\
46	0\\
47	0\\
48	0\\
49	0\\
50	0\\
51	0\\
52	0\\
53	0\\
54	0\\
55	0\\
56	0\\
57	0\\
58	0\\
59	0\\
60	0\\
61	0\\
62	0\\
63	0\\
64	0\\
65	0\\
66	0\\
67	0\\
68	0\\
69	0\\
70	0\\
71	0\\
72	0\\
73	0\\
74	0\\
75	0\\
76	0\\
77	0\\
78	0\\
79	0\\
80	0\\
81	0\\
82	0\\
83	0\\
84	0\\
85	0\\
86	0\\
87	0\\
88	0\\
89	0\\
90	0\\
91	0\\
92	0\\
93	0\\
94	0\\
95	0\\
96	0\\
97	0\\
98	0\\
99	0\\
100	0\\
101	0\\
102	0\\
103	0\\
104	0\\
105	0\\
106	0\\
107	0\\
108	0\\
109	0\\
110	0\\
111	0\\
112	0\\
113	0\\
114	0\\
115	0\\
116	0\\
117	0\\
118	0\\
119	0\\
120	0\\
121	0\\
122	0\\
123	0\\
124	0\\
125	0\\
126	0\\
127	0\\
128	0\\
129	0\\
130	0\\
131	0\\
132	0\\
133	0\\
134	0\\
135	0\\
136	0\\
137	0\\
138	0\\
139	0\\
140	0\\
141	0\\
142	0\\
143	0\\
144	0\\
145	0\\
146	0\\
147	0\\
148	0\\
149	0\\
150	0\\
151	0\\
152	0\\
153	0\\
154	0\\
155	0\\
156	0\\
157	0\\
158	0\\
159	0\\
160	0\\
161	0\\
162	0\\
163	0\\
164	0\\
165	0\\
166	0\\
167	0\\
168	0\\
169	0\\
170	0\\
171	0\\
172	0\\
173	0\\
174	0\\
175	0\\
176	0\\
177	0\\
178	0\\
179	0\\
180	0\\
181	0\\
182	0\\
183	0\\
184	0\\
185	0\\
186	0\\
187	0\\
188	0\\
189	0\\
190	0\\
191	0\\
192	0\\
193	0\\
194	0\\
195	0\\
196	0\\
197	0\\
198	0\\
199	0\\
200	0\\
201	0\\
202	0\\
203	0\\
204	0\\
205	0\\
206	0\\
207	0\\
208	0\\
209	0\\
210	0\\
211	0\\
212	0\\
213	0\\
214	0\\
215	0\\
216	0\\
217	0\\
218	0\\
219	0\\
220	0\\
221	0\\
222	0\\
223	0\\
224	0\\
225	0\\
226	0\\
227	0\\
228	0\\
229	0\\
230	0\\
231	0\\
232	0\\
233	0\\
234	0\\
235	0\\
236	0\\
237	0\\
238	0\\
239	0\\
240	0\\
241	0\\
242	0\\
243	0\\
244	0\\
245	0\\
246	0\\
247	0\\
248	0\\
249	0\\
250	0\\
251	0\\
252	0\\
253	0\\
254	0\\
255	0\\
256	0\\
257	0\\
258	0\\
259	0\\
260	0\\
261	0\\
262	0\\
263	0\\
264	0\\
265	0\\
266	0\\
267	0\\
268	0\\
269	0\\
270	0\\
271	0\\
272	0\\
273	0\\
274	0\\
275	0\\
276	0\\
277	0\\
278	0\\
279	0\\
280	0\\
281	0\\
282	0\\
283	0\\
284	0\\
285	0\\
286	0\\
287	0\\
288	0\\
289	0\\
290	0\\
291	0\\
292	0\\
293	0\\
294	0\\
295	0\\
296	0\\
297	0\\
298	0\\
299	0\\
300	0\\
301	0\\
302	0\\
303	0\\
304	0\\
305	0\\
306	0\\
307	0\\
308	0\\
309	0\\
310	0\\
311	0\\
312	0\\
313	0\\
314	0\\
315	0\\
316	0\\
317	0\\
318	0\\
319	0\\
320	0\\
321	0\\
322	0\\
323	0\\
324	0\\
325	0\\
326	0\\
327	0\\
328	0\\
329	0\\
330	0\\
331	0\\
332	0\\
333	0\\
334	0\\
335	0\\
336	0\\
337	0\\
338	0\\
339	0\\
340	0\\
341	0\\
342	0\\
343	0\\
344	0\\
345	0\\
346	0\\
347	0\\
348	0\\
349	0\\
350	0\\
351	0\\
352	0\\
353	0\\
354	0\\
355	0\\
356	0\\
357	0\\
358	0\\
359	0\\
360	0\\
361	0\\
362	0\\
363	0\\
364	0\\
365	0\\
366	0\\
367	0\\
368	0\\
369	0\\
370	0\\
371	0\\
372	0\\
373	0\\
374	0\\
375	0\\
376	0\\
377	0\\
378	0\\
379	0\\
380	0\\
381	0\\
382	0\\
383	0\\
384	0\\
385	0\\
386	0\\
387	0\\
388	0\\
389	0\\
390	0\\
391	0\\
392	0\\
393	0\\
394	0\\
395	0\\
396	0\\
397	0\\
398	0\\
399	0\\
400	0\\
401	0\\
402	0\\
403	0\\
404	0\\
405	0\\
406	0\\
407	0\\
408	0\\
409	0\\
410	0\\
411	0\\
412	0\\
413	0\\
414	0\\
415	0\\
416	0\\
417	0\\
418	0\\
419	0\\
420	0\\
421	0\\
422	0\\
423	0\\
424	0\\
425	0\\
426	0\\
427	0\\
428	0\\
429	0\\
430	0\\
431	0\\
432	0\\
433	0\\
434	0\\
435	0\\
436	0\\
437	0\\
438	0\\
439	0\\
440	0\\
441	0\\
442	0\\
443	0\\
444	0\\
445	0\\
446	0\\
447	0\\
448	0\\
449	0\\
450	0\\
451	0\\
452	0\\
453	0\\
454	0\\
455	0\\
456	0\\
457	0\\
458	0\\
459	0\\
460	0\\
461	0\\
462	0\\
463	0\\
464	0\\
465	0\\
466	0\\
467	0\\
468	0\\
469	0\\
470	0\\
471	0\\
472	0\\
473	0\\
474	0\\
475	0\\
476	0\\
477	0\\
478	0\\
479	0\\
480	0\\
481	0\\
482	0\\
483	0\\
484	0\\
485	0\\
486	0\\
487	0\\
488	0\\
489	0\\
490	0\\
491	0\\
492	0\\
493	0\\
494	0\\
495	0\\
496	0\\
497	0\\
498	0\\
499	0\\
500	0\\
501	0\\
502	0\\
503	0\\
504	0\\
505	0\\
506	0\\
507	0\\
508	0\\
509	0\\
510	0\\
511	0\\
512	0\\
513	0\\
514	0\\
515	0\\
516	0\\
517	0\\
518	0\\
519	0\\
520	0\\
521	0\\
522	0\\
523	0\\
524	0\\
525	0\\
526	0\\
527	0\\
528	0\\
529	0\\
530	0\\
531	0\\
532	0\\
533	0\\
534	0\\
535	0\\
536	0\\
537	0\\
538	0\\
539	1.95527641225376e-05\\
540	4.5298850109529e-05\\
541	7.16515787047177e-05\\
542	9.87260037163583e-05\\
543	0.000126597716204775\\
544	0.000154828213300624\\
545	0.000183516284188123\\
546	0.000212652451694723\\
547	0.000242445887004352\\
548	0.000272977208656028\\
549	0.000304273916678095\\
550	0.000336360067194731\\
551	0.000369260162993882\\
552	0.000402999655010358\\
553	0.000437605098067932\\
554	0.000473104294736734\\
555	0.000509526629431762\\
556	0.000547836300656453\\
557	0.00058668916353381\\
558	0.00062520296878696\\
559	0.000663720153911229\\
560	0.00070323559382204\\
561	0.000743809025123238\\
562	0.000785486013167821\\
563	0.000865379867363339\\
564	0.00126850553815116\\
565	0.00149693684286139\\
566	0.00155900187926996\\
567	0.00162212034359134\\
568	0.00168634680537145\\
569	0.00175171104531127\\
570	0.00181824424140386\\
571	0.00188597918426003\\
572	0.00195495039828782\\
573	0.00202519426953807\\
574	0.0020967491834911\\
575	0.00216965567387592\\
576	0.00224395658348052\\
577	0.00231969723802151\\
578	0.0023969256343423\\
579	0.00247569264466642\\
580	0.00255605223972909\\
581	0.00263806173637132\\
582	0.00272178208245877\\
583	0.00280727821124775\\
584	0.00289461954867252\\
585	0.00298388089415187\\
586	0.00307514426197734\\
587	0.00316850325005475\\
588	0.00326407412228819\\
589	0.00336202479534929\\
590	0.00346265165082637\\
591	0.00356658418066895\\
592	0.00367533141522135\\
593	0.00379274227372929\\
594	0.0039289098819834\\
595	0.00411043972693672\\
596	0.00440815817174734\\
597	0.00501118703192877\\
598	0.00642488516645657\\
599	0\\
600	0\\
};
\addplot [color=mycolor1,solid,forget plot]
  table[row sep=crcr]{%
1	0\\
2	0\\
3	0\\
4	0\\
5	0\\
6	0\\
7	0\\
8	0\\
9	0\\
10	0\\
11	0\\
12	0\\
13	0\\
14	0\\
15	0\\
16	0\\
17	0\\
18	0\\
19	0\\
20	0\\
21	0\\
22	0\\
23	0\\
24	0\\
25	0\\
26	0\\
27	0\\
28	0\\
29	0\\
30	0\\
31	0\\
32	0\\
33	0\\
34	0\\
35	0\\
36	0\\
37	0\\
38	0\\
39	0\\
40	0\\
41	0\\
42	0\\
43	0\\
44	0\\
45	0\\
46	0\\
47	0\\
48	0\\
49	0\\
50	0\\
51	0\\
52	0\\
53	0\\
54	0\\
55	0\\
56	0\\
57	0\\
58	0\\
59	0\\
60	0\\
61	0\\
62	0\\
63	0\\
64	0\\
65	0\\
66	0\\
67	0\\
68	0\\
69	0\\
70	0\\
71	0\\
72	0\\
73	0\\
74	0\\
75	0\\
76	0\\
77	0\\
78	0\\
79	0\\
80	0\\
81	0\\
82	0\\
83	0\\
84	0\\
85	0\\
86	0\\
87	0\\
88	0\\
89	0\\
90	0\\
91	0\\
92	0\\
93	0\\
94	0\\
95	0\\
96	0\\
97	0\\
98	0\\
99	0\\
100	0\\
101	0\\
102	0\\
103	0\\
104	0\\
105	0\\
106	0\\
107	0\\
108	0\\
109	0\\
110	0\\
111	0\\
112	0\\
113	0\\
114	0\\
115	0\\
116	0\\
117	0\\
118	0\\
119	0\\
120	0\\
121	0\\
122	0\\
123	0\\
124	0\\
125	0\\
126	0\\
127	0\\
128	0\\
129	0\\
130	0\\
131	0\\
132	0\\
133	0\\
134	0\\
135	0\\
136	0\\
137	0\\
138	0\\
139	0\\
140	0\\
141	0\\
142	0\\
143	0\\
144	0\\
145	0\\
146	0\\
147	0\\
148	0\\
149	0\\
150	0\\
151	0\\
152	0\\
153	0\\
154	0\\
155	0\\
156	0\\
157	0\\
158	0\\
159	0\\
160	0\\
161	0\\
162	0\\
163	0\\
164	0\\
165	0\\
166	0\\
167	0\\
168	0\\
169	0\\
170	0\\
171	0\\
172	0\\
173	0\\
174	0\\
175	0\\
176	0\\
177	0\\
178	0\\
179	0\\
180	0\\
181	0\\
182	0\\
183	0\\
184	0\\
185	0\\
186	0\\
187	0\\
188	0\\
189	0\\
190	0\\
191	0\\
192	0\\
193	0\\
194	0\\
195	0\\
196	0\\
197	0\\
198	0\\
199	0\\
200	0\\
201	0\\
202	0\\
203	0\\
204	0\\
205	0\\
206	0\\
207	0\\
208	0\\
209	0\\
210	0\\
211	0\\
212	0\\
213	0\\
214	0\\
215	0\\
216	0\\
217	0\\
218	0\\
219	0\\
220	0\\
221	0\\
222	0\\
223	0\\
224	0\\
225	0\\
226	0\\
227	0\\
228	0\\
229	0\\
230	0\\
231	0\\
232	0\\
233	0\\
234	0\\
235	0\\
236	0\\
237	0\\
238	0\\
239	0\\
240	0\\
241	0\\
242	0\\
243	0\\
244	0\\
245	0\\
246	0\\
247	0\\
248	0\\
249	0\\
250	0\\
251	0\\
252	0\\
253	0\\
254	0\\
255	0\\
256	0\\
257	0\\
258	0\\
259	0\\
260	0\\
261	0\\
262	0\\
263	0\\
264	0\\
265	0\\
266	0\\
267	0\\
268	0\\
269	0\\
270	0\\
271	0\\
272	0\\
273	0\\
274	0\\
275	0\\
276	0\\
277	0\\
278	0\\
279	0\\
280	0\\
281	0\\
282	0\\
283	0\\
284	0\\
285	0\\
286	0\\
287	0\\
288	0\\
289	0\\
290	0\\
291	0\\
292	0\\
293	0\\
294	0\\
295	0\\
296	0\\
297	0\\
298	0\\
299	0\\
300	0\\
301	0\\
302	0\\
303	0\\
304	0\\
305	0\\
306	0\\
307	0\\
308	0\\
309	0\\
310	0\\
311	0\\
312	0\\
313	0\\
314	0\\
315	0\\
316	0\\
317	0\\
318	0\\
319	0\\
320	0\\
321	0\\
322	0\\
323	0\\
324	0\\
325	0\\
326	0\\
327	0\\
328	0\\
329	0\\
330	0\\
331	0\\
332	0\\
333	0\\
334	0\\
335	0\\
336	0\\
337	0\\
338	0\\
339	0\\
340	0\\
341	0\\
342	0\\
343	0\\
344	0\\
345	0\\
346	0\\
347	0\\
348	0\\
349	0\\
350	0\\
351	0\\
352	0\\
353	0\\
354	0\\
355	0\\
356	0\\
357	0\\
358	0\\
359	0\\
360	0\\
361	0\\
362	0\\
363	0\\
364	0\\
365	0\\
366	0\\
367	0\\
368	0\\
369	0\\
370	0\\
371	0\\
372	0\\
373	0\\
374	0\\
375	0\\
376	0\\
377	0\\
378	0\\
379	0\\
380	0\\
381	0\\
382	0\\
383	0\\
384	0\\
385	0\\
386	0\\
387	0\\
388	0\\
389	0\\
390	0\\
391	0\\
392	0\\
393	0\\
394	0\\
395	0\\
396	0\\
397	0\\
398	0\\
399	0\\
400	0\\
401	0\\
402	0\\
403	0\\
404	0\\
405	0\\
406	0\\
407	0\\
408	0\\
409	0\\
410	0\\
411	0\\
412	0\\
413	0\\
414	0\\
415	0\\
416	0\\
417	0\\
418	0\\
419	0\\
420	0\\
421	0\\
422	0\\
423	0\\
424	0\\
425	0\\
426	0\\
427	0\\
428	0\\
429	0\\
430	0\\
431	0\\
432	0\\
433	0\\
434	0\\
435	0\\
436	0\\
437	0\\
438	0\\
439	0\\
440	0\\
441	0\\
442	0\\
443	0\\
444	0\\
445	0\\
446	0\\
447	0\\
448	0\\
449	0\\
450	0\\
451	0\\
452	0\\
453	0\\
454	0\\
455	0\\
456	0\\
457	0\\
458	0\\
459	0\\
460	0\\
461	0\\
462	0\\
463	0\\
464	0\\
465	0\\
466	0\\
467	0\\
468	0\\
469	0\\
470	0\\
471	0\\
472	0\\
473	0\\
474	0\\
475	0\\
476	0\\
477	0\\
478	0\\
479	0\\
480	0\\
481	0\\
482	0\\
483	0\\
484	0\\
485	0\\
486	0\\
487	0\\
488	0\\
489	0\\
490	0\\
491	0\\
492	0\\
493	0\\
494	0\\
495	0\\
496	0\\
497	0\\
498	0\\
499	0\\
500	0\\
501	0\\
502	0\\
503	0\\
504	0\\
505	0\\
506	0\\
507	0\\
508	0\\
509	0\\
510	0\\
511	0\\
512	0\\
513	0\\
514	0\\
515	0\\
516	0\\
517	0\\
518	0\\
519	0\\
520	0\\
521	0\\
522	0\\
523	0\\
524	0\\
525	0\\
526	0\\
527	0\\
528	0\\
529	0\\
530	0\\
531	0\\
532	0\\
533	0\\
534	0\\
535	0\\
536	0\\
537	0\\
538	0\\
539	1.37376150958274e-05\\
540	3.93767248397528e-05\\
541	6.56239748089781e-05\\
542	9.24933624265191e-05\\
543	0.000119999855829102\\
544	0.000148363699355699\\
545	0.000177377641200535\\
546	0.000206774145788462\\
547	0.000236720771405114\\
548	0.000267152374965129\\
549	0.00029829523360433\\
550	0.000330215002525051\\
551	0.000362941523011504\\
552	0.000396501057709032\\
553	0.000430920114469899\\
554	0.000466226213023714\\
555	0.000502448117088533\\
556	0.000539615657443634\\
557	0.000578457004892515\\
558	0.000618368790079295\\
559	0.000658014352548393\\
560	0.0006974686375873\\
561	0.000737812075177008\\
562	0.000779241486122308\\
563	0.000821805097695177\\
564	0.000946087951490863\\
565	0.00132524351514214\\
566	0.00155876358354603\\
567	0.0016221192151821\\
568	0.00168634675322425\\
569	0.00175171103773517\\
570	0.00181824423981386\\
571	0.00188597918379922\\
572	0.00195495039810385\\
573	0.00202519426944808\\
574	0.0020967491834472\\
575	0.00216965567385279\\
576	0.00224395658346852\\
577	0.00231969723801546\\
578	0.00239692563433929\\
579	0.00247569264466521\\
580	0.00255605223972864\\
581	0.00263806173637133\\
582	0.00272178208245879\\
583	0.00280727821124775\\
584	0.00289461954867253\\
585	0.00298388089415188\\
586	0.00307514426197735\\
587	0.00316850325005475\\
588	0.00326407412228818\\
589	0.0033620247953493\\
590	0.00346265165082638\\
591	0.00356658418066896\\
592	0.00367533141522136\\
593	0.00379274227372929\\
594	0.00392890988198341\\
595	0.00411043972693672\\
596	0.00440815817174735\\
597	0.00501118703192878\\
598	0.00642488516645657\\
599	0\\
600	0\\
};
\addplot [color=mycolor2,solid,forget plot]
  table[row sep=crcr]{%
1	0\\
2	0\\
3	0\\
4	0\\
5	0\\
6	0\\
7	0\\
8	0\\
9	0\\
10	0\\
11	0\\
12	0\\
13	0\\
14	0\\
15	0\\
16	0\\
17	0\\
18	0\\
19	0\\
20	0\\
21	0\\
22	0\\
23	0\\
24	0\\
25	0\\
26	0\\
27	0\\
28	0\\
29	0\\
30	0\\
31	0\\
32	0\\
33	0\\
34	0\\
35	0\\
36	0\\
37	0\\
38	0\\
39	0\\
40	0\\
41	0\\
42	0\\
43	0\\
44	0\\
45	0\\
46	0\\
47	0\\
48	0\\
49	0\\
50	0\\
51	0\\
52	0\\
53	0\\
54	0\\
55	0\\
56	0\\
57	0\\
58	0\\
59	0\\
60	0\\
61	0\\
62	0\\
63	0\\
64	0\\
65	0\\
66	0\\
67	0\\
68	0\\
69	0\\
70	0\\
71	0\\
72	0\\
73	0\\
74	0\\
75	0\\
76	0\\
77	0\\
78	0\\
79	0\\
80	0\\
81	0\\
82	0\\
83	0\\
84	0\\
85	0\\
86	0\\
87	0\\
88	0\\
89	0\\
90	0\\
91	0\\
92	0\\
93	0\\
94	0\\
95	0\\
96	0\\
97	0\\
98	0\\
99	0\\
100	0\\
101	0\\
102	0\\
103	0\\
104	0\\
105	0\\
106	0\\
107	0\\
108	0\\
109	0\\
110	0\\
111	0\\
112	0\\
113	0\\
114	0\\
115	0\\
116	0\\
117	0\\
118	0\\
119	0\\
120	0\\
121	0\\
122	0\\
123	0\\
124	0\\
125	0\\
126	0\\
127	0\\
128	0\\
129	0\\
130	0\\
131	0\\
132	0\\
133	0\\
134	0\\
135	0\\
136	0\\
137	0\\
138	0\\
139	0\\
140	0\\
141	0\\
142	0\\
143	0\\
144	0\\
145	0\\
146	0\\
147	0\\
148	0\\
149	0\\
150	0\\
151	0\\
152	0\\
153	0\\
154	0\\
155	0\\
156	0\\
157	0\\
158	0\\
159	0\\
160	0\\
161	0\\
162	0\\
163	0\\
164	0\\
165	0\\
166	0\\
167	0\\
168	0\\
169	0\\
170	0\\
171	0\\
172	0\\
173	0\\
174	0\\
175	0\\
176	0\\
177	0\\
178	0\\
179	0\\
180	0\\
181	0\\
182	0\\
183	0\\
184	0\\
185	0\\
186	0\\
187	0\\
188	0\\
189	0\\
190	0\\
191	0\\
192	0\\
193	0\\
194	0\\
195	0\\
196	0\\
197	0\\
198	0\\
199	0\\
200	0\\
201	0\\
202	0\\
203	0\\
204	0\\
205	0\\
206	0\\
207	0\\
208	0\\
209	0\\
210	0\\
211	0\\
212	0\\
213	0\\
214	0\\
215	0\\
216	0\\
217	0\\
218	0\\
219	0\\
220	0\\
221	0\\
222	0\\
223	0\\
224	0\\
225	0\\
226	0\\
227	0\\
228	0\\
229	0\\
230	0\\
231	0\\
232	0\\
233	0\\
234	0\\
235	0\\
236	0\\
237	0\\
238	0\\
239	0\\
240	0\\
241	0\\
242	0\\
243	0\\
244	0\\
245	0\\
246	0\\
247	0\\
248	0\\
249	0\\
250	0\\
251	0\\
252	0\\
253	0\\
254	0\\
255	0\\
256	0\\
257	0\\
258	0\\
259	0\\
260	0\\
261	0\\
262	0\\
263	0\\
264	0\\
265	0\\
266	0\\
267	0\\
268	0\\
269	0\\
270	0\\
271	0\\
272	0\\
273	0\\
274	0\\
275	0\\
276	0\\
277	0\\
278	0\\
279	0\\
280	0\\
281	0\\
282	0\\
283	0\\
284	0\\
285	0\\
286	0\\
287	0\\
288	0\\
289	0\\
290	0\\
291	0\\
292	0\\
293	0\\
294	0\\
295	0\\
296	0\\
297	0\\
298	0\\
299	0\\
300	0\\
301	0\\
302	0\\
303	0\\
304	0\\
305	0\\
306	0\\
307	0\\
308	0\\
309	0\\
310	0\\
311	0\\
312	0\\
313	0\\
314	0\\
315	0\\
316	0\\
317	0\\
318	0\\
319	0\\
320	0\\
321	0\\
322	0\\
323	0\\
324	0\\
325	0\\
326	0\\
327	0\\
328	0\\
329	0\\
330	0\\
331	0\\
332	0\\
333	0\\
334	0\\
335	0\\
336	0\\
337	0\\
338	0\\
339	0\\
340	0\\
341	0\\
342	0\\
343	0\\
344	0\\
345	0\\
346	0\\
347	0\\
348	0\\
349	0\\
350	0\\
351	0\\
352	0\\
353	0\\
354	0\\
355	0\\
356	0\\
357	0\\
358	0\\
359	0\\
360	0\\
361	0\\
362	0\\
363	0\\
364	0\\
365	0\\
366	0\\
367	0\\
368	0\\
369	0\\
370	0\\
371	0\\
372	0\\
373	0\\
374	0\\
375	0\\
376	0\\
377	0\\
378	0\\
379	0\\
380	0\\
381	0\\
382	0\\
383	0\\
384	0\\
385	0\\
386	0\\
387	0\\
388	0\\
389	0\\
390	0\\
391	0\\
392	0\\
393	0\\
394	0\\
395	0\\
396	0\\
397	0\\
398	0\\
399	0\\
400	0\\
401	0\\
402	0\\
403	0\\
404	0\\
405	0\\
406	0\\
407	0\\
408	0\\
409	0\\
410	0\\
411	0\\
412	0\\
413	0\\
414	0\\
415	0\\
416	0\\
417	0\\
418	0\\
419	0\\
420	0\\
421	0\\
422	0\\
423	0\\
424	0\\
425	0\\
426	0\\
427	0\\
428	0\\
429	0\\
430	0\\
431	0\\
432	0\\
433	0\\
434	0\\
435	0\\
436	0\\
437	0\\
438	0\\
439	0\\
440	0\\
441	0\\
442	0\\
443	0\\
444	0\\
445	0\\
446	0\\
447	0\\
448	0\\
449	0\\
450	0\\
451	0\\
452	0\\
453	0\\
454	0\\
455	0\\
456	0\\
457	0\\
458	0\\
459	0\\
460	0\\
461	0\\
462	0\\
463	0\\
464	0\\
465	0\\
466	0\\
467	0\\
468	0\\
469	0\\
470	0\\
471	0\\
472	0\\
473	0\\
474	0\\
475	0\\
476	0\\
477	0\\
478	0\\
479	0\\
480	0\\
481	0\\
482	0\\
483	0\\
484	0\\
485	0\\
486	0\\
487	0\\
488	0\\
489	0\\
490	0\\
491	0\\
492	0\\
493	0\\
494	0\\
495	0\\
496	0\\
497	0\\
498	0\\
499	0\\
500	0\\
501	0\\
502	0\\
503	0\\
504	0\\
505	0\\
506	0\\
507	0\\
508	0\\
509	0\\
510	0\\
511	0\\
512	0\\
513	0\\
514	0\\
515	0\\
516	0\\
517	0\\
518	0\\
519	0\\
520	0\\
521	0\\
522	0\\
523	0\\
524	0\\
525	0\\
526	0\\
527	0\\
528	0\\
529	0\\
530	0\\
531	0\\
532	0\\
533	0\\
534	0\\
535	0\\
536	0\\
537	0\\
538	0\\
539	6.8109836946266e-06\\
540	3.2333550628703e-05\\
541	5.84639163104344e-05\\
542	8.52179610591799e-05\\
543	0.000112610529629426\\
544	0.000140656593656372\\
545	0.000169372996211992\\
546	0.000199020774807513\\
547	0.000229285591716688\\
548	0.00025996464680747\\
549	0.000291246962245206\\
550	0.000323061482549782\\
551	0.000355620274831104\\
552	0.000388995178994957\\
553	0.000423221419650544\\
554	0.000458328095444309\\
555	0.000494344032598358\\
556	0.000531299105939698\\
557	0.000569224327827854\\
558	0.000608484400997936\\
559	0.000649558790110594\\
560	0.000690283174206512\\
561	0.000731090493962589\\
562	0.000772286971587725\\
563	0.000814588754116958\\
564	0.000858056127894556\\
565	0.000992351393135649\\
566	0.00136836372751276\\
567	0.00162000548846045\\
568	0.00168633694558227\\
569	0.00175171060027825\\
570	0.001818244178077\\
571	0.00188597917131468\\
572	0.00195495039465754\\
573	0.00202519426812732\\
574	0.00209674918280833\\
575	0.00216965567354768\\
576	0.00224395658330853\\
577	0.00231969723793263\\
578	0.00239692563429661\\
579	0.00247569264464508\\
580	0.00255605223972036\\
581	0.00263806173636886\\
582	0.00272178208245834\\
583	0.00280727821124774\\
584	0.00289461954867251\\
585	0.00298388089415188\\
586	0.00307514426197732\\
587	0.00316850325005475\\
588	0.00326407412228818\\
589	0.0033620247953493\\
590	0.00346265165082637\\
591	0.00356658418066895\\
592	0.00367533141522134\\
593	0.00379274227372929\\
594	0.00392890988198339\\
595	0.00411043972693671\\
596	0.00440815817174734\\
597	0.00501118703192877\\
598	0.00642488516645657\\
599	0\\
600	0\\
};
\addplot [color=mycolor3,solid,forget plot]
  table[row sep=crcr]{%
1	0\\
2	0\\
3	0\\
4	0\\
5	0\\
6	0\\
7	0\\
8	0\\
9	0\\
10	0\\
11	0\\
12	0\\
13	0\\
14	0\\
15	0\\
16	0\\
17	0\\
18	0\\
19	0\\
20	0\\
21	0\\
22	0\\
23	0\\
24	0\\
25	0\\
26	0\\
27	0\\
28	0\\
29	0\\
30	0\\
31	0\\
32	0\\
33	0\\
34	0\\
35	0\\
36	0\\
37	0\\
38	0\\
39	0\\
40	0\\
41	0\\
42	0\\
43	0\\
44	0\\
45	0\\
46	0\\
47	0\\
48	0\\
49	0\\
50	0\\
51	0\\
52	0\\
53	0\\
54	0\\
55	0\\
56	0\\
57	0\\
58	0\\
59	0\\
60	0\\
61	0\\
62	0\\
63	0\\
64	0\\
65	0\\
66	0\\
67	0\\
68	0\\
69	0\\
70	0\\
71	0\\
72	0\\
73	0\\
74	0\\
75	0\\
76	0\\
77	0\\
78	0\\
79	0\\
80	0\\
81	0\\
82	0\\
83	0\\
84	0\\
85	0\\
86	0\\
87	0\\
88	0\\
89	0\\
90	0\\
91	0\\
92	0\\
93	0\\
94	0\\
95	0\\
96	0\\
97	0\\
98	0\\
99	0\\
100	0\\
101	0\\
102	0\\
103	0\\
104	0\\
105	0\\
106	0\\
107	0\\
108	0\\
109	0\\
110	0\\
111	0\\
112	0\\
113	0\\
114	0\\
115	0\\
116	0\\
117	0\\
118	0\\
119	0\\
120	0\\
121	0\\
122	0\\
123	0\\
124	0\\
125	0\\
126	0\\
127	0\\
128	0\\
129	0\\
130	0\\
131	0\\
132	0\\
133	0\\
134	0\\
135	0\\
136	0\\
137	0\\
138	0\\
139	0\\
140	0\\
141	0\\
142	0\\
143	0\\
144	0\\
145	0\\
146	0\\
147	0\\
148	0\\
149	0\\
150	0\\
151	0\\
152	0\\
153	0\\
154	0\\
155	0\\
156	0\\
157	0\\
158	0\\
159	0\\
160	0\\
161	0\\
162	0\\
163	0\\
164	0\\
165	0\\
166	0\\
167	0\\
168	0\\
169	0\\
170	0\\
171	0\\
172	0\\
173	0\\
174	0\\
175	0\\
176	0\\
177	0\\
178	0\\
179	0\\
180	0\\
181	0\\
182	0\\
183	0\\
184	0\\
185	0\\
186	0\\
187	0\\
188	0\\
189	0\\
190	0\\
191	0\\
192	0\\
193	0\\
194	0\\
195	0\\
196	0\\
197	0\\
198	0\\
199	0\\
200	0\\
201	0\\
202	0\\
203	0\\
204	0\\
205	0\\
206	0\\
207	0\\
208	0\\
209	0\\
210	0\\
211	0\\
212	0\\
213	0\\
214	0\\
215	0\\
216	0\\
217	0\\
218	0\\
219	0\\
220	0\\
221	0\\
222	0\\
223	0\\
224	0\\
225	0\\
226	0\\
227	0\\
228	0\\
229	0\\
230	0\\
231	0\\
232	0\\
233	0\\
234	0\\
235	0\\
236	0\\
237	0\\
238	0\\
239	0\\
240	0\\
241	0\\
242	0\\
243	0\\
244	0\\
245	0\\
246	0\\
247	0\\
248	0\\
249	0\\
250	0\\
251	0\\
252	0\\
253	0\\
254	0\\
255	0\\
256	0\\
257	0\\
258	0\\
259	0\\
260	0\\
261	0\\
262	0\\
263	0\\
264	0\\
265	0\\
266	0\\
267	0\\
268	0\\
269	0\\
270	0\\
271	0\\
272	0\\
273	0\\
274	0\\
275	0\\
276	0\\
277	0\\
278	0\\
279	0\\
280	0\\
281	0\\
282	0\\
283	0\\
284	0\\
285	0\\
286	0\\
287	0\\
288	0\\
289	0\\
290	0\\
291	0\\
292	0\\
293	0\\
294	0\\
295	0\\
296	0\\
297	0\\
298	0\\
299	0\\
300	0\\
301	0\\
302	0\\
303	0\\
304	0\\
305	0\\
306	0\\
307	0\\
308	0\\
309	0\\
310	0\\
311	0\\
312	0\\
313	0\\
314	0\\
315	0\\
316	0\\
317	0\\
318	0\\
319	0\\
320	0\\
321	0\\
322	0\\
323	0\\
324	0\\
325	0\\
326	0\\
327	0\\
328	0\\
329	0\\
330	0\\
331	0\\
332	0\\
333	0\\
334	0\\
335	0\\
336	0\\
337	0\\
338	0\\
339	0\\
340	0\\
341	0\\
342	0\\
343	0\\
344	0\\
345	0\\
346	0\\
347	0\\
348	0\\
349	0\\
350	0\\
351	0\\
352	0\\
353	0\\
354	0\\
355	0\\
356	0\\
357	0\\
358	0\\
359	0\\
360	0\\
361	0\\
362	0\\
363	0\\
364	0\\
365	0\\
366	0\\
367	0\\
368	0\\
369	0\\
370	0\\
371	0\\
372	0\\
373	0\\
374	0\\
375	0\\
376	0\\
377	0\\
378	0\\
379	0\\
380	0\\
381	0\\
382	0\\
383	0\\
384	0\\
385	0\\
386	0\\
387	0\\
388	0\\
389	0\\
390	0\\
391	0\\
392	0\\
393	0\\
394	0\\
395	0\\
396	0\\
397	0\\
398	0\\
399	0\\
400	0\\
401	0\\
402	0\\
403	0\\
404	0\\
405	0\\
406	0\\
407	0\\
408	0\\
409	0\\
410	0\\
411	0\\
412	0\\
413	0\\
414	0\\
415	0\\
416	0\\
417	0\\
418	0\\
419	0\\
420	0\\
421	0\\
422	0\\
423	0\\
424	0\\
425	0\\
426	0\\
427	0\\
428	0\\
429	0\\
430	0\\
431	0\\
432	0\\
433	0\\
434	0\\
435	0\\
436	0\\
437	0\\
438	0\\
439	0\\
440	0\\
441	0\\
442	0\\
443	0\\
444	0\\
445	0\\
446	0\\
447	0\\
448	0\\
449	0\\
450	0\\
451	0\\
452	0\\
453	0\\
454	0\\
455	0\\
456	0\\
457	0\\
458	0\\
459	0\\
460	0\\
461	0\\
462	0\\
463	0\\
464	0\\
465	0\\
466	0\\
467	0\\
468	0\\
469	0\\
470	0\\
471	0\\
472	0\\
473	0\\
474	0\\
475	0\\
476	0\\
477	0\\
478	0\\
479	0\\
480	0\\
481	0\\
482	0\\
483	0\\
484	0\\
485	0\\
486	0\\
487	0\\
488	0\\
489	0\\
490	0\\
491	0\\
492	0\\
493	0\\
494	0\\
495	0\\
496	0\\
497	0\\
498	0\\
499	0\\
500	0\\
501	0\\
502	0\\
503	0\\
504	0\\
505	0\\
506	0\\
507	0\\
508	0\\
509	0\\
510	0\\
511	0\\
512	0\\
513	0\\
514	0\\
515	0\\
516	0\\
517	0\\
518	0\\
519	0\\
520	0\\
521	0\\
522	0\\
523	0\\
524	0\\
525	0\\
526	0\\
527	0\\
528	0\\
529	0\\
530	0\\
531	0\\
532	0\\
533	0\\
534	0\\
535	0\\
536	0\\
537	0\\
538	0\\
539	0\\
540	2.45283203031442e-05\\
541	5.05305641840116e-05\\
542	7.7151130188513e-05\\
543	0.000104409775999626\\
544	0.000132322707189817\\
545	0.000160906048590909\\
546	0.000190176201071463\\
547	0.000220151473717069\\
548	0.000251106602530764\\
549	0.000282689628183228\\
550	0.00031473882969767\\
551	0.000347426881058723\\
552	0.000380710692019464\\
553	0.000414748901432071\\
554	0.000449641734149903\\
555	0.000485433046687176\\
556	0.000522155524764944\\
557	0.000559840336988405\\
558	0.000598520034534356\\
559	0.00063823058415102\\
560	0.000680175666195175\\
561	0.000722313760005713\\
562	0.000764531447846155\\
563	0.000806672231281403\\
564	0.000849854681485079\\
565	0.000894229023457877\\
566	0.00101511639260571\\
567	0.00139640472273508\\
568	0.00166773442770165\\
569	0.00175162603362615\\
570	0.00181824053534768\\
571	0.00188597867140269\\
572	0.00195495029706772\\
573	0.0020251942424614\\
574	0.00209674917340268\\
575	0.00216965566904425\\
576	0.00224395658121081\\
577	0.00231969723683898\\
578	0.0023969256337334\\
579	0.00247569264435266\\
580	0.00255605223958102\\
581	0.00263806173630904\\
582	0.00272178208244085\\
583	0.0028072782112451\\
584	0.00289461954867251\\
585	0.00298388089415188\\
586	0.00307514426197733\\
587	0.00316850325005475\\
588	0.00326407412228818\\
589	0.00336202479534929\\
590	0.00346265165082637\\
591	0.00356658418066895\\
592	0.00367533141522135\\
593	0.00379274227372929\\
594	0.0039289098819834\\
595	0.00411043972693672\\
596	0.00440815817174734\\
597	0.00501118703192877\\
598	0.00642488516645657\\
599	0\\
600	0\\
};
\addplot [color=mycolor4,solid,forget plot]
  table[row sep=crcr]{%
1	0\\
2	0\\
3	0\\
4	0\\
5	0\\
6	0\\
7	0\\
8	0\\
9	0\\
10	0\\
11	0\\
12	0\\
13	0\\
14	0\\
15	0\\
16	0\\
17	0\\
18	0\\
19	0\\
20	0\\
21	0\\
22	0\\
23	0\\
24	0\\
25	0\\
26	0\\
27	0\\
28	0\\
29	0\\
30	0\\
31	0\\
32	0\\
33	0\\
34	0\\
35	0\\
36	0\\
37	0\\
38	0\\
39	0\\
40	0\\
41	0\\
42	0\\
43	0\\
44	0\\
45	0\\
46	0\\
47	0\\
48	0\\
49	0\\
50	0\\
51	0\\
52	0\\
53	0\\
54	0\\
55	0\\
56	0\\
57	0\\
58	0\\
59	0\\
60	0\\
61	0\\
62	0\\
63	0\\
64	0\\
65	0\\
66	0\\
67	0\\
68	0\\
69	0\\
70	0\\
71	0\\
72	0\\
73	0\\
74	0\\
75	0\\
76	0\\
77	0\\
78	0\\
79	0\\
80	0\\
81	0\\
82	0\\
83	0\\
84	0\\
85	0\\
86	0\\
87	0\\
88	0\\
89	0\\
90	0\\
91	0\\
92	0\\
93	0\\
94	0\\
95	0\\
96	0\\
97	0\\
98	0\\
99	0\\
100	0\\
101	0\\
102	0\\
103	0\\
104	0\\
105	0\\
106	0\\
107	0\\
108	0\\
109	0\\
110	0\\
111	0\\
112	0\\
113	0\\
114	0\\
115	0\\
116	0\\
117	0\\
118	0\\
119	0\\
120	0\\
121	0\\
122	0\\
123	0\\
124	0\\
125	0\\
126	0\\
127	0\\
128	0\\
129	0\\
130	0\\
131	0\\
132	0\\
133	0\\
134	0\\
135	0\\
136	0\\
137	0\\
138	0\\
139	0\\
140	0\\
141	0\\
142	0\\
143	0\\
144	0\\
145	0\\
146	0\\
147	0\\
148	0\\
149	0\\
150	0\\
151	0\\
152	0\\
153	0\\
154	0\\
155	0\\
156	0\\
157	0\\
158	0\\
159	0\\
160	0\\
161	0\\
162	0\\
163	0\\
164	0\\
165	0\\
166	0\\
167	0\\
168	0\\
169	0\\
170	0\\
171	0\\
172	0\\
173	0\\
174	0\\
175	0\\
176	0\\
177	0\\
178	0\\
179	0\\
180	0\\
181	0\\
182	0\\
183	0\\
184	0\\
185	0\\
186	0\\
187	0\\
188	0\\
189	0\\
190	0\\
191	0\\
192	0\\
193	0\\
194	0\\
195	0\\
196	0\\
197	0\\
198	0\\
199	0\\
200	0\\
201	0\\
202	0\\
203	0\\
204	0\\
205	0\\
206	0\\
207	0\\
208	0\\
209	0\\
210	0\\
211	0\\
212	0\\
213	0\\
214	0\\
215	0\\
216	0\\
217	0\\
218	0\\
219	0\\
220	0\\
221	0\\
222	0\\
223	0\\
224	0\\
225	0\\
226	0\\
227	0\\
228	0\\
229	0\\
230	0\\
231	0\\
232	0\\
233	0\\
234	0\\
235	0\\
236	0\\
237	0\\
238	0\\
239	0\\
240	0\\
241	0\\
242	0\\
243	0\\
244	0\\
245	0\\
246	0\\
247	0\\
248	0\\
249	0\\
250	0\\
251	0\\
252	0\\
253	0\\
254	0\\
255	0\\
256	0\\
257	0\\
258	0\\
259	0\\
260	0\\
261	0\\
262	0\\
263	0\\
264	0\\
265	0\\
266	0\\
267	0\\
268	0\\
269	0\\
270	0\\
271	0\\
272	0\\
273	0\\
274	0\\
275	0\\
276	0\\
277	0\\
278	0\\
279	0\\
280	0\\
281	0\\
282	0\\
283	0\\
284	0\\
285	0\\
286	0\\
287	0\\
288	0\\
289	0\\
290	0\\
291	0\\
292	0\\
293	0\\
294	0\\
295	0\\
296	0\\
297	0\\
298	0\\
299	0\\
300	0\\
301	0\\
302	0\\
303	0\\
304	0\\
305	0\\
306	0\\
307	0\\
308	0\\
309	0\\
310	0\\
311	0\\
312	0\\
313	0\\
314	0\\
315	0\\
316	0\\
317	0\\
318	0\\
319	0\\
320	0\\
321	0\\
322	0\\
323	0\\
324	0\\
325	0\\
326	0\\
327	0\\
328	0\\
329	0\\
330	0\\
331	0\\
332	0\\
333	0\\
334	0\\
335	0\\
336	0\\
337	0\\
338	0\\
339	0\\
340	0\\
341	0\\
342	0\\
343	0\\
344	0\\
345	0\\
346	0\\
347	0\\
348	0\\
349	0\\
350	0\\
351	0\\
352	0\\
353	0\\
354	0\\
355	0\\
356	0\\
357	0\\
358	0\\
359	0\\
360	0\\
361	0\\
362	0\\
363	0\\
364	0\\
365	0\\
366	0\\
367	0\\
368	0\\
369	0\\
370	0\\
371	0\\
372	0\\
373	0\\
374	0\\
375	0\\
376	0\\
377	0\\
378	0\\
379	0\\
380	0\\
381	0\\
382	0\\
383	0\\
384	0\\
385	0\\
386	0\\
387	0\\
388	0\\
389	0\\
390	0\\
391	0\\
392	0\\
393	0\\
394	0\\
395	0\\
396	0\\
397	0\\
398	0\\
399	0\\
400	0\\
401	0\\
402	0\\
403	0\\
404	0\\
405	0\\
406	0\\
407	0\\
408	0\\
409	0\\
410	0\\
411	0\\
412	0\\
413	0\\
414	0\\
415	0\\
416	0\\
417	0\\
418	0\\
419	0\\
420	0\\
421	0\\
422	0\\
423	0\\
424	0\\
425	0\\
426	0\\
427	0\\
428	0\\
429	0\\
430	0\\
431	0\\
432	0\\
433	0\\
434	0\\
435	0\\
436	0\\
437	0\\
438	0\\
439	0\\
440	0\\
441	0\\
442	0\\
443	0\\
444	0\\
445	0\\
446	0\\
447	0\\
448	0\\
449	0\\
450	0\\
451	0\\
452	0\\
453	0\\
454	0\\
455	0\\
456	0\\
457	0\\
458	0\\
459	0\\
460	0\\
461	0\\
462	0\\
463	0\\
464	0\\
465	0\\
466	0\\
467	0\\
468	0\\
469	0\\
470	0\\
471	0\\
472	0\\
473	0\\
474	0\\
475	0\\
476	0\\
477	0\\
478	0\\
479	0\\
480	0\\
481	0\\
482	0\\
483	0\\
484	0\\
485	0\\
486	0\\
487	0\\
488	0\\
489	0\\
490	0\\
491	0\\
492	0\\
493	0\\
494	0\\
495	0\\
496	0\\
497	0\\
498	0\\
499	0\\
500	0\\
501	0\\
502	0\\
503	0\\
504	0\\
505	0\\
506	0\\
507	0\\
508	0\\
509	0\\
510	0\\
511	0\\
512	0\\
513	0\\
514	0\\
515	0\\
516	0\\
517	0\\
518	0\\
519	0\\
520	0\\
521	0\\
522	0\\
523	0\\
524	0\\
525	0\\
526	0\\
527	0\\
528	0\\
529	0\\
530	0\\
531	0\\
532	0\\
533	0\\
534	0\\
535	0\\
536	0\\
537	0\\
538	0\\
539	0\\
540	1.57801742067035e-05\\
541	4.17209330634838e-05\\
542	6.82521651627544e-05\\
543	9.53742265120166e-05\\
544	0.000123140289849838\\
545	0.000151571903397981\\
546	0.000180687612467819\\
547	0.000210507578298351\\
548	0.000241049124398555\\
549	0.000272332494289059\\
550	0.000304617640894825\\
551	0.000337587086677401\\
552	0.000371095663930957\\
553	0.000405260843325515\\
554	0.000440102586296729\\
555	0.000475683908134913\\
556	0.000512157386629192\\
557	0.000549580513042214\\
558	0.000587986730254416\\
559	0.000627411471795976\\
560	0.00066788979602979\\
561	0.000710048071235101\\
562	0.0007536191055055\\
563	0.000796982169508077\\
564	0.000840690466772013\\
565	0.000884782514078633\\
566	0.000930038853850304\\
567	0.00101562515945223\\
568	0.00140105236470372\\
569	0.00167444769533238\\
570	0.00181751715607354\\
571	0.00188594856162508\\
572	0.00195494627428676\\
573	0.00202519348264098\\
574	0.00209674898273335\\
575	0.00216965560246406\\
576	0.00224395654960764\\
577	0.00231969722252966\\
578	0.0023969256263201\\
579	0.00247569264056459\\
580	0.00255605223759985\\
581	0.00263806173535728\\
582	0.00272178208202001\\
583	0.00280727821111812\\
584	0.00289461954865299\\
585	0.00298388089415187\\
586	0.00307514426197734\\
587	0.00316850325005476\\
588	0.00326407412228819\\
589	0.0033620247953493\\
590	0.00346265165082637\\
591	0.00356658418066895\\
592	0.00367533141522135\\
593	0.0037927422737293\\
594	0.0039289098819834\\
595	0.00411043972693671\\
596	0.00440815817174734\\
597	0.00501118703192877\\
598	0.00642488516645657\\
599	0\\
600	0\\
};
\addplot [color=mycolor5,solid,forget plot]
  table[row sep=crcr]{%
1	0\\
2	0\\
3	0\\
4	0\\
5	0\\
6	0\\
7	0\\
8	0\\
9	0\\
10	0\\
11	0\\
12	0\\
13	0\\
14	0\\
15	0\\
16	0\\
17	0\\
18	0\\
19	0\\
20	0\\
21	0\\
22	0\\
23	0\\
24	0\\
25	0\\
26	0\\
27	0\\
28	0\\
29	0\\
30	0\\
31	0\\
32	0\\
33	0\\
34	0\\
35	0\\
36	0\\
37	0\\
38	0\\
39	0\\
40	0\\
41	0\\
42	0\\
43	0\\
44	0\\
45	0\\
46	0\\
47	0\\
48	0\\
49	0\\
50	0\\
51	0\\
52	0\\
53	0\\
54	0\\
55	0\\
56	0\\
57	0\\
58	0\\
59	0\\
60	0\\
61	0\\
62	0\\
63	0\\
64	0\\
65	0\\
66	0\\
67	0\\
68	0\\
69	0\\
70	0\\
71	0\\
72	0\\
73	0\\
74	0\\
75	0\\
76	0\\
77	0\\
78	0\\
79	0\\
80	0\\
81	0\\
82	0\\
83	0\\
84	0\\
85	0\\
86	0\\
87	0\\
88	0\\
89	0\\
90	0\\
91	0\\
92	0\\
93	0\\
94	0\\
95	0\\
96	0\\
97	0\\
98	0\\
99	0\\
100	0\\
101	0\\
102	0\\
103	0\\
104	0\\
105	0\\
106	0\\
107	0\\
108	0\\
109	0\\
110	0\\
111	0\\
112	0\\
113	0\\
114	0\\
115	0\\
116	0\\
117	0\\
118	0\\
119	0\\
120	0\\
121	0\\
122	0\\
123	0\\
124	0\\
125	0\\
126	0\\
127	0\\
128	0\\
129	0\\
130	0\\
131	0\\
132	0\\
133	0\\
134	0\\
135	0\\
136	0\\
137	0\\
138	0\\
139	0\\
140	0\\
141	0\\
142	0\\
143	0\\
144	0\\
145	0\\
146	0\\
147	0\\
148	0\\
149	0\\
150	0\\
151	0\\
152	0\\
153	0\\
154	0\\
155	0\\
156	0\\
157	0\\
158	0\\
159	0\\
160	0\\
161	0\\
162	0\\
163	0\\
164	0\\
165	0\\
166	0\\
167	0\\
168	0\\
169	0\\
170	0\\
171	0\\
172	0\\
173	0\\
174	0\\
175	0\\
176	0\\
177	0\\
178	0\\
179	0\\
180	0\\
181	0\\
182	0\\
183	0\\
184	0\\
185	0\\
186	0\\
187	0\\
188	0\\
189	0\\
190	0\\
191	0\\
192	0\\
193	0\\
194	0\\
195	0\\
196	0\\
197	0\\
198	0\\
199	0\\
200	0\\
201	0\\
202	0\\
203	0\\
204	0\\
205	0\\
206	0\\
207	0\\
208	0\\
209	0\\
210	0\\
211	0\\
212	0\\
213	0\\
214	0\\
215	0\\
216	0\\
217	0\\
218	0\\
219	0\\
220	0\\
221	0\\
222	0\\
223	0\\
224	0\\
225	0\\
226	0\\
227	0\\
228	0\\
229	0\\
230	0\\
231	0\\
232	0\\
233	0\\
234	0\\
235	0\\
236	0\\
237	0\\
238	0\\
239	0\\
240	0\\
241	0\\
242	0\\
243	0\\
244	0\\
245	0\\
246	0\\
247	0\\
248	0\\
249	0\\
250	0\\
251	0\\
252	0\\
253	0\\
254	0\\
255	0\\
256	0\\
257	0\\
258	0\\
259	0\\
260	0\\
261	0\\
262	0\\
263	0\\
264	0\\
265	0\\
266	0\\
267	0\\
268	0\\
269	0\\
270	0\\
271	0\\
272	0\\
273	0\\
274	0\\
275	0\\
276	0\\
277	0\\
278	0\\
279	0\\
280	0\\
281	0\\
282	0\\
283	0\\
284	0\\
285	0\\
286	0\\
287	0\\
288	0\\
289	0\\
290	0\\
291	0\\
292	0\\
293	0\\
294	0\\
295	0\\
296	0\\
297	0\\
298	0\\
299	0\\
300	0\\
301	0\\
302	0\\
303	0\\
304	0\\
305	0\\
306	0\\
307	0\\
308	0\\
309	0\\
310	0\\
311	0\\
312	0\\
313	0\\
314	0\\
315	0\\
316	0\\
317	0\\
318	0\\
319	0\\
320	0\\
321	0\\
322	0\\
323	0\\
324	0\\
325	0\\
326	0\\
327	0\\
328	0\\
329	0\\
330	0\\
331	0\\
332	0\\
333	0\\
334	0\\
335	0\\
336	0\\
337	0\\
338	0\\
339	0\\
340	0\\
341	0\\
342	0\\
343	0\\
344	0\\
345	0\\
346	0\\
347	0\\
348	0\\
349	0\\
350	0\\
351	0\\
352	0\\
353	0\\
354	0\\
355	0\\
356	0\\
357	0\\
358	0\\
359	0\\
360	0\\
361	0\\
362	0\\
363	0\\
364	0\\
365	0\\
366	0\\
367	0\\
368	0\\
369	0\\
370	0\\
371	0\\
372	0\\
373	0\\
374	0\\
375	0\\
376	0\\
377	0\\
378	0\\
379	0\\
380	0\\
381	0\\
382	0\\
383	0\\
384	0\\
385	0\\
386	0\\
387	0\\
388	0\\
389	0\\
390	0\\
391	0\\
392	0\\
393	0\\
394	0\\
395	0\\
396	0\\
397	0\\
398	0\\
399	0\\
400	0\\
401	0\\
402	0\\
403	0\\
404	0\\
405	0\\
406	0\\
407	0\\
408	0\\
409	0\\
410	0\\
411	0\\
412	0\\
413	0\\
414	0\\
415	0\\
416	0\\
417	0\\
418	0\\
419	0\\
420	0\\
421	0\\
422	0\\
423	0\\
424	0\\
425	0\\
426	0\\
427	0\\
428	0\\
429	0\\
430	0\\
431	0\\
432	0\\
433	0\\
434	0\\
435	0\\
436	0\\
437	0\\
438	0\\
439	0\\
440	0\\
441	0\\
442	0\\
443	0\\
444	0\\
445	0\\
446	0\\
447	0\\
448	0\\
449	0\\
450	0\\
451	0\\
452	0\\
453	0\\
454	0\\
455	0\\
456	0\\
457	0\\
458	0\\
459	0\\
460	0\\
461	0\\
462	0\\
463	0\\
464	0\\
465	0\\
466	0\\
467	0\\
468	0\\
469	0\\
470	0\\
471	0\\
472	0\\
473	0\\
474	0\\
475	0\\
476	0\\
477	0\\
478	0\\
479	0\\
480	0\\
481	0\\
482	0\\
483	0\\
484	0\\
485	0\\
486	0\\
487	0\\
488	0\\
489	0\\
490	0\\
491	0\\
492	0\\
493	0\\
494	0\\
495	0\\
496	0\\
497	0\\
498	0\\
499	0\\
500	0\\
501	0\\
502	0\\
503	0\\
504	0\\
505	0\\
506	0\\
507	0\\
508	0\\
509	0\\
510	0\\
511	0\\
512	0\\
513	0\\
514	0\\
515	0\\
516	0\\
517	0\\
518	0\\
519	0\\
520	0\\
521	0\\
522	0\\
523	0\\
524	0\\
525	0\\
526	0\\
527	0\\
528	0\\
529	0\\
530	0\\
531	0\\
532	0\\
533	0\\
534	0\\
535	0\\
536	0\\
537	0\\
538	0\\
539	0\\
540	5.67779163272291e-06\\
541	3.16144865046584e-05\\
542	5.81200036704712e-05\\
543	8.51921790854403e-05\\
544	0.00011288255189914\\
545	0.000141212350434799\\
546	0.000170189245831659\\
547	0.000199839716824939\\
548	0.000230207356742342\\
549	0.000261309781838766\\
550	0.000293170189040735\\
551	0.000325810513721937\\
552	0.000359445923410564\\
553	0.000393869735973008\\
554	0.000428926673279003\\
555	0.000464639289161119\\
556	0.000501125425574296\\
557	0.00053830708211444\\
558	0.000576434759629047\\
559	0.000615550561680942\\
560	0.000655704173597092\\
561	0.000696934033709092\\
562	0.000739282894716805\\
563	0.000783943930247253\\
564	0.000828931810503418\\
565	0.000874071000879365\\
566	0.000919506942758544\\
567	0.000965676871722533\\
568	0.00101311411328194\\
569	0.00136536274941772\\
570	0.00166568078770398\\
571	0.00187981036938081\\
572	0.00195469927586354\\
573	0.00202516132606539\\
574	0.00209674309368475\\
575	0.00216965418970599\\
576	0.00224395608129885\\
577	0.00231969700168746\\
578	0.00239692552947569\\
579	0.00247569259072984\\
580	0.00255605221240535\\
581	0.00263806172208846\\
582	0.00272178207560973\\
583	0.00280727820819447\\
584	0.00289461954774638\\
585	0.00298388089400896\\
586	0.00307514426197733\\
587	0.00316850325005475\\
588	0.00326407412228819\\
589	0.0033620247953493\\
590	0.00346265165082638\\
591	0.00356658418066895\\
592	0.00367533141522135\\
593	0.00379274227372929\\
594	0.00392890988198341\\
595	0.00411043972693671\\
596	0.00440815817174734\\
597	0.00501118703192877\\
598	0.00642488516645657\\
599	0\\
600	0\\
};
\addplot [color=mycolor6,solid,forget plot]
  table[row sep=crcr]{%
1	0\\
2	0\\
3	0\\
4	0\\
5	0\\
6	0\\
7	0\\
8	0\\
9	0\\
10	0\\
11	0\\
12	0\\
13	0\\
14	0\\
15	0\\
16	0\\
17	0\\
18	0\\
19	0\\
20	0\\
21	0\\
22	0\\
23	0\\
24	0\\
25	0\\
26	0\\
27	0\\
28	0\\
29	0\\
30	0\\
31	0\\
32	0\\
33	0\\
34	0\\
35	0\\
36	0\\
37	0\\
38	0\\
39	0\\
40	0\\
41	0\\
42	0\\
43	0\\
44	0\\
45	0\\
46	0\\
47	0\\
48	0\\
49	0\\
50	0\\
51	0\\
52	0\\
53	0\\
54	0\\
55	0\\
56	0\\
57	0\\
58	0\\
59	0\\
60	0\\
61	0\\
62	0\\
63	0\\
64	0\\
65	0\\
66	0\\
67	0\\
68	0\\
69	0\\
70	0\\
71	0\\
72	0\\
73	0\\
74	0\\
75	0\\
76	0\\
77	0\\
78	0\\
79	0\\
80	0\\
81	0\\
82	0\\
83	0\\
84	0\\
85	0\\
86	0\\
87	0\\
88	0\\
89	0\\
90	0\\
91	0\\
92	0\\
93	0\\
94	0\\
95	0\\
96	0\\
97	0\\
98	0\\
99	0\\
100	0\\
101	0\\
102	0\\
103	0\\
104	0\\
105	0\\
106	0\\
107	0\\
108	0\\
109	0\\
110	0\\
111	0\\
112	0\\
113	0\\
114	0\\
115	0\\
116	0\\
117	0\\
118	0\\
119	0\\
120	0\\
121	0\\
122	0\\
123	0\\
124	0\\
125	0\\
126	0\\
127	0\\
128	0\\
129	0\\
130	0\\
131	0\\
132	0\\
133	0\\
134	0\\
135	0\\
136	0\\
137	0\\
138	0\\
139	0\\
140	0\\
141	0\\
142	0\\
143	0\\
144	0\\
145	0\\
146	0\\
147	0\\
148	0\\
149	0\\
150	0\\
151	0\\
152	0\\
153	0\\
154	0\\
155	0\\
156	0\\
157	0\\
158	0\\
159	0\\
160	0\\
161	0\\
162	0\\
163	0\\
164	0\\
165	0\\
166	0\\
167	0\\
168	0\\
169	0\\
170	0\\
171	0\\
172	0\\
173	0\\
174	0\\
175	0\\
176	0\\
177	0\\
178	0\\
179	0\\
180	0\\
181	0\\
182	0\\
183	0\\
184	0\\
185	0\\
186	0\\
187	0\\
188	0\\
189	0\\
190	0\\
191	0\\
192	0\\
193	0\\
194	0\\
195	0\\
196	0\\
197	0\\
198	0\\
199	0\\
200	0\\
201	0\\
202	0\\
203	0\\
204	0\\
205	0\\
206	0\\
207	0\\
208	0\\
209	0\\
210	0\\
211	0\\
212	0\\
213	0\\
214	0\\
215	0\\
216	0\\
217	0\\
218	0\\
219	0\\
220	0\\
221	0\\
222	0\\
223	0\\
224	0\\
225	0\\
226	0\\
227	0\\
228	0\\
229	0\\
230	0\\
231	0\\
232	0\\
233	0\\
234	0\\
235	0\\
236	0\\
237	0\\
238	0\\
239	0\\
240	0\\
241	0\\
242	0\\
243	0\\
244	0\\
245	0\\
246	0\\
247	0\\
248	0\\
249	0\\
250	0\\
251	0\\
252	0\\
253	0\\
254	0\\
255	0\\
256	0\\
257	0\\
258	0\\
259	0\\
260	0\\
261	0\\
262	0\\
263	0\\
264	0\\
265	0\\
266	0\\
267	0\\
268	0\\
269	0\\
270	0\\
271	0\\
272	0\\
273	0\\
274	0\\
275	0\\
276	0\\
277	0\\
278	0\\
279	0\\
280	0\\
281	0\\
282	0\\
283	0\\
284	0\\
285	0\\
286	0\\
287	0\\
288	0\\
289	0\\
290	0\\
291	0\\
292	0\\
293	0\\
294	0\\
295	0\\
296	0\\
297	0\\
298	0\\
299	0\\
300	0\\
301	0\\
302	0\\
303	0\\
304	0\\
305	0\\
306	0\\
307	0\\
308	0\\
309	0\\
310	0\\
311	0\\
312	0\\
313	0\\
314	0\\
315	0\\
316	0\\
317	0\\
318	0\\
319	0\\
320	0\\
321	0\\
322	0\\
323	0\\
324	0\\
325	0\\
326	0\\
327	0\\
328	0\\
329	0\\
330	0\\
331	0\\
332	0\\
333	0\\
334	0\\
335	0\\
336	0\\
337	0\\
338	0\\
339	0\\
340	0\\
341	0\\
342	0\\
343	0\\
344	0\\
345	0\\
346	0\\
347	0\\
348	0\\
349	0\\
350	0\\
351	0\\
352	0\\
353	0\\
354	0\\
355	0\\
356	0\\
357	0\\
358	0\\
359	0\\
360	0\\
361	0\\
362	0\\
363	0\\
364	0\\
365	0\\
366	0\\
367	0\\
368	0\\
369	0\\
370	0\\
371	0\\
372	0\\
373	0\\
374	0\\
375	0\\
376	0\\
377	0\\
378	0\\
379	0\\
380	0\\
381	0\\
382	0\\
383	0\\
384	0\\
385	0\\
386	0\\
387	0\\
388	0\\
389	0\\
390	0\\
391	0\\
392	0\\
393	0\\
394	0\\
395	0\\
396	0\\
397	0\\
398	0\\
399	0\\
400	0\\
401	0\\
402	0\\
403	0\\
404	0\\
405	0\\
406	0\\
407	0\\
408	0\\
409	0\\
410	0\\
411	0\\
412	0\\
413	0\\
414	0\\
415	0\\
416	0\\
417	0\\
418	0\\
419	0\\
420	0\\
421	0\\
422	0\\
423	0\\
424	0\\
425	0\\
426	0\\
427	0\\
428	0\\
429	0\\
430	0\\
431	0\\
432	0\\
433	0\\
434	0\\
435	0\\
436	0\\
437	0\\
438	0\\
439	0\\
440	0\\
441	0\\
442	0\\
443	0\\
444	0\\
445	0\\
446	0\\
447	0\\
448	0\\
449	0\\
450	0\\
451	0\\
452	0\\
453	0\\
454	0\\
455	0\\
456	0\\
457	0\\
458	0\\
459	0\\
460	0\\
461	0\\
462	0\\
463	0\\
464	0\\
465	0\\
466	0\\
467	0\\
468	0\\
469	0\\
470	0\\
471	0\\
472	0\\
473	0\\
474	0\\
475	0\\
476	0\\
477	0\\
478	0\\
479	0\\
480	0\\
481	0\\
482	0\\
483	0\\
484	0\\
485	0\\
486	0\\
487	0\\
488	0\\
489	0\\
490	0\\
491	0\\
492	0\\
493	0\\
494	0\\
495	0\\
496	0\\
497	0\\
498	0\\
499	0\\
500	0\\
501	0\\
502	0\\
503	0\\
504	0\\
505	0\\
506	0\\
507	0\\
508	0\\
509	0\\
510	0\\
511	0\\
512	0\\
513	0\\
514	0\\
515	0\\
516	0\\
517	0\\
518	0\\
519	0\\
520	0\\
521	0\\
522	0\\
523	0\\
524	0\\
525	0\\
526	0\\
527	0\\
528	0\\
529	0\\
530	0\\
531	0\\
532	0\\
533	0\\
534	0\\
535	0\\
536	0\\
537	0\\
538	0\\
539	0\\
540	0\\
541	1.99792072081089e-05\\
542	4.63727525165937e-05\\
543	7.34212262876605e-05\\
544	0.000101084696376327\\
545	0.000129373228443587\\
546	0.000158292503572227\\
547	0.000187861500603815\\
548	0.000218122519240035\\
549	0.00024909286214484\\
550	0.000280775360288263\\
551	0.000313225176815619\\
552	0.000346470143581298\\
553	0.00038053214301298\\
554	0.000415556128944413\\
555	0.000451524249748653\\
556	0.00048824208518114\\
557	0.000525592789678638\\
558	0.000563780851805092\\
559	0.000602776718153804\\
560	0.00064265699652178\\
561	0.000683574413147921\\
562	0.000725585257827588\\
563	0.000768736465274667\\
564	0.000813472406506441\\
565	0.000859930031446736\\
566	0.000906633225702796\\
567	0.000953771917621697\\
568	0.00100124266870638\\
569	0.00104967338643609\\
570	0.0013043068138501\\
571	0.00163930374758022\\
572	0.00190309861278883\\
573	0.00202315071336391\\
574	0.00209648783523968\\
575	0.00216960877787299\\
576	0.00224394564255947\\
577	0.00231969373006616\\
578	0.00239692399237447\\
579	0.00247569194043164\\
580	0.00255605188006998\\
581	0.00263806155640494\\
582	0.0027217819877466\\
583	0.0028072781656434\\
584	0.00289461952769629\\
585	0.00298388088761337\\
586	0.00307514426094347\\
587	0.00316850325005476\\
588	0.00326407412228819\\
589	0.00336202479534929\\
590	0.00346265165082637\\
591	0.00356658418066895\\
592	0.00367533141522134\\
593	0.00379274227372929\\
594	0.0039289098819834\\
595	0.00411043972693671\\
596	0.00440815817174734\\
597	0.00501118703192877\\
598	0.00642488516645657\\
599	0\\
600	0\\
};
\addplot [color=mycolor7,solid,forget plot]
  table[row sep=crcr]{%
1	0\\
2	0\\
3	0\\
4	0\\
5	0\\
6	0\\
7	0\\
8	0\\
9	0\\
10	0\\
11	0\\
12	0\\
13	0\\
14	0\\
15	0\\
16	0\\
17	0\\
18	0\\
19	0\\
20	0\\
21	0\\
22	0\\
23	0\\
24	0\\
25	0\\
26	0\\
27	0\\
28	0\\
29	0\\
30	0\\
31	0\\
32	0\\
33	0\\
34	0\\
35	0\\
36	0\\
37	0\\
38	0\\
39	0\\
40	0\\
41	0\\
42	0\\
43	0\\
44	0\\
45	0\\
46	0\\
47	0\\
48	0\\
49	0\\
50	0\\
51	0\\
52	0\\
53	0\\
54	0\\
55	0\\
56	0\\
57	0\\
58	0\\
59	0\\
60	0\\
61	0\\
62	0\\
63	0\\
64	0\\
65	0\\
66	0\\
67	0\\
68	0\\
69	0\\
70	0\\
71	0\\
72	0\\
73	0\\
74	0\\
75	0\\
76	0\\
77	0\\
78	0\\
79	0\\
80	0\\
81	0\\
82	0\\
83	0\\
84	0\\
85	0\\
86	0\\
87	0\\
88	0\\
89	0\\
90	0\\
91	0\\
92	0\\
93	0\\
94	0\\
95	0\\
96	0\\
97	0\\
98	0\\
99	0\\
100	0\\
101	0\\
102	0\\
103	0\\
104	0\\
105	0\\
106	0\\
107	0\\
108	0\\
109	0\\
110	0\\
111	0\\
112	0\\
113	0\\
114	0\\
115	0\\
116	0\\
117	0\\
118	0\\
119	0\\
120	0\\
121	0\\
122	0\\
123	0\\
124	0\\
125	0\\
126	0\\
127	0\\
128	0\\
129	0\\
130	0\\
131	0\\
132	0\\
133	0\\
134	0\\
135	0\\
136	0\\
137	0\\
138	0\\
139	0\\
140	0\\
141	0\\
142	0\\
143	0\\
144	0\\
145	0\\
146	0\\
147	0\\
148	0\\
149	0\\
150	0\\
151	0\\
152	0\\
153	0\\
154	0\\
155	0\\
156	0\\
157	0\\
158	0\\
159	0\\
160	0\\
161	0\\
162	0\\
163	0\\
164	0\\
165	0\\
166	0\\
167	0\\
168	0\\
169	0\\
170	0\\
171	0\\
172	0\\
173	0\\
174	0\\
175	0\\
176	0\\
177	0\\
178	0\\
179	0\\
180	0\\
181	0\\
182	0\\
183	0\\
184	0\\
185	0\\
186	0\\
187	0\\
188	0\\
189	0\\
190	0\\
191	0\\
192	0\\
193	0\\
194	0\\
195	0\\
196	0\\
197	0\\
198	0\\
199	0\\
200	0\\
201	0\\
202	0\\
203	0\\
204	0\\
205	0\\
206	0\\
207	0\\
208	0\\
209	0\\
210	0\\
211	0\\
212	0\\
213	0\\
214	0\\
215	0\\
216	0\\
217	0\\
218	0\\
219	0\\
220	0\\
221	0\\
222	0\\
223	0\\
224	0\\
225	0\\
226	0\\
227	0\\
228	0\\
229	0\\
230	0\\
231	0\\
232	0\\
233	0\\
234	0\\
235	0\\
236	0\\
237	0\\
238	0\\
239	0\\
240	0\\
241	0\\
242	0\\
243	0\\
244	0\\
245	0\\
246	0\\
247	0\\
248	0\\
249	0\\
250	0\\
251	0\\
252	0\\
253	0\\
254	0\\
255	0\\
256	0\\
257	0\\
258	0\\
259	0\\
260	0\\
261	0\\
262	0\\
263	0\\
264	0\\
265	0\\
266	0\\
267	0\\
268	0\\
269	0\\
270	0\\
271	0\\
272	0\\
273	0\\
274	0\\
275	0\\
276	0\\
277	0\\
278	0\\
279	0\\
280	0\\
281	0\\
282	0\\
283	0\\
284	0\\
285	0\\
286	0\\
287	0\\
288	0\\
289	0\\
290	0\\
291	0\\
292	0\\
293	0\\
294	0\\
295	0\\
296	0\\
297	0\\
298	0\\
299	0\\
300	0\\
301	0\\
302	0\\
303	0\\
304	0\\
305	0\\
306	0\\
307	0\\
308	0\\
309	0\\
310	0\\
311	0\\
312	0\\
313	0\\
314	0\\
315	0\\
316	0\\
317	0\\
318	0\\
319	0\\
320	0\\
321	0\\
322	0\\
323	0\\
324	0\\
325	0\\
326	0\\
327	0\\
328	0\\
329	0\\
330	0\\
331	0\\
332	0\\
333	0\\
334	0\\
335	0\\
336	0\\
337	0\\
338	0\\
339	0\\
340	0\\
341	0\\
342	0\\
343	0\\
344	0\\
345	0\\
346	0\\
347	0\\
348	0\\
349	0\\
350	0\\
351	0\\
352	0\\
353	0\\
354	0\\
355	0\\
356	0\\
357	0\\
358	0\\
359	0\\
360	0\\
361	0\\
362	0\\
363	0\\
364	0\\
365	0\\
366	0\\
367	0\\
368	0\\
369	0\\
370	0\\
371	0\\
372	0\\
373	0\\
374	0\\
375	0\\
376	0\\
377	0\\
378	0\\
379	0\\
380	0\\
381	0\\
382	0\\
383	0\\
384	0\\
385	0\\
386	0\\
387	0\\
388	0\\
389	0\\
390	0\\
391	0\\
392	0\\
393	0\\
394	0\\
395	0\\
396	0\\
397	0\\
398	0\\
399	0\\
400	0\\
401	0\\
402	0\\
403	0\\
404	0\\
405	0\\
406	0\\
407	0\\
408	0\\
409	0\\
410	0\\
411	0\\
412	0\\
413	0\\
414	0\\
415	0\\
416	0\\
417	0\\
418	0\\
419	0\\
420	0\\
421	0\\
422	0\\
423	0\\
424	0\\
425	0\\
426	0\\
427	0\\
428	0\\
429	0\\
430	0\\
431	0\\
432	0\\
433	0\\
434	0\\
435	0\\
436	0\\
437	0\\
438	0\\
439	0\\
440	0\\
441	0\\
442	0\\
443	0\\
444	0\\
445	0\\
446	0\\
447	0\\
448	0\\
449	0\\
450	0\\
451	0\\
452	0\\
453	0\\
454	0\\
455	0\\
456	0\\
457	0\\
458	0\\
459	0\\
460	0\\
461	0\\
462	0\\
463	0\\
464	0\\
465	0\\
466	0\\
467	0\\
468	0\\
469	0\\
470	0\\
471	0\\
472	0\\
473	0\\
474	0\\
475	0\\
476	0\\
477	0\\
478	0\\
479	0\\
480	0\\
481	0\\
482	0\\
483	0\\
484	0\\
485	0\\
486	0\\
487	0\\
488	0\\
489	0\\
490	0\\
491	0\\
492	0\\
493	0\\
494	0\\
495	0\\
496	0\\
497	0\\
498	0\\
499	0\\
500	0\\
501	0\\
502	0\\
503	0\\
504	0\\
505	0\\
506	0\\
507	0\\
508	0\\
509	0\\
510	0\\
511	0\\
512	0\\
513	0\\
514	0\\
515	0\\
516	0\\
517	0\\
518	0\\
519	0\\
520	0\\
521	0\\
522	0\\
523	0\\
524	0\\
525	0\\
526	0\\
527	0\\
528	0\\
529	0\\
530	0\\
531	0\\
532	0\\
533	0\\
534	0\\
535	0\\
536	0\\
537	0\\
538	0\\
539	0\\
540	0\\
541	6.76881387420085e-06\\
542	3.30164558502364e-05\\
543	5.98611904804831e-05\\
544	8.7334079349987e-05\\
545	0.000115457856912956\\
546	0.000144277166012184\\
547	0.000173793755482025\\
548	0.000203981423139992\\
549	0.000234866214895428\\
550	0.000266443042311468\\
551	0.00029876073672643\\
552	0.000331847287909978\\
553	0.000365721898928622\\
554	0.000400391322970895\\
555	0.000435922616984166\\
556	0.000472350252675263\\
557	0.000509932211154596\\
558	0.000548310927633405\\
559	0.000587456131965322\\
560	0.000627403331006158\\
561	0.000668274334551636\\
562	0.000710017142892241\\
563	0.000752802951629782\\
564	0.000796728665476221\\
565	0.000841860817331624\\
566	0.000888906921322659\\
567	0.000937293589642778\\
568	0.000985960275563987\\
569	0.0010352558317612\\
570	0.00108500923535093\\
571	0.00122235714586162\\
572	0.00157519138430146\\
573	0.0018518308403065\\
574	0.00208025282367438\\
575	0.00216759714565874\\
576	0.00224359741223466\\
577	0.0023196168261704\\
578	0.00239690130315676\\
579	0.00247568128212639\\
580	0.00255604754780296\\
581	0.00263805935698486\\
582	0.00272178091061339\\
583	0.00280727759029156\\
584	0.00289461924949013\\
585	0.00298388075192783\\
586	0.00307514421637755\\
587	0.00316850324266257\\
588	0.00326407412228818\\
589	0.00336202479534929\\
590	0.00346265165082637\\
591	0.00356658418066895\\
592	0.00367533141522135\\
593	0.00379274227372929\\
594	0.0039289098819834\\
595	0.00411043972693672\\
596	0.00440815817174734\\
597	0.00501118703192877\\
598	0.00642488516645657\\
599	0\\
600	0\\
};
\addplot [color=mycolor8,solid,forget plot]
  table[row sep=crcr]{%
1	0\\
2	0\\
3	0\\
4	0\\
5	0\\
6	0\\
7	0\\
8	0\\
9	0\\
10	0\\
11	0\\
12	0\\
13	0\\
14	0\\
15	0\\
16	0\\
17	0\\
18	0\\
19	0\\
20	0\\
21	0\\
22	0\\
23	0\\
24	0\\
25	0\\
26	0\\
27	0\\
28	0\\
29	0\\
30	0\\
31	0\\
32	0\\
33	0\\
34	0\\
35	0\\
36	0\\
37	0\\
38	0\\
39	0\\
40	0\\
41	0\\
42	0\\
43	0\\
44	0\\
45	0\\
46	0\\
47	0\\
48	0\\
49	0\\
50	0\\
51	0\\
52	0\\
53	0\\
54	0\\
55	0\\
56	0\\
57	0\\
58	0\\
59	0\\
60	0\\
61	0\\
62	0\\
63	0\\
64	0\\
65	0\\
66	0\\
67	0\\
68	0\\
69	0\\
70	0\\
71	0\\
72	0\\
73	0\\
74	0\\
75	0\\
76	0\\
77	0\\
78	0\\
79	0\\
80	0\\
81	0\\
82	0\\
83	0\\
84	0\\
85	0\\
86	0\\
87	0\\
88	0\\
89	0\\
90	0\\
91	0\\
92	0\\
93	0\\
94	0\\
95	0\\
96	0\\
97	0\\
98	0\\
99	0\\
100	0\\
101	0\\
102	0\\
103	0\\
104	0\\
105	0\\
106	0\\
107	0\\
108	0\\
109	0\\
110	0\\
111	0\\
112	0\\
113	0\\
114	0\\
115	0\\
116	0\\
117	0\\
118	0\\
119	0\\
120	0\\
121	0\\
122	0\\
123	0\\
124	0\\
125	0\\
126	0\\
127	0\\
128	0\\
129	0\\
130	0\\
131	0\\
132	0\\
133	0\\
134	0\\
135	0\\
136	0\\
137	0\\
138	0\\
139	0\\
140	0\\
141	0\\
142	0\\
143	0\\
144	0\\
145	0\\
146	0\\
147	0\\
148	0\\
149	0\\
150	0\\
151	0\\
152	0\\
153	0\\
154	0\\
155	0\\
156	0\\
157	0\\
158	0\\
159	0\\
160	0\\
161	0\\
162	0\\
163	0\\
164	0\\
165	0\\
166	0\\
167	0\\
168	0\\
169	0\\
170	0\\
171	0\\
172	0\\
173	0\\
174	0\\
175	0\\
176	0\\
177	0\\
178	0\\
179	0\\
180	0\\
181	0\\
182	0\\
183	0\\
184	0\\
185	0\\
186	0\\
187	0\\
188	0\\
189	0\\
190	0\\
191	0\\
192	0\\
193	0\\
194	0\\
195	0\\
196	0\\
197	0\\
198	0\\
199	0\\
200	0\\
201	0\\
202	0\\
203	0\\
204	0\\
205	0\\
206	0\\
207	0\\
208	0\\
209	0\\
210	0\\
211	0\\
212	0\\
213	0\\
214	0\\
215	0\\
216	0\\
217	0\\
218	0\\
219	0\\
220	0\\
221	0\\
222	0\\
223	0\\
224	0\\
225	0\\
226	0\\
227	0\\
228	0\\
229	0\\
230	0\\
231	0\\
232	0\\
233	0\\
234	0\\
235	0\\
236	0\\
237	0\\
238	0\\
239	0\\
240	0\\
241	0\\
242	0\\
243	0\\
244	0\\
245	0\\
246	0\\
247	0\\
248	0\\
249	0\\
250	0\\
251	0\\
252	0\\
253	0\\
254	0\\
255	0\\
256	0\\
257	0\\
258	0\\
259	0\\
260	0\\
261	0\\
262	0\\
263	0\\
264	0\\
265	0\\
266	0\\
267	0\\
268	0\\
269	0\\
270	0\\
271	0\\
272	0\\
273	0\\
274	0\\
275	0\\
276	0\\
277	0\\
278	0\\
279	0\\
280	0\\
281	0\\
282	0\\
283	0\\
284	0\\
285	0\\
286	0\\
287	0\\
288	0\\
289	0\\
290	0\\
291	0\\
292	0\\
293	0\\
294	0\\
295	0\\
296	0\\
297	0\\
298	0\\
299	0\\
300	0\\
301	0\\
302	0\\
303	0\\
304	0\\
305	0\\
306	0\\
307	0\\
308	0\\
309	0\\
310	0\\
311	0\\
312	0\\
313	0\\
314	0\\
315	0\\
316	0\\
317	0\\
318	0\\
319	0\\
320	0\\
321	0\\
322	0\\
323	0\\
324	0\\
325	0\\
326	0\\
327	0\\
328	0\\
329	0\\
330	0\\
331	0\\
332	0\\
333	0\\
334	0\\
335	0\\
336	0\\
337	0\\
338	0\\
339	0\\
340	0\\
341	0\\
342	0\\
343	0\\
344	0\\
345	0\\
346	0\\
347	0\\
348	0\\
349	0\\
350	0\\
351	0\\
352	0\\
353	0\\
354	0\\
355	0\\
356	0\\
357	0\\
358	0\\
359	0\\
360	0\\
361	0\\
362	0\\
363	0\\
364	0\\
365	0\\
366	0\\
367	0\\
368	0\\
369	0\\
370	0\\
371	0\\
372	0\\
373	0\\
374	0\\
375	0\\
376	0\\
377	0\\
378	0\\
379	0\\
380	0\\
381	0\\
382	0\\
383	0\\
384	0\\
385	0\\
386	0\\
387	0\\
388	0\\
389	0\\
390	0\\
391	0\\
392	0\\
393	0\\
394	0\\
395	0\\
396	0\\
397	0\\
398	0\\
399	0\\
400	0\\
401	0\\
402	0\\
403	0\\
404	0\\
405	0\\
406	0\\
407	0\\
408	0\\
409	0\\
410	0\\
411	0\\
412	0\\
413	0\\
414	0\\
415	0\\
416	0\\
417	0\\
418	0\\
419	0\\
420	0\\
421	0\\
422	0\\
423	0\\
424	0\\
425	0\\
426	0\\
427	0\\
428	0\\
429	0\\
430	0\\
431	0\\
432	0\\
433	0\\
434	0\\
435	0\\
436	0\\
437	0\\
438	0\\
439	0\\
440	0\\
441	0\\
442	0\\
443	0\\
444	0\\
445	0\\
446	0\\
447	0\\
448	0\\
449	0\\
450	0\\
451	0\\
452	0\\
453	0\\
454	0\\
455	0\\
456	0\\
457	0\\
458	0\\
459	0\\
460	0\\
461	0\\
462	0\\
463	0\\
464	0\\
465	0\\
466	0\\
467	0\\
468	0\\
469	0\\
470	0\\
471	0\\
472	0\\
473	0\\
474	0\\
475	0\\
476	0\\
477	0\\
478	0\\
479	0\\
480	0\\
481	0\\
482	0\\
483	0\\
484	0\\
485	0\\
486	0\\
487	0\\
488	0\\
489	0\\
490	0\\
491	0\\
492	0\\
493	0\\
494	0\\
495	0\\
496	0\\
497	0\\
498	0\\
499	0\\
500	0\\
501	0\\
502	0\\
503	0\\
504	0\\
505	0\\
506	0\\
507	0\\
508	0\\
509	0\\
510	0\\
511	0\\
512	0\\
513	0\\
514	0\\
515	0\\
516	0\\
517	0\\
518	0\\
519	0\\
520	0\\
521	0\\
522	0\\
523	0\\
524	0\\
525	0\\
526	0\\
527	0\\
528	0\\
529	0\\
530	0\\
531	0\\
532	0\\
533	0\\
534	0\\
535	0\\
536	0\\
537	0\\
538	0\\
539	0\\
540	0\\
541	0\\
542	1.77994563498494e-05\\
543	4.44690418034569e-05\\
544	7.17504882016815e-05\\
545	9.96653124370045e-05\\
546	0.000128247224210947\\
547	0.000157508764958893\\
548	0.000187477130071588\\
549	0.000218174077853842\\
550	0.00024965718467097\\
551	0.000281882535055856\\
552	0.000314854645582878\\
553	0.000348597040202349\\
554	0.000383112689282817\\
555	0.00041846178624356\\
556	0.000454668625667846\\
557	0.000491753884948208\\
558	0.000529732799945901\\
559	0.000568792410221848\\
560	0.000608906511575984\\
561	0.000649904672830707\\
562	0.000691724311605649\\
563	0.000734489579471478\\
564	0.000778277104886136\\
565	0.00082303473115782\\
566	0.000868959770003612\\
567	0.000916145326157647\\
568	0.000965414946849161\\
569	0.0010158882089616\\
570	0.00106677817567325\\
571	0.00111839899996879\\
572	0.00117070764637285\\
573	0.00145155287937048\\
574	0.00177963609047075\\
575	0.00203805474797816\\
576	0.00222786492014083\\
577	0.00231696253384747\\
578	0.00239633649011414\\
579	0.00247552515937604\\
580	0.00255597390481131\\
581	0.0026380307249917\\
582	0.00272176645872677\\
583	0.00280727066893297\\
584	0.00289461552253682\\
585	0.00298387896151858\\
586	0.00307514331064204\\
587	0.00316850293601066\\
588	0.00326407407006435\\
589	0.00336202479534929\\
590	0.00346265165082637\\
591	0.00356658418066895\\
592	0.00367533141522135\\
593	0.0037927422737293\\
594	0.00392890988198341\\
595	0.00411043972693672\\
596	0.00440815817174735\\
597	0.00501118703192877\\
598	0.00642488516645657\\
599	0\\
600	0\\
};
\addplot [color=blue!25!mycolor7,solid,forget plot]
  table[row sep=crcr]{%
1	0\\
2	0\\
3	0\\
4	0\\
5	0\\
6	0\\
7	0\\
8	0\\
9	0\\
10	0\\
11	0\\
12	0\\
13	0\\
14	0\\
15	0\\
16	0\\
17	0\\
18	0\\
19	0\\
20	0\\
21	0\\
22	0\\
23	0\\
24	0\\
25	0\\
26	0\\
27	0\\
28	0\\
29	0\\
30	0\\
31	0\\
32	0\\
33	0\\
34	0\\
35	0\\
36	0\\
37	0\\
38	0\\
39	0\\
40	0\\
41	0\\
42	0\\
43	0\\
44	0\\
45	0\\
46	0\\
47	0\\
48	0\\
49	0\\
50	0\\
51	0\\
52	0\\
53	0\\
54	0\\
55	0\\
56	0\\
57	0\\
58	0\\
59	0\\
60	0\\
61	0\\
62	0\\
63	0\\
64	0\\
65	0\\
66	0\\
67	0\\
68	0\\
69	0\\
70	0\\
71	0\\
72	0\\
73	0\\
74	0\\
75	0\\
76	0\\
77	0\\
78	0\\
79	0\\
80	0\\
81	0\\
82	0\\
83	0\\
84	0\\
85	0\\
86	0\\
87	0\\
88	0\\
89	0\\
90	0\\
91	0\\
92	0\\
93	0\\
94	0\\
95	0\\
96	0\\
97	0\\
98	0\\
99	0\\
100	0\\
101	0\\
102	0\\
103	0\\
104	0\\
105	0\\
106	0\\
107	0\\
108	0\\
109	0\\
110	0\\
111	0\\
112	0\\
113	0\\
114	0\\
115	0\\
116	0\\
117	0\\
118	0\\
119	0\\
120	0\\
121	0\\
122	0\\
123	0\\
124	0\\
125	0\\
126	0\\
127	0\\
128	0\\
129	0\\
130	0\\
131	0\\
132	0\\
133	0\\
134	0\\
135	0\\
136	0\\
137	0\\
138	0\\
139	0\\
140	0\\
141	0\\
142	0\\
143	0\\
144	0\\
145	0\\
146	0\\
147	0\\
148	0\\
149	0\\
150	0\\
151	0\\
152	0\\
153	0\\
154	0\\
155	0\\
156	0\\
157	0\\
158	0\\
159	0\\
160	0\\
161	0\\
162	0\\
163	0\\
164	0\\
165	0\\
166	0\\
167	0\\
168	0\\
169	0\\
170	0\\
171	0\\
172	0\\
173	0\\
174	0\\
175	0\\
176	0\\
177	0\\
178	0\\
179	0\\
180	0\\
181	0\\
182	0\\
183	0\\
184	0\\
185	0\\
186	0\\
187	0\\
188	0\\
189	0\\
190	0\\
191	0\\
192	0\\
193	0\\
194	0\\
195	0\\
196	0\\
197	0\\
198	0\\
199	0\\
200	0\\
201	0\\
202	0\\
203	0\\
204	0\\
205	0\\
206	0\\
207	0\\
208	0\\
209	0\\
210	0\\
211	0\\
212	0\\
213	0\\
214	0\\
215	0\\
216	0\\
217	0\\
218	0\\
219	0\\
220	0\\
221	0\\
222	0\\
223	0\\
224	0\\
225	0\\
226	0\\
227	0\\
228	0\\
229	0\\
230	0\\
231	0\\
232	0\\
233	0\\
234	0\\
235	0\\
236	0\\
237	0\\
238	0\\
239	0\\
240	0\\
241	0\\
242	0\\
243	0\\
244	0\\
245	0\\
246	0\\
247	0\\
248	0\\
249	0\\
250	0\\
251	0\\
252	0\\
253	0\\
254	0\\
255	0\\
256	0\\
257	0\\
258	0\\
259	0\\
260	0\\
261	0\\
262	0\\
263	0\\
264	0\\
265	0\\
266	0\\
267	0\\
268	0\\
269	0\\
270	0\\
271	0\\
272	0\\
273	0\\
274	0\\
275	0\\
276	0\\
277	0\\
278	0\\
279	0\\
280	0\\
281	0\\
282	0\\
283	0\\
284	0\\
285	0\\
286	0\\
287	0\\
288	0\\
289	0\\
290	0\\
291	0\\
292	0\\
293	0\\
294	0\\
295	0\\
296	0\\
297	0\\
298	0\\
299	0\\
300	0\\
301	0\\
302	0\\
303	0\\
304	0\\
305	0\\
306	0\\
307	0\\
308	0\\
309	0\\
310	0\\
311	0\\
312	0\\
313	0\\
314	0\\
315	0\\
316	0\\
317	0\\
318	0\\
319	0\\
320	0\\
321	0\\
322	0\\
323	0\\
324	0\\
325	0\\
326	0\\
327	0\\
328	0\\
329	0\\
330	0\\
331	0\\
332	0\\
333	0\\
334	0\\
335	0\\
336	0\\
337	0\\
338	0\\
339	0\\
340	0\\
341	0\\
342	0\\
343	0\\
344	0\\
345	0\\
346	0\\
347	0\\
348	0\\
349	0\\
350	0\\
351	0\\
352	0\\
353	0\\
354	0\\
355	0\\
356	0\\
357	0\\
358	0\\
359	0\\
360	0\\
361	0\\
362	0\\
363	0\\
364	0\\
365	0\\
366	0\\
367	0\\
368	0\\
369	0\\
370	0\\
371	0\\
372	0\\
373	0\\
374	0\\
375	0\\
376	0\\
377	0\\
378	0\\
379	0\\
380	0\\
381	0\\
382	0\\
383	0\\
384	0\\
385	0\\
386	0\\
387	0\\
388	0\\
389	0\\
390	0\\
391	0\\
392	0\\
393	0\\
394	0\\
395	0\\
396	0\\
397	0\\
398	0\\
399	0\\
400	0\\
401	0\\
402	0\\
403	0\\
404	0\\
405	0\\
406	0\\
407	0\\
408	0\\
409	0\\
410	0\\
411	0\\
412	0\\
413	0\\
414	0\\
415	0\\
416	0\\
417	0\\
418	0\\
419	0\\
420	0\\
421	0\\
422	0\\
423	0\\
424	0\\
425	0\\
426	0\\
427	0\\
428	0\\
429	0\\
430	0\\
431	0\\
432	0\\
433	0\\
434	0\\
435	0\\
436	0\\
437	0\\
438	0\\
439	0\\
440	0\\
441	0\\
442	0\\
443	0\\
444	0\\
445	0\\
446	0\\
447	0\\
448	0\\
449	0\\
450	0\\
451	0\\
452	0\\
453	0\\
454	0\\
455	0\\
456	0\\
457	0\\
458	0\\
459	0\\
460	0\\
461	0\\
462	0\\
463	0\\
464	0\\
465	0\\
466	0\\
467	0\\
468	0\\
469	0\\
470	0\\
471	0\\
472	0\\
473	0\\
474	0\\
475	0\\
476	0\\
477	0\\
478	0\\
479	0\\
480	0\\
481	0\\
482	0\\
483	0\\
484	0\\
485	0\\
486	0\\
487	0\\
488	0\\
489	0\\
490	0\\
491	0\\
492	0\\
493	0\\
494	0\\
495	0\\
496	0\\
497	0\\
498	0\\
499	0\\
500	0\\
501	0\\
502	0\\
503	0\\
504	0\\
505	0\\
506	0\\
507	0\\
508	0\\
509	0\\
510	0\\
511	0\\
512	0\\
513	0\\
514	0\\
515	0\\
516	0\\
517	0\\
518	0\\
519	0\\
520	0\\
521	0\\
522	0\\
523	0\\
524	0\\
525	0\\
526	0\\
527	0\\
528	0\\
529	0\\
530	0\\
531	0\\
532	0\\
533	0\\
534	0\\
535	0\\
536	0\\
537	0\\
538	0\\
539	0\\
540	0\\
541	0\\
542	6.11660056486632e-07\\
543	2.701048695734e-05\\
544	5.40246794947436e-05\\
545	8.16731202693553e-05\\
546	0.000109974832324423\\
547	0.000138958562454918\\
548	0.00016865085298941\\
549	0.000199069928818032\\
550	0.000230234589378527\\
551	0.00026216528174636\\
552	0.000294879489820562\\
553	0.000328399703159835\\
554	0.000362790215985211\\
555	0.00039799532627114\\
556	0.000434040398948903\\
557	0.000470950458242298\\
558	0.000508722127414138\\
559	0.00054742699063001\\
560	0.000587094676229419\\
561	0.000627754501768088\\
562	0.000669596686142599\\
563	0.000712473576967077\\
564	0.000756336153151856\\
565	0.000801114955820075\\
566	0.000846955852793247\\
567	0.000893934541161798\\
568	0.000942022580164056\\
569	0.00099137139308498\\
570	0.00104281851857678\\
571	0.00109557130842463\\
572	0.00114898652968873\\
573	0.00120314728583361\\
574	0.00131396473799518\\
575	0.00163850815209253\\
576	0.00193799204557724\\
577	0.00219526366421932\\
578	0.00237623984097034\\
579	0.00247139029517359\\
580	0.00255490860490003\\
581	0.00263752357935765\\
582	0.00272157874905703\\
583	0.00280717633476708\\
584	0.00289457157399351\\
585	0.00298385506818827\\
586	0.0030751319777398\\
587	0.00316849697477764\\
588	0.00326407198718924\\
589	0.00336202443091807\\
590	0.00346265165082637\\
591	0.00356658418066895\\
592	0.00367533141522134\\
593	0.00379274227372929\\
594	0.00392890988198339\\
595	0.00411043972693671\\
596	0.00440815817174734\\
597	0.00501118703192877\\
598	0.00642488516645657\\
599	0\\
600	0\\
};
\addplot [color=mycolor9,solid,forget plot]
  table[row sep=crcr]{%
1	0\\
2	0\\
3	0\\
4	0\\
5	0\\
6	0\\
7	0\\
8	0\\
9	0\\
10	0\\
11	0\\
12	0\\
13	0\\
14	0\\
15	0\\
16	0\\
17	0\\
18	0\\
19	0\\
20	0\\
21	0\\
22	0\\
23	0\\
24	0\\
25	0\\
26	0\\
27	0\\
28	0\\
29	0\\
30	0\\
31	0\\
32	0\\
33	0\\
34	0\\
35	0\\
36	0\\
37	0\\
38	0\\
39	0\\
40	0\\
41	0\\
42	0\\
43	0\\
44	0\\
45	0\\
46	0\\
47	0\\
48	0\\
49	0\\
50	0\\
51	0\\
52	0\\
53	0\\
54	0\\
55	0\\
56	0\\
57	0\\
58	0\\
59	0\\
60	0\\
61	0\\
62	0\\
63	0\\
64	0\\
65	0\\
66	0\\
67	0\\
68	0\\
69	0\\
70	0\\
71	0\\
72	0\\
73	0\\
74	0\\
75	0\\
76	0\\
77	0\\
78	0\\
79	0\\
80	0\\
81	0\\
82	0\\
83	0\\
84	0\\
85	0\\
86	0\\
87	0\\
88	0\\
89	0\\
90	0\\
91	0\\
92	0\\
93	0\\
94	0\\
95	0\\
96	0\\
97	0\\
98	0\\
99	0\\
100	0\\
101	0\\
102	0\\
103	0\\
104	0\\
105	0\\
106	0\\
107	0\\
108	0\\
109	0\\
110	0\\
111	0\\
112	0\\
113	0\\
114	0\\
115	0\\
116	0\\
117	0\\
118	0\\
119	0\\
120	0\\
121	0\\
122	0\\
123	0\\
124	0\\
125	0\\
126	0\\
127	0\\
128	0\\
129	0\\
130	0\\
131	0\\
132	0\\
133	0\\
134	0\\
135	0\\
136	0\\
137	0\\
138	0\\
139	0\\
140	0\\
141	0\\
142	0\\
143	0\\
144	0\\
145	0\\
146	0\\
147	0\\
148	0\\
149	0\\
150	0\\
151	0\\
152	0\\
153	0\\
154	0\\
155	0\\
156	0\\
157	0\\
158	0\\
159	0\\
160	0\\
161	0\\
162	0\\
163	0\\
164	0\\
165	0\\
166	0\\
167	0\\
168	0\\
169	0\\
170	0\\
171	0\\
172	0\\
173	0\\
174	0\\
175	0\\
176	0\\
177	0\\
178	0\\
179	0\\
180	0\\
181	0\\
182	0\\
183	0\\
184	0\\
185	0\\
186	0\\
187	0\\
188	0\\
189	0\\
190	0\\
191	0\\
192	0\\
193	0\\
194	0\\
195	0\\
196	0\\
197	0\\
198	0\\
199	0\\
200	0\\
201	0\\
202	0\\
203	0\\
204	0\\
205	0\\
206	0\\
207	0\\
208	0\\
209	0\\
210	0\\
211	0\\
212	0\\
213	0\\
214	0\\
215	0\\
216	0\\
217	0\\
218	0\\
219	0\\
220	0\\
221	0\\
222	0\\
223	0\\
224	0\\
225	0\\
226	0\\
227	0\\
228	0\\
229	0\\
230	0\\
231	0\\
232	0\\
233	0\\
234	0\\
235	0\\
236	0\\
237	0\\
238	0\\
239	0\\
240	0\\
241	0\\
242	0\\
243	0\\
244	0\\
245	0\\
246	0\\
247	0\\
248	0\\
249	0\\
250	0\\
251	0\\
252	0\\
253	0\\
254	0\\
255	0\\
256	0\\
257	0\\
258	0\\
259	0\\
260	0\\
261	0\\
262	0\\
263	0\\
264	0\\
265	0\\
266	0\\
267	0\\
268	0\\
269	0\\
270	0\\
271	0\\
272	0\\
273	0\\
274	0\\
275	0\\
276	0\\
277	0\\
278	0\\
279	0\\
280	0\\
281	0\\
282	0\\
283	0\\
284	0\\
285	0\\
286	0\\
287	0\\
288	0\\
289	0\\
290	0\\
291	0\\
292	0\\
293	0\\
294	0\\
295	0\\
296	0\\
297	0\\
298	0\\
299	0\\
300	0\\
301	0\\
302	0\\
303	0\\
304	0\\
305	0\\
306	0\\
307	0\\
308	0\\
309	0\\
310	0\\
311	0\\
312	0\\
313	0\\
314	0\\
315	0\\
316	0\\
317	0\\
318	0\\
319	0\\
320	0\\
321	0\\
322	0\\
323	0\\
324	0\\
325	0\\
326	0\\
327	0\\
328	0\\
329	0\\
330	0\\
331	0\\
332	0\\
333	0\\
334	0\\
335	0\\
336	0\\
337	0\\
338	0\\
339	0\\
340	0\\
341	0\\
342	0\\
343	0\\
344	0\\
345	0\\
346	0\\
347	0\\
348	0\\
349	0\\
350	0\\
351	0\\
352	0\\
353	0\\
354	0\\
355	0\\
356	0\\
357	0\\
358	0\\
359	0\\
360	0\\
361	0\\
362	0\\
363	0\\
364	0\\
365	0\\
366	0\\
367	0\\
368	0\\
369	0\\
370	0\\
371	0\\
372	0\\
373	0\\
374	0\\
375	0\\
376	0\\
377	0\\
378	0\\
379	0\\
380	0\\
381	0\\
382	0\\
383	0\\
384	0\\
385	0\\
386	0\\
387	0\\
388	0\\
389	0\\
390	0\\
391	0\\
392	0\\
393	0\\
394	0\\
395	0\\
396	0\\
397	0\\
398	0\\
399	0\\
400	0\\
401	0\\
402	0\\
403	0\\
404	0\\
405	0\\
406	0\\
407	0\\
408	0\\
409	0\\
410	0\\
411	0\\
412	0\\
413	0\\
414	0\\
415	0\\
416	0\\
417	0\\
418	0\\
419	0\\
420	0\\
421	0\\
422	0\\
423	0\\
424	0\\
425	0\\
426	0\\
427	0\\
428	0\\
429	0\\
430	0\\
431	0\\
432	0\\
433	0\\
434	0\\
435	0\\
436	0\\
437	0\\
438	0\\
439	0\\
440	0\\
441	0\\
442	0\\
443	0\\
444	0\\
445	0\\
446	0\\
447	0\\
448	0\\
449	0\\
450	0\\
451	0\\
452	0\\
453	0\\
454	0\\
455	0\\
456	0\\
457	0\\
458	0\\
459	0\\
460	0\\
461	0\\
462	0\\
463	0\\
464	0\\
465	0\\
466	0\\
467	0\\
468	0\\
469	0\\
470	0\\
471	0\\
472	0\\
473	0\\
474	0\\
475	0\\
476	0\\
477	0\\
478	0\\
479	0\\
480	0\\
481	0\\
482	0\\
483	0\\
484	0\\
485	0\\
486	0\\
487	0\\
488	0\\
489	0\\
490	0\\
491	0\\
492	0\\
493	0\\
494	0\\
495	0\\
496	0\\
497	0\\
498	0\\
499	0\\
500	0\\
501	0\\
502	0\\
503	0\\
504	0\\
505	0\\
506	0\\
507	0\\
508	0\\
509	0\\
510	0\\
511	0\\
512	0\\
513	0\\
514	0\\
515	0\\
516	0\\
517	0\\
518	0\\
519	0\\
520	0\\
521	0\\
522	0\\
523	0\\
524	0\\
525	0\\
526	0\\
527	0\\
528	0\\
529	0\\
530	0\\
531	0\\
532	0\\
533	0\\
534	0\\
535	0\\
536	0\\
537	0\\
538	0\\
539	0\\
540	0\\
541	0\\
542	0\\
543	7.19050061245648e-06\\
544	3.38811395008324e-05\\
545	6.11995138419879e-05\\
546	8.91653520785335e-05\\
547	0.000117798896383763\\
548	0.0001471207591222\\
549	0.000177163848134191\\
550	0.000207954234817611\\
551	0.000239511711651359\\
552	0.000271863001773644\\
553	0.000305030078629033\\
554	0.000339024023324224\\
555	0.00037386790771121\\
556	0.000409585805568485\\
557	0.000446207799016567\\
558	0.00048378405404937\\
559	0.000522274282432132\\
560	0.000561707740752937\\
561	0.000602117791849755\\
562	0.000643509855471186\\
563	0.000685944198495142\\
564	0.000729459648213524\\
565	0.000774285458757708\\
566	0.000820194696357515\\
567	0.000867170853535688\\
568	0.000915208070350155\\
569	0.00096440285599\\
570	0.00101486356606715\\
571	0.00106662268470533\\
572	0.00112017026618547\\
573	0.00117541917271621\\
574	0.00123165468208164\\
575	0.00128853771559882\\
576	0.00145855591256311\\
577	0.00177609907745312\\
578	0.00206426274766966\\
579	0.00232087238374528\\
580	0.00252475298378954\\
581	0.00263031863751007\\
582	0.00271809688623168\\
583	0.00280595565159673\\
584	0.00289395943431418\\
585	0.00298357938090699\\
586	0.00307498027389778\\
587	0.00316842647759219\\
588	0.00326403331960031\\
589	0.00336201047031448\\
590	0.00346264913972014\\
591	0.00356658418066895\\
592	0.00367533141522134\\
593	0.00379274227372928\\
594	0.00392890988198339\\
595	0.00411043972693671\\
596	0.00440815817174734\\
597	0.00501118703192877\\
598	0.00642488516645657\\
599	0\\
600	0\\
};
\addplot [color=blue!50!mycolor7,solid,forget plot]
  table[row sep=crcr]{%
1	0\\
2	0\\
3	0\\
4	0\\
5	0\\
6	0\\
7	0\\
8	0\\
9	0\\
10	0\\
11	0\\
12	0\\
13	0\\
14	0\\
15	0\\
16	0\\
17	0\\
18	0\\
19	0\\
20	0\\
21	0\\
22	0\\
23	0\\
24	0\\
25	0\\
26	0\\
27	0\\
28	0\\
29	0\\
30	0\\
31	0\\
32	0\\
33	0\\
34	0\\
35	0\\
36	0\\
37	0\\
38	0\\
39	0\\
40	0\\
41	0\\
42	0\\
43	0\\
44	0\\
45	0\\
46	0\\
47	0\\
48	0\\
49	0\\
50	0\\
51	0\\
52	0\\
53	0\\
54	0\\
55	0\\
56	0\\
57	0\\
58	0\\
59	0\\
60	0\\
61	0\\
62	0\\
63	0\\
64	0\\
65	0\\
66	0\\
67	0\\
68	0\\
69	0\\
70	0\\
71	0\\
72	0\\
73	0\\
74	0\\
75	0\\
76	0\\
77	0\\
78	0\\
79	0\\
80	0\\
81	0\\
82	0\\
83	0\\
84	0\\
85	0\\
86	0\\
87	0\\
88	0\\
89	0\\
90	0\\
91	0\\
92	0\\
93	0\\
94	0\\
95	0\\
96	0\\
97	0\\
98	0\\
99	0\\
100	0\\
101	0\\
102	0\\
103	0\\
104	0\\
105	0\\
106	0\\
107	0\\
108	0\\
109	0\\
110	0\\
111	0\\
112	0\\
113	0\\
114	0\\
115	0\\
116	0\\
117	0\\
118	0\\
119	0\\
120	0\\
121	0\\
122	0\\
123	0\\
124	0\\
125	0\\
126	0\\
127	0\\
128	0\\
129	0\\
130	0\\
131	0\\
132	0\\
133	0\\
134	0\\
135	0\\
136	0\\
137	0\\
138	0\\
139	0\\
140	0\\
141	0\\
142	0\\
143	0\\
144	0\\
145	0\\
146	0\\
147	0\\
148	0\\
149	0\\
150	0\\
151	0\\
152	0\\
153	0\\
154	0\\
155	0\\
156	0\\
157	0\\
158	0\\
159	0\\
160	0\\
161	0\\
162	0\\
163	0\\
164	0\\
165	0\\
166	0\\
167	0\\
168	0\\
169	0\\
170	0\\
171	0\\
172	0\\
173	0\\
174	0\\
175	0\\
176	0\\
177	0\\
178	0\\
179	0\\
180	0\\
181	0\\
182	0\\
183	0\\
184	0\\
185	0\\
186	0\\
187	0\\
188	0\\
189	0\\
190	0\\
191	0\\
192	0\\
193	0\\
194	0\\
195	0\\
196	0\\
197	0\\
198	0\\
199	0\\
200	0\\
201	0\\
202	0\\
203	0\\
204	0\\
205	0\\
206	0\\
207	0\\
208	0\\
209	0\\
210	0\\
211	0\\
212	0\\
213	0\\
214	0\\
215	0\\
216	0\\
217	0\\
218	0\\
219	0\\
220	0\\
221	0\\
222	0\\
223	0\\
224	0\\
225	0\\
226	0\\
227	0\\
228	0\\
229	0\\
230	0\\
231	0\\
232	0\\
233	0\\
234	0\\
235	0\\
236	0\\
237	0\\
238	0\\
239	0\\
240	0\\
241	0\\
242	0\\
243	0\\
244	0\\
245	0\\
246	0\\
247	0\\
248	0\\
249	0\\
250	0\\
251	0\\
252	0\\
253	0\\
254	0\\
255	0\\
256	0\\
257	0\\
258	0\\
259	0\\
260	0\\
261	0\\
262	0\\
263	0\\
264	0\\
265	0\\
266	0\\
267	0\\
268	0\\
269	0\\
270	0\\
271	0\\
272	0\\
273	0\\
274	0\\
275	0\\
276	0\\
277	0\\
278	0\\
279	0\\
280	0\\
281	0\\
282	0\\
283	0\\
284	0\\
285	0\\
286	0\\
287	0\\
288	0\\
289	0\\
290	0\\
291	0\\
292	0\\
293	0\\
294	0\\
295	0\\
296	0\\
297	0\\
298	0\\
299	0\\
300	0\\
301	0\\
302	0\\
303	0\\
304	0\\
305	0\\
306	0\\
307	0\\
308	0\\
309	0\\
310	0\\
311	0\\
312	0\\
313	0\\
314	0\\
315	0\\
316	0\\
317	0\\
318	0\\
319	0\\
320	0\\
321	0\\
322	0\\
323	0\\
324	0\\
325	0\\
326	0\\
327	0\\
328	0\\
329	0\\
330	0\\
331	0\\
332	0\\
333	0\\
334	0\\
335	0\\
336	0\\
337	0\\
338	0\\
339	0\\
340	0\\
341	0\\
342	0\\
343	0\\
344	0\\
345	0\\
346	0\\
347	0\\
348	0\\
349	0\\
350	0\\
351	0\\
352	0\\
353	0\\
354	0\\
355	0\\
356	0\\
357	0\\
358	0\\
359	0\\
360	0\\
361	0\\
362	0\\
363	0\\
364	0\\
365	0\\
366	0\\
367	0\\
368	0\\
369	0\\
370	0\\
371	0\\
372	0\\
373	0\\
374	0\\
375	0\\
376	0\\
377	0\\
378	0\\
379	0\\
380	0\\
381	0\\
382	0\\
383	0\\
384	0\\
385	0\\
386	0\\
387	0\\
388	0\\
389	0\\
390	0\\
391	0\\
392	0\\
393	0\\
394	0\\
395	0\\
396	0\\
397	0\\
398	0\\
399	0\\
400	0\\
401	0\\
402	0\\
403	0\\
404	0\\
405	0\\
406	0\\
407	0\\
408	0\\
409	0\\
410	0\\
411	0\\
412	0\\
413	0\\
414	0\\
415	0\\
416	0\\
417	0\\
418	0\\
419	0\\
420	0\\
421	0\\
422	0\\
423	0\\
424	0\\
425	0\\
426	0\\
427	0\\
428	0\\
429	0\\
430	0\\
431	0\\
432	0\\
433	0\\
434	0\\
435	0\\
436	0\\
437	0\\
438	0\\
439	0\\
440	0\\
441	0\\
442	0\\
443	0\\
444	0\\
445	0\\
446	0\\
447	0\\
448	0\\
449	0\\
450	0\\
451	0\\
452	0\\
453	0\\
454	0\\
455	0\\
456	0\\
457	0\\
458	0\\
459	0\\
460	0\\
461	0\\
462	0\\
463	0\\
464	0\\
465	0\\
466	0\\
467	0\\
468	0\\
469	0\\
470	0\\
471	0\\
472	0\\
473	0\\
474	0\\
475	0\\
476	0\\
477	0\\
478	0\\
479	0\\
480	0\\
481	0\\
482	0\\
483	0\\
484	0\\
485	0\\
486	0\\
487	0\\
488	0\\
489	0\\
490	0\\
491	0\\
492	0\\
493	0\\
494	0\\
495	0\\
496	0\\
497	0\\
498	0\\
499	0\\
500	0\\
501	0\\
502	0\\
503	0\\
504	0\\
505	0\\
506	0\\
507	0\\
508	0\\
509	0\\
510	0\\
511	0\\
512	0\\
513	0\\
514	0\\
515	0\\
516	0\\
517	0\\
518	0\\
519	0\\
520	0\\
521	0\\
522	0\\
523	0\\
524	0\\
525	0\\
526	0\\
527	0\\
528	0\\
529	0\\
530	0\\
531	0\\
532	0\\
533	0\\
534	0\\
535	0\\
536	0\\
537	0\\
538	0\\
539	0\\
540	0\\
541	0\\
542	0\\
543	0\\
544	1.05627776958363e-05\\
545	3.75304772690883e-05\\
546	6.51312718678484e-05\\
547	9.33863630317708e-05\\
548	0.000122317550313663\\
549	0.00015194702081486\\
550	0.000182308666181335\\
551	0.00021342974811385\\
552	0.000245331657737849\\
553	0.000278036116023763\\
554	0.000311565519271933\\
555	0.000345943015786943\\
556	0.000381192545748294\\
557	0.000417339269889016\\
558	0.000454422980542451\\
559	0.00049245289508931\\
560	0.00053145713940947\\
561	0.000571465034329317\\
562	0.000612547638993974\\
563	0.000654664794599318\\
564	0.000697840632010934\\
565	0.000742117586682551\\
566	0.000787524528073936\\
567	0.000834101450324763\\
568	0.000882052764535135\\
569	0.000931270793437822\\
570	0.000981668890123943\\
571	0.00103328290710104\\
572	0.00108611346496739\\
573	0.00114036010457888\\
574	0.00119618222241278\\
575	0.00125424290438048\\
576	0.00131331951945714\\
577	0.00137341947531424\\
578	0.00155650097528339\\
579	0.001863805897274\\
580	0.00215685965284502\\
581	0.00241246675897658\\
582	0.0026698348651085\\
583	0.00278212107437134\\
584	0.00288608491603381\\
585	0.00297962692672493\\
586	0.00307327221777276\\
587	0.00316747165964346\\
588	0.00326360274193222\\
589	0.00336176338818523\\
590	0.00346255683773199\\
591	0.00356656710209231\\
592	0.00367533141522134\\
593	0.00379274227372929\\
594	0.00392890988198339\\
595	0.00411043972693671\\
596	0.00440815817174734\\
597	0.00501118703192877\\
598	0.00642488516645657\\
599	0\\
600	0\\
};
\addplot [color=blue!40!mycolor9,solid,forget plot]
  table[row sep=crcr]{%
1	0\\
2	0\\
3	0\\
4	0\\
5	0\\
6	0\\
7	0\\
8	0\\
9	0\\
10	0\\
11	0\\
12	0\\
13	0\\
14	0\\
15	0\\
16	0\\
17	0\\
18	0\\
19	0\\
20	0\\
21	0\\
22	0\\
23	0\\
24	0\\
25	0\\
26	0\\
27	0\\
28	0\\
29	0\\
30	0\\
31	0\\
32	0\\
33	0\\
34	0\\
35	0\\
36	0\\
37	0\\
38	0\\
39	0\\
40	0\\
41	0\\
42	0\\
43	0\\
44	0\\
45	0\\
46	0\\
47	0\\
48	0\\
49	0\\
50	0\\
51	0\\
52	0\\
53	0\\
54	0\\
55	0\\
56	0\\
57	0\\
58	0\\
59	0\\
60	0\\
61	0\\
62	0\\
63	0\\
64	0\\
65	0\\
66	0\\
67	0\\
68	0\\
69	0\\
70	0\\
71	0\\
72	0\\
73	0\\
74	0\\
75	0\\
76	0\\
77	0\\
78	0\\
79	0\\
80	0\\
81	0\\
82	0\\
83	0\\
84	0\\
85	0\\
86	0\\
87	0\\
88	0\\
89	0\\
90	0\\
91	0\\
92	0\\
93	0\\
94	0\\
95	0\\
96	0\\
97	0\\
98	0\\
99	0\\
100	0\\
101	0\\
102	0\\
103	0\\
104	0\\
105	0\\
106	0\\
107	0\\
108	0\\
109	0\\
110	0\\
111	0\\
112	0\\
113	0\\
114	0\\
115	0\\
116	0\\
117	0\\
118	0\\
119	0\\
120	0\\
121	0\\
122	0\\
123	0\\
124	0\\
125	0\\
126	0\\
127	0\\
128	0\\
129	0\\
130	0\\
131	0\\
132	0\\
133	0\\
134	0\\
135	0\\
136	0\\
137	0\\
138	0\\
139	0\\
140	0\\
141	0\\
142	0\\
143	0\\
144	0\\
145	0\\
146	0\\
147	0\\
148	0\\
149	0\\
150	0\\
151	0\\
152	0\\
153	0\\
154	0\\
155	0\\
156	0\\
157	0\\
158	0\\
159	0\\
160	0\\
161	0\\
162	0\\
163	0\\
164	0\\
165	0\\
166	0\\
167	0\\
168	0\\
169	0\\
170	0\\
171	0\\
172	0\\
173	0\\
174	0\\
175	0\\
176	0\\
177	0\\
178	0\\
179	0\\
180	0\\
181	0\\
182	0\\
183	0\\
184	0\\
185	0\\
186	0\\
187	0\\
188	0\\
189	0\\
190	0\\
191	0\\
192	0\\
193	0\\
194	0\\
195	0\\
196	0\\
197	0\\
198	0\\
199	0\\
200	0\\
201	0\\
202	0\\
203	0\\
204	0\\
205	0\\
206	0\\
207	0\\
208	0\\
209	0\\
210	0\\
211	0\\
212	0\\
213	0\\
214	0\\
215	0\\
216	0\\
217	0\\
218	0\\
219	0\\
220	0\\
221	0\\
222	0\\
223	0\\
224	0\\
225	0\\
226	0\\
227	0\\
228	0\\
229	0\\
230	0\\
231	0\\
232	0\\
233	0\\
234	0\\
235	0\\
236	0\\
237	0\\
238	0\\
239	0\\
240	0\\
241	0\\
242	0\\
243	0\\
244	0\\
245	0\\
246	0\\
247	0\\
248	0\\
249	0\\
250	0\\
251	0\\
252	0\\
253	0\\
254	0\\
255	0\\
256	0\\
257	0\\
258	0\\
259	0\\
260	0\\
261	0\\
262	0\\
263	0\\
264	0\\
265	0\\
266	0\\
267	0\\
268	0\\
269	0\\
270	0\\
271	0\\
272	0\\
273	0\\
274	0\\
275	0\\
276	0\\
277	0\\
278	0\\
279	0\\
280	0\\
281	0\\
282	0\\
283	0\\
284	0\\
285	0\\
286	0\\
287	0\\
288	0\\
289	0\\
290	0\\
291	0\\
292	0\\
293	0\\
294	0\\
295	0\\
296	0\\
297	0\\
298	0\\
299	0\\
300	0\\
301	0\\
302	0\\
303	0\\
304	0\\
305	0\\
306	0\\
307	0\\
308	0\\
309	0\\
310	0\\
311	0\\
312	0\\
313	0\\
314	0\\
315	0\\
316	0\\
317	0\\
318	0\\
319	0\\
320	0\\
321	0\\
322	0\\
323	0\\
324	0\\
325	0\\
326	0\\
327	0\\
328	0\\
329	0\\
330	0\\
331	0\\
332	0\\
333	0\\
334	0\\
335	0\\
336	0\\
337	0\\
338	0\\
339	0\\
340	0\\
341	0\\
342	0\\
343	0\\
344	0\\
345	0\\
346	0\\
347	0\\
348	0\\
349	0\\
350	0\\
351	0\\
352	0\\
353	0\\
354	0\\
355	0\\
356	0\\
357	0\\
358	0\\
359	0\\
360	0\\
361	0\\
362	0\\
363	0\\
364	0\\
365	0\\
366	0\\
367	0\\
368	0\\
369	0\\
370	0\\
371	0\\
372	0\\
373	0\\
374	0\\
375	0\\
376	0\\
377	0\\
378	0\\
379	0\\
380	0\\
381	0\\
382	0\\
383	0\\
384	0\\
385	0\\
386	0\\
387	0\\
388	0\\
389	0\\
390	0\\
391	0\\
392	0\\
393	0\\
394	0\\
395	0\\
396	0\\
397	0\\
398	0\\
399	0\\
400	0\\
401	0\\
402	0\\
403	0\\
404	0\\
405	0\\
406	0\\
407	0\\
408	0\\
409	0\\
410	0\\
411	0\\
412	0\\
413	0\\
414	0\\
415	0\\
416	0\\
417	0\\
418	0\\
419	0\\
420	0\\
421	0\\
422	0\\
423	0\\
424	0\\
425	0\\
426	0\\
427	0\\
428	0\\
429	0\\
430	0\\
431	0\\
432	0\\
433	0\\
434	0\\
435	0\\
436	0\\
437	0\\
438	0\\
439	0\\
440	0\\
441	0\\
442	0\\
443	0\\
444	0\\
445	0\\
446	0\\
447	0\\
448	0\\
449	0\\
450	0\\
451	0\\
452	0\\
453	0\\
454	0\\
455	0\\
456	0\\
457	0\\
458	0\\
459	0\\
460	0\\
461	0\\
462	0\\
463	0\\
464	0\\
465	0\\
466	0\\
467	0\\
468	0\\
469	0\\
470	0\\
471	0\\
472	0\\
473	0\\
474	0\\
475	0\\
476	0\\
477	0\\
478	0\\
479	0\\
480	0\\
481	0\\
482	0\\
483	0\\
484	0\\
485	0\\
486	0\\
487	0\\
488	0\\
489	0\\
490	0\\
491	0\\
492	0\\
493	0\\
494	0\\
495	0\\
496	0\\
497	0\\
498	0\\
499	0\\
500	0\\
501	0\\
502	0\\
503	0\\
504	0\\
505	0\\
506	0\\
507	0\\
508	0\\
509	0\\
510	0\\
511	0\\
512	0\\
513	0\\
514	0\\
515	0\\
516	0\\
517	0\\
518	0\\
519	0\\
520	0\\
521	0\\
522	0\\
523	0\\
524	0\\
525	0\\
526	0\\
527	0\\
528	0\\
529	0\\
530	0\\
531	0\\
532	0\\
533	0\\
534	0\\
535	0\\
536	0\\
537	0\\
538	0\\
539	0\\
540	0\\
541	0\\
542	0\\
543	0\\
544	0\\
545	8.42502412765558e-06\\
546	3.5851947618167e-05\\
547	6.38863155815981e-05\\
548	9.25528110374975e-05\\
549	0.000121877183064925\\
550	0.000151885824432349\\
551	0.000182616850935656\\
552	0.000214101549367516\\
553	0.000246366661718656\\
554	0.000279437159386335\\
555	0.000313337079486906\\
556	0.000348091356271088\\
557	0.000383725850763514\\
558	0.000420267380630395\\
559	0.000457743782503094\\
560	0.000496183903776342\\
561	0.000535617637525919\\
562	0.000576075963802652\\
563	0.00061759716129537\\
564	0.000660217067372072\\
565	0.000703959760100343\\
566	0.000748880601024024\\
567	0.000795008079541074\\
568	0.000842337955796574\\
569	0.00089091219139234\\
570	0.000940779053632871\\
571	0.000992018378974306\\
572	0.00104480256110106\\
573	0.00109890666525564\\
574	0.00115440158344068\\
575	0.0012113083263446\\
576	0.0012696758029509\\
577	0.00133007165307071\\
578	0.00139239776289234\\
579	0.00145595886199769\\
580	0.00160714494923117\\
581	0.001899590420333\\
582	0.00221326571071451\\
583	0.00246366209819108\\
584	0.00272385949410461\\
585	0.00292924427885274\\
586	0.00304785416029217\\
587	0.00315702143525764\\
588	0.00325763804119556\\
589	0.00335918365979025\\
590	0.00346100219613368\\
591	0.0035659653520596\\
592	0.00367521681883594\\
593	0.00379274227372929\\
594	0.0039289098819834\\
595	0.00411043972693671\\
596	0.00440815817174734\\
597	0.00501118703192877\\
598	0.00642488516645657\\
599	0\\
600	0\\
};
\addplot [color=blue!75!mycolor7,solid,forget plot]
  table[row sep=crcr]{%
1	0\\
2	0\\
3	0\\
4	0\\
5	0\\
6	0\\
7	0\\
8	0\\
9	0\\
10	0\\
11	0\\
12	0\\
13	0\\
14	0\\
15	0\\
16	0\\
17	0\\
18	0\\
19	0\\
20	0\\
21	0\\
22	0\\
23	0\\
24	0\\
25	0\\
26	0\\
27	0\\
28	0\\
29	0\\
30	0\\
31	0\\
32	0\\
33	0\\
34	0\\
35	0\\
36	0\\
37	0\\
38	0\\
39	0\\
40	0\\
41	0\\
42	0\\
43	0\\
44	0\\
45	0\\
46	0\\
47	0\\
48	0\\
49	0\\
50	0\\
51	0\\
52	0\\
53	0\\
54	0\\
55	0\\
56	0\\
57	0\\
58	0\\
59	0\\
60	0\\
61	0\\
62	0\\
63	0\\
64	0\\
65	0\\
66	0\\
67	0\\
68	0\\
69	0\\
70	0\\
71	0\\
72	0\\
73	0\\
74	0\\
75	0\\
76	0\\
77	0\\
78	0\\
79	0\\
80	0\\
81	0\\
82	0\\
83	0\\
84	0\\
85	0\\
86	0\\
87	0\\
88	0\\
89	0\\
90	0\\
91	0\\
92	0\\
93	0\\
94	0\\
95	0\\
96	0\\
97	0\\
98	0\\
99	0\\
100	0\\
101	0\\
102	0\\
103	0\\
104	0\\
105	0\\
106	0\\
107	0\\
108	0\\
109	0\\
110	0\\
111	0\\
112	0\\
113	0\\
114	0\\
115	0\\
116	0\\
117	0\\
118	0\\
119	0\\
120	0\\
121	0\\
122	0\\
123	0\\
124	0\\
125	0\\
126	0\\
127	0\\
128	0\\
129	0\\
130	0\\
131	0\\
132	0\\
133	0\\
134	0\\
135	0\\
136	0\\
137	0\\
138	0\\
139	0\\
140	0\\
141	0\\
142	0\\
143	0\\
144	0\\
145	0\\
146	0\\
147	0\\
148	0\\
149	0\\
150	0\\
151	0\\
152	0\\
153	0\\
154	0\\
155	0\\
156	0\\
157	0\\
158	0\\
159	0\\
160	0\\
161	0\\
162	0\\
163	0\\
164	0\\
165	0\\
166	0\\
167	0\\
168	0\\
169	0\\
170	0\\
171	0\\
172	0\\
173	0\\
174	0\\
175	0\\
176	0\\
177	0\\
178	0\\
179	0\\
180	0\\
181	0\\
182	0\\
183	0\\
184	0\\
185	0\\
186	0\\
187	0\\
188	0\\
189	0\\
190	0\\
191	0\\
192	0\\
193	0\\
194	0\\
195	0\\
196	0\\
197	0\\
198	0\\
199	0\\
200	0\\
201	0\\
202	0\\
203	0\\
204	0\\
205	0\\
206	0\\
207	0\\
208	0\\
209	0\\
210	0\\
211	0\\
212	0\\
213	0\\
214	0\\
215	0\\
216	0\\
217	0\\
218	0\\
219	0\\
220	0\\
221	0\\
222	0\\
223	0\\
224	0\\
225	0\\
226	0\\
227	0\\
228	0\\
229	0\\
230	0\\
231	0\\
232	0\\
233	0\\
234	0\\
235	0\\
236	0\\
237	0\\
238	0\\
239	0\\
240	0\\
241	0\\
242	0\\
243	0\\
244	0\\
245	0\\
246	0\\
247	0\\
248	0\\
249	0\\
250	0\\
251	0\\
252	0\\
253	0\\
254	0\\
255	0\\
256	0\\
257	0\\
258	0\\
259	0\\
260	0\\
261	0\\
262	0\\
263	0\\
264	0\\
265	0\\
266	0\\
267	0\\
268	0\\
269	0\\
270	0\\
271	0\\
272	0\\
273	0\\
274	0\\
275	0\\
276	0\\
277	0\\
278	0\\
279	0\\
280	0\\
281	0\\
282	0\\
283	0\\
284	0\\
285	0\\
286	0\\
287	0\\
288	0\\
289	0\\
290	0\\
291	0\\
292	0\\
293	0\\
294	0\\
295	0\\
296	0\\
297	0\\
298	0\\
299	0\\
300	0\\
301	0\\
302	0\\
303	0\\
304	0\\
305	0\\
306	0\\
307	0\\
308	0\\
309	0\\
310	0\\
311	0\\
312	0\\
313	0\\
314	0\\
315	0\\
316	0\\
317	0\\
318	0\\
319	0\\
320	0\\
321	0\\
322	0\\
323	0\\
324	0\\
325	0\\
326	0\\
327	0\\
328	0\\
329	0\\
330	0\\
331	0\\
332	0\\
333	0\\
334	0\\
335	0\\
336	0\\
337	0\\
338	0\\
339	0\\
340	0\\
341	0\\
342	0\\
343	0\\
344	0\\
345	0\\
346	0\\
347	0\\
348	0\\
349	0\\
350	0\\
351	0\\
352	0\\
353	0\\
354	0\\
355	0\\
356	0\\
357	0\\
358	0\\
359	0\\
360	0\\
361	0\\
362	0\\
363	0\\
364	0\\
365	0\\
366	0\\
367	0\\
368	0\\
369	0\\
370	0\\
371	0\\
372	0\\
373	0\\
374	0\\
375	0\\
376	0\\
377	0\\
378	0\\
379	0\\
380	0\\
381	0\\
382	0\\
383	0\\
384	0\\
385	0\\
386	0\\
387	0\\
388	0\\
389	0\\
390	0\\
391	0\\
392	0\\
393	0\\
394	0\\
395	0\\
396	0\\
397	0\\
398	0\\
399	0\\
400	0\\
401	0\\
402	0\\
403	0\\
404	0\\
405	0\\
406	0\\
407	0\\
408	0\\
409	0\\
410	0\\
411	0\\
412	0\\
413	0\\
414	0\\
415	0\\
416	0\\
417	0\\
418	0\\
419	0\\
420	0\\
421	0\\
422	0\\
423	0\\
424	0\\
425	0\\
426	0\\
427	0\\
428	0\\
429	0\\
430	0\\
431	0\\
432	0\\
433	0\\
434	0\\
435	0\\
436	0\\
437	0\\
438	0\\
439	0\\
440	0\\
441	0\\
442	0\\
443	0\\
444	0\\
445	0\\
446	0\\
447	0\\
448	0\\
449	0\\
450	0\\
451	0\\
452	0\\
453	0\\
454	0\\
455	0\\
456	0\\
457	0\\
458	0\\
459	0\\
460	0\\
461	0\\
462	0\\
463	0\\
464	0\\
465	0\\
466	0\\
467	0\\
468	0\\
469	0\\
470	0\\
471	0\\
472	0\\
473	0\\
474	0\\
475	0\\
476	0\\
477	0\\
478	0\\
479	0\\
480	0\\
481	0\\
482	0\\
483	0\\
484	0\\
485	0\\
486	0\\
487	0\\
488	0\\
489	0\\
490	0\\
491	0\\
492	0\\
493	0\\
494	0\\
495	0\\
496	0\\
497	0\\
498	0\\
499	0\\
500	0\\
501	0\\
502	0\\
503	0\\
504	0\\
505	0\\
506	0\\
507	0\\
508	0\\
509	0\\
510	0\\
511	0\\
512	0\\
513	0\\
514	0\\
515	0\\
516	0\\
517	0\\
518	0\\
519	0\\
520	0\\
521	0\\
522	0\\
523	0\\
524	0\\
525	0\\
526	0\\
527	0\\
528	0\\
529	0\\
530	0\\
531	0\\
532	0\\
533	0\\
534	0\\
535	0\\
536	0\\
537	0\\
538	0\\
539	0\\
540	0\\
541	0\\
542	0\\
543	0\\
544	0\\
545	0\\
546	0\\
547	2.01299093932324e-05\\
548	4.97180538846728e-05\\
549	7.97809066500282e-05\\
550	0.000110346808772033\\
551	0.000141449544534373\\
552	0.000173138917481353\\
553	0.000205467689239195\\
554	0.000238497409334684\\
555	0.000272296809122876\\
556	0.000306891010061317\\
557	0.000342306630210377\\
558	0.000378571867476797\\
559	0.000415716567682697\\
560	0.000453772268996213\\
561	0.000492772207995115\\
562	0.000532751271474939\\
563	0.000573745863855526\\
564	0.000615793687390193\\
565	0.000658933366835831\\
566	0.000703203836142204\\
567	0.00074864369322039\\
568	0.000795293618467427\\
569	0.000843202487028982\\
570	0.000892415829948461\\
571	0.000942997007940762\\
572	0.000994962562184872\\
573	0.0010483381877448\\
574	0.00110318229396275\\
575	0.00115964812226496\\
576	0.00121775736770317\\
577	0.00127738546819808\\
578	0.00133863477863552\\
579	0.00140157076039025\\
580	0.001466687528102\\
581	0.00153378580448497\\
582	0.00160982745781871\\
583	0.00189508529007164\\
584	0.00219138160047944\\
585	0.00247449504394679\\
586	0.0027294908751432\\
587	0.00299438591461619\\
588	0.00319452135862246\\
589	0.0033221530258415\\
590	0.00344585440730691\\
591	0.00355633781678836\\
592	0.00367135035665401\\
593	0.00379198418312597\\
594	0.0039289098819834\\
595	0.00411043972693671\\
596	0.00440815817174734\\
597	0.00501118703192877\\
598	0.00642488516645657\\
599	0\\
600	0\\
};
\addplot [color=blue!80!mycolor9,solid,forget plot]
  table[row sep=crcr]{%
1	0\\
2	0\\
3	0\\
4	0\\
5	0\\
6	0\\
7	0\\
8	0\\
9	0\\
10	0\\
11	0\\
12	0\\
13	0\\
14	0\\
15	0\\
16	0\\
17	0\\
18	0\\
19	0\\
20	0\\
21	0\\
22	0\\
23	0\\
24	0\\
25	0\\
26	0\\
27	0\\
28	0\\
29	0\\
30	0\\
31	0\\
32	0\\
33	0\\
34	0\\
35	0\\
36	0\\
37	0\\
38	0\\
39	0\\
40	0\\
41	0\\
42	0\\
43	0\\
44	0\\
45	0\\
46	0\\
47	0\\
48	0\\
49	0\\
50	0\\
51	0\\
52	0\\
53	0\\
54	0\\
55	0\\
56	0\\
57	0\\
58	0\\
59	0\\
60	0\\
61	0\\
62	0\\
63	0\\
64	0\\
65	0\\
66	0\\
67	0\\
68	0\\
69	0\\
70	0\\
71	0\\
72	0\\
73	0\\
74	0\\
75	0\\
76	0\\
77	0\\
78	0\\
79	0\\
80	0\\
81	0\\
82	0\\
83	0\\
84	0\\
85	0\\
86	0\\
87	0\\
88	0\\
89	0\\
90	0\\
91	0\\
92	0\\
93	0\\
94	0\\
95	0\\
96	0\\
97	0\\
98	0\\
99	0\\
100	0\\
101	0\\
102	0\\
103	0\\
104	0\\
105	0\\
106	0\\
107	0\\
108	0\\
109	0\\
110	0\\
111	0\\
112	0\\
113	0\\
114	0\\
115	0\\
116	0\\
117	0\\
118	0\\
119	0\\
120	0\\
121	0\\
122	0\\
123	0\\
124	0\\
125	0\\
126	0\\
127	0\\
128	0\\
129	0\\
130	0\\
131	0\\
132	0\\
133	0\\
134	0\\
135	0\\
136	0\\
137	0\\
138	0\\
139	0\\
140	0\\
141	0\\
142	0\\
143	0\\
144	0\\
145	0\\
146	0\\
147	0\\
148	0\\
149	0\\
150	0\\
151	0\\
152	0\\
153	0\\
154	0\\
155	0\\
156	0\\
157	0\\
158	0\\
159	0\\
160	0\\
161	0\\
162	0\\
163	0\\
164	0\\
165	0\\
166	0\\
167	0\\
168	0\\
169	0\\
170	0\\
171	0\\
172	0\\
173	0\\
174	0\\
175	0\\
176	0\\
177	0\\
178	0\\
179	0\\
180	0\\
181	0\\
182	0\\
183	0\\
184	0\\
185	0\\
186	0\\
187	0\\
188	0\\
189	0\\
190	0\\
191	0\\
192	0\\
193	0\\
194	0\\
195	0\\
196	0\\
197	0\\
198	0\\
199	0\\
200	0\\
201	0\\
202	0\\
203	0\\
204	0\\
205	0\\
206	0\\
207	0\\
208	0\\
209	0\\
210	0\\
211	0\\
212	0\\
213	0\\
214	0\\
215	0\\
216	0\\
217	0\\
218	0\\
219	0\\
220	0\\
221	0\\
222	0\\
223	0\\
224	0\\
225	0\\
226	0\\
227	0\\
228	0\\
229	0\\
230	0\\
231	0\\
232	0\\
233	0\\
234	0\\
235	0\\
236	0\\
237	0\\
238	0\\
239	0\\
240	0\\
241	0\\
242	0\\
243	0\\
244	0\\
245	0\\
246	0\\
247	0\\
248	0\\
249	0\\
250	0\\
251	0\\
252	0\\
253	0\\
254	0\\
255	0\\
256	0\\
257	0\\
258	0\\
259	0\\
260	0\\
261	0\\
262	0\\
263	0\\
264	0\\
265	0\\
266	0\\
267	0\\
268	0\\
269	0\\
270	0\\
271	0\\
272	0\\
273	0\\
274	0\\
275	0\\
276	0\\
277	0\\
278	0\\
279	0\\
280	0\\
281	0\\
282	0\\
283	0\\
284	0\\
285	0\\
286	0\\
287	0\\
288	0\\
289	0\\
290	0\\
291	0\\
292	0\\
293	0\\
294	0\\
295	0\\
296	0\\
297	0\\
298	0\\
299	0\\
300	0\\
301	0\\
302	0\\
303	0\\
304	0\\
305	0\\
306	0\\
307	0\\
308	0\\
309	0\\
310	0\\
311	0\\
312	0\\
313	0\\
314	0\\
315	0\\
316	0\\
317	0\\
318	0\\
319	0\\
320	0\\
321	0\\
322	0\\
323	0\\
324	0\\
325	0\\
326	0\\
327	0\\
328	0\\
329	0\\
330	0\\
331	0\\
332	0\\
333	0\\
334	0\\
335	0\\
336	0\\
337	0\\
338	0\\
339	0\\
340	0\\
341	0\\
342	0\\
343	0\\
344	0\\
345	0\\
346	0\\
347	0\\
348	0\\
349	0\\
350	0\\
351	0\\
352	0\\
353	0\\
354	0\\
355	0\\
356	0\\
357	0\\
358	0\\
359	0\\
360	0\\
361	0\\
362	0\\
363	0\\
364	0\\
365	0\\
366	0\\
367	0\\
368	0\\
369	0\\
370	0\\
371	0\\
372	0\\
373	0\\
374	0\\
375	0\\
376	0\\
377	0\\
378	0\\
379	0\\
380	0\\
381	0\\
382	0\\
383	0\\
384	0\\
385	0\\
386	0\\
387	0\\
388	0\\
389	0\\
390	0\\
391	0\\
392	0\\
393	0\\
394	0\\
395	0\\
396	0\\
397	0\\
398	0\\
399	0\\
400	0\\
401	0\\
402	0\\
403	0\\
404	0\\
405	0\\
406	0\\
407	0\\
408	0\\
409	0\\
410	0\\
411	0\\
412	0\\
413	0\\
414	0\\
415	0\\
416	0\\
417	0\\
418	0\\
419	0\\
420	0\\
421	0\\
422	0\\
423	0\\
424	0\\
425	0\\
426	0\\
427	0\\
428	0\\
429	0\\
430	0\\
431	0\\
432	0\\
433	0\\
434	0\\
435	0\\
436	0\\
437	0\\
438	0\\
439	0\\
440	0\\
441	0\\
442	0\\
443	0\\
444	0\\
445	0\\
446	0\\
447	0\\
448	0\\
449	0\\
450	0\\
451	0\\
452	0\\
453	0\\
454	0\\
455	0\\
456	0\\
457	0\\
458	0\\
459	0\\
460	0\\
461	0\\
462	0\\
463	0\\
464	0\\
465	0\\
466	0\\
467	0\\
468	0\\
469	0\\
470	0\\
471	0\\
472	0\\
473	0\\
474	0\\
475	0\\
476	0\\
477	0\\
478	0\\
479	0\\
480	0\\
481	0\\
482	0\\
483	0\\
484	0\\
485	0\\
486	0\\
487	0\\
488	0\\
489	0\\
490	0\\
491	0\\
492	0\\
493	0\\
494	0\\
495	0\\
496	0\\
497	0\\
498	0\\
499	0\\
500	0\\
501	0\\
502	0\\
503	0\\
504	0\\
505	0\\
506	0\\
507	0\\
508	0\\
509	0\\
510	0\\
511	0\\
512	0\\
513	0\\
514	0\\
515	0\\
516	0\\
517	0\\
518	0\\
519	0\\
520	0\\
521	0\\
522	0\\
523	0\\
524	0\\
525	0\\
526	0\\
527	0\\
528	0\\
529	0\\
530	0\\
531	0\\
532	0\\
533	0\\
534	0\\
535	0\\
536	0\\
537	0\\
538	0\\
539	0\\
540	0\\
541	0\\
542	0\\
543	0\\
544	0\\
545	0\\
546	0\\
547	0\\
548	0\\
549	0\\
550	2.12039339737889e-05\\
551	5.84050361664144e-05\\
552	9.54010943067283e-05\\
553	0.000132096979318932\\
554	0.000168388025923539\\
555	0.000204174616590858\\
556	0.000240606685821581\\
557	0.000277698174390048\\
558	0.000315465232300687\\
559	0.000353926827843293\\
560	0.000393105503571735\\
561	0.000433028368811907\\
562	0.000473728260877721\\
563	0.000515245178121573\\
564	0.000557628044582771\\
565	0.000600936877891869\\
566	0.000645245449218651\\
567	0.000690644491509104\\
568	0.000737174475960431\\
569	0.000784875954349228\\
570	0.000833792095532848\\
571	0.000883968867581733\\
572	0.00093545555079639\\
573	0.000988305024199897\\
574	0.00104257376943079\\
575	0.00109832432258913\\
576	0.00115564793896351\\
577	0.00121458956000823\\
578	0.0012751791441012\\
579	0.00133755345032623\\
580	0.00140183211419833\\
581	0.00146787349040303\\
582	0.00153576033905729\\
583	0.00160587300607344\\
584	0.00167839160337079\\
585	0.00183894072237567\\
586	0.00211968813859839\\
587	0.00241319308767275\\
588	0.00268990374966985\\
589	0.0029472638427176\\
590	0.00321775581666242\\
591	0.00346925274976254\\
592	0.00361272863110318\\
593	0.00376752264049869\\
594	0.00392397211113342\\
595	0.00411043972693671\\
596	0.00440815817174734\\
597	0.00501118703192877\\
598	0.00642488516645657\\
599	0\\
600	0\\
};
\addplot [color=blue,solid,forget plot]
  table[row sep=crcr]{%
1	0\\
2	0\\
3	0\\
4	0\\
5	0\\
6	0\\
7	0\\
8	0\\
9	0\\
10	0\\
11	0\\
12	0\\
13	0\\
14	0\\
15	0\\
16	0\\
17	0\\
18	0\\
19	0\\
20	0\\
21	0\\
22	0\\
23	0\\
24	0\\
25	0\\
26	0\\
27	0\\
28	0\\
29	0\\
30	0\\
31	0\\
32	0\\
33	0\\
34	0\\
35	0\\
36	0\\
37	0\\
38	0\\
39	0\\
40	0\\
41	0\\
42	0\\
43	0\\
44	0\\
45	0\\
46	0\\
47	0\\
48	0\\
49	0\\
50	0\\
51	0\\
52	0\\
53	0\\
54	0\\
55	0\\
56	0\\
57	0\\
58	0\\
59	0\\
60	0\\
61	0\\
62	0\\
63	0\\
64	0\\
65	0\\
66	0\\
67	0\\
68	0\\
69	0\\
70	0\\
71	0\\
72	0\\
73	0\\
74	0\\
75	0\\
76	0\\
77	0\\
78	0\\
79	0\\
80	0\\
81	0\\
82	0\\
83	0\\
84	0\\
85	0\\
86	0\\
87	0\\
88	0\\
89	0\\
90	0\\
91	0\\
92	0\\
93	0\\
94	0\\
95	0\\
96	0\\
97	0\\
98	0\\
99	0\\
100	0\\
101	0\\
102	0\\
103	0\\
104	0\\
105	0\\
106	0\\
107	0\\
108	0\\
109	0\\
110	0\\
111	0\\
112	0\\
113	0\\
114	0\\
115	0\\
116	0\\
117	0\\
118	0\\
119	0\\
120	0\\
121	0\\
122	0\\
123	0\\
124	0\\
125	0\\
126	0\\
127	0\\
128	0\\
129	0\\
130	0\\
131	0\\
132	0\\
133	0\\
134	0\\
135	0\\
136	0\\
137	0\\
138	0\\
139	0\\
140	0\\
141	0\\
142	0\\
143	0\\
144	0\\
145	0\\
146	0\\
147	0\\
148	0\\
149	0\\
150	0\\
151	0\\
152	0\\
153	0\\
154	0\\
155	0\\
156	0\\
157	0\\
158	0\\
159	0\\
160	0\\
161	0\\
162	0\\
163	0\\
164	0\\
165	0\\
166	0\\
167	0\\
168	0\\
169	0\\
170	0\\
171	0\\
172	0\\
173	0\\
174	0\\
175	0\\
176	0\\
177	0\\
178	0\\
179	0\\
180	0\\
181	0\\
182	0\\
183	0\\
184	0\\
185	0\\
186	0\\
187	0\\
188	0\\
189	0\\
190	0\\
191	0\\
192	0\\
193	0\\
194	0\\
195	0\\
196	0\\
197	0\\
198	0\\
199	0\\
200	0\\
201	0\\
202	0\\
203	0\\
204	0\\
205	0\\
206	0\\
207	0\\
208	0\\
209	0\\
210	0\\
211	0\\
212	0\\
213	0\\
214	0\\
215	0\\
216	0\\
217	0\\
218	0\\
219	0\\
220	0\\
221	0\\
222	0\\
223	0\\
224	0\\
225	0\\
226	0\\
227	0\\
228	0\\
229	0\\
230	0\\
231	0\\
232	0\\
233	0\\
234	0\\
235	0\\
236	0\\
237	0\\
238	0\\
239	0\\
240	0\\
241	0\\
242	0\\
243	0\\
244	0\\
245	0\\
246	0\\
247	0\\
248	0\\
249	0\\
250	0\\
251	0\\
252	0\\
253	0\\
254	0\\
255	0\\
256	0\\
257	0\\
258	0\\
259	0\\
260	0\\
261	0\\
262	0\\
263	0\\
264	0\\
265	0\\
266	0\\
267	0\\
268	0\\
269	0\\
270	0\\
271	0\\
272	0\\
273	0\\
274	0\\
275	0\\
276	0\\
277	0\\
278	0\\
279	0\\
280	0\\
281	0\\
282	0\\
283	0\\
284	0\\
285	0\\
286	0\\
287	0\\
288	0\\
289	0\\
290	0\\
291	0\\
292	0\\
293	0\\
294	0\\
295	0\\
296	0\\
297	0\\
298	0\\
299	0\\
300	0\\
301	0\\
302	0\\
303	0\\
304	0\\
305	0\\
306	0\\
307	0\\
308	0\\
309	0\\
310	0\\
311	0\\
312	0\\
313	0\\
314	0\\
315	0\\
316	0\\
317	0\\
318	0\\
319	0\\
320	0\\
321	0\\
322	0\\
323	0\\
324	0\\
325	0\\
326	0\\
327	0\\
328	0\\
329	0\\
330	0\\
331	0\\
332	0\\
333	0\\
334	0\\
335	0\\
336	0\\
337	0\\
338	0\\
339	0\\
340	0\\
341	0\\
342	0\\
343	0\\
344	0\\
345	0\\
346	0\\
347	0\\
348	0\\
349	0\\
350	0\\
351	0\\
352	0\\
353	0\\
354	0\\
355	0\\
356	0\\
357	0\\
358	0\\
359	0\\
360	0\\
361	0\\
362	0\\
363	0\\
364	0\\
365	0\\
366	0\\
367	0\\
368	0\\
369	0\\
370	0\\
371	0\\
372	0\\
373	0\\
374	0\\
375	0\\
376	0\\
377	0\\
378	0\\
379	0\\
380	0\\
381	0\\
382	0\\
383	0\\
384	0\\
385	0\\
386	0\\
387	0\\
388	0\\
389	0\\
390	0\\
391	0\\
392	0\\
393	0\\
394	0\\
395	0\\
396	0\\
397	0\\
398	0\\
399	0\\
400	0\\
401	0\\
402	0\\
403	0\\
404	0\\
405	0\\
406	0\\
407	0\\
408	0\\
409	0\\
410	0\\
411	0\\
412	0\\
413	0\\
414	0\\
415	0\\
416	0\\
417	0\\
418	0\\
419	0\\
420	0\\
421	0\\
422	0\\
423	0\\
424	0\\
425	0\\
426	0\\
427	0\\
428	0\\
429	0\\
430	0\\
431	0\\
432	0\\
433	0\\
434	0\\
435	0\\
436	0\\
437	0\\
438	0\\
439	0\\
440	0\\
441	0\\
442	0\\
443	0\\
444	0\\
445	0\\
446	0\\
447	0\\
448	0\\
449	0\\
450	0\\
451	0\\
452	0\\
453	0\\
454	0\\
455	0\\
456	0\\
457	0\\
458	0\\
459	0\\
460	0\\
461	0\\
462	0\\
463	0\\
464	0\\
465	0\\
466	0\\
467	0\\
468	0\\
469	0\\
470	0\\
471	0\\
472	0\\
473	0\\
474	0\\
475	0\\
476	0\\
477	0\\
478	0\\
479	0\\
480	0\\
481	0\\
482	0\\
483	0\\
484	0\\
485	0\\
486	0\\
487	0\\
488	0\\
489	0\\
490	0\\
491	0\\
492	0\\
493	0\\
494	0\\
495	0\\
496	0\\
497	0\\
498	0\\
499	0\\
500	0\\
501	0\\
502	0\\
503	0\\
504	0\\
505	0\\
506	0\\
507	0\\
508	0\\
509	0\\
510	0\\
511	0\\
512	0\\
513	0\\
514	0\\
515	0\\
516	0\\
517	0\\
518	0\\
519	0\\
520	0\\
521	0\\
522	0\\
523	0\\
524	0\\
525	0\\
526	0\\
527	0\\
528	0\\
529	0\\
530	0\\
531	0\\
532	0\\
533	0\\
534	0\\
535	0\\
536	0\\
537	0\\
538	0\\
539	0\\
540	0\\
541	0\\
542	0\\
543	0\\
544	0\\
545	0\\
546	0\\
547	0\\
548	0\\
549	0\\
550	0\\
551	0\\
552	0\\
553	0\\
554	0\\
555	1.91719487111611e-05\\
556	6.51916416175343e-05\\
557	0.000111751777533538\\
558	0.000158812301352876\\
559	0.000206325699048006\\
560	0.000254245520571927\\
561	0.000302505267507824\\
562	0.000351027678648465\\
563	0.000399723539571551\\
564	0.000448490252455449\\
565	0.000497210249201547\\
566	0.000545749008894194\\
567	0.00059395425796242\\
568	0.000643093628485419\\
569	0.000693247352774227\\
570	0.000744444297273541\\
571	0.000796717437008355\\
572	0.000850102988926268\\
573	0.000904635303798376\\
574	0.00096035463702777\\
575	0.00101730890690584\\
576	0.0010755559393491\\
577	0.0011351665831607\\
578	0.00119622852090244\\
579	0.00125885100335975\\
580	0.00132315112991813\\
581	0.00138919553049845\\
582	0.00145709190305839\\
583	0.00152687282332745\\
584	0.00159873739035049\\
585	0.00167260810899154\\
586	0.00174851745901825\\
587	0.00182712492469281\\
588	0.0020052356741675\\
589	0.00228207394121861\\
590	0.0025666370940843\\
591	0.00286021103851328\\
592	0.00312453732154687\\
593	0.00341862421613178\\
594	0.00377220464640417\\
595	0.00407888616628598\\
596	0.00440815817174734\\
597	0.00501118703192877\\
598	0.00642488516645657\\
599	0\\
600	0\\
};
\addplot [color=mycolor10,solid,forget plot]
  table[row sep=crcr]{%
1	0\\
2	0\\
3	0\\
4	0\\
5	0\\
6	0\\
7	0\\
8	0\\
9	0\\
10	0\\
11	0\\
12	0\\
13	0\\
14	0\\
15	0\\
16	0\\
17	0\\
18	0\\
19	0\\
20	0\\
21	0\\
22	0\\
23	0\\
24	0\\
25	0\\
26	0\\
27	0\\
28	0\\
29	0\\
30	0\\
31	0\\
32	0\\
33	0\\
34	0\\
35	0\\
36	0\\
37	0\\
38	0\\
39	0\\
40	0\\
41	0\\
42	0\\
43	0\\
44	0\\
45	0\\
46	0\\
47	0\\
48	0\\
49	0\\
50	0\\
51	0\\
52	0\\
53	0\\
54	0\\
55	0\\
56	0\\
57	0\\
58	0\\
59	0\\
60	0\\
61	0\\
62	0\\
63	0\\
64	0\\
65	0\\
66	0\\
67	0\\
68	0\\
69	0\\
70	0\\
71	0\\
72	0\\
73	0\\
74	0\\
75	0\\
76	0\\
77	0\\
78	0\\
79	0\\
80	0\\
81	0\\
82	0\\
83	0\\
84	0\\
85	0\\
86	0\\
87	0\\
88	0\\
89	0\\
90	0\\
91	0\\
92	0\\
93	0\\
94	0\\
95	0\\
96	0\\
97	0\\
98	0\\
99	0\\
100	0\\
101	0\\
102	0\\
103	0\\
104	0\\
105	0\\
106	0\\
107	0\\
108	0\\
109	0\\
110	0\\
111	0\\
112	0\\
113	0\\
114	0\\
115	0\\
116	0\\
117	0\\
118	0\\
119	0\\
120	0\\
121	0\\
122	0\\
123	0\\
124	0\\
125	0\\
126	0\\
127	0\\
128	0\\
129	0\\
130	0\\
131	0\\
132	0\\
133	0\\
134	0\\
135	0\\
136	0\\
137	0\\
138	0\\
139	0\\
140	0\\
141	0\\
142	0\\
143	0\\
144	0\\
145	0\\
146	0\\
147	0\\
148	0\\
149	0\\
150	0\\
151	0\\
152	0\\
153	0\\
154	0\\
155	0\\
156	0\\
157	0\\
158	0\\
159	0\\
160	0\\
161	0\\
162	0\\
163	0\\
164	0\\
165	0\\
166	0\\
167	0\\
168	0\\
169	0\\
170	0\\
171	0\\
172	0\\
173	0\\
174	0\\
175	0\\
176	0\\
177	0\\
178	0\\
179	0\\
180	0\\
181	0\\
182	0\\
183	0\\
184	0\\
185	0\\
186	0\\
187	0\\
188	0\\
189	0\\
190	0\\
191	0\\
192	0\\
193	0\\
194	0\\
195	0\\
196	0\\
197	0\\
198	0\\
199	0\\
200	0\\
201	0\\
202	0\\
203	0\\
204	0\\
205	0\\
206	0\\
207	0\\
208	0\\
209	0\\
210	0\\
211	0\\
212	0\\
213	0\\
214	0\\
215	0\\
216	0\\
217	0\\
218	0\\
219	0\\
220	0\\
221	0\\
222	0\\
223	0\\
224	0\\
225	0\\
226	0\\
227	0\\
228	0\\
229	0\\
230	0\\
231	0\\
232	0\\
233	0\\
234	0\\
235	0\\
236	0\\
237	0\\
238	0\\
239	0\\
240	0\\
241	0\\
242	0\\
243	0\\
244	0\\
245	0\\
246	0\\
247	0\\
248	0\\
249	0\\
250	0\\
251	0\\
252	0\\
253	0\\
254	0\\
255	0\\
256	0\\
257	0\\
258	0\\
259	0\\
260	0\\
261	0\\
262	0\\
263	0\\
264	0\\
265	0\\
266	0\\
267	0\\
268	0\\
269	0\\
270	0\\
271	0\\
272	0\\
273	0\\
274	0\\
275	0\\
276	0\\
277	0\\
278	0\\
279	0\\
280	0\\
281	0\\
282	0\\
283	0\\
284	0\\
285	0\\
286	0\\
287	0\\
288	0\\
289	0\\
290	0\\
291	0\\
292	0\\
293	0\\
294	0\\
295	0\\
296	0\\
297	0\\
298	0\\
299	0\\
300	0\\
301	0\\
302	0\\
303	0\\
304	0\\
305	0\\
306	0\\
307	0\\
308	0\\
309	0\\
310	0\\
311	0\\
312	0\\
313	0\\
314	0\\
315	0\\
316	0\\
317	0\\
318	0\\
319	0\\
320	0\\
321	0\\
322	0\\
323	0\\
324	0\\
325	0\\
326	0\\
327	0\\
328	0\\
329	0\\
330	0\\
331	0\\
332	0\\
333	0\\
334	0\\
335	0\\
336	0\\
337	0\\
338	0\\
339	0\\
340	0\\
341	0\\
342	0\\
343	0\\
344	0\\
345	0\\
346	0\\
347	0\\
348	0\\
349	0\\
350	0\\
351	0\\
352	0\\
353	0\\
354	0\\
355	0\\
356	0\\
357	0\\
358	0\\
359	0\\
360	0\\
361	0\\
362	0\\
363	0\\
364	0\\
365	0\\
366	0\\
367	0\\
368	0\\
369	0\\
370	0\\
371	0\\
372	0\\
373	0\\
374	0\\
375	0\\
376	0\\
377	0\\
378	0\\
379	0\\
380	0\\
381	0\\
382	0\\
383	0\\
384	0\\
385	0\\
386	0\\
387	0\\
388	0\\
389	0\\
390	0\\
391	0\\
392	0\\
393	0\\
394	0\\
395	0\\
396	0\\
397	0\\
398	0\\
399	0\\
400	0\\
401	0\\
402	0\\
403	0\\
404	0\\
405	0\\
406	0\\
407	0\\
408	0\\
409	0\\
410	0\\
411	0\\
412	0\\
413	0\\
414	0\\
415	0\\
416	0\\
417	0\\
418	0\\
419	0\\
420	0\\
421	0\\
422	0\\
423	0\\
424	0\\
425	0\\
426	0\\
427	0\\
428	0\\
429	0\\
430	0\\
431	0\\
432	0\\
433	0\\
434	0\\
435	0\\
436	0\\
437	0\\
438	0\\
439	0\\
440	0\\
441	0\\
442	0\\
443	0\\
444	0\\
445	0\\
446	0\\
447	0\\
448	0\\
449	0\\
450	0\\
451	0\\
452	0\\
453	0\\
454	0\\
455	0\\
456	0\\
457	0\\
458	0\\
459	0\\
460	0\\
461	0\\
462	0\\
463	0\\
464	0\\
465	0\\
466	0\\
467	0\\
468	0\\
469	0\\
470	0\\
471	0\\
472	0\\
473	0\\
474	0\\
475	0\\
476	0\\
477	0\\
478	0\\
479	0\\
480	0\\
481	0\\
482	0\\
483	0\\
484	0\\
485	0\\
486	0\\
487	0\\
488	0\\
489	0\\
490	0\\
491	0\\
492	0\\
493	0\\
494	0\\
495	0\\
496	0\\
497	0\\
498	0\\
499	0\\
500	0\\
501	0\\
502	0\\
503	0\\
504	0\\
505	0\\
506	0\\
507	0\\
508	0\\
509	0\\
510	0\\
511	0\\
512	0\\
513	0\\
514	0\\
515	0\\
516	0\\
517	0\\
518	0\\
519	0\\
520	0\\
521	0\\
522	0\\
523	0\\
524	0\\
525	0\\
526	0\\
527	0\\
528	0\\
529	0\\
530	0\\
531	0\\
532	0\\
533	0\\
534	0\\
535	0\\
536	0\\
537	0\\
538	0\\
539	0\\
540	0\\
541	0\\
542	0\\
543	0\\
544	0\\
545	0\\
546	0\\
547	0\\
548	0\\
549	0\\
550	0\\
551	0\\
552	0\\
553	0\\
554	0\\
555	0\\
556	0\\
557	0\\
558	0\\
559	0\\
560	0\\
561	0\\
562	0\\
563	0\\
564	6.38099865833115e-05\\
565	0.000154253607646718\\
566	0.000245683738061437\\
567	0.000337871408199173\\
568	0.000399914179502145\\
569	0.000461561946766633\\
570	0.00052410013148521\\
571	0.000587469940331828\\
572	0.000651657865042651\\
573	0.000716829079787043\\
574	0.000782921384134708\\
575	0.000849856385345376\\
576	0.000917537151764115\\
577	0.000985844560883639\\
578	0.00105463204067166\\
579	0.00112371753012356\\
580	0.00119324941240428\\
581	0.00126444776909183\\
582	0.00133734372880463\\
583	0.00141196851593021\\
584	0.00148835333464273\\
585	0.00156653037063608\\
586	0.00164653267396629\\
587	0.00172839481985068\\
588	0.00181217273894298\\
589	0.00189801407111492\\
590	0.00198587444025641\\
591	0.00211599393384859\\
592	0.00239672135457705\\
593	0.00268923432027998\\
594	0.003042228405148\\
595	0.00347514822745281\\
596	0.00421265278177493\\
597	0.00501118703192877\\
598	0.00642488516645657\\
599	0\\
600	0\\
};
\addplot [color=mycolor11,solid,forget plot]
  table[row sep=crcr]{%
1	0\\
2	0\\
3	0\\
4	0\\
5	0\\
6	0\\
7	0\\
8	0\\
9	0\\
10	0\\
11	0\\
12	0\\
13	0\\
14	0\\
15	0\\
16	0\\
17	0\\
18	0\\
19	0\\
20	0\\
21	0\\
22	0\\
23	0\\
24	0\\
25	0\\
26	0\\
27	0\\
28	0\\
29	0\\
30	0\\
31	0\\
32	0\\
33	0\\
34	0\\
35	0\\
36	0\\
37	0\\
38	0\\
39	0\\
40	0\\
41	0\\
42	0\\
43	0\\
44	0\\
45	0\\
46	0\\
47	0\\
48	0\\
49	0\\
50	0\\
51	0\\
52	0\\
53	0\\
54	0\\
55	0\\
56	0\\
57	0\\
58	0\\
59	0\\
60	0\\
61	0\\
62	0\\
63	0\\
64	0\\
65	0\\
66	0\\
67	0\\
68	0\\
69	0\\
70	0\\
71	0\\
72	0\\
73	0\\
74	0\\
75	0\\
76	0\\
77	0\\
78	0\\
79	0\\
80	0\\
81	0\\
82	0\\
83	0\\
84	0\\
85	0\\
86	0\\
87	0\\
88	0\\
89	0\\
90	0\\
91	0\\
92	0\\
93	0\\
94	0\\
95	0\\
96	0\\
97	0\\
98	0\\
99	0\\
100	0\\
101	0\\
102	0\\
103	0\\
104	0\\
105	0\\
106	0\\
107	0\\
108	0\\
109	0\\
110	0\\
111	0\\
112	0\\
113	0\\
114	0\\
115	0\\
116	0\\
117	0\\
118	0\\
119	0\\
120	0\\
121	0\\
122	0\\
123	0\\
124	0\\
125	0\\
126	0\\
127	0\\
128	0\\
129	0\\
130	0\\
131	0\\
132	0\\
133	0\\
134	0\\
135	0\\
136	0\\
137	0\\
138	0\\
139	0\\
140	0\\
141	0\\
142	0\\
143	0\\
144	0\\
145	0\\
146	0\\
147	0\\
148	0\\
149	0\\
150	0\\
151	0\\
152	0\\
153	0\\
154	0\\
155	0\\
156	0\\
157	0\\
158	0\\
159	0\\
160	0\\
161	0\\
162	0\\
163	0\\
164	0\\
165	0\\
166	0\\
167	0\\
168	0\\
169	0\\
170	0\\
171	0\\
172	0\\
173	0\\
174	0\\
175	0\\
176	0\\
177	0\\
178	0\\
179	0\\
180	0\\
181	0\\
182	0\\
183	0\\
184	0\\
185	0\\
186	0\\
187	0\\
188	0\\
189	0\\
190	0\\
191	0\\
192	0\\
193	0\\
194	0\\
195	0\\
196	0\\
197	0\\
198	0\\
199	0\\
200	0\\
201	0\\
202	0\\
203	0\\
204	0\\
205	0\\
206	0\\
207	0\\
208	0\\
209	0\\
210	0\\
211	0\\
212	0\\
213	0\\
214	0\\
215	0\\
216	0\\
217	0\\
218	0\\
219	0\\
220	0\\
221	0\\
222	0\\
223	0\\
224	0\\
225	0\\
226	0\\
227	0\\
228	0\\
229	0\\
230	0\\
231	0\\
232	0\\
233	0\\
234	0\\
235	0\\
236	0\\
237	0\\
238	0\\
239	0\\
240	0\\
241	0\\
242	0\\
243	0\\
244	0\\
245	0\\
246	0\\
247	0\\
248	0\\
249	0\\
250	0\\
251	0\\
252	0\\
253	0\\
254	0\\
255	0\\
256	0\\
257	0\\
258	0\\
259	0\\
260	0\\
261	0\\
262	0\\
263	0\\
264	0\\
265	0\\
266	0\\
267	0\\
268	0\\
269	0\\
270	0\\
271	0\\
272	0\\
273	0\\
274	0\\
275	0\\
276	0\\
277	0\\
278	0\\
279	0\\
280	0\\
281	0\\
282	0\\
283	0\\
284	0\\
285	0\\
286	0\\
287	0\\
288	0\\
289	0\\
290	0\\
291	0\\
292	0\\
293	0\\
294	0\\
295	0\\
296	0\\
297	0\\
298	0\\
299	0\\
300	0\\
301	0\\
302	0\\
303	0\\
304	0\\
305	0\\
306	0\\
307	0\\
308	0\\
309	0\\
310	0\\
311	0\\
312	0\\
313	0\\
314	0\\
315	0\\
316	0\\
317	0\\
318	0\\
319	0\\
320	0\\
321	0\\
322	0\\
323	0\\
324	0\\
325	0\\
326	0\\
327	0\\
328	0\\
329	0\\
330	0\\
331	0\\
332	0\\
333	0\\
334	0\\
335	0\\
336	0\\
337	0\\
338	0\\
339	0\\
340	0\\
341	0\\
342	0\\
343	0\\
344	0\\
345	0\\
346	0\\
347	0\\
348	0\\
349	0\\
350	0\\
351	0\\
352	0\\
353	0\\
354	0\\
355	0\\
356	0\\
357	0\\
358	0\\
359	0\\
360	0\\
361	0\\
362	0\\
363	0\\
364	0\\
365	0\\
366	0\\
367	0\\
368	0\\
369	0\\
370	0\\
371	0\\
372	0\\
373	0\\
374	0\\
375	0\\
376	0\\
377	0\\
378	0\\
379	0\\
380	0\\
381	0\\
382	0\\
383	0\\
384	0\\
385	0\\
386	0\\
387	0\\
388	0\\
389	0\\
390	0\\
391	0\\
392	0\\
393	0\\
394	0\\
395	0\\
396	0\\
397	0\\
398	0\\
399	0\\
400	0\\
401	0\\
402	0\\
403	0\\
404	0\\
405	0\\
406	0\\
407	0\\
408	0\\
409	0\\
410	0\\
411	0\\
412	0\\
413	0\\
414	0\\
415	0\\
416	0\\
417	0\\
418	0\\
419	0\\
420	0\\
421	0\\
422	0\\
423	0\\
424	0\\
425	0\\
426	0\\
427	0\\
428	0\\
429	0\\
430	0\\
431	0\\
432	0\\
433	0\\
434	0\\
435	0\\
436	0\\
437	0\\
438	0\\
439	0\\
440	0\\
441	0\\
442	0\\
443	0\\
444	0\\
445	0\\
446	0\\
447	0\\
448	0\\
449	0\\
450	0\\
451	0\\
452	0\\
453	0\\
454	0\\
455	0\\
456	0\\
457	0\\
458	0\\
459	0\\
460	0\\
461	0\\
462	0\\
463	0\\
464	0\\
465	0\\
466	0\\
467	0\\
468	0\\
469	0\\
470	0\\
471	0\\
472	0\\
473	0\\
474	0\\
475	0\\
476	0\\
477	0\\
478	0\\
479	0\\
480	0\\
481	0\\
482	0\\
483	0\\
484	0\\
485	0\\
486	0\\
487	0\\
488	0\\
489	0\\
490	0\\
491	0\\
492	0\\
493	0\\
494	0\\
495	0\\
496	0\\
497	0\\
498	0\\
499	0\\
500	0\\
501	0\\
502	0\\
503	0\\
504	0\\
505	0\\
506	0\\
507	0\\
508	0\\
509	0\\
510	0\\
511	0\\
512	0\\
513	0\\
514	0\\
515	0\\
516	0\\
517	0\\
518	0\\
519	0\\
520	0\\
521	0\\
522	0\\
523	0\\
524	0\\
525	0\\
526	0\\
527	0\\
528	0\\
529	0\\
530	0\\
531	0\\
532	0\\
533	0\\
534	0\\
535	0\\
536	0\\
537	0\\
538	0\\
539	0\\
540	0\\
541	0\\
542	0\\
543	0\\
544	0\\
545	0\\
546	0\\
547	0\\
548	0\\
549	0\\
550	0\\
551	0\\
552	0\\
553	0\\
554	0\\
555	0\\
556	0\\
557	0\\
558	0\\
559	0\\
560	0\\
561	0\\
562	0\\
563	0\\
564	0\\
565	0\\
566	0\\
567	0\\
568	0\\
569	0\\
570	0\\
571	0\\
572	0\\
573	4.04709118961138e-05\\
574	0.000146135873699704\\
575	0.000254345046972783\\
576	0.000365036226188813\\
577	0.000477991922738407\\
578	0.000592630547932805\\
579	0.000709819060000769\\
580	0.000822303469911085\\
581	0.000910282263958814\\
582	0.00100058720699734\\
583	0.00109325009451827\\
584	0.00118829487178716\\
585	0.0012857341382992\\
586	0.00138559035948951\\
587	0.00148783487413243\\
588	0.00159241035075032\\
589	0.0016992221476438\\
590	0.00180812756853647\\
591	0.00191892108123612\\
592	0.00203131802967084\\
593	0.0021449343752569\\
594	0.00225924019373236\\
595	0.0025990765750797\\
596	0.00331669012914152\\
597	0.00466497804100969\\
598	0.00642488516645657\\
599	0\\
600	0\\
};
\addplot [color=mycolor12,solid,forget plot]
  table[row sep=crcr]{%
1	0\\
2	0\\
3	0\\
4	0\\
5	0\\
6	0\\
7	0\\
8	0\\
9	0\\
10	0\\
11	0\\
12	0\\
13	0\\
14	0\\
15	0\\
16	0\\
17	0\\
18	0\\
19	0\\
20	0\\
21	0\\
22	0\\
23	0\\
24	0\\
25	0\\
26	0\\
27	0\\
28	0\\
29	0\\
30	0\\
31	0\\
32	0\\
33	0\\
34	0\\
35	0\\
36	0\\
37	0\\
38	0\\
39	0\\
40	0\\
41	0\\
42	0\\
43	0\\
44	0\\
45	0\\
46	0\\
47	0\\
48	0\\
49	0\\
50	0\\
51	0\\
52	0\\
53	0\\
54	0\\
55	0\\
56	0\\
57	0\\
58	0\\
59	0\\
60	0\\
61	0\\
62	0\\
63	0\\
64	0\\
65	0\\
66	0\\
67	0\\
68	0\\
69	0\\
70	0\\
71	0\\
72	0\\
73	0\\
74	0\\
75	0\\
76	0\\
77	0\\
78	0\\
79	0\\
80	0\\
81	0\\
82	0\\
83	0\\
84	0\\
85	0\\
86	0\\
87	0\\
88	0\\
89	0\\
90	0\\
91	0\\
92	0\\
93	0\\
94	0\\
95	0\\
96	0\\
97	0\\
98	0\\
99	0\\
100	0\\
101	0\\
102	0\\
103	0\\
104	0\\
105	0\\
106	0\\
107	0\\
108	0\\
109	0\\
110	0\\
111	0\\
112	0\\
113	0\\
114	0\\
115	0\\
116	0\\
117	0\\
118	0\\
119	0\\
120	0\\
121	0\\
122	0\\
123	0\\
124	0\\
125	0\\
126	0\\
127	0\\
128	0\\
129	0\\
130	0\\
131	0\\
132	0\\
133	0\\
134	0\\
135	0\\
136	0\\
137	0\\
138	0\\
139	0\\
140	0\\
141	0\\
142	0\\
143	0\\
144	0\\
145	0\\
146	0\\
147	0\\
148	0\\
149	0\\
150	0\\
151	0\\
152	0\\
153	0\\
154	0\\
155	0\\
156	0\\
157	0\\
158	0\\
159	0\\
160	0\\
161	0\\
162	0\\
163	0\\
164	0\\
165	0\\
166	0\\
167	0\\
168	0\\
169	0\\
170	0\\
171	0\\
172	0\\
173	0\\
174	0\\
175	0\\
176	0\\
177	0\\
178	0\\
179	0\\
180	0\\
181	0\\
182	0\\
183	0\\
184	0\\
185	0\\
186	0\\
187	0\\
188	0\\
189	0\\
190	0\\
191	0\\
192	0\\
193	0\\
194	0\\
195	0\\
196	0\\
197	0\\
198	0\\
199	0\\
200	0\\
201	0\\
202	0\\
203	0\\
204	0\\
205	0\\
206	0\\
207	0\\
208	0\\
209	0\\
210	0\\
211	0\\
212	0\\
213	0\\
214	0\\
215	0\\
216	0\\
217	0\\
218	0\\
219	0\\
220	0\\
221	0\\
222	0\\
223	0\\
224	0\\
225	0\\
226	0\\
227	0\\
228	0\\
229	0\\
230	0\\
231	0\\
232	0\\
233	0\\
234	0\\
235	0\\
236	0\\
237	0\\
238	0\\
239	0\\
240	0\\
241	0\\
242	0\\
243	0\\
244	0\\
245	0\\
246	0\\
247	0\\
248	0\\
249	0\\
250	0\\
251	0\\
252	0\\
253	0\\
254	0\\
255	0\\
256	0\\
257	0\\
258	0\\
259	0\\
260	0\\
261	0\\
262	0\\
263	0\\
264	0\\
265	0\\
266	0\\
267	0\\
268	0\\
269	0\\
270	0\\
271	0\\
272	0\\
273	0\\
274	0\\
275	0\\
276	0\\
277	0\\
278	0\\
279	0\\
280	0\\
281	0\\
282	0\\
283	0\\
284	0\\
285	0\\
286	0\\
287	0\\
288	0\\
289	0\\
290	0\\
291	0\\
292	0\\
293	0\\
294	0\\
295	0\\
296	0\\
297	0\\
298	0\\
299	0\\
300	0\\
301	0\\
302	0\\
303	0\\
304	0\\
305	0\\
306	0\\
307	0\\
308	0\\
309	0\\
310	0\\
311	0\\
312	0\\
313	0\\
314	0\\
315	0\\
316	0\\
317	0\\
318	0\\
319	0\\
320	0\\
321	0\\
322	0\\
323	0\\
324	0\\
325	0\\
326	0\\
327	0\\
328	0\\
329	0\\
330	0\\
331	0\\
332	0\\
333	0\\
334	0\\
335	0\\
336	0\\
337	0\\
338	0\\
339	0\\
340	0\\
341	0\\
342	0\\
343	0\\
344	0\\
345	0\\
346	0\\
347	0\\
348	0\\
349	0\\
350	0\\
351	0\\
352	0\\
353	0\\
354	0\\
355	0\\
356	0\\
357	0\\
358	0\\
359	0\\
360	0\\
361	0\\
362	0\\
363	0\\
364	0\\
365	0\\
366	0\\
367	0\\
368	0\\
369	0\\
370	0\\
371	0\\
372	0\\
373	0\\
374	0\\
375	0\\
376	0\\
377	0\\
378	0\\
379	0\\
380	0\\
381	0\\
382	0\\
383	0\\
384	0\\
385	0\\
386	0\\
387	0\\
388	0\\
389	0\\
390	0\\
391	0\\
392	0\\
393	0\\
394	0\\
395	0\\
396	0\\
397	0\\
398	0\\
399	0\\
400	0\\
401	0\\
402	0\\
403	0\\
404	0\\
405	0\\
406	0\\
407	0\\
408	0\\
409	0\\
410	0\\
411	0\\
412	0\\
413	0\\
414	0\\
415	0\\
416	0\\
417	0\\
418	0\\
419	0\\
420	0\\
421	0\\
422	0\\
423	0\\
424	0\\
425	0\\
426	0\\
427	0\\
428	0\\
429	0\\
430	0\\
431	0\\
432	0\\
433	0\\
434	0\\
435	0\\
436	0\\
437	0\\
438	0\\
439	0\\
440	0\\
441	0\\
442	0\\
443	0\\
444	0\\
445	0\\
446	0\\
447	0\\
448	0\\
449	0\\
450	0\\
451	0\\
452	0\\
453	0\\
454	0\\
455	0\\
456	0\\
457	0\\
458	0\\
459	0\\
460	0\\
461	0\\
462	0\\
463	0\\
464	0\\
465	0\\
466	0\\
467	0\\
468	0\\
469	0\\
470	0\\
471	0\\
472	0\\
473	0\\
474	0\\
475	0\\
476	0\\
477	0\\
478	0\\
479	0\\
480	0\\
481	0\\
482	0\\
483	0\\
484	0\\
485	0\\
486	0\\
487	0\\
488	0\\
489	0\\
490	0\\
491	0\\
492	0\\
493	0\\
494	0\\
495	0\\
496	0\\
497	0\\
498	0\\
499	0\\
500	0\\
501	0\\
502	0\\
503	0\\
504	0\\
505	0\\
506	0\\
507	0\\
508	0\\
509	0\\
510	0\\
511	0\\
512	0\\
513	0\\
514	0\\
515	0\\
516	0\\
517	0\\
518	0\\
519	0\\
520	0\\
521	0\\
522	0\\
523	0\\
524	0\\
525	0\\
526	0\\
527	0\\
528	0\\
529	0\\
530	0\\
531	0\\
532	0\\
533	0\\
534	0\\
535	0\\
536	0\\
537	0\\
538	0\\
539	0\\
540	0\\
541	0\\
542	0\\
543	0\\
544	0\\
545	0\\
546	0\\
547	0\\
548	0\\
549	0\\
550	0\\
551	0\\
552	0\\
553	0\\
554	0\\
555	0\\
556	0\\
557	0\\
558	0\\
559	0\\
560	0\\
561	0\\
562	0\\
563	0\\
564	0\\
565	0\\
566	0\\
567	0\\
568	0\\
569	0\\
570	0\\
571	0\\
572	0\\
573	0\\
574	0\\
575	0\\
576	0\\
577	0\\
578	0\\
579	0\\
580	0\\
581	0\\
582	0\\
583	0\\
584	0\\
585	0.00013774192519883\\
586	0.000284032666100375\\
587	0.000435941232690743\\
588	0.000593865412635933\\
589	0.000758286829767887\\
590	0.000929880038087214\\
591	0.00110901579535917\\
592	0.0012962244624629\\
593	0.00149227987384335\\
594	0.00169817885362274\\
595	0.0019152966930724\\
596	0.00214579709295045\\
597	0.00338203886614602\\
598	0.00642488516645657\\
599	0\\
600	0\\
};
\addplot [color=mycolor13,solid,forget plot]
  table[row sep=crcr]{%
1	0.0013780462490059\\
2	0.0013780462490059\\
3	0.0013780462490059\\
4	0.0013780462490059\\
5	0.0013780462490059\\
6	0.0013780462490059\\
7	0.0013780462490059\\
8	0.0013780462490059\\
9	0.0013780462490059\\
10	0.0013780462490059\\
11	0.0013780462490059\\
12	0.0013780462490059\\
13	0.0013780462490059\\
14	0.0013780462490059\\
15	0.0013780462490059\\
16	0.0013780462490059\\
17	0.0013780462490059\\
18	0.0013780462490059\\
19	0.0013780462490059\\
20	0.0013780462490059\\
21	0.0013780462490059\\
22	0.0013780462490059\\
23	0.0013780462490059\\
24	0.0013780462490059\\
25	0.0013780462490059\\
26	0.0013780462490059\\
27	0.0013780462490059\\
28	0.0013780462490059\\
29	0.0013780462490059\\
30	0.0013780462490059\\
31	0.0013780462490059\\
32	0.0013780462490059\\
33	0.0013780462490059\\
34	0.0013780462490059\\
35	0.0013780462490059\\
36	0.0013780462490059\\
37	0.0013780462490059\\
38	0.0013780462490059\\
39	0.0013780462490059\\
40	0.0013780462490059\\
41	0.0013780462490059\\
42	0.0013780462490059\\
43	0.0013780462490059\\
44	0.0013780462490059\\
45	0.0013780462490059\\
46	0.0013780462490059\\
47	0.0013780462490059\\
48	0.0013780462490059\\
49	0.0013780462490059\\
50	0.0013780462490059\\
51	0.0013780462490059\\
52	0.0013780462490059\\
53	0.0013780462490059\\
54	0.0013780462490059\\
55	0.0013780462490059\\
56	0.0013780462490059\\
57	0.0013780462490059\\
58	0.0013780462490059\\
59	0.0013780462490059\\
60	0.0013780462490059\\
61	0.0013780462490059\\
62	0.0013780462490059\\
63	0.0013780462490059\\
64	0.0013780462490059\\
65	0.0013780462490059\\
66	0.0013780462490059\\
67	0.0013780462490059\\
68	0.0013780462490059\\
69	0.0013780462490059\\
70	0.0013780462490059\\
71	0.0013780462490059\\
72	0.0013780462490059\\
73	0.0013780462490059\\
74	0.0013780462490059\\
75	0.0013780462490059\\
76	0.0013780462490059\\
77	0.0013780462490059\\
78	0.0013780462490059\\
79	0.0013780462490059\\
80	0.0013780462490059\\
81	0.0013780462490059\\
82	0.0013780462490059\\
83	0.0013780462490059\\
84	0.0013780462490059\\
85	0.0013780462490059\\
86	0.0013780462490059\\
87	0.0013780462490059\\
88	0.0013780462490059\\
89	0.0013780462490059\\
90	0.0013780462490059\\
91	0.0013780462490059\\
92	0.0013780462490059\\
93	0.0013780462490059\\
94	0.0013780462490059\\
95	0.0013780462490059\\
96	0.0013780462490059\\
97	0.0013780462490059\\
98	0.0013780462490059\\
99	0.0013780462490059\\
100	0.0013780462490059\\
101	0.0013780462490059\\
102	0.0013780462490059\\
103	0.0013780462490059\\
104	0.0013780462490059\\
105	0.0013780462490059\\
106	0.0013780462490059\\
107	0.0013780462490059\\
108	0.0013780462490059\\
109	0.0013780462490059\\
110	0.0013780462490059\\
111	0.0013780462490059\\
112	0.0013780462490059\\
113	0.0013780462490059\\
114	0.0013780462490059\\
115	0.0013780462490059\\
116	0.0013780462490059\\
117	0.0013780462490059\\
118	0.0013780462490059\\
119	0.0013780462490059\\
120	0.0013780462490059\\
121	0.0013780462490059\\
122	0.0013780462490059\\
123	0.0013780462490059\\
124	0.0013780462490059\\
125	0.0013780462490059\\
126	0.0013780462490059\\
127	0.0013780462490059\\
128	0.0013780462490059\\
129	0.0013780462490059\\
130	0.0013780462490059\\
131	0.0013780462490059\\
132	0.0013780462490059\\
133	0.0013780462490059\\
134	0.0013780462490059\\
135	0.0013780462490059\\
136	0.0013780462490059\\
137	0.0013780462490059\\
138	0.0013780462490059\\
139	0.0013780462490059\\
140	0.0013780462490059\\
141	0.0013780462490059\\
142	0.0013780462490059\\
143	0.0013780462490059\\
144	0.0013780462490059\\
145	0.0013780462490059\\
146	0.0013780462490059\\
147	0.0013780462490059\\
148	0.0013780462490059\\
149	0.0013780462490059\\
150	0.0013780462490059\\
151	0.0013780462490059\\
152	0.0013780462490059\\
153	0.0013780462490059\\
154	0.0013780462490059\\
155	0.0013780462490059\\
156	0.0013780462490059\\
157	0.0013780462490059\\
158	0.0013780462490059\\
159	0.0013780462490059\\
160	0.0013780462490059\\
161	0.0013780462490059\\
162	0.0013780462490059\\
163	0.0013780462490059\\
164	0.0013780462490059\\
165	0.0013780462490059\\
166	0.0013780462490059\\
167	0.0013780462490059\\
168	0.0013780462490059\\
169	0.0013780462490059\\
170	0.0013780462490059\\
171	0.0013780462490059\\
172	0.0013780462490059\\
173	0.0013780462490059\\
174	0.0013780462490059\\
175	0.0013780462490059\\
176	0.0013780462490059\\
177	0.0013780462490059\\
178	0.0013780462490059\\
179	0.0013780462490059\\
180	0.0013780462490059\\
181	0.0013780462490059\\
182	0.0013780462490059\\
183	0.0013780462490059\\
184	0.0013780462490059\\
185	0.0013780462490059\\
186	0.0013780462490059\\
187	0.0013780462490059\\
188	0.0013780462490059\\
189	0.0013780462490059\\
190	0.0013780462490059\\
191	0.0013780462490059\\
192	0.0013780462490059\\
193	0.0013780462490059\\
194	0.0013780462490059\\
195	0.0013780462490059\\
196	0.0013780462490059\\
197	0.0013780462490059\\
198	0.0013780462490059\\
199	0.0013780462490059\\
200	0.0013780462490059\\
201	0.0013780462490059\\
202	0.0013780462490059\\
203	0.0013780462490059\\
204	0.0013780462490059\\
205	0.0013780462490059\\
206	0.0013780462490059\\
207	0.0013780462490059\\
208	0.0013780462490059\\
209	0.0013780462490059\\
210	0.0013780462490059\\
211	0.0013780462490059\\
212	0.0013780462490059\\
213	0.0013780462490059\\
214	0.0013780462490059\\
215	0.0013780462490059\\
216	0.0013780462490059\\
217	0.0013780462490059\\
218	0.0013780462490059\\
219	0.0013780462490059\\
220	0.0013780462490059\\
221	0.0013780462490059\\
222	0.0013780462490059\\
223	0.0013780462490059\\
224	0.0013780462490059\\
225	0.0013780462490059\\
226	0.0013780462490059\\
227	0.0013780462490059\\
228	0.0013780462490059\\
229	0.0013780462490059\\
230	0.0013780462490059\\
231	0.0013780462490059\\
232	0.0013780462490059\\
233	0.0013780462490059\\
234	0.0013780462490059\\
235	0.0013780462490059\\
236	0.0013780462490059\\
237	0.0013780462490059\\
238	0.0013780462490059\\
239	0.0013780462490059\\
240	0.0013780462490059\\
241	0.0013780462490059\\
242	0.0013780462490059\\
243	0.0013780462490059\\
244	0.0013780462490059\\
245	0.0013780462490059\\
246	0.0013780462490059\\
247	0.0013780462490059\\
248	0.0013780462490059\\
249	0.0013780462490059\\
250	0.0013780462490059\\
251	0.0013780462490059\\
252	0.0013780462490059\\
253	0.0013780462490059\\
254	0.0013780462490059\\
255	0.0013780462490059\\
256	0.0013780462490059\\
257	0.0013780462490059\\
258	0.0013780462490059\\
259	0.0013780462490059\\
260	0.0013780462490059\\
261	0.0013780462490059\\
262	0.0013780462490059\\
263	0.0013780462490059\\
264	0.0013780462490059\\
265	0.0013780462490059\\
266	0.0013780462490059\\
267	0.0013780462490059\\
268	0.0013780462490059\\
269	0.0013780462490059\\
270	0.0013780462490059\\
271	0.0013780462490059\\
272	0.0013780462490059\\
273	0.0013780462490059\\
274	0.0013780462490059\\
275	0.0013780462490059\\
276	0.0013780462490059\\
277	0.0013780462490059\\
278	0.0013780462490059\\
279	0.0013780462490059\\
280	0.0013780462490059\\
281	0.0013780462490059\\
282	0.0013780462490059\\
283	0.0013780462490059\\
284	0.0013780462490059\\
285	0.0013780462490059\\
286	0.0013780462490059\\
287	0.0013780462490059\\
288	0.0013780462490059\\
289	0.0013780462490059\\
290	0.0013780462490059\\
291	0.0013780462490059\\
292	0.0013780462490059\\
293	0.0013780462490059\\
294	0.0013780462490059\\
295	0.0013780462490059\\
296	0.0013780462490059\\
297	0.0013780462490059\\
298	0.0013780462490059\\
299	0.0013780462490059\\
300	0.0013780462490059\\
301	0.0013780462490059\\
302	0.0013780462490059\\
303	0.0013780462490059\\
304	0.0013780462490059\\
305	0.0013780462490059\\
306	0.0013780462490059\\
307	0.0013780462490059\\
308	0.0013780462490059\\
309	0.0013780462490059\\
310	0.0013780462490059\\
311	0.0013780462490059\\
312	0.0013780462490059\\
313	0.0013780462490059\\
314	0.0013780462490059\\
315	0.0013780462490059\\
316	0.0013780462490059\\
317	0.0013780462490059\\
318	0.0013780462490059\\
319	0.0013780462490059\\
320	0.0013780462490059\\
321	0.0013780462490059\\
322	0.0013780462490059\\
323	0.0013780462490059\\
324	0.0013780462490059\\
325	0.0013780462490059\\
326	0.0013780462490059\\
327	0.0013780462490059\\
328	0.0013780462490059\\
329	0.0013780462490059\\
330	0.0013780462490059\\
331	0.0013780462490059\\
332	0.0013780462490059\\
333	0.0013780462490059\\
334	0.0013780462490059\\
335	0.0013780462490059\\
336	0.0013780462490059\\
337	0.0013780462490059\\
338	0.0013780462490059\\
339	0.0013780462490059\\
340	0.0013780462490059\\
341	0.0013780462490059\\
342	0.0013780462490059\\
343	0.0013780462490059\\
344	0.0013780462490059\\
345	0.0013780462490059\\
346	0.0013780462490059\\
347	0.0013780462490059\\
348	0.0013780462490059\\
349	0.0013780462490059\\
350	0.0013780462490059\\
351	0.0013780462490059\\
352	0.0013780462490059\\
353	0.0013780462490059\\
354	0.0013780462490059\\
355	0.0013780462490059\\
356	0.0013780462490059\\
357	0.0013780462490059\\
358	0.0013780462490059\\
359	0.0013780462490059\\
360	0.0013780462490059\\
361	0.0013780462490059\\
362	0.0013780462490059\\
363	0.0013780462490059\\
364	0.0013780462490059\\
365	0.0013780462490059\\
366	0.0013780462490059\\
367	0.0013780462490059\\
368	0.0013780462490059\\
369	0.0013780462490059\\
370	0.0013780462490059\\
371	0.0013780462490059\\
372	0.0013780462490059\\
373	0.0013780462490059\\
374	0.0013780462490059\\
375	0.0013780462490059\\
376	0.0013780462490059\\
377	0.0013780462490059\\
378	0.0013780462490059\\
379	0.0013780462490059\\
380	0.0013780462490059\\
381	0.0013780462490059\\
382	0.0013780462490059\\
383	0.0013780462490059\\
384	0.0013780462490059\\
385	0.0013780462490059\\
386	0.0013780462490059\\
387	0.0013780462490059\\
388	0.0013780462490059\\
389	0.0013780462490059\\
390	0.0013780462490059\\
391	0.0013780462490059\\
392	0.0013780462490059\\
393	0.0013780462490059\\
394	0.0013780462490059\\
395	0.0013780462490059\\
396	0.0013780462490059\\
397	0.0013780462490059\\
398	0.0013780462490059\\
399	0.0013780462490059\\
400	0.0013780462490059\\
401	0.0013780462490059\\
402	0.0013780462490059\\
403	0.0013780462490059\\
404	0.0013780462490059\\
405	0.0013780462490059\\
406	0.0013780462490059\\
407	0.0013780462490059\\
408	0.0013780462490059\\
409	0.0013780462490059\\
410	0.0013780462490059\\
411	0.0013780462490059\\
412	0.0013780462490059\\
413	0.0013780462490059\\
414	0.0013780462490059\\
415	0.0013780462490059\\
416	0.0013780462490059\\
417	0.0013780462490059\\
418	0.0013780462490059\\
419	0.0013780462490059\\
420	0.0013780462490059\\
421	0.0013780462490059\\
422	0.0013780462490059\\
423	0.0013780462490059\\
424	0.0013780462490059\\
425	0.0013780462490059\\
426	0.0013780462490059\\
427	0.0013780462490059\\
428	0.0013780462490059\\
429	0.0013780462490059\\
430	0.0013780462490059\\
431	0.0013780462490059\\
432	0.0013780462490059\\
433	0.0013780462490059\\
434	0.0013780462490059\\
435	0.0013780462490059\\
436	0.0013780462490059\\
437	0.0013780462490059\\
438	0.0013780462490059\\
439	0.0013780462490059\\
440	0.0013780462490059\\
441	0.0013780462490059\\
442	0.0013780462490059\\
443	0.0013780462490059\\
444	0.0013780462490059\\
445	0.0013780462490059\\
446	0.0013780462490059\\
447	0.0013780462490059\\
448	0.0013780462490059\\
449	0.0013780462490059\\
450	0.0013780462490059\\
451	0.0013780462490059\\
452	0.0013780462490059\\
453	0.0013780462490059\\
454	0.0013780462490059\\
455	0.0013780462490059\\
456	0.0013780462490059\\
457	0.0013780462490059\\
458	0.0013780462490059\\
459	0.0013780462490059\\
460	0.0013780462490059\\
461	0.0013780462490059\\
462	0.0013780462490059\\
463	0.0013780462490059\\
464	0.0013780462490059\\
465	0.0013780462490059\\
466	0.0013780462490059\\
467	0.0013780462490059\\
468	0.0013780462490059\\
469	0.0013780462490059\\
470	0.0013780462490059\\
471	0.0013780462490059\\
472	0.0013780462490059\\
473	0.0013780462490059\\
474	0.0013780462490059\\
475	0.0013780462490059\\
476	0.0013780462490059\\
477	0.0013780462490059\\
478	0.0013780462490059\\
479	0.0013780462490059\\
480	0.0013780462490059\\
481	0.0013780462490059\\
482	0.0013780462490059\\
483	0.0013780462490059\\
484	0.0013780462490059\\
485	0.0013780462490059\\
486	0.0013780462490059\\
487	0.0013780462490059\\
488	0.0013780462490059\\
489	0.0013780462490059\\
490	0.0013780462490059\\
491	0.0013780462490059\\
492	0.0013780462490059\\
493	0.0013780462490059\\
494	0.0013780462490059\\
495	0.0013780462490059\\
496	0.0013780462490059\\
497	0.0013780462490059\\
498	0.0013780462490059\\
499	0.0013780462490059\\
500	0.0013780462490059\\
501	0.0013780462490059\\
502	0.0013780462490059\\
503	0.0013780462490059\\
504	0.0013780462490059\\
505	0.0013780462490059\\
506	0.0013780462490059\\
507	0.0013780462490059\\
508	0.0013780462490059\\
509	0.0013780462490059\\
510	0.0013780462490059\\
511	0.0013780462490059\\
512	0.0013780462490059\\
513	0.0013780462490059\\
514	0.0013780462490059\\
515	0.0013780462490059\\
516	0.0013780462490059\\
517	0.0013780462490059\\
518	0.0013780462490059\\
519	0.0013780462490059\\
520	0.0013780462490059\\
521	0.0013780462490059\\
522	0.0013780462490059\\
523	0.0013780462490059\\
524	0.0013780462490059\\
525	0.0013780462490059\\
526	0.0013780462490059\\
527	0.0013780462490059\\
528	0.0013780462490059\\
529	0.0013780462490059\\
530	0.0013780462490059\\
531	0.0013780462490059\\
532	0.0013780462490059\\
533	0.0013780462490059\\
534	0.0013780462490059\\
535	0.0013780462490059\\
536	0.0013780462490059\\
537	0.0013780462490059\\
538	0.0013780462490059\\
539	0.0013780462490059\\
540	0.0013780462490059\\
541	0.0013780462490059\\
542	0.0013780462490059\\
543	0.0013780462490059\\
544	0.0013780462490059\\
545	0.0013780462490059\\
546	0.0013780462490059\\
547	0.0013780462490059\\
548	0.0013780462490059\\
549	0.0013780462490059\\
550	0.0013780462490059\\
551	0.0013780462490059\\
552	0.0013780462490059\\
553	0.0013780462490059\\
554	0.0013780462490059\\
555	0.0013780462490059\\
556	0.0013780462490059\\
557	0.0013780462490059\\
558	0.0013780462490059\\
559	0.0013780462490059\\
560	0.0013780462490059\\
561	0.0013780462490059\\
562	0.0013780462490059\\
563	0.0013780462490059\\
564	0.0013780462490059\\
565	0.00136279015443986\\
566	0.00125827733001416\\
567	0.00115323348010358\\
568	0.00104696223543405\\
569	0.000939078664145309\\
570	0.000831870952691528\\
571	0.000725446290622915\\
572	0.000624556670773367\\
573	0.00053496420760559\\
574	0.000390445210304555\\
575	0.000253767653105659\\
576	0.000124813533253903\\
577	3.73689878182394e-06\\
578	0\\
579	0\\
580	0\\
581	0\\
582	0\\
583	0\\
584	0\\
585	0\\
586	0\\
587	0\\
588	0\\
589	0\\
590	0\\
591	0\\
592	6.50501568911646e-05\\
593	0.000171129827925065\\
594	0.000294096086145507\\
595	0.000433306443467276\\
596	0.000584745759468008\\
597	0.000739344758132349\\
598	0.00345717029705862\\
599	0\\
600	0\\
};
\addplot [color=mycolor14,solid,forget plot]
  table[row sep=crcr]{%
1	0.00634705736923244\\
2	0.00634705707591813\\
3	0.00634705677729562\\
4	0.00634705647326904\\
5	0.00634705616374076\\
6	0.00634705584861141\\
7	0.00634705552777983\\
8	0.00634705520114302\\
9	0.00634705486859616\\
10	0.00634705453003251\\
11	0.00634705418534343\\
12	0.00634705383441831\\
13	0.00634705347714455\\
14	0.00634705311340754\\
15	0.00634705274309059\\
16	0.00634705236607491\\
17	0.0063470519822396\\
18	0.00634705159146155\\
19	0.00634705119361545\\
20	0.00634705078857375\\
21	0.00634705037620657\\
22	0.00634704995638174\\
23	0.00634704952896467\\
24	0.00634704909381839\\
25	0.00634704865080343\\
26	0.00634704819977784\\
27	0.00634704774059711\\
28	0.00634704727311413\\
29	0.00634704679717915\\
30	0.00634704631263973\\
31	0.00634704581934069\\
32	0.00634704531712405\\
33	0.00634704480582902\\
34	0.00634704428529189\\
35	0.00634704375534603\\
36	0.00634704321582182\\
37	0.00634704266654658\\
38	0.00634704210734454\\
39	0.00634704153803677\\
40	0.00634704095844114\\
41	0.00634704036837222\\
42	0.00634703976764128\\
43	0.00634703915605621\\
44	0.00634703853342143\\
45	0.00634703789953786\\
46	0.00634703725420284\\
47	0.0063470365972101\\
48	0.00634703592834965\\
49	0.00634703524740774\\
50	0.00634703455416678\\
51	0.00634703384840529\\
52	0.00634703312989782\\
53	0.00634703239841486\\
54	0.00634703165372281\\
55	0.00634703089558388\\
56	0.006347030123756\\
57	0.00634702933799278\\
58	0.00634702853804341\\
59	0.00634702772365259\\
60	0.00634702689456044\\
61	0.00634702605050245\\
62	0.00634702519120935\\
63	0.00634702431640704\\
64	0.00634702342581654\\
65	0.00634702251915386\\
66	0.00634702159612992\\
67	0.0063470206564505\\
68	0.00634701969981608\\
69	0.00634701872592179\\
70	0.0063470177344573\\
71	0.00634701672510676\\
72	0.00634701569754863\\
73	0.00634701465145565\\
74	0.00634701358649468\\
75	0.00634701250232665\\
76	0.00634701139860642\\
77	0.00634701027498266\\
78	0.0063470091310978\\
79	0.00634700796658785\\
80	0.00634700678108232\\
81	0.00634700557420411\\
82	0.00634700434556936\\
83	0.00634700309478739\\
84	0.00634700182146052\\
85	0.00634700052518396\\
86	0.00634699920554571\\
87	0.0063469978621264\\
88	0.00634699649449918\\
89	0.00634699510222959\\
90	0.00634699368487539\\
91	0.00634699224198647\\
92	0.00634699077310468\\
93	0.0063469892777637\\
94	0.00634698775548889\\
95	0.00634698620579713\\
96	0.00634698462819671\\
97	0.00634698302218713\\
98	0.00634698138725896\\
99	0.00634697972289372\\
100	0.00634697802856366\\
101	0.00634697630373162\\
102	0.00634697454785088\\
103	0.00634697276036498\\
104	0.00634697094070752\\
105	0.00634696908830205\\
106	0.0063469672025618\\
107	0.00634696528288959\\
108	0.0063469633286776\\
109	0.00634696133930715\\
110	0.00634695931414859\\
111	0.00634695725256103\\
112	0.00634695515389217\\
113	0.00634695301747813\\
114	0.00634695084264318\\
115	0.00634694862869959\\
116	0.00634694637494736\\
117	0.00634694408067408\\
118	0.00634694174515463\\
119	0.006346939367651\\
120	0.00634693694741208\\
121	0.00634693448367337\\
122	0.00634693197565678\\
123	0.00634692942257041\\
124	0.00634692682360826\\
125	0.00634692417795001\\
126	0.00634692148476076\\
127	0.00634691874319076\\
128	0.00634691595237519\\
129	0.00634691311143384\\
130	0.00634691021947086\\
131	0.0063469072755745\\
132	0.00634690427881681\\
133	0.00634690122825337\\
134	0.00634689812292295\\
135	0.00634689496184731\\
136	0.00634689174403079\\
137	0.00634688846846009\\
138	0.00634688513410389\\
139	0.0063468817399126\\
140	0.00634687828481799\\
141	0.00634687476773284\\
142	0.0063468711875507\\
143	0.00634686754314543\\
144	0.00634686383337096\\
145	0.00634686005706085\\
146	0.00634685621302799\\
147	0.00634685230006422\\
148	0.00634684831693993\\
149	0.00634684426240374\\
150	0.00634684013518206\\
151	0.00634683593397872\\
152	0.00634683165747458\\
153	0.0063468273043271\\
154	0.00634682287316997\\
155	0.00634681836261263\\
156	0.0063468137712399\\
157	0.00634680909761152\\
158	0.00634680434026169\\
159	0.00634679949769865\\
160	0.00634679456840423\\
161	0.00634678955083332\\
162	0.00634678444341348\\
163	0.00634677924454438\\
164	0.00634677395259738\\
165	0.00634676856591495\\
166	0.00634676308281026\\
167	0.00634675750156654\\
168	0.00634675182043668\\
169	0.00634674603764258\\
170	0.00634674015137471\\
171	0.00634673415979143\\
172	0.00634672806101856\\
173	0.00634672185314869\\
174	0.00634671553424066\\
175	0.00634670910231894\\
176	0.00634670255537304\\
177	0.00634669589135686\\
178	0.00634668910818808\\
179	0.00634668220374752\\
180	0.0063466751758785\\
181	0.00634666802238616\\
182	0.00634666074103676\\
183	0.00634665332955706\\
184	0.00634664578563356\\
185	0.00634663810691181\\
186	0.00634663029099569\\
187	0.00634662233544665\\
188	0.006346614237783\\
189	0.00634660599547909\\
190	0.00634659760596459\\
191	0.00634658906662365\\
192	0.00634658037479413\\
193	0.00634657152776673\\
194	0.00634656252278421\\
195	0.00634655335704053\\
196	0.00634654402767992\\
197	0.00634653453179609\\
198	0.00634652486643127\\
199	0.0063465150285753\\
200	0.00634650501516476\\
201	0.00634649482308192\\
202	0.00634648444915386\\
203	0.00634647389015147\\
204	0.00634646314278842\\
205	0.00634645220372016\\
206	0.00634644106954291\\
207	0.00634642973679255\\
208	0.00634641820194359\\
209	0.00634640646140806\\
210	0.0063463945115344\\
211	0.00634638234860632\\
212	0.00634636996884166\\
213	0.00634635736839118\\
214	0.00634634454333739\\
215	0.00634633148969334\\
216	0.00634631820340137\\
217	0.00634630468033181\\
218	0.00634629091628174\\
219	0.00634627690697368\\
220	0.00634626264805419\\
221	0.0063462481350926\\
222	0.00634623336357957\\
223	0.00634621832892568\\
224	0.00634620302646004\\
225	0.00634618745142875\\
226	0.00634617159899349\\
227	0.00634615546422996\\
228	0.00634613904212636\\
229	0.00634612232758177\\
230	0.00634610531540462\\
231	0.006346088000311\\
232	0.00634607037692303\\
233	0.00634605243976716\\
234	0.00634603418327243\\
235	0.00634601560176877\\
236	0.00634599668948512\\
237	0.00634597744054773\\
238	0.00634595784897819\\
239	0.00634593790869163\\
240	0.00634591761349475\\
241	0.00634589695708388\\
242	0.00634587593304298\\
243	0.00634585453484162\\
244	0.00634583275583291\\
245	0.00634581058925138\\
246	0.00634578802821086\\
247	0.00634576506570227\\
248	0.00634574169459141\\
249	0.00634571790761672\\
250	0.00634569369738691\\
251	0.00634566905637868\\
252	0.00634564397693429\\
253	0.00634561845125914\\
254	0.00634559247141927\\
255	0.00634556602933886\\
256	0.00634553911679764\\
257	0.00634551172542829\\
258	0.00634548384671375\\
259	0.00634545547198453\\
260	0.00634542659241593\\
261	0.00634539719902521\\
262	0.00634536728266877\\
263	0.00634533683403918\\
264	0.00634530584366226\\
265	0.00634527430189401\\
266	0.00634524219891754\\
267	0.00634520952473997\\
268	0.00634517626918919\\
269	0.00634514242191067\\
270	0.00634510797236414\\
271	0.00634507290982028\\
272	0.00634503722335723\\
273	0.00634500090185709\\
274	0.00634496393400233\\
275	0.00634492630827227\\
276	0.00634488801293938\\
277	0.0063448490360656\\
278	0.00634480936549852\\
279	0.0063447689888675\\
280	0.00634472789357982\\
281	0.00634468606681663\\
282	0.00634464349552894\\
283	0.00634460016643353\\
284	0.00634455606600872\\
285	0.0063445111804902\\
286	0.00634446549586669\\
287	0.00634441899787558\\
288	0.00634437167199854\\
289	0.00634432350345699\\
290	0.00634427447720758\\
291	0.00634422457793762\\
292	0.00634417379006036\\
293	0.0063441220977103\\
294	0.00634406948473844\\
295	0.00634401593470744\\
296	0.00634396143088677\\
297	0.00634390595624778\\
298	0.00634384949345877\\
299	0.00634379202488\\
300	0.00634373353255865\\
301	0.00634367399822384\\
302	0.00634361340328147\\
303	0.00634355172880927\\
304	0.00634348895555162\\
305	0.00634342506391455\\
306	0.0063433600339606\\
307	0.00634329384540378\\
308	0.00634322647760438\\
309	0.00634315790956398\\
310	0.00634308811992051\\
311	0.00634301708694385\\
312	0.00634294478853137\\
313	0.00634287120220282\\
314	0.0063427963050959\\
315	0.00634272007396187\\
316	0.00634264248516149\\
317	0.00634256351466105\\
318	0.00634248313802878\\
319	0.0063424013304314\\
320	0.00634231806663113\\
321	0.00634223332098303\\
322	0.00634214706743279\\
323	0.00634205927951495\\
324	0.00634196993035181\\
325	0.00634187899265279\\
326	0.00634178643871461\\
327	0.00634169224042198\\
328	0.00634159636924884\\
329	0.0063414987962596\\
330	0.00634139949210948\\
331	0.00634129842704191\\
332	0.00634119557088033\\
333	0.00634109089301242\\
334	0.00634098436237369\\
335	0.00634087594746614\\
336	0.00634076561650549\\
337	0.00634065333780671\\
338	0.00634053908012841\\
339	0.00634042281148967\\
340	0.0063403044970337\\
341	0.00634018409871652\\
342	0.00634006157113215\\
343	0.00633993685233052\\
344	0.00633980984603337\\
345	0.00633968039571144\\
346	0.00633954827735552\\
347	0.00633941331734338\\
348	0.00633927576477506\\
349	0.00633913590350577\\
350	0.00633899369546465\\
351	0.00633884910201422\\
352	0.00633870208394732\\
353	0.00633855260148489\\
354	0.00633840061427464\\
355	0.00633824608138921\\
356	0.0063380889613211\\
357	0.00633792921197611\\
358	0.00633776679067756\\
359	0.00633760165416524\\
360	0.00633743375859476\\
361	0.00633726305953701\\
362	0.00633708951197813\\
363	0.0063369130703195\\
364	0.00633673368837812\\
365	0.00633655131938696\\
366	0.00633636591599548\\
367	0.0063361774302701\\
368	0.00633598581369455\\
369	0.00633579101716998\\
370	0.00633559299101447\\
371	0.00633539168496156\\
372	0.00633518704815668\\
373	0.00633497902915057\\
374	0.00633476757589027\\
375	0.00633455263571372\\
376	0.00633433415536172\\
377	0.0063341120810081\\
378	0.00633388635826129\\
379	0.00633365693222053\\
380	0.00633342374756986\\
381	0.00633318674867619\\
382	0.00633294587952931\\
383	0.00633270108308434\\
384	0.00633245229929575\\
385	0.00633219946197002\\
386	0.00633194249927014\\
387	0.00633168134762682\\
388	0.00633141594441352\\
389	0.00633114622526075\\
390	0.00633087212386279\\
391	0.00633059357183474\\
392	0.00633031049874965\\
393	0.00633002283268106\\
394	0.00632973050203992\\
395	0.00632943344055455\\
396	0.00632913159953022\\
397	0.00632882497602954\\
398	0.00632851367297699\\
399	0.00632819801383586\\
400	0.00632787871871675\\
401	0.00632755703509414\\
402	0.00632723434369704\\
403	0.00632691022043075\\
404	0.00632658003640803\\
405	0.00632624367882569\\
406	0.00632590102995405\\
407	0.00632555196672422\\
408	0.00632519636035737\\
409	0.00632483407667918\\
410	0.00632446497754873\\
411	0.00632408892036726\\
412	0.00632370574530547\\
413	0.0063233152870529\\
414	0.00632291737454029\\
415	0.00632251183064566\\
416	0.00632209847188187\\
417	0.00632167710805953\\
418	0.00632124754191151\\
419	0.00632080956865844\\
420	0.00632036297551085\\
421	0.00631990754118923\\
422	0.00631944303564629\\
423	0.00631896921980199\\
424	0.00631848584474056\\
425	0.00631799265114201\\
426	0.00631748936867251\\
427	0.00631697571532665\\
428	0.00631645139670927\\
429	0.00631591610523212\\
430	0.00631536951917585\\
431	0.00631481130152096\\
432	0.0063142410983804\\
433	0.00631365853681534\\
434	0.00631306322201186\\
435	0.00631245473496562\\
436	0.00631183263545229\\
437	0.00631119648081164\\
438	0.00631054585902489\\
439	0.00630988033286531\\
440	0.00630919937951012\\
441	0.00630850244862298\\
442	0.00630778896062646\\
443	0.00630705830485506\\
444	0.00630630983758596\\
445	0.00630554287987795\\
446	0.00630475671460947\\
447	0.00630395057978746\\
448	0.00630312364952706\\
449	0.00630227499286354\\
450	0.00630140346092567\\
451	0.00630050737731262\\
452	0.00629958369754549\\
453	0.00629862575552163\\
454	0.00629761732521094\\
455	0.0062965172609967\\
456	0.00629522043131399\\
457	0.00629346826565445\\
458	0.00629058504385814\\
459	0.0062873984227585\\
460	0.00628414460028995\\
461	0.00628080773056847\\
462	0.00627737393016521\\
463	0.00627388356313013\\
464	0.00627033595250894\\
465	0.00626672881184384\\
466	0.00626305953089114\\
467	0.00625932510476634\\
468	0.00625552214579273\\
469	0.0062516473709978\\
470	0.00624769936191294\\
471	0.00624367602190079\\
472	0.00623957699600573\\
473	0.00623540724679653\\
474	0.00623118227627135\\
475	0.00622693215559939\\
476	0.00622268415786011\\
477	0.00621837197795093\\
478	0.00621391858504133\\
479	0.00620931339029792\\
480	0.00620454353957697\\
481	0.00619959164428468\\
482	0.0061944298689289\\
483	0.00618900452079593\\
484	0.00618319741935313\\
485	0.00617674152536256\\
486	0.00616948027811247\\
487	0.00616210182564511\\
488	0.00615460347021238\\
489	0.00614698244999517\\
490	0.00613923586265543\\
491	0.00613136059652419\\
492	0.0061233530577195\\
493	0.00611520758579711\\
494	0.00610691552533023\\
495	0.00609848237957893\\
496	0.00608990449054164\\
497	0.00608117670253305\\
498	0.00607229238821892\\
499	0.00606324736646706\\
500	0.00605403705724364\\
501	0.00604465615743606\\
502	0.00603509786883715\\
503	0.00602535233006483\\
504	0.00601540563529222\\
505	0.00600525878536194\\
506	0.00599488587033331\\
507	0.00598422125494825\\
508	0.00597312631007669\\
509	0.00596172141658392\\
510	0.00595018659363723\\
511	0.0059384445843753\\
512	0.00592631108743226\\
513	0.00591346613950271\\
514	0.00590030505807963\\
515	0.00588707956134307\\
516	0.00587379765610018\\
517	0.00586048147386627\\
518	0.00584718801494137\\
519	0.00583405829195754\\
520	0.00582141659957693\\
521	0.00580989012708187\\
522	0.00580008189998949\\
523	0.00578994549122259\\
524	0.00577913384386583\\
525	0.00576657393904838\\
526	0.00574897062406128\\
527	0.00570949806036672\\
528	0.00566622002656595\\
529	0.00562168348095022\\
530	0.00557566290236694\\
531	0.00552774050188853\\
532	0.00547700589151219\\
533	0.00541874111612479\\
534	0.00535561486132491\\
535	0.00529235810299489\\
536	0.00522897636497744\\
537	0.00516547212723779\\
538	0.00510184062425884\\
539	0.00503805005930613\\
540	0.0049739785608551\\
541	0.00490919741497271\\
542	0.00484167426677888\\
543	0.0047750592880797\\
544	0.00470938112578479\\
545	0.00464466913499734\\
546	0.00458095349021259\\
547	0.00451826420357372\\
548	0.00445662654337949\\
549	0.00439604589693207\\
550	0.00433645820630629\\
551	0.00427756785872679\\
552	0.00421831745834273\\
553	0.00415492967898489\\
554	0.0040927650695408\\
555	0.00403290499091029\\
556	0.00397537262231718\\
557	0.00392025786722828\\
558	0.00386761667012469\\
559	0.00381749584867485\\
560	0.0037699710534416\\
561	0.00372520839465025\\
562	0.00368359691573897\\
563	0.00364607931640829\\
564	0.00361495875009759\\
565	0.00358407587802502\\
566	0.00350180811894708\\
567	0.00341627094766928\\
568	0.00332212922189586\\
569	0.00322378609530446\\
570	0.00312258660790876\\
571	0.00301842823208778\\
572	0.00291083880439416\\
573	0.00279199464559352\\
574	0.00262090950248387\\
575	0.00244700241101585\\
576	0.00227017807912024\\
577	0.00209042395100323\\
578	0.0019079951662922\\
579	0.00172275411076606\\
580	0.00153451315287464\\
581	0.00134245890765045\\
582	0.00114797364967888\\
583	0.000951303160335593\\
584	0.000752186592492726\\
585	0.000549776472909163\\
586	0.000341218417480763\\
587	0.000117616391847642\\
588	0\\
589	0\\
590	0\\
591	0\\
592	0\\
593	0\\
594	0\\
595	0\\
596	0\\
597	0\\
598	0\\
599	0\\
600	0\\
};
\addplot [color=mycolor15,solid,forget plot]
  table[row sep=crcr]{%
1	0.0050928946278291\\
2	0.00509289365214289\\
3	0.00509289265879869\\
4	0.00509289164747749\\
5	0.00509289061785456\\
6	0.00509288956959928\\
7	0.00509288850237511\\
8	0.00509288741583942\\
9	0.00509288630964345\\
10	0.00509288518343212\\
11	0.00509288403684399\\
12	0.00509288286951108\\
13	0.00509288168105882\\
14	0.00509288047110587\\
15	0.00509287923926405\\
16	0.00509287798513818\\
17	0.00509287670832597\\
18	0.00509287540841791\\
19	0.0050928740849971\\
20	0.00509287273763914\\
21	0.00509287136591202\\
22	0.00509286996937594\\
23	0.00509286854758319\\
24	0.00509286710007804\\
25	0.00509286562639651\\
26	0.00509286412606633\\
27	0.00509286259860671\\
28	0.00509286104352824\\
29	0.0050928594603327\\
30	0.00509285784851292\\
31	0.0050928562075526\\
32	0.0050928545369262\\
33	0.0050928528360987\\
34	0.00509285110452549\\
35	0.00509284934165218\\
36	0.00509284754691442\\
37	0.0050928457197377\\
38	0.00509284385953725\\
39	0.00509284196571773\\
40	0.00509284003767317\\
41	0.0050928380747867\\
42	0.00509283607643036\\
43	0.00509283404196496\\
44	0.0050928319707398\\
45	0.00509282986209254\\
46	0.00509282771534892\\
47	0.00509282552982261\\
48	0.00509282330481497\\
49	0.00509282103961481\\
50	0.0050928187334982\\
51	0.00509281638572822\\
52	0.00509281399555476\\
53	0.00509281156221422\\
54	0.00509280908492936\\
55	0.00509280656290897\\
56	0.00509280399534768\\
57	0.00509280138142569\\
58	0.0050927987203085\\
59	0.00509279601114666\\
60	0.00509279325307551\\
61	0.0050927904452149\\
62	0.00509278758666891\\
63	0.0050927846765256\\
64	0.00509278171385666\\
65	0.0050927786977172\\
66	0.00509277562714539\\
67	0.00509277250116219\\
68	0.00509276931877105\\
69	0.00509276607895757\\
70	0.0050927627806892\\
71	0.00509275942291493\\
72	0.00509275600456492\\
73	0.00509275252455024\\
74	0.00509274898176244\\
75	0.00509274537507326\\
76	0.00509274170333429\\
77	0.00509273796537655\\
78	0.00509273416001019\\
79	0.00509273028602408\\
80	0.00509272634218544\\
81	0.00509272232723947\\
82	0.00509271823990895\\
83	0.00509271407889383\\
84	0.00509270984287085\\
85	0.00509270553049308\\
86	0.00509270114038959\\
87	0.00509269667116491\\
88	0.0050926921213987\\
89	0.00509268748964523\\
90	0.00509268277443298\\
91	0.00509267797426414\\
92	0.00509267308761421\\
93	0.00509266811293145\\
94	0.00509266304863643\\
95	0.00509265789312156\\
96	0.00509265264475058\\
97	0.00509264730185801\\
98	0.0050926418627487\\
99	0.00509263632569725\\
100	0.00509263068894751\\
101	0.00509262495071201\\
102	0.00509261910917141\\
103	0.00509261316247397\\
104	0.00509260710873491\\
105	0.0050926009460359\\
106	0.00509259467242439\\
107	0.00509258828591309\\
108	0.00509258178447927\\
109	0.00509257516606419\\
110	0.00509256842857244\\
111	0.00509256156987129\\
112	0.00509255458779004\\
113	0.00509254748011936\\
114	0.00509254024461056\\
115	0.00509253287897496\\
116	0.00509252538088311\\
117	0.00509251774796413\\
118	0.00509250997780495\\
119	0.00509250206794957\\
120	0.0050924940158983\\
121	0.00509248581910697\\
122	0.00509247747498618\\
123	0.00509246898090049\\
124	0.00509246033416757\\
125	0.00509245153205742\\
126	0.0050924425717915\\
127	0.00509243345054189\\
128	0.00509242416543039\\
129	0.00509241471352766\\
130	0.0050924050918523\\
131	0.00509239529736995\\
132	0.00509238532699234\\
133	0.00509237517757631\\
134	0.00509236484592291\\
135	0.00509235432877635\\
136	0.00509234362282302\\
137	0.0050923327246905\\
138	0.00509232163094645\\
139	0.00509231033809762\\
140	0.00509229884258875\\
141	0.00509228714080147\\
142	0.00509227522905317\\
143	0.00509226310359592\\
144	0.00509225076061526\\
145	0.00509223819622906\\
146	0.00509222540648632\\
147	0.00509221238736592\\
148	0.00509219913477545\\
149	0.00509218564454988\\
150	0.00509217191245034\\
151	0.00509215793416276\\
152	0.00509214370529661\\
153	0.00509212922138346\\
154	0.00509211447787569\\
155	0.00509209947014505\\
156	0.00509208419348126\\
157	0.00509206864309053\\
158	0.0050920528140941\\
159	0.00509203670152675\\
160	0.00509202030033528\\
161	0.00509200360537694\\
162	0.00509198661141783\\
163	0.00509196931313133\\
164	0.00509195170509646\\
165	0.00509193378179618\\
166	0.00509191553761573\\
167	0.00509189696684087\\
168	0.00509187806365615\\
169	0.00509185882214313\\
170	0.00509183923627852\\
171	0.00509181929993237\\
172	0.00509179900686614\\
173	0.00509177835073085\\
174	0.00509175732506505\\
175	0.0050917359232929\\
176	0.00509171413872208\\
177	0.0050916919645418\\
178	0.00509166939382065\\
179	0.00509164641950451\\
180	0.00509162303441435\\
181	0.00509159923124401\\
182	0.005091575002558\\
183	0.00509155034078917\\
184	0.00509152523823638\\
185	0.00509149968706214\\
186	0.00509147367929021\\
187	0.00509144720680311\\
188	0.00509142026133964\\
189	0.00509139283449233\\
190	0.00509136491770483\\
191	0.00509133650226931\\
192	0.00509130757932373\\
193	0.00509127813984914\\
194	0.00509124817466685\\
195	0.00509121767443567\\
196	0.00509118662964895\\
197	0.00509115503063168\\
198	0.00509112286753751\\
199	0.00509109013034565\\
200	0.00509105680885785\\
201	0.00509102289269518\\
202	0.00509098837129484\\
203	0.00509095323390692\\
204	0.005090917469591\\
205	0.00509088106721283\\
206	0.00509084401544081\\
207	0.00509080630274256\\
208	0.00509076791738125\\
209	0.00509072884741202\\
210	0.00509068908067825\\
211	0.00509064860480777\\
212	0.00509060740720903\\
213	0.00509056547506718\\
214	0.00509052279534006\\
215	0.00509047935475418\\
216	0.00509043513980055\\
217	0.00509039013673047\\
218	0.00509034433155126\\
219	0.00509029771002189\\
220	0.0050902502576485\\
221	0.00509020195967992\\
222	0.00509015280110303\\
223	0.00509010276663806\\
224	0.00509005184073383\\
225	0.00509000000756285\\
226	0.00508994725101637\\
227	0.00508989355469935\\
228	0.00508983890192529\\
229	0.00508978327571099\\
230	0.00508972665877122\\
231	0.0050896690335133\\
232	0.00508961038203154\\
233	0.00508955068610163\\
234	0.00508948992717488\\
235	0.00508942808637236\\
236	0.00508936514447898\\
237	0.00508930108193742\\
238	0.00508923587884193\\
239	0.00508916951493205\\
240	0.00508910196958621\\
241	0.00508903322181522\\
242	0.00508896325025557\\
243	0.00508889203316274\\
244	0.00508881954840423\\
245	0.00508874577345261\\
246	0.00508867068537832\\
247	0.0050885942608424\\
248	0.0050885164760891\\
249	0.0050884373069383\\
250	0.00508835672877782\\
251	0.00508827471655561\\
252	0.00508819124477175\\
253	0.00508810628747034\\
254	0.00508801981823121\\
255	0.00508793181016152\\
256	0.00508784223588716\\
257	0.00508775106754401\\
258	0.00508765827676908\\
259	0.0050875638346914\\
260	0.00508746771192282\\
261	0.0050873698785486\\
262	0.00508727030411788\\
263	0.0050871689576339\\
264	0.00508706580754409\\
265	0.00508696082172997\\
266	0.00508685396749688\\
267	0.00508674521156349\\
268	0.00508663452005114\\
269	0.00508652185847301\\
270	0.00508640719172311\\
271	0.00508629048406506\\
272	0.00508617169912081\\
273	0.0050860507998589\\
274	0.00508592774858237\\
275	0.00508580250691664\\
276	0.00508567503579743\\
277	0.00508554529545836\\
278	0.00508541324541817\\
279	0.00508527884446789\\
280	0.00508514205065768\\
281	0.0050850028212835\\
282	0.00508486111287353\\
283	0.00508471688117443\\
284	0.00508457008113731\\
285	0.00508442066690353\\
286	0.00508426859179028\\
287	0.00508411380827591\\
288	0.00508395626798505\\
289	0.00508379592167355\\
290	0.00508363271921317\\
291	0.00508346660957611\\
292	0.00508329754081924\\
293	0.00508312546006827\\
294	0.00508295031350161\\
295	0.00508277204633408\\
296	0.00508259060280047\\
297	0.0050824059261389\\
298	0.00508221795857403\\
299	0.00508202664130014\\
300	0.00508183191446406\\
301	0.00508163371714798\\
302	0.00508143198735219\\
303	0.00508122666197778\\
304	0.00508101767680916\\
305	0.00508080496649673\\
306	0.00508058846453944\\
307	0.00508036810326731\\
308	0.005080143813824\\
309	0.00507991552614912\\
310	0.0050796831689607\\
311	0.00507944666973814\\
312	0.00507920595470679\\
313	0.00507896094882321\\
314	0.00507871157575707\\
315	0.00507845775787563\\
316	0.00507819941622867\\
317	0.00507793647053405\\
318	0.00507766883916395\\
319	0.00507739643913194\\
320	0.00507711918608097\\
321	0.00507683699427252\\
322	0.00507654977657687\\
323	0.00507625744446493\\
324	0.00507595990800155\\
325	0.00507565707584084\\
326	0.00507534885522337\\
327	0.00507503515197581\\
328	0.00507471587051285\\
329	0.00507439091384142\\
330	0.00507406018356655\\
331	0.00507372357989712\\
332	0.00507338100164791\\
333	0.00507303234623229\\
334	0.00507267750964002\\
335	0.00507231638640881\\
336	0.00507194886965547\\
337	0.00507157485138313\\
338	0.00507119422348189\\
339	0.00507080687923724\\
340	0.00507041271038931\\
341	0.00507001160023853\\
342	0.00506960342422547\\
343	0.00506918803930844\\
344	0.00506876525882263\\
345	0.0050683347975141\\
346	0.00506789616893734\\
347	0.00506744856867696\\
348	0.00506699109119338\\
349	0.00506652456440268\\
350	0.0050660501975954\\
351	0.00506556786113188\\
352	0.0050650774234284\\
353	0.00506457875094549\\
354	0.00506407170817921\\
355	0.00506355615765624\\
356	0.00506303195992941\\
357	0.00506249897356119\\
358	0.00506195705509287\\
359	0.00506140605906212\\
360	0.00506084583799855\\
361	0.0050602762424204\\
362	0.00505969712083228\\
363	0.00505910831972386\\
364	0.00505850968356943\\
365	0.00505790105482817\\
366	0.00505728227394488\\
367	0.00505665317935105\\
368	0.00505601360746598\\
369	0.00505536339269754\\
370	0.00505470236744231\\
371	0.00505403036208429\\
372	0.00505334720499054\\
373	0.00505265272250077\\
374	0.00505194673890596\\
375	0.00505122907641344\\
376	0.00505049955511283\\
377	0.00504975799299523\\
378	0.00504900420605909\\
379	0.00504823800823944\\
380	0.00504745921149303\\
381	0.00504666762595958\\
382	0.00504586306014246\\
383	0.00504504532078459\\
384	0.00504421421144329\\
385	0.00504336952770216\\
386	0.00504251104749455\\
387	0.00504163852689711\\
388	0.00504075174925823\\
389	0.0050398505012029\\
390	0.00503893456352778\\
391	0.00503800371050569\\
392	0.00503705770926875\\
393	0.00503609631957294\\
394	0.00503511929472342\\
395	0.00503412638559884\\
396	0.00503311735245732\\
397	0.00503209199545522\\
398	0.00503105022820599\\
399	0.00502999224452367\\
400	0.00502891886763283\\
401	0.00502783218799178\\
402	0.00502673639327816\\
403	0.00502563753529514\\
404	0.00502453721029399\\
405	0.00502341626063081\\
406	0.00502227430254632\\
407	0.00502111093574222\\
408	0.00501992574202896\\
409	0.00501871828365271\\
410	0.00501748810334104\\
411	0.00501623472918392\\
412	0.005014957679507\\
413	0.00501365640947008\\
414	0.00501233035555447\\
415	0.00501097893461764\\
416	0.00500960154289323\\
417	0.00500819755493185\\
418	0.00500676632246699\\
419	0.0050053071731649\\
420	0.00500381940917676\\
421	0.0050023023054178\\
422	0.00500075510777094\\
423	0.00499917703210561\\
424	0.00499756726373227\\
425	0.00499592495449141\\
426	0.0049942492208264\\
427	0.0049925391417214\\
428	0.00499079375648711\\
429	0.00498901206236689\\
430	0.00498719301191277\\
431	0.00498533551003465\\
432	0.00498343841053814\\
433	0.00498150051182483\\
434	0.00497952055128766\\
435	0.00497749719812845\\
436	0.00497542904612344\\
437	0.00497331461496192\\
438	0.00497115238815756\\
439	0.00496894092847353\\
440	0.00496667873037992\\
441	0.00496436401488461\\
442	0.00496199490964288\\
443	0.00495956944314037\\
444	0.00495708553846413\\
445	0.00495454100667428\\
446	0.00495193353985444\\
447	0.0049492607032568\\
448	0.00494651992026552\\
449	0.00494370842112155\\
450	0.00494082314810791\\
451	0.00493786049973932\\
452	0.00493481563215763\\
453	0.00493168056649731\\
454	0.00492843908689531\\
455	0.0049250531026715\\
456	0.00492142645084939\\
457	0.00491730494594139\\
458	0.00491208803876988\\
459	0.00490239900538334\\
460	0.00489149692452099\\
461	0.00488036978683903\\
462	0.0048689610936534\\
463	0.00485718523999083\\
464	0.00484520977544474\\
465	0.00483303324960802\\
466	0.0048206477514237\\
467	0.00480804428243009\\
468	0.0047952124548571\\
469	0.00478214021535745\\
470	0.00476881468701787\\
471	0.00475523106750443\\
472	0.00474138065763046\\
473	0.00472725851969782\\
474	0.00471287490790399\\
475	0.00469827434654054\\
476	0.00468356348793886\\
477	0.00466889699916233\\
478	0.00465405026924246\\
479	0.00463871239017449\\
480	0.00462284724770989\\
481	0.00460641219022524\\
482	0.00458935239600222\\
483	0.00457158616356049\\
484	0.0045529655160597\\
485	0.0045331696140612\\
486	0.00451120800814773\\
487	0.00448604223992208\\
488	0.00446045227420199\\
489	0.00443442838203219\\
490	0.00440796066997701\\
491	0.00438103880866148\\
492	0.00435365189049488\\
493	0.00432578813853256\\
494	0.00429743538016299\\
495	0.00426858429943873\\
496	0.00423922283061158\\
497	0.00420933758393212\\
498	0.00417891159500239\\
499	0.00414792434834379\\
500	0.00411636181189269\\
501	0.00408420885099331\\
502	0.0040514483152974\\
503	0.00401805870402856\\
504	0.00398400845011968\\
505	0.00394924445222225\\
506	0.00391378241291492\\
507	0.00387754851258355\\
508	0.00384034278626771\\
509	0.00380159692797307\\
510	0.00376171459281732\\
511	0.00372140524121531\\
512	0.00368042783019527\\
513	0.00363816404977489\\
514	0.00359328710604181\\
515	0.00354725849525978\\
516	0.00350104439854883\\
517	0.0034546796590398\\
518	0.0034082466826054\\
519	0.00336194673080687\\
520	0.00331627115305347\\
521	0.00327235601772224\\
522	0.00323245081899571\\
523	0.00319880639447882\\
524	0.00316409774624987\\
525	0.00312711704512529\\
526	0.00308409729865228\\
527	0.00302322049211327\\
528	0.00295799210260085\\
529	0.00289023238240467\\
530	0.00281957843518099\\
531	0.0027452515402818\\
532	0.00266566068754643\\
533	0.0025771460147355\\
534	0.00248201373084442\\
535	0.00238431533313198\\
536	0.00228393946136431\\
537	0.00218076254050936\\
538	0.00207463993340179\\
539	0.00196537226973244\\
540	0.00185259646353105\\
541	0.00173543479749105\\
542	0.00161171504284901\\
543	0.00148532245630597\\
544	0.0013561727032017\\
545	0.00122418802429166\\
546	0.00108928435702427\\
547	0.00095136984335789\\
548	0.000810335032611802\\
549	0.000666015326258075\\
550	0.00051806260568251\\
551	0.000365504845979632\\
552	0.000205227792137007\\
553	2.68864273520115e-05\\
554	0\\
555	0\\
556	0\\
557	0\\
558	0\\
559	0\\
560	0\\
561	0\\
562	0\\
563	0\\
564	0\\
565	0\\
566	0\\
567	0\\
568	0\\
569	0\\
570	0\\
571	0\\
572	0\\
573	0\\
574	0\\
575	0\\
576	0\\
577	0\\
578	0\\
579	0\\
580	0\\
581	0\\
582	0\\
583	0\\
584	0\\
585	0\\
586	0\\
587	0\\
588	0\\
589	0\\
590	0\\
591	0\\
592	0\\
593	0\\
594	0\\
595	0\\
596	0\\
597	0\\
598	0\\
599	0\\
600	0\\
};
\addplot [color=mycolor16,solid,forget plot]
  table[row sep=crcr]{%
1	0.00344286792547887\\
2	0.0034428627298262\\
3	0.00344285744013911\\
4	0.00344285205471872\\
5	0.00344284657183551\\
6	0.00344284098972885\\
7	0.00344283530660638\\
8	0.00344282952064345\\
9	0.00344282362998257\\
10	0.00344281763273278\\
11	0.00344281152696909\\
12	0.00344280531073186\\
13	0.00344279898202617\\
14	0.00344279253882116\\
15	0.00344278597904946\\
16	0.00344277930060647\\
17	0.00344277250134973\\
18	0.00344276557909821\\
19	0.00344275853163165\\
20	0.00344275135668987\\
21	0.00344274405197199\\
22	0.00344273661513577\\
23	0.00344272904379685\\
24	0.00344272133552797\\
25	0.00344271348785824\\
26	0.00344270549827233\\
27	0.00344269736420971\\
28	0.00344268908306379\\
29	0.00344268065218116\\
30	0.00344267206886071\\
31	0.00344266333035277\\
32	0.00344265443385828\\
33	0.00344264537652789\\
34	0.00344263615546104\\
35	0.00344262676770509\\
36	0.00344261721025434\\
37	0.00344260748004913\\
38	0.00344259757397484\\
39	0.00344258748886092\\
40	0.0034425772214799\\
41	0.00344256676854639\\
42	0.00344255612671598\\
43	0.00344254529258426\\
44	0.00344253426268573\\
45	0.00344252303349268\\
46	0.0034425116014141\\
47	0.00344249996279459\\
48	0.00344248811391314\\
49	0.00344247605098199\\
50	0.00344246377014548\\
51	0.00344245126747877\\
52	0.00344243853898668\\
53	0.00344242558060235\\
54	0.00344241238818607\\
55	0.00344239895752387\\
56	0.00344238528432629\\
57	0.00344237136422697\\
58	0.00344235719278133\\
59	0.00344234276546515\\
60	0.00344232807767315\\
61	0.00344231312471755\\
62	0.00344229790182662\\
63	0.00344228240414316\\
64	0.00344226662672301\\
65	0.00344225056453347\\
66	0.00344223421245173\\
67	0.00344221756526331\\
68	0.00344220061766035\\
69	0.00344218336424005\\
70	0.00344216579950289\\
71	0.00344214791785095\\
72	0.00344212971358619\\
73	0.00344211118090861\\
74	0.00344209231391449\\
75	0.00344207310659451\\
76	0.00344205355283191\\
77	0.00344203364640054\\
78	0.00344201338096296\\
79	0.00344199275006843\\
80	0.00344197174715092\\
81	0.00344195036552705\\
82	0.00344192859839402\\
83	0.00344190643882747\\
84	0.00344188387977934\\
85	0.00344186091407569\\
86	0.00344183753441441\\
87	0.00344181373336301\\
88	0.00344178950335629\\
89	0.00344176483669394\\
90	0.00344173972553823\\
91	0.00344171416191148\\
92	0.00344168813769367\\
93	0.00344166164461982\\
94	0.00344163467427752\\
95	0.00344160721810423\\
96	0.0034415792673847\\
97	0.0034415508132482\\
98	0.00344152184666578\\
99	0.00344149235844749\\
100	0.00344146233923953\\
101	0.00344143177952131\\
102	0.00344140066960253\\
103	0.00344136899962017\\
104	0.00344133675953543\\
105	0.00344130393913062\\
106	0.00344127052800598\\
107	0.00344123651557647\\
108	0.00344120189106851\\
109	0.00344116664351662\\
110	0.00344113076176001\\
111	0.00344109423443917\\
112	0.00344105704999233\\
113	0.00344101919665188\\
114	0.00344098066244072\\
115	0.00344094143516861\\
116	0.00344090150242832\\
117	0.00344086085159187\\
118	0.00344081946980659\\
119	0.00344077734399114\\
120	0.0034407344608315\\
121	0.00344069080677685\\
122	0.00344064636803533\\
123	0.00344060113056991\\
124	0.00344055508009391\\
125	0.00344050820206669\\
126	0.00344046048168914\\
127	0.0034404119038991\\
128	0.00344036245336672\\
129	0.00344031211448975\\
130	0.00344026087138871\\
131	0.003440208707902\\
132	0.00344015560758095\\
133	0.0034401015536847\\
134	0.00344004652917507\\
135	0.00343999051671133\\
136	0.00343993349864483\\
137	0.00343987545701357\\
138	0.0034398163735367\\
139	0.00343975622960885\\
140	0.00343969500629445\\
141	0.00343963268432187\\
142	0.00343956924407749\\
143	0.0034395046655997\\
144	0.00343943892857273\\
145	0.00343937201232041\\
146	0.00343930389579982\\
147	0.00343923455759482\\
148	0.00343916397590946\\
149	0.00343909212856127\\
150	0.0034390189929745\\
151	0.00343894454617309\\
152	0.00343886876477369\\
153	0.00343879162497845\\
154	0.00343871310256769\\
155	0.00343863317289249\\
156	0.00343855181086712\\
157	0.0034384689909613\\
158	0.00343838468719239\\
159	0.00343829887311741\\
160	0.00343821152182493\\
161	0.00343812260592678\\
162	0.00343803209754964\\
163	0.00343793996832654\\
164	0.00343784618938807\\
165	0.00343775073135359\\
166	0.00343765356432215\\
167	0.00343755465786337\\
168	0.00343745398100804\\
169	0.00343735150223867\\
170	0.00343724718947977\\
171	0.00343714101008803\\
172	0.0034370329308423\\
173	0.00343692291793336\\
174	0.00343681093695358\\
175	0.00343669695288634\\
176	0.00343658093009526\\
177	0.00343646283231331\\
178	0.00343634262263159\\
179	0.0034362202634881\\
180	0.0034360957166561\\
181	0.00343596894323245\\
182	0.00343583990362558\\
183	0.00343570855754343\\
184	0.00343557486398096\\
185	0.00343543878120763\\
186	0.00343530026675451\\
187	0.00343515927740128\\
188	0.00343501576916291\\
189	0.00343486969727613\\
190	0.00343472101618568\\
191	0.00343456967953029\\
192	0.0034344156401284\\
193	0.00343425884996365\\
194	0.0034340992601701\\
195	0.00343393682101718\\
196	0.00343377148189436\\
197	0.00343360319129557\\
198	0.00343343189680334\\
199	0.0034332575450726\\
200	0.00343308008181429\\
201	0.00343289945177855\\
202	0.00343271559873771\\
203	0.00343252846546893\\
204	0.00343233799373655\\
205	0.00343214412427403\\
206	0.00343194679676573\\
207	0.00343174594982822\\
208	0.00343154152099128\\
209	0.00343133344667862\\
210	0.00343112166218817\\
211	0.00343090610167205\\
212	0.00343068669811622\\
213	0.00343046338331963\\
214	0.00343023608787317\\
215	0.00343000474113806\\
216	0.00342976927122399\\
217	0.00342952960496678\\
218	0.00342928566790563\\
219	0.00342903738426005\\
220	0.00342878467690622\\
221	0.00342852746735306\\
222	0.00342826567571775\\
223	0.00342799922070092\\
224	0.00342772801956126\\
225	0.00342745198808977\\
226	0.00342717104058347\\
227	0.00342688508981867\\
228	0.00342659404702375\\
229	0.0034262978218514\\
230	0.00342599632235041\\
231	0.00342568945493688\\
232	0.00342537712436497\\
233	0.00342505923369709\\
234	0.00342473568427346\\
235	0.0034244063756813\\
236	0.00342407120572325\\
237	0.00342373007038532\\
238	0.0034233828638043\\
239	0.0034230294782344\\
240	0.00342266980401349\\
241	0.00342230372952856\\
242	0.0034219311411806\\
243	0.00342155192334889\\
244	0.00342116595835452\\
245	0.00342077312642337\\
246	0.00342037330564831\\
247	0.00341996637195078\\
248	0.0034195521990416\\
249	0.00341913065838114\\
250	0.00341870161913868\\
251	0.00341826494815112\\
252	0.00341782050988081\\
253	0.00341736816637278\\
254	0.00341690777721102\\
255	0.00341643919947406\\
256	0.00341596228768975\\
257	0.00341547689378915\\
258	0.00341498286705964\\
259	0.00341448005409716\\
260	0.00341396829875758\\
261	0.00341344744210718\\
262	0.00341291732237227\\
263	0.00341237777488788\\
264	0.00341182863204548\\
265	0.00341126972323985\\
266	0.00341070087481491\\
267	0.0034101219100086\\
268	0.00340953264889679\\
269	0.00340893290833617\\
270	0.00340832250190613\\
271	0.00340770123984972\\
272	0.00340706892901375\\
273	0.00340642537278796\\
274	0.00340577037104266\\
275	0.00340510372006404\\
276	0.00340442521248928\\
277	0.00340373463724212\\
278	0.00340303177946615\\
279	0.00340231642045696\\
280	0.00340158833759315\\
281	0.00340084730426624\\
282	0.00340009308980937\\
283	0.00339932545942495\\
284	0.00339854417411105\\
285	0.00339774899058674\\
286	0.00339693966121618\\
287	0.00339611593393218\\
288	0.0033952775521561\\
289	0.0033944242547214\\
290	0.00339355577579138\\
291	0.00339267184477818\\
292	0.00339177218626017\\
293	0.00339085651989832\\
294	0.00338992456035151\\
295	0.00338897601719089\\
296	0.00338801059481318\\
297	0.00338702799235318\\
298	0.00338602790359528\\
299	0.00338501001688428\\
300	0.00338397401503539\\
301	0.0033829195752436\\
302	0.00338184636899246\\
303	0.00338075406196239\\
304	0.00337964231393863\\
305	0.00337851077871895\\
306	0.00337735910402129\\
307	0.00337618693139139\\
308	0.00337499389611056\\
309	0.00337377962710332\\
310	0.00337254374684477\\
311	0.00337128587126779\\
312	0.00337000560967265\\
313	0.00336870256464419\\
314	0.00336737633197372\\
315	0.00336602650056444\\
316	0.00336465265234928\\
317	0.00336325436221154\\
318	0.00336183119790872\\
319	0.00336038272000004\\
320	0.0033589084817784\\
321	0.00335740802920719\\
322	0.00335588090086295\\
323	0.00335432662788449\\
324	0.00335274473392965\\
325	0.00335113473514055\\
326	0.00334949614011866\\
327	0.00334782844991109\\
328	0.0033461311580096\\
329	0.00334440375036411\\
330	0.00334264570541242\\
331	0.00334085649412775\\
332	0.00333903558008377\\
333	0.00333718241953204\\
334	0.00333529646147408\\
335	0.00333337714768695\\
336	0.00333142391265077\\
337	0.00332943618348289\\
338	0.00332741338083925\\
339	0.00332535492385863\\
340	0.00332326024002062\\
341	0.0033211287537288\\
342	0.00331895985646738\\
343	0.00331675292394826\\
344	0.00331450728973434\\
345	0.00331222216048866\\
346	0.00330989635828335\\
347	0.00330752761579761\\
348	0.00330511112844656\\
349	0.00330263941456339\\
350	0.00330011690336177\\
351	0.00329755216031542\\
352	0.00329494449277718\\
353	0.00329229319792663\\
354	0.00328959756271786\\
355	0.00328685686384107\\
356	0.00328407036770518\\
357	0.00328123733043031\\
358	0.00327835699778407\\
359	0.00327542860503736\\
360	0.00327245137706731\\
361	0.00326942452835106\\
362	0.00326634726296647\\
363	0.00326321877459954\\
364	0.00326003824655825\\
365	0.00325680485179212\\
366	0.00325351775291687\\
367	0.00325017610224328\\
368	0.00324677904180891\\
369	0.00324332570341137\\
370	0.00323981520864163\\
371	0.00323624666891518\\
372	0.0032326191854982\\
373	0.00322893184952281\\
374	0.00322518374197804\\
375	0.00322137393365025\\
376	0.00321750148498322\\
377	0.00321356544589632\\
378	0.00320956485581192\\
379	0.00320549874406713\\
380	0.00320136612929491\\
381	0.00319716601939614\\
382	0.00319289741177549\\
383	0.00318855929398756\\
384	0.00318415064427167\\
385	0.00317967042827671\\
386	0.00317511757946406\\
387	0.00317049094444681\\
388	0.00316578925304264\\
389	0.00316101137882746\\
390	0.00315615621444637\\
391	0.00315122262383991\\
392	0.00314620943821516\\
393	0.00314111545160758\\
394	0.00313593941641775\\
395	0.00313068004017766\\
396	0.00312533598732079\\
397	0.00311990589695682\\
398	0.00311438844816519\\
399	0.00310878256139036\\
400	0.00310308797710424\\
401	0.00309730682743635\\
402	0.00309144755358902\\
403	0.00308553294398344\\
404	0.00307960777443576\\
405	0.00307370305247482\\
406	0.00306768880373024\\
407	0.0030615630565163\\
408	0.00305532375840407\\
409	0.0030489687691679\\
410	0.0030424958507213\\
411	0.00303590266293422\\
412	0.00302918678566082\\
413	0.00302234573389074\\
414	0.0030153766995041\\
415	0.00300827678087732\\
416	0.00300104297822203\\
417	0.00299367218865368\\
418	0.00298616120097232\\
419	0.00297850669012333\\
420	0.00297070521120253\\
421	0.00296275319266422\\
422	0.0029546469282637\\
423	0.00294638256839115\\
424	0.0029379561147681\\
425	0.00292936341705382\\
426	0.0029206001594658\\
427	0.00291166185151239\\
428	0.00290254381812317\\
429	0.00289324118913661\\
430	0.00288374888809989\\
431	0.00287406162032386\\
432	0.00286417386009392\\
433	0.00285407983679185\\
434	0.00284377351921915\\
435	0.00283324859613197\\
436	0.00282249844842589\\
437	0.00281151610819758\\
438	0.00280029423099783\\
439	0.00278882526319504\\
440	0.0027771019541476\\
441	0.00276511668582589\\
442	0.00275286060858101\\
443	0.00274032442670874\\
444	0.00272749837285342\\
445	0.00271437218112794\\
446	0.00270093505952688\\
447	0.0026871756644159\\
448	0.00267308208307675\\
449	0.00265864181460575\\
450	0.00264384168309426\\
451	0.00262866779095149\\
452	0.00261310541108611\\
453	0.00259713867596304\\
454	0.00258074962229649\\
455	0.00256391505904901\\
456	0.00254659623316061\\
457	0.0025286955854366\\
458	0.00250987886118515\\
459	0.00248796942335234\\
460	0.00246493431717584\\
461	0.00244142489791835\\
462	0.00241733485852565\\
463	0.00239232361604092\\
464	0.00236569411982794\\
465	0.00233856268707768\\
466	0.00231095686283289\\
467	0.00228286699853765\\
468	0.0022542819120225\\
469	0.00222518696093254\\
470	0.00219555975230399\\
471	0.00216537201516941\\
472	0.00213465618073841\\
473	0.00210341748129568\\
474	0.0020716692733659\\
475	0.0020394910917054\\
476	0.00200716271405622\\
477	0.0019754970633491\\
478	0.00194601804071996\\
479	0.00191765058876644\\
480	0.00188845229302002\\
481	0.00185837562380705\\
482	0.0018273670959291\\
483	0.00179536361596924\\
484	0.00176227937297323\\
485	0.00172794340871963\\
486	0.00169145087460265\\
487	0.00165134325332256\\
488	0.00161026073528203\\
489	0.00156816972602933\\
490	0.00152503488596659\\
491	0.00148081997832478\\
492	0.0014354873087811\\
493	0.00138899963751852\\
494	0.00134132222049671\\
495	0.00129240247908461\\
496	0.00124219552545212\\
497	0.00119065577442196\\
498	0.00113773508033689\\
499	0.00108336920262801\\
500	0.00102747393336099\\
501	0.000969990569839575\\
502	0.000910857118539138\\
503	0.000850006359514929\\
504	0.000787355549633204\\
505	0.00072274295263537\\
506	0.000655701571956031\\
507	0.00058665174801009\\
508	0.000515530777339095\\
509	0.000441372331336617\\
510	0.000364445223296124\\
511	0.000285746035416068\\
512	0.000205074854842097\\
513	0.000121623393036926\\
514	3.08717202713595e-05\\
515	0\\
516	0\\
517	0\\
518	0\\
519	0\\
520	0\\
521	0\\
522	0\\
523	0\\
524	0\\
525	0\\
526	0\\
527	0\\
528	0\\
529	0\\
530	0\\
531	0\\
532	0\\
533	0\\
534	0\\
535	0\\
536	0\\
537	0\\
538	0\\
539	0\\
540	0\\
541	0\\
542	0\\
543	0\\
544	0\\
545	0\\
546	0\\
547	0\\
548	0\\
549	0\\
550	0\\
551	0\\
552	0\\
553	0\\
554	0\\
555	0\\
556	0\\
557	0\\
558	0\\
559	0\\
560	0\\
561	0\\
562	0\\
563	0\\
564	0\\
565	0\\
566	0\\
567	0\\
568	0\\
569	0\\
570	0\\
571	0\\
572	0\\
573	0\\
574	0\\
575	0\\
576	0\\
577	0\\
578	0\\
579	0\\
580	0\\
581	0\\
582	0\\
583	0\\
584	0\\
585	0\\
586	0\\
587	0\\
588	0\\
589	0\\
590	0\\
591	0\\
592	0\\
593	0\\
594	0\\
595	0\\
596	0\\
597	0\\
598	0\\
599	0\\
600	0\\
};
\addplot [color=mycolor17,solid,forget plot]
  table[row sep=crcr]{%
1	0.00189995238561724\\
2	0.0018999437136439\\
3	0.00189993488465496\\
4	0.00189992589581167\\
5	0.00189991674422404\\
6	0.00189990742695\\
7	0.0018998979409944\\
8	0.00189988828330809\\
9	0.00189987845078691\\
10	0.00189986844027078\\
11	0.00189985824854262\\
12	0.00189984787232735\\
13	0.00189983730829086\\
14	0.00189982655303896\\
15	0.00189981560311625\\
16	0.00189980445500509\\
17	0.00189979310512444\\
18	0.00189978154982872\\
19	0.00189976978540668\\
20	0.00189975780808019\\
21	0.0018997456140031\\
22	0.00189973319925991\\
23	0.00189972055986466\\
24	0.00189970769175955\\
25	0.00189969459081374\\
26	0.00189968125282197\\
27	0.00189966767350329\\
28	0.00189965384849965\\
29	0.00189963977337455\\
30	0.00189962544361162\\
31	0.00189961085461321\\
32	0.00189959600169889\\
33	0.00189958088010403\\
34	0.00189956548497822\\
35	0.00189954981138381\\
36	0.00189953385429427\\
37	0.00189951760859267\\
38	0.00189950106906999\\
39	0.00189948423042356\\
40	0.0018994670872553\\
41	0.00189944963407005\\
42	0.00189943186527383\\
43	0.00189941377517207\\
44	0.00189939535796783\\
45	0.00189937660775992\\
46	0.00189935751854108\\
47	0.00189933808419607\\
48	0.00189931829849972\\
49	0.001899298155115\\
50	0.00189927764759099\\
51	0.00189925676936085\\
52	0.00189923551373976\\
53	0.00189921387392279\\
54	0.00189919184298278\\
55	0.00189916941386813\\
56	0.0018991465794006\\
57	0.00189912333227302\\
58	0.00189909966504699\\
59	0.00189907557015058\\
60	0.0018990510398759\\
61	0.00189902606637667\\
62	0.00189900064166575\\
63	0.00189897475761267\\
64	0.00189894840594101\\
65	0.00189892157822584\\
66	0.00189889426589102\\
67	0.00189886646020656\\
68	0.00189883815228581\\
69	0.00189880933308273\\
70	0.00189877999338901\\
71	0.00189875012383114\\
72	0.00189871971486753\\
73	0.00189868875678549\\
74	0.00189865723969814\\
75	0.00189862515354133\\
76	0.00189859248807052\\
77	0.00189855923285746\\
78	0.00189852537728702\\
79	0.00189849091055379\\
80	0.00189845582165872\\
81	0.00189842009940564\\
82	0.00189838373239779\\
83	0.00189834670903417\\
84	0.00189830901750601\\
85	0.00189827064579293\\
86	0.00189823158165929\\
87	0.00189819181265027\\
88	0.00189815132608804\\
89	0.00189811010906769\\
90	0.00189806814845327\\
91	0.00189802543087364\\
92	0.00189798194271827\\
93	0.00189793767013299\\
94	0.00189789259901563\\
95	0.00189784671501165\\
96	0.00189780000350958\\
97	0.00189775244963649\\
98	0.00189770403825333\\
99	0.00189765475395014\\
100	0.00189760458104132\\
101	0.00189755350356063\\
102	0.00189750150525623\\
103	0.00189744856958562\\
104	0.00189739467971041\\
105	0.00189733981849109\\
106	0.00189728396848168\\
107	0.00189722711192423\\
108	0.0018971692307433\\
109	0.00189711030654029\\
110	0.00189705032058771\\
111	0.00189698925382329\\
112	0.00189692708684406\\
113	0.00189686379990026\\
114	0.00189679937288919\\
115	0.00189673378534888\\
116	0.00189666701645176\\
117	0.00189659904499809\\
118	0.00189652984940941\\
119	0.00189645940772171\\
120	0.00189638769757864\\
121	0.00189631469622449\\
122	0.0018962403804971\\
123	0.00189616472682061\\
124	0.00189608771119812\\
125	0.00189600930920417\\
126	0.00189592949597712\\
127	0.00189584824621138\\
128	0.00189576553414956\\
129	0.00189568133357432\\
130	0.00189559561780031\\
131	0.00189550835966573\\
132	0.00189541953152391\\
133	0.00189532910523467\\
134	0.00189523705215551\\
135	0.00189514334313268\\
136	0.00189504794849211\\
137	0.00189495083803007\\
138	0.0018948519810038\\
139	0.0018947513461219\\
140	0.00189464890153454\\
141	0.00189454461482352\\
142	0.00189443845299216\\
143	0.00189433038245497\\
144	0.00189422036902718\\
145	0.00189410837791406\\
146	0.00189399437370003\\
147	0.00189387832033764\\
148	0.00189376018113623\\
149	0.00189363991875057\\
150	0.00189351749516907\\
151	0.00189339287170204\\
152	0.00189326600896943\\
153	0.00189313686688867\\
154	0.00189300540466206\\
155	0.001892871580764\\
156	0.00189273535292805\\
157	0.00189259667813369\\
158	0.00189245551259283\\
159	0.00189231181173616\\
160	0.00189216553019915\\
161	0.00189201662180786\\
162	0.00189186503956452\\
163	0.00189171073563275\\
164	0.00189155366132262\\
165	0.00189139376707535\\
166	0.00189123100244783\\
167	0.00189106531609673\\
168	0.00189089665576248\\
169	0.00189072496825281\\
170	0.00189055019942608\\
171	0.00189037229417427\\
172	0.00189019119640573\\
173	0.00189000684902745\\
174	0.00188981919392722\\
175	0.00188962817195532\\
176	0.00188943372290589\\
177	0.00188923578549803\\
178	0.00188903429735649\\
179	0.00188882919499201\\
180	0.00188862041378136\\
181	0.00188840788794691\\
182	0.00188819155053594\\
183	0.00188797133339948\\
184	0.0018877471671708\\
185	0.00188751898124351\\
186	0.00188728670374925\\
187	0.00188705026153492\\
188	0.00188680958013959\\
189	0.00188656458377089\\
190	0.001886315195281\\
191	0.0018860613361422\\
192	0.00188580292642196\\
193	0.00188553988475757\\
194	0.00188527212833025\\
195	0.00188499957283889\\
196	0.00188472213247319\\
197	0.00188443971988633\\
198	0.00188415224616716\\
199	0.00188385962081185\\
200	0.00188356175169499\\
201	0.00188325854504014\\
202	0.00188294990538991\\
203	0.00188263573557535\\
204	0.00188231593668489\\
205	0.00188199040803256\\
206	0.00188165904712573\\
207	0.0018813217496322\\
208	0.00188097840934658\\
209	0.00188062891815617\\
210	0.00188027316600614\\
211	0.00187991104086394\\
212	0.00187954242868321\\
213	0.00187916721336689\\
214	0.00187878527672963\\
215	0.00187839649845949\\
216	0.00187800075607892\\
217	0.00187759792490499\\
218	0.0018771878780088\\
219	0.0018767704861742\\
220	0.00187634561785559\\
221	0.00187591313913506\\
222	0.0018754729136785\\
223	0.00187502480269105\\
224	0.00187456866487158\\
225	0.00187410435636625\\
226	0.00187363173072126\\
227	0.00187315063883462\\
228	0.00187266092890697\\
229	0.00187216244639142\\
230	0.0018716550339425\\
231	0.00187113853136392\\
232	0.00187061277555552\\
233	0.00187007760045899\\
234	0.0018695328370026\\
235	0.00186897831304481\\
236	0.0018684138533168\\
237	0.00186783927936376\\
238	0.00186725440948511\\
239	0.0018666590586735\\
240	0.00186605303855252\\
241	0.00186543615731325\\
242	0.0018648082196495\\
243	0.00186416902669172\\
244	0.00186351837593962\\
245	0.00186285606119344\\
246	0.00186218187248376\\
247	0.00186149559599996\\
248	0.0018607970140172\\
249	0.00186008590482191\\
250	0.00185936204263577\\
251	0.00185862519753818\\
252	0.00185787513538704\\
253	0.00185711161773808\\
254	0.00185633440176234\\
255	0.00185554324016213\\
256	0.00185473788108515\\
257	0.00185391806803691\\
258	0.00185308353979131\\
259	0.00185223403029937\\
260	0.00185136926859611\\
261	0.00185048897870546\\
262	0.00184959287954318\\
263	0.00184868068481791\\
264	0.00184775210292986\\
265	0.00184680683686772\\
266	0.00184584458410315\\
267	0.0018448650364832\\
268	0.00184386788012037\\
269	0.00184285279528038\\
270	0.00184181945626755\\
271	0.00184076753130771\\
272	0.00183969668242903\\
273	0.00183860656534083\\
274	0.0018374968293106\\
275	0.00183636711703653\\
276	0.00183521706451204\\
277	0.00183404630089088\\
278	0.00183285444835665\\
279	0.00183164112198385\\
280	0.00183040592959619\\
281	0.00182914847162169\\
282	0.00182786834094467\\
283	0.00182656512275454\\
284	0.0018252383943913\\
285	0.00182388772518768\\
286	0.00182251267630782\\
287	0.00182111280058288\\
288	0.0018196876423409\\
289	0.00181823673723737\\
290	0.00181675961207759\\
291	0.00181525578463748\\
292	0.00181372476348\\
293	0.00181216604776756\\
294	0.00181057912707035\\
295	0.00180896348117056\\
296	0.00180731857986229\\
297	0.00180564388274715\\
298	0.00180393883902548\\
299	0.00180220288728304\\
300	0.0018004354552731\\
301	0.00179863595969402\\
302	0.00179680380596191\\
303	0.00179493838797864\\
304	0.00179303908789491\\
305	0.00179110527586846\\
306	0.0017891363098173\\
307	0.00178713153516803\\
308	0.00178509028459898\\
309	0.00178301187777807\\
310	0.00178089562109412\\
311	0.00177874080737972\\
312	0.00177654671562439\\
313	0.00177431261068608\\
314	0.00177203774302466\\
315	0.00176972134844518\\
316	0.00176736264775351\\
317	0.00176496084645114\\
318	0.00176251513442429\\
319	0.00176002468562693\\
320	0.00175748865775794\\
321	0.00175490619193196\\
322	0.00175227641234405\\
323	0.00174959842592766\\
324	0.0017468713220059\\
325	0.00174409417193571\\
326	0.00174126602874465\\
327	0.00173838592675978\\
328	0.00173545288122827\\
329	0.00173246588792904\\
330	0.00172942392277472\\
331	0.0017263259414026\\
332	0.00172317087875161\\
333	0.00171995764861706\\
334	0.00171668514315773\\
335	0.00171335223228456\\
336	0.00170995776275204\\
337	0.0017065005565978\\
338	0.00170297940867129\\
339	0.00169939308558086\\
340	0.00169574034130069\\
341	0.00169201997073723\\
342	0.0016882307717239\\
343	0.00168437138920443\\
344	0.00168044040580117\\
345	0.00167643625454292\\
346	0.00167235691241596\\
347	0.00166819882434387\\
348	0.00166395318686893\\
349	0.00165959423041928\\
350	0.00165506096071799\\
351	0.00165037745262354\\
352	0.00164561111846711\\
353	0.00164076042838519\\
354	0.00163582382245267\\
355	0.00163079971010773\\
356	0.00162568646959972\\
357	0.0016204824475013\\
358	0.00161518595826011\\
359	0.00160979528345505\\
360	0.00160430867059299\\
361	0.00159872433309202\\
362	0.00159304044970079\\
363	0.00158725516392286\\
364	0.00158136658344821\\
365	0.00157537277959397\\
366	0.00156927178675731\\
367	0.00156306160188326\\
368	0.00155674018395112\\
369	0.00155030545348344\\
370	0.00154375529208251\\
371	0.00153708754200058\\
372	0.00153030000575121\\
373	0.00152339044577001\\
374	0.00151635658412768\\
375	0.00150919610227706\\
376	0.00150190664075992\\
377	0.00149448579874129\\
378	0.00148693113347432\\
379	0.00147924016103994\\
380	0.00147141035952659\\
381	0.00146343916733997\\
382	0.00145532398482506\\
383	0.00144706217760111\\
384	0.00143865108289257\\
385	0.00143008801900107\\
386	0.00142137028793421\\
387	0.00141249511799778\\
388	0.00140345938882302\\
389	0.00139425929207139\\
390	0.00138489183209175\\
391	0.00137535422876928\\
392	0.00136564368716252\\
393	0.00135575740034698\\
394	0.00134569255308525\\
395	0.00133544632776129\\
396	0.00132501591701025\\
397	0.00131439855660885\\
398	0.00130359161994592\\
399	0.00129259289934463\\
400	0.0012814014516963\\
401	0.0012700201346766\\
402	0.00125846313425795\\
403	0.00124677777455407\\
404	0.00123510367046939\\
405	0.00122379391174075\\
406	0.00121331495775664\\
407	0.00120261856048567\\
408	0.00119170054973482\\
409	0.001180556648175\\
410	0.00116918246115891\\
411	0.00115757344745319\\
412	0.00114572489073096\\
413	0.0011336319868634\\
414	0.00112128991524153\\
415	0.00110869242013178\\
416	0.00109583303037424\\
417	0.00108270504925331\\
418	0.00106930154381129\\
419	0.00105561533360234\\
420	0.00104163897886202\\
421	0.00102736476786103\\
422	0.0010127847024471\\
423	0.000997890479493102\\
424	0.000982673469571751\\
425	0.000967124709500084\\
426	0.000951234904115045\\
427	0.000934994389298189\\
428	0.000918393112486593\\
429	0.000901420611983254\\
430	0.000884065994998693\\
431	0.000866317914362073\\
432	0.000848164543847834\\
433	0.00082959355204378\\
434	0.000810592074483101\\
435	0.000791146682864857\\
436	0.000771243346814465\\
437	0.000750867373037908\\
438	0.000730003280363605\\
439	0.000708634566704641\\
440	0.000686744057229842\\
441	0.00066431680296066\\
442	0.000641337978190259\\
443	0.000617788359155828\\
444	0.000593647776015725\\
445	0.000568895056808599\\
446	0.000543507967089672\\
447	0.000517463145007269\\
448	0.000490736039535908\\
449	0.000463300891635387\\
450	0.000435130742918302\\
451	0.000406197158852362\\
452	0.000376470248027807\\
453	0.000345918755290155\\
454	0.000314510300852118\\
455	0.00028221207140335\\
456	0.000248992681573801\\
457	0.000214839035451101\\
458	0.000179918523768187\\
459	0.000151007207087082\\
460	0.000123087413694654\\
461	9.45158457683247e-05\\
462	6.51988540410677e-05\\
463	3.47459928719172e-05\\
464	1.0433240778689e-06\\
465	0\\
466	0\\
467	0\\
468	0\\
469	0\\
470	0\\
471	0\\
472	0\\
473	0\\
474	0\\
475	0\\
476	0\\
477	0\\
478	0\\
479	0\\
480	0\\
481	0\\
482	0\\
483	0\\
484	0\\
485	0\\
486	0\\
487	0\\
488	0\\
489	0\\
490	0\\
491	0\\
492	0\\
493	0\\
494	0\\
495	0\\
496	0\\
497	0\\
498	0\\
499	0\\
500	0\\
501	0\\
502	0\\
503	0\\
504	0\\
505	0\\
506	0\\
507	0\\
508	0\\
509	0\\
510	0\\
511	0\\
512	0\\
513	0\\
514	0\\
515	0\\
516	0\\
517	0\\
518	0\\
519	0\\
520	0\\
521	0\\
522	0\\
523	0\\
524	0\\
525	0\\
526	0\\
527	0\\
528	0\\
529	0\\
530	0\\
531	0\\
532	0\\
533	0\\
534	0\\
535	0\\
536	0\\
537	0\\
538	0\\
539	0\\
540	0\\
541	0\\
542	0\\
543	0\\
544	0\\
545	0\\
546	0\\
547	0\\
548	0\\
549	0\\
550	0\\
551	0\\
552	0\\
553	0\\
554	0\\
555	0\\
556	0\\
557	0\\
558	0\\
559	0\\
560	0\\
561	0\\
562	0\\
563	0\\
564	0\\
565	0\\
566	0\\
567	0\\
568	0\\
569	0\\
570	0\\
571	0\\
572	0\\
573	0\\
574	0\\
575	0\\
576	0\\
577	0\\
578	0\\
579	0\\
580	0\\
581	0\\
582	0\\
583	0\\
584	0\\
585	0\\
586	0\\
587	0\\
588	0\\
589	0\\
590	0\\
591	0\\
592	0\\
593	0\\
594	0\\
595	0\\
596	0\\
597	0\\
598	0\\
599	0\\
600	0\\
};
\addplot [color=mycolor18,solid,forget plot]
  table[row sep=crcr]{%
1	0.000185619121464617\\
2	0.000185612706249818\\
3	0.000185606174741565\\
4	0.00018559952483321\\
5	0.000185592754379972\\
6	0.000185585861198282\\
7	0.000185578843065035\\
8	0.000185571697716924\\
9	0.000185564422849684\\
10	0.000185557016117363\\
11	0.000185549475131572\\
12	0.000185541797460714\\
13	0.00018553398062921\\
14	0.000185526022116701\\
15	0.000185517919357245\\
16	0.00018550966973849\\
17	0.000185501270600826\\
18	0.000185492719236572\\
19	0.000185484012889063\\
20	0.000185475148751796\\
21	0.000185466123967508\\
22	0.0001854569356273\\
23	0.000185447580769662\\
24	0.000185438056379553\\
25	0.000185428359387416\\
26	0.000185418486668229\\
27	0.000185408435040469\\
28	0.000185398201265113\\
29	0.000185387782044588\\
30	0.00018537717402175\\
31	0.000185366373778767\\
32	0.000185355377836053\\
33	0.00018534418265115\\
34	0.000185332784617606\\
35	0.000185321180063799\\
36	0.000185309365251772\\
37	0.000185297336376068\\
38	0.000185285089562478\\
39	0.000185272620866811\\
40	0.000185259926273652\\
41	0.000185247001695066\\
42	0.000185233842969297\\
43	0.000185220445859436\\
44	0.000185206806052073\\
45	0.000185192919155937\\
46	0.000185178780700454\\
47	0.000185164386134368\\
48	0.000185149730824251\\
49	0.00018513481005306\\
50	0.000185119619018611\\
51	0.000185104152832056\\
52	0.000185088406516322\\
53	0.000185072375004529\\
54	0.000185056053138373\\
55	0.000185039435666491\\
56	0.000185022517242776\\
57	0.000185005292424688\\
58	0.000184987755671518\\
59	0.000184969901342615\\
60	0.000184951723695606\\
61	0.00018493321688456\\
62	0.000184914374958128\\
63	0.000184895191857661\\
64	0.000184875661415266\\
65	0.000184855777351865\\
66	0.000184835533275179\\
67	0.000184814922677716\\
68	0.000184793938934687\\
69	0.000184772575301912\\
70	0.000184750824913678\\
71	0.000184728680780546\\
72	0.000184706135787143\\
73	0.000184683182689907\\
74	0.000184659814114774\\
75	0.000184636022554853\\
76	0.000184611800368026\\
77	0.000184587139774541\\
78	0.00018456203285453\\
79	0.000184536471545518\\
80	0.000184510447639833\\
81	0.000184483952782041\\
82	0.000184456978466267\\
83	0.000184429516033519\\
84	0.000184401556668935\\
85	0.000184373091398991\\
86	0.00018434411108867\\
87	0.000184314606438536\\
88	0.000184284567981833\\
89	0.000184253986081458\\
90	0.000184222850926915\\
91	0.000184191152531218\\
92	0.000184158880727728\\
93	0.00018412602516692\\
94	0.000184092575313138\\
95	0.000184058520441248\\
96	0.000184023849633228\\
97	0.000183988551774742\\
98	0.000183952615551622\\
99	0.000183916029446278\\
100	0.000183878781734071\\
101	0.000183840860479602\\
102	0.00018380225353293\\
103	0.000183762948525753\\
104	0.000183722932867498\\
105	0.000183682193741331\\
106	0.000183640718100111\\
107	0.000183598492662304\\
108	0.000183555503907729\\
109	0.000183511738073346\\
110	0.000183467181148893\\
111	0.000183421818872449\\
112	0.000183375636725948\\
113	0.000183328619930616\\
114	0.000183280753442277\\
115	0.000183232021946637\\
116	0.000183182409854425\\
117	0.000183131901296501\\
118	0.000183080480118849\\
119	0.000183028129877477\\
120	0.000182974833833243\\
121	0.000182920574946577\\
122	0.000182865335872101\\
123	0.000182809098953183\\
124	0.000182751846216364\\
125	0.000182693559365692\\
126	0.000182634219776969\\
127	0.000182573808491879\\
128	0.000182512306212025\\
129	0.000182449693292846\\
130	0.000182385949737433\\
131	0.000182321055190257\\
132	0.000182254988930714\\
133	0.000182187729866642\\
134	0.000182119256527671\\
135	0.000182049547058475\\
136	0.000181978579211895\\
137	0.000181906330341912\\
138	0.000181832777396575\\
139	0.000181757896910689\\
140	0.000181681664998503\\
141	0.000181604057346123\\
142	0.000181525049203929\\
143	0.000181444615378757\\
144	0.000181362730225995\\
145	0.000181279367641508\\
146	0.00018119450105343\\
147	0.00018110810341383\\
148	0.000181020147190189\\
149	0.000180930604356744\\
150	0.000180839446385702\\
151	0.000180746644238246\\
152	0.000180652168355429\\
153	0.000180555988648863\\
154	0.000180458074491286\\
155	0.000180358394706937\\
156	0.000180256917561722\\
157	0.000180153610753286\\
158	0.000180048441400838\\
159	0.000179941376034829\\
160	0.000179832380586422\\
161	0.000179721420376791\\
162	0.00017960846010624\\
163	0.000179493463843082\\
164	0.000179376395012368\\
165	0.000179257216384383\\
166	0.000179135890062951\\
167	0.000179012377473527\\
168	0.000178886639351057\\
169	0.000178758635727662\\
170	0.00017862832592005\\
171	0.00017849566851675\\
172	0.000178360621365075\\
173	0.00017822314155788\\
174	0.000178083185420068\\
175	0.000177940708494852\\
176	0.000177795665529789\\
177	0.000177648010462527\\
178	0.000177497696406333\\
179	0.000177344675635338\\
180	0.000177188899569507\\
181	0.000177030318759379\\
182	0.000176868882870475\\
183	0.00017670454066747\\
184	0.000176537239998063\\
185	0.000176366927776539\\
186	0.000176193549967079\\
187	0.000176017051566696\\
188	0.000175837376587934\\
189	0.000175654468041207\\
190	0.000175468267916847\\
191	0.000175278717166785\\
192	0.000175085755685953\\
193	0.000174889322293305\\
194	0.000174689354712498\\
195	0.00017448578955225\\
196	0.000174278562286291\\
197	0.000174067607233022\\
198	0.00017385285753469\\
199	0.000173634245136292\\
200	0.000173411700764045\\
201	0.000173185153903429\\
202	0.000172954532776905\\
203	0.000172719764321141\\
204	0.000172480774163874\\
205	0.000172237486600319\\
206	0.000171989824569154\\
207	0.000171737709628052\\
208	0.000171481061928769\\
209	0.000171219800191751\\
210	0.000170953841680305\\
211	0.000170683102174234\\
212	0.000170407495942966\\
213	0.000170126935718529\\
214	0.00016984133266724\\
215	0.000169550596361666\\
216	0.000169254634751657\\
217	0.000168953354134901\\
218	0.000168646659126945\\
219	0.000168334452630624\\
220	0.00016801663580494\\
221	0.000167693108033341\\
222	0.000167363766891356\\
223	0.000167028508113683\\
224	0.000166687225560611\\
225	0.000166339811183783\\
226	0.000165986154991326\\
227	0.000165626145012311\\
228	0.000165259667260515\\
229	0.000164886605697482\\
230	0.000164506842194881\\
231	0.000164120256496109\\
232	0.000163726726177173\\
233	0.000163326126606785\\
234	0.00016291833090568\\
235	0.00016250320990514\\
236	0.000162080632104701\\
237	0.000161650463628998\\
238	0.000161212568183803\\
239	0.000160766807011131\\
240	0.000160313038843497\\
241	0.000159851119857239\\
242	0.000159380903624893\\
243	0.000158902241066628\\
244	0.000158414980400708\\
245	0.0001579189670929\\
246	0.000157414043804938\\
247	0.000156900050341865\\
248	0.000156376823598324\\
249	0.0001558441975038\\
250	0.000155302002966622\\
251	0.000154750067816939\\
252	0.000154188216748465\\
253	0.000153616271258996\\
254	0.000153034049589741\\
255	0.000152441366663364\\
256	0.000151838034020765\\
257	0.000151223859756513\\
258	0.000150598648452949\\
259	0.000149962201112916\\
260	0.000149314315091053\\
261	0.000148654784023675\\
262	0.000147983397757145\\
263	0.000147299942275301\\
264	0.000146604199622824\\
265	0.000145895947831796\\
266	0.00014517496084202\\
267	0.000144441008421891\\
268	0.000143693856087161\\
269	0.000142933265017849\\
270	0.000142158991973321\\
271	0.000141370789205363\\
272	0.000140568404369321\\
273	0.000139751580433838\\
274	0.000138920055590997\\
275	0.000138073563169448\\
276	0.000137211831543223\\
277	0.00013633458400787\\
278	0.000135441538663405\\
279	0.000134532408336163\\
280	0.000133606900471857\\
281	0.000132664717025979\\
282	0.000131705554351407\\
283	0.00013072910308324\\
284	0.000129735048020551\\
285	0.000128723068005081\\
286	0.000127692835796773\\
287	0.00012664401794577\\
288	0.000125576274661166\\
289	0.000124489259675891\\
290	0.000123382620107833\\
291	0.000122255996316973\\
292	0.000121109021758344\\
293	0.000119941322830555\\
294	0.000118752518719733\\
295	0.000117542221238646\\
296	0.000116310034660682\\
297	0.000115055555548544\\
298	0.00011377837257723\\
299	0.000112478066351131\\
300	0.000111154209214821\\
301	0.000109806365057201\\
302	0.000108434089108701\\
303	0.000107036927731037\\
304	0.000105614418199145\\
305	0.000104166088474863\\
306	0.000102691456971949\\
307	0.00010119003231192\\
308	9.96613130704809e-05\\
309	9.81047875139772e-05\\
310	9.65199333248541e-05\\
311	9.49062173128027e-05\\
312	9.32630951025622e-05\\
313	9.15900107841814e-05\\
314	8.98863965375841e-05\\
315	8.81516723557525e-05\\
316	8.63852459281013e-05\\
317	8.4586512040245e-05\\
318	8.27548521819326e-05\\
319	8.08896341350972e-05\\
320	7.8990211540252e-05\\
321	7.70559234394652e-05\\
322	7.50860937938349e-05\\
323	7.30800309730493e-05\\
324	7.10370272144234e-05\\
325	6.89563580484116e-05\\
326	6.68372816871838e-05\\
327	6.46790383722825e-05\\
328	6.24808496769509e-05\\
329	6.02419177578719e-05\\
330	5.79614245501665e-05\\
331	5.56385308980601e-05\\
332	5.32723756105729e-05\\
333	5.0862074424749e-05\\
334	4.84067188397502e-05\\
335	4.59053747312909e-05\\
336	4.33570805044987e-05\\
337	4.0760844132857e-05\\
338	3.81156374116781e-05\\
339	3.54203838004354e-05\\
340	3.26739372049016e-05\\
341	2.98751043135728e-05\\
342	2.70229872211389e-05\\
343	2.41169932287785e-05\\
344	2.11558559878758e-05\\
345	1.81381856318131e-05\\
346	1.50622635655824e-05\\
347	1.19252419144098e-05\\
348	8.71976743864639e-06\\
349	5.41806378954149e-06\\
350	1.88170861355089e-06\\
351	0\\
352	0\\
353	0\\
354	0\\
355	0\\
356	0\\
357	0\\
358	0\\
359	0\\
360	0\\
361	0\\
362	0\\
363	0\\
364	0\\
365	0\\
366	0\\
367	0\\
368	0\\
369	0\\
370	0\\
371	0\\
372	0\\
373	0\\
374	0\\
375	0\\
376	0\\
377	0\\
378	0\\
379	0\\
380	0\\
381	0\\
382	0\\
383	0\\
384	0\\
385	0\\
386	0\\
387	0\\
388	0\\
389	0\\
390	0\\
391	0\\
392	0\\
393	0\\
394	0\\
395	0\\
396	0\\
397	0\\
398	0\\
399	0\\
400	0\\
401	0\\
402	0\\
403	0\\
404	0\\
405	0\\
406	0\\
407	0\\
408	0\\
409	0\\
410	0\\
411	0\\
412	0\\
413	0\\
414	0\\
415	0\\
416	0\\
417	0\\
418	0\\
419	0\\
420	0\\
421	0\\
422	0\\
423	0\\
424	0\\
425	0\\
426	0\\
427	0\\
428	0\\
429	0\\
430	0\\
431	0\\
432	0\\
433	0\\
434	0\\
435	0\\
436	0\\
437	0\\
438	0\\
439	0\\
440	0\\
441	0\\
442	0\\
443	0\\
444	0\\
445	0\\
446	0\\
447	0\\
448	0\\
449	0\\
450	0\\
451	0\\
452	0\\
453	0\\
454	0\\
455	0\\
456	0\\
457	0\\
458	0\\
459	0\\
460	0\\
461	0\\
462	0\\
463	0\\
464	0\\
465	0\\
466	0\\
467	0\\
468	0\\
469	0\\
470	0\\
471	0\\
472	0\\
473	0\\
474	0\\
475	0\\
476	0\\
477	0\\
478	0\\
479	0\\
480	0\\
481	0\\
482	0\\
483	0\\
484	0\\
485	0\\
486	0\\
487	0\\
488	0\\
489	0\\
490	0\\
491	0\\
492	0\\
493	0\\
494	0\\
495	0\\
496	0\\
497	0\\
498	0\\
499	0\\
500	0\\
501	0\\
502	0\\
503	0\\
504	0\\
505	0\\
506	0\\
507	0\\
508	0\\
509	0\\
510	0\\
511	0\\
512	0\\
513	0\\
514	0\\
515	0\\
516	0\\
517	0\\
518	0\\
519	0\\
520	0\\
521	0\\
522	0\\
523	0\\
524	0\\
525	0\\
526	0\\
527	0\\
528	0\\
529	0\\
530	0\\
531	0\\
532	0\\
533	0\\
534	0\\
535	0\\
536	0\\
537	0\\
538	0\\
539	0\\
540	0\\
541	0\\
542	0\\
543	0\\
544	0\\
545	0\\
546	0\\
547	0\\
548	0\\
549	0\\
550	0\\
551	0\\
552	0\\
553	0\\
554	0\\
555	0\\
556	0\\
557	0\\
558	0\\
559	0\\
560	0\\
561	0\\
562	0\\
563	0\\
564	0\\
565	0\\
566	0\\
567	0\\
568	0\\
569	0\\
570	0\\
571	0\\
572	0\\
573	0\\
574	0\\
575	0\\
576	0\\
577	0\\
578	0\\
579	0\\
580	0\\
581	0\\
582	0\\
583	0\\
584	0\\
585	0\\
586	0\\
587	0\\
588	0\\
589	0\\
590	0\\
591	0\\
592	0\\
593	0\\
594	0\\
595	0\\
596	0\\
597	0\\
598	0\\
599	0\\
600	0\\
};
\addplot [color=red!25!mycolor17,solid,forget plot]
  table[row sep=crcr]{%
1	0\\
2	0\\
3	0\\
4	0\\
5	0\\
6	0\\
7	0\\
8	0\\
9	0\\
10	0\\
11	0\\
12	0\\
13	0\\
14	0\\
15	0\\
16	0\\
17	0\\
18	0\\
19	0\\
20	0\\
21	0\\
22	0\\
23	0\\
24	0\\
25	0\\
26	0\\
27	0\\
28	0\\
29	0\\
30	0\\
31	0\\
32	0\\
33	0\\
34	0\\
35	0\\
36	0\\
37	0\\
38	0\\
39	0\\
40	0\\
41	0\\
42	0\\
43	0\\
44	0\\
45	0\\
46	0\\
47	0\\
48	0\\
49	0\\
50	0\\
51	0\\
52	0\\
53	0\\
54	0\\
55	0\\
56	0\\
57	0\\
58	0\\
59	0\\
60	0\\
61	0\\
62	0\\
63	0\\
64	0\\
65	0\\
66	0\\
67	0\\
68	0\\
69	0\\
70	0\\
71	0\\
72	0\\
73	0\\
74	0\\
75	0\\
76	0\\
77	0\\
78	0\\
79	0\\
80	0\\
81	0\\
82	0\\
83	0\\
84	0\\
85	0\\
86	0\\
87	0\\
88	0\\
89	0\\
90	0\\
91	0\\
92	0\\
93	0\\
94	0\\
95	0\\
96	0\\
97	0\\
98	0\\
99	0\\
100	0\\
101	0\\
102	0\\
103	0\\
104	0\\
105	0\\
106	0\\
107	0\\
108	0\\
109	0\\
110	0\\
111	0\\
112	0\\
113	0\\
114	0\\
115	0\\
116	0\\
117	0\\
118	0\\
119	0\\
120	0\\
121	0\\
122	0\\
123	0\\
124	0\\
125	0\\
126	0\\
127	0\\
128	0\\
129	0\\
130	0\\
131	0\\
132	0\\
133	0\\
134	0\\
135	0\\
136	0\\
137	0\\
138	0\\
139	0\\
140	0\\
141	0\\
142	0\\
143	0\\
144	0\\
145	0\\
146	0\\
147	0\\
148	0\\
149	0\\
150	0\\
151	0\\
152	0\\
153	0\\
154	0\\
155	0\\
156	0\\
157	0\\
158	0\\
159	0\\
160	0\\
161	0\\
162	0\\
163	0\\
164	0\\
165	0\\
166	0\\
167	0\\
168	0\\
169	0\\
170	0\\
171	0\\
172	0\\
173	0\\
174	0\\
175	0\\
176	0\\
177	0\\
178	0\\
179	0\\
180	0\\
181	0\\
182	0\\
183	0\\
184	0\\
185	0\\
186	0\\
187	0\\
188	0\\
189	0\\
190	0\\
191	0\\
192	0\\
193	0\\
194	0\\
195	0\\
196	0\\
197	0\\
198	0\\
199	0\\
200	0\\
201	0\\
202	0\\
203	0\\
204	0\\
205	0\\
206	0\\
207	0\\
208	0\\
209	0\\
210	0\\
211	0\\
212	0\\
213	0\\
214	0\\
215	0\\
216	0\\
217	0\\
218	0\\
219	0\\
220	0\\
221	0\\
222	0\\
223	0\\
224	0\\
225	0\\
226	0\\
227	0\\
228	0\\
229	0\\
230	0\\
231	0\\
232	0\\
233	0\\
234	0\\
235	0\\
236	0\\
237	0\\
238	0\\
239	0\\
240	0\\
241	0\\
242	0\\
243	0\\
244	0\\
245	0\\
246	0\\
247	0\\
248	0\\
249	0\\
250	0\\
251	0\\
252	0\\
253	0\\
254	0\\
255	0\\
256	0\\
257	0\\
258	0\\
259	0\\
260	0\\
261	0\\
262	0\\
263	0\\
264	0\\
265	0\\
266	0\\
267	0\\
268	0\\
269	0\\
270	0\\
271	0\\
272	0\\
273	0\\
274	0\\
275	0\\
276	0\\
277	0\\
278	0\\
279	0\\
280	0\\
281	0\\
282	0\\
283	0\\
284	0\\
285	0\\
286	0\\
287	0\\
288	0\\
289	0\\
290	0\\
291	0\\
292	0\\
293	0\\
294	0\\
295	0\\
296	0\\
297	0\\
298	0\\
299	0\\
300	0\\
301	0\\
302	0\\
303	0\\
304	0\\
305	0\\
306	0\\
307	0\\
308	0\\
309	0\\
310	0\\
311	0\\
312	0\\
313	0\\
314	0\\
315	0\\
316	0\\
317	0\\
318	0\\
319	0\\
320	0\\
321	0\\
322	0\\
323	0\\
324	0\\
325	0\\
326	0\\
327	0\\
328	0\\
329	0\\
330	0\\
331	0\\
332	0\\
333	0\\
334	0\\
335	0\\
336	0\\
337	0\\
338	0\\
339	0\\
340	0\\
341	0\\
342	0\\
343	0\\
344	0\\
345	0\\
346	0\\
347	0\\
348	0\\
349	0\\
350	0\\
351	0\\
352	0\\
353	0\\
354	0\\
355	0\\
356	0\\
357	0\\
358	0\\
359	0\\
360	0\\
361	0\\
362	0\\
363	0\\
364	0\\
365	0\\
366	0\\
367	0\\
368	0\\
369	0\\
370	0\\
371	0\\
372	0\\
373	0\\
374	0\\
375	0\\
376	0\\
377	0\\
378	0\\
379	0\\
380	0\\
381	0\\
382	0\\
383	0\\
384	0\\
385	0\\
386	0\\
387	0\\
388	0\\
389	0\\
390	0\\
391	0\\
392	0\\
393	0\\
394	0\\
395	0\\
396	0\\
397	0\\
398	0\\
399	0\\
400	0\\
401	0\\
402	0\\
403	0\\
404	0\\
405	0\\
406	0\\
407	0\\
408	0\\
409	0\\
410	0\\
411	0\\
412	0\\
413	0\\
414	0\\
415	0\\
416	0\\
417	0\\
418	0\\
419	0\\
420	0\\
421	0\\
422	0\\
423	0\\
424	0\\
425	0\\
426	0\\
427	0\\
428	0\\
429	0\\
430	0\\
431	0\\
432	0\\
433	0\\
434	0\\
435	0\\
436	0\\
437	0\\
438	0\\
439	0\\
440	0\\
441	0\\
442	0\\
443	0\\
444	0\\
445	0\\
446	0\\
447	0\\
448	0\\
449	0\\
450	0\\
451	0\\
452	0\\
453	0\\
454	0\\
455	0\\
456	0\\
457	0\\
458	0\\
459	0\\
460	0\\
461	0\\
462	0\\
463	0\\
464	0\\
465	0\\
466	0\\
467	0\\
468	0\\
469	0\\
470	0\\
471	0\\
472	0\\
473	0\\
474	0\\
475	0\\
476	0\\
477	0\\
478	0\\
479	0\\
480	0\\
481	0\\
482	0\\
483	0\\
484	0\\
485	0\\
486	0\\
487	0\\
488	0\\
489	0\\
490	0\\
491	0\\
492	0\\
493	0\\
494	0\\
495	0\\
496	0\\
497	0\\
498	0\\
499	0\\
500	0\\
501	0\\
502	0\\
503	0\\
504	0\\
505	0\\
506	0\\
507	0\\
508	0\\
509	0\\
510	0\\
511	0\\
512	0\\
513	0\\
514	0\\
515	0\\
516	0\\
517	0\\
518	0\\
519	0\\
520	0\\
521	0\\
522	0\\
523	0\\
524	0\\
525	0\\
526	0\\
527	0\\
528	0\\
529	0\\
530	0\\
531	0\\
532	0\\
533	0\\
534	0\\
535	0\\
536	0\\
537	0\\
538	0\\
539	0\\
540	0\\
541	0\\
542	0\\
543	0\\
544	0\\
545	0\\
546	0\\
547	0\\
548	0\\
549	0\\
550	0\\
551	0\\
552	0\\
553	0\\
554	0\\
555	0\\
556	0\\
557	0\\
558	0\\
559	0\\
560	0\\
561	0\\
562	0\\
563	0\\
564	0\\
565	0\\
566	0\\
567	0\\
568	0\\
569	0\\
570	0\\
571	0\\
572	0\\
573	0\\
574	0\\
575	0\\
576	0\\
577	0\\
578	0\\
579	0\\
580	0\\
581	0\\
582	0\\
583	0\\
584	0\\
585	0\\
586	0\\
587	0\\
588	0\\
589	0\\
590	0\\
591	0\\
592	0\\
593	0\\
594	0\\
595	0\\
596	0\\
597	0\\
598	0\\
599	0\\
600	0\\
};
\addplot [color=mycolor19,solid,forget plot]
  table[row sep=crcr]{%
1	0\\
2	0\\
3	0\\
4	0\\
5	0\\
6	0\\
7	0\\
8	0\\
9	0\\
10	0\\
11	0\\
12	0\\
13	0\\
14	0\\
15	0\\
16	0\\
17	0\\
18	0\\
19	0\\
20	0\\
21	0\\
22	0\\
23	0\\
24	0\\
25	0\\
26	0\\
27	0\\
28	0\\
29	0\\
30	0\\
31	0\\
32	0\\
33	0\\
34	0\\
35	0\\
36	0\\
37	0\\
38	0\\
39	0\\
40	0\\
41	0\\
42	0\\
43	0\\
44	0\\
45	0\\
46	0\\
47	0\\
48	0\\
49	0\\
50	0\\
51	0\\
52	0\\
53	0\\
54	0\\
55	0\\
56	0\\
57	0\\
58	0\\
59	0\\
60	0\\
61	0\\
62	0\\
63	0\\
64	0\\
65	0\\
66	0\\
67	0\\
68	0\\
69	0\\
70	0\\
71	0\\
72	0\\
73	0\\
74	0\\
75	0\\
76	0\\
77	0\\
78	0\\
79	0\\
80	0\\
81	0\\
82	0\\
83	0\\
84	0\\
85	0\\
86	0\\
87	0\\
88	0\\
89	0\\
90	0\\
91	0\\
92	0\\
93	0\\
94	0\\
95	0\\
96	0\\
97	0\\
98	0\\
99	0\\
100	0\\
101	0\\
102	0\\
103	0\\
104	0\\
105	0\\
106	0\\
107	0\\
108	0\\
109	0\\
110	0\\
111	0\\
112	0\\
113	0\\
114	0\\
115	0\\
116	0\\
117	0\\
118	0\\
119	0\\
120	0\\
121	0\\
122	0\\
123	0\\
124	0\\
125	0\\
126	0\\
127	0\\
128	0\\
129	0\\
130	0\\
131	0\\
132	0\\
133	0\\
134	0\\
135	0\\
136	0\\
137	0\\
138	0\\
139	0\\
140	0\\
141	0\\
142	0\\
143	0\\
144	0\\
145	0\\
146	0\\
147	0\\
148	0\\
149	0\\
150	0\\
151	0\\
152	0\\
153	0\\
154	0\\
155	0\\
156	0\\
157	0\\
158	0\\
159	0\\
160	0\\
161	0\\
162	0\\
163	0\\
164	0\\
165	0\\
166	0\\
167	0\\
168	0\\
169	0\\
170	0\\
171	0\\
172	0\\
173	0\\
174	0\\
175	0\\
176	0\\
177	0\\
178	0\\
179	0\\
180	0\\
181	0\\
182	0\\
183	0\\
184	0\\
185	0\\
186	0\\
187	0\\
188	0\\
189	0\\
190	0\\
191	0\\
192	0\\
193	0\\
194	0\\
195	0\\
196	0\\
197	0\\
198	0\\
199	0\\
200	0\\
201	0\\
202	0\\
203	0\\
204	0\\
205	0\\
206	0\\
207	0\\
208	0\\
209	0\\
210	0\\
211	0\\
212	0\\
213	0\\
214	0\\
215	0\\
216	0\\
217	0\\
218	0\\
219	0\\
220	0\\
221	0\\
222	0\\
223	0\\
224	0\\
225	0\\
226	0\\
227	0\\
228	0\\
229	0\\
230	0\\
231	0\\
232	0\\
233	0\\
234	0\\
235	0\\
236	0\\
237	0\\
238	0\\
239	0\\
240	0\\
241	0\\
242	0\\
243	0\\
244	0\\
245	0\\
246	0\\
247	0\\
248	0\\
249	0\\
250	0\\
251	0\\
252	0\\
253	0\\
254	0\\
255	0\\
256	0\\
257	0\\
258	0\\
259	0\\
260	0\\
261	0\\
262	0\\
263	0\\
264	0\\
265	0\\
266	0\\
267	0\\
268	0\\
269	0\\
270	0\\
271	0\\
272	0\\
273	0\\
274	0\\
275	0\\
276	0\\
277	0\\
278	0\\
279	0\\
280	0\\
281	0\\
282	0\\
283	0\\
284	0\\
285	0\\
286	0\\
287	0\\
288	0\\
289	0\\
290	0\\
291	0\\
292	0\\
293	0\\
294	0\\
295	0\\
296	0\\
297	0\\
298	0\\
299	0\\
300	0\\
301	0\\
302	0\\
303	0\\
304	0\\
305	0\\
306	0\\
307	0\\
308	0\\
309	0\\
310	0\\
311	0\\
312	0\\
313	0\\
314	0\\
315	0\\
316	0\\
317	0\\
318	0\\
319	0\\
320	0\\
321	0\\
322	0\\
323	0\\
324	0\\
325	0\\
326	0\\
327	0\\
328	0\\
329	0\\
330	0\\
331	0\\
332	0\\
333	0\\
334	0\\
335	0\\
336	0\\
337	0\\
338	0\\
339	0\\
340	0\\
341	0\\
342	0\\
343	0\\
344	0\\
345	0\\
346	0\\
347	0\\
348	0\\
349	0\\
350	0\\
351	0\\
352	0\\
353	0\\
354	0\\
355	0\\
356	0\\
357	0\\
358	0\\
359	0\\
360	0\\
361	0\\
362	0\\
363	0\\
364	0\\
365	0\\
366	0\\
367	0\\
368	0\\
369	0\\
370	0\\
371	0\\
372	0\\
373	0\\
374	0\\
375	0\\
376	0\\
377	0\\
378	0\\
379	0\\
380	0\\
381	0\\
382	0\\
383	0\\
384	0\\
385	0\\
386	0\\
387	0\\
388	0\\
389	0\\
390	0\\
391	0\\
392	0\\
393	0\\
394	0\\
395	0\\
396	0\\
397	0\\
398	0\\
399	0\\
400	0\\
401	0\\
402	0\\
403	0\\
404	0\\
405	0\\
406	0\\
407	0\\
408	0\\
409	0\\
410	0\\
411	0\\
412	0\\
413	0\\
414	0\\
415	0\\
416	0\\
417	0\\
418	0\\
419	0\\
420	0\\
421	0\\
422	0\\
423	0\\
424	0\\
425	0\\
426	0\\
427	0\\
428	0\\
429	0\\
430	0\\
431	0\\
432	0\\
433	0\\
434	0\\
435	0\\
436	0\\
437	0\\
438	0\\
439	0\\
440	0\\
441	0\\
442	0\\
443	0\\
444	0\\
445	0\\
446	0\\
447	0\\
448	0\\
449	0\\
450	0\\
451	0\\
452	0\\
453	0\\
454	0\\
455	0\\
456	0\\
457	0\\
458	0\\
459	0\\
460	0\\
461	0\\
462	0\\
463	0\\
464	0\\
465	0\\
466	0\\
467	0\\
468	0\\
469	0\\
470	0\\
471	0\\
472	0\\
473	0\\
474	0\\
475	0\\
476	0\\
477	0\\
478	0\\
479	0\\
480	0\\
481	0\\
482	0\\
483	0\\
484	0\\
485	0\\
486	0\\
487	0\\
488	0\\
489	0\\
490	0\\
491	0\\
492	0\\
493	0\\
494	0\\
495	0\\
496	0\\
497	0\\
498	0\\
499	0\\
500	0\\
501	0\\
502	0\\
503	0\\
504	0\\
505	0\\
506	0\\
507	0\\
508	0\\
509	0\\
510	0\\
511	0\\
512	0\\
513	0\\
514	0\\
515	0\\
516	0\\
517	0\\
518	0\\
519	0\\
520	0\\
521	0\\
522	0\\
523	0\\
524	0\\
525	0\\
526	0\\
527	0\\
528	0\\
529	0\\
530	0\\
531	0\\
532	0\\
533	0\\
534	0\\
535	0\\
536	0\\
537	0\\
538	0\\
539	0\\
540	0\\
541	0\\
542	0\\
543	0\\
544	0\\
545	0\\
546	0\\
547	0\\
548	0\\
549	0\\
550	0\\
551	0\\
552	0\\
553	0\\
554	0\\
555	0\\
556	0\\
557	0\\
558	0\\
559	0\\
560	0\\
561	0\\
562	0\\
563	0\\
564	0\\
565	0\\
566	0\\
567	0\\
568	0\\
569	0\\
570	0\\
571	0\\
572	0\\
573	0\\
574	0\\
575	0\\
576	0\\
577	0\\
578	0\\
579	0\\
580	0\\
581	0\\
582	0\\
583	0\\
584	0\\
585	0\\
586	0\\
587	0\\
588	0\\
589	0\\
590	0\\
591	0\\
592	0\\
593	0\\
594	0\\
595	0\\
596	0\\
597	0\\
598	0\\
599	0\\
600	0\\
};
\addplot [color=red!50!mycolor17,solid,forget plot]
  table[row sep=crcr]{%
1	0\\
2	0\\
3	0\\
4	0\\
5	0\\
6	0\\
7	0\\
8	0\\
9	0\\
10	0\\
11	0\\
12	0\\
13	0\\
14	0\\
15	0\\
16	0\\
17	0\\
18	0\\
19	0\\
20	0\\
21	0\\
22	0\\
23	0\\
24	0\\
25	0\\
26	0\\
27	0\\
28	0\\
29	0\\
30	0\\
31	0\\
32	0\\
33	0\\
34	0\\
35	0\\
36	0\\
37	0\\
38	0\\
39	0\\
40	0\\
41	0\\
42	0\\
43	0\\
44	0\\
45	0\\
46	0\\
47	0\\
48	0\\
49	0\\
50	0\\
51	0\\
52	0\\
53	0\\
54	0\\
55	0\\
56	0\\
57	0\\
58	0\\
59	0\\
60	0\\
61	0\\
62	0\\
63	0\\
64	0\\
65	0\\
66	0\\
67	0\\
68	0\\
69	0\\
70	0\\
71	0\\
72	0\\
73	0\\
74	0\\
75	0\\
76	0\\
77	0\\
78	0\\
79	0\\
80	0\\
81	0\\
82	0\\
83	0\\
84	0\\
85	0\\
86	0\\
87	0\\
88	0\\
89	0\\
90	0\\
91	0\\
92	0\\
93	0\\
94	0\\
95	0\\
96	0\\
97	0\\
98	0\\
99	0\\
100	0\\
101	0\\
102	0\\
103	0\\
104	0\\
105	0\\
106	0\\
107	0\\
108	0\\
109	0\\
110	0\\
111	0\\
112	0\\
113	0\\
114	0\\
115	0\\
116	0\\
117	0\\
118	0\\
119	0\\
120	0\\
121	0\\
122	0\\
123	0\\
124	0\\
125	0\\
126	0\\
127	0\\
128	0\\
129	0\\
130	0\\
131	0\\
132	0\\
133	0\\
134	0\\
135	0\\
136	0\\
137	0\\
138	0\\
139	0\\
140	0\\
141	0\\
142	0\\
143	0\\
144	0\\
145	0\\
146	0\\
147	0\\
148	0\\
149	0\\
150	0\\
151	0\\
152	0\\
153	0\\
154	0\\
155	0\\
156	0\\
157	0\\
158	0\\
159	0\\
160	0\\
161	0\\
162	0\\
163	0\\
164	0\\
165	0\\
166	0\\
167	0\\
168	0\\
169	0\\
170	0\\
171	0\\
172	0\\
173	0\\
174	0\\
175	0\\
176	0\\
177	0\\
178	0\\
179	0\\
180	0\\
181	0\\
182	0\\
183	0\\
184	0\\
185	0\\
186	0\\
187	0\\
188	0\\
189	0\\
190	0\\
191	0\\
192	0\\
193	0\\
194	0\\
195	0\\
196	0\\
197	0\\
198	0\\
199	0\\
200	0\\
201	0\\
202	0\\
203	0\\
204	0\\
205	0\\
206	0\\
207	0\\
208	0\\
209	0\\
210	0\\
211	0\\
212	0\\
213	0\\
214	0\\
215	0\\
216	0\\
217	0\\
218	0\\
219	0\\
220	0\\
221	0\\
222	0\\
223	0\\
224	0\\
225	0\\
226	0\\
227	0\\
228	0\\
229	0\\
230	0\\
231	0\\
232	0\\
233	0\\
234	0\\
235	0\\
236	0\\
237	0\\
238	0\\
239	0\\
240	0\\
241	0\\
242	0\\
243	0\\
244	0\\
245	0\\
246	0\\
247	0\\
248	0\\
249	0\\
250	0\\
251	0\\
252	0\\
253	0\\
254	0\\
255	0\\
256	0\\
257	0\\
258	0\\
259	0\\
260	0\\
261	0\\
262	0\\
263	0\\
264	0\\
265	0\\
266	0\\
267	0\\
268	0\\
269	0\\
270	0\\
271	0\\
272	0\\
273	0\\
274	0\\
275	0\\
276	0\\
277	0\\
278	0\\
279	0\\
280	0\\
281	0\\
282	0\\
283	0\\
284	0\\
285	0\\
286	0\\
287	0\\
288	0\\
289	0\\
290	0\\
291	0\\
292	0\\
293	0\\
294	0\\
295	0\\
296	0\\
297	0\\
298	0\\
299	0\\
300	0\\
301	0\\
302	0\\
303	0\\
304	0\\
305	0\\
306	0\\
307	0\\
308	0\\
309	0\\
310	0\\
311	0\\
312	0\\
313	0\\
314	0\\
315	0\\
316	0\\
317	0\\
318	0\\
319	0\\
320	0\\
321	0\\
322	0\\
323	0\\
324	0\\
325	0\\
326	0\\
327	0\\
328	0\\
329	0\\
330	0\\
331	0\\
332	0\\
333	0\\
334	0\\
335	0\\
336	0\\
337	0\\
338	0\\
339	0\\
340	0\\
341	0\\
342	0\\
343	0\\
344	0\\
345	0\\
346	0\\
347	0\\
348	0\\
349	0\\
350	0\\
351	0\\
352	0\\
353	0\\
354	0\\
355	0\\
356	0\\
357	0\\
358	0\\
359	0\\
360	0\\
361	0\\
362	0\\
363	0\\
364	0\\
365	0\\
366	0\\
367	0\\
368	0\\
369	0\\
370	0\\
371	0\\
372	0\\
373	0\\
374	0\\
375	0\\
376	0\\
377	0\\
378	0\\
379	0\\
380	0\\
381	0\\
382	0\\
383	0\\
384	0\\
385	0\\
386	0\\
387	0\\
388	0\\
389	0\\
390	0\\
391	0\\
392	0\\
393	0\\
394	0\\
395	0\\
396	0\\
397	0\\
398	0\\
399	0\\
400	0\\
401	0\\
402	0\\
403	0\\
404	0\\
405	0\\
406	0\\
407	0\\
408	0\\
409	0\\
410	0\\
411	0\\
412	0\\
413	0\\
414	0\\
415	0\\
416	0\\
417	0\\
418	0\\
419	0\\
420	0\\
421	0\\
422	0\\
423	0\\
424	0\\
425	0\\
426	0\\
427	0\\
428	0\\
429	0\\
430	0\\
431	0\\
432	0\\
433	0\\
434	0\\
435	0\\
436	0\\
437	0\\
438	0\\
439	0\\
440	0\\
441	0\\
442	0\\
443	0\\
444	0\\
445	0\\
446	0\\
447	0\\
448	0\\
449	0\\
450	0\\
451	0\\
452	0\\
453	0\\
454	0\\
455	0\\
456	0\\
457	0\\
458	0\\
459	0\\
460	0\\
461	0\\
462	0\\
463	0\\
464	0\\
465	0\\
466	0\\
467	0\\
468	0\\
469	0\\
470	0\\
471	0\\
472	0\\
473	0\\
474	0\\
475	0\\
476	0\\
477	0\\
478	0\\
479	0\\
480	0\\
481	0\\
482	0\\
483	0\\
484	0\\
485	0\\
486	0\\
487	0\\
488	0\\
489	0\\
490	0\\
491	0\\
492	0\\
493	0\\
494	0\\
495	0\\
496	0\\
497	0\\
498	0\\
499	0\\
500	0\\
501	0\\
502	0\\
503	0\\
504	0\\
505	0\\
506	0\\
507	0\\
508	0\\
509	0\\
510	0\\
511	0\\
512	0\\
513	0\\
514	0\\
515	0\\
516	0\\
517	0\\
518	0\\
519	0\\
520	0\\
521	0\\
522	0\\
523	0\\
524	0\\
525	0\\
526	0\\
527	0\\
528	0\\
529	0\\
530	0\\
531	0\\
532	0\\
533	0\\
534	0\\
535	0\\
536	0\\
537	0\\
538	0\\
539	0\\
540	0\\
541	0\\
542	0\\
543	0\\
544	0\\
545	0\\
546	0\\
547	0\\
548	0\\
549	0\\
550	0\\
551	0\\
552	0\\
553	0\\
554	0\\
555	0\\
556	0\\
557	0\\
558	0\\
559	0\\
560	0\\
561	0\\
562	0\\
563	0\\
564	0\\
565	0\\
566	0\\
567	0\\
568	0\\
569	0\\
570	0\\
571	0\\
572	0\\
573	0\\
574	0\\
575	0\\
576	0\\
577	0\\
578	0\\
579	0\\
580	0\\
581	0\\
582	0\\
583	0\\
584	0\\
585	0\\
586	0\\
587	0\\
588	0\\
589	0\\
590	0\\
591	0\\
592	0\\
593	0\\
594	0\\
595	0\\
596	0\\
597	0\\
598	0\\
599	0\\
600	0\\
};
\addplot [color=red!40!mycolor19,solid,forget plot]
  table[row sep=crcr]{%
1	0\\
2	0\\
3	0\\
4	0\\
5	0\\
6	0\\
7	0\\
8	0\\
9	0\\
10	0\\
11	0\\
12	0\\
13	0\\
14	0\\
15	0\\
16	0\\
17	0\\
18	0\\
19	0\\
20	0\\
21	0\\
22	0\\
23	0\\
24	0\\
25	0\\
26	0\\
27	0\\
28	0\\
29	0\\
30	0\\
31	0\\
32	0\\
33	0\\
34	0\\
35	0\\
36	0\\
37	0\\
38	0\\
39	0\\
40	0\\
41	0\\
42	0\\
43	0\\
44	0\\
45	0\\
46	0\\
47	0\\
48	0\\
49	0\\
50	0\\
51	0\\
52	0\\
53	0\\
54	0\\
55	0\\
56	0\\
57	0\\
58	0\\
59	0\\
60	0\\
61	0\\
62	0\\
63	0\\
64	0\\
65	0\\
66	0\\
67	0\\
68	0\\
69	0\\
70	0\\
71	0\\
72	0\\
73	0\\
74	0\\
75	0\\
76	0\\
77	0\\
78	0\\
79	0\\
80	0\\
81	0\\
82	0\\
83	0\\
84	0\\
85	0\\
86	0\\
87	0\\
88	0\\
89	0\\
90	0\\
91	0\\
92	0\\
93	0\\
94	0\\
95	0\\
96	0\\
97	0\\
98	0\\
99	0\\
100	0\\
101	0\\
102	0\\
103	0\\
104	0\\
105	0\\
106	0\\
107	0\\
108	0\\
109	0\\
110	0\\
111	0\\
112	0\\
113	0\\
114	0\\
115	0\\
116	0\\
117	0\\
118	0\\
119	0\\
120	0\\
121	0\\
122	0\\
123	0\\
124	0\\
125	0\\
126	0\\
127	0\\
128	0\\
129	0\\
130	0\\
131	0\\
132	0\\
133	0\\
134	0\\
135	0\\
136	0\\
137	0\\
138	0\\
139	0\\
140	0\\
141	0\\
142	0\\
143	0\\
144	0\\
145	0\\
146	0\\
147	0\\
148	0\\
149	0\\
150	0\\
151	0\\
152	0\\
153	0\\
154	0\\
155	0\\
156	0\\
157	0\\
158	0\\
159	0\\
160	0\\
161	0\\
162	0\\
163	0\\
164	0\\
165	0\\
166	0\\
167	0\\
168	0\\
169	0\\
170	0\\
171	0\\
172	0\\
173	0\\
174	0\\
175	0\\
176	0\\
177	0\\
178	0\\
179	0\\
180	0\\
181	0\\
182	0\\
183	0\\
184	0\\
185	0\\
186	0\\
187	0\\
188	0\\
189	0\\
190	0\\
191	0\\
192	0\\
193	0\\
194	0\\
195	0\\
196	0\\
197	0\\
198	0\\
199	0\\
200	0\\
201	0\\
202	0\\
203	0\\
204	0\\
205	0\\
206	0\\
207	0\\
208	0\\
209	0\\
210	0\\
211	0\\
212	0\\
213	0\\
214	0\\
215	0\\
216	0\\
217	0\\
218	0\\
219	0\\
220	0\\
221	0\\
222	0\\
223	0\\
224	0\\
225	0\\
226	0\\
227	0\\
228	0\\
229	0\\
230	0\\
231	0\\
232	0\\
233	0\\
234	0\\
235	0\\
236	0\\
237	0\\
238	0\\
239	0\\
240	0\\
241	0\\
242	0\\
243	0\\
244	0\\
245	0\\
246	0\\
247	0\\
248	0\\
249	0\\
250	0\\
251	0\\
252	0\\
253	0\\
254	0\\
255	0\\
256	0\\
257	0\\
258	0\\
259	0\\
260	0\\
261	0\\
262	0\\
263	0\\
264	0\\
265	0\\
266	0\\
267	0\\
268	0\\
269	0\\
270	0\\
271	0\\
272	0\\
273	0\\
274	0\\
275	0\\
276	0\\
277	0\\
278	0\\
279	0\\
280	0\\
281	0\\
282	0\\
283	0\\
284	0\\
285	0\\
286	0\\
287	0\\
288	0\\
289	0\\
290	0\\
291	0\\
292	0\\
293	0\\
294	0\\
295	0\\
296	0\\
297	0\\
298	0\\
299	0\\
300	0\\
301	0\\
302	0\\
303	0\\
304	0\\
305	0\\
306	0\\
307	0\\
308	0\\
309	0\\
310	0\\
311	0\\
312	0\\
313	0\\
314	0\\
315	0\\
316	0\\
317	0\\
318	0\\
319	0\\
320	0\\
321	0\\
322	0\\
323	0\\
324	0\\
325	0\\
326	0\\
327	0\\
328	0\\
329	0\\
330	0\\
331	0\\
332	0\\
333	0\\
334	0\\
335	0\\
336	0\\
337	0\\
338	0\\
339	0\\
340	0\\
341	0\\
342	0\\
343	0\\
344	0\\
345	0\\
346	0\\
347	0\\
348	0\\
349	0\\
350	0\\
351	0\\
352	0\\
353	0\\
354	0\\
355	0\\
356	0\\
357	0\\
358	0\\
359	0\\
360	0\\
361	0\\
362	0\\
363	0\\
364	0\\
365	0\\
366	0\\
367	0\\
368	0\\
369	0\\
370	0\\
371	0\\
372	0\\
373	0\\
374	0\\
375	0\\
376	0\\
377	0\\
378	0\\
379	0\\
380	0\\
381	0\\
382	0\\
383	0\\
384	0\\
385	0\\
386	0\\
387	0\\
388	0\\
389	0\\
390	0\\
391	0\\
392	0\\
393	0\\
394	0\\
395	0\\
396	0\\
397	0\\
398	0\\
399	0\\
400	0\\
401	0\\
402	0\\
403	0\\
404	0\\
405	0\\
406	0\\
407	0\\
408	0\\
409	0\\
410	0\\
411	0\\
412	0\\
413	0\\
414	0\\
415	0\\
416	0\\
417	0\\
418	0\\
419	0\\
420	0\\
421	0\\
422	0\\
423	0\\
424	0\\
425	0\\
426	0\\
427	0\\
428	0\\
429	0\\
430	0\\
431	0\\
432	0\\
433	0\\
434	0\\
435	0\\
436	0\\
437	0\\
438	0\\
439	0\\
440	0\\
441	0\\
442	0\\
443	0\\
444	0\\
445	0\\
446	0\\
447	0\\
448	0\\
449	0\\
450	0\\
451	0\\
452	0\\
453	0\\
454	0\\
455	0\\
456	0\\
457	0\\
458	0\\
459	0\\
460	0\\
461	0\\
462	0\\
463	0\\
464	0\\
465	0\\
466	0\\
467	0\\
468	0\\
469	0\\
470	0\\
471	0\\
472	0\\
473	0\\
474	0\\
475	0\\
476	0\\
477	0\\
478	0\\
479	0\\
480	0\\
481	0\\
482	0\\
483	0\\
484	0\\
485	0\\
486	0\\
487	0\\
488	0\\
489	0\\
490	0\\
491	0\\
492	0\\
493	0\\
494	0\\
495	0\\
496	0\\
497	0\\
498	0\\
499	0\\
500	0\\
501	0\\
502	0\\
503	0\\
504	0\\
505	0\\
506	0\\
507	0\\
508	0\\
509	0\\
510	0\\
511	0\\
512	0\\
513	0\\
514	0\\
515	0\\
516	0\\
517	0\\
518	0\\
519	0\\
520	0\\
521	0\\
522	0\\
523	0\\
524	0\\
525	0\\
526	0\\
527	0\\
528	0\\
529	0\\
530	0\\
531	0\\
532	0\\
533	0\\
534	0\\
535	0\\
536	0\\
537	0\\
538	0\\
539	0\\
540	0\\
541	0\\
542	0\\
543	0\\
544	0\\
545	0\\
546	0\\
547	0\\
548	0\\
549	0\\
550	0\\
551	0\\
552	0\\
553	0\\
554	0\\
555	0\\
556	0\\
557	0\\
558	0\\
559	0\\
560	0\\
561	0\\
562	0\\
563	0\\
564	0\\
565	0\\
566	0\\
567	0\\
568	0\\
569	0\\
570	0\\
571	0\\
572	0\\
573	0\\
574	0\\
575	0\\
576	0\\
577	0\\
578	0\\
579	0\\
580	0\\
581	0\\
582	0\\
583	0\\
584	0\\
585	0\\
586	0\\
587	0\\
588	0\\
589	0\\
590	0\\
591	0\\
592	0\\
593	0\\
594	0\\
595	0\\
596	0\\
597	0\\
598	0\\
599	0\\
600	0\\
};
\addplot [color=red!75!mycolor17,solid,forget plot]
  table[row sep=crcr]{%
1	0\\
2	0\\
3	0\\
4	0\\
5	0\\
6	0\\
7	0\\
8	0\\
9	0\\
10	0\\
11	0\\
12	0\\
13	0\\
14	0\\
15	0\\
16	0\\
17	0\\
18	0\\
19	0\\
20	0\\
21	0\\
22	0\\
23	0\\
24	0\\
25	0\\
26	0\\
27	0\\
28	0\\
29	0\\
30	0\\
31	0\\
32	0\\
33	0\\
34	0\\
35	0\\
36	0\\
37	0\\
38	0\\
39	0\\
40	0\\
41	0\\
42	0\\
43	0\\
44	0\\
45	0\\
46	0\\
47	0\\
48	0\\
49	0\\
50	0\\
51	0\\
52	0\\
53	0\\
54	0\\
55	0\\
56	0\\
57	0\\
58	0\\
59	0\\
60	0\\
61	0\\
62	0\\
63	0\\
64	0\\
65	0\\
66	0\\
67	0\\
68	0\\
69	0\\
70	0\\
71	0\\
72	0\\
73	0\\
74	0\\
75	0\\
76	0\\
77	0\\
78	0\\
79	0\\
80	0\\
81	0\\
82	0\\
83	0\\
84	0\\
85	0\\
86	0\\
87	0\\
88	0\\
89	0\\
90	0\\
91	0\\
92	0\\
93	0\\
94	0\\
95	0\\
96	0\\
97	0\\
98	0\\
99	0\\
100	0\\
101	0\\
102	0\\
103	0\\
104	0\\
105	0\\
106	0\\
107	0\\
108	0\\
109	0\\
110	0\\
111	0\\
112	0\\
113	0\\
114	0\\
115	0\\
116	0\\
117	0\\
118	0\\
119	0\\
120	0\\
121	0\\
122	0\\
123	0\\
124	0\\
125	0\\
126	0\\
127	0\\
128	0\\
129	0\\
130	0\\
131	0\\
132	0\\
133	0\\
134	0\\
135	0\\
136	0\\
137	0\\
138	0\\
139	0\\
140	0\\
141	0\\
142	0\\
143	0\\
144	0\\
145	0\\
146	0\\
147	0\\
148	0\\
149	0\\
150	0\\
151	0\\
152	0\\
153	0\\
154	0\\
155	0\\
156	0\\
157	0\\
158	0\\
159	0\\
160	0\\
161	0\\
162	0\\
163	0\\
164	0\\
165	0\\
166	0\\
167	0\\
168	0\\
169	0\\
170	0\\
171	0\\
172	0\\
173	0\\
174	0\\
175	0\\
176	0\\
177	0\\
178	0\\
179	0\\
180	0\\
181	0\\
182	0\\
183	0\\
184	0\\
185	0\\
186	0\\
187	0\\
188	0\\
189	0\\
190	0\\
191	0\\
192	0\\
193	0\\
194	0\\
195	0\\
196	0\\
197	0\\
198	0\\
199	0\\
200	0\\
201	0\\
202	0\\
203	0\\
204	0\\
205	0\\
206	0\\
207	0\\
208	0\\
209	0\\
210	0\\
211	0\\
212	0\\
213	0\\
214	0\\
215	0\\
216	0\\
217	0\\
218	0\\
219	0\\
220	0\\
221	0\\
222	0\\
223	0\\
224	0\\
225	0\\
226	0\\
227	0\\
228	0\\
229	0\\
230	0\\
231	0\\
232	0\\
233	0\\
234	0\\
235	0\\
236	0\\
237	0\\
238	0\\
239	0\\
240	0\\
241	0\\
242	0\\
243	0\\
244	0\\
245	0\\
246	0\\
247	0\\
248	0\\
249	0\\
250	0\\
251	0\\
252	0\\
253	0\\
254	0\\
255	0\\
256	0\\
257	0\\
258	0\\
259	0\\
260	0\\
261	0\\
262	0\\
263	0\\
264	0\\
265	0\\
266	0\\
267	0\\
268	0\\
269	0\\
270	0\\
271	0\\
272	0\\
273	0\\
274	0\\
275	0\\
276	0\\
277	0\\
278	0\\
279	0\\
280	0\\
281	0\\
282	0\\
283	0\\
284	0\\
285	0\\
286	0\\
287	0\\
288	0\\
289	0\\
290	0\\
291	0\\
292	0\\
293	0\\
294	0\\
295	0\\
296	0\\
297	0\\
298	0\\
299	0\\
300	0\\
301	0\\
302	0\\
303	0\\
304	0\\
305	0\\
306	0\\
307	0\\
308	0\\
309	0\\
310	0\\
311	0\\
312	0\\
313	0\\
314	0\\
315	0\\
316	0\\
317	0\\
318	0\\
319	0\\
320	0\\
321	0\\
322	0\\
323	0\\
324	0\\
325	0\\
326	0\\
327	0\\
328	0\\
329	0\\
330	0\\
331	0\\
332	0\\
333	0\\
334	0\\
335	0\\
336	0\\
337	0\\
338	0\\
339	0\\
340	0\\
341	0\\
342	0\\
343	0\\
344	0\\
345	0\\
346	0\\
347	0\\
348	0\\
349	0\\
350	0\\
351	0\\
352	0\\
353	0\\
354	0\\
355	0\\
356	0\\
357	0\\
358	0\\
359	0\\
360	0\\
361	0\\
362	0\\
363	0\\
364	0\\
365	0\\
366	0\\
367	0\\
368	0\\
369	0\\
370	0\\
371	0\\
372	0\\
373	0\\
374	0\\
375	0\\
376	0\\
377	0\\
378	0\\
379	0\\
380	0\\
381	0\\
382	0\\
383	0\\
384	0\\
385	0\\
386	0\\
387	0\\
388	0\\
389	0\\
390	0\\
391	0\\
392	0\\
393	0\\
394	0\\
395	0\\
396	0\\
397	0\\
398	0\\
399	0\\
400	0\\
401	0\\
402	0\\
403	0\\
404	0\\
405	0\\
406	0\\
407	0\\
408	0\\
409	0\\
410	0\\
411	0\\
412	0\\
413	0\\
414	0\\
415	0\\
416	0\\
417	0\\
418	0\\
419	0\\
420	0\\
421	0\\
422	0\\
423	0\\
424	0\\
425	0\\
426	0\\
427	0\\
428	0\\
429	0\\
430	0\\
431	0\\
432	0\\
433	0\\
434	0\\
435	0\\
436	0\\
437	0\\
438	0\\
439	0\\
440	0\\
441	0\\
442	0\\
443	0\\
444	0\\
445	0\\
446	0\\
447	0\\
448	0\\
449	0\\
450	0\\
451	0\\
452	0\\
453	0\\
454	0\\
455	0\\
456	0\\
457	0\\
458	0\\
459	0\\
460	0\\
461	0\\
462	0\\
463	0\\
464	0\\
465	0\\
466	0\\
467	0\\
468	0\\
469	0\\
470	0\\
471	0\\
472	0\\
473	0\\
474	0\\
475	0\\
476	0\\
477	0\\
478	0\\
479	0\\
480	0\\
481	0\\
482	0\\
483	0\\
484	0\\
485	0\\
486	0\\
487	0\\
488	0\\
489	0\\
490	0\\
491	0\\
492	0\\
493	0\\
494	0\\
495	0\\
496	0\\
497	0\\
498	0\\
499	0\\
500	0\\
501	0\\
502	0\\
503	0\\
504	0\\
505	0\\
506	0\\
507	0\\
508	0\\
509	0\\
510	0\\
511	0\\
512	0\\
513	0\\
514	0\\
515	0\\
516	0\\
517	0\\
518	0\\
519	0\\
520	0\\
521	0\\
522	0\\
523	0\\
524	0\\
525	0\\
526	0\\
527	0\\
528	0\\
529	0\\
530	0\\
531	0\\
532	0\\
533	0\\
534	0\\
535	0\\
536	0\\
537	0\\
538	0\\
539	0\\
540	0\\
541	0\\
542	0\\
543	0\\
544	0\\
545	0\\
546	0\\
547	0\\
548	0\\
549	0\\
550	0\\
551	0\\
552	0\\
553	0\\
554	0\\
555	0\\
556	0\\
557	0\\
558	0\\
559	0\\
560	0\\
561	0\\
562	0\\
563	0\\
564	0\\
565	0\\
566	0\\
567	0\\
568	0\\
569	0\\
570	0\\
571	0\\
572	0\\
573	0\\
574	0\\
575	0\\
576	0\\
577	0\\
578	0\\
579	0\\
580	0\\
581	0\\
582	0\\
583	0\\
584	0\\
585	0\\
586	0\\
587	0\\
588	0\\
589	0\\
590	0\\
591	0\\
592	0\\
593	0\\
594	0\\
595	0\\
596	0\\
597	0\\
598	0\\
599	0\\
600	0\\
};
\addplot [color=red!80!mycolor19,solid,forget plot]
  table[row sep=crcr]{%
1	0\\
2	0\\
3	0\\
4	0\\
5	0\\
6	0\\
7	0\\
8	0\\
9	0\\
10	0\\
11	0\\
12	0\\
13	0\\
14	0\\
15	0\\
16	0\\
17	0\\
18	0\\
19	0\\
20	0\\
21	0\\
22	0\\
23	0\\
24	0\\
25	0\\
26	0\\
27	0\\
28	0\\
29	0\\
30	0\\
31	0\\
32	0\\
33	0\\
34	0\\
35	0\\
36	0\\
37	0\\
38	0\\
39	0\\
40	0\\
41	0\\
42	0\\
43	0\\
44	0\\
45	0\\
46	0\\
47	0\\
48	0\\
49	0\\
50	0\\
51	0\\
52	0\\
53	0\\
54	0\\
55	0\\
56	0\\
57	0\\
58	0\\
59	0\\
60	0\\
61	0\\
62	0\\
63	0\\
64	0\\
65	0\\
66	0\\
67	0\\
68	0\\
69	0\\
70	0\\
71	0\\
72	0\\
73	0\\
74	0\\
75	0\\
76	0\\
77	0\\
78	0\\
79	0\\
80	0\\
81	0\\
82	0\\
83	0\\
84	0\\
85	0\\
86	0\\
87	0\\
88	0\\
89	0\\
90	0\\
91	0\\
92	0\\
93	0\\
94	0\\
95	0\\
96	0\\
97	0\\
98	0\\
99	0\\
100	0\\
101	0\\
102	0\\
103	0\\
104	0\\
105	0\\
106	0\\
107	0\\
108	0\\
109	0\\
110	0\\
111	0\\
112	0\\
113	0\\
114	0\\
115	0\\
116	0\\
117	0\\
118	0\\
119	0\\
120	0\\
121	0\\
122	0\\
123	0\\
124	0\\
125	0\\
126	0\\
127	0\\
128	0\\
129	0\\
130	0\\
131	0\\
132	0\\
133	0\\
134	0\\
135	0\\
136	0\\
137	0\\
138	0\\
139	0\\
140	0\\
141	0\\
142	0\\
143	0\\
144	0\\
145	0\\
146	0\\
147	0\\
148	0\\
149	0\\
150	0\\
151	0\\
152	0\\
153	0\\
154	0\\
155	0\\
156	0\\
157	0\\
158	0\\
159	0\\
160	0\\
161	0\\
162	0\\
163	0\\
164	0\\
165	0\\
166	0\\
167	0\\
168	0\\
169	0\\
170	0\\
171	0\\
172	0\\
173	0\\
174	0\\
175	0\\
176	0\\
177	0\\
178	0\\
179	0\\
180	0\\
181	0\\
182	0\\
183	0\\
184	0\\
185	0\\
186	0\\
187	0\\
188	0\\
189	0\\
190	0\\
191	0\\
192	0\\
193	0\\
194	0\\
195	0\\
196	0\\
197	0\\
198	0\\
199	0\\
200	0\\
201	0\\
202	0\\
203	0\\
204	0\\
205	0\\
206	0\\
207	0\\
208	0\\
209	0\\
210	0\\
211	0\\
212	0\\
213	0\\
214	0\\
215	0\\
216	0\\
217	0\\
218	0\\
219	0\\
220	0\\
221	0\\
222	0\\
223	0\\
224	0\\
225	0\\
226	0\\
227	0\\
228	0\\
229	0\\
230	0\\
231	0\\
232	0\\
233	0\\
234	0\\
235	0\\
236	0\\
237	0\\
238	0\\
239	0\\
240	0\\
241	0\\
242	0\\
243	0\\
244	0\\
245	0\\
246	0\\
247	0\\
248	0\\
249	0\\
250	0\\
251	0\\
252	0\\
253	0\\
254	0\\
255	0\\
256	0\\
257	0\\
258	0\\
259	0\\
260	0\\
261	0\\
262	0\\
263	0\\
264	0\\
265	0\\
266	0\\
267	0\\
268	0\\
269	0\\
270	0\\
271	0\\
272	0\\
273	0\\
274	0\\
275	0\\
276	0\\
277	0\\
278	0\\
279	0\\
280	0\\
281	0\\
282	0\\
283	0\\
284	0\\
285	0\\
286	0\\
287	0\\
288	0\\
289	0\\
290	0\\
291	0\\
292	0\\
293	0\\
294	0\\
295	0\\
296	0\\
297	0\\
298	0\\
299	0\\
300	0\\
301	0\\
302	0\\
303	0\\
304	0\\
305	0\\
306	0\\
307	0\\
308	0\\
309	0\\
310	0\\
311	0\\
312	0\\
313	0\\
314	0\\
315	0\\
316	0\\
317	0\\
318	0\\
319	0\\
320	0\\
321	0\\
322	0\\
323	0\\
324	0\\
325	0\\
326	0\\
327	0\\
328	0\\
329	0\\
330	0\\
331	0\\
332	0\\
333	0\\
334	0\\
335	0\\
336	0\\
337	0\\
338	0\\
339	0\\
340	0\\
341	0\\
342	0\\
343	0\\
344	0\\
345	0\\
346	0\\
347	0\\
348	0\\
349	0\\
350	0\\
351	0\\
352	0\\
353	0\\
354	0\\
355	0\\
356	0\\
357	0\\
358	0\\
359	0\\
360	0\\
361	0\\
362	0\\
363	0\\
364	0\\
365	0\\
366	0\\
367	0\\
368	0\\
369	0\\
370	0\\
371	0\\
372	0\\
373	0\\
374	0\\
375	0\\
376	0\\
377	0\\
378	0\\
379	0\\
380	0\\
381	0\\
382	0\\
383	0\\
384	0\\
385	0\\
386	0\\
387	0\\
388	0\\
389	0\\
390	0\\
391	0\\
392	0\\
393	0\\
394	0\\
395	0\\
396	0\\
397	0\\
398	0\\
399	0\\
400	0\\
401	0\\
402	0\\
403	0\\
404	0\\
405	0\\
406	0\\
407	0\\
408	0\\
409	0\\
410	0\\
411	0\\
412	0\\
413	0\\
414	0\\
415	0\\
416	0\\
417	0\\
418	0\\
419	0\\
420	0\\
421	0\\
422	0\\
423	0\\
424	0\\
425	0\\
426	0\\
427	0\\
428	0\\
429	0\\
430	0\\
431	0\\
432	0\\
433	0\\
434	0\\
435	0\\
436	0\\
437	0\\
438	0\\
439	0\\
440	0\\
441	0\\
442	0\\
443	0\\
444	0\\
445	0\\
446	0\\
447	0\\
448	0\\
449	0\\
450	0\\
451	0\\
452	0\\
453	0\\
454	0\\
455	0\\
456	0\\
457	0\\
458	0\\
459	0\\
460	0\\
461	0\\
462	0\\
463	0\\
464	0\\
465	0\\
466	0\\
467	0\\
468	0\\
469	0\\
470	0\\
471	0\\
472	0\\
473	0\\
474	0\\
475	0\\
476	0\\
477	0\\
478	0\\
479	0\\
480	0\\
481	0\\
482	0\\
483	0\\
484	0\\
485	0\\
486	0\\
487	0\\
488	0\\
489	0\\
490	0\\
491	0\\
492	0\\
493	0\\
494	0\\
495	0\\
496	0\\
497	0\\
498	0\\
499	0\\
500	0\\
501	0\\
502	0\\
503	0\\
504	0\\
505	0\\
506	0\\
507	0\\
508	0\\
509	0\\
510	0\\
511	0\\
512	0\\
513	0\\
514	0\\
515	0\\
516	0\\
517	0\\
518	0\\
519	0\\
520	0\\
521	0\\
522	0\\
523	0\\
524	0\\
525	0\\
526	0\\
527	0\\
528	0\\
529	0\\
530	0\\
531	0\\
532	0\\
533	0\\
534	0\\
535	0\\
536	0\\
537	0\\
538	0\\
539	0\\
540	0\\
541	0\\
542	0\\
543	0\\
544	0\\
545	0\\
546	0\\
547	0\\
548	0\\
549	0\\
550	0\\
551	0\\
552	0\\
553	0\\
554	0\\
555	0\\
556	0\\
557	0\\
558	0\\
559	0\\
560	0\\
561	0\\
562	0\\
563	0\\
564	0\\
565	0\\
566	0\\
567	0\\
568	0\\
569	0\\
570	0\\
571	0\\
572	0\\
573	0\\
574	0\\
575	0\\
576	0\\
577	0\\
578	0\\
579	0\\
580	0\\
581	0\\
582	0\\
583	0\\
584	0\\
585	0\\
586	0\\
587	0\\
588	0\\
589	0\\
590	0\\
591	0\\
592	0\\
593	0\\
594	0\\
595	0\\
596	0\\
597	0\\
598	0\\
599	0\\
600	0\\
};
\addplot [color=red,solid,forget plot]
  table[row sep=crcr]{%
1	0\\
2	0\\
3	0\\
4	0\\
5	0\\
6	0\\
7	0\\
8	0\\
9	0\\
10	0\\
11	0\\
12	0\\
13	0\\
14	0\\
15	0\\
16	0\\
17	0\\
18	0\\
19	0\\
20	0\\
21	0\\
22	0\\
23	0\\
24	0\\
25	0\\
26	0\\
27	0\\
28	0\\
29	0\\
30	0\\
31	0\\
32	0\\
33	0\\
34	0\\
35	0\\
36	0\\
37	0\\
38	0\\
39	0\\
40	0\\
41	0\\
42	0\\
43	0\\
44	0\\
45	0\\
46	0\\
47	0\\
48	0\\
49	0\\
50	0\\
51	0\\
52	0\\
53	0\\
54	0\\
55	0\\
56	0\\
57	0\\
58	0\\
59	0\\
60	0\\
61	0\\
62	0\\
63	0\\
64	0\\
65	0\\
66	0\\
67	0\\
68	0\\
69	0\\
70	0\\
71	0\\
72	0\\
73	0\\
74	0\\
75	0\\
76	0\\
77	0\\
78	0\\
79	0\\
80	0\\
81	0\\
82	0\\
83	0\\
84	0\\
85	0\\
86	0\\
87	0\\
88	0\\
89	0\\
90	0\\
91	0\\
92	0\\
93	0\\
94	0\\
95	0\\
96	0\\
97	0\\
98	0\\
99	0\\
100	0\\
101	0\\
102	0\\
103	0\\
104	0\\
105	0\\
106	0\\
107	0\\
108	0\\
109	0\\
110	0\\
111	0\\
112	0\\
113	0\\
114	0\\
115	0\\
116	0\\
117	0\\
118	0\\
119	0\\
120	0\\
121	0\\
122	0\\
123	0\\
124	0\\
125	0\\
126	0\\
127	0\\
128	0\\
129	0\\
130	0\\
131	0\\
132	0\\
133	0\\
134	0\\
135	0\\
136	0\\
137	0\\
138	0\\
139	0\\
140	0\\
141	0\\
142	0\\
143	0\\
144	0\\
145	0\\
146	0\\
147	0\\
148	0\\
149	0\\
150	0\\
151	0\\
152	0\\
153	0\\
154	0\\
155	0\\
156	0\\
157	0\\
158	0\\
159	0\\
160	0\\
161	0\\
162	0\\
163	0\\
164	0\\
165	0\\
166	0\\
167	0\\
168	0\\
169	0\\
170	0\\
171	0\\
172	0\\
173	0\\
174	0\\
175	0\\
176	0\\
177	0\\
178	0\\
179	0\\
180	0\\
181	0\\
182	0\\
183	0\\
184	0\\
185	0\\
186	0\\
187	0\\
188	0\\
189	0\\
190	0\\
191	0\\
192	0\\
193	0\\
194	0\\
195	0\\
196	0\\
197	0\\
198	0\\
199	0\\
200	0\\
201	0\\
202	0\\
203	0\\
204	0\\
205	0\\
206	0\\
207	0\\
208	0\\
209	0\\
210	0\\
211	0\\
212	0\\
213	0\\
214	0\\
215	0\\
216	0\\
217	0\\
218	0\\
219	0\\
220	0\\
221	0\\
222	0\\
223	0\\
224	0\\
225	0\\
226	0\\
227	0\\
228	0\\
229	0\\
230	0\\
231	0\\
232	0\\
233	0\\
234	0\\
235	0\\
236	0\\
237	0\\
238	0\\
239	0\\
240	0\\
241	0\\
242	0\\
243	0\\
244	0\\
245	0\\
246	0\\
247	0\\
248	0\\
249	0\\
250	0\\
251	0\\
252	0\\
253	0\\
254	0\\
255	0\\
256	0\\
257	0\\
258	0\\
259	0\\
260	0\\
261	0\\
262	0\\
263	0\\
264	0\\
265	0\\
266	0\\
267	0\\
268	0\\
269	0\\
270	0\\
271	0\\
272	0\\
273	0\\
274	0\\
275	0\\
276	0\\
277	0\\
278	0\\
279	0\\
280	0\\
281	0\\
282	0\\
283	0\\
284	0\\
285	0\\
286	0\\
287	0\\
288	0\\
289	0\\
290	0\\
291	0\\
292	0\\
293	0\\
294	0\\
295	0\\
296	0\\
297	0\\
298	0\\
299	0\\
300	0\\
301	0\\
302	0\\
303	0\\
304	0\\
305	0\\
306	0\\
307	0\\
308	0\\
309	0\\
310	0\\
311	0\\
312	0\\
313	0\\
314	0\\
315	0\\
316	0\\
317	0\\
318	0\\
319	0\\
320	0\\
321	0\\
322	0\\
323	0\\
324	0\\
325	0\\
326	0\\
327	0\\
328	0\\
329	0\\
330	0\\
331	0\\
332	0\\
333	0\\
334	0\\
335	0\\
336	0\\
337	0\\
338	0\\
339	0\\
340	0\\
341	0\\
342	0\\
343	0\\
344	0\\
345	0\\
346	0\\
347	0\\
348	0\\
349	0\\
350	0\\
351	0\\
352	0\\
353	0\\
354	0\\
355	0\\
356	0\\
357	0\\
358	0\\
359	0\\
360	0\\
361	0\\
362	0\\
363	0\\
364	0\\
365	0\\
366	0\\
367	0\\
368	0\\
369	0\\
370	0\\
371	0\\
372	0\\
373	0\\
374	0\\
375	0\\
376	0\\
377	0\\
378	0\\
379	0\\
380	0\\
381	0\\
382	0\\
383	0\\
384	0\\
385	0\\
386	0\\
387	0\\
388	0\\
389	0\\
390	0\\
391	0\\
392	0\\
393	0\\
394	0\\
395	0\\
396	0\\
397	0\\
398	0\\
399	0\\
400	0\\
401	0\\
402	0\\
403	0\\
404	0\\
405	0\\
406	0\\
407	0\\
408	0\\
409	0\\
410	0\\
411	0\\
412	0\\
413	0\\
414	0\\
415	0\\
416	0\\
417	0\\
418	0\\
419	0\\
420	0\\
421	0\\
422	0\\
423	0\\
424	0\\
425	0\\
426	0\\
427	0\\
428	0\\
429	0\\
430	0\\
431	0\\
432	0\\
433	0\\
434	0\\
435	0\\
436	0\\
437	0\\
438	0\\
439	0\\
440	0\\
441	0\\
442	0\\
443	0\\
444	0\\
445	0\\
446	0\\
447	0\\
448	0\\
449	0\\
450	0\\
451	0\\
452	0\\
453	0\\
454	0\\
455	0\\
456	0\\
457	0\\
458	0\\
459	0\\
460	0\\
461	0\\
462	0\\
463	0\\
464	0\\
465	0\\
466	0\\
467	0\\
468	0\\
469	0\\
470	0\\
471	0\\
472	0\\
473	0\\
474	0\\
475	0\\
476	0\\
477	0\\
478	0\\
479	0\\
480	0\\
481	0\\
482	0\\
483	0\\
484	0\\
485	0\\
486	0\\
487	0\\
488	0\\
489	0\\
490	0\\
491	0\\
492	0\\
493	0\\
494	0\\
495	0\\
496	0\\
497	0\\
498	0\\
499	0\\
500	0\\
501	0\\
502	0\\
503	0\\
504	0\\
505	0\\
506	0\\
507	0\\
508	0\\
509	0\\
510	0\\
511	0\\
512	0\\
513	0\\
514	0\\
515	0\\
516	0\\
517	0\\
518	0\\
519	0\\
520	0\\
521	0\\
522	0\\
523	0\\
524	0\\
525	0\\
526	0\\
527	0\\
528	0\\
529	0\\
530	0\\
531	0\\
532	0\\
533	0\\
534	0\\
535	0\\
536	0\\
537	0\\
538	0\\
539	0\\
540	0\\
541	0\\
542	0\\
543	0\\
544	0\\
545	0\\
546	0\\
547	0\\
548	0\\
549	0\\
550	0\\
551	0\\
552	0\\
553	0\\
554	0\\
555	0\\
556	0\\
557	0\\
558	0\\
559	0\\
560	0\\
561	0\\
562	0\\
563	0\\
564	0\\
565	0\\
566	0\\
567	0\\
568	0\\
569	0\\
570	0\\
571	0\\
572	0\\
573	0\\
574	0\\
575	0\\
576	0\\
577	0\\
578	0\\
579	0\\
580	0\\
581	0\\
582	0\\
583	0\\
584	0\\
585	0\\
586	0\\
587	0\\
588	0\\
589	0\\
590	0\\
591	0\\
592	0\\
593	0\\
594	0\\
595	0\\
596	0\\
597	0\\
598	0\\
599	0\\
600	0\\
};
\addplot [color=mycolor20,solid,forget plot]
  table[row sep=crcr]{%
1	0\\
2	0\\
3	0\\
4	0\\
5	0\\
6	0\\
7	0\\
8	0\\
9	0\\
10	0\\
11	0\\
12	0\\
13	0\\
14	0\\
15	0\\
16	0\\
17	0\\
18	0\\
19	0\\
20	0\\
21	0\\
22	0\\
23	0\\
24	0\\
25	0\\
26	0\\
27	0\\
28	0\\
29	0\\
30	0\\
31	0\\
32	0\\
33	0\\
34	0\\
35	0\\
36	0\\
37	0\\
38	0\\
39	0\\
40	0\\
41	0\\
42	0\\
43	0\\
44	0\\
45	0\\
46	0\\
47	0\\
48	0\\
49	0\\
50	0\\
51	0\\
52	0\\
53	0\\
54	0\\
55	0\\
56	0\\
57	0\\
58	0\\
59	0\\
60	0\\
61	0\\
62	0\\
63	0\\
64	0\\
65	0\\
66	0\\
67	0\\
68	0\\
69	0\\
70	0\\
71	0\\
72	0\\
73	0\\
74	0\\
75	0\\
76	0\\
77	0\\
78	0\\
79	0\\
80	0\\
81	0\\
82	0\\
83	0\\
84	0\\
85	0\\
86	0\\
87	0\\
88	0\\
89	0\\
90	0\\
91	0\\
92	0\\
93	0\\
94	0\\
95	0\\
96	0\\
97	0\\
98	0\\
99	0\\
100	0\\
101	0\\
102	0\\
103	0\\
104	0\\
105	0\\
106	0\\
107	0\\
108	0\\
109	0\\
110	0\\
111	0\\
112	0\\
113	0\\
114	0\\
115	0\\
116	0\\
117	0\\
118	0\\
119	0\\
120	0\\
121	0\\
122	0\\
123	0\\
124	0\\
125	0\\
126	0\\
127	0\\
128	0\\
129	0\\
130	0\\
131	0\\
132	0\\
133	0\\
134	0\\
135	0\\
136	0\\
137	0\\
138	0\\
139	0\\
140	0\\
141	0\\
142	0\\
143	0\\
144	0\\
145	0\\
146	0\\
147	0\\
148	0\\
149	0\\
150	0\\
151	0\\
152	0\\
153	0\\
154	0\\
155	0\\
156	0\\
157	0\\
158	0\\
159	0\\
160	0\\
161	0\\
162	0\\
163	0\\
164	0\\
165	0\\
166	0\\
167	0\\
168	0\\
169	0\\
170	0\\
171	0\\
172	0\\
173	0\\
174	0\\
175	0\\
176	0\\
177	0\\
178	0\\
179	0\\
180	0\\
181	0\\
182	0\\
183	0\\
184	0\\
185	0\\
186	0\\
187	0\\
188	0\\
189	0\\
190	0\\
191	0\\
192	0\\
193	0\\
194	0\\
195	0\\
196	0\\
197	0\\
198	0\\
199	0\\
200	0\\
201	0\\
202	0\\
203	0\\
204	0\\
205	0\\
206	0\\
207	0\\
208	0\\
209	0\\
210	0\\
211	0\\
212	0\\
213	0\\
214	0\\
215	0\\
216	0\\
217	0\\
218	0\\
219	0\\
220	0\\
221	0\\
222	0\\
223	0\\
224	0\\
225	0\\
226	0\\
227	0\\
228	0\\
229	0\\
230	0\\
231	0\\
232	0\\
233	0\\
234	0\\
235	0\\
236	0\\
237	0\\
238	0\\
239	0\\
240	0\\
241	0\\
242	0\\
243	0\\
244	0\\
245	0\\
246	0\\
247	0\\
248	0\\
249	0\\
250	0\\
251	0\\
252	0\\
253	0\\
254	0\\
255	0\\
256	0\\
257	0\\
258	0\\
259	0\\
260	0\\
261	0\\
262	0\\
263	0\\
264	0\\
265	0\\
266	0\\
267	0\\
268	0\\
269	0\\
270	0\\
271	0\\
272	0\\
273	0\\
274	0\\
275	0\\
276	0\\
277	0\\
278	0\\
279	0\\
280	0\\
281	0\\
282	0\\
283	0\\
284	0\\
285	0\\
286	0\\
287	0\\
288	0\\
289	0\\
290	0\\
291	0\\
292	0\\
293	0\\
294	0\\
295	0\\
296	0\\
297	0\\
298	0\\
299	0\\
300	0\\
301	0\\
302	0\\
303	0\\
304	0\\
305	0\\
306	0\\
307	0\\
308	0\\
309	0\\
310	0\\
311	0\\
312	0\\
313	0\\
314	0\\
315	0\\
316	0\\
317	0\\
318	0\\
319	0\\
320	0\\
321	0\\
322	0\\
323	0\\
324	0\\
325	0\\
326	0\\
327	0\\
328	0\\
329	0\\
330	0\\
331	0\\
332	0\\
333	0\\
334	0\\
335	0\\
336	0\\
337	0\\
338	0\\
339	0\\
340	0\\
341	0\\
342	0\\
343	0\\
344	0\\
345	0\\
346	0\\
347	0\\
348	0\\
349	0\\
350	0\\
351	0\\
352	0\\
353	0\\
354	0\\
355	0\\
356	0\\
357	0\\
358	0\\
359	0\\
360	0\\
361	0\\
362	0\\
363	0\\
364	0\\
365	0\\
366	0\\
367	0\\
368	0\\
369	0\\
370	0\\
371	0\\
372	0\\
373	0\\
374	0\\
375	0\\
376	0\\
377	0\\
378	0\\
379	0\\
380	0\\
381	0\\
382	0\\
383	0\\
384	0\\
385	0\\
386	0\\
387	0\\
388	0\\
389	0\\
390	0\\
391	0\\
392	0\\
393	0\\
394	0\\
395	0\\
396	0\\
397	0\\
398	0\\
399	0\\
400	0\\
401	0\\
402	0\\
403	0\\
404	0\\
405	0\\
406	0\\
407	0\\
408	0\\
409	0\\
410	0\\
411	0\\
412	0\\
413	0\\
414	0\\
415	0\\
416	0\\
417	0\\
418	0\\
419	0\\
420	0\\
421	0\\
422	0\\
423	0\\
424	0\\
425	0\\
426	0\\
427	0\\
428	0\\
429	0\\
430	0\\
431	0\\
432	0\\
433	0\\
434	0\\
435	0\\
436	0\\
437	0\\
438	0\\
439	0\\
440	0\\
441	0\\
442	0\\
443	0\\
444	0\\
445	0\\
446	0\\
447	0\\
448	0\\
449	0\\
450	0\\
451	0\\
452	0\\
453	0\\
454	0\\
455	0\\
456	0\\
457	0\\
458	0\\
459	0\\
460	0\\
461	0\\
462	0\\
463	0\\
464	0\\
465	0\\
466	0\\
467	0\\
468	0\\
469	0\\
470	0\\
471	0\\
472	0\\
473	0\\
474	0\\
475	0\\
476	0\\
477	0\\
478	0\\
479	0\\
480	0\\
481	0\\
482	0\\
483	0\\
484	0\\
485	0\\
486	0\\
487	0\\
488	0\\
489	0\\
490	0\\
491	0\\
492	0\\
493	0\\
494	0\\
495	0\\
496	0\\
497	0\\
498	0\\
499	0\\
500	0\\
501	0\\
502	0\\
503	0\\
504	0\\
505	0\\
506	0\\
507	0\\
508	0\\
509	0\\
510	0\\
511	0\\
512	0\\
513	0\\
514	0\\
515	0\\
516	0\\
517	0\\
518	0\\
519	0\\
520	0\\
521	0\\
522	0\\
523	0\\
524	0\\
525	0\\
526	0\\
527	0\\
528	0\\
529	0\\
530	0\\
531	0\\
532	0\\
533	0\\
534	0\\
535	0\\
536	0\\
537	0\\
538	0\\
539	0\\
540	0\\
541	0\\
542	0\\
543	0\\
544	0\\
545	0\\
546	0\\
547	0\\
548	0\\
549	0\\
550	0\\
551	0\\
552	0\\
553	0\\
554	0\\
555	0\\
556	0\\
557	0\\
558	0\\
559	0\\
560	0\\
561	0\\
562	0\\
563	0\\
564	0\\
565	0\\
566	0\\
567	0\\
568	0\\
569	0\\
570	0\\
571	0\\
572	0\\
573	0\\
574	0\\
575	0\\
576	0\\
577	0\\
578	0\\
579	0\\
580	0\\
581	0\\
582	0\\
583	0\\
584	0\\
585	0\\
586	0\\
587	0\\
588	0\\
589	0\\
590	0\\
591	0\\
592	0\\
593	0\\
594	0\\
595	0\\
596	0\\
597	0\\
598	0\\
599	0\\
600	0\\
};
\addplot [color=mycolor21,solid,forget plot]
  table[row sep=crcr]{%
1	0\\
2	0\\
3	0\\
4	0\\
5	0\\
6	0\\
7	0\\
8	0\\
9	0\\
10	0\\
11	0\\
12	0\\
13	0\\
14	0\\
15	0\\
16	0\\
17	0\\
18	0\\
19	0\\
20	0\\
21	0\\
22	0\\
23	0\\
24	0\\
25	0\\
26	0\\
27	0\\
28	0\\
29	0\\
30	0\\
31	0\\
32	0\\
33	0\\
34	0\\
35	0\\
36	0\\
37	0\\
38	0\\
39	0\\
40	0\\
41	0\\
42	0\\
43	0\\
44	0\\
45	0\\
46	0\\
47	0\\
48	0\\
49	0\\
50	0\\
51	0\\
52	0\\
53	0\\
54	0\\
55	0\\
56	0\\
57	0\\
58	0\\
59	0\\
60	0\\
61	0\\
62	0\\
63	0\\
64	0\\
65	0\\
66	0\\
67	0\\
68	0\\
69	0\\
70	0\\
71	0\\
72	0\\
73	0\\
74	0\\
75	0\\
76	0\\
77	0\\
78	0\\
79	0\\
80	0\\
81	0\\
82	0\\
83	0\\
84	0\\
85	0\\
86	0\\
87	0\\
88	0\\
89	0\\
90	0\\
91	0\\
92	0\\
93	0\\
94	0\\
95	0\\
96	0\\
97	0\\
98	0\\
99	0\\
100	0\\
101	0\\
102	0\\
103	0\\
104	0\\
105	0\\
106	0\\
107	0\\
108	0\\
109	0\\
110	0\\
111	0\\
112	0\\
113	0\\
114	0\\
115	0\\
116	0\\
117	0\\
118	0\\
119	0\\
120	0\\
121	0\\
122	0\\
123	0\\
124	0\\
125	0\\
126	0\\
127	0\\
128	0\\
129	0\\
130	0\\
131	0\\
132	0\\
133	0\\
134	0\\
135	0\\
136	0\\
137	0\\
138	0\\
139	0\\
140	0\\
141	0\\
142	0\\
143	0\\
144	0\\
145	0\\
146	0\\
147	0\\
148	0\\
149	0\\
150	0\\
151	0\\
152	0\\
153	0\\
154	0\\
155	0\\
156	0\\
157	0\\
158	0\\
159	0\\
160	0\\
161	0\\
162	0\\
163	0\\
164	0\\
165	0\\
166	0\\
167	0\\
168	0\\
169	0\\
170	0\\
171	0\\
172	0\\
173	0\\
174	0\\
175	0\\
176	0\\
177	0\\
178	0\\
179	0\\
180	0\\
181	0\\
182	0\\
183	0\\
184	0\\
185	0\\
186	0\\
187	0\\
188	0\\
189	0\\
190	0\\
191	0\\
192	0\\
193	0\\
194	0\\
195	0\\
196	0\\
197	0\\
198	0\\
199	0\\
200	0\\
201	0\\
202	0\\
203	0\\
204	0\\
205	0\\
206	0\\
207	0\\
208	0\\
209	0\\
210	0\\
211	0\\
212	0\\
213	0\\
214	0\\
215	0\\
216	0\\
217	0\\
218	0\\
219	0\\
220	0\\
221	0\\
222	0\\
223	0\\
224	0\\
225	0\\
226	0\\
227	0\\
228	0\\
229	0\\
230	0\\
231	0\\
232	0\\
233	0\\
234	0\\
235	0\\
236	0\\
237	0\\
238	0\\
239	0\\
240	0\\
241	0\\
242	0\\
243	0\\
244	0\\
245	0\\
246	0\\
247	0\\
248	0\\
249	0\\
250	0\\
251	0\\
252	0\\
253	0\\
254	0\\
255	0\\
256	0\\
257	0\\
258	0\\
259	0\\
260	0\\
261	0\\
262	0\\
263	0\\
264	0\\
265	0\\
266	0\\
267	0\\
268	0\\
269	0\\
270	0\\
271	0\\
272	0\\
273	0\\
274	0\\
275	0\\
276	0\\
277	0\\
278	0\\
279	0\\
280	0\\
281	0\\
282	0\\
283	0\\
284	0\\
285	0\\
286	0\\
287	0\\
288	0\\
289	0\\
290	0\\
291	0\\
292	0\\
293	0\\
294	0\\
295	0\\
296	0\\
297	0\\
298	0\\
299	0\\
300	0\\
301	0\\
302	0\\
303	0\\
304	0\\
305	0\\
306	0\\
307	0\\
308	0\\
309	0\\
310	0\\
311	0\\
312	0\\
313	0\\
314	0\\
315	0\\
316	0\\
317	0\\
318	0\\
319	0\\
320	0\\
321	0\\
322	0\\
323	0\\
324	0\\
325	0\\
326	0\\
327	0\\
328	0\\
329	0\\
330	0\\
331	0\\
332	0\\
333	0\\
334	0\\
335	0\\
336	0\\
337	0\\
338	0\\
339	0\\
340	0\\
341	0\\
342	0\\
343	0\\
344	0\\
345	0\\
346	0\\
347	0\\
348	0\\
349	0\\
350	0\\
351	0\\
352	0\\
353	0\\
354	0\\
355	0\\
356	0\\
357	0\\
358	0\\
359	0\\
360	0\\
361	0\\
362	0\\
363	0\\
364	0\\
365	0\\
366	0\\
367	0\\
368	0\\
369	0\\
370	0\\
371	0\\
372	0\\
373	0\\
374	0\\
375	0\\
376	0\\
377	0\\
378	0\\
379	0\\
380	0\\
381	0\\
382	0\\
383	0\\
384	0\\
385	0\\
386	0\\
387	0\\
388	0\\
389	0\\
390	0\\
391	0\\
392	0\\
393	0\\
394	0\\
395	0\\
396	0\\
397	0\\
398	0\\
399	0\\
400	0\\
401	0\\
402	0\\
403	0\\
404	0\\
405	0\\
406	0\\
407	0\\
408	0\\
409	0\\
410	0\\
411	0\\
412	0\\
413	0\\
414	0\\
415	0\\
416	0\\
417	0\\
418	0\\
419	0\\
420	0\\
421	0\\
422	0\\
423	0\\
424	0\\
425	0\\
426	0\\
427	0\\
428	0\\
429	0\\
430	0\\
431	0\\
432	0\\
433	0\\
434	0\\
435	0\\
436	0\\
437	0\\
438	0\\
439	0\\
440	0\\
441	0\\
442	0\\
443	0\\
444	0\\
445	0\\
446	0\\
447	0\\
448	0\\
449	0\\
450	0\\
451	0\\
452	0\\
453	0\\
454	0\\
455	0\\
456	0\\
457	0\\
458	0\\
459	0\\
460	0\\
461	0\\
462	0\\
463	0\\
464	0\\
465	0\\
466	0\\
467	0\\
468	0\\
469	0\\
470	0\\
471	0\\
472	0\\
473	0\\
474	0\\
475	0\\
476	0\\
477	0\\
478	0\\
479	0\\
480	0\\
481	0\\
482	0\\
483	0\\
484	0\\
485	0\\
486	0\\
487	0\\
488	0\\
489	0\\
490	0\\
491	0\\
492	0\\
493	0\\
494	0\\
495	0\\
496	0\\
497	0\\
498	0\\
499	0\\
500	0\\
501	0\\
502	0\\
503	0\\
504	0\\
505	0\\
506	0\\
507	0\\
508	0\\
509	0\\
510	0\\
511	0\\
512	0\\
513	0\\
514	0\\
515	0\\
516	0\\
517	0\\
518	0\\
519	0\\
520	0\\
521	0\\
522	0\\
523	0\\
524	0\\
525	0\\
526	0\\
527	0\\
528	0\\
529	0\\
530	0\\
531	0\\
532	0\\
533	0\\
534	0\\
535	0\\
536	0\\
537	0\\
538	0\\
539	0\\
540	0\\
541	0\\
542	0\\
543	0\\
544	0\\
545	0\\
546	0\\
547	0\\
548	0\\
549	0\\
550	0\\
551	0\\
552	0\\
553	0\\
554	0\\
555	0\\
556	0\\
557	0\\
558	0\\
559	0\\
560	0\\
561	0\\
562	0\\
563	0\\
564	0\\
565	0\\
566	0\\
567	0\\
568	0\\
569	0\\
570	0\\
571	0\\
572	0\\
573	0\\
574	0\\
575	0\\
576	0\\
577	0\\
578	0\\
579	0\\
580	0\\
581	0\\
582	0\\
583	0\\
584	0\\
585	0\\
586	0\\
587	0\\
588	0\\
589	0\\
590	0\\
591	0\\
592	0\\
593	0\\
594	0\\
595	0\\
596	0\\
597	0\\
598	0\\
599	0\\
600	0\\
};
\addplot [color=black!20!mycolor21,solid,forget plot]
  table[row sep=crcr]{%
1	0\\
2	0\\
3	0\\
4	0\\
5	0\\
6	0\\
7	0\\
8	0\\
9	0\\
10	0\\
11	0\\
12	0\\
13	0\\
14	0\\
15	0\\
16	0\\
17	0\\
18	0\\
19	0\\
20	0\\
21	0\\
22	0\\
23	0\\
24	0\\
25	0\\
26	0\\
27	0\\
28	0\\
29	0\\
30	0\\
31	0\\
32	0\\
33	0\\
34	0\\
35	0\\
36	0\\
37	0\\
38	0\\
39	0\\
40	0\\
41	0\\
42	0\\
43	0\\
44	0\\
45	0\\
46	0\\
47	0\\
48	0\\
49	0\\
50	0\\
51	0\\
52	0\\
53	0\\
54	0\\
55	0\\
56	0\\
57	0\\
58	0\\
59	0\\
60	0\\
61	0\\
62	0\\
63	0\\
64	0\\
65	0\\
66	0\\
67	0\\
68	0\\
69	0\\
70	0\\
71	0\\
72	0\\
73	0\\
74	0\\
75	0\\
76	0\\
77	0\\
78	0\\
79	0\\
80	0\\
81	0\\
82	0\\
83	0\\
84	0\\
85	0\\
86	0\\
87	0\\
88	0\\
89	0\\
90	0\\
91	0\\
92	0\\
93	0\\
94	0\\
95	0\\
96	0\\
97	0\\
98	0\\
99	0\\
100	0\\
101	0\\
102	0\\
103	0\\
104	0\\
105	0\\
106	0\\
107	0\\
108	0\\
109	0\\
110	0\\
111	0\\
112	0\\
113	0\\
114	0\\
115	0\\
116	0\\
117	0\\
118	0\\
119	0\\
120	0\\
121	0\\
122	0\\
123	0\\
124	0\\
125	0\\
126	0\\
127	0\\
128	0\\
129	0\\
130	0\\
131	0\\
132	0\\
133	0\\
134	0\\
135	0\\
136	0\\
137	0\\
138	0\\
139	0\\
140	0\\
141	0\\
142	0\\
143	0\\
144	0\\
145	0\\
146	0\\
147	0\\
148	0\\
149	0\\
150	0\\
151	0\\
152	0\\
153	0\\
154	0\\
155	0\\
156	0\\
157	0\\
158	0\\
159	0\\
160	0\\
161	0\\
162	0\\
163	0\\
164	0\\
165	0\\
166	0\\
167	0\\
168	0\\
169	0\\
170	0\\
171	0\\
172	0\\
173	0\\
174	0\\
175	0\\
176	0\\
177	0\\
178	0\\
179	0\\
180	0\\
181	0\\
182	0\\
183	0\\
184	0\\
185	0\\
186	0\\
187	0\\
188	0\\
189	0\\
190	0\\
191	0\\
192	0\\
193	0\\
194	0\\
195	0\\
196	0\\
197	0\\
198	0\\
199	0\\
200	0\\
201	0\\
202	0\\
203	0\\
204	0\\
205	0\\
206	0\\
207	0\\
208	0\\
209	0\\
210	0\\
211	0\\
212	0\\
213	0\\
214	0\\
215	0\\
216	0\\
217	0\\
218	0\\
219	0\\
220	0\\
221	0\\
222	0\\
223	0\\
224	0\\
225	0\\
226	0\\
227	0\\
228	0\\
229	0\\
230	0\\
231	0\\
232	0\\
233	0\\
234	0\\
235	0\\
236	0\\
237	0\\
238	0\\
239	0\\
240	0\\
241	0\\
242	0\\
243	0\\
244	0\\
245	0\\
246	0\\
247	0\\
248	0\\
249	0\\
250	0\\
251	0\\
252	0\\
253	0\\
254	0\\
255	0\\
256	0\\
257	0\\
258	0\\
259	0\\
260	0\\
261	0\\
262	0\\
263	0\\
264	0\\
265	0\\
266	0\\
267	0\\
268	0\\
269	0\\
270	0\\
271	0\\
272	0\\
273	0\\
274	0\\
275	0\\
276	0\\
277	0\\
278	0\\
279	0\\
280	0\\
281	0\\
282	0\\
283	0\\
284	0\\
285	0\\
286	0\\
287	0\\
288	0\\
289	0\\
290	0\\
291	0\\
292	0\\
293	0\\
294	0\\
295	0\\
296	0\\
297	0\\
298	0\\
299	0\\
300	0\\
301	0\\
302	0\\
303	0\\
304	0\\
305	0\\
306	0\\
307	0\\
308	0\\
309	0\\
310	0\\
311	0\\
312	0\\
313	0\\
314	0\\
315	0\\
316	0\\
317	0\\
318	0\\
319	0\\
320	0\\
321	0\\
322	0\\
323	0\\
324	0\\
325	0\\
326	0\\
327	0\\
328	0\\
329	0\\
330	0\\
331	0\\
332	0\\
333	0\\
334	0\\
335	0\\
336	0\\
337	0\\
338	0\\
339	0\\
340	0\\
341	0\\
342	0\\
343	0\\
344	0\\
345	0\\
346	0\\
347	0\\
348	0\\
349	0\\
350	0\\
351	0\\
352	0\\
353	0\\
354	0\\
355	0\\
356	0\\
357	0\\
358	0\\
359	0\\
360	0\\
361	0\\
362	0\\
363	0\\
364	0\\
365	0\\
366	0\\
367	0\\
368	0\\
369	0\\
370	0\\
371	0\\
372	0\\
373	0\\
374	0\\
375	0\\
376	0\\
377	0\\
378	0\\
379	0\\
380	0\\
381	0\\
382	0\\
383	0\\
384	0\\
385	0\\
386	0\\
387	0\\
388	0\\
389	0\\
390	0\\
391	0\\
392	0\\
393	0\\
394	0\\
395	0\\
396	0\\
397	0\\
398	0\\
399	0\\
400	0\\
401	0\\
402	0\\
403	0\\
404	0\\
405	0\\
406	0\\
407	0\\
408	0\\
409	0\\
410	0\\
411	0\\
412	0\\
413	0\\
414	0\\
415	0\\
416	0\\
417	0\\
418	0\\
419	0\\
420	0\\
421	0\\
422	0\\
423	0\\
424	0\\
425	0\\
426	0\\
427	0\\
428	0\\
429	0\\
430	0\\
431	0\\
432	0\\
433	0\\
434	0\\
435	0\\
436	0\\
437	0\\
438	0\\
439	0\\
440	0\\
441	0\\
442	0\\
443	0\\
444	0\\
445	0\\
446	0\\
447	0\\
448	0\\
449	0\\
450	0\\
451	0\\
452	0\\
453	0\\
454	0\\
455	0\\
456	0\\
457	0\\
458	0\\
459	0\\
460	0\\
461	0\\
462	0\\
463	0\\
464	0\\
465	0\\
466	0\\
467	0\\
468	0\\
469	0\\
470	0\\
471	0\\
472	0\\
473	0\\
474	0\\
475	0\\
476	0\\
477	0\\
478	0\\
479	0\\
480	0\\
481	0\\
482	0\\
483	0\\
484	0\\
485	0\\
486	0\\
487	0\\
488	0\\
489	0\\
490	0\\
491	0\\
492	0\\
493	0\\
494	0\\
495	0\\
496	0\\
497	0\\
498	0\\
499	0\\
500	0\\
501	0\\
502	0\\
503	0\\
504	0\\
505	0\\
506	0\\
507	0\\
508	0\\
509	0\\
510	0\\
511	0\\
512	0\\
513	0\\
514	0\\
515	0\\
516	0\\
517	0\\
518	0\\
519	0\\
520	0\\
521	0\\
522	0\\
523	0\\
524	0\\
525	0\\
526	0\\
527	0\\
528	0\\
529	0\\
530	0\\
531	0\\
532	0\\
533	0\\
534	0\\
535	0\\
536	0\\
537	0\\
538	0\\
539	0\\
540	0\\
541	0\\
542	0\\
543	0\\
544	0\\
545	0\\
546	0\\
547	0\\
548	0\\
549	0\\
550	0\\
551	0\\
552	0\\
553	0\\
554	0\\
555	0\\
556	0\\
557	0\\
558	0\\
559	0\\
560	0\\
561	0\\
562	0\\
563	0\\
564	0\\
565	0\\
566	0\\
567	0\\
568	0\\
569	0\\
570	0\\
571	0\\
572	0\\
573	0\\
574	0\\
575	0\\
576	0\\
577	0\\
578	0\\
579	0\\
580	0\\
581	0\\
582	0\\
583	0\\
584	0\\
585	0\\
586	0\\
587	0\\
588	0\\
589	0\\
590	0\\
591	0\\
592	0\\
593	0\\
594	0\\
595	0\\
596	0\\
597	0\\
598	0\\
599	0\\
600	0\\
};
\addplot [color=black!50!mycolor20,solid,forget plot]
  table[row sep=crcr]{%
1	0\\
2	0\\
3	0\\
4	0\\
5	0\\
6	0\\
7	0\\
8	0\\
9	0\\
10	0\\
11	0\\
12	0\\
13	0\\
14	0\\
15	0\\
16	0\\
17	0\\
18	0\\
19	0\\
20	0\\
21	0\\
22	0\\
23	0\\
24	0\\
25	0\\
26	0\\
27	0\\
28	0\\
29	0\\
30	0\\
31	0\\
32	0\\
33	0\\
34	0\\
35	0\\
36	0\\
37	0\\
38	0\\
39	0\\
40	0\\
41	0\\
42	0\\
43	0\\
44	0\\
45	0\\
46	0\\
47	0\\
48	0\\
49	0\\
50	0\\
51	0\\
52	0\\
53	0\\
54	0\\
55	0\\
56	0\\
57	0\\
58	0\\
59	0\\
60	0\\
61	0\\
62	0\\
63	0\\
64	0\\
65	0\\
66	0\\
67	0\\
68	0\\
69	0\\
70	0\\
71	0\\
72	0\\
73	0\\
74	0\\
75	0\\
76	0\\
77	0\\
78	0\\
79	0\\
80	0\\
81	0\\
82	0\\
83	0\\
84	0\\
85	0\\
86	0\\
87	0\\
88	0\\
89	0\\
90	0\\
91	0\\
92	0\\
93	0\\
94	0\\
95	0\\
96	0\\
97	0\\
98	0\\
99	0\\
100	0\\
101	0\\
102	0\\
103	0\\
104	0\\
105	0\\
106	0\\
107	0\\
108	0\\
109	0\\
110	0\\
111	0\\
112	0\\
113	0\\
114	0\\
115	0\\
116	0\\
117	0\\
118	0\\
119	0\\
120	0\\
121	0\\
122	0\\
123	0\\
124	0\\
125	0\\
126	0\\
127	0\\
128	0\\
129	0\\
130	0\\
131	0\\
132	0\\
133	0\\
134	0\\
135	0\\
136	0\\
137	0\\
138	0\\
139	0\\
140	0\\
141	0\\
142	0\\
143	0\\
144	0\\
145	0\\
146	0\\
147	0\\
148	0\\
149	0\\
150	0\\
151	0\\
152	0\\
153	0\\
154	0\\
155	0\\
156	0\\
157	0\\
158	0\\
159	0\\
160	0\\
161	0\\
162	0\\
163	0\\
164	0\\
165	0\\
166	0\\
167	0\\
168	0\\
169	0\\
170	0\\
171	0\\
172	0\\
173	0\\
174	0\\
175	0\\
176	0\\
177	0\\
178	0\\
179	0\\
180	0\\
181	0\\
182	0\\
183	0\\
184	0\\
185	0\\
186	0\\
187	0\\
188	0\\
189	0\\
190	0\\
191	0\\
192	0\\
193	0\\
194	0\\
195	0\\
196	0\\
197	0\\
198	0\\
199	0\\
200	0\\
201	0\\
202	0\\
203	0\\
204	0\\
205	0\\
206	0\\
207	0\\
208	0\\
209	0\\
210	0\\
211	0\\
212	0\\
213	0\\
214	0\\
215	0\\
216	0\\
217	0\\
218	0\\
219	0\\
220	0\\
221	0\\
222	0\\
223	0\\
224	0\\
225	0\\
226	0\\
227	0\\
228	0\\
229	0\\
230	0\\
231	0\\
232	0\\
233	0\\
234	0\\
235	0\\
236	0\\
237	0\\
238	0\\
239	0\\
240	0\\
241	0\\
242	0\\
243	0\\
244	0\\
245	0\\
246	0\\
247	0\\
248	0\\
249	0\\
250	0\\
251	0\\
252	0\\
253	0\\
254	0\\
255	0\\
256	0\\
257	0\\
258	0\\
259	0\\
260	0\\
261	0\\
262	0\\
263	0\\
264	0\\
265	0\\
266	0\\
267	0\\
268	0\\
269	0\\
270	0\\
271	0\\
272	0\\
273	0\\
274	0\\
275	0\\
276	0\\
277	0\\
278	0\\
279	0\\
280	0\\
281	0\\
282	0\\
283	0\\
284	0\\
285	0\\
286	0\\
287	0\\
288	0\\
289	0\\
290	0\\
291	0\\
292	0\\
293	0\\
294	0\\
295	0\\
296	0\\
297	0\\
298	0\\
299	0\\
300	0\\
301	0\\
302	0\\
303	0\\
304	0\\
305	0\\
306	0\\
307	0\\
308	0\\
309	0\\
310	0\\
311	0\\
312	0\\
313	0\\
314	0\\
315	0\\
316	0\\
317	0\\
318	0\\
319	0\\
320	0\\
321	0\\
322	0\\
323	0\\
324	0\\
325	0\\
326	0\\
327	0\\
328	0\\
329	0\\
330	0\\
331	0\\
332	0\\
333	0\\
334	0\\
335	0\\
336	0\\
337	0\\
338	0\\
339	0\\
340	0\\
341	0\\
342	0\\
343	0\\
344	0\\
345	0\\
346	0\\
347	0\\
348	0\\
349	0\\
350	0\\
351	0\\
352	0\\
353	0\\
354	0\\
355	0\\
356	0\\
357	0\\
358	0\\
359	0\\
360	0\\
361	0\\
362	0\\
363	0\\
364	0\\
365	0\\
366	0\\
367	0\\
368	0\\
369	0\\
370	0\\
371	0\\
372	0\\
373	0\\
374	0\\
375	0\\
376	0\\
377	0\\
378	0\\
379	0\\
380	0\\
381	0\\
382	0\\
383	0\\
384	0\\
385	0\\
386	0\\
387	0\\
388	0\\
389	0\\
390	0\\
391	0\\
392	0\\
393	0\\
394	0\\
395	0\\
396	0\\
397	0\\
398	0\\
399	0\\
400	0\\
401	0\\
402	0\\
403	0\\
404	0\\
405	0\\
406	0\\
407	0\\
408	0\\
409	0\\
410	0\\
411	0\\
412	0\\
413	0\\
414	0\\
415	0\\
416	0\\
417	0\\
418	0\\
419	0\\
420	0\\
421	0\\
422	0\\
423	0\\
424	0\\
425	0\\
426	0\\
427	0\\
428	0\\
429	0\\
430	0\\
431	0\\
432	0\\
433	0\\
434	0\\
435	0\\
436	0\\
437	0\\
438	0\\
439	0\\
440	0\\
441	0\\
442	0\\
443	0\\
444	0\\
445	0\\
446	0\\
447	0\\
448	0\\
449	0\\
450	0\\
451	0\\
452	0\\
453	0\\
454	0\\
455	0\\
456	0\\
457	0\\
458	0\\
459	0\\
460	0\\
461	0\\
462	0\\
463	0\\
464	0\\
465	0\\
466	0\\
467	0\\
468	0\\
469	0\\
470	0\\
471	0\\
472	0\\
473	0\\
474	0\\
475	0\\
476	0\\
477	0\\
478	0\\
479	0\\
480	0\\
481	0\\
482	0\\
483	0\\
484	0\\
485	0\\
486	0\\
487	0\\
488	0\\
489	0\\
490	0\\
491	0\\
492	0\\
493	0\\
494	0\\
495	0\\
496	0\\
497	0\\
498	0\\
499	0\\
500	0\\
501	0\\
502	0\\
503	0\\
504	0\\
505	0\\
506	0\\
507	0\\
508	0\\
509	0\\
510	0\\
511	0\\
512	0\\
513	0\\
514	0\\
515	0\\
516	0\\
517	0\\
518	0\\
519	0\\
520	0\\
521	0\\
522	0\\
523	0\\
524	0\\
525	0\\
526	0\\
527	0\\
528	0\\
529	0\\
530	0\\
531	0\\
532	0\\
533	0\\
534	0\\
535	0\\
536	0\\
537	0\\
538	0\\
539	0\\
540	0\\
541	0\\
542	0\\
543	0\\
544	0\\
545	0\\
546	0\\
547	0\\
548	0\\
549	0\\
550	0\\
551	0\\
552	0\\
553	0\\
554	0\\
555	0\\
556	0\\
557	0\\
558	0\\
559	0\\
560	0\\
561	0\\
562	0\\
563	0\\
564	0\\
565	0\\
566	0\\
567	0\\
568	0\\
569	0\\
570	0\\
571	0\\
572	0\\
573	0\\
574	0\\
575	0\\
576	0\\
577	0\\
578	0\\
579	0\\
580	0\\
581	0\\
582	0\\
583	0\\
584	0\\
585	0\\
586	0\\
587	0\\
588	0\\
589	0\\
590	0\\
591	0\\
592	0\\
593	0\\
594	0\\
595	0\\
596	0\\
597	0\\
598	0\\
599	0\\
600	0\\
};
\addplot [color=black!60!mycolor21,solid,forget plot]
  table[row sep=crcr]{%
1	0\\
2	0\\
3	0\\
4	0\\
5	0\\
6	0\\
7	0\\
8	0\\
9	0\\
10	0\\
11	0\\
12	0\\
13	0\\
14	0\\
15	0\\
16	0\\
17	0\\
18	0\\
19	0\\
20	0\\
21	0\\
22	0\\
23	0\\
24	0\\
25	0\\
26	0\\
27	0\\
28	0\\
29	0\\
30	0\\
31	0\\
32	0\\
33	0\\
34	0\\
35	0\\
36	0\\
37	0\\
38	0\\
39	0\\
40	0\\
41	0\\
42	0\\
43	0\\
44	0\\
45	0\\
46	0\\
47	0\\
48	0\\
49	0\\
50	0\\
51	0\\
52	0\\
53	0\\
54	0\\
55	0\\
56	0\\
57	0\\
58	0\\
59	0\\
60	0\\
61	0\\
62	0\\
63	0\\
64	0\\
65	0\\
66	0\\
67	0\\
68	0\\
69	0\\
70	0\\
71	0\\
72	0\\
73	0\\
74	0\\
75	0\\
76	0\\
77	0\\
78	0\\
79	0\\
80	0\\
81	0\\
82	0\\
83	0\\
84	0\\
85	0\\
86	0\\
87	0\\
88	0\\
89	0\\
90	0\\
91	0\\
92	0\\
93	0\\
94	0\\
95	0\\
96	0\\
97	0\\
98	0\\
99	0\\
100	0\\
101	0\\
102	0\\
103	0\\
104	0\\
105	0\\
106	0\\
107	0\\
108	0\\
109	0\\
110	0\\
111	0\\
112	0\\
113	0\\
114	0\\
115	0\\
116	0\\
117	0\\
118	0\\
119	0\\
120	0\\
121	0\\
122	0\\
123	0\\
124	0\\
125	0\\
126	0\\
127	0\\
128	0\\
129	0\\
130	0\\
131	0\\
132	0\\
133	0\\
134	0\\
135	0\\
136	0\\
137	0\\
138	0\\
139	0\\
140	0\\
141	0\\
142	0\\
143	0\\
144	0\\
145	0\\
146	0\\
147	0\\
148	0\\
149	0\\
150	0\\
151	0\\
152	0\\
153	0\\
154	0\\
155	0\\
156	0\\
157	0\\
158	0\\
159	0\\
160	0\\
161	0\\
162	0\\
163	0\\
164	0\\
165	0\\
166	0\\
167	0\\
168	0\\
169	0\\
170	0\\
171	0\\
172	0\\
173	0\\
174	0\\
175	0\\
176	0\\
177	0\\
178	0\\
179	0\\
180	0\\
181	0\\
182	0\\
183	0\\
184	0\\
185	0\\
186	0\\
187	0\\
188	0\\
189	0\\
190	0\\
191	0\\
192	0\\
193	0\\
194	0\\
195	0\\
196	0\\
197	0\\
198	0\\
199	0\\
200	0\\
201	0\\
202	0\\
203	0\\
204	0\\
205	0\\
206	0\\
207	0\\
208	0\\
209	0\\
210	0\\
211	0\\
212	0\\
213	0\\
214	0\\
215	0\\
216	0\\
217	0\\
218	0\\
219	0\\
220	0\\
221	0\\
222	0\\
223	0\\
224	0\\
225	0\\
226	0\\
227	0\\
228	0\\
229	0\\
230	0\\
231	0\\
232	0\\
233	0\\
234	0\\
235	0\\
236	0\\
237	0\\
238	0\\
239	0\\
240	0\\
241	0\\
242	0\\
243	0\\
244	0\\
245	0\\
246	0\\
247	0\\
248	0\\
249	0\\
250	0\\
251	0\\
252	0\\
253	0\\
254	0\\
255	0\\
256	0\\
257	0\\
258	0\\
259	0\\
260	0\\
261	0\\
262	0\\
263	0\\
264	0\\
265	0\\
266	0\\
267	0\\
268	0\\
269	0\\
270	0\\
271	0\\
272	0\\
273	0\\
274	0\\
275	0\\
276	0\\
277	0\\
278	0\\
279	0\\
280	0\\
281	0\\
282	0\\
283	0\\
284	0\\
285	0\\
286	0\\
287	0\\
288	0\\
289	0\\
290	0\\
291	0\\
292	0\\
293	0\\
294	0\\
295	0\\
296	0\\
297	0\\
298	0\\
299	0\\
300	0\\
301	0\\
302	0\\
303	0\\
304	0\\
305	0\\
306	0\\
307	0\\
308	0\\
309	0\\
310	0\\
311	0\\
312	0\\
313	0\\
314	0\\
315	0\\
316	0\\
317	0\\
318	0\\
319	0\\
320	0\\
321	0\\
322	0\\
323	0\\
324	0\\
325	0\\
326	0\\
327	0\\
328	0\\
329	0\\
330	0\\
331	0\\
332	0\\
333	0\\
334	0\\
335	0\\
336	0\\
337	0\\
338	0\\
339	0\\
340	0\\
341	0\\
342	0\\
343	0\\
344	0\\
345	0\\
346	0\\
347	0\\
348	0\\
349	0\\
350	0\\
351	0\\
352	0\\
353	0\\
354	0\\
355	0\\
356	0\\
357	0\\
358	0\\
359	0\\
360	0\\
361	0\\
362	0\\
363	0\\
364	0\\
365	0\\
366	0\\
367	0\\
368	0\\
369	0\\
370	0\\
371	0\\
372	0\\
373	0\\
374	0\\
375	0\\
376	0\\
377	0\\
378	0\\
379	0\\
380	0\\
381	0\\
382	0\\
383	0\\
384	0\\
385	0\\
386	0\\
387	0\\
388	0\\
389	0\\
390	0\\
391	0\\
392	0\\
393	0\\
394	0\\
395	0\\
396	0\\
397	0\\
398	0\\
399	0\\
400	0\\
401	0\\
402	0\\
403	0\\
404	0\\
405	0\\
406	0\\
407	0\\
408	0\\
409	0\\
410	0\\
411	0\\
412	0\\
413	0\\
414	0\\
415	0\\
416	0\\
417	0\\
418	0\\
419	0\\
420	0\\
421	0\\
422	0\\
423	0\\
424	0\\
425	0\\
426	0\\
427	0\\
428	0\\
429	0\\
430	0\\
431	0\\
432	0\\
433	0\\
434	0\\
435	0\\
436	0\\
437	0\\
438	0\\
439	0\\
440	0\\
441	0\\
442	0\\
443	0\\
444	0\\
445	0\\
446	0\\
447	0\\
448	0\\
449	0\\
450	0\\
451	0\\
452	0\\
453	0\\
454	0\\
455	0\\
456	0\\
457	0\\
458	0\\
459	0\\
460	0\\
461	0\\
462	0\\
463	0\\
464	0\\
465	0\\
466	0\\
467	0\\
468	0\\
469	0\\
470	0\\
471	0\\
472	0\\
473	0\\
474	0\\
475	0\\
476	0\\
477	0\\
478	0\\
479	0\\
480	0\\
481	0\\
482	0\\
483	0\\
484	0\\
485	0\\
486	0\\
487	0\\
488	0\\
489	0\\
490	0\\
491	0\\
492	0\\
493	0\\
494	0\\
495	0\\
496	0\\
497	0\\
498	0\\
499	0\\
500	0\\
501	0\\
502	0\\
503	0\\
504	0\\
505	0\\
506	0\\
507	0\\
508	0\\
509	0\\
510	0\\
511	0\\
512	0\\
513	0\\
514	0\\
515	0\\
516	0\\
517	0\\
518	0\\
519	0\\
520	0\\
521	0\\
522	0\\
523	0\\
524	0\\
525	0\\
526	0\\
527	0\\
528	0\\
529	0\\
530	0\\
531	0\\
532	0\\
533	0\\
534	0\\
535	0\\
536	0\\
537	0\\
538	0\\
539	0\\
540	0\\
541	0\\
542	0\\
543	0\\
544	0\\
545	0\\
546	0\\
547	0\\
548	0\\
549	0\\
550	0\\
551	0\\
552	0\\
553	0\\
554	0\\
555	0\\
556	0\\
557	0\\
558	0\\
559	0\\
560	0\\
561	0\\
562	0\\
563	0\\
564	0\\
565	0\\
566	0\\
567	0\\
568	0\\
569	0\\
570	0\\
571	0\\
572	0\\
573	0\\
574	0\\
575	0\\
576	0\\
577	0\\
578	0\\
579	0\\
580	0\\
581	0\\
582	0\\
583	0\\
584	0\\
585	0\\
586	0\\
587	0\\
588	0\\
589	0\\
590	0\\
591	0\\
592	0\\
593	0\\
594	0\\
595	0\\
596	0\\
597	0\\
598	0\\
599	0\\
600	0\\
};
\addplot [color=black!80!mycolor21,solid,forget plot]
  table[row sep=crcr]{%
1	0\\
2	0\\
3	0\\
4	0\\
5	0\\
6	0\\
7	0\\
8	0\\
9	0\\
10	0\\
11	0\\
12	0\\
13	0\\
14	0\\
15	0\\
16	0\\
17	0\\
18	0\\
19	0\\
20	0\\
21	0\\
22	0\\
23	0\\
24	0\\
25	0\\
26	0\\
27	0\\
28	0\\
29	0\\
30	0\\
31	0\\
32	0\\
33	0\\
34	0\\
35	0\\
36	0\\
37	0\\
38	0\\
39	0\\
40	0\\
41	0\\
42	0\\
43	0\\
44	0\\
45	0\\
46	0\\
47	0\\
48	0\\
49	0\\
50	0\\
51	0\\
52	0\\
53	0\\
54	0\\
55	0\\
56	0\\
57	0\\
58	0\\
59	0\\
60	0\\
61	0\\
62	0\\
63	0\\
64	0\\
65	0\\
66	0\\
67	0\\
68	0\\
69	0\\
70	0\\
71	0\\
72	0\\
73	0\\
74	0\\
75	0\\
76	0\\
77	0\\
78	0\\
79	0\\
80	0\\
81	0\\
82	0\\
83	0\\
84	0\\
85	0\\
86	0\\
87	0\\
88	0\\
89	0\\
90	0\\
91	0\\
92	0\\
93	0\\
94	0\\
95	0\\
96	0\\
97	0\\
98	0\\
99	0\\
100	0\\
101	0\\
102	0\\
103	0\\
104	0\\
105	0\\
106	0\\
107	0\\
108	0\\
109	0\\
110	0\\
111	0\\
112	0\\
113	0\\
114	0\\
115	0\\
116	0\\
117	0\\
118	0\\
119	0\\
120	0\\
121	0\\
122	0\\
123	0\\
124	0\\
125	0\\
126	0\\
127	0\\
128	0\\
129	0\\
130	0\\
131	0\\
132	0\\
133	0\\
134	0\\
135	0\\
136	0\\
137	0\\
138	0\\
139	0\\
140	0\\
141	0\\
142	0\\
143	0\\
144	0\\
145	0\\
146	0\\
147	0\\
148	0\\
149	0\\
150	0\\
151	0\\
152	0\\
153	0\\
154	0\\
155	0\\
156	0\\
157	0\\
158	0\\
159	0\\
160	0\\
161	0\\
162	0\\
163	0\\
164	0\\
165	0\\
166	0\\
167	0\\
168	0\\
169	0\\
170	0\\
171	0\\
172	0\\
173	0\\
174	0\\
175	0\\
176	0\\
177	0\\
178	0\\
179	0\\
180	0\\
181	0\\
182	0\\
183	0\\
184	0\\
185	0\\
186	0\\
187	0\\
188	0\\
189	0\\
190	0\\
191	0\\
192	0\\
193	0\\
194	0\\
195	0\\
196	0\\
197	0\\
198	0\\
199	0\\
200	0\\
201	0\\
202	0\\
203	0\\
204	0\\
205	0\\
206	0\\
207	0\\
208	0\\
209	0\\
210	0\\
211	0\\
212	0\\
213	0\\
214	0\\
215	0\\
216	0\\
217	0\\
218	0\\
219	0\\
220	0\\
221	0\\
222	0\\
223	0\\
224	0\\
225	0\\
226	0\\
227	0\\
228	0\\
229	0\\
230	0\\
231	0\\
232	0\\
233	0\\
234	0\\
235	0\\
236	0\\
237	0\\
238	0\\
239	0\\
240	0\\
241	0\\
242	0\\
243	0\\
244	0\\
245	0\\
246	0\\
247	0\\
248	0\\
249	0\\
250	0\\
251	0\\
252	0\\
253	0\\
254	0\\
255	0\\
256	0\\
257	0\\
258	0\\
259	0\\
260	0\\
261	0\\
262	0\\
263	0\\
264	0\\
265	0\\
266	0\\
267	0\\
268	0\\
269	0\\
270	0\\
271	0\\
272	0\\
273	0\\
274	0\\
275	0\\
276	0\\
277	0\\
278	0\\
279	0\\
280	0\\
281	0\\
282	0\\
283	0\\
284	0\\
285	0\\
286	0\\
287	0\\
288	0\\
289	0\\
290	0\\
291	0\\
292	0\\
293	0\\
294	0\\
295	0\\
296	0\\
297	0\\
298	0\\
299	0\\
300	0\\
301	0\\
302	0\\
303	0\\
304	0\\
305	0\\
306	0\\
307	0\\
308	0\\
309	0\\
310	0\\
311	0\\
312	0\\
313	0\\
314	0\\
315	0\\
316	0\\
317	0\\
318	0\\
319	0\\
320	0\\
321	0\\
322	0\\
323	0\\
324	0\\
325	0\\
326	0\\
327	0\\
328	0\\
329	0\\
330	0\\
331	0\\
332	0\\
333	0\\
334	0\\
335	0\\
336	0\\
337	0\\
338	0\\
339	0\\
340	0\\
341	0\\
342	0\\
343	0\\
344	0\\
345	0\\
346	0\\
347	0\\
348	0\\
349	0\\
350	0\\
351	0\\
352	0\\
353	0\\
354	0\\
355	0\\
356	0\\
357	0\\
358	0\\
359	0\\
360	0\\
361	0\\
362	0\\
363	0\\
364	0\\
365	0\\
366	0\\
367	0\\
368	0\\
369	0\\
370	0\\
371	0\\
372	0\\
373	0\\
374	0\\
375	0\\
376	0\\
377	0\\
378	0\\
379	0\\
380	0\\
381	0\\
382	0\\
383	0\\
384	0\\
385	0\\
386	0\\
387	0\\
388	0\\
389	0\\
390	0\\
391	0\\
392	0\\
393	0\\
394	0\\
395	0\\
396	0\\
397	0\\
398	0\\
399	0\\
400	0\\
401	0\\
402	0\\
403	0\\
404	0\\
405	0\\
406	0\\
407	0\\
408	0\\
409	0\\
410	0\\
411	0\\
412	0\\
413	0\\
414	0\\
415	0\\
416	0\\
417	0\\
418	0\\
419	0\\
420	0\\
421	0\\
422	0\\
423	0\\
424	0\\
425	0\\
426	0\\
427	0\\
428	0\\
429	0\\
430	0\\
431	0\\
432	0\\
433	0\\
434	0\\
435	0\\
436	0\\
437	0\\
438	0\\
439	0\\
440	0\\
441	0\\
442	0\\
443	0\\
444	0\\
445	0\\
446	0\\
447	0\\
448	0\\
449	0\\
450	0\\
451	0\\
452	0\\
453	0\\
454	0\\
455	0\\
456	0\\
457	0\\
458	0\\
459	0\\
460	0\\
461	0\\
462	0\\
463	0\\
464	0\\
465	0\\
466	0\\
467	0\\
468	0\\
469	0\\
470	0\\
471	0\\
472	0\\
473	0\\
474	0\\
475	0\\
476	0\\
477	0\\
478	0\\
479	0\\
480	0\\
481	0\\
482	0\\
483	0\\
484	0\\
485	0\\
486	0\\
487	0\\
488	0\\
489	0\\
490	0\\
491	0\\
492	0\\
493	0\\
494	0\\
495	0\\
496	0\\
497	0\\
498	0\\
499	0\\
500	0\\
501	0\\
502	0\\
503	0\\
504	0\\
505	0\\
506	0\\
507	0\\
508	0\\
509	0\\
510	0\\
511	0\\
512	0\\
513	0\\
514	0\\
515	0\\
516	0\\
517	0\\
518	0\\
519	0\\
520	0\\
521	0\\
522	0\\
523	0\\
524	0\\
525	0\\
526	0\\
527	0\\
528	0\\
529	0\\
530	0\\
531	0\\
532	0\\
533	0\\
534	0\\
535	0\\
536	0\\
537	0\\
538	0\\
539	0\\
540	0\\
541	0\\
542	0\\
543	0\\
544	0\\
545	0\\
546	0\\
547	0\\
548	0\\
549	0\\
550	0\\
551	0\\
552	0\\
553	0\\
554	0\\
555	0\\
556	0\\
557	0\\
558	0\\
559	0\\
560	0\\
561	0\\
562	0\\
563	0\\
564	0\\
565	0\\
566	0\\
567	0\\
568	0\\
569	0\\
570	0\\
571	0\\
572	0\\
573	0\\
574	0\\
575	0\\
576	0\\
577	0\\
578	0\\
579	0\\
580	0\\
581	0\\
582	0\\
583	0\\
584	0\\
585	0\\
586	0\\
587	0\\
588	0\\
589	0\\
590	0\\
591	0\\
592	0\\
593	0\\
594	0\\
595	0\\
596	0\\
597	0\\
598	0\\
599	0\\
600	0\\
};
\addplot [color=black,solid,forget plot]
  table[row sep=crcr]{%
1	0\\
2	0\\
3	0\\
4	0\\
5	0\\
6	0\\
7	0\\
8	0\\
9	0\\
10	0\\
11	0\\
12	0\\
13	0\\
14	0\\
15	0\\
16	0\\
17	0\\
18	0\\
19	0\\
20	0\\
21	0\\
22	0\\
23	0\\
24	0\\
25	0\\
26	0\\
27	0\\
28	0\\
29	0\\
30	0\\
31	0\\
32	0\\
33	0\\
34	0\\
35	0\\
36	0\\
37	0\\
38	0\\
39	0\\
40	0\\
41	0\\
42	0\\
43	0\\
44	0\\
45	0\\
46	0\\
47	0\\
48	0\\
49	0\\
50	0\\
51	0\\
52	0\\
53	0\\
54	0\\
55	0\\
56	0\\
57	0\\
58	0\\
59	0\\
60	0\\
61	0\\
62	0\\
63	0\\
64	0\\
65	0\\
66	0\\
67	0\\
68	0\\
69	0\\
70	0\\
71	0\\
72	0\\
73	0\\
74	0\\
75	0\\
76	0\\
77	0\\
78	0\\
79	0\\
80	0\\
81	0\\
82	0\\
83	0\\
84	0\\
85	0\\
86	0\\
87	0\\
88	0\\
89	0\\
90	0\\
91	0\\
92	0\\
93	0\\
94	0\\
95	0\\
96	0\\
97	0\\
98	0\\
99	0\\
100	0\\
101	0\\
102	0\\
103	0\\
104	0\\
105	0\\
106	0\\
107	0\\
108	0\\
109	0\\
110	0\\
111	0\\
112	0\\
113	0\\
114	0\\
115	0\\
116	0\\
117	0\\
118	0\\
119	0\\
120	0\\
121	0\\
122	0\\
123	0\\
124	0\\
125	0\\
126	0\\
127	0\\
128	0\\
129	0\\
130	0\\
131	0\\
132	0\\
133	0\\
134	0\\
135	0\\
136	0\\
137	0\\
138	0\\
139	0\\
140	0\\
141	0\\
142	0\\
143	0\\
144	0\\
145	0\\
146	0\\
147	0\\
148	0\\
149	0\\
150	0\\
151	0\\
152	0\\
153	0\\
154	0\\
155	0\\
156	0\\
157	0\\
158	0\\
159	0\\
160	0\\
161	0\\
162	0\\
163	0\\
164	0\\
165	0\\
166	0\\
167	0\\
168	0\\
169	0\\
170	0\\
171	0\\
172	0\\
173	0\\
174	0\\
175	0\\
176	0\\
177	0\\
178	0\\
179	0\\
180	0\\
181	0\\
182	0\\
183	0\\
184	0\\
185	0\\
186	0\\
187	0\\
188	0\\
189	0\\
190	0\\
191	0\\
192	0\\
193	0\\
194	0\\
195	0\\
196	0\\
197	0\\
198	0\\
199	0\\
200	0\\
201	0\\
202	0\\
203	0\\
204	0\\
205	0\\
206	0\\
207	0\\
208	0\\
209	0\\
210	0\\
211	0\\
212	0\\
213	0\\
214	0\\
215	0\\
216	0\\
217	0\\
218	0\\
219	0\\
220	0\\
221	0\\
222	0\\
223	0\\
224	0\\
225	0\\
226	0\\
227	0\\
228	0\\
229	0\\
230	0\\
231	0\\
232	0\\
233	0\\
234	0\\
235	0\\
236	0\\
237	0\\
238	0\\
239	0\\
240	0\\
241	0\\
242	0\\
243	0\\
244	0\\
245	0\\
246	0\\
247	0\\
248	0\\
249	0\\
250	0\\
251	0\\
252	0\\
253	0\\
254	0\\
255	0\\
256	0\\
257	0\\
258	0\\
259	0\\
260	0\\
261	0\\
262	0\\
263	0\\
264	0\\
265	0\\
266	0\\
267	0\\
268	0\\
269	0\\
270	0\\
271	0\\
272	0\\
273	0\\
274	0\\
275	0\\
276	0\\
277	0\\
278	0\\
279	0\\
280	0\\
281	0\\
282	0\\
283	0\\
284	0\\
285	0\\
286	0\\
287	0\\
288	0\\
289	0\\
290	0\\
291	0\\
292	0\\
293	0\\
294	0\\
295	0\\
296	0\\
297	0\\
298	0\\
299	0\\
300	0\\
301	0\\
302	0\\
303	0\\
304	0\\
305	0\\
306	0\\
307	0\\
308	0\\
309	0\\
310	0\\
311	0\\
312	0\\
313	0\\
314	0\\
315	0\\
316	0\\
317	0\\
318	0\\
319	0\\
320	0\\
321	0\\
322	0\\
323	0\\
324	0\\
325	0\\
326	0\\
327	0\\
328	0\\
329	0\\
330	0\\
331	0\\
332	0\\
333	0\\
334	0\\
335	0\\
336	0\\
337	0\\
338	0\\
339	0\\
340	0\\
341	0\\
342	0\\
343	0\\
344	0\\
345	0\\
346	0\\
347	0\\
348	0\\
349	0\\
350	0\\
351	0\\
352	0\\
353	0\\
354	0\\
355	0\\
356	0\\
357	0\\
358	0\\
359	0\\
360	0\\
361	0\\
362	0\\
363	0\\
364	0\\
365	0\\
366	0\\
367	0\\
368	0\\
369	0\\
370	0\\
371	0\\
372	0\\
373	0\\
374	0\\
375	0\\
376	0\\
377	0\\
378	0\\
379	0\\
380	0\\
381	0\\
382	0\\
383	0\\
384	0\\
385	0\\
386	0\\
387	0\\
388	0\\
389	0\\
390	0\\
391	0\\
392	0\\
393	0\\
394	0\\
395	0\\
396	0\\
397	0\\
398	0\\
399	0\\
400	0\\
401	0\\
402	0\\
403	0\\
404	0\\
405	0\\
406	0\\
407	0\\
408	0\\
409	0\\
410	0\\
411	0\\
412	0\\
413	0\\
414	0\\
415	0\\
416	0\\
417	0\\
418	0\\
419	0\\
420	0\\
421	0\\
422	0\\
423	0\\
424	0\\
425	0\\
426	0\\
427	0\\
428	0\\
429	0\\
430	0\\
431	0\\
432	0\\
433	0\\
434	0\\
435	0\\
436	0\\
437	0\\
438	0\\
439	0\\
440	0\\
441	0\\
442	0\\
443	0\\
444	0\\
445	0\\
446	0\\
447	0\\
448	0\\
449	0\\
450	0\\
451	0\\
452	0\\
453	0\\
454	0\\
455	0\\
456	0\\
457	0\\
458	0\\
459	0\\
460	0\\
461	0\\
462	0\\
463	0\\
464	0\\
465	0\\
466	0\\
467	0\\
468	0\\
469	0\\
470	0\\
471	0\\
472	0\\
473	0\\
474	0\\
475	0\\
476	0\\
477	0\\
478	0\\
479	0\\
480	0\\
481	0\\
482	0\\
483	0\\
484	0\\
485	0\\
486	0\\
487	0\\
488	0\\
489	0\\
490	0\\
491	0\\
492	0\\
493	0\\
494	0\\
495	0\\
496	0\\
497	0\\
498	0\\
499	0\\
500	0\\
501	0\\
502	0\\
503	0\\
504	0\\
505	0\\
506	0\\
507	0\\
508	0\\
509	0\\
510	0\\
511	0\\
512	0\\
513	0\\
514	0\\
515	0\\
516	0\\
517	0\\
518	0\\
519	0\\
520	0\\
521	0\\
522	0\\
523	0\\
524	0\\
525	0\\
526	0\\
527	0\\
528	0\\
529	0\\
530	0\\
531	0\\
532	0\\
533	0\\
534	0\\
535	0\\
536	0\\
537	0\\
538	0\\
539	0\\
540	0\\
541	0\\
542	0\\
543	0\\
544	0\\
545	0\\
546	0\\
547	0\\
548	0\\
549	0\\
550	0\\
551	0\\
552	0\\
553	0\\
554	0\\
555	0\\
556	0\\
557	0\\
558	0\\
559	0\\
560	0\\
561	0\\
562	0\\
563	0\\
564	0\\
565	0\\
566	0\\
567	0\\
568	0\\
569	0\\
570	0\\
571	0\\
572	0\\
573	0\\
574	0\\
575	0\\
576	0\\
577	0\\
578	0\\
579	0\\
580	0\\
581	0\\
582	0\\
583	0\\
584	0\\
585	0\\
586	0\\
587	0\\
588	0\\
589	0\\
590	0\\
591	0\\
592	0\\
593	0\\
594	0\\
595	0\\
596	0\\
597	0\\
598	0\\
599	0\\
600	0\\
};
\end{axis}
\end{tikzpicture}% 
%  \caption{Discrete Time w/ nFPC}
%\end{subfigure}\\
%
%\leavevmode\smash{\makebox[0pt]{\hspace{-7em}% HORIZONTAL POSITION           
%  \rotatebox[origin=l]{90}{\hspace{20em}% VERTICAL POSITION
%    Depth $\delta^-$}%
%}}\hspace{0pt plus 1filll}\null
%
%Time (s)
%
%\vspace{1cm}
%\begin{subfigure}{\linewidth}
%  \centering
%  \tikzsetnextfilename{deltalegend}
%  \definecolor{mycolor1}{rgb}{1.00000,0.00000,1.00000}%
\begin{tikzpicture}[framed]
    \begingroup
    % inits/clears the lists (which might be populated from previous
    % axes):
    \csname pgfplots@init@cleared@structures\endcsname
    \pgfplotsset{legend style={at={(0,1)},anchor=north west},legend columns=-1,legend style={draw=none,column sep=1ex},legend entries={$q=-4$,$q=-3$,$q=-2$,$q=-1$}}%
    
    \csname pgfplots@addlegendimage\endcsname{thick,green,dashed,sharp plot}
    \csname pgfplots@addlegendimage\endcsname{thick,mycolor1,dashed,sharp plot}
    \csname pgfplots@addlegendimage\endcsname{thick,red,dashed,sharp plot}
    \csname pgfplots@addlegendimage\endcsname{thick,blue,dashed,sharp plot}

    % draws the legend:
    \csname pgfplots@createlegend\endcsname
    \endgroup

    \begingroup
    % inits/clears the lists (which might be populated from previous
    % axes):
    \csname pgfplots@init@cleared@structures\endcsname
    \pgfplotsset{legend style={at={(3.75,0.5)},anchor=north west},legend columns=-1,legend style={draw=none,column sep=1ex},legend entries={$q=0$}}%

    \csname pgfplots@addlegendimage\endcsname{thick,black,sharp plot}

    % draws the legend:
    \csname pgfplots@createlegend\endcsname
    \endgroup

    \begingroup
    % inits/clears the lists (which might be populated from previous
    % axes):
    \csname pgfplots@init@cleared@structures\endcsname
    \pgfplotsset{legend style={at={(0,0)},anchor=north west},legend columns=-1,legend style={draw=none,column sep=1ex},legend entries={$q=+4$,$q=+3$,$q=+2$,$q=+1$}}%
    
    \csname pgfplots@addlegendimage\endcsname{thick,green,sharp plot}
    \csname pgfplots@addlegendimage\endcsname{thick,mycolor1,sharp plot}
    \csname pgfplots@addlegendimage\endcsname{thick,red,sharp plot}
    \csname pgfplots@addlegendimage\endcsname{thick,blue,sharp plot}

    % draws the legend:
    \csname pgfplots@createlegend\endcsname
    \endgroup
\end{tikzpicture} 
%\end{subfigure}%
%  \caption{Optimal sell depths $\delta^{-}$ for Markov state $Z=(\rho = -1, \Delta S = -1)$, implying heavy imbalance in favor of sell pressure, and having previously seen a downward price change. We expect the midprice to fall.}
%  \label{fig:comp_dm_z1}
%\end{figure}
%
%\begin{figure}
%\centering
%\begin{subfigure}{.45\linewidth}
%  \centering
%  \setlength\figureheight{\linewidth} 
%  \setlength\figurewidth{\linewidth}
%  \tikzsetnextfilename{dm_cts_z8}
%  % This file was created by matlab2tikz.
%
%The latest updates can be retrieved from
%  http://www.mathworks.com/matlabcentral/fileexchange/22022-matlab2tikz-matlab2tikz
%where you can also make suggestions and rate matlab2tikz.
%
\definecolor{mycolor1}{rgb}{0.00000,1.00000,0.14286}%
\definecolor{mycolor2}{rgb}{0.00000,1.00000,0.28571}%
\definecolor{mycolor3}{rgb}{0.00000,1.00000,0.42857}%
\definecolor{mycolor4}{rgb}{0.00000,1.00000,0.57143}%
\definecolor{mycolor5}{rgb}{0.00000,1.00000,0.71429}%
\definecolor{mycolor6}{rgb}{0.00000,1.00000,0.85714}%
\definecolor{mycolor7}{rgb}{0.00000,1.00000,1.00000}%
\definecolor{mycolor8}{rgb}{0.00000,0.87500,1.00000}%
\definecolor{mycolor9}{rgb}{0.00000,0.62500,1.00000}%
\definecolor{mycolor10}{rgb}{0.12500,0.00000,1.00000}%
\definecolor{mycolor11}{rgb}{0.25000,0.00000,1.00000}%
\definecolor{mycolor12}{rgb}{0.37500,0.00000,1.00000}%
\definecolor{mycolor13}{rgb}{0.50000,0.00000,1.00000}%
\definecolor{mycolor14}{rgb}{0.62500,0.00000,1.00000}%
\definecolor{mycolor15}{rgb}{0.75000,0.00000,1.00000}%
\definecolor{mycolor16}{rgb}{0.87500,0.00000,1.00000}%
\definecolor{mycolor17}{rgb}{1.00000,0.00000,1.00000}%
\definecolor{mycolor18}{rgb}{1.00000,0.00000,0.87500}%
\definecolor{mycolor19}{rgb}{1.00000,0.00000,0.62500}%
\definecolor{mycolor20}{rgb}{0.85714,0.00000,0.00000}%
\definecolor{mycolor21}{rgb}{0.71429,0.00000,0.00000}%
%
\begin{tikzpicture}

\begin{axis}[%
width=4.1in,
height=3.803in,
at={(0.809in,0.513in)},
scale only axis,
point meta min=0,
point meta max=1,
every outer x axis line/.append style={black},
every x tick label/.append style={font=\color{black}},
xmin=0,
xmax=600,
every outer y axis line/.append style={black},
every y tick label/.append style={font=\color{black}},
ymin=0,
ymax=0.012,
axis background/.style={fill=white},
axis x line*=bottom,
axis y line*=left,
colormap={mymap}{[1pt] rgb(0pt)=(0,1,0); rgb(7pt)=(0,1,1); rgb(15pt)=(0,0,1); rgb(23pt)=(1,0,1); rgb(31pt)=(1,0,0); rgb(38pt)=(0,0,0)},
colorbar,
colorbar style={separate axis lines,every outer x axis line/.append style={black},every x tick label/.append style={font=\color{black}},every outer y axis line/.append style={black},every y tick label/.append style={font=\color{black}},yticklabels={{-19},{-17},{-15},{-13},{-11},{-9},{-7},{-5},{-3},{-1},{1},{3},{5},{7},{9},{11},{13},{15},{17},{19}}}
]
\addplot [color=green,solid,forget plot]
  table[row sep=crcr]{%
0.01	0.00503700863926798\\
1.01	0.0050370095564963\\
2.01	0.00503701049210395\\
3.01	0.0050370114464563\\
4.01	0.00503701241992682\\
5.01	0.00503701341289536\\
6.01	0.00503701442575027\\
7.01	0.00503701545888663\\
8.01	0.00503701651270766\\
9.01	0.00503701758762468\\
10.01	0.00503701868405695\\
11.01	0.00503701980243221\\
12.01	0.00503702094318642\\
13.01	0.00503702210676399\\
14.01	0.00503702329361844\\
15.01	0.00503702450421181\\
16.01	0.0050370257390156\\
17.01	0.00503702699851016\\
18.01	0.00503702828318539\\
19.01	0.00503702959354146\\
20.01	0.00503703093008762\\
21.01	0.00503703229334332\\
22.01	0.00503703368383834\\
23.01	0.00503703510211298\\
24.01	0.00503703654871751\\
25.01	0.00503703802421365\\
26.01	0.00503703952917398\\
27.01	0.00503704106418201\\
28.01	0.00503704262983323\\
29.01	0.00503704422673434\\
30.01	0.00503704585550413\\
31.01	0.00503704751677359\\
32.01	0.00503704921118597\\
33.01	0.00503705093939702\\
34.01	0.00503705270207538\\
35.01	0.00503705449990291\\
36.01	0.00503705633357451\\
37.01	0.0050370582037991\\
38.01	0.00503706011129942\\
39.01	0.00503706205681168\\
40.01	0.00503706404108728\\
41.01	0.00503706606489202\\
42.01	0.00503706812900628\\
43.01	0.00503707023422634\\
44.01	0.00503707238136358\\
45.01	0.0050370745712455\\
46.01	0.00503707680471538\\
47.01	0.00503707908263312\\
48.01	0.00503708140587569\\
49.01	0.00503708377533685\\
50.01	0.00503708619192778\\
51.01	0.00503708865657709\\
52.01	0.0050370911702326\\
53.01	0.00503709373385891\\
54.01	0.00503709634844085\\
55.01	0.00503709901498176\\
56.01	0.00503710173450458\\
57.01	0.00503710450805214\\
58.01	0.00503710733668758\\
59.01	0.00503711022149461\\
60.01	0.00503711316357836\\
61.01	0.00503711616406474\\
62.01	0.00503711922410222\\
63.01	0.00503712234486101\\
64.01	0.00503712552753433\\
65.01	0.0050371287733387\\
66.01	0.00503713208351358\\
67.01	0.00503713545932324\\
68.01	0.00503713890205601\\
69.01	0.00503714241302494\\
70.01	0.00503714599356911\\
71.01	0.00503714964505306\\
72.01	0.00503715336886795\\
73.01	0.00503715716643199\\
74.01	0.00503716103919033\\
75.01	0.00503716498861637\\
76.01	0.00503716901621196\\
77.01	0.00503717312350804\\
78.01	0.00503717731206504\\
79.01	0.00503718158347313\\
80.01	0.00503718593935376\\
81.01	0.00503719038135941\\
82.01	0.00503719491117427\\
83.01	0.00503719953051524\\
84.01	0.00503720424113244\\
85.01	0.00503720904480873\\
86.01	0.00503721394336243\\
87.01	0.00503721893864667\\
88.01	0.00503722403254947\\
89.01	0.00503722922699576\\
90.01	0.00503723452394766\\
91.01	0.00503723992540453\\
92.01	0.00503724543340422\\
93.01	0.00503725105002359\\
94.01	0.00503725677737982\\
95.01	0.00503726261763057\\
96.01	0.00503726857297443\\
97.01	0.0050372746456524\\
98.01	0.00503728083794855\\
99.01	0.00503728715219068\\
100.01	0.00503729359075072\\
101.01	0.00503730015604673\\
102.01	0.00503730685054238\\
103.01	0.00503731367674878\\
104.01	0.00503732063722473\\
105.01	0.00503732773457818\\
106.01	0.00503733497146674\\
107.01	0.00503734235059897\\
108.01	0.00503734987473498\\
109.01	0.00503735754668719\\
110.01	0.00503736536932195\\
111.01	0.00503737334556062\\
112.01	0.00503738147837934\\
113.01	0.00503738977081139\\
114.01	0.00503739822594776\\
115.01	0.00503740684693799\\
116.01	0.00503741563699175\\
117.01	0.00503742459937948\\
118.01	0.00503743373743401\\
119.01	0.00503744305455114\\
120.01	0.00503745255419133\\
121.01	0.00503746223988039\\
122.01	0.00503747211521105\\
123.01	0.00503748218384462\\
124.01	0.00503749244951107\\
125.01	0.00503750291601124\\
126.01	0.0050375135872181\\
127.01	0.00503752446707711\\
128.01	0.00503753555960904\\
129.01	0.00503754686891023\\
130.01	0.00503755839915435\\
131.01	0.00503757015459461\\
132.01	0.00503758213956329\\
133.01	0.00503759435847503\\
134.01	0.00503760681582774\\
135.01	0.00503761951620343\\
136.01	0.00503763246427077\\
137.01	0.00503764566478632\\
138.01	0.00503765912259596\\
139.01	0.00503767284263604\\
140.01	0.00503768682993702\\
141.01	0.00503770108962251\\
142.01	0.0050377156269131\\
143.01	0.00503773044712685\\
144.01	0.00503774555568146\\
145.01	0.00503776095809616\\
146.01	0.00503777665999397\\
147.01	0.00503779266710264\\
148.01	0.00503780898525674\\
149.01	0.00503782562040073\\
150.01	0.00503784257858947\\
151.01	0.00503785986599095\\
152.01	0.00503787748888843\\
153.01	0.00503789545368261\\
154.01	0.00503791376689258\\
155.01	0.00503793243515976\\
156.01	0.00503795146524905\\
157.01	0.005037970864051\\
158.01	0.00503799063858401\\
159.01	0.00503801079599758\\
160.01	0.00503803134357367\\
161.01	0.00503805228872931\\
162.01	0.00503807363901934\\
163.01	0.00503809540213851\\
164.01	0.00503811758592421\\
165.01	0.00503814019835902\\
166.01	0.00503816324757281\\
167.01	0.00503818674184675\\
168.01	0.00503821068961455\\
169.01	0.00503823509946524\\
170.01	0.00503825998014745\\
171.01	0.00503828534057066\\
172.01	0.0050383111898083\\
173.01	0.00503833753710173\\
174.01	0.00503836439186212\\
175.01	0.00503839176367404\\
176.01	0.00503841966229843\\
177.01	0.00503844809767484\\
178.01	0.00503847707992691\\
179.01	0.00503850661936314\\
180.01	0.00503853672648163\\
181.01	0.00503856741197301\\
182.01	0.00503859868672335\\
183.01	0.00503863056181852\\
184.01	0.00503866304854753\\
185.01	0.00503869615840566\\
186.01	0.00503872990309845\\
187.01	0.00503876429454511\\
188.01	0.00503879934488299\\
189.01	0.00503883506647076\\
190.01	0.00503887147189222\\
191.01	0.00503890857396075\\
192.01	0.00503894638572358\\
193.01	0.00503898492046473\\
194.01	0.00503902419171071\\
195.01	0.00503906421323363\\
196.01	0.00503910499905608\\
197.01	0.00503914656345553\\
198.01	0.00503918892096808\\
199.01	0.00503923208639412\\
200.01	0.0050392760748024\\
201.01	0.0050393209015345\\
202.01	0.00503936658221077\\
203.01	0.00503941313273334\\
204.01	0.00503946056929269\\
205.01	0.0050395089083721\\
206.01	0.00503955816675239\\
207.01	0.00503960836151857\\
208.01	0.00503965951006308\\
209.01	0.0050397116300932\\
210.01	0.00503976473963538\\
211.01	0.00503981885704116\\
212.01	0.00503987400099298\\
213.01	0.00503993019050977\\
214.01	0.00503998744495318\\
215.01	0.00504004578403339\\
216.01	0.00504010522781573\\
217.01	0.00504016579672581\\
218.01	0.00504022751155711\\
219.01	0.00504029039347662\\
220.01	0.00504035446403217\\
221.01	0.00504041974515843\\
222.01	0.00504048625918432\\
223.01	0.00504055402883963\\
224.01	0.00504062307726214\\
225.01	0.00504069342800531\\
226.01	0.005040765105045\\
227.01	0.0050408381327873\\
228.01	0.00504091253607636\\
229.01	0.00504098834020186\\
230.01	0.0050410655709073\\
231.01	0.00504114425439822\\
232.01	0.00504122441734989\\
233.01	0.00504130608691561\\
234.01	0.0050413892907368\\
235.01	0.00504147405694988\\
236.01	0.00504156041419651\\
237.01	0.00504164839163246\\
238.01	0.00504173801893631\\
239.01	0.00504182932631966\\
240.01	0.00504192234453665\\
241.01	0.00504201710489405\\
242.01	0.00504211363926055\\
243.01	0.00504221198007862\\
244.01	0.00504231216037326\\
245.01	0.00504241421376425\\
246.01	0.00504251817447671\\
247.01	0.00504262407735151\\
248.01	0.00504273195785787\\
249.01	0.00504284185210524\\
250.01	0.00504295379685413\\
251.01	0.00504306782952962\\
252.01	0.00504318398823272\\
253.01	0.00504330231175437\\
254.01	0.00504342283958797\\
255.01	0.00504354561194312\\
256.01	0.00504367066975919\\
257.01	0.00504379805471941\\
258.01	0.00504392780926499\\
259.01	0.00504405997661062\\
260.01	0.00504419460075892\\
261.01	0.00504433172651703\\
262.01	0.00504447139951025\\
263.01	0.00504461366620026\\
264.01	0.00504475857390158\\
265.01	0.0050449061707972\\
266.01	0.00504505650595788\\
267.01	0.00504520962935875\\
268.01	0.00504536559189853\\
269.01	0.00504552444541745\\
270.01	0.00504568624271717\\
271.01	0.0050458510375806\\
272.01	0.00504601888479148\\
273.01	0.00504618984015631\\
274.01	0.00504636396052505\\
275.01	0.00504654130381252\\
276.01	0.00504672192902139\\
277.01	0.00504690589626572\\
278.01	0.00504709326679303\\
279.01	0.00504728410301017\\
280.01	0.0050474784685072\\
281.01	0.00504767642808349\\
282.01	0.00504787804777293\\
283.01	0.00504808339487145\\
284.01	0.00504829253796438\\
285.01	0.00504850554695443\\
286.01	0.00504872249309057\\
287.01	0.00504894344899795\\
288.01	0.00504916848870818\\
289.01	0.00504939768769\\
290.01	0.00504963112288198\\
291.01	0.00504986887272421\\
292.01	0.00505011101719227\\
293.01	0.00505035763783143\\
294.01	0.00505060881779185\\
295.01	0.00505086464186458\\
296.01	0.00505112519651726\\
297.01	0.00505139056993306\\
298.01	0.00505166085204839\\
299.01	0.00505193613459221\\
300.01	0.00505221651112603\\
301.01	0.00505250207708521\\
302.01	0.00505279292981986\\
303.01	0.00505308916863922\\
304.01	0.00505339089485271\\
305.01	0.00505369821181673\\
306.01	0.00505401122497884\\
307.01	0.00505433004192291\\
308.01	0.00505465477241729\\
309.01	0.00505498552846184\\
310.01	0.00505532242433601\\
311.01	0.00505566557664891\\
312.01	0.00505601510438797\\
313.01	0.00505637112897089\\
314.01	0.00505673377429641\\
315.01	0.00505710316679639\\
316.01	0.00505747943548865\\
317.01	0.0050578627120301\\
318.01	0.0050582531307706\\
319.01	0.00505865082880838\\
320.01	0.00505905594604403\\
321.01	0.00505946862523698\\
322.01	0.00505988901206109\\
323.01	0.00506031725516179\\
324.01	0.0050607535062126\\
325.01	0.00506119791997345\\
326.01	0.00506165065434799\\
327.01	0.00506211187044248\\
328.01	0.00506258173262414\\
329.01	0.00506306040858084\\
330.01	0.00506354806938109\\
331.01	0.00506404488953358\\
332.01	0.00506455104704839\\
333.01	0.00506506672349805\\
334.01	0.0050655921040795\\
335.01	0.00506612737767675\\
336.01	0.00506667273692421\\
337.01	0.00506722837827064\\
338.01	0.00506779450204467\\
339.01	0.00506837131252053\\
340.01	0.00506895901798665\\
341.01	0.00506955783081389\\
342.01	0.00507016796752677\\
343.01	0.00507078964887561\\
344.01	0.00507142309991152\\
345.01	0.00507206855006312\\
346.01	0.00507272623321661\\
347.01	0.00507339638779827\\
348.01	0.00507407925685907\\
349.01	0.00507477508816512\\
350.01	0.00507548413428993\\
351.01	0.00507620665271169\\
352.01	0.00507694290591373\\
353.01	0.00507769316149151\\
354.01	0.00507845769226323\\
355.01	0.0050792367763855\\
356.01	0.00508003069747492\\
357.01	0.00508083974473519\\
358.01	0.0050816642130905\\
359.01	0.00508250440332336\\
360.01	0.00508336062221926\\
361.01	0.00508423318271767\\
362.01	0.00508512240406883\\
363.01	0.00508602861199517\\
364.01	0.00508695213885965\\
365.01	0.00508789332383966\\
366.01	0.00508885251310515\\
367.01	0.00508983006000338\\
368.01	0.00509082632524624\\
369.01	0.00509184167710397\\
370.01	0.00509287649160212\\
371.01	0.00509393115272302\\
372.01	0.00509500605261015\\
373.01	0.00509610159177717\\
374.01	0.00509721817931821\\
375.01	0.00509835623312491\\
376.01	0.00509951618010449\\
377.01	0.00510069845640224\\
378.01	0.00510190350762779\\
379.01	0.00510313178908584\\
380.01	0.0051043837660103\\
381.01	0.00510565991380521\\
382.01	0.00510696071828762\\
383.01	0.00510828667593788\\
384.01	0.00510963829415476\\
385.01	0.00511101609151532\\
386.01	0.00511242059804085\\
387.01	0.00511385235546691\\
388.01	0.00511531191752215\\
389.01	0.00511679985020833\\
390.01	0.00511831673208837\\
391.01	0.0051198631545798\\
392.01	0.00512143972225235\\
393.01	0.00512304705313218\\
394.01	0.00512468577901079\\
395.01	0.00512635654575985\\
396.01	0.00512806001365004\\
397.01	0.00512979685767662\\
398.01	0.00513156776788948\\
399.01	0.0051333734497285\\
400.01	0.00513521462436362\\
401.01	0.00513709202904053\\
402.01	0.00513900641743134\\
403.01	0.00514095855999003\\
404.01	0.0051429492443123\\
405.01	0.00514497927550131\\
406.01	0.00514704947653821\\
407.01	0.00514916068865695\\
408.01	0.0051513137717242\\
409.01	0.00515350960462505\\
410.01	0.00515574908565231\\
411.01	0.005158033132902\\
412.01	0.00516036268467382\\
413.01	0.00516273869987566\\
414.01	0.00516516215843577\\
415.01	0.00516763406171819\\
416.01	0.00517015543294619\\
417.01	0.00517272731762952\\
418.01	0.00517535078399994\\
419.01	0.0051780269234516\\
420.01	0.00518075685099006\\
421.01	0.00518354170568777\\
422.01	0.00518638265114635\\
423.01	0.00518928087596993\\
424.01	0.00519223759424342\\
425.01	0.00519525404602364\\
426.01	0.00519833149783658\\
427.01	0.00520147124318873\\
428.01	0.00520467460308546\\
429.01	0.00520794292656374\\
430.01	0.00521127759123499\\
431.01	0.00521468000384159\\
432.01	0.00521815160082545\\
433.01	0.00522169384891116\\
434.01	0.00522530824570214\\
435.01	0.00522899632029247\\
436.01	0.00523275963389283\\
437.01	0.00523659978047139\\
438.01	0.00524051838741119\\
439.01	0.0052445171161831\\
440.01	0.00524859766303384\\
441.01	0.00525276175969047\\
442.01	0.00525701117408137\\
443.01	0.00526134771107234\\
444.01	0.00526577321321897\\
445.01	0.00527028956153447\\
446.01	0.00527489867627398\\
447.01	0.00527960251773282\\
448.01	0.00528440308706105\\
449.01	0.0052893024270928\\
450.01	0.00529430262319144\\
451.01	0.00529940580410992\\
452.01	0.00530461414286562\\
453.01	0.00530992985763248\\
454.01	0.00531535521264853\\
455.01	0.00532089251913995\\
456.01	0.00532654413626477\\
457.01	0.00533231247207272\\
458.01	0.00533819998448709\\
459.01	0.00534420918230508\\
460.01	0.00535034262622169\\
461.01	0.00535660292987603\\
462.01	0.00536299276092225\\
463.01	0.00536951484212522\\
464.01	0.00537617195248202\\
465.01	0.00538296692837173\\
466.01	0.00538990266473194\\
467.01	0.00539698211626374\\
468.01	0.00540420829866611\\
469.01	0.00541158428989791\\
470.01	0.00541911323147179\\
471.01	0.00542679832977569\\
472.01	0.00543464285742695\\
473.01	0.00544265015465428\\
474.01	0.00545082363071245\\
475.01	0.00545916676532751\\
476.01	0.00546768311017262\\
477.01	0.00547637629037634\\
478.01	0.00548525000606218\\
479.01	0.00549430803392156\\
480.01	0.00550355422881765\\
481.01	0.00551299252542421\\
482.01	0.00552262693989639\\
483.01	0.005532461571577\\
484.01	0.0055425006047365\\
485.01	0.00555274831034856\\
486.01	0.00556320904790121\\
487.01	0.0055738872672441\\
488.01	0.00558478751047235\\
489.01	0.00559591441384842\\
490.01	0.00560727270975982\\
491.01	0.00561886722871592\\
492.01	0.00563070290138235\\
493.01	0.00564278476065359\\
494.01	0.00565511794376293\\
495.01	0.00566770769443155\\
496.01	0.00568055936505529\\
497.01	0.00569367841892915\\
498.01	0.00570707043250933\\
499.01	0.00572074109771233\\
500.01	0.00573469622425185\\
501.01	0.00574894174201072\\
502.01	0.0057634837034497\\
503.01	0.00577832828605053\\
504.01	0.00579348179479285\\
505.01	0.00580895066466464\\
506.01	0.00582474146320375\\
507.01	0.00584086089306883\\
508.01	0.00585731579463925\\
509.01	0.00587411314864009\\
510.01	0.00589126007879083\\
511.01	0.00590876385447447\\
512.01	0.00592663189342303\\
513.01	0.00594487176441871\\
514.01	0.00596349119000116\\
515.01	0.00598249804918133\\
516.01	0.00600190038015398\\
517.01	0.00602170638300233\\
518.01	0.00604192442238996\\
519.01	0.00606256303023013\\
520.01	0.006083630908326\\
521.01	0.0061051369309701\\
522.01	0.0061270901474948\\
523.01	0.00614949978475927\\
524.01	0.00617237524956166\\
525.01	0.0061957261309602\\
526.01	0.00621956220248883\\
527.01	0.0062438934242455\\
528.01	0.00626872994483725\\
529.01	0.00629408210315595\\
530.01	0.0063199604299614\\
531.01	0.00634637564924387\\
532.01	0.00637333867933546\\
533.01	0.00640086063373574\\
534.01	0.00642895282161518\\
535.01	0.00645762674795358\\
536.01	0.00648689411326941\\
537.01	0.00651676681288722\\
538.01	0.00654725693568947\\
539.01	0.00657837676228909\\
540.01	0.0066101387625562\\
541.01	0.00664255559242306\\
542.01	0.00667564008988554\\
543.01	0.00670940527010898\\
544.01	0.00674386431953959\\
545.01	0.00677903058890984\\
546.01	0.00681491758501747\\
547.01	0.00685153896114556\\
548.01	0.00688890850597589\\
549.01	0.00692704013083688\\
550.01	0.00696594785510919\\
551.01	0.00700564578959692\\
552.01	0.00704614811765319\\
553.01	0.00708746907382994\\
554.01	0.00712962291979972\\
555.01	0.00717262391727598\\
556.01	0.00721648629763107\\
557.01	0.00726122422788934\\
558.01	0.00730685177274028\\
559.01	0.00735338285219186\\
560.01	0.00740083119445041\\
561.01	0.00744921028358358\\
562.01	0.00749853330148946\\
563.01	0.00754881306366298\\
564.01	0.0076000619482173\\
565.01	0.00765229181758725\\
566.01	0.00770551393231232\\
567.01	0.00775973885627156\\
568.01	0.00781497635272441\\
569.01	0.00787123527049889\\
570.01	0.00792852341966897\\
571.01	0.00798684743608042\\
572.01	0.0080462126341193\\
573.01	0.00810662284718149\\
574.01	0.0081680802554003\\
575.01	0.00823058520033459\\
576.01	0.00829413598651979\\
577.01	0.00835872867006437\\
578.01	0.00842435683484077\\
579.01	0.0084910113573111\\
580.01	0.00855868016166438\\
581.01	0.00862734796776696\\
582.01	0.00869699603548486\\
583.01	0.0087676019102873\\
584.01	0.00883913917675417\\
585.01	0.00891157722877423\\
586.01	0.00898488106795022\\
587.01	0.00905901114515074\\
588.01	0.00913392326443695\\
589.01	0.00920956857395242\\
590.01	0.00928589367504629\\
591.01	0.00936284088921878\\
592.01	0.0094403487328199\\
593.01	0.00951835266226869\\
594.01	0.00959678616847739\\
595.01	0.00967558231888201\\
596.01	0.00975467586988015\\
597.01	0.00983387214752878\\
598.01	0.0099086620184807\\
599.01	0.00997087280416276\\
599.02	0.00997138072163725\\
599.03	0.00997188557625799\\
599.04	0.00997238733820211\\
599.05	0.00997288597735272\\
599.06	0.00997338146329604\\
599.07	0.00997387376531845\\
599.08	0.00997436285240349\\
599.09	0.00997484869322892\\
599.1	0.00997533125616361\\
599.11	0.00997581050926455\\
599.12	0.00997628642027372\\
599.13	0.00997675895661497\\
599.14	0.0099772280853909\\
599.15	0.00997769377337961\\
599.16	0.00997815598703157\\
599.17	0.00997861469246631\\
599.18	0.00997906985546915\\
599.19	0.00997952144148792\\
599.2	0.00997996941562957\\
599.21	0.00998041374265684\\
599.22	0.0099808543869848\\
599.23	0.00998129131267744\\
599.24	0.00998172448344418\\
599.25	0.00998215386263636\\
599.26	0.00998257941258354\\
599.27	0.00998300109279825\\
599.28	0.00998341886238981\\
599.29	0.00998383268006024\\
599.3	0.00998424250410032\\
599.31	0.00998464829238543\\
599.32	0.00998505000237151\\
599.33	0.00998544759109087\\
599.34	0.00998584101514799\\
599.35	0.00998623023071532\\
599.36	0.00998661519352895\\
599.37	0.00998699585888436\\
599.38	0.00998737218163197\\
599.39	0.00998774411617281\\
599.4	0.00998811161645403\\
599.41	0.00998847463596443\\
599.42	0.0099888331277299\\
599.43	0.00998918704430885\\
599.44	0.00998953633778757\\
599.45	0.00998988095977558\\
599.46	0.00999022086140088\\
599.47	0.00999055599330522\\
599.48	0.00999088630563925\\
599.49	0.00999121174805768\\
599.5	0.00999153226971435\\
599.51	0.0099918478192573\\
599.52	0.00999215834482374\\
599.53	0.00999246379403503\\
599.54	0.0099927641139915\\
599.55	0.00999305925126738\\
599.56	0.00999334915190553\\
599.57	0.0099936337614122\\
599.58	0.00999391302475173\\
599.59	0.00999418688634118\\
599.6	0.00999445529004489\\
599.61	0.00999471817916904\\
599.62	0.00999497549645613\\
599.63	0.00999522718407935\\
599.64	0.009995473183637\\
599.65	0.00999571343614678\\
599.66	0.00999594788204004\\
599.67	0.00999617646115598\\
599.68	0.0099963991127358\\
599.69	0.00999661577541675\\
599.7	0.00999682638722618\\
599.71	0.00999703088557551\\
599.72	0.00999722920725411\\
599.73	0.00999742128842315\\
599.74	0.00999760706460942\\
599.75	0.00999778647069897\\
599.76	0.00999795944093086\\
599.77	0.0099981259088907\\
599.78	0.0099982858075042\\
599.79	0.00999843906903064\\
599.8	0.00999858562505627\\
599.81	0.00999872540648766\\
599.82	0.00999885834354498\\
599.83	0.00999898436575515\\
599.84	0.00999910340194508\\
599.85	0.00999921538023465\\
599.86	0.00999932022802977\\
599.87	0.00999941787201528\\
599.88	0.00999950823814785\\
599.89	0.00999959125164875\\
599.9	0.00999966683699656\\
599.91	0.00999973491791987\\
599.92	0.0099997954173898\\
599.93	0.00999984825761255\\
599.94	0.00999989336002181\\
599.95	0.00999993064527112\\
599.96	0.00999996003322615\\
599.97	0.00999998144295691\\
599.98	0.00999999479272987\\
599.99	0.01\\
600	0.01\\
};
\addplot [color=mycolor1,solid,forget plot]
  table[row sep=crcr]{%
0.01	0.00503201950426191\\
1.01	0.00503202043624351\\
2.01	0.00503202138695444\\
3.01	0.00503202235676875\\
4.01	0.00503202334606778\\
5.01	0.00503202435524097\\
6.01	0.00503202538468486\\
7.01	0.00503202643480416\\
8.01	0.0050320275060117\\
9.01	0.00503202859872831\\
10.01	0.00503202971338315\\
11.01	0.00503203085041398\\
12.01	0.00503203201026672\\
13.01	0.00503203319339726\\
14.01	0.005032034400269\\
15.01	0.00503203563135604\\
16.01	0.00503203688714068\\
17.01	0.00503203816811504\\
18.01	0.00503203947478162\\
19.01	0.00503204080765182\\
20.01	0.00503204216724792\\
21.01	0.00503204355410197\\
22.01	0.00503204496875709\\
23.01	0.00503204641176672\\
24.01	0.00503204788369532\\
25.01	0.00503204938511864\\
26.01	0.00503205091662345\\
27.01	0.00503205247880859\\
28.01	0.00503205407228416\\
29.01	0.00503205569767265\\
30.01	0.00503205735560877\\
31.01	0.0050320590467394\\
32.01	0.00503206077172466\\
33.01	0.00503206253123722\\
34.01	0.00503206432596379\\
35.01	0.00503206615660365\\
36.01	0.00503206802387021\\
37.01	0.00503206992849109\\
38.01	0.0050320718712078\\
39.01	0.00503207385277674\\
40.01	0.00503207587396953\\
41.01	0.00503207793557234\\
42.01	0.0050320800383872\\
43.01	0.00503208218323159\\
44.01	0.00503208437093899\\
45.01	0.00503208660235979\\
46.01	0.00503208887836051\\
47.01	0.00503209119982492\\
48.01	0.0050320935676543\\
49.01	0.00503209598276713\\
50.01	0.00503209844610009\\
51.01	0.00503210095860906\\
52.01	0.00503210352126715\\
53.01	0.00503210613506777\\
54.01	0.00503210880102339\\
55.01	0.00503211152016583\\
56.01	0.00503211429354786\\
57.01	0.00503211712224252\\
58.01	0.00503212000734368\\
59.01	0.0050321229499671\\
60.01	0.00503212595124972\\
61.01	0.00503212901235128\\
62.01	0.00503213213445397\\
63.01	0.00503213531876315\\
64.01	0.00503213856650761\\
65.01	0.00503214187894002\\
66.01	0.00503214525733772\\
67.01	0.00503214870300306\\
68.01	0.00503215221726366\\
69.01	0.00503215580147326\\
70.01	0.00503215945701149\\
71.01	0.00503216318528557\\
72.01	0.00503216698772946\\
73.01	0.00503217086580512\\
74.01	0.00503217482100379\\
75.01	0.005032178854845\\
76.01	0.00503218296887792\\
77.01	0.00503218716468218\\
78.01	0.00503219144386768\\
79.01	0.00503219580807637\\
80.01	0.00503220025898121\\
81.01	0.00503220479828845\\
82.01	0.00503220942773705\\
83.01	0.00503221414910002\\
84.01	0.00503221896418446\\
85.01	0.00503222387483279\\
86.01	0.00503222888292315\\
87.01	0.00503223399036967\\
88.01	0.00503223919912457\\
89.01	0.00503224451117715\\
90.01	0.00503224992855547\\
91.01	0.00503225545332672\\
92.01	0.0050322610875983\\
93.01	0.00503226683351865\\
94.01	0.00503227269327697\\
95.01	0.00503227866910558\\
96.01	0.00503228476327977\\
97.01	0.00503229097811875\\
98.01	0.00503229731598653\\
99.01	0.00503230377929245\\
100.01	0.00503231037049318\\
101.01	0.00503231709209198\\
102.01	0.0050323239466408\\
103.01	0.00503233093674088\\
104.01	0.00503233806504341\\
105.01	0.00503234533425098\\
106.01	0.00503235274711802\\
107.01	0.00503236030645168\\
108.01	0.00503236801511339\\
109.01	0.00503237587602001\\
110.01	0.00503238389214397\\
111.01	0.00503239206651512\\
112.01	0.00503240040222153\\
113.01	0.00503240890241055\\
114.01	0.00503241757029019\\
115.01	0.00503242640912957\\
116.01	0.00503243542226084\\
117.01	0.00503244461308017\\
118.01	0.00503245398504823\\
119.01	0.00503246354169318\\
120.01	0.005032473286609\\
121.01	0.00503248322346042\\
122.01	0.00503249335598105\\
123.01	0.00503250368797589\\
124.01	0.00503251422332365\\
125.01	0.00503252496597636\\
126.01	0.00503253591996161\\
127.01	0.00503254708938419\\
128.01	0.00503255847842715\\
129.01	0.0050325700913533\\
130.01	0.00503258193250708\\
131.01	0.00503259400631453\\
132.01	0.00503260631728797\\
133.01	0.00503261887002382\\
134.01	0.00503263166920667\\
135.01	0.00503264471961051\\
136.01	0.00503265802610012\\
137.01	0.00503267159363238\\
138.01	0.00503268542725815\\
139.01	0.00503269953212504\\
140.01	0.00503271391347715\\
141.01	0.00503272857665937\\
142.01	0.00503274352711694\\
143.01	0.00503275877039866\\
144.01	0.00503277431215837\\
145.01	0.005032790158157\\
146.01	0.00503280631426428\\
147.01	0.00503282278646114\\
148.01	0.00503283958084161\\
149.01	0.00503285670361445\\
150.01	0.00503287416110612\\
151.01	0.00503289195976253\\
152.01	0.0050329101061502\\
153.01	0.00503292860696047\\
154.01	0.00503294746901059\\
155.01	0.00503296669924563\\
156.01	0.00503298630474189\\
157.01	0.0050330062927082\\
158.01	0.00503302667048941\\
159.01	0.0050330474455681\\
160.01	0.00503306862556732\\
161.01	0.00503309021825343\\
162.01	0.00503311223153793\\
163.01	0.00503313467348104\\
164.01	0.00503315755229414\\
165.01	0.00503318087634153\\
166.01	0.00503320465414483\\
167.01	0.00503322889438429\\
168.01	0.00503325360590273\\
169.01	0.00503327879770787\\
170.01	0.00503330447897571\\
171.01	0.0050333306590534\\
172.01	0.00503335734746257\\
173.01	0.00503338455390173\\
174.01	0.00503341228824994\\
175.01	0.00503344056057023\\
176.01	0.00503346938111257\\
177.01	0.00503349876031783\\
178.01	0.00503352870881993\\
179.01	0.00503355923745083\\
180.01	0.0050335903572428\\
181.01	0.00503362207943275\\
182.01	0.0050336544154656\\
183.01	0.00503368737699827\\
184.01	0.00503372097590274\\
185.01	0.00503375522427091\\
186.01	0.00503379013441749\\
187.01	0.00503382571888486\\
188.01	0.00503386199044611\\
189.01	0.00503389896211015\\
190.01	0.00503393664712548\\
191.01	0.00503397505898429\\
192.01	0.00503401421142635\\
193.01	0.00503405411844454\\
194.01	0.00503409479428832\\
195.01	0.00503413625346871\\
196.01	0.00503417851076279\\
197.01	0.00503422158121802\\
198.01	0.00503426548015751\\
199.01	0.00503431022318523\\
200.01	0.00503435582618919\\
201.01	0.00503440230534838\\
202.01	0.00503444967713718\\
203.01	0.00503449795832997\\
204.01	0.00503454716600721\\
205.01	0.00503459731756009\\
206.01	0.00503464843069654\\
207.01	0.00503470052344598\\
208.01	0.00503475361416586\\
209.01	0.00503480772154641\\
210.01	0.00503486286461741\\
211.01	0.00503491906275336\\
212.01	0.00503497633567945\\
213.01	0.00503503470347832\\
214.01	0.00503509418659531\\
215.01	0.0050351548058459\\
216.01	0.00503521658242043\\
217.01	0.00503527953789218\\
218.01	0.00503534369422357\\
219.01	0.00503540907377234\\
220.01	0.00503547569929882\\
221.01	0.00503554359397246\\
222.01	0.00503561278137921\\
223.01	0.00503568328552928\\
224.01	0.00503575513086363\\
225.01	0.00503582834226052\\
226.01	0.00503590294504541\\
227.01	0.00503597896499697\\
228.01	0.00503605642835475\\
229.01	0.00503613536182793\\
230.01	0.0050362157926031\\
231.01	0.00503629774835188\\
232.01	0.00503638125724035\\
233.01	0.00503646634793695\\
234.01	0.00503655304962039\\
235.01	0.00503664139199012\\
236.01	0.00503673140527389\\
237.01	0.00503682312023715\\
238.01	0.00503691656819238\\
239.01	0.00503701178100915\\
240.01	0.00503710879112251\\
241.01	0.00503720763154361\\
242.01	0.00503730833586963\\
243.01	0.00503741093829291\\
244.01	0.00503751547361275\\
245.01	0.00503762197724492\\
246.01	0.00503773048523254\\
247.01	0.00503784103425756\\
248.01	0.00503795366165097\\
249.01	0.0050380684054043\\
250.01	0.00503818530418168\\
251.01	0.00503830439733127\\
252.01	0.00503842572489746\\
253.01	0.00503854932763265\\
254.01	0.00503867524701027\\
255.01	0.00503880352523717\\
256.01	0.00503893420526672\\
257.01	0.00503906733081197\\
258.01	0.0050392029463599\\
259.01	0.00503934109718421\\
260.01	0.00503948182936034\\
261.01	0.00503962518977843\\
262.01	0.00503977122616076\\
263.01	0.00503991998707443\\
264.01	0.00504007152194746\\
265.01	0.0050402258810856\\
266.01	0.0050403831156868\\
267.01	0.0050405432778593\\
268.01	0.00504070642063711\\
269.01	0.00504087259799862\\
270.01	0.0050410418648838\\
271.01	0.00504121427721224\\
272.01	0.00504138989190245\\
273.01	0.0050415687668901\\
274.01	0.00504175096114795\\
275.01	0.00504193653470614\\
276.01	0.00504212554867273\\
277.01	0.00504231806525504\\
278.01	0.00504251414778109\\
279.01	0.00504271386072164\\
280.01	0.00504291726971426\\
281.01	0.00504312444158475\\
282.01	0.0050433354443741\\
283.01	0.00504355034736101\\
284.01	0.00504376922108885\\
285.01	0.00504399213739071\\
286.01	0.0050442191694183\\
287.01	0.00504445039166762\\
288.01	0.0050446858800087\\
289.01	0.00504492571171544\\
290.01	0.00504516996549426\\
291.01	0.00504541872151704\\
292.01	0.00504567206145204\\
293.01	0.00504593006849774\\
294.01	0.00504619282741553\\
295.01	0.00504646042456572\\
296.01	0.00504673294794333\\
297.01	0.00504701048721468\\
298.01	0.00504729313375593\\
299.01	0.00504758098069216\\
300.01	0.00504787412293708\\
301.01	0.00504817265723545\\
302.01	0.00504847668220459\\
303.01	0.00504878629837929\\
304.01	0.00504910160825628\\
305.01	0.0050494227163401\\
306.01	0.00504974972919078\\
307.01	0.00505008275547381\\
308.01	0.00505042190600799\\
309.01	0.00505076729381831\\
310.01	0.00505111903418812\\
311.01	0.0050514772447123\\
312.01	0.00505184204535378\\
313.01	0.00505221355849816\\
314.01	0.00505259190901264\\
315.01	0.00505297722430356\\
316.01	0.00505336963437772\\
317.01	0.00505376927190221\\
318.01	0.00505417627226815\\
319.01	0.00505459077365233\\
320.01	0.0050550129170826\\
321.01	0.00505544284650278\\
322.01	0.00505588070883858\\
323.01	0.00505632665406469\\
324.01	0.00505678083527231\\
325.01	0.00505724340873737\\
326.01	0.00505771453398907\\
327.01	0.00505819437387861\\
328.01	0.00505868309464988\\
329.01	0.00505918086600726\\
330.01	0.00505968786118643\\
331.01	0.00506020425702283\\
332.01	0.00506073023402123\\
333.01	0.00506126597642489\\
334.01	0.0050618116722832\\
335.01	0.00506236751351999\\
336.01	0.00506293369599944\\
337.01	0.0050635104195933\\
338.01	0.00506409788824524\\
339.01	0.00506469631003512\\
340.01	0.00506530589724219\\
341.01	0.00506592686640633\\
342.01	0.00506655943838927\\
343.01	0.00506720383843395\\
344.01	0.00506786029622303\\
345.01	0.00506852904593738\\
346.01	0.00506921032631205\\
347.01	0.00506990438069319\\
348.01	0.00507061145709439\\
349.01	0.00507133180825325\\
350.01	0.00507206569168783\\
351.01	0.0050728133697556\\
352.01	0.00507357510971221\\
353.01	0.0050743511837737\\
354.01	0.00507514186918129\\
355.01	0.0050759474482705\\
356.01	0.00507676820854269\\
357.01	0.00507760444274413\\
358.01	0.00507845644895008\\
359.01	0.00507932453065577\\
360.01	0.00508020899687583\\
361.01	0.00508111016225213\\
362.01	0.0050820283471692\\
363.01	0.0050829638778836\\
364.01	0.00508391708665953\\
365.01	0.00508488831191731\\
366.01	0.00508587789839304\\
367.01	0.00508688619730879\\
368.01	0.00508791356655353\\
369.01	0.00508896037087408\\
370.01	0.00509002698207614\\
371.01	0.00509111377923371\\
372.01	0.00509222114890623\\
373.01	0.00509334948536214\\
374.01	0.00509449919080782\\
375.01	0.00509567067562119\\
376.01	0.00509686435858675\\
377.01	0.00509808066713371\\
378.01	0.00509932003757677\\
379.01	0.00510058291535667\\
380.01	0.00510186975528346\\
381.01	0.00510318102178189\\
382.01	0.00510451718913998\\
383.01	0.00510587874176205\\
384.01	0.00510726617442703\\
385.01	0.00510867999255332\\
386.01	0.0051101207124698\\
387.01	0.00511158886169583\\
388.01	0.00511308497922627\\
389.01	0.00511460961582658\\
390.01	0.00511616333433346\\
391.01	0.00511774670996351\\
392.01	0.00511936033063096\\
393.01	0.00512100479727096\\
394.01	0.00512268072417241\\
395.01	0.00512438873931648\\
396.01	0.00512612948472518\\
397.01	0.0051279036168155\\
398.01	0.00512971180676239\\
399.01	0.00513155474086818\\
400.01	0.00513343312093995\\
401.01	0.00513534766467416\\
402.01	0.0051372991060481\\
403.01	0.00513928819571818\\
404.01	0.00514131570142495\\
405.01	0.00514338240840512\\
406.01	0.00514548911980908\\
407.01	0.00514763665712437\\
408.01	0.00514982586060625\\
409.01	0.00515205758971196\\
410.01	0.00515433272354165\\
411.01	0.00515665216128347\\
412.01	0.00515901682266326\\
413.01	0.00516142764839961\\
414.01	0.00516388560066147\\
415.01	0.00516639166353125\\
416.01	0.00516894684347027\\
417.01	0.00517155216978837\\
418.01	0.0051742086951167\\
419.01	0.00517691749588425\\
420.01	0.00517967967279609\\
421.01	0.00518249635131593\\
422.01	0.00518536868215205\\
423.01	0.00518829784174599\\
424.01	0.00519128503276489\\
425.01	0.00519433148459841\\
426.01	0.00519743845385992\\
427.01	0.00520060722489213\\
428.01	0.00520383911027941\\
429.01	0.00520713545136559\\
430.01	0.00521049761877979\\
431.01	0.0052139270129701\\
432.01	0.00521742506474721\\
433.01	0.00522099323583805\\
434.01	0.00522463301945253\\
435.01	0.00522834594086097\\
436.01	0.00523213355798921\\
437.01	0.00523599746202724\\
438.01	0.0052399392780561\\
439.01	0.0052439606656921\\
440.01	0.00524806331975258\\
441.01	0.00525224897094034\\
442.01	0.00525651938654981\\
443.01	0.00526087637119615\\
444.01	0.00526532176756633\\
445.01	0.00526985745719233\\
446.01	0.00527448536124653\\
447.01	0.00527920744135943\\
448.01	0.00528402570045861\\
449.01	0.00528894218362613\\
450.01	0.00529395897897561\\
451.01	0.00529907821854578\\
452.01	0.00530430207920979\\
453.01	0.00530963278359722\\
454.01	0.00531507260102934\\
455.01	0.00532062384846525\\
456.01	0.00532628889145623\\
457.01	0.00533207014511108\\
458.01	0.00533797007507009\\
459.01	0.00534399119848931\\
460.01	0.0053501360850358\\
461.01	0.0053564073578961\\
462.01	0.00536280769480051\\
463.01	0.00536933982906571\\
464.01	0.0053760065506588\\
465.01	0.00538281070728559\\
466.01	0.00538975520550605\\
467.01	0.00539684301187904\\
468.01	0.00540407715413795\\
469.01	0.00541146072239986\\
470.01	0.00541899687040678\\
471.01	0.00542668881680251\\
472.01	0.00543453984644367\\
473.01	0.0054425533117462\\
474.01	0.00545073263406792\\
475.01	0.00545908130512543\\
476.01	0.00546760288844794\\
477.01	0.00547630102086585\\
478.01	0.0054851794140342\\
479.01	0.00549424185599155\\
480.01	0.0055034922127532\\
481.01	0.00551293442993825\\
482.01	0.00552257253443218\\
483.01	0.00553241063608223\\
484.01	0.00554245292942729\\
485.01	0.00555270369546343\\
486.01	0.00556316730344353\\
487.01	0.00557384821271302\\
488.01	0.00558475097458304\\
489.01	0.00559588023423913\\
490.01	0.00560724073269073\\
491.01	0.00561883730875751\\
492.01	0.00563067490109745\\
493.01	0.00564275855027375\\
494.01	0.00565509340086368\\
495.01	0.00566768470360667\\
496.01	0.00568053781759384\\
497.01	0.00569365821249661\\
498.01	0.00570705147083497\\
499.01	0.00572072329028427\\
500.01	0.00573467948601941\\
501.01	0.00574892599309709\\
502.01	0.00576346886887259\\
503.01	0.00577831429545238\\
504.01	0.00579346858218174\\
505.01	0.0058089381681635\\
506.01	0.00582472962480907\\
507.01	0.00584084965842006\\
508.01	0.00585730511279649\\
509.01	0.00587410297187216\\
510.01	0.00589125036237267\\
511.01	0.00590875455649485\\
512.01	0.00592662297460268\\
513.01	0.00594486318793586\\
514.01	0.00596348292132978\\
515.01	0.0059824900559373\\
516.01	0.00600189263195081\\
517.01	0.00602169885131572\\
518.01	0.00604191708043086\\
519.01	0.00606255585282599\\
520.01	0.00608362387180911\\
521.01	0.00610513001307378\\
522.01	0.00612708332725578\\
523.01	0.00614949304242679\\
524.01	0.00617236856651236\\
525.01	0.006195719489619\\
526.01	0.00621955558625398\\
527.01	0.00624388681741989\\
528.01	0.00626872333256309\\
529.01	0.00629407547135422\\
530.01	0.00631995376527479\\
531.01	0.00634636893898359\\
532.01	0.00637333191143073\\
533.01	0.00640085379668688\\
534.01	0.00642894590444845\\
535.01	0.00645761974017953\\
536.01	0.00648688700484237\\
537.01	0.00651675959416793\\
538.01	0.00654724959740859\\
539.01	0.00657836929551299\\
540.01	0.00661013115865439\\
541.01	0.00664254784303731\\
542.01	0.0066756321869002\\
543.01	0.00670939720562297\\
544.01	0.00674385608583911\\
545.01	0.00677902217844316\\
546.01	0.00681490899037128\\
547.01	0.00685153017502185\\
548.01	0.00688889952117103\\
549.01	0.00692703094022179\\
550.01	0.00696593845161156\\
551.01	0.00700563616618453\\
552.01	0.00704613826731923\\
553.01	0.00708745898958019\\
554.01	0.00712961259464204\\
555.01	0.00717261334421132\\
556.01	0.0072164754696476\\
557.01	0.00726121313795847\\
558.01	0.00730684041381582\\
559.01	0.0073533712172122\\
560.01	0.00740081927634337\\
561.01	0.00744919807527555\\
562.01	0.00749852079591776\\
563.01	0.00754880025379286\\
564.01	0.00760004882706339\\
565.01	0.00765227837824005\\
566.01	0.0077055001679697\\
567.01	0.00775972476027739\\
568.01	0.00781496191861332\\
569.01	0.00787122049204771\\
570.01	0.00792850829095721\\
571.01	0.00798683195155923\\
572.01	0.00804619678868979\\
573.01	0.00810660663628399\\
574.01	0.00816806367511503\\
575.01	0.0082305682474943\\
576.01	0.00829411865883634\\
577.01	0.00835871096626884\\
578.01	0.00842433875483839\\
579.01	0.00849099290235135\\
580.01	0.00855866133452602\\
581.01	0.008627328772957\\
582.01	0.00869697647944997\\
583.01	0.00876758200163454\\
584.01	0.00883911892647557\\
585.01	0.00891155665046791\\
586.01	0.0089848601780259\\
587.01	0.00905898996300263\\
588.01	0.00913390181256045\\
589.01	0.00920954687797058\\
590.01	0.00928587176359865\\
591.01	0.0093628187936477\\
592.01	0.00944032648656475\\
593.01	0.00951833029984481\\
594.01	0.00959676372387219\\
595.01	0.00967555982313978\\
596.01	0.00975465334756447\\
597.01	0.00983385942577576\\
598.01	0.00990866201848071\\
599.01	0.00997087280416276\\
599.02	0.00997138072163725\\
599.03	0.009971885576258\\
599.04	0.00997238733820211\\
599.05	0.00997288597735272\\
599.06	0.00997338146329604\\
599.07	0.00997387376531845\\
599.08	0.00997436285240349\\
599.09	0.00997484869322892\\
599.1	0.00997533125616361\\
599.11	0.00997581050926455\\
599.12	0.00997628642027372\\
599.13	0.00997675895661498\\
599.14	0.0099772280853909\\
599.15	0.00997769377337961\\
599.16	0.00997815598703157\\
599.17	0.00997861469246631\\
599.18	0.00997906985546915\\
599.19	0.00997952144148792\\
599.2	0.00997996941562957\\
599.21	0.00998041374265684\\
599.22	0.0099808543869848\\
599.23	0.00998129131267744\\
599.24	0.00998172448344418\\
599.25	0.00998215386263636\\
599.26	0.00998257941258353\\
599.27	0.00998300109279825\\
599.28	0.00998341886238981\\
599.29	0.00998383268006024\\
599.3	0.00998424250410032\\
599.31	0.00998464829238543\\
599.32	0.00998505000237151\\
599.33	0.00998544759109087\\
599.34	0.00998584101514799\\
599.35	0.00998623023071532\\
599.36	0.00998661519352896\\
599.37	0.00998699585888436\\
599.38	0.00998737218163197\\
599.39	0.00998774411617281\\
599.4	0.00998811161645403\\
599.41	0.00998847463596443\\
599.42	0.0099888331277299\\
599.43	0.00998918704430885\\
599.44	0.00998953633778757\\
599.45	0.00998988095977558\\
599.46	0.00999022086140088\\
599.47	0.00999055599330522\\
599.48	0.00999088630563925\\
599.49	0.00999121174805768\\
599.5	0.00999153226971435\\
599.51	0.0099918478192573\\
599.52	0.00999215834482374\\
599.53	0.00999246379403503\\
599.54	0.0099927641139915\\
599.55	0.00999305925126738\\
599.56	0.00999334915190553\\
599.57	0.0099936337614122\\
599.58	0.00999391302475173\\
599.59	0.00999418688634118\\
599.6	0.00999445529004489\\
599.61	0.00999471817916904\\
599.62	0.00999497549645613\\
599.63	0.00999522718407934\\
599.64	0.009995473183637\\
599.65	0.00999571343614678\\
599.66	0.00999594788204004\\
599.67	0.00999617646115599\\
599.68	0.0099963991127358\\
599.69	0.00999661577541675\\
599.7	0.00999682638722618\\
599.71	0.00999703088557551\\
599.72	0.00999722920725411\\
599.73	0.00999742128842315\\
599.74	0.00999760706460942\\
599.75	0.00999778647069897\\
599.76	0.00999795944093086\\
599.77	0.0099981259088907\\
599.78	0.0099982858075042\\
599.79	0.00999843906903064\\
599.8	0.00999858562505627\\
599.81	0.00999872540648767\\
599.82	0.00999885834354498\\
599.83	0.00999898436575515\\
599.84	0.00999910340194508\\
599.85	0.00999921538023465\\
599.86	0.00999932022802977\\
599.87	0.00999941787201528\\
599.88	0.00999950823814785\\
599.89	0.00999959125164875\\
599.9	0.00999966683699656\\
599.91	0.00999973491791987\\
599.92	0.0099997954173898\\
599.93	0.00999984825761255\\
599.94	0.00999989336002181\\
599.95	0.00999993064527112\\
599.96	0.00999996003322615\\
599.97	0.00999998144295691\\
599.98	0.00999999479272987\\
599.99	0.01\\
600	0.01\\
};
\addplot [color=mycolor2,solid,forget plot]
  table[row sep=crcr]{%
0.01	0.00502120505551015\\
1.01	0.0050212060149824\\
2.01	0.00502120699383299\\
3.01	0.00502120799245165\\
4.01	0.00502120901123571\\
5.01	0.00502121005059042\\
6.01	0.00502121111092924\\
7.01	0.00502121219267379\\
8.01	0.00502121329625405\\
9.01	0.00502121442210887\\
10.01	0.00502121557068572\\
11.01	0.00502121674244047\\
12.01	0.00502121793783911\\
13.01	0.0050212191573559\\
14.01	0.00502122040147554\\
15.01	0.00502122167069138\\
16.01	0.00502122296550742\\
17.01	0.00502122428643781\\
18.01	0.00502122563400595\\
19.01	0.0050212270087466\\
20.01	0.00502122841120478\\
21.01	0.00502122984193676\\
22.01	0.00502123130150919\\
23.01	0.00502123279050046\\
24.01	0.00502123430950069\\
25.01	0.00502123585911135\\
26.01	0.00502123743994609\\
27.01	0.00502123905263077\\
28.01	0.00502124069780355\\
29.01	0.00502124237611543\\
30.01	0.00502124408823032\\
31.01	0.00502124583482541\\
32.01	0.00502124761659137\\
33.01	0.00502124943423254\\
34.01	0.0050212512884672\\
35.01	0.00502125318002777\\
36.01	0.00502125510966178\\
37.01	0.0050212570781312\\
38.01	0.00502125908621306\\
39.01	0.00502126113470028\\
40.01	0.00502126322440086\\
41.01	0.00502126535613939\\
42.01	0.00502126753075657\\
43.01	0.00502126974910991\\
44.01	0.00502127201207395\\
45.01	0.00502127432054039\\
46.01	0.00502127667541896\\
47.01	0.00502127907763728\\
48.01	0.00502128152814088\\
49.01	0.0050212840278949\\
50.01	0.00502128657788314\\
51.01	0.00502128917910871\\
52.01	0.00502129183259509\\
53.01	0.0050212945393856\\
54.01	0.00502129730054438\\
55.01	0.00502130011715682\\
56.01	0.00502130299032957\\
57.01	0.00502130592119115\\
58.01	0.00502130891089264\\
59.01	0.00502131196060768\\
60.01	0.00502131507153325\\
61.01	0.00502131824489013\\
62.01	0.00502132148192286\\
63.01	0.00502132478390088\\
64.01	0.00502132815211859\\
65.01	0.00502133158789611\\
66.01	0.00502133509257949\\
67.01	0.00502133866754131\\
68.01	0.00502134231418158\\
69.01	0.00502134603392737\\
70.01	0.0050213498282346\\
71.01	0.00502135369858719\\
72.01	0.00502135764649908\\
73.01	0.00502136167351383\\
74.01	0.00502136578120502\\
75.01	0.00502136997117758\\
76.01	0.00502137424506807\\
77.01	0.00502137860454521\\
78.01	0.00502138305131081\\
79.01	0.00502138758709988\\
80.01	0.00502139221368189\\
81.01	0.00502139693286069\\
82.01	0.00502140174647622\\
83.01	0.0050214066564041\\
84.01	0.00502141166455702\\
85.01	0.00502141677288556\\
86.01	0.00502142198337838\\
87.01	0.00502142729806312\\
88.01	0.00502143271900729\\
89.01	0.00502143824831927\\
90.01	0.00502144388814851\\
91.01	0.00502144964068685\\
92.01	0.00502145550816918\\
93.01	0.00502146149287397\\
94.01	0.00502146759712493\\
95.01	0.00502147382329066\\
96.01	0.00502148017378673\\
97.01	0.00502148665107566\\
98.01	0.00502149325766843\\
99.01	0.00502149999612531\\
100.01	0.00502150686905647\\
101.01	0.0050215138791232\\
102.01	0.00502152102903899\\
103.01	0.00502152832157038\\
104.01	0.00502153575953804\\
105.01	0.00502154334581734\\
106.01	0.00502155108334049\\
107.01	0.00502155897509676\\
108.01	0.00502156702413392\\
109.01	0.00502157523355901\\
110.01	0.00502158360653984\\
111.01	0.00502159214630588\\
112.01	0.00502160085614986\\
113.01	0.00502160973942875\\
114.01	0.00502161879956461\\
115.01	0.00502162804004638\\
116.01	0.0050216374644311\\
117.01	0.00502164707634492\\
118.01	0.00502165687948466\\
119.01	0.00502166687761887\\
120.01	0.00502167707459003\\
121.01	0.00502168747431439\\
122.01	0.00502169808078512\\
123.01	0.00502170889807277\\
124.01	0.00502171993032674\\
125.01	0.00502173118177726\\
126.01	0.00502174265673663\\
127.01	0.00502175435960078\\
128.01	0.00502176629485096\\
129.01	0.00502177846705515\\
130.01	0.00502179088086999\\
131.01	0.0050218035410424\\
132.01	0.00502181645241076\\
133.01	0.00502182961990796\\
134.01	0.00502184304856201\\
135.01	0.00502185674349784\\
136.01	0.00502187070993954\\
137.01	0.00502188495321224\\
138.01	0.00502189947874446\\
139.01	0.00502191429206877\\
140.01	0.00502192939882506\\
141.01	0.00502194480476151\\
142.01	0.00502196051573778\\
143.01	0.0050219765377259\\
144.01	0.00502199287681351\\
145.01	0.00502200953920504\\
146.01	0.00502202653122452\\
147.01	0.00502204385931768\\
148.01	0.00502206153005422\\
149.01	0.00502207955013015\\
150.01	0.00502209792636996\\
151.01	0.0050221166657293\\
152.01	0.0050221357752977\\
153.01	0.00502215526229993\\
154.01	0.00502217513409978\\
155.01	0.00502219539820223\\
156.01	0.00502221606225553\\
157.01	0.00502223713405503\\
158.01	0.0050222586215449\\
159.01	0.00502228053282122\\
160.01	0.00502230287613462\\
161.01	0.00502232565989372\\
162.01	0.00502234889266762\\
163.01	0.0050223725831888\\
164.01	0.00502239674035607\\
165.01	0.00502242137323838\\
166.01	0.005022446491077\\
167.01	0.00502247210328907\\
168.01	0.00502249821947123\\
169.01	0.00502252484940274\\
170.01	0.00502255200304808\\
171.01	0.00502257969056166\\
172.01	0.00502260792228945\\
173.01	0.00502263670877492\\
174.01	0.00502266606076066\\
175.01	0.00502269598919311\\
176.01	0.00502272650522542\\
177.01	0.00502275762022222\\
178.01	0.00502278934576256\\
179.01	0.00502282169364454\\
180.01	0.00502285467588862\\
181.01	0.00502288830474204\\
182.01	0.00502292259268314\\
183.01	0.00502295755242499\\
184.01	0.00502299319692021\\
185.01	0.00502302953936454\\
186.01	0.00502306659320215\\
187.01	0.00502310437212936\\
188.01	0.00502314289009939\\
189.01	0.00502318216132727\\
190.01	0.00502322220029433\\
191.01	0.00502326302175243\\
192.01	0.00502330464072964\\
193.01	0.00502334707253499\\
194.01	0.00502339033276278\\
195.01	0.00502343443729858\\
196.01	0.00502347940232345\\
197.01	0.00502352524432037\\
198.01	0.00502357198007852\\
199.01	0.00502361962669878\\
200.01	0.00502366820160027\\
201.01	0.0050237177225248\\
202.01	0.00502376820754208\\
203.01	0.00502381967505755\\
204.01	0.00502387214381636\\
205.01	0.00502392563291001\\
206.01	0.00502398016178238\\
207.01	0.00502403575023566\\
208.01	0.00502409241843702\\
209.01	0.00502415018692418\\
210.01	0.00502420907661233\\
211.01	0.00502426910880058\\
212.01	0.00502433030517863\\
213.01	0.00502439268783314\\
214.01	0.00502445627925474\\
215.01	0.005024521102345\\
216.01	0.00502458718042365\\
217.01	0.00502465453723511\\
218.01	0.00502472319695607\\
219.01	0.00502479318420265\\
220.01	0.00502486452403789\\
221.01	0.00502493724197978\\
222.01	0.0050250113640079\\
223.01	0.00502508691657197\\
224.01	0.00502516392659885\\
225.01	0.0050252424215021\\
226.01	0.00502532242918824\\
227.01	0.00502540397806611\\
228.01	0.00502548709705464\\
229.01	0.00502557181559166\\
230.01	0.00502565816364192\\
231.01	0.00502574617170654\\
232.01	0.005025835870831\\
233.01	0.00502592729261411\\
234.01	0.00502602046921832\\
235.01	0.0050261154333765\\
236.01	0.00502621221840324\\
237.01	0.00502631085820338\\
238.01	0.00502641138728205\\
239.01	0.00502651384075328\\
240.01	0.00502661825435088\\
241.01	0.00502672466443788\\
242.01	0.00502683310801658\\
243.01	0.00502694362273876\\
244.01	0.0050270562469161\\
245.01	0.00502717101953057\\
246.01	0.00502728798024471\\
247.01	0.00502740716941269\\
248.01	0.0050275286280913\\
249.01	0.005027652398051\\
250.01	0.00502777852178654\\
251.01	0.00502790704252892\\
252.01	0.00502803800425672\\
253.01	0.00502817145170778\\
254.01	0.00502830743039084\\
255.01	0.00502844598659744\\
256.01	0.00502858716741425\\
257.01	0.00502873102073589\\
258.01	0.00502887759527657\\
259.01	0.00502902694058344\\
260.01	0.005029179107049\\
261.01	0.00502933414592505\\
262.01	0.0050294921093347\\
263.01	0.00502965305028719\\
264.01	0.00502981702269076\\
265.01	0.0050299840813669\\
266.01	0.00503015428206463\\
267.01	0.00503032768147485\\
268.01	0.00503050433724536\\
269.01	0.00503068430799553\\
270.01	0.00503086765333136\\
271.01	0.00503105443386162\\
272.01	0.00503124471121294\\
273.01	0.00503143854804688\\
274.01	0.00503163600807607\\
275.01	0.00503183715608083\\
276.01	0.00503204205792718\\
277.01	0.00503225078058336\\
278.01	0.00503246339213911\\
279.01	0.00503267996182363\\
280.01	0.00503290056002414\\
281.01	0.00503312525830624\\
282.01	0.00503335412943298\\
283.01	0.0050335872473862\\
284.01	0.00503382468738714\\
285.01	0.00503406652591845\\
286.01	0.00503431284074593\\
287.01	0.00503456371094249\\
288.01	0.00503481921691118\\
289.01	0.00503507944040973\\
290.01	0.00503534446457649\\
291.01	0.00503561437395558\\
292.01	0.00503588925452533\\
293.01	0.00503616919372432\\
294.01	0.00503645428048247\\
295.01	0.00503674460524972\\
296.01	0.00503704026002744\\
297.01	0.00503734133840114\\
298.01	0.00503764793557371\\
299.01	0.00503796014839986\\
300.01	0.00503827807542342\\
301.01	0.00503860181691385\\
302.01	0.00503893147490631\\
303.01	0.00503926715324135\\
304.01	0.00503960895760799\\
305.01	0.00503995699558779\\
306.01	0.0050403113767003\\
307.01	0.00504067221245086\\
308.01	0.00504103961638069\\
309.01	0.00504141370411773\\
310.01	0.00504179459343115\\
311.01	0.00504218240428678\\
312.01	0.00504257725890531\\
313.01	0.00504297928182318\\
314.01	0.00504338859995435\\
315.01	0.00504380534265644\\
316.01	0.00504422964179752\\
317.01	0.00504466163182693\\
318.01	0.00504510144984702\\
319.01	0.00504554923568865\\
320.01	0.00504600513198888\\
321.01	0.00504646928427072\\
322.01	0.00504694184102603\\
323.01	0.00504742295380053\\
324.01	0.00504791277728073\\
325.01	0.00504841146938398\\
326.01	0.00504891919135017\\
327.01	0.00504943610783593\\
328.01	0.00504996238700892\\
329.01	0.00505049820064668\\
330.01	0.00505104372423423\\
331.01	0.00505159913706429\\
332.01	0.00505216462233776\\
333.01	0.0050527403672645\\
334.01	0.00505332656316482\\
335.01	0.00505392340556957\\
336.01	0.00505453109432047\\
337.01	0.00505514983366831\\
338.01	0.00505577983236974\\
339.01	0.00505642130378112\\
340.01	0.00505707446594919\\
341.01	0.00505773954169936\\
342.01	0.00505841675871721\\
343.01	0.00505910634962739\\
344.01	0.005059808552065\\
345.01	0.0050605236087411\\
346.01	0.00506125176750191\\
347.01	0.00506199328137963\\
348.01	0.00506274840863571\\
349.01	0.00506351741279459\\
350.01	0.00506430056267015\\
351.01	0.00506509813238215\\
352.01	0.00506591040136409\\
353.01	0.00506673765436222\\
354.01	0.00506758018142595\\
355.01	0.00506843827789061\\
356.01	0.00506931224435403\\
357.01	0.00507020238664743\\
358.01	0.00507110901580171\\
359.01	0.00507203244801305\\
360.01	0.00507297300460754\\
361.01	0.0050739310120094\\
362.01	0.00507490680171485\\
363.01	0.00507590071027472\\
364.01	0.00507691307928929\\
365.01	0.00507794425541877\\
366.01	0.00507899459041269\\
367.01	0.00508006444116119\\
368.01	0.00508115416977156\\
369.01	0.00508226414367324\\
370.01	0.00508339473575106\\
371.01	0.00508454632450858\\
372.01	0.00508571929426307\\
373.01	0.00508691403536712\\
374.01	0.00508813094445901\\
375.01	0.00508937042473226\\
376.01	0.00509063288622533\\
377.01	0.00509191874612255\\
378.01	0.00509322842905835\\
379.01	0.00509456236742253\\
380.01	0.0050959210016579\\
381.01	0.00509730478054768\\
382.01	0.00509871416149152\\
383.01	0.00510014961076941\\
384.01	0.00510161160379846\\
385.01	0.00510310062538774\\
386.01	0.00510461716999747\\
387.01	0.00510616174200523\\
388.01	0.00510773485598404\\
389.01	0.0051093370369904\\
390.01	0.00511096882086284\\
391.01	0.00511263075453339\\
392.01	0.00511432339634935\\
393.01	0.00511604731640772\\
394.01	0.00511780309690188\\
395.01	0.0051195913324808\\
396.01	0.00512141263062116\\
397.01	0.00512326761201211\\
398.01	0.00512515691095339\\
399.01	0.0051270811757666\\
400.01	0.00512904106921996\\
401.01	0.00513103726896684\\
402.01	0.00513307046799628\\
403.01	0.00513514137509836\\
404.01	0.00513725071534207\\
405.01	0.00513939923056583\\
406.01	0.00514158767988127\\
407.01	0.00514381684018901\\
408.01	0.00514608750670632\\
409.01	0.00514840049350604\\
410.01	0.00515075663406668\\
411.01	0.00515315678183237\\
412.01	0.00515560181078219\\
413.01	0.00515809261600762\\
414.01	0.00516063011429847\\
415.01	0.00516321524473404\\
416.01	0.00516584896928096\\
417.01	0.00516853227339409\\
418.01	0.00517126616662132\\
419.01	0.0051740516832099\\
420.01	0.00517688988271299\\
421.01	0.00517978185059545\\
422.01	0.0051827286988378\\
423.01	0.00518573156653616\\
424.01	0.00518879162049825\\
425.01	0.0051919100558333\\
426.01	0.00519508809653527\\
427.01	0.00519832699605861\\
428.01	0.005201628037886\\
429.01	0.00520499253608783\\
430.01	0.0052084218358726\\
431.01	0.00521191731412986\\
432.01	0.00521548037996471\\
433.01	0.00521911247522648\\
434.01	0.00522281507503175\\
435.01	0.00522658968828514\\
436.01	0.005230437858198\\
437.01	0.00523436116281045\\
438.01	0.00523836121551853\\
439.01	0.0052424396656111\\
440.01	0.00524659819881913\\
441.01	0.00525083853788445\\
442.01	0.00525516244315059\\
443.01	0.00525957171318059\\
444.01	0.00526406818540756\\
445.01	0.0052686537368216\\
446.01	0.0052733302846973\\
447.01	0.00527809978736513\\
448.01	0.00528296424502851\\
449.01	0.00528792570062933\\
450.01	0.00529298624076026\\
451.01	0.00529814799662302\\
452.01	0.00530341314502969\\
453.01	0.0053087839094425\\
454.01	0.00531426256104464\\
455.01	0.00531985141983556\\
456.01	0.00532555285574184\\
457.01	0.0053313692897333\\
458.01	0.00533730319493714\\
459.01	0.00534335709773871\\
460.01	0.00534953357886434\\
461.01	0.00535583527443989\\
462.01	0.00536226487702372\\
463.01	0.00536882513661484\\
464.01	0.00537551886164229\\
465.01	0.00538234891994208\\
466.01	0.0053893182397339\\
467.01	0.0053964298106058\\
468.01	0.00540368668452083\\
469.01	0.00541109197685086\\
470.01	0.0054186488674464\\
471.01	0.00542636060174487\\
472.01	0.00543423049192164\\
473.01	0.00544226191808676\\
474.01	0.00545045832952867\\
475.01	0.00545882324600859\\
476.01	0.00546736025910498\\
477.01	0.00547607303360939\\
478.01	0.00548496530897408\\
479.01	0.00549404090080845\\
480.01	0.00550330370242497\\
481.01	0.00551275768643106\\
482.01	0.00552240690636421\\
483.01	0.00553225549836771\\
484.01	0.00554230768290534\\
485.01	0.00555256776651032\\
486.01	0.00556304014356669\\
487.01	0.00557372929812386\\
488.01	0.00558463980574016\\
489.01	0.00559577633535974\\
490.01	0.00560714365122079\\
491.01	0.00561874661479977\\
492.01	0.00563059018679275\\
493.01	0.00564267942913979\\
494.01	0.00565501950709213\\
495.01	0.00566761569132816\\
496.01	0.00568047336011608\\
497.01	0.00569359800152755\\
498.01	0.00570699521569809\\
499.01	0.00572067071713506\\
500.01	0.0057346303370701\\
501.01	0.00574888002585278\\
502.01	0.00576342585538613\\
503.01	0.00577827402159808\\
504.01	0.00579343084694856\\
505.01	0.00580890278297078\\
506.01	0.00582469641284311\\
507.01	0.0058408184539908\\
508.01	0.00585727576071554\\
509.01	0.00587407532685036\\
510.01	0.00589122428843934\\
511.01	0.00590872992643619\\
512.01	0.00592659966942349\\
513.01	0.0059448410963448\\
514.01	0.00596346193924731\\
515.01	0.00598247008602941\\
516.01	0.00600187358318675\\
517.01	0.00602168063855298\\
518.01	0.0060418996240226\\
519.01	0.00606253907825346\\
520.01	0.00608360770933741\\
521.01	0.00610511439742895\\
522.01	0.00612706819732234\\
523.01	0.00614947834096546\\
524.01	0.00617235423989611\\
525.01	0.00619570548758692\\
526.01	0.00621954186168196\\
527.01	0.0062438733261071\\
528.01	0.00626871003303373\\
529.01	0.00629406232467292\\
530.01	0.00631994073487588\\
531.01	0.00634635599051178\\
532.01	0.00637331901259404\\
533.01	0.00640084091711891\\
534.01	0.00642893301558123\\
535.01	0.00645760681512446\\
536.01	0.00648687401827952\\
537.01	0.00651674652224228\\
538.01	0.00654723641763366\\
539.01	0.00657835598668036\\
540.01	0.00661011770074879\\
541.01	0.00664253421715752\\
542.01	0.00667561837518443\\
543.01	0.00670938319117929\\
544.01	0.00674384185268058\\
545.01	0.00677900771142645\\
546.01	0.00681489427513986\\
547.01	0.00685151519795303\\
548.01	0.00688888426932745\\
549.01	0.00692701540130676\\
550.01	0.00696592261392825\\
551.01	0.00700562001860003\\
552.01	0.00704612179923281\\
553.01	0.00708744219089684\\
554.01	0.00712959545574979\\
555.01	0.00717259585596489\\
556.01	0.00721645762335742\\
557.01	0.00726119492538552\\
558.01	0.00730682182717329\\
559.01	0.00735335224917305\\
560.01	0.00740079992005648\\
561.01	0.0074491783243876\\
562.01	0.0074985006446055\\
563.01	0.00754877969680245\\
564.01	0.00760002785975936\\
565.01	0.00765225699666378\\
566.01	0.00770547836890862\\
567.01	0.00775970254134385\\
568.01	0.00781493927833469\\
569.01	0.00787119742996866\\
570.01	0.00792848480775271\\
571.01	0.00798680804915989\\
572.01	0.00804617247041955\\
573.01	0.00810658190700895\\
574.01	0.00816803854140444\\
575.01	0.00823054271779196\\
576.01	0.00829409274364186\\
577.01	0.00835868467832643\\
578.01	0.00842431210933126\\
579.01	0.00849096591709788\\
580.01	0.0085586340301731\\
581.01	0.00862730117316442\\
582.01	0.00869694861105789\\
583.01	0.0087675538948034\\
584.01	0.00883909061478503\\
585.01	0.00891152817095808\\
586.01	0.00898483157115984\\
587.01	0.00905896127252333\\
588.01	0.00913387308520954\\
589.01	0.00920951816302855\\
590.01	0.00928584311219743\\
591.01	0.00936279025779631\\
592.01	0.00944029811781628\\
593.01	0.00951830214751989\\
594.01	0.00959673583273945\\
595.01	0.00967553223043691\\
596.01	0.00975462607922802\\
597.01	0.00983384409165019\\
598.01	0.00990866201848071\\
599.01	0.00997087280416276\\
599.02	0.00997138072163725\\
599.03	0.00997188557625799\\
599.04	0.00997238733820211\\
599.05	0.00997288597735272\\
599.06	0.00997338146329604\\
599.07	0.00997387376531845\\
599.08	0.00997436285240349\\
599.09	0.00997484869322892\\
599.1	0.00997533125616361\\
599.11	0.00997581050926455\\
599.12	0.00997628642027372\\
599.13	0.00997675895661497\\
599.14	0.0099772280853909\\
599.15	0.00997769377337961\\
599.16	0.00997815598703157\\
599.17	0.00997861469246631\\
599.18	0.00997906985546915\\
599.19	0.00997952144148792\\
599.2	0.00997996941562957\\
599.21	0.00998041374265684\\
599.22	0.0099808543869848\\
599.23	0.00998129131267744\\
599.24	0.00998172448344418\\
599.25	0.00998215386263636\\
599.26	0.00998257941258354\\
599.27	0.00998300109279825\\
599.28	0.00998341886238981\\
599.29	0.00998383268006024\\
599.3	0.00998424250410032\\
599.31	0.00998464829238543\\
599.32	0.00998505000237151\\
599.33	0.00998544759109087\\
599.34	0.00998584101514799\\
599.35	0.00998623023071532\\
599.36	0.00998661519352895\\
599.37	0.00998699585888436\\
599.38	0.00998737218163197\\
599.39	0.00998774411617281\\
599.4	0.00998811161645403\\
599.41	0.00998847463596443\\
599.42	0.0099888331277299\\
599.43	0.00998918704430885\\
599.44	0.00998953633778757\\
599.45	0.00998988095977558\\
599.46	0.00999022086140088\\
599.47	0.00999055599330522\\
599.48	0.00999088630563925\\
599.49	0.00999121174805768\\
599.5	0.00999153226971435\\
599.51	0.0099918478192573\\
599.52	0.00999215834482374\\
599.53	0.00999246379403503\\
599.54	0.0099927641139915\\
599.55	0.00999305925126738\\
599.56	0.00999334915190553\\
599.57	0.0099936337614122\\
599.58	0.00999391302475173\\
599.59	0.00999418688634118\\
599.6	0.00999445529004489\\
599.61	0.00999471817916904\\
599.62	0.00999497549645613\\
599.63	0.00999522718407935\\
599.64	0.009995473183637\\
599.65	0.00999571343614678\\
599.66	0.00999594788204004\\
599.67	0.00999617646115598\\
599.68	0.0099963991127358\\
599.69	0.00999661577541675\\
599.7	0.00999682638722618\\
599.71	0.00999703088557551\\
599.72	0.00999722920725411\\
599.73	0.00999742128842315\\
599.74	0.00999760706460942\\
599.75	0.00999778647069897\\
599.76	0.00999795944093086\\
599.77	0.0099981259088907\\
599.78	0.0099982858075042\\
599.79	0.00999843906903064\\
599.8	0.00999858562505627\\
599.81	0.00999872540648767\\
599.82	0.00999885834354498\\
599.83	0.00999898436575515\\
599.84	0.00999910340194508\\
599.85	0.00999921538023465\\
599.86	0.00999932022802977\\
599.87	0.00999941787201528\\
599.88	0.00999950823814785\\
599.89	0.00999959125164875\\
599.9	0.00999966683699656\\
599.91	0.00999973491791987\\
599.92	0.0099997954173898\\
599.93	0.00999984825761255\\
599.94	0.00999989336002181\\
599.95	0.00999993064527112\\
599.96	0.00999996003322615\\
599.97	0.00999998144295691\\
599.98	0.00999999479272987\\
599.99	0.01\\
600	0.01\\
};
\addplot [color=mycolor3,solid,forget plot]
  table[row sep=crcr]{%
0.01	0.0050016280518921\\
1.01	0.00500162904994192\\
2.01	0.00500163006827589\\
3.01	0.00500163110730512\\
4.01	0.00500163216744873\\
5.01	0.00500163324913459\\
6.01	0.00500163435279902\\
7.01	0.00500163547888721\\
8.01	0.00500163662785354\\
9.01	0.00500163780016109\\
10.01	0.00500163899628275\\
11.01	0.00500164021670073\\
12.01	0.0050016414619069\\
13.01	0.00500164273240315\\
14.01	0.00500164402870142\\
15.01	0.00500164535132418\\
16.01	0.00500164670080426\\
17.01	0.00500164807768509\\
18.01	0.00500164948252126\\
19.01	0.00500165091587854\\
20.01	0.00500165237833407\\
21.01	0.00500165387047644\\
22.01	0.00500165539290649\\
23.01	0.00500165694623676\\
24.01	0.00500165853109226\\
25.01	0.00500166014811082\\
26.01	0.00500166179794266\\
27.01	0.00500166348125161\\
28.01	0.0050016651987143\\
29.01	0.00500166695102154\\
30.01	0.00500166873887767\\
31.01	0.0050016705630014\\
32.01	0.00500167242412572\\
33.01	0.00500167432299854\\
34.01	0.0050016762603827\\
35.01	0.00500167823705644\\
36.01	0.00500168025381365\\
37.01	0.00500168231146401\\
38.01	0.00500168441083404\\
39.01	0.00500168655276619\\
40.01	0.00500168873812026\\
41.01	0.00500169096777317\\
42.01	0.00500169324261952\\
43.01	0.00500169556357194\\
44.01	0.00500169793156127\\
45.01	0.00500170034753701\\
46.01	0.00500170281246805\\
47.01	0.00500170532734243\\
48.01	0.00500170789316817\\
49.01	0.00500171051097331\\
50.01	0.00500171318180696\\
51.01	0.00500171590673891\\
52.01	0.00500171868686054\\
53.01	0.00500172152328528\\
54.01	0.00500172441714855\\
55.01	0.00500172736960895\\
56.01	0.00500173038184804\\
57.01	0.00500173345507127\\
58.01	0.00500173659050805\\
59.01	0.00500173978941269\\
60.01	0.00500174305306444\\
61.01	0.00500174638276826\\
62.01	0.00500174977985532\\
63.01	0.0050017532456833\\
64.01	0.00500175678163739\\
65.01	0.00500176038913015\\
66.01	0.00500176406960282\\
67.01	0.00500176782452535\\
68.01	0.00500177165539681\\
69.01	0.00500177556374673\\
70.01	0.00500177955113498\\
71.01	0.00500178361915304\\
72.01	0.00500178776942375\\
73.01	0.00500179200360257\\
74.01	0.00500179632337816\\
75.01	0.00500180073047306\\
76.01	0.00500180522664397\\
77.01	0.00500180981368298\\
78.01	0.00500181449341772\\
79.01	0.00500181926771266\\
80.01	0.0050018241384694\\
81.01	0.00500182910762757\\
82.01	0.00500183417716536\\
83.01	0.0050018393491007\\
84.01	0.00500184462549172\\
85.01	0.00500185000843775\\
86.01	0.00500185550007982\\
87.01	0.00500186110260194\\
88.01	0.00500186681823142\\
89.01	0.00500187264924018\\
90.01	0.00500187859794544\\
91.01	0.00500188466671074\\
92.01	0.00500189085794667\\
93.01	0.00500189717411179\\
94.01	0.00500190361771366\\
95.01	0.00500191019131004\\
96.01	0.00500191689750928\\
97.01	0.00500192373897206\\
98.01	0.00500193071841186\\
99.01	0.00500193783859608\\
100.01	0.00500194510234757\\
101.01	0.00500195251254507\\
102.01	0.00500196007212453\\
103.01	0.0050019677840805\\
104.01	0.00500197565146715\\
105.01	0.0050019836773994\\
106.01	0.00500199186505378\\
107.01	0.00500200021767026\\
108.01	0.00500200873855308\\
109.01	0.00500201743107212\\
110.01	0.00500202629866431\\
111.01	0.00500203534483475\\
112.01	0.00500204457315789\\
113.01	0.00500205398727957\\
114.01	0.00500206359091773\\
115.01	0.00500207338786419\\
116.01	0.00500208338198575\\
117.01	0.00500209357722599\\
118.01	0.00500210397760696\\
119.01	0.00500211458722999\\
120.01	0.00500212541027791\\
121.01	0.00500213645101627\\
122.01	0.00500214771379517\\
123.01	0.00500215920305093\\
124.01	0.00500217092330749\\
125.01	0.00500218287917836\\
126.01	0.00500219507536824\\
127.01	0.00500220751667511\\
128.01	0.00500222020799147\\
129.01	0.0050022331543068\\
130.01	0.00500224636070901\\
131.01	0.00500225983238653\\
132.01	0.00500227357463011\\
133.01	0.00500228759283523\\
134.01	0.00500230189250367\\
135.01	0.00500231647924548\\
136.01	0.00500233135878173\\
137.01	0.00500234653694627\\
138.01	0.00500236201968751\\
139.01	0.00500237781307142\\
140.01	0.00500239392328344\\
141.01	0.00500241035663061\\
142.01	0.00500242711954409\\
143.01	0.00500244421858152\\
144.01	0.00500246166042968\\
145.01	0.00500247945190674\\
146.01	0.00500249759996471\\
147.01	0.00500251611169228\\
148.01	0.00500253499431723\\
149.01	0.0050025542552094\\
150.01	0.00500257390188305\\
151.01	0.00500259394199984\\
152.01	0.0050026143833716\\
153.01	0.00500263523396326\\
154.01	0.00500265650189594\\
155.01	0.00500267819544939\\
156.01	0.00500270032306574\\
157.01	0.005002722893352\\
158.01	0.00500274591508375\\
159.01	0.0050027693972076\\
160.01	0.00500279334884582\\
161.01	0.0050028177792977\\
162.01	0.00500284269804473\\
163.01	0.00500286811475308\\
164.01	0.00500289403927738\\
165.01	0.00500292048166448\\
166.01	0.00500294745215671\\
167.01	0.00500297496119584\\
168.01	0.0050030030194267\\
169.01	0.00500303163770119\\
170.01	0.00500306082708216\\
171.01	0.00500309059884714\\
172.01	0.00500312096449286\\
173.01	0.00500315193573883\\
174.01	0.00500318352453204\\
175.01	0.00500321574305071\\
176.01	0.005003248603709\\
177.01	0.00500328211916144\\
178.01	0.00500331630230723\\
179.01	0.00500335116629466\\
180.01	0.00500338672452627\\
181.01	0.00500342299066324\\
182.01	0.00500345997863021\\
183.01	0.00500349770262021\\
184.01	0.00500353617709963\\
185.01	0.00500357541681333\\
186.01	0.00500361543678975\\
187.01	0.00500365625234608\\
188.01	0.00500369787909372\\
189.01	0.00500374033294362\\
190.01	0.00500378363011175\\
191.01	0.00500382778712488\\
192.01	0.00500387282082589\\
193.01	0.0050039187483798\\
194.01	0.00500396558727978\\
195.01	0.00500401335535278\\
196.01	0.00500406207076591\\
197.01	0.00500411175203229\\
198.01	0.00500416241801774\\
199.01	0.00500421408794673\\
200.01	0.00500426678140897\\
201.01	0.00500432051836609\\
202.01	0.00500437531915836\\
203.01	0.00500443120451104\\
204.01	0.00500448819554175\\
205.01	0.00500454631376729\\
206.01	0.00500460558111075\\
207.01	0.00500466601990832\\
208.01	0.00500472765291744\\
209.01	0.00500479050332339\\
210.01	0.00500485459474725\\
211.01	0.00500491995125339\\
212.01	0.00500498659735742\\
213.01	0.00500505455803364\\
214.01	0.0050051238587237\\
215.01	0.00500519452534379\\
216.01	0.0050052665842935\\
217.01	0.0050053400624642\\
218.01	0.0050054149872469\\
219.01	0.00500549138654141\\
220.01	0.00500556928876456\\
221.01	0.00500564872285925\\
222.01	0.00500572971830321\\
223.01	0.00500581230511794\\
224.01	0.00500589651387844\\
225.01	0.00500598237572125\\
226.01	0.00500606992235524\\
227.01	0.00500615918607009\\
228.01	0.00500625019974606\\
229.01	0.00500634299686419\\
230.01	0.00500643761151554\\
231.01	0.00500653407841152\\
232.01	0.00500663243289368\\
233.01	0.00500673271094423\\
234.01	0.00500683494919568\\
235.01	0.00500693918494234\\
236.01	0.00500704545614977\\
237.01	0.0050071538014658\\
238.01	0.00500726426023136\\
239.01	0.00500737687249117\\
240.01	0.00500749167900487\\
241.01	0.00500760872125791\\
242.01	0.00500772804147255\\
243.01	0.00500784968261975\\
244.01	0.00500797368842978\\
245.01	0.00500810010340398\\
246.01	0.0050082289728267\\
247.01	0.00500836034277632\\
248.01	0.00500849426013733\\
249.01	0.00500863077261224\\
250.01	0.00500876992873327\\
251.01	0.00500891177787448\\
252.01	0.00500905637026401\\
253.01	0.00500920375699584\\
254.01	0.00500935399004249\\
255.01	0.0050095071222674\\
256.01	0.00500966320743702\\
257.01	0.00500982230023315\\
258.01	0.00500998445626602\\
259.01	0.00501014973208669\\
260.01	0.00501031818519961\\
261.01	0.00501048987407515\\
262.01	0.00501066485816323\\
263.01	0.00501084319790523\\
264.01	0.00501102495474763\\
265.01	0.00501121019115461\\
266.01	0.00501139897062119\\
267.01	0.00501159135768638\\
268.01	0.00501178741794621\\
269.01	0.00501198721806694\\
270.01	0.00501219082579847\\
271.01	0.00501239830998744\\
272.01	0.00501260974059091\\
273.01	0.00501282518868935\\
274.01	0.00501304472650031\\
275.01	0.0050132684273924\\
276.01	0.00501349636589818\\
277.01	0.00501372861772856\\
278.01	0.00501396525978616\\
279.01	0.00501420637017932\\
280.01	0.00501445202823633\\
281.01	0.00501470231451923\\
282.01	0.00501495731083836\\
283.01	0.00501521710026688\\
284.01	0.00501548176715487\\
285.01	0.00501575139714465\\
286.01	0.00501602607718557\\
287.01	0.00501630589554915\\
288.01	0.00501659094184514\\
289.01	0.00501688130703674\\
290.01	0.00501717708345732\\
291.01	0.00501747836482667\\
292.01	0.00501778524626806\\
293.01	0.00501809782432616\\
294.01	0.00501841619698477\\
295.01	0.00501874046368529\\
296.01	0.00501907072534664\\
297.01	0.00501940708438503\\
298.01	0.00501974964473477\\
299.01	0.00502009851187034\\
300.01	0.00502045379282888\\
301.01	0.00502081559623435\\
302.01	0.00502118403232309\\
303.01	0.00502155921296964\\
304.01	0.00502194125171489\\
305.01	0.00502233026379545\\
306.01	0.00502272636617557\\
307.01	0.00502312967757916\\
308.01	0.00502354031852612\\
309.01	0.00502395841136923\\
310.01	0.00502438408033382\\
311.01	0.00502481745156044\\
312.01	0.00502525865315002\\
313.01	0.00502570781521222\\
314.01	0.00502616506991675\\
315.01	0.00502663055154874\\
316.01	0.00502710439656688\\
317.01	0.00502758674366617\\
318.01	0.00502807773384525\\
319.01	0.00502857751047717\\
320.01	0.00502908621938583\\
321.01	0.00502960400892682\\
322.01	0.00503013103007399\\
323.01	0.00503066743651082\\
324.01	0.0050312133847286\\
325.01	0.00503176903412964\\
326.01	0.00503233454713715\\
327.01	0.00503291008931162\\
328.01	0.00503349582947334\\
329.01	0.00503409193983205\\
330.01	0.00503469859612314\\
331.01	0.005035315977751\\
332.01	0.00503594426793894\\
333.01	0.00503658365388542\\
334.01	0.00503723432692794\\
335.01	0.00503789648271236\\
336.01	0.00503857032136812\\
337.01	0.00503925604768955\\
338.01	0.00503995387132164\\
339.01	0.00504066400694936\\
340.01	0.00504138667449157\\
341.01	0.00504212209929511\\
342.01	0.00504287051233135\\
343.01	0.00504363215039098\\
344.01	0.00504440725627697\\
345.01	0.00504519607899295\\
346.01	0.00504599887392596\\
347.01	0.00504681590301874\\
348.01	0.00504764743493299\\
349.01	0.00504849374519711\\
350.01	0.00504935511633733\\
351.01	0.00505023183798901\\
352.01	0.00505112420698506\\
353.01	0.00505203252741686\\
354.01	0.00505295711066694\\
355.01	0.0050538982754085\\
356.01	0.00505485634756882\\
357.01	0.00505583166025526\\
358.01	0.00505682455364158\\
359.01	0.00505783537481175\\
360.01	0.00505886447756317\\
361.01	0.00505991222216809\\
362.01	0.0050609789750954\\
363.01	0.00506206510869642\\
364.01	0.00506317100085968\\
365.01	0.00506429703464094\\
366.01	0.00506544359787774\\
367.01	0.00506661108279842\\
368.01	0.00506779988563986\\
369.01	0.00506901040628567\\
370.01	0.00507024304794354\\
371.01	0.00507149821687677\\
372.01	0.00507277632220685\\
373.01	0.00507407777580396\\
374.01	0.0050754029922775\\
375.01	0.00507675238907809\\
376.01	0.0050781263867148\\
377.01	0.00507952540908478\\
378.01	0.00508094988390589\\
379.01	0.00508240024323037\\
380.01	0.0050838769240124\\
381.01	0.00508538036869029\\
382.01	0.00508691102574624\\
383.01	0.00508846935020078\\
384.01	0.00509005580401857\\
385.01	0.00509167085641588\\
386.01	0.0050933149840912\\
387.01	0.0050949886714239\\
388.01	0.00509669241067484\\
389.01	0.00509842670220022\\
390.01	0.00510019205467784\\
391.01	0.00510198898534765\\
392.01	0.00510381802026706\\
393.01	0.00510567969458225\\
394.01	0.0051075745528162\\
395.01	0.0051095031491746\\
396.01	0.0051114660478695\\
397.01	0.00511346382346355\\
398.01	0.00511549706123428\\
399.01	0.00511756635755942\\
400.01	0.00511967232032567\\
401.01	0.00512181556935933\\
402.01	0.00512399673688282\\
403.01	0.00512621646799463\\
404.01	0.00512847542117596\\
405.01	0.00513077426882418\\
406.01	0.00513311369781239\\
407.01	0.00513549441007806\\
408.01	0.00513791712323855\\
409.01	0.00514038257123667\\
410.01	0.00514289150501269\\
411.01	0.00514544469320723\\
412.01	0.00514804292289033\\
413.01	0.0051506870003202\\
414.01	0.00515337775172785\\
415.01	0.00515611602412834\\
416.01	0.005158902686157\\
417.01	0.00516173862892902\\
418.01	0.00516462476692038\\
419.01	0.00516756203886763\\
420.01	0.00517055140868483\\
421.01	0.00517359386639368\\
422.01	0.00517669042906353\\
423.01	0.00517984214175886\\
424.01	0.00518305007848732\\
425.01	0.00518631534314723\\
426.01	0.0051896390704676\\
427.01	0.00519302242693516\\
428.01	0.00519646661170586\\
429.01	0.00519997285749194\\
430.01	0.0052035424314217\\
431.01	0.00520717663586574\\
432.01	0.00521087680922418\\
433.01	0.00521464432667026\\
434.01	0.00521848060084656\\
435.01	0.00522238708250908\\
436.01	0.00522636526111817\\
437.01	0.00523041666537323\\
438.01	0.00523454286369303\\
439.01	0.00523874546464077\\
440.01	0.00524302611730065\\
441.01	0.00524738651160832\\
442.01	0.00525182837864539\\
443.01	0.00525635349090743\\
444.01	0.00526096366255775\\
445.01	0.00526566074968343\\
446.01	0.00527044665056876\\
447.01	0.00527532330600824\\
448.01	0.00528029269967614\\
449.01	0.00528535685857795\\
450.01	0.00529051785360128\\
451.01	0.00529577780018857\\
452.01	0.00530113885914815\\
453.01	0.00530660323761675\\
454.01	0.00531217319018241\\
455.01	0.005317851020168\\
456.01	0.00532363908106857\\
457.01	0.0053295397781255\\
458.01	0.00533555557001072\\
459.01	0.005341688970586\\
460.01	0.00534794255069161\\
461.01	0.00535431893991593\\
462.01	0.00536082082829249\\
463.01	0.00536745096787895\\
464.01	0.00537421217417684\\
465.01	0.00538110732737072\\
466.01	0.00538813937338328\\
467.01	0.00539531132476963\\
468.01	0.00540262626148751\\
469.01	0.00541008733159785\\
470.01	0.00541769775194074\\
471.01	0.00542546080881408\\
472.01	0.0054333798586714\\
473.01	0.00544145832885058\\
474.01	0.00544969971834659\\
475.01	0.00545810759864\\
476.01	0.00546668561459478\\
477.01	0.00547543748543522\\
478.01	0.00548436700581247\\
479.01	0.00549347804696805\\
480.01	0.00550277455799925\\
481.01	0.00551226056722783\\
482.01	0.0055219401836714\\
483.01	0.00553181759861202\\
484.01	0.00554189708725276\\
485.01	0.00555218301045088\\
486.01	0.00556267981651445\\
487.01	0.00557339204304411\\
488.01	0.00558432431880738\\
489.01	0.00559548136562726\\
490.01	0.0056068680002765\\
491.01	0.00561848913636992\\
492.01	0.00563034978625274\\
493.01	0.0056424550628901\\
494.01	0.00565481018177132\\
495.01	0.00566742046284165\\
496.01	0.00568029133248303\\
497.01	0.00569342832555668\\
498.01	0.00570683708752279\\
499.01	0.0057205233766401\\
500.01	0.00573449306624481\\
501.01	0.00574875214710439\\
502.01	0.00576330672983821\\
503.01	0.00577816304739932\\
504.01	0.00579332745760815\\
505.01	0.00580880644572997\\
506.01	0.00582460662709058\\
507.01	0.00584073474972176\\
508.01	0.00585719769703225\\
509.01	0.00587400249050098\\
510.01	0.00589115629238746\\
511.01	0.00590866640846227\\
512.01	0.00592654029075282\\
513.01	0.00594478554030617\\
514.01	0.00596340990996545\\
515.01	0.0059824213071583\\
516.01	0.00600182779669088\\
517.01	0.00602163760354031\\
518.01	0.00604185911563766\\
519.01	0.00606250088662946\\
520.01	0.00608357163860813\\
521.01	0.00610508026480032\\
522.01	0.00612703583220095\\
523.01	0.00614944758414021\\
524.01	0.00617232494277072\\
525.01	0.00619567751146097\\
526.01	0.00621951507707789\\
527.01	0.00624384761214108\\
528.01	0.00626868527682911\\
529.01	0.00629403842081533\\
530.01	0.00631991758490741\\
531.01	0.00634633350246382\\
532.01	0.00637329710055433\\
533.01	0.00640081950083352\\
534.01	0.00642891202008611\\
535.01	0.00645758617040558\\
536.01	0.00648685365895875\\
537.01	0.00651672638728716\\
538.01	0.00654721645008825\\
539.01	0.00657833613341638\\
540.01	0.00661009791223496\\
541.01	0.00664251444724473\\
542.01	0.00667559858090686\\
543.01	0.00670936333256887\\
544.01	0.00674382189259384\\
545.01	0.00677898761538313\\
546.01	0.00681487401117106\\
547.01	0.00685149473646011\\
548.01	0.00688886358294855\\
549.01	0.00692699446479224\\
550.01	0.00696590140402426\\
551.01	0.00700559851393984\\
552.01	0.0070460999802358\\
553.01	0.00708742003967412\\
554.01	0.00712957295601932\\
555.01	0.00717257299297346\\
556.01	0.00721643438381055\\
557.01	0.00726117129738661\\
558.01	0.00730679780017085\\
559.01	0.00735332781391837\\
560.01	0.00740077506856995\\
561.01	0.00744915304993686\\
562.01	0.0074984749416927\\
563.01	0.00754875356116406\\
564.01	0.00760000128837737\\
565.01	0.00765222998778898\\
566.01	0.00770545092209638\\
567.01	0.00775967465750245\\
568.01	0.0078149109597867\\
569.01	0.00787116868052467\\
570.01	0.00792845563279784\\
571.01	0.0079867784557522\\
572.01	0.00804614246739978\\
573.01	0.00810655150512116\\
574.01	0.00816800775342491\\
575.01	0.00823051155866612\\
576.01	0.00829406123062584\\
577.01	0.00835865283113084\\
578.01	0.00842427995026335\\
579.01	0.00849093347119724\\
580.01	0.00855860132533547\\
581.01	0.00862726824024612\\
582.01	0.00869691548395165\\
583.01	0.00876752061047502\\
584.01	0.00883905721325699\\
585.01	0.00891149469522319\\
586.01	0.00898479806700421\\
587.01	0.00905892778823375\\
588.01	0.0091338396711344\\
589.01	0.00920948487095468\\
590.01	0.00928580999449757\\
591.01	0.00936275736629181\\
592.01	0.00944026550228819\\
593.01	0.00951826985378823\\
594.01	0.00959670390021289\\
595.01	0.00967550068901628\\
596.01	0.00975459494542325\\
597.01	0.00983382663319375\\
598.01	0.00990866201848071\\
599.01	0.00997087280416276\\
599.02	0.00997138072163725\\
599.03	0.009971885576258\\
599.04	0.00997238733820211\\
599.05	0.00997288597735272\\
599.06	0.00997338146329604\\
599.07	0.00997387376531845\\
599.08	0.00997436285240349\\
599.09	0.00997484869322892\\
599.1	0.00997533125616361\\
599.11	0.00997581050926455\\
599.12	0.00997628642027372\\
599.13	0.00997675895661498\\
599.14	0.0099772280853909\\
599.15	0.00997769377337961\\
599.16	0.00997815598703157\\
599.17	0.00997861469246631\\
599.18	0.00997906985546915\\
599.19	0.00997952144148792\\
599.2	0.00997996941562957\\
599.21	0.00998041374265684\\
599.22	0.0099808543869848\\
599.23	0.00998129131267744\\
599.24	0.00998172448344418\\
599.25	0.00998215386263636\\
599.26	0.00998257941258353\\
599.27	0.00998300109279825\\
599.28	0.00998341886238981\\
599.29	0.00998383268006024\\
599.3	0.00998424250410032\\
599.31	0.00998464829238543\\
599.32	0.00998505000237151\\
599.33	0.00998544759109087\\
599.34	0.00998584101514799\\
599.35	0.00998623023071532\\
599.36	0.00998661519352895\\
599.37	0.00998699585888436\\
599.38	0.00998737218163197\\
599.39	0.00998774411617281\\
599.4	0.00998811161645403\\
599.41	0.00998847463596443\\
599.42	0.0099888331277299\\
599.43	0.00998918704430885\\
599.44	0.00998953633778757\\
599.45	0.00998988095977558\\
599.46	0.00999022086140088\\
599.47	0.00999055599330522\\
599.48	0.00999088630563925\\
599.49	0.00999121174805768\\
599.5	0.00999153226971435\\
599.51	0.0099918478192573\\
599.52	0.00999215834482374\\
599.53	0.00999246379403503\\
599.54	0.0099927641139915\\
599.55	0.00999305925126738\\
599.56	0.00999334915190553\\
599.57	0.0099936337614122\\
599.58	0.00999391302475173\\
599.59	0.00999418688634118\\
599.6	0.00999445529004489\\
599.61	0.00999471817916904\\
599.62	0.00999497549645613\\
599.63	0.00999522718407934\\
599.64	0.009995473183637\\
599.65	0.00999571343614678\\
599.66	0.00999594788204004\\
599.67	0.00999617646115599\\
599.68	0.0099963991127358\\
599.69	0.00999661577541675\\
599.7	0.00999682638722618\\
599.71	0.00999703088557551\\
599.72	0.00999722920725411\\
599.73	0.00999742128842315\\
599.74	0.00999760706460942\\
599.75	0.00999778647069897\\
599.76	0.00999795944093086\\
599.77	0.0099981259088907\\
599.78	0.0099982858075042\\
599.79	0.00999843906903064\\
599.8	0.00999858562505627\\
599.81	0.00999872540648767\\
599.82	0.00999885834354498\\
599.83	0.00999898436575515\\
599.84	0.00999910340194508\\
599.85	0.00999921538023465\\
599.86	0.00999932022802977\\
599.87	0.00999941787201528\\
599.88	0.00999950823814785\\
599.89	0.00999959125164875\\
599.9	0.00999966683699656\\
599.91	0.00999973491791987\\
599.92	0.0099997954173898\\
599.93	0.00999984825761255\\
599.94	0.00999989336002181\\
599.95	0.00999993064527112\\
599.96	0.00999996003322615\\
599.97	0.00999998144295691\\
599.98	0.00999999479272987\\
599.99	0.01\\
600	0.01\\
};
\addplot [color=mycolor4,solid,forget plot]
  table[row sep=crcr]{%
0.01	0.00496755019417924\\
1.01	0.0049675512405473\\
2.01	0.00496755230832586\\
3.01	0.00496755339795235\\
4.01	0.00496755450987339\\
5.01	0.00496755564454414\\
6.01	0.00496755680242982\\
7.01	0.00496755798400442\\
8.01	0.00496755918975203\\
9.01	0.00496756042016645\\
10.01	0.0049675616757516\\
11.01	0.0049675629570216\\
12.01	0.00496756426450115\\
13.01	0.00496756559872562\\
14.01	0.00496756696024118\\
15.01	0.00496756834960534\\
16.01	0.00496756976738685\\
17.01	0.00496757121416612\\
18.01	0.0049675726905352\\
19.01	0.00496757419709846\\
20.01	0.00496757573447246\\
21.01	0.00496757730328634\\
22.01	0.00496757890418198\\
23.01	0.00496758053781456\\
24.01	0.00496758220485252\\
25.01	0.0049675839059775\\
26.01	0.00496758564188582\\
27.01	0.00496758741328719\\
28.01	0.00496758922090651\\
29.01	0.00496759106548292\\
30.01	0.0049675929477707\\
31.01	0.00496759486853966\\
32.01	0.00496759682857525\\
33.01	0.00496759882867888\\
34.01	0.00496760086966812\\
35.01	0.00496760295237748\\
36.01	0.00496760507765827\\
37.01	0.00496760724637924\\
38.01	0.00496760945942672\\
39.01	0.00496761171770513\\
40.01	0.00496761402213719\\
41.01	0.0049676163736647\\
42.01	0.00496761877324842\\
43.01	0.00496762122186858\\
44.01	0.0049676237205255\\
45.01	0.00496762627023999\\
46.01	0.00496762887205328\\
47.01	0.00496763152702795\\
48.01	0.0049676342362482\\
49.01	0.00496763700082062\\
50.01	0.00496763982187355\\
51.01	0.00496764270055901\\
52.01	0.00496764563805219\\
53.01	0.004967648635552\\
54.01	0.00496765169428197\\
55.01	0.00496765481549043\\
56.01	0.00496765800045121\\
57.01	0.00496766125046365\\
58.01	0.00496766456685403\\
59.01	0.00496766795097521\\
60.01	0.0049676714042078\\
61.01	0.0049676749279603\\
62.01	0.00496767852366987\\
63.01	0.00496768219280302\\
64.01	0.0049676859368558\\
65.01	0.00496768975735483\\
66.01	0.00496769365585771\\
67.01	0.00496769763395352\\
68.01	0.00496770169326417\\
69.01	0.00496770583544404\\
70.01	0.00496771006218102\\
71.01	0.00496771437519765\\
72.01	0.0049677187762512\\
73.01	0.00496772326713463\\
74.01	0.00496772784967736\\
75.01	0.00496773252574605\\
76.01	0.00496773729724513\\
77.01	0.00496774216611764\\
78.01	0.00496774713434632\\
79.01	0.00496775220395372\\
80.01	0.0049677573770037\\
81.01	0.004967762655602\\
82.01	0.00496776804189705\\
83.01	0.00496777353808089\\
84.01	0.00496777914638992\\
85.01	0.00496778486910593\\
86.01	0.00496779070855712\\
87.01	0.00496779666711872\\
88.01	0.0049678027472143\\
89.01	0.0049678089513164\\
90.01	0.00496781528194765\\
91.01	0.00496782174168196\\
92.01	0.00496782833314528\\
93.01	0.00496783505901684\\
94.01	0.00496784192203031\\
95.01	0.00496784892497434\\
96.01	0.00496785607069442\\
97.01	0.00496786336209352\\
98.01	0.00496787080213341\\
99.01	0.00496787839383603\\
100.01	0.00496788614028412\\
101.01	0.00496789404462306\\
102.01	0.00496790211006197\\
103.01	0.00496791033987477\\
104.01	0.00496791873740164\\
105.01	0.00496792730605047\\
106.01	0.00496793604929797\\
107.01	0.00496794497069106\\
108.01	0.0049679540738487\\
109.01	0.00496796336246286\\
110.01	0.00496797284030015\\
111.01	0.00496798251120333\\
112.01	0.00496799237909287\\
113.01	0.00496800244796837\\
114.01	0.0049680127219105\\
115.01	0.004968023205082\\
116.01	0.00496803390173005\\
117.01	0.00496804481618734\\
118.01	0.00496805595287435\\
119.01	0.00496806731630038\\
120.01	0.00496807891106613\\
121.01	0.00496809074186499\\
122.01	0.00496810281348509\\
123.01	0.00496811513081118\\
124.01	0.00496812769882653\\
125.01	0.00496814052261496\\
126.01	0.00496815360736277\\
127.01	0.00496816695836078\\
128.01	0.00496818058100656\\
129.01	0.00496819448080647\\
130.01	0.00496820866337745\\
131.01	0.00496822313445001\\
132.01	0.00496823789987005\\
133.01	0.00496825296560104\\
134.01	0.00496826833772641\\
135.01	0.00496828402245247\\
136.01	0.00496830002610989\\
137.01	0.00496831635515706\\
138.01	0.00496833301618216\\
139.01	0.0049683500159058\\
140.01	0.0049683673611837\\
141.01	0.0049683850590095\\
142.01	0.00496840311651728\\
143.01	0.00496842154098463\\
144.01	0.00496844033983493\\
145.01	0.00496845952064086\\
146.01	0.0049684790911274\\
147.01	0.00496849905917386\\
148.01	0.00496851943281836\\
149.01	0.00496854022025964\\
150.01	0.00496856142986134\\
151.01	0.00496858307015448\\
152.01	0.00496860514984107\\
153.01	0.00496862767779743\\
154.01	0.00496865066307772\\
155.01	0.00496867411491754\\
156.01	0.004968698042737\\
157.01	0.00496872245614495\\
158.01	0.00496874736494229\\
159.01	0.00496877277912618\\
160.01	0.0049687987088929\\
161.01	0.00496882516464336\\
162.01	0.00496885215698574\\
163.01	0.00496887969674001\\
164.01	0.00496890779494216\\
165.01	0.00496893646284843\\
166.01	0.00496896571193949\\
167.01	0.00496899555392501\\
168.01	0.00496902600074794\\
169.01	0.00496905706458917\\
170.01	0.00496908875787194\\
171.01	0.00496912109326691\\
172.01	0.0049691540836969\\
173.01	0.00496918774234145\\
174.01	0.00496922208264239\\
175.01	0.00496925711830809\\
176.01	0.00496929286331974\\
177.01	0.00496932933193543\\
178.01	0.00496936653869663\\
179.01	0.00496940449843285\\
180.01	0.00496944322626755\\
181.01	0.00496948273762356\\
182.01	0.0049695230482293\\
183.01	0.0049695641741242\\
184.01	0.00496960613166485\\
185.01	0.00496964893753113\\
186.01	0.00496969260873233\\
187.01	0.00496973716261352\\
188.01	0.00496978261686203\\
189.01	0.00496982898951351\\
190.01	0.0049698762989593\\
191.01	0.00496992456395258\\
192.01	0.00496997380361565\\
193.01	0.00497002403744673\\
194.01	0.00497007528532704\\
195.01	0.00497012756752824\\
196.01	0.00497018090471969\\
197.01	0.00497023531797575\\
198.01	0.00497029082878378\\
199.01	0.00497034745905163\\
200.01	0.00497040523111546\\
201.01	0.00497046416774798\\
202.01	0.00497052429216663\\
203.01	0.00497058562804158\\
204.01	0.00497064819950424\\
205.01	0.00497071203115605\\
206.01	0.00497077714807698\\
207.01	0.0049708435758344\\
208.01	0.0049709113404918\\
209.01	0.00497098046861857\\
210.01	0.00497105098729865\\
211.01	0.00497112292414026\\
212.01	0.00497119630728528\\
213.01	0.00497127116541907\\
214.01	0.00497134752778053\\
215.01	0.00497142542417174\\
216.01	0.00497150488496835\\
217.01	0.00497158594112979\\
218.01	0.00497166862420986\\
219.01	0.00497175296636723\\
220.01	0.00497183900037647\\
221.01	0.00497192675963833\\
222.01	0.00497201627819194\\
223.01	0.00497210759072489\\
224.01	0.00497220073258527\\
225.01	0.00497229573979346\\
226.01	0.00497239264905308\\
227.01	0.00497249149776331\\
228.01	0.00497259232403127\\
229.01	0.00497269516668333\\
230.01	0.00497280006527831\\
231.01	0.00497290706011921\\
232.01	0.00497301619226651\\
233.01	0.00497312750355068\\
234.01	0.00497324103658504\\
235.01	0.00497335683477877\\
236.01	0.00497347494235049\\
237.01	0.00497359540434152\\
238.01	0.00497371826662901\\
239.01	0.00497384357594036\\
240.01	0.00497397137986627\\
241.01	0.00497410172687499\\
242.01	0.0049742346663265\\
243.01	0.00497437024848646\\
244.01	0.00497450852454065\\
245.01	0.00497464954660931\\
246.01	0.00497479336776174\\
247.01	0.00497494004203085\\
248.01	0.00497508962442793\\
249.01	0.00497524217095781\\
250.01	0.00497539773863309\\
251.01	0.00497555638549001\\
252.01	0.00497571817060258\\
253.01	0.00497588315409864\\
254.01	0.0049760513971745\\
255.01	0.00497622296211021\\
256.01	0.00497639791228509\\
257.01	0.00497657631219313\\
258.01	0.00497675822745788\\
259.01	0.00497694372484804\\
260.01	0.00497713287229314\\
261.01	0.00497732573889832\\
262.01	0.00497752239496002\\
263.01	0.00497772291198105\\
264.01	0.00497792736268579\\
265.01	0.00497813582103557\\
266.01	0.0049783483622434\\
267.01	0.00497856506278914\\
268.01	0.00497878600043404\\
269.01	0.00497901125423583\\
270.01	0.00497924090456307\\
271.01	0.00497947503310968\\
272.01	0.00497971372290859\\
273.01	0.00497995705834638\\
274.01	0.00498020512517662\\
275.01	0.00498045801053344\\
276.01	0.00498071580294483\\
277.01	0.00498097859234527\\
278.01	0.00498124647008918\\
279.01	0.00498151952896214\\
280.01	0.00498179786319315\\
281.01	0.00498208156846677\\
282.01	0.00498237074193363\\
283.01	0.00498266548222114\\
284.01	0.00498296588944425\\
285.01	0.00498327206521498\\
286.01	0.00498358411265164\\
287.01	0.00498390213638834\\
288.01	0.00498422624258228\\
289.01	0.00498455653892249\\
290.01	0.00498489313463672\\
291.01	0.0049852361404982\\
292.01	0.00498558566883212\\
293.01	0.00498594183352066\\
294.01	0.00498630475000855\\
295.01	0.0049866745353076\\
296.01	0.0049870513080004\\
297.01	0.00498743518824385\\
298.01	0.00498782629777228\\
299.01	0.00498822475989989\\
300.01	0.00498863069952266\\
301.01	0.00498904424312037\\
302.01	0.00498946551875726\\
303.01	0.00498989465608378\\
304.01	0.00499033178633712\\
305.01	0.00499077704234251\\
306.01	0.00499123055851343\\
307.01	0.00499169247085336\\
308.01	0.00499216291695664\\
309.01	0.00499264203601017\\
310.01	0.00499312996879676\\
311.01	0.00499362685769801\\
312.01	0.00499413284669889\\
313.01	0.0049946480813938\\
314.01	0.0049951727089943\\
315.01	0.00499570687833882\\
316.01	0.00499625073990526\\
317.01	0.00499680444582594\\
318.01	0.00499736814990544\\
319.01	0.00499794200764365\\
320.01	0.00499852617626204\\
321.01	0.00499912081473511\\
322.01	0.00499972608382755\\
323.01	0.00500034214613818\\
324.01	0.00500096916614998\\
325.01	0.00500160731028902\\
326.01	0.00500225674699113\\
327.01	0.0050029176467794\\
328.01	0.00500359018235161\\
329.01	0.00500427452867935\\
330.01	0.00500497086312048\\
331.01	0.00500567936554518\\
332.01	0.00500640021847691\\
333.01	0.00500713360725059\\
334.01	0.00500787972018706\\
335.01	0.00500863874878736\\
336.01	0.00500941088794696\\
337.01	0.00501019633619076\\
338.01	0.00501099529593065\\
339.01	0.00501180797374665\\
340.01	0.0050126345806918\\
341.01	0.0050134753326232\\
342.01	0.00501433045055741\\
343.01	0.00501520016105224\\
344.01	0.00501608469661437\\
345.01	0.0050169842961323\\
346.01	0.0050178992053329\\
347.01	0.00501882967726136\\
348.01	0.00501977597278112\\
349.01	0.00502073836109178\\
350.01	0.00502171712025976\\
351.01	0.0050227125377582\\
352.01	0.00502372491100889\\
353.01	0.00502475454791997\\
354.01	0.00502580176740999\\
355.01	0.00502686689990904\\
356.01	0.00502795028782525\\
357.01	0.00502905228596373\\
358.01	0.00503017326188299\\
359.01	0.0050313135961752\\
360.01	0.00503247368265083\\
361.01	0.00503365392841168\\
362.01	0.00503485475379381\\
363.01	0.00503607659216244\\
364.01	0.00503731988954131\\
365.01	0.00503858510406292\\
366.01	0.00503987270522489\\
367.01	0.00504118317294898\\
368.01	0.005042516996438\\
369.01	0.00504387467284059\\
370.01	0.00504525670573876\\
371.01	0.00504666360348999\\
372.01	0.00504809587746541\\
373.01	0.00504955404024386\\
374.01	0.00505103860383685\\
375.01	0.0050525500780346\\
376.01	0.0050540889689753\\
377.01	0.0050556557780482\\
378.01	0.00505725100123884\\
379.01	0.0050588751290104\\
380.01	0.0050605286467818\\
381.01	0.00506221203601141\\
382.01	0.00506392577581308\\
383.01	0.00506567034493595\\
384.01	0.00506744622382201\\
385.01	0.00506925389636752\\
386.01	0.00507109385100737\\
387.01	0.00507296658105322\\
388.01	0.0050748725847341\\
389.01	0.00507681236521759\\
390.01	0.00507878643064001\\
391.01	0.00508079529414431\\
392.01	0.00508283947392883\\
393.01	0.00508491949330739\\
394.01	0.0050870358807824\\
395.01	0.00508918917013286\\
396.01	0.00509137990051904\\
397.01	0.00509360861660459\\
398.01	0.00509587586870029\\
399.01	0.00509818221292972\\
400.01	0.00510052821141903\\
401.01	0.00510291443251547\\
402.01	0.005105341451034\\
403.01	0.00510780984853788\\
404.01	0.00511032021365325\\
405.01	0.00511287314242206\\
406.01	0.00511546923869705\\
407.01	0.00511810911457931\\
408.01	0.00512079339090511\\
409.01	0.00512352269778132\\
410.01	0.00512629767517689\\
411.01	0.00512911897356907\\
412.01	0.00513198725465157\\
413.01	0.00513490319210574\\
414.01	0.00513786747243726\\
415.01	0.00514088079588267\\
416.01	0.00514394387738673\\
417.01	0.00514705744765431\\
418.01	0.00515022225427615\\
419.01	0.00515343906293231\\
420.01	0.00515670865867184\\
421.01	0.0051600318472699\\
422.01	0.00516340945666009\\
423.01	0.00516684233844038\\
424.01	0.00517033136945048\\
425.01	0.00517387745341429\\
426.01	0.00517748152264334\\
427.01	0.00518114453979394\\
428.01	0.00518486749966721\\
429.01	0.00518865143104338\\
430.01	0.00519249739853767\\
431.01	0.00519640650446129\\
432.01	0.00520037989067396\\
433.01	0.00520441874040711\\
434.01	0.00520852428003953\\
435.01	0.00521269778080247\\
436.01	0.00521694056039095\\
437.01	0.0052212539844582\\
438.01	0.00522563946796589\\
439.01	0.00523009847636763\\
440.01	0.00523463252660038\\
441.01	0.00523924318786104\\
442.01	0.00524393208215018\\
443.01	0.00524870088456564\\
444.01	0.00525355132333839\\
445.01	0.00525848517960635\\
446.01	0.00526350428693462\\
447.01	0.00526861053059645\\
448.01	0.00527380584664806\\
449.01	0.00527909222083719\\
450.01	0.00528447168740513\\
451.01	0.00528994632785534\\
452.01	0.00529551826977522\\
453.01	0.00530118968581521\\
454.01	0.0053069627929332\\
455.01	0.0053128398520253\\
456.01	0.00531882316805671\\
457.01	0.00532491509080059\\
458.01	0.00533111801626969\\
459.01	0.00533743438889314\\
460.01	0.00534386670444288\\
461.01	0.00535041751365456\\
462.01	0.00535708942641741\\
463.01	0.00536388511633094\\
464.01	0.0053708073253608\\
465.01	0.00537785886827462\\
466.01	0.0053850426365309\\
467.01	0.00539236160134585\\
468.01	0.00539981881579216\\
469.01	0.00540741741598878\\
470.01	0.00541516062166209\\
471.01	0.00542305173640308\\
472.01	0.00543109414774612\\
473.01	0.00543929132709937\\
474.01	0.00544764682954967\\
475.01	0.00545616429357375\\
476.01	0.00546484744069134\\
477.01	0.00547370007510466\\
478.01	0.00548272608336978\\
479.01	0.00549192943415226\\
480.01	0.00550131417811883\\
481.01	0.00551088444801663\\
482.01	0.00552064445898752\\
483.01	0.00553059850915716\\
484.01	0.00554075098052845\\
485.01	0.00555110634019344\\
486.01	0.00556166914186174\\
487.01	0.00557244402768026\\
488.01	0.0055834357303013\\
489.01	0.00559464907513395\\
490.01	0.00560608898269654\\
491.01	0.00561776047097949\\
492.01	0.0056296686577266\\
493.01	0.00564181876255559\\
494.01	0.00565421610886396\\
495.01	0.00566686612550699\\
496.01	0.00567977434828099\\
497.01	0.00569294642129253\\
498.01	0.00570638809832281\\
499.01	0.00572010524429577\\
500.01	0.0057341038369189\\
501.01	0.00574838996851797\\
502.01	0.00576296984806557\\
503.01	0.00577784980339512\\
504.01	0.00579303628358452\\
505.01	0.00580853586148709\\
506.01	0.00582435523638102\\
507.01	0.00584050123670836\\
508.01	0.0058569808228702\\
509.01	0.00587380109004865\\
510.01	0.00589096927103139\\
511.01	0.00590849273902176\\
512.01	0.00592637901042681\\
513.01	0.00594463574762613\\
514.01	0.00596327076173313\\
515.01	0.00598229201536413\\
516.01	0.00600170762542973\\
517.01	0.0060215258659571\\
518.01	0.00604175517093986\\
519.01	0.00606240413720248\\
520.01	0.00608348152725796\\
521.01	0.00610499627213941\\
522.01	0.00612695747418071\\
523.01	0.00614937440973107\\
524.01	0.0061722565317805\\
525.01	0.00619561347248153\\
526.01	0.00621945504555035\\
527.01	0.00624379124853081\\
528.01	0.00626863226490573\\
529.01	0.00629398846603425\\
530.01	0.0063198704128945\\
531.01	0.00634628885760222\\
532.01	0.00637325474467523\\
533.01	0.00640077921200628\\
534.01	0.00642887359150497\\
535.01	0.00645754940936542\\
536.01	0.00648681838591415\\
537.01	0.00651669243498613\\
538.01	0.00654718366277532\\
539.01	0.00657830436609863\\
540.01	0.00661006703000588\\
541.01	0.00664248432466176\\
542.01	0.00667556910141767\\
543.01	0.00670933438798234\\
544.01	0.00674379338259237\\
545.01	0.00677895944707115\\
546.01	0.00681484609865593\\
547.01	0.00685146700046133\\
548.01	0.00688883595043263\\
549.01	0.00692696686862976\\
550.01	0.00696587378266646\\
551.01	0.00700557081111293\\
552.01	0.00704607214465176\\
553.01	0.00708739202475621\\
554.01	0.0071295447196411\\
555.01	0.00717254449721021\\
556.01	0.00721640559470368\\
557.01	0.00726114218471995\\
558.01	0.00730676833725931\\
559.01	0.00735329797740891\\
560.01	0.00740074483825644\\
561.01	0.0074491224085882\\
562.01	0.00749844387489634\\
563.01	0.00754872205718511\\
564.01	0.00759996933803445\\
565.01	0.00765219758434847\\
566.01	0.00770541806118489\\
567.01	0.00775964133703997\\
568.01	0.00781487717993988\\
569.01	0.00787113444368163\\
570.01	0.00792842094356467\\
571.01	0.00798674332097159\\
572.01	0.00804610689619114\\
573.01	0.00810651550894171\\
574.01	0.00816797134615157\\
575.01	0.00823047475669562\\
576.01	0.00829402405299119\\
577.01	0.00835861529963226\\
578.01	0.00842424208960891\\
579.01	0.00849089530914933\\
580.01	0.00855856289285696\\
581.01	0.00862722957163953\\
582.01	0.00869687661698168\\
583.01	0.00876748158646345\\
584.01	0.00883901807713575\\
585.01	0.00891145549552764\\
586.01	0.00898475885578457\\
587.01	0.00905888862085642\\
588.01	0.00913380060593759\\
589.01	0.00920944596871416\\
590.01	0.00928577131764595\\
591.01	0.00936271897782057\\
592.01	0.00944022746424321\\
593.01	0.00951823222524592\\
594.01	0.00959666673459481\\
595.01	0.00967546403056262\\
596.01	0.00975455882459834\\
597.01	0.00983380640043464\\
598.01	0.00990866201848071\\
599.01	0.00997087280416276\\
599.02	0.00997138072163725\\
599.03	0.009971885576258\\
599.04	0.00997238733820211\\
599.05	0.00997288597735272\\
599.06	0.00997338146329604\\
599.07	0.00997387376531845\\
599.08	0.00997436285240349\\
599.09	0.00997484869322892\\
599.1	0.00997533125616361\\
599.11	0.00997581050926455\\
599.12	0.00997628642027372\\
599.13	0.00997675895661497\\
599.14	0.0099772280853909\\
599.15	0.00997769377337961\\
599.16	0.00997815598703157\\
599.17	0.00997861469246631\\
599.18	0.00997906985546915\\
599.19	0.00997952144148792\\
599.2	0.00997996941562957\\
599.21	0.00998041374265684\\
599.22	0.0099808543869848\\
599.23	0.00998129131267744\\
599.24	0.00998172448344418\\
599.25	0.00998215386263636\\
599.26	0.00998257941258354\\
599.27	0.00998300109279825\\
599.28	0.00998341886238981\\
599.29	0.00998383268006024\\
599.3	0.00998424250410032\\
599.31	0.00998464829238543\\
599.32	0.00998505000237151\\
599.33	0.00998544759109087\\
599.34	0.00998584101514799\\
599.35	0.00998623023071532\\
599.36	0.00998661519352896\\
599.37	0.00998699585888436\\
599.38	0.00998737218163197\\
599.39	0.00998774411617281\\
599.4	0.00998811161645403\\
599.41	0.00998847463596443\\
599.42	0.0099888331277299\\
599.43	0.00998918704430885\\
599.44	0.00998953633778757\\
599.45	0.00998988095977558\\
599.46	0.00999022086140088\\
599.47	0.00999055599330522\\
599.48	0.00999088630563925\\
599.49	0.00999121174805768\\
599.5	0.00999153226971435\\
599.51	0.0099918478192573\\
599.52	0.00999215834482374\\
599.53	0.00999246379403503\\
599.54	0.0099927641139915\\
599.55	0.00999305925126738\\
599.56	0.00999334915190553\\
599.57	0.0099936337614122\\
599.58	0.00999391302475173\\
599.59	0.00999418688634118\\
599.6	0.00999445529004489\\
599.61	0.00999471817916904\\
599.62	0.00999497549645613\\
599.63	0.00999522718407934\\
599.64	0.009995473183637\\
599.65	0.00999571343614678\\
599.66	0.00999594788204004\\
599.67	0.00999617646115598\\
599.68	0.0099963991127358\\
599.69	0.00999661577541675\\
599.7	0.00999682638722618\\
599.71	0.00999703088557551\\
599.72	0.00999722920725411\\
599.73	0.00999742128842315\\
599.74	0.00999760706460942\\
599.75	0.00999778647069897\\
599.76	0.00999795944093086\\
599.77	0.0099981259088907\\
599.78	0.0099982858075042\\
599.79	0.00999843906903064\\
599.8	0.00999858562505627\\
599.81	0.00999872540648767\\
599.82	0.00999885834354498\\
599.83	0.00999898436575515\\
599.84	0.00999910340194508\\
599.85	0.00999921538023465\\
599.86	0.00999932022802977\\
599.87	0.00999941787201528\\
599.88	0.00999950823814785\\
599.89	0.00999959125164875\\
599.9	0.00999966683699656\\
599.91	0.00999973491791987\\
599.92	0.0099997954173898\\
599.93	0.00999984825761255\\
599.94	0.00999989336002181\\
599.95	0.00999993064527112\\
599.96	0.00999996003322615\\
599.97	0.00999998144295691\\
599.98	0.00999999479272987\\
599.99	0.01\\
600	0.01\\
};
\addplot [color=mycolor5,solid,forget plot]
  table[row sep=crcr]{%
0.01	0.00490905749971206\\
1.01	0.00490905860197839\\
2.01	0.00490905972694872\\
3.01	0.00490906087509092\\
4.01	0.00490906204688202\\
5.01	0.00490906324280954\\
6.01	0.00490906446337047\\
7.01	0.00490906570907243\\
8.01	0.00490906698043314\\
9.01	0.00490906827798124\\
10.01	0.00490906960225622\\
11.01	0.00490907095380874\\
12.01	0.00490907233320062\\
13.01	0.00490907374100548\\
14.01	0.00490907517780848\\
15.01	0.00490907664420701\\
16.01	0.00490907814081074\\
17.01	0.00490907966824188\\
18.01	0.00490908122713578\\
19.01	0.0049090828181402\\
20.01	0.00490908444191679\\
21.01	0.00490908609914072\\
22.01	0.00490908779050088\\
23.01	0.00490908951670066\\
24.01	0.00490909127845751\\
25.01	0.00490909307650426\\
26.01	0.00490909491158817\\
27.01	0.00490909678447236\\
28.01	0.00490909869593543\\
29.01	0.00490910064677222\\
30.01	0.00490910263779389\\
31.01	0.00490910466982809\\
32.01	0.00490910674371986\\
33.01	0.00490910886033126\\
34.01	0.00490911102054242\\
35.01	0.00490911322525142\\
36.01	0.00490911547537501\\
37.01	0.0049091177718488\\
38.01	0.0049091201156274\\
39.01	0.00490912250768544\\
40.01	0.00490912494901743\\
41.01	0.00490912744063825\\
42.01	0.00490912998358396\\
43.01	0.0049091325789118\\
44.01	0.00490913522770087\\
45.01	0.00490913793105228\\
46.01	0.00490914069009012\\
47.01	0.00490914350596146\\
48.01	0.00490914637983708\\
49.01	0.00490914931291167\\
50.01	0.00490915230640494\\
51.01	0.0049091553615612\\
52.01	0.00490915847965061\\
53.01	0.00490916166196961\\
54.01	0.00490916490984129\\
55.01	0.00490916822461597\\
56.01	0.00490917160767151\\
57.01	0.00490917506041461\\
58.01	0.00490917858428062\\
59.01	0.00490918218073452\\
60.01	0.00490918585127144\\
61.01	0.00490918959741736\\
62.01	0.00490919342072964\\
63.01	0.00490919732279765\\
64.01	0.00490920130524378\\
65.01	0.00490920536972335\\
66.01	0.00490920951792605\\
67.01	0.00490921375157641\\
68.01	0.00490921807243412\\
69.01	0.00490922248229551\\
70.01	0.00490922698299386\\
71.01	0.00490923157639973\\
72.01	0.00490923626442272\\
73.01	0.00490924104901155\\
74.01	0.00490924593215501\\
75.01	0.00490925091588273\\
76.01	0.00490925600226642\\
77.01	0.00490926119342014\\
78.01	0.00490926649150164\\
79.01	0.0049092718987131\\
80.01	0.00490927741730177\\
81.01	0.00490928304956145\\
82.01	0.00490928879783287\\
83.01	0.00490929466450511\\
84.01	0.00490930065201656\\
85.01	0.00490930676285549\\
86.01	0.00490931299956143\\
87.01	0.00490931936472637\\
88.01	0.00490932586099546\\
89.01	0.0049093324910683\\
90.01	0.00490933925770003\\
91.01	0.00490934616370248\\
92.01	0.00490935321194527\\
93.01	0.00490936040535726\\
94.01	0.00490936774692727\\
95.01	0.00490937523970579\\
96.01	0.00490938288680609\\
97.01	0.00490939069140538\\
98.01	0.00490939865674618\\
99.01	0.00490940678613766\\
100.01	0.00490941508295725\\
101.01	0.00490942355065171\\
102.01	0.00490943219273871\\
103.01	0.00490944101280809\\
104.01	0.00490945001452382\\
105.01	0.00490945920162493\\
106.01	0.00490946857792736\\
107.01	0.00490947814732567\\
108.01	0.00490948791379411\\
109.01	0.00490949788138902\\
110.01	0.0049095080542499\\
111.01	0.0049095184366014\\
112.01	0.00490952903275502\\
113.01	0.00490953984711079\\
114.01	0.00490955088415899\\
115.01	0.00490956214848251\\
116.01	0.00490957364475825\\
117.01	0.00490958537775917\\
118.01	0.00490959735235603\\
119.01	0.00490960957352014\\
120.01	0.00490962204632456\\
121.01	0.00490963477594673\\
122.01	0.00490964776767029\\
123.01	0.00490966102688729\\
124.01	0.00490967455910074\\
125.01	0.00490968836992635\\
126.01	0.00490970246509535\\
127.01	0.00490971685045637\\
128.01	0.00490973153197836\\
129.01	0.00490974651575266\\
130.01	0.00490976180799554\\
131.01	0.00490977741505096\\
132.01	0.0049097933433928\\
133.01	0.00490980959962782\\
134.01	0.00490982619049835\\
135.01	0.00490984312288513\\
136.01	0.00490986040380985\\
137.01	0.00490987804043802\\
138.01	0.00490989604008256\\
139.01	0.00490991441020583\\
140.01	0.00490993315842334\\
141.01	0.00490995229250674\\
142.01	0.00490997182038666\\
143.01	0.00490999175015648\\
144.01	0.00491001209007534\\
145.01	0.00491003284857165\\
146.01	0.00491005403424595\\
147.01	0.00491007565587574\\
148.01	0.00491009772241746\\
149.01	0.0049101202430113\\
150.01	0.00491014322698433\\
151.01	0.00491016668385458\\
152.01	0.00491019062333502\\
153.01	0.00491021505533708\\
154.01	0.00491023998997492\\
155.01	0.00491026543756963\\
156.01	0.00491029140865341\\
157.01	0.00491031791397366\\
158.01	0.00491034496449737\\
159.01	0.00491037257141587\\
160.01	0.00491040074614905\\
161.01	0.00491042950034998\\
162.01	0.00491045884590979\\
163.01	0.00491048879496239\\
164.01	0.00491051935988952\\
165.01	0.0049105505533255\\
166.01	0.00491058238816239\\
167.01	0.00491061487755542\\
168.01	0.00491064803492815\\
169.01	0.00491068187397732\\
170.01	0.0049107164086796\\
171.01	0.00491075165329606\\
172.01	0.00491078762237854\\
173.01	0.00491082433077526\\
174.01	0.00491086179363659\\
175.01	0.00491090002642164\\
176.01	0.004910939044904\\
177.01	0.00491097886517806\\
178.01	0.00491101950366565\\
179.01	0.00491106097712254\\
180.01	0.00491110330264509\\
181.01	0.00491114649767722\\
182.01	0.00491119058001678\\
183.01	0.00491123556782333\\
184.01	0.00491128147962502\\
185.01	0.00491132833432602\\
186.01	0.00491137615121377\\
187.01	0.00491142494996694\\
188.01	0.00491147475066307\\
189.01	0.0049115255737865\\
190.01	0.00491157744023633\\
191.01	0.00491163037133473\\
192.01	0.0049116843888356\\
193.01	0.00491173951493254\\
194.01	0.00491179577226811\\
195.01	0.00491185318394217\\
196.01	0.00491191177352146\\
197.01	0.00491197156504844\\
198.01	0.00491203258305056\\
199.01	0.00491209485255008\\
200.01	0.00491215839907388\\
201.01	0.00491222324866275\\
202.01	0.00491228942788217\\
203.01	0.00491235696383208\\
204.01	0.00491242588415776\\
205.01	0.00491249621705986\\
206.01	0.00491256799130579\\
207.01	0.00491264123624047\\
208.01	0.0049127159817977\\
209.01	0.00491279225851149\\
210.01	0.00491287009752768\\
211.01	0.00491294953061565\\
212.01	0.00491303059018065\\
213.01	0.00491311330927587\\
214.01	0.00491319772161466\\
215.01	0.00491328386158379\\
216.01	0.00491337176425578\\
217.01	0.00491346146540234\\
218.01	0.00491355300150757\\
219.01	0.0049136464097818\\
220.01	0.00491374172817487\\
221.01	0.00491383899539088\\
222.01	0.00491393825090128\\
223.01	0.00491403953496078\\
224.01	0.00491414288862115\\
225.01	0.00491424835374625\\
226.01	0.00491435597302772\\
227.01	0.00491446578999996\\
228.01	0.00491457784905612\\
229.01	0.00491469219546373\\
230.01	0.00491480887538116\\
231.01	0.00491492793587396\\
232.01	0.00491504942493175\\
233.01	0.00491517339148445\\
234.01	0.0049152998854203\\
235.01	0.00491542895760302\\
236.01	0.00491556065988899\\
237.01	0.00491569504514588\\
238.01	0.00491583216727048\\
239.01	0.00491597208120721\\
240.01	0.00491611484296665\\
241.01	0.0049162605096445\\
242.01	0.00491640913944097\\
243.01	0.00491656079167989\\
244.01	0.00491671552682842\\
245.01	0.00491687340651699\\
246.01	0.00491703449355928\\
247.01	0.00491719885197298\\
248.01	0.00491736654699965\\
249.01	0.00491753764512574\\
250.01	0.00491771221410402\\
251.01	0.00491789032297398\\
252.01	0.00491807204208403\\
253.01	0.00491825744311238\\
254.01	0.00491844659908905\\
255.01	0.00491863958441796\\
256.01	0.00491883647489887\\
257.01	0.00491903734774932\\
258.01	0.00491924228162745\\
259.01	0.00491945135665417\\
260.01	0.00491966465443579\\
261.01	0.0049198822580869\\
262.01	0.00492010425225277\\
263.01	0.0049203307231323\\
264.01	0.004920561758501\\
265.01	0.00492079744773356\\
266.01	0.0049210378818268\\
267.01	0.00492128315342236\\
268.01	0.00492153335682981\\
269.01	0.00492178858804879\\
270.01	0.00492204894479158\\
271.01	0.00492231452650571\\
272.01	0.00492258543439587\\
273.01	0.00492286177144605\\
274.01	0.00492314364244112\\
275.01	0.00492343115398804\\
276.01	0.00492372441453698\\
277.01	0.00492402353440209\\
278.01	0.0049243286257809\\
279.01	0.00492463980277491\\
280.01	0.00492495718140823\\
281.01	0.00492528087964596\\
282.01	0.00492561101741196\\
283.01	0.00492594771660614\\
284.01	0.00492629110112015\\
285.01	0.0049266412968533\\
286.01	0.00492699843172646\\
287.01	0.00492736263569535\\
288.01	0.00492773404076295\\
289.01	0.00492811278099051\\
290.01	0.00492849899250662\\
291.01	0.00492889281351602\\
292.01	0.00492929438430578\\
293.01	0.00492970384725112\\
294.01	0.00493012134681801\\
295.01	0.0049305470295654\\
296.01	0.00493098104414412\\
297.01	0.00493142354129508\\
298.01	0.00493187467384421\\
299.01	0.00493233459669588\\
300.01	0.0049328034668234\\
301.01	0.004933281443257\\
302.01	0.00493376868706956\\
303.01	0.00493426536135884\\
304.01	0.00493477163122784\\
305.01	0.00493528766376086\\
306.01	0.00493581362799739\\
307.01	0.00493634969490228\\
308.01	0.00493689603733261\\
309.01	0.00493745283000145\\
310.01	0.00493802024943731\\
311.01	0.00493859847394061\\
312.01	0.00493918768353642\\
313.01	0.0049397880599234\\
314.01	0.00494039978641915\\
315.01	0.00494102304790169\\
316.01	0.0049416580307475\\
317.01	0.00494230492276611\\
318.01	0.00494296391313159\\
319.01	0.00494363519231056\\
320.01	0.00494431895198727\\
321.01	0.00494501538498655\\
322.01	0.00494572468519489\\
323.01	0.00494644704747903\\
324.01	0.00494718266760483\\
325.01	0.00494793174215515\\
326.01	0.00494869446844877\\
327.01	0.00494947104446108\\
328.01	0.00495026166874761\\
329.01	0.00495106654037195\\
330.01	0.00495188585883968\\
331.01	0.0049527198240398\\
332.01	0.00495356863619648\\
333.01	0.00495443249583227\\
334.01	0.0049553116037469\\
335.01	0.00495620616101282\\
336.01	0.00495711636899286\\
337.01	0.00495804242938172\\
338.01	0.00495898454427672\\
339.01	0.00495994291628156\\
340.01	0.00496091774864825\\
341.01	0.00496190924546254\\
342.01	0.00496291761187816\\
343.01	0.00496394305440709\\
344.01	0.00496498578127114\\
345.01	0.00496604600282269\\
346.01	0.0049671239320417\\
347.01	0.00496821978511656\\
348.01	0.00496933378211588\\
349.01	0.00497046614775908\\
350.01	0.00497161711229437\\
351.01	0.00497278691248819\\
352.01	0.00497397579273538\\
353.01	0.004975184006293\\
354.01	0.0049764118166411\\
355.01	0.00497765949897297\\
356.01	0.00497892734181093\\
357.01	0.00498021564874448\\
358.01	0.00498152474027903\\
359.01	0.00498285495578051\\
360.01	0.00498420665549278\\
361.01	0.00498558022259997\\
362.01	0.00498697606529111\\
363.01	0.00498839461877956\\
364.01	0.00498983634721632\\
365.01	0.00499130174541977\\
366.01	0.00499279134033873\\
367.01	0.00499430569214429\\
368.01	0.00499584539483784\\
369.01	0.00499741107625103\\
370.01	0.00499900339730523\\
371.01	0.00500062305039763\\
372.01	0.00500227075678872\\
373.01	0.0050039472628861\\
374.01	0.00500565333535642\\
375.01	0.00500738975505523\\
376.01	0.0050091573098522\\
377.01	0.00501095678654535\\
378.01	0.00501278896220887\\
379.01	0.00501465459550808\\
380.01	0.00501655441872212\\
381.01	0.00501848913142981\\
382.01	0.00502045939698054\\
383.01	0.00502246584291349\\
384.01	0.00502450906625633\\
385.01	0.00502658964389123\\
386.01	0.00502870814599819\\
387.01	0.00503086514465358\\
388.01	0.00503306121496463\\
389.01	0.00503529693482319\\
390.01	0.00503757288463709\\
391.01	0.00503988964705035\\
392.01	0.00504224780665428\\
393.01	0.0050446479496886\\
394.01	0.0050470906637351\\
395.01	0.00504957653740315\\
396.01	0.00505210616001035\\
397.01	0.00505468012125833\\
398.01	0.00505729901090658\\
399.01	0.00505996341844481\\
400.01	0.00506267393276891\\
401.01	0.00506543114186003\\
402.01	0.00506823563247182\\
403.01	0.00507108798982919\\
404.01	0.00507398879734204\\
405.01	0.00507693863633824\\
406.01	0.00507993808582029\\
407.01	0.00508298772225244\\
408.01	0.00508608811938221\\
409.01	0.00508923984810478\\
410.01	0.00509244347637487\\
411.01	0.00509569956917674\\
412.01	0.00509900868855837\\
413.01	0.00510237139373884\\
414.01	0.00510578824130067\\
415.01	0.00510925978547561\\
416.01	0.0051127865785352\\
417.01	0.00511636917129822\\
418.01	0.00512000811376698\\
419.01	0.00512370395590563\\
420.01	0.00512745724857272\\
421.01	0.00513126854462224\\
422.01	0.00513513840018684\\
423.01	0.00513906737615664\\
424.01	0.00514305603986628\\
425.01	0.00514710496700421\\
426.01	0.00515121474375491\\
427.01	0.00515538596918477\\
428.01	0.00515961925788036\\
429.01	0.00516391524284567\\
430.01	0.00516827457866102\\
431.01	0.00517269794490338\\
432.01	0.00517718604982302\\
433.01	0.00518173963426656\\
434.01	0.00518635947582922\\
435.01	0.00519104639321241\\
436.01	0.00519580125075539\\
437.01	0.00520062496309655\\
438.01	0.00520551849991628\\
439.01	0.0052104828906948\\
440.01	0.00521551922941084\\
441.01	0.00522062867909467\\
442.01	0.00522581247613339\\
443.01	0.00523107193421826\\
444.01	0.00523640844780934\\
445.01	0.00524182349498784\\
446.01	0.00524731863955634\\
447.01	0.0052528955322519\\
448.01	0.00525855591093484\\
449.01	0.00526430159963491\\
450.01	0.00527013450635724\\
451.01	0.00527605661958334\\
452.01	0.00528207000345424\\
453.01	0.00528817679168332\\
454.01	0.00529437918033213\\
455.01	0.00530067941967452\\
456.01	0.00530707980549043\\
457.01	0.00531358267025249\\
458.01	0.00532019037479467\\
459.01	0.00532690530116499\\
460.01	0.00533372984745722\\
461.01	0.00534066642544533\\
462.01	0.00534771746179938\\
463.01	0.00535488540348701\\
464.01	0.00536217272763402\\
465.01	0.00536958195559539\\
466.01	0.00537711567026543\\
467.01	0.00538477653479423\\
468.01	0.00539256731002991\\
469.01	0.00540049086756045\\
470.01	0.00540855019647486\\
471.01	0.00541674840612985\\
472.01	0.00542508872779501\\
473.01	0.00543357451562931\\
474.01	0.00544220924694998\\
475.01	0.00545099652176344\\
476.01	0.00545994006154736\\
477.01	0.00546904370729777\\
478.01	0.00547831141688372\\
479.01	0.00548774726178407\\
480.01	0.00549735542331861\\
481.01	0.005507140188521\\
482.01	0.00551710594584133\\
483.01	0.00552725718089843\\
484.01	0.00553759847253217\\
485.01	0.00554813448942038\\
486.01	0.0055588699875275\\
487.01	0.00556980980863281\\
488.01	0.00558095888013528\\
489.01	0.00559232221625795\\
490.01	0.00560390492066476\\
491.01	0.00561571219036122\\
492.01	0.00562774932059444\\
493.01	0.00564002171030427\\
494.01	0.0056525348675442\\
495.01	0.00566529441422003\\
496.01	0.00567830608954503\\
497.01	0.00569157575182161\\
498.01	0.00570510937855931\\
499.01	0.00571891306547014\\
500.01	0.0057329930251922\\
501.01	0.0057473555862806\\
502.01	0.00576200719262769\\
503.01	0.00577695440339838\\
504.01	0.00579220389354149\\
505.01	0.0058077624549108\\
506.01	0.00582363699799712\\
507.01	0.00583983455422826\\
508.01	0.00585636227876019\\
509.01	0.00587322745364077\\
510.01	0.00589043749120116\\
511.01	0.00590799993751283\\
512.01	0.00592592247575652\\
513.01	0.00594421292937826\\
514.01	0.00596287926496326\\
515.01	0.00598192959483183\\
516.01	0.00600137217944308\\
517.01	0.00602121542974826\\
518.01	0.00604146790964455\\
519.01	0.00606213833861916\\
520.01	0.00608323559458114\\
521.01	0.00610476871683385\\
522.01	0.0061267469091303\\
523.01	0.00614917954274147\\
524.01	0.00617207615947925\\
525.01	0.00619544647461083\\
526.01	0.00621930037961972\\
527.01	0.00624364794477872\\
528.01	0.0062684994215141\\
529.01	0.00629386524455719\\
530.01	0.00631975603388037\\
531.01	0.00634618259641703\\
532.01	0.00637315592754992\\
533.01	0.00640068721233276\\
534.01	0.00642878782640081\\
535.01	0.00645746933650783\\
536.01	0.00648674350063502\\
537.01	0.00651662226761109\\
538.01	0.00654711777618407\\
539.01	0.00657824235348439\\
540.01	0.00661000851281471\\
541.01	0.00664242895069685\\
542.01	0.00667551654309728\\
543.01	0.00670928434074414\\
544.01	0.00674374556343396\\
545.01	0.00677891359321937\\
546.01	0.00681480196635297\\
547.01	0.00685142436385382\\
548.01	0.00688879460055168\\
549.01	0.00692692661244959\\
550.01	0.00696583444223101\\
551.01	0.00700553222272127\\
552.01	0.00704603415809471\\
553.01	0.00708735450259801\\
554.01	0.00712950753653874\\
555.01	0.00717250753926676\\
556.01	0.00721636875885068\\
557.01	0.00726110537812294\\
558.01	0.00730673147674622\\
559.01	0.00735326098891716\\
560.01	0.00740070765629824\\
561.01	0.00744908497573459\\
562.01	0.00749840614127946\\
563.01	0.00754868398002041\\
564.01	0.00759993088116463\\
565.01	0.00765215871781027\\
566.01	0.00770537876080272\\
567.01	0.00775960158404742\\
568.01	0.00781483696063322\\
569.01	0.00787109374910713\\
570.01	0.00792837976924348\\
571.01	0.00798670166666285\\
572.01	0.00804606476569629\\
573.01	0.00810647290995056\\
574.01	0.00816792829013013\\
575.01	0.00823043125881488\\
576.01	0.00829398013209581\\
577.01	0.00835857097824435\\
578.01	0.00842419739396354\\
579.01	0.00849085026925531\\
580.01	0.00855851754257446\\
581.01	0.0086271839487638\\
582.01	0.00869683076332219\\
583.01	0.00876743554790117\\
584.01	0.00883897190364118\\
585.01	0.00891140924111708\\
586.01	0.00898471257838705\\
587.01	0.00905884238205851\\
588.01	0.0091337544705663\\
589.01	0.00920940000420826\\
590.01	0.00928572559315688\\
591.01	0.00936267356296942\\
592.01	0.00944018242744499\\
593.01	0.00951818763149205\\
594.01	0.00959662264256165\\
595.01	0.00967542048888486\\
596.01	0.00975451586711174\\
597.01	0.00983378252435979\\
598.01	0.00990866201848071\\
599.01	0.00997087280416276\\
599.02	0.00997138072163725\\
599.03	0.00997188557625799\\
599.04	0.00997238733820211\\
599.05	0.00997288597735272\\
599.06	0.00997338146329604\\
599.07	0.00997387376531845\\
599.08	0.00997436285240349\\
599.09	0.00997484869322892\\
599.1	0.00997533125616361\\
599.11	0.00997581050926455\\
599.12	0.00997628642027372\\
599.13	0.00997675895661497\\
599.14	0.0099772280853909\\
599.15	0.00997769377337961\\
599.16	0.00997815598703157\\
599.17	0.00997861469246631\\
599.18	0.00997906985546915\\
599.19	0.00997952144148792\\
599.2	0.00997996941562957\\
599.21	0.00998041374265684\\
599.22	0.0099808543869848\\
599.23	0.00998129131267744\\
599.24	0.00998172448344418\\
599.25	0.00998215386263636\\
599.26	0.00998257941258354\\
599.27	0.00998300109279825\\
599.28	0.00998341886238981\\
599.29	0.00998383268006024\\
599.3	0.00998424250410032\\
599.31	0.00998464829238543\\
599.32	0.00998505000237151\\
599.33	0.00998544759109087\\
599.34	0.00998584101514799\\
599.35	0.00998623023071532\\
599.36	0.00998661519352895\\
599.37	0.00998699585888436\\
599.38	0.00998737218163197\\
599.39	0.00998774411617281\\
599.4	0.00998811161645403\\
599.41	0.00998847463596443\\
599.42	0.0099888331277299\\
599.43	0.00998918704430885\\
599.44	0.00998953633778757\\
599.45	0.00998988095977558\\
599.46	0.00999022086140088\\
599.47	0.00999055599330522\\
599.48	0.00999088630563925\\
599.49	0.00999121174805768\\
599.5	0.00999153226971435\\
599.51	0.0099918478192573\\
599.52	0.00999215834482374\\
599.53	0.00999246379403503\\
599.54	0.0099927641139915\\
599.55	0.00999305925126738\\
599.56	0.00999334915190553\\
599.57	0.0099936337614122\\
599.58	0.00999391302475173\\
599.59	0.00999418688634118\\
599.6	0.00999445529004489\\
599.61	0.00999471817916904\\
599.62	0.00999497549645613\\
599.63	0.00999522718407935\\
599.64	0.009995473183637\\
599.65	0.00999571343614678\\
599.66	0.00999594788204004\\
599.67	0.00999617646115598\\
599.68	0.0099963991127358\\
599.69	0.00999661577541675\\
599.7	0.00999682638722618\\
599.71	0.00999703088557551\\
599.72	0.00999722920725411\\
599.73	0.00999742128842315\\
599.74	0.00999760706460942\\
599.75	0.00999778647069897\\
599.76	0.00999795944093086\\
599.77	0.0099981259088907\\
599.78	0.0099982858075042\\
599.79	0.00999843906903064\\
599.8	0.00999858562505627\\
599.81	0.00999872540648766\\
599.82	0.00999885834354498\\
599.83	0.00999898436575515\\
599.84	0.00999910340194508\\
599.85	0.00999921538023465\\
599.86	0.00999932022802977\\
599.87	0.00999941787201528\\
599.88	0.00999950823814785\\
599.89	0.00999959125164875\\
599.9	0.00999966683699656\\
599.91	0.00999973491791987\\
599.92	0.0099997954173898\\
599.93	0.00999984825761255\\
599.94	0.00999989336002181\\
599.95	0.00999993064527112\\
599.96	0.00999996003322615\\
599.97	0.00999998144295691\\
599.98	0.00999999479272987\\
599.99	0.01\\
600	0.01\\
};
\addplot [color=mycolor6,solid,forget plot]
  table[row sep=crcr]{%
0.01	0.00481014954134405\\
1.01	0.00481015070071254\\
2.01	0.00481015188410282\\
3.01	0.00481015309201312\\
4.01	0.0048101543249523\\
5.01	0.00481015558343964\\
6.01	0.0048101568680053\\
7.01	0.00481015817919046\\
8.01	0.00481015951754755\\
9.01	0.00481016088364039\\
10.01	0.00481016227804448\\
11.01	0.00481016370134746\\
12.01	0.00481016515414917\\
13.01	0.00481016663706167\\
14.01	0.0048101681507102\\
15.01	0.00481016969573249\\
16.01	0.0048101712727798\\
17.01	0.00481017288251699\\
18.01	0.00481017452562242\\
19.01	0.00481017620278913\\
20.01	0.00481017791472395\\
21.01	0.00481017966214872\\
22.01	0.00481018144580031\\
23.01	0.00481018326643066\\
24.01	0.00481018512480773\\
25.01	0.00481018702171504\\
26.01	0.00481018895795289\\
27.01	0.00481019093433795\\
28.01	0.00481019295170368\\
29.01	0.00481019501090106\\
30.01	0.00481019711279888\\
31.01	0.00481019925828392\\
32.01	0.00481020144826122\\
33.01	0.00481020368365494\\
34.01	0.0048102059654084\\
35.01	0.00481020829448453\\
36.01	0.00481021067186604\\
37.01	0.00481021309855627\\
38.01	0.00481021557557966\\
39.01	0.00481021810398161\\
40.01	0.0048102206848295\\
41.01	0.00481022331921278\\
42.01	0.00481022600824366\\
43.01	0.00481022875305748\\
44.01	0.00481023155481318\\
45.01	0.00481023441469378\\
46.01	0.00481023733390705\\
47.01	0.00481024031368568\\
48.01	0.00481024335528829\\
49.01	0.00481024645999943\\
50.01	0.00481024962913047\\
51.01	0.00481025286402029\\
52.01	0.00481025616603531\\
53.01	0.00481025953657056\\
54.01	0.00481026297705017\\
55.01	0.0048102664889277\\
56.01	0.00481027007368723\\
57.01	0.00481027373284361\\
58.01	0.0048102774679431\\
59.01	0.00481028128056438\\
60.01	0.00481028517231888\\
61.01	0.00481028914485161\\
62.01	0.0048102931998418\\
63.01	0.00481029733900368\\
64.01	0.00481030156408702\\
65.01	0.00481030587687843\\
66.01	0.00481031027920126\\
67.01	0.00481031477291716\\
68.01	0.00481031935992633\\
69.01	0.00481032404216857\\
70.01	0.00481032882162426\\
71.01	0.00481033370031486\\
72.01	0.00481033868030385\\
73.01	0.00481034376369797\\
74.01	0.00481034895264766\\
75.01	0.00481035424934825\\
76.01	0.00481035965604056\\
77.01	0.00481036517501233\\
78.01	0.00481037080859879\\
79.01	0.00481037655918383\\
80.01	0.00481038242920138\\
81.01	0.00481038842113546\\
82.01	0.00481039453752234\\
83.01	0.00481040078095088\\
84.01	0.00481040715406386\\
85.01	0.00481041365955931\\
86.01	0.00481042030019124\\
87.01	0.00481042707877116\\
88.01	0.00481043399816905\\
89.01	0.00481044106131481\\
90.01	0.00481044827119945\\
91.01	0.00481045563087631\\
92.01	0.0048104631434622\\
93.01	0.00481047081213884\\
94.01	0.00481047864015437\\
95.01	0.00481048663082471\\
96.01	0.00481049478753461\\
97.01	0.00481050311373948\\
98.01	0.00481051161296683\\
99.01	0.00481052028881776\\
100.01	0.00481052914496813\\
101.01	0.00481053818517066\\
102.01	0.00481054741325607\\
103.01	0.00481055683313504\\
104.01	0.00481056644879967\\
105.01	0.00481057626432532\\
106.01	0.00481058628387246\\
107.01	0.004810596511688\\
108.01	0.00481060695210768\\
109.01	0.00481061760955709\\
110.01	0.00481062848855444\\
111.01	0.00481063959371215\\
112.01	0.00481065092973859\\
113.01	0.00481066250144034\\
114.01	0.00481067431372413\\
115.01	0.0048106863715992\\
116.01	0.00481069868017873\\
117.01	0.00481071124468282\\
118.01	0.00481072407044042\\
119.01	0.00481073716289135\\
120.01	0.00481075052758896\\
121.01	0.00481076417020221\\
122.01	0.0048107780965182\\
123.01	0.00481079231244502\\
124.01	0.00481080682401336\\
125.01	0.0048108216373801\\
126.01	0.00481083675883011\\
127.01	0.0048108521947795\\
128.01	0.0048108679517779\\
129.01	0.00481088403651141\\
130.01	0.0048109004558058\\
131.01	0.00481091721662877\\
132.01	0.00481093432609353\\
133.01	0.00481095179146135\\
134.01	0.00481096962014495\\
135.01	0.00481098781971125\\
136.01	0.0048110063978853\\
137.01	0.0048110253625528\\
138.01	0.00481104472176394\\
139.01	0.00481106448373662\\
140.01	0.00481108465686003\\
141.01	0.00481110524969797\\
142.01	0.00481112627099319\\
143.01	0.00481114772967009\\
144.01	0.0048111696348393\\
145.01	0.00481119199580117\\
146.01	0.00481121482204984\\
147.01	0.00481123812327723\\
148.01	0.00481126190937731\\
149.01	0.00481128619045\\
150.01	0.00481131097680572\\
151.01	0.00481133627896967\\
152.01	0.00481136210768619\\
153.01	0.00481138847392362\\
154.01	0.0048114153888786\\
155.01	0.00481144286398083\\
156.01	0.00481147091089847\\
157.01	0.00481149954154214\\
158.01	0.00481152876807094\\
159.01	0.00481155860289674\\
160.01	0.00481158905869027\\
161.01	0.00481162014838588\\
162.01	0.00481165188518705\\
163.01	0.00481168428257234\\
164.01	0.00481171735430082\\
165.01	0.00481175111441771\\
166.01	0.00481178557726118\\
167.01	0.0048118207574671\\
168.01	0.00481185666997605\\
169.01	0.00481189333003998\\
170.01	0.00481193075322769\\
171.01	0.00481196895543208\\
172.01	0.00481200795287645\\
173.01	0.00481204776212178\\
174.01	0.00481208840007345\\
175.01	0.00481212988398846\\
176.01	0.00481217223148248\\
177.01	0.00481221546053767\\
178.01	0.00481225958950987\\
179.01	0.00481230463713658\\
180.01	0.00481235062254469\\
181.01	0.00481239756525864\\
182.01	0.00481244548520862\\
183.01	0.00481249440273894\\
184.01	0.0048125443386166\\
185.01	0.00481259531403974\\
186.01	0.00481264735064701\\
187.01	0.00481270047052633\\
188.01	0.00481275469622423\\
189.01	0.00481281005075523\\
190.01	0.0048128665576118\\
191.01	0.0048129242407738\\
192.01	0.00481298312471849\\
193.01	0.00481304323443138\\
194.01	0.00481310459541592\\
195.01	0.00481316723370447\\
196.01	0.00481323117586938\\
197.01	0.00481329644903362\\
198.01	0.00481336308088273\\
199.01	0.00481343109967566\\
200.01	0.00481350053425693\\
201.01	0.00481357141406868\\
202.01	0.00481364376916264\\
203.01	0.00481371763021285\\
204.01	0.0048137930285282\\
205.01	0.00481386999606585\\
206.01	0.0048139485654437\\
207.01	0.00481402876995489\\
208.01	0.00481411064358053\\
209.01	0.00481419422100451\\
210.01	0.00481427953762747\\
211.01	0.00481436662958172\\
212.01	0.00481445553374579\\
213.01	0.00481454628775976\\
214.01	0.00481463893004085\\
215.01	0.00481473349979893\\
216.01	0.00481483003705285\\
217.01	0.004814928582647\\
218.01	0.00481502917826748\\
219.01	0.00481513186645949\\
220.01	0.00481523669064492\\
221.01	0.00481534369513964\\
222.01	0.00481545292517209\\
223.01	0.00481556442690091\\
224.01	0.0048156782474344\\
225.01	0.00481579443484924\\
226.01	0.00481591303820997\\
227.01	0.0048160341075892\\
228.01	0.00481615769408726\\
229.01	0.00481628384985322\\
230.01	0.00481641262810552\\
231.01	0.0048165440831537\\
232.01	0.00481667827041972\\
233.01	0.00481681524646065\\
234.01	0.00481695506899074\\
235.01	0.00481709779690459\\
236.01	0.00481724349030065\\
237.01	0.00481739221050485\\
238.01	0.0048175440200949\\
239.01	0.00481769898292503\\
240.01	0.00481785716415088\\
241.01	0.00481801863025545\\
242.01	0.00481818344907474\\
243.01	0.00481835168982407\\
244.01	0.00481852342312557\\
245.01	0.00481869872103495\\
246.01	0.00481887765706959\\
247.01	0.00481906030623662\\
248.01	0.00481924674506191\\
249.01	0.0048194370516192\\
250.01	0.00481963130555982\\
251.01	0.00481982958814241\\
252.01	0.00482003198226424\\
253.01	0.00482023857249204\\
254.01	0.00482044944509324\\
255.01	0.00482066468806844\\
256.01	0.00482088439118374\\
257.01	0.00482110864600375\\
258.01	0.00482133754592508\\
259.01	0.00482157118621034\\
260.01	0.0048218096640221\\
261.01	0.0048220530784582\\
262.01	0.00482230153058678\\
263.01	0.00482255512348202\\
264.01	0.00482281396226006\\
265.01	0.00482307815411577\\
266.01	0.00482334780835965\\
267.01	0.00482362303645502\\
268.01	0.00482390395205565\\
269.01	0.00482419067104402\\
270.01	0.00482448331156941\\
271.01	0.00482478199408652\\
272.01	0.00482508684139468\\
273.01	0.00482539797867679\\
274.01	0.00482571553353858\\
275.01	0.0048260396360484\\
276.01	0.00482637041877684\\
277.01	0.00482670801683639\\
278.01	0.0048270525679215\\
279.01	0.0048274042123484\\
280.01	0.00482776309309454\\
281.01	0.00482812935583878\\
282.01	0.00482850314900027\\
283.01	0.00482888462377816\\
284.01	0.00482927393419044\\
285.01	0.00482967123711206\\
286.01	0.00483007669231331\\
287.01	0.00483049046249705\\
288.01	0.00483091271333523\\
289.01	0.00483134361350471\\
290.01	0.00483178333472237\\
291.01	0.00483223205177861\\
292.01	0.00483268994257\\
293.01	0.00483315718813013\\
294.01	0.00483363397265954\\
295.01	0.00483412048355308\\
296.01	0.00483461691142635\\
297.01	0.0048351234501388\\
298.01	0.00483564029681527\\
299.01	0.00483616765186451\\
300.01	0.00483670571899504\\
301.01	0.00483725470522769\\
302.01	0.00483781482090476\\
303.01	0.004838386279695\\
304.01	0.00483896929859511\\
305.01	0.00483956409792581\\
306.01	0.00484017090132424\\
307.01	0.00484078993572963\\
308.01	0.00484142143136446\\
309.01	0.00484206562170798\\
310.01	0.00484272274346456\\
311.01	0.00484339303652351\\
312.01	0.00484407674391166\\
313.01	0.00484477411173703\\
314.01	0.00484548538912395\\
315.01	0.00484621082813835\\
316.01	0.00484695068370171\\
317.01	0.00484770521349595\\
318.01	0.00484847467785354\\
319.01	0.00484925933963751\\
320.01	0.00485005946410659\\
321.01	0.00485087531876649\\
322.01	0.00485170717320614\\
323.01	0.00485255529891743\\
324.01	0.0048534199690986\\
325.01	0.0048543014584399\\
326.01	0.00485520004289085\\
327.01	0.00485611599940768\\
328.01	0.00485704960568156\\
329.01	0.00485800113984573\\
330.01	0.00485897088016164\\
331.01	0.00485995910468322\\
332.01	0.0048609660908997\\
333.01	0.00486199211535643\\
334.01	0.00486303745325403\\
335.01	0.00486410237802692\\
336.01	0.00486518716090109\\
337.01	0.00486629207043406\\
338.01	0.00486741737203751\\
339.01	0.00486856332748696\\
340.01	0.00486973019442055\\
341.01	0.00487091822583183\\
342.01	0.00487212766956283\\
343.01	0.00487335876780355\\
344.01	0.00487461175660618\\
345.01	0.00487588686542503\\
346.01	0.00487718431669312\\
347.01	0.00487850432545109\\
348.01	0.00487984709904424\\
349.01	0.00488121283690827\\
350.01	0.00488260173046647\\
351.01	0.00488401396316547\\
352.01	0.00488544971067974\\
353.01	0.00488690914131983\\
354.01	0.00488839241668426\\
355.01	0.00488989969259808\\
356.01	0.00489143112038896\\
357.01	0.00489298684855277\\
358.01	0.00489456702486797\\
359.01	0.00489617179902103\\
360.01	0.00489780132580704\\
361.01	0.00489945576897079\\
362.01	0.00490113530575392\\
363.01	0.00490284013220595\\
364.01	0.0049045704693079\\
365.01	0.00490632656994297\\
366.01	0.00490810872672136\\
367.01	0.00490991728063166\\
368.01	0.00491175263044519\\
369.01	0.00491361524273022\\
370.01	0.0049155056622537\\
371.01	0.00491742452243475\\
372.01	0.00491937255538659\\
373.01	0.00492135060091869\\
374.01	0.00492335961368574\\
375.01	0.00492540066746114\\
376.01	0.00492747495529421\\
377.01	0.00492958378410742\\
378.01	0.00493172856214356\\
379.01	0.00493391077765498\\
380.01	0.00493613196744747\\
381.01	0.00493839367453429\\
382.01	0.00494069739548804\\
383.01	0.00494304452051163\\
384.01	0.00494543627341122\\
385.01	0.0049478736656129\\
386.01	0.00495035751517274\\
387.01	0.00495288858523196\\
388.01	0.00495546764281225\\
389.01	0.00495809545977625\\
390.01	0.00496077281234379\\
391.01	0.00496350048057054\\
392.01	0.00496627924778534\\
393.01	0.00496910989998675\\
394.01	0.00497199322519443\\
395.01	0.00497493001275608\\
396.01	0.0049779210526074\\
397.01	0.00498096713448271\\
398.01	0.00498406904707539\\
399.01	0.00498722757714884\\
400.01	0.00499044350859192\\
401.01	0.00499371762142346\\
402.01	0.0049970506907422\\
403.01	0.00500044348562147\\
404.01	0.0050038967679506\\
405.01	0.00500741129122183\\
406.01	0.00501098779926511\\
407.01	0.0050146270249312\\
408.01	0.00501832968872651\\
409.01	0.00502209649740228\\
410.01	0.00502592814250234\\
411.01	0.00502982529887465\\
412.01	0.00503378862315343\\
413.01	0.00503781875221904\\
414.01	0.00504191630164617\\
415.01	0.00504608186414991\\
416.01	0.00505031600804466\\
417.01	0.00505461927572939\\
418.01	0.00505899218221903\\
419.01	0.0050634352137407\\
420.01	0.00506794882641948\\
421.01	0.00507253344507952\\
422.01	0.00507718946219067\\
423.01	0.00508191723699541\\
424.01	0.0050867170948518\\
425.01	0.00509158932683731\\
426.01	0.00509653418965687\\
427.01	0.00510155190590753\\
428.01	0.00510664266475552\\
429.01	0.00511180662308321\\
430.01	0.00511704390717192\\
431.01	0.0051223546149865\\
432.01	0.00512773881913381\\
433.01	0.00513319657056684\\
434.01	0.00513872790310789\\
435.01	0.00514433283886527\\
436.01	0.00515001139461045\\
437.01	0.00515576358918318\\
438.01	0.00516158945197483\\
439.01	0.00516748903253372\\
440.01	0.00517346241131204\\
441.01	0.00517950971155169\\
442.01	0.00518563111227167\\
443.01	0.0051918268622821\\
444.01	0.00519809729509522\\
445.01	0.00520444284454731\\
446.01	0.00521086406087209\\
447.01	0.00521736162688377\\
448.01	0.00522393637383599\\
449.01	0.00523058929641744\\
450.01	0.005237321566239\\
451.01	0.0052441345430564\\
452.01	0.00525102978286511\\
453.01	0.00525800904192067\\
454.01	0.00526507427567648\\
455.01	0.00527222763162986\\
456.01	0.00527947143513814\\
457.01	0.00528680816744723\\
458.01	0.00529424043550476\\
459.01	0.00530177093365197\\
460.01	0.00530940239803307\\
461.01	0.00531713755558978\\
462.01	0.00532497907080687\\
463.01	0.0053329294949338\\
464.01	0.0053409912240989\\
465.01	0.00534916647429395\\
466.01	0.00535745728213249\\
467.01	0.00536586553963554\\
468.01	0.00537439306754374\\
469.01	0.00538304172107079\\
470.01	0.00539181348436453\\
471.01	0.00540071050516695\\
472.01	0.00540973510763672\\
473.01	0.00541888980385066\\
474.01	0.00542817730434367\\
475.01	0.00543760052735374\\
476.01	0.00544716260641621\\
477.01	0.0054568668959398\\
478.01	0.00546671697440167\\
479.01	0.00547671664481987\\
480.01	0.00548686993220562\\
481.01	0.00549718107777489\\
482.01	0.00550765452980419\\
483.01	0.00551829493116719\\
484.01	0.00552910710377401\\
485.01	0.00554009603037293\\
486.01	0.00555126683443509\\
487.01	0.00556262475914229\\
488.01	0.00557417514679082\\
489.01	0.00558592342019337\\
490.01	0.00559787506784526\\
491.01	0.00561003563466669\\
492.01	0.00562241071994367\\
493.01	0.00563500598358412\\
494.01	0.00564782716088615\\
495.01	0.00566088008463324\\
496.01	0.0056741707115195\\
497.01	0.00568770514791241\\
498.01	0.00570148966840113\\
499.01	0.00571553072125539\\
500.01	0.0057298349224963\\
501.01	0.00574440904645443\\
502.01	0.00575926001579341\\
503.01	0.00577439489153707\\
504.01	0.00578982086358361\\
505.01	0.00580554524222467\\
506.01	0.00582157545117937\\
507.01	0.00583791902261407\\
508.01	0.00585458359451178\\
509.01	0.00587157691059846\\
510.01	0.00588890682281078\\
511.01	0.00590658129601832\\
512.01	0.00592460841441665\\
513.01	0.0059429963887318\\
514.01	0.00596175356318732\\
515.01	0.00598088842117465\\
516.01	0.00600040958882743\\
517.01	0.00602032583630079\\
518.01	0.00604064607744553\\
519.01	0.00606137936926249\\
520.01	0.0060825349121271\\
521.01	0.00610412205098902\\
522.01	0.00612615027750338\\
523.01	0.00614862923297689\\
524.01	0.00617156871194051\\
525.01	0.00619497866610692\\
526.01	0.00621886920843724\\
527.01	0.00624325061704201\\
528.01	0.00626813333868453\\
529.01	0.00629352799173694\\
530.01	0.00631944536855689\\
531.01	0.00634589643737332\\
532.01	0.00637289234385549\\
533.01	0.00640044441253146\\
534.01	0.00642856414810232\\
535.01	0.00645726323657654\\
536.01	0.00648655354609155\\
537.01	0.00651644712729382\\
538.01	0.0065469562131532\\
539.01	0.00657809321810346\\
540.01	0.00660987073641983\\
541.01	0.00664230153976614\\
542.01	0.00667539857385353\\
543.01	0.00670917495415527\\
544.01	0.0067436439606068\\
545.01	0.00677881903119029\\
546.01	0.00681471375427505\\
547.01	0.00685134185956257\\
548.01	0.00688871720747539\\
549.01	0.00692685377682434\\
550.01	0.00696576565057831\\
551.01	0.00700546699954922\\
552.01	0.00704597206379078\\
553.01	0.00708729513148984\\
554.01	0.00712945051510259\\
555.01	0.00717245252446584\\
556.01	0.00721631543658319\\
557.01	0.00726105346176332\\
558.01	0.00730668070575833\\
559.01	0.00735321112752477\\
560.01	0.0074006584921987\\
561.01	0.00744903631884569\\
562.01	0.00749835782251346\\
563.01	0.00754863585008057\\
564.01	0.00759988280936222\\
565.01	0.00765211059090243\\
566.01	0.00770533048185168\\
567.01	0.0077595530713051\\
568.01	0.00781478814645575\\
569.01	0.00787104457890665\\
570.01	0.00792833020048318\\
571.01	0.0079866516679058\\
572.01	0.00804601431571488\\
573.01	0.00810642199690455\\
574.01	0.0081678769108208\\
575.01	0.00823037941802071\\
576.01	0.00829392784199252\\
577.01	0.0083585182579118\\
578.01	0.00842414426897704\\
579.01	0.00849079677135603\\
580.01	0.00855846370941088\\
581.01	0.00862712982369113\\
582.01	0.0086967763952409\\
583.01	0.00876738099111419\\
584.01	0.00883891721770022\\
585.01	0.00891135449062501\\
586.01	0.00898465783271585\\
587.01	0.00905878771493402\\
588.01	0.00913369995946198\\
589.01	0.00920934572947927\\
590.01	0.00928567163683064\\
591.01	0.00936262000709249\\
592.01	0.00944012935186362\\
593.01	0.00951813511091667\\
594.01	0.00959657074273127\\
595.01	0.00967536926160508\\
596.01	0.00975446534388769\\
597.01	0.0098337539663526\\
598.01	0.00990866201848071\\
599.01	0.00997087280416276\\
599.02	0.00997138072163725\\
599.03	0.009971885576258\\
599.04	0.00997238733820211\\
599.05	0.00997288597735272\\
599.06	0.00997338146329604\\
599.07	0.00997387376531845\\
599.08	0.00997436285240349\\
599.09	0.00997484869322892\\
599.1	0.00997533125616361\\
599.11	0.00997581050926455\\
599.12	0.00997628642027372\\
599.13	0.00997675895661497\\
599.14	0.0099772280853909\\
599.15	0.00997769377337961\\
599.16	0.00997815598703157\\
599.17	0.00997861469246631\\
599.18	0.00997906985546915\\
599.19	0.00997952144148792\\
599.2	0.00997996941562957\\
599.21	0.00998041374265684\\
599.22	0.0099808543869848\\
599.23	0.00998129131267744\\
599.24	0.00998172448344418\\
599.25	0.00998215386263636\\
599.26	0.00998257941258353\\
599.27	0.00998300109279825\\
599.28	0.00998341886238981\\
599.29	0.00998383268006024\\
599.3	0.00998424250410032\\
599.31	0.00998464829238543\\
599.32	0.00998505000237151\\
599.33	0.00998544759109087\\
599.34	0.00998584101514799\\
599.35	0.00998623023071532\\
599.36	0.00998661519352895\\
599.37	0.00998699585888436\\
599.38	0.00998737218163197\\
599.39	0.00998774411617281\\
599.4	0.00998811161645403\\
599.41	0.00998847463596443\\
599.42	0.0099888331277299\\
599.43	0.00998918704430885\\
599.44	0.00998953633778757\\
599.45	0.00998988095977558\\
599.46	0.00999022086140088\\
599.47	0.00999055599330522\\
599.48	0.00999088630563925\\
599.49	0.00999121174805768\\
599.5	0.00999153226971435\\
599.51	0.0099918478192573\\
599.52	0.00999215834482374\\
599.53	0.00999246379403503\\
599.54	0.0099927641139915\\
599.55	0.00999305925126738\\
599.56	0.00999334915190553\\
599.57	0.0099936337614122\\
599.58	0.00999391302475173\\
599.59	0.00999418688634118\\
599.6	0.00999445529004489\\
599.61	0.00999471817916904\\
599.62	0.00999497549645613\\
599.63	0.00999522718407934\\
599.64	0.009995473183637\\
599.65	0.00999571343614678\\
599.66	0.00999594788204004\\
599.67	0.00999617646115599\\
599.68	0.0099963991127358\\
599.69	0.00999661577541675\\
599.7	0.00999682638722618\\
599.71	0.00999703088557551\\
599.72	0.00999722920725411\\
599.73	0.00999742128842315\\
599.74	0.00999760706460942\\
599.75	0.00999778647069897\\
599.76	0.00999795944093086\\
599.77	0.0099981259088907\\
599.78	0.0099982858075042\\
599.79	0.00999843906903064\\
599.8	0.00999858562505627\\
599.81	0.00999872540648767\\
599.82	0.00999885834354498\\
599.83	0.00999898436575515\\
599.84	0.00999910340194508\\
599.85	0.00999921538023465\\
599.86	0.00999932022802977\\
599.87	0.00999941787201528\\
599.88	0.00999950823814785\\
599.89	0.00999959125164875\\
599.9	0.00999966683699656\\
599.91	0.00999973491791987\\
599.92	0.0099997954173898\\
599.93	0.00999984825761255\\
599.94	0.00999989336002181\\
599.95	0.00999993064527112\\
599.96	0.00999996003322615\\
599.97	0.00999998144295691\\
599.98	0.00999999479272987\\
599.99	0.01\\
600	0.01\\
};
\addplot [color=mycolor7,solid,forget plot]
  table[row sep=crcr]{%
0.01	0.00464834445259417\\
1.01	0.00464834564822231\\
2.01	0.00464834686873988\\
3.01	0.0046483481146663\\
4.01	0.00464834938653171\\
5.01	0.00464835068487754\\
6.01	0.00464835201025615\\
7.01	0.00464835336323188\\
8.01	0.00464835474438058\\
9.01	0.00464835615429034\\
10.01	0.00464835759356162\\
11.01	0.00464835906280725\\
12.01	0.00464836056265287\\
13.01	0.00464836209373737\\
14.01	0.00464836365671296\\
15.01	0.00464836525224556\\
16.01	0.00464836688101485\\
17.01	0.00464836854371492\\
18.01	0.00464837024105429\\
19.01	0.00464837197375634\\
20.01	0.00464837374255971\\
21.01	0.00464837554821833\\
22.01	0.0046483773915021\\
23.01	0.00464837927319689\\
24.01	0.00464838119410517\\
25.01	0.00464838315504608\\
26.01	0.00464838515685599\\
27.01	0.00464838720038867\\
28.01	0.0046483892865162\\
29.01	0.00464839141612844\\
30.01	0.0046483935901341\\
31.01	0.00464839580946098\\
32.01	0.00464839807505615\\
33.01	0.00464840038788687\\
34.01	0.00464840274894025\\
35.01	0.00464840515922424\\
36.01	0.00464840761976822\\
37.01	0.00464841013162275\\
38.01	0.00464841269586058\\
39.01	0.00464841531357703\\
40.01	0.00464841798589027\\
41.01	0.00464842071394191\\
42.01	0.00464842349889752\\
43.01	0.00464842634194722\\
44.01	0.0046484292443059\\
45.01	0.00464843220721413\\
46.01	0.00464843523193825\\
47.01	0.00464843831977162\\
48.01	0.00464844147203429\\
49.01	0.00464844469007421\\
50.01	0.00464844797526748\\
51.01	0.0046484513290191\\
52.01	0.00464845475276383\\
53.01	0.00464845824796627\\
54.01	0.00464846181612165\\
55.01	0.00464846545875701\\
56.01	0.00464846917743099\\
57.01	0.00464847297373523\\
58.01	0.00464847684929467\\
59.01	0.00464848080576842\\
60.01	0.00464848484485038\\
61.01	0.00464848896826998\\
62.01	0.00464849317779312\\
63.01	0.00464849747522268\\
64.01	0.0046485018623994\\
65.01	0.00464850634120256\\
66.01	0.00464851091355128\\
67.01	0.00464851558140465\\
68.01	0.00464852034676305\\
69.01	0.00464852521166893\\
70.01	0.00464853017820752\\
71.01	0.00464853524850789\\
72.01	0.00464854042474395\\
73.01	0.00464854570913519\\
74.01	0.00464855110394777\\
75.01	0.00464855661149552\\
76.01	0.00464856223414092\\
77.01	0.00464856797429605\\
78.01	0.00464857383442366\\
79.01	0.00464857981703851\\
80.01	0.00464858592470777\\
81.01	0.00464859216005296\\
82.01	0.00464859852575081\\
83.01	0.00464860502453399\\
84.01	0.00464861165919278\\
85.01	0.00464861843257618\\
86.01	0.0046486253475931\\
87.01	0.0046486324072136\\
88.01	0.00464863961447035\\
89.01	0.00464864697245969\\
90.01	0.00464865448434311\\
91.01	0.00464866215334847\\
92.01	0.00464866998277179\\
93.01	0.00464867797597848\\
94.01	0.0046486861364045\\
95.01	0.00464869446755817\\
96.01	0.00464870297302199\\
97.01	0.00464871165645349\\
98.01	0.00464872052158731\\
99.01	0.00464872957223671\\
100.01	0.00464873881229512\\
101.01	0.00464874824573808\\
102.01	0.00464875787662486\\
103.01	0.00464876770910008\\
104.01	0.00464877774739562\\
105.01	0.00464878799583257\\
106.01	0.00464879845882279\\
107.01	0.00464880914087151\\
108.01	0.00464882004657829\\
109.01	0.00464883118064019\\
110.01	0.00464884254785258\\
111.01	0.00464885415311218\\
112.01	0.00464886600141887\\
113.01	0.00464887809787795\\
114.01	0.00464889044770213\\
115.01	0.00464890305621371\\
116.01	0.00464891592884785\\
117.01	0.00464892907115352\\
118.01	0.00464894248879703\\
119.01	0.00464895618756395\\
120.01	0.00464897017336184\\
121.01	0.00464898445222281\\
122.01	0.0046489990303062\\
123.01	0.00464901391390093\\
124.01	0.00464902910942883\\
125.01	0.00464904462344685\\
126.01	0.00464906046265044\\
127.01	0.00464907663387601\\
128.01	0.00464909314410422\\
129.01	0.00464911000046296\\
130.01	0.00464912721023041\\
131.01	0.00464914478083822\\
132.01	0.00464916271987477\\
133.01	0.00464918103508856\\
134.01	0.00464919973439162\\
135.01	0.00464921882586271\\
136.01	0.0046492383177511\\
137.01	0.00464925821848014\\
138.01	0.00464927853665087\\
139.01	0.00464929928104579\\
140.01	0.0046493204606327\\
141.01	0.00464934208456876\\
142.01	0.00464936416220404\\
143.01	0.00464938670308643\\
144.01	0.00464940971696495\\
145.01	0.00464943321379452\\
146.01	0.0046494572037403\\
147.01	0.00464948169718185\\
148.01	0.00464950670471789\\
149.01	0.00464953223717089\\
150.01	0.00464955830559181\\
151.01	0.00464958492126498\\
152.01	0.00464961209571292\\
153.01	0.00464963984070133\\
154.01	0.00464966816824472\\
155.01	0.004649697090611\\
156.01	0.00464972662032705\\
157.01	0.00464975677018478\\
158.01	0.00464978755324576\\
159.01	0.0046498189828479\\
160.01	0.00464985107261046\\
161.01	0.00464988383644051\\
162.01	0.00464991728853896\\
163.01	0.0046499514434067\\
164.01	0.00464998631585091\\
165.01	0.00465002192099168\\
166.01	0.00465005827426837\\
167.01	0.00465009539144685\\
168.01	0.00465013328862596\\
169.01	0.00465017198224481\\
170.01	0.00465021148908977\\
171.01	0.00465025182630202\\
172.01	0.0046502930113851\\
173.01	0.0046503350622123\\
174.01	0.0046503779970348\\
175.01	0.00465042183448953\\
176.01	0.00465046659360724\\
177.01	0.00465051229382124\\
178.01	0.00465055895497543\\
179.01	0.00465060659733316\\
180.01	0.00465065524158635\\
181.01	0.00465070490886434\\
182.01	0.0046507556207433\\
183.01	0.00465080739925518\\
184.01	0.00465086026689822\\
185.01	0.00465091424664639\\
186.01	0.00465096936195923\\
187.01	0.00465102563679255\\
188.01	0.00465108309560855\\
189.01	0.00465114176338727\\
190.01	0.00465120166563647\\
191.01	0.00465126282840389\\
192.01	0.00465132527828799\\
193.01	0.00465138904244983\\
194.01	0.00465145414862494\\
195.01	0.00465152062513614\\
196.01	0.00465158850090458\\
197.01	0.00465165780546387\\
198.01	0.00465172856897206\\
199.01	0.00465180082222541\\
200.01	0.0046518745966718\\
201.01	0.00465194992442446\\
202.01	0.00465202683827612\\
203.01	0.0046521053717135\\
204.01	0.00465218555893182\\
205.01	0.00465226743484987\\
206.01	0.00465235103512567\\
207.01	0.0046524363961717\\
208.01	0.0046525235551711\\
209.01	0.00465261255009412\\
210.01	0.00465270341971445\\
211.01	0.00465279620362667\\
212.01	0.00465289094226341\\
213.01	0.00465298767691322\\
214.01	0.00465308644973861\\
215.01	0.00465318730379469\\
216.01	0.00465329028304812\\
217.01	0.00465339543239622\\
218.01	0.00465350279768705\\
219.01	0.00465361242573939\\
220.01	0.00465372436436307\\
221.01	0.00465383866238059\\
222.01	0.00465395536964798\\
223.01	0.00465407453707718\\
224.01	0.00465419621665816\\
225.01	0.00465432046148176\\
226.01	0.00465444732576344\\
227.01	0.00465457686486646\\
228.01	0.00465470913532673\\
229.01	0.00465484419487774\\
230.01	0.00465498210247563\\
231.01	0.00465512291832523\\
232.01	0.00465526670390652\\
233.01	0.00465541352200186\\
234.01	0.00465556343672337\\
235.01	0.00465571651354121\\
236.01	0.00465587281931247\\
237.01	0.0046560324223103\\
238.01	0.00465619539225437\\
239.01	0.00465636180034099\\
240.01	0.00465653171927498\\
241.01	0.00465670522330118\\
242.01	0.00465688238823729\\
243.01	0.00465706329150724\\
244.01	0.00465724801217507\\
245.01	0.00465743663097969\\
246.01	0.00465762923037064\\
247.01	0.00465782589454374\\
248.01	0.00465802670947869\\
249.01	0.0046582317629764\\
250.01	0.00465844114469749\\
251.01	0.00465865494620205\\
252.01	0.0046588732609894\\
253.01	0.00465909618453894\\
254.01	0.00465932381435288\\
255.01	0.00465955624999767\\
256.01	0.00465979359314861\\
257.01	0.00466003594763367\\
258.01	0.00466028341947942\\
259.01	0.00466053611695691\\
260.01	0.00466079415062931\\
261.01	0.00466105763339973\\
262.01	0.00466132668056077\\
263.01	0.00466160140984465\\
264.01	0.00466188194147421\\
265.01	0.0046621683982155\\
266.01	0.00466246090543077\\
267.01	0.00466275959113274\\
268.01	0.00466306458604027\\
269.01	0.00466337602363437\\
270.01	0.00466369404021644\\
271.01	0.00466401877496626\\
272.01	0.00466435037000225\\
273.01	0.00466468897044188\\
274.01	0.00466503472446445\\
275.01	0.00466538778337351\\
276.01	0.00466574830166176\\
277.01	0.00466611643707615\\
278.01	0.00466649235068493\\
279.01	0.00466687620694507\\
280.01	0.00466726817377179\\
281.01	0.00466766842260811\\
282.01	0.00466807712849658\\
283.01	0.00466849447015159\\
284.01	0.00466892063003287\\
285.01	0.00466935579442034\\
286.01	0.00466980015348983\\
287.01	0.00467025390138999\\
288.01	0.00467071723632011\\
289.01	0.00467119036060909\\
290.01	0.00467167348079517\\
291.01	0.00467216680770687\\
292.01	0.00467267055654463\\
293.01	0.00467318494696309\\
294.01	0.00467371020315407\\
295.01	0.00467424655393085\\
296.01	0.0046747942328116\\
297.01	0.00467535347810466\\
298.01	0.00467592453299313\\
299.01	0.0046765076456199\\
300.01	0.00467710306917288\\
301.01	0.0046777110619695\\
302.01	0.00467833188754116\\
303.01	0.00467896581471694\\
304.01	0.00467961311770609\\
305.01	0.0046802740761803\\
306.01	0.00468094897535276\\
307.01	0.00468163810605712\\
308.01	0.0046823417648227\\
309.01	0.00468306025394839\\
310.01	0.00468379388157174\\
311.01	0.0046845429617358\\
312.01	0.00468530781445062\\
313.01	0.0046860887657502\\
314.01	0.00468688614774332\\
315.01	0.00468770029865807\\
316.01	0.004688531562879\\
317.01	0.00468938029097496\\
318.01	0.00469024683971839\\
319.01	0.0046911315720933\\
320.01	0.00469203485729067\\
321.01	0.00469295707069094\\
322.01	0.00469389859383011\\
323.01	0.00469485981435\\
324.01	0.00469584112592815\\
325.01	0.00469684292818687\\
326.01	0.00469786562657859\\
327.01	0.00469890963224451\\
328.01	0.00469997536184386\\
329.01	0.00470106323735046\\
330.01	0.00470217368581355\\
331.01	0.00470330713907838\\
332.01	0.00470446403346253\\
333.01	0.00470564480938383\\
334.01	0.0047068499109341\\
335.01	0.00470807978539418\\
336.01	0.00470933488268387\\
337.01	0.00471061565474067\\
338.01	0.00471192255482068\\
339.01	0.00471325603671385\\
340.01	0.00471461655386707\\
341.01	0.00471600455840571\\
342.01	0.00471742050004591\\
343.01	0.00471886482488885\\
344.01	0.0047203379740879\\
345.01	0.0047218403823789\\
346.01	0.0047233724764662\\
347.01	0.00472493467325462\\
348.01	0.00472652737792041\\
349.01	0.00472815098181462\\
350.01	0.00472980586019424\\
351.01	0.00473149236977929\\
352.01	0.00473321084613727\\
353.01	0.00473496160090216\\
354.01	0.00473674491883976\\
355.01	0.0047385610547811\\
356.01	0.00474041023045556\\
357.01	0.00474229263126748\\
358.01	0.00474420840307992\\
359.01	0.00474615764908549\\
360.01	0.00474814042687612\\
361.01	0.00475015674584822\\
362.01	0.00475220656512587\\
363.01	0.00475428979222349\\
364.01	0.00475640628273112\\
365.01	0.00475855584136699\\
366.01	0.00476073822481682\\
367.01	0.00476295314686802\\
368.01	0.00476520028643795\\
369.01	0.00476747929920025\\
370.01	0.00476978983361274\\
371.01	0.00477213155224933\\
372.01	0.00477450415940651\\
373.01	0.00477690743598608\\
374.01	0.00477934128259452\\
375.01	0.00478180577161051\\
376.01	0.00478430120855663\\
377.01	0.00478682820236268\\
378.01	0.00478938774284642\\
379.01	0.00479198128171347\\
380.01	0.00479461081023144\\
381.01	0.00479727892195163\\
382.01	0.00479998884170042\\
383.01	0.00480274439151386\\
384.01	0.00480554984876812\\
385.01	0.00480840960580881\\
386.01	0.00481132663638383\\
387.01	0.0048143021350969\\
388.01	0.00481733718293635\\
389.01	0.00482043287460097\\
390.01	0.00482359031817933\\
391.01	0.00482681063478051\\
392.01	0.00483009495811014\\
393.01	0.00483344443398835\\
394.01	0.00483686021980408\\
395.01	0.00484034348390203\\
396.01	0.00484389540489522\\
397.01	0.00484751717089908\\
398.01	0.00485120997867994\\
399.01	0.00485497503271119\\
400.01	0.00485881354413251\\
401.01	0.0048627267296011\\
402.01	0.00486671581003152\\
403.01	0.00487078200921302\\
404.01	0.00487492655229703\\
405.01	0.00487915066414647\\
406.01	0.00488345556753692\\
407.01	0.00488784248120079\\
408.01	0.00489231261770388\\
409.01	0.00489686718114493\\
410.01	0.00490150736466798\\
411.01	0.00490623434777614\\
412.01	0.00491104929343796\\
413.01	0.00491595334497498\\
414.01	0.0049209476227208\\
415.01	0.00492603322044267\\
416.01	0.00493121120151686\\
417.01	0.00493648259485009\\
418.01	0.00494184839054121\\
419.01	0.00494730953527837\\
420.01	0.00495286692747109\\
421.01	0.00495852141211699\\
422.01	0.00496427377540952\\
423.01	0.00497012473909408\\
424.01	0.00497607495459103\\
425.01	0.00498212499690281\\
426.01	0.00498827535833897\\
427.01	0.00499452644209535\\
428.01	0.00500087855573869\\
429.01	0.00500733190466011\\
430.01	0.00501388658557486\\
431.01	0.00502054258016595\\
432.01	0.00502729974898626\\
433.01	0.00503415782575883\\
434.01	0.00504111641224168\\
435.01	0.00504817497385036\\
436.01	0.00505533283626679\\
437.01	0.00506258918329626\\
438.01	0.00506994305627546\\
439.01	0.00507739335537467\\
440.01	0.00508493884317916\\
441.01	0.00509257815097973\\
442.01	0.00510030978824515\\
443.01	0.00510813215578212\\
444.01	0.00511604356312524\\
445.01	0.00512404225070982\\
446.01	0.00513212641738443\\
447.01	0.00514029425378412\\
448.01	0.00514854398202269\\
449.01	0.00515687390203937\\
450.01	0.00516528244474876\\
451.01	0.0051737682318643\\
452.01	0.00518233014188449\\
453.01	0.00519096738119998\\
454.01	0.00519967955859978\\
455.01	0.00520846676056924\\
456.01	0.00521732962368783\\
457.01	0.00522626939911519\\
458.01	0.00523528800263791\\
459.01	0.00524438804207301\\
460.01	0.00525357281213593\\
461.01	0.00526284624540482\\
462.01	0.00527221280719085\\
463.01	0.00528167732264719\\
464.01	0.00529124472744835\\
465.01	0.00530091974060239\\
466.01	0.0053107064721192\\
467.01	0.00532060800344458\\
468.01	0.00533062602087934\\
469.01	0.00534076072198664\\
470.01	0.00535101162228553\\
471.01	0.00536137827533253\\
472.01	0.0053718603558916\\
473.01	0.00538245769485676\\
474.01	0.00539317031688376\\
475.01	0.0054039984803502\\
476.01	0.00541494271909335\\
477.01	0.00542600388516891\\
478.01	0.00543718319164449\\
479.01	0.00544848225418493\\
480.01	0.00545990312991866\\
481.01	0.00547144835179354\\
482.01	0.00548312095635677\\
483.01	0.00549492450265347\\
484.01	0.00550686307978838\\
485.01	0.00551894130066258\\
486.01	0.00553116427957259\\
487.01	0.00554353759180509\\
488.01	0.00555606721418994\\
489.01	0.00556875944688499\\
490.01	0.00558162081857003\\
491.01	0.0055946579797886\\
492.01	0.00560787759240927\\
493.01	0.00562128622692798\\
494.01	0.00563489028319811\\
495.01	0.00564869595329159\\
496.01	0.00566270924591865\\
497.01	0.00567693608728224\\
498.01	0.00569138249822781\\
499.01	0.00570605479287023\\
500.01	0.00572095966238625\\
501.01	0.00573610416924741\\
502.01	0.00575149572772143\\
503.01	0.00576714207787485\\
504.01	0.00578305125354241\\
505.01	0.00579923154518509\\
506.01	0.00581569145909275\\
507.01	0.00583243967495409\\
508.01	0.00584948500437797\\
509.01	0.00586683635341503\\
510.01	0.00588450269239885\\
511.01	0.00590249303634946\\
512.01	0.00592081643858883\\
513.01	0.00593948199892681\\
514.01	0.00595849888564073\\
515.01	0.00597787636749476\\
516.01	0.00599762384849975\\
517.01	0.0060177508949207\\
518.01	0.00603826724368112\\
519.01	0.00605918279183401\\
520.01	0.00608050758068721\\
521.01	0.00610225178171825\\
522.01	0.006124425685514\\
523.01	0.00614703969441464\\
524.01	0.00617010431929348\\
525.01	0.00619363018054338\\
526.01	0.00621762801289089\\
527.01	0.00624210867316072\\
528.01	0.00626708314964169\\
529.01	0.00629256257138429\\
530.01	0.00631855821573893\\
531.01	0.00634508151289306\\
532.01	0.00637214404718316\\
533.01	0.00639975755639791\\
534.01	0.00642793393108233\\
535.01	0.00645668521483311\\
536.01	0.00648602360550507\\
537.01	0.00651596145698026\\
538.01	0.00654651128106563\\
539.01	0.00657768574904344\\
540.01	0.0066094976924182\\
541.01	0.0066419601024892\\
542.01	0.00667508612853107\\
543.01	0.00670888907454924\\
544.01	0.00674338239474022\\
545.01	0.00677857968783409\\
546.01	0.00681449469037443\\
547.01	0.00685114126877864\\
548.01	0.00688853340991637\\
549.01	0.00692668520993533\\
550.01	0.00696561086108091\\
551.01	0.00700532463627988\\
552.01	0.00704584087128132\\
553.01	0.00708717394416014\\
554.01	0.00712933825198563\\
555.01	0.00717234818441871\\
556.01	0.00721621809395735\\
557.01	0.00726096226249309\\
558.01	0.00730659486381002\\
559.01	0.00735312992163711\\
560.01	0.00740058126284496\\
561.01	0.00744896246535642\\
562.01	0.00749828680030934\\
563.01	0.00754856716798113\\
564.01	0.00759981602694357\\
565.01	0.00765204531588342\\
566.01	0.00770526636749114\\
567.01	0.00775948981379601\\
568.01	0.00781472548230801\\
569.01	0.00787098228231643\\
570.01	0.00792826808069656\\
571.01	0.00798658956658903\\
572.01	0.00804595210435032\\
573.01	0.00810635957423606\\
574.01	0.00816781420037203\\
575.01	0.0082303163657128\\
576.01	0.00829386441388749\\
577.01	0.00835845443810524\\
578.01	0.00842408005766179\\
579.01	0.00849073218307326\\
580.01	0.00855839877149774\\
581.01	0.00862706457492604\\
582.01	0.00869671088467823\\
583.01	0.00876731527708814\\
584.01	0.00883885136696868\\
585.01	0.00891128857761011\\
586.01	0.00898459193878753\\
587.01	0.00905872192767049\\
588.01	0.00913363437180803\\
589.01	0.00920928043871075\\
590.01	0.00928560674321885\\
591.01	0.00936255561214592\\
592.01	0.00944006555600452\\
593.01	0.00951807201042488\\
594.01	0.00959650842575684\\
595.01	0.00967530780301053\\
596.01	0.00975440479863077\\
597.01	0.00983371536751284\\
598.01	0.00990866201848071\\
599.01	0.00997087280416276\\
599.02	0.00997138072163725\\
599.03	0.009971885576258\\
599.04	0.00997238733820211\\
599.05	0.00997288597735272\\
599.06	0.00997338146329604\\
599.07	0.00997387376531845\\
599.08	0.00997436285240349\\
599.09	0.00997484869322892\\
599.1	0.00997533125616361\\
599.11	0.00997581050926455\\
599.12	0.00997628642027372\\
599.13	0.00997675895661497\\
599.14	0.0099772280853909\\
599.15	0.00997769377337961\\
599.16	0.00997815598703157\\
599.17	0.00997861469246631\\
599.18	0.00997906985546915\\
599.19	0.00997952144148792\\
599.2	0.00997996941562957\\
599.21	0.00998041374265684\\
599.22	0.0099808543869848\\
599.23	0.00998129131267744\\
599.24	0.00998172448344418\\
599.25	0.00998215386263636\\
599.26	0.00998257941258354\\
599.27	0.00998300109279825\\
599.28	0.00998341886238981\\
599.29	0.00998383268006024\\
599.3	0.00998424250410032\\
599.31	0.00998464829238543\\
599.32	0.00998505000237151\\
599.33	0.00998544759109087\\
599.34	0.00998584101514799\\
599.35	0.00998623023071532\\
599.36	0.00998661519352896\\
599.37	0.00998699585888436\\
599.38	0.00998737218163197\\
599.39	0.00998774411617281\\
599.4	0.00998811161645403\\
599.41	0.00998847463596443\\
599.42	0.0099888331277299\\
599.43	0.00998918704430885\\
599.44	0.00998953633778757\\
599.45	0.00998988095977558\\
599.46	0.00999022086140088\\
599.47	0.00999055599330522\\
599.48	0.00999088630563925\\
599.49	0.00999121174805768\\
599.5	0.00999153226971435\\
599.51	0.0099918478192573\\
599.52	0.00999215834482375\\
599.53	0.00999246379403503\\
599.54	0.0099927641139915\\
599.55	0.00999305925126738\\
599.56	0.00999334915190553\\
599.57	0.0099936337614122\\
599.58	0.00999391302475173\\
599.59	0.00999418688634118\\
599.6	0.00999445529004489\\
599.61	0.00999471817916904\\
599.62	0.00999497549645613\\
599.63	0.00999522718407935\\
599.64	0.009995473183637\\
599.65	0.00999571343614678\\
599.66	0.00999594788204004\\
599.67	0.00999617646115598\\
599.68	0.0099963991127358\\
599.69	0.00999661577541675\\
599.7	0.00999682638722618\\
599.71	0.00999703088557551\\
599.72	0.00999722920725411\\
599.73	0.00999742128842315\\
599.74	0.00999760706460942\\
599.75	0.00999778647069897\\
599.76	0.00999795944093086\\
599.77	0.0099981259088907\\
599.78	0.0099982858075042\\
599.79	0.00999843906903064\\
599.8	0.00999858562505627\\
599.81	0.00999872540648767\\
599.82	0.00999885834354498\\
599.83	0.00999898436575515\\
599.84	0.00999910340194508\\
599.85	0.00999921538023465\\
599.86	0.00999932022802977\\
599.87	0.00999941787201528\\
599.88	0.00999950823814785\\
599.89	0.00999959125164875\\
599.9	0.00999966683699656\\
599.91	0.00999973491791987\\
599.92	0.0099997954173898\\
599.93	0.00999984825761255\\
599.94	0.00999989336002181\\
599.95	0.00999993064527112\\
599.96	0.00999996003322615\\
599.97	0.00999998144295691\\
599.98	0.00999999479272987\\
599.99	0.01\\
600	0.01\\
};
\addplot [color=mycolor8,solid,forget plot]
  table[row sep=crcr]{%
0.01	0.00440626308531185\\
1.01	0.00440626422127448\\
2.01	0.00440626538095373\\
3.01	0.0044062665648462\\
4.01	0.00440626777345891\\
5.01	0.00440626900730949\\
6.01	0.00440627026692658\\
7.01	0.00440627155284969\\
8.01	0.00440627286562986\\
9.01	0.00440627420582962\\
10.01	0.0044062755740233\\
11.01	0.00440627697079723\\
12.01	0.00440627839675041\\
13.01	0.00440627985249417\\
14.01	0.00440628133865258\\
15.01	0.004406282855863\\
16.01	0.00440628440477632\\
17.01	0.00440628598605691\\
18.01	0.0044062876003831\\
19.01	0.00440628924844742\\
20.01	0.0044062909309573\\
21.01	0.00440629264863485\\
22.01	0.00440629440221726\\
23.01	0.00440629619245744\\
24.01	0.00440629802012393\\
25.01	0.00440629988600161\\
26.01	0.00440630179089169\\
27.01	0.0044063037356125\\
28.01	0.00440630572099936\\
29.01	0.0044063077479054\\
30.01	0.00440630981720147\\
31.01	0.00440631192977683\\
32.01	0.0044063140865396\\
33.01	0.00440631628841669\\
34.01	0.00440631853635493\\
35.01	0.00440632083132096\\
36.01	0.00440632317430162\\
37.01	0.00440632556630473\\
38.01	0.00440632800835908\\
39.01	0.00440633050151536\\
40.01	0.00440633304684623\\
41.01	0.00440633564544717\\
42.01	0.00440633829843672\\
43.01	0.00440634100695664\\
44.01	0.00440634377217311\\
45.01	0.0044063465952769\\
46.01	0.0044063494774837\\
47.01	0.00440635242003476\\
48.01	0.00440635542419768\\
49.01	0.00440635849126676\\
50.01	0.00440636162256351\\
51.01	0.0044063648194374\\
52.01	0.00440636808326623\\
53.01	0.00440637141545691\\
54.01	0.00440637481744614\\
55.01	0.00440637829070072\\
56.01	0.00440638183671864\\
57.01	0.00440638545702918\\
58.01	0.00440638915319415\\
59.01	0.00440639292680821\\
60.01	0.0044063967794996\\
61.01	0.00440640071293111\\
62.01	0.0044064047288004\\
63.01	0.00440640882884109\\
64.01	0.00440641301482347\\
65.01	0.00440641728855513\\
66.01	0.00440642165188182\\
67.01	0.00440642610668807\\
68.01	0.0044064306548985\\
69.01	0.0044064352984782\\
70.01	0.00440644003943368\\
71.01	0.00440644487981423\\
72.01	0.00440644982171214\\
73.01	0.00440645486726378\\
74.01	0.00440646001865092\\
75.01	0.00440646527810121\\
76.01	0.00440647064788945\\
77.01	0.00440647613033868\\
78.01	0.00440648172782072\\
79.01	0.00440648744275766\\
80.01	0.00440649327762303\\
81.01	0.00440649923494237\\
82.01	0.00440650531729445\\
83.01	0.004406511527313\\
84.01	0.00440651786768717\\
85.01	0.00440652434116291\\
86.01	0.0044065309505447\\
87.01	0.00440653769869574\\
88.01	0.00440654458854012\\
89.01	0.00440655162306375\\
90.01	0.00440655880531568\\
91.01	0.00440656613840951\\
92.01	0.00440657362552467\\
93.01	0.00440658126990784\\
94.01	0.00440658907487454\\
95.01	0.00440659704381034\\
96.01	0.00440660518017254\\
97.01	0.00440661348749162\\
98.01	0.00440662196937291\\
99.01	0.00440663062949775\\
100.01	0.00440663947162587\\
101.01	0.0044066484995962\\
102.01	0.00440665771732905\\
103.01	0.00440666712882812\\
104.01	0.00440667673818164\\
105.01	0.00440668654956447\\
106.01	0.00440669656723994\\
107.01	0.00440670679556165\\
108.01	0.00440671723897552\\
109.01	0.00440672790202165\\
110.01	0.00440673878933663\\
111.01	0.00440674990565476\\
112.01	0.00440676125581123\\
113.01	0.00440677284474333\\
114.01	0.00440678467749334\\
115.01	0.00440679675921023\\
116.01	0.0044068090951522\\
117.01	0.00440682169068929\\
118.01	0.00440683455130486\\
119.01	0.00440684768259891\\
120.01	0.00440686109029029\\
121.01	0.00440687478021913\\
122.01	0.00440688875834937\\
123.01	0.00440690303077154\\
124.01	0.00440691760370533\\
125.01	0.00440693248350259\\
126.01	0.00440694767664984\\
127.01	0.00440696318977146\\
128.01	0.00440697902963225\\
129.01	0.00440699520314074\\
130.01	0.00440701171735211\\
131.01	0.00440702857947142\\
132.01	0.00440704579685672\\
133.01	0.00440706337702228\\
134.01	0.00440708132764198\\
135.01	0.00440709965655289\\
136.01	0.00440711837175821\\
137.01	0.0044071374814314\\
138.01	0.00440715699391962\\
139.01	0.00440717691774739\\
140.01	0.00440719726162048\\
141.01	0.00440721803442939\\
142.01	0.0044072392452541\\
143.01	0.00440726090336699\\
144.01	0.00440728301823819\\
145.01	0.00440730559953875\\
146.01	0.00440732865714562\\
147.01	0.00440735220114562\\
148.01	0.00440737624184034\\
149.01	0.0044074007897504\\
150.01	0.00440742585562011\\
151.01	0.00440745145042237\\
152.01	0.00440747758536389\\
153.01	0.00440750427188961\\
154.01	0.00440753152168795\\
155.01	0.00440755934669618\\
156.01	0.00440758775910581\\
157.01	0.00440761677136776\\
158.01	0.00440764639619827\\
159.01	0.00440767664658397\\
160.01	0.00440770753578837\\
161.01	0.00440773907735721\\
162.01	0.00440777128512501\\
163.01	0.00440780417322095\\
164.01	0.00440783775607496\\
165.01	0.00440787204842495\\
166.01	0.00440790706532249\\
167.01	0.00440794282213992\\
168.01	0.00440797933457754\\
169.01	0.00440801661867002\\
170.01	0.00440805469079412\\
171.01	0.00440809356767564\\
172.01	0.0044081332663972\\
173.01	0.00440817380440565\\
174.01	0.00440821519951994\\
175.01	0.00440825746993932\\
176.01	0.00440830063425132\\
177.01	0.00440834471143978\\
178.01	0.00440838972089398\\
179.01	0.0044084356824169\\
180.01	0.0044084826162344\\
181.01	0.00440853054300401\\
182.01	0.00440857948382432\\
183.01	0.00440862946024473\\
184.01	0.00440868049427444\\
185.01	0.00440873260839318\\
186.01	0.00440878582556084\\
187.01	0.00440884016922796\\
188.01	0.00440889566334617\\
189.01	0.00440895233237887\\
190.01	0.00440901020131275\\
191.01	0.00440906929566838\\
192.01	0.00440912964151229\\
193.01	0.00440919126546833\\
194.01	0.00440925419473025\\
195.01	0.00440931845707281\\
196.01	0.00440938408086572\\
197.01	0.00440945109508563\\
198.01	0.00440951952932955\\
199.01	0.00440958941382805\\
200.01	0.00440966077945887\\
201.01	0.00440973365776112\\
202.01	0.00440980808094972\\
203.01	0.00440988408192954\\
204.01	0.00440996169431064\\
205.01	0.00441004095242328\\
206.01	0.00441012189133402\\
207.01	0.00441020454686088\\
208.01	0.00441028895559045\\
209.01	0.0044103751548937\\
210.01	0.00441046318294396\\
211.01	0.0044105530787333\\
212.01	0.00441064488209088\\
213.01	0.0044107386337011\\
214.01	0.00441083437512203\\
215.01	0.00441093214880439\\
216.01	0.00441103199811108\\
217.01	0.00441113396733675\\
218.01	0.00441123810172846\\
219.01	0.00441134444750611\\
220.01	0.00441145305188383\\
221.01	0.00441156396309121\\
222.01	0.00441167723039612\\
223.01	0.0044117929041269\\
224.01	0.00441191103569558\\
225.01	0.00441203167762177\\
226.01	0.00441215488355656\\
227.01	0.00441228070830745\\
228.01	0.00441240920786376\\
229.01	0.00441254043942216\\
230.01	0.00441267446141355\\
231.01	0.00441281133352994\\
232.01	0.00441295111675212\\
233.01	0.00441309387337799\\
234.01	0.00441323966705155\\
235.01	0.00441338856279257\\
236.01	0.00441354062702679\\
237.01	0.00441369592761702\\
238.01	0.00441385453389485\\
239.01	0.00441401651669299\\
240.01	0.00441418194837837\\
241.01	0.00441435090288646\\
242.01	0.00441452345575567\\
243.01	0.00441469968416305\\
244.01	0.00441487966696047\\
245.01	0.00441506348471231\\
246.01	0.00441525121973288\\
247.01	0.00441544295612602\\
248.01	0.0044156387798247\\
249.01	0.00441583877863176\\
250.01	0.00441604304226205\\
251.01	0.00441625166238495\\
252.01	0.00441646473266812\\
253.01	0.00441668234882274\\
254.01	0.00441690460864883\\
255.01	0.00441713161208299\\
256.01	0.00441736346124592\\
257.01	0.00441760026049218\\
258.01	0.00441784211646047\\
259.01	0.00441808913812528\\
260.01	0.00441834143685001\\
261.01	0.00441859912644132\\
262.01	0.00441886232320449\\
263.01	0.00441913114600051\\
264.01	0.00441940571630475\\
265.01	0.00441968615826624\\
266.01	0.00441997259876925\\
267.01	0.00442026516749622\\
268.01	0.00442056399699201\\
269.01	0.00442086922273013\\
270.01	0.0044211809831799\\
271.01	0.00442149941987666\\
272.01	0.00442182467749223\\
273.01	0.00442215690390849\\
274.01	0.00442249625029138\\
275.01	0.00442284287116884\\
276.01	0.00442319692450853\\
277.01	0.00442355857179931\\
278.01	0.00442392797813407\\
279.01	0.00442430531229437\\
280.01	0.00442469074683832\\
281.01	0.00442508445819004\\
282.01	0.0044254866267318\\
283.01	0.00442589743689811\\
284.01	0.0044263170772733\\
285.01	0.00442674574069093\\
286.01	0.00442718362433586\\
287.01	0.00442763092984971\\
288.01	0.004428087863439\\
289.01	0.00442855463598554\\
290.01	0.00442903146316111\\
291.01	0.00442951856554408\\
292.01	0.00443001616874006\\
293.01	0.00443052450350619\\
294.01	0.00443104380587758\\
295.01	0.00443157431729853\\
296.01	0.00443211628475747\\
297.01	0.00443266996092481\\
298.01	0.00443323560429575\\
299.01	0.00443381347933643\\
300.01	0.00443440385663482\\
301.01	0.00443500701305513\\
302.01	0.00443562323189807\\
303.01	0.00443625280306416\\
304.01	0.00443689602322331\\
305.01	0.00443755319598767\\
306.01	0.00443822463209094\\
307.01	0.00443891064957251\\
308.01	0.0044396115739662\\
309.01	0.00444032773849545\\
310.01	0.00444105948427355\\
311.01	0.00444180716050999\\
312.01	0.00444257112472224\\
313.01	0.00444335174295431\\
314.01	0.00444414939000094\\
315.01	0.00444496444963761\\
316.01	0.00444579731485817\\
317.01	0.00444664838811785\\
318.01	0.0044475180815835\\
319.01	0.00444840681738912\\
320.01	0.00444931502789954\\
321.01	0.00445024315597932\\
322.01	0.0044511916552685\\
323.01	0.00445216099046462\\
324.01	0.00445315163761006\\
325.01	0.0044541640843857\\
326.01	0.00445519883040953\\
327.01	0.00445625638753941\\
328.01	0.0044573372801805\\
329.01	0.00445844204559541\\
330.01	0.00445957123421578\\
331.01	0.00446072540995518\\
332.01	0.00446190515052064\\
333.01	0.00446311104772117\\
334.01	0.00446434370777075\\
335.01	0.00446560375158356\\
336.01	0.0044668918150572\\
337.01	0.00446820854933985\\
338.01	0.00446955462107711\\
339.01	0.00447093071263302\\
340.01	0.00447233752227598\\
341.01	0.00447377576432524\\
342.01	0.00447524616924487\\
343.01	0.00447674948367514\\
344.01	0.00447828647038827\\
345.01	0.00447985790815145\\
346.01	0.00448146459147909\\
347.01	0.0044831073302514\\
348.01	0.0044847869491738\\
349.01	0.00448650428704578\\
350.01	0.00448826019580327\\
351.01	0.00449005553929223\\
352.01	0.00449189119172355\\
353.01	0.00449376803575088\\
354.01	0.00449568696010501\\
355.01	0.00449764885670435\\
356.01	0.00449965461715125\\
357.01	0.00450170512850824\\
358.01	0.0045038012682342\\
359.01	0.00450594389814001\\
360.01	0.00450813385720904\\
361.01	0.00451037195310497\\
362.01	0.00451265895217097\\
363.01	0.00451499556770892\\
364.01	0.00451738244630782\\
365.01	0.00451982015198824\\
366.01	0.00452230914792892\\
367.01	0.00452484977556643\\
368.01	0.00452744223090766\\
369.01	0.00453008653798694\\
370.01	0.00453278251955661\\
371.01	0.00453552976534603\\
372.01	0.00453832759860198\\
373.01	0.00454117504219044\\
374.01	0.00454407078637301\\
375.01	0.00454701316158006\\
376.01	0.00455000012124152\\
377.01	0.00455302924221239\\
378.01	0.00455609775384222\\
379.01	0.00455920261167952\\
380.01	0.00456234063874891\\
381.01	0.00456550876703976\\
382.01	0.00456870442538744\\
383.01	0.00457192613875745\\
384.01	0.00457517443009159\\
385.01	0.00457845526305113\\
386.01	0.0045817974796559\\
387.01	0.00458520946202622\\
388.01	0.00458869264373901\\
389.01	0.00459224848572395\\
390.01	0.00459587847657204\\
391.01	0.00459958413282364\\
392.01	0.00460336699923236\\
393.01	0.00460722864900213\\
394.01	0.00461117068399346\\
395.01	0.00461519473489354\\
396.01	0.00461930246134645\\
397.01	0.00462349555203738\\
398.01	0.00462777572472496\\
399.01	0.0046321447262158\\
400.01	0.00463660433227225\\
401.01	0.00464115634744748\\
402.01	0.00464580260483666\\
403.01	0.00465054496573504\\
404.01	0.00465538531919338\\
405.01	0.00466032558145577\\
406.01	0.00466536769526826\\
407.01	0.00467051362904262\\
408.01	0.0046757653758587\\
409.01	0.00468112495228747\\
410.01	0.00468659439701416\\
411.01	0.00469217576923976\\
412.01	0.00469787114683555\\
413.01	0.00470368262422466\\
414.01	0.00470961230996053\\
415.01	0.00471566232396985\\
416.01	0.00472183479442334\\
417.01	0.00472813185419708\\
418.01	0.00473455563688004\\
419.01	0.00474110827228083\\
420.01	0.00474779188138451\\
421.01	0.00475460857070206\\
422.01	0.00476156042595463\\
423.01	0.0047686495050266\\
424.01	0.00477587783011849\\
425.01	0.00478324737902673\\
426.01	0.00479076007546984\\
427.01	0.00479841777838035\\
428.01	0.00480622227007421\\
429.01	0.00481417524321116\\
430.01	0.00482227828645251\\
431.01	0.00483053286872971\\
432.01	0.00483894032203472\\
433.01	0.00484750182265346\\
434.01	0.00485621837077113\\
435.01	0.00486509076839704\\
436.01	0.00487411959558005\\
437.01	0.00488330518491606\\
438.01	0.00489264759439711\\
439.01	0.00490214657870898\\
440.01	0.00491180155916121\\
441.01	0.00492161159253393\\
442.01	0.00493157533925231\\
443.01	0.00494169103145929\\
444.01	0.00495195644175643\\
445.01	0.00496236885363065\\
446.01	0.00497292503488503\\
447.01	0.00498362121575813\\
448.01	0.00499445307385384\\
449.01	0.00500541572852012\\
450.01	0.00501650374791437\\
451.01	0.00502771117267692\\
452.01	0.00503903156088574\\
453.01	0.00505045805977243\\
454.01	0.00506198351047833\\
455.01	0.00507360059285489\\
456.01	0.00508530201781257\\
457.01	0.00509708077479794\\
458.01	0.00510893044128637\\
459.01	0.00512084555923308\\
460.01	0.00513282207947717\\
461.01	0.00514485786805845\\
462.01	0.00515695325669842\\
463.01	0.00516911160101229\\
464.01	0.00518133978105872\\
465.01	0.00519364853487436\\
466.01	0.00520605244994091\\
467.01	0.00521856934055163\\
468.01	0.00523121859675642\\
469.01	0.00524401443556734\\
470.01	0.00525695595333581\\
471.01	0.00527003895708189\\
472.01	0.0052832589415233\\
473.01	0.00529661111701986\\
474.01	0.00531009044759288\\
475.01	0.00532369170046118\\
476.01	0.00533740950855351\\
477.01	0.0053512384473979\\
478.01	0.00536517312763309\\
479.01	0.00537920830408881\\
480.01	0.00539333900192615\\
481.01	0.00540756065978887\\
482.01	0.00542186928905644\\
483.01	0.00543626164696971\\
484.01	0.00545073541969959\\
485.01	0.00546528940926458\\
486.01	0.00547992371550203\\
487.01	0.00549463990104536\\
488.01	0.00550944112351877\\
489.01	0.00552433221510523\\
490.01	0.00553931968568339\\
491.01	0.00555441162261239\\
492.01	0.00556961745924792\\
493.01	0.00558494758736217\\
494.01	0.0056004127988741\\
495.01	0.00561602356490333\\
496.01	0.00563178920303264\\
497.01	0.00564771705855759\\
498.01	0.00566381198048466\\
499.01	0.00568007724959183\\
500.01	0.00569651658050933\\
501.01	0.00571313456604973\\
502.01	0.00572993672726638\\
503.01	0.00574692954501591\\
504.01	0.0057641204693081\\
505.01	0.00578151790174496\\
506.01	0.00579913114680775\\
507.01	0.00581697032881338\\
508.01	0.00583504627322513\\
509.01	0.00585337035387697\\
510.01	0.00587195431173997\\
511.01	0.0058908100562076\\
512.01	0.00590994946638894\\
513.01	0.00592938421704352\\
514.01	0.00594912566032621\\
515.01	0.005969184797927\\
516.01	0.00598957237400632\\
517.01	0.00601029909958091\\
518.01	0.00603137594750737\\
519.01	0.00605281429317925\\
520.01	0.00607462587650683\\
521.01	0.00609682273422721\\
522.01	0.00611941712827492\\
523.01	0.00614242147440214\\
524.01	0.00616584827601105\\
525.01	0.00618971006870538\\
526.01	0.00621401938102251\\
527.01	0.00623878871588064\\
528.01	0.00626403055516378\\
529.01	0.00628975738633661\\
530.01	0.00631598174502204\\
531.01	0.0063427162616497\\
532.01	0.00636997369532064\\
533.01	0.00639776693951445\\
534.01	0.00642610900545185\\
535.01	0.00645501300392987\\
536.01	0.00648449213262774\\
537.01	0.00651455966959223\\
538.01	0.00654522897277823\\
539.01	0.00657651348475956\\
540.01	0.00660842674092671\\
541.01	0.00664098237881587\\
542.01	0.00667419414588415\\
543.01	0.00670807590334808\\
544.01	0.00674264162486408\\
545.01	0.00677790539086053\\
546.01	0.00681388138114321\\
547.01	0.00685058386763962\\
548.01	0.00688802720720362\\
549.01	0.00692622583376501\\
550.01	0.00696519424901141\\
551.01	0.00700494701080015\\
552.01	0.00704549871862432\\
553.01	0.00708686399568258\\
554.01	0.00712905746738202\\
555.01	0.00717209373633373\\
556.01	0.00721598735393222\\
557.01	0.00726075278837836\\
558.01	0.00730640438870886\\
559.01	0.0073529563442945\\
560.01	0.0074004226392819\\
561.01	0.00744881700148302\\
562.01	0.0074981528452527\\
563.01	0.00754844320790874\\
564.01	0.00759970067923708\\
565.01	0.00765193732357281\\
566.01	0.00770516459387936\\
567.01	0.00775939323719052\\
568.01	0.0078146331907652\\
569.01	0.00787089346830959\\
570.01	0.00792818203563622\\
571.01	0.00798650567515498\\
572.01	0.00804586983862646\\
573.01	0.00810627848766415\\
574.01	0.00816773392156107\\
575.01	0.00823023659215375\\
576.01	0.00829378490563244\\
577.01	0.00835837501148391\\
578.01	0.00842400057911648\\
579.01	0.00849065256320018\\
580.01	0.00855831895938081\\
581.01	0.00862698455284182\\
582.01	0.00869663066323415\\
583.01	0.00876723489083508\\
584.01	0.00883877087050155\\
585.01	0.00891120804214121\\
586.01	0.00898451144914387\\
587.01	0.00905864157963132\\
588.01	0.00913355426966545\\
589.01	0.00920920069290074\\
590.01	0.0092855274678398\\
591.01	0.00936247692215228\\
592.01	0.00943998756383507\\
593.01	0.00951799482179795\\
594.01	0.00959643213433056\\
595.01	0.00967523248356592\\
596.01	0.00975433049838055\\
597.01	0.00983365065339071\\
598.01	0.00990866201848071\\
599.01	0.00997087280416276\\
599.02	0.00997138072163725\\
599.03	0.009971885576258\\
599.04	0.00997238733820211\\
599.05	0.00997288597735272\\
599.06	0.00997338146329604\\
599.07	0.00997387376531845\\
599.08	0.00997436285240349\\
599.09	0.00997484869322892\\
599.1	0.00997533125616361\\
599.11	0.00997581050926455\\
599.12	0.00997628642027372\\
599.13	0.00997675895661498\\
599.14	0.0099772280853909\\
599.15	0.00997769377337961\\
599.16	0.00997815598703157\\
599.17	0.00997861469246631\\
599.18	0.00997906985546915\\
599.19	0.00997952144148792\\
599.2	0.00997996941562957\\
599.21	0.00998041374265684\\
599.22	0.0099808543869848\\
599.23	0.00998129131267744\\
599.24	0.00998172448344418\\
599.25	0.00998215386263636\\
599.26	0.00998257941258353\\
599.27	0.00998300109279825\\
599.28	0.00998341886238981\\
599.29	0.00998383268006024\\
599.3	0.00998424250410032\\
599.31	0.00998464829238543\\
599.32	0.00998505000237151\\
599.33	0.00998544759109087\\
599.34	0.00998584101514799\\
599.35	0.00998623023071532\\
599.36	0.00998661519352895\\
599.37	0.00998699585888436\\
599.38	0.00998737218163197\\
599.39	0.00998774411617281\\
599.4	0.00998811161645403\\
599.41	0.00998847463596443\\
599.42	0.0099888331277299\\
599.43	0.00998918704430885\\
599.44	0.00998953633778757\\
599.45	0.00998988095977558\\
599.46	0.00999022086140088\\
599.47	0.00999055599330522\\
599.48	0.00999088630563925\\
599.49	0.00999121174805768\\
599.5	0.00999153226971435\\
599.51	0.0099918478192573\\
599.52	0.00999215834482374\\
599.53	0.00999246379403503\\
599.54	0.0099927641139915\\
599.55	0.00999305925126738\\
599.56	0.00999334915190553\\
599.57	0.0099936337614122\\
599.58	0.00999391302475173\\
599.59	0.00999418688634118\\
599.6	0.00999445529004489\\
599.61	0.00999471817916904\\
599.62	0.00999497549645613\\
599.63	0.00999522718407934\\
599.64	0.009995473183637\\
599.65	0.00999571343614678\\
599.66	0.00999594788204004\\
599.67	0.00999617646115599\\
599.68	0.0099963991127358\\
599.69	0.00999661577541675\\
599.7	0.00999682638722618\\
599.71	0.00999703088557551\\
599.72	0.00999722920725411\\
599.73	0.00999742128842315\\
599.74	0.00999760706460942\\
599.75	0.00999778647069897\\
599.76	0.00999795944093086\\
599.77	0.0099981259088907\\
599.78	0.0099982858075042\\
599.79	0.00999843906903064\\
599.8	0.00999858562505627\\
599.81	0.00999872540648767\\
599.82	0.00999885834354498\\
599.83	0.00999898436575515\\
599.84	0.00999910340194508\\
599.85	0.00999921538023465\\
599.86	0.00999932022802977\\
599.87	0.00999941787201528\\
599.88	0.00999950823814785\\
599.89	0.00999959125164875\\
599.9	0.00999966683699656\\
599.91	0.00999973491791987\\
599.92	0.0099997954173898\\
599.93	0.00999984825761255\\
599.94	0.00999989336002181\\
599.95	0.00999993064527112\\
599.96	0.00999996003322615\\
599.97	0.00999998144295691\\
599.98	0.00999999479272987\\
599.99	0.01\\
600	0.01\\
};
\addplot [color=blue!25!mycolor7,solid,forget plot]
  table[row sep=crcr]{%
0.01	0.00413645223902454\\
1.01	0.00413645300246634\\
2.01	0.00413645378183859\\
3.01	0.00413645457747443\\
4.01	0.0041364553897143\\
5.01	0.00413645621890542\\
6.01	0.00413645706540267\\
7.01	0.00413645792956811\\
8.01	0.00413645881177166\\
9.01	0.00413645971239082\\
10.01	0.00413646063181095\\
11.01	0.00413646157042567\\
12.01	0.00413646252863661\\
13.01	0.00413646350685418\\
14.01	0.00413646450549729\\
15.01	0.00413646552499354\\
16.01	0.00413646656577953\\
17.01	0.00413646762830092\\
18.01	0.00413646871301321\\
19.01	0.00413646982038111\\
20.01	0.00413647095087882\\
21.01	0.00413647210499092\\
22.01	0.00413647328321212\\
23.01	0.00413647448604743\\
24.01	0.0041364757140125\\
25.01	0.0041364769676339\\
26.01	0.00413647824744907\\
27.01	0.00413647955400687\\
28.01	0.00413648088786767\\
29.01	0.0041364822496036\\
30.01	0.00413648363979888\\
31.01	0.00413648505905004\\
32.01	0.00413648650796607\\
33.01	0.00413648798716877\\
34.01	0.00413648949729303\\
35.01	0.00413649103898717\\
36.01	0.00413649261291286\\
37.01	0.00413649421974589\\
38.01	0.00413649586017656\\
39.01	0.00413649753490919\\
40.01	0.0041364992446631\\
41.01	0.00413650099017301\\
42.01	0.00413650277218841\\
43.01	0.00413650459147534\\
44.01	0.00413650644881562\\
45.01	0.00413650834500721\\
46.01	0.00413651028086549\\
47.01	0.00413651225722261\\
48.01	0.00413651427492816\\
49.01	0.00413651633484986\\
50.01	0.0041365184378737\\
51.01	0.00413652058490413\\
52.01	0.00413652277686458\\
53.01	0.0041365250146983\\
54.01	0.00413652729936799\\
55.01	0.00413652963185687\\
56.01	0.00413653201316876\\
57.01	0.00413653444432874\\
58.01	0.00413653692638334\\
59.01	0.00413653946040121\\
60.01	0.00413654204747346\\
61.01	0.00413654468871429\\
62.01	0.00413654738526122\\
63.01	0.00413655013827585\\
64.01	0.00413655294894416\\
65.01	0.00413655581847719\\
66.01	0.00413655874811143\\
67.01	0.00413656173910959\\
68.01	0.00413656479276098\\
69.01	0.00413656791038176\\
70.01	0.00413657109331628\\
71.01	0.00413657434293679\\
72.01	0.00413657766064482\\
73.01	0.00413658104787133\\
74.01	0.00413658450607736\\
75.01	0.0041365880367549\\
76.01	0.00413659164142743\\
77.01	0.0041365953216502\\
78.01	0.0041365990790117\\
79.01	0.00413660291513383\\
80.01	0.00413660683167236\\
81.01	0.00413661083031852\\
82.01	0.00413661491279878\\
83.01	0.00413661908087617\\
84.01	0.00413662333635101\\
85.01	0.00413662768106166\\
86.01	0.00413663211688484\\
87.01	0.00413663664573743\\
88.01	0.00413664126957637\\
89.01	0.00413664599039996\\
90.01	0.00413665081024897\\
91.01	0.00413665573120693\\
92.01	0.00413666075540151\\
93.01	0.00413666588500525\\
94.01	0.00413667112223645\\
95.01	0.00413667646936043\\
96.01	0.00413668192869014\\
97.01	0.00413668750258765\\
98.01	0.00413669319346451\\
99.01	0.00413669900378365\\
100.01	0.00413670493605961\\
101.01	0.00413671099286049\\
102.01	0.00413671717680839\\
103.01	0.00413672349058052\\
104.01	0.00413672993691108\\
105.01	0.00413673651859182\\
106.01	0.0041367432384736\\
107.01	0.00413675009946735\\
108.01	0.00413675710454571\\
109.01	0.00413676425674391\\
110.01	0.00413677155916145\\
111.01	0.00413677901496344\\
112.01	0.0041367866273818\\
113.01	0.00413679439971668\\
114.01	0.00413680233533802\\
115.01	0.00413681043768732\\
116.01	0.00413681871027842\\
117.01	0.0041368271566995\\
118.01	0.00413683578061494\\
119.01	0.00413684458576649\\
120.01	0.00413685357597467\\
121.01	0.00413686275514111\\
122.01	0.00413687212725002\\
123.01	0.00413688169636965\\
124.01	0.00413689146665469\\
125.01	0.00413690144234731\\
126.01	0.00413691162777957\\
127.01	0.00413692202737526\\
128.01	0.00413693264565181\\
129.01	0.00413694348722218\\
130.01	0.00413695455679676\\
131.01	0.00413696585918608\\
132.01	0.00413697739930197\\
133.01	0.0041369891821606\\
134.01	0.00413700121288418\\
135.01	0.00413701349670324\\
136.01	0.00413702603895943\\
137.01	0.0041370388451072\\
138.01	0.00413705192071638\\
139.01	0.00413706527147512\\
140.01	0.00413707890319162\\
141.01	0.00413709282179756\\
142.01	0.00413710703334966\\
143.01	0.00413712154403365\\
144.01	0.00413713636016562\\
145.01	0.00413715148819595\\
146.01	0.00413716693471143\\
147.01	0.00413718270643845\\
148.01	0.00413719881024602\\
149.01	0.0041372152531487\\
150.01	0.00413723204230966\\
151.01	0.00413724918504406\\
152.01	0.00413726668882178\\
153.01	0.0041372845612714\\
154.01	0.00413730281018307\\
155.01	0.00413732144351218\\
156.01	0.00413734046938279\\
157.01	0.00413735989609126\\
158.01	0.00413737973210988\\
159.01	0.00413739998609051\\
160.01	0.00413742066686876\\
161.01	0.00413744178346762\\
162.01	0.0041374633451015\\
163.01	0.00413748536118023\\
164.01	0.0041375078413135\\
165.01	0.00413753079531483\\
166.01	0.00413755423320596\\
167.01	0.00413757816522158\\
168.01	0.00413760260181349\\
169.01	0.0041376275536554\\
170.01	0.0041376530316477\\
171.01	0.00413767904692233\\
172.01	0.00413770561084746\\
173.01	0.00413773273503304\\
174.01	0.00413776043133533\\
175.01	0.00413778871186272\\
176.01	0.00413781758898084\\
177.01	0.00413784707531809\\
178.01	0.0041378771837714\\
179.01	0.00413790792751159\\
180.01	0.00413793931998961\\
181.01	0.00413797137494241\\
182.01	0.00413800410639917\\
183.01	0.00413803752868733\\
184.01	0.0041380716564393\\
185.01	0.00413810650459848\\
186.01	0.00413814208842662\\
187.01	0.0041381784235099\\
188.01	0.00413821552576664\\
189.01	0.00413825341145374\\
190.01	0.00413829209717424\\
191.01	0.00413833159988498\\
192.01	0.00413837193690397\\
193.01	0.00413841312591806\\
194.01	0.00413845518499083\\
195.01	0.00413849813257113\\
196.01	0.00413854198750095\\
197.01	0.00413858676902401\\
198.01	0.0041386324967943\\
199.01	0.00413867919088515\\
200.01	0.0041387268717981\\
201.01	0.00413877556047228\\
202.01	0.00413882527829372\\
203.01	0.00413887604710475\\
204.01	0.0041389278892144\\
205.01	0.00413898082740843\\
206.01	0.00413903488495899\\
207.01	0.00413909008563577\\
208.01	0.0041391464537166\\
209.01	0.00413920401399849\\
210.01	0.00413926279180882\\
211.01	0.00413932281301681\\
212.01	0.00413938410404531\\
213.01	0.00413944669188301\\
214.01	0.00413951060409641\\
215.01	0.00413957586884263\\
216.01	0.00413964251488226\\
217.01	0.00413971057159257\\
218.01	0.00413978006898054\\
219.01	0.00413985103769716\\
220.01	0.00413992350905104\\
221.01	0.00413999751502343\\
222.01	0.00414007308828199\\
223.01	0.00414015026219657\\
224.01	0.00414022907085443\\
225.01	0.00414030954907614\\
226.01	0.0041403917324311\\
227.01	0.00414047565725467\\
228.01	0.00414056136066472\\
229.01	0.00414064888057912\\
230.01	0.0041407382557329\\
231.01	0.00414082952569686\\
232.01	0.00414092273089588\\
233.01	0.00414101791262771\\
234.01	0.00414111511308253\\
235.01	0.00414121437536263\\
236.01	0.00414131574350272\\
237.01	0.00414141926249077\\
238.01	0.00414152497828936\\
239.01	0.00414163293785735\\
240.01	0.00414174318917206\\
241.01	0.00414185578125228\\
242.01	0.00414197076418168\\
243.01	0.00414208818913275\\
244.01	0.00414220810839117\\
245.01	0.00414233057538117\\
246.01	0.00414245564469116\\
247.01	0.00414258337210011\\
248.01	0.00414271381460452\\
249.01	0.0041428470304464\\
250.01	0.00414298307914145\\
251.01	0.00414312202150835\\
252.01	0.00414326391969813\\
253.01	0.00414340883722551\\
254.01	0.00414355683899971\\
255.01	0.00414370799135663\\
256.01	0.0041438623620923\\
257.01	0.00414402002049635\\
258.01	0.00414418103738673\\
259.01	0.00414434548514565\\
260.01	0.00414451343775553\\
261.01	0.00414468497083743\\
262.01	0.00414486016168879\\
263.01	0.00414503908932352\\
264.01	0.0041452218345121\\
265.01	0.00414540847982364\\
266.01	0.00414559910966864\\
267.01	0.00414579381034308\\
268.01	0.00414599267007339\\
269.01	0.00414619577906319\\
270.01	0.00414640322954103\\
271.01	0.00414661511580915\\
272.01	0.00414683153429449\\
273.01	0.00414705258360024\\
274.01	0.00414727836455965\\
275.01	0.00414750898029047\\
276.01	0.00414774453625195\\
277.01	0.00414798514030318\\
278.01	0.00414823090276233\\
279.01	0.00414848193646948\\
280.01	0.0041487383568493\\
281.01	0.00414900028197704\\
282.01	0.00414926783264594\\
283.01	0.00414954113243737\\
284.01	0.00414982030779178\\
285.01	0.00415010548808338\\
286.01	0.00415039680569666\\
287.01	0.00415069439610491\\
288.01	0.00415099839795185\\
289.01	0.0041513089531358\\
290.01	0.00415162620689657\\
291.01	0.00415195030790523\\
292.01	0.00415228140835719\\
293.01	0.00415261966406762\\
294.01	0.00415296523457169\\
295.01	0.00415331828322656\\
296.01	0.00415367897731833\\
297.01	0.00415404748817211\\
298.01	0.0041544239912662\\
299.01	0.00415480866635073\\
300.01	0.00415520169757005\\
301.01	0.00415560327359081\\
302.01	0.00415601358773349\\
303.01	0.00415643283811012\\
304.01	0.00415686122776703\\
305.01	0.00415729896483309\\
306.01	0.00415774626267394\\
307.01	0.0041582033400524\\
308.01	0.00415867042129625\\
309.01	0.00415914773647154\\
310.01	0.00415963552156495\\
311.01	0.00416013401867235\\
312.01	0.00416064347619738\\
313.01	0.00416116414905673\\
314.01	0.0041616962988969\\
315.01	0.0041622401943196\\
316.01	0.00416279611111757\\
317.01	0.0041633643325221\\
318.01	0.00416394514946271\\
319.01	0.00416453886083896\\
320.01	0.00416514577380607\\
321.01	0.00416576620407566\\
322.01	0.00416640047623104\\
323.01	0.00416704892406029\\
324.01	0.00416771189090702\\
325.01	0.00416838973004008\\
326.01	0.00416908280504448\\
327.01	0.00416979149023516\\
328.01	0.0041705161710944\\
329.01	0.00417125724473598\\
330.01	0.00417201512039816\\
331.01	0.00417279021996729\\
332.01	0.00417358297853551\\
333.01	0.00417439384499517\\
334.01	0.00417522328267431\\
335.01	0.00417607177001537\\
336.01	0.00417693980130271\\
337.01	0.00417782788744424\\
338.01	0.00417873655681118\\
339.01	0.00417966635614245\\
340.01	0.00418061785152207\\
341.01	0.00418159162943539\\
342.01	0.00418258829791465\\
343.01	0.00418360848778307\\
344.01	0.00418465285400919\\
345.01	0.0041857220771854\\
346.01	0.00418681686514386\\
347.01	0.00418793795472863\\
348.01	0.00418908611374225\\
349.01	0.00419026214308968\\
350.01	0.00419146687914455\\
351.01	0.0041927011963669\\
352.01	0.00419396601020493\\
353.01	0.00419526228031948\\
354.01	0.00419659101417189\\
355.01	0.00419795327102597\\
356.01	0.00419935016641657\\
357.01	0.00420078287714831\\
358.01	0.00420225264688966\\
359.01	0.00420376079244129\\
360.01	0.00420530871075799\\
361.01	0.00420689788681412\\
362.01	0.0042085299024041\\
363.01	0.0042102064459706\\
364.01	0.0042119293235499\\
365.01	0.00421370047090788\\
366.01	0.00421552196691868\\
367.01	0.00421739604819074\\
368.01	0.00421932512487386\\
369.01	0.00422131179747372\\
370.01	0.0042233588743284\\
371.01	0.00422546938915921\\
372.01	0.00422764661774756\\
373.01	0.00422989409227113\\
374.01	0.00423221561109659\\
375.01	0.00423461524077492\\
376.01	0.00423709730550745\\
377.01	0.00423966635727248\\
378.01	0.00424232711689224\\
379.01	0.00424508437225969\\
380.01	0.00424794281429006\\
381.01	0.00425090678330682\\
382.01	0.0042539798876709\\
383.01	0.00425716444136141\\
384.01	0.00426046064632322\\
385.01	0.00426386332049204\\
386.01	0.0042673453438255\\
387.01	0.00427090121902333\\
388.01	0.0042745325387893\\
389.01	0.00427824093201318\\
390.01	0.00428202806470326\\
391.01	0.00428589564094661\\
392.01	0.0042898454039004\\
393.01	0.00429387913681263\\
394.01	0.0042979986640759\\
395.01	0.00430220585231304\\
396.01	0.00430650261149555\\
397.01	0.00431089089609782\\
398.01	0.00431537270628513\\
399.01	0.00431995008913716\\
400.01	0.00432462513990922\\
401.01	0.00432940000332797\\
402.01	0.0043342768749257\\
403.01	0.00433925800241129\\
404.01	0.00434434568707703\\
405.01	0.00434954228524249\\
406.01	0.00435485020973335\\
407.01	0.00436027193139435\\
408.01	0.00436580998063417\\
409.01	0.00437146694899987\\
410.01	0.0043772454907771\\
411.01	0.00438314832461265\\
412.01	0.00438917823515285\\
413.01	0.00439533807469185\\
414.01	0.00440163076482111\\
415.01	0.00440805929806949\\
416.01	0.00441462673952259\\
417.01	0.0044213362284047\\
418.01	0.00442819097960613\\
419.01	0.00443519428513427\\
420.01	0.00444234951546026\\
421.01	0.00444966012073209\\
422.01	0.00445712963181495\\
423.01	0.00446476166111412\\
424.01	0.00447255990312809\\
425.01	0.00448052813466672\\
426.01	0.00448867021466183\\
427.01	0.00449699008347793\\
428.01	0.00450549176162084\\
429.01	0.0045141793477156\\
430.01	0.00452305701561066\\
431.01	0.00453212901043024\\
432.01	0.00454139964337136\\
433.01	0.00455087328500416\\
434.01	0.00456055435678991\\
435.01	0.00457044732048396\\
436.01	0.0045805566650301\\
437.01	0.00459088689049059\\
438.01	0.00460144248847508\\
439.01	0.0046122279184453\\
440.01	0.00462324757917482\\
441.01	0.00463450577452604\\
442.01	0.00464600667258623\\
443.01	0.00465775425706393\\
444.01	0.00466975226970179\\
445.01	0.00468200414230838\\
446.01	0.00469451291685857\\
447.01	0.00470728115197526\\
448.01	0.00472031081399361\\
449.01	0.00473360315075769\\
450.01	0.00474715854634261\\
451.01	0.00476097635509615\\
452.01	0.00477505471382782\\
453.01	0.00478939033177071\\
454.01	0.00480397825926507\\
455.01	0.00481881163820742\\
456.01	0.00483388144051397\\
457.01	0.00484917620564145\\
458.01	0.00486468179525587\\
459.01	0.00488038119337847\\
460.01	0.00489625439505458\\
461.01	0.00491227844760229\\
462.01	0.00492842773829337\\
463.01	0.00494467466437917\\
464.01	0.00496099088050544\\
465.01	0.00497734940141575\\
466.01	0.00499372795367321\\
467.01	0.00501011413174371\\
468.01	0.0050265133947317\\
469.01	0.00504306504391844\\
470.01	0.0050598695421144\\
471.01	0.00507692297182101\\
472.01	0.00509422048189447\\
473.01	0.00511175620985842\\
474.01	0.00512952320483855\\
475.01	0.0051475133534548\\
476.01	0.00516571731177395\\
477.01	0.00518412444739152\\
478.01	0.00520272279690785\\
479.01	0.00522149904552283\\
480.01	0.00524043853601257\\
481.01	0.00525952531454729\\
482.01	0.00527874222508858\\
483.01	0.00529807106619583\\
484.01	0.00531749282573263\\
485.01	0.00533698801071057\\
486.01	0.0053565370906694\\
487.01	0.00537612107299407\\
488.01	0.00539572222640368\\
489.01	0.0054153249630382\\
490.01	0.00543491687789839\\
491.01	0.00545448992356011\\
492.01	0.00547404166447779\\
493.01	0.00549357650763554\\
494.01	0.00551310673116578\\
495.01	0.00553265301150467\\
496.01	0.00555224397090921\\
497.01	0.00557191400421153\\
498.01	0.00559169612135395\\
499.01	0.00561159653081639\\
500.01	0.0056316070469707\\
501.01	0.00565172015606437\\
502.01	0.00567192950010262\\
503.01	0.0056922301992143\\
504.01	0.00571261918965564\\
505.01	0.00573309556467543\\
506.01	0.00575366090006868\\
507.01	0.00577431953984645\\
508.01	0.00579507881024807\\
509.01	0.00581594912282832\\
510.01	0.00583694392057781\\
511.01	0.00585807941681996\\
512.01	0.00587937407805668\\
513.01	0.00590084781407138\\
514.01	0.00592252086944919\\
515.01	0.00594441247270583\\
516.01	0.00596653941150554\\
517.01	0.00598891490954435\\
518.01	0.00601154936829866\\
519.01	0.00603445364319628\\
520.01	0.00605764003084672\\
521.01	0.00608112221141253\\
522.01	0.00610491512470457\\
523.01	0.00612903478785888\\
524.01	0.00615349805588999\\
525.01	0.00617832233322134\\
526.01	0.00620352525356942\\
527.01	0.00622912435644384\\
528.01	0.0062551368000985\\
529.01	0.00628157916136718\\
530.01	0.00630846737794379\\
531.01	0.00633581688014802\\
532.01	0.00636364292158714\\
533.01	0.00639196096373634\\
534.01	0.00642078677220527\\
535.01	0.00645013631912694\\
536.01	0.00648002567432941\\
537.01	0.00651047090700989\\
538.01	0.00654148800648144\\
539.01	0.00657309283000611\\
540.01	0.00660530108354772\\
541.01	0.00663812833697643\\
542.01	0.00667159006859329\\
543.01	0.00670570172519962\\
544.01	0.00674047877514396\\
545.01	0.00677593672836852\\
546.01	0.00681209111868202\\
547.01	0.00684895747719921\\
548.01	0.00688655131297821\\
549.01	0.00692488810181778\\
550.01	0.00696398328188797\\
551.01	0.00700385225360844\\
552.01	0.00704451038011664\\
553.01	0.00708597298416507\\
554.01	0.00712825533781296\\
555.01	0.00717137264321357\\
556.01	0.00721534000599387\\
557.01	0.00726017240483853\\
558.01	0.00730588465886365\\
559.01	0.00735249139193321\\
560.01	0.0074000069925621\\
561.01	0.00744844556806246\\
562.01	0.00749782089178799\\
563.01	0.00754814634269311\\
564.01	0.00759943483685668\\
565.01	0.00765169875093423\\
566.01	0.00770494983745968\\
567.01	0.0077591991315086\\
568.01	0.0078144568479118\\
569.01	0.00787073226819176\\
570.01	0.00792803361649234\\
571.01	0.00798636792389737\\
572.01	0.00804574088066074\\
573.01	0.00810615667597409\\
574.01	0.00816761782497417\\
575.01	0.0082301249827724\\
576.01	0.00829367674544343\\
577.01	0.00835826943818673\\
578.01	0.00842389689126298\\
579.01	0.00849055020480459\\
580.01	0.00855821750423072\\
581.01	0.00862688368879396\\
582.01	0.00869653017680659\\
583.01	0.00876713465240274\\
584.01	0.00883867082037334\\
585.01	0.00891110817774947\\
586.01	0.00898441181351145\\
587.01	0.0090585422512016\\
588.01	0.00913345535348011\\
589.01	0.00920910231300336\\
590.01	0.00928542976066912\\
591.01	0.00936238003058418\\
592.01	0.00943989163144331\\
593.01	0.00951789998683473\\
594.01	0.00959633852288064\\
595.01	0.00967514020129159\\
596.01	0.00975423962023931\\
597.01	0.00983357023250315\\
598.01	0.00990866201848071\\
599.01	0.00997087280416276\\
599.02	0.00997138072163725\\
599.03	0.00997188557625799\\
599.04	0.00997238733820211\\
599.05	0.00997288597735272\\
599.06	0.00997338146329604\\
599.07	0.00997387376531845\\
599.08	0.00997436285240349\\
599.09	0.00997484869322892\\
599.1	0.00997533125616361\\
599.11	0.00997581050926455\\
599.12	0.00997628642027372\\
599.13	0.00997675895661497\\
599.14	0.0099772280853909\\
599.15	0.00997769377337961\\
599.16	0.00997815598703157\\
599.17	0.00997861469246631\\
599.18	0.00997906985546915\\
599.19	0.00997952144148792\\
599.2	0.00997996941562957\\
599.21	0.00998041374265684\\
599.22	0.0099808543869848\\
599.23	0.00998129131267744\\
599.24	0.00998172448344418\\
599.25	0.00998215386263636\\
599.26	0.00998257941258354\\
599.27	0.00998300109279825\\
599.28	0.00998341886238981\\
599.29	0.00998383268006024\\
599.3	0.00998424250410032\\
599.31	0.00998464829238543\\
599.32	0.00998505000237151\\
599.33	0.00998544759109087\\
599.34	0.00998584101514799\\
599.35	0.00998623023071532\\
599.36	0.00998661519352896\\
599.37	0.00998699585888436\\
599.38	0.00998737218163197\\
599.39	0.00998774411617281\\
599.4	0.00998811161645403\\
599.41	0.00998847463596443\\
599.42	0.0099888331277299\\
599.43	0.00998918704430885\\
599.44	0.00998953633778757\\
599.45	0.00998988095977558\\
599.46	0.00999022086140088\\
599.47	0.00999055599330522\\
599.48	0.00999088630563925\\
599.49	0.00999121174805768\\
599.5	0.00999153226971435\\
599.51	0.0099918478192573\\
599.52	0.00999215834482374\\
599.53	0.00999246379403503\\
599.54	0.0099927641139915\\
599.55	0.00999305925126738\\
599.56	0.00999334915190553\\
599.57	0.0099936337614122\\
599.58	0.00999391302475173\\
599.59	0.00999418688634118\\
599.6	0.00999445529004489\\
599.61	0.00999471817916904\\
599.62	0.00999497549645613\\
599.63	0.00999522718407934\\
599.64	0.009995473183637\\
599.65	0.00999571343614678\\
599.66	0.00999594788204004\\
599.67	0.00999617646115598\\
599.68	0.0099963991127358\\
599.69	0.00999661577541675\\
599.7	0.00999682638722618\\
599.71	0.00999703088557551\\
599.72	0.00999722920725411\\
599.73	0.00999742128842315\\
599.74	0.00999760706460942\\
599.75	0.00999778647069897\\
599.76	0.00999795944093086\\
599.77	0.0099981259088907\\
599.78	0.0099982858075042\\
599.79	0.00999843906903064\\
599.8	0.00999858562505627\\
599.81	0.00999872540648767\\
599.82	0.00999885834354498\\
599.83	0.00999898436575515\\
599.84	0.00999910340194508\\
599.85	0.00999921538023465\\
599.86	0.00999932022802977\\
599.87	0.00999941787201528\\
599.88	0.00999950823814785\\
599.89	0.00999959125164875\\
599.9	0.00999966683699656\\
599.91	0.00999973491791987\\
599.92	0.0099997954173898\\
599.93	0.00999984825761255\\
599.94	0.00999989336002181\\
599.95	0.00999993064527112\\
599.96	0.00999996003322615\\
599.97	0.00999998144295691\\
599.98	0.00999999479272987\\
599.99	0.01\\
600	0.01\\
};
\addplot [color=mycolor9,solid,forget plot]
  table[row sep=crcr]{%
0.01	0.00400079225488101\\
1.01	0.00400079282381871\\
2.01	0.00400079340463527\\
3.01	0.00400079399757922\\
4.01	0.00400079460290462\\
5.01	0.00400079522087064\\
6.01	0.00400079585174193\\
7.01	0.00400079649578867\\
8.01	0.00400079715328687\\
9.01	0.0040007978245182\\
10.01	0.00400079850977031\\
11.01	0.00400079920933689\\
12.01	0.00400079992351778\\
13.01	0.00400080065261907\\
14.01	0.00400080139695326\\
15.01	0.00400080215683971\\
16.01	0.00400080293260409\\
17.01	0.00400080372457927\\
18.01	0.00400080453310467\\
19.01	0.00400080535852729\\
20.01	0.00400080620120127\\
21.01	0.0040008070614881\\
22.01	0.00400080793975696\\
23.01	0.00400080883638475\\
24.01	0.00400080975175637\\
25.01	0.00400081068626465\\
26.01	0.00400081164031092\\
27.01	0.00400081261430475\\
28.01	0.00400081360866447\\
29.01	0.00400081462381706\\
30.01	0.00400081566019855\\
31.01	0.00400081671825407\\
32.01	0.00400081779843835\\
33.01	0.00400081890121556\\
34.01	0.00400082002705943\\
35.01	0.00400082117645384\\
36.01	0.00400082234989289\\
37.01	0.00400082354788113\\
38.01	0.00400082477093339\\
39.01	0.00400082601957574\\
40.01	0.0040008272943452\\
41.01	0.00400082859578976\\
42.01	0.00400082992446962\\
43.01	0.0040008312809561\\
44.01	0.00400083266583293\\
45.01	0.00400083407969611\\
46.01	0.00400083552315401\\
47.01	0.00400083699682792\\
48.01	0.00400083850135223\\
49.01	0.00400084003737461\\
50.01	0.00400084160555644\\
51.01	0.00400084320657304\\
52.01	0.00400084484111402\\
53.01	0.00400084650988335\\
54.01	0.00400084821359988\\
55.01	0.00400084995299756\\
56.01	0.00400085172882597\\
57.01	0.00400085354185044\\
58.01	0.0040008553928523\\
59.01	0.00400085728262939\\
60.01	0.00400085921199654\\
61.01	0.00400086118178546\\
62.01	0.00400086319284565\\
63.01	0.00400086524604443\\
64.01	0.00400086734226725\\
65.01	0.0040008694824184\\
66.01	0.00400087166742125\\
67.01	0.00400087389821861\\
68.01	0.00400087617577295\\
69.01	0.0040008785010673\\
70.01	0.0040008808751055\\
71.01	0.00400088329891232\\
72.01	0.00400088577353426\\
73.01	0.00400088830003978\\
74.01	0.00400089087951998\\
75.01	0.00400089351308901\\
76.01	0.0040008962018844\\
77.01	0.00400089894706777\\
78.01	0.00400090174982534\\
79.01	0.00400090461136811\\
80.01	0.00400090753293288\\
81.01	0.00400091051578234\\
82.01	0.00400091356120622\\
83.01	0.00400091667052092\\
84.01	0.00400091984507096\\
85.01	0.00400092308622907\\
86.01	0.00400092639539736\\
87.01	0.00400092977400694\\
88.01	0.00400093322351967\\
89.01	0.00400093674542806\\
90.01	0.00400094034125598\\
91.01	0.00400094401255949\\
92.01	0.00400094776092788\\
93.01	0.00400095158798371\\
94.01	0.00400095549538389\\
95.01	0.00400095948482035\\
96.01	0.00400096355802075\\
97.01	0.00400096771674907\\
98.01	0.00400097196280687\\
99.01	0.00400097629803352\\
100.01	0.00400098072430735\\
101.01	0.00400098524354638\\
102.01	0.00400098985770925\\
103.01	0.00400099456879582\\
104.01	0.00400099937884846\\
105.01	0.0040010042899525\\
106.01	0.00400100930423727\\
107.01	0.00400101442387746\\
108.01	0.00400101965109331\\
109.01	0.00400102498815241\\
110.01	0.00400103043737005\\
111.01	0.00400103600111069\\
112.01	0.00400104168178841\\
113.01	0.00400104748186875\\
114.01	0.00400105340386914\\
115.01	0.00400105945036039\\
116.01	0.00400106562396758\\
117.01	0.00400107192737163\\
118.01	0.00400107836330978\\
119.01	0.00400108493457729\\
120.01	0.00400109164402875\\
121.01	0.00400109849457905\\
122.01	0.00400110548920472\\
123.01	0.00400111263094542\\
124.01	0.00400111992290493\\
125.01	0.00400112736825299\\
126.01	0.00400113497022629\\
127.01	0.00400114273213006\\
128.01	0.00400115065733947\\
129.01	0.00400115874930135\\
130.01	0.00400116701153531\\
131.01	0.00400117544763559\\
132.01	0.00400118406127273\\
133.01	0.00400119285619467\\
134.01	0.00400120183622904\\
135.01	0.00400121100528446\\
136.01	0.0040012203673522\\
137.01	0.0040012299265086\\
138.01	0.00400123968691588\\
139.01	0.00400124965282467\\
140.01	0.00400125982857562\\
141.01	0.00400127021860153\\
142.01	0.00400128082742915\\
143.01	0.00400129165968113\\
144.01	0.00400130272007827\\
145.01	0.00400131401344138\\
146.01	0.00400132554469354\\
147.01	0.00400133731886239\\
148.01	0.00400134934108201\\
149.01	0.00400136161659541\\
150.01	0.00400137415075716\\
151.01	0.00400138694903507\\
152.01	0.00400140001701337\\
153.01	0.00400141336039447\\
154.01	0.00400142698500212\\
155.01	0.00400144089678368\\
156.01	0.00400145510181263\\
157.01	0.00400146960629152\\
158.01	0.00400148441655476\\
159.01	0.00400149953907127\\
160.01	0.00400151498044732\\
161.01	0.00400153074742948\\
162.01	0.00400154684690811\\
163.01	0.00400156328591966\\
164.01	0.00400158007165034\\
165.01	0.00400159721143896\\
166.01	0.00400161471278061\\
167.01	0.00400163258332972\\
168.01	0.00400165083090352\\
169.01	0.00400166946348571\\
170.01	0.00400168848922954\\
171.01	0.0040017079164623\\
172.01	0.00400172775368818\\
173.01	0.00400174800959244\\
174.01	0.00400176869304568\\
175.01	0.00400178981310693\\
176.01	0.00400181137902857\\
177.01	0.00400183340025986\\
178.01	0.00400185588645174\\
179.01	0.00400187884746085\\
180.01	0.004001902293354\\
181.01	0.00400192623441251\\
182.01	0.00400195068113719\\
183.01	0.00400197564425293\\
184.01	0.00400200113471347\\
185.01	0.00400202716370644\\
186.01	0.00400205374265833\\
187.01	0.00400208088323978\\
188.01	0.00400210859737063\\
189.01	0.00400213689722557\\
190.01	0.00400216579523964\\
191.01	0.00400219530411368\\
192.01	0.00400222543682022\\
193.01	0.00400225620660934\\
194.01	0.00400228762701485\\
195.01	0.00400231971186019\\
196.01	0.00400235247526493\\
197.01	0.00400238593165128\\
198.01	0.00400242009575025\\
199.01	0.00400245498260905\\
200.01	0.00400249060759751\\
201.01	0.00400252698641519\\
202.01	0.0040025641350986\\
203.01	0.00400260207002887\\
204.01	0.00400264080793888\\
205.01	0.00400268036592107\\
206.01	0.00400272076143546\\
207.01	0.00400276201231771\\
208.01	0.0040028041367873\\
209.01	0.00400284715345606\\
210.01	0.00400289108133653\\
211.01	0.0040029359398511\\
212.01	0.00400298174884117\\
213.01	0.00400302852857604\\
214.01	0.00400307629976253\\
215.01	0.00400312508355492\\
216.01	0.00400317490156429\\
217.01	0.00400322577586912\\
218.01	0.00400327772902583\\
219.01	0.00400333078407886\\
220.01	0.0040033849645722\\
221.01	0.00400344029455958\\
222.01	0.00400349679861733\\
223.01	0.0040035545018547\\
224.01	0.00400361342992643\\
225.01	0.00400367360904473\\
226.01	0.00400373506599244\\
227.01	0.00400379782813527\\
228.01	0.00400386192343491\\
229.01	0.00400392738046294\\
230.01	0.0040039942284144\\
231.01	0.00400406249712196\\
232.01	0.00400413221707023\\
233.01	0.00400420341941109\\
234.01	0.00400427613597826\\
235.01	0.00400435039930348\\
236.01	0.00400442624263209\\
237.01	0.00400450369993956\\
238.01	0.00400458280594816\\
239.01	0.00400466359614446\\
240.01	0.00400474610679673\\
241.01	0.00400483037497298\\
242.01	0.00400491643855967\\
243.01	0.00400500433628072\\
244.01	0.00400509410771696\\
245.01	0.00400518579332625\\
246.01	0.00400527943446398\\
247.01	0.00400537507340401\\
248.01	0.00400547275336068\\
249.01	0.00400557251851065\\
250.01	0.00400567441401559\\
251.01	0.00400577848604586\\
252.01	0.0040058847818048\\
253.01	0.00400599334955253\\
254.01	0.00400610423863214\\
255.01	0.00400621749949535\\
256.01	0.00400633318372889\\
257.01	0.00400645134408252\\
258.01	0.00400657203449714\\
259.01	0.0040066953101334\\
260.01	0.00400682122740204\\
261.01	0.00400694984399408\\
262.01	0.00400708121891265\\
263.01	0.00400721541250496\\
264.01	0.0040073524864961\\
265.01	0.00400749250402305\\
266.01	0.00400763552967005\\
267.01	0.00400778162950483\\
268.01	0.00400793087111605\\
269.01	0.0040080833236515\\
270.01	0.00400823905785821\\
271.01	0.00400839814612274\\
272.01	0.00400856066251321\\
273.01	0.00400872668282293\\
274.01	0.00400889628461461\\
275.01	0.00400906954726678\\
276.01	0.00400924655202069\\
277.01	0.00400942738202939\\
278.01	0.00400961212240829\\
279.01	0.00400980086028671\\
280.01	0.00400999368486213\\
281.01	0.00401019068745494\\
282.01	0.00401039196156595\\
283.01	0.00401059760293537\\
284.01	0.00401080770960394\\
285.01	0.00401102238197561\\
286.01	0.00401124172288256\\
287.01	0.00401146583765299\\
288.01	0.00401169483418017\\
289.01	0.00401192882299489\\
290.01	0.00401216791733948\\
291.01	0.00401241223324518\\
292.01	0.00401266188961152\\
293.01	0.00401291700828935\\
294.01	0.00401317771416633\\
295.01	0.00401344413525533\\
296.01	0.00401371640278654\\
297.01	0.00401399465130289\\
298.01	0.0040142790187586\\
299.01	0.00401456964662193\\
300.01	0.00401486667998169\\
301.01	0.00401517026765781\\
302.01	0.00401548056231579\\
303.01	0.00401579772058686\\
304.01	0.00401612190319132\\
305.01	0.00401645327506792\\
306.01	0.00401679200550804\\
307.01	0.00401713826829567\\
308.01	0.00401749224185236\\
309.01	0.00401785410938972\\
310.01	0.00401822405906641\\
311.01	0.0040186022841539\\
312.01	0.00401898898320749\\
313.01	0.00401938436024643\\
314.01	0.00401978862494078\\
315.01	0.00402020199280703\\
316.01	0.0040206246854131\\
317.01	0.00402105693059175\\
318.01	0.00402149896266428\\
319.01	0.00402195102267505\\
320.01	0.00402241335863672\\
321.01	0.00402288622578666\\
322.01	0.00402336988685704\\
323.01	0.00402386461235664\\
324.01	0.00402437068086765\\
325.01	0.00402488837935699\\
326.01	0.00402541800350273\\
327.01	0.00402595985803753\\
328.01	0.00402651425711005\\
329.01	0.00402708152466413\\
330.01	0.00402766199483845\\
331.01	0.0040282560123867\\
332.01	0.00402886393312019\\
333.01	0.00402948612437388\\
334.01	0.00403012296549693\\
335.01	0.00403077484837001\\
336.01	0.00403144217794987\\
337.01	0.00403212537284327\\
338.01	0.00403282486591065\\
339.01	0.00403354110490371\\
340.01	0.00403427455313465\\
341.01	0.00403502569018123\\
342.01	0.00403579501262745\\
343.01	0.00403658303484196\\
344.01	0.00403739028979434\\
345.01	0.00403821732990874\\
346.01	0.00403906472795776\\
347.01	0.00403993307799215\\
348.01	0.0040408229963084\\
349.01	0.00404173512244888\\
350.01	0.00404267012023247\\
351.01	0.00404362867880947\\
352.01	0.00404461151373312\\
353.01	0.00404561936803739\\
354.01	0.00404665301330805\\
355.01	0.00404771325072806\\
356.01	0.00404880091207617\\
357.01	0.00404991686064848\\
358.01	0.00405106199206755\\
359.01	0.00405223723493328\\
360.01	0.00405344355125863\\
361.01	0.00405468193662199\\
362.01	0.00405595341995024\\
363.01	0.00405725906282824\\
364.01	0.00405859995821199\\
365.01	0.00405997722839344\\
366.01	0.00406139202204257\\
367.01	0.00406284551011538\\
368.01	0.00406433888039251\\
369.01	0.00406587333037041\\
370.01	0.00406745005820723\\
371.01	0.00406907025139548\\
372.01	0.00407073507283152\\
373.01	0.00407244564396998\\
374.01	0.00407420302481734\\
375.01	0.00407600819065625\\
376.01	0.00407786200563846\\
377.01	0.00407976519380208\\
378.01	0.00408171830873572\\
379.01	0.00408372170414938\\
380.01	0.00408577550919545\\
381.01	0.00408787961475132\\
382.01	0.00409003368038524\\
383.01	0.00409223717687271\\
384.01	0.00409448948661441\\
385.01	0.00409679010368409\\
386.01	0.00409913926378894\\
387.01	0.00410153796860382\\
388.01	0.00410398729457805\\
389.01	0.00410648834409962\\
390.01	0.00410904224630156\\
391.01	0.00411165015790546\\
392.01	0.00411431326410397\\
393.01	0.00411703277948469\\
394.01	0.00411980994899869\\
395.01	0.00412264604897609\\
396.01	0.00412554238819288\\
397.01	0.00412850030899023\\
398.01	0.0041315211884525\\
399.01	0.00413460643964725\\
400.01	0.00413775751292993\\
401.01	0.00414097589732168\\
402.01	0.00414426312196102\\
403.01	0.00414762075763914\\
404.01	0.00415105041842263\\
405.01	0.00415455376337169\\
406.01	0.00415813249836025\\
407.01	0.00416178837800729\\
408.01	0.00416552320772841\\
409.01	0.0041693388459159\\
410.01	0.00417323720626121\\
411.01	0.00417722026022958\\
412.01	0.00418129003970098\\
413.01	0.00418544863979343\\
414.01	0.00418969822188282\\
415.01	0.00419404101684077\\
416.01	0.00419847932850825\\
417.01	0.00420301553742987\\
418.01	0.00420765210487349\\
419.01	0.00421239157716445\\
420.01	0.00421723659036616\\
421.01	0.00422218987534434\\
422.01	0.00422725426325424\\
423.01	0.00423243269149921\\
424.01	0.00423772821021023\\
425.01	0.00424314398930774\\
426.01	0.0042486833262106\\
427.01	0.00425434965427011\\
428.01	0.00426014655201326\\
429.01	0.00426607775329521\\
430.01	0.00427214715847105\\
431.01	0.00427835884671442\\
432.01	0.00428471708962851\\
433.01	0.00429122636631332\\
434.01	0.00429789138007754\\
435.01	0.00430471707700844\\
436.01	0.00431170866664313\\
437.01	0.00431887164501668\\
438.01	0.00432621182039702\\
439.01	0.00433373534206047\\
440.01	0.00434144873249774\\
441.01	0.00434935892349288\\
442.01	0.00435747329655405\\
443.01	0.00436579972822673\\
444.01	0.00437434664085377\\
445.01	0.00438312305937479\\
446.01	0.00439213867476621\\
447.01	0.00440140391469866\\
448.01	0.00441093002191358\\
449.01	0.00442072914067407\\
450.01	0.00443081441138396\\
451.01	0.00444120007304674\\
452.01	0.00445190157258024\\
453.01	0.00446293567901471\\
454.01	0.00447432059913423\\
455.01	0.00448607608899386\\
456.01	0.00449822355266942\\
457.01	0.00451078611521795\\
458.01	0.00452378865060674\\
459.01	0.00453725773660674\\
460.01	0.00455122149634399\\
461.01	0.00456570926899146\\
462.01	0.00458075102808025\\
463.01	0.00459637643250742\\
464.01	0.00461261334892561\\
465.01	0.00462948561986687\\
466.01	0.00464700976283944\\
467.01	0.00466519016233086\\
468.01	0.004684011890909\\
469.01	0.00470332858609318\\
470.01	0.00472304126243356\\
471.01	0.00474315610270316\\
472.01	0.00476367892864757\\
473.01	0.0047846150933752\\
474.01	0.00480596935236996\\
475.01	0.00482774570927307\\
476.01	0.00484994723197578\\
477.01	0.00487257583379748\\
478.01	0.00489563201320967\\
479.01	0.00491911454605194\\
480.01	0.00494302015020838\\
481.01	0.00496734311061718\\
482.01	0.00499207483835325\\
483.01	0.00501720336700964\\
484.01	0.00504271278708326\\
485.01	0.0050685826294028\\
486.01	0.00509478721452446\\
487.01	0.00512129499896691\\
488.01	0.00514806797106716\\
489.01	0.00517506118207071\\
490.01	0.00520222254669445\\
491.01	0.00522949311615405\\
492.01	0.00525680805092996\\
493.01	0.0052840985061533\\
494.01	0.00531129494555366\\
495.01	0.00533833263621925\\
496.01	0.005365160328844\\
497.01	0.00539175353915022\\
498.01	0.00541822952524772\\
499.01	0.00544489036454156\\
500.01	0.00547172941255036\\
501.01	0.00549871773255214\\
502.01	0.00552582449500173\\
503.01	0.00555301731921735\\
504.01	0.00558026276534436\\
505.01	0.00560752700751449\\
506.01	0.00563477672031228\\
507.01	0.00566198020904772\\
508.01	0.00568910880783696\\
509.01	0.00571613855510147\\
510.01	0.00574305212934537\\
511.01	0.00576984098227466\\
512.01	0.00579650753175162\\
513.01	0.00582306715959326\\
514.01	0.00584954957850267\\
515.01	0.00587599885929561\\
516.01	0.00590247100249808\\
517.01	0.00592902591337369\\
518.01	0.00595568984274648\\
519.01	0.00598245979477857\\
520.01	0.00600933477015154\\
521.01	0.00603631748495947\\
522.01	0.00606341483578303\\
523.01	0.00609063826637363\\
524.01	0.00611800397038684\\
525.01	0.00614553284666085\\
526.01	0.00617325011730924\\
527.01	0.00620118453538543\\
528.01	0.00622936712841125\\
529.01	0.00625782947214204\\
530.01	0.00628660159347257\\
531.01	0.00631570979295472\\
532.01	0.00634517511355617\\
533.01	0.00637501547558893\\
534.01	0.00640524991762986\\
535.01	0.00643589910486549\\
536.01	0.00646698502056707\\
537.01	0.00649853057532576\\
538.01	0.006530559152647\\
539.01	0.00656309412379254\\
540.01	0.00659615838242811\\
541.01	0.0066297739676066\\
542.01	0.0066639618567796\\
543.01	0.00669874200887199\\
544.01	0.00673413370293458\\
545.01	0.00677015607463429\\
546.01	0.00680682837909413\\
547.01	0.00684416988246932\\
548.01	0.00688219971110505\\
549.01	0.00692093672659515\\
550.01	0.00696039943911544\\
551.01	0.00700060596789915\\
552.01	0.00704157405109965\\
553.01	0.00708332109679305\\
554.01	0.00712586425334399\\
555.01	0.0071692204641202\\
556.01	0.00721340647046484\\
557.01	0.00725843876908058\\
558.01	0.00730433356515286\\
559.01	0.0073511067342187\\
560.01	0.00739877379108111\\
561.01	0.00744734986172339\\
562.01	0.00749684965260175\\
563.01	0.00754728741102208\\
564.01	0.00759867687142414\\
565.01	0.00765103118587222\\
566.01	0.00770436284199182\\
567.01	0.00775868357290257\\
568.01	0.007814004259616\\
569.01	0.00787033482401756\\
570.01	0.0079276841103895\\
571.01	0.0079860597537612\\
572.01	0.00804546803401966\\
573.01	0.00810591371553068\\
574.01	0.00816739987270856\\
575.01	0.00822992770213072\\
576.01	0.00829349632141887\\
577.01	0.00835810255497104\\
578.01	0.00842374070703372\\
579.01	0.00849040232326687\\
580.01	0.0085580759427603\\
581.01	0.00862674684338341\\
582.01	0.00869639678438916\\
583.01	0.00876700375140449\\
584.01	0.00883854171047707\\
585.01	0.0089109803798979\\
586.01	0.00898428503117805\\
587.01	0.00905841633390994\\
588.01	0.00913333026343812\\
589.01	0.00920897809551196\\
590.01	0.00928530651866069\\
591.01	0.00936225790326229\\
592.01	0.00943977077658321\\
593.01	0.00951778056592611\\
594.01	0.00959622068800083\\
595.01	0.00967502408241994\\
596.01	0.00975412531166755\\
597.01	0.00983346338244348\\
598.01	0.0099086620184807\\
599.01	0.00997087280416276\\
599.02	0.00997138072163725\\
599.03	0.009971885576258\\
599.04	0.00997238733820211\\
599.05	0.00997288597735272\\
599.06	0.00997338146329604\\
599.07	0.00997387376531845\\
599.08	0.00997436285240349\\
599.09	0.00997484869322892\\
599.1	0.00997533125616361\\
599.11	0.00997581050926455\\
599.12	0.00997628642027372\\
599.13	0.00997675895661497\\
599.14	0.0099772280853909\\
599.15	0.00997769377337961\\
599.16	0.00997815598703157\\
599.17	0.00997861469246631\\
599.18	0.00997906985546915\\
599.19	0.00997952144148792\\
599.2	0.00997996941562957\\
599.21	0.00998041374265684\\
599.22	0.0099808543869848\\
599.23	0.00998129131267744\\
599.24	0.00998172448344418\\
599.25	0.00998215386263636\\
599.26	0.00998257941258353\\
599.27	0.00998300109279825\\
599.28	0.00998341886238981\\
599.29	0.00998383268006024\\
599.3	0.00998424250410032\\
599.31	0.00998464829238543\\
599.32	0.00998505000237151\\
599.33	0.00998544759109087\\
599.34	0.00998584101514799\\
599.35	0.00998623023071532\\
599.36	0.00998661519352895\\
599.37	0.00998699585888436\\
599.38	0.00998737218163197\\
599.39	0.00998774411617281\\
599.4	0.00998811161645403\\
599.41	0.00998847463596443\\
599.42	0.0099888331277299\\
599.43	0.00998918704430885\\
599.44	0.00998953633778757\\
599.45	0.00998988095977558\\
599.46	0.00999022086140088\\
599.47	0.00999055599330522\\
599.48	0.00999088630563925\\
599.49	0.00999121174805768\\
599.5	0.00999153226971435\\
599.51	0.0099918478192573\\
599.52	0.00999215834482374\\
599.53	0.00999246379403503\\
599.54	0.0099927641139915\\
599.55	0.00999305925126738\\
599.56	0.00999334915190553\\
599.57	0.0099936337614122\\
599.58	0.00999391302475173\\
599.59	0.00999418688634118\\
599.6	0.00999445529004489\\
599.61	0.00999471817916904\\
599.62	0.00999497549645613\\
599.63	0.00999522718407935\\
599.64	0.009995473183637\\
599.65	0.00999571343614678\\
599.66	0.00999594788204004\\
599.67	0.00999617646115599\\
599.68	0.0099963991127358\\
599.69	0.00999661577541675\\
599.7	0.00999682638722618\\
599.71	0.00999703088557551\\
599.72	0.00999722920725411\\
599.73	0.00999742128842315\\
599.74	0.00999760706460942\\
599.75	0.00999778647069897\\
599.76	0.00999795944093086\\
599.77	0.0099981259088907\\
599.78	0.0099982858075042\\
599.79	0.00999843906903064\\
599.8	0.00999858562505627\\
599.81	0.00999872540648766\\
599.82	0.00999885834354498\\
599.83	0.00999898436575515\\
599.84	0.00999910340194508\\
599.85	0.00999921538023465\\
599.86	0.00999932022802977\\
599.87	0.00999941787201528\\
599.88	0.00999950823814785\\
599.89	0.00999959125164875\\
599.9	0.00999966683699656\\
599.91	0.00999973491791987\\
599.92	0.0099997954173898\\
599.93	0.00999984825761255\\
599.94	0.00999989336002181\\
599.95	0.00999993064527112\\
599.96	0.00999996003322615\\
599.97	0.00999998144295691\\
599.98	0.00999999479272987\\
599.99	0.01\\
600	0.01\\
};
\addplot [color=blue!50!mycolor7,solid,forget plot]
  table[row sep=crcr]{%
0.01	0.00386858074314488\\
1.01	0.00386858122147336\\
2.01	0.00386858170981677\\
3.01	0.00386858220838564\\
4.01	0.0038685827173946\\
5.01	0.00386858323706303\\
6.01	0.00386858376761477\\
7.01	0.00386858430927869\\
8.01	0.00386858486228814\\
9.01	0.0038685854268815\\
10.01	0.00386858600330241\\
11.01	0.00386858659179941\\
12.01	0.00386858719262637\\
13.01	0.00386858780604245\\
14.01	0.00386858843231234\\
15.01	0.00386858907170616\\
16.01	0.00386858972449997\\
17.01	0.00386859039097541\\
18.01	0.00386859107142044\\
19.01	0.00386859176612868\\
20.01	0.00386859247540022\\
21.01	0.0038685931995414\\
22.01	0.00386859393886488\\
23.01	0.00386859469369029\\
24.01	0.00386859546434348\\
25.01	0.00386859625115768\\
26.01	0.00386859705447302\\
27.01	0.00386859787463661\\
28.01	0.00386859871200307\\
29.01	0.00386859956693467\\
30.01	0.00386860043980092\\
31.01	0.00386860133097955\\
32.01	0.00386860224085585\\
33.01	0.00386860316982367\\
34.01	0.00386860411828495\\
35.01	0.0038686050866502\\
36.01	0.00386860607533868\\
37.01	0.00386860708477837\\
38.01	0.00386860811540639\\
39.01	0.00386860916766897\\
40.01	0.00386861024202195\\
41.01	0.00386861133893077\\
42.01	0.00386861245887059\\
43.01	0.00386861360232681\\
44.01	0.00386861476979494\\
45.01	0.00386861596178115\\
46.01	0.00386861717880201\\
47.01	0.00386861842138536\\
48.01	0.00386861969007007\\
49.01	0.00386862098540641\\
50.01	0.00386862230795627\\
51.01	0.00386862365829355\\
52.01	0.00386862503700421\\
53.01	0.00386862644468644\\
54.01	0.00386862788195155\\
55.01	0.00386862934942353\\
56.01	0.00386863084773952\\
57.01	0.0038686323775501\\
58.01	0.00386863393951985\\
59.01	0.0038686355343274\\
60.01	0.0038686371626656\\
61.01	0.00386863882524223\\
62.01	0.0038686405227797\\
63.01	0.00386864225601603\\
64.01	0.00386864402570479\\
65.01	0.00386864583261561\\
66.01	0.00386864767753425\\
67.01	0.00386864956126334\\
68.01	0.00386865148462237\\
69.01	0.00386865344844835\\
70.01	0.00386865545359587\\
71.01	0.00386865750093775\\
72.01	0.00386865959136522\\
73.01	0.00386866172578866\\
74.01	0.00386866390513758\\
75.01	0.00386866613036128\\
76.01	0.00386866840242901\\
77.01	0.00386867072233096\\
78.01	0.00386867309107809\\
79.01	0.0038686755097028\\
80.01	0.00386867797925949\\
81.01	0.00386868050082491\\
82.01	0.00386868307549844\\
83.01	0.00386868570440344\\
84.01	0.00386868838868656\\
85.01	0.0038686911295191\\
86.01	0.00386869392809693\\
87.01	0.00386869678564163\\
88.01	0.00386869970340065\\
89.01	0.00386870268264786\\
90.01	0.00386870572468433\\
91.01	0.00386870883083892\\
92.01	0.00386871200246827\\
93.01	0.00386871524095812\\
94.01	0.00386871854772387\\
95.01	0.00386872192421076\\
96.01	0.00386872537189478\\
97.01	0.00386872889228353\\
98.01	0.00386873248691645\\
99.01	0.00386873615736574\\
100.01	0.00386873990523717\\
101.01	0.00386874373217061\\
102.01	0.00386874763984076\\
103.01	0.00386875162995819\\
104.01	0.00386875570426959\\
105.01	0.00386875986455903\\
106.01	0.00386876411264855\\
107.01	0.00386876845039894\\
108.01	0.0038687728797106\\
109.01	0.00386877740252452\\
110.01	0.00386878202082291\\
111.01	0.00386878673663024\\
112.01	0.00386879155201419\\
113.01	0.00386879646908631\\
114.01	0.00386880149000345\\
115.01	0.00386880661696813\\
116.01	0.00386881185223024\\
117.01	0.00386881719808707\\
118.01	0.00386882265688533\\
119.01	0.00386882823102166\\
120.01	0.00386883392294382\\
121.01	0.00386883973515184\\
122.01	0.00386884567019892\\
123.01	0.00386885173069293\\
124.01	0.00386885791929734\\
125.01	0.00386886423873228\\
126.01	0.00386887069177644\\
127.01	0.00386887728126729\\
128.01	0.00386888401010324\\
129.01	0.00386889088124455\\
130.01	0.00386889789771479\\
131.01	0.00386890506260176\\
132.01	0.0038689123790597\\
133.01	0.00386891985031\\
134.01	0.00386892747964292\\
135.01	0.00386893527041925\\
136.01	0.00386894322607161\\
137.01	0.00386895135010566\\
138.01	0.00386895964610259\\
139.01	0.00386896811771983\\
140.01	0.00386897676869327\\
141.01	0.0038689856028387\\
142.01	0.0038689946240536\\
143.01	0.00386900383631873\\
144.01	0.00386901324370049\\
145.01	0.00386902285035202\\
146.01	0.00386903266051551\\
147.01	0.00386904267852417\\
148.01	0.00386905290880393\\
149.01	0.00386906335587582\\
150.01	0.0038690740243575\\
151.01	0.00386908491896602\\
152.01	0.00386909604451913\\
153.01	0.00386910740593839\\
154.01	0.00386911900825068\\
155.01	0.003869130856591\\
156.01	0.00386914295620451\\
157.01	0.00386915531244907\\
158.01	0.00386916793079733\\
159.01	0.00386918081684007\\
160.01	0.00386919397628767\\
161.01	0.00386920741497371\\
162.01	0.00386922113885681\\
163.01	0.00386923515402379\\
164.01	0.00386924946669284\\
165.01	0.00386926408321553\\
166.01	0.00386927901008027\\
167.01	0.00386929425391519\\
168.01	0.00386930982149108\\
169.01	0.0038693257197249\\
170.01	0.00386934195568241\\
171.01	0.00386935853658167\\
172.01	0.0038693754697964\\
173.01	0.00386939276285952\\
174.01	0.00386941042346642\\
175.01	0.00386942845947846\\
176.01	0.00386944687892686\\
177.01	0.00386946569001652\\
178.01	0.00386948490112914\\
179.01	0.00386950452082786\\
180.01	0.00386952455786106\\
181.01	0.00386954502116617\\
182.01	0.00386956591987418\\
183.01	0.00386958726331355\\
184.01	0.00386960906101484\\
185.01	0.00386963132271513\\
186.01	0.00386965405836236\\
187.01	0.00386967727812036\\
188.01	0.0038697009923734\\
189.01	0.0038697252117309\\
190.01	0.00386974994703293\\
191.01	0.00386977520935472\\
192.01	0.00386980101001223\\
193.01	0.00386982736056742\\
194.01	0.00386985427283396\\
195.01	0.00386988175888232\\
196.01	0.00386990983104603\\
197.01	0.00386993850192668\\
198.01	0.00386996778440132\\
199.01	0.00386999769162739\\
200.01	0.00387002823704935\\
201.01	0.00387005943440528\\
202.01	0.00387009129773321\\
203.01	0.003870123841378\\
204.01	0.00387015707999833\\
205.01	0.00387019102857358\\
206.01	0.00387022570241097\\
207.01	0.00387026111715313\\
208.01	0.00387029728878577\\
209.01	0.00387033423364494\\
210.01	0.00387037196842557\\
211.01	0.00387041051018915\\
212.01	0.00387044987637228\\
213.01	0.00387049008479498\\
214.01	0.0038705311536697\\
215.01	0.00387057310160979\\
216.01	0.00387061594763929\\
217.01	0.00387065971120169\\
218.01	0.00387070441216977\\
219.01	0.00387075007085552\\
220.01	0.00387079670802004\\
221.01	0.00387084434488412\\
222.01	0.0038708930031381\\
223.01	0.00387094270495389\\
224.01	0.00387099347299485\\
225.01	0.00387104533042792\\
226.01	0.0038710983009351\\
227.01	0.00387115240872488\\
228.01	0.00387120767854547\\
229.01	0.00387126413569633\\
230.01	0.00387132180604194\\
231.01	0.003871380716024\\
232.01	0.0038714408926759\\
233.01	0.00387150236363576\\
234.01	0.00387156515716117\\
235.01	0.00387162930214342\\
236.01	0.0038716948281229\\
237.01	0.00387176176530395\\
238.01	0.00387183014457057\\
239.01	0.00387189999750268\\
240.01	0.00387197135639282\\
241.01	0.00387204425426282\\
242.01	0.00387211872488141\\
243.01	0.00387219480278177\\
244.01	0.00387227252328017\\
245.01	0.00387235192249445\\
246.01	0.00387243303736348\\
247.01	0.0038725159056671\\
248.01	0.00387260056604614\\
249.01	0.00387268705802328\\
250.01	0.00387277542202481\\
251.01	0.00387286569940221\\
252.01	0.00387295793245487\\
253.01	0.00387305216445341\\
254.01	0.00387314843966292\\
255.01	0.00387324680336815\\
256.01	0.0038733473018983\\
257.01	0.00387344998265256\\
258.01	0.00387355489412662\\
259.01	0.00387366208594028\\
260.01	0.00387377160886529\\
261.01	0.00387388351485358\\
262.01	0.00387399785706723\\
263.01	0.00387411468990886\\
264.01	0.00387423406905245\\
265.01	0.00387435605147568\\
266.01	0.00387448069549264\\
267.01	0.00387460806078781\\
268.01	0.00387473820845088\\
269.01	0.00387487120101261\\
270.01	0.00387500710248137\\
271.01	0.00387514597838114\\
272.01	0.00387528789579067\\
273.01	0.00387543292338315\\
274.01	0.00387558113146752\\
275.01	0.00387573259203067\\
276.01	0.00387588737878133\\
277.01	0.0038760455671947\\
278.01	0.00387620723455863\\
279.01	0.00387637246002108\\
280.01	0.0038765413246388\\
281.01	0.0038767139114279\\
282.01	0.00387689030541528\\
283.01	0.00387707059369199\\
284.01	0.00387725486546811\\
285.01	0.00387744321212886\\
286.01	0.0038776357272933\\
287.01	0.00387783250687326\\
288.01	0.00387803364913558\\
289.01	0.00387823925476498\\
290.01	0.00387844942692965\\
291.01	0.00387866427134811\\
292.01	0.0038788838963589\\
293.01	0.00387910841299116\\
294.01	0.00387933793503815\\
295.01	0.00387957257913316\\
296.01	0.00387981246482685\\
297.01	0.00388005771466718\\
298.01	0.00388030845428254\\
299.01	0.00388056481246622\\
300.01	0.00388082692126358\\
301.01	0.00388109491606271\\
302.01	0.00388136893568684\\
303.01	0.00388164912248991\\
304.01	0.00388193562245431\\
305.01	0.00388222858529291\\
306.01	0.00388252816455287\\
307.01	0.00388283451772257\\
308.01	0.00388314780634237\\
309.01	0.00388346819611796\\
310.01	0.00388379585703746\\
311.01	0.00388413096349123\\
312.01	0.00388447369439597\\
313.01	0.00388482423332162\\
314.01	0.003885182768622\\
315.01	0.00388554949356952\\
316.01	0.0038859246064932\\
317.01	0.00388630831092017\\
318.01	0.00388670081572171\\
319.01	0.00388710233526245\\
320.01	0.00388751308955389\\
321.01	0.00388793330441166\\
322.01	0.00388836321161637\\
323.01	0.00388880304907896\\
324.01	0.0038892530610095\\
325.01	0.00388971349808938\\
326.01	0.0038901846176479\\
327.01	0.00389066668384264\\
328.01	0.00389115996784191\\
329.01	0.00389166474801204\\
330.01	0.00389218131010673\\
331.01	0.00389270994745958\\
332.01	0.00389325096117865\\
333.01	0.00389380466034321\\
334.01	0.00389437136220222\\
335.01	0.00389495139237302\\
336.01	0.00389554508504091\\
337.01	0.00389615278315711\\
338.01	0.00389677483863729\\
339.01	0.00389741161255426\\
340.01	0.00389806347532944\\
341.01	0.00389873080691741\\
342.01	0.00389941399698415\\
343.01	0.0039001134450754\\
344.01	0.00390082956077403\\
345.01	0.00390156276384374\\
346.01	0.00390231348435555\\
347.01	0.00390308216279489\\
348.01	0.00390386925014351\\
349.01	0.00390467520793431\\
350.01	0.00390550050827339\\
351.01	0.00390634563382347\\
352.01	0.00390721107774351\\
353.01	0.0039080973435775\\
354.01	0.00390900494508627\\
355.01	0.00390993440601353\\
356.01	0.00391088625977783\\
357.01	0.00391186104908239\\
358.01	0.00391285932543314\\
359.01	0.00391388164855558\\
360.01	0.00391492858570175\\
361.01	0.00391600071083822\\
362.01	0.00391709860370817\\
363.01	0.00391822284876306\\
364.01	0.0039193740339603\\
365.01	0.00392055274943226\\
366.01	0.00392175958603289\\
367.01	0.00392299513378343\\
368.01	0.00392425998024432\\
369.01	0.00392555470886186\\
370.01	0.0039268798973519\\
371.01	0.00392823611621257\\
372.01	0.00392962392748478\\
373.01	0.00393104388391531\\
374.01	0.00393249652871842\\
375.01	0.00393398239617307\\
376.01	0.00393550201333963\\
377.01	0.00393705590321207\\
378.01	0.0039386445896398\\
379.01	0.0039402686043306\\
380.01	0.00394192849615328\\
381.01	0.00394362484274391\\
382.01	0.0039453582640135\\
383.01	0.00394712943642735\\
384.01	0.00394893910572075\\
385.01	0.00395078809372049\\
386.01	0.00395267728833708\\
387.01	0.00395460761274427\\
388.01	0.00395658001442778\\
389.01	0.00395859546571165\\
390.01	0.00396065496457514\\
391.01	0.0039627595355054\\
392.01	0.0039649102303875\\
393.01	0.00396710812943429\\
394.01	0.00396935434215751\\
395.01	0.00397165000838327\\
396.01	0.00397399629931416\\
397.01	0.0039763944186407\\
398.01	0.00397884560370544\\
399.01	0.00398135112672141\\
400.01	0.00398391229605037\\
401.01	0.00398653045754237\\
402.01	0.00398920699594149\\
403.01	0.00399194333636183\\
404.01	0.00399474094583709\\
405.01	0.00399760133494935\\
406.01	0.00400052605954149\\
407.01	0.00400351672251856\\
408.01	0.00400657497574333\\
409.01	0.0040097025220328\\
410.01	0.0040129011172612\\
411.01	0.00401617257257576\\
412.01	0.00401951875673466\\
413.01	0.00402294159857138\\
414.01	0.00402644308959658\\
415.01	0.00403002528674379\\
416.01	0.00403369031526886\\
417.01	0.00403744037181229\\
418.01	0.00404127772763399\\
419.01	0.00404520473203011\\
420.01	0.00404922381594372\\
421.01	0.00405333749577814\\
422.01	0.00405754837742431\\
423.01	0.00406185916051362\\
424.01	0.00406627264290478\\
425.01	0.00407079172541597\\
426.01	0.00407541941681033\\
427.01	0.0040801588390427\\
428.01	0.00408501323277298\\
429.01	0.00408998596314955\\
430.01	0.00409508052586133\\
431.01	0.00410030055345542\\
432.01	0.00410564982190721\\
433.01	0.00411113225742545\\
434.01	0.00411675194346493\\
435.01	0.00412251312790421\\
436.01	0.00412842023033567\\
437.01	0.00413447784938988\\
438.01	0.00414069076999928\\
439.01	0.00414706397047156\\
440.01	0.00415360262921131\\
441.01	0.00416031213088211\\
442.01	0.00416719807175235\\
443.01	0.00417426626390027\\
444.01	0.00418152273788162\\
445.01	0.00418897374337146\\
446.01	0.00419662574718474\\
447.01	0.00420448542795937\\
448.01	0.00421255966664272\\
449.01	0.00422085553176207\\
450.01	0.00422938025828778\\
451.01	0.0042381412187074\\
452.01	0.00424714588475078\\
453.01	0.00425640177802644\\
454.01	0.00426591640771269\\
455.01	0.0042756971934038\\
456.01	0.00428575137133055\\
457.01	0.00429608588254402\\
458.01	0.004306707242431\\
459.01	0.00431762139232646\\
460.01	0.00432883353633232\\
461.01	0.00434034797020055\\
462.01	0.00435216791495188\\
463.01	0.00436429537673048\\
464.01	0.00437673106759208\\
465.01	0.00438947444140124\\
466.01	0.00440252392748826\\
467.01	0.00441587748602856\\
468.01	0.00442953366884713\\
469.01	0.00444349452215706\\
470.01	0.00445776866461103\\
471.01	0.00447236656830442\\
472.01	0.00448729958656169\\
473.01	0.0045025800846319\\
474.01	0.00451822158966254\\
475.01	0.0045342389610175\\
476.01	0.00455064858089547\\
477.01	0.00456746856324833\\
478.01	0.00458471897573415\\
479.01	0.00460242206412984\\
480.01	0.00462060245982847\\
481.01	0.00463928734521334\\
482.01	0.00465850675671346\\
483.01	0.00467829418219433\\
484.01	0.00469868696563419\\
485.01	0.00471972658808292\\
486.01	0.00474145882382038\\
487.01	0.00476393367448086\\
488.01	0.00478720493526354\\
489.01	0.00481132917779667\\
490.01	0.00483636383485271\\
491.01	0.00486236422364897\\
492.01	0.0048893823801086\\
493.01	0.00491746667181199\\
494.01	0.00494665806710051\\
495.01	0.00497698424535896\\
496.01	0.00500845104106382\\
497.01	0.00504103014834045\\
498.01	0.00507454796308061\\
499.01	0.00510865853264725\\
500.01	0.00514332840972172\\
501.01	0.00517853973412254\\
502.01	0.00521426915877894\\
503.01	0.00525048700255988\\
504.01	0.00528715633502748\\
505.01	0.00532423201429295\\
506.01	0.00536165971570869\\
507.01	0.00539937501559937\\
508.01	0.00543730263400269\\
509.01	0.00547535599460236\\
510.01	0.00551343733817398\\
511.01	0.00555143873823\\
512.01	0.00558924452729373\\
513.01	0.00562673586838863\\
514.01	0.00566379852542793\\
515.01	0.00570033533525394\\
516.01	0.00573628551453981\\
517.01	0.00577173599208902\\
518.01	0.00580717753185216\\
519.01	0.005842642439381\\
520.01	0.00587808362130181\\
521.01	0.00591345447457746\\
522.01	0.00594871041434022\\
523.01	0.0059838108718764\\
524.01	0.00601872188111489\\
525.01	0.00605341932537118\\
526.01	0.00608789247573951\\
527.01	0.0061221474258837\\
528.01	0.00615621015398192\\
529.01	0.00619012854416304\\
530.01	0.00622397222772657\\
531.01	0.00625782842397304\\
532.01	0.00629178452997938\\
533.01	0.00632587055864795\\
534.01	0.00636009328503807\\
535.01	0.00639446549593649\\
536.01	0.00642900656512509\\
537.01	0.00646374254740046\\
538.01	0.00649870585890901\\
539.01	0.00653393442389336\\
540.01	0.00656947018386905\\
541.01	0.00660535691740503\\
542.01	0.00664163743538566\\
543.01	0.00667835044132781\\
544.01	0.00671552776347704\\
545.01	0.00675319498961828\\
546.01	0.00679137757058428\\
547.01	0.00683010261535099\\
548.01	0.00686939836821782\\
549.01	0.00690929355948017\\
550.01	0.00694981670841965\\
551.01	0.0069909954519336\\
552.01	0.00703285600150236\\
553.01	0.0070754228525436\\
554.01	0.00711871886371888\\
555.01	0.0071627657603584\\
556.01	0.00720758481787483\\
557.01	0.00725319703853558\\
558.01	0.00729962296373621\\
559.01	0.00734688249195777\\
560.01	0.0073949947477996\\
561.01	0.00744397801310743\\
562.01	0.00749384972275357\\
563.01	0.00754462651135841\\
564.01	0.00759632427494854\\
565.01	0.00764895819286281\\
566.01	0.00770254266877708\\
567.01	0.00775709122588741\\
568.01	0.00781261640385305\\
569.01	0.00786912966267725\\
570.01	0.00792664128800444\\
571.01	0.00798516028951031\\
572.01	0.00804469428287664\\
573.01	0.0081052493475248\\
574.01	0.0081668298578033\\
575.01	0.00822943829317042\\
576.01	0.00829307503476616\\
577.01	0.00835773814990506\\
578.01	0.00842342316335163\\
579.01	0.00849012281477226\\
580.01	0.00855782680318764\\
581.01	0.00862652152130265\\
582.01	0.0086961897849244\\
583.01	0.0087668105646243\\
584.01	0.00883835872788432\\
585.01	0.00891080480112841\\
586.01	0.00898411476341371\\
587.01	0.0090582498869396\\
588.01	0.00913316664376106\\
589.01	0.00920881670326823\\
590.01	0.00928514705127742\\
591.01	0.00936210026927741\\
592.01	0.00943961502205398\\
593.01	0.00951762681432733\\
594.01	0.00959606909286117\\
595.01	0.0096748747904041\\
596.01	0.00975397843263973\\
597.01	0.00983331896004518\\
598.01	0.0099086620184803\\
599.01	0.00997087280416276\\
599.02	0.00997138072163725\\
599.03	0.00997188557625799\\
599.04	0.00997238733820211\\
599.05	0.00997288597735272\\
599.06	0.00997338146329604\\
599.07	0.00997387376531845\\
599.08	0.00997436285240349\\
599.09	0.00997484869322892\\
599.1	0.00997533125616361\\
599.11	0.00997581050926455\\
599.12	0.00997628642027372\\
599.13	0.00997675895661498\\
599.14	0.0099772280853909\\
599.15	0.00997769377337961\\
599.16	0.00997815598703157\\
599.17	0.00997861469246631\\
599.18	0.00997906985546915\\
599.19	0.00997952144148792\\
599.2	0.00997996941562957\\
599.21	0.00998041374265684\\
599.22	0.0099808543869848\\
599.23	0.00998129131267744\\
599.24	0.00998172448344418\\
599.25	0.00998215386263636\\
599.26	0.00998257941258354\\
599.27	0.00998300109279825\\
599.28	0.00998341886238981\\
599.29	0.00998383268006024\\
599.3	0.00998424250410032\\
599.31	0.00998464829238543\\
599.32	0.00998505000237151\\
599.33	0.00998544759109087\\
599.34	0.00998584101514799\\
599.35	0.00998623023071532\\
599.36	0.00998661519352895\\
599.37	0.00998699585888436\\
599.38	0.00998737218163197\\
599.39	0.00998774411617281\\
599.4	0.00998811161645403\\
599.41	0.00998847463596444\\
599.42	0.0099888331277299\\
599.43	0.00998918704430885\\
599.44	0.00998953633778757\\
599.45	0.00998988095977558\\
599.46	0.00999022086140088\\
599.47	0.00999055599330522\\
599.48	0.00999088630563925\\
599.49	0.00999121174805768\\
599.5	0.00999153226971435\\
599.51	0.0099918478192573\\
599.52	0.00999215834482374\\
599.53	0.00999246379403503\\
599.54	0.0099927641139915\\
599.55	0.00999305925126738\\
599.56	0.00999334915190553\\
599.57	0.0099936337614122\\
599.58	0.00999391302475173\\
599.59	0.00999418688634118\\
599.6	0.00999445529004489\\
599.61	0.00999471817916904\\
599.62	0.00999497549645613\\
599.63	0.00999522718407934\\
599.64	0.009995473183637\\
599.65	0.00999571343614678\\
599.66	0.00999594788204004\\
599.67	0.00999617646115598\\
599.68	0.0099963991127358\\
599.69	0.00999661577541675\\
599.7	0.00999682638722618\\
599.71	0.00999703088557551\\
599.72	0.00999722920725411\\
599.73	0.00999742128842315\\
599.74	0.00999760706460942\\
599.75	0.00999778647069897\\
599.76	0.00999795944093086\\
599.77	0.0099981259088907\\
599.78	0.0099982858075042\\
599.79	0.00999843906903064\\
599.8	0.00999858562505627\\
599.81	0.00999872540648767\\
599.82	0.00999885834354498\\
599.83	0.00999898436575515\\
599.84	0.00999910340194508\\
599.85	0.00999921538023465\\
599.86	0.00999932022802977\\
599.87	0.00999941787201528\\
599.88	0.00999950823814785\\
599.89	0.00999959125164875\\
599.9	0.00999966683699656\\
599.91	0.00999973491791987\\
599.92	0.0099997954173898\\
599.93	0.00999984825761255\\
599.94	0.00999989336002181\\
599.95	0.00999993064527112\\
599.96	0.00999996003322615\\
599.97	0.00999998144295691\\
599.98	0.00999999479272987\\
599.99	0.01\\
600	0.01\\
};
\addplot [color=blue!40!mycolor9,solid,forget plot]
  table[row sep=crcr]{%
0.01	0.00356935369650614\\
1.01	0.0035693541789565\\
2.01	0.00356935467156345\\
3.01	0.00356935517454188\\
4.01	0.003569355688111\\
5.01	0.00356935621249488\\
6.01	0.00356935674792234\\
7.01	0.00356935729462701\\
8.01	0.00356935785284748\\
9.01	0.00356935842282738\\
10.01	0.00356935900481559\\
11.01	0.00356935959906622\\
12.01	0.00356936020583877\\
13.01	0.00356936082539841\\
14.01	0.00356936145801577\\
15.01	0.00356936210396751\\
16.01	0.00356936276353592\\
17.01	0.00356936343700942\\
18.01	0.00356936412468239\\
19.01	0.00356936482685569\\
20.01	0.00356936554383659\\
21.01	0.00356936627593862\\
22.01	0.00356936702348254\\
23.01	0.00356936778679537\\
24.01	0.00356936856621158\\
25.01	0.00356936936207233\\
26.01	0.00356937017472626\\
27.01	0.00356937100452949\\
28.01	0.00356937185184578\\
29.01	0.00356937271704624\\
30.01	0.00356937360051053\\
31.01	0.00356937450262593\\
32.01	0.0035693754237882\\
33.01	0.00356937636440138\\
34.01	0.0035693773248783\\
35.01	0.00356937830564054\\
36.01	0.00356937930711877\\
37.01	0.00356938032975258\\
38.01	0.00356938137399115\\
39.01	0.00356938244029334\\
40.01	0.00356938352912765\\
41.01	0.00356938464097266\\
42.01	0.00356938577631713\\
43.01	0.00356938693566027\\
44.01	0.00356938811951205\\
45.01	0.00356938932839322\\
46.01	0.00356939056283574\\
47.01	0.00356939182338285\\
48.01	0.00356939311058957\\
49.01	0.0035693944250226\\
50.01	0.00356939576726092\\
51.01	0.00356939713789588\\
52.01	0.0035693985375314\\
53.01	0.0035693999667844\\
54.01	0.00356940142628485\\
55.01	0.00356940291667644\\
56.01	0.00356940443861649\\
57.01	0.00356940599277643\\
58.01	0.00356940757984218\\
59.01	0.00356940920051417\\
60.01	0.00356941085550791\\
61.01	0.00356941254555425\\
62.01	0.00356941427139984\\
63.01	0.00356941603380698\\
64.01	0.00356941783355474\\
65.01	0.00356941967143858\\
66.01	0.00356942154827111\\
67.01	0.00356942346488234\\
68.01	0.00356942542212033\\
69.01	0.00356942742085103\\
70.01	0.00356942946195912\\
71.01	0.0035694315463482\\
72.01	0.0035694336749415\\
73.01	0.00356943584868165\\
74.01	0.00356943806853179\\
75.01	0.00356944033547574\\
76.01	0.00356944265051855\\
77.01	0.00356944501468662\\
78.01	0.00356944742902841\\
79.01	0.00356944989461534\\
80.01	0.00356945241254138\\
81.01	0.00356945498392423\\
82.01	0.00356945760990569\\
83.01	0.00356946029165188\\
84.01	0.00356946303035427\\
85.01	0.00356946582722982\\
86.01	0.00356946868352175\\
87.01	0.0035694716005001\\
88.01	0.00356947457946204\\
89.01	0.00356947762173299\\
90.01	0.00356948072866653\\
91.01	0.0035694839016457\\
92.01	0.00356948714208323\\
93.01	0.00356949045142237\\
94.01	0.00356949383113733\\
95.01	0.0035694972827342\\
96.01	0.00356950080775159\\
97.01	0.00356950440776119\\
98.01	0.00356950808436861\\
99.01	0.00356951183921431\\
100.01	0.00356951567397391\\
101.01	0.00356951959035923\\
102.01	0.00356952359011911\\
103.01	0.00356952767504026\\
104.01	0.00356953184694772\\
105.01	0.00356953610770608\\
106.01	0.00356954045922046\\
107.01	0.00356954490343674\\
108.01	0.00356954944234325\\
109.01	0.0035695540779709\\
110.01	0.00356955881239492\\
111.01	0.00356956364773511\\
112.01	0.00356956858615737\\
113.01	0.00356957362987451\\
114.01	0.00356957878114701\\
115.01	0.00356958404228451\\
116.01	0.00356958941564639\\
117.01	0.00356959490364335\\
118.01	0.00356960050873852\\
119.01	0.00356960623344788\\
120.01	0.00356961208034247\\
121.01	0.00356961805204864\\
122.01	0.00356962415125012\\
123.01	0.00356963038068857\\
124.01	0.00356963674316516\\
125.01	0.00356964324154208\\
126.01	0.00356964987874328\\
127.01	0.00356965665775656\\
128.01	0.00356966358163448\\
129.01	0.00356967065349578\\
130.01	0.00356967787652725\\
131.01	0.00356968525398468\\
132.01	0.00356969278919485\\
133.01	0.0035697004855569\\
134.01	0.00356970834654367\\
135.01	0.00356971637570373\\
136.01	0.00356972457666296\\
137.01	0.00356973295312622\\
138.01	0.0035697415088789\\
139.01	0.00356975024778874\\
140.01	0.00356975917380811\\
141.01	0.00356976829097526\\
142.01	0.00356977760341658\\
143.01	0.00356978711534861\\
144.01	0.00356979683107955\\
145.01	0.00356980675501176\\
146.01	0.00356981689164382\\
147.01	0.00356982724557248\\
148.01	0.00356983782149475\\
149.01	0.00356984862421054\\
150.01	0.00356985965862446\\
151.01	0.00356987092974825\\
152.01	0.00356988244270362\\
153.01	0.00356989420272411\\
154.01	0.00356990621515779\\
155.01	0.00356991848546973\\
156.01	0.00356993101924495\\
157.01	0.00356994382219063\\
158.01	0.00356995690013892\\
159.01	0.00356997025904997\\
160.01	0.00356998390501463\\
161.01	0.00356999784425741\\
162.01	0.00357001208313922\\
163.01	0.00357002662816094\\
164.01	0.00357004148596586\\
165.01	0.00357005666334344\\
166.01	0.00357007216723233\\
167.01	0.00357008800472368\\
168.01	0.00357010418306469\\
169.01	0.00357012070966183\\
170.01	0.00357013759208466\\
171.01	0.00357015483806954\\
172.01	0.00357017245552338\\
173.01	0.003570190452527\\
174.01	0.00357020883733943\\
175.01	0.00357022761840199\\
176.01	0.00357024680434229\\
177.01	0.00357026640397823\\
178.01	0.00357028642632264\\
179.01	0.0035703068805874\\
180.01	0.00357032777618782\\
181.01	0.00357034912274765\\
182.01	0.00357037093010336\\
183.01	0.00357039320830949\\
184.01	0.00357041596764265\\
185.01	0.00357043921860747\\
186.01	0.00357046297194154\\
187.01	0.00357048723862011\\
188.01	0.00357051202986224\\
189.01	0.00357053735713599\\
190.01	0.00357056323216381\\
191.01	0.00357058966692904\\
192.01	0.00357061667368105\\
193.01	0.00357064426494189\\
194.01	0.00357067245351184\\
195.01	0.00357070125247674\\
196.01	0.00357073067521334\\
197.01	0.00357076073539701\\
198.01	0.0035707914470078\\
199.01	0.00357082282433764\\
200.01	0.00357085488199765\\
201.01	0.00357088763492534\\
202.01	0.0035709210983919\\
203.01	0.00357095528801027\\
204.01	0.00357099021974247\\
205.01	0.00357102590990781\\
206.01	0.00357106237519164\\
207.01	0.00357109963265299\\
208.01	0.0035711376997334\\
209.01	0.00357117659426605\\
210.01	0.0035712163344844\\
211.01	0.00357125693903206\\
212.01	0.00357129842697129\\
213.01	0.00357134081779364\\
214.01	0.00357138413142935\\
215.01	0.00357142838825765\\
216.01	0.0035714736091173\\
217.01	0.00357151981531727\\
218.01	0.0035715670286473\\
219.01	0.00357161527138948\\
220.01	0.00357166456632982\\
221.01	0.00357171493676962\\
222.01	0.00357176640653782\\
223.01	0.00357181900000292\\
224.01	0.00357187274208633\\
225.01	0.00357192765827472\\
226.01	0.00357198377463296\\
227.01	0.00357204111781907\\
228.01	0.00357209971509662\\
229.01	0.00357215959434987\\
230.01	0.00357222078409804\\
231.01	0.00357228331351042\\
232.01	0.00357234721242168\\
233.01	0.00357241251134764\\
234.01	0.00357247924150125\\
235.01	0.0035725474348093\\
236.01	0.00357261712392903\\
237.01	0.00357268834226578\\
238.01	0.00357276112399071\\
239.01	0.00357283550405872\\
240.01	0.00357291151822756\\
241.01	0.00357298920307673\\
242.01	0.00357306859602675\\
243.01	0.00357314973536013\\
244.01	0.00357323266024112\\
245.01	0.00357331741073738\\
246.01	0.0035734040278412\\
247.01	0.00357349255349197\\
248.01	0.00357358303059889\\
249.01	0.00357367550306435\\
250.01	0.00357377001580761\\
251.01	0.00357386661478948\\
252.01	0.00357396534703736\\
253.01	0.00357406626067107\\
254.01	0.00357416940492954\\
255.01	0.00357427483019705\\
256.01	0.0035743825880318\\
257.01	0.00357449273119416\\
258.01	0.00357460531367555\\
259.01	0.00357472039072872\\
260.01	0.00357483801889809\\
261.01	0.00357495825605129\\
262.01	0.00357508116141174\\
263.01	0.00357520679559105\\
264.01	0.0035753352206235\\
265.01	0.0035754665000003\\
266.01	0.00357560069870552\\
267.01	0.00357573788325241\\
268.01	0.00357587812172097\\
269.01	0.00357602148379602\\
270.01	0.00357616804080665\\
271.01	0.0035763178657669\\
272.01	0.00357647103341641\\
273.01	0.00357662762026308\\
274.01	0.00357678770462662\\
275.01	0.0035769513666829\\
276.01	0.00357711868850955\\
277.01	0.00357728975413245\\
278.01	0.0035774646495737\\
279.01	0.00357764346290101\\
280.01	0.0035778262842775\\
281.01	0.00357801320601334\\
282.01	0.0035782043226186\\
283.01	0.00357839973085708\\
284.01	0.00357859952980157\\
285.01	0.00357880382089096\\
286.01	0.00357901270798767\\
287.01	0.00357922629743751\\
288.01	0.00357944469813003\\
289.01	0.00357966802156129\\
290.01	0.003579896381897\\
291.01	0.00358012989603849\\
292.01	0.00358036868368844\\
293.01	0.00358061286742002\\
294.01	0.00358086257274595\\
295.01	0.0035811179281903\\
296.01	0.00358137906536138\\
297.01	0.00358164611902638\\
298.01	0.00358191922718731\\
299.01	0.00358219853115907\\
300.01	0.00358248417564937\\
301.01	0.00358277630883916\\
302.01	0.00358307508246659\\
303.01	0.0035833806519107\\
304.01	0.0035836931762786\\
305.01	0.00358401281849324\\
306.01	0.00358433974538321\\
307.01	0.00358467412777473\\
308.01	0.00358501614058483\\
309.01	0.00358536596291655\\
310.01	0.00358572377815553\\
311.01	0.00358608977406931\\
312.01	0.00358646414290688\\
313.01	0.0035868470815013\\
314.01	0.00358723879137322\\
315.01	0.00358763947883652\\
316.01	0.00358804935510501\\
317.01	0.00358846863640165\\
318.01	0.00358889754406856\\
319.01	0.00358933630467896\\
320.01	0.00358978515015054\\
321.01	0.00359024431786005\\
322.01	0.00359071405076002\\
323.01	0.00359119459749599\\
324.01	0.00359168621252533\\
325.01	0.00359218915623769\\
326.01	0.00359270369507596\\
327.01	0.00359323010165832\\
328.01	0.00359376865490199\\
329.01	0.00359431964014711\\
330.01	0.00359488334928176\\
331.01	0.0035954600808672\\
332.01	0.00359605014026447\\
333.01	0.00359665383976118\\
334.01	0.00359727149869771\\
335.01	0.00359790344359505\\
336.01	0.00359855000828187\\
337.01	0.00359921153402205\\
338.01	0.00359988836964127\\
339.01	0.00360058087165412\\
340.01	0.00360128940439082\\
341.01	0.00360201434012253\\
342.01	0.00360275605918707\\
343.01	0.00360351495011342\\
344.01	0.00360429140974572\\
345.01	0.00360508584336625\\
346.01	0.00360589866481749\\
347.01	0.00360673029662416\\
348.01	0.00360758117011349\\
349.01	0.0036084517255366\\
350.01	0.00360934241218824\\
351.01	0.00361025368852749\\
352.01	0.00361118602229971\\
353.01	0.0036121398906598\\
354.01	0.0036131157802978\\
355.01	0.0036141141875686\\
356.01	0.00361513561862792\\
357.01	0.00361618058957471\\
358.01	0.00361724962660467\\
359.01	0.00361834326617612\\
360.01	0.00361946205519366\\
361.01	0.0036206065512127\\
362.01	0.00362177732267134\\
363.01	0.00362297494915492\\
364.01	0.00362420002170184\\
365.01	0.00362545314315821\\
366.01	0.00362673492859153\\
367.01	0.00362804600577428\\
368.01	0.00362938701574894\\
369.01	0.00363075861348704\\
370.01	0.00363216146865467\\
371.01	0.00363359626649709\\
372.01	0.00363506370885205\\
373.01	0.0036365645152997\\
374.01	0.00363809942445012\\
375.01	0.00363966919536403\\
376.01	0.00364127460908717\\
377.01	0.00364291647026863\\
378.01	0.00364459560880954\\
379.01	0.0036463128814699\\
380.01	0.00364806917333388\\
381.01	0.00364986539901338\\
382.01	0.00365170250345209\\
383.01	0.00365358146220757\\
384.01	0.00365550328113723\\
385.01	0.00365746899556441\\
386.01	0.00365947966931873\\
387.01	0.00366153639456435\\
388.01	0.00366364029248849\\
389.01	0.00366579251415143\\
390.01	0.00366799424136459\\
391.01	0.00367024668759425\\
392.01	0.00367255109889375\\
393.01	0.00367490875486343\\
394.01	0.00367732096963948\\
395.01	0.00367978909291161\\
396.01	0.00368231451097226\\
397.01	0.00368489864779506\\
398.01	0.00368754296614511\\
399.01	0.00369024896872178\\
400.01	0.00369301819933206\\
401.01	0.00369585224409821\\
402.01	0.003698752732698\\
403.01	0.003701721339638\\
404.01	0.00370475978556105\\
405.01	0.00370786983858665\\
406.01	0.0037110533156855\\
407.01	0.00371431208408747\\
408.01	0.00371764806272219\\
409.01	0.00372106322369291\\
410.01	0.00372455959378086\\
411.01	0.00372813925598164\\
412.01	0.00373180435106996\\
413.01	0.00373555707919232\\
414.01	0.00373939970148496\\
415.01	0.00374333454171481\\
416.01	0.00374736398794043\\
417.01	0.00375149049418813\\
418.01	0.00375571658214024\\
419.01	0.00376004484283006\\
420.01	0.00376447793833572\\
421.01	0.00376901860346752\\
422.01	0.00377366964744026\\
423.01	0.00377843395551854\\
424.01	0.00378331449062667\\
425.01	0.00378831429490753\\
426.01	0.00379343649121612\\
427.01	0.0037986842845302\\
428.01	0.00380406096325764\\
429.01	0.00380956990041652\\
430.01	0.00381521455466308\\
431.01	0.00382099847113472\\
432.01	0.00382692528207579\\
433.01	0.00383299870720497\\
434.01	0.00383922255378029\\
435.01	0.00384560071631176\\
436.01	0.00385213717586352\\
437.01	0.00385883599888315\\
438.01	0.00386570133548747\\
439.01	0.00387273741712474\\
440.01	0.00387994855352876\\
441.01	0.00388733912886977\\
442.01	0.00389491359700014\\
443.01	0.00390267647568868\\
444.01	0.0039106323397283\\
445.01	0.00391878581280282\\
446.01	0.00392714155799803\\
447.01	0.00393570426684846\\
448.01	0.00394447864682365\\
449.01	0.003953469407182\\
450.01	0.00396268124315192\\
451.01	0.00397211881845402\\
452.01	0.00398178674624092\\
453.01	0.00399168956863305\\
454.01	0.0040018317351434\\
455.01	0.00401221758044108\\
456.01	0.00402285130209053\\
457.01	0.00403373693912723\\
458.01	0.00404487835258123\\
459.01	0.00405627920932505\\
460.01	0.00406794297087019\\
461.01	0.00407987288890881\\
462.01	0.00409207200940272\\
463.01	0.00410454318670068\\
464.01	0.00411728910826407\\
465.01	0.00413031232870952\\
466.01	0.00414361530839268\\
467.01	0.00415720044562973\\
468.01	0.00417107008107845\\
469.01	0.0041852264260813\\
470.01	0.0041996713398888\\
471.01	0.00421440617148017\\
472.01	0.00422943173661071\\
473.01	0.00424474833547754\\
474.01	0.00426035582131055\\
475.01	0.00427625374672212\\
476.01	0.00429244162620634\\
477.01	0.00430891936952346\\
478.01	0.00432568796411437\\
479.01	0.00434275051881251\\
480.01	0.00436011382294717\\
481.01	0.00437778925709713\\
482.01	0.00439578057004691\\
483.01	0.00441408517387566\\
484.01	0.00443270191090969\\
485.01	0.00445163272657016\\
486.01	0.00447088499173598\\
487.01	0.00449047488246039\\
488.01	0.00451043223607692\\
489.01	0.0045308074601669\\
490.01	0.00455168128593993\\
491.01	0.00457315422736754\\
492.01	0.00459526008907832\\
493.01	0.0046180086043931\\
494.01	0.00464140788574584\\
495.01	0.00466546479504304\\
496.01	0.00469018583076554\\
497.01	0.00471557883003723\\
498.01	0.00474165656460672\\
499.01	0.00476844518523685\\
500.01	0.00479597870150315\\
501.01	0.00482429399915098\\
502.01	0.00485343100202368\\
503.01	0.0048834328864565\\
504.01	0.00491434627623896\\
505.01	0.00494622140426403\\
506.01	0.00497911220262801\\
507.01	0.00501307619044987\\
508.01	0.00504817404249849\\
509.01	0.00508446872486107\\
510.01	0.00512202399523604\\
511.01	0.00516090197633048\\
512.01	0.00520115939160515\\
513.01	0.0052428418864781\\
514.01	0.0052859756268194\\
515.01	0.00533055504474222\\
516.01	0.00537652515399289\\
517.01	0.00542367506389353\\
518.01	0.00547140676547196\\
519.01	0.0055195941845584\\
520.01	0.00556818235367103\\
521.01	0.00561710597037564\\
522.01	0.00566628677968869\\
523.01	0.00571562983441296\\
524.01	0.00576501816609789\\
525.01	0.00581430985838814\\
526.01	0.00586334494528416\\
527.01	0.00591195183244157\\
528.01	0.00595995539606151\\
529.01	0.00600718933154067\\
530.01	0.00605351576115472\\
531.01	0.00609885554544969\\
532.01	0.00614347792173343\\
533.01	0.00618796876807136\\
534.01	0.00623229035747514\\
535.01	0.00627638553022832\\
536.01	0.00632020584718589\\
537.01	0.00636371544694091\\
538.01	0.00640689538419013\\
539.01	0.00644974820779094\\
540.01	0.00649230232892437\\
541.01	0.00653461539186144\\
542.01	0.00657677531915586\\
543.01	0.00661889690504216\\
544.01	0.00666110859425139\\
545.01	0.00670348372223335\\
546.01	0.00674604040021065\\
547.01	0.00678880258411917\\
548.01	0.00683180313316688\\
549.01	0.00687508335988836\\
550.01	0.00691869185821967\\
551.01	0.00696268244849074\\
552.01	0.00700711107313386\\
553.01	0.0070520317686205\\
554.01	0.0070974922108247\\
555.01	0.00714353011714465\\
556.01	0.00719017612575834\\
557.01	0.0072374611684107\\
558.01	0.00728541715086486\\
559.01	0.00733407609564801\\
560.01	0.00738346924975821\\
561.01	0.00743362623227398\\
562.01	0.00748457434096044\\
563.01	0.00753633822330531\\
564.01	0.00758894009010531\\
565.01	0.00764240049063996\\
566.01	0.00769673903354819\\
567.01	0.00775197432695984\\
568.01	0.00780812368455662\\
569.01	0.00786520288823177\\
570.01	0.00792322603603263\\
571.01	0.00798220548225692\\
572.01	0.00804215184843037\\
573.01	0.00810307405076856\\
574.01	0.00816497926209387\\
575.01	0.0082278727525132\\
576.01	0.00829175766618825\\
577.01	0.00835663479827209\\
578.01	0.00842250237713441\\
579.01	0.00848935584384751\\
580.01	0.00855718761791611\\
581.01	0.00862598683882235\\
582.01	0.0086957390795814\\
583.01	0.00876642604196961\\
584.01	0.00883802525422727\\
585.01	0.00891050978858627\\
586.01	0.00898384801195749\\
587.01	0.00905800338572644\\
588.01	0.00913293433529357\\
589.01	0.00920859421638682\\
590.01	0.00928493141303536\\
591.01	0.0093618896103455\\
592.01	0.00943940829294075\\
593.01	0.00951742352873278\\
594.01	0.00959586911059195\\
595.01	0.0096746781463672\\
596.01	0.00975378521137988\\
597.01	0.00983312920843281\\
598.01	0.00990866201846347\\
599.01	0.00997087280416267\\
599.02	0.00997138072163717\\
599.03	0.00997188557625792\\
599.04	0.00997238733820205\\
599.05	0.00997288597735266\\
599.06	0.00997338146329598\\
599.07	0.00997387376531839\\
599.08	0.00997436285240344\\
599.09	0.00997484869322887\\
599.1	0.00997533125616357\\
599.11	0.00997581050926451\\
599.12	0.00997628642027369\\
599.13	0.00997675895661494\\
599.14	0.00997722808539087\\
599.15	0.00997769377337959\\
599.16	0.00997815598703155\\
599.17	0.00997861469246629\\
599.18	0.00997906985546913\\
599.19	0.0099795214414879\\
599.2	0.00997996941562956\\
599.21	0.00998041374265682\\
599.22	0.00998085438698478\\
599.23	0.00998129131267743\\
599.24	0.00998172448344417\\
599.25	0.00998215386263635\\
599.26	0.00998257941258352\\
599.27	0.00998300109279824\\
599.28	0.0099834188623898\\
599.29	0.00998383268006024\\
599.3	0.00998424250410031\\
599.31	0.00998464829238542\\
599.32	0.0099850500023715\\
599.33	0.00998544759109086\\
599.34	0.00998584101514799\\
599.35	0.00998623023071531\\
599.36	0.00998661519352895\\
599.37	0.00998699585888435\\
599.38	0.00998737218163196\\
599.39	0.00998774411617281\\
599.4	0.00998811161645403\\
599.41	0.00998847463596443\\
599.42	0.0099888331277299\\
599.43	0.00998918704430885\\
599.44	0.00998953633778757\\
599.45	0.00998988095977558\\
599.46	0.00999022086140088\\
599.47	0.00999055599330522\\
599.48	0.00999088630563925\\
599.49	0.00999121174805768\\
599.5	0.00999153226971434\\
599.51	0.0099918478192573\\
599.52	0.00999215834482374\\
599.53	0.00999246379403503\\
599.54	0.0099927641139915\\
599.55	0.00999305925126738\\
599.56	0.00999334915190552\\
599.57	0.0099936337614122\\
599.58	0.00999391302475173\\
599.59	0.00999418688634118\\
599.6	0.00999445529004489\\
599.61	0.00999471817916904\\
599.62	0.00999497549645613\\
599.63	0.00999522718407935\\
599.64	0.009995473183637\\
599.65	0.00999571343614678\\
599.66	0.00999594788204004\\
599.67	0.00999617646115599\\
599.68	0.0099963991127358\\
599.69	0.00999661577541675\\
599.7	0.00999682638722618\\
599.71	0.00999703088557551\\
599.72	0.00999722920725411\\
599.73	0.00999742128842315\\
599.74	0.00999760706460942\\
599.75	0.00999778647069897\\
599.76	0.00999795944093086\\
599.77	0.0099981259088907\\
599.78	0.0099982858075042\\
599.79	0.00999843906903064\\
599.8	0.00999858562505627\\
599.81	0.00999872540648767\\
599.82	0.00999885834354498\\
599.83	0.00999898436575515\\
599.84	0.00999910340194508\\
599.85	0.00999921538023465\\
599.86	0.00999932022802977\\
599.87	0.00999941787201528\\
599.88	0.00999950823814785\\
599.89	0.00999959125164875\\
599.9	0.00999966683699656\\
599.91	0.00999973491791987\\
599.92	0.0099997954173898\\
599.93	0.00999984825761255\\
599.94	0.00999989336002181\\
599.95	0.00999993064527112\\
599.96	0.00999996003322615\\
599.97	0.00999998144295691\\
599.98	0.00999999479272987\\
599.99	0.01\\
600	0.01\\
};
\addplot [color=blue!75!mycolor7,solid,forget plot]
  table[row sep=crcr]{%
0.01	0.00270247467503103\\
1.01	0.00270247533429627\\
2.01	0.00270247600752455\\
3.01	0.00270247669501311\\
4.01	0.00270247739706596\\
5.01	0.00270247811399338\\
6.01	0.00270247884611237\\
7.01	0.00270247959374664\\
8.01	0.00270248035722696\\
9.01	0.00270248113689134\\
10.01	0.00270248193308464\\
11.01	0.00270248274615945\\
12.01	0.00270248357647591\\
13.01	0.00270248442440169\\
14.01	0.00270248529031265\\
15.01	0.0027024861745924\\
16.01	0.00270248707763303\\
17.01	0.00270248799983511\\
18.01	0.00270248894160763\\
19.01	0.00270248990336849\\
20.01	0.00270249088554445\\
21.01	0.00270249188857176\\
22.01	0.00270249291289565\\
23.01	0.00270249395897115\\
24.01	0.00270249502726301\\
25.01	0.0027024961182462\\
26.01	0.00270249723240559\\
27.01	0.00270249837023682\\
28.01	0.00270249953224588\\
29.01	0.00270250071895002\\
30.01	0.00270250193087732\\
31.01	0.00270250316856732\\
32.01	0.0027025044325714\\
33.01	0.00270250572345257\\
34.01	0.0027025070417861\\
35.01	0.00270250838815958\\
36.01	0.00270250976317325\\
37.01	0.00270251116744064\\
38.01	0.00270251260158821\\
39.01	0.00270251406625591\\
40.01	0.00270251556209768\\
41.01	0.00270251708978153\\
42.01	0.00270251864998984\\
43.01	0.0027025202434199\\
44.01	0.00270252187078381\\
45.01	0.00270252353280925\\
46.01	0.00270252523023966\\
47.01	0.00270252696383441\\
48.01	0.00270252873436937\\
49.01	0.00270253054263721\\
50.01	0.00270253238944777\\
51.01	0.00270253427562822\\
52.01	0.00270253620202393\\
53.01	0.00270253816949839\\
54.01	0.00270254017893395\\
55.01	0.00270254223123183\\
56.01	0.00270254432731296\\
57.01	0.00270254646811821\\
58.01	0.00270254865460874\\
59.01	0.00270255088776668\\
60.01	0.00270255316859545\\
61.01	0.00270255549812008\\
62.01	0.00270255787738798\\
63.01	0.00270256030746932\\
64.01	0.00270256278945737\\
65.01	0.00270256532446929\\
66.01	0.00270256791364634\\
67.01	0.00270257055815473\\
68.01	0.00270257325918585\\
69.01	0.00270257601795711\\
70.01	0.00270257883571216\\
71.01	0.00270258171372205\\
72.01	0.00270258465328506\\
73.01	0.00270258765572813\\
74.01	0.00270259072240689\\
75.01	0.00270259385470629\\
76.01	0.00270259705404162\\
77.01	0.00270260032185902\\
78.01	0.00270260365963614\\
79.01	0.00270260706888252\\
80.01	0.00270261055114097\\
81.01	0.00270261410798775\\
82.01	0.00270261774103349\\
83.01	0.00270262145192399\\
84.01	0.00270262524234094\\
85.01	0.00270262911400267\\
86.01	0.00270263306866504\\
87.01	0.00270263710812213\\
88.01	0.00270264123420753\\
89.01	0.00270264544879445\\
90.01	0.00270264975379733\\
91.01	0.0027026541511723\\
92.01	0.00270265864291827\\
93.01	0.0027026632310779\\
94.01	0.00270266791773846\\
95.01	0.00270267270503299\\
96.01	0.00270267759514119\\
97.01	0.00270268259029019\\
98.01	0.00270268769275631\\
99.01	0.00270269290486544\\
100.01	0.00270269822899444\\
101.01	0.00270270366757241\\
102.01	0.00270270922308156\\
103.01	0.00270271489805848\\
104.01	0.00270272069509551\\
105.01	0.00270272661684193\\
106.01	0.00270273266600483\\
107.01	0.00270273884535099\\
108.01	0.00270274515770798\\
109.01	0.00270275160596534\\
110.01	0.00270275819307605\\
111.01	0.00270276492205816\\
112.01	0.00270277179599596\\
113.01	0.00270277881804147\\
114.01	0.00270278599141625\\
115.01	0.00270279331941242\\
116.01	0.00270280080539494\\
117.01	0.00270280845280267\\
118.01	0.00270281626514995\\
119.01	0.00270282424602916\\
120.01	0.00270283239911125\\
121.01	0.00270284072814832\\
122.01	0.00270284923697524\\
123.01	0.00270285792951139\\
124.01	0.00270286680976269\\
125.01	0.00270287588182341\\
126.01	0.00270288514987828\\
127.01	0.00270289461820446\\
128.01	0.00270290429117342\\
129.01	0.00270291417325338\\
130.01	0.00270292426901132\\
131.01	0.00270293458311537\\
132.01	0.00270294512033654\\
133.01	0.00270295588555169\\
134.01	0.0027029668837457\\
135.01	0.00270297812001373\\
136.01	0.0027029895995636\\
137.01	0.00270300132771877\\
138.01	0.00270301330992052\\
139.01	0.0027030255517311\\
140.01	0.00270303805883587\\
141.01	0.00270305083704628\\
142.01	0.00270306389230295\\
143.01	0.00270307723067827\\
144.01	0.00270309085837955\\
145.01	0.00270310478175213\\
146.01	0.00270311900728221\\
147.01	0.00270313354160019\\
148.01	0.00270314839148419\\
149.01	0.0027031635638628\\
150.01	0.00270317906581906\\
151.01	0.00270319490459373\\
152.01	0.00270321108758879\\
153.01	0.00270322762237106\\
154.01	0.00270324451667618\\
155.01	0.00270326177841231\\
156.01	0.0027032794156638\\
157.01	0.00270329743669538\\
158.01	0.00270331584995673\\
159.01	0.00270333466408561\\
160.01	0.00270335388791323\\
161.01	0.00270337353046794\\
162.01	0.00270339360097988\\
163.01	0.00270341410888571\\
164.01	0.00270343506383317\\
165.01	0.00270345647568597\\
166.01	0.00270347835452865\\
167.01	0.00270350071067162\\
168.01	0.00270352355465635\\
169.01	0.00270354689726052\\
170.01	0.00270357074950371\\
171.01	0.00270359512265228\\
172.01	0.00270362002822581\\
173.01	0.00270364547800236\\
174.01	0.00270367148402458\\
175.01	0.00270369805860558\\
176.01	0.00270372521433532\\
177.01	0.00270375296408672\\
178.01	0.00270378132102239\\
179.01	0.0027038102986011\\
180.01	0.00270383991058468\\
181.01	0.00270387017104476\\
182.01	0.00270390109437012\\
183.01	0.00270393269527364\\
184.01	0.00270396498880028\\
185.01	0.00270399799033418\\
186.01	0.0027040317156065\\
187.01	0.00270406618070387\\
188.01	0.00270410140207589\\
189.01	0.00270413739654395\\
190.01	0.00270417418130976\\
191.01	0.00270421177396382\\
192.01	0.00270425019249451\\
193.01	0.00270428945529744\\
194.01	0.00270432958118458\\
195.01	0.00270437058939391\\
196.01	0.00270441249959952\\
197.01	0.00270445533192122\\
198.01	0.0027044991069352\\
199.01	0.00270454384568428\\
200.01	0.00270458956968893\\
201.01	0.00270463630095816\\
202.01	0.0027046840620009\\
203.01	0.00270473287583753\\
204.01	0.00270478276601178\\
205.01	0.00270483375660273\\
206.01	0.00270488587223702\\
207.01	0.00270493913810216\\
208.01	0.00270499357995892\\
209.01	0.00270504922415492\\
210.01	0.0027051060976382\\
211.01	0.00270516422797092\\
212.01	0.00270522364334395\\
213.01	0.00270528437259129\\
214.01	0.002705346445205\\
215.01	0.00270540989135065\\
216.01	0.00270547474188272\\
217.01	0.00270554102836068\\
218.01	0.00270560878306568\\
219.01	0.00270567803901699\\
220.01	0.00270574882998924\\
221.01	0.00270582119053019\\
222.01	0.00270589515597848\\
223.01	0.00270597076248241\\
224.01	0.00270604804701824\\
225.01	0.00270612704741006\\
226.01	0.00270620780234948\\
227.01	0.00270629035141534\\
228.01	0.00270637473509498\\
229.01	0.0027064609948053\\
230.01	0.00270654917291427\\
231.01	0.00270663931276335\\
232.01	0.00270673145869008\\
233.01	0.0027068256560514\\
234.01	0.00270692195124765\\
235.01	0.00270702039174651\\
236.01	0.00270712102610837\\
237.01	0.00270722390401164\\
238.01	0.00270732907627895\\
239.01	0.00270743659490411\\
240.01	0.0027075465130791\\
241.01	0.00270765888522241\\
242.01	0.00270777376700783\\
243.01	0.00270789121539326\\
244.01	0.00270801128865152\\
245.01	0.00270813404640037\\
246.01	0.00270825954963483\\
247.01	0.00270838786075876\\
248.01	0.00270851904361798\\
249.01	0.00270865316353411\\
250.01	0.00270879028733922\\
251.01	0.00270893048341113\\
252.01	0.00270907382170958\\
253.01	0.00270922037381315\\
254.01	0.00270937021295713\\
255.01	0.00270952341407281\\
256.01	0.00270968005382667\\
257.01	0.00270984021066105\\
258.01	0.00271000396483605\\
259.01	0.00271017139847174\\
260.01	0.00271034259559182\\
261.01	0.00271051764216831\\
262.01	0.00271069662616669\\
263.01	0.00271087963759313\\
264.01	0.00271106676854173\\
265.01	0.00271125811324362\\
266.01	0.00271145376811681\\
267.01	0.0027116538318173\\
268.01	0.00271185840529134\\
269.01	0.00271206759182897\\
270.01	0.00271228149711895\\
271.01	0.00271250022930422\\
272.01	0.00271272389903977\\
273.01	0.0027129526195508\\
274.01	0.00271318650669294\\
275.01	0.00271342567901324\\
276.01	0.00271367025781309\\
277.01	0.00271392036721234\\
278.01	0.00271417613421482\\
279.01	0.00271443768877547\\
280.01	0.00271470516386876\\
281.01	0.00271497869555914\\
282.01	0.0027152584230726\\
283.01	0.00271554448886996\\
284.01	0.00271583703872232\\
285.01	0.00271613622178692\\
286.01	0.00271644219068644\\
287.01	0.00271675510158866\\
288.01	0.00271707511428881\\
289.01	0.00271740239229304\\
290.01	0.00271773710290442\\
291.01	0.00271807941731017\\
292.01	0.00271842951067162\\
293.01	0.00271878756221533\\
294.01	0.00271915375532732\\
295.01	0.0027195282776481\\
296.01	0.00271991132117057\\
297.01	0.00272030308234024\\
298.01	0.00272070376215712\\
299.01	0.00272111356628033\\
300.01	0.00272153270513456\\
301.01	0.00272196139401958\\
302.01	0.00272239985322109\\
303.01	0.00272284830812539\\
304.01	0.00272330698933557\\
305.01	0.00272377613279026\\
306.01	0.00272425597988599\\
307.01	0.00272474677760093\\
308.01	0.00272524877862255\\
309.01	0.00272576224147726\\
310.01	0.00272628743066379\\
311.01	0.00272682461678848\\
312.01	0.00272737407670492\\
313.01	0.00272793609365547\\
314.01	0.00272851095741694\\
315.01	0.0027290989644492\\
316.01	0.00272970041804706\\
317.01	0.00273031562849621\\
318.01	0.00273094491323209\\
319.01	0.00273158859700301\\
320.01	0.00273224701203695\\
321.01	0.0027329204982125\\
322.01	0.00273360940323363\\
323.01	0.00273431408280884\\
324.01	0.00273503490083495\\
325.01	0.00273577222958521\\
326.01	0.00273652644990233\\
327.01	0.00273729795139636\\
328.01	0.00273808713264778\\
329.01	0.00273889440141571\\
330.01	0.0027397201748519\\
331.01	0.00274056487972089\\
332.01	0.00274142895262519\\
333.01	0.00274231284023753\\
334.01	0.00274321699953993\\
335.01	0.00274414189806882\\
336.01	0.00274508801416797\\
337.01	0.00274605583724873\\
338.01	0.00274704586805798\\
339.01	0.00274805861895508\\
340.01	0.00274909461419602\\
341.01	0.00275015439022826\\
342.01	0.00275123849599369\\
343.01	0.00275234749324232\\
344.01	0.00275348195685582\\
345.01	0.00275464247518248\\
346.01	0.00275582965038318\\
347.01	0.00275704409878906\\
348.01	0.00275828645127285\\
349.01	0.00275955735363188\\
350.01	0.00276085746698575\\
351.01	0.00276218746818835\\
352.01	0.00276354805025458\\
353.01	0.00276493992280309\\
354.01	0.00276636381251567\\
355.01	0.00276782046361405\\
356.01	0.00276931063835421\\
357.01	0.00277083511754045\\
358.01	0.00277239470105849\\
359.01	0.00277399020842996\\
360.01	0.0027756224793874\\
361.01	0.00277729237447215\\
362.01	0.00277900077565511\\
363.01	0.00278074858698127\\
364.01	0.0027825367352384\\
365.01	0.00278436617065053\\
366.01	0.00278623786759656\\
367.01	0.00278815282535375\\
368.01	0.00279011206886554\\
369.01	0.00279211664953301\\
370.01	0.00279416764602901\\
371.01	0.002796266165132\\
372.01	0.00279841334257751\\
373.01	0.00280061034392369\\
374.01	0.00280285836542661\\
375.01	0.00280515863492006\\
376.01	0.00280751241269528\\
377.01	0.00280992099237441\\
378.01	0.00281238570177332\\
379.01	0.00281490790375073\\
380.01	0.00281748899704232\\
381.01	0.00282013041708387\\
382.01	0.00282283363683311\\
383.01	0.00282560016760509\\
384.01	0.00282843155994579\\
385.01	0.00283132940456879\\
386.01	0.00283429533337751\\
387.01	0.00283733102056727\\
388.01	0.00284043818377587\\
389.01	0.00284361858526741\\
390.01	0.00284687403314821\\
391.01	0.00285020638261679\\
392.01	0.00285361753724704\\
393.01	0.00285710945030708\\
394.01	0.00286068412611297\\
395.01	0.00286434362142014\\
396.01	0.00286809004685051\\
397.01	0.00287192556835872\\
398.01	0.00287585240873706\\
399.01	0.00287987284915896\\
400.01	0.00288398923076455\\
401.01	0.0028882039562859\\
402.01	0.00289251949171476\\
403.01	0.00289693836801229\\
404.01	0.00290146318286257\\
405.01	0.00290609660246954\\
406.01	0.00291084136339893\\
407.01	0.00291570027446451\\
408.01	0.00292067621866084\\
409.01	0.00292577215514149\\
410.01	0.00293099112124419\\
411.01	0.00293633623456342\\
412.01	0.0029418106950693\\
413.01	0.00294741778727514\\
414.01	0.00295316088245261\\
415.01	0.00295904344089451\\
416.01	0.0029650690142259\\
417.01	0.00297124124776349\\
418.01	0.00297756388292232\\
419.01	0.0029840407596702\\
420.01	0.00299067581902952\\
421.01	0.00299747310562577\\
422.01	0.0030044367702815\\
423.01	0.00301157107265605\\
424.01	0.00301888038392892\\
425.01	0.00302636918952615\\
426.01	0.00303404209188808\\
427.01	0.00304190381327682\\
428.01	0.00304995919862031\\
429.01	0.00305821321839271\\
430.01	0.00306667097152531\\
431.01	0.00307533768834659\\
432.01	0.0030842187335459\\
433.01	0.00309331960915568\\
434.01	0.00310264595754579\\
435.01	0.00311220356442228\\
436.01	0.00312199836182031\\
437.01	0.00313203643107907\\
438.01	0.00314232400578267\\
439.01	0.00315286747464798\\
440.01	0.00316367338433383\\
441.01	0.00317474844213898\\
442.01	0.00318609951854902\\
443.01	0.00319773364957773\\
444.01	0.00320965803883519\\
445.01	0.00322188005923436\\
446.01	0.00323440725422265\\
447.01	0.00324724733839032\\
448.01	0.00326040819726636\\
449.01	0.0032738978860566\\
450.01	0.00328772462700695\\
451.01	0.00330189680498473\\
452.01	0.00331642296075368\\
453.01	0.00333131178126965\\
454.01	0.00334657208613542\\
455.01	0.00336221280911142\\
456.01	0.00337824297328074\\
457.01	0.00339467165808292\\
458.01	0.00341150795596004\\
459.01	0.0034287609157695\\
460.01	0.00344643946939566\\
461.01	0.00346455233711562\\
462.01	0.00348310790622741\\
463.01	0.00350211407624967\\
464.01	0.00352157806272528\\
465.01	0.00354150615061395\\
466.01	0.00356190338840462\\
467.01	0.00358277321850817\\
468.01	0.00360411705825751\\
469.01	0.00362593384889105\\
470.01	0.0036482193487216\\
471.01	0.00367096518644888\\
472.01	0.00369415773920015\\
473.01	0.00371777671173217\\
474.01	0.00374179332594245\\
475.01	0.00376616800608949\\
476.01	0.0037908474080284\\
477.01	0.00381576058788272\\
478.01	0.0038408140226982\\
479.01	0.00386588508688511\\
480.01	0.00389081376234303\\
481.01	0.00391560395586256\\
482.01	0.00394074112993927\\
483.01	0.00396620506208416\\
484.01	0.00399191528349777\\
485.01	0.00401776461077528\\
486.01	0.00404361142247187\\
487.01	0.00406926963312608\\
488.01	0.00409449564862648\\
489.01	0.00411897135558789\\
490.01	0.00414228188906971\\
491.01	0.00416507254868226\\
492.01	0.00418846898558721\\
493.01	0.00421248742566058\\
494.01	0.00423714381453705\\
495.01	0.00426245500524932\\
496.01	0.00428843889910915\\
497.01	0.00431511453216295\\
498.01	0.00434250203415325\\
499.01	0.00437062228851151\\
500.01	0.00439949642593\\
501.01	0.0044291454734143\\
502.01	0.00445958989481118\\
503.01	0.00449084888929291\\
504.01	0.00452293933632795\\
505.01	0.00455587427112269\\
506.01	0.00458966202379704\\
507.01	0.00462430698135259\\
508.01	0.00465980928779375\\
509.01	0.00469616417384544\\
510.01	0.0047333613215893\\
511.01	0.00477138434497626\\
512.01	0.0048102105109371\\
513.01	0.00484981089078331\\
514.01	0.00489015122506367\\
515.01	0.00493119391866482\\
516.01	0.00497290177298885\\
517.01	0.00501524473887385\\
518.01	0.00505821584207151\\
519.01	0.00510183294455671\\
520.01	0.00514613671083838\\
521.01	0.00519120211992206\\
522.01	0.00523715530414039\\
523.01	0.00528419723132036\\
524.01	0.0053326260424256\\
525.01	0.00538267931871982\\
526.01	0.00543447963477701\\
527.01	0.00548809594440868\\
528.01	0.00554356281551956\\
529.01	0.00560087933159319\\
530.01	0.0056599898495372\\
531.01	0.00572075662910118\\
532.01	0.00578268190428666\\
533.01	0.00584498022083387\\
534.01	0.00590749574688542\\
535.01	0.00597007948607768\\
536.01	0.00603256371658051\\
537.01	0.00609476253691515\\
538.01	0.00615647411242743\\
539.01	0.00621748539801371\\
540.01	0.00627758036634497\\
541.01	0.00633655347722414\\
542.01	0.0063942310381018\\
543.01	0.00645050344564633\\
544.01	0.00650549634327943\\
545.01	0.00656008411933888\\
546.01	0.00661435765076117\\
547.01	0.00666825438150137\\
548.01	0.00672172896272959\\
549.01	0.00677475834979131\\
550.01	0.00682734663152718\\
551.01	0.0068795317334653\\
552.01	0.00693139203259288\\
553.01	0.00698304779253108\\
554.01	0.00703465559071412\\
555.01	0.00708638066172824\\
556.01	0.00713829638688321\\
557.01	0.00719043503387839\\
558.01	0.00724283853598048\\
559.01	0.00729555771687146\\
560.01	0.00734865055334202\\
561.01	0.00740218005541401\\
562.01	0.0074562106142838\\
563.01	0.00751080248597193\\
564.01	0.00756600583168564\\
565.01	0.00762185735902811\\
566.01	0.00767838795473285\\
567.01	0.00773562904201511\\
568.01	0.00779361189922099\\
569.01	0.00785236643307945\\
570.01	0.00791191989642814\\
571.01	0.00797229587071889\\
572.01	0.00803351381482855\\
573.01	0.00809558940754691\\
574.01	0.00815853564761196\\
575.01	0.00822236373310811\\
576.01	0.00828708286240763\\
577.01	0.00835269977667169\\
578.01	0.00841921839411719\\
579.01	0.00848663956920399\\
580.01	0.00855496097497914\\
581.01	0.00862417706125201\\
582.01	0.00869427898579362\\
583.01	0.00876525439463777\\
584.01	0.00883708706023637\\
585.01	0.00890975649722161\\
586.01	0.00898323760124607\\
587.01	0.00905750032875147\\
588.01	0.00913250943662461\\
589.01	0.00920822430118101\\
590.01	0.00928459884902855\\
591.01	0.00936158165898825\\
592.01	0.00943911632123402\\
593.01	0.00951714214254039\\
594.01	0.00959559528261798\\
595.01	0.00967441041276702\\
596.01	0.00975352299930713\\
597.01	0.00983287233211095\\
598.01	0.00990866201774166\\
599.01	0.0099708728041569\\
599.02	0.00997138072163179\\
599.03	0.0099718855762529\\
599.04	0.00997238733819737\\
599.05	0.0099728859773483\\
599.06	0.00997338146329193\\
599.07	0.00997387376531463\\
599.08	0.00997436285239995\\
599.09	0.00997484869322563\\
599.1	0.00997533125616056\\
599.11	0.00997581050926173\\
599.12	0.00997628642027111\\
599.13	0.00997675895661256\\
599.14	0.00997722808538867\\
599.15	0.00997769377337755\\
599.16	0.00997815598702968\\
599.17	0.00997861469246456\\
599.18	0.00997906985546755\\
599.19	0.00997952144148644\\
599.2	0.00997996941562822\\
599.21	0.0099804137426556\\
599.22	0.00998085438698366\\
599.23	0.0099812913126764\\
599.24	0.00998172448344324\\
599.25	0.0099821538626355\\
599.26	0.00998257941258275\\
599.27	0.00998300109279754\\
599.28	0.00998341886238916\\
599.29	0.00998383268005965\\
599.3	0.00998424250409978\\
599.31	0.00998464829238495\\
599.32	0.00998505000237108\\
599.33	0.00998544759109047\\
599.34	0.00998584101514764\\
599.35	0.009986230230715\\
599.36	0.00998661519352867\\
599.37	0.0099869958588841\\
599.38	0.00998737218163174\\
599.39	0.00998774411617261\\
599.4	0.00998811161645385\\
599.41	0.00998847463596427\\
599.42	0.00998883312772976\\
599.43	0.00998918704430873\\
599.44	0.00998953633778746\\
599.45	0.00998988095977548\\
599.46	0.0099902208614008\\
599.47	0.00999055599330515\\
599.48	0.00999088630563919\\
599.49	0.00999121174805762\\
599.5	0.0099915322697143\\
599.51	0.00999184781925726\\
599.52	0.00999215834482371\\
599.53	0.009992463794035\\
599.54	0.00999276411399148\\
599.55	0.00999305925126736\\
599.56	0.00999334915190551\\
599.57	0.00999363376141218\\
599.58	0.00999391302475172\\
599.59	0.00999418688634117\\
599.6	0.00999445529004488\\
599.61	0.00999471817916904\\
599.62	0.00999497549645612\\
599.63	0.00999522718407934\\
599.64	0.00999547318363699\\
599.65	0.00999571343614677\\
599.66	0.00999594788204004\\
599.67	0.00999617646115598\\
599.68	0.0099963991127358\\
599.69	0.00999661577541675\\
599.7	0.00999682638722618\\
599.71	0.00999703088557551\\
599.72	0.00999722920725411\\
599.73	0.00999742128842315\\
599.74	0.00999760706460942\\
599.75	0.00999778647069897\\
599.76	0.00999795944093086\\
599.77	0.0099981259088907\\
599.78	0.0099982858075042\\
599.79	0.00999843906903064\\
599.8	0.00999858562505627\\
599.81	0.00999872540648767\\
599.82	0.00999885834354498\\
599.83	0.00999898436575515\\
599.84	0.00999910340194508\\
599.85	0.00999921538023465\\
599.86	0.00999932022802977\\
599.87	0.00999941787201528\\
599.88	0.00999950823814785\\
599.89	0.00999959125164875\\
599.9	0.00999966683699656\\
599.91	0.00999973491791987\\
599.92	0.0099997954173898\\
599.93	0.00999984825761255\\
599.94	0.00999989336002181\\
599.95	0.00999993064527112\\
599.96	0.00999996003322615\\
599.97	0.00999998144295691\\
599.98	0.00999999479272987\\
599.99	0.01\\
600	0.01\\
};
\addplot [color=blue!80!mycolor9,solid,forget plot]
  table[row sep=crcr]{%
0.01	0.000977852700539981\\
1.01	0.000977853634422134\\
2.01	0.000977854588141196\\
3.01	0.000977855562121314\\
4.01	0.000977856556795567\\
5.01	0.000977857572606509\\
6.01	0.000977858610006203\\
7.01	0.000977859669456425\\
8.01	0.000977860751428916\\
9.01	0.000977861856405533\\
10.01	0.000977862984878675\\
11.01	0.000977864137351257\\
12.01	0.000977865314337065\\
13.01	0.000977866516360955\\
14.01	0.000977867743959124\\
15.01	0.000977868997679321\\
16.01	0.000977870278081203\\
17.01	0.00097787158573636\\
18.01	0.000977872921228896\\
19.01	0.00097787428515541\\
20.01	0.000977875678125496\\
21.01	0.000977877100761824\\
22.01	0.000977878553700559\\
23.01	0.000977880037591639\\
24.01	0.000977881553099115\\
25.01	0.000977883100901272\\
26.01	0.000977884681691195\\
27.01	0.000977886296176872\\
28.01	0.000977887945081642\\
29.01	0.000977889629144538\\
30.01	0.000977891349120574\\
31.01	0.000977893105781148\\
32.01	0.000977894899914282\\
33.01	0.000977896732325137\\
34.01	0.000977898603836395\\
35.01	0.000977900515288496\\
36.01	0.000977902467540072\\
37.01	0.000977904461468414\\
38.01	0.00097790649796984\\
39.01	0.000977908577960193\\
40.01	0.000977910702375052\\
41.01	0.000977912872170437\\
42.01	0.00097791508832306\\
43.01	0.00097791735183083\\
44.01	0.000977919663713349\\
45.01	0.000977922025012408\\
46.01	0.000977924436792413\\
47.01	0.000977926900140917\\
48.01	0.000977929416169157\\
49.01	0.000977931986012488\\
50.01	0.000977934610830962\\
51.01	0.000977937291809956\\
52.01	0.000977940030160482\\
53.01	0.000977942827120125\\
54.01	0.000977945683953263\\
55.01	0.000977948601951856\\
56.01	0.000977951582436049\\
57.01	0.000977954626754755\\
58.01	0.00097795773628627\\
59.01	0.000977960912438923\\
60.01	0.000977964156651745\\
61.01	0.000977967470395219\\
62.01	0.000977970855171879\\
63.01	0.000977974312517095\\
64.01	0.000977977843999757\\
65.01	0.000977981451222902\\
66.01	0.000977985135824741\\
67.01	0.000977988899479122\\
68.01	0.000977992743896573\\
69.01	0.000977996670825036\\
70.01	0.000978000682050673\\
71.01	0.000978004779398585\\
72.01	0.000978008964733969\\
73.01	0.000978013239962737\\
74.01	0.000978017607032526\\
75.01	0.000978022067933695\\
76.01	0.000978026624700125\\
77.01	0.000978031279410277\\
78.01	0.00097803603418814\\
79.01	0.000978040891204311\\
80.01	0.000978045852676905\\
81.01	0.000978050920872761\\
82.01	0.000978056098108464\\
83.01	0.000978061386751328\\
84.01	0.000978066789220709\\
85.01	0.00097807230798911\\
86.01	0.000978077945583244\\
87.01	0.000978083704585513\\
88.01	0.000978089587634796\\
89.01	0.000978095597428223\\
90.01	0.000978101736722178\\
91.01	0.000978108008333575\\
92.01	0.000978114415141458\\
93.01	0.000978120960088147\\
94.01	0.000978127646180729\\
95.01	0.000978134476492525\\
96.01	0.000978141454164527\\
97.01	0.000978148582407093\\
98.01	0.000978155864501125\\
99.01	0.000978163303799975\\
100.01	0.000978170903731014\\
101.01	0.000978178667797105\\
102.01	0.000978186599578532\\
103.01	0.000978194702734539\\
104.01	0.000978202981005198\\
105.01	0.000978211438213146\\
106.01	0.000978220078265618\\
107.01	0.000978228905156135\\
108.01	0.000978237922966425\\
109.01	0.000978247135868679\\
110.01	0.000978256548127249\\
111.01	0.000978266164100912\\
112.01	0.000978275988244924\\
113.01	0.000978286025113302\\
114.01	0.000978296279360795\\
115.01	0.000978306755745513\\
116.01	0.000978317459130768\\
117.01	0.000978328394487936\\
118.01	0.00097833956689862\\
119.01	0.00097835098155704\\
120.01	0.000978362643772886\\
121.01	0.00097837455897367\\
122.01	0.000978386732707413\\
123.01	0.000978399170645523\\
124.01	0.00097841187858537\\
125.01	0.000978424862453239\\
126.01	0.000978438128307189\\
127.01	0.000978451682340092\\
128.01	0.000978465530882731\\
129.01	0.000978479680406708\\
130.01	0.000978494137527749\\
131.01	0.000978508909009045\\
132.01	0.000978524001764527\\
133.01	0.000978539422862211\\
134.01	0.000978555179527725\\
135.01	0.000978571279147918\\
136.01	0.000978587729274553\\
137.01	0.000978604537627846\\
138.01	0.0009786217121006\\
139.01	0.00097863926076177\\
140.01	0.000978657191860723\\
141.01	0.000978675513831254\\
142.01	0.00097869423529575\\
143.01	0.000978713365069467\\
144.01	0.000978732912164923\\
145.01	0.000978752885796436\\
146.01	0.000978773295384684\\
147.01	0.000978794150561374\\
148.01	0.000978815461174058\\
149.01	0.0009788372372911\\
150.01	0.00097885948920659\\
151.01	0.000978882227445584\\
152.01	0.000978905462769339\\
153.01	0.000978929206180701\\
154.01	0.000978953468929587\\
155.01	0.000978978262518707\\
156.01	0.000979003598709241\\
157.01	0.000979029489526842\\
158.01	0.000979055947267465\\
159.01	0.000979082984503894\\
160.01	0.000979110614091706\\
161.01	0.000979138849175894\\
162.01	0.000979167703197628\\
163.01	0.00097919718990077\\
164.01	0.000979227323338959\\
165.01	0.000979258117882749\\
166.01	0.00097928958822667\\
167.01	0.000979321749396848\\
168.01	0.000979354616758486\\
169.01	0.000979388206023696\\
170.01	0.000979422533259416\\
171.01	0.000979457614895658\\
172.01	0.000979493467733579\\
173.01	0.000979530108954289\\
174.01	0.00097956755612745\\
175.01	0.000979605827220183\\
176.01	0.000979644940606206\\
177.01	0.000979684915075218\\
178.01	0.000979725769842385\\
179.01	0.000979767524558271\\
180.01	0.000979810199318601\\
181.01	0.000979853814674782\\
182.01	0.000979898391644148\\
183.01	0.000979943951720866\\
184.01	0.000979990516886818\\
185.01	0.000980038109622841\\
186.01	0.000980086752920183\\
187.01	0.000980136470292346\\
188.01	0.000980187285787068\\
189.01	0.00098023922399873\\
190.01	0.000980292310080719\\
191.01	0.000980346569758483\\
192.01	0.000980402029342894\\
193.01	0.000980458715743308\\
194.01	0.000980516656481835\\
195.01	0.000980575879707124\\
196.01	0.000980636414208966\\
197.01	0.000980698289433195\\
198.01	0.000980761535496581\\
199.01	0.000980826183202589\\
200.01	0.000980892264057037\\
201.01	0.000980959810284399\\
202.01	0.000981028854844412\\
203.01	0.00098109943144898\\
204.01	0.000981171574579554\\
205.01	0.000981245319505095\\
206.01	0.000981320702300066\\
207.01	0.00098139775986307\\
208.01	0.000981476529935958\\
209.01	0.000981557051123218\\
210.01	0.000981639362912061\\
211.01	0.000981723505692672\\
212.01	0.000981809520779184\\
213.01	0.000981897450430896\\
214.01	0.00098198733787426\\
215.01	0.000982079227325149\\
216.01	0.000982173164011738\\
217.01	0.000982269194197881\\
218.01	0.000982367365207112\\
219.01	0.000982467725447043\\
220.01	0.00098257032443437\\
221.01	0.000982675212820631\\
222.01	0.000982782442418388\\
223.01	0.00098289206622787\\
224.01	0.000983004138464708\\
225.01	0.000983118714587679\\
226.01	0.000983235851327688\\
227.01	0.000983355606716904\\
228.01	0.000983478040118967\\
229.01	0.000983603212259584\\
230.01	0.000983731185257984\\
231.01	0.000983862022659251\\
232.01	0.000983995789466889\\
233.01	0.000984132552176664\\
234.01	0.000984272378810921\\
235.01	0.000984415338953833\\
236.01	0.000984561503787354\\
237.01	0.000984710946128009\\
238.01	0.000984863740464532\\
239.01	0.000985019962996402\\
240.01	0.000985179691673145\\
241.01	0.000985343006234672\\
242.01	0.00098550998825242\\
243.01	0.000985680721171445\\
244.01	0.000985855290353568\\
245.01	0.000986033783121336\\
246.01	0.000986216288803014\\
247.01	0.000986402898778722\\
248.01	0.000986593706527553\\
249.01	0.000986788807675575\\
250.01	0.000986988300045274\\
251.01	0.000987192283705654\\
252.01	0.000987400861023835\\
253.01	0.000987614136717815\\
254.01	0.00098783221791017\\
255.01	0.000988055214183013\\
256.01	0.000988283237634266\\
257.01	0.000988516402935354\\
258.01	0.000988754827389832\\
259.01	0.000988998630993609\\
260.01	0.000989247936496365\\
261.01	0.000989502869464469\\
262.01	0.000989763558345291\\
263.01	0.000990030134532631\\
264.01	0.000990302732434202\\
265.01	0.00099058148954017\\
266.01	0.000990866546493385\\
267.01	0.000991158047161086\\
268.01	0.000991456138708338\\
269.01	0.000991760971673181\\
270.01	0.000992072700043062\\
271.01	0.000992391481333565\\
272.01	0.000992717476668362\\
273.01	0.000993050850861391\\
274.01	0.000993391772500428\\
275.01	0.000993740414033099\\
276.01	0.000994096951854213\\
277.01	0.000994461566395627\\
278.01	0.000994834442217797\\
279.01	0.00099521576810329\\
280.01	0.000995605737152967\\
281.01	0.000996004546883714\\
282.01	0.000996412399328725\\
283.01	0.000996829501140058\\
284.01	0.000997256063693289\\
285.01	0.00099769230319513\\
286.01	0.000998138440792759\\
287.01	0.00099859470268631\\
288.01	0.000999061320243591\\
289.01	0.000999538530117548\\
290.01	0.00100002657436661\\
291.01	0.00100052570057756\\
292.01	0.00100103616199158\\
293.01	0.00100155821763315\\
294.01	0.00100209213244173\\
295.01	0.00100263817740722\\
296.01	0.00100319662970799\\
297.01	0.00100376777285256\\
298.01	0.00100435189682466\\
299.01	0.00100494929823174\\
300.01	0.00100556028045719\\
301.01	0.00100618515381621\\
302.01	0.00100682423571566\\
303.01	0.00100747785081769\\
304.01	0.00100814633120754\\
305.01	0.00100883001656578\\
306.01	0.00100952925434449\\
307.01	0.00101024439994826\\
308.01	0.00101097581691967\\
309.01	0.00101172387712957\\
310.01	0.00101248896097245\\
311.01	0.00101327145756668\\
312.01	0.00101407176496008\\
313.01	0.0010148902903412\\
314.01	0.00101572745025582\\
315.01	0.00101658367082933\\
316.01	0.00101745938799547\\
317.01	0.00101835504773071\\
318.01	0.00101927110629547\\
319.01	0.00102020803048173\\
320.01	0.00102116629786739\\
321.01	0.00102214639707782\\
322.01	0.00102314882805461\\
323.01	0.00102417410233171\\
324.01	0.00102522274331941\\
325.01	0.00102629528659612\\
326.01	0.00102739228020864\\
327.01	0.0010285142849805\\
328.01	0.00102966187482923\\
329.01	0.00103083563709257\\
330.01	0.00103203617286379\\
331.01	0.00103326409733641\\
332.01	0.001034520040159\\
333.01	0.00103580464579946\\
334.01	0.00103711857391993\\
335.01	0.00103846249976194\\
336.01	0.00103983711454256\\
337.01	0.00104124312586123\\
338.01	0.00104268125811837\\
339.01	0.00104415225294487\\
340.01	0.00104565686964395\\
341.01	0.00104719588564468\\
342.01	0.00104877009696788\\
343.01	0.00105038031870457\\
344.01	0.00105202738550707\\
345.01	0.00105371215209317\\
346.01	0.00105543549376341\\
347.01	0.00105719830693203\\
348.01	0.00105900150967126\\
349.01	0.00106084604226982\\
350.01	0.00106273286780535\\
351.01	0.00106466297273126\\
352.01	0.0010666373674783\\
353.01	0.00106865708707073\\
354.01	0.00107072319175775\\
355.01	0.00107283676766021\\
356.01	0.00107499892743283\\
357.01	0.00107721081094243\\
358.01	0.0010794735859621\\
359.01	0.00108178844888184\\
360.01	0.00108415662543624\\
361.01	0.00108657937144881\\
362.01	0.00108905797359376\\
363.01	0.00109159375017568\\
364.01	0.0010941880519273\\
365.01	0.00109684226282558\\
366.01	0.00109955780092683\\
367.01	0.00110233611922096\\
368.01	0.00110517870650565\\
369.01	0.00110808708828062\\
370.01	0.00111106282766278\\
371.01	0.00111410752632268\\
372.01	0.001117222825443\\
373.01	0.00112041040669993\\
374.01	0.00112367199326822\\
375.01	0.0011270093508514\\
376.01	0.00113042428873812\\
377.01	0.00113391866088624\\
378.01	0.00113749436703708\\
379.01	0.00114115335386133\\
380.01	0.00114489761613958\\
381.01	0.00114872919797993\\
382.01	0.00115265019407521\\
383.01	0.00115666275100261\\
384.01	0.00116076906856723\\
385.01	0.00116497140119103\\
386.01	0.00116927205934747\\
387.01	0.0011736734110416\\
388.01	0.00117817788333639\\
389.01	0.00118278796392682\\
390.01	0.00118750620276415\\
391.01	0.00119233521373234\\
392.01	0.00119727767637918\\
393.01	0.00120233633770445\\
394.01	0.00120751401400818\\
395.01	0.00121281359280129\\
396.01	0.00121823803478228\\
397.01	0.00122379037588293\\
398.01	0.00122947372938625\\
399.01	0.00123529128812105\\
400.01	0.00124124632673615\\
401.01	0.0012473422040592\\
402.01	0.00125358236554408\\
403.01	0.0012599703458121\\
404.01	0.00126650977129149\\
405.01	0.00127320436296136\\
406.01	0.00128005793920519\\
407.01	0.00128707441878058\\
408.01	0.00129425782391166\\
409.01	0.00130161228351112\\
410.01	0.00130914203653993\\
411.01	0.00131685143551185\\
412.01	0.00132474495015258\\
413.01	0.00133282717122183\\
414.01	0.00134110281450882\\
415.01	0.00134957672501127\\
416.01	0.00135825388130991\\
417.01	0.00136713940015008\\
418.01	0.00137623854124377\\
419.01	0.00138555671230634\\
420.01	0.00139509947434261\\
421.01	0.00140487254719901\\
422.01	0.00141488181539902\\
423.01	0.00142513333428104\\
424.01	0.00143563333645922\\
425.01	0.00144638823862921\\
426.01	0.00145740464874334\\
427.01	0.00146868937358098\\
428.01	0.00148024942674369\\
429.01	0.00149209203710532\\
430.01	0.00150422465775273\\
431.01	0.00151665497545458\\
432.01	0.00152939092070017\\
433.01	0.00154244067835579\\
434.01	0.00155581269899002\\
435.01	0.00156951571092695\\
436.01	0.00158355873309278\\
437.01	0.00159795108873086\\
438.01	0.00161270242006914\\
439.01	0.00162782270403704\\
440.01	0.00164332226914268\\
441.01	0.00165921181363873\\
442.01	0.00167550242512398\\
443.01	0.00169220560175436\\
444.01	0.00170933327526415\\
445.01	0.00172689783603421\\
446.01	0.00174491216048532\\
447.01	0.00176338964112644\\
448.01	0.00178234421964791\\
449.01	0.0018017904235238\\
450.01	0.00182174340667754\\
451.01	0.00184221899487168\\
452.01	0.0018632337366148\\
453.01	0.00188480496053472\\
454.01	0.00190695084035944\\
455.01	0.0019296904688766\\
456.01	0.00195304394251991\\
457.01	0.00197703245856441\\
458.01	0.0020016784273107\\
459.01	0.00202700560210914\\
460.01	0.00205303923061743\\
461.01	0.0020798062312703\\
462.01	0.00210733539945335\\
463.01	0.00213565764792342\\
464.01	0.00216480628436238\\
465.01	0.00219481732160913\\
466.01	0.00222572979017696\\
467.01	0.00225758593534509\\
468.01	0.0022904309084886\\
469.01	0.00232431367757898\\
470.01	0.00235929070988804\\
471.01	0.00239542522304157\\
472.01	0.00243278801778327\\
473.01	0.00247145889961402\\
474.01	0.00251152843611212\\
475.01	0.00255310014955491\\
476.01	0.00259629329740276\\
477.01	0.00264124651736053\\
478.01	0.00268812296161684\\
479.01	0.00273711620747841\\
480.01	0.00278845274894908\\
481.01	0.00281809485375913\\
482.01	0.0028376904564289\\
483.01	0.00285824238600901\\
484.01	0.00287986986848995\\
485.01	0.00290271791097438\\
486.01	0.00292696401033721\\
487.01	0.00295282676371047\\
488.01	0.00298057695173238\\
489.01	0.0030105518446611\\
490.01	0.00304317371901346\\
491.01	0.00307779714308335\\
492.01	0.00311332206216349\\
493.01	0.00314976706439995\\
494.01	0.00318715176354267\\
495.01	0.00322549583333774\\
496.01	0.00326481914899174\\
497.01	0.00330514204057174\\
498.01	0.00334648571336444\\
499.01	0.00338887291518974\\
500.01	0.0034323289608935\\
501.01	0.00347688324818993\\
502.01	0.00352257144897241\\
503.01	0.00356943863488352\\
504.01	0.00361754369033495\\
505.01	0.00366696072314491\\
506.01	0.00371773565702078\\
507.01	0.00376988797378531\\
508.01	0.00382343127861638\\
509.01	0.00387837187091133\\
510.01	0.00393470653278658\\
511.01	0.00399241970744769\\
512.01	0.00405147988646246\\
513.01	0.00411183496672934\\
514.01	0.0041734062597363\\
515.01	0.00423608073040822\\
516.01	0.00429970090066552\\
517.01	0.00436405165869855\\
518.01	0.00442884288366734\\
519.01	0.00449368678115327\\
520.01	0.00455806900599307\\
521.01	0.00462131128482143\\
522.01	0.0046825227041285\\
523.01	0.00474053607862413\\
524.01	0.00479468739552544\\
525.01	0.00484775780185083\\
526.01	0.00490099915561815\\
527.01	0.00495569460839533\\
528.01	0.005011842622378\\
529.01	0.00506943446390541\\
530.01	0.00512845593205903\\
531.01	0.00518889007741343\\
532.01	0.00525072401889197\\
533.01	0.00531396417467579\\
534.01	0.00537862213181257\\
535.01	0.00544470528652228\\
536.01	0.00551221652046107\\
537.01	0.00558115353795019\\
538.01	0.00565150795059695\\
539.01	0.00572326408073132\\
540.01	0.00579639733595701\\
541.01	0.0058708719582017\\
542.01	0.00594663896412132\\
543.01	0.006023640817064\\
544.01	0.00610169629838397\\
545.01	0.0061799446459423\\
546.01	0.00625806301414588\\
547.01	0.0063358244001688\\
548.01	0.00641299880601097\\
549.01	0.00648937082796436\\
550.01	0.00656473392672981\\
551.01	0.00663879945807873\\
552.01	0.00671127018401746\\
553.01	0.00678189713821906\\
554.01	0.00685053104426765\\
555.01	0.00691756867300738\\
556.01	0.00698399436934487\\
557.01	0.00704980690070247\\
558.01	0.0071149853986964\\
559.01	0.00717953865275502\\
560.01	0.00724349912485966\\
561.01	0.00730692267095094\\
562.01	0.00736991220277727\\
563.01	0.00743262416484024\\
564.01	0.00749525849673979\\
565.01	0.00755799012829877\\
566.01	0.00762087397561231\\
567.01	0.00768395414595888\\
568.01	0.00774728354746372\\
569.01	0.00781092427126301\\
570.01	0.0078749452850801\\
571.01	0.00793941585126919\\
572.01	0.00800439739752404\\
573.01	0.00806993562399001\\
574.01	0.00813605795685841\\
575.01	0.00820278543061315\\
576.01	0.00827013997487067\\
577.01	0.00833814309414947\\
578.01	0.0084068139649348\\
579.01	0.00847616764322697\\
580.01	0.00854621376178146\\
581.01	0.00861695616792148\\
582.01	0.00868839387545263\\
583.01	0.00876052282070629\\
584.01	0.00883333648987411\\
585.01	0.0089068255990963\\
586.01	0.00898097767697374\\
587.01	0.00905577659279556\\
588.01	0.00913120213809157\\
589.01	0.00920722970013294\\
590.01	0.00928382994806625\\
591.01	0.00936096842126078\\
592.01	0.00943860510469645\\
593.01	0.00951669429991475\\
594.01	0.00959518502752448\\
595.01	0.00967402215932641\\
596.01	0.00975314847738075\\
597.01	0.00983250783722213\\
598.01	0.00990866197942267\\
599.01	0.00997087280370268\\
599.02	0.00997138072120452\\
599.03	0.00997188557585126\\
599.04	0.00997238733782006\\
599.05	0.0099728859769941\\
599.06	0.00997338146295965\\
599.07	0.00997387376500313\\
599.08	0.00997436285210814\\
599.09	0.00997484869295248\\
599.1	0.00997533125590507\\
599.11	0.00997581050902294\\
599.12	0.00997628642004811\\
599.13	0.00997675895640448\\
599.14	0.00997722808519467\\
599.15	0.00997769377319684\\
599.16	0.00997815598686149\\
599.17	0.00997861469230817\\
599.18	0.00997906985532226\\
599.19	0.0099795214413516\\
599.2	0.0099799694155032\\
599.21	0.0099804137425398\\
599.22	0.00998085438687651\\
599.23	0.00998129131257736\\
599.24	0.00998172448335179\\
599.25	0.00998215386255116\\
599.26	0.00998257941250506\\
599.27	0.00998300109272606\\
599.28	0.00998341886232347\\
599.29	0.00998383267999937\\
599.3	0.00998424250404453\\
599.31	0.00998464829233438\\
599.32	0.00998505000232485\\
599.33	0.00998544759104829\\
599.34	0.00998584101510919\\
599.35	0.00998623023068002\\
599.36	0.00998661519349689\\
599.37	0.00998699585885528\\
599.38	0.00998737218160564\\
599.39	0.00998774411614902\\
599.4	0.00998811161643258\\
599.41	0.00998847463594512\\
599.42	0.00998883312771255\\
599.43	0.00998918704429328\\
599.44	0.00998953633777364\\
599.45	0.00998988095976314\\
599.46	0.00999022086138981\\
599.47	0.00999055599329538\\
599.48	0.00999088630563053\\
599.49	0.00999121174804997\\
599.5	0.00999153226970755\\
599.51	0.00999184781925132\\
599.52	0.0099921583448185\\
599.53	0.00999246379403045\\
599.54	0.00999276411398751\\
599.55	0.00999305925126392\\
599.56	0.00999334915190253\\
599.57	0.00999363376140962\\
599.58	0.00999391302474952\\
599.59	0.00999418688633929\\
599.6	0.00999445529004328\\
599.61	0.00999471817916768\\
599.62	0.00999497549645498\\
599.63	0.00999522718407838\\
599.64	0.00999547318363619\\
599.65	0.00999571343614611\\
599.66	0.00999594788203949\\
599.67	0.00999617646115553\\
599.68	0.00999639911273543\\
599.69	0.00999661577541645\\
599.7	0.00999682638722594\\
599.71	0.00999703088557532\\
599.72	0.00999722920725396\\
599.73	0.00999742128842304\\
599.74	0.00999760706460933\\
599.75	0.0099977864706989\\
599.76	0.00999795944093081\\
599.77	0.00999812590889066\\
599.78	0.00999828580750417\\
599.79	0.00999843906903062\\
599.8	0.00999858562505626\\
599.81	0.00999872540648766\\
599.82	0.00999885834354497\\
599.83	0.00999898436575515\\
599.84	0.00999910340194508\\
599.85	0.00999921538023465\\
599.86	0.00999932022802977\\
599.87	0.00999941787201528\\
599.88	0.00999950823814785\\
599.89	0.00999959125164875\\
599.9	0.00999966683699656\\
599.91	0.00999973491791987\\
599.92	0.0099997954173898\\
599.93	0.00999984825761255\\
599.94	0.00999989336002181\\
599.95	0.00999993064527112\\
599.96	0.00999996003322615\\
599.97	0.00999998144295691\\
599.98	0.00999999479272987\\
599.99	0.01\\
600	0.01\\
};
\addplot [color=blue,solid,forget plot]
  table[row sep=crcr]{%
0.01	0\\
1.01	0\\
2.01	0\\
3.01	0\\
4.01	0\\
5.01	0\\
6.01	0\\
7.01	0\\
8.01	0\\
9.01	0\\
10.01	0\\
11.01	0\\
12.01	0\\
13.01	0\\
14.01	0\\
15.01	0\\
16.01	0\\
17.01	0\\
18.01	0\\
19.01	0\\
20.01	0\\
21.01	0\\
22.01	0\\
23.01	0\\
24.01	0\\
25.01	0\\
26.01	0\\
27.01	0\\
28.01	0\\
29.01	0\\
30.01	0\\
31.01	0\\
32.01	0\\
33.01	0\\
34.01	0\\
35.01	0\\
36.01	0\\
37.01	0\\
38.01	0\\
39.01	0\\
40.01	0\\
41.01	0\\
42.01	0\\
43.01	0\\
44.01	0\\
45.01	0\\
46.01	0\\
47.01	0\\
48.01	0\\
49.01	0\\
50.01	0\\
51.01	0\\
52.01	0\\
53.01	0\\
54.01	0\\
55.01	0\\
56.01	0\\
57.01	0\\
58.01	0\\
59.01	0\\
60.01	0\\
61.01	0\\
62.01	0\\
63.01	0\\
64.01	0\\
65.01	0\\
66.01	0\\
67.01	0\\
68.01	0\\
69.01	0\\
70.01	0\\
71.01	0\\
72.01	0\\
73.01	0\\
74.01	0\\
75.01	0\\
76.01	0\\
77.01	0\\
78.01	0\\
79.01	0\\
80.01	0\\
81.01	0\\
82.01	0\\
83.01	0\\
84.01	0\\
85.01	0\\
86.01	0\\
87.01	0\\
88.01	0\\
89.01	0\\
90.01	0\\
91.01	0\\
92.01	0\\
93.01	0\\
94.01	0\\
95.01	0\\
96.01	0\\
97.01	0\\
98.01	0\\
99.01	0\\
100.01	0\\
101.01	0\\
102.01	0\\
103.01	0\\
104.01	0\\
105.01	0\\
106.01	0\\
107.01	0\\
108.01	0\\
109.01	0\\
110.01	0\\
111.01	0\\
112.01	0\\
113.01	0\\
114.01	0\\
115.01	0\\
116.01	0\\
117.01	0\\
118.01	0\\
119.01	0\\
120.01	0\\
121.01	0\\
122.01	0\\
123.01	0\\
124.01	0\\
125.01	0\\
126.01	0\\
127.01	0\\
128.01	0\\
129.01	0\\
130.01	0\\
131.01	0\\
132.01	0\\
133.01	0\\
134.01	0\\
135.01	0\\
136.01	0\\
137.01	0\\
138.01	0\\
139.01	0\\
140.01	0\\
141.01	0\\
142.01	0\\
143.01	0\\
144.01	0\\
145.01	0\\
146.01	0\\
147.01	0\\
148.01	0\\
149.01	0\\
150.01	0\\
151.01	0\\
152.01	0\\
153.01	0\\
154.01	0\\
155.01	0\\
156.01	0\\
157.01	0\\
158.01	0\\
159.01	0\\
160.01	0\\
161.01	0\\
162.01	0\\
163.01	0\\
164.01	0\\
165.01	0\\
166.01	0\\
167.01	0\\
168.01	0\\
169.01	0\\
170.01	0\\
171.01	0\\
172.01	0\\
173.01	0\\
174.01	0\\
175.01	0\\
176.01	0\\
177.01	0\\
178.01	0\\
179.01	0\\
180.01	0\\
181.01	0\\
182.01	0\\
183.01	0\\
184.01	0\\
185.01	0\\
186.01	0\\
187.01	0\\
188.01	0\\
189.01	0\\
190.01	0\\
191.01	0\\
192.01	0\\
193.01	0\\
194.01	0\\
195.01	0\\
196.01	0\\
197.01	0\\
198.01	0\\
199.01	0\\
200.01	0\\
201.01	0\\
202.01	0\\
203.01	0\\
204.01	0\\
205.01	0\\
206.01	0\\
207.01	0\\
208.01	0\\
209.01	0\\
210.01	0\\
211.01	0\\
212.01	0\\
213.01	0\\
214.01	0\\
215.01	0\\
216.01	0\\
217.01	0\\
218.01	0\\
219.01	0\\
220.01	0\\
221.01	0\\
222.01	0\\
223.01	0\\
224.01	0\\
225.01	0\\
226.01	0\\
227.01	0\\
228.01	0\\
229.01	0\\
230.01	0\\
231.01	0\\
232.01	0\\
233.01	0\\
234.01	0\\
235.01	0\\
236.01	0\\
237.01	0\\
238.01	0\\
239.01	0\\
240.01	0\\
241.01	0\\
242.01	0\\
243.01	0\\
244.01	0\\
245.01	0\\
246.01	0\\
247.01	0\\
248.01	0\\
249.01	0\\
250.01	0\\
251.01	0\\
252.01	0\\
253.01	0\\
254.01	0\\
255.01	0\\
256.01	0\\
257.01	0\\
258.01	0\\
259.01	0\\
260.01	0\\
261.01	0\\
262.01	0\\
263.01	0\\
264.01	0\\
265.01	0\\
266.01	0\\
267.01	0\\
268.01	0\\
269.01	0\\
270.01	0\\
271.01	0\\
272.01	0\\
273.01	0\\
274.01	0\\
275.01	0\\
276.01	0\\
277.01	0\\
278.01	0\\
279.01	0\\
280.01	0\\
281.01	0\\
282.01	0\\
283.01	0\\
284.01	0\\
285.01	0\\
286.01	0\\
287.01	0\\
288.01	0\\
289.01	0\\
290.01	0\\
291.01	0\\
292.01	0\\
293.01	0\\
294.01	0\\
295.01	0\\
296.01	0\\
297.01	0\\
298.01	0\\
299.01	0\\
300.01	0\\
301.01	0\\
302.01	0\\
303.01	0\\
304.01	0\\
305.01	0\\
306.01	0\\
307.01	0\\
308.01	0\\
309.01	0\\
310.01	0\\
311.01	0\\
312.01	0\\
313.01	0\\
314.01	0\\
315.01	0\\
316.01	0\\
317.01	0\\
318.01	0\\
319.01	0\\
320.01	0\\
321.01	0\\
322.01	0\\
323.01	0\\
324.01	0\\
325.01	0\\
326.01	0\\
327.01	0\\
328.01	0\\
329.01	0\\
330.01	0\\
331.01	0\\
332.01	0\\
333.01	0\\
334.01	0\\
335.01	0\\
336.01	0\\
337.01	0\\
338.01	0\\
339.01	0\\
340.01	0\\
341.01	0\\
342.01	0\\
343.01	0\\
344.01	0\\
345.01	0\\
346.01	0\\
347.01	0\\
348.01	0\\
349.01	0\\
350.01	0\\
351.01	0\\
352.01	0\\
353.01	0\\
354.01	0\\
355.01	0\\
356.01	0\\
357.01	0\\
358.01	0\\
359.01	0\\
360.01	0\\
361.01	0\\
362.01	0\\
363.01	0\\
364.01	0\\
365.01	0\\
366.01	0\\
367.01	0\\
368.01	0\\
369.01	0\\
370.01	0\\
371.01	0\\
372.01	0\\
373.01	0\\
374.01	0\\
375.01	0\\
376.01	0\\
377.01	0\\
378.01	0\\
379.01	0\\
380.01	0\\
381.01	0\\
382.01	0\\
383.01	0\\
384.01	0\\
385.01	0\\
386.01	0\\
387.01	0\\
388.01	0\\
389.01	0\\
390.01	0\\
391.01	0\\
392.01	0\\
393.01	0\\
394.01	0\\
395.01	0\\
396.01	0\\
397.01	0\\
398.01	0\\
399.01	0\\
400.01	0\\
401.01	0\\
402.01	0\\
403.01	0\\
404.01	0\\
405.01	0\\
406.01	0\\
407.01	0\\
408.01	0\\
409.01	0\\
410.01	0\\
411.01	0\\
412.01	0\\
413.01	0\\
414.01	0\\
415.01	0\\
416.01	0\\
417.01	0\\
418.01	0\\
419.01	0\\
420.01	0\\
421.01	0\\
422.01	0\\
423.01	0\\
424.01	0\\
425.01	0\\
426.01	0\\
427.01	0\\
428.01	0\\
429.01	0\\
430.01	0\\
431.01	0\\
432.01	0\\
433.01	0\\
434.01	0\\
435.01	0\\
436.01	0\\
437.01	0\\
438.01	0\\
439.01	0\\
440.01	0\\
441.01	0\\
442.01	0\\
443.01	0\\
444.01	0\\
445.01	0\\
446.01	0\\
447.01	0\\
448.01	0\\
449.01	0\\
450.01	0\\
451.01	0\\
452.01	0\\
453.01	0\\
454.01	0\\
455.01	0\\
456.01	0\\
457.01	0\\
458.01	0\\
459.01	0\\
460.01	0\\
461.01	0\\
462.01	0\\
463.01	0\\
464.01	0\\
465.01	0\\
466.01	0\\
467.01	0\\
468.01	0\\
469.01	0\\
470.01	0\\
471.01	0\\
472.01	0\\
473.01	0\\
474.01	0\\
475.01	0\\
476.01	0\\
477.01	0\\
478.01	0\\
479.01	0\\
480.01	0\\
481.01	2.40966981226928e-05\\
482.01	6.02289196438011e-05\\
483.01	9.7493952585008e-05\\
484.01	0.000135932639077595\\
485.01	0.000175584678312843\\
486.01	0.000216487278826116\\
487.01	0.000258673283014495\\
488.01	0.000302168582289146\\
489.01	0.000346988580984628\\
490.01	0.000393133387252401\\
491.01	0.000440595944165442\\
492.01	0.000489414579453555\\
493.01	0.000539644681450131\\
494.01	0.000591344781451793\\
495.01	0.000644576616767242\\
496.01	0.000699405125739842\\
497.01	0.000755898342664832\\
498.01	0.000814127149159516\\
499.01	0.000874164822961092\\
500.01	0.000936086304175951\\
501.01	0.000999967071387352\\
502.01	0.00106588148366777\\
503.01	0.00113390039593665\\
504.01	0.00120408779034123\\
505.01	0.00127650080732459\\
506.01	0.00135123384775706\\
507.01	0.0014284158805291\\
508.01	0.00150818998615476\\
509.01	0.00159071524288537\\
510.01	0.00167616935849231\\
511.01	0.00176475183676328\\
512.01	0.00185668780418655\\
513.01	0.00195223265526322\\
514.01	0.00205167771738155\\
515.01	0.00215535719120302\\
516.01	0.00226365669406223\\
517.01	0.00237702382739495\\
518.01	0.00249598131293208\\
519.01	0.00262114340450183\\
520.01	0.00275323647292278\\
521.01	0.00289312492573644\\
522.01	0.00304184399238899\\
523.01	0.00320064139004236\\
524.01	0.00337017847018786\\
525.01	0.0035477461113088\\
526.01	0.0036295125370641\\
527.01	0.0037108927160592\\
528.01	0.00379456895392187\\
529.01	0.00388056948962633\\
530.01	0.00396891275840736\\
531.01	0.0040596039183237\\
532.01	0.00415263008682738\\
533.01	0.00424795372319946\\
534.01	0.00434550378487202\\
535.01	0.00444515876139365\\
536.01	0.0045467365748054\\
537.01	0.0046499857123393\\
538.01	0.00475456809396017\\
539.01	0.00486003862294787\\
540.01	0.00496581932279654\\
541.01	0.00507116688167566\\
542.01	0.00517506758599043\\
543.01	0.00527604888611295\\
544.01	0.00537221070707498\\
545.01	0.00546672253319988\\
546.01	0.00556220625235845\\
547.01	0.0056580924038923\\
548.01	0.0057535950559224\\
549.01	0.00584763774075489\\
550.01	0.00594170309216324\\
551.01	0.00603796482129104\\
552.01	0.00613623750855436\\
553.01	0.00623623587041443\\
554.01	0.00633753842167594\\
555.01	0.00643917690572694\\
556.01	0.00653969205861012\\
557.01	0.00663870292660752\\
558.01	0.00673595224939041\\
559.01	0.00683132728671051\\
560.01	0.00692492472489956\\
561.01	0.00701661563739462\\
562.01	0.00710600980656644\\
563.01	0.00719278358715181\\
564.01	0.00727677208888639\\
565.01	0.00735899562298691\\
566.01	0.00744011141004657\\
567.01	0.00752017065907459\\
568.01	0.00759925750663107\\
569.01	0.00767738576415853\\
570.01	0.00775462864345971\\
571.01	0.00783113453106925\\
572.01	0.00790711791574717\\
573.01	0.00798282551971004\\
574.01	0.00805842551950888\\
575.01	0.00813393580891355\\
576.01	0.00820937321546196\\
577.01	0.00828477503550238\\
578.01	0.00836019599880814\\
579.01	0.00843570205622002\\
580.01	0.00851136051977954\\
581.01	0.0085872267444925\\
582.01	0.00866332969700664\\
583.01	0.00873967262081048\\
584.01	0.00881625188759261\\
585.01	0.00889306268112347\\
586.01	0.00897010042272349\\
587.01	0.00904736164167313\\
588.01	0.00912484327393588\\
589.01	0.00920254246997357\\
590.01	0.00928045754230262\\
591.01	0.00935858943802847\\
592.01	0.00943694120142937\\
593.01	0.00951551569543855\\
594.01	0.00959431296143754\\
595.01	0.00967332770823237\\
596.01	0.00975254751940146\\
597.01	0.00983195256120927\\
598.01	0.00990865761438377\\
599.01	0.00997087276439033\\
599.02	0.009971380683866\\
599.03	0.00997188554040684\\
599.04	0.00997238730419245\\
599.05	0.00997288594510838\\
599.06	0.00997338143274323\\
599.07	0.00997387373638571\\
599.08	0.00997436282502164\\
599.09	0.00997484866733099\\
599.1	0.00997533123168485\\
599.11	0.0099758104861423\\
599.12	0.00997628639844743\\
599.13	0.00997675893602614\\
599.14	0.00997722806598298\\
599.15	0.00997769375509804\\
599.16	0.00997815596982367\\
599.17	0.00997861467628126\\
599.18	0.00997906984025795\\
599.19	0.00997952142720333\\
599.2	0.00997996940222608\\
599.21	0.00998041373009062\\
599.22	0.00998085437521367\\
599.23	0.00998129130166082\\
599.24	0.00998172447314304\\
599.25	0.0099821538530132\\
599.26	0.00998257940360238\\
599.27	0.00998300108442457\\
599.28	0.00998341885459047\\
599.29	0.00998383267280348\\
599.3	0.00998424249735567\\
599.31	0.00998464828612373\\
599.32	0.00998504999656486\\
599.33	0.00998544758571256\\
599.34	0.00998584101017251\\
599.35	0.00998623022611829\\
599.36	0.00998661518928712\\
599.37	0.00998699585497554\\
599.38	0.00998737217803502\\
599.39	0.0099877441128676\\
599.4	0.00998811161342143\\
599.41	0.00998847463318623\\
599.42	0.00998883312518882\\
599.43	0.00998918704198849\\
599.44	0.00998953633567239\\
599.45	0.00998988095785088\\
599.46	0.00999022085965275\\
599.47	0.0099905559917205\\
599.48	0.00999088630420554\\
599.49	0.00999121174676329\\
599.5	0.00999153226854827\\
599.51	0.00999184781820919\\
599.52	0.00999215834388389\\
599.53	0.00999246379319433\\
599.54	0.00999276411324143\\
599.55	0.00999305925059996\\
599.56	0.00999334915131334\\
599.57	0.00999363376088831\\
599.58	0.0099939130242897\\
599.59	0.00999418688593504\\
599.6	0.0099944552896891\\
599.61	0.00999471817885849\\
599.62	0.00999497549618609\\
599.63	0.00999522718384549\\
599.64	0.00999547318343534\\
599.65	0.00999571343597366\\
599.66	0.00999594788189215\\
599.67	0.00999617646103028\\
599.68	0.00999639911262954\\
599.69	0.00999661577532744\\
599.7	0.00999682638715159\\
599.71	0.00999703088551362\\
599.72	0.00999722920720313\\
599.73	0.00999742128838149\\
599.74	0.00999760706457565\\
599.75	0.00999778647067185\\
599.76	0.00999795944090929\\
599.77	0.00999812590887373\\
599.78	0.00999828580749101\\
599.79	0.00999843906902052\\
599.8	0.00999858562504862\\
599.81	0.00999872540648197\\
599.82	0.00999885834354081\\
599.83	0.00999898436575217\\
599.84	0.009999103401943\\
599.85	0.00999921538023323\\
599.86	0.00999932022802883\\
599.87	0.00999941787201469\\
599.88	0.00999950823814749\\
599.89	0.00999959125164854\\
599.9	0.00999966683699645\\
599.91	0.00999973491791981\\
599.92	0.00999979541738977\\
599.93	0.00999984825761254\\
599.94	0.00999989336002181\\
599.95	0.00999993064527112\\
599.96	0.00999996003322615\\
599.97	0.00999998144295691\\
599.98	0.00999999479272987\\
599.99	0.01\\
600	0.01\\
};
\addplot [color=mycolor10,solid,forget plot]
  table[row sep=crcr]{%
0.01	0\\
1.01	0\\
2.01	0\\
3.01	0\\
4.01	0\\
5.01	0\\
6.01	0\\
7.01	0\\
8.01	0\\
9.01	0\\
10.01	0\\
11.01	0\\
12.01	0\\
13.01	0\\
14.01	0\\
15.01	0\\
16.01	0\\
17.01	0\\
18.01	0\\
19.01	0\\
20.01	0\\
21.01	0\\
22.01	0\\
23.01	0\\
24.01	0\\
25.01	0\\
26.01	0\\
27.01	0\\
28.01	0\\
29.01	0\\
30.01	0\\
31.01	0\\
32.01	0\\
33.01	0\\
34.01	0\\
35.01	0\\
36.01	0\\
37.01	0\\
38.01	0\\
39.01	0\\
40.01	0\\
41.01	0\\
42.01	0\\
43.01	0\\
44.01	0\\
45.01	0\\
46.01	0\\
47.01	0\\
48.01	0\\
49.01	0\\
50.01	0\\
51.01	0\\
52.01	0\\
53.01	0\\
54.01	0\\
55.01	0\\
56.01	0\\
57.01	0\\
58.01	0\\
59.01	0\\
60.01	0\\
61.01	0\\
62.01	0\\
63.01	0\\
64.01	0\\
65.01	0\\
66.01	0\\
67.01	0\\
68.01	0\\
69.01	0\\
70.01	0\\
71.01	0\\
72.01	0\\
73.01	0\\
74.01	0\\
75.01	0\\
76.01	0\\
77.01	0\\
78.01	0\\
79.01	0\\
80.01	0\\
81.01	0\\
82.01	0\\
83.01	0\\
84.01	0\\
85.01	0\\
86.01	0\\
87.01	0\\
88.01	0\\
89.01	0\\
90.01	0\\
91.01	0\\
92.01	0\\
93.01	0\\
94.01	0\\
95.01	0\\
96.01	0\\
97.01	0\\
98.01	0\\
99.01	0\\
100.01	0\\
101.01	0\\
102.01	0\\
103.01	0\\
104.01	0\\
105.01	0\\
106.01	0\\
107.01	0\\
108.01	0\\
109.01	0\\
110.01	0\\
111.01	0\\
112.01	0\\
113.01	0\\
114.01	0\\
115.01	0\\
116.01	0\\
117.01	0\\
118.01	0\\
119.01	0\\
120.01	0\\
121.01	0\\
122.01	0\\
123.01	0\\
124.01	0\\
125.01	0\\
126.01	0\\
127.01	0\\
128.01	0\\
129.01	0\\
130.01	0\\
131.01	0\\
132.01	0\\
133.01	0\\
134.01	0\\
135.01	0\\
136.01	0\\
137.01	0\\
138.01	0\\
139.01	0\\
140.01	0\\
141.01	0\\
142.01	0\\
143.01	0\\
144.01	0\\
145.01	0\\
146.01	0\\
147.01	0\\
148.01	0\\
149.01	0\\
150.01	0\\
151.01	0\\
152.01	0\\
153.01	0\\
154.01	0\\
155.01	0\\
156.01	0\\
157.01	0\\
158.01	0\\
159.01	0\\
160.01	0\\
161.01	0\\
162.01	0\\
163.01	0\\
164.01	0\\
165.01	0\\
166.01	0\\
167.01	0\\
168.01	0\\
169.01	0\\
170.01	0\\
171.01	0\\
172.01	0\\
173.01	0\\
174.01	0\\
175.01	0\\
176.01	0\\
177.01	0\\
178.01	0\\
179.01	0\\
180.01	0\\
181.01	0\\
182.01	0\\
183.01	0\\
184.01	0\\
185.01	0\\
186.01	0\\
187.01	0\\
188.01	0\\
189.01	0\\
190.01	0\\
191.01	0\\
192.01	0\\
193.01	0\\
194.01	0\\
195.01	0\\
196.01	0\\
197.01	0\\
198.01	0\\
199.01	0\\
200.01	0\\
201.01	0\\
202.01	0\\
203.01	0\\
204.01	0\\
205.01	0\\
206.01	0\\
207.01	0\\
208.01	0\\
209.01	0\\
210.01	0\\
211.01	0\\
212.01	0\\
213.01	0\\
214.01	0\\
215.01	0\\
216.01	0\\
217.01	0\\
218.01	0\\
219.01	0\\
220.01	0\\
221.01	0\\
222.01	0\\
223.01	0\\
224.01	0\\
225.01	0\\
226.01	0\\
227.01	0\\
228.01	0\\
229.01	0\\
230.01	0\\
231.01	0\\
232.01	0\\
233.01	0\\
234.01	0\\
235.01	0\\
236.01	0\\
237.01	0\\
238.01	0\\
239.01	0\\
240.01	0\\
241.01	0\\
242.01	0\\
243.01	0\\
244.01	0\\
245.01	0\\
246.01	0\\
247.01	0\\
248.01	0\\
249.01	0\\
250.01	0\\
251.01	0\\
252.01	0\\
253.01	0\\
254.01	0\\
255.01	0\\
256.01	0\\
257.01	0\\
258.01	0\\
259.01	0\\
260.01	0\\
261.01	0\\
262.01	0\\
263.01	0\\
264.01	0\\
265.01	0\\
266.01	0\\
267.01	0\\
268.01	0\\
269.01	0\\
270.01	0\\
271.01	0\\
272.01	0\\
273.01	0\\
274.01	0\\
275.01	0\\
276.01	0\\
277.01	0\\
278.01	0\\
279.01	0\\
280.01	0\\
281.01	0\\
282.01	0\\
283.01	0\\
284.01	0\\
285.01	0\\
286.01	0\\
287.01	0\\
288.01	0\\
289.01	0\\
290.01	0\\
291.01	0\\
292.01	0\\
293.01	0\\
294.01	0\\
295.01	0\\
296.01	0\\
297.01	0\\
298.01	0\\
299.01	0\\
300.01	0\\
301.01	0\\
302.01	0\\
303.01	0\\
304.01	0\\
305.01	0\\
306.01	0\\
307.01	0\\
308.01	0\\
309.01	0\\
310.01	0\\
311.01	0\\
312.01	0\\
313.01	0\\
314.01	0\\
315.01	0\\
316.01	0\\
317.01	0\\
318.01	0\\
319.01	0\\
320.01	0\\
321.01	0\\
322.01	0\\
323.01	0\\
324.01	0\\
325.01	0\\
326.01	0\\
327.01	0\\
328.01	0\\
329.01	0\\
330.01	0\\
331.01	0\\
332.01	0\\
333.01	0\\
334.01	0\\
335.01	0\\
336.01	0\\
337.01	0\\
338.01	0\\
339.01	0\\
340.01	0\\
341.01	0\\
342.01	0\\
343.01	0\\
344.01	0\\
345.01	0\\
346.01	0\\
347.01	0\\
348.01	0\\
349.01	0\\
350.01	0\\
351.01	0\\
352.01	0\\
353.01	0\\
354.01	0\\
355.01	0\\
356.01	0\\
357.01	0\\
358.01	0\\
359.01	0\\
360.01	0\\
361.01	0\\
362.01	0\\
363.01	0\\
364.01	0\\
365.01	0\\
366.01	0\\
367.01	0\\
368.01	0\\
369.01	0\\
370.01	0\\
371.01	0\\
372.01	0\\
373.01	0\\
374.01	0\\
375.01	0\\
376.01	0\\
377.01	0\\
378.01	0\\
379.01	0\\
380.01	0\\
381.01	0\\
382.01	0\\
383.01	0\\
384.01	0\\
385.01	0\\
386.01	0\\
387.01	0\\
388.01	0\\
389.01	0\\
390.01	0\\
391.01	0\\
392.01	0\\
393.01	0\\
394.01	0\\
395.01	0\\
396.01	0\\
397.01	0\\
398.01	0\\
399.01	0\\
400.01	0\\
401.01	0\\
402.01	0\\
403.01	0\\
404.01	0\\
405.01	0\\
406.01	0\\
407.01	0\\
408.01	0\\
409.01	0\\
410.01	0\\
411.01	0\\
412.01	0\\
413.01	0\\
414.01	0\\
415.01	0\\
416.01	0\\
417.01	0\\
418.01	0\\
419.01	0\\
420.01	0\\
421.01	0\\
422.01	0\\
423.01	0\\
424.01	0\\
425.01	0\\
426.01	0\\
427.01	0\\
428.01	0\\
429.01	0\\
430.01	0\\
431.01	0\\
432.01	0\\
433.01	0\\
434.01	0\\
435.01	0\\
436.01	0\\
437.01	0\\
438.01	0\\
439.01	0\\
440.01	0\\
441.01	0\\
442.01	0\\
443.01	0\\
444.01	0\\
445.01	0\\
446.01	0\\
447.01	0\\
448.01	0\\
449.01	0\\
450.01	0\\
451.01	0\\
452.01	0\\
453.01	0\\
454.01	0\\
455.01	0\\
456.01	0\\
457.01	0\\
458.01	0\\
459.01	0\\
460.01	0\\
461.01	0\\
462.01	0\\
463.01	0\\
464.01	0\\
465.01	0\\
466.01	0\\
467.01	0\\
468.01	0\\
469.01	0\\
470.01	0\\
471.01	0\\
472.01	0\\
473.01	0\\
474.01	0\\
475.01	0\\
476.01	0\\
477.01	0\\
478.01	0\\
479.01	0\\
480.01	0\\
481.01	0\\
482.01	0\\
483.01	0\\
484.01	0\\
485.01	0\\
486.01	0\\
487.01	0\\
488.01	0\\
489.01	0\\
490.01	0\\
491.01	0\\
492.01	0\\
493.01	0\\
494.01	0\\
495.01	0\\
496.01	0\\
497.01	0\\
498.01	0\\
499.01	0\\
500.01	0\\
501.01	0\\
502.01	0\\
503.01	0\\
504.01	0\\
505.01	0\\
506.01	0\\
507.01	0\\
508.01	0\\
509.01	0\\
510.01	0\\
511.01	0\\
512.01	0\\
513.01	0\\
514.01	0\\
515.01	0\\
516.01	0\\
517.01	0\\
518.01	0\\
519.01	0\\
520.01	0\\
521.01	0\\
522.01	0\\
523.01	0\\
524.01	0\\
525.01	0\\
526.01	0.000102771456501963\\
527.01	0.000211857262483357\\
528.01	0.000324838345335268\\
529.01	0.000441966255021686\\
530.01	0.000563519085941433\\
531.01	0.000689805325113445\\
532.01	0.000821168123154713\\
533.01	0.000957989226061683\\
534.01	0.00110069167556698\\
535.01	0.00124976707994646\\
536.01	0.00140579226223662\\
537.01	0.00156943036835827\\
538.01	0.00174142679592595\\
539.01	0.0019226250528362\\
540.01	0.00211398992820282\\
541.01	0.00231662768418949\\
542.01	0.00253187134003461\\
543.01	0.00276144006226553\\
544.01	0.00300736772181577\\
545.01	0.00326661567286447\\
546.01	0.00353699838186093\\
547.01	0.00381957633107237\\
548.01	0.00411560683043353\\
549.01	0.00442528142982412\\
550.01	0.00456342194511371\\
551.01	0.00470501977180571\\
552.01	0.00484979652204437\\
553.01	0.00499735440866724\\
554.01	0.00514714434222182\\
555.01	0.00529842819406245\\
556.01	0.00545023857511322\\
557.01	0.00560127475048913\\
558.01	0.00574977686249003\\
559.01	0.00589337759184622\\
560.01	0.00603144302138599\\
561.01	0.00617116532806554\\
562.01	0.00631279169818998\\
563.01	0.00645550940284293\\
564.01	0.00659814217849057\\
565.01	0.00673791067745923\\
566.01	0.0068719890954267\\
567.01	0.00699830368032526\\
568.01	0.00712095802857818\\
569.01	0.00724086556742704\\
570.01	0.00735745019262945\\
571.01	0.00747029237527268\\
572.01	0.00757928265660215\\
573.01	0.00768484450268323\\
574.01	0.00778859580472871\\
575.01	0.0078909719541174\\
576.01	0.00799171655542844\\
577.01	0.00809063074719604\\
578.01	0.00818760060298669\\
579.01	0.0082826282025498\\
580.01	0.00837586389043174\\
581.01	0.00846763456953067\\
582.01	0.00855842645664378\\
583.01	0.0086485452226052\\
584.01	0.0087380156008509\\
585.01	0.00882682224290697\\
586.01	0.00891491374940033\\
587.01	0.00900224063450917\\
588.01	0.00908876407379003\\
589.01	0.0091744495113924\\
590.01	0.00925926074743837\\
591.01	0.00934317099294246\\
592.01	0.0094261874329281\\
593.01	0.00950835739313356\\
594.01	0.0095897680639356\\
595.01	0.00967054251212487\\
596.01	0.00975083116190794\\
597.01	0.00983079836948479\\
598.01	0.00990857284332932\\
599.01	0.00997086936970872\\
599.02	0.0099713774254769\\
599.03	0.00997188241419166\\
599.04	0.00997238430611518\\
599.05	0.00997288307121482\\
599.06	0.00997337867916017\\
599.07	0.00997387109932015\\
599.08	0.00997436030076002\\
599.09	0.00997484625223838\\
599.1	0.00997532892220413\\
599.11	0.0099758082787934\\
599.12	0.00997628428982652\\
599.13	0.00997675692280479\\
599.14	0.00997722614490744\\
599.15	0.00997769192298835\\
599.16	0.0099781542235729\\
599.17	0.0099786130128547\\
599.18	0.00997906825669229\\
599.19	0.00997951992060585\\
599.2	0.00997996796977384\\
599.21	0.00998041236902966\\
599.22	0.00998085308285816\\
599.23	0.00998129007539227\\
599.24	0.00998172331040948\\
599.25	0.00998215275132834\\
599.26	0.00998257836054724\\
599.27	0.00998300009764374\\
599.28	0.00998341792179137\\
599.29	0.00998383179175552\\
599.3	0.00998424166588949\\
599.31	0.00998464750213039\\
599.32	0.00998504925799502\\
599.33	0.00998544689057571\\
599.34	0.00998584035653615\\
599.35	0.00998622961210713\\
599.36	0.00998661461308229\\
599.37	0.00998699531481375\\
599.38	0.00998737167220779\\
599.39	0.00998774363972043\\
599.4	0.00998811117135299\\
599.41	0.00998847422064756\\
599.42	0.0099888327406825\\
599.43	0.00998918668406783\\
599.44	0.00998953600294064\\
599.45	0.00998988064896035\\
599.46	0.00999022057330405\\
599.47	0.0099905557266617\\
599.48	0.00999088605923132\\
599.49	0.00999121152071412\\
599.5	0.0099915320603096\\
599.51	0.00999184762671058\\
599.52	0.0099921581680982\\
599.53	0.00999246363213685\\
599.54	0.00999276396596905\\
599.55	0.00999305911621034\\
599.56	0.00999334902894398\\
599.57	0.0099936336497158\\
599.58	0.00999391292352879\\
599.59	0.00999418679483778\\
599.6	0.009994455207544\\
599.61	0.00999471810498963\\
599.62	0.00999497542995225\\
599.63	0.00999522712463928\\
599.64	0.00999547313068229\\
599.65	0.00999571338913138\\
599.66	0.00999594784044938\\
599.67	0.00999617642450606\\
599.68	0.00999639908057226\\
599.69	0.00999661574731397\\
599.7	0.00999682636278636\\
599.71	0.00999703086442773\\
599.72	0.00999722918905342\\
599.73	0.00999742127284961\\
599.74	0.00999760705136717\\
599.75	0.00999778645951533\\
599.76	0.00999795943155535\\
599.77	0.00999812590109411\\
599.78	0.00999828580107764\\
599.79	0.0099984390637846\\
599.8	0.00999858562081968\\
599.81	0.00999872540310692\\
599.82	0.009998858340883\\
599.83	0.00999898436369044\\
599.84	0.00999910340037074\\
599.85	0.00999921537905747\\
599.86	0.00999932022716924\\
599.87	0.00999941787140266\\
599.88	0.0099995082377252\\
599.89	0.00999959125136801\\
599.9	0.00999966683681859\\
599.91	0.00999973491781351\\
599.92	0.00999979541733096\\
599.93	0.00999984825758326\\
599.94	0.00999989336000932\\
599.95	0.00999993064526697\\
599.96	0.00999996003322533\\
599.97	0.00999998144295691\\
599.98	0.00999999479272987\\
599.99	0.01\\
600	0.01\\
};
\addplot [color=mycolor11,solid,forget plot]
  table[row sep=crcr]{%
0.01	0\\
1.01	0\\
2.01	0\\
3.01	0\\
4.01	0\\
5.01	0\\
6.01	0\\
7.01	0\\
8.01	0\\
9.01	0\\
10.01	0\\
11.01	0\\
12.01	0\\
13.01	0\\
14.01	0\\
15.01	0\\
16.01	0\\
17.01	0\\
18.01	0\\
19.01	0\\
20.01	0\\
21.01	0\\
22.01	0\\
23.01	0\\
24.01	0\\
25.01	0\\
26.01	0\\
27.01	0\\
28.01	0\\
29.01	0\\
30.01	0\\
31.01	0\\
32.01	0\\
33.01	0\\
34.01	0\\
35.01	0\\
36.01	0\\
37.01	0\\
38.01	0\\
39.01	0\\
40.01	0\\
41.01	0\\
42.01	0\\
43.01	0\\
44.01	0\\
45.01	0\\
46.01	0\\
47.01	0\\
48.01	0\\
49.01	0\\
50.01	0\\
51.01	0\\
52.01	0\\
53.01	0\\
54.01	0\\
55.01	0\\
56.01	0\\
57.01	0\\
58.01	0\\
59.01	0\\
60.01	0\\
61.01	0\\
62.01	0\\
63.01	0\\
64.01	0\\
65.01	0\\
66.01	0\\
67.01	0\\
68.01	0\\
69.01	0\\
70.01	0\\
71.01	0\\
72.01	0\\
73.01	0\\
74.01	0\\
75.01	0\\
76.01	0\\
77.01	0\\
78.01	0\\
79.01	0\\
80.01	0\\
81.01	0\\
82.01	0\\
83.01	0\\
84.01	0\\
85.01	0\\
86.01	0\\
87.01	0\\
88.01	0\\
89.01	0\\
90.01	0\\
91.01	0\\
92.01	0\\
93.01	0\\
94.01	0\\
95.01	0\\
96.01	0\\
97.01	0\\
98.01	0\\
99.01	0\\
100.01	0\\
101.01	0\\
102.01	0\\
103.01	0\\
104.01	0\\
105.01	0\\
106.01	0\\
107.01	0\\
108.01	0\\
109.01	0\\
110.01	0\\
111.01	0\\
112.01	0\\
113.01	0\\
114.01	0\\
115.01	0\\
116.01	0\\
117.01	0\\
118.01	0\\
119.01	0\\
120.01	0\\
121.01	0\\
122.01	0\\
123.01	0\\
124.01	0\\
125.01	0\\
126.01	0\\
127.01	0\\
128.01	0\\
129.01	0\\
130.01	0\\
131.01	0\\
132.01	0\\
133.01	0\\
134.01	0\\
135.01	0\\
136.01	0\\
137.01	0\\
138.01	0\\
139.01	0\\
140.01	0\\
141.01	0\\
142.01	0\\
143.01	0\\
144.01	0\\
145.01	0\\
146.01	0\\
147.01	0\\
148.01	0\\
149.01	0\\
150.01	0\\
151.01	0\\
152.01	0\\
153.01	0\\
154.01	0\\
155.01	0\\
156.01	0\\
157.01	0\\
158.01	0\\
159.01	0\\
160.01	0\\
161.01	0\\
162.01	0\\
163.01	0\\
164.01	0\\
165.01	0\\
166.01	0\\
167.01	0\\
168.01	0\\
169.01	0\\
170.01	0\\
171.01	0\\
172.01	0\\
173.01	0\\
174.01	0\\
175.01	0\\
176.01	0\\
177.01	0\\
178.01	0\\
179.01	0\\
180.01	0\\
181.01	0\\
182.01	0\\
183.01	0\\
184.01	0\\
185.01	0\\
186.01	0\\
187.01	0\\
188.01	0\\
189.01	0\\
190.01	0\\
191.01	0\\
192.01	0\\
193.01	0\\
194.01	0\\
195.01	0\\
196.01	0\\
197.01	0\\
198.01	0\\
199.01	0\\
200.01	0\\
201.01	0\\
202.01	0\\
203.01	0\\
204.01	0\\
205.01	0\\
206.01	0\\
207.01	0\\
208.01	0\\
209.01	0\\
210.01	0\\
211.01	0\\
212.01	0\\
213.01	0\\
214.01	0\\
215.01	0\\
216.01	0\\
217.01	0\\
218.01	0\\
219.01	0\\
220.01	0\\
221.01	0\\
222.01	0\\
223.01	0\\
224.01	0\\
225.01	0\\
226.01	0\\
227.01	0\\
228.01	0\\
229.01	0\\
230.01	0\\
231.01	0\\
232.01	0\\
233.01	0\\
234.01	0\\
235.01	0\\
236.01	0\\
237.01	0\\
238.01	0\\
239.01	0\\
240.01	0\\
241.01	0\\
242.01	0\\
243.01	0\\
244.01	0\\
245.01	0\\
246.01	0\\
247.01	0\\
248.01	0\\
249.01	0\\
250.01	0\\
251.01	0\\
252.01	0\\
253.01	0\\
254.01	0\\
255.01	0\\
256.01	0\\
257.01	0\\
258.01	0\\
259.01	0\\
260.01	0\\
261.01	0\\
262.01	0\\
263.01	0\\
264.01	0\\
265.01	0\\
266.01	0\\
267.01	0\\
268.01	0\\
269.01	0\\
270.01	0\\
271.01	0\\
272.01	0\\
273.01	0\\
274.01	0\\
275.01	0\\
276.01	0\\
277.01	0\\
278.01	0\\
279.01	0\\
280.01	0\\
281.01	0\\
282.01	0\\
283.01	0\\
284.01	0\\
285.01	0\\
286.01	0\\
287.01	0\\
288.01	0\\
289.01	0\\
290.01	0\\
291.01	0\\
292.01	0\\
293.01	0\\
294.01	0\\
295.01	0\\
296.01	0\\
297.01	0\\
298.01	0\\
299.01	0\\
300.01	0\\
301.01	0\\
302.01	0\\
303.01	0\\
304.01	0\\
305.01	0\\
306.01	0\\
307.01	0\\
308.01	0\\
309.01	0\\
310.01	0\\
311.01	0\\
312.01	0\\
313.01	0\\
314.01	0\\
315.01	0\\
316.01	0\\
317.01	0\\
318.01	0\\
319.01	0\\
320.01	0\\
321.01	0\\
322.01	0\\
323.01	0\\
324.01	0\\
325.01	0\\
326.01	0\\
327.01	0\\
328.01	0\\
329.01	0\\
330.01	0\\
331.01	0\\
332.01	0\\
333.01	0\\
334.01	0\\
335.01	0\\
336.01	0\\
337.01	0\\
338.01	0\\
339.01	0\\
340.01	0\\
341.01	0\\
342.01	0\\
343.01	0\\
344.01	0\\
345.01	0\\
346.01	0\\
347.01	0\\
348.01	0\\
349.01	0\\
350.01	0\\
351.01	0\\
352.01	0\\
353.01	0\\
354.01	0\\
355.01	0\\
356.01	0\\
357.01	0\\
358.01	0\\
359.01	0\\
360.01	0\\
361.01	0\\
362.01	0\\
363.01	0\\
364.01	0\\
365.01	0\\
366.01	0\\
367.01	0\\
368.01	0\\
369.01	0\\
370.01	0\\
371.01	0\\
372.01	0\\
373.01	0\\
374.01	0\\
375.01	0\\
376.01	0\\
377.01	0\\
378.01	0\\
379.01	0\\
380.01	0\\
381.01	0\\
382.01	0\\
383.01	0\\
384.01	0\\
385.01	0\\
386.01	0\\
387.01	0\\
388.01	0\\
389.01	0\\
390.01	0\\
391.01	0\\
392.01	0\\
393.01	0\\
394.01	0\\
395.01	0\\
396.01	0\\
397.01	0\\
398.01	0\\
399.01	0\\
400.01	0\\
401.01	0\\
402.01	0\\
403.01	0\\
404.01	0\\
405.01	0\\
406.01	0\\
407.01	0\\
408.01	0\\
409.01	0\\
410.01	0\\
411.01	0\\
412.01	0\\
413.01	0\\
414.01	0\\
415.01	0\\
416.01	0\\
417.01	0\\
418.01	0\\
419.01	0\\
420.01	0\\
421.01	0\\
422.01	0\\
423.01	0\\
424.01	0\\
425.01	0\\
426.01	0\\
427.01	0\\
428.01	0\\
429.01	0\\
430.01	0\\
431.01	0\\
432.01	0\\
433.01	0\\
434.01	0\\
435.01	0\\
436.01	0\\
437.01	0\\
438.01	0\\
439.01	0\\
440.01	0\\
441.01	0\\
442.01	0\\
443.01	0\\
444.01	0\\
445.01	0\\
446.01	0\\
447.01	0\\
448.01	0\\
449.01	0\\
450.01	0\\
451.01	0\\
452.01	0\\
453.01	0\\
454.01	0\\
455.01	0\\
456.01	0\\
457.01	0\\
458.01	0\\
459.01	0\\
460.01	0\\
461.01	0\\
462.01	0\\
463.01	0\\
464.01	0\\
465.01	0\\
466.01	0\\
467.01	0\\
468.01	0\\
469.01	0\\
470.01	0\\
471.01	0\\
472.01	0\\
473.01	0\\
474.01	0\\
475.01	0\\
476.01	0\\
477.01	0\\
478.01	0\\
479.01	0\\
480.01	0\\
481.01	0\\
482.01	0\\
483.01	0\\
484.01	0\\
485.01	0\\
486.01	0\\
487.01	0\\
488.01	0\\
489.01	0\\
490.01	0\\
491.01	0\\
492.01	0\\
493.01	0\\
494.01	0\\
495.01	0\\
496.01	0\\
497.01	0\\
498.01	0\\
499.01	0\\
500.01	0\\
501.01	0\\
502.01	0\\
503.01	0\\
504.01	0\\
505.01	0\\
506.01	0\\
507.01	0\\
508.01	0\\
509.01	0\\
510.01	0\\
511.01	0\\
512.01	0\\
513.01	0\\
514.01	0\\
515.01	0\\
516.01	0\\
517.01	0\\
518.01	0\\
519.01	0\\
520.01	0\\
521.01	0\\
522.01	0\\
523.01	0\\
524.01	0\\
525.01	0\\
526.01	0\\
527.01	0\\
528.01	0\\
529.01	0\\
530.01	0\\
531.01	0\\
532.01	0\\
533.01	0\\
534.01	0\\
535.01	0\\
536.01	0\\
537.01	0\\
538.01	0\\
539.01	0\\
540.01	0\\
541.01	0\\
542.01	0\\
543.01	0\\
544.01	0\\
545.01	0\\
546.01	0\\
547.01	0\\
548.01	0\\
549.01	1.3150276878017e-06\\
550.01	0.00018803212887869\\
551.01	0.000383425867251676\\
552.01	0.000588371123930863\\
553.01	0.000803879090139129\\
554.01	0.0010311236007458\\
555.01	0.00127147336229072\\
556.01	0.00152653133905252\\
557.01	0.00179818381624366\\
558.01	0.00208866277061284\\
559.01	0.00240062447287979\\
560.01	0.0027347493126044\\
561.01	0.00308407921609736\\
562.01	0.00344878121259429\\
563.01	0.0038298958305099\\
564.01	0.00422869759739618\\
565.01	0.00464705663062651\\
566.01	0.00508738754435159\\
567.01	0.00544973364055435\\
568.01	0.0056475991624296\\
569.01	0.00584743223013972\\
570.01	0.00604748900368386\\
571.01	0.00624531985607343\\
572.01	0.00643752645254089\\
573.01	0.0066221749281171\\
574.01	0.00680685556915087\\
575.01	0.00699128669887769\\
576.01	0.00717427665445123\\
577.01	0.00735438709921835\\
578.01	0.00752989876416336\\
579.01	0.00769877883743499\\
580.01	0.0078586552305761\\
581.01	0.00800680768344173\\
582.01	0.00814164657742384\\
583.01	0.00827128831769647\\
584.01	0.00839855560001231\\
585.01	0.0085240333202042\\
586.01	0.00864790941724483\\
587.01	0.0087698867552374\\
588.01	0.00888970311215465\\
589.01	0.00900714619236641\\
590.01	0.00912201300988473\\
591.01	0.00923390318299627\\
592.01	0.00934234304250483\\
593.01	0.00944692082238488\\
594.01	0.00954732441074327\\
595.01	0.00964338499452397\\
596.01	0.00973512605081985\\
597.01	0.00982281460498365\\
598.01	0.00990633521989775\\
599.01	0.00997061088345206\\
599.02	0.00997112665722618\\
599.03	0.00997163921121569\\
599.04	0.00997214851709544\\
599.05	0.00997265454624966\\
599.06	0.00997315726976898\\
599.07	0.00997365665844757\\
599.08	0.00997415268278013\\
599.09	0.0099746453129589\\
599.1	0.00997513451887065\\
599.11	0.0099756202700936\\
599.12	0.00997610253589435\\
599.13	0.00997658128522475\\
599.14	0.00997705648671877\\
599.15	0.0099775281086893\\
599.16	0.00997799611912494\\
599.17	0.00997846048568677\\
599.18	0.00997892117570508\\
599.19	0.00997937815617603\\
599.2	0.00997983139375833\\
599.21	0.00998028085476989\\
599.22	0.00998072650518434\\
599.23	0.00998116831062769\\
599.24	0.00998160623637473\\
599.25	0.00998204024734566\\
599.26	0.00998247030788024\\
599.27	0.00998289637903019\\
599.28	0.0099833184214462\\
599.29	0.00998373639537388\\
599.3	0.0099841502606498\\
599.31	0.00998455997669731\\
599.32	0.00998496550252246\\
599.33	0.00998536679670981\\
599.34	0.00998576381741821\\
599.35	0.00998615652237657\\
599.36	0.00998654486887954\\
599.37	0.00998692881378318\\
599.38	0.00998730831350056\\
599.39	0.00998768332399733\\
599.4	0.00998805380078729\\
599.41	0.00998841969892779\\
599.42	0.00998878097301526\\
599.43	0.00998913757718051\\
599.44	0.00998948946508413\\
599.45	0.00998983658991178\\
599.46	0.00999017890437104\\
599.47	0.00999051636068749\\
599.48	0.00999084891059986\\
599.49	0.00999117650535519\\
599.5	0.00999149909570389\\
599.51	0.00999181663189475\\
599.52	0.00999212906366997\\
599.53	0.00999243634026004\\
599.54	0.00999273841037864\\
599.55	0.00999303522221745\\
599.56	0.00999332672344092\\
599.57	0.00999361286118097\\
599.58	0.00999389358203168\\
599.59	0.00999416883204387\\
599.6	0.00999443855671962\\
599.61	0.00999470270100682\\
599.62	0.00999496120929356\\
599.63	0.00999521402540248\\
599.64	0.00999546109258514\\
599.65	0.00999570235351621\\
599.66	0.00999593775028769\\
599.67	0.00999616722440306\\
599.68	0.00999639071677128\\
599.69	0.00999660816770083\\
599.7	0.00999681951689364\\
599.71	0.00999702470343895\\
599.72	0.00999722366580711\\
599.73	0.0099974163418433\\
599.74	0.00999760266876119\\
599.75	0.00999778258313656\\
599.76	0.00999795602090076\\
599.77	0.00999812291733421\\
599.78	0.00999828320705971\\
599.79	0.00999843682403578\\
599.8	0.00999858370154984\\
599.81	0.00999872377221136\\
599.82	0.00999885696794489\\
599.83	0.00999898321998304\\
599.84	0.00999910245885937\\
599.85	0.00999921461440118\\
599.86	0.00999931961572222\\
599.87	0.00999941739121527\\
599.88	0.00999950786854473\\
599.89	0.00999959097463901\\
599.9	0.00999966663568284\\
599.91	0.00999973477710955\\
599.92	0.00999979532359316\\
599.93	0.00999984819904042\\
599.94	0.00999989332658271\\
599.95	0.00999993062856786\\
599.96	0.00999996002655181\\
599.97	0.00999998144129022\\
599.98	0.00999999479272987\\
599.99	0.01\\
600	0.01\\
};
\addplot [color=mycolor12,solid,forget plot]
  table[row sep=crcr]{%
0.01	0\\
1.01	0\\
2.01	0\\
3.01	0\\
4.01	0\\
5.01	0\\
6.01	0\\
7.01	0\\
8.01	0\\
9.01	0\\
10.01	0\\
11.01	0\\
12.01	0\\
13.01	0\\
14.01	0\\
15.01	0\\
16.01	0\\
17.01	0\\
18.01	0\\
19.01	0\\
20.01	0\\
21.01	0\\
22.01	0\\
23.01	0\\
24.01	0\\
25.01	0\\
26.01	0\\
27.01	0\\
28.01	0\\
29.01	0\\
30.01	0\\
31.01	0\\
32.01	0\\
33.01	0\\
34.01	0\\
35.01	0\\
36.01	0\\
37.01	0\\
38.01	0\\
39.01	0\\
40.01	0\\
41.01	0\\
42.01	0\\
43.01	0\\
44.01	0\\
45.01	0\\
46.01	0\\
47.01	0\\
48.01	0\\
49.01	0\\
50.01	0\\
51.01	0\\
52.01	0\\
53.01	0\\
54.01	0\\
55.01	0\\
56.01	0\\
57.01	0\\
58.01	0\\
59.01	0\\
60.01	0\\
61.01	0\\
62.01	0\\
63.01	0\\
64.01	0\\
65.01	0\\
66.01	0\\
67.01	0\\
68.01	0\\
69.01	0\\
70.01	0\\
71.01	0\\
72.01	0\\
73.01	0\\
74.01	0\\
75.01	0\\
76.01	0\\
77.01	0\\
78.01	0\\
79.01	0\\
80.01	0\\
81.01	0\\
82.01	0\\
83.01	0\\
84.01	0\\
85.01	0\\
86.01	0\\
87.01	0\\
88.01	0\\
89.01	0\\
90.01	0\\
91.01	0\\
92.01	0\\
93.01	0\\
94.01	0\\
95.01	0\\
96.01	0\\
97.01	0\\
98.01	0\\
99.01	0\\
100.01	0\\
101.01	0\\
102.01	0\\
103.01	0\\
104.01	0\\
105.01	0\\
106.01	0\\
107.01	0\\
108.01	0\\
109.01	0\\
110.01	0\\
111.01	0\\
112.01	0\\
113.01	0\\
114.01	0\\
115.01	0\\
116.01	0\\
117.01	0\\
118.01	0\\
119.01	0\\
120.01	0\\
121.01	0\\
122.01	0\\
123.01	0\\
124.01	0\\
125.01	0\\
126.01	0\\
127.01	0\\
128.01	0\\
129.01	0\\
130.01	0\\
131.01	0\\
132.01	0\\
133.01	0\\
134.01	0\\
135.01	0\\
136.01	0\\
137.01	0\\
138.01	0\\
139.01	0\\
140.01	0\\
141.01	0\\
142.01	0\\
143.01	0\\
144.01	0\\
145.01	0\\
146.01	0\\
147.01	0\\
148.01	0\\
149.01	0\\
150.01	0\\
151.01	0\\
152.01	0\\
153.01	0\\
154.01	0\\
155.01	0\\
156.01	0\\
157.01	0\\
158.01	0\\
159.01	0\\
160.01	0\\
161.01	0\\
162.01	0\\
163.01	0\\
164.01	0\\
165.01	0\\
166.01	0\\
167.01	0\\
168.01	0\\
169.01	0\\
170.01	0\\
171.01	0\\
172.01	0\\
173.01	0\\
174.01	0\\
175.01	0\\
176.01	0\\
177.01	0\\
178.01	0\\
179.01	0\\
180.01	0\\
181.01	0\\
182.01	0\\
183.01	0\\
184.01	0\\
185.01	0\\
186.01	0\\
187.01	0\\
188.01	0\\
189.01	0\\
190.01	0\\
191.01	0\\
192.01	0\\
193.01	0\\
194.01	0\\
195.01	0\\
196.01	0\\
197.01	0\\
198.01	0\\
199.01	0\\
200.01	0\\
201.01	0\\
202.01	0\\
203.01	0\\
204.01	0\\
205.01	0\\
206.01	0\\
207.01	0\\
208.01	0\\
209.01	0\\
210.01	0\\
211.01	0\\
212.01	0\\
213.01	0\\
214.01	0\\
215.01	0\\
216.01	0\\
217.01	0\\
218.01	0\\
219.01	0\\
220.01	0\\
221.01	0\\
222.01	0\\
223.01	0\\
224.01	0\\
225.01	0\\
226.01	0\\
227.01	0\\
228.01	0\\
229.01	0\\
230.01	0\\
231.01	0\\
232.01	0\\
233.01	0\\
234.01	0\\
235.01	0\\
236.01	0\\
237.01	0\\
238.01	0\\
239.01	0\\
240.01	0\\
241.01	0\\
242.01	0\\
243.01	0\\
244.01	0\\
245.01	0\\
246.01	0\\
247.01	0\\
248.01	0\\
249.01	0\\
250.01	0\\
251.01	0\\
252.01	0\\
253.01	0\\
254.01	0\\
255.01	0\\
256.01	0\\
257.01	0\\
258.01	0\\
259.01	0\\
260.01	0\\
261.01	0\\
262.01	0\\
263.01	0\\
264.01	0\\
265.01	0\\
266.01	0\\
267.01	0\\
268.01	0\\
269.01	0\\
270.01	0\\
271.01	0\\
272.01	0\\
273.01	0\\
274.01	0\\
275.01	0\\
276.01	0\\
277.01	0\\
278.01	0\\
279.01	0\\
280.01	0\\
281.01	0\\
282.01	0\\
283.01	0\\
284.01	0\\
285.01	0\\
286.01	0\\
287.01	0\\
288.01	0\\
289.01	0\\
290.01	0\\
291.01	0\\
292.01	0\\
293.01	0\\
294.01	0\\
295.01	0\\
296.01	0\\
297.01	0\\
298.01	0\\
299.01	0\\
300.01	0\\
301.01	0\\
302.01	0\\
303.01	0\\
304.01	0\\
305.01	0\\
306.01	0\\
307.01	0\\
308.01	0\\
309.01	0\\
310.01	0\\
311.01	0\\
312.01	0\\
313.01	0\\
314.01	0\\
315.01	0\\
316.01	0\\
317.01	0\\
318.01	0\\
319.01	0\\
320.01	0\\
321.01	0\\
322.01	0\\
323.01	0\\
324.01	0\\
325.01	0\\
326.01	0\\
327.01	0\\
328.01	0\\
329.01	0\\
330.01	0\\
331.01	0\\
332.01	0\\
333.01	0\\
334.01	0\\
335.01	0\\
336.01	0\\
337.01	0\\
338.01	0\\
339.01	0\\
340.01	0\\
341.01	0\\
342.01	0\\
343.01	0\\
344.01	0\\
345.01	0\\
346.01	0\\
347.01	0\\
348.01	0\\
349.01	0\\
350.01	0\\
351.01	0\\
352.01	0\\
353.01	0\\
354.01	0\\
355.01	0\\
356.01	0\\
357.01	0\\
358.01	0\\
359.01	0\\
360.01	0\\
361.01	0\\
362.01	0\\
363.01	0\\
364.01	0\\
365.01	0\\
366.01	0\\
367.01	0\\
368.01	0\\
369.01	0\\
370.01	0\\
371.01	0\\
372.01	0\\
373.01	0\\
374.01	0\\
375.01	0\\
376.01	0\\
377.01	0\\
378.01	0\\
379.01	0\\
380.01	0\\
381.01	0\\
382.01	0\\
383.01	0\\
384.01	0\\
385.01	0\\
386.01	0\\
387.01	0\\
388.01	0\\
389.01	0\\
390.01	0\\
391.01	0\\
392.01	0\\
393.01	0\\
394.01	0\\
395.01	0\\
396.01	0\\
397.01	0\\
398.01	0\\
399.01	0\\
400.01	0\\
401.01	0\\
402.01	0\\
403.01	0\\
404.01	0\\
405.01	0\\
406.01	0\\
407.01	0\\
408.01	0\\
409.01	0\\
410.01	0\\
411.01	0\\
412.01	0\\
413.01	0\\
414.01	0\\
415.01	0\\
416.01	0\\
417.01	0\\
418.01	0\\
419.01	0\\
420.01	0\\
421.01	0\\
422.01	0\\
423.01	0\\
424.01	0\\
425.01	0\\
426.01	0\\
427.01	0\\
428.01	0\\
429.01	0\\
430.01	0\\
431.01	0\\
432.01	0\\
433.01	0\\
434.01	0\\
435.01	0\\
436.01	0\\
437.01	0\\
438.01	0\\
439.01	0\\
440.01	0\\
441.01	0\\
442.01	0\\
443.01	0\\
444.01	0\\
445.01	0\\
446.01	0\\
447.01	0\\
448.01	0\\
449.01	0\\
450.01	0\\
451.01	0\\
452.01	0\\
453.01	0\\
454.01	0\\
455.01	0\\
456.01	0\\
457.01	0\\
458.01	0\\
459.01	0\\
460.01	0\\
461.01	0\\
462.01	0\\
463.01	0\\
464.01	0\\
465.01	0\\
466.01	0\\
467.01	0\\
468.01	0\\
469.01	0\\
470.01	0\\
471.01	0\\
472.01	0\\
473.01	0\\
474.01	0\\
475.01	0\\
476.01	0\\
477.01	0\\
478.01	0\\
479.01	0\\
480.01	0\\
481.01	0\\
482.01	0\\
483.01	0\\
484.01	0\\
485.01	0\\
486.01	0\\
487.01	0\\
488.01	0\\
489.01	0\\
490.01	0\\
491.01	0\\
492.01	0\\
493.01	0\\
494.01	0\\
495.01	0\\
496.01	0\\
497.01	0\\
498.01	0\\
499.01	0\\
500.01	0\\
501.01	0\\
502.01	0\\
503.01	0\\
504.01	0\\
505.01	0\\
506.01	0\\
507.01	0\\
508.01	0\\
509.01	0\\
510.01	0\\
511.01	0\\
512.01	0\\
513.01	0\\
514.01	0\\
515.01	0\\
516.01	0\\
517.01	0\\
518.01	0\\
519.01	0\\
520.01	0\\
521.01	0\\
522.01	0\\
523.01	0\\
524.01	0\\
525.01	0\\
526.01	0\\
527.01	0\\
528.01	0\\
529.01	0\\
530.01	0\\
531.01	0\\
532.01	0\\
533.01	0\\
534.01	0\\
535.01	0\\
536.01	0\\
537.01	0\\
538.01	0\\
539.01	0\\
540.01	0\\
541.01	0\\
542.01	0\\
543.01	0\\
544.01	0\\
545.01	0\\
546.01	0\\
547.01	0\\
548.01	0\\
549.01	0\\
550.01	0\\
551.01	0\\
552.01	0\\
553.01	0\\
554.01	0\\
555.01	0\\
556.01	0\\
557.01	0\\
558.01	0\\
559.01	0\\
560.01	0\\
561.01	0\\
562.01	0\\
563.01	0\\
564.01	0\\
565.01	0\\
566.01	0\\
567.01	0.000102186357564371\\
568.01	0.000389004268560401\\
569.01	0.000693210272731084\\
570.01	0.00101718055313735\\
571.01	0.00136374365253529\\
572.01	0.0017362802601048\\
573.01	0.00213614104658714\\
574.01	0.00255403495792057\\
575.01	0.00298985795617562\\
576.01	0.00344508022503881\\
577.01	0.00392130871497483\\
578.01	0.00442029093212064\\
579.01	0.00494390684300543\\
580.01	0.00549412735162226\\
581.01	0.00607287639058549\\
582.01	0.00655902377829562\\
583.01	0.00679815156285074\\
584.01	0.00702626828205187\\
585.01	0.00724284999880616\\
586.01	0.00745955259200264\\
587.01	0.00767677270493367\\
588.01	0.00789364530608335\\
589.01	0.00810926959771742\\
590.01	0.00832353557737281\\
591.01	0.00853710703215436\\
592.01	0.00874855389543367\\
593.01	0.00895606871113526\\
594.01	0.00915750548245555\\
595.01	0.0093503362862595\\
596.01	0.0095316075842035\\
597.01	0.00969789970826579\\
598.01	0.00984529451390937\\
599.01	0.00995515823496999\\
599.02	0.00995597699786755\\
599.03	0.00995678973044666\\
599.04	0.00995759639776036\\
599.05	0.0099583969645674\\
599.06	0.00995919139532951\\
599.07	0.00995997965420866\\
599.08	0.00996076170506433\\
599.09	0.00996153751145075\\
599.1	0.00996230703661406\\
599.11	0.00996307024348954\\
599.12	0.00996382709469877\\
599.13	0.00996457755254676\\
599.14	0.00996532157901905\\
599.15	0.00996605913577888\\
599.16	0.00996679018416421\\
599.17	0.00996751468518479\\
599.18	0.00996823259951922\\
599.19	0.00996894388751196\\
599.2	0.00996964850917033\\
599.21	0.00997034642416144\\
599.22	0.00997103759180925\\
599.23	0.00997172197109139\\
599.24	0.00997239952063618\\
599.25	0.00997307019871944\\
599.26	0.00997373396326153\\
599.27	0.0099743907718255\\
599.28	0.00997504058161395\\
599.29	0.00997568334946589\\
599.3	0.00997631903185352\\
599.31	0.00997694758487909\\
599.32	0.00997756896427161\\
599.33	0.00997818312538369\\
599.34	0.00997879002318826\\
599.35	0.0099793896122753\\
599.36	0.00997998184684856\\
599.37	0.00998056668072227\\
599.38	0.00998114406731784\\
599.39	0.00998171395966052\\
599.4	0.00998227631037606\\
599.41	0.00998283107168736\\
599.42	0.00998337819541113\\
599.43	0.00998391763295447\\
599.44	0.00998444933531151\\
599.45	0.00998497325306004\\
599.46	0.00998548933306155\\
599.47	0.00998599752004539\\
599.48	0.00998649775826485\\
599.49	0.00998698999149323\\
599.5	0.00998747416301998\\
599.51	0.0099879502156468\\
599.52	0.00998841809168377\\
599.53	0.00998887773294543\\
599.54	0.00998932908074684\\
599.55	0.00998977207589977\\
599.56	0.00999020665870868\\
599.57	0.00999063276896691\\
599.58	0.00999105034595272\\
599.59	0.00999145932842542\\
599.6	0.00999185965462149\\
599.61	0.00999225126225067\\
599.62	0.00999263408849213\\
599.63	0.00999300806999059\\
599.64	0.0099933731428525\\
599.65	0.00999372924264221\\
599.66	0.00999407630437819\\
599.67	0.00999441426252925\\
599.68	0.00999474305101081\\
599.69	0.00999506260318117\\
599.7	0.00999537285183787\\
599.71	0.009995673729214\\
599.72	0.00999596516697468\\
599.73	0.00999624709621344\\
599.74	0.00999651944744879\\
599.75	0.00999678215062071\\
599.76	0.00999703513508736\\
599.77	0.0099972783296217\\
599.78	0.00999751166240827\\
599.79	0.00999773506104004\\
599.8	0.00999794845251533\\
599.81	0.00999815176323479\\
599.82	0.00999834491899857\\
599.83	0.00999852784500348\\
599.84	0.00999870046584035\\
599.85	0.00999886270549145\\
599.86	0.00999901448732809\\
599.87	0.00999915573410834\\
599.88	0.00999928636797488\\
599.89	0.00999940631045301\\
599.9	0.00999951548244887\\
599.91	0.00999961380424782\\
599.92	0.00999970119551297\\
599.93	0.00999977757528401\\
599.94	0.00999984286197615\\
599.95	0.00999989697337941\\
599.96	0.00999993982665806\\
599.97	0.0099999713383504\\
599.98	0.00999999142436879\\
599.99	0.01\\
600	0.01\\
};
\addplot [color=mycolor13,solid,forget plot]
  table[row sep=crcr]{%
0.01	0\\
1.01	0\\
2.01	0\\
3.01	0\\
4.01	0\\
5.01	0\\
6.01	0\\
7.01	0\\
8.01	0\\
9.01	0\\
10.01	0\\
11.01	0\\
12.01	0\\
13.01	0\\
14.01	0\\
15.01	0\\
16.01	0\\
17.01	0\\
18.01	0\\
19.01	0\\
20.01	0\\
21.01	0\\
22.01	0\\
23.01	0\\
24.01	0\\
25.01	0\\
26.01	0\\
27.01	0\\
28.01	0\\
29.01	0\\
30.01	0\\
31.01	0\\
32.01	0\\
33.01	0\\
34.01	0\\
35.01	0\\
36.01	0\\
37.01	0\\
38.01	0\\
39.01	0\\
40.01	0\\
41.01	0\\
42.01	0\\
43.01	0\\
44.01	0\\
45.01	0\\
46.01	0\\
47.01	0\\
48.01	0\\
49.01	0\\
50.01	0\\
51.01	0\\
52.01	0\\
53.01	0\\
54.01	0\\
55.01	0\\
56.01	0\\
57.01	0\\
58.01	0\\
59.01	0\\
60.01	0\\
61.01	0\\
62.01	0\\
63.01	0\\
64.01	0\\
65.01	0\\
66.01	0\\
67.01	0\\
68.01	0\\
69.01	0\\
70.01	0\\
71.01	0\\
72.01	0\\
73.01	0\\
74.01	0\\
75.01	0\\
76.01	0\\
77.01	0\\
78.01	0\\
79.01	0\\
80.01	0\\
81.01	0\\
82.01	0\\
83.01	0\\
84.01	0\\
85.01	0\\
86.01	0\\
87.01	0\\
88.01	0\\
89.01	0\\
90.01	0\\
91.01	0\\
92.01	0\\
93.01	0\\
94.01	0\\
95.01	0\\
96.01	0\\
97.01	0\\
98.01	0\\
99.01	0\\
100.01	0\\
101.01	0\\
102.01	0\\
103.01	0\\
104.01	0\\
105.01	0\\
106.01	0\\
107.01	0\\
108.01	0\\
109.01	0\\
110.01	0\\
111.01	0\\
112.01	0\\
113.01	0\\
114.01	0\\
115.01	0\\
116.01	0\\
117.01	0\\
118.01	0\\
119.01	0\\
120.01	0\\
121.01	0\\
122.01	0\\
123.01	0\\
124.01	0\\
125.01	0\\
126.01	0\\
127.01	0\\
128.01	0\\
129.01	0\\
130.01	0\\
131.01	0\\
132.01	0\\
133.01	0\\
134.01	0\\
135.01	0\\
136.01	0\\
137.01	0\\
138.01	0\\
139.01	0\\
140.01	0\\
141.01	0\\
142.01	0\\
143.01	0\\
144.01	0\\
145.01	0\\
146.01	0\\
147.01	0\\
148.01	0\\
149.01	0\\
150.01	0\\
151.01	0\\
152.01	0\\
153.01	0\\
154.01	0\\
155.01	0\\
156.01	0\\
157.01	0\\
158.01	0\\
159.01	0\\
160.01	0\\
161.01	0\\
162.01	0\\
163.01	0\\
164.01	0\\
165.01	0\\
166.01	0\\
167.01	0\\
168.01	0\\
169.01	0\\
170.01	0\\
171.01	0\\
172.01	0\\
173.01	0\\
174.01	0\\
175.01	0\\
176.01	0\\
177.01	0\\
178.01	0\\
179.01	0\\
180.01	0\\
181.01	0\\
182.01	0\\
183.01	0\\
184.01	0\\
185.01	0\\
186.01	0\\
187.01	0\\
188.01	0\\
189.01	0\\
190.01	0\\
191.01	0\\
192.01	0\\
193.01	0\\
194.01	0\\
195.01	0\\
196.01	0\\
197.01	0\\
198.01	0\\
199.01	0\\
200.01	0\\
201.01	0\\
202.01	0\\
203.01	0\\
204.01	0\\
205.01	0\\
206.01	0\\
207.01	0\\
208.01	0\\
209.01	0\\
210.01	0\\
211.01	0\\
212.01	0\\
213.01	0\\
214.01	0\\
215.01	0\\
216.01	0\\
217.01	0\\
218.01	0\\
219.01	0\\
220.01	0\\
221.01	0\\
222.01	0\\
223.01	0\\
224.01	0\\
225.01	0\\
226.01	0\\
227.01	0\\
228.01	0\\
229.01	0\\
230.01	0\\
231.01	0\\
232.01	0\\
233.01	0\\
234.01	0\\
235.01	0\\
236.01	0\\
237.01	0\\
238.01	0\\
239.01	0\\
240.01	0\\
241.01	0\\
242.01	0\\
243.01	0\\
244.01	0\\
245.01	0\\
246.01	0\\
247.01	0\\
248.01	0\\
249.01	0\\
250.01	0\\
251.01	0\\
252.01	0\\
253.01	0\\
254.01	0\\
255.01	0\\
256.01	0\\
257.01	0\\
258.01	0\\
259.01	0\\
260.01	0\\
261.01	0\\
262.01	0\\
263.01	0\\
264.01	0\\
265.01	0\\
266.01	0\\
267.01	0\\
268.01	0\\
269.01	0\\
270.01	0\\
271.01	0\\
272.01	0\\
273.01	0\\
274.01	0\\
275.01	0\\
276.01	0\\
277.01	0\\
278.01	0\\
279.01	0\\
280.01	0\\
281.01	0\\
282.01	0\\
283.01	0\\
284.01	0\\
285.01	0\\
286.01	0\\
287.01	0\\
288.01	0\\
289.01	0\\
290.01	0\\
291.01	0\\
292.01	0\\
293.01	0\\
294.01	0\\
295.01	0\\
296.01	0\\
297.01	0\\
298.01	0\\
299.01	0\\
300.01	0\\
301.01	0\\
302.01	0\\
303.01	0\\
304.01	0\\
305.01	0\\
306.01	0\\
307.01	0\\
308.01	0\\
309.01	0\\
310.01	0\\
311.01	0\\
312.01	0\\
313.01	0\\
314.01	0\\
315.01	0\\
316.01	0\\
317.01	0\\
318.01	0\\
319.01	0\\
320.01	0\\
321.01	0\\
322.01	0\\
323.01	0\\
324.01	0\\
325.01	0\\
326.01	0\\
327.01	0\\
328.01	0\\
329.01	0\\
330.01	0\\
331.01	0\\
332.01	0\\
333.01	0\\
334.01	0\\
335.01	0\\
336.01	0\\
337.01	0\\
338.01	0\\
339.01	0\\
340.01	0\\
341.01	0\\
342.01	0\\
343.01	0\\
344.01	0\\
345.01	0\\
346.01	0\\
347.01	0\\
348.01	0\\
349.01	0\\
350.01	0\\
351.01	0\\
352.01	0\\
353.01	0\\
354.01	0\\
355.01	0\\
356.01	0\\
357.01	0\\
358.01	0\\
359.01	0\\
360.01	0\\
361.01	0\\
362.01	0\\
363.01	0\\
364.01	0\\
365.01	0\\
366.01	0\\
367.01	0\\
368.01	0\\
369.01	0\\
370.01	0\\
371.01	0\\
372.01	0\\
373.01	0\\
374.01	0\\
375.01	0\\
376.01	0\\
377.01	0\\
378.01	0\\
379.01	0\\
380.01	0\\
381.01	0\\
382.01	0\\
383.01	0\\
384.01	0\\
385.01	0\\
386.01	0\\
387.01	0\\
388.01	0\\
389.01	0\\
390.01	0\\
391.01	0\\
392.01	0\\
393.01	0\\
394.01	0\\
395.01	0\\
396.01	0\\
397.01	0\\
398.01	0\\
399.01	0\\
400.01	0\\
401.01	0\\
402.01	0\\
403.01	0\\
404.01	0\\
405.01	0\\
406.01	0\\
407.01	0\\
408.01	0\\
409.01	0\\
410.01	0\\
411.01	0\\
412.01	0\\
413.01	0\\
414.01	0\\
415.01	0\\
416.01	0\\
417.01	0\\
418.01	0\\
419.01	0\\
420.01	0\\
421.01	0\\
422.01	0\\
423.01	0\\
424.01	0\\
425.01	0\\
426.01	0\\
427.01	0\\
428.01	0\\
429.01	0\\
430.01	0\\
431.01	0\\
432.01	0\\
433.01	0\\
434.01	0\\
435.01	0\\
436.01	0\\
437.01	0\\
438.01	0\\
439.01	0\\
440.01	0\\
441.01	0\\
442.01	0\\
443.01	0\\
444.01	0\\
445.01	0\\
446.01	0\\
447.01	0\\
448.01	0\\
449.01	0\\
450.01	0\\
451.01	0\\
452.01	0\\
453.01	0\\
454.01	0\\
455.01	0\\
456.01	0\\
457.01	0\\
458.01	0\\
459.01	0\\
460.01	0\\
461.01	0\\
462.01	0\\
463.01	0\\
464.01	0\\
465.01	0\\
466.01	0\\
467.01	0\\
468.01	0\\
469.01	0\\
470.01	0\\
471.01	0\\
472.01	0\\
473.01	0\\
474.01	0\\
475.01	0\\
476.01	0\\
477.01	0\\
478.01	0\\
479.01	0\\
480.01	0\\
481.01	0\\
482.01	0\\
483.01	0\\
484.01	0\\
485.01	0\\
486.01	0\\
487.01	0\\
488.01	0\\
489.01	0\\
490.01	0\\
491.01	0\\
492.01	0\\
493.01	0\\
494.01	0\\
495.01	0\\
496.01	0\\
497.01	0\\
498.01	0\\
499.01	0\\
500.01	0\\
501.01	0\\
502.01	0\\
503.01	0\\
504.01	0\\
505.01	0\\
506.01	0\\
507.01	0\\
508.01	0\\
509.01	0\\
510.01	0\\
511.01	0\\
512.01	0\\
513.01	0\\
514.01	0\\
515.01	0\\
516.01	0\\
517.01	0\\
518.01	0\\
519.01	0\\
520.01	0\\
521.01	0\\
522.01	0\\
523.01	0\\
524.01	0\\
525.01	0\\
526.01	0\\
527.01	0\\
528.01	0\\
529.01	0\\
530.01	0\\
531.01	0\\
532.01	0\\
533.01	0\\
534.01	0\\
535.01	0\\
536.01	0\\
537.01	0\\
538.01	0\\
539.01	0\\
540.01	0\\
541.01	0\\
542.01	0\\
543.01	0\\
544.01	0\\
545.01	0\\
546.01	0\\
547.01	0\\
548.01	0\\
549.01	0\\
550.01	0\\
551.01	0\\
552.01	0\\
553.01	0\\
554.01	0\\
555.01	0\\
556.01	0\\
557.01	0\\
558.01	0\\
559.01	0\\
560.01	0\\
561.01	0\\
562.01	0\\
563.01	0\\
564.01	0\\
565.01	0\\
566.01	0\\
567.01	0\\
568.01	0\\
569.01	0\\
570.01	0\\
571.01	0\\
572.01	0\\
573.01	0\\
574.01	0\\
575.01	0\\
576.01	0\\
577.01	0\\
578.01	0\\
579.01	0\\
580.01	0\\
581.01	0\\
582.01	0.000121361606057675\\
583.01	0.00050917585157843\\
584.01	0.000923898505224239\\
585.01	0.00136483889801735\\
586.01	0.00181970423639984\\
587.01	0.00228837891239878\\
588.01	0.00277241611548145\\
589.01	0.00327301277390237\\
590.01	0.00379000974525\\
591.01	0.00432249455715031\\
592.01	0.00487168325353377\\
593.01	0.0054388733982834\\
594.01	0.00602544747888225\\
595.01	0.00663295416224377\\
596.01	0.00726313661607713\\
597.01	0.00791796813678225\\
598.01	0.00859969802437893\\
599.01	0.00930685224832755\\
599.02	0.00931395529900888\\
599.03	0.00932105763771179\\
599.04	0.00932815923365039\\
599.05	0.00933526005568012\\
599.06	0.0093423600722933\\
599.07	0.00934945925161451\\
599.08	0.00935655756139602\\
599.09	0.00936365496901308\\
599.1	0.00937075144145913\\
599.11	0.009377846945341\\
599.12	0.00938494144687396\\
599.13	0.00939203491187677\\
599.14	0.00939912730576658\\
599.15	0.00940621859355381\\
599.16	0.00941330873983692\\
599.17	0.00942039770879713\\
599.18	0.00942748546419297\\
599.19	0.00943457196935488\\
599.2	0.00944165718717962\\
599.21	0.00944874108012463\\
599.22	0.00945582361020228\\
599.23	0.00946290473897406\\
599.24	0.00946998442754468\\
599.25	0.00947706263655601\\
599.26	0.00948413932618099\\
599.27	0.00949121445611743\\
599.28	0.00949828798558168\\
599.29	0.00950535987330221\\
599.3	0.0095124300775131\\
599.31	0.00951949855594741\\
599.32	0.00952656526583043\\
599.33	0.00953363016387283\\
599.34	0.00954069320626369\\
599.35	0.00954775434866342\\
599.36	0.00955481354619657\\
599.37	0.00956187075344444\\
599.38	0.0095689259244377\\
599.39	0.00957597901264877\\
599.4	0.00958302997098412\\
599.41	0.0095900787517764\\
599.42	0.00959712530677647\\
599.43	0.0096041695871453\\
599.44	0.00961121154344564\\
599.45	0.00961825112563368\\
599.46	0.0096252882830519\\
599.47	0.00963232296442116\\
599.48	0.00963935511783186\\
599.49	0.00964638469073488\\
599.5	0.00965341162993244\\
599.51	0.00966043588156874\\
599.52	0.00966745739112043\\
599.53	0.0096744761033869\\
599.54	0.00968149196248044\\
599.55	0.00968850491181611\\
599.56	0.00969551489410154\\
599.57	0.00970252185132643\\
599.58	0.00970952572475191\\
599.59	0.00971652645489968\\
599.6	0.0097235239815409\\
599.61	0.00973051824368496\\
599.62	0.00973750917956788\\
599.63	0.00974449672664063\\
599.64	0.00975148082155711\\
599.65	0.00975846140016193\\
599.66	0.00976543839747794\\
599.67	0.00977241174769349\\
599.68	0.00977938138414944\\
599.69	0.0097863472393259\\
599.7	0.00979330924482867\\
599.71	0.00980026733137542\\
599.72	0.00980722142878157\\
599.73	0.00981417146594583\\
599.74	0.00982111737083552\\
599.75	0.00982805907047141\\
599.76	0.00983499649091242\\
599.77	0.00984192955723982\\
599.78	0.00984885819354115\\
599.79	0.00985578232289379\\
599.8	0.0098627018673481\\
599.81	0.00986961674791019\\
599.82	0.00987652688452436\\
599.83	0.009883432196055\\
599.84	0.00989033260026818\\
599.85	0.00989722801381275\\
599.86	0.00990411835220096\\
599.87	0.00991100352978864\\
599.88	0.00991788345975494\\
599.89	0.00992475805408144\\
599.9	0.0099316272235309\\
599.91	0.00993849087762531\\
599.92	0.0099453489246235\\
599.93	0.00995220127149813\\
599.94	0.00995904782391208\\
599.95	0.00996588848619419\\
599.96	0.00997272316131441\\
599.97	0.00997955175085828\\
599.98	0.00998637415500062\\
599.99	0.00999319027247863\\
600	0.01\\
};
\addplot [color=mycolor14,solid,forget plot]
  table[row sep=crcr]{%
0.01	0.01\\
1.01	0.01\\
2.01	0.01\\
3.01	0.01\\
4.01	0.01\\
5.01	0.01\\
6.01	0.01\\
7.01	0.01\\
8.01	0.01\\
9.01	0.01\\
10.01	0.01\\
11.01	0.01\\
12.01	0.01\\
13.01	0.01\\
14.01	0.01\\
15.01	0.01\\
16.01	0.01\\
17.01	0.01\\
18.01	0.01\\
19.01	0.01\\
20.01	0.01\\
21.01	0.01\\
22.01	0.01\\
23.01	0.01\\
24.01	0.01\\
25.01	0.01\\
26.01	0.01\\
27.01	0.01\\
28.01	0.01\\
29.01	0.01\\
30.01	0.01\\
31.01	0.01\\
32.01	0.01\\
33.01	0.01\\
34.01	0.01\\
35.01	0.01\\
36.01	0.01\\
37.01	0.01\\
38.01	0.01\\
39.01	0.01\\
40.01	0.01\\
41.01	0.01\\
42.01	0.01\\
43.01	0.01\\
44.01	0.01\\
45.01	0.01\\
46.01	0.01\\
47.01	0.01\\
48.01	0.01\\
49.01	0.01\\
50.01	0.01\\
51.01	0.01\\
52.01	0.01\\
53.01	0.01\\
54.01	0.01\\
55.01	0.01\\
56.01	0.01\\
57.01	0.01\\
58.01	0.01\\
59.01	0.01\\
60.01	0.01\\
61.01	0.01\\
62.01	0.01\\
63.01	0.01\\
64.01	0.01\\
65.01	0.01\\
66.01	0.01\\
67.01	0.01\\
68.01	0.01\\
69.01	0.01\\
70.01	0.01\\
71.01	0.01\\
72.01	0.01\\
73.01	0.01\\
74.01	0.01\\
75.01	0.01\\
76.01	0.01\\
77.01	0.01\\
78.01	0.01\\
79.01	0.01\\
80.01	0.01\\
81.01	0.01\\
82.01	0.01\\
83.01	0.01\\
84.01	0.01\\
85.01	0.01\\
86.01	0.01\\
87.01	0.01\\
88.01	0.01\\
89.01	0.01\\
90.01	0.01\\
91.01	0.01\\
92.01	0.01\\
93.01	0.01\\
94.01	0.01\\
95.01	0.01\\
96.01	0.01\\
97.01	0.01\\
98.01	0.01\\
99.01	0.01\\
100.01	0.01\\
101.01	0.01\\
102.01	0.01\\
103.01	0.01\\
104.01	0.01\\
105.01	0.01\\
106.01	0.01\\
107.01	0.01\\
108.01	0.01\\
109.01	0.01\\
110.01	0.01\\
111.01	0.01\\
112.01	0.01\\
113.01	0.01\\
114.01	0.01\\
115.01	0.01\\
116.01	0.01\\
117.01	0.01\\
118.01	0.01\\
119.01	0.01\\
120.01	0.01\\
121.01	0.01\\
122.01	0.01\\
123.01	0.01\\
124.01	0.01\\
125.01	0.01\\
126.01	0.01\\
127.01	0.01\\
128.01	0.01\\
129.01	0.01\\
130.01	0.01\\
131.01	0.01\\
132.01	0.01\\
133.01	0.01\\
134.01	0.01\\
135.01	0.01\\
136.01	0.01\\
137.01	0.01\\
138.01	0.01\\
139.01	0.01\\
140.01	0.01\\
141.01	0.01\\
142.01	0.01\\
143.01	0.01\\
144.01	0.01\\
145.01	0.01\\
146.01	0.01\\
147.01	0.01\\
148.01	0.01\\
149.01	0.01\\
150.01	0.01\\
151.01	0.01\\
152.01	0.01\\
153.01	0.01\\
154.01	0.01\\
155.01	0.01\\
156.01	0.01\\
157.01	0.01\\
158.01	0.01\\
159.01	0.01\\
160.01	0.01\\
161.01	0.01\\
162.01	0.01\\
163.01	0.01\\
164.01	0.01\\
165.01	0.01\\
166.01	0.01\\
167.01	0.01\\
168.01	0.01\\
169.01	0.01\\
170.01	0.01\\
171.01	0.01\\
172.01	0.01\\
173.01	0.01\\
174.01	0.01\\
175.01	0.01\\
176.01	0.01\\
177.01	0.01\\
178.01	0.01\\
179.01	0.01\\
180.01	0.01\\
181.01	0.01\\
182.01	0.01\\
183.01	0.01\\
184.01	0.01\\
185.01	0.01\\
186.01	0.01\\
187.01	0.01\\
188.01	0.01\\
189.01	0.01\\
190.01	0.01\\
191.01	0.01\\
192.01	0.01\\
193.01	0.01\\
194.01	0.01\\
195.01	0.01\\
196.01	0.01\\
197.01	0.01\\
198.01	0.01\\
199.01	0.01\\
200.01	0.01\\
201.01	0.01\\
202.01	0.01\\
203.01	0.01\\
204.01	0.01\\
205.01	0.01\\
206.01	0.01\\
207.01	0.01\\
208.01	0.01\\
209.01	0.01\\
210.01	0.01\\
211.01	0.01\\
212.01	0.01\\
213.01	0.01\\
214.01	0.01\\
215.01	0.01\\
216.01	0.01\\
217.01	0.01\\
218.01	0.01\\
219.01	0.01\\
220.01	0.01\\
221.01	0.01\\
222.01	0.01\\
223.01	0.01\\
224.01	0.01\\
225.01	0.01\\
226.01	0.01\\
227.01	0.01\\
228.01	0.01\\
229.01	0.01\\
230.01	0.01\\
231.01	0.01\\
232.01	0.01\\
233.01	0.01\\
234.01	0.01\\
235.01	0.01\\
236.01	0.01\\
237.01	0.01\\
238.01	0.01\\
239.01	0.01\\
240.01	0.01\\
241.01	0.01\\
242.01	0.01\\
243.01	0.01\\
244.01	0.01\\
245.01	0.01\\
246.01	0.01\\
247.01	0.01\\
248.01	0.01\\
249.01	0.01\\
250.01	0.01\\
251.01	0.01\\
252.01	0.01\\
253.01	0.01\\
254.01	0.01\\
255.01	0.01\\
256.01	0.01\\
257.01	0.01\\
258.01	0.01\\
259.01	0.01\\
260.01	0.01\\
261.01	0.01\\
262.01	0.01\\
263.01	0.01\\
264.01	0.01\\
265.01	0.01\\
266.01	0.01\\
267.01	0.01\\
268.01	0.01\\
269.01	0.01\\
270.01	0.01\\
271.01	0.01\\
272.01	0.01\\
273.01	0.01\\
274.01	0.01\\
275.01	0.01\\
276.01	0.01\\
277.01	0.01\\
278.01	0.01\\
279.01	0.01\\
280.01	0.01\\
281.01	0.01\\
282.01	0.01\\
283.01	0.01\\
284.01	0.01\\
285.01	0.01\\
286.01	0.01\\
287.01	0.01\\
288.01	0.01\\
289.01	0.01\\
290.01	0.01\\
291.01	0.01\\
292.01	0.01\\
293.01	0.01\\
294.01	0.01\\
295.01	0.01\\
296.01	0.01\\
297.01	0.01\\
298.01	0.01\\
299.01	0.01\\
300.01	0.01\\
301.01	0.01\\
302.01	0.01\\
303.01	0.01\\
304.01	0.01\\
305.01	0.01\\
306.01	0.01\\
307.01	0.01\\
308.01	0.01\\
309.01	0.01\\
310.01	0.01\\
311.01	0.01\\
312.01	0.01\\
313.01	0.01\\
314.01	0.01\\
315.01	0.01\\
316.01	0.01\\
317.01	0.01\\
318.01	0.01\\
319.01	0.01\\
320.01	0.01\\
321.01	0.01\\
322.01	0.01\\
323.01	0.01\\
324.01	0.01\\
325.01	0.01\\
326.01	0.01\\
327.01	0.01\\
328.01	0.01\\
329.01	0.01\\
330.01	0.01\\
331.01	0.01\\
332.01	0.01\\
333.01	0.01\\
334.01	0.01\\
335.01	0.01\\
336.01	0.01\\
337.01	0.01\\
338.01	0.01\\
339.01	0.01\\
340.01	0.01\\
341.01	0.01\\
342.01	0.01\\
343.01	0.01\\
344.01	0.01\\
345.01	0.01\\
346.01	0.01\\
347.01	0.01\\
348.01	0.01\\
349.01	0.01\\
350.01	0.01\\
351.01	0.01\\
352.01	0.01\\
353.01	0.01\\
354.01	0.01\\
355.01	0.01\\
356.01	0.01\\
357.01	0.01\\
358.01	0.01\\
359.01	0.01\\
360.01	0.01\\
361.01	0.01\\
362.01	0.01\\
363.01	0.01\\
364.01	0.01\\
365.01	0.01\\
366.01	0.01\\
367.01	0.01\\
368.01	0.01\\
369.01	0.01\\
370.01	0.01\\
371.01	0.01\\
372.01	0.01\\
373.01	0.01\\
374.01	0.01\\
375.01	0.01\\
376.01	0.01\\
377.01	0.01\\
378.01	0.01\\
379.01	0.01\\
380.01	0.01\\
381.01	0.01\\
382.01	0.01\\
383.01	0.01\\
384.01	0.01\\
385.01	0.01\\
386.01	0.01\\
387.01	0.01\\
388.01	0.01\\
389.01	0.01\\
390.01	0.01\\
391.01	0.01\\
392.01	0.01\\
393.01	0.01\\
394.01	0.01\\
395.01	0.01\\
396.01	0.01\\
397.01	0.01\\
398.01	0.01\\
399.01	0.01\\
400.01	0.01\\
401.01	0.01\\
402.01	0.01\\
403.01	0.01\\
404.01	0.01\\
405.01	0.01\\
406.01	0.01\\
407.01	0.01\\
408.01	0.01\\
409.01	0.01\\
410.01	0.01\\
411.01	0.01\\
412.01	0.01\\
413.01	0.01\\
414.01	0.01\\
415.01	0.01\\
416.01	0.01\\
417.01	0.01\\
418.01	0.01\\
419.01	0.01\\
420.01	0.01\\
421.01	0.01\\
422.01	0.01\\
423.01	0.01\\
424.01	0.01\\
425.01	0.01\\
426.01	0.01\\
427.01	0.01\\
428.01	0.01\\
429.01	0.01\\
430.01	0.01\\
431.01	0.01\\
432.01	0.01\\
433.01	0.01\\
434.01	0.01\\
435.01	0.01\\
436.01	0.01\\
437.01	0.01\\
438.01	0.01\\
439.01	0.01\\
440.01	0.01\\
441.01	0.01\\
442.01	0.01\\
443.01	0.01\\
444.01	0.01\\
445.01	0.01\\
446.01	0.01\\
447.01	0.01\\
448.01	0.01\\
449.01	0.01\\
450.01	0.01\\
451.01	0.01\\
452.01	0.01\\
453.01	0.01\\
454.01	0.01\\
455.01	0.01\\
456.01	0.01\\
457.01	0.01\\
458.01	0.01\\
459.01	0.01\\
460.01	0.01\\
461.01	0.01\\
462.01	0.01\\
463.01	0.01\\
464.01	0.01\\
465.01	0.01\\
466.01	0.01\\
467.01	0.01\\
468.01	0.01\\
469.01	0.01\\
470.01	0.01\\
471.01	0.01\\
472.01	0.01\\
473.01	0.01\\
474.01	0.01\\
475.01	0.01\\
476.01	0.01\\
477.01	0.01\\
478.01	0.01\\
479.01	0.01\\
480.01	0.01\\
481.01	0.01\\
482.01	0.01\\
483.01	0.01\\
484.01	0.01\\
485.01	0.01\\
486.01	0.01\\
487.01	0.01\\
488.01	0.01\\
489.01	0.01\\
490.01	0.01\\
491.01	0.01\\
492.01	0.01\\
493.01	0.01\\
494.01	0.01\\
495.01	0.01\\
496.01	0.01\\
497.01	0.01\\
498.01	0.01\\
499.01	0.01\\
500.01	0.01\\
501.01	0.01\\
502.01	0.01\\
503.01	0.01\\
504.01	0.01\\
505.01	0.01\\
506.01	0.01\\
507.01	0.01\\
508.01	0.01\\
509.01	0.01\\
510.01	0.01\\
511.01	0.01\\
512.01	0.01\\
513.01	0.01\\
514.01	0.01\\
515.01	0.01\\
516.01	0.01\\
517.01	0.01\\
518.01	0.01\\
519.01	0.01\\
520.01	0.01\\
521.01	0.01\\
522.01	0.01\\
523.01	0.01\\
524.01	0.01\\
525.01	0.01\\
526.01	0.01\\
527.01	0.01\\
528.01	0.01\\
529.01	0.01\\
530.01	0.01\\
531.01	0.01\\
532.01	0.01\\
533.01	0.01\\
534.01	0.01\\
535.01	0.01\\
536.01	0.01\\
537.01	0.01\\
538.01	0.01\\
539.01	0.01\\
540.01	0.01\\
541.01	0.01\\
542.01	0.01\\
543.01	0.01\\
544.01	0.01\\
545.01	0.01\\
546.01	0.01\\
547.01	0.01\\
548.01	0.01\\
549.01	0.01\\
550.01	0.01\\
551.01	0.01\\
552.01	0.01\\
553.01	0.01\\
554.01	0.01\\
555.01	0.01\\
556.01	0.01\\
557.01	0.01\\
558.01	0.01\\
559.01	0.01\\
560.01	0.01\\
561.01	0.01\\
562.01	0.01\\
563.01	0.01\\
564.01	0.01\\
565.01	0.01\\
566.01	0.01\\
567.01	0.01\\
568.01	0.01\\
569.01	0.01\\
570.01	0.01\\
571.01	0.01\\
572.01	0.01\\
573.01	0.01\\
574.01	0.01\\
575.01	0.01\\
576.01	0.01\\
577.01	0.01\\
578.01	0.01\\
579.01	0.01\\
580.01	0.01\\
581.01	0.01\\
582.01	0.00992989647703613\\
583.01	0.00954177372982102\\
584.01	0.00912547726397663\\
585.01	0.00868439512189853\\
586.01	0.00822984681208113\\
587.01	0.00776098456227372\\
588.01	0.00727618774155096\\
589.01	0.00677424947721351\\
590.01	0.00625539153484212\\
591.01	0.00572055713570469\\
592.01	0.00516858725663456\\
593.01	0.00459821690406403\\
594.01	0.00400803771374159\\
595.01	0.00339646929396734\\
596.01	0.00276172792797194\\
597.01	0.00210178615661351\\
598.01	0.00141431937259587\\
599.01	0.000700643040783771\\
599.02	0.000693469184667535\\
599.03	0.00068629596350653\\
599.04	0.000679123407609256\\
599.05	0.000671951547635673\\
599.06	0.000664780414601591\\
599.07	0.000657610039883112\\
599.08	0.00065044045522116\\
599.09	0.00064327169272606\\
599.1	0.000636103784882198\\
599.11	0.000628936764552753\\
599.12	0.000621770664984485\\
599.13	0.000614605519812619\\
599.14	0.000607441363065791\\
599.15	0.000600278229171067\\
599.16	0.000593116152959046\\
599.17	0.000585955169669047\\
599.18	0.000578795314954366\\
599.19	0.000571636624887621\\
599.2	0.000564479135966182\\
599.21	0.000557322885117686\\
599.22	0.000550167909705638\\
599.23	0.000543014247535096\\
599.24	0.000535861936858464\\
599.25	0.000528711016381361\\
599.26	0.000521561525268591\\
599.27	0.000514413503150207\\
599.28	0.000507266990127684\\
599.29	0.000500122026780177\\
599.3	0.0004929786541709\\
599.31	0.000485836913853588\\
599.32	0.000478696847879087\\
599.33	0.000471558498802045\\
599.34	0.000464421909687714\\
599.35	0.00045728712411886\\
599.36	0.000450154186202811\\
599.37	0.000443023140578602\\
599.38	0.000435894032424248\\
599.39	0.000428766907464151\\
599.4	0.000421641811976625\\
599.41	0.000414518792801556\\
599.42	0.000407397897348187\\
599.43	0.000400279173603059\\
599.44	0.000393162670138064\\
599.45	0.000386048436118664\\
599.46	0.000378936521312195\\
599.47	0.000371826976094117\\
599.48	0.000364719851456648\\
599.49	0.000357615199017553\\
599.5	0.000350513071029101\\
599.51	0.000343413520387173\\
599.52	0.000336316600640549\\
599.53	0.000329222366000361\\
599.54	0.000322130871349731\\
599.55	0.000315042172253572\\
599.56	0.000307956324968602\\
599.57	0.000300873386453522\\
599.58	0.000293793414379399\\
599.59	0.000286716467140258\\
599.6	0.000279642603863864\\
599.61	0.000272571884422727\\
599.62	0.000265504369445314\\
599.63	0.000258440120327493\\
599.64	0.000251379199244197\\
599.65	0.000244321669161319\\
599.66	0.00023726759384787\\
599.67	0.000230217037888347\\
599.68	0.000223170066695397\\
599.69	0.000216126746522712\\
599.7	0.000209087144478199\\
599.71	0.00020205132853744\\
599.72	0.000195019367557427\\
599.73	0.000187991331290582\\
599.74	0.000180967290399096\\
599.75	0.000173947316469567\\
599.76	0.000166931482027961\\
599.77	0.000159919860554901\\
599.78	0.000152912526501302\\
599.79	0.000145909555304351\\
599.8	0.00013891102340384\\
599.81	0.000131917008258898\\
599.82	0.000124927588365074\\
599.83	0.000117942843271846\\
599.84	0.000110962853600519\\
599.85	0.000103987701062559\\
599.86	9.70174684783599e-05\\
599.87	9.00522397964624e-05\\
599.88	8.30921001132405e-05\\
599.89	7.61371356930683e-05\\
599.9	6.91874339889959e-05\\
599.91	6.22430836639334e-05\\
599.92	5.53041746123722e-05\\
599.93	4.83707979826668e-05\\
599.94	4.14430461998863e-05\\
599.95	3.45210129892633e-05\\
599.96	2.76047934002557e-05\\
599.97	2.06944838312597e-05\\
599.98	1.37901820549801e-05\\
599.99	6.891987244486e-06\\
600	0\\
};
\addplot [color=mycolor15,solid,forget plot]
  table[row sep=crcr]{%
0.01	0.01\\
1.01	0.01\\
2.01	0.01\\
3.01	0.01\\
4.01	0.01\\
5.01	0.01\\
6.01	0.01\\
7.01	0.01\\
8.01	0.01\\
9.01	0.01\\
10.01	0.01\\
11.01	0.01\\
12.01	0.01\\
13.01	0.01\\
14.01	0.01\\
15.01	0.01\\
16.01	0.01\\
17.01	0.01\\
18.01	0.01\\
19.01	0.01\\
20.01	0.01\\
21.01	0.01\\
22.01	0.01\\
23.01	0.01\\
24.01	0.01\\
25.01	0.01\\
26.01	0.01\\
27.01	0.01\\
28.01	0.01\\
29.01	0.01\\
30.01	0.01\\
31.01	0.01\\
32.01	0.01\\
33.01	0.01\\
34.01	0.01\\
35.01	0.01\\
36.01	0.01\\
37.01	0.01\\
38.01	0.01\\
39.01	0.01\\
40.01	0.01\\
41.01	0.01\\
42.01	0.01\\
43.01	0.01\\
44.01	0.01\\
45.01	0.01\\
46.01	0.01\\
47.01	0.01\\
48.01	0.01\\
49.01	0.01\\
50.01	0.01\\
51.01	0.01\\
52.01	0.01\\
53.01	0.01\\
54.01	0.01\\
55.01	0.01\\
56.01	0.01\\
57.01	0.01\\
58.01	0.01\\
59.01	0.01\\
60.01	0.01\\
61.01	0.01\\
62.01	0.01\\
63.01	0.01\\
64.01	0.01\\
65.01	0.01\\
66.01	0.01\\
67.01	0.01\\
68.01	0.01\\
69.01	0.01\\
70.01	0.01\\
71.01	0.01\\
72.01	0.01\\
73.01	0.01\\
74.01	0.01\\
75.01	0.01\\
76.01	0.01\\
77.01	0.01\\
78.01	0.01\\
79.01	0.01\\
80.01	0.01\\
81.01	0.01\\
82.01	0.01\\
83.01	0.01\\
84.01	0.01\\
85.01	0.01\\
86.01	0.01\\
87.01	0.01\\
88.01	0.01\\
89.01	0.01\\
90.01	0.01\\
91.01	0.01\\
92.01	0.01\\
93.01	0.01\\
94.01	0.01\\
95.01	0.01\\
96.01	0.01\\
97.01	0.01\\
98.01	0.01\\
99.01	0.01\\
100.01	0.01\\
101.01	0.01\\
102.01	0.01\\
103.01	0.01\\
104.01	0.01\\
105.01	0.01\\
106.01	0.01\\
107.01	0.01\\
108.01	0.01\\
109.01	0.01\\
110.01	0.01\\
111.01	0.01\\
112.01	0.01\\
113.01	0.01\\
114.01	0.01\\
115.01	0.01\\
116.01	0.01\\
117.01	0.01\\
118.01	0.01\\
119.01	0.01\\
120.01	0.01\\
121.01	0.01\\
122.01	0.01\\
123.01	0.01\\
124.01	0.01\\
125.01	0.01\\
126.01	0.01\\
127.01	0.01\\
128.01	0.01\\
129.01	0.01\\
130.01	0.01\\
131.01	0.01\\
132.01	0.01\\
133.01	0.01\\
134.01	0.01\\
135.01	0.01\\
136.01	0.01\\
137.01	0.01\\
138.01	0.01\\
139.01	0.01\\
140.01	0.01\\
141.01	0.01\\
142.01	0.01\\
143.01	0.01\\
144.01	0.01\\
145.01	0.01\\
146.01	0.01\\
147.01	0.01\\
148.01	0.01\\
149.01	0.01\\
150.01	0.01\\
151.01	0.01\\
152.01	0.01\\
153.01	0.01\\
154.01	0.01\\
155.01	0.01\\
156.01	0.01\\
157.01	0.01\\
158.01	0.01\\
159.01	0.01\\
160.01	0.01\\
161.01	0.01\\
162.01	0.01\\
163.01	0.01\\
164.01	0.01\\
165.01	0.01\\
166.01	0.01\\
167.01	0.01\\
168.01	0.01\\
169.01	0.01\\
170.01	0.01\\
171.01	0.01\\
172.01	0.01\\
173.01	0.01\\
174.01	0.01\\
175.01	0.01\\
176.01	0.01\\
177.01	0.01\\
178.01	0.01\\
179.01	0.01\\
180.01	0.01\\
181.01	0.01\\
182.01	0.01\\
183.01	0.01\\
184.01	0.01\\
185.01	0.01\\
186.01	0.01\\
187.01	0.01\\
188.01	0.01\\
189.01	0.01\\
190.01	0.01\\
191.01	0.01\\
192.01	0.01\\
193.01	0.01\\
194.01	0.01\\
195.01	0.01\\
196.01	0.01\\
197.01	0.01\\
198.01	0.01\\
199.01	0.01\\
200.01	0.01\\
201.01	0.01\\
202.01	0.01\\
203.01	0.01\\
204.01	0.01\\
205.01	0.01\\
206.01	0.01\\
207.01	0.01\\
208.01	0.01\\
209.01	0.01\\
210.01	0.01\\
211.01	0.01\\
212.01	0.01\\
213.01	0.01\\
214.01	0.01\\
215.01	0.01\\
216.01	0.01\\
217.01	0.01\\
218.01	0.01\\
219.01	0.01\\
220.01	0.01\\
221.01	0.01\\
222.01	0.01\\
223.01	0.01\\
224.01	0.01\\
225.01	0.01\\
226.01	0.01\\
227.01	0.01\\
228.01	0.01\\
229.01	0.01\\
230.01	0.01\\
231.01	0.01\\
232.01	0.01\\
233.01	0.01\\
234.01	0.01\\
235.01	0.01\\
236.01	0.01\\
237.01	0.01\\
238.01	0.01\\
239.01	0.01\\
240.01	0.01\\
241.01	0.01\\
242.01	0.01\\
243.01	0.01\\
244.01	0.01\\
245.01	0.01\\
246.01	0.01\\
247.01	0.01\\
248.01	0.01\\
249.01	0.01\\
250.01	0.01\\
251.01	0.01\\
252.01	0.01\\
253.01	0.01\\
254.01	0.01\\
255.01	0.01\\
256.01	0.01\\
257.01	0.01\\
258.01	0.01\\
259.01	0.01\\
260.01	0.01\\
261.01	0.01\\
262.01	0.01\\
263.01	0.01\\
264.01	0.01\\
265.01	0.01\\
266.01	0.01\\
267.01	0.01\\
268.01	0.01\\
269.01	0.01\\
270.01	0.01\\
271.01	0.01\\
272.01	0.01\\
273.01	0.01\\
274.01	0.01\\
275.01	0.01\\
276.01	0.01\\
277.01	0.01\\
278.01	0.01\\
279.01	0.01\\
280.01	0.01\\
281.01	0.01\\
282.01	0.01\\
283.01	0.01\\
284.01	0.01\\
285.01	0.01\\
286.01	0.01\\
287.01	0.01\\
288.01	0.01\\
289.01	0.01\\
290.01	0.01\\
291.01	0.01\\
292.01	0.01\\
293.01	0.01\\
294.01	0.01\\
295.01	0.01\\
296.01	0.01\\
297.01	0.01\\
298.01	0.01\\
299.01	0.01\\
300.01	0.01\\
301.01	0.01\\
302.01	0.01\\
303.01	0.01\\
304.01	0.01\\
305.01	0.01\\
306.01	0.01\\
307.01	0.01\\
308.01	0.01\\
309.01	0.01\\
310.01	0.01\\
311.01	0.01\\
312.01	0.01\\
313.01	0.01\\
314.01	0.01\\
315.01	0.01\\
316.01	0.01\\
317.01	0.01\\
318.01	0.01\\
319.01	0.01\\
320.01	0.01\\
321.01	0.01\\
322.01	0.01\\
323.01	0.01\\
324.01	0.01\\
325.01	0.01\\
326.01	0.01\\
327.01	0.01\\
328.01	0.01\\
329.01	0.01\\
330.01	0.01\\
331.01	0.01\\
332.01	0.01\\
333.01	0.01\\
334.01	0.01\\
335.01	0.01\\
336.01	0.01\\
337.01	0.01\\
338.01	0.01\\
339.01	0.01\\
340.01	0.01\\
341.01	0.01\\
342.01	0.01\\
343.01	0.01\\
344.01	0.01\\
345.01	0.01\\
346.01	0.01\\
347.01	0.01\\
348.01	0.01\\
349.01	0.01\\
350.01	0.01\\
351.01	0.01\\
352.01	0.01\\
353.01	0.01\\
354.01	0.01\\
355.01	0.01\\
356.01	0.01\\
357.01	0.01\\
358.01	0.01\\
359.01	0.01\\
360.01	0.01\\
361.01	0.01\\
362.01	0.01\\
363.01	0.01\\
364.01	0.01\\
365.01	0.01\\
366.01	0.01\\
367.01	0.01\\
368.01	0.01\\
369.01	0.01\\
370.01	0.01\\
371.01	0.01\\
372.01	0.01\\
373.01	0.01\\
374.01	0.01\\
375.01	0.01\\
376.01	0.01\\
377.01	0.01\\
378.01	0.01\\
379.01	0.01\\
380.01	0.01\\
381.01	0.01\\
382.01	0.01\\
383.01	0.01\\
384.01	0.01\\
385.01	0.01\\
386.01	0.01\\
387.01	0.01\\
388.01	0.01\\
389.01	0.01\\
390.01	0.01\\
391.01	0.01\\
392.01	0.01\\
393.01	0.01\\
394.01	0.01\\
395.01	0.01\\
396.01	0.01\\
397.01	0.01\\
398.01	0.01\\
399.01	0.01\\
400.01	0.01\\
401.01	0.01\\
402.01	0.01\\
403.01	0.01\\
404.01	0.01\\
405.01	0.01\\
406.01	0.01\\
407.01	0.01\\
408.01	0.01\\
409.01	0.01\\
410.01	0.01\\
411.01	0.01\\
412.01	0.01\\
413.01	0.01\\
414.01	0.01\\
415.01	0.01\\
416.01	0.01\\
417.01	0.01\\
418.01	0.01\\
419.01	0.01\\
420.01	0.01\\
421.01	0.01\\
422.01	0.01\\
423.01	0.01\\
424.01	0.01\\
425.01	0.01\\
426.01	0.01\\
427.01	0.01\\
428.01	0.01\\
429.01	0.01\\
430.01	0.01\\
431.01	0.01\\
432.01	0.01\\
433.01	0.01\\
434.01	0.01\\
435.01	0.01\\
436.01	0.01\\
437.01	0.01\\
438.01	0.01\\
439.01	0.01\\
440.01	0.01\\
441.01	0.01\\
442.01	0.01\\
443.01	0.01\\
444.01	0.01\\
445.01	0.01\\
446.01	0.01\\
447.01	0.01\\
448.01	0.01\\
449.01	0.01\\
450.01	0.01\\
451.01	0.01\\
452.01	0.01\\
453.01	0.01\\
454.01	0.01\\
455.01	0.01\\
456.01	0.01\\
457.01	0.01\\
458.01	0.01\\
459.01	0.01\\
460.01	0.01\\
461.01	0.01\\
462.01	0.01\\
463.01	0.01\\
464.01	0.01\\
465.01	0.01\\
466.01	0.01\\
467.01	0.01\\
468.01	0.01\\
469.01	0.01\\
470.01	0.01\\
471.01	0.01\\
472.01	0.01\\
473.01	0.01\\
474.01	0.01\\
475.01	0.01\\
476.01	0.01\\
477.01	0.01\\
478.01	0.01\\
479.01	0.01\\
480.01	0.01\\
481.01	0.01\\
482.01	0.01\\
483.01	0.01\\
484.01	0.01\\
485.01	0.01\\
486.01	0.01\\
487.01	0.01\\
488.01	0.01\\
489.01	0.01\\
490.01	0.01\\
491.01	0.01\\
492.01	0.01\\
493.01	0.01\\
494.01	0.01\\
495.01	0.01\\
496.01	0.01\\
497.01	0.01\\
498.01	0.01\\
499.01	0.01\\
500.01	0.01\\
501.01	0.01\\
502.01	0.01\\
503.01	0.01\\
504.01	0.01\\
505.01	0.01\\
506.01	0.01\\
507.01	0.01\\
508.01	0.01\\
509.01	0.01\\
510.01	0.01\\
511.01	0.01\\
512.01	0.01\\
513.01	0.01\\
514.01	0.01\\
515.01	0.01\\
516.01	0.01\\
517.01	0.01\\
518.01	0.01\\
519.01	0.01\\
520.01	0.01\\
521.01	0.01\\
522.01	0.01\\
523.01	0.01\\
524.01	0.01\\
525.01	0.01\\
526.01	0.01\\
527.01	0.01\\
528.01	0.01\\
529.01	0.01\\
530.01	0.01\\
531.01	0.01\\
532.01	0.01\\
533.01	0.01\\
534.01	0.01\\
535.01	0.01\\
536.01	0.01\\
537.01	0.01\\
538.01	0.01\\
539.01	0.01\\
540.01	0.01\\
541.01	0.01\\
542.01	0.01\\
543.01	0.01\\
544.01	0.01\\
545.01	0.01\\
546.01	0.01\\
547.01	0.01\\
548.01	0.01\\
549.01	0.01\\
550.01	0.01\\
551.01	0.01\\
552.01	0.01\\
553.01	0.01\\
554.01	0.01\\
555.01	0.01\\
556.01	0.01\\
557.01	0.01\\
558.01	0.01\\
559.01	0.01\\
560.01	0.01\\
561.01	0.01\\
562.01	0.01\\
563.01	0.01\\
564.01	0.01\\
565.01	0.01\\
566.01	0.01\\
567.01	0.01\\
568.01	0.00972744916310699\\
569.01	0.00942656800190368\\
570.01	0.00910533817373198\\
571.01	0.00876083290411074\\
572.01	0.00838953411780288\\
573.01	0.00799085123339294\\
574.01	0.00757422759961413\\
575.01	0.00713929514049568\\
576.01	0.00668456449300758\\
577.01	0.0062084103816373\\
578.01	0.00570906766945888\\
579.01	0.00518463758754371\\
580.01	0.00463312160520521\\
581.01	0.00405253293841158\\
582.01	0.00351244908859151\\
583.01	0.00326931276710382\\
584.01	0.00303822528933792\\
585.01	0.00281707885798722\\
586.01	0.00259498529778048\\
587.01	0.00237242508948119\\
588.01	0.00215029934370932\\
589.01	0.00192954919884637\\
590.01	0.00171025695550095\\
591.01	0.00149175973833009\\
592.01	0.00127554651852854\\
593.01	0.00106347889233958\\
594.01	0.000857764094569244\\
595.01	0.000660999733626294\\
596.01	0.000476218771598161\\
597.01	0.000306932213913688\\
598.01	0.000157169919638113\\
599.01	4.57135485354369e-05\\
599.02	4.48827722135618e-05\\
599.03	4.40580970898427e-05\\
599.04	4.32395580149871e-05\\
599.05	4.2427190131129e-05\\
599.06	4.16210288745004e-05\\
599.07	4.08211099781359e-05\\
599.08	4.00274694745805e-05\\
599.09	3.92401436986305e-05\\
599.1	3.84591692900969e-05\\
599.11	3.76845831965792e-05\\
599.12	3.69164226762652e-05\\
599.13	3.61547253007622e-05\\
599.14	3.53995289579228e-05\\
599.15	3.46508718547297e-05\\
599.16	3.39087925201615e-05\\
599.17	3.31733298081088e-05\\
599.18	3.24445229002955e-05\\
599.19	3.17224113092294e-05\\
599.2	3.100703488116e-05\\
599.21	3.02984337990763e-05\\
599.22	2.95966485857075e-05\\
599.23	2.89017201065538e-05\\
599.24	2.82136895729344e-05\\
599.25	2.75325985450488e-05\\
599.26	2.68584889344392e-05\\
599.27	2.61914030062933e-05\\
599.28	2.55313833825432e-05\\
599.29	2.48784730449945e-05\\
599.3	2.4232715338468e-05\\
599.31	2.35941539739638e-05\\
599.32	2.29628330318411e-05\\
599.33	2.23387969650098e-05\\
599.34	2.17220906021491e-05\\
599.35	2.11127591509385e-05\\
599.36	2.05108482012935e-05\\
599.37	1.99164037286426e-05\\
599.38	1.93294720971914e-05\\
599.39	1.87501000632195e-05\\
599.4	1.81783347783862e-05\\
599.41	1.76142237930405e-05\\
599.42	1.70578150595693e-05\\
599.43	1.65091569357192e-05\\
599.44	1.59682981879673e-05\\
599.45	1.54352879948774e-05\\
599.46	1.49101760638821e-05\\
599.47	1.43930175409875e-05\\
599.48	1.38838680450361e-05\\
599.49	1.33827836715731e-05\\
599.5	1.28898209967015e-05\\
599.51	1.24050370809564e-05\\
599.52	1.1928489473189e-05\\
599.53	1.14602362144486e-05\\
599.54	1.10003358418636e-05\\
599.55	1.05488473925511e-05\\
599.56	1.01058304074926e-05\\
599.57	9.67134493544235e-06\\
599.58	9.24545153681115e-06\\
599.59	8.82821128755755e-06\\
599.6	8.41968578306831e-06\\
599.61	8.01993714203558e-06\\
599.62	7.62902801032875e-06\\
599.63	7.24702156483861e-06\\
599.64	6.87398151732153e-06\\
599.65	6.50997211822796e-06\\
599.66	6.15505816049279e-06\\
599.67	5.80930498333791e-06\\
599.68	5.47277847600704e-06\\
599.69	5.14554508150751e-06\\
599.7	4.82767180030874e-06\\
599.71	4.51922619398859e-06\\
599.72	4.22027638887629e-06\\
599.73	3.93089107961382e-06\\
599.74	3.65113953270692e-06\\
599.75	3.38109158999275e-06\\
599.76	3.12081767206776e-06\\
599.77	2.87038878166342e-06\\
599.78	2.62987650693179e-06\\
599.79	2.39935302467388e-06\\
599.8	2.17889110350374e-06\\
599.81	1.96856410689117e-06\\
599.82	1.76844599615936e-06\\
599.83	1.57861133334887e-06\\
599.84	1.3991352839967e-06\\
599.85	1.23009361979905e-06\\
599.86	1.07156272114578e-06\\
599.87	9.23619579542082e-07\\
599.88	7.8634179987401e-07\\
599.89	6.59807602540474e-07\\
599.9	5.44095825409999e-07\\
599.91	4.39285925628308e-07\\
599.92	3.45457981228148e-07\\
599.93	2.62692692548291e-07\\
599.94	1.91071383456518e-07\\
599.95	1.30676002336669e-07\\
599.96	8.158912285193e-08\\
599.97	4.38939444496328e-08\\
599.98	1.76742926006473e-08\\
599.99	3.01461875948372e-09\\
600	0\\
};
\addplot [color=mycolor16,solid,forget plot]
  table[row sep=crcr]{%
0.01	0.01\\
1.01	0.01\\
2.01	0.01\\
3.01	0.01\\
4.01	0.01\\
5.01	0.01\\
6.01	0.01\\
7.01	0.01\\
8.01	0.01\\
9.01	0.01\\
10.01	0.01\\
11.01	0.01\\
12.01	0.01\\
13.01	0.01\\
14.01	0.01\\
15.01	0.01\\
16.01	0.01\\
17.01	0.01\\
18.01	0.01\\
19.01	0.01\\
20.01	0.01\\
21.01	0.01\\
22.01	0.01\\
23.01	0.01\\
24.01	0.01\\
25.01	0.01\\
26.01	0.01\\
27.01	0.01\\
28.01	0.01\\
29.01	0.01\\
30.01	0.01\\
31.01	0.01\\
32.01	0.01\\
33.01	0.01\\
34.01	0.01\\
35.01	0.01\\
36.01	0.01\\
37.01	0.01\\
38.01	0.01\\
39.01	0.01\\
40.01	0.01\\
41.01	0.01\\
42.01	0.01\\
43.01	0.01\\
44.01	0.01\\
45.01	0.01\\
46.01	0.01\\
47.01	0.01\\
48.01	0.01\\
49.01	0.01\\
50.01	0.01\\
51.01	0.01\\
52.01	0.01\\
53.01	0.01\\
54.01	0.01\\
55.01	0.01\\
56.01	0.01\\
57.01	0.01\\
58.01	0.01\\
59.01	0.01\\
60.01	0.01\\
61.01	0.01\\
62.01	0.01\\
63.01	0.01\\
64.01	0.01\\
65.01	0.01\\
66.01	0.01\\
67.01	0.01\\
68.01	0.01\\
69.01	0.01\\
70.01	0.01\\
71.01	0.01\\
72.01	0.01\\
73.01	0.01\\
74.01	0.01\\
75.01	0.01\\
76.01	0.01\\
77.01	0.01\\
78.01	0.01\\
79.01	0.01\\
80.01	0.01\\
81.01	0.01\\
82.01	0.01\\
83.01	0.01\\
84.01	0.01\\
85.01	0.01\\
86.01	0.01\\
87.01	0.01\\
88.01	0.01\\
89.01	0.01\\
90.01	0.01\\
91.01	0.01\\
92.01	0.01\\
93.01	0.01\\
94.01	0.01\\
95.01	0.01\\
96.01	0.01\\
97.01	0.01\\
98.01	0.01\\
99.01	0.01\\
100.01	0.01\\
101.01	0.01\\
102.01	0.01\\
103.01	0.01\\
104.01	0.01\\
105.01	0.01\\
106.01	0.01\\
107.01	0.01\\
108.01	0.01\\
109.01	0.01\\
110.01	0.01\\
111.01	0.01\\
112.01	0.01\\
113.01	0.01\\
114.01	0.01\\
115.01	0.01\\
116.01	0.01\\
117.01	0.01\\
118.01	0.01\\
119.01	0.01\\
120.01	0.01\\
121.01	0.01\\
122.01	0.01\\
123.01	0.01\\
124.01	0.01\\
125.01	0.01\\
126.01	0.01\\
127.01	0.01\\
128.01	0.01\\
129.01	0.01\\
130.01	0.01\\
131.01	0.01\\
132.01	0.01\\
133.01	0.01\\
134.01	0.01\\
135.01	0.01\\
136.01	0.01\\
137.01	0.01\\
138.01	0.01\\
139.01	0.01\\
140.01	0.01\\
141.01	0.01\\
142.01	0.01\\
143.01	0.01\\
144.01	0.01\\
145.01	0.01\\
146.01	0.01\\
147.01	0.01\\
148.01	0.01\\
149.01	0.01\\
150.01	0.01\\
151.01	0.01\\
152.01	0.01\\
153.01	0.01\\
154.01	0.01\\
155.01	0.01\\
156.01	0.01\\
157.01	0.01\\
158.01	0.01\\
159.01	0.01\\
160.01	0.01\\
161.01	0.01\\
162.01	0.01\\
163.01	0.01\\
164.01	0.01\\
165.01	0.01\\
166.01	0.01\\
167.01	0.01\\
168.01	0.01\\
169.01	0.01\\
170.01	0.01\\
171.01	0.01\\
172.01	0.01\\
173.01	0.01\\
174.01	0.01\\
175.01	0.01\\
176.01	0.01\\
177.01	0.01\\
178.01	0.01\\
179.01	0.01\\
180.01	0.01\\
181.01	0.01\\
182.01	0.01\\
183.01	0.01\\
184.01	0.01\\
185.01	0.01\\
186.01	0.01\\
187.01	0.01\\
188.01	0.01\\
189.01	0.01\\
190.01	0.01\\
191.01	0.01\\
192.01	0.01\\
193.01	0.01\\
194.01	0.01\\
195.01	0.01\\
196.01	0.01\\
197.01	0.01\\
198.01	0.01\\
199.01	0.01\\
200.01	0.01\\
201.01	0.01\\
202.01	0.01\\
203.01	0.01\\
204.01	0.01\\
205.01	0.01\\
206.01	0.01\\
207.01	0.01\\
208.01	0.01\\
209.01	0.01\\
210.01	0.01\\
211.01	0.01\\
212.01	0.01\\
213.01	0.01\\
214.01	0.01\\
215.01	0.01\\
216.01	0.01\\
217.01	0.01\\
218.01	0.01\\
219.01	0.01\\
220.01	0.01\\
221.01	0.01\\
222.01	0.01\\
223.01	0.01\\
224.01	0.01\\
225.01	0.01\\
226.01	0.01\\
227.01	0.01\\
228.01	0.01\\
229.01	0.01\\
230.01	0.01\\
231.01	0.01\\
232.01	0.01\\
233.01	0.01\\
234.01	0.01\\
235.01	0.01\\
236.01	0.01\\
237.01	0.01\\
238.01	0.01\\
239.01	0.01\\
240.01	0.01\\
241.01	0.01\\
242.01	0.01\\
243.01	0.01\\
244.01	0.01\\
245.01	0.01\\
246.01	0.01\\
247.01	0.01\\
248.01	0.01\\
249.01	0.01\\
250.01	0.01\\
251.01	0.01\\
252.01	0.01\\
253.01	0.01\\
254.01	0.01\\
255.01	0.01\\
256.01	0.01\\
257.01	0.01\\
258.01	0.01\\
259.01	0.01\\
260.01	0.01\\
261.01	0.01\\
262.01	0.01\\
263.01	0.01\\
264.01	0.01\\
265.01	0.01\\
266.01	0.01\\
267.01	0.01\\
268.01	0.01\\
269.01	0.01\\
270.01	0.01\\
271.01	0.01\\
272.01	0.01\\
273.01	0.01\\
274.01	0.01\\
275.01	0.01\\
276.01	0.01\\
277.01	0.01\\
278.01	0.01\\
279.01	0.01\\
280.01	0.01\\
281.01	0.01\\
282.01	0.01\\
283.01	0.01\\
284.01	0.01\\
285.01	0.01\\
286.01	0.01\\
287.01	0.01\\
288.01	0.01\\
289.01	0.01\\
290.01	0.01\\
291.01	0.01\\
292.01	0.01\\
293.01	0.01\\
294.01	0.01\\
295.01	0.01\\
296.01	0.01\\
297.01	0.01\\
298.01	0.01\\
299.01	0.01\\
300.01	0.01\\
301.01	0.01\\
302.01	0.01\\
303.01	0.01\\
304.01	0.01\\
305.01	0.01\\
306.01	0.01\\
307.01	0.01\\
308.01	0.01\\
309.01	0.01\\
310.01	0.01\\
311.01	0.01\\
312.01	0.01\\
313.01	0.01\\
314.01	0.01\\
315.01	0.01\\
316.01	0.01\\
317.01	0.01\\
318.01	0.01\\
319.01	0.01\\
320.01	0.01\\
321.01	0.01\\
322.01	0.01\\
323.01	0.01\\
324.01	0.01\\
325.01	0.01\\
326.01	0.01\\
327.01	0.01\\
328.01	0.01\\
329.01	0.01\\
330.01	0.01\\
331.01	0.01\\
332.01	0.01\\
333.01	0.01\\
334.01	0.01\\
335.01	0.01\\
336.01	0.01\\
337.01	0.01\\
338.01	0.01\\
339.01	0.01\\
340.01	0.01\\
341.01	0.01\\
342.01	0.01\\
343.01	0.01\\
344.01	0.01\\
345.01	0.01\\
346.01	0.01\\
347.01	0.01\\
348.01	0.01\\
349.01	0.01\\
350.01	0.01\\
351.01	0.01\\
352.01	0.01\\
353.01	0.01\\
354.01	0.01\\
355.01	0.01\\
356.01	0.01\\
357.01	0.01\\
358.01	0.01\\
359.01	0.01\\
360.01	0.01\\
361.01	0.01\\
362.01	0.01\\
363.01	0.01\\
364.01	0.01\\
365.01	0.01\\
366.01	0.01\\
367.01	0.01\\
368.01	0.01\\
369.01	0.01\\
370.01	0.01\\
371.01	0.01\\
372.01	0.01\\
373.01	0.01\\
374.01	0.01\\
375.01	0.01\\
376.01	0.01\\
377.01	0.01\\
378.01	0.01\\
379.01	0.01\\
380.01	0.01\\
381.01	0.01\\
382.01	0.01\\
383.01	0.01\\
384.01	0.01\\
385.01	0.01\\
386.01	0.01\\
387.01	0.01\\
388.01	0.01\\
389.01	0.01\\
390.01	0.01\\
391.01	0.01\\
392.01	0.01\\
393.01	0.01\\
394.01	0.01\\
395.01	0.01\\
396.01	0.01\\
397.01	0.01\\
398.01	0.01\\
399.01	0.01\\
400.01	0.01\\
401.01	0.01\\
402.01	0.01\\
403.01	0.01\\
404.01	0.01\\
405.01	0.01\\
406.01	0.01\\
407.01	0.01\\
408.01	0.01\\
409.01	0.01\\
410.01	0.01\\
411.01	0.01\\
412.01	0.01\\
413.01	0.01\\
414.01	0.01\\
415.01	0.01\\
416.01	0.01\\
417.01	0.01\\
418.01	0.01\\
419.01	0.01\\
420.01	0.01\\
421.01	0.01\\
422.01	0.01\\
423.01	0.01\\
424.01	0.01\\
425.01	0.01\\
426.01	0.01\\
427.01	0.01\\
428.01	0.01\\
429.01	0.01\\
430.01	0.01\\
431.01	0.01\\
432.01	0.01\\
433.01	0.01\\
434.01	0.01\\
435.01	0.01\\
436.01	0.01\\
437.01	0.01\\
438.01	0.01\\
439.01	0.01\\
440.01	0.01\\
441.01	0.01\\
442.01	0.01\\
443.01	0.01\\
444.01	0.01\\
445.01	0.01\\
446.01	0.01\\
447.01	0.01\\
448.01	0.01\\
449.01	0.01\\
450.01	0.01\\
451.01	0.01\\
452.01	0.01\\
453.01	0.01\\
454.01	0.01\\
455.01	0.01\\
456.01	0.01\\
457.01	0.01\\
458.01	0.01\\
459.01	0.01\\
460.01	0.01\\
461.01	0.01\\
462.01	0.01\\
463.01	0.01\\
464.01	0.01\\
465.01	0.01\\
466.01	0.01\\
467.01	0.01\\
468.01	0.01\\
469.01	0.01\\
470.01	0.01\\
471.01	0.01\\
472.01	0.01\\
473.01	0.01\\
474.01	0.01\\
475.01	0.01\\
476.01	0.01\\
477.01	0.01\\
478.01	0.01\\
479.01	0.01\\
480.01	0.01\\
481.01	0.01\\
482.01	0.01\\
483.01	0.01\\
484.01	0.01\\
485.01	0.01\\
486.01	0.01\\
487.01	0.01\\
488.01	0.01\\
489.01	0.01\\
490.01	0.01\\
491.01	0.01\\
492.01	0.01\\
493.01	0.01\\
494.01	0.01\\
495.01	0.01\\
496.01	0.01\\
497.01	0.01\\
498.01	0.01\\
499.01	0.01\\
500.01	0.01\\
501.01	0.01\\
502.01	0.01\\
503.01	0.01\\
504.01	0.01\\
505.01	0.01\\
506.01	0.01\\
507.01	0.01\\
508.01	0.01\\
509.01	0.01\\
510.01	0.01\\
511.01	0.01\\
512.01	0.01\\
513.01	0.01\\
514.01	0.01\\
515.01	0.01\\
516.01	0.01\\
517.01	0.01\\
518.01	0.01\\
519.01	0.01\\
520.01	0.01\\
521.01	0.01\\
522.01	0.01\\
523.01	0.01\\
524.01	0.01\\
525.01	0.01\\
526.01	0.01\\
527.01	0.01\\
528.01	0.01\\
529.01	0.01\\
530.01	0.01\\
531.01	0.01\\
532.01	0.01\\
533.01	0.01\\
534.01	0.01\\
535.01	0.01\\
536.01	0.01\\
537.01	0.01\\
538.01	0.01\\
539.01	0.01\\
540.01	0.01\\
541.01	0.01\\
542.01	0.01\\
543.01	0.01\\
544.01	0.01\\
545.01	0.01\\
546.01	0.01\\
547.01	0.01\\
548.01	0.01\\
549.01	0.01\\
550.01	0.00995817214987554\\
551.01	0.00976920720650903\\
552.01	0.00957068161494685\\
553.01	0.00936157648640411\\
554.01	0.00914070877388854\\
555.01	0.00890669871536501\\
556.01	0.00865793000953965\\
557.01	0.00839250036389308\\
558.01	0.00810815851763091\\
559.01	0.00780222450397486\\
560.01	0.00747301733874219\\
561.01	0.00712763658868991\\
562.01	0.00676663990071911\\
563.01	0.00638887747094233\\
564.01	0.00599300805243266\\
565.01	0.00557739361006792\\
566.01	0.00513978932111589\\
567.01	0.0046775026391738\\
568.01	0.00446437522935085\\
569.01	0.00425959342731657\\
570.01	0.00405456668876184\\
571.01	0.00385186976766412\\
572.01	0.00365507220738019\\
573.01	0.00346541136082133\\
574.01	0.00327522104056992\\
575.01	0.00308519632578407\\
576.01	0.00289654874182385\\
577.01	0.00271074026193477\\
578.01	0.00252951755813305\\
579.01	0.00235494432006656\\
580.01	0.00218942634918826\\
581.01	0.00203571962313973\\
582.01	0.00189569368076647\\
583.01	0.00176221935349623\\
584.01	0.00163117395006303\\
585.01	0.00150194649748184\\
586.01	0.00137451254708096\\
587.01	0.00124919829444105\\
588.01	0.001126282592439\\
589.01	0.00100599016945428\\
590.01	0.000888534916031865\\
591.01	0.00077433810196983\\
592.01	0.000663891166351075\\
593.01	0.000557617301030397\\
594.01	0.000455831854103557\\
595.01	0.000358696458436331\\
596.01	0.000266167733632852\\
597.01	0.000177943768206052\\
598.01	9.42984683149437e-05\\
599.01	2.97433029915143e-05\\
599.02	2.92255074340608e-05\\
599.03	2.87109129547599e-05\\
599.04	2.81995475120318e-05\\
599.05	2.76914393515584e-05\\
599.06	2.71866170091662e-05\\
599.07	2.66851093137094e-05\\
599.08	2.61869453900155e-05\\
599.09	2.56921546618414e-05\\
599.1	2.52007668548517e-05\\
599.11	2.47128119996477e-05\\
599.12	2.42283204348152e-05\\
599.13	2.37473228099933e-05\\
599.14	2.32698500889934e-05\\
599.15	2.2795933552925e-05\\
599.16	2.23256048033898e-05\\
599.17	2.18588957656557e-05\\
599.18	2.13958386919115e-05\\
599.19	2.09364661645194e-05\\
599.2	2.04808110993264e-05\\
599.21	2.00289067489832e-05\\
599.22	1.95807867063164e-05\\
599.23	1.91364849077228e-05\\
599.24	1.86960356366034e-05\\
599.25	1.82594735268233e-05\\
599.26	1.78268349628249e-05\\
599.27	1.73981584822864e-05\\
599.28	1.69734830200441e-05\\
599.29	1.65528479120682e-05\\
599.3	1.61362928994564e-05\\
599.31	1.57238581324892e-05\\
599.32	1.53155841747174e-05\\
599.33	1.49115120070802e-05\\
599.34	1.45116830320754e-05\\
599.35	1.41161390779781e-05\\
599.36	1.37249224030859e-05\\
599.37	1.33380757000104e-05\\
599.38	1.29556421000227e-05\\
599.39	1.2577665177432e-05\\
599.4	1.22041889540053e-05\\
599.41	1.18352579034451e-05\\
599.42	1.14709169558962e-05\\
599.43	1.11112115025128e-05\\
599.44	1.07561874000597e-05\\
599.45	1.04058909755692e-05\\
599.46	1.00603690309786e-05\\
599.47	9.7196688453368e-06\\
599.48	9.38383817956912e-06\\
599.49	9.05292528129154e-06\\
599.5	8.72697888967475e-06\\
599.51	8.40604824036036e-06\\
599.52	8.09018307042046e-06\\
599.53	7.77943362337274e-06\\
599.54	7.47385065425279e-06\\
599.55	7.17348543472461e-06\\
599.56	6.87838975826048e-06\\
599.57	6.58861594535556e-06\\
599.58	6.30421684882053e-06\\
599.59	6.02524585910542e-06\\
599.6	5.75175690969466e-06\\
599.61	5.4838044825558e-06\\
599.62	5.22144361363858e-06\\
599.63	4.96472989843823e-06\\
599.64	4.71371949762288e-06\\
599.65	4.46846914270785e-06\\
599.66	4.22903614179931e-06\\
599.67	3.9954783853987e-06\\
599.68	3.76785435226429e-06\\
599.69	3.54622311534575e-06\\
599.7	3.33064434777409e-06\\
599.71	3.12117832892804e-06\\
599.72	2.91788595054021e-06\\
599.73	2.72082872291787e-06\\
599.74	2.53006878117752e-06\\
599.75	2.34566889159918e-06\\
599.76	2.16769245801711e-06\\
599.77	1.99620352829381e-06\\
599.78	1.83126680087728e-06\\
599.79	1.67294763142416e-06\\
599.8	1.52131203949059e-06\\
599.81	1.37642671532712e-06\\
599.82	1.23835902672391e-06\\
599.83	1.10717702595832e-06\\
599.84	9.82949456815319e-07\\
599.85	8.65745761698122e-07\\
599.86	7.55636088823827e-07\\
599.87	6.52691299497105e-07\\
599.88	5.56982975498388e-07\\
599.89	4.6858342654145e-07\\
599.9	3.87565697841999e-07\\
599.91	3.14003577771282e-07\\
599.92	2.47971605622441e-07\\
599.93	1.89545079472275e-07\\
599.94	1.38800064147099e-07\\
599.95	9.58133993030769e-08\\
599.96	6.06627076158578e-08\\
599.97	3.34264030846937e-08\\
599.98	1.4183699454523e-08\\
599.99	3.01461875948372e-09\\
600	0\\
};
\addplot [color=mycolor17,solid,forget plot]
  table[row sep=crcr]{%
0.01	0.01\\
1.01	0.01\\
2.01	0.01\\
3.01	0.01\\
4.01	0.01\\
5.01	0.01\\
6.01	0.01\\
7.01	0.01\\
8.01	0.01\\
9.01	0.01\\
10.01	0.01\\
11.01	0.01\\
12.01	0.01\\
13.01	0.01\\
14.01	0.01\\
15.01	0.01\\
16.01	0.01\\
17.01	0.01\\
18.01	0.01\\
19.01	0.01\\
20.01	0.01\\
21.01	0.01\\
22.01	0.01\\
23.01	0.01\\
24.01	0.01\\
25.01	0.01\\
26.01	0.01\\
27.01	0.01\\
28.01	0.01\\
29.01	0.01\\
30.01	0.01\\
31.01	0.01\\
32.01	0.01\\
33.01	0.01\\
34.01	0.01\\
35.01	0.01\\
36.01	0.01\\
37.01	0.01\\
38.01	0.01\\
39.01	0.01\\
40.01	0.01\\
41.01	0.01\\
42.01	0.01\\
43.01	0.01\\
44.01	0.01\\
45.01	0.01\\
46.01	0.01\\
47.01	0.01\\
48.01	0.01\\
49.01	0.01\\
50.01	0.01\\
51.01	0.01\\
52.01	0.01\\
53.01	0.01\\
54.01	0.01\\
55.01	0.01\\
56.01	0.01\\
57.01	0.01\\
58.01	0.01\\
59.01	0.01\\
60.01	0.01\\
61.01	0.01\\
62.01	0.01\\
63.01	0.01\\
64.01	0.01\\
65.01	0.01\\
66.01	0.01\\
67.01	0.01\\
68.01	0.01\\
69.01	0.01\\
70.01	0.01\\
71.01	0.01\\
72.01	0.01\\
73.01	0.01\\
74.01	0.01\\
75.01	0.01\\
76.01	0.01\\
77.01	0.01\\
78.01	0.01\\
79.01	0.01\\
80.01	0.01\\
81.01	0.01\\
82.01	0.01\\
83.01	0.01\\
84.01	0.01\\
85.01	0.01\\
86.01	0.01\\
87.01	0.01\\
88.01	0.01\\
89.01	0.01\\
90.01	0.01\\
91.01	0.01\\
92.01	0.01\\
93.01	0.01\\
94.01	0.01\\
95.01	0.01\\
96.01	0.01\\
97.01	0.01\\
98.01	0.01\\
99.01	0.01\\
100.01	0.01\\
101.01	0.01\\
102.01	0.01\\
103.01	0.01\\
104.01	0.01\\
105.01	0.01\\
106.01	0.01\\
107.01	0.01\\
108.01	0.01\\
109.01	0.01\\
110.01	0.01\\
111.01	0.01\\
112.01	0.01\\
113.01	0.01\\
114.01	0.01\\
115.01	0.01\\
116.01	0.01\\
117.01	0.01\\
118.01	0.01\\
119.01	0.01\\
120.01	0.01\\
121.01	0.01\\
122.01	0.01\\
123.01	0.01\\
124.01	0.01\\
125.01	0.01\\
126.01	0.01\\
127.01	0.01\\
128.01	0.01\\
129.01	0.01\\
130.01	0.01\\
131.01	0.01\\
132.01	0.01\\
133.01	0.01\\
134.01	0.01\\
135.01	0.01\\
136.01	0.01\\
137.01	0.01\\
138.01	0.01\\
139.01	0.01\\
140.01	0.01\\
141.01	0.01\\
142.01	0.01\\
143.01	0.01\\
144.01	0.01\\
145.01	0.01\\
146.01	0.01\\
147.01	0.01\\
148.01	0.01\\
149.01	0.01\\
150.01	0.01\\
151.01	0.01\\
152.01	0.01\\
153.01	0.01\\
154.01	0.01\\
155.01	0.01\\
156.01	0.01\\
157.01	0.01\\
158.01	0.01\\
159.01	0.01\\
160.01	0.01\\
161.01	0.01\\
162.01	0.01\\
163.01	0.01\\
164.01	0.01\\
165.01	0.01\\
166.01	0.01\\
167.01	0.01\\
168.01	0.01\\
169.01	0.01\\
170.01	0.01\\
171.01	0.01\\
172.01	0.01\\
173.01	0.01\\
174.01	0.01\\
175.01	0.01\\
176.01	0.01\\
177.01	0.01\\
178.01	0.01\\
179.01	0.01\\
180.01	0.01\\
181.01	0.01\\
182.01	0.01\\
183.01	0.01\\
184.01	0.01\\
185.01	0.01\\
186.01	0.01\\
187.01	0.01\\
188.01	0.01\\
189.01	0.01\\
190.01	0.01\\
191.01	0.01\\
192.01	0.01\\
193.01	0.01\\
194.01	0.01\\
195.01	0.01\\
196.01	0.01\\
197.01	0.01\\
198.01	0.01\\
199.01	0.01\\
200.01	0.01\\
201.01	0.01\\
202.01	0.01\\
203.01	0.01\\
204.01	0.01\\
205.01	0.01\\
206.01	0.01\\
207.01	0.01\\
208.01	0.01\\
209.01	0.01\\
210.01	0.01\\
211.01	0.01\\
212.01	0.01\\
213.01	0.01\\
214.01	0.01\\
215.01	0.01\\
216.01	0.01\\
217.01	0.01\\
218.01	0.01\\
219.01	0.01\\
220.01	0.01\\
221.01	0.01\\
222.01	0.01\\
223.01	0.01\\
224.01	0.01\\
225.01	0.01\\
226.01	0.01\\
227.01	0.01\\
228.01	0.01\\
229.01	0.01\\
230.01	0.01\\
231.01	0.01\\
232.01	0.01\\
233.01	0.01\\
234.01	0.01\\
235.01	0.01\\
236.01	0.01\\
237.01	0.01\\
238.01	0.01\\
239.01	0.01\\
240.01	0.01\\
241.01	0.01\\
242.01	0.01\\
243.01	0.01\\
244.01	0.01\\
245.01	0.01\\
246.01	0.01\\
247.01	0.01\\
248.01	0.01\\
249.01	0.01\\
250.01	0.01\\
251.01	0.01\\
252.01	0.01\\
253.01	0.01\\
254.01	0.01\\
255.01	0.01\\
256.01	0.01\\
257.01	0.01\\
258.01	0.01\\
259.01	0.01\\
260.01	0.01\\
261.01	0.01\\
262.01	0.01\\
263.01	0.01\\
264.01	0.01\\
265.01	0.01\\
266.01	0.01\\
267.01	0.01\\
268.01	0.01\\
269.01	0.01\\
270.01	0.01\\
271.01	0.01\\
272.01	0.01\\
273.01	0.01\\
274.01	0.01\\
275.01	0.01\\
276.01	0.01\\
277.01	0.01\\
278.01	0.01\\
279.01	0.01\\
280.01	0.01\\
281.01	0.01\\
282.01	0.01\\
283.01	0.01\\
284.01	0.01\\
285.01	0.01\\
286.01	0.01\\
287.01	0.01\\
288.01	0.01\\
289.01	0.01\\
290.01	0.01\\
291.01	0.01\\
292.01	0.01\\
293.01	0.01\\
294.01	0.01\\
295.01	0.01\\
296.01	0.01\\
297.01	0.01\\
298.01	0.01\\
299.01	0.01\\
300.01	0.01\\
301.01	0.01\\
302.01	0.01\\
303.01	0.01\\
304.01	0.01\\
305.01	0.01\\
306.01	0.01\\
307.01	0.01\\
308.01	0.01\\
309.01	0.01\\
310.01	0.01\\
311.01	0.01\\
312.01	0.01\\
313.01	0.01\\
314.01	0.01\\
315.01	0.01\\
316.01	0.01\\
317.01	0.01\\
318.01	0.01\\
319.01	0.01\\
320.01	0.01\\
321.01	0.01\\
322.01	0.01\\
323.01	0.01\\
324.01	0.01\\
325.01	0.01\\
326.01	0.01\\
327.01	0.01\\
328.01	0.01\\
329.01	0.01\\
330.01	0.01\\
331.01	0.01\\
332.01	0.01\\
333.01	0.01\\
334.01	0.01\\
335.01	0.01\\
336.01	0.01\\
337.01	0.01\\
338.01	0.01\\
339.01	0.01\\
340.01	0.01\\
341.01	0.01\\
342.01	0.01\\
343.01	0.01\\
344.01	0.01\\
345.01	0.01\\
346.01	0.01\\
347.01	0.01\\
348.01	0.01\\
349.01	0.01\\
350.01	0.01\\
351.01	0.01\\
352.01	0.01\\
353.01	0.01\\
354.01	0.01\\
355.01	0.01\\
356.01	0.01\\
357.01	0.01\\
358.01	0.01\\
359.01	0.01\\
360.01	0.01\\
361.01	0.01\\
362.01	0.01\\
363.01	0.01\\
364.01	0.01\\
365.01	0.01\\
366.01	0.01\\
367.01	0.01\\
368.01	0.01\\
369.01	0.01\\
370.01	0.01\\
371.01	0.01\\
372.01	0.01\\
373.01	0.01\\
374.01	0.01\\
375.01	0.01\\
376.01	0.01\\
377.01	0.01\\
378.01	0.01\\
379.01	0.01\\
380.01	0.01\\
381.01	0.01\\
382.01	0.01\\
383.01	0.01\\
384.01	0.01\\
385.01	0.01\\
386.01	0.01\\
387.01	0.01\\
388.01	0.01\\
389.01	0.01\\
390.01	0.01\\
391.01	0.01\\
392.01	0.01\\
393.01	0.01\\
394.01	0.01\\
395.01	0.01\\
396.01	0.01\\
397.01	0.01\\
398.01	0.01\\
399.01	0.01\\
400.01	0.01\\
401.01	0.01\\
402.01	0.01\\
403.01	0.01\\
404.01	0.01\\
405.01	0.01\\
406.01	0.01\\
407.01	0.01\\
408.01	0.01\\
409.01	0.01\\
410.01	0.01\\
411.01	0.01\\
412.01	0.01\\
413.01	0.01\\
414.01	0.01\\
415.01	0.01\\
416.01	0.01\\
417.01	0.01\\
418.01	0.01\\
419.01	0.01\\
420.01	0.01\\
421.01	0.01\\
422.01	0.01\\
423.01	0.01\\
424.01	0.01\\
425.01	0.01\\
426.01	0.01\\
427.01	0.01\\
428.01	0.01\\
429.01	0.01\\
430.01	0.01\\
431.01	0.01\\
432.01	0.01\\
433.01	0.01\\
434.01	0.01\\
435.01	0.01\\
436.01	0.01\\
437.01	0.01\\
438.01	0.01\\
439.01	0.01\\
440.01	0.01\\
441.01	0.01\\
442.01	0.01\\
443.01	0.01\\
444.01	0.01\\
445.01	0.01\\
446.01	0.01\\
447.01	0.01\\
448.01	0.01\\
449.01	0.01\\
450.01	0.01\\
451.01	0.01\\
452.01	0.01\\
453.01	0.01\\
454.01	0.01\\
455.01	0.01\\
456.01	0.01\\
457.01	0.01\\
458.01	0.01\\
459.01	0.01\\
460.01	0.01\\
461.01	0.01\\
462.01	0.01\\
463.01	0.01\\
464.01	0.01\\
465.01	0.01\\
466.01	0.01\\
467.01	0.01\\
468.01	0.01\\
469.01	0.01\\
470.01	0.01\\
471.01	0.01\\
472.01	0.01\\
473.01	0.01\\
474.01	0.01\\
475.01	0.01\\
476.01	0.01\\
477.01	0.01\\
478.01	0.01\\
479.01	0.01\\
480.01	0.01\\
481.01	0.01\\
482.01	0.01\\
483.01	0.01\\
484.01	0.01\\
485.01	0.01\\
486.01	0.01\\
487.01	0.01\\
488.01	0.01\\
489.01	0.01\\
490.01	0.01\\
491.01	0.01\\
492.01	0.01\\
493.01	0.01\\
494.01	0.01\\
495.01	0.01\\
496.01	0.01\\
497.01	0.01\\
498.01	0.01\\
499.01	0.01\\
500.01	0.01\\
501.01	0.01\\
502.01	0.01\\
503.01	0.01\\
504.01	0.01\\
505.01	0.01\\
506.01	0.01\\
507.01	0.01\\
508.01	0.01\\
509.01	0.01\\
510.01	0.01\\
511.01	0.01\\
512.01	0.01\\
513.01	0.01\\
514.01	0.01\\
515.01	0.01\\
516.01	0.01\\
517.01	0.01\\
518.01	0.01\\
519.01	0.01\\
520.01	0.01\\
521.01	0.01\\
522.01	0.01\\
523.01	0.01\\
524.01	0.01\\
525.01	0.01\\
526.01	0.01\\
527.01	0.0099241290970491\\
528.01	0.0098175255165026\\
529.01	0.0097068930893769\\
530.01	0.00959196270396107\\
531.01	0.00947243515181103\\
532.01	0.00934797596358879\\
533.01	0.00921820807741072\\
534.01	0.00908269772191878\\
535.01	0.00894095105251903\\
536.01	0.00879240950820449\\
537.01	0.00863643967948476\\
538.01	0.00847232140769048\\
539.01	0.00829923383611116\\
540.01	0.00811623904417037\\
541.01	0.00792226282616251\\
542.01	0.00771606180810179\\
543.01	0.00749606450831156\\
544.01	0.00726030556795162\\
545.01	0.0070079519472063\\
546.01	0.00674388486960165\\
547.01	0.00646787322976985\\
548.01	0.00617876222036522\\
549.01	0.00587518142076962\\
550.01	0.00559746032953064\\
551.01	0.00545298715323179\\
552.01	0.00530500792545318\\
553.01	0.00515390011393403\\
554.01	0.00500019249536979\\
555.01	0.0048446042748905\\
556.01	0.00468808397530697\\
557.01	0.00453190880521706\\
558.01	0.00437781468838097\\
559.01	0.00422814498614577\\
560.01	0.00408450061202286\\
561.01	0.00393959397258884\\
562.01	0.0037924112109357\\
563.01	0.00364377063441716\\
564.01	0.00349484266244423\\
565.01	0.00334828400578513\\
566.01	0.0032068365243211\\
567.01	0.00307305339271171\\
568.01	0.00294563612557173\\
569.01	0.00282111130465503\\
570.01	0.00270007470687016\\
571.01	0.00258295637848159\\
572.01	0.00246985659109791\\
573.01	0.00236031264984525\\
574.01	0.00225264487338503\\
575.01	0.00214649048949318\\
576.01	0.00204212467871798\\
577.01	0.00193976321903908\\
578.01	0.00183953410827614\\
579.01	0.00174144548087779\\
580.01	0.00164535244959273\\
581.01	0.00155092827496037\\
582.01	0.00145768949973365\\
583.01	0.0013652994855999\\
584.01	0.00127372346543224\\
585.01	0.0011829917503887\\
586.01	0.00109316523826044\\
587.01	0.00100429568118757\\
588.01	0.00091642184966745\\
589.01	0.000829576198374293\\
590.01	0.000743791048770606\\
591.01	0.000659086383332572\\
592.01	0.000575443770475836\\
593.01	0.000492800020771381\\
594.01	0.000411047558901598\\
595.01	0.000330038786353392\\
596.01	0.000249595277727267\\
597.01	0.000169522162970183\\
598.01	9.18966740472912e-05\\
599.01	2.94706190942413e-05\\
599.02	2.89609613509639e-05\\
599.03	2.84543437073619e-05\\
599.04	2.79507956110585e-05\\
599.05	2.74503468013707e-05\\
599.06	2.69530273121939e-05\\
599.07	2.64588674749062e-05\\
599.08	2.59678979212998e-05\\
599.09	2.54801495865476e-05\\
599.1	2.49956537121988e-05\\
599.11	2.45144418491942e-05\\
599.12	2.40365458609258e-05\\
599.13	2.35619979263146e-05\\
599.14	2.30908305429295e-05\\
599.15	2.26230765301254e-05\\
599.16	2.21587690322159e-05\\
599.17	2.1697941521695e-05\\
599.18	2.12406278024529e-05\\
599.19	2.07868620130686e-05\\
599.2	2.03366786300958e-05\\
599.21	1.98901124714036e-05\\
599.22	1.9447198699547e-05\\
599.23	1.90079728251708e-05\\
599.24	1.85724707104321e-05\\
599.25	1.81407285724869e-05\\
599.26	1.77127848684563e-05\\
599.27	1.72886796869724e-05\\
599.28	1.68684535171004e-05\\
599.29	1.64521472522886e-05\\
599.3	1.60398021943637e-05\\
599.31	1.56314600575445e-05\\
599.32	1.52271629725205e-05\\
599.33	1.48269534905581e-05\\
599.34	1.44308745876499e-05\\
599.35	1.40389696686987e-05\\
599.36	1.36512825717559e-05\\
599.37	1.32678575722866e-05\\
599.38	1.28887393874896e-05\\
599.39	1.25139731806528e-05\\
599.4	1.21436045655531e-05\\
599.41	1.17776796109033e-05\\
599.42	1.14162448448338e-05\\
599.43	1.10593472594303e-05\\
599.44	1.07070343153062e-05\\
599.45	1.03593539462262e-05\\
599.46	1.00163545637792e-05\\
599.47	9.67808506207625e-06\\
599.48	9.34459482253812e-06\\
599.49	9.01593371867813e-06\\
599.5	8.69215212097318e-06\\
599.51	8.37330090176433e-06\\
599.52	8.05943144020597e-06\\
599.53	7.75059562726881e-06\\
599.54	7.44684587079478e-06\\
599.55	7.14823510058853e-06\\
599.56	6.85481677357644e-06\\
599.57	6.56664487900041e-06\\
599.58	6.28377394367753e-06\\
599.59	6.00625903730313e-06\\
599.6	5.73415577780415e-06\\
599.61	5.46752033675491e-06\\
599.62	5.20640944483286e-06\\
599.63	4.9508803973454e-06\\
599.64	4.70099105979137e-06\\
599.65	4.45679987349373e-06\\
599.66	4.21836586128252e-06\\
599.67	3.98574863322981e-06\\
599.68	3.75900839244374e-06\\
599.69	3.53820594092488e-06\\
599.7	3.32340268546956e-06\\
599.71	3.11466064364246e-06\\
599.72	2.9120424498031e-06\\
599.73	2.71561136118426e-06\\
599.74	2.52543126405373e-06\\
599.75	2.34156667990038e-06\\
599.76	2.16408277171551e-06\\
599.77	1.99304535031947e-06\\
599.78	1.82852088075058e-06\\
599.79	1.67057648871489e-06\\
599.8	1.51927996711911e-06\\
599.81	1.37469978262958e-06\\
599.82	1.23690508234062e-06\\
599.83	1.10596570046875e-06\\
599.84	9.81952165138994e-07\\
599.85	8.64935705228304e-07\\
599.86	7.54988257262862e-07\\
599.87	6.5218247242288e-07\\
599.88	5.56591723556085e-07\\
599.89	4.6829011231958e-07\\
599.9	3.87352476345984e-07\\
599.91	3.13854396508453e-07\\
599.92	2.47872204241217e-07\\
599.93	1.89482988934703e-07\\
599.94	1.38764605403519e-07\\
599.95	9.57956814255645e-08\\
599.96	6.06556253574669e-08\\
599.97	3.34246338211386e-08\\
599.98	1.4183699454523e-08\\
599.99	3.01461875948372e-09\\
600	0\\
};
\addplot [color=mycolor18,solid,forget plot]
  table[row sep=crcr]{%
0.01	0.01\\
1.01	0.01\\
2.01	0.01\\
3.01	0.01\\
4.01	0.01\\
5.01	0.01\\
6.01	0.01\\
7.01	0.01\\
8.01	0.01\\
9.01	0.01\\
10.01	0.01\\
11.01	0.01\\
12.01	0.01\\
13.01	0.01\\
14.01	0.01\\
15.01	0.01\\
16.01	0.01\\
17.01	0.01\\
18.01	0.01\\
19.01	0.01\\
20.01	0.01\\
21.01	0.01\\
22.01	0.01\\
23.01	0.01\\
24.01	0.01\\
25.01	0.01\\
26.01	0.01\\
27.01	0.01\\
28.01	0.01\\
29.01	0.01\\
30.01	0.01\\
31.01	0.01\\
32.01	0.01\\
33.01	0.01\\
34.01	0.01\\
35.01	0.01\\
36.01	0.01\\
37.01	0.01\\
38.01	0.01\\
39.01	0.01\\
40.01	0.01\\
41.01	0.01\\
42.01	0.01\\
43.01	0.01\\
44.01	0.01\\
45.01	0.01\\
46.01	0.01\\
47.01	0.01\\
48.01	0.01\\
49.01	0.01\\
50.01	0.01\\
51.01	0.01\\
52.01	0.01\\
53.01	0.01\\
54.01	0.01\\
55.01	0.01\\
56.01	0.01\\
57.01	0.01\\
58.01	0.01\\
59.01	0.01\\
60.01	0.01\\
61.01	0.01\\
62.01	0.01\\
63.01	0.01\\
64.01	0.01\\
65.01	0.01\\
66.01	0.01\\
67.01	0.01\\
68.01	0.01\\
69.01	0.01\\
70.01	0.01\\
71.01	0.01\\
72.01	0.01\\
73.01	0.01\\
74.01	0.01\\
75.01	0.01\\
76.01	0.01\\
77.01	0.01\\
78.01	0.01\\
79.01	0.01\\
80.01	0.01\\
81.01	0.01\\
82.01	0.01\\
83.01	0.01\\
84.01	0.01\\
85.01	0.01\\
86.01	0.01\\
87.01	0.01\\
88.01	0.01\\
89.01	0.01\\
90.01	0.01\\
91.01	0.01\\
92.01	0.01\\
93.01	0.01\\
94.01	0.01\\
95.01	0.01\\
96.01	0.01\\
97.01	0.01\\
98.01	0.01\\
99.01	0.01\\
100.01	0.01\\
101.01	0.01\\
102.01	0.01\\
103.01	0.01\\
104.01	0.01\\
105.01	0.01\\
106.01	0.01\\
107.01	0.01\\
108.01	0.01\\
109.01	0.01\\
110.01	0.01\\
111.01	0.01\\
112.01	0.01\\
113.01	0.01\\
114.01	0.01\\
115.01	0.01\\
116.01	0.01\\
117.01	0.01\\
118.01	0.01\\
119.01	0.01\\
120.01	0.01\\
121.01	0.01\\
122.01	0.01\\
123.01	0.01\\
124.01	0.01\\
125.01	0.01\\
126.01	0.01\\
127.01	0.01\\
128.01	0.01\\
129.01	0.01\\
130.01	0.01\\
131.01	0.01\\
132.01	0.01\\
133.01	0.01\\
134.01	0.01\\
135.01	0.01\\
136.01	0.01\\
137.01	0.01\\
138.01	0.01\\
139.01	0.01\\
140.01	0.01\\
141.01	0.01\\
142.01	0.01\\
143.01	0.01\\
144.01	0.01\\
145.01	0.01\\
146.01	0.01\\
147.01	0.01\\
148.01	0.01\\
149.01	0.01\\
150.01	0.01\\
151.01	0.01\\
152.01	0.01\\
153.01	0.01\\
154.01	0.01\\
155.01	0.01\\
156.01	0.01\\
157.01	0.01\\
158.01	0.01\\
159.01	0.01\\
160.01	0.01\\
161.01	0.01\\
162.01	0.01\\
163.01	0.01\\
164.01	0.01\\
165.01	0.01\\
166.01	0.01\\
167.01	0.01\\
168.01	0.01\\
169.01	0.01\\
170.01	0.01\\
171.01	0.01\\
172.01	0.01\\
173.01	0.01\\
174.01	0.01\\
175.01	0.01\\
176.01	0.01\\
177.01	0.01\\
178.01	0.01\\
179.01	0.01\\
180.01	0.01\\
181.01	0.01\\
182.01	0.01\\
183.01	0.01\\
184.01	0.01\\
185.01	0.01\\
186.01	0.01\\
187.01	0.01\\
188.01	0.01\\
189.01	0.01\\
190.01	0.01\\
191.01	0.01\\
192.01	0.01\\
193.01	0.01\\
194.01	0.01\\
195.01	0.01\\
196.01	0.01\\
197.01	0.01\\
198.01	0.01\\
199.01	0.01\\
200.01	0.01\\
201.01	0.01\\
202.01	0.01\\
203.01	0.01\\
204.01	0.01\\
205.01	0.01\\
206.01	0.01\\
207.01	0.01\\
208.01	0.01\\
209.01	0.01\\
210.01	0.01\\
211.01	0.01\\
212.01	0.01\\
213.01	0.01\\
214.01	0.01\\
215.01	0.01\\
216.01	0.01\\
217.01	0.01\\
218.01	0.01\\
219.01	0.01\\
220.01	0.01\\
221.01	0.01\\
222.01	0.01\\
223.01	0.01\\
224.01	0.01\\
225.01	0.01\\
226.01	0.01\\
227.01	0.01\\
228.01	0.01\\
229.01	0.01\\
230.01	0.01\\
231.01	0.01\\
232.01	0.01\\
233.01	0.01\\
234.01	0.01\\
235.01	0.01\\
236.01	0.01\\
237.01	0.01\\
238.01	0.01\\
239.01	0.01\\
240.01	0.01\\
241.01	0.01\\
242.01	0.01\\
243.01	0.01\\
244.01	0.01\\
245.01	0.01\\
246.01	0.01\\
247.01	0.01\\
248.01	0.01\\
249.01	0.01\\
250.01	0.01\\
251.01	0.01\\
252.01	0.01\\
253.01	0.01\\
254.01	0.01\\
255.01	0.01\\
256.01	0.01\\
257.01	0.01\\
258.01	0.01\\
259.01	0.01\\
260.01	0.01\\
261.01	0.01\\
262.01	0.01\\
263.01	0.01\\
264.01	0.01\\
265.01	0.01\\
266.01	0.01\\
267.01	0.01\\
268.01	0.01\\
269.01	0.01\\
270.01	0.01\\
271.01	0.01\\
272.01	0.01\\
273.01	0.01\\
274.01	0.01\\
275.01	0.01\\
276.01	0.01\\
277.01	0.01\\
278.01	0.01\\
279.01	0.01\\
280.01	0.01\\
281.01	0.01\\
282.01	0.01\\
283.01	0.01\\
284.01	0.01\\
285.01	0.01\\
286.01	0.01\\
287.01	0.01\\
288.01	0.01\\
289.01	0.01\\
290.01	0.01\\
291.01	0.01\\
292.01	0.01\\
293.01	0.01\\
294.01	0.01\\
295.01	0.01\\
296.01	0.01\\
297.01	0.01\\
298.01	0.01\\
299.01	0.01\\
300.01	0.01\\
301.01	0.01\\
302.01	0.01\\
303.01	0.01\\
304.01	0.01\\
305.01	0.01\\
306.01	0.01\\
307.01	0.01\\
308.01	0.01\\
309.01	0.01\\
310.01	0.01\\
311.01	0.01\\
312.01	0.01\\
313.01	0.01\\
314.01	0.01\\
315.01	0.01\\
316.01	0.01\\
317.01	0.01\\
318.01	0.01\\
319.01	0.01\\
320.01	0.01\\
321.01	0.01\\
322.01	0.01\\
323.01	0.01\\
324.01	0.01\\
325.01	0.01\\
326.01	0.01\\
327.01	0.01\\
328.01	0.01\\
329.01	0.01\\
330.01	0.01\\
331.01	0.01\\
332.01	0.01\\
333.01	0.01\\
334.01	0.01\\
335.01	0.01\\
336.01	0.01\\
337.01	0.01\\
338.01	0.01\\
339.01	0.01\\
340.01	0.01\\
341.01	0.01\\
342.01	0.01\\
343.01	0.01\\
344.01	0.01\\
345.01	0.01\\
346.01	0.01\\
347.01	0.01\\
348.01	0.01\\
349.01	0.01\\
350.01	0.01\\
351.01	0.01\\
352.01	0.01\\
353.01	0.01\\
354.01	0.01\\
355.01	0.01\\
356.01	0.01\\
357.01	0.01\\
358.01	0.01\\
359.01	0.01\\
360.01	0.01\\
361.01	0.01\\
362.01	0.01\\
363.01	0.01\\
364.01	0.01\\
365.01	0.01\\
366.01	0.01\\
367.01	0.01\\
368.01	0.01\\
369.01	0.01\\
370.01	0.01\\
371.01	0.01\\
372.01	0.01\\
373.01	0.01\\
374.01	0.01\\
375.01	0.01\\
376.01	0.01\\
377.01	0.01\\
378.01	0.01\\
379.01	0.01\\
380.01	0.01\\
381.01	0.01\\
382.01	0.01\\
383.01	0.01\\
384.01	0.01\\
385.01	0.01\\
386.01	0.01\\
387.01	0.01\\
388.01	0.01\\
389.01	0.01\\
390.01	0.01\\
391.01	0.01\\
392.01	0.01\\
393.01	0.01\\
394.01	0.01\\
395.01	0.01\\
396.01	0.01\\
397.01	0.01\\
398.01	0.01\\
399.01	0.01\\
400.01	0.01\\
401.01	0.01\\
402.01	0.01\\
403.01	0.01\\
404.01	0.01\\
405.01	0.01\\
406.01	0.01\\
407.01	0.01\\
408.01	0.01\\
409.01	0.01\\
410.01	0.01\\
411.01	0.01\\
412.01	0.01\\
413.01	0.01\\
414.01	0.01\\
415.01	0.01\\
416.01	0.01\\
417.01	0.01\\
418.01	0.01\\
419.01	0.01\\
420.01	0.01\\
421.01	0.01\\
422.01	0.01\\
423.01	0.01\\
424.01	0.01\\
425.01	0.01\\
426.01	0.01\\
427.01	0.01\\
428.01	0.01\\
429.01	0.01\\
430.01	0.01\\
431.01	0.01\\
432.01	0.01\\
433.01	0.01\\
434.01	0.01\\
435.01	0.01\\
436.01	0.01\\
437.01	0.01\\
438.01	0.01\\
439.01	0.01\\
440.01	0.01\\
441.01	0.01\\
442.01	0.01\\
443.01	0.01\\
444.01	0.01\\
445.01	0.01\\
446.01	0.01\\
447.01	0.01\\
448.01	0.01\\
449.01	0.01\\
450.01	0.01\\
451.01	0.01\\
452.01	0.01\\
453.01	0.01\\
454.01	0.01\\
455.01	0.01\\
456.01	0.01\\
457.01	0.01\\
458.01	0.01\\
459.01	0.01\\
460.01	0.01\\
461.01	0.01\\
462.01	0.01\\
463.01	0.01\\
464.01	0.01\\
465.01	0.01\\
466.01	0.01\\
467.01	0.01\\
468.01	0.01\\
469.01	0.01\\
470.01	0.01\\
471.01	0.01\\
472.01	0.01\\
473.01	0.01\\
474.01	0.01\\
475.01	0.01\\
476.01	0.01\\
477.01	0.01\\
478.01	0.01\\
479.01	0.01\\
480.01	0.01\\
481.01	0.01\\
482.01	0.00997587649458391\\
483.01	0.00994281436826373\\
484.01	0.00990865576868689\\
485.01	0.00987335571535595\\
486.01	0.00983686864441155\\
487.01	0.00979914916206342\\
488.01	0.00976015312757781\\
489.01	0.0097198391823294\\
490.01	0.0096781708806394\\
491.01	0.0096351196302989\\
492.01	0.00959066872002735\\
493.01	0.00954481854877986\\
494.01	0.00949756230674207\\
495.01	0.00944884687152364\\
496.01	0.00939861087884548\\
497.01	0.00934679062273646\\
498.01	0.00929332052445587\\
499.01	0.0092381338739154\\
500.01	0.00918116394599725\\
501.01	0.00912234562935636\\
502.01	0.00906161775233871\\
503.01	0.00899892635353021\\
504.01	0.00893422402364727\\
505.01	0.00886742921998204\\
506.01	0.00879843256617422\\
507.01	0.00872711390830633\\
508.01	0.00865334091708237\\
509.01	0.00857696723504866\\
510.01	0.00849783027010523\\
511.01	0.00841574855662225\\
512.01	0.00833051858576653\\
513.01	0.00824191098148276\\
514.01	0.00814966586632992\\
515.01	0.00805348721986368\\
516.01	0.00795303597859557\\
517.01	0.00784792155686998\\
518.01	0.00773769137706917\\
519.01	0.00762181787816733\\
520.01	0.00749968231209421\\
521.01	0.00737055443836958\\
522.01	0.00723356698045552\\
523.01	0.00708768334921067\\
524.01	0.00693165666092587\\
525.01	0.00676423745727685\\
526.01	0.00658790083961235\\
527.01	0.00647971681111041\\
528.01	0.00639593571072038\\
529.01	0.00630965465279673\\
530.01	0.00622083657043316\\
531.01	0.00612945588072651\\
532.01	0.00603550315539981\\
533.01	0.00593899257231927\\
534.01	0.00583997738401568\\
535.01	0.00573855520496804\\
536.01	0.00563487457077089\\
537.01	0.00552914787384191\\
538.01	0.00542166736493526\\
539.01	0.00531282489136661\\
540.01	0.00520313623617997\\
541.01	0.00509327112131359\\
542.01	0.00498410056830288\\
543.01	0.00487687893575851\\
544.01	0.00477334731136774\\
545.01	0.00467422638434592\\
546.01	0.0045743314764834\\
547.01	0.00447337191168162\\
548.01	0.00437198627604522\\
549.01	0.00427105996371826\\
550.01	0.00417166426615998\\
551.01	0.00407132678206081\\
552.01	0.00396880225878627\\
553.01	0.00386438111845274\\
554.01	0.00375849222447324\\
555.01	0.00365203488305964\\
556.01	0.0035466544291015\\
557.01	0.00344282389172102\\
558.01	0.00334084018973639\\
559.01	0.00324086386752923\\
560.01	0.00314283492651795\\
561.01	0.00304683398660745\\
562.01	0.00295326966246949\\
563.01	0.00286249569317403\\
564.01	0.00277470624413143\\
565.01	0.00268886161041842\\
566.01	0.00260428765967366\\
567.01	0.00252097709498086\\
568.01	0.00243880615864478\\
569.01	0.00235772436845609\\
570.01	0.00227765771045678\\
571.01	0.00219845346857509\\
572.01	0.00211989051998918\\
573.01	0.0020417157060724\\
574.01	0.00196376477038438\\
575.01	0.0018860263484256\\
576.01	0.00180848634643038\\
577.01	0.00173110876709557\\
578.01	0.00165383891558031\\
579.01	0.00157660995977223\\
580.01	0.00149935332576446\\
581.01	0.00142201270423878\\
582.01	0.00134455890324395\\
583.01	0.00126698852963796\\
584.01	0.00118930562924654\\
585.01	0.00111151429061311\\
586.01	0.00103361667856622\\
587.01	0.000955612541206028\\
588.01	0.000877500028226702\\
589.01	0.000799275845239126\\
590.01	0.000720934182363075\\
591.01	0.000642465092883062\\
592.01	0.000563855076711632\\
593.01	0.00048508951620752\\
594.01	0.000406155478895408\\
595.01	0.000327044352021353\\
596.01	0.000247753662207873\\
597.01	0.000168287229632226\\
598.01	9.17536481397484e-05\\
599.01	2.94669794711974e-05\\
599.02	2.89574678339153e-05\\
599.03	2.84509918799507e-05\\
599.04	2.79475811456025e-05\\
599.05	2.74472654580235e-05\\
599.06	2.69500749380813e-05\\
599.07	2.64560400032848e-05\\
599.08	2.59651913706981e-05\\
599.09	2.54775600599076e-05\\
599.1	2.49931773960189e-05\\
599.11	2.45120750126723e-05\\
599.12	2.40342848550935e-05\\
599.13	2.35598391831755e-05\\
599.14	2.30887705745895e-05\\
599.15	2.26211119279309e-05\\
599.16	2.21568964658882e-05\\
599.17	2.16961577384437e-05\\
599.18	2.12389296261259e-05\\
599.19	2.07852463432537e-05\\
599.2	2.03351424412582e-05\\
599.21	1.98886528120085e-05\\
599.22	1.94458126911764e-05\\
599.23	1.90066576616355e-05\\
599.24	1.85712236569005e-05\\
599.25	1.81395469645886e-05\\
599.26	1.77116661135811e-05\\
599.27	1.72876212607056e-05\\
599.28	1.68674529623658e-05\\
599.29	1.64512021784951e-05\\
599.3	1.60389102765428e-05\\
599.31	1.56306190355025e-05\\
599.32	1.52263706499779e-05\\
599.33	1.48262077342837e-05\\
599.34	1.44301733266006e-05\\
599.35	1.40383108931594e-05\\
599.36	1.36506643324697e-05\\
599.37	1.3267277979595e-05\\
599.38	1.28881966104562e-05\\
599.39	1.25134654462011e-05\\
599.4	1.21431301575905e-05\\
599.41	1.17772368694487e-05\\
599.42	1.14158321651448e-05\\
599.43	1.10589630911238e-05\\
599.44	1.07066771614876e-05\\
599.45	1.03590223626028e-05\\
599.46	1.00160471577775e-05\\
599.47	9.67780049196745e-06\\
599.48	9.3443317965361e-06\\
599.49	9.01569099405995e-06\\
599.5	8.6919285031805e-06\\
599.51	8.37309524350481e-06\\
599.52	8.05924264055125e-06\\
599.53	7.75042263075933e-06\\
599.54	7.44668766651697e-06\\
599.55	7.14809072127447e-06\\
599.56	6.85468529468269e-06\\
599.57	6.56652541779733e-06\\
599.58	6.28366565832289e-06\\
599.59	6.00616112590886e-06\\
599.6	5.73406747752039e-06\\
599.61	5.46744092282461e-06\\
599.62	5.20633822966195e-06\\
599.63	4.95081672955734e-06\\
599.64	4.70093432328351e-06\\
599.65	4.45674948649534e-06\\
599.66	4.21832127539025e-06\\
599.67	3.98570933245884e-06\\
599.68	3.75897389226325e-06\\
599.69	3.53817578729702e-06\\
599.7	3.32337645388148e-06\\
599.71	3.11463793812965e-06\\
599.72	2.9120229019762e-06\\
599.73	2.71559462925938e-06\\
599.74	2.52541703184977e-06\\
599.75	2.34155465586756e-06\\
599.76	2.16407268794142e-06\\
599.77	1.99303696153501e-06\\
599.78	1.82851396333256e-06\\
599.79	1.67057083969718e-06\\
599.8	1.51927540317613e-06\\
599.81	1.37469613908231e-06\\
599.82	1.23690221214869e-06\\
599.83	1.10596347322606e-06\\
599.84	9.81950466062351e-07\\
599.85	8.64934434142636e-07\\
599.86	7.54987327612408e-07\\
599.87	6.52181810235561e-07\\
599.88	5.56591266461653e-07\\
599.89	4.68289808524397e-07\\
599.9	3.87352283644227e-07\\
599.91	3.13854281279446e-07\\
599.92	2.47872140451966e-07\\
599.93	1.89482957158038e-07\\
599.94	1.38764591836246e-07\\
599.95	9.57956769204876e-08\\
599.96	6.06556244623496e-08\\
599.97	3.34246338194039e-08\\
599.98	1.4183699454523e-08\\
599.99	3.014618757749e-09\\
600	0\\
};
\addplot [color=red!25!mycolor17,solid,forget plot]
  table[row sep=crcr]{%
0.01	0.00894151321774357\\
1.01	0.00894151261464888\\
2.01	0.00894151199851174\\
3.01	0.00894151136904826\\
4.01	0.00894151072596831\\
5.01	0.00894151006897544\\
6.01	0.00894150939776663\\
7.01	0.0089415087120323\\
8.01	0.00894150801145603\\
9.01	0.00894150729571435\\
10.01	0.00894150656447688\\
11.01	0.00894150581740579\\
12.01	0.00894150505415598\\
13.01	0.00894150427437465\\
14.01	0.00894150347770131\\
15.01	0.00894150266376757\\
16.01	0.00894150183219687\\
17.01	0.00894150098260437\\
18.01	0.00894150011459686\\
19.01	0.00894149922777238\\
20.01	0.00894149832172012\\
21.01	0.00894149739602027\\
22.01	0.00894149645024375\\
23.01	0.00894149548395206\\
24.01	0.00894149449669706\\
25.01	0.00894149348802066\\
26.01	0.0089414924574547\\
27.01	0.00894149140452075\\
28.01	0.00894149032872973\\
29.01	0.00894148922958186\\
30.01	0.00894148810656636\\
31.01	0.00894148695916115\\
32.01	0.00894148578683254\\
33.01	0.00894148458903512\\
34.01	0.00894148336521152\\
35.01	0.00894148211479195\\
36.01	0.00894148083719403\\
37.01	0.00894147953182262\\
38.01	0.00894147819806935\\
39.01	0.00894147683531239\\
40.01	0.00894147544291619\\
41.01	0.00894147402023112\\
42.01	0.00894147256659323\\
43.01	0.0089414710813238\\
44.01	0.00894146956372919\\
45.01	0.00894146801310022\\
46.01	0.00894146642871228\\
47.01	0.00894146480982446\\
48.01	0.00894146315567963\\
49.01	0.00894146146550371\\
50.01	0.00894145973850553\\
51.01	0.00894145797387646\\
52.01	0.00894145617078981\\
53.01	0.00894145432840061\\
54.01	0.00894145244584514\\
55.01	0.0089414505222405\\
56.01	0.00894144855668418\\
57.01	0.00894144654825368\\
58.01	0.00894144449600596\\
59.01	0.00894144239897699\\
60.01	0.00894144025618145\\
61.01	0.00894143806661196\\
62.01	0.00894143582923881\\
63.01	0.00894143354300944\\
64.01	0.00894143120684771\\
65.01	0.00894142881965376\\
66.01	0.00894142638030302\\
67.01	0.00894142388764601\\
68.01	0.00894142134050764\\
69.01	0.00894141873768662\\
70.01	0.00894141607795485\\
71.01	0.00894141336005689\\
72.01	0.0089414105827093\\
73.01	0.00894140774460003\\
74.01	0.00894140484438768\\
75.01	0.00894140188070104\\
76.01	0.00894139885213815\\
77.01	0.00894139575726586\\
78.01	0.00894139259461894\\
79.01	0.00894138936269947\\
80.01	0.00894138605997596\\
81.01	0.00894138268488271\\
82.01	0.00894137923581905\\
83.01	0.0089413757111485\\
84.01	0.00894137210919787\\
85.01	0.00894136842825662\\
86.01	0.00894136466657578\\
87.01	0.00894136082236734\\
88.01	0.0089413568938032\\
89.01	0.00894135287901419\\
90.01	0.00894134877608929\\
91.01	0.00894134458307466\\
92.01	0.00894134029797261\\
93.01	0.0089413359187405\\
94.01	0.00894133144328999\\
95.01	0.00894132686948578\\
96.01	0.00894132219514458\\
97.01	0.00894131741803402\\
98.01	0.00894131253587157\\
99.01	0.00894130754632342\\
100.01	0.00894130244700323\\
101.01	0.00894129723547087\\
102.01	0.00894129190923148\\
103.01	0.00894128646573383\\
104.01	0.00894128090236932\\
105.01	0.00894127521647054\\
106.01	0.00894126940530998\\
107.01	0.0089412634660986\\
108.01	0.00894125739598443\\
109.01	0.0089412511920512\\
110.01	0.00894124485131678\\
111.01	0.00894123837073165\\
112.01	0.00894123174717752\\
113.01	0.00894122497746555\\
114.01	0.0089412180583348\\
115.01	0.0089412109864507\\
116.01	0.00894120375840312\\
117.01	0.00894119637070482\\
118.01	0.00894118881978959\\
119.01	0.00894118110201047\\
120.01	0.00894117321363788\\
121.01	0.00894116515085766\\
122.01	0.00894115690976925\\
123.01	0.00894114848638355\\
124.01	0.00894113987662105\\
125.01	0.0089411310763095\\
126.01	0.00894112208118201\\
127.01	0.00894111288687472\\
128.01	0.00894110348892462\\
129.01	0.0089410938827672\\
130.01	0.0089410840637342\\
131.01	0.00894107402705106\\
132.01	0.0089410637678346\\
133.01	0.0089410532810905\\
134.01	0.00894104256171054\\
135.01	0.00894103160447012\\
136.01	0.00894102040402564\\
137.01	0.00894100895491155\\
138.01	0.00894099725153761\\
139.01	0.00894098528818587\\
140.01	0.00894097305900795\\
141.01	0.0089409605580218\\
142.01	0.00894094777910862\\
143.01	0.00894093471600977\\
144.01	0.00894092136232341\\
145.01	0.00894090771150117\\
146.01	0.00894089375684481\\
147.01	0.00894087949150265\\
148.01	0.00894086490846605\\
149.01	0.00894085000056567\\
150.01	0.00894083476046775\\
151.01	0.00894081918067024\\
152.01	0.00894080325349894\\
153.01	0.00894078697110343\\
154.01	0.00894077032545292\\
155.01	0.00894075330833198\\
156.01	0.00894073591133631\\
157.01	0.0089407181258683\\
158.01	0.00894069994313242\\
159.01	0.00894068135413064\\
160.01	0.00894066234965762\\
161.01	0.00894064292029593\\
162.01	0.00894062305641088\\
163.01	0.00894060274814567\\
164.01	0.00894058198541587\\
165.01	0.00894056075790433\\
166.01	0.00894053905505555\\
167.01	0.00894051686606997\\
168.01	0.00894049417989846\\
169.01	0.00894047098523631\\
170.01	0.00894044727051697\\
171.01	0.00894042302390632\\
172.01	0.00894039823329595\\
173.01	0.00894037288629684\\
174.01	0.0089403469702327\\
175.01	0.00894032047213323\\
176.01	0.00894029337872692\\
177.01	0.00894026567643424\\
178.01	0.00894023735136012\\
179.01	0.00894020838928645\\
180.01	0.00894017877566452\\
181.01	0.00894014849560717\\
182.01	0.00894011753388064\\
183.01	0.00894008587489632\\
184.01	0.0089400535027025\\
185.01	0.00894002040097549\\
186.01	0.00893998655301091\\
187.01	0.00893995194171456\\
188.01	0.00893991654959314\\
189.01	0.00893988035874466\\
190.01	0.00893984335084885\\
191.01	0.00893980550715699\\
192.01	0.00893976680848176\\
193.01	0.00893972723518686\\
194.01	0.00893968676717593\\
195.01	0.0089396453838819\\
196.01	0.00893960306425557\\
197.01	0.00893955978675409\\
198.01	0.00893951552932906\\
199.01	0.00893947026941452\\
200.01	0.00893942398391432\\
201.01	0.00893937664918967\\
202.01	0.00893932824104591\\
203.01	0.00893927873471925\\
204.01	0.00893922810486304\\
205.01	0.00893917632553366\\
206.01	0.00893912337017641\\
207.01	0.00893906921161051\\
208.01	0.00893901382201416\\
209.01	0.00893895717290917\\
210.01	0.0089388992351449\\
211.01	0.00893883997888221\\
212.01	0.00893877937357691\\
213.01	0.00893871738796242\\
214.01	0.00893865399003264\\
215.01	0.00893858914702397\\
216.01	0.00893852282539686\\
217.01	0.00893845499081734\\
218.01	0.0089383856081374\\
219.01	0.00893831464137563\\
220.01	0.00893824205369682\\
221.01	0.00893816780739137\\
222.01	0.0089380918638539\\
223.01	0.00893801418356186\\
224.01	0.00893793472605301\\
225.01	0.00893785344990261\\
226.01	0.00893777031270021\\
227.01	0.00893768527102556\\
228.01	0.00893759828042407\\
229.01	0.00893750929538178\\
230.01	0.00893741826929937\\
231.01	0.00893732515446611\\
232.01	0.00893722990203235\\
233.01	0.00893713246198218\\
234.01	0.00893703278310474\\
235.01	0.00893693081296528\\
236.01	0.00893682649787527\\
237.01	0.00893671978286185\\
238.01	0.00893661061163641\\
239.01	0.00893649892656259\\
240.01	0.0089363846686233\\
241.01	0.00893626777738715\\
242.01	0.00893614819097369\\
243.01	0.00893602584601828\\
244.01	0.00893590067763559\\
245.01	0.00893577261938262\\
246.01	0.00893564160322045\\
247.01	0.00893550755947528\\
248.01	0.00893537041679852\\
249.01	0.0089352301021258\\
250.01	0.00893508654063487\\
251.01	0.00893493965570274\\
252.01	0.00893478936886143\\
253.01	0.00893463559975303\\
254.01	0.00893447826608312\\
255.01	0.00893431728357363\\
256.01	0.00893415256591417\\
257.01	0.00893398402471211\\
258.01	0.00893381156944177\\
259.01	0.00893363510739197\\
260.01	0.00893345454361264\\
261.01	0.00893326978085985\\
262.01	0.00893308071953978\\
263.01	0.00893288725765067\\
264.01	0.00893268929072435\\
265.01	0.00893248671176548\\
266.01	0.00893227941118956\\
267.01	0.0089320672767597\\
268.01	0.00893185019352134\\
269.01	0.00893162804373585\\
270.01	0.00893140070681221\\
271.01	0.008931168059237\\
272.01	0.00893092997450288\\
273.01	0.00893068632303518\\
274.01	0.00893043697211676\\
275.01	0.00893018178581092\\
276.01	0.00892992062488261\\
277.01	0.00892965334671749\\
278.01	0.00892937980523929\\
279.01	0.00892909985082486\\
280.01	0.00892881333021744\\
281.01	0.00892852008643756\\
282.01	0.00892821995869189\\
283.01	0.00892791278227995\\
284.01	0.0089275983884984\\
285.01	0.00892727660454309\\
286.01	0.0089269472534087\\
287.01	0.0089266101537859\\
288.01	0.00892626511995626\\
289.01	0.00892591196168419\\
290.01	0.00892555048410659\\
291.01	0.00892518048761965\\
292.01	0.00892480176776308\\
293.01	0.00892441411510132\\
294.01	0.00892401731510211\\
295.01	0.00892361114801182\\
296.01	0.00892319538872798\\
297.01	0.00892276980666865\\
298.01	0.00892233416563886\\
299.01	0.00892188822369321\\
300.01	0.00892143173299576\\
301.01	0.00892096443967636\\
302.01	0.00892048608368312\\
303.01	0.00891999639863201\\
304.01	0.00891949511165199\\
305.01	0.00891898194322718\\
306.01	0.00891845660703449\\
307.01	0.00891791880977794\\
308.01	0.00891736825101831\\
309.01	0.00891680462299919\\
310.01	0.00891622761046853\\
311.01	0.00891563689049579\\
312.01	0.00891503213228483\\
313.01	0.00891441299698199\\
314.01	0.00891377913747952\\
315.01	0.0089131301982144\\
316.01	0.00891246581496189\\
317.01	0.00891178561462424\\
318.01	0.00891108921501403\\
319.01	0.00891037622463211\\
320.01	0.00890964624244024\\
321.01	0.00890889885762785\\
322.01	0.00890813364937315\\
323.01	0.008907350186598\\
324.01	0.00890654802771699\\
325.01	0.00890572672037978\\
326.01	0.00890488580120724\\
327.01	0.00890402479552079\\
328.01	0.00890314321706468\\
329.01	0.00890224056772141\\
330.01	0.00890131633721959\\
331.01	0.00890037000283452\\
332.01	0.00889940102908055\\
333.01	0.00889840886739597\\
334.01	0.0088973929558192\\
335.01	0.00889635271865654\\
336.01	0.00889528756614137\\
337.01	0.00889419689408379\\
338.01	0.00889308008351116\\
339.01	0.00889193650029912\\
340.01	0.00889076549479213\\
341.01	0.00888956640141393\\
342.01	0.00888833853826738\\
343.01	0.00888708120672304\\
344.01	0.00888579369099642\\
345.01	0.00888447525771335\\
346.01	0.00888312515546309\\
347.01	0.00888174261433902\\
348.01	0.0088803268454657\\
349.01	0.00887887704051307\\
350.01	0.00887739237119596\\
351.01	0.00887587198875959\\
352.01	0.00887431502344956\\
353.01	0.0088727205839668\\
354.01	0.00887108775690598\\
355.01	0.0088694156061774\\
356.01	0.00886770317241179\\
357.01	0.00886594947234695\\
358.01	0.00886415349819599\\
359.01	0.00886231421699673\\
360.01	0.00886043056994076\\
361.01	0.00885850147168265\\
362.01	0.00885652580962747\\
363.01	0.00885450244319672\\
364.01	0.00885243020307152\\
365.01	0.00885030789041235\\
366.01	0.00884813427605492\\
367.01	0.00884590809968062\\
368.01	0.00884362806896159\\
369.01	0.00884129285867894\\
370.01	0.00883890110981364\\
371.01	0.008836451428609\\
372.01	0.00883394238560412\\
373.01	0.00883137251463696\\
374.01	0.00882874031181643\\
375.01	0.00882604423446269\\
376.01	0.00882328270001422\\
377.01	0.00882045408490118\\
378.01	0.00881755672338373\\
379.01	0.00881458890635412\\
380.01	0.00881154888010224\\
381.01	0.00880843484504269\\
382.01	0.00880524495440249\\
383.01	0.00880197731286887\\
384.01	0.00879862997519489\\
385.01	0.00879520094476274\\
386.01	0.0087916881721026\\
387.01	0.00878808955336624\\
388.01	0.00878440292875404\\
389.01	0.0087806260808934\\
390.01	0.00877675673316796\\
391.01	0.00877279254799538\\
392.01	0.00876873112505233\\
393.01	0.00876456999944523\\
394.01	0.00876030663982492\\
395.01	0.00875593844644316\\
396.01	0.00875146274914971\\
397.01	0.00874687680532766\\
398.01	0.00874217779776539\\
399.01	0.0087373628324628\\
400.01	0.00873242893637046\\
401.01	0.00872737305505869\\
402.01	0.00872219205031533\\
403.01	0.00871688269766945\\
404.01	0.00871144168383938\\
405.01	0.0087058656041022\\
406.01	0.0087001509595828\\
407.01	0.00869429415446006\\
408.01	0.00868829149308723\\
409.01	0.00868213917702424\\
410.01	0.00867583330197849\\
411.01	0.00866936985465104\\
412.01	0.00866274470948433\\
413.01	0.0086559536253075\\
414.01	0.00864899224187483\\
415.01	0.00864185607629205\\
416.01	0.00863454051932611\\
417.01	0.00862704083159274\\
418.01	0.00861935213961748\\
419.01	0.00861146943176589\\
420.01	0.00860338755403901\\
421.01	0.00859510120573041\\
422.01	0.00858660493493794\\
423.01	0.00857789313391867\\
424.01	0.0085689600342695\\
425.01	0.00855979970191195\\
426.01	0.00855040603185129\\
427.01	0.00854077274267304\\
428.01	0.00853089337072811\\
429.01	0.00852076126394522\\
430.01	0.00851036957519363\\
431.01	0.00849971125510028\\
432.01	0.0084887790442035\\
433.01	0.00847756546430114\\
434.01	0.00846606280882325\\
435.01	0.00845426313203126\\
436.01	0.00844215823681791\\
437.01	0.00842973966085852\\
438.01	0.00841699866085121\\
439.01	0.00840392619458725\\
440.01	0.00839051290063101\\
441.01	0.00837674907547543\\
442.01	0.00836262464820484\\
443.01	0.00834812915298423\\
444.01	0.00833325170015707\\
445.01	0.00831798094745767\\
446.01	0.00830230507491702\\
447.01	0.00828621177179482\\
448.01	0.00826968821763505\\
449.01	0.00825272104924269\\
450.01	0.00823529632331078\\
451.01	0.00821739947509464\\
452.01	0.00819901527257593\\
453.01	0.00818012776546474\\
454.01	0.00816072022827056\\
455.01	0.00814077509653561\\
456.01	0.00812027389515372\\
457.01	0.00809919715749363\\
458.01	0.00807752433379312\\
459.01	0.00805523368698338\\
460.01	0.00803230217372145\\
461.01	0.00800870530793988\\
462.01	0.00798441700363633\\
463.01	0.00795940939289735\\
464.01	0.00793365261423636\\
465.01	0.00790711456517616\\
466.01	0.0078797606115569\\
467.01	0.00785155324421342\\
468.01	0.00782245167133338\\
469.01	0.00779241133183461\\
470.01	0.00776138331130186\\
471.01	0.0077293136371557\\
472.01	0.00769614242347331\\
473.01	0.00766180282783474\\
474.01	0.00762621977218363\\
475.01	0.00758930836627548\\
476.01	0.00755097195491648\\
477.01	0.00751109968769789\\
478.01	0.00746956348057245\\
479.01	0.00742621419992225\\
480.01	0.00738087684906441\\
481.01	0.00733334447066516\\
482.01	0.00730772838409598\\
483.01	0.00728921539375883\\
484.01	0.00726986604526977\\
485.01	0.0072495895299529\\
486.01	0.00722827672678798\\
487.01	0.00720579559218599\\
488.01	0.00718198527132034\\
489.01	0.00715664855337426\\
490.01	0.00712954217595951\\
491.01	0.00710036432905067\\
492.01	0.00706873850339288\\
493.01	0.00703426993795644\\
494.01	0.00699822049880224\\
495.01	0.006961198131097\\
496.01	0.00692317937685264\\
497.01	0.00688413949676074\\
498.01	0.00684405178854448\\
499.01	0.00680288658077105\\
500.01	0.00676060976968188\\
501.01	0.00671718071535282\\
502.01	0.00667254924184988\\
503.01	0.00662665139217403\\
504.01	0.0065794087358317\\
505.01	0.00653076848116818\\
506.01	0.00648069800907385\\
507.01	0.00642916762080523\\
508.01	0.00637615143522\\
509.01	0.00632162873443743\\
510.01	0.00626558567286279\\
511.01	0.00620801745221732\\
512.01	0.00614893109665368\\
513.01	0.00608834900380557\\
514.01	0.00602631350328169\\
515.01	0.00596289272854014\\
516.01	0.00589818820786404\\
517.01	0.00583234471434273\\
518.01	0.00576556309567665\\
519.01	0.00569811707126496\\
520.01	0.00563037536487652\\
521.01	0.00556283014420392\\
522.01	0.00549613352115219\\
523.01	0.0054311450531315\\
524.01	0.00536899375642301\\
525.01	0.00531089743934325\\
526.01	0.00525433400642232\\
527.01	0.00519790383802797\\
528.01	0.00513996246353109\\
529.01	0.00508042133421524\\
530.01	0.00501928811296119\\
531.01	0.0049565761873247\\
532.01	0.00489230154984199\\
533.01	0.00482647312224399\\
534.01	0.00475908001949976\\
535.01	0.00469011082236774\\
536.01	0.00461955953879067\\
537.01	0.0045474267640274\\
538.01	0.00447372112767133\\
539.01	0.00439846114084963\\
540.01	0.00432167761323271\\
541.01	0.00424341689448369\\
542.01	0.00416374521879202\\
543.01	0.00408275070957415\\
544.01	0.00400054871057547\\
545.01	0.00391793968658502\\
546.01	0.00383545025872607\\
547.01	0.00375332823407465\\
548.01	0.00367184062787143\\
549.01	0.0035912528713037\\
550.01	0.00351180585965046\\
551.01	0.0034337427002255\\
552.01	0.00335737485324567\\
553.01	0.00328298039254646\\
554.01	0.00321074134821009\\
555.01	0.00314037875271919\\
556.01	0.00307077499893151\\
557.01	0.00300188263613312\\
558.01	0.00293373027196458\\
559.01	0.00286631379211223\\
560.01	0.00279960126959131\\
561.01	0.00273353630284404\\
562.01	0.0026680150261463\\
563.01	0.0026028766296019\\
564.01	0.00253791314161076\\
565.01	0.00247294065430676\\
566.01	0.00240790287881855\\
567.01	0.00234275431907017\\
568.01	0.00227743915861326\\
569.01	0.00221189383001435\\
570.01	0.00214604795923127\\
571.01	0.00207983098516851\\
572.01	0.00201318079937052\\
573.01	0.00194605206981758\\
574.01	0.0018784185498138\\
575.01	0.00181025995554062\\
576.01	0.00174155473393376\\
577.01	0.0016722815447536\\
578.01	0.00160242126650577\\
579.01	0.00153195886904355\\
580.01	0.00146088474238344\\
581.01	0.00138919499581868\\
582.01	0.0013168903329978\\
583.01	0.00124397413150743\\
584.01	0.00117045174146673\\
585.01	0.00109633076541331\\
586.01	0.00102162149322053\\
587.01	0.000946337383085976\\
588.01	0.000870495457990504\\
589.01	0.000794116593283583\\
590.01	0.00071722578280283\\
591.01	0.000639852503519138\\
592.01	0.00056203108043296\\
593.01	0.000483800710259085\\
594.01	0.000405204886350921\\
595.01	0.000326290009844911\\
596.01	0.000247102975742894\\
597.01	0.000167687548540458\\
598.01	9.17367679171607e-05\\
599.01	2.94669369274633e-05\\
599.02	2.89574274290674e-05\\
599.03	2.84509535274195e-05\\
599.04	2.7947544761462e-05\\
599.05	2.7447230960936e-05\\
599.06	2.69500422492523e-05\\
599.07	2.64560090463949e-05\\
599.08	2.5965162071857e-05\\
599.09	2.54775323476081e-05\\
599.1	2.49931512010769e-05\\
599.11	2.45120502681759e-05\\
599.12	2.40342614963567e-05\\
599.13	2.35598171476892e-05\\
599.14	2.30887498019697e-05\\
599.15	2.26210923598685e-05\\
599.16	2.21568780461034e-05\\
599.17	2.16961404126415e-05\\
599.18	2.12389133419366e-05\\
599.19	2.07852310502073e-05\\
599.2	2.03351280907219e-05\\
599.21	1.98886393571462e-05\\
599.22	1.94458000869063e-05\\
599.23	1.90066458645843e-05\\
599.24	1.85712126253585e-05\\
599.25	1.8139536658468e-05\\
599.26	1.77116564944301e-05\\
599.27	1.72876122915989e-05\\
599.28	1.68674446078751e-05\\
599.29	1.64511944046493e-05\\
599.3	1.60389030507795e-05\\
599.31	1.56306123266399e-05\\
599.32	1.52263644281631e-05\\
599.33	1.48262019709685e-05\\
599.34	1.44301679944961e-05\\
599.35	1.40383059661995e-05\\
599.36	1.36506597857768e-05\\
599.37	1.32672737894416e-05\\
599.38	1.28881927542353e-05\\
599.39	1.25134619023814e-05\\
599.4	1.21431269056952e-05\\
599.41	1.17772338900073e-05\\
599.42	1.14158294396705e-05\\
599.43	1.10589606020818e-05\\
599.44	1.07066748922489e-05\\
599.45	1.03590202974301e-05\\
599.46	1.00160452817816e-05\\
599.47	9.67779879108323e-06\\
599.48	9.34433025748943e-06\\
599.49	9.01568960434002e-06\\
599.5	8.69192725100679e-06\\
599.51	8.37309411780285e-06\\
599.52	8.0592416309283e-06\\
599.53	7.75042172746279e-06\\
599.54	7.4466868604222e-06\\
599.55	7.14809000385182e-06\\
599.56	6.85468465797499e-06\\
599.57	6.5665248543869e-06\\
599.58	6.283665161309e-06\\
599.59	6.00616068889559e-06\\
599.6	5.73406709457672e-06\\
599.61	5.46744058846881e-06\\
599.62	5.20633793883903e-06\\
599.63	4.95081647761304e-06\\
599.64	4.70093410595215e-06\\
599.65	4.45674929985991e-06\\
599.66	4.21832111588243e-06\\
599.67	3.98570919682602e-06\\
599.68	3.75897377755986e-06\\
599.69	3.5381756908516e-06\\
599.7	3.32337637328622e-06\\
599.71	3.11463787123004e-06\\
599.72	2.91202284684322e-06\\
599.73	2.71559458416872e-06\\
599.74	2.52541699527833e-06\\
599.75	2.34155462647441e-06\\
599.76	2.16407266454867e-06\\
599.77	1.99303694311918e-06\\
599.78	1.82851394900548e-06\\
599.79	1.67057082869557e-06\\
599.8	1.51927539484599e-06\\
599.81	1.37469613288067e-06\\
599.82	1.23690220761412e-06\\
599.83	1.10596346997692e-06\\
599.84	9.81950463786394e-07\\
599.85	8.64934432593528e-07\\
599.86	7.54987326588921e-07\\
599.87	6.5218180958504e-07\\
599.88	5.56591266064402e-07\\
599.89	4.68289808293679e-07\\
599.9	3.87352283521061e-07\\
599.91	3.13854281218731e-07\\
599.92	2.47872140425945e-07\\
599.93	1.89482957149364e-07\\
599.94	1.38764591834512e-07\\
599.95	9.57956769222224e-08\\
599.96	6.06556244606149e-08\\
599.97	3.34246338211386e-08\\
599.98	1.4183699454523e-08\\
599.99	3.01461875948372e-09\\
600	0\\
};
\addplot [color=mycolor19,solid,forget plot]
  table[row sep=crcr]{%
0.01	0.00739460781638744\\
1.01	0.00739460735721567\\
2.01	0.00739460688814162\\
3.01	0.00739460640895054\\
4.01	0.00739460591942291\\
5.01	0.00739460541933461\\
6.01	0.00739460490845621\\
7.01	0.00739460438655362\\
8.01	0.00739460385338746\\
9.01	0.00739460330871333\\
10.01	0.00739460275228116\\
11.01	0.00739460218383558\\
12.01	0.00739460160311581\\
13.01	0.00739460100985508\\
14.01	0.00739460040378089\\
15.01	0.00739459978461482\\
16.01	0.00739459915207223\\
17.01	0.00739459850586236\\
18.01	0.00739459784568802\\
19.01	0.00739459717124554\\
20.01	0.00739459648222471\\
21.01	0.00739459577830832\\
22.01	0.00739459505917233\\
23.01	0.00739459432448543\\
24.01	0.00739459357390921\\
25.01	0.00739459280709796\\
26.01	0.00739459202369806\\
27.01	0.00739459122334836\\
28.01	0.00739459040567979\\
29.01	0.00739458957031496\\
30.01	0.00739458871686853\\
31.01	0.00739458784494637\\
32.01	0.00739458695414586\\
33.01	0.0073945860440556\\
34.01	0.00739458511425492\\
35.01	0.00739458416431424\\
36.01	0.00739458319379423\\
37.01	0.00739458220224598\\
38.01	0.00739458118921078\\
39.01	0.0073945801542198\\
40.01	0.00739457909679383\\
41.01	0.0073945780164433\\
42.01	0.00739457691266755\\
43.01	0.00739457578495535\\
44.01	0.00739457463278377\\
45.01	0.00739457345561862\\
46.01	0.00739457225291381\\
47.01	0.00739457102411131\\
48.01	0.00739456976864065\\
49.01	0.00739456848591905\\
50.01	0.00739456717535065\\
51.01	0.00739456583632645\\
52.01	0.0073945644682242\\
53.01	0.0073945630704077\\
54.01	0.00739456164222689\\
55.01	0.00739456018301725\\
56.01	0.00739455869209964\\
57.01	0.00739455716877984\\
58.01	0.00739455561234832\\
59.01	0.00739455402208016\\
60.01	0.00739455239723384\\
61.01	0.00739455073705207\\
62.01	0.0073945490407604\\
63.01	0.00739454730756729\\
64.01	0.00739454553666394\\
65.01	0.00739454372722328\\
66.01	0.00739454187840033\\
67.01	0.0073945399893311\\
68.01	0.00739453805913265\\
69.01	0.00739453608690237\\
70.01	0.0073945340717177\\
71.01	0.00739453201263567\\
72.01	0.00739452990869233\\
73.01	0.00739452775890244\\
74.01	0.00739452556225888\\
75.01	0.00739452331773208\\
76.01	0.00739452102426993\\
77.01	0.00739451868079665\\
78.01	0.00739451628621268\\
79.01	0.00739451383939399\\
80.01	0.00739451133919172\\
81.01	0.00739450878443146\\
82.01	0.00739450617391255\\
83.01	0.00739450350640777\\
84.01	0.00739450078066259\\
85.01	0.00739449799539457\\
86.01	0.00739449514929286\\
87.01	0.00739449224101712\\
88.01	0.00739448926919739\\
89.01	0.00739448623243329\\
90.01	0.00739448312929309\\
91.01	0.00739447995831325\\
92.01	0.00739447671799745\\
93.01	0.00739447340681626\\
94.01	0.00739447002320606\\
95.01	0.00739446656556814\\
96.01	0.00739446303226845\\
97.01	0.00739445942163625\\
98.01	0.00739445573196368\\
99.01	0.00739445196150452\\
100.01	0.00739444810847374\\
101.01	0.00739444417104638\\
102.01	0.00739444014735652\\
103.01	0.00739443603549671\\
104.01	0.00739443183351683\\
105.01	0.00739442753942302\\
106.01	0.00739442315117697\\
107.01	0.00739441866669458\\
108.01	0.00739441408384518\\
109.01	0.00739440940045021\\
110.01	0.00739440461428254\\
111.01	0.00739439972306501\\
112.01	0.00739439472446939\\
113.01	0.00739438961611521\\
114.01	0.00739438439556867\\
115.01	0.00739437906034121\\
116.01	0.00739437360788864\\
117.01	0.00739436803560947\\
118.01	0.00739436234084363\\
119.01	0.00739435652087173\\
120.01	0.00739435057291278\\
121.01	0.00739434449412337\\
122.01	0.0073943382815963\\
123.01	0.00739433193235886\\
124.01	0.00739432544337121\\
125.01	0.00739431881152551\\
126.01	0.00739431203364364\\
127.01	0.0073943051064758\\
128.01	0.00739429802669893\\
129.01	0.00739429079091534\\
130.01	0.00739428339565034\\
131.01	0.00739427583735108\\
132.01	0.0073942681123843\\
133.01	0.00739426021703493\\
134.01	0.00739425214750385\\
135.01	0.00739424389990599\\
136.01	0.00739423547026865\\
137.01	0.00739422685452913\\
138.01	0.00739421804853298\\
139.01	0.00739420904803167\\
140.01	0.00739419984868031\\
141.01	0.00739419044603592\\
142.01	0.00739418083555458\\
143.01	0.00739417101258945\\
144.01	0.00739416097238848\\
145.01	0.00739415071009162\\
146.01	0.00739414022072869\\
147.01	0.00739412949921663\\
148.01	0.00739411854035687\\
149.01	0.00739410733883286\\
150.01	0.00739409588920699\\
151.01	0.00739408418591852\\
152.01	0.00739407222327968\\
153.01	0.00739405999547342\\
154.01	0.00739404749655035\\
155.01	0.00739403472042565\\
156.01	0.00739402166087577\\
157.01	0.00739400831153534\\
158.01	0.00739399466589381\\
159.01	0.0073939807172921\\
160.01	0.00739396645891918\\
161.01	0.00739395188380848\\
162.01	0.00739393698483459\\
163.01	0.00739392175470884\\
164.01	0.00739390618597618\\
165.01	0.00739389027101107\\
166.01	0.00739387400201339\\
167.01	0.00739385737100459\\
168.01	0.00739384036982332\\
169.01	0.0073938229901212\\
170.01	0.00739380522335864\\
171.01	0.00739378706080012\\
172.01	0.00739376849350976\\
173.01	0.00739374951234659\\
174.01	0.00739373010795965\\
175.01	0.0073937102707832\\
176.01	0.00739368999103181\\
177.01	0.00739366925869482\\
178.01	0.00739364806353133\\
179.01	0.00739362639506489\\
180.01	0.00739360424257758\\
181.01	0.00739358159510467\\
182.01	0.00739355844142866\\
183.01	0.0073935347700736\\
184.01	0.00739351056929848\\
185.01	0.00739348582709131\\
186.01	0.00739346053116302\\
187.01	0.00739343466894026\\
188.01	0.00739340822755921\\
189.01	0.00739338119385852\\
190.01	0.00739335355437231\\
191.01	0.0073933252953229\\
192.01	0.00739329640261368\\
193.01	0.00739326686182099\\
194.01	0.00739323665818721\\
195.01	0.00739320577661214\\
196.01	0.00739317420164503\\
197.01	0.00739314191747661\\
198.01	0.00739310890793014\\
199.01	0.00739307515645301\\
200.01	0.00739304064610744\\
201.01	0.00739300535956169\\
202.01	0.00739296927908054\\
203.01	0.00739293238651537\\
204.01	0.00739289466329505\\
205.01	0.00739285609041505\\
206.01	0.00739281664842748\\
207.01	0.0073927763174307\\
208.01	0.00739273507705801\\
209.01	0.00739269290646691\\
210.01	0.00739264978432763\\
211.01	0.00739260568881128\\
212.01	0.00739256059757806\\
213.01	0.00739251448776509\\
214.01	0.00739246733597363\\
215.01	0.0073924191182565\\
216.01	0.00739236981010455\\
217.01	0.00739231938643358\\
218.01	0.00739226782157008\\
219.01	0.00739221508923747\\
220.01	0.00739216116254139\\
221.01	0.00739210601395458\\
222.01	0.00739204961530218\\
223.01	0.00739199193774552\\
224.01	0.00739193295176645\\
225.01	0.00739187262715094\\
226.01	0.00739181093297207\\
227.01	0.00739174783757285\\
228.01	0.00739168330854878\\
229.01	0.0073916173127293\\
230.01	0.00739154981615984\\
231.01	0.00739148078408217\\
232.01	0.00739141018091542\\
233.01	0.0073913379702357\\
234.01	0.00739126411475604\\
235.01	0.00739118857630508\\
236.01	0.00739111131580584\\
237.01	0.00739103229325327\\
238.01	0.00739095146769215\\
239.01	0.00739086879719365\\
240.01	0.00739078423883171\\
241.01	0.0073906977486585\\
242.01	0.00739060928167972\\
243.01	0.00739051879182908\\
244.01	0.00739042623194195\\
245.01	0.00739033155372833\\
246.01	0.00739023470774575\\
247.01	0.0073901356433708\\
248.01	0.00739003430877038\\
249.01	0.00738993065087156\\
250.01	0.00738982461533181\\
251.01	0.00738971614650737\\
252.01	0.0073896051874218\\
253.01	0.00738949167973259\\
254.01	0.00738937556369788\\
255.01	0.00738925677814244\\
256.01	0.00738913526042168\\
257.01	0.00738901094638614\\
258.01	0.00738888377034369\\
259.01	0.00738875366502219\\
260.01	0.00738862056153028\\
261.01	0.00738848438931714\\
262.01	0.00738834507613185\\
263.01	0.00738820254798125\\
264.01	0.00738805672908689\\
265.01	0.00738790754184095\\
266.01	0.00738775490676085\\
267.01	0.00738759874244298\\
268.01	0.00738743896551507\\
269.01	0.00738727549058717\\
270.01	0.0073871082302019\\
271.01	0.00738693709478314\\
272.01	0.00738676199258338\\
273.01	0.00738658282962967\\
274.01	0.00738639950966869\\
275.01	0.00738621193410975\\
276.01	0.00738602000196675\\
277.01	0.00738582360979866\\
278.01	0.00738562265164842\\
279.01	0.00738541701898025\\
280.01	0.00738520660061536\\
281.01	0.00738499128266623\\
282.01	0.00738477094846914\\
283.01	0.00738454547851474\\
284.01	0.00738431475037708\\
285.01	0.00738407863864118\\
286.01	0.00738383701482791\\
287.01	0.00738358974731779\\
288.01	0.00738333670127216\\
289.01	0.00738307773855345\\
290.01	0.0073828127176421\\
291.01	0.00738254149355236\\
292.01	0.0073822639177452\\
293.01	0.00738197983803984\\
294.01	0.00738168909852228\\
295.01	0.007381391539452\\
296.01	0.00738108699716632\\
297.01	0.00738077530398194\\
298.01	0.0073804562880943\\
299.01	0.00738012977347433\\
300.01	0.00737979557976289\\
301.01	0.0073794535221616\\
302.01	0.00737910341132239\\
303.01	0.00737874505323258\\
304.01	0.00737837824909876\\
305.01	0.00737800279522633\\
306.01	0.00737761848289708\\
307.01	0.00737722509824278\\
308.01	0.00737682242211633\\
309.01	0.00737641022995919\\
310.01	0.00737598829166552\\
311.01	0.00737555637144354\\
312.01	0.00737511422767252\\
313.01	0.00737466161275675\\
314.01	0.00737419827297554\\
315.01	0.00737372394832979\\
316.01	0.0073732383723844\\
317.01	0.00737274127210699\\
318.01	0.00737223236770241\\
319.01	0.00737171137244328\\
320.01	0.00737117799249619\\
321.01	0.00737063192674344\\
322.01	0.00737007286660059\\
323.01	0.00736950049582947\\
324.01	0.00736891449034603\\
325.01	0.00736831451802444\\
326.01	0.0073677002384949\\
327.01	0.00736707130293777\\
328.01	0.00736642735387172\\
329.01	0.00736576802493683\\
330.01	0.00736509294067259\\
331.01	0.00736440171628967\\
332.01	0.00736369395743714\\
333.01	0.00736296925996266\\
334.01	0.0073622272096674\\
335.01	0.00736146738205507\\
336.01	0.00736068934207347\\
337.01	0.00735989264385179\\
338.01	0.00735907683042887\\
339.01	0.00735824143347624\\
340.01	0.00735738597301394\\
341.01	0.00735650995711878\\
342.01	0.00735561288162585\\
343.01	0.00735469422982191\\
344.01	0.00735375347213177\\
345.01	0.00735279006579585\\
346.01	0.00735180345454027\\
347.01	0.00735079306823823\\
348.01	0.00734975832256253\\
349.01	0.00734869861862977\\
350.01	0.00734761334263451\\
351.01	0.00734650186547459\\
352.01	0.00734536354236683\\
353.01	0.00734419771245164\\
354.01	0.00734300369838831\\
355.01	0.00734178080593891\\
356.01	0.00734052832354109\\
357.01	0.00733924552186971\\
358.01	0.00733793165338621\\
359.01	0.00733658595187573\\
360.01	0.00733520763197159\\
361.01	0.00733379588866597\\
362.01	0.00733234989680765\\
363.01	0.00733086881058492\\
364.01	0.0073293517629934\\
365.01	0.00732779786528957\\
366.01	0.00732620620642631\\
367.01	0.00732457585247379\\
368.01	0.0073229058460219\\
369.01	0.00732119520556453\\
370.01	0.00731944292486566\\
371.01	0.00731764797230505\\
372.01	0.00731580929020395\\
373.01	0.00731392579412926\\
374.01	0.00731199637217588\\
375.01	0.00731001988422509\\
376.01	0.00730799516117935\\
377.01	0.00730592100417133\\
378.01	0.00730379618374717\\
379.01	0.00730161943902166\\
380.01	0.0072993894768053\\
381.01	0.00729710497070073\\
382.01	0.00729476456016879\\
383.01	0.00729236684956112\\
384.01	0.00728991040711958\\
385.01	0.00728739376393997\\
386.01	0.0072848154128984\\
387.01	0.00728217380754028\\
388.01	0.00727946736092822\\
389.01	0.00727669444444872\\
390.01	0.00727385338657547\\
391.01	0.00727094247158749\\
392.01	0.00726795993824026\\
393.01	0.00726490397838857\\
394.01	0.00726177273555814\\
395.01	0.00725856430346603\\
396.01	0.00725527672448628\\
397.01	0.00725190798806082\\
398.01	0.00724845602905235\\
399.01	0.00724491872604027\\
400.01	0.00724129389955609\\
401.01	0.00723757931025992\\
402.01	0.00723377265705657\\
403.01	0.00722987157515321\\
404.01	0.00722587363405743\\
405.01	0.00722177633552025\\
406.01	0.00721757711142594\\
407.01	0.00721327332163232\\
408.01	0.00720886225176726\\
409.01	0.00720434111098601\\
410.01	0.00719970702969591\\
411.01	0.00719495705725344\\
412.01	0.00719008815963761\\
413.01	0.00718509721710061\\
414.01	0.00717998102179087\\
415.01	0.00717473627533758\\
416.01	0.00716935958637324\\
417.01	0.00716384746795955\\
418.01	0.00715819633486804\\
419.01	0.00715240250065579\\
420.01	0.00714646217447976\\
421.01	0.00714037145762384\\
422.01	0.00713412633977827\\
423.01	0.00712772269510845\\
424.01	0.00712115627811266\\
425.01	0.00711442271926145\\
426.01	0.00710751752041582\\
427.01	0.00710043605002663\\
428.01	0.00709317353812675\\
429.01	0.00708572507113625\\
430.01	0.00707808558651451\\
431.01	0.00707024986731098\\
432.01	0.00706221253668515\\
433.01	0.00705396805249208\\
434.01	0.00704551070205724\\
435.01	0.00703683459729696\\
436.01	0.00702793367037132\\
437.01	0.0070188016700894\\
438.01	0.00700943215930772\\
439.01	0.00699981851357164\\
440.01	0.0069899539212232\\
441.01	0.00697983138512249\\
442.01	0.00696944372596999\\
443.01	0.00695878358691883\\
444.01	0.00694784343866876\\
445.01	0.00693661558342861\\
446.01	0.00692509215389056\\
447.01	0.00691326509820568\\
448.01	0.00690112616826287\\
449.01	0.00688866691980275\\
450.01	0.00687587871469878\\
451.01	0.00686275272483023\\
452.01	0.00684927993790207\\
453.01	0.0068354511656438\\
454.01	0.0068212570549198\\
455.01	0.00680668810240477\\
456.01	0.00679173467363303\\
457.01	0.00677638702742366\\
458.01	0.00676063534693086\\
459.01	0.00674446977887691\\
460.01	0.00672788048292408\\
461.01	0.00671085769364399\\
462.01	0.00669339179818848\\
463.01	0.00667547343359382\\
464.01	0.00665709360870891\\
465.01	0.00663824385710418\\
466.01	0.00661891642907578\\
467.01	0.00659910453312369\\
468.01	0.00657880264020722\\
469.01	0.00655800686785727\\
470.01	0.00653671546609859\\
471.01	0.00651492943343259\\
472.01	0.00649265329926033\\
473.01	0.00646989611961106\\
474.01	0.0064466727465601\\
475.01	0.00642300544911735\\
476.01	0.00639892598551904\\
477.01	0.00637447825521281\\
478.01	0.00634972170595724\\
479.01	0.00632473573363495\\
480.01	0.00629962538319443\\
481.01	0.00627452875737553\\
482.01	0.00624939051251928\\
483.01	0.00622376064740071\\
484.01	0.00619765015738636\\
485.01	0.0061711067485383\\
486.01	0.00614419621701035\\
487.01	0.00611700764458573\\
488.01	0.00608966011335451\\
489.01	0.00606231140357749\\
490.01	0.00603516928644871\\
491.01	0.00600850621982586\\
492.01	0.00598267851658965\\
493.01	0.00595807385714244\\
494.01	0.00593342118060778\\
495.01	0.00590807476816829\\
496.01	0.00588201498763912\\
497.01	0.00585522118092143\\
498.01	0.00582767155076743\\
499.01	0.00579934311057551\\
500.01	0.00577021176562901\\
501.01	0.00574025263987014\\
502.01	0.00570944089116093\\
503.01	0.00567775295885528\\
504.01	0.00564516795050089\\
505.01	0.00561166826048236\\
506.01	0.00557723879844062\\
507.01	0.00554186725854646\\
508.01	0.0055055447350111\\
509.01	0.00546826638667768\\
510.01	0.00543003212621563\\
511.01	0.00539084729183767\\
512.01	0.00535072323266179\\
513.01	0.00530967769962473\\
514.01	0.00526773487704865\\
515.01	0.00522492480832201\\
516.01	0.00518128185258883\\
517.01	0.00513684164376462\\
518.01	0.0050916357889881\\
519.01	0.00504568026289477\\
520.01	0.00499895988309242\\
521.01	0.00495143501123638\\
522.01	0.00490303435970988\\
523.01	0.00485363928552798\\
524.01	0.00480306166516005\\
525.01	0.0047510147258005\\
526.01	0.004697213408747\\
527.01	0.00464152182251868\\
528.01	0.0045838662559028\\
529.01	0.00452422283921419\\
530.01	0.00446261138137171\\
531.01	0.00439911741448504\\
532.01	0.00433392363083164\\
533.01	0.00426778874389639\\
534.01	0.00420137881425066\\
535.01	0.00413485118336642\\
536.01	0.00406838381103923\\
537.01	0.00400217499259854\\
538.01	0.00393644160379946\\
539.01	0.0038714150870976\\
540.01	0.00380733405547405\\
541.01	0.00374443191699418\\
542.01	0.00368291726258625\\
543.01	0.00362294390816676\\
544.01	0.00356455659307963\\
545.01	0.00350696477086615\\
546.01	0.00344975170545223\\
547.01	0.00339298951158971\\
548.01	0.00333673215405864\\
549.01	0.00328100951661619\\
550.01	0.00322582173290015\\
551.01	0.00317113412249031\\
552.01	0.00311687052003248\\
553.01	0.00306290970465843\\
554.01	0.00300908991410397\\
555.01	0.00295523201111278\\
556.01	0.00290124563021432\\
557.01	0.002847098072094\\
558.01	0.00279274702632913\\
559.01	0.00273814046510795\\
560.01	0.00268321858787583\\
561.01	0.00262791619871552\\
562.01	0.00257216649574095\\
563.01	0.00251590691454799\\
564.01	0.00245908556494755\\
565.01	0.00240166502778602\\
566.01	0.00234361427178739\\
567.01	0.0022849016921835\\
568.01	0.0022254958877592\\
569.01	0.00216536696107141\\
570.01	0.0021044878379189\\
571.01	0.00204283533477184\\
572.01	0.00198039060692598\\
573.01	0.00191713873061543\\
574.01	0.0018530674700027\\
575.01	0.00178816633815506\\
576.01	0.00172242682014106\\
577.01	0.00165584284131613\\
578.01	0.001588411129313\\
579.01	0.00152013143583996\\
580.01	0.00145100662154961\\
581.01	0.00138104265774608\\
582.01	0.00131024865991341\\
583.01	0.0012386370874182\\
584.01	0.00116622409072038\\
585.01	0.00109302987562263\\
586.01	0.001019079034026\\
587.01	0.000944400819187679\\
588.01	0.000869029345008453\\
589.01	0.000793003687887234\\
590.01	0.000716367854795577\\
591.01	0.000639170551898716\\
592.01	0.000561464658762556\\
593.01	0.000483306311701305\\
594.01	0.000404753505687408\\
595.01	0.000325864119007519\\
596.01	0.000246693254416338\\
597.01	0.000167289772990669\\
598.01	9.17366978681194e-05\\
599.01	2.94669364335823e-05\\
599.02	2.89574269646022e-05\\
599.03	2.84509530909024e-05\\
599.04	2.79475443514862e-05\\
599.05	2.74472305761501e-05\\
599.06	2.69500418883586e-05\\
599.07	2.64560087081516e-05\\
599.08	2.59651617550757e-05\\
599.09	2.54775320511456e-05\\
599.1	2.49931509238403e-05\\
599.11	2.45120500091244e-05\\
599.12	2.40342612544929e-05\\
599.13	2.35598169220554e-05\\
599.14	2.30887495916553e-05\\
599.15	2.26210921640043e-05\\
599.16	2.21568778638551e-05\\
599.17	2.16961402432179e-05\\
599.18	2.12389131845833e-05\\
599.19	2.07852309042009e-05\\
599.2	2.03351279553787e-05\\
599.21	1.98886392318159e-05\\
599.22	1.94457999709644e-05\\
599.23	1.90066457574443e-05\\
599.24	1.85712125264602e-05\\
599.25	1.81395365672823e-05\\
599.26	1.77116564104504e-05\\
599.27	1.72876122143499e-05\\
599.28	1.68674445369076e-05\\
599.29	1.64511943395312e-05\\
599.3	1.60389029911102e-05\\
599.31	1.56306122720377e-05\\
599.32	1.52263643782673e-05\\
599.33	1.48262019254389e-05\\
599.34	1.44301679530119e-05\\
599.35	1.40383059284606e-05\\
599.36	1.36506597515022e-05\\
599.37	1.32672737583623e-05\\
599.38	1.28881927261015e-05\\
599.39	1.25134618769607e-05\\
599.4	1.21431268827639e-05\\
599.41	1.17772338693641e-05\\
599.42	1.14158294211263e-05\\
599.43	1.10589605854527e-05\\
599.44	1.0706674877372e-05\\
599.45	1.03590202841473e-05\\
599.46	1.00160452699508e-05\\
599.47	9.6777987805708e-06\\
599.48	9.34433024817223e-06\\
599.49	9.01568959610181e-06\\
599.5	8.6919272437435e-06\\
599.51	8.37309411141907e-06\\
599.52	8.05924162533209e-06\\
599.53	7.75042172257087e-06\\
599.54	7.44668685616172e-06\\
599.55	7.14809000015512e-06\\
599.56	6.85468465477443e-06\\
599.57	6.56652485162869e-06\\
599.58	6.28366515894284e-06\\
599.59	6.00616068687464e-06\\
599.6	5.73406709285587e-06\\
599.61	5.4674405870099e-06\\
599.62	5.20633793760911e-06\\
599.63	4.95081647658262e-06\\
599.64	4.70093410509e-06\\
599.65	4.45674929914694e-06\\
599.66	4.21832111529435e-06\\
599.67	3.9857091963455e-06\\
599.68	3.75897377716608e-06\\
599.69	3.53817569053415e-06\\
599.7	3.32337637303469e-06\\
599.71	3.11463787103054e-06\\
599.72	2.91202284668536e-06\\
599.73	2.71559458404555e-06\\
599.74	2.52541699518466e-06\\
599.75	2.34155462640155e-06\\
599.76	2.16407266449489e-06\\
599.77	1.99303694307755e-06\\
599.78	1.82851394897598e-06\\
599.79	1.67057082867302e-06\\
599.8	1.51927539483211e-06\\
599.81	1.37469613287027e-06\\
599.82	1.23690220760718e-06\\
599.83	1.10596346996998e-06\\
599.84	9.81950463782924e-07\\
599.85	8.64934432591793e-07\\
599.86	7.54987326588921e-07\\
599.87	6.5218180958504e-07\\
599.88	5.56591266064402e-07\\
599.89	4.68289808295413e-07\\
599.9	3.87352283521061e-07\\
599.91	3.13854281216996e-07\\
599.92	2.4787214042421e-07\\
599.93	1.8948295714763e-07\\
599.94	1.38764591834512e-07\\
599.95	9.57956769204876e-08\\
599.96	6.06556244606149e-08\\
599.97	3.34246338194039e-08\\
599.98	1.41836994527883e-08\\
599.99	3.01461875948372e-09\\
600	0\\
};
\addplot [color=red!50!mycolor17,solid,forget plot]
  table[row sep=crcr]{%
0.01	0.00656440582368625\\
1.01	0.00656440546540022\\
2.01	0.00656440509943144\\
3.01	0.00656440472561442\\
4.01	0.00656440434378001\\
5.01	0.00656440395375523\\
6.01	0.00656440355536398\\
7.01	0.00656440314842576\\
8.01	0.00656440273275647\\
9.01	0.00656440230816771\\
10.01	0.00656440187446719\\
11.01	0.00656440143145865\\
12.01	0.00656440097894109\\
13.01	0.00656440051670957\\
14.01	0.00656440004455453\\
15.01	0.00656439956226184\\
16.01	0.00656439906961291\\
17.01	0.0065643985663842\\
18.01	0.00656439805234737\\
19.01	0.00656439752726918\\
20.01	0.00656439699091115\\
21.01	0.00656439644302987\\
22.01	0.00656439588337666\\
23.01	0.00656439531169723\\
24.01	0.00656439472773186\\
25.01	0.00656439413121509\\
26.01	0.00656439352187586\\
27.01	0.00656439289943711\\
28.01	0.00656439226361582\\
29.01	0.00656439161412292\\
30.01	0.00656439095066264\\
31.01	0.00656439027293327\\
32.01	0.0065643895806263\\
33.01	0.00656438887342651\\
34.01	0.00656438815101185\\
35.01	0.00656438741305317\\
36.01	0.00656438665921446\\
37.01	0.00656438588915204\\
38.01	0.00656438510251502\\
39.01	0.00656438429894477\\
40.01	0.00656438347807471\\
41.01	0.00656438263953042\\
42.01	0.00656438178292943\\
43.01	0.00656438090788055\\
44.01	0.00656438001398459\\
45.01	0.0065643791008333\\
46.01	0.00656437816800967\\
47.01	0.00656437721508744\\
48.01	0.00656437624163129\\
49.01	0.00656437524719617\\
50.01	0.00656437423132769\\
51.01	0.0065643731935612\\
52.01	0.00656437213342191\\
53.01	0.00656437105042498\\
54.01	0.00656436994407461\\
55.01	0.00656436881386433\\
56.01	0.00656436765927691\\
57.01	0.00656436647978339\\
58.01	0.00656436527484341\\
59.01	0.00656436404390475\\
60.01	0.00656436278640348\\
61.01	0.0065643615017627\\
62.01	0.0065643601893935\\
63.01	0.0065643588486938\\
64.01	0.00656435747904851\\
65.01	0.00656435607982905\\
66.01	0.0065643546503928\\
67.01	0.00656435319008354\\
68.01	0.00656435169823045\\
69.01	0.00656435017414805\\
70.01	0.00656434861713593\\
71.01	0.00656434702647833\\
72.01	0.00656434540144379\\
73.01	0.00656434374128471\\
74.01	0.00656434204523739\\
75.01	0.00656434031252119\\
76.01	0.00656433854233837\\
77.01	0.00656433673387389\\
78.01	0.00656433488629476\\
79.01	0.00656433299874963\\
80.01	0.00656433107036859\\
81.01	0.00656432910026258\\
82.01	0.00656432708752323\\
83.01	0.00656432503122192\\
84.01	0.00656432293040998\\
85.01	0.00656432078411769\\
86.01	0.0065643185913541\\
87.01	0.00656431635110668\\
88.01	0.00656431406234046\\
89.01	0.00656431172399769\\
90.01	0.00656430933499771\\
91.01	0.0065643068942357\\
92.01	0.00656430440058294\\
93.01	0.00656430185288571\\
94.01	0.00656429924996476\\
95.01	0.00656429659061535\\
96.01	0.0065642938736059\\
97.01	0.00656429109767786\\
98.01	0.00656428826154488\\
99.01	0.00656428536389257\\
100.01	0.00656428240337719\\
101.01	0.00656427937862584\\
102.01	0.0065642762882352\\
103.01	0.00656427313077095\\
104.01	0.00656426990476732\\
105.01	0.00656426660872626\\
106.01	0.00656426324111655\\
107.01	0.00656425980037325\\
108.01	0.00656425628489711\\
109.01	0.00656425269305348\\
110.01	0.00656424902317158\\
111.01	0.00656424527354389\\
112.01	0.00656424144242516\\
113.01	0.00656423752803153\\
114.01	0.00656423352854006\\
115.01	0.00656422944208717\\
116.01	0.00656422526676833\\
117.01	0.00656422100063694\\
118.01	0.00656421664170345\\
119.01	0.00656421218793391\\
120.01	0.00656420763725004\\
121.01	0.00656420298752725\\
122.01	0.00656419823659398\\
123.01	0.00656419338223057\\
124.01	0.0065641884221685\\
125.01	0.00656418335408863\\
126.01	0.00656417817562074\\
127.01	0.0065641728843422\\
128.01	0.00656416747777677\\
129.01	0.00656416195339289\\
130.01	0.00656415630860342\\
131.01	0.00656415054076357\\
132.01	0.00656414464717013\\
133.01	0.00656413862505968\\
134.01	0.00656413247160771\\
135.01	0.00656412618392702\\
136.01	0.00656411975906615\\
137.01	0.00656411319400827\\
138.01	0.00656410648566962\\
139.01	0.00656409963089772\\
140.01	0.00656409262647038\\
141.01	0.00656408546909342\\
142.01	0.00656407815539972\\
143.01	0.00656407068194717\\
144.01	0.00656406304521711\\
145.01	0.00656405524161271\\
146.01	0.00656404726745693\\
147.01	0.00656403911899101\\
148.01	0.00656403079237245\\
149.01	0.00656402228367297\\
150.01	0.00656401358887737\\
151.01	0.00656400470388014\\
152.01	0.00656399562448489\\
153.01	0.00656398634640142\\
154.01	0.00656397686524381\\
155.01	0.00656396717652843\\
156.01	0.00656395727567134\\
157.01	0.00656394715798651\\
158.01	0.00656393681868305\\
159.01	0.00656392625286333\\
160.01	0.00656391545552006\\
161.01	0.0065639044215341\\
162.01	0.0065638931456716\\
163.01	0.00656388162258207\\
164.01	0.00656386984679521\\
165.01	0.00656385781271792\\
166.01	0.00656384551463225\\
167.01	0.00656383294669207\\
168.01	0.0065638201029203\\
169.01	0.00656380697720591\\
170.01	0.0065637935633009\\
171.01	0.00656377985481698\\
172.01	0.00656376584522262\\
173.01	0.00656375152784004\\
174.01	0.00656373689584109\\
175.01	0.00656372194224441\\
176.01	0.00656370665991189\\
177.01	0.00656369104154491\\
178.01	0.00656367507968084\\
179.01	0.00656365876668904\\
180.01	0.00656364209476747\\
181.01	0.00656362505593814\\
182.01	0.00656360764204395\\
183.01	0.00656358984474339\\
184.01	0.00656357165550749\\
185.01	0.00656355306561492\\
186.01	0.00656353406614752\\
187.01	0.00656351464798604\\
188.01	0.00656349480180533\\
189.01	0.00656347451806996\\
190.01	0.00656345378702875\\
191.01	0.00656343259871038\\
192.01	0.00656341094291802\\
193.01	0.00656338880922434\\
194.01	0.00656336618696599\\
195.01	0.00656334306523831\\
196.01	0.00656331943288979\\
197.01	0.00656329527851595\\
198.01	0.00656327059045398\\
199.01	0.00656324535677658\\
200.01	0.00656321956528616\\
201.01	0.00656319320350774\\
202.01	0.00656316625868347\\
203.01	0.00656313871776556\\
204.01	0.00656311056740936\\
205.01	0.00656308179396725\\
206.01	0.00656305238348084\\
207.01	0.00656302232167389\\
208.01	0.00656299159394528\\
209.01	0.00656296018536091\\
210.01	0.00656292808064631\\
211.01	0.00656289526417888\\
212.01	0.00656286171997907\\
213.01	0.0065628274317028\\
214.01	0.00656279238263234\\
215.01	0.0065627565556678\\
216.01	0.0065627199333185\\
217.01	0.00656268249769313\\
218.01	0.00656264423049121\\
219.01	0.00656260511299261\\
220.01	0.00656256512604795\\
221.01	0.00656252425006915\\
222.01	0.00656248246501819\\
223.01	0.00656243975039708\\
224.01	0.00656239608523671\\
225.01	0.00656235144808599\\
226.01	0.00656230581700023\\
227.01	0.00656225916952966\\
228.01	0.00656221148270753\\
229.01	0.00656216273303747\\
230.01	0.00656211289648114\\
231.01	0.00656206194844563\\
232.01	0.00656200986377009\\
233.01	0.00656195661671203\\
234.01	0.00656190218093363\\
235.01	0.00656184652948781\\
236.01	0.00656178963480315\\
237.01	0.00656173146866952\\
238.01	0.00656167200222242\\
239.01	0.00656161120592764\\
240.01	0.00656154904956486\\
241.01	0.00656148550221161\\
242.01	0.00656142053222612\\
243.01	0.00656135410723006\\
244.01	0.0065612861940909\\
245.01	0.00656121675890383\\
246.01	0.00656114576697296\\
247.01	0.00656107318279228\\
248.01	0.00656099897002596\\
249.01	0.00656092309148856\\
250.01	0.00656084550912443\\
251.01	0.00656076618398621\\
252.01	0.00656068507621333\\
253.01	0.00656060214501012\\
254.01	0.00656051734862315\\
255.01	0.00656043064431713\\
256.01	0.00656034198835127\\
257.01	0.00656025133595515\\
258.01	0.00656015864130328\\
259.01	0.0065600638574892\\
260.01	0.00655996693649861\\
261.01	0.00655986782918266\\
262.01	0.00655976648523\\
263.01	0.00655966285313819\\
264.01	0.00655955688018397\\
265.01	0.00655944851239334\\
266.01	0.00655933769451097\\
267.01	0.00655922436996773\\
268.01	0.00655910848084881\\
269.01	0.00655898996786009\\
270.01	0.00655886877029374\\
271.01	0.00655874482599301\\
272.01	0.00655861807131657\\
273.01	0.00655848844110089\\
274.01	0.00655835586862248\\
275.01	0.00655822028555866\\
276.01	0.00655808162194776\\
277.01	0.00655793980614784\\
278.01	0.00655779476479435\\
279.01	0.00655764642275691\\
280.01	0.00655749470309495\\
281.01	0.00655733952701165\\
282.01	0.00655718081380714\\
283.01	0.00655701848083072\\
284.01	0.00655685244343093\\
285.01	0.00655668261490464\\
286.01	0.00655650890644569\\
287.01	0.00655633122709056\\
288.01	0.00655614948366405\\
289.01	0.0065559635807221\\
290.01	0.00655577342049454\\
291.01	0.00655557890282469\\
292.01	0.00655537992510951\\
293.01	0.00655517638223547\\
294.01	0.00655496816651483\\
295.01	0.00655475516761933\\
296.01	0.00655453727251183\\
297.01	0.00655431436537697\\
298.01	0.0065540863275489\\
299.01	0.00655385303743793\\
300.01	0.00655361437045451\\
301.01	0.00655337019893227\\
302.01	0.00655312039204702\\
303.01	0.00655286481573576\\
304.01	0.00655260333261192\\
305.01	0.00655233580187901\\
306.01	0.00655206207924174\\
307.01	0.00655178201681469\\
308.01	0.00655149546302884\\
309.01	0.00655120226253515\\
310.01	0.00655090225610571\\
311.01	0.00655059528053253\\
312.01	0.0065502811685226\\
313.01	0.00654995974859157\\
314.01	0.00654963084495366\\
315.01	0.00654929427740841\\
316.01	0.0065489498612254\\
317.01	0.00654859740702488\\
318.01	0.00654823672065572\\
319.01	0.00654786760307014\\
320.01	0.00654748985019475\\
321.01	0.00654710325279886\\
322.01	0.0065467075963584\\
323.01	0.00654630266091722\\
324.01	0.00654588822094412\\
325.01	0.00654546404518628\\
326.01	0.00654502989651933\\
327.01	0.00654458553179271\\
328.01	0.00654413070167154\\
329.01	0.0065436651504752\\
330.01	0.00654318861601012\\
331.01	0.00654270082939996\\
332.01	0.0065422015149103\\
333.01	0.00654169038976984\\
334.01	0.00654116716398685\\
335.01	0.00654063154016085\\
336.01	0.00654008321329051\\
337.01	0.00653952187057521\\
338.01	0.00653894719121438\\
339.01	0.00653835884619936\\
340.01	0.00653775649810243\\
341.01	0.00653713980086034\\
342.01	0.00653650839955177\\
343.01	0.00653586193017142\\
344.01	0.0065352000193979\\
345.01	0.00653452228435681\\
346.01	0.00653382833237883\\
347.01	0.00653311776075192\\
348.01	0.00653239015646936\\
349.01	0.00653164509597112\\
350.01	0.00653088214488131\\
351.01	0.00653010085773923\\
352.01	0.00652930077772563\\
353.01	0.00652848143638347\\
354.01	0.00652764235333345\\
355.01	0.0065267830359844\\
356.01	0.00652590297923764\\
357.01	0.00652500166518708\\
358.01	0.00652407856281347\\
359.01	0.00652313312767335\\
360.01	0.00652216480158361\\
361.01	0.00652117301230029\\
362.01	0.00652015717319252\\
363.01	0.00651911668291189\\
364.01	0.00651805092505683\\
365.01	0.0065169592678316\\
366.01	0.00651584106370219\\
367.01	0.0065146956490462\\
368.01	0.00651352234379845\\
369.01	0.00651232045109326\\
370.01	0.00651108925690145\\
371.01	0.00650982802966368\\
372.01	0.00650853601991949\\
373.01	0.00650721245993259\\
374.01	0.00650585656331152\\
375.01	0.0065044675246271\\
376.01	0.00650304451902536\\
377.01	0.00650158670183648\\
378.01	0.00650009320817941\\
379.01	0.00649856315256223\\
380.01	0.00649699562847678\\
381.01	0.00649538970798896\\
382.01	0.0064937444413224\\
383.01	0.00649205885643525\\
384.01	0.00649033195858945\\
385.01	0.00648856272991013\\
386.01	0.00648675012893443\\
387.01	0.00648489309014677\\
388.01	0.00648299052349833\\
389.01	0.00648104131390808\\
390.01	0.00647904432073992\\
391.01	0.00647699837725221\\
392.01	0.00647490229001441\\
393.01	0.00647275483828189\\
394.01	0.00647055477332394\\
395.01	0.00646830081769172\\
396.01	0.00646599166441805\\
397.01	0.00646362597613286\\
398.01	0.00646120238408305\\
399.01	0.00645871948703488\\
400.01	0.00645617585004426\\
401.01	0.00645357000307023\\
402.01	0.00645090043941078\\
403.01	0.00644816561393521\\
404.01	0.00644536394109067\\
405.01	0.00644249379265812\\
406.01	0.00643955349523729\\
407.01	0.00643654132744707\\
408.01	0.00643345551683574\\
409.01	0.00643029423650974\\
410.01	0.00642705560150948\\
411.01	0.00642373766498846\\
412.01	0.00642033841428345\\
413.01	0.00641685576700698\\
414.01	0.00641328756733783\\
415.01	0.00640963158272969\\
416.01	0.00640588550129285\\
417.01	0.00640204693010357\\
418.01	0.00639811339462549\\
419.01	0.00639408233923406\\
420.01	0.00638995112838307\\
421.01	0.00638571704692442\\
422.01	0.00638137729839664\\
423.01	0.00637692900261399\\
424.01	0.00637236919322113\\
425.01	0.00636769481525123\\
426.01	0.00636290272270502\\
427.01	0.00635798967617175\\
428.01	0.00635295234051363\\
429.01	0.00634778728264398\\
430.01	0.00634249096942608\\
431.01	0.006337059765727\\
432.01	0.00633148993266037\\
433.01	0.00632577762605561\\
434.01	0.00631991889518877\\
435.01	0.00631390968181004\\
436.01	0.00630774581949997\\
437.01	0.00630142303337773\\
438.01	0.00629493694017861\\
439.01	0.00628828304870329\\
440.01	0.00628145676063154\\
441.01	0.00627445337167707\\
442.01	0.00626726807305398\\
443.01	0.00625989595323021\\
444.01	0.00625233199997218\\
445.01	0.0062445711027621\\
446.01	0.00623660805584039\\
447.01	0.00622843756253448\\
448.01	0.00622005424148167\\
449.01	0.00621145263441838\\
450.01	0.00620262721545311\\
451.01	0.00619357240196595\\
452.01	0.00618428256728609\\
453.01	0.00617475205529724\\
454.01	0.00616497519710713\\
455.01	0.00615494632990128\\
456.01	0.00614465981806608\\
457.01	0.00613411007662122\\
458.01	0.00612329159692982\\
459.01	0.00611219897456385\\
460.01	0.00610082693907493\\
461.01	0.00608917038525529\\
462.01	0.00607722440526771\\
463.01	0.00606498432075847\\
464.01	0.00605244571375353\\
465.01	0.00603960445476142\\
466.01	0.00602645672608229\\
467.01	0.00601299903786675\\
468.01	0.00599922823402013\\
469.01	0.00598514148467885\\
470.01	0.00597073626182786\\
471.01	0.00595601029487695\\
472.01	0.00594096150400683\\
473.01	0.0059255879113317\\
474.01	0.00590988753420292\\
475.01	0.00589385827255953\\
476.01	0.00587749783778725\\
477.01	0.00586080373908369\\
478.01	0.00584377299093302\\
479.01	0.00582640150876262\\
480.01	0.0058086831876211\\
481.01	0.00579060853200369\\
482.01	0.00577216432972964\\
483.01	0.00575334582220765\\
484.01	0.00573415424939939\\
485.01	0.00571459079020263\\
486.01	0.00569465572021814\\
487.01	0.00567434718468358\\
488.01	0.00565365936006162\\
489.01	0.00563257975556609\\
490.01	0.00561108531160021\\
491.01	0.00558913682306404\\
492.01	0.00556667103921158\\
493.01	0.00554358997176287\\
494.01	0.00551979880560138\\
495.01	0.00549527325324063\\
496.01	0.00546999979739121\\
497.01	0.00544396646201648\\
498.01	0.00541716238647811\\
499.01	0.00538957684242895\\
500.01	0.00536119737711359\\
501.01	0.00533200587711989\\
502.01	0.00530196903533774\\
503.01	0.00527104449029588\\
504.01	0.0052391864232501\\
505.01	0.00520634535208319\\
506.01	0.00517246793809063\\
507.01	0.00513749692894901\\
508.01	0.00510137127455286\\
509.01	0.00506402650617743\\
510.01	0.00502539551200544\\
511.01	0.00498540989747207\\
512.01	0.0049440021960149\\
513.01	0.0049011093030496\\
514.01	0.00485667765489779\\
515.01	0.0048106708810639\\
516.01	0.00476308094499622\\
517.01	0.00471394418628692\\
518.01	0.00466336422930108\\
519.01	0.00461186793412728\\
520.01	0.0045599046924575\\
521.01	0.00450754211297316\\
522.01	0.00445485964823526\\
523.01	0.00440195105852912\\
524.01	0.00434892780644088\\
525.01	0.00429592382681327\\
526.01	0.00424309974104571\\
527.01	0.00419063659086737\\
528.01	0.00413872705740767\\
529.01	0.0040875654696395\\
530.01	0.00403733280501714\\
531.01	0.00398817333943129\\
532.01	0.00394015889021307\\
533.01	0.0038927890665121\\
534.01	0.0038456196551268\\
535.01	0.00379871775713688\\
536.01	0.00375214226384924\\
537.01	0.00370593984677847\\
538.01	0.00366014038052297\\
539.01	0.00361475198846775\\
540.01	0.0035697561020783\\
541.01	0.00352510323923424\\
542.01	0.00348071068926054\\
543.01	0.00343646401783639\\
544.01	0.00339222542361496\\
545.01	0.00334788219760175\\
546.01	0.00330340572108235\\
547.01	0.00325877291055184\\
548.01	0.00321395096245914\\
549.01	0.00316889777513852\\
550.01	0.0031235631575099\\
551.01	0.00307789096322181\\
552.01	0.00303182227862506\\
553.01	0.00298529967226709\\
554.01	0.00293827203692452\\
555.01	0.00289069883880613\\
556.01	0.00284254826966974\\
557.01	0.00279378882254609\\
558.01	0.0027443879232907\\
559.01	0.00269431286133482\\
560.01	0.00264353176750194\\
561.01	0.0025920145436866\\
562.01	0.00253973362071961\\
563.01	0.00248666432825441\\
564.01	0.00243278467556005\\
565.01	0.00237807450916724\\
566.01	0.00232251470725555\\
567.01	0.00226608722974788\\
568.01	0.00220877542917444\\
569.01	0.00215056429547898\\
570.01	0.0020914406062683\\
571.01	0.00203139297475291\\
572.01	0.00197041181888403\\
573.01	0.00190848931254016\\
574.01	0.00184561940995461\\
575.01	0.00178179800211486\\
576.01	0.00171702313729112\\
577.01	0.00165129523694475\\
578.01	0.00158461730238309\\
579.01	0.00151699512093703\\
580.01	0.00144843748360971\\
581.01	0.00137895642524441\\
582.01	0.00130856749049479\\
583.01	0.00123729001347867\\
584.01	0.00116514738699767\\
585.01	0.00109216730167909\\
586.01	0.00101838193976332\\
587.01	0.000943828105516345\\
588.01	0.000868547269223433\\
589.01	0.000792585494808212\\
590.01	0.000715993212731169\\
591.01	0.000638824791183846\\
592.01	0.000561137850788497\\
593.01	0.000482992259190443\\
594.01	0.000404448728692637\\
595.01	0.000325566921518283\\
596.01	0.000246402942558918\\
597.01	0.000167006059591517\\
598.01	9.17366970263709e-05\\
599.01	2.94669364272089e-05\\
599.02	2.89574269586573e-05\\
599.03	2.84509530853617e-05\\
599.04	2.79475443463271e-05\\
599.05	2.74472305713484e-05\\
599.06	2.69500418838952e-05\\
599.07	2.64560087040022e-05\\
599.08	2.59651617512246e-05\\
599.09	2.54775320475721e-05\\
599.1	2.49931509205322e-05\\
599.11	2.45120500060592e-05\\
599.12	2.40342612516566e-05\\
599.13	2.35598169194343e-05\\
599.14	2.30887495892354e-05\\
599.15	2.262109216177e-05\\
599.16	2.21568778617977e-05\\
599.17	2.16961402413236e-05\\
599.18	2.12389131828417e-05\\
599.19	2.07852309026015e-05\\
599.2	2.03351279539129e-05\\
599.21	1.98886392304715e-05\\
599.22	1.94457999697362e-05\\
599.23	1.90066457563202e-05\\
599.24	1.85712125254332e-05\\
599.25	1.81395365663438e-05\\
599.26	1.77116564095969e-05\\
599.27	1.72876122135745e-05\\
599.28	1.68674445362033e-05\\
599.29	1.64511943388929e-05\\
599.3	1.60389029905342e-05\\
599.31	1.56306122715139e-05\\
599.32	1.52263643777972e-05\\
599.33	1.48262019250139e-05\\
599.34	1.4430167952632e-05\\
599.35	1.40383059281188e-05\\
599.36	1.36506597511951e-05\\
599.37	1.32672737580882e-05\\
599.38	1.28881927258552e-05\\
599.39	1.25134618767422e-05\\
599.4	1.21431268825713e-05\\
599.41	1.17772338691941e-05\\
599.42	1.14158294209736e-05\\
599.43	1.10589605853192e-05\\
599.44	1.0706674877254e-05\\
599.45	1.03590202840433e-05\\
599.46	1.00160452698606e-05\\
599.47	9.67779878049101e-06\\
599.48	9.34433024810284e-06\\
599.49	9.01568959604283e-06\\
599.5	8.69192724369319e-06\\
599.51	8.37309411137396e-06\\
599.52	8.05924162529219e-06\\
599.53	7.75042172253965e-06\\
599.54	7.4466868561357e-06\\
599.55	7.14809000013084e-06\\
599.56	6.85468465475708e-06\\
599.57	6.56652485161308e-06\\
599.58	6.2836651589307e-06\\
599.59	6.00616068686249e-06\\
599.6	5.73406709284546e-06\\
599.61	5.46744058700296e-06\\
599.62	5.20633793760217e-06\\
599.63	4.95081647657741e-06\\
599.64	4.70093410508653e-06\\
599.65	4.45674929914174e-06\\
599.66	4.21832111529089e-06\\
599.67	3.98570919634203e-06\\
599.68	3.75897377716608e-06\\
599.69	3.53817569053415e-06\\
599.7	3.32337637303469e-06\\
599.71	3.11463787102881e-06\\
599.72	2.91202284668363e-06\\
599.73	2.71559458404382e-06\\
599.74	2.52541699518292e-06\\
599.75	2.34155462640155e-06\\
599.76	2.16407266449663e-06\\
599.77	1.99303694307928e-06\\
599.78	1.82851394897598e-06\\
599.79	1.67057082867302e-06\\
599.8	1.51927539483211e-06\\
599.81	1.37469613287027e-06\\
599.82	1.23690220760544e-06\\
599.83	1.10596346997172e-06\\
599.84	9.81950463782924e-07\\
599.85	8.64934432591793e-07\\
599.86	7.54987326587186e-07\\
599.87	6.5218180958504e-07\\
599.88	5.56591266062667e-07\\
599.89	4.68289808295413e-07\\
599.9	3.87352283521061e-07\\
599.91	3.13854281218731e-07\\
599.92	2.47872140425945e-07\\
599.93	1.89482957149364e-07\\
599.94	1.38764591834512e-07\\
599.95	9.57956769204876e-08\\
599.96	6.06556244606149e-08\\
599.97	3.34246338211386e-08\\
599.98	1.4183699454523e-08\\
599.99	3.014618757749e-09\\
600	0\\
};
\addplot [color=red!40!mycolor19,solid,forget plot]
  table[row sep=crcr]{%
0.01	0.00627501982079631\\
1.01	0.00627501944692668\\
2.01	0.00627501906508025\\
3.01	0.00627501867508635\\
4.01	0.00627501827677046\\
5.01	0.00627501786995453\\
6.01	0.00627501745445643\\
7.01	0.00627501703009011\\
8.01	0.00627501659666559\\
9.01	0.00627501615398896\\
10.01	0.00627501570186209\\
11.01	0.00627501524008224\\
12.01	0.00627501476844279\\
13.01	0.00627501428673242\\
14.01	0.0062750137947353\\
15.01	0.00627501329223091\\
16.01	0.00627501277899408\\
17.01	0.00627501225479479\\
18.01	0.00627501171939803\\
19.01	0.00627501117256362\\
20.01	0.00627501061404649\\
21.01	0.00627501004359601\\
22.01	0.00627500946095607\\
23.01	0.00627500886586557\\
24.01	0.00627500825805735\\
25.01	0.00627500763725846\\
26.01	0.00627500700319044\\
27.01	0.00627500635556846\\
28.01	0.00627500569410162\\
29.01	0.00627500501849288\\
30.01	0.00627500432843883\\
31.01	0.00627500362362923\\
32.01	0.00627500290374755\\
33.01	0.00627500216847012\\
34.01	0.00627500141746654\\
35.01	0.00627500065039911\\
36.01	0.00627499986692271\\
37.01	0.00627499906668511\\
38.01	0.00627499824932634\\
39.01	0.00627499741447863\\
40.01	0.00627499656176637\\
41.01	0.00627499569080558\\
42.01	0.0062749948012044\\
43.01	0.00627499389256226\\
44.01	0.00627499296446983\\
45.01	0.00627499201650924\\
46.01	0.00627499104825335\\
47.01	0.00627499005926611\\
48.01	0.00627498904910168\\
49.01	0.00627498801730491\\
50.01	0.00627498696341052\\
51.01	0.00627498588694327\\
52.01	0.0062749847874179\\
53.01	0.00627498366433837\\
54.01	0.00627498251719804\\
55.01	0.00627498134547947\\
56.01	0.00627498014865353\\
57.01	0.00627497892618032\\
58.01	0.00627497767750798\\
59.01	0.00627497640207243\\
60.01	0.00627497509929776\\
61.01	0.00627497376859569\\
62.01	0.00627497240936482\\
63.01	0.00627497102099114\\
64.01	0.00627496960284693\\
65.01	0.00627496815429119\\
66.01	0.00627496667466925\\
67.01	0.00627496516331188\\
68.01	0.00627496361953534\\
69.01	0.00627496204264142\\
70.01	0.00627496043191656\\
71.01	0.00627495878663203\\
72.01	0.00627495710604279\\
73.01	0.00627495538938831\\
74.01	0.00627495363589117\\
75.01	0.00627495184475727\\
76.01	0.00627495001517523\\
77.01	0.00627494814631603\\
78.01	0.00627494623733275\\
79.01	0.00627494428736043\\
80.01	0.00627494229551492\\
81.01	0.00627494026089314\\
82.01	0.00627493818257228\\
83.01	0.00627493605960982\\
84.01	0.00627493389104248\\
85.01	0.0062749316758862\\
86.01	0.00627492941313573\\
87.01	0.00627492710176376\\
88.01	0.00627492474072093\\
89.01	0.00627492232893507\\
90.01	0.00627491986531064\\
91.01	0.00627491734872842\\
92.01	0.00627491477804482\\
93.01	0.00627491215209159\\
94.01	0.006274909469675\\
95.01	0.00627490672957539\\
96.01	0.00627490393054669\\
97.01	0.00627490107131582\\
98.01	0.00627489815058196\\
99.01	0.00627489516701594\\
100.01	0.00627489211926003\\
101.01	0.00627488900592669\\
102.01	0.00627488582559843\\
103.01	0.00627488257682694\\
104.01	0.00627487925813229\\
105.01	0.00627487586800252\\
106.01	0.00627487240489285\\
107.01	0.00627486886722501\\
108.01	0.00627486525338602\\
109.01	0.00627486156172819\\
110.01	0.00627485779056785\\
111.01	0.00627485393818469\\
112.01	0.00627485000282107\\
113.01	0.00627484598268125\\
114.01	0.00627484187593009\\
115.01	0.00627483768069272\\
116.01	0.00627483339505341\\
117.01	0.00627482901705461\\
118.01	0.00627482454469642\\
119.01	0.00627481997593549\\
120.01	0.00627481530868351\\
121.01	0.00627481054080715\\
122.01	0.00627480567012622\\
123.01	0.00627480069441342\\
124.01	0.00627479561139276\\
125.01	0.00627479041873867\\
126.01	0.00627478511407482\\
127.01	0.00627477969497318\\
128.01	0.00627477415895298\\
129.01	0.00627476850347911\\
130.01	0.00627476272596112\\
131.01	0.00627475682375241\\
132.01	0.00627475079414839\\
133.01	0.00627474463438554\\
134.01	0.0062747383416401\\
135.01	0.00627473191302679\\
136.01	0.00627472534559735\\
137.01	0.00627471863633906\\
138.01	0.00627471178217341\\
139.01	0.00627470477995492\\
140.01	0.00627469762646923\\
141.01	0.00627469031843194\\
142.01	0.00627468285248695\\
143.01	0.00627467522520483\\
144.01	0.00627466743308101\\
145.01	0.00627465947253462\\
146.01	0.0062746513399064\\
147.01	0.00627464303145715\\
148.01	0.00627463454336603\\
149.01	0.00627462587172868\\
150.01	0.00627461701255539\\
151.01	0.00627460796176902\\
152.01	0.00627459871520338\\
153.01	0.00627458926860131\\
154.01	0.00627457961761231\\
155.01	0.00627456975779074\\
156.01	0.00627455968459374\\
157.01	0.00627454939337881\\
158.01	0.00627453887940217\\
159.01	0.0062745281378158\\
160.01	0.00627451716366581\\
161.01	0.0062745059518896\\
162.01	0.00627449449731388\\
163.01	0.00627448279465192\\
164.01	0.00627447083850095\\
165.01	0.00627445862334025\\
166.01	0.00627444614352765\\
167.01	0.0062744333932974\\
168.01	0.00627442036675739\\
169.01	0.00627440705788619\\
170.01	0.00627439346053031\\
171.01	0.00627437956840123\\
172.01	0.00627436537507274\\
173.01	0.0062743508739768\\
174.01	0.00627433605840198\\
175.01	0.00627432092148916\\
176.01	0.00627430545622857\\
177.01	0.00627428965545641\\
178.01	0.00627427351185153\\
179.01	0.00627425701793191\\
180.01	0.00627424016605103\\
181.01	0.00627422294839444\\
182.01	0.00627420535697528\\
183.01	0.00627418738363186\\
184.01	0.00627416902002222\\
185.01	0.00627415025762123\\
186.01	0.00627413108771593\\
187.01	0.00627411150140158\\
188.01	0.00627409148957742\\
189.01	0.00627407104294207\\
190.01	0.00627405015198938\\
191.01	0.00627402880700365\\
192.01	0.00627400699805497\\
193.01	0.00627398471499471\\
194.01	0.00627396194745011\\
195.01	0.00627393868481967\\
196.01	0.00627391491626796\\
197.01	0.00627389063072046\\
198.01	0.00627386581685818\\
199.01	0.00627384046311181\\
200.01	0.00627381455765685\\
201.01	0.00627378808840736\\
202.01	0.00627376104301\\
203.01	0.0062737334088386\\
204.01	0.00627370517298764\\
205.01	0.00627367632226584\\
206.01	0.00627364684319003\\
207.01	0.00627361672197865\\
208.01	0.00627358594454488\\
209.01	0.00627355449648979\\
210.01	0.00627352236309501\\
211.01	0.00627348952931607\\
212.01	0.00627345597977495\\
213.01	0.00627342169875202\\
214.01	0.00627338667017911\\
215.01	0.00627335087763107\\
216.01	0.00627331430431797\\
217.01	0.00627327693307661\\
218.01	0.0062732387463622\\
219.01	0.00627319972623976\\
220.01	0.00627315985437529\\
221.01	0.00627311911202647\\
222.01	0.00627307748003373\\
223.01	0.00627303493881053\\
224.01	0.0062729914683338\\
225.01	0.00627294704813376\\
226.01	0.0062729016572838\\
227.01	0.00627285527439051\\
228.01	0.00627280787758198\\
229.01	0.00627275944449813\\
230.01	0.00627270995227875\\
231.01	0.00627265937755205\\
232.01	0.00627260769642319\\
233.01	0.00627255488446222\\
234.01	0.00627250091669176\\
235.01	0.00627244576757419\\
236.01	0.00627238941099914\\
237.01	0.00627233182026998\\
238.01	0.00627227296809058\\
239.01	0.00627221282655082\\
240.01	0.00627215136711321\\
241.01	0.00627208856059781\\
242.01	0.00627202437716745\\
243.01	0.00627195878631262\\
244.01	0.00627189175683558\\
245.01	0.00627182325683449\\
246.01	0.00627175325368689\\
247.01	0.00627168171403308\\
248.01	0.00627160860375854\\
249.01	0.0062715338879766\\
250.01	0.00627145753100993\\
251.01	0.00627137949637256\\
252.01	0.00627129974675016\\
253.01	0.00627121824398126\\
254.01	0.00627113494903623\\
255.01	0.00627104982199781\\
256.01	0.00627096282203976\\
257.01	0.00627087390740491\\
258.01	0.00627078303538332\\
259.01	0.00627069016228966\\
260.01	0.00627059524344006\\
261.01	0.00627049823312814\\
262.01	0.00627039908460019\\
263.01	0.00627029775003053\\
264.01	0.00627019418049573\\
265.01	0.00627008832594792\\
266.01	0.00626998013518788\\
267.01	0.0062698695558371\\
268.01	0.00626975653430933\\
269.01	0.00626964101578122\\
270.01	0.00626952294416235\\
271.01	0.00626940226206434\\
272.01	0.00626927891076863\\
273.01	0.00626915283019496\\
274.01	0.00626902395886698\\
275.01	0.00626889223387859\\
276.01	0.00626875759085857\\
277.01	0.00626861996393404\\
278.01	0.00626847928569379\\
279.01	0.00626833548714984\\
280.01	0.00626818849769801\\
281.01	0.00626803824507829\\
282.01	0.00626788465533277\\
283.01	0.00626772765276327\\
284.01	0.00626756715988795\\
285.01	0.00626740309739626\\
286.01	0.00626723538410234\\
287.01	0.00626706393689815\\
288.01	0.00626688867070387\\
289.01	0.00626670949841873\\
290.01	0.00626652633086838\\
291.01	0.00626633907675284\\
292.01	0.0062661476425907\\
293.01	0.00626595193266438\\
294.01	0.00626575184896115\\
295.01	0.00626554729111439\\
296.01	0.00626533815634235\\
297.01	0.00626512433938498\\
298.01	0.00626490573243966\\
299.01	0.00626468222509425\\
300.01	0.00626445370425851\\
301.01	0.00626422005409342\\
302.01	0.0062639811559389\\
303.01	0.00626373688823883\\
304.01	0.00626348712646358\\
305.01	0.00626323174303124\\
306.01	0.00626297060722554\\
307.01	0.00626270358511182\\
308.01	0.00626243053945027\\
309.01	0.00626215132960651\\
310.01	0.00626186581146005\\
311.01	0.0062615738373084\\
312.01	0.0062612752557708\\
313.01	0.00626096991168605\\
314.01	0.00626065764600972\\
315.01	0.00626033829570683\\
316.01	0.0062600116936413\\
317.01	0.00625967766846247\\
318.01	0.00625933604448791\\
319.01	0.00625898664158239\\
320.01	0.00625862927503338\\
321.01	0.00625826375542209\\
322.01	0.0062578898884914\\
323.01	0.00625750747500859\\
324.01	0.00625711631062481\\
325.01	0.00625671618572884\\
326.01	0.00625630688529746\\
327.01	0.00625588818874049\\
328.01	0.00625545986974115\\
329.01	0.00625502169609042\\
330.01	0.00625457342951782\\
331.01	0.00625411482551534\\
332.01	0.00625364563315641\\
333.01	0.00625316559490887\\
334.01	0.00625267444644229\\
335.01	0.00625217191642857\\
336.01	0.00625165772633658\\
337.01	0.0062511315902205\\
338.01	0.00625059321450026\\
339.01	0.00625004229773692\\
340.01	0.00624947853039881\\
341.01	0.00624890159462121\\
342.01	0.00624831116395897\\
343.01	0.00624770690313027\\
344.01	0.00624708846775251\\
345.01	0.00624645550407053\\
346.01	0.00624580764867563\\
347.01	0.00624514452821613\\
348.01	0.00624446575909892\\
349.01	0.00624377094718198\\
350.01	0.00624305968745715\\
351.01	0.00624233156372373\\
352.01	0.00624158614825195\\
353.01	0.00624082300143695\\
354.01	0.00624004167144166\\
355.01	0.00623924169383069\\
356.01	0.00623842259119347\\
357.01	0.00623758387275602\\
358.01	0.00623672503398323\\
359.01	0.00623584555617051\\
360.01	0.00623494490602452\\
361.01	0.006234022535234\\
362.01	0.00623307788003002\\
363.01	0.00623211036073653\\
364.01	0.00623111938131073\\
365.01	0.00623010432887544\\
366.01	0.00622906457324128\\
367.01	0.00622799946642182\\
368.01	0.00622690834214188\\
369.01	0.00622579051533872\\
370.01	0.0062246452816595\\
371.01	0.00622347191695422\\
372.01	0.00622226967676832\\
373.01	0.00622103779583443\\
374.01	0.00621977548756801\\
375.01	0.00621848194356779\\
376.01	0.00621715633312541\\
377.01	0.00621579780274746\\
378.01	0.00621440547569387\\
379.01	0.00621297845153867\\
380.01	0.00621151580575779\\
381.01	0.00621001658934978\\
382.01	0.00620847982849738\\
383.01	0.00620690452427863\\
384.01	0.00620528965243403\\
385.01	0.00620363416320286\\
386.01	0.00620193698123763\\
387.01	0.00620019700560957\\
388.01	0.00619841310991871\\
389.01	0.00619658414252233\\
390.01	0.00619470892689802\\
391.01	0.00619278626215702\\
392.01	0.00619081492372418\\
393.01	0.00618879366420226\\
394.01	0.00618672121443493\\
395.01	0.00618459628478509\\
396.01	0.00618241756663875\\
397.01	0.00618018373414341\\
398.01	0.00617789344617994\\
399.01	0.00617554534856301\\
400.01	0.00617313807644601\\
401.01	0.00617067025689468\\
402.01	0.00616814051156823\\
403.01	0.00616554745941866\\
404.01	0.00616288971928154\\
405.01	0.00616016591218665\\
406.01	0.00615737466316107\\
407.01	0.00615451460222969\\
408.01	0.00615158436424575\\
409.01	0.00614858258710004\\
410.01	0.00614550790777582\\
411.01	0.00614235895564246\\
412.01	0.00613913434234199\\
413.01	0.00613583264763858\\
414.01	0.00613245240073966\\
415.01	0.00612899205692615\\
416.01	0.00612544996998135\\
417.01	0.00612182436207277\\
418.01	0.00611811329471075\\
419.01	0.00611431464759695\\
420.01	0.00611042611998717\\
421.01	0.00610644528345852\\
422.01	0.00610236962267547\\
423.01	0.00609819653509054\\
424.01	0.00609392332698215\\
425.01	0.00608954720927813\\
426.01	0.00608506529315244\\
427.01	0.00608047458538285\\
428.01	0.00607577198345746\\
429.01	0.00607095427041315\\
430.01	0.00606601810939559\\
431.01	0.0060609600379238\\
432.01	0.0060557764618456\\
433.01	0.00605046364896941\\
434.01	0.00604501772236019\\
435.01	0.00603943465328612\\
436.01	0.00603371025380779\\
437.01	0.00602784016900473\\
438.01	0.00602181986883974\\
439.01	0.00601564463967026\\
440.01	0.00600930957542628\\
441.01	0.00600280956849132\\
442.01	0.00599613930034045\\
443.01	0.00598929323201499\\
444.01	0.0059822655945433\\
445.01	0.0059750503794517\\
446.01	0.00596764132954422\\
447.01	0.0059600319301649\\
448.01	0.00595221540118198\\
449.01	0.00594418468999795\\
450.01	0.00593593246598146\\
451.01	0.00592745111681121\\
452.01	0.00591873274732945\\
453.01	0.00590976918162688\\
454.01	0.00590055196922702\\
455.01	0.00589107239639559\\
456.01	0.00588132150378528\\
457.01	0.00587129011181471\\
458.01	0.00586096885537762\\
459.01	0.00585034822966616\\
460.01	0.00583941864904041\\
461.01	0.00582817052095897\\
462.01	0.00581659433692928\\
463.01	0.00580468078216617\\
464.01	0.00579242086502429\\
465.01	0.00577980606610628\\
466.01	0.0057668285049653\\
467.01	0.00575348111911687\\
468.01	0.0057397578450755\\
469.01	0.00572565378352102\\
470.01	0.00571116531930866\\
471.01	0.00569629015022579\\
472.01	0.00568102715383578\\
473.01	0.00566537598617683\\
474.01	0.00564933625488107\\
475.01	0.00563290599850248\\
476.01	0.00561607703700144\\
477.01	0.00559883378959261\\
478.01	0.00558115863084415\\
479.01	0.00556303221632345\\
480.01	0.00554443329402765\\
481.01	0.00552533853042991\\
482.01	0.00550572236828859\\
483.01	0.00548555671834599\\
484.01	0.00546481037634793\\
485.01	0.0054434485964233\\
486.01	0.00542143271600303\\
487.01	0.00539871981429891\\
488.01	0.00537526245314433\\
489.01	0.0053510085754977\\
490.01	0.00532590167516816\\
491.01	0.00529988140633078\\
492.01	0.00527288488008312\\
493.01	0.00524484900592452\\
494.01	0.00521571354443454\\
495.01	0.00518542144114009\\
496.01	0.00515392022106027\\
497.01	0.00512116654246566\\
498.01	0.00508713295858186\\
499.01	0.00505181766150161\\
500.01	0.00501525838772721\\
501.01	0.00497766164855981\\
502.01	0.00493943528264369\\
503.01	0.00490062690988339\\
504.01	0.00486126761608939\\
505.01	0.00482139625878809\\
506.01	0.00478106053220624\\
507.01	0.00474031806455034\\
508.01	0.00469923749003392\\
509.01	0.00465789940415897\\
510.01	0.00461639706089578\\
511.01	0.00457483659870346\\
512.01	0.00453333647999093\\
513.01	0.0044920256832665\\
514.01	0.00445103998170064\\
515.01	0.00441051535217822\\
516.01	0.00437057715191148\\
517.01	0.00433132312909362\\
518.01	0.00429279753583172\\
519.01	0.00425462094943984\\
520.01	0.00421646388263337\\
521.01	0.00417838022721257\\
522.01	0.004140422850745\\
523.01	0.00410264178097458\\
524.01	0.00406508188635623\\
525.01	0.00402777995207816\\
526.01	0.00399076106794841\\
527.01	0.00395403462590486\\
528.01	0.00391759050822663\\
529.01	0.00388139584061086\\
530.01	0.00384539310379418\\
531.01	0.00380950096592413\\
532.01	0.00377361999606866\\
533.01	0.00373766221781281\\
534.01	0.00370160676146456\\
535.01	0.00366544339044638\\
536.01	0.00362915467612006\\
537.01	0.00359271576305016\\
538.01	0.00355609454479962\\
539.01	0.00351925238193505\\
540.01	0.00348214548655068\\
541.01	0.00344472705759396\\
542.01	0.00340695015404107\\
543.01	0.0033687711001213\\
544.01	0.0033301528691409\\
545.01	0.0032910665487296\\
546.01	0.00325148580905002\\
547.01	0.00321138269631842\\
548.01	0.0031707279562313\\
549.01	0.00312949174391565\\
550.01	0.00308764439343728\\
551.01	0.00304515716922659\\
552.01	0.00300200289208221\\
553.01	0.00295815630627339\\
554.01	0.00291359405328688\\
555.01	0.00286829417430089\\
556.01	0.00282223533915401\\
557.01	0.0027753965662475\\
558.01	0.00272775741296249\\
559.01	0.00267929817756782\\
560.01	0.00263000004395317\\
561.01	0.00257984515678967\\
562.01	0.00252881662416956\\
563.01	0.00247689846218247\\
564.01	0.00242407552049739\\
565.01	0.00237033344952677\\
566.01	0.00231565875634426\\
567.01	0.00226003891137813\\
568.01	0.0022034624518856\\
569.01	0.0021459190761402\\
570.01	0.0020873997344778\\
571.01	0.00202789672639428\\
572.01	0.00196740381416456\\
573.01	0.00190591636143106\\
574.01	0.00184343149881933\\
575.01	0.00177994830972369\\
576.01	0.00171546802767963\\
577.01	0.00164999424345148\\
578.01	0.00158353312275398\\
579.01	0.00151609363483733\\
580.01	0.00144768779049829\\
581.01	0.0013783308857227\\
582.01	0.00130804174453238\\
583.01	0.00123684295248754\\
584.01	0.00116476107120689\\
585.01	0.00109182682306618\\
586.01	0.00101807523268008\\
587.01	0.000943545708142529\\
588.01	0.000868282040487625\\
589.01	0.000792332294341591\\
590.01	0.000715748556112592\\
591.01	0.000638586497983857\\
592.01	0.00056090470582257\\
593.01	0.00048276370607613\\
594.01	0.000404224610087212\\
595.01	0.000325347273344155\\
596.01	0.00024618784114796\\
597.01	0.000166807993572935\\
598.01	9.17366970068743e-05\\
599.01	2.94669364271152e-05\\
599.02	2.89574269585705e-05\\
599.03	2.84509530852819e-05\\
599.04	2.79475443462525e-05\\
599.05	2.7447230571279e-05\\
599.06	2.69500418838293e-05\\
599.07	2.64560087039449e-05\\
599.08	2.59651617511691e-05\\
599.09	2.54775320475235e-05\\
599.1	2.49931509204836e-05\\
599.11	2.45120500060158e-05\\
599.12	2.40342612516167e-05\\
599.13	2.35598169193996e-05\\
599.14	2.30887495892024e-05\\
599.15	2.26210921617405e-05\\
599.16	2.215687786177e-05\\
599.17	2.16961402412976e-05\\
599.18	2.12389131828191e-05\\
599.19	2.07852309025824e-05\\
599.2	2.03351279538938e-05\\
599.21	1.98886392304524e-05\\
599.22	1.94457999697188e-05\\
599.23	1.90066457563063e-05\\
599.24	1.85712125254194e-05\\
599.25	1.81395365663334e-05\\
599.26	1.77116564095865e-05\\
599.27	1.72876122135658e-05\\
599.28	1.68674445361946e-05\\
599.29	1.64511943388859e-05\\
599.3	1.60389029905273e-05\\
599.31	1.56306122715104e-05\\
599.32	1.52263643777902e-05\\
599.33	1.48262019250105e-05\\
599.34	1.44301679526268e-05\\
599.35	1.40383059281154e-05\\
599.36	1.36506597511899e-05\\
599.37	1.32672737580847e-05\\
599.38	1.28881927258535e-05\\
599.39	1.25134618767404e-05\\
599.4	1.21431268825696e-05\\
599.41	1.17772338691924e-05\\
599.42	1.14158294209736e-05\\
599.43	1.10589605853174e-05\\
599.44	1.07066748772523e-05\\
599.45	1.03590202840433e-05\\
599.46	1.00160452698589e-05\\
599.47	9.67779878049101e-06\\
599.48	9.34433024810284e-06\\
599.49	9.0156895960411e-06\\
599.5	8.69192724369319e-06\\
599.51	8.37309411137396e-06\\
599.52	8.05924162529392e-06\\
599.53	7.75042172253965e-06\\
599.54	7.44668685613396e-06\\
599.55	7.14809000013084e-06\\
599.56	6.85468465475535e-06\\
599.57	6.56652485161134e-06\\
599.58	6.28366515892896e-06\\
599.59	6.00616068686249e-06\\
599.6	5.73406709284546e-06\\
599.61	5.46744058700296e-06\\
599.62	5.2063379376039e-06\\
599.63	4.95081647657741e-06\\
599.64	4.70093410508653e-06\\
599.65	4.45674929914347e-06\\
599.66	4.21832111529262e-06\\
599.67	3.98570919634376e-06\\
599.68	3.75897377716608e-06\\
599.69	3.53817569053241e-06\\
599.7	3.32337637303295e-06\\
599.71	3.11463787102881e-06\\
599.72	2.91202284668363e-06\\
599.73	2.71559458404555e-06\\
599.74	2.52541699518466e-06\\
599.75	2.34155462640329e-06\\
599.76	2.16407266449489e-06\\
599.77	1.99303694307755e-06\\
599.78	1.82851394897772e-06\\
599.79	1.67057082867475e-06\\
599.8	1.51927539483385e-06\\
599.81	1.37469613287027e-06\\
599.82	1.23690220760718e-06\\
599.83	1.10596346997172e-06\\
599.84	9.81950463784659e-07\\
599.85	8.64934432591793e-07\\
599.86	7.54987326587186e-07\\
599.87	6.52181809583305e-07\\
599.88	5.56591266064402e-07\\
599.89	4.68289808295413e-07\\
599.9	3.87352283521061e-07\\
599.91	3.13854281216996e-07\\
599.92	2.4787214042421e-07\\
599.93	1.8948295714763e-07\\
599.94	1.38764591834512e-07\\
599.95	9.57956769204876e-08\\
599.96	6.06556244606149e-08\\
599.97	3.34246338194039e-08\\
599.98	1.4183699454523e-08\\
599.99	3.01461875948372e-09\\
600	0\\
};
\addplot [color=red!75!mycolor17,solid,forget plot]
  table[row sep=crcr]{%
0.01	0.00615124513121968\\
1.01	0.00615124466954397\\
2.01	0.00615124419804395\\
3.01	0.00615124371650986\\
4.01	0.00615124322472784\\
5.01	0.00615124272247878\\
6.01	0.00615124220953934\\
7.01	0.00615124168568137\\
8.01	0.00615124115067172\\
9.01	0.00615124060427232\\
10.01	0.00615124004623988\\
11.01	0.00615123947632624\\
12.01	0.00615123889427741\\
13.01	0.00615123829983411\\
14.01	0.00615123769273202\\
15.01	0.00615123707270048\\
16.01	0.00615123643946319\\
17.01	0.00615123579273817\\
18.01	0.00615123513223719\\
19.01	0.00615123445766603\\
20.01	0.00615123376872386\\
21.01	0.00615123306510371\\
22.01	0.00615123234649195\\
23.01	0.00615123161256789\\
24.01	0.00615123086300448\\
25.01	0.00615123009746729\\
26.01	0.00615122931561487\\
27.01	0.00615122851709837\\
28.01	0.00615122770156156\\
29.01	0.00615122686864046\\
30.01	0.00615122601796326\\
31.01	0.00615122514915043\\
32.01	0.00615122426181394\\
33.01	0.00615122335555755\\
34.01	0.0061512224299766\\
35.01	0.00615122148465763\\
36.01	0.00615122051917858\\
37.01	0.00615121953310805\\
38.01	0.00615121852600525\\
39.01	0.00615121749742009\\
40.01	0.00615121644689298\\
41.01	0.00615121537395443\\
42.01	0.00615121427812442\\
43.01	0.00615121315891327\\
44.01	0.00615121201582032\\
45.01	0.00615121084833429\\
46.01	0.00615120965593289\\
47.01	0.00615120843808264\\
48.01	0.00615120719423868\\
49.01	0.00615120592384418\\
50.01	0.00615120462633052\\
51.01	0.00615120330111693\\
52.01	0.00615120194761009\\
53.01	0.0061512005652038\\
54.01	0.00615119915327905\\
55.01	0.00615119771120327\\
56.01	0.00615119623833049\\
57.01	0.00615119473400065\\
58.01	0.00615119319753965\\
59.01	0.00615119162825901\\
60.01	0.00615119002545504\\
61.01	0.00615118838840926\\
62.01	0.00615118671638754\\
63.01	0.00615118500864007\\
64.01	0.00615118326440093\\
65.01	0.00615118148288763\\
66.01	0.00615117966330086\\
67.01	0.00615117780482403\\
68.01	0.00615117590662313\\
69.01	0.00615117396784602\\
70.01	0.00615117198762241\\
71.01	0.00615116996506288\\
72.01	0.00615116789925952\\
73.01	0.00615116578928429\\
74.01	0.00615116363418927\\
75.01	0.00615116143300638\\
76.01	0.00615115918474637\\
77.01	0.00615115688839902\\
78.01	0.00615115454293195\\
79.01	0.00615115214729076\\
80.01	0.00615114970039842\\
81.01	0.00615114720115439\\
82.01	0.00615114464843474\\
83.01	0.0061511420410911\\
84.01	0.00615113937795039\\
85.01	0.00615113665781445\\
86.01	0.00615113387945888\\
87.01	0.0061511310416332\\
88.01	0.00615112814305988\\
89.01	0.00615112518243382\\
90.01	0.00615112215842168\\
91.01	0.00615111906966155\\
92.01	0.00615111591476198\\
93.01	0.00615111269230171\\
94.01	0.00615110940082874\\
95.01	0.00615110603885955\\
96.01	0.00615110260487881\\
97.01	0.0061510990973384\\
98.01	0.00615109551465685\\
99.01	0.00615109185521853\\
100.01	0.00615108811737304\\
101.01	0.00615108429943421\\
102.01	0.0061510803996794\\
103.01	0.00615107641634907\\
104.01	0.0061510723476455\\
105.01	0.00615106819173216\\
106.01	0.00615106394673287\\
107.01	0.00615105961073085\\
108.01	0.00615105518176808\\
109.01	0.00615105065784422\\
110.01	0.00615104603691575\\
111.01	0.00615104131689492\\
112.01	0.00615103649564875\\
113.01	0.0061510315709984\\
114.01	0.00615102654071772\\
115.01	0.00615102140253287\\
116.01	0.00615101615412047\\
117.01	0.00615101079310716\\
118.01	0.00615100531706819\\
119.01	0.00615099972352604\\
120.01	0.0061509940099504\\
121.01	0.00615098817375558\\
122.01	0.00615098221230042\\
123.01	0.00615097612288638\\
124.01	0.00615096990275661\\
125.01	0.00615096354909479\\
126.01	0.00615095705902356\\
127.01	0.00615095042960346\\
128.01	0.00615094365783104\\
129.01	0.00615093674063835\\
130.01	0.00615092967489107\\
131.01	0.00615092245738647\\
132.01	0.00615091508485339\\
133.01	0.00615090755394924\\
134.01	0.00615089986125927\\
135.01	0.00615089200329484\\
136.01	0.00615088397649178\\
137.01	0.00615087577720894\\
138.01	0.00615086740172604\\
139.01	0.0061508588462425\\
140.01	0.00615085010687546\\
141.01	0.00615084117965773\\
142.01	0.00615083206053641\\
143.01	0.00615082274537083\\
144.01	0.00615081322993062\\
145.01	0.00615080350989381\\
146.01	0.00615079358084492\\
147.01	0.00615078343827253\\
148.01	0.00615077307756771\\
149.01	0.00615076249402159\\
150.01	0.00615075168282305\\
151.01	0.00615074063905718\\
152.01	0.00615072935770238\\
153.01	0.00615071783362799\\
154.01	0.00615070606159245\\
155.01	0.0061506940362403\\
156.01	0.00615068175210023\\
157.01	0.00615066920358219\\
158.01	0.00615065638497509\\
159.01	0.00615064329044373\\
160.01	0.00615062991402655\\
161.01	0.00615061624963279\\
162.01	0.00615060229103937\\
163.01	0.00615058803188819\\
164.01	0.00615057346568362\\
165.01	0.00615055858578872\\
166.01	0.00615054338542289\\
167.01	0.00615052785765803\\
168.01	0.00615051199541592\\
169.01	0.00615049579146478\\
170.01	0.00615047923841574\\
171.01	0.00615046232871981\\
172.01	0.00615044505466359\\
173.01	0.00615042740836686\\
174.01	0.006150409381778\\
175.01	0.00615039096667045\\
176.01	0.00615037215463921\\
177.01	0.00615035293709682\\
178.01	0.00615033330526905\\
179.01	0.00615031325019131\\
180.01	0.00615029276270415\\
181.01	0.00615027183344897\\
182.01	0.00615025045286421\\
183.01	0.00615022861118025\\
184.01	0.00615020629841525\\
185.01	0.00615018350437032\\
186.01	0.00615016021862495\\
187.01	0.0061501364305322\\
188.01	0.00615011212921345\\
189.01	0.00615008730355363\\
190.01	0.00615006194219591\\
191.01	0.00615003603353649\\
192.01	0.00615000956571926\\
193.01	0.00614998252662989\\
194.01	0.00614995490389073\\
195.01	0.00614992668485472\\
196.01	0.00614989785659962\\
197.01	0.00614986840592174\\
198.01	0.00614983831933027\\
199.01	0.0061498075830404\\
200.01	0.00614977618296703\\
201.01	0.00614974410471874\\
202.01	0.0061497113335904\\
203.01	0.00614967785455659\\
204.01	0.00614964365226432\\
205.01	0.0061496087110263\\
206.01	0.00614957301481319\\
207.01	0.00614953654724642\\
208.01	0.00614949929158999\\
209.01	0.00614946123074312\\
210.01	0.00614942234723224\\
211.01	0.00614938262320241\\
212.01	0.00614934204040888\\
213.01	0.00614930058020941\\
214.01	0.00614925822355434\\
215.01	0.00614921495097818\\
216.01	0.0061491707425906\\
217.01	0.00614912557806666\\
218.01	0.00614907943663756\\
219.01	0.00614903229708059\\
220.01	0.00614898413770928\\
221.01	0.00614893493636313\\
222.01	0.00614888467039695\\
223.01	0.0061488333166703\\
224.01	0.00614878085153668\\
225.01	0.00614872725083177\\
226.01	0.00614867248986285\\
227.01	0.00614861654339601\\
228.01	0.00614855938564508\\
229.01	0.00614850099025888\\
230.01	0.00614844133030864\\
231.01	0.00614838037827548\\
232.01	0.0061483181060369\\
233.01	0.00614825448485373\\
234.01	0.00614818948535598\\
235.01	0.00614812307752912\\
236.01	0.00614805523069937\\
237.01	0.00614798591351937\\
238.01	0.00614791509395259\\
239.01	0.00614784273925873\\
240.01	0.00614776881597704\\
241.01	0.00614769328991072\\
242.01	0.00614761612611038\\
243.01	0.00614753728885672\\
244.01	0.00614745674164379\\
245.01	0.00614737444716088\\
246.01	0.00614729036727486\\
247.01	0.00614720446301071\\
248.01	0.00614711669453367\\
249.01	0.00614702702112888\\
250.01	0.00614693540118201\\
251.01	0.00614684179215854\\
252.01	0.00614674615058323\\
253.01	0.00614664843201824\\
254.01	0.00614654859104191\\
255.01	0.00614644658122558\\
256.01	0.00614634235511138\\
257.01	0.00614623586418795\\
258.01	0.00614612705886717\\
259.01	0.00614601588845865\\
260.01	0.00614590230114513\\
261.01	0.00614578624395606\\
262.01	0.00614566766274146\\
263.01	0.00614554650214442\\
264.01	0.0061454227055733\\
265.01	0.00614529621517315\\
266.01	0.00614516697179641\\
267.01	0.00614503491497296\\
268.01	0.00614489998287925\\
269.01	0.0061447621123068\\
270.01	0.00614462123862972\\
271.01	0.00614447729577154\\
272.01	0.00614433021617141\\
273.01	0.00614417993074869\\
274.01	0.00614402636886761\\
275.01	0.00614386945830026\\
276.01	0.00614370912518887\\
277.01	0.00614354529400727\\
278.01	0.00614337788752106\\
279.01	0.00614320682674676\\
280.01	0.00614303203091033\\
281.01	0.00614285341740365\\
282.01	0.00614267090174094\\
283.01	0.00614248439751313\\
284.01	0.00614229381634143\\
285.01	0.00614209906782946\\
286.01	0.00614190005951409\\
287.01	0.00614169669681516\\
288.01	0.00614148888298399\\
289.01	0.00614127651904899\\
290.01	0.00614105950376205\\
291.01	0.00614083773354134\\
292.01	0.00614061110241417\\
293.01	0.00614037950195624\\
294.01	0.00614014282123195\\
295.01	0.00613990094672996\\
296.01	0.00613965376229902\\
297.01	0.00613940114908108\\
298.01	0.00613914298544244\\
299.01	0.00613887914690307\\
300.01	0.0061386095060638\\
301.01	0.00613833393253115\\
302.01	0.00613805229283985\\
303.01	0.00613776445037347\\
304.01	0.00613747026528237\\
305.01	0.00613716959439855\\
306.01	0.00613686229114891\\
307.01	0.00613654820546506\\
308.01	0.00613622718369033\\
309.01	0.00613589906848424\\
310.01	0.00613556369872345\\
311.01	0.00613522090939985\\
312.01	0.00613487053151501\\
313.01	0.00613451239197168\\
314.01	0.00613414631346101\\
315.01	0.00613377211434705\\
316.01	0.00613338960854596\\
317.01	0.00613299860540252\\
318.01	0.00613259890956122\\
319.01	0.00613219032083414\\
320.01	0.00613177263406312\\
321.01	0.00613134563897781\\
322.01	0.00613090912004864\\
323.01	0.00613046285633399\\
324.01	0.00613000662132245\\
325.01	0.00612954018276931\\
326.01	0.00612906330252674\\
327.01	0.00612857573636803\\
328.01	0.00612807723380498\\
329.01	0.00612756753789875\\
330.01	0.00612704638506286\\
331.01	0.00612651350485966\\
332.01	0.00612596861978793\\
333.01	0.0061254114450629\\
334.01	0.0061248416883865\\
335.01	0.0061242590497101\\
336.01	0.00612366322098599\\
337.01	0.00612305388590973\\
338.01	0.00612243071965169\\
339.01	0.00612179338857729\\
340.01	0.00612114154995517\\
341.01	0.00612047485165395\\
342.01	0.00611979293182497\\
343.01	0.00611909541857141\\
344.01	0.00611838192960334\\
345.01	0.00611765207187742\\
346.01	0.00611690544121983\\
347.01	0.00611614162193266\\
348.01	0.00611536018638253\\
349.01	0.00611456069456946\\
350.01	0.00611374269367594\\
351.01	0.00611290571759487\\
352.01	0.00611204928643453\\
353.01	0.00611117290600028\\
354.01	0.00611027606725035\\
355.01	0.00610935824572522\\
356.01	0.00610841890094807\\
357.01	0.00610745747579621\\
358.01	0.0061064733958397\\
359.01	0.00610546606864639\\
360.01	0.00610443488305109\\
361.01	0.00610337920838679\\
362.01	0.00610229839367529\\
363.01	0.00610119176677524\\
364.01	0.00610005863348544\\
365.01	0.00609889827659873\\
366.01	0.00609770995490624\\
367.01	0.00609649290214711\\
368.01	0.00609524632590198\\
369.01	0.00609396940642553\\
370.01	0.00609266129541588\\
371.01	0.0060913211147181\\
372.01	0.00608994795495598\\
373.01	0.00608854087409203\\
374.01	0.00608709889590906\\
375.01	0.0060856210084131\\
376.01	0.00608410616215264\\
377.01	0.00608255326845319\\
378.01	0.00608096119756439\\
379.01	0.0060793287767183\\
380.01	0.00607765478809893\\
381.01	0.00607593796672505\\
382.01	0.00607417699824758\\
383.01	0.00607237051666689\\
384.01	0.00607051710197895\\
385.01	0.00606861527775943\\
386.01	0.00606666350870381\\
387.01	0.00606466019814232\\
388.01	0.0060626036855588\\
389.01	0.00606049224414927\\
390.01	0.00605832407846529\\
391.01	0.00605609732220219\\
392.01	0.00605381003620425\\
393.01	0.00605146020677961\\
394.01	0.00604904574443693\\
395.01	0.00604656448318412\\
396.01	0.00604401418055822\\
397.01	0.00604139251859296\\
398.01	0.00603869710597121\\
399.01	0.00603592548165764\\
400.01	0.00603307512036123\\
401.01	0.00603014344023608\\
402.01	0.00602712781329085\\
403.01	0.00602402557904241\\
404.01	0.00602083406200576\\
405.01	0.00601755059365846\\
406.01	0.00601417253953499\\
407.01	0.00601069733207518\\
408.01	0.00600712250973919\\
409.01	0.00600344576266342\\
410.01	0.0059996649846932\\
411.01	0.00599577833089774\\
412.01	0.00599178427849172\\
413.01	0.00598768168726121\\
414.01	0.00598346985280637\\
415.01	0.0059791485417527\\
416.01	0.00597471799193696\\
417.01	0.00597017885158271\\
418.01	0.00596553201843817\\
419.01	0.00596077832103965\\
420.01	0.00595591762523831\\
421.01	0.0059509476872939\\
422.01	0.00594586574767648\\
423.01	0.00594066895397797\\
424.01	0.00593535435606941\\
425.01	0.00592991890088308\\
426.01	0.00592435942678017\\
427.01	0.00591867265746136\\
428.01	0.00591285519537157\\
429.01	0.00590690351454442\\
430.01	0.00590081395282439\\
431.01	0.005894582703397\\
432.01	0.00588820580554942\\
433.01	0.00588167913457138\\
434.01	0.00587499839069569\\
435.01	0.00586815908696614\\
436.01	0.00586115653590091\\
437.01	0.0058539858348075\\
438.01	0.00584664184958114\\
439.01	0.00583911919679941\\
440.01	0.00583141222389821\\
441.01	0.00582351498718718\\
442.01	0.00581542122743015\\
443.01	0.00580712434267734\\
444.01	0.00579861735800195\\
445.01	0.00578989289174168\\
446.01	0.00578094311780526\\
447.01	0.00577175972354865\\
448.01	0.00576233386267786\\
449.01	0.00575265610258605\\
450.01	0.00574271636548881\\
451.01	0.00573250386268671\\
452.01	0.00572200702127258\\
453.01	0.00571121340261374\\
454.01	0.00570010961200627\\
455.01	0.00568868119903684\\
456.01	0.00567691254843265\\
457.01	0.00566478676158142\\
458.01	0.00565228552953682\\
459.01	0.00563938899926999\\
460.01	0.00562607563633821\\
461.01	0.00561232208917969\\
462.01	0.0055981030631865\\
463.01	0.00558339121688668\\
464.01	0.0055681570984796\\
465.01	0.0055523691492855\\
466.01	0.00553599381229492\\
467.01	0.00551899580022513\\
468.01	0.00550133860003272\\
469.01	0.00548298532209655\\
470.01	0.00546390004556607\\
471.01	0.00544404987119809\\
472.01	0.00542340797560858\\
473.01	0.00540195807482421\\
474.01	0.00537970086212681\\
475.01	0.00535667491186894\\
476.01	0.00533310316495461\\
477.01	0.00530905067600642\\
478.01	0.00528451324571092\\
479.01	0.00525948785542397\\
480.01	0.00523397292823664\\
481.01	0.00520796863302586\\
482.01	0.00518147723529276\\
483.01	0.00515450349988047\\
484.01	0.0051270551583203\\
485.01	0.00509914344624079\\
486.01	0.00507078370929517\\
487.01	0.00504199607500807\\
488.01	0.00501280618139569\\
489.01	0.00498324594232345\\
490.01	0.00495335431279067\\
491.01	0.00492317799198879\\
492.01	0.00489277196421238\\
493.01	0.00486219972203605\\
494.01	0.00483153294461179\\
495.01	0.00480085041618139\\
496.01	0.00477023590868901\\
497.01	0.0047397744102359\\
498.01	0.00470954587410876\\
499.01	0.00467961535984617\\
500.01	0.00465001797757927\\
501.01	0.00462062509242448\\
502.01	0.00459109249044627\\
503.01	0.00456143234565688\\
504.01	0.0045316820202934\\
505.01	0.00450188043189105\\
506.01	0.00447206739146352\\
507.01	0.00444228271748359\\
508.01	0.00441256508991101\\
509.01	0.00438295061273709\\
510.01	0.00435347106457543\\
511.01	0.00432415183879997\\
512.01	0.00429500961361187\\
513.01	0.00426604985698718\\
514.01	0.00423726437445556\\
515.01	0.00420862926745604\\
516.01	0.00418010391291735\\
517.01	0.00415163193823281\\
518.01	0.00412314570316703\\
519.01	0.00409458785638967\\
520.01	0.00406595094842425\\
521.01	0.00403723620157672\\
522.01	0.00400844050139541\\
523.01	0.00397955589621132\\
524.01	0.00395056922292003\\
525.01	0.00392146193374226\\
526.01	0.00389221021685914\\
527.01	0.00386278551309164\\
528.01	0.0038331555110655\\
529.01	0.00380328567100357\\
530.01	0.00377314126510747\\
531.01	0.0037426897917219\\
532.01	0.00371190338476699\\
533.01	0.00368076009306139\\
534.01	0.00364924023746064\\
535.01	0.00361732258221762\\
536.01	0.00358498429767557\\
537.01	0.00355220137212305\\
538.01	0.00351894910426884\\
539.01	0.00348520264598917\\
540.01	0.00345093754616732\\
541.01	0.00341613022608219\\
542.01	0.0033807582992128\\
543.01	0.00334480064213531\\
544.01	0.00330823714412511\\
545.01	0.00327104815352956\\
546.01	0.00323321401108179\\
547.01	0.00319471507490733\\
548.01	0.0031555318906373\\
549.01	0.00311564533753994\\
550.01	0.00307503673297587\\
551.01	0.00303368788459745\\
552.01	0.00299158108659081\\
553.01	0.00294869906736923\\
554.01	0.00290502491097292\\
555.01	0.00286054198984103\\
556.01	0.00281523395154636\\
557.01	0.00276908476086833\\
558.01	0.0027220787504204\\
559.01	0.00267420066112334\\
560.01	0.00262543567394102\\
561.01	0.00257576943734479\\
562.01	0.00252518809674998\\
563.01	0.00247367833295417\\
564.01	0.00242122741540764\\
565.01	0.0023678232722642\\
566.01	0.00231345457350701\\
567.01	0.00225811082177995\\
568.01	0.00220178245017577\\
569.01	0.00214446092893834\\
570.01	0.00208613888320133\\
571.01	0.00202681022346972\\
572.01	0.00196647028977617\\
573.01	0.00190511600947689\\
574.01	0.00184274606784671\\
575.01	0.00177936109045576\\
576.01	0.00171496383673893\\
577.01	0.00164955940429165\\
578.01	0.00158315544293594\\
579.01	0.00151576237682872\\
580.01	0.00144739363196437\\
581.01	0.00137806586538633\\
582.01	0.0013077991912756\\
583.01	0.00123661739776533\\
584.01	0.0011645481466601\\
585.01	0.00109162314601654\\
586.01	0.00101787828266926\\
587.01	0.000943353698183423\\
588.01	0.000868093787227074\\
589.01	0.000792147091763009\\
590.01	0.000715566057480832\\
591.01	0.000638406610165536\\
592.01	0.000560727498800357\\
593.01	0.000482589338632919\\
594.01	0.000404053270619676\\
595.01	0.000325179132900562\\
596.01	0.000246023014372898\\
597.01	0.000166666784466021\\
598.01	9.17366970063904e-05\\
599.01	2.94669364271135e-05\\
599.02	2.89574269585705e-05\\
599.03	2.84509530852801e-05\\
599.04	2.79475443462508e-05\\
599.05	2.7447230571279e-05\\
599.06	2.69500418838293e-05\\
599.07	2.64560087039432e-05\\
599.08	2.59651617511691e-05\\
599.09	2.54775320475235e-05\\
599.1	2.49931509204854e-05\\
599.11	2.45120500060175e-05\\
599.12	2.40342612516185e-05\\
599.13	2.35598169193978e-05\\
599.14	2.30887495892024e-05\\
599.15	2.26210921617422e-05\\
599.16	2.21568778617717e-05\\
599.17	2.16961402412993e-05\\
599.18	2.12389131828191e-05\\
599.19	2.07852309025824e-05\\
599.2	2.03351279538938e-05\\
599.21	1.98886392304542e-05\\
599.22	1.94457999697188e-05\\
599.23	1.90066457563046e-05\\
599.24	1.85712125254211e-05\\
599.25	1.81395365663351e-05\\
599.26	1.77116564095865e-05\\
599.27	1.72876122135658e-05\\
599.28	1.68674445361946e-05\\
599.29	1.64511943388859e-05\\
599.3	1.60389029905273e-05\\
599.31	1.56306122715087e-05\\
599.32	1.5226364377792e-05\\
599.33	1.48262019250105e-05\\
599.34	1.44301679526268e-05\\
599.35	1.40383059281154e-05\\
599.36	1.36506597511916e-05\\
599.37	1.32672737580847e-05\\
599.38	1.28881927258552e-05\\
599.39	1.25134618767404e-05\\
599.4	1.21431268825696e-05\\
599.41	1.17772338691924e-05\\
599.42	1.14158294209719e-05\\
599.43	1.10589605853192e-05\\
599.44	1.07066748772523e-05\\
599.45	1.03590202840433e-05\\
599.46	1.00160452698606e-05\\
599.47	9.67779878049274e-06\\
599.48	9.34433024810284e-06\\
599.49	9.01568959604283e-06\\
599.5	8.69192724369319e-06\\
599.51	8.3730941113757e-06\\
599.52	8.05924162529219e-06\\
599.53	7.75042172253965e-06\\
599.54	7.4466868561357e-06\\
599.55	7.14809000013084e-06\\
599.56	6.85468465475535e-06\\
599.57	6.56652485161308e-06\\
599.58	6.2836651589307e-06\\
599.59	6.00616068686076e-06\\
599.6	5.73406709284546e-06\\
599.61	5.46744058700123e-06\\
599.62	5.2063379376039e-06\\
599.63	4.95081647657741e-06\\
599.64	4.70093410508653e-06\\
599.65	4.45674929914347e-06\\
599.66	4.21832111529089e-06\\
599.67	3.98570919634376e-06\\
599.68	3.75897377716435e-06\\
599.69	3.53817569053415e-06\\
599.7	3.32337637303295e-06\\
599.71	3.11463787103054e-06\\
599.72	2.91202284668536e-06\\
599.73	2.71559458404555e-06\\
599.74	2.52541699518292e-06\\
599.75	2.34155462640155e-06\\
599.76	2.16407266449489e-06\\
599.77	1.99303694307928e-06\\
599.78	1.82851394897598e-06\\
599.79	1.67057082867302e-06\\
599.8	1.51927539483211e-06\\
599.81	1.37469613287027e-06\\
599.82	1.23690220760718e-06\\
599.83	1.10596346997172e-06\\
599.84	9.81950463782924e-07\\
599.85	8.64934432591793e-07\\
599.86	7.54987326587186e-07\\
599.87	6.5218180958504e-07\\
599.88	5.56591266064402e-07\\
599.89	4.68289808295413e-07\\
599.9	3.87352283521061e-07\\
599.91	3.13854281218731e-07\\
599.92	2.47872140425945e-07\\
599.93	1.89482957149364e-07\\
599.94	1.38764591834512e-07\\
599.95	9.57956769222224e-08\\
599.96	6.06556244606149e-08\\
599.97	3.34246338211386e-08\\
599.98	1.4183699454523e-08\\
599.99	3.01461875948372e-09\\
600	0\\
};
\addplot [color=red!80!mycolor19,solid,forget plot]
  table[row sep=crcr]{%
0.01	0.0060323088673989\\
1.01	0.00603230822979061\\
2.01	0.0060323075786237\\
3.01	0.00603230691360877\\
4.01	0.00603230623445046\\
5.01	0.00603230554084724\\
6.01	0.00603230483249098\\
7.01	0.00603230410906711\\
8.01	0.00603230337025409\\
9.01	0.00603230261572357\\
10.01	0.00603230184514019\\
11.01	0.00603230105816139\\
12.01	0.00603230025443744\\
13.01	0.00603229943361105\\
14.01	0.00603229859531677\\
15.01	0.00603229773918224\\
16.01	0.00603229686482659\\
17.01	0.00603229597186072\\
18.01	0.00603229505988737\\
19.01	0.00603229412850074\\
20.01	0.00603229317728638\\
21.01	0.00603229220582086\\
22.01	0.00603229121367175\\
23.01	0.0060322902003973\\
24.01	0.0060322891655463\\
25.01	0.00603228810865806\\
26.01	0.0060322870292616\\
27.01	0.00603228592687629\\
28.01	0.006032284801011\\
29.01	0.00603228365116394\\
30.01	0.00603228247682287\\
31.01	0.0060322812774641\\
32.01	0.00603228005255319\\
33.01	0.00603227880154409\\
34.01	0.00603227752387895\\
35.01	0.00603227621898803\\
36.01	0.00603227488628917\\
37.01	0.00603227352518803\\
38.01	0.00603227213507735\\
39.01	0.00603227071533685\\
40.01	0.00603226926533295\\
41.01	0.00603226778441839\\
42.01	0.00603226627193235\\
43.01	0.00603226472719929\\
44.01	0.0060322631495296\\
45.01	0.0060322615382187\\
46.01	0.00603225989254695\\
47.01	0.00603225821177918\\
48.01	0.00603225649516439\\
49.01	0.00603225474193552\\
50.01	0.00603225295130922\\
51.01	0.00603225112248514\\
52.01	0.00603224925464555\\
53.01	0.00603224734695528\\
54.01	0.00603224539856145\\
55.01	0.00603224340859254\\
56.01	0.00603224137615865\\
57.01	0.00603223930035039\\
58.01	0.00603223718023908\\
59.01	0.0060322350148757\\
60.01	0.00603223280329144\\
61.01	0.00603223054449612\\
62.01	0.00603222823747868\\
63.01	0.00603222588120604\\
64.01	0.00603222347462281\\
65.01	0.00603222101665137\\
66.01	0.00603221850619037\\
67.01	0.00603221594211521\\
68.01	0.00603221332327691\\
69.01	0.00603221064850169\\
70.01	0.00603220791659062\\
71.01	0.00603220512631906\\
72.01	0.00603220227643587\\
73.01	0.00603219936566315\\
74.01	0.00603219639269535\\
75.01	0.0060321933561987\\
76.01	0.00603219025481103\\
77.01	0.00603218708714056\\
78.01	0.00603218385176591\\
79.01	0.00603218054723475\\
80.01	0.0060321771720635\\
81.01	0.00603217372473684\\
82.01	0.00603217020370665\\
83.01	0.00603216660739147\\
84.01	0.00603216293417598\\
85.01	0.00603215918240966\\
86.01	0.00603215535040686\\
87.01	0.00603215143644542\\
88.01	0.00603214743876594\\
89.01	0.00603214335557143\\
90.01	0.0060321391850259\\
91.01	0.00603213492525388\\
92.01	0.0060321305743395\\
93.01	0.00603212613032547\\
94.01	0.00603212159121241\\
95.01	0.00603211695495789\\
96.01	0.00603211221947524\\
97.01	0.00603210738263305\\
98.01	0.00603210244225382\\
99.01	0.00603209739611303\\
100.01	0.00603209224193806\\
101.01	0.00603208697740756\\
102.01	0.00603208160015007\\
103.01	0.00603207610774268\\
104.01	0.00603207049771058\\
105.01	0.00603206476752528\\
106.01	0.00603205891460374\\
107.01	0.00603205293630724\\
108.01	0.00603204682994021\\
109.01	0.00603204059274872\\
110.01	0.00603203422191947\\
111.01	0.00603202771457836\\
112.01	0.00603202106778945\\
113.01	0.00603201427855305\\
114.01	0.0060320073438051\\
115.01	0.00603200026041496\\
116.01	0.00603199302518452\\
117.01	0.00603198563484681\\
118.01	0.00603197808606388\\
119.01	0.0060319703754262\\
120.01	0.00603196249944992\\
121.01	0.00603195445457652\\
122.01	0.00603194623717027\\
123.01	0.00603193784351689\\
124.01	0.00603192926982196\\
125.01	0.00603192051220891\\
126.01	0.00603191156671731\\
127.01	0.00603190242930123\\
128.01	0.0060318930958273\\
129.01	0.0060318835620728\\
130.01	0.00603187382372352\\
131.01	0.00603186387637232\\
132.01	0.00603185371551617\\
133.01	0.00603184333655526\\
134.01	0.00603183273478989\\
135.01	0.00603182190541884\\
136.01	0.00603181084353705\\
137.01	0.00603179954413329\\
138.01	0.0060317880020879\\
139.01	0.00603177621217043\\
140.01	0.00603176416903718\\
141.01	0.00603175186722916\\
142.01	0.00603173930116886\\
143.01	0.00603172646515809\\
144.01	0.00603171335337539\\
145.01	0.00603169995987326\\
146.01	0.00603168627857525\\
147.01	0.00603167230327385\\
148.01	0.0060316580276267\\
149.01	0.0060316434451543\\
150.01	0.00603162854923662\\
151.01	0.00603161333311041\\
152.01	0.0060315977898659\\
153.01	0.00603158191244361\\
154.01	0.0060315656936313\\
155.01	0.0060315491260604\\
156.01	0.00603153220220251\\
157.01	0.00603151491436623\\
158.01	0.00603149725469341\\
159.01	0.00603147921515594\\
160.01	0.00603146078755138\\
161.01	0.0060314419634995\\
162.01	0.00603142273443861\\
163.01	0.0060314030916214\\
164.01	0.00603138302611078\\
165.01	0.00603136252877606\\
166.01	0.00603134159028829\\
167.01	0.00603132020111665\\
168.01	0.00603129835152348\\
169.01	0.0060312760315597\\
170.01	0.00603125323106085\\
171.01	0.00603122993964158\\
172.01	0.0060312061466916\\
173.01	0.0060311818413703\\
174.01	0.00603115701260169\\
175.01	0.00603113164907001\\
176.01	0.00603110573921339\\
177.01	0.00603107927121927\\
178.01	0.00603105223301893\\
179.01	0.00603102461228137\\
180.01	0.00603099639640822\\
181.01	0.00603096757252764\\
182.01	0.00603093812748813\\
183.01	0.00603090804785276\\
184.01	0.00603087731989301\\
185.01	0.00603084592958219\\
186.01	0.00603081386258891\\
187.01	0.00603078110427054\\
188.01	0.00603074763966666\\
189.01	0.00603071345349168\\
190.01	0.00603067853012803\\
191.01	0.00603064285361903\\
192.01	0.00603060640766096\\
193.01	0.00603056917559613\\
194.01	0.00603053114040492\\
195.01	0.00603049228469773\\
196.01	0.00603045259070696\\
197.01	0.00603041204027912\\
198.01	0.00603037061486567\\
199.01	0.00603032829551533\\
200.01	0.00603028506286447\\
201.01	0.00603024089712847\\
202.01	0.00603019577809261\\
203.01	0.00603014968510258\\
204.01	0.0060301025970547\\
205.01	0.00603005449238637\\
206.01	0.00603000534906598\\
207.01	0.00602995514458247\\
208.01	0.006029903855935\\
209.01	0.00602985145962232\\
210.01	0.00602979793163174\\
211.01	0.00602974324742779\\
212.01	0.00602968738194122\\
213.01	0.00602963030955659\\
214.01	0.00602957200410113\\
215.01	0.00602951243883215\\
216.01	0.00602945158642428\\
217.01	0.00602938941895736\\
218.01	0.00602932590790274\\
219.01	0.00602926102411058\\
220.01	0.00602919473779578\\
221.01	0.00602912701852399\\
222.01	0.0060290578351978\\
223.01	0.00602898715604198\\
224.01	0.00602891494858824\\
225.01	0.0060288411796609\\
226.01	0.00602876581536028\\
227.01	0.00602868882104774\\
228.01	0.00602861016132892\\
229.01	0.00602852980003709\\
230.01	0.00602844770021664\\
231.01	0.00602836382410504\\
232.01	0.00602827813311566\\
233.01	0.00602819058781932\\
234.01	0.00602810114792604\\
235.01	0.00602800977226561\\
236.01	0.00602791641876861\\
237.01	0.0060278210444464\\
238.01	0.00602772360537084\\
239.01	0.00602762405665334\\
240.01	0.00602752235242419\\
241.01	0.00602741844581051\\
242.01	0.00602731228891381\\
243.01	0.00602720383278795\\
244.01	0.00602709302741537\\
245.01	0.00602697982168388\\
246.01	0.00602686416336153\\
247.01	0.00602674599907317\\
248.01	0.00602662527427365\\
249.01	0.00602650193322271\\
250.01	0.00602637591895818\\
251.01	0.00602624717326863\\
252.01	0.00602611563666591\\
253.01	0.00602598124835641\\
254.01	0.00602584394621202\\
255.01	0.00602570366674065\\
256.01	0.00602556034505534\\
257.01	0.00602541391484351\\
258.01	0.00602526430833456\\
259.01	0.00602511145626771\\
260.01	0.00602495528785824\\
261.01	0.00602479573076368\\
262.01	0.00602463271104836\\
263.01	0.00602446615314793\\
264.01	0.0060242959798326\\
265.01	0.00602412211216966\\
266.01	0.00602394446948496\\
267.01	0.0060237629693236\\
268.01	0.00602357752740974\\
269.01	0.00602338805760507\\
270.01	0.00602319447186705\\
271.01	0.00602299668020533\\
272.01	0.00602279459063738\\
273.01	0.00602258810914331\\
274.01	0.00602237713961935\\
275.01	0.00602216158383037\\
276.01	0.00602194134136069\\
277.01	0.00602171630956459\\
278.01	0.00602148638351458\\
279.01	0.00602125145594939\\
280.01	0.00602101141721989\\
281.01	0.00602076615523391\\
282.01	0.0060205155554001\\
283.01	0.00602025950056934\\
284.01	0.00601999787097592\\
285.01	0.0060197305441762\\
286.01	0.00601945739498642\\
287.01	0.00601917829541845\\
288.01	0.00601889311461384\\
289.01	0.00601860171877663\\
290.01	0.00601830397110407\\
291.01	0.00601799973171564\\
292.01	0.00601768885757962\\
293.01	0.00601737120243923\\
294.01	0.00601704661673481\\
295.01	0.00601671494752553\\
296.01	0.00601637603840791\\
297.01	0.00601602972943313\\
298.01	0.00601567585702081\\
299.01	0.0060153142538717\\
300.01	0.00601494474887713\\
301.01	0.00601456716702615\\
302.01	0.00601418132931047\\
303.01	0.00601378705262579\\
304.01	0.00601338414967121\\
305.01	0.00601297242884532\\
306.01	0.00601255169413933\\
307.01	0.00601212174502735\\
308.01	0.00601168237635293\\
309.01	0.00601123337821282\\
310.01	0.00601077453583678\\
311.01	0.00601030562946407\\
312.01	0.00600982643421604\\
313.01	0.00600933671996531\\
314.01	0.00600883625119968\\
315.01	0.00600832478688312\\
316.01	0.00600780208031201\\
317.01	0.0060072678789662\\
318.01	0.00600672192435638\\
319.01	0.00600616395186524\\
320.01	0.00600559369058467\\
321.01	0.00600501086314687\\
322.01	0.00600441518554961\\
323.01	0.00600380636697669\\
324.01	0.00600318410961098\\
325.01	0.00600254810844194\\
326.01	0.00600189805106614\\
327.01	0.00600123361748045\\
328.01	0.0060005544798687\\
329.01	0.00599986030237971\\
330.01	0.00599915074089787\\
331.01	0.00599842544280482\\
332.01	0.00599768404673316\\
333.01	0.00599692618230945\\
334.01	0.00599615146988878\\
335.01	0.00599535952027785\\
336.01	0.00599454993444854\\
337.01	0.00599372230323855\\
338.01	0.00599287620704164\\
339.01	0.00599201121548363\\
340.01	0.00599112688708665\\
341.01	0.00599022276891755\\
342.01	0.00598929839622296\\
343.01	0.00598835329204795\\
344.01	0.00598738696683767\\
345.01	0.00598639891802182\\
346.01	0.00598538862958029\\
347.01	0.00598435557158891\\
348.01	0.00598329919974391\\
349.01	0.00598221895486372\\
350.01	0.00598111426236792\\
351.01	0.00597998453172951\\
352.01	0.00597882915590072\\
353.01	0.00597764751070914\\
354.01	0.00597643895422407\\
355.01	0.00597520282608819\\
356.01	0.00597393844681501\\
357.01	0.005972645117047\\
358.01	0.00597132211677355\\
359.01	0.0059699687045044\\
360.01	0.00596858411639517\\
361.01	0.00596716756532282\\
362.01	0.00596571823990379\\
363.01	0.00596423530345384\\
364.01	0.00596271789288124\\
365.01	0.00596116511751003\\
366.01	0.00595957605782627\\
367.01	0.00595794976414015\\
368.01	0.0059562852551558\\
369.01	0.00595458151644177\\
370.01	0.00595283749879069\\
371.01	0.00595105211645827\\
372.01	0.00594922424526912\\
373.01	0.00594735272057594\\
374.01	0.00594543633505635\\
375.01	0.00594347383632999\\
376.01	0.00594146392437586\\
377.01	0.00593940524872865\\
378.01	0.00593729640542735\\
379.01	0.00593513593368828\\
380.01	0.00593292231227016\\
381.01	0.00593065395549255\\
382.01	0.00592832920886761\\
383.01	0.00592594634429556\\
384.01	0.00592350355476975\\
385.01	0.00592099894852966\\
386.01	0.00591843054258989\\
387.01	0.00591579625556634\\
388.01	0.00591309389970917\\
389.01	0.00591032117203818\\
390.01	0.00590747564446821\\
391.01	0.00590455475279379\\
392.01	0.00590155578439159\\
393.01	0.00589847586448267\\
394.01	0.00589531194078513\\
395.01	0.00589206076637255\\
396.01	0.00588871888055167\\
397.01	0.00588528258756502\\
398.01	0.00588174793293934\\
399.01	0.00587811067732352\\
400.01	0.00587436626771316\\
401.01	0.0058705098060435\\
402.01	0.0058665360152756\\
403.01	0.00586243920331443\\
404.01	0.00585821322542409\\
405.01	0.00585385144627785\\
406.01	0.005849346703474\\
407.01	0.00584469127533662\\
408.01	0.00583987685722575\\
409.01	0.00583489455256717\\
410.01	0.00582973488760384\\
411.01	0.00582438786276795\\
412.01	0.00581884305901457\\
413.01	0.00581308982501403\\
414.01	0.00580711758158385\\
415.01	0.00580091629425014\\
416.01	0.00579447718488472\\
417.01	0.00578779378104929\\
418.01	0.0057808634398562\\
419.01	0.00577368953577726\\
420.01	0.00576631687256892\\
421.01	0.00575878074051462\\
422.01	0.00575107735371502\\
423.01	0.0057432028327119\\
424.01	0.00573515320250625\\
425.01	0.00572692439066269\\
426.01	0.00571851222553454\\
427.01	0.00570991243464606\\
428.01	0.00570112064327694\\
429.01	0.00569213237330246\\
430.01	0.00568294304235371\\
431.01	0.00567354796337234\\
432.01	0.00566394234465059\\
433.01	0.00565412129046134\\
434.01	0.00564407980240533\\
435.01	0.00563381278162197\\
436.01	0.00562331503203997\\
437.01	0.0056125812648738\\
438.01	0.00560160610460731\\
439.01	0.00559038409675234\\
440.01	0.00557890971771511\\
441.01	0.00556717738716608\\
442.01	0.00555518148337079\\
443.01	0.00554291636202407\\
444.01	0.00553037637921097\\
445.01	0.00551755591922929\\
446.01	0.00550444942811777\\
447.01	0.00549105145387235\\
448.01	0.00547735669447731\\
449.01	0.00546336005504276\\
450.01	0.00544905671551638\\
451.01	0.00543444221062791\\
452.01	0.00541951252391299\\
453.01	0.0054042641978536\\
454.01	0.00538869446233392\\
455.01	0.0053728013837293\\
456.01	0.00535658403697551\\
457.01	0.00534004270286315\\
458.01	0.00532317909247237\\
459.01	0.00530599660000874\\
460.01	0.00528850058414319\\
461.01	0.00527069867609095\\
462.01	0.00525260110973188\\
463.01	0.00523422106466096\\
464.01	0.00521557500650392\\
465.01	0.00519668299926648\\
466.01	0.00517756895069433\\
467.01	0.00515826073188677\\
468.01	0.00513879008439024\\
469.01	0.00511919218843788\\
470.01	0.00509950471039199\\
471.01	0.00507976606957098\\
472.01	0.00506001255591459\\
473.01	0.00504027377851333\\
474.01	0.00502056571458162\\
475.01	0.00500086854867461\\
476.01	0.00498097279191445\\
477.01	0.0049608181976079\\
478.01	0.00494041518376095\\
479.01	0.00491977568273822\\
480.01	0.00489891321125773\\
481.01	0.00487784292242705\\
482.01	0.00485658163246677\\
483.01	0.00483514781308942\\
484.01	0.0048135615384342\\
485.01	0.00479184437305596\\
486.01	0.00477001918516127\\
487.01	0.00474810986696565\\
488.01	0.00472614094195711\\
489.01	0.00470413703747452\\
490.01	0.00468212220105582\\
491.01	0.00466011904158858\\
492.01	0.00463814768309418\\
493.01	0.00461622453253979\\
494.01	0.00459436088709398\\
495.01	0.00457256144371503\\
496.01	0.00455082282733694\\
497.01	0.00452913234196017\\
498.01	0.00450746728654395\\
499.01	0.00448579538001983\\
500.01	0.00446407713748063\\
501.01	0.00444227403134053\\
502.01	0.00442037839858218\\
503.01	0.00439839898302056\\
504.01	0.00437634322091555\\
505.01	0.00435421663797677\\
506.01	0.00433202245018829\\
507.01	0.00430976117040764\\
508.01	0.00428743024502981\\
509.01	0.00426502375278557\\
510.01	0.00424253220619599\\
511.01	0.00421994250452162\\
512.01	0.00419723809369644\\
513.01	0.00417439939107087\\
514.01	0.00415140452641962\\
515.01	0.00412823042857954\\
516.01	0.00410485423826421\\
517.01	0.00408125493514216\\
518.01	0.00405741490545617\\
519.01	0.00403332066448421\\
520.01	0.00400896013218617\\
521.01	0.00398431983134321\\
522.01	0.0039593847362626\\
523.01	0.00393413843157042\\
524.01	0.00390856333801112\\
525.01	0.0038826410022877\\
526.01	0.0038563524398284\\
527.01	0.0038296785084463\\
528.01	0.00380260027802109\\
529.01	0.00377509934822052\\
530.01	0.00374715805511664\\
531.01	0.00371875950392218\\
532.01	0.00368988737917599\\
533.01	0.0036605255388462\\
534.01	0.00363065763087537\\
535.01	0.00360026706979577\\
536.01	0.00356933716041013\\
537.01	0.00353785121829806\\
538.01	0.00350579267201484\\
539.01	0.0034731451374028\\
540.01	0.00343989245617953\\
541.01	0.0034060186949134\\
542.01	0.003371508107079\\
543.01	0.0033363450700003\\
544.01	0.00330051401884971\\
545.01	0.00326399940755064\\
546.01	0.00322678571573412\\
547.01	0.00318885747810126\\
548.01	0.00315019930833668\\
549.01	0.00311079591342006\\
550.01	0.00307063209961312\\
551.01	0.00302969277287801\\
552.01	0.00298796293775365\\
553.01	0.0029454276994024\\
554.01	0.00290207227312257\\
555.01	0.00285788200362993\\
556.01	0.0028128423928549\\
557.01	0.00276693913212747\\
558.01	0.00272015813645232\\
559.01	0.00267248558161509\\
560.01	0.00262390794559945\\
561.01	0.00257441205578024\\
562.01	0.00252398514312515\\
563.01	0.00247261490420945\\
564.01	0.00242028957133667\\
565.01	0.00236699799067106\\
566.01	0.00231272970830979\\
567.01	0.00225747506467842\\
568.01	0.0022012252979938\\
569.01	0.00214397265754098\\
570.01	0.0020857105273739\\
571.01	0.00202643356088241\\
572.01	0.00196613782650036\\
573.01	0.00190482096468999\\
574.01	0.00184248235623363\\
575.01	0.00177912330175362\\
576.01	0.00171474721218609\\
577.01	0.00164935980960222\\
578.01	0.00158296933731052\\
579.01	0.001515586777603\\
580.01	0.00144722607478906\\
581.01	0.0013779043602682\\
582.01	0.00130764217525753\\
583.01	0.00123646368534738\\
584.01	0.00116439687921647\\
585.01	0.00109147374150863\\
586.01	0.00101773038694345\\
587.01	0.000943207139070279\\
588.01	0.000867948532496969\\
589.01	0.000792003211719553\\
590.01	0.000715423692577445\\
591.01	0.000638265943538507\\
592.01	0.000560588733085159\\
593.01	0.000482452675945771\\
594.01	0.00040391889420434\\
595.01	0.000325047188695244\\
596.01	0.000245893590649514\\
597.01	0.000166581383471795\\
598.01	9.17366970063782e-05\\
599.01	2.94669364271152e-05\\
599.02	2.89574269585705e-05\\
599.03	2.84509530852819e-05\\
599.04	2.79475443462525e-05\\
599.05	2.74472305712807e-05\\
599.06	2.6950041883831e-05\\
599.07	2.64560087039449e-05\\
599.08	2.59651617511691e-05\\
599.09	2.54775320475218e-05\\
599.1	2.49931509204854e-05\\
599.11	2.45120500060175e-05\\
599.12	2.40342612516167e-05\\
599.13	2.35598169193996e-05\\
599.14	2.30887495892024e-05\\
599.15	2.26210921617405e-05\\
599.16	2.215687786177e-05\\
599.17	2.16961402412993e-05\\
599.18	2.12389131828191e-05\\
599.19	2.07852309025824e-05\\
599.2	2.03351279538938e-05\\
599.21	1.98886392304542e-05\\
599.22	1.94457999697206e-05\\
599.23	1.90066457563063e-05\\
599.24	1.85712125254211e-05\\
599.25	1.81395365663334e-05\\
599.26	1.77116564095865e-05\\
599.27	1.72876122135658e-05\\
599.28	1.68674445361946e-05\\
599.29	1.64511943388877e-05\\
599.3	1.60389029905273e-05\\
599.31	1.56306122715087e-05\\
599.32	1.52263643777902e-05\\
599.33	1.48262019250105e-05\\
599.34	1.44301679526268e-05\\
599.35	1.40383059281154e-05\\
599.36	1.36506597511916e-05\\
599.37	1.32672737580865e-05\\
599.38	1.28881927258535e-05\\
599.39	1.25134618767404e-05\\
599.4	1.21431268825679e-05\\
599.41	1.17772338691924e-05\\
599.42	1.14158294209736e-05\\
599.43	1.10589605853174e-05\\
599.44	1.07066748772523e-05\\
599.45	1.03590202840433e-05\\
599.46	1.00160452698589e-05\\
599.47	9.67779878049101e-06\\
599.48	9.34433024810284e-06\\
599.49	9.0156895960411e-06\\
599.5	8.69192724369146e-06\\
599.51	8.37309411137396e-06\\
599.52	8.05924162529392e-06\\
599.53	7.75042172253965e-06\\
599.54	7.44668685613396e-06\\
599.55	7.14809000013084e-06\\
599.56	6.85468465475535e-06\\
599.57	6.56652485161308e-06\\
599.58	6.28366515892896e-06\\
599.59	6.00616068686249e-06\\
599.6	5.73406709284546e-06\\
599.61	5.46744058700296e-06\\
599.62	5.20633793760217e-06\\
599.63	4.95081647657741e-06\\
599.64	4.70093410508653e-06\\
599.65	4.45674929914347e-06\\
599.66	4.21832111529262e-06\\
599.67	3.98570919634203e-06\\
599.68	3.75897377716608e-06\\
599.69	3.53817569053415e-06\\
599.7	3.32337637303469e-06\\
599.71	3.11463787102881e-06\\
599.72	2.91202284668363e-06\\
599.73	2.71559458404382e-06\\
599.74	2.52541699518466e-06\\
599.75	2.34155462640329e-06\\
599.76	2.16407266449489e-06\\
599.77	1.99303694307928e-06\\
599.78	1.82851394897598e-06\\
599.79	1.67057082867302e-06\\
599.8	1.51927539483211e-06\\
599.81	1.37469613287027e-06\\
599.82	1.23690220760544e-06\\
599.83	1.10596346997172e-06\\
599.84	9.81950463782924e-07\\
599.85	8.64934432591793e-07\\
599.86	7.54987326588921e-07\\
599.87	6.5218180958504e-07\\
599.88	5.56591266064402e-07\\
599.89	4.68289808295413e-07\\
599.9	3.87352283521061e-07\\
599.91	3.13854281216996e-07\\
599.92	2.47872140425945e-07\\
599.93	1.8948295714763e-07\\
599.94	1.38764591834512e-07\\
599.95	9.57956769204876e-08\\
599.96	6.06556244606149e-08\\
599.97	3.34246338194039e-08\\
599.98	1.4183699454523e-08\\
599.99	3.014618757749e-09\\
600	0\\
};
\addplot [color=red,solid,forget plot]
  table[row sep=crcr]{%
0.01	0.00579827411309351\\
1.01	0.00579827314537753\\
2.01	0.00579827215707644\\
3.01	0.00579827114775116\\
4.01	0.00579827011695319\\
5.01	0.00579826906422434\\
6.01	0.00579826798909636\\
7.01	0.00579826689109139\\
8.01	0.00579826576972118\\
9.01	0.0057982646244869\\
10.01	0.00579826345487936\\
11.01	0.00579826226037807\\
12.01	0.00579826104045152\\
13.01	0.00579825979455707\\
14.01	0.00579825852214016\\
15.01	0.0057982572226343\\
16.01	0.00579825589546109\\
17.01	0.00579825454002963\\
18.01	0.00579825315573628\\
19.01	0.00579825174196466\\
20.01	0.00579825029808493\\
21.01	0.00579824882345383\\
22.01	0.0057982473174144\\
23.01	0.00579824577929556\\
24.01	0.00579824420841174\\
25.01	0.00579824260406261\\
26.01	0.0057982409655332\\
27.01	0.00579823929209277\\
28.01	0.00579823758299505\\
29.01	0.00579823583747786\\
30.01	0.00579823405476249\\
31.01	0.00579823223405377\\
32.01	0.00579823037453908\\
33.01	0.00579822847538871\\
34.01	0.00579822653575484\\
35.01	0.00579822455477176\\
36.01	0.00579822253155499\\
37.01	0.00579822046520106\\
38.01	0.0057982183547872\\
39.01	0.00579821619937075\\
40.01	0.00579821399798874\\
41.01	0.00579821174965755\\
42.01	0.00579820945337251\\
43.01	0.00579820710810723\\
44.01	0.00579820471281346\\
45.01	0.00579820226642009\\
46.01	0.00579819976783343\\
47.01	0.00579819721593598\\
48.01	0.00579819460958627\\
49.01	0.0057981919476184\\
50.01	0.00579818922884112\\
51.01	0.00579818645203786\\
52.01	0.00579818361596585\\
53.01	0.00579818071935561\\
54.01	0.00579817776091003\\
55.01	0.00579817473930455\\
56.01	0.00579817165318576\\
57.01	0.00579816850117141\\
58.01	0.0057981652818493\\
59.01	0.00579816199377695\\
60.01	0.00579815863548088\\
61.01	0.00579815520545563\\
62.01	0.00579815170216373\\
63.01	0.00579814812403414\\
64.01	0.00579814446946259\\
65.01	0.00579814073680959\\
66.01	0.00579813692440098\\
67.01	0.0057981330305261\\
68.01	0.00579812905343762\\
69.01	0.00579812499135079\\
70.01	0.00579812084244191\\
71.01	0.00579811660484832\\
72.01	0.00579811227666712\\
73.01	0.00579810785595438\\
74.01	0.00579810334072441\\
75.01	0.0057980987289487\\
76.01	0.00579809401855495\\
77.01	0.00579808920742611\\
78.01	0.0057980842933996\\
79.01	0.00579807927426606\\
80.01	0.0057980741477687\\
81.01	0.00579806891160204\\
82.01	0.00579806356341071\\
83.01	0.0057980581007887\\
84.01	0.00579805252127776\\
85.01	0.00579804682236706\\
86.01	0.00579804100149125\\
87.01	0.00579803505602974\\
88.01	0.00579802898330541\\
89.01	0.00579802278058322\\
90.01	0.00579801644506926\\
91.01	0.00579800997390934\\
92.01	0.00579800336418742\\
93.01	0.00579799661292453\\
94.01	0.00579798971707748\\
95.01	0.00579798267353733\\
96.01	0.00579797547912801\\
97.01	0.00579796813060459\\
98.01	0.00579796062465232\\
99.01	0.00579795295788471\\
100.01	0.00579794512684214\\
101.01	0.0057979371279901\\
102.01	0.00579792895771785\\
103.01	0.00579792061233645\\
104.01	0.00579791208807725\\
105.01	0.00579790338109002\\
106.01	0.00579789448744148\\
107.01	0.00579788540311315\\
108.01	0.0057978761239998\\
109.01	0.00579786664590704\\
110.01	0.00579785696455001\\
111.01	0.00579784707555114\\
112.01	0.00579783697443797\\
113.01	0.00579782665664153\\
114.01	0.00579781611749366\\
115.01	0.00579780535222541\\
116.01	0.00579779435596459\\
117.01	0.00579778312373305\\
118.01	0.00579777165044552\\
119.01	0.00579775993090621\\
120.01	0.00579774795980698\\
121.01	0.00579773573172455\\
122.01	0.00579772324111822\\
123.01	0.00579771048232739\\
124.01	0.00579769744956855\\
125.01	0.00579768413693305\\
126.01	0.00579767053838438\\
127.01	0.0057976566477548\\
128.01	0.00579764245874324\\
129.01	0.00579762796491163\\
130.01	0.00579761315968266\\
131.01	0.00579759803633612\\
132.01	0.00579758258800648\\
133.01	0.0057975668076788\\
134.01	0.00579755068818649\\
135.01	0.00579753422220721\\
136.01	0.00579751740225975\\
137.01	0.00579750022070049\\
138.01	0.00579748266972034\\
139.01	0.00579746474134026\\
140.01	0.00579744642740845\\
141.01	0.0057974277195958\\
142.01	0.00579740860939252\\
143.01	0.00579738908810407\\
144.01	0.00579736914684723\\
145.01	0.00579734877654559\\
146.01	0.00579732796792604\\
147.01	0.00579730671151351\\
148.01	0.00579728499762742\\
149.01	0.00579726281637671\\
150.01	0.00579724015765566\\
151.01	0.00579721701113865\\
152.01	0.00579719336627584\\
153.01	0.00579716921228806\\
154.01	0.00579714453816153\\
155.01	0.00579711933264326\\
156.01	0.00579709358423587\\
157.01	0.00579706728119147\\
158.01	0.00579704041150698\\
159.01	0.00579701296291797\\
160.01	0.0057969849228934\\
161.01	0.00579695627862977\\
162.01	0.00579692701704473\\
163.01	0.00579689712477128\\
164.01	0.00579686658815173\\
165.01	0.00579683539323069\\
166.01	0.00579680352574962\\
167.01	0.00579677097113907\\
168.01	0.00579673771451243\\
169.01	0.00579670374065922\\
170.01	0.00579666903403737\\
171.01	0.00579663357876651\\
172.01	0.00579659735862035\\
173.01	0.0057965603570189\\
174.01	0.00579652255702124\\
175.01	0.00579648394131712\\
176.01	0.00579644449221892\\
177.01	0.00579640419165386\\
178.01	0.00579636302115499\\
179.01	0.0057963209618527\\
180.01	0.00579627799446601\\
181.01	0.00579623409929379\\
182.01	0.00579618925620505\\
183.01	0.00579614344462948\\
184.01	0.00579609664354852\\
185.01	0.00579604883148476\\
186.01	0.00579599998649227\\
187.01	0.00579595008614644\\
188.01	0.0057958991075329\\
189.01	0.00579584702723753\\
190.01	0.00579579382133497\\
191.01	0.00579573946537734\\
192.01	0.00579568393438316\\
193.01	0.00579562720282547\\
194.01	0.00579556924461956\\
195.01	0.00579551003311128\\
196.01	0.005795449541064\\
197.01	0.00579538774064614\\
198.01	0.00579532460341794\\
199.01	0.0057952601003182\\
200.01	0.00579519420165056\\
201.01	0.00579512687706953\\
202.01	0.00579505809556621\\
203.01	0.0057949878254534\\
204.01	0.00579491603435147\\
205.01	0.00579484268917235\\
206.01	0.00579476775610444\\
207.01	0.00579469120059632\\
208.01	0.00579461298734054\\
209.01	0.00579453308025713\\
210.01	0.00579445144247664\\
211.01	0.00579436803632249\\
212.01	0.00579428282329334\\
213.01	0.00579419576404477\\
214.01	0.00579410681837108\\
215.01	0.00579401594518572\\
216.01	0.00579392310250235\\
217.01	0.00579382824741506\\
218.01	0.00579373133607749\\
219.01	0.00579363232368251\\
220.01	0.00579353116444109\\
221.01	0.00579342781156036\\
222.01	0.00579332221722175\\
223.01	0.00579321433255784\\
224.01	0.00579310410762988\\
225.01	0.00579299149140342\\
226.01	0.00579287643172464\\
227.01	0.00579275887529527\\
228.01	0.00579263876764722\\
229.01	0.00579251605311728\\
230.01	0.00579239067481969\\
231.01	0.00579226257462006\\
232.01	0.00579213169310728\\
233.01	0.00579199796956499\\
234.01	0.00579186134194323\\
235.01	0.00579172174682847\\
236.01	0.00579157911941347\\
237.01	0.00579143339346638\\
238.01	0.00579128450129919\\
239.01	0.00579113237373528\\
240.01	0.00579097694007667\\
241.01	0.00579081812806965\\
242.01	0.00579065586387095\\
243.01	0.00579049007201172\\
244.01	0.00579032067536187\\
245.01	0.00579014759509287\\
246.01	0.00578997075063997\\
247.01	0.00578979005966341\\
248.01	0.00578960543800922\\
249.01	0.00578941679966862\\
250.01	0.0057892240567364\\
251.01	0.00578902711936903\\
252.01	0.00578882589574102\\
253.01	0.00578862029200088\\
254.01	0.00578841021222596\\
255.01	0.00578819555837566\\
256.01	0.00578797623024446\\
257.01	0.0057877521254132\\
258.01	0.00578752313919994\\
259.01	0.00578728916460873\\
260.01	0.0057870500922779\\
261.01	0.00578680581042684\\
262.01	0.00578655620480192\\
263.01	0.00578630115862098\\
264.01	0.0057860405525161\\
265.01	0.0057857742644753\\
266.01	0.00578550216978347\\
267.01	0.00578522414096144\\
268.01	0.00578494004770291\\
269.01	0.00578464975681151\\
270.01	0.00578435313213467\\
271.01	0.00578405003449716\\
272.01	0.00578374032163303\\
273.01	0.00578342384811487\\
274.01	0.00578310046528246\\
275.01	0.00578277002116907\\
276.01	0.0057824323604267\\
277.01	0.00578208732424903\\
278.01	0.00578173475029253\\
279.01	0.00578137447259588\\
280.01	0.00578100632149719\\
281.01	0.00578063012354972\\
282.01	0.0057802457014351\\
283.01	0.00577985287387483\\
284.01	0.00577945145553875\\
285.01	0.00577904125695298\\
286.01	0.00577862208440381\\
287.01	0.00577819373984031\\
288.01	0.00577775602077444\\
289.01	0.00577730872017823\\
290.01	0.00577685162637893\\
291.01	0.00577638452295131\\
292.01	0.00577590718860751\\
293.01	0.00577541939708381\\
294.01	0.00577492091702463\\
295.01	0.00577441151186419\\
296.01	0.00577389093970455\\
297.01	0.00577335895319051\\
298.01	0.00577281529938182\\
299.01	0.00577225971962204\\
300.01	0.00577169194940389\\
301.01	0.00577111171823091\\
302.01	0.00577051874947646\\
303.01	0.00576991276023777\\
304.01	0.00576929346118716\\
305.01	0.00576866055641959\\
306.01	0.00576801374329498\\
307.01	0.00576735271227751\\
308.01	0.00576667714677085\\
309.01	0.00576598672294736\\
310.01	0.00576528110957491\\
311.01	0.0057645599678377\\
312.01	0.00576382295115243\\
313.01	0.00576306970497961\\
314.01	0.00576229986663068\\
315.01	0.00576151306506771\\
316.01	0.00576070892069997\\
317.01	0.00575988704517318\\
318.01	0.00575904704115399\\
319.01	0.0057581885021081\\
320.01	0.00575731101207152\\
321.01	0.00575641414541665\\
322.01	0.00575549746661117\\
323.01	0.00575456052996957\\
324.01	0.00575360287939838\\
325.01	0.005752624048134\\
326.01	0.00575162355847247\\
327.01	0.00575060092149164\\
328.01	0.00574955563676572\\
329.01	0.00574848719207092\\
330.01	0.00574739506308279\\
331.01	0.00574627871306462\\
332.01	0.00574513759254621\\
333.01	0.00574397113899348\\
334.01	0.00574277877646861\\
335.01	0.0057415599152792\\
336.01	0.00574031395161721\\
337.01	0.0057390402671877\\
338.01	0.00573773822882488\\
339.01	0.00573640718809818\\
340.01	0.00573504648090548\\
341.01	0.00573365542705417\\
342.01	0.00573223332982955\\
343.01	0.00573077947555017\\
344.01	0.00572929313311008\\
345.01	0.00572777355350623\\
346.01	0.00572621996935286\\
347.01	0.00572463159438094\\
348.01	0.00572300762292222\\
349.01	0.00572134722937925\\
350.01	0.00571964956767855\\
351.01	0.00571791377070902\\
352.01	0.00571613894974415\\
353.01	0.00571432419384813\\
354.01	0.00571246856926574\\
355.01	0.00571057111879649\\
356.01	0.00570863086115212\\
357.01	0.00570664679029909\\
358.01	0.00570461787478514\\
359.01	0.00570254305705154\\
360.01	0.00570042125273152\\
361.01	0.00569825134993498\\
362.01	0.0056960322085237\\
363.01	0.00569376265937483\\
364.01	0.00569144150363884\\
365.01	0.0056890675119921\\
366.01	0.00568663942388841\\
367.01	0.00568415594681422\\
368.01	0.00568161575555128\\
369.01	0.00567901749145439\\
370.01	0.00567635976175055\\
371.01	0.00567364113886814\\
372.01	0.0056708601598084\\
373.01	0.00566801532556912\\
374.01	0.00566510510063793\\
375.01	0.00566212791257145\\
376.01	0.00565908215168273\\
377.01	0.00565596617085954\\
378.01	0.00565277828554576\\
379.01	0.00564951677391856\\
380.01	0.00564617987730409\\
381.01	0.00564276580087828\\
382.01	0.00563927271471126\\
383.01	0.00563569875522094\\
384.01	0.00563204202711578\\
385.01	0.00562830060591658\\
386.01	0.00562447254116693\\
387.01	0.00562055586045687\\
388.01	0.00561654857440534\\
389.01	0.00561244868277319\\
390.01	0.00560825418190138\\
391.01	0.00560396307370325\\
392.01	0.00559957337646969\\
393.01	0.00559508313778593\\
394.01	0.00559049044989649\\
395.01	0.00558579346789798\\
396.01	0.00558099043117951\\
397.01	0.00557607968857266\\
398.01	0.00557105972770409\\
399.01	0.00556592920906667\\
400.01	0.00556068700532121\\
401.01	0.00555533224630952\\
402.01	0.00554986437016528\\
403.01	0.00554428318074\\
404.01	0.0055385889112695\\
405.01	0.00553278229373854\\
406.01	0.00552686463268604\\
407.01	0.00552083788111343\\
408.01	0.00551470471457791\\
409.01	0.0055084685972551\\
410.01	0.00550213383045587\\
411.01	0.00549570556939092\\
412.01	0.00548918978734055\\
413.01	0.0054825931570487\\
414.01	0.00547592280609357\\
415.01	0.00546918588474558\\
416.01	0.00546238885944615\\
417.01	0.00545553640981832\\
418.01	0.00544862975833197\\
419.01	0.00544166419427558\\
420.01	0.00543459249843498\\
421.01	0.00542737374615082\\
422.01	0.00542000527256111\\
423.01	0.00541248440613606\\
424.01	0.00540480847372568\\
425.01	0.00539697480624451\\
426.01	0.00538898074505829\\
427.01	0.0053808236491421\\
428.01	0.00537250090308864\\
429.01	0.00536400992604847\\
430.01	0.00535534818169118\\
431.01	0.00534651318928491\\
432.01	0.00533750253599418\\
433.01	0.00532831389050632\\
434.01	0.00531894501809819\\
435.01	0.00530939379726229\\
436.01	0.00529965823801376\\
437.01	0.00528973650200004\\
438.01	0.00527962692453595\\
439.01	0.00526932803867838\\
440.01	0.00525883860145041\\
441.01	0.00524815762230244\\
442.01	0.00523728439388126\\
443.01	0.00522621852513802\\
444.01	0.00521495997676421\\
445.01	0.00520350909887938\\
446.01	0.00519186667081365\\
447.01	0.00518003394271962\\
448.01	0.00516801267861244\\
449.01	0.00515580520026193\\
450.01	0.00514341443114205\\
451.01	0.00513084393937215\\
452.01	0.00511809797825092\\
453.01	0.00510518152257694\\
454.01	0.0050921002984596\\
455.01	0.00507886080374135\\
456.01	0.00506547031546919\\
457.01	0.00505193688006244\\
458.01	0.00503826928093613\\
459.01	0.00502447697736356\\
460.01	0.00501057000734561\\
461.01	0.0049965588462518\\
462.01	0.00498245421214625\\
463.01	0.00496826680818446\\
464.01	0.00495400699256695\\
465.01	0.00493968436772454\\
466.01	0.00492530728334791\\
467.01	0.00491088225358604\\
468.01	0.00489641329867669\\
469.01	0.00488190123757946\\
470.01	0.00486734298392155\\
471.01	0.0048527309371367\\
472.01	0.00483805262034577\\
473.01	0.0048232908051748\\
474.01	0.00480842449390443\\
475.01	0.00479343143458618\\
476.01	0.00477830033022567\\
477.01	0.00476303646284201\\
478.01	0.00474764684725635\\
479.01	0.00473213867838251\\
480.01	0.00471651923707135\\
481.01	0.00470079577686061\\
482.01	0.00468497539039543\\
483.01	0.00466906485480435\\
484.01	0.00465307045607682\\
485.01	0.00463699779360229\\
486.01	0.00462085156753712\\
487.01	0.00460463535367401\\
488.01	0.00458835137305685\\
489.01	0.00457200026678467\\
490.01	0.00455558089029488\\
491.01	0.0045390901458345\\
492.01	0.00452252287660589\\
493.01	0.0045058718507412\\
494.01	0.00448912786693964\\
495.01	0.00447228001483836\\
496.01	0.00445531611978078\\
497.01	0.00443822338997197\\
498.01	0.00442098925806168\\
499.01	0.00440360236002616\\
500.01	0.00438605350842562\\
501.01	0.00436833633741155\\
502.01	0.00435044623385012\\
503.01	0.00433237797366445\\
504.01	0.00431412521693101\\
505.01	0.00429568048089654\\
506.01	0.00427703514314936\\
507.01	0.00425817947909612\\
508.01	0.00423910273824293\\
509.01	0.00421979326213203\\
510.01	0.00420023864409566\\
511.01	0.00418042592702429\\
512.01	0.00416034182994515\\
513.01	0.00413997298735496\\
514.01	0.00411930617726896\\
515.01	0.00409832850577516\\
516.01	0.00407702750949197\\
517.01	0.00405539113648135\\
518.01	0.00403340757746375\\
519.01	0.00401106495778801\\
520.01	0.00398835105619131\\
521.01	0.00396525327661902\\
522.01	0.00394175872708765\\
523.01	0.00391785430582586\\
524.01	0.00389352678603324\\
525.01	0.00386876289331736\\
526.01	0.00384354936936821\\
527.01	0.00381787301565537\\
528.01	0.00379172071226016\\
529.01	0.00376507940970956\\
530.01	0.00373793609609847\\
531.01	0.00371027774778276\\
532.01	0.00368209127865747\\
533.01	0.00365336350813047\\
534.01	0.00362408116246383\\
535.01	0.00359423089568912\\
536.01	0.0035637993083127\\
537.01	0.00353277295895014\\
538.01	0.00350113836867015\\
539.01	0.00346888201873439\\
540.01	0.00343599034334621\\
541.01	0.00340244971988049\\
542.01	0.0033682464596358\\
543.01	0.00333336680210739\\
544.01	0.00329779691482852\\
545.01	0.00326152289880476\\
546.01	0.00322453079706372\\
547.01	0.00318680660347261\\
548.01	0.00314833627127922\\
549.01	0.00310910572210278\\
550.01	0.0030691008562936\\
551.01	0.00302830756557809\\
552.01	0.00298671174877585\\
553.01	0.00294429933111708\\
554.01	0.00290105628736226\\
555.01	0.00285696866863559\\
556.01	0.002812022632798\\
557.01	0.00276620447842027\\
558.01	0.00271950068276585\\
559.01	0.0026718979443291\\
560.01	0.0026233832304573\\
561.01	0.00257394383053661\\
562.01	0.00252356741517104\\
563.01	0.00247224210175046\\
564.01	0.00241995652680611\\
565.01	0.00236669992559503\\
566.01	0.00231246221942173\\
567.01	0.00225723411125625\\
568.01	0.00220100719020869\\
569.01	0.00214377404539634\\
570.01	0.00208552838970178\\
571.01	0.00202626519387332\\
572.01	0.00196598083135666\\
573.01	0.00190467323416349\\
574.01	0.00184234205996039\\
575.01	0.00177898887038548\\
576.01	0.00171461732035613\\
577.01	0.00164923335779841\\
578.01	0.00158284543280214\\
579.01	0.00151546471464598\\
580.01	0.00144710531442435\\
581.01	0.00137778451008954\\
582.01	0.00130752296955477\\
583.01	0.00123634496602094\\
584.01	0.00116427857781718\\
585.01	0.00109135586269028\\
586.01	0.00101761299352925\\
587.01	0.000943090338822979\\
588.01	0.000867832466552738\\
589.01	0.000791888044500708\\
590.01	0.000715309602850512\\
591.01	0.00063815311614427\\
592.01	0.000560477350747159\\
593.01	0.000482342910469856\\
594.01	0.000403810896310433\\
595.01	0.000324941075665367\\
596.01	0.00024578943091586\\
597.01	0.000166525623744458\\
598.01	9.17366970063782e-05\\
599.01	2.94669364271135e-05\\
599.02	2.89574269585688e-05\\
599.03	2.84509530852801e-05\\
599.04	2.79475443462508e-05\\
599.05	2.7447230571279e-05\\
599.06	2.69500418838293e-05\\
599.07	2.64560087039432e-05\\
599.08	2.59651617511691e-05\\
599.09	2.54775320475235e-05\\
599.1	2.49931509204836e-05\\
599.11	2.45120500060175e-05\\
599.12	2.40342612516185e-05\\
599.13	2.35598169193996e-05\\
599.14	2.30887495892024e-05\\
599.15	2.26210921617405e-05\\
599.16	2.215687786177e-05\\
599.17	2.16961402412976e-05\\
599.18	2.12389131828191e-05\\
599.19	2.07852309025824e-05\\
599.2	2.03351279538938e-05\\
599.21	1.98886392304542e-05\\
599.22	1.94457999697188e-05\\
599.23	1.90066457563063e-05\\
599.24	1.85712125254211e-05\\
599.25	1.81395365663334e-05\\
599.26	1.77116564095865e-05\\
599.27	1.72876122135658e-05\\
599.28	1.68674445361946e-05\\
599.29	1.64511943388859e-05\\
599.3	1.60389029905273e-05\\
599.31	1.56306122715087e-05\\
599.32	1.5226364377792e-05\\
599.33	1.48262019250105e-05\\
599.34	1.44301679526268e-05\\
599.35	1.40383059281154e-05\\
599.36	1.36506597511916e-05\\
599.37	1.32672737580847e-05\\
599.38	1.28881927258535e-05\\
599.39	1.25134618767404e-05\\
599.4	1.21431268825696e-05\\
599.41	1.17772338691924e-05\\
599.42	1.14158294209719e-05\\
599.43	1.10589605853174e-05\\
599.44	1.07066748772523e-05\\
599.45	1.03590202840433e-05\\
599.46	1.00160452698606e-05\\
599.47	9.67779878049101e-06\\
599.48	9.34433024810111e-06\\
599.49	9.01568959604283e-06\\
599.5	8.69192724369319e-06\\
599.51	8.37309411137396e-06\\
599.52	8.05924162529219e-06\\
599.53	7.75042172253791e-06\\
599.54	7.4466868561357e-06\\
599.55	7.14809000013084e-06\\
599.56	6.85468465475535e-06\\
599.57	6.56652485161134e-06\\
599.58	6.28366515892896e-06\\
599.59	6.00616068686249e-06\\
599.6	5.73406709284546e-06\\
599.61	5.46744058700296e-06\\
599.62	5.2063379376039e-06\\
599.63	4.95081647657741e-06\\
599.64	4.70093410508479e-06\\
599.65	4.45674929914174e-06\\
599.66	4.21832111529089e-06\\
599.67	3.98570919634376e-06\\
599.68	3.75897377716608e-06\\
599.69	3.53817569053241e-06\\
599.7	3.32337637303295e-06\\
599.71	3.11463787102881e-06\\
599.72	2.91202284668363e-06\\
599.73	2.71559458404555e-06\\
599.74	2.52541699518292e-06\\
599.75	2.34155462640155e-06\\
599.76	2.16407266449663e-06\\
599.77	1.99303694307755e-06\\
599.78	1.82851394897598e-06\\
599.79	1.67057082867475e-06\\
599.8	1.51927539483211e-06\\
599.81	1.37469613287027e-06\\
599.82	1.23690220760718e-06\\
599.83	1.10596346997172e-06\\
599.84	9.81950463784659e-07\\
599.85	8.64934432590059e-07\\
599.86	7.54987326587186e-07\\
599.87	6.52181809583305e-07\\
599.88	5.56591266064402e-07\\
599.89	4.68289808295413e-07\\
599.9	3.87352283521061e-07\\
599.91	3.13854281218731e-07\\
599.92	2.4787214042421e-07\\
599.93	1.8948295714763e-07\\
599.94	1.38764591834512e-07\\
599.95	9.57956769204876e-08\\
599.96	6.06556244606149e-08\\
599.97	3.34246338194039e-08\\
599.98	1.4183699454523e-08\\
599.99	3.01461875948372e-09\\
600	0\\
};
\addplot [color=mycolor20,solid,forget plot]
  table[row sep=crcr]{%
0.01	0.00553720637056298\\
1.01	0.00553720532562888\\
2.01	0.00553720425853025\\
3.01	0.00553720316879582\\
4.01	0.0055372020559441\\
5.01	0.00553720091948356\\
6.01	0.00553719975891235\\
7.01	0.00553719857371746\\
8.01	0.00553719736337522\\
9.01	0.00553719612735087\\
10.01	0.00553719486509796\\
11.01	0.00553719357605868\\
12.01	0.0055371922596631\\
13.01	0.00553719091532918\\
14.01	0.00553718954246257\\
15.01	0.00553718814045607\\
16.01	0.00553718670868974\\
17.01	0.00553718524653015\\
18.01	0.00553718375333059\\
19.01	0.00553718222843034\\
20.01	0.00553718067115473\\
21.01	0.00553717908081458\\
22.01	0.0055371774567061\\
23.01	0.00553717579811048\\
24.01	0.00553717410429338\\
25.01	0.005537172374505\\
26.01	0.00553717060797924\\
27.01	0.00553716880393391\\
28.01	0.00553716696157009\\
29.01	0.0055371650800717\\
30.01	0.00553716315860519\\
31.01	0.0055371611963193\\
32.01	0.00553715919234463\\
33.01	0.00553715714579308\\
34.01	0.0055371550557576\\
35.01	0.00553715292131169\\
36.01	0.00553715074150919\\
37.01	0.00553714851538364\\
38.01	0.00553714624194788\\
39.01	0.00553714392019385\\
40.01	0.00553714154909165\\
41.01	0.00553713912758965\\
42.01	0.00553713665461338\\
43.01	0.00553713412906581\\
44.01	0.00553713154982601\\
45.01	0.00553712891574957\\
46.01	0.00553712622566704\\
47.01	0.00553712347838437\\
48.01	0.00553712067268193\\
49.01	0.00553711780731382\\
50.01	0.00553711488100765\\
51.01	0.00553711189246362\\
52.01	0.00553710884035421\\
53.01	0.0055371057233236\\
54.01	0.00553710253998694\\
55.01	0.00553709928892959\\
56.01	0.00553709596870648\\
57.01	0.00553709257784203\\
58.01	0.00553708911482873\\
59.01	0.0055370855781272\\
60.01	0.00553708196616446\\
61.01	0.0055370782773344\\
62.01	0.00553707450999615\\
63.01	0.00553707066247405\\
64.01	0.00553706673305628\\
65.01	0.00553706271999447\\
66.01	0.00553705862150265\\
67.01	0.00553705443575672\\
68.01	0.00553705016089349\\
69.01	0.00553704579500971\\
70.01	0.00553704133616168\\
71.01	0.00553703678236353\\
72.01	0.0055370321315874\\
73.01	0.00553702738176148\\
74.01	0.0055370225307697\\
75.01	0.00553701757645073\\
76.01	0.00553701251659681\\
77.01	0.00553700734895287\\
78.01	0.00553700207121549\\
79.01	0.00553699668103181\\
80.01	0.0055369911759986\\
81.01	0.00553698555366081\\
82.01	0.00553697981151105\\
83.01	0.0055369739469879\\
84.01	0.00553696795747521\\
85.01	0.00553696184030048\\
86.01	0.00553695559273409\\
87.01	0.00553694921198756\\
88.01	0.00553694269521272\\
89.01	0.00553693603950039\\
90.01	0.0055369292418787\\
91.01	0.00553692229931204\\
92.01	0.00553691520869987\\
93.01	0.00553690796687468\\
94.01	0.00553690057060118\\
95.01	0.00553689301657463\\
96.01	0.00553688530141914\\
97.01	0.00553687742168652\\
98.01	0.00553686937385437\\
99.01	0.00553686115432469\\
100.01	0.00553685275942206\\
101.01	0.00553684418539229\\
102.01	0.00553683542840027\\
103.01	0.00553682648452875\\
104.01	0.00553681734977601\\
105.01	0.00553680802005464\\
106.01	0.00553679849118908\\
107.01	0.0055367887589142\\
108.01	0.00553677881887289\\
109.01	0.00553676866661483\\
110.01	0.00553675829759354\\
111.01	0.00553674770716508\\
112.01	0.00553673689058559\\
113.01	0.00553672584300898\\
114.01	0.00553671455948508\\
115.01	0.00553670303495715\\
116.01	0.00553669126425974\\
117.01	0.00553667924211642\\
118.01	0.00553666696313701\\
119.01	0.00553665442181534\\
120.01	0.00553664161252666\\
121.01	0.00553662852952538\\
122.01	0.00553661516694214\\
123.01	0.00553660151878125\\
124.01	0.00553658757891789\\
125.01	0.00553657334109543\\
126.01	0.00553655879892246\\
127.01	0.00553654394587013\\
128.01	0.00553652877526866\\
129.01	0.00553651328030499\\
130.01	0.00553649745401897\\
131.01	0.00553648128930096\\
132.01	0.00553646477888751\\
133.01	0.00553644791535928\\
134.01	0.00553643069113668\\
135.01	0.00553641309847707\\
136.01	0.00553639512947085\\
137.01	0.00553637677603802\\
138.01	0.00553635802992419\\
139.01	0.00553633888269745\\
140.01	0.0055363193257438\\
141.01	0.0055362993502634\\
142.01	0.0055362789472671\\
143.01	0.00553625810757184\\
144.01	0.0055362368217962\\
145.01	0.0055362150803568\\
146.01	0.00553619287346342\\
147.01	0.00553617019111457\\
148.01	0.00553614702309314\\
149.01	0.00553612335896175\\
150.01	0.00553609918805758\\
151.01	0.00553607449948804\\
152.01	0.00553604928212504\\
153.01	0.00553602352460071\\
154.01	0.00553599721530188\\
155.01	0.00553597034236451\\
156.01	0.00553594289366837\\
157.01	0.00553591485683214\\
158.01	0.00553588621920657\\
159.01	0.00553585696786985\\
160.01	0.00553582708962081\\
161.01	0.0055357965709733\\
162.01	0.00553576539814999\\
163.01	0.00553573355707596\\
164.01	0.00553570103337233\\
165.01	0.0055356678123497\\
166.01	0.00553563387900109\\
167.01	0.00553559921799565\\
168.01	0.00553556381367109\\
169.01	0.00553552765002675\\
170.01	0.0055354907107165\\
171.01	0.00553545297904048\\
172.01	0.00553541443793807\\
173.01	0.00553537506998019\\
174.01	0.00553533485736078\\
175.01	0.00553529378188889\\
176.01	0.00553525182498051\\
177.01	0.00553520896764966\\
178.01	0.00553516519049994\\
179.01	0.00553512047371552\\
180.01	0.00553507479705218\\
181.01	0.00553502813982761\\
182.01	0.00553498048091255\\
183.01	0.00553493179872054\\
184.01	0.00553488207119826\\
185.01	0.00553483127581537\\
186.01	0.00553477938955405\\
187.01	0.00553472638889845\\
188.01	0.00553467224982427\\
189.01	0.00553461694778693\\
190.01	0.00553456045771075\\
191.01	0.00553450275397771\\
192.01	0.00553444381041517\\
193.01	0.0055343836002841\\
194.01	0.00553432209626664\\
195.01	0.00553425927045353\\
196.01	0.00553419509433156\\
197.01	0.00553412953876997\\
198.01	0.0055340625740075\\
199.01	0.00553399416963834\\
200.01	0.00553392429459836\\
201.01	0.0055338529171507\\
202.01	0.00553378000487124\\
203.01	0.00553370552463382\\
204.01	0.00553362944259458\\
205.01	0.00553355172417663\\
206.01	0.00553347233405399\\
207.01	0.0055333912361354\\
208.01	0.00553330839354808\\
209.01	0.00553322376862004\\
210.01	0.00553313732286324\\
211.01	0.00553304901695543\\
212.01	0.00553295881072252\\
213.01	0.00553286666311976\\
214.01	0.00553277253221284\\
215.01	0.00553267637515866\\
216.01	0.00553257814818546\\
217.01	0.00553247780657256\\
218.01	0.00553237530463007\\
219.01	0.00553227059567756\\
220.01	0.00553216363202278\\
221.01	0.00553205436493915\\
222.01	0.00553194274464385\\
223.01	0.00553182872027455\\
224.01	0.00553171223986592\\
225.01	0.0055315932503259\\
226.01	0.00553147169741097\\
227.01	0.00553134752570107\\
228.01	0.00553122067857429\\
229.01	0.00553109109818053\\
230.01	0.00553095872541492\\
231.01	0.00553082349989038\\
232.01	0.00553068535990977\\
233.01	0.00553054424243772\\
234.01	0.00553040008307083\\
235.01	0.00553025281600856\\
236.01	0.00553010237402275\\
237.01	0.00552994868842619\\
238.01	0.00552979168904128\\
239.01	0.00552963130416764\\
240.01	0.00552946746054834\\
241.01	0.00552930008333689\\
242.01	0.00552912909606222\\
243.01	0.00552895442059329\\
244.01	0.00552877597710296\\
245.01	0.0055285936840316\\
246.01	0.00552840745804875\\
247.01	0.00552821721401476\\
248.01	0.00552802286494169\\
249.01	0.00552782432195278\\
250.01	0.00552762149424138\\
251.01	0.00552741428902916\\
252.01	0.00552720261152324\\
253.01	0.00552698636487199\\
254.01	0.00552676545012045\\
255.01	0.00552653976616447\\
256.01	0.00552630920970405\\
257.01	0.00552607367519561\\
258.01	0.00552583305480285\\
259.01	0.00552558723834706\\
260.01	0.0055253361132561\\
261.01	0.00552507956451263\\
262.01	0.00552481747460019\\
263.01	0.00552454972344926\\
264.01	0.00552427618838158\\
265.01	0.00552399674405377\\
266.01	0.00552371126239857\\
267.01	0.00552341961256598\\
268.01	0.00552312166086285\\
269.01	0.00552281727069066\\
270.01	0.00552250630248259\\
271.01	0.00552218861363884\\
272.01	0.00552186405846033\\
273.01	0.00552153248808176\\
274.01	0.00552119375040224\\
275.01	0.00552084769001537\\
276.01	0.00552049414813723\\
277.01	0.00552013296253245\\
278.01	0.00551976396743991\\
279.01	0.00551938699349545\\
280.01	0.00551900186765388\\
281.01	0.0055186084131092\\
282.01	0.00551820644921202\\
283.01	0.00551779579138705\\
284.01	0.00551737625104743\\
285.01	0.00551694763550737\\
286.01	0.00551650974789376\\
287.01	0.00551606238705486\\
288.01	0.00551560534746743\\
289.01	0.00551513841914275\\
290.01	0.00551466138752897\\
291.01	0.00551417403341258\\
292.01	0.00551367613281776\\
293.01	0.00551316745690295\\
294.01	0.00551264777185604\\
295.01	0.00551211683878689\\
296.01	0.00551157441361747\\
297.01	0.00551102024697017\\
298.01	0.00551045408405385\\
299.01	0.00550987566454698\\
300.01	0.00550928472247866\\
301.01	0.00550868098610768\\
302.01	0.00550806417779831\\
303.01	0.00550743401389427\\
304.01	0.00550679020459034\\
305.01	0.00550613245380003\\
306.01	0.00550546045902275\\
307.01	0.00550477391120687\\
308.01	0.00550407249461015\\
309.01	0.00550335588665917\\
310.01	0.00550262375780388\\
311.01	0.00550187577137102\\
312.01	0.00550111158341431\\
313.01	0.00550033084256196\\
314.01	0.00549953318986141\\
315.01	0.00549871825862176\\
316.01	0.00549788567425313\\
317.01	0.00549703505410395\\
318.01	0.00549616600729461\\
319.01	0.00549527813454928\\
320.01	0.0054943710280256\\
321.01	0.00549344427114077\\
322.01	0.00549249743839581\\
323.01	0.0054915300951978\\
324.01	0.00549054179767959\\
325.01	0.0054895320925169\\
326.01	0.00548850051674458\\
327.01	0.00548744659757026\\
328.01	0.00548636985218627\\
329.01	0.00548526978758043\\
330.01	0.00548414590034619\\
331.01	0.00548299767649076\\
332.01	0.00548182459124306\\
333.01	0.00548062610886197\\
334.01	0.00547940168244379\\
335.01	0.0054781507537308\\
336.01	0.00547687275292085\\
337.01	0.00547556709847771\\
338.01	0.00547423319694501\\
339.01	0.00547287044276185\\
340.01	0.00547147821808205\\
341.01	0.00547005589259885\\
342.01	0.00546860282337389\\
343.01	0.00546711835467275\\
344.01	0.00546560181780855\\
345.01	0.00546405253099437\\
346.01	0.00546246979920531\\
347.01	0.0054608529140528\\
348.01	0.0054592011536728\\
349.01	0.00545751378262847\\
350.01	0.00545579005183191\\
351.01	0.00545402919848476\\
352.01	0.00545223044604209\\
353.01	0.00545039300420147\\
354.01	0.00544851606892138\\
355.01	0.00544659882247069\\
356.01	0.00544464043351565\\
357.01	0.00544264005724628\\
358.01	0.00544059683554877\\
359.01	0.0054385098972271\\
360.01	0.00543637835828238\\
361.01	0.00543420132225272\\
362.01	0.00543197788062396\\
363.01	0.00542970711331608\\
364.01	0.00542738808925562\\
365.01	0.00542501986704131\\
366.01	0.00542260149571503\\
367.01	0.0054201320156464\\
368.01	0.00541761045954511\\
369.01	0.00541503585361179\\
370.01	0.00541240721884241\\
371.01	0.00540972357250119\\
372.01	0.00540698392977664\\
373.01	0.00540418730563962\\
374.01	0.00540133271692067\\
375.01	0.00539841918462585\\
376.01	0.00539544573651215\\
377.01	0.00539241140994294\\
378.01	0.00538931525504518\\
379.01	0.0053861563381919\\
380.01	0.00538293374582873\\
381.01	0.00537964658867044\\
382.01	0.00537629400628247\\
383.01	0.00537287517206913\\
384.01	0.00536938929867904\\
385.01	0.0053658356438392\\
386.01	0.00536221351661703\\
387.01	0.0053585222841025\\
388.01	0.00535476137848928\\
389.01	0.00535093030451537\\
390.01	0.00534702864720414\\
391.01	0.00534305607981669\\
392.01	0.00533901237189306\\
393.01	0.00533489739721644\\
394.01	0.00533071114147749\\
395.01	0.00532645370935609\\
396.01	0.00532212533064996\\
397.01	0.00531772636498756\\
398.01	0.00531325730454725\\
399.01	0.00530871877406405\\
400.01	0.00530411152725457\\
401.01	0.00529943643860297\\
402.01	0.00529469448925469\\
403.01	0.00528988674554758\\
404.01	0.00528501432848217\\
405.01	0.00528007837222782\\
406.01	0.00527507996957568\\
407.01	0.00527002010216799\\
408.01	0.00526489955338293\\
409.01	0.00525971880207216\\
410.01	0.00525447789606453\\
411.01	0.0052491763057043\\
412.01	0.00524381275999656\\
413.01	0.00523838507168268\\
414.01	0.00523288996340713\\
415.01	0.00522732291605919\\
416.01	0.00522167807372655\\
417.01	0.00521594825942429\\
418.01	0.00521012518459015\\
419.01	0.00520419997708913\\
420.01	0.00519816522500772\\
421.01	0.00519201847717648\\
422.01	0.0051857584488522\\
423.01	0.00517938390285342\\
424.01	0.00517289365542751\\
425.01	0.00516628658248699\\
426.01	0.00515956162622112\\
427.01	0.00515271780208505\\
428.01	0.00514575420616415\\
429.01	0.00513867002290691\\
430.01	0.00513146453321292\\
431.01	0.00512413712285335\\
432.01	0.00511668729119532\\
433.01	0.00510911466018588\\
434.01	0.00510141898354174\\
435.01	0.0050936001560743\\
436.01	0.00508565822306024\\
437.01	0.00507759338954814\\
438.01	0.00506940602946684\\
439.01	0.00506109669437276\\
440.01	0.00505266612164202\\
441.01	0.00504411524187867\\
442.01	0.00503544518526552\\
443.01	0.00502665728654644\\
444.01	0.00501775308827338\\
445.01	0.00500873434190417\\
446.01	0.00499960300627798\\
447.01	0.0049903612429404\\
448.01	0.00498101140773014\\
449.01	0.00497155603798732\\
450.01	0.00496199783469326\\
451.01	0.00495233963881487\\
452.01	0.00494258440110989\\
453.01	0.00493273514465536\\
454.01	0.00492279491941333\\
455.01	0.00491276674824503\\
456.01	0.00490265356395906\\
457.01	0.00489245813724484\\
458.01	0.00488218299572643\\
459.01	0.00487183033490665\\
460.01	0.00486140192248767\\
461.01	0.00485089899848782\\
462.01	0.00484032217475886\\
463.01	0.00482967133896136\\
464.01	0.00481894556979449\\
465.01	0.00480814307225671\\
466.01	0.00479726114385588\\
467.01	0.00478629618479182\\
468.01	0.00477524376685169\\
469.01	0.00476409877648323\\
470.01	0.00475285564626067\\
471.01	0.00474150868417128\\
472.01	0.00473005249939949\\
473.01	0.00471848250288801\\
474.01	0.00470679542534797\\
475.01	0.00469498973555899\\
476.01	0.00468306554036529\\
477.01	0.00467102333681361\\
478.01	0.00465886335633616\\
479.01	0.00464658549313939\\
480.01	0.00463418925027865\\
481.01	0.00462167368557904\\
482.01	0.00460903735892477\\
483.01	0.00459627828277447\\
484.01	0.00458339387811139\\
485.01	0.00457038093836366\\
486.01	0.00455723560412103\\
487.01	0.00454395335166238\\
488.01	0.00453052899834742\\
489.01	0.00451695672772696\\
490.01	0.00450323013668929\\
491.01	0.00448934230598205\\
492.01	0.00447528589388215\\
493.01	0.00446105325052688\\
494.01	0.00444663654737016\\
495.01	0.00443202791238837\\
496.01	0.00441721955719538\\
497.01	0.0044022038776158\\
498.01	0.00438697350549951\\
499.01	0.00437152128856624\\
500.01	0.00435584018016472\\
501.01	0.00433992303804354\\
502.01	0.00432376239648625\\
503.01	0.00430735037702502\\
504.01	0.00429067870800565\\
505.01	0.00427373876057116\\
506.01	0.00425652159158788\\
507.01	0.00423901799212023\\
508.01	0.00422121853940893\\
509.01	0.00420311364962767\\
510.01	0.00418469362806572\\
511.01	0.00416594871292255\\
512.01	0.00414686910874527\\
513.01	0.00412744500587848\\
514.01	0.00410766658334731\\
515.01	0.0040875239945441\\
516.01	0.00406700733806003\\
517.01	0.00404610661987399\\
518.01	0.00402481171727932\\
519.01	0.00400311235781672\\
520.01	0.00398099812237176\\
521.01	0.00395845846277685\\
522.01	0.00393548271928023\\
523.01	0.0039120601342729\\
524.01	0.00388817986146693\\
525.01	0.00386383097008843\\
526.01	0.00383900244409569\\
527.01	0.00381368317698033\\
528.01	0.00378786196329684\\
529.01	0.0037615274885997\\
530.01	0.00373466831978622\\
531.01	0.00370727289777221\\
532.01	0.00367932953377338\\
533.01	0.00365082640913506\\
534.01	0.00362175157696972\\
535.01	0.00359209296338246\\
536.01	0.00356183836755163\\
537.01	0.00353097546093829\\
538.01	0.00349949178610511\\
539.01	0.00346737475569416\\
540.01	0.00343461165211991\\
541.01	0.00340118962845988\\
542.01	0.00336709571087169\\
543.01	0.00333231680265383\\
544.01	0.00329683968985111\\
545.01	0.0032606510481884\\
546.01	0.00322373745118108\\
547.01	0.00318608537950776\\
548.01	0.00314768123191225\\
549.01	0.00310851133792719\\
550.01	0.00306856197269586\\
551.01	0.0030278193741351\\
552.01	0.00298626976264715\\
553.01	0.00294389936356977\\
554.01	0.0029006944325494\\
555.01	0.0028566412840523\\
556.01	0.00281172632328039\\
557.01	0.00276593608181439\\
558.01	0.00271925725734165\\
559.01	0.0026716767578422\\
560.01	0.00262318175062\\
561.01	0.00257375971658235\\
562.01	0.0025233985101925\\
563.01	0.00247208642554797\\
564.01	0.0024198122690649\\
565.01	0.00236656543928021\\
566.01	0.00231233601430398\\
567.01	0.00225711484746698\\
568.01	0.0022008936717135\\
569.01	0.00214366521328238\\
570.01	0.00208542331519992\\
571.01	0.0020261630710739\\
572.01	0.0019658809696171\\
573.01	0.00190457505024007\\
574.01	0.00184224506991961\\
575.01	0.00177889268136865\\
576.01	0.00171452162228293\\
577.01	0.00164913791511233\\
578.01	0.00158275007636962\\
579.01	0.00151536933392877\\
580.01	0.00144700985004043\\
581.01	0.0013776889468657\\
582.01	0.0013074273301515\\
583.01	0.00123624930517962\\
584.01	0.00116418297724289\\
585.01	0.00109126042654351\\
586.01	0.00101751784445324\\
587.01	0.000942995614387319\\
588.01	0.000867738315945504\\
589.01	0.000791794625255108\\
590.01	0.00071521707734773\\
591.01	0.000638061647592614\\
592.01	0.000560387098297354\\
593.01	0.000482254023083268\\
594.01	0.000403723504947367\\
595.01	0.000324855283297511\\
596.01	0.000245705299785164\\
597.01	0.000166480802755067\\
598.01	9.17366970063765e-05\\
599.01	2.94669364271135e-05\\
599.02	2.89574269585705e-05\\
599.03	2.84509530852801e-05\\
599.04	2.79475443462508e-05\\
599.05	2.7447230571279e-05\\
599.06	2.69500418838293e-05\\
599.07	2.64560087039432e-05\\
599.08	2.59651617511673e-05\\
599.09	2.54775320475218e-05\\
599.1	2.49931509204836e-05\\
599.11	2.45120500060158e-05\\
599.12	2.40342612516167e-05\\
599.13	2.35598169193978e-05\\
599.14	2.30887495892024e-05\\
599.15	2.26210921617405e-05\\
599.16	2.215687786177e-05\\
599.17	2.16961402412993e-05\\
599.18	2.12389131828191e-05\\
599.19	2.07852309025806e-05\\
599.2	2.03351279538921e-05\\
599.21	1.98886392304542e-05\\
599.22	1.94457999697188e-05\\
599.23	1.90066457563046e-05\\
599.24	1.85712125254211e-05\\
599.25	1.81395365663334e-05\\
599.26	1.77116564095865e-05\\
599.27	1.72876122135658e-05\\
599.28	1.68674445361946e-05\\
599.29	1.64511943388859e-05\\
599.3	1.60389029905273e-05\\
599.31	1.56306122715087e-05\\
599.32	1.52263643777902e-05\\
599.33	1.48262019250087e-05\\
599.34	1.44301679526268e-05\\
599.35	1.40383059281154e-05\\
599.36	1.36506597511916e-05\\
599.37	1.32672737580847e-05\\
599.38	1.28881927258535e-05\\
599.39	1.25134618767404e-05\\
599.4	1.21431268825696e-05\\
599.41	1.17772338691924e-05\\
599.42	1.14158294209719e-05\\
599.43	1.10589605853174e-05\\
599.44	1.0706674877254e-05\\
599.45	1.03590202840433e-05\\
599.46	1.00160452698589e-05\\
599.47	9.67779878049101e-06\\
599.48	9.34433024810284e-06\\
599.49	9.0156895960411e-06\\
599.5	8.69192724369146e-06\\
599.51	8.37309411137396e-06\\
599.52	8.05924162529392e-06\\
599.53	7.75042172253965e-06\\
599.54	7.44668685613396e-06\\
599.55	7.14809000013084e-06\\
599.56	6.85468465475708e-06\\
599.57	6.56652485161308e-06\\
599.58	6.2836651589307e-06\\
599.59	6.00616068686249e-06\\
599.6	5.7340670928472e-06\\
599.61	5.46744058700296e-06\\
599.62	5.20633793760217e-06\\
599.63	4.95081647657741e-06\\
599.64	4.70093410508653e-06\\
599.65	4.45674929914347e-06\\
599.66	4.21832111529089e-06\\
599.67	3.98570919634376e-06\\
599.68	3.75897377716608e-06\\
599.69	3.53817569053415e-06\\
599.7	3.32337637303469e-06\\
599.71	3.11463787102881e-06\\
599.72	2.91202284668536e-06\\
599.73	2.71559458404555e-06\\
599.74	2.52541699518466e-06\\
599.75	2.34155462640155e-06\\
599.76	2.16407266449489e-06\\
599.77	1.99303694307928e-06\\
599.78	1.82851394897598e-06\\
599.79	1.67057082867302e-06\\
599.8	1.51927539483211e-06\\
599.81	1.37469613287027e-06\\
599.82	1.23690220760718e-06\\
599.83	1.10596346997172e-06\\
599.84	9.81950463782924e-07\\
599.85	8.64934432591793e-07\\
599.86	7.54987326587186e-07\\
599.87	6.5218180958504e-07\\
599.88	5.56591266062667e-07\\
599.89	4.68289808295413e-07\\
599.9	3.87352283521061e-07\\
599.91	3.13854281216996e-07\\
599.92	2.47872140425945e-07\\
599.93	1.89482957149364e-07\\
599.94	1.38764591834512e-07\\
599.95	9.57956769222224e-08\\
599.96	6.06556244623496e-08\\
599.97	3.34246338211386e-08\\
599.98	1.4183699454523e-08\\
599.99	3.01461875948372e-09\\
600	0\\
};
\addplot [color=mycolor21,solid,forget plot]
  table[row sep=crcr]{%
0.01	0.00535026209346621\\
1.01	0.00535026104829641\\
2.01	0.00535025998105595\\
3.01	0.00535025889127814\\
4.01	0.0053502577784865\\
5.01	0.00535025664219418\\
6.01	0.00535025548190425\\
7.01	0.00535025429710914\\
8.01	0.00535025308729064\\
9.01	0.00535025185191931\\
10.01	0.00535025059045471\\
11.01	0.0053502493023449\\
12.01	0.00535024798702625\\
13.01	0.00535024664392316\\
14.01	0.00535024527244783\\
15.01	0.0053502438720001\\
16.01	0.00535024244196682\\
17.01	0.00535024098172227\\
18.01	0.005350239490627\\
19.01	0.00535023796802848\\
20.01	0.00535023641325996\\
21.01	0.0053502348256408\\
22.01	0.00535023320447572\\
23.01	0.00535023154905479\\
24.01	0.00535022985865317\\
25.01	0.00535022813253049\\
26.01	0.00535022636993082\\
27.01	0.00535022457008205\\
28.01	0.00535022273219562\\
29.01	0.00535022085546657\\
30.01	0.00535021893907261\\
31.01	0.00535021698217404\\
32.01	0.0053502149839134\\
33.01	0.00535021294341511\\
34.01	0.005350210859785\\
35.01	0.00535020873210986\\
36.01	0.00535020655945719\\
37.01	0.00535020434087453\\
38.01	0.0053502020753894\\
39.01	0.00535019976200858\\
40.01	0.00535019739971781\\
41.01	0.00535019498748146\\
42.01	0.00535019252424177\\
43.01	0.00535019000891849\\
44.01	0.00535018744040847\\
45.01	0.00535018481758538\\
46.01	0.0053501821392988\\
47.01	0.00535017940437382\\
48.01	0.00535017661161067\\
49.01	0.0053501737597842\\
50.01	0.00535017084764323\\
51.01	0.00535016787390984\\
52.01	0.00535016483727916\\
53.01	0.00535016173641865\\
54.01	0.00535015856996733\\
55.01	0.00535015533653546\\
56.01	0.00535015203470388\\
57.01	0.00535014866302299\\
58.01	0.00535014522001257\\
59.01	0.00535014170416103\\
60.01	0.0053501381139247\\
61.01	0.00535013444772711\\
62.01	0.00535013070395823\\
63.01	0.00535012688097388\\
64.01	0.00535012297709494\\
65.01	0.00535011899060661\\
66.01	0.00535011491975781\\
67.01	0.00535011076276013\\
68.01	0.00535010651778722\\
69.01	0.00535010218297388\\
70.01	0.00535009775641527\\
71.01	0.00535009323616632\\
72.01	0.00535008862024022\\
73.01	0.00535008390660834\\
74.01	0.0053500790931987\\
75.01	0.00535007417789551\\
76.01	0.00535006915853775\\
77.01	0.00535006403291869\\
78.01	0.00535005879878457\\
79.01	0.00535005345383386\\
80.01	0.00535004799571588\\
81.01	0.00535004242203036\\
82.01	0.00535003673032574\\
83.01	0.00535003091809843\\
84.01	0.00535002498279168\\
85.01	0.00535001892179418\\
86.01	0.0053500127324394\\
87.01	0.00535000641200381\\
88.01	0.00534999995770629\\
89.01	0.00534999336670627\\
90.01	0.00534998663610307\\
91.01	0.0053499797629341\\
92.01	0.0053499727441738\\
93.01	0.00534996557673259\\
94.01	0.00534995825745473\\
95.01	0.00534995078311778\\
96.01	0.00534994315043042\\
97.01	0.00534993535603153\\
98.01	0.00534992739648842\\
99.01	0.00534991926829532\\
100.01	0.00534991096787215\\
101.01	0.00534990249156239\\
102.01	0.00534989383563186\\
103.01	0.00534988499626685\\
104.01	0.00534987596957271\\
105.01	0.00534986675157164\\
106.01	0.00534985733820153\\
107.01	0.00534984772531359\\
108.01	0.00534983790867085\\
109.01	0.00534982788394607\\
110.01	0.00534981764672022\\
111.01	0.0053498071924799\\
112.01	0.00534979651661583\\
113.01	0.00534978561442046\\
114.01	0.00534977448108632\\
115.01	0.00534976311170351\\
116.01	0.00534975150125743\\
117.01	0.00534973964462699\\
118.01	0.00534972753658188\\
119.01	0.00534971517178064\\
120.01	0.00534970254476804\\
121.01	0.00534968964997274\\
122.01	0.00534967648170465\\
123.01	0.00534966303415268\\
124.01	0.00534964930138204\\
125.01	0.00534963527733151\\
126.01	0.00534962095581067\\
127.01	0.00534960633049756\\
128.01	0.00534959139493537\\
129.01	0.00534957614252992\\
130.01	0.00534956056654668\\
131.01	0.00534954466010754\\
132.01	0.00534952841618806\\
133.01	0.00534951182761399\\
134.01	0.00534949488705868\\
135.01	0.00534947758703889\\
136.01	0.00534945991991232\\
137.01	0.00534944187787386\\
138.01	0.00534942345295226\\
139.01	0.00534940463700621\\
140.01	0.00534938542172121\\
141.01	0.00534936579860556\\
142.01	0.00534934575898676\\
143.01	0.00534932529400738\\
144.01	0.00534930439462162\\
145.01	0.00534928305159063\\
146.01	0.00534926125547865\\
147.01	0.0053492389966496\\
148.01	0.00534921626526133\\
149.01	0.00534919305126237\\
150.01	0.00534916934438704\\
151.01	0.00534914513415085\\
152.01	0.00534912040984623\\
153.01	0.00534909516053715\\
154.01	0.00534906937505456\\
155.01	0.00534904304199172\\
156.01	0.00534901614969872\\
157.01	0.00534898868627732\\
158.01	0.0053489606395759\\
159.01	0.00534893199718371\\
160.01	0.00534890274642584\\
161.01	0.00534887287435704\\
162.01	0.00534884236775645\\
163.01	0.00534881121312145\\
164.01	0.0053487793966615\\
165.01	0.00534874690429258\\
166.01	0.0053487137216302\\
167.01	0.00534867983398361\\
168.01	0.0053486452263491\\
169.01	0.00534860988340292\\
170.01	0.00534857378949483\\
171.01	0.00534853692864163\\
172.01	0.00534849928451919\\
173.01	0.00534846084045555\\
174.01	0.0053484215794237\\
175.01	0.00534838148403353\\
176.01	0.00534834053652487\\
177.01	0.00534829871875869\\
178.01	0.00534825601220972\\
179.01	0.00534821239795825\\
180.01	0.00534816785668118\\
181.01	0.00534812236864375\\
182.01	0.00534807591369094\\
183.01	0.00534802847123837\\
184.01	0.00534798002026301\\
185.01	0.00534793053929424\\
186.01	0.00534788000640404\\
187.01	0.00534782839919714\\
188.01	0.00534777569480136\\
189.01	0.00534772186985746\\
190.01	0.00534766690050877\\
191.01	0.00534761076239052\\
192.01	0.00534755343061893\\
193.01	0.00534749487978045\\
194.01	0.0053474350839205\\
195.01	0.0053473740165318\\
196.01	0.00534731165054265\\
197.01	0.00534724795830508\\
198.01	0.00534718291158238\\
199.01	0.00534711648153705\\
200.01	0.00534704863871773\\
201.01	0.00534697935304602\\
202.01	0.00534690859380354\\
203.01	0.00534683632961815\\
204.01	0.00534676252845023\\
205.01	0.00534668715757838\\
206.01	0.00534661018358502\\
207.01	0.00534653157234164\\
208.01	0.00534645128899402\\
209.01	0.0053463692979464\\
210.01	0.00534628556284588\\
211.01	0.00534620004656689\\
212.01	0.00534611271119413\\
213.01	0.00534602351800639\\
214.01	0.00534593242745903\\
215.01	0.00534583939916739\\
216.01	0.00534574439188814\\
217.01	0.00534564736350164\\
218.01	0.00534554827099344\\
219.01	0.00534544707043504\\
220.01	0.00534534371696496\\
221.01	0.00534523816476892\\
222.01	0.0053451303670598\\
223.01	0.00534502027605716\\
224.01	0.0053449078429661\\
225.01	0.00534479301795617\\
226.01	0.00534467575013941\\
227.01	0.00534455598754815\\
228.01	0.0053444336771125\\
229.01	0.00534430876463645\\
230.01	0.00534418119477529\\
231.01	0.00534405091101063\\
232.01	0.00534391785562617\\
233.01	0.00534378196968224\\
234.01	0.00534364319299068\\
235.01	0.00534350146408818\\
236.01	0.00534335672020978\\
237.01	0.00534320889726148\\
238.01	0.00534305792979283\\
239.01	0.00534290375096808\\
240.01	0.00534274629253751\\
241.01	0.00534258548480774\\
242.01	0.00534242125661182\\
243.01	0.00534225353527838\\
244.01	0.00534208224660047\\
245.01	0.00534190731480294\\
246.01	0.00534172866251073\\
247.01	0.00534154621071516\\
248.01	0.00534135987873995\\
249.01	0.00534116958420659\\
250.01	0.00534097524299939\\
251.01	0.00534077676922931\\
252.01	0.00534057407519673\\
253.01	0.00534036707135504\\
254.01	0.00534015566627153\\
255.01	0.0053399397665895\\
256.01	0.00533971927698778\\
257.01	0.00533949410014039\\
258.01	0.00533926413667565\\
259.01	0.00533902928513409\\
260.01	0.00533878944192549\\
261.01	0.00533854450128538\\
262.01	0.00533829435523067\\
263.01	0.00533803889351393\\
264.01	0.0053377780035775\\
265.01	0.00533751157050692\\
266.01	0.00533723947698208\\
267.01	0.00533696160322902\\
268.01	0.00533667782697017\\
269.01	0.0053363880233736\\
270.01	0.00533609206500142\\
271.01	0.00533578982175755\\
272.01	0.00533548116083401\\
273.01	0.00533516594665643\\
274.01	0.00533484404082944\\
275.01	0.00533451530207933\\
276.01	0.00533417958619711\\
277.01	0.00533383674598056\\
278.01	0.00533348663117434\\
279.01	0.00533312908841\\
280.01	0.00533276396114464\\
281.01	0.00533239108959815\\
282.01	0.00533201031069036\\
283.01	0.00533162145797615\\
284.01	0.00533122436158038\\
285.01	0.00533081884813113\\
286.01	0.00533040474069229\\
287.01	0.00532998185869494\\
288.01	0.00532955001786797\\
289.01	0.00532910903016721\\
290.01	0.0053286587037039\\
291.01	0.00532819884267224\\
292.01	0.00532772924727555\\
293.01	0.00532724971365193\\
294.01	0.0053267600337985\\
295.01	0.00532625999549482\\
296.01	0.00532574938222581\\
297.01	0.00532522797310312\\
298.01	0.00532469554278615\\
299.01	0.00532415186140171\\
300.01	0.00532359669446365\\
301.01	0.00532302980279103\\
302.01	0.00532245094242565\\
303.01	0.00532185986454963\\
304.01	0.00532125631540115\\
305.01	0.00532064003619124\\
306.01	0.00532001076301838\\
307.01	0.00531936822678417\\
308.01	0.005318712153108\\
309.01	0.00531804226224141\\
310.01	0.00531735826898274\\
311.01	0.00531665988259147\\
312.01	0.00531594680670288\\
313.01	0.00531521873924296\\
314.01	0.00531447537234359\\
315.01	0.0053137163922588\\
316.01	0.00531294147928049\\
317.01	0.00531215030765688\\
318.01	0.00531134254551053\\
319.01	0.00531051785475856\\
320.01	0.00530967589103442\\
321.01	0.00530881630361108\\
322.01	0.00530793873532671\\
323.01	0.00530704282251277\\
324.01	0.00530612819492473\\
325.01	0.00530519447567572\\
326.01	0.00530424128117427\\
327.01	0.00530326822106537\\
328.01	0.00530227489817605\\
329.01	0.00530126090846623\\
330.01	0.00530022584098391\\
331.01	0.00529916927782694\\
332.01	0.00529809079411157\\
333.01	0.00529698995794719\\
334.01	0.00529586633041939\\
335.01	0.00529471946558195\\
336.01	0.00529354891045676\\
337.01	0.00529235420504582\\
338.01	0.00529113488235237\\
339.01	0.00528989046841502\\
340.01	0.00528862048235526\\
341.01	0.00528732443643789\\
342.01	0.00528600183614752\\
343.01	0.00528465218028151\\
344.01	0.00528327496106036\\
345.01	0.0052818696642579\\
346.01	0.00528043576935145\\
347.01	0.00527897274969461\\
348.01	0.0052774800727138\\
349.01	0.00527595720013088\\
350.01	0.0052744035882121\\
351.01	0.00527281868804799\\
352.01	0.00527120194586318\\
353.01	0.00526955280336065\\
354.01	0.00526787069810039\\
355.01	0.00526615506391757\\
356.01	0.00526440533137949\\
357.01	0.0052626209282854\\
358.01	0.00526080128021128\\
359.01	0.00525894581110196\\
360.01	0.00525705394391194\\
361.01	0.00525512510129937\\
362.01	0.00525315870637325\\
363.01	0.0052511541834974\\
364.01	0.00524911095915187\\
365.01	0.00524702846285501\\
366.01	0.00524490612814607\\
367.01	0.00524274339362976\\
368.01	0.00524053970408339\\
369.01	0.00523829451162582\\
370.01	0.00523600727694703\\
371.01	0.00523367747059659\\
372.01	0.00523130457432675\\
373.01	0.00522888808248597\\
374.01	0.00522642750345643\\
375.01	0.0052239223611262\\
376.01	0.00522137219638459\\
377.01	0.00521877656862848\\
378.01	0.00521613505726135\\
379.01	0.00521344726316258\\
380.01	0.00521071281010516\\
381.01	0.0052079313460878\\
382.01	0.00520510254454846\\
383.01	0.00520222610541506\\
384.01	0.00519930175594483\\
385.01	0.00519632925129438\\
386.01	0.00519330837475483\\
387.01	0.00519023893757661\\
388.01	0.00518712077829755\\
389.01	0.00518395376147891\\
390.01	0.00518073777574154\\
391.01	0.00517747273098405\\
392.01	0.00517415855465568\\
393.01	0.00517079518694496\\
394.01	0.00516738257474272\\
395.01	0.00516392066422929\\
396.01	0.00516040939194155\\
397.01	0.00515684867418499\\
398.01	0.00515323839467018\\
399.01	0.00514957839029082\\
400.01	0.00514586843500483\\
401.01	0.0051421082218543\\
402.01	0.00513829734325826\\
403.01	0.0051344352698442\\
404.01	0.00513052132825987\\
405.01	0.00512655467862695\\
406.01	0.00512253429258049\\
407.01	0.00511845893316631\\
408.01	0.00511432713827346\\
409.01	0.00511013720971888\\
410.01	0.00510588721058433\\
411.01	0.00510157497386776\\
412.01	0.00509719812589269\\
413.01	0.00509275412807227\\
414.01	0.00508824034035754\\
415.01	0.00508365410866688\\
416.01	0.00507899287629741\\
417.01	0.00507425431498538\\
418.01	0.00506943646375203\\
419.01	0.00506453785125295\\
420.01	0.00505955753787001\\
421.01	0.00505449487564457\\
422.01	0.0050493492743097\\
423.01	0.00504412018361756\\
424.01	0.00503880709561506\\
425.01	0.00503340954685702\\
426.01	0.00502792712052649\\
427.01	0.00502235944842922\\
428.01	0.00501670621282421\\
429.01	0.00501096714804755\\
430.01	0.00500514204188019\\
431.01	0.00499923073660854\\
432.01	0.00499323312971387\\
433.01	0.00498714917412894\\
434.01	0.00498097887798744\\
435.01	0.00497472230378842\\
436.01	0.00496837956689141\\
437.01	0.00496195083325152\\
438.01	0.00495543631629968\\
439.01	0.0049488362728684\\
440.01	0.00494215099806004\\
441.01	0.00493538081895403\\
442.01	0.00492852608705271\\
443.01	0.0049215871693661\\
444.01	0.00491456443804755\\
445.01	0.0049074582585057\\
446.01	0.00490026897593585\\
447.01	0.00489299690024187\\
448.01	0.00488564228935582\\
449.01	0.00487820533100495\\
450.01	0.00487068612303634\\
451.01	0.00486308465247448\\
452.01	0.00485540077357311\\
453.01	0.00484763418522\\
454.01	0.00483978440816255\\
455.01	0.00483185076265124\\
456.01	0.00482383234723436\\
457.01	0.00481572801957892\\
458.01	0.00480753638033996\\
459.01	0.00479925576123245\\
460.01	0.00479088421856735\\
461.01	0.00478241953357406\\
462.01	0.00477385922082033\\
463.01	0.0047652005459163\\
464.01	0.00475644055341732\\
465.01	0.00474757610536369\\
466.01	0.00473860393016235\\
467.01	0.00472952068047446\\
468.01	0.00472032299737828\\
469.01	0.0047110075763275\\
470.01	0.00470157122838082\\
471.01	0.00469201092801985\\
472.01	0.00468232383700174\\
473.01	0.0046725072928645\\
474.01	0.00466255875221294\\
475.01	0.00465247568499771\\
476.01	0.00464225543447259\\
477.01	0.00463189511510558\\
478.01	0.0046213915849398\\
479.01	0.00461074143567458\\
480.01	0.00459994098549038\\
481.01	0.00458898627489505\\
482.01	0.00457787306603472\\
483.01	0.00456659684584823\\
484.01	0.00455515283335924\\
485.01	0.00454353599126477\\
486.01	0.00453174104179636\\
487.01	0.00451976248659717\\
488.01	0.0045075946300765\\
489.01	0.00449523160536918\\
490.01	0.00448266740167535\\
491.01	0.00446989589138234\\
492.01	0.00445691085505024\\
493.01	0.00444370600210405\\
494.01	0.00443027498501948\\
495.01	0.00441661140500323\\
496.01	0.00440270880777203\\
497.01	0.00438856066912667\\
498.01	0.00437416037167617\\
499.01	0.00435950117623876\\
500.01	0.0043445761938544\\
501.01	0.00432937836627371\\
502.01	0.00431390046205232\\
503.01	0.00429813508615042\\
504.01	0.00428207469322905\\
505.01	0.00426571160085169\\
506.01	0.00424903800183912\\
507.01	0.0042320459751404\\
508.01	0.00421472749460594\\
509.01	0.00419707443513539\\
510.01	0.00417907857582396\\
511.01	0.0041607315999603\\
512.01	0.0041420250920226\\
513.01	0.00412295053216629\\
514.01	0.00410349928904926\\
515.01	0.00408366261213326\\
516.01	0.00406343162474818\\
517.01	0.00404279731908349\\
518.01	0.00402175055378084\\
519.01	0.00400028205391089\\
520.01	0.00397838241203012\\
521.01	0.00395604208873859\\
522.01	0.00393325141211583\\
523.01	0.00391000057605093\\
524.01	0.00388627963762727\\
525.01	0.00386207851380362\\
526.01	0.00383738697769382\\
527.01	0.00381219465478078\\
528.01	0.00378649101939319\\
529.01	0.00376026539170914\\
530.01	0.00373350693544559\\
531.01	0.00370620465624647\\
532.01	0.00367834740064358\\
533.01	0.00364992385538304\\
534.01	0.00362092254694498\\
535.01	0.00359133184124472\\
536.01	0.00356113994363432\\
537.01	0.00353033489935747\\
538.01	0.00349890459460845\\
539.01	0.00346683675832488\\
540.01	0.00343411896482426\\
541.01	0.00340073863736593\\
542.01	0.00336668305270029\\
543.01	0.0033319393466588\\
544.01	0.00329649452084877\\
545.01	0.00326033545053969\\
546.01	0.00322344889386798\\
547.01	0.00318582150251138\\
548.01	0.00314743983399592\\
549.01	0.00310829036580649\\
550.01	0.00306835951147481\\
551.01	0.0030276336388282\\
552.01	0.00298609909059895\\
553.01	0.00294374220760762\\
554.01	0.00290054935475943\\
555.01	0.0028565069501179\\
556.01	0.00281160149734305\\
557.01	0.0027658196218078\\
558.01	0.0027191481107284\\
559.01	0.00267157395766957\\
560.01	0.00262308441180904\\
561.01	0.00257366703237543\\
562.01	0.00252330974869932\\
563.01	0.00247200092634584\\
564.01	0.00241972943982337\\
565.01	0.00236648475238619\\
566.01	0.00231225700346725\\
567.01	0.00225703710429318\\
568.01	0.00220081684223704\\
569.01	0.00214358899446181\\
570.01	0.00208534745138918\\
571.01	0.00202608735049004\\
572.01	0.00196580522083402\\
573.01	0.00190449913874023\\
574.01	0.0018421688947402\\
575.01	0.00177881617187809\\
576.01	0.00171444473512429\\
577.01	0.00164906063134602\\
578.01	0.00158267239884402\\
579.01	0.00151529128489964\\
580.01	0.00144693146904958\\
581.01	0.00137761028887722\\
582.01	0.00130734846393035\\
583.01	0.00123617031188133\\
584.01	0.00116410394916785\\
585.01	0.00109118146599105\\
586.01	0.00101743906259415\\
587.01	0.000942917130054623\\
588.01	0.00086766025422433\\
589.01	0.000791717115732183\\
590.01	0.000715140251857945\\
591.01	0.000637985637273675\\
592.01	0.00056031202973124\\
593.01	0.000482180013262905\\
594.01	0.000403650654755804\\
595.01	0.00032478366912282\\
596.01	0.000245634962812777\\
597.01	0.000166444281554918\\
598.01	9.17366970063782e-05\\
599.01	2.94669364271135e-05\\
599.02	2.89574269585688e-05\\
599.03	2.84509530852819e-05\\
599.04	2.79475443462525e-05\\
599.05	2.7447230571279e-05\\
599.06	2.69500418838293e-05\\
599.07	2.64560087039432e-05\\
599.08	2.59651617511691e-05\\
599.09	2.54775320475235e-05\\
599.1	2.49931509204854e-05\\
599.11	2.45120500060175e-05\\
599.12	2.40342612516167e-05\\
599.13	2.35598169193996e-05\\
599.14	2.30887495892042e-05\\
599.15	2.26210921617422e-05\\
599.16	2.215687786177e-05\\
599.17	2.16961402412976e-05\\
599.18	2.12389131828191e-05\\
599.19	2.07852309025824e-05\\
599.2	2.03351279538938e-05\\
599.21	1.98886392304542e-05\\
599.22	1.94457999697188e-05\\
599.23	1.90066457563063e-05\\
599.24	1.85712125254194e-05\\
599.25	1.81395365663334e-05\\
599.26	1.77116564095883e-05\\
599.27	1.72876122135658e-05\\
599.28	1.68674445361946e-05\\
599.29	1.64511943388859e-05\\
599.3	1.60389029905273e-05\\
599.31	1.56306122715104e-05\\
599.32	1.5226364377792e-05\\
599.33	1.48262019250105e-05\\
599.34	1.44301679526285e-05\\
599.35	1.40383059281154e-05\\
599.36	1.36506597511916e-05\\
599.37	1.32672737580847e-05\\
599.38	1.28881927258552e-05\\
599.39	1.25134618767404e-05\\
599.4	1.21431268825696e-05\\
599.41	1.17772338691924e-05\\
599.42	1.14158294209736e-05\\
599.43	1.10589605853174e-05\\
599.44	1.07066748772523e-05\\
599.45	1.03590202840433e-05\\
599.46	1.00160452698606e-05\\
599.47	9.67779878049101e-06\\
599.48	9.34433024810284e-06\\
599.49	9.01568959604283e-06\\
599.5	8.69192724369319e-06\\
599.51	8.37309411137396e-06\\
599.52	8.05924162529219e-06\\
599.53	7.75042172253965e-06\\
599.54	7.4466868561357e-06\\
599.55	7.14809000013084e-06\\
599.56	6.85468465475535e-06\\
599.57	6.56652485161308e-06\\
599.58	6.28366515892896e-06\\
599.59	6.00616068686249e-06\\
599.6	5.7340670928472e-06\\
599.61	5.46744058700296e-06\\
599.62	5.20633793760217e-06\\
599.63	4.95081647657741e-06\\
599.64	4.70093410508653e-06\\
599.65	4.45674929914347e-06\\
599.66	4.21832111529262e-06\\
599.67	3.98570919634203e-06\\
599.68	3.75897377716435e-06\\
599.69	3.53817569053415e-06\\
599.7	3.32337637303295e-06\\
599.71	3.11463787103054e-06\\
599.72	2.91202284668363e-06\\
599.73	2.71559458404382e-06\\
599.74	2.52541699518292e-06\\
599.75	2.34155462640329e-06\\
599.76	2.16407266449489e-06\\
599.77	1.99303694307928e-06\\
599.78	1.82851394897598e-06\\
599.79	1.67057082867302e-06\\
599.8	1.51927539483211e-06\\
599.81	1.37469613287027e-06\\
599.82	1.23690220760718e-06\\
599.83	1.10596346997172e-06\\
599.84	9.81950463782924e-07\\
599.85	8.64934432591793e-07\\
599.86	7.54987326588921e-07\\
599.87	6.5218180958504e-07\\
599.88	5.56591266064402e-07\\
599.89	4.68289808295413e-07\\
599.9	3.87352283521061e-07\\
599.91	3.13854281218731e-07\\
599.92	2.4787214042421e-07\\
599.93	1.8948295714763e-07\\
599.94	1.38764591834512e-07\\
599.95	9.57956769204876e-08\\
599.96	6.06556244606149e-08\\
599.97	3.34246338194039e-08\\
599.98	1.41836994527883e-08\\
599.99	3.01461875948372e-09\\
600	0\\
};
\addplot [color=black!20!mycolor21,solid,forget plot]
  table[row sep=crcr]{%
0.01	0.00523138383269897\\
1.01	0.005231382805886\\
2.01	0.00523138175751092\\
3.01	0.0052313806871207\\
4.01	0.00523137959425308\\
5.01	0.00523137847843602\\
6.01	0.0052313773391874\\
7.01	0.0052313761760151\\
8.01	0.00523137498841659\\
9.01	0.00523137377587905\\
10.01	0.00523137253787854\\
11.01	0.00523137127388031\\
12.01	0.00523136998333838\\
13.01	0.00523136866569534\\
14.01	0.00523136732038186\\
15.01	0.00523136594681675\\
16.01	0.00523136454440681\\
17.01	0.00523136311254593\\
18.01	0.00523136165061591\\
19.01	0.00523136015798483\\
20.01	0.00523135863400801\\
21.01	0.0052313570780272\\
22.01	0.00523135548937016\\
23.01	0.00523135386735057\\
24.01	0.00523135221126765\\
25.01	0.00523135052040608\\
26.01	0.00523134879403519\\
27.01	0.00523134703140926\\
28.01	0.00523134523176695\\
29.01	0.00523134339433056\\
30.01	0.00523134151830637\\
31.01	0.00523133960288403\\
32.01	0.00523133764723592\\
33.01	0.00523133565051707\\
34.01	0.00523133361186495\\
35.01	0.00523133153039868\\
36.01	0.00523132940521904\\
37.01	0.00523132723540796\\
38.01	0.00523132502002802\\
39.01	0.005231322758122\\
40.01	0.00523132044871281\\
41.01	0.0052313180908025\\
42.01	0.00523131568337245\\
43.01	0.0052313132253825\\
44.01	0.00523131071577083\\
45.01	0.00523130815345305\\
46.01	0.00523130553732208\\
47.01	0.0052313028662477\\
48.01	0.00523130013907582\\
49.01	0.00523129735462829\\
50.01	0.00523129451170188\\
51.01	0.0052312916090685\\
52.01	0.00523128864547392\\
53.01	0.00523128561963773\\
54.01	0.00523128253025265\\
55.01	0.00523127937598381\\
56.01	0.00523127615546846\\
57.01	0.00523127286731504\\
58.01	0.0052312695101031\\
59.01	0.00523126608238191\\
60.01	0.00523126258267048\\
61.01	0.00523125900945685\\
62.01	0.00523125536119704\\
63.01	0.00523125163631499\\
64.01	0.00523124783320103\\
65.01	0.00523124395021218\\
66.01	0.00523123998567065\\
67.01	0.00523123593786356\\
68.01	0.00523123180504193\\
69.01	0.00523122758542023\\
70.01	0.0052312232771752\\
71.01	0.00523121887844556\\
72.01	0.00523121438733066\\
73.01	0.00523120980189001\\
74.01	0.00523120512014266\\
75.01	0.00523120034006559\\
76.01	0.00523119545959362\\
77.01	0.00523119047661807\\
78.01	0.00523118538898599\\
79.01	0.00523118019449925\\
80.01	0.00523117489091362\\
81.01	0.0052311694759376\\
82.01	0.00523116394723152\\
83.01	0.0052311583024069\\
84.01	0.00523115253902459\\
85.01	0.00523114665459489\\
86.01	0.00523114064657517\\
87.01	0.00523113451236992\\
88.01	0.00523112824932879\\
89.01	0.00523112185474595\\
90.01	0.00523111532585859\\
91.01	0.00523110865984609\\
92.01	0.00523110185382852\\
93.01	0.00523109490486543\\
94.01	0.00523108780995487\\
95.01	0.00523108056603126\\
96.01	0.00523107316996528\\
97.01	0.00523106561856186\\
98.01	0.0052310579085586\\
99.01	0.00523105003662475\\
100.01	0.00523104199935942\\
101.01	0.00523103379329062\\
102.01	0.00523102541487337\\
103.01	0.00523101686048825\\
104.01	0.00523100812643983\\
105.01	0.00523099920895525\\
106.01	0.00523099010418216\\
107.01	0.00523098080818766\\
108.01	0.00523097131695629\\
109.01	0.00523096162638815\\
110.01	0.00523095173229713\\
111.01	0.00523094163040974\\
112.01	0.00523093131636236\\
113.01	0.00523092078570006\\
114.01	0.0052309100338744\\
115.01	0.00523089905624127\\
116.01	0.00523088784805948\\
117.01	0.00523087640448797\\
118.01	0.00523086472058439\\
119.01	0.00523085279130236\\
120.01	0.00523084061149009\\
121.01	0.00523082817588724\\
122.01	0.00523081547912343\\
123.01	0.00523080251571542\\
124.01	0.00523078928006506\\
125.01	0.00523077576645655\\
126.01	0.00523076196905461\\
127.01	0.00523074788190097\\
128.01	0.00523073349891317\\
129.01	0.00523071881388052\\
130.01	0.00523070382046229\\
131.01	0.00523068851218465\\
132.01	0.00523067288243826\\
133.01	0.00523065692447512\\
134.01	0.0052306406314054\\
135.01	0.00523062399619537\\
136.01	0.00523060701166355\\
137.01	0.00523058967047768\\
138.01	0.00523057196515207\\
139.01	0.00523055388804402\\
140.01	0.00523053543135064\\
141.01	0.0052305165871056\\
142.01	0.00523049734717545\\
143.01	0.00523047770325624\\
144.01	0.00523045764687\\
145.01	0.00523043716936136\\
146.01	0.00523041626189342\\
147.01	0.00523039491544394\\
148.01	0.00523037312080204\\
149.01	0.00523035086856357\\
150.01	0.00523032814912757\\
151.01	0.00523030495269195\\
152.01	0.00523028126924912\\
153.01	0.00523025708858244\\
154.01	0.00523023240026072\\
155.01	0.00523020719363465\\
156.01	0.00523018145783191\\
157.01	0.00523015518175259\\
158.01	0.00523012835406437\\
159.01	0.00523010096319786\\
160.01	0.00523007299734137\\
161.01	0.00523004444443611\\
162.01	0.00523001529217109\\
163.01	0.00522998552797743\\
164.01	0.00522995513902352\\
165.01	0.00522992411220928\\
166.01	0.00522989243416079\\
167.01	0.00522986009122422\\
168.01	0.00522982706946012\\
169.01	0.00522979335463813\\
170.01	0.00522975893222984\\
171.01	0.00522972378740324\\
172.01	0.00522968790501672\\
173.01	0.00522965126961176\\
174.01	0.00522961386540708\\
175.01	0.00522957567629173\\
176.01	0.00522953668581796\\
177.01	0.00522949687719485\\
178.01	0.00522945623328061\\
179.01	0.0052294147365757\\
180.01	0.00522937236921498\\
181.01	0.00522932911296079\\
182.01	0.00522928494919445\\
183.01	0.00522923985890923\\
184.01	0.00522919382270152\\
185.01	0.00522914682076305\\
186.01	0.00522909883287243\\
187.01	0.00522904983838685\\
188.01	0.00522899981623294\\
189.01	0.00522894874489822\\
190.01	0.00522889660242194\\
191.01	0.00522884336638555\\
192.01	0.0052287890139039\\
193.01	0.00522873352161505\\
194.01	0.00522867686567052\\
195.01	0.00522861902172563\\
196.01	0.00522855996492882\\
197.01	0.00522849966991163\\
198.01	0.00522843811077804\\
199.01	0.00522837526109304\\
200.01	0.00522831109387224\\
201.01	0.00522824558157061\\
202.01	0.00522817869607035\\
203.01	0.0052281104086697\\
204.01	0.00522804069007087\\
205.01	0.00522796951036762\\
206.01	0.00522789683903318\\
207.01	0.00522782264490732\\
208.01	0.00522774689618307\\
209.01	0.00522766956039422\\
210.01	0.00522759060440158\\
211.01	0.00522750999437894\\
212.01	0.00522742769579952\\
213.01	0.00522734367342148\\
214.01	0.00522725789127358\\
215.01	0.00522717031263984\\
216.01	0.00522708090004511\\
217.01	0.00522698961523909\\
218.01	0.00522689641918104\\
219.01	0.00522680127202353\\
220.01	0.0052267041330961\\
221.01	0.005226604960889\\
222.01	0.00522650371303547\\
223.01	0.00522640034629519\\
224.01	0.00522629481653646\\
225.01	0.00522618707871809\\
226.01	0.00522607708687117\\
227.01	0.00522596479408068\\
228.01	0.00522585015246619\\
229.01	0.00522573311316276\\
230.01	0.00522561362630088\\
231.01	0.00522549164098697\\
232.01	0.00522536710528238\\
233.01	0.00522523996618295\\
234.01	0.0052251101695975\\
235.01	0.00522497766032645\\
236.01	0.00522484238203957\\
237.01	0.0052247042772541\\
238.01	0.00522456328731119\\
239.01	0.00522441935235325\\
240.01	0.00522427241130017\\
241.01	0.00522412240182509\\
242.01	0.00522396926033021\\
243.01	0.0052238129219215\\
244.01	0.00522365332038351\\
245.01	0.00522349038815377\\
246.01	0.00522332405629596\\
247.01	0.00522315425447371\\
248.01	0.00522298091092315\\
249.01	0.0052228039524253\\
250.01	0.00522262330427802\\
251.01	0.00522243889026721\\
252.01	0.00522225063263816\\
253.01	0.00522205845206537\\
254.01	0.00522186226762313\\
255.01	0.00522166199675438\\
256.01	0.00522145755524\\
257.01	0.00522124885716726\\
258.01	0.00522103581489799\\
259.01	0.00522081833903531\\
260.01	0.00522059633839147\\
261.01	0.00522036971995348\\
262.01	0.00522013838884953\\
263.01	0.00521990224831446\\
264.01	0.00521966119965464\\
265.01	0.00521941514221146\\
266.01	0.00521916397332687\\
267.01	0.00521890758830544\\
268.01	0.0052186458803775\\
269.01	0.00521837874066163\\
270.01	0.00521810605812694\\
271.01	0.00521782771955339\\
272.01	0.00521754360949365\\
273.01	0.00521725361023293\\
274.01	0.00521695760174888\\
275.01	0.00521665546167114\\
276.01	0.00521634706524061\\
277.01	0.00521603228526704\\
278.01	0.00521571099208779\\
279.01	0.00521538305352528\\
280.01	0.00521504833484423\\
281.01	0.00521470669870827\\
282.01	0.00521435800513644\\
283.01	0.00521400211145921\\
284.01	0.00521363887227421\\
285.01	0.00521326813940134\\
286.01	0.00521288976183818\\
287.01	0.00521250358571453\\
288.01	0.00521210945424713\\
289.01	0.0052117072076936\\
290.01	0.00521129668330694\\
291.01	0.00521087771528946\\
292.01	0.0052104501347468\\
293.01	0.00521001376964111\\
294.01	0.00520956844474569\\
295.01	0.0052091139815982\\
296.01	0.00520865019845492\\
297.01	0.00520817691024477\\
298.01	0.00520769392852357\\
299.01	0.0052072010614285\\
300.01	0.00520669811363309\\
301.01	0.00520618488630217\\
302.01	0.00520566117704823\\
303.01	0.00520512677988722\\
304.01	0.00520458148519613\\
305.01	0.0052040250796701\\
306.01	0.0052034573462818\\
307.01	0.00520287806424025\\
308.01	0.00520228700895233\\
309.01	0.00520168395198414\\
310.01	0.00520106866102452\\
311.01	0.00520044089984959\\
312.01	0.00519980042828956\\
313.01	0.00519914700219645\\
314.01	0.00519848037341427\\
315.01	0.00519780028975126\\
316.01	0.00519710649495468\\
317.01	0.00519639872868714\\
318.01	0.0051956767265069\\
319.01	0.00519494021985052\\
320.01	0.00519418893601805\\
321.01	0.00519342259816257\\
322.01	0.00519264092528269\\
323.01	0.00519184363221911\\
324.01	0.00519103042965536\\
325.01	0.00519020102412334\\
326.01	0.00518935511801243\\
327.01	0.00518849240958488\\
328.01	0.00518761259299581\\
329.01	0.00518671535831935\\
330.01	0.00518580039158017\\
331.01	0.00518486737479209\\
332.01	0.00518391598600277\\
333.01	0.00518294589934636\\
334.01	0.0051819567851026\\
335.01	0.00518094830976426\\
336.01	0.00517992013611334\\
337.01	0.00517887192330526\\
338.01	0.00517780332696261\\
339.01	0.00517671399927806\\
340.01	0.00517560358912744\\
341.01	0.00517447174219251\\
342.01	0.00517331810109557\\
343.01	0.00517214230554414\\
344.01	0.00517094399248761\\
345.01	0.00516972279628462\\
346.01	0.005168478348884\\
347.01	0.00516721028001698\\
348.01	0.00516591821740183\\
349.01	0.00516460178696177\\
350.01	0.0051632606130549\\
351.01	0.00516189431871695\\
352.01	0.00516050252591668\\
353.01	0.00515908485582376\\
354.01	0.00515764092908807\\
355.01	0.00515617036613068\\
356.01	0.00515467278744542\\
357.01	0.0051531478139106\\
358.01	0.00515159506710816\\
359.01	0.00515001416965157\\
360.01	0.00514840474551805\\
361.01	0.00514676642038459\\
362.01	0.00514509882196492\\
363.01	0.00514340158034452\\
364.01	0.00514167432831077\\
365.01	0.00513991670167382\\
366.01	0.00513812833957383\\
367.01	0.00513630888477042\\
368.01	0.00513445798390796\\
369.01	0.00513257528774949\\
370.01	0.00513066045137389\\
371.01	0.00512871313432686\\
372.01	0.00512673300071721\\
373.01	0.00512471971924839\\
374.01	0.00512267296317452\\
375.01	0.00512059241016839\\
376.01	0.00511847774208999\\
377.01	0.00511632864463985\\
378.01	0.00511414480688257\\
379.01	0.00511192592062577\\
380.01	0.00510967167963675\\
381.01	0.00510738177867964\\
382.01	0.00510505591235568\\
383.01	0.0051026937737284\\
384.01	0.00510029505271672\\
385.01	0.00509785943423936\\
386.01	0.00509538659609526\\
387.01	0.00509287620656777\\
388.01	0.00509032792174346\\
389.01	0.00508774138254064\\
390.01	0.00508511621144959\\
391.01	0.00508245200899375\\
392.01	0.00507974834993066\\
393.01	0.00507700477922407\\
394.01	0.00507422080783227\\
395.01	0.00507139590837533\\
396.01	0.00506852951076469\\
397.01	0.00506562099789917\\
398.01	0.0050626697015592\\
399.01	0.00505967489865909\\
400.01	0.00505663580804531\\
401.01	0.00505355158806207\\
402.01	0.00505042133513194\\
403.01	0.00504724408362658\\
404.01	0.0050440188073223\\
405.01	0.0050407444227402\\
406.01	0.00503741979466024\\
407.01	0.00503404374406445\\
408.01	0.0050306150586897\\
409.01	0.00502713250625813\\
410.01	0.00502359485027725\\
411.01	0.00502000086806385\\
412.01	0.00501634937032668\\
413.01	0.00501263922124967\\
414.01	0.00500886935755939\\
415.01	0.00500503880457567\\
416.01	0.00500114668680975\\
417.01	0.00499719223044989\\
418.01	0.00499317475531407\\
419.01	0.004989093654993\\
420.01	0.00498494836696045\\
421.01	0.00498073834559584\\
422.01	0.00497646305411321\\
423.01	0.00497212196381056\\
424.01	0.00496771455341656\\
425.01	0.00496324030825039\\
426.01	0.00495869871917461\\
427.01	0.00495408928132304\\
428.01	0.00494941149258343\\
429.01	0.00494466485181496\\
430.01	0.00493984885678136\\
431.01	0.00493496300177866\\
432.01	0.00493000677494162\\
433.01	0.00492497965520942\\
434.01	0.00491988110893609\\
435.01	0.004914710586133\\
436.01	0.00490946751633421\\
437.01	0.00490415130407918\\
438.01	0.0048987613240141\\
439.01	0.00489329691561693\\
440.01	0.00488775737756222\\
441.01	0.00488214196174756\\
442.01	0.00487644986701617\\
443.01	0.00487068023262009\\
444.01	0.00486483213148382\\
445.01	0.00485890456333915\\
446.01	0.00485289644782183\\
447.01	0.0048468066176342\\
448.01	0.00484063381189665\\
449.01	0.00483437666982691\\
450.01	0.00482803372490231\\
451.01	0.00482160339967746\\
452.01	0.00481508400143666\\
453.01	0.00480847371887101\\
454.01	0.00480177061996946\\
455.01	0.00479497265130526\\
456.01	0.00478807763887849\\
457.01	0.00478108329064709\\
458.01	0.0047739872008241\\
459.01	0.00476678685595524\\
460.01	0.00475947964270048\\
461.01	0.00475206285713371\\
462.01	0.00474453371524032\\
463.01	0.00473688936414579\\
464.01	0.00472912689344517\\
465.01	0.00472124334584061\\
466.01	0.00471323572615556\\
467.01	0.0047051010076903\\
468.01	0.00469683613486702\\
469.01	0.00468843802121081\\
470.01	0.00467990354198637\\
471.01	0.00467122952129828\\
472.01	0.00466241271420965\\
473.01	0.00465344978541999\\
474.01	0.00464433728719487\\
475.01	0.00463507164027726\\
476.01	0.00462564912182952\\
477.01	0.00461606586177708\\
478.01	0.00460631784347362\\
479.01	0.00459640090568819\\
480.01	0.00458631074558279\\
481.01	0.00457604292263263\\
482.01	0.0045655928633934\\
483.01	0.00455495586698143\\
484.01	0.00454412711108051\\
485.01	0.00453310165824177\\
486.01	0.0045218744621983\\
487.01	0.00451044037388256\\
488.01	0.00449879414680735\\
489.01	0.00448693044147396\\
490.01	0.00447484382848845\\
491.01	0.00446252879012912\\
492.01	0.00444997972019801\\
493.01	0.00443719092212448\\
494.01	0.00442415660545863\\
495.01	0.00441087088108571\\
496.01	0.00439732775569151\\
497.01	0.00438352112617594\\
498.01	0.00436944477479478\\
499.01	0.00435509236575418\\
500.01	0.00434045744372663\\
501.01	0.00432553343426664\\
502.01	0.00431031364545572\\
503.01	0.00429479126971478\\
504.01	0.00427895938512484\\
505.01	0.00426281095607702\\
506.01	0.0042463388331861\\
507.01	0.00422953575244153\\
508.01	0.0042123943336166\\
509.01	0.00419490707800405\\
510.01	0.00417706636559825\\
511.01	0.00415886445188262\\
512.01	0.00414029346441456\\
513.01	0.00412134539940507\\
514.01	0.00410201211847112\\
515.01	0.0040822853456925\\
516.01	0.00406215666502128\\
517.01	0.00404161751800122\\
518.01	0.00402065920166672\\
519.01	0.00399927286644217\\
520.01	0.00397744951389662\\
521.01	0.00395517999431117\\
522.01	0.00393245500410627\\
523.01	0.00390926508320447\\
524.01	0.00388560061240984\\
525.01	0.00386145181088166\\
526.01	0.00383680873377442\\
527.01	0.0038116612700997\\
528.01	0.00378599914084991\\
529.01	0.00375981189740243\\
530.01	0.00373308892021277\\
531.01	0.00370581941779547\\
532.01	0.00367799242599846\\
533.01	0.00364959680759006\\
534.01	0.00362062125220449\\
535.01	0.00359105427670131\\
536.01	0.0035608842260056\\
537.01	0.00353009927449551\\
538.01	0.00349868742799898\\
539.01	0.00346663652646789\\
540.01	0.00343393424739245\\
541.01	0.00340056811002963\\
542.01	0.00336652548052344\\
543.01	0.00333179357800268\\
544.01	0.00329635948175882\\
545.01	0.00326021013961405\\
546.01	0.00322333237760449\\
547.01	0.00318571291111079\\
548.01	0.0031473383575855\\
549.01	0.00310819525103227\\
550.01	0.00306827005840969\\
551.01	0.00302754919814918\\
552.01	0.00298601906098984\\
553.01	0.00294366603335671\\
554.01	0.0029004765235269\\
555.01	0.00285643699084808\\
556.01	0.0028115339783004\\
557.01	0.00276575414871214\\
558.01	0.00271908432496719\\
559.01	0.00267151153456668\\
560.01	0.00262302305893529\\
561.01	0.00257360648788791\\
562.01	0.00252324977970384\\
563.01	0.00247194132727781\\
564.01	0.00241967003084647\\
565.01	0.00236642537780996\\
566.01	0.00231219753018783\\
567.01	0.0022569774202631\\
568.01	0.00220075685497212\\
569.01	0.00214352862959637\\
570.01	0.00208528665129069\\
571.01	0.00202602607294628\\
572.01	0.00196574343782424\\
573.01	0.00190443683530311\\
574.01	0.00184210606794752\\
575.01	0.00177875282992286\\
576.01	0.00171438089652806\\
577.01	0.00164899632428734\\
578.01	0.00158260766060568\\
579.01	0.00151522616142658\\
580.01	0.00144686601460507\\
581.01	0.00137754456577855\\
582.01	0.00130728254233795\\
583.01	0.00123610426960943\\
584.01	0.00116403787147582\\
585.01	0.00109111544530627\\
586.01	0.00101737319810725\\
587.01	0.00094285152711592\\
588.01	0.000867595023456734\\
589.01	0.000791652371757953\\
590.01	0.000715076111516234\\
591.01	0.000637922217178835\\
592.01	0.000560249442989076\\
593.01	0.000482118365120262\\
594.01	0.00040359003690792\\
595.01	0.000324724152338418\\
596.01	0.000245576587458076\\
597.01	0.000166416924230011\\
598.01	9.17366970063765e-05\\
599.01	2.94669364271135e-05\\
599.02	2.89574269585705e-05\\
599.03	2.84509530852801e-05\\
599.04	2.79475443462508e-05\\
599.05	2.7447230571279e-05\\
599.06	2.69500418838293e-05\\
599.07	2.64560087039449e-05\\
599.08	2.59651617511691e-05\\
599.09	2.54775320475218e-05\\
599.1	2.49931509204836e-05\\
599.11	2.45120500060175e-05\\
599.12	2.40342612516185e-05\\
599.13	2.35598169193978e-05\\
599.14	2.30887495892007e-05\\
599.15	2.26210921617405e-05\\
599.16	2.215687786177e-05\\
599.17	2.16961402412993e-05\\
599.18	2.12389131828174e-05\\
599.19	2.07852309025806e-05\\
599.2	2.03351279538938e-05\\
599.21	1.98886392304524e-05\\
599.22	1.94457999697188e-05\\
599.23	1.90066457563046e-05\\
599.24	1.85712125254211e-05\\
599.25	1.81395365663334e-05\\
599.26	1.77116564095865e-05\\
599.27	1.72876122135658e-05\\
599.28	1.68674445361946e-05\\
599.29	1.64511943388859e-05\\
599.3	1.60389029905273e-05\\
599.31	1.56306122715087e-05\\
599.32	1.52263643777902e-05\\
599.33	1.48262019250105e-05\\
599.34	1.44301679526268e-05\\
599.35	1.40383059281154e-05\\
599.36	1.36506597511916e-05\\
599.37	1.32672737580847e-05\\
599.38	1.28881927258535e-05\\
599.39	1.25134618767404e-05\\
599.4	1.21431268825696e-05\\
599.41	1.17772338691906e-05\\
599.42	1.14158294209719e-05\\
599.43	1.10589605853174e-05\\
599.44	1.07066748772523e-05\\
599.45	1.03590202840433e-05\\
599.46	1.00160452698589e-05\\
599.47	9.67779878049101e-06\\
599.48	9.34433024810284e-06\\
599.49	9.0156895960411e-06\\
599.5	8.69192724369319e-06\\
599.51	8.37309411137396e-06\\
599.52	8.05924162529392e-06\\
599.53	7.75042172253965e-06\\
599.54	7.44668685613396e-06\\
599.55	7.14809000013084e-06\\
599.56	6.85468465475535e-06\\
599.57	6.56652485161134e-06\\
599.58	6.2836651589307e-06\\
599.59	6.00616068686249e-06\\
599.6	5.73406709284546e-06\\
599.61	5.46744058700123e-06\\
599.62	5.2063379376039e-06\\
599.63	4.95081647657741e-06\\
599.64	4.70093410508653e-06\\
599.65	4.45674929914174e-06\\
599.66	4.21832111529089e-06\\
599.67	3.98570919634376e-06\\
599.68	3.75897377716608e-06\\
599.69	3.53817569053241e-06\\
599.7	3.32337637303295e-06\\
599.71	3.11463787102881e-06\\
599.72	2.91202284668363e-06\\
599.73	2.71559458404555e-06\\
599.74	2.52541699518466e-06\\
599.75	2.34155462640155e-06\\
599.76	2.16407266449663e-06\\
599.77	1.99303694307755e-06\\
599.78	1.82851394897772e-06\\
599.79	1.67057082867475e-06\\
599.8	1.51927539483211e-06\\
599.81	1.37469613287027e-06\\
599.82	1.23690220760544e-06\\
599.83	1.10596346997172e-06\\
599.84	9.81950463784659e-07\\
599.85	8.64934432591793e-07\\
599.86	7.54987326587186e-07\\
599.87	6.52181809583305e-07\\
599.88	5.56591266064402e-07\\
599.89	4.68289808293679e-07\\
599.9	3.87352283521061e-07\\
599.91	3.13854281216996e-07\\
599.92	2.47872140425945e-07\\
599.93	1.89482957149364e-07\\
599.94	1.38764591834512e-07\\
599.95	9.57956769204876e-08\\
599.96	6.06556244606149e-08\\
599.97	3.34246338211386e-08\\
599.98	1.4183699454523e-08\\
599.99	3.014618757749e-09\\
600	0\\
};
\addplot [color=black!50!mycolor20,solid,forget plot]
  table[row sep=crcr]{%
0.01	0.00515885702990981\\
1.01	0.0051588560241\\
2.01	0.00515885499729988\\
3.01	0.00515885394907215\\
4.01	0.00515885287897005\\
5.01	0.00515885178653789\\
6.01	0.00515885067131022\\
7.01	0.00515884953281223\\
8.01	0.00515884837055896\\
9.01	0.00515884718405548\\
10.01	0.0051588459727966\\
11.01	0.00515884473626641\\
12.01	0.00515884347393856\\
13.01	0.00515884218527537\\
14.01	0.00515884086972846\\
15.01	0.00515883952673772\\
16.01	0.00515883815573141\\
17.01	0.0051588367561263\\
18.01	0.0051588353273264\\
19.01	0.00515883386872376\\
20.01	0.00515883237969774\\
21.01	0.00515883085961459\\
22.01	0.00515882930782761\\
23.01	0.00515882772367658\\
24.01	0.00515882610648755\\
25.01	0.00515882445557254\\
26.01	0.00515882277022942\\
27.01	0.00515882104974142\\
28.01	0.00515881929337658\\
29.01	0.00515881750038814\\
30.01	0.00515881567001362\\
31.01	0.00515881380147459\\
32.01	0.00515881189397682\\
33.01	0.00515880994670914\\
34.01	0.00515880795884389\\
35.01	0.00515880592953614\\
36.01	0.00515880385792326\\
37.01	0.0051588017431248\\
38.01	0.00515879958424202\\
39.01	0.00515879738035744\\
40.01	0.0051587951305346\\
41.01	0.00515879283381773\\
42.01	0.00515879048923105\\
43.01	0.00515878809577853\\
44.01	0.00515878565244348\\
45.01	0.00515878315818808\\
46.01	0.00515878061195318\\
47.01	0.00515877801265732\\
48.01	0.00515877535919682\\
49.01	0.00515877265044496\\
50.01	0.00515876988525172\\
51.01	0.00515876706244326\\
52.01	0.00515876418082143\\
53.01	0.00515876123916288\\
54.01	0.00515875823621926\\
55.01	0.00515875517071615\\
56.01	0.00515875204135281\\
57.01	0.00515874884680145\\
58.01	0.00515874558570655\\
59.01	0.00515874225668481\\
60.01	0.00515873885832414\\
61.01	0.00515873538918293\\
62.01	0.00515873184779\\
63.01	0.0051587282326435\\
64.01	0.00515872454221057\\
65.01	0.00515872077492647\\
66.01	0.00515871692919399\\
67.01	0.00515871300338292\\
68.01	0.00515870899582923\\
69.01	0.00515870490483436\\
70.01	0.00515870072866464\\
71.01	0.00515869646555036\\
72.01	0.0051586921136853\\
73.01	0.00515868767122566\\
74.01	0.0051586831362895\\
75.01	0.00515867850695594\\
76.01	0.00515867378126429\\
77.01	0.00515866895721325\\
78.01	0.00515866403276003\\
79.01	0.00515865900581964\\
80.01	0.00515865387426385\\
81.01	0.00515864863592042\\
82.01	0.00515864328857214\\
83.01	0.00515863782995577\\
84.01	0.0051586322577615\\
85.01	0.0051586265696312\\
86.01	0.00515862076315848\\
87.01	0.005158614835887\\
88.01	0.00515860878530952\\
89.01	0.00515860260886694\\
90.01	0.00515859630394725\\
91.01	0.00515858986788435\\
92.01	0.00515858329795716\\
93.01	0.00515857659138814\\
94.01	0.00515856974534228\\
95.01	0.00515856275692627\\
96.01	0.00515855562318678\\
97.01	0.00515854834110911\\
98.01	0.00515854090761676\\
99.01	0.00515853331956932\\
100.01	0.00515852557376181\\
101.01	0.00515851766692258\\
102.01	0.00515850959571263\\
103.01	0.00515850135672406\\
104.01	0.00515849294647825\\
105.01	0.005158484361425\\
106.01	0.00515847559794085\\
107.01	0.00515846665232715\\
108.01	0.00515845752080916\\
109.01	0.0051584481995342\\
110.01	0.00515843868457001\\
111.01	0.00515842897190299\\
112.01	0.00515841905743692\\
113.01	0.00515840893699111\\
114.01	0.00515839860629824\\
115.01	0.00515838806100324\\
116.01	0.00515837729666118\\
117.01	0.00515836630873546\\
118.01	0.0051583550925958\\
119.01	0.00515834364351641\\
120.01	0.00515833195667411\\
121.01	0.00515832002714636\\
122.01	0.00515830784990911\\
123.01	0.00515829541983468\\
124.01	0.00515828273168972\\
125.01	0.0051582697801332\\
126.01	0.00515825655971399\\
127.01	0.00515824306486858\\
128.01	0.00515822928991865\\
129.01	0.00515821522906947\\
130.01	0.00515820087640657\\
131.01	0.00515818622589406\\
132.01	0.00515817127137142\\
133.01	0.00515815600655153\\
134.01	0.00515814042501787\\
135.01	0.005158124520222\\
136.01	0.00515810828548049\\
137.01	0.0051580917139727\\
138.01	0.0051580747987376\\
139.01	0.00515805753267124\\
140.01	0.00515803990852314\\
141.01	0.00515802191889429\\
142.01	0.00515800355623319\\
143.01	0.00515798481283354\\
144.01	0.00515796568083053\\
145.01	0.00515794615219766\\
146.01	0.00515792621874401\\
147.01	0.00515790587211008\\
148.01	0.00515788510376496\\
149.01	0.00515786390500255\\
150.01	0.00515784226693803\\
151.01	0.00515782018050432\\
152.01	0.00515779763644852\\
153.01	0.00515777462532752\\
154.01	0.00515775113750518\\
155.01	0.00515772716314745\\
156.01	0.00515770269221862\\
157.01	0.00515767771447749\\
158.01	0.00515765221947307\\
159.01	0.00515762619654012\\
160.01	0.00515759963479528\\
161.01	0.00515757252313203\\
162.01	0.00515754485021663\\
163.01	0.00515751660448361\\
164.01	0.00515748777413071\\
165.01	0.00515745834711416\\
166.01	0.00515742831114403\\
167.01	0.00515739765367906\\
168.01	0.00515736636192184\\
169.01	0.00515733442281314\\
170.01	0.00515730182302728\\
171.01	0.00515726854896632\\
172.01	0.00515723458675454\\
173.01	0.00515719992223326\\
174.01	0.00515716454095461\\
175.01	0.00515712842817631\\
176.01	0.00515709156885521\\
177.01	0.00515705394764147\\
178.01	0.00515701554887255\\
179.01	0.0051569763565666\\
180.01	0.00515693635441651\\
181.01	0.00515689552578291\\
182.01	0.00515685385368791\\
183.01	0.0051568113208079\\
184.01	0.00515676790946716\\
185.01	0.00515672360163037\\
186.01	0.00515667837889592\\
187.01	0.00515663222248803\\
188.01	0.00515658511324966\\
189.01	0.00515653703163458\\
190.01	0.00515648795770014\\
191.01	0.0051564378710991\\
192.01	0.00515638675107132\\
193.01	0.005156334576436\\
194.01	0.00515628132558318\\
195.01	0.00515622697646506\\
196.01	0.00515617150658773\\
197.01	0.00515611489300189\\
198.01	0.00515605711229397\\
199.01	0.00515599814057718\\
200.01	0.00515593795348198\\
201.01	0.00515587652614653\\
202.01	0.00515581383320716\\
203.01	0.00515574984878829\\
204.01	0.00515568454649235\\
205.01	0.00515561789939006\\
206.01	0.00515554988000918\\
207.01	0.00515548046032457\\
208.01	0.00515540961174691\\
209.01	0.00515533730511193\\
210.01	0.00515526351066906\\
211.01	0.00515518819807012\\
212.01	0.00515511133635769\\
213.01	0.00515503289395306\\
214.01	0.00515495283864426\\
215.01	0.00515487113757402\\
216.01	0.00515478775722689\\
217.01	0.0051547026634168\\
218.01	0.00515461582127406\\
219.01	0.00515452719523198\\
220.01	0.00515443674901382\\
221.01	0.00515434444561874\\
222.01	0.00515425024730838\\
223.01	0.00515415411559234\\
224.01	0.00515405601121396\\
225.01	0.00515395589413587\\
226.01	0.00515385372352475\\
227.01	0.00515374945773651\\
228.01	0.00515364305430075\\
229.01	0.00515353446990535\\
230.01	0.00515342366038035\\
231.01	0.00515331058068166\\
232.01	0.00515319518487507\\
233.01	0.00515307742611925\\
234.01	0.00515295725664875\\
235.01	0.005152834627757\\
236.01	0.0051527094897785\\
237.01	0.0051525817920708\\
238.01	0.00515245148299724\\
239.01	0.00515231850990778\\
240.01	0.00515218281912065\\
241.01	0.00515204435590355\\
242.01	0.00515190306445377\\
243.01	0.00515175888787945\\
244.01	0.00515161176817957\\
245.01	0.00515146164622308\\
246.01	0.0051513084617296\\
247.01	0.00515115215324801\\
248.01	0.00515099265813542\\
249.01	0.00515082991253633\\
250.01	0.00515066385136038\\
251.01	0.00515049440826115\\
252.01	0.00515032151561338\\
253.01	0.00515014510449054\\
254.01	0.00514996510464259\\
255.01	0.00514978144447204\\
256.01	0.00514959405101135\\
257.01	0.00514940284989878\\
258.01	0.00514920776535424\\
259.01	0.00514900872015569\\
260.01	0.00514880563561422\\
261.01	0.0051485984315492\\
262.01	0.00514838702626342\\
263.01	0.00514817133651743\\
264.01	0.00514795127750393\\
265.01	0.00514772676282249\\
266.01	0.00514749770445233\\
267.01	0.00514726401272705\\
268.01	0.00514702559630736\\
269.01	0.00514678236215444\\
270.01	0.00514653421550262\\
271.01	0.00514628105983266\\
272.01	0.00514602279684371\\
273.01	0.00514575932642596\\
274.01	0.0051454905466326\\
275.01	0.005145216353652\\
276.01	0.0051449366417792\\
277.01	0.00514465130338831\\
278.01	0.00514436022890354\\
279.01	0.0051440633067712\\
280.01	0.00514376042343024\\
281.01	0.0051434514632844\\
282.01	0.0051431363086736\\
283.01	0.0051428148398448\\
284.01	0.00514248693492345\\
285.01	0.00514215246988519\\
286.01	0.00514181131852681\\
287.01	0.00514146335243798\\
288.01	0.00514110844097311\\
289.01	0.00514074645122274\\
290.01	0.00514037724798586\\
291.01	0.0051400006937419\\
292.01	0.00513961664862325\\
293.01	0.00513922497038804\\
294.01	0.00513882551439329\\
295.01	0.00513841813356863\\
296.01	0.00513800267838955\\
297.01	0.00513757899685242\\
298.01	0.00513714693444902\\
299.01	0.00513670633414256\\
300.01	0.00513625703634312\\
301.01	0.0051357988788851\\
302.01	0.00513533169700429\\
303.01	0.00513485532331667\\
304.01	0.00513436958779732\\
305.01	0.00513387431776059\\
306.01	0.00513336933784176\\
307.01	0.00513285446997861\\
308.01	0.00513232953339542\\
309.01	0.0051317943445872\\
310.01	0.00513124871730558\\
311.01	0.0051306924625467\\
312.01	0.00513012538853882\\
313.01	0.00512954730073386\\
314.01	0.00512895800179832\\
315.01	0.00512835729160743\\
316.01	0.00512774496724011\\
317.01	0.00512712082297672\\
318.01	0.00512648465029781\\
319.01	0.00512583623788549\\
320.01	0.00512517537162689\\
321.01	0.00512450183461948\\
322.01	0.0051238154071795\\
323.01	0.00512311586685151\\
324.01	0.00512240298842167\\
325.01	0.0051216765439323\\
326.01	0.00512093630270024\\
327.01	0.00512018203133715\\
328.01	0.00511941349377248\\
329.01	0.00511863045127929\\
330.01	0.00511783266250308\\
331.01	0.00511701988349325\\
332.01	0.00511619186773709\\
333.01	0.00511534836619702\\
334.01	0.00511448912735015\\
335.01	0.00511361389723113\\
336.01	0.0051127224194771\\
337.01	0.00511181443537549\\
338.01	0.005110889683914\\
339.01	0.0051099479018334\\
340.01	0.00510898882368156\\
341.01	0.00510801218187047\\
342.01	0.00510701770673342\\
343.01	0.00510600512658526\\
344.01	0.00510497416778141\\
345.01	0.00510392455477939\\
346.01	0.00510285601019871\\
347.01	0.00510176825488083\\
348.01	0.00510066100794834\\
349.01	0.00509953398686104\\
350.01	0.00509838690747049\\
351.01	0.00509721948406981\\
352.01	0.0050960314294396\\
353.01	0.00509482245488743\\
354.01	0.00509359227028119\\
355.01	0.00509234058407305\\
356.01	0.00509106710331452\\
357.01	0.00508977153365951\\
358.01	0.00508845357935487\\
359.01	0.00508711294321496\\
360.01	0.00508574932658015\\
361.01	0.00508436242925602\\
362.01	0.00508295194943009\\
363.01	0.0050815175835661\\
364.01	0.00508005902627021\\
365.01	0.0050785759701294\\
366.01	0.00507706810551666\\
367.01	0.00507553512036194\\
368.01	0.00507397669988443\\
369.01	0.00507239252628455\\
370.01	0.0050707822783914\\
371.01	0.00506914563126312\\
372.01	0.00506748225573749\\
373.01	0.00506579181792998\\
374.01	0.00506407397867616\\
375.01	0.00506232839291728\\
376.01	0.00506055470902655\\
377.01	0.0050587525680755\\
378.01	0.00505692160303954\\
379.01	0.00505506143794429\\
380.01	0.00505317168695244\\
381.01	0.00505125195339577\\
382.01	0.00504930182875586\\
383.01	0.00504732089159889\\
384.01	0.00504530870647419\\
385.01	0.00504326482278501\\
386.01	0.00504118877364544\\
387.01	0.00503908007473788\\
388.01	0.00503693822319072\\
389.01	0.00503476269649633\\
390.01	0.00503255295149568\\
391.01	0.00503030842345703\\
392.01	0.00502802852527992\\
393.01	0.0050257126468598\\
394.01	0.00502336015465031\\
395.01	0.00502097039146361\\
396.01	0.00501854267654705\\
397.01	0.00501607630598036\\
398.01	0.00501357055342879\\
399.01	0.00501102467128963\\
400.01	0.00500843789225892\\
401.01	0.00500580943133779\\
402.01	0.00500313848828397\\
403.01	0.00500042425049423\\
404.01	0.00499766589628516\\
405.01	0.00499486259850927\\
406.01	0.0049920135284137\\
407.01	0.00498911785961365\\
408.01	0.00498617477201683\\
409.01	0.00498318345549585\\
410.01	0.00498014311307841\\
411.01	0.0049770529633992\\
412.01	0.00497391224215602\\
413.01	0.00497072020233393\\
414.01	0.00496747611302017\\
415.01	0.00496417925673757\\
416.01	0.00496082892538658\\
417.01	0.00495742441509799\\
418.01	0.00495396502055228\\
419.01	0.00495045002956129\\
420.01	0.00494687871883495\\
421.01	0.0049432503514938\\
422.01	0.00493956417562139\\
423.01	0.0049358194228934\\
424.01	0.00493201530710189\\
425.01	0.00492815102256774\\
426.01	0.00492422574244058\\
427.01	0.00492023861688482\\
428.01	0.0049161887711512\\
429.01	0.00491207530353777\\
430.01	0.00490789728324046\\
431.01	0.00490365374809945\\
432.01	0.00489934370224745\\
433.01	0.00489496611366758\\
434.01	0.00489051991167313\\
435.01	0.00488600398431918\\
436.01	0.00488141717576553\\
437.01	0.00487675828360596\\
438.01	0.00487202605618598\\
439.01	0.00486721918993338\\
440.01	0.00486233632672787\\
441.01	0.0048573760513388\\
442.01	0.00485233688896245\\
443.01	0.00484721730289378\\
444.01	0.00484201569236629\\
445.01	0.00483673039059846\\
446.01	0.00483135966308048\\
447.01	0.00482590170613808\\
448.01	0.00482035464580481\\
449.01	0.00481471653703252\\
450.01	0.00480898536326192\\
451.01	0.00480315903636792\\
452.01	0.00479723539698434\\
453.01	0.00479121221520126\\
454.01	0.0047850871916114\\
455.01	0.0047788579586666\\
456.01	0.00477252208228742\\
457.01	0.00476607706364531\\
458.01	0.00475952034102221\\
459.01	0.00475284929162807\\
460.01	0.00474606123324042\\
461.01	0.0047391534255208\\
462.01	0.00473212307085284\\
463.01	0.00472496731455358\\
464.01	0.0047176832443234\\
465.01	0.00471026788883426\\
466.01	0.00470271821540123\\
467.01	0.00469503112674995\\
468.01	0.00468720345697479\\
469.01	0.0046792319668779\\
470.01	0.00467111333897339\\
471.01	0.00466284417252799\\
472.01	0.00465442097905768\\
473.01	0.00464584017869519\\
474.01	0.00463709809774316\\
475.01	0.0046281909675301\\
476.01	0.00461911492439009\\
477.01	0.00460986601030792\\
478.01	0.00460044017382117\\
479.01	0.00459083327102632\\
480.01	0.00458104106663141\\
481.01	0.00457105923499075\\
482.01	0.00456088336105611\\
483.01	0.00455050894117474\\
484.01	0.00453993138366703\\
485.01	0.0045291460091181\\
486.01	0.00451814805033159\\
487.01	0.00450693265190246\\
488.01	0.00449549486938637\\
489.01	0.00448382966806533\\
490.01	0.00447193192133378\\
491.01	0.00445979640875732\\
492.01	0.00444741781388005\\
493.01	0.0044347907218801\\
494.01	0.00442190961718438\\
495.01	0.00440876888115547\\
496.01	0.00439536278994963\\
497.01	0.00438168551261193\\
498.01	0.00436773110942803\\
499.01	0.00435349353049303\\
500.01	0.00433896661440256\\
501.01	0.00432414408693396\\
502.01	0.00430901955959005\\
503.01	0.00429358652792935\\
504.01	0.00427783836966785\\
505.01	0.00426176834256679\\
506.01	0.0042453695821297\\
507.01	0.00422863509914185\\
508.01	0.00421155777708945\\
509.01	0.0041941303695008\\
510.01	0.00417634549724843\\
511.01	0.00415819564585181\\
512.01	0.0041396731628067\\
513.01	0.00412077025496171\\
514.01	0.00410147898594817\\
515.01	0.00408179127365571\\
516.01	0.00406169888774293\\
517.01	0.00404119344716381\\
518.01	0.00402026641769837\\
519.01	0.00399890910948619\\
520.01	0.00397711267457466\\
521.01	0.00395486810450582\\
522.01	0.00393216622796758\\
523.01	0.00390899770853628\\
524.01	0.00388535304253463\\
525.01	0.00386122255702903\\
526.01	0.00383659640798494\\
527.01	0.00381146457860206\\
528.01	0.00378581687784591\\
529.01	0.00375964293919767\\
530.01	0.00373293221964467\\
531.01	0.00370567399893881\\
532.01	0.0036778573791551\\
533.01	0.00364947128458843\\
534.01	0.00362050446202891\\
535.01	0.00359094548146376\\
536.01	0.00356078273725293\\
537.01	0.0035300044498321\\
538.01	0.0034985986679996\\
539.01	0.00346655327184927\\
540.01	0.00343385597641933\\
541.01	0.00340049433613169\\
542.01	0.00336645575010434\\
543.01	0.00333172746843132\\
544.01	0.00329629659952941\\
545.01	0.0032601501186634\\
546.01	0.00322327487777092\\
547.01	0.00318565761672175\\
548.01	0.00314728497615331\\
549.01	0.00310814351204539\\
550.01	0.00306821971220547\\
551.01	0.00302750001485491\\
552.01	0.00298597082952428\\
553.01	0.00294361856048271\\
554.01	0.0029004296329475\\
555.01	0.00285639052234093\\
556.01	0.00281148778688329\\
557.01	0.00276570810383654\\
558.01	0.00271903830973548\\
559.01	0.0026714654449714\\
560.01	0.00262297680311917\\
561.01	0.00257355998542694\\
562.01	0.00252320296091246\\
563.01	0.00247189413254027\\
564.01	0.00241962240997658\\
565.01	0.00236637728944275\\
566.01	0.00231214894120763\\
567.01	0.0022569283052712\\
568.01	0.00220070719580021\\
569.01	0.00214347841486824\\
570.01	0.0020852358760371\\
571.01	0.00202597473827579\\
572.01	0.00196569155065329\\
573.01	0.0019043844081454\\
574.01	0.00184205311876436\\
575.01	0.00177869938203216\\
576.01	0.0017143269785692\\
577.01	0.00164894197023678\\
578.01	0.00158255290983592\\
579.01	0.00151517105879892\\
580.01	0.00144681061058329\\
581.01	0.00137748891654737\\
582.01	0.00130722670990687\\
583.01	0.00123604832187718\\
584.01	0.0011639818822259\\
585.01	0.00109105949409744\\
586.01	0.0010173173700143\\
587.01	0.000942795912266653\\
588.01	0.000867539716298419\\
589.01	0.000791597469971103\\
590.01	0.000715021714474689\\
591.01	0.000637868423831121\\
592.01	0.000560196349006645\\
593.01	0.000482066059121876\\
594.01	0.000403538595524574\\
595.01	0.000324673633827242\\
596.01	0.000245527023505882\\
597.01	0.000166394005355891\\
598.01	9.17366970063799e-05\\
599.01	2.94669364271135e-05\\
599.02	2.89574269585688e-05\\
599.03	2.84509530852819e-05\\
599.04	2.79475443462508e-05\\
599.05	2.7447230571279e-05\\
599.06	2.69500418838293e-05\\
599.07	2.64560087039432e-05\\
599.08	2.59651617511691e-05\\
599.09	2.54775320475218e-05\\
599.1	2.49931509204854e-05\\
599.11	2.45120500060158e-05\\
599.12	2.40342612516167e-05\\
599.13	2.35598169193996e-05\\
599.14	2.30887495892042e-05\\
599.15	2.26210921617405e-05\\
599.16	2.21568778617717e-05\\
599.17	2.16961402412976e-05\\
599.18	2.12389131828191e-05\\
599.19	2.07852309025806e-05\\
599.2	2.03351279538938e-05\\
599.21	1.98886392304542e-05\\
599.22	1.94457999697206e-05\\
599.23	1.90066457563046e-05\\
599.24	1.85712125254211e-05\\
599.25	1.81395365663334e-05\\
599.26	1.77116564095865e-05\\
599.27	1.72876122135641e-05\\
599.28	1.68674445361946e-05\\
599.29	1.64511943388859e-05\\
599.3	1.60389029905273e-05\\
599.31	1.56306122715087e-05\\
599.32	1.52263643777902e-05\\
599.33	1.48262019250087e-05\\
599.34	1.44301679526268e-05\\
599.35	1.40383059281154e-05\\
599.36	1.36506597511899e-05\\
599.37	1.32672737580847e-05\\
599.38	1.28881927258535e-05\\
599.39	1.25134618767387e-05\\
599.4	1.21431268825679e-05\\
599.41	1.17772338691924e-05\\
599.42	1.14158294209736e-05\\
599.43	1.10589605853174e-05\\
599.44	1.07066748772523e-05\\
599.45	1.03590202840433e-05\\
599.46	1.00160452698606e-05\\
599.47	9.67779878049101e-06\\
599.48	9.34433024810284e-06\\
599.49	9.0156895960411e-06\\
599.5	8.69192724369146e-06\\
599.51	8.37309411137396e-06\\
599.52	8.05924162529219e-06\\
599.53	7.75042172253791e-06\\
599.54	7.4466868561357e-06\\
599.55	7.14809000013084e-06\\
599.56	6.85468465475535e-06\\
599.57	6.56652485161308e-06\\
599.58	6.28366515892896e-06\\
599.59	6.00616068686249e-06\\
599.6	5.73406709284546e-06\\
599.61	5.46744058700296e-06\\
599.62	5.20633793760217e-06\\
599.63	4.95081647657568e-06\\
599.64	4.70093410508653e-06\\
599.65	4.45674929914347e-06\\
599.66	4.21832111529262e-06\\
599.67	3.98570919634203e-06\\
599.68	3.75897377716608e-06\\
599.69	3.53817569053415e-06\\
599.7	3.32337637303469e-06\\
599.71	3.11463787102881e-06\\
599.72	2.91202284668363e-06\\
599.73	2.71559458404555e-06\\
599.74	2.52541699518292e-06\\
599.75	2.34155462640155e-06\\
599.76	2.16407266449489e-06\\
599.77	1.99303694307928e-06\\
599.78	1.82851394897598e-06\\
599.79	1.67057082867302e-06\\
599.8	1.51927539483211e-06\\
599.81	1.37469613287027e-06\\
599.82	1.23690220760718e-06\\
599.83	1.10596346997172e-06\\
599.84	9.81950463782924e-07\\
599.85	8.64934432591793e-07\\
599.86	7.54987326587186e-07\\
599.87	6.5218180958504e-07\\
599.88	5.56591266064402e-07\\
599.89	4.68289808295413e-07\\
599.9	3.87352283521061e-07\\
599.91	3.13854281218731e-07\\
599.92	2.4787214042421e-07\\
599.93	1.8948295714763e-07\\
599.94	1.38764591834512e-07\\
599.95	9.57956769222224e-08\\
599.96	6.06556244606149e-08\\
599.97	3.34246338194039e-08\\
599.98	1.4183699454523e-08\\
599.99	3.01461875948372e-09\\
600	0\\
};
\addplot [color=black!60!mycolor21,solid,forget plot]
  table[row sep=crcr]{%
0.01	0.00511537456508619\\
1.01	0.00511537357866577\\
2.01	0.00511537257178985\\
3.01	0.00511537154403555\\
4.01	0.00511537049497141\\
5.01	0.00511536942415678\\
6.01	0.00511536833114231\\
7.01	0.00511536721546909\\
8.01	0.00511536607666884\\
9.01	0.00511536491426357\\
10.01	0.00511536372776555\\
11.01	0.00511536251667717\\
12.01	0.00511536128049021\\
13.01	0.0051153600186863\\
14.01	0.00511535873073629\\
15.01	0.00511535741610018\\
16.01	0.00511535607422688\\
17.01	0.00511535470455369\\
18.01	0.00511535330650676\\
19.01	0.00511535187950028\\
20.01	0.0051153504229363\\
21.01	0.00511534893620466\\
22.01	0.00511534741868266\\
23.01	0.00511534586973479\\
24.01	0.00511534428871243\\
25.01	0.0051153426749539\\
26.01	0.00511534102778349\\
27.01	0.00511533934651185\\
28.01	0.00511533763043549\\
29.01	0.00511533587883644\\
30.01	0.00511533409098209\\
31.01	0.00511533226612451\\
32.01	0.00511533040350066\\
33.01	0.00511532850233169\\
34.01	0.00511532656182295\\
35.01	0.0051153245811633\\
36.01	0.00511532255952498\\
37.01	0.00511532049606332\\
38.01	0.00511531838991627\\
39.01	0.00511531624020418\\
40.01	0.00511531404602932\\
41.01	0.00511531180647548\\
42.01	0.00511530952060763\\
43.01	0.00511530718747187\\
44.01	0.00511530480609428\\
45.01	0.00511530237548142\\
46.01	0.00511529989461918\\
47.01	0.00511529736247276\\
48.01	0.00511529477798627\\
49.01	0.00511529214008206\\
50.01	0.00511528944766045\\
51.01	0.00511528669959924\\
52.01	0.00511528389475322\\
53.01	0.00511528103195392\\
54.01	0.00511527811000864\\
55.01	0.00511527512770054\\
56.01	0.0051152720837875\\
57.01	0.00511526897700234\\
58.01	0.00511526580605207\\
59.01	0.0051152625696169\\
60.01	0.00511525926635029\\
61.01	0.00511525589487797\\
62.01	0.00511525245379763\\
63.01	0.00511524894167829\\
64.01	0.00511524535705981\\
65.01	0.00511524169845221\\
66.01	0.00511523796433488\\
67.01	0.00511523415315624\\
68.01	0.00511523026333304\\
69.01	0.0051152262932497\\
70.01	0.00511522224125771\\
71.01	0.0051152181056747\\
72.01	0.00511521388478418\\
73.01	0.0051152095768344\\
74.01	0.00511520518003802\\
75.01	0.00511520069257119\\
76.01	0.00511519611257275\\
77.01	0.00511519143814377\\
78.01	0.0051151866673465\\
79.01	0.00511518179820347\\
80.01	0.00511517682869724\\
81.01	0.00511517175676903\\
82.01	0.00511516658031813\\
83.01	0.005115161297201\\
84.01	0.00511515590523046\\
85.01	0.00511515040217494\\
86.01	0.00511514478575718\\
87.01	0.00511513905365357\\
88.01	0.00511513320349328\\
89.01	0.00511512723285709\\
90.01	0.00511512113927664\\
91.01	0.00511511492023344\\
92.01	0.00511510857315768\\
93.01	0.00511510209542713\\
94.01	0.00511509548436636\\
95.01	0.00511508873724558\\
96.01	0.00511508185127942\\
97.01	0.00511507482362601\\
98.01	0.00511506765138546\\
99.01	0.00511506033159934\\
100.01	0.00511505286124885\\
101.01	0.00511504523725388\\
102.01	0.00511503745647194\\
103.01	0.00511502951569646\\
104.01	0.00511502141165627\\
105.01	0.00511501314101339\\
106.01	0.0051150047003622\\
107.01	0.00511499608622811\\
108.01	0.00511498729506608\\
109.01	0.00511497832325891\\
110.01	0.00511496916711637\\
111.01	0.00511495982287342\\
112.01	0.00511495028668861\\
113.01	0.00511494055464267\\
114.01	0.00511493062273714\\
115.01	0.00511492048689247\\
116.01	0.00511491014294623\\
117.01	0.00511489958665213\\
118.01	0.00511488881367781\\
119.01	0.00511487781960355\\
120.01	0.00511486659991981\\
121.01	0.00511485515002609\\
122.01	0.00511484346522868\\
123.01	0.00511483154073941\\
124.01	0.00511481937167308\\
125.01	0.0051148069530458\\
126.01	0.00511479427977302\\
127.01	0.00511478134666765\\
128.01	0.00511476814843793\\
129.01	0.00511475467968496\\
130.01	0.00511474093490148\\
131.01	0.00511472690846857\\
132.01	0.00511471259465452\\
133.01	0.00511469798761181\\
134.01	0.00511468308137513\\
135.01	0.00511466786985886\\
136.01	0.005114652346855\\
137.01	0.00511463650603059\\
138.01	0.00511462034092494\\
139.01	0.00511460384494737\\
140.01	0.00511458701137482\\
141.01	0.0051145698333489\\
142.01	0.0051145523038734\\
143.01	0.00511453441581127\\
144.01	0.00511451616188224\\
145.01	0.00511449753465995\\
146.01	0.0051144785265685\\
147.01	0.00511445912988006\\
148.01	0.00511443933671178\\
149.01	0.00511441913902264\\
150.01	0.00511439852861036\\
151.01	0.00511437749710821\\
152.01	0.00511435603598174\\
153.01	0.00511433413652551\\
154.01	0.0051143117898597\\
155.01	0.00511428898692678\\
156.01	0.00511426571848785\\
157.01	0.00511424197511929\\
158.01	0.00511421774720892\\
159.01	0.00511419302495233\\
160.01	0.00511416779834923\\
161.01	0.00511414205719964\\
162.01	0.00511411579109974\\
163.01	0.00511408898943806\\
164.01	0.0051140616413913\\
165.01	0.00511403373592053\\
166.01	0.00511400526176639\\
167.01	0.00511397620744537\\
168.01	0.00511394656124498\\
169.01	0.00511391631121967\\
170.01	0.00511388544518601\\
171.01	0.00511385395071814\\
172.01	0.00511382181514319\\
173.01	0.00511378902553609\\
174.01	0.00511375556871525\\
175.01	0.00511372143123701\\
176.01	0.00511368659939112\\
177.01	0.00511365105919513\\
178.01	0.00511361479638916\\
179.01	0.00511357779643073\\
180.01	0.00511354004448925\\
181.01	0.00511350152544029\\
182.01	0.00511346222386012\\
183.01	0.00511342212402013\\
184.01	0.00511338120988018\\
185.01	0.00511333946508353\\
186.01	0.0051132968729502\\
187.01	0.00511325341647084\\
188.01	0.00511320907830058\\
189.01	0.00511316384075266\\
190.01	0.00511311768579164\\
191.01	0.00511307059502656\\
192.01	0.00511302254970484\\
193.01	0.00511297353070479\\
194.01	0.00511292351852883\\
195.01	0.00511287249329615\\
196.01	0.0051128204347357\\
197.01	0.00511276732217853\\
198.01	0.00511271313455037\\
199.01	0.00511265785036407\\
200.01	0.00511260144771163\\
201.01	0.00511254390425591\\
202.01	0.00511248519722357\\
203.01	0.00511242530339583\\
204.01	0.00511236419910063\\
205.01	0.00511230186020375\\
206.01	0.00511223826210062\\
207.01	0.00511217337970698\\
208.01	0.00511210718745071\\
209.01	0.00511203965926171\\
210.01	0.00511197076856337\\
211.01	0.00511190048826304\\
212.01	0.00511182879074205\\
213.01	0.00511175564784649\\
214.01	0.00511168103087718\\
215.01	0.00511160491057922\\
216.01	0.00511152725713227\\
217.01	0.00511144804013977\\
218.01	0.00511136722861875\\
219.01	0.00511128479098874\\
220.01	0.00511120069506087\\
221.01	0.00511111490802715\\
222.01	0.00511102739644873\\
223.01	0.00511093812624481\\
224.01	0.005110847062681\\
225.01	0.00511075417035707\\
226.01	0.00511065941319563\\
227.01	0.00511056275442924\\
228.01	0.00511046415658882\\
229.01	0.00511036358149029\\
230.01	0.00511026099022245\\
231.01	0.00511015634313397\\
232.01	0.00511004959981986\\
233.01	0.00510994071910844\\
234.01	0.00510982965904792\\
235.01	0.00510971637689247\\
236.01	0.00510960082908849\\
237.01	0.00510948297126058\\
238.01	0.0051093627581968\\
239.01	0.00510924014383445\\
240.01	0.00510911508124556\\
241.01	0.00510898752262192\\
242.01	0.00510885741925982\\
243.01	0.00510872472154461\\
244.01	0.00510858937893547\\
245.01	0.00510845133994987\\
246.01	0.00510831055214733\\
247.01	0.00510816696211345\\
248.01	0.00510802051544365\\
249.01	0.00510787115672668\\
250.01	0.00510771882952772\\
251.01	0.00510756347637195\\
252.01	0.00510740503872715\\
253.01	0.0051072434569865\\
254.01	0.00510707867045102\\
255.01	0.00510691061731235\\
256.01	0.00510673923463431\\
257.01	0.00510656445833546\\
258.01	0.00510638622317047\\
259.01	0.00510620446271231\\
260.01	0.0051060191093331\\
261.01	0.00510583009418576\\
262.01	0.0051056373471849\\
263.01	0.00510544079698826\\
264.01	0.00510524037097728\\
265.01	0.00510503599523753\\
266.01	0.00510482759453972\\
267.01	0.00510461509231971\\
268.01	0.00510439841065887\\
269.01	0.0051041774702647\\
270.01	0.00510395219044992\\
271.01	0.00510372248911315\\
272.01	0.00510348828271854\\
273.01	0.00510324948627538\\
274.01	0.00510300601331802\\
275.01	0.00510275777588507\\
276.01	0.00510250468449915\\
277.01	0.00510224664814656\\
278.01	0.00510198357425603\\
279.01	0.00510171536867864\\
280.01	0.00510144193566733\\
281.01	0.0051011631778558\\
282.01	0.0051008789962381\\
283.01	0.00510058929014804\\
284.01	0.00510029395723858\\
285.01	0.00509999289346137\\
286.01	0.0050996859930462\\
287.01	0.00509937314848069\\
288.01	0.00509905425048967\\
289.01	0.00509872918801566\\
290.01	0.00509839784819816\\
291.01	0.00509806011635408\\
292.01	0.0050977158759578\\
293.01	0.00509736500862192\\
294.01	0.00509700739407746\\
295.01	0.00509664291015461\\
296.01	0.00509627143276457\\
297.01	0.00509589283587995\\
298.01	0.00509550699151661\\
299.01	0.00509511376971576\\
300.01	0.00509471303852548\\
301.01	0.0050943046639842\\
302.01	0.00509388851010228\\
303.01	0.00509346443884578\\
304.01	0.00509303231011988\\
305.01	0.00509259198175275\\
306.01	0.00509214330947942\\
307.01	0.00509168614692717\\
308.01	0.00509122034559952\\
309.01	0.00509074575486255\\
310.01	0.00509026222193018\\
311.01	0.00508976959185067\\
312.01	0.00508926770749355\\
313.01	0.00508875640953611\\
314.01	0.00508823553645157\\
315.01	0.00508770492449654\\
316.01	0.00508716440769957\\
317.01	0.00508661381784954\\
318.01	0.00508605298448518\\
319.01	0.00508548173488376\\
320.01	0.00508489989405098\\
321.01	0.0050843072847109\\
322.01	0.0050837037272957\\
323.01	0.0050830890399361\\
324.01	0.00508246303845125\\
325.01	0.00508182553633909\\
326.01	0.00508117634476631\\
327.01	0.00508051527255817\\
328.01	0.00507984212618831\\
329.01	0.00507915670976748\\
330.01	0.00507845882503223\\
331.01	0.00507774827133237\\
332.01	0.00507702484561849\\
333.01	0.00507628834242714\\
334.01	0.00507553855386577\\
335.01	0.00507477526959526\\
336.01	0.00507399827681169\\
337.01	0.00507320736022492\\
338.01	0.0050724023020362\\
339.01	0.00507158288191181\\
340.01	0.00507074887695489\\
341.01	0.00506990006167322\\
342.01	0.00506903620794389\\
343.01	0.00506815708497276\\
344.01	0.00506726245925113\\
345.01	0.00506635209450567\\
346.01	0.00506542575164405\\
347.01	0.00506448318869422\\
348.01	0.00506352416073646\\
349.01	0.00506254841982928\\
350.01	0.00506155571492676\\
351.01	0.00506054579178812\\
352.01	0.00505951839287799\\
353.01	0.00505847325725697\\
354.01	0.00505741012046191\\
355.01	0.00505632871437514\\
356.01	0.00505522876708182\\
357.01	0.0050541100027152\\
358.01	0.00505297214128824\\
359.01	0.00505181489851223\\
360.01	0.00505063798560022\\
361.01	0.00504944110905546\\
362.01	0.00504822397044529\\
363.01	0.00504698626615752\\
364.01	0.00504572768714187\\
365.01	0.00504444791863355\\
366.01	0.00504314663986067\\
367.01	0.00504182352373438\\
368.01	0.00504047823652271\\
369.01	0.00503911043750747\\
370.01	0.00503771977862624\\
371.01	0.0050363059040985\\
372.01	0.00503486845003876\\
373.01	0.00503340704405683\\
374.01	0.00503192130484766\\
375.01	0.00503041084177334\\
376.01	0.00502887525443856\\
377.01	0.00502731413226412\\
378.01	0.00502572705406087\\
379.01	0.00502411358760904\\
380.01	0.0050224732892465\\
381.01	0.00502080570347132\\
382.01	0.00501911036256501\\
383.01	0.00501738678624098\\
384.01	0.00501563448132577\\
385.01	0.00501385294147864\\
386.01	0.00501204164695771\\
387.01	0.00501020006443945\\
388.01	0.00500832764689743\\
389.01	0.00500642383354889\\
390.01	0.00500448804987441\\
391.01	0.00500251970771589\\
392.01	0.00500051820545903\\
393.01	0.00499848292830042\\
394.01	0.00499641324860389\\
395.01	0.00499430852634055\\
396.01	0.0049921681096128\\
397.01	0.00498999133525027\\
398.01	0.00498777752947023\\
399.01	0.0049855260085841\\
400.01	0.00498323607973127\\
401.01	0.00498090704161521\\
402.01	0.00497853818521132\\
403.01	0.00497612879441471\\
404.01	0.00497367814658854\\
405.01	0.00497118551297381\\
406.01	0.00496865015892014\\
407.01	0.00496607134389796\\
408.01	0.00496344832125691\\
409.01	0.00496078033770405\\
410.01	0.00495806663248456\\
411.01	0.00495530643626621\\
412.01	0.00495249896974442\\
413.01	0.00494964344200882\\
414.01	0.0049467390487322\\
415.01	0.00494378497026461\\
416.01	0.004940780369726\\
417.01	0.00493772439119356\\
418.01	0.0049346161580621\\
419.01	0.00493145477161677\\
420.01	0.0049282393097965\\
421.01	0.00492496882606195\\
422.01	0.00492164234826773\\
423.01	0.00491825887750382\\
424.01	0.00491481738690305\\
425.01	0.00491131682041853\\
426.01	0.00490775609157437\\
427.01	0.0049041340821939\\
428.01	0.00490044964111023\\
429.01	0.00489670158286241\\
430.01	0.0048928886863865\\
431.01	0.00488900969370451\\
432.01	0.00488506330862008\\
433.01	0.00488104819542758\\
434.01	0.00487696297764154\\
435.01	0.00487280623675731\\
436.01	0.0048685765110472\\
437.01	0.00486427229440418\\
438.01	0.00485989203523984\\
439.01	0.00485543413544438\\
440.01	0.00485089694941698\\
441.01	0.00484627878317314\\
442.01	0.00484157789353523\\
443.01	0.00483679248740917\\
444.01	0.00483192072115264\\
445.01	0.00482696070003325\\
446.01	0.00482191047777615\\
447.01	0.0048167680561974\\
448.01	0.00481153138491524\\
449.01	0.00480619836112931\\
450.01	0.00480076682945332\\
451.01	0.00479523458178626\\
452.01	0.0047895993571997\\
453.01	0.00478385884181872\\
454.01	0.00477801066867047\\
455.01	0.00477205241747283\\
456.01	0.00476598161433339\\
457.01	0.00475979573133239\\
458.01	0.00475349218596161\\
459.01	0.00474706834039562\\
460.01	0.00474052150058208\\
461.01	0.00473384891513581\\
462.01	0.00472704777404177\\
463.01	0.00472011520717495\\
464.01	0.0047130482826625\\
465.01	0.00470584400512422\\
466.01	0.00469849931383928\\
467.01	0.00469101108089903\\
468.01	0.00468337610940962\\
469.01	0.00467559113180777\\
470.01	0.00466765280834874\\
471.01	0.00465955772580652\\
472.01	0.00465130239640618\\
473.01	0.00464288325697488\\
474.01	0.00463429666827228\\
475.01	0.00462553891443265\\
476.01	0.00461660620244293\\
477.01	0.00460749466159679\\
478.01	0.00459820034288695\\
479.01	0.00458871921831872\\
480.01	0.00457904718012815\\
481.01	0.00456918003989079\\
482.01	0.00455911352751145\\
483.01	0.00454884329008707\\
484.01	0.00453836489063851\\
485.01	0.00452767380671381\\
486.01	0.00451676542886776\\
487.01	0.00450563505902704\\
488.01	0.00449427790875835\\
489.01	0.00448268909745497\\
490.01	0.00447086365046334\\
491.01	0.00445879649716923\\
492.01	0.00444648246906463\\
493.01	0.00443391629780925\\
494.01	0.00442109261329795\\
495.01	0.00440800594173683\\
496.01	0.00439465070372407\\
497.01	0.00438102121232231\\
498.01	0.00436711167110619\\
499.01	0.00435291617216462\\
500.01	0.00433842869404037\\
501.01	0.00432364309959411\\
502.01	0.00430855313379167\\
503.01	0.00429315242141763\\
504.01	0.00427743446472644\\
505.01	0.0042613926410404\\
506.01	0.0042450202003061\\
507.01	0.00422831026262157\\
508.01	0.00421125581574242\\
509.01	0.00419384971257737\\
510.01	0.00417608466867955\\
511.01	0.00415795325973936\\
512.01	0.00413944791908216\\
513.01	0.00412056093517416\\
514.01	0.00410128444913751\\
515.01	0.00408161045227904\\
516.01	0.0040615307836341\\
517.01	0.00404103712753402\\
518.01	0.00402012101120428\\
519.01	0.00399877380240454\\
520.01	0.00397698670712463\\
521.01	0.00395475076734716\\
522.01	0.00393205685889359\\
523.01	0.00390889568936628\\
524.01	0.00388525779620339\\
525.01	0.00386113354486115\\
526.01	0.00383651312714225\\
527.01	0.00381138655968965\\
528.01	0.00378574368266509\\
529.01	0.00375957415863976\\
530.01	0.00373286747171988\\
531.01	0.00370561292693941\\
532.01	0.00367779964995124\\
533.01	0.0036494165870535\\
534.01	0.00362045250559075\\
535.01	0.0035908959947719\\
536.01	0.0035607354669546\\
537.01	0.00352995915944617\\
538.01	0.00349855513688045\\
539.01	0.00346651129423165\\
540.01	0.00343381536053716\\
541.01	0.0034004549034037\\
542.01	0.00336641733438228\\
543.01	0.00333168991530363\\
544.01	0.00329625976567544\\
545.01	0.00326011387125328\\
546.01	0.00322323909390545\\
547.01	0.00318562218290602\\
548.01	0.00314724978780293\\
549.01	0.00310810847301851\\
550.01	0.00306818473435897\\
551.01	0.00302746501762232\\
552.01	0.00298593573951309\\
553.01	0.00294358331108954\\
554.01	0.00290039416398957\\
555.01	0.00285635477970408\\
556.01	0.00281145172218551\\
557.01	0.00276567167410694\\
558.01	0.00271900147710948\\
559.01	0.00267142817640317\\
560.01	0.00262293907011199\\
561.01	0.00257352176378272\\
562.01	0.00252316423050231\\
563.01	0.00247185487709762\\
564.01	0.00241958261691408\\
565.01	0.00236633694969541\\
566.01	0.00231210804910259\\
567.01	0.00225688685842675\\
568.01	0.00220066519505408\\
569.01	0.00214343586423708\\
570.01	0.00208519278270673\\
571.01	0.0020259311126231\\
572.01	0.00196564740629848\\
573.01	0.00190433976203401\\
574.01	0.00184200799127701\\
575.01	0.00177865379711897\\
576.01	0.00171428096390598\\
577.01	0.00164889555739739\\
578.01	0.00158250613447483\\
579.01	0.00151512396083625\\
580.01	0.00144676323438244\\
581.01	0.00137744131107293\\
582.01	0.00130717892884751\\
583.01	0.00123600042371503\\
584.01	0.00116393393022847\\
585.01	0.00109101155620331\\
586.01	0.00101726951857635\\
587.01	0.000942748223607881\\
588.01	0.00086749227002462\\
589.01	0.000791550347971492\\
590.01	0.000714974999524649\\
591.01	0.000637822197691413\\
592.01	0.000560150689888671\\
593.01	0.000482021038358199\\
594.01	0.000403494273246938\\
595.01	0.000324630053410709\\
596.01	0.000245484204483332\\
597.01	0.00016637427046296\\
598.01	9.17366970063765e-05\\
599.01	2.94669364271135e-05\\
599.02	2.89574269585705e-05\\
599.03	2.84509530852819e-05\\
599.04	2.79475443462508e-05\\
599.05	2.7447230571279e-05\\
599.06	2.69500418838293e-05\\
599.07	2.64560087039432e-05\\
599.08	2.59651617511691e-05\\
599.09	2.54775320475235e-05\\
599.1	2.49931509204836e-05\\
599.11	2.45120500060158e-05\\
599.12	2.40342612516167e-05\\
599.13	2.35598169193978e-05\\
599.14	2.30887495892024e-05\\
599.15	2.26210921617405e-05\\
599.16	2.215687786177e-05\\
599.17	2.16961402412993e-05\\
599.18	2.12389131828191e-05\\
599.19	2.07852309025824e-05\\
599.2	2.03351279538938e-05\\
599.21	1.98886392304542e-05\\
599.22	1.94457999697206e-05\\
599.23	1.90066457563063e-05\\
599.24	1.85712125254211e-05\\
599.25	1.81395365663334e-05\\
599.26	1.77116564095865e-05\\
599.27	1.72876122135658e-05\\
599.28	1.68674445361963e-05\\
599.29	1.64511943388859e-05\\
599.3	1.60389029905273e-05\\
599.31	1.56306122715087e-05\\
599.32	1.5226364377792e-05\\
599.33	1.48262019250105e-05\\
599.34	1.44301679526268e-05\\
599.35	1.40383059281154e-05\\
599.36	1.36506597511916e-05\\
599.37	1.32672737580847e-05\\
599.38	1.28881927258535e-05\\
599.39	1.25134618767404e-05\\
599.4	1.21431268825696e-05\\
599.41	1.17772338691924e-05\\
599.42	1.14158294209719e-05\\
599.43	1.10589605853174e-05\\
599.44	1.0706674877254e-05\\
599.45	1.03590202840433e-05\\
599.46	1.00160452698589e-05\\
599.47	9.67779878049101e-06\\
599.48	9.34433024810284e-06\\
599.49	9.01568959604283e-06\\
599.5	8.69192724369319e-06\\
599.51	8.37309411137396e-06\\
599.52	8.05924162529392e-06\\
599.53	7.75042172253791e-06\\
599.54	7.44668685613396e-06\\
599.55	7.14809000013084e-06\\
599.56	6.85468465475708e-06\\
599.57	6.56652485161308e-06\\
599.58	6.28366515892896e-06\\
599.59	6.00616068686076e-06\\
599.6	5.73406709284546e-06\\
599.61	5.46744058700296e-06\\
599.62	5.2063379376039e-06\\
599.63	4.95081647657741e-06\\
599.64	4.70093410508653e-06\\
599.65	4.45674929914347e-06\\
599.66	4.21832111529089e-06\\
599.67	3.98570919634376e-06\\
599.68	3.75897377716608e-06\\
599.69	3.53817569053415e-06\\
599.7	3.32337637303295e-06\\
599.71	3.11463787102881e-06\\
599.72	2.91202284668536e-06\\
599.73	2.71559458404382e-06\\
599.74	2.52541699518466e-06\\
599.75	2.34155462640329e-06\\
599.76	2.16407266449489e-06\\
599.77	1.99303694307755e-06\\
599.78	1.82851394897598e-06\\
599.79	1.67057082867302e-06\\
599.8	1.51927539483211e-06\\
599.81	1.37469613287027e-06\\
599.82	1.23690220760718e-06\\
599.83	1.10596346997172e-06\\
599.84	9.81950463782924e-07\\
599.85	8.64934432591793e-07\\
599.86	7.54987326587186e-07\\
599.87	6.5218180958504e-07\\
599.88	5.56591266064402e-07\\
599.89	4.68289808295413e-07\\
599.9	3.87352283521061e-07\\
599.91	3.13854281216996e-07\\
599.92	2.47872140425945e-07\\
599.93	1.89482957149364e-07\\
599.94	1.38764591834512e-07\\
599.95	9.57956769204876e-08\\
599.96	6.06556244606149e-08\\
599.97	3.34246338211386e-08\\
599.98	1.4183699454523e-08\\
599.99	3.01461875948372e-09\\
600	0\\
};
\addplot [color=black!80!mycolor21,solid,forget plot]
  table[row sep=crcr]{%
0.01	0.00508968489854863\\
1.01	0.00508968392840915\\
2.01	0.00508968293827033\\
3.01	0.00508968192772168\\
4.01	0.00508968089634424\\
5.01	0.00508967984371108\\
6.01	0.00508967876938601\\
7.01	0.00508967767292402\\
8.01	0.0050896765538712\\
9.01	0.00508967541176465\\
10.01	0.00508967424613145\\
11.01	0.00508967305648953\\
12.01	0.00508967184234672\\
13.01	0.00508967060320077\\
14.01	0.00508966933853946\\
15.01	0.00508966804784016\\
16.01	0.00508966673056913\\
17.01	0.00508966538618245\\
18.01	0.00508966401412461\\
19.01	0.00508966261382906\\
20.01	0.00508966118471765\\
21.01	0.00508965972620041\\
22.01	0.00508965823767534\\
23.01	0.00508965671852852\\
24.01	0.00508965516813317\\
25.01	0.00508965358584988\\
26.01	0.00508965197102632\\
27.01	0.00508965032299697\\
28.01	0.00508964864108255\\
29.01	0.00508964692459043\\
30.01	0.00508964517281322\\
31.01	0.0050896433850298\\
32.01	0.00508964156050412\\
33.01	0.00508963969848547\\
34.01	0.00508963779820734\\
35.01	0.00508963585888803\\
36.01	0.00508963387973035\\
37.01	0.00508963185992034\\
38.01	0.00508962979862791\\
39.01	0.00508962769500595\\
40.01	0.00508962554819032\\
41.01	0.00508962335729921\\
42.01	0.00508962112143281\\
43.01	0.00508961883967355\\
44.01	0.00508961651108512\\
45.01	0.00508961413471206\\
46.01	0.00508961170957968\\
47.01	0.00508960923469374\\
48.01	0.0050896067090395\\
49.01	0.005089604131582\\
50.01	0.00508960150126519\\
51.01	0.00508959881701179\\
52.01	0.00508959607772295\\
53.01	0.00508959328227709\\
54.01	0.00508959042953037\\
55.01	0.00508958751831546\\
56.01	0.00508958454744193\\
57.01	0.0050895815156951\\
58.01	0.00508957842183554\\
59.01	0.00508957526459921\\
60.01	0.00508957204269618\\
61.01	0.00508956875481088\\
62.01	0.00508956539960106\\
63.01	0.00508956197569726\\
64.01	0.00508955848170247\\
65.01	0.00508955491619142\\
66.01	0.00508955127771036\\
67.01	0.00508954756477626\\
68.01	0.00508954377587622\\
69.01	0.00508953990946661\\
70.01	0.00508953596397271\\
71.01	0.00508953193778857\\
72.01	0.00508952782927572\\
73.01	0.00508952363676241\\
74.01	0.0050895193585434\\
75.01	0.0050895149928794\\
76.01	0.00508951053799624\\
77.01	0.00508950599208363\\
78.01	0.00508950135329541\\
79.01	0.00508949661974818\\
80.01	0.00508949178952017\\
81.01	0.00508948686065187\\
82.01	0.00508948183114403\\
83.01	0.00508947669895704\\
84.01	0.00508947146201124\\
85.01	0.00508946611818433\\
86.01	0.00508946066531193\\
87.01	0.00508945510118605\\
88.01	0.00508944942355465\\
89.01	0.00508944363012057\\
90.01	0.00508943771854086\\
91.01	0.00508943168642519\\
92.01	0.00508942553133553\\
93.01	0.00508941925078526\\
94.01	0.005089412842238\\
95.01	0.00508940630310644\\
96.01	0.00508939963075168\\
97.01	0.0050893928224821\\
98.01	0.00508938587555247\\
99.01	0.00508937878716225\\
100.01	0.00508937155445535\\
101.01	0.0050893641745188\\
102.01	0.00508935664438125\\
103.01	0.0050893489610122\\
104.01	0.00508934112132043\\
105.01	0.00508933312215349\\
106.01	0.00508932496029615\\
107.01	0.00508931663246878\\
108.01	0.00508930813532626\\
109.01	0.00508929946545756\\
110.01	0.00508929061938363\\
111.01	0.00508928159355538\\
112.01	0.00508927238435405\\
113.01	0.00508926298808852\\
114.01	0.00508925340099421\\
115.01	0.00508924361923217\\
116.01	0.00508923363888714\\
117.01	0.00508922345596582\\
118.01	0.0050892130663955\\
119.01	0.00508920246602324\\
120.01	0.00508919165061311\\
121.01	0.00508918061584572\\
122.01	0.00508916935731579\\
123.01	0.00508915787053103\\
124.01	0.00508914615090978\\
125.01	0.00508913419377994\\
126.01	0.00508912199437698\\
127.01	0.00508910954784219\\
128.01	0.0050890968492203\\
129.01	0.00508908389345868\\
130.01	0.00508907067540444\\
131.01	0.00508905718980313\\
132.01	0.00508904343129645\\
133.01	0.00508902939442083\\
134.01	0.00508901507360411\\
135.01	0.00508900046316471\\
136.01	0.00508898555730928\\
137.01	0.00508897035012952\\
138.01	0.00508895483560167\\
139.01	0.00508893900758307\\
140.01	0.00508892285981006\\
141.01	0.00508890638589551\\
142.01	0.00508888957932736\\
143.01	0.00508887243346507\\
144.01	0.00508885494153837\\
145.01	0.00508883709664292\\
146.01	0.00508881889173978\\
147.01	0.00508880031965169\\
148.01	0.00508878137306073\\
149.01	0.00508876204450545\\
150.01	0.00508874232637809\\
151.01	0.00508872221092231\\
152.01	0.00508870169022961\\
153.01	0.00508868075623708\\
154.01	0.00508865940072392\\
155.01	0.00508863761530899\\
156.01	0.00508861539144725\\
157.01	0.00508859272042685\\
158.01	0.00508856959336592\\
159.01	0.00508854600120961\\
160.01	0.00508852193472651\\
161.01	0.00508849738450504\\
162.01	0.00508847234095076\\
163.01	0.00508844679428235\\
164.01	0.00508842073452835\\
165.01	0.00508839415152334\\
166.01	0.00508836703490458\\
167.01	0.00508833937410815\\
168.01	0.00508831115836474\\
169.01	0.00508828237669681\\
170.01	0.00508825301791326\\
171.01	0.00508822307060691\\
172.01	0.00508819252314936\\
173.01	0.00508816136368744\\
174.01	0.00508812958013865\\
175.01	0.00508809716018666\\
176.01	0.00508806409127808\\
177.01	0.00508803036061665\\
178.01	0.00508799595515917\\
179.01	0.00508796086161161\\
180.01	0.00508792506642342\\
181.01	0.00508788855578339\\
182.01	0.00508785131561455\\
183.01	0.00508781333156922\\
184.01	0.00508777458902433\\
185.01	0.00508773507307595\\
186.01	0.00508769476853422\\
187.01	0.00508765365991789\\
188.01	0.00508761173144972\\
189.01	0.00508756896704989\\
190.01	0.00508752535033114\\
191.01	0.0050874808645932\\
192.01	0.00508743549281676\\
193.01	0.00508738921765766\\
194.01	0.00508734202144088\\
195.01	0.00508729388615485\\
196.01	0.00508724479344496\\
197.01	0.00508719472460746\\
198.01	0.00508714366058295\\
199.01	0.00508709158195026\\
200.01	0.00508703846891954\\
201.01	0.00508698430132592\\
202.01	0.0050869290586222\\
203.01	0.00508687271987262\\
204.01	0.00508681526374536\\
205.01	0.00508675666850569\\
206.01	0.00508669691200885\\
207.01	0.0050866359716921\\
208.01	0.00508657382456766\\
209.01	0.0050865104472155\\
210.01	0.0050864458157747\\
211.01	0.00508637990593653\\
212.01	0.00508631269293606\\
213.01	0.00508624415154391\\
214.01	0.0050861742560585\\
215.01	0.00508610298029722\\
216.01	0.00508603029758829\\
217.01	0.00508595618076212\\
218.01	0.00508588060214226\\
219.01	0.00508580353353721\\
220.01	0.00508572494623056\\
221.01	0.00508564481097246\\
222.01	0.00508556309797024\\
223.01	0.00508547977687867\\
224.01	0.00508539481679076\\
225.01	0.00508530818622798\\
226.01	0.00508521985313024\\
227.01	0.00508512978484633\\
228.01	0.00508503794812291\\
229.01	0.00508494430909511\\
230.01	0.00508484883327607\\
231.01	0.00508475148554574\\
232.01	0.00508465223014012\\
233.01	0.00508455103064164\\
234.01	0.00508444784996679\\
235.01	0.00508434265035488\\
236.01	0.00508423539335794\\
237.01	0.00508412603982814\\
238.01	0.00508401454990652\\
239.01	0.00508390088301113\\
240.01	0.00508378499782545\\
241.01	0.00508366685228511\\
242.01	0.00508354640356693\\
243.01	0.00508342360807613\\
244.01	0.00508329842143316\\
245.01	0.00508317079846144\\
246.01	0.00508304069317468\\
247.01	0.00508290805876296\\
248.01	0.00508277284758022\\
249.01	0.00508263501113062\\
250.01	0.00508249450005546\\
251.01	0.00508235126411846\\
252.01	0.00508220525219239\\
253.01	0.00508205641224551\\
254.01	0.00508190469132674\\
255.01	0.00508175003555108\\
256.01	0.00508159239008613\\
257.01	0.00508143169913625\\
258.01	0.00508126790592883\\
259.01	0.005081100952698\\
260.01	0.0050809307806709\\
261.01	0.0050807573300511\\
262.01	0.00508058054000401\\
263.01	0.00508040034864073\\
264.01	0.00508021669300259\\
265.01	0.00508002950904491\\
266.01	0.00507983873162133\\
267.01	0.00507964429446718\\
268.01	0.00507944613018312\\
269.01	0.00507924417021882\\
270.01	0.00507903834485659\\
271.01	0.00507882858319344\\
272.01	0.00507861481312517\\
273.01	0.00507839696132857\\
274.01	0.00507817495324472\\
275.01	0.00507794871306106\\
276.01	0.00507771816369434\\
277.01	0.00507748322677191\\
278.01	0.00507724382261517\\
279.01	0.0050769998702209\\
280.01	0.00507675128724319\\
281.01	0.00507649798997494\\
282.01	0.00507623989332975\\
283.01	0.0050759769108233\\
284.01	0.00507570895455487\\
285.01	0.00507543593518795\\
286.01	0.00507515776193144\\
287.01	0.00507487434252075\\
288.01	0.0050745855831988\\
289.01	0.00507429138869547\\
290.01	0.00507399166220907\\
291.01	0.00507368630538611\\
292.01	0.00507337521830149\\
293.01	0.0050730582994385\\
294.01	0.00507273544566812\\
295.01	0.00507240655222904\\
296.01	0.00507207151270666\\
297.01	0.00507173021901234\\
298.01	0.0050713825613626\\
299.01	0.00507102842825736\\
300.01	0.00507066770645871\\
301.01	0.00507030028096876\\
302.01	0.00506992603500801\\
303.01	0.00506954484999268\\
304.01	0.00506915660551168\\
305.01	0.00506876117930481\\
306.01	0.0050683584472374\\
307.01	0.00506794828327829\\
308.01	0.00506753055947467\\
309.01	0.00506710514592742\\
310.01	0.00506667191076681\\
311.01	0.00506623072012522\\
312.01	0.00506578143811225\\
313.01	0.00506532392678706\\
314.01	0.00506485804613066\\
315.01	0.00506438365401823\\
316.01	0.00506390060618953\\
317.01	0.00506340875621869\\
318.01	0.00506290795548339\\
319.01	0.0050623980531334\\
320.01	0.0050618788960569\\
321.01	0.0050613503288471\\
322.01	0.00506081219376555\\
323.01	0.00506026433070663\\
324.01	0.00505970657715841\\
325.01	0.00505913876816366\\
326.01	0.00505856073627759\\
327.01	0.00505797231152495\\
328.01	0.00505737332135447\\
329.01	0.00505676359059195\\
330.01	0.00505614294139056\\
331.01	0.00505551119317966\\
332.01	0.00505486816260919\\
333.01	0.00505421366349356\\
334.01	0.0050535475067512\\
335.01	0.00505286950034228\\
336.01	0.00505217944920215\\
337.01	0.00505147715517194\\
338.01	0.00505076241692552\\
339.01	0.00505003502989245\\
340.01	0.00504929478617756\\
341.01	0.00504854147447524\\
342.01	0.00504777487998067\\
343.01	0.00504699478429536\\
344.01	0.00504620096532795\\
345.01	0.00504539319719129\\
346.01	0.00504457125009211\\
347.01	0.00504373489021737\\
348.01	0.00504288387961356\\
349.01	0.00504201797606057\\
350.01	0.00504113693294076\\
351.01	0.00504024049909947\\
352.01	0.00503932841870174\\
353.01	0.00503840043108143\\
354.01	0.00503745627058352\\
355.01	0.00503649566640135\\
356.01	0.00503551834240592\\
357.01	0.00503452401696963\\
358.01	0.00503351240278326\\
359.01	0.00503248320666641\\
360.01	0.00503143612937219\\
361.01	0.00503037086538597\\
362.01	0.00502928710271854\\
363.01	0.00502818452269397\\
364.01	0.00502706279973244\\
365.01	0.00502592160112882\\
366.01	0.00502476058682842\\
367.01	0.00502357940919909\\
368.01	0.00502237771280032\\
369.01	0.00502115513415367\\
370.01	0.00501991130150968\\
371.01	0.00501864583461723\\
372.01	0.00501735834449525\\
373.01	0.00501604843320617\\
374.01	0.00501471569363434\\
375.01	0.00501335970927069\\
376.01	0.0050119800540043\\
377.01	0.0050105762919218\\
378.01	0.00500914797711773\\
379.01	0.00500769465351591\\
380.01	0.00500621585470338\\
381.01	0.00500471110377882\\
382.01	0.00500317991321475\\
383.01	0.00500162178473713\\
384.01	0.00500003620922055\\
385.01	0.00499842266659988\\
386.01	0.00499678062579989\\
387.01	0.004995109544679\\
388.01	0.00499340886998947\\
389.01	0.00499167803735109\\
390.01	0.00498991647123458\\
391.01	0.00498812358495372\\
392.01	0.00498629878066159\\
393.01	0.00498444144934696\\
394.01	0.00498255097082289\\
395.01	0.00498062671370551\\
396.01	0.00497866803537291\\
397.01	0.00497667428189747\\
398.01	0.00497464478794445\\
399.01	0.00497257887662799\\
400.01	0.00497047585931557\\
401.01	0.00496833503537484\\
402.01	0.00496615569185451\\
403.01	0.00496393710309253\\
404.01	0.00496167853024835\\
405.01	0.00495937922075627\\
406.01	0.00495703840769956\\
407.01	0.00495465530910754\\
408.01	0.00495222912718346\\
409.01	0.00494975904747006\\
410.01	0.00494724423796623\\
411.01	0.00494468384820806\\
412.01	0.00494207700833176\\
413.01	0.00493942282813428\\
414.01	0.0049367203961457\\
415.01	0.00493396877872884\\
416.01	0.00493116701921221\\
417.01	0.00492831413705652\\
418.01	0.00492540912705394\\
419.01	0.00492245095854786\\
420.01	0.00491943857466022\\
421.01	0.00491637089151961\\
422.01	0.00491324679748207\\
423.01	0.00491006515234761\\
424.01	0.00490682478657479\\
425.01	0.00490352450049337\\
426.01	0.00490016306351852\\
427.01	0.00489673921336825\\
428.01	0.00489325165528674\\
429.01	0.00488969906127556\\
430.01	0.00488608006933446\\
431.01	0.00488239328271397\\
432.01	0.00487863726918218\\
433.01	0.00487481056030704\\
434.01	0.00487091165075614\\
435.01	0.00486693899761418\\
436.01	0.00486289101971994\\
437.01	0.00485876609702326\\
438.01	0.00485456256996054\\
439.01	0.00485027873885053\\
440.01	0.00484591286330686\\
441.01	0.00484146316166711\\
442.01	0.00483692781043501\\
443.01	0.00483230494373362\\
444.01	0.00482759265276427\\
445.01	0.004822788985267\\
446.01	0.0048178919449794\\
447.01	0.00481289949108397\\
448.01	0.00480780953764226\\
449.01	0.00480261995300564\\
450.01	0.00479732855919909\\
451.01	0.00479193313126877\\
452.01	0.00478643139658867\\
453.01	0.00478082103412083\\
454.01	0.0047750996736238\\
455.01	0.00476926489480607\\
456.01	0.004763314226423\\
457.01	0.00475724514531626\\
458.01	0.00475105507539817\\
459.01	0.00474474138658562\\
460.01	0.00473830139368701\\
461.01	0.00473173235525485\\
462.01	0.00472503147240865\\
463.01	0.00471819588764355\\
464.01	0.00471122268363353\\
465.01	0.00470410888204121\\
466.01	0.00469685144234548\\
467.01	0.00468944726069261\\
468.01	0.00468189316877687\\
469.01	0.00467418593275002\\
470.01	0.00466632225215585\\
471.01	0.00465829875888062\\
472.01	0.00465011201610793\\
473.01	0.00464175851726228\\
474.01	0.0046332346849264\\
475.01	0.00462453686972014\\
476.01	0.00461566134912937\\
477.01	0.00460660432628032\\
478.01	0.00459736192865624\\
479.01	0.0045879302067535\\
480.01	0.00457830513267719\\
481.01	0.00456848259867723\\
482.01	0.0045584584156257\\
483.01	0.00454822831143751\\
484.01	0.00453778792943787\\
485.01	0.00452713282668092\\
486.01	0.00451625847222222\\
487.01	0.00450516024535199\\
488.01	0.00449383343379027\\
489.01	0.00448227323184985\\
490.01	0.0044704747385684\\
491.01	0.00445843295581342\\
492.01	0.00444614278635746\\
493.01	0.00443359903192642\\
494.01	0.00442079639121603\\
495.01	0.00440772945787658\\
496.01	0.00439439271846087\\
497.01	0.00438078055033382\\
498.01	0.00436688721954141\\
499.01	0.00435270687863867\\
500.01	0.00433823356447654\\
501.01	0.00432346119595061\\
502.01	0.00430838357171449\\
503.01	0.0042929943678623\\
504.01	0.00427728713558209\\
505.01	0.00426125529878641\\
506.01	0.00424489215172238\\
507.01	0.00422819085656428\\
508.01	0.00421114444099322\\
509.01	0.00419374579576618\\
510.01	0.0041759876722773\\
511.01	0.00415786268011639\\
512.01	0.00413936328462638\\
513.01	0.00412048180446544\\
514.01	0.00410121040917822\\
515.01	0.00408154111678125\\
516.01	0.00406146579137078\\
517.01	0.0040409761407585\\
518.01	0.00402006371414559\\
519.01	0.0039987198998435\\
520.01	0.00397693592305065\\
521.01	0.00395470284369803\\
522.01	0.00393201155437433\\
523.01	0.00390885277834494\\
524.01	0.00388521706767742\\
525.01	0.00386109480149263\\
526.01	0.00383647618435695\\
527.01	0.00381135124483494\\
528.01	0.00378570983422737\\
529.01	0.00375954162551551\\
530.01	0.00373283611253975\\
531.01	0.00370558260944227\\
532.01	0.0036777702504053\\
533.01	0.0036493879897215\\
534.01	0.00362042460223506\\
535.01	0.00359086868419784\\
536.01	0.00356070865458678\\
537.01	0.0035299327569359\\
538.01	0.00349852906174065\\
539.01	0.00346648546949845\\
540.01	0.00343378971445346\\
541.01	0.00340042936912447\\
542.01	0.00336639184969894\\
543.01	0.00333166442238612\\
544.01	0.00329623421082995\\
545.01	0.00326008820469374\\
546.01	0.00322321326953876\\
547.01	0.00318559615812999\\
548.01	0.00314722352331424\\
549.01	0.00310808193263235\\
550.01	0.00306815788483875\\
551.01	0.00302743782851966\\
552.01	0.00298590818301772\\
553.01	0.00294355536188978\\
554.01	0.00290036579914364\\
555.01	0.00285632597852142\\
556.01	0.00281142246612037\\
557.01	0.00276564194666362\\
558.01	0.00271897126376054\\
559.01	0.00267139746452063\\
560.01	0.00262290784891359\\
561.01	0.0025734900242925\\
562.01	0.00252313196552769\\
563.01	0.00247182208122277\\
564.01	0.00241954928651093\\
565.01	0.00236630308295206\\
566.01	0.00231207364607109\\
567.01	0.00225685192108934\\
568.01	0.00220062972740853\\
569.01	0.0021433998724014\\
570.01	0.00208515627504275\\
571.01	0.0020258940998783\\
572.01	0.00196560990176544\\
573.01	0.00190430178172618\\
574.01	0.00184196955411858\\
575.01	0.00177861492514706\\
576.01	0.00171424168248111\\
577.01	0.00164885589541889\\
578.01	0.00158246612459554\\
579.01	0.00151508363966999\\
580.01	0.00144672264269536\\
581.01	0.00137740049394915\\
582.01	0.0013071379358148\\
583.01	0.00123595930881451\\
584.01	0.00116389275200805\\
585.01	0.00109097037760846\\
586.01	0.00101722840670688\\
587.01	0.000942707249300885\\
588.01	0.000867451507214028\\
589.01	0.000791509872763252\\
590.01	0.000714934888910885\\
591.01	0.000637782527808906\\
592.01	0.000560111533702402\\
593.01	0.000481982462622189\\
594.01	0.000403456334556148\\
595.01	0.000324592793109807\\
596.01	0.000245447642136467\\
597.01	0.00016636800413272\\
598.01	9.17366970063782e-05\\
599.01	2.94669364271135e-05\\
599.02	2.89574269585688e-05\\
599.03	2.84509530852801e-05\\
599.04	2.79475443462508e-05\\
599.05	2.7447230571279e-05\\
599.06	2.69500418838293e-05\\
599.07	2.64560087039432e-05\\
599.08	2.59651617511691e-05\\
599.09	2.54775320475218e-05\\
599.1	2.49931509204854e-05\\
599.11	2.45120500060175e-05\\
599.12	2.40342612516185e-05\\
599.13	2.35598169193996e-05\\
599.14	2.30887495892024e-05\\
599.15	2.26210921617405e-05\\
599.16	2.215687786177e-05\\
599.17	2.16961402412976e-05\\
599.18	2.12389131828191e-05\\
599.19	2.07852309025824e-05\\
599.2	2.03351279538938e-05\\
599.21	1.98886392304542e-05\\
599.22	1.94457999697188e-05\\
599.23	1.90066457563046e-05\\
599.24	1.85712125254194e-05\\
599.25	1.81395365663351e-05\\
599.26	1.77116564095865e-05\\
599.27	1.72876122135658e-05\\
599.28	1.68674445361946e-05\\
599.29	1.64511943388859e-05\\
599.3	1.60389029905273e-05\\
599.31	1.56306122715087e-05\\
599.32	1.52263643777902e-05\\
599.33	1.48262019250105e-05\\
599.34	1.44301679526268e-05\\
599.35	1.40383059281154e-05\\
599.36	1.36506597511916e-05\\
599.37	1.32672737580847e-05\\
599.38	1.28881927258552e-05\\
599.39	1.25134618767404e-05\\
599.4	1.21431268825696e-05\\
599.41	1.17772338691924e-05\\
599.42	1.14158294209719e-05\\
599.43	1.10589605853174e-05\\
599.44	1.07066748772523e-05\\
599.45	1.03590202840433e-05\\
599.46	1.00160452698606e-05\\
599.47	9.67779878049101e-06\\
599.48	9.34433024810284e-06\\
599.49	9.01568959604283e-06\\
599.5	8.69192724369146e-06\\
599.51	8.3730941113757e-06\\
599.52	8.05924162529219e-06\\
599.53	7.75042172253965e-06\\
599.54	7.4466868561357e-06\\
599.55	7.14809000013257e-06\\
599.56	6.85468465475535e-06\\
599.57	6.56652485161134e-06\\
599.58	6.2836651589307e-06\\
599.59	6.00616068686249e-06\\
599.6	5.73406709284546e-06\\
599.61	5.46744058700296e-06\\
599.62	5.20633793760217e-06\\
599.63	4.95081647657741e-06\\
599.64	4.70093410508653e-06\\
599.65	4.45674929914347e-06\\
599.66	4.21832111529262e-06\\
599.67	3.98570919634376e-06\\
599.68	3.75897377716435e-06\\
599.69	3.53817569053241e-06\\
599.7	3.32337637303295e-06\\
599.71	3.11463787103054e-06\\
599.72	2.91202284668363e-06\\
599.73	2.71559458404555e-06\\
599.74	2.52541699518292e-06\\
599.75	2.34155462640155e-06\\
599.76	2.16407266449489e-06\\
599.77	1.99303694307928e-06\\
599.78	1.82851394897598e-06\\
599.79	1.67057082867475e-06\\
599.8	1.51927539483211e-06\\
599.81	1.37469613287027e-06\\
599.82	1.23690220760544e-06\\
599.83	1.10596346997172e-06\\
599.84	9.81950463784659e-07\\
599.85	8.64934432591793e-07\\
599.86	7.54987326588921e-07\\
599.87	6.52181809583305e-07\\
599.88	5.56591266062667e-07\\
599.89	4.68289808295413e-07\\
599.9	3.87352283521061e-07\\
599.91	3.13854281218731e-07\\
599.92	2.47872140425945e-07\\
599.93	1.8948295714763e-07\\
599.94	1.38764591834512e-07\\
599.95	9.57956769204876e-08\\
599.96	6.06556244606149e-08\\
599.97	3.34246338194039e-08\\
599.98	1.4183699454523e-08\\
599.99	3.014618757749e-09\\
600	0\\
};
\addplot [color=black,solid,forget plot]
  table[row sep=crcr]{%
0.01	0.00507509332097565\\
1.01	0.0050750923628945\\
2.01	0.00507509138515603\\
3.01	0.00507509038735957\\
4.01	0.00507508936909632\\
5.01	0.00507508832994861\\
6.01	0.00507508726949108\\
7.01	0.00507508618728923\\
8.01	0.00507508508290031\\
9.01	0.00507508395587187\\
10.01	0.00507508280574299\\
11.01	0.00507508163204302\\
12.01	0.00507508043429207\\
13.01	0.0050750792120006\\
14.01	0.00507507796466875\\
15.01	0.00507507669178709\\
16.01	0.00507507539283559\\
17.01	0.00507507406728384\\
18.01	0.00507507271459088\\
19.01	0.00507507133420444\\
20.01	0.00507506992556135\\
21.01	0.00507506848808715\\
22.01	0.00507506702119588\\
23.01	0.0050750655242895\\
24.01	0.00507506399675818\\
25.01	0.00507506243797951\\
26.01	0.00507506084731916\\
27.01	0.00507505922412902\\
28.01	0.0050750575677491\\
29.01	0.00507505587750485\\
30.01	0.00507505415270949\\
31.01	0.0050750523926618\\
32.01	0.0050750505966461\\
33.01	0.00507504876393276\\
34.01	0.00507504689377778\\
35.01	0.0050750449854215\\
36.01	0.00507504303808951\\
37.01	0.00507504105099175\\
38.01	0.00507503902332194\\
39.01	0.00507503695425819\\
40.01	0.00507503484296174\\
41.01	0.00507503268857721\\
42.01	0.00507503049023215\\
43.01	0.00507502824703633\\
44.01	0.00507502595808177\\
45.01	0.0050750236224424\\
46.01	0.00507502123917342\\
47.01	0.00507501880731144\\
48.01	0.00507501632587355\\
49.01	0.00507501379385709\\
50.01	0.0050750112102396\\
51.01	0.00507500857397792\\
52.01	0.00507500588400792\\
53.01	0.00507500313924481\\
54.01	0.00507500033858133\\
55.01	0.00507499748088886\\
56.01	0.00507499456501574\\
57.01	0.00507499158978734\\
58.01	0.00507498855400582\\
59.01	0.00507498545644934\\
60.01	0.00507498229587151\\
61.01	0.0050749790710015\\
62.01	0.0050749757805424\\
63.01	0.00507497242317229\\
64.01	0.00507496899754251\\
65.01	0.00507496550227766\\
66.01	0.00507496193597477\\
67.01	0.00507495829720301\\
68.01	0.00507495458450303\\
69.01	0.00507495079638682\\
70.01	0.00507494693133646\\
71.01	0.00507494298780371\\
72.01	0.00507493896420967\\
73.01	0.00507493485894434\\
74.01	0.00507493067036567\\
75.01	0.00507492639679894\\
76.01	0.0050749220365359\\
77.01	0.00507491758783472\\
78.01	0.00507491304891886\\
79.01	0.00507490841797665\\
80.01	0.00507490369316095\\
81.01	0.00507489887258731\\
82.01	0.00507489395433428\\
83.01	0.00507488893644271\\
84.01	0.00507488381691369\\
85.01	0.00507487859370976\\
86.01	0.00507487326475279\\
87.01	0.00507486782792364\\
88.01	0.00507486228106111\\
89.01	0.00507485662196151\\
90.01	0.00507485084837759\\
91.01	0.00507484495801778\\
92.01	0.00507483894854511\\
93.01	0.00507483281757652\\
94.01	0.00507482656268228\\
95.01	0.00507482018138451\\
96.01	0.00507481367115676\\
97.01	0.00507480702942243\\
98.01	0.00507480025355429\\
99.01	0.005074793340874\\
100.01	0.00507478628864997\\
101.01	0.00507477909409691\\
102.01	0.00507477175437477\\
103.01	0.00507476426658802\\
104.01	0.00507475662778401\\
105.01	0.00507474883495236\\
106.01	0.0050747408850232\\
107.01	0.00507473277486687\\
108.01	0.00507472450129263\\
109.01	0.00507471606104641\\
110.01	0.00507470745081088\\
111.01	0.00507469866720385\\
112.01	0.00507468970677664\\
113.01	0.00507468056601352\\
114.01	0.00507467124132977\\
115.01	0.00507466172907038\\
116.01	0.00507465202550914\\
117.01	0.0050746421268475\\
118.01	0.00507463202921244\\
119.01	0.00507462172865515\\
120.01	0.00507461122115039\\
121.01	0.00507460050259413\\
122.01	0.00507458956880223\\
123.01	0.00507457841550933\\
124.01	0.00507456703836704\\
125.01	0.00507455543294281\\
126.01	0.00507454359471697\\
127.01	0.00507453151908264\\
128.01	0.00507451920134384\\
129.01	0.00507450663671286\\
130.01	0.0050744938203094\\
131.01	0.00507448074715839\\
132.01	0.00507446741218821\\
133.01	0.0050744538102292\\
134.01	0.00507443993601179\\
135.01	0.00507442578416418\\
136.01	0.00507441134921064\\
137.01	0.00507439662557017\\
138.01	0.00507438160755303\\
139.01	0.00507436628936035\\
140.01	0.00507435066508125\\
141.01	0.00507433472869107\\
142.01	0.00507431847404841\\
143.01	0.00507430189489389\\
144.01	0.00507428498484734\\
145.01	0.00507426773740678\\
146.01	0.00507425014594418\\
147.01	0.00507423220370468\\
148.01	0.00507421390380356\\
149.01	0.00507419523922374\\
150.01	0.00507417620281398\\
151.01	0.00507415678728538\\
152.01	0.00507413698520982\\
153.01	0.00507411678901688\\
154.01	0.00507409619099173\\
155.01	0.005074075183271\\
156.01	0.00507405375784154\\
157.01	0.0050740319065375\\
158.01	0.00507400962103689\\
159.01	0.00507398689285877\\
160.01	0.00507396371336092\\
161.01	0.0050739400737363\\
162.01	0.00507391596501045\\
163.01	0.00507389137803769\\
164.01	0.00507386630349933\\
165.01	0.00507384073189902\\
166.01	0.00507381465356017\\
167.01	0.00507378805862289\\
168.01	0.00507376093704049\\
169.01	0.00507373327857531\\
170.01	0.00507370507279723\\
171.01	0.00507367630907735\\
172.01	0.0050736469765867\\
173.01	0.00507361706429175\\
174.01	0.00507358656095062\\
175.01	0.00507355545510958\\
176.01	0.00507352373509888\\
177.01	0.00507349138902891\\
178.01	0.00507345840478705\\
179.01	0.00507342477003229\\
180.01	0.00507339047219217\\
181.01	0.00507335549845812\\
182.01	0.00507331983578156\\
183.01	0.00507328347086898\\
184.01	0.00507324639017846\\
185.01	0.00507320857991437\\
186.01	0.00507317002602343\\
187.01	0.00507313071418995\\
188.01	0.00507309062983087\\
189.01	0.00507304975809168\\
190.01	0.00507300808384057\\
191.01	0.00507296559166482\\
192.01	0.00507292226586437\\
193.01	0.00507287809044797\\
194.01	0.00507283304912774\\
195.01	0.00507278712531304\\
196.01	0.00507274030210645\\
197.01	0.00507269256229743\\
198.01	0.00507264388835758\\
199.01	0.00507259426243429\\
200.01	0.00507254366634564\\
201.01	0.00507249208157421\\
202.01	0.00507243948926193\\
203.01	0.00507238587020346\\
204.01	0.00507233120484011\\
205.01	0.00507227547325413\\
206.01	0.00507221865516198\\
207.01	0.00507216072990919\\
208.01	0.00507210167646214\\
209.01	0.00507204147340324\\
210.01	0.00507198009892319\\
211.01	0.00507191753081424\\
212.01	0.00507185374646416\\
213.01	0.00507178872284889\\
214.01	0.00507172243652521\\
215.01	0.00507165486362391\\
216.01	0.00507158597984246\\
217.01	0.0050715157604375\\
218.01	0.00507144418021761\\
219.01	0.00507137121353524\\
220.01	0.0050712968342801\\
221.01	0.00507122101586983\\
222.01	0.00507114373124269\\
223.01	0.00507106495284985\\
224.01	0.005070984652647\\
225.01	0.00507090280208587\\
226.01	0.00507081937210558\\
227.01	0.00507073433312447\\
228.01	0.00507064765503149\\
229.01	0.00507055930717735\\
230.01	0.00507046925836458\\
231.01	0.00507037747684004\\
232.01	0.00507028393028489\\
233.01	0.00507018858580489\\
234.01	0.00507009140992138\\
235.01	0.00506999236856199\\
236.01	0.00506989142705026\\
237.01	0.00506978855009589\\
238.01	0.00506968370178452\\
239.01	0.00506957684556842\\
240.01	0.00506946794425484\\
241.01	0.00506935695999656\\
242.01	0.00506924385428079\\
243.01	0.00506912858791837\\
244.01	0.0050690111210332\\
245.01	0.00506889141305054\\
246.01	0.00506876942268595\\
247.01	0.00506864510793435\\
248.01	0.005068518426058\\
249.01	0.00506838933357479\\
250.01	0.00506825778624604\\
251.01	0.00506812373906494\\
252.01	0.00506798714624405\\
253.01	0.00506784796120308\\
254.01	0.00506770613655556\\
255.01	0.00506756162409706\\
256.01	0.00506741437479153\\
257.01	0.00506726433875824\\
258.01	0.00506711146525844\\
259.01	0.00506695570268196\\
260.01	0.00506679699853297\\
261.01	0.00506663529941704\\
262.01	0.00506647055102603\\
263.01	0.00506630269812392\\
264.01	0.00506613168453275\\
265.01	0.00506595745311698\\
266.01	0.00506577994576957\\
267.01	0.00506559910339558\\
268.01	0.00506541486589755\\
269.01	0.00506522717215955\\
270.01	0.00506503596003048\\
271.01	0.00506484116630971\\
272.01	0.00506464272672869\\
273.01	0.00506444057593556\\
274.01	0.00506423464747703\\
275.01	0.00506402487378258\\
276.01	0.00506381118614522\\
277.01	0.00506359351470526\\
278.01	0.00506337178843148\\
279.01	0.00506314593510227\\
280.01	0.00506291588128722\\
281.01	0.00506268155232887\\
282.01	0.00506244287232206\\
283.01	0.00506219976409476\\
284.01	0.00506195214918757\\
285.01	0.00506169994783382\\
286.01	0.00506144307893834\\
287.01	0.00506118146005622\\
288.01	0.0050609150073707\\
289.01	0.00506064363567161\\
290.01	0.00506036725833246\\
291.01	0.00506008578728751\\
292.01	0.00505979913300801\\
293.01	0.00505950720447802\\
294.01	0.00505920990917059\\
295.01	0.00505890715302186\\
296.01	0.00505859884040531\\
297.01	0.00505828487410596\\
298.01	0.00505796515529287\\
299.01	0.00505763958349196\\
300.01	0.00505730805655781\\
301.01	0.00505697047064422\\
302.01	0.00505662672017483\\
303.01	0.00505627669781245\\
304.01	0.00505592029442809\\
305.01	0.00505555739906764\\
306.01	0.00505518789892074\\
307.01	0.00505481167928527\\
308.01	0.00505442862353273\\
309.01	0.00505403861307289\\
310.01	0.00505364152731531\\
311.01	0.00505323724363369\\
312.01	0.00505282563732405\\
313.01	0.00505240658156526\\
314.01	0.00505197994737795\\
315.01	0.00505154560358038\\
316.01	0.0050511034167442\\
317.01	0.00505065325114879\\
318.01	0.00505019496873414\\
319.01	0.00504972842905117\\
320.01	0.00504925348921163\\
321.01	0.00504877000383523\\
322.01	0.00504827782499646\\
323.01	0.00504777680216768\\
324.01	0.00504726678216167\\
325.01	0.00504674760907182\\
326.01	0.00504621912421047\\
327.01	0.00504568116604355\\
328.01	0.00504513357012616\\
329.01	0.00504457616903229\\
330.01	0.00504400879228462\\
331.01	0.00504343126628027\\
332.01	0.00504284341421566\\
333.01	0.00504224505600726\\
334.01	0.00504163600820968\\
335.01	0.00504101608393246\\
336.01	0.00504038509275196\\
337.01	0.00503974284062201\\
338.01	0.00503908912978047\\
339.01	0.00503842375865283\\
340.01	0.00503774652175355\\
341.01	0.00503705720958282\\
342.01	0.00503635560852107\\
343.01	0.00503564150071973\\
344.01	0.00503491466398973\\
345.01	0.00503417487168396\\
346.01	0.00503342189258006\\
347.01	0.0050326554907563\\
348.01	0.00503187542546633\\
349.01	0.00503108145101045\\
350.01	0.00503027331660137\\
351.01	0.0050294507662298\\
352.01	0.00502861353852478\\
353.01	0.00502776136661102\\
354.01	0.00502689397796417\\
355.01	0.00502601109426194\\
356.01	0.00502511243123333\\
357.01	0.00502419769850433\\
358.01	0.00502326659944203\\
359.01	0.00502231883099567\\
360.01	0.00502135408353597\\
361.01	0.00502037204069237\\
362.01	0.00501937237918879\\
363.01	0.00501835476867738\\
364.01	0.00501731887157184\\
365.01	0.00501626434288031\\
366.01	0.00501519083003552\\
367.01	0.00501409797272657\\
368.01	0.00501298540273072\\
369.01	0.00501185274374292\\
370.01	0.00501069961120957\\
371.01	0.0050095256121601\\
372.01	0.00500833034503979\\
373.01	0.00500711339954497\\
374.01	0.00500587435645806\\
375.01	0.00500461278748491\\
376.01	0.00500332825509143\\
377.01	0.00500202031234377\\
378.01	0.00500068850274819\\
379.01	0.00499933236009091\\
380.01	0.00499795140827909\\
381.01	0.00499654516118146\\
382.01	0.00499511312246765\\
383.01	0.00499365478544587\\
384.01	0.00499216963289657\\
385.01	0.00499065713690462\\
386.01	0.00498911675868314\\
387.01	0.00498754794839435\\
388.01	0.00498595014495849\\
389.01	0.0049843227758546\\
390.01	0.00498266525690864\\
391.01	0.00498097699207024\\
392.01	0.0049792573731698\\
393.01	0.00497750577966067\\
394.01	0.00497572157834173\\
395.01	0.0049739041230573\\
396.01	0.00497205275437444\\
397.01	0.00497016679923682\\
398.01	0.00496824557059137\\
399.01	0.0049662883669906\\
400.01	0.00496429447216763\\
401.01	0.00496226315458532\\
402.01	0.00496019366695966\\
403.01	0.0049580852457607\\
404.01	0.0049559371106904\\
405.01	0.00495374846414219\\
406.01	0.00495151849064386\\
407.01	0.00494924635628814\\
408.01	0.00494693120815443\\
409.01	0.00494457217372436\\
410.01	0.00494216836029669\\
411.01	0.0049397188544016\\
412.01	0.00493722272122053\\
413.01	0.00493467900400838\\
414.01	0.00493208672352417\\
415.01	0.0049294448774652\\
416.01	0.00492675243990482\\
417.01	0.00492400836073521\\
418.01	0.00492121156510803\\
419.01	0.00491836095287491\\
420.01	0.00491545539802683\\
421.01	0.00491249374812972\\
422.01	0.00490947482375937\\
423.01	0.00490639741793495\\
424.01	0.00490326029554925\\
425.01	0.00490006219279999\\
426.01	0.00489680181662181\\
427.01	0.0048934778441154\\
428.01	0.00489008892197846\\
429.01	0.00488663366593589\\
430.01	0.00488311066016935\\
431.01	0.00487951845674879\\
432.01	0.0048758555750596\\
433.01	0.00487212050123231\\
434.01	0.00486831168756726\\
435.01	0.00486442755195876\\
436.01	0.00486046647731391\\
437.01	0.00485642681096679\\
438.01	0.00485230686408699\\
439.01	0.00484810491107931\\
440.01	0.00484381918897482\\
441.01	0.00483944789680955\\
442.01	0.00483498919499099\\
443.01	0.00483044120464899\\
444.01	0.0048258020069697\\
445.01	0.00482106964251116\\
446.01	0.00481624211049723\\
447.01	0.00481131736808979\\
448.01	0.00480629332963679\\
449.01	0.00480116786589522\\
450.01	0.00479593880322778\\
451.01	0.00479060392277257\\
452.01	0.00478516095958617\\
453.01	0.00477960760175957\\
454.01	0.00477394148950731\\
455.01	0.00476816021423241\\
456.01	0.0047622613175661\\
457.01	0.00475624229038595\\
458.01	0.00475010057181329\\
459.01	0.00474383354819188\\
460.01	0.00473743855205174\\
461.01	0.00473091286105579\\
462.01	0.0047242536969366\\
463.01	0.00471745822441844\\
464.01	0.00471052355013044\\
465.01	0.00470344672150507\\
466.01	0.0046962247256663\\
467.01	0.00468885448830111\\
468.01	0.0046813328725142\\
469.01	0.00467365667766195\\
470.01	0.00466582263816132\\
471.01	0.00465782742227181\\
472.01	0.00464966763084477\\
473.01	0.0046413397960396\\
474.01	0.00463284038000248\\
475.01	0.00462416577350806\\
476.01	0.00461531229456245\\
477.01	0.00460627618696779\\
478.01	0.00459705361884733\\
479.01	0.00458764068113383\\
480.01	0.00457803338601985\\
481.01	0.00456822766537132\\
482.01	0.00455821936910527\\
483.01	0.00454800426353266\\
484.01	0.00453757802966714\\
485.01	0.00452693626150026\\
486.01	0.00451607446424404\\
487.01	0.00450498805254091\\
488.01	0.00449367234864202\\
489.01	0.00448212258055315\\
490.01	0.00447033388014911\\
491.01	0.00445830128125419\\
492.01	0.00444601971769131\\
493.01	0.00443348402129692\\
494.01	0.00442068891990277\\
495.01	0.00440762903528331\\
496.01	0.00439429888107034\\
497.01	0.0043806928606338\\
498.01	0.0043668052649308\\
499.01	0.00435263027032252\\
500.01	0.0043381619363625\\
501.01	0.00432339420355518\\
502.01	0.00430832089108876\\
503.01	0.00429293569454225\\
504.01	0.00427723218357085\\
505.01	0.0042612037995689\\
506.01	0.00424484385331487\\
507.01	0.00422814552259942\\
508.01	0.0042111018498394\\
509.01	0.004193705739681\\
510.01	0.00417594995659516\\
511.01	0.00415782712246773\\
512.01	0.00413932971419067\\
513.01	0.00412045006125712\\
514.01	0.0041011803433656\\
515.01	0.00408151258804065\\
516.01	0.00406143866827475\\
517.01	0.00404095030019947\\
518.01	0.00402003904079373\\
519.01	0.00399869628563757\\
520.01	0.00397691326672204\\
521.01	0.0039546810503256\\
522.01	0.00393199053496824\\
523.01	0.00390883244945772\\
524.01	0.00388519735104193\\
525.01	0.00386107562368351\\
526.01	0.00383645747647488\\
527.01	0.00381133294221396\\
528.01	0.00378569187616113\\
529.01	0.00375952395500273\\
530.01	0.00373281867604811\\
531.01	0.00370556535668826\\
532.01	0.00367775313414919\\
533.01	0.00364937096557665\\
534.01	0.00362040762848954\\
535.01	0.00359085172164746\\
536.01	0.00356069166637926\\
537.01	0.00352991570842443\\
538.01	0.00349851192034694\\
539.01	0.0034664682045829\\
540.01	0.00343377229719395\\
541.01	0.00340041177240193\\
542.01	0.00336637404798946\\
543.01	0.00333164639165858\\
544.01	0.00329621592845059\\
545.01	0.00326006964933632\\
546.01	0.00322319442110026\\
547.01	0.00318557699765106\\
548.01	0.0031472040329063\\
549.01	0.00310806209541009\\
550.01	0.00306813768485848\\
551.01	0.00302741725072389\\
552.01	0.0029858872131864\\
553.01	0.00294353398659807\\
554.01	0.00290034400572704\\
555.01	0.00285630375504905\\
556.01	0.00281139980137633\\
557.01	0.00276561883013688\\
558.01	0.00271894768564463\\
559.01	0.00267137341572391\\
560.01	0.00262288332107934\\
561.01	0.00257346500983139\\
562.01	0.00252310645766198\\
563.01	0.00247179607404328\\
564.01	0.00241952277504812\\
565.01	0.00236627606326161\\
566.01	0.0023120461153342\\
567.01	0.0022568238777289\\
568.01	0.00220060117122217\\
569.01	0.00214337080471112\\
570.01	0.00208512669886227\\
571.01	0.00202586402009816\\
572.01	0.00196557932535577\\
573.01	0.00190427071795702\\
574.01	0.00184193801479703\\
575.01	0.00177858292487004\\
576.01	0.00171420923890147\\
577.01	0.00164882302952239\\
578.01	0.00158243286098413\\
579.01	0.00151505000684564\\
580.01	0.00144668867333727\\
581.01	0.00137736622517327\\
582.01	0.00130710340940336\\
583.01	0.00123592457139783\\
584.01	0.00116385785517657\\
585.01	0.00109093537792825\\
586.01	0.0010171933656031\\
587.01	0.0009426722327669\\
588.01	0.000867416585292753\\
589.01	0.000791475118736057\\
590.01	0.000714900378115771\\
591.01	0.000637748335993645\\
592.01	0.000560077734801639\\
593.01	0.000481949125828402\\
594.01	0.000403423520535174\\
595.01	0.000324560549193819\\
596.01	0.000245415996312425\\
597.01	0.000166366053398785\\
598.01	9.17366970063782e-05\\
599.01	2.94669364271135e-05\\
599.02	2.89574269585705e-05\\
599.03	2.84509530852801e-05\\
599.04	2.79475443462508e-05\\
599.05	2.7447230571279e-05\\
599.06	2.69500418838293e-05\\
599.07	2.64560087039432e-05\\
599.08	2.59651617511691e-05\\
599.09	2.54775320475235e-05\\
599.1	2.49931509204836e-05\\
599.11	2.45120500060175e-05\\
599.12	2.40342612516167e-05\\
599.13	2.35598169193978e-05\\
599.14	2.30887495892024e-05\\
599.15	2.26210921617405e-05\\
599.16	2.215687786177e-05\\
599.17	2.16961402412976e-05\\
599.18	2.12389131828191e-05\\
599.19	2.07852309025824e-05\\
599.2	2.03351279538938e-05\\
599.21	1.98886392304524e-05\\
599.22	1.94457999697188e-05\\
599.23	1.90066457563063e-05\\
599.24	1.85712125254211e-05\\
599.25	1.81395365663334e-05\\
599.26	1.77116564095865e-05\\
599.27	1.72876122135658e-05\\
599.28	1.68674445361946e-05\\
599.29	1.64511943388859e-05\\
599.3	1.60389029905273e-05\\
599.31	1.56306122715087e-05\\
599.32	1.5226364377792e-05\\
599.33	1.48262019250105e-05\\
599.34	1.44301679526268e-05\\
599.35	1.40383059281154e-05\\
599.36	1.36506597511916e-05\\
599.37	1.32672737580847e-05\\
599.38	1.28881927258535e-05\\
599.39	1.25134618767404e-05\\
599.4	1.21431268825696e-05\\
599.41	1.17772338691924e-05\\
599.42	1.14158294209736e-05\\
599.43	1.10589605853174e-05\\
599.44	1.07066748772523e-05\\
599.45	1.03590202840433e-05\\
599.46	1.00160452698589e-05\\
599.47	9.67779878049101e-06\\
599.48	9.34433024810111e-06\\
599.49	9.0156895960411e-06\\
599.5	8.69192724369319e-06\\
599.51	8.37309411137396e-06\\
599.52	8.05924162529392e-06\\
599.53	7.75042172253965e-06\\
599.54	7.44668685613396e-06\\
599.55	7.14809000013084e-06\\
599.56	6.85468465475535e-06\\
599.57	6.56652485161308e-06\\
599.58	6.28366515892896e-06\\
599.59	6.00616068686249e-06\\
599.6	5.73406709284546e-06\\
599.61	5.46744058700296e-06\\
599.62	5.20633793760217e-06\\
599.63	4.95081647657741e-06\\
599.64	4.70093410508653e-06\\
599.65	4.45674929914174e-06\\
599.66	4.21832111529089e-06\\
599.67	3.98570919634203e-06\\
599.68	3.75897377716608e-06\\
599.69	3.53817569053415e-06\\
599.7	3.32337637303469e-06\\
599.71	3.11463787102881e-06\\
599.72	2.91202284668363e-06\\
599.73	2.71559458404382e-06\\
599.74	2.52541699518466e-06\\
599.75	2.34155462640329e-06\\
599.76	2.16407266449663e-06\\
599.77	1.99303694307928e-06\\
599.78	1.82851394897598e-06\\
599.79	1.67057082867302e-06\\
599.8	1.51927539483385e-06\\
599.81	1.37469613287027e-06\\
599.82	1.23690220760718e-06\\
599.83	1.10596346997172e-06\\
599.84	9.81950463782924e-07\\
599.85	8.64934432591793e-07\\
599.86	7.54987326587186e-07\\
599.87	6.5218180958504e-07\\
599.88	5.56591266064402e-07\\
599.89	4.68289808295413e-07\\
599.9	3.87352283521061e-07\\
599.91	3.13854281216996e-07\\
599.92	2.4787214042421e-07\\
599.93	1.89482957149364e-07\\
599.94	1.38764591832777e-07\\
599.95	9.57956769222224e-08\\
599.96	6.06556244606149e-08\\
599.97	3.34246338211386e-08\\
599.98	1.4183699454523e-08\\
599.99	3.01461875948372e-09\\
600	0\\
};
\end{axis}
\end{tikzpicture}%
%  \caption{Continuous Time}
%\end{subfigure}%
%\hfill%
%\begin{subfigure}{.45\linewidth}
%  \centering
%  \setlength\figureheight{\linewidth} 
%  \setlength\figurewidth{\linewidth}
%  \tikzsetnextfilename{dm_dscr_z8}
%  % This file was created by matlab2tikz.
%
%The latest updates can be retrieved from
%  http://www.mathworks.com/matlabcentral/fileexchange/22022-matlab2tikz-matlab2tikz
%where you can also make suggestions and rate matlab2tikz.
%
\definecolor{mycolor1}{rgb}{1.00000,0.00000,1.00000}%
%
\begin{tikzpicture}

\begin{axis}[%
width=4.564in,
height=3.803in,
at={(1.067in,0.513in)},
scale only axis,
every outer x axis line/.append style={black},
every x tick label/.append style={font=\color{black}},
xmin=0,
xmax=100,
xlabel={Time},
every outer y axis line/.append style={black},
every y tick label/.append style={font=\color{black}},
ymin=0,
ymax=0.012,
ylabel={Depth $\delta$},
axis background/.style={fill=white},
title={Z=8},
axis x line*=bottom,
axis y line*=left,
legend style={legend cell align=left,align=left,draw=black}
]
\addplot [color=green,dashed]
  table[row sep=crcr]{%
1	0.00485945625675347\\
2	0.00487414302033052\\
3	0.00488936844963023\\
4	0.00490515306501651\\
5	0.00492151829663004\\
6	0.00493848654848153\\
7	0.00495608127189559\\
8	0.00497432704549034\\
9	0.0049932496508163\\
10	0.00501287614094745\\
11	0.00503323490651359\\
12	0.00505435575266095\\
13	0.00507627004656115\\
14	0.00509901110469301\\
15	0.00512261431087908\\
16	0.00514711692825322\\
17	0.00517255821158281\\
18	0.0051989795554708\\
19	0.00522642469838481\\
20	0.00525494031771952\\
21	0.00528457648587319\\
22	0.0053153870786072\\
23	0.00534742886513508\\
24	0.00538076028949751\\
25	0.0054154428220149\\
26	0.00545154020211779\\
27	0.00548911714246375\\
28	0.00552823129860305\\
29	0.00556894135895538\\
30	0.00561131173166151\\
31	0.00565541678573689\\
32	0.00570134233290593\\
33	0.00574918961030762\\
34	0.00579908018002111\\
35	0.005851164268531\\
36	0.00590565702760265\\
37	0.00596283833729376\\
38	0.00602291072182752\\
39	0.00608596820594332\\
40	0.00615208768624582\\
41	0.0062213147623831\\
42	0.00629362362321532\\
43	0.00636892654015613\\
44	0.00644722868781644\\
45	0.00652843920469626\\
46	0.00661232934652509\\
47	0.00669847046861788\\
48	0.00678615428450335\\
49	0.00687421974785775\\
50	0.00696067182424244\\
51	0.00704414543345934\\
52	0.00712427704869411\\
53	0.00720078458073156\\
54	0.00727326795226155\\
55	0.0073418697025762\\
56	0.00740757939773257\\
57	0.00747194638628693\\
58	0.00753616146270302\\
59	0.00760032443049048\\
60	0.00766456970083794\\
61	0.00772906707473294\\
62	0.00779402631784734\\
63	0.00785968044069715\\
64	0.00792626260473561\\
65	0.00799388326307111\\
66	0.00806258634939459\\
67	0.00813239846694426\\
68	0.00820281961516284\\
69	0.00827383024153404\\
70	0.00834539769073058\\
71	0.00841756757162911\\
72	0.00849054969104642\\
73	0.00856434016815509\\
74	0.00863893108773656\\
75	0.00871432309957891\\
76	0.00878982547481302\\
77	0.00886405134220572\\
78	0.00893671335213983\\
79	0.00900822354657033\\
80	0.00907847647648609\\
81	0.00914730827594116\\
82	0.00921440840949971\\
83	0.00928113934875132\\
84	0.00934779119356061\\
85	0.00941413354923462\\
86	0.0094798880781265\\
87	0.00954466555764447\\
88	0.00960799533874711\\
89	0.00966853872526029\\
90	0.00972565262017696\\
91	0.0097788671886327\\
92	0.0098278806934644\\
93	0.00987342441910115\\
94	0.00991480090004276\\
95	0.00995105460587982\\
96	0.00998065038037144\\
97	0.00999982980159223\\
98	0.01003324910555\\
99	0\\
100	0\\
};
\addlegendentry{$q=-4$};

\addplot [color=mycolor1,dashed]
  table[row sep=crcr]{%
1	0.00382394873317695\\
2	0.00384957111565333\\
3	0.00387611528866614\\
4	0.00390361249529857\\
5	0.0039320947298463\\
6	0.00396159469728134\\
7	0.00399214575585568\\
8	0.00402378183377548\\
9	0.00405653729889385\\
10	0.00409044670018183\\
11	0.0041255440385175\\
12	0.00416185989878204\\
13	0.00419940650464184\\
14	0.00423819597083052\\
15	0.00427825536483802\\
16	0.0043196096312672\\
17	0.00436228095842971\\
18	0.00440628812960201\\
19	0.00445164616064467\\
20	0.00449836727525809\\
21	0.00454646698192176\\
22	0.0045959897301589\\
23	0.00464704069773572\\
24	0.00469960624769317\\
25	0.00475365559312346\\
26	0.00480913602535157\\
27	0.00486596706169554\\
28	0.00492403367726455\\
29	0.00498317662968881\\
30	0.00504318022982533\\
31	0.00510375560939573\\
32	0.00516451886391788\\
33	0.00522496301408269\\
34	0.00528442237628088\\
35	0.00534202975236024\\
36	0.00539678989436052\\
37	0.00544837002297262\\
38	0.00550237551216994\\
39	0.0055589315185265\\
40	0.00561814043264873\\
41	0.00568009263451653\\
42	0.00574487272443725\\
43	0.00581257793596128\\
44	0.00588329827868223\\
45	0.00595711328056412\\
46	0.00603408787606269\\
47	0.00611426621538027\\
48	0.00619765173247439\\
49	0.0062841498977199\\
50	0.00637368789242256\\
51	0.00646622029331727\\
52	0.00656160117960447\\
53	0.00665954007715095\\
54	0.00675956807606328\\
55	0.0068609611061804\\
56	0.00696252346518636\\
57	0.00706227997466448\\
58	0.00715870268656942\\
59	0.00725154359529958\\
60	0.00734071115791681\\
61	0.00742600917742923\\
62	0.00750754664013129\\
63	0.00758595983147457\\
64	0.00766200587575343\\
65	0.00773753850067737\\
66	0.00781290984532616\\
67	0.0078881605406168\\
68	0.00796339019400955\\
69	0.00803874861910076\\
70	0.00811444087980434\\
71	0.00819062585220229\\
72	0.00826742321998385\\
73	0.00834501196542576\\
74	0.00842343896169291\\
75	0.0085027251122091\\
76	0.00858289961770601\\
77	0.00866399899362369\\
78	0.00874604952977821\\
79	0.00882844448104465\\
80	0.00891112708521404\\
81	0.00899409332830582\\
82	0.0090773568188397\\
83	0.00915945101398878\\
84	0.009239854675922\\
85	0.00931847594409736\\
86	0.00939551870782303\\
87	0.00947072995214786\\
88	0.00954351336420434\\
89	0.00961296437622768\\
90	0.00967974901988489\\
91	0.00974411148075818\\
92	0.0098051085698454\\
93	0.00986066688361775\\
94	0.00990963950258614\\
95	0.00995008790550502\\
96	0.00998057518199083\\
97	0.00999982980159223\\
98	0.01003324910555\\
99	0\\
100	0\\
};
\addlegendentry{$q=-3$};

\addplot [color=red,dashed]
  table[row sep=crcr]{%
1	0.00203625637388802\\
2	0.00205694682399506\\
3	0.00207847386511244\\
4	0.00210087662914663\\
5	0.00212419664426941\\
6	0.00214847803912337\\
7	0.00217376776201011\\
8	0.00220011580521964\\
9	0.00222757541446491\\
10	0.00225620326863204\\
11	0.00228605967015089\\
12	0.00231720809827092\\
13	0.00234971785761731\\
14	0.00238366585758567\\
15	0.00241913551019996\\
16	0.00245621760525557\\
17	0.00249501140621176\\
18	0.00253562607725969\\
19	0.00257818256996886\\
20	0.00262281598900947\\
21	0.00266967873293951\\
22	0.00271894350197721\\
23	0.00277079045151589\\
24	0.00282542156688777\\
25	0.00288306442435626\\
26	0.00294397666663905\\
27	0.00300845123391652\\
28	0.00307682238123111\\
29	0.00314947455760223\\
30	0.00322683619825089\\
31	0.0033094322528504\\
32	0.00339788772366867\\
33	0.003492941051719\\
34	0.00359546346540763\\
35	0.00370646241761729\\
36	0.00377583934464846\\
37	0.00385317360833428\\
38	0.00393324370364177\\
39	0.0040161639545401\\
40	0.00410210460056693\\
41	0.00419125142992333\\
42	0.00428368399359528\\
43	0.00437941286843043\\
44	0.0044784804383629\\
45	0.0045809385558749\\
46	0.004686820361371\\
47	0.00479613230001219\\
48	0.00490884668681802\\
49	0.00502491892166917\\
50	0.00514431096295905\\
51	0.00526709981183514\\
52	0.00539322086161395\\
53	0.00552239190252712\\
54	0.00565414961068421\\
55	0.00578780616934352\\
56	0.00592231962241305\\
57	0.00605633882844064\\
58	0.00618818471261372\\
59	0.0063156027855923\\
60	0.00643720690847549\\
61	0.00656262720770935\\
62	0.00669131339088365\\
63	0.0068223129141614\\
64	0.00695409983086691\\
65	0.00708409457700272\\
66	0.00721129009152353\\
67	0.00733495466027041\\
68	0.00745411623004709\\
69	0.00756773135676126\\
70	0.00767512187930786\\
71	0.00777734561538589\\
72	0.00787627928465611\\
73	0.00797309686954974\\
74	0.00806886123862369\\
75	0.00816351733955394\\
76	0.00825716402085118\\
77	0.00835021851031631\\
78	0.00844302389172513\\
79	0.00853581364536144\\
80	0.00862862608066721\\
81	0.00872151736171665\\
82	0.00881453042058036\\
83	0.00890764015363901\\
84	0.00900077402725062\\
85	0.00909385280462877\\
86	0.00918659398642789\\
87	0.00927874325742761\\
88	0.00937027399818891\\
89	0.00946124141899078\\
90	0.0095502828761426\\
91	0.00963604389549451\\
92	0.00971781643994538\\
93	0.00979475046813423\\
94	0.00986566787131446\\
95	0.00992847447705118\\
96	0.00997640299214361\\
97	0.00999961158897599\\
98	0.01003324910555\\
99	0\\
100	0\\
};
\addlegendentry{$q=-2$};

\addplot [color=blue,dashed]
  table[row sep=crcr]{%
1	0.000107277425565821\\
2	0.00010829576775512\\
3	0.000109357246700352\\
4	0.000110464014507547\\
5	0.000111618360836398\\
6	0.000112822727060759\\
7	0.000114079725526992\\
8	0.000115392166628401\\
9	0.000116763089083499\\
10	0.000118195743580936\\
11	0.000119693361035447\\
12	0.000121259375659868\\
13	0.000122897678162007\\
14	0.000124612447359982\\
15	0.000126408170924103\\
16	0.000128289659307511\\
17	0.000130262047800479\\
18	0.000132330798449739\\
19	0.000134501802044341\\
20	0.000136781921283393\\
21	0.000139179763237535\\
22	0.000141703525482586\\
23	0.000144362268124243\\
24	0.000147166064002637\\
25	0.000150126205431619\\
26	0.000153255505478384\\
27	0.000156568692226079\\
28	0.000160082524410796\\
29	0.000163812506886159\\
30	0.00016777896427553\\
31	0.000172006473451556\\
32	0.000176524358632362\\
33	0.000181369318582451\\
34	0.000186592626833443\\
35	0.000192285669283197\\
36	0.00024942253577002\\
37	0.000309145709392737\\
38	0.000371348848368564\\
39	0.00043620111965971\\
40	0.000503882269590036\\
41	0.00057457384528812\\
42	0.00064848171472686\\
43	0.000725843514025343\\
44	0.000806928536617441\\
45	0.000892039518267811\\
46	0.000981518305163587\\
47	0.00107575357524396\\
48	0.0011751923918723\\
49	0.00128035430398567\\
50	0.00139184649354964\\
51	0.00151034982277297\\
52	0.00163648204936279\\
53	0.00177110975924253\\
54	0.00191537073237776\\
55	0.00207046770012352\\
56	0.0022378391613308\\
57	0.00241919027646704\\
58	0.00261652910209623\\
59	0.00283222966545185\\
60	0.00306752599786927\\
61	0.00331283086515275\\
62	0.0035685996048372\\
63	0.00383519421471796\\
64	0.00411302621380756\\
65	0.00440274342922439\\
66	0.00469935368877752\\
67	0.00500583342817488\\
68	0.00532532718734629\\
69	0.00565876112690701\\
70	0.00593065199045151\\
71	0.00613615367870856\\
72	0.00634160416831506\\
73	0.00654370918544362\\
74	0.00673930839805359\\
75	0.00692640629703696\\
76	0.00710211369599329\\
77	0.00726192851247405\\
78	0.00741896330693372\\
79	0.00757405866803044\\
80	0.00772673960027738\\
81	0.0078768892415683\\
82	0.00802476112604144\\
83	0.00817170332587985\\
84	0.00831818108566537\\
85	0.00846462780503413\\
86	0.00861055071139941\\
87	0.00875558871699461\\
88	0.00889949761240524\\
89	0.00904149002310103\\
90	0.00918069225954357\\
91	0.00931615360006014\\
92	0.00944684615622117\\
93	0.00957168006435625\\
94	0.00968954914319653\\
95	0.00979940168610245\\
96	0.00989957393262904\\
97	0.00998069825602133\\
98	0.01003324910555\\
99	0\\
100	0\\
};
\addlegendentry{$q=-1$};

\addplot [color=black,solid]
  table[row sep=crcr]{%
1	0.000315132141961445\\
2	0.000315132141961445\\
3	0.000315132141961445\\
4	0.000315132141961445\\
5	0.000315132141961445\\
6	0.000315132141961445\\
7	0.000315132141961445\\
8	0.000315132141961445\\
9	0.000315132141961445\\
10	0.000315132141961445\\
11	0.000315132141961445\\
12	0.000315132141961445\\
13	0.000315132141961445\\
14	0.000315132141961445\\
15	0.000315132141961445\\
16	0.000315132141961445\\
17	0.000315132141961445\\
18	0.000315132141961445\\
19	0.000315132141961445\\
20	0.000315132141961445\\
21	0.000315132141961445\\
22	0.000315132141961445\\
23	0.000315132141961445\\
24	0.000315132141961445\\
25	0.000315132141961445\\
26	0.000315132141961445\\
27	0.000315132141961445\\
28	0.000315132141961445\\
29	0.000315132141961445\\
30	0.000315132141961445\\
31	0.000315132141961445\\
32	0.000315132141961445\\
33	0.000315132141961445\\
34	0.000315132141961445\\
35	0.000315132141961445\\
36	0.000315132141961445\\
37	0.000315132141961445\\
38	0.000315132141961445\\
39	0.000315132141961445\\
40	0.000315132141961445\\
41	0.000315132141961445\\
42	0.000315132141961445\\
43	0.000315132141961445\\
44	0.000315132141961445\\
45	0.000315132141961445\\
46	0.000315132141961445\\
47	0.000315132141961445\\
48	0.000315132141961445\\
49	0.000315132141961445\\
50	0.000315132141961445\\
51	0.000315132141961445\\
52	0.00031527825582075\\
53	0.00031555428017459\\
54	0.000315849297560266\\
55	0.000316166125854495\\
56	0.000316507903973797\\
57	0.000316878207644637\\
58	0.000317281267679093\\
59	0.000317722098240475\\
60	0.000318206664581177\\
61	0.000318742094533239\\
62	0.000319336814178377\\
63	0.000320000854997598\\
64	0.000320746086345837\\
65	0.000321586520854692\\
66	0.000327984279529535\\
67	0.000337550585480914\\
68	0.00034797886066513\\
69	0.000359407450804621\\
70	0.000446271320100876\\
71	0.000611671498222364\\
72	0.000790051572150104\\
73	0.000983252379655302\\
74	0.00119225648897744\\
75	0.00141097874546424\\
76	0.00165196551615618\\
77	0.00191908662026477\\
78	0.00219924705869884\\
79	0.00249135195420683\\
80	0.00279464276648611\\
81	0.00310906201374306\\
82	0.00343464900349301\\
83	0.0037701413982495\\
84	0.004115488203838\\
85	0.00447109773233466\\
86	0.00483655278955692\\
87	0.00521198560788567\\
88	0.0055971677510235\\
89	0.00599221072788071\\
90	0.00639689196057587\\
91	0.00681115249885264\\
92	0.00723501370613978\\
93	0.00766825212328673\\
94	0.00811008000553154\\
95	0.0085593774169704\\
96	0.00901438057914194\\
97	0.00947206379995425\\
98	0.00992590828135393\\
99	0\\
100	0\\
};
\addlegendentry{$q=0$};

\addplot [color=blue,solid]
  table[row sep=crcr]{%
1	0.0100329692748296\\
2	0.0100329679438438\\
3	0.0100329665737844\\
4	0.0100329651631603\\
5	0.0100329637110275\\
6	0.0100329622162635\\
7	0.0100329606765574\\
8	0.010032959088764\\
9	0.0100329574488552\\
10	0.0100329557554473\\
11	0.0100329540110098\\
12	0.0100329522124844\\
13	0.010032950355083\\
14	0.0100329484286329\\
15	0.0100329464042331\\
16	0.0100329441798771\\
17	0.010032937934459\\
18	0.0100329249582351\\
19	0.01003291153935\\
20	0.0100328976522816\\
21	0.0100328832665241\\
22	0.0100328683403523\\
23	0.0100328528016462\\
24	0.0100328364917738\\
25	0.0100328190387565\\
26	0.0100327999966034\\
27	0.0100327802429298\\
28	0.0100327597344465\\
29	0.0100327384240231\\
30	0.0100327162602481\\
31	0.0100326931868577\\
32	0.0100326691418516\\
33	0.0100326440557504\\
34	0.0100326178472242\\
35	0.0100325904097408\\
36	0.0100325615631244\\
37	0.0100325308450213\\
38	0.0100324459543204\\
39	0.0100322214910233\\
40	0.0100319901987928\\
41	0.0100317511363865\\
42	0.0100315037528158\\
43	0.0100312475296138\\
44	0.010030981844076\\
45	0.0100307060610967\\
46	0.0100304193007889\\
47	0.0100301205485006\\
48	0.0100298086290865\\
49	0.0100294821760859\\
50	0.010029139593409\\
51	0.0100287789973107\\
52	0.0100283980729242\\
53	0.0100279935061273\\
54	0.0100275602526793\\
55	0.0100241158619052\\
56	0.0100199123580354\\
57	0.0100156641671768\\
58	0.0100113729002276\\
59	0.0100070372259259\\
60	0.0100026557934925\\
61	0.00999822745059608\\
62	0.0099937513919967\\
63	0.00998922738064705\\
64	0.00998465609120485\\
65	0.00998003948099702\\
66	0.00997460734725493\\
67	0.00996882978578957\\
68	0.00996295928471298\\
69	0.00995700828182999\\
70	0.00995115051904177\\
71	0.00993198468323465\\
72	0.00991179204224949\\
73	0.00989059140574361\\
74	0.00984791950631454\\
75	0.00962443667249636\\
76	0.00938631826922074\\
77	0.00913157865020818\\
78	0.0088578625968306\\
79	0.0085623508134461\\
80	0.0082418646173091\\
81	0.00789585237346233\\
82	0.00753610471222871\\
83	0.00716308979915167\\
84	0.00677636210378066\\
85	0.00637565744842168\\
86	0.00596114399340499\\
87	0.00553417378650248\\
88	0.00509423606601865\\
89	0.00464085403357268\\
90	0.00417361846537467\\
91	0.00369226049843334\\
92	0.00319748084124855\\
93	0.00269022915655069\\
94	0.00217160322227635\\
95	0.00164283062467496\\
96	0.00110643888137445\\
97	0.000567408530202185\\
98	3.32491055499654e-05\\
99	0\\
100	0\\
};
\addlegendentry{$q=1$};

\addplot [color=red,solid]
  table[row sep=crcr]{%
1	0.0100187058662163\\
2	0.0100185634774708\\
3	0.0100184167401897\\
4	0.0100182654762648\\
5	0.0100181095187468\\
6	0.0100179486955663\\
7	0.0100177827906476\\
8	0.0100176115426449\\
9	0.0100174346532581\\
10	0.0100172519066996\\
11	0.010017063213596\\
12	0.0100168682805138\\
13	0.0100166667049456\\
14	0.01001645788312\\
15	0.0100162405664583\\
16	0.0100160110702597\\
17	0.0100156438207477\\
18	0.0100150496634009\\
19	0.0100144342467735\\
20	0.0100137961191358\\
21	0.0100131335578139\\
22	0.0100124443370166\\
23	0.0100117250547756\\
24	0.0100109692032847\\
25	0.0100101627608925\\
26	0.0100090618504057\\
27	0.010006307215976\\
28	0.0100034896443133\\
29	0.0100006071590769\\
30	0.0099976576951472\\
31	0.00999463910218179\\
32	0.00999154914522113\\
33	0.00998838548652868\\
34	0.00998514559313926\\
35	0.00998182636436354\\
36	0.00997842262206861\\
37	0.00997492033449082\\
38	0.00996959056129169\\
39	0.00995959564403076\\
40	0.00994930541761534\\
41	0.00993870269318898\\
42	0.00992776844428659\\
43	0.00991648522397518\\
44	0.00990483303003661\\
45	0.00989278905253359\\
46	0.00988032052939186\\
47	0.00986738869252152\\
48	0.00985394858649275\\
49	0.00983994787608937\\
50	0.0098253253659455\\
51	0.00981000880022007\\
52	0.00979390964062349\\
53	0.00977690293422598\\
54	0.00975880086933501\\
55	0.00963403039976769\\
56	0.00947791061205357\\
57	0.00931475232544556\\
58	0.00914409334018826\\
59	0.00896527233445575\\
60	0.00877745433557311\\
61	0.00857967616664015\\
62	0.00837082206220613\\
63	0.00814959444555788\\
64	0.00791448054645849\\
65	0.00766370763610572\\
66	0.00739599210340636\\
67	0.0071083790920695\\
68	0.00679782284173191\\
69	0.00646119601050876\\
70	0.00610733115161804\\
71	0.00575262518513171\\
72	0.00538348894161203\\
73	0.00499873019055509\\
74	0.00461820533023241\\
75	0.00440450791375652\\
76	0.00418881841356822\\
77	0.00397235882276841\\
78	0.00375714245292512\\
79	0.00354664900457645\\
80	0.00334590530539139\\
81	0.0031572889882801\\
82	0.00296807243696754\\
83	0.00277736910648468\\
84	0.00258547823036496\\
85	0.00239267130567504\\
86	0.00219898690736822\\
87	0.00200317573278665\\
88	0.00180613206651937\\
89	0.00160871409912125\\
90	0.00141197570640385\\
91	0.00121677147072099\\
92	0.00102360464204711\\
93	0.000833141672008094\\
94	0.000646658220394614\\
95	0.000465861915824526\\
96	0.000292511834892549\\
97	0.00012959792500039\\
98	3.32491055499654e-05\\
99	0\\
100	0\\
};
\addlegendentry{$q=2$};

\addplot [color=mycolor1,solid]
  table[row sep=crcr]{%
1	0.00981860164552879\\
2	0.00981423735354565\\
3	0.00980974584851086\\
4	0.00980512224211095\\
5	0.0098003613757449\\
6	0.00979545810592951\\
7	0.00979040713881327\\
8	0.0097852025101279\\
9	0.0097798378894794\\
10	0.00977430665705891\\
11	0.00976860199779076\\
12	0.00976271640359355\\
13	0.00975664051604021\\
14	0.00975036421452315\\
15	0.00974387684648715\\
16	0.00973716875959182\\
17	0.00973035749903079\\
18	0.00972352322598488\\
19	0.00971643447848896\\
20	0.00970906919908387\\
21	0.00970140117514127\\
22	0.00969339953279936\\
23	0.00968502776192527\\
24	0.00967624238960126\\
25	0.00966698842236478\\
26	0.00964914125260309\\
27	0.00957274674989259\\
28	0.00949373601077428\\
29	0.00941198094100749\\
30	0.00932734405865233\\
31	0.00923967947397971\\
32	0.00914883301142084\\
33	0.0090546427260522\\
34	0.00895694005674083\\
35	0.00885555211217539\\
36	0.00875030643872931\\
37	0.00864104329795993\\
38	0.00852939674446276\\
39	0.00841815502170258\\
40	0.00830257718119142\\
41	0.00818242267763374\\
42	0.00805744616901231\\
43	0.00792740735238815\\
44	0.00779210186077065\\
45	0.00765128427700737\\
46	0.00750457647174612\\
47	0.00735143723509152\\
48	0.00719123946373411\\
49	0.00702325178588119\\
50	0.00684661528406853\\
51	0.00666031387530546\\
52	0.00646313762077005\\
53	0.00625364552867741\\
54	0.00603010672697628\\
55	0.00589894414857172\\
56	0.00578182254651292\\
57	0.00565159408914177\\
58	0.00551322823895752\\
59	0.00537184143962644\\
60	0.00522766866812109\\
61	0.00508104792573206\\
62	0.00493244932777306\\
63	0.00478250303054305\\
64	0.00463204718550654\\
65	0.00448222704670039\\
66	0.0043345728255961\\
67	0.00419112611387241\\
68	0.00405446500596207\\
69	0.00392790552331875\\
70	0.00380235521941472\\
71	0.00367457296327706\\
72	0.00354499292623339\\
73	0.00341423479445978\\
74	0.0032833149303601\\
75	0.00315159803372358\\
76	0.00302107454113035\\
77	0.00289199695409881\\
78	0.00276442662866803\\
79	0.00263800975207846\\
80	0.00251120114192371\\
81	0.00238203468679405\\
82	0.00225060925979982\\
83	0.00211690448951394\\
84	0.00198066918796629\\
85	0.00184174376140796\\
86	0.00170018422525516\\
87	0.00155632507068512\\
88	0.00141057483317854\\
89	0.00126294383989656\\
90	0.00111371401515057\\
91	0.000963208188100662\\
92	0.000811803711308885\\
93	0.00066005319163681\\
94	0.000509922248582303\\
95	0.0003623900613677\\
96	0.000222877382777326\\
97	0.0001131259618254\\
98	3.32491055499654e-05\\
99	0\\
100	0\\
};
\addlegendentry{$q=3$};

\addplot [color=green,solid]
  table[row sep=crcr]{%
1	0.00840850314635316\\
2	0.00836390536715778\\
3	0.00831801231030496\\
4	0.00827078217763323\\
5	0.0082221715057822\\
6	0.00817213507356084\\
7	0.0081206261136224\\
8	0.00806759381786456\\
9	0.00801298003986253\\
10	0.00795672111441184\\
11	0.00789874790436701\\
12	0.00783898535219299\\
13	0.00777735299267128\\
14	0.00771376447120918\\
15	0.00764812709959048\\
16	0.00758034147920659\\
17	0.00751029825504354\\
18	0.00743787883048643\\
19	0.00736296469479454\\
20	0.00728543408216624\\
21	0.00720513501774509\\
22	0.00712190115643649\\
23	0.00703555046877898\\
24	0.00694588526332273\\
25	0.0068526981975587\\
26	0.00676406609247885\\
27	0.00673119653239127\\
28	0.00669706169314112\\
29	0.00666164813517396\\
30	0.0066249502220872\\
31	0.00658684636954535\\
32	0.00654719118990135\\
33	0.00650580937341389\\
34	0.00646248777838751\\
35	0.00641696515639261\\
36	0.00636891871664562\\
37	0.00631794634792286\\
38	0.00626352317128828\\
39	0.00620498797956653\\
40	0.00614165712404742\\
41	0.00607269254849635\\
42	0.00599706175588608\\
43	0.00591348891830688\\
44	0.0058204177575719\\
45	0.00572246012464679\\
46	0.00562288682616604\\
47	0.00552189755763095\\
48	0.00541976259189682\\
49	0.00531684291696741\\
50	0.00521361638470062\\
51	0.00511071176009554\\
52	0.00500895302494531\\
53	0.0049094164795163\\
54	0.00481350405759531\\
55	0.00472144456513887\\
56	0.00463472119686986\\
57	0.00455560038458885\\
58	0.00447917291126602\\
59	0.00440035548125985\\
60	0.00431910495801014\\
61	0.0042354110809133\\
62	0.00414929315612524\\
63	0.00406081832404118\\
64	0.00396998039615607\\
65	0.00387657656643398\\
66	0.00378054699318786\\
67	0.00368198916379013\\
68	0.0035808306845737\\
69	0.00347693953677897\\
70	0.00337084635289038\\
71	0.00326451824871775\\
72	0.00315937428408911\\
73	0.00305537219184628\\
74	0.00295213854074137\\
75	0.00284825403748608\\
76	0.00274209284132924\\
77	0.00263373892104426\\
78	0.00252305301400646\\
79	0.00240975409464889\\
80	0.00229343741535956\\
81	0.00217409690567752\\
82	0.00205173954490793\\
83	0.00192639870728597\\
84	0.00179814115984751\\
85	0.00166709702916354\\
86	0.001533442602948\\
87	0.00139862000009753\\
88	0.00126255262385255\\
89	0.00112557608806227\\
90	0.000988077718204886\\
91	0.000850509502407808\\
92	0.000713404021036185\\
93	0.00057879574612941\\
94	0.000447390496825131\\
95	0.000321946478778335\\
96	0.000210513689464598\\
97	0.0001131259618254\\
98	3.32491055499654e-05\\
99	0\\
100	0\\
};
\addlegendentry{$q=4$};

\end{axis}
\end{tikzpicture}% 
%  \caption{Discrete Time}
%\end{subfigure}\\
%\vspace{1cm}
%\begin{subfigure}{.45\linewidth}
%  \centering
%  \setlength\figureheight{\linewidth} 
%  \setlength\figurewidth{\linewidth}
%  \tikzsetnextfilename{dm_cts_nFPC_z8}
%  % This file was created by matlab2tikz.
%
%The latest updates can be retrieved from
%  http://www.mathworks.com/matlabcentral/fileexchange/22022-matlab2tikz-matlab2tikz
%where you can also make suggestions and rate matlab2tikz.
%
\definecolor{mycolor1}{rgb}{0.00000,1.00000,0.14286}%
\definecolor{mycolor2}{rgb}{0.00000,1.00000,0.28571}%
\definecolor{mycolor3}{rgb}{0.00000,1.00000,0.42857}%
\definecolor{mycolor4}{rgb}{0.00000,1.00000,0.57143}%
\definecolor{mycolor5}{rgb}{0.00000,1.00000,0.71429}%
\definecolor{mycolor6}{rgb}{0.00000,1.00000,0.85714}%
\definecolor{mycolor7}{rgb}{0.00000,1.00000,1.00000}%
\definecolor{mycolor8}{rgb}{0.00000,0.87500,1.00000}%
\definecolor{mycolor9}{rgb}{0.00000,0.62500,1.00000}%
\definecolor{mycolor10}{rgb}{0.12500,0.00000,1.00000}%
\definecolor{mycolor11}{rgb}{0.25000,0.00000,1.00000}%
\definecolor{mycolor12}{rgb}{0.37500,0.00000,1.00000}%
\definecolor{mycolor13}{rgb}{0.50000,0.00000,1.00000}%
\definecolor{mycolor14}{rgb}{0.62500,0.00000,1.00000}%
\definecolor{mycolor15}{rgb}{0.75000,0.00000,1.00000}%
\definecolor{mycolor16}{rgb}{0.87500,0.00000,1.00000}%
\definecolor{mycolor17}{rgb}{1.00000,0.00000,1.00000}%
\definecolor{mycolor18}{rgb}{1.00000,0.00000,0.87500}%
\definecolor{mycolor19}{rgb}{1.00000,0.00000,0.62500}%
\definecolor{mycolor20}{rgb}{0.85714,0.00000,0.00000}%
\definecolor{mycolor21}{rgb}{0.71429,0.00000,0.00000}%
%
\begin{tikzpicture}

\begin{axis}[%
width=4.1in,
height=3.803in,
at={(0.809in,0.513in)},
scale only axis,
point meta min=0,
point meta max=1,
every outer x axis line/.append style={black},
every x tick label/.append style={font=\color{black}},
xmin=0,
xmax=600,
every outer y axis line/.append style={black},
every y tick label/.append style={font=\color{black}},
ymin=0,
ymax=0.012,
axis background/.style={fill=white},
axis x line*=bottom,
axis y line*=left,
colormap={mymap}{[1pt] rgb(0pt)=(0,1,0); rgb(7pt)=(0,1,1); rgb(15pt)=(0,0,1); rgb(23pt)=(1,0,1); rgb(31pt)=(1,0,0); rgb(38pt)=(0,0,0)},
colorbar,
colorbar style={separate axis lines,every outer x axis line/.append style={black},every x tick label/.append style={font=\color{black}},every outer y axis line/.append style={black},every y tick label/.append style={font=\color{black}},yticklabels={{-19},{-17},{-15},{-13},{-11},{-9},{-7},{-5},{-3},{-1},{1},{3},{5},{7},{9},{11},{13},{15},{17},{19}}}
]
\addplot [color=green,solid,forget plot]
  table[row sep=crcr]{%
0.01	0.0050253436469414\\
1.01	0.00502534430234539\\
2.01	0.00502534497054888\\
3.01	0.00502534565180165\\
4.01	0.0050253463463589\\
5.01	0.00502534705448057\\
6.01	0.00502534777643188\\
7.01	0.00502534851248307\\
8.01	0.00502534926290973\\
9.01	0.00502535002799284\\
10.01	0.00502535080801872\\
11.01	0.00502535160327991\\
12.01	0.00502535241407387\\
13.01	0.00502535324070441\\
14.01	0.00502535408348131\\
15.01	0.00502535494272008\\
16.01	0.00502535581874267\\
17.01	0.00502535671187737\\
18.01	0.00502535762245873\\
19.01	0.00502535855082814\\
20.01	0.00502535949733349\\
21.01	0.00502536046232946\\
22.01	0.00502536144617782\\
23.01	0.00502536244924762\\
24.01	0.00502536347191462\\
25.01	0.00502536451456237\\
26.01	0.00502536557758182\\
27.01	0.00502536666137174\\
28.01	0.00502536776633886\\
29.01	0.00502536889289751\\
30.01	0.00502537004147008\\
31.01	0.00502537121248772\\
32.01	0.00502537240638989\\
33.01	0.00502537362362432\\
34.01	0.00502537486464871\\
35.01	0.00502537612992798\\
36.01	0.00502537741993778\\
37.01	0.00502537873516229\\
38.01	0.00502538007609546\\
39.01	0.00502538144324083\\
40.01	0.00502538283711204\\
41.01	0.00502538425823262\\
42.01	0.00502538570713662\\
43.01	0.00502538718436846\\
44.01	0.00502538869048358\\
45.01	0.00502539022604787\\
46.01	0.00502539179163891\\
47.01	0.005025393387845\\
48.01	0.00502539501526664\\
49.01	0.00502539667451579\\
50.01	0.00502539836621682\\
51.01	0.00502540009100638\\
52.01	0.00502540184953321\\
53.01	0.00502540364245924\\
54.01	0.00502540547045979\\
55.01	0.00502540733422271\\
56.01	0.00502540923444972\\
57.01	0.00502541117185656\\
58.01	0.00502541314717336\\
59.01	0.00502541516114388\\
60.01	0.00502541721452765\\
61.01	0.00502541930809802\\
62.01	0.00502542144264437\\
63.01	0.00502542361897154\\
64.01	0.00502542583790042\\
65.01	0.00502542810026785\\
66.01	0.00502543040692743\\
67.01	0.00502543275874937\\
68.01	0.00502543515662152\\
69.01	0.00502543760144908\\
70.01	0.00502544009415481\\
71.01	0.00502544263568036\\
72.01	0.0050254452269857\\
73.01	0.00502544786904979\\
74.01	0.00502545056287121\\
75.01	0.00502545330946821\\
76.01	0.00502545610987915\\
77.01	0.00502545896516299\\
78.01	0.00502546187639957\\
79.01	0.00502546484469038\\
80.01	0.00502546787115871\\
81.01	0.00502547095695013\\
82.01	0.00502547410323306\\
83.01	0.00502547731119857\\
84.01	0.00502548058206204\\
85.01	0.00502548391706273\\
86.01	0.00502548731746463\\
87.01	0.0050254907845565\\
88.01	0.00502549431965296\\
89.01	0.00502549792409521\\
90.01	0.00502550159925003\\
91.01	0.00502550534651219\\
92.01	0.0050255091673041\\
93.01	0.00502551306307628\\
94.01	0.0050255170353086\\
95.01	0.00502552108550974\\
96.01	0.0050255252152184\\
97.01	0.00502552942600487\\
98.01	0.00502553371946982\\
99.01	0.00502553809724601\\
100.01	0.00502554256099869\\
101.01	0.00502554711242673\\
102.01	0.00502555175326247\\
103.01	0.00502555648527314\\
104.01	0.00502556131026061\\
105.01	0.00502556623006328\\
106.01	0.00502557124655656\\
107.01	0.00502557636165213\\
108.01	0.00502558157730099\\
109.01	0.00502558689549265\\
110.01	0.00502559231825647\\
111.01	0.00502559784766227\\
112.01	0.00502560348582148\\
113.01	0.00502560923488735\\
114.01	0.00502561509705665\\
115.01	0.00502562107456972\\
116.01	0.00502562716971211\\
117.01	0.00502563338481501\\
118.01	0.00502563972225592\\
119.01	0.0050256461844608\\
120.01	0.00502565277390346\\
121.01	0.00502565949310754\\
122.01	0.00502566634464752\\
123.01	0.00502567333114898\\
124.01	0.00502568045529087\\
125.01	0.00502568771980505\\
126.01	0.00502569512747891\\
127.01	0.00502570268115568\\
128.01	0.00502571038373564\\
129.01	0.00502571823817732\\
130.01	0.00502572624749856\\
131.01	0.00502573441477791\\
132.01	0.00502574274315543\\
133.01	0.00502575123583529\\
134.01	0.00502575989608575\\
135.01	0.00502576872724045\\
136.01	0.00502577773270074\\
137.01	0.00502578691593569\\
138.01	0.00502579628048477\\
139.01	0.00502580582995865\\
140.01	0.0050258155680408\\
141.01	0.0050258254984889\\
142.01	0.00502583562513637\\
143.01	0.00502584595189431\\
144.01	0.00502585648275183\\
145.01	0.00502586722177942\\
146.01	0.00502587817312949\\
147.01	0.00502588934103815\\
148.01	0.00502590072982748\\
149.01	0.00502591234390623\\
150.01	0.00502592418777278\\
151.01	0.00502593626601633\\
152.01	0.00502594858331887\\
153.01	0.00502596114445725\\
154.01	0.00502597395430432\\
155.01	0.00502598701783252\\
156.01	0.00502600034011449\\
157.01	0.00502601392632548\\
158.01	0.00502602778174588\\
159.01	0.00502604191176259\\
160.01	0.00502605632187225\\
161.01	0.00502607101768271\\
162.01	0.0050260860049155\\
163.01	0.00502610128940839\\
164.01	0.00502611687711763\\
165.01	0.00502613277412037\\
166.01	0.00502614898661696\\
167.01	0.00502616552093366\\
168.01	0.00502618238352512\\
169.01	0.00502619958097786\\
170.01	0.00502621712001197\\
171.01	0.00502623500748392\\
172.01	0.00502625325038977\\
173.01	0.00502627185586785\\
174.01	0.00502629083120179\\
175.01	0.0050263101838236\\
176.01	0.00502632992131696\\
177.01	0.00502635005141905\\
178.01	0.00502637058202549\\
179.01	0.00502639152119269\\
180.01	0.00502641287714084\\
181.01	0.00502643465825794\\
182.01	0.00502645687310335\\
183.01	0.00502647953041124\\
184.01	0.00502650263909296\\
185.01	0.00502652620824279\\
186.01	0.00502655024714027\\
187.01	0.00502657476525428\\
188.01	0.00502659977224754\\
189.01	0.00502662527797978\\
190.01	0.00502665129251281\\
191.01	0.00502667782611341\\
192.01	0.00502670488925874\\
193.01	0.00502673249264007\\
194.01	0.00502676064716748\\
195.01	0.00502678936397455\\
196.01	0.0050268186544222\\
197.01	0.00502684853010429\\
198.01	0.00502687900285209\\
199.01	0.00502691008473879\\
200.01	0.00502694178808552\\
201.01	0.00502697412546577\\
202.01	0.00502700710971057\\
203.01	0.00502704075391375\\
204.01	0.00502707507143804\\
205.01	0.00502711007592019\\
206.01	0.00502714578127678\\
207.01	0.00502718220170939\\
208.01	0.00502721935171177\\
209.01	0.00502725724607461\\
210.01	0.0050272958998929\\
211.01	0.00502733532857103\\
212.01	0.00502737554782998\\
213.01	0.00502741657371389\\
214.01	0.00502745842259641\\
215.01	0.00502750111118755\\
216.01	0.00502754465654134\\
217.01	0.0050275890760616\\
218.01	0.00502763438751083\\
219.01	0.00502768060901643\\
220.01	0.00502772775907867\\
221.01	0.00502777585657851\\
222.01	0.00502782492078555\\
223.01	0.00502787497136614\\
224.01	0.00502792602839108\\
225.01	0.00502797811234429\\
226.01	0.00502803124413168\\
227.01	0.00502808544508973\\
228.01	0.0050281407369944\\
229.01	0.00502819714206984\\
230.01	0.00502825468299856\\
231.01	0.00502831338293011\\
232.01	0.00502837326549152\\
233.01	0.0050284343547963\\
234.01	0.00502849667545505\\
235.01	0.00502856025258627\\
236.01	0.00502862511182556\\
237.01	0.00502869127933764\\
238.01	0.00502875878182693\\
239.01	0.00502882764654813\\
240.01	0.00502889790131839\\
241.01	0.00502896957452861\\
242.01	0.00502904269515575\\
243.01	0.0050291172927743\\
244.01	0.0050291933975688\\
245.01	0.00502927104034726\\
246.01	0.0050293502525528\\
247.01	0.00502943106627787\\
248.01	0.00502951351427761\\
249.01	0.00502959762998303\\
250.01	0.00502968344751517\\
251.01	0.00502977100169932\\
252.01	0.00502986032808012\\
253.01	0.00502995146293608\\
254.01	0.00503004444329483\\
255.01	0.00503013930694796\\
256.01	0.00503023609246817\\
257.01	0.00503033483922449\\
258.01	0.0050304355873991\\
259.01	0.00503053837800363\\
260.01	0.00503064325289665\\
261.01	0.00503075025480176\\
262.01	0.00503085942732462\\
263.01	0.00503097081497172\\
264.01	0.00503108446316855\\
265.01	0.00503120041827904\\
266.01	0.0050313187276243\\
267.01	0.00503143943950354\\
268.01	0.0050315626032135\\
269.01	0.00503168826906858\\
270.01	0.00503181648842269\\
271.01	0.00503194731369064\\
272.01	0.0050320807983697\\
273.01	0.00503221699706208\\
274.01	0.00503235596549806\\
275.01	0.00503249776055859\\
276.01	0.00503264244029958\\
277.01	0.00503279006397625\\
278.01	0.00503294069206728\\
279.01	0.0050330943863007\\
280.01	0.00503325120967907\\
281.01	0.0050334112265065\\
282.01	0.00503357450241435\\
283.01	0.00503374110439042\\
284.01	0.00503391110080476\\
285.01	0.00503408456144014\\
286.01	0.0050342615575199\\
287.01	0.00503444216173804\\
288.01	0.0050346264482895\\
289.01	0.00503481449290169\\
290.01	0.00503500637286478\\
291.01	0.00503520216706568\\
292.01	0.0050354019560189\\
293.01	0.00503560582190185\\
294.01	0.00503581384858895\\
295.01	0.00503602612168494\\
296.01	0.00503624272856248\\
297.01	0.00503646375839809\\
298.01	0.0050366893022086\\
299.01	0.00503691945289003\\
300.01	0.00503715430525531\\
301.01	0.00503739395607522\\
302.01	0.00503763850411706\\
303.01	0.00503788805018743\\
304.01	0.00503814269717304\\
305.01	0.0050384025500842\\
306.01	0.00503866771609884\\
307.01	0.00503893830460631\\
308.01	0.005039214427254\\
309.01	0.00503949619799313\\
310.01	0.00503978373312652\\
311.01	0.00504007715135721\\
312.01	0.00504037657383813\\
313.01	0.00504068212422235\\
314.01	0.0050409939287139\\
315.01	0.00504131211612241\\
316.01	0.00504163681791495\\
317.01	0.0050419681682714\\
318.01	0.00504230630414115\\
319.01	0.00504265136529913\\
320.01	0.00504300349440465\\
321.01	0.005043362837061\\
322.01	0.00504372954187635\\
323.01	0.00504410376052492\\
324.01	0.00504448564781135\\
325.01	0.00504487536173437\\
326.01	0.0050452730635529\\
327.01	0.00504567891785457\\
328.01	0.00504609309262244\\
329.01	0.00504651575930629\\
330.01	0.00504694709289338\\
331.01	0.00504738727198278\\
332.01	0.00504783647885829\\
333.01	0.00504829489956601\\
334.01	0.00504876272398996\\
335.01	0.00504924014593339\\
336.01	0.00504972736319814\\
337.01	0.00505022457766829\\
338.01	0.00505073199539322\\
339.01	0.00505124982667407\\
340.01	0.00505177828615133\\
341.01	0.00505231759289476\\
342.01	0.00505286797049365\\
343.01	0.0050534296471508\\
344.01	0.00505400285577759\\
345.01	0.00505458783409077\\
346.01	0.00505518482471143\\
347.01	0.00505579407526607\\
348.01	0.00505641583849006\\
349.01	0.00505705037233223\\
350.01	0.00505769794006256\\
351.01	0.00505835881038222\\
352.01	0.00505903325753451\\
353.01	0.00505972156141962\\
354.01	0.00506042400771045\\
355.01	0.00506114088797194\\
356.01	0.00506187249978191\\
357.01	0.00506261914685557\\
358.01	0.00506338113917071\\
359.01	0.00506415879309688\\
360.01	0.00506495243152727\\
361.01	0.00506576238401224\\
362.01	0.00506658898689749\\
363.01	0.00506743258346209\\
364.01	0.00506829352406188\\
365.01	0.00506917216627517\\
366.01	0.00507006887505074\\
367.01	0.00507098402286013\\
368.01	0.00507191798985146\\
369.01	0.00507287116400743\\
370.01	0.00507384394130747\\
371.01	0.00507483672589013\\
372.01	0.00507584993022273\\
373.01	0.00507688397527178\\
374.01	0.00507793929067735\\
375.01	0.00507901631493216\\
376.01	0.00508011549556292\\
377.01	0.00508123728931636\\
378.01	0.00508238216234905\\
379.01	0.00508355059042037\\
380.01	0.00508474305909051\\
381.01	0.00508596006392134\\
382.01	0.00508720211068256\\
383.01	0.00508846971556157\\
384.01	0.00508976340537737\\
385.01	0.00509108371779993\\
386.01	0.00509243120157252\\
387.01	0.00509380641673987\\
388.01	0.00509520993488072\\
389.01	0.0050966423393447\\
390.01	0.00509810422549539\\
391.01	0.00509959620095607\\
392.01	0.00510111888586348\\
393.01	0.00510267291312402\\
394.01	0.00510425892867744\\
395.01	0.00510587759176476\\
396.01	0.00510752957520158\\
397.01	0.00510921556565833\\
398.01	0.00511093626394493\\
399.01	0.00511269238530185\\
400.01	0.00511448465969741\\
401.01	0.00511631383213107\\
402.01	0.00511818066294276\\
403.01	0.0051200859281289\\
404.01	0.00512203041966497\\
405.01	0.00512401494583434\\
406.01	0.00512604033156455\\
407.01	0.00512810741876957\\
408.01	0.00513021706670055\\
409.01	0.00513237015230281\\
410.01	0.00513456757058009\\
411.01	0.00513681023496659\\
412.01	0.00513909907770765\\
413.01	0.00514143505024678\\
414.01	0.00514381912362155\\
415.01	0.0051462522888675\\
416.01	0.0051487355574302\\
417.01	0.00515126996158641\\
418.01	0.00515385655487299\\
419.01	0.00515649641252641\\
420.01	0.00515919063192849\\
421.01	0.005161940333065\\
422.01	0.00516474665899053\\
423.01	0.00516761077630448\\
424.01	0.00517053387563707\\
425.01	0.00517351717214504\\
426.01	0.00517656190601712\\
427.01	0.00517966934299052\\
428.01	0.00518284077487896\\
429.01	0.00518607752010913\\
430.01	0.00518938092427124\\
431.01	0.00519275236067859\\
432.01	0.00519619323094003\\
433.01	0.00519970496554423\\
434.01	0.00520328902445524\\
435.01	0.00520694689772078\\
436.01	0.00521068010609343\\
437.01	0.00521449020166515\\
438.01	0.00521837876851267\\
439.01	0.00522234742335928\\
440.01	0.00522639781624823\\
441.01	0.00523053163123074\\
442.01	0.00523475058706865\\
443.01	0.00523905643795085\\
444.01	0.00524345097422511\\
445.01	0.00524793602314479\\
446.01	0.00525251344963172\\
447.01	0.00525718515705385\\
448.01	0.00526195308801961\\
449.01	0.00526681922518889\\
450.01	0.0052717855921007\\
451.01	0.00527685425401767\\
452.01	0.00528202731878902\\
453.01	0.00528730693772994\\
454.01	0.00529269530652099\\
455.01	0.00529819466612468\\
456.01	0.00530380730372196\\
457.01	0.00530953555366792\\
458.01	0.00531538179846745\\
459.01	0.0053213484697711\\
460.01	0.00532743804939127\\
461.01	0.00533365307034027\\
462.01	0.00533999611788943\\
463.01	0.00534646983065033\\
464.01	0.00535307690167858\\
465.01	0.00535982007960048\\
466.01	0.0053667021697633\\
467.01	0.00537372603540943\\
468.01	0.00538089459887411\\
469.01	0.00538821084280963\\
470.01	0.00539567781143413\\
471.01	0.00540329861180545\\
472.01	0.00541107641512373\\
473.01	0.00541901445805865\\
474.01	0.00542711604410569\\
475.01	0.00543538454497058\\
476.01	0.00544382340198156\\
477.01	0.00545243612753163\\
478.01	0.00546122630655084\\
479.01	0.00547019759800861\\
480.01	0.00547935373644722\\
481.01	0.00548869853354755\\
482.01	0.00549823587972666\\
483.01	0.00550796974576846\\
484.01	0.00551790418448807\\
485.01	0.00552804333243061\\
486.01	0.00553839141160435\\
487.01	0.00554895273125022\\
488.01	0.00555973168964703\\
489.01	0.00557073277595338\\
490.01	0.00558196057208785\\
491.01	0.00559341975464639\\
492.01	0.00560511509685933\\
493.01	0.00561705147058765\\
494.01	0.00562923384835898\\
495.01	0.00564166730544473\\
496.01	0.00565435702197845\\
497.01	0.00566730828511622\\
498.01	0.00568052649123867\\
499.01	0.00569401714819741\\
500.01	0.00570778587760391\\
501.01	0.00572183841716205\\
502.01	0.0057361806230467\\
503.01	0.00575081847232464\\
504.01	0.0057657580654227\\
505.01	0.00578100562863958\\
506.01	0.00579656751670357\\
507.01	0.00581245021537605\\
508.01	0.00582866034409943\\
509.01	0.00584520465869111\\
510.01	0.00586209005408149\\
511.01	0.00587932356709678\\
512.01	0.00589691237928515\\
513.01	0.00591486381978583\\
514.01	0.00593318536823928\\
515.01	0.00595188465773874\\
516.01	0.00597096947781858\\
517.01	0.0059904477774809\\
518.01	0.00601032766825516\\
519.01	0.00603061742729019\\
520.01	0.0060513255004731\\
521.01	0.00607246050557363\\
522.01	0.0060940312354081\\
523.01	0.00611604666101788\\
524.01	0.00613851593485767\\
525.01	0.00616144839398609\\
526.01	0.00618485356325117\\
527.01	0.006208741158463\\
528.01	0.00623312108954262\\
529.01	0.00625800346363684\\
530.01	0.0062833985881868\\
531.01	0.00630931697393645\\
532.01	0.00633576933786364\\
533.01	0.00636276660601857\\
534.01	0.00639031991624708\\
535.01	0.00641844062077823\\
536.01	0.00644714028865036\\
537.01	0.00647643070794703\\
538.01	0.00650632388781151\\
539.01	0.00653683206020528\\
540.01	0.00656796768136911\\
541.01	0.00659974343294424\\
542.01	0.00663217222270325\\
543.01	0.00666526718483568\\
544.01	0.00669904167972642\\
545.01	0.00673350929315926\\
546.01	0.00676868383486742\\
547.01	0.00680457933634701\\
548.01	0.00684121004783805\\
549.01	0.00687859043436699\\
550.01	0.00691673517073471\\
551.01	0.0069556591353172\\
552.01	0.00699537740253582\\
553.01	0.00703590523383653\\
554.01	0.0070772580669986\\
555.01	0.00711945150357646\\
556.01	0.00716250129425623\\
557.01	0.00720642332188554\\
558.01	0.00725123358191065\\
559.01	0.00729694815992683\\
560.01	0.00734358320602013\\
561.01	0.00739115490554365\\
562.01	0.00743967944594207\\
563.01	0.00748917297919549\\
564.01	0.00753965157942145\\
565.01	0.00759113119512742\\
566.01	0.00764362759556816\\
567.01	0.00769715631061607\\
568.01	0.00775173256351049\\
569.01	0.0078073711958089\\
570.01	0.00786408658382248\\
571.01	0.00792189254578262\\
572.01	0.00798080223895675\\
573.01	0.00804082804591287\\
574.01	0.00810198144913145\\
575.01	0.00816427289318073\\
576.01	0.00822771163372017\\
577.01	0.00829230557268554\\
578.01	0.00835806107914579\\
579.01	0.00842498279553232\\
580.01	0.00849307342923551\\
581.01	0.00856233352997421\\
582.01	0.00863276125390358\\
583.01	0.00870435211617224\\
584.01	0.00877709873462935\\
585.01	0.00885099056867839\\
586.01	0.00892601365895706\\
587.01	0.00900215037570352\\
588.01	0.00907937918646837\\
589.01	0.0091576744574184\\
590.01	0.00923700630705446\\
591.01	0.00931734053698447\\
592.01	0.00939863867177418\\
593.01	0.00948085814924172\\
594.01	0.00956395271435918\\
595.01	0.00964787308480385\\
596.01	0.00973256797493054\\
597.01	0.0098179855884858\\
598.01	0.00990285876644176\\
599.01	0.00996919203046377\\
599.02	0.00996973314335038\\
599.03	0.0099702709571694\\
599.04	0.00997080543920336\\
599.05	0.0099713365564138\\
599.06	0.00997186427543808\\
599.07	0.0099723885625862\\
599.08	0.00997290938383756\\
599.09	0.00997342670483771\\
599.1	0.00997394049089505\\
599.11	0.00997445070697751\\
599.12	0.00997495731770918\\
599.13	0.00997546028736696\\
599.14	0.00997595957987707\\
599.15	0.00997645515881165\\
599.16	0.00997694698738526\\
599.17	0.00997743502845131\\
599.18	0.00997791924449856\\
599.19	0.00997839959764748\\
599.2	0.00997887604964662\\
599.21	0.00997934856186899\\
599.22	0.00997981709530829\\
599.23	0.00998028161057521\\
599.24	0.00998074206789365\\
599.25	0.0099811984270969\\
599.26	0.00998165064762378\\
599.27	0.00998209868851477\\
599.28	0.00998254250840806\\
599.29	0.00998298206553559\\
599.3	0.00998341731771905\\
599.31	0.00998384822236584\\
599.32	0.00998427473646497\\
599.33	0.00998469681658292\\
599.34	0.00998511441885952\\
599.35	0.0099855274990037\\
599.36	0.00998593601228926\\
599.37	0.00998633991355056\\
599.38	0.00998673915717821\\
599.39	0.00998713369711469\\
599.4	0.00998752348684992\\
599.41	0.00998790847811157\\
599.42	0.00998828861974491\\
599.43	0.00998866386008804\\
599.44	0.00998903414696681\\
599.45	0.00998939942768985\\
599.46	0.00998975964904344\\
599.47	0.00999011475728636\\
599.48	0.00999046469814472\\
599.49	0.00999080941680669\\
599.5	0.00999114885791725\\
599.51	0.00999148296557278\\
599.52	0.0099918116833157\\
599.53	0.00999213495412901\\
599.54	0.00999245272043077\\
599.55	0.00999276492406855\\
599.56	0.00999307150631382\\
599.57	0.00999337240785624\\
599.58	0.00999366756879799\\
599.59	0.00999395692864795\\
599.6	0.00999424042631586\\
599.61	0.00999451800010642\\
599.62	0.00999478958771336\\
599.63	0.0099950551262134\\
599.64	0.00999531455206019\\
599.65	0.00999556780107816\\
599.66	0.00999581480845634\\
599.67	0.00999605550874211\\
599.68	0.00999628983583487\\
599.69	0.00999651772297967\\
599.7	0.00999673910276076\\
599.71	0.00999695390709509\\
599.72	0.00999716206722577\\
599.73	0.00999736351371538\\
599.74	0.00999755817643933\\
599.75	0.00999774598457904\\
599.76	0.00999792686661517\\
599.77	0.00999810075032068\\
599.78	0.00999826756275386\\
599.79	0.00999842723025133\\
599.8	0.00999857967842092\\
599.81	0.0099987248321345\\
599.82	0.00999886261552071\\
599.83	0.00999899295195769\\
599.84	0.00999911576406567\\
599.85	0.00999923097369951\\
599.86	0.00999933850194117\\
599.87	0.00999943826909211\\
599.88	0.00999953019466558\\
599.89	0.00999961419737891\\
599.9	0.00999969019514566\\
599.91	0.00999975810506767\\
599.92	0.00999981784342713\\
599.93	0.00999986932567848\\
599.94	0.00999991246644028\\
599.95	0.00999994717948697\\
599.96	0.00999997337774056\\
599.97	0.00999999097326228\\
599.98	0.00999999987724406\\
599.99	0.01\\
600	0.01\\
};
\addplot [color=mycolor1,solid,forget plot]
  table[row sep=crcr]{%
0.01	0.00502505708478154\\
1.01	0.00502505776987136\\
2.01	0.00502505846833001\\
3.01	0.00502505918041851\\
4.01	0.00502505990640237\\
5.01	0.00502506064655234\\
6.01	0.00502506140114448\\
7.01	0.00502506217046036\\
8.01	0.00502506295478708\\
9.01	0.00502506375441727\\
10.01	0.0050250645696491\\
11.01	0.00502506540078659\\
12.01	0.00502506624814011\\
13.01	0.00502506711202561\\
14.01	0.00502506799276525\\
15.01	0.00502506889068779\\
16.01	0.00502506980612805\\
17.01	0.00502507073942768\\
18.01	0.00502507169093472\\
19.01	0.00502507266100416\\
20.01	0.00502507364999782\\
21.01	0.0050250746582847\\
22.01	0.00502507568624078\\
23.01	0.00502507673424957\\
24.01	0.0050250778027022\\
25.01	0.0050250788919973\\
26.01	0.00502508000254118\\
27.01	0.00502508113474812\\
28.01	0.00502508228904022\\
29.01	0.00502508346584857\\
30.01	0.00502508466561202\\
31.01	0.00502508588877826\\
32.01	0.00502508713580357\\
33.01	0.00502508840715338\\
34.01	0.00502508970330179\\
35.01	0.00502509102473275\\
36.01	0.00502509237193918\\
37.01	0.00502509374542376\\
38.01	0.0050250951456991\\
39.01	0.00502509657328767\\
40.01	0.00502509802872239\\
41.01	0.00502509951254626\\
42.01	0.00502510102531317\\
43.01	0.0050251025675877\\
44.01	0.00502510413994553\\
45.01	0.00502510574297341\\
46.01	0.00502510737726991\\
47.01	0.00502510904344494\\
48.01	0.00502511074212078\\
49.01	0.00502511247393153\\
50.01	0.00502511423952365\\
51.01	0.00502511603955635\\
52.01	0.00502511787470195\\
53.01	0.00502511974564554\\
54.01	0.00502512165308583\\
55.01	0.0050251235977353\\
56.01	0.00502512558032045\\
57.01	0.00502512760158177\\
58.01	0.00502512966227386\\
59.01	0.00502513176316688\\
60.01	0.00502513390504561\\
61.01	0.00502513608871036\\
62.01	0.00502513831497725\\
63.01	0.00502514058467798\\
64.01	0.00502514289866096\\
65.01	0.0050251452577908\\
66.01	0.00502514766294929\\
67.01	0.00502515011503596\\
68.01	0.00502515261496719\\
69.01	0.00502515516367757\\
70.01	0.00502515776212048\\
71.01	0.00502516041126738\\
72.01	0.0050251631121091\\
73.01	0.00502516586565591\\
74.01	0.00502516867293762\\
75.01	0.0050251715350047\\
76.01	0.00502517445292788\\
77.01	0.00502517742779912\\
78.01	0.00502518046073192\\
79.01	0.00502518355286122\\
80.01	0.00502518670534463\\
81.01	0.00502518991936224\\
82.01	0.00502519319611752\\
83.01	0.00502519653683742\\
84.01	0.0050251999427732\\
85.01	0.00502520341520069\\
86.01	0.00502520695542052\\
87.01	0.00502521056475919\\
88.01	0.00502521424456948\\
89.01	0.00502521799622993\\
90.01	0.00502522182114742\\
91.01	0.00502522572075539\\
92.01	0.00502522969651603\\
93.01	0.00502523374992067\\
94.01	0.00502523788248893\\
95.01	0.00502524209577135\\
96.01	0.00502524639134888\\
97.01	0.00502525077083307\\
98.01	0.00502525523586747\\
99.01	0.00502525978812848\\
100.01	0.00502526442932494\\
101.01	0.00502526916119955\\
102.01	0.00502527398552965\\
103.01	0.00502527890412721\\
104.01	0.0050252839188403\\
105.01	0.00502528903155341\\
106.01	0.00502529424418774\\
107.01	0.00502529955870326\\
108.01	0.00502530497709808\\
109.01	0.00502531050140967\\
110.01	0.00502531613371601\\
111.01	0.00502532187613616\\
112.01	0.00502532773083089\\
113.01	0.00502533370000388\\
114.01	0.00502533978590211\\
115.01	0.00502534599081714\\
116.01	0.00502535231708595\\
117.01	0.00502535876709181\\
118.01	0.00502536534326498\\
119.01	0.00502537204808393\\
120.01	0.005025378884076\\
121.01	0.00502538585381879\\
122.01	0.00502539295994089\\
123.01	0.00502540020512304\\
124.01	0.00502540759209896\\
125.01	0.0050254151236567\\
126.01	0.00502542280263979\\
127.01	0.0050254306319476\\
128.01	0.00502543861453753\\
129.01	0.00502544675342539\\
130.01	0.00502545505168694\\
131.01	0.00502546351245933\\
132.01	0.00502547213894153\\
133.01	0.00502548093439648\\
134.01	0.00502548990215143\\
135.01	0.00502549904560015\\
136.01	0.00502550836820367\\
137.01	0.0050255178734924\\
138.01	0.00502552756506625\\
139.01	0.00502553744659737\\
140.01	0.00502554752183065\\
141.01	0.00502555779458538\\
142.01	0.00502556826875752\\
143.01	0.00502557894831987\\
144.01	0.0050255898373251\\
145.01	0.00502560093990664\\
146.01	0.00502561226027967\\
147.01	0.00502562380274427\\
148.01	0.00502563557168551\\
149.01	0.00502564757157675\\
150.01	0.00502565980698014\\
151.01	0.00502567228254919\\
152.01	0.00502568500303027\\
153.01	0.00502569797326444\\
154.01	0.00502571119819006\\
155.01	0.00502572468284337\\
156.01	0.00502573843236212\\
157.01	0.0050257524519868\\
158.01	0.00502576674706218\\
159.01	0.00502578132304063\\
160.01	0.00502579618548352\\
161.01	0.00502581134006361\\
162.01	0.00502582679256727\\
163.01	0.00502584254889701\\
164.01	0.00502585861507368\\
165.01	0.00502587499723875\\
166.01	0.00502589170165724\\
167.01	0.00502590873471953\\
168.01	0.00502592610294486\\
169.01	0.00502594381298288\\
170.01	0.00502596187161705\\
171.01	0.00502598028576712\\
172.01	0.0050259990624921\\
173.01	0.00502601820899304\\
174.01	0.00502603773261556\\
175.01	0.00502605764085359\\
176.01	0.00502607794135135\\
177.01	0.00502609864190751\\
178.01	0.00502611975047776\\
179.01	0.00502614127517792\\
180.01	0.00502616322428776\\
181.01	0.00502618560625406\\
182.01	0.00502620842969345\\
183.01	0.00502623170339703\\
184.01	0.00502625543633339\\
185.01	0.00502627963765204\\
186.01	0.00502630431668719\\
187.01	0.00502632948296225\\
188.01	0.0050263551461926\\
189.01	0.00502638131629046\\
190.01	0.00502640800336845\\
191.01	0.00502643521774387\\
192.01	0.00502646296994291\\
193.01	0.00502649127070517\\
194.01	0.00502652013098762\\
195.01	0.00502654956196918\\
196.01	0.00502657957505573\\
197.01	0.00502661018188422\\
198.01	0.00502664139432787\\
199.01	0.00502667322450087\\
200.01	0.00502670568476324\\
201.01	0.00502673878772611\\
202.01	0.00502677254625666\\
203.01	0.00502680697348418\\
204.01	0.00502684208280435\\
205.01	0.00502687788788525\\
206.01	0.00502691440267317\\
207.01	0.00502695164139846\\
208.01	0.00502698961858103\\
209.01	0.0050270283490364\\
210.01	0.00502706784788198\\
211.01	0.00502710813054341\\
212.01	0.00502714921276085\\
213.01	0.00502719111059502\\
214.01	0.0050272338404346\\
215.01	0.00502727741900305\\
216.01	0.00502732186336444\\
217.01	0.005027367190932\\
218.01	0.00502741341947435\\
219.01	0.00502746056712336\\
220.01	0.00502750865238133\\
221.01	0.00502755769412885\\
222.01	0.00502760771163299\\
223.01	0.00502765872455443\\
224.01	0.0050277107529568\\
225.01	0.00502776381731429\\
226.01	0.00502781793852039\\
227.01	0.00502787313789665\\
228.01	0.00502792943720123\\
229.01	0.00502798685863878\\
230.01	0.0050280454248687\\
231.01	0.00502810515901558\\
232.01	0.00502816608467785\\
233.01	0.00502822822593832\\
234.01	0.00502829160737461\\
235.01	0.00502835625406842\\
236.01	0.00502842219161706\\
237.01	0.00502848944614304\\
238.01	0.00502855804430619\\
239.01	0.00502862801331419\\
240.01	0.00502869938093389\\
241.01	0.00502877217550329\\
242.01	0.00502884642594303\\
243.01	0.00502892216176901\\
244.01	0.00502899941310453\\
245.01	0.00502907821069254\\
246.01	0.0050291585859096\\
247.01	0.00502924057077767\\
248.01	0.00502932419797852\\
249.01	0.00502940950086742\\
250.01	0.0050294965134865\\
251.01	0.00502958527058009\\
252.01	0.00502967580760794\\
253.01	0.00502976816076126\\
254.01	0.00502986236697753\\
255.01	0.00502995846395607\\
256.01	0.00503005649017382\\
257.01	0.00503015648490135\\
258.01	0.00503025848821978\\
259.01	0.0050303625410372\\
260.01	0.00503046868510645\\
261.01	0.00503057696304171\\
262.01	0.00503068741833726\\
263.01	0.00503080009538562\\
264.01	0.00503091503949554\\
265.01	0.00503103229691196\\
266.01	0.00503115191483511\\
267.01	0.00503127394143976\\
268.01	0.00503139842589623\\
269.01	0.00503152541839064\\
270.01	0.00503165497014652\\
271.01	0.00503178713344521\\
272.01	0.00503192196164903\\
273.01	0.00503205950922278\\
274.01	0.00503219983175696\\
275.01	0.00503234298599126\\
276.01	0.00503248902983819\\
277.01	0.0050326380224073\\
278.01	0.00503279002402979\\
279.01	0.00503294509628419\\
280.01	0.00503310330202217\\
281.01	0.00503326470539428\\
282.01	0.00503342937187773\\
283.01	0.00503359736830303\\
284.01	0.00503376876288231\\
285.01	0.00503394362523771\\
286.01	0.00503412202643089\\
287.01	0.0050343040389923\\
288.01	0.00503448973695184\\
289.01	0.00503467919586953\\
290.01	0.00503487249286755\\
291.01	0.00503506970666168\\
292.01	0.00503527091759529\\
293.01	0.00503547620767214\\
294.01	0.00503568566059007\\
295.01	0.0050358993617779\\
296.01	0.00503611739842902\\
297.01	0.00503633985953877\\
298.01	0.0050365668359417\\
299.01	0.00503679842034899\\
300.01	0.00503703470738753\\
301.01	0.00503727579363877\\
302.01	0.00503752177767999\\
303.01	0.00503777276012465\\
304.01	0.00503802884366436\\
305.01	0.00503829013311213\\
306.01	0.00503855673544537\\
307.01	0.00503882875985134\\
308.01	0.00503910631777216\\
309.01	0.00503938952295147\\
310.01	0.00503967849148132\\
311.01	0.00503997334185133\\
312.01	0.00504027419499739\\
313.01	0.00504058117435266\\
314.01	0.00504089440589889\\
315.01	0.00504121401821822\\
316.01	0.0050415401425481\\
317.01	0.005041872912835\\
318.01	0.00504221246579045\\
319.01	0.00504255894094829\\
320.01	0.00504291248072317\\
321.01	0.00504327323046944\\
322.01	0.0050436413385419\\
323.01	0.00504401695635803\\
324.01	0.00504440023846103\\
325.01	0.00504479134258454\\
326.01	0.00504519042971829\\
327.01	0.00504559766417539\\
328.01	0.00504601321366138\\
329.01	0.00504643724934425\\
330.01	0.00504686994592582\\
331.01	0.00504731148171481\\
332.01	0.00504776203870178\\
333.01	0.00504822180263492\\
334.01	0.00504869096309848\\
335.01	0.00504916971359129\\
336.01	0.00504965825160864\\
337.01	0.00505015677872389\\
338.01	0.00505066550067459\\
339.01	0.00505118462744747\\
340.01	0.00505171437336652\\
341.01	0.00505225495718289\\
342.01	0.00505280660216778\\
343.01	0.00505336953620408\\
344.01	0.00505394399188328\\
345.01	0.00505453020660221\\
346.01	0.00505512842266326\\
347.01	0.00505573888737543\\
348.01	0.00505636185315794\\
349.01	0.00505699757764599\\
350.01	0.00505764632379935\\
351.01	0.00505830836001134\\
352.01	0.00505898396022197\\
353.01	0.00505967340403249\\
354.01	0.00506037697682216\\
355.01	0.0050610949698679\\
356.01	0.0050618276804657\\
357.01	0.00506257541205518\\
358.01	0.00506333847434595\\
359.01	0.00506411718344768\\
360.01	0.00506491186200134\\
361.01	0.00506572283931435\\
362.01	0.00506655045149716\\
363.01	0.00506739504160418\\
364.01	0.00506825695977667\\
365.01	0.00506913656338806\\
366.01	0.00507003421719339\\
367.01	0.00507095029348086\\
368.01	0.00507188517222661\\
369.01	0.0050728392412534\\
370.01	0.00507381289639131\\
371.01	0.00507480654164263\\
372.01	0.00507582058934995\\
373.01	0.00507685546036712\\
374.01	0.0050779115842346\\
375.01	0.00507898939935726\\
376.01	0.0050800893531875\\
377.01	0.00508121190241012\\
378.01	0.00508235751313217\\
379.01	0.00508352666107692\\
380.01	0.00508471983178049\\
381.01	0.00508593752079405\\
382.01	0.0050871802338892\\
383.01	0.00508844848726744\\
384.01	0.0050897428077748\\
385.01	0.00509106373312024\\
386.01	0.00509241181209835\\
387.01	0.00509378760481723\\
388.01	0.00509519168293052\\
389.01	0.00509662462987447\\
390.01	0.00509808704111012\\
391.01	0.00509957952436935\\
392.01	0.00510110269990748\\
393.01	0.00510265720076039\\
394.01	0.00510424367300651\\
395.01	0.00510586277603481\\
396.01	0.00510751518281828\\
397.01	0.0051092015801926\\
398.01	0.00511092266914112\\
399.01	0.00511267916508518\\
400.01	0.00511447179818102\\
401.01	0.00511630131362208\\
402.01	0.00511816847194851\\
403.01	0.00512007404936213\\
404.01	0.00512201883804843\\
405.01	0.00512400364650505\\
406.01	0.00512602929987734\\
407.01	0.00512809664030063\\
408.01	0.00513020652724942\\
409.01	0.00513235983789442\\
410.01	0.00513455746746602\\
411.01	0.00513680032962638\\
412.01	0.00513908935684861\\
413.01	0.00514142550080382\\
414.01	0.00514380973275642\\
415.01	0.00514624304396753\\
416.01	0.00514872644610682\\
417.01	0.00515126097167247\\
418.01	0.00515384767442111\\
419.01	0.00515648762980477\\
420.01	0.00515918193541862\\
421.01	0.00516193171145674\\
422.01	0.00516473810117871\\
423.01	0.00516760227138448\\
424.01	0.0051705254129001\\
425.01	0.00517350874107243\\
426.01	0.00517655349627588\\
427.01	0.00517966094442791\\
428.01	0.00518283237751574\\
429.01	0.00518606911413511\\
430.01	0.00518937250003826\\
431.01	0.00519274390869479\\
432.01	0.00519618474186347\\
433.01	0.00519969643017593\\
434.01	0.00520328043373308\\
435.01	0.00520693824271272\\
436.01	0.00521067137799075\\
437.01	0.00521448139177509\\
438.01	0.00521836986825194\\
439.01	0.00522233842424669\\
440.01	0.0052263887098974\\
441.01	0.0052305224093428\\
442.01	0.00523474124142474\\
443.01	0.00523904696040461\\
444.01	0.00524344135669504\\
445.01	0.00524792625760664\\
446.01	0.00525250352811009\\
447.01	0.00525717507161445\\
448.01	0.00526194283076127\\
449.01	0.00526680878823532\\
450.01	0.005271774967592\\
451.01	0.00527684343410191\\
452.01	0.00528201629561358\\
453.01	0.00528729570343331\\
454.01	0.00529268385322357\\
455.01	0.00529818298592039\\
456.01	0.00530379538866937\\
457.01	0.00530952339578142\\
458.01	0.00531536938970828\\
459.01	0.00532133580203822\\
460.01	0.0053274251145128\\
461.01	0.00533363986006455\\
462.01	0.00533998262387593\\
463.01	0.00534645604446064\\
464.01	0.00535306281476762\\
465.01	0.00535980568330775\\
466.01	0.00536668745530369\\
467.01	0.00537371099386401\\
468.01	0.00538087922118197\\
469.01	0.00538819511975855\\
470.01	0.00539566173365116\\
471.01	0.00540328216974893\\
472.01	0.00541105959907338\\
473.01	0.00541899725810693\\
474.01	0.00542709845014895\\
475.01	0.00543536654669927\\
476.01	0.00544380498887161\\
477.01	0.005452417288835\\
478.01	0.00546120703128584\\
479.01	0.00547017787495054\\
480.01	0.00547933355411916\\
481.01	0.00548867788021037\\
482.01	0.00549821474336881\\
483.01	0.00550794811409631\\
484.01	0.00551788204491581\\
485.01	0.00552802067206939\\
486.01	0.00553836821725223\\
487.01	0.00554892898938124\\
488.01	0.00555970738639985\\
489.01	0.00557070789712073\\
490.01	0.00558193510310437\\
491.01	0.00559339368057722\\
492.01	0.00560508840238808\\
493.01	0.00561702414000388\\
494.01	0.00562920586554576\\
495.01	0.00564163865386569\\
496.01	0.00565432768466423\\
497.01	0.00566727824465042\\
498.01	0.00568049572974454\\
499.01	0.00569398564732272\\
500.01	0.00570775361850585\\
501.01	0.00572180538049252\\
502.01	0.00573614678893519\\
503.01	0.00575078382036302\\
504.01	0.00576572257464746\\
505.01	0.00578096927751429\\
506.01	0.00579653028310122\\
507.01	0.00581241207655984\\
508.01	0.0058286212767039\\
509.01	0.00584516463870154\\
510.01	0.0058620490568133\\
511.01	0.00587928156717432\\
512.01	0.0058968693506196\\
513.01	0.00591481973555158\\
514.01	0.00593314020085063\\
515.01	0.00595183837882469\\
516.01	0.00597092205819774\\
517.01	0.00599039918713468\\
518.01	0.00601027787630053\\
519.01	0.00603056640195079\\
520.01	0.00605127320905013\\
521.01	0.00607240691441507\\
522.01	0.00609397630987676\\
523.01	0.00611599036545887\\
524.01	0.00613845823256408\\
525.01	0.00616138924716375\\
526.01	0.00618479293298224\\
527.01	0.00620867900466769\\
528.01	0.0062330573709399\\
529.01	0.00625793813770395\\
530.01	0.00628333161111733\\
531.01	0.00630924830059647\\
532.01	0.00633569892174679\\
533.01	0.00636269439919962\\
534.01	0.00639024586933368\\
535.01	0.00641836468286175\\
536.01	0.0064470624072543\\
537.01	0.00647635082897466\\
538.01	0.00650624195549183\\
539.01	0.00653674801703709\\
540.01	0.00656788146806421\\
541.01	0.00659965498836909\\
542.01	0.0066320814838192\\
543.01	0.00666517408663819\\
544.01	0.0066989461551837\\
545.01	0.00673341127314891\\
546.01	0.0067685832481139\\
547.01	0.0068044761093578\\
548.01	0.00684110410483998\\
549.01	0.00687848169724375\\
550.01	0.0069166235589634\\
551.01	0.00695554456590763\\
552.01	0.00699525978997213\\
553.01	0.00703578449001997\\
554.01	0.00707713410119494\\
555.01	0.00711932422236754\\
556.01	0.00716237060149713\\
557.01	0.00720628911866866\\
558.01	0.00725109576653823\\
559.01	0.00729680662789368\\
560.01	0.00734343785000768\\
561.01	0.00739100561542945\\
562.01	0.00743952610882453\\
563.01	0.00748901547943973\\
564.01	0.00753948979872679\\
565.01	0.00759096501262174\\
566.01	0.00764345688793244\\
567.01	0.00769698095224389\\
568.01	0.0077515524267073\\
569.01	0.00780718615103681\\
570.01	0.00786389649999601\\
571.01	0.00792169729062253\\
572.01	0.00798060167940823\\
573.01	0.00804062204863615\\
574.01	0.00810176988107205\\
575.01	0.00816405562222705\\
576.01	0.00822748852945653\\
577.01	0.00829207650724693\\
578.01	0.00835782592818142\\
579.01	0.0084247414392816\\
580.01	0.00849282575371788\\
581.01	0.0085620794282911\\
582.01	0.00863250062764312\\
583.01	0.00870408487690098\\
584.01	0.00877682480544311\\
585.01	0.00885070988576821\\
586.01	0.00892572617312741\\
587.01	0.00900185605375053\\
588.01	0.00907907801228958\\
589.01	0.00915736643267762\\
590.01	0.00923669145116018\\
591.01	0.00931701888605708\\
592.01	0.00939831027616845\\
593.01	0.00948052306904656\\
594.01	0.00956361101211311\\
595.01	0.00964752481442215\\
596.01	0.00973221316552893\\
597.01	0.00981762422138093\\
598.01	0.00990285624732339\\
599.01	0.00996919203046377\\
599.02	0.00996973314335038\\
599.03	0.0099702709571694\\
599.04	0.00997080543920336\\
599.05	0.0099713365564138\\
599.06	0.00997186427543808\\
599.07	0.0099723885625862\\
599.08	0.00997290938383756\\
599.09	0.00997342670483771\\
599.1	0.00997394049089505\\
599.11	0.00997445070697751\\
599.12	0.00997495731770918\\
599.13	0.00997546028736696\\
599.14	0.00997595957987707\\
599.15	0.00997645515881165\\
599.16	0.00997694698738526\\
599.17	0.00997743502845131\\
599.18	0.00997791924449856\\
599.19	0.00997839959764748\\
599.2	0.00997887604964662\\
599.21	0.00997934856186899\\
599.22	0.00997981709530829\\
599.23	0.00998028161057521\\
599.24	0.00998074206789365\\
599.25	0.0099811984270969\\
599.26	0.00998165064762378\\
599.27	0.00998209868851477\\
599.28	0.00998254250840806\\
599.29	0.00998298206553559\\
599.3	0.00998341731771905\\
599.31	0.00998384822236584\\
599.32	0.00998427473646497\\
599.33	0.00998469681658292\\
599.34	0.00998511441885952\\
599.35	0.0099855274990037\\
599.36	0.00998593601228926\\
599.37	0.00998633991355056\\
599.38	0.00998673915717821\\
599.39	0.00998713369711469\\
599.4	0.00998752348684992\\
599.41	0.00998790847811156\\
599.42	0.00998828861974491\\
599.43	0.00998866386008803\\
599.44	0.00998903414696681\\
599.45	0.00998939942768985\\
599.46	0.00998975964904344\\
599.47	0.00999011475728636\\
599.48	0.00999046469814471\\
599.49	0.00999080941680669\\
599.5	0.00999114885791725\\
599.51	0.00999148296557278\\
599.52	0.0099918116833157\\
599.53	0.00999213495412901\\
599.54	0.00999245272043077\\
599.55	0.00999276492406855\\
599.56	0.00999307150631381\\
599.57	0.00999337240785624\\
599.58	0.00999366756879799\\
599.59	0.00999395692864795\\
599.6	0.00999424042631585\\
599.61	0.00999451800010642\\
599.62	0.00999478958771336\\
599.63	0.0099950551262134\\
599.64	0.00999531455206019\\
599.65	0.00999556780107816\\
599.66	0.00999581480845634\\
599.67	0.00999605550874211\\
599.68	0.00999628983583487\\
599.69	0.00999651772297967\\
599.7	0.00999673910276076\\
599.71	0.00999695390709509\\
599.72	0.00999716206722577\\
599.73	0.00999736351371538\\
599.74	0.00999755817643933\\
599.75	0.00999774598457904\\
599.76	0.00999792686661517\\
599.77	0.00999810075032068\\
599.78	0.00999826756275386\\
599.79	0.00999842723025133\\
599.8	0.00999857967842092\\
599.81	0.0099987248321345\\
599.82	0.00999886261552071\\
599.83	0.00999899295195769\\
599.84	0.00999911576406567\\
599.85	0.00999923097369951\\
599.86	0.00999933850194117\\
599.87	0.00999943826909211\\
599.88	0.00999953019466558\\
599.89	0.00999961419737891\\
599.9	0.00999969019514566\\
599.91	0.00999975810506767\\
599.92	0.00999981784342713\\
599.93	0.00999986932567848\\
599.94	0.00999991246644028\\
599.95	0.00999994717948697\\
599.96	0.00999997337774056\\
599.97	0.00999999097326228\\
599.98	0.00999999987724406\\
599.99	0.01\\
600	0.01\\
};
\addplot [color=mycolor2,solid,forget plot]
  table[row sep=crcr]{%
0.01	0.00502433004032531\\
1.01	0.00502433078290415\\
2.01	0.00502433153996987\\
3.01	0.00502433231180427\\
4.01	0.00502433309869423\\
5.01	0.00502433390093257\\
6.01	0.00502433471881732\\
7.01	0.00502433555265265\\
8.01	0.00502433640274838\\
9.01	0.00502433726942045\\
10.01	0.00502433815299091\\
11.01	0.00502433905378807\\
12.01	0.00502433997214664\\
13.01	0.0050243409084075\\
14.01	0.00502434186291843\\
15.01	0.00502434283603398\\
16.01	0.00502434382811538\\
17.01	0.00502434483953081\\
18.01	0.00502434587065569\\
19.01	0.00502434692187287\\
20.01	0.00502434799357236\\
21.01	0.00502434908615175\\
22.01	0.00502435020001639\\
23.01	0.00502435133557945\\
24.01	0.00502435249326209\\
25.01	0.00502435367349358\\
26.01	0.00502435487671149\\
27.01	0.0050243561033619\\
28.01	0.00502435735389965\\
29.01	0.00502435862878807\\
30.01	0.00502435992849977\\
31.01	0.00502436125351623\\
32.01	0.00502436260432844\\
33.01	0.00502436398143678\\
34.01	0.00502436538535125\\
35.01	0.00502436681659189\\
36.01	0.00502436827568868\\
37.01	0.00502436976318171\\
38.01	0.00502437127962201\\
39.01	0.00502437282557061\\
40.01	0.00502437440159964\\
41.01	0.00502437600829278\\
42.01	0.00502437764624425\\
43.01	0.00502437931606002\\
44.01	0.00502438101835799\\
45.01	0.00502438275376783\\
46.01	0.00502438452293126\\
47.01	0.00502438632650271\\
48.01	0.00502438816514894\\
49.01	0.00502439003954984\\
50.01	0.0050243919503983\\
51.01	0.00502439389840066\\
52.01	0.00502439588427681\\
53.01	0.00502439790876083\\
54.01	0.00502439997260064\\
55.01	0.00502440207655885\\
56.01	0.00502440422141273\\
57.01	0.00502440640795451\\
58.01	0.00502440863699193\\
59.01	0.00502441090934809\\
60.01	0.00502441322586211\\
61.01	0.00502441558738942\\
62.01	0.0050244179948018\\
63.01	0.00502442044898818\\
64.01	0.00502442295085417\\
65.01	0.00502442550132326\\
66.01	0.00502442810133673\\
67.01	0.00502443075185414\\
68.01	0.00502443345385328\\
69.01	0.00502443620833122\\
70.01	0.00502443901630395\\
71.01	0.00502444187880739\\
72.01	0.00502444479689727\\
73.01	0.00502444777164984\\
74.01	0.00502445080416218\\
75.01	0.00502445389555228\\
76.01	0.00502445704696021\\
77.01	0.00502446025954768\\
78.01	0.00502446353449897\\
79.01	0.00502446687302142\\
80.01	0.00502447027634557\\
81.01	0.00502447374572551\\
82.01	0.00502447728244004\\
83.01	0.00502448088779259\\
84.01	0.00502448456311146\\
85.01	0.00502448830975082\\
86.01	0.0050244921290912\\
87.01	0.00502449602253954\\
88.01	0.00502449999152989\\
89.01	0.00502450403752445\\
90.01	0.00502450816201346\\
91.01	0.00502451236651598\\
92.01	0.00502451665258054\\
93.01	0.0050245210217854\\
94.01	0.00502452547573952\\
95.01	0.00502453001608305\\
96.01	0.00502453464448767\\
97.01	0.0050245393626575\\
98.01	0.00502454417232982\\
99.01	0.00502454907527516\\
100.01	0.00502455407329852\\
101.01	0.00502455916823986\\
102.01	0.00502456436197452\\
103.01	0.00502456965641446\\
104.01	0.00502457505350838\\
105.01	0.00502458055524285\\
106.01	0.00502458616364285\\
107.01	0.00502459188077253\\
108.01	0.00502459770873598\\
109.01	0.00502460364967827\\
110.01	0.00502460970578573\\
111.01	0.00502461587928722\\
112.01	0.00502462217245472\\
113.01	0.00502462858760408\\
114.01	0.00502463512709636\\
115.01	0.00502464179333817\\
116.01	0.00502464858878258\\
117.01	0.0050246555159305\\
118.01	0.00502466257733118\\
119.01	0.00502466977558323\\
120.01	0.00502467711333579\\
121.01	0.00502468459328946\\
122.01	0.00502469221819687\\
123.01	0.00502469999086443\\
124.01	0.0050247079141528\\
125.01	0.0050247159909782\\
126.01	0.00502472422431341\\
127.01	0.0050247326171889\\
128.01	0.0050247411726939\\
129.01	0.00502474989397785\\
130.01	0.00502475878425128\\
131.01	0.00502476784678701\\
132.01	0.00502477708492131\\
133.01	0.00502478650205564\\
134.01	0.00502479610165719\\
135.01	0.00502480588726081\\
136.01	0.0050248158624699\\
137.01	0.00502482603095807\\
138.01	0.00502483639647029\\
139.01	0.00502484696282447\\
140.01	0.00502485773391259\\
141.01	0.00502486871370277\\
142.01	0.00502487990624001\\
143.01	0.00502489131564829\\
144.01	0.00502490294613174\\
145.01	0.00502491480197648\\
146.01	0.00502492688755225\\
147.01	0.00502493920731371\\
148.01	0.00502495176580255\\
149.01	0.00502496456764895\\
150.01	0.00502497761757341\\
151.01	0.00502499092038851\\
152.01	0.00502500448100068\\
153.01	0.00502501830441233\\
154.01	0.00502503239572324\\
155.01	0.00502504676013321\\
156.01	0.0050250614029432\\
157.01	0.00502507632955788\\
158.01	0.00502509154548772\\
159.01	0.0050251070563508\\
160.01	0.00502512286787522\\
161.01	0.0050251389859009\\
162.01	0.00502515541638252\\
163.01	0.00502517216539074\\
164.01	0.00502518923911544\\
165.01	0.0050252066438678\\
166.01	0.00502522438608252\\
167.01	0.00502524247232077\\
168.01	0.00502526090927197\\
169.01	0.00502527970375694\\
170.01	0.00502529886273037\\
171.01	0.00502531839328352\\
172.01	0.00502533830264698\\
173.01	0.00502535859819302\\
174.01	0.00502537928743922\\
175.01	0.00502540037805077\\
176.01	0.00502542187784366\\
177.01	0.0050254437947877\\
178.01	0.00502546613700938\\
179.01	0.0050254889127955\\
180.01	0.00502551213059595\\
181.01	0.00502553579902699\\
182.01	0.00502555992687535\\
183.01	0.00502558452310024\\
184.01	0.00502560959683822\\
185.01	0.00502563515740609\\
186.01	0.00502566121430463\\
187.01	0.00502568777722227\\
188.01	0.00502571485603907\\
189.01	0.00502574246083047\\
190.01	0.00502577060187115\\
191.01	0.00502579928963928\\
192.01	0.00502582853482043\\
193.01	0.00502585834831201\\
194.01	0.00502588874122717\\
195.01	0.00502591972489998\\
196.01	0.00502595131088869\\
197.01	0.00502598351098166\\
198.01	0.00502601633720089\\
199.01	0.00502604980180751\\
200.01	0.00502608391730629\\
201.01	0.00502611869645081\\
202.01	0.0050261541522485\\
203.01	0.00502619029796553\\
204.01	0.0050262271471328\\
205.01	0.00502626471355053\\
206.01	0.00502630301129413\\
207.01	0.00502634205472018\\
208.01	0.00502638185847148\\
209.01	0.00502642243748359\\
210.01	0.00502646380699059\\
211.01	0.00502650598253092\\
212.01	0.00502654897995404\\
213.01	0.00502659281542676\\
214.01	0.00502663750543959\\
215.01	0.0050266830668135\\
216.01	0.00502672951670692\\
217.01	0.00502677687262224\\
218.01	0.00502682515241343\\
219.01	0.00502687437429299\\
220.01	0.00502692455683951\\
221.01	0.00502697571900508\\
222.01	0.00502702788012314\\
223.01	0.00502708105991628\\
224.01	0.00502713527850461\\
225.01	0.00502719055641328\\
226.01	0.00502724691458187\\
227.01	0.00502730437437221\\
228.01	0.00502736295757766\\
229.01	0.00502742268643143\\
230.01	0.00502748358361636\\
231.01	0.00502754567227415\\
232.01	0.00502760897601477\\
233.01	0.00502767351892637\\
234.01	0.00502773932558503\\
235.01	0.005027806421065\\
236.01	0.00502787483094894\\
237.01	0.00502794458133929\\
238.01	0.00502801569886797\\
239.01	0.00502808821070845\\
240.01	0.00502816214458639\\
241.01	0.0050282375287917\\
242.01	0.00502831439218987\\
243.01	0.00502839276423442\\
244.01	0.00502847267497887\\
245.01	0.00502855415508958\\
246.01	0.00502863723585828\\
247.01	0.00502872194921569\\
248.01	0.00502880832774445\\
249.01	0.00502889640469281\\
250.01	0.00502898621398901\\
251.01	0.00502907779025501\\
252.01	0.00502917116882139\\
253.01	0.00502926638574215\\
254.01	0.00502936347780983\\
255.01	0.00502946248257113\\
256.01	0.00502956343834272\\
257.01	0.00502966638422728\\
258.01	0.00502977136013015\\
259.01	0.00502987840677609\\
260.01	0.00502998756572667\\
261.01	0.00503009887939768\\
262.01	0.00503021239107712\\
263.01	0.00503032814494354\\
264.01	0.00503044618608481\\
265.01	0.00503056656051706\\
266.01	0.00503068931520437\\
267.01	0.00503081449807827\\
268.01	0.00503094215805883\\
269.01	0.00503107234507448\\
270.01	0.00503120511008398\\
271.01	0.00503134050509758\\
272.01	0.00503147858319909\\
273.01	0.00503161939856864\\
274.01	0.00503176300650546\\
275.01	0.00503190946345133\\
276.01	0.00503205882701463\\
277.01	0.00503221115599456\\
278.01	0.00503236651040657\\
279.01	0.0050325249515072\\
280.01	0.00503268654182049\\
281.01	0.00503285134516392\\
282.01	0.00503301942667632\\
283.01	0.00503319085284439\\
284.01	0.005033365691532\\
285.01	0.00503354401200766\\
286.01	0.00503372588497485\\
287.01	0.00503391138260131\\
288.01	0.00503410057854997\\
289.01	0.00503429354800942\\
290.01	0.00503449036772634\\
291.01	0.00503469111603755\\
292.01	0.00503489587290304\\
293.01	0.0050351047199394\\
294.01	0.00503531774045483\\
295.01	0.00503553501948349\\
296.01	0.00503575664382178\\
297.01	0.00503598270206433\\
298.01	0.0050362132846417\\
299.01	0.00503644848385791\\
300.01	0.00503668839392943\\
301.01	0.00503693311102424\\
302.01	0.00503718273330268\\
303.01	0.0050374373609581\\
304.01	0.00503769709625872\\
305.01	0.00503796204359043\\
306.01	0.00503823230950071\\
307.01	0.00503850800274236\\
308.01	0.00503878923431933\\
309.01	0.00503907611753321\\
310.01	0.00503936876802981\\
311.01	0.00503966730384769\\
312.01	0.00503997184546696\\
313.01	0.00504028251585979\\
314.01	0.00504059944054119\\
315.01	0.00504092274762133\\
316.01	0.0050412525678584\\
317.01	0.00504158903471354\\
318.01	0.00504193228440589\\
319.01	0.00504228245596902\\
320.01	0.00504263969130883\\
321.01	0.00504300413526243\\
322.01	0.00504337593565841\\
323.01	0.00504375524337804\\
324.01	0.00504414221241793\\
325.01	0.00504453699995414\\
326.01	0.00504493976640799\\
327.01	0.00504535067551151\\
328.01	0.00504576989437699\\
329.01	0.00504619759356566\\
330.01	0.00504663394715922\\
331.01	0.00504707913283256\\
332.01	0.00504753333192737\\
333.01	0.0050479967295282\\
334.01	0.00504846951454033\\
335.01	0.00504895187976848\\
336.01	0.00504944402199728\\
337.01	0.00504994614207464\\
338.01	0.00505045844499537\\
339.01	0.00505098113998795\\
340.01	0.00505151444060255\\
341.01	0.00505205856480088\\
342.01	0.00505261373504817\\
343.01	0.00505318017840738\\
344.01	0.00505375812663522\\
345.01	0.00505434781628047\\
346.01	0.00505494948878352\\
347.01	0.00505556339057943\\
348.01	0.0050561897732026\\
349.01	0.00505682889339344\\
350.01	0.00505748101320719\\
351.01	0.00505814640012613\\
352.01	0.0050588253271726\\
353.01	0.00505951807302557\\
354.01	0.00506022492213949\\
355.01	0.0050609461648646\\
356.01	0.00506168209757112\\
357.01	0.00506243302277486\\
358.01	0.0050631992492661\\
359.01	0.00506398109224075\\
360.01	0.00506477887343372\\
361.01	0.00506559292125611\\
362.01	0.00506642357093377\\
363.01	0.0050672711646495\\
364.01	0.00506813605168775\\
365.01	0.00506901858858232\\
366.01	0.00506991913926619\\
367.01	0.00507083807522581\\
368.01	0.00507177577565693\\
369.01	0.00507273262762459\\
370.01	0.0050737090262252\\
371.01	0.00507470537475273\\
372.01	0.00507572208486796\\
373.01	0.0050767595767709\\
374.01	0.00507781827937674\\
375.01	0.00507889863049542\\
376.01	0.00508000107701456\\
377.01	0.00508112607508641\\
378.01	0.00508227409031825\\
379.01	0.00508344559796643\\
380.01	0.00508464108313535\\
381.01	0.00508586104097879\\
382.01	0.00508710597690695\\
383.01	0.00508837640679624\\
384.01	0.0050896728572046\\
385.01	0.00509099586558949\\
386.01	0.00509234598053222\\
387.01	0.00509372376196553\\
388.01	0.00509512978140565\\
389.01	0.00509656462218988\\
390.01	0.0050980288797183\\
391.01	0.00509952316170056\\
392.01	0.00510104808840786\\
393.01	0.00510260429292941\\
394.01	0.00510419242143478\\
395.01	0.00510581313344144\\
396.01	0.00510746710208676\\
397.01	0.00510915501440734\\
398.01	0.00511087757162237\\
399.01	0.00511263548942352\\
400.01	0.00511442949827068\\
401.01	0.00511626034369345\\
402.01	0.0051181287865992\\
403.01	0.00512003560358712\\
404.01	0.00512198158726905\\
405.01	0.00512396754659643\\
406.01	0.00512599430719415\\
407.01	0.00512806271170159\\
408.01	0.00513017362012046\\
409.01	0.00513232791016949\\
410.01	0.00513452647764723\\
411.01	0.00513677023680156\\
412.01	0.00513906012070751\\
413.01	0.0051413970816525\\
414.01	0.00514378209152984\\
415.01	0.00514621614224034\\
416.01	0.00514870024610219\\
417.01	0.00515123543627008\\
418.01	0.00515382276716167\\
419.01	0.00515646331489446\\
420.01	0.00515915817773094\\
421.01	0.00516190847653376\\
422.01	0.00516471535522936\\
423.01	0.0051675799812823\\
424.01	0.00517050354617916\\
425.01	0.00517348726592291\\
426.01	0.00517653238153705\\
427.01	0.00517964015958057\\
428.01	0.00518281189267433\\
429.01	0.00518604890003725\\
430.01	0.00518935252803483\\
431.01	0.00519272415073881\\
432.01	0.00519616517049813\\
433.01	0.00519967701852253\\
434.01	0.00520326115547738\\
435.01	0.00520691907209199\\
436.01	0.00521065228978022\\
437.01	0.00521446236127313\\
438.01	0.00521835087126643\\
439.01	0.00522231943707985\\
440.01	0.0052263697093313\\
441.01	0.00523050337262421\\
442.01	0.00523472214624948\\
443.01	0.00523902778490222\\
444.01	0.00524342207941315\\
445.01	0.00524790685749485\\
446.01	0.00525248398450374\\
447.01	0.00525715536421802\\
448.01	0.00526192293963146\\
449.01	0.00526678869376357\\
450.01	0.00527175465048671\\
451.01	0.00527682287537118\\
452.01	0.00528199547654581\\
453.01	0.0052872746055784\\
454.01	0.00529266245837408\\
455.01	0.00529816127609135\\
456.01	0.00530377334607839\\
457.01	0.00530950100282836\\
458.01	0.00531534662895433\\
459.01	0.00532131265618531\\
460.01	0.00532740156638227\\
461.01	0.00533361589257574\\
462.01	0.0053399582200252\\
463.01	0.00534643118730034\\
464.01	0.0053530374873846\\
465.01	0.00535977986880204\\
466.01	0.00536666113676787\\
467.01	0.00537368415436255\\
468.01	0.00538085184373044\\
469.01	0.00538816718730339\\
470.01	0.00539563322904991\\
471.01	0.00540325307574993\\
472.01	0.00541102989829671\\
473.01	0.00541896693302511\\
474.01	0.00542706748306775\\
475.01	0.00543533491973951\\
476.01	0.00544377268394998\\
477.01	0.00545238428764663\\
478.01	0.00546117331528623\\
479.01	0.0054701434253375\\
480.01	0.0054792983518152\\
481.01	0.00548864190584508\\
482.01	0.00549817797726253\\
483.01	0.00550791053624219\\
484.01	0.00551784363496273\\
485.01	0.0055279814093054\\
486.01	0.00553832808058766\\
487.01	0.00554888795733173\\
488.01	0.0055596654370699\\
489.01	0.00557066500818651\\
490.01	0.00558189125179769\\
491.01	0.00559334884366847\\
492.01	0.00560504255616977\\
493.01	0.00561697726027399\\
494.01	0.00562915792759081\\
495.01	0.00564158963244404\\
496.01	0.00565427755398963\\
497.01	0.00566722697837492\\
498.01	0.00568044330094133\\
499.01	0.00569393202846911\\
500.01	0.00570769878146611\\
501.01	0.00572174929650032\\
502.01	0.00573608942857665\\
503.01	0.00575072515355798\\
504.01	0.00576566257063221\\
505.01	0.00578090790482362\\
506.01	0.00579646750954958\\
507.01	0.00581234786922323\\
508.01	0.00582855560190046\\
509.01	0.00584509746197324\\
510.01	0.00586198034290671\\
511.01	0.00587921128002072\\
512.01	0.00589679745331509\\
513.01	0.00591474619033792\\
514.01	0.00593306496909482\\
515.01	0.00595176142099829\\
516.01	0.0059708433338565\\
517.01	0.00599031865489785\\
518.01	0.00601019549382988\\
519.01	0.00603048212592962\\
520.01	0.00605118699516193\\
521.01	0.00607231871732236\\
522.01	0.0060938860831997\\
523.01	0.00611589806175381\\
524.01	0.00613836380330246\\
525.01	0.00616129264271094\\
526.01	0.00618469410257619\\
527.01	0.00620857789639841\\
528.01	0.006232953931729\\
529.01	0.00625783231328467\\
530.01	0.00628322334601478\\
531.01	0.00630913753810876\\
532.01	0.0063355856039277\\
533.01	0.00636257846684055\\
534.01	0.00639012726194826\\
535.01	0.00641824333867082\\
536.01	0.00644693826317301\\
537.01	0.00647622382060102\\
538.01	0.00650611201709725\\
539.01	0.00653661508155973\\
540.01	0.00656774546710449\\
541.01	0.00659951585218799\\
542.01	0.00663193914134093\\
543.01	0.00666502846545589\\
544.01	0.00669879718156972\\
545.01	0.00673325887207067\\
546.01	0.00676842734325422\\
547.01	0.00680431662314229\\
548.01	0.00684094095847129\\
549.01	0.00687831481074262\\
550.01	0.00691645285122013\\
551.01	0.00695536995474208\\
552.01	0.00699508119220391\\
553.01	0.00703560182155209\\
554.01	0.00707694727710953\\
555.01	0.00711913315703655\\
556.01	0.00716217520870922\\
557.01	0.00720608931177373\\
558.01	0.00725089145861135\\
559.01	0.0072965977319211\\
560.01	0.00734322427909608\\
561.01	0.00739078728304045\\
562.01	0.00743930292903837\\
563.01	0.00748878736724826\\
564.01	0.00753925667036025\\
565.01	0.00759072678591111\\
566.01	0.0076432134827101\\
567.01	0.00769673229078503\\
568.01	0.00775129843421511\\
569.01	0.00780692675617325\\
570.01	0.00786363163546109\\
571.01	0.0079214268937827\\
572.01	0.00798032569297568\\
573.01	0.00804034042139869\\
574.01	0.00810148256867251\\
575.01	0.00816376258798929\\
576.01	0.00822718974525323\\
577.01	0.00829177195440151\\
578.01	0.00835751559839206\\
579.01	0.00842442533555191\\
580.01	0.00849250389127248\\
581.01	0.00856175183544577\\
582.01	0.00863216734659208\\
583.01	0.00870374596436989\\
584.01	0.00877648033314344\\
585.01	0.00885035994056966\\
586.01	0.00892537085684264\\
587.01	0.0090014954823987\\
588.01	0.00907871231467107\\
589.01	0.0091569957480491\\
590.01	0.00923631592574849\\
591.01	0.00931663866808596\\
592.01	0.00939792550899438\\
593.01	0.00948013388190589\\
594.01	0.00956321750786664\\
595.01	0.00964712705354462\\
596.01	0.00973181114542127\\
597.01	0.00981721784988034\\
598.01	0.00990284350259786\\
599.01	0.00996919203046377\\
599.02	0.00996973314335038\\
599.03	0.0099702709571694\\
599.04	0.00997080543920336\\
599.05	0.0099713365564138\\
599.06	0.00997186427543808\\
599.07	0.0099723885625862\\
599.08	0.00997290938383756\\
599.09	0.00997342670483771\\
599.1	0.00997394049089505\\
599.11	0.00997445070697751\\
599.12	0.00997495731770918\\
599.13	0.00997546028736696\\
599.14	0.00997595957987707\\
599.15	0.00997645515881166\\
599.16	0.00997694698738526\\
599.17	0.00997743502845131\\
599.18	0.00997791924449856\\
599.19	0.00997839959764748\\
599.2	0.00997887604964662\\
599.21	0.00997934856186899\\
599.22	0.00997981709530829\\
599.23	0.00998028161057521\\
599.24	0.00998074206789365\\
599.25	0.0099811984270969\\
599.26	0.00998165064762378\\
599.27	0.00998209868851477\\
599.28	0.00998254250840806\\
599.29	0.00998298206553559\\
599.3	0.00998341731771905\\
599.31	0.00998384822236584\\
599.32	0.00998427473646497\\
599.33	0.00998469681658292\\
599.34	0.00998511441885952\\
599.35	0.0099855274990037\\
599.36	0.00998593601228926\\
599.37	0.00998633991355056\\
599.38	0.00998673915717821\\
599.39	0.00998713369711469\\
599.4	0.00998752348684992\\
599.41	0.00998790847811157\\
599.42	0.00998828861974491\\
599.43	0.00998866386008804\\
599.44	0.00998903414696681\\
599.45	0.00998939942768986\\
599.46	0.00998975964904344\\
599.47	0.00999011475728636\\
599.48	0.00999046469814472\\
599.49	0.00999080941680669\\
599.5	0.00999114885791725\\
599.51	0.00999148296557278\\
599.52	0.0099918116833157\\
599.53	0.00999213495412901\\
599.54	0.00999245272043077\\
599.55	0.00999276492406855\\
599.56	0.00999307150631381\\
599.57	0.00999337240785624\\
599.58	0.00999366756879799\\
599.59	0.00999395692864795\\
599.6	0.00999424042631586\\
599.61	0.00999451800010642\\
599.62	0.00999478958771336\\
599.63	0.0099950551262134\\
599.64	0.00999531455206019\\
599.65	0.00999556780107816\\
599.66	0.00999581480845634\\
599.67	0.00999605550874211\\
599.68	0.00999628983583487\\
599.69	0.00999651772297967\\
599.7	0.00999673910276076\\
599.71	0.00999695390709509\\
599.72	0.00999716206722577\\
599.73	0.00999736351371538\\
599.74	0.00999755817643933\\
599.75	0.00999774598457904\\
599.76	0.00999792686661517\\
599.77	0.00999810075032068\\
599.78	0.00999826756275386\\
599.79	0.00999842723025133\\
599.8	0.00999857967842092\\
599.81	0.0099987248321345\\
599.82	0.00999886261552071\\
599.83	0.00999899295195769\\
599.84	0.00999911576406567\\
599.85	0.00999923097369951\\
599.86	0.00999933850194117\\
599.87	0.00999943826909211\\
599.88	0.00999953019466558\\
599.89	0.00999961419737891\\
599.9	0.00999969019514566\\
599.91	0.00999975810506767\\
599.92	0.00999981784342713\\
599.93	0.00999986932567848\\
599.94	0.00999991246644028\\
599.95	0.00999994717948697\\
599.96	0.00999997337774056\\
599.97	0.00999999097326229\\
599.98	0.00999999987724406\\
599.99	0.01\\
600	0.01\\
};
\addplot [color=mycolor3,solid,forget plot]
  table[row sep=crcr]{%
0.01	0.00502269192807735\\
1.01	0.00502269275816619\\
2.01	0.00502269360447714\\
3.01	0.00502269446732508\\
4.01	0.00502269534703129\\
5.01	0.00502269624392311\\
6.01	0.00502269715833426\\
7.01	0.00502269809060478\\
8.01	0.00502269904108132\\
9.01	0.00502270001011726\\
10.01	0.00502270099807273\\
11.01	0.00502270200531477\\
12.01	0.00502270303221747\\
13.01	0.00502270407916229\\
14.01	0.00502270514653784\\
15.01	0.00502270623474015\\
16.01	0.00502270734417315\\
17.01	0.00502270847524855\\
18.01	0.00502270962838561\\
19.01	0.00502271080401182\\
20.01	0.00502271200256298\\
21.01	0.00502271322448329\\
22.01	0.00502271447022537\\
23.01	0.00502271574025047\\
24.01	0.00502271703502895\\
25.01	0.00502271835504012\\
26.01	0.00502271970077223\\
27.01	0.00502272107272335\\
28.01	0.00502272247140077\\
29.01	0.00502272389732167\\
30.01	0.0050227253510132\\
31.01	0.00502272683301236\\
32.01	0.00502272834386676\\
33.01	0.00502272988413437\\
34.01	0.00502273145438391\\
35.01	0.00502273305519495\\
36.01	0.00502273468715808\\
37.01	0.00502273635087555\\
38.01	0.00502273804696077\\
39.01	0.00502273977603926\\
40.01	0.00502274153874836\\
41.01	0.00502274333573718\\
42.01	0.00502274516766807\\
43.01	0.00502274703521559\\
44.01	0.00502274893906715\\
45.01	0.00502275087992363\\
46.01	0.00502275285849908\\
47.01	0.00502275487552123\\
48.01	0.00502275693173183\\
49.01	0.00502275902788675\\
50.01	0.00502276116475632\\
51.01	0.00502276334312574\\
52.01	0.00502276556379504\\
53.01	0.00502276782757959\\
54.01	0.00502277013531031\\
55.01	0.00502277248783412\\
56.01	0.00502277488601388\\
57.01	0.00502277733072924\\
58.01	0.00502277982287644\\
59.01	0.00502278236336892\\
60.01	0.00502278495313754\\
61.01	0.00502278759313086\\
62.01	0.00502279028431567\\
63.01	0.00502279302767693\\
64.01	0.00502279582421873\\
65.01	0.00502279867496436\\
66.01	0.0050228015809561\\
67.01	0.00502280454325642\\
68.01	0.00502280756294829\\
69.01	0.00502281064113506\\
70.01	0.00502281377894102\\
71.01	0.0050228169775119\\
72.01	0.00502282023801544\\
73.01	0.00502282356164153\\
74.01	0.00502282694960268\\
75.01	0.00502283040313457\\
76.01	0.00502283392349639\\
77.01	0.00502283751197109\\
78.01	0.00502284116986636\\
79.01	0.00502284489851453\\
80.01	0.00502284869927325\\
81.01	0.00502285257352628\\
82.01	0.00502285652268339\\
83.01	0.00502286054818123\\
84.01	0.00502286465148379\\
85.01	0.00502286883408294\\
86.01	0.00502287309749881\\
87.01	0.00502287744328052\\
88.01	0.00502288187300651\\
89.01	0.00502288638828523\\
90.01	0.00502289099075573\\
91.01	0.00502289568208819\\
92.01	0.00502290046398435\\
93.01	0.00502290533817826\\
94.01	0.00502291030643741\\
95.01	0.00502291537056186\\
96.01	0.00502292053238662\\
97.01	0.00502292579378125\\
98.01	0.00502293115665067\\
99.01	0.00502293662293596\\
100.01	0.00502294219461531\\
101.01	0.00502294787370401\\
102.01	0.0050229536622559\\
103.01	0.00502295956236332\\
104.01	0.00502296557615868\\
105.01	0.00502297170581455\\
106.01	0.00502297795354477\\
107.01	0.00502298432160513\\
108.01	0.00502299081229411\\
109.01	0.00502299742795358\\
110.01	0.00502300417096997\\
111.01	0.00502301104377507\\
112.01	0.00502301804884617\\
113.01	0.00502302518870795\\
114.01	0.00502303246593248\\
115.01	0.00502303988314082\\
116.01	0.00502304744300355\\
117.01	0.00502305514824167\\
118.01	0.00502306300162776\\
119.01	0.0050230710059868\\
120.01	0.00502307916419732\\
121.01	0.00502308747919222\\
122.01	0.00502309595395981\\
123.01	0.00502310459154491\\
124.01	0.00502311339505002\\
125.01	0.00502312236763623\\
126.01	0.00502313151252447\\
127.01	0.00502314083299653\\
128.01	0.00502315033239616\\
129.01	0.00502316001413047\\
130.01	0.00502316988167072\\
131.01	0.00502317993855393\\
132.01	0.00502319018838419\\
133.01	0.00502320063483327\\
134.01	0.00502321128164261\\
135.01	0.00502322213262422\\
136.01	0.00502323319166223\\
137.01	0.00502324446271411\\
138.01	0.00502325594981212\\
139.01	0.00502326765706478\\
140.01	0.0050232795886581\\
141.01	0.0050232917488572\\
142.01	0.00502330414200786\\
143.01	0.00502331677253788\\
144.01	0.00502332964495891\\
145.01	0.00502334276386766\\
146.01	0.00502335613394771\\
147.01	0.00502336975997114\\
148.01	0.0050233836468003\\
149.01	0.0050233977993893\\
150.01	0.00502341222278592\\
151.01	0.0050234269221333\\
152.01	0.00502344190267173\\
153.01	0.00502345716974069\\
154.01	0.00502347272878047\\
155.01	0.00502348858533407\\
156.01	0.00502350474504953\\
157.01	0.00502352121368163\\
158.01	0.00502353799709374\\
159.01	0.00502355510126019\\
160.01	0.00502357253226837\\
161.01	0.0050235902963208\\
162.01	0.00502360839973709\\
163.01	0.00502362684895667\\
164.01	0.00502364565054074\\
165.01	0.00502366481117458\\
166.01	0.00502368433766998\\
167.01	0.00502370423696777\\
168.01	0.00502372451614012\\
169.01	0.00502374518239317\\
170.01	0.0050237662430692\\
171.01	0.00502378770564997\\
172.01	0.00502380957775894\\
173.01	0.00502383186716404\\
174.01	0.00502385458178017\\
175.01	0.0050238777296725\\
176.01	0.00502390131905903\\
177.01	0.00502392535831374\\
178.01	0.00502394985596955\\
179.01	0.00502397482072109\\
180.01	0.0050240002614282\\
181.01	0.00502402618711899\\
182.01	0.00502405260699289\\
183.01	0.00502407953042421\\
184.01	0.00502410696696538\\
185.01	0.0050241349263505\\
186.01	0.00502416341849879\\
187.01	0.00502419245351826\\
188.01	0.00502422204170916\\
189.01	0.00502425219356792\\
190.01	0.00502428291979097\\
191.01	0.00502431423127846\\
192.01	0.00502434613913834\\
193.01	0.00502437865469038\\
194.01	0.00502441178947051\\
195.01	0.00502444555523426\\
196.01	0.00502447996396256\\
197.01	0.0050245150278644\\
198.01	0.00502455075938266\\
199.01	0.00502458717119787\\
200.01	0.00502462427623343\\
201.01	0.00502466208765993\\
202.01	0.00502470061890026\\
203.01	0.00502473988363467\\
204.01	0.00502477989580546\\
205.01	0.00502482066962262\\
206.01	0.00502486221956912\\
207.01	0.00502490456040573\\
208.01	0.00502494770717721\\
209.01	0.00502499167521754\\
210.01	0.00502503648015548\\
211.01	0.00502508213792142\\
212.01	0.00502512866475206\\
213.01	0.00502517607719771\\
214.01	0.0050252243921276\\
215.01	0.00502527362673716\\
216.01	0.00502532379855358\\
217.01	0.00502537492544335\\
218.01	0.00502542702561851\\
219.01	0.00502548011764381\\
220.01	0.00502553422044383\\
221.01	0.00502558935331021\\
222.01	0.00502564553590917\\
223.01	0.00502570278828886\\
224.01	0.00502576113088737\\
225.01	0.00502582058454044\\
226.01	0.00502588117048948\\
227.01	0.00502594291039012\\
228.01	0.0050260058263203\\
229.01	0.00502606994078937\\
230.01	0.00502613527674647\\
231.01	0.00502620185758955\\
232.01	0.00502626970717483\\
233.01	0.00502633884982615\\
234.01	0.00502640931034432\\
235.01	0.0050264811140172\\
236.01	0.00502655428662993\\
237.01	0.00502662885447454\\
238.01	0.00502670484436112\\
239.01	0.00502678228362818\\
240.01	0.00502686120015372\\
241.01	0.00502694162236637\\
242.01	0.005027023579257\\
243.01	0.0050271071003903\\
244.01	0.00502719221591665\\
245.01	0.00502727895658453\\
246.01	0.00502736735375309\\
247.01	0.00502745743940455\\
248.01	0.00502754924615793\\
249.01	0.00502764280728168\\
250.01	0.00502773815670789\\
251.01	0.00502783532904608\\
252.01	0.00502793435959722\\
253.01	0.00502803528436875\\
254.01	0.00502813814008939\\
255.01	0.00502824296422411\\
256.01	0.0050283497949901\\
257.01	0.00502845867137251\\
258.01	0.00502856963314093\\
259.01	0.00502868272086569\\
260.01	0.00502879797593533\\
261.01	0.00502891544057366\\
262.01	0.00502903515785794\\
263.01	0.0050291571717366\\
264.01	0.00502928152704825\\
265.01	0.00502940826954061\\
266.01	0.00502953744588969\\
267.01	0.00502966910371992\\
268.01	0.00502980329162449\\
269.01	0.00502994005918601\\
270.01	0.00503007945699756\\
271.01	0.00503022153668439\\
272.01	0.00503036635092638\\
273.01	0.00503051395348038\\
274.01	0.00503066439920342\\
275.01	0.00503081774407646\\
276.01	0.00503097404522812\\
277.01	0.00503113336095971\\
278.01	0.00503129575077038\\
279.01	0.0050314612753828\\
280.01	0.00503162999676922\\
281.01	0.00503180197817885\\
282.01	0.00503197728416456\\
283.01	0.00503215598061148\\
284.01	0.00503233813476518\\
285.01	0.00503252381526131\\
286.01	0.00503271309215458\\
287.01	0.00503290603695017\\
288.01	0.00503310272263373\\
289.01	0.00503330322370403\\
290.01	0.00503350761620454\\
291.01	0.00503371597775681\\
292.01	0.00503392838759395\\
293.01	0.00503414492659505\\
294.01	0.00503436567732025\\
295.01	0.00503459072404615\\
296.01	0.00503482015280256\\
297.01	0.00503505405140962\\
298.01	0.00503529250951544\\
299.01	0.00503553561863494\\
300.01	0.005035783472189\\
301.01	0.005036036165545\\
302.01	0.00503629379605716\\
303.01	0.00503655646310849\\
304.01	0.00503682426815327\\
305.01	0.00503709731476035\\
306.01	0.00503737570865702\\
307.01	0.00503765955777422\\
308.01	0.005037948972292\\
309.01	0.00503824406468613\\
310.01	0.00503854494977575\\
311.01	0.00503885174477148\\
312.01	0.00503916456932479\\
313.01	0.00503948354557794\\
314.01	0.00503980879821511\\
315.01	0.00504014045451451\\
316.01	0.0050404786444011\\
317.01	0.00504082350050034\\
318.01	0.00504117515819326\\
319.01	0.00504153375567223\\
320.01	0.00504189943399799\\
321.01	0.00504227233715752\\
322.01	0.00504265261212303\\
323.01	0.00504304040891196\\
324.01	0.00504343588064847\\
325.01	0.00504383918362587\\
326.01	0.00504425047736964\\
327.01	0.00504466992470332\\
328.01	0.00504509769181376\\
329.01	0.00504553394831879\\
330.01	0.00504597886733602\\
331.01	0.00504643262555302\\
332.01	0.00504689540329863\\
333.01	0.00504736738461671\\
334.01	0.00504784875733993\\
335.01	0.00504833971316666\\
336.01	0.0050488404477388\\
337.01	0.00504935116072139\\
338.01	0.00504987205588395\\
339.01	0.00505040334118401\\
340.01	0.00505094522885204\\
341.01	0.00505149793547943\\
342.01	0.00505206168210733\\
343.01	0.00505263669431827\\
344.01	0.00505322320233027\\
345.01	0.00505382144109255\\
346.01	0.00505443165038472\\
347.01	0.00505505407491719\\
348.01	0.0050556889644352\\
349.01	0.00505633657382473\\
350.01	0.0050569971632219\\
351.01	0.00505767099812442\\
352.01	0.00505835834950605\\
353.01	0.00505905949393413\\
354.01	0.00505977471368994\\
355.01	0.00506050429689232\\
356.01	0.00506124853762378\\
357.01	0.0050620077360596\\
358.01	0.00506278219860108\\
359.01	0.00506357223801002\\
360.01	0.00506437817354806\\
361.01	0.00506520033111743\\
362.01	0.0050660390434064\\
363.01	0.00506689465003635\\
364.01	0.00506776749771209\\
365.01	0.00506865794037575\\
366.01	0.00506956633936299\\
367.01	0.00507049306356233\\
368.01	0.00507143848957758\\
369.01	0.00507240300189217\\
370.01	0.00507338699303843\\
371.01	0.00507439086376758\\
372.01	0.00507541502322463\\
373.01	0.00507645988912479\\
374.01	0.00507752588793453\\
375.01	0.00507861345505498\\
376.01	0.0050797230350091\\
377.01	0.00508085508163242\\
378.01	0.00508201005826769\\
379.01	0.00508318843796281\\
380.01	0.00508439070367284\\
381.01	0.00508561734846642\\
382.01	0.00508686887573492\\
383.01	0.00508814579940786\\
384.01	0.00508944864417014\\
385.01	0.00509077794568545\\
386.01	0.00509213425082232\\
387.01	0.00509351811788605\\
388.01	0.00509493011685437\\
389.01	0.0050963708296172\\
390.01	0.00509784085022219\\
391.01	0.00509934078512408\\
392.01	0.00510087125343914\\
393.01	0.00510243288720465\\
394.01	0.00510402633164277\\
395.01	0.00510565224543046\\
396.01	0.0051073113009738\\
397.01	0.00510900418468772\\
398.01	0.00511073159728134\\
399.01	0.00511249425404832\\
400.01	0.00511429288516371\\
401.01	0.0051161282359854\\
402.01	0.00511800106736243\\
403.01	0.00511991215594831\\
404.01	0.00512186229452122\\
405.01	0.00512385229231005\\
406.01	0.00512588297532704\\
407.01	0.00512795518670606\\
408.01	0.00513006978704923\\
409.01	0.00513222765477883\\
410.01	0.00513442968649672\\
411.01	0.00513667679735155\\
412.01	0.00513896992141193\\
413.01	0.00514131001204859\\
414.01	0.00514369804232345\\
415.01	0.00514613500538642\\
416.01	0.005148621914881\\
417.01	0.00515115980535756\\
418.01	0.00515374973269582\\
419.01	0.00515639277453533\\
420.01	0.00515909003071588\\
421.01	0.00516184262372579\\
422.01	0.00516465169916107\\
423.01	0.00516751842619395\\
424.01	0.00517044399805046\\
425.01	0.00517342963249932\\
426.01	0.00517647657235074\\
427.01	0.00517958608596648\\
428.01	0.0051827594677801\\
429.01	0.00518599803882874\\
430.01	0.00518930314729744\\
431.01	0.00519267616907353\\
432.01	0.00519611850831445\\
433.01	0.00519963159802705\\
434.01	0.00520321690065996\\
435.01	0.00520687590870869\\
436.01	0.00521061014533324\\
437.01	0.00521442116498987\\
438.01	0.00521831055407579\\
439.01	0.00522227993158805\\
440.01	0.00522633094979617\\
441.01	0.00523046529492927\\
442.01	0.00523468468787772\\
443.01	0.00523899088490927\\
444.01	0.00524338567839996\\
445.01	0.00524787089758146\\
446.01	0.00525244840930239\\
447.01	0.00525712011880645\\
448.01	0.0052618879705259\\
449.01	0.00526675394889162\\
450.01	0.00527172007916012\\
451.01	0.00527678842825666\\
452.01	0.0052819611056364\\
453.01	0.00528724026416325\\
454.01	0.00529262810100554\\
455.01	0.00529812685855215\\
456.01	0.00530373882534568\\
457.01	0.00530946633703643\\
458.01	0.00531531177735521\\
459.01	0.00532127757910709\\
460.01	0.00532736622518554\\
461.01	0.00533358024960859\\
462.01	0.00533992223857581\\
463.01	0.00534639483154867\\
464.01	0.00535300072235285\\
465.01	0.00535974266030467\\
466.01	0.00536662345136033\\
467.01	0.00537364595929006\\
468.01	0.00538081310687699\\
469.01	0.00538812787714035\\
470.01	0.00539559331458557\\
471.01	0.00540321252647991\\
472.01	0.00541098868415421\\
473.01	0.00541892502433317\\
474.01	0.00542702485049187\\
475.01	0.00543529153424156\\
476.01	0.00544372851674385\\
477.01	0.00545233931015376\\
478.01	0.00546112749909385\\
479.01	0.00547009674215762\\
480.01	0.00547925077344463\\
481.01	0.0054885934041268\\
482.01	0.00549812852404753\\
483.01	0.00550786010335333\\
484.01	0.00551779219415929\\
485.01	0.00552792893224861\\
486.01	0.0055382745388064\\
487.01	0.00554883332219024\\
488.01	0.00555960967973582\\
489.01	0.00557060809959947\\
490.01	0.00558183316263904\\
491.01	0.00559328954433222\\
492.01	0.00560498201673348\\
493.01	0.00561691545047095\\
494.01	0.00562909481678298\\
495.01	0.00564152518959594\\
496.01	0.00565421174764243\\
497.01	0.0056671597766221\\
498.01	0.00568037467140432\\
499.01	0.00569386193827418\\
500.01	0.00570762719722163\\
501.01	0.00572167618427445\\
502.01	0.00573601475387593\\
503.01	0.00575064888130682\\
504.01	0.00576558466515187\\
505.01	0.00578082832981236\\
506.01	0.0057963862280633\\
507.01	0.00581226484365609\\
508.01	0.00582847079396645\\
509.01	0.00584501083268776\\
510.01	0.00586189185256834\\
511.01	0.00587912088819392\\
512.01	0.0058967051188127\\
513.01	0.00591465187120406\\
514.01	0.00593296862258812\\
515.01	0.00595166300357612\\
516.01	0.0059707428011586\\
517.01	0.00599021596173088\\
518.01	0.00601009059415211\\
519.01	0.00603037497283576\\
520.01	0.00605107754086892\\
521.01	0.0060722069131553\\
522.01	0.00609377187957812\\
523.01	0.00611578140817863\\
524.01	0.00613824464834352\\
525.01	0.00616117093399438\\
526.01	0.00618456978677384\\
527.01	0.00620845091921686\\
528.01	0.00623282423789978\\
529.01	0.00625769984655496\\
530.01	0.00628308804914007\\
531.01	0.00630899935284569\\
532.01	0.00633544447102788\\
533.01	0.00636243432604691\\
534.01	0.00638998005199212\\
535.01	0.00641809299727018\\
536.01	0.00644678472703378\\
537.01	0.00647606702541922\\
538.01	0.00650595189756438\\
539.01	0.00653645157136962\\
540.01	0.00656757849896372\\
541.01	0.00659934535782969\\
542.01	0.00663176505154073\\
543.01	0.00666485071005252\\
544.01	0.00669861568948972\\
545.01	0.00673307357135754\\
546.01	0.00676823816110224\\
547.01	0.00680412348593676\\
548.01	0.00684074379183394\\
549.01	0.00687811353958502\\
550.01	0.00691624739980339\\
551.01	0.0069551602467454\\
552.01	0.0069948671508024\\
553.01	0.00703538336950402\\
554.01	0.00707672433685472\\
555.01	0.00711890565080693\\
556.01	0.00716194305865242\\
557.01	0.00720585244009135\\
558.01	0.00725064978771305\\
559.01	0.00729635118459537\\
560.01	0.00734297277870056\\
561.01	0.00739053075371283\\
562.01	0.00743904129592875\\
563.01	0.00748852055677669\\
564.01	0.00753898461049995\\
565.01	0.00759044940649993\\
566.01	0.00764293071579204\\
567.01	0.00769644407098446\\
568.01	0.00775100469914526\\
569.01	0.00780662744688079\\
570.01	0.0078633266969083\\
571.01	0.00792111627536951\\
572.01	0.00798000934910111\\
573.01	0.00804001831206291\\
574.01	0.00810115466011737\\
575.01	0.0081634288533764\\
576.01	0.00822685016537466\\
577.01	0.00829142651841632\\
578.01	0.00835716430458038\\
579.01	0.00842406819207157\\
580.01	0.00849214091689975\\
581.01	0.0085613830602753\\
582.01	0.00863179281266063\\
583.01	0.00870336572616018\\
584.01	0.00877609445790773\\
585.01	0.00884996850839709\\
586.01	0.00892497396037102\\
587.01	0.00900109322604318\\
588.01	0.00907830481320621\\
589.01	0.00915658312433641\\
590.01	0.00923589830734327\\
591.01	0.00931621618238686\\
592.01	0.00939749827650748\\
593.01	0.00947970200708216\\
594.01	0.00956278106682799\\
595.01	0.0096466860778335\\
596.01	0.00973136560068165\\
597.01	0.00981676760809251\\
598.01	0.00990268918671883\\
599.01	0.00996919203046377\\
599.02	0.00996973314335038\\
599.03	0.0099702709571694\\
599.04	0.00997080543920336\\
599.05	0.0099713365564138\\
599.06	0.00997186427543808\\
599.07	0.0099723885625862\\
599.08	0.00997290938383756\\
599.09	0.00997342670483771\\
599.1	0.00997394049089505\\
599.11	0.00997445070697751\\
599.12	0.00997495731770918\\
599.13	0.00997546028736696\\
599.14	0.00997595957987707\\
599.15	0.00997645515881165\\
599.16	0.00997694698738526\\
599.17	0.00997743502845131\\
599.18	0.00997791924449856\\
599.19	0.00997839959764748\\
599.2	0.00997887604964662\\
599.21	0.00997934856186899\\
599.22	0.00997981709530829\\
599.23	0.00998028161057521\\
599.24	0.00998074206789365\\
599.25	0.0099811984270969\\
599.26	0.00998165064762378\\
599.27	0.00998209868851477\\
599.28	0.00998254250840806\\
599.29	0.00998298206553559\\
599.3	0.00998341731771905\\
599.31	0.00998384822236584\\
599.32	0.00998427473646497\\
599.33	0.00998469681658292\\
599.34	0.00998511441885952\\
599.35	0.0099855274990037\\
599.36	0.00998593601228926\\
599.37	0.00998633991355056\\
599.38	0.00998673915717821\\
599.39	0.00998713369711469\\
599.4	0.00998752348684992\\
599.41	0.00998790847811157\\
599.42	0.00998828861974491\\
599.43	0.00998866386008804\\
599.44	0.00998903414696681\\
599.45	0.00998939942768985\\
599.46	0.00998975964904344\\
599.47	0.00999011475728636\\
599.48	0.00999046469814472\\
599.49	0.00999080941680669\\
599.5	0.00999114885791725\\
599.51	0.00999148296557278\\
599.52	0.0099918116833157\\
599.53	0.00999213495412901\\
599.54	0.00999245272043077\\
599.55	0.00999276492406855\\
599.56	0.00999307150631382\\
599.57	0.00999337240785624\\
599.58	0.00999366756879799\\
599.59	0.00999395692864795\\
599.6	0.00999424042631585\\
599.61	0.00999451800010642\\
599.62	0.00999478958771336\\
599.63	0.0099950551262134\\
599.64	0.00999531455206019\\
599.65	0.00999556780107816\\
599.66	0.00999581480845634\\
599.67	0.00999605550874211\\
599.68	0.00999628983583487\\
599.69	0.00999651772297967\\
599.7	0.00999673910276076\\
599.71	0.00999695390709509\\
599.72	0.00999716206722577\\
599.73	0.00999736351371538\\
599.74	0.00999755817643933\\
599.75	0.00999774598457904\\
599.76	0.00999792686661517\\
599.77	0.00999810075032068\\
599.78	0.00999826756275386\\
599.79	0.00999842723025133\\
599.8	0.00999857967842092\\
599.81	0.0099987248321345\\
599.82	0.00999886261552071\\
599.83	0.00999899295195769\\
599.84	0.00999911576406567\\
599.85	0.00999923097369951\\
599.86	0.00999933850194117\\
599.87	0.00999943826909211\\
599.88	0.00999953019466558\\
599.89	0.00999961419737891\\
599.9	0.00999969019514566\\
599.91	0.00999975810506767\\
599.92	0.00999981784342713\\
599.93	0.00999986932567848\\
599.94	0.00999991246644028\\
599.95	0.00999994717948697\\
599.96	0.00999997337774056\\
599.97	0.00999999097326228\\
599.98	0.00999999987724406\\
599.99	0.01\\
600	0.01\\
};
\addplot [color=mycolor4,solid,forget plot]
  table[row sep=crcr]{%
0.01	0.0050190636862812\\
1.01	0.00501906463782538\\
2.01	0.00501906560805189\\
3.01	0.00501906659732479\\
4.01	0.00501906760601505\\
5.01	0.00501906863450116\\
6.01	0.00501906968316851\\
7.01	0.00501907075241023\\
8.01	0.00501907184262679\\
9.01	0.00501907295422666\\
10.01	0.00501907408762585\\
11.01	0.00501907524324856\\
12.01	0.00501907642152723\\
13.01	0.00501907762290241\\
14.01	0.00501907884782346\\
15.01	0.00501908009674813\\
16.01	0.00501908137014302\\
17.01	0.00501908266848361\\
18.01	0.00501908399225469\\
19.01	0.00501908534195042\\
20.01	0.0050190867180743\\
21.01	0.00501908812113945\\
22.01	0.00501908955166902\\
23.01	0.00501909101019615\\
24.01	0.0050190924972642\\
25.01	0.0050190940134268\\
26.01	0.00501909555924866\\
27.01	0.00501909713530482\\
28.01	0.00501909874218165\\
29.01	0.00501910038047652\\
30.01	0.00501910205079835\\
31.01	0.00501910375376788\\
32.01	0.00501910549001755\\
33.01	0.005019107260192\\
34.01	0.00501910906494818\\
35.01	0.00501911090495559\\
36.01	0.00501911278089671\\
37.01	0.00501911469346675\\
38.01	0.00501911664337441\\
39.01	0.0050191186313419\\
40.01	0.0050191206581052\\
41.01	0.00501912272441454\\
42.01	0.00501912483103419\\
43.01	0.0050191269787432\\
44.01	0.0050191291683354\\
45.01	0.00501913140061969\\
46.01	0.00501913367642064\\
47.01	0.00501913599657836\\
48.01	0.00501913836194895\\
49.01	0.005019140773405\\
50.01	0.00501914323183551\\
51.01	0.00501914573814656\\
52.01	0.00501914829326133\\
53.01	0.00501915089812062\\
54.01	0.00501915355368324\\
55.01	0.00501915626092608\\
56.01	0.00501915902084474\\
57.01	0.0050191618344535\\
58.01	0.00501916470278627\\
59.01	0.00501916762689629\\
60.01	0.00501917060785684\\
61.01	0.00501917364676173\\
62.01	0.00501917674472523\\
63.01	0.00501917990288314\\
64.01	0.00501918312239248\\
65.01	0.00501918640443227\\
66.01	0.00501918975020393\\
67.01	0.00501919316093166\\
68.01	0.00501919663786256\\
69.01	0.00501920018226763\\
70.01	0.0050192037954418\\
71.01	0.00501920747870464\\
72.01	0.00501921123340041\\
73.01	0.00501921506089889\\
74.01	0.00501921896259575\\
75.01	0.00501922293991293\\
76.01	0.00501922699429926\\
77.01	0.00501923112723107\\
78.01	0.00501923534021226\\
79.01	0.00501923963477527\\
80.01	0.00501924401248134\\
81.01	0.0050192484749212\\
82.01	0.00501925302371536\\
83.01	0.00501925766051513\\
84.01	0.00501926238700277\\
85.01	0.00501926720489194\\
86.01	0.00501927211592888\\
87.01	0.00501927712189236\\
88.01	0.0050192822245949\\
89.01	0.00501928742588281\\
90.01	0.00501929272763717\\
91.01	0.00501929813177436\\
92.01	0.00501930364024664\\
93.01	0.00501930925504299\\
94.01	0.00501931497818932\\
95.01	0.00501932081175031\\
96.01	0.00501932675782853\\
97.01	0.00501933281856613\\
98.01	0.00501933899614557\\
99.01	0.00501934529278994\\
100.01	0.0050193517107639\\
101.01	0.00501935825237451\\
102.01	0.00501936491997196\\
103.01	0.00501937171595036\\
104.01	0.00501937864274822\\
105.01	0.00501938570285006\\
106.01	0.0050193928987864\\
107.01	0.00501940023313502\\
108.01	0.00501940770852174\\
109.01	0.0050194153276215\\
110.01	0.00501942309315881\\
111.01	0.00501943100790882\\
112.01	0.00501943907469866\\
113.01	0.00501944729640778\\
114.01	0.00501945567596943\\
115.01	0.00501946421637105\\
116.01	0.0050194729206558\\
117.01	0.00501948179192331\\
118.01	0.00501949083333078\\
119.01	0.0050195000480939\\
120.01	0.00501950943948809\\
121.01	0.00501951901084931\\
122.01	0.00501952876557563\\
123.01	0.00501953870712786\\
124.01	0.00501954883903089\\
125.01	0.00501955916487502\\
126.01	0.00501956968831682\\
127.01	0.00501958041308039\\
128.01	0.00501959134295896\\
129.01	0.00501960248181554\\
130.01	0.00501961383358478\\
131.01	0.0050196254022737\\
132.01	0.00501963719196331\\
133.01	0.00501964920680996\\
134.01	0.00501966145104675\\
135.01	0.00501967392898464\\
136.01	0.00501968664501412\\
137.01	0.00501969960360645\\
138.01	0.00501971280931506\\
139.01	0.00501972626677753\\
140.01	0.00501973998071634\\
141.01	0.00501975395594097\\
142.01	0.00501976819734937\\
143.01	0.00501978270992932\\
144.01	0.00501979749876019\\
145.01	0.00501981256901469\\
146.01	0.00501982792596037\\
147.01	0.00501984357496139\\
148.01	0.00501985952148028\\
149.01	0.00501987577107953\\
150.01	0.0050198923294236\\
151.01	0.0050199092022809\\
152.01	0.00501992639552516\\
153.01	0.00501994391513751\\
154.01	0.00501996176720869\\
155.01	0.00501997995794085\\
156.01	0.00501999849364934\\
157.01	0.00502001738076462\\
158.01	0.00502003662583522\\
159.01	0.0050200562355288\\
160.01	0.00502007621663457\\
161.01	0.00502009657606595\\
162.01	0.00502011732086234\\
163.01	0.00502013845819144\\
164.01	0.00502015999535146\\
165.01	0.00502018193977367\\
166.01	0.00502020429902479\\
167.01	0.00502022708080903\\
168.01	0.00502025029297101\\
169.01	0.00502027394349808\\
170.01	0.00502029804052302\\
171.01	0.00502032259232618\\
172.01	0.00502034760733859\\
173.01	0.00502037309414435\\
174.01	0.00502039906148381\\
175.01	0.00502042551825597\\
176.01	0.0050204524735213\\
177.01	0.00502047993650487\\
178.01	0.00502050791659914\\
179.01	0.00502053642336707\\
180.01	0.0050205654665449\\
181.01	0.00502059505604584\\
182.01	0.00502062520196246\\
183.01	0.00502065591457078\\
184.01	0.00502068720433252\\
185.01	0.00502071908189955\\
186.01	0.00502075155811632\\
187.01	0.0050207846440241\\
188.01	0.00502081835086399\\
189.01	0.00502085269008092\\
190.01	0.00502088767332684\\
191.01	0.00502092331246481\\
192.01	0.00502095961957291\\
193.01	0.00502099660694766\\
194.01	0.00502103428710834\\
195.01	0.00502107267280101\\
196.01	0.00502111177700215\\
197.01	0.00502115161292357\\
198.01	0.00502119219401589\\
199.01	0.0050212335339737\\
200.01	0.00502127564673925\\
201.01	0.00502131854650727\\
202.01	0.00502136224772966\\
203.01	0.00502140676511986\\
204.01	0.005021452113658\\
205.01	0.00502149830859555\\
206.01	0.00502154536546001\\
207.01	0.00502159330006058\\
208.01	0.00502164212849283\\
209.01	0.00502169186714428\\
210.01	0.0050217425326993\\
211.01	0.0050217941421451\\
212.01	0.00502184671277701\\
213.01	0.00502190026220412\\
214.01	0.00502195480835549\\
215.01	0.00502201036948524\\
216.01	0.0050220669641797\\
217.01	0.0050221246113627\\
218.01	0.00502218333030202\\
219.01	0.00502224314061629\\
220.01	0.00502230406228095\\
221.01	0.00502236611563531\\
222.01	0.00502242932138887\\
223.01	0.00502249370062913\\
224.01	0.00502255927482793\\
225.01	0.00502262606584909\\
226.01	0.00502269409595579\\
227.01	0.00502276338781804\\
228.01	0.00502283396452048\\
229.01	0.00502290584957027\\
230.01	0.00502297906690523\\
231.01	0.005023053640902\\
232.01	0.00502312959638448\\
233.01	0.00502320695863241\\
234.01	0.00502328575339044\\
235.01	0.00502336600687672\\
236.01	0.00502344774579229\\
237.01	0.00502353099733051\\
238.01	0.00502361578918678\\
239.01	0.00502370214956807\\
240.01	0.00502379010720296\\
241.01	0.00502387969135258\\
242.01	0.00502397093182022\\
243.01	0.00502406385896251\\
244.01	0.00502415850370061\\
245.01	0.00502425489753115\\
246.01	0.00502435307253787\\
247.01	0.00502445306140335\\
248.01	0.00502455489742119\\
249.01	0.00502465861450822\\
250.01	0.0050247642472171\\
251.01	0.00502487183074932\\
252.01	0.00502498140096876\\
253.01	0.00502509299441485\\
254.01	0.00502520664831621\\
255.01	0.00502532240060563\\
256.01	0.0050254402899341\\
257.01	0.00502556035568567\\
258.01	0.005025682637993\\
259.01	0.00502580717775299\\
260.01	0.00502593401664239\\
261.01	0.00502606319713503\\
262.01	0.00502619476251753\\
263.01	0.00502632875690788\\
264.01	0.00502646522527174\\
265.01	0.00502660421344144\\
266.01	0.00502674576813428\\
267.01	0.00502688993697173\\
268.01	0.00502703676849824\\
269.01	0.00502718631220181\\
270.01	0.00502733861853444\\
271.01	0.00502749373893292\\
272.01	0.00502765172584038\\
273.01	0.00502781263272795\\
274.01	0.0050279765141182\\
275.01	0.00502814342560737\\
276.01	0.00502831342388961\\
277.01	0.00502848656678119\\
278.01	0.00502866291324529\\
279.01	0.00502884252341747\\
280.01	0.00502902545863208\\
281.01	0.00502921178144883\\
282.01	0.00502940155568045\\
283.01	0.00502959484642063\\
284.01	0.00502979172007285\\
285.01	0.00502999224438008\\
286.01	0.00503019648845489\\
287.01	0.00503040452281049\\
288.01	0.00503061641939237\\
289.01	0.00503083225161066\\
290.01	0.00503105209437366\\
291.01	0.00503127602412171\\
292.01	0.00503150411886187\\
293.01	0.00503173645820396\\
294.01	0.00503197312339637\\
295.01	0.00503221419736403\\
296.01	0.00503245976474617\\
297.01	0.0050327099119353\\
298.01	0.0050329647271169\\
299.01	0.00503322430031067\\
300.01	0.00503348872341158\\
301.01	0.00503375809023245\\
302.01	0.00503403249654735\\
303.01	0.00503431204013617\\
304.01	0.00503459682082916\\
305.01	0.00503488694055343\\
306.01	0.00503518250337986\\
307.01	0.00503548361557084\\
308.01	0.00503579038562912\\
309.01	0.0050361029243475\\
310.01	0.00503642134485913\\
311.01	0.00503674576268956\\
312.01	0.00503707629580859\\
313.01	0.00503741306468343\\
314.01	0.00503775619233326\\
315.01	0.00503810580438374\\
316.01	0.00503846202912313\\
317.01	0.00503882499755876\\
318.01	0.00503919484347427\\
319.01	0.00503957170348862\\
320.01	0.00503995571711445\\
321.01	0.00504034702681803\\
322.01	0.00504074577808022\\
323.01	0.00504115211945731\\
324.01	0.0050415662026435\\
325.01	0.00504198818253367\\
326.01	0.00504241821728675\\
327.01	0.00504285646839001\\
328.01	0.00504330310072384\\
329.01	0.00504375828262761\\
330.01	0.00504422218596549\\
331.01	0.00504469498619389\\
332.01	0.00504517686242904\\
333.01	0.00504566799751503\\
334.01	0.00504616857809335\\
335.01	0.00504667879467241\\
336.01	0.0050471988416986\\
337.01	0.00504772891762756\\
338.01	0.00504826922499717\\
339.01	0.00504881997050075\\
340.01	0.0050493813650619\\
341.01	0.00504995362391013\\
342.01	0.00505053696665841\\
343.01	0.00505113161738205\\
344.01	0.00505173780469856\\
345.01	0.00505235576185045\\
346.01	0.00505298572678886\\
347.01	0.00505362794226062\\
348.01	0.00505428265589696\\
349.01	0.00505495012030505\\
350.01	0.00505563059316302\\
351.01	0.00505632433731748\\
352.01	0.00505703162088522\\
353.01	0.00505775271735856\\
354.01	0.00505848790571424\\
355.01	0.00505923747052754\\
356.01	0.00506000170209087\\
357.01	0.0050607808965366\\
358.01	0.00506157535596578\\
359.01	0.005062385388582\\
360.01	0.00506321130883047\\
361.01	0.00506405343754281\\
362.01	0.0050649121020872\\
363.01	0.00506578763652464\\
364.01	0.00506668038176997\\
365.01	0.00506759068575829\\
366.01	0.00506851890361704\\
367.01	0.00506946539784152\\
368.01	0.00507043053847556\\
369.01	0.00507141470329618\\
370.01	0.00507241827800047\\
371.01	0.00507344165639692\\
372.01	0.00507448524059727\\
373.01	0.00507554944121311\\
374.01	0.00507663467755196\\
375.01	0.00507774137781693\\
376.01	0.00507886997930695\\
377.01	0.0050800209286208\\
378.01	0.00508119468186213\\
379.01	0.00508239170484881\\
380.01	0.00508361247332564\\
381.01	0.00508485747318098\\
382.01	0.00508612720066827\\
383.01	0.00508742216263094\\
384.01	0.00508874287673265\\
385.01	0.00509008987169214\\
386.01	0.00509146368752236\\
387.01	0.00509286487577354\\
388.01	0.00509429399978219\\
389.01	0.00509575163492443\\
390.01	0.00509723836887327\\
391.01	0.00509875480186192\\
392.01	0.00510030154695075\\
393.01	0.00510187923030017\\
394.01	0.00510348849144823\\
395.01	0.00510512998359208\\
396.01	0.00510680437387648\\
397.01	0.00510851234368537\\
398.01	0.00511025458893963\\
399.01	0.00511203182040038\\
400.01	0.00511384476397572\\
401.01	0.00511569416103477\\
402.01	0.00511758076872551\\
403.01	0.00511950536029861\\
404.01	0.00512146872543671\\
405.01	0.00512347167058867\\
406.01	0.00512551501930966\\
407.01	0.00512759961260707\\
408.01	0.0051297263092914\\
409.01	0.00513189598633347\\
410.01	0.00513410953922723\\
411.01	0.00513636788235847\\
412.01	0.00513867194937997\\
413.01	0.00514102269359204\\
414.01	0.00514342108833014\\
415.01	0.005145868127359\\
416.01	0.00514836482527282\\
417.01	0.00515091221790266\\
418.01	0.00515351136273091\\
419.01	0.00515616333931325\\
420.01	0.0051588692497074\\
421.01	0.00516163021891089\\
422.01	0.00516444739530594\\
423.01	0.00516732195111302\\
424.01	0.00517025508285339\\
425.01	0.00517324801182083\\
426.01	0.00517630198456262\\
427.01	0.00517941827337016\\
428.01	0.00518259817678116\\
429.01	0.00518584302009149\\
430.01	0.00518915415587828\\
431.01	0.00519253296453566\\
432.01	0.00519598085482181\\
433.01	0.00519949926441959\\
434.01	0.00520308966050941\\
435.01	0.0052067535403565\\
436.01	0.00521049243191227\\
437.01	0.0052143078944302\\
438.01	0.00521820151909683\\
439.01	0.00522217492967812\\
440.01	0.00522622978318196\\
441.01	0.00523036777053668\\
442.01	0.00523459061728625\\
443.01	0.00523890008430183\\
444.01	0.00524329796851115\\
445.01	0.00524778610364351\\
446.01	0.00525236636099316\\
447.01	0.00525704065019965\\
448.01	0.00526181092004457\\
449.01	0.00526667915926614\\
450.01	0.00527164739739069\\
451.01	0.00527671770558057\\
452.01	0.00528189219749942\\
453.01	0.00528717303019413\\
454.01	0.00529256240499335\\
455.01	0.00529806256842333\\
456.01	0.00530367581314065\\
457.01	0.00530940447888219\\
458.01	0.00531525095343356\\
459.01	0.0053212176736156\\
460.01	0.00532730712629066\\
461.01	0.00533352184938808\\
462.01	0.00533986443295212\\
463.01	0.00534633752021037\\
464.01	0.00535294380866658\\
465.01	0.00535968605121613\\
466.01	0.00536656705728716\\
467.01	0.00537358969400614\\
468.01	0.00538075688738979\\
469.01	0.00538807162356364\\
470.01	0.00539553695000666\\
471.01	0.00540315597682403\\
472.01	0.00541093187804739\\
473.01	0.00541886789296336\\
474.01	0.00542696732747124\\
475.01	0.00543523355546909\\
476.01	0.00544367002027044\\
477.01	0.00545228023605022\\
478.01	0.00546106778932162\\
479.01	0.00547003634044396\\
480.01	0.00547918962516099\\
481.01	0.00548853145617276\\
482.01	0.00549806572473786\\
483.01	0.00550779640231014\\
484.01	0.00551772754220767\\
485.01	0.00552786328131574\\
486.01	0.00553820784182571\\
487.01	0.00554876553300686\\
488.01	0.00555954075301628\\
489.01	0.00557053799074402\\
490.01	0.00558176182769582\\
491.01	0.00559321693991399\\
492.01	0.00560490809993744\\
493.01	0.00561684017880044\\
494.01	0.0056290181480723\\
495.01	0.0056414470819366\\
496.01	0.00565413215931353\\
497.01	0.00566707866602291\\
498.01	0.00568029199698998\\
499.01	0.0056937776584946\\
500.01	0.00570754127046308\\
501.01	0.00572158856880466\\
502.01	0.00573592540779119\\
503.01	0.00575055776248193\\
504.01	0.00576549173119376\\
505.01	0.00578073353801487\\
506.01	0.00579628953536469\\
507.01	0.00581216620659866\\
508.01	0.0058283701686583\\
509.01	0.00584490817476563\\
510.01	0.00586178711716329\\
511.01	0.00587901402989748\\
512.01	0.00589659609164545\\
513.01	0.00591454062858528\\
514.01	0.0059328551173065\\
515.01	0.00595154718776236\\
516.01	0.00597062462625876\\
517.01	0.00599009537848122\\
518.01	0.00600996755255487\\
519.01	0.00603024942213667\\
520.01	0.00605094942953484\\
521.01	0.00607207618885311\\
522.01	0.00609363848915531\\
523.01	0.00611564529764388\\
524.01	0.00613810576284839\\
525.01	0.00616102921781665\\
526.01	0.00618442518330081\\
527.01	0.00620830337092997\\
528.01	0.00623267368636038\\
529.01	0.00625754623239091\\
530.01	0.00628293131203315\\
531.01	0.00630883943152065\\
532.01	0.00633528130324207\\
533.01	0.00636226784858186\\
534.01	0.00638981020064584\\
535.01	0.00641791970685204\\
536.01	0.00644660793135991\\
537.01	0.00647588665731036\\
538.01	0.00650576788884577\\
539.01	0.0065362638528734\\
540.01	0.00656738700053343\\
541.01	0.00659915000832749\\
542.01	0.00663156577885837\\
543.01	0.00666464744112576\\
544.01	0.00669840835031567\\
545.01	0.00673286208701748\\
546.01	0.00676802245578906\\
547.01	0.00680390348298683\\
548.01	0.00684051941376644\\
549.01	0.00687788470814671\\
550.01	0.0069160140360215\\
551.01	0.0069549222709883\\
552.01	0.00699462448284956\\
553.01	0.00703513592862633\\
554.01	0.00707647204190588\\
555.01	0.00711864842032717\\
556.01	0.00716168081098622\\
557.01	0.00720558509352023\\
558.01	0.00725037726060528\\
559.01	0.00729607339557415\\
560.01	0.0073426896468327\\
561.01	0.00739024219871955\\
562.01	0.00743874723842159\\
563.01	0.00748822091851965\\
564.01	0.00753867931470128\\
565.01	0.00759013837813551\\
566.01	0.00764261388196354\\
567.01	0.0076961213613144\\
568.01	0.00775067604621158\\
569.01	0.00780629278669433\\
570.01	0.00786298596943516\\
571.01	0.00792076942509971\\
572.01	0.0079796563256667\\
573.01	0.00803965907090454\\
574.01	0.00810078916320144\\
575.01	0.00816305706995899\\
576.01	0.00822647207281021\\
577.01	0.00829104210300508\\
578.01	0.00835677356244437\\
579.01	0.00842367113004633\\
580.01	0.00849173755342209\\
581.01	0.00856097342624057\\
582.01	0.00863137695221383\\
583.01	0.00870294369737224\\
584.01	0.00877566633327528\\
585.01	0.00884953437508313\\
586.01	0.00892453392008026\\
587.01	0.00900064739439612\\
588.01	0.00907785331843724\\
589.01	0.00915612610509321\\
590.01	0.00923543590930589\\
591.01	0.00931574855334877\\
592.01	0.00939702555946881\\
593.01	0.00947922433078829\\
594.01	0.00956229853304221\\
595.01	0.00964619874445287\\
596.01	0.00973087345958245\\
597.01	0.0098162705563152\\
598.01	0.00990228776786639\\
599.01	0.00996919203046377\\
599.02	0.00996973314335038\\
599.03	0.0099702709571694\\
599.04	0.00997080543920336\\
599.05	0.0099713365564138\\
599.06	0.00997186427543808\\
599.07	0.0099723885625862\\
599.08	0.00997290938383756\\
599.09	0.00997342670483771\\
599.1	0.00997394049089505\\
599.11	0.00997445070697751\\
599.12	0.00997495731770918\\
599.13	0.00997546028736696\\
599.14	0.00997595957987707\\
599.15	0.00997645515881166\\
599.16	0.00997694698738526\\
599.17	0.00997743502845132\\
599.18	0.00997791924449856\\
599.19	0.00997839959764748\\
599.2	0.00997887604964662\\
599.21	0.00997934856186899\\
599.22	0.00997981709530829\\
599.23	0.00998028161057521\\
599.24	0.00998074206789365\\
599.25	0.0099811984270969\\
599.26	0.00998165064762378\\
599.27	0.00998209868851477\\
599.28	0.00998254250840806\\
599.29	0.00998298206553559\\
599.3	0.00998341731771905\\
599.31	0.00998384822236584\\
599.32	0.00998427473646497\\
599.33	0.00998469681658292\\
599.34	0.00998511441885952\\
599.35	0.0099855274990037\\
599.36	0.00998593601228926\\
599.37	0.00998633991355056\\
599.38	0.00998673915717821\\
599.39	0.00998713369711469\\
599.4	0.00998752348684992\\
599.41	0.00998790847811157\\
599.42	0.00998828861974491\\
599.43	0.00998866386008804\\
599.44	0.00998903414696681\\
599.45	0.00998939942768985\\
599.46	0.00998975964904344\\
599.47	0.00999011475728636\\
599.48	0.00999046469814472\\
599.49	0.00999080941680669\\
599.5	0.00999114885791725\\
599.51	0.00999148296557278\\
599.52	0.0099918116833157\\
599.53	0.00999213495412901\\
599.54	0.00999245272043077\\
599.55	0.00999276492406855\\
599.56	0.00999307150631381\\
599.57	0.00999337240785624\\
599.58	0.00999366756879799\\
599.59	0.00999395692864795\\
599.6	0.00999424042631586\\
599.61	0.00999451800010642\\
599.62	0.00999478958771336\\
599.63	0.0099950551262134\\
599.64	0.00999531455206019\\
599.65	0.00999556780107816\\
599.66	0.00999581480845634\\
599.67	0.00999605550874211\\
599.68	0.00999628983583487\\
599.69	0.00999651772297967\\
599.7	0.00999673910276075\\
599.71	0.00999695390709509\\
599.72	0.00999716206722577\\
599.73	0.00999736351371538\\
599.74	0.00999755817643933\\
599.75	0.00999774598457904\\
599.76	0.00999792686661517\\
599.77	0.00999810075032068\\
599.78	0.00999826756275386\\
599.79	0.00999842723025133\\
599.8	0.00999857967842092\\
599.81	0.0099987248321345\\
599.82	0.00999886261552071\\
599.83	0.00999899295195769\\
599.84	0.00999911576406567\\
599.85	0.00999923097369951\\
599.86	0.00999933850194117\\
599.87	0.00999943826909211\\
599.88	0.00999953019466558\\
599.89	0.00999961419737891\\
599.9	0.00999969019514566\\
599.91	0.00999975810506767\\
599.92	0.00999981784342713\\
599.93	0.00999986932567848\\
599.94	0.00999991246644028\\
599.95	0.00999994717948697\\
599.96	0.00999997337774056\\
599.97	0.00999999097326228\\
599.98	0.00999999987724406\\
599.99	0.01\\
600	0.01\\
};
\addplot [color=mycolor5,solid,forget plot]
  table[row sep=crcr]{%
0.01	0.00501105323809203\\
1.01	0.00501105434915347\\
2.01	0.00501105548219559\\
3.01	0.00501105663765027\\
4.01	0.00501105781595748\\
5.01	0.00501105901756574\\
6.01	0.00501106024293257\\
7.01	0.00501106149252414\\
8.01	0.00501106276681591\\
9.01	0.00501106406629223\\
10.01	0.00501106539144701\\
11.01	0.00501106674278391\\
12.01	0.00501106812081607\\
13.01	0.00501106952606685\\
14.01	0.00501107095906948\\
15.01	0.0050110724203678\\
16.01	0.00501107391051601\\
17.01	0.00501107543007902\\
18.01	0.00501107697963291\\
19.01	0.0050110785597646\\
20.01	0.00501108017107262\\
21.01	0.00501108181416682\\
22.01	0.00501108348966919\\
23.01	0.00501108519821353\\
24.01	0.00501108694044591\\
25.01	0.00501108871702494\\
26.01	0.00501109052862167\\
27.01	0.00501109237592055\\
28.01	0.00501109425961895\\
29.01	0.00501109618042773\\
30.01	0.00501109813907175\\
31.01	0.00501110013628946\\
32.01	0.00501110217283358\\
33.01	0.00501110424947158\\
34.01	0.00501110636698546\\
35.01	0.00501110852617254\\
36.01	0.00501111072784531\\
37.01	0.00501111297283197\\
38.01	0.00501111526197658\\
39.01	0.00501111759613956\\
40.01	0.00501111997619762\\
41.01	0.00501112240304449\\
42.01	0.00501112487759112\\
43.01	0.00501112740076579\\
44.01	0.00501112997351484\\
45.01	0.00501113259680265\\
46.01	0.00501113527161187\\
47.01	0.00501113799894434\\
48.01	0.00501114077982097\\
49.01	0.00501114361528207\\
50.01	0.00501114650638813\\
51.01	0.00501114945421972\\
52.01	0.00501115245987828\\
53.01	0.00501115552448617\\
54.01	0.00501115864918734\\
55.01	0.00501116183514756\\
56.01	0.00501116508355486\\
57.01	0.0050111683956198\\
58.01	0.00501117177257621\\
59.01	0.00501117521568152\\
60.01	0.00501117872621715\\
61.01	0.00501118230548872\\
62.01	0.00501118595482709\\
63.01	0.00501118967558804\\
64.01	0.00501119346915346\\
65.01	0.00501119733693153\\
66.01	0.00501120128035717\\
67.01	0.00501120530089255\\
68.01	0.00501120940002762\\
69.01	0.00501121357928062\\
70.01	0.00501121784019883\\
71.01	0.00501122218435855\\
72.01	0.00501122661336638\\
73.01	0.00501123112885897\\
74.01	0.0050112357325043\\
75.01	0.00501124042600191\\
76.01	0.00501124521108336\\
77.01	0.00501125008951314\\
78.01	0.00501125506308905\\
79.01	0.005011260133643\\
80.01	0.0050112653030413\\
81.01	0.00501127057318572\\
82.01	0.00501127594601383\\
83.01	0.00501128142349977\\
84.01	0.0050112870076547\\
85.01	0.00501129270052825\\
86.01	0.0050112985042082\\
87.01	0.00501130442082177\\
88.01	0.00501131045253633\\
89.01	0.00501131660155973\\
90.01	0.0050113228701416\\
91.01	0.00501132926057374\\
92.01	0.00501133577519107\\
93.01	0.00501134241637217\\
94.01	0.00501134918654049\\
95.01	0.00501135608816462\\
96.01	0.00501136312375958\\
97.01	0.00501137029588755\\
98.01	0.00501137760715856\\
99.01	0.00501138506023138\\
100.01	0.00501139265781471\\
101.01	0.00501140040266766\\
102.01	0.00501140829760099\\
103.01	0.00501141634547772\\
104.01	0.00501142454921466\\
105.01	0.00501143291178256\\
106.01	0.0050114414362077\\
107.01	0.00501145012557279\\
108.01	0.00501145898301772\\
109.01	0.00501146801174081\\
110.01	0.00501147721499982\\
111.01	0.00501148659611307\\
112.01	0.00501149615846029\\
113.01	0.00501150590548408\\
114.01	0.00501151584069057\\
115.01	0.00501152596765125\\
116.01	0.00501153629000348\\
117.01	0.00501154681145168\\
118.01	0.00501155753576905\\
119.01	0.00501156846679843\\
120.01	0.0050115796084534\\
121.01	0.00501159096471993\\
122.01	0.00501160253965742\\
123.01	0.00501161433740002\\
124.01	0.00501162636215804\\
125.01	0.00501163861821899\\
126.01	0.00501165110994933\\
127.01	0.00501166384179594\\
128.01	0.0050116768182871\\
129.01	0.00501169004403393\\
130.01	0.00501170352373236\\
131.01	0.0050117172621643\\
132.01	0.0050117312641991\\
133.01	0.00501174553479515\\
134.01	0.00501176007900125\\
135.01	0.00501177490195867\\
136.01	0.00501179000890223\\
137.01	0.00501180540516227\\
138.01	0.0050118210961662\\
139.01	0.00501183708743999\\
140.01	0.00501185338461066\\
141.01	0.00501186999340709\\
142.01	0.00501188691966192\\
143.01	0.00501190416931392\\
144.01	0.00501192174840956\\
145.01	0.00501193966310458\\
146.01	0.00501195791966616\\
147.01	0.00501197652447485\\
148.01	0.00501199548402642\\
149.01	0.00501201480493388\\
150.01	0.00501203449392944\\
151.01	0.00501205455786659\\
152.01	0.00501207500372232\\
153.01	0.00501209583859905\\
154.01	0.00501211706972662\\
155.01	0.00501213870446485\\
156.01	0.00501216075030556\\
157.01	0.00501218321487495\\
158.01	0.00501220610593539\\
159.01	0.0050122294313884\\
160.01	0.00501225319927683\\
161.01	0.00501227741778669\\
162.01	0.00501230209525023\\
163.01	0.00501232724014815\\
164.01	0.00501235286111248\\
165.01	0.00501237896692812\\
166.01	0.00501240556653661\\
167.01	0.00501243266903807\\
168.01	0.00501246028369394\\
169.01	0.00501248841992995\\
170.01	0.00501251708733844\\
171.01	0.00501254629568146\\
172.01	0.00501257605489341\\
173.01	0.00501260637508434\\
174.01	0.00501263726654223\\
175.01	0.00501266873973628\\
176.01	0.00501270080532015\\
177.01	0.00501273347413449\\
178.01	0.00501276675721027\\
179.01	0.00501280066577221\\
180.01	0.00501283521124173\\
181.01	0.00501287040523979\\
182.01	0.00501290625959114\\
183.01	0.00501294278632643\\
184.01	0.00501297999768707\\
185.01	0.00501301790612706\\
186.01	0.00501305652431797\\
187.01	0.00501309586515127\\
188.01	0.00501313594174276\\
189.01	0.00501317676743585\\
190.01	0.00501321835580532\\
191.01	0.00501326072066127\\
192.01	0.00501330387605255\\
193.01	0.00501334783627098\\
194.01	0.00501339261585526\\
195.01	0.00501343822959488\\
196.01	0.00501348469253411\\
197.01	0.00501353201997619\\
198.01	0.00501358022748774\\
199.01	0.00501362933090251\\
200.01	0.00501367934632607\\
201.01	0.00501373029014007\\
202.01	0.00501378217900687\\
203.01	0.00501383502987365\\
204.01	0.00501388885997742\\
205.01	0.00501394368684922\\
206.01	0.0050139995283193\\
207.01	0.00501405640252163\\
208.01	0.00501411432789881\\
209.01	0.00501417332320697\\
210.01	0.00501423340752125\\
211.01	0.00501429460023999\\
212.01	0.00501435692109077\\
213.01	0.00501442039013514\\
214.01	0.00501448502777423\\
215.01	0.00501455085475419\\
216.01	0.00501461789217139\\
217.01	0.00501468616147842\\
218.01	0.00501475568448942\\
219.01	0.00501482648338628\\
220.01	0.00501489858072407\\
221.01	0.00501497199943747\\
222.01	0.00501504676284646\\
223.01	0.00501512289466254\\
224.01	0.00501520041899522\\
225.01	0.00501527936035853\\
226.01	0.00501535974367673\\
227.01	0.00501544159429182\\
228.01	0.00501552493796999\\
229.01	0.00501560980090822\\
230.01	0.0050156962097412\\
231.01	0.00501578419154891\\
232.01	0.00501587377386335\\
233.01	0.00501596498467613\\
234.01	0.00501605785244565\\
235.01	0.00501615240610501\\
236.01	0.00501624867506979\\
237.01	0.00501634668924585\\
238.01	0.00501644647903703\\
239.01	0.00501654807535393\\
240.01	0.00501665150962216\\
241.01	0.00501675681379021\\
242.01	0.00501686402033923\\
243.01	0.00501697316229131\\
244.01	0.0050170842732184\\
245.01	0.00501719738725173\\
246.01	0.00501731253909156\\
247.01	0.0050174297640166\\
248.01	0.00501754909789333\\
249.01	0.00501767057718719\\
250.01	0.00501779423897193\\
251.01	0.00501792012094072\\
252.01	0.00501804826141644\\
253.01	0.00501817869936318\\
254.01	0.00501831147439737\\
255.01	0.00501844662679896\\
256.01	0.00501858419752384\\
257.01	0.00501872422821566\\
258.01	0.00501886676121815\\
259.01	0.00501901183958807\\
260.01	0.0050191595071083\\
261.01	0.00501930980830093\\
262.01	0.00501946278844137\\
263.01	0.00501961849357192\\
264.01	0.00501977697051723\\
265.01	0.00501993826689813\\
266.01	0.00502010243114755\\
267.01	0.00502026951252586\\
268.01	0.0050204395611374\\
269.01	0.0050206126279468\\
270.01	0.00502078876479636\\
271.01	0.00502096802442337\\
272.01	0.0050211504604781\\
273.01	0.00502133612754316\\
274.01	0.00502152508115168\\
275.01	0.0050217173778079\\
276.01	0.00502191307500735\\
277.01	0.00502211223125749\\
278.01	0.00502231490609989\\
279.01	0.00502252116013228\\
280.01	0.0050227310550317\\
281.01	0.00502294465357837\\
282.01	0.00502316201967989\\
283.01	0.00502338321839679\\
284.01	0.00502360831596875\\
285.01	0.0050238373798413\\
286.01	0.00502407047869415\\
287.01	0.00502430768246917\\
288.01	0.00502454906240107\\
289.01	0.0050247946910473\\
290.01	0.00502504464232019\\
291.01	0.00502529899151921\\
292.01	0.00502555781536547\\
293.01	0.00502582119203565\\
294.01	0.00502608920119921\\
295.01	0.00502636192405458\\
296.01	0.00502663944336823\\
297.01	0.00502692184351397\\
298.01	0.00502720921051425\\
299.01	0.00502750163208222\\
300.01	0.00502779919766547\\
301.01	0.00502810199849096\\
302.01	0.00502841012761156\\
303.01	0.00502872367995369\\
304.01	0.00502904275236694\\
305.01	0.00502936744367477\\
306.01	0.00502969785472668\\
307.01	0.00503003408845215\\
308.01	0.00503037624991633\\
309.01	0.00503072444637663\\
310.01	0.00503107878734196\\
311.01	0.00503143938463199\\
312.01	0.00503180635243938\\
313.01	0.00503217980739329\\
314.01	0.00503255986862387\\
315.01	0.00503294665782869\\
316.01	0.00503334029934097\\
317.01	0.00503374092019886\\
318.01	0.00503414865021638\\
319.01	0.0050345636220554\\
320.01	0.0050349859712993\\
321.01	0.0050354158365284\\
322.01	0.00503585335939509\\
323.01	0.00503629868470189\\
324.01	0.00503675196047861\\
325.01	0.00503721333806179\\
326.01	0.00503768297217426\\
327.01	0.00503816102100541\\
328.01	0.00503864764629162\\
329.01	0.00503914301339753\\
330.01	0.00503964729139655\\
331.01	0.00504016065315194\\
332.01	0.00504068327539701\\
333.01	0.00504121533881544\\
334.01	0.00504175702811992\\
335.01	0.00504230853213031\\
336.01	0.00504287004385055\\
337.01	0.00504344176054386\\
338.01	0.00504402388380591\\
339.01	0.00504461661963666\\
340.01	0.00504522017850959\\
341.01	0.00504583477543853\\
342.01	0.00504646063004185\\
343.01	0.00504709796660381\\
344.01	0.00504774701413349\\
345.01	0.00504840800641998\\
346.01	0.00504908118208582\\
347.01	0.00504976678463623\\
348.01	0.00505046506250637\\
349.01	0.0050511762691063\\
350.01	0.0050519006628631\\
351.01	0.00505263850726183\\
352.01	0.00505339007088509\\
353.01	0.00505415562745272\\
354.01	0.00505493545586137\\
355.01	0.00505572984022609\\
356.01	0.00505653906992413\\
357.01	0.0050573634396435\\
358.01	0.00505820324943602\\
359.01	0.00505905880477907\\
360.01	0.00505993041664424\\
361.01	0.00506081840157829\\
362.01	0.00506172308179569\\
363.01	0.00506264478528499\\
364.01	0.00506358384593185\\
365.01	0.00506454060365845\\
366.01	0.00506551540458144\\
367.01	0.00506650860118923\\
368.01	0.00506752055253759\\
369.01	0.00506855162446482\\
370.01	0.00506960218982375\\
371.01	0.0050706726287294\\
372.01	0.00507176332881949\\
373.01	0.00507287468552395\\
374.01	0.00507400710233857\\
375.01	0.00507516099110038\\
376.01	0.00507633677225852\\
377.01	0.00507753487513728\\
378.01	0.00507875573819004\\
379.01	0.0050799998092432\\
380.01	0.00508126754573331\\
381.01	0.00508255941494219\\
382.01	0.00508387589423588\\
383.01	0.00508521747130959\\
384.01	0.00508658464443858\\
385.01	0.00508797792273444\\
386.01	0.0050893978264083\\
387.01	0.00509084488704023\\
388.01	0.00509231964785468\\
389.01	0.00509382266400238\\
390.01	0.00509535450284851\\
391.01	0.00509691574426789\\
392.01	0.00509850698094573\\
393.01	0.00510012881868571\\
394.01	0.00510178187672403\\
395.01	0.0051034667880506\\
396.01	0.00510518419973573\\
397.01	0.00510693477326443\\
398.01	0.00510871918487622\\
399.01	0.00511053812591178\\
400.01	0.00511239230316565\\
401.01	0.00511428243924573\\
402.01	0.00511620927293786\\
403.01	0.0051181735595774\\
404.01	0.00512017607142572\\
405.01	0.00512221759805365\\
406.01	0.0051242989467291\\
407.01	0.00512642094281104\\
408.01	0.00512858443014799\\
409.01	0.00513079027148199\\
410.01	0.00513303934885734\\
411.01	0.00513533256403319\\
412.01	0.00513767083890113\\
413.01	0.00514005511590683\\
414.01	0.00514248635847511\\
415.01	0.00514496555143869\\
416.01	0.00514749370147027\\
417.01	0.00515007183751852\\
418.01	0.00515270101124604\\
419.01	0.00515538229747044\\
420.01	0.00515811679460831\\
421.01	0.00516090562512184\\
422.01	0.00516374993596685\\
423.01	0.00516665089904494\\
424.01	0.0051696097116572\\
425.01	0.00517262759696037\\
426.01	0.00517570580442675\\
427.01	0.00517884561030678\\
428.01	0.0051820483180947\\
429.01	0.00518531525899951\\
430.01	0.00518864779242004\\
431.01	0.00519204730642486\\
432.01	0.0051955152182399\\
433.01	0.00519905297474199\\
434.01	0.00520266205296244\\
435.01	0.00520634396059898\\
436.01	0.00521010023654003\\
437.01	0.005213932451401\\
438.01	0.00521784220807515\\
439.01	0.00522183114230099\\
440.01	0.00522590092324672\\
441.01	0.00523005325411513\\
442.01	0.00523428987276905\\
443.01	0.00523861255238114\\
444.01	0.00524302310210695\\
445.01	0.00524752336778546\\
446.01	0.00525211523266651\\
447.01	0.00525680061816582\\
448.01	0.00526158148465021\\
449.01	0.00526645983225105\\
450.01	0.00527143770170575\\
451.01	0.00527651717522777\\
452.01	0.00528170037740337\\
453.01	0.00528698947611138\\
454.01	0.00529238668346678\\
455.01	0.00529789425678259\\
456.01	0.00530351449954898\\
457.01	0.00530924976242536\\
458.01	0.00531510244424384\\
459.01	0.00532107499302052\\
460.01	0.00532716990697359\\
461.01	0.00533338973554799\\
462.01	0.00533973708044658\\
463.01	0.00534621459667049\\
464.01	0.00535282499357081\\
465.01	0.00535957103591755\\
466.01	0.00536645554498819\\
467.01	0.00537348139968213\\
468.01	0.00538065153766244\\
469.01	0.00538796895652568\\
470.01	0.0053954367150031\\
471.01	0.00540305793419209\\
472.01	0.00541083579882057\\
473.01	0.00541877355854389\\
474.01	0.00542687452927569\\
475.01	0.00543514209455356\\
476.01	0.0054435797069395\\
477.01	0.0054521908894558\\
478.01	0.00546097923705703\\
479.01	0.0054699484181375\\
480.01	0.0054791021760752\\
481.01	0.00548844433081055\\
482.01	0.00549797878046185\\
483.01	0.00550770950297491\\
484.01	0.00551764055780818\\
485.01	0.00552777608765277\\
486.01	0.00553812032018593\\
487.01	0.00554867756985955\\
488.01	0.00555945223972305\\
489.01	0.00557044882328042\\
490.01	0.00558167190638299\\
491.01	0.00559312616915737\\
492.01	0.00560481638797075\\
493.01	0.00561674743743441\\
494.01	0.00562892429244649\\
495.01	0.00564135203027566\\
496.01	0.00565403583268666\\
497.01	0.00566698098810914\\
498.01	0.00568019289385005\\
499.01	0.00569367705835026\\
500.01	0.00570743910348621\\
501.01	0.00572148476691542\\
502.01	0.00573581990446716\\
503.01	0.00575045049257814\\
504.01	0.005765382630772\\
505.01	0.0057806225441852\\
506.01	0.00579617658613588\\
507.01	0.0058120512407385\\
508.01	0.00582825312556255\\
509.01	0.00584478899433546\\
510.01	0.0058616657396896\\
511.01	0.00587889039595234\\
512.01	0.00589647014198016\\
513.01	0.00591441230403314\\
514.01	0.00593272435869262\\
515.01	0.00595141393581679\\
516.01	0.0059704888215366\\
517.01	0.00598995696128612\\
518.01	0.00600982646286827\\
519.01	0.00603010559955078\\
520.01	0.00605080281319069\\
521.01	0.00607192671738277\\
522.01	0.00609348610062707\\
523.01	0.00611548992951208\\
524.01	0.00613794735190604\\
525.01	0.00616086770015153\\
526.01	0.0061842604942537\\
527.01	0.00620813544505668\\
528.01	0.00623250245739489\\
529.01	0.0062573716332129\\
530.01	0.00628275327463575\\
531.01	0.00630865788698202\\
532.01	0.00633509618169938\\
533.01	0.00636207907920586\\
534.01	0.00638961771161965\\
535.01	0.00641772342535149\\
536.01	0.00644640778353617\\
537.01	0.00647568256827581\\
538.01	0.00650555978266212\\
539.01	0.00653605165254259\\
540.01	0.00656717062799204\\
541.01	0.00659892938444444\\
542.01	0.00663134082343626\\
543.01	0.00666441807290583\\
544.01	0.00669817448698827\\
545.01	0.00673262364523612\\
546.01	0.00676777935119077\\
547.01	0.00680365563021905\\
548.01	0.00684026672651947\\
549.01	0.00687762709919433\\
550.01	0.00691575141726969\\
551.01	0.00695465455353335\\
552.01	0.00699435157704655\\
553.01	0.00703485774416845\\
554.01	0.00707618848791713\\
555.01	0.00711835940546861\\
556.01	0.00716138624357784\\
557.01	0.00720528488168006\\
558.01	0.00725007131240749\\
559.01	0.00729576161922826\\
560.01	0.00734237195088536\\
561.01	0.00738991849228227\\
562.01	0.00743841743142593\\
563.01	0.00748788492200297\\
564.01	0.00753833704112559\\
565.01	0.00758978974174271\\
566.01	0.0076422587991696\\
567.01	0.0076957597511456\\
568.01	0.00775030783078647\\
569.01	0.00780591789175387\\
570.01	0.007862604324924\\
571.01	0.00792038096580129\\
572.01	0.00797926099189406\\
573.01	0.00803925680924829\\
574.01	0.00810037992733435\\
575.01	0.00816264082149692\\
576.01	0.00822604878222494\\
577.01	0.00829061175058319\\
578.01	0.00835633613928231\\
579.01	0.0084232266390671\\
580.01	0.00849128601039259\\
581.01	0.00856051486076035\\
582.01	0.00863091140863674\\
583.01	0.00870247123560845\\
584.01	0.0087751870294043\\
585.01	0.00884904832168768\\
586.01	0.00892404122618365\\
587.01	0.00900014818485135\\
588.01	0.00907734773257349\\
589.01	0.00915561429437061\\
590.01	0.00923491803366146\\
591.01	0.00931522477583024\\
592.01	0.00939649603864537\\
593.01	0.00947868921029156\\
594.01	0.00956175792742154\\
595.01	0.00964565272031552\\
596.01	0.00973032201072283\\
597.01	0.00981571357120332\\
598.01	0.00990176516613374\\
599.01	0.00996919203046377\\
599.02	0.00996973314335038\\
599.03	0.0099702709571694\\
599.04	0.00997080543920336\\
599.05	0.0099713365564138\\
599.06	0.00997186427543808\\
599.07	0.0099723885625862\\
599.08	0.00997290938383756\\
599.09	0.00997342670483771\\
599.1	0.00997394049089505\\
599.11	0.00997445070697751\\
599.12	0.00997495731770918\\
599.13	0.00997546028736696\\
599.14	0.00997595957987707\\
599.15	0.00997645515881165\\
599.16	0.00997694698738526\\
599.17	0.00997743502845131\\
599.18	0.00997791924449856\\
599.19	0.00997839959764748\\
599.2	0.00997887604964662\\
599.21	0.00997934856186899\\
599.22	0.00997981709530829\\
599.23	0.00998028161057521\\
599.24	0.00998074206789365\\
599.25	0.0099811984270969\\
599.26	0.00998165064762378\\
599.27	0.00998209868851477\\
599.28	0.00998254250840806\\
599.29	0.00998298206553559\\
599.3	0.00998341731771905\\
599.31	0.00998384822236584\\
599.32	0.00998427473646497\\
599.33	0.00998469681658292\\
599.34	0.00998511441885952\\
599.35	0.0099855274990037\\
599.36	0.00998593601228926\\
599.37	0.00998633991355056\\
599.38	0.00998673915717821\\
599.39	0.00998713369711469\\
599.4	0.00998752348684992\\
599.41	0.00998790847811157\\
599.42	0.00998828861974491\\
599.43	0.00998866386008803\\
599.44	0.00998903414696681\\
599.45	0.00998939942768986\\
599.46	0.00998975964904344\\
599.47	0.00999011475728636\\
599.48	0.00999046469814472\\
599.49	0.00999080941680669\\
599.5	0.00999114885791725\\
599.51	0.00999148296557278\\
599.52	0.0099918116833157\\
599.53	0.00999213495412901\\
599.54	0.00999245272043077\\
599.55	0.00999276492406855\\
599.56	0.00999307150631381\\
599.57	0.00999337240785624\\
599.58	0.00999366756879799\\
599.59	0.00999395692864795\\
599.6	0.00999424042631585\\
599.61	0.00999451800010642\\
599.62	0.00999478958771336\\
599.63	0.0099950551262134\\
599.64	0.00999531455206019\\
599.65	0.00999556780107816\\
599.66	0.00999581480845634\\
599.67	0.00999605550874211\\
599.68	0.00999628983583487\\
599.69	0.00999651772297967\\
599.7	0.00999673910276076\\
599.71	0.00999695390709509\\
599.72	0.00999716206722577\\
599.73	0.00999736351371538\\
599.74	0.00999755817643933\\
599.75	0.00999774598457904\\
599.76	0.00999792686661517\\
599.77	0.00999810075032068\\
599.78	0.00999826756275386\\
599.79	0.00999842723025133\\
599.8	0.00999857967842092\\
599.81	0.0099987248321345\\
599.82	0.00999886261552071\\
599.83	0.00999899295195769\\
599.84	0.00999911576406567\\
599.85	0.00999923097369951\\
599.86	0.00999933850194117\\
599.87	0.00999943826909211\\
599.88	0.00999953019466558\\
599.89	0.00999961419737891\\
599.9	0.00999969019514566\\
599.91	0.00999975810506767\\
599.92	0.00999981784342713\\
599.93	0.00999986932567848\\
599.94	0.00999991246644028\\
599.95	0.00999994717948697\\
599.96	0.00999997337774056\\
599.97	0.00999999097326228\\
599.98	0.00999999987724406\\
599.99	0.01\\
600	0.01\\
};
\addplot [color=mycolor6,solid,forget plot]
  table[row sep=crcr]{%
0.01	0.00499338655419056\\
1.01	0.00499338786632048\\
2.01	0.00499338920466236\\
3.01	0.00499339056973667\\
4.01	0.00499339196207427\\
5.01	0.00499339338221642\\
6.01	0.00499339483071506\\
7.01	0.00499339630813272\\
8.01	0.00499339781504335\\
9.01	0.0049933993520319\\
10.01	0.00499340091969511\\
11.01	0.00499340251864137\\
12.01	0.00499340414949088\\
13.01	0.00499340581287615\\
14.01	0.00499340750944205\\
15.01	0.00499340923984605\\
16.01	0.00499341100475871\\
17.01	0.00499341280486372\\
18.01	0.00499341464085801\\
19.01	0.00499341651345235\\
20.01	0.00499341842337139\\
21.01	0.00499342037135406\\
22.01	0.00499342235815352\\
23.01	0.00499342438453793\\
24.01	0.00499342645129039\\
25.01	0.00499342855920965\\
26.01	0.00499343070910956\\
27.01	0.00499343290182029\\
28.01	0.00499343513818806\\
29.01	0.00499343741907576\\
30.01	0.00499343974536304\\
31.01	0.00499344211794697\\
32.01	0.00499344453774193\\
33.01	0.00499344700568019\\
34.01	0.00499344952271234\\
35.01	0.00499345208980734\\
36.01	0.00499345470795319\\
37.01	0.00499345737815723\\
38.01	0.00499346010144638\\
39.01	0.00499346287886737\\
40.01	0.0049934657114879\\
41.01	0.00499346860039604\\
42.01	0.00499347154670089\\
43.01	0.00499347455153364\\
44.01	0.00499347761604714\\
45.01	0.00499348074141685\\
46.01	0.00499348392884111\\
47.01	0.0049934871795414\\
48.01	0.00499349049476293\\
49.01	0.00499349387577538\\
50.01	0.00499349732387294\\
51.01	0.00499350084037486\\
52.01	0.00499350442662611\\
53.01	0.00499350808399769\\
54.01	0.00499351181388723\\
55.01	0.00499351561771942\\
56.01	0.00499351949694649\\
57.01	0.00499352345304914\\
58.01	0.00499352748753636\\
59.01	0.00499353160194656\\
60.01	0.00499353579784786\\
61.01	0.00499354007683879\\
62.01	0.00499354444054856\\
63.01	0.0049935488906382\\
64.01	0.0049935534288007\\
65.01	0.00499355805676154\\
66.01	0.00499356277627991\\
67.01	0.0049935675891484\\
68.01	0.0049935724971947\\
69.01	0.00499357750228167\\
70.01	0.00499358260630776\\
71.01	0.00499358781120819\\
72.01	0.00499359311895536\\
73.01	0.00499359853155977\\
74.01	0.00499360405107056\\
75.01	0.00499360967957618\\
76.01	0.00499361541920532\\
77.01	0.00499362127212724\\
78.01	0.0049936272405534\\
79.01	0.00499363332673711\\
80.01	0.00499363953297549\\
81.01	0.00499364586160916\\
82.01	0.00499365231502408\\
83.01	0.00499365889565152\\
84.01	0.00499366560596964\\
85.01	0.0049936724485036\\
86.01	0.00499367942582712\\
87.01	0.00499368654056336\\
88.01	0.00499369379538488\\
89.01	0.00499370119301604\\
90.01	0.00499370873623257\\
91.01	0.00499371642786353\\
92.01	0.00499372427079173\\
93.01	0.00499373226795513\\
94.01	0.00499374042234736\\
95.01	0.00499374873701904\\
96.01	0.00499375721507912\\
97.01	0.00499376585969533\\
98.01	0.00499377467409542\\
99.01	0.004993783661569\\
100.01	0.00499379282546757\\
101.01	0.00499380216920652\\
102.01	0.00499381169626573\\
103.01	0.00499382141019117\\
104.01	0.00499383131459584\\
105.01	0.00499384141316096\\
106.01	0.0049938517096375\\
107.01	0.00499386220784727\\
108.01	0.0049938729116842\\
109.01	0.00499388382511565\\
110.01	0.00499389495218394\\
111.01	0.00499390629700729\\
112.01	0.0049939178637817\\
113.01	0.00499392965678191\\
114.01	0.00499394168036323\\
115.01	0.00499395393896263\\
116.01	0.00499396643710051\\
117.01	0.00499397917938196\\
118.01	0.00499399217049852\\
119.01	0.00499400541522942\\
120.01	0.00499401891844364\\
121.01	0.00499403268510105\\
122.01	0.0049940467202543\\
123.01	0.00499406102905034\\
124.01	0.00499407561673199\\
125.01	0.00499409048864025\\
126.01	0.00499410565021534\\
127.01	0.00499412110699898\\
128.01	0.00499413686463544\\
129.01	0.00499415292887462\\
130.01	0.00499416930557263\\
131.01	0.00499418600069457\\
132.01	0.0049942030203159\\
133.01	0.00499422037062471\\
134.01	0.00499423805792357\\
135.01	0.00499425608863156\\
136.01	0.00499427446928641\\
137.01	0.00499429320654646\\
138.01	0.00499431230719284\\
139.01	0.00499433177813158\\
140.01	0.00499435162639551\\
141.01	0.00499437185914724\\
142.01	0.00499439248368068\\
143.01	0.00499441350742372\\
144.01	0.00499443493794008\\
145.01	0.00499445678293226\\
146.01	0.0049944790502436\\
147.01	0.00499450174786076\\
148.01	0.00499452488391619\\
149.01	0.00499454846669074\\
150.01	0.00499457250461605\\
151.01	0.00499459700627705\\
152.01	0.00499462198041467\\
153.01	0.00499464743592884\\
154.01	0.00499467338188067\\
155.01	0.00499469982749551\\
156.01	0.00499472678216545\\
157.01	0.00499475425545247\\
158.01	0.00499478225709107\\
159.01	0.00499481079699134\\
160.01	0.00499483988524153\\
161.01	0.00499486953211163\\
162.01	0.0049948997480561\\
163.01	0.00499493054371672\\
164.01	0.00499496192992582\\
165.01	0.00499499391771016\\
166.01	0.00499502651829281\\
167.01	0.00499505974309748\\
168.01	0.00499509360375137\\
169.01	0.00499512811208846\\
170.01	0.00499516328015329\\
171.01	0.00499519912020387\\
172.01	0.00499523564471568\\
173.01	0.00499527286638476\\
174.01	0.00499531079813151\\
175.01	0.00499534945310437\\
176.01	0.00499538884468316\\
177.01	0.00499542898648321\\
178.01	0.00499546989235899\\
179.01	0.00499551157640772\\
180.01	0.00499555405297349\\
181.01	0.00499559733665106\\
182.01	0.00499564144228974\\
183.01	0.00499568638499785\\
184.01	0.00499573218014604\\
185.01	0.00499577884337205\\
186.01	0.00499582639058445\\
187.01	0.00499587483796736\\
188.01	0.00499592420198405\\
189.01	0.00499597449938159\\
190.01	0.00499602574719532\\
191.01	0.00499607796275328\\
192.01	0.00499613116368033\\
193.01	0.00499618536790311\\
194.01	0.00499624059365412\\
195.01	0.00499629685947682\\
196.01	0.00499635418423008\\
197.01	0.00499641258709284\\
198.01	0.00499647208756888\\
199.01	0.00499653270549189\\
200.01	0.00499659446103007\\
201.01	0.00499665737469117\\
202.01	0.00499672146732727\\
203.01	0.00499678676014038\\
204.01	0.00499685327468653\\
205.01	0.00499692103288189\\
206.01	0.0049969900570075\\
207.01	0.00499706036971419\\
208.01	0.00499713199402853\\
209.01	0.00499720495335762\\
210.01	0.00499727927149465\\
211.01	0.0049973549726242\\
212.01	0.00499743208132786\\
213.01	0.0049975106225897\\
214.01	0.00499759062180158\\
215.01	0.00499767210476922\\
216.01	0.00499775509771747\\
217.01	0.00499783962729614\\
218.01	0.00499792572058577\\
219.01	0.00499801340510302\\
220.01	0.00499810270880725\\
221.01	0.00499819366010555\\
222.01	0.00499828628785939\\
223.01	0.00499838062139017\\
224.01	0.004998476690485\\
225.01	0.00499857452540325\\
226.01	0.00499867415688249\\
227.01	0.00499877561614422\\
228.01	0.00499887893490024\\
229.01	0.00499898414535897\\
230.01	0.0049990912802317\\
231.01	0.00499920037273871\\
232.01	0.00499931145661536\\
233.01	0.00499942456611892\\
234.01	0.00499953973603448\\
235.01	0.00499965700168177\\
236.01	0.00499977639892117\\
237.01	0.00499989796416045\\
238.01	0.00500002173436129\\
239.01	0.00500014774704567\\
240.01	0.00500027604030258\\
241.01	0.00500040665279465\\
242.01	0.00500053962376464\\
243.01	0.00500067499304205\\
244.01	0.0050008128010505\\
245.01	0.00500095308881363\\
246.01	0.00500109589796245\\
247.01	0.00500124127074171\\
248.01	0.00500138925001742\\
249.01	0.0050015398792831\\
250.01	0.00500169320266722\\
251.01	0.00500184926494003\\
252.01	0.00500200811152057\\
253.01	0.00500216978848361\\
254.01	0.00500233434256725\\
255.01	0.00500250182118003\\
256.01	0.00500267227240769\\
257.01	0.00500284574502116\\
258.01	0.00500302228848373\\
259.01	0.00500320195295877\\
260.01	0.00500338478931665\\
261.01	0.00500357084914305\\
262.01	0.0050037601847466\\
263.01	0.00500395284916681\\
264.01	0.00500414889618174\\
265.01	0.00500434838031662\\
266.01	0.00500455135685159\\
267.01	0.00500475788183073\\
268.01	0.00500496801207019\\
269.01	0.00500518180516701\\
270.01	0.0050053993195083\\
271.01	0.0050056206142802\\
272.01	0.00500584574947749\\
273.01	0.00500607478591301\\
274.01	0.00500630778522779\\
275.01	0.00500654480990131\\
276.01	0.0050067859232618\\
277.01	0.00500703118949763\\
278.01	0.00500728067366785\\
279.01	0.00500753444171483\\
280.01	0.0050077925604757\\
281.01	0.0050080550976951\\
282.01	0.00500832212203871\\
283.01	0.00500859370310652\\
284.01	0.00500886991144747\\
285.01	0.00500915081857438\\
286.01	0.00500943649697945\\
287.01	0.00500972702015111\\
288.01	0.00501002246259069\\
289.01	0.00501032289983132\\
290.01	0.00501062840845641\\
291.01	0.00501093906612013\\
292.01	0.0050112549515685\\
293.01	0.00501157614466213\\
294.01	0.00501190272639887\\
295.01	0.00501223477894006\\
296.01	0.00501257238563586\\
297.01	0.00501291563105405\\
298.01	0.00501326460100887\\
299.01	0.00501361938259264\\
300.01	0.00501398006420875\\
301.01	0.00501434673560681\\
302.01	0.00501471948791993\\
303.01	0.00501509841370372\\
304.01	0.00501548360697839\\
305.01	0.00501587516327254\\
306.01	0.00501627317967014\\
307.01	0.00501667775486028\\
308.01	0.00501708898918908\\
309.01	0.00501750698471545\\
310.01	0.00501793184526991\\
311.01	0.00501836367651623\\
312.01	0.00501880258601761\\
313.01	0.00501924868330537\\
314.01	0.00501970207995231\\
315.01	0.00502016288964997\\
316.01	0.00502063122828927\\
317.01	0.00502110721404658\\
318.01	0.00502159096747391\\
319.01	0.00502208261159244\\
320.01	0.00502258227199301\\
321.01	0.00502309007693881\\
322.01	0.00502360615747541\\
323.01	0.00502413064754381\\
324.01	0.0050246636840998\\
325.01	0.00502520540723725\\
326.01	0.00502575596031725\\
327.01	0.00502631549010182\\
328.01	0.00502688414689204\\
329.01	0.00502746208467171\\
330.01	0.00502804946125478\\
331.01	0.00502864643843685\\
332.01	0.00502925318215132\\
333.01	0.00502986986262773\\
334.01	0.00503049665455403\\
335.01	0.00503113373724073\\
336.01	0.00503178129478642\\
337.01	0.00503243951624402\\
338.01	0.00503310859578748\\
339.01	0.0050337887328762\\
340.01	0.00503448013241763\\
341.01	0.00503518300492596\\
342.01	0.00503589756667509\\
343.01	0.00503662403984562\\
344.01	0.0050373626526614\\
345.01	0.00503811363951756\\
346.01	0.00503887724109417\\
347.01	0.0050396537044564\\
348.01	0.00504044328313653\\
349.01	0.00504124623719773\\
350.01	0.00504206283327525\\
351.01	0.00504289334459476\\
352.01	0.00504373805096422\\
353.01	0.00504459723873756\\
354.01	0.00504547120074937\\
355.01	0.00504636023621806\\
356.01	0.00504726465061738\\
357.01	0.00504818475551667\\
358.01	0.0050491208683893\\
359.01	0.00505007331239125\\
360.01	0.00505104241611436\\
361.01	0.00505202851331574\\
362.01	0.00505303194263147\\
363.01	0.00505405304728081\\
364.01	0.00505509217477058\\
365.01	0.00505614967661133\\
366.01	0.00505722590805743\\
367.01	0.00505832122788587\\
368.01	0.00505943599823176\\
369.01	0.00506057058449409\\
370.01	0.00506172535533017\\
371.01	0.0050629006827536\\
372.01	0.00506409694234744\\
373.01	0.00506531451359821\\
374.01	0.0050665537803536\\
375.01	0.00506781513138806\\
376.01	0.00506909896105726\\
377.01	0.00507040567000097\\
378.01	0.0050717356658435\\
379.01	0.00507308936382878\\
380.01	0.00507446718732461\\
381.01	0.0050758695681664\\
382.01	0.00507729694690821\\
383.01	0.00507874977306559\\
384.01	0.00508022850536756\\
385.01	0.00508173361201872\\
386.01	0.00508326557097012\\
387.01	0.00508482487020107\\
388.01	0.0050864120080109\\
389.01	0.0050880274933212\\
390.01	0.0050896718459903\\
391.01	0.00509134559713759\\
392.01	0.00509304928948121\\
393.01	0.00509478347768592\\
394.01	0.00509654872872523\\
395.01	0.00509834562225397\\
396.01	0.00510017475099574\\
397.01	0.00510203672114171\\
398.01	0.00510393215276361\\
399.01	0.00510586168023992\\
400.01	0.00510782595269503\\
401.01	0.00510982563445289\\
402.01	0.00511186140550353\\
403.01	0.00511393396198424\\
404.01	0.00511604401667314\\
405.01	0.00511819229949688\\
406.01	0.00512037955805184\\
407.01	0.00512260655813837\\
408.01	0.00512487408430706\\
409.01	0.00512718294041817\\
410.01	0.00512953395021211\\
411.01	0.00513192795789152\\
412.01	0.00513436582871407\\
413.01	0.0051368484495937\\
414.01	0.00513937672971189\\
415.01	0.00514195160113549\\
416.01	0.00514457401944182\\
417.01	0.00514724496434761\\
418.01	0.00514996544034281\\
419.01	0.00515273647732563\\
420.01	0.00515555913123816\\
421.01	0.00515843448470016\\
422.01	0.00516136364764033\\
423.01	0.00516434775792068\\
424.01	0.00516738798195487\\
425.01	0.00517048551531558\\
426.01	0.00517364158333061\\
427.01	0.00517685744166467\\
428.01	0.00518013437688505\\
429.01	0.00518347370700885\\
430.01	0.00518687678202942\\
431.01	0.00519034498442225\\
432.01	0.0051938797296254\\
433.01	0.00519748246649712\\
434.01	0.00520115467774707\\
435.01	0.00520489788034244\\
436.01	0.00520871362588934\\
437.01	0.00521260350098952\\
438.01	0.00521656912757588\\
439.01	0.00522061216322811\\
440.01	0.00522473430147351\\
441.01	0.0052289372720771\\
442.01	0.00523322284132761\\
443.01	0.00523759281232565\\
444.01	0.00524204902528289\\
445.01	0.00524659335784246\\
446.01	0.00525122772542844\\
447.01	0.00525595408163848\\
448.01	0.00526077441868902\\
449.01	0.00526569076792553\\
450.01	0.00527070520040915\\
451.01	0.00527581982759093\\
452.01	0.0052810368020802\\
453.01	0.00528635831851634\\
454.01	0.00529178661454525\\
455.01	0.0052973239718996\\
456.01	0.00530297271757832\\
457.01	0.00530873522511274\\
458.01	0.00531461391590231\\
459.01	0.00532061126059696\\
460.01	0.00532672978049861\\
461.01	0.00533297204894941\\
462.01	0.00533934069267547\\
463.01	0.00534583839305854\\
464.01	0.00535246788731461\\
465.01	0.00535923196957572\\
466.01	0.00536613349188794\\
467.01	0.00537317536516092\\
468.01	0.00538036056011246\\
469.01	0.00538769210822991\\
470.01	0.00539517310275549\\
471.01	0.00540280669969907\\
472.01	0.00541059611888273\\
473.01	0.00541854464502098\\
474.01	0.00542665562884202\\
475.01	0.00543493248825357\\
476.01	0.00544337870955725\\
477.01	0.00545199784871682\\
478.01	0.00546079353268274\\
479.01	0.00546976946077612\\
480.01	0.0054789294061358\\
481.01	0.00548827721722787\\
482.01	0.00549781681941935\\
483.01	0.00550755221661568\\
484.01	0.00551748749295894\\
485.01	0.00552762681458464\\
486.01	0.00553797443143256\\
487.01	0.00554853467910639\\
488.01	0.00555931198077608\\
489.01	0.0055703108491171\\
490.01	0.00558153588827973\\
491.01	0.00559299179588486\\
492.01	0.00560468336504057\\
493.01	0.00561661548637882\\
494.01	0.00562879315011237\\
495.01	0.00564122144811656\\
496.01	0.0056539055760409\\
497.01	0.0056668508354585\\
498.01	0.00568006263606116\\
499.01	0.0056935464979061\\
500.01	0.00570730805371605\\
501.01	0.00572135305123445\\
502.01	0.00573568735563307\\
503.01	0.00575031695197263\\
504.01	0.00576524794771362\\
505.01	0.00578048657527726\\
506.01	0.0057960391946542\\
507.01	0.00581191229605887\\
508.01	0.00582811250262841\\
509.01	0.00584464657316448\\
510.01	0.00586152140491583\\
511.01	0.00587874403640189\\
512.01	0.0058963216502738\\
513.01	0.00591426157621598\\
514.01	0.00593257129388382\\
515.01	0.00595125843587969\\
516.01	0.00597033079076532\\
517.01	0.00598979630611013\\
518.01	0.00600966309157334\\
519.01	0.00602993942201802\\
520.01	0.00605063374065308\\
521.01	0.00607175466219925\\
522.01	0.00609331097607468\\
523.01	0.00611531164959501\\
524.01	0.00613776583118063\\
525.01	0.00616068285356548\\
526.01	0.00618407223699981\\
527.01	0.00620794369243782\\
528.01	0.00623230712470173\\
529.01	0.00625717263560928\\
530.01	0.00628255052705605\\
531.01	0.00630845130403507\\
532.01	0.00633488567758089\\
533.01	0.00636186456761978\\
534.01	0.00638939910570525\\
535.01	0.0064175006376183\\
536.01	0.0064461807258063\\
537.01	0.00647545115163257\\
538.01	0.0065053239174051\\
539.01	0.00653581124814943\\
540.01	0.00656692559308616\\
541.01	0.00659867962676944\\
542.01	0.00663108624983591\\
543.01	0.00666415858931164\\
544.01	0.00669790999841302\\
545.01	0.00673235405577483\\
546.01	0.00676750456402943\\
547.01	0.00680337554765068\\
548.01	0.00683998124996985\\
549.01	0.00687733612925698\\
550.01	0.00691545485375123\\
551.01	0.0069543522955103\\
552.01	0.00699404352293434\\
553.01	0.00703454379180462\\
554.01	0.00707586853465878\\
555.01	0.00711803334830747\\
556.01	0.0071610539792728\\
557.01	0.00720494630691036\\
558.01	0.00724972632394758\\
559.01	0.00729541011414746\\
560.01	0.00734201382677425\\
561.01	0.00738955364750814\\
562.01	0.0074380457654204\\
563.01	0.00748750633558497\\
564.01	0.00753795143686202\\
565.01	0.00758939702435083\\
566.01	0.00764185887596373\\
567.01	0.00769535253253253\\
568.01	0.00774989323081213\\
569.01	0.00780549582870473\\
570.01	0.00786217472198644\\
571.01	0.007919943751782\\
572.01	0.00797881610200229\\
573.01	0.00803880418594278\\
574.01	0.0080999195212333\\
575.01	0.00816217259234991\\
576.01	0.00822557269994251\\
577.01	0.00829012779631636\\
578.01	0.00835584430654082\\
579.01	0.00842272693485917\\
580.01	0.00849077845636217\\
581.01	0.00855999949428942\\
582.01	0.00863038828386762\\
583.01	0.00870194042432736\\
584.01	0.0087746486217084\\
585.01	0.00884850242633413\\
586.01	0.00892348797048993\\
587.01	0.00899958771397793\\
588.01	0.00907678020797329\\
589.01	0.00915503989113028\\
590.01	0.0092343369363852\\
591.01	0.00931463717262456\\
592.01	0.00939590211264666\\
593.01	0.00947808912803386\\
594.01	0.00956115182316124\\
595.01	0.00964504067520493\\
596.01	0.00972970402544469\\
597.01	0.0098150895303299\\
598.01	0.00990114621169971\\
599.01	0.00996919203046377\\
599.02	0.00996973314335038\\
599.03	0.0099702709571694\\
599.04	0.00997080543920336\\
599.05	0.0099713365564138\\
599.06	0.00997186427543808\\
599.07	0.0099723885625862\\
599.08	0.00997290938383756\\
599.09	0.00997342670483771\\
599.1	0.00997394049089505\\
599.11	0.00997445070697751\\
599.12	0.00997495731770918\\
599.13	0.00997546028736696\\
599.14	0.00997595957987707\\
599.15	0.00997645515881166\\
599.16	0.00997694698738526\\
599.17	0.00997743502845131\\
599.18	0.00997791924449856\\
599.19	0.00997839959764748\\
599.2	0.00997887604964662\\
599.21	0.00997934856186899\\
599.22	0.00997981709530829\\
599.23	0.00998028161057521\\
599.24	0.00998074206789365\\
599.25	0.0099811984270969\\
599.26	0.00998165064762378\\
599.27	0.00998209868851477\\
599.28	0.00998254250840806\\
599.29	0.00998298206553559\\
599.3	0.00998341731771905\\
599.31	0.00998384822236584\\
599.32	0.00998427473646497\\
599.33	0.00998469681658292\\
599.34	0.00998511441885952\\
599.35	0.0099855274990037\\
599.36	0.00998593601228926\\
599.37	0.00998633991355056\\
599.38	0.00998673915717821\\
599.39	0.00998713369711469\\
599.4	0.00998752348684992\\
599.41	0.00998790847811156\\
599.42	0.00998828861974491\\
599.43	0.00998866386008804\\
599.44	0.00998903414696681\\
599.45	0.00998939942768985\\
599.46	0.00998975964904344\\
599.47	0.00999011475728636\\
599.48	0.00999046469814472\\
599.49	0.00999080941680669\\
599.5	0.00999114885791725\\
599.51	0.00999148296557278\\
599.52	0.0099918116833157\\
599.53	0.00999213495412901\\
599.54	0.00999245272043077\\
599.55	0.00999276492406855\\
599.56	0.00999307150631382\\
599.57	0.00999337240785624\\
599.58	0.00999366756879799\\
599.59	0.00999395692864795\\
599.6	0.00999424042631585\\
599.61	0.00999451800010642\\
599.62	0.00999478958771336\\
599.63	0.0099950551262134\\
599.64	0.00999531455206019\\
599.65	0.00999556780107816\\
599.66	0.00999581480845634\\
599.67	0.00999605550874211\\
599.68	0.00999628983583487\\
599.69	0.00999651772297967\\
599.7	0.00999673910276076\\
599.71	0.00999695390709509\\
599.72	0.00999716206722577\\
599.73	0.00999736351371538\\
599.74	0.00999755817643933\\
599.75	0.00999774598457904\\
599.76	0.00999792686661517\\
599.77	0.00999810075032068\\
599.78	0.00999826756275386\\
599.79	0.00999842723025133\\
599.8	0.00999857967842092\\
599.81	0.0099987248321345\\
599.82	0.00999886261552071\\
599.83	0.00999899295195769\\
599.84	0.00999911576406567\\
599.85	0.00999923097369951\\
599.86	0.00999933850194117\\
599.87	0.00999943826909211\\
599.88	0.00999953019466558\\
599.89	0.00999961419737891\\
599.9	0.00999969019514566\\
599.91	0.00999975810506767\\
599.92	0.00999981784342713\\
599.93	0.00999986932567848\\
599.94	0.00999991246644028\\
599.95	0.00999994717948697\\
599.96	0.00999997337774056\\
599.97	0.00999999097326228\\
599.98	0.00999999987724406\\
599.99	0.01\\
600	0.01\\
};
\addplot [color=mycolor7,solid,forget plot]
  table[row sep=crcr]{%
0.01	0.00495449873497758\\
1.01	0.00495450029197152\\
2.01	0.00495450188039694\\
3.01	0.00495450350088551\\
4.01	0.00495450515408258\\
5.01	0.00495450684064533\\
6.01	0.0049545085612448\\
7.01	0.00495451031656516\\
8.01	0.0049545121073042\\
9.01	0.00495451393417384\\
10.01	0.00495451579790009\\
11.01	0.00495451769922321\\
12.01	0.00495451963889873\\
13.01	0.00495452161769698\\
14.01	0.00495452363640381\\
15.01	0.00495452569582075\\
16.01	0.00495452779676507\\
17.01	0.00495452994007062\\
18.01	0.00495453212658796\\
19.01	0.00495453435718417\\
20.01	0.00495453663274401\\
21.01	0.00495453895416986\\
22.01	0.00495454132238195\\
23.01	0.00495454373831894\\
24.01	0.0049545462029381\\
25.01	0.00495454871721564\\
26.01	0.00495455128214774\\
27.01	0.00495455389874973\\
28.01	0.00495455656805779\\
29.01	0.00495455929112838\\
30.01	0.00495456206903892\\
31.01	0.00495456490288834\\
32.01	0.00495456779379762\\
33.01	0.00495457074290986\\
34.01	0.0049545737513909\\
35.01	0.00495457682042971\\
36.01	0.00495457995123902\\
37.01	0.00495458314505552\\
38.01	0.00495458640314056\\
39.01	0.0049545897267807\\
40.01	0.00495459311728746\\
41.01	0.00495459657599899\\
42.01	0.00495460010427994\\
43.01	0.0049546037035217\\
44.01	0.00495460737514361\\
45.01	0.00495461112059283\\
46.01	0.0049546149413455\\
47.01	0.00495461883890681\\
48.01	0.0049546228148121\\
49.01	0.0049546268706267\\
50.01	0.00495463100794719\\
51.01	0.00495463522840174\\
52.01	0.00495463953365069\\
53.01	0.00495464392538732\\
54.01	0.00495464840533831\\
55.01	0.00495465297526451\\
56.01	0.00495465763696179\\
57.01	0.00495466239226126\\
58.01	0.00495466724303043\\
59.01	0.00495467219117345\\
60.01	0.00495467723863223\\
61.01	0.00495468238738704\\
62.01	0.00495468763945725\\
63.01	0.00495469299690213\\
64.01	0.00495469846182128\\
65.01	0.00495470403635614\\
66.01	0.00495470972268981\\
67.01	0.00495471552304907\\
68.01	0.00495472143970416\\
69.01	0.00495472747497004\\
70.01	0.00495473363120746\\
71.01	0.00495473991082342\\
72.01	0.00495474631627246\\
73.01	0.00495475285005728\\
74.01	0.00495475951472986\\
75.01	0.00495476631289217\\
76.01	0.00495477324719747\\
77.01	0.00495478032035129\\
78.01	0.00495478753511203\\
79.01	0.0049547948942924\\
80.01	0.00495480240076\\
81.01	0.00495481005743911\\
82.01	0.00495481786731085\\
83.01	0.00495482583341533\\
84.01	0.00495483395885171\\
85.01	0.00495484224678019\\
86.01	0.0049548507004227\\
87.01	0.00495485932306419\\
88.01	0.00495486811805396\\
89.01	0.0049548770888066\\
90.01	0.00495488623880375\\
91.01	0.00495489557159466\\
92.01	0.00495490509079834\\
93.01	0.00495491480010417\\
94.01	0.00495492470327344\\
95.01	0.00495493480414112\\
96.01	0.00495494510661651\\
97.01	0.00495495561468547\\
98.01	0.00495496633241142\\
99.01	0.00495497726393669\\
100.01	0.00495498841348437\\
101.01	0.00495499978535961\\
102.01	0.00495501138395121\\
103.01	0.00495502321373342\\
104.01	0.0049550352792672\\
105.01	0.00495504758520215\\
106.01	0.00495506013627819\\
107.01	0.00495507293732686\\
108.01	0.00495508599327349\\
109.01	0.00495509930913893\\
110.01	0.00495511289004101\\
111.01	0.00495512674119682\\
112.01	0.00495514086792431\\
113.01	0.00495515527564409\\
114.01	0.0049551699698817\\
115.01	0.00495518495626897\\
116.01	0.00495520024054676\\
117.01	0.00495521582856674\\
118.01	0.004955231726293\\
119.01	0.00495524793980481\\
120.01	0.00495526447529829\\
121.01	0.00495528133908892\\
122.01	0.00495529853761351\\
123.01	0.00495531607743251\\
124.01	0.00495533396523258\\
125.01	0.00495535220782854\\
126.01	0.00495537081216579\\
127.01	0.00495538978532296\\
128.01	0.00495540913451452\\
129.01	0.00495542886709247\\
130.01	0.00495544899054995\\
131.01	0.0049554695125229\\
132.01	0.00495549044079345\\
133.01	0.00495551178329211\\
134.01	0.00495553354810058\\
135.01	0.00495555574345463\\
136.01	0.00495557837774689\\
137.01	0.00495560145952961\\
138.01	0.00495562499751775\\
139.01	0.00495564900059182\\
140.01	0.00495567347780106\\
141.01	0.00495569843836585\\
142.01	0.00495572389168203\\
143.01	0.00495574984732285\\
144.01	0.0049557763150429\\
145.01	0.00495580330478111\\
146.01	0.0049558308266643\\
147.01	0.00495585889101022\\
148.01	0.00495588750833134\\
149.01	0.00495591668933791\\
150.01	0.00495594644494192\\
151.01	0.00495597678626046\\
152.01	0.00495600772461968\\
153.01	0.00495603927155806\\
154.01	0.00495607143883059\\
155.01	0.00495610423841204\\
156.01	0.0049561376825016\\
157.01	0.00495617178352639\\
158.01	0.00495620655414555\\
159.01	0.00495624200725424\\
160.01	0.00495627815598809\\
161.01	0.00495631501372726\\
162.01	0.00495635259410016\\
163.01	0.00495639091098891\\
164.01	0.00495642997853274\\
165.01	0.00495646981113275\\
166.01	0.00495651042345693\\
167.01	0.00495655183044394\\
168.01	0.00495659404730854\\
169.01	0.00495663708954593\\
170.01	0.00495668097293626\\
171.01	0.00495672571355051\\
172.01	0.00495677132775403\\
173.01	0.00495681783221298\\
174.01	0.00495686524389857\\
175.01	0.00495691358009234\\
176.01	0.00495696285839146\\
177.01	0.00495701309671413\\
178.01	0.00495706431330482\\
179.01	0.00495711652673982\\
180.01	0.0049571697559327\\
181.01	0.00495722402013991\\
182.01	0.00495727933896652\\
183.01	0.00495733573237175\\
184.01	0.00495739322067508\\
185.01	0.00495745182456218\\
186.01	0.00495751156509033\\
187.01	0.00495757246369501\\
188.01	0.00495763454219597\\
189.01	0.00495769782280309\\
190.01	0.00495776232812312\\
191.01	0.00495782808116539\\
192.01	0.00495789510534889\\
193.01	0.00495796342450822\\
194.01	0.00495803306290048\\
195.01	0.00495810404521171\\
196.01	0.00495817639656367\\
197.01	0.00495825014252073\\
198.01	0.00495832530909653\\
199.01	0.00495840192276086\\
200.01	0.00495848001044674\\
201.01	0.00495855959955767\\
202.01	0.00495864071797426\\
203.01	0.00495872339406181\\
204.01	0.00495880765667776\\
205.01	0.00495889353517825\\
206.01	0.00495898105942588\\
207.01	0.00495907025979767\\
208.01	0.00495916116719171\\
209.01	0.00495925381303512\\
210.01	0.00495934822929152\\
211.01	0.00495944444846912\\
212.01	0.00495954250362785\\
213.01	0.00495964242838731\\
214.01	0.00495974425693477\\
215.01	0.00495984802403256\\
216.01	0.00495995376502656\\
217.01	0.00496006151585361\\
218.01	0.00496017131304975\\
219.01	0.00496028319375823\\
220.01	0.00496039719573714\\
221.01	0.00496051335736789\\
222.01	0.00496063171766306\\
223.01	0.00496075231627453\\
224.01	0.00496087519350162\\
225.01	0.00496100039029876\\
226.01	0.00496112794828414\\
227.01	0.0049612579097475\\
228.01	0.00496139031765823\\
229.01	0.00496152521567356\\
230.01	0.00496166264814635\\
231.01	0.00496180266013309\\
232.01	0.00496194529740224\\
233.01	0.00496209060644175\\
234.01	0.00496223863446724\\
235.01	0.00496238942942935\\
236.01	0.00496254304002211\\
237.01	0.0049626995156903\\
238.01	0.00496285890663714\\
239.01	0.00496302126383162\\
240.01	0.00496318663901635\\
241.01	0.0049633550847142\\
242.01	0.00496352665423608\\
243.01	0.00496370140168802\\
244.01	0.00496387938197736\\
245.01	0.00496406065082051\\
246.01	0.00496424526474883\\
247.01	0.00496443328111536\\
248.01	0.00496462475810122\\
249.01	0.00496481975472149\\
250.01	0.00496501833083085\\
251.01	0.00496522054712963\\
252.01	0.00496542646516901\\
253.01	0.00496563614735633\\
254.01	0.00496584965695947\\
255.01	0.00496606705811198\\
256.01	0.0049662884158171\\
257.01	0.004966513795952\\
258.01	0.00496674326527102\\
259.01	0.00496697689140936\\
260.01	0.0049672147428861\\
261.01	0.00496745688910649\\
262.01	0.00496770340036452\\
263.01	0.00496795434784431\\
264.01	0.00496820980362171\\
265.01	0.00496846984066524\\
266.01	0.00496873453283674\\
267.01	0.00496900395489093\\
268.01	0.00496927818247525\\
269.01	0.00496955729212913\\
270.01	0.0049698413612816\\
271.01	0.00497013046825032\\
272.01	0.0049704246922382\\
273.01	0.0049707241133307\\
274.01	0.00497102881249172\\
275.01	0.00497133887155958\\
276.01	0.00497165437324192\\
277.01	0.00497197540111015\\
278.01	0.00497230203959358\\
279.01	0.00497263437397219\\
280.01	0.0049729724903692\\
281.01	0.00497331647574369\\
282.01	0.00497366641788095\\
283.01	0.00497402240538389\\
284.01	0.00497438452766298\\
285.01	0.00497475287492517\\
286.01	0.00497512753816343\\
287.01	0.00497550860914453\\
288.01	0.00497589618039711\\
289.01	0.00497629034519845\\
290.01	0.00497669119756142\\
291.01	0.00497709883222086\\
292.01	0.004977513344619\\
293.01	0.00497793483089126\\
294.01	0.00497836338785153\\
295.01	0.00497879911297684\\
296.01	0.0049792421043927\\
297.01	0.00497969246085737\\
298.01	0.00498015028174784\\
299.01	0.00498061566704373\\
300.01	0.00498108871731409\\
301.01	0.00498156953370295\\
302.01	0.00498205821791588\\
303.01	0.00498255487220821\\
304.01	0.00498305959937317\\
305.01	0.00498357250273236\\
306.01	0.00498409368612773\\
307.01	0.00498462325391459\\
308.01	0.00498516131095768\\
309.01	0.00498570796262973\\
310.01	0.00498626331481203\\
311.01	0.0049868274738999\\
312.01	0.00498740054681021\\
313.01	0.00498798264099452\\
314.01	0.0049885738644562\\
315.01	0.00498917432577321\\
316.01	0.00498978413412694\\
317.01	0.00499040339933726\\
318.01	0.00499103223190508\\
319.01	0.00499167074306354\\
320.01	0.00499231904483693\\
321.01	0.00499297725011064\\
322.01	0.00499364547271063\\
323.01	0.00499432382749534\\
324.01	0.00499501243046012\\
325.01	0.00499571139885524\\
326.01	0.00499642085131839\\
327.01	0.00499714090802421\\
328.01	0.00499787169085036\\
329.01	0.00499861332356202\\
330.01	0.00499936593201675\\
331.01	0.00500012964438984\\
332.01	0.00500090459142272\\
333.01	0.0050016909066939\\
334.01	0.00500248872691636\\
335.01	0.00500329819225946\\
336.01	0.00500411944669937\\
337.01	0.00500495263839654\\
338.01	0.00500579792010264\\
339.01	0.00500665544959673\\
340.01	0.00500752539015163\\
341.01	0.00500840791102919\\
342.01	0.0050093031880058\\
343.01	0.00501021140392467\\
344.01	0.00501113274927688\\
345.01	0.00501206742280441\\
346.01	0.0050130156321261\\
347.01	0.0050139775943794\\
348.01	0.00501495353687549\\
349.01	0.00501594369775796\\
350.01	0.00501694832666061\\
351.01	0.00501796768535226\\
352.01	0.00501900204835869\\
353.01	0.00502005170354786\\
354.01	0.00502111695266431\\
355.01	0.00502219811179455\\
356.01	0.00502329551174445\\
357.01	0.00502440949830703\\
358.01	0.00502554043239732\\
359.01	0.00502668869002935\\
360.01	0.00502785466210898\\
361.01	0.00502903875401514\\
362.01	0.00503024138494439\\
363.01	0.00503146298699349\\
364.01	0.0050327040039593\\
365.01	0.00503396488984265\\
366.01	0.00503524610704879\\
367.01	0.00503654812429304\\
368.01	0.00503787141423349\\
369.01	0.00503921645087724\\
370.01	0.00504058370683111\\
371.01	0.0050419736505011\\
372.01	0.00504338674337992\\
373.01	0.00504482343760372\\
374.01	0.00504628417399551\\
375.01	0.00504776938084786\\
376.01	0.00504927947371088\\
377.01	0.00505081485643616\\
378.01	0.00505237592364963\\
379.01	0.00505396306465568\\
380.01	0.00505557666840886\\
381.01	0.00505721712776508\\
382.01	0.00505888484080708\\
383.01	0.00506058021101743\\
384.01	0.00506230364741797\\
385.01	0.00506405556471838\\
386.01	0.00506583638347773\\
387.01	0.00506764653027661\\
388.01	0.00506948643790339\\
389.01	0.00507135654555459\\
390.01	0.00507325729904906\\
391.01	0.00507518915106002\\
392.01	0.00507715256136289\\
393.01	0.00507914799710259\\
394.01	0.00508117593307975\\
395.01	0.0050832368520585\\
396.01	0.00508533124509594\\
397.01	0.00508745961189566\\
398.01	0.00508962246118645\\
399.01	0.00509182031112708\\
400.01	0.00509405368973993\\
401.01	0.00509632313537381\\
402.01	0.00509862919719781\\
403.01	0.00510097243572839\\
404.01	0.00510335342339101\\
405.01	0.00510577274511721\\
406.01	0.00510823099897995\\
407.01	0.00511072879686754\\
408.01	0.0051132667651991\\
409.01	0.00511584554568112\\
410.01	0.00511846579610811\\
411.01	0.0051211281912082\\
412.01	0.0051238334235334\\
413.01	0.00512658220439763\\
414.01	0.00512937526486082\\
415.01	0.00513221335676202\\
416.01	0.00513509725379858\\
417.01	0.00513802775265401\\
418.01	0.00514100567417107\\
419.01	0.00514403186457146\\
420.01	0.00514710719671833\\
421.01	0.00515023257141975\\
422.01	0.00515340891877078\\
423.01	0.00515663719952925\\
424.01	0.00515991840652104\\
425.01	0.00516325356606923\\
426.01	0.00516664373944022\\
427.01	0.00517009002429979\\
428.01	0.00517359355616892\\
429.01	0.00517715550987126\\
430.01	0.00518077710095871\\
431.01	0.00518445958710489\\
432.01	0.00518820426945177\\
433.01	0.00519201249389373\\
434.01	0.00519588565228472\\
435.01	0.00519982518355067\\
436.01	0.00520383257468861\\
437.01	0.00520790936163687\\
438.01	0.00521205712999526\\
439.01	0.00521627751557836\\
440.01	0.00522057220478593\\
441.01	0.00522494293477301\\
442.01	0.00522939149340879\\
443.01	0.00523391971901313\\
444.01	0.00523852949986739\\
445.01	0.00524322277349935\\
446.01	0.00524800152575124\\
447.01	0.00525286778964769\\
448.01	0.0052578236440903\\
449.01	0.005262871212418\\
450.01	0.00526801266088528\\
451.01	0.00527325019712209\\
452.01	0.00527858606865597\\
453.01	0.00528402256158897\\
454.01	0.0052895619995329\\
455.01	0.00529520674291668\\
456.01	0.00530095918878035\\
457.01	0.00530682177116795\\
458.01	0.00531279696221322\\
459.01	0.00531888727398484\\
460.01	0.0053250952611094\\
461.01	0.00533142352412674\\
462.01	0.00533787471344524\\
463.01	0.00534445153366397\\
464.01	0.0053511567479194\\
465.01	0.00535799318182135\\
466.01	0.00536496372649997\\
467.01	0.00537207134042046\\
468.01	0.00537931905026122\\
469.01	0.00538670995152291\\
470.01	0.00539424720906119\\
471.01	0.00540193405755518\\
472.01	0.00540977380191732\\
473.01	0.00541776981765332\\
474.01	0.00542592555118307\\
475.01	0.00543424452013756\\
476.01	0.00544273031364788\\
477.01	0.00545138659264675\\
478.01	0.00546021709020519\\
479.01	0.00546922561192885\\
480.01	0.00547841603644079\\
481.01	0.00548779231597983\\
482.01	0.0054973584771407\\
483.01	0.00550711862178339\\
484.01	0.00551707692813627\\
485.01	0.00552723765211148\\
486.01	0.00553760512884571\\
487.01	0.005548183774471\\
488.01	0.0055589780881089\\
489.01	0.00556999265406935\\
490.01	0.00558123214422312\\
491.01	0.00559270132050174\\
492.01	0.00560440503747062\\
493.01	0.00561634824491018\\
494.01	0.00562853599034233\\
495.01	0.00564097342144393\\
496.01	0.00565366578831375\\
497.01	0.00566661844558945\\
498.01	0.00567983685445803\\
499.01	0.00569332658464414\\
500.01	0.00570709331646155\\
501.01	0.00572114284296089\\
502.01	0.00573548107217994\\
503.01	0.00575011402949795\\
504.01	0.00576504786009478\\
505.01	0.0057802888315108\\
506.01	0.00579584333630428\\
507.01	0.00581171789479883\\
508.01	0.00582791915791098\\
509.01	0.0058444539100473\\
510.01	0.00586132907205845\\
511.01	0.00587855170423747\\
512.01	0.00589612900934987\\
513.01	0.0059140683356845\\
514.01	0.00593237718011849\\
515.01	0.00595106319119111\\
516.01	0.00597013417218746\\
517.01	0.0059895980842368\\
518.01	0.00600946304943295\\
519.01	0.00602973735398422\\
520.01	0.00605042945139851\\
521.01	0.00607154796570211\\
522.01	0.00609310169468649\\
523.01	0.00611509961317347\\
524.01	0.00613755087629289\\
525.01	0.00616046482276039\\
526.01	0.00618385097814808\\
527.01	0.00620771905813523\\
528.01	0.00623207897172903\\
529.01	0.00625694082444449\\
530.01	0.00628231492143\\
531.01	0.00630821177052611\\
532.01	0.00633464208524313\\
533.01	0.00636161678764127\\
534.01	0.00638914701109523\\
535.01	0.00641724410292135\\
536.01	0.00644591962684281\\
537.01	0.00647518536526556\\
538.01	0.00650505332133199\\
539.01	0.00653553572071758\\
540.01	0.00656664501313042\\
541.01	0.00659839387346958\\
542.01	0.00663079520259363\\
543.01	0.00666386212764396\\
544.01	0.00669760800186221\\
545.01	0.00673204640383408\\
546.01	0.00676719113608257\\
547.01	0.00680305622292752\\
548.01	0.00683965590751582\\
549.01	0.00687700464791808\\
550.01	0.00691511711217481\\
551.01	0.00695400817216172\\
552.01	0.00699369289613081\\
553.01	0.00703418653976667\\
554.01	0.00707550453558119\\
555.01	0.00711766248044977\\
556.01	0.00716067612107247\\
557.01	0.00720456133711941\\
558.01	0.00724933412179551\\
559.01	0.0072950105595322\\
560.01	0.00734160680048446\\
561.01	0.00738913903147977\\
562.01	0.00743762344303022\\
563.01	0.00748707619198446\\
564.01	0.00753751335935612\\
565.01	0.00758895090282426\\
566.01	0.00764140460335994\\
567.01	0.00769489000538857\\
568.01	0.00774942234985408\\
569.01	0.00780501649950793\\
570.01	0.00786168685570438\\
571.01	0.00791944726594716\\
572.01	0.00797831092140276\\
573.01	0.00803829024357548\\
574.01	0.00809939675933612\\
575.01	0.0081616409635105\\
576.01	0.00822503216828076\\
577.01	0.00828957833873377\\
578.01	0.00835528591402531\\
579.01	0.00842215961382854\\
580.01	0.00849020223002198\\
581.01	0.00855941440397067\\
582.01	0.00862979439029825\\
583.01	0.00870133780877493\\
584.01	0.00877403738691154\\
585.01	0.00884788269711461\\
586.01	0.0089228598939045\\
587.01	0.00899895145882848\\
588.01	0.00907613596344161\\
589.01	0.00915438786423933\\
590.01	0.00923367734790629\\
591.01	0.00931397025094757\\
592.01	0.00939522808500194\\
593.01	0.00947740820829374\\
594.01	0.00956046419524734\\
595.01	0.00964434647087689\\
596.01	0.00972900329493185\\
597.01	0.0098143822038764\\
598.01	0.00990043204779049\\
599.01	0.00996919203046377\\
599.02	0.00996973314335038\\
599.03	0.0099702709571694\\
599.04	0.00997080543920336\\
599.05	0.0099713365564138\\
599.06	0.00997186427543808\\
599.07	0.0099723885625862\\
599.08	0.00997290938383756\\
599.09	0.00997342670483771\\
599.1	0.00997394049089505\\
599.11	0.00997445070697751\\
599.12	0.00997495731770918\\
599.13	0.00997546028736696\\
599.14	0.00997595957987707\\
599.15	0.00997645515881165\\
599.16	0.00997694698738526\\
599.17	0.00997743502845131\\
599.18	0.00997791924449856\\
599.19	0.00997839959764748\\
599.2	0.00997887604964662\\
599.21	0.00997934856186899\\
599.22	0.00997981709530829\\
599.23	0.00998028161057521\\
599.24	0.00998074206789365\\
599.25	0.0099811984270969\\
599.26	0.00998165064762378\\
599.27	0.00998209868851477\\
599.28	0.00998254250840806\\
599.29	0.00998298206553559\\
599.3	0.00998341731771905\\
599.31	0.00998384822236584\\
599.32	0.00998427473646497\\
599.33	0.00998469681658292\\
599.34	0.00998511441885952\\
599.35	0.0099855274990037\\
599.36	0.00998593601228926\\
599.37	0.00998633991355056\\
599.38	0.00998673915717821\\
599.39	0.0099871336971147\\
599.4	0.00998752348684992\\
599.41	0.00998790847811157\\
599.42	0.00998828861974491\\
599.43	0.00998866386008804\\
599.44	0.00998903414696681\\
599.45	0.00998939942768985\\
599.46	0.00998975964904344\\
599.47	0.00999011475728636\\
599.48	0.00999046469814472\\
599.49	0.00999080941680669\\
599.5	0.00999114885791725\\
599.51	0.00999148296557278\\
599.52	0.0099918116833157\\
599.53	0.00999213495412901\\
599.54	0.00999245272043077\\
599.55	0.00999276492406855\\
599.56	0.00999307150631381\\
599.57	0.00999337240785624\\
599.58	0.00999366756879799\\
599.59	0.00999395692864795\\
599.6	0.00999424042631586\\
599.61	0.00999451800010642\\
599.62	0.00999478958771336\\
599.63	0.0099950551262134\\
599.64	0.00999531455206019\\
599.65	0.00999556780107816\\
599.66	0.00999581480845634\\
599.67	0.00999605550874211\\
599.68	0.00999628983583487\\
599.69	0.00999651772297967\\
599.7	0.00999673910276076\\
599.71	0.00999695390709509\\
599.72	0.00999716206722577\\
599.73	0.00999736351371538\\
599.74	0.00999755817643933\\
599.75	0.00999774598457904\\
599.76	0.00999792686661517\\
599.77	0.00999810075032067\\
599.78	0.00999826756275386\\
599.79	0.00999842723025133\\
599.8	0.00999857967842092\\
599.81	0.0099987248321345\\
599.82	0.00999886261552071\\
599.83	0.00999899295195769\\
599.84	0.00999911576406567\\
599.85	0.00999923097369951\\
599.86	0.00999933850194117\\
599.87	0.00999943826909211\\
599.88	0.00999953019466558\\
599.89	0.00999961419737891\\
599.9	0.00999969019514566\\
599.91	0.00999975810506767\\
599.92	0.00999981784342713\\
599.93	0.00999986932567848\\
599.94	0.00999991246644028\\
599.95	0.00999994717948697\\
599.96	0.00999997337774056\\
599.97	0.00999999097326228\\
599.98	0.00999999987724406\\
599.99	0.01\\
600	0.01\\
};
\addplot [color=mycolor8,solid,forget plot]
  table[row sep=crcr]{%
0.01	0.00486947107861205\\
1.01	0.00486947292236736\\
2.01	0.00486947480371144\\
3.01	0.00486947672340973\\
4.01	0.00486947868224241\\
5.01	0.00486948068100652\\
6.01	0.00486948272051449\\
7.01	0.00486948480159557\\
8.01	0.00486948692509585\\
9.01	0.00486948909187865\\
10.01	0.00486949130282463\\
11.01	0.00486949355883243\\
12.01	0.00486949586081895\\
13.01	0.0048694982097197\\
14.01	0.00486950060648905\\
15.01	0.0048695030521007\\
16.01	0.00486950554754846\\
17.01	0.00486950809384586\\
18.01	0.00486951069202711\\
19.01	0.00486951334314763\\
20.01	0.0048695160482839\\
21.01	0.00486951880853456\\
22.01	0.00486952162502026\\
23.01	0.00486952449888436\\
24.01	0.00486952743129361\\
25.01	0.00486953042343835\\
26.01	0.00486953347653281\\
27.01	0.0048695365918162\\
28.01	0.00486953977055238\\
29.01	0.00486954301403122\\
30.01	0.0048695463235686\\
31.01	0.00486954970050693\\
32.01	0.00486955314621586\\
33.01	0.00486955666209285\\
34.01	0.00486956024956363\\
35.01	0.00486956391008279\\
36.01	0.00486956764513434\\
37.01	0.00486957145623229\\
38.01	0.00486957534492136\\
39.01	0.00486957931277746\\
40.01	0.00486958336140846\\
41.01	0.00486958749245478\\
42.01	0.00486959170758994\\
43.01	0.00486959600852142\\
44.01	0.00486960039699094\\
45.01	0.00486960487477578\\
46.01	0.00486960944368901\\
47.01	0.00486961410558036\\
48.01	0.00486961886233696\\
49.01	0.00486962371588401\\
50.01	0.00486962866818569\\
51.01	0.00486963372124574\\
52.01	0.00486963887710859\\
53.01	0.00486964413785981\\
54.01	0.00486964950562714\\
55.01	0.0048696549825813\\
56.01	0.00486966057093674\\
57.01	0.00486966627295266\\
58.01	0.00486967209093384\\
59.01	0.00486967802723172\\
60.01	0.004869684084245\\
61.01	0.0048696902644208\\
62.01	0.00486969657025571\\
63.01	0.00486970300429641\\
64.01	0.00486970956914139\\
65.01	0.00486971626744086\\
66.01	0.00486972310189916\\
67.01	0.00486973007527472\\
68.01	0.00486973719038161\\
69.01	0.00486974445009067\\
70.01	0.00486975185733038\\
71.01	0.00486975941508849\\
72.01	0.00486976712641271\\
73.01	0.00486977499441213\\
74.01	0.00486978302225829\\
75.01	0.00486979121318698\\
76.01	0.00486979957049866\\
77.01	0.00486980809756034\\
78.01	0.00486981679780672\\
79.01	0.00486982567474153\\
80.01	0.00486983473193921\\
81.01	0.00486984397304571\\
82.01	0.00486985340178046\\
83.01	0.0048698630219374\\
84.01	0.00486987283738692\\
85.01	0.00486988285207705\\
86.01	0.00486989307003499\\
87.01	0.00486990349536882\\
88.01	0.0048699141322692\\
89.01	0.00486992498501079\\
90.01	0.00486993605795404\\
91.01	0.00486994735554673\\
92.01	0.00486995888232605\\
93.01	0.00486997064291982\\
94.01	0.0048699826420489\\
95.01	0.00486999488452863\\
96.01	0.00487000737527089\\
97.01	0.00487002011928564\\
98.01	0.00487003312168349\\
99.01	0.00487004638767716\\
100.01	0.00487005992258352\\
101.01	0.00487007373182632\\
102.01	0.00487008782093738\\
103.01	0.00487010219555918\\
104.01	0.00487011686144692\\
105.01	0.00487013182447101\\
106.01	0.00487014709061895\\
107.01	0.00487016266599795\\
108.01	0.00487017855683707\\
109.01	0.00487019476948977\\
110.01	0.00487021131043615\\
111.01	0.0048702281862858\\
112.01	0.00487024540377978\\
113.01	0.00487026296979401\\
114.01	0.0048702808913408\\
115.01	0.00487029917557284\\
116.01	0.00487031782978466\\
117.01	0.00487033686141632\\
118.01	0.00487035627805595\\
119.01	0.00487037608744272\\
120.01	0.00487039629746952\\
121.01	0.00487041691618612\\
122.01	0.00487043795180245\\
123.01	0.0048704594126915\\
124.01	0.00487048130739251\\
125.01	0.0048705036446141\\
126.01	0.00487052643323813\\
127.01	0.00487054968232194\\
128.01	0.0048705734011032\\
129.01	0.00487059759900224\\
130.01	0.0048706222856261\\
131.01	0.00487064747077201\\
132.01	0.00487067316443125\\
133.01	0.00487069937679273\\
134.01	0.00487072611824707\\
135.01	0.00487075339939004\\
136.01	0.00487078123102694\\
137.01	0.00487080962417641\\
138.01	0.0048708385900747\\
139.01	0.00487086814017983\\
140.01	0.00487089828617572\\
141.01	0.00487092903997673\\
142.01	0.00487096041373185\\
143.01	0.00487099241982938\\
144.01	0.00487102507090169\\
145.01	0.00487105837982942\\
146.01	0.00487109235974664\\
147.01	0.00487112702404564\\
148.01	0.00487116238638149\\
149.01	0.00487119846067759\\
150.01	0.0048712352611306\\
151.01	0.00487127280221532\\
152.01	0.00487131109869036\\
153.01	0.00487135016560362\\
154.01	0.00487139001829714\\
155.01	0.00487143067241363\\
156.01	0.00487147214390136\\
157.01	0.00487151444902012\\
158.01	0.0048715576043474\\
159.01	0.00487160162678421\\
160.01	0.00487164653356111\\
161.01	0.0048716923422442\\
162.01	0.00487173907074211\\
163.01	0.00487178673731159\\
164.01	0.00487183536056479\\
165.01	0.00487188495947542\\
166.01	0.0048719355533856\\
167.01	0.00487198716201284\\
168.01	0.00487203980545693\\
169.01	0.00487209350420692\\
170.01	0.004872148279149\\
171.01	0.00487220415157283\\
172.01	0.00487226114318008\\
173.01	0.00487231927609102\\
174.01	0.00487237857285301\\
175.01	0.00487243905644807\\
176.01	0.00487250075030086\\
177.01	0.0048725636782868\\
178.01	0.00487262786474045\\
179.01	0.00487269333446377\\
180.01	0.00487276011273462\\
181.01	0.00487282822531551\\
182.01	0.00487289769846232\\
183.01	0.00487296855893347\\
184.01	0.00487304083399875\\
185.01	0.00487311455144837\\
186.01	0.00487318973960289\\
187.01	0.00487326642732225\\
188.01	0.00487334464401561\\
189.01	0.00487342441965137\\
190.01	0.00487350578476671\\
191.01	0.00487358877047799\\
192.01	0.00487367340849093\\
193.01	0.00487375973111112\\
194.01	0.00487384777125442\\
195.01	0.00487393756245795\\
196.01	0.0048740291388905\\
197.01	0.00487412253536421\\
198.01	0.00487421778734529\\
199.01	0.00487431493096541\\
200.01	0.00487441400303351\\
201.01	0.00487451504104718\\
202.01	0.00487461808320468\\
203.01	0.00487472316841675\\
204.01	0.00487483033631851\\
205.01	0.00487493962728282\\
206.01	0.00487505108243155\\
207.01	0.00487516474364866\\
208.01	0.00487528065359315\\
209.01	0.00487539885571175\\
210.01	0.00487551939425231\\
211.01	0.00487564231427645\\
212.01	0.00487576766167358\\
213.01	0.00487589548317438\\
214.01	0.00487602582636409\\
215.01	0.00487615873969677\\
216.01	0.004876294272509\\
217.01	0.00487643247503433\\
218.01	0.00487657339841745\\
219.01	0.00487671709472827\\
220.01	0.00487686361697712\\
221.01	0.00487701301912899\\
222.01	0.00487716535611812\\
223.01	0.00487732068386365\\
224.01	0.00487747905928396\\
225.01	0.00487764054031238\\
226.01	0.00487780518591223\\
227.01	0.00487797305609189\\
228.01	0.00487814421192095\\
229.01	0.004878318715545\\
230.01	0.00487849663020176\\
231.01	0.00487867802023668\\
232.01	0.00487886295111854\\
233.01	0.0048790514894554\\
234.01	0.00487924370301043\\
235.01	0.00487943966071792\\
236.01	0.00487963943269878\\
237.01	0.00487984309027684\\
238.01	0.00488005070599444\\
239.01	0.00488026235362868\\
240.01	0.00488047810820687\\
241.01	0.00488069804602276\\
242.01	0.00488092224465167\\
243.01	0.00488115078296653\\
244.01	0.00488138374115347\\
245.01	0.00488162120072699\\
246.01	0.00488186324454526\\
247.01	0.00488210995682524\\
248.01	0.00488236142315752\\
249.01	0.00488261773052098\\
250.01	0.00488287896729749\\
251.01	0.00488314522328534\\
252.01	0.00488341658971392\\
253.01	0.00488369315925632\\
254.01	0.0048839750260436\\
255.01	0.00488426228567626\\
256.01	0.00488455503523741\\
257.01	0.0048848533733041\\
258.01	0.00488515739995884\\
259.01	0.00488546721679987\\
260.01	0.00488578292695181\\
261.01	0.0048861046350747\\
262.01	0.00488643244737243\\
263.01	0.00488676647160112\\
264.01	0.00488710681707634\\
265.01	0.00488745359467868\\
266.01	0.00488780691685956\\
267.01	0.00488816689764514\\
268.01	0.00488853365263966\\
269.01	0.00488890729902719\\
270.01	0.00488928795557286\\
271.01	0.00488967574262189\\
272.01	0.00489007078209797\\
273.01	0.0048904731974995\\
274.01	0.00489088311389492\\
275.01	0.00489130065791585\\
276.01	0.00489172595774905\\
277.01	0.00489215914312559\\
278.01	0.004892600345309\\
279.01	0.00489304969708074\\
280.01	0.0048935073327239\\
281.01	0.00489397338800394\\
282.01	0.00489444800014771\\
283.01	0.00489493130781972\\
284.01	0.0048954234510946\\
285.01	0.00489592457142875\\
286.01	0.00489643481162716\\
287.01	0.00489695431580782\\
288.01	0.00489748322936308\\
289.01	0.00489802169891686\\
290.01	0.00489856987227907\\
291.01	0.004899127898395\\
292.01	0.00489969592729207\\
293.01	0.00490027411002165\\
294.01	0.00490086259859672\\
295.01	0.00490146154592501\\
296.01	0.0049020711057375\\
297.01	0.00490269143251219\\
298.01	0.00490332268139221\\
299.01	0.00490396500809982\\
300.01	0.00490461856884361\\
301.01	0.00490528352022102\\
302.01	0.00490596001911514\\
303.01	0.00490664822258498\\
304.01	0.00490734828775043\\
305.01	0.00490806037167041\\
306.01	0.00490878463121465\\
307.01	0.00490952122292969\\
308.01	0.0049102703028981\\
309.01	0.00491103202659038\\
310.01	0.00491180654871179\\
311.01	0.00491259402304131\\
312.01	0.00491339460226485\\
313.01	0.00491420843780183\\
314.01	0.0049150356796258\\
315.01	0.00491587647607902\\
316.01	0.00491673097368153\\
317.01	0.00491759931693518\\
318.01	0.00491848164812264\\
319.01	0.00491937810710343\\
320.01	0.00492028883110579\\
321.01	0.00492121395451707\\
322.01	0.00492215360867281\\
323.01	0.00492310792164599\\
324.01	0.0049240770180385\\
325.01	0.00492506101877592\\
326.01	0.00492606004090838\\
327.01	0.00492707419741913\\
328.01	0.0049281035970444\\
329.01	0.00492914834410706\\
330.01	0.00493020853836761\\
331.01	0.00493128427489808\\
332.01	0.00493237564397986\\
333.01	0.00493348273103555\\
334.01	0.00493460561659563\\
335.01	0.00493574437631042\\
336.01	0.00493689908101126\\
337.01	0.00493806979683226\\
338.01	0.00493925658539787\\
339.01	0.00494045950408954\\
340.01	0.00494167860639888\\
341.01	0.00494291394238186\\
342.01	0.00494416555922468\\
343.01	0.00494543350193688\\
344.01	0.00494671781418379\\
345.01	0.00494801853927719\\
346.01	0.00494933572133761\\
347.01	0.00495066940664869\\
348.01	0.0049520196452166\\
349.01	0.0049533864925576\\
350.01	0.00495477001172624\\
351.01	0.00495617027560363\\
352.01	0.00495758736945942\\
353.01	0.0049590213937989\\
354.01	0.00496047246750404\\
355.01	0.00496194073127067\\
356.01	0.00496342635133638\\
357.01	0.00496492952348586\\
358.01	0.00496645047730471\\
359.01	0.00496798948064016\\
360.01	0.00496954684420485\\
361.01	0.00497112292623784\\
362.01	0.00497271813710551\\
363.01	0.00497433294369125\\
364.01	0.00497596787338372\\
365.01	0.00497762351742256\\
366.01	0.00497930053331316\\
367.01	0.00498099964596406\\
368.01	0.00498272164714229\\
369.01	0.00498446739279117\\
370.01	0.00498623779770917\\
371.01	0.0049880338270704\\
372.01	0.00498985648428585\\
373.01	0.00499170679479151\\
374.01	0.00499358578553845\\
375.01	0.0049954944603099\\
376.01	0.00499743377157359\\
377.01	0.00499940459050881\\
378.01	0.00500140767827624\\
379.01	0.00500344366373888\\
380.01	0.00500551304083137\\
381.01	0.00500761623451497\\
382.01	0.00500975366440684\\
383.01	0.00501192575025035\\
384.01	0.00501413291170537\\
385.01	0.00501637556813499\\
386.01	0.00501865413838863\\
387.01	0.00502096904058337\\
388.01	0.00502332069188344\\
389.01	0.0050257095082792\\
390.01	0.00502813590436763\\
391.01	0.00503060029313336\\
392.01	0.00503310308573499\\
393.01	0.00503564469129482\\
394.01	0.00503822551669646\\
395.01	0.00504084596639151\\
396.01	0.00504350644221681\\
397.01	0.00504620734322635\\
398.01	0.00504894906553885\\
399.01	0.0050517320022056\\
400.01	0.00505455654310108\\
401.01	0.00505742307484021\\
402.01	0.00506033198072661\\
403.01	0.00506328364073562\\
404.01	0.00506627843153779\\
405.01	0.00506931672656746\\
406.01	0.00507239889614258\\
407.01	0.00507552530764104\\
408.01	0.00507869632574179\\
409.01	0.00508191231273589\\
410.01	0.00508517362891761\\
411.01	0.00508848063306101\\
412.01	0.00509183368299319\\
413.01	0.00509523313627251\\
414.01	0.00509867935098175\\
415.01	0.0051021726866462\\
416.01	0.00510571350528854\\
417.01	0.00510930217263104\\
418.01	0.00511293905945774\\
419.01	0.00511662454314692\\
420.01	0.00512035900938866\\
421.01	0.00512414285409804\\
422.01	0.00512797648553649\\
423.01	0.00513186032665466\\
424.01	0.00513579481766586\\
425.01	0.00513978041886369\\
426.01	0.00514381761369105\\
427.01	0.00514790691206936\\
428.01	0.00515204885399412\\
429.01	0.0051562440133987\\
430.01	0.00516049300228848\\
431.01	0.0051647964751378\\
432.01	0.0051691551335438\\
433.01	0.0051735697311199\\
434.01	0.00517804107860816\\
435.01	0.00518257004917931\\
436.01	0.00518715758388222\\
437.01	0.00519180469719159\\
438.01	0.00519651248259237\\
439.01	0.005201282118126\\
440.01	0.00520611487180661\\
441.01	0.00521101210680318\\
442.01	0.00521597528626405\\
443.01	0.00522100597764519\\
444.01	0.00522610585638446\\
445.01	0.0052312767087502\\
446.01	0.00523652043367955\\
447.01	0.00524183904340754\\
448.01	0.0052472346626893\\
449.01	0.00525270952641858\\
450.01	0.00525826597545891\\
451.01	0.00526390645053876\\
452.01	0.00526963348410613\\
453.01	0.00527544969011183\\
454.01	0.00528135775179404\\
455.01	0.00528736040766942\\
456.01	0.00529346043611076\\
457.01	0.00529966063910327\\
458.01	0.00530596382602569\\
459.01	0.00531237279858311\\
460.01	0.00531889033831225\\
461.01	0.00532551919835066\\
462.01	0.00533226210133951\\
463.01	0.00533912174532289\\
464.01	0.00534610081914363\\
465.01	0.00535320202787512\\
466.01	0.00536042812681975\\
467.01	0.00536778195505186\\
468.01	0.00537526645014033\\
469.01	0.0053828846510761\\
470.01	0.00539063970006425\\
471.01	0.00539853484395138\\
472.01	0.00540657343524228\\
473.01	0.00541475893266269\\
474.01	0.0054230949012279\\
475.01	0.00543158501178372\\
476.01	0.00544023303999507\\
477.01	0.00544904286476906\\
478.01	0.00545801846611518\\
479.01	0.00546716392246428\\
480.01	0.00547648340748907\\
481.01	0.00548598118649742\\
482.01	0.00549566161249708\\
483.01	0.00550552912206398\\
484.01	0.00551558823117855\\
485.01	0.00552584353122865\\
486.01	0.00553629968540818\\
487.01	0.00554696142576086\\
488.01	0.00555783355113311\\
489.01	0.00556892092629246\\
490.01	0.0055802284824364\\
491.01	0.00559176121925617\\
492.01	0.00560352420861528\\
493.01	0.0056155225997567\\
494.01	0.00562776162575973\\
495.01	0.00564024661073782\\
496.01	0.00565298297702609\\
497.01	0.00566597625140927\\
498.01	0.00567923206939053\\
499.01	0.00569275617702374\\
500.01	0.00570655443140132\\
501.01	0.00572063280103759\\
502.01	0.00573499736640269\\
503.01	0.00574965432066073\\
504.01	0.00576460997066185\\
505.01	0.00577987073823816\\
506.01	0.00579544316184305\\
507.01	0.00581133389856794\\
508.01	0.00582754972655403\\
509.01	0.00584409754780098\\
510.01	0.00586098439135369\\
511.01	0.00587821741682283\\
512.01	0.00589580391817397\\
513.01	0.00591375132769374\\
514.01	0.0059320672200273\\
515.01	0.00595075931617198\\
516.01	0.00596983548732184\\
517.01	0.00598930375848478\\
518.01	0.00600917231184658\\
519.01	0.00602944948993105\\
520.01	0.00605014379868112\\
521.01	0.00607126391059673\\
522.01	0.00609281866798034\\
523.01	0.00611481708628279\\
524.01	0.00613726835753446\\
525.01	0.00616018185383922\\
526.01	0.00618356713090683\\
527.01	0.00620743393159559\\
528.01	0.00623179218943586\\
529.01	0.00625665203210274\\
530.01	0.00628202378480992\\
531.01	0.0063079179735996\\
532.01	0.00633434532850766\\
533.01	0.00636131678658915\\
534.01	0.00638884349479479\\
535.01	0.00641693681269148\\
536.01	0.00644560831501638\\
537.01	0.00647486979404623\\
538.01	0.0065047332617518\\
539.01	0.00653521095169769\\
540.01	0.00656631532064435\\
541.01	0.00659805904980486\\
542.01	0.00663045504570371\\
543.01	0.00666351644058184\\
544.01	0.00669725659228586\\
545.01	0.00673168908357341\\
546.01	0.00676682772076062\\
547.01	0.00680268653162876\\
548.01	0.00683927976249817\\
549.01	0.00687662187436601\\
550.01	0.00691472753799263\\
551.01	0.0069536116278066\\
552.01	0.00699328921448385\\
553.01	0.00703377555604178\\
554.01	0.00707508608726937\\
555.01	0.00711723640729895\\
556.01	0.00716024226510179\\
557.01	0.00720411954266781\\
558.01	0.00724888423560591\\
559.01	0.00729455243087272\\
560.01	0.00734114028130855\\
561.01	0.00738866397662728\\
562.01	0.00743713971047418\\
563.01	0.00748658364312575\\
564.01	0.00753701185936998\\
565.01	0.00758844032106287\\
566.01	0.00764088481381529\\
567.01	0.00769436088721993\\
568.01	0.00774888378798493\\
569.01	0.00780446838529657\\
570.01	0.00786112908769292\\
571.01	0.00791887975069271\\
572.01	0.00797773357439382\\
573.01	0.00803770299023592\\
574.01	0.00809879953611645\\
575.01	0.00816103371906509\\
576.01	0.00822441486472502\\
577.01	0.00828895095297273\\
578.01	0.00835464843913835\\
579.01	0.00842151206048843\\
580.01	0.00848954462791702\\
581.01	0.00855874680318726\\
582.01	0.0086291168626057\\
583.01	0.00870065044873541\\
584.01	0.00877334031271359\\
585.01	0.00884717605099796\\
586.01	0.00892214384200484\\
587.01	0.0089982261902219\\
588.01	0.00907540168810686\\
589.01	0.00915364480957812\\
590.01	0.00923292575336488\\
591.01	0.0093132103601611\\
592.01	0.00939446013472989\\
593.01	0.00947663241322402\\
594.01	0.00955968072750807\\
595.01	0.00964355543279665\\
596.01	0.00972820468321698\\
597.01	0.00981357586291021\\
598.01	0.00989961760918075\\
599.01	0.00996919203046377\\
599.02	0.00996973314335038\\
599.03	0.0099702709571694\\
599.04	0.00997080543920336\\
599.05	0.0099713365564138\\
599.06	0.00997186427543808\\
599.07	0.0099723885625862\\
599.08	0.00997290938383756\\
599.09	0.00997342670483771\\
599.1	0.00997394049089505\\
599.11	0.00997445070697751\\
599.12	0.00997495731770918\\
599.13	0.00997546028736696\\
599.14	0.00997595957987707\\
599.15	0.00997645515881166\\
599.16	0.00997694698738526\\
599.17	0.00997743502845131\\
599.18	0.00997791924449856\\
599.19	0.00997839959764748\\
599.2	0.00997887604964662\\
599.21	0.00997934856186899\\
599.22	0.00997981709530829\\
599.23	0.00998028161057521\\
599.24	0.00998074206789365\\
599.25	0.0099811984270969\\
599.26	0.00998165064762378\\
599.27	0.00998209868851477\\
599.28	0.00998254250840806\\
599.29	0.00998298206553559\\
599.3	0.00998341731771905\\
599.31	0.00998384822236584\\
599.32	0.00998427473646497\\
599.33	0.00998469681658292\\
599.34	0.00998511441885952\\
599.35	0.0099855274990037\\
599.36	0.00998593601228926\\
599.37	0.00998633991355056\\
599.38	0.00998673915717821\\
599.39	0.00998713369711469\\
599.4	0.00998752348684992\\
599.41	0.00998790847811157\\
599.42	0.00998828861974491\\
599.43	0.00998866386008804\\
599.44	0.00998903414696681\\
599.45	0.00998939942768985\\
599.46	0.00998975964904344\\
599.47	0.00999011475728636\\
599.48	0.00999046469814472\\
599.49	0.00999080941680669\\
599.5	0.00999114885791725\\
599.51	0.00999148296557278\\
599.52	0.0099918116833157\\
599.53	0.00999213495412901\\
599.54	0.00999245272043077\\
599.55	0.00999276492406855\\
599.56	0.00999307150631381\\
599.57	0.00999337240785624\\
599.58	0.00999366756879799\\
599.59	0.00999395692864795\\
599.6	0.00999424042631585\\
599.61	0.00999451800010642\\
599.62	0.00999478958771336\\
599.63	0.0099950551262134\\
599.64	0.00999531455206019\\
599.65	0.00999556780107816\\
599.66	0.00999581480845634\\
599.67	0.00999605550874211\\
599.68	0.00999628983583487\\
599.69	0.00999651772297967\\
599.7	0.00999673910276076\\
599.71	0.00999695390709509\\
599.72	0.00999716206722577\\
599.73	0.00999736351371538\\
599.74	0.00999755817643933\\
599.75	0.00999774598457904\\
599.76	0.00999792686661517\\
599.77	0.00999810075032068\\
599.78	0.00999826756275386\\
599.79	0.00999842723025133\\
599.8	0.00999857967842092\\
599.81	0.0099987248321345\\
599.82	0.00999886261552071\\
599.83	0.00999899295195769\\
599.84	0.00999911576406567\\
599.85	0.00999923097369951\\
599.86	0.00999933850194117\\
599.87	0.00999943826909211\\
599.88	0.00999953019466558\\
599.89	0.00999961419737891\\
599.9	0.00999969019514566\\
599.91	0.00999975810506767\\
599.92	0.00999981784342713\\
599.93	0.00999986932567848\\
599.94	0.00999991246644028\\
599.95	0.00999994717948697\\
599.96	0.00999997337774056\\
599.97	0.00999999097326228\\
599.98	0.00999999987724406\\
599.99	0.01\\
600	0.01\\
};
\addplot [color=blue!25!mycolor7,solid,forget plot]
  table[row sep=crcr]{%
0.01	0.0046877913708233\\
1.01	0.00468779351620759\\
2.01	0.00468779570568027\\
3.01	0.00468779794014776\\
4.01	0.00468780022053526\\
5.01	0.00468780254778671\\
6.01	0.00468780492286568\\
7.01	0.00468780734675557\\
8.01	0.00468780982045992\\
9.01	0.00468781234500285\\
10.01	0.00468781492142956\\
11.01	0.00468781755080697\\
12.01	0.00468782023422366\\
13.01	0.00468782297279062\\
14.01	0.00468782576764179\\
15.01	0.0046878286199345\\
16.01	0.0046878315308496\\
17.01	0.00468783450159244\\
18.01	0.00468783753339311\\
19.01	0.00468784062750699\\
20.01	0.00468784378521535\\
21.01	0.00468784700782547\\
22.01	0.00468785029667193\\
23.01	0.00468785365311665\\
24.01	0.00468785707854939\\
25.01	0.00468786057438855\\
26.01	0.00468786414208172\\
27.01	0.00468786778310613\\
28.01	0.0046878714989697\\
29.01	0.00468787529121094\\
30.01	0.0046878791614004\\
31.01	0.0046878831111405\\
32.01	0.00468788714206692\\
33.01	0.0046878912558488\\
34.01	0.00468789545418967\\
35.01	0.00468789973882809\\
36.01	0.0046879041115383\\
37.01	0.00468790857413097\\
38.01	0.004687913128454\\
39.01	0.00468791777639346\\
40.01	0.00468792251987409\\
41.01	0.00468792736085995\\
42.01	0.00468793230135572\\
43.01	0.00468793734340721\\
44.01	0.00468794248910237\\
45.01	0.00468794774057203\\
46.01	0.00468795309999054\\
47.01	0.00468795856957747\\
48.01	0.00468796415159733\\
49.01	0.0046879698483618\\
50.01	0.00468797566222959\\
51.01	0.00468798159560826\\
52.01	0.00468798765095454\\
53.01	0.00468799383077587\\
54.01	0.00468800013763086\\
55.01	0.00468800657413113\\
56.01	0.00468801314294176\\
57.01	0.00468801984678247\\
58.01	0.00468802668842902\\
59.01	0.00468803367071403\\
60.01	0.00468804079652857\\
61.01	0.00468804806882284\\
62.01	0.0046880554906078\\
63.01	0.00468806306495617\\
64.01	0.00468807079500363\\
65.01	0.00468807868395074\\
66.01	0.00468808673506337\\
67.01	0.00468809495167452\\
68.01	0.00468810333718582\\
69.01	0.00468811189506857\\
70.01	0.00468812062886539\\
71.01	0.00468812954219165\\
72.01	0.00468813863873707\\
73.01	0.00468814792226696\\
74.01	0.00468815739662418\\
75.01	0.00468816706573004\\
76.01	0.00468817693358677\\
77.01	0.00468818700427871\\
78.01	0.0046881972819738\\
79.01	0.00468820777092574\\
80.01	0.00468821847547531\\
81.01	0.00468822940005259\\
82.01	0.00468824054917864\\
83.01	0.00468825192746695\\
84.01	0.00468826353962619\\
85.01	0.00468827539046124\\
86.01	0.00468828748487558\\
87.01	0.00468829982787353\\
88.01	0.00468831242456196\\
89.01	0.00468832528015235\\
90.01	0.00468833839996327\\
91.01	0.00468835178942247\\
92.01	0.00468836545406862\\
93.01	0.00468837939955452\\
94.01	0.00468839363164836\\
95.01	0.00468840815623696\\
96.01	0.00468842297932769\\
97.01	0.00468843810705127\\
98.01	0.0046884535456638\\
99.01	0.0046884693015498\\
100.01	0.00468848538122497\\
101.01	0.0046885017913378\\
102.01	0.00468851853867376\\
103.01	0.00468853563015703\\
104.01	0.00468855307285371\\
105.01	0.00468857087397483\\
106.01	0.00468858904087895\\
107.01	0.00468860758107535\\
108.01	0.00468862650222744\\
109.01	0.00468864581215518\\
110.01	0.00468866551883915\\
111.01	0.00468868563042284\\
112.01	0.00468870615521692\\
113.01	0.00468872710170185\\
114.01	0.00468874847853218\\
115.01	0.00468877029453905\\
116.01	0.00468879255873471\\
117.01	0.0046888152803158\\
118.01	0.00468883846866718\\
119.01	0.00468886213336556\\
120.01	0.00468888628418372\\
121.01	0.0046889109310944\\
122.01	0.00468893608427406\\
123.01	0.00468896175410764\\
124.01	0.00468898795119226\\
125.01	0.00468901468634156\\
126.01	0.00468904197059049\\
127.01	0.00468906981519962\\
128.01	0.00468909823165932\\
129.01	0.00468912723169509\\
130.01	0.00468915682727183\\
131.01	0.00468918703059898\\
132.01	0.00468921785413555\\
133.01	0.00468924931059461\\
134.01	0.00468928141294914\\
135.01	0.00468931417443692\\
136.01	0.00468934760856594\\
137.01	0.00468938172912008\\
138.01	0.00468941655016439\\
139.01	0.00468945208605075\\
140.01	0.00468948835142417\\
141.01	0.00468952536122824\\
142.01	0.00468956313071139\\
143.01	0.00468960167543311\\
144.01	0.00468964101126965\\
145.01	0.00468968115442142\\
146.01	0.00468972212141856\\
147.01	0.00468976392912843\\
148.01	0.00468980659476177\\
149.01	0.00468985013587997\\
150.01	0.00468989457040206\\
151.01	0.004689939916612\\
152.01	0.00468998619316581\\
153.01	0.0046900334190991\\
154.01	0.00469008161383513\\
155.01	0.00469013079719205\\
156.01	0.00469018098939114\\
157.01	0.00469023221106503\\
158.01	0.00469028448326561\\
159.01	0.00469033782747277\\
160.01	0.00469039226560272\\
161.01	0.0046904478200171\\
162.01	0.00469050451353166\\
163.01	0.00469056236942537\\
164.01	0.00469062141144987\\
165.01	0.00469068166383869\\
166.01	0.00469074315131712\\
167.01	0.00469080589911186\\
168.01	0.00469086993296136\\
169.01	0.00469093527912584\\
170.01	0.00469100196439738\\
171.01	0.00469107001611126\\
172.01	0.00469113946215627\\
173.01	0.00469121033098589\\
174.01	0.00469128265162952\\
175.01	0.00469135645370398\\
176.01	0.00469143176742538\\
177.01	0.00469150862362071\\
178.01	0.00469158705374005\\
179.01	0.00469166708986911\\
180.01	0.0046917487647417\\
181.01	0.00469183211175282\\
182.01	0.00469191716497146\\
183.01	0.00469200395915406\\
184.01	0.0046920925297582\\
185.01	0.00469218291295631\\
186.01	0.00469227514564995\\
187.01	0.00469236926548416\\
188.01	0.00469246531086198\\
189.01	0.00469256332095967\\
190.01	0.0046926633357415\\
191.01	0.00469276539597568\\
192.01	0.00469286954325\\
193.01	0.00469297581998768\\
194.01	0.00469308426946434\\
195.01	0.00469319493582397\\
196.01	0.00469330786409664\\
197.01	0.00469342310021517\\
198.01	0.00469354069103299\\
199.01	0.00469366068434245\\
200.01	0.00469378312889261\\
201.01	0.00469390807440817\\
202.01	0.00469403557160822\\
203.01	0.00469416567222568\\
204.01	0.00469429842902691\\
205.01	0.00469443389583178\\
206.01	0.00469457212753394\\
207.01	0.00469471318012162\\
208.01	0.00469485711069827\\
209.01	0.00469500397750496\\
210.01	0.00469515383994136\\
211.01	0.00469530675858849\\
212.01	0.00469546279523128\\
213.01	0.00469562201288101\\
214.01	0.00469578447579952\\
215.01	0.00469595024952263\\
216.01	0.00469611940088441\\
217.01	0.00469629199804179\\
218.01	0.00469646811049951\\
219.01	0.00469664780913606\\
220.01	0.00469683116622897\\
221.01	0.00469701825548141\\
222.01	0.0046972091520493\\
223.01	0.00469740393256773\\
224.01	0.00469760267517942\\
225.01	0.00469780545956214\\
226.01	0.00469801236695772\\
227.01	0.00469822348020087\\
228.01	0.00469843888374873\\
229.01	0.00469865866371036\\
230.01	0.00469888290787765\\
231.01	0.00469911170575583\\
232.01	0.00469934514859474\\
233.01	0.00469958332942075\\
234.01	0.00469982634306883\\
235.01	0.00470007428621501\\
236.01	0.00470032725741021\\
237.01	0.004700585357113\\
238.01	0.00470084868772419\\
239.01	0.00470111735362118\\
240.01	0.00470139146119281\\
241.01	0.00470167111887497\\
242.01	0.00470195643718639\\
243.01	0.00470224752876482\\
244.01	0.00470254450840389\\
245.01	0.00470284749309043\\
246.01	0.00470315660204168\\
247.01	0.00470347195674359\\
248.01	0.00470379368098893\\
249.01	0.00470412190091624\\
250.01	0.00470445674504887\\
251.01	0.00470479834433482\\
252.01	0.00470514683218584\\
253.01	0.00470550234451828\\
254.01	0.00470586501979306\\
255.01	0.00470623499905659\\
256.01	0.00470661242598161\\
257.01	0.00470699744690858\\
258.01	0.00470739021088667\\
259.01	0.00470779086971571\\
260.01	0.00470819957798718\\
261.01	0.00470861649312641\\
262.01	0.00470904177543439\\
263.01	0.00470947558812912\\
264.01	0.00470991809738729\\
265.01	0.00471036947238596\\
266.01	0.0047108298853437\\
267.01	0.00471129951156185\\
268.01	0.0047117785294653\\
269.01	0.00471226712064268\\
270.01	0.00471276546988608\\
271.01	0.00471327376523074\\
272.01	0.00471379219799338\\
273.01	0.00471432096281013\\
274.01	0.00471486025767372\\
275.01	0.00471541028396903\\
276.01	0.00471597124650817\\
277.01	0.00471654335356393\\
278.01	0.00471712681690171\\
279.01	0.00471772185181048\\
280.01	0.004718328677131\\
281.01	0.00471894751528298\\
282.01	0.00471957859228921\\
283.01	0.00472022213779801\\
284.01	0.00472087838510299\\
285.01	0.00472154757115958\\
286.01	0.00472222993659834\\
287.01	0.00472292572573532\\
288.01	0.00472363518657853\\
289.01	0.00472435857082978\\
290.01	0.00472509613388258\\
291.01	0.00472584813481511\\
292.01	0.00472661483637805\\
293.01	0.00472739650497579\\
294.01	0.00472819341064325\\
295.01	0.00472900582701362\\
296.01	0.00472983403128104\\
297.01	0.00473067830415359\\
298.01	0.00473153892979923\\
299.01	0.00473241619578142\\
300.01	0.00473331039298522\\
301.01	0.00473422181553319\\
302.01	0.00473515076068848\\
303.01	0.00473609752874667\\
304.01	0.00473706242291323\\
305.01	0.00473804574916711\\
306.01	0.00473904781610854\\
307.01	0.00474006893478995\\
308.01	0.00474110941852864\\
309.01	0.00474216958270086\\
310.01	0.00474324974451443\\
311.01	0.00474435022275863\\
312.01	0.0047454713375312\\
313.01	0.00474661340993866\\
314.01	0.00474777676176948\\
315.01	0.00474896171513786\\
316.01	0.00475016859209591\\
317.01	0.00475139771421119\\
318.01	0.00475264940211023\\
319.01	0.00475392397498133\\
320.01	0.00475522175003794\\
321.01	0.00475654304193771\\
322.01	0.0047578881621554\\
323.01	0.0047592574183065\\
324.01	0.00476065111341886\\
325.01	0.00476206954514856\\
326.01	0.0047635130049386\\
327.01	0.00476498177711507\\
328.01	0.00476647613791956\\
329.01	0.00476799635447413\\
330.01	0.00476954268367623\\
331.01	0.00477111537101982\\
332.01	0.00477271464934223\\
333.01	0.0047743407374924\\
334.01	0.00477599383892136\\
335.01	0.0047776741401921\\
336.01	0.00477938180941025\\
337.01	0.00478111699457616\\
338.01	0.00478287982186117\\
339.01	0.00478467039381171\\
340.01	0.00478648878748918\\
341.01	0.00478833505255249\\
342.01	0.00479020920929768\\
343.01	0.00479211124666848\\
344.01	0.00479404112026045\\
345.01	0.00479599875034337\\
346.01	0.00479798401993539\\
347.01	0.00479999677296886\\
348.01	0.00480203681259828\\
349.01	0.0048041038997108\\
350.01	0.00480619775171291\\
351.01	0.00480831804168072\\
352.01	0.00481046439798116\\
353.01	0.00481263640448791\\
354.01	0.00481483360154139\\
355.01	0.00481705548782523\\
356.01	0.00481930152336384\\
357.01	0.00482157113387317\\
358.01	0.00482386371673542\\
359.01	0.00482617864890168\\
360.01	0.00482851529706731\\
361.01	0.0048308730304962\\
362.01	0.00483325123690904\\
363.01	0.0048356493418704\\
364.01	0.00483806683212031\\
365.01	0.00484050328328288\\
366.01	0.00484295839233395\\
367.01	0.00484543201510167\\
368.01	0.00484792420889242\\
369.01	0.00485043528003107\\
370.01	0.00485296583564418\\
371.01	0.00485551683832332\\
372.01	0.00485808966130107\\
373.01	0.00486068614033032\\
374.01	0.00486330861641787\\
375.01	0.00486595996071022\\
376.01	0.00486864356886456\\
377.01	0.0048713633067735\\
378.01	0.00487412338201897\\
379.01	0.00487692810520714\\
380.01	0.0048797810365406\\
381.01	0.00488268339855548\\
382.01	0.00488563585314032\\
383.01	0.00488863906102592\\
384.01	0.00489169368111608\\
385.01	0.00489480036977825\\
386.01	0.00489795978009008\\
387.01	0.00490117256104255\\
388.01	0.00490443935669549\\
389.01	0.00490776080528592\\
390.01	0.00491113753828591\\
391.01	0.00491457017940904\\
392.01	0.00491805934356283\\
393.01	0.00492160563574746\\
394.01	0.00492520964989636\\
395.01	0.00492887196766008\\
396.01	0.00493259315713027\\
397.01	0.00493637377150284\\
398.01	0.00494021434768074\\
399.01	0.00494411540481332\\
400.01	0.00494807744277466\\
401.01	0.00495210094057746\\
402.01	0.00495618635472721\\
403.01	0.00496033411751279\\
404.01	0.00496454463523903\\
405.01	0.0049688182864005\\
406.01	0.00497315541980127\\
407.01	0.0049775563526234\\
408.01	0.00498202136844939\\
409.01	0.00498655071524534\\
410.01	0.00499114460331061\\
411.01	0.00499580320320435\\
412.01	0.00500052664365926\\
413.01	0.0050053150094949\\
414.01	0.00501016833954575\\
415.01	0.00501508662462151\\
416.01	0.00502006980552037\\
417.01	0.00502511777111758\\
418.01	0.00503023035655757\\
419.01	0.00503540734158045\\
420.01	0.00504064844901714\\
421.01	0.00504595334349552\\
422.01	0.00505132163040094\\
423.01	0.00505675285514366\\
424.01	0.00506224650279165\\
425.01	0.00506780199813149\\
426.01	0.00507341870623117\\
427.01	0.00507909593358397\\
428.01	0.00508483292992171\\
429.01	0.0050906288907933\\
430.01	0.00509648296101652\\
431.01	0.00510239423911465\\
432.01	0.00510836178286296\\
433.01	0.0051143846160757\\
434.01	0.00512046173677136\\
435.01	0.00512659212686058\\
436.01	0.00513277476350303\\
437.01	0.00513900863227999\\
438.01	0.00514529274232582\\
439.01	0.00515162614354973\\
440.01	0.00515800794606375\\
441.01	0.00516443734190444\\
442.01	0.00517091362909716\\
443.01	0.00517743623805843\\
444.01	0.00518400476026008\\
445.01	0.00519061897898486\\
446.01	0.00519727890188222\\
447.01	0.00520398479488479\\
448.01	0.00521073721685766\\
449.01	0.00521753705412668\\
450.01	0.00522438555376617\\
451.01	0.00523128435420685\\
452.01	0.00523823551137073\\
453.01	0.00524524151814018\\
454.01	0.00525230531454857\\
455.01	0.00525943028566056\\
456.01	0.00526662024373692\\
457.01	0.00527387939102032\\
458.01	0.00528121225944443\\
459.01	0.00528862362391745\\
460.01	0.00529611838680955\\
461.01	0.00530370143321333\\
462.01	0.00531137745996266\\
463.01	0.0053191507869801\\
464.01	0.00532702516825855\\
465.01	0.0053350036328947\\
466.01	0.00534308841881588\\
467.01	0.00535128129746901\\
468.01	0.00535958409596056\\
469.01	0.00536799879274197\\
470.01	0.00537652753053824\\
471.01	0.00538517262929297\\
472.01	0.00539393659903768\\
473.01	0.00540282215246299\\
474.01	0.00541183221693191\\
475.01	0.00542096994562932\\
476.01	0.00543023872749784\\
477.01	0.00543964219556525\\
478.01	0.00544918423322939\\
479.01	0.00545886897804068\\
480.01	0.00546870082249888\\
481.01	0.00547868441138057\\
482.01	0.00548882463513328\\
483.01	0.00549912661892601\\
484.01	0.00550959570704213\\
485.01	0.00552023744245141\\
486.01	0.00553105754161531\\
487.01	0.00554206186487474\\
488.01	0.00555325638315463\\
489.01	0.00556464714220292\\
490.01	0.00557624022615136\\
491.01	0.00558804172283474\\
492.01	0.00560005769397902\\
493.01	0.00561229415398543\\
494.01	0.00562475706143512\\
495.01	0.00563745232737498\\
496.01	0.00565038584350466\\
497.01	0.00566356353096031\\
498.01	0.00567699140473343\\
499.01	0.00569067562622107\\
500.01	0.00570462251382754\\
501.01	0.0057188385381076\\
502.01	0.00573333031557464\\
503.01	0.00574810460175816\\
504.01	0.0057631682837062\\
505.01	0.00577852837218618\\
506.01	0.00579419199390473\\
507.01	0.00581016638412089\\
508.01	0.00582645888008326\\
509.01	0.00584307691575363\\
510.01	0.00586002801828694\\
511.01	0.00587731980670701\\
512.01	0.00589495999313232\\
513.01	0.00591295638675328\\
514.01	0.00593131690053231\\
515.01	0.00595004956028283\\
516.01	0.00596916251539123\\
517.01	0.0059886640500192\\
518.01	0.00600856259324166\\
519.01	0.00602886672640598\\
520.01	0.00604958518667911\\
521.01	0.00607072686832795\\
522.01	0.00609230082395634\\
523.01	0.00611431626621421\\
524.01	0.00613678257003705\\
525.01	0.00615970927544816\\
526.01	0.00618310609092311\\
527.01	0.00620698289728304\\
528.01	0.00623134975204244\\
529.01	0.00625621689409586\\
530.01	0.00628159474858923\\
531.01	0.00630749393178962\\
532.01	0.00633392525575795\\
533.01	0.00636089973264133\\
534.01	0.00638842857845291\\
535.01	0.00641652321629788\\
536.01	0.00644519527912794\\
537.01	0.00647445661221022\\
538.01	0.00650431927544885\\
539.01	0.00653479554555353\\
540.01	0.0065658979179944\\
541.01	0.00659763910867128\\
542.01	0.00663003205521878\\
543.01	0.00666308991785967\\
544.01	0.00669682607971868\\
545.01	0.00673125414650544\\
546.01	0.00676638794547803\\
547.01	0.00680224152360036\\
548.01	0.00683882914480995\\
549.01	0.00687616528631111\\
550.01	0.00691426463379902\\
551.01	0.00695314207550284\\
552.01	0.00699281269490666\\
553.01	0.00703329176198434\\
554.01	0.00707459472276593\\
555.01	0.00711673718703461\\
556.01	0.00715973491393524\\
557.01	0.00720360379525376\\
558.01	0.00724835983610392\\
559.01	0.00729401913273179\\
560.01	0.00734059784712131\\
561.01	0.00738811217805081\\
562.01	0.00743657832821639\\
563.01	0.00748601246700066\\
564.01	0.0075364306884236\\
565.01	0.00758784896377284\\
566.01	0.00764028308836689\\
567.01	0.00769374862186187\\
568.01	0.00774826082146843\\
569.01	0.00780383456740263\\
570.01	0.00786048427985252\\
571.01	0.00791822382670581\\
572.01	0.00797706642125281\\
573.01	0.00803702450905831\\
574.01	0.00809810964319064\\
575.01	0.00816033234701034\\
576.01	0.00822370196376449\\
577.01	0.0082882264923125\\
578.01	0.00835391240844111\\
579.01	0.00842076447142201\\
580.01	0.00848878551574813\\
581.01	0.008557976228378\\
582.01	0.00862833491235338\\
583.01	0.0086998572383756\\
584.01	0.00877253598687894\\
585.01	0.00884636078439139\\
586.01	0.00892131783960339\\
587.01	0.00899738968667349\\
588.01	0.00907455494601469\\
589.01	0.00915278811628361\\
590.01	0.0092320594157346\\
591.01	0.00931233469675008\\
592.01	0.00939357546453003\\
593.01	0.00947573903999945\\
594.01	0.00955877891846273\\
595.01	0.00964264538999877\\
596.01	0.00972728650580666\\
597.01	0.00981264949761991\\
598.01	0.00989868278608211\\
599.01	0.00996919203046377\\
599.02	0.00996973314335038\\
599.03	0.0099702709571694\\
599.04	0.00997080543920336\\
599.05	0.0099713365564138\\
599.06	0.00997186427543808\\
599.07	0.0099723885625862\\
599.08	0.00997290938383756\\
599.09	0.00997342670483771\\
599.1	0.00997394049089505\\
599.11	0.00997445070697751\\
599.12	0.00997495731770918\\
599.13	0.00997546028736696\\
599.14	0.00997595957987707\\
599.15	0.00997645515881165\\
599.16	0.00997694698738526\\
599.17	0.00997743502845131\\
599.18	0.00997791924449856\\
599.19	0.00997839959764748\\
599.2	0.00997887604964662\\
599.21	0.00997934856186899\\
599.22	0.00997981709530829\\
599.23	0.00998028161057521\\
599.24	0.00998074206789365\\
599.25	0.0099811984270969\\
599.26	0.00998165064762378\\
599.27	0.00998209868851477\\
599.28	0.00998254250840806\\
599.29	0.00998298206553559\\
599.3	0.00998341731771905\\
599.31	0.00998384822236584\\
599.32	0.00998427473646497\\
599.33	0.00998469681658292\\
599.34	0.00998511441885952\\
599.35	0.0099855274990037\\
599.36	0.00998593601228926\\
599.37	0.00998633991355056\\
599.38	0.00998673915717821\\
599.39	0.00998713369711469\\
599.4	0.00998752348684992\\
599.41	0.00998790847811157\\
599.42	0.00998828861974491\\
599.43	0.00998866386008803\\
599.44	0.00998903414696681\\
599.45	0.00998939942768986\\
599.46	0.00998975964904344\\
599.47	0.00999011475728636\\
599.48	0.00999046469814472\\
599.49	0.00999080941680669\\
599.5	0.00999114885791725\\
599.51	0.00999148296557278\\
599.52	0.0099918116833157\\
599.53	0.00999213495412901\\
599.54	0.00999245272043077\\
599.55	0.00999276492406855\\
599.56	0.00999307150631382\\
599.57	0.00999337240785624\\
599.58	0.00999366756879799\\
599.59	0.00999395692864795\\
599.6	0.00999424042631586\\
599.61	0.00999451800010642\\
599.62	0.00999478958771336\\
599.63	0.0099950551262134\\
599.64	0.00999531455206019\\
599.65	0.00999556780107816\\
599.66	0.00999581480845634\\
599.67	0.00999605550874211\\
599.68	0.00999628983583487\\
599.69	0.00999651772297967\\
599.7	0.00999673910276075\\
599.71	0.00999695390709509\\
599.72	0.00999716206722577\\
599.73	0.00999736351371538\\
599.74	0.00999755817643933\\
599.75	0.00999774598457904\\
599.76	0.00999792686661517\\
599.77	0.00999810075032068\\
599.78	0.00999826756275386\\
599.79	0.00999842723025133\\
599.8	0.00999857967842092\\
599.81	0.0099987248321345\\
599.82	0.00999886261552071\\
599.83	0.00999899295195769\\
599.84	0.00999911576406567\\
599.85	0.00999923097369951\\
599.86	0.00999933850194117\\
599.87	0.00999943826909211\\
599.88	0.00999953019466558\\
599.89	0.00999961419737891\\
599.9	0.00999969019514566\\
599.91	0.00999975810506767\\
599.92	0.00999981784342713\\
599.93	0.00999986932567848\\
599.94	0.00999991246644028\\
599.95	0.00999994717948697\\
599.96	0.00999997337774056\\
599.97	0.00999999097326228\\
599.98	0.00999999987724406\\
599.99	0.01\\
600	0.01\\
};
\addplot [color=mycolor9,solid,forget plot]
  table[row sep=crcr]{%
0.01	0.00433017084612224\\
1.01	0.00433017312500381\\
2.01	0.00433017545095497\\
3.01	0.00433017782494978\\
4.01	0.00433018024798222\\
5.01	0.00433018272106704\\
6.01	0.00433018524523977\\
7.01	0.00433018782155786\\
8.01	0.00433019045110017\\
9.01	0.00433019313496824\\
10.01	0.00433019587428649\\
11.01	0.00433019867020241\\
12.01	0.00433020152388747\\
13.01	0.00433020443653742\\
14.01	0.00433020740937283\\
15.01	0.00433021044363945\\
16.01	0.00433021354060889\\
17.01	0.00433021670157932\\
18.01	0.00433021992787565\\
19.01	0.00433022322085035\\
20.01	0.00433022658188384\\
21.01	0.0043302300123854\\
22.01	0.00433023351379334\\
23.01	0.00433023708757577\\
24.01	0.00433024073523149\\
25.01	0.00433024445829021\\
26.01	0.00433024825831346\\
27.01	0.00433025213689534\\
28.01	0.00433025609566255\\
29.01	0.00433026013627608\\
30.01	0.0043302642604309\\
31.01	0.00433026846985764\\
32.01	0.0043302727663225\\
33.01	0.00433027715162843\\
34.01	0.00433028162761594\\
35.01	0.00433028619616342\\
36.01	0.00433029085918867\\
37.01	0.00433029561864896\\
38.01	0.00433030047654247\\
39.01	0.00433030543490852\\
40.01	0.0043303104958291\\
41.01	0.00433031566142937\\
42.01	0.00433032093387836\\
43.01	0.00433032631539047\\
44.01	0.00433033180822573\\
45.01	0.00433033741469135\\
46.01	0.0043303431371427\\
47.01	0.00433034897798339\\
48.01	0.00433035493966745\\
49.01	0.00433036102469984\\
50.01	0.00433036723563757\\
51.01	0.00433037357509045\\
52.01	0.00433038004572285\\
53.01	0.00433038665025423\\
54.01	0.00433039339146085\\
55.01	0.00433040027217621\\
56.01	0.00433040729529295\\
57.01	0.004330414463764\\
58.01	0.00433042178060344\\
59.01	0.00433042924888788\\
60.01	0.00433043687175783\\
61.01	0.00433044465241937\\
62.01	0.00433045259414487\\
63.01	0.00433046070027501\\
64.01	0.00433046897421975\\
65.01	0.00433047741945979\\
66.01	0.00433048603954836\\
67.01	0.00433049483811269\\
68.01	0.00433050381885497\\
69.01	0.00433051298555459\\
70.01	0.0043305223420696\\
71.01	0.00433053189233826\\
72.01	0.00433054164038044\\
73.01	0.00433055159029981\\
74.01	0.00433056174628528\\
75.01	0.00433057211261298\\
76.01	0.00433058269364797\\
77.01	0.00433059349384581\\
78.01	0.00433060451775481\\
79.01	0.00433061577001809\\
80.01	0.00433062725537522\\
81.01	0.00433063897866427\\
82.01	0.00433065094482382\\
83.01	0.0043306631588957\\
84.01	0.00433067562602585\\
85.01	0.00433068835146781\\
86.01	0.00433070134058434\\
87.01	0.00433071459884948\\
88.01	0.00433072813185141\\
89.01	0.00433074194529469\\
90.01	0.00433075604500216\\
91.01	0.00433077043691803\\
92.01	0.00433078512711026\\
93.01	0.00433080012177299\\
94.01	0.00433081542722946\\
95.01	0.00433083104993409\\
96.01	0.00433084699647591\\
97.01	0.004330863273581\\
98.01	0.00433087988811552\\
99.01	0.00433089684708847\\
100.01	0.00433091415765466\\
101.01	0.00433093182711798\\
102.01	0.00433094986293427\\
103.01	0.00433096827271437\\
104.01	0.00433098706422815\\
105.01	0.00433100624540667\\
106.01	0.00433102582434618\\
107.01	0.00433104580931194\\
108.01	0.00433106620874071\\
109.01	0.00433108703124521\\
110.01	0.00433110828561756\\
111.01	0.0043311299808329\\
112.01	0.00433115212605309\\
113.01	0.00433117473063106\\
114.01	0.00433119780411428\\
115.01	0.00433122135624922\\
116.01	0.00433124539698553\\
117.01	0.00433126993647953\\
118.01	0.00433129498509968\\
119.01	0.0043313205534299\\
120.01	0.00433134665227493\\
121.01	0.00433137329266459\\
122.01	0.00433140048585827\\
123.01	0.00433142824335001\\
124.01	0.0043314565768733\\
125.01	0.00433148549840623\\
126.01	0.00433151502017609\\
127.01	0.00433154515466544\\
128.01	0.0043315759146164\\
129.01	0.00433160731303699\\
130.01	0.00433163936320614\\
131.01	0.00433167207867974\\
132.01	0.00433170547329559\\
133.01	0.00433173956118046\\
134.01	0.00433177435675522\\
135.01	0.00433180987474144\\
136.01	0.00433184613016742\\
137.01	0.0043318831383747\\
138.01	0.00433192091502473\\
139.01	0.00433195947610543\\
140.01	0.00433199883793768\\
141.01	0.00433203901718285\\
142.01	0.00433208003084949\\
143.01	0.00433212189630079\\
144.01	0.00433216463126229\\
145.01	0.00433220825382846\\
146.01	0.00433225278247137\\
147.01	0.00433229823604812\\
148.01	0.00433234463380919\\
149.01	0.00433239199540626\\
150.01	0.00433244034090077\\
151.01	0.00433248969077268\\
152.01	0.00433254006592888\\
153.01	0.00433259148771226\\
154.01	0.00433264397791094\\
155.01	0.00433269755876733\\
156.01	0.00433275225298779\\
157.01	0.00433280808375204\\
158.01	0.00433286507472368\\
159.01	0.0043329232500595\\
160.01	0.00433298263442053\\
161.01	0.00433304325298182\\
162.01	0.00433310513144383\\
163.01	0.00433316829604315\\
164.01	0.00433323277356362\\
165.01	0.00433329859134778\\
166.01	0.00433336577730876\\
167.01	0.00433343435994199\\
168.01	0.00433350436833752\\
169.01	0.00433357583219236\\
170.01	0.00433364878182305\\
171.01	0.0043337232481787\\
172.01	0.00433379926285424\\
173.01	0.00433387685810409\\
174.01	0.00433395606685556\\
175.01	0.00433403692272322\\
176.01	0.00433411946002318\\
177.01	0.00433420371378762\\
178.01	0.00433428971978003\\
179.01	0.00433437751451023\\
180.01	0.00433446713525018\\
181.01	0.00433455862004985\\
182.01	0.00433465200775332\\
183.01	0.00433474733801562\\
184.01	0.00433484465131949\\
185.01	0.00433494398899309\\
186.01	0.00433504539322707\\
187.01	0.00433514890709322\\
188.01	0.00433525457456245\\
189.01	0.00433536244052353\\
190.01	0.00433547255080315\\
191.01	0.00433558495218429\\
192.01	0.00433569969242707\\
193.01	0.00433581682028934\\
194.01	0.00433593638554661\\
195.01	0.00433605843901434\\
196.01	0.00433618303256922\\
197.01	0.00433631021917127\\
198.01	0.00433644005288711\\
199.01	0.00433657258891254\\
200.01	0.00433670788359651\\
201.01	0.00433684599446493\\
202.01	0.00433698698024577\\
203.01	0.00433713090089369\\
204.01	0.00433727781761672\\
205.01	0.00433742779290112\\
206.01	0.00433758089053933\\
207.01	0.00433773717565676\\
208.01	0.00433789671474018\\
209.01	0.0043380595756658\\
210.01	0.00433822582772853\\
211.01	0.00433839554167183\\
212.01	0.00433856878971793\\
213.01	0.00433874564559936\\
214.01	0.00433892618459006\\
215.01	0.00433911048353805\\
216.01	0.00433929862089806\\
217.01	0.00433949067676585\\
218.01	0.00433968673291245\\
219.01	0.00433988687281911\\
220.01	0.00434009118171333\\
221.01	0.00434029974660607\\
222.01	0.00434051265632852\\
223.01	0.00434073000157076\\
224.01	0.00434095187492078\\
225.01	0.0043411783709045\\
226.01	0.00434140958602644\\
227.01	0.00434164561881134\\
228.01	0.00434188656984646\\
229.01	0.0043421325418261\\
230.01	0.0043423836395948\\
231.01	0.00434263997019303\\
232.01	0.00434290164290378\\
233.01	0.00434316876929976\\
234.01	0.00434344146329135\\
235.01	0.0043437198411766\\
236.01	0.00434400402169097\\
237.01	0.00434429412605936\\
238.01	0.00434459027804859\\
239.01	0.00434489260402099\\
240.01	0.00434520123298968\\
241.01	0.0043455162966741\\
242.01	0.00434583792955812\\
243.01	0.00434616626894793\\
244.01	0.00434650145503243\\
245.01	0.00434684363094361\\
246.01	0.00434719294281992\\
247.01	0.00434754953986957\\
248.01	0.00434791357443598\\
249.01	0.00434828520206429\\
250.01	0.00434866458156952\\
251.01	0.00434905187510615\\
252.01	0.00434944724823953\\
253.01	0.00434985087001818\\
254.01	0.0043502629130481\\
255.01	0.00435068355356909\\
256.01	0.00435111297153221\\
257.01	0.00435155135067847\\
258.01	0.00435199887862077\\
259.01	0.00435245574692603\\
260.01	0.00435292215120055\\
261.01	0.00435339829117608\\
262.01	0.00435388437079812\\
263.01	0.00435438059831655\\
264.01	0.00435488718637828\\
265.01	0.00435540435212144\\
266.01	0.00435593231727177\\
267.01	0.00435647130824146\\
268.01	0.00435702155622998\\
269.01	0.00435758329732734\\
270.01	0.00435815677261919\\
271.01	0.0043587422282943\\
272.01	0.00435933991575502\\
273.01	0.00435995009172968\\
274.01	0.00436057301838702\\
275.01	0.0043612089634542\\
276.01	0.00436185820033605\\
277.01	0.00436252100823858\\
278.01	0.0043631976722935\\
279.01	0.00436388848368592\\
280.01	0.00436459373978564\\
281.01	0.00436531374427969\\
282.01	0.00436604880730937\\
283.01	0.00436679924560854\\
284.01	0.00436756538264553\\
285.01	0.00436834754876796\\
286.01	0.00436914608135021\\
287.01	0.0043699613249442\\
288.01	0.00437079363143217\\
289.01	0.00437164336018346\\
290.01	0.00437251087821332\\
291.01	0.0043733965603453\\
292.01	0.00437430078937547\\
293.01	0.00437522395624056\\
294.01	0.00437616646018743\\
295.01	0.00437712870894692\\
296.01	0.00437811111890782\\
297.01	0.00437911411529571\\
298.01	0.00438013813235176\\
299.01	0.00438118361351485\\
300.01	0.00438225101160518\\
301.01	0.00438334078900816\\
302.01	0.00438445341786145\\
303.01	0.00438558938024082\\
304.01	0.00438674916834735\\
305.01	0.0043879332846942\\
306.01	0.0043891422422918\\
307.01	0.00439037656483271\\
308.01	0.0043916367868736\\
309.01	0.00439292345401367\\
310.01	0.00439423712306931\\
311.01	0.00439557836224365\\
312.01	0.00439694775128867\\
313.01	0.00439834588165999\\
314.01	0.00439977335666125\\
315.01	0.00440123079157653\\
316.01	0.00440271881378908\\
317.01	0.00440423806288427\\
318.01	0.0044057891907312\\
319.01	0.00440737286154438\\
320.01	0.00440898975191703\\
321.01	0.00441064055082571\\
322.01	0.00441232595959958\\
323.01	0.00441404669184965\\
324.01	0.00441580347335209\\
325.01	0.00441759704187901\\
326.01	0.00441942814696891\\
327.01	0.00442129754962872\\
328.01	0.00442320602195728\\
329.01	0.00442515434667984\\
330.01	0.00442714331658105\\
331.01	0.00442917373382191\\
332.01	0.004431246409127\\
333.01	0.00443336216082205\\
334.01	0.00443552181370352\\
335.01	0.00443772619771644\\
336.01	0.004439976146417\\
337.01	0.00444227249518929\\
338.01	0.00444461607918624\\
339.01	0.00444700773095674\\
340.01	0.00444944827772067\\
341.01	0.00445193853824481\\
342.01	0.00445447931927011\\
343.01	0.00445707141143288\\
344.01	0.00445971558461861\\
345.01	0.00446241258267614\\
346.01	0.00446516311741744\\
347.01	0.00446796786181537\\
348.01	0.00447082744230989\\
349.01	0.00447374243011707\\
350.01	0.00447671333143547\\
351.01	0.00447974057643231\\
352.01	0.00448282450688689\\
353.01	0.00448596536236742\\
354.01	0.0044891632648139\\
355.01	0.00449241820140781\\
356.01	0.00449573000562052\\
357.01	0.00449909833635545\\
358.01	0.00450252265513718\\
359.01	0.00450600220135811\\
360.01	0.00450953596567819\\
361.01	0.00451312266179822\\
362.01	0.00451676069699428\\
363.01	0.00452044814204629\\
364.01	0.00452418270151704\\
365.01	0.00452796168578788\\
366.01	0.00453178198685936\\
367.01	0.00453564006073842\\
368.01	0.00453953192031815\\
369.01	0.00454345314410495\\
370.01	0.00454739890806999\\
371.01	0.0045513640504516\\
372.01	0.00455534318270358\\
373.01	0.00455933086423309\\
374.01	0.00456332186443281\\
375.01	0.00456731154322568\\
376.01	0.00457129639147949\\
377.01	0.00457527478596882\\
378.01	0.00457924803103825\\
379.01	0.00458322178203278\\
380.01	0.00458723576922909\\
381.01	0.00459132580643362\\
382.01	0.00459549317978798\\
383.01	0.00459973918071012\\
384.01	0.00460406511206603\\
385.01	0.00460847228753523\\
386.01	0.00461296203090729\\
387.01	0.00461753567530385\\
388.01	0.00462219456232112\\
389.01	0.00462694004108527\\
390.01	0.00463177346721592\\
391.01	0.00463669620168919\\
392.01	0.00464170960959405\\
393.01	0.00464681505877265\\
394.01	0.00465201391833729\\
395.01	0.00465730755705253\\
396.01	0.00466269734157579\\
397.01	0.00466818463454245\\
398.01	0.00467377079248614\\
399.01	0.00467945716358194\\
400.01	0.00468524508519763\\
401.01	0.00469113588124237\\
402.01	0.00469713085929439\\
403.01	0.00470323130749577\\
404.01	0.00470943849119407\\
405.01	0.00471575364931583\\
406.01	0.0047221779904517\\
407.01	0.00472871268863441\\
408.01	0.00473535887878805\\
409.01	0.00474211765182792\\
410.01	0.00474899004938528\\
411.01	0.00475597705813798\\
412.01	0.00476307960371755\\
413.01	0.00477029854417019\\
414.01	0.00477763466294448\\
415.01	0.0047850886613799\\
416.01	0.00479266115066852\\
417.01	0.00480035264326593\\
418.01	0.00480816354372278\\
419.01	0.00481609413891522\\
420.01	0.00482414458765092\\
421.01	0.00483231490963051\\
422.01	0.00484060497375127\\
423.01	0.00484901448574133\\
424.01	0.00485754297512131\\
425.01	0.00486618978150076\\
426.01	0.00487495404022541\\
427.01	0.00488383466740644\\
428.01	0.00489283034438197\\
429.01	0.00490193950168054\\
430.01	0.00491116030258371\\
431.01	0.00492049062641793\\
432.01	0.00492992805174296\\
433.01	0.00493946983965368\\
434.01	0.00494911291746512\\
435.01	0.00495885386312005\\
436.01	0.00496868889073661\\
437.01	0.00497861383780683\\
438.01	0.0049886241546663\\
439.01	0.00499871489698278\\
440.01	0.00500888072215998\\
441.01	0.00501911589071902\\
442.01	0.00502941427391444\\
443.01	0.00503976936905594\\
444.01	0.00505017432424469\\
445.01	0.00506062197449125\\
446.01	0.00507110489145547\\
447.01	0.00508161544932473\\
448.01	0.00509214590961422\\
449.01	0.00510268852790492\\
450.01	0.00511323568569115\\
451.01	0.00512378005054773\\
452.01	0.00513431476765311\\
453.01	0.00514483368523116\\
454.01	0.00515533161552699\\
455.01	0.00516580463131715\\
456.01	0.00517625039536382\\
457.01	0.00518666851626712\\
458.01	0.00519706091828341\\
459.01	0.005207432204131\\
460.01	0.00521778997756235\\
461.01	0.00522814507518973\\
462.01	0.00523851163286017\\
463.01	0.00524890687837509\\
464.01	0.00525935049808428\\
465.01	0.00526986336656013\\
466.01	0.00528046390900781\\
467.01	0.00529115632719554\\
468.01	0.00530193845264243\\
469.01	0.00531280807546486\\
470.01	0.00532376301529709\\
471.01	0.00533480115650591\\
472.01	0.00534592048846137\\
473.01	0.00535711915115739\\
474.01	0.00536839548638489\\
475.01	0.00537974809454407\\
476.01	0.00539117589704932\\
477.01	0.00540267820407779\\
478.01	0.00541425478715471\\
479.01	0.00542590595576906\\
480.01	0.00543763263680076\\
481.01	0.00544943645503797\\
482.01	0.00546131981246499\\
483.01	0.00547328596328735\\
484.01	0.0054853390808436\\
485.01	0.00549748431163628\\
486.01	0.00550972781073364\\
487.01	0.00552207675180393\\
488.01	0.00553453930414448\\
489.01	0.00554712456842048\\
490.01	0.00555984246268859\\
491.01	0.00557270355102107\\
492.01	0.00558571880923359\\
493.01	0.00559889932668814\\
494.01	0.00561225595107349\\
495.01	0.00562579889621909\\
496.01	0.00563953735363371\\
497.01	0.00565347917996437\\
498.01	0.00566763086460868\\
499.01	0.00568199845586227\\
500.01	0.00569658835651467\\
501.01	0.00571140738657194\\
502.01	0.00572646278248946\\
503.01	0.00574176219000061\\
504.01	0.0057573136497279\\
505.01	0.0057731255748838\\
506.01	0.00578920672058626\\
507.01	0.00580556614463656\\
508.01	0.0058222131600648\\
509.01	0.00583915728036193\\
510.01	0.00585640815910021\\
511.01	0.00587397552660074\\
512.01	0.00589186912741288\\
513.01	0.00591009866356706\\
514.01	0.00592867374970717\\
515.01	0.00594760388707689\\
516.01	0.00596689846351392\\
517.01	0.00598656678545\\
518.01	0.00600661814441272\\
519.01	0.00602706191228984\\
520.01	0.00604790762593617\\
521.01	0.00606916500494155\\
522.01	0.00609084394021554\\
523.01	0.00611295448121869\\
524.01	0.006135506823636\\
525.01	0.00615851129817362\\
526.01	0.00618197836122347\\
527.01	0.00620591858815658\\
528.01	0.00623034266995729\\
529.01	0.00625526141377142\\
530.01	0.00628068574768584\\
531.01	0.00630662672966783\\
532.01	0.00633309556005505\\
533.01	0.00636010359633393\\
534.01	0.00638766236823543\\
535.01	0.00641578359059092\\
536.01	0.00644447917130381\\
537.01	0.00647376121386944\\
538.01	0.00650364201786405\\
539.01	0.00653413407973057\\
540.01	0.00656525009409123\\
541.01	0.00659700295554078\\
542.01	0.00662940576081271\\
543.01	0.00666247181114391\\
544.01	0.00669621461459394\\
545.01	0.00673064788801325\\
546.01	0.00676578555831429\\
547.01	0.0068016417626947\\
548.01	0.00683823084750839\\
549.01	0.00687556736559398\\
550.01	0.00691366607204055\\
551.01	0.00695254191854659\\
552.01	0.00699221004650432\\
553.01	0.0070326857787142\\
554.01	0.00707398460951927\\
555.01	0.00711612219311656\\
556.01	0.00715911432978177\\
557.01	0.00720297694972537\\
558.01	0.00724772609428095\\
559.01	0.00729337789411719\\
560.01	0.00733994854415049\\
561.01	0.00738745427482568\\
562.01	0.00743591131940887\\
563.01	0.00748533587690427\\
564.01	0.00753574407015443\\
565.01	0.00758715189862288\\
566.01	0.00763957518530753\\
567.01	0.00769302951718983\\
568.01	0.00774753017858247\\
569.01	0.00780309207669819\\
570.01	0.00785972965872284\\
571.01	0.00791745681964229\\
572.01	0.00797628680004186\\
573.01	0.00803623207307779\\
574.01	0.0080973042198116\\
575.01	0.00815951379211108\\
576.01	0.00822287016236105\\
577.01	0.00828738135930615\\
578.01	0.00835305388947689\\
579.01	0.0084198925438449\\
580.01	0.00848790018963276\\
581.01	0.00855707754759355\\
582.01	0.00862742295560682\\
583.01	0.00869893212015025\\
584.01	0.00877159785815303\\
585.01	0.00884540983297857\\
586.01	0.00892035428990249\\
587.01	0.00899641379854713\\
588.01	0.00907356701243149\\
589.01	0.00915178845925224\\
590.01	0.00923104837992613\\
591.01	0.00931131264004169\\
592.01	0.00939254274449733\\
593.01	0.00947469599512985\\
594.01	0.00955772584254327\\
595.01	0.00964158249773262\\
596.01	0.00972621388721412\\
597.01	0.00981156705815533\\
598.01	0.00989759016861823\\
599.01	0.00996919203046376\\
599.02	0.00996973314335037\\
599.03	0.00997027095716939\\
599.04	0.00997080543920335\\
599.05	0.00997133655641379\\
599.06	0.00997186427543808\\
599.07	0.0099723885625862\\
599.08	0.00997290938383756\\
599.09	0.00997342670483771\\
599.1	0.00997394049089505\\
599.11	0.00997445070697751\\
599.12	0.00997495731770918\\
599.13	0.00997546028736696\\
599.14	0.00997595957987706\\
599.15	0.00997645515881165\\
599.16	0.00997694698738526\\
599.17	0.00997743502845131\\
599.18	0.00997791924449856\\
599.19	0.00997839959764748\\
599.2	0.00997887604964662\\
599.21	0.00997934856186899\\
599.22	0.00997981709530829\\
599.23	0.00998028161057521\\
599.24	0.00998074206789365\\
599.25	0.0099811984270969\\
599.26	0.00998165064762378\\
599.27	0.00998209868851477\\
599.28	0.00998254250840806\\
599.29	0.00998298206553559\\
599.3	0.00998341731771905\\
599.31	0.00998384822236584\\
599.32	0.00998427473646497\\
599.33	0.00998469681658292\\
599.34	0.00998511441885952\\
599.35	0.0099855274990037\\
599.36	0.00998593601228926\\
599.37	0.00998633991355056\\
599.38	0.00998673915717821\\
599.39	0.00998713369711469\\
599.4	0.00998752348684992\\
599.41	0.00998790847811157\\
599.42	0.00998828861974491\\
599.43	0.00998866386008804\\
599.44	0.00998903414696681\\
599.45	0.00998939942768985\\
599.46	0.00998975964904344\\
599.47	0.00999011475728636\\
599.48	0.00999046469814472\\
599.49	0.00999080941680669\\
599.5	0.00999114885791725\\
599.51	0.00999148296557278\\
599.52	0.0099918116833157\\
599.53	0.00999213495412901\\
599.54	0.00999245272043077\\
599.55	0.00999276492406855\\
599.56	0.00999307150631381\\
599.57	0.00999337240785624\\
599.58	0.00999366756879799\\
599.59	0.00999395692864795\\
599.6	0.00999424042631585\\
599.61	0.00999451800010642\\
599.62	0.00999478958771336\\
599.63	0.0099950551262134\\
599.64	0.00999531455206019\\
599.65	0.00999556780107816\\
599.66	0.00999581480845634\\
599.67	0.00999605550874211\\
599.68	0.00999628983583487\\
599.69	0.00999651772297967\\
599.7	0.00999673910276076\\
599.71	0.00999695390709509\\
599.72	0.00999716206722577\\
599.73	0.00999736351371538\\
599.74	0.00999755817643933\\
599.75	0.00999774598457904\\
599.76	0.00999792686661517\\
599.77	0.00999810075032068\\
599.78	0.00999826756275386\\
599.79	0.00999842723025133\\
599.8	0.00999857967842092\\
599.81	0.0099987248321345\\
599.82	0.00999886261552071\\
599.83	0.00999899295195769\\
599.84	0.00999911576406567\\
599.85	0.00999923097369951\\
599.86	0.00999933850194117\\
599.87	0.00999943826909211\\
599.88	0.00999953019466558\\
599.89	0.00999961419737891\\
599.9	0.00999969019514566\\
599.91	0.00999975810506767\\
599.92	0.00999981784342713\\
599.93	0.00999986932567848\\
599.94	0.00999991246644028\\
599.95	0.00999994717948697\\
599.96	0.00999997337774056\\
599.97	0.00999999097326228\\
599.98	0.00999999987724406\\
599.99	0.01\\
600	0.01\\
};
\addplot [color=blue!50!mycolor7,solid,forget plot]
  table[row sep=crcr]{%
0.01	0.00384759528872361\\
1.01	0.00384759650035796\\
2.01	0.00384759773699071\\
3.01	0.00384759899913818\\
4.01	0.0038476002873275\\
5.01	0.00384760160209639\\
6.01	0.00384760294399437\\
7.01	0.00384760431358136\\
8.01	0.00384760571142964\\
9.01	0.003847607138123\\
10.01	0.00384760859425731\\
11.01	0.00384761008044081\\
12.01	0.00384761159729444\\
13.01	0.00384761314545166\\
14.01	0.00384761472555935\\
15.01	0.00384761633827756\\
16.01	0.00384761798428016\\
17.01	0.00384761966425487\\
18.01	0.00384762137890358\\
19.01	0.00384762312894275\\
20.01	0.0038476249151037\\
21.01	0.00384762673813283\\
22.01	0.00384762859879202\\
23.01	0.00384763049785899\\
24.01	0.00384763243612715\\
25.01	0.00384763441440672\\
26.01	0.00384763643352468\\
27.01	0.00384763849432476\\
28.01	0.00384764059766865\\
29.01	0.00384764274443537\\
30.01	0.00384764493552242\\
31.01	0.00384764717184573\\
32.01	0.00384764945434041\\
33.01	0.00384765178396063\\
34.01	0.00384765416168033\\
35.01	0.00384765658849384\\
36.01	0.00384765906541576\\
37.01	0.00384766159348192\\
38.01	0.00384766417374949\\
39.01	0.00384766680729738\\
40.01	0.00384766949522697\\
41.01	0.00384767223866228\\
42.01	0.0038476750387509\\
43.01	0.00384767789666381\\
44.01	0.00384768081359639\\
45.01	0.00384768379076875\\
46.01	0.00384768682942599\\
47.01	0.00384768993083938\\
48.01	0.00384769309630635\\
49.01	0.00384769632715087\\
50.01	0.00384769962472466\\
51.01	0.00384770299040733\\
52.01	0.00384770642560689\\
53.01	0.00384770993176066\\
54.01	0.0038477135103356\\
55.01	0.00384771716282899\\
56.01	0.00384772089076922\\
57.01	0.00384772469571612\\
58.01	0.00384772857926173\\
59.01	0.00384773254303135\\
60.01	0.00384773658868366\\
61.01	0.0038477407179118\\
62.01	0.00384774493244371\\
63.01	0.00384774923404323\\
64.01	0.00384775362451083\\
65.01	0.00384775810568398\\
66.01	0.0038477626794382\\
67.01	0.00384776734768782\\
68.01	0.00384777211238706\\
69.01	0.00384777697553007\\
70.01	0.00384778193915269\\
71.01	0.00384778700533263\\
72.01	0.00384779217619059\\
73.01	0.0038477974538913\\
74.01	0.00384780284064408\\
75.01	0.00384780833870411\\
76.01	0.00384781395037312\\
77.01	0.00384781967800038\\
78.01	0.00384782552398412\\
79.01	0.00384783149077164\\
80.01	0.00384783758086119\\
81.01	0.00384784379680263\\
82.01	0.00384785014119883\\
83.01	0.00384785661670597\\
84.01	0.00384786322603557\\
85.01	0.00384786997195508\\
86.01	0.00384787685728921\\
87.01	0.00384788388492122\\
88.01	0.00384789105779376\\
89.01	0.00384789837891045\\
90.01	0.00384790585133714\\
91.01	0.00384791347820311\\
92.01	0.0038479212627022\\
93.01	0.00384792920809417\\
94.01	0.00384793731770647\\
95.01	0.00384794559493538\\
96.01	0.0038479540432473\\
97.01	0.00384796266618023\\
98.01	0.00384797146734549\\
99.01	0.00384798045042906\\
100.01	0.00384798961919297\\
101.01	0.00384799897747733\\
102.01	0.0038480085292014\\
103.01	0.00384801827836574\\
104.01	0.00384802822905329\\
105.01	0.00384803838543156\\
106.01	0.00384804875175446\\
107.01	0.00384805933236342\\
108.01	0.00384807013168999\\
109.01	0.0038480811542573\\
110.01	0.00384809240468159\\
111.01	0.00384810388767497\\
112.01	0.00384811560804697\\
113.01	0.00384812757070631\\
114.01	0.00384813978066323\\
115.01	0.00384815224303175\\
116.01	0.0038481649630312\\
117.01	0.00384817794598918\\
118.01	0.0038481911973432\\
119.01	0.00384820472264338\\
120.01	0.00384821852755422\\
121.01	0.00384823261785741\\
122.01	0.00384824699945434\\
123.01	0.00384826167836812\\
124.01	0.00384827666074634\\
125.01	0.00384829195286389\\
126.01	0.00384830756112499\\
127.01	0.00384832349206626\\
128.01	0.00384833975235986\\
129.01	0.00384835634881517\\
130.01	0.00384837328838253\\
131.01	0.00384839057815589\\
132.01	0.00384840822537588\\
133.01	0.00384842623743248\\
134.01	0.00384844462186852\\
135.01	0.00384846338638264\\
136.01	0.00384848253883255\\
137.01	0.00384850208723819\\
138.01	0.00384852203978511\\
139.01	0.00384854240482807\\
140.01	0.00384856319089436\\
141.01	0.00384858440668753\\
142.01	0.00384860606109089\\
143.01	0.0038486281631708\\
144.01	0.00384865072218134\\
145.01	0.00384867374756762\\
146.01	0.00384869724896983\\
147.01	0.00384872123622711\\
148.01	0.00384874571938182\\
149.01	0.00384877070868388\\
150.01	0.00384879621459475\\
151.01	0.00384882224779173\\
152.01	0.00384884881917291\\
153.01	0.0038488759398613\\
154.01	0.00384890362120956\\
155.01	0.00384893187480476\\
156.01	0.00384896071247302\\
157.01	0.00384899014628526\\
158.01	0.00384902018856088\\
159.01	0.00384905085187411\\
160.01	0.00384908214905842\\
161.01	0.00384911409321263\\
162.01	0.00384914669770575\\
163.01	0.00384917997618305\\
164.01	0.00384921394257109\\
165.01	0.00384924861108442\\
166.01	0.00384928399623073\\
167.01	0.00384932011281745\\
168.01	0.00384935697595744\\
169.01	0.00384939460107557\\
170.01	0.00384943300391552\\
171.01	0.00384947220054573\\
172.01	0.00384951220736654\\
173.01	0.00384955304111667\\
174.01	0.00384959471888087\\
175.01	0.00384963725809606\\
176.01	0.00384968067655957\\
177.01	0.0038497249924361\\
178.01	0.00384977022426548\\
179.01	0.00384981639097029\\
180.01	0.00384986351186402\\
181.01	0.0038499116066588\\
182.01	0.00384996069547404\\
183.01	0.0038500107988446\\
184.01	0.00385006193772938\\
185.01	0.00385011413352032\\
186.01	0.00385016740805139\\
187.01	0.00385022178360732\\
188.01	0.00385027728293341\\
189.01	0.00385033392924493\\
190.01	0.0038503917462368\\
191.01	0.00385045075809383\\
192.01	0.00385051098950047\\
193.01	0.00385057246565138\\
194.01	0.00385063521226227\\
195.01	0.00385069925558031\\
196.01	0.00385076462239561\\
197.01	0.00385083134005224\\
198.01	0.00385089943645991\\
199.01	0.00385096894010575\\
200.01	0.00385103988006619\\
201.01	0.00385111228601959\\
202.01	0.00385118618825844\\
203.01	0.00385126161770273\\
204.01	0.00385133860591234\\
205.01	0.00385141718510092\\
206.01	0.00385149738814972\\
207.01	0.00385157924862121\\
208.01	0.00385166280077365\\
209.01	0.00385174807957529\\
210.01	0.00385183512072013\\
211.01	0.00385192396064234\\
212.01	0.0038520146365324\\
213.01	0.00385210718635255\\
214.01	0.00385220164885365\\
215.01	0.00385229806359161\\
216.01	0.00385239647094437\\
217.01	0.00385249691212945\\
218.01	0.00385259942922155\\
219.01	0.00385270406517098\\
220.01	0.00385281086382266\\
221.01	0.00385291986993412\\
222.01	0.00385303112919588\\
223.01	0.00385314468825143\\
224.01	0.00385326059471697\\
225.01	0.00385337889720312\\
226.01	0.00385349964533548\\
227.01	0.0038536228897771\\
228.01	0.00385374868225075\\
229.01	0.0038538770755614\\
230.01	0.00385400812362009\\
231.01	0.0038541418814681\\
232.01	0.00385427840530114\\
233.01	0.00385441775249454\\
234.01	0.00385455998162938\\
235.01	0.00385470515251875\\
236.01	0.0038548533262347\\
237.01	0.00385500456513629\\
238.01	0.00385515893289778\\
239.01	0.00385531649453771\\
240.01	0.00385547731644899\\
241.01	0.0038556414664296\\
242.01	0.00385580901371367\\
243.01	0.00385598002900422\\
244.01	0.00385615458450575\\
245.01	0.00385633275395872\\
246.01	0.00385651461267368\\
247.01	0.00385670023756761\\
248.01	0.00385688970720062\\
249.01	0.00385708310181328\\
250.01	0.00385728050336587\\
251.01	0.00385748199557781\\
252.01	0.00385768766396863\\
253.01	0.00385789759590026\\
254.01	0.00385811188062017\\
255.01	0.003858330609306\\
256.01	0.0038585538751109\\
257.01	0.00385878177321131\\
258.01	0.00385901440085491\\
259.01	0.00385925185741097\\
260.01	0.00385949424442153\\
261.01	0.00385974166565466\\
262.01	0.00385999422715901\\
263.01	0.00386025203732057\\
264.01	0.00386051520691995\\
265.01	0.00386078384919294\\
266.01	0.00386105807989293\\
267.01	0.00386133801735403\\
268.01	0.00386162378255764\\
269.01	0.00386191549920036\\
270.01	0.00386221329376521\\
271.01	0.00386251729559428\\
272.01	0.00386282763696461\\
273.01	0.00386314445316586\\
274.01	0.00386346788258228\\
275.01	0.00386379806677595\\
276.01	0.0038641351505745\\
277.01	0.00386447928216084\\
278.01	0.00386483061316766\\
279.01	0.0038651892987745\\
280.01	0.00386555549780873\\
281.01	0.00386592937285116\\
282.01	0.0038663110903453\\
283.01	0.00386670082071099\\
284.01	0.00386709873846357\\
285.01	0.00386750502233688\\
286.01	0.00386791985541253\\
287.01	0.00386834342525398\\
288.01	0.00386877592404709\\
289.01	0.00386921754874664\\
290.01	0.00386966850122977\\
291.01	0.00387012898845597\\
292.01	0.0038705992226349\\
293.01	0.00387107942140262\\
294.01	0.00387156980800543\\
295.01	0.003872070611493\\
296.01	0.00387258206692204\\
297.01	0.00387310441556857\\
298.01	0.00387363790515272\\
299.01	0.0038741827900744\\
300.01	0.00387473933166152\\
301.01	0.00387530779843209\\
302.01	0.00387588846636999\\
303.01	0.00387648161921647\\
304.01	0.00387708754877832\\
305.01	0.00387770655525267\\
306.01	0.00387833894757257\\
307.01	0.00387898504377129\\
308.01	0.00387964517136917\\
309.01	0.00388031966778415\\
310.01	0.00388100888076767\\
311.01	0.003881713168868\\
312.01	0.00388243290192311\\
313.01	0.00388316846158542\\
314.01	0.00388392024188122\\
315.01	0.00388468864980734\\
316.01	0.0038854741059684\\
317.01	0.00388627704525726\\
318.01	0.0038870979175842\\
319.01	0.00388793718865625\\
320.01	0.00388879534081335\\
321.01	0.00388967287392496\\
322.01	0.00389057030635291\\
323.01	0.00389148817598656\\
324.01	0.00389242704135562\\
325.01	0.00389338748282962\\
326.01	0.00389437010390933\\
327.01	0.00389537553262136\\
328.01	0.0038964044230234\\
329.01	0.003897457456831\\
330.01	0.00389853534517806\\
331.01	0.00389963883052295\\
332.01	0.00390076868871406\\
333.01	0.00390192573123038\\
334.01	0.00390311080761456\\
335.01	0.00390432480811607\\
336.01	0.00390556866656577\\
337.01	0.00390684336350376\\
338.01	0.0039081499295861\\
339.01	0.00390948944929785\\
340.01	0.00391086306500041\\
341.01	0.00391227198134919\\
342.01	0.00391371747011414\\
343.01	0.00391520087544454\\
344.01	0.00391672361961845\\
345.01	0.00391828720932369\\
346.01	0.00391989324251817\\
347.01	0.00392154341592235\\
348.01	0.00392323953319734\\
349.01	0.00392498351386607\\
350.01	0.00392677740303493\\
351.01	0.00392862338197279\\
352.01	0.00393052377960143\\
353.01	0.00393248108494643\\
354.01	0.00393449796058669\\
355.01	0.00393657725712421\\
356.01	0.00393872202867239\\
357.01	0.00394093554932545\\
358.01	0.00394322133052201\\
359.01	0.00394558313914724\\
360.01	0.00394802501612099\\
361.01	0.00395055129509342\\
362.01	0.00395316662069238\\
363.01	0.00395587596553301\\
364.01	0.00395868464488944\\
365.01	0.00396159832751057\\
366.01	0.00396462304051402\\
367.01	0.00396776516556191\\
368.01	0.00397103142256452\\
369.01	0.00397442883589473\\
370.01	0.00397796467643419\\
371.01	0.00398164637059396\\
372.01	0.00398548136459353\\
373.01	0.00398947692853466\\
374.01	0.0039936398799023\\
375.01	0.00399797619969865\\
376.01	0.00400249050601793\\
377.01	0.00400718533887115\\
378.01	0.0040120601956959\\
379.01	0.00401711023818462\\
380.01	0.00402229707812446\\
381.01	0.00402758779734362\\
382.01	0.00403298441935207\\
383.01	0.00403848901199207\\
384.01	0.00404410368126233\\
385.01	0.00404983057187891\\
386.01	0.00405567186783042\\
387.01	0.0040616297929237\\
388.01	0.00406770661131813\\
389.01	0.00407390462804669\\
390.01	0.00408022618951885\\
391.01	0.00408667368400339\\
392.01	0.00409324954208565\\
393.01	0.00409995623709576\\
394.01	0.00410679628550188\\
395.01	0.0041137722472635\\
396.01	0.00412088672613654\\
397.01	0.00412814236992487\\
398.01	0.00413554187066803\\
399.01	0.00414308796475642\\
400.01	0.00415078343296359\\
401.01	0.00415863110038205\\
402.01	0.00416663383625075\\
403.01	0.00417479455365705\\
404.01	0.00418311620909673\\
405.01	0.00419160180187202\\
406.01	0.00420025437330558\\
407.01	0.0042090770057451\\
408.01	0.00421807282133005\\
409.01	0.00422724498048821\\
410.01	0.00423659668012834\\
411.01	0.00424613115148417\\
412.01	0.00425585165756716\\
413.01	0.00426576149017367\\
414.01	0.00427586396639013\\
415.01	0.00428616242452759\\
416.01	0.00429666021941308\\
417.01	0.00430736071695087\\
418.01	0.00431826728786092\\
419.01	0.00432938330048456\\
420.01	0.00434071211253735\\
421.01	0.00435225706167172\\
422.01	0.00436402145469421\\
423.01	0.00437600855526338\\
424.01	0.00438822156987256\\
425.01	0.00440066363189556\\
426.01	0.00441333778344674\\
427.01	0.00442624695477689\\
428.01	0.00443939394089102\\
429.01	0.00445278137503738\\
430.01	0.00446641169867595\\
431.01	0.00448028712748935\\
432.01	0.00449440961295119\\
433.01	0.00450878079891455\\
434.01	0.00452340197262956\\
435.01	0.00453827400954186\\
436.01	0.00455339731116888\\
437.01	0.00456877173529571\\
438.01	0.00458439651768355\\
439.01	0.00460027018444709\\
440.01	0.00461639045423427\\
441.01	0.00463275412934851\\
442.01	0.00464935697499815\\
443.01	0.00466619358595628\\
444.01	0.00468325724009715\\
445.01	0.00470053973856403\\
446.01	0.00471803123276715\\
447.01	0.00473572003906111\\
448.01	0.00475359244288599\\
449.01	0.0047716324954655\\
450.01	0.00478982180798224\\
451.01	0.00480813935064878\\
452.01	0.00482656126750973\\
453.01	0.00484506072241868\\
454.01	0.00486360779785242\\
455.01	0.00488216947654946\\
456.01	0.00490070974708109\\
457.01	0.00491918988926615\\
458.01	0.00493756901500489\\
459.01	0.00495580496616116\\
460.01	0.00497385570559329\\
461.01	0.00499168138295437\\
462.01	0.00500924731669823\\
463.01	0.00502652820708054\\
464.01	0.0050435138947469\\
465.01	0.00506021730225566\\
466.01	0.00507679117790721\\
467.01	0.00509351189056388\\
468.01	0.00511038239909231\\
469.01	0.00512739407767977\\
470.01	0.00514453748828496\\
471.01	0.00516180235492443\\
472.01	0.00517917754523689\\
473.01	0.00519665106255998\\
474.01	0.00521421005259325\\
475.01	0.00523184082933595\\
476.01	0.0052495289236258\\
477.01	0.00526725915335834\\
478.01	0.00528501571822812\\
479.01	0.00530278232776542\\
480.01	0.00532054236939011\\
481.01	0.00533827912325906\\
482.01	0.00535597603111575\\
483.01	0.00537361702653526\\
484.01	0.00539118693371128\\
485.01	0.00540867194101191\\
486.01	0.00542606015360278\\
487.01	0.00544334222601611\\
488.01	0.00546051206996545\\
489.01	0.00547756762406072\\
490.01	0.00549451165910218\\
491.01	0.00551135257363967\\
492.01	0.00552810510700925\\
493.01	0.0055447908580748\\
494.01	0.00556143844260698\\
495.01	0.005578083045178\\
496.01	0.00559476501949797\\
497.01	0.00561152704485662\\
498.01	0.00562840221187111\\
499.01	0.00564539569062671\\
500.01	0.0056625077707454\\
501.01	0.00567973968740582\\
502.01	0.00569709375686213\\
503.01	0.00571457351239865\\
504.01	0.00573218383614435\\
505.01	0.00574993108078917\\
506.01	0.00576782317370089\\
507.01	0.00578586969427461\\
508.01	0.00580408191360225\\
509.01	0.00582247278388372\\
510.01	0.00584105686366401\\
511.01	0.00585985016435045\\
512.01	0.00587886990413575\\
513.01	0.0058981341583032\\
514.01	0.00591766140122166\\
515.01	0.00593746994702705\\
516.01	0.00595757731575038\\
517.01	0.00597799958337351\\
518.01	0.00599875082345979\\
519.01	0.00601984291558474\\
520.01	0.00604128680950413\\
521.01	0.00606309389102688\\
522.01	0.00608527608585403\\
523.01	0.00610784581405038\\
524.01	0.00613081593090994\\
525.01	0.00615419965464391\\
526.01	0.00617801048227137\\
527.01	0.00620226209636413\\
528.01	0.00622696826687023\\
529.01	0.006252142754069\\
530.01	0.00627779922067693\\
531.01	0.0063039511629907\\
532.01	0.00633061187228706\\
533.01	0.00635779443775675\\
534.01	0.00638551179984718\\
535.01	0.00641377685612876\\
536.01	0.00644260260165693\\
537.01	0.00647200221739751\\
538.01	0.00650198906876377\\
539.01	0.00653257668666739\\
540.01	0.00656377875008843\\
541.01	0.00659560907141205\\
542.01	0.0066280815855853\\
543.01	0.00666121034404836\\
544.01	0.00669500951413639\\
545.01	0.00672949338414971\\
546.01	0.00676467637354784\\
547.01	0.00680057304676548\\
548.01	0.00683719812803159\\
549.01	0.00687456651353929\\
550.01	0.00691269327682495\\
551.01	0.00695159366529483\\
552.01	0.00699128309192888\\
553.01	0.00703177712610352\\
554.01	0.00707309148393236\\
555.01	0.00711524201785119\\
556.01	0.00715824470504815\\
557.01	0.00720211563422044\\
558.01	0.0072468709900294\\
559.01	0.0072925270345521\\
560.01	0.00733910008501349\\
561.01	0.00738660648715271\\
562.01	0.00743506258375307\\
563.01	0.00748448467812454\\
564.01	0.0075348889925328\\
565.01	0.00758629162138968\\
566.01	0.00763870847868745\\
567.01	0.00769215523903229\\
568.01	0.00774664727157911\\
569.01	0.00780219956613101\\
570.01	0.00785882665063752\\
571.01	0.00791654249931098\\
572.01	0.00797536043057803\\
573.01	0.00803529299409064\\
574.01	0.0080963518460355\\
575.01	0.00815854761199431\\
576.01	0.00822188973662608\\
577.01	0.00828638631949463\\
578.01	0.00835204393648558\\
579.01	0.00841886744645257\\
580.01	0.00848685978301172\\
581.01	0.00855602173179519\\
582.01	0.00862635169400304\\
583.01	0.00869784543780366\\
584.01	0.00877049584006834\\
585.01	0.00884429262215488\\
586.01	0.00891922208505276\\
587.01	0.00899526685127645\\
588.01	0.00907240562356496\\
589.01	0.00915061297387856\\
590.01	0.00922985918056657\\
591.01	0.00931011013716259\\
592.01	0.00939132736335054\\
593.01	0.00947346815761936\\
594.01	0.00955648594246635\\
595.01	0.00964033086731408\\
596.01	0.0097249507523202\\
597.01	0.00981029247891265\\
598.01	0.00989630396133466\\
599.01	0.00996919203046332\\
599.02	0.00996973314335002\\
599.03	0.00997027095716909\\
599.04	0.00997080543920308\\
599.05	0.00997133655641355\\
599.06	0.00997186427543785\\
599.07	0.00997238856258599\\
599.08	0.00997290938383737\\
599.09	0.00997342670483753\\
599.1	0.00997394049089489\\
599.11	0.00997445070697736\\
599.12	0.00997495731770904\\
599.13	0.00997546028736683\\
599.14	0.00997595957987695\\
599.15	0.00997645515881154\\
599.16	0.00997694698738516\\
599.17	0.00997743502845122\\
599.18	0.00997791924449848\\
599.19	0.0099783995976474\\
599.2	0.00997887604964655\\
599.21	0.00997934856186892\\
599.22	0.00997981709530823\\
599.23	0.00998028161057515\\
599.24	0.0099807420678936\\
599.25	0.00998119842709685\\
599.26	0.00998165064762374\\
599.27	0.00998209868851473\\
599.28	0.00998254250840802\\
599.29	0.00998298206553556\\
599.3	0.00998341731771902\\
599.31	0.00998384822236582\\
599.32	0.00998427473646495\\
599.33	0.0099846968165829\\
599.34	0.00998511441885951\\
599.35	0.00998552749900369\\
599.36	0.00998593601228924\\
599.37	0.00998633991355054\\
599.38	0.0099867391571782\\
599.39	0.00998713369711469\\
599.4	0.00998752348684992\\
599.41	0.00998790847811156\\
599.42	0.0099882886197449\\
599.43	0.00998866386008803\\
599.44	0.00998903414696681\\
599.45	0.00998939942768985\\
599.46	0.00998975964904344\\
599.47	0.00999011475728636\\
599.48	0.00999046469814471\\
599.49	0.00999080941680669\\
599.5	0.00999114885791725\\
599.51	0.00999148296557278\\
599.52	0.0099918116833157\\
599.53	0.00999213495412901\\
599.54	0.00999245272043077\\
599.55	0.00999276492406855\\
599.56	0.00999307150631381\\
599.57	0.00999337240785624\\
599.58	0.00999366756879799\\
599.59	0.00999395692864795\\
599.6	0.00999424042631585\\
599.61	0.00999451800010642\\
599.62	0.00999478958771336\\
599.63	0.0099950551262134\\
599.64	0.00999531455206019\\
599.65	0.00999556780107816\\
599.66	0.00999581480845634\\
599.67	0.00999605550874211\\
599.68	0.00999628983583487\\
599.69	0.00999651772297967\\
599.7	0.00999673910276076\\
599.71	0.00999695390709509\\
599.72	0.00999716206722577\\
599.73	0.00999736351371538\\
599.74	0.00999755817643933\\
599.75	0.00999774598457904\\
599.76	0.00999792686661517\\
599.77	0.00999810075032068\\
599.78	0.00999826756275386\\
599.79	0.00999842723025133\\
599.8	0.00999857967842092\\
599.81	0.0099987248321345\\
599.82	0.00999886261552071\\
599.83	0.00999899295195769\\
599.84	0.00999911576406567\\
599.85	0.00999923097369951\\
599.86	0.00999933850194117\\
599.87	0.00999943826909211\\
599.88	0.00999953019466558\\
599.89	0.00999961419737891\\
599.9	0.00999969019514566\\
599.91	0.00999975810506767\\
599.92	0.00999981784342713\\
599.93	0.00999986932567848\\
599.94	0.00999991246644028\\
599.95	0.00999994717948697\\
599.96	0.00999997337774056\\
599.97	0.00999999097326228\\
599.98	0.00999999987724406\\
599.99	0.01\\
600	0.01\\
};
\addplot [color=blue!40!mycolor9,solid,forget plot]
  table[row sep=crcr]{%
0.01	0.00367405832583603\\
1.01	0.00367405912102549\\
2.01	0.00367405993263262\\
3.01	0.00367406076099672\\
4.01	0.00367406160646402\\
5.01	0.00367406246938859\\
6.01	0.00367406335013067\\
7.01	0.00367406424905903\\
8.01	0.00367406516654958\\
9.01	0.00367406610298597\\
10.01	0.00367406705875989\\
11.01	0.00367406803427119\\
12.01	0.00367406902992784\\
13.01	0.00367407004614641\\
14.01	0.00367407108335191\\
15.01	0.00367407214197835\\
16.01	0.00367407322246871\\
17.01	0.00367407432527495\\
18.01	0.0036740754508585\\
19.01	0.00367407659969039\\
20.01	0.00367407777225132\\
21.01	0.00367407896903182\\
22.01	0.00367408019053285\\
23.01	0.00367408143726555\\
24.01	0.00367408270975178\\
25.01	0.00367408400852423\\
26.01	0.00367408533412621\\
27.01	0.00367408668711273\\
28.01	0.00367408806805\\
29.01	0.00367408947751607\\
30.01	0.00367409091610117\\
31.01	0.00367409238440748\\
32.01	0.00367409388304958\\
33.01	0.00367409541265495\\
34.01	0.00367409697386401\\
35.01	0.00367409856733053\\
36.01	0.0036741001937215\\
37.01	0.00367410185371803\\
38.01	0.00367410354801519\\
39.01	0.00367410527732256\\
40.01	0.00367410704236429\\
41.01	0.00367410884387957\\
42.01	0.00367411068262281\\
43.01	0.00367411255936412\\
44.01	0.00367411447488969\\
45.01	0.00367411643000158\\
46.01	0.00367411842551899\\
47.01	0.00367412046227759\\
48.01	0.00367412254113066\\
49.01	0.00367412466294911\\
50.01	0.00367412682862167\\
51.01	0.0036741290390556\\
52.01	0.0036741312951771\\
53.01	0.00367413359793124\\
54.01	0.00367413594828284\\
55.01	0.00367413834721683\\
56.01	0.00367414079573823\\
57.01	0.00367414329487306\\
58.01	0.00367414584566859\\
59.01	0.00367414844919367\\
60.01	0.00367415110653928\\
61.01	0.0036741538188189\\
62.01	0.00367415658716955\\
63.01	0.00367415941275123\\
64.01	0.00367416229674802\\
65.01	0.00367416524036898\\
66.01	0.0036741682448477\\
67.01	0.0036741713114434\\
68.01	0.00367417444144151\\
69.01	0.00367417763615423\\
70.01	0.00367418089692056\\
71.01	0.00367418422510739\\
72.01	0.00367418762210989\\
73.01	0.00367419108935223\\
74.01	0.00367419462828794\\
75.01	0.00367419824040066\\
76.01	0.0036742019272047\\
77.01	0.00367420569024587\\
78.01	0.00367420953110184\\
79.01	0.00367421345138316\\
80.01	0.00367421745273361\\
81.01	0.00367422153683078\\
82.01	0.00367422570538732\\
83.01	0.00367422996015113\\
84.01	0.00367423430290657\\
85.01	0.00367423873547459\\
86.01	0.00367424325971415\\
87.01	0.00367424787752242\\
88.01	0.00367425259083611\\
89.01	0.00367425740163188\\
90.01	0.00367426231192744\\
91.01	0.00367426732378213\\
92.01	0.00367427243929801\\
93.01	0.0036742776606208\\
94.01	0.00367428298994044\\
95.01	0.00367428842949226\\
96.01	0.00367429398155808\\
97.01	0.00367429964846672\\
98.01	0.00367430543259524\\
99.01	0.0036743113363702\\
100.01	0.00367431736226802\\
101.01	0.00367432351281664\\
102.01	0.0036743297905963\\
103.01	0.00367433619824085\\
104.01	0.00367434273843835\\
105.01	0.00367434941393292\\
106.01	0.00367435622752513\\
107.01	0.00367436318207376\\
108.01	0.00367437028049679\\
109.01	0.00367437752577268\\
110.01	0.00367438492094152\\
111.01	0.00367439246910635\\
112.01	0.00367440017343455\\
113.01	0.00367440803715885\\
114.01	0.00367441606357934\\
115.01	0.00367442425606429\\
116.01	0.00367443261805184\\
117.01	0.00367444115305116\\
118.01	0.00367444986464413\\
119.01	0.0036744587564871\\
120.01	0.00367446783231197\\
121.01	0.00367447709592808\\
122.01	0.00367448655122356\\
123.01	0.00367449620216714\\
124.01	0.00367450605281003\\
125.01	0.00367451610728699\\
126.01	0.00367452636981899\\
127.01	0.00367453684471392\\
128.01	0.0036745475363691\\
129.01	0.00367455844927339\\
130.01	0.00367456958800837\\
131.01	0.00367458095725051\\
132.01	0.00367459256177353\\
133.01	0.00367460440644998\\
134.01	0.00367461649625347\\
135.01	0.00367462883626063\\
136.01	0.00367464143165344\\
137.01	0.00367465428772142\\
138.01	0.0036746674098637\\
139.01	0.00367468080359164\\
140.01	0.00367469447453068\\
141.01	0.00367470842842302\\
142.01	0.00367472267113002\\
143.01	0.00367473720863488\\
144.01	0.00367475204704466\\
145.01	0.00367476719259334\\
146.01	0.00367478265164431\\
147.01	0.0036747984306931\\
148.01	0.00367481453637007\\
149.01	0.00367483097544331\\
150.01	0.00367484775482131\\
151.01	0.00367486488155649\\
152.01	0.00367488236284745\\
153.01	0.0036749002060424\\
154.01	0.0036749184186423\\
155.01	0.00367493700830412\\
156.01	0.00367495598284417\\
157.01	0.00367497535024085\\
158.01	0.00367499511863893\\
159.01	0.00367501529635234\\
160.01	0.00367503589186823\\
161.01	0.00367505691385027\\
162.01	0.00367507837114245\\
163.01	0.0036751002727728\\
164.01	0.00367512262795758\\
165.01	0.003675145446105\\
166.01	0.00367516873681918\\
167.01	0.0036751925099046\\
168.01	0.00367521677536986\\
169.01	0.00367524154343279\\
170.01	0.00367526682452371\\
171.01	0.00367529262929117\\
172.01	0.00367531896860572\\
173.01	0.00367534585356497\\
174.01	0.00367537329549816\\
175.01	0.00367540130597163\\
176.01	0.00367542989679323\\
177.01	0.00367545908001773\\
178.01	0.00367548886795219\\
179.01	0.00367551927316123\\
180.01	0.00367555030847222\\
181.01	0.00367558198698174\\
182.01	0.00367561432206047\\
183.01	0.00367564732735948\\
184.01	0.00367568101681631\\
185.01	0.00367571540466079\\
186.01	0.00367575050542167\\
187.01	0.00367578633393296\\
188.01	0.00367582290534023\\
189.01	0.00367586023510806\\
190.01	0.00367589833902585\\
191.01	0.00367593723321591\\
192.01	0.00367597693414015\\
193.01	0.00367601745860749\\
194.01	0.00367605882378159\\
195.01	0.00367610104718842\\
196.01	0.00367614414672417\\
197.01	0.00367618814066356\\
198.01	0.00367623304766794\\
199.01	0.00367627888679362\\
200.01	0.00367632567750083\\
201.01	0.00367637343966273\\
202.01	0.00367642219357428\\
203.01	0.0036764719599615\\
204.01	0.00367652275999129\\
205.01	0.00367657461528113\\
206.01	0.00367662754790893\\
207.01	0.00367668158042372\\
208.01	0.00367673673585571\\
209.01	0.00367679303772729\\
210.01	0.00367685051006431\\
211.01	0.00367690917740703\\
212.01	0.00367696906482211\\
213.01	0.00367703019791429\\
214.01	0.00367709260283866\\
215.01	0.00367715630631309\\
216.01	0.00367722133563136\\
217.01	0.00367728771867616\\
218.01	0.00367735548393265\\
219.01	0.00367742466050223\\
220.01	0.00367749527811696\\
221.01	0.00367756736715395\\
222.01	0.00367764095865095\\
223.01	0.00367771608432077\\
224.01	0.00367779277656807\\
225.01	0.00367787106850474\\
226.01	0.00367795099396734\\
227.01	0.00367803258753387\\
228.01	0.00367811588454109\\
229.01	0.00367820092110299\\
230.01	0.00367828773412955\\
231.01	0.00367837636134524\\
232.01	0.00367846684130912\\
233.01	0.00367855921343473\\
234.01	0.00367865351801115\\
235.01	0.00367874979622426\\
236.01	0.0036788480901784\\
237.01	0.00367894844291927\\
238.01	0.00367905089845701\\
239.01	0.0036791555017904\\
240.01	0.00367926229893116\\
241.01	0.0036793713369296\\
242.01	0.00367948266390042\\
243.01	0.00367959632904982\\
244.01	0.00367971238270315\\
245.01	0.00367983087633339\\
246.01	0.00367995186259069\\
247.01	0.00368007539533263\\
248.01	0.00368020152965538\\
249.01	0.00368033032192623\\
250.01	0.00368046182981627\\
251.01	0.00368059611233512\\
252.01	0.00368073322986634\\
253.01	0.00368087324420359\\
254.01	0.0036810162185886\\
255.01	0.00368116221774978\\
256.01	0.00368131130794246\\
257.01	0.00368146355699048\\
258.01	0.00368161903432902\\
259.01	0.00368177781104863\\
260.01	0.00368193995994104\\
261.01	0.00368210555554632\\
262.01	0.00368227467420189\\
263.01	0.00368244739409282\\
264.01	0.00368262379530425\\
265.01	0.00368280395987521\\
266.01	0.00368298797185448\\
267.01	0.00368317591735849\\
268.01	0.00368336788463104\\
269.01	0.00368356396410541\\
270.01	0.00368376424846817\\
271.01	0.00368396883272569\\
272.01	0.00368417781427288\\
273.01	0.00368439129296444\\
274.01	0.00368460937118845\\
275.01	0.00368483215394291\\
276.01	0.003685059748915\\
277.01	0.00368529226656313\\
278.01	0.00368552982020205\\
279.01	0.00368577252609094\\
280.01	0.00368602050352533\\
281.01	0.00368627387493163\\
282.01	0.00368653276596563\\
283.01	0.003686797305615\\
284.01	0.00368706762630486\\
285.01	0.00368734386400814\\
286.01	0.00368762615835948\\
287.01	0.00368791465277396\\
288.01	0.00368820949456992\\
289.01	0.00368851083509693\\
290.01	0.00368881882986823\\
291.01	0.00368913363869844\\
292.01	0.003689455425847\\
293.01	0.00368978436016636\\
294.01	0.00369012061525605\\
295.01	0.00369046436962391\\
296.01	0.00369081580685172\\
297.01	0.00369117511576859\\
298.01	0.00369154249063074\\
299.01	0.00369191813130812\\
300.01	0.00369230224347841\\
301.01	0.00369269503882901\\
302.01	0.00369309673526556\\
303.01	0.00369350755713052\\
304.01	0.00369392773542858\\
305.01	0.00369435750806205\\
306.01	0.00369479712007419\\
307.01	0.00369524682390303\\
308.01	0.00369570687964417\\
309.01	0.00369617755532457\\
310.01	0.00369665912718591\\
311.01	0.00369715187997963\\
312.01	0.00369765610727224\\
313.01	0.0036981721117632\\
314.01	0.00369870020561382\\
315.01	0.00369924071078917\\
316.01	0.003699793959412\\
317.01	0.00370036029413032\\
318.01	0.0037009400684974\\
319.01	0.00370153364736611\\
320.01	0.0037021414072972\\
321.01	0.00370276373698146\\
322.01	0.00370340103767629\\
323.01	0.00370405372365683\\
324.01	0.00370472222268241\\
325.01	0.00370540697647703\\
326.01	0.00370610844122576\\
327.01	0.00370682708808467\\
328.01	0.00370756340370593\\
329.01	0.00370831789077751\\
330.01	0.00370909106857538\\
331.01	0.00370988347352987\\
332.01	0.00371069565980338\\
333.01	0.00371152819987965\\
334.01	0.0037123816851611\\
335.01	0.00371325672657484\\
336.01	0.00371415395518282\\
337.01	0.00371507402279454\\
338.01	0.00371601760257771\\
339.01	0.00371698538966353\\
340.01	0.00371797810174047\\
341.01	0.0037189964796306\\
342.01	0.00372004128784083\\
343.01	0.00372111331508027\\
344.01	0.00372221337473191\\
345.01	0.00372334230526716\\
346.01	0.00372450097058653\\
347.01	0.0037256902602693\\
348.01	0.00372691108970988\\
349.01	0.00372816440011696\\
350.01	0.00372945115834382\\
351.01	0.00373077235651685\\
352.01	0.00373212901142059\\
353.01	0.00373352216359135\\
354.01	0.00373495287606578\\
355.01	0.0037364222327198\\
356.01	0.00373793133612502\\
357.01	0.00373948130483995\\
358.01	0.00374107327004145\\
359.01	0.00374270837139101\\
360.01	0.00374438775202041\\
361.01	0.00374611255250908\\
362.01	0.00374788390372034\\
363.01	0.00374970291835895\\
364.01	0.00375157068111615\\
365.01	0.00375348823728412\\
366.01	0.00375545657975034\\
367.01	0.00375747663434159\\
368.01	0.00375954924357592\\
369.01	0.0037616751490212\\
370.01	0.00376385497267603\\
371.01	0.003766089198096\\
372.01	0.00376837815244509\\
373.01	0.00377072199129273\\
374.01	0.00377312068888773\\
375.01	0.00377557403791379\\
376.01	0.00377808166449803\\
377.01	0.00378064306669181\\
378.01	0.00378325768800604\\
379.01	0.00378592504219442\\
380.01	0.00378864505725178\\
381.01	0.00379141857738417\\
382.01	0.00379424667125867\\
383.01	0.00379713043234045\\
384.01	0.00380007097979201\\
385.01	0.00380306945942864\\
386.01	0.00380612704473182\\
387.01	0.00380924493792692\\
388.01	0.00381242437112994\\
389.01	0.00381566660756871\\
390.01	0.00381897294288394\\
391.01	0.00382234470651839\\
392.01	0.0038257832631996\\
393.01	0.00382929001452471\\
394.01	0.00383286640065623\\
395.01	0.00383651390213738\\
396.01	0.00384023404183742\\
397.01	0.0038440283870385\\
398.01	0.00384789855167558\\
399.01	0.00385184619874275\\
400.01	0.00385587304288089\\
401.01	0.00385998085316258\\
402.01	0.00386417145609101\\
403.01	0.00386844673883276\\
404.01	0.00387280865270531\\
405.01	0.00387725921694236\\
406.01	0.00388180052276163\\
407.01	0.00388643473776389\\
408.01	0.00389116411069319\\
409.01	0.00389599097659257\\
410.01	0.00390091776239113\\
411.01	0.00390594699296457\\
412.01	0.00391108129771316\\
413.01	0.0039163234177077\\
414.01	0.00392167621345645\\
415.01	0.00392714267335492\\
416.01	0.00393272592288359\\
417.01	0.0039384292346282\\
418.01	0.00394425603920165\\
419.01	0.00395020993715844\\
420.01	0.00395629471199882\\
421.01	0.00396251434437153\\
422.01	0.00396887302759433\\
423.01	0.00397537518462468\\
424.01	0.00398202548662442\\
425.01	0.00398882887327795\\
426.01	0.00399579057503845\\
427.01	0.00400291613749372\\
428.01	0.00401021144805891\\
429.01	0.00401768276522525\\
430.01	0.00402533675060855\\
431.01	0.00403318050406516\\
432.01	0.00404122160215854\\
433.01	0.00404946814027847\\
434.01	0.00405792877873196\\
435.01	0.00406661279313371\\
436.01	0.00407553012943204\\
437.01	0.0040846914639002\\
438.01	0.0040941082684095\\
439.01	0.00410379288126253\\
440.01	0.00411375858380633\\
441.01	0.00412401968295002\\
442.01	0.00413459159956747\\
443.01	0.00414549096256252\\
444.01	0.00415673570807933\\
445.01	0.00416834518293846\\
446.01	0.00418034025082298\\
447.01	0.00419274339898484\\
448.01	0.00420557884222234\\
449.01	0.00421887261951838\\
450.01	0.00423265267690792\\
451.01	0.00424694892772715\\
452.01	0.00426179327819508\\
453.01	0.00427721960205298\\
454.01	0.00429326364241243\\
455.01	0.00430996281163371\\
456.01	0.00432735585042339\\
457.01	0.00434548229469846\\
458.01	0.00436438168218982\\
459.01	0.00438409240903823\\
460.01	0.00440465011819025\\
461.01	0.00442608546416438\\
462.01	0.00444842105000842\\
463.01	0.00447166726820816\\
464.01	0.00449581668842845\\
465.01	0.00452083653810631\\
466.01	0.00454655444585295\\
467.01	0.00457268258452102\\
468.01	0.00459920934856123\\
469.01	0.00462613306159242\\
470.01	0.00465345098640395\\
471.01	0.0046811591223869\\
472.01	0.0047092519385362\\
473.01	0.00473772202655891\\
474.01	0.00476655965321753\\
475.01	0.00479575222489502\\
476.01	0.00482528394100766\\
477.01	0.00485513585982052\\
478.01	0.00488528566086753\\
479.01	0.0049157072618335\\
480.01	0.00494637041136256\\
481.01	0.00497724026453776\\
482.01	0.0050082769508859\\
483.01	0.00503943514996553\\
484.01	0.00507066369672383\\
485.01	0.00510190524850224\\
486.01	0.0051330960586651\\
487.01	0.00516416591944975\\
488.01	0.00519503836025188\\
489.01	0.00522563121910394\\
490.01	0.00525585774707416\\
491.01	0.00528562846197784\\
492.01	0.00531485405192902\\
493.01	0.00534344970529017\\
494.01	0.00537134142467356\\
495.01	0.00539847497274243\\
496.01	0.00542482833968094\\
497.01	0.00545042992074787\\
498.01	0.00547574300765019\\
499.01	0.00550109153729324\\
500.01	0.00552645060857317\\
501.01	0.00555179473582924\\
502.01	0.00557709816769944\\
503.01	0.00560233528928952\\
504.01	0.00562748112029061\\
505.01	0.00565251192163366\\
506.01	0.00567740592120446\\
507.01	0.00570214416678325\\
508.01	0.00572671150974177\\
509.01	0.0057510977150463\\
510.01	0.00577529868053916\\
511.01	0.00579931772940393\\
512.01	0.00582316691157837\\
513.01	0.0058468682091317\\
514.01	0.00587045448251295\\
515.01	0.00589396991263982\\
516.01	0.00591746957905275\\
517.01	0.00594101765388125\\
518.01	0.00596468346858382\\
519.01	0.00598852627082647\\
520.01	0.00601256288541119\\
521.01	0.00603680034606036\\
522.01	0.00606124780133964\\
523.01	0.00608591668244098\\
524.01	0.00611082082606813\\
525.01	0.00613597653886037\\
526.01	0.00616140258431862\\
527.01	0.00618712007040617\\
528.01	0.00621315221465982\\
529.01	0.00623952396483905\\
530.01	0.00626626145807431\\
531.01	0.00629339131242648\\
532.01	0.00632093976509019\\
533.01	0.00634893170585903\\
534.01	0.00637738970975967\\
535.01	0.00640633326085757\\
536.01	0.00643577895491486\\
537.01	0.00646574287119041\\
538.01	0.00649624165212671\\
539.01	0.00652729244507207\\
540.01	0.0065589127943315\\
541.01	0.0065911205186732\\
542.01	0.00662393357819652\\
543.01	0.00665736993624814\\
544.01	0.00669144742622071\\
545.01	0.00672618363750042\\
546.01	0.00676159583667395\\
547.01	0.00679770094139254\\
548.01	0.00683451556229975\\
549.01	0.00687205612044749\\
550.01	0.00691033902574197\\
551.01	0.00694938081321948\\
552.01	0.00698919814651896\\
553.01	0.00702980778368401\\
554.01	0.0070712265449191\\
555.01	0.00711347128465088\\
556.01	0.00715655886917984\\
557.01	0.00720050616072122\\
558.01	0.0072453300078212\\
559.01	0.00729104724095515\\
560.01	0.00733767467060229\\
561.01	0.00738522908340952\\
562.01	0.00743372723060418\\
563.01	0.00748318580258283\\
564.01	0.00753362138874788\\
565.01	0.00758505042934002\\
566.01	0.00763748916253181\\
567.01	0.00769095356644288\\
568.01	0.00774545929526543\\
569.01	0.00780102160845901\\
570.01	0.0078576552917938\\
571.01	0.00791537456890428\\
572.01	0.00797419300200297\\
573.01	0.00803412338054319\\
574.01	0.00809517759694714\\
575.01	0.00815736650899904\\
576.01	0.00822069978879914\\
577.01	0.00828518575791731\\
578.01	0.00835083120818599\\
579.01	0.00841764120771106\\
580.01	0.00848561889195915\\
581.01	0.00855476524018159\\
582.01	0.00862507883799715\\
583.01	0.00869655562770745\\
584.01	0.00876918864890651\\
585.01	0.0088429677732069\\
586.01	0.00891787943850125\\
587.01	0.00899390639018665\\
588.01	0.00907102743936527\\
589.01	0.00914921725139712\\
590.01	0.00922844618250661\\
591.01	0.0093086801876622\\
592.01	0.00938988082996722\\
593.01	0.00947200543070414\\
594.01	0.00955500741044542\\
595.01	0.00963883688587809\\
596.01	0.00972344160493188\\
597.01	0.00980876832537324\\
598.01	0.00989476477037853\\
599.01	0.00996919203040513\\
599.02	0.00996973314331371\\
599.03	0.00997027095714612\\
599.04	0.00997080543918755\\
599.05	0.00997133655640176\\
599.06	0.00997186427542786\\
599.07	0.00997238856257705\\
599.08	0.00997290938382928\\
599.09	0.00997342670483015\\
599.1	0.00997394049088811\\
599.11	0.0099744507069711\\
599.12	0.00997495731770325\\
599.13	0.00997546028736146\\
599.14	0.00997595957987197\\
599.15	0.00997645515880693\\
599.16	0.00997694698738088\\
599.17	0.00997743502844726\\
599.18	0.00997791924449481\\
599.19	0.00997839959764401\\
599.2	0.00997887604964342\\
599.21	0.00997934856186603\\
599.22	0.00997981709530556\\
599.23	0.0099802816105727\\
599.24	0.00998074206789134\\
599.25	0.00998119842709478\\
599.26	0.00998165064762184\\
599.27	0.00998209868851299\\
599.28	0.00998254250840643\\
599.29	0.00998298206553411\\
599.3	0.0099834173177177\\
599.31	0.00998384822236462\\
599.32	0.00998427473646385\\
599.33	0.00998469681658191\\
599.34	0.00998511441885861\\
599.35	0.00998552749900288\\
599.36	0.00998593601228852\\
599.37	0.00998633991354989\\
599.38	0.00998673915717762\\
599.39	0.00998713369711416\\
599.4	0.00998752348684945\\
599.41	0.00998790847811114\\
599.42	0.00998828861974453\\
599.43	0.0099886638600877\\
599.44	0.00998903414696652\\
599.45	0.0099893994276896\\
599.46	0.00998975964904322\\
599.47	0.00999011475728616\\
599.48	0.00999046469814455\\
599.49	0.00999080941680655\\
599.5	0.00999114885791712\\
599.51	0.00999148296557267\\
599.52	0.00999181168331561\\
599.53	0.00999213495412893\\
599.54	0.0099924527204307\\
599.55	0.00999276492406849\\
599.56	0.00999307150631377\\
599.57	0.0099933724078562\\
599.58	0.00999366756879796\\
599.59	0.00999395692864792\\
599.6	0.00999424042631583\\
599.61	0.0099945180001064\\
599.62	0.00999478958771334\\
599.63	0.00999505512621339\\
599.64	0.00999531455206018\\
599.65	0.00999556780107815\\
599.66	0.00999581480845634\\
599.67	0.00999605550874211\\
599.68	0.00999628983583487\\
599.69	0.00999651772297967\\
599.7	0.00999673910276075\\
599.71	0.00999695390709509\\
599.72	0.00999716206722577\\
599.73	0.00999736351371538\\
599.74	0.00999755817643933\\
599.75	0.00999774598457904\\
599.76	0.00999792686661517\\
599.77	0.00999810075032068\\
599.78	0.00999826756275386\\
599.79	0.00999842723025133\\
599.8	0.00999857967842092\\
599.81	0.0099987248321345\\
599.82	0.00999886261552071\\
599.83	0.00999899295195769\\
599.84	0.00999911576406567\\
599.85	0.00999923097369951\\
599.86	0.00999933850194117\\
599.87	0.00999943826909211\\
599.88	0.00999953019466558\\
599.89	0.00999961419737891\\
599.9	0.00999969019514566\\
599.91	0.00999975810506767\\
599.92	0.00999981784342713\\
599.93	0.00999986932567848\\
599.94	0.00999991246644028\\
599.95	0.00999994717948697\\
599.96	0.00999997337774056\\
599.97	0.00999999097326228\\
599.98	0.00999999987724406\\
599.99	0.01\\
600	0.01\\
};
\addplot [color=blue!75!mycolor7,solid,forget plot]
  table[row sep=crcr]{%
0.01	0.00346027296821295\\
1.01	0.00346027362420681\\
2.01	0.00346027429381217\\
3.01	0.0034602749773122\\
4.01	0.00346027567499618\\
5.01	0.0034602763871589\\
6.01	0.00346027711410207\\
7.01	0.00346027785613308\\
8.01	0.00346027861356599\\
9.01	0.00346027938672151\\
10.01	0.00346028017592695\\
11.01	0.00346028098151639\\
12.01	0.00346028180383087\\
13.01	0.00346028264321883\\
14.01	0.00346028350003563\\
15.01	0.00346028437464441\\
16.01	0.00346028526741551\\
17.01	0.00346028617872732\\
18.01	0.00346028710896605\\
19.01	0.0034602880585259\\
20.01	0.00346028902780944\\
21.01	0.00346029001722768\\
22.01	0.00346029102720004\\
23.01	0.00346029205815479\\
24.01	0.00346029311052924\\
25.01	0.00346029418476959\\
26.01	0.00346029528133186\\
27.01	0.00346029640068117\\
28.01	0.00346029754329262\\
29.01	0.00346029870965113\\
30.01	0.00346029990025172\\
31.01	0.00346030111559985\\
32.01	0.00346030235621163\\
33.01	0.00346030362261389\\
34.01	0.00346030491534453\\
35.01	0.00346030623495253\\
36.01	0.00346030758199867\\
37.01	0.00346030895705529\\
38.01	0.00346031036070667\\
39.01	0.00346031179354944\\
40.01	0.00346031325619285\\
41.01	0.00346031474925859\\
42.01	0.00346031627338167\\
43.01	0.00346031782921026\\
44.01	0.00346031941740603\\
45.01	0.00346032103864478\\
46.01	0.00346032269361626\\
47.01	0.00346032438302477\\
48.01	0.00346032610758935\\
49.01	0.00346032786804411\\
50.01	0.0034603296651386\\
51.01	0.00346033149963806\\
52.01	0.00346033337232378\\
53.01	0.00346033528399348\\
54.01	0.00346033723546163\\
55.01	0.00346033922755966\\
56.01	0.00346034126113662\\
57.01	0.00346034333705923\\
58.01	0.00346034545621254\\
59.01	0.00346034761950002\\
60.01	0.00346034982784436\\
61.01	0.00346035208218756\\
62.01	0.0034603543834911\\
63.01	0.00346035673273698\\
64.01	0.00346035913092776\\
65.01	0.00346036157908705\\
66.01	0.00346036407826\\
67.01	0.00346036662951375\\
68.01	0.00346036923393772\\
69.01	0.00346037189264434\\
70.01	0.00346037460676944\\
71.01	0.00346037737747276\\
72.01	0.00346038020593843\\
73.01	0.00346038309337537\\
74.01	0.00346038604101807\\
75.01	0.00346038905012697\\
76.01	0.00346039212198898\\
77.01	0.00346039525791822\\
78.01	0.00346039845925639\\
79.01	0.00346040172737334\\
80.01	0.00346040506366795\\
81.01	0.00346040846956863\\
82.01	0.0034604119465335\\
83.01	0.00346041549605182\\
84.01	0.00346041911964391\\
85.01	0.00346042281886236\\
86.01	0.00346042659529237\\
87.01	0.00346043045055275\\
88.01	0.00346043438629623\\
89.01	0.00346043840421047\\
90.01	0.00346044250601864\\
91.01	0.00346044669348052\\
92.01	0.00346045096839281\\
93.01	0.00346045533259015\\
94.01	0.00346045978794601\\
95.01	0.00346046433637333\\
96.01	0.00346046897982541\\
97.01	0.00346047372029699\\
98.01	0.0034604785598249\\
99.01	0.0034604835004888\\
100.01	0.00346048854441262\\
101.01	0.003460493693765\\
102.01	0.00346049895076038\\
103.01	0.00346050431766014\\
104.01	0.00346050979677362\\
105.01	0.00346051539045875\\
106.01	0.00346052110112362\\
107.01	0.0034605269312271\\
108.01	0.00346053288328019\\
109.01	0.00346053895984705\\
110.01	0.00346054516354626\\
111.01	0.00346055149705174\\
112.01	0.00346055796309411\\
113.01	0.00346056456446209\\
114.01	0.00346057130400326\\
115.01	0.00346057818462554\\
116.01	0.00346058520929884\\
117.01	0.00346059238105594\\
118.01	0.00346059970299396\\
119.01	0.00346060717827577\\
120.01	0.00346061481013159\\
121.01	0.00346062260185992\\
122.01	0.0034606305568297\\
123.01	0.00346063867848137\\
124.01	0.0034606469703283\\
125.01	0.00346065543595889\\
126.01	0.00346066407903767\\
127.01	0.00346067290330737\\
128.01	0.00346068191259023\\
129.01	0.0034606911107898\\
130.01	0.00346070050189307\\
131.01	0.00346071008997172\\
132.01	0.00346071987918425\\
133.01	0.00346072987377806\\
134.01	0.00346074007809076\\
135.01	0.00346075049655294\\
136.01	0.00346076113368944\\
137.01	0.00346077199412187\\
138.01	0.00346078308257053\\
139.01	0.00346079440385637\\
140.01	0.00346080596290387\\
141.01	0.00346081776474244\\
142.01	0.00346082981450914\\
143.01	0.00346084211745108\\
144.01	0.00346085467892764\\
145.01	0.00346086750441305\\
146.01	0.00346088059949883\\
147.01	0.00346089396989629\\
148.01	0.00346090762143927\\
149.01	0.0034609215600868\\
150.01	0.00346093579192595\\
151.01	0.00346095032317421\\
152.01	0.00346096516018295\\
153.01	0.00346098030944\\
154.01	0.00346099577757264\\
155.01	0.00346101157135083\\
156.01	0.00346102769769017\\
157.01	0.00346104416365536\\
158.01	0.0034610609764632\\
159.01	0.00346107814348626\\
160.01	0.003461095672256\\
161.01	0.00346111357046652\\
162.01	0.0034611318459782\\
163.01	0.00346115050682108\\
164.01	0.00346116956119898\\
165.01	0.00346118901749312\\
166.01	0.00346120888426622\\
167.01	0.00346122917026641\\
168.01	0.0034612498844316\\
169.01	0.00346127103589311\\
170.01	0.00346129263398104\\
171.01	0.00346131468822759\\
172.01	0.00346133720837241\\
173.01	0.00346136020436684\\
174.01	0.00346138368637866\\
175.01	0.00346140766479729\\
176.01	0.00346143215023823\\
177.01	0.00346145715354887\\
178.01	0.00346148268581301\\
179.01	0.00346150875835664\\
180.01	0.00346153538275346\\
181.01	0.00346156257082997\\
182.01	0.00346159033467176\\
183.01	0.00346161868662928\\
184.01	0.00346164763932366\\
185.01	0.00346167720565309\\
186.01	0.00346170739879893\\
187.01	0.00346173823223225\\
188.01	0.00346176971972081\\
189.01	0.00346180187533525\\
190.01	0.0034618347134566\\
191.01	0.00346186824878295\\
192.01	0.00346190249633714\\
193.01	0.00346193747147391\\
194.01	0.00346197318988783\\
195.01	0.00346200966762107\\
196.01	0.00346204692107138\\
197.01	0.00346208496700048\\
198.01	0.00346212382254231\\
199.01	0.00346216350521213\\
200.01	0.00346220403291492\\
201.01	0.00346224542395469\\
202.01	0.00346228769704397\\
203.01	0.00346233087131317\\
204.01	0.00346237496632039\\
205.01	0.00346242000206184\\
206.01	0.00346246599898185\\
207.01	0.00346251297798326\\
208.01	0.00346256096043894\\
209.01	0.00346260996820243\\
210.01	0.00346266002361927\\
211.01	0.00346271114953928\\
212.01	0.00346276336932792\\
213.01	0.00346281670687917\\
214.01	0.00346287118662797\\
215.01	0.00346292683356303\\
216.01	0.00346298367324049\\
217.01	0.00346304173179739\\
218.01	0.00346310103596568\\
219.01	0.00346316161308691\\
220.01	0.0034632234911264\\
221.01	0.00346328669868928\\
222.01	0.00346335126503503\\
223.01	0.00346341722009433\\
224.01	0.00346348459448502\\
225.01	0.00346355341952927\\
226.01	0.00346362372727028\\
227.01	0.00346369555049068\\
228.01	0.0034637689227305\\
229.01	0.00346384387830587\\
230.01	0.00346392045232814\\
231.01	0.00346399868072374\\
232.01	0.00346407860025449\\
233.01	0.00346416024853842\\
234.01	0.00346424366407096\\
235.01	0.00346432888624702\\
236.01	0.00346441595538365\\
237.01	0.00346450491274311\\
238.01	0.00346459580055654\\
239.01	0.00346468866204866\\
240.01	0.00346478354146264\\
241.01	0.00346488048408608\\
242.01	0.0034649795362774\\
243.01	0.00346508074549308\\
244.01	0.00346518416031558\\
245.01	0.00346528983048211\\
246.01	0.00346539780691429\\
247.01	0.00346550814174827\\
248.01	0.00346562088836583\\
249.01	0.00346573610142654\\
250.01	0.00346585383690075\\
251.01	0.00346597415210312\\
252.01	0.0034660971057273\\
253.01	0.0034662227578818\\
254.01	0.00346635117012642\\
255.01	0.00346648240550983\\
256.01	0.00346661652860854\\
257.01	0.00346675360556607\\
258.01	0.00346689370413411\\
259.01	0.00346703689371411\\
260.01	0.00346718324540039\\
261.01	0.0034673328320244\\
262.01	0.00346748572819977\\
263.01	0.00346764201036888\\
264.01	0.00346780175685081\\
265.01	0.00346796504789028\\
266.01	0.00346813196570774\\
267.01	0.00346830259455144\\
268.01	0.00346847702075015\\
269.01	0.00346865533276736\\
270.01	0.00346883762125744\\
271.01	0.00346902397912266\\
272.01	0.00346921450157171\\
273.01	0.00346940928617991\\
274.01	0.00346960843295105\\
275.01	0.00346981204438045\\
276.01	0.00347002022551944\\
277.01	0.00347023308404214\\
278.01	0.00347045073031277\\
279.01	0.00347067327745584\\
280.01	0.00347090084142684\\
281.01	0.00347113354108534\\
282.01	0.0034713714982694\\
283.01	0.00347161483787171\\
284.01	0.00347186368791751\\
285.01	0.0034721181796441\\
286.01	0.00347237844758251\\
287.01	0.00347264462964009\\
288.01	0.00347291686718541\\
289.01	0.00347319530513482\\
290.01	0.00347348009204033\\
291.01	0.00347377138017986\\
292.01	0.00347406932564839\\
293.01	0.0034743740884513\\
294.01	0.00347468583259949\\
295.01	0.00347500472620522\\
296.01	0.00347533094158058\\
297.01	0.00347566465533702\\
298.01	0.0034760060484862\\
299.01	0.00347635530654243\\
300.01	0.00347671261962672\\
301.01	0.00347707818257161\\
302.01	0.00347745219502794\\
303.01	0.00347783486157195\\
304.01	0.00347822639181405\\
305.01	0.00347862700050824\\
306.01	0.00347903690766271\\
307.01	0.00347945633865043\\
308.01	0.00347988552432155\\
309.01	0.00348032470111508\\
310.01	0.00348077411117149\\
311.01	0.00348123400244509\\
312.01	0.00348170462881735\\
313.01	0.00348218625020862\\
314.01	0.00348267913269065\\
315.01	0.00348318354859799\\
316.01	0.00348369977663887\\
317.01	0.00348422810200456\\
318.01	0.00348476881647847\\
319.01	0.00348532221854259\\
320.01	0.00348588861348323\\
321.01	0.00348646831349399\\
322.01	0.00348706163777694\\
323.01	0.00348766891264135\\
324.01	0.00348829047159897\\
325.01	0.00348892665545671\\
326.01	0.00348957781240554\\
327.01	0.00349024429810557\\
328.01	0.00349092647576742\\
329.01	0.00349162471622844\\
330.01	0.00349233939802451\\
331.01	0.00349307090745658\\
332.01	0.00349381963865134\\
333.01	0.00349458599361593\\
334.01	0.00349537038228661\\
335.01	0.00349617322257001\\
336.01	0.00349699494037756\\
337.01	0.00349783596965168\\
338.01	0.00349869675238413\\
339.01	0.00349957773862557\\
340.01	0.00350047938648575\\
341.01	0.00350140216212408\\
342.01	0.00350234653972973\\
343.01	0.00350331300149104\\
344.01	0.00350430203755365\\
345.01	0.00350531414596637\\
346.01	0.003506349832615\\
347.01	0.00350740961114285\\
348.01	0.00350849400285846\\
349.01	0.0035096035366293\\
350.01	0.00351073874876223\\
351.01	0.00351190018287018\\
352.01	0.0035130883897261\\
353.01	0.00351430392710535\\
354.01	0.0035155473596173\\
355.01	0.00351681925852976\\
356.01	0.00351812020158951\\
357.01	0.0035194507728439\\
358.01	0.00352081156247081\\
359.01	0.00352220316662644\\
360.01	0.00352362618732319\\
361.01	0.00352508123235327\\
362.01	0.00352656891527919\\
363.01	0.0035280898555162\\
364.01	0.00352964467853894\\
365.01	0.00353123401625111\\
366.01	0.00353285850756602\\
367.01	0.00353451879925318\\
368.01	0.00353621554711559\\
369.01	0.00353794941757132\\
370.01	0.003539721089716\\
371.01	0.00354153125794724\\
372.01	0.00354338063522171\\
373.01	0.00354526995699678\\
374.01	0.0035471999858638\\
375.01	0.00354917151680359\\
376.01	0.0035511853828645\\
377.01	0.00355324246085807\\
378.01	0.0035553436763458\\
379.01	0.00355749000667126\\
380.01	0.00355968247959745\\
381.01	0.00356192216159083\\
382.01	0.00356421014844351\\
383.01	0.00356654756533187\\
384.01	0.0035689355677666\\
385.01	0.00357137534257635\\
386.01	0.00357386810892851\\
387.01	0.00357641511938682\\
388.01	0.00357901766100757\\
389.01	0.00358167705647555\\
390.01	0.00358439466528152\\
391.01	0.00358717188494148\\
392.01	0.00359001015226043\\
393.01	0.00359291094464055\\
394.01	0.00359587578143637\\
395.01	0.00359890622535709\\
396.01	0.00360200388391832\\
397.01	0.00360517041094378\\
398.01	0.00360840750811844\\
399.01	0.00361171692659451\\
400.01	0.00361510046865083\\
401.01	0.00361855998940695\\
402.01	0.00362209739859295\\
403.01	0.00362571466237528\\
404.01	0.00362941380523974\\
405.01	0.00363319691193145\\
406.01	0.0036370661294524\\
407.01	0.00364102366911618\\
408.01	0.00364507180865957\\
409.01	0.00364921289440978\\
410.01	0.00365344934350697\\
411.01	0.00365778364617858\\
412.01	0.00366221836806465\\
413.01	0.00366675615258946\\
414.01	0.00367139972337641\\
415.01	0.00367615188669986\\
416.01	0.00368101553396871\\
417.01	0.00368599364423231\\
418.01	0.00369108928670067\\
419.01	0.00369630562326653\\
420.01	0.00370164591101524\\
421.01	0.00370711350470643\\
422.01	0.00371271185920789\\
423.01	0.00371844453185686\\
424.01	0.00372431518472269\\
425.01	0.00373032758673646\\
426.01	0.00373648561564981\\
427.01	0.00374279325977601\\
428.01	0.00374925461946055\\
429.01	0.00375587390821697\\
430.01	0.00376265545345388\\
431.01	0.00376960369670651\\
432.01	0.00377672319327018\\
433.01	0.00378401861111711\\
434.01	0.00379149472895806\\
435.01	0.00379915643328761\\
436.01	0.00380700871422563\\
437.01	0.00381505665993977\\
438.01	0.00382330544939905\\
439.01	0.00383176034317328\\
440.01	0.00384042667195154\\
441.01	0.00384930982241018\\
442.01	0.00385841522001223\\
443.01	0.00386774830827336\\
444.01	0.00387731452398092\\
445.01	0.00388711926780829\\
446.01	0.00389716786973208\\
447.01	0.00390746554863938\\
448.01	0.00391801736552\\
449.01	0.00392882816968527\\
450.01	0.00393990253756635\\
451.01	0.00395124470384521\\
452.01	0.00396285848500437\\
453.01	0.0039747471959013\\
454.01	0.00398691356075675\\
455.01	0.00399935962109656\\
456.01	0.00401208664484539\\
457.01	0.0040250950431314\\
458.01	0.00403838430467857\\
459.01	0.00405195296229161\\
460.01	0.00406579861235158\\
461.01	0.00407991801718984\\
462.01	0.00409430733516124\\
463.01	0.00410896262159766\\
464.01	0.00412388169678813\\
465.01	0.0041390635464146\\
466.01	0.00415450916369327\\
467.01	0.00417022703877926\\
468.01	0.00418623020788076\\
469.01	0.00420253300612251\\
470.01	0.00421915334458593\\
471.01	0.00423611450437305\\
472.01	0.00425344692776813\\
473.01	0.00427119058098699\\
474.01	0.0042893980666307\\
475.01	0.00430813595353832\\
476.01	0.00432745418754948\\
477.01	0.00434738642238951\\
478.01	0.00436796854694536\\
479.01	0.00438923923852827\\
480.01	0.00441124012149316\\
481.01	0.00443401590478815\\
482.01	0.00445761448463721\\
483.01	0.00448208699358519\\
484.01	0.00450748777056427\\
485.01	0.00453387421800094\\
486.01	0.00456130650062177\\
487.01	0.00458984702570931\\
488.01	0.0046195596249929\\
489.01	0.00465050833221185\\
490.01	0.00468275559761284\\
491.01	0.00471635938155702\\
492.01	0.0047513690821035\\
493.01	0.00478782101251431\\
494.01	0.0048257319804129\\
495.01	0.00486509058339598\\
496.01	0.00490584537563689\\
497.01	0.00494788813973441\\
498.01	0.00499067706248237\\
499.01	0.00503382368061261\\
500.01	0.00507728600037356\\
501.01	0.00512101589736558\\
502.01	0.00516495853105634\\
503.01	0.00520905175454348\\
504.01	0.00525322552146334\\
505.01	0.00529740134029413\\
506.01	0.00534149182733723\\
507.01	0.00538540041562128\\
508.01	0.00542902130718709\\
509.01	0.005472239789007\\
510.01	0.00551493307597316\\
511.01	0.00555697190167869\\
512.01	0.00559822315355755\\
513.01	0.00563855394918638\\
514.01	0.0056778376830278\\
515.01	0.00571596274960215\\
516.01	0.00575284489717085\\
517.01	0.00578844451789892\\
518.01	0.00582279122548661\\
519.01	0.00585647215374869\\
520.01	0.00589007821401159\\
521.01	0.00592357682065511\\
522.01	0.00595693559919004\\
523.01	0.00599012481542081\\
524.01	0.00602311876244421\\
525.01	0.00605589728395156\\
526.01	0.00608844743812646\\
527.01	0.00612076532603694\\
528.01	0.00615285802945342\\
529.01	0.00618474555417383\\
530.01	0.00621646260793039\\
531.01	0.00624805994354959\\
532.01	0.00627960485971661\\
533.01	0.00631118026364552\\
534.01	0.00634288143230977\\
535.01	0.00637480875100246\\
536.01	0.00640702911071041\\
537.01	0.00643956106745507\\
538.01	0.00647242193593152\\
539.01	0.00650563211117126\\
540.01	0.00653921506614063\\
541.01	0.00657319725365189\\
542.01	0.00660760790790732\\
543.01	0.00664247870742235\\
544.01	0.00667784317785525\\
545.01	0.00671373582867647\\
546.01	0.0067501910560746\\
547.01	0.00678724184788092\\
548.01	0.00682491841768709\\
549.01	0.00686324700395446\\
550.01	0.00690224961801737\\
551.01	0.00694194683434955\\
552.01	0.00698235970957576\\
553.01	0.00702350973072825\\
554.01	0.00706541862943193\\
555.01	0.00710810817972993\\
556.01	0.00715159999085044\\
557.01	0.00719591531061176\\
558.01	0.00724107485952969\\
559.01	0.00728709871898849\\
560.01	0.00733400629729445\\
561.01	0.00738181639206545\\
562.01	0.00743054735256967\\
563.01	0.00748021729887154\\
564.01	0.00753084421884847\\
565.01	0.00758244591252368\\
566.01	0.00763503991220713\\
567.01	0.00768864340185235\\
568.01	0.00774327313825653\\
569.01	0.00779894537555497\\
570.01	0.00785567579287152\\
571.01	0.00791347942325609\\
572.01	0.00797237057980226\\
573.01	0.00803236277225495\\
574.01	0.00809346860516428\\
575.01	0.00815569964939458\\
576.01	0.00821906628915308\\
577.01	0.00828357755347728\\
578.01	0.0083492409343018\\
579.01	0.00841606219068275\\
580.01	0.00848404513879882\\
581.01	0.00855319142743901\\
582.01	0.00862350029902764\\
583.01	0.00869496833695104\\
584.01	0.00876758920113668\\
585.01	0.00884135335563897\\
586.01	0.00891624779447431\\
587.01	0.00899225577465261\\
588.01	0.00906935656751973\\
589.01	0.00914752524228344\\
590.01	0.00922673249973243\\
591.01	0.00930694457949549\\
592.01	0.00938812327098054\\
593.01	0.00947022606676473\\
594.01	0.00955320650816933\\
595.01	0.00963701478667503\\
596.01	0.00972159868250433\\
597.01	0.00980690494409953\\
598.01	0.00989288124056736\\
599.01	0.00996919201787552\\
599.02	0.00996973313515437\\
599.03	0.00997027095217347\\
599.04	0.00997080543638652\\
599.05	0.00997133655493895\\
599.06	0.00997186427466602\\
599.07	0.00997238856209093\\
599.08	0.00997290938342301\\
599.09	0.00997342670448527\\
599.1	0.00997394049058981\\
599.11	0.00997445070670796\\
599.12	0.00997495731746685\\
599.13	0.00997546028714593\\
599.14	0.00997595957967355\\
599.15	0.00997645515862339\\
599.16	0.00997694698721088\\
599.17	0.00997743502828965\\
599.18	0.0099779192443486\\
599.19	0.00997839959750833\\
599.2	0.0099788760495175\\
599.21	0.00997934856174921\\
599.22	0.00997981709519721\\
599.23	0.00998028161047228\\
599.24	0.00998074206779834\\
599.25	0.00998119842700871\\
599.26	0.00998165064754226\\
599.27	0.00998209868843949\\
599.28	0.00998254250833861\\
599.29	0.00998298206547159\\
599.3	0.00998341731766015\\
599.31	0.00998384822231171\\
599.32	0.00998427473641528\\
599.33	0.00998469681653738\\
599.34	0.00998511441881785\\
599.35	0.00998552749896564\\
599.36	0.00998593601225455\\
599.37	0.00998633991351897\\
599.38	0.00998673915714952\\
599.39	0.00998713369708868\\
599.4	0.0099875234868264\\
599.41	0.00998790847809034\\
599.42	0.0099882886197258\\
599.43	0.00998866386007088\\
599.44	0.00998903414695145\\
599.45	0.00998939942767614\\
599.46	0.00998975964903123\\
599.47	0.00999011475727552\\
599.48	0.00999046469813513\\
599.49	0.00999080941679823\\
599.5	0.00999114885790982\\
599.51	0.00999148296556627\\
599.52	0.00999181168331002\\
599.53	0.00999213495412407\\
599.54	0.0099924527204265\\
599.55	0.00999276492406487\\
599.56	0.00999307150631066\\
599.57	0.00999337240785355\\
599.58	0.00999366756879571\\
599.59	0.00999395692864602\\
599.6	0.00999424042631423\\
599.61	0.00999451800010506\\
599.62	0.00999478958771224\\
599.63	0.00999505512621248\\
599.64	0.00999531455205943\\
599.65	0.00999556780107754\\
599.66	0.00999581480845585\\
599.67	0.00999605550874172\\
599.68	0.00999628983583456\\
599.69	0.00999651772297942\\
599.7	0.00999673910276056\\
599.71	0.00999695390709495\\
599.72	0.00999716206722566\\
599.73	0.0099973635137153\\
599.74	0.00999755817643927\\
599.75	0.009997745984579\\
599.76	0.00999792686661514\\
599.77	0.00999810075032065\\
599.78	0.00999826756275384\\
599.79	0.00999842723025132\\
599.8	0.00999857967842091\\
599.81	0.00999872483213449\\
599.82	0.0099988626155207\\
599.83	0.00999899295195769\\
599.84	0.00999911576406567\\
599.85	0.00999923097369951\\
599.86	0.00999933850194117\\
599.87	0.00999943826909211\\
599.88	0.00999953019466558\\
599.89	0.00999961419737891\\
599.9	0.00999969019514566\\
599.91	0.00999975810506767\\
599.92	0.00999981784342713\\
599.93	0.00999986932567848\\
599.94	0.00999991246644028\\
599.95	0.00999994717948697\\
599.96	0.00999997337774056\\
599.97	0.00999999097326228\\
599.98	0.00999999987724406\\
599.99	0.01\\
600	0.01\\
};
\addplot [color=blue!80!mycolor9,solid,forget plot]
  table[row sep=crcr]{%
0.01	0.00266457163794791\\
1.01	0.0026645724869681\\
2.01	0.00266457335376111\\
3.01	0.00266457423870125\\
4.01	0.00266457514217061\\
5.01	0.00266457606455949\\
6.01	0.00266457700626637\\
7.01	0.0026645779676983\\
8.01	0.00266457894927092\\
9.01	0.0026645799514086\\
10.01	0.00266458097454493\\
11.01	0.00266458201912257\\
12.01	0.00266458308559365\\
13.01	0.00266458417441982\\
14.01	0.00266458528607275\\
15.01	0.00266458642103385\\
16.01	0.00266458757979494\\
17.01	0.0026645887628583\\
18.01	0.00266458997073694\\
19.01	0.0026645912039547\\
20.01	0.00266459246304654\\
21.01	0.0026645937485589\\
22.01	0.00266459506104984\\
23.01	0.00266459640108933\\
24.01	0.00266459776925942\\
25.01	0.00266459916615464\\
26.01	0.00266460059238206\\
27.01	0.00266460204856192\\
28.01	0.00266460353532749\\
29.01	0.00266460505332568\\
30.01	0.00266460660321709\\
31.01	0.00266460818567651\\
32.01	0.00266460980139311\\
33.01	0.00266461145107075\\
34.01	0.00266461313542835\\
35.01	0.00266461485520027\\
36.01	0.00266461661113639\\
37.01	0.00266461840400285\\
38.01	0.00266462023458196\\
39.01	0.0026646221036729\\
40.01	0.0026646240120918\\
41.01	0.00266462596067241\\
42.01	0.00266462795026632\\
43.01	0.00266462998174324\\
44.01	0.0026646320559917\\
45.01	0.00266463417391909\\
46.01	0.00266463633645225\\
47.01	0.00266463854453793\\
48.01	0.00266464079914327\\
49.01	0.00266464310125601\\
50.01	0.00266464545188521\\
51.01	0.00266464785206155\\
52.01	0.00266465030283769\\
53.01	0.00266465280528916\\
54.01	0.00266465536051439\\
55.01	0.0026646579696355\\
56.01	0.00266466063379865\\
57.01	0.00266466335417471\\
58.01	0.0026646661319597\\
59.01	0.00266466896837545\\
60.01	0.00266467186467\\
61.01	0.00266467482211827\\
62.01	0.00266467784202264\\
63.01	0.00266468092571354\\
64.01	0.00266468407455017\\
65.01	0.00266468728992068\\
66.01	0.0026646905732436\\
67.01	0.00266469392596776\\
68.01	0.00266469734957341\\
69.01	0.00266470084557262\\
70.01	0.00266470441551018\\
71.01	0.00266470806096418\\
72.01	0.0026647117835469\\
73.01	0.00266471558490531\\
74.01	0.00266471946672216\\
75.01	0.00266472343071654\\
76.01	0.00266472747864462\\
77.01	0.00266473161230064\\
78.01	0.00266473583351774\\
79.01	0.00266474014416871\\
80.01	0.00266474454616691\\
81.01	0.00266474904146714\\
82.01	0.00266475363206663\\
83.01	0.00266475832000586\\
84.01	0.00266476310736967\\
85.01	0.00266476799628808\\
86.01	0.00266477298893737\\
87.01	0.00266477808754098\\
88.01	0.00266478329437085\\
89.01	0.00266478861174824\\
90.01	0.00266479404204483\\
91.01	0.00266479958768383\\
92.01	0.00266480525114132\\
93.01	0.00266481103494726\\
94.01	0.00266481694168662\\
95.01	0.00266482297400069\\
96.01	0.00266482913458842\\
97.01	0.00266483542620755\\
98.01	0.00266484185167591\\
99.01	0.00266484841387294\\
100.01	0.00266485511574084\\
101.01	0.00266486196028608\\
102.01	0.00266486895058091\\
103.01	0.0026648760897645\\
104.01	0.00266488338104495\\
105.01	0.00266489082770029\\
106.01	0.00266489843308033\\
107.01	0.00266490620060843\\
108.01	0.0026649141337827\\
109.01	0.0026649222361779\\
110.01	0.00266493051144728\\
111.01	0.00266493896332422\\
112.01	0.0026649475956238\\
113.01	0.00266495641224506\\
114.01	0.00266496541717257\\
115.01	0.00266497461447854\\
116.01	0.00266498400832446\\
117.01	0.00266499360296354\\
118.01	0.00266500340274252\\
119.01	0.00266501341210373\\
120.01	0.00266502363558734\\
121.01	0.00266503407783369\\
122.01	0.00266504474358513\\
123.01	0.00266505563768876\\
124.01	0.0026650667650986\\
125.01	0.00266507813087803\\
126.01	0.00266508974020212\\
127.01	0.00266510159836031\\
128.01	0.00266511371075902\\
129.01	0.00266512608292429\\
130.01	0.00266513872050413\\
131.01	0.00266515162927195\\
132.01	0.00266516481512874\\
133.01	0.00266517828410628\\
134.01	0.00266519204237022\\
135.01	0.00266520609622276\\
136.01	0.00266522045210615\\
137.01	0.0026652351166055\\
138.01	0.00266525009645228\\
139.01	0.00266526539852762\\
140.01	0.00266528102986538\\
141.01	0.00266529699765627\\
142.01	0.00266531330925086\\
143.01	0.00266532997216343\\
144.01	0.00266534699407573\\
145.01	0.00266536438284079\\
146.01	0.00266538214648667\\
147.01	0.00266540029322075\\
148.01	0.00266541883143365\\
149.01	0.00266543776970332\\
150.01	0.00266545711679963\\
151.01	0.00266547688168856\\
152.01	0.00266549707353662\\
153.01	0.00266551770171561\\
154.01	0.00266553877580733\\
155.01	0.00266556030560825\\
156.01	0.00266558230113468\\
157.01	0.00266560477262746\\
158.01	0.00266562773055756\\
159.01	0.00266565118563111\\
160.01	0.00266567514879483\\
161.01	0.00266569963124173\\
162.01	0.00266572464441652\\
163.01	0.00266575020002176\\
164.01	0.00266577631002354\\
165.01	0.0026658029866576\\
166.01	0.00266583024243576\\
167.01	0.0026658580901521\\
168.01	0.00266588654288963\\
169.01	0.0026659156140269\\
170.01	0.00266594531724485\\
171.01	0.00266597566653391\\
172.01	0.00266600667620102\\
173.01	0.00266603836087716\\
174.01	0.00266607073552483\\
175.01	0.00266610381544564\\
176.01	0.00266613761628846\\
177.01	0.00266617215405712\\
178.01	0.002666207445119\\
179.01	0.00266624350621338\\
180.01	0.00266628035446018\\
181.01	0.00266631800736872\\
182.01	0.00266635648284699\\
183.01	0.00266639579921062\\
184.01	0.00266643597519291\\
185.01	0.00266647702995419\\
186.01	0.002666518983092\\
187.01	0.0026665618546513\\
188.01	0.00266660566513491\\
189.01	0.00266665043551419\\
190.01	0.00266669618724028\\
191.01	0.00266674294225503\\
192.01	0.00266679072300276\\
193.01	0.00266683955244196\\
194.01	0.00266688945405739\\
195.01	0.00266694045187248\\
196.01	0.00266699257046196\\
197.01	0.0026670458349648\\
198.01	0.00266710027109773\\
199.01	0.00266715590516838\\
200.01	0.00266721276408991\\
201.01	0.00266727087539453\\
202.01	0.00266733026724861\\
203.01	0.00266739096846752\\
204.01	0.00266745300853088\\
205.01	0.00266751641759831\\
206.01	0.00266758122652549\\
207.01	0.00266764746688065\\
208.01	0.00266771517096133\\
209.01	0.00266778437181175\\
210.01	0.00266785510324043\\
211.01	0.00266792739983816\\
212.01	0.00266800129699683\\
213.01	0.00266807683092807\\
214.01	0.00266815403868285\\
215.01	0.00266823295817135\\
216.01	0.00266831362818346\\
217.01	0.00266839608840921\\
218.01	0.00266848037946069\\
219.01	0.00266856654289356\\
220.01	0.00266865462122943\\
221.01	0.00266874465797903\\
222.01	0.00266883669766536\\
223.01	0.00266893078584811\\
224.01	0.00266902696914791\\
225.01	0.00266912529527159\\
226.01	0.0026692258130382\\
227.01	0.00266932857240515\\
228.01	0.00266943362449527\\
229.01	0.00266954102162476\\
230.01	0.00266965081733107\\
231.01	0.00266976306640226\\
232.01	0.00266987782490644\\
233.01	0.00266999515022235\\
234.01	0.00267011510107036\\
235.01	0.00267023773754418\\
236.01	0.00267036312114371\\
237.01	0.00267049131480807\\
238.01	0.00267062238294997\\
239.01	0.0026707563914905\\
240.01	0.00267089340789491\\
241.01	0.00267103350120903\\
242.01	0.00267117674209699\\
243.01	0.00267132320287911\\
244.01	0.0026714729575714\\
245.01	0.00267162608192548\\
246.01	0.00267178265346955\\
247.01	0.00267194275155039\\
248.01	0.00267210645737619\\
249.01	0.0026722738540605\\
250.01	0.002672445026667\\
251.01	0.00267262006225551\\
252.01	0.00267279904992885\\
253.01	0.00267298208088099\\
254.01	0.00267316924844579\\
255.01	0.00267336064814749\\
256.01	0.00267355637775195\\
257.01	0.00267375653731905\\
258.01	0.00267396122925628\\
259.01	0.00267417055837368\\
260.01	0.00267438463193963\\
261.01	0.00267460355973817\\
262.01	0.0026748274541278\\
263.01	0.00267505643010083\\
264.01	0.00267529060534465\\
265.01	0.00267553010030422\\
266.01	0.00267577503824559\\
267.01	0.00267602554532119\\
268.01	0.00267628175063627\\
269.01	0.00267654378631698\\
270.01	0.0026768117875793\\
271.01	0.00267708589280041\\
272.01	0.00267736624359056\\
273.01	0.00267765298486741\\
274.01	0.00267794626493075\\
275.01	0.00267824623553987\\
276.01	0.00267855305199202\\
277.01	0.00267886687320239\\
278.01	0.00267918786178601\\
279.01	0.00267951618414114\\
280.01	0.00267985201053431\\
281.01	0.00268019551518725\\
282.01	0.00268054687636552\\
283.01	0.00268090627646873\\
284.01	0.00268127390212285\\
285.01	0.0026816499442741\\
286.01	0.00268203459828478\\
287.01	0.00268242806403111\\
288.01	0.00268283054600282\\
289.01	0.00268324225340482\\
290.01	0.00268366340026081\\
291.01	0.00268409420551895\\
292.01	0.00268453489315959\\
293.01	0.00268498569230494\\
294.01	0.00268544683733134\\
295.01	0.00268591856798311\\
296.01	0.00268640112948912\\
297.01	0.00268689477268145\\
298.01	0.00268739975411637\\
299.01	0.00268791633619792\\
300.01	0.00268844478730357\\
301.01	0.00268898538191259\\
302.01	0.00268953840073737\\
303.01	0.00269010413085666\\
304.01	0.00269068286585223\\
305.01	0.00269127490594787\\
306.01	0.0026918805581516\\
307.01	0.00269250013640087\\
308.01	0.00269313396171067\\
309.01	0.00269378236232487\\
310.01	0.00269444567387148\\
311.01	0.00269512423952074\\
312.01	0.00269581841014684\\
313.01	0.00269652854449395\\
314.01	0.00269725500934594\\
315.01	0.00269799817969983\\
316.01	0.00269875843894426\\
317.01	0.00269953617904219\\
318.01	0.0027003318007179\\
319.01	0.00270114571364983\\
320.01	0.00270197833666783\\
321.01	0.00270283009795648\\
322.01	0.00270370143526415\\
323.01	0.00270459279611743\\
324.01	0.0027055046380432\\
325.01	0.0027064374287965\\
326.01	0.00270739164659582\\
327.01	0.00270836778036614\\
328.01	0.00270936632998942\\
329.01	0.00271038780656391\\
330.01	0.00271143273267192\\
331.01	0.00271250164265704\\
332.01	0.0027135950829113\\
333.01	0.002714713612172\\
334.01	0.00271585780182998\\
335.01	0.00271702823624914\\
336.01	0.00271822551309752\\
337.01	0.00271945024369141\\
338.01	0.00272070305335274\\
339.01	0.0027219845817795\\
340.01	0.00272329548343155\\
341.01	0.00272463642793109\\
342.01	0.00272600810047945\\
343.01	0.00272741120228998\\
344.01	0.00272884645103871\\
345.01	0.00273031458133332\\
346.01	0.00273181634520065\\
347.01	0.0027333525125948\\
348.01	0.00273492387192541\\
349.01	0.00273653123060816\\
350.01	0.00273817541563807\\
351.01	0.00273985727418651\\
352.01	0.00274157767422373\\
353.01	0.00274333750516745\\
354.01	0.00274513767855961\\
355.01	0.00274697912877253\\
356.01	0.00274886281374583\\
357.01	0.00275078971575658\\
358.01	0.00275276084222435\\
359.01	0.00275477722655294\\
360.01	0.00275683992901134\\
361.01	0.00275895003765677\\
362.01	0.00276110866930049\\
363.01	0.00276331697052061\\
364.01	0.00276557611872193\\
365.01	0.00276788732324527\\
366.01	0.00277025182652645\\
367.01	0.00277267090530398\\
368.01	0.00277514587187318\\
369.01	0.00277767807538037\\
370.01	0.00278026890314976\\
371.01	0.00278291978202763\\
372.01	0.00278563217972543\\
373.01	0.00278840760613281\\
374.01	0.00279124761456527\\
375.01	0.00279415380289599\\
376.01	0.00279712781451333\\
377.01	0.00280017133904137\\
378.01	0.00280328611321068\\
379.01	0.00280647392864265\\
380.01	0.00280973660924248\\
381.01	0.00281307600841563\\
382.01	0.00281649402727172\\
383.01	0.00281999261600058\\
384.01	0.00282357377511652\\
385.01	0.00282723955672774\\
386.01	0.00283099206583052\\
387.01	0.00283483346162763\\
388.01	0.00283876595887161\\
389.01	0.00284279182923128\\
390.01	0.00284691340268196\\
391.01	0.00285113306891834\\
392.01	0.00285545327878948\\
393.01	0.00285987654575489\\
394.01	0.00286440544736059\\
395.01	0.00286904262673481\\
396.01	0.00287379079410072\\
397.01	0.00287865272830584\\
398.01	0.00288363127836563\\
399.01	0.00288872936501952\\
400.01	0.00289394998229737\\
401.01	0.00289929619909338\\
402.01	0.00290477116074526\\
403.01	0.00291037809061451\\
404.01	0.00291612029166492\\
405.01	0.002922001148035\\
406.01	0.00292802412659954\\
407.01	0.00293419277851558\\
408.01	0.00294051074074617\\
409.01	0.00294698173755663\\
410.01	0.00295360958197426\\
411.01	0.00296039817720486\\
412.01	0.00296735151799532\\
413.01	0.00297447369193275\\
414.01	0.00298176888066713\\
415.01	0.00298924136104482\\
416.01	0.00299689550613684\\
417.01	0.00300473578614513\\
418.01	0.00301276676916589\\
419.01	0.00302099312178881\\
420.01	0.00302941960950472\\
421.01	0.00303805109689367\\
422.01	0.0030468925475587\\
423.01	0.00305594902376694\\
424.01	0.00306522568575372\\
425.01	0.00307472779063846\\
426.01	0.00308446069089321\\
427.01	0.00309442983229672\\
428.01	0.00310464075129446\\
429.01	0.00311509907167435\\
430.01	0.00312581050045253\\
431.01	0.00313678082284618\\
432.01	0.00314801589619097\\
433.01	0.00315952164263663\\
434.01	0.0031713040404252\\
435.01	0.00318336911352479\\
436.01	0.00319572291935036\\
437.01	0.00320837153425779\\
438.01	0.00322132103643843\\
439.01	0.00323457748577748\\
440.01	0.00324814690015546\\
441.01	0.00326203522757654\\
442.01	0.003276248313389\\
443.01	0.00329079186172066\\
444.01	0.00330567139007952\\
445.01	0.00332089217585739\\
446.01	0.00333645919321636\\
447.01	0.00335237703852002\\
448.01	0.00336864984208066\\
449.01	0.00338528116350889\\
450.01	0.00340227386735189\\
451.01	0.00341962997495885\\
452.01	0.0034373504875775\\
453.01	0.0034554351745154\\
454.01	0.00347388231872536\\
455.01	0.00349268841031794\\
456.01	0.00351184777615066\\
457.01	0.00353135213066103\\
458.01	0.0035511900293104\\
459.01	0.00357134620116009\\
460.01	0.00359180073090039\\
461.01	0.00361252805274208\\
462.01	0.00363349570902957\\
463.01	0.00365466282199656\\
464.01	0.00367597820322375\\
465.01	0.00369737782856295\\
466.01	0.00371878167771662\\
467.01	0.00374009055359666\\
468.01	0.00376133808360597\\
469.01	0.00378254449084704\\
470.01	0.00380359765481982\\
471.01	0.00382435754725771\\
472.01	0.00384465007892143\\
473.01	0.00386425891196347\\
474.01	0.00388291464302406\\
475.01	0.00390069266259455\\
476.01	0.00391881844986454\\
477.01	0.00393741767035533\\
478.01	0.00395650237704295\\
479.01	0.00397608452442232\\
480.01	0.00399617587633663\\
481.01	0.00401678789648693\\
482.01	0.00403793161924702\\
483.01	0.0040596174982915\\
484.01	0.00408185523052651\\
485.01	0.00410465355293726\\
486.01	0.00412802001034002\\
487.01	0.00415196069297844\\
488.01	0.00417647994903544\\
489.01	0.00420158020761564\\
490.01	0.00422726721482038\\
491.01	0.00425360121734022\\
492.01	0.00428062631288843\\
493.01	0.00430837070158088\\
494.01	0.00433684816062326\\
495.01	0.00436607012923727\\
496.01	0.00439604936910906\\
497.01	0.00442680147837339\\
498.01	0.00445834901184761\\
499.01	0.00449072614764581\\
500.01	0.00452397116594763\\
501.01	0.00455812478068281\\
502.01	0.00459323015236275\\
503.01	0.00462933283209169\\
504.01	0.00466648068637506\\
505.01	0.0047047241659741\\
506.01	0.0047441163056711\\
507.01	0.00478471255337544\\
508.01	0.00482657046799717\\
509.01	0.00486974924138452\\
510.01	0.00491430898617083\\
511.01	0.00496030971408751\\
512.01	0.00500780990689396\\
513.01	0.0050568645528759\\
514.01	0.00510752248267552\\
515.01	0.00515982277813339\\
516.01	0.00521378994643276\\
517.01	0.00526942495828889\\
518.01	0.00532661479683448\\
519.01	0.00538461916995028\\
520.01	0.00544272241137332\\
521.01	0.00550084980280282\\
522.01	0.00555891123659179\\
523.01	0.00561679111518276\\
524.01	0.00567435363040023\\
525.01	0.0057314487426998\\
526.01	0.0057879137030987\\
527.01	0.00584357440185067\\
528.01	0.00589824790083561\\
529.01	0.0059517466270772\\
530.01	0.00600388487332472\\
531.01	0.00605448868008126\\
532.01	0.00610341014093586\\
533.01	0.00615054761026743\\
534.01	0.00619587398233221\\
535.01	0.0062395504608343\\
536.01	0.00628272466409035\\
537.01	0.00632570687442321\\
538.01	0.0063684708154812\\
539.01	0.00641099749854468\\
540.01	0.00645327643854359\\
541.01	0.00649530646326558\\
542.01	0.0065370960188697\\
543.01	0.0065786662053737\\
544.01	0.00662005583566263\\
545.01	0.00666132366051709\\
546.01	0.00670254971458035\\
547.01	0.00674383537412062\\
548.01	0.00678530020261345\\
549.01	0.00682707436024067\\
550.01	0.00686925950890378\\
551.01	0.00691188719112618\\
552.01	0.00695498218311738\\
553.01	0.00699857348386164\\
554.01	0.00704269416314167\\
555.01	0.00708738096435513\\
556.01	0.00713267362631148\\
557.01	0.007178613887444\\
558.01	0.0072252441598414\\
559.01	0.00727260590629534\\
560.01	0.00732073783306641\\
561.01	0.00736967414027251\\
562.01	0.007419443185512\\
563.01	0.00747006821770754\\
564.01	0.00752157146830121\\
565.01	0.00757397540609622\\
566.01	0.00762730254286453\\
567.01	0.00768157517579493\\
568.01	0.00773681508916095\\
569.01	0.00779304325438374\\
570.01	0.0078502795596901\\
571.01	0.00790854260152617\\
572.01	0.007967849572551\\
573.01	0.0080282162773056\\
574.01	0.00808965727958438\\
575.01	0.00815218607059426\\
576.01	0.00821581501954256\\
577.01	0.00828055518630606\\
578.01	0.00834641611642781\\
579.01	0.00841340562488602\\
580.01	0.00848152957296922\\
581.01	0.00855079164303971\\
582.01	0.00862119311141858\\
583.01	0.00869273261663716\\
584.01	0.00876540591708722\\
585.01	0.00883920562907762\\
586.01	0.00891412093710792\\
587.01	0.00899013728670779\\
588.01	0.00906723608772262\\
589.01	0.00914539444740412\\
590.01	0.00922458495535043\\
591.01	0.00930477554784545\\
592.01	0.00938592948552506\\
593.01	0.00946800548660199\\
594.01	0.0095509580681256\\
595.01	0.00963473816035478\\
596.01	0.00971929407498429\\
597.01	0.00980457292772622\\
598.01	0.00989052264111993\\
599.01	0.0099691898561019\\
599.02	0.00996973151746593\\
599.03	0.00997026979807466\\
599.04	0.00997080466579245\\
599.05	0.00997133608806853\\
599.06	0.0099718640319399\\
599.07	0.00997238846402112\\
599.08	0.00997290935048918\\
599.09	0.00997342667879889\\
599.1	0.00997394047065652\\
599.11	0.00997445069119892\\
599.12	0.00997495730522922\\
599.13	0.0099754602772142\\
599.14	0.00997595957128175\\
599.15	0.00997645515121832\\
599.16	0.00997694698046649\\
599.17	0.00997743502212201\\
599.18	0.00997791923868709\\
599.19	0.00997839959229404\\
599.2	0.00997887604470164\\
599.21	0.00997934855729153\\
599.22	0.00997981709106457\\
599.23	0.00998028160663717\\
599.24	0.0099807420642377\\
599.25	0.00998119842370277\\
599.26	0.00998165064447353\\
599.27	0.00998209868559199\\
599.28	0.00998254250569747\\
599.29	0.00998298206302299\\
599.3	0.00998341731539121\\
599.31	0.00998384822021048\\
599.32	0.00998427473447066\\
599.33	0.00998469681473907\\
599.34	0.00998511441715626\\
599.35	0.00998552749743186\\
599.36	0.00998593601084028\\
599.37	0.00998633991221646\\
599.38	0.00998673915595153\\
599.39	0.00998713369598844\\
599.4	0.00998752348581752\\
599.41	0.00998790847716685\\
599.42	0.00998828861888209\\
599.43	0.0099886638593016\\
599.44	0.00998903414625157\\
599.45	0.00998939942704085\\
599.46	0.00998975964845599\\
599.47	0.00999011475675598\\
599.48	0.00999046469766715\\
599.49	0.0099908094163779\\
599.5	0.00999114885753339\\
599.51	0.00999148296523019\\
599.52	0.00999181168301094\\
599.53	0.00999213495385881\\
599.54	0.00999245272019206\\
599.55	0.00999276492385844\\
599.56	0.0099930715061296\\
599.57	0.00999337240769538\\
599.58	0.00999366756865814\\
599.59	0.00999395692852691\\
599.6	0.00999424042621159\\
599.61	0.00999451800001706\\
599.62	0.00999478958763719\\
599.63	0.00999505512614884\\
599.64	0.00999531455200579\\
599.65	0.00999556780103262\\
599.66	0.00999581480841847\\
599.67	0.00999605550871085\\
599.68	0.00999628983580927\\
599.69	0.00999651772295887\\
599.7	0.00999673910274401\\
599.71	0.00999695390708174\\
599.72	0.00999716206721524\\
599.73	0.00999736351370716\\
599.74	0.009997558176433\\
599.75	0.00999774598457423\\
599.76	0.00999792686661157\\
599.77	0.00999810075031802\\
599.78	0.00999826756275193\\
599.79	0.00999842723024996\\
599.8	0.00999857967841997\\
599.81	0.00999872483213385\\
599.82	0.00999886261552029\\
599.83	0.00999899295195742\\
599.84	0.00999911576406551\\
599.85	0.00999923097369942\\
599.86	0.00999933850194112\\
599.87	0.00999943826909208\\
599.88	0.00999953019466556\\
599.89	0.00999961419737891\\
599.9	0.00999969019514565\\
599.91	0.00999975810506767\\
599.92	0.00999981784342713\\
599.93	0.00999986932567848\\
599.94	0.00999991246644028\\
599.95	0.00999994717948697\\
599.96	0.00999997337774056\\
599.97	0.00999999097326229\\
599.98	0.00999999987724406\\
599.99	0.01\\
600	0.01\\
};
\addplot [color=blue,solid,forget plot]
  table[row sep=crcr]{%
0.01	0.00125393423089431\\
1.01	0.00125393505353844\\
2.01	0.00125393589347361\\
3.01	0.00125393675106585\\
4.01	0.00125393762668902\\
5.01	0.00125393852072497\\
6.01	0.00125393943356374\\
7.01	0.00125394036560372\\
8.01	0.00125394131725177\\
9.01	0.00125394228892352\\
10.01	0.00125394328104338\\
11.01	0.001253944294045\\
12.01	0.00125394532837118\\
13.01	0.00125394638447431\\
14.01	0.00125394746281636\\
15.01	0.00125394856386934\\
16.01	0.00125394968811527\\
17.01	0.0012539508360466\\
18.01	0.00125395200816618\\
19.01	0.00125395320498783\\
20.01	0.00125395442703635\\
21.01	0.00125395567484773\\
22.01	0.00125395694896955\\
23.01	0.00125395824996109\\
24.01	0.00125395957839373\\
25.01	0.00125396093485108\\
26.01	0.0012539623199293\\
27.01	0.00125396373423735\\
28.01	0.00125396517839733\\
29.01	0.00125396665304475\\
30.01	0.00125396815882872\\
31.01	0.00125396969641239\\
32.01	0.00125397126647316\\
33.01	0.00125397286970307\\
34.01	0.00125397450680905\\
35.01	0.00125397617851321\\
36.01	0.00125397788555339\\
37.01	0.00125397962868316\\
38.01	0.00125398140867252\\
39.01	0.00125398322630799\\
40.01	0.0012539850823931\\
41.01	0.00125398697774879\\
42.01	0.00125398891321365\\
43.01	0.00125399088964447\\
44.01	0.00125399290791648\\
45.01	0.00125399496892396\\
46.01	0.00125399707358046\\
47.01	0.00125399922281941\\
48.01	0.00125400141759426\\
49.01	0.0012540036588793\\
50.01	0.00125400594766984\\
51.01	0.00125400828498275\\
52.01	0.00125401067185706\\
53.01	0.00125401310935421\\
54.01	0.0012540155985587\\
55.01	0.00125401814057864\\
56.01	0.00125402073654616\\
57.01	0.00125402338761801\\
58.01	0.00125402609497604\\
59.01	0.00125402885982782\\
60.01	0.00125403168340715\\
61.01	0.00125403456697468\\
62.01	0.00125403751181852\\
63.01	0.00125404051925487\\
64.01	0.00125404359062846\\
65.01	0.0012540467273135\\
66.01	0.00125404993071397\\
67.01	0.00125405320226462\\
68.01	0.00125405654343137\\
69.01	0.00125405995571222\\
70.01	0.00125406344063787\\
71.01	0.00125406699977242\\
72.01	0.00125407063471412\\
73.01	0.00125407434709629\\
74.01	0.00125407813858777\\
75.01	0.00125408201089396\\
76.01	0.00125408596575772\\
77.01	0.00125409000495993\\
78.01	0.00125409413032049\\
79.01	0.00125409834369923\\
80.01	0.0012541026469967\\
81.01	0.00125410704215514\\
82.01	0.00125411153115942\\
83.01	0.00125411611603796\\
84.01	0.00125412079886368\\
85.01	0.00125412558175515\\
86.01	0.00125413046687735\\
87.01	0.00125413545644301\\
88.01	0.00125414055271341\\
89.01	0.0012541457579996\\
90.01	0.0012541510746636\\
91.01	0.00125415650511938\\
92.01	0.00125416205183409\\
93.01	0.00125416771732924\\
94.01	0.00125417350418203\\
95.01	0.00125417941502655\\
96.01	0.00125418545255487\\
97.01	0.00125419161951863\\
98.01	0.00125419791873025\\
99.01	0.00125420435306429\\
100.01	0.0012542109254589\\
101.01	0.0012542176389172\\
102.01	0.00125422449650878\\
103.01	0.00125423150137134\\
104.01	0.0012542386567118\\
105.01	0.00125424596580841\\
106.01	0.00125425343201209\\
107.01	0.00125426105874794\\
108.01	0.00125426884951726\\
109.01	0.00125427680789904\\
110.01	0.00125428493755174\\
111.01	0.00125429324221514\\
112.01	0.00125430172571228\\
113.01	0.00125431039195117\\
114.01	0.00125431924492683\\
115.01	0.00125432828872321\\
116.01	0.00125433752751532\\
117.01	0.0012543469655712\\
118.01	0.00125435660725406\\
119.01	0.0012543664570245\\
120.01	0.00125437651944268\\
121.01	0.00125438679917059\\
122.01	0.00125439730097445\\
123.01	0.00125440802972698\\
124.01	0.00125441899040989\\
125.01	0.00125443018811639\\
126.01	0.00125444162805375\\
127.01	0.00125445331554588\\
128.01	0.00125446525603598\\
129.01	0.00125447745508923\\
130.01	0.00125448991839586\\
131.01	0.00125450265177357\\
132.01	0.00125451566117089\\
133.01	0.00125452895266993\\
134.01	0.00125454253248947\\
135.01	0.00125455640698822\\
136.01	0.00125457058266792\\
137.01	0.0012545850661767\\
138.01	0.00125459986431244\\
139.01	0.00125461498402614\\
140.01	0.00125463043242562\\
141.01	0.00125464621677888\\
142.01	0.00125466234451806\\
143.01	0.00125467882324309\\
144.01	0.00125469566072556\\
145.01	0.00125471286491268\\
146.01	0.00125473044393147\\
147.01	0.00125474840609271\\
148.01	0.00125476675989535\\
149.01	0.00125478551403082\\
150.01	0.00125480467738739\\
151.01	0.00125482425905484\\
152.01	0.00125484426832918\\
153.01	0.00125486471471729\\
154.01	0.00125488560794186\\
155.01	0.00125490695794642\\
156.01	0.00125492877490042\\
157.01	0.00125495106920464\\
158.01	0.00125497385149625\\
159.01	0.0012549971326546\\
160.01	0.00125502092380669\\
161.01	0.00125504523633294\\
162.01	0.00125507008187316\\
163.01	0.00125509547233247\\
164.01	0.00125512141988763\\
165.01	0.00125514793699322\\
166.01	0.0012551750363882\\
167.01	0.00125520273110249\\
168.01	0.00125523103446372\\
169.01	0.00125525996010425\\
170.01	0.00125528952196834\\
171.01	0.00125531973431916\\
172.01	0.00125535061174647\\
173.01	0.00125538216917418\\
174.01	0.00125541442186815\\
175.01	0.00125544738544411\\
176.01	0.0012554810758759\\
177.01	0.00125551550950389\\
178.01	0.00125555070304341\\
179.01	0.00125558667359361\\
180.01	0.00125562343864633\\
181.01	0.00125566101609541\\
182.01	0.00125569942424602\\
183.01	0.00125573868182436\\
184.01	0.00125577880798727\\
185.01	0.00125581982233263\\
186.01	0.00125586174490941\\
187.01	0.00125590459622835\\
188.01	0.00125594839727284\\
189.01	0.0012559931695098\\
190.01	0.00125603893490105\\
191.01	0.001256085715915\\
192.01	0.00125613353553838\\
193.01	0.00125618241728846\\
194.01	0.00125623238522533\\
195.01	0.00125628346396483\\
196.01	0.00125633567869134\\
197.01	0.00125638905517123\\
198.01	0.00125644361976628\\
199.01	0.00125649939944798\\
200.01	0.00125655642181124\\
201.01	0.00125661471508962\\
202.01	0.00125667430816975\\
203.01	0.00125673523060681\\
204.01	0.0012567975126402\\
205.01	0.00125686118520943\\
206.01	0.00125692627997055\\
207.01	0.00125699282931286\\
208.01	0.00125706086637609\\
209.01	0.00125713042506786\\
210.01	0.00125720154008163\\
211.01	0.00125727424691511\\
212.01	0.00125734858188885\\
213.01	0.00125742458216568\\
214.01	0.00125750228577012\\
215.01	0.00125758173160866\\
216.01	0.00125766295949011\\
217.01	0.00125774601014684\\
218.01	0.00125783092525605\\
219.01	0.00125791774746189\\
220.01	0.00125800652039799\\
221.01	0.00125809728871031\\
222.01	0.0012581900980809\\
223.01	0.0012582849952516\\
224.01	0.001258382028049\\
225.01	0.00125848124540939\\
226.01	0.00125858269740458\\
227.01	0.00125868643526809\\
228.01	0.00125879251142234\\
229.01	0.00125890097950571\\
230.01	0.00125901189440101\\
231.01	0.00125912531226417\\
232.01	0.00125924129055357\\
233.01	0.00125935988806013\\
234.01	0.00125948116493796\\
235.01	0.00125960518273595\\
236.01	0.00125973200442973\\
237.01	0.00125986169445445\\
238.01	0.00125999431873861\\
239.01	0.00126012994473805\\
240.01	0.0012602686414711\\
241.01	0.0012604104795545\\
242.01	0.00126055553123994\\
243.01	0.00126070387045153\\
244.01	0.00126085557282399\\
245.01	0.00126101071574184\\
246.01	0.00126116937837926\\
247.01	0.00126133164174099\\
248.01	0.00126149758870402\\
249.01	0.00126166730406024\\
250.01	0.00126184087456004\\
251.01	0.00126201838895677\\
252.01	0.00126219993805231\\
253.01	0.00126238561474343\\
254.01	0.00126257551406953\\
255.01	0.001262769733261\\
256.01	0.00126296837178885\\
257.01	0.00126317153141541\\
258.01	0.00126337931624609\\
259.01	0.00126359183278239\\
260.01	0.00126380918997596\\
261.01	0.00126403149928375\\
262.01	0.00126425887472449\\
263.01	0.00126449143293652\\
264.01	0.00126472929323666\\
265.01	0.00126497257768039\\
266.01	0.00126522141112361\\
267.01	0.00126547592128547\\
268.01	0.0012657362388127\\
269.01	0.00126600249734532\\
270.01	0.00126627483358403\\
271.01	0.00126655338735864\\
272.01	0.00126683830169855\\
273.01	0.00126712972290432\\
274.01	0.00126742780062136\\
275.01	0.00126773268791475\\
276.01	0.00126804454134626\\
277.01	0.00126836352105265\\
278.01	0.00126868979082625\\
279.01	0.00126902351819685\\
280.01	0.00126936487451616\\
281.01	0.00126971403504353\\
282.01	0.00127007117903408\\
283.01	0.00127043648982901\\
284.01	0.00127081015494774\\
285.01	0.00127119236618251\\
286.01	0.00127158331969509\\
287.01	0.00127198321611599\\
288.01	0.00127239226064606\\
289.01	0.00127281066316061\\
290.01	0.00127323863831603\\
291.01	0.00127367640565934\\
292.01	0.00127412418974026\\
293.01	0.00127458222022641\\
294.01	0.00127505073202128\\
295.01	0.00127552996538557\\
296.01	0.0012760201660615\\
297.01	0.00127652158540051\\
298.01	0.00127703448049461\\
299.01	0.00127755911431093\\
300.01	0.00127809575583051\\
301.01	0.00127864468019063\\
302.01	0.00127920616883109\\
303.01	0.00127978050964508\\
304.01	0.00128036799713387\\
305.01	0.00128096893256652\\
306.01	0.00128158362414391\\
307.01	0.00128221238716784\\
308.01	0.00128285554421517\\
309.01	0.00128351342531733\\
310.01	0.00128418636814499\\
311.01	0.00128487471819885\\
312.01	0.0012855788290061\\
313.01	0.00128629906232317\\
314.01	0.00128703578834468\\
315.01	0.00128778938591937\\
316.01	0.00128856024277257\\
317.01	0.001289348755736\\
318.01	0.00129015533098513\\
319.01	0.00129098038428395\\
320.01	0.00129182434123811\\
321.01	0.00129268763755599\\
322.01	0.00129357071931841\\
323.01	0.00129447404325767\\
324.01	0.00129539807704496\\
325.01	0.00129634329958798\\
326.01	0.00129731020133809\\
327.01	0.00129829928460753\\
328.01	0.00129931106389727\\
329.01	0.00130034606623538\\
330.01	0.00130140483152689\\
331.01	0.00130248791291458\\
332.01	0.00130359587715185\\
333.01	0.00130472930498773\\
334.01	0.00130588879156403\\
335.01	0.0013070749468253\\
336.01	0.00130828839594193\\
337.01	0.00130952977974665\\
338.01	0.00131079975518428\\
339.01	0.00131209899577613\\
340.01	0.00131342819209829\\
341.01	0.00131478805227443\\
342.01	0.00131617930248375\\
343.01	0.00131760268748405\\
344.01	0.0013190589711502\\
345.01	0.00132054893702836\\
346.01	0.00132207338890637\\
347.01	0.00132363315140022\\
348.01	0.00132522907055724\\
349.01	0.00132686201447599\\
350.01	0.00132853287394308\\
351.01	0.00133024256308745\\
352.01	0.00133199202005168\\
353.01	0.00133378220768134\\
354.01	0.00133561411423148\\
355.01	0.00133748875409111\\
356.01	0.00133940716852558\\
357.01	0.00134137042643621\\
358.01	0.00134337962513773\\
359.01	0.00134543589115287\\
360.01	0.00134754038102385\\
361.01	0.0013496942821399\\
362.01	0.00135189881358083\\
363.01	0.00135415522697461\\
364.01	0.00135646480736899\\
365.01	0.00135882887411507\\
366.01	0.00136124878176148\\
367.01	0.00136372592095738\\
368.01	0.00136626171936242\\
369.01	0.00136885764256187\\
370.01	0.00137151519498482\\
371.01	0.00137423592082448\\
372.01	0.00137702140495972\\
373.01	0.00137987327387897\\
374.01	0.00138279319660863\\
375.01	0.00138578288565305\\
376.01	0.00138884409795629\\
377.01	0.0013919786359096\\
378.01	0.00139518834852271\\
379.01	0.00139847513283203\\
380.01	0.00140184093410652\\
381.01	0.00140528774874939\\
382.01	0.00140881762657582\\
383.01	0.00141243267238211\\
384.01	0.00141613504756924\\
385.01	0.00141992697182572\\
386.01	0.00142381072487216\\
387.01	0.00142778864827148\\
388.01	0.00143186314730736\\
389.01	0.00143603669293546\\
390.01	0.00144031182381085\\
391.01	0.00144469114839608\\
392.01	0.00144917734715422\\
393.01	0.00145377317483215\\
394.01	0.00145848146283897\\
395.01	0.0014633051217252\\
396.01	0.00146824714376911\\
397.01	0.00147331060567626\\
398.01	0.00147849867139957\\
399.01	0.00148381459508746\\
400.01	0.00148926172416791\\
401.01	0.00149484350257767\\
402.01	0.00150056347414533\\
403.01	0.00150642528613958\\
404.01	0.00151243269299284\\
405.01	0.00151858956021261\\
406.01	0.00152489986849356\\
407.01	0.0015313677180441\\
408.01	0.00153799733314351\\
409.01	0.00154479306694518\\
410.01	0.00155175940654517\\
411.01	0.00155890097833498\\
412.01	0.00156622255366081\\
413.01	0.00157372905481189\\
414.01	0.00158142556136492\\
415.01	0.00158931731691189\\
416.01	0.00159740973620281\\
417.01	0.00160570841273723\\
418.01	0.00161421912684314\\
419.01	0.00162294785428316\\
420.01	0.00163190077543637\\
421.01	0.00164108428510503\\
422.01	0.00165050500300411\\
423.01	0.00166016978499704\\
424.01	0.00167008573514798\\
425.01	0.00168026021867062\\
426.01	0.00169070087586256\\
427.01	0.00170141563712507\\
428.01	0.00171241273918239\\
429.01	0.00172370074262813\\
430.01	0.00173528855094479\\
431.01	0.00174718543116167\\
432.01	0.00175940103634059\\
433.01	0.00177194543010548\\
434.01	0.00178482911346454\\
435.01	0.00179806305421074\\
436.01	0.0018116587192307\\
437.01	0.00182562811010398\\
438.01	0.00183998380243701\\
439.01	0.00185473898944803\\
440.01	0.00186990753040684\\
441.01	0.00188550400463645\\
442.01	0.00190154377190793\\
443.01	0.00191804304020712\\
444.01	0.0019350189420306\\
445.01	0.0019524896205829\\
446.01	0.00197047432750603\\
447.01	0.00198899353408612\\
448.01	0.00200806905826303\\
449.01	0.0020277242102326\\
450.01	0.00204798395999738\\
451.01	0.0020688751309135\\
452.01	0.00209042662413107\\
453.01	0.00211266967986786\\
454.01	0.00213563818274333\\
455.01	0.00215936901998632\\
456.01	0.00218390250329652\\
457.01	0.00220928286757602\\
458.01	0.00223555886277866\\
459.01	0.00226278445889772\\
460.01	0.0022910196888251\\
461.01	0.00232033165971263\\
462.01	0.00235079577092562\\
463.01	0.00238249718617517\\
464.01	0.00241553261370436\\
465.01	0.00245001243512748\\
466.01	0.0024860629897989\\
467.01	0.00252269340217711\\
468.01	0.00253901728385071\\
469.01	0.00255639392450184\\
470.01	0.00257496450888221\\
471.01	0.00259489690471044\\
472.01	0.00261639053285141\\
473.01	0.00263968364758491\\
474.01	0.00266506226992916\\
475.01	0.00269245505943368\\
476.01	0.00272065093610824\\
477.01	0.00274955271826105\\
478.01	0.00277917729609491\\
479.01	0.00280954178593348\\
480.01	0.00284066351017588\\
481.01	0.00287255997420967\\
482.01	0.00290524884014917\\
483.01	0.00293874789733586\\
484.01	0.00297307502964344\\
485.01	0.00300824817977751\\
486.01	0.00304428531097289\\
487.01	0.00308120436689807\\
488.01	0.00311902323296728\\
489.01	0.00315775973843944\\
490.01	0.00319743226223462\\
491.01	0.00323806066098094\\
492.01	0.00327966220576215\\
493.01	0.00332225262627554\\
494.01	0.00336584605081814\\
495.01	0.00341045552030027\\
496.01	0.00345609286794123\\
497.01	0.00350276851295833\\
498.01	0.00355049120673921\\
499.01	0.00359926768736594\\
500.01	0.00364910239797952\\
501.01	0.00369999726787668\\
502.01	0.0037519516255841\\
503.01	0.00380496318885829\\
504.01	0.00385900734789926\\
505.01	0.00391403660138614\\
506.01	0.00396998585551344\\
507.01	0.0040267684552152\\
508.01	0.00408427124200197\\
509.01	0.00414234847335277\\
510.01	0.00420081432960689\\
511.01	0.00425943366533543\\
512.01	0.00431791057508681\\
513.01	0.00437587423268418\\
514.01	0.00443286131603846\\
515.01	0.00448829405724582\\
516.01	0.00454145169655468\\
517.01	0.00459201346058562\\
518.01	0.0046429781323353\\
519.01	0.00469489782545895\\
520.01	0.00474771890165974\\
521.01	0.00480136321045815\\
522.01	0.00485574293994415\\
523.01	0.00491145818565206\\
524.01	0.00496885945429269\\
525.01	0.00502801076693878\\
526.01	0.00508897183480263\\
527.01	0.00515179490913153\\
528.01	0.00521652055630504\\
529.01	0.00528317329708926\\
530.01	0.00535175362761153\\
531.01	0.00542221561762005\\
532.01	0.00549445951575484\\
533.01	0.00556831438034086\\
534.01	0.00564352082639238\\
535.01	0.00571974964221054\\
536.01	0.0057957442445397\\
537.01	0.00587105093686107\\
538.01	0.00594551194541186\\
539.01	0.00601897841781172\\
540.01	0.00609132568218902\\
541.01	0.00616247150904156\\
542.01	0.00623236290012066\\
543.01	0.0063008176648035\\
544.01	0.00636760942669748\\
545.01	0.00643254230701142\\
546.01	0.00649542408219803\\
547.01	0.00655610162007356\\
548.01	0.006614507589475\\
549.01	0.0066707307889387\\
550.01	0.00672605603223849\\
551.01	0.00678112490934816\\
552.01	0.00683590420357794\\
553.01	0.0068903704202434\\
554.01	0.00694451627906765\\
555.01	0.00699835431625151\\
556.01	0.00705192041560494\\
557.01	0.00710527686155195\\
558.01	0.00715851427507727\\
559.01	0.00721175147058358\\
560.01	0.00726513178502448\\
561.01	0.00731881411340836\\
562.01	0.00737295989880207\\
563.01	0.00742766298174074\\
564.01	0.00748295075754973\\
565.01	0.00753885104528757\\
566.01	0.0075953952343396\\
567.01	0.00765261854464693\\
568.01	0.00771055980364574\\
569.01	0.0077692603192376\\
570.01	0.00782876227912249\\
571.01	0.00788910669409165\\
572.01	0.0079503309275909\\
573.01	0.00801246605430344\\
574.01	0.00807553530800077\\
575.01	0.00813955656820071\\
576.01	0.00820454654838254\\
577.01	0.00827052111535247\\
578.01	0.00833749503822894\\
579.01	0.00840548173779629\\
580.01	0.00847449285543066\\
581.01	0.00854453779037137\\
582.01	0.00861562329403154\\
583.01	0.00868775317389058\\
584.01	0.00876092815854303\\
585.01	0.00883514596130418\\
586.01	0.00891040144580925\\
587.01	0.00898668645072877\\
588.01	0.00906398946084543\\
589.01	0.00914229527513433\\
590.01	0.0092215846004733\\
591.01	0.00930183366668292\\
592.01	0.00938301391778946\\
593.01	0.00946509184244681\\
594.01	0.00954802902026272\\
595.01	0.00963178247925906\\
596.01	0.00971630547999516\\
597.01	0.00980154886224196\\
598.01	0.00988746310858009\\
599.01	0.00996918176309228\\
599.02	0.00996972426151389\\
599.03	0.00997026325652431\\
599.04	0.00997079875308772\\
599.05	0.00997133075863836\\
599.06	0.00997185927537436\\
599.07	0.00997238430589235\\
599.08	0.00997290585509465\\
599.09	0.00997342387048931\\
599.1	0.00997393826406157\\
599.11	0.00997444900322816\\
599.12	0.00997495605368425\\
599.13	0.00997545938076358\\
599.14	0.00997595894942923\\
599.15	0.0099764547242638\\
599.16	0.00997694666945895\\
599.17	0.00997743474895895\\
599.18	0.00997791899742809\\
599.19	0.00997839937780886\\
599.2	0.00997887585262738\\
599.21	0.00997934838398374\\
599.22	0.00997981693354265\\
599.23	0.00998028146252\\
599.24	0.00998074193166859\\
599.25	0.00998119830126832\\
599.26	0.00998165053112998\\
599.27	0.0099820985805808\\
599.28	0.00998254240840773\\
599.29	0.00998298197288523\\
599.3	0.00998341723187397\\
599.31	0.00998384814281939\\
599.32	0.0099842746627474\\
599.33	0.00998469674826017\\
599.34	0.00998511435553172\\
599.35	0.00998552744030362\\
599.36	0.00998593595788055\\
599.37	0.00998633986312586\\
599.38	0.00998673911045709\\
599.39	0.00998713365384143\\
599.4	0.00998752344679118\\
599.41	0.00998790844105529\\
599.42	0.00998828858549677\\
599.43	0.0099886638284696\\
599.44	0.00998903411781367\\
599.45	0.00998939940084989\\
599.46	0.0099897596243751\\
599.47	0.00999011473465703\\
599.48	0.00999046467742929\\
599.49	0.00999080939788633\\
599.5	0.00999114884067832\\
599.51	0.00999148294990612\\
599.52	0.0099918116691162\\
599.53	0.00999213494129551\\
599.54	0.00999245270886605\\
599.55	0.0099927649136794\\
599.56	0.009993071497011\\
599.57	0.00999337239955457\\
599.58	0.00999366756141627\\
599.59	0.00999395692210898\\
599.6	0.00999424042054647\\
599.61	0.00999451799503739\\
599.62	0.00999478958327942\\
599.63	0.00999505512235319\\
599.64	0.00999531454871618\\
599.65	0.00999556779819663\\
599.66	0.00999581480598729\\
599.67	0.00999605550663917\\
599.68	0.00999628983405521\\
599.69	0.0099965177214839\\
599.7	0.00999673910151283\\
599.71	0.00999695390606217\\
599.72	0.00999716206637809\\
599.73	0.00999736351302614\\
599.74	0.00999755817588451\\
599.75	0.00999774598413729\\
599.76	0.00999792686626763\\
599.77	0.00999810075005082\\
599.78	0.00999826756254734\\
599.79	0.0099984272300958\\
599.8	0.00999857967830587\\
599.81	0.00999872483205108\\
599.82	0.00999886261546159\\
599.83	0.00999899295191686\\
599.84	0.0099991157640383\\
599.85	0.00999923097368178\\
599.86	0.00999933850193015\\
599.87	0.00999943826908557\\
599.88	0.00999953019466193\\
599.89	0.00999961419737702\\
599.9	0.00999969019514477\\
599.91	0.0099997581050673\\
599.92	0.009999817843427\\
599.93	0.00999986932567845\\
599.94	0.00999991246644027\\
599.95	0.00999994717948697\\
599.96	0.00999997337774056\\
599.97	0.00999999097326228\\
599.98	0.00999999987724406\\
599.99	0.01\\
600	0.01\\
};
\addplot [color=mycolor10,solid,forget plot]
  table[row sep=crcr]{%
0.01	0\\
1.01	0\\
2.01	0\\
3.01	0\\
4.01	0\\
5.01	0\\
6.01	0\\
7.01	0\\
8.01	0\\
9.01	0\\
10.01	0\\
11.01	0\\
12.01	0\\
13.01	0\\
14.01	0\\
15.01	0\\
16.01	0\\
17.01	0\\
18.01	0\\
19.01	0\\
20.01	0\\
21.01	0\\
22.01	0\\
23.01	0\\
24.01	0\\
25.01	0\\
26.01	0\\
27.01	0\\
28.01	0\\
29.01	0\\
30.01	0\\
31.01	0\\
32.01	0\\
33.01	0\\
34.01	0\\
35.01	0\\
36.01	0\\
37.01	0\\
38.01	0\\
39.01	0\\
40.01	0\\
41.01	0\\
42.01	0\\
43.01	0\\
44.01	0\\
45.01	0\\
46.01	0\\
47.01	0\\
48.01	0\\
49.01	0\\
50.01	0\\
51.01	0\\
52.01	0\\
53.01	0\\
54.01	0\\
55.01	0\\
56.01	0\\
57.01	0\\
58.01	0\\
59.01	0\\
60.01	0\\
61.01	0\\
62.01	0\\
63.01	0\\
64.01	0\\
65.01	0\\
66.01	0\\
67.01	0\\
68.01	0\\
69.01	0\\
70.01	0\\
71.01	0\\
72.01	0\\
73.01	0\\
74.01	0\\
75.01	0\\
76.01	0\\
77.01	0\\
78.01	0\\
79.01	0\\
80.01	0\\
81.01	0\\
82.01	0\\
83.01	0\\
84.01	0\\
85.01	0\\
86.01	0\\
87.01	0\\
88.01	0\\
89.01	0\\
90.01	0\\
91.01	0\\
92.01	0\\
93.01	0\\
94.01	0\\
95.01	0\\
96.01	0\\
97.01	0\\
98.01	0\\
99.01	0\\
100.01	0\\
101.01	0\\
102.01	0\\
103.01	0\\
104.01	0\\
105.01	0\\
106.01	0\\
107.01	0\\
108.01	0\\
109.01	0\\
110.01	0\\
111.01	0\\
112.01	0\\
113.01	0\\
114.01	0\\
115.01	0\\
116.01	0\\
117.01	0\\
118.01	0\\
119.01	0\\
120.01	0\\
121.01	0\\
122.01	0\\
123.01	0\\
124.01	0\\
125.01	0\\
126.01	0\\
127.01	0\\
128.01	0\\
129.01	0\\
130.01	0\\
131.01	0\\
132.01	0\\
133.01	0\\
134.01	0\\
135.01	0\\
136.01	0\\
137.01	0\\
138.01	0\\
139.01	0\\
140.01	0\\
141.01	0\\
142.01	0\\
143.01	0\\
144.01	0\\
145.01	0\\
146.01	0\\
147.01	0\\
148.01	0\\
149.01	0\\
150.01	0\\
151.01	0\\
152.01	0\\
153.01	0\\
154.01	0\\
155.01	0\\
156.01	0\\
157.01	0\\
158.01	0\\
159.01	0\\
160.01	0\\
161.01	0\\
162.01	0\\
163.01	0\\
164.01	0\\
165.01	0\\
166.01	0\\
167.01	0\\
168.01	0\\
169.01	0\\
170.01	0\\
171.01	0\\
172.01	0\\
173.01	0\\
174.01	0\\
175.01	0\\
176.01	0\\
177.01	0\\
178.01	0\\
179.01	0\\
180.01	0\\
181.01	0\\
182.01	0\\
183.01	0\\
184.01	0\\
185.01	0\\
186.01	0\\
187.01	0\\
188.01	0\\
189.01	0\\
190.01	0\\
191.01	0\\
192.01	0\\
193.01	0\\
194.01	0\\
195.01	0\\
196.01	0\\
197.01	0\\
198.01	0\\
199.01	0\\
200.01	0\\
201.01	0\\
202.01	0\\
203.01	0\\
204.01	0\\
205.01	0\\
206.01	0\\
207.01	0\\
208.01	0\\
209.01	0\\
210.01	0\\
211.01	0\\
212.01	0\\
213.01	0\\
214.01	0\\
215.01	0\\
216.01	0\\
217.01	0\\
218.01	0\\
219.01	0\\
220.01	0\\
221.01	0\\
222.01	0\\
223.01	0\\
224.01	0\\
225.01	0\\
226.01	0\\
227.01	0\\
228.01	0\\
229.01	0\\
230.01	0\\
231.01	0\\
232.01	0\\
233.01	0\\
234.01	0\\
235.01	0\\
236.01	0\\
237.01	0\\
238.01	0\\
239.01	0\\
240.01	0\\
241.01	0\\
242.01	0\\
243.01	0\\
244.01	0\\
245.01	0\\
246.01	0\\
247.01	0\\
248.01	0\\
249.01	0\\
250.01	0\\
251.01	0\\
252.01	0\\
253.01	0\\
254.01	0\\
255.01	0\\
256.01	0\\
257.01	0\\
258.01	0\\
259.01	0\\
260.01	0\\
261.01	0\\
262.01	0\\
263.01	0\\
264.01	0\\
265.01	0\\
266.01	0\\
267.01	0\\
268.01	0\\
269.01	0\\
270.01	0\\
271.01	0\\
272.01	0\\
273.01	0\\
274.01	0\\
275.01	0\\
276.01	0\\
277.01	0\\
278.01	0\\
279.01	0\\
280.01	0\\
281.01	0\\
282.01	0\\
283.01	0\\
284.01	0\\
285.01	0\\
286.01	0\\
287.01	0\\
288.01	0\\
289.01	0\\
290.01	0\\
291.01	0\\
292.01	0\\
293.01	0\\
294.01	0\\
295.01	0\\
296.01	0\\
297.01	0\\
298.01	0\\
299.01	0\\
300.01	0\\
301.01	0\\
302.01	0\\
303.01	0\\
304.01	0\\
305.01	0\\
306.01	0\\
307.01	0\\
308.01	0\\
309.01	0\\
310.01	0\\
311.01	0\\
312.01	0\\
313.01	0\\
314.01	0\\
315.01	0\\
316.01	0\\
317.01	0\\
318.01	0\\
319.01	0\\
320.01	0\\
321.01	0\\
322.01	0\\
323.01	0\\
324.01	0\\
325.01	0\\
326.01	0\\
327.01	0\\
328.01	0\\
329.01	0\\
330.01	0\\
331.01	0\\
332.01	0\\
333.01	0\\
334.01	0\\
335.01	0\\
336.01	0\\
337.01	0\\
338.01	0\\
339.01	0\\
340.01	0\\
341.01	0\\
342.01	0\\
343.01	0\\
344.01	0\\
345.01	0\\
346.01	0\\
347.01	0\\
348.01	0\\
349.01	0\\
350.01	0\\
351.01	0\\
352.01	0\\
353.01	0\\
354.01	0\\
355.01	0\\
356.01	0\\
357.01	0\\
358.01	0\\
359.01	0\\
360.01	0\\
361.01	0\\
362.01	0\\
363.01	0\\
364.01	0\\
365.01	0\\
366.01	0\\
367.01	0\\
368.01	0\\
369.01	0\\
370.01	0\\
371.01	0\\
372.01	0\\
373.01	0\\
374.01	0\\
375.01	0\\
376.01	0\\
377.01	0\\
378.01	0\\
379.01	0\\
380.01	0\\
381.01	0\\
382.01	0\\
383.01	0\\
384.01	0\\
385.01	0\\
386.01	0\\
387.01	0\\
388.01	0\\
389.01	0\\
390.01	0\\
391.01	0\\
392.01	0\\
393.01	0\\
394.01	0\\
395.01	0\\
396.01	0\\
397.01	0\\
398.01	0\\
399.01	0\\
400.01	0\\
401.01	0\\
402.01	0\\
403.01	0\\
404.01	0\\
405.01	0\\
406.01	0\\
407.01	0\\
408.01	0\\
409.01	0\\
410.01	0\\
411.01	0\\
412.01	0\\
413.01	0\\
414.01	0\\
415.01	0\\
416.01	0\\
417.01	0\\
418.01	0\\
419.01	0\\
420.01	0\\
421.01	0\\
422.01	0\\
423.01	0\\
424.01	0\\
425.01	0\\
426.01	0\\
427.01	0\\
428.01	0\\
429.01	0\\
430.01	0\\
431.01	0\\
432.01	0\\
433.01	0\\
434.01	0\\
435.01	0\\
436.01	0\\
437.01	0\\
438.01	0\\
439.01	0\\
440.01	0\\
441.01	0\\
442.01	0\\
443.01	0\\
444.01	0\\
445.01	0\\
446.01	0\\
447.01	0\\
448.01	0\\
449.01	0\\
450.01	0\\
451.01	0\\
452.01	0\\
453.01	0\\
454.01	0\\
455.01	0\\
456.01	0\\
457.01	0\\
458.01	0\\
459.01	0\\
460.01	0\\
461.01	0\\
462.01	0\\
463.01	0\\
464.01	0\\
465.01	0\\
466.01	0\\
467.01	1.1329020402559e-06\\
468.01	2.42926371570376e-05\\
469.01	4.81479683581976e-05\\
470.01	7.27151256976269e-05\\
471.01	9.80079158148517e-05\\
472.01	0.000124038584063818\\
473.01	0.000150816227182535\\
474.01	0.000178344407502577\\
475.01	0.000206620404597957\\
476.01	0.000235655901361494\\
477.01	0.000265476405587259\\
478.01	0.000296108967598012\\
479.01	0.000327582024237015\\
480.01	0.000359925502228145\\
481.01	0.000393170931133675\\
482.01	0.000427351566845549\\
483.01	0.000462502526611258\\
484.01	0.000498660936648215\\
485.01	0.000535866093441591\\
486.01	0.000574159639844557\\
487.01	0.000613585757155623\\
488.01	0.000654191375036341\\
489.01	0.000696026407974198\\
490.01	0.000739144056248947\\
491.01	0.000783601041649637\\
492.01	0.000829457753426376\\
493.01	0.000876778638348154\\
494.01	0.000925632540756657\\
495.01	0.000976093074296254\\
496.01	0.00102823898237879\\
497.01	0.00108215450488519\\
498.01	0.00113792974319596\\
499.01	0.00119566100888369\\
500.01	0.00125545111606378\\
501.01	0.00131740941963204\\
502.01	0.00138165037152721\\
503.01	0.00144829225281667\\
504.01	0.00151748425710271\\
505.01	0.00158940412526166\\
506.01	0.00166425130362097\\
507.01	0.00174225050833373\\
508.01	0.00182365606282622\\
509.01	0.00190875711176604\\
510.01	0.00199788390795892\\
511.01	0.00209141541598479\\
512.01	0.002189788535845\\
513.01	0.00229350932384645\\
514.01	0.00240316667304261\\
515.01	0.0025194489415711\\
516.01	0.00264316302433761\\
517.01	0.00277466880452097\\
518.01	0.00291105160323856\\
519.01	0.00305194372531061\\
520.01	0.00319760921929432\\
521.01	0.00334834332533889\\
522.01	0.00348683251894904\\
523.01	0.00356545994298809\\
524.01	0.00364603069415331\\
525.01	0.00372855121145225\\
526.01	0.0038130173250395\\
527.01	0.00389941178465692\\
528.01	0.00398770129590559\\
529.01	0.00407783303432947\\
530.01	0.00416973012800455\\
531.01	0.00426328637259697\\
532.01	0.00435836060467695\\
533.01	0.00445476958820691\\
534.01	0.00455228061471653\\
535.01	0.00465060558407056\\
536.01	0.00474938505180781\\
537.01	0.00484816680571053\\
538.01	0.00494636987829088\\
539.01	0.00504324604238324\\
540.01	0.00513783824991967\\
541.01	0.00522892631659753\\
542.01	0.00531858003379235\\
543.01	0.00541077469880275\\
544.01	0.00550551210428055\\
545.01	0.0056026673440873\\
546.01	0.00570201569717175\\
547.01	0.00580320740074056\\
548.01	0.00590572513662101\\
549.01	0.0060087956203401\\
550.01	0.00611035611335997\\
551.01	0.00621030214124193\\
552.01	0.00630902143547821\\
553.01	0.00640619490681774\\
554.01	0.00650148779051777\\
555.01	0.00659455811486975\\
556.01	0.00668506994227237\\
557.01	0.00677271322077196\\
558.01	0.00685723252977279\\
559.01	0.00693846697486137\\
560.01	0.00701641341732511\\
561.01	0.00709125918268363\\
562.01	0.0071634980266294\\
563.01	0.00723508800698468\\
564.01	0.00730639425146508\\
565.01	0.00737738964046476\\
566.01	0.00744807960110319\\
567.01	0.00751845268206095\\
568.01	0.00758851847485299\\
569.01	0.00765831772946058\\
570.01	0.00772792707527122\\
571.01	0.00779746294572383\\
572.01	0.00786708620901358\\
573.01	0.00793698753582011\\
574.01	0.00800732822400679\\
575.01	0.00807816822687876\\
576.01	0.00814952977669954\\
577.01	0.00822143569195578\\
578.01	0.00829390508513462\\
579.01	0.00836695756900242\\
580.01	0.00844061618955115\\
581.01	0.00851490610843392\\
582.01	0.008589852586896\\
583.01	0.00866547846795157\\
584.01	0.00874180134816968\\
585.01	0.00881883011789336\\
586.01	0.00889656636788261\\
587.01	0.00897501123941693\\
588.01	0.00905415976584804\\
589.01	0.00913400639453102\\
590.01	0.00921454649402135\\
591.01	0.00929577647899807\\
592.01	0.00937769332881967\\
593.01	0.00946029382921912\\
594.01	0.00954357359616978\\
595.01	0.00962752597645235\\
596.01	0.00971214096674654\\
597.01	0.00979740439062092\\
598.01	0.00988329769391883\\
599.01	0.00996873201918203\\
599.02	0.00996938150051711\\
599.03	0.00997000982619995\\
599.04	0.00997061680203572\\
599.05	0.00997120222994154\\
599.06	0.00997176894407421\\
599.07	0.00997231748063729\\
599.08	0.00997284766231445\\
599.09	0.00997337282663754\\
599.1	0.00997389327698541\\
599.11	0.00997440918165476\\
599.12	0.00997492099195222\\
599.13	0.00997542869053242\\
599.14	0.00997593226134337\\
599.15	0.00997643168971941\\
599.16	0.00997692696247924\\
599.17	0.00997741806787515\\
599.18	0.00997790492496225\\
599.19	0.00997838752592332\\
599.2	0.00997886586487288\\
599.21	0.00997933993670913\\
599.22	0.00997980973680635\\
599.23	0.00998027526269526\\
599.24	0.00998073651420421\\
599.25	0.00998119349148986\\
599.26	0.00998164618964857\\
599.27	0.00998209460591699\\
599.28	0.00998253874521366\\
599.29	0.00998297859807655\\
599.3	0.00998341412419777\\
599.31	0.0099838452819738\\
599.32	0.00998427202941182\\
599.33	0.00998469432412354\\
599.34	0.00998511212331871\\
599.35	0.00998552538379823\\
599.36	0.00998593406194686\\
599.37	0.00998633811372554\\
599.38	0.00998673749466321\\
599.39	0.00998713215984817\\
599.4	0.00998752206391895\\
599.41	0.00998790715979874\\
599.42	0.00998828739732171\\
599.43	0.00998866272571617\\
599.44	0.00998903309366109\\
599.45	0.00998939844927219\\
599.46	0.00998975874008727\\
599.47	0.00999011391305059\\
599.48	0.00999046391449633\\
599.49	0.00999080869013592\\
599.5	0.00999114818504226\\
599.51	0.0099914823436306\\
599.52	0.0099918111096529\\
599.53	0.00999213442619165\\
599.54	0.00999245223572696\\
599.55	0.00999276448016512\\
599.56	0.00999307110083833\\
599.57	0.00999337203849892\\
599.58	0.00999366723331349\\
599.59	0.00999395662485697\\
599.6	0.00999424015210655\\
599.61	0.00999451775343565\\
599.62	0.00999478936660834\\
599.63	0.00999505492877341\\
599.64	0.00999531437645813\\
599.65	0.00999556764556198\\
599.66	0.00999581467135026\\
599.67	0.0099960553884477\\
599.68	0.00999628973083186\\
599.69	0.00999651763182663\\
599.7	0.00999673902409553\\
599.71	0.00999695383963498\\
599.72	0.00999716200976746\\
599.73	0.00999736346513468\\
599.74	0.00999755813569058\\
599.75	0.00999774595069428\\
599.76	0.00999792683870299\\
599.77	0.00999810072756478\\
599.78	0.00999826754441137\\
599.79	0.00999842721565074\\
599.8	0.00999857966695976\\
599.81	0.00999872482327668\\
599.82	0.00999886260879365\\
599.83	0.00999899294694907\\
599.84	0.00999911576041993\\
599.85	0.00999923097111411\\
599.86	0.00999933850016262\\
599.87	0.00999943826791175\\
599.88	0.00999953019391525\\
599.89	0.00999961419692642\\
599.9	0.0099996901948902\\
599.91	0.00999975810493522\\
599.92	0.00999981784336591\\
599.93	0.00999986932565446\\
599.94	0.009999912466433\\
599.95	0.00999994717948561\\
599.96	0.00999997337774052\\
599.97	0.00999999097326228\\
599.98	0.00999999987724406\\
599.99	0.01\\
600	0.01\\
};
\addplot [color=mycolor11,solid,forget plot]
  table[row sep=crcr]{%
0.01	0\\
1.01	0\\
2.01	0\\
3.01	0\\
4.01	0\\
5.01	0\\
6.01	0\\
7.01	0\\
8.01	0\\
9.01	0\\
10.01	0\\
11.01	0\\
12.01	0\\
13.01	0\\
14.01	0\\
15.01	0\\
16.01	0\\
17.01	0\\
18.01	0\\
19.01	0\\
20.01	0\\
21.01	0\\
22.01	0\\
23.01	0\\
24.01	0\\
25.01	0\\
26.01	0\\
27.01	0\\
28.01	0\\
29.01	0\\
30.01	0\\
31.01	0\\
32.01	0\\
33.01	0\\
34.01	0\\
35.01	0\\
36.01	0\\
37.01	0\\
38.01	0\\
39.01	0\\
40.01	0\\
41.01	0\\
42.01	0\\
43.01	0\\
44.01	0\\
45.01	0\\
46.01	0\\
47.01	0\\
48.01	0\\
49.01	0\\
50.01	0\\
51.01	0\\
52.01	0\\
53.01	0\\
54.01	0\\
55.01	0\\
56.01	0\\
57.01	0\\
58.01	0\\
59.01	0\\
60.01	0\\
61.01	0\\
62.01	0\\
63.01	0\\
64.01	0\\
65.01	0\\
66.01	0\\
67.01	0\\
68.01	0\\
69.01	0\\
70.01	0\\
71.01	0\\
72.01	0\\
73.01	0\\
74.01	0\\
75.01	0\\
76.01	0\\
77.01	0\\
78.01	0\\
79.01	0\\
80.01	0\\
81.01	0\\
82.01	0\\
83.01	0\\
84.01	0\\
85.01	0\\
86.01	0\\
87.01	0\\
88.01	0\\
89.01	0\\
90.01	0\\
91.01	0\\
92.01	0\\
93.01	0\\
94.01	0\\
95.01	0\\
96.01	0\\
97.01	0\\
98.01	0\\
99.01	0\\
100.01	0\\
101.01	0\\
102.01	0\\
103.01	0\\
104.01	0\\
105.01	0\\
106.01	0\\
107.01	0\\
108.01	0\\
109.01	0\\
110.01	0\\
111.01	0\\
112.01	0\\
113.01	0\\
114.01	0\\
115.01	0\\
116.01	0\\
117.01	0\\
118.01	0\\
119.01	0\\
120.01	0\\
121.01	0\\
122.01	0\\
123.01	0\\
124.01	0\\
125.01	0\\
126.01	0\\
127.01	0\\
128.01	0\\
129.01	0\\
130.01	0\\
131.01	0\\
132.01	0\\
133.01	0\\
134.01	0\\
135.01	0\\
136.01	0\\
137.01	0\\
138.01	0\\
139.01	0\\
140.01	0\\
141.01	0\\
142.01	0\\
143.01	0\\
144.01	0\\
145.01	0\\
146.01	0\\
147.01	0\\
148.01	0\\
149.01	0\\
150.01	0\\
151.01	0\\
152.01	0\\
153.01	0\\
154.01	0\\
155.01	0\\
156.01	0\\
157.01	0\\
158.01	0\\
159.01	0\\
160.01	0\\
161.01	0\\
162.01	0\\
163.01	0\\
164.01	0\\
165.01	0\\
166.01	0\\
167.01	0\\
168.01	0\\
169.01	0\\
170.01	0\\
171.01	0\\
172.01	0\\
173.01	0\\
174.01	0\\
175.01	0\\
176.01	0\\
177.01	0\\
178.01	0\\
179.01	0\\
180.01	0\\
181.01	0\\
182.01	0\\
183.01	0\\
184.01	0\\
185.01	0\\
186.01	0\\
187.01	0\\
188.01	0\\
189.01	0\\
190.01	0\\
191.01	0\\
192.01	0\\
193.01	0\\
194.01	0\\
195.01	0\\
196.01	0\\
197.01	0\\
198.01	0\\
199.01	0\\
200.01	0\\
201.01	0\\
202.01	0\\
203.01	0\\
204.01	0\\
205.01	0\\
206.01	0\\
207.01	0\\
208.01	0\\
209.01	0\\
210.01	0\\
211.01	0\\
212.01	0\\
213.01	0\\
214.01	0\\
215.01	0\\
216.01	0\\
217.01	0\\
218.01	0\\
219.01	0\\
220.01	0\\
221.01	0\\
222.01	0\\
223.01	0\\
224.01	0\\
225.01	0\\
226.01	0\\
227.01	0\\
228.01	0\\
229.01	0\\
230.01	0\\
231.01	0\\
232.01	0\\
233.01	0\\
234.01	0\\
235.01	0\\
236.01	0\\
237.01	0\\
238.01	0\\
239.01	0\\
240.01	0\\
241.01	0\\
242.01	0\\
243.01	0\\
244.01	0\\
245.01	0\\
246.01	0\\
247.01	0\\
248.01	0\\
249.01	0\\
250.01	0\\
251.01	0\\
252.01	0\\
253.01	0\\
254.01	0\\
255.01	0\\
256.01	0\\
257.01	0\\
258.01	0\\
259.01	0\\
260.01	0\\
261.01	0\\
262.01	0\\
263.01	0\\
264.01	0\\
265.01	0\\
266.01	0\\
267.01	0\\
268.01	0\\
269.01	0\\
270.01	0\\
271.01	0\\
272.01	0\\
273.01	0\\
274.01	0\\
275.01	0\\
276.01	0\\
277.01	0\\
278.01	0\\
279.01	0\\
280.01	0\\
281.01	0\\
282.01	0\\
283.01	0\\
284.01	0\\
285.01	0\\
286.01	0\\
287.01	0\\
288.01	0\\
289.01	0\\
290.01	0\\
291.01	0\\
292.01	0\\
293.01	0\\
294.01	0\\
295.01	0\\
296.01	0\\
297.01	0\\
298.01	0\\
299.01	0\\
300.01	0\\
301.01	0\\
302.01	0\\
303.01	0\\
304.01	0\\
305.01	0\\
306.01	0\\
307.01	0\\
308.01	0\\
309.01	0\\
310.01	0\\
311.01	0\\
312.01	0\\
313.01	0\\
314.01	0\\
315.01	0\\
316.01	0\\
317.01	0\\
318.01	0\\
319.01	0\\
320.01	0\\
321.01	0\\
322.01	0\\
323.01	0\\
324.01	0\\
325.01	0\\
326.01	0\\
327.01	0\\
328.01	0\\
329.01	0\\
330.01	0\\
331.01	0\\
332.01	0\\
333.01	0\\
334.01	0\\
335.01	0\\
336.01	0\\
337.01	0\\
338.01	0\\
339.01	0\\
340.01	0\\
341.01	0\\
342.01	0\\
343.01	0\\
344.01	0\\
345.01	0\\
346.01	0\\
347.01	0\\
348.01	0\\
349.01	0\\
350.01	0\\
351.01	0\\
352.01	0\\
353.01	0\\
354.01	0\\
355.01	0\\
356.01	0\\
357.01	0\\
358.01	0\\
359.01	0\\
360.01	0\\
361.01	0\\
362.01	0\\
363.01	0\\
364.01	0\\
365.01	0\\
366.01	0\\
367.01	0\\
368.01	0\\
369.01	0\\
370.01	0\\
371.01	0\\
372.01	0\\
373.01	0\\
374.01	0\\
375.01	0\\
376.01	0\\
377.01	0\\
378.01	0\\
379.01	0\\
380.01	0\\
381.01	0\\
382.01	0\\
383.01	0\\
384.01	0\\
385.01	0\\
386.01	0\\
387.01	0\\
388.01	0\\
389.01	0\\
390.01	0\\
391.01	0\\
392.01	0\\
393.01	0\\
394.01	0\\
395.01	0\\
396.01	0\\
397.01	0\\
398.01	0\\
399.01	0\\
400.01	0\\
401.01	0\\
402.01	0\\
403.01	0\\
404.01	0\\
405.01	0\\
406.01	0\\
407.01	0\\
408.01	0\\
409.01	0\\
410.01	0\\
411.01	0\\
412.01	0\\
413.01	0\\
414.01	0\\
415.01	0\\
416.01	0\\
417.01	0\\
418.01	0\\
419.01	0\\
420.01	0\\
421.01	0\\
422.01	0\\
423.01	0\\
424.01	0\\
425.01	0\\
426.01	0\\
427.01	0\\
428.01	0\\
429.01	0\\
430.01	0\\
431.01	0\\
432.01	0\\
433.01	0\\
434.01	0\\
435.01	0\\
436.01	0\\
437.01	0\\
438.01	0\\
439.01	0\\
440.01	0\\
441.01	0\\
442.01	0\\
443.01	0\\
444.01	0\\
445.01	0\\
446.01	0\\
447.01	0\\
448.01	0\\
449.01	0\\
450.01	0\\
451.01	0\\
452.01	0\\
453.01	0\\
454.01	0\\
455.01	0\\
456.01	0\\
457.01	0\\
458.01	0\\
459.01	0\\
460.01	0\\
461.01	0\\
462.01	0\\
463.01	0\\
464.01	0\\
465.01	0\\
466.01	0\\
467.01	0\\
468.01	0\\
469.01	0\\
470.01	0\\
471.01	0\\
472.01	0\\
473.01	0\\
474.01	0\\
475.01	0\\
476.01	0\\
477.01	0\\
478.01	0\\
479.01	0\\
480.01	0\\
481.01	0\\
482.01	0\\
483.01	0\\
484.01	0\\
485.01	0\\
486.01	0\\
487.01	0\\
488.01	0\\
489.01	0\\
490.01	0\\
491.01	0\\
492.01	0\\
493.01	0\\
494.01	0\\
495.01	0\\
496.01	0\\
497.01	0\\
498.01	0\\
499.01	0\\
500.01	0\\
501.01	0\\
502.01	0\\
503.01	0\\
504.01	0\\
505.01	0\\
506.01	0\\
507.01	0\\
508.01	0\\
509.01	0\\
510.01	0\\
511.01	0\\
512.01	0\\
513.01	0\\
514.01	0\\
515.01	0\\
516.01	0\\
517.01	0\\
518.01	0\\
519.01	0\\
520.01	0\\
521.01	0\\
522.01	1.76072062530383e-05\\
523.01	0.0001000531427582\\
524.01	0.000185428064959862\\
525.01	0.000273915626818395\\
526.01	0.000365718517235863\\
527.01	0.000461061166109798\\
528.01	0.000560192925065827\\
529.01	0.000663391817041978\\
530.01	0.000770968941636186\\
531.01	0.000883273738164617\\
532.01	0.00100070026104646\\
533.01	0.00112369466667597\\
534.01	0.00125276429462983\\
535.01	0.00138848841820502\\
536.01	0.00153153040644095\\
537.01	0.00168265200432605\\
538.01	0.00184273081035254\\
539.01	0.00201279319736091\\
540.01	0.00219403790053492\\
541.01	0.00238785532069443\\
542.01	0.00259226846870504\\
543.01	0.00280354376610866\\
544.01	0.0030219682635569\\
545.01	0.00324793355738788\\
546.01	0.00348188162292224\\
547.01	0.00372430787917213\\
548.01	0.00397577250198962\\
549.01	0.0042369103544857\\
550.01	0.00449075671486131\\
551.01	0.00462618461453768\\
552.01	0.00476421212870169\\
553.01	0.00490458243399326\\
554.01	0.00504693095053299\\
555.01	0.00519075452171885\\
556.01	0.00533537205239804\\
557.01	0.00547987416083386\\
558.01	0.00562305849481418\\
559.01	0.0057633450823129\\
560.01	0.00589868845148751\\
561.01	0.00603278110898117\\
562.01	0.00616897852510308\\
563.01	0.00630509330493087\\
564.01	0.00644045192015123\\
565.01	0.00657462204057506\\
566.01	0.00670711983291309\\
567.01	0.00683740812363565\\
568.01	0.0069648990353388\\
569.01	0.00708895843143271\\
570.01	0.00720891484742703\\
571.01	0.00732402511855551\\
572.01	0.00743343423629905\\
573.01	0.00753834307265663\\
574.01	0.00764066192485596\\
575.01	0.00774196963664896\\
576.01	0.00784232616627326\\
577.01	0.00794188335541293\\
578.01	0.00804077814675698\\
579.01	0.00813893755468454\\
580.01	0.00823628502515087\\
581.01	0.00833276801027717\\
582.01	0.00842836104623377\\
583.01	0.0085230664726219\\
584.01	0.00861691980849836\\
585.01	0.00871008730799935\\
586.01	0.00880256534465678\\
587.01	0.00889458764294722\\
588.01	0.00898632359220033\\
589.01	0.00907770521085289\\
590.01	0.00916865124237691\\
591.01	0.00925908767877544\\
592.01	0.00934895807582376\\
593.01	0.00943823031869177\\
594.01	0.00952690140896054\\
595.01	0.00961500159478332\\
596.01	0.00970259742227558\\
597.01	0.00978979309540728\\
598.01	0.00987673031330844\\
599.01	0.00996359465763311\\
599.02	0.00996446178033267\\
599.03	0.00996532763967337\\
599.04	0.00996619223192198\\
599.05	0.00996705555325088\\
599.06	0.00996791457106968\\
599.07	0.00996876854874041\\
599.08	0.00996961745928199\\
599.09	0.00997044779379745\\
599.1	0.00997125907535553\\
599.11	0.00997205095980902\\
599.12	0.00997282281992336\\
599.13	0.00997357449570154\\
599.14	0.00997430582442724\\
599.15	0.00997501664056067\\
599.16	0.00997570677562952\\
599.17	0.00997637605811486\\
599.18	0.00997702431333591\\
599.19	0.00997765136331818\\
599.2	0.00997825702666225\\
599.21	0.00997884164734969\\
599.22	0.00997940571276279\\
599.23	0.00997994904050891\\
599.24	0.00998047144445088\\
599.25	0.00998097350251616\\
599.26	0.00998145771758323\\
599.27	0.00998192391942239\\
599.28	0.00998238161802606\\
599.29	0.00998283374884875\\
599.3	0.00998328059264519\\
599.31	0.009983722405159\\
599.32	0.00998415915282055\\
599.33	0.0099845908027081\\
599.34	0.00998501732260721\\
599.35	0.00998543868107325\\
599.36	0.00998585484749724\\
599.37	0.00998626579217519\\
599.38	0.00998667148638112\\
599.39	0.00998707190244393\\
599.4	0.00998746701382841\\
599.41	0.00998785679522082\\
599.42	0.0099882412571037\\
599.43	0.00998862039174518\\
599.44	0.00998899417396288\\
599.45	0.00998936258018044\\
599.46	0.00998972558854081\\
599.47	0.00999008317902579\\
599.48	0.00999043533358202\\
599.49	0.00999078203429291\\
599.5	0.00999112326444559\\
599.51	0.00999145900953319\\
599.52	0.00999178925246696\\
599.53	0.00999211397373816\\
599.54	0.00999243312917881\\
599.55	0.00999274666306492\\
599.56	0.00999305451751694\\
599.57	0.00999335663412312\\
599.58	0.00999365295393464\\
599.59	0.00999394341746056\\
599.6	0.00999422796466266\\
599.61	0.00999450653495009\\
599.62	0.00999477906702091\\
599.63	0.00999504549894827\\
599.64	0.0099953057682215\\
599.65	0.00999555981174296\\
599.66	0.0099958075658224\\
599.67	0.00999604896616963\\
599.68	0.00999628394788682\\
599.69	0.00999651244546047\\
599.7	0.00999673439275298\\
599.71	0.0099969497229938\\
599.72	0.00999715836877016\\
599.73	0.00999736026201729\\
599.74	0.00999755533403083\\
599.75	0.00999774351546318\\
599.76	0.00999792473631449\\
599.77	0.00999809892592316\\
599.78	0.00999826601295585\\
599.79	0.00999842592539837\\
599.8	0.00999857859054469\\
599.81	0.00999872393498536\\
599.82	0.00999886188459496\\
599.83	0.00999899236451894\\
599.84	0.0099991152991595\\
599.85	0.00999923061216063\\
599.86	0.00999933822639223\\
599.87	0.00999943806393315\\
599.88	0.00999953004605316\\
599.89	0.00999961409319372\\
599.9	0.00999969012494744\\
599.91	0.00999975806003624\\
599.92	0.0099998178162879\\
599.93	0.00999986931061113\\
599.94	0.00999991245896887\\
599.95	0.00999994717634967\\
599.96	0.00999997337673722\\
599.97	0.00999999097307755\\
599.98	0.00999999987724406\\
599.99	0.01\\
600	0.01\\
};
\addplot [color=mycolor12,solid,forget plot]
  table[row sep=crcr]{%
0.01	0\\
1.01	0\\
2.01	0\\
3.01	0\\
4.01	0\\
5.01	0\\
6.01	0\\
7.01	0\\
8.01	0\\
9.01	0\\
10.01	0\\
11.01	0\\
12.01	0\\
13.01	0\\
14.01	0\\
15.01	0\\
16.01	0\\
17.01	0\\
18.01	0\\
19.01	0\\
20.01	0\\
21.01	0\\
22.01	0\\
23.01	0\\
24.01	0\\
25.01	0\\
26.01	0\\
27.01	0\\
28.01	0\\
29.01	0\\
30.01	0\\
31.01	0\\
32.01	0\\
33.01	0\\
34.01	0\\
35.01	0\\
36.01	0\\
37.01	0\\
38.01	0\\
39.01	0\\
40.01	0\\
41.01	0\\
42.01	0\\
43.01	0\\
44.01	0\\
45.01	0\\
46.01	0\\
47.01	0\\
48.01	0\\
49.01	0\\
50.01	0\\
51.01	0\\
52.01	0\\
53.01	0\\
54.01	0\\
55.01	0\\
56.01	0\\
57.01	0\\
58.01	0\\
59.01	0\\
60.01	0\\
61.01	0\\
62.01	0\\
63.01	0\\
64.01	0\\
65.01	0\\
66.01	0\\
67.01	0\\
68.01	0\\
69.01	0\\
70.01	0\\
71.01	0\\
72.01	0\\
73.01	0\\
74.01	0\\
75.01	0\\
76.01	0\\
77.01	0\\
78.01	0\\
79.01	0\\
80.01	0\\
81.01	0\\
82.01	0\\
83.01	0\\
84.01	0\\
85.01	0\\
86.01	0\\
87.01	0\\
88.01	0\\
89.01	0\\
90.01	0\\
91.01	0\\
92.01	0\\
93.01	0\\
94.01	0\\
95.01	0\\
96.01	0\\
97.01	0\\
98.01	0\\
99.01	0\\
100.01	0\\
101.01	0\\
102.01	0\\
103.01	0\\
104.01	0\\
105.01	0\\
106.01	0\\
107.01	0\\
108.01	0\\
109.01	0\\
110.01	0\\
111.01	0\\
112.01	0\\
113.01	0\\
114.01	0\\
115.01	0\\
116.01	0\\
117.01	0\\
118.01	0\\
119.01	0\\
120.01	0\\
121.01	0\\
122.01	0\\
123.01	0\\
124.01	0\\
125.01	0\\
126.01	0\\
127.01	0\\
128.01	0\\
129.01	0\\
130.01	0\\
131.01	0\\
132.01	0\\
133.01	0\\
134.01	0\\
135.01	0\\
136.01	0\\
137.01	0\\
138.01	0\\
139.01	0\\
140.01	0\\
141.01	0\\
142.01	0\\
143.01	0\\
144.01	0\\
145.01	0\\
146.01	0\\
147.01	0\\
148.01	0\\
149.01	0\\
150.01	0\\
151.01	0\\
152.01	0\\
153.01	0\\
154.01	0\\
155.01	0\\
156.01	0\\
157.01	0\\
158.01	0\\
159.01	0\\
160.01	0\\
161.01	0\\
162.01	0\\
163.01	0\\
164.01	0\\
165.01	0\\
166.01	0\\
167.01	0\\
168.01	0\\
169.01	0\\
170.01	0\\
171.01	0\\
172.01	0\\
173.01	0\\
174.01	0\\
175.01	0\\
176.01	0\\
177.01	0\\
178.01	0\\
179.01	0\\
180.01	0\\
181.01	0\\
182.01	0\\
183.01	0\\
184.01	0\\
185.01	0\\
186.01	0\\
187.01	0\\
188.01	0\\
189.01	0\\
190.01	0\\
191.01	0\\
192.01	0\\
193.01	0\\
194.01	0\\
195.01	0\\
196.01	0\\
197.01	0\\
198.01	0\\
199.01	0\\
200.01	0\\
201.01	0\\
202.01	0\\
203.01	0\\
204.01	0\\
205.01	0\\
206.01	0\\
207.01	0\\
208.01	0\\
209.01	0\\
210.01	0\\
211.01	0\\
212.01	0\\
213.01	0\\
214.01	0\\
215.01	0\\
216.01	0\\
217.01	0\\
218.01	0\\
219.01	0\\
220.01	0\\
221.01	0\\
222.01	0\\
223.01	0\\
224.01	0\\
225.01	0\\
226.01	0\\
227.01	0\\
228.01	0\\
229.01	0\\
230.01	0\\
231.01	0\\
232.01	0\\
233.01	0\\
234.01	0\\
235.01	0\\
236.01	0\\
237.01	0\\
238.01	0\\
239.01	0\\
240.01	0\\
241.01	0\\
242.01	0\\
243.01	0\\
244.01	0\\
245.01	0\\
246.01	0\\
247.01	0\\
248.01	0\\
249.01	0\\
250.01	0\\
251.01	0\\
252.01	0\\
253.01	0\\
254.01	0\\
255.01	0\\
256.01	0\\
257.01	0\\
258.01	0\\
259.01	0\\
260.01	0\\
261.01	0\\
262.01	0\\
263.01	0\\
264.01	0\\
265.01	0\\
266.01	0\\
267.01	0\\
268.01	0\\
269.01	0\\
270.01	0\\
271.01	0\\
272.01	0\\
273.01	0\\
274.01	0\\
275.01	0\\
276.01	0\\
277.01	0\\
278.01	0\\
279.01	0\\
280.01	0\\
281.01	0\\
282.01	0\\
283.01	0\\
284.01	0\\
285.01	0\\
286.01	0\\
287.01	0\\
288.01	0\\
289.01	0\\
290.01	0\\
291.01	0\\
292.01	0\\
293.01	0\\
294.01	0\\
295.01	0\\
296.01	0\\
297.01	0\\
298.01	0\\
299.01	0\\
300.01	0\\
301.01	0\\
302.01	0\\
303.01	0\\
304.01	0\\
305.01	0\\
306.01	0\\
307.01	0\\
308.01	0\\
309.01	0\\
310.01	0\\
311.01	0\\
312.01	0\\
313.01	0\\
314.01	0\\
315.01	0\\
316.01	0\\
317.01	0\\
318.01	0\\
319.01	0\\
320.01	0\\
321.01	0\\
322.01	0\\
323.01	0\\
324.01	0\\
325.01	0\\
326.01	0\\
327.01	0\\
328.01	0\\
329.01	0\\
330.01	0\\
331.01	0\\
332.01	0\\
333.01	0\\
334.01	0\\
335.01	0\\
336.01	0\\
337.01	0\\
338.01	0\\
339.01	0\\
340.01	0\\
341.01	0\\
342.01	0\\
343.01	0\\
344.01	0\\
345.01	0\\
346.01	0\\
347.01	0\\
348.01	0\\
349.01	0\\
350.01	0\\
351.01	0\\
352.01	0\\
353.01	0\\
354.01	0\\
355.01	0\\
356.01	0\\
357.01	0\\
358.01	0\\
359.01	0\\
360.01	0\\
361.01	0\\
362.01	0\\
363.01	0\\
364.01	0\\
365.01	0\\
366.01	0\\
367.01	0\\
368.01	0\\
369.01	0\\
370.01	0\\
371.01	0\\
372.01	0\\
373.01	0\\
374.01	0\\
375.01	0\\
376.01	0\\
377.01	0\\
378.01	0\\
379.01	0\\
380.01	0\\
381.01	0\\
382.01	0\\
383.01	0\\
384.01	0\\
385.01	0\\
386.01	0\\
387.01	0\\
388.01	0\\
389.01	0\\
390.01	0\\
391.01	0\\
392.01	0\\
393.01	0\\
394.01	0\\
395.01	0\\
396.01	0\\
397.01	0\\
398.01	0\\
399.01	0\\
400.01	0\\
401.01	0\\
402.01	0\\
403.01	0\\
404.01	0\\
405.01	0\\
406.01	0\\
407.01	0\\
408.01	0\\
409.01	0\\
410.01	0\\
411.01	0\\
412.01	0\\
413.01	0\\
414.01	0\\
415.01	0\\
416.01	0\\
417.01	0\\
418.01	0\\
419.01	0\\
420.01	0\\
421.01	0\\
422.01	0\\
423.01	0\\
424.01	0\\
425.01	0\\
426.01	0\\
427.01	0\\
428.01	0\\
429.01	0\\
430.01	0\\
431.01	0\\
432.01	0\\
433.01	0\\
434.01	0\\
435.01	0\\
436.01	0\\
437.01	0\\
438.01	0\\
439.01	0\\
440.01	0\\
441.01	0\\
442.01	0\\
443.01	0\\
444.01	0\\
445.01	0\\
446.01	0\\
447.01	0\\
448.01	0\\
449.01	0\\
450.01	0\\
451.01	0\\
452.01	0\\
453.01	0\\
454.01	0\\
455.01	0\\
456.01	0\\
457.01	0\\
458.01	0\\
459.01	0\\
460.01	0\\
461.01	0\\
462.01	0\\
463.01	0\\
464.01	0\\
465.01	0\\
466.01	0\\
467.01	0\\
468.01	0\\
469.01	0\\
470.01	0\\
471.01	0\\
472.01	0\\
473.01	0\\
474.01	0\\
475.01	0\\
476.01	0\\
477.01	0\\
478.01	0\\
479.01	0\\
480.01	0\\
481.01	0\\
482.01	0\\
483.01	0\\
484.01	0\\
485.01	0\\
486.01	0\\
487.01	0\\
488.01	0\\
489.01	0\\
490.01	0\\
491.01	0\\
492.01	0\\
493.01	0\\
494.01	0\\
495.01	0\\
496.01	0\\
497.01	0\\
498.01	0\\
499.01	0\\
500.01	0\\
501.01	0\\
502.01	0\\
503.01	0\\
504.01	0\\
505.01	0\\
506.01	0\\
507.01	0\\
508.01	0\\
509.01	0\\
510.01	0\\
511.01	0\\
512.01	0\\
513.01	0\\
514.01	0\\
515.01	0\\
516.01	0\\
517.01	0\\
518.01	0\\
519.01	0\\
520.01	0\\
521.01	0\\
522.01	0\\
523.01	0\\
524.01	0\\
525.01	0\\
526.01	0\\
527.01	0\\
528.01	0\\
529.01	0\\
530.01	0\\
531.01	0\\
532.01	0\\
533.01	0\\
534.01	0\\
535.01	0\\
536.01	0\\
537.01	0\\
538.01	0\\
539.01	0\\
540.01	0\\
541.01	0\\
542.01	0\\
543.01	0\\
544.01	0\\
545.01	0\\
546.01	0\\
547.01	0\\
548.01	0\\
549.01	0\\
550.01	1.76577693769484e-05\\
551.01	0.000163601594765743\\
552.01	0.000316552654936606\\
553.01	0.000477202157931674\\
554.01	0.000646343853199616\\
555.01	0.000824893090787756\\
556.01	0.00101390996811625\\
557.01	0.0012146275223706\\
558.01	0.00142848601941258\\
559.01	0.00165717309688124\\
560.01	0.00190263038585375\\
561.01	0.00216086094684066\\
562.01	0.00242801592023746\\
563.01	0.00270459001499955\\
564.01	0.00299112571883046\\
565.01	0.00328819303298395\\
566.01	0.00359638629634986\\
567.01	0.00391632018100313\\
568.01	0.00424862352259442\\
569.01	0.0045939301115959\\
570.01	0.00495286578552835\\
571.01	0.00532608000731284\\
572.01	0.00568646434170741\\
573.01	0.00588160416759255\\
574.01	0.00607494172253639\\
575.01	0.00626304604111949\\
576.01	0.00644372525962685\\
577.01	0.0066149467502195\\
578.01	0.00678349529972556\\
579.01	0.00695169599783461\\
580.01	0.00711915538111959\\
581.01	0.00728547230152028\\
582.01	0.00745025912556778\\
583.01	0.00761317424151823\\
584.01	0.00777397103082797\\
585.01	0.00793312457297383\\
586.01	0.00809207708383814\\
587.01	0.00825067850764279\\
588.01	0.0084088581555231\\
589.01	0.00856615818895188\\
590.01	0.00872204584445263\\
591.01	0.00887592796155797\\
592.01	0.00902715589244679\\
593.01	0.00917502985911509\\
594.01	0.00931880522180199\\
595.01	0.0094577020919197\\
596.01	0.00959091850522286\\
597.01	0.00971763773133404\\
598.01	0.00983693415442488\\
599.01	0.00994672806842535\\
599.02	0.00994775855856473\\
599.03	0.00994878730830944\\
599.04	0.00994981430099912\\
599.05	0.0099508395197916\\
599.06	0.00995186294794709\\
599.07	0.00995288456862204\\
599.08	0.00995390436481037\\
599.09	0.00995492232059677\\
599.1	0.00995593841996281\\
599.11	0.00995695264678405\\
599.12	0.00995796498486604\\
599.13	0.00995897541791202\\
599.14	0.00995998392953529\\
599.15	0.0099609905032724\\
599.16	0.00996199512259775\\
599.17	0.00996299777093928\\
599.18	0.00996399843169559\\
599.19	0.00996499708825443\\
599.2	0.00996599372401278\\
599.21	0.00996698779596926\\
599.22	0.00996797861736903\\
599.23	0.00996896616765423\\
599.24	0.00996995042631169\\
599.25	0.00997093060729154\\
599.26	0.00997190400413337\\
599.27	0.00997287058004875\\
599.28	0.00997382061001036\\
599.29	0.00997475095702614\\
599.3	0.00997566116920662\\
599.31	0.00997655082052863\\
599.32	0.0099774197729459\\
599.33	0.00997826788498202\\
599.34	0.00997909501386427\\
599.35	0.00997990101551993\\
599.36	0.00998068574457414\\
599.37	0.00998144905434973\\
599.38	0.00998219079686904\\
599.39	0.00998291082285798\\
599.4	0.0099836089817526\\
599.41	0.00998428512170819\\
599.42	0.00998493905288865\\
599.43	0.00998557060322691\\
599.44	0.00998617961841143\\
599.45	0.00998676594296101\\
599.46	0.0099873294202526\\
599.47	0.00998786989255357\\
599.48	0.00998838720105844\\
599.49	0.00998888190498132\\
599.5	0.00998935417965473\\
599.51	0.00998980387068191\\
599.52	0.0099902324022735\\
599.53	0.00999064107301789\\
599.54	0.00999103756789193\\
599.55	0.0099914252526318\\
599.56	0.0099918045563071\\
599.57	0.00999217544917753\\
599.58	0.00999253790543692\\
599.59	0.00999289190353504\\
599.6	0.00999323742652058\\
599.61	0.00999357446240663\\
599.62	0.00999390303937479\\
599.63	0.00999422316746411\\
599.64	0.00999453485133593\\
599.65	0.00999483808574109\\
599.66	0.00999513286499583\\
599.67	0.00999541918934847\\
599.68	0.00999569706543861\\
599.69	0.00999596650678687\\
599.7	0.009996227534317\\
599.71	0.00999648017691256\\
599.72	0.00999672447201024\\
599.73	0.0099969604662323\\
599.74	0.00999718812015448\\
599.75	0.00999740737458996\\
599.76	0.00999761817122877\\
599.77	0.00999782045273265\\
599.78	0.00999801416283588\\
599.79	0.00999819924009764\\
599.8	0.00999837562306064\\
599.81	0.00999854325056222\\
599.82	0.00999870206199949\\
599.83	0.00999885199763102\\
599.84	0.00999899299866258\\
599.85	0.00999912500733794\\
599.86	0.009999247967035\\
599.87	0.0099993618223675\\
599.88	0.00999946651929288\\
599.89	0.0099995620052265\\
599.9	0.00999964822916273\\
599.91	0.00999972514180333\\
599.92	0.00999979269569369\\
599.93	0.00999985084536732\\
599.94	0.00999989954749936\\
599.95	0.00999993876106953\\
599.96	0.00999996844753539\\
599.97	0.00999998857101644\\
599.98	0.00999999909849005\\
599.99	0.01\\
600	0.01\\
};
\addplot [color=mycolor13,solid,forget plot]
  table[row sep=crcr]{%
0.01	0\\
1.01	0\\
2.01	0\\
3.01	0\\
4.01	0\\
5.01	0\\
6.01	0\\
7.01	0\\
8.01	0\\
9.01	0\\
10.01	0\\
11.01	0\\
12.01	0\\
13.01	0\\
14.01	0\\
15.01	0\\
16.01	0\\
17.01	0\\
18.01	0\\
19.01	0\\
20.01	0\\
21.01	0\\
22.01	0\\
23.01	0\\
24.01	0\\
25.01	0\\
26.01	0\\
27.01	0\\
28.01	0\\
29.01	0\\
30.01	0\\
31.01	0\\
32.01	0\\
33.01	0\\
34.01	0\\
35.01	0\\
36.01	0\\
37.01	0\\
38.01	0\\
39.01	0\\
40.01	0\\
41.01	0\\
42.01	0\\
43.01	0\\
44.01	0\\
45.01	0\\
46.01	0\\
47.01	0\\
48.01	0\\
49.01	0\\
50.01	0\\
51.01	0\\
52.01	0\\
53.01	0\\
54.01	0\\
55.01	0\\
56.01	0\\
57.01	0\\
58.01	0\\
59.01	0\\
60.01	0\\
61.01	0\\
62.01	0\\
63.01	0\\
64.01	0\\
65.01	0\\
66.01	0\\
67.01	0\\
68.01	0\\
69.01	0\\
70.01	0\\
71.01	0\\
72.01	0\\
73.01	0\\
74.01	0\\
75.01	0\\
76.01	0\\
77.01	0\\
78.01	0\\
79.01	0\\
80.01	0\\
81.01	0\\
82.01	0\\
83.01	0\\
84.01	0\\
85.01	0\\
86.01	0\\
87.01	0\\
88.01	0\\
89.01	0\\
90.01	0\\
91.01	0\\
92.01	0\\
93.01	0\\
94.01	0\\
95.01	0\\
96.01	0\\
97.01	0\\
98.01	0\\
99.01	0\\
100.01	0\\
101.01	0\\
102.01	0\\
103.01	0\\
104.01	0\\
105.01	0\\
106.01	0\\
107.01	0\\
108.01	0\\
109.01	0\\
110.01	0\\
111.01	0\\
112.01	0\\
113.01	0\\
114.01	0\\
115.01	0\\
116.01	0\\
117.01	0\\
118.01	0\\
119.01	0\\
120.01	0\\
121.01	0\\
122.01	0\\
123.01	0\\
124.01	0\\
125.01	0\\
126.01	0\\
127.01	0\\
128.01	0\\
129.01	0\\
130.01	0\\
131.01	0\\
132.01	0\\
133.01	0\\
134.01	0\\
135.01	0\\
136.01	0\\
137.01	0\\
138.01	0\\
139.01	0\\
140.01	0\\
141.01	0\\
142.01	0\\
143.01	0\\
144.01	0\\
145.01	0\\
146.01	0\\
147.01	0\\
148.01	0\\
149.01	0\\
150.01	0\\
151.01	0\\
152.01	0\\
153.01	0\\
154.01	0\\
155.01	0\\
156.01	0\\
157.01	0\\
158.01	0\\
159.01	0\\
160.01	0\\
161.01	0\\
162.01	0\\
163.01	0\\
164.01	0\\
165.01	0\\
166.01	0\\
167.01	0\\
168.01	0\\
169.01	0\\
170.01	0\\
171.01	0\\
172.01	0\\
173.01	0\\
174.01	0\\
175.01	0\\
176.01	0\\
177.01	0\\
178.01	0\\
179.01	0\\
180.01	0\\
181.01	0\\
182.01	0\\
183.01	0\\
184.01	0\\
185.01	0\\
186.01	0\\
187.01	0\\
188.01	0\\
189.01	0\\
190.01	0\\
191.01	0\\
192.01	0\\
193.01	0\\
194.01	0\\
195.01	0\\
196.01	0\\
197.01	0\\
198.01	0\\
199.01	0\\
200.01	0\\
201.01	0\\
202.01	0\\
203.01	0\\
204.01	0\\
205.01	0\\
206.01	0\\
207.01	0\\
208.01	0\\
209.01	0\\
210.01	0\\
211.01	0\\
212.01	0\\
213.01	0\\
214.01	0\\
215.01	0\\
216.01	0\\
217.01	0\\
218.01	0\\
219.01	0\\
220.01	0\\
221.01	0\\
222.01	0\\
223.01	0\\
224.01	0\\
225.01	0\\
226.01	0\\
227.01	0\\
228.01	0\\
229.01	0\\
230.01	0\\
231.01	0\\
232.01	0\\
233.01	0\\
234.01	0\\
235.01	0\\
236.01	0\\
237.01	0\\
238.01	0\\
239.01	0\\
240.01	0\\
241.01	0\\
242.01	0\\
243.01	0\\
244.01	0\\
245.01	0\\
246.01	0\\
247.01	0\\
248.01	0\\
249.01	0\\
250.01	0\\
251.01	0\\
252.01	0\\
253.01	0\\
254.01	0\\
255.01	0\\
256.01	0\\
257.01	0\\
258.01	0\\
259.01	0\\
260.01	0\\
261.01	0\\
262.01	0\\
263.01	0\\
264.01	0\\
265.01	0\\
266.01	0\\
267.01	0\\
268.01	0\\
269.01	0\\
270.01	0\\
271.01	0\\
272.01	0\\
273.01	0\\
274.01	0\\
275.01	0\\
276.01	0\\
277.01	0\\
278.01	0\\
279.01	0\\
280.01	0\\
281.01	0\\
282.01	0\\
283.01	0\\
284.01	0\\
285.01	0\\
286.01	0\\
287.01	0\\
288.01	0\\
289.01	0\\
290.01	0\\
291.01	0\\
292.01	0\\
293.01	0\\
294.01	0\\
295.01	0\\
296.01	0\\
297.01	0\\
298.01	0\\
299.01	0\\
300.01	0\\
301.01	0\\
302.01	0\\
303.01	0\\
304.01	0\\
305.01	0\\
306.01	0\\
307.01	0\\
308.01	0\\
309.01	0\\
310.01	0\\
311.01	0\\
312.01	0\\
313.01	0\\
314.01	0\\
315.01	0\\
316.01	0\\
317.01	0\\
318.01	0\\
319.01	0\\
320.01	0\\
321.01	0\\
322.01	0\\
323.01	0\\
324.01	0\\
325.01	0\\
326.01	0\\
327.01	0\\
328.01	0\\
329.01	0\\
330.01	0\\
331.01	0\\
332.01	0\\
333.01	0\\
334.01	0\\
335.01	0\\
336.01	0\\
337.01	0\\
338.01	0\\
339.01	0\\
340.01	0\\
341.01	0\\
342.01	0\\
343.01	0\\
344.01	0\\
345.01	0\\
346.01	0\\
347.01	0\\
348.01	0\\
349.01	0\\
350.01	0\\
351.01	0\\
352.01	0\\
353.01	0\\
354.01	0\\
355.01	0\\
356.01	0\\
357.01	0\\
358.01	0\\
359.01	0\\
360.01	0\\
361.01	0\\
362.01	0\\
363.01	0\\
364.01	0\\
365.01	0\\
366.01	0\\
367.01	0\\
368.01	0\\
369.01	0\\
370.01	0\\
371.01	0\\
372.01	0\\
373.01	0\\
374.01	0\\
375.01	0\\
376.01	0\\
377.01	0\\
378.01	0\\
379.01	0\\
380.01	0\\
381.01	0\\
382.01	0\\
383.01	0\\
384.01	0\\
385.01	0\\
386.01	0\\
387.01	0\\
388.01	0\\
389.01	0\\
390.01	0\\
391.01	0\\
392.01	0\\
393.01	0\\
394.01	0\\
395.01	0\\
396.01	0\\
397.01	0\\
398.01	0\\
399.01	0\\
400.01	0\\
401.01	0\\
402.01	0\\
403.01	0\\
404.01	0\\
405.01	0\\
406.01	0\\
407.01	0\\
408.01	0\\
409.01	0\\
410.01	0\\
411.01	0\\
412.01	0\\
413.01	0\\
414.01	0\\
415.01	0\\
416.01	0\\
417.01	0\\
418.01	0\\
419.01	0\\
420.01	0\\
421.01	0\\
422.01	0\\
423.01	0\\
424.01	0\\
425.01	0\\
426.01	0\\
427.01	0\\
428.01	0\\
429.01	0\\
430.01	0\\
431.01	0\\
432.01	0\\
433.01	0\\
434.01	0\\
435.01	0\\
436.01	0\\
437.01	0\\
438.01	0\\
439.01	0\\
440.01	0\\
441.01	0\\
442.01	0\\
443.01	0\\
444.01	0\\
445.01	0\\
446.01	0\\
447.01	0\\
448.01	0\\
449.01	0\\
450.01	0\\
451.01	0\\
452.01	0\\
453.01	0\\
454.01	0\\
455.01	0\\
456.01	0\\
457.01	0\\
458.01	0\\
459.01	0\\
460.01	0\\
461.01	0\\
462.01	0\\
463.01	0\\
464.01	0\\
465.01	0\\
466.01	0\\
467.01	0\\
468.01	0\\
469.01	0\\
470.01	0\\
471.01	0\\
472.01	0\\
473.01	0\\
474.01	0\\
475.01	0\\
476.01	0\\
477.01	0\\
478.01	0\\
479.01	0\\
480.01	0\\
481.01	0\\
482.01	0\\
483.01	0\\
484.01	0\\
485.01	0\\
486.01	0\\
487.01	0\\
488.01	0\\
489.01	0\\
490.01	0\\
491.01	0\\
492.01	0\\
493.01	0\\
494.01	0\\
495.01	0\\
496.01	0\\
497.01	0\\
498.01	0\\
499.01	0\\
500.01	0\\
501.01	0\\
502.01	0\\
503.01	0\\
504.01	0\\
505.01	0\\
506.01	0\\
507.01	0\\
508.01	0\\
509.01	0\\
510.01	0\\
511.01	0\\
512.01	0\\
513.01	0\\
514.01	0\\
515.01	0\\
516.01	0\\
517.01	0\\
518.01	0\\
519.01	0\\
520.01	0\\
521.01	0\\
522.01	0\\
523.01	0\\
524.01	0\\
525.01	0\\
526.01	0\\
527.01	0\\
528.01	0\\
529.01	0\\
530.01	0\\
531.01	0\\
532.01	0\\
533.01	0\\
534.01	0\\
535.01	0\\
536.01	0\\
537.01	0\\
538.01	0\\
539.01	0\\
540.01	0\\
541.01	0\\
542.01	0\\
543.01	0\\
544.01	0\\
545.01	0\\
546.01	0\\
547.01	0\\
548.01	0\\
549.01	0\\
550.01	0\\
551.01	0\\
552.01	0\\
553.01	0\\
554.01	0\\
555.01	0\\
556.01	0\\
557.01	0\\
558.01	0\\
559.01	0\\
560.01	0\\
561.01	0\\
562.01	0\\
563.01	0\\
564.01	0\\
565.01	0\\
566.01	0\\
567.01	0\\
568.01	0\\
569.01	0\\
570.01	0\\
571.01	0\\
572.01	2.7816904777541e-05\\
573.01	0.000234469092053179\\
574.01	0.000454399221483506\\
575.01	0.000689399833623931\\
576.01	0.0009415921916204\\
577.01	0.00121282930990188\\
578.01	0.00149614830095804\\
579.01	0.00178923528638872\\
580.01	0.00209247197993115\\
581.01	0.00240619622796399\\
582.01	0.00273067721320507\\
583.01	0.00306608229716627\\
584.01	0.00341243284835259\\
585.01	0.00376898156286783\\
586.01	0.00413388772680092\\
587.01	0.00450662171783373\\
588.01	0.00488671348865318\\
589.01	0.00527426155529603\\
590.01	0.0056693963527141\\
591.01	0.00607223737509095\\
592.01	0.00648288259282408\\
593.01	0.00690139908379193\\
594.01	0.00732781618824651\\
595.01	0.00776211870274235\\
596.01	0.00820423876877381\\
597.01	0.00865403604580581\\
598.01	0.0091111651447944\\
599.01	0.00957383789362019\\
599.02	0.0095784699445533\\
599.03	0.00958310152356088\\
599.04	0.009587732601833\\
599.05	0.00959236314982192\\
599.06	0.00959699313722105\\
599.07	0.00960162253294322\\
599.08	0.00960625130509822\\
599.09	0.00961087942096949\\
599.1	0.00961550684699015\\
599.11	0.00962013354871818\\
599.12	0.00962475949081084\\
599.13	0.00962938463699814\\
599.14	0.00963400895005545\\
599.15	0.00963863239177514\\
599.16	0.00964325492293724\\
599.17	0.00964787650327905\\
599.18	0.00965249709146373\\
599.19	0.00965711664504773\\
599.2	0.00966173512044705\\
599.21	0.00966635247295013\\
599.22	0.00967096865669379\\
599.23	0.0096755836245642\\
599.24	0.00968019732815788\\
599.25	0.00968480971781115\\
599.26	0.00968942074273276\\
599.27	0.00969403035071384\\
599.28	0.00969863848897433\\
599.29	0.00970324510350724\\
599.3	0.00970785013878026\\
599.31	0.00971245353768087\\
599.32	0.0097170552414366\\
599.33	0.00972165518956062\\
599.34	0.00972625331979421\\
599.35	0.00973084956804691\\
599.36	0.00973544386833419\\
599.37	0.00974003615271259\\
599.38	0.00974462635121213\\
599.39	0.00974921439176584\\
599.4	0.00975380020013637\\
599.41	0.00975838369983937\\
599.42	0.00976296481206596\\
599.43	0.00976754345559703\\
599.44	0.00977211954671424\\
599.45	0.00977669299910894\\
599.46	0.00978126372378694\\
599.47	0.00978583162896891\\
599.48	0.00979039661998624\\
599.49	0.00979495788351871\\
599.5	0.00979951498770784\\
599.51	0.00980406782641168\\
599.52	0.00980861471572478\\
599.53	0.00981315408788168\\
599.54	0.00981767801023239\\
599.55	0.00982218287897037\\
599.56	0.00982666802718657\\
599.57	0.00983113324532572\\
599.58	0.00983557831854249\\
599.59	0.00984000302649006\\
599.6	0.00984440714309827\\
599.61	0.00984879043634082\\
599.62	0.00985315263350748\\
599.63	0.00985749347945079\\
599.64	0.00986181272388278\\
599.65	0.00986611010955735\\
599.66	0.00987038537198554\\
599.67	0.00987463823912518\\
599.68	0.00987886843105376\\
599.69	0.00988307565962328\\
599.7	0.00988725962809584\\
599.71	0.00989142003075855\\
599.72	0.00989555655251621\\
599.73	0.00989966886846011\\
599.74	0.00990375664357885\\
599.75	0.00990781953233169\\
599.76	0.00991185717815917\\
599.77	0.00991586921296373\\
599.78	0.00991985525655799\\
599.79	0.00992381491609016\\
599.8	0.00992774778542166\\
599.81	0.00993165344446322\\
599.82	0.00993553145846567\\
599.83	0.00993938137726134\\
599.84	0.00994320273445221\\
599.85	0.00994699504653957\\
599.86	0.00995075781198986\\
599.87	0.00995449051023017\\
599.88	0.00995819260056652\\
599.89	0.00996186352101685\\
599.9	0.00996550268704979\\
599.91	0.00996910949021901\\
599.92	0.00997268329668161\\
599.93	0.00997622344558731\\
599.94	0.00997972924732342\\
599.95	0.00998319998159831\\
599.96	0.00998663489534342\\
599.97	0.0099900332004109\\
599.98	0.00999339407104016\\
599.99	0.00999671664106234\\
600	0.01\\
};
\addplot [color=mycolor14,solid,forget plot]
  table[row sep=crcr]{%
0.01	0.01\\
1.01	0.01\\
2.01	0.01\\
3.01	0.01\\
4.01	0.01\\
5.01	0.01\\
6.01	0.01\\
7.01	0.01\\
8.01	0.01\\
9.01	0.01\\
10.01	0.01\\
11.01	0.01\\
12.01	0.01\\
13.01	0.01\\
14.01	0.01\\
15.01	0.01\\
16.01	0.01\\
17.01	0.01\\
18.01	0.01\\
19.01	0.01\\
20.01	0.01\\
21.01	0.01\\
22.01	0.01\\
23.01	0.01\\
24.01	0.01\\
25.01	0.01\\
26.01	0.01\\
27.01	0.01\\
28.01	0.01\\
29.01	0.01\\
30.01	0.01\\
31.01	0.01\\
32.01	0.01\\
33.01	0.01\\
34.01	0.01\\
35.01	0.01\\
36.01	0.01\\
37.01	0.01\\
38.01	0.01\\
39.01	0.01\\
40.01	0.01\\
41.01	0.01\\
42.01	0.01\\
43.01	0.01\\
44.01	0.01\\
45.01	0.01\\
46.01	0.01\\
47.01	0.01\\
48.01	0.01\\
49.01	0.01\\
50.01	0.01\\
51.01	0.01\\
52.01	0.01\\
53.01	0.01\\
54.01	0.01\\
55.01	0.01\\
56.01	0.01\\
57.01	0.01\\
58.01	0.01\\
59.01	0.01\\
60.01	0.01\\
61.01	0.01\\
62.01	0.01\\
63.01	0.01\\
64.01	0.01\\
65.01	0.01\\
66.01	0.01\\
67.01	0.01\\
68.01	0.01\\
69.01	0.01\\
70.01	0.01\\
71.01	0.01\\
72.01	0.01\\
73.01	0.01\\
74.01	0.01\\
75.01	0.01\\
76.01	0.01\\
77.01	0.01\\
78.01	0.01\\
79.01	0.01\\
80.01	0.01\\
81.01	0.01\\
82.01	0.01\\
83.01	0.01\\
84.01	0.01\\
85.01	0.01\\
86.01	0.01\\
87.01	0.01\\
88.01	0.01\\
89.01	0.01\\
90.01	0.01\\
91.01	0.01\\
92.01	0.01\\
93.01	0.01\\
94.01	0.01\\
95.01	0.01\\
96.01	0.01\\
97.01	0.01\\
98.01	0.01\\
99.01	0.01\\
100.01	0.01\\
101.01	0.01\\
102.01	0.01\\
103.01	0.01\\
104.01	0.01\\
105.01	0.01\\
106.01	0.01\\
107.01	0.01\\
108.01	0.01\\
109.01	0.01\\
110.01	0.01\\
111.01	0.01\\
112.01	0.01\\
113.01	0.01\\
114.01	0.01\\
115.01	0.01\\
116.01	0.01\\
117.01	0.01\\
118.01	0.01\\
119.01	0.01\\
120.01	0.01\\
121.01	0.01\\
122.01	0.01\\
123.01	0.01\\
124.01	0.01\\
125.01	0.01\\
126.01	0.01\\
127.01	0.01\\
128.01	0.01\\
129.01	0.01\\
130.01	0.01\\
131.01	0.01\\
132.01	0.01\\
133.01	0.01\\
134.01	0.01\\
135.01	0.01\\
136.01	0.01\\
137.01	0.01\\
138.01	0.01\\
139.01	0.01\\
140.01	0.01\\
141.01	0.01\\
142.01	0.01\\
143.01	0.01\\
144.01	0.01\\
145.01	0.01\\
146.01	0.01\\
147.01	0.01\\
148.01	0.01\\
149.01	0.01\\
150.01	0.01\\
151.01	0.01\\
152.01	0.01\\
153.01	0.01\\
154.01	0.01\\
155.01	0.01\\
156.01	0.01\\
157.01	0.01\\
158.01	0.01\\
159.01	0.01\\
160.01	0.01\\
161.01	0.01\\
162.01	0.01\\
163.01	0.01\\
164.01	0.01\\
165.01	0.01\\
166.01	0.01\\
167.01	0.01\\
168.01	0.01\\
169.01	0.01\\
170.01	0.01\\
171.01	0.01\\
172.01	0.01\\
173.01	0.01\\
174.01	0.01\\
175.01	0.01\\
176.01	0.01\\
177.01	0.01\\
178.01	0.01\\
179.01	0.01\\
180.01	0.01\\
181.01	0.01\\
182.01	0.01\\
183.01	0.01\\
184.01	0.01\\
185.01	0.01\\
186.01	0.01\\
187.01	0.01\\
188.01	0.01\\
189.01	0.01\\
190.01	0.01\\
191.01	0.01\\
192.01	0.01\\
193.01	0.01\\
194.01	0.01\\
195.01	0.01\\
196.01	0.01\\
197.01	0.01\\
198.01	0.01\\
199.01	0.01\\
200.01	0.01\\
201.01	0.01\\
202.01	0.01\\
203.01	0.01\\
204.01	0.01\\
205.01	0.01\\
206.01	0.01\\
207.01	0.01\\
208.01	0.01\\
209.01	0.01\\
210.01	0.01\\
211.01	0.01\\
212.01	0.01\\
213.01	0.01\\
214.01	0.01\\
215.01	0.01\\
216.01	0.01\\
217.01	0.01\\
218.01	0.01\\
219.01	0.01\\
220.01	0.01\\
221.01	0.01\\
222.01	0.01\\
223.01	0.01\\
224.01	0.01\\
225.01	0.01\\
226.01	0.01\\
227.01	0.01\\
228.01	0.01\\
229.01	0.01\\
230.01	0.01\\
231.01	0.01\\
232.01	0.01\\
233.01	0.01\\
234.01	0.01\\
235.01	0.01\\
236.01	0.01\\
237.01	0.01\\
238.01	0.01\\
239.01	0.01\\
240.01	0.01\\
241.01	0.01\\
242.01	0.01\\
243.01	0.01\\
244.01	0.01\\
245.01	0.01\\
246.01	0.01\\
247.01	0.01\\
248.01	0.01\\
249.01	0.01\\
250.01	0.01\\
251.01	0.01\\
252.01	0.01\\
253.01	0.01\\
254.01	0.01\\
255.01	0.01\\
256.01	0.01\\
257.01	0.01\\
258.01	0.01\\
259.01	0.01\\
260.01	0.01\\
261.01	0.01\\
262.01	0.01\\
263.01	0.01\\
264.01	0.01\\
265.01	0.01\\
266.01	0.01\\
267.01	0.01\\
268.01	0.01\\
269.01	0.01\\
270.01	0.01\\
271.01	0.01\\
272.01	0.01\\
273.01	0.01\\
274.01	0.01\\
275.01	0.01\\
276.01	0.01\\
277.01	0.01\\
278.01	0.01\\
279.01	0.01\\
280.01	0.01\\
281.01	0.01\\
282.01	0.01\\
283.01	0.01\\
284.01	0.01\\
285.01	0.01\\
286.01	0.01\\
287.01	0.01\\
288.01	0.01\\
289.01	0.01\\
290.01	0.01\\
291.01	0.01\\
292.01	0.01\\
293.01	0.01\\
294.01	0.01\\
295.01	0.01\\
296.01	0.01\\
297.01	0.01\\
298.01	0.01\\
299.01	0.01\\
300.01	0.01\\
301.01	0.01\\
302.01	0.01\\
303.01	0.01\\
304.01	0.01\\
305.01	0.01\\
306.01	0.01\\
307.01	0.01\\
308.01	0.01\\
309.01	0.01\\
310.01	0.01\\
311.01	0.01\\
312.01	0.01\\
313.01	0.01\\
314.01	0.01\\
315.01	0.01\\
316.01	0.01\\
317.01	0.01\\
318.01	0.01\\
319.01	0.01\\
320.01	0.01\\
321.01	0.01\\
322.01	0.01\\
323.01	0.01\\
324.01	0.01\\
325.01	0.01\\
326.01	0.01\\
327.01	0.01\\
328.01	0.01\\
329.01	0.01\\
330.01	0.01\\
331.01	0.01\\
332.01	0.01\\
333.01	0.01\\
334.01	0.01\\
335.01	0.01\\
336.01	0.01\\
337.01	0.01\\
338.01	0.01\\
339.01	0.01\\
340.01	0.01\\
341.01	0.01\\
342.01	0.01\\
343.01	0.01\\
344.01	0.01\\
345.01	0.01\\
346.01	0.01\\
347.01	0.01\\
348.01	0.01\\
349.01	0.01\\
350.01	0.01\\
351.01	0.01\\
352.01	0.01\\
353.01	0.01\\
354.01	0.01\\
355.01	0.01\\
356.01	0.01\\
357.01	0.01\\
358.01	0.01\\
359.01	0.01\\
360.01	0.01\\
361.01	0.01\\
362.01	0.01\\
363.01	0.01\\
364.01	0.01\\
365.01	0.01\\
366.01	0.01\\
367.01	0.01\\
368.01	0.01\\
369.01	0.01\\
370.01	0.01\\
371.01	0.01\\
372.01	0.01\\
373.01	0.01\\
374.01	0.01\\
375.01	0.01\\
376.01	0.01\\
377.01	0.01\\
378.01	0.01\\
379.01	0.01\\
380.01	0.01\\
381.01	0.01\\
382.01	0.01\\
383.01	0.01\\
384.01	0.01\\
385.01	0.01\\
386.01	0.01\\
387.01	0.01\\
388.01	0.01\\
389.01	0.01\\
390.01	0.01\\
391.01	0.01\\
392.01	0.01\\
393.01	0.01\\
394.01	0.01\\
395.01	0.01\\
396.01	0.01\\
397.01	0.01\\
398.01	0.01\\
399.01	0.01\\
400.01	0.01\\
401.01	0.01\\
402.01	0.01\\
403.01	0.01\\
404.01	0.01\\
405.01	0.01\\
406.01	0.01\\
407.01	0.01\\
408.01	0.01\\
409.01	0.01\\
410.01	0.01\\
411.01	0.01\\
412.01	0.01\\
413.01	0.01\\
414.01	0.01\\
415.01	0.01\\
416.01	0.01\\
417.01	0.01\\
418.01	0.01\\
419.01	0.01\\
420.01	0.01\\
421.01	0.01\\
422.01	0.01\\
423.01	0.01\\
424.01	0.01\\
425.01	0.01\\
426.01	0.01\\
427.01	0.01\\
428.01	0.01\\
429.01	0.01\\
430.01	0.01\\
431.01	0.01\\
432.01	0.01\\
433.01	0.01\\
434.01	0.01\\
435.01	0.01\\
436.01	0.01\\
437.01	0.01\\
438.01	0.01\\
439.01	0.01\\
440.01	0.01\\
441.01	0.01\\
442.01	0.01\\
443.01	0.01\\
444.01	0.01\\
445.01	0.01\\
446.01	0.01\\
447.01	0.01\\
448.01	0.01\\
449.01	0.01\\
450.01	0.01\\
451.01	0.01\\
452.01	0.01\\
453.01	0.01\\
454.01	0.01\\
455.01	0.01\\
456.01	0.01\\
457.01	0.01\\
458.01	0.01\\
459.01	0.01\\
460.01	0.01\\
461.01	0.01\\
462.01	0.01\\
463.01	0.01\\
464.01	0.01\\
465.01	0.01\\
466.01	0.01\\
467.01	0.01\\
468.01	0.01\\
469.01	0.01\\
470.01	0.01\\
471.01	0.01\\
472.01	0.01\\
473.01	0.01\\
474.01	0.01\\
475.01	0.01\\
476.01	0.01\\
477.01	0.01\\
478.01	0.01\\
479.01	0.01\\
480.01	0.01\\
481.01	0.01\\
482.01	0.01\\
483.01	0.01\\
484.01	0.01\\
485.01	0.01\\
486.01	0.01\\
487.01	0.01\\
488.01	0.01\\
489.01	0.01\\
490.01	0.01\\
491.01	0.01\\
492.01	0.01\\
493.01	0.01\\
494.01	0.01\\
495.01	0.01\\
496.01	0.01\\
497.01	0.01\\
498.01	0.01\\
499.01	0.01\\
500.01	0.01\\
501.01	0.01\\
502.01	0.01\\
503.01	0.01\\
504.01	0.01\\
505.01	0.01\\
506.01	0.01\\
507.01	0.01\\
508.01	0.01\\
509.01	0.01\\
510.01	0.01\\
511.01	0.01\\
512.01	0.01\\
513.01	0.01\\
514.01	0.01\\
515.01	0.01\\
516.01	0.01\\
517.01	0.01\\
518.01	0.01\\
519.01	0.01\\
520.01	0.01\\
521.01	0.01\\
522.01	0.01\\
523.01	0.01\\
524.01	0.01\\
525.01	0.01\\
526.01	0.01\\
527.01	0.01\\
528.01	0.01\\
529.01	0.01\\
530.01	0.01\\
531.01	0.01\\
532.01	0.01\\
533.01	0.01\\
534.01	0.01\\
535.01	0.01\\
536.01	0.01\\
537.01	0.01\\
538.01	0.01\\
539.01	0.01\\
540.01	0.01\\
541.01	0.01\\
542.01	0.01\\
543.01	0.01\\
544.01	0.01\\
545.01	0.01\\
546.01	0.01\\
547.01	0.01\\
548.01	0.01\\
549.01	0.01\\
550.01	0.01\\
551.01	0.01\\
552.01	0.01\\
553.01	0.01\\
554.01	0.01\\
555.01	0.01\\
556.01	0.01\\
557.01	0.01\\
558.01	0.01\\
559.01	0.01\\
560.01	0.01\\
561.01	0.01\\
562.01	0.01\\
563.01	0.01\\
564.01	0.01\\
565.01	0.01\\
566.01	0.01\\
567.01	0.01\\
568.01	0.01\\
569.01	0.01\\
570.01	0.01\\
571.01	0.01\\
572.01	0.01\\
573.01	0.00980894049358069\\
574.01	0.00959285206314057\\
575.01	0.00936137092261357\\
576.01	0.00911232543663533\\
577.01	0.00884392085869094\\
578.01	0.00856331316716572\\
579.01	0.00827261631885943\\
580.01	0.00797142148655063\\
581.01	0.00765935959990316\\
582.01	0.00733612494795841\\
583.01	0.00700150675312477\\
584.01	0.00665543123614698\\
585.01	0.00629825648550703\\
586.01	0.00593194486745375\\
587.01	0.0055575347412609\\
588.01	0.00517546743094326\\
589.01	0.00478557032423368\\
590.01	0.00438767896675604\\
591.01	0.00398164747908303\\
592.01	0.00356735318276342\\
593.01	0.00314470158774184\\
594.01	0.00271363155295263\\
595.01	0.00227412039702548\\
596.01	0.00182618962595358\\
597.01	0.00136992176993503\\
598.01	0.000905592002350849\\
599.01	0.00043492981576735\\
599.02	0.000430213974982772\\
599.03	0.000425498536119657\\
599.04	0.000420783528401717\\
599.05	0.000416068981802759\\
599.06	0.000411354927068031\\
599.07	0.000406641395736292\\
599.08	0.000401928420162703\\
599.09	0.000397216033542409\\
599.1	0.000392504269934873\\
599.11	0.000387793164289036\\
599.12	0.000383082752469323\\
599.13	0.000378373071282544\\
599.14	0.000373664158505704\\
599.15	0.000368956052914794\\
599.16	0.000364248794314553\\
599.17	0.000359542423569297\\
599.18	0.000354836982634823\\
599.19	0.000350132514591449\\
599.2	0.000345429063678236\\
599.21	0.000340726675289876\\
599.22	0.000336025395992838\\
599.23	0.000331325273624394\\
599.24	0.000326626357332171\\
599.25	0.000321928697615173\\
599.26	0.000317232346366351\\
599.27	0.000312537356328739\\
599.28	0.000307843780926497\\
599.29	0.000303151675113411\\
599.3	0.000298461095422742\\
599.31	0.000293772099994554\\
599.32	0.000289084748632946\\
599.33	0.000284399102886653\\
599.34	0.000279715226107495\\
599.35	0.000275033183508976\\
599.36	0.000270353042232248\\
599.37	0.000265674871413527\\
599.38	0.000260998742252962\\
599.39	0.00025632472808642\\
599.4	0.000251652904460315\\
599.41	0.000246983349209667\\
599.42	0.000242316142539528\\
599.43	0.000237651367110008\\
599.44	0.000232989108125029\\
599.45	0.000228329453425088\\
599.46	0.000223672493584178\\
599.47	0.000219018322011167\\
599.48	0.000214367035055838\\
599.49	0.000209719387470494\\
599.5	0.000205075830177353\\
599.51	0.00020043647066856\\
599.52	0.000195801419884652\\
599.53	0.000191173145721217\\
599.54	0.000186563422008259\\
599.55	0.000181972450344473\\
599.56	0.000177400620217833\\
599.57	0.00017284847721002\\
599.58	0.000168316234915521\\
599.59	0.000163804156364234\\
599.6	0.00015931246726116\\
599.61	0.000154841397385808\\
599.62	0.000150391182695489\\
599.63	0.000145962065581585\\
599.64	0.000141554295139023\\
599.65	0.000137168127449856\\
599.66	0.000132803825881878\\
599.67	0.000128461661373839\\
599.68	0.000124141912752868\\
599.69	0.000119844867080403\\
599.7	0.00011557082001749\\
599.71	0.00011132007621082\\
599.72	0.000107092949700736\\
599.73	0.000102889764201437\\
599.74	9.87108534851212e-05\\
599.75	9.45565618428821e-05\\
599.76	9.04272445685847e-05\\
599.77	8.6323268469345e-05\\
599.78	8.22450124151018e-05\\
599.79	7.81928679230454e-05\\
599.8	7.41672397797984e-05\\
599.81	7.01685467046336e-05\\
599.82	6.6197222057314e-05\\
599.83	6.22537145945345e-05\\
599.84	5.83384892794431e-05\\
599.85	5.44520281491948e-05\\
599.86	5.05948312460661e-05\\
599.87	4.67674176184013e-05\\
599.88	4.29703263983248e-05\\
599.89	3.92041179641221e-05\\
599.9	3.54693751961673e-05\\
599.91	3.17667048364539e-05\\
599.92	2.80967389631615e-05\\
599.93	2.44601365932736e-05\\
599.94	2.08575854280871e-05\\
599.95	1.72898037587021e-05\\
599.96	1.37575425510246e-05\\
599.97	1.02615877329605e-05\\
599.98	6.80276270998044e-06\\
599.99	3.38193113953895e-06\\
600	0\\
};
\addplot [color=mycolor15,solid,forget plot]
  table[row sep=crcr]{%
0.01	0.01\\
1.01	0.01\\
2.01	0.01\\
3.01	0.01\\
4.01	0.01\\
5.01	0.01\\
6.01	0.01\\
7.01	0.01\\
8.01	0.01\\
9.01	0.01\\
10.01	0.01\\
11.01	0.01\\
12.01	0.01\\
13.01	0.01\\
14.01	0.01\\
15.01	0.01\\
16.01	0.01\\
17.01	0.01\\
18.01	0.01\\
19.01	0.01\\
20.01	0.01\\
21.01	0.01\\
22.01	0.01\\
23.01	0.01\\
24.01	0.01\\
25.01	0.01\\
26.01	0.01\\
27.01	0.01\\
28.01	0.01\\
29.01	0.01\\
30.01	0.01\\
31.01	0.01\\
32.01	0.01\\
33.01	0.01\\
34.01	0.01\\
35.01	0.01\\
36.01	0.01\\
37.01	0.01\\
38.01	0.01\\
39.01	0.01\\
40.01	0.01\\
41.01	0.01\\
42.01	0.01\\
43.01	0.01\\
44.01	0.01\\
45.01	0.01\\
46.01	0.01\\
47.01	0.01\\
48.01	0.01\\
49.01	0.01\\
50.01	0.01\\
51.01	0.01\\
52.01	0.01\\
53.01	0.01\\
54.01	0.01\\
55.01	0.01\\
56.01	0.01\\
57.01	0.01\\
58.01	0.01\\
59.01	0.01\\
60.01	0.01\\
61.01	0.01\\
62.01	0.01\\
63.01	0.01\\
64.01	0.01\\
65.01	0.01\\
66.01	0.01\\
67.01	0.01\\
68.01	0.01\\
69.01	0.01\\
70.01	0.01\\
71.01	0.01\\
72.01	0.01\\
73.01	0.01\\
74.01	0.01\\
75.01	0.01\\
76.01	0.01\\
77.01	0.01\\
78.01	0.01\\
79.01	0.01\\
80.01	0.01\\
81.01	0.01\\
82.01	0.01\\
83.01	0.01\\
84.01	0.01\\
85.01	0.01\\
86.01	0.01\\
87.01	0.01\\
88.01	0.01\\
89.01	0.01\\
90.01	0.01\\
91.01	0.01\\
92.01	0.01\\
93.01	0.01\\
94.01	0.01\\
95.01	0.01\\
96.01	0.01\\
97.01	0.01\\
98.01	0.01\\
99.01	0.01\\
100.01	0.01\\
101.01	0.01\\
102.01	0.01\\
103.01	0.01\\
104.01	0.01\\
105.01	0.01\\
106.01	0.01\\
107.01	0.01\\
108.01	0.01\\
109.01	0.01\\
110.01	0.01\\
111.01	0.01\\
112.01	0.01\\
113.01	0.01\\
114.01	0.01\\
115.01	0.01\\
116.01	0.01\\
117.01	0.01\\
118.01	0.01\\
119.01	0.01\\
120.01	0.01\\
121.01	0.01\\
122.01	0.01\\
123.01	0.01\\
124.01	0.01\\
125.01	0.01\\
126.01	0.01\\
127.01	0.01\\
128.01	0.01\\
129.01	0.01\\
130.01	0.01\\
131.01	0.01\\
132.01	0.01\\
133.01	0.01\\
134.01	0.01\\
135.01	0.01\\
136.01	0.01\\
137.01	0.01\\
138.01	0.01\\
139.01	0.01\\
140.01	0.01\\
141.01	0.01\\
142.01	0.01\\
143.01	0.01\\
144.01	0.01\\
145.01	0.01\\
146.01	0.01\\
147.01	0.01\\
148.01	0.01\\
149.01	0.01\\
150.01	0.01\\
151.01	0.01\\
152.01	0.01\\
153.01	0.01\\
154.01	0.01\\
155.01	0.01\\
156.01	0.01\\
157.01	0.01\\
158.01	0.01\\
159.01	0.01\\
160.01	0.01\\
161.01	0.01\\
162.01	0.01\\
163.01	0.01\\
164.01	0.01\\
165.01	0.01\\
166.01	0.01\\
167.01	0.01\\
168.01	0.01\\
169.01	0.01\\
170.01	0.01\\
171.01	0.01\\
172.01	0.01\\
173.01	0.01\\
174.01	0.01\\
175.01	0.01\\
176.01	0.01\\
177.01	0.01\\
178.01	0.01\\
179.01	0.01\\
180.01	0.01\\
181.01	0.01\\
182.01	0.01\\
183.01	0.01\\
184.01	0.01\\
185.01	0.01\\
186.01	0.01\\
187.01	0.01\\
188.01	0.01\\
189.01	0.01\\
190.01	0.01\\
191.01	0.01\\
192.01	0.01\\
193.01	0.01\\
194.01	0.01\\
195.01	0.01\\
196.01	0.01\\
197.01	0.01\\
198.01	0.01\\
199.01	0.01\\
200.01	0.01\\
201.01	0.01\\
202.01	0.01\\
203.01	0.01\\
204.01	0.01\\
205.01	0.01\\
206.01	0.01\\
207.01	0.01\\
208.01	0.01\\
209.01	0.01\\
210.01	0.01\\
211.01	0.01\\
212.01	0.01\\
213.01	0.01\\
214.01	0.01\\
215.01	0.01\\
216.01	0.01\\
217.01	0.01\\
218.01	0.01\\
219.01	0.01\\
220.01	0.01\\
221.01	0.01\\
222.01	0.01\\
223.01	0.01\\
224.01	0.01\\
225.01	0.01\\
226.01	0.01\\
227.01	0.01\\
228.01	0.01\\
229.01	0.01\\
230.01	0.01\\
231.01	0.01\\
232.01	0.01\\
233.01	0.01\\
234.01	0.01\\
235.01	0.01\\
236.01	0.01\\
237.01	0.01\\
238.01	0.01\\
239.01	0.01\\
240.01	0.01\\
241.01	0.01\\
242.01	0.01\\
243.01	0.01\\
244.01	0.01\\
245.01	0.01\\
246.01	0.01\\
247.01	0.01\\
248.01	0.01\\
249.01	0.01\\
250.01	0.01\\
251.01	0.01\\
252.01	0.01\\
253.01	0.01\\
254.01	0.01\\
255.01	0.01\\
256.01	0.01\\
257.01	0.01\\
258.01	0.01\\
259.01	0.01\\
260.01	0.01\\
261.01	0.01\\
262.01	0.01\\
263.01	0.01\\
264.01	0.01\\
265.01	0.01\\
266.01	0.01\\
267.01	0.01\\
268.01	0.01\\
269.01	0.01\\
270.01	0.01\\
271.01	0.01\\
272.01	0.01\\
273.01	0.01\\
274.01	0.01\\
275.01	0.01\\
276.01	0.01\\
277.01	0.01\\
278.01	0.01\\
279.01	0.01\\
280.01	0.01\\
281.01	0.01\\
282.01	0.01\\
283.01	0.01\\
284.01	0.01\\
285.01	0.01\\
286.01	0.01\\
287.01	0.01\\
288.01	0.01\\
289.01	0.01\\
290.01	0.01\\
291.01	0.01\\
292.01	0.01\\
293.01	0.01\\
294.01	0.01\\
295.01	0.01\\
296.01	0.01\\
297.01	0.01\\
298.01	0.01\\
299.01	0.01\\
300.01	0.01\\
301.01	0.01\\
302.01	0.01\\
303.01	0.01\\
304.01	0.01\\
305.01	0.01\\
306.01	0.01\\
307.01	0.01\\
308.01	0.01\\
309.01	0.01\\
310.01	0.01\\
311.01	0.01\\
312.01	0.01\\
313.01	0.01\\
314.01	0.01\\
315.01	0.01\\
316.01	0.01\\
317.01	0.01\\
318.01	0.01\\
319.01	0.01\\
320.01	0.01\\
321.01	0.01\\
322.01	0.01\\
323.01	0.01\\
324.01	0.01\\
325.01	0.01\\
326.01	0.01\\
327.01	0.01\\
328.01	0.01\\
329.01	0.01\\
330.01	0.01\\
331.01	0.01\\
332.01	0.01\\
333.01	0.01\\
334.01	0.01\\
335.01	0.01\\
336.01	0.01\\
337.01	0.01\\
338.01	0.01\\
339.01	0.01\\
340.01	0.01\\
341.01	0.01\\
342.01	0.01\\
343.01	0.01\\
344.01	0.01\\
345.01	0.01\\
346.01	0.01\\
347.01	0.01\\
348.01	0.01\\
349.01	0.01\\
350.01	0.01\\
351.01	0.01\\
352.01	0.01\\
353.01	0.01\\
354.01	0.01\\
355.01	0.01\\
356.01	0.01\\
357.01	0.01\\
358.01	0.01\\
359.01	0.01\\
360.01	0.01\\
361.01	0.01\\
362.01	0.01\\
363.01	0.01\\
364.01	0.01\\
365.01	0.01\\
366.01	0.01\\
367.01	0.01\\
368.01	0.01\\
369.01	0.01\\
370.01	0.01\\
371.01	0.01\\
372.01	0.01\\
373.01	0.01\\
374.01	0.01\\
375.01	0.01\\
376.01	0.01\\
377.01	0.01\\
378.01	0.01\\
379.01	0.01\\
380.01	0.01\\
381.01	0.01\\
382.01	0.01\\
383.01	0.01\\
384.01	0.01\\
385.01	0.01\\
386.01	0.01\\
387.01	0.01\\
388.01	0.01\\
389.01	0.01\\
390.01	0.01\\
391.01	0.01\\
392.01	0.01\\
393.01	0.01\\
394.01	0.01\\
395.01	0.01\\
396.01	0.01\\
397.01	0.01\\
398.01	0.01\\
399.01	0.01\\
400.01	0.01\\
401.01	0.01\\
402.01	0.01\\
403.01	0.01\\
404.01	0.01\\
405.01	0.01\\
406.01	0.01\\
407.01	0.01\\
408.01	0.01\\
409.01	0.01\\
410.01	0.01\\
411.01	0.01\\
412.01	0.01\\
413.01	0.01\\
414.01	0.01\\
415.01	0.01\\
416.01	0.01\\
417.01	0.01\\
418.01	0.01\\
419.01	0.01\\
420.01	0.01\\
421.01	0.01\\
422.01	0.01\\
423.01	0.01\\
424.01	0.01\\
425.01	0.01\\
426.01	0.01\\
427.01	0.01\\
428.01	0.01\\
429.01	0.01\\
430.01	0.01\\
431.01	0.01\\
432.01	0.01\\
433.01	0.01\\
434.01	0.01\\
435.01	0.01\\
436.01	0.01\\
437.01	0.01\\
438.01	0.01\\
439.01	0.01\\
440.01	0.01\\
441.01	0.01\\
442.01	0.01\\
443.01	0.01\\
444.01	0.01\\
445.01	0.01\\
446.01	0.01\\
447.01	0.01\\
448.01	0.01\\
449.01	0.01\\
450.01	0.01\\
451.01	0.01\\
452.01	0.01\\
453.01	0.01\\
454.01	0.01\\
455.01	0.01\\
456.01	0.01\\
457.01	0.01\\
458.01	0.01\\
459.01	0.01\\
460.01	0.01\\
461.01	0.01\\
462.01	0.01\\
463.01	0.01\\
464.01	0.01\\
465.01	0.01\\
466.01	0.01\\
467.01	0.01\\
468.01	0.01\\
469.01	0.01\\
470.01	0.01\\
471.01	0.01\\
472.01	0.01\\
473.01	0.01\\
474.01	0.01\\
475.01	0.01\\
476.01	0.01\\
477.01	0.01\\
478.01	0.01\\
479.01	0.01\\
480.01	0.01\\
481.01	0.01\\
482.01	0.01\\
483.01	0.01\\
484.01	0.01\\
485.01	0.01\\
486.01	0.01\\
487.01	0.01\\
488.01	0.01\\
489.01	0.01\\
490.01	0.01\\
491.01	0.01\\
492.01	0.01\\
493.01	0.01\\
494.01	0.01\\
495.01	0.01\\
496.01	0.01\\
497.01	0.01\\
498.01	0.01\\
499.01	0.01\\
500.01	0.01\\
501.01	0.01\\
502.01	0.01\\
503.01	0.01\\
504.01	0.01\\
505.01	0.01\\
506.01	0.01\\
507.01	0.01\\
508.01	0.01\\
509.01	0.01\\
510.01	0.01\\
511.01	0.01\\
512.01	0.01\\
513.01	0.01\\
514.01	0.01\\
515.01	0.01\\
516.01	0.01\\
517.01	0.01\\
518.01	0.01\\
519.01	0.01\\
520.01	0.01\\
521.01	0.01\\
522.01	0.01\\
523.01	0.01\\
524.01	0.01\\
525.01	0.01\\
526.01	0.01\\
527.01	0.01\\
528.01	0.01\\
529.01	0.01\\
530.01	0.01\\
531.01	0.01\\
532.01	0.01\\
533.01	0.01\\
534.01	0.01\\
535.01	0.01\\
536.01	0.01\\
537.01	0.01\\
538.01	0.01\\
539.01	0.01\\
540.01	0.01\\
541.01	0.01\\
542.01	0.01\\
543.01	0.01\\
544.01	0.01\\
545.01	0.01\\
546.01	0.01\\
547.01	0.01\\
548.01	0.01\\
549.01	0.01\\
550.01	0.01\\
551.01	0.00990204942161797\\
552.01	0.00975551902864949\\
553.01	0.00960130073362049\\
554.01	0.0094386003805723\\
555.01	0.009266503152353\\
556.01	0.0090839507765319\\
557.01	0.00888971377982333\\
558.01	0.008682357705687\\
559.01	0.00846020294590994\\
560.01	0.00822128900639638\\
561.01	0.00796766659822521\\
562.01	0.0077048159941505\\
563.01	0.00743232512672041\\
564.01	0.0071496454824594\\
565.01	0.0068561899957042\\
566.01	0.0065513453559236\\
567.01	0.00623447560958786\\
568.01	0.00590492764055903\\
569.01	0.00556203943909303\\
570.01	0.00520515136201691\\
571.01	0.0048335653439925\\
572.01	0.00444648371089207\\
573.01	0.0042357046706426\\
574.01	0.00403720111108808\\
575.01	0.00384342391724351\\
576.01	0.00365706307610982\\
577.01	0.00348009927715622\\
578.01	0.00330555158339844\\
579.01	0.00313128876087222\\
580.01	0.00295772959046831\\
581.01	0.00278530784327323\\
582.01	0.0026144529462283\\
583.01	0.00244555985981434\\
584.01	0.00227894406209801\\
585.01	0.00211453903896056\\
586.01	0.00195076928643856\\
587.01	0.00178744637180153\\
588.01	0.00162464815568858\\
589.01	0.00146288912824371\\
590.01	0.00130274409803109\\
591.01	0.00114484388588358\\
592.01	0.000989873810182567\\
593.01	0.00083856959646709\\
594.01	0.00069170989414428\\
595.01	0.000550104571421454\\
596.01	0.000414578932401176\\
597.01	0.000285963596768515\\
598.01	0.000165188062198949\\
599.01	5.4345015548133e-05\\
599.02	5.33064624635531e-05\\
599.03	5.22697048745222e-05\\
599.04	5.12347597904136e-05\\
599.05	5.0201644409836e-05\\
599.06	4.91703761153605e-05\\
599.07	4.8140972467612e-05\\
599.08	4.71134500464859e-05\\
599.09	4.60878252401993e-05\\
599.1	4.50641145578815e-05\\
599.11	4.40423346413817e-05\\
599.12	4.30225022459152e-05\\
599.13	4.20046341915222e-05\\
599.14	4.0988747398104e-05\\
599.15	3.99748588722423e-05\\
599.16	3.89629856928746e-05\\
599.17	3.79531449942027e-05\\
599.18	3.69453539466515e-05\\
599.19	3.59396297421356e-05\\
599.2	3.49359895741794e-05\\
599.21	3.39348997364097e-05\\
599.22	3.29370757084346e-05\\
599.23	3.19425380165637e-05\\
599.24	3.09513071439539e-05\\
599.25	2.99634035031259e-05\\
599.26	2.89788474063021e-05\\
599.27	2.80044066980276e-05\\
599.28	2.70493587152636e-05\\
599.29	2.61138364157729e-05\\
599.3	2.51979740088139e-05\\
599.31	2.43022091112599e-05\\
599.32	2.34269645671102e-05\\
599.33	2.25723778285291e-05\\
599.34	2.17385877173907e-05\\
599.35	2.09257721965651e-05\\
599.36	2.01340862337751e-05\\
599.37	1.9363671100418e-05\\
599.38	1.86146691965421e-05\\
599.39	1.78872240385961e-05\\
599.4	1.71814802439572e-05\\
599.41	1.64975835122289e-05\\
599.42	1.58356806037336e-05\\
599.43	1.5195919314372e-05\\
599.44	1.457844844663e-05\\
599.45	1.39834177763486e-05\\
599.46	1.34109780150146e-05\\
599.47	1.28612807671867e-05\\
599.48	1.23344784827405e-05\\
599.49	1.18300658477049e-05\\
599.5	1.13478434777131e-05\\
599.51	1.08879592333778e-05\\
599.52	1.04505664172659e-05\\
599.53	1.00334667578663e-05\\
599.54	9.62511880027439e-06\\
599.55	9.2255549021085e-06\\
599.56	8.83462029344952e-06\\
599.57	8.45200412851778e-06\\
599.58	8.07772879304891e-06\\
599.59	7.71176793122087e-06\\
599.6	7.35413349658119e-06\\
599.61	7.00483423372172e-06\\
599.62	6.66387342551991e-06\\
599.63	6.33124847425164e-06\\
599.64	6.00695045552616e-06\\
599.65	5.69096364331188e-06\\
599.66	5.38326500425235e-06\\
599.67	5.08384138973167e-06\\
599.68	4.79267948952887e-06\\
599.69	4.50975886944101e-06\\
599.7	4.23505144031729e-06\\
599.71	3.96852089248369e-06\\
599.72	3.71012224967097e-06\\
599.73	3.45988538647847e-06\\
599.74	3.21786860717557e-06\\
599.75	2.98412945139333e-06\\
599.76	2.75872706875617e-06\\
599.77	2.54172382291654e-06\\
599.78	2.33318159911268e-06\\
599.79	2.13316174173424e-06\\
599.8	1.94172498835417e-06\\
599.81	1.75893139995643e-06\\
599.82	1.58484028716689e-06\\
599.83	1.41951013224277e-06\\
599.84	1.2629985065276e-06\\
599.85	1.11536198312695e-06\\
599.86	9.76656044501464e-07\\
599.87	8.46934984602421e-07\\
599.88	7.26251805284461e-07\\
599.89	6.14658106537491e-07\\
599.9	5.12203970199493e-07\\
599.91	4.18937836669728e-07\\
599.92	3.34906374173036e-07\\
599.93	2.60154340070429e-07\\
599.94	1.94724433657395e-07\\
599.95	1.38657139853171e-07\\
599.96	9.19905631651535e-08\\
599.97	5.47602512050716e-08\\
599.98	2.6999006997111e-08\\
599.99	8.73668930083393e-09\\
600	0\\
};
\addplot [color=mycolor16,solid,forget plot]
  table[row sep=crcr]{%
0.01	0.01\\
1.01	0.01\\
2.01	0.01\\
3.01	0.01\\
4.01	0.01\\
5.01	0.01\\
6.01	0.01\\
7.01	0.01\\
8.01	0.01\\
9.01	0.01\\
10.01	0.01\\
11.01	0.01\\
12.01	0.01\\
13.01	0.01\\
14.01	0.01\\
15.01	0.01\\
16.01	0.01\\
17.01	0.01\\
18.01	0.01\\
19.01	0.01\\
20.01	0.01\\
21.01	0.01\\
22.01	0.01\\
23.01	0.01\\
24.01	0.01\\
25.01	0.01\\
26.01	0.01\\
27.01	0.01\\
28.01	0.01\\
29.01	0.01\\
30.01	0.01\\
31.01	0.01\\
32.01	0.01\\
33.01	0.01\\
34.01	0.01\\
35.01	0.01\\
36.01	0.01\\
37.01	0.01\\
38.01	0.01\\
39.01	0.01\\
40.01	0.01\\
41.01	0.01\\
42.01	0.01\\
43.01	0.01\\
44.01	0.01\\
45.01	0.01\\
46.01	0.01\\
47.01	0.01\\
48.01	0.01\\
49.01	0.01\\
50.01	0.01\\
51.01	0.01\\
52.01	0.01\\
53.01	0.01\\
54.01	0.01\\
55.01	0.01\\
56.01	0.01\\
57.01	0.01\\
58.01	0.01\\
59.01	0.01\\
60.01	0.01\\
61.01	0.01\\
62.01	0.01\\
63.01	0.01\\
64.01	0.01\\
65.01	0.01\\
66.01	0.01\\
67.01	0.01\\
68.01	0.01\\
69.01	0.01\\
70.01	0.01\\
71.01	0.01\\
72.01	0.01\\
73.01	0.01\\
74.01	0.01\\
75.01	0.01\\
76.01	0.01\\
77.01	0.01\\
78.01	0.01\\
79.01	0.01\\
80.01	0.01\\
81.01	0.01\\
82.01	0.01\\
83.01	0.01\\
84.01	0.01\\
85.01	0.01\\
86.01	0.01\\
87.01	0.01\\
88.01	0.01\\
89.01	0.01\\
90.01	0.01\\
91.01	0.01\\
92.01	0.01\\
93.01	0.01\\
94.01	0.01\\
95.01	0.01\\
96.01	0.01\\
97.01	0.01\\
98.01	0.01\\
99.01	0.01\\
100.01	0.01\\
101.01	0.01\\
102.01	0.01\\
103.01	0.01\\
104.01	0.01\\
105.01	0.01\\
106.01	0.01\\
107.01	0.01\\
108.01	0.01\\
109.01	0.01\\
110.01	0.01\\
111.01	0.01\\
112.01	0.01\\
113.01	0.01\\
114.01	0.01\\
115.01	0.01\\
116.01	0.01\\
117.01	0.01\\
118.01	0.01\\
119.01	0.01\\
120.01	0.01\\
121.01	0.01\\
122.01	0.01\\
123.01	0.01\\
124.01	0.01\\
125.01	0.01\\
126.01	0.01\\
127.01	0.01\\
128.01	0.01\\
129.01	0.01\\
130.01	0.01\\
131.01	0.01\\
132.01	0.01\\
133.01	0.01\\
134.01	0.01\\
135.01	0.01\\
136.01	0.01\\
137.01	0.01\\
138.01	0.01\\
139.01	0.01\\
140.01	0.01\\
141.01	0.01\\
142.01	0.01\\
143.01	0.01\\
144.01	0.01\\
145.01	0.01\\
146.01	0.01\\
147.01	0.01\\
148.01	0.01\\
149.01	0.01\\
150.01	0.01\\
151.01	0.01\\
152.01	0.01\\
153.01	0.01\\
154.01	0.01\\
155.01	0.01\\
156.01	0.01\\
157.01	0.01\\
158.01	0.01\\
159.01	0.01\\
160.01	0.01\\
161.01	0.01\\
162.01	0.01\\
163.01	0.01\\
164.01	0.01\\
165.01	0.01\\
166.01	0.01\\
167.01	0.01\\
168.01	0.01\\
169.01	0.01\\
170.01	0.01\\
171.01	0.01\\
172.01	0.01\\
173.01	0.01\\
174.01	0.01\\
175.01	0.01\\
176.01	0.01\\
177.01	0.01\\
178.01	0.01\\
179.01	0.01\\
180.01	0.01\\
181.01	0.01\\
182.01	0.01\\
183.01	0.01\\
184.01	0.01\\
185.01	0.01\\
186.01	0.01\\
187.01	0.01\\
188.01	0.01\\
189.01	0.01\\
190.01	0.01\\
191.01	0.01\\
192.01	0.01\\
193.01	0.01\\
194.01	0.01\\
195.01	0.01\\
196.01	0.01\\
197.01	0.01\\
198.01	0.01\\
199.01	0.01\\
200.01	0.01\\
201.01	0.01\\
202.01	0.01\\
203.01	0.01\\
204.01	0.01\\
205.01	0.01\\
206.01	0.01\\
207.01	0.01\\
208.01	0.01\\
209.01	0.01\\
210.01	0.01\\
211.01	0.01\\
212.01	0.01\\
213.01	0.01\\
214.01	0.01\\
215.01	0.01\\
216.01	0.01\\
217.01	0.01\\
218.01	0.01\\
219.01	0.01\\
220.01	0.01\\
221.01	0.01\\
222.01	0.01\\
223.01	0.01\\
224.01	0.01\\
225.01	0.01\\
226.01	0.01\\
227.01	0.01\\
228.01	0.01\\
229.01	0.01\\
230.01	0.01\\
231.01	0.01\\
232.01	0.01\\
233.01	0.01\\
234.01	0.01\\
235.01	0.01\\
236.01	0.01\\
237.01	0.01\\
238.01	0.01\\
239.01	0.01\\
240.01	0.01\\
241.01	0.01\\
242.01	0.01\\
243.01	0.01\\
244.01	0.01\\
245.01	0.01\\
246.01	0.01\\
247.01	0.01\\
248.01	0.01\\
249.01	0.01\\
250.01	0.01\\
251.01	0.01\\
252.01	0.01\\
253.01	0.01\\
254.01	0.01\\
255.01	0.01\\
256.01	0.01\\
257.01	0.01\\
258.01	0.01\\
259.01	0.01\\
260.01	0.01\\
261.01	0.01\\
262.01	0.01\\
263.01	0.01\\
264.01	0.01\\
265.01	0.01\\
266.01	0.01\\
267.01	0.01\\
268.01	0.01\\
269.01	0.01\\
270.01	0.01\\
271.01	0.01\\
272.01	0.01\\
273.01	0.01\\
274.01	0.01\\
275.01	0.01\\
276.01	0.01\\
277.01	0.01\\
278.01	0.01\\
279.01	0.01\\
280.01	0.01\\
281.01	0.01\\
282.01	0.01\\
283.01	0.01\\
284.01	0.01\\
285.01	0.01\\
286.01	0.01\\
287.01	0.01\\
288.01	0.01\\
289.01	0.01\\
290.01	0.01\\
291.01	0.01\\
292.01	0.01\\
293.01	0.01\\
294.01	0.01\\
295.01	0.01\\
296.01	0.01\\
297.01	0.01\\
298.01	0.01\\
299.01	0.01\\
300.01	0.01\\
301.01	0.01\\
302.01	0.01\\
303.01	0.01\\
304.01	0.01\\
305.01	0.01\\
306.01	0.01\\
307.01	0.01\\
308.01	0.01\\
309.01	0.01\\
310.01	0.01\\
311.01	0.01\\
312.01	0.01\\
313.01	0.01\\
314.01	0.01\\
315.01	0.01\\
316.01	0.01\\
317.01	0.01\\
318.01	0.01\\
319.01	0.01\\
320.01	0.01\\
321.01	0.01\\
322.01	0.01\\
323.01	0.01\\
324.01	0.01\\
325.01	0.01\\
326.01	0.01\\
327.01	0.01\\
328.01	0.01\\
329.01	0.01\\
330.01	0.01\\
331.01	0.01\\
332.01	0.01\\
333.01	0.01\\
334.01	0.01\\
335.01	0.01\\
336.01	0.01\\
337.01	0.01\\
338.01	0.01\\
339.01	0.01\\
340.01	0.01\\
341.01	0.01\\
342.01	0.01\\
343.01	0.01\\
344.01	0.01\\
345.01	0.01\\
346.01	0.01\\
347.01	0.01\\
348.01	0.01\\
349.01	0.01\\
350.01	0.01\\
351.01	0.01\\
352.01	0.01\\
353.01	0.01\\
354.01	0.01\\
355.01	0.01\\
356.01	0.01\\
357.01	0.01\\
358.01	0.01\\
359.01	0.01\\
360.01	0.01\\
361.01	0.01\\
362.01	0.01\\
363.01	0.01\\
364.01	0.01\\
365.01	0.01\\
366.01	0.01\\
367.01	0.01\\
368.01	0.01\\
369.01	0.01\\
370.01	0.01\\
371.01	0.01\\
372.01	0.01\\
373.01	0.01\\
374.01	0.01\\
375.01	0.01\\
376.01	0.01\\
377.01	0.01\\
378.01	0.01\\
379.01	0.01\\
380.01	0.01\\
381.01	0.01\\
382.01	0.01\\
383.01	0.01\\
384.01	0.01\\
385.01	0.01\\
386.01	0.01\\
387.01	0.01\\
388.01	0.01\\
389.01	0.01\\
390.01	0.01\\
391.01	0.01\\
392.01	0.01\\
393.01	0.01\\
394.01	0.01\\
395.01	0.01\\
396.01	0.01\\
397.01	0.01\\
398.01	0.01\\
399.01	0.01\\
400.01	0.01\\
401.01	0.01\\
402.01	0.01\\
403.01	0.01\\
404.01	0.01\\
405.01	0.01\\
406.01	0.01\\
407.01	0.01\\
408.01	0.01\\
409.01	0.01\\
410.01	0.01\\
411.01	0.01\\
412.01	0.01\\
413.01	0.01\\
414.01	0.01\\
415.01	0.01\\
416.01	0.01\\
417.01	0.01\\
418.01	0.01\\
419.01	0.01\\
420.01	0.01\\
421.01	0.01\\
422.01	0.01\\
423.01	0.01\\
424.01	0.01\\
425.01	0.01\\
426.01	0.01\\
427.01	0.01\\
428.01	0.01\\
429.01	0.01\\
430.01	0.01\\
431.01	0.01\\
432.01	0.01\\
433.01	0.01\\
434.01	0.01\\
435.01	0.01\\
436.01	0.01\\
437.01	0.01\\
438.01	0.01\\
439.01	0.01\\
440.01	0.01\\
441.01	0.01\\
442.01	0.01\\
443.01	0.01\\
444.01	0.01\\
445.01	0.01\\
446.01	0.01\\
447.01	0.01\\
448.01	0.01\\
449.01	0.01\\
450.01	0.01\\
451.01	0.01\\
452.01	0.01\\
453.01	0.01\\
454.01	0.01\\
455.01	0.01\\
456.01	0.01\\
457.01	0.01\\
458.01	0.01\\
459.01	0.01\\
460.01	0.01\\
461.01	0.01\\
462.01	0.01\\
463.01	0.01\\
464.01	0.01\\
465.01	0.01\\
466.01	0.01\\
467.01	0.01\\
468.01	0.01\\
469.01	0.01\\
470.01	0.01\\
471.01	0.01\\
472.01	0.01\\
473.01	0.01\\
474.01	0.01\\
475.01	0.01\\
476.01	0.01\\
477.01	0.01\\
478.01	0.01\\
479.01	0.01\\
480.01	0.01\\
481.01	0.01\\
482.01	0.01\\
483.01	0.01\\
484.01	0.01\\
485.01	0.01\\
486.01	0.01\\
487.01	0.01\\
488.01	0.01\\
489.01	0.01\\
490.01	0.01\\
491.01	0.01\\
492.01	0.01\\
493.01	0.01\\
494.01	0.01\\
495.01	0.01\\
496.01	0.01\\
497.01	0.01\\
498.01	0.01\\
499.01	0.01\\
500.01	0.01\\
501.01	0.01\\
502.01	0.01\\
503.01	0.01\\
504.01	0.01\\
505.01	0.01\\
506.01	0.01\\
507.01	0.01\\
508.01	0.01\\
509.01	0.01\\
510.01	0.01\\
511.01	0.01\\
512.01	0.01\\
513.01	0.01\\
514.01	0.01\\
515.01	0.01\\
516.01	0.01\\
517.01	0.01\\
518.01	0.01\\
519.01	0.01\\
520.01	0.01\\
521.01	0.01\\
522.01	0.01\\
523.01	0.00992575245227274\\
524.01	0.00984652916315449\\
525.01	0.00976427383091153\\
526.01	0.00967878858387452\\
527.01	0.00958985481377644\\
528.01	0.00949723016296407\\
529.01	0.00940064498293353\\
530.01	0.00929979815948588\\
531.01	0.00919435215483429\\
532.01	0.0090839270942965\\
533.01	0.00896809366624845\\
534.01	0.00884636485433692\\
535.01	0.00871818833204663\\
536.01	0.00858293155538475\\
537.01	0.00843986048094458\\
538.01	0.00828812853295346\\
539.01	0.00812675686309659\\
540.01	0.00795460963928865\\
541.01	0.00777036785781529\\
542.01	0.00757342059949487\\
543.01	0.00736856329333439\\
544.01	0.00715647415153917\\
545.01	0.00693675136318322\\
546.01	0.00670895106381417\\
547.01	0.00647257909973894\\
548.01	0.00622708057036442\\
549.01	0.00597182819023422\\
550.01	0.00570611914029568\\
551.01	0.00552764479099582\\
552.01	0.00538785028227558\\
553.01	0.00524547405651935\\
554.01	0.00510085683299053\\
555.01	0.0049544724765762\\
556.01	0.00480696566205194\\
557.01	0.00465919998370261\\
558.01	0.00451231962311478\\
559.01	0.00436782949647694\\
560.01	0.00422770128462472\\
561.01	0.00409012220592474\\
562.01	0.0039497661840341\\
563.01	0.00380802511746821\\
564.01	0.00366686467040957\\
565.01	0.00352675473052659\\
566.01	0.00338818459516706\\
567.01	0.00325170044485083\\
568.01	0.00311790483298779\\
569.01	0.00298745287269209\\
570.01	0.00286104575543987\\
571.01	0.00273947549593472\\
572.01	0.00262369203361567\\
573.01	0.00251316420804345\\
574.01	0.00240612417636698\\
575.01	0.00230064181470498\\
576.01	0.00219622952468568\\
577.01	0.00209274219424811\\
578.01	0.00199006113334727\\
579.01	0.00188828358359086\\
580.01	0.00178750904497525\\
581.01	0.00168781344225515\\
582.01	0.00158924687363906\\
583.01	0.0014918307579822\\
584.01	0.00139556043494016\\
585.01	0.00130039603260936\\
586.01	0.00120615033170115\\
587.01	0.00111236845038909\\
588.01	0.00101905521743938\\
589.01	0.000926321771227706\\
590.01	0.000834264424821173\\
591.01	0.000742959206678368\\
592.01	0.000652455943327473\\
593.01	0.000562772011437012\\
594.01	0.000473886068201661\\
595.01	0.00038573221032095\\
596.01	0.000298195161474923\\
597.01	0.000211107141103397\\
598.01	0.000124246194188974\\
599.01	3.7327034981641e-05\\
599.02	3.64581259681349e-05\\
599.03	3.55904023650409e-05\\
599.04	3.47238671974722e-05\\
599.05	3.38585235776824e-05\\
599.06	3.29943747087724e-05\\
599.07	3.21314238885661e-05\\
599.08	3.12828146845964e-05\\
599.09	3.04526072935907e-05\\
599.1	2.96411999832092e-05\\
599.11	2.8848744456229e-05\\
599.12	2.80755012468988e-05\\
599.13	2.73221646003644e-05\\
599.14	2.65888933353459e-05\\
599.15	2.58758490175037e-05\\
599.16	2.51831960653721e-05\\
599.17	2.45111189409661e-05\\
599.18	2.38598295667809e-05\\
599.19	2.3229501472391e-05\\
599.2	2.26203115193009e-05\\
599.21	2.20319886654764e-05\\
599.22	2.14640126570079e-05\\
599.23	2.09165616798686e-05\\
599.24	2.03898175673723e-05\\
599.25	1.98839659503354e-05\\
599.26	1.93991964147905e-05\\
599.27	1.89289341987753e-05\\
599.28	1.8464068823297e-05\\
599.29	1.80046351070922e-05\\
599.3	1.75506675047163e-05\\
599.31	1.71018958724945e-05\\
599.32	1.66580635847045e-05\\
599.33	1.62192026103292e-05\\
599.34	1.5785344236359e-05\\
599.35	1.53564802825266e-05\\
599.36	1.49326256067388e-05\\
599.37	1.4513809013501e-05\\
599.38	1.41000583480724e-05\\
599.39	1.36914004212514e-05\\
599.4	1.32878626875211e-05\\
599.41	1.28894738098337e-05\\
599.42	1.24962611940758e-05\\
599.43	1.21082508957385e-05\\
599.44	1.17254675215759e-05\\
599.45	1.13479341259243e-05\\
599.46	1.09756721014351e-05\\
599.47	1.06087010638255e-05\\
599.48	1.02470387303398e-05\\
599.49	9.8907026131103e-06\\
599.5	9.53970915088496e-06\\
599.51	9.19407265268304e-06\\
599.52	8.85380515453817e-06\\
599.53	8.51892385799261e-06\\
599.54	8.18948273669656e-06\\
599.55	7.86553617346253e-06\\
599.56	7.54713965199667e-06\\
599.57	7.23435039426401e-06\\
599.58	6.92722614961672e-06\\
599.59	6.6258253988271e-06\\
599.6	6.33020718107669e-06\\
599.61	6.04043109214518e-06\\
599.62	5.75655729072831e-06\\
599.63	5.47864650500898e-06\\
599.64	5.2067600394886e-06\\
599.65	4.94095978209944e-06\\
599.66	4.68130821164141e-06\\
599.67	4.42786840111967e-06\\
599.68	4.18070402385777e-06\\
599.69	3.93987936155027e-06\\
599.7	3.70545931271775e-06\\
599.71	3.477509401573e-06\\
599.72	3.25609578728098e-06\\
599.73	3.04128525271952e-06\\
599.74	2.83314520516977e-06\\
599.75	2.63174368469489e-06\\
599.76	2.43714937234532e-06\\
599.77	2.24943159845246e-06\\
599.78	2.06866035230673e-06\\
599.79	1.89490629239931e-06\\
599.8	1.72824075721396e-06\\
599.81	1.56873577666743e-06\\
599.82	1.41646408419877e-06\\
599.83	1.27149912958896e-06\\
599.84	1.13391509257328e-06\\
599.85	1.00378689725684e-06\\
599.86	8.81190227465176e-07\\
599.87	7.66201543043674e-07\\
599.88	6.58898097204846e-07\\
599.89	5.59357955015258e-07\\
599.9	4.67660013056884e-07\\
599.91	3.8388402042247e-07\\
599.92	3.08110601103875e-07\\
599.93	2.40421277865333e-07\\
599.94	1.80898497769907e-07\\
599.95	1.29625659426799e-07\\
599.96	8.66871421312254e-08\\
599.97	5.21683370079129e-08\\
599.98	2.61556803542173e-08\\
599.99	8.73668930083393e-09\\
600	0\\
};
\addplot [color=mycolor17,solid,forget plot]
  table[row sep=crcr]{%
0.01	0.01\\
1.01	0.01\\
2.01	0.01\\
3.01	0.01\\
4.01	0.01\\
5.01	0.01\\
6.01	0.01\\
7.01	0.01\\
8.01	0.01\\
9.01	0.01\\
10.01	0.01\\
11.01	0.01\\
12.01	0.01\\
13.01	0.01\\
14.01	0.01\\
15.01	0.01\\
16.01	0.01\\
17.01	0.01\\
18.01	0.01\\
19.01	0.01\\
20.01	0.01\\
21.01	0.01\\
22.01	0.01\\
23.01	0.01\\
24.01	0.01\\
25.01	0.01\\
26.01	0.01\\
27.01	0.01\\
28.01	0.01\\
29.01	0.01\\
30.01	0.01\\
31.01	0.01\\
32.01	0.01\\
33.01	0.01\\
34.01	0.01\\
35.01	0.01\\
36.01	0.01\\
37.01	0.01\\
38.01	0.01\\
39.01	0.01\\
40.01	0.01\\
41.01	0.01\\
42.01	0.01\\
43.01	0.01\\
44.01	0.01\\
45.01	0.01\\
46.01	0.01\\
47.01	0.01\\
48.01	0.01\\
49.01	0.01\\
50.01	0.01\\
51.01	0.01\\
52.01	0.01\\
53.01	0.01\\
54.01	0.01\\
55.01	0.01\\
56.01	0.01\\
57.01	0.01\\
58.01	0.01\\
59.01	0.01\\
60.01	0.01\\
61.01	0.01\\
62.01	0.01\\
63.01	0.01\\
64.01	0.01\\
65.01	0.01\\
66.01	0.01\\
67.01	0.01\\
68.01	0.01\\
69.01	0.01\\
70.01	0.01\\
71.01	0.01\\
72.01	0.01\\
73.01	0.01\\
74.01	0.01\\
75.01	0.01\\
76.01	0.01\\
77.01	0.01\\
78.01	0.01\\
79.01	0.01\\
80.01	0.01\\
81.01	0.01\\
82.01	0.01\\
83.01	0.01\\
84.01	0.01\\
85.01	0.01\\
86.01	0.01\\
87.01	0.01\\
88.01	0.01\\
89.01	0.01\\
90.01	0.01\\
91.01	0.01\\
92.01	0.01\\
93.01	0.01\\
94.01	0.01\\
95.01	0.01\\
96.01	0.01\\
97.01	0.01\\
98.01	0.01\\
99.01	0.01\\
100.01	0.01\\
101.01	0.01\\
102.01	0.01\\
103.01	0.01\\
104.01	0.01\\
105.01	0.01\\
106.01	0.01\\
107.01	0.01\\
108.01	0.01\\
109.01	0.01\\
110.01	0.01\\
111.01	0.01\\
112.01	0.01\\
113.01	0.01\\
114.01	0.01\\
115.01	0.01\\
116.01	0.01\\
117.01	0.01\\
118.01	0.01\\
119.01	0.01\\
120.01	0.01\\
121.01	0.01\\
122.01	0.01\\
123.01	0.01\\
124.01	0.01\\
125.01	0.01\\
126.01	0.01\\
127.01	0.01\\
128.01	0.01\\
129.01	0.01\\
130.01	0.01\\
131.01	0.01\\
132.01	0.01\\
133.01	0.01\\
134.01	0.01\\
135.01	0.01\\
136.01	0.01\\
137.01	0.01\\
138.01	0.01\\
139.01	0.01\\
140.01	0.01\\
141.01	0.01\\
142.01	0.01\\
143.01	0.01\\
144.01	0.01\\
145.01	0.01\\
146.01	0.01\\
147.01	0.01\\
148.01	0.01\\
149.01	0.01\\
150.01	0.01\\
151.01	0.01\\
152.01	0.01\\
153.01	0.01\\
154.01	0.01\\
155.01	0.01\\
156.01	0.01\\
157.01	0.01\\
158.01	0.01\\
159.01	0.01\\
160.01	0.01\\
161.01	0.01\\
162.01	0.01\\
163.01	0.01\\
164.01	0.01\\
165.01	0.01\\
166.01	0.01\\
167.01	0.01\\
168.01	0.01\\
169.01	0.01\\
170.01	0.01\\
171.01	0.01\\
172.01	0.01\\
173.01	0.01\\
174.01	0.01\\
175.01	0.01\\
176.01	0.01\\
177.01	0.01\\
178.01	0.01\\
179.01	0.01\\
180.01	0.01\\
181.01	0.01\\
182.01	0.01\\
183.01	0.01\\
184.01	0.01\\
185.01	0.01\\
186.01	0.01\\
187.01	0.01\\
188.01	0.01\\
189.01	0.01\\
190.01	0.01\\
191.01	0.01\\
192.01	0.01\\
193.01	0.01\\
194.01	0.01\\
195.01	0.01\\
196.01	0.01\\
197.01	0.01\\
198.01	0.01\\
199.01	0.01\\
200.01	0.01\\
201.01	0.01\\
202.01	0.01\\
203.01	0.01\\
204.01	0.01\\
205.01	0.01\\
206.01	0.01\\
207.01	0.01\\
208.01	0.01\\
209.01	0.01\\
210.01	0.01\\
211.01	0.01\\
212.01	0.01\\
213.01	0.01\\
214.01	0.01\\
215.01	0.01\\
216.01	0.01\\
217.01	0.01\\
218.01	0.01\\
219.01	0.01\\
220.01	0.01\\
221.01	0.01\\
222.01	0.01\\
223.01	0.01\\
224.01	0.01\\
225.01	0.01\\
226.01	0.01\\
227.01	0.01\\
228.01	0.01\\
229.01	0.01\\
230.01	0.01\\
231.01	0.01\\
232.01	0.01\\
233.01	0.01\\
234.01	0.01\\
235.01	0.01\\
236.01	0.01\\
237.01	0.01\\
238.01	0.01\\
239.01	0.01\\
240.01	0.01\\
241.01	0.01\\
242.01	0.01\\
243.01	0.01\\
244.01	0.01\\
245.01	0.01\\
246.01	0.01\\
247.01	0.01\\
248.01	0.01\\
249.01	0.01\\
250.01	0.01\\
251.01	0.01\\
252.01	0.01\\
253.01	0.01\\
254.01	0.01\\
255.01	0.01\\
256.01	0.01\\
257.01	0.01\\
258.01	0.01\\
259.01	0.01\\
260.01	0.01\\
261.01	0.01\\
262.01	0.01\\
263.01	0.01\\
264.01	0.01\\
265.01	0.01\\
266.01	0.01\\
267.01	0.01\\
268.01	0.01\\
269.01	0.01\\
270.01	0.01\\
271.01	0.01\\
272.01	0.01\\
273.01	0.01\\
274.01	0.01\\
275.01	0.01\\
276.01	0.01\\
277.01	0.01\\
278.01	0.01\\
279.01	0.01\\
280.01	0.01\\
281.01	0.01\\
282.01	0.01\\
283.01	0.01\\
284.01	0.01\\
285.01	0.01\\
286.01	0.01\\
287.01	0.01\\
288.01	0.01\\
289.01	0.01\\
290.01	0.01\\
291.01	0.01\\
292.01	0.01\\
293.01	0.01\\
294.01	0.01\\
295.01	0.01\\
296.01	0.01\\
297.01	0.01\\
298.01	0.01\\
299.01	0.01\\
300.01	0.01\\
301.01	0.01\\
302.01	0.01\\
303.01	0.01\\
304.01	0.01\\
305.01	0.01\\
306.01	0.01\\
307.01	0.01\\
308.01	0.01\\
309.01	0.01\\
310.01	0.01\\
311.01	0.01\\
312.01	0.01\\
313.01	0.01\\
314.01	0.01\\
315.01	0.01\\
316.01	0.01\\
317.01	0.01\\
318.01	0.01\\
319.01	0.01\\
320.01	0.01\\
321.01	0.01\\
322.01	0.01\\
323.01	0.01\\
324.01	0.01\\
325.01	0.01\\
326.01	0.01\\
327.01	0.01\\
328.01	0.01\\
329.01	0.01\\
330.01	0.01\\
331.01	0.01\\
332.01	0.01\\
333.01	0.01\\
334.01	0.01\\
335.01	0.01\\
336.01	0.01\\
337.01	0.01\\
338.01	0.01\\
339.01	0.01\\
340.01	0.01\\
341.01	0.01\\
342.01	0.01\\
343.01	0.01\\
344.01	0.01\\
345.01	0.01\\
346.01	0.01\\
347.01	0.01\\
348.01	0.01\\
349.01	0.01\\
350.01	0.01\\
351.01	0.01\\
352.01	0.01\\
353.01	0.01\\
354.01	0.01\\
355.01	0.01\\
356.01	0.01\\
357.01	0.01\\
358.01	0.01\\
359.01	0.01\\
360.01	0.01\\
361.01	0.01\\
362.01	0.01\\
363.01	0.01\\
364.01	0.01\\
365.01	0.01\\
366.01	0.01\\
367.01	0.01\\
368.01	0.01\\
369.01	0.01\\
370.01	0.01\\
371.01	0.01\\
372.01	0.01\\
373.01	0.01\\
374.01	0.01\\
375.01	0.01\\
376.01	0.01\\
377.01	0.01\\
378.01	0.01\\
379.01	0.01\\
380.01	0.01\\
381.01	0.01\\
382.01	0.01\\
383.01	0.01\\
384.01	0.01\\
385.01	0.01\\
386.01	0.01\\
387.01	0.01\\
388.01	0.01\\
389.01	0.01\\
390.01	0.01\\
391.01	0.01\\
392.01	0.01\\
393.01	0.01\\
394.01	0.01\\
395.01	0.01\\
396.01	0.01\\
397.01	0.01\\
398.01	0.01\\
399.01	0.01\\
400.01	0.01\\
401.01	0.01\\
402.01	0.01\\
403.01	0.01\\
404.01	0.01\\
405.01	0.01\\
406.01	0.01\\
407.01	0.01\\
408.01	0.01\\
409.01	0.01\\
410.01	0.01\\
411.01	0.01\\
412.01	0.01\\
413.01	0.01\\
414.01	0.01\\
415.01	0.01\\
416.01	0.01\\
417.01	0.01\\
418.01	0.01\\
419.01	0.01\\
420.01	0.01\\
421.01	0.01\\
422.01	0.01\\
423.01	0.01\\
424.01	0.01\\
425.01	0.01\\
426.01	0.01\\
427.01	0.01\\
428.01	0.01\\
429.01	0.01\\
430.01	0.01\\
431.01	0.01\\
432.01	0.01\\
433.01	0.01\\
434.01	0.01\\
435.01	0.01\\
436.01	0.01\\
437.01	0.01\\
438.01	0.01\\
439.01	0.01\\
440.01	0.01\\
441.01	0.01\\
442.01	0.01\\
443.01	0.01\\
444.01	0.01\\
445.01	0.01\\
446.01	0.01\\
447.01	0.01\\
448.01	0.01\\
449.01	0.01\\
450.01	0.01\\
451.01	0.01\\
452.01	0.01\\
453.01	0.01\\
454.01	0.01\\
455.01	0.01\\
456.01	0.01\\
457.01	0.01\\
458.01	0.01\\
459.01	0.01\\
460.01	0.01\\
461.01	0.00999078095185203\\
462.01	0.00997419417939734\\
463.01	0.009957082611733\\
464.01	0.00993942808881208\\
465.01	0.00992121208632641\\
466.01	0.00990241583879067\\
467.01	0.00988302051352144\\
468.01	0.00986300745183245\\
469.01	0.00984235849873459\\
470.01	0.00982105644896362\\
471.01	0.0097990856459315\\
472.01	0.00977643278174111\\
473.01	0.00975308797251662\\
474.01	0.00972904630606034\\
475.01	0.00970430778655313\\
476.01	0.00967885773352897\\
477.01	0.00965267082451434\\
478.01	0.00962572037708996\\
479.01	0.00959797835263745\\
480.01	0.00956941526577682\\
481.01	0.00954000008832665\\
482.01	0.00950970014810938\\
483.01	0.00947848102321919\\
484.01	0.00944630643266237\\
485.01	0.00941313812349246\\
486.01	0.00937893574377376\\
487.01	0.00934365667464908\\
488.01	0.00930725604099166\\
489.01	0.00926968669288044\\
490.01	0.00923089904542123\\
491.01	0.00919084104334817\\
492.01	0.0091494581801977\\
493.01	0.00910669359017154\\
494.01	0.00906248826125746\\
495.01	0.00901678156923852\\
496.01	0.0089695132699148\\
497.01	0.00892062368141041\\
498.01	0.0088700333650678\\
499.01	0.00881764971181154\\
500.01	0.00876337149402897\\
501.01	0.0087070876953619\\
502.01	0.0086486761348062\\
503.01	0.00858800184576233\\
504.01	0.00852491515234896\\
505.01	0.00845924938584853\\
506.01	0.00839081825084369\\
507.01	0.00831941260721302\\
508.01	0.00824479661368092\\
509.01	0.00816670313252492\\
510.01	0.00808482820202401\\
511.01	0.00799882436587168\\
512.01	0.00790829259769038\\
513.01	0.00781277249428077\\
514.01	0.00771173033149313\\
515.01	0.00760454449104528\\
516.01	0.00749048781583137\\
517.01	0.00736870869453522\\
518.01	0.00723963818854821\\
519.01	0.00710597811992642\\
520.01	0.00696760229751159\\
521.01	0.00682422000099556\\
522.01	0.00667551254793456\\
523.01	0.00659587815347017\\
524.01	0.00651633098827587\\
525.01	0.00643473006207726\\
526.01	0.00635106660776621\\
527.01	0.00626534366701773\\
528.01	0.00617757882137513\\
529.01	0.00608780722930537\\
530.01	0.00599608567697558\\
531.01	0.00590249750060823\\
532.01	0.00580715853831556\\
533.01	0.00571022394240866\\
534.01	0.0056118929793264\\
535.01	0.00551242338902114\\
536.01	0.00541215254983005\\
537.01	0.00531151088970588\\
538.01	0.00521102795614884\\
539.01	0.00511137278512313\\
540.01	0.00501339837207893\\
541.01	0.00491819638302972\\
542.01	0.00482625485291065\\
543.01	0.0047326019711171\\
544.01	0.00463623227679512\\
545.01	0.00453723161982669\\
546.01	0.00443576164286419\\
547.01	0.00433208802327658\\
548.01	0.00422661498624084\\
549.01	0.00411993151685858\\
550.01	0.00401290011585353\\
551.01	0.00390721634275333\\
552.01	0.0038026967159869\\
553.01	0.00369958105289584\\
554.01	0.00359821419173106\\
555.01	0.00349895359268961\\
556.01	0.00340215802666711\\
557.01	0.00330817029183982\\
558.01	0.00321729180630797\\
559.01	0.00312974662100248\\
560.01	0.00304562319959933\\
561.01	0.0029648310383562\\
562.01	0.00288723971163634\\
563.01	0.00281168088604776\\
564.01	0.00273659112181641\\
565.01	0.00266192996599381\\
566.01	0.00258771427401347\\
567.01	0.00251394505248555\\
568.01	0.00244063027530275\\
569.01	0.00236774713459566\\
570.01	0.00229523575951096\\
571.01	0.00222299396948937\\
572.01	0.00215087054873885\\
573.01	0.00207867101495317\\
574.01	0.00200620911203144\\
575.01	0.00193337366729925\\
576.01	0.00186014255824003\\
577.01	0.00178649907178294\\
578.01	0.00171242998631992\\
579.01	0.0016379206162627\\
580.01	0.0015629515820556\\
581.01	0.00148749997079724\\
582.01	0.00141154117821388\\
583.01	0.00133505131726857\\
584.01	0.00125800997726646\\
585.01	0.00118040287913029\\
586.01	0.00110222304799643\\
587.01	0.00102347123973567\\
588.01	0.00094415206962192\\
589.01	0.000864266394493165\\
590.01	0.00078381051162838\\
591.01	0.00070277603982059\\
592.01	0.000621149948535656\\
593.01	0.000538914720361251\\
594.01	0.000456048589420978\\
595.01	0.000372525732669495\\
596.01	0.000288316195459062\\
597.01	0.00020338520208954\\
598.01	0.000117691340829394\\
599.01	3.2249382697221e-05\\
599.02	3.15980747054457e-05\\
599.03	3.09676859235018e-05\\
599.04	3.03584576151362e-05\\
599.05	2.97705837207663e-05\\
599.06	2.92042621379707e-05\\
599.07	2.86596948884162e-05\\
599.08	2.8123912998973e-05\\
599.09	2.75930227461897e-05\\
599.1	2.70667931478105e-05\\
599.11	2.65452441665006e-05\\
599.12	2.60282881461519e-05\\
599.13	2.55154035494164e-05\\
599.14	2.50066056569926e-05\\
599.15	2.45019084058242e-05\\
599.16	2.40013242945435e-05\\
599.17	2.35048471433803e-05\\
599.18	2.30124445947163e-05\\
599.19	2.25241241168985e-05\\
599.2	2.20398912888203e-05\\
599.21	2.15597507792053e-05\\
599.22	2.10837069131182e-05\\
599.23	2.0611761905133e-05\\
599.24	2.01439157217088e-05\\
599.25	1.96801659357645e-05\\
599.26	1.92205075730031e-05\\
599.27	1.87649527270909e-05\\
599.28	1.83135392298733e-05\\
599.29	1.78663051708071e-05\\
599.3	1.74232888957779e-05\\
599.31	1.6984530027039e-05\\
599.32	1.65500694928791e-05\\
599.33	1.61199486044834e-05\\
599.34	1.56942090620352e-05\\
599.35	1.52728931138019e-05\\
599.36	1.48560434799095e-05\\
599.37	1.44437033046136e-05\\
599.38	1.40359161648531e-05\\
599.39	1.36327268429586e-05\\
599.4	1.32341836384959e-05\\
599.41	1.28403353542913e-05\\
599.42	1.24512313076607e-05\\
599.43	1.2066921342313e-05\\
599.44	1.16874558409707e-05\\
599.45	1.13128857387526e-05\\
599.46	1.09432625373523e-05\\
599.47	1.05786383200971e-05\\
599.48	1.02190657679229e-05\\
599.49	9.86459817164075e-06\\
599.5	9.51528944691422e-06\\
599.51	9.17119415258169e-06\\
599.52	8.83236751010076e-06\\
599.53	8.4988654011544e-06\\
599.54	8.17074426504201e-06\\
599.55	7.84806110491927e-06\\
599.56	7.53087349160408e-06\\
599.57	7.21923956508233e-06\\
599.58	6.91321804026715e-06\\
599.59	6.6128682119828e-06\\
599.6	6.31824996075517e-06\\
599.61	6.02942375867339e-06\\
599.62	5.74645067535212e-06\\
599.63	5.46939238393711e-06\\
599.64	5.19831116718369e-06\\
599.65	4.93326992361681e-06\\
599.66	4.67433217375343e-06\\
599.67	4.42156206638573e-06\\
599.68	4.17502438496142e-06\\
599.69	3.93478455401261e-06\\
599.7	3.70090864566797e-06\\
599.71	3.47346338623079e-06\\
599.72	3.2525161628473e-06\\
599.73	3.03813503022173e-06\\
599.74	2.83038871743899e-06\\
599.75	2.62934663483247e-06\\
599.76	2.43507888095067e-06\\
599.77	2.2476562495672e-06\\
599.78	2.06715023678798e-06\\
599.79	1.89363304820347e-06\\
599.8	1.72717760612943e-06\\
599.81	1.56785755689968e-06\\
599.82	1.41574727823349e-06\\
599.83	1.27092188665495e-06\\
599.84	1.1334572449697e-06\\
599.85	1.00342996981438e-06\\
599.86	8.80917439232201e-07\\
599.87	7.65997800316123e-07\\
599.88	6.58749976879813e-07\\
599.89	5.59253677163279e-07\\
599.9	4.67589401585353e-07\\
599.91	3.83838450489227e-07\\
599.92	3.08082931920958e-07\\
599.93	2.40405769407967e-07\\
599.94	1.8089070970978e-07\\
599.95	1.2962233057745e-07\\
599.96	8.66860484539933e-08\\
599.97	5.21681261331924e-08\\
599.98	2.61556803542173e-08\\
599.99	8.73668930083393e-09\\
600	0\\
};
\addplot [color=mycolor18,solid,forget plot]
  table[row sep=crcr]{%
0.01	0.0085282693244888\\
1.01	0.00852826887287535\\
2.01	0.00852826841151541\\
3.01	0.00852826794019706\\
4.01	0.00852826745870373\\
5.01	0.00852826696681405\\
6.01	0.00852826646430188\\
7.01	0.00852826595093601\\
8.01	0.00852826542648029\\
9.01	0.00852826489069324\\
10.01	0.0085282643433281\\
11.01	0.00852826378413266\\
12.01	0.00852826321284936\\
13.01	0.0085282626292147\\
14.01	0.00852826203295951\\
15.01	0.00852826142380871\\
16.01	0.00852826080148116\\
17.01	0.00852826016568943\\
18.01	0.00852825951613987\\
19.01	0.00852825885253234\\
20.01	0.00852825817456001\\
21.01	0.0085282574819093\\
22.01	0.00852825677425974\\
23.01	0.00852825605128382\\
24.01	0.00852825531264676\\
25.01	0.00852825455800638\\
26.01	0.00852825378701293\\
27.01	0.00852825299930898\\
28.01	0.00852825219452916\\
29.01	0.00852825137229997\\
30.01	0.0085282505322398\\
31.01	0.00852824967395849\\
32.01	0.00852824879705723\\
33.01	0.00852824790112842\\
34.01	0.00852824698575541\\
35.01	0.00852824605051238\\
36.01	0.008528245094964\\
37.01	0.00852824411866531\\
38.01	0.00852824312116146\\
39.01	0.00852824210198756\\
40.01	0.00852824106066839\\
41.01	0.00852823999671815\\
42.01	0.00852823890964029\\
43.01	0.00852823779892717\\
44.01	0.00852823666405996\\
45.01	0.00852823550450822\\
46.01	0.00852823431972977\\
47.01	0.00852823310917022\\
48.01	0.00852823187226311\\
49.01	0.00852823060842911\\
50.01	0.00852822931707615\\
51.01	0.00852822799759891\\
52.01	0.00852822664937859\\
53.01	0.00852822527178259\\
54.01	0.00852822386416421\\
55.01	0.00852822242586225\\
56.01	0.00852822095620088\\
57.01	0.00852821945448912\\
58.01	0.00852821792002046\\
59.01	0.00852821635207279\\
60.01	0.00852821474990769\\
61.01	0.0085282131127703\\
62.01	0.00852821143988891\\
63.01	0.00852820973047442\\
64.01	0.0085282079837202\\
65.01	0.0085282061988014\\
66.01	0.00852820437487486\\
67.01	0.0085282025110784\\
68.01	0.00852820060653054\\
69.01	0.00852819866033\\
70.01	0.0085281966715553\\
71.01	0.00852819463926426\\
72.01	0.00852819256249349\\
73.01	0.00852819044025805\\
74.01	0.00852818827155076\\
75.01	0.00852818605534186\\
76.01	0.0085281837905783\\
77.01	0.00852818147618344\\
78.01	0.00852817911105634\\
79.01	0.0085281766940713\\
80.01	0.00852817422407719\\
81.01	0.00852817169989689\\
82.01	0.0085281691203268\\
83.01	0.00852816648413615\\
84.01	0.00852816379006628\\
85.01	0.00852816103683019\\
86.01	0.00852815822311177\\
87.01	0.00852815534756507\\
88.01	0.00852815240881381\\
89.01	0.00852814940545044\\
90.01	0.00852814633603559\\
91.01	0.00852814319909731\\
92.01	0.00852813999313021\\
93.01	0.00852813671659478\\
94.01	0.00852813336791666\\
95.01	0.00852812994548565\\
96.01	0.00852812644765504\\
97.01	0.00852812287274061\\
98.01	0.00852811921901989\\
99.01	0.00852811548473132\\
100.01	0.00852811166807319\\
101.01	0.00852810776720269\\
102.01	0.00852810378023512\\
103.01	0.00852809970524282\\
104.01	0.00852809554025416\\
105.01	0.00852809128325249\\
106.01	0.0085280869321752\\
107.01	0.00852808248491249\\
108.01	0.00852807793930634\\
109.01	0.00852807329314944\\
110.01	0.00852806854418401\\
111.01	0.0085280636901005\\
112.01	0.00852805872853654\\
113.01	0.00852805365707559\\
114.01	0.0085280484732458\\
115.01	0.0085280431745185\\
116.01	0.00852803775830707\\
117.01	0.00852803222196547\\
118.01	0.0085280265627868\\
119.01	0.008528020778002\\
120.01	0.00852801486477833\\
121.01	0.00852800882021782\\
122.01	0.00852800264135578\\
123.01	0.00852799632515918\\
124.01	0.00852798986852503\\
125.01	0.00852798326827892\\
126.01	0.00852797652117303\\
127.01	0.0085279696238846\\
128.01	0.00852796257301403\\
129.01	0.00852795536508316\\
130.01	0.00852794799653337\\
131.01	0.00852794046372366\\
132.01	0.0085279327629287\\
133.01	0.00852792489033672\\
134.01	0.0085279168420477\\
135.01	0.00852790861407104\\
136.01	0.00852790020232355\\
137.01	0.00852789160262709\\
138.01	0.00852788281070649\\
139.01	0.0085278738221871\\
140.01	0.00852786463259244\\
141.01	0.00852785523734174\\
142.01	0.00852784563174758\\
143.01	0.00852783581101312\\
144.01	0.00852782577022966\\
145.01	0.00852781550437395\\
146.01	0.00852780500830533\\
147.01	0.00852779427676297\\
148.01	0.00852778330436295\\
149.01	0.00852777208559542\\
150.01	0.00852776061482139\\
151.01	0.00852774888626977\\
152.01	0.00852773689403411\\
153.01	0.00852772463206939\\
154.01	0.00852771209418864\\
155.01	0.00852769927405947\\
156.01	0.00852768616520067\\
157.01	0.0085276727609785\\
158.01	0.00852765905460308\\
159.01	0.00852764503912454\\
160.01	0.0085276307074292\\
161.01	0.00852761605223562\\
162.01	0.00852760106609042\\
163.01	0.00852758574136424\\
164.01	0.00852757007024735\\
165.01	0.00852755404474531\\
166.01	0.00852753765667452\\
167.01	0.00852752089765755\\
168.01	0.00852750375911848\\
169.01	0.00852748623227795\\
170.01	0.00852746830814835\\
171.01	0.00852744997752857\\
172.01	0.008527431230999\\
173.01	0.00852741205891598\\
174.01	0.00852739245140631\\
175.01	0.00852737239836189\\
176.01	0.00852735188943368\\
177.01	0.00852733091402596\\
178.01	0.00852730946129019\\
179.01	0.0085272875201189\\
180.01	0.00852726507913922\\
181.01	0.00852724212670645\\
182.01	0.00852721865089733\\
183.01	0.00852719463950314\\
184.01	0.00852717008002282\\
185.01	0.00852714495965563\\
186.01	0.00852711926529373\\
187.01	0.00852709298351479\\
188.01	0.00852706610057399\\
189.01	0.00852703860239624\\
190.01	0.00852701047456802\\
191.01	0.00852698170232879\\
192.01	0.00852695227056261\\
193.01	0.00852692216378937\\
194.01	0.00852689136615564\\
195.01	0.00852685986142535\\
196.01	0.00852682763297052\\
197.01	0.00852679466376136\\
198.01	0.0085267609363564\\
199.01	0.00852672643289213\\
200.01	0.00852669113507269\\
201.01	0.00852665502415893\\
202.01	0.00852661808095746\\
203.01	0.00852658028580938\\
204.01	0.00852654161857869\\
205.01	0.00852650205864015\\
206.01	0.0085264615848675\\
207.01	0.00852642017562057\\
208.01	0.00852637780873278\\
209.01	0.00852633446149765\\
210.01	0.00852629011065562\\
211.01	0.00852624473238001\\
212.01	0.00852619830226295\\
213.01	0.00852615079530072\\
214.01	0.00852610218587904\\
215.01	0.0085260524477576\\
216.01	0.00852600155405443\\
217.01	0.00852594947722981\\
218.01	0.0085258961890697\\
219.01	0.00852584166066901\\
220.01	0.00852578586241413\\
221.01	0.00852572876396513\\
222.01	0.00852567033423764\\
223.01	0.00852561054138394\\
224.01	0.00852554935277405\\
225.01	0.00852548673497577\\
226.01	0.00852542265373473\\
227.01	0.00852535707395371\\
228.01	0.00852528995967124\\
229.01	0.00852522127404003\\
230.01	0.0085251509793045\\
231.01	0.00852507903677823\\
232.01	0.0085250054068201\\
233.01	0.00852493004881053\\
234.01	0.00852485292112682\\
235.01	0.00852477398111765\\
236.01	0.00852469318507734\\
237.01	0.00852461048821923\\
238.01	0.00852452584464841\\
239.01	0.00852443920733379\\
240.01	0.00852435052807936\\
241.01	0.00852425975749496\\
242.01	0.00852416684496602\\
243.01	0.00852407173862272\\
244.01	0.00852397438530831\\
245.01	0.0085238747305466\\
246.01	0.00852377271850874\\
247.01	0.00852366829197907\\
248.01	0.00852356139232001\\
249.01	0.0085234519594363\\
250.01	0.00852333993173823\\
251.01	0.00852322524610382\\
252.01	0.00852310783784022\\
253.01	0.00852298764064399\\
254.01	0.00852286458656052\\
255.01	0.00852273860594251\\
256.01	0.00852260962740696\\
257.01	0.00852247757779165\\
258.01	0.00852234238211008\\
259.01	0.00852220396350558\\
260.01	0.00852206224320412\\
261.01	0.00852191714046591\\
262.01	0.00852176857253592\\
263.01	0.00852161645459301\\
264.01	0.00852146069969794\\
265.01	0.00852130121873993\\
266.01	0.00852113792038201\\
267.01	0.00852097071100482\\
268.01	0.0085207994946494\\
269.01	0.00852062417295801\\
270.01	0.00852044464511388\\
271.01	0.00852026080777938\\
272.01	0.00852007255503248\\
273.01	0.00851987977830176\\
274.01	0.00851968236629997\\
275.01	0.00851948020495553\\
276.01	0.00851927317734264\\
277.01	0.00851906116360957\\
278.01	0.00851884404090515\\
279.01	0.00851862168330348\\
280.01	0.0085183939617267\\
281.01	0.00851816074386581\\
282.01	0.00851792189409971\\
283.01	0.00851767727341198\\
284.01	0.00851742673930586\\
285.01	0.00851717014571683\\
286.01	0.00851690734292317\\
287.01	0.00851663817745436\\
288.01	0.00851636249199709\\
289.01	0.00851608012529895\\
290.01	0.00851579091206974\\
291.01	0.00851549468288027\\
292.01	0.00851519126405875\\
293.01	0.00851488047758453\\
294.01	0.00851456214097902\\
295.01	0.00851423606719427\\
296.01	0.00851390206449821\\
297.01	0.00851355993635757\\
298.01	0.00851320948131761\\
299.01	0.00851285049287855\\
300.01	0.00851248275936947\\
301.01	0.00851210606381833\\
302.01	0.00851172018381939\\
303.01	0.00851132489139673\\
304.01	0.00851091995286449\\
305.01	0.00851050512868361\\
306.01	0.00851008017331476\\
307.01	0.00850964483506748\\
308.01	0.00850919885594564\\
309.01	0.00850874197148854\\
310.01	0.00850827391060823\\
311.01	0.00850779439542229\\
312.01	0.00850730314108252\\
313.01	0.0085067998555988\\
314.01	0.00850628423965839\\
315.01	0.00850575598644074\\
316.01	0.00850521478142667\\
317.01	0.00850466030220307\\
318.01	0.00850409221826213\\
319.01	0.00850351019079479\\
320.01	0.00850291387247883\\
321.01	0.00850230290726128\\
322.01	0.00850167693013432\\
323.01	0.0085010355669054\\
324.01	0.00850037843396062\\
325.01	0.00849970513802145\\
326.01	0.0084990152758946\\
327.01	0.00849830843421463\\
328.01	0.00849758418917936\\
329.01	0.00849684210627718\\
330.01	0.00849608174000655\\
331.01	0.00849530263358736\\
332.01	0.00849450431866355\\
333.01	0.00849368631499666\\
334.01	0.0084928481301504\\
335.01	0.00849198925916552\\
336.01	0.00849110918422448\\
337.01	0.00849020737430603\\
338.01	0.00848928328482888\\
339.01	0.00848833635728427\\
340.01	0.00848736601885668\\
341.01	0.00848637168203296\\
342.01	0.00848535274419822\\
343.01	0.00848430858721917\\
344.01	0.00848323857701382\\
345.01	0.00848214206310694\\
346.01	0.00848101837817107\\
347.01	0.00847986683755223\\
348.01	0.00847868673877964\\
349.01	0.00847747736105926\\
350.01	0.00847623796475034\\
351.01	0.00847496779082389\\
352.01	0.00847366606030291\\
353.01	0.00847233197368387\\
354.01	0.00847096471033751\\
355.01	0.00846956342789008\\
356.01	0.00846812726158228\\
357.01	0.00846665532360672\\
358.01	0.00846514670242193\\
359.01	0.00846360046204281\\
360.01	0.00846201564130635\\
361.01	0.00846039125311161\\
362.01	0.00845872628363337\\
363.01	0.00845701969150814\\
364.01	0.00845527040699201\\
365.01	0.00845347733108871\\
366.01	0.00845163933464759\\
367.01	0.00844975525743003\\
368.01	0.00844782390714294\\
369.01	0.00844584405843904\\
370.01	0.00844381445188201\\
371.01	0.00844173379287559\\
372.01	0.00843960075055563\\
373.01	0.00843741395664327\\
374.01	0.0084351720042585\\
375.01	0.00843287344669222\\
376.01	0.00843051679613575\\
377.01	0.00842810052236599\\
378.01	0.00842562305138452\\
379.01	0.00842308276400934\\
380.01	0.00842047799441703\\
381.01	0.00841780702863358\\
382.01	0.00841506810297179\\
383.01	0.00841225940241283\\
384.01	0.00840937905892985\\
385.01	0.00840642514975087\\
386.01	0.00840339569555807\\
387.01	0.00840028865862046\\
388.01	0.00839710194085653\\
389.01	0.00839383338182355\\
390.01	0.00839048075662889\\
391.01	0.00838704177375985\\
392.01	0.00838351407282648\\
393.01	0.00837989522221236\\
394.01	0.00837618271662836\\
395.01	0.00837237397456203\\
396.01	0.00836846633561692\\
397.01	0.0083644570577336\\
398.01	0.00836034331428519\\
399.01	0.00835612219103809\\
400.01	0.00835179068296904\\
401.01	0.00834734569092825\\
402.01	0.00834278401813774\\
403.01	0.0083381023665139\\
404.01	0.00833329733280185\\
405.01	0.00832836540450924\\
406.01	0.00832330295562662\\
407.01	0.00831810624212161\\
408.01	0.00831277139719408\\
409.01	0.00830729442628117\\
410.01	0.0083016712018026\\
411.01	0.00829589745764005\\
412.01	0.00828996878335004\\
413.01	0.00828388061811719\\
414.01	0.00827762824445846\\
415.01	0.00827120678153902\\
416.01	0.00826461117320351\\
417.01	0.0082578361846318\\
418.01	0.00825087643389198\\
419.01	0.00824372635047618\\
420.01	0.00823638016567071\\
421.01	0.00822883193369906\\
422.01	0.00822107557696015\\
423.01	0.00821310472571571\\
424.01	0.00820491269494183\\
425.01	0.00819649253760645\\
426.01	0.00818783703140281\\
427.01	0.00817893866291245\\
428.01	0.00816978961040823\\
429.01	0.00816038172513833\\
430.01	0.00815070651090978\\
431.01	0.00814075510176216\\
432.01	0.00813051823749057\\
433.01	0.00811998623674044\\
434.01	0.00810914896735115\\
435.01	0.00809799581357453\\
436.01	0.00808651563973082\\
437.01	0.00807469674979188\\
438.01	0.00806252684229236\\
439.01	0.00804999295986385\\
440.01	0.00803708143255964\\
441.01	0.00802377781398531\\
442.01	0.00801006680906401\\
443.01	0.00799593219204271\\
444.01	0.00798135671307224\\
445.01	0.00796632199136345\\
446.01	0.00795080839251616\\
447.01	0.00793479488712468\\
448.01	0.00791825888715257\\
449.01	0.00790117605582211\\
450.01	0.00788352008580184\\
451.01	0.00786526243873771\\
452.01	0.00784637203025313\\
453.01	0.00782681487280826\\
454.01	0.00780655371765629\\
455.01	0.00778554744100704\\
456.01	0.00776375037486257\\
457.01	0.00774111152438106\\
458.01	0.00771757362418356\\
459.01	0.007693072062799\\
460.01	0.00766753430450629\\
461.01	0.00765013319258523\\
462.01	0.00763908808592548\\
463.01	0.00762750964586218\\
464.01	0.00761534584623699\\
465.01	0.00760253654626921\\
466.01	0.00758901191196083\\
467.01	0.00757469048882546\\
468.01	0.0075594768419738\\
469.01	0.00754325865786523\\
470.01	0.00752590317422022\\
471.01	0.0075072527753302\\
472.01	0.0074871195413191\\
473.01	0.00746527851243589\\
474.01	0.00744145998050491\\
475.01	0.00741583934100435\\
476.01	0.00738948680289242\\
477.01	0.00736244307460696\\
478.01	0.00733469087315198\\
479.01	0.00730621265144998\\
480.01	0.0072769906088408\\
481.01	0.0072470067004919\\
482.01	0.00721624264423079\\
483.01	0.00718467992263655\\
484.01	0.00715229977706487\\
485.01	0.00711908318482394\\
486.01	0.00708501073440597\\
487.01	0.00705006155323819\\
488.01	0.00701421368066979\\
489.01	0.00697744713463668\\
490.01	0.00693974183472198\\
491.01	0.00690107722893671\\
492.01	0.00686143209391965\\
493.01	0.00682078424647268\\
494.01	0.00677911014682784\\
495.01	0.0067363844159549\\
496.01	0.00669257897406453\\
497.01	0.00664766209734195\\
498.01	0.00660161409848995\\
499.01	0.006554423046006\\
500.01	0.00650608067788641\\
501.01	0.00645658341378783\\
502.01	0.0064059336528664\\
503.01	0.00635414145503738\\
504.01	0.00630122645225831\\
505.01	0.00624721976066043\\
506.01	0.00619216778140031\\
507.01	0.00613613602116493\\
508.01	0.00607921259138708\\
509.01	0.00602151344421225\\
510.01	0.00596318885914\\
511.01	0.00590443146170561\\
512.01	0.0058454861538424\\
513.01	0.00578666243426488\\
514.01	0.00572834971386359\\
515.01	0.00567103640614396\\
516.01	0.00561533396415977\\
517.01	0.0055620088737059\\
518.01	0.00551059015308814\\
519.01	0.00545824927896969\\
520.01	0.0054049202581716\\
521.01	0.00535069205361565\\
522.01	0.00529567320000898\\
523.01	0.00523949329040571\\
524.01	0.00518161246188112\\
525.01	0.00512196036795267\\
526.01	0.00506046576669617\\
527.01	0.00499706214664321\\
528.01	0.00493169862185053\\
529.01	0.0048643375711972\\
530.01	0.0047949600813584\\
531.01	0.00472357368012475\\
532.01	0.00465022245915389\\
533.01	0.00457499852317799\\
534.01	0.00449799425098407\\
535.01	0.00441925191459439\\
536.01	0.00433906268995177\\
537.01	0.00425833640930586\\
538.01	0.00417838431595103\\
539.01	0.00409939265699864\\
540.01	0.00402143966493794\\
541.01	0.00394459238813641\\
542.01	0.00386891012544723\\
543.01	0.00379451669365658\\
544.01	0.00372165678674702\\
545.01	0.00365058060888114\\
546.01	0.00358150836090378\\
547.01	0.00351463948028555\\
548.01	0.00345014330083666\\
549.01	0.00338809864123182\\
550.01	0.00332842287084302\\
551.01	0.00326978705576702\\
552.01	0.0032115050237175\\
553.01	0.0031536159779063\\
554.01	0.00309614412897075\\
555.01	0.0030390947138821\\
556.01	0.00298245005115764\\
557.01	0.00292616583595391\\
558.01	0.00287016822054579\\
559.01	0.00281435251935063\\
560.01	0.00275858482690329\\
561.01	0.00270270824872089\\
562.01	0.00264655033766167\\
563.01	0.00258995010948459\\
564.01	0.00253284519234311\\
565.01	0.00247521006594279\\
566.01	0.00241701529277463\\
567.01	0.00235822741438245\\
568.01	0.00229880868214501\\
569.01	0.00223871784692923\\
570.01	0.00217791174383267\\
571.01	0.00211634742053402\\
572.01	0.0020539848162932\\
573.01	0.00199078985346574\\
574.01	0.00192673690423219\\
575.01	0.00186180846663631\\
576.01	0.00179598983847555\\
577.01	0.00172926700343405\\
578.01	0.00166162680595677\\
579.01	0.00159305719081834\\
580.01	0.00152354764606489\\
581.01	0.00145308967998356\\
582.01	0.0013816772350871\\
583.01	0.00130930697684359\\
584.01	0.00123597839125735\\
585.01	0.00116169364078332\\
586.01	0.00108645721274621\\
587.01	0.00101027566830084\\
588.01	0.000933157587584601\\
589.01	0.000855113653909664\\
590.01	0.000776156728517496\\
591.01	0.000696301791282395\\
592.01	0.000615565671439002\\
593.01	0.000533966471584386\\
594.01	0.000451522561386629\\
595.01	0.000368250987049191\\
596.01	0.000284165109520231\\
597.01	0.000199271250183436\\
598.01	0.000113564089302298\\
599.01	3.18261402467001e-05\\
599.02	3.12794051966058e-05\\
599.03	3.07361362371396e-05\\
599.04	3.01962768802927e-05\\
599.05	2.96598193405082e-05\\
599.06	2.91267532562291e-05\\
599.07	2.85970655242371e-05\\
599.08	2.80707764810702e-05\\
599.09	2.75479164762726e-05\\
599.1	2.70285167089327e-05\\
599.11	2.65126085881601e-05\\
599.12	2.60002240642203e-05\\
599.13	2.54913970200877e-05\\
599.14	2.49861616816692e-05\\
599.15	2.44845526258131e-05\\
599.16	2.39866047888303e-05\\
599.17	2.34923535378934e-05\\
599.18	2.30018347767025e-05\\
599.19	2.251508481242e-05\\
599.2	2.20321403675509e-05\\
599.21	2.15530385901663e-05\\
599.22	2.10778170631979e-05\\
599.23	2.06065138181385e-05\\
599.24	2.01391673495636e-05\\
599.25	1.96758166306349e-05\\
599.26	1.92165011295622e-05\\
599.27	1.87612607734713e-05\\
599.28	1.8310135886844e-05\\
599.29	1.78631671960919e-05\\
599.3	1.74203958342063e-05\\
599.31	1.69818633422789e-05\\
599.32	1.65476116708516e-05\\
599.33	1.61176831841251e-05\\
599.34	1.56921206641847e-05\\
599.35	1.52709673147209e-05\\
599.36	1.485426676507e-05\\
599.37	1.44420630744727e-05\\
599.38	1.40344007363812e-05\\
599.39	1.36313255098709e-05\\
599.4	1.3232886592919e-05\\
599.41	1.28391336731322e-05\\
599.42	1.24501169325771e-05\\
599.43	1.20658870526348e-05\\
599.44	1.16864952188697e-05\\
599.45	1.13119931259391e-05\\
599.46	1.09424329824918e-05\\
599.47	1.05778675161157e-05\\
599.48	1.02183499782645e-05\\
599.49	9.86393414922036e-06\\
599.5	9.51467434305575e-06\\
599.51	9.1706254125995e-06\\
599.52	8.83184275438965e-06\\
599.53	8.4983823136868e-06\\
599.54	8.17030058985514e-06\\
599.55	7.84765464179223e-06\\
599.56	7.53050209342451e-06\\
599.57	7.21890113926371e-06\\
599.58	6.91291055002208e-06\\
599.59	6.61258967827975e-06\\
599.6	6.31799846421283e-06\\
599.61	6.02919744138383e-06\\
599.62	5.74624774257564e-06\\
599.63	5.46921110568602e-06\\
599.64	5.19814987969512e-06\\
599.65	4.9331270306676e-06\\
599.66	4.67420614782936e-06\\
599.67	4.42145144970672e-06\\
599.68	4.17492779030207e-06\\
599.69	3.93470066536311e-06\\
599.7	3.70083621868862e-06\\
599.71	3.47340124850354e-06\\
599.72	3.25246321389479e-06\\
599.73	3.03809024131303e-06\\
599.74	2.83035113113513e-06\\
599.75	2.62931536429249e-06\\
599.76	2.43505310895328e-06\\
599.77	2.24763522728433e-06\\
599.78	2.06713328227207e-06\\
599.79	1.89361954460419e-06\\
599.8	1.72716699962938e-06\\
599.81	1.56784935437075e-06\\
599.82	1.41574104461396e-06\\
599.83	1.27091724206295e-06\\
599.84	1.13345386157197e-06\\
599.85	1.00342756843147e-06\\
599.86	8.80915785733682e-07\\
599.87	7.65996701814625e-07\\
599.88	6.58749277755027e-07\\
599.89	5.59253254961076e-07\\
599.9	4.67589162821483e-07\\
599.91	3.83838326433947e-07\\
599.92	3.08082874409671e-07\\
599.93	2.40405746740335e-07\\
599.94	1.80890702781294e-07\\
599.95	1.2962232925906e-07\\
599.96	8.66860484019516e-08\\
599.97	5.21681261349272e-08\\
599.98	2.61556803542173e-08\\
599.99	8.73668930083393e-09\\
600	0\\
};
\addplot [color=red!25!mycolor17,solid,forget plot]
  table[row sep=crcr]{%
0.01	0.00734126507339155\\
1.01	0.00734126455459817\\
2.01	0.00734126402465449\\
3.01	0.00734126348331942\\
4.01	0.00734126293034655\\
5.01	0.00734126236548402\\
6.01	0.0073412617884747\\
7.01	0.00734126119905562\\
8.01	0.00734126059695818\\
9.01	0.00734125998190793\\
10.01	0.00734125935362445\\
11.01	0.00734125871182122\\
12.01	0.00734125805620525\\
13.01	0.00734125738647739\\
14.01	0.00734125670233179\\
15.01	0.00734125600345608\\
16.01	0.00734125528953093\\
17.01	0.00734125456023013\\
18.01	0.00734125381522025\\
19.01	0.00734125305416054\\
20.01	0.00734125227670295\\
21.01	0.00734125148249162\\
22.01	0.00734125067116316\\
23.01	0.00734124984234594\\
24.01	0.00734124899566034\\
25.01	0.00734124813071843\\
26.01	0.00734124724712374\\
27.01	0.00734124634447111\\
28.01	0.0073412454223465\\
29.01	0.00734124448032689\\
30.01	0.00734124351797983\\
31.01	0.00734124253486344\\
32.01	0.00734124153052618\\
33.01	0.00734124050450658\\
34.01	0.00734123945633304\\
35.01	0.00734123838552357\\
36.01	0.00734123729158562\\
37.01	0.0073412361740158\\
38.01	0.00734123503229976\\
39.01	0.00734123386591169\\
40.01	0.0073412326743143\\
41.01	0.00734123145695845\\
42.01	0.0073412302132829\\
43.01	0.00734122894271422\\
44.01	0.00734122764466611\\
45.01	0.0073412263185395\\
46.01	0.0073412249637221\\
47.01	0.00734122357958816\\
48.01	0.00734122216549811\\
49.01	0.00734122072079827\\
50.01	0.00734121924482052\\
51.01	0.00734121773688207\\
52.01	0.00734121619628499\\
53.01	0.00734121462231603\\
54.01	0.00734121301424613\\
55.01	0.00734121137133014\\
56.01	0.00734120969280643\\
57.01	0.00734120797789652\\
58.01	0.00734120622580483\\
59.01	0.00734120443571808\\
60.01	0.00734120260680509\\
61.01	0.00734120073821619\\
62.01	0.00734119882908289\\
63.01	0.00734119687851762\\
64.01	0.00734119488561302\\
65.01	0.00734119284944171\\
66.01	0.0073411907690556\\
67.01	0.00734118864348575\\
68.01	0.00734118647174166\\
69.01	0.00734118425281084\\
70.01	0.00734118198565839\\
71.01	0.00734117966922635\\
72.01	0.00734117730243338\\
73.01	0.00734117488417396\\
74.01	0.0073411724133181\\
75.01	0.00734116988871071\\
76.01	0.00734116730917097\\
77.01	0.00734116467349179\\
78.01	0.00734116198043919\\
79.01	0.00734115922875178\\
80.01	0.00734115641713994\\
81.01	0.00734115354428552\\
82.01	0.00734115060884088\\
83.01	0.00734114760942827\\
84.01	0.00734114454463934\\
85.01	0.00734114141303422\\
86.01	0.00734113821314096\\
87.01	0.00734113494345461\\
88.01	0.00734113160243682\\
89.01	0.00734112818851467\\
90.01	0.00734112470008018\\
91.01	0.00734112113548928\\
92.01	0.00734111749306128\\
93.01	0.00734111377107779\\
94.01	0.00734110996778191\\
95.01	0.00734110608137747\\
96.01	0.00734110211002788\\
97.01	0.00734109805185562\\
98.01	0.00734109390494087\\
99.01	0.00734108966732074\\
100.01	0.00734108533698838\\
101.01	0.00734108091189177\\
102.01	0.00734107638993288\\
103.01	0.00734107176896641\\
104.01	0.00734106704679891\\
105.01	0.00734106222118752\\
106.01	0.00734105728983888\\
107.01	0.00734105225040815\\
108.01	0.00734104710049755\\
109.01	0.00734104183765527\\
110.01	0.0073410364593742\\
111.01	0.00734103096309071\\
112.01	0.00734102534618341\\
113.01	0.00734101960597159\\
114.01	0.00734101373971391\\
115.01	0.00734100774460734\\
116.01	0.00734100161778521\\
117.01	0.00734099535631623\\
118.01	0.00734098895720262\\
119.01	0.00734098241737884\\
120.01	0.00734097573370982\\
121.01	0.00734096890298946\\
122.01	0.00734096192193895\\
123.01	0.0073409547872052\\
124.01	0.00734094749535894\\
125.01	0.00734094004289301\\
126.01	0.00734093242622056\\
127.01	0.00734092464167328\\
128.01	0.0073409166854994\\
129.01	0.00734090855386181\\
130.01	0.00734090024283592\\
131.01	0.00734089174840787\\
132.01	0.00734088306647234\\
133.01	0.00734087419283038\\
134.01	0.00734086512318723\\
135.01	0.00734085585315009\\
136.01	0.00734084637822589\\
137.01	0.00734083669381879\\
138.01	0.00734082679522806\\
139.01	0.00734081667764527\\
140.01	0.00734080633615198\\
141.01	0.00734079576571723\\
142.01	0.00734078496119466\\
143.01	0.00734077391732002\\
144.01	0.00734076262870825\\
145.01	0.00734075108985067\\
146.01	0.00734073929511214\\
147.01	0.00734072723872807\\
148.01	0.00734071491480133\\
149.01	0.0073407023172991\\
150.01	0.00734068944004967\\
151.01	0.00734067627673933\\
152.01	0.00734066282090877\\
153.01	0.00734064906594976\\
154.01	0.00734063500510159\\
155.01	0.00734062063144758\\
156.01	0.00734060593791108\\
157.01	0.00734059091725212\\
158.01	0.00734057556206308\\
159.01	0.00734055986476504\\
160.01	0.0073405438176036\\
161.01	0.00734052741264453\\
162.01	0.00734051064176986\\
163.01	0.00734049349667319\\
164.01	0.00734047596885525\\
165.01	0.00734045804961941\\
166.01	0.00734043973006696\\
167.01	0.00734042100109207\\
168.01	0.00734040185337714\\
169.01	0.00734038227738751\\
170.01	0.00734036226336636\\
171.01	0.00734034180132939\\
172.01	0.00734032088105916\\
173.01	0.00734029949209986\\
174.01	0.00734027762375134\\
175.01	0.00734025526506323\\
176.01	0.00734023240482905\\
177.01	0.00734020903157987\\
178.01	0.00734018513357814\\
179.01	0.00734016069881111\\
180.01	0.00734013571498427\\
181.01	0.00734011016951444\\
182.01	0.00734008404952291\\
183.01	0.00734005734182818\\
184.01	0.00734003003293865\\
185.01	0.00734000210904514\\
186.01	0.00733997355601325\\
187.01	0.00733994435937539\\
188.01	0.00733991450432257\\
189.01	0.00733988397569632\\
190.01	0.00733985275797997\\
191.01	0.00733982083529023\\
192.01	0.00733978819136796\\
193.01	0.00733975480956899\\
194.01	0.00733972067285493\\
195.01	0.00733968576378361\\
196.01	0.00733965006449877\\
197.01	0.00733961355672036\\
198.01	0.00733957622173397\\
199.01	0.0073395380403803\\
200.01	0.00733949899304402\\
201.01	0.00733945905964283\\
202.01	0.00733941821961589\\
203.01	0.00733937645191195\\
204.01	0.00733933373497738\\
205.01	0.00733929004674399\\
206.01	0.00733924536461594\\
207.01	0.00733919966545706\\
208.01	0.00733915292557724\\
209.01	0.00733910512071911\\
210.01	0.00733905622604368\\
211.01	0.00733900621611606\\
212.01	0.00733895506489082\\
213.01	0.0073389027456968\\
214.01	0.00733884923122136\\
215.01	0.00733879449349471\\
216.01	0.00733873850387362\\
217.01	0.00733868123302454\\
218.01	0.00733862265090651\\
219.01	0.00733856272675349\\
220.01	0.00733850142905631\\
221.01	0.00733843872554425\\
222.01	0.00733837458316583\\
223.01	0.00733830896806951\\
224.01	0.00733824184558348\\
225.01	0.00733817318019561\\
226.01	0.0073381029355318\\
227.01	0.00733803107433481\\
228.01	0.00733795755844195\\
229.01	0.00733788234876259\\
230.01	0.00733780540525474\\
231.01	0.00733772668690099\\
232.01	0.00733764615168425\\
233.01	0.00733756375656269\\
234.01	0.00733747945744349\\
235.01	0.00733739320915682\\
236.01	0.00733730496542871\\
237.01	0.00733721467885293\\
238.01	0.00733712230086257\\
239.01	0.00733702778170085\\
240.01	0.00733693107039089\\
241.01	0.00733683211470503\\
242.01	0.00733673086113305\\
243.01	0.00733662725484983\\
244.01	0.00733652123968203\\
245.01	0.00733641275807399\\
246.01	0.00733630175105267\\
247.01	0.00733618815819157\\
248.01	0.00733607191757416\\
249.01	0.00733595296575578\\
250.01	0.00733583123772501\\
251.01	0.00733570666686371\\
252.01	0.00733557918490631\\
253.01	0.00733544872189793\\
254.01	0.00733531520615136\\
255.01	0.00733517856420289\\
256.01	0.00733503872076728\\
257.01	0.00733489559869112\\
258.01	0.00733474911890551\\
259.01	0.00733459920037698\\
260.01	0.00733444576005756\\
261.01	0.00733428871283346\\
262.01	0.00733412797147213\\
263.01	0.00733396344656839\\
264.01	0.00733379504648899\\
265.01	0.00733362267731562\\
266.01	0.00733344624278665\\
267.01	0.0073332656442373\\
268.01	0.00733308078053815\\
269.01	0.00733289154803233\\
270.01	0.0073326978404708\\
271.01	0.00733249954894618\\
272.01	0.00733229656182496\\
273.01	0.00733208876467764\\
274.01	0.00733187604020731\\
275.01	0.00733165826817652\\
276.01	0.00733143532533209\\
277.01	0.007331207085328\\
278.01	0.00733097341864639\\
279.01	0.00733073419251666\\
280.01	0.00733048927083215\\
281.01	0.00733023851406531\\
282.01	0.00732998177918005\\
283.01	0.00732971891954244\\
284.01	0.00732944978482864\\
285.01	0.00732917422093099\\
286.01	0.00732889206986159\\
287.01	0.00732860316965309\\
288.01	0.00732830735425762\\
289.01	0.00732800445344265\\
290.01	0.00732769429268461\\
291.01	0.00732737669305965\\
292.01	0.00732705147113174\\
293.01	0.00732671843883795\\
294.01	0.0073263774033711\\
295.01	0.0073260281670591\\
296.01	0.00732567052724178\\
297.01	0.00732530427614428\\
298.01	0.00732492920074749\\
299.01	0.0073245450826555\\
300.01	0.00732415169795939\\
301.01	0.00732374881709819\\
302.01	0.00732333620471588\\
303.01	0.00732291361951545\\
304.01	0.00732248081410905\\
305.01	0.00732203753486475\\
306.01	0.00732158352174942\\
307.01	0.00732111850816789\\
308.01	0.00732064222079828\\
309.01	0.00732015437942341\\
310.01	0.00731965469675795\\
311.01	0.00731914287827173\\
312.01	0.00731861862200856\\
313.01	0.007318081618401\\
314.01	0.00731753155008062\\
315.01	0.00731696809168352\\
316.01	0.00731639090965177\\
317.01	0.00731579966202974\\
318.01	0.00731519399825555\\
319.01	0.00731457355894801\\
320.01	0.00731393797568816\\
321.01	0.00731328687079602\\
322.01	0.00731261985710168\\
323.01	0.00731193653771122\\
324.01	0.00731123650576725\\
325.01	0.00731051934420369\\
326.01	0.00730978462549459\\
327.01	0.00730903191139745\\
328.01	0.00730826075269011\\
329.01	0.00730747068890161\\
330.01	0.00730666124803663\\
331.01	0.00730583194629368\\
332.01	0.00730498228777586\\
333.01	0.00730411176419571\\
334.01	0.00730321985457197\\
335.01	0.0073023060249201\\
336.01	0.00730136972793427\\
337.01	0.00730041040266253\\
338.01	0.00729942747417348\\
339.01	0.00729842035321495\\
340.01	0.00729738843586434\\
341.01	0.00729633110316978\\
342.01	0.00729524772078298\\
343.01	0.00729413763858231\\
344.01	0.00729300019028617\\
345.01	0.00729183469305672\\
346.01	0.00729064044709288\\
347.01	0.00728941673521282\\
348.01	0.00728816282242548\\
349.01	0.00728687795549005\\
350.01	0.00728556136246381\\
351.01	0.00728421225223741\\
352.01	0.00728282981405698\\
353.01	0.00728141321703262\\
354.01	0.00727996160963335\\
355.01	0.00727847411916644\\
356.01	0.00727694985124217\\
357.01	0.00727538788922212\\
358.01	0.00727378729365114\\
359.01	0.00727214710167144\\
360.01	0.00727046632641884\\
361.01	0.00726874395639984\\
362.01	0.00726697895484907\\
363.01	0.00726517025906573\\
364.01	0.00726331677972892\\
365.01	0.00726141740019008\\
366.01	0.00725947097574201\\
367.01	0.00725747633286347\\
368.01	0.00725543226843849\\
369.01	0.0072533375489489\\
370.01	0.00725119090963921\\
371.01	0.00724899105365316\\
372.01	0.00724673665113992\\
373.01	0.00724442633833004\\
374.01	0.00724205871657875\\
375.01	0.0072396323513764\\
376.01	0.00723714577132455\\
377.01	0.00723459746707628\\
378.01	0.0072319858902405\\
379.01	0.00722930945224778\\
380.01	0.00722656652317783\\
381.01	0.00722375543054717\\
382.01	0.00722087445805516\\
383.01	0.00721792184428871\\
384.01	0.00721489578138365\\
385.01	0.00721179441364176\\
386.01	0.00720861583610336\\
387.01	0.00720535809307382\\
388.01	0.00720201917660352\\
389.01	0.00719859702492024\\
390.01	0.00719508952081353\\
391.01	0.00719149448997062\\
392.01	0.00718780969926303\\
393.01	0.00718403285498466\\
394.01	0.00718016160104026\\
395.01	0.00717619351708595\\
396.01	0.00717212611662225\\
397.01	0.00716795684504191\\
398.01	0.00716368307763514\\
399.01	0.0071593021175567\\
400.01	0.00715481119376046\\
401.01	0.00715020745890973\\
402.01	0.00714548798727503\\
403.01	0.00714064977263219\\
404.01	0.00713568972618098\\
405.01	0.00713060467450701\\
406.01	0.00712539135761644\\
407.01	0.00712004642707825\\
408.01	0.00711456644431503\\
409.01	0.0071089478790868\\
410.01	0.00710318710821348\\
411.01	0.00709728041457593\\
412.01	0.00709122398641762\\
413.01	0.00708501391693231\\
414.01	0.0070786462040412\\
415.01	0.00707211675003738\\
416.01	0.00706542136092544\\
417.01	0.00705855574717348\\
418.01	0.00705151552320142\\
419.01	0.00704429620649164\\
420.01	0.00703689321959463\\
421.01	0.00702930188928812\\
422.01	0.00702151739972556\\
423.01	0.00701353482583495\\
424.01	0.00700534917614831\\
425.01	0.0069969553614563\\
426.01	0.00698834819560703\\
427.01	0.00697952239786452\\
428.01	0.00697047259597068\\
429.01	0.00696119333002084\\
430.01	0.00695167905728106\\
431.01	0.00694192415809635\\
432.01	0.00693192294306247\\
433.01	0.00692166966166322\\
434.01	0.00691115851260798\\
435.01	0.00690038365614526\\
436.01	0.00688933922867446\\
437.01	0.0068780193600339\\
438.01	0.00686641819391086\\
439.01	0.00685452991189808\\
440.01	0.00684234876181789\\
441.01	0.00682986909104756\\
442.01	0.00681708538572061\\
443.01	0.00680399231684287\\
444.01	0.00679058479456501\\
445.01	0.0067768580321002\\
446.01	0.00676280762107438\\
447.01	0.00674842962046204\\
448.01	0.00673372066171481\\
449.01	0.00671867807323846\\
450.01	0.00670330002799456\\
451.01	0.00668758571715203\\
452.01	0.00667153548351915\\
453.01	0.00665515057052235\\
454.01	0.00663843559358089\\
455.01	0.00662139865022594\\
456.01	0.00660405128096307\\
457.01	0.00658640926690019\\
458.01	0.00656849362613499\\
459.01	0.00655033190451136\\
460.01	0.00653195996748447\\
461.01	0.00651338992196601\\
462.01	0.00649451268265961\\
463.01	0.00647533204807501\\
464.01	0.00645587440892889\\
465.01	0.00643617364606535\\
466.01	0.00641627280342872\\
467.01	0.00639622615410085\\
468.01	0.0063761017581098\\
469.01	0.00635598463514874\\
470.01	0.00633598069084061\\
471.01	0.00631622165165349\\
472.01	0.00629687134529433\\
473.01	0.00627813341218427\\
474.01	0.0062602617677022\\
475.01	0.00624307687454638\\
476.01	0.00622549082442006\\
477.01	0.00620743337667525\\
478.01	0.00618889187776189\\
479.01	0.00616985371567746\\
480.01	0.00615030642372044\\
481.01	0.00613023781053281\\
482.01	0.00610963612259572\\
483.01	0.00608849024679421\\
484.01	0.00606678996198266\\
485.01	0.00604452623973263\\
486.01	0.00602169128206165\\
487.01	0.00599826555399007\\
488.01	0.00597418206558548\\
489.01	0.00594941305773239\\
490.01	0.00592394139900088\\
491.01	0.00589775033755464\\
492.01	0.00587082362165473\\
493.01	0.00584314561672445\\
494.01	0.00581470141027783\\
495.01	0.00578547689168251\\
496.01	0.00575545871032362\\
497.01	0.00572463408788436\\
498.01	0.00569299073822983\\
499.01	0.00566051586224345\\
500.01	0.00562719492563089\\
501.01	0.00559300990457205\\
502.01	0.0055579343979256\\
503.01	0.00552192924888774\\
504.01	0.00548494980648591\\
505.01	0.00544693740411984\\
506.01	0.0054078213643012\\
507.01	0.00536754783778978\\
508.01	0.005326061155258\\
509.01	0.00528330235990979\\
510.01	0.00523920912376\\
511.01	0.00519371569541818\\
512.01	0.00514675288793071\\
513.01	0.00509824811625097\\
514.01	0.00504812549505705\\
515.01	0.00499630601086775\\
516.01	0.00494270780002012\\
517.01	0.00488724652526266\\
518.01	0.00482984735845951\\
519.01	0.00477055288280926\\
520.01	0.00470952603271803\\
521.01	0.0046472138863196\\
522.01	0.00458470131205031\\
523.01	0.00452219569378065\\
524.01	0.00445980107877253\\
525.01	0.00439765771346164\\
526.01	0.00433592197309556\\
527.01	0.00427476476752623\\
528.01	0.00421437048237607\\
529.01	0.00415493489511919\\
530.01	0.00409666162626346\\
531.01	0.00403975646328572\\
532.01	0.00398441874170262\\
533.01	0.00393082869873044\\
534.01	0.00387912966662672\\
535.01	0.00382940493481397\\
536.01	0.00378164249179029\\
537.01	0.00373524000500358\\
538.01	0.00368912833644244\\
539.01	0.00364330189677589\\
540.01	0.00359778390829597\\
541.01	0.00355259025757311\\
542.01	0.00350772801655309\\
543.01	0.00346319391049507\\
544.01	0.00341896868234572\\
545.01	0.00337501297581907\\
546.01	0.00333126511313421\\
547.01	0.00328763977111955\\
548.01	0.00324402773980095\\
549.01	0.00320029915127878\\
550.01	0.00315631167800961\\
551.01	0.00311195285554549\\
552.01	0.0030671882490935\\
553.01	0.00302199127603092\\
554.01	0.00297633053814613\\
555.01	0.00293017008017734\\
556.01	0.00288346993778933\\
557.01	0.00283618702641363\\
558.01	0.00278827640582232\\
559.01	0.0027396929194969\\
560.01	0.00269039314200024\\
561.01	0.00264033745611435\\
562.01	0.00258949198921666\\
563.01	0.00253782990454083\\
564.01	0.0024853288845688\\
565.01	0.00243196691153569\\
566.01	0.00237772185534557\\
567.01	0.00232257175221729\\
568.01	0.00226649513730902\\
569.01	0.00220947140606029\\
570.01	0.00215148115625655\\
571.01	0.00209250647178937\\
572.01	0.00203253110496798\\
573.01	0.00197154051305623\\
574.01	0.00190952172343432\\
575.01	0.00184646307785447\\
576.01	0.00178235412443806\\
577.01	0.0017171857660873\\
578.01	0.00165095045995029\\
579.01	0.00158364242165109\\
580.01	0.00151525782603943\\
581.01	0.00144579499572334\\
582.01	0.00137525457316008\\
583.01	0.00130363967477813\\
584.01	0.00123095602868033\\
585.01	0.00115721210051214\\
586.01	0.00108241921223445\\
587.01	0.00100659164275123\\
588.01	0.000929746671129862\\
589.01	0.000851904520258075\\
590.01	0.000773088158107163\\
591.01	0.000693322906705438\\
592.01	0.000612635797368357\\
593.01	0.000531054596121653\\
594.01	0.000448606405509172\\
595.01	0.00036531572727916\\
596.01	0.000281201843626103\\
597.01	0.000196275341074087\\
598.01	0.000110533558374905\\
599.01	3.18230760274988e-05\\
599.02	3.12771225888381e-05\\
599.03	3.07343852024352e-05\\
599.04	3.01948954524109e-05\\
599.05	2.96586853198184e-05\\
599.06	2.91257871969947e-05\\
599.07	2.85962339028577e-05\\
599.08	2.8070058607518e-05\\
599.09	2.75472948133501e-05\\
599.1	2.70279763573446e-05\\
599.11	2.65121374149781e-05\\
599.12	2.59998125032215e-05\\
599.13	2.54910364795111e-05\\
599.14	2.49858445451751e-05\\
599.15	2.44842722489057e-05\\
599.16	2.39863554902346e-05\\
599.17	2.34921305228321e-05\\
599.18	2.30016339574767e-05\\
599.19	2.25149027655146e-05\\
599.2	2.20319742823254e-05\\
599.21	2.15528862108072e-05\\
599.22	2.10776766248668e-05\\
599.23	2.06063839729136e-05\\
599.24	2.01390470813568e-05\\
599.25	1.96757051580971e-05\\
599.26	1.92163977960017e-05\\
599.27	1.87611649765159e-05\\
599.28	1.83100470734845e-05\\
599.29	1.78630848569894e-05\\
599.3	1.74203194972418e-05\\
599.31	1.69817925685135e-05\\
599.32	1.65475460531112e-05\\
599.33	1.61176223453848e-05\\
599.34	1.56920642557928e-05\\
599.35	1.52709150149869e-05\\
599.36	1.48542182779494e-05\\
599.37	1.44420181281777e-05\\
599.38	1.40343590818905e-05\\
599.39	1.36312869210512e-05\\
599.4	1.32328508630885e-05\\
599.41	1.2839100613395e-05\\
599.42	1.24500863701339e-05\\
599.43	1.20658588290742e-05\\
599.44	1.16864691884749e-05\\
599.45	1.13119691540261e-05\\
599.46	1.09424109438309e-05\\
599.47	1.05778472934363e-05\\
599.48	1.02183314609188e-05\\
599.49	9.86391723201314e-06\\
599.5	9.51465892530708e-06\\
599.51	9.17061139746149e-06\\
599.52	8.83183004850643e-06\\
599.53	8.49837082718408e-06\\
599.54	8.17029023633675e-06\\
599.55	7.84764533835573e-06\\
599.56	7.53049376068203e-06\\
599.57	7.21889370135405e-06\\
599.58	6.91290393461941e-06\\
599.59	6.61258381660576e-06\\
599.6	6.31799329102282e-06\\
599.61	6.02919289495116e-06\\
599.62	5.74624376466562e-06\\
599.63	5.46920764152997e-06\\
599.64	5.19814687793482e-06\\
599.65	4.93312444331713e-06\\
599.66	4.67420393020224e-06\\
599.67	4.42144956034653e-06\\
599.68	4.17492619090958e-06\\
599.69	3.93469932070262e-06\\
599.7	3.70083509648733e-06\\
599.71	3.47340031935442e-06\\
599.72	3.25246245113688e-06\\
599.73	3.03808962091867e-06\\
599.74	2.8303506315764e-06\\
599.75	2.62931496640938e-06\\
599.76	2.43505279581835e-06\\
599.77	2.24763498406222e-06\\
599.78	2.06713309606858e-06\\
599.79	1.89361940432578e-06\\
599.8	1.72716689582353e-06\\
599.81	1.56784927908028e-06\\
599.82	1.41574099122785e-06\\
599.83	1.27091720517752e-06\\
599.84	1.13345383683135e-06\\
599.85	1.00342755240089e-06\\
599.86	8.80915775762492e-07\\
599.87	7.65996695906157e-07\\
599.88	6.58749274453849e-07\\
599.89	5.59253253248904e-07\\
599.9	4.67589162013102e-07\\
599.91	3.8383832610088e-07\\
599.92	3.0808287429171e-07\\
599.93	2.4040574671258e-07\\
599.94	1.80890702777825e-07\\
599.95	1.2962232925906e-07\\
599.96	8.66860484019516e-08\\
599.97	5.21681261331924e-08\\
599.98	2.61556803542173e-08\\
599.99	8.73668929909921e-09\\
600	0\\
};
\addplot [color=mycolor19,solid,forget plot]
  table[row sep=crcr]{%
0.01	0.00662800561349471\\
1.01	0.00662800516698634\\
2.01	0.00662800471096339\\
3.01	0.00662800424522229\\
4.01	0.00662800376955526\\
5.01	0.00662800328375014\\
6.01	0.00662800278758982\\
7.01	0.00662800228085294\\
8.01	0.00662800176331315\\
9.01	0.00662800123473928\\
10.01	0.00662800069489526\\
11.01	0.00662800014353982\\
12.01	0.00662799958042655\\
13.01	0.00662799900530394\\
14.01	0.00662799841791487\\
15.01	0.00662799781799666\\
16.01	0.00662799720528108\\
17.01	0.00662799657949403\\
18.01	0.00662799594035567\\
19.01	0.00662799528758004\\
20.01	0.006627994620875\\
21.01	0.0066279939399423\\
22.01	0.00662799324447699\\
23.01	0.00662799253416775\\
24.01	0.00662799180869655\\
25.01	0.00662799106773842\\
26.01	0.00662799031096156\\
27.01	0.00662798953802681\\
28.01	0.00662798874858787\\
29.01	0.00662798794229095\\
30.01	0.00662798711877457\\
31.01	0.0066279862776695\\
32.01	0.00662798541859861\\
33.01	0.00662798454117659\\
34.01	0.00662798364500984\\
35.01	0.00662798272969625\\
36.01	0.00662798179482509\\
37.01	0.00662798083997678\\
38.01	0.00662797986472263\\
39.01	0.00662797886862471\\
40.01	0.00662797785123572\\
41.01	0.00662797681209865\\
42.01	0.00662797575074674\\
43.01	0.00662797466670292\\
44.01	0.0066279735594801\\
45.01	0.00662797242858051\\
46.01	0.00662797127349571\\
47.01	0.0066279700937063\\
48.01	0.0066279688886815\\
49.01	0.00662796765787931\\
50.01	0.00662796640074588\\
51.01	0.0066279651167154\\
52.01	0.00662796380520997\\
53.01	0.00662796246563897\\
54.01	0.00662796109739918\\
55.01	0.00662795969987448\\
56.01	0.00662795827243515\\
57.01	0.00662795681443804\\
58.01	0.00662795532522601\\
59.01	0.0066279538041277\\
60.01	0.00662795225045732\\
61.01	0.00662795066351422\\
62.01	0.00662794904258259\\
63.01	0.00662794738693112\\
64.01	0.00662794569581267\\
65.01	0.00662794396846396\\
66.01	0.00662794220410519\\
67.01	0.00662794040193965\\
68.01	0.00662793856115338\\
69.01	0.00662793668091491\\
70.01	0.0066279347603746\\
71.01	0.00662793279866449\\
72.01	0.00662793079489781\\
73.01	0.0066279287481686\\
74.01	0.00662792665755128\\
75.01	0.00662792452210008\\
76.01	0.00662792234084891\\
77.01	0.00662792011281058\\
78.01	0.00662791783697666\\
79.01	0.00662791551231669\\
80.01	0.00662791313777803\\
81.01	0.00662791071228504\\
82.01	0.00662790823473874\\
83.01	0.00662790570401648\\
84.01	0.00662790311897112\\
85.01	0.00662790047843062\\
86.01	0.0066278977811975\\
87.01	0.00662789502604836\\
88.01	0.00662789221173306\\
89.01	0.00662788933697446\\
90.01	0.00662788640046762\\
91.01	0.00662788340087927\\
92.01	0.00662788033684711\\
93.01	0.00662787720697921\\
94.01	0.00662787400985347\\
95.01	0.00662787074401677\\
96.01	0.0066278674079845\\
97.01	0.0066278640002396\\
98.01	0.00662786051923218\\
99.01	0.00662785696337853\\
100.01	0.00662785333106043\\
101.01	0.0066278496206245\\
102.01	0.00662784583038131\\
103.01	0.0066278419586047\\
104.01	0.00662783800353083\\
105.01	0.00662783396335757\\
106.01	0.00662782983624339\\
107.01	0.00662782562030659\\
108.01	0.0066278213136246\\
109.01	0.00662781691423285\\
110.01	0.00662781242012386\\
111.01	0.00662780782924653\\
112.01	0.00662780313950477\\
113.01	0.00662779834875697\\
114.01	0.00662779345481474\\
115.01	0.00662778845544178\\
116.01	0.0066277833483531\\
117.01	0.00662777813121369\\
118.01	0.00662777280163762\\
119.01	0.00662776735718675\\
120.01	0.00662776179536972\\
121.01	0.00662775611364071\\
122.01	0.00662775030939814\\
123.01	0.00662774437998358\\
124.01	0.00662773832268047\\
125.01	0.00662773213471277\\
126.01	0.00662772581324374\\
127.01	0.00662771935537439\\
128.01	0.00662771275814245\\
129.01	0.00662770601852059\\
130.01	0.00662769913341532\\
131.01	0.00662769209966521\\
132.01	0.00662768491403954\\
133.01	0.00662767757323686\\
134.01	0.0066276700738832\\
135.01	0.00662766241253068\\
136.01	0.00662765458565565\\
137.01	0.00662764658965726\\
138.01	0.00662763842085535\\
139.01	0.0066276300754892\\
140.01	0.00662762154971541\\
141.01	0.00662761283960593\\
142.01	0.00662760394114659\\
143.01	0.00662759485023485\\
144.01	0.00662758556267798\\
145.01	0.00662757607419107\\
146.01	0.00662756638039477\\
147.01	0.0066275564768134\\
148.01	0.00662754635887267\\
149.01	0.00662753602189763\\
150.01	0.00662752546111022\\
151.01	0.00662751467162706\\
152.01	0.00662750364845716\\
153.01	0.00662749238649931\\
154.01	0.00662748088053993\\
155.01	0.00662746912525013\\
156.01	0.00662745711518362\\
157.01	0.00662744484477359\\
158.01	0.00662743230833038\\
159.01	0.00662741950003849\\
160.01	0.0066274064139538\\
161.01	0.00662739304400081\\
162.01	0.00662737938396948\\
163.01	0.00662736542751227\\
164.01	0.00662735116814124\\
165.01	0.00662733659922458\\
166.01	0.0066273217139835\\
167.01	0.00662730650548892\\
168.01	0.00662729096665803\\
169.01	0.00662727509025096\\
170.01	0.00662725886886697\\
171.01	0.00662724229494104\\
172.01	0.00662722536074013\\
173.01	0.00662720805835907\\
174.01	0.00662719037971722\\
175.01	0.0066271723165538\\
176.01	0.00662715386042436\\
177.01	0.00662713500269641\\
178.01	0.00662711573454498\\
179.01	0.00662709604694843\\
180.01	0.00662707593068403\\
181.01	0.00662705537632315\\
182.01	0.00662703437422661\\
183.01	0.00662701291454014\\
184.01	0.00662699098718905\\
185.01	0.00662696858187351\\
186.01	0.00662694568806321\\
187.01	0.00662692229499199\\
188.01	0.00662689839165287\\
189.01	0.006626873966792\\
190.01	0.0066268490089033\\
191.01	0.00662682350622247\\
192.01	0.0066267974467212\\
193.01	0.00662677081810105\\
194.01	0.00662674360778704\\
195.01	0.00662671580292151\\
196.01	0.00662668739035743\\
197.01	0.00662665835665177\\
198.01	0.00662662868805853\\
199.01	0.00662659837052181\\
200.01	0.00662656738966869\\
201.01	0.0066265357308016\\
202.01	0.00662650337889101\\
203.01	0.00662647031856767\\
204.01	0.00662643653411447\\
205.01	0.00662640200945852\\
206.01	0.00662636672816279\\
207.01	0.00662633067341761\\
208.01	0.00662629382803171\\
209.01	0.0066262561744237\\
210.01	0.00662621769461239\\
211.01	0.00662617837020773\\
212.01	0.00662613818240104\\
213.01	0.00662609711195519\\
214.01	0.00662605513919448\\
215.01	0.00662601224399421\\
216.01	0.00662596840576998\\
217.01	0.00662592360346703\\
218.01	0.0066258778155487\\
219.01	0.00662583101998533\\
220.01	0.00662578319424236\\
221.01	0.00662573431526805\\
222.01	0.00662568435948147\\
223.01	0.00662563330275989\\
224.01	0.00662558112042532\\
225.01	0.0066255277872315\\
226.01	0.00662547327735037\\
227.01	0.00662541756435768\\
228.01	0.00662536062121904\\
229.01	0.0066253024202747\\
230.01	0.00662524293322482\\
231.01	0.00662518213111368\\
232.01	0.00662511998431401\\
233.01	0.00662505646251033\\
234.01	0.00662499153468248\\
235.01	0.00662492516908842\\
236.01	0.00662485733324617\\
237.01	0.00662478799391618\\
238.01	0.00662471711708254\\
239.01	0.00662464466793375\\
240.01	0.00662457061084337\\
241.01	0.00662449490934975\\
242.01	0.00662441752613552\\
243.01	0.00662433842300622\\
244.01	0.0066242575608687\\
245.01	0.00662417489970877\\
246.01	0.00662409039856812\\
247.01	0.00662400401552091\\
248.01	0.00662391570764941\\
249.01	0.00662382543101951\\
250.01	0.00662373314065458\\
251.01	0.00662363879051004\\
252.01	0.00662354233344579\\
253.01	0.00662344372119895\\
254.01	0.00662334290435512\\
255.01	0.00662323983231958\\
256.01	0.00662313445328695\\
257.01	0.00662302671421067\\
258.01	0.00662291656077115\\
259.01	0.00662280393734327\\
260.01	0.00662268878696307\\
261.01	0.00662257105129317\\
262.01	0.00662245067058778\\
263.01	0.00662232758365616\\
264.01	0.00662220172782539\\
265.01	0.00662207303890209\\
266.01	0.00662194145113291\\
267.01	0.00662180689716418\\
268.01	0.00662166930800013\\
269.01	0.00662152861296002\\
270.01	0.00662138473963445\\
271.01	0.00662123761383974\\
272.01	0.00662108715957155\\
273.01	0.00662093329895719\\
274.01	0.00662077595220619\\
275.01	0.00662061503755995\\
276.01	0.00662045047123949\\
277.01	0.00662028216739223\\
278.01	0.00662011003803687\\
279.01	0.0066199339930068\\
280.01	0.00661975393989208\\
281.01	0.00661956978397959\\
282.01	0.00661938142819155\\
283.01	0.00661918877302219\\
284.01	0.00661899171647292\\
285.01	0.00661879015398522\\
286.01	0.00661858397837213\\
287.01	0.00661837307974748\\
288.01	0.00661815734545293\\
289.01	0.00661793665998358\\
290.01	0.00661771090491081\\
291.01	0.0066174799588034\\
292.01	0.00661724369714626\\
293.01	0.00661700199225676\\
294.01	0.00661675471319886\\
295.01	0.00661650172569491\\
296.01	0.0066162428920347\\
297.01	0.00661597807098249\\
298.01	0.00661570711768067\\
299.01	0.00661542988355159\\
300.01	0.00661514621619602\\
301.01	0.00661485595928926\\
302.01	0.00661455895247416\\
303.01	0.00661425503125132\\
304.01	0.00661394402686626\\
305.01	0.00661362576619343\\
306.01	0.0066133000716175\\
307.01	0.00661296676091094\\
308.01	0.00661262564710884\\
309.01	0.00661227653837986\\
310.01	0.00661191923789445\\
311.01	0.00661155354368929\\
312.01	0.00661117924852798\\
313.01	0.0066107961397587\\
314.01	0.00661040399916789\\
315.01	0.00661000260283024\\
316.01	0.00660959172095541\\
317.01	0.0066091711177301\\
318.01	0.0066087405511571\\
319.01	0.00660829977289013\\
320.01	0.00660784852806491\\
321.01	0.00660738655512567\\
322.01	0.00660691358564859\\
323.01	0.00660642934416056\\
324.01	0.00660593354795405\\
325.01	0.00660542590689797\\
326.01	0.00660490612324442\\
327.01	0.00660437389143127\\
328.01	0.00660382889788074\\
329.01	0.00660327082079387\\
330.01	0.00660269932994098\\
331.01	0.00660211408644781\\
332.01	0.00660151474257809\\
333.01	0.00660090094151163\\
334.01	0.0066002723171192\\
335.01	0.00659962849373248\\
336.01	0.00659896908591152\\
337.01	0.00659829369820749\\
338.01	0.00659760192492218\\
339.01	0.00659689334986413\\
340.01	0.00659616754610136\\
341.01	0.00659542407571067\\
342.01	0.00659466248952397\\
343.01	0.00659388232687187\\
344.01	0.00659308311532434\\
345.01	0.00659226437042884\\
346.01	0.00659142559544619\\
347.01	0.00659056628108438\\
348.01	0.00658968590523029\\
349.01	0.00658878393268019\\
350.01	0.0065878598148684\\
351.01	0.00658691298959533\\
352.01	0.00658594288075464\\
353.01	0.00658494889805978\\
354.01	0.00658393043677062\\
355.01	0.00658288687742027\\
356.01	0.00658181758554252\\
357.01	0.00658072191140011\\
358.01	0.00657959918971456\\
359.01	0.00657844873939757\\
360.01	0.00657726986328474\\
361.01	0.00657606184787175\\
362.01	0.00657482396305372\\
363.01	0.00657355546186785\\
364.01	0.00657225558024\\
365.01	0.006570923536736\\
366.01	0.00656955853231751\\
367.01	0.00656815975010318\\
368.01	0.00656672635513631\\
369.01	0.00656525749415812\\
370.01	0.00656375229538862\\
371.01	0.00656220986831459\\
372.01	0.00656062930348597\\
373.01	0.00655900967232032\\
374.01	0.00655735002691722\\
375.01	0.00655564939988173\\
376.01	0.00655390680415868\\
377.01	0.00655212123287765\\
378.01	0.00655029165920917\\
379.01	0.00654841703623332\\
380.01	0.00654649629681985\\
381.01	0.00654452835352096\\
382.01	0.00654251209847646\\
383.01	0.00654044640333092\\
384.01	0.00653833011916143\\
385.01	0.00653616207641647\\
386.01	0.0065339410848613\\
387.01	0.00653166593352889\\
388.01	0.0065293353906704\\
389.01	0.0065269482037002\\
390.01	0.00652450309912636\\
391.01	0.00652199878245545\\
392.01	0.0065194339380574\\
393.01	0.00651680722897081\\
394.01	0.00651411729662454\\
395.01	0.00651136276044442\\
396.01	0.00650854221730586\\
397.01	0.00650565424078453\\
398.01	0.00650269738014472\\
399.01	0.00649967015899487\\
400.01	0.00649657107352375\\
401.01	0.00649339859021859\\
402.01	0.00649015114294966\\
403.01	0.00648682712929694\\
404.01	0.00648342490598466\\
405.01	0.00647994278329268\\
406.01	0.00647637901833192\\
407.01	0.00647273180711472\\
408.01	0.00646899927543674\\
409.01	0.00646517946873548\\
410.01	0.00646127034133285\\
411.01	0.00645726974585314\\
412.01	0.00645317542419395\\
413.01	0.00644898500231479\\
414.01	0.00644469599438852\\
415.01	0.00644030582556887\\
416.01	0.00643581184422117\\
417.01	0.00643121132031199\\
418.01	0.00642650144298454\\
419.01	0.00642167931811403\\
420.01	0.00641674196582409\\
421.01	0.00641168631758996\\
422.01	0.0064065092131186\\
423.01	0.00640120740019603\\
424.01	0.00639577753187652\\
425.01	0.00639021616335015\\
426.01	0.00638451974956823\\
427.01	0.00637868464293872\\
428.01	0.00637270709110753\\
429.01	0.00636658323485393\\
430.01	0.00636030910613026\\
431.01	0.00635388062628287\\
432.01	0.00634729360449606\\
433.01	0.00634054373650708\\
434.01	0.00633362660364876\\
435.01	0.00632653767228254\\
436.01	0.00631927229369635\\
437.01	0.00631182570455096\\
438.01	0.00630419302797106\\
439.01	0.00629636927538985\\
440.01	0.00628834934927172\\
441.01	0.00628012804685388\\
442.01	0.00627170006506416\\
443.01	0.00626306000679248\\
444.01	0.00625420238871374\\
445.01	0.00624512165087826\\
446.01	0.00623581216830759\\
447.01	0.00622626826485076\\
448.01	0.00621648422956873\\
449.01	0.00620645433592308\\
450.01	0.00619617286403988\\
451.01	0.00618563412629655\\
452.01	0.00617483249659878\\
453.01	0.00616376244593114\\
454.01	0.00615241856387239\\
455.01	0.00614079555763997\\
456.01	0.00612888826101543\\
457.01	0.00611669163481851\\
458.01	0.0061042007499225\\
459.01	0.00609141074431236\\
460.01	0.00607831673480827\\
461.01	0.00606491372901676\\
462.01	0.00605119794332509\\
463.01	0.00603716701194948\\
464.01	0.00602281903691963\\
465.01	0.00600815244782415\\
466.01	0.00599316579941203\\
467.01	0.00597785743865201\\
468.01	0.00596222498751151\\
469.01	0.00594626455940978\\
470.01	0.00592996931183786\\
471.01	0.00591332644919716\\
472.01	0.0058963183014063\\
473.01	0.00587891840890463\\
474.01	0.00586108519619017\\
475.01	0.00584275945284479\\
476.01	0.0058239000907944\\
477.01	0.00580447904511014\\
478.01	0.00578446622209283\\
479.01	0.00576382917878065\\
480.01	0.00574253294140469\\
481.01	0.00572053981903716\\
482.01	0.00569780921596359\\
483.01	0.00567429744785275\\
484.01	0.00564995756883459\\
485.01	0.00562473921933519\\
486.01	0.00559858850938103\\
487.01	0.0055714480080374\\
488.01	0.00554325732314974\\
489.01	0.00551395347978441\\
490.01	0.00548347094837184\\
491.01	0.00545174230923784\\
492.01	0.00541869929752596\\
493.01	0.00538427432238069\\
494.01	0.00534840261886492\\
495.01	0.0053110252402656\\
496.01	0.00527209316299654\\
497.01	0.00523157286128748\\
498.01	0.00518945381409247\\
499.01	0.00514575855936362\\
500.01	0.00510055609813132\\
501.01	0.00505398507554129\\
502.01	0.00500669829198533\\
503.01	0.00495908387073525\\
504.01	0.00491119851801213\\
505.01	0.00486310693772908\\
506.01	0.00481488274983092\\
507.01	0.00476660888573337\\
508.01	0.0047183779636502\\
509.01	0.00467029270123617\\
510.01	0.00462246611531557\\
511.01	0.00457502139463206\\
512.01	0.00452809129427338\\
513.01	0.00448181684757209\\
514.01	0.00443634512148415\\
515.01	0.00439182564908723\\
516.01	0.0043484050477132\\
517.01	0.00430621914513456\\
518.01	0.00426538166528717\\
519.01	0.00422596707374778\\
520.01	0.00418798502878103\\
521.01	0.00415117282828407\\
522.01	0.004114607957536\\
523.01	0.00407821214291625\\
524.01	0.00404202278944255\\
525.01	0.00400607380377639\\
526.01	0.00397039386513776\\
527.01	0.00393500465022193\\
528.01	0.00389991884824442\\
529.01	0.00386513800886344\\
530.01	0.0038306503075186\\
531.01	0.00379642837598923\\
532.01	0.00376242743600688\\
533.01	0.00372858410164934\\
534.01	0.00369481639498225\\
535.01	0.00366102575010299\\
536.01	0.00362710213479891\\
537.01	0.00359294119869851\\
538.01	0.00355850046695407\\
539.01	0.00352376222467012\\
540.01	0.00348870536509431\\
541.01	0.00345330520045406\\
542.01	0.00341753352031443\\
543.01	0.00338135877420527\\
544.01	0.00334474647734787\\
545.01	0.00330765993147755\\
546.01	0.00327006123083538\\
547.01	0.00323191254095152\\
548.01	0.00319317761547406\\
549.01	0.0031538234145313\\
550.01	0.00311382155102045\\
551.01	0.0030731486693184\\
552.01	0.0030317832424425\\
553.01	0.00298970328206819\\
554.01	0.00294688640917473\\
555.01	0.0029033100905392\\
556.01	0.00285895189237939\\
557.01	0.00281378973570623\\
558.01	0.00276780213238578\\
559.01	0.00272096837574328\\
560.01	0.00267326865627233\\
561.01	0.00262468407428047\\
562.01	0.00257519653052143\\
563.01	0.00252478849624054\\
564.01	0.00247344276509564\\
565.01	0.00242114241434334\\
566.01	0.00236787089249708\\
567.01	0.0023136121155352\\
568.01	0.00225835056070582\\
569.01	0.00220207135471073\\
570.01	0.00214476035493563\\
571.01	0.00208640422450415\\
572.01	0.00202699050448087\\
573.01	0.00196650768972543\\
574.01	0.00190494531817138\\
575.01	0.00184229408482995\\
576.01	0.00177854598418229\\
577.01	0.00171369446888605\\
578.01	0.0016477346162388\\
579.01	0.0015806633005132\\
580.01	0.00151247936952008\\
581.01	0.00144318382328261\\
582.01	0.00137277999184947\\
583.01	0.00130127370788135\\
584.01	0.0012286734675554\\
585.01	0.00115499057031396\\
586.01	0.00108023922386147\\
587.01	0.00100443659586242\\
588.01	0.000927602789133309\\
589.01	0.000849760712020215\\
590.01	0.000770935808457237\\
591.01	0.00069115560281989\\
592.01	0.000610449002857873\\
593.01	0.000528845289120537\\
594.01	0.000446372700577078\\
595.01	0.000363056502572266\\
596.01	0.000278916393581779\\
597.01	0.000193963069867737\\
598.01	0.000108193720153206\\
599.01	3.18230447902152e-05\\
599.02	3.12770961984968e-05\\
599.03	3.07343624924919e-05\\
599.04	3.0194875613938e-05\\
599.05	2.96586678183493e-05\\
599.06	2.91257716778671e-05\\
599.07	2.85962200841493e-05\\
599.08	2.80700462514365e-05\\
599.09	2.75472837197206e-05\\
599.1	2.70279663579388e-05\\
599.11	2.65121283671912e-05\\
599.12	2.59998042840028e-05\\
599.13	2.54910289836183e-05\\
599.14	2.49858376833341e-05\\
599.15	2.44842659458625e-05\\
599.16	2.39863496827274e-05\\
599.17	2.34921251576915e-05\\
599.18	2.30016289902261e-05\\
599.19	2.2514898159005e-05\\
599.2	2.20319700054462e-05\\
599.21	2.15528822372704e-05\\
599.22	2.10776729321089e-05\\
599.23	2.06063805411449e-05\\
599.24	2.01390438927945e-05\\
599.25	1.96757021964072e-05\\
599.26	1.92163950460209e-05\\
599.27	1.87611624241565e-05\\
599.28	1.83100447056338e-05\\
599.29	1.78630826614314e-05\\
599.3	1.74203174626037e-05\\
599.31	1.69817906841944e-05\\
599.32	1.65475443092387e-05\\
599.33	1.61176207327651e-05\\
599.34	1.56920627658545e-05\\
599.35	1.52709136397415e-05\\
599.36	1.48542170099464e-05\\
599.37	1.44420169604503e-05\\
599.38	1.40343580079232e-05\\
599.39	1.36312859347863e-05\\
599.4	1.32328499588292e-05\\
599.41	1.28390997857844e-05\\
599.42	1.24500856141119e-05\\
599.43	1.20658581398547e-05\\
599.44	1.16864685615215e-05\\
599.45	1.13119685850316e-05\\
599.46	1.09424104286961e-05\\
599.47	1.05778468282615e-05\\
599.48	1.02183310419918e-05\\
599.49	9.86391685580193e-06\\
599.5	9.5146585884498e-06\\
599.51	9.17061109676973e-06\\
599.52	8.83182978096524e-06\\
599.53	8.49837058993982e-06\\
599.54	8.17029002670236e-06\\
599.55	7.84764515380544e-06\\
599.56	7.53049359884274e-06\\
599.57	7.21889356001225e-06\\
599.58	6.91290381171078e-06\\
599.59	6.61258371020476e-06\\
599.6	6.31799319935483e-06\\
599.61	6.02919281637339e-06\\
599.62	5.74624369766713e-06\\
599.63	5.46920758472819e-06\\
599.64	5.1981468300686e-06\\
599.65	4.93312440323461e-06\\
599.66	4.67420389686606e-06\\
599.67	4.42144953282167e-06\\
599.68	4.1749261683599e-06\\
599.69	3.93469930238047e-06\\
599.7	3.70083508173524e-06\\
599.71	3.47340030758779e-06\\
599.72	3.25246244185264e-06\\
599.73	3.03808961367273e-06\\
599.74	2.83035062599232e-06\\
599.75	2.62931496216104e-06\\
599.76	2.43505279263513e-06\\
599.77	2.24763498171514e-06\\
599.78	2.06713309437029e-06\\
599.79	1.89361940312015e-06\\
599.8	1.72716689498566e-06\\
599.81	1.56784927851129e-06\\
599.82	1.41574099085662e-06\\
599.83	1.27091720493987e-06\\
599.84	1.13345383668736e-06\\
599.85	1.00342755231589e-06\\
599.86	8.8091577571392e-07\\
599.87	7.65996695880136e-07\\
599.88	6.58749274441706e-07\\
599.89	5.59253253243699e-07\\
599.9	4.67589162013102e-07\\
599.91	3.8383832609741e-07\\
599.92	3.0808287429171e-07\\
599.93	2.40405746710845e-07\\
599.94	1.8089070277609e-07\\
599.95	1.2962232925906e-07\\
599.96	8.66860484002169e-08\\
599.97	5.21681261331924e-08\\
599.98	2.61556803542173e-08\\
599.99	8.73668930083393e-09\\
600	0\\
};
\addplot [color=red!50!mycolor17,solid,forget plot]
  table[row sep=crcr]{%
0.01	0.00643330023985397\\
1.01	0.00643329966361264\\
2.01	0.00643329907515522\\
3.01	0.00643329847422214\\
4.01	0.00643329786054826\\
5.01	0.00643329723386302\\
6.01	0.00643329659388985\\
7.01	0.00643329594034644\\
8.01	0.00643329527294438\\
9.01	0.00643329459138911\\
10.01	0.00643329389537988\\
11.01	0.00643329318460946\\
12.01	0.0064332924587642\\
13.01	0.00643329171752365\\
14.01	0.00643329096056051\\
15.01	0.0064332901875405\\
16.01	0.00643328939812239\\
17.01	0.00643328859195762\\
18.01	0.00643328776869002\\
19.01	0.00643328692795602\\
20.01	0.00643328606938434\\
21.01	0.00643328519259551\\
22.01	0.00643328429720226\\
23.01	0.00643328338280896\\
24.01	0.00643328244901161\\
25.01	0.0064332814953974\\
26.01	0.0064332805215449\\
27.01	0.0064332795270238\\
28.01	0.00643327851139429\\
29.01	0.00643327747420746\\
30.01	0.00643327641500483\\
31.01	0.006433275333318\\
32.01	0.00643327422866862\\
33.01	0.00643327310056819\\
34.01	0.00643327194851781\\
35.01	0.00643327077200789\\
36.01	0.0064332695705179\\
37.01	0.0064332683435164\\
38.01	0.00643326709046047\\
39.01	0.00643326581079568\\
40.01	0.00643326450395578\\
41.01	0.00643326316936238\\
42.01	0.00643326180642484\\
43.01	0.00643326041453989\\
44.01	0.00643325899309134\\
45.01	0.00643325754144988\\
46.01	0.00643325605897279\\
47.01	0.00643325454500375\\
48.01	0.00643325299887223\\
49.01	0.00643325141989364\\
50.01	0.00643324980736839\\
51.01	0.00643324816058252\\
52.01	0.00643324647880629\\
53.01	0.00643324476129471\\
54.01	0.00643324300728684\\
55.01	0.00643324121600541\\
56.01	0.00643323938665682\\
57.01	0.00643323751843024\\
58.01	0.00643323561049773\\
59.01	0.00643323366201344\\
60.01	0.00643323167211386\\
61.01	0.00643322963991663\\
62.01	0.00643322756452081\\
63.01	0.00643322544500617\\
64.01	0.00643322328043285\\
65.01	0.00643322106984085\\
66.01	0.00643321881224981\\
67.01	0.0064332165066584\\
68.01	0.00643321415204387\\
69.01	0.00643321174736173\\
70.01	0.00643320929154495\\
71.01	0.00643320678350395\\
72.01	0.0064332042221259\\
73.01	0.00643320160627412\\
74.01	0.00643319893478762\\
75.01	0.00643319620648064\\
76.01	0.00643319342014218\\
77.01	0.00643319057453537\\
78.01	0.00643318766839686\\
79.01	0.00643318470043639\\
80.01	0.00643318166933613\\
81.01	0.00643317857375005\\
82.01	0.00643317541230344\\
83.01	0.00643317218359219\\
84.01	0.00643316888618216\\
85.01	0.00643316551860874\\
86.01	0.00643316207937583\\
87.01	0.00643315856695557\\
88.01	0.0064331549797873\\
89.01	0.00643315131627714\\
90.01	0.00643314757479708\\
91.01	0.0064331437536844\\
92.01	0.00643313985124065\\
93.01	0.00643313586573155\\
94.01	0.00643313179538536\\
95.01	0.00643312763839263\\
96.01	0.00643312339290536\\
97.01	0.0064331190570357\\
98.01	0.0064331146288558\\
99.01	0.00643311010639661\\
100.01	0.00643310548764689\\
101.01	0.00643310077055248\\
102.01	0.00643309595301526\\
103.01	0.00643309103289239\\
104.01	0.00643308600799515\\
105.01	0.00643308087608805\\
106.01	0.00643307563488783\\
107.01	0.00643307028206251\\
108.01	0.0064330648152302\\
109.01	0.00643305923195813\\
110.01	0.00643305352976162\\
111.01	0.00643304770610254\\
112.01	0.00643304175838908\\
113.01	0.00643303568397349\\
114.01	0.00643302948015176\\
115.01	0.00643302314416188\\
116.01	0.00643301667318305\\
117.01	0.0064330100643339\\
118.01	0.0064330033146717\\
119.01	0.00643299642119072\\
120.01	0.00643298938082084\\
121.01	0.00643298219042644\\
122.01	0.00643297484680501\\
123.01	0.00643296734668524\\
124.01	0.00643295968672614\\
125.01	0.00643295186351517\\
126.01	0.00643294387356687\\
127.01	0.00643293571332134\\
128.01	0.00643292737914249\\
129.01	0.00643291886731646\\
130.01	0.00643291017404996\\
131.01	0.00643290129546863\\
132.01	0.00643289222761531\\
133.01	0.00643288296644819\\
134.01	0.0064328735078389\\
135.01	0.00643286384757091\\
136.01	0.00643285398133756\\
137.01	0.00643284390474001\\
138.01	0.00643283361328532\\
139.01	0.00643282310238452\\
140.01	0.00643281236735039\\
141.01	0.00643280140339574\\
142.01	0.00643279020563064\\
143.01	0.00643277876906069\\
144.01	0.0064327670885847\\
145.01	0.00643275515899224\\
146.01	0.0064327429749616\\
147.01	0.0064327305310571\\
148.01	0.0064327178217268\\
149.01	0.00643270484129986\\
150.01	0.0064326915839843\\
151.01	0.00643267804386418\\
152.01	0.00643266421489673\\
153.01	0.00643265009091029\\
154.01	0.00643263566560078\\
155.01	0.00643262093252944\\
156.01	0.0064326058851197\\
157.01	0.00643259051665394\\
158.01	0.0064325748202712\\
159.01	0.00643255878896333\\
160.01	0.00643254241557224\\
161.01	0.00643252569278664\\
162.01	0.00643250861313882\\
163.01	0.00643249116900114\\
164.01	0.00643247335258249\\
165.01	0.00643245515592519\\
166.01	0.00643243657090099\\
167.01	0.00643241758920761\\
168.01	0.00643239820236508\\
169.01	0.00643237840171157\\
170.01	0.00643235817839993\\
171.01	0.00643233752339339\\
172.01	0.00643231642746167\\
173.01	0.00643229488117671\\
174.01	0.0064322728749082\\
175.01	0.00643225039881996\\
176.01	0.00643222744286454\\
177.01	0.00643220399677939\\
178.01	0.00643218005008191\\
179.01	0.00643215559206491\\
180.01	0.00643213061179134\\
181.01	0.00643210509808991\\
182.01	0.00643207903954972\\
183.01	0.00643205242451492\\
184.01	0.00643202524108001\\
185.01	0.00643199747708377\\
186.01	0.00643196912010412\\
187.01	0.0064319401574526\\
188.01	0.00643191057616808\\
189.01	0.00643188036301151\\
190.01	0.00643184950445939\\
191.01	0.00643181798669802\\
192.01	0.00643178579561694\\
193.01	0.00643175291680245\\
194.01	0.00643171933553146\\
195.01	0.00643168503676419\\
196.01	0.00643165000513781\\
197.01	0.00643161422495921\\
198.01	0.00643157768019776\\
199.01	0.00643154035447815\\
200.01	0.00643150223107265\\
201.01	0.00643146329289373\\
202.01	0.00643142352248596\\
203.01	0.00643138290201833\\
204.01	0.00643134141327574\\
205.01	0.00643129903765055\\
206.01	0.00643125575613437\\
207.01	0.00643121154930904\\
208.01	0.00643116639733792\\
209.01	0.00643112027995636\\
210.01	0.00643107317646256\\
211.01	0.006431025065708\\
212.01	0.00643097592608755\\
213.01	0.00643092573552946\\
214.01	0.0064308744714851\\
215.01	0.00643082211091856\\
216.01	0.00643076863029572\\
217.01	0.00643071400557331\\
218.01	0.00643065821218798\\
219.01	0.00643060122504442\\
220.01	0.00643054301850361\\
221.01	0.00643048356637099\\
222.01	0.0064304228418841\\
223.01	0.00643036081769961\\
224.01	0.0064302974658809\\
225.01	0.00643023275788465\\
226.01	0.00643016666454723\\
227.01	0.00643009915607106\\
228.01	0.00643003020200987\\
229.01	0.00642995977125506\\
230.01	0.00642988783202002\\
231.01	0.00642981435182537\\
232.01	0.00642973929748326\\
233.01	0.00642966263508117\\
234.01	0.00642958432996606\\
235.01	0.00642950434672698\\
236.01	0.00642942264917838\\
237.01	0.00642933920034242\\
238.01	0.00642925396243062\\
239.01	0.00642916689682585\\
240.01	0.00642907796406301\\
241.01	0.00642898712380994\\
242.01	0.00642889433484729\\
243.01	0.0064287995550484\\
244.01	0.00642870274135785\\
245.01	0.00642860384977059\\
246.01	0.0064285028353096\\
247.01	0.00642839965200366\\
248.01	0.00642829425286382\\
249.01	0.00642818658985994\\
250.01	0.00642807661389634\\
251.01	0.00642796427478643\\
252.01	0.00642784952122762\\
253.01	0.00642773230077468\\
254.01	0.00642761255981288\\
255.01	0.0064274902435299\\
256.01	0.00642736529588781\\
257.01	0.00642723765959359\\
258.01	0.00642710727606899\\
259.01	0.00642697408541999\\
260.01	0.00642683802640458\\
261.01	0.0064266990364007\\
262.01	0.0064265570513725\\
263.01	0.00642641200583604\\
264.01	0.0064262638328237\\
265.01	0.0064261124638481\\
266.01	0.00642595782886436\\
267.01	0.00642579985623197\\
268.01	0.00642563847267495\\
269.01	0.00642547360324142\\
270.01	0.00642530517126144\\
271.01	0.00642513309830385\\
272.01	0.0064249573041322\\
273.01	0.00642477770665875\\
274.01	0.00642459422189782\\
275.01	0.0064244067639166\\
276.01	0.00642421524478602\\
277.01	0.0064240195745288\\
278.01	0.00642381966106671\\
279.01	0.00642361541016589\\
280.01	0.00642340672538038\\
281.01	0.00642319350799416\\
282.01	0.00642297565696161\\
283.01	0.0064227530688452\\
284.01	0.00642252563775196\\
285.01	0.00642229325526805\\
286.01	0.00642205581039044\\
287.01	0.00642181318945708\\
288.01	0.0064215652760747\\
289.01	0.00642131195104404\\
290.01	0.00642105309228254\\
291.01	0.00642078857474497\\
292.01	0.00642051827034081\\
293.01	0.00642024204784927\\
294.01	0.00641995977283155\\
295.01	0.00641967130753948\\
296.01	0.00641937651082184\\
297.01	0.00641907523802669\\
298.01	0.00641876734090096\\
299.01	0.00641845266748652\\
300.01	0.00641813106201194\\
301.01	0.00641780236478143\\
302.01	0.00641746641205873\\
303.01	0.00641712303594808\\
304.01	0.00641677206427006\\
305.01	0.00641641332043338\\
306.01	0.00641604662330142\\
307.01	0.00641567178705515\\
308.01	0.00641528862104976\\
309.01	0.00641489692966672\\
310.01	0.00641449651216027\\
311.01	0.00641408716249786\\
312.01	0.00641366866919556\\
313.01	0.00641324081514603\\
314.01	0.00641280337744097\\
315.01	0.00641235612718674\\
316.01	0.00641189882931247\\
317.01	0.00641143124237128\\
318.01	0.00641095311833422\\
319.01	0.00641046420237525\\
320.01	0.00640996423264893\\
321.01	0.0064094529400591\\
322.01	0.00640893004801868\\
323.01	0.00640839527219962\\
324.01	0.00640784832027409\\
325.01	0.00640728889164443\\
326.01	0.00640671667716335\\
327.01	0.00640613135884255\\
328.01	0.00640553260954994\\
329.01	0.00640492009269518\\
330.01	0.00640429346190234\\
331.01	0.00640365236066948\\
332.01	0.00640299642201571\\
333.01	0.00640232526811258\\
334.01	0.00640163850990212\\
335.01	0.00640093574669889\\
336.01	0.00640021656577682\\
337.01	0.0063994805419384\\
338.01	0.00639872723706807\\
339.01	0.00639795619966709\\
340.01	0.00639716696436969\\
341.01	0.00639635905144049\\
342.01	0.00639553196625191\\
343.01	0.00639468519874068\\
344.01	0.00639381822284263\\
345.01	0.00639293049590642\\
346.01	0.00639202145808291\\
347.01	0.00639109053169153\\
348.01	0.00639013712056225\\
349.01	0.00638916060935153\\
350.01	0.00638816036283289\\
351.01	0.0063871357251601\\
352.01	0.00638608601910334\\
353.01	0.00638501054525607\\
354.01	0.00638390858121396\\
355.01	0.0063827793807233\\
356.01	0.00638162217279975\\
357.01	0.00638043616081613\\
358.01	0.00637922052155926\\
359.01	0.00637797440425547\\
360.01	0.00637669692956446\\
361.01	0.00637538718854211\\
362.01	0.00637404424157169\\
363.01	0.00637266711726575\\
364.01	0.00637125481133701\\
365.01	0.00636980628544193\\
366.01	0.00636832046599739\\
367.01	0.00636679624297341\\
368.01	0.00636523246866418\\
369.01	0.00636362795644195\\
370.01	0.00636198147949785\\
371.01	0.00636029176957551\\
372.01	0.00635855751570492\\
373.01	0.00635677736294391\\
374.01	0.00635494991113921\\
375.01	0.00635307371371823\\
376.01	0.00635114727652669\\
377.01	0.00634916905673055\\
378.01	0.00634713746180239\\
379.01	0.0063450508486184\\
380.01	0.0063429075226946\\
381.01	0.00634070573759887\\
382.01	0.00633844369457892\\
383.01	0.00633611954245547\\
384.01	0.00633373137783847\\
385.01	0.00633127724573051\\
386.01	0.00632875514059788\\
387.01	0.00632616300799689\\
388.01	0.00632349874686038\\
389.01	0.00632076021256291\\
390.01	0.00631794522090061\\
391.01	0.0063150515531395\\
392.01	0.00631207696230492\\
393.01	0.00630901918090426\\
394.01	0.00630587593029035\\
395.01	0.00630264493189209\\
396.01	0.00629932392054452\\
397.01	0.00629591066015217\\
398.01	0.00629240296190524\\
399.01	0.0062887987052299\\
400.01	0.00628509586158552\\
401.01	0.0062812925210984\\
402.01	0.0062773869218394\\
403.01	0.00627337748125873\\
404.01	0.0062692628288728\\
405.01	0.00626504183867491\\
406.01	0.00626071365886577\\
407.01	0.00625627773524493\\
408.01	0.00625173382286064\\
409.01	0.0062470819780779\\
410.01	0.00624232251986272\\
411.01	0.00623745594444802\\
412.01	0.00623248277120422\\
413.01	0.00622740328765818\\
414.01	0.00622221673713735\\
415.01	0.00621692070537143\\
416.01	0.00621151246139671\\
417.01	0.00620598916715711\\
418.01	0.0062003478704892\\
419.01	0.00619458549751251\\
420.01	0.0061886988443706\\
421.01	0.00618268456826306\\
422.01	0.00617653917770893\\
423.01	0.00617025902193658\\
424.01	0.0061638402793036\\
425.01	0.00615727894469201\\
426.01	0.00615057081576944\\
427.01	0.00614371147799952\\
428.01	0.00613669628828092\\
429.01	0.00612952035707672\\
430.01	0.00612217852888811\\
431.01	0.00611466536090582\\
432.01	0.00610697509966166\\
433.01	0.00609910165548131\\
434.01	0.00609103857452474\\
435.01	0.00608277900817753\\
436.01	0.00607431567953827\\
437.01	0.00606564084672496\\
438.01	0.00605674626270089\\
439.01	0.00604762313130064\\
440.01	0.00603826205911502\\
441.01	0.00602865300287525\\
442.01	0.00601878521196404\\
443.01	0.00600864716567108\\
444.01	0.00599822650481147\\
445.01	0.00598750995734001\\
446.01	0.00597648325762419\\
447.01	0.00596513105909743\\
448.01	0.00595343684010651\\
449.01	0.00594138280290273\\
450.01	0.005928949765932\\
451.01	0.00591611704986275\\
452.01	0.0059028623581921\\
453.01	0.00588916165380416\\
454.01	0.00587498903373414\\
455.01	0.00586031660584085\\
456.01	0.00584511437208053\\
457.01	0.00582935012488772\\
458.01	0.00581298936587319\\
459.01	0.00579599525945634\\
460.01	0.00577832863859256\\
461.01	0.00575994808581527\\
462.01	0.00574081011055646\\
463.01	0.00572086944900231\\
464.01	0.00570007956420959\\
465.01	0.00567839341078409\\
466.01	0.00565576454955953\\
467.01	0.00563214873078398\\
468.01	0.00560750610165938\\
469.01	0.0055818042429304\\
470.01	0.00555502230411261\\
471.01	0.0055271565999938\\
472.01	0.00549822810063586\\
473.01	0.00546829475838497\\
474.01	0.00543770048875961\\
475.01	0.0054066488083951\\
476.01	0.00537514743510029\\
477.01	0.00534320637975664\\
478.01	0.00531083814295213\\
479.01	0.00527805803414741\\
480.01	0.00524488452308555\\
481.01	0.00521133962399908\\
482.01	0.00517744931233549\\
483.01	0.00514324397220679\\
484.01	0.00510875887058511\\
485.01	0.00507403465119368\\
486.01	0.00503911783676843\\
487.01	0.0050040613225054\\
488.01	0.00496892483227736\\
489.01	0.00493377529014779\\
490.01	0.00489868707071135\\
491.01	0.00486374205840157\\
492.01	0.00482902941053039\\
493.01	0.00479464488819632\\
494.01	0.00476068957112154\\
495.01	0.00472726770782663\\
496.01	0.00469448336651602\\
497.01	0.00466243544789561\\
498.01	0.00463121047938355\\
499.01	0.00460087238312644\\
500.01	0.00457144816890996\\
501.01	0.00454290277306364\\
502.01	0.00451468377798736\\
503.01	0.00448648666934954\\
504.01	0.00445834215488247\\
505.01	0.00443028109400982\\
506.01	0.00440233396611083\\
507.01	0.00437453021896771\\
508.01	0.00434689749021075\\
509.01	0.00431946068556158\\
510.01	0.00429224090559129\\
511.01	0.00426525422123244\\
512.01	0.00423851031014622\\
513.01	0.00421201098427128\\
514.01	0.0041857486660573\\
515.01	0.00415970491050224\\
516.01	0.00413384912698003\\
517.01	0.00410813773590309\\
518.01	0.00408251411026784\\
519.01	0.00405690981719585\\
520.01	0.00403124793559125\\
521.01	0.00400545145824772\\
522.01	0.00397948220161627\\
523.01	0.00395333152360722\\
524.01	0.00392698916657118\\
525.01	0.00390044210047198\\
526.01	0.00387367439700243\\
527.01	0.00384666718283707\\
528.01	0.00381939869609618\\
529.01	0.00379184447350555\\
530.01	0.00376397769607122\\
531.01	0.003735769718089\\
532.01	0.00370719079578369\\
533.01	0.00367821101462431\\
534.01	0.0036488013837574\\
535.01	0.00361893501550884\\
536.01	0.00358858822822191\\
537.01	0.00355774121106488\\
538.01	0.00352637650554267\\
539.01	0.00349447638476056\\
540.01	0.00346202253866048\\
541.01	0.00342899620714643\\
542.01	0.00339537833345592\\
543.01	0.00336114973292802\\
544.01	0.00332629126957986\\
545.01	0.0032907840270283\\
546.01	0.00325460945664299\\
547.01	0.00321774948331401\\
548.01	0.00318018654712834\\
549.01	0.00314190356106398\\
550.01	0.00310288377422265\\
551.01	0.00306311055643913\\
552.01	0.00302256723038261\\
553.01	0.00298123706917909\\
554.01	0.00293910334223729\\
555.01	0.00289614935795801\\
556.01	0.00285235850019123\\
557.01	0.00280771425681943\\
558.01	0.00276220023958686\\
559.01	0.00271580019548417\\
560.01	0.00266849801167274\\
561.01	0.00262027771800452\\
562.01	0.00257112349337822\\
563.01	0.00252101968387742\\
564.01	0.00246995083963625\\
565.01	0.00241790176743256\\
566.01	0.00236485758999453\\
567.01	0.00231080380985966\\
568.01	0.00225572637833465\\
569.01	0.00219961177044417\\
570.01	0.0021424470669859\\
571.01	0.00208422004494936\\
572.01	0.00202491927756344\\
573.01	0.00196453424505074\\
574.01	0.00190305545671095\\
575.01	0.00184047458421142\\
576.01	0.00177678460515771\\
577.01	0.00171197995585397\\
578.01	0.00164605669237123\\
579.01	0.0015790126587281\\
580.01	0.00151084766032477\\
581.01	0.00144156363987375\\
582.01	0.0013711648518907\\
583.01	0.00129965803028412\\
584.01	0.00122705254163708\\
585.01	0.0011533605143295\\
586.01	0.00107859693060129\\
587.01	0.00100277966487854\\
588.01	0.000925929446941498\\
589.01	0.000848069722467409\\
590.01	0.00076922637580672\\
591.01	0.000689427270172824\\
592.01	0.000608701548246961\\
593.01	0.000527078620894876\\
594.01	0.000444586752481062\\
595.01	0.000361251127183708\\
596.01	0.000277091250558454\\
597.01	0.000192117502885737\\
598.01	0.00010632661372072\\
599.01	3.1823044268001e-05\\
599.02	3.12770957282913e-05\\
599.03	3.07343620667162e-05\\
599.04	3.01948752267685e-05\\
599.05	2.96586674650972e-05\\
599.06	2.91257713545909e-05\\
599.07	2.85962197874873e-05\\
599.08	2.80700459785194e-05\\
599.09	2.75472834680903e-05\\
599.1	2.70279661254754e-05\\
599.11	2.65121281520647e-05\\
599.12	2.59998040846362e-05\\
599.13	2.54910287986413e-05\\
599.14	2.49858375115618e-05\\
599.15	2.44842657862541e-05\\
599.16	2.39863495343721e-05\\
599.17	2.34921250197827e-05\\
599.18	2.30016288620422e-05\\
599.19	2.25148980398954e-05\\
599.2	2.2031969894816e-05\\
599.21	2.15528821345731e-05\\
599.22	2.10776728368396e-05\\
599.23	2.06063804528353e-05\\
599.24	2.01390438110006e-05\\
599.25	1.96757021207125e-05\\
599.26	1.92163949760387e-05\\
599.27	1.8761162359519e-05\\
599.28	1.8310044645994e-05\\
599.29	1.78630826064737e-05\\
599.3	1.7420317412014e-05\\
599.31	1.69817906376935e-05\\
599.32	1.65475442665524e-05\\
599.33	1.61176206936401e-05\\
599.34	1.5692062730048e-05\\
599.35	1.52709136070281e-05\\
599.36	1.48542169801109e-05\\
599.37	1.44420169332915e-05\\
599.38	1.40343579832485e-05\\
599.39	1.36312859124171e-05\\
599.4	1.32328499385954e-05\\
599.41	1.2839099767523e-05\\
599.42	1.24500855976737e-05\\
599.43	1.20658581250939e-05\\
599.44	1.16864685483029e-05\\
599.45	1.13119685732268e-05\\
599.46	1.09424104181854e-05\\
599.47	1.05778468189321e-05\\
599.48	1.02183310337345e-05\\
599.49	9.86391684851956e-06\\
599.5	9.51465858204867e-06\\
599.51	9.17061109116483e-06\\
599.52	8.83182977607332e-06\\
599.53	8.49837058568975e-06\\
599.54	8.17029002302475e-06\\
599.55	7.84764515063437e-06\\
599.56	7.53049359612443e-06\\
599.57	7.21889355768945e-06\\
599.58	6.91290380973666e-06\\
599.59	6.61258370854116e-06\\
599.6	6.31799319795665e-06\\
599.61	6.02919281520418e-06\\
599.62	5.74624369669915e-06\\
599.63	5.46920758392848e-06\\
599.64	5.19814682941461e-06\\
599.65	4.93312440270205e-06\\
599.66	4.67420389643758e-06\\
599.67	4.42144953247993e-06\\
599.68	4.17492616808929e-06\\
599.69	3.93469930216883e-06\\
599.7	3.70083508157044e-06\\
599.71	3.47340030746289e-06\\
599.72	3.25246244175723e-06\\
599.73	3.03808961360334e-06\\
599.74	2.83035062593855e-06\\
599.75	2.62931496212288e-06\\
599.76	2.43505279260738e-06\\
599.77	2.24763498169432e-06\\
599.78	2.06713309435641e-06\\
599.79	1.89361940310974e-06\\
599.8	1.72716689497872e-06\\
599.81	1.56784927850956e-06\\
599.82	1.41574099085315e-06\\
599.83	1.27091720493813e-06\\
599.84	1.13345383668563e-06\\
599.85	1.00342755231589e-06\\
599.86	8.8091577571392e-07\\
599.87	7.65996695880136e-07\\
599.88	6.58749274441706e-07\\
599.89	5.59253253243699e-07\\
599.9	4.67589162011367e-07\\
599.91	3.83838326099145e-07\\
599.92	3.08082874293444e-07\\
599.93	2.40405746710845e-07\\
599.94	1.80890702777825e-07\\
599.95	1.2962232925906e-07\\
599.96	8.66860484019516e-08\\
599.97	5.21681261349272e-08\\
599.98	2.61556803542173e-08\\
599.99	8.73668930083393e-09\\
600	0\\
};
\addplot [color=red!40!mycolor19,solid,forget plot]
  table[row sep=crcr]{%
0.01	0.0062874344136244\\
1.01	0.00628743349800273\\
2.01	0.00628743256299474\\
3.01	0.00628743160818932\\
4.01	0.00628743063316646\\
5.01	0.00628742963749692\\
6.01	0.00628742862074276\\
7.01	0.0062874275824567\\
8.01	0.00628742652218161\\
9.01	0.00628742543945105\\
10.01	0.00628742433378832\\
11.01	0.0062874232047069\\
12.01	0.00628742205170972\\
13.01	0.00628742087428894\\
14.01	0.00628741967192625\\
15.01	0.00628741844409226\\
16.01	0.00628741719024629\\
17.01	0.00628741590983598\\
18.01	0.00628741460229756\\
19.01	0.00628741326705499\\
20.01	0.00628741190351998\\
21.01	0.00628741051109212\\
22.01	0.00628740908915761\\
23.01	0.00628740763709008\\
24.01	0.00628740615424945\\
25.01	0.00628740463998247\\
26.01	0.0062874030936216\\
27.01	0.00628740151448521\\
28.01	0.00628739990187734\\
29.01	0.00628739825508692\\
30.01	0.0062873965733878\\
31.01	0.00628739485603868\\
32.01	0.00628739310228218\\
33.01	0.00628739131134478\\
34.01	0.00628738948243671\\
35.01	0.00628738761475131\\
36.01	0.0062873857074647\\
37.01	0.00628738375973543\\
38.01	0.00628738177070424\\
39.01	0.00628737973949339\\
40.01	0.00628737766520673\\
41.01	0.00628737554692879\\
42.01	0.00628737338372471\\
43.01	0.0062873711746397\\
44.01	0.00628736891869874\\
45.01	0.00628736661490585\\
46.01	0.00628736426224395\\
47.01	0.00628736185967426\\
48.01	0.00628735940613593\\
49.01	0.0062873569005455\\
50.01	0.00628735434179671\\
51.01	0.006287351728759\\
52.01	0.00628734906027848\\
53.01	0.00628734633517628\\
54.01	0.00628734355224853\\
55.01	0.0062873407102658\\
56.01	0.00628733780797224\\
57.01	0.00628733484408533\\
58.01	0.00628733181729537\\
59.01	0.00628732872626476\\
60.01	0.00628732556962706\\
61.01	0.0062873223459872\\
62.01	0.00628731905392003\\
63.01	0.00628731569197016\\
64.01	0.00628731225865114\\
65.01	0.00628730875244501\\
66.01	0.00628730517180127\\
67.01	0.0062873015151364\\
68.01	0.00628729778083321\\
69.01	0.00628729396723982\\
70.01	0.00628729007266971\\
71.01	0.00628728609539976\\
72.01	0.00628728203367046\\
73.01	0.00628727788568475\\
74.01	0.00628727364960721\\
75.01	0.00628726932356354\\
76.01	0.00628726490563911\\
77.01	0.00628726039387883\\
78.01	0.00628725578628547\\
79.01	0.00628725108081969\\
80.01	0.00628724627539847\\
81.01	0.00628724136789457\\
82.01	0.00628723635613503\\
83.01	0.00628723123790091\\
84.01	0.00628722601092583\\
85.01	0.00628722067289502\\
86.01	0.00628721522144452\\
87.01	0.00628720965415978\\
88.01	0.0062872039685749\\
89.01	0.00628719816217146\\
90.01	0.00628719223237714\\
91.01	0.00628718617656494\\
92.01	0.00628717999205191\\
93.01	0.00628717367609755\\
94.01	0.00628716722590323\\
95.01	0.00628716063861062\\
96.01	0.00628715391130024\\
97.01	0.0062871470409906\\
98.01	0.00628714002463641\\
99.01	0.00628713285912764\\
100.01	0.0062871255412879\\
101.01	0.00628711806787282\\
102.01	0.00628711043556921\\
103.01	0.00628710264099285\\
104.01	0.00628709468068761\\
105.01	0.00628708655112356\\
106.01	0.0062870782486955\\
107.01	0.00628706976972144\\
108.01	0.00628706111044066\\
109.01	0.00628705226701221\\
110.01	0.00628704323551351\\
111.01	0.00628703401193834\\
112.01	0.00628702459219453\\
113.01	0.00628701497210318\\
114.01	0.00628700514739588\\
115.01	0.00628699511371355\\
116.01	0.00628698486660368\\
117.01	0.00628697440151922\\
118.01	0.00628696371381592\\
119.01	0.00628695279875048\\
120.01	0.00628694165147855\\
121.01	0.00628693026705234\\
122.01	0.00628691864041858\\
123.01	0.00628690676641635\\
124.01	0.00628689463977442\\
125.01	0.00628688225510939\\
126.01	0.00628686960692276\\
127.01	0.00628685668959899\\
128.01	0.00628684349740249\\
129.01	0.00628683002447583\\
130.01	0.00628681626483625\\
131.01	0.00628680221237351\\
132.01	0.00628678786084703\\
133.01	0.00628677320388323\\
134.01	0.00628675823497253\\
135.01	0.00628674294746665\\
136.01	0.00628672733457538\\
137.01	0.00628671138936365\\
138.01	0.00628669510474871\\
139.01	0.00628667847349666\\
140.01	0.00628666148821946\\
141.01	0.00628664414137117\\
142.01	0.00628662642524527\\
143.01	0.00628660833197088\\
144.01	0.0062865898535092\\
145.01	0.00628657098165012\\
146.01	0.00628655170800836\\
147.01	0.00628653202401986\\
148.01	0.00628651192093804\\
149.01	0.00628649138982988\\
150.01	0.00628647042157179\\
151.01	0.00628644900684564\\
152.01	0.0062864271361349\\
153.01	0.00628640479971971\\
154.01	0.00628638198767357\\
155.01	0.00628635868985792\\
156.01	0.00628633489591809\\
157.01	0.00628631059527905\\
158.01	0.00628628577713991\\
159.01	0.00628626043047003\\
160.01	0.00628623454400331\\
161.01	0.00628620810623378\\
162.01	0.00628618110540997\\
163.01	0.00628615352953008\\
164.01	0.00628612536633663\\
165.01	0.00628609660331089\\
166.01	0.00628606722766725\\
167.01	0.00628603722634769\\
168.01	0.00628600658601601\\
169.01	0.00628597529305184\\
170.01	0.00628594333354455\\
171.01	0.00628591069328737\\
172.01	0.0062858773577703\\
173.01	0.00628584331217474\\
174.01	0.00628580854136631\\
175.01	0.00628577302988811\\
176.01	0.00628573676195391\\
177.01	0.00628569972144138\\
178.01	0.00628566189188479\\
179.01	0.00628562325646768\\
180.01	0.00628558379801538\\
181.01	0.00628554349898735\\
182.01	0.00628550234146987\\
183.01	0.00628546030716749\\
184.01	0.006285417377395\\
185.01	0.00628537353306957\\
186.01	0.00628532875470208\\
187.01	0.00628528302238823\\
188.01	0.00628523631580016\\
189.01	0.00628518861417705\\
190.01	0.00628513989631629\\
191.01	0.00628509014056373\\
192.01	0.00628503932480434\\
193.01	0.00628498742645229\\
194.01	0.00628493442244111\\
195.01	0.0062848802892134\\
196.01	0.00628482500271029\\
197.01	0.00628476853836071\\
198.01	0.00628471087107083\\
199.01	0.00628465197521284\\
200.01	0.00628459182461342\\
201.01	0.0062845303925423\\
202.01	0.00628446765170066\\
203.01	0.00628440357420835\\
204.01	0.00628433813159232\\
205.01	0.00628427129477378\\
206.01	0.00628420303405488\\
207.01	0.00628413331910632\\
208.01	0.00628406211895315\\
209.01	0.0062839894019618\\
210.01	0.00628391513582533\\
211.01	0.00628383928754987\\
212.01	0.00628376182343942\\
213.01	0.00628368270908119\\
214.01	0.0062836019093302\\
215.01	0.00628351938829409\\
216.01	0.00628343510931667\\
217.01	0.00628334903496229\\
218.01	0.0062832611269986\\
219.01	0.00628317134637997\\
220.01	0.00628307965323034\\
221.01	0.0062829860068251\\
222.01	0.00628289036557343\\
223.01	0.00628279268699974\\
224.01	0.00628269292772447\\
225.01	0.0062825910434453\\
226.01	0.00628248698891729\\
227.01	0.0062823807179326\\
228.01	0.00628227218330053\\
229.01	0.00628216133682583\\
230.01	0.00628204812928764\\
231.01	0.00628193251041765\\
232.01	0.00628181442887756\\
233.01	0.00628169383223631\\
234.01	0.00628157066694637\\
235.01	0.00628144487832066\\
236.01	0.0062813164105073\\
237.01	0.00628118520646496\\
238.01	0.00628105120793778\\
239.01	0.00628091435542867\\
240.01	0.00628077458817327\\
241.01	0.00628063184411234\\
242.01	0.00628048605986412\\
243.01	0.00628033717069563\\
244.01	0.00628018511049424\\
245.01	0.0062800298117374\\
246.01	0.00627987120546236\\
247.01	0.00627970922123501\\
248.01	0.00627954378711856\\
249.01	0.00627937482964052\\
250.01	0.00627920227375994\\
251.01	0.00627902604283324\\
252.01	0.00627884605857934\\
253.01	0.00627866224104423\\
254.01	0.00627847450856465\\
255.01	0.00627828277773102\\
256.01	0.00627808696334902\\
257.01	0.00627788697840107\\
258.01	0.00627768273400627\\
259.01	0.0062774741393794\\
260.01	0.00627726110178995\\
261.01	0.00627704352651842\\
262.01	0.00627682131681328\\
263.01	0.00627659437384595\\
264.01	0.00627636259666503\\
265.01	0.00627612588214916\\
266.01	0.00627588412495935\\
267.01	0.00627563721748919\\
268.01	0.00627538504981479\\
269.01	0.00627512750964293\\
270.01	0.006274864482258\\
271.01	0.00627459585046796\\
272.01	0.00627432149454862\\
273.01	0.00627404129218614\\
274.01	0.00627375511841894\\
275.01	0.0062734628455778\\
276.01	0.00627316434322387\\
277.01	0.00627285947808595\\
278.01	0.00627254811399527\\
279.01	0.00627223011181933\\
280.01	0.0062719053293935\\
281.01	0.00627157362145101\\
282.01	0.00627123483955036\\
283.01	0.00627088883200205\\
284.01	0.00627053544379253\\
285.01	0.00627017451650581\\
286.01	0.00626980588824328\\
287.01	0.0062694293935412\\
288.01	0.00626904486328561\\
289.01	0.00626865212462511\\
290.01	0.00626825100088108\\
291.01	0.00626784131145485\\
292.01	0.00626742287173261\\
293.01	0.0062669954929873\\
294.01	0.0062665589822768\\
295.01	0.00626611314234064\\
296.01	0.00626565777149172\\
297.01	0.00626519266350579\\
298.01	0.00626471760750708\\
299.01	0.00626423238784969\\
300.01	0.00626373678399616\\
301.01	0.00626323057039097\\
302.01	0.0062627135163308\\
303.01	0.00626218538582935\\
304.01	0.00626164593747834\\
305.01	0.00626109492430317\\
306.01	0.00626053209361404\\
307.01	0.00625995718685075\\
308.01	0.00625936993942274\\
309.01	0.00625877008054312\\
310.01	0.00625815733305593\\
311.01	0.00625753141325752\\
312.01	0.00625689203071042\\
313.01	0.00625623888805078\\
314.01	0.00625557168078717\\
315.01	0.00625489009709179\\
316.01	0.00625419381758304\\
317.01	0.00625348251509964\\
318.01	0.00625275585446366\\
319.01	0.00625201349223544\\
320.01	0.00625125507645604\\
321.01	0.00625048024637951\\
322.01	0.00624968863219228\\
323.01	0.00624887985472068\\
324.01	0.00624805352512373\\
325.01	0.0062472092445725\\
326.01	0.00624634660391358\\
327.01	0.00624546518331611\\
328.01	0.00624456455190193\\
329.01	0.00624364426735621\\
330.01	0.0062427038755192\\
331.01	0.00624174290995664\\
332.01	0.00624076089150639\\
333.01	0.00623975732780229\\
334.01	0.00623873171277178\\
335.01	0.00623768352610487\\
336.01	0.00623661223269421\\
337.01	0.0062355172820435\\
338.01	0.0062343981076407\\
339.01	0.006233254126295\\
340.01	0.00623208473743404\\
341.01	0.00623088932235893\\
342.01	0.00622966724345284\\
343.01	0.00622841784334071\\
344.01	0.00622714044399542\\
345.01	0.00622583434578652\\
346.01	0.0062244988264667\\
347.01	0.00622313314009197\\
348.01	0.00622173651586795\\
349.01	0.0062203081569188\\
350.01	0.00621884723897052\\
351.01	0.00621735290894193\\
352.01	0.00621582428343583\\
353.01	0.00621426044712168\\
354.01	0.00621266045099962\\
355.01	0.00621102331053706\\
356.01	0.00620934800366509\\
357.01	0.00620763346862314\\
358.01	0.00620587860163797\\
359.01	0.00620408225442216\\
360.01	0.006202243231476\\
361.01	0.00620036028717386\\
362.01	0.00619843212261682\\
363.01	0.00619645738222726\\
364.01	0.006194434650064\\
365.01	0.00619236244583001\\
366.01	0.00619023922054431\\
367.01	0.0061880633518454\\
368.01	0.00618583313889235\\
369.01	0.00618354679682352\\
370.01	0.00618120245073172\\
371.01	0.00617879812910809\\
372.01	0.00617633175670501\\
373.01	0.00617380114676127\\
374.01	0.00617120399252893\\
375.01	0.00616853785803526\\
376.01	0.00616580016800745\\
377.01	0.0061629881968826\\
378.01	0.00616009905681787\\
379.01	0.00615712968461129\\
380.01	0.00615407682743772\\
381.01	0.00615093702729765\\
382.01	0.00614770660407561\\
383.01	0.00614438163709984\\
384.01	0.00614095794509658\\
385.01	0.00613743106443638\\
386.01	0.0061337962255768\\
387.01	0.00613004832762434\\
388.01	0.00612618191096159\\
389.01	0.00612219112792623\\
390.01	0.00611806971158247\\
391.01	0.00611381094270715\\
392.01	0.00610940761522144\\
393.01	0.00610485200045085\\
394.01	0.00610013581080529\\
395.01	0.00609525016374377\\
396.01	0.00609018554726357\\
397.01	0.00608493178864853\\
398.01	0.00607947802886572\\
399.01	0.00607381270586999\\
400.01	0.00606792355121726\\
401.01	0.00606179760589656\\
402.01	0.00605542126325994\\
403.01	0.00604878034952735\\
404.01	0.00604186025572631\\
405.01	0.00603464613937136\\
406.01	0.00602712321997983\\
407.01	0.00601927720010481\\
408.01	0.00601109485345431\\
409.01	0.00600256483459055\\
410.01	0.00599367878155951\\
411.01	0.00598443280481278\\
412.01	0.00597482948450378\\
413.01	0.00596488193307401\\
414.01	0.0059546831236043\\
415.01	0.00594427709668789\\
416.01	0.00593365990536041\\
417.01	0.00592282755153628\\
418.01	0.00591177598822489\\
419.01	0.00590050112219411\\
420.01	0.00588899881712395\\
421.01	0.00587726489722895\\
422.01	0.00586529515155599\\
423.01	0.00585308533949892\\
424.01	0.00584063119717485\\
425.01	0.00582792844452006\\
426.01	0.00581497279344729\\
427.01	0.00580175995723419\\
428.01	0.00578828566130119\\
429.01	0.00577454565556098\\
430.01	0.00576053572854219\\
431.01	0.0057462517235175\\
432.01	0.0057316895568933\\
433.01	0.00571684523915225\\
434.01	0.00570171489867311\\
435.01	0.00568629480879477\\
436.01	0.00567058141853382\\
437.01	0.00565457138741205\\
438.01	0.00563826162490599\\
439.01	0.00562164933508589\\
440.01	0.00560473206707327\\
441.01	0.00558750777201248\\
442.01	0.00556997486731927\\
443.01	0.00555213230904023\\
444.01	0.00553397967322594\\
445.01	0.00551551724728434\\
446.01	0.00549674613234149\\
447.01	0.00547766835767428\\
448.01	0.00545828700829854\\
449.01	0.00543860636677846\\
450.01	0.00541863207024607\\
451.01	0.00539837128347489\\
452.01	0.00537783288860757\\
453.01	0.00535702769195026\\
454.01	0.00533596864670675\\
455.01	0.00531467108993984\\
456.01	0.0052931529927509\\
457.01	0.0052714352185777\\
458.01	0.00524954178253281\\
459.01	0.00522750010109665\\
460.01	0.0052053412154063\\
461.01	0.00518309996732541\\
462.01	0.00516081511544892\\
463.01	0.00513852934235167\\
464.01	0.00511628908477032\\
465.01	0.00509414411971495\\
466.01	0.00507214680710524\\
467.01	0.00505035085397365\\
468.01	0.0050288094198075\\
469.01	0.00500757232323613\\
470.01	0.00498668204924411\\
471.01	0.00496616831028183\\
472.01	0.00494604015159451\\
473.01	0.00492627232442627\\
474.01	0.00490655094030001\\
475.01	0.00488669277603261\\
476.01	0.0048667116707059\\
477.01	0.00484662252716021\\
478.01	0.00482644130395952\\
479.01	0.00480618498903938\\
480.01	0.00478587155017494\\
481.01	0.00476551985709734\\
482.01	0.00474514956932396\\
483.01	0.00472478098298111\\
484.01	0.00470443482912806\\
485.01	0.00468413201539438\\
486.01	0.00466389330222162\\
487.01	0.00464373890478968\\
488.01	0.00462368801200742\\
489.01	0.0046037582152793\\
490.01	0.00458396484218238\\
491.01	0.00456432019387871\\
492.01	0.0045448326919775\\
493.01	0.00452550595109121\\
494.01	0.00450633780899151\\
495.01	0.00448731936920279\\
496.01	0.00446843414388941\\
497.01	0.00444965743157864\\
498.01	0.00443095612914953\\
499.01	0.00441228926890717\\
500.01	0.0043936096982971\\
501.01	0.00437486750573409\\
502.01	0.00435602565748428\\
503.01	0.0043370789330409\\
504.01	0.00431802629035681\\
505.01	0.00429886530309774\\
506.01	0.00427959199346607\\
507.01	0.00426020067367799\\
508.01	0.0042406838042687\\
509.01	0.00422103187942214\\
510.01	0.00420123335174307\\
511.01	0.00418127461102447\\
512.01	0.00416114003344601\\
513.01	0.00414081211889399\\
514.01	0.00412027173411878\\
515.01	0.00409949847735075\\
516.01	0.00407847117443859\\
517.01	0.00405716850560663\\
518.01	0.00403556974269061\\
519.01	0.00401365554521901\\
520.01	0.00399140871398024\\
521.01	0.00396881471001509\\
522.01	0.00394586099374016\\
523.01	0.00392253494191472\\
524.01	0.0038988233106298\\
525.01	0.00387471229518079\\
526.01	0.00385018761236385\\
527.01	0.00382523459925919\\
528.01	0.00379983832618786\\
529.01	0.00377398371974654\\
530.01	0.00374765568967166\\
531.01	0.00372083925084878\\
532.01	0.00369351962926699\\
533.01	0.00366568233851864\\
534.01	0.00363731321224822\\
535.01	0.00360839837891926\\
536.01	0.00357892417022308\\
537.01	0.00354887696689357\\
538.01	0.00351824303743295\\
539.01	0.00348700849082071\\
540.01	0.00345515929959017\\
541.01	0.00342268132576639\\
542.01	0.0033895603433927\\
543.01	0.00335578205619657\\
544.01	0.00332133210906197\\
545.01	0.00328619609232783\\
546.01	0.00325035953859222\\
547.01	0.00321380791264442\\
548.01	0.00317652659640092\\
549.01	0.00313850087220274\\
550.01	0.00309971590924588\\
551.01	0.00306015675865827\\
552.01	0.00301980836014317\\
553.01	0.00297865555498463\\
554.01	0.00293668310007627\\
555.01	0.00289387568239544\\
556.01	0.00285021793428497\\
557.01	0.00280569445007147\\
558.01	0.00276028980470851\\
559.01	0.00271398857526332\\
560.01	0.0026667753661384\\
561.01	0.0026186348388955\\
562.01	0.0025695517473923\\
563.01	0.00251951097863612\\
564.01	0.00246849759935086\\
565.01	0.00241649690806831\\
566.01	0.0023634944929314\\
567.01	0.0023094762957353\\
568.01	0.00225442868280603\\
569.01	0.00219833852332488\\
570.01	0.00214119327569023\\
571.01	0.00208298108246332\\
572.01	0.00202369087436417\\
573.01	0.00196331248367459\\
574.01	0.00190183676726109\\
575.01	0.0018392557392658\\
576.01	0.0017755627133218\\
577.01	0.00171075245390312\\
578.01	0.00164482133605403\\
579.01	0.00157776751222645\\
580.01	0.00150959108427172\\
581.01	0.00144029427774478\\
582.01	0.00136988161453246\\
583.01	0.001298360078346\\
584.01	0.00122573926573864\\
585.01	0.00115203151291554\\
586.01	0.0010772519855635\\
587.01	0.00100141871507007\\
588.01	0.000924552559616688\\
589.01	0.000846677062455354\\
590.01	0.000767818171894426\\
591.01	0.000688003777723883\\
592.01	0.000607263006503161\\
593.01	0.000525625202691551\\
594.01	0.000443118503243908\\
595.01	0.000359767889058513\\
596.01	0.000275592566341293\\
597.01	0.000190602493045661\\
598.01	0.000104793818189718\\
599.01	3.18230442566507e-05\\
599.02	3.12770957178118e-05\\
599.03	3.07343620570243e-05\\
599.04	3.01948752177914e-05\\
599.05	2.96586674567757e-05\\
599.06	2.9125771346868e-05\\
599.07	2.8596219780316e-05\\
599.08	2.80700459718581e-05\\
599.09	2.75472834618973e-05\\
599.1	2.70279661197196e-05\\
599.11	2.65121281467183e-05\\
599.12	2.59998040796645e-05\\
599.13	2.54910287940252e-05\\
599.14	2.49858375072735e-05\\
599.15	2.44842657822781e-05\\
599.16	2.39863495306858e-05\\
599.17	2.34921250163653e-05\\
599.18	2.30016288588781e-05\\
599.19	2.2514898036969e-05\\
599.2	2.20319698921133e-05\\
599.21	2.15528821320803e-05\\
599.22	2.10776728345428e-05\\
599.23	2.06063804507207e-05\\
599.24	2.01390438090542e-05\\
599.25	1.96757021189257e-05\\
599.26	1.92163949743994e-05\\
599.27	1.87611623580202e-05\\
599.28	1.83100446446254e-05\\
599.29	1.78630826052212e-05\\
599.3	1.7420317410876e-05\\
599.31	1.69817906366561e-05\\
599.32	1.65475442656122e-05\\
599.33	1.61176206927866e-05\\
599.34	1.56920627292778e-05\\
599.35	1.52709136063342e-05\\
599.36	1.48542169794846e-05\\
599.37	1.44420169327295e-05\\
599.38	1.40343579827472e-05\\
599.39	1.36312859119678e-05\\
599.4	1.32328499381947e-05\\
599.41	1.28390997671674e-05\\
599.42	1.24500855973562e-05\\
599.43	1.20658581248146e-05\\
599.44	1.16864685480583e-05\\
599.45	1.13119685730117e-05\\
599.46	1.09424104179964e-05\\
599.47	1.05778468187656e-05\\
599.48	1.02183310335905e-05\\
599.49	9.86391684839466e-06\\
599.5	9.51465858194285e-06\\
599.51	9.17061109107463e-06\\
599.52	8.83182977599699e-06\\
599.53	8.49837058562383e-06\\
599.54	8.1702900229675e-06\\
599.55	7.84764515058753e-06\\
599.56	7.53049359608279e-06\\
599.57	7.21889355765649e-06\\
599.58	6.91290380971064e-06\\
599.59	6.61258370851861e-06\\
599.6	6.31799319793756e-06\\
599.61	6.02919281519031e-06\\
599.62	5.74624369668701e-06\\
599.63	5.46920758391981e-06\\
599.64	5.19814682940593e-06\\
599.65	4.93312440269685e-06\\
599.66	4.67420389643411e-06\\
599.67	4.42144953247646e-06\\
599.68	4.17492616808582e-06\\
599.69	3.9346993021671e-06\\
599.7	3.70083508156871e-06\\
599.71	3.47340030746289e-06\\
599.72	3.25246244175549e-06\\
599.73	3.03808961359987e-06\\
599.74	2.83035062593855e-06\\
599.75	2.62931496212288e-06\\
599.76	2.43505279260738e-06\\
599.77	2.24763498169606e-06\\
599.78	2.06713309435641e-06\\
599.79	1.89361940310974e-06\\
599.8	1.72716689498045e-06\\
599.81	1.56784927850782e-06\\
599.82	1.41574099085315e-06\\
599.83	1.27091720493813e-06\\
599.84	1.13345383668736e-06\\
599.85	1.00342755231415e-06\\
599.86	8.8091577571392e-07\\
599.87	7.65996695881871e-07\\
599.88	6.58749274441706e-07\\
599.89	5.59253253241965e-07\\
599.9	4.67589162013102e-07\\
599.91	3.83838326099145e-07\\
599.92	3.0808287429171e-07\\
599.93	2.4040574671258e-07\\
599.94	1.80890702777825e-07\\
599.95	1.29622329257326e-07\\
599.96	8.66860484019516e-08\\
599.97	5.21681261331924e-08\\
599.98	2.61556803542173e-08\\
599.99	8.73668930083393e-09\\
600	0\\
};
\addplot [color=red!75!mycolor17,solid,forget plot]
  table[row sep=crcr]{%
0.01	0.00588381158434445\\
1.01	0.00588380981397887\\
2.01	0.00588380800610954\\
3.01	0.00588380615994001\\
4.01	0.0058838042746568\\
5.01	0.00588380234942946\\
6.01	0.0058838003834096\\
7.01	0.00588379837573055\\
8.01	0.00588379632550778\\
9.01	0.00588379423183732\\
10.01	0.00588379209379631\\
11.01	0.00588378991044209\\
12.01	0.00588378768081189\\
13.01	0.00588378540392255\\
14.01	0.00588378307877\\
15.01	0.00588378070432864\\
16.01	0.00588377827955094\\
17.01	0.00588377580336728\\
18.01	0.00588377327468511\\
19.01	0.0058837706923886\\
20.01	0.00588376805533812\\
21.01	0.00588376536236967\\
22.01	0.00588376261229484\\
23.01	0.00588375980389944\\
24.01	0.00588375693594367\\
25.01	0.00588375400716098\\
26.01	0.00588375101625819\\
27.01	0.0058837479619143\\
28.01	0.00588374484278038\\
29.01	0.00588374165747836\\
30.01	0.00588373840460112\\
31.01	0.00588373508271131\\
32.01	0.00588373169034102\\
33.01	0.00588372822599129\\
34.01	0.0058837246881306\\
35.01	0.00588372107519516\\
36.01	0.00588371738558779\\
37.01	0.00588371361767722\\
38.01	0.00588370976979719\\
39.01	0.00588370584024625\\
40.01	0.00588370182728623\\
41.01	0.00588369772914211\\
42.01	0.00588369354400089\\
43.01	0.00588368927001094\\
44.01	0.00588368490528103\\
45.01	0.00588368044787969\\
46.01	0.00588367589583411\\
47.01	0.00588367124712947\\
48.01	0.00588366649970772\\
49.01	0.00588366165146701\\
50.01	0.00588365670026041\\
51.01	0.00588365164389562\\
52.01	0.00588364648013299\\
53.01	0.00588364120668539\\
54.01	0.00588363582121674\\
55.01	0.00588363032134091\\
56.01	0.00588362470462086\\
57.01	0.00588361896856785\\
58.01	0.00588361311063934\\
59.01	0.00588360712823887\\
60.01	0.00588360101871454\\
61.01	0.00588359477935745\\
62.01	0.00588358840740108\\
63.01	0.00588358190001976\\
64.01	0.00588357525432742\\
65.01	0.00588356846737608\\
66.01	0.00588356153615494\\
67.01	0.00588355445758881\\
68.01	0.00588354722853688\\
69.01	0.00588353984579109\\
70.01	0.00588353230607453\\
71.01	0.0058835246060407\\
72.01	0.00588351674227109\\
73.01	0.00588350871127429\\
74.01	0.0058835005094842\\
75.01	0.00588349213325842\\
76.01	0.00588348357887669\\
77.01	0.00588347484253906\\
78.01	0.00588346592036446\\
79.01	0.00588345680838855\\
80.01	0.00588344750256232\\
81.01	0.00588343799874999\\
82.01	0.00588342829272758\\
83.01	0.00588341838018034\\
84.01	0.00588340825670129\\
85.01	0.00588339791778932\\
86.01	0.00588338735884676\\
87.01	0.00588337657517755\\
88.01	0.00588336556198514\\
89.01	0.00588335431437017\\
90.01	0.00588334282732862\\
91.01	0.00588333109574909\\
92.01	0.00588331911441085\\
93.01	0.00588330687798141\\
94.01	0.00588329438101373\\
95.01	0.00588328161794454\\
96.01	0.00588326858309107\\
97.01	0.00588325527064932\\
98.01	0.00588324167469059\\
99.01	0.00588322778915878\\
100.01	0.00588321360786858\\
101.01	0.00588319912450226\\
102.01	0.00588318433260617\\
103.01	0.00588316922558886\\
104.01	0.00588315379671724\\
105.01	0.00588313803911415\\
106.01	0.00588312194575496\\
107.01	0.00588310550946438\\
108.01	0.00588308872291343\\
109.01	0.00588307157861613\\
110.01	0.00588305406892553\\
111.01	0.00588303618603111\\
112.01	0.00588301792195478\\
113.01	0.00588299926854728\\
114.01	0.00588298021748495\\
115.01	0.00588296076026503\\
116.01	0.005882940888203\\
117.01	0.00588292059242756\\
118.01	0.00588289986387734\\
119.01	0.00588287869329652\\
120.01	0.00588285707123074\\
121.01	0.00588283498802288\\
122.01	0.00588281243380844\\
123.01	0.00588278939851127\\
124.01	0.00588276587183914\\
125.01	0.00588274184327876\\
126.01	0.0058827173020915\\
127.01	0.00588269223730786\\
128.01	0.00588266663772314\\
129.01	0.00588264049189196\\
130.01	0.00588261378812332\\
131.01	0.0058825865144752\\
132.01	0.00588255865874927\\
133.01	0.00588253020848514\\
134.01	0.00588250115095503\\
135.01	0.00588247147315758\\
136.01	0.00588244116181235\\
137.01	0.00588241020335382\\
138.01	0.00588237858392497\\
139.01	0.00588234628937106\\
140.01	0.00588231330523328\\
141.01	0.00588227961674249\\
142.01	0.00588224520881197\\
143.01	0.00588221006603104\\
144.01	0.00588217417265786\\
145.01	0.00588213751261204\\
146.01	0.00588210006946808\\
147.01	0.0058820618264473\\
148.01	0.00588202276641042\\
149.01	0.00588198287184971\\
150.01	0.00588194212488134\\
151.01	0.00588190050723683\\
152.01	0.0058818580002552\\
153.01	0.00588181458487425\\
154.01	0.00588177024162174\\
155.01	0.00588172495060694\\
156.01	0.00588167869151168\\
157.01	0.00588163144358043\\
158.01	0.00588158318561174\\
159.01	0.00588153389594783\\
160.01	0.00588148355246559\\
161.01	0.00588143213256567\\
162.01	0.00588137961316285\\
163.01	0.00588132597067528\\
164.01	0.00588127118101404\\
165.01	0.0058812152195715\\
166.01	0.0058811580612111\\
167.01	0.00588109968025551\\
168.01	0.00588104005047472\\
169.01	0.00588097914507433\\
170.01	0.00588091693668332\\
171.01	0.0058808533973417\\
172.01	0.00588078849848801\\
173.01	0.00588072221094589\\
174.01	0.00588065450491102\\
175.01	0.0058805853499375\\
176.01	0.00588051471492441\\
177.01	0.00588044256810102\\
178.01	0.00588036887701293\\
179.01	0.00588029360850673\\
180.01	0.00588021672871559\\
181.01	0.00588013820304342\\
182.01	0.00588005799614911\\
183.01	0.0058799760719306\\
184.01	0.00587989239350861\\
185.01	0.00587980692320948\\
186.01	0.00587971962254822\\
187.01	0.00587963045221116\\
188.01	0.00587953937203801\\
189.01	0.00587944634100323\\
190.01	0.00587935131719767\\
191.01	0.00587925425780959\\
192.01	0.0058791551191048\\
193.01	0.00587905385640724\\
194.01	0.00587895042407826\\
195.01	0.00587884477549585\\
196.01	0.00587873686303343\\
197.01	0.00587862663803878\\
198.01	0.00587851405081081\\
199.01	0.00587839905057767\\
200.01	0.00587828158547348\\
201.01	0.00587816160251482\\
202.01	0.0058780390475759\\
203.01	0.00587791386536484\\
204.01	0.00587778599939807\\
205.01	0.0058776553919744\\
206.01	0.00587752198414932\\
207.01	0.00587738571570717\\
208.01	0.0058772465251347\\
209.01	0.00587710434959241\\
210.01	0.00587695912488636\\
211.01	0.0058768107854383\\
212.01	0.00587665926425642\\
213.01	0.00587650449290441\\
214.01	0.00587634640147051\\
215.01	0.00587618491853521\\
216.01	0.00587601997113923\\
217.01	0.00587585148474994\\
218.01	0.00587567938322747\\
219.01	0.00587550358878984\\
220.01	0.00587532402197739\\
221.01	0.0058751406016171\\
222.01	0.00587495324478494\\
223.01	0.00587476186676826\\
224.01	0.00587456638102727\\
225.01	0.00587436669915539\\
226.01	0.00587416273083883\\
227.01	0.0058739543838156\\
228.01	0.00587374156383323\\
229.01	0.00587352417460554\\
230.01	0.00587330211776942\\
231.01	0.00587307529283898\\
232.01	0.00587284359716003\\
233.01	0.00587260692586363\\
234.01	0.00587236517181778\\
235.01	0.00587211822557813\\
236.01	0.00587186597533873\\
237.01	0.00587160830688041\\
238.01	0.00587134510351866\\
239.01	0.00587107624605066\\
240.01	0.00587080161270023\\
241.01	0.00587052107906241\\
242.01	0.00587023451804657\\
243.01	0.00586994179981858\\
244.01	0.00586964279174073\\
245.01	0.00586933735831147\\
246.01	0.00586902536110367\\
247.01	0.00586870665870073\\
248.01	0.00586838110663213\\
249.01	0.00586804855730707\\
250.01	0.00586770885994708\\
251.01	0.00586736186051688\\
252.01	0.00586700740165385\\
253.01	0.00586664532259585\\
254.01	0.00586627545910731\\
255.01	0.00586589764340432\\
256.01	0.00586551170407741\\
257.01	0.00586511746601293\\
258.01	0.00586471475031261\\
259.01	0.00586430337421192\\
260.01	0.00586388315099517\\
261.01	0.00586345388991064\\
262.01	0.00586301539608245\\
263.01	0.00586256747042118\\
264.01	0.00586210990953192\\
265.01	0.00586164250562143\\
266.01	0.00586116504640177\\
267.01	0.00586067731499283\\
268.01	0.00586017908982239\\
269.01	0.00585967014452416\\
270.01	0.00585915024783305\\
271.01	0.00585861916347889\\
272.01	0.00585807665007678\\
273.01	0.0058575224610167\\
274.01	0.00585695634434821\\
275.01	0.00585637804266499\\
276.01	0.00585578729298533\\
277.01	0.00585518382663051\\
278.01	0.00585456736910046\\
279.01	0.0058539376399463\\
280.01	0.00585329435264057\\
281.01	0.00585263721444388\\
282.01	0.00585196592626974\\
283.01	0.00585128018254474\\
284.01	0.00585057967106704\\
285.01	0.00584986407286065\\
286.01	0.00584913306202712\\
287.01	0.00584838630559376\\
288.01	0.00584762346335831\\
289.01	0.0058468441877298\\
290.01	0.00584604812356651\\
291.01	0.0058452349080099\\
292.01	0.00584440417031455\\
293.01	0.00584355553167466\\
294.01	0.00584268860504691\\
295.01	0.00584180299496801\\
296.01	0.00584089829736983\\
297.01	0.00583997409938889\\
298.01	0.00583902997917247\\
299.01	0.00583806550567998\\
300.01	0.00583708023847964\\
301.01	0.00583607372754137\\
302.01	0.0058350455130235\\
303.01	0.00583399512505619\\
304.01	0.00583292208351904\\
305.01	0.00583182589781418\\
306.01	0.00583070606663385\\
307.01	0.00582956207772283\\
308.01	0.00582839340763591\\
309.01	0.00582719952148923\\
310.01	0.00582597987270716\\
311.01	0.00582473390276236\\
312.01	0.00582346104091066\\
313.01	0.00582216070392023\\
314.01	0.00582083229579483\\
315.01	0.00581947520749061\\
316.01	0.00581808881662733\\
317.01	0.00581667248719307\\
318.01	0.00581522556924281\\
319.01	0.00581374739859083\\
320.01	0.00581223729649667\\
321.01	0.0058106945693446\\
322.01	0.00580911850831705\\
323.01	0.00580750838906135\\
324.01	0.00580586347135024\\
325.01	0.00580418299873598\\
326.01	0.00580246619819846\\
327.01	0.00580071227978684\\
328.01	0.00579892043625547\\
329.01	0.00579708984269393\\
330.01	0.00579521965615169\\
331.01	0.0057933090152569\\
332.01	0.00579135703983091\\
333.01	0.00578936283049848\\
334.01	0.00578732546829255\\
335.01	0.00578524401425749\\
336.01	0.00578311750904827\\
337.01	0.00578094497252816\\
338.01	0.00577872540336539\\
339.01	0.00577645777862977\\
340.01	0.00577414105339103\\
341.01	0.00577177416031952\\
342.01	0.00576935600929145\\
343.01	0.0057668854870002\\
344.01	0.00576436145657629\\
345.01	0.0057617827572184\\
346.01	0.00575914820383806\\
347.01	0.00575645658672056\\
348.01	0.00575370667120748\\
349.01	0.00575089719740307\\
350.01	0.0057480268799102\\
351.01	0.00574509440760137\\
352.01	0.00574209844342953\\
353.01	0.00573903762428736\\
354.01	0.00573591056092131\\
355.01	0.00573271583790994\\
356.01	0.0057294520137157\\
357.01	0.00572611762082254\\
358.01	0.00572271116597093\\
359.01	0.00571923113050552\\
360.01	0.00571567597085059\\
361.01	0.00571204411913296\\
362.01	0.00570833398397189\\
363.01	0.00570454395145984\\
364.01	0.00570067238636121\\
365.01	0.00569671763355712\\
366.01	0.0056926780197718\\
367.01	0.00568855185561727\\
368.01	0.00568433743799914\\
369.01	0.00568003305293321\\
370.01	0.00567563697882439\\
371.01	0.00567114749027247\\
372.01	0.00566656286246969\\
373.01	0.0056618813762707\\
374.01	0.0056571013240185\\
375.01	0.00565222101622497\\
376.01	0.0056472387892136\\
377.01	0.00564215301384461\\
378.01	0.00563696210545766\\
379.01	0.00563166453517971\\
380.01	0.00562625884276317\\
381.01	0.00562074365113432\\
382.01	0.0056151176828499\\
383.01	0.00560937977867574\\
384.01	0.00560352891851695\\
385.01	0.00559756424494493\\
386.01	0.00559148508957717\\
387.01	0.00558529100257136\\
388.01	0.00557898178549608\\
389.01	0.00557255752782516\\
390.01	0.00556601864727755\\
391.01	0.00555936593417019\\
392.01	0.00555260059987158\\
393.01	0.00554572432931722\\
394.01	0.0055387393373643\\
395.01	0.00553164842850102\\
396.01	0.00552445505905949\\
397.01	0.00551716340056851\\
398.01	0.00550977840219078\\
399.01	0.00550230584923837\\
400.01	0.00549475241348927\\
401.01	0.00548712568931336\\
402.01	0.0054794342073349\\
403.01	0.00547168741431669\\
404.01	0.00546389560392106\\
405.01	0.00545606977766861\\
406.01	0.00544822140836945\\
407.01	0.00544036206901984\\
408.01	0.00543250287793826\\
409.01	0.00542465369485019\\
410.01	0.00541682198154169\\
411.01	0.0054090112130263\\
412.01	0.00540121868888936\\
413.01	0.0053934311467888\\
414.01	0.00538555388893473\\
415.01	0.00537753794871848\\
416.01	0.00536938187750997\\
417.01	0.00536108428583244\\
418.01	0.00535264385052405\\
419.01	0.00534405932245016\\
420.01	0.00533532953480055\\
421.01	0.005326453412012\\
422.01	0.0053174299793551\\
423.01	0.00530825837321148\\
424.01	0.00529893785206253\\
425.01	0.00528946780823533\\
426.01	0.00527984778044697\\
427.01	0.0052700774671813\\
428.01	0.00526015674092809\\
429.01	0.00525008566331355\\
430.01	0.00523986450114147\\
431.01	0.00522949374336115\\
432.01	0.00521897411896488\\
433.01	0.00520830661580828\\
434.01	0.00519749250032951\\
435.01	0.00518653333812438\\
436.01	0.00517543101530906\\
437.01	0.00516418776057607\\
438.01	0.00515280616780604\\
439.01	0.00514128921905997\\
440.01	0.00512964030771935\\
441.01	0.00511786326147795\\
442.01	0.00510596236481325\\
443.01	0.005093942380472\\
444.01	0.00508180856939705\\
445.01	0.00506956670839642\\
446.01	0.00505722310470398\\
447.01	0.00504478460641066\\
448.01	0.00503225860754379\\
449.01	0.00501965304634434\\
450.01	0.00500697639503538\\
451.01	0.00499423763908425\\
452.01	0.00498144624364622\\
453.01	0.00496861210453071\\
454.01	0.00495574548067791\\
455.01	0.00494285690482716\\
456.01	0.00492995706867163\\
457.01	0.00491705667849878\\
458.01	0.00490416627718298\\
459.01	0.00489129602842898\\
460.01	0.00487845545954164\\
461.01	0.00486565315984728\\
462.01	0.00485289643286296\\
463.01	0.00484019090226389\\
464.01	0.00482754007636114\\
465.01	0.00481494488149143\\
466.01	0.00480240318289198\\
467.01	0.00478990932375952\\
468.01	0.00477745373048349\\
469.01	0.0047650226563763\\
470.01	0.00475259816988718\\
471.01	0.00474015853571868\\
472.01	0.00472767920367289\\
473.01	0.00471513474777203\\
474.01	0.00470250725175043\\
475.01	0.00468979689859931\\
476.01	0.00467700686081363\\
477.01	0.00466414014221827\\
478.01	0.00465119951291316\\
479.01	0.00463818743693199\\
480.01	0.00462510599264845\\
481.01	0.00461195678621757\\
482.01	0.00459874085866007\\
483.01	0.00458545858761119\\
484.01	0.00457210958527451\\
485.01	0.00455869259475305\\
486.01	0.00454520538770111\\
487.01	0.00453164466714042\\
488.01	0.00451800598032893\\
489.01	0.00450428364772914\\
490.01	0.00449047071535862\\
491.01	0.00447655893905665\\
492.01	0.00446253881033878\\
493.01	0.00444839963432259\\
494.01	0.00443412967037021\\
495.01	0.00441971634512523\\
496.01	0.00440514654481864\\
497.01	0.00439040698802298\\
498.01	0.00437548466997626\\
499.01	0.00436036735305318\\
500.01	0.00434504405182897\\
501.01	0.00432950542122603\\
502.01	0.00431374376253249\\
503.01	0.00429775181588751\\
504.01	0.00428152185964032\\
505.01	0.00426504566107352\\
506.01	0.00424831449026248\\
507.01	0.00423131914387567\\
508.01	0.00421404997969988\\
509.01	0.00419649696233378\\
510.01	0.00417864971998277\\
511.01	0.00416049761158845\\
512.01	0.00414202980261649\\
513.01	0.00412323534667832\\
514.01	0.00410410326881245\\
515.01	0.00408462264473276\\
516.01	0.00406478266880003\\
517.01	0.00404457270213406\\
518.01	0.00402398229155767\\
519.01	0.00400300115064997\\
520.01	0.00398161909713912\\
521.01	0.00395982594793969\\
522.01	0.00393761139778053\\
523.01	0.0039149649645009\\
524.01	0.00389187599995229\\
525.01	0.00386833370872558\\
526.01	0.00384432716607575\\
527.01	0.00381984533406512\\
528.01	0.00379487707489989\\
529.01	0.00376941116044047\\
530.01	0.00374343627697373\\
531.01	0.00371694102457945\\
532.01	0.00368991391083733\\
533.01	0.00366234333922554\\
534.01	0.00363421759335883\\
535.01	0.00360552481914565\\
536.01	0.00357625300787387\\
537.01	0.00354638998385786\\
538.01	0.0035159233995706\\
539.01	0.00348484073635985\\
540.01	0.00345312930589523\\
541.01	0.00342077625093994\\
542.01	0.0033877685454502\\
543.01	0.00335409299413567\\
544.01	0.00331973623170642\\
545.01	0.00328468472213207\\
546.01	0.00324892475832306\\
547.01	0.0032124424627048\\
548.01	0.00317522378916534\\
549.01	0.00313725452679205\\
550.01	0.00309852030566009\\
551.01	0.00305900660468921\\
552.01	0.00301869876133101\\
553.01	0.0029775819828656\\
554.01	0.00293564135940069\\
555.01	0.00289286187882892\\
556.01	0.00284922844403485\\
557.01	0.00280472589266526\\
558.01	0.00275933901978964\\
559.01	0.00271305260378651\\
560.01	0.00266585143579169\\
561.01	0.00261772035304431\\
562.01	0.00256864427646527\\
563.01	0.00251860825281521\\
564.01	0.00246759750180826\\
565.01	0.00241559746860962\\
566.01	0.00236259388219452\\
567.01	0.00230857282007239\\
568.01	0.00225352077989341\\
569.01	0.00219742475845609\\
570.01	0.00214027233862706\\
571.01	0.00208205178465884\\
572.01	0.00202275214634875\\
573.01	0.00196236337241117\\
574.01	0.00190087643333459\\
575.01	0.00183828345384417\\
576.01	0.00177457785488343\\
577.01	0.00170975450473966\\
578.01	0.00164380987854659\\
579.01	0.00157674222487883\\
580.01	0.00150855173747032\\
581.01	0.00143924072920082\\
582.01	0.00136881380434555\\
583.01	0.00129727802360513\\
584.01	0.00122464305453996\\
585.01	0.00115092129761855\\
586.01	0.00107612797501737\\
587.01	0.00100028116541664\\
588.01	0.000923401763108929\\
589.01	0.000845513333520894\\
590.01	0.000766641829415248\\
591.01	0.000686815122188048\\
592.01	0.000606062290302432\\
593.01	0.000524412591375313\\
594.01	0.000441894024979966\\
595.01	0.000358531368867966\\
596.01	0.000274343540840818\\
597.01	0.0001893401004031\\
598.01	0.000103516656727439\\
599.01	3.18230442563801e-05\\
599.02	3.1277095717562e-05\\
599.03	3.07343620567901e-05\\
599.04	3.01948752175762e-05\\
599.05	2.96586674565728e-05\\
599.06	2.91257713466823e-05\\
599.07	2.85962197801425e-05\\
599.08	2.8070045971695e-05\\
599.09	2.75472834617499e-05\\
599.1	2.70279661195826e-05\\
599.11	2.65121281465882e-05\\
599.12	2.59998040795482e-05\\
599.13	2.54910287939159e-05\\
599.14	2.49858375071747e-05\\
599.15	2.44842657821844e-05\\
599.16	2.39863495305973e-05\\
599.17	2.34921250162872e-05\\
599.18	2.30016288588052e-05\\
599.19	2.25148980369048e-05\\
599.2	2.20319698920526e-05\\
599.21	2.1552882132023e-05\\
599.22	2.10776728344908e-05\\
599.23	2.06063804506721e-05\\
599.24	2.01390438090126e-05\\
599.25	1.96757021188858e-05\\
599.26	1.92163949743647e-05\\
599.27	1.87611623579872e-05\\
599.28	1.83100446445959e-05\\
599.29	1.78630826051952e-05\\
599.3	1.74203174108534e-05\\
599.31	1.69817906366353e-05\\
599.32	1.65475442655931e-05\\
599.33	1.61176206927693e-05\\
599.34	1.56920627292639e-05\\
599.35	1.52709136063203e-05\\
599.36	1.48542169794742e-05\\
599.37	1.44420169327208e-05\\
599.38	1.40343579827368e-05\\
599.39	1.36312859119591e-05\\
599.4	1.32328499381877e-05\\
599.41	1.28390997671604e-05\\
599.42	1.24500855973528e-05\\
599.43	1.20658581248094e-05\\
599.44	1.16864685480531e-05\\
599.45	1.13119685730065e-05\\
599.46	1.09424104179946e-05\\
599.47	1.05778468187639e-05\\
599.48	1.02183310335888e-05\\
599.49	9.86391684839466e-06\\
599.5	9.51465858194112e-06\\
599.51	9.17061109107116e-06\\
599.52	8.83182977599525e-06\\
599.53	8.4983705856221e-06\\
599.54	8.1702900229675e-06\\
599.55	7.84764515058753e-06\\
599.56	7.53049359608453e-06\\
599.57	7.21889355765649e-06\\
599.58	6.91290380971064e-06\\
599.59	6.61258370851688e-06\\
599.6	6.31799319793756e-06\\
599.61	6.02919281519031e-06\\
599.62	5.74624369668701e-06\\
599.63	5.46920758391981e-06\\
599.64	5.19814682940767e-06\\
599.65	4.93312440269685e-06\\
599.66	4.67420389643411e-06\\
599.67	4.42144953247646e-06\\
599.68	4.17492616808582e-06\\
599.69	3.93469930216536e-06\\
599.7	3.70083508157044e-06\\
599.71	3.47340030746116e-06\\
599.72	3.25246244175549e-06\\
599.73	3.03808961360161e-06\\
599.74	2.83035062593855e-06\\
599.75	2.62931496212288e-06\\
599.76	2.43505279260738e-06\\
599.77	2.24763498169606e-06\\
599.78	2.06713309435641e-06\\
599.79	1.89361940311147e-06\\
599.8	1.72716689497872e-06\\
599.81	1.56784927850956e-06\\
599.82	1.41574099085315e-06\\
599.83	1.27091720493987e-06\\
599.84	1.13345383668563e-06\\
599.85	1.00342755231589e-06\\
599.86	8.8091577571392e-07\\
599.87	7.65996695880136e-07\\
599.88	6.5874927444344e-07\\
599.89	5.59253253243699e-07\\
599.9	4.67589162011367e-07\\
599.91	3.83838326099145e-07\\
599.92	3.0808287429171e-07\\
599.93	2.40405746710845e-07\\
599.94	1.8089070277609e-07\\
599.95	1.2962232925906e-07\\
599.96	8.66860484002169e-08\\
599.97	5.21681261331924e-08\\
599.98	2.61556803542173e-08\\
599.99	8.73668930083393e-09\\
600	0\\
};
\addplot [color=red!80!mycolor19,solid,forget plot]
  table[row sep=crcr]{%
0.01	0.00547444890358919\\
1.01	0.00547444712667968\\
2.01	0.00547444531233606\\
3.01	0.00547444345976878\\
4.01	0.0054744415681716\\
5.01	0.00547443963672114\\
6.01	0.00547443766457662\\
7.01	0.00547443565087979\\
8.01	0.00547443359475392\\
9.01	0.005474431495304\\
10.01	0.00547442935161593\\
11.01	0.00547442716275643\\
12.01	0.00547442492777244\\
13.01	0.005474422645691\\
14.01	0.00547442031551823\\
15.01	0.00547441793623944\\
16.01	0.00547441550681854\\
17.01	0.00547441302619754\\
18.01	0.00547441049329609\\
19.01	0.00547440790701098\\
20.01	0.00547440526621585\\
21.01	0.00547440256976035\\
22.01	0.00547439981646993\\
23.01	0.00547439700514498\\
24.01	0.00547439413456091\\
25.01	0.00547439120346722\\
26.01	0.0054743882105868\\
27.01	0.00547438515461574\\
28.01	0.00547438203422226\\
29.01	0.00547437884804688\\
30.01	0.00547437559470116\\
31.01	0.00547437227276731\\
32.01	0.00547436888079773\\
33.01	0.0054743654173139\\
34.01	0.00547436188080657\\
35.01	0.00547435826973421\\
36.01	0.00547435458252294\\
37.01	0.0054743508175653\\
38.01	0.00547434697322043\\
39.01	0.00547434304781212\\
40.01	0.00547433903962925\\
41.01	0.00547433494692446\\
42.01	0.00547433076791327\\
43.01	0.00547432650077363\\
44.01	0.00547432214364497\\
45.01	0.00547431769462733\\
46.01	0.00547431315178085\\
47.01	0.00547430851312424\\
48.01	0.00547430377663492\\
49.01	0.00547429894024707\\
50.01	0.00547429400185142\\
51.01	0.00547428895929409\\
52.01	0.00547428381037574\\
53.01	0.00547427855285056\\
54.01	0.00547427318442522\\
55.01	0.00547426770275792\\
56.01	0.00547426210545742\\
57.01	0.00547425639008182\\
58.01	0.00547425055413781\\
59.01	0.00547424459507934\\
60.01	0.00547423851030633\\
61.01	0.0054742322971641\\
62.01	0.0054742259529418\\
63.01	0.00547421947487096\\
64.01	0.00547421286012485\\
65.01	0.00547420610581714\\
66.01	0.00547419920900045\\
67.01	0.00547419216666483\\
68.01	0.0054741849757369\\
69.01	0.00547417763307836\\
70.01	0.00547417013548458\\
71.01	0.00547416247968332\\
72.01	0.00547415466233293\\
73.01	0.00547414668002145\\
74.01	0.00547413852926446\\
75.01	0.00547413020650412\\
76.01	0.00547412170810745\\
77.01	0.00547411303036458\\
78.01	0.0054741041694873\\
79.01	0.00547409512160737\\
80.01	0.00547408588277475\\
81.01	0.00547407644895593\\
82.01	0.00547406681603214\\
83.01	0.00547405697979773\\
84.01	0.00547404693595819\\
85.01	0.00547403668012787\\
86.01	0.00547402620782881\\
87.01	0.00547401551448836\\
88.01	0.00547400459543732\\
89.01	0.00547399344590764\\
90.01	0.00547398206103065\\
91.01	0.00547397043583482\\
92.01	0.00547395856524361\\
93.01	0.00547394644407307\\
94.01	0.00547393406702997\\
95.01	0.00547392142870883\\
96.01	0.00547390852359046\\
97.01	0.00547389534603869\\
98.01	0.00547388189029847\\
99.01	0.00547386815049334\\
100.01	0.00547385412062224\\
101.01	0.0054738397945576\\
102.01	0.0054738251660425\\
103.01	0.0054738102286877\\
104.01	0.00547379497596908\\
105.01	0.00547377940122467\\
106.01	0.00547376349765189\\
107.01	0.00547374725830422\\
108.01	0.00547373067608866\\
109.01	0.00547371374376237\\
110.01	0.00547369645392947\\
111.01	0.00547367879903787\\
112.01	0.00547366077137625\\
113.01	0.00547364236307018\\
114.01	0.00547362356607897\\
115.01	0.00547360437219226\\
116.01	0.00547358477302621\\
117.01	0.00547356476002007\\
118.01	0.00547354432443229\\
119.01	0.00547352345733684\\
120.01	0.00547350214961929\\
121.01	0.00547348039197256\\
122.01	0.00547345817489328\\
123.01	0.00547343548867756\\
124.01	0.00547341232341654\\
125.01	0.00547338866899223\\
126.01	0.00547336451507322\\
127.01	0.00547333985110985\\
128.01	0.0054733146663301\\
129.01	0.00547328894973431\\
130.01	0.00547326269009112\\
131.01	0.00547323587593198\\
132.01	0.00547320849554641\\
133.01	0.00547318053697676\\
134.01	0.00547315198801326\\
135.01	0.00547312283618873\\
136.01	0.0054730930687728\\
137.01	0.00547306267276675\\
138.01	0.00547303163489744\\
139.01	0.00547299994161216\\
140.01	0.00547296757907228\\
141.01	0.00547293453314739\\
142.01	0.00547290078940917\\
143.01	0.00547286633312511\\
144.01	0.00547283114925213\\
145.01	0.00547279522243025\\
146.01	0.00547275853697545\\
147.01	0.00547272107687336\\
148.01	0.00547268282577222\\
149.01	0.00547264376697546\\
150.01	0.00547260388343509\\
151.01	0.00547256315774372\\
152.01	0.00547252157212722\\
153.01	0.00547247910843723\\
154.01	0.00547243574814315\\
155.01	0.00547239147232418\\
156.01	0.00547234626166093\\
157.01	0.00547230009642765\\
158.01	0.00547225295648282\\
159.01	0.00547220482126146\\
160.01	0.00547215566976527\\
161.01	0.00547210548055443\\
162.01	0.00547205423173795\\
163.01	0.00547200190096412\\
164.01	0.005471948465411\\
165.01	0.00547189390177681\\
166.01	0.00547183818626952\\
167.01	0.00547178129459688\\
168.01	0.0054717232019556\\
169.01	0.00547166388302144\\
170.01	0.0054716033119376\\
171.01	0.00547154146230374\\
172.01	0.00547147830716507\\
173.01	0.00547141381900007\\
174.01	0.00547134796970932\\
175.01	0.00547128073060318\\
176.01	0.00547121207238956\\
177.01	0.00547114196516117\\
178.01	0.00547107037838286\\
179.01	0.00547099728087881\\
180.01	0.0054709226408186\\
181.01	0.00547084642570411\\
182.01	0.00547076860235531\\
183.01	0.00547068913689645\\
184.01	0.00547060799474112\\
185.01	0.00547052514057796\\
186.01	0.00547044053835526\\
187.01	0.00547035415126584\\
188.01	0.00547026594173106\\
189.01	0.00547017587138508\\
190.01	0.00547008390105858\\
191.01	0.00546998999076199\\
192.01	0.00546989409966854\\
193.01	0.00546979618609676\\
194.01	0.00546969620749292\\
195.01	0.00546959412041315\\
196.01	0.00546948988050503\\
197.01	0.0054693834424882\\
198.01	0.00546927476013626\\
199.01	0.00546916378625628\\
200.01	0.00546905047266936\\
201.01	0.00546893477018997\\
202.01	0.00546881662860582\\
203.01	0.00546869599665605\\
204.01	0.0054685728220099\\
205.01	0.0054684470512447\\
206.01	0.00546831862982342\\
207.01	0.00546818750207195\\
208.01	0.00546805361115537\\
209.01	0.00546791689905432\\
210.01	0.00546777730654086\\
211.01	0.00546763477315352\\
212.01	0.00546748923717205\\
213.01	0.00546734063559167\\
214.01	0.0054671889040968\\
215.01	0.00546703397703456\\
216.01	0.00546687578738701\\
217.01	0.0054667142667433\\
218.01	0.00546654934527175\\
219.01	0.00546638095169039\\
220.01	0.00546620901323807\\
221.01	0.00546603345564368\\
222.01	0.00546585420309565\\
223.01	0.00546567117821107\\
224.01	0.0054654843020036\\
225.01	0.00546529349385084\\
226.01	0.00546509867146126\\
227.01	0.00546489975084056\\
228.01	0.0054646966462573\\
229.01	0.00546448927020777\\
230.01	0.00546427753338016\\
231.01	0.0054640613446185\\
232.01	0.00546384061088514\\
233.01	0.00546361523722327\\
234.01	0.00546338512671819\\
235.01	0.00546315018045823\\
236.01	0.0054629102974949\\
237.01	0.00546266537480206\\
238.01	0.00546241530723413\\
239.01	0.00546215998748435\\
240.01	0.00546189930604151\\
241.01	0.00546163315114636\\
242.01	0.00546136140874677\\
243.01	0.00546108396245219\\
244.01	0.0054608006934879\\
245.01	0.00546051148064738\\
246.01	0.00546021620024458\\
247.01	0.00545991472606548\\
248.01	0.00545960692931768\\
249.01	0.0054592926785807\\
250.01	0.00545897183975395\\
251.01	0.00545864427600442\\
252.01	0.00545830984771375\\
253.01	0.00545796841242399\\
254.01	0.00545761982478237\\
255.01	0.00545726393648532\\
256.01	0.00545690059622154\\
257.01	0.00545652964961366\\
258.01	0.00545615093916022\\
259.01	0.00545576430417472\\
260.01	0.00545536958072562\\
261.01	0.00545496660157401\\
262.01	0.00545455519611091\\
263.01	0.00545413519029307\\
264.01	0.00545370640657917\\
265.01	0.00545326866386236\\
266.01	0.00545282177740459\\
267.01	0.0054523655587686\\
268.01	0.00545189981574863\\
269.01	0.00545142435230043\\
270.01	0.00545093896847087\\
271.01	0.00545044346032559\\
272.01	0.00544993761987627\\
273.01	0.00544942123500623\\
274.01	0.00544889408939605\\
275.01	0.00544835596244741\\
276.01	0.00544780662920622\\
277.01	0.00544724586028455\\
278.01	0.00544667342178206\\
279.01	0.00544608907520625\\
280.01	0.00544549257739157\\
281.01	0.00544488368041826\\
282.01	0.00544426213152959\\
283.01	0.00544362767304888\\
284.01	0.00544298004229537\\
285.01	0.00544231897149949\\
286.01	0.00544164418771732\\
287.01	0.0054409554127445\\
288.01	0.00544025236302916\\
289.01	0.00543953474958493\\
290.01	0.00543880227790276\\
291.01	0.00543805464786266\\
292.01	0.00543729155364504\\
293.01	0.00543651268364177\\
294.01	0.00543571772036664\\
295.01	0.00543490634036652\\
296.01	0.00543407821413152\\
297.01	0.00543323300600598\\
298.01	0.00543237037409957\\
299.01	0.00543148997019855\\
300.01	0.00543059143967702\\
301.01	0.00542967442140992\\
302.01	0.00542873854768631\\
303.01	0.00542778344412324\\
304.01	0.00542680872958117\\
305.01	0.00542581401608091\\
306.01	0.00542479890872185\\
307.01	0.0054237630056023\\
308.01	0.00542270589774171\\
309.01	0.00542162716900558\\
310.01	0.00542052639603238\\
311.01	0.00541940314816471\\
312.01	0.00541825698738258\\
313.01	0.00541708746824154\\
314.01	0.00541589413781383\\
315.01	0.00541467653563516\\
316.01	0.00541343419365607\\
317.01	0.00541216663619915\\
318.01	0.0054108733799222\\
319.01	0.00540955393378804\\
320.01	0.0054082077990423\\
321.01	0.0054068344691983\\
322.01	0.00540543343003169\\
323.01	0.00540400415958425\\
324.01	0.0054025461281779\\
325.01	0.0054010587984407\\
326.01	0.00539954162534438\\
327.01	0.00539799405625615\\
328.01	0.00539641553100435\\
329.01	0.00539480548195996\\
330.01	0.00539316333413541\\
331.01	0.00539148850530168\\
332.01	0.00538978040612542\\
333.01	0.0053880384403271\\
334.01	0.00538626200486351\\
335.01	0.00538445049013432\\
336.01	0.00538260328021608\\
337.01	0.00538071975312591\\
338.01	0.0053787992811161\\
339.01	0.00537684123100312\\
340.01	0.00537484496453274\\
341.01	0.00537280983878454\\
342.01	0.00537073520661919\\
343.01	0.00536862041717076\\
344.01	0.0053664648163883\\
345.01	0.00536426774762926\\
346.01	0.00536202855230991\\
347.01	0.00535974657061605\\
348.01	0.00535742114227824\\
349.01	0.00535505160741688\\
350.01	0.0053526373074612\\
351.01	0.00535017758614789\\
352.01	0.0053476717906051\\
353.01	0.00534511927252624\\
354.01	0.00534251938944185\\
355.01	0.00533987150609316\\
356.01	0.00533717499591671\\
357.01	0.0053344292426446\\
358.01	0.00533163364202873\\
359.01	0.00532878760369663\\
360.01	0.00532589055314491\\
361.01	0.00532294193387944\\
362.01	0.00531994120970843\\
363.01	0.00531688786719797\\
364.01	0.00531378141829418\\
365.01	0.00531062140312217\\
366.01	0.0053074073929661\\
367.01	0.00530413899343624\\
368.01	0.00530081584782815\\
369.01	0.00529743764067618\\
370.01	0.00529400410150313\\
371.01	0.00529051500876392\\
372.01	0.0052869701939804\\
373.01	0.00528336954605725\\
374.01	0.0052797130157684\\
375.01	0.00527600062039373\\
376.01	0.00527223244848314\\
377.01	0.00526840866471069\\
378.01	0.00526452951477912\\
379.01	0.00526059533031518\\
380.01	0.00525660653368822\\
381.01	0.00525256364266209\\
382.01	0.00524846727477379\\
383.01	0.00524431815130667\\
384.01	0.00524011710069707\\
385.01	0.00523586506118336\\
386.01	0.00523156308246611\\
387.01	0.00522721232610426\\
388.01	0.00522281406432562\\
389.01	0.00521836967687051\\
390.01	0.00521388064542686\\
391.01	0.00520934854514482\\
392.01	0.00520477503264535\\
393.01	0.00520016182985849\\
394.01	0.0051955107029452\\
395.01	0.0051908234354836\\
396.01	0.00518610179502812\\
397.01	0.00518134749210581\\
398.01	0.00517656213069732\\
399.01	0.00517174714929188\\
400.01	0.00516690375172526\\
401.01	0.00516203282725931\\
402.01	0.00515713485978329\\
403.01	0.00515220982670024\\
404.01	0.00514725708910238\\
405.01	0.00514227527638556\\
406.01	0.00513726217069819\\
407.01	0.00513221459982413\\
408.01	0.00512712835162468\\
409.01	0.00512199812947355\\
410.01	0.00511681757686981\\
411.01	0.00511157941144495\\
412.01	0.00510627572504719\\
413.01	0.00510089853303709\\
414.01	0.00509544215081351\\
415.01	0.00508990559050473\\
416.01	0.00508428859435006\\
417.01	0.00507859095793022\\
418.01	0.0050728125332189\\
419.01	0.00506695323167737\\
420.01	0.00506101302737739\\
421.01	0.00505499196013406\\
422.01	0.00504889013862831\\
423.01	0.00504270774349266\\
424.01	0.00503644503033259\\
425.01	0.00503010233265198\\
426.01	0.00502368006464182\\
427.01	0.0050171787237906\\
428.01	0.00501059889326493\\
429.01	0.00500394124400285\\
430.01	0.00499720653645617\\
431.01	0.00499039562190678\\
432.01	0.00498350944327549\\
433.01	0.00497654903533095\\
434.01	0.00496951552419494\\
435.01	0.00496241012603001\\
436.01	0.00495523414478367\\
437.01	0.00494798896884916\\
438.01	0.00494067606649338\\
439.01	0.0049332969798862\\
440.01	0.00492585331755582\\
441.01	0.00491834674508316\\
442.01	0.00491077897383633\\
443.01	0.00490315174754159\\
444.01	0.00489546682648113\\
445.01	0.00488772596910899\\
446.01	0.00487993091088409\\
447.01	0.00487208334013407\\
448.01	0.00486418487078903\\
449.01	0.00485623701186795\\
450.01	0.00484824113365184\\
451.01	0.00484019843056258\\
452.01	0.00483210988086774\\
453.01	0.00482397620346859\\
454.01	0.0048157978122043\\
455.01	0.00480757476831897\\
456.01	0.00479930673200534\\
457.01	0.00479099291426222\\
458.01	0.00478263203068424\\
459.01	0.00477422225924417\\
460.01	0.00476576120462847\\
461.01	0.00475724587222647\\
462.01	0.00474867265544567\\
463.01	0.00474003734060925\\
464.01	0.0047313351341798\\
465.01	0.00472256071731719\\
466.01	0.00471370833270468\\
467.01	0.00470477190790311\\
468.01	0.00469574521786123\\
469.01	0.00468662208610703\\
470.01	0.00467739661883009\\
471.01	0.00466806345757149\\
472.01	0.00465861802321931\\
473.01	0.00464905670308286\\
474.01	0.00463937683481917\\
475.01	0.00462957605184824\\
476.01	0.00461965177964405\\
477.01	0.00460960118805203\\
478.01	0.00459942117493768\\
479.01	0.00458910835096083\\
480.01	0.00457865902595065\\
481.01	0.00456806919740823\\
482.01	0.0045573345417218\\
483.01	0.00454645040872085\\
484.01	0.0045354118202245\\
485.01	0.00452421347324612\\
486.01	0.00451284974848832\\
487.01	0.00450131472469421\\
488.01	0.00448960219928953\\
489.01	0.00447770571555365\\
490.01	0.00446561859626462\\
491.01	0.00445333398337029\\
492.01	0.00444084488272215\\
493.01	0.00442814421226662\\
494.01	0.00441522485132207\\
495.01	0.00440207968771001\\
496.01	0.00438870165860932\\
497.01	0.00437508378018656\\
498.01	0.00436121916053661\\
499.01	0.00434710099056351\\
500.01	0.00433272250870228\\
501.01	0.00431807693864201\\
502.01	0.00430315740715636\\
503.01	0.00428795688086955\\
504.01	0.00427246815788195\\
505.01	0.00425668387727377\\
506.01	0.00424059652999392\\
507.01	0.00422419847041546\\
508.01	0.0042074819281634\\
509.01	0.00419043901973585\\
510.01	0.00417306175936144\\
511.01	0.00415534206847596\\
512.01	0.00413727178316525\\
513.01	0.00411884265894367\\
514.01	0.00410004637230826\\
515.01	0.00408087451867664\\
516.01	0.00406131860657965\\
517.01	0.00404137004835897\\
518.01	0.00402102014811284\\
519.01	0.00400026008820573\\
520.01	0.00397908091623087\\
521.01	0.00395747353471852\\
522.01	0.00393542869566511\\
523.01	0.00391293699939012\\
524.01	0.00388998889433544\\
525.01	0.00386657467628649\\
526.01	0.00384268448686651\\
527.01	0.00381830831127323\\
528.01	0.00379343597527271\\
529.01	0.00376805714151781\\
530.01	0.00374216130531905\\
531.01	0.0037157377900536\\
532.01	0.00368877574245122\\
533.01	0.00366126412803118\\
534.01	0.00363319172696691\\
535.01	0.00360454713061691\\
536.01	0.00357531873886363\\
537.01	0.00354549475825077\\
538.01	0.00351506320072506\\
539.01	0.0034840118827168\\
540.01	0.00345232842446228\\
541.01	0.00342000024963096\\
542.01	0.00338701458535218\\
543.01	0.00335335846274843\\
544.01	0.00331901871809061\\
545.01	0.00328398199469261\\
546.01	0.00324823474566422\\
547.01	0.00321176323763672\\
548.01	0.00317455355556667\\
549.01	0.00313659160871879\\
550.01	0.00309786313792477\\
551.01	0.00305835372422298\\
552.01	0.00301804879900678\\
553.01	0.00297693365584456\\
554.01	0.00293499346416344\\
555.01	0.00289221328500843\\
556.01	0.00284857808910646\\
557.01	0.00280407277748279\\
558.01	0.00275868220489704\\
559.01	0.00271239120638774\\
560.01	0.00266518462723621\\
561.01	0.00261704735668648\\
562.01	0.0025679643657854\\
563.01	0.00251792074973548\\
564.01	0.00246690177518275\\
565.01	0.00241489293289015\\
566.01	0.00236187999627078\\
567.01	0.00230784908627568\\
568.01	0.00225278674314507\\
569.01	0.00219668000553777\\
570.01	0.00213951649754955\\
571.01	0.00208128452410928\\
572.01	0.00202197317520257\\
573.01	0.00196157243930301\\
574.01	0.00190007332628658\\
575.01	0.00183746799995258\\
576.01	0.0017737499200606\\
577.01	0.00170891399350014\\
578.01	0.00164295673381588\\
579.01	0.00157587642778816\\
580.01	0.00150767330708272\\
581.01	0.00143834972208804\\
582.01	0.0013679103139034\\
583.01	0.00129636217895157\\
584.01	0.0012237150187842\\
585.01	0.00114998126521834\\
586.01	0.00107517616785102\\
587.01	0.000999317827082614\\
588.01	0.000922427150825461\\
589.01	0.000844527706822454\\
590.01	0.000765645434626115\\
591.01	0.000685808171383964\\
592.01	0.000605044933136151\\
593.01	0.000523384877725838\\
594.01	0.000440855855864791\\
595.01	0.00035748243240945\\
596.01	0.00027328322926679\\
597.01	0.000188267403047507\\
598.01	0.000102430022728519\\
599.01	3.18230442563731e-05\\
599.02	3.12770957175551e-05\\
599.03	3.07343620567866e-05\\
599.04	3.0194875217571e-05\\
599.05	2.96586674565693e-05\\
599.06	2.91257713466754e-05\\
599.07	2.85962197801391e-05\\
599.08	2.80700459716916e-05\\
599.09	2.75472834617447e-05\\
599.1	2.70279661195773e-05\\
599.11	2.65121281465865e-05\\
599.12	2.5999804079543e-05\\
599.13	2.54910287939124e-05\\
599.14	2.49858375071695e-05\\
599.15	2.4484265782181e-05\\
599.16	2.39863495305973e-05\\
599.17	2.34921250162837e-05\\
599.18	2.30016288588035e-05\\
599.19	2.25148980369013e-05\\
599.2	2.20319698920508e-05\\
599.21	2.1552882132023e-05\\
599.22	2.10776728344908e-05\\
599.23	2.06063804506721e-05\\
599.24	2.01390438090109e-05\\
599.25	1.96757021188858e-05\\
599.26	1.92163949743647e-05\\
599.27	1.87611623579855e-05\\
599.28	1.83100446445959e-05\\
599.29	1.78630826051952e-05\\
599.3	1.74203174108517e-05\\
599.31	1.69817906366353e-05\\
599.32	1.65475442655931e-05\\
599.33	1.61176206927693e-05\\
599.34	1.56920627292622e-05\\
599.35	1.52709136063186e-05\\
599.36	1.48542169794725e-05\\
599.37	1.4442016932719e-05\\
599.38	1.40343579827368e-05\\
599.39	1.36312859119591e-05\\
599.4	1.32328499381877e-05\\
599.41	1.28390997671621e-05\\
599.42	1.2450085597351e-05\\
599.43	1.20658581248094e-05\\
599.44	1.16864685480531e-05\\
599.45	1.13119685730082e-05\\
599.46	1.09424104179929e-05\\
599.47	1.05778468187639e-05\\
599.48	1.02183310335905e-05\\
599.49	9.86391684839293e-06\\
599.5	9.51465858193938e-06\\
599.51	9.17061109107289e-06\\
599.52	8.83182977599352e-06\\
599.53	8.4983705856221e-06\\
599.54	8.1702900229675e-06\\
599.55	7.84764515058753e-06\\
599.56	7.53049359608279e-06\\
599.57	7.21889355765649e-06\\
599.58	6.91290380971064e-06\\
599.59	6.61258370851688e-06\\
599.6	6.31799319793756e-06\\
599.61	6.02919281519031e-06\\
599.62	5.74624369668701e-06\\
599.63	5.46920758391981e-06\\
599.64	5.19814682940593e-06\\
599.65	4.93312440269685e-06\\
599.66	4.67420389643237e-06\\
599.67	4.42144953247646e-06\\
599.68	4.17492616808755e-06\\
599.69	3.9346993021671e-06\\
599.7	3.70083508156871e-06\\
599.71	3.47340030746116e-06\\
599.72	3.25246244175723e-06\\
599.73	3.03808961360161e-06\\
599.74	2.83035062593855e-06\\
599.75	2.62931496212288e-06\\
599.76	2.43505279260738e-06\\
599.77	2.24763498169606e-06\\
599.78	2.06713309435641e-06\\
599.79	1.89361940310974e-06\\
599.8	1.72716689498045e-06\\
599.81	1.56784927850782e-06\\
599.82	1.41574099085488e-06\\
599.83	1.27091720493813e-06\\
599.84	1.13345383668563e-06\\
599.85	1.00342755231589e-06\\
599.86	8.8091577571392e-07\\
599.87	7.65996695880136e-07\\
599.88	6.58749274441706e-07\\
599.89	5.59253253243699e-07\\
599.9	4.67589162013102e-07\\
599.91	3.83838326099145e-07\\
599.92	3.0808287429171e-07\\
599.93	2.40405746710845e-07\\
599.94	1.80890702777825e-07\\
599.95	1.2962232925906e-07\\
599.96	8.66860484019516e-08\\
599.97	5.21681261331924e-08\\
599.98	2.61556803542173e-08\\
599.99	8.73668930083393e-09\\
600	0\\
};
\addplot [color=red,solid,forget plot]
  table[row sep=crcr]{%
0.01	0.00524634702484204\\
1.01	0.00524634538167428\\
2.01	0.00524634370419065\\
3.01	0.00524634199167481\\
4.01	0.00524634024339566\\
5.01	0.00524633845860666\\
6.01	0.00524633663654615\\
7.01	0.00524633477643592\\
8.01	0.00524633287748189\\
9.01	0.0052463309388734\\
10.01	0.00524632895978309\\
11.01	0.00524632693936595\\
12.01	0.00524632487675974\\
13.01	0.00524632277108397\\
14.01	0.0052463206214401\\
15.01	0.00524631842691058\\
16.01	0.00524631618655892\\
17.01	0.00524631389942906\\
18.01	0.00524631156454481\\
19.01	0.00524630918091006\\
20.01	0.00524630674750763\\
21.01	0.00524630426329937\\
22.01	0.00524630172722527\\
23.01	0.00524629913820358\\
24.01	0.0052462964951298\\
25.01	0.00524629379687638\\
26.01	0.00524629104229229\\
27.01	0.00524628823020258\\
28.01	0.00524628535940793\\
29.01	0.00524628242868409\\
30.01	0.00524627943678111\\
31.01	0.00524627638242321\\
32.01	0.00524627326430783\\
33.01	0.00524627008110561\\
34.01	0.00524626683145922\\
35.01	0.00524626351398313\\
36.01	0.00524626012726317\\
37.01	0.00524625666985588\\
38.01	0.00524625314028707\\
39.01	0.00524624953705275\\
40.01	0.00524624585861699\\
41.01	0.00524624210341223\\
42.01	0.0052462382698383\\
43.01	0.0052462343562615\\
44.01	0.00524623036101445\\
45.01	0.00524622628239498\\
46.01	0.00524622211866539\\
47.01	0.00524621786805216\\
48.01	0.0052462135287446\\
49.01	0.0052462090988945\\
50.01	0.00524620457661531\\
51.01	0.00524619995998124\\
52.01	0.00524619524702633\\
53.01	0.00524619043574392\\
54.01	0.00524618552408546\\
55.01	0.00524618050996003\\
56.01	0.00524617539123337\\
57.01	0.00524617016572659\\
58.01	0.0052461648312158\\
59.01	0.00524615938543075\\
60.01	0.00524615382605415\\
61.01	0.00524614815072038\\
62.01	0.00524614235701494\\
63.01	0.0052461364424732\\
64.01	0.00524613040457932\\
65.01	0.0052461242407651\\
66.01	0.00524611794840926\\
67.01	0.00524611152483606\\
68.01	0.00524610496731396\\
69.01	0.00524609827305526\\
70.01	0.00524609143921402\\
71.01	0.0052460844628854\\
72.01	0.00524607734110442\\
73.01	0.00524607007084434\\
74.01	0.00524606264901603\\
75.01	0.00524605507246604\\
76.01	0.00524604733797553\\
77.01	0.00524603944225902\\
78.01	0.00524603138196313\\
79.01	0.00524602315366473\\
80.01	0.0052460147538698\\
81.01	0.00524600617901195\\
82.01	0.00524599742545109\\
83.01	0.00524598848947141\\
84.01	0.00524597936728023\\
85.01	0.00524597005500673\\
86.01	0.00524596054869953\\
87.01	0.00524595084432566\\
88.01	0.00524594093776854\\
89.01	0.00524593082482648\\
90.01	0.00524592050121083\\
91.01	0.0052459099625443\\
92.01	0.00524589920435879\\
93.01	0.00524588822209397\\
94.01	0.00524587701109538\\
95.01	0.00524586556661231\\
96.01	0.00524585388379547\\
97.01	0.00524584195769568\\
98.01	0.00524582978326137\\
99.01	0.00524581735533652\\
100.01	0.00524580466865875\\
101.01	0.00524579171785691\\
102.01	0.00524577849744897\\
103.01	0.00524576500183954\\
104.01	0.00524575122531796\\
105.01	0.00524573716205573\\
106.01	0.0052457228061036\\
107.01	0.00524570815139027\\
108.01	0.00524569319171874\\
109.01	0.00524567792076409\\
110.01	0.00524566233207136\\
111.01	0.00524564641905242\\
112.01	0.00524563017498314\\
113.01	0.00524561359300103\\
114.01	0.00524559666610192\\
115.01	0.00524557938713776\\
116.01	0.00524556174881299\\
117.01	0.00524554374368194\\
118.01	0.00524552536414564\\
119.01	0.00524550660244865\\
120.01	0.00524548745067598\\
121.01	0.00524546790074993\\
122.01	0.00524544794442652\\
123.01	0.00524542757329218\\
124.01	0.0052454067787606\\
125.01	0.00524538555206882\\
126.01	0.00524536388427362\\
127.01	0.0052453417662487\\
128.01	0.00524531918867973\\
129.01	0.00524529614206134\\
130.01	0.00524527261669292\\
131.01	0.00524524860267488\\
132.01	0.0052452240899045\\
133.01	0.00524519906807203\\
134.01	0.00524517352665601\\
135.01	0.0052451474549193\\
136.01	0.00524512084190489\\
137.01	0.00524509367643101\\
138.01	0.00524506594708726\\
139.01	0.00524503764222927\\
140.01	0.00524500874997448\\
141.01	0.00524497925819697\\
142.01	0.00524494915452292\\
143.01	0.00524491842632542\\
144.01	0.00524488706071916\\
145.01	0.00524485504455592\\
146.01	0.00524482236441842\\
147.01	0.00524478900661545\\
148.01	0.00524475495717632\\
149.01	0.00524472020184529\\
150.01	0.00524468472607508\\
151.01	0.00524464851502255\\
152.01	0.00524461155354148\\
153.01	0.0052445738261769\\
154.01	0.00524453531715894\\
155.01	0.00524449601039674\\
156.01	0.00524445588947163\\
157.01	0.0052444149376306\\
158.01	0.00524437313778031\\
159.01	0.00524433047247939\\
160.01	0.00524428692393208\\
161.01	0.00524424247398101\\
162.01	0.00524419710409988\\
163.01	0.00524415079538634\\
164.01	0.00524410352855446\\
165.01	0.00524405528392705\\
166.01	0.00524400604142793\\
167.01	0.00524395578057405\\
168.01	0.0052439044804675\\
169.01	0.00524385211978692\\
170.01	0.00524379867677972\\
171.01	0.00524374412925308\\
172.01	0.00524368845456543\\
173.01	0.00524363162961788\\
174.01	0.00524357363084466\\
175.01	0.00524351443420435\\
176.01	0.00524345401517028\\
177.01	0.00524339234872135\\
178.01	0.00524332940933197\\
179.01	0.00524326517096204\\
180.01	0.00524319960704773\\
181.01	0.00524313269049009\\
182.01	0.00524306439364537\\
183.01	0.00524299468831421\\
184.01	0.00524292354573063\\
185.01	0.00524285093655134\\
186.01	0.00524277683084407\\
187.01	0.0052427011980764\\
188.01	0.00524262400710397\\
189.01	0.00524254522615907\\
190.01	0.00524246482283786\\
191.01	0.00524238276408858\\
192.01	0.00524229901619896\\
193.01	0.00524221354478338\\
194.01	0.00524212631476993\\
195.01	0.00524203729038728\\
196.01	0.00524194643515123\\
197.01	0.00524185371185099\\
198.01	0.00524175908253533\\
199.01	0.00524166250849836\\
200.01	0.00524156395026513\\
201.01	0.00524146336757735\\
202.01	0.00524136071937776\\
203.01	0.00524125596379558\\
204.01	0.00524114905813088\\
205.01	0.00524103995883895\\
206.01	0.00524092862151417\\
207.01	0.00524081500087397\\
208.01	0.00524069905074223\\
209.01	0.00524058072403255\\
210.01	0.00524045997273106\\
211.01	0.00524033674787933\\
212.01	0.00524021099955651\\
213.01	0.0052400826768615\\
214.01	0.00523995172789479\\
215.01	0.00523981809973951\\
216.01	0.00523968173844319\\
217.01	0.00523954258899855\\
218.01	0.00523940059532382\\
219.01	0.00523925570024301\\
220.01	0.00523910784546591\\
221.01	0.00523895697156793\\
222.01	0.00523880301796943\\
223.01	0.00523864592291425\\
224.01	0.00523848562344873\\
225.01	0.00523832205540018\\
226.01	0.00523815515335461\\
227.01	0.00523798485063467\\
228.01	0.00523781107927687\\
229.01	0.00523763377000877\\
230.01	0.00523745285222539\\
231.01	0.00523726825396603\\
232.01	0.00523707990188999\\
233.01	0.00523688772125235\\
234.01	0.0052366916358793\\
235.01	0.00523649156814334\\
236.01	0.00523628743893765\\
237.01	0.00523607916765078\\
238.01	0.00523586667214054\\
239.01	0.00523564986870783\\
240.01	0.00523542867206966\\
241.01	0.00523520299533229\\
242.01	0.00523497274996426\\
243.01	0.00523473784576858\\
244.01	0.00523449819085457\\
245.01	0.00523425369160987\\
246.01	0.00523400425267175\\
247.01	0.00523374977689813\\
248.01	0.00523349016533854\\
249.01	0.00523322531720449\\
250.01	0.00523295512983983\\
251.01	0.00523267949869064\\
252.01	0.00523239831727478\\
253.01	0.0052321114771511\\
254.01	0.0052318188678893\\
255.01	0.00523152037703769\\
256.01	0.00523121589009264\\
257.01	0.00523090529046706\\
258.01	0.00523058845945782\\
259.01	0.00523026527621435\\
260.01	0.00522993561770634\\
261.01	0.00522959935869157\\
262.01	0.00522925637168265\\
263.01	0.00522890652691559\\
264.01	0.00522854969231577\\
265.01	0.00522818573346614\\
266.01	0.00522781451357373\\
267.01	0.0052274358934371\\
268.01	0.00522704973141289\\
269.01	0.00522665588338346\\
270.01	0.00522625420272365\\
271.01	0.00522584454026808\\
272.01	0.0052254267442784\\
273.01	0.00522500066041105\\
274.01	0.0052245661316846\\
275.01	0.00522412299844759\\
276.01	0.00522367109834689\\
277.01	0.00522321026629624\\
278.01	0.00522274033444507\\
279.01	0.00522226113214746\\
280.01	0.00522177248593254\\
281.01	0.00522127421947445\\
282.01	0.00522076615356294\\
283.01	0.00522024810607567\\
284.01	0.00521971989194984\\
285.01	0.00521918132315574\\
286.01	0.00521863220867095\\
287.01	0.0052180723544547\\
288.01	0.00521750156342459\\
289.01	0.00521691963543351\\
290.01	0.00521632636724819\\
291.01	0.00521572155252897\\
292.01	0.00521510498181107\\
293.01	0.00521447644248759\\
294.01	0.00521383571879398\\
295.01	0.00521318259179452\\
296.01	0.00521251683937033\\
297.01	0.00521183823621045\\
298.01	0.00521114655380373\\
299.01	0.00521044156043415\\
300.01	0.00520972302117921\\
301.01	0.00520899069790897\\
302.01	0.00520824434929016\\
303.01	0.00520748373079238\\
304.01	0.00520670859469767\\
305.01	0.00520591869011355\\
306.01	0.00520511376299\\
307.01	0.00520429355614022\\
308.01	0.00520345780926567\\
309.01	0.00520260625898526\\
310.01	0.00520173863886974\\
311.01	0.00520085467948036\\
312.01	0.00519995410841374\\
313.01	0.00519903665035063\\
314.01	0.00519810202711284\\
315.01	0.00519714995772436\\
316.01	0.00519618015847998\\
317.01	0.00519519234302001\\
318.01	0.00519418622241354\\
319.01	0.00519316150524767\\
320.01	0.00519211789772595\\
321.01	0.00519105510377478\\
322.01	0.00518997282515848\\
323.01	0.00518887076160365\\
324.01	0.00518774861093347\\
325.01	0.00518660606921215\\
326.01	0.00518544283089948\\
327.01	0.0051842585890171\\
328.01	0.00518305303532577\\
329.01	0.0051818258605159\\
330.01	0.00518057675440898\\
331.01	0.00517930540617347\\
332.01	0.00517801150455399\\
333.01	0.00517669473811432\\
334.01	0.00517535479549591\\
335.01	0.00517399136569088\\
336.01	0.00517260413833114\\
337.01	0.00517119280399358\\
338.01	0.00516975705452242\\
339.01	0.00516829658336771\\
340.01	0.00516681108594328\\
341.01	0.00516530026000083\\
342.01	0.00516376380602336\\
343.01	0.00516220142763736\\
344.01	0.00516061283204347\\
345.01	0.0051589977304664\\
346.01	0.00515735583862371\\
347.01	0.00515568687721378\\
348.01	0.00515399057242243\\
349.01	0.00515226665644796\\
350.01	0.00515051486804406\\
351.01	0.00514873495308003\\
352.01	0.00514692666511627\\
353.01	0.00514508976599623\\
354.01	0.0051432240264493\\
355.01	0.00514132922670752\\
356.01	0.00513940515712955\\
357.01	0.00513745161883129\\
358.01	0.00513546842431997\\
359.01	0.00513345539812616\\
360.01	0.00513141237743133\\
361.01	0.00512933921268395\\
362.01	0.00512723576819906\\
363.01	0.00512510192273381\\
364.01	0.00512293757003164\\
365.01	0.00512074261932499\\
366.01	0.00511851699578688\\
367.01	0.00511626064092056\\
368.01	0.00511397351287178\\
369.01	0.00511165558665115\\
370.01	0.00510930685424922\\
371.01	0.00510692732462547\\
372.01	0.00510451702355187\\
373.01	0.00510207599328724\\
374.01	0.00509960429205898\\
375.01	0.00509710199332412\\
376.01	0.00509456918477933\\
377.01	0.00509200596709101\\
378.01	0.00508941245230719\\
379.01	0.00508678876191842\\
380.01	0.00508413502452625\\
381.01	0.00508145137308095\\
382.01	0.00507873794164737\\
383.01	0.00507599486165573\\
384.01	0.00507322225759954\\
385.01	0.00507042024214\\
386.01	0.00506758891058257\\
387.01	0.00506472833469952\\
388.01	0.00506183855587652\\
389.01	0.00505891957757931\\
390.01	0.00505597135715017\\
391.01	0.00505299379696817\\
392.01	0.00504998673503601\\
393.01	0.00504694993509147\\
394.01	0.00504388307639118\\
395.01	0.00504078574336494\\
396.01	0.00503765741541047\\
397.01	0.00503449745717789\\
398.01	0.00503130510978448\\
399.01	0.00502807948350922\\
400.01	0.00502481955263314\\
401.01	0.00502152415321797\\
402.01	0.00501819198474338\\
403.01	0.00501482161663915\\
404.01	0.0050114115008428\\
405.01	0.00500795999154164\\
406.01	0.00500446537320313\\
407.01	0.00500092589777272\\
408.01	0.00499733983145012\\
409.01	0.00499370551062219\\
410.01	0.00499002140512592\\
411.01	0.00498628618482141\\
412.01	0.00498249878208409\\
413.01	0.00497865843779457\\
414.01	0.00497476469135668\\
415.01	0.00497081721153147\\
416.01	0.00496681568223637\\
417.01	0.00496275979613698\\
418.01	0.00495864925421746\\
419.01	0.00495448376522575\\
420.01	0.00495026304497878\\
421.01	0.00494598681551536\\
422.01	0.00494165480407954\\
423.01	0.00493726674192088\\
424.01	0.00493282236289352\\
425.01	0.00492832140183725\\
426.01	0.00492376359272367\\
427.01	0.00491914866654613\\
428.01	0.0049144763489378\\
429.01	0.00490974635749719\\
430.01	0.0049049583988014\\
431.01	0.00490011216509085\\
432.01	0.00489520733060504\\
433.01	0.00489024354755414\\
434.01	0.00488522044171163\\
435.01	0.00488013760761407\\
436.01	0.00487499460336023\\
437.01	0.00486979094500281\\
438.01	0.00486452610053278\\
439.01	0.00485919948346191\\
440.01	0.00485381044601682\\
441.01	0.00484835827196602\\
442.01	0.00484284216911223\\
443.01	0.00483726126149552\\
444.01	0.00483161458136402\\
445.01	0.00482590106098931\\
446.01	0.00482011952441736\\
447.01	0.0048142686792673\\
448.01	0.00480834710871583\\
449.01	0.00480235326382089\\
450.01	0.00479628545637138\\
451.01	0.00479014185246842\\
452.01	0.00478392046707218\\
453.01	0.00477761915976888\\
454.01	0.00477123563203365\\
455.01	0.00476476742627694\\
456.01	0.00475821192696775\\
457.01	0.00475156636411687\\
458.01	0.00474482781937998\\
459.01	0.00473799323499074\\
460.01	0.00473105942565689\\
461.01	0.00472402309344056\\
462.01	0.00471688084548702\\
463.01	0.00470962921426139\\
464.01	0.0047022646796878\\
465.01	0.00469478369226967\\
466.01	0.00468718269589198\\
467.01	0.00467945814860222\\
468.01	0.00467160653924801\\
469.01	0.00466362439749715\\
470.01	0.00465550829456611\\
471.01	0.00464725483211195\\
472.01	0.00463886061745229\\
473.01	0.00463032222497244\\
474.01	0.00462163614761239\\
475.01	0.00461279875818079\\
476.01	0.00460380629909003\\
477.01	0.00459465488160419\\
478.01	0.00458534048603782\\
479.01	0.00457585896272411\\
480.01	0.00456620603379292\\
481.01	0.00455637729577941\\
482.01	0.00454636822306456\\
483.01	0.00453617417212127\\
484.01	0.00452579038650785\\
485.01	0.00451521200251422\\
486.01	0.00450443405532398\\
487.01	0.0044934514855084\\
488.01	0.0044822591456219\\
489.01	0.00447085180662029\\
490.01	0.00445922416378062\\
491.01	0.00444737084176718\\
492.01	0.00443528639847168\\
493.01	0.00442296532726427\\
494.01	0.0044104020573332\\
495.01	0.00439759095188005\\
496.01	0.00438452630407865\\
497.01	0.00437120233091795\\
498.01	0.00435761316531334\\
499.01	0.00434375284719876\\
500.01	0.00432961531465032\\
501.01	0.00431519439636444\\
502.01	0.00430048380688169\\
503.01	0.00428547714512883\\
504.01	0.00427016789463693\\
505.01	0.00425454942377817\\
506.01	0.00423861498570822\\
507.01	0.00422235771793727\\
508.01	0.00420577064146486\\
509.01	0.00418884665942247\\
510.01	0.00417157855518518\\
511.01	0.00415395898993742\\
512.01	0.00413598049970358\\
513.01	0.00411763549188932\\
514.01	0.00409891624141305\\
515.01	0.00407981488654231\\
516.01	0.00406032342458033\\
517.01	0.00404043370756525\\
518.01	0.00402013743814512\\
519.01	0.00399942616576509\\
520.01	0.00397829128324327\\
521.01	0.00395672402371646\\
522.01	0.00393471545782475\\
523.01	0.00391225649092619\\
524.01	0.00388933786022006\\
525.01	0.00386595013177418\\
526.01	0.0038420836974821\\
527.01	0.003817728771985\\
528.01	0.00379287538959846\\
529.01	0.00376751340128798\\
530.01	0.00374163247173836\\
531.01	0.00371522207656307\\
532.01	0.00368827149969528\\
533.01	0.00366076983099453\\
534.01	0.00363270596409685\\
535.01	0.00360406859452609\\
536.01	0.00357484621807641\\
537.01	0.00354502712947658\\
538.01	0.00351459942135183\\
539.01	0.00348355098351638\\
540.01	0.00345186950264478\\
541.01	0.00341954246237786\\
542.01	0.00338655714392597\\
543.01	0.00335290062723492\\
544.01	0.00331855979278612\\
545.01	0.00328352132410759\\
546.01	0.00324777171107945\\
547.01	0.00321129725412398\\
548.01	0.00317408406938027\\
549.01	0.0031361180949736\\
550.01	0.00309738509850105\\
551.01	0.00305787068587207\\
552.01	0.00301756031165513\\
553.01	0.00297643929110013\\
554.01	0.00293449281402208\\
555.01	0.00289170596074975\\
556.01	0.00284806372036218\\
557.01	0.00280355101145738\\
558.01	0.00275815270571891\\
559.01	0.0027118536545707\\
560.01	0.00266463871923601\\
561.01	0.00261649280454222\\
562.01	0.00256740089684071\\
563.01	0.00251734810643931\\
564.01	0.00246631971497154\\
565.01	0.00241430122815276\\
566.01	0.00236127843439859\\
567.01	0.00230723746979893\\
568.01	0.00225216488995792\\
569.01	0.00219604774921472\\
570.01	0.00213887368775576\\
571.01	0.00208063102710875\\
572.01	0.00202130887446654\\
573.01	0.00196089723622006\\
574.01	0.00189938714097235\\
575.01	0.00183677077215304\\
576.01	0.00177304161013619\\
577.01	0.00170819458346989\\
578.01	0.00164222622842918\\
579.01	0.00157513485557815\\
580.01	0.0015069207213362\\
581.01	0.00143758620164401\\
582.01	0.0013671359636618\\
583.01	0.00129557712993561\\
584.01	0.0012229194275526\\
585.01	0.00114917531236232\\
586.01	0.00107436005523633\\
587.01	0.000998491773401186\\
588.01	0.000921591384903622\\
589.01	0.000843682457983908\\
590.01	0.00076479091922294\\
591.01	0.00068494457437703\\
592.01	0.000604172383317383\\
593.01	0.000522503414813216\\
594.01	0.000439965387250307\\
595.01	0.000356582676778022\\
596.01	0.000272373643602759\\
597.01	0.000187347088672941\\
598.01	0.000101497604931054\\
599.01	3.18230442563749e-05\\
599.02	3.12770957175551e-05\\
599.03	3.07343620567849e-05\\
599.04	3.0194875217571e-05\\
599.05	2.96586674565693e-05\\
599.06	2.91257713466771e-05\\
599.07	2.85962197801391e-05\\
599.08	2.80700459716916e-05\\
599.09	2.75472834617447e-05\\
599.1	2.70279661195791e-05\\
599.11	2.65121281465865e-05\\
599.12	2.59998040795448e-05\\
599.13	2.54910287939142e-05\\
599.14	2.49858375071712e-05\\
599.15	2.44842657821827e-05\\
599.16	2.39863495305973e-05\\
599.17	2.34921250162855e-05\\
599.18	2.30016288588035e-05\\
599.19	2.25148980369013e-05\\
599.2	2.20319698920526e-05\\
599.21	2.1552882132023e-05\\
599.22	2.10776728344891e-05\\
599.23	2.06063804506738e-05\\
599.24	2.01390438090109e-05\\
599.25	1.96757021188876e-05\\
599.26	1.92163949743647e-05\\
599.27	1.87611623579872e-05\\
599.28	1.83100446445959e-05\\
599.29	1.78630826051952e-05\\
599.3	1.74203174108517e-05\\
599.31	1.69817906366335e-05\\
599.32	1.65475442655914e-05\\
599.33	1.61176206927693e-05\\
599.34	1.56920627292639e-05\\
599.35	1.52709136063203e-05\\
599.36	1.48542169794742e-05\\
599.37	1.4442016932719e-05\\
599.38	1.40343579827385e-05\\
599.39	1.36312859119591e-05\\
599.4	1.32328499381877e-05\\
599.41	1.28390997671604e-05\\
599.42	1.24500855973528e-05\\
599.43	1.20658581248094e-05\\
599.44	1.16864685480531e-05\\
599.45	1.13119685730065e-05\\
599.46	1.09424104179929e-05\\
599.47	1.05778468187639e-05\\
599.48	1.02183310335888e-05\\
599.49	9.86391684839293e-06\\
599.5	9.51465858194112e-06\\
599.51	9.17061109107116e-06\\
599.52	8.83182977599525e-06\\
599.53	8.4983705856221e-06\\
599.54	8.1702900229675e-06\\
599.55	7.84764515058579e-06\\
599.56	7.53049359608453e-06\\
599.57	7.21889355765649e-06\\
599.58	6.91290380970891e-06\\
599.59	6.61258370851861e-06\\
599.6	6.31799319793756e-06\\
599.61	6.02919281519031e-06\\
599.62	5.74624369668701e-06\\
599.63	5.46920758391807e-06\\
599.64	5.19814682940593e-06\\
599.65	4.93312440269685e-06\\
599.66	4.67420389643411e-06\\
599.67	4.42144953247646e-06\\
599.68	4.17492616808582e-06\\
599.69	3.93469930216536e-06\\
599.7	3.70083508156871e-06\\
599.71	3.47340030746116e-06\\
599.72	3.25246244175549e-06\\
599.73	3.03808961360161e-06\\
599.74	2.83035062593855e-06\\
599.75	2.62931496212288e-06\\
599.76	2.43505279260911e-06\\
599.77	2.24763498169606e-06\\
599.78	2.06713309435641e-06\\
599.79	1.89361940310974e-06\\
599.8	1.72716689497872e-06\\
599.81	1.56784927850956e-06\\
599.82	1.41574099085315e-06\\
599.83	1.27091720493813e-06\\
599.84	1.13345383668736e-06\\
599.85	1.00342755231589e-06\\
599.86	8.8091577571392e-07\\
599.87	7.65996695880136e-07\\
599.88	6.58749274441706e-07\\
599.89	5.59253253243699e-07\\
599.9	4.67589162011367e-07\\
599.91	3.83838326099145e-07\\
599.92	3.08082874293444e-07\\
599.93	2.4040574671258e-07\\
599.94	1.80890702777825e-07\\
599.95	1.2962232925906e-07\\
599.96	8.66860484019516e-08\\
599.97	5.21681261349272e-08\\
599.98	2.61556803542173e-08\\
599.99	8.73668929909921e-09\\
600	0\\
};
\addplot [color=mycolor20,solid,forget plot]
  table[row sep=crcr]{%
0.01	0.00513241483564837\\
1.01	0.00513241334424749\\
2.01	0.00513241182204282\\
3.01	0.00513241026839975\\
4.01	0.00513240868267067\\
5.01	0.00513240706419493\\
6.01	0.00513240541229786\\
7.01	0.00513240372629167\\
8.01	0.00513240200547385\\
9.01	0.00513240024912786\\
10.01	0.00513239845652213\\
11.01	0.0051323966269105\\
12.01	0.00513239475953114\\
13.01	0.00513239285360692\\
14.01	0.00513239090834441\\
15.01	0.00513238892293442\\
16.01	0.00513238689655086\\
17.01	0.00513238482835078\\
18.01	0.00513238271747402\\
19.01	0.00513238056304285\\
20.01	0.00513237836416136\\
21.01	0.00513237611991541\\
22.01	0.0051323738293723\\
23.01	0.00513237149158016\\
24.01	0.00513236910556749\\
25.01	0.00513236667034322\\
26.01	0.00513236418489582\\
27.01	0.00513236164819297\\
28.01	0.0051323590591816\\
29.01	0.00513235641678669\\
30.01	0.00513235371991137\\
31.01	0.00513235096743661\\
32.01	0.0051323481582204\\
33.01	0.00513234529109728\\
34.01	0.00513234236487794\\
35.01	0.0051323393783491\\
36.01	0.00513233633027242\\
37.01	0.00513233321938416\\
38.01	0.00513233004439521\\
39.01	0.00513232680398992\\
40.01	0.00513232349682574\\
41.01	0.00513232012153271\\
42.01	0.00513231667671307\\
43.01	0.00513231316094059\\
44.01	0.00513230957275992\\
45.01	0.00513230591068593\\
46.01	0.00513230217320327\\
47.01	0.00513229835876573\\
48.01	0.00513229446579563\\
49.01	0.00513229049268308\\
50.01	0.00513228643778559\\
51.01	0.00513228229942684\\
52.01	0.00513227807589697\\
53.01	0.00513227376545062\\
54.01	0.00513226936630757\\
55.01	0.00513226487665105\\
56.01	0.00513226029462723\\
57.01	0.0051322556183449\\
58.01	0.00513225084587409\\
59.01	0.00513224597524579\\
60.01	0.00513224100445096\\
61.01	0.00513223593143984\\
62.01	0.0051322307541208\\
63.01	0.00513222547035982\\
64.01	0.00513222007797958\\
65.01	0.00513221457475855\\
66.01	0.00513220895843004\\
67.01	0.00513220322668146\\
68.01	0.00513219737715356\\
69.01	0.0051321914074386\\
70.01	0.00513218531508069\\
71.01	0.00513217909757361\\
72.01	0.00513217275236081\\
73.01	0.00513216627683364\\
74.01	0.00513215966833091\\
75.01	0.00513215292413726\\
76.01	0.00513214604148243\\
77.01	0.00513213901754015\\
78.01	0.00513213184942692\\
79.01	0.00513212453420093\\
80.01	0.00513211706886089\\
81.01	0.0051321094503449\\
82.01	0.00513210167552917\\
83.01	0.0051320937412265\\
84.01	0.00513208564418569\\
85.01	0.00513207738108971\\
86.01	0.00513206894855464\\
87.01	0.00513206034312811\\
88.01	0.00513205156128827\\
89.01	0.00513204259944226\\
90.01	0.00513203345392471\\
91.01	0.00513202412099615\\
92.01	0.00513201459684214\\
93.01	0.00513200487757133\\
94.01	0.0051319949592137\\
95.01	0.00513198483771948\\
96.01	0.00513197450895748\\
97.01	0.0051319639687133\\
98.01	0.00513195321268775\\
99.01	0.00513194223649546\\
100.01	0.00513193103566274\\
101.01	0.00513191960562597\\
102.01	0.00513190794172987\\
103.01	0.00513189603922598\\
104.01	0.00513188389327018\\
105.01	0.00513187149892124\\
106.01	0.00513185885113905\\
107.01	0.00513184594478222\\
108.01	0.00513183277460645\\
109.01	0.00513181933526249\\
110.01	0.00513180562129378\\
111.01	0.00513179162713463\\
112.01	0.00513177734710791\\
113.01	0.00513176277542302\\
114.01	0.00513174790617369\\
115.01	0.00513173273333546\\
116.01	0.00513171725076347\\
117.01	0.00513170145219042\\
118.01	0.00513168533122378\\
119.01	0.00513166888134339\\
120.01	0.00513165209589928\\
121.01	0.00513163496810875\\
122.01	0.00513161749105399\\
123.01	0.00513159965767963\\
124.01	0.00513158146078958\\
125.01	0.00513156289304461\\
126.01	0.00513154394695976\\
127.01	0.00513152461490098\\
128.01	0.00513150488908276\\
129.01	0.00513148476156499\\
130.01	0.00513146422424991\\
131.01	0.00513144326887912\\
132.01	0.00513142188703037\\
133.01	0.0051314000701145\\
134.01	0.00513137780937233\\
135.01	0.00513135509587135\\
136.01	0.00513133192050217\\
137.01	0.00513130827397524\\
138.01	0.00513128414681747\\
139.01	0.00513125952936845\\
140.01	0.0051312344117773\\
141.01	0.00513120878399871\\
142.01	0.00513118263578901\\
143.01	0.00513115595670288\\
144.01	0.00513112873608927\\
145.01	0.00513110096308722\\
146.01	0.00513107262662244\\
147.01	0.00513104371540271\\
148.01	0.00513101421791389\\
149.01	0.00513098412241576\\
150.01	0.00513095341693789\\
151.01	0.00513092208927484\\
152.01	0.0051308901269819\\
153.01	0.00513085751737094\\
154.01	0.00513082424750541\\
155.01	0.00513079030419562\\
156.01	0.00513075567399406\\
157.01	0.00513072034319089\\
158.01	0.00513068429780831\\
159.01	0.00513064752359607\\
160.01	0.00513061000602636\\
161.01	0.0051305717302882\\
162.01	0.00513053268128248\\
163.01	0.00513049284361671\\
164.01	0.00513045220159896\\
165.01	0.00513041073923299\\
166.01	0.00513036844021219\\
167.01	0.00513032528791378\\
168.01	0.00513028126539313\\
169.01	0.00513023635537786\\
170.01	0.00513019054026152\\
171.01	0.00513014380209775\\
172.01	0.00513009612259363\\
173.01	0.00513004748310344\\
174.01	0.00512999786462264\\
175.01	0.00512994724778056\\
176.01	0.00512989561283425\\
177.01	0.00512984293966133\\
178.01	0.00512978920775333\\
179.01	0.00512973439620871\\
180.01	0.00512967848372522\\
181.01	0.00512962144859345\\
182.01	0.00512956326868863\\
183.01	0.00512950392146356\\
184.01	0.00512944338394127\\
185.01	0.00512938163270644\\
186.01	0.00512931864389846\\
187.01	0.00512925439320302\\
188.01	0.00512918885584404\\
189.01	0.0051291220065753\\
190.01	0.00512905381967257\\
191.01	0.00512898426892433\\
192.01	0.00512891332762384\\
193.01	0.00512884096856025\\
194.01	0.00512876716400947\\
195.01	0.00512869188572549\\
196.01	0.00512861510493096\\
197.01	0.00512853679230802\\
198.01	0.00512845691798884\\
199.01	0.00512837545154612\\
200.01	0.00512829236198347\\
201.01	0.00512820761772528\\
202.01	0.00512812118660717\\
203.01	0.00512803303586553\\
204.01	0.00512794313212738\\
205.01	0.00512785144140001\\
206.01	0.00512775792906014\\
207.01	0.00512766255984398\\
208.01	0.00512756529783547\\
209.01	0.005127466106456\\
210.01	0.00512736494845286\\
211.01	0.00512726178588829\\
212.01	0.00512715658012785\\
213.01	0.00512704929182884\\
214.01	0.00512693988092886\\
215.01	0.00512682830663378\\
216.01	0.00512671452740574\\
217.01	0.00512659850095092\\
218.01	0.0051264801842074\\
219.01	0.00512635953333281\\
220.01	0.00512623650369184\\
221.01	0.00512611104984295\\
222.01	0.00512598312552607\\
223.01	0.00512585268364942\\
224.01	0.00512571967627659\\
225.01	0.00512558405461285\\
226.01	0.00512544576899208\\
227.01	0.00512530476886323\\
228.01	0.00512516100277649\\
229.01	0.00512501441836955\\
230.01	0.00512486496235384\\
231.01	0.0051247125804999\\
232.01	0.00512455721762403\\
233.01	0.00512439881757345\\
234.01	0.00512423732321212\\
235.01	0.00512407267640616\\
236.01	0.00512390481800906\\
237.01	0.00512373368784713\\
238.01	0.00512355922470507\\
239.01	0.00512338136630993\\
240.01	0.00512320004931743\\
241.01	0.00512301520929611\\
242.01	0.00512282678071231\\
243.01	0.00512263469691465\\
244.01	0.00512243889011949\\
245.01	0.00512223929139481\\
246.01	0.00512203583064516\\
247.01	0.00512182843659607\\
248.01	0.00512161703677903\\
249.01	0.00512140155751532\\
250.01	0.00512118192390098\\
251.01	0.00512095805979098\\
252.01	0.00512072988778389\\
253.01	0.00512049732920648\\
254.01	0.00512026030409788\\
255.01	0.00512001873119458\\
256.01	0.00511977252791487\\
257.01	0.00511952161034348\\
258.01	0.00511926589321649\\
259.01	0.00511900528990631\\
260.01	0.00511873971240635\\
261.01	0.00511846907131616\\
262.01	0.00511819327582711\\
263.01	0.00511791223370725\\
264.01	0.00511762585128748\\
265.01	0.00511733403344705\\
266.01	0.00511703668359962\\
267.01	0.00511673370367922\\
268.01	0.00511642499412766\\
269.01	0.00511611045388039\\
270.01	0.0051157899803543\\
271.01	0.00511546346943497\\
272.01	0.00511513081546445\\
273.01	0.00511479191122941\\
274.01	0.00511444664795006\\
275.01	0.00511409491526913\\
276.01	0.0051137366012415\\
277.01	0.00511337159232413\\
278.01	0.00511299977336665\\
279.01	0.00511262102760302\\
280.01	0.00511223523664279\\
281.01	0.00511184228046374\\
282.01	0.00511144203740521\\
283.01	0.00511103438416184\\
284.01	0.00511061919577816\\
285.01	0.00511019634564426\\
286.01	0.0051097657054915\\
287.01	0.0051093271453905\\
288.01	0.00510888053374866\\
289.01	0.00510842573730956\\
290.01	0.00510796262115269\\
291.01	0.0051074910486952\\
292.01	0.00510701088169404\\
293.01	0.00510652198024906\\
294.01	0.0051060242028085\\
295.01	0.00510551740617429\\
296.01	0.00510500144550995\\
297.01	0.00510447617434908\\
298.01	0.0051039414446056\\
299.01	0.00510339710658558\\
300.01	0.00510284300900028\\
301.01	0.00510227899898119\\
302.01	0.0051017049220967\\
303.01	0.00510112062237015\\
304.01	0.00510052594230007\\
305.01	0.00509992072288226\\
306.01	0.00509930480363314\\
307.01	0.00509867802261618\\
308.01	0.005098040216469\\
309.01	0.00509739122043389\\
310.01	0.00509673086838949\\
311.01	0.00509605899288485\\
312.01	0.00509537542517597\\
313.01	0.0050946799952644\\
314.01	0.00509397253193796\\
315.01	0.00509325286281404\\
316.01	0.00509252081438535\\
317.01	0.0050917762120676\\
318.01	0.00509101888024977\\
319.01	0.00509024864234706\\
320.01	0.00508946532085563\\
321.01	0.00508866873741051\\
322.01	0.00508785871284507\\
323.01	0.00508703506725312\\
324.01	0.00508619762005377\\
325.01	0.00508534619005734\\
326.01	0.00508448059553462\\
327.01	0.00508360065428733\\
328.01	0.00508270618372094\\
329.01	0.0050817970009187\\
330.01	0.00508087292271784\\
331.01	0.00507993376578699\\
332.01	0.00507897934670416\\
333.01	0.00507800948203693\\
334.01	0.00507702398842144\\
335.01	0.00507602268264328\\
336.01	0.00507500538171723\\
337.01	0.00507397190296693\\
338.01	0.00507292206410296\\
339.01	0.00507185568330056\\
340.01	0.00507077257927324\\
341.01	0.00506967257134569\\
342.01	0.00506855547952166\\
343.01	0.00506742112454792\\
344.01	0.00506626932797324\\
345.01	0.00506509991220067\\
346.01	0.00506391270053324\\
347.01	0.00506270751721066\\
348.01	0.0050614841874369\\
349.01	0.00506024253739589\\
350.01	0.00505898239425578\\
351.01	0.00505770358615814\\
352.01	0.00505640594219222\\
353.01	0.00505508929235078\\
354.01	0.00505375346746671\\
355.01	0.00505239829912762\\
356.01	0.00505102361956614\\
357.01	0.00504962926152432\\
358.01	0.00504821505808848\\
359.01	0.00504678084249199\\
360.01	0.00504532644788424\\
361.01	0.00504385170706125\\
362.01	0.00504235645215611\\
363.01	0.00504084051428484\\
364.01	0.00503930372314545\\
365.01	0.00503774590656626\\
366.01	0.00503616689000066\\
367.01	0.00503456649596352\\
368.01	0.00503294454340806\\
369.01	0.0050313008470378\\
370.01	0.00502963521655258\\
371.01	0.00502794745582441\\
372.01	0.00502623736200119\\
373.01	0.00502450472453784\\
374.01	0.00502274932415067\\
375.01	0.00502097093169802\\
376.01	0.00501916930698559\\
377.01	0.00501734419749892\\
378.01	0.00501549533706614\\
379.01	0.00501362244445593\\
380.01	0.00501172522191803\\
381.01	0.0050098033536758\\
382.01	0.00500785650438091\\
383.01	0.00500588431754847\\
384.01	0.00500388641398753\\
385.01	0.00500186239025184\\
386.01	0.00499981181713583\\
387.01	0.00499773423824771\\
388.01	0.00499562916869507\\
389.01	0.00499349609392439\\
390.01	0.00499133446875998\\
391.01	0.00498914371669356\\
392.01	0.00498692322948159\\
393.01	0.00498467236710779\\
394.01	0.0049823904581755\\
395.01	0.00498007680079112\\
396.01	0.00497773066400296\\
397.01	0.00497535128984978\\
398.01	0.00497293789606803\\
399.01	0.00497048967948952\\
400.01	0.00496800582013984\\
401.01	0.00496548548601524\\
402.01	0.00496292783847709\\
403.01	0.00496033203814928\\
404.01	0.00495769725113726\\
405.01	0.00495502265531409\\
406.01	0.00495230744632645\\
407.01	0.00494955084288241\\
408.01	0.00494675209079259\\
409.01	0.00494391046515843\\
410.01	0.00494102527007969\\
411.01	0.00493809583530152\\
412.01	0.0049351215094292\\
413.01	0.00493210164978179\\
414.01	0.00492903560998622\\
415.01	0.00492592273033172\\
416.01	0.00492276233481905\\
417.01	0.00491955373006865\\
418.01	0.00491629620420586\\
419.01	0.00491298902567613\\
420.01	0.00490963144199091\\
421.01	0.00490622267840105\\
422.01	0.00490276193649748\\
423.01	0.00489924839273843\\
424.01	0.00489568119690254\\
425.01	0.00489205947046875\\
426.01	0.0048883823049225\\
427.01	0.00488464875999325\\
428.01	0.00488085786182232\\
429.01	0.00487700860106707\\
430.01	0.00487309993094574\\
431.01	0.00486913076522753\\
432.01	0.00486509997617781\\
433.01	0.00486100639246517\\
434.01	0.00485684879704068\\
435.01	0.00485262592500365\\
436.01	0.00484833646146494\\
437.01	0.00484397903942548\\
438.01	0.00483955223768783\\
439.01	0.00483505457881895\\
440.01	0.00483048452718812\\
441.01	0.00482584048710082\\
442.01	0.00482112080105842\\
443.01	0.00481632374816491\\
444.01	0.0048114475427145\\
445.01	0.00480649033298484\\
446.01	0.00480145020026725\\
447.01	0.00479632515816302\\
448.01	0.00479111315217012\\
449.01	0.00478581205958852\\
450.01	0.00478041968976229\\
451.01	0.00477493378467433\\
452.01	0.00476935201990197\\
453.01	0.00476367200593159\\
454.01	0.00475789128981889\\
455.01	0.00475200735717024\\
456.01	0.00474601763439902\\
457.01	0.004739919491201\\
458.01	0.00473371024316399\\
459.01	0.0047273871544136\\
460.01	0.00472094744016993\\
461.01	0.00471438826907376\\
462.01	0.00470770676512423\\
463.01	0.00470090000905348\\
464.01	0.00469396503897045\\
465.01	0.00468689885010838\\
466.01	0.00467969839354542\\
467.01	0.00467236057381463\\
468.01	0.00466488224539955\\
469.01	0.00465726020821576\\
470.01	0.00464949120231721\\
471.01	0.00464157190222172\\
472.01	0.00463349891140976\\
473.01	0.00462526875768144\\
474.01	0.00461687789006565\\
475.01	0.00460832267757614\\
476.01	0.00459959940900615\\
477.01	0.00459070429294342\\
478.01	0.00458163345786526\\
479.01	0.00457238295229263\\
480.01	0.00456294874497674\\
481.01	0.00455332672509053\\
482.01	0.00454351270239028\\
483.01	0.00453350240731102\\
484.01	0.00452329149095487\\
485.01	0.0045128755249317\\
486.01	0.00450225000100838\\
487.01	0.00449141033052669\\
488.01	0.00448035184355369\\
489.01	0.00446906978773375\\
490.01	0.00445755932682209\\
491.01	0.00444581553889245\\
492.01	0.00443383341422656\\
493.01	0.00442160785291254\\
494.01	0.00440913366219645\\
495.01	0.00439640555365504\\
496.01	0.00438341814026985\\
497.01	0.00437016593349746\\
498.01	0.00435664334043017\\
499.01	0.00434284466113117\\
500.01	0.00432876408619353\\
501.01	0.00431439569452768\\
502.01	0.00429973345131089\\
503.01	0.00428477120597454\\
504.01	0.00426950269010674\\
505.01	0.00425392151523104\\
506.01	0.0042380211704564\\
507.01	0.00422179502000055\\
508.01	0.00420523630059349\\
509.01	0.00418833811877135\\
510.01	0.00417109344807316\\
511.01	0.00415349512615779\\
512.01	0.00413553585186063\\
513.01	0.00411720818221027\\
514.01	0.00409850452942619\\
515.01	0.00407941715791657\\
516.01	0.00405993818129196\\
517.01	0.00404005955940527\\
518.01	0.00401977309542275\\
519.01	0.0039990704329228\\
520.01	0.00397794305301479\\
521.01	0.00395638227146918\\
522.01	0.00393437923584688\\
523.01	0.00391192492263153\\
524.01	0.00388901013437162\\
525.01	0.00386562549684443\\
526.01	0.00384176145625882\\
527.01	0.00381740827651028\\
528.01	0.00379255603650427\\
529.01	0.00376719462756544\\
530.01	0.00374131375094899\\
531.01	0.00371490291547166\\
532.01	0.0036879514352818\\
533.01	0.00366044842778692\\
534.01	0.00363238281176114\\
535.01	0.00360374330565611\\
536.01	0.0035745184261425\\
537.01	0.00354469648691371\\
538.01	0.00351426559778857\\
539.01	0.00348321366415254\\
540.01	0.0034515283867838\\
541.01	0.00341919726211485\\
542.01	0.00338620758298481\\
543.01	0.00335254643994394\\
544.01	0.00331820072317836\\
545.01	0.00328315712513108\\
546.01	0.00324740214390159\\
547.01	0.0032109220875175\\
548.01	0.00317370307917983\\
549.01	0.00313573106359633\\
550.01	0.00309699181452752\\
551.01	0.00305747094368483\\
552.01	0.00301715391113351\\
553.01	0.00297602603736896\\
554.01	0.00293407251725179\\
555.01	0.00289127843600545\\
556.01	0.00284762878749973\\
557.01	0.0028031084950645\\
558.01	0.00275770243510144\\
559.01	0.00271139546378433\\
560.01	0.00266417244716499\\
561.01	0.0026160182950273\\
562.01	0.00256691799886021\\
563.01	0.00251685667434632\\
564.01	0.00246581960879106\\
565.01	0.00241379231394375\\
566.01	0.00236076058468414\\
567.01	0.00230671056407004\\
568.01	0.00225162881525418\\
569.01	0.00219550240078614\\
570.01	0.00213831896980923\\
571.01	0.00208006685364094\\
572.01	0.0020207351701836\\
573.01	0.00196031393754173\\
574.01	0.00189879419711535\\
575.01	0.0018361681462836\\
576.01	0.00177242928057545\\
577.01	0.00170757254492862\\
578.01	0.00164159449323789\\
579.01	0.00157449345486604\\
580.01	0.00150626970609533\\
581.01	0.00143692564359493\\
582.01	0.00136646595580921\\
583.01	0.00129489778666962\\
584.01	0.00122223088410795\\
585.01	0.00114847772339413\\
586.01	0.00107365359220303\\
587.01	0.000997776620360649\\
588.01	0.000920867732220132\\
589.01	0.000842950493310577\\
590.01	0.000764050814954753\\
591.01	0.000684196470559372\\
592.01	0.000603416364730805\\
593.01	0.000521739480625437\\
594.01	0.000439193411214457\\
595.01	0.000355802355441642\\
596.01	0.00027158442935076\\
597.01	0.000186548103627336\\
598.01	0.000100687530733697\\
599.01	3.18230442563749e-05\\
599.02	3.12770957175551e-05\\
599.03	3.07343620567866e-05\\
599.04	3.0194875217571e-05\\
599.05	2.96586674565693e-05\\
599.06	2.91257713466771e-05\\
599.07	2.85962197801391e-05\\
599.08	2.80700459716933e-05\\
599.09	2.75472834617464e-05\\
599.1	2.70279661195791e-05\\
599.11	2.65121281465865e-05\\
599.12	2.59998040795448e-05\\
599.13	2.54910287939124e-05\\
599.14	2.49858375071712e-05\\
599.15	2.44842657821827e-05\\
599.16	2.39863495305956e-05\\
599.17	2.34921250162837e-05\\
599.18	2.30016288588052e-05\\
599.19	2.25148980369013e-05\\
599.2	2.20319698920508e-05\\
599.21	2.1552882132023e-05\\
599.22	2.10776728344908e-05\\
599.23	2.06063804506721e-05\\
599.24	2.01390438090126e-05\\
599.25	1.96757021188858e-05\\
599.26	1.9216394974363e-05\\
599.27	1.87611623579872e-05\\
599.28	1.83100446445959e-05\\
599.29	1.78630826051952e-05\\
599.3	1.74203174108517e-05\\
599.31	1.69817906366353e-05\\
599.32	1.65475442655931e-05\\
599.33	1.61176206927693e-05\\
599.34	1.56920627292622e-05\\
599.35	1.52709136063203e-05\\
599.36	1.48542169794725e-05\\
599.37	1.44420169327208e-05\\
599.38	1.40343579827368e-05\\
599.39	1.36312859119591e-05\\
599.4	1.32328499381877e-05\\
599.41	1.28390997671604e-05\\
599.42	1.2450085597351e-05\\
599.43	1.20658581248111e-05\\
599.44	1.16864685480531e-05\\
599.45	1.13119685730082e-05\\
599.46	1.09424104179929e-05\\
599.47	1.05778468187639e-05\\
599.48	1.02183310335888e-05\\
599.49	9.86391684839293e-06\\
599.5	9.51465858193938e-06\\
599.51	9.17061109107289e-06\\
599.52	8.83182977599525e-06\\
599.53	8.4983705856221e-06\\
599.54	8.1702900229675e-06\\
599.55	7.84764515058753e-06\\
599.56	7.53049359608279e-06\\
599.57	7.21889355765649e-06\\
599.58	6.91290380971064e-06\\
599.59	6.61258370851688e-06\\
599.6	6.31799319793756e-06\\
599.61	6.02919281519031e-06\\
599.62	5.74624369668528e-06\\
599.63	5.46920758391981e-06\\
599.64	5.19814682940593e-06\\
599.65	4.93312440269685e-06\\
599.66	4.67420389643237e-06\\
599.67	4.42144953247819e-06\\
599.68	4.17492616808582e-06\\
599.69	3.9346993021671e-06\\
599.7	3.70083508156871e-06\\
599.71	3.47340030746289e-06\\
599.72	3.25246244175549e-06\\
599.73	3.03808961359987e-06\\
599.74	2.83035062593855e-06\\
599.75	2.62931496212288e-06\\
599.76	2.43505279260738e-06\\
599.77	2.24763498169606e-06\\
599.78	2.06713309435641e-06\\
599.79	1.89361940311147e-06\\
599.8	1.72716689497872e-06\\
599.81	1.56784927850956e-06\\
599.82	1.41574099085315e-06\\
599.83	1.27091720493813e-06\\
599.84	1.13345383668563e-06\\
599.85	1.00342755231415e-06\\
599.86	8.8091577571392e-07\\
599.87	7.65996695880136e-07\\
599.88	6.58749274441706e-07\\
599.89	5.59253253241965e-07\\
599.9	4.67589162013102e-07\\
599.91	3.83838326099145e-07\\
599.92	3.0808287429171e-07\\
599.93	2.40405746710845e-07\\
599.94	1.8089070277609e-07\\
599.95	1.29622329257326e-07\\
599.96	8.66860484002169e-08\\
599.97	5.21681261331924e-08\\
599.98	2.61556803542173e-08\\
599.99	8.73668930083393e-09\\
600	0\\
};
\addplot [color=mycolor21,solid,forget plot]
  table[row sep=crcr]{%
0.01	0.00507708887705905\\
1.01	0.00507708753621324\\
2.01	0.0050770861680176\\
3.01	0.00507708477191713\\
4.01	0.0050770833473457\\
5.01	0.00507708189372535\\
6.01	0.00507708041046706\\
7.01	0.00507707889696935\\
8.01	0.00507707735261923\\
9.01	0.00507707577679103\\
10.01	0.00507707416884656\\
11.01	0.00507707252813467\\
12.01	0.00507707085399132\\
13.01	0.00507706914573891\\
14.01	0.0050770674026864\\
15.01	0.00507706562412882\\
16.01	0.0050770638093468\\
17.01	0.00507706195760675\\
18.01	0.00507706006816042\\
19.01	0.00507705814024422\\
20.01	0.00507705617307949\\
21.01	0.00507705416587195\\
22.01	0.00507705211781113\\
23.01	0.00507705002807054\\
24.01	0.0050770478958069\\
25.01	0.00507704572016018\\
26.01	0.00507704350025313\\
27.01	0.00507704123519085\\
28.01	0.00507703892406036\\
29.01	0.00507703656593057\\
30.01	0.00507703415985188\\
31.01	0.0050770317048553\\
32.01	0.00507702919995268\\
33.01	0.00507702664413585\\
34.01	0.00507702403637687\\
35.01	0.00507702137562703\\
36.01	0.00507701866081639\\
37.01	0.00507701589085413\\
38.01	0.0050770130646273\\
39.01	0.0050770101810007\\
40.01	0.00507700723881643\\
41.01	0.00507700423689375\\
42.01	0.00507700117402806\\
43.01	0.00507699804899074\\
44.01	0.00507699486052866\\
45.01	0.00507699160736389\\
46.01	0.0050769882881928\\
47.01	0.00507698490168587\\
48.01	0.00507698144648692\\
49.01	0.00507697792121297\\
50.01	0.00507697432445339\\
51.01	0.00507697065476941\\
52.01	0.00507696691069357\\
53.01	0.00507696309072958\\
54.01	0.00507695919335097\\
55.01	0.00507695521700095\\
56.01	0.0050769511600919\\
57.01	0.00507694702100466\\
58.01	0.00507694279808788\\
59.01	0.00507693848965749\\
60.01	0.00507693409399573\\
61.01	0.00507692960935092\\
62.01	0.00507692503393644\\
63.01	0.00507692036593068\\
64.01	0.00507691560347542\\
65.01	0.00507691074467593\\
66.01	0.00507690578759975\\
67.01	0.0050769007302762\\
68.01	0.00507689557069561\\
69.01	0.00507689030680855\\
70.01	0.00507688493652499\\
71.01	0.00507687945771347\\
72.01	0.00507687386820064\\
73.01	0.00507686816577011\\
74.01	0.0050768623481615\\
75.01	0.00507685641306991\\
76.01	0.00507685035814514\\
77.01	0.00507684418099038\\
78.01	0.00507683787916161\\
79.01	0.00507683145016669\\
80.01	0.00507682489146434\\
81.01	0.0050768182004631\\
82.01	0.00507681137452044\\
83.01	0.00507680441094229\\
84.01	0.00507679730698098\\
85.01	0.00507679005983503\\
86.01	0.00507678266664797\\
87.01	0.00507677512450732\\
88.01	0.00507676743044301\\
89.01	0.00507675958142686\\
90.01	0.00507675157437126\\
91.01	0.00507674340612815\\
92.01	0.00507673507348772\\
93.01	0.00507672657317712\\
94.01	0.00507671790185935\\
95.01	0.00507670905613247\\
96.01	0.00507670003252765\\
97.01	0.00507669082750827\\
98.01	0.00507668143746858\\
99.01	0.00507667185873229\\
100.01	0.0050766620875516\\
101.01	0.00507665212010514\\
102.01	0.00507664195249747\\
103.01	0.00507663158075676\\
104.01	0.00507662100083391\\
105.01	0.0050766102086011\\
106.01	0.00507659919985008\\
107.01	0.00507658797029066\\
108.01	0.00507657651554915\\
109.01	0.00507656483116712\\
110.01	0.0050765529125992\\
111.01	0.00507654075521193\\
112.01	0.00507652835428203\\
113.01	0.00507651570499456\\
114.01	0.00507650280244127\\
115.01	0.00507648964161869\\
116.01	0.00507647621742685\\
117.01	0.00507646252466667\\
118.01	0.00507644855803879\\
119.01	0.00507643431214153\\
120.01	0.00507641978146884\\
121.01	0.00507640496040837\\
122.01	0.00507638984323938\\
123.01	0.00507637442413115\\
124.01	0.0050763586971405\\
125.01	0.00507634265621009\\
126.01	0.00507632629516564\\
127.01	0.0050763096077144\\
128.01	0.00507629258744297\\
129.01	0.00507627522781448\\
130.01	0.00507625752216674\\
131.01	0.00507623946371005\\
132.01	0.00507622104552452\\
133.01	0.00507620226055775\\
134.01	0.00507618310162244\\
135.01	0.00507616356139414\\
136.01	0.00507614363240815\\
137.01	0.00507612330705765\\
138.01	0.00507610257759048\\
139.01	0.005076081436107\\
140.01	0.00507605987455689\\
141.01	0.00507603788473684\\
142.01	0.00507601545828773\\
143.01	0.00507599258669128\\
144.01	0.00507596926126792\\
145.01	0.00507594547317327\\
146.01	0.00507592121339545\\
147.01	0.00507589647275183\\
148.01	0.00507587124188621\\
149.01	0.00507584551126559\\
150.01	0.00507581927117713\\
151.01	0.00507579251172424\\
152.01	0.00507576522282425\\
153.01	0.0050757373942047\\
154.01	0.00507570901539941\\
155.01	0.00507568007574582\\
156.01	0.00507565056438109\\
157.01	0.00507562047023848\\
158.01	0.00507558978204405\\
159.01	0.00507555848831241\\
160.01	0.00507552657734379\\
161.01	0.00507549403721958\\
162.01	0.00507546085579868\\
163.01	0.0050754270207135\\
164.01	0.00507539251936635\\
165.01	0.00507535733892502\\
166.01	0.00507532146631874\\
167.01	0.00507528488823392\\
168.01	0.00507524759111015\\
169.01	0.00507520956113589\\
170.01	0.00507517078424397\\
171.01	0.00507513124610713\\
172.01	0.00507509093213366\\
173.01	0.00507504982746286\\
174.01	0.00507500791695994\\
175.01	0.0050749651852123\\
176.01	0.00507492161652357\\
177.01	0.00507487719490985\\
178.01	0.005074831904094\\
179.01	0.0050747857275009\\
180.01	0.00507473864825254\\
181.01	0.00507469064916273\\
182.01	0.00507464171273208\\
183.01	0.00507459182114255\\
184.01	0.00507454095625181\\
185.01	0.00507448909958861\\
186.01	0.0050744362323465\\
187.01	0.00507438233537851\\
188.01	0.0050743273891917\\
189.01	0.005074271373941\\
190.01	0.00507421426942365\\
191.01	0.00507415605507348\\
192.01	0.0050740967099546\\
193.01	0.00507403621275545\\
194.01	0.00507397454178257\\
195.01	0.0050739116749546\\
196.01	0.00507384758979588\\
197.01	0.00507378226342988\\
198.01	0.00507371567257321\\
199.01	0.00507364779352859\\
200.01	0.00507357860217837\\
201.01	0.00507350807397796\\
202.01	0.00507343618394885\\
203.01	0.00507336290667201\\
204.01	0.00507328821628047\\
205.01	0.00507321208645276\\
206.01	0.00507313449040569\\
207.01	0.00507305540088659\\
208.01	0.00507297479016699\\
209.01	0.00507289263003433\\
210.01	0.00507280889178514\\
211.01	0.00507272354621716\\
212.01	0.0050726365636218\\
213.01	0.00507254791377652\\
214.01	0.00507245756593692\\
215.01	0.00507236548882902\\
216.01	0.00507227165064126\\
217.01	0.00507217601901641\\
218.01	0.00507207856104334\\
219.01	0.00507197924324943\\
220.01	0.00507187803159139\\
221.01	0.00507177489144776\\
222.01	0.0050716697876099\\
223.01	0.00507156268427412\\
224.01	0.00507145354503242\\
225.01	0.00507134233286413\\
226.01	0.00507122901012767\\
227.01	0.00507111353855088\\
228.01	0.00507099587922289\\
229.01	0.00507087599258482\\
230.01	0.00507075383842098\\
231.01	0.00507062937585008\\
232.01	0.00507050256331542\\
233.01	0.00507037335857621\\
234.01	0.00507024171869829\\
235.01	0.00507010760004478\\
236.01	0.00506997095826677\\
237.01	0.00506983174829409\\
238.01	0.00506968992432524\\
239.01	0.00506954543981895\\
240.01	0.00506939824748368\\
241.01	0.00506924829926839\\
242.01	0.00506909554635304\\
243.01	0.00506893993913896\\
244.01	0.00506878142723886\\
245.01	0.00506861995946761\\
246.01	0.00506845548383187\\
247.01	0.0050682879475211\\
248.01	0.00506811729689692\\
249.01	0.00506794347748404\\
250.01	0.00506776643395975\\
251.01	0.005067586110145\\
252.01	0.00506740244899392\\
253.01	0.00506721539258424\\
254.01	0.00506702488210727\\
255.01	0.00506683085785847\\
256.01	0.00506663325922733\\
257.01	0.00506643202468773\\
258.01	0.0050662270917882\\
259.01	0.00506601839714222\\
260.01	0.00506580587641847\\
261.01	0.00506558946433108\\
262.01	0.00506536909463067\\
263.01	0.00506514470009385\\
264.01	0.00506491621251443\\
265.01	0.00506468356269365\\
266.01	0.005064446680431\\
267.01	0.0050642054945148\\
268.01	0.00506395993271301\\
269.01	0.00506370992176399\\
270.01	0.00506345538736767\\
271.01	0.00506319625417625\\
272.01	0.00506293244578591\\
273.01	0.00506266388472717\\
274.01	0.00506239049245704\\
275.01	0.00506211218935009\\
276.01	0.00506182889469009\\
277.01	0.00506154052666153\\
278.01	0.0050612470023417\\
279.01	0.00506094823769231\\
280.01	0.00506064414755202\\
281.01	0.00506033464562809\\
282.01	0.00506001964448908\\
283.01	0.00505969905555732\\
284.01	0.00505937278910131\\
285.01	0.00505904075422845\\
286.01	0.00505870285887829\\
287.01	0.00505835900981521\\
288.01	0.00505800911262169\\
289.01	0.00505765307169174\\
290.01	0.00505729079022411\\
291.01	0.00505692217021622\\
292.01	0.00505654711245727\\
293.01	0.00505616551652256\\
294.01	0.00505577728076712\\
295.01	0.00505538230231984\\
296.01	0.00505498047707758\\
297.01	0.00505457169969901\\
298.01	0.0050541558635993\\
299.01	0.00505373286094394\\
300.01	0.00505330258264308\\
301.01	0.00505286491834604\\
302.01	0.00505241975643513\\
303.01	0.00505196698401995\\
304.01	0.00505150648693165\\
305.01	0.00505103814971642\\
306.01	0.00505056185562958\\
307.01	0.00505007748662879\\
308.01	0.00504958492336787\\
309.01	0.00504908404518902\\
310.01	0.00504857473011588\\
311.01	0.00504805685484564\\
312.01	0.00504753029474055\\
313.01	0.00504699492381921\\
314.01	0.0050464506147466\\
315.01	0.00504589723882434\\
316.01	0.00504533466597911\\
317.01	0.00504476276475078\\
318.01	0.00504418140227924\\
319.01	0.00504359044429042\\
320.01	0.00504298975508064\\
321.01	0.00504237919749954\\
322.01	0.00504175863293179\\
323.01	0.0050411279212771\\
324.01	0.00504048692092766\\
325.01	0.00503983548874477\\
326.01	0.00503917348003241\\
327.01	0.00503850074850792\\
328.01	0.00503781714627182\\
329.01	0.00503712252377266\\
330.01	0.00503641672977052\\
331.01	0.00503569961129585\\
332.01	0.00503497101360607\\
333.01	0.00503423078013689\\
334.01	0.00503347875245049\\
335.01	0.00503271477017883\\
336.01	0.00503193867096227\\
337.01	0.00503115029038325\\
338.01	0.00503034946189406\\
339.01	0.00502953601673909\\
340.01	0.00502870978387105\\
341.01	0.00502787058985985\\
342.01	0.00502701825879568\\
343.01	0.00502615261218312\\
344.01	0.00502527346882931\\
345.01	0.00502438064472175\\
346.01	0.00502347395289909\\
347.01	0.00502255320331151\\
348.01	0.005021618202672\\
349.01	0.00502066875429794\\
350.01	0.00501970465794117\\
351.01	0.0050187257096079\\
352.01	0.00501773170136676\\
353.01	0.00501672242114493\\
354.01	0.00501569765251243\\
355.01	0.00501465717445341\\
356.01	0.00501360076112476\\
357.01	0.00501252818160172\\
358.01	0.00501143919960925\\
359.01	0.00501033357324104\\
360.01	0.00500921105466373\\
361.01	0.00500807138980852\\
362.01	0.00500691431804838\\
363.01	0.00500573957186251\\
364.01	0.00500454687648826\\
365.01	0.0050033359495601\\
366.01	0.0050021065007374\\
367.01	0.00500085823132129\\
368.01	0.00499959083386242\\
369.01	0.00499830399176034\\
370.01	0.00499699737885614\\
371.01	0.00499567065902101\\
372.01	0.00499432348574205\\
373.01	0.00499295550170825\\
374.01	0.00499156633840018\\
375.01	0.00499015561568501\\
376.01	0.00498872294142227\\
377.01	0.00498726791108275\\
378.01	0.00498579010738665\\
379.01	0.00498428909996395\\
380.01	0.00498276444504269\\
381.01	0.00498121568517083\\
382.01	0.00497964234897718\\
383.01	0.00497804395097536\\
384.01	0.00497641999142021\\
385.01	0.00497476995621788\\
386.01	0.00497309331689844\\
387.01	0.00497138953065381\\
388.01	0.00496965804044674\\
389.01	0.00496789827519218\\
390.01	0.00496610965001477\\
391.01	0.00496429156658303\\
392.01	0.00496244341351421\\
393.01	0.00496056456685244\\
394.01	0.00495865439060511\\
395.01	0.00495671223733239\\
396.01	0.00495473744877186\\
397.01	0.00495272935648013\\
398.01	0.00495068728246698\\
399.01	0.00494861053979386\\
400.01	0.00494649843310342\\
401.01	0.00494435025904248\\
402.01	0.00494216530653852\\
403.01	0.00493994285688795\\
404.01	0.0049376821836139\\
405.01	0.00493538255205873\\
406.01	0.00493304321868069\\
407.01	0.0049306634300419\\
408.01	0.00492824242149082\\
409.01	0.00492577941557181\\
410.01	0.00492327362022339\\
411.01	0.00492072422686674\\
412.01	0.00491813040851883\\
413.01	0.00491549131808978\\
414.01	0.00491280608702397\\
415.01	0.00491007382433465\\
416.01	0.00490729361582734\\
417.01	0.00490446452333173\\
418.01	0.00490158558391598\\
419.01	0.00489865580908632\\
420.01	0.00489567418397143\\
421.01	0.00489263966649393\\
422.01	0.00488955118653039\\
423.01	0.00488640764506158\\
424.01	0.00488320791331547\\
425.01	0.00487995083190449\\
426.01	0.00487663520996096\\
427.01	0.00487325982427065\\
428.01	0.00486982341841193\\
429.01	0.00486632470189805\\
430.01	0.00486276234933108\\
431.01	0.00485913499956767\\
432.01	0.00485544125490201\\
433.01	0.00485167968026843\\
434.01	0.00484784880246914\\
435.01	0.00484394710942952\\
436.01	0.00483997304948575\\
437.01	0.00483592503070799\\
438.01	0.00483180142026234\\
439.01	0.0048276005438161\\
440.01	0.00482332068498685\\
441.01	0.00481896008484052\\
442.01	0.00481451694143632\\
443.01	0.00480998940942305\\
444.01	0.0048053755996833\\
445.01	0.00480067357902606\\
446.01	0.00479588136992385\\
447.01	0.00479099695029043\\
448.01	0.00478601825329393\\
449.01	0.00478094316719583\\
450.01	0.00477576953520854\\
451.01	0.00477049515535841\\
452.01	0.00476511778034158\\
453.01	0.0047596351173582\\
454.01	0.00475404482790781\\
455.01	0.00474834452752741\\
456.01	0.00474253178545619\\
457.01	0.00473660412420473\\
458.01	0.00473055901901403\\
459.01	0.00472439389718501\\
460.01	0.00471810613726825\\
461.01	0.0047116930681009\\
462.01	0.00470515196768929\\
463.01	0.00469848006193911\\
464.01	0.00469167452324305\\
465.01	0.00468473246894722\\
466.01	0.00467765095972418\\
467.01	0.00467042699789038\\
468.01	0.00466305752571401\\
469.01	0.00465553942376162\\
470.01	0.00464786950933107\\
471.01	0.00464004453501251\\
472.01	0.0046320611874001\\
473.01	0.00462391608595133\\
474.01	0.00461560578196254\\
475.01	0.00460712675759189\\
476.01	0.0045984754248689\\
477.01	0.00458964812466511\\
478.01	0.00458064112561716\\
479.01	0.00457145062299611\\
480.01	0.00456207273751556\\
481.01	0.00455250351407303\\
482.01	0.00454273892041895\\
483.01	0.00453277484574797\\
484.01	0.00452260709921037\\
485.01	0.00451223140833936\\
486.01	0.00450164341739547\\
487.01	0.00449083868562842\\
488.01	0.00447981268545985\\
489.01	0.00446856080059189\\
490.01	0.00445707832404862\\
491.01	0.00444536045615936\\
492.01	0.00443340230249307\\
493.01	0.00442119887175426\\
494.01	0.00440874507365103\\
495.01	0.00439603571674383\\
496.01	0.0043830655062815\\
497.01	0.00436982904202923\\
498.01	0.00435632081608763\\
499.01	0.00434253521069777\\
500.01	0.0043284664960249\\
501.01	0.00431410882791098\\
502.01	0.00429945624558325\\
503.01	0.00428450266931315\\
504.01	0.0042692418980233\\
505.01	0.00425366760684348\\
506.01	0.00423777334461912\\
507.01	0.004221552531376\\
508.01	0.00420499845574363\\
509.01	0.00418810427234193\\
510.01	0.00417086299913536\\
511.01	0.00415326751475782\\
512.01	0.0041353105558127\\
513.01	0.00411698471414996\\
514.01	0.00409828243412553\\
515.01	0.00407919600984361\\
516.01	0.00405971758238504\\
517.01	0.00403983913702328\\
518.01	0.00401955250043088\\
519.01	0.0039988493378783\\
520.01	0.00397772115042922\\
521.01	0.00395615927213539\\
522.01	0.00393415486723955\\
523.01	0.00391169892739118\\
524.01	0.00388878226888378\\
525.01	0.00386539552992501\\
526.01	0.00384152916794622\\
527.01	0.00381717345696531\\
528.01	0.00379231848501426\\
529.01	0.00376695415164407\\
530.01	0.00374107016552353\\
531.01	0.00371465604214729\\
532.01	0.00368770110167236\\
533.01	0.00366019446690445\\
534.01	0.00363212506145671\\
535.01	0.0036034816081067\\
536.01	0.00357425262738278\\
537.01	0.00354442643640999\\
538.01	0.00351399114805384\\
539.01	0.0034829346704015\\
540.01	0.00345124470662508\\
541.01	0.00341890875527855\\
542.01	0.00338591411108143\\
543.01	0.00335224786625271\\
544.01	0.00331789691246243\\
545.01	0.00328284794347674\\
546.01	0.00324708745858068\\
547.01	0.00321060176687146\\
548.01	0.00317337699252519\\
549.01	0.00313539908115085\\
550.01	0.00309665380735802\\
551.01	0.00305712678367617\\
552.01	0.00301680347097966\\
553.01	0.00297566919058672\\
554.01	0.00293370913821819\\
555.01	0.00289090840001998\\
556.01	0.00284725197087291\\
557.01	0.00280272477523499\\
558.01	0.00275731169078333\\
559.01	0.00271099757514768\\
560.01	0.0026637672960521\\
561.01	0.00261560576520825\\
562.01	0.00256649797633006\\
563.01	0.00251642904766814\\
564.01	0.00246538426948843\\
565.01	0.00241334915694547\\
566.01	0.0023603095088256\\
567.01	0.00230625147265386\\
568.01	0.00225116161667357\\
569.01	0.00219502700921337\\
570.01	0.00213783530595022\\
571.01	0.00207957484555631\\
572.01	0.00202023475417473\\
573.01	0.00195980505909759\\
574.01	0.00189827681191261\\
575.01	0.00183564222122825\\
576.01	0.00177189479486902\\
577.01	0.00170702949113364\\
578.01	0.00164104287830919\\
579.01	0.00157393330110091\\
580.01	0.00150570105194178\\
581.01	0.00143634854423568\\
582.01	0.0013658804834154\\
583.01	0.00129430403018606\\
584.01	0.00122162894839109\\
585.01	0.00114786772747427\\
586.01	0.00107303566637796\\
587.01	0.000997150901747018\\
588.01	0.000920234358289583\\
589.01	0.00084230959280966\\
590.01	0.000763402495449801\\
591.01	0.000683540801648747\\
592.01	0.000602753355717998\\
593.01	0.000521069051135661\\
594.01	0.000438515352848756\\
595.01	0.000355116282076462\\
596.01	0.000270889713084794\\
597.01	0.000185843792620275\\
598.01	9.99722432115345e-05\\
599.01	3.18230442563731e-05\\
599.02	3.12770957175551e-05\\
599.03	3.07343620567866e-05\\
599.04	3.0194875217571e-05\\
599.05	2.96586674565693e-05\\
599.06	2.91257713466771e-05\\
599.07	2.85962197801391e-05\\
599.08	2.80700459716916e-05\\
599.09	2.75472834617447e-05\\
599.1	2.70279661195773e-05\\
599.11	2.65121281465865e-05\\
599.12	2.59998040795448e-05\\
599.13	2.54910287939142e-05\\
599.14	2.49858375071729e-05\\
599.15	2.44842657821827e-05\\
599.16	2.39863495305973e-05\\
599.17	2.34921250162855e-05\\
599.18	2.30016288588035e-05\\
599.19	2.2514898036903e-05\\
599.2	2.20319698920508e-05\\
599.21	2.1552882132023e-05\\
599.22	2.10776728344908e-05\\
599.23	2.06063804506721e-05\\
599.24	2.01390438090109e-05\\
599.25	1.96757021188858e-05\\
599.26	1.92163949743647e-05\\
599.27	1.87611623579855e-05\\
599.28	1.83100446445959e-05\\
599.29	1.78630826051952e-05\\
599.3	1.74203174108517e-05\\
599.31	1.69817906366353e-05\\
599.32	1.65475442655931e-05\\
599.33	1.61176206927693e-05\\
599.34	1.56920627292639e-05\\
599.35	1.52709136063203e-05\\
599.36	1.48542169794742e-05\\
599.37	1.4442016932719e-05\\
599.38	1.40343579827368e-05\\
599.39	1.36312859119591e-05\\
599.4	1.32328499381877e-05\\
599.41	1.28390997671604e-05\\
599.42	1.2450085597351e-05\\
599.43	1.20658581248094e-05\\
599.44	1.16864685480531e-05\\
599.45	1.13119685730065e-05\\
599.46	1.09424104179929e-05\\
599.47	1.05778468187639e-05\\
599.48	1.02183310335888e-05\\
599.49	9.86391684839293e-06\\
599.5	9.51465858194112e-06\\
599.51	9.17061109107116e-06\\
599.52	8.83182977599352e-06\\
599.53	8.49837058562383e-06\\
599.54	8.17029002296576e-06\\
599.55	7.84764515058753e-06\\
599.56	7.53049359608453e-06\\
599.57	7.21889355765649e-06\\
599.58	6.91290380971064e-06\\
599.59	6.61258370851688e-06\\
599.6	6.31799319793756e-06\\
599.61	6.02919281518857e-06\\
599.62	5.74624369668701e-06\\
599.63	5.46920758391981e-06\\
599.64	5.19814682940767e-06\\
599.65	4.93312440269685e-06\\
599.66	4.67420389643411e-06\\
599.67	4.42144953247646e-06\\
599.68	4.17492616808755e-06\\
599.69	3.9346993021671e-06\\
599.7	3.70083508157044e-06\\
599.71	3.47340030746116e-06\\
599.72	3.25246244175723e-06\\
599.73	3.03808961360161e-06\\
599.74	2.83035062593855e-06\\
599.75	2.62931496212288e-06\\
599.76	2.43505279260738e-06\\
599.77	2.24763498169432e-06\\
599.78	2.06713309435815e-06\\
599.79	1.89361940310974e-06\\
599.8	1.72716689498045e-06\\
599.81	1.56784927850782e-06\\
599.82	1.41574099085315e-06\\
599.83	1.27091720493987e-06\\
599.84	1.13345383668563e-06\\
599.85	1.00342755231589e-06\\
599.86	8.8091577571392e-07\\
599.87	7.65996695880136e-07\\
599.88	6.58749274441706e-07\\
599.89	5.59253253243699e-07\\
599.9	4.67589162011367e-07\\
599.91	3.83838326099145e-07\\
599.92	3.0808287429171e-07\\
599.93	2.4040574671258e-07\\
599.94	1.80890702777825e-07\\
599.95	1.2962232925906e-07\\
599.96	8.66860484019516e-08\\
599.97	5.21681261331924e-08\\
599.98	2.61556803542173e-08\\
599.99	8.73668930083393e-09\\
600	0\\
};
\addplot [color=black!20!mycolor21,solid,forget plot]
  table[row sep=crcr]{%
0.01	0.00505041622436337\\
1.01	0.00505041502687961\\
2.01	0.00505041380528635\\
3.01	0.00505041255910172\\
4.01	0.00505041128783399\\
5.01	0.00505040999098198\\
6.01	0.00505040866803445\\
7.01	0.0050504073184702\\
8.01	0.00505040594175756\\
9.01	0.0050504045373543\\
10.01	0.00505040310470765\\
11.01	0.00505040164325379\\
12.01	0.00505040015241783\\
13.01	0.00505039863161348\\
14.01	0.00505039708024294\\
15.01	0.00505039549769652\\
16.01	0.00505039388335266\\
17.01	0.00505039223657732\\
18.01	0.00505039055672402\\
19.01	0.00505038884313369\\
20.01	0.00505038709513408\\
21.01	0.00505038531203984\\
22.01	0.00505038349315189\\
23.01	0.00505038163775768\\
24.01	0.00505037974513043\\
25.01	0.00505037781452916\\
26.01	0.00505037584519808\\
27.01	0.00505037383636694\\
28.01	0.00505037178725006\\
29.01	0.0050503696970464\\
30.01	0.00505036756493908\\
31.01	0.00505036539009538\\
32.01	0.00505036317166596\\
33.01	0.00505036090878514\\
34.01	0.00505035860056998\\
35.01	0.00505035624612024\\
36.01	0.00505035384451818\\
37.01	0.00505035139482785\\
38.01	0.0050503488960952\\
39.01	0.00505034634734725\\
40.01	0.00505034374759241\\
41.01	0.00505034109581914\\
42.01	0.00505033839099619\\
43.01	0.0050503356320726\\
44.01	0.00505033281797631\\
45.01	0.00505032994761456\\
46.01	0.00505032701987323\\
47.01	0.0050503240336164\\
48.01	0.00505032098768594\\
49.01	0.00505031788090126\\
50.01	0.00505031471205845\\
51.01	0.00505031147993047\\
52.01	0.00505030818326621\\
53.01	0.00505030482078999\\
54.01	0.0050503013912013\\
55.01	0.00505029789317477\\
56.01	0.00505029432535876\\
57.01	0.00505029068637558\\
58.01	0.00505028697482046\\
59.01	0.00505028318926152\\
60.01	0.00505027932823905\\
61.01	0.005050275390265\\
62.01	0.00505027137382239\\
63.01	0.00505026727736469\\
64.01	0.00505026309931556\\
65.01	0.00505025883806804\\
66.01	0.00505025449198377\\
67.01	0.00505025005939313\\
68.01	0.00505024553859374\\
69.01	0.0050502409278504\\
70.01	0.00505023622539427\\
71.01	0.00505023142942263\\
72.01	0.00505022653809739\\
73.01	0.00505022154954516\\
74.01	0.00505021646185647\\
75.01	0.00505021127308494\\
76.01	0.00505020598124666\\
77.01	0.00505020058431944\\
78.01	0.00505019508024193\\
79.01	0.00505018946691318\\
80.01	0.00505018374219198\\
81.01	0.00505017790389557\\
82.01	0.00505017194979935\\
83.01	0.00505016587763594\\
84.01	0.00505015968509456\\
85.01	0.00505015336981967\\
86.01	0.00505014692941063\\
87.01	0.00505014036142088\\
88.01	0.00505013366335681\\
89.01	0.00505012683267706\\
90.01	0.00505011986679147\\
91.01	0.00505011276306055\\
92.01	0.00505010551879388\\
93.01	0.00505009813124976\\
94.01	0.00505009059763433\\
95.01	0.0050500829150999\\
96.01	0.00505007508074471\\
97.01	0.00505006709161143\\
98.01	0.00505005894468656\\
99.01	0.00505005063689899\\
100.01	0.00505004216511883\\
101.01	0.00505003352615715\\
102.01	0.00505002471676396\\
103.01	0.00505001573362758\\
104.01	0.00505000657337335\\
105.01	0.00504999723256242\\
106.01	0.00504998770769082\\
107.01	0.00504997799518791\\
108.01	0.00504996809141548\\
109.01	0.00504995799266634\\
110.01	0.00504994769516324\\
111.01	0.00504993719505728\\
112.01	0.00504992648842675\\
113.01	0.00504991557127588\\
114.01	0.00504990443953338\\
115.01	0.00504989308905139\\
116.01	0.00504988151560328\\
117.01	0.0050498697148834\\
118.01	0.00504985768250443\\
119.01	0.00504984541399691\\
120.01	0.00504983290480681\\
121.01	0.00504982015029495\\
122.01	0.00504980714573495\\
123.01	0.00504979388631146\\
124.01	0.00504978036711917\\
125.01	0.0050497665831604\\
126.01	0.0050497525293443\\
127.01	0.00504973820048466\\
128.01	0.00504972359129796\\
129.01	0.00504970869640226\\
130.01	0.00504969351031514\\
131.01	0.00504967802745166\\
132.01	0.00504966224212308\\
133.01	0.00504964614853445\\
134.01	0.00504962974078308\\
135.01	0.00504961301285629\\
136.01	0.0050495959586299\\
137.01	0.00504957857186603\\
138.01	0.00504956084621088\\
139.01	0.00504954277519308\\
140.01	0.00504952435222127\\
141.01	0.00504950557058214\\
142.01	0.00504948642343829\\
143.01	0.00504946690382603\\
144.01	0.00504944700465291\\
145.01	0.00504942671869607\\
146.01	0.00504940603859908\\
147.01	0.00504938495687064\\
148.01	0.00504936346588136\\
149.01	0.00504934155786169\\
150.01	0.00504931922489933\\
151.01	0.00504929645893711\\
152.01	0.00504927325177016\\
153.01	0.00504924959504327\\
154.01	0.00504922548024874\\
155.01	0.00504920089872304\\
156.01	0.00504917584164474\\
157.01	0.00504915030003159\\
158.01	0.00504912426473755\\
159.01	0.00504909772645034\\
160.01	0.0050490706756881\\
161.01	0.00504904310279709\\
162.01	0.00504901499794829\\
163.01	0.00504898635113455\\
164.01	0.00504895715216749\\
165.01	0.00504892739067457\\
166.01	0.0050488970560957\\
167.01	0.00504886613768058\\
168.01	0.00504883462448481\\
169.01	0.00504880250536705\\
170.01	0.00504876976898552\\
171.01	0.0050487364037947\\
172.01	0.00504870239804215\\
173.01	0.00504866773976427\\
174.01	0.00504863241678364\\
175.01	0.00504859641670479\\
176.01	0.00504855972691134\\
177.01	0.00504852233456132\\
178.01	0.00504848422658404\\
179.01	0.00504844538967644\\
180.01	0.00504840581029883\\
181.01	0.00504836547467099\\
182.01	0.00504832436876866\\
183.01	0.00504828247831899\\
184.01	0.00504823978879698\\
185.01	0.0050481962854207\\
186.01	0.00504815195314776\\
187.01	0.00504810677667052\\
188.01	0.00504806074041221\\
189.01	0.00504801382852224\\
190.01	0.00504796602487202\\
191.01	0.00504791731305022\\
192.01	0.00504786767635818\\
193.01	0.00504781709780566\\
194.01	0.00504776556010587\\
195.01	0.00504771304567071\\
196.01	0.00504765953660593\\
197.01	0.00504760501470664\\
198.01	0.00504754946145155\\
199.01	0.005047492857999\\
200.01	0.0050474351851811\\
201.01	0.00504737642349863\\
202.01	0.00504731655311634\\
203.01	0.00504725555385704\\
204.01	0.00504719340519661\\
205.01	0.00504713008625858\\
206.01	0.00504706557580845\\
207.01	0.00504699985224826\\
208.01	0.00504693289361077\\
209.01	0.00504686467755405\\
210.01	0.00504679518135547\\
211.01	0.00504672438190567\\
212.01	0.00504665225570302\\
213.01	0.00504657877884728\\
214.01	0.00504650392703367\\
215.01	0.00504642767554666\\
216.01	0.00504634999925358\\
217.01	0.00504627087259842\\
218.01	0.00504619026959572\\
219.01	0.00504610816382335\\
220.01	0.00504602452841662\\
221.01	0.00504593933606141\\
222.01	0.00504585255898741\\
223.01	0.00504576416896103\\
224.01	0.00504567413727911\\
225.01	0.00504558243476158\\
226.01	0.00504548903174409\\
227.01	0.00504539389807151\\
228.01	0.00504529700309017\\
229.01	0.00504519831564083\\
230.01	0.00504509780405106\\
231.01	0.00504499543612776\\
232.01	0.00504489117914962\\
233.01	0.00504478499985944\\
234.01	0.0050446768644564\\
235.01	0.00504456673858798\\
236.01	0.00504445458734198\\
237.01	0.00504434037523866\\
238.01	0.00504422406622267\\
239.01	0.00504410562365426\\
240.01	0.0050439850103012\\
241.01	0.00504386218833065\\
242.01	0.00504373711929989\\
243.01	0.00504360976414827\\
244.01	0.00504348008318793\\
245.01	0.00504334803609521\\
246.01	0.00504321358190158\\
247.01	0.00504307667898427\\
248.01	0.00504293728505743\\
249.01	0.00504279535716248\\
250.01	0.00504265085165886\\
251.01	0.00504250372421412\\
252.01	0.00504235392979442\\
253.01	0.00504220142265456\\
254.01	0.00504204615632818\\
255.01	0.00504188808361737\\
256.01	0.00504172715658271\\
257.01	0.00504156332653268\\
258.01	0.00504139654401304\\
259.01	0.00504122675879636\\
260.01	0.00504105391987104\\
261.01	0.00504087797543049\\
262.01	0.00504069887286166\\
263.01	0.00504051655873399\\
264.01	0.00504033097878772\\
265.01	0.00504014207792223\\
266.01	0.00503994980018425\\
267.01	0.00503975408875577\\
268.01	0.00503955488594175\\
269.01	0.00503935213315789\\
270.01	0.00503914577091756\\
271.01	0.00503893573881922\\
272.01	0.00503872197553281\\
273.01	0.00503850441878714\\
274.01	0.00503828300535608\\
275.01	0.00503805767104399\\
276.01	0.0050378283506724\\
277.01	0.00503759497806512\\
278.01	0.0050373574860338\\
279.01	0.00503711580636279\\
280.01	0.0050368698697938\\
281.01	0.00503661960601027\\
282.01	0.00503636494362115\\
283.01	0.00503610581014482\\
284.01	0.00503584213199201\\
285.01	0.00503557383444909\\
286.01	0.00503530084165991\\
287.01	0.00503502307660814\\
288.01	0.00503474046109883\\
289.01	0.00503445291573879\\
290.01	0.00503416035991835\\
291.01	0.00503386271178994\\
292.01	0.00503355988824878\\
293.01	0.00503325180491076\\
294.01	0.00503293837609126\\
295.01	0.00503261951478264\\
296.01	0.00503229513263092\\
297.01	0.00503196513991238\\
298.01	0.0050316294455087\\
299.01	0.0050312879568819\\
300.01	0.00503094058004804\\
301.01	0.00503058721955057\\
302.01	0.00503022777843219\\
303.01	0.00502986215820591\\
304.01	0.00502949025882556\\
305.01	0.00502911197865478\\
306.01	0.00502872721443513\\
307.01	0.00502833586125309\\
308.01	0.00502793781250557\\
309.01	0.00502753295986485\\
310.01	0.00502712119324115\\
311.01	0.00502670240074511\\
312.01	0.0050262764686474\\
313.01	0.00502584328133831\\
314.01	0.00502540272128466\\
315.01	0.00502495466898543\\
316.01	0.00502449900292578\\
317.01	0.00502403559952918\\
318.01	0.00502356433310781\\
319.01	0.00502308507581059\\
320.01	0.00502259769756933\\
321.01	0.00502210206604349\\
322.01	0.00502159804656149\\
323.01	0.00502108550206058\\
324.01	0.00502056429302427\\
325.01	0.00502003427741649\\
326.01	0.00501949531061407\\
327.01	0.00501894724533653\\
328.01	0.00501838993157195\\
329.01	0.0050178232165013\\
330.01	0.005017246944419\\
331.01	0.00501666095665098\\
332.01	0.00501606509146876\\
333.01	0.00501545918400077\\
334.01	0.00501484306614038\\
335.01	0.00501421656644987\\
336.01	0.00501357951006183\\
337.01	0.00501293171857526\\
338.01	0.00501227300994974\\
339.01	0.00501160319839449\\
340.01	0.00501092209425414\\
341.01	0.00501022950389015\\
342.01	0.00500952522955797\\
343.01	0.00500880906928109\\
344.01	0.00500808081671953\\
345.01	0.00500734026103516\\
346.01	0.00500658718675241\\
347.01	0.00500582137361472\\
348.01	0.00500504259643691\\
349.01	0.00500425062495316\\
350.01	0.00500344522366115\\
351.01	0.00500262615166124\\
352.01	0.00500179316249306\\
353.01	0.00500094600396683\\
354.01	0.0050000844179919\\
355.01	0.00499920814040097\\
356.01	0.00499831690077129\\
357.01	0.00499741042224258\\
358.01	0.00499648842133201\\
359.01	0.0049955506077466\\
360.01	0.00499459668419335\\
361.01	0.00499362634618713\\
362.01	0.00499263928185728\\
363.01	0.00499163517175377\\
364.01	0.00499061368865077\\
365.01	0.00498957449735275\\
366.01	0.00498851725449865\\
367.01	0.00498744160836883\\
368.01	0.00498634719869235\\
369.01	0.00498523365645737\\
370.01	0.00498410060372413\\
371.01	0.00498294765344128\\
372.01	0.004981774409267\\
373.01	0.00498058046539411\\
374.01	0.00497936540638164\\
375.01	0.00497812880699148\\
376.01	0.00497687023203234\\
377.01	0.00497558923621056\\
378.01	0.00497428536398801\\
379.01	0.00497295814944752\\
380.01	0.00497160711616688\\
381.01	0.00497023177709904\\
382.01	0.00496883163446029\\
383.01	0.00496740617962532\\
384.01	0.00496595489302728\\
385.01	0.00496447724406343\\
386.01	0.00496297269100318\\
387.01	0.00496144068089814\\
388.01	0.00495988064949051\\
389.01	0.0049582920211187\\
390.01	0.00495667420861591\\
391.01	0.00495502661319822\\
392.01	0.00495334862434032\\
393.01	0.00495163961963088\\
394.01	0.0049498989646079\\
395.01	0.00494812601256582\\
396.01	0.00494632010433079\\
397.01	0.00494448056800078\\
398.01	0.00494260671864545\\
399.01	0.00494069785796225\\
400.01	0.00493875327388629\\
401.01	0.00493677224015327\\
402.01	0.00493475401581408\\
403.01	0.00493269784470363\\
404.01	0.00493060295486799\\
405.01	0.00492846855795484\\
406.01	0.00492629384857654\\
407.01	0.00492407800365421\\
408.01	0.00492182018175633\\
409.01	0.00491951952244241\\
410.01	0.00491717514562545\\
411.01	0.00491478615096051\\
412.01	0.0049123516172678\\
413.01	0.00490987060198812\\
414.01	0.0049073421406633\\
415.01	0.00490476524642877\\
416.01	0.00490213890950422\\
417.01	0.00489946209668025\\
418.01	0.00489673375080099\\
419.01	0.00489395279024194\\
420.01	0.00489111810838552\\
421.01	0.00488822857309285\\
422.01	0.0048852830261742\\
423.01	0.00488228028285686\\
424.01	0.00487921913125238\\
425.01	0.00487609833182316\\
426.01	0.0048729166168476\\
427.01	0.00486967268988787\\
428.01	0.00486636522525517\\
429.01	0.00486299286747915\\
430.01	0.00485955423077612\\
431.01	0.00485604789852002\\
432.01	0.00485247242271422\\
433.01	0.00484882632346511\\
434.01	0.00484510808845668\\
435.01	0.0048413161724264\\
436.01	0.00483744899664086\\
437.01	0.00483350494837175\\
438.01	0.00482948238037028\\
439.01	0.00482537961033915\\
440.01	0.00482119492040131\\
441.01	0.00481692655656285\\
442.01	0.00481257272816965\\
443.01	0.00480813160735409\\
444.01	0.00480360132847085\\
445.01	0.00479897998751858\\
446.01	0.00479426564154536\\
447.01	0.00478945630803472\\
448.01	0.00478454996426917\\
449.01	0.00477954454666905\\
450.01	0.00477443795010257\\
451.01	0.00476922802716536\\
452.01	0.00476391258742675\\
453.01	0.0047584893966391\\
454.01	0.00475295617590996\\
455.01	0.0047473106008352\\
456.01	0.00474155030059124\\
457.01	0.00473567285698744\\
458.01	0.00472967580347889\\
459.01	0.00472355662414201\\
460.01	0.00471731275261322\\
461.01	0.00471094157099698\\
462.01	0.00470444040874411\\
463.01	0.00469780654150765\\
464.01	0.00469103718997992\\
465.01	0.00468412951871555\\
466.01	0.00467708063494494\\
467.01	0.00466988758738365\\
468.01	0.00466254736503701\\
469.01	0.00465505689600287\\
470.01	0.0046474130462699\\
471.01	0.00463961261850789\\
472.01	0.00463165235084363\\
473.01	0.00462352891561738\\
474.01	0.00461523891810933\\
475.01	0.00460677889523471\\
476.01	0.00459814531420051\\
477.01	0.0045893345711252\\
478.01	0.00458034298961881\\
479.01	0.00457116681932295\\
480.01	0.00456180223441107\\
481.01	0.0045522453320476\\
482.01	0.00454249213080608\\
483.01	0.00453253856904761\\
484.01	0.00452238050325833\\
485.01	0.00451201370634854\\
486.01	0.00450143386591313\\
487.01	0.00449063658245445\\
488.01	0.0044796173675692\\
489.01	0.00446837164210053\\
490.01	0.00445689473425607\\
491.01	0.00444518187769261\\
492.01	0.00443322820956902\\
493.01	0.00442102876856632\\
494.01	0.00440857849287668\\
495.01	0.00439587221815883\\
496.01	0.00438290467546104\\
497.01	0.00436967048910916\\
498.01	0.00435616417455901\\
499.01	0.0043423801362124\\
500.01	0.00432831266519476\\
501.01	0.0043139559370941\\
502.01	0.00429930400966176\\
503.01	0.00428435082047391\\
504.01	0.0042690901845561\\
505.01	0.00425351579197043\\
506.01	0.00423762120536703\\
507.01	0.00422139985750059\\
508.01	0.00420484504871337\\
509.01	0.00418794994438539\\
510.01	0.00417070757235356\\
511.01	0.00415311082030088\\
512.01	0.00413515243311747\\
513.01	0.00411682501023545\\
514.01	0.00409812100293835\\
515.01	0.00407903271164887\\
516.01	0.00405955228319665\\
517.01	0.00403967170806943\\
518.01	0.00401938281765047\\
519.01	0.00399867728144732\\
520.01	0.00397754660431566\\
521.01	0.00395598212368432\\
522.01	0.00393397500678653\\
523.01	0.00391151624790538\\
524.01	0.0038885966656406\\
525.01	0.00386520690020487\\
526.01	0.0038413374107603\\
527.01	0.00381697847280496\\
528.01	0.0037921201756219\\
529.01	0.00376675241980425\\
530.01	0.00374086491487127\\
531.01	0.00371444717699319\\
532.01	0.00368748852684237\\
533.01	0.00365997808759294\\
534.01	0.00363190478309231\\
535.01	0.00360325733623105\\
536.01	0.00357402426753979\\
537.01	0.0035441938940475\\
538.01	0.00351375432843491\\
539.01	0.00348269347852603\\
540.01	0.00345099904716186\\
541.01	0.00341865853250594\\
542.01	0.00338565922883782\\
543.01	0.00335198822789567\\
544.01	0.00331763242083754\\
545.01	0.00328257850089516\\
546.01	0.00324681296680706\\
547.01	0.00321032212712155\\
548.01	0.00317309210547439\\
549.01	0.00313510884695503\\
550.01	0.00309635812568573\\
551.01	0.00305682555375531\\
552.01	0.00301649659165863\\
553.01	0.00297535656041113\\
554.01	0.0029333906555258\\
555.01	0.00289058396305458\\
556.01	0.00284692147791951\\
557.01	0.00280238812477814\\
558.01	0.00275696878169113\\
559.01	0.00271064830688349\\
560.01	0.00266341156891711\\
561.01	0.00261524348061712\\
562.01	0.00256612903712336\\
563.01	0.00251605335846367\\
564.01	0.00246500173707516\\
565.01	0.00241295969072369\\
566.01	0.00235991302129535\\
567.01	0.0023058478799553\\
568.01	0.00225075083918186\\
569.01	0.00219460897218979\\
570.01	0.0021374099402512\\
571.01	0.00207914208840051\\
572.01	0.00201979454996673\\
573.01	0.00195935736030474\\
574.01	0.00189782157998909\\
575.01	0.00183517942757606\\
576.01	0.00177142442182105\\
577.01	0.00170655153293824\\
578.01	0.00164055734208577\\
579.01	0.0015734402077281\\
580.01	0.00150520043682321\\
581.01	0.00143584045787427\\
582.01	0.00136536499170297\\
583.01	0.00129378121428826\\
584.01	0.00122109890407295\\
585.01	0.00114733056366669\\
586.01	0.0010724915027304\\
587.01	0.000996599864842329\\
588.01	0.000919676576107574\\
589.01	0.000841745186917533\\
590.01	0.000762831570259378\\
591.01	0.00068296342990758\\
592.01	0.00060216955918722\\
593.01	0.000520478775138188\\
594.01	0.000437918433036224\\
595.01	0.000354512401346254\\
596.01	0.000270278346057853\\
597.01	0.000185224134443608\\
598.01	9.93507029036552e-05\\
599.01	3.18230442563749e-05\\
599.02	3.12770957175551e-05\\
599.03	3.07343620567849e-05\\
599.04	3.01948752175693e-05\\
599.05	2.96586674565693e-05\\
599.06	2.91257713466771e-05\\
599.07	2.85962197801373e-05\\
599.08	2.80700459716933e-05\\
599.09	2.75472834617447e-05\\
599.1	2.70279661195791e-05\\
599.11	2.65121281465847e-05\\
599.12	2.59998040795448e-05\\
599.13	2.54910287939124e-05\\
599.14	2.49858375071695e-05\\
599.15	2.4484265782181e-05\\
599.16	2.39863495305973e-05\\
599.17	2.34921250162837e-05\\
599.18	2.30016288588035e-05\\
599.19	2.25148980369013e-05\\
599.2	2.20319698920508e-05\\
599.21	2.1552882132023e-05\\
599.22	2.10776728344891e-05\\
599.23	2.06063804506721e-05\\
599.24	2.01390438090109e-05\\
599.25	1.96757021188876e-05\\
599.26	1.92163949743647e-05\\
599.27	1.87611623579872e-05\\
599.28	1.83100446445941e-05\\
599.29	1.78630826051934e-05\\
599.3	1.74203174108517e-05\\
599.31	1.69817906366353e-05\\
599.32	1.65475442655914e-05\\
599.33	1.61176206927693e-05\\
599.34	1.56920627292622e-05\\
599.35	1.52709136063186e-05\\
599.36	1.48542169794725e-05\\
599.37	1.4442016932719e-05\\
599.38	1.40343579827368e-05\\
599.39	1.36312859119591e-05\\
599.4	1.32328499381877e-05\\
599.41	1.28390997671621e-05\\
599.42	1.2450085597351e-05\\
599.43	1.20658581248094e-05\\
599.44	1.16864685480531e-05\\
599.45	1.13119685730082e-05\\
599.46	1.09424104179929e-05\\
599.47	1.05778468187639e-05\\
599.48	1.02183310335888e-05\\
599.49	9.86391684839466e-06\\
599.5	9.51465858194112e-06\\
599.51	9.17061109107116e-06\\
599.52	8.83182977599352e-06\\
599.53	8.4983705856221e-06\\
599.54	8.1702900229675e-06\\
599.55	7.84764515058579e-06\\
599.56	7.53049359608279e-06\\
599.57	7.21889355765649e-06\\
599.58	6.91290380971064e-06\\
599.59	6.61258370851861e-06\\
599.6	6.31799319793756e-06\\
599.61	6.02919281519031e-06\\
599.62	5.74624369668701e-06\\
599.63	5.46920758391981e-06\\
599.64	5.19814682940593e-06\\
599.65	4.93312440269685e-06\\
599.66	4.67420389643411e-06\\
599.67	4.42144953247646e-06\\
599.68	4.17492616808582e-06\\
599.69	3.93469930216536e-06\\
599.7	3.70083508156871e-06\\
599.71	3.47340030746116e-06\\
599.72	3.25246244175549e-06\\
599.73	3.03808961360161e-06\\
599.74	2.83035062593855e-06\\
599.75	2.62931496212288e-06\\
599.76	2.43505279260738e-06\\
599.77	2.24763498169606e-06\\
599.78	2.06713309435641e-06\\
599.79	1.89361940310974e-06\\
599.8	1.72716689497872e-06\\
599.81	1.56784927850956e-06\\
599.82	1.41574099085488e-06\\
599.83	1.27091720493813e-06\\
599.84	1.13345383668563e-06\\
599.85	1.00342755231589e-06\\
599.86	8.8091577571392e-07\\
599.87	7.65996695881871e-07\\
599.88	6.58749274441706e-07\\
599.89	5.59253253243699e-07\\
599.9	4.67589162013102e-07\\
599.91	3.83838326099145e-07\\
599.92	3.08082874293444e-07\\
599.93	2.40405746710845e-07\\
599.94	1.8089070277609e-07\\
599.95	1.2962232925906e-07\\
599.96	8.66860484019516e-08\\
599.97	5.21681261349272e-08\\
599.98	2.61556803542173e-08\\
599.99	8.73668930083393e-09\\
600	0\\
};
\addplot [color=black!50!mycolor20,solid,forget plot]
  table[row sep=crcr]{%
0.01	0.00503757637221215\\
1.01	0.00503757530562437\\
2.01	0.00503757421782917\\
3.01	0.00503757310840824\\
4.01	0.00503757197693535\\
5.01	0.00503757082297613\\
6.01	0.00503756964608718\\
7.01	0.00503756844581672\\
8.01	0.00503756722170427\\
9.01	0.00503756597328036\\
10.01	0.00503756470006631\\
11.01	0.00503756340157425\\
12.01	0.00503756207730662\\
13.01	0.0050375607267563\\
14.01	0.00503755934940636\\
15.01	0.00503755794472965\\
16.01	0.0050375565121889\\
17.01	0.00503755505123641\\
18.01	0.00503755356131391\\
19.01	0.00503755204185207\\
20.01	0.0050375504922708\\
21.01	0.00503754891197831\\
22.01	0.00503754730037193\\
23.01	0.00503754565683676\\
24.01	0.00503754398074654\\
25.01	0.00503754227146226\\
26.01	0.00503754052833307\\
27.01	0.00503753875069508\\
28.01	0.00503753693787184\\
29.01	0.00503753508917366\\
30.01	0.0050375332038976\\
31.01	0.00503753128132708\\
32.01	0.00503752932073171\\
33.01	0.00503752732136696\\
34.01	0.00503752528247392\\
35.01	0.00503752320327906\\
36.01	0.00503752108299393\\
37.01	0.00503751892081476\\
38.01	0.00503751671592218\\
39.01	0.0050375144674814\\
40.01	0.00503751217464121\\
41.01	0.00503750983653403\\
42.01	0.00503750745227585\\
43.01	0.00503750502096533\\
44.01	0.00503750254168386\\
45.01	0.00503750001349514\\
46.01	0.00503749743544505\\
47.01	0.00503749480656079\\
48.01	0.00503749212585129\\
49.01	0.00503748939230592\\
50.01	0.00503748660489518\\
51.01	0.00503748376256947\\
52.01	0.00503748086425903\\
53.01	0.00503747790887379\\
54.01	0.00503747489530284\\
55.01	0.00503747182241355\\
56.01	0.00503746868905205\\
57.01	0.00503746549404199\\
58.01	0.00503746223618486\\
59.01	0.00503745891425895\\
60.01	0.00503745552701947\\
61.01	0.0050374520731974\\
62.01	0.00503744855150008\\
63.01	0.00503744496060934\\
64.01	0.00503744129918257\\
65.01	0.00503743756585114\\
66.01	0.00503743375922066\\
67.01	0.00503742987786966\\
68.01	0.00503742592034982\\
69.01	0.00503742188518542\\
70.01	0.00503741777087245\\
71.01	0.00503741357587815\\
72.01	0.00503740929864081\\
73.01	0.00503740493756893\\
74.01	0.00503740049104078\\
75.01	0.00503739595740386\\
76.01	0.00503739133497405\\
77.01	0.00503738662203546\\
78.01	0.00503738181683963\\
79.01	0.00503737691760501\\
80.01	0.00503737192251585\\
81.01	0.0050373668297227\\
82.01	0.00503736163734053\\
83.01	0.00503735634344888\\
84.01	0.00503735094609075\\
85.01	0.00503734544327243\\
86.01	0.00503733983296231\\
87.01	0.00503733411309024\\
88.01	0.00503732828154737\\
89.01	0.00503732233618485\\
90.01	0.00503731627481322\\
91.01	0.00503731009520176\\
92.01	0.00503730379507781\\
93.01	0.00503729737212582\\
94.01	0.00503729082398685\\
95.01	0.00503728414825724\\
96.01	0.00503727734248828\\
97.01	0.00503727040418533\\
98.01	0.00503726333080657\\
99.01	0.00503725611976281\\
100.01	0.00503724876841621\\
101.01	0.00503724127407932\\
102.01	0.00503723363401428\\
103.01	0.00503722584543207\\
104.01	0.00503721790549134\\
105.01	0.00503720981129765\\
106.01	0.00503720155990228\\
107.01	0.00503719314830158\\
108.01	0.0050371845734357\\
109.01	0.00503717583218735\\
110.01	0.00503716692138161\\
111.01	0.00503715783778399\\
112.01	0.00503714857809979\\
113.01	0.0050371391389728\\
114.01	0.00503712951698459\\
115.01	0.00503711970865275\\
116.01	0.0050371097104304\\
117.01	0.00503709951870469\\
118.01	0.00503708912979552\\
119.01	0.00503707853995445\\
120.01	0.00503706774536366\\
121.01	0.00503705674213456\\
122.01	0.00503704552630625\\
123.01	0.00503703409384467\\
124.01	0.00503702244064106\\
125.01	0.0050370105625105\\
126.01	0.00503699845519107\\
127.01	0.00503698611434173\\
128.01	0.00503697353554149\\
129.01	0.00503696071428767\\
130.01	0.00503694764599466\\
131.01	0.0050369343259925\\
132.01	0.00503692074952483\\
133.01	0.00503690691174793\\
134.01	0.00503689280772931\\
135.01	0.00503687843244535\\
136.01	0.00503686378078044\\
137.01	0.00503684884752489\\
138.01	0.00503683362737365\\
139.01	0.00503681811492427\\
140.01	0.0050368023046754\\
141.01	0.00503678619102482\\
142.01	0.0050367697682679\\
143.01	0.00503675303059582\\
144.01	0.00503673597209362\\
145.01	0.00503671858673818\\
146.01	0.0050367008683967\\
147.01	0.00503668281082442\\
148.01	0.0050366644076628\\
149.01	0.00503664565243758\\
150.01	0.00503662653855665\\
151.01	0.00503660705930822\\
152.01	0.0050365872078584\\
153.01	0.00503656697724912\\
154.01	0.00503654636039616\\
155.01	0.00503652535008687\\
156.01	0.00503650393897796\\
157.01	0.00503648211959321\\
158.01	0.00503645988432092\\
159.01	0.00503643722541213\\
160.01	0.00503641413497769\\
161.01	0.00503639060498585\\
162.01	0.00503636662726021\\
163.01	0.00503634219347696\\
164.01	0.00503631729516229\\
165.01	0.00503629192368973\\
166.01	0.00503626607027794\\
167.01	0.00503623972598729\\
168.01	0.00503621288171818\\
169.01	0.00503618552820714\\
170.01	0.00503615765602495\\
171.01	0.00503612925557314\\
172.01	0.00503610031708122\\
173.01	0.00503607083060432\\
174.01	0.00503604078601939\\
175.01	0.00503601017302279\\
176.01	0.00503597898112645\\
177.01	0.00503594719965542\\
178.01	0.00503591481774434\\
179.01	0.00503588182433441\\
180.01	0.00503584820816981\\
181.01	0.00503581395779453\\
182.01	0.00503577906154872\\
183.01	0.0050357435075658\\
184.01	0.00503570728376812\\
185.01	0.00503567037786405\\
186.01	0.00503563277734397\\
187.01	0.00503559446947682\\
188.01	0.00503555544130603\\
189.01	0.00503551567964594\\
190.01	0.0050354751710778\\
191.01	0.00503543390194573\\
192.01	0.00503539185835323\\
193.01	0.0050353490261584\\
194.01	0.00503530539097027\\
195.01	0.00503526093814411\\
196.01	0.00503521565277792\\
197.01	0.00503516951970722\\
198.01	0.00503512252350134\\
199.01	0.00503507464845814\\
200.01	0.00503502587860027\\
201.01	0.00503497619767009\\
202.01	0.00503492558912484\\
203.01	0.00503487403613193\\
204.01	0.00503482152156468\\
205.01	0.00503476802799605\\
206.01	0.00503471353769469\\
207.01	0.00503465803261952\\
208.01	0.00503460149441431\\
209.01	0.00503454390440248\\
210.01	0.00503448524358185\\
211.01	0.00503442549261899\\
212.01	0.00503436463184354\\
213.01	0.00503430264124295\\
214.01	0.00503423950045613\\
215.01	0.00503417518876822\\
216.01	0.00503410968510407\\
217.01	0.00503404296802266\\
218.01	0.00503397501571035\\
219.01	0.00503390580597508\\
220.01	0.00503383531623984\\
221.01	0.00503376352353603\\
222.01	0.00503369040449694\\
223.01	0.00503361593535149\\
224.01	0.00503354009191626\\
225.01	0.00503346284958959\\
226.01	0.00503338418334417\\
227.01	0.00503330406771952\\
228.01	0.00503322247681503\\
229.01	0.00503313938428232\\
230.01	0.00503305476331757\\
231.01	0.00503296858665383\\
232.01	0.00503288082655331\\
233.01	0.00503279145479899\\
234.01	0.00503270044268673\\
235.01	0.00503260776101671\\
236.01	0.00503251338008532\\
237.01	0.00503241726967597\\
238.01	0.00503231939905068\\
239.01	0.00503221973694097\\
240.01	0.00503211825153886\\
241.01	0.00503201491048714\\
242.01	0.00503190968087044\\
243.01	0.00503180252920504\\
244.01	0.00503169342142931\\
245.01	0.00503158232289353\\
246.01	0.00503146919834986\\
247.01	0.00503135401194118\\
248.01	0.0050312367271912\\
249.01	0.00503111730699308\\
250.01	0.00503099571359866\\
251.01	0.00503087190860672\\
252.01	0.00503074585295193\\
253.01	0.00503061750689229\\
254.01	0.00503048682999773\\
255.01	0.00503035378113773\\
256.01	0.00503021831846828\\
257.01	0.00503008039941939\\
258.01	0.00502993998068197\\
259.01	0.00502979701819428\\
260.01	0.0050296514671284\\
261.01	0.00502950328187615\\
262.01	0.00502935241603455\\
263.01	0.00502919882239165\\
264.01	0.00502904245291109\\
265.01	0.00502888325871732\\
266.01	0.00502872119007941\\
267.01	0.00502855619639554\\
268.01	0.00502838822617612\\
269.01	0.00502821722702749\\
270.01	0.00502804314563448\\
271.01	0.00502786592774291\\
272.01	0.00502768551814172\\
273.01	0.00502750186064476\\
274.01	0.00502731489807084\\
275.01	0.00502712457222613\\
276.01	0.00502693082388276\\
277.01	0.00502673359275983\\
278.01	0.0050265328175017\\
279.01	0.00502632843565699\\
280.01	0.0050261203836568\\
281.01	0.00502590859679258\\
282.01	0.00502569300919249\\
283.01	0.00502547355379831\\
284.01	0.00502525016234141\\
285.01	0.00502502276531742\\
286.01	0.00502479129196121\\
287.01	0.00502455567022076\\
288.01	0.00502431582673\\
289.01	0.00502407168678189\\
290.01	0.00502382317429955\\
291.01	0.00502357021180749\\
292.01	0.00502331272040208\\
293.01	0.0050230506197208\\
294.01	0.0050227838279105\\
295.01	0.00502251226159539\\
296.01	0.00502223583584404\\
297.01	0.00502195446413479\\
298.01	0.00502166805832143\\
299.01	0.0050213765285966\\
300.01	0.00502107978345517\\
301.01	0.00502077772965604\\
302.01	0.0050204702721828\\
303.01	0.0050201573142045\\
304.01	0.00501983875703335\\
305.01	0.00501951450008285\\
306.01	0.00501918444082401\\
307.01	0.00501884847474066\\
308.01	0.005018506495283\\
309.01	0.00501815839382062\\
310.01	0.00501780405959352\\
311.01	0.00501744337966192\\
312.01	0.00501707623885547\\
313.01	0.00501670251971957\\
314.01	0.00501632210246132\\
315.01	0.00501593486489365\\
316.01	0.00501554068237779\\
317.01	0.00501513942776458\\
318.01	0.00501473097133332\\
319.01	0.00501431518072992\\
320.01	0.00501389192090298\\
321.01	0.00501346105403746\\
322.01	0.00501302243948797\\
323.01	0.00501257593370905\\
324.01	0.00501212139018412\\
325.01	0.00501165865935276\\
326.01	0.00501118758853553\\
327.01	0.00501070802185731\\
328.01	0.00501021980016853\\
329.01	0.00500972276096439\\
330.01	0.0050092167383022\\
331.01	0.00500870156271605\\
332.01	0.00500817706113033\\
333.01	0.00500764305677077\\
334.01	0.00500709936907302\\
335.01	0.0050065458135899\\
336.01	0.00500598220189525\\
337.01	0.00500540834148718\\
338.01	0.0050048240356881\\
339.01	0.00500422908354254\\
340.01	0.00500362327971341\\
341.01	0.00500300641437589\\
342.01	0.0050023782731091\\
343.01	0.00500173863678528\\
344.01	0.00500108728145748\\
345.01	0.00500042397824491\\
346.01	0.00499974849321595\\
347.01	0.00499906058726931\\
348.01	0.00499836001601358\\
349.01	0.0049976465296438\\
350.01	0.00499691987281695\\
351.01	0.00499617978452547\\
352.01	0.00499542599796851\\
353.01	0.00499465824042179\\
354.01	0.00499387623310557\\
355.01	0.00499307969105064\\
356.01	0.00499226832296384\\
357.01	0.0049914418310903\\
358.01	0.00499059991107576\\
359.01	0.00498974225182652\\
360.01	0.00498886853536857\\
361.01	0.00498797843670493\\
362.01	0.00498707162367234\\
363.01	0.00498614775679577\\
364.01	0.00498520648914337\\
365.01	0.00498424746617824\\
366.01	0.00498327032561077\\
367.01	0.00498227469724855\\
368.01	0.00498126020284642\\
369.01	0.00498022645595387\\
370.01	0.00497917306176225\\
371.01	0.00497809961694975\\
372.01	0.00497700570952573\\
373.01	0.00497589091867262\\
374.01	0.00497475481458582\\
375.01	0.0049735969583125\\
376.01	0.00497241690158675\\
377.01	0.00497121418666233\\
378.01	0.00496998834614267\\
379.01	0.00496873890280737\\
380.01	0.00496746536943322\\
381.01	0.00496616724861297\\
382.01	0.00496484403256673\\
383.01	0.00496349520294922\\
384.01	0.0049621202306498\\
385.01	0.00496071857558527\\
386.01	0.00495928968648604\\
387.01	0.00495783300067191\\
388.01	0.00495634794382075\\
389.01	0.00495483392972596\\
390.01	0.00495329036004352\\
391.01	0.00495171662402862\\
392.01	0.00495011209825996\\
393.01	0.00494847614635249\\
394.01	0.00494680811865763\\
395.01	0.0049451073519513\\
396.01	0.00494337316910967\\
397.01	0.00494160487877243\\
398.01	0.00493980177499517\\
399.01	0.00493796313689028\\
400.01	0.00493608822825808\\
401.01	0.00493417629720918\\
402.01	0.00493222657577901\\
403.01	0.00493023827953653\\
404.01	0.00492821060718771\\
405.01	0.00492614274017615\\
406.01	0.00492403384228107\\
407.01	0.00492188305921548\\
408.01	0.00491968951822292\\
409.01	0.0049174523276755\\
410.01	0.00491517057667094\\
411.01	0.00491284333463032\\
412.01	0.00491046965089229\\
413.01	0.00490804855430651\\
414.01	0.0049055790528206\\
415.01	0.00490306013306335\\
416.01	0.00490049075992149\\
417.01	0.00489786987611105\\
418.01	0.00489519640174224\\
419.01	0.00489246923387906\\
420.01	0.0048896872460923\\
421.01	0.00488684928800671\\
422.01	0.00488395418484163\\
423.01	0.00488100073694514\\
424.01	0.00487798771932154\\
425.01	0.0048749138811516\\
426.01	0.00487177794530663\\
427.01	0.00486857860785368\\
428.01	0.00486531453755403\\
429.01	0.00486198437535299\\
430.01	0.00485858673386123\\
431.01	0.00485512019682707\\
432.01	0.00485158331859944\\
433.01	0.00484797462358041\\
434.01	0.00484429260566794\\
435.01	0.00484053572768602\\
436.01	0.00483670242080498\\
437.01	0.00483279108394687\\
438.01	0.00482880008317991\\
439.01	0.00482472775109747\\
440.01	0.00482057238618295\\
441.01	0.00481633225215964\\
442.01	0.00481200557732326\\
443.01	0.0048075905538585\\
444.01	0.00480308533713759\\
445.01	0.00479848804500103\\
446.01	0.00479379675701847\\
447.01	0.00478900951373037\\
448.01	0.00478412431586967\\
449.01	0.00477913912356268\\
450.01	0.00477405185550922\\
451.01	0.00476886038814144\\
452.01	0.00476356255476095\\
453.01	0.0047581561446548\\
454.01	0.00475263890219039\\
455.01	0.00474700852588799\\
456.01	0.00474126266747335\\
457.01	0.00473539893090912\\
458.01	0.00472941487140539\\
459.01	0.00472330799441088\\
460.01	0.00471707575458347\\
461.01	0.00471071555474243\\
462.01	0.00470422474480017\\
463.01	0.00469760062067582\\
464.01	0.00469084042318818\\
465.01	0.00468394133692999\\
466.01	0.00467690048912133\\
467.01	0.0046697149484417\\
468.01	0.00466238172384067\\
469.01	0.00465489776332429\\
470.01	0.00464725995271764\\
471.01	0.00463946511440073\\
472.01	0.00463151000601805\\
473.01	0.00462339131915957\\
474.01	0.00461510567801282\\
475.01	0.00460664963798493\\
476.01	0.00459801968429604\\
477.01	0.00458921223054078\\
478.01	0.00458022361722044\\
479.01	0.00457105011024327\\
480.01	0.00456168789939413\\
481.01	0.00455213309677256\\
482.01	0.00454238173519966\\
483.01	0.00453242976659276\\
484.01	0.00452227306030811\\
485.01	0.0045119074014521\\
486.01	0.00450132848915897\\
487.01	0.004490531934837\\
488.01	0.00447951326038137\\
489.01	0.00446826789635372\\
490.01	0.00445679118012794\\
491.01	0.00444507835400288\\
492.01	0.00443312456327982\\
493.01	0.00442092485430535\\
494.01	0.00440847417247915\\
495.01	0.00439576736022571\\
496.01	0.00438279915493058\\
497.01	0.0043695641868392\\
498.01	0.00435605697691972\\
499.01	0.00434227193468841\\
500.01	0.00432820335599812\\
501.01	0.00431384542079002\\
502.01	0.0042991921908077\\
503.01	0.00428423760727525\\
504.01	0.00426897548853848\\
505.01	0.00425339952767048\\
506.01	0.0042375032900411\\
507.01	0.00422128021085203\\
508.01	0.00420472359263741\\
509.01	0.00418782660273114\\
510.01	0.00417058227070207\\
511.01	0.00415298348575844\\
512.01	0.00413502299412217\\
513.01	0.00411669339637638\\
514.01	0.00409798714478656\\
515.01	0.0040788965405988\\
516.01	0.00405941373131771\\
517.01	0.00403953070796672\\
518.01	0.00401923930233447\\
519.01	0.00399853118421253\\
520.01	0.00397739785862717\\
521.01	0.00395583066307222\\
522.01	0.00393382076474854\\
523.01	0.00391135915781609\\
524.01	0.00388843666066737\\
525.01	0.00386504391323036\\
526.01	0.00384117137431015\\
527.01	0.00381680931898115\\
528.01	0.00379194783604039\\
529.01	0.00376657682553818\\
530.01	0.00374068599639765\\
531.01	0.0037142648641438\\
532.01	0.00368730274875859\\
533.01	0.00365978877268431\\
534.01	0.00363171185899915\\
535.01	0.00360306072979041\\
536.01	0.00357382390475556\\
537.01	0.00354398970006315\\
538.01	0.00351354622751146\\
539.01	0.0034824813940233\\
540.01	0.00345078290152427\\
541.01	0.00341843824725186\\
542.01	0.00338543472455447\\
543.01	0.00335175942423902\\
544.01	0.00331739923653709\\
545.01	0.00328234085376592\\
546.01	0.00324657077376695\\
547.01	0.00321007530421679\\
548.01	0.0031728405679133\\
549.01	0.00313485250914984\\
550.01	0.00309609690130523\\
551.01	0.00305655935578795\\
552.01	0.0030162253324876\\
553.01	0.00297508015190375\\
554.01	0.00293310900913688\\
555.01	0.00289029698994688\\
556.01	0.00284662908910202\\
557.01	0.00280209023126455\\
558.01	0.00275666529468013\\
559.01	0.00271033913796359\\
560.01	0.00266309663029763\\
561.01	0.00261492268538809\\
562.01	0.00256580229954686\\
563.01	0.0025157205942994\\
564.01	0.00246466286394252\\
565.01	0.00241261462850278\\
566.01	0.00235956169257031\\
567.01	0.00230549021050174\\
568.01	0.00225038675850025\\
569.01	0.00219423841408641\\
570.01	0.00213703284346752\\
571.01	0.00207875839729022\\
572.01	0.00201940421521853\\
573.01	0.00195896033970692\\
574.01	0.00189741783922914\\
575.01	0.00183476894106481\\
576.01	0.00177100717352635\\
577.01	0.00170612751720696\\
578.01	0.00164012656442551\\
579.01	0.0015730026855086\\
580.01	0.00150475619984723\\
581.01	0.00143538954874788\\
582.01	0.00136490746591783\\
583.01	0.00129331713990011\\
584.01	0.0012206283608285\\
585.01	0.00114685364138917\\
586.01	0.00107200829872149\\
587.01	0.000996110479992625\\
588.01	0.000919181109325396\\
589.01	0.000841243727382117\\
590.01	0.000762324186875217\\
591.01	0.000682450157175529\\
592.01	0.000601650378505039\\
593.01	0.000519953590291149\\
594.01	0.00043738703832388\\
595.01	0.00035397444039822\\
596.01	0.000269733258902364\\
597.01	0.000184671089782441\\
598.01	9.88225819852275e-05\\
599.01	3.18230442563731e-05\\
599.02	3.12770957175551e-05\\
599.03	3.07343620567849e-05\\
599.04	3.0194875217571e-05\\
599.05	2.96586674565693e-05\\
599.06	2.91257713466771e-05\\
599.07	2.85962197801391e-05\\
599.08	2.80700459716916e-05\\
599.09	2.75472834617447e-05\\
599.1	2.70279661195773e-05\\
599.11	2.65121281465865e-05\\
599.12	2.5999804079543e-05\\
599.13	2.54910287939142e-05\\
599.14	2.49858375071712e-05\\
599.15	2.44842657821827e-05\\
599.16	2.39863495305956e-05\\
599.17	2.34921250162855e-05\\
599.18	2.30016288588035e-05\\
599.19	2.25148980369013e-05\\
599.2	2.20319698920508e-05\\
599.21	2.15528821320213e-05\\
599.22	2.10776728344908e-05\\
599.23	2.06063804506721e-05\\
599.24	2.01390438090109e-05\\
599.25	1.96757021188858e-05\\
599.26	1.9216394974363e-05\\
599.27	1.87611623579855e-05\\
599.28	1.83100446445959e-05\\
599.29	1.78630826051952e-05\\
599.3	1.74203174108517e-05\\
599.31	1.69817906366353e-05\\
599.32	1.65475442655931e-05\\
599.33	1.61176206927693e-05\\
599.34	1.56920627292622e-05\\
599.35	1.52709136063203e-05\\
599.36	1.48542169794742e-05\\
599.37	1.4442016932719e-05\\
599.38	1.40343579827368e-05\\
599.39	1.36312859119591e-05\\
599.4	1.32328499381877e-05\\
599.41	1.28390997671604e-05\\
599.42	1.2450085597351e-05\\
599.43	1.20658581248094e-05\\
599.44	1.16864685480531e-05\\
599.45	1.13119685730065e-05\\
599.46	1.09424104179929e-05\\
599.47	1.05778468187639e-05\\
599.48	1.02183310335888e-05\\
599.49	9.86391684839293e-06\\
599.5	9.51465858193938e-06\\
599.51	9.17061109107289e-06\\
599.52	8.83182977599525e-06\\
599.53	8.4983705856221e-06\\
599.54	8.1702900229675e-06\\
599.55	7.84764515058753e-06\\
599.56	7.53049359608453e-06\\
599.57	7.21889355765649e-06\\
599.58	6.91290380970891e-06\\
599.59	6.61258370851688e-06\\
599.6	6.31799319793756e-06\\
599.61	6.02919281519031e-06\\
599.62	5.74624369668701e-06\\
599.63	5.46920758391807e-06\\
599.64	5.19814682940593e-06\\
599.65	4.93312440269685e-06\\
599.66	4.67420389643237e-06\\
599.67	4.42144953247646e-06\\
599.68	4.17492616808582e-06\\
599.69	3.9346993021671e-06\\
599.7	3.70083508156871e-06\\
599.71	3.47340030746289e-06\\
599.72	3.25246244175723e-06\\
599.73	3.03808961360161e-06\\
599.74	2.83035062593855e-06\\
599.75	2.62931496212288e-06\\
599.76	2.43505279260738e-06\\
599.77	2.24763498169606e-06\\
599.78	2.06713309435641e-06\\
599.79	1.89361940311147e-06\\
599.8	1.72716689498045e-06\\
599.81	1.56784927850782e-06\\
599.82	1.41574099085315e-06\\
599.83	1.27091720493813e-06\\
599.84	1.13345383668736e-06\\
599.85	1.00342755231415e-06\\
599.86	8.8091577571392e-07\\
599.87	7.65996695880136e-07\\
599.88	6.58749274441706e-07\\
599.89	5.59253253243699e-07\\
599.9	4.67589162011367e-07\\
599.91	3.83838326099145e-07\\
599.92	3.0808287429171e-07\\
599.93	2.40405746710845e-07\\
599.94	1.80890702777825e-07\\
599.95	1.2962232925906e-07\\
599.96	8.66860484002169e-08\\
599.97	5.21681261331924e-08\\
599.98	2.61556803542173e-08\\
599.99	8.73668930083393e-09\\
600	0\\
};
\addplot [color=black!60!mycolor21,solid,forget plot]
  table[row sep=crcr]{%
0.01	0.00503139859049536\\
1.01	0.00503139763734625\\
2.01	0.00503139666545196\\
3.01	0.00503139567444704\\
4.01	0.00503139466395896\\
5.01	0.00503139363360786\\
6.01	0.00503139258300674\\
7.01	0.00503139151176115\\
8.01	0.00503139041946888\\
9.01	0.00503138930572007\\
10.01	0.00503138817009698\\
11.01	0.00503138701217382\\
12.01	0.00503138583151661\\
13.01	0.005031384627683\\
14.01	0.00503138340022211\\
15.01	0.00503138214867432\\
16.01	0.00503138087257144\\
17.01	0.00503137957143602\\
18.01	0.00503137824478144\\
19.01	0.00503137689211191\\
20.01	0.00503137551292208\\
21.01	0.00503137410669669\\
22.01	0.00503137267291078\\
23.01	0.00503137121102927\\
24.01	0.00503136972050663\\
25.01	0.00503136820078729\\
26.01	0.0050313666513046\\
27.01	0.00503136507148146\\
28.01	0.00503136346072934\\
29.01	0.00503136181844857\\
30.01	0.00503136014402813\\
31.01	0.00503135843684492\\
32.01	0.00503135669626435\\
33.01	0.00503135492163941\\
34.01	0.0050313531123108\\
35.01	0.00503135126760659\\
36.01	0.0050313493868421\\
37.01	0.00503134746931938\\
38.01	0.00503134551432744\\
39.01	0.00503134352114138\\
40.01	0.00503134148902256\\
41.01	0.00503133941721832\\
42.01	0.00503133730496162\\
43.01	0.00503133515147055\\
44.01	0.00503133295594866\\
45.01	0.00503133071758422\\
46.01	0.00503132843554987\\
47.01	0.00503132610900244\\
48.01	0.00503132373708287\\
49.01	0.00503132131891577\\
50.01	0.00503131885360891\\
51.01	0.00503131634025312\\
52.01	0.00503131377792208\\
53.01	0.00503131116567176\\
54.01	0.0050313085025401\\
55.01	0.00503130578754686\\
56.01	0.00503130301969303\\
57.01	0.00503130019796089\\
58.01	0.00503129732131309\\
59.01	0.00503129438869273\\
60.01	0.0050312913990228\\
61.01	0.00503128835120597\\
62.01	0.00503128524412393\\
63.01	0.00503128207663747\\
64.01	0.0050312788475854\\
65.01	0.00503127555578462\\
66.01	0.00503127220002975\\
67.01	0.00503126877909265\\
68.01	0.0050312652917218\\
69.01	0.00503126173664195\\
70.01	0.00503125811255386\\
71.01	0.00503125441813366\\
72.01	0.0050312506520323\\
73.01	0.00503124681287535\\
74.01	0.00503124289926258\\
75.01	0.00503123890976706\\
76.01	0.0050312348429351\\
77.01	0.00503123069728553\\
78.01	0.00503122647130935\\
79.01	0.00503122216346869\\
80.01	0.00503121777219725\\
81.01	0.00503121329589863\\
82.01	0.00503120873294689\\
83.01	0.005031204081685\\
84.01	0.00503119934042499\\
85.01	0.00503119450744717\\
86.01	0.00503118958099935\\
87.01	0.0050311845592965\\
88.01	0.00503117944051983\\
89.01	0.00503117422281647\\
90.01	0.00503116890429895\\
91.01	0.00503116348304399\\
92.01	0.00503115795709251\\
93.01	0.00503115232444854\\
94.01	0.00503114658307843\\
95.01	0.00503114073091068\\
96.01	0.00503113476583492\\
97.01	0.0050311286857011\\
98.01	0.00503112248831905\\
99.01	0.00503111617145736\\
100.01	0.005031109732843\\
101.01	0.00503110317016036\\
102.01	0.00503109648105029\\
103.01	0.00503108966310976\\
104.01	0.00503108271389102\\
105.01	0.00503107563090009\\
106.01	0.00503106841159678\\
107.01	0.00503106105339315\\
108.01	0.005031053553653\\
109.01	0.00503104590969137\\
110.01	0.00503103811877243\\
111.01	0.00503103017810982\\
112.01	0.00503102208486511\\
113.01	0.00503101383614706\\
114.01	0.00503100542901042\\
115.01	0.00503099686045517\\
116.01	0.00503098812742551\\
117.01	0.00503097922680829\\
118.01	0.00503097015543318\\
119.01	0.00503096091007028\\
120.01	0.00503095148743003\\
121.01	0.00503094188416136\\
122.01	0.0050309320968513\\
123.01	0.00503092212202324\\
124.01	0.0050309119561359\\
125.01	0.00503090159558271\\
126.01	0.00503089103668979\\
127.01	0.00503088027571513\\
128.01	0.00503086930884764\\
129.01	0.00503085813220537\\
130.01	0.00503084674183433\\
131.01	0.00503083513370743\\
132.01	0.00503082330372289\\
133.01	0.00503081124770333\\
134.01	0.00503079896139353\\
135.01	0.00503078644045992\\
136.01	0.00503077368048869\\
137.01	0.00503076067698438\\
138.01	0.00503074742536853\\
139.01	0.00503073392097806\\
140.01	0.00503072015906363\\
141.01	0.00503070613478845\\
142.01	0.00503069184322624\\
143.01	0.00503067727935992\\
144.01	0.00503066243807978\\
145.01	0.00503064731418194\\
146.01	0.00503063190236643\\
147.01	0.00503061619723582\\
148.01	0.00503060019329304\\
149.01	0.00503058388493984\\
150.01	0.00503056726647493\\
151.01	0.00503055033209187\\
152.01	0.00503053307587752\\
153.01	0.00503051549180999\\
154.01	0.00503049757375628\\
155.01	0.00503047931547104\\
156.01	0.00503046071059376\\
157.01	0.00503044175264716\\
158.01	0.00503042243503509\\
159.01	0.00503040275103985\\
160.01	0.00503038269382059\\
161.01	0.00503036225641073\\
162.01	0.00503034143171598\\
163.01	0.00503032021251167\\
164.01	0.0050302985914408\\
165.01	0.00503027656101107\\
166.01	0.00503025411359278\\
167.01	0.00503023124141671\\
168.01	0.00503020793657069\\
169.01	0.00503018419099782\\
170.01	0.00503015999649337\\
171.01	0.00503013534470243\\
172.01	0.00503011022711691\\
173.01	0.00503008463507281\\
174.01	0.00503005855974766\\
175.01	0.00503003199215706\\
176.01	0.00503000492315242\\
177.01	0.00502997734341756\\
178.01	0.00502994924346589\\
179.01	0.00502992061363698\\
180.01	0.00502989144409368\\
181.01	0.0050298617248189\\
182.01	0.00502983144561235\\
183.01	0.00502980059608679\\
184.01	0.00502976916566513\\
185.01	0.00502973714357697\\
186.01	0.00502970451885467\\
187.01	0.00502967128032999\\
188.01	0.0050296374166304\\
189.01	0.00502960291617576\\
190.01	0.0050295677671735\\
191.01	0.00502953195761594\\
192.01	0.00502949547527527\\
193.01	0.00502945830770057\\
194.01	0.00502942044221267\\
195.01	0.00502938186590086\\
196.01	0.00502934256561803\\
197.01	0.00502930252797667\\
198.01	0.00502926173934435\\
199.01	0.0050292201858392\\
200.01	0.00502917785332523\\
201.01	0.00502913472740798\\
202.01	0.00502909079342936\\
203.01	0.00502904603646299\\
204.01	0.00502900044130903\\
205.01	0.0050289539924897\\
206.01	0.00502890667424345\\
207.01	0.00502885847051987\\
208.01	0.0050288093649748\\
209.01	0.00502875934096441\\
210.01	0.00502870838153951\\
211.01	0.00502865646944035\\
212.01	0.00502860358709064\\
213.01	0.00502854971659139\\
214.01	0.00502849483971531\\
215.01	0.0050284389379003\\
216.01	0.00502838199224336\\
217.01	0.00502832398349412\\
218.01	0.00502826489204852\\
219.01	0.00502820469794217\\
220.01	0.00502814338084295\\
221.01	0.00502808092004487\\
222.01	0.00502801729446075\\
223.01	0.0050279524826145\\
224.01	0.00502788646263481\\
225.01	0.00502781921224657\\
226.01	0.00502775070876387\\
227.01	0.00502768092908203\\
228.01	0.00502760984966948\\
229.01	0.00502753744655966\\
230.01	0.0050274636953428\\
231.01	0.00502738857115743\\
232.01	0.00502731204868156\\
233.01	0.005027234102124\\
234.01	0.00502715470521516\\
235.01	0.00502707383119783\\
236.01	0.00502699145281805\\
237.01	0.00502690754231499\\
238.01	0.00502682207141134\\
239.01	0.00502673501130347\\
240.01	0.00502664633265087\\
241.01	0.00502655600556584\\
242.01	0.0050264639996026\\
243.01	0.0050263702837467\\
244.01	0.00502627482640345\\
245.01	0.00502617759538634\\
246.01	0.00502607855790602\\
247.01	0.00502597768055802\\
248.01	0.00502587492931026\\
249.01	0.00502577026949106\\
250.01	0.00502566366577619\\
251.01	0.00502555508217566\\
252.01	0.00502544448202057\\
253.01	0.00502533182794971\\
254.01	0.0050252170818954\\
255.01	0.00502510020506895\\
256.01	0.00502498115794663\\
257.01	0.0050248599002545\\
258.01	0.00502473639095347\\
259.01	0.00502461058822324\\
260.01	0.00502448244944653\\
261.01	0.00502435193119314\\
262.01	0.00502421898920295\\
263.01	0.00502408357836889\\
264.01	0.00502394565271949\\
265.01	0.00502380516540097\\
266.01	0.0050236620686594\\
267.01	0.00502351631382134\\
268.01	0.00502336785127532\\
269.01	0.00502321663045207\\
270.01	0.00502306259980404\\
271.01	0.00502290570678556\\
272.01	0.00502274589783178\\
273.01	0.0050225831183366\\
274.01	0.00502241731263146\\
275.01	0.00502224842396228\\
276.01	0.00502207639446676\\
277.01	0.00502190116515079\\
278.01	0.0050217226758641\\
279.01	0.00502154086527589\\
280.01	0.00502135567084918\\
281.01	0.00502116702881495\\
282.01	0.00502097487414639\\
283.01	0.00502077914053067\\
284.01	0.00502057976034186\\
285.01	0.00502037666461219\\
286.01	0.00502016978300333\\
287.01	0.00501995904377588\\
288.01	0.00501974437375977\\
289.01	0.00501952569832223\\
290.01	0.00501930294133615\\
291.01	0.00501907602514742\\
292.01	0.00501884487054113\\
293.01	0.00501860939670745\\
294.01	0.00501836952120628\\
295.01	0.00501812515993157\\
296.01	0.00501787622707447\\
297.01	0.00501762263508545\\
298.01	0.00501736429463557\\
299.01	0.00501710111457737\\
300.01	0.00501683300190418\\
301.01	0.00501655986170874\\
302.01	0.00501628159714126\\
303.01	0.00501599810936545\\
304.01	0.00501570929751445\\
305.01	0.00501541505864538\\
306.01	0.00501511528769338\\
307.01	0.00501480987742318\\
308.01	0.00501449871838138\\
309.01	0.00501418169884616\\
310.01	0.00501385870477673\\
311.01	0.0050135296197612\\
312.01	0.0050131943249632\\
313.01	0.00501285269906782\\
314.01	0.00501250461822576\\
315.01	0.00501214995599656\\
316.01	0.00501178858329041\\
317.01	0.00501142036830881\\
318.01	0.00501104517648384\\
319.01	0.00501066287041591\\
320.01	0.00501027330981074\\
321.01	0.00500987635141417\\
322.01	0.00500947184894646\\
323.01	0.00500905965303438\\
324.01	0.00500863961114235\\
325.01	0.00500821156750197\\
326.01	0.0050077753630402\\
327.01	0.00500733083530581\\
328.01	0.0050068778183947\\
329.01	0.00500641614287336\\
330.01	0.00500594563570093\\
331.01	0.00500546612014977\\
332.01	0.00500497741572432\\
333.01	0.00500447933807829\\
334.01	0.00500397169893088\\
335.01	0.00500345430598014\\
336.01	0.00500292696281612\\
337.01	0.00500238946883131\\
338.01	0.00500184161912962\\
339.01	0.0050012832044345\\
340.01	0.00500071401099379\\
341.01	0.00500013382048451\\
342.01	0.00499954240991468\\
343.01	0.00499893955152418\\
344.01	0.00499832501268307\\
345.01	0.0049976985557892\\
346.01	0.00499705993816283\\
347.01	0.0049964089119405\\
348.01	0.0049957452239657\\
349.01	0.00499506861567949\\
350.01	0.00499437882300773\\
351.01	0.0049936755762471\\
352.01	0.00499295859994831\\
353.01	0.00499222761279911\\
354.01	0.00499148232750315\\
355.01	0.00499072245065837\\
356.01	0.00498994768263262\\
357.01	0.00498915771743696\\
358.01	0.0049883522425973\\
359.01	0.00498753093902332\\
360.01	0.00498669348087552\\
361.01	0.00498583953542931\\
362.01	0.00498496876293744\\
363.01	0.00498408081648879\\
364.01	0.00498317534186573\\
365.01	0.00498225197739787\\
366.01	0.00498131035381339\\
367.01	0.0049803500940877\\
368.01	0.00497937081328793\\
369.01	0.00497837211841585\\
370.01	0.0049773536082462\\
371.01	0.00497631487316221\\
372.01	0.0049752554949872\\
373.01	0.0049741750468126\\
374.01	0.00497307309282207\\
375.01	0.00497194918811139\\
376.01	0.00497080287850422\\
377.01	0.00496963370036354\\
378.01	0.00496844118039817\\
379.01	0.00496722483546439\\
380.01	0.00496598417236356\\
381.01	0.00496471868763301\\
382.01	0.00496342786733346\\
383.01	0.00496211118682948\\
384.01	0.00496076811056514\\
385.01	0.00495939809183382\\
386.01	0.0049580005725418\\
387.01	0.00495657498296681\\
388.01	0.0049551207415093\\
389.01	0.00495363725443867\\
390.01	0.0049521239156333\\
391.01	0.00495058010631375\\
392.01	0.0049490051947711\\
393.01	0.00494739853608841\\
394.01	0.00494575947185656\\
395.01	0.00494408732988409\\
396.01	0.00494238142390204\\
397.01	0.00494064105326258\\
398.01	0.0049388655026322\\
399.01	0.00493705404168015\\
400.01	0.00493520592476195\\
401.01	0.00493332039059669\\
402.01	0.00493139666194075\\
403.01	0.00492943394525598\\
404.01	0.00492743143037305\\
405.01	0.00492538829015008\\
406.01	0.00492330368012597\\
407.01	0.00492117673816835\\
408.01	0.00491900658411652\\
409.01	0.00491679231941834\\
410.01	0.00491453302676086\\
411.01	0.00491222776969481\\
412.01	0.00490987559225262\\
413.01	0.0049074755185588\\
414.01	0.00490502655243384\\
415.01	0.00490252767699035\\
416.01	0.00489997785422148\\
417.01	0.00489737602458166\\
418.01	0.00489472110655943\\
419.01	0.00489201199624185\\
420.01	0.00488924756687061\\
421.01	0.00488642666838923\\
422.01	0.00488354812698167\\
423.01	0.00488061074460138\\
424.01	0.00487761329849112\\
425.01	0.0048745545406932\\
426.01	0.00487143319754917\\
427.01	0.00486824796918974\\
428.01	0.00486499752901395\\
429.01	0.00486168052315687\\
430.01	0.00485829556994715\\
431.01	0.00485484125935185\\
432.01	0.00485131615240987\\
433.01	0.00484771878065343\\
434.01	0.00484404764551618\\
435.01	0.00484030121772918\\
436.01	0.00483647793670301\\
437.01	0.00483257620989627\\
438.01	0.00482859441217073\\
439.01	0.00482453088513136\\
440.01	0.00482038393645248\\
441.01	0.00481615183918814\\
442.01	0.00481183283106798\\
443.01	0.00480742511377696\\
444.01	0.00480292685222\\
445.01	0.00479833617376934\\
446.01	0.00479365116749608\\
447.01	0.00478886988338537\\
448.01	0.00478399033153439\\
449.01	0.00477901048133291\\
450.01	0.00477392826062676\\
451.01	0.00476874155486356\\
452.01	0.00476344820622057\\
453.01	0.0047580460127148\\
454.01	0.00475253272729457\\
455.01	0.00474690605691219\\
456.01	0.00474116366157828\\
457.01	0.00473530315339724\\
458.01	0.00472932209558291\\
459.01	0.00472321800145414\\
460.01	0.0047169883334113\\
461.01	0.00471063050189128\\
462.01	0.00470414186430254\\
463.01	0.0046975197239378\\
464.01	0.00469076132886535\\
465.01	0.00468386387079798\\
466.01	0.00467682448393921\\
467.01	0.00466964024380527\\
468.01	0.00466230816602322\\
469.01	0.00465482520510529\\
470.01	0.00464718825319733\\
471.01	0.00463939413880171\\
472.01	0.00463143962547456\\
473.01	0.00462332141049567\\
474.01	0.00461503612351281\\
475.01	0.0046065803251571\\
476.01	0.00459795050563165\\
477.01	0.00458914308327175\\
478.01	0.00458015440307578\\
479.01	0.00457098073520809\\
480.01	0.00456161827347155\\
481.01	0.00455206313375069\\
482.01	0.00454231135242416\\
483.01	0.00453235888474604\\
484.01	0.00452220160319721\\
485.01	0.00451183529580319\\
486.01	0.00450125566442095\\
487.01	0.00449045832299321\\
488.01	0.00447943879576871\\
489.01	0.00446819251548963\\
490.01	0.00445671482154523\\
491.01	0.00444500095808993\\
492.01	0.00443304607212781\\
493.01	0.00442084521156108\\
494.01	0.00440839332320334\\
495.01	0.00439568525075659\\
496.01	0.00438271573275181\\
497.01	0.0043694794004541\\
498.01	0.00435597077572967\\
499.01	0.0043421842688769\\
500.01	0.00432811417642031\\
501.01	0.0043137546788676\\
502.01	0.00429909983842902\\
503.01	0.0042841435967016\\
504.01	0.00426887977231503\\
505.01	0.00425330205854258\\
506.01	0.00423740402087563\\
507.01	0.00422117909456325\\
508.01	0.00420462058211687\\
509.01	0.00418772165078221\\
510.01	0.00417047532997789\\
511.01	0.00415287450870299\\
512.01	0.00413491193291534\\
513.01	0.00411658020288142\\
514.01	0.00409787177050073\\
515.01	0.00407877893660669\\
516.01	0.00405929384824654\\
517.01	0.00403940849594464\\
518.01	0.00401911471095102\\
519.01	0.00399840416248059\\
520.01	0.00397726835494781\\
521.01	0.00395569862520106\\
522.01	0.00393368613976365\\
523.01	0.00391122189208767\\
524.01	0.00388829669982948\\
525.01	0.00386490120215339\\
526.01	0.00384102585707547\\
527.01	0.00381666093885656\\
528.01	0.00379179653545806\\
529.01	0.00376642254607224\\
530.01	0.00374052867874502\\
531.01	0.003714104448105\\
532.01	0.00368713917322073\\
533.01	0.00365962197560618\\
534.01	0.00363154177739699\\
535.01	0.00360288729972611\\
536.01	0.00357364706132669\\
537.01	0.00354380937739529\\
538.01	0.00351336235875247\\
539.01	0.00348229391134084\\
540.01	0.00345059173610505\\
541.01	0.00341824332930561\\
542.01	0.00338523598332041\\
543.01	0.00335155678799634\\
544.01	0.00331719263262028\\
545.01	0.00328213020858442\\
546.01	0.00324635601283099\\
547.01	0.00320985635216906\\
548.01	0.00317261734856691\\
549.01	0.0031346249455342\\
550.01	0.00309586491572126\\
551.01	0.00305632286987125\\
552.01	0.00301598426728306\\
553.01	0.00297483442795171\\
554.01	0.00293285854657185\\
555.01	0.00289004170861091\\
556.01	0.00284636890867354\\
557.01	0.00280182507140488\\
558.01	0.00275639507519943\\
559.01	0.00271006377900744\\
560.01	0.0026628160525571\\
561.01	0.00261463681033538\\
562.01	0.00256551104969777\\
563.01	0.00251542389350631\\
564.01	0.00246436063771941\\
565.01	0.00241230680438499\\
566.01	0.00235924820051124\\
567.01	0.00230517098330829\\
568.01	0.00225006173230929\\
569.01	0.00219390752888354\\
570.01	0.00213669604364802\\
571.01	0.00207841563226292\\
572.01	0.00201905544004984\\
573.01	0.00195860551580188\\
574.01	0.00189705693504289\\
575.01	0.0018344019328357\\
576.01	0.00177063404601639\\
577.01	0.00170574826442997\\
578.01	0.0016397411903371\\
579.01	0.00157261120462172\\
580.01	0.00150435863772479\\
581.01	0.00143498594231021\\
582.01	0.00136449786348202\\
583.01	0.00129290160084457\\
584.01	0.00122020695474439\\
585.01	0.00114642644654167\\
586.01	0.00107157539959435\\
587.01	0.000995671963626618\\
588.01	0.000918737060083706\\
589.01	0.000840794219677079\\
590.01	0.000761869275267243\\
591.01	0.000681989863100538\\
592.01	0.000601184672692693\\
593.01	0.000519482369694076\\
594.01	0.000436910096074035\\
595.01	0.000353491426927423\\
596.01	0.000269243631889871\\
597.01	0.000184174050000235\\
598.01	9.85575539648349e-05\\
599.01	3.18230442563731e-05\\
599.02	3.12770957175568e-05\\
599.03	3.07343620567866e-05\\
599.04	3.0194875217571e-05\\
599.05	2.96586674565693e-05\\
599.06	2.91257713466771e-05\\
599.07	2.85962197801391e-05\\
599.08	2.80700459716933e-05\\
599.09	2.75472834617464e-05\\
599.1	2.70279661195791e-05\\
599.11	2.65121281465865e-05\\
599.12	2.59998040795448e-05\\
599.13	2.54910287939124e-05\\
599.14	2.49858375071712e-05\\
599.15	2.44842657821827e-05\\
599.16	2.39863495305973e-05\\
599.17	2.34921250162837e-05\\
599.18	2.30016288588035e-05\\
599.19	2.25148980369013e-05\\
599.2	2.20319698920526e-05\\
599.21	2.15528821320247e-05\\
599.22	2.10776728344908e-05\\
599.23	2.06063804506721e-05\\
599.24	2.01390438090126e-05\\
599.25	1.96757021188858e-05\\
599.26	1.92163949743647e-05\\
599.27	1.87611623579872e-05\\
599.28	1.83100446445959e-05\\
599.29	1.78630826051952e-05\\
599.3	1.74203174108517e-05\\
599.31	1.69817906366353e-05\\
599.32	1.65475442655931e-05\\
599.33	1.61176206927693e-05\\
599.34	1.56920627292639e-05\\
599.35	1.52709136063203e-05\\
599.36	1.48542169794725e-05\\
599.37	1.44420169327208e-05\\
599.38	1.40343579827368e-05\\
599.39	1.36312859119591e-05\\
599.4	1.32328499381877e-05\\
599.41	1.28390997671604e-05\\
599.42	1.24500855973528e-05\\
599.43	1.20658581248111e-05\\
599.44	1.16864685480531e-05\\
599.45	1.13119685730082e-05\\
599.46	1.09424104179929e-05\\
599.47	1.05778468187639e-05\\
599.48	1.02183310335888e-05\\
599.49	9.86391684839293e-06\\
599.5	9.51465858194112e-06\\
599.51	9.17061109107116e-06\\
599.52	8.83182977599525e-06\\
599.53	8.4983705856221e-06\\
599.54	8.1702900229675e-06\\
599.55	7.84764515058753e-06\\
599.56	7.53049359608279e-06\\
599.57	7.21889355765649e-06\\
599.58	6.91290380971064e-06\\
599.59	6.61258370851688e-06\\
599.6	6.31799319793583e-06\\
599.61	6.02919281519031e-06\\
599.62	5.74624369668701e-06\\
599.63	5.46920758391981e-06\\
599.64	5.19814682940593e-06\\
599.65	4.93312440269511e-06\\
599.66	4.67420389643411e-06\\
599.67	4.42144953247646e-06\\
599.68	4.17492616808755e-06\\
599.69	3.9346993021671e-06\\
599.7	3.70083508156871e-06\\
599.71	3.47340030746116e-06\\
599.72	3.25246244175549e-06\\
599.73	3.03808961359987e-06\\
599.74	2.83035062593855e-06\\
599.75	2.62931496212288e-06\\
599.76	2.43505279260738e-06\\
599.77	2.24763498169606e-06\\
599.78	2.06713309435641e-06\\
599.79	1.89361940310974e-06\\
599.8	1.72716689497872e-06\\
599.81	1.56784927850956e-06\\
599.82	1.41574099085315e-06\\
599.83	1.27091720493987e-06\\
599.84	1.13345383668563e-06\\
599.85	1.00342755231589e-06\\
599.86	8.80915775712185e-07\\
599.87	7.65996695880136e-07\\
599.88	6.5874927444344e-07\\
599.89	5.59253253243699e-07\\
599.9	4.67589162013102e-07\\
599.91	3.8383832609741e-07\\
599.92	3.0808287429171e-07\\
599.93	2.4040574671258e-07\\
599.94	1.80890702777825e-07\\
599.95	1.2962232925906e-07\\
599.96	8.66860484019516e-08\\
599.97	5.21681261331924e-08\\
599.98	2.61556803542173e-08\\
599.99	8.73668929909921e-09\\
600	0\\
};
\addplot [color=black!80!mycolor21,solid,forget plot]
  table[row sep=crcr]{%
0.01	0.00502843436055783\\
1.01	0.00502843349929418\\
2.01	0.00502843262123861\\
3.01	0.0050284317260663\\
4.01	0.00502843081344614\\
5.01	0.00502842988304058\\
6.01	0.00502842893450587\\
7.01	0.00502842796749136\\
8.01	0.00502842698163999\\
9.01	0.00502842597658766\\
10.01	0.00502842495196327\\
11.01	0.00502842390738872\\
12.01	0.0050284228424786\\
13.01	0.00502842175684009\\
14.01	0.00502842065007276\\
15.01	0.00502841952176892\\
16.01	0.0050284183715127\\
17.01	0.00502841719888041\\
18.01	0.00502841600344017\\
19.01	0.00502841478475189\\
20.01	0.00502841354236696\\
21.01	0.00502841227582853\\
22.01	0.00502841098467075\\
23.01	0.00502840966841888\\
24.01	0.00502840832658917\\
25.01	0.00502840695868844\\
26.01	0.00502840556421434\\
27.01	0.00502840414265467\\
28.01	0.00502840269348778\\
29.01	0.00502840121618183\\
30.01	0.0050283997101948\\
31.01	0.00502839817497453\\
32.01	0.00502839660995804\\
33.01	0.00502839501457191\\
34.01	0.00502839338823163\\
35.01	0.00502839173034153\\
36.01	0.00502839004029431\\
37.01	0.00502838831747177\\
38.01	0.00502838656124318\\
39.01	0.00502838477096642\\
40.01	0.00502838294598677\\
41.01	0.00502838108563712\\
42.01	0.00502837918923754\\
43.01	0.00502837725609535\\
44.01	0.0050283752855046\\
45.01	0.00502837327674568\\
46.01	0.00502837122908562\\
47.01	0.00502836914177741\\
48.01	0.00502836701405978\\
49.01	0.00502836484515692\\
50.01	0.00502836263427834\\
51.01	0.00502836038061843\\
52.01	0.00502835808335623\\
53.01	0.00502835574165529\\
54.01	0.00502835335466321\\
55.01	0.00502835092151147\\
56.01	0.00502834844131502\\
57.01	0.00502834591317168\\
58.01	0.00502834333616256\\
59.01	0.00502834070935102\\
60.01	0.00502833803178272\\
61.01	0.00502833530248539\\
62.01	0.00502833252046801\\
63.01	0.0050283296847209\\
64.01	0.0050283267942151\\
65.01	0.00502832384790237\\
66.01	0.0050283208447142\\
67.01	0.00502831778356208\\
68.01	0.00502831466333673\\
69.01	0.00502831148290791\\
70.01	0.00502830824112375\\
71.01	0.00502830493681093\\
72.01	0.00502830156877361\\
73.01	0.0050282981357933\\
74.01	0.00502829463662843\\
75.01	0.00502829107001377\\
76.01	0.00502828743466048\\
77.01	0.00502828372925505\\
78.01	0.00502827995245877\\
79.01	0.0050282761029083\\
80.01	0.00502827217921379\\
81.01	0.00502826817995968\\
82.01	0.00502826410370298\\
83.01	0.00502825994897383\\
84.01	0.0050282557142745\\
85.01	0.00502825139807869\\
86.01	0.00502824699883142\\
87.01	0.0050282425149483\\
88.01	0.00502823794481481\\
89.01	0.00502823328678589\\
90.01	0.00502822853918555\\
91.01	0.00502822370030597\\
92.01	0.00502821876840701\\
93.01	0.00502821374171555\\
94.01	0.00502820861842523\\
95.01	0.00502820339669533\\
96.01	0.00502819807465029\\
97.01	0.00502819265037929\\
98.01	0.00502818712193531\\
99.01	0.00502818148733451\\
100.01	0.00502817574455551\\
101.01	0.00502816989153885\\
102.01	0.00502816392618621\\
103.01	0.00502815784635948\\
104.01	0.00502815164988006\\
105.01	0.00502814533452847\\
106.01	0.00502813889804326\\
107.01	0.00502813233812017\\
108.01	0.00502812565241156\\
109.01	0.0050281188385253\\
110.01	0.00502811189402423\\
111.01	0.00502810481642512\\
112.01	0.00502809760319793\\
113.01	0.00502809025176476\\
114.01	0.00502808275949907\\
115.01	0.00502807512372475\\
116.01	0.00502806734171527\\
117.01	0.00502805941069259\\
118.01	0.00502805132782616\\
119.01	0.0050280430902323\\
120.01	0.00502803469497239\\
121.01	0.00502802613905277\\
122.01	0.00502801741942322\\
123.01	0.00502800853297598\\
124.01	0.00502799947654461\\
125.01	0.00502799024690289\\
126.01	0.00502798084076383\\
127.01	0.00502797125477838\\
128.01	0.00502796148553406\\
129.01	0.00502795152955438\\
130.01	0.00502794138329717\\
131.01	0.00502793104315328\\
132.01	0.00502792050544566\\
133.01	0.00502790976642754\\
134.01	0.00502789882228181\\
135.01	0.00502788766911928\\
136.01	0.00502787630297728\\
137.01	0.00502786471981833\\
138.01	0.00502785291552887\\
139.01	0.00502784088591756\\
140.01	0.00502782862671417\\
141.01	0.00502781613356767\\
142.01	0.00502780340204514\\
143.01	0.00502779042762982\\
144.01	0.00502777720571961\\
145.01	0.0050277637316257\\
146.01	0.00502775000057084\\
147.01	0.00502773600768727\\
148.01	0.00502772174801552\\
149.01	0.00502770721650234\\
150.01	0.00502769240799919\\
151.01	0.00502767731726028\\
152.01	0.00502766193894071\\
153.01	0.00502764626759434\\
154.01	0.00502763029767243\\
155.01	0.00502761402352129\\
156.01	0.00502759743938044\\
157.01	0.00502758053938022\\
158.01	0.00502756331754031\\
159.01	0.00502754576776728\\
160.01	0.00502752788385231\\
161.01	0.00502750965946929\\
162.01	0.00502749108817226\\
163.01	0.00502747216339341\\
164.01	0.00502745287844059\\
165.01	0.00502743322649504\\
166.01	0.00502741320060887\\
167.01	0.00502739279370249\\
168.01	0.00502737199856242\\
169.01	0.00502735080783846\\
170.01	0.00502732921404082\\
171.01	0.00502730720953808\\
172.01	0.00502728478655421\\
173.01	0.00502726193716557\\
174.01	0.00502723865329805\\
175.01	0.00502721492672447\\
176.01	0.00502719074906172\\
177.01	0.00502716611176715\\
178.01	0.00502714100613615\\
179.01	0.00502711542329865\\
180.01	0.00502708935421598\\
181.01	0.00502706278967793\\
182.01	0.00502703572029884\\
183.01	0.00502700813651456\\
184.01	0.00502698002857915\\
185.01	0.00502695138656097\\
186.01	0.00502692220033937\\
187.01	0.00502689245960104\\
188.01	0.00502686215383585\\
189.01	0.0050268312723334\\
190.01	0.00502679980417929\\
191.01	0.00502676773825054\\
192.01	0.00502673506321213\\
193.01	0.00502670176751231\\
194.01	0.00502666783937896\\
195.01	0.00502663326681498\\
196.01	0.00502659803759368\\
197.01	0.00502656213925446\\
198.01	0.00502652555909849\\
199.01	0.00502648828418351\\
200.01	0.00502645030131966\\
201.01	0.0050264115970639\\
202.01	0.00502637215771571\\
203.01	0.00502633196931158\\
204.01	0.0050262910176199\\
205.01	0.0050262492881358\\
206.01	0.00502620676607569\\
207.01	0.00502616343637166\\
208.01	0.0050261192836659\\
209.01	0.00502607429230486\\
210.01	0.00502602844633365\\
211.01	0.00502598172948986\\
212.01	0.0050259341251974\\
213.01	0.00502588561656033\\
214.01	0.00502583618635638\\
215.01	0.00502578581703054\\
216.01	0.00502573449068861\\
217.01	0.00502568218908995\\
218.01	0.00502562889364079\\
219.01	0.00502557458538716\\
220.01	0.00502551924500772\\
221.01	0.0050254628528063\\
222.01	0.0050254053887043\\
223.01	0.00502534683223293\\
224.01	0.00502528716252563\\
225.01	0.00502522635830983\\
226.01	0.00502516439789862\\
227.01	0.00502510125918259\\
228.01	0.00502503691962126\\
229.01	0.00502497135623418\\
230.01	0.00502490454559219\\
231.01	0.00502483646380783\\
232.01	0.00502476708652681\\
233.01	0.00502469638891764\\
234.01	0.00502462434566235\\
235.01	0.00502455093094678\\
236.01	0.00502447611844968\\
237.01	0.0050243998813329\\
238.01	0.005024322192231\\
239.01	0.00502424302323956\\
240.01	0.00502416234590481\\
241.01	0.00502408013121223\\
242.01	0.0050239963495747\\
243.01	0.00502391097082081\\
244.01	0.00502382396418274\\
245.01	0.00502373529828444\\
246.01	0.00502364494112798\\
247.01	0.00502355286008181\\
248.01	0.00502345902186676\\
249.01	0.005023363392543\\
250.01	0.00502326593749602\\
251.01	0.00502316662142288\\
252.01	0.00502306540831744\\
253.01	0.00502296226145618\\
254.01	0.00502285714338281\\
255.01	0.00502275001589302\\
256.01	0.00502264084001892\\
257.01	0.00502252957601304\\
258.01	0.00502241618333144\\
259.01	0.00502230062061778\\
260.01	0.00502218284568591\\
261.01	0.00502206281550182\\
262.01	0.00502194048616658\\
263.01	0.00502181581289769\\
264.01	0.00502168875001045\\
265.01	0.00502155925089892\\
266.01	0.005021427268016\\
267.01	0.00502129275285441\\
268.01	0.00502115565592534\\
269.01	0.00502101592673802\\
270.01	0.00502087351377883\\
271.01	0.00502072836448901\\
272.01	0.00502058042524221\\
273.01	0.00502042964132244\\
274.01	0.00502027595690047\\
275.01	0.00502011931500993\\
276.01	0.00501995965752344\\
277.01	0.00501979692512705\\
278.01	0.00501963105729606\\
279.01	0.00501946199226796\\
280.01	0.00501928966701663\\
281.01	0.00501911401722497\\
282.01	0.00501893497725712\\
283.01	0.00501875248013079\\
284.01	0.00501856645748749\\
285.01	0.00501837683956397\\
286.01	0.00501818355516119\\
287.01	0.00501798653161407\\
288.01	0.00501778569475948\\
289.01	0.00501758096890471\\
290.01	0.00501737227679403\\
291.01	0.00501715953957528\\
292.01	0.00501694267676547\\
293.01	0.00501672160621599\\
294.01	0.00501649624407649\\
295.01	0.00501626650475819\\
296.01	0.00501603230089676\\
297.01	0.00501579354331405\\
298.01	0.00501555014097906\\
299.01	0.005015302000968\\
300.01	0.00501504902842402\\
301.01	0.00501479112651473\\
302.01	0.00501452819639081\\
303.01	0.00501426013714194\\
304.01	0.00501398684575311\\
305.01	0.00501370821705909\\
306.01	0.00501342414369801\\
307.01	0.00501313451606501\\
308.01	0.00501283922226349\\
309.01	0.0050125381480563\\
310.01	0.00501223117681559\\
311.01	0.00501191818947187\\
312.01	0.00501159906446138\\
313.01	0.00501127367767312\\
314.01	0.00501094190239442\\
315.01	0.00501060360925566\\
316.01	0.00501025866617318\\
317.01	0.00500990693829178\\
318.01	0.00500954828792586\\
319.01	0.00500918257449919\\
320.01	0.00500880965448356\\
321.01	0.00500842938133637\\
322.01	0.00500804160543641\\
323.01	0.00500764617401919\\
324.01	0.00500724293111037\\
325.01	0.00500683171745786\\
326.01	0.005006412370463\\
327.01	0.00500598472411009\\
328.01	0.00500554860889441\\
329.01	0.00500510385174934\\
330.01	0.005004650275971\\
331.01	0.00500418770114295\\
332.01	0.00500371594305808\\
333.01	0.00500323481363959\\
334.01	0.00500274412086025\\
335.01	0.00500224366866026\\
336.01	0.0050017332568633\\
337.01	0.0050012126810911\\
338.01	0.0050006817326763\\
339.01	0.0050001401985734\\
340.01	0.00499958786126887\\
341.01	0.00499902449868812\\
342.01	0.00499844988410169\\
343.01	0.00499786378602948\\
344.01	0.00499726596814258\\
345.01	0.00499665618916393\\
346.01	0.00499603420276663\\
347.01	0.00499539975746997\\
348.01	0.00499475259653424\\
349.01	0.0049940924578527\\
350.01	0.00499341907384166\\
351.01	0.00499273217132882\\
352.01	0.00499203147143882\\
353.01	0.00499131668947634\\
354.01	0.00499058753480788\\
355.01	0.00498984371074027\\
356.01	0.0049890849143966\\
357.01	0.00498831083659057\\
358.01	0.00498752116169746\\
359.01	0.00498671556752294\\
360.01	0.00498589372516844\\
361.01	0.00498505529889489\\
362.01	0.00498419994598239\\
363.01	0.00498332731658825\\
364.01	0.00498243705360065\\
365.01	0.00498152879249033\\
366.01	0.00498060216115879\\
367.01	0.00497965677978277\\
368.01	0.00497869226065643\\
369.01	0.00497770820802893\\
370.01	0.00497670421793944\\
371.01	0.00497567987804857\\
372.01	0.00497463476746508\\
373.01	0.00497356845657062\\
374.01	0.0049724805068389\\
375.01	0.00497137047065233\\
376.01	0.00497023789111349\\
377.01	0.00496908230185381\\
378.01	0.00496790322683697\\
379.01	0.00496670018015875\\
380.01	0.00496547266584231\\
381.01	0.00496422017762947\\
382.01	0.00496294219876715\\
383.01	0.00496163820178908\\
384.01	0.00496030764829392\\
385.01	0.00495894998871806\\
386.01	0.00495756466210336\\
387.01	0.00495615109586128\\
388.01	0.00495470870553127\\
389.01	0.00495323689453462\\
390.01	0.0049517350539238\\
391.01	0.00495020256212605\\
392.01	0.00494863878468264\\
393.01	0.00494704307398262\\
394.01	0.00494541476899205\\
395.01	0.00494375319497745\\
396.01	0.00494205766322433\\
397.01	0.00494032747074996\\
398.01	0.00493856190001135\\
399.01	0.00493676021860735\\
400.01	0.00493492167897478\\
401.01	0.00493304551807963\\
402.01	0.00493113095710157\\
403.01	0.00492917720111273\\
404.01	0.00492718343875042\\
405.01	0.00492514884188331\\
406.01	0.00492307256527112\\
407.01	0.00492095374621756\\
408.01	0.00491879150421652\\
409.01	0.004916584940591\\
410.01	0.00491433313812519\\
411.01	0.00491203516068845\\
412.01	0.0049096900528525\\
413.01	0.00490729683949968\\
414.01	0.00490485452542453\\
415.01	0.0049023620949264\\
416.01	0.00489981851139374\\
417.01	0.00489722271688028\\
418.01	0.00489457363167176\\
419.01	0.00489187015384463\\
420.01	0.00488911115881449\\
421.01	0.00488629549887635\\
422.01	0.00488342200273419\\
423.01	0.00488048947502156\\
424.01	0.00487749669581191\\
425.01	0.00487444242011829\\
426.01	0.00487132537738289\\
427.01	0.00486814427095616\\
428.01	0.00486489777756404\\
429.01	0.00486158454676542\\
430.01	0.00485820320039651\\
431.01	0.0048547523320048\\
432.01	0.0048512305062708\\
433.01	0.00484763625841695\\
434.01	0.00484396809360503\\
435.01	0.00484022448632022\\
436.01	0.00483640387974204\\
437.01	0.00483250468510353\\
438.01	0.00482852528103479\\
439.01	0.00482446401289458\\
440.01	0.00482031919208718\\
441.01	0.00481608909536478\\
442.01	0.0048117719641162\\
443.01	0.00480736600364003\\
444.01	0.00480286938240266\\
445.01	0.0047982802312817\\
446.01	0.00479359664279346\\
447.01	0.00478881667030427\\
448.01	0.00478393832722535\\
449.01	0.00477895958619237\\
450.01	0.00477387837822654\\
451.01	0.00476869259188027\\
452.01	0.00476340007236439\\
453.01	0.00475799862065733\\
454.01	0.00475248599259701\\
455.01	0.00474685989795386\\
456.01	0.00474111799948513\\
457.01	0.00473525791196947\\
458.01	0.00472927720122272\\
459.01	0.00472317338309335\\
460.01	0.00471694392243755\\
461.01	0.00471058623207308\\
462.01	0.00470409767171277\\
463.01	0.00469747554687555\\
464.01	0.00469071710777627\\
465.01	0.00468381954819199\\
466.01	0.00467678000430614\\
467.01	0.00466959555352891\\
468.01	0.00466226321329406\\
469.01	0.00465477993983148\\
470.01	0.00464714262691423\\
471.01	0.00463934810458196\\
472.01	0.00463139313783721\\
473.01	0.00462327442531657\\
474.01	0.00461498859793493\\
475.01	0.00460653221750294\\
476.01	0.00459790177531674\\
477.01	0.00458909369072005\\
478.01	0.0045801043096378\\
479.01	0.00457092990308097\\
480.01	0.00456156666562179\\
481.01	0.00455201071383896\\
482.01	0.00454225808473339\\
483.01	0.00453230473411217\\
484.01	0.00452214653494158\\
485.01	0.00451177927566853\\
486.01	0.00450119865850974\\
487.01	0.00449040029770726\\
488.01	0.00447937971775215\\
489.01	0.00446813235157375\\
490.01	0.00445665353869464\\
491.01	0.00444493852335173\\
492.01	0.00443298245258193\\
493.01	0.00442078037427268\\
494.01	0.00440832723517685\\
495.01	0.00439561787889161\\
496.01	0.00438264704380088\\
497.01	0.00436940936098138\\
498.01	0.00435589935207233\\
499.01	0.00434211142710735\\
500.01	0.00432803988231031\\
501.01	0.00431367889785326\\
502.01	0.00429902253557849\\
503.01	0.00428406473668267\\
504.01	0.00426879931936507\\
505.01	0.00425321997643889\\
506.01	0.00423732027290725\\
507.01	0.00422109364350371\\
508.01	0.00420453339019825\\
509.01	0.00418763267966946\\
510.01	0.00417038454074419\\
511.01	0.0041527818618063\\
512.01	0.00413481738817535\\
513.01	0.0041164837194568\\
514.01	0.00409777330686782\\
515.01	0.00407867845053805\\
516.01	0.00405919129679085\\
517.01	0.00403930383540636\\
518.01	0.00401900789687119\\
519.01	0.00399829514961824\\
520.01	0.00397715709726165\\
521.01	0.00395558507583142\\
522.01	0.00393357025101552\\
523.01	0.00391110361541448\\
524.01	0.00388817598581671\\
525.01	0.0038647780005037\\
526.01	0.00384090011659382\\
527.01	0.00381653260743666\\
528.01	0.00379166556006887\\
529.01	0.00376628887274622\\
530.01	0.00374039225256658\\
531.01	0.00371396521320085\\
532.01	0.00368699707275148\\
533.01	0.00365947695175801\\
534.01	0.00363139377137604\\
535.01	0.00360273625175423\\
536.01	0.00357349291063878\\
537.01	0.00354365206223945\\
538.01	0.00351320181639204\\
539.01	0.00348213007805842\\
540.01	0.00345042454721088\\
541.01	0.00341807271914801\\
542.01	0.00338506188529999\\
543.01	0.00335137913458498\\
544.01	0.00331701135538371\\
545.01	0.00328194523821044\\
546.01	0.00324616727916235\\
547.01	0.00320966378424351\\
548.01	0.00317242087466379\\
549.01	0.00313442449322877\\
550.01	0.00309566041194589\\
551.01	0.00305611424098788\\
552.01	0.00301577143916516\\
553.01	0.0029746173260766\\
554.01	0.00293263709612681\\
555.01	0.00288981583461184\\
556.01	0.0028461385360993\\
557.01	0.00280159012534766\\
558.01	0.00275615548103306\\
559.01	0.00270981946257547\\
560.01	0.00266256694038211\\
561.01	0.0026143828298511\\
562.01	0.00256525212950647\\
563.01	0.00251515996366286\\
564.01	0.00246409163004402\\
565.01	0.00241203265280651\\
566.01	0.00235896884144259\\
567.01	0.00230488635605555\\
568.01	0.00224977177951561\\
569.01	0.00219361219700808\\
570.01	0.00213639528348043\\
571.01	0.00207810939947119\\
572.01	0.00201874369576008\\
573.01	0.00195828822720497\\
574.01	0.00189673407602165\\
575.01	0.00183407348460219\\
576.01	0.0017702999977462\\
577.01	0.00170540861387421\\
578.01	0.00163939594438615\\
579.01	0.00157226037978655\\
580.01	0.00150400226049134\\
581.01	0.00143462404930831\\
582.01	0.00136413050139218\\
583.01	0.00129252882594572\\
584.01	0.00121982883197683\\
585.01	0.00114604304792577\\
586.01	0.00107118680180153\\
587.01	0.000995278244445114\\
588.01	0.000918338293453008\\
589.01	0.0008403904688796\\
590.01	0.000761460583759315\\
591.01	0.000681576242330336\\
592.01	0.00060076608608719\\
593.01	0.000519058711787984\\
594.01	0.000436481165494431\\
595.01	0.000353056891624152\\
596.01	0.000268802984598084\\
597.01	0.000183726551418536\\
598.01	9.84661228899224e-05\\
599.01	3.18230442563749e-05\\
599.02	3.12770957175551e-05\\
599.03	3.07343620567866e-05\\
599.04	3.0194875217571e-05\\
599.05	2.9658667456571e-05\\
599.06	2.91257713466771e-05\\
599.07	2.85962197801391e-05\\
599.08	2.80700459716916e-05\\
599.09	2.75472834617447e-05\\
599.1	2.70279661195791e-05\\
599.11	2.65121281465865e-05\\
599.12	2.59998040795448e-05\\
599.13	2.54910287939142e-05\\
599.14	2.49858375071712e-05\\
599.15	2.4484265782181e-05\\
599.16	2.39863495305973e-05\\
599.17	2.34921250162855e-05\\
599.18	2.30016288588052e-05\\
599.19	2.25148980369013e-05\\
599.2	2.20319698920508e-05\\
599.21	2.1552882132023e-05\\
599.22	2.10776728344891e-05\\
599.23	2.06063804506721e-05\\
599.24	2.01390438090109e-05\\
599.25	1.96757021188858e-05\\
599.26	1.92163949743647e-05\\
599.27	1.87611623579872e-05\\
599.28	1.83100446445959e-05\\
599.29	1.78630826051952e-05\\
599.3	1.74203174108517e-05\\
599.31	1.69817906366335e-05\\
599.32	1.65475442655914e-05\\
599.33	1.61176206927693e-05\\
599.34	1.56920627292622e-05\\
599.35	1.52709136063186e-05\\
599.36	1.48542169794742e-05\\
599.37	1.4442016932719e-05\\
599.38	1.40343579827385e-05\\
599.39	1.36312859119608e-05\\
599.4	1.3232849938186e-05\\
599.41	1.28390997671621e-05\\
599.42	1.2450085597351e-05\\
599.43	1.20658581248094e-05\\
599.44	1.16864685480531e-05\\
599.45	1.13119685730065e-05\\
599.46	1.09424104179946e-05\\
599.47	1.05778468187639e-05\\
599.48	1.02183310335905e-05\\
599.49	9.86391684839293e-06\\
599.5	9.51465858193938e-06\\
599.51	9.17061109107289e-06\\
599.52	8.83182977599352e-06\\
599.53	8.4983705856221e-06\\
599.54	8.1702900229675e-06\\
599.55	7.84764515058579e-06\\
599.56	7.53049359608279e-06\\
599.57	7.21889355765649e-06\\
599.58	6.91290380971064e-06\\
599.59	6.61258370851688e-06\\
599.6	6.31799319793756e-06\\
599.61	6.02919281519031e-06\\
599.62	5.74624369668528e-06\\
599.63	5.46920758391981e-06\\
599.64	5.19814682940767e-06\\
599.65	4.93312440269685e-06\\
599.66	4.67420389643237e-06\\
599.67	4.42144953247646e-06\\
599.68	4.17492616808582e-06\\
599.69	3.93469930216536e-06\\
599.7	3.70083508156871e-06\\
599.71	3.47340030746116e-06\\
599.72	3.25246244175549e-06\\
599.73	3.03808961360161e-06\\
599.74	2.83035062593855e-06\\
599.75	2.62931496212288e-06\\
599.76	2.43505279260911e-06\\
599.77	2.24763498169606e-06\\
599.78	2.06713309435641e-06\\
599.79	1.89361940310974e-06\\
599.8	1.72716689498045e-06\\
599.81	1.56784927850782e-06\\
599.82	1.41574099085315e-06\\
599.83	1.27091720493813e-06\\
599.84	1.13345383668563e-06\\
599.85	1.00342755231589e-06\\
599.86	8.8091577571392e-07\\
599.87	7.65996695880136e-07\\
599.88	6.58749274441706e-07\\
599.89	5.59253253241965e-07\\
599.9	4.67589162011367e-07\\
599.91	3.83838326099145e-07\\
599.92	3.0808287429171e-07\\
599.93	2.40405746710845e-07\\
599.94	1.8089070277609e-07\\
599.95	1.29622329257326e-07\\
599.96	8.66860484019516e-08\\
599.97	5.21681261349272e-08\\
599.98	2.61556803542173e-08\\
599.99	8.73668930083393e-09\\
600	0\\
};
\addplot [color=black,solid,forget plot]
  table[row sep=crcr]{%
0.01	0.00502703576604918\\
1.01	0.00502703497172564\\
2.01	0.00502703416200814\\
3.01	0.00502703333659998\\
4.01	0.0050270324951992\\
5.01	0.00502703163749766\\
6.01	0.00502703076318172\\
7.01	0.00502702987193134\\
8.01	0.0050270289634203\\
9.01	0.00502702803731684\\
10.01	0.00502702709328209\\
11.01	0.00502702613097098\\
12.01	0.00502702515003172\\
13.01	0.00502702415010615\\
14.01	0.00502702313082892\\
15.01	0.00502702209182774\\
16.01	0.00502702103272314\\
17.01	0.00502701995312872\\
18.01	0.00502701885265022\\
19.01	0.00502701773088608\\
20.01	0.00502701658742719\\
21.01	0.00502701542185629\\
22.01	0.00502701423374847\\
23.01	0.00502701302267043\\
24.01	0.00502701178818072\\
25.01	0.00502701052982919\\
26.01	0.00502700924715755\\
27.01	0.00502700793969848\\
28.01	0.00502700660697566\\
29.01	0.00502700524850406\\
30.01	0.00502700386378847\\
31.01	0.00502700245232527\\
32.01	0.00502700101360067\\
33.01	0.00502699954709104\\
34.01	0.0050269980522628\\
35.01	0.00502699652857234\\
36.01	0.00502699497546558\\
37.01	0.00502699339237751\\
38.01	0.0050269917787328\\
39.01	0.00502699013394496\\
40.01	0.00502698845741587\\
41.01	0.00502698674853659\\
42.01	0.00502698500668635\\
43.01	0.00502698323123265\\
44.01	0.0050269814215304\\
45.01	0.0050269795769228\\
46.01	0.00502697769673993\\
47.01	0.00502697578029949\\
48.01	0.00502697382690603\\
49.01	0.00502697183585049\\
50.01	0.00502696980641097\\
51.01	0.00502696773785102\\
52.01	0.00502696562942034\\
53.01	0.00502696348035482\\
54.01	0.00502696128987477\\
55.01	0.00502695905718655\\
56.01	0.00502695678148061\\
57.01	0.00502695446193257\\
58.01	0.0050269520977019\\
59.01	0.00502694968793225\\
60.01	0.00502694723175042\\
61.01	0.00502694472826718\\
62.01	0.00502694217657556\\
63.01	0.0050269395757521\\
64.01	0.00502693692485474\\
65.01	0.00502693422292425\\
66.01	0.00502693146898289\\
67.01	0.00502692866203387\\
68.01	0.00502692580106153\\
69.01	0.0050269228850307\\
70.01	0.00502691991288679\\
71.01	0.00502691688355467\\
72.01	0.00502691379593868\\
73.01	0.00502691064892259\\
74.01	0.00502690744136828\\
75.01	0.00502690417211634\\
76.01	0.00502690083998477\\
77.01	0.00502689744376923\\
78.01	0.00502689398224234\\
79.01	0.00502689045415315\\
80.01	0.00502688685822671\\
81.01	0.00502688319316384\\
82.01	0.00502687945764056\\
83.01	0.00502687565030747\\
84.01	0.00502687176978926\\
85.01	0.00502686781468428\\
86.01	0.00502686378356412\\
87.01	0.00502685967497303\\
88.01	0.00502685548742783\\
89.01	0.00502685121941631\\
90.01	0.00502684686939733\\
91.01	0.00502684243580049\\
92.01	0.00502683791702561\\
93.01	0.00502683331144142\\
94.01	0.00502682861738518\\
95.01	0.00502682383316288\\
96.01	0.00502681895704779\\
97.01	0.00502681398728019\\
98.01	0.00502680892206636\\
99.01	0.00502680375957853\\
100.01	0.00502679849795363\\
101.01	0.00502679313529323\\
102.01	0.0050267876696622\\
103.01	0.00502678209908856\\
104.01	0.00502677642156221\\
105.01	0.00502677063503451\\
106.01	0.00502676473741767\\
107.01	0.00502675872658393\\
108.01	0.00502675260036436\\
109.01	0.00502674635654883\\
110.01	0.00502673999288452\\
111.01	0.0050267335070753\\
112.01	0.0050267268967811\\
113.01	0.00502672015961691\\
114.01	0.00502671329315179\\
115.01	0.00502670629490811\\
116.01	0.00502669916236056\\
117.01	0.00502669189293583\\
118.01	0.00502668448401019\\
119.01	0.00502667693291004\\
120.01	0.00502666923691057\\
121.01	0.00502666139323414\\
122.01	0.00502665339904985\\
123.01	0.00502664525147199\\
124.01	0.00502663694755967\\
125.01	0.00502662848431528\\
126.01	0.00502661985868368\\
127.01	0.00502661106755034\\
128.01	0.00502660210774116\\
129.01	0.00502659297602076\\
130.01	0.00502658366909147\\
131.01	0.00502657418359178\\
132.01	0.00502656451609557\\
133.01	0.00502655466311061\\
134.01	0.00502654462107741\\
135.01	0.00502653438636722\\
136.01	0.00502652395528167\\
137.01	0.00502651332405101\\
138.01	0.00502650248883249\\
139.01	0.00502649144570913\\
140.01	0.00502648019068817\\
141.01	0.00502646871969995\\
142.01	0.00502645702859572\\
143.01	0.00502644511314656\\
144.01	0.00502643296904195\\
145.01	0.00502642059188779\\
146.01	0.00502640797720459\\
147.01	0.00502639512042618\\
148.01	0.00502638201689851\\
149.01	0.00502636866187672\\
150.01	0.00502635505052398\\
151.01	0.00502634117790974\\
152.01	0.00502632703900747\\
153.01	0.00502631262869365\\
154.01	0.00502629794174505\\
155.01	0.00502628297283685\\
156.01	0.00502626771654075\\
157.01	0.00502625216732305\\
158.01	0.00502623631954242\\
159.01	0.00502622016744737\\
160.01	0.00502620370517524\\
161.01	0.00502618692674866\\
162.01	0.00502616982607423\\
163.01	0.00502615239693952\\
164.01	0.005026134633011\\
165.01	0.00502611652783171\\
166.01	0.00502609807481878\\
167.01	0.00502607926726106\\
168.01	0.00502606009831574\\
169.01	0.00502604056100712\\
170.01	0.00502602064822275\\
171.01	0.00502600035271122\\
172.01	0.00502597966707928\\
173.01	0.00502595858378888\\
174.01	0.00502593709515498\\
175.01	0.0050259151933413\\
176.01	0.00502589287035859\\
177.01	0.00502587011806067\\
178.01	0.00502584692814166\\
179.01	0.00502582329213277\\
180.01	0.00502579920139933\\
181.01	0.00502577464713643\\
182.01	0.0050257496203664\\
183.01	0.0050257241119352\\
184.01	0.00502569811250916\\
185.01	0.00502567161257017\\
186.01	0.00502564460241322\\
187.01	0.00502561707214224\\
188.01	0.00502558901166555\\
189.01	0.00502556041069319\\
190.01	0.00502553125873202\\
191.01	0.00502550154508167\\
192.01	0.00502547125883069\\
193.01	0.00502544038885155\\
194.01	0.00502540892379753\\
195.01	0.00502537685209701\\
196.01	0.00502534416194964\\
197.01	0.00502531084132103\\
198.01	0.005025276877939\\
199.01	0.00502524225928753\\
200.01	0.00502520697260286\\
201.01	0.00502517100486792\\
202.01	0.00502513434280697\\
203.01	0.00502509697288092\\
204.01	0.00502505888128121\\
205.01	0.00502502005392489\\
206.01	0.00502498047644921\\
207.01	0.00502494013420463\\
208.01	0.00502489901225038\\
209.01	0.00502485709534807\\
210.01	0.00502481436795496\\
211.01	0.00502477081421839\\
212.01	0.00502472641796924\\
213.01	0.00502468116271534\\
214.01	0.00502463503163499\\
215.01	0.00502458800756995\\
216.01	0.00502454007301841\\
217.01	0.00502449121012867\\
218.01	0.00502444140069091\\
219.01	0.00502439062613034\\
220.01	0.0050243388674997\\
221.01	0.00502428610547143\\
222.01	0.00502423232032982\\
223.01	0.00502417749196284\\
224.01	0.00502412159985483\\
225.01	0.005024064623076\\
226.01	0.00502400654027658\\
227.01	0.00502394732967589\\
228.01	0.00502388696905482\\
229.01	0.0050238254357458\\
230.01	0.00502376270662396\\
231.01	0.00502369875809768\\
232.01	0.00502363356609869\\
233.01	0.00502356710607221\\
234.01	0.00502349935296707\\
235.01	0.00502343028122427\\
236.01	0.0050233598647678\\
237.01	0.00502328807699373\\
238.01	0.00502321489075774\\
239.01	0.00502314027836531\\
240.01	0.00502306421156023\\
241.01	0.00502298666151143\\
242.01	0.00502290759880255\\
243.01	0.00502282699341911\\
244.01	0.00502274481473582\\
245.01	0.00502266103150399\\
246.01	0.00502257561183809\\
247.01	0.005022488523203\\
248.01	0.00502239973240027\\
249.01	0.00502230920555365\\
250.01	0.00502221690809563\\
251.01	0.00502212280475224\\
252.01	0.00502202685952898\\
253.01	0.00502192903569476\\
254.01	0.0050218292957669\\
255.01	0.00502172760149603\\
256.01	0.00502162391384869\\
257.01	0.00502151819299161\\
258.01	0.0050214103982748\\
259.01	0.00502130048821416\\
260.01	0.00502118842047428\\
261.01	0.00502107415184989\\
262.01	0.00502095763824812\\
263.01	0.00502083883466974\\
264.01	0.00502071769518958\\
265.01	0.00502059417293812\\
266.01	0.00502046822008034\\
267.01	0.00502033978779554\\
268.01	0.00502020882625775\\
269.01	0.00502007528461342\\
270.01	0.00501993911095965\\
271.01	0.0050198002523227\\
272.01	0.00501965865463546\\
273.01	0.0050195142627137\\
274.01	0.00501936702023292\\
275.01	0.00501921686970466\\
276.01	0.00501906375245166\\
277.01	0.00501890760858225\\
278.01	0.00501874837696522\\
279.01	0.00501858599520404\\
280.01	0.00501842039960989\\
281.01	0.00501825152517415\\
282.01	0.00501807930554093\\
283.01	0.00501790367297823\\
284.01	0.00501772455834978\\
285.01	0.0050175418910846\\
286.01	0.00501735559914695\\
287.01	0.00501716560900614\\
288.01	0.00501697184560448\\
289.01	0.00501677423232486\\
290.01	0.0050165726909586\\
291.01	0.00501636714167152\\
292.01	0.00501615750297024\\
293.01	0.00501594369166666\\
294.01	0.00501572562284251\\
295.01	0.00501550320981382\\
296.01	0.00501527636409247\\
297.01	0.00501504499534952\\
298.01	0.00501480901137572\\
299.01	0.0050145683180425\\
300.01	0.00501432281926165\\
301.01	0.00501407241694434\\
302.01	0.00501381701095937\\
303.01	0.00501355649908949\\
304.01	0.00501329077698911\\
305.01	0.00501301973813854\\
306.01	0.00501274327379982\\
307.01	0.00501246127296886\\
308.01	0.00501217362232905\\
309.01	0.00501188020620328\\
310.01	0.00501158090650419\\
311.01	0.00501127560268378\\
312.01	0.00501096417168221\\
313.01	0.00501064648787578\\
314.01	0.00501032242302336\\
315.01	0.00500999184621224\\
316.01	0.00500965462380213\\
317.01	0.00500931061936889\\
318.01	0.00500895969364617\\
319.01	0.00500860170446727\\
320.01	0.00500823650670423\\
321.01	0.00500786395220709\\
322.01	0.00500748388974067\\
323.01	0.0050070961649207\\
324.01	0.00500670062014944\\
325.01	0.00500629709454859\\
326.01	0.00500588542389132\\
327.01	0.00500546544053379\\
328.01	0.00500503697334403\\
329.01	0.00500459984762995\\
330.01	0.0050041538850663\\
331.01	0.00500369890361948\\
332.01	0.00500323471747099\\
333.01	0.00500276113694053\\
334.01	0.0050022779684046\\
335.01	0.00500178501421676\\
336.01	0.00500128207262463\\
337.01	0.00500076893768534\\
338.01	0.00500024539917948\\
339.01	0.00499971124252294\\
340.01	0.00499916624867749\\
341.01	0.004998610194059\\
342.01	0.00499804285044488\\
343.01	0.00499746398487761\\
344.01	0.00499687335956869\\
345.01	0.00499627073179867\\
346.01	0.00499565585381676\\
347.01	0.00499502847273668\\
348.01	0.0049943883304321\\
349.01	0.00499373516342843\\
350.01	0.00499306870279381\\
351.01	0.00499238867402622\\
352.01	0.00499169479694037\\
353.01	0.00499098678554984\\
354.01	0.00499026434794896\\
355.01	0.00498952718619072\\
356.01	0.00498877499616313\\
357.01	0.00498800746746252\\
358.01	0.004987224283264\\
359.01	0.00498642512019002\\
360.01	0.00498560964817559\\
361.01	0.00498477753033095\\
362.01	0.004983928422801\\
363.01	0.00498306197462264\\
364.01	0.00498217782757774\\
365.01	0.00498127561604488\\
366.01	0.00498035496684642\\
367.01	0.00497941549909322\\
368.01	0.00497845682402568\\
369.01	0.00497747854485262\\
370.01	0.0049764802565847\\
371.01	0.00497546154586643\\
372.01	0.00497442199080368\\
373.01	0.00497336116078744\\
374.01	0.00497227861631527\\
375.01	0.00497117390880699\\
376.01	0.00497004658041888\\
377.01	0.00496889616385164\\
378.01	0.00496772218215657\\
379.01	0.00496652414853671\\
380.01	0.0049653015661442\\
381.01	0.00496405392787294\\
382.01	0.00496278071614793\\
383.01	0.00496148140270978\\
384.01	0.00496015544839467\\
385.01	0.0049588023029107\\
386.01	0.00495742140460907\\
387.01	0.00495601218025018\\
388.01	0.00495457404476648\\
389.01	0.00495310640101909\\
390.01	0.0049516086395499\\
391.01	0.00495008013832966\\
392.01	0.00494852026249938\\
393.01	0.00494692836410812\\
394.01	0.00494530378184428\\
395.01	0.00494364584076202\\
396.01	0.00494195385200213\\
397.01	0.00494022711250749\\
398.01	0.00493846490473222\\
399.01	0.00493666649634533\\
400.01	0.00493483113992809\\
401.01	0.00493295807266601\\
402.01	0.00493104651603314\\
403.01	0.00492909567547131\\
404.01	0.00492710474006213\\
405.01	0.00492507288219215\\
406.01	0.0049229992572118\\
407.01	0.00492088300308632\\
408.01	0.0049187232400408\\
409.01	0.00491651907019639\\
410.01	0.0049142695772005\\
411.01	0.00491197382584832\\
412.01	0.00490963086169665\\
413.01	0.00490723971067068\\
414.01	0.00490479937866128\\
415.01	0.00490230885111476\\
416.01	0.0048997670926143\\
417.01	0.00489717304645162\\
418.01	0.00489452563419164\\
419.01	0.00489182375522598\\
420.01	0.00488906628631888\\
421.01	0.00488625208114297\\
422.01	0.00488337996980528\\
423.01	0.00488044875836417\\
424.01	0.00487745722833499\\
425.01	0.00487440413618674\\
426.01	0.00487128821282793\\
427.01	0.00486810816308112\\
428.01	0.00486486266514787\\
429.01	0.00486155037006109\\
430.01	0.00485816990112722\\
431.01	0.00485471985335676\\
432.01	0.00485119879288194\\
433.01	0.00484760525636404\\
434.01	0.00484393775038724\\
435.01	0.00484019475084038\\
436.01	0.00483637470228587\\
437.01	0.00483247601731603\\
438.01	0.00482849707589545\\
439.01	0.00482443622469048\\
440.01	0.00482029177638453\\
441.01	0.00481606200897971\\
442.01	0.00481174516508364\\
443.01	0.00480733945118229\\
444.01	0.00480284303689729\\
445.01	0.00479825405422835\\
446.01	0.00479357059678031\\
447.01	0.00478879071897395\\
448.01	0.00478391243524159\\
449.01	0.00477893371920487\\
450.01	0.0047738525028374\\
451.01	0.00476866667560869\\
452.01	0.0047633740836119\\
453.01	0.00475797252867356\\
454.01	0.00475245976744409\\
455.01	0.00474683351047193\\
456.01	0.00474109142125671\\
457.01	0.00473523111528438\\
458.01	0.00472925015904235\\
459.01	0.00472314606901494\\
460.01	0.00471691631065769\\
461.01	0.00471055829735238\\
462.01	0.00470406938933889\\
463.01	0.00469744689262706\\
464.01	0.00469068805788581\\
465.01	0.00468379007931023\\
466.01	0.00467675009346508\\
467.01	0.00466956517810593\\
468.01	0.00466223235097598\\
469.01	0.00465474856857865\\
470.01	0.00464711072492618\\
471.01	0.00463931565026219\\
472.01	0.0046313601097596\\
473.01	0.00462324080219216\\
474.01	0.00461495435857925\\
475.01	0.00460649734080428\\
476.01	0.00459786624020519\\
477.01	0.00458905747613741\\
478.01	0.00458006739450801\\
479.01	0.0045708922662815\\
480.01	0.00456152828595533\\
481.01	0.00455197157000686\\
482.01	0.00454221815530865\\
483.01	0.00453226399751377\\
484.01	0.00452210496940937\\
485.01	0.00451173685923857\\
486.01	0.00450115536898976\\
487.01	0.00449035611265384\\
488.01	0.00447933461444723\\
489.01	0.00446808630700198\\
490.01	0.00445660652952146\\
491.01	0.00444489052590145\\
492.01	0.00443293344281625\\
493.01	0.00442073032776978\\
494.01	0.0044082761271104\\
495.01	0.00439556568401039\\
496.01	0.00438259373640875\\
497.01	0.00436935491491723\\
498.01	0.00435584374068995\\
499.01	0.00434205462325613\\
500.01	0.00432798185831573\\
501.01	0.00431361962549771\\
502.01	0.00429896198608176\\
503.01	0.0042840028806827\\
504.01	0.00426873612689873\\
505.01	0.00425315541692276\\
506.01	0.00423725431511873\\
507.01	0.0042210262555614\\
508.01	0.00420446453954272\\
509.01	0.00418756233304362\\
510.01	0.00417031266417407\\
511.01	0.0041527084205811\\
512.01	0.00413474234682745\\
513.01	0.00411640704174263\\
514.01	0.00409769495574692\\
515.01	0.00407859838815333\\
516.01	0.00405910948444813\\
517.01	0.00403922023355392\\
518.01	0.00401892246507926\\
519.01	0.00399820784655814\\
520.01	0.00397706788068469\\
521.01	0.00395549390254862\\
522.01	0.00393347707687628\\
523.01	0.00391100839528585\\
524.01	0.00388807867356233\\
525.01	0.00386467854896303\\
526.01	0.00384079847756167\\
527.01	0.00381642873164236\\
528.01	0.00379155939715641\\
529.01	0.00376618037125456\\
530.01	0.00374028135991029\\
531.01	0.00371385187565177\\
532.01	0.00368688123541993\\
533.01	0.00365935855857643\\
534.01	0.00363127276508319\\
535.01	0.00360261257388014\\
536.01	0.00357336650149188\\
537.01	0.00354352286089486\\
538.01	0.00351306976068174\\
539.01	0.00348199510456496\\
540.01	0.00345028659126204\\
541.01	0.00341793171481548\\
542.01	0.00338491776540139\\
543.01	0.0033512318306892\\
544.01	0.00331686079782177\\
545.01	0.00328179135609021\\
546.01	0.00324601000039032\\
547.01	0.00320950303555117\\
548.01	0.00317225658164214\\
549.01	0.00313425658037082\\
550.01	0.00309548880269848\\
551.01	0.0030559388578128\\
552.01	0.0030155922036114\\
553.01	0.00297443415886602\\
554.01	0.0029324499172527\\
555.01	0.00288962456345301\\
556.01	0.00284594309155133\\
557.01	0.00280139042597268\\
558.01	0.00275595144523016\\
559.01	0.00270961100877377\\
560.01	0.0026623539872585\\
561.01	0.00261416529657479\\
562.01	0.0025650299360136\\
563.01	0.00251493303096204\\
564.01	0.00246385988055685\\
565.01	0.00241179601074453\\
566.01	0.00235872723322304\\
567.01	0.0023046397107587\\
568.01	0.00224952002938472\\
569.01	0.0021933552779938\\
570.01	0.00213613313583009\\
571.01	0.00207784196836236\\
572.01	0.00201847093197674\\
573.01	0.00195801008785242\\
574.01	0.00189645052527396\\
575.01	0.00183378449447307\\
576.01	0.00177000554886907\\
577.01	0.00170510869627301\\
578.01	0.00163909055821012\\
579.01	0.00157194953597514\\
580.01	0.00150368598132186\\
581.01	0.00143430236876668\\
582.01	0.00136380346528834\\
583.01	0.00129219649167196\\
584.01	0.00121949126778043\\
585.01	0.00114570033153021\\
586.01	0.00107083901816671\\
587.01	0.000994925482401572\\
588.01	0.000917980640875388\\
589.01	0.000840028005979048\\
590.01	0.000761093373966642\\
591.01	0.000681204320110056\\
592.01	0.000600389440856771\\
593.01	0.000518677266913736\\
594.01	0.000436094751083286\\
595.01	0.000352665209519579\\
596.01	0.000268405563604855\\
597.01	0.000183322690309106\\
598.01	9.83859705868326e-05\\
599.01	3.18230442563731e-05\\
599.02	3.12770957175551e-05\\
599.03	3.07343620567849e-05\\
599.04	3.0194875217571e-05\\
599.05	2.96586674565693e-05\\
599.06	2.91257713466771e-05\\
599.07	2.85962197801391e-05\\
599.08	2.80700459716933e-05\\
599.09	2.75472834617447e-05\\
599.1	2.70279661195773e-05\\
599.11	2.65121281465865e-05\\
599.12	2.59998040795448e-05\\
599.13	2.54910287939124e-05\\
599.14	2.49858375071712e-05\\
599.15	2.44842657821827e-05\\
599.16	2.39863495305956e-05\\
599.17	2.34921250162837e-05\\
599.18	2.30016288588035e-05\\
599.19	2.2514898036903e-05\\
599.2	2.20319698920508e-05\\
599.21	2.1552882132023e-05\\
599.22	2.10776728344908e-05\\
599.23	2.06063804506721e-05\\
599.24	2.01390438090109e-05\\
599.25	1.96757021188876e-05\\
599.26	1.9216394974363e-05\\
599.27	1.87611623579855e-05\\
599.28	1.83100446445959e-05\\
599.29	1.78630826051952e-05\\
599.3	1.74203174108517e-05\\
599.31	1.69817906366335e-05\\
599.32	1.65475442655914e-05\\
599.33	1.61176206927693e-05\\
599.34	1.56920627292622e-05\\
599.35	1.52709136063186e-05\\
599.36	1.48542169794725e-05\\
599.37	1.4442016932719e-05\\
599.38	1.40343579827368e-05\\
599.39	1.36312859119591e-05\\
599.4	1.3232849938186e-05\\
599.41	1.28390997671604e-05\\
599.42	1.24500855973528e-05\\
599.43	1.20658581248094e-05\\
599.44	1.16864685480531e-05\\
599.45	1.13119685730082e-05\\
599.46	1.09424104179929e-05\\
599.47	1.05778468187639e-05\\
599.48	1.02183310335888e-05\\
599.49	9.86391684839293e-06\\
599.5	9.51465858194112e-06\\
599.51	9.17061109107116e-06\\
599.52	8.83182977599525e-06\\
599.53	8.4983705856221e-06\\
599.54	8.1702900229675e-06\\
599.55	7.84764515058753e-06\\
599.56	7.53049359608453e-06\\
599.57	7.21889355765649e-06\\
599.58	6.91290380970891e-06\\
599.59	6.61258370851688e-06\\
599.6	6.31799319793756e-06\\
599.61	6.02919281519031e-06\\
599.62	5.74624369668701e-06\\
599.63	5.46920758391981e-06\\
599.64	5.19814682940593e-06\\
599.65	4.93312440269685e-06\\
599.66	4.67420389643411e-06\\
599.67	4.42144953247646e-06\\
599.68	4.17492616808582e-06\\
599.69	3.9346993021671e-06\\
599.7	3.70083508157044e-06\\
599.71	3.47340030746116e-06\\
599.72	3.25246244175723e-06\\
599.73	3.03808961360161e-06\\
599.74	2.83035062593855e-06\\
599.75	2.62931496212288e-06\\
599.76	2.43505279260738e-06\\
599.77	2.24763498169432e-06\\
599.78	2.06713309435641e-06\\
599.79	1.89361940311147e-06\\
599.8	1.72716689497872e-06\\
599.81	1.56784927850956e-06\\
599.82	1.41574099085488e-06\\
599.83	1.27091720493813e-06\\
599.84	1.13345383668736e-06\\
599.85	1.00342755231589e-06\\
599.86	8.8091577571392e-07\\
599.87	7.65996695880136e-07\\
599.88	6.58749274441706e-07\\
599.89	5.59253253243699e-07\\
599.9	4.67589162013102e-07\\
599.91	3.83838326099145e-07\\
599.92	3.08082874293444e-07\\
599.93	2.40405746710845e-07\\
599.94	1.80890702777825e-07\\
599.95	1.2962232925906e-07\\
599.96	8.66860484019516e-08\\
599.97	5.21681261331924e-08\\
599.98	2.61556803542173e-08\\
599.99	8.73668930083393e-09\\
600	0\\
};
\end{axis}
\end{tikzpicture}% 
%  \caption{Continuous Time w/ nFPC}
%\end{subfigure}%
%\hfill%
%\begin{subfigure}{.45\linewidth}
%  \centering
%  \setlength\figureheight{\linewidth} 
%  \setlength\figurewidth{\linewidth}
%  \tikzsetnextfilename{dm_dscr_nFPC_z8}
%  % This file was created by matlab2tikz.
%
%The latest updates can be retrieved from
%  http://www.mathworks.com/matlabcentral/fileexchange/22022-matlab2tikz-matlab2tikz
%where you can also make suggestions and rate matlab2tikz.
%
\definecolor{mycolor1}{rgb}{0.00000,1.00000,0.14286}%
\definecolor{mycolor2}{rgb}{0.00000,1.00000,0.28571}%
\definecolor{mycolor3}{rgb}{0.00000,1.00000,0.42857}%
\definecolor{mycolor4}{rgb}{0.00000,1.00000,0.57143}%
\definecolor{mycolor5}{rgb}{0.00000,1.00000,0.71429}%
\definecolor{mycolor6}{rgb}{0.00000,1.00000,0.85714}%
\definecolor{mycolor7}{rgb}{0.00000,1.00000,1.00000}%
\definecolor{mycolor8}{rgb}{0.00000,0.87500,1.00000}%
\definecolor{mycolor9}{rgb}{0.00000,0.62500,1.00000}%
\definecolor{mycolor10}{rgb}{0.12500,0.00000,1.00000}%
\definecolor{mycolor11}{rgb}{0.25000,0.00000,1.00000}%
\definecolor{mycolor12}{rgb}{0.37500,0.00000,1.00000}%
\definecolor{mycolor13}{rgb}{0.50000,0.00000,1.00000}%
\definecolor{mycolor14}{rgb}{0.62500,0.00000,1.00000}%
\definecolor{mycolor15}{rgb}{0.75000,0.00000,1.00000}%
\definecolor{mycolor16}{rgb}{0.87500,0.00000,1.00000}%
\definecolor{mycolor17}{rgb}{1.00000,0.00000,1.00000}%
\definecolor{mycolor18}{rgb}{1.00000,0.00000,0.87500}%
\definecolor{mycolor19}{rgb}{1.00000,0.00000,0.62500}%
\definecolor{mycolor20}{rgb}{0.85714,0.00000,0.00000}%
\definecolor{mycolor21}{rgb}{0.71429,0.00000,0.00000}%
%
\begin{tikzpicture}

\begin{axis}[%
width=4.1in,
height=3.803in,
at={(0.809in,0.513in)},
scale only axis,
point meta min=0,
point meta max=1,
every outer x axis line/.append style={black},
every x tick label/.append style={font=\color{black}},
xmin=0,
xmax=600,
every outer y axis line/.append style={black},
every y tick label/.append style={font=\color{black}},
ymin=0,
ymax=0.012,
axis background/.style={fill=white},
axis x line*=bottom,
axis y line*=left,
colormap={mymap}{[1pt] rgb(0pt)=(0,1,0); rgb(7pt)=(0,1,1); rgb(15pt)=(0,0,1); rgb(23pt)=(1,0,1); rgb(31pt)=(1,0,0); rgb(38pt)=(0,0,0)},
colorbar,
colorbar style={separate axis lines,every outer x axis line/.append style={black},every x tick label/.append style={font=\color{black}},every outer y axis line/.append style={black},every y tick label/.append style={font=\color{black}},yticklabels={{-19},{-17},{-15},{-13},{-11},{-9},{-7},{-5},{-3},{-1},{1},{3},{5},{7},{9},{11},{13},{15},{17},{19}}}
]
\addplot [color=green,solid,forget plot]
  table[row sep=crcr]{%
1	0.00409784475759984\\
2	0.00409792030352774\\
3	0.00409799759515541\\
4	0.00409807667318205\\
5	0.00409815757926939\\
6	0.00409824035606549\\
7	0.00409832504722819\\
8	0.00409841169745005\\
9	0.00409850035248332\\
10	0.00409859105916587\\
11	0.00409868386544764\\
12	0.00409877882041777\\
13	0.00409887597433279\\
14	0.0040989753786449\\
15	0.0040990770860314\\
16	0.00409918115042499\\
17	0.00409928762704437\\
18	0.00409939657242646\\
19	0.00409950804445848\\
20	0.00409962210241159\\
21	0.00409973880697533\\
22	0.00409985822029268\\
23	0.00409998040599627\\
24	0.00410010542924576\\
25	0.00410023335676568\\
26	0.00410036425688493\\
27	0.00410049819957701\\
28	0.00410063525650136\\
29	0.00410077550104588\\
30	0.00410091900837079\\
31	0.00410106585545345\\
32	0.00410121612113476\\
33	0.00410136988616654\\
34	0.0041015272332605\\
35	0.00410168824713853\\
36	0.00410185301458445\\
37	0.00410202162449739\\
38	0.00410219416794652\\
39	0.00410237073822733\\
40	0.00410255143092025\\
41	0.00410273634395007\\
42	0.00410292557764804\\
43	0.00410311923481519\\
44	0.00410331742078817\\
45	0.00410352024350664\\
46	0.00410372781358327\\
47	0.00410394024437562\\
48	0.0041041576520604\\
49	0.00410438015571046\\
50	0.00410460787737368\\
51	0.00410484094215535\\
52	0.00410507947830251\\
53	0.00410532361729158\\
54	0.00410557349391918\\
55	0.00410582924639567\\
56	0.00410609101644245\\
57	0.00410635894939252\\
58	0.00410663319429459\\
59	0.00410691390402163\\
60	0.00410720123538273\\
61	0.00410749534923982\\
62	0.0041077964106286\\
63	0.00410810458888434\\
64	0.0041084200577726\\
65	0.00410874299562529\\
66	0.00410907358548255\\
67	0.00410941201524031\\
68	0.00410975847780406\\
69	0.0041101131712496\\
70	0.00411047629899072\\
71	0.00411084806995413\\
72	0.00411122869876268\\
73	0.00411161840592728\\
74	0.00411201741804687\\
75	0.00411242596801909\\
76	0.00411284429526073\\
77	0.00411327264593887\\
78	0.00411371127321411\\
79	0.00411416043749602\\
80	0.00411462040671191\\
81	0.00411509145658996\\
82	0.00411557387095742\\
83	0.00411606794205526\\
84	0.00411657397087042\\
85	0.00411709226748659\\
86	0.00411762315145538\\
87	0.00411816695218918\\
88	0.00411872400937669\\
89	0.00411929467342452\\
90	0.00411987930592472\\
91	0.00412047828015157\\
92	0.00412109198158957\\
93	0.00412172080849527\\
94	0.00412236517249433\\
95	0.0041230254992186\\
96	0.00412370222898441\\
97	0.00412439581751649\\
98	0.0041251067367204\\
99	0.00412583547550783\\
100	0.00412658254067848\\
101	0.00412734845786299\\
102	0.00412813377253208\\
103	0.00412893905107647\\
104	0.00412976488196353\\
105	0.00413061187697616\\
106	0.00413148067254064\\
107	0.00413237193114969\\
108	0.00413328634288785\\
109	0.00413422462706693\\
110	0.00413518753397913\\
111	0.00413617584677592\\
112	0.00413719038348127\\
113	0.00413823199914742\\
114	0.00413930158816231\\
115	0.00414040008671661\\
116	0.0041415284754392\\
117	0.00414268778220749\\
118	0.00414387908514009\\
119	0.00414510351577602\\
120	0.004146362262443\\
121	0.00414765657381456\\
122	0.00414898776265122\\
123	0.00415035720971535\\
124	0.00415176636784119\\
125	0.00415321676613401\\
126	0.00415471001425678\\
127	0.00415624780674915\\
128	0.00415783192730104\\
129	0.00415946425287718\\
130	0.00416114675755497\\
131	0.00416288151589485\\
132	0.00416467070560643\\
133	0.00416651660920656\\
134	0.0041684216142744\\
135	0.00417038821180032\\
136	0.00417241899198365\\
137	0.00417451663665705\\
138	0.00417668390729333\\
139	0.00417892362726813\\
140	0.00418123865669847\\
141	0.00418363185772974\\
142	0.00418610604758351\\
143	0.00418866393596887\\
144	0.00419130804257029\\
145	0.00419404058920214\\
146	0.00419686335980469\\
147	0.00419977751967251\\
148	0.00420278338304558\\
149	0.00420588011530566\\
150	0.0042090653522319\\
151	0.00421233471331403\\
152	0.00421568117537134\\
153	0.00421906581522593\\
154	0.00422248597450611\\
155	0.00422594201091482\\
156	0.00422943428503874\\
157	0.00423296316033096\\
158	0.00423652900308989\\
159	0.00424013218243538\\
160	0.00424377307028067\\
161	0.00424745204130091\\
162	0.00425116947289712\\
163	0.00425492574515596\\
164	0.00425872124080423\\
165	0.00426255634515837\\
166	0.0042664314460684\\
167	0.00427034693385525\\
168	0.00427430320124195\\
169	0.00427830064327726\\
170	0.00428233965725188\\
171	0.00428642064260612\\
172	0.00429054400082835\\
173	0.00429471013534397\\
174	0.00429891945139335\\
175	0.00430317235589869\\
176	0.00430746925731783\\
177	0.00431181056548482\\
178	0.00431619669143569\\
179	0.00432062804721811\\
180	0.00432510504568366\\
181	0.00432962810026109\\
182	0.00433419762470916\\
183	0.00433881403284646\\
184	0.00434347773825736\\
185	0.00434818915397096\\
186	0.00435294869211118\\
187	0.00435775676351291\\
188	0.00436261377730807\\
189	0.00436752014046877\\
190	0.00437247625731131\\
191	0.00437748252895453\\
192	0.00438253935272789\\
193	0.00438764712152498\\
194	0.00439280622309771\\
195	0.00439801703928436\\
196	0.00440327994516592\\
197	0.00440859530814329\\
198	0.00441396348692798\\
199	0.00441938483043717\\
200	0.00442485967658419\\
201	0.00443038835095385\\
202	0.00443597116535115\\
203	0.00444160841621043\\
204	0.00444730038285186\\
205	0.00445304732556862\\
206	0.00445884948352894\\
207	0.00446470707247404\\
208	0.00447062028219071\\
209	0.00447658927373774\\
210	0.0044826141763994\\
211	0.00448869508434091\\
212	0.00449483205293488\\
213	0.0045010250947277\\
214	0.00450727417500996\\
215	0.00451357920695347\\
216	0.00451994004627456\\
217	0.00452635648537871\\
218	0.00453282824694102\\
219	0.00453935497687283\\
220	0.00454593623662294\\
221	0.00455257149476086\\
222	0.00455926011778804\\
223	0.00456600136012463\\
224	0.00457279435322196\\
225	0.00457963809375516\\
226	0.00458653143086177\\
227	0.00459347305240365\\
228	0.00460046147025117\\
229	0.00460749500461732\\
230	0.00461457176750347\\
231	0.00462168964535071\\
232	0.00462884628093494\\
233	0.0046360390539445\\
234	0.00464326505527298\\
235	0.0046505210069576\\
236	0.00465780349876262\\
237	0.00466510876667863\\
238	0.00467243266248855\\
239	0.00467977062581579\\
240	0.00468711765648356\\
241	0.00469446828812083\\
242	0.00470181656430447\\
243	0.00470915601898387\\
244	0.00471647966353329\\
245	0.00472377998355318\\
246	0.00473104894955389\\
247	0.00473827804697045\\
248	0.0047454583326852\\
249	0.00475258052754034\\
250	0.00475963515751682\\
251	0.00476661276115319\\
252	0.00477350419045295\\
253	0.00478030099699438\\
254	0.00478699576098433\\
255	0.00479358310053279\\
256	0.00480006068343935\\
257	0.00480643056821392\\
258	0.00481270097541078\\
259	0.00481890168130836\\
260	0.00482513036488285\\
261	0.00483138572166826\\
262	0.00483766632757198\\
263	0.00484397062201996\\
264	0.00485029688657508\\
265	0.00485664321544166\\
266	0.00486300746246877\\
267	0.00486938705377548\\
268	0.0048757800140581\\
269	0.00488218440010652\\
270	0.00488859816855402\\
271	0.00489501917381586\\
272	0.00490144516664088\\
273	0.00490787379348034\\
274	0.00491430259692962\\
275	0.00492072901756244\\
276	0.00492715039755593\\
277	0.00493356398660151\\
278	0.00493996695070905\\
279	0.00494635638464548\\
280	0.00495272932888653\\
281	0.00495908279218016\\
282	0.00496541378042456\\
283	0.00497171933271478\\
284	0.00497799656666794\\
285	0.00498424273248062\\
286	0.00499045527173479\\
287	0.00499663188759957\\
288	0.00500276885455785\\
289	0.00500886179727862\\
290	0.00501490622294663\\
291	0.0050208975384141\\
292	0.00502683107286921\\
293	0.00503270210805545\\
294	0.00503850591519979\\
295	0.00504423779490658\\
296	0.00504989312489541\\
297	0.00505546741658601\\
298	0.00506095638156396\\
299	0.00506635600894974\\
300	0.00507166265461885\\
301	0.00507687314304434\\
302	0.00508198488219067\\
303	0.00508699599120746\\
304	0.00509190543908644\\
305	0.00509671318596088\\
306	0.00510142031598759\\
307	0.00510602917595886\\
308	0.00511054350191797\\
309	0.00511496851579151\\
310	0.00511931096968429\\
311	0.00512357910642704\\
312	0.00512778249307162\\
313	0.00513193166910957\\
314	0.00513603753568379\\
315	0.00514011038915024\\
316	0.00514415339170176\\
317	0.00514816558872805\\
318	0.00515214555600522\\
319	0.00515609192551052\\
320	0.00516000339726006\\
321	0.0051638787522915\\
322	0.00516771686684419\\
323	0.00517151672770696\\
324	0.00517527744867228\\
325	0.00517899828803098\\
326	0.00518267866697795\\
327	0.00518631818872863\\
328	0.00518991665801778\\
329	0.00519347410061639\\
330	0.00519699078257065\\
331	0.00520046722872537\\
332	0.00520390423989118\\
333	0.00520730290787344\\
334	0.00521066462749302\\
335	0.0052139911047977\\
336	0.00521728436050494\\
337	0.00522054672698069\\
338	0.00522378083724281\\
339	0.00522698960436026\\
340	0.00523017618951947\\
341	0.00523334395679822\\
342	0.00523649641238523\\
343	0.00523963712698979\\
344	0.00524276964205906\\
345	0.00524589736136628\\
346	0.00524902343209221\\
347	0.00525215062342403\\
348	0.00525528121650131\\
349	0.0052584169280298\\
350	0.00526155892451648\\
351	0.00526470821934065\\
352	0.0052678659099201\\
353	0.00527103317704476\\
354	0.0052742112832648\\
355	0.00527740157022393\\
356	0.00528060545483306\\
357	0.00528382442418496\\
358	0.00528706002911852\\
359	0.0052903138763541\\
360	0.00529358761915325\\
361	0.00529688294650302\\
362	0.0053002015708826\\
363	0.00530354521475302\\
364	0.00530691559602828\\
365	0.00531031441291814\\
366	0.00531374332865523\\
367	0.00531720395678034\\
368	0.00532069784783065\\
369	0.00532422647844399\\
370	0.00532779124403392\\
371	0.00533139345626187\\
372	0.00533503434648635\\
373	0.00533871507609726\\
374	0.00534243675401099\\
375	0.00534620046040236\\
376	0.00535000727230451\\
377	0.00535385827275344\\
378	0.0053577545480272\\
379	0.00536169718486027\\
380	0.00536568726769117\\
381	0.00536972587601087\\
382	0.00537381408188833\\
383	0.00537795294775574\\
384	0.00538214352454116\\
385	0.00538638685023603\\
386	0.00539068394898272\\
387	0.00539503583075542\\
388	0.00539944349168897\\
389	0.00540390791508142\\
390	0.00540843007305335\\
391	0.0054130109287946\\
392	0.00541765143926053\\
393	0.00542235255810581\\
394	0.00542711523856707\\
395	0.00543194043594388\\
396	0.00543682910931025\\
397	0.00544178222215611\\
398	0.00544680074245792\\
399	0.00545188564276676\\
400	0.00545703790040008\\
401	0.00546225849773876\\
402	0.00546754842262698\\
403	0.00547290866886857\\
404	0.00547834023680947\\
405	0.00548384413399068\\
406	0.00548942137585202\\
407	0.00549507298646269\\
408	0.00550079999925233\\
409	0.00550660345771517\\
410	0.00551248441606048\\
411	0.00551844393979057\\
412	0.00552448310619344\\
413	0.00553060300475403\\
414	0.00553680473750273\\
415	0.00554308941933771\\
416	0.00554945817835165\\
417	0.00555591215616768\\
418	0.00556245250828261\\
419	0.00556908040441308\\
420	0.00557579702884114\\
421	0.0055826035807562\\
422	0.00558950127459017\\
423	0.00559649134034323\\
424	0.00560357502389917\\
425	0.00561075358732925\\
426	0.00561802830918566\\
427	0.00562540048478577\\
428	0.00563287142648992\\
429	0.00564044246397542\\
430	0.00564811494450953\\
431	0.00565589023322228\\
432	0.00566376971337998\\
433	0.00567175478665855\\
434	0.00567984687341632\\
435	0.00568804741296627\\
436	0.00569635786384754\\
437	0.00570477970409607\\
438	0.00571331443151478\\
439	0.00572196356394295\\
440	0.00573072863952553\\
441	0.00573961121698208\\
442	0.00574861287587567\\
443	0.00575773521688192\\
444	0.00576697986205737\\
445	0.00577634845510774\\
446	0.00578584266165521\\
447	0.00579546416950444\\
448	0.00580521468890684\\
449	0.00581509595282272\\
450	0.00582510971718042\\
451	0.00583525776113227\\
452	0.00584554188730608\\
453	0.0058559639220517\\
454	0.00586652571568156\\
455	0.00587722914270388\\
456	0.00588807610204783\\
457	0.00589906851727852\\
458	0.00591020833680067\\
459	0.00592149753404935\\
460	0.00593293810766529\\
461	0.00594453208165292\\
462	0.00595628150551875\\
463	0.00596818845438693\\
464	0.00598025502908955\\
465	0.00599248335622785\\
466	0.00600487558820092\\
467	0.00601743390319778\\
468	0.00603016050514805\\
469	0.00604305762362689\\
470	0.00605612751370817\\
471	0.00606937245576052\\
472	0.00608279475517958\\
473	0.00609639674204979\\
474	0.00611018077072839\\
475	0.00612414921934346\\
476	0.00613830448919851\\
477	0.00615264900407409\\
478	0.00616718520941869\\
479	0.00618191557141915\\
480	0.00619684257594279\\
481	0.00621196872734266\\
482	0.00622729654711955\\
483	0.00624282857243494\\
484	0.00625856735447267\\
485	0.00627451545665015\\
486	0.00629067545268511\\
487	0.00630704992453122\\
488	0.00632364146020511\\
489	0.00634045265154001\\
490	0.00635748609191801\\
491	0.00637474437405498\\
492	0.00639223008794037\\
493	0.00640994581907178\\
494	0.0064278941471719\\
495	0.00644607764563728\\
496	0.00646449888204806\\
497	0.00648316042016958\\
498	0.00650206482400709\\
499	0.00652121466464175\\
500	0.00654061253078801\\
501	0.00656026104428372\\
502	0.00658016288206822\\
503	0.00660032080664221\\
504	0.00662073770756038\\
505	0.00664141665721423\\
506	0.00666236098506164\\
507	0.00668357437559948\\
508	0.00670506099682256\\
509	0.00672682566774802\\
510	0.00674886776502838\\
511	0.00677119585749369\\
512	0.00679382297813684\\
513	0.00681676657528694\\
514	0.00684004385018383\\
515	0.00686366344368807\\
516	0.00688763434733743\\
517	0.00691196385407253\\
518	0.00693665747087436\\
519	0.00696171964759408\\
520	0.00698715326698022\\
521	0.00701295919276243\\
522	0.00703913570363854\\
523	0.00706567763267601\\
524	0.00709257527891002\\
525	0.00711981302177701\\
526	0.0071473288879066\\
527	0.00717508630917037\\
528	0.00720315062282531\\
529	0.00723170825117882\\
530	0.00726080771265837\\
531	0.00729050452473134\\
532	0.00732083845906349\\
533	0.00735185058042227\\
534	0.00738358424139605\\
535	0.00741608727970591\\
536	0.0074494127711794\\
537	0.00748362000688747\\
538	0.0075187754227737\\
539	0.00755495371725533\\
540	0.00759223914403396\\
541	0.00763072064444217\\
542	0.00767050041992187\\
543	0.00771053336520066\\
544	0.00774814489342967\\
545	0.00778501922575503\\
546	0.00782234883417097\\
547	0.00786028395490776\\
548	0.00789883697694226\\
549	0.00793799584592659\\
550	0.00797774149567381\\
551	0.00801804850376936\\
552	0.00805888399757698\\
553	0.00810020597452262\\
554	0.00814196096787807\\
555	0.00818408040193254\\
556	0.00822647299553078\\
557	0.00826814713066404\\
558	0.00830933123458027\\
559	0.00835081763203065\\
560	0.00839260575580856\\
561	0.00843466894641207\\
562	0.00847698026063976\\
563	0.00851951301986419\\
564	0.00856218669075257\\
565	0.00860460466750246\\
566	0.00864738981450226\\
567	0.00869053926768215\\
568	0.00873403631435603\\
569	0.00877786229262354\\
570	0.00882199645485595\\
571	0.0088664158231443\\
572	0.00891109504152541\\
573	0.00895600622686392\\
574	0.0090011188209025\\
575	0.00904639944683076\\
576	0.00909181177477994\\
577	0.00913731640198181\\
578	0.00918287075500257\\
579	0.00922842902355833\\
580	0.0092739421380386\\
581	0.00931935780612553\\
582	0.00936462062793334\\
583	0.00940967231402828\\
584	0.00945445203657523\\
585	0.0094988969504773\\
586	0.00954294292772307\\
587	0.00958652555102227\\
588	0.00962958140299686\\
589	0.00967204963943729\\
590	0.0097138736812812\\
591	0.0097550024176802\\
592	0.00979538909575469\\
593	0.00983498278816331\\
594	0.00987369853931868\\
595	0.00991118387968948\\
596	0.00994658651044256\\
597	0.00997788999445116\\
598	0.010000292044645\\
599	0\\
600	0\\
};
\addplot [color=mycolor1,solid,forget plot]
  table[row sep=crcr]{%
1	0.00409772721480342\\
2	0.00409779894368203\\
3	0.00409787229004294\\
4	0.00409794729009925\\
5	0.00409802398086176\\
6	0.00409810240015601\\
7	0.00409818258663963\\
8	0.00409826457981992\\
9	0.00409834842007161\\
10	0.00409843414865529\\
11	0.00409852180773607\\
12	0.00409861144040237\\
13	0.00409870309068543\\
14	0.00409879680357872\\
15	0.00409889262505827\\
16	0.00409899060210289\\
17	0.00409909078271514\\
18	0.00409919321594221\\
19	0.00409929795189766\\
20	0.00409940504178319\\
21	0.00409951453791119\\
22	0.00409962649372711\\
23	0.00409974096383279\\
24	0.00409985800400979\\
25	0.00409997767124344\\
26	0.00410010002374688\\
27	0.00410022512098599\\
28	0.00410035302370438\\
29	0.00410048379394882\\
30	0.00410061749509548\\
31	0.0041007541918758\\
32	0.00410089395040352\\
33	0.00410103683820191\\
34	0.00410118292423112\\
35	0.0041013322789164\\
36	0.00410148497417633\\
37	0.00410164108345169\\
38	0.00410180068173495\\
39	0.00410196384559975\\
40	0.00410213065323089\\
41	0.00410230118445509\\
42	0.00410247552077149\\
43	0.00410265374538353\\
44	0.00410283594323015\\
45	0.00410302220101832\\
46	0.0041032126072553\\
47	0.00410340725228164\\
48	0.00410360622830462\\
49	0.00410380962943171\\
50	0.00410401755170488\\
51	0.00410423009313479\\
52	0.00410444735373589\\
53	0.00410466943556137\\
54	0.00410489644273873\\
55	0.00410512848150574\\
56	0.00410536566024641\\
57	0.00410560808952759\\
58	0.00410585588213612\\
59	0.00410610915311544\\
60	0.00410636801980324\\
61	0.00410663260186913\\
62	0.00410690302135278\\
63	0.00410717940270171\\
64	0.00410746187281014\\
65	0.00410775056105755\\
66	0.00410804559934759\\
67	0.00410834712214699\\
68	0.0041086552665253\\
69	0.00410897017219387\\
70	0.0041092919815457\\
71	0.00410962083969529\\
72	0.00410995689451837\\
73	0.00411030029669195\\
74	0.00411065119973452\\
75	0.00411100976004612\\
76	0.00411137613694869\\
77	0.00411175049272651\\
78	0.00411213299266626\\
79	0.00411252380509742\\
80	0.00411292310143299\\
81	0.00411333105620942\\
82	0.00411374784712704\\
83	0.00411417365509031\\
84	0.00411460866424798\\
85	0.00411505306203289\\
86	0.00411550703920199\\
87	0.00411597078987582\\
88	0.00411644451157802\\
89	0.00411692840527417\\
90	0.00411742267541044\\
91	0.00411792752995182\\
92	0.00411844318041973\\
93	0.00411896984192869\\
94	0.00411950773322249\\
95	0.0041200570767091\\
96	0.00412061809849419\\
97	0.00412119102841346\\
98	0.00412177610006287\\
99	0.00412237355082669\\
100	0.00412298362190276\\
101	0.0041236065583249\\
102	0.00412424260898072\\
103	0.00412489202662549\\
104	0.00412555506788985\\
105	0.00412623199328115\\
106	0.0041269230671765\\
107	0.00412762855780644\\
108	0.00412834873722706\\
109	0.00412908388127877\\
110	0.0041298342695286\\
111	0.00413060018519369\\
112	0.00413138191504216\\
113	0.00413217974926719\\
114	0.00413299398132975\\
115	0.00413382490776419\\
116	0.00413467282794\\
117	0.004135538043773\\
118	0.00413642085937625\\
119	0.00413732158064065\\
120	0.00413824051473449\\
121	0.00413917796950705\\
122	0.00414013425278189\\
123	0.00414110967152146\\
124	0.00414210453084355\\
125	0.00414311913286581\\
126	0.00414415377535373\\
127	0.00414520875014204\\
128	0.00414628434129883\\
129	0.0041473808229959\\
130	0.00414849845704804\\
131	0.00414963749007916\\
132	0.00415079815027464\\
133	0.00415198064367354\\
134	0.00415318514996139\\
135	0.00415441181772467\\
136	0.00415566075913799\\
137	0.00415693204407067\\
138	0.00415822569362178\\
139	0.00415954167312933\\
140	0.00416087988475102\\
141	0.00416224015978999\\
142	0.0041636222510447\\
143	0.00416502582561039\\
144	0.00416645045876467\\
145	0.00416789562984994\\
146	0.00416936072144634\\
147	0.00417084502364335\\
148	0.00417234774590553\\
149	0.00417386803993707\\
150	0.00417540503813451\\
151	0.0041769579138338\\
152	0.00417852597327299\\
153	0.00418010928885501\\
154	0.00418170799276501\\
155	0.00418332221827263\\
156	0.00418495209976837\\
157	0.0041865977728027\\
158	0.00418825937412943\\
159	0.00418993704175247\\
160	0.00419163091497726\\
161	0.00419334113446629\\
162	0.00419506784230025\\
163	0.00419681118204391\\
164	0.00419857129881815\\
165	0.00420034833937799\\
166	0.00420214245219747\\
167	0.00420395378756188\\
168	0.00420578249766757\\
169	0.00420762873673033\\
170	0.00420949266110283\\
171	0.00421137442940206\\
172	0.00421327420264693\\
173	0.00421519214440761\\
174	0.00421712842096694\\
175	0.00421908320149493\\
176	0.00422105665823793\\
177	0.00422304896672268\\
178	0.00422506030597716\\
179	0.00422709085876897\\
180	0.00422914081186331\\
181	0.00423121035630107\\
182	0.00423329968769973\\
183	0.00423540900657803\\
184	0.00423753851870656\\
185	0.00423968843548626\\
186	0.00424185897435704\\
187	0.00424405035923917\\
188	0.00424626282100948\\
189	0.00424849659801558\\
190	0.00425075193663102\\
191	0.00425302909185485\\
192	0.00425532832795867\\
193	0.00425764991918607\\
194	0.00425999415050744\\
195	0.00426236131843595\\
196	0.00426475173190895\\
197	0.00426716571324103\\
198	0.00426960359915399\\
199	0.00427206574189107\\
200	0.00427455251042226\\
201	0.00427706429174896\\
202	0.00427960149231631\\
203	0.00428216453954314\\
204	0.00428475388347959\\
205	0.00428736999860489\\
206	0.00429001338577685\\
207	0.00429268457434783\\
208	0.0042953841244628\\
209	0.00429811262955571\\
210	0.00430087071906375\\
211	0.00430365906137983\\
212	0.00430647836706625\\
213	0.0043093293923543\\
214	0.00431221294295856\\
215	0.0043151298782353\\
216	0.00431808111571905\\
217	0.00432106763607472\\
218	0.00432409048850475\\
219	0.00432715079665611\\
220	0.00433024976507524\\
221	0.0043333886862628\\
222	0.00433656894838491\\
223	0.00433979204370076\\
224	0.00434305957777054\\
225	0.00434637327951035\\
226	0.00434973501216295\\
227	0.00435314678525194\\
228	0.00435661076758406\\
229	0.00436012930135168\\
230	0.00436370491735915\\
231	0.00436734035132117\\
232	0.00437103856100468\\
233	0.00437480274356612\\
234	0.00437863635279915\\
235	0.00438254314485878\\
236	0.00438652718250342\\
237	0.00439059285877664\\
238	0.00439474492172882\\
239	0.00439898849948371\\
240	0.00440332912502909\\
241	0.00440777275985302\\
242	0.0044123258151979\\
243	0.00441699516925015\\
244	0.00442178817798137\\
245	0.00442671267656985\\
246	0.00443177696730284\\
247	0.00443698978851966\\
248	0.0044423602574157\\
249	0.00444789777725692\\
250	0.00445361189653253\\
251	0.00445951210304723\\
252	0.00446560752534391\\
253	0.00447190650305558\\
254	0.00447841604892367\\
255	0.00448514108665174\\
256	0.00449208342506626\\
257	0.00449924038604222\\
258	0.00450660297522872\\
259	0.00451414025698812\\
260	0.0045217525252719\\
261	0.00452944028791647\\
262	0.00453720402905694\\
263	0.00454504420549145\\
264	0.00455296124256425\\
265	0.00456095552955954\\
266	0.00456902741892628\\
267	0.00457717730623837\\
268	0.00458540556282764\\
269	0.00459371251732887\\
270	0.00460209845009204\\
271	0.00461056358702463\\
272	0.00461910809282056\\
273	0.00462773206353394\\
274	0.00463643551845733\\
275	0.00464521839126953\\
276	0.00465408052042462\\
277	0.00466302163876009\\
278	0.00467204136227265\\
279	0.00468113917773462\\
280	0.0046903144267936\\
281	0.00469956626631459\\
282	0.00470889371913389\\
283	0.00471829568279564\\
284	0.00472777082691935\\
285	0.00473731755503774\\
286	0.00474693394218696\\
287	0.00475661756059487\\
288	0.00476636552940042\\
289	0.00477617451557519\\
290	0.00478604067846271\\
291	0.0047959596220509\\
292	0.0048059263359506\\
293	0.00481593480188697\\
294	0.00482597820079109\\
295	0.0048360488286401\\
296	0.00484613800659609\\
297	0.00485623598621501\\
298	0.00486633185105257\\
299	0.00487641341673388\\
300	0.00488646713248571\\
301	0.00489647798800039\\
302	0.00490642942862883\\
303	0.00491630326797561\\
304	0.00492607946572709\\
305	0.00493573738047742\\
306	0.00494525536311096\\
307	0.00495461088938635\\
308	0.00496378103654142\\
309	0.00497274319842527\\
310	0.00498147612422024\\
311	0.0049899613955635\\
312	0.00499818550441449\\
313	0.00500614280671684\\
314	0.0050138383289735\\
315	0.00502129149402076\\
316	0.00502870260229165\\
317	0.00503610288391933\\
318	0.00504348774286446\\
319	0.00505085233759865\\
320	0.0050581915765222\\
321	0.00506550011410386\\
322	0.00507277234802151\\
323	0.00508000242071015\\
324	0.00508718422583651\\
325	0.00509431141980478\\
326	0.00510137743966654\\
327	0.00510837552893968\\
328	0.00511529877420358\\
329	0.00512214015415874\\
330	0.00512889259727722\\
331	0.00513554905037802\\
332	0.00514210256268315\\
333	0.00514854638708865\\
334	0.00515487409757838\\
335	0.00516107971488429\\
336	0.00516715784227585\\
337	0.00517310383639541\\
338	0.00517891399802226\\
339	0.00518458578150427\\
340	0.0051901180217842\\
341	0.00519551118469681\\
342	0.00520076765133354\\
343	0.00520589199788326\\
344	0.0052108912081264\\
345	0.00521577481157448\\
346	0.0052205548933601\\
347	0.00522524589882257\\
348	0.00522986412393199\\
349	0.0052344267388042\\
350	0.00523894945645597\\
351	0.00524343605940803\\
352	0.00524788560074577\\
353	0.00525229734303664\\
354	0.00525667078600199\\
355	0.00526100569488241\\
356	0.0052653021289482\\
357	0.00526956046970549\\
358	0.00527378144828387\\
359	0.00527796617157201\\
360	0.00528211614594839\\
361	0.00528623329710285\\
362	0.00529031998433898\\
363	0.00529437900711757\\
364	0.00529841360059086\\
365	0.00530242741703889\\
366	0.00530642449123779\\
367	0.00531040918682558\\
368	0.00531438612091678\\
369	0.00531836006481072\\
370	0.00532233581986472\\
371	0.00532631806978906\\
372	0.00533031121401844\\
373	0.00533431919234488\\
374	0.00533834531948073\\
375	0.00534239216076209\\
376	0.0053464615391883\\
377	0.00535055505978985\\
378	0.0053546744226983\\
379	0.00535882141702043\\
380	0.00536299791282706\\
381	0.00536720585112944\\
382	0.005371447231776\\
383	0.00537572409929912\\
384	0.00538003852682219\\
385	0.00538439259824991\\
386	0.00538878838911245\\
387	0.00539322794662035\\
388	0.0053977132697017\\
389	0.00540224629003136\\
390	0.00540682885531752\\
391	0.00541146271635344\\
392	0.00541614951952548\\
393	0.00542089080652173\\
394	0.00542568802278514\\
395	0.00543054253561985\\
396	0.00543545566152567\\
397	0.00544042869990902\\
398	0.00544546294960872\\
399	0.00545055970798066\\
400	0.00545572026769364\\
401	0.00546094591375229\\
402	0.00546623792085765\\
403	0.00547159755122089\\
404	0.00547702605294676\\
405	0.00548252465909634\\
406	0.00548809458752136\\
407	0.00549373704153303\\
408	0.00549945321142147\\
409	0.00550524427678036\\
410	0.00551111140950813\\
411	0.00551705577726307\\
412	0.00552307854704392\\
413	0.0055291808884754\\
414	0.00553536397632439\\
415	0.00554162899181428\\
416	0.00554797712303811\\
417	0.00555440956521041\\
418	0.00556092752104443\\
419	0.00556753220125155\\
420	0.00557422482515346\\
421	0.00558100662139226\\
422	0.00558787882871715\\
423	0.00559484269682172\\
424	0.00560189948720002\\
425	0.0056090504739877\\
426	0.00561629694475481\\
427	0.00562364020122091\\
428	0.00563108155987295\\
429	0.0056386223524815\\
430	0.00564626392653087\\
431	0.00565400764560148\\
432	0.00566185488974582\\
433	0.00566980705587261\\
434	0.00567786555813514\\
435	0.00568603182831905\\
436	0.00569430731622558\\
437	0.00570269349004649\\
438	0.00571119183672758\\
439	0.00571980386231872\\
440	0.00572853109230956\\
441	0.00573737507195137\\
442	0.00574633736656675\\
443	0.00575541956184987\\
444	0.00576462326416059\\
445	0.00577395010081486\\
446	0.00578340172037297\\
447	0.00579297979292529\\
448	0.00580268601037476\\
449	0.00581252208671591\\
450	0.00582248975830964\\
451	0.00583259078415376\\
452	0.00584282694614889\\
453	0.00585320004935955\\
454	0.00586371192227051\\
455	0.00587436441703796\\
456	0.00588515940973543\\
457	0.00589609880059389\\
458	0.0059071845142356\\
459	0.0059184184999007\\
460	0.00592980273166581\\
461	0.00594133920865384\\
462	0.00595302995523355\\
463	0.00596487702120814\\
464	0.00597688248199155\\
465	0.00598904843877122\\
466	0.00600137701865596\\
467	0.00601387037480744\\
468	0.00602653068655392\\
469	0.00603936015948441\\
470	0.00605236102552181\\
471	0.00606553554297312\\
472	0.00607888599655534\\
473	0.00609241469739513\\
474	0.00610612398300072\\
475	0.00612001621720504\\
476	0.00613409379007817\\
477	0.00614835911780911\\
478	0.00616281464255603\\
479	0.00617746283226529\\
480	0.00619230618046065\\
481	0.00620734720600465\\
482	0.00622258845283574\\
483	0.00623803248968672\\
484	0.00625368190979207\\
485	0.00626953933059434\\
486	0.00628560739346397\\
487	0.00630188876345045\\
488	0.00631838612908839\\
489	0.00633510220228862\\
490	0.00635203971835239\\
491	0.00636920143615627\\
492	0.00638659013856709\\
493	0.00640420863316077\\
494	0.00642205975333513\\
495	0.00644014635992797\\
496	0.00645847134347519\\
497	0.006477037627273\\
498	0.00649584817144181\\
499	0.0065149059782292\\
500	0.00653421409883596\\
501	0.00655377564210219\\
502	0.00657359378545241\\
503	0.00659367178856756\\
504	0.00661401301032988\\
505	0.00663462092967143\\
506	0.00665549917104909\\
507	0.00667665153536294\\
508	0.0066980820372294\\
509	0.00671979494960482\\
510	0.00674179493719375\\
511	0.0067640868956916\\
512	0.0067866759769133\\
513	0.00680956400369565\\
514	0.00683275491124096\\
515	0.00685626196973593\\
516	0.0068800986408156\\
517	0.00690428490325568\\
518	0.00692883527291861\\
519	0.00695375870382165\\
520	0.00697906425632247\\
521	0.00700475949871382\\
522	0.00703085045342747\\
523	0.00705734188569942\\
524	0.00708423681480593\\
525	0.00711153593740922\\
526	0.00713923746350955\\
527	0.00716733497321819\\
528	0.00719581077674005\\
529	0.00722455855703865\\
530	0.00725363372862162\\
531	0.00728313100693132\\
532	0.00731320423426021\\
533	0.00734390257698169\\
534	0.00737527918989368\\
535	0.00740737901823451\\
536	0.00744024906925833\\
537	0.00747393941753061\\
538	0.00750850490350621\\
539	0.00754400577180493\\
540	0.00758050847644705\\
541	0.00761808214679199\\
542	0.00765680105430943\\
543	0.00769676637028498\\
544	0.00773813039112718\\
545	0.00777868539256093\\
546	0.0078168880070114\\
547	0.00785492834535674\\
548	0.00789347294193416\\
549	0.00793261873474102\\
550	0.0079723685954058\\
551	0.00801270533222008\\
552	0.00805360340878692\\
553	0.00809502949182715\\
554	0.00813694106531913\\
555	0.00817928405928756\\
556	0.0082219892856467\\
557	0.00826498033629833\\
558	0.00830749515365131\\
559	0.00834919568350826\\
560	0.00839115862383602\\
561	0.00843340529491727\\
562	0.00847591105247397\\
563	0.00851864731722591\\
564	0.00856158658383183\\
565	0.0086045920915919\\
566	0.00864738978534756\\
567	0.00869053926739589\\
568	0.00873403631432268\\
569	0.00877786229261397\\
570	0.00882199645485205\\
571	0.00886641582314234\\
572	0.00891109504152458\\
573	0.00895600622686346\\
574	0.00900111882090225\\
575	0.00904639944683062\\
576	0.00909181177477987\\
577	0.00913731640198179\\
578	0.00918287075500256\\
579	0.00922842902355833\\
580	0.00927394213803861\\
581	0.00931935780612554\\
582	0.00936462062793335\\
583	0.00940967231402828\\
584	0.00945445203657523\\
585	0.0094988969504773\\
586	0.00954294292772307\\
587	0.00958652555102228\\
588	0.00962958140299686\\
589	0.0096720496394373\\
590	0.0097138736812812\\
591	0.0097550024176802\\
592	0.00979538909575469\\
593	0.00983498278816331\\
594	0.00987369853931868\\
595	0.00991118387968948\\
596	0.00994658651044256\\
597	0.00997788999445116\\
598	0.010000292044645\\
599	0\\
600	0\\
};
\addplot [color=mycolor2,solid,forget plot]
  table[row sep=crcr]{%
1	0.00409747161189642\\
2	0.00409753539813578\\
3	0.00409760055013061\\
4	0.00409766709594947\\
5	0.00409773506418996\\
6	0.00409780448398646\\
7	0.00409787538501809\\
8	0.00409794779751668\\
9	0.00409802175227469\\
10	0.00409809728065326\\
11	0.00409817441458998\\
12	0.00409825318660705\\
13	0.00409833362981889\\
14	0.00409841577794048\\
15	0.00409849966529485\\
16	0.00409858532682111\\
17	0.00409867279808217\\
18	0.00409876211527228\\
19	0.00409885331522501\\
20	0.00409894643542064\\
21	0.00409904151399345\\
22	0.0040991385897394\\
23	0.00409923770212313\\
24	0.00409933889128525\\
25	0.00409944219804905\\
26	0.00409954766392779\\
27	0.00409965533113078\\
28	0.00409976524257016\\
29	0.00409987744186714\\
30	0.00409999197335779\\
31	0.00410010888209922\\
32	0.00410022821387476\\
33	0.0041003500151994\\
34	0.00410047433332463\\
35	0.00410060121624315\\
36	0.00410073071269329\\
37	0.00410086287216279\\
38	0.00410099774489229\\
39	0.00410113538187862\\
40	0.00410127583487759\\
41	0.00410141915640617\\
42	0.00410156539974421\\
43	0.00410171461893575\\
44	0.00410186686878994\\
45	0.00410202220488092\\
46	0.00410218068354748\\
47	0.0041023423618923\\
48	0.00410250729777974\\
49	0.00410267554983383\\
50	0.00410284717743505\\
51	0.0041030222407164\\
52	0.00410320080055873\\
53	0.00410338291858535\\
54	0.00410356865715563\\
55	0.00410375807935773\\
56	0.00410395124900057\\
57	0.00410414823060455\\
58	0.00410434908939137\\
59	0.00410455389127309\\
60	0.00410476270283974\\
61	0.00410497559134594\\
62	0.00410519262469638\\
63	0.0041054138714302\\
64	0.00410563940070404\\
65	0.00410586928227383\\
66	0.00410610358647515\\
67	0.00410634238420284\\
68	0.00410658574688835\\
69	0.00410683374647658\\
70	0.0041070864554006\\
71	0.00410734394655529\\
72	0.00410760629326943\\
73	0.00410787356927631\\
74	0.00410814584868262\\
75	0.00410842320593601\\
76	0.00410870571579089\\
77	0.00410899345327286\\
78	0.00410928649364127\\
79	0.00410958491235057\\
80	0.00410988878500935\\
81	0.0041101981873385\\
82	0.00411051319512743\\
83	0.00411083388418873\\
84	0.00411116033031089\\
85	0.00411149260921013\\
86	0.00411183079648004\\
87	0.00411217496753997\\
88	0.00411252519758179\\
89	0.00411288156151534\\
90	0.00411324413391216\\
91	0.00411361298894819\\
92	0.00411398820034483\\
93	0.0041143698413086\\
94	0.00411475798447002\\
95	0.00411515270182075\\
96	0.00411555406465005\\
97	0.00411596214347976\\
98	0.00411637700799863\\
99	0.00411679872699537\\
100	0.00411722736829093\\
101	0.00411766299867002\\
102	0.0041181056838117\\
103	0.00411855548821906\\
104	0.00411901247514837\\
105	0.00411947670653741\\
106	0.0041199482429331\\
107	0.00412042714341815\\
108	0.0041209134655373\\
109	0.0041214072652224\\
110	0.00412190859671682\\
111	0.00412241751249883\\
112	0.00412293406320339\\
113	0.00412345829754346\\
114	0.00412399026222934\\
115	0.0041245300018866\\
116	0.00412507755897211\\
117	0.00412563297368823\\
118	0.00412619628389468\\
119	0.00412676752501859\\
120	0.00412734672996138\\
121	0.00412793392900395\\
122	0.00412852914970932\\
123	0.00412913241682325\\
124	0.00412974375217359\\
125	0.00413036317456882\\
126	0.00413099069969742\\
127	0.00413162634002976\\
128	0.00413227010472435\\
129	0.00413292199954316\\
130	0.00413358202677864\\
131	0.00413425018519948\\
132	0.00413492647002104\\
133	0.00413561087291054\\
134	0.00413630338203718\\
135	0.00413700398218096\\
136	0.00413771265491638\\
137	0.00413842937889044\\
138	0.00413915413021722\\
139	0.00413988688301409\\
140	0.00414062761010833\\
141	0.00414137628394292\\
142	0.00414213287771258\\
143	0.00414289736675549\\
144	0.00414366973022058\\
145	0.00414444995301399\\
146	0.0041452380280014\\
147	0.00414603395840073\\
148	0.00414683776023207\\
149	0.00414764946459766\\
150	0.00414846911946697\\
151	0.00414929679070874\\
152	0.00415013256280046\\
153	0.00415097652319167\\
154	0.00415182876115608\\
155	0.00415268936787204\\
156	0.00415355843650752\\
157	0.00415443606230929\\
158	0.00415532234269612\\
159	0.00415621737735695\\
160	0.00415712126835361\\
161	0.00415803412022869\\
162	0.00415895604011845\\
163	0.00415988713787103\\
164	0.00416082752617065\\
165	0.00416177732066735\\
166	0.00416273664011272\\
167	0.00416370560650229\\
168	0.00416468434522427\\
169	0.00416567298521502\\
170	0.00416667165912175\\
171	0.00416768050347225\\
172	0.00416869965885266\\
173	0.0041697292700925\\
174	0.00417076948645847\\
175	0.00417182046185595\\
176	0.00417288235503981\\
177	0.00417395532983361\\
178	0.00417503955535845\\
179	0.00417613520627111\\
180	0.00417724246301193\\
181	0.0041783615120632\\
182	0.00417949254621711\\
183	0.0041806357648553\\
184	0.00418179137423853\\
185	0.00418295958780796\\
186	0.00418414062649785\\
187	0.00418533471905979\\
188	0.00418654210239906\\
189	0.00418776302192296\\
190	0.00418899773190199\\
191	0.00419024649584326\\
192	0.00419150958687672\\
193	0.00419278728815461\\
194	0.00419407989326373\\
195	0.00419538770665124\\
196	0.00419671104406349\\
197	0.00419805023299822\\
198	0.00419940561317019\\
199	0.00420077753698996\\
200	0.00420216637005581\\
201	0.00420357249165847\\
202	0.00420499629529858\\
203	0.00420643818921634\\
204	0.0042078985969331\\
205	0.00420937795780344\\
206	0.00421087672757834\\
207	0.00421239537897701\\
208	0.00421393440226728\\
209	0.00421549430585238\\
210	0.0042170756168629\\
211	0.00421867888175132\\
212	0.00422030466688705\\
213	0.00422195355914864\\
214	0.0042236261665097\\
215	0.00422532311861412\\
216	0.00422704506733573\\
217	0.00422879268731582\\
218	0.00423056667647189\\
219	0.00423236775646867\\
220	0.00423419667314141\\
221	0.004236054196859\\
222	0.00423794112281342\\
223	0.0042398582712181\\
224	0.0042418064873953\\
225	0.00424378664172982\\
226	0.0042457996294605\\
227	0.00424784637027782\\
228	0.00424992780768809\\
229	0.00425204490809677\\
230	0.00425419865955228\\
231	0.00425639007007598\\
232	0.0042586201655153\\
233	0.00426088998707641\\
234	0.00426320058987421\\
235	0.00426555303748082\\
236	0.00426794839720525\\
237	0.00427038773444267\\
238	0.00427287210592968\\
239	0.0042754025517495\\
240	0.00427798008592344\\
241	0.00428060568541552\\
242	0.00428328027738017\\
243	0.00428600472449318\\
244	0.00428877980823606\\
245	0.0042916062100587\\
246	0.0042944844904334\\
247	0.00429741506595932\\
248	0.00430039818488352\\
249	0.00430343390169871\\
250	0.0043065220518256\\
251	0.00430966222770246\\
252	0.00431285375894432\\
253	0.00431609570450518\\
254	0.00431938685736227\\
255	0.00432272577187187\\
256	0.00432611082522832\\
257	0.0043295403286064\\
258	0.00433301270856113\\
259	0.0043365270143948\\
260	0.0043400842428449\\
261	0.00434368544098068\\
262	0.00434733170972222\\
263	0.00435102420769006\\
264	0.00435476415560266\\
265	0.00435855284221954\\
266	0.00436239163542775\\
267	0.00436628198165968\\
268	0.00437022541053829\\
269	0.00437422354125197\\
270	0.00437827808949937\\
271	0.0043823908750584\\
272	0.00438656383003684\\
273	0.00439079900786693\\
274	0.00439509859310615\\
275	0.00439946491211647\\
276	0.00440390044467257\\
277	0.00440840783653867\\
278	0.00441298991295949\\
279	0.00441764969279\\
280	0.00442239040296108\\
281	0.00442721550334059\\
282	0.00443212870843422\\
283	0.0044371340008761\\
284	0.00444223565292052\\
285	0.00444743824742085\\
286	0.00445274669560388\\
287	0.00445816627621191\\
288	0.00446370267254473\\
289	0.00446936200821571\\
290	0.00447515088662875\\
291	0.00448107642829197\\
292	0.00448714628446144\\
293	0.00449336870794162\\
294	0.00449975259858886\\
295	0.00450630754965302\\
296	0.00451304389387529\\
297	0.00451997274778971\\
298	0.004527106052044\\
299	0.00453445660470557\\
300	0.00454203808334817\\
301	0.00454986505006974\\
302	0.00455795293147298\\
303	0.00456631796775024\\
304	0.00457497720938437\\
305	0.00458394828581726\\
306	0.00459324913323066\\
307	0.00460289761933221\\
308	0.00461291100682783\\
309	0.00462330520184831\\
310	0.00463409371679639\\
311	0.00464528625435607\\
312	0.00465688678276178\\
313	0.00466889085781066\\
314	0.00468128210320214\\
315	0.00469402777432611\\
316	0.00470691739611307\\
317	0.00471991242935172\\
318	0.00473301002317151\\
319	0.00474620688294381\\
320	0.00475949917767092\\
321	0.00477288264150297\\
322	0.0047863524755296\\
323	0.00479990320100424\\
324	0.00481352857623578\\
325	0.00482722150450891\\
326	0.00484097393982608\\
327	0.0048547768328975\\
328	0.00486861970165124\\
329	0.00488249054713525\\
330	0.00489637589406776\\
331	0.00491026063622077\\
332	0.00492412785784585\\
333	0.00493795855675552\\
334	0.00495173168165891\\
335	0.00496542528834626\\
336	0.00497901542468126\\
337	0.00499247611671266\\
338	0.00500577942968237\\
339	0.00501889565646387\\
340	0.00503179379203509\\
341	0.00504444043675032\\
342	0.005056798614081\\
343	0.00506883063675311\\
344	0.00508049923158478\\
345	0.0050917691756241\\
346	0.00510260962124097\\
347	0.00511299734576308\\
348	0.00512292124026069\\
349	0.00513238847118869\\
350	0.00514145356279036\\
351	0.00515044292812221\\
352	0.00515934757340538\\
353	0.0051681582470874\\
354	0.00517686548853945\\
355	0.00518545969417062\\
356	0.00519393120286888\\
357	0.00520227039350545\\
358	0.00521046779163548\\
359	0.00521851417751173\\
360	0.00522640072029858\\
361	0.00523411914484358\\
362	0.00524166192587091\\
363	0.00524902252478635\\
364	0.00525619569649764\\
365	0.00526317784285559\\
366	0.00526996735421263\\
367	0.00527656497134352\\
368	0.00528297415389167\\
369	0.00528920143293488\\
370	0.0052952567130429\\
371	0.00530115347201255\\
372	0.00530690878823694\\
373	0.00531254308939831\\
374	0.00531807947203472\\
375	0.00532354238354427\\
376	0.00532895418314075\\
377	0.00533432142320478\\
378	0.0053396445335948\\
379	0.00534492439119021\\
380	0.00535016235681699\\
381	0.00535536030967686\\
382	0.0053605206769865\\
383	0.0053656464553484\\
384	0.00537074122179324\\
385	0.00537580913203551\\
386	0.00538085490254376\\
387	0.00538588377271432\\
388	0.00539090144328955\\
389	0.00539591398731883\\
390	0.00540092773045593\\
391	0.00540594909860276\\
392	0.00541098443331722\\
393	0.00541603977941393\\
394	0.00542112065567026\\
395	0.00542623182967747\\
396	0.00543137713273228\\
397	0.00543655937249408\\
398	0.00544178091217529\\
399	0.0054470441437416\\
400	0.00545235153771615\\
401	0.0054577056261782\\
402	0.00546310898322712\\
403	0.00546856420313171\\
404	0.0054740738765625\\
405	0.00547964056552351\\
406	0.0054852667778617\\
407	0.0054909549425352\\
408	0.00549670738714891\\
409	0.00550252631958955\\
410	0.00550841381585341\\
411	0.00551437181627652\\
412	0.00552040213222794\\
413	0.00552650646469684\\
414	0.00553268643478951\\
415	0.00553894362349785\\
416	0.00554527959897722\\
417	0.00555169592110127\\
418	0.00555819413806102\\
419	0.00556477578347433\\
420	0.00557144237415245\\
421	0.00557819540866512\\
422	0.00558503636682932\\
423	0.00559196671021332\\
424	0.00559898788369828\\
425	0.00560610131806688\\
426	0.00561330843349454\\
427	0.00562061064370411\\
428	0.0056280093604178\\
429	0.00563550599761747\\
430	0.00564310197503964\\
431	0.00565079872034391\\
432	0.00565859766997983\\
433	0.00566650026965315\\
434	0.00567450797493938\\
435	0.00568262225203333\\
436	0.00569084457861544\\
437	0.00569917644480915\\
438	0.00570761935419601\\
439	0.00571617482485011\\
440	0.00572484439035047\\
441	0.00573362960073054\\
442	0.0057425320233305\\
443	0.00575155324352999\\
444	0.00576069486535867\\
445	0.00576995851200706\\
446	0.00577934582628691\\
447	0.00578885847108561\\
448	0.00579849812982527\\
449	0.00580826650692118\\
450	0.00581816532823506\\
451	0.00582819634151819\\
452	0.00583836131684128\\
453	0.00584866204700868\\
454	0.00585910034795612\\
455	0.00586967805913298\\
456	0.00588039704387136\\
457	0.00589125918974586\\
458	0.0059022664089277\\
459	0.00591342063853709\\
460	0.00592472384099479\\
461	0.00593617800437307\\
462	0.00594778514274589\\
463	0.00595954729653819\\
464	0.00597146653287447\\
465	0.00598354494592692\\
466	0.0059957846572637\\
467	0.00600818781619825\\
468	0.00602075660013988\\
469	0.00603349321494683\\
470	0.00604639989528218\\
471	0.00605947890497346\\
472	0.00607273253737647\\
473	0.00608616311574416\\
474	0.00609977299360119\\
475	0.00611356455512523\\
476	0.00612754021553649\\
477	0.00614170242149662\\
478	0.00615605365151884\\
479	0.00617059641639156\\
480	0.00618533325961788\\
481	0.00620026675787385\\
482	0.00621539952148958\\
483	0.00623073419495676\\
484	0.00624627345746848\\
485	0.00626202002349724\\
486	0.00627797664341869\\
487	0.0062941461041901\\
488	0.0063105312300946\\
489	0.00632713488356401\\
490	0.00634395996609632\\
491	0.00636100941928663\\
492	0.00637828622599498\\
493	0.00639579341167863\\
494	0.00641353404592295\\
495	0.00643151124421204\\
496	0.00644972816998979\\
497	0.00646818803707225\\
498	0.00648689411248712\\
499	0.0065058497198329\\
500	0.00652505824327141\\
501	0.006544523132295\\
502	0.006564247907443\\
503	0.00658423616718477\\
504	0.00660449159624057\\
505	0.0066250179756797\\
506	0.0066458191952222\\
507	0.00666689926828206\\
508	0.00668826235043057\\
509	0.00670991276214244\\
510	0.00673185501589693\\
511	0.00675409385627773\\
512	0.00677663430915819\\
513	0.00679948178453945\\
514	0.00682264204388124\\
515	0.00684612107772275\\
516	0.00686992525165649\\
517	0.00689405601162315\\
518	0.00691852182093339\\
519	0.00694333742202316\\
520	0.00696851780686047\\
521	0.006994085086136\\
522	0.00702005457319023\\
523	0.00704643675525956\\
524	0.00707324204297694\\
525	0.0071004793565474\\
526	0.00712815559940787\\
527	0.00715627592819553\\
528	0.00718484326852532\\
529	0.00721385844412268\\
530	0.0072433155463926\\
531	0.00727317878533223\\
532	0.00730338078918359\\
533	0.0073339813864525\\
534	0.00736509082672553\\
535	0.00739685283269898\\
536	0.00742932163048537\\
537	0.00746255622633911\\
538	0.00749660705933109\\
539	0.00753152688765596\\
540	0.00756737104112477\\
541	0.00760419757173407\\
542	0.00764206275967574\\
543	0.0076810318607664\\
544	0.00772118083695201\\
545	0.00776262635070966\\
546	0.00780552562472109\\
547	0.00784719944470293\\
548	0.00788651753070341\\
549	0.00792583956011556\\
550	0.00796564674856095\\
551	0.00800603755448718\\
552	0.00804701590222305\\
553	0.00808856038361576\\
554	0.00813063840062174\\
555	0.00817320762573787\\
556	0.00821621465504425\\
557	0.00825959134358754\\
558	0.00830325988408064\\
559	0.00834680859180681\\
560	0.00838920854817912\\
561	0.00843169352148518\\
562	0.00847443197159942\\
563	0.00851741118575558\\
564	0.00856060116250899\\
565	0.0086039734318009\\
566	0.00864737435478978\\
567	0.00869053904167139\\
568	0.00873403631210821\\
569	0.00877786229236826\\
570	0.00882199645478541\\
571	0.00886641582311596\\
572	0.00891109504151141\\
573	0.00895600622685637\\
574	0.00900111882089835\\
575	0.00904639944682846\\
576	0.0090918117747785\\
577	0.00913731640198132\\
578	0.00918287075500235\\
579	0.00922842902355825\\
580	0.00927394213803858\\
581	0.00931935780612553\\
582	0.00936462062793333\\
583	0.00940967231402827\\
584	0.00945445203657523\\
585	0.00949889695047729\\
586	0.00954294292772306\\
587	0.00958652555102227\\
588	0.00962958140299686\\
589	0.00967204963943729\\
590	0.0097138736812812\\
591	0.00975500241768019\\
592	0.00979538909575469\\
593	0.00983498278816331\\
594	0.00987369853931868\\
595	0.00991118387968948\\
596	0.00994658651044256\\
597	0.00997788999445116\\
598	0.010000292044645\\
599	0\\
600	0\\
};
\addplot [color=mycolor3,solid,forget plot]
  table[row sep=crcr]{%
1	0.00409709505721719\\
2	0.00409714790196123\\
3	0.00409720179506604\\
4	0.00409725675550612\\
5	0.00409731280253287\\
6	0.00409736995567597\\
7	0.00409742823474466\\
8	0.00409748765982912\\
9	0.0040975482513015\\
10	0.0040976100298169\\
11	0.00409767301631428\\
12	0.00409773723201712\\
13	0.00409780269843404\\
14	0.00409786943735913\\
15	0.00409793747087224\\
16	0.00409800682133909\\
17	0.00409807751141115\\
18	0.0040981495640253\\
19	0.00409822300240346\\
20	0.00409829785005182\\
21	0.00409837413075997\\
22	0.00409845186859983\\
23	0.00409853108792437\\
24	0.0040986118133659\\
25	0.00409869406983447\\
26	0.00409877788251555\\
27	0.00409886327686802\\
28	0.00409895027862135\\
29	0.00409903891377286\\
30	0.0040991292085846\\
31	0.00409922118957973\\
32	0.00409931488353903\\
33	0.00409941031749665\\
34	0.00409950751873585\\
35	0.00409960651478426\\
36	0.00409970733340878\\
37	0.00409981000261029\\
38	0.00409991455061785\\
39	0.00410002100588264\\
40	0.00410012939707149\\
41	0.00410023975305994\\
42	0.00410035210292515\\
43	0.00410046647593821\\
44	0.00410058290155609\\
45	0.00410070140941337\\
46	0.00410082202931329\\
47	0.00410094479121861\\
48	0.00410106972524196\\
49	0.00410119686163573\\
50	0.00410132623078173\\
51	0.00410145786318031\\
52	0.00410159178943899\\
53	0.00410172804026081\\
54	0.00410186664643225\\
55	0.00410200763881073\\
56	0.00410215104831164\\
57	0.00410229690589511\\
58	0.00410244524255241\\
59	0.00410259608929171\\
60	0.0041027494771238\\
61	0.00410290543704742\\
62	0.00410306400003411\\
63	0.00410322519701274\\
64	0.0041033890588541\\
65	0.00410355561635478\\
66	0.00410372490022111\\
67	0.00410389694105278\\
68	0.00410407176932631\\
69	0.0041042494153785\\
70	0.00410442990938947\\
71	0.00410461328136599\\
72	0.00410479956112467\\
73	0.00410498877827502\\
74	0.00410518096220291\\
75	0.00410537614205391\\
76	0.00410557434671693\\
77	0.00410577560480814\\
78	0.00410597994465524\\
79	0.00410618739428196\\
80	0.00410639798139328\\
81	0.00410661173336094\\
82	0.00410682867720988\\
83	0.00410704883960502\\
84	0.00410727224683913\\
85	0.00410749892482147\\
86	0.00410772889906749\\
87	0.00410796219468941\\
88	0.00410819883638821\\
89	0.00410843884844677\\
90	0.0041086822547244\\
91	0.00410892907865275\\
92	0.00410917934323363\\
93	0.00410943307103807\\
94	0.00410969028420758\\
95	0.00410995100445693\\
96	0.00411021525307929\\
97	0.00411048305095331\\
98	0.00411075441855213\\
99	0.00411102937595528\\
100	0.00411130794286244\\
101	0.00411159013860977\\
102	0.00411187598218897\\
103	0.00411216549226877\\
104	0.00411245868721916\\
105	0.00411275558513829\\
106	0.00411305620388219\\
107	0.00411336056109718\\
108	0.00411366867425543\\
109	0.00411398056069289\\
110	0.00411429623765032\\
111	0.00411461572231712\\
112	0.00411493903187795\\
113	0.00411526618356198\\
114	0.00411559719469509\\
115	0.00411593208275438\\
116	0.00411627086542568\\
117	0.00411661356066322\\
118	0.00411696018675189\\
119	0.00411731076237185\\
120	0.00411766530666538\\
121	0.00411802383930604\\
122	0.00411838638056982\\
123	0.00411875295140849\\
124	0.00411912357352497\\
125	0.00411949826945069\\
126	0.00411987706262516\\
127	0.00412025997747728\\
128	0.00412064703950929\\
129	0.00412103827538233\\
130	0.0041214337130048\\
131	0.00412183338162291\\
132	0.0041222373119138\\
133	0.0041226455360812\\
134	0.00412305808795372\\
135	0.00412347500308553\\
136	0.00412389631885935\\
137	0.00412432207459104\\
138	0.00412475231163499\\
139	0.004125187073489\\
140	0.00412562640589676\\
141	0.00412607035694534\\
142	0.00412651897715428\\
143	0.00412697231955158\\
144	0.00412743043973162\\
145	0.00412789339588805\\
146	0.00412836124881516\\
147	0.0041288340618709\\
148	0.00412931190089873\\
149	0.00412979483411735\\
150	0.00413028293201533\\
151	0.00413077626729512\\
152	0.0041312749144199\\
153	0.0041317789496272\\
154	0.00413228845098237\\
155	0.00413280349843347\\
156	0.00413332417386732\\
157	0.0041338505611666\\
158	0.00413438274626842\\
159	0.00413492081722397\\
160	0.00413546486425925\\
161	0.00413601497983702\\
162	0.00413657125871991\\
163	0.00413713379803445\\
164	0.00413770269733617\\
165	0.00413827805867575\\
166	0.00413885998666593\\
167	0.00413944858854946\\
168	0.00414004397426772\\
169	0.00414064625652988\\
170	0.00414125555088306\\
171	0.00414187197578273\\
172	0.00414249565266374\\
173	0.00414312670601149\\
174	0.00414376526343341\\
175	0.00414441145573073\\
176	0.0041450654169698\\
177	0.00414572728455374\\
178	0.00414639719929339\\
179	0.00414707530547805\\
180	0.00414776175094538\\
181	0.00414845668715073\\
182	0.00414916026923534\\
183	0.0041498726560934\\
184	0.00415059401043789\\
185	0.00415132449886459\\
186	0.00415206429191454\\
187	0.00415281356413448\\
188	0.00415357249413494\\
189	0.0041543412646461\\
190	0.0041551200625707\\
191	0.00415590907903425\\
192	0.00415670850943206\\
193	0.00415751855347266\\
194	0.0041583394152178\\
195	0.00415917130311833\\
196	0.00416001443004602\\
197	0.00416086901332105\\
198	0.00416173527473464\\
199	0.00416261344056694\\
200	0.00416350374159974\\
201	0.00416440641312374\\
202	0.00416532169494035\\
203	0.00416624983135754\\
204	0.00416719107117978\\
205	0.00416814566769192\\
206	0.00416911387863618\\
207	0.00417009596618309\\
208	0.00417109219689547\\
209	0.00417210284168541\\
210	0.00417312817576454\\
211	0.00417416847858684\\
212	0.00417522403378432\\
213	0.0041762951290953\\
214	0.00417738205628507\\
215	0.00417848511105917\\
216	0.00417960459296875\\
217	0.0041807408053083\\
218	0.00418189405500533\\
219	0.00418306465250219\\
220	0.00418425291162984\\
221	0.00418545914947342\\
222	0.0041866836862296\\
223	0.00418792684505579\\
224	0.00418918895191122\\
225	0.00419047033538989\\
226	0.00419177132654538\\
227	0.00419309225870856\\
228	0.00419443346729708\\
229	0.00419579528961866\\
230	0.00419717806466829\\
231	0.00419858213292799\\
232	0.00420000783619921\\
233	0.00420145551752562\\
234	0.00420292552087986\\
235	0.00420441819099777\\
236	0.00420593387324132\\
237	0.00420747291350122\\
238	0.00420903565815524\\
239	0.00421062245410225\\
240	0.00421223364889825\\
241	0.00421386959102594\\
242	0.00421553063033765\\
243	0.00421721711871981\\
244	0.00421892941103672\\
245	0.00422066786642197\\
246	0.00422243284999639\\
247	0.00422422473510007\\
248	0.00422604390613041\\
249	0.0042278907620704\\
250	0.00422976572077321\\
251	0.00423166922412071\\
252	0.00423360174433245\\
253	0.00423556379097977\\
254	0.00423755591868842\\
255	0.00423957873513579\\
256	0.00424163290862752\\
257	0.00424371917420263\\
258	0.00424583833787788\\
259	0.00424799127431861\\
260	0.00425017888885837\\
261	0.00425240211861172\\
262	0.00425466193363533\\
263	0.00425695933817204\\
264	0.0042592953720731\\
265	0.00426167111248506\\
266	0.00426408767458076\\
267	0.00426654621277113\\
268	0.00426904792205369\\
269	0.0042715940393867\\
270	0.00427418584508419\\
271	0.00427682466422699\\
272	0.00427951186808263\\
273	0.00428224887552587\\
274	0.00428503715444896\\
275	0.00428787822314751\\
276	0.0042907736516625\\
277	0.00429372506305063\\
278	0.0042967341345539\\
279	0.00429980259872177\\
280	0.00430293224504238\\
281	0.00430612492086036\\
282	0.00430938253134836\\
283	0.00431270703968716\\
284	0.00431610046668569\\
285	0.00431956488984633\\
286	0.00432310244353248\\
287	0.00432671531807215\\
288	0.00433040575779906\\
289	0.00433417605820851\\
290	0.004338028561485\\
291	0.00434196564953624\\
292	0.00434598973981118\\
293	0.00435010327728252\\
294	0.00435430872458364\\
295	0.0043586085500224\\
296	0.00436300521317298\\
297	0.00436750114773264\\
298	0.00437209874132715\\
299	0.00437680031196629\\
300	0.00438160808091853\\
301	0.00438652414202639\\
302	0.00439155042855168\\
303	0.00439668868214047\\
304	0.00440194040551478\\
305	0.00440730681827399\\
306	0.00441278881462466\\
307	0.00441838692466119\\
308	0.00442410128403887\\
309	0.00442993161913903\\
310	0.00443587725775034\\
311	0.00444193717853756\\
312	0.00444811011538356\\
313	0.00445439475816206\\
314	0.00446079010033279\\
315	0.0044672959489268\\
316	0.00447391631848287\\
317	0.00448065625056135\\
318	0.00448752117149148\\
319	0.00449451692340825\\
320	0.00450164981628869\\
321	0.00450892666435019\\
322	0.0045163548226895\\
323	0.00452394223476899\\
324	0.00453169748434551\\
325	0.0045396298526327\\
326	0.0045477493794233\\
327	0.00455606689500467\\
328	0.0045645940897534\\
329	0.00457334359700911\\
330	0.00458232905924628\\
331	0.0045915651923621\\
332	0.00460106784652621\\
333	0.00461085409850468\\
334	0.00462094238577422\\
335	0.00463135242098504\\
336	0.00464210518561074\\
337	0.00465322288161534\\
338	0.00466472882292711\\
339	0.00467664723130864\\
340	0.00468900278185558\\
341	0.00470182005770181\\
342	0.0047151231142723\\
343	0.00472893456204501\\
344	0.00474327429492926\\
345	0.00475815774915058\\
346	0.00477359354148361\\
347	0.00478958028699191\\
348	0.00480610232394138\\
349	0.0048231239137538\\
350	0.00484056158046822\\
351	0.00485806520159102\\
352	0.00487562323831201\\
353	0.00489322270900446\\
354	0.0049108491103253\\
355	0.00492848572021073\\
356	0.00494611359557722\\
357	0.00496371126707528\\
358	0.00498125594075324\\
359	0.0049987233684562\\
360	0.00501608709894146\\
361	0.00503331849331765\\
362	0.00505038698214896\\
363	0.0050672585025896\\
364	0.00508389293770087\\
365	0.00510024698592675\\
366	0.00511627433362389\\
367	0.00513192602333867\\
368	0.00514715109622411\\
369	0.00516189761431189\\
370	0.00517611420720388\\
371	0.00518975233876229\\
372	0.00520276938265742\\
373	0.00521513302496895\\
374	0.00522682733997493\\
375	0.00523786108158864\\
376	0.00524831508627264\\
377	0.00525861736023971\\
378	0.00526875578457428\\
379	0.00527871832643526\\
380	0.00528849317371806\\
381	0.00529806890299139\\
382	0.00530743470690838\\
383	0.00531658071304657\\
384	0.00532549831821052\\
385	0.00533418054159597\\
386	0.00534262242059119\\
387	0.0053508214454658\\
388	0.00535877802430025\\
389	0.00536649596191278\\
390	0.00537398293026189\\
391	0.00538125089273225\\
392	0.00538831642273599\\
393	0.00539520083064398\\
394	0.00540192997276148\\
395	0.00540853355941611\\
396	0.00541504371526052\\
397	0.00542149244662168\\
398	0.00542789359962197\\
399	0.00543425068402017\\
400	0.00544056657450049\\
401	0.00544684484705187\\
402	0.00545308979767841\\
403	0.00545930644642226\\
404	0.00546550052258368\\
405	0.00547167842660904\\
406	0.0054778471637682\\
407	0.00548401424463447\\
408	0.00549018754787465\\
409	0.0054963751420991\\
410	0.00550258506634743\\
411	0.0055088250732971\\
412	0.00551510234635268\\
413	0.00552142321306698\\
414	0.00552779289425235\\
415	0.00553421535301886\\
416	0.00554069376033234\\
417	0.00554723114868104\\
418	0.00555383060849965\\
419	0.00556049526235636\\
420	0.0055672282365609\\
421	0.00557403263085659\\
422	0.00558091148717831\\
423	0.00558786775882935\\
424	0.00559490428182437\\
425	0.00560202375054027\\
426	0.00560922870016655\\
427	0.00561652149865555\\
428	0.00562390435080193\\
429	0.00563137931650108\\
430	0.00563894834380563\\
431	0.00564661331460629\\
432	0.00565437608506419\\
433	0.00566223849751214\\
434	0.00567020237707114\\
435	0.00567826952907412\\
436	0.00568644173746758\\
437	0.00569472076433976\\
438	0.0057031083506825\\
439	0.00571160621843063\\
440	0.00572021607373206\\
441	0.00572893961128431\\
442	0.00573777851943147\\
443	0.00574673448555957\\
444	0.00575580920118335\\
445	0.00576500436602482\\
446	0.00577432169042485\\
447	0.00578376289640306\\
448	0.00579332971842165\\
449	0.00580302390431187\\
450	0.00581284721633995\\
451	0.0058228014323806\\
452	0.00583288834715807\\
453	0.00584310977350766\\
454	0.0058534675436082\\
455	0.005863963510137\\
456	0.00587459954730718\\
457	0.00588537755176319\\
458	0.00589629944333533\\
459	0.00590736716568488\\
460	0.00591858268690471\\
461	0.00592994800012101\\
462	0.00594146512410113\\
463	0.00595313610386228\\
464	0.00596496301127626\\
465	0.00597694794566633\\
466	0.00598909303439411\\
467	0.00600140043343626\\
468	0.00601387232795212\\
469	0.00602651093284681\\
470	0.00603931849333446\\
471	0.00605229728550793\\
472	0.00606544961692059\\
473	0.0060787778271834\\
474	0.0060922842885795\\
475	0.00610597140669768\\
476	0.00611984162108701\\
477	0.00613389740593545\\
478	0.00614814127077531\\
479	0.00616257576121957\\
480	0.00617720345973286\\
481	0.006192026986442\\
482	0.00620704899999094\\
483	0.00622227219844596\\
484	0.00623769932025699\\
485	0.00625333314528213\\
486	0.00626917649588283\\
487	0.00628523223809936\\
488	0.00630150328291617\\
489	0.00631799258762984\\
490	0.00633470315733296\\
491	0.00635163804653024\\
492	0.00636880036090543\\
493	0.00638619325926091\\
494	0.00640381995565534\\
495	0.00642168372176941\\
496	0.00643978788953463\\
497	0.00645813585406661\\
498	0.00647673107695201\\
499	0.00649557708994683\\
500	0.00651467749915488\\
501	0.00653403598976808\\
502	0.00655365633146543\\
503	0.00657354238458662\\
504	0.00659369810721729\\
505	0.00661412756335069\\
506	0.00663483493232087\\
507	0.00665582451974051\\
508	0.00667710077022156\\
509	0.0066986682822085\\
510	0.00672053182533012\\
511	0.00674269636055962\\
512	0.00676516706373502\\
513	0.00678794935267998\\
514	0.00681104892293287\\
515	0.0068344717947585\\
516	0.00685822436532986\\
517	0.00688231353410591\\
518	0.00690674657611931\\
519	0.0069315311009031\\
520	0.00695667520078294\\
521	0.00698218171358456\\
522	0.00700806184724271\\
523	0.00703433235251492\\
524	0.0070610103146217\\
525	0.0070881189831278\\
526	0.00711567716246021\\
527	0.00714369741365174\\
528	0.00717219191737849\\
529	0.00720117163858217\\
530	0.00723064483866799\\
531	0.00726061777558896\\
532	0.00729109401196795\\
533	0.00732206999733117\\
534	0.00735350410895862\\
535	0.00738534794863182\\
536	0.00741766494491498\\
537	0.00745056597663264\\
538	0.00748420149227407\\
539	0.0075186282540824\\
540	0.00755391134287273\\
541	0.00759010429212257\\
542	0.00762725429437914\\
543	0.00766541139405159\\
544	0.00770463275832746\\
545	0.00774498450574268\\
546	0.00778654350476932\\
547	0.00782943284053057\\
548	0.007873812470974\\
549	0.00791685142016626\\
550	0.00795760885345326\\
551	0.00799826335611184\\
552	0.008039343319122\\
553	0.00808097441322759\\
554	0.00812317053076772\\
555	0.00816590702824844\\
556	0.00820914385485902\\
557	0.00825282989559372\\
558	0.00829690046744554\\
559	0.00834127797993642\\
560	0.00838587427252687\\
561	0.00842944747538469\\
562	0.00847248750936672\\
563	0.00851573631428289\\
564	0.00855920311781243\\
565	0.00860285881927415\\
566	0.00864667312030307\\
567	0.00869051061764692\\
568	0.00873403472335589\\
569	0.00877786227543177\\
570	0.00882199645300022\\
571	0.00886641582265344\\
572	0.00891109504133455\\
573	0.00895600622677124\\
574	0.00900111882085289\\
575	0.00904639944680352\\
576	0.00909181177476488\\
577	0.00913731640197366\\
578	0.00918287075499839\\
579	0.00922842902355639\\
580	0.00927394213803805\\
581	0.00931935780612538\\
582	0.00936462062793332\\
583	0.00940967231402828\\
584	0.00945445203657523\\
585	0.0094988969504773\\
586	0.00954294292772307\\
587	0.00958652555102228\\
588	0.00962958140299686\\
589	0.0096720496394373\\
590	0.0097138736812812\\
591	0.0097550024176802\\
592	0.00979538909575469\\
593	0.00983498278816331\\
594	0.00987369853931868\\
595	0.00991118387968948\\
596	0.00994658651044256\\
597	0.00997788999445116\\
598	0.010000292044645\\
599	0\\
600	0\\
};
\addplot [color=mycolor4,solid,forget plot]
  table[row sep=crcr]{%
1	0.00409664868257122\\
2	0.00409668962381598\\
3	0.00409673130488141\\
4	0.00409677373721799\\
5	0.00409681693239205\\
6	0.00409686090208509\\
7	0.00409690565809265\\
8	0.00409695121232338\\
9	0.00409699757679793\\
10	0.00409704476364767\\
11	0.00409709278511336\\
12	0.00409714165354387\\
13	0.00409719138139446\\
14	0.00409724198122542\\
15	0.00409729346570032\\
16	0.00409734584758411\\
17	0.00409739913974129\\
18	0.00409745335513396\\
19	0.0040975085068197\\
20	0.00409756460794935\\
21	0.0040976216717648\\
22	0.00409767971159646\\
23	0.00409773874086082\\
24	0.00409779877305787\\
25	0.00409785982176835\\
26	0.00409792190065095\\
27	0.00409798502343938\\
28	0.0040980492039394\\
29	0.0040981144560256\\
30	0.00409818079363827\\
31	0.00409824823078012\\
32	0.00409831678151281\\
33	0.00409838645995345\\
34	0.00409845728027101\\
35	0.00409852925668279\\
36	0.00409860240345046\\
37	0.00409867673487639\\
38	0.00409875226529966\\
39	0.0040988290090922\\
40	0.00409890698065467\\
41	0.00409898619441248\\
42	0.00409906666481163\\
43	0.00409914840631443\\
44	0.00409923143339559\\
45	0.00409931576053771\\
46	0.00409940140222727\\
47	0.00409948837295028\\
48	0.00409957668718817\\
49	0.00409966635941348\\
50	0.00409975740408583\\
51	0.00409984983564766\\
52	0.00409994366852032\\
53	0.00410003891709999\\
54	0.0041001355957539\\
55	0.0041002337188164\\
56	0.00410033330058548\\
57	0.00410043435531914\\
58	0.0041005368972321\\
59	0.00410064094049262\\
60	0.00410074649921956\\
61	0.00410085358747956\\
62	0.00410096221928472\\
63	0.00410107240859023\\
64	0.00410118416929251\\
65	0.00410129751522758\\
66	0.00410141246016985\\
67	0.00410152901783121\\
68	0.00410164720186052\\
69	0.00410176702584359\\
70	0.00410188850330351\\
71	0.00410201164770159\\
72	0.0041021364724387\\
73	0.00410226299085719\\
74	0.00410239121624346\\
75	0.00410252116183092\\
76	0.00410265284080381\\
77	0.00410278626630155\\
78	0.00410292145142377\\
79	0.00410305840923608\\
80	0.00410319715277667\\
81	0.00410333769506352\\
82	0.00410348004910242\\
83	0.00410362422789598\\
84	0.00410377024445325\\
85	0.00410391811180042\\
86	0.00410406784299215\\
87	0.00410421945112408\\
88	0.00410437294934605\\
89	0.00410452835087618\\
90	0.0041046856690162\\
91	0.00410484491716742\\
92	0.00410500610884766\\
93	0.00410516925770937\\
94	0.00410533437755831\\
95	0.00410550148237347\\
96	0.00410567058632778\\
97	0.00410584170380952\\
98	0.00410601484944506\\
99	0.0041061900381219\\
100	0.00410636728501282\\
101	0.00410654660560078\\
102	0.00410672801570444\\
103	0.00410691153150435\\
104	0.00410709716956986\\
105	0.00410728494688636\\
106	0.00410747488088329\\
107	0.00410766698946234\\
108	0.00410786129102603\\
109	0.00410805780450654\\
110	0.00410825654939485\\
111	0.00410845754576972\\
112	0.00410866081432693\\
113	0.00410886637640819\\
114	0.00410907425402994\\
115	0.00410928446991207\\
116	0.0041094970475059\\
117	0.00410971201102203\\
118	0.00410992938545747\\
119	0.0041101491966223\\
120	0.00411037147116538\\
121	0.00411059623659954\\
122	0.00411082352132581\\
123	0.00411105335465676\\
124	0.00411128576683904\\
125	0.00411152078907476\\
126	0.00411175845354187\\
127	0.00411199879341387\\
128	0.00411224184287793\\
129	0.00411248763715254\\
130	0.00411273621250376\\
131	0.00411298760626071\\
132	0.00411324185682997\\
133	0.00411349900370904\\
134	0.00411375908749892\\
135	0.00411402214991591\\
136	0.00411428823380235\\
137	0.00411455738313678\\
138	0.00411482964304308\\
139	0.00411510505979911\\
140	0.00411538368084452\\
141	0.00411566555478786\\
142	0.00411595073141336\\
143	0.00411623926168708\\
144	0.00411653119776312\\
145	0.00411682659298973\\
146	0.00411712550191647\\
147	0.00411742798030309\\
148	0.00411773408513234\\
149	0.00411804387462939\\
150	0.0041183574082853\\
151	0.00411867474685224\\
152	0.00411899595236577\\
153	0.00411932108816943\\
154	0.00411965021893952\\
155	0.00411998341070969\\
156	0.00412032073089582\\
157	0.00412066224832075\\
158	0.004121008033239\\
159	0.00412135815736145\\
160	0.0041217126938801\\
161	0.00412207171749242\\
162	0.00412243530442588\\
163	0.00412280353246209\\
164	0.004123176480961\\
165	0.00412355423088474\\
166	0.0041239368648212\\
167	0.00412432446700746\\
168	0.00412471712335291\\
169	0.00412511492146215\\
170	0.00412551795065742\\
171	0.00412592630200085\\
172	0.00412634006831619\\
173	0.0041267593442104\\
174	0.00412718422609469\\
175	0.00412761481220526\\
176	0.00412805120262364\\
177	0.00412849349929665\\
178	0.00412894180605595\\
179	0.00412939622863741\\
180	0.00412985687469984\\
181	0.00413032385384373\\
182	0.00413079727762944\\
183	0.00413127725959528\\
184	0.00413176391527536\\
185	0.00413225736221742\\
186	0.00413275772000041\\
187	0.00413326511025216\\
188	0.00413377965666715\\
189	0.00413430148502457\\
190	0.00413483072320659\\
191	0.00413536750121703\\
192	0.00413591195120078\\
193	0.00413646420746377\\
194	0.00413702440649402\\
195	0.00413759268698367\\
196	0.00413816918985233\\
197	0.00413875405827195\\
198	0.00413934743769338\\
199	0.00413994947587498\\
200	0.00414056032291342\\
201	0.00414118013127704\\
202	0.00414180905584204\\
203	0.00414244725393175\\
204	0.00414309488535929\\
205	0.00414375211247407\\
206	0.00414441910021256\\
207	0.00414509601615324\\
208	0.00414578303057657\\
209	0.0041464803165302\\
210	0.0041471880498995\\
211	0.0041479064094845\\
212	0.00414863557708264\\
213	0.00414937573757863\\
214	0.00415012707904113\\
215	0.00415088979282709\\
216	0.00415166407369351\\
217	0.00415245011991762\\
218	0.00415324813342542\\
219	0.00415405831992878\\
220	0.00415488088907169\\
221	0.00415571605458558\\
222	0.00415656403445428\\
223	0.00415742505108853\\
224	0.00415829933151019\\
225	0.00415918710754658\\
226	0.0041600886160349\\
227	0.00416100409903648\\
228	0.0041619338040616\\
229	0.00416287798430444\\
230	0.00416383689888962\\
231	0.00416481081313237\\
232	0.00416579999881265\\
233	0.00416680473444185\\
234	0.00416782530555101\\
235	0.00416886200499009\\
236	0.00416991513323826\\
237	0.00417098499872553\\
238	0.0041720719181655\\
239	0.00417317621689987\\
240	0.00417429822925369\\
241	0.00417543829890231\\
242	0.00417659677924825\\
243	0.00417777403380749\\
244	0.00417897043660341\\
245	0.00418018637256503\\
246	0.00418142223792556\\
247	0.00418267844061465\\
248	0.00418395540063496\\
249	0.0041852535504115\\
250	0.00418657333510657\\
251	0.00418791521289753\\
252	0.0041892796551678\\
253	0.00419066714660251\\
254	0.00419207818516755\\
255	0.00419351328196016\\
256	0.00419497296095737\\
257	0.00419645775879288\\
258	0.0041979682245596\\
259	0.00419950491873774\\
260	0.00420106841334051\\
261	0.004202659292063\\
262	0.00420427815043769\\
263	0.00420592559600409\\
264	0.00420760224848415\\
265	0.00420930873987836\\
266	0.00421104571460549\\
267	0.00421281382965351\\
268	0.0042146137547326\\
269	0.00421644617243015\\
270	0.00421831177836837\\
271	0.00422021128136477\\
272	0.0042221454035958\\
273	0.00422411488076409\\
274	0.00422612046226925\\
275	0.00422816291138236\\
276	0.00423024300542409\\
277	0.00423236153594855\\
278	0.00423451930894415\\
279	0.00423671714508118\\
280	0.00423895587990949\\
281	0.00424123636400916\\
282	0.00424355946318336\\
283	0.00424592605864222\\
284	0.00424833704720788\\
285	0.0042507933416628\\
286	0.00425329587106279\\
287	0.00425584558103343\\
288	0.00425844343407693\\
289	0.00426109040985437\\
290	0.00426378750546487\\
291	0.00426653573608827\\
292	0.00426933613552122\\
293	0.00427218975681514\\
294	0.00427509767304613\\
295	0.00427806097825504\\
296	0.00428108078860607\\
297	0.00428415824382355\\
298	0.00428729450898293\\
299	0.00429049077675482\\
300	0.00429374827024759\\
301	0.00429706824668784\\
302	0.00430045200220975\\
303	0.00430390087654047\\
304	0.00430741625931877\\
305	0.00431099959792094\\
306	0.00431465240683752\\
307	0.00431837627874854\\
308	0.0043221728973601\\
309	0.00432604405188818\\
310	0.00432999165274966\\
311	0.00433401774763531\\
312	0.00433812453795846\\
313	0.00434231439422532\\
314	0.00434658986551328\\
315	0.00435095368955819\\
316	0.00435540871945394\\
317	0.00435995791231748\\
318	0.00436460433143802\\
319	0.00436935114925647\\
320	0.00437420164829873\\
321	0.00437915922141272\\
322	0.00438422737192456\\
323	0.00438940971298996\\
324	0.00439470996591586\\
325	0.00440013195682813\\
326	0.00440567960975303\\
327	0.00441135694093633\\
328	0.00441716805169503\\
329	0.00442311711736991\\
330	0.00442920837349606\\
331	0.00443544609938092\\
332	0.00444183460149077\\
333	0.00444837819425593\\
334	0.00445508116082882\\
335	0.00446194771879704\\
336	0.00446898198085834\\
337	0.00447618791006925\\
338	0.00448356926800846\\
339	0.0044911295503311\\
340	0.00449887193144283\\
341	0.00450679923471691\\
342	0.00451491389170535\\
343	0.00452321791763498\\
344	0.00453171291502352\\
345	0.00454040012236478\\
346	0.00454928053170496\\
347	0.0045583551079548\\
348	0.00456762515584229\\
349	0.0045770929316558\\
350	0.00458676283704058\\
351	0.00459664640722626\\
352	0.00460675613519573\\
353	0.00461710555285139\\
354	0.00462770926379756\\
355	0.00463858304664908\\
356	0.00464974395497892\\
357	0.00466121052009476\\
358	0.0046730027990592\\
359	0.00468514236738451\\
360	0.00469765236112168\\
361	0.00471055747325224\\
362	0.00472388374758736\\
363	0.00473765833306152\\
364	0.00475190959924341\\
365	0.00476666685219704\\
366	0.00478195989826624\\
367	0.00479781840288387\\
368	0.00481427097438884\\
369	0.00483134387867026\\
370	0.00484905924751179\\
371	0.00486743258499649\\
372	0.00488646990496595\\
373	0.00490616315581023\\
374	0.00492648442273783\\
375	0.00494737823307882\\
376	0.00496871592348503\\
377	0.00499002714010541\\
378	0.00501128758581625\\
379	0.00503247096782956\\
380	0.0050535488223626\\
381	0.00507449094119751\\
382	0.00509526083085377\\
383	0.0051158164758645\\
384	0.00513611189221272\\
385	0.00515609703013776\\
386	0.0051757177862393\\
387	0.00519491617700118\\
388	0.00521363074694414\\
389	0.00523179732370554\\
390	0.00524935012226\\
391	0.00526622344716711\\
392	0.00528235423550805\\
393	0.00529768567804759\\
394	0.00531217237850484\\
395	0.00532578763345195\\
396	0.0053385332321474\\
397	0.00535045281232267\\
398	0.00536206547182922\\
399	0.00537344406308905\\
400	0.00538457529139072\\
401	0.00539544694147929\\
402	0.00540604824494743\\
403	0.0054163703009139\\
404	0.00542640655005739\\
405	0.0054361532980586\\
406	0.00544561028227294\\
407	0.00545478126853225\\
408	0.0054636746509502\\
409	0.00547230401387973\\
410	0.0054806885824226\\
411	0.00548885346400712\\
412	0.00549682954230167\\
413	0.00550465281782058\\
414	0.00551236291050277\\
415	0.00552000033082267\\
416	0.00552758999632444\\
417	0.00553514042149178\\
418	0.00554265655480703\\
419	0.00555014421767857\\
420	0.00555761009506642\\
421	0.00556506169942287\\
422	0.00557250730179434\\
423	0.00557995582371182\\
424	0.00558741668426262\\
425	0.00559489959813635\\
426	0.00560241432305374\\
427	0.00560997035970928\\
428	0.00561757661503771\\
429	0.00562524105176726\\
430	0.00563297036571746\\
431	0.00564076975981897\\
432	0.00564864322931405\\
433	0.00565659441994265\\
434	0.0056646270121865\\
435	0.00567274468719499\\
436	0.00568095109050692\\
437	0.00568924979480068\\
438	0.00569764426333909\\
439	0.00570613781625333\\
440	0.00571473360229106\\
441	0.00572343457906517\\
442	0.00573224350506446\\
443	0.00574116294654159\\
444	0.00575019530158278\\
445	0.00575934284175239\\
446	0.0057686077680368\\
447	0.00577799225250322\\
448	0.00578749844787233\\
449	0.00579712848460786\\
450	0.00580688446918496\\
451	0.00581676848371286\\
452	0.00582678258703619\\
453	0.00583692881735942\\
454	0.00584720919632644\\
455	0.00585762573434246\\
456	0.00586818043675091\\
457	0.0058788753102902\\
458	0.00588971236908199\\
459	0.00590069363930562\\
460	0.00591182116179189\\
461	0.00592309699327209\\
462	0.00593452320750874\\
463	0.00594610189660377\\
464	0.00595783517244888\\
465	0.00596972516827276\\
466	0.00598177404023063\\
467	0.00599398396897853\\
468	0.00600635716117462\\
469	0.00601889585085943\\
470	0.00603160230068508\\
471	0.00604447880299224\\
472	0.00605752768077079\\
473	0.00607075128858001\\
474	0.00608415201348621\\
475	0.00609773227602922\\
476	0.0061114945312139\\
477	0.00612544126952421\\
478	0.00613957501795931\\
479	0.00615389834109338\\
480	0.00616841384216404\\
481	0.00618312416419666\\
482	0.00619803199117528\\
483	0.00621314004927233\\
484	0.00622845110815102\\
485	0.00624396798235272\\
486	0.00625969353278066\\
487	0.00627563066829013\\
488	0.00629178234739776\\
489	0.00630815158012391\\
490	0.00632474142998397\\
491	0.00634155501614687\\
492	0.00635859551578096\\
493	0.00637586616661024\\
494	0.0063933702697064\\
495	0.00641111119254501\\
496	0.00642909237235806\\
497	0.0064473173198182\\
498	0.00646578962309558\\
499	0.00648451295233245\\
500	0.00650349106458752\\
501	0.00652272780930867\\
502	0.00654222713440037\\
503	0.00656199309296133\\
504	0.00658202985077848\\
505	0.00660234169467454\\
506	0.00662293304182132\\
507	0.00664380845014584\\
508	0.00666497262997551\\
509	0.00668643045708999\\
510	0.00670818698737221\\
511	0.00673024747328626\\
512	0.00675261738244383\\
513	0.00677530241856389\\
514	0.00679830854505638\\
515	0.00682164201148414\\
516	0.00684530938344261\\
517	0.0068693175757188\\
518	0.00689367389591976\\
519	0.00691838609654099\\
520	0.00694346243095817\\
521	0.00696891178565863\\
522	0.00699474352995971\\
523	0.00702096749344242\\
524	0.00704759410736137\\
525	0.00707462948156095\\
526	0.00710208597353999\\
527	0.00712998298566222\\
528	0.00715834050109882\\
529	0.00718718223334858\\
530	0.00721653322220688\\
531	0.00724640874563171\\
532	0.00727682321657789\\
533	0.00730779033496002\\
534	0.00733932054955617\\
535	0.00737142094251038\\
536	0.00740409155219721\\
537	0.00743729626273568\\
538	0.00747098540990441\\
539	0.00750522364908087\\
540	0.00754010540771196\\
541	0.00757580349715638\\
542	0.00761236890146435\\
543	0.00764985959963282\\
544	0.00768832507234716\\
545	0.00772781694935591\\
546	0.00776839367582466\\
547	0.00781012138020359\\
548	0.00785307687559534\\
549	0.00789738450562302\\
550	0.0079432034083031\\
551	0.00798788543240736\\
552	0.00803044603523655\\
553	0.00807251022837668\\
554	0.00811489868865496\\
555	0.00815779179765849\\
556	0.0082012227307077\\
557	0.00824515999795583\\
558	0.00828955575295806\\
559	0.00833434738316874\\
560	0.00837945454762876\\
561	0.00842479121386117\\
562	0.0084696778803419\\
563	0.00851341295005585\\
564	0.00855721779447572\\
565	0.00860119556072858\\
566	0.00864533797554444\\
567	0.00868961368428026\\
568	0.00873393789191277\\
569	0.00877785590949144\\
570	0.00882199632513923\\
571	0.0088664158097615\\
572	0.00891109503814012\\
573	0.00895600622559429\\
574	0.00900111882029819\\
575	0.00904639944650971\\
576	0.00909181177460439\\
577	0.00913731640188469\\
578	0.00918287075494969\\
579	0.00922842902353094\\
580	0.00927394213802573\\
581	0.00931935780612039\\
582	0.00936462062793178\\
583	0.00940967231402812\\
584	0.00945445203657523\\
585	0.00949889695047729\\
586	0.00954294292772307\\
587	0.00958652555102227\\
588	0.00962958140299686\\
589	0.00967204963943729\\
590	0.0097138736812812\\
591	0.0097550024176802\\
592	0.00979538909575469\\
593	0.00983498278816331\\
594	0.00987369853931868\\
595	0.00991118387968948\\
596	0.00994658651044256\\
597	0.00997788999445116\\
598	0.010000292044645\\
599	0\\
600	0\\
};
\addplot [color=mycolor5,solid,forget plot]
  table[row sep=crcr]{%
1	0.00409619160294684\\
2	0.00409622158226429\\
3	0.00409625205221089\\
4	0.00409628301926326\\
5	0.00409631448994264\\
6	0.00409634647081406\\
7	0.00409637896848576\\
8	0.00409641198960823\\
9	0.0040964455408736\\
10	0.0040964796290148\\
11	0.00409651426080469\\
12	0.00409654944305536\\
13	0.00409658518261726\\
14	0.0040966214863783\\
15	0.00409665836126297\\
16	0.00409669581423155\\
17	0.00409673385227931\\
18	0.00409677248243541\\
19	0.00409681171176222\\
20	0.00409685154735435\\
21	0.00409689199633775\\
22	0.00409693306586891\\
23	0.00409697476313393\\
24	0.00409701709534774\\
25	0.00409706006975311\\
26	0.00409710369362\\
27	0.00409714797424461\\
28	0.00409719291894871\\
29	0.00409723853507883\\
30	0.00409728483000557\\
31	0.00409733181112282\\
32	0.00409737948584722\\
33	0.00409742786161753\\
34	0.00409747694589413\\
35	0.00409752674615837\\
36	0.00409757726991232\\
37	0.00409762852467825\\
38	0.00409768051799848\\
39	0.00409773325743501\\
40	0.00409778675056946\\
41	0.00409784100500304\\
42	0.00409789602835649\\
43	0.0040979518282704\\
44	0.00409800841240532\\
45	0.00409806578844213\\
46	0.00409812396408264\\
47	0.00409818294705012\\
48	0.00409824274509012\\
49	0.00409830336597133\\
50	0.00409836481748665\\
51	0.00409842710745441\\
52	0.00409849024371973\\
53	0.00409855423415614\\
54	0.00409861908666724\\
55	0.00409868480918863\\
56	0.00409875140969016\\
57	0.00409881889617818\\
58	0.00409888727669801\\
59	0.00409895655933686\\
60	0.0040990267522268\\
61	0.00409909786354797\\
62	0.00409916990153199\\
63	0.00409924287446588\\
64	0.00409931679069591\\
65	0.00409939165863193\\
66	0.00409946748675191\\
67	0.00409954428360668\\
68	0.00409962205782512\\
69	0.00409970081811947\\
70	0.00409978057329107\\
71	0.00409986133223633\\
72	0.00409994310395293\\
73	0.00410002589754655\\
74	0.00410010972223754\\
75	0.00410019458736832\\
76	0.00410028050241081\\
77	0.00410036747697415\\
78	0.00410045552081292\\
79	0.00410054464383554\\
80	0.00410063485611294\\
81	0.00410072616788758\\
82	0.00410081858958281\\
83	0.00410091213181236\\
84	0.00410100680539028\\
85	0.00410110262134104\\
86	0.00410119959090989\\
87	0.00410129772557349\\
88	0.0041013970370508\\
89	0.00410149753731418\\
90	0.00410159923860063\\
91	0.00410170215342322\\
92	0.00410180629458288\\
93	0.00410191167518001\\
94	0.0041020183086265\\
95	0.00410212620865782\\
96	0.00410223538934498\\
97	0.00410234586510692\\
98	0.00410245765072262\\
99	0.00410257076134334\\
100	0.00410268521250494\\
101	0.00410280102014015\\
102	0.00410291820059057\\
103	0.00410303677061892\\
104	0.004103156747421\\
105	0.00410327814863759\\
106	0.00410340099236612\\
107	0.00410352529717227\\
108	0.00410365108210135\\
109	0.00410377836668947\\
110	0.00410390717097453\\
111	0.00410403751550685\\
112	0.00410416942135973\\
113	0.00410430291013961\\
114	0.00410443800399612\\
115	0.00410457472563163\\
116	0.00410471309831071\\
117	0.00410485314586941\\
118	0.00410499489272402\\
119	0.00410513836387979\\
120	0.00410528358493932\\
121	0.00410543058211083\\
122	0.00410557938221616\\
123	0.00410573001269859\\
124	0.00410588250163056\\
125	0.0041060368777213\\
126	0.00410619317032435\\
127	0.00410635140944498\\
128	0.00410651162574777\\
129	0.00410667385056405\\
130	0.00410683811589949\\
131	0.00410700445444173\\
132	0.00410717289956831\\
133	0.00410734348535462\\
134	0.00410751624658221\\
135	0.00410769121874715\\
136	0.00410786843806889\\
137	0.00410804794149926\\
138	0.00410822976673195\\
139	0.0041084139522122\\
140	0.00410860053714694\\
141	0.00410878956151536\\
142	0.00410898106607968\\
143	0.00410917509239668\\
144	0.00410937168282914\\
145	0.00410957088055834\\
146	0.00410977272959654\\
147	0.00410997727480027\\
148	0.00411018456188428\\
149	0.00411039463743527\\
150	0.0041106075489235\\
151	0.0041108233447165\\
152	0.00411104207409314\\
153	0.00411126378725806\\
154	0.00411148853535573\\
155	0.00411171637048515\\
156	0.00411194734571431\\
157	0.00411218151509491\\
158	0.00411241893367731\\
159	0.00411265965752546\\
160	0.00411290374373211\\
161	0.00411315125043418\\
162	0.00411340223682819\\
163	0.00411365676318623\\
164	0.00411391489087168\\
165	0.00411417668235555\\
166	0.00411444220123294\\
167	0.00411471151223977\\
168	0.00411498468126977\\
169	0.00411526177539199\\
170	0.0041155428628684\\
171	0.00411582801317217\\
172	0.00411611729700612\\
173	0.00411641078632185\\
174	0.0041167085543392\\
175	0.00411701067556636\\
176	0.00411731722582052\\
177	0.00411762828224914\\
178	0.00411794392335213\\
179	0.00411826422900421\\
180	0.00411858928047872\\
181	0.00411891916047184\\
182	0.00411925395312786\\
183	0.00411959374406541\\
184	0.00411993862040482\\
185	0.00412028867079633\\
186	0.00412064398544972\\
187	0.00412100465616507\\
188	0.00412137077636482\\
189	0.00412174244112731\\
190	0.00412211974722161\\
191	0.00412250279314403\\
192	0.00412289167915623\\
193	0.00412328650732495\\
194	0.00412368738156349\\
195	0.00412409440767505\\
196	0.00412450769339802\\
197	0.00412492734845314\\
198	0.00412535348459278\\
199	0.00412578621565219\\
200	0.00412622565760295\\
201	0.00412667192860869\\
202	0.00412712514908281\\
203	0.00412758544174877\\
204	0.00412805293170244\\
205	0.00412852774647686\\
206	0.00412901001610912\\
207	0.00412949987320979\\
208	0.00412999745303449\\
209	0.00413050289355773\\
210	0.00413101633554895\\
211	0.00413153792265059\\
212	0.00413206780145841\\
213	0.00413260612160365\\
214	0.00413315303583702\\
215	0.00413370870011443\\
216	0.00413427327368446\\
217	0.00413484691917709\\
218	0.00413542980269359\\
219	0.00413602209389782\\
220	0.004136623966108\\
221	0.00413723559638941\\
222	0.0041378571656474\\
223	0.00413848885872059\\
224	0.00413913086447404\\
225	0.00413978337589208\\
226	0.00414044659017064\\
227	0.00414112070880878\\
228	0.0041418059376992\\
229	0.00414250248721764\\
230	0.00414321057231111\\
231	0.00414393041258449\\
232	0.00414466223238378\\
233	0.00414540626087849\\
234	0.00414616273214168\\
235	0.00414693188522777\\
236	0.00414771396424824\\
237	0.00414850921844467\\
238	0.00414931790225951\\
239	0.00415014027540443\\
240	0.00415097660292645\\
241	0.00415182715527154\\
242	0.00415269220834638\\
243	0.0041535720435779\\
244	0.00415446694797091\\
245	0.00415537721416414\\
246	0.00415630314048469\\
247	0.00415724503100112\\
248	0.00415820319557572\\
249	0.00415917794991717\\
250	0.00416016961563463\\
251	0.00416117852029278\\
252	0.00416220499746934\\
253	0.0041632493868176\\
254	0.0041643120341376\\
255	0.00416539329146222\\
256	0.00416649351716763\\
257	0.00416761307609297\\
258	0.00416875233960254\\
259	0.00416991168572639\\
260	0.00417109149930953\\
261	0.00417229217217076\\
262	0.00417351410327181\\
263	0.00417475769889585\\
264	0.00417602337283\\
265	0.00417731154656293\\
266	0.00417862264949541\\
267	0.00417995711916321\\
268	0.00418131540147287\\
269	0.00418269795095136\\
270	0.00418410523100984\\
271	0.00418553771422237\\
272	0.00418699588261993\\
273	0.00418848022800048\\
274	0.00418999125225548\\
275	0.00419152946771326\\
276	0.00419309539750048\\
277	0.00419468957592257\\
278	0.0041963125488651\\
279	0.00419796487420833\\
280	0.00419964712225698\\
281	0.00420135987619224\\
282	0.00420310373254267\\
283	0.00420487930167866\\
284	0.00420668720833811\\
285	0.00420852809216799\\
286	0.00421040260828383\\
287	0.00421231142784932\\
288	0.00421425523867425\\
289	0.00421623474583657\\
290	0.00421825067235086\\
291	0.00422030375984715\\
292	0.00422239476927642\\
293	0.00422452448164367\\
294	0.00422669369876994\\
295	0.00422890324408457\\
296	0.0042311539634484\\
297	0.00423344672600954\\
298	0.00423578242509367\\
299	0.00423816197913298\\
300	0.00424058633263962\\
301	0.00424305645721312\\
302	0.00424557335247821\\
303	0.00424813804707692\\
304	0.00425075159967099\\
305	0.00425341509992398\\
306	0.00425612966943876\\
307	0.00425889646261729\\
308	0.00426171666739985\\
309	0.0042645915058305\\
310	0.00426752223441344\\
311	0.00427051014433791\\
312	0.00427355656159489\\
313	0.00427666284695063\\
314	0.00427983039637195\\
315	0.00428306063945929\\
316	0.00428635503992115\\
317	0.00428971509653306\\
318	0.0042931423442123\\
319	0.00429663835501139\\
320	0.00430020473916595\\
321	0.00430384314625455\\
322	0.00430755526643084\\
323	0.0043113428317262\\
324	0.0043152076173955\\
325	0.0043191514432586\\
326	0.00432317617543228\\
327	0.00432728372825828\\
328	0.00433147606632453\\
329	0.00433575520676757\\
330	0.00434012322199784\\
331	0.00434458224309119\\
332	0.00434913446355973\\
333	0.0043537821425117\\
334	0.00435852760938808\\
335	0.00436337326953708\\
336	0.00436832161071977\\
337	0.00437337521052818\\
338	0.00437853674471211\\
339	0.00438380899845113\\
340	0.00438919488129389\\
341	0.00439469744265264\\
342	0.00440031988971178\\
343	0.00440606560777859\\
344	0.00441193818286184\\
345	0.00441794142585865\\
346	0.00442407939712489\\
347	0.00443035642955957\\
348	0.0044367771487174\\
349	0.00444334648174883\\
350	0.00445006965806816\\
351	0.00445695208320719\\
352	0.00446399933593276\\
353	0.00447121716027577\\
354	0.00447861146097566\\
355	0.00448618829689002\\
356	0.00449395387742492\\
357	0.00450191454159263\\
358	0.00451007673028543\\
359	0.00451844695920616\\
360	0.00452703178309679\\
361	0.00453583774289785\\
362	0.00454487131592899\\
363	0.00455413889370712\\
364	0.00456364672977185\\
365	0.00457340088478789\\
366	0.00458340717185425\\
367	0.00459367110664716\\
368	0.00460419786932714\\
369	0.00461499228825761\\
370	0.00462605886356467\\
371	0.00463740188191018\\
372	0.00464902557505931\\
373	0.00466093443897203\\
374	0.00467313375313041\\
375	0.00468563037627639\\
376	0.00469843443609457\\
377	0.00471156587778469\\
378	0.00472504627848721\\
379	0.00473889890666229\\
380	0.00475314875338926\\
381	0.00476782220581069\\
382	0.0047829472136154\\
383	0.00479855342091387\\
384	0.00481467205339902\\
385	0.00483133570831505\\
386	0.00484857801096762\\
387	0.00486643308849613\\
388	0.004884934788385\\
389	0.0049041154916689\\
390	0.0049240048555595\\
391	0.00494462772721307\\
392	0.00496600122010846\\
393	0.00498813086022231\\
394	0.00501100534714339\\
395	0.00503459170974232\\
396	0.00505882507912041\\
397	0.00508359712951041\\
398	0.00510833073767857\\
399	0.00513289684048401\\
400	0.005157250316384\\
401	0.00518134153147043\\
402	0.005205116122047\\
403	0.00522851485600113\\
404	0.00525147361912384\\
405	0.00527392360208589\\
406	0.00529579166536875\\
407	0.00531700101044047\\
408	0.00533747237227751\\
409	0.00535712580524692\\
410	0.00537588365823375\\
411	0.00539367467105489\\
412	0.00541043974944722\\
413	0.00542614015003393\\
414	0.00544076868404582\\
415	0.00545436489319491\\
416	0.00546738929631827\\
417	0.00548011849802279\\
418	0.00549253944262353\\
419	0.00550464114701441\\
420	0.00551641524832511\\
421	0.00552785661896403\\
422	0.00553896404323548\\
423	0.00554974094209379\\
424	0.00556019610678231\\
425	0.00557034439458198\\
426	0.00558020731548305\\
427	0.00558981340369295\\
428	0.00559919822137929\\
429	0.00560840377312078\\
430	0.00561747701797148\\
431	0.00562646704137285\\
432	0.00563541158760834\\
433	0.00564432490145999\\
434	0.00565321393937883\\
435	0.00566208666998028\\
436	0.00567095202740168\\
437	0.00567981982455819\\
438	0.00568870061929596\\
439	0.00569760552686288\\
440	0.00570654597380089\\
441	0.00571553339180765\\
442	0.00572457885615251\\
443	0.00573369268309565\\
444	0.00574288401618527\\
445	0.00575216045473461\\
446	0.00576152781243046\\
447	0.00577099067349807\\
448	0.00578055331278669\\
449	0.00579022000283565\\
450	0.00579999497122763\\
451	0.00580988235693061\\
452	0.00581988616766082\\
453	0.0058300102408789\\
454	0.00584025821160249\\
455	0.00585063349069623\\
456	0.00586113925753365\\
457	0.00587177847068472\\
458	0.00588255389919582\\
459	0.00589346817452325\\
460	0.00590452385842724\\
461	0.00591572348402295\\
462	0.00592706956128874\\
463	0.00593856457489556\\
464	0.00595021098360879\\
465	0.00596201122142354\\
466	0.00597396770050443\\
467	0.00598608281587155\\
468	0.00599835895160588\\
469	0.00601079848814156\\
470	0.00602340380998847\\
471	0.00603617731301882\\
472	0.00604912141032395\\
473	0.00606223853571863\\
474	0.00607553114562047\\
475	0.00608900172073521\\
476	0.00610265276796723\\
477	0.00611648682251159\\
478	0.00613050645007231\\
479	0.00614471424914408\\
480	0.00615911285329214\\
481	0.00617370493337021\\
482	0.00618849319963286\\
483	0.00620348040372628\\
484	0.00621866934058261\\
485	0.00623406285029019\\
486	0.00624966382004183\\
487	0.00626547518621064\\
488	0.00628149993655961\\
489	0.0062977411125919\\
490	0.00631420181205383\\
491	0.00633088519160514\\
492	0.00634779446967697\\
493	0.00636493292954258\\
494	0.00638230392263143\\
495	0.0063999108721217\\
496	0.00641775727685009\\
497	0.00643584671557906\\
498	0.00645418285166201\\
499	0.00647276943814956\\
500	0.00649161032338459\\
501	0.0065107094571397\\
502	0.00653007089735626\\
503	0.00654969881755083\\
504	0.0065695975149618\\
505	0.00658977141951615\\
506	0.00661022510370496\\
507	0.00663096329346413\\
508	0.00665199088016667\\
509	0.0066733129338431\\
510	0.00669493471775744\\
511	0.00671686170447795\\
512	0.00673909959359459\\
513	0.00676165433124745\\
514	0.00678453213164819\\
515	0.00680773950079382\\
516	0.00683128326258385\\
517	0.00685517058756828\\
518	0.0068794090243893\\
519	0.00690400653407585\\
520	0.00692897152752586\\
521	0.00695431290571268\\
522	0.00698004010997143\\
523	0.00700616317875087\\
524	0.00703269280624844\\
525	0.00705964046356102\\
526	0.00708701828899484\\
527	0.00711483901941279\\
528	0.00714311610684903\\
529	0.00717186079668715\\
530	0.00720108516360538\\
531	0.00723081189144472\\
532	0.0072610644887466\\
533	0.00729186641857461\\
534	0.00732325074755106\\
535	0.00735523634621418\\
536	0.0073878402515663\\
537	0.00742107811239321\\
538	0.00745496230375221\\
539	0.0074894968293927\\
540	0.00752466168945731\\
541	0.00756038385752855\\
542	0.00759672502313432\\
543	0.00763374757337947\\
544	0.00767162181685216\\
545	0.00771041237843783\\
546	0.00775017685925997\\
547	0.00779097741675524\\
548	0.0078328729291118\\
549	0.00787593015589296\\
550	0.00792022484569437\\
551	0.00796587725701316\\
552	0.00801304218089454\\
553	0.00805960843508264\\
554	0.00810430047028959\\
555	0.00814779918778151\\
556	0.00819148112952877\\
557	0.00823562173607292\\
558	0.00828025250104444\\
559	0.00832535069346646\\
560	0.00837086503244462\\
561	0.00841672430456974\\
562	0.00846284579668691\\
563	0.00850912918133008\\
564	0.00855432580474309\\
565	0.00859878079862082\\
566	0.00864330641528182\\
567	0.00868795759871341\\
568	0.00873271555494383\\
569	0.00877755218112524\\
570	0.00882198003310067\\
571	0.00886641486864619\\
572	0.00891109494574295\\
573	0.00895600620365328\\
574	0.00900111881251587\\
575	0.00904639944292441\\
576	0.00909181177272146\\
577	0.0091373164008612\\
578	0.00918287075438226\\
579	0.00922842902321877\\
580	0.00927394213786092\\
581	0.00931935780604082\\
582	0.00936462062789891\\
583	0.00940967231401762\\
584	0.00945445203657308\\
585	0.0094988969504773\\
586	0.00954294292772307\\
587	0.00958652555102228\\
588	0.00962958140299686\\
589	0.00967204963943729\\
590	0.0097138736812812\\
591	0.00975500241768019\\
592	0.00979538909575469\\
593	0.00983498278816331\\
594	0.00987369853931868\\
595	0.00991118387968948\\
596	0.00994658651044256\\
597	0.00997788999445116\\
598	0.010000292044645\\
599	0\\
600	0\\
};
\addplot [color=mycolor6,solid,forget plot]
  table[row sep=crcr]{%
1	0.0040957575257232\\
2	0.00409577859341713\\
3	0.00409579997628435\\
4	0.004095821678038\\
5	0.00409584370241615\\
6	0.00409586605318159\\
7	0.00409588873412207\\
8	0.00409591174905029\\
9	0.00409593510180391\\
10	0.00409595879624581\\
11	0.00409598283626408\\
12	0.00409600722577218\\
13	0.00409603196870914\\
14	0.00409605706903975\\
15	0.00409608253075475\\
16	0.00409610835787113\\
17	0.00409613455443227\\
18	0.00409616112450847\\
19	0.00409618807219714\\
20	0.00409621540162322\\
21	0.00409624311693958\\
22	0.00409627122232759\\
23	0.00409629972199747\\
24	0.00409632862018893\\
25	0.00409635792117177\\
26	0.00409638762924648\\
27	0.00409641774874495\\
28	0.00409644828403125\\
29	0.00409647923950236\\
30	0.00409651061958913\\
31	0.00409654242875714\\
32	0.00409657467150769\\
33	0.00409660735237884\\
34	0.00409664047594657\\
35	0.00409667404682594\\
36	0.00409670806967228\\
37	0.00409674254918267\\
38	0.00409677749009724\\
39	0.00409681289720065\\
40	0.00409684877532375\\
41	0.00409688512934512\\
42	0.00409692196419291\\
43	0.00409695928484662\\
44	0.00409699709633897\\
45	0.00409703540375805\\
46	0.00409707421224927\\
47	0.00409711352701767\\
48	0.00409715335333018\\
49	0.00409719369651803\\
50	0.00409723456197922\\
51	0.00409727595518121\\
52	0.00409731788166357\\
53	0.00409736034704081\\
54	0.00409740335700531\\
55	0.00409744691733042\\
56	0.00409749103387354\\
57	0.00409753571257934\\
58	0.00409758095948331\\
59	0.00409762678071508\\
60	0.00409767318250206\\
61	0.00409772017117324\\
62	0.00409776775316286\\
63	0.00409781593501451\\
64	0.00409786472338505\\
65	0.00409791412504892\\
66	0.00409796414690219\\
67	0.00409801479596717\\
68	0.00409806607939678\\
69	0.00409811800447917\\
70	0.00409817057864237\\
71	0.00409822380945914\\
72	0.0040982777046519\\
73	0.00409833227209759\\
74	0.00409838751983293\\
75	0.00409844345605942\\
76	0.00409850008914866\\
77	0.00409855742764775\\
78	0.00409861548028459\\
79	0.00409867425597347\\
80	0.00409873376382058\\
81	0.00409879401312954\\
82	0.00409885501340726\\
83	0.00409891677436947\\
84	0.00409897930594667\\
85	0.00409904261828975\\
86	0.00409910672177596\\
87	0.0040991716270148\\
88	0.00409923734485389\\
89	0.00409930388638487\\
90	0.00409937126294934\\
91	0.00409943948614486\\
92	0.00409950856783085\\
93	0.00409957852013454\\
94	0.00409964935545692\\
95	0.00409972108647858\\
96	0.00409979372616584\\
97	0.00409986728777638\\
98	0.00409994178486525\\
99	0.00410001723129069\\
100	0.00410009364121998\\
101	0.0041001710291351\\
102	0.00410024940983868\\
103	0.00410032879845951\\
104	0.00410040921045839\\
105	0.00410049066163375\\
106	0.00410057316812724\\
107	0.0041006567464294\\
108	0.00410074141338527\\
109	0.00410082718619986\\
110	0.00410091408244383\\
111	0.00410100212005902\\
112	0.00410109131736399\\
113	0.00410118169305965\\
114	0.00410127326623474\\
115	0.0041013660563716\\
116	0.00410146008335181\\
117	0.00410155536746177\\
118	0.00410165192939864\\
119	0.00410174979027601\\
120	0.00410184897163003\\
121	0.00410194949542532\\
122	0.00410205138406099\\
123	0.00410215466037715\\
124	0.00410225934766104\\
125	0.00410236546965366\\
126	0.00410247305055639\\
127	0.00410258211503788\\
128	0.00410269268824098\\
129	0.00410280479578989\\
130	0.00410291846379755\\
131	0.00410303371887327\\
132	0.0041031505881303\\
133	0.00410326909919388\\
134	0.00410338928020944\\
135	0.00410351115985074\\
136	0.00410363476732871\\
137	0.00410376013240006\\
138	0.00410388728537628\\
139	0.00410401625713288\\
140	0.00410414707911871\\
141	0.0041042797833657\\
142	0.00410441440249856\\
143	0.00410455096974481\\
144	0.00410468951894524\\
145	0.00410483008456399\\
146	0.00410497270169956\\
147	0.00410511740609558\\
148	0.00410526423415193\\
149	0.00410541322293573\\
150	0.00410556441019279\\
151	0.00410571783435943\\
152	0.00410587353457443\\
153	0.0041060315506911\\
154	0.00410619192328988\\
155	0.00410635469369109\\
156	0.0041065199039677\\
157	0.00410668759695891\\
158	0.00410685781628351\\
159	0.004107030606354\\
160	0.00410720601239067\\
161	0.00410738408043628\\
162	0.00410756485737098\\
163	0.00410774839092757\\
164	0.00410793472970733\\
165	0.00410812392319595\\
166	0.00410831602178023\\
167	0.00410851107676491\\
168	0.00410870914039013\\
169	0.00410891026584936\\
170	0.00410911450730776\\
171	0.00410932191992095\\
172	0.00410953255985457\\
173	0.00410974648430407\\
174	0.00410996375151537\\
175	0.00411018442080578\\
176	0.0041104085525858\\
177	0.0041106362083814\\
178	0.00411086745085676\\
179	0.00411110234383806\\
180	0.00411134095233764\\
181	0.00411158334257883\\
182	0.00411182958202163\\
183	0.004112079739389\\
184	0.00411233388469389\\
185	0.004112592089267\\
186	0.00411285442578532\\
187	0.00411312096830136\\
188	0.00411339179227324\\
189	0.00411366697459539\\
190	0.00411394659363018\\
191	0.00411423072924033\\
192	0.0041145194628219\\
193	0.00411481287733823\\
194	0.00411511105735464\\
195	0.00411541408907388\\
196	0.00411572206037222\\
197	0.0041160350608364\\
198	0.00411635318180128\\
199	0.00411667651638819\\
200	0.004117005159544\\
201	0.00411733920808075\\
202	0.00411767876071612\\
203	0.00411802391811416\\
204	0.00411837478292706\\
205	0.00411873145983705\\
206	0.00411909405559909\\
207	0.00411946267908387\\
208	0.00411983744132134\\
209	0.00412021845554453\\
210	0.00412060583723384\\
211	0.00412099970416156\\
212	0.00412140017643675\\
213	0.00412180737655021\\
214	0.00412222142941977\\
215	0.00412264246243567\\
216	0.00412307060550606\\
217	0.00412350599110271\\
218	0.0041239487543066\\
219	0.00412439903285371\\
220	0.00412485696718078\\
221	0.00412532270047123\\
222	0.00412579637870083\\
223	0.00412627815068381\\
224	0.00412676816811873\\
225	0.00412726658563466\\
226	0.00412777356083737\\
227	0.00412828925435592\\
228	0.0041288138298894\\
229	0.00412934745425412\\
230	0.00412989029743132\\
231	0.00413044253261524\\
232	0.0041310043362622\\
233	0.00413157588814017\\
234	0.00413215737137989\\
235	0.00413274897252666\\
236	0.00413335088159374\\
237	0.00413396329211722\\
238	0.00413458640121268\\
239	0.00413522040963379\\
240	0.00413586552183297\\
241	0.00413652194602454\\
242	0.00413718989425032\\
243	0.00413786958244821\\
244	0.00413856123052352\\
245	0.00413926506242377\\
246	0.00413998130621683\\
247	0.00414071019417249\\
248	0.00414145196284843\\
249	0.00414220685317966\\
250	0.00414297511057278\\
251	0.00414375698500402\\
252	0.00414455273112235\\
253	0.00414536260835737\\
254	0.00414618688103258\\
255	0.00414702581848413\\
256	0.00414787969518288\\
257	0.00414874879085554\\
258	0.00414963339061607\\
259	0.00415053378510179\\
260	0.00415145027061411\\
261	0.00415238314926421\\
262	0.00415333272912317\\
263	0.00415429932437663\\
264	0.00415528325548441\\
265	0.00415628484934529\\
266	0.00415730443946647\\
267	0.00415834236613792\\
268	0.00415939897661162\\
269	0.00416047462528526\\
270	0.00416156967389093\\
271	0.00416268449168826\\
272	0.00416381945566199\\
273	0.00416497495072398\\
274	0.00416615136991966\\
275	0.00416734911463855\\
276	0.00416856859482894\\
277	0.00416981022921671\\
278	0.00417107444552737\\
279	0.00417236168071149\\
280	0.00417367238117411\\
281	0.00417500700300752\\
282	0.00417636601222807\\
283	0.00417774988501727\\
284	0.00417915910796586\\
285	0.00418059417832096\\
286	0.00418205560423672\\
287	0.00418354390502853\\
288	0.00418505961143148\\
289	0.00418660326586487\\
290	0.00418817542270023\\
291	0.00418977664853424\\
292	0.00419140752246745\\
293	0.00419306863638921\\
294	0.00419476059526954\\
295	0.00419648401745842\\
296	0.00419823953499375\\
297	0.00420002779391894\\
298	0.0042018494546114\\
299	0.0042037051921232\\
300	0.00420559569653311\\
301	0.00420752167330453\\
302	0.00420948384366157\\
303	0.00421148294498171\\
304	0.00421351973120494\\
305	0.00421559497326092\\
306	0.00421770945951509\\
307	0.0042198639962358\\
308	0.00422205940808414\\
309	0.0042242965386328\\
310	0.00422657625092935\\
311	0.00422889942811147\\
312	0.00423126697407714\\
313	0.00423367981422548\\
314	0.00423613889609186\\
315	0.00423864519013073\\
316	0.00424119969056351\\
317	0.00424380341627011\\
318	0.00424645741171143\\
319	0.0042491627478955\\
320	0.00425192052339256\\
321	0.00425473186539667\\
322	0.00425759793083365\\
323	0.0042605199075153\\
324	0.00426349901534109\\
325	0.00426653650757716\\
326	0.00426963367219471\\
327	0.0042727918332629\\
328	0.0042760123524123\\
329	0.00427929663038039\\
330	0.00428264610864861\\
331	0.00428606227112797\\
332	0.0042895466458308\\
333	0.00429310080669641\\
334	0.00429672637548867\\
335	0.00430042502375289\\
336	0.00430419847480939\\
337	0.00430804850578537\\
338	0.00431197694982626\\
339	0.0043159856984577\\
340	0.00432007670380599\\
341	0.0043242519807938\\
342	0.00432851360926847\\
343	0.00433286373601202\\
344	0.00433730457657591\\
345	0.00434183841689476\\
346	0.00434646761468339\\
347	0.00435119460075521\\
348	0.00435602188021876\\
349	0.00436095203435385\\
350	0.00436598771960515\\
351	0.00437113167052693\\
352	0.00437638670281178\\
353	0.00438175571708934\\
354	0.00438724170317217\\
355	0.00439284774500271\\
356	0.00439857702484328\\
357	0.00440443282788267\\
358	0.00441041854789761\\
359	0.00441653769327272\\
360	0.00442279389322759\\
361	0.0044291909064023\\
362	0.00443573263321883\\
363	0.00444242312760875\\
364	0.0044492666106289\\
365	0.0044562674862465\\
366	0.00446343035955969\\
367	0.00447076005766673\\
368	0.0044782616533322\\
369	0.0044859404917794\\
370	0.00449380222149397\\
371	0.00450185282282893\\
372	0.00451009863813425\\
373	0.00451854639854589\\
374	0.00452720324248943\\
375	0.00453607672148651\\
376	0.00454517480924487\\
377	0.00455450571499868\\
378	0.00456407786561144\\
379	0.00457389987876267\\
380	0.00458398051178957\\
381	0.00459432865007819\\
382	0.00460495328678225\\
383	0.00461586348087892\\
384	0.00462706830927145\\
385	0.0046385768137451\\
386	0.00465039794434209\\
387	0.0046625405016459\\
388	0.00467501308273088\\
389	0.00468782406537107\\
390	0.0047009815918066\\
391	0.00471449359032724\\
392	0.00472836787429916\\
393	0.00474261236282055\\
394	0.00475723558076776\\
395	0.00477224709037745\\
396	0.00478765835628073\\
397	0.00480348381186787\\
398	0.00481974914418876\\
399	0.00483648354993576\\
400	0.00485371815958732\\
401	0.00487148598747839\\
402	0.00488982180268676\\
403	0.00490876188850308\\
404	0.00492834364062747\\
405	0.00494860488779465\\
406	0.00496958322414589\\
407	0.00499131486987985\\
408	0.00501383256332424\\
409	0.00503716515917807\\
410	0.00506133397257322\\
411	0.00508634837559588\\
412	0.00511220029646462\\
413	0.00513884973395452\\
414	0.00516621948017444\\
415	0.00519418140474314\\
416	0.00522219113231119\\
417	0.00524989647293191\\
418	0.00527723538116368\\
419	0.00530414018582434\\
420	0.00533053757369377\\
421	0.00535634855725208\\
422	0.00538148875596453\\
423	0.00540586895196347\\
424	0.00542939656222645\\
425	0.00545197769857029\\
426	0.00547352034188589\\
427	0.00549393886548258\\
428	0.00551316036094833\\
429	0.00553113346425168\\
430	0.00554784052366552\\
431	0.00556331422444391\\
432	0.00557791451485326\\
433	0.00559216759613327\\
434	0.00560606131561664\\
435	0.00561958666055824\\
436	0.00563273850432294\\
437	0.0056455164115504\\
438	0.00565792548096981\\
439	0.00566997719468297\\
440	0.00568169021734538\\
441	0.00569309105650054\\
442	0.00570421445207726\\
443	0.00571510330308959\\
444	0.00572580785695178\\
445	0.00573638377343506\\
446	0.00574688852537332\\
447	0.00575736051133426\\
448	0.00576781473905054\\
449	0.00577826051033063\\
450	0.00578870823467299\\
451	0.00579916932351866\\
452	0.00580965602840855\\
453	0.00582018121534943\\
454	0.00583075806991991\\
455	0.00584139973202969\\
456	0.00585211886682199\\
457	0.00586292719038153\\
458	0.00587383498817624\\
459	0.0058848506935462\\
460	0.00589598063712908\\
461	0.00590722996532073\\
462	0.00591860360210382\\
463	0.00593010643016358\\
464	0.00594174324038562\\
465	0.0059535186821428\\
466	0.0059654372173562\\
467	0.00597750308200778\\
468	0.00598972025937734\\
469	0.00600209246959655\\
470	0.00601462317989138\\
471	0.00602731563870397\\
472	0.00604017293410104\\
473	0.00605319807152112\\
474	0.00606639402431047\\
475	0.00607976374298815\\
476	0.00609331015374547\\
477	0.00610703615890761\\
478	0.00612094463951541\\
479	0.00613503846005787\\
480	0.00614932047521305\\
481	0.00616379353823352\\
482	0.00617846051035483\\
483	0.00619332427033949\\
484	0.00620838772305117\\
485	0.00622365380589142\\
486	0.00623912549261596\\
487	0.00625480579621778\\
488	0.00627069777218987\\
489	0.00628680452212941\\
490	0.00630312919763236\\
491	0.00631967500441896\\
492	0.00633644520662642\\
493	0.00635344313121247\\
494	0.00637067217243157\\
495	0.00638813579637969\\
496	0.00640583754565251\\
497	0.00642378104422246\\
498	0.00644197000268395\\
499	0.00646040822396084\\
500	0.00647909960951374\\
501	0.006498048166091\\
502	0.00651725801307465\\
503	0.00653673339048063\\
504	0.00655647866768176\\
505	0.0065764983529322\\
506	0.00659679710378154\\
507	0.00661737973847604\\
508	0.00663825124845157\\
509	0.00665941681202778\\
510	0.00668088180941628\\
511	0.00670265183916187\\
512	0.00672473273614287\\
513	0.00674713059126367\\
514	0.0067698517729772\\
515	0.00679290295077906\\
516	0.00681629112081456\\
517	0.00684002363373771\\
518	0.00686410822495721\\
519	0.00688855304739165\\
520	0.00691336670683068\\
521	0.00693855829996578\\
522	0.00696413745491618\\
523	0.00699011437411914\\
524	0.00701649987951138\\
525	0.00704330545914029\\
526	0.00707054332066159\\
527	0.00709822644926038\\
528	0.00712636866359132\\
529	0.0071549847035114\\
530	0.00718409021224089\\
531	0.00721370155228759\\
532	0.00724383586893235\\
533	0.00727451118922645\\
534	0.00730573830354627\\
535	0.00733754289382939\\
536	0.00736995197449274\\
537	0.00740299261041146\\
538	0.00743669497365421\\
539	0.00747108869579545\\
540	0.00750619254437731\\
541	0.00754202239752112\\
542	0.00757858696594791\\
543	0.00761588370558736\\
544	0.00765381442335629\\
545	0.00769240422359834\\
546	0.00773170952485417\\
547	0.00777184860363677\\
548	0.0078129512996561\\
549	0.0078550743470806\\
550	0.00789829083241618\\
551	0.00794266950789453\\
552	0.00798828443682109\\
553	0.00803524805689781\\
554	0.00808370675822525\\
555	0.00813245686176951\\
556	0.00817966759564629\\
557	0.00822469162758563\\
558	0.00826987878715762\\
559	0.00831532757736804\\
560	0.00836118747187362\\
561	0.00840745876100177\\
562	0.00845408897122118\\
563	0.00850099755665664\\
564	0.00854809439039289\\
565	0.00859487531085683\\
566	0.00864051776728843\\
567	0.00868567121298489\\
568	0.00873085196361323\\
569	0.00877609289333616\\
570	0.00882137592721109\\
571	0.00886636048870486\\
572	0.00891108851230718\\
573	0.00895600554864665\\
574	0.00900111866282342\\
575	0.00904639939181835\\
576	0.00909181174974404\\
577	0.00913731638890149\\
578	0.00918287074791749\\
579	0.00922842901963624\\
580	0.00927394213588079\\
581	0.00931935780498493\\
582	0.00936462062738148\\
583	0.00940967231379966\\
584	0.00945445203650304\\
585	0.00949889695046246\\
586	0.00954294292772186\\
587	0.00958652555102227\\
588	0.00962958140299686\\
589	0.00967204963943729\\
590	0.0097138736812812\\
591	0.00975500241768019\\
592	0.00979538909575469\\
593	0.00983498278816331\\
594	0.00987369853931868\\
595	0.00991118387968948\\
596	0.00994658651044256\\
597	0.00997788999445116\\
598	0.010000292044645\\
599	0\\
600	0\\
};
\addplot [color=mycolor7,solid,forget plot]
  table[row sep=crcr]{%
1	0.00409529888158573\\
2	0.00409531335916626\\
3	0.00409532803996346\\
4	0.00409534292632739\\
5	0.00409535802063159\\
6	0.00409537332527361\\
7	0.00409538884267553\\
8	0.0040954045752844\\
9	0.00409542052557287\\
10	0.00409543669603968\\
11	0.00409545308921039\\
12	0.00409546970763792\\
13	0.00409548655390317\\
14	0.0040955036306158\\
15	0.00409552094041491\\
16	0.00409553848596972\\
17	0.00409555626998042\\
18	0.00409557429517898\\
19	0.00409559256432993\\
20	0.00409561108023117\\
21	0.00409562984571509\\
22	0.00409564886364924\\
23	0.00409566813693751\\
24	0.00409568766852107\\
25	0.00409570746137935\\
26	0.00409572751853122\\
27	0.00409574784303607\\
28	0.00409576843799497\\
29	0.00409578930655189\\
30	0.00409581045189494\\
31	0.00409583187725764\\
32	0.00409585358592025\\
33	0.00409587558121114\\
34	0.00409589786650818\\
35	0.0040959204452403\\
36	0.00409594332088887\\
37	0.00409596649698929\\
38	0.00409598997713262\\
39	0.00409601376496716\\
40	0.00409603786420016\\
41	0.00409606227859958\\
42	0.00409608701199582\\
43	0.00409611206828361\\
44	0.00409613745142382\\
45	0.00409616316544542\\
46	0.00409618921444747\\
47	0.00409621560260106\\
48	0.00409624233415152\\
49	0.00409626941342034\\
50	0.00409629684480756\\
51	0.00409632463279382\\
52	0.00409635278194263\\
53	0.00409638129690279\\
54	0.00409641018241064\\
55	0.00409643944329256\\
56	0.00409646908446722\\
57	0.00409649911094831\\
58	0.004096529527847\\
59	0.00409656034037446\\
60	0.00409659155384456\\
61	0.00409662317367653\\
62	0.00409665520539764\\
63	0.00409668765464605\\
64	0.0040967205271735\\
65	0.00409675382884819\\
66	0.0040967875656577\\
67	0.00409682174371186\\
68	0.00409685636924571\\
69	0.00409689144862251\\
70	0.00409692698833673\\
71	0.00409696299501717\\
72	0.00409699947543001\\
73	0.0040970364364819\\
74	0.00409707388522319\\
75	0.00409711182885107\\
76	0.00409715027471284\\
77	0.00409718923030907\\
78	0.00409722870329695\\
79	0.00409726870149347\\
80	0.00409730923287888\\
81	0.00409735030559997\\
82	0.00409739192797341\\
83	0.00409743410848921\\
84	0.00409747685581403\\
85	0.00409752017879478\\
86	0.00409756408646196\\
87	0.00409760858803316\\
88	0.00409765369291657\\
89	0.0040976994107146\\
90	0.00409774575122724\\
91	0.00409779272445585\\
92	0.00409784034060656\\
93	0.00409788861009412\\
94	0.00409793754354529\\
95	0.00409798715180282\\
96	0.00409803744592888\\
97	0.00409808843720899\\
98	0.00409814013715571\\
99	0.00409819255751257\\
100	0.00409824571025767\\
101	0.00409829960760789\\
102	0.00409835426202258\\
103	0.00409840968620775\\
104	0.00409846589311991\\
105	0.0040985228959703\\
106	0.00409858070822895\\
107	0.00409863934362897\\
108	0.00409869881617072\\
109	0.00409875914012631\\
110	0.00409882033004383\\
111	0.004098882400752\\
112	0.00409894536736461\\
113	0.00409900924528532\\
114	0.00409907405021224\\
115	0.00409913979814298\\
116	0.00409920650537936\\
117	0.00409927418853265\\
118	0.00409934286452856\\
119	0.00409941255061264\\
120	0.00409948326435549\\
121	0.00409955502365831\\
122	0.00409962784675861\\
123	0.00409970175223569\\
124	0.00409977675901671\\
125	0.00409985288638252\\
126	0.00409993015397387\\
127	0.00410000858179754\\
128	0.00410008819023275\\
129	0.00410016900003767\\
130	0.00410025103235615\\
131	0.0041003343087243\\
132	0.00410041885107761\\
133	0.00410050468175801\\
134	0.00410059182352093\\
135	0.00410068029954293\\
136	0.00410077013342902\\
137	0.00410086134922046\\
138	0.00410095397140261\\
139	0.00410104802491292\\
140	0.00410114353514911\\
141	0.00410124052797757\\
142	0.00410133902974192\\
143	0.00410143906727174\\
144	0.00410154066789134\\
145	0.00410164385942924\\
146	0.00410174867022732\\
147	0.00410185512915045\\
148	0.00410196326559641\\
149	0.00410207310950645\\
150	0.0041021846913758\\
151	0.00410229804226414\\
152	0.00410241319380625\\
153	0.00410253017822295\\
154	0.00410264902833234\\
155	0.00410276977756104\\
156	0.00410289245995604\\
157	0.00410301711019652\\
158	0.00410314376360608\\
159	0.00410327245616516\\
160	0.00410340322452384\\
161	0.00410353610601481\\
162	0.00410367113866676\\
163	0.00410380836121798\\
164	0.00410394781313014\\
165	0.00410408953460283\\
166	0.00410423356658778\\
167	0.00410437995080393\\
168	0.00410452872975262\\
169	0.00410467994673294\\
170	0.00410483364585779\\
171	0.00410498987207004\\
172	0.00410514867115905\\
173	0.0041053100897776\\
174	0.00410547417545914\\
175	0.00410564097663547\\
176	0.00410581054265464\\
177	0.00410598292379933\\
178	0.00410615817130577\\
179	0.00410633633738256\\
180	0.0041065174752303\\
181	0.0041067016390614\\
182	0.00410688888412046\\
183	0.0041070792667047\\
184	0.00410727284418501\\
185	0.0041074696750275\\
186	0.00410766981881499\\
187	0.00410787333626945\\
188	0.00410808028927429\\
189	0.00410829074089746\\
190	0.00410850475541469\\
191	0.0041087223983331\\
192	0.00410894373641531\\
193	0.00410916883770393\\
194	0.00410939777154627\\
195	0.00410963060861962\\
196	0.00410986742095669\\
197	0.00411010828197159\\
198	0.00411035326648619\\
199	0.00411060245075664\\
200	0.00411085591250051\\
201	0.00411111373092417\\
202	0.0041113759867505\\
203	0.00411164276224718\\
204	0.00411191414125503\\
205	0.004112190209217\\
206	0.00411247105320746\\
207	0.00411275676196188\\
208	0.00411304742590691\\
209	0.00411334313719079\\
210	0.00411364398971436\\
211	0.00411395007916245\\
212	0.00411426150303559\\
213	0.00411457836068234\\
214	0.00411490075333212\\
215	0.00411522878412847\\
216	0.00411556255816302\\
217	0.00411590218250975\\
218	0.0041162477662602\\
219	0.00411659942055925\\
220	0.00411695725864131\\
221	0.00411732139586763\\
222	0.00411769194976415\\
223	0.00411806904006028\\
224	0.00411845278872851\\
225	0.00411884332002493\\
226	0.00411924076053081\\
227	0.00411964523919514\\
228	0.00412005688737832\\
229	0.00412047583889703\\
230	0.00412090223007023\\
231	0.00412133619976646\\
232	0.00412177788945259\\
233	0.00412222744324388\\
234	0.00412268500795542\\
235	0.0041231507331553\\
236	0.00412362477121914\\
237	0.00412410727738639\\
238	0.00412459840981831\\
239	0.00412509832965749\\
240	0.00412560720108942\\
241	0.00412612519140568\\
242	0.004126652471069\\
243	0.00412718921378027\\
244	0.00412773559654744\\
245	0.00412829179975632\\
246	0.00412885800724327\\
247	0.0041294344063702\\
248	0.00413002118810105\\
249	0.00413061854708067\\
250	0.00413122668171561\\
251	0.00413184579425673\\
252	0.00413247609088421\\
253	0.00413311778179431\\
254	0.00413377108128832\\
255	0.00413443620786323\\
256	0.00413511338430401\\
257	0.00413580283777837\\
258	0.00413650479993368\\
259	0.00413721950699552\\
260	0.00413794719986878\\
261	0.00413868812424015\\
262	0.0041394425306831\\
263	0.00414021067476449\\
264	0.00414099281715355\\
265	0.00414178922373251\\
266	0.00414260016570967\\
267	0.00414342591973415\\
268	0.00414426676801282\\
269	0.00414512299842949\\
270	0.00414599490466596\\
271	0.00414688278632531\\
272	0.00414778694905744\\
273	0.00414870770468688\\
274	0.00414964537134285\\
275	0.00415060027359177\\
276	0.00415157274257249\\
277	0.00415256311613383\\
278	0.00415357173897492\\
279	0.00415459896278864\\
280	0.0041556451464081\\
281	0.00415671065595639\\
282	0.00415779586500009\\
283	0.00415890115470628\\
284	0.0041600269140035\\
285	0.00416117353974716\\
286	0.00416234143688919\\
287	0.00416353101865285\\
288	0.00416474270671286\\
289	0.00416597693138086\\
290	0.00416723413179686\\
291	0.00416851475612708\\
292	0.00416981926176857\\
293	0.00417114811556082\\
294	0.00417250179400487\\
295	0.00417388078349076\\
296	0.00417528558053294\\
297	0.0041767166920149\\
298	0.00417817463544284\\
299	0.00417965993920883\\
300	0.00418117314286336\\
301	0.00418271479739847\\
302	0.00418428546554185\\
303	0.00418588572206159\\
304	0.00418751615408255\\
305	0.00418917736141384\\
306	0.00419086995688843\\
307	0.00419259456671431\\
308	0.00419435183083869\\
309	0.00419614240332592\\
310	0.00419796695274931\\
311	0.00419982616259705\\
312	0.0042017207316897\\
313	0.00420365137459837\\
314	0.00420561882208432\\
315	0.00420762382155358\\
316	0.0042096671375247\\
317	0.00421174955210941\\
318	0.00421387186550712\\
319	0.00421603489651504\\
320	0.00421823948305327\\
321	0.00422048648270602\\
322	0.00422277677327956\\
323	0.00422511125337804\\
324	0.00422749084300021\\
325	0.00422991648415631\\
326	0.00423238914150635\\
327	0.00423490980302193\\
328	0.00423747948067404\\
329	0.00424009921114819\\
330	0.00424277005658351\\
331	0.00424549310533493\\
332	0.00424826947277318\\
333	0.00425110030211756\\
334	0.00425398676530208\\
335	0.0042569300638763\\
336	0.00425993142994667\\
337	0.00426299212717308\\
338	0.00426611345181838\\
339	0.00426929673383262\\
340	0.00427254333798882\\
341	0.00427585466507403\\
342	0.00427923215314009\\
343	0.00428267727882064\\
344	0.00428619155872267\\
345	0.00428977655090712\\
346	0.00429343385647985\\
347	0.00429716512129608\\
348	0.00430097203779398\\
349	0.00430485634670329\\
350	0.0043088198390626\\
351	0.00431286435833666\\
352	0.00431699180268599\\
353	0.00432120412736427\\
354	0.00432550334724409\\
355	0.00432989153936209\\
356	0.00433437084559166\\
357	0.00433894347548184\\
358	0.00434361170919735\\
359	0.00434837790057111\\
360	0.0043532444804495\\
361	0.00435821396036348\\
362	0.00436328893615132\\
363	0.00436847209172436\\
364	0.00437376620296244\\
365	0.00437917414171798\\
366	0.00438469887990066\\
367	0.00439034349361296\\
368	0.00439611116732586\\
369	0.00440200519807622\\
370	0.00440802899928185\\
371	0.00441418610453587\\
372	0.00442048017121563\\
373	0.00442691498401626\\
374	0.00443349445901006\\
375	0.00444022264981431\\
376	0.00444710374764763\\
377	0.00445414208774742\\
378	0.00446134215594629\\
379	0.00446870859513666\\
380	0.00447624621648438\\
381	0.00448396001123058\\
382	0.00449185516245293\\
383	0.00449993705836801\\
384	0.00450821130742561\\
385	0.00451668375545995\\
386	0.00452536050518122\\
387	0.00453424793849747\\
388	0.00454335274328614\\
389	0.00455268194107426\\
390	0.00456224291751835\\
391	0.00457204345661088\\
392	0.00458209177891305\\
393	0.00459239658516182\\
394	0.00460296707186981\\
395	0.00461381294564437\\
396	0.00462494440577857\\
397	0.00463637213268345\\
398	0.00464810711780087\\
399	0.00466016060587126\\
400	0.00467254406173833\\
401	0.00468526913030364\\
402	0.00469834758954639\\
403	0.00471179129669854\\
404	0.00472561212869725\\
405	0.004739821942918\\
406	0.0047544325287009\\
407	0.00476945555926563\\
408	0.00478490272671644\\
409	0.0048007856878958\\
410	0.00481711609581658\\
411	0.00483390569989827\\
412	0.00485116620336472\\
413	0.0048689103631968\\
414	0.00488715290178423\\
415	0.0049059118337603\\
416	0.00492521605889702\\
417	0.00494510181564877\\
418	0.0049656072975093\\
419	0.0049867723737793\\
420	0.00500863798590982\\
421	0.00503124585634181\\
422	0.0050546384992814\\
423	0.00507885959850083\\
424	0.00510395023558282\\
425	0.00512994484037258\\
426	0.00515686433680294\\
427	0.00518471516750045\\
428	0.00521348289821372\\
429	0.00524312368771077\\
430	0.00527355295684454\\
431	0.0053046303856445\\
432	0.00533589051850966\\
433	0.00536671291745671\\
434	0.00539701960726992\\
435	0.00542672580803933\\
436	0.00545573994399862\\
437	0.00548396439442113\\
438	0.00551129656975341\\
439	0.00553763057692947\\
440	0.00556285975532503\\
441	0.00558688044713349\\
442	0.00560959744130947\\
443	0.00563093167375537\\
444	0.00565083095726243\\
445	0.00566928477518812\\
446	0.00568634437505773\\
447	0.00570257952405059\\
448	0.00571841830284484\\
449	0.00573385115307405\\
450	0.00574887307694424\\
451	0.00576348456606036\\
452	0.0057776925779305\\
453	0.00579151151564601\\
454	0.00580496413961544\\
455	0.00581808230232539\\
456	0.00583090734542928\\
457	0.00584348992964762\\
458	0.00585588896168109\\
459	0.00586816914156737\\
460	0.00588039646581119\\
461	0.00589260790387998\\
462	0.00590481832838029\\
463	0.00591703939841473\\
464	0.0059292839423036\\
465	0.0059415657811149\\
466	0.00595389947914521\\
467	0.00596630001433183\\
468	0.00597878236640431\\
469	0.00599136102925191\\
470	0.00600404946788492\\
471	0.00601685956198455\\
472	0.00602980111100503\\
473	0.00604288152490197\\
474	0.00605610677287527\\
475	0.00606948250346558\\
476	0.00608301430127271\\
477	0.00609670762906814\\
478	0.00611056777217856\\
479	0.00612459978900502\\
480	0.00613880847232134\\
481	0.00615319832658378\\
482	0.00616777356663\\
483	0.0061825381424517\\
484	0.00619749579257747\\
485	0.00621265012407515\\
486	0.00622800469780967\\
487	0.00624356306178102\\
488	0.00625932874999612\\
489	0.00627530528369641\\
490	0.00629149617516352\\
491	0.00630790493418527\\
492	0.00632453507706829\\
493	0.00634139013782821\\
494	0.00635847368088586\\
495	0.00637578931427811\\
496	0.00639334070211891\\
497	0.00641113157493753\\
498	0.00642916573711289\\
499	0.00644744707338765\\
500	0.00646597955615283\\
501	0.00648476725348922\\
502	0.00650381433793958\\
503	0.00652312509597778\\
504	0.00654270393814206\\
505	0.00656255540981227\\
506	0.00658268420263934\\
507	0.00660309516668271\\
508	0.00662379332337688\\
509	0.00664478387952437\\
510	0.00666607224253939\\
511	0.00668766403709905\\
512	0.00670956512331254\\
513	0.00673178161652677\\
514	0.00675431990889423\\
515	0.00677718669283627\\
516	0.00680038898653935\\
517	0.00682393416162442\\
518	0.00684782997312543\\
519	0.0068720845919022\\
520	0.00689670663959\\
521	0.00692170522615219\\
522	0.00694708999005904\\
523	0.00697287114105492\\
524	0.00699905950538988\\
525	0.00702566657328029\\
526	0.00705270454805575\\
527	0.00708018639631153\\
528	0.00710812589829871\\
529	0.00713653769710684\\
530	0.00716543734818967\\
531	0.00719484137021395\\
532	0.007224767287208\\
533	0.00725523365595185\\
534	0.00728626016991656\\
535	0.00731786730671243\\
536	0.0073500762855338\\
537	0.00738290903487387\\
538	0.00741638488680999\\
539	0.00745052410967822\\
540	0.00748535622462389\\
541	0.00752091025166277\\
542	0.00755721342335093\\
543	0.00759429613161743\\
544	0.00763218088623283\\
545	0.00767086853324643\\
546	0.00771034715895772\\
547	0.00775055806174717\\
548	0.00779145697827102\\
549	0.00783309498128678\\
550	0.00787553646602525\\
551	0.00791895405846008\\
552	0.00796342795168289\\
553	0.00800902558206601\\
554	0.00805583223394632\\
555	0.00810394296610672\\
556	0.00815348917904342\\
557	0.00820452788108392\\
558	0.00825427676315796\\
559	0.00830217862287868\\
560	0.00834893731572075\\
561	0.00839579436605613\\
562	0.00844291676212203\\
563	0.00849036691366464\\
564	0.00853811179111934\\
565	0.00858607445145949\\
566	0.00863415405798469\\
567	0.00868157869679225\\
568	0.00872782524480766\\
569	0.00877366426417583\\
570	0.00881945432030142\\
571	0.00886522995351837\\
572	0.00891084129040892\\
573	0.00895597597182961\\
574	0.00900111411185122\\
575	0.00904639837989421\\
576	0.00909181141669918\\
577	0.00913731624296646\\
578	0.00918287067265086\\
579	0.00922842897919994\\
580	0.00927394211349327\\
581	0.00931935779255734\\
582	0.0093646206206887\\
583	0.00940967231047064\\
584	0.00945445203507202\\
585	0.00949889694999284\\
586	0.00954294292762137\\
587	0.00958652555101391\\
588	0.00962958140299686\\
589	0.00967204963943729\\
590	0.0097138736812812\\
591	0.00975500241768019\\
592	0.00979538909575469\\
593	0.00983498278816331\\
594	0.00987369853931868\\
595	0.00991118387968948\\
596	0.00994658651044256\\
597	0.00997788999445116\\
598	0.010000292044645\\
599	0\\
600	0\\
};
\addplot [color=mycolor8,solid,forget plot]
  table[row sep=crcr]{%
1	0.0040944850682663\\
2	0.00409449503326179\\
3	0.00409450513478612\\
4	0.00409451537454594\\
5	0.0040945257542724\\
6	0.00409453627572179\\
7	0.00409454694067606\\
8	0.00409455775094352\\
9	0.00409456870835943\\
10	0.00409457981478679\\
11	0.00409459107211679\\
12	0.00409460248226963\\
13	0.00409461404719534\\
14	0.0040946257688743\\
15	0.00409463764931816\\
16	0.00409464969057058\\
17	0.00409466189470794\\
18	0.00409467426384026\\
19	0.00409468680011199\\
20	0.00409469950570288\\
21	0.00409471238282884\\
22	0.00409472543374286\\
23	0.00409473866073591\\
24	0.00409475206613794\\
25	0.00409476565231875\\
26	0.00409477942168906\\
27	0.00409479337670149\\
28	0.00409480751985164\\
29	0.00409482185367903\\
30	0.00409483638076831\\
31	0.00409485110375029\\
32	0.00409486602530311\\
33	0.00409488114815337\\
34	0.00409489647507731\\
35	0.00409491200890193\\
36	0.00409492775250638\\
37	0.00409494370882305\\
38	0.00409495988083891\\
39	0.00409497627159686\\
40	0.0040949928841969\\
41	0.00409500972179762\\
42	0.00409502678761747\\
43	0.00409504408493619\\
44	0.00409506161709623\\
45	0.0040950793875042\\
46	0.00409509739963229\\
47	0.0040951156570198\\
48	0.00409513416327465\\
49	0.00409515292207488\\
50	0.00409517193717029\\
51	0.00409519121238387\\
52	0.00409521075161362\\
53	0.00409523055883402\\
54	0.00409525063809774\\
55	0.00409527099353731\\
56	0.00409529162936692\\
57	0.00409531254988402\\
58	0.00409533375947105\\
59	0.00409535526259738\\
60	0.00409537706382103\\
61	0.0040953991677904\\
62	0.00409542157924627\\
63	0.00409544430302357\\
64	0.00409546734405334\\
65	0.00409549070736458\\
66	0.0040955143980863\\
67	0.00409553842144936\\
68	0.00409556278278859\\
69	0.00409558748754472\\
70	0.0040956125412665\\
71	0.00409563794961268\\
72	0.00409566371835417\\
73	0.00409568985337613\\
74	0.0040957163606802\\
75	0.00409574324638658\\
76	0.00409577051673624\\
77	0.0040957981780932\\
78	0.00409582623694681\\
79	0.00409585469991399\\
80	0.00409588357374152\\
81	0.0040959128653085\\
82	0.00409594258162867\\
83	0.00409597272985288\\
84	0.00409600331727147\\
85	0.00409603435131681\\
86	0.00409606583956581\\
87	0.00409609778974255\\
88	0.00409613020972078\\
89	0.00409616310752666\\
90	0.00409619649134142\\
91	0.0040962303695041\\
92	0.00409626475051434\\
93	0.00409629964303513\\
94	0.00409633505589592\\
95	0.00409637099809522\\
96	0.00409640747880389\\
97	0.004096444507368\\
98	0.00409648209331201\\
99	0.00409652024634192\\
100	0.00409655897634848\\
101	0.00409659829341052\\
102	0.00409663820779827\\
103	0.00409667872997674\\
104	0.00409671987060928\\
105	0.00409676164056121\\
106	0.00409680405090336\\
107	0.00409684711291586\\
108	0.00409689083809193\\
109	0.00409693523814174\\
110	0.00409698032499651\\
111	0.00409702611081243\\
112	0.00409707260797493\\
113	0.00409711982910283\\
114	0.0040971677870529\\
115	0.00409721649492399\\
116	0.00409726596606184\\
117	0.00409731621406359\\
118	0.00409736725278263\\
119	0.00409741909633328\\
120	0.004097471759096\\
121	0.00409752525572223\\
122	0.00409757960113968\\
123	0.00409763481055758\\
124	0.00409769089947206\\
125	0.0040977478836717\\
126	0.00409780577924305\\
127	0.0040978646025764\\
128	0.00409792437037164\\
129	0.00409798509964415\\
130	0.00409804680773077\\
131	0.00409810951229613\\
132	0.00409817323133875\\
133	0.00409823798319739\\
134	0.0040983037865577\\
135	0.00409837066045853\\
136	0.00409843862429871\\
137	0.00409850769784378\\
138	0.00409857790123282\\
139	0.00409864925498518\\
140	0.00409872178000752\\
141	0.00409879549760041\\
142	0.00409887042946505\\
143	0.00409894659770956\\
144	0.00409902402485492\\
145	0.00409910273383964\\
146	0.00409918274802353\\
147	0.00409926409118955\\
148	0.00409934678754345\\
149	0.00409943086171322\\
150	0.00409951633875851\\
151	0.00409960324420857\\
152	0.00409969160407124\\
153	0.00409978144484217\\
154	0.00409987279351427\\
155	0.00409996567758716\\
156	0.00410006012507712\\
157	0.00410015616452687\\
158	0.00410025382501584\\
159	0.00410035313617052\\
160	0.00410045412817502\\
161	0.00410055683178177\\
162	0.00410066127832264\\
163	0.00410076749971998\\
164	0.00410087552849824\\
165	0.0041009853977953\\
166	0.00410109714137453\\
167	0.00410121079363689\\
168	0.00410132638963302\\
169	0.00410144396507606\\
170	0.00410156355635425\\
171	0.00410168520054384\\
172	0.00410180893542254\\
173	0.00410193479948294\\
174	0.00410206283194616\\
175	0.00410219307277598\\
176	0.00410232556269296\\
177	0.00410246034318903\\
178	0.00410259745654214\\
179	0.00410273694583135\\
180	0.00410287885495212\\
181	0.00410302322863191\\
182	0.00410317011244582\\
183	0.00410331955283289\\
184	0.00410347159711248\\
185	0.00410362629350077\\
186	0.00410378369112799\\
187	0.00410394384005544\\
188	0.00410410679129315\\
189	0.00410427259681782\\
190	0.00410444130959084\\
191	0.00410461298357691\\
192	0.00410478767376283\\
193	0.00410496543617664\\
194	0.00410514632790705\\
195	0.00410533040712331\\
196	0.00410551773309538\\
197	0.00410570836621455\\
198	0.00410590236801416\\
199	0.00410609980119111\\
200	0.00410630072962732\\
201	0.00410650521841196\\
202	0.0041067133338639\\
203	0.00410692514355454\\
204	0.00410714071633124\\
205	0.00410736012234113\\
206	0.00410758343305527\\
207	0.00410781072129342\\
208	0.00410804206124933\\
209	0.00410827752851641\\
210	0.00410851720011405\\
211	0.00410876115451436\\
212	0.00410900947166952\\
213	0.00410926223303989\\
214	0.00410951952162238\\
215	0.00410978142197975\\
216	0.00411004802027031\\
217	0.00411031940427848\\
218	0.00411059566344596\\
219	0.00411087688890343\\
220	0.0041111631735034\\
221	0.00411145461185331\\
222	0.00411175130034976\\
223	0.0041120533372133\\
224	0.00411236082252425\\
225	0.00411267385825921\\
226	0.00411299254832841\\
227	0.00411331699861403\\
228	0.00411364731700936\\
229	0.00411398361345889\\
230	0.00411432599999937\\
231	0.00411467459080176\\
232	0.00411502950221417\\
233	0.00411539085280594\\
234	0.00411575876341246\\
235	0.00411613335718126\\
236	0.00411651475961895\\
237	0.00411690309863952\\
238	0.00411729850461322\\
239	0.00411770111041709\\
240	0.00411811105148613\\
241	0.00411852846586585\\
242	0.00411895349426588\\
243	0.00411938628011473\\
244	0.00411982696961559\\
245	0.00412027571180354\\
246	0.00412073265860372\\
247	0.00412119796489075\\
248	0.00412167178854948\\
249	0.00412215429053686\\
250	0.004122645634945\\
251	0.00412314598906578\\
252	0.00412365552345631\\
253	0.00412417441200613\\
254	0.00412470283200538\\
255	0.00412524096421461\\
256	0.00412578899293563\\
257	0.0041263471060841\\
258	0.00412691549526344\\
259	0.00412749435584006\\
260	0.00412808388702039\\
261	0.00412868429192916\\
262	0.00412929577768935\\
263	0.00412991855550383\\
264	0.0041305528407385\\
265	0.00413119885300734\\
266	0.00413185681625898\\
267	0.00413252695886524\\
268	0.00413320951371146\\
269	0.0041339047182888\\
270	0.00413461281478836\\
271	0.00413533405019761\\
272	0.00413606867639873\\
273	0.0041368169502692\\
274	0.00413757913378469\\
275	0.00413835549412417\\
276	0.00413914630377745\\
277	0.00413995184065516\\
278	0.00414077238820099\\
279	0.00414160823550564\\
280	0.00414245967742329\\
281	0.00414332701469304\\
282	0.00414421055406293\\
283	0.00414511060841741\\
284	0.00414602749690825\\
285	0.00414696154508837\\
286	0.00414791308504986\\
287	0.00414888245556494\\
288	0.00414987000223126\\
289	0.00415087607762071\\
290	0.00415190104143217\\
291	0.0041529452606486\\
292	0.00415400910969802\\
293	0.00415509297061887\\
294	0.00415619723322993\\
295	0.00415732229530426\\
296	0.00415846856274837\\
297	0.00415963644978543\\
298	0.00416082637914369\\
299	0.00416203878224955\\
300	0.00416327409942557\\
301	0.00416453278009381\\
302	0.00416581528298421\\
303	0.00416712207634835\\
304	0.00416845363817869\\
305	0.00416981045643338\\
306	0.00417119302926692\\
307	0.00417260186526686\\
308	0.0041740374836967\\
309	0.00417550041474544\\
310	0.00417699119978404\\
311	0.00417851039162818\\
312	0.00418005855480817\\
313	0.00418163626584737\\
314	0.00418324411354955\\
315	0.00418488269929651\\
316	0.00418655263735967\\
317	0.0041882545552216\\
318	0.00418998909390109\\
319	0.00419175690828564\\
320	0.00419355866747476\\
321	0.00419539505513425\\
322	0.00419726676986236\\
323	0.00419917452556858\\
324	0.00420111905186532\\
325	0.00420310109447362\\
326	0.0042051214156433\\
327	0.0042071807945886\\
328	0.00420928002793997\\
329	0.00421141993021198\\
330	0.00421360133428882\\
331	0.00421582509192851\\
332	0.00421809207428652\\
333	0.0042204031724594\\
334	0.0042227592980493\\
335	0.00422516138375087\\
336	0.00422761038396227\\
337	0.00423010727542065\\
338	0.00423265305786158\\
339	0.00423524875470434\\
340	0.00423789541376437\\
341	0.00424059410799402\\
342	0.00424334593625302\\
343	0.00424615202410991\\
344	0.00424901352467674\\
345	0.00425193161947885\\
346	0.00425490751936006\\
347	0.00425794246542194\\
348	0.0042610377299823\\
349	0.00426419461758929\\
350	0.00426741446607708\\
351	0.00427069864766853\\
352	0.00427404857012499\\
353	0.00427746567794334\\
354	0.00428095145359514\\
355	0.00428450741882037\\
356	0.00428813513598078\\
357	0.00429183620947113\\
358	0.00429561228719527\\
359	0.00429946506212598\\
360	0.00430339627395013\\
361	0.00430740771077397\\
362	0.00431150121091045\\
363	0.00431567866475392\\
364	0.00431994201674703\\
365	0.00432429326744639\\
366	0.00432873447569465\\
367	0.00433326776090921\\
368	0.00433789530549518\\
369	0.00434261935737215\\
370	0.00434744223265894\\
371	0.00435236631852469\\
372	0.00435739407623897\\
373	0.00436252804447881\\
374	0.00436777084290759\\
375	0.00437312517544296\\
376	0.00437859383406985\\
377	0.00438417970281689\\
378	0.00438988576193537\\
379	0.00439571509262823\\
380	0.00440167088194655\\
381	0.00440775642781222\\
382	0.00441397514429105\\
383	0.00442033056711756\\
384	0.00442682635946839\\
385	0.00443346631798028\\
386	0.00444025437902488\\
387	0.00444719462527896\\
388	0.00445429129230295\\
389	0.0044615487752745\\
390	0.00446897163596021\\
391	0.00447656460997009\\
392	0.00448433261412566\\
393	0.00449228075180321\\
394	0.00450041431890918\\
395	0.0045087388093455\\
396	0.0045172599237471\\
397	0.00452598357799719\\
398	0.00453491591445068\\
399	0.00454406331532495\\
400	0.0045534324175252\\
401	0.00456303012911759\\
402	0.00457286364769764\\
403	0.00458294048112205\\
404	0.00459326847221748\\
405	0.00460385582532282\\
406	0.00461471113660434\\
407	0.00462584343692504\\
408	0.00463726221812296\\
409	0.00464897746180932\\
410	0.00466099966391952\\
411	0.00467333983998608\\
412	0.00468600960271405\\
413	0.00469902118314442\\
414	0.00471238743222883\\
415	0.00472612181762712\\
416	0.00474023824665073\\
417	0.00475475092648928\\
418	0.00476967431722962\\
419	0.00478502307559469\\
420	0.00480081204463977\\
421	0.0048170562845384\\
422	0.00483377108440233\\
423	0.00485097166490057\\
424	0.00486867295982357\\
425	0.00488688942746511\\
426	0.00490563551255536\\
427	0.00492492582990952\\
428	0.00494477550822037\\
429	0.00496520076900319\\
430	0.00498621983945126\\
431	0.00500785429668989\\
432	0.00503013499635789\\
433	0.00505310424657145\\
434	0.00507680680091776\\
435	0.0051012920114171\\
436	0.00512661182744845\\
437	0.00515281394145678\\
438	0.00517994117337858\\
439	0.00520803153272416\\
440	0.00523711506502273\\
441	0.00526720957702407\\
442	0.00529831488517551\\
443	0.00533040512244193\\
444	0.00536341849519898\\
445	0.00539724369300036\\
446	0.00543170189834466\\
447	0.00546610176344987\\
448	0.00549990610319642\\
449	0.00553301837589037\\
450	0.00556533487705187\\
451	0.00559674536991488\\
452	0.0056271342825192\\
453	0.00565638269071956\\
454	0.00568437138183165\\
455	0.00571098540078678\\
456	0.00573612057040539\\
457	0.00575969257333471\\
458	0.00578164973338271\\
459	0.00580199069287926\\
460	0.00582078861599557\\
461	0.0058388742305532\\
462	0.00585652106058518\\
463	0.00587372288773997\\
464	0.00589047957807101\\
465	0.00590679819241133\\
466	0.00592269410962667\\
467	0.00593819208772081\\
468	0.00595332714098071\\
469	0.00596814504838323\\
470	0.00598270223198533\\
471	0.00599706463399752\\
472	0.00601130506551497\\
473	0.00602549828895608\\
474	0.00603968883362203\\
475	0.00605389505768314\\
476	0.00606813068370529\\
477	0.00608241069479026\\
478	0.00609675109911606\\
479	0.00611116860606077\\
480	0.00612568020806276\\
481	0.00614030266958773\\
482	0.00615505193696179\\
483	0.00616994250272472\\
484	0.00618498678894013\\
485	0.0062001946602503\\
486	0.00621557354234828\\
487	0.00623112997508471\\
488	0.00624687045373967\\
489	0.006262801363535\\
490	0.00627892891613285\\
491	0.00629525909250054\\
492	0.00631179759747023\\
493	0.00632854983207803\\
494	0.00634552089004974\\
495	0.00636271558416575\\
496	0.00638013850598619\\
497	0.00639779411750585\\
498	0.00641568685551596\\
499	0.00643382117631433\\
500	0.00645220155775853\\
501	0.00647083250442214\\
502	0.00648971855613801\\
503	0.00650886430004107\\
504	0.0065282743859806\\
505	0.00654795354486264\\
506	0.00656790660911695\\
507	0.00658813853410985\\
508	0.00660865441902504\\
509	0.00662945952567692\\
510	0.00665055929520593\\
511	0.00667195936502683\\
512	0.00669366558756179\\
513	0.00671568405080497\\
514	0.00673802110075266\\
515	0.00676068336572467\\
516	0.00678367778260441\\
517	0.00680701162504017\\
518	0.00683069253368399\\
519	0.00685472854859688\\
520	0.00687912814401729\\
521	0.00690390026575907\\
522	0.00692905437143166\\
523	0.00695460047353777\\
524	0.00698054918542238\\
525	0.00700691176997008\\
526	0.00703370019084742\\
527	0.00706092716595188\\
528	0.00708860622254237\\
529	0.00711675175328499\\
530	0.00714537907203752\\
531	0.00717450446771347\\
532	0.00720414525424005\\
533	0.00723431981403947\\
534	0.00726504763054851\\
535	0.00729634931502445\\
536	0.00732824661246005\\
537	0.00736076237665303\\
538	0.0073939205406289\\
539	0.00742774592141068\\
540	0.00746226382161746\\
541	0.007497499781614\\
542	0.00753347926021568\\
543	0.00757022206150404\\
544	0.00760775241840647\\
545	0.00764609839092532\\
546	0.00768528247896232\\
547	0.00772530972612136\\
548	0.00776618996457804\\
549	0.00780792246057549\\
550	0.00785049159263132\\
551	0.00789380041828882\\
552	0.00793786967130199\\
553	0.00798275837360501\\
554	0.00802854609916383\\
555	0.00807542371984085\\
556	0.00812345795096185\\
557	0.00817273046972689\\
558	0.00822337605108118\\
559	0.0082755419680626\\
560	0.00832801029844381\\
561	0.00837909318940157\\
562	0.00842813555899623\\
563	0.00847661969636791\\
564	0.00852511945157862\\
565	0.00857380264738345\\
566	0.00862269013955515\\
567	0.00867172225479524\\
568	0.00872080160460923\\
569	0.00876904162934258\\
570	0.00881606796714063\\
571	0.00886258479454568\\
572	0.00890894303025856\\
573	0.00895519463224402\\
574	0.00900097900874467\\
575	0.00904636842012797\\
576	0.00909180468302204\\
577	0.00913731409301758\\
578	0.00918286975492669\\
579	0.0092284285101118\\
580	0.00927394186312772\\
581	0.00931935765412942\\
582	0.00936462054353618\\
583	0.00940967226850926\\
584	0.00945445201388313\\
585	0.00949889694069117\\
586	0.00954294292450232\\
587	0.00958652555033089\\
588	0.00962958140293929\\
589	0.0096720496394373\\
590	0.0097138736812812\\
591	0.0097550024176802\\
592	0.00979538909575469\\
593	0.00983498278816331\\
594	0.00987369853931868\\
595	0.00991118387968948\\
596	0.00994658651044256\\
597	0.00997788999445116\\
598	0.010000292044645\\
599	0\\
600	0\\
};
\addplot [color=blue!25!mycolor7,solid,forget plot]
  table[row sep=crcr]{%
1	0.00409179167663971\\
2	0.00409179884488707\\
3	0.00409180611331869\\
4	0.00409181348334759\\
5	0.00409182095641101\\
6	0.00409182853397098\\
7	0.00409183621751489\\
8	0.00409184400855593\\
9	0.00409185190863384\\
10	0.00409185991931529\\
11	0.00409186804219466\\
12	0.00409187627889456\\
13	0.00409188463106638\\
14	0.00409189310039107\\
15	0.00409190168857962\\
16	0.00409191039737383\\
17	0.004091919228547\\
18	0.00409192818390445\\
19	0.00409193726528435\\
20	0.0040919464745584\\
21	0.00409195581363252\\
22	0.00409196528444763\\
23	0.00409197488898028\\
24	0.00409198462924354\\
25	0.00409199450728775\\
26	0.00409200452520124\\
27	0.0040920146851111\\
28	0.00409202498918411\\
29	0.00409203543962756\\
30	0.00409204603868992\\
31	0.00409205678866186\\
32	0.00409206769187707\\
33	0.0040920787507131\\
34	0.00409208996759235\\
35	0.00409210134498289\\
36	0.00409211288539939\\
37	0.00409212459140411\\
38	0.00409213646560782\\
39	0.00409214851067072\\
40	0.00409216072930356\\
41	0.00409217312426847\\
42	0.00409218569838013\\
43	0.00409219845450664\\
44	0.00409221139557072\\
45	0.0040922245245506\\
46	0.00409223784448126\\
47	0.00409225135845542\\
48	0.00409226506962462\\
49	0.00409227898120043\\
50	0.00409229309645551\\
51	0.00409230741872489\\
52	0.00409232195140688\\
53	0.00409233669796452\\
54	0.00409235166192667\\
55	0.00409236684688922\\
56	0.00409238225651635\\
57	0.00409239789454178\\
58	0.00409241376477004\\
59	0.00409242987107773\\
60	0.00409244621741486\\
61	0.00409246280780622\\
62	0.00409247964635263\\
63	0.00409249673723235\\
64	0.00409251408470245\\
65	0.00409253169310027\\
66	0.00409254956684473\\
67	0.00409256771043788\\
68	0.00409258612846634\\
69	0.00409260482560274\\
70	0.00409262380660726\\
71	0.00409264307632917\\
72	0.00409266263970835\\
73	0.00409268250177698\\
74	0.00409270266766095\\
75	0.0040927231425816\\
76	0.00409274393185739\\
77	0.00409276504090553\\
78	0.00409278647524373\\
79	0.00409280824049184\\
80	0.00409283034237376\\
81	0.00409285278671906\\
82	0.00409287557946492\\
83	0.00409289872665787\\
84	0.00409292223445579\\
85	0.00409294610912974\\
86	0.004092970357066\\
87	0.00409299498476787\\
88	0.00409301999885785\\
89	0.0040930454060796\\
90	0.00409307121330013\\
91	0.00409309742751178\\
92	0.0040931240558345\\
93	0.00409315110551802\\
94	0.00409317858394403\\
95	0.00409320649862865\\
96	0.00409323485722457\\
97	0.00409326366752356\\
98	0.00409329293745885\\
99	0.00409332267510763\\
100	0.00409335288869359\\
101	0.00409338358658947\\
102	0.00409341477731978\\
103	0.00409344646956346\\
104	0.00409347867215667\\
105	0.00409351139409552\\
106	0.00409354464453912\\
107	0.00409357843281241\\
108	0.00409361276840935\\
109	0.00409364766099587\\
110	0.00409368312041317\\
111	0.00409371915668098\\
112	0.00409375578000092\\
113	0.00409379300076002\\
114	0.00409383082953415\\
115	0.00409386927709188\\
116	0.00409390835439814\\
117	0.00409394807261812\\
118	0.00409398844312139\\
119	0.00409402947748602\\
120	0.00409407118750284\\
121	0.00409411358518003\\
122	0.00409415668274764\\
123	0.00409420049266249\\
124	0.00409424502761317\\
125	0.0040942903005252\\
126	0.00409433632456659\\
127	0.00409438311315353\\
128	0.00409443067995644\\
129	0.00409447903890625\\
130	0.00409452820420132\\
131	0.00409457819031449\\
132	0.00409462901200102\\
133	0.00409468068430702\\
134	0.00409473322257879\\
135	0.00409478664247332\\
136	0.00409484095997023\\
137	0.00409489619138556\\
138	0.00409495235338789\\
139	0.00409500946301823\\
140	0.00409506753771362\\
141	0.00409512659533744\\
142	0.00409518665421704\\
143	0.00409524773319237\\
144	0.00409530985167809\\
145	0.0040953730297432\\
146	0.00409543728820971\\
147	0.00409550264876417\\
148	0.00409556913404742\\
149	0.00409563676758475\\
150	0.00409570557304781\\
151	0.00409577557197914\\
152	0.00409584678632554\\
153	0.00409591923844574\\
154	0.00409599295111822\\
155	0.00409606794754915\\
156	0.00409614425138048\\
157	0.00409622188669833\\
158	0.00409630087804125\\
159	0.0040963812504089\\
160	0.00409646302927072\\
161	0.00409654624057492\\
162	0.00409663091075744\\
163	0.00409671706675124\\
164	0.00409680473599569\\
165	0.00409689394644617\\
166	0.00409698472658383\\
167	0.00409707710542542\\
168	0.00409717111253366\\
169	0.00409726677802718\\
170	0.00409736413259134\\
171	0.0040974632074888\\
172	0.00409756403457039\\
173	0.00409766664628621\\
174	0.00409777107569697\\
175	0.0040978773564854\\
176	0.00409798552296808\\
177	0.00409809561010729\\
178	0.00409820765352321\\
179	0.00409832168950623\\
180	0.00409843775502965\\
181	0.00409855588776244\\
182	0.00409867612608244\\
183	0.00409879850908959\\
184	0.00409892307661945\\
185	0.00409904986925723\\
186	0.00409917892835166\\
187	0.00409931029602955\\
188	0.0040994440152103\\
189	0.00409958012962086\\
190	0.00409971868381087\\
191	0.00409985972316823\\
192	0.00410000329393484\\
193	0.00410014944322269\\
194	0.00410029821903024\\
195	0.00410044967025921\\
196	0.00410060384673164\\
197	0.00410076079920708\\
198	0.00410092057940066\\
199	0.00410108324000084\\
200	0.00410124883468806\\
201	0.00410141741815344\\
202	0.00410158904611804\\
203	0.00410176377535236\\
204	0.00410194166369637\\
205	0.00410212277007982\\
206	0.00410230715454314\\
207	0.00410249487825852\\
208	0.00410268600355158\\
209	0.00410288059392352\\
210	0.0041030787140736\\
211	0.00410328042992225\\
212	0.00410348580863442\\
213	0.00410369491864363\\
214	0.00410390782967637\\
215	0.00410412461277718\\
216	0.00410434534033399\\
217	0.00410457008610425\\
218	0.00410479892524139\\
219	0.004105031934322\\
220	0.00410526919137341\\
221	0.004105510775902\\
222	0.00410575676892189\\
223	0.00410600725298448\\
224	0.00410626231220832\\
225	0.00410652203230979\\
226	0.00410678650063431\\
227	0.00410705580618822\\
228	0.00410733003967132\\
229	0.00410760929350995\\
230	0.00410789366189082\\
231	0.00410818324079571\\
232	0.00410847812803645\\
233	0.00410877842329085\\
234	0.00410908422813944\\
235	0.00410939564610263\\
236	0.00410971278267897\\
237	0.00411003574538369\\
238	0.00411036464378859\\
239	0.00411069958956213\\
240	0.00411104069651084\\
241	0.00411138808062097\\
242	0.00411174186010148\\
243	0.00411210215542755\\
244	0.00411246908938508\\
245	0.00411284278711603\\
246	0.00411322337616464\\
247	0.0041136109865246\\
248	0.00411400575068721\\
249	0.00411440780369023\\
250	0.0041148172831682\\
251	0.00411523432940332\\
252	0.00411565908537769\\
253	0.00411609169682644\\
254	0.00411653231229203\\
255	0.00411698108317972\\
256	0.00411743816381432\\
257	0.00411790371149797\\
258	0.00411837788656926\\
259	0.00411886085246391\\
260	0.0041193527757763\\
261	0.00411985382632294\\
262	0.0041203641772071\\
263	0.00412088400488505\\
264	0.00412141348923386\\
265	0.00412195281362084\\
266	0.00412250216497463\\
267	0.00412306173385831\\
268	0.00412363171454399\\
269	0.00412421230508999\\
270	0.00412480370741964\\
271	0.00412540612740305\\
272	0.00412601977494126\\
273	0.00412664486405411\\
274	0.00412728161297198\\
275	0.00412793024423284\\
276	0.00412859098478615\\
277	0.0041292640661058\\
278	0.00412994972431493\\
279	0.00413064820032278\\
280	0.00413135973994347\\
281	0.00413208459388185\\
282	0.00413282301783096\\
283	0.00413357527257914\\
284	0.00413434162411919\\
285	0.00413512234376086\\
286	0.00413591770824562\\
287	0.00413672799986431\\
288	0.00413755350657742\\
289	0.00413839452213839\\
290	0.00413925134621974\\
291	0.00414012428454199\\
292	0.00414101364900559\\
293	0.00414191975782595\\
294	0.00414284293567111\\
295	0.00414378351380293\\
296	0.00414474183022088\\
297	0.00414571822980925\\
298	0.00414671306448721\\
299	0.00414772669336217\\
300	0.0041487594828863\\
301	0.00414981180701613\\
302	0.00415088404737535\\
303	0.00415197659342082\\
304	0.00415308984261149\\
305	0.00415422420058033\\
306	0.00415538008130855\\
307	0.00415655790730176\\
308	0.0041577581097668\\
309	0.00415898112878767\\
310	0.00416022741349748\\
311	0.00416149742224266\\
312	0.00416279162273294\\
313	0.00416411049216903\\
314	0.00416545451733653\\
315	0.00416682419465783\\
316	0.00416822003021616\\
317	0.00416964253990388\\
318	0.004171092250146\\
319	0.0041725696983368\\
320	0.00417407543309548\\
321	0.00417561001452936\\
322	0.00417717401450484\\
323	0.0041787680169263\\
324	0.00418039261802365\\
325	0.00418204842664817\\
326	0.00418373606457772\\
327	0.00418545616683093\\
328	0.00418720938199121\\
329	0.00418899637254059\\
330	0.00419081781520407\\
331	0.00419267440130464\\
332	0.0041945668371294\\
333	0.00419649584430726\\
334	0.00419846216019875\\
335	0.00420046653829827\\
336	0.00420250974864935\\
337	0.00420459257827343\\
338	0.00420671583161246\\
339	0.00420888033098623\\
340	0.00421108691706498\\
341	0.0042133364493579\\
342	0.0042156298067184\\
343	0.00421796788786684\\
344	0.00422035161193176\\
345	0.00422278191901008\\
346	0.00422525977074702\\
347	0.00422778615093572\\
348	0.00423036206614031\\
349	0.00423298854634276\\
350	0.00423566664561411\\
351	0.00423839744281184\\
352	0.00424118204230375\\
353	0.00424402157471958\\
354	0.00424691719773253\\
355	0.00424987009687203\\
356	0.00425288148636925\\
357	0.00425595261003717\\
358	0.00425908474218753\\
359	0.0042622791885842\\
360	0.004265537287431\\
361	0.00426886041040458\\
362	0.00427224996373771\\
363	0.00427570738934931\\
364	0.00427923416602433\\
365	0.00428283181064584\\
366	0.00428650187948255\\
367	0.00429024596953427\\
368	0.00429406571993667\\
369	0.00429796281343218\\
370	0.00430193897790976\\
371	0.00430599598801906\\
372	0.0043101356668646\\
373	0.00431435988777598\\
374	0.00431867057611981\\
375	0.0043230697112264\\
376	0.00432755932840648\\
377	0.00433214152107077\\
378	0.00433681844298267\\
379	0.00434159231061924\\
380	0.00434646540565764\\
381	0.00435144007758906\\
382	0.00435651874643855\\
383	0.00436170390561014\\
384	0.00436699812487807\\
385	0.00437240405353485\\
386	0.00437792442370775\\
387	0.00438356205383751\\
388	0.00438931985234851\\
389	0.00439520082153542\\
390	0.00440120806168268\\
391	0.00440734477539559\\
392	0.00441361427201974\\
393	0.00442001997239376\\
394	0.00442656541389017\\
395	0.00443325425600352\\
396	0.00444009028617354\\
397	0.00444707742590148\\
398	0.0044542197372455\\
399	0.00446152142965565\\
400	0.00446898686715969\\
401	0.00447662057591346\\
402	0.00448442725215078\\
403	0.00449241177061241\\
404	0.00450057919329857\\
405	0.00450893477873799\\
406	0.00451748399207861\\
407	0.00452623251381179\\
408	0.00453518624866284\\
409	0.00454435133425066\\
410	0.00455373414953379\\
411	0.00456334132928628\\
412	0.00457317977488908\\
413	0.00458325666578606\\
414	0.00459357947365316\\
415	0.00460415597434657\\
416	0.00461499426294936\\
417	0.00462610277355903\\
418	0.00463749030177741\\
419	0.00464916603393684\\
420	0.00466113958139602\\
421	0.00467342101085839\\
422	0.0046860208531977\\
423	0.00469895012246663\\
424	0.00471222034629779\\
425	0.00472584364112071\\
426	0.00473983275497101\\
427	0.00475420110916848\\
428	0.00476896283486819\\
429	0.00478413279786332\\
430	0.00479972660194332\\
431	0.0048157605760717\\
432	0.00483225164319992\\
433	0.00484921720001487\\
434	0.00486667521852141\\
435	0.00488464392529949\\
436	0.0049031412572039\\
437	0.00492218498644744\\
438	0.00494179291144521\\
439	0.00496198283679481\\
440	0.00498277260286462\\
441	0.00500418019598847\\
442	0.00502622398219888\\
443	0.00504892311777252\\
444	0.00507229816531374\\
445	0.00509637169284442\\
446	0.00512117354920067\\
447	0.00514674900909566\\
448	0.00517315033241187\\
449	0.00520042753936902\\
450	0.00522863063663961\\
451	0.00525780818888461\\
452	0.00528800531821086\\
453	0.00531926094482009\\
454	0.00535160402406627\\
455	0.00538504845803528\\
456	0.00541958625039071\\
457	0.00545517834741904\\
458	0.00549174252478596\\
459	0.00552913740957662\\
460	0.00556714135810725\\
461	0.005604787298736\\
462	0.0056416798000527\\
463	0.00567770459455718\\
464	0.00571274007328253\\
465	0.00574665857487867\\
466	0.00577932847560934\\
467	0.0058106174084599\\
468	0.00584039718465546\\
469	0.0058685510370182\\
470	0.00589498376403355\\
471	0.00591963580200271\\
472	0.00594250260714197\\
473	0.00596366113022463\\
474	0.00598397845686839\\
475	0.0060038223638901\\
476	0.00602318874475772\\
477	0.00604208086658422\\
478	0.00606051065279006\\
479	0.00607849994106377\\
480	0.00609608160931803\\
481	0.00611330040582075\\
482	0.00613021324010889\\
483	0.00614688858232525\\
484	0.00616340446907811\\
485	0.00617984441050601\\
486	0.00619628223156762\\
487	0.00621274872484579\\
488	0.00622925904839289\\
489	0.00624582983842187\\
490	0.006262478959408\\
491	0.00627922515411315\\
492	0.00629608758567791\\
493	0.00631308527169584\\
494	0.00633023642363993\\
495	0.00634755772678505\\
496	0.00636506362978485\\
497	0.00638276576432228\\
498	0.00640067291806857\\
499	0.00641879282478134\\
500	0.00643713319965338\\
501	0.00645570166760095\\
502	0.00647450569462391\\
503	0.00649355252754852\\
504	0.00651284914856596\\
505	0.00653240225181333\\
506	0.00655221824944383\\
507	0.00657230331361228\\
508	0.00659266345765685\\
509	0.00661330465311988\\
510	0.00663423294085632\\
511	0.00665545447750711\\
512	0.00667697554746031\\
513	0.00669880257938798\\
514	0.00672094216772658\\
515	0.00674340109923942\\
516	0.00676618638448121\\
517	0.00678930529357986\\
518	0.00681276539528721\\
519	0.00683657459779465\\
520	0.00686074118949649\\
521	0.00688527387795266\\
522	0.00691018182879633\\
523	0.00693547470735929\\
524	0.00696116272365154\\
525	0.0069872566806296\\
526	0.00701376802560303\\
527	0.00704070890452965\\
528	0.00706809221883977\\
529	0.00709593168430487\\
530	0.00712424189132716\\
531	0.00715303836586031\\
532	0.00718233762994827\\
533	0.00721215726049923\\
534	0.00724251594426931\\
535	0.00727343352600266\\
536	0.00730493104605926\\
537	0.00733703076285003\\
538	0.00736975615367599\\
539	0.00740313188943341\\
540	0.00743718377564411\\
541	0.00747193864038859\\
542	0.00750742415076012\\
543	0.00754366859468992\\
544	0.00758070031442702\\
545	0.00761854702852742\\
546	0.00765723521324297\\
547	0.00769678967550589\\
548	0.00773722468672789\\
549	0.00777854347761477\\
550	0.00782075245913859\\
551	0.00786385640142318\\
552	0.00790785477206057\\
553	0.0079527453250826\\
554	0.00799851308237961\\
555	0.00804505598680104\\
556	0.00809242117741512\\
557	0.00814067103610812\\
558	0.00818989521339174\\
559	0.00824028601165865\\
560	0.00829193340389175\\
561	0.00834496567632439\\
562	0.00839955265381671\\
563	0.00845374671421613\\
564	0.00850651646442516\\
565	0.00855718422615864\\
566	0.00860737522808454\\
567	0.00865747862960277\\
568	0.0087075839716849\\
569	0.00875773555704555\\
570	0.00880783885424744\\
571	0.00885708033400435\\
572	0.0089050682940259\\
573	0.00895224789310436\\
574	0.00899912084127209\\
575	0.00904569281043743\\
576	0.00909165223986215\\
577	0.00913727099022529\\
578	0.00918285607127283\\
579	0.00922842280592135\\
580	0.00927393897059496\\
581	0.00931935612056587\\
582	0.00936461969699102\\
583	0.00940967179489983\\
584	0.00945445175374296\\
585	0.00949889680731318\\
586	0.00954294286465648\\
587	0.00958652552981098\\
588	0.00962958139833628\\
589	0.00967204963904421\\
590	0.0097138736812812\\
591	0.00975500241768019\\
592	0.00979538909575469\\
593	0.00983498278816331\\
594	0.00987369853931868\\
595	0.00991118387968948\\
596	0.00994658651044256\\
597	0.00997788999445116\\
598	0.010000292044645\\
599	0\\
600	0\\
};
\addplot [color=mycolor9,solid,forget plot]
  table[row sep=crcr]{%
1	0.00408032568010838\\
2	0.0040803317214586\\
3	0.00408033785212888\\
4	0.00408034407351985\\
5	0.00408035038705741\\
6	0.00408035679419308\\
7	0.00408036329640468\\
8	0.00408036989519683\\
9	0.00408037659210129\\
10	0.00408038338867769\\
11	0.00408039028651398\\
12	0.00408039728722694\\
13	0.00408040439246285\\
14	0.00408041160389798\\
15	0.00408041892323913\\
16	0.00408042635222434\\
17	0.00408043389262328\\
18	0.00408044154623809\\
19	0.00408044931490382\\
20	0.00408045720048911\\
21	0.00408046520489688\\
22	0.00408047333006484\\
23	0.00408048157796627\\
24	0.00408048995061064\\
25	0.00408049845004424\\
26	0.00408050707835092\\
27	0.00408051583765277\\
28	0.00408052473011081\\
29	0.00408053375792565\\
30	0.00408054292333841\\
31	0.00408055222863123\\
32	0.00408056167612811\\
33	0.00408057126819572\\
34	0.00408058100724404\\
35	0.00408059089572735\\
36	0.00408060093614478\\
37	0.00408061113104128\\
38	0.0040806214830084\\
39	0.00408063199468506\\
40	0.00408064266875844\\
41	0.00408065350796486\\
42	0.00408066451509057\\
43	0.00408067569297266\\
44	0.00408068704449997\\
45	0.00408069857261397\\
46	0.00408071028030962\\
47	0.00408072217063644\\
48	0.00408073424669928\\
49	0.0040807465116594\\
50	0.00408075896873532\\
51	0.00408077162120389\\
52	0.00408078447240126\\
53	0.00408079752572385\\
54	0.00408081078462939\\
55	0.00408082425263794\\
56	0.00408083793333293\\
57	0.00408085183036224\\
58	0.00408086594743926\\
59	0.00408088028834398\\
60	0.00408089485692403\\
61	0.00408090965709584\\
62	0.00408092469284576\\
63	0.00408093996823122\\
64	0.00408095548738186\\
65	0.00408097125450065\\
66	0.00408098727386516\\
67	0.00408100354982865\\
68	0.00408102008682132\\
69	0.00408103688935157\\
70	0.00408105396200713\\
71	0.00408107130945636\\
72	0.00408108893644952\\
73	0.00408110684781996\\
74	0.00408112504848546\\
75	0.00408114354344953\\
76	0.0040811623378026\\
77	0.00408118143672343\\
78	0.0040812008454803\\
79	0.0040812205694325\\
80	0.00408124061403147\\
81	0.00408126098482225\\
82	0.00408128168744471\\
83	0.00408130272763498\\
84	0.00408132411122675\\
85	0.00408134584415252\\
86	0.00408136793244503\\
87	0.00408139038223856\\
88	0.00408141319977018\\
89	0.00408143639138112\\
90	0.00408145996351798\\
91	0.00408148392273409\\
92	0.00408150827569066\\
93	0.00408153302915806\\
94	0.00408155819001696\\
95	0.00408158376525952\\
96	0.00408160976199053\\
97	0.00408163618742834\\
98	0.00408166304890601\\
99	0.00408169035387211\\
100	0.00408171810989168\\
101	0.00408174632464702\\
102	0.00408177500593827\\
103	0.0040818041616841\\
104	0.00408183379992208\\
105	0.00408186392880907\\
106	0.00408189455662128\\
107	0.00408192569175447\\
108	0.00408195734272342\\
109	0.00408198951816175\\
110	0.00408202222682114\\
111	0.00408205547757034\\
112	0.00408208927939387\\
113	0.00408212364139036\\
114	0.00408215857277041\\
115	0.00408219408285411\\
116	0.00408223018106779\\
117	0.00408226687694039\\
118	0.0040823041800988\\
119	0.00408234210026281\\
120	0.00408238064723854\\
121	0.00408241983091128\\
122	0.00408245966123659\\
123	0.00408250014823023\\
124	0.00408254130195599\\
125	0.00408258313251164\\
126	0.00408262565001214\\
127	0.00408266886457015\\
128	0.00408271278627252\\
129	0.0040827574251529\\
130	0.00408280279115896\\
131	0.00408284889411325\\
132	0.00408289574366644\\
133	0.00408294334924135\\
134	0.00408299171996512\\
135	0.00408304086458768\\
136	0.00408309079138236\\
137	0.00408314150802506\\
138	0.00408319302144679\\
139	0.00408324533765281\\
140	0.00408329846150164\\
141	0.00408335239643531\\
142	0.00408340714415597\\
143	0.00408346270425434\\
144	0.00408351907383572\\
145	0.00408357624731534\\
146	0.00408363421695109\\
147	0.00408369297595483\\
148	0.00408375253033698\\
149	0.00408381294198024\\
150	0.00408387442250217\\
151	0.00408393699150774\\
152	0.0040840006689656\\
153	0.00408406547521487\\
154	0.004084131430972\\
155	0.0040841985573378\\
156	0.00408426687580454\\
157	0.00408433640826325\\
158	0.00408440717701115\\
159	0.0040844792047591\\
160	0.00408455251463939\\
161	0.00408462713021347\\
162	0.00408470307548001\\
163	0.00408478037488287\\
164	0.00408485905331948\\
165	0.00408493913614922\\
166	0.00408502064920193\\
167	0.00408510361878671\\
168	0.00408518807170073\\
169	0.00408527403523841\\
170	0.00408536153720045\\
171	0.00408545060590334\\
172	0.00408554127018892\\
173	0.00408563355943405\\
174	0.00408572750356054\\
175	0.00408582313304534\\
176	0.0040859204789307\\
177	0.00408601957283474\\
178	0.00408612044696201\\
179	0.00408622313411457\\
180	0.0040863276677028\\
181	0.00408643408175689\\
182	0.00408654241093819\\
183	0.00408665269055101\\
184	0.00408676495655458\\
185	0.00408687924557497\\
186	0.00408699559491781\\
187	0.00408711404258058\\
188	0.00408723462726558\\
189	0.00408735738839309\\
190	0.00408748236611461\\
191	0.00408760960132647\\
192	0.00408773913568356\\
193	0.00408787101161365\\
194	0.00408800527233164\\
195	0.00408814196185414\\
196	0.00408828112501439\\
197	0.00408842280747765\\
198	0.00408856705575633\\
199	0.00408871391722615\\
200	0.0040888634401419\\
201	0.00408901567365398\\
202	0.00408917066782506\\
203	0.00408932847364704\\
204	0.00408948914305838\\
205	0.00408965272896175\\
206	0.00408981928524199\\
207	0.00408998886678448\\
208	0.0040901615294938\\
209	0.00409033733031263\\
210	0.00409051632724121\\
211	0.00409069857935697\\
212	0.00409088414683478\\
213	0.00409107309096724\\
214	0.00409126547418565\\
215	0.00409146136008117\\
216	0.00409166081342647\\
217	0.0040918639001977\\
218	0.00409207068759704\\
219	0.0040922812440754\\
220	0.00409249563935562\\
221	0.00409271394445633\\
222	0.00409293623171591\\
223	0.00409316257481705\\
224	0.00409339304881169\\
225	0.00409362773014645\\
226	0.00409386669668852\\
227	0.0040941100277519\\
228	0.0040943578041243\\
229	0.00409461010809421\\
230	0.00409486702347883\\
231	0.00409512863565205\\
232	0.00409539503157325\\
233	0.00409566629981644\\
234	0.00409594253059986\\
235	0.00409622381581618\\
236	0.00409651024906302\\
237	0.00409680192567429\\
238	0.00409709894275164\\
239	0.00409740139919666\\
240	0.00409770939574349\\
241	0.00409802303499203\\
242	0.00409834242144152\\
243	0.00409866766152481\\
244	0.00409899886364285\\
245	0.00409933613819994\\
246	0.00409967959763939\\
247	0.00410002935647955\\
248	0.00410038553135043\\
249	0.00410074824103093\\
250	0.00410111760648614\\
251	0.00410149375090549\\
252	0.00410187679974104\\
253	0.00410226688074624\\
254	0.00410266412401525\\
255	0.00410306866202235\\
256	0.00410348062966169\\
257	0.00410390016428735\\
258	0.00410432740575361\\
259	0.0041047624964549\\
260	0.00410520558136627\\
261	0.00410565680808287\\
262	0.00410611632685941\\
263	0.0041065842906483\\
264	0.00410706085513649\\
265	0.00410754617878015\\
266	0.00410804042283614\\
267	0.00410854375138903\\
268	0.00410905633137184\\
269	0.00410957833257757\\
270	0.00411010992765883\\
271	0.00411065129211007\\
272	0.00411120260422655\\
273	0.00411176404503058\\
274	0.00411233579815451\\
275	0.00411291804966501\\
276	0.00411351098781273\\
277	0.0041141148026981\\
278	0.00411472968589211\\
279	0.00411535583028356\\
280	0.00411599343159979\\
281	0.00411664269946865\\
282	0.0041173038480168\\
283	0.00411797709534109\\
284	0.00411866266359079\\
285	0.00411936077905226\\
286	0.0041200716722368\\
287	0.00412079557797152\\
288	0.00412153273549352\\
289	0.0041222833885479\\
290	0.00412304778548947\\
291	0.00412382617938882\\
292	0.00412461882814302\\
293	0.00412542599459136\\
294	0.004126247946637\\
295	0.00412708495737457\\
296	0.00412793730522538\\
297	0.0041288052740808\\
298	0.00412968915345553\\
299	0.00413058923865261\\
300	0.0041315058309427\\
301	0.00413243923776124\\
302	0.00413338977292813\\
303	0.00413435775689608\\
304	0.00413534351703653\\
305	0.00413634738797513\\
306	0.00413736971199214\\
307	0.00413841083951022\\
308	0.00413947112969897\\
309	0.00414055095123541\\
310	0.00414165068327278\\
311	0.00414277071668319\\
312	0.00414391145564883\\
313	0.00414507331965988\\
314	0.0041462567458689\\
315	0.00414746219133093\\
316	0.00414869013318215\\
317	0.00414994105951313\\
318	0.00415121542405593\\
319	0.00415251367288903\\
320	0.00415383626087322\\
321	0.00415518365184186\\
322	0.00415655631879631\\
323	0.00415795474410643\\
324	0.00415937941971667\\
325	0.00416083084735821\\
326	0.00416230953876658\\
327	0.0041638160159058\\
328	0.0041653508111988\\
329	0.00416691446776466\\
330	0.00416850753966274\\
331	0.00417013059214414\\
332	0.00417178420191086\\
333	0.00417346895738277\\
334	0.00417518545897301\\
335	0.00417693431937199\\
336	0.0041787161638408\\
337	0.00418053163051362\\
338	0.00418238137071067\\
339	0.0041842660492613\\
340	0.00418618634483813\\
341	0.00418814295030285\\
342	0.00419013657306392\\
343	0.00419216793544708\\
344	0.00419423777507914\\
345	0.00419634684528539\\
346	0.00419849591550199\\
347	0.00420068577170344\\
348	0.0042029172168468\\
349	0.00420519107133266\\
350	0.00420750817348505\\
351	0.00420986938005095\\
352	0.00421227556672175\\
353	0.00421472762867913\\
354	0.00421722648116924\\
355	0.00421977306011006\\
356	0.00422236832273927\\
357	0.00422501324831298\\
358	0.00422770883886822\\
359	0.00423045612005951\\
360	0.00423325614202024\\
361	0.00423610997994662\\
362	0.00423901873440536\\
363	0.00424198353210752\\
364	0.00424500552671295\\
365	0.00424808589966584\\
366	0.00425122586106436\\
367	0.00425442665056831\\
368	0.00425768953834222\\
369	0.00426101582603336\\
370	0.0042644068477877\\
371	0.00426786397130313\\
372	0.004271388598917\\
373	0.00427498216872125\\
374	0.00427864615570287\\
375	0.00428238207289374\\
376	0.00428619147251039\\
377	0.00429007594705438\\
378	0.00429403713033155\\
379	0.00429807669835758\\
380	0.00430219637020581\\
381	0.00430639790936694\\
382	0.00431068312732309\\
383	0.00431505388634855\\
384	0.00431951210153059\\
385	0.00432405974289698\\
386	0.0043286988376544\\
387	0.00433343147254496\\
388	0.00433825979632698\\
389	0.00434318602238383\\
390	0.00434821243146024\\
391	0.00435334137452092\\
392	0.00435857527575316\\
393	0.00436391663571122\\
394	0.00436936803461629\\
395	0.00437493213578097\\
396	0.00438061168915676\\
397	0.0043864095350035\\
398	0.00439232860766343\\
399	0.00439837193942105\\
400	0.00440454266442583\\
401	0.00441084402264989\\
402	0.0044172793638488\\
403	0.00442385215147291\\
404	0.00443056596649915\\
405	0.00443742451113101\\
406	0.00444443161216516\\
407	0.00445159122412042\\
408	0.00445890743212007\\
409	0.00446638445468958\\
410	0.00447402664703967\\
411	0.00448183850433959\\
412	0.004489824665813\\
413	0.00449798992021728\\
414	0.00450633921284684\\
415	0.00451487766131778\\
416	0.00452361057509217\\
417	0.00453254346895203\\
418	0.00454168207405567\\
419	0.00455103234957194\\
420	0.00456060049389352\\
421	0.00457039295449526\\
422	0.00458041643914397\\
423	0.00459067792893476\\
424	0.00460118469484066\\
425	0.00461194431180844\\
426	0.00462296467320176\\
427	0.00463425400557573\\
428	0.00464582088389187\\
429	0.00465767424757893\\
430	0.00466982341915739\\
431	0.00468227812477799\\
432	0.0046950485232572\\
433	0.00470814523954862\\
434	0.00472157936550173\\
435	0.00473536245780382\\
436	0.00474950659521184\\
437	0.0047640244387787\\
438	0.00477892927470935\\
439	0.00479423506031151\\
440	0.00480995647212761\\
441	0.00482610895435044\\
442	0.00484270876391218\\
443	0.00485977300663505\\
444	0.00487731967260649\\
445	0.00489536790457525\\
446	0.00491393775780624\\
447	0.00493304950806499\\
448	0.00495272349413835\\
449	0.00497298037922986\\
450	0.00499384109334846\\
451	0.00501532678326504\\
452	0.00503745878020474\\
453	0.00506025859515393\\
454	0.0050837479230485\\
455	0.00510794836847699\\
456	0.0051328848498848\\
457	0.00515858542537479\\
458	0.00518507829680298\\
459	0.0052123916363027\\
460	0.00524055774871919\\
461	0.00526962519665851\\
462	0.0052996483229499\\
463	0.00533068136551119\\
464	0.00536277687710101\\
465	0.00539598350622364\\
466	0.00543034292173908\\
467	0.00546588564894863\\
468	0.00550262551694845\\
469	0.00554055226288177\\
470	0.00557962158440338\\
471	0.00561974192124584\\
472	0.00566075695097854\\
473	0.00570242247316536\\
474	0.00574371723299744\\
475	0.00578413666240843\\
476	0.0058235514348476\\
477	0.00586182452730329\\
478	0.00589881302973534\\
479	0.00593437122793735\\
480	0.00596835508548402\\
481	0.0060006288329282\\
482	0.00603107436035096\\
483	0.00605960441782868\\
484	0.00608618090950803\\
485	0.0061108399790797\\
486	0.00613394827269195\\
487	0.0061565572231749\\
488	0.00617866106963987\\
489	0.00620026206261332\\
490	0.00622137190575537\\
491	0.00624201318766271\\
492	0.00626222068822813\\
493	0.00628204238213333\\
494	0.00630153987527973\\
495	0.00632078789161669\\
496	0.00633987226080195\\
497	0.00635888563866321\\
498	0.00637791389053429\\
499	0.00639699321982514\\
500	0.00641614092023951\\
501	0.00643537595941995\\
502	0.00645471868676816\\
503	0.00647419042522651\\
504	0.00649381293908639\\
505	0.00651360777983297\\
506	0.00653359552902565\\
507	0.00655379498480331\\
508	0.00657422238156807\\
509	0.00659489079711179\\
510	0.00661581066893441\\
511	0.00663699164239693\\
512	0.00665844341738469\\
513	0.00668017567478128\\
514	0.00670219800864469\\
515	0.00672451987082082\\
516	0.00674715053602441\\
517	0.00677009909628874\\
518	0.0067933744936005\\
519	0.00681698559766519\\
520	0.00684094133084513\\
521	0.00686525083245867\\
522	0.00688992357169755\\
523	0.006914969393402\\
524	0.00694039855405188\\
525	0.00696622176460758\\
526	0.00699245024049316\\
527	0.00701909575863972\\
528	0.00704617072100075\\
529	0.00707368822332517\\
530	0.00710166212726932\\
531	0.00713010713325534\\
532	0.00715903885104884\\
533	0.00718847386631718\\
534	0.00721842980511202\\
535	0.00724892539692609\\
536	0.00727998053398537\\
537	0.00731161632367419\\
538	0.00734385513004147\\
539	0.00737672059906469\\
540	0.00741023766083625\\
541	0.00744443250022344\\
542	0.00747933248540618\\
543	0.00751496604033652\\
544	0.00755136244984742\\
545	0.00758855157424061\\
546	0.00762656343832719\\
547	0.00766542764552213\\
548	0.00770517274565856\\
549	0.00774582545318689\\
550	0.00778739842297575\\
551	0.00782990245782332\\
552	0.00787334632801186\\
553	0.00791773109971343\\
554	0.00796305422576511\\
555	0.00800931980616587\\
556	0.00805652647189953\\
557	0.0081046653681536\\
558	0.00815370949778185\\
559	0.00820357552247344\\
560	0.00825431025459932\\
561	0.00830597583091507\\
562	0.00835865057764933\\
563	0.00841255528549486\\
564	0.00846782554576636\\
565	0.00852461550995907\\
566	0.00858084970818208\\
567	0.00863568580905793\\
568	0.00868850380505972\\
569	0.00874038008329295\\
570	0.00879204607992877\\
571	0.00884343635142392\\
572	0.00889467572782135\\
573	0.00894516958355532\\
574	0.00899434902706773\\
575	0.00904222600225824\\
576	0.00908965295883001\\
577	0.00913649196733126\\
578	0.00918262170103313\\
579	0.00922833831212848\\
580	0.00927390407257964\\
581	0.00931933850652799\\
582	0.00936461041460092\\
583	0.00940966667844402\\
584	0.00945444888056296\\
585	0.00949889521326668\\
586	0.00954294203465611\\
587	0.00958652514879544\\
588	0.00962958126464585\\
589	0.00967204960829027\\
590	0.00971387367861981\\
591	0.0097550024176802\\
592	0.00979538909575469\\
593	0.00983498278816331\\
594	0.00987369853931868\\
595	0.00991118387968948\\
596	0.00994658651044256\\
597	0.00997788999445116\\
598	0.010000292044645\\
599	0\\
600	0\\
};
\addplot [color=blue!50!mycolor7,solid,forget plot]
  table[row sep=crcr]{%
1	0.00402964426831132\\
2	0.00402965034379505\\
3	0.00402965651562647\\
4	0.00402966278541615\\
5	0.0040296691548033\\
6	0.00402967562545648\\
7	0.00402968219907398\\
8	0.0040296888773844\\
9	0.00402969566214725\\
10	0.00402970255515347\\
11	0.00402970955822588\\
12	0.00402971667321998\\
13	0.00402972390202436\\
14	0.00402973124656131\\
15	0.00402973870878748\\
16	0.00402974629069442\\
17	0.00402975399430933\\
18	0.00402976182169553\\
19	0.00402976977495321\\
20	0.00402977785622004\\
21	0.00402978606767185\\
22	0.00402979441152328\\
23	0.00402980289002851\\
24	0.00402981150548188\\
25	0.00402982026021867\\
26	0.00402982915661576\\
27	0.00402983819709248\\
28	0.00402984738411119\\
29	0.00402985672017811\\
30	0.00402986620784412\\
31	0.00402987584970546\\
32	0.00402988564840465\\
33	0.0040298956066311\\
34	0.00402990572712215\\
35	0.0040299160126637\\
36	0.00402992646609121\\
37	0.00402993709029046\\
38	0.00402994788819844\\
39	0.00402995886280432\\
40	0.00402997001715026\\
41	0.0040299813543323\\
42	0.00402999287750143\\
43	0.00403000458986439\\
44	0.00403001649468467\\
45	0.00403002859528357\\
46	0.00403004089504111\\
47	0.00403005339739706\\
48	0.00403006610585196\\
49	0.00403007902396819\\
50	0.00403009215537106\\
51	0.00403010550374975\\
52	0.00403011907285858\\
53	0.00403013286651801\\
54	0.00403014688861583\\
55	0.00403016114310827\\
56	0.00403017563402121\\
57	0.00403019036545137\\
58	0.00403020534156747\\
59	0.00403022056661152\\
60	0.00403023604490003\\
61	0.00403025178082535\\
62	0.00403026777885687\\
63	0.00403028404354242\\
64	0.00403030057950953\\
65	0.00403031739146684\\
66	0.0040303344842055\\
67	0.00403035186260053\\
68	0.00403036953161226\\
69	0.00403038749628784\\
70	0.00403040576176262\\
71	0.00403042433326177\\
72	0.00403044321610175\\
73	0.00403046241569182\\
74	0.00403048193753568\\
75	0.0040305017872331\\
76	0.00403052197048147\\
77	0.0040305424930776\\
78	0.00403056336091931\\
79	0.00403058458000714\\
80	0.00403060615644616\\
81	0.00403062809644772\\
82	0.00403065040633129\\
83	0.0040306730925263\\
84	0.00403069616157396\\
85	0.00403071962012924\\
86	0.00403074347496271\\
87	0.00403076773296263\\
88	0.00403079240113689\\
89	0.00403081748661507\\
90	0.00403084299665048\\
91	0.00403086893862231\\
92	0.00403089532003773\\
93	0.00403092214853418\\
94	0.00403094943188149\\
95	0.00403097717798414\\
96	0.00403100539488353\\
97	0.00403103409076045\\
98	0.00403106327393735\\
99	0.00403109295288078\\
100	0.00403112313620386\\
101	0.00403115383266873\\
102	0.00403118505118914\\
103	0.00403121680083304\\
104	0.00403124909082514\\
105	0.00403128193054971\\
106	0.00403131532955316\\
107	0.00403134929754687\\
108	0.00403138384440994\\
109	0.00403141898019218\\
110	0.00403145471511682\\
111	0.00403149105958357\\
112	0.0040315280241716\\
113	0.00403156561964252\\
114	0.00403160385694349\\
115	0.0040316427472103\\
116	0.00403168230177054\\
117	0.00403172253214682\\
118	0.00403176345005995\\
119	0.00403180506743214\\
120	0.00403184739639037\\
121	0.00403189044926957\\
122	0.00403193423861598\\
123	0.00403197877719031\\
124	0.00403202407797098\\
125	0.00403207015415745\\
126	0.00403211701917317\\
127	0.00403216468666858\\
128	0.00403221317052421\\
129	0.00403226248485323\\
130	0.00403231264400393\\
131	0.00403236366256223\\
132	0.00403241555535348\\
133	0.00403246833744424\\
134	0.00403252202414366\\
135	0.00403257663100445\\
136	0.00403263217382357\\
137	0.00403268866864269\\
138	0.00403274613174887\\
139	0.00403280457967544\\
140	0.00403286402920493\\
141	0.00403292449737621\\
142	0.00403298600150241\\
143	0.00403304855921572\\
144	0.00403311218858156\\
145	0.00403317690839188\\
146	0.00403324273891722\\
147	0.00403330970375947\\
148	0.0040333778337388\\
149	0.004033447168939\\
150	0.00403351773105495\\
151	0.00403358954217357\\
152	0.00403366262478084\\
153	0.00403373700176913\\
154	0.0040338126964444\\
155	0.00403388973253367\\
156	0.0040339681341927\\
157	0.00403404792601353\\
158	0.00403412913303239\\
159	0.00403421178073775\\
160	0.00403429589507838\\
161	0.0040343815024716\\
162	0.00403446862981177\\
163	0.0040345573044788\\
164	0.00403464755434689\\
165	0.00403473940779341\\
166	0.0040348328937079\\
167	0.0040349280415013\\
168	0.00403502488111525\\
169	0.00403512344303169\\
170	0.00403522375828242\\
171	0.0040353258584591\\
172	0.0040354297757232\\
173	0.00403553554281623\\
174	0.00403564319307008\\
175	0.00403575276041768\\
176	0.00403586427940372\\
177	0.00403597778519553\\
178	0.00403609331359445\\
179	0.00403621090104681\\
180	0.00403633058465587\\
181	0.00403645240219323\\
182	0.00403657639211106\\
183	0.00403670259355399\\
184	0.0040368310463717\\
185	0.00403696179113145\\
186	0.00403709486913082\\
187	0.00403723032241092\\
188	0.00403736819376953\\
189	0.00403750852677466\\
190	0.00403765136577822\\
191	0.00403779675593019\\
192	0.00403794474319268\\
193	0.00403809537435453\\
194	0.00403824869704588\\
195	0.00403840475975344\\
196	0.00403856361183554\\
197	0.00403872530353779\\
198	0.00403888988600878\\
199	0.00403905741131628\\
200	0.00403922793246357\\
201	0.00403940150340611\\
202	0.00403957817906847\\
203	0.00403975801536169\\
204	0.00403994106920072\\
205	0.00404012739852243\\
206	0.00404031706230372\\
207	0.00404051012058006\\
208	0.00404070663446437\\
209	0.00404090666616626\\
210	0.00404111027901159\\
211	0.00404131753746227\\
212	0.00404152850713655\\
213	0.00404174325482977\\
214	0.00404196184853525\\
215	0.00404218435746575\\
216	0.00404241085207537\\
217	0.00404264140408154\\
218	0.0040428760864879\\
219	0.00404311497360726\\
220	0.00404335814108511\\
221	0.00404360566592364\\
222	0.00404385762650622\\
223	0.00404411410262221\\
224	0.00404437517549263\\
225	0.00404464092779601\\
226	0.00404491144369495\\
227	0.00404518680886314\\
228	0.00404546711051317\\
229	0.00404575243742481\\
230	0.00404604287997398\\
231	0.00404633853016232\\
232	0.00404663948164759\\
233	0.00404694582977477\\
234	0.00404725767160795\\
235	0.00404757510596315\\
236	0.00404789823344184\\
237	0.00404822715646565\\
238	0.00404856197931195\\
239	0.00404890280815061\\
240	0.00404924975108192\\
241	0.00404960291817575\\
242	0.0040499624215123\\
243	0.00405032837522407\\
244	0.00405070089553996\\
245	0.00405108010083074\\
246	0.00405146611165699\\
247	0.00405185905081918\\
248	0.00405225904341034\\
249	0.00405266621687171\\
250	0.00405308070105178\\
251	0.00405350262826892\\
252	0.00405393213337873\\
253	0.00405436935384623\\
254	0.004054814429824\\
255	0.00405526750423721\\
256	0.00405572872287694\\
257	0.00405619823450304\\
258	0.00405667619095848\\
259	0.00405716274729766\\
260	0.00405765806193113\\
261	0.00405816229679059\\
262	0.00405867561751827\\
263	0.00405919819368625\\
264	0.00405973019905237\\
265	0.00406027181186149\\
266	0.0040608232152023\\
267	0.00406138459743367\\
268	0.00406195615269679\\
269	0.0040625380815348\\
270	0.00406313059164622\\
271	0.00406373389880394\\
272	0.00406434822797799\\
273	0.00406497381469557\\
274	0.00406561090663889\\
275	0.00406625976532493\\
276	0.00406692066708692\\
277	0.0040675939000439\\
278	0.00406827974262886\\
279	0.00406897835323118\\
280	0.00406968914749888\\
281	0.00407041227445104\\
282	0.00407114793169407\\
283	0.0040718963188144\\
284	0.00407265763728611\\
285	0.0040734320903632\\
286	0.00407421988295385\\
287	0.00407502122147461\\
288	0.00407583631368072\\
289	0.00407666536846943\\
290	0.00407750859565094\\
291	0.00407836620568241\\
292	0.00407923840935802\\
293	0.00408012541744706\\
294	0.00408102744027068\\
295	0.00408194468720526\\
296	0.00408287736609797\\
297	0.00408382568257669\\
298	0.0040847898392321\\
299	0.00408577003464479\\
300	0.00408676646222342\\
301	0.00408777930881153\\
302	0.00408880875300992\\
303	0.00408985496314912\\
304	0.0040909180948289\\
305	0.00409199828792148\\
306	0.00409309566291238\\
307	0.00409421031642586\\
308	0.00409534231576585\\
309	0.00409649169232576\\
310	0.00409765843387609\\
311	0.00409884247631765\\
312	0.00410004369736373\\
313	0.00410126192051685\\
314	0.00410249695646168\\
315	0.0041037487710781\\
316	0.00410501809269451\\
317	0.00410630872083941\\
318	0.00410762228403794\\
319	0.00410895917827685\\
320	0.00411031980594528\\
321	0.00411170457592415\\
322	0.00411311390367631\\
323	0.0041145482113375\\
324	0.00411600792780806\\
325	0.00411749348884517\\
326	0.00411900533715594\\
327	0.00412054392249108\\
328	0.00412210970173916\\
329	0.00412370313902142\\
330	0.00412532470578721\\
331	0.00412697488090991\\
332	0.00412865415078315\\
333	0.00413036300941769\\
334	0.00413210195853834\\
335	0.00413387150768126\\
336	0.00413567217429118\\
337	0.00413750448381872\\
338	0.00413936896981707\\
339	0.00414126617403825\\
340	0.00414319664652806\\
341	0.004145160945719\\
342	0.00414715963852069\\
343	0.0041491933004056\\
344	0.00415126251548892\\
345	0.0041533678765995\\
346	0.00415550998533789\\
347	0.00415768945211674\\
348	0.00415990689617547\\
349	0.00416216294555975\\
350	0.00416445823705093\\
351	0.00416679341602682\\
352	0.00416916913622703\\
353	0.00417158605938683\\
354	0.00417404485469147\\
355	0.00417654619798789\\
356	0.00417909077067757\\
357	0.00418167925822084\\
358	0.00418431234829119\\
359	0.00418699072918654\\
360	0.00418971509198084\\
361	0.00419248615450633\\
362	0.00419530468493859\\
363	0.00419817146485468\\
364	0.00420108728974278\\
365	0.00420405296957705\\
366	0.00420706932946481\\
367	0.00421013721038834\\
368	0.00421325747015759\\
369	0.00421643098468567\\
370	0.00421965864940848\\
371	0.00422294138104568\\
372	0.00422628011982449\\
373	0.00422967583232177\\
374	0.00423312951512707\\
375	0.00423664219958411\\
376	0.00424021495790487\\
377	0.00424384891089465\\
378	0.00424754523712741\\
379	0.00425130518182945\\
380	0.00425513005820619\\
381	0.00425902121396172\\
382	0.0042629798657054\\
383	0.0042670071798537\\
384	0.00427110434913889\\
385	0.00427527259430024\\
386	0.00427951316601193\\
387	0.00428382734708664\\
388	0.00428821645499868\\
389	0.00429268184477737\\
390	0.00429722491232846\\
391	0.00430184709825241\\
392	0.00430654989223478\\
393	0.00431133483809401\\
394	0.00431620353957894\\
395	0.00432115766701866\\
396	0.00432619896493554\\
397	0.00433132926073578\\
398	0.00433655047459088\\
399	0.00434186463061658\\
400	0.00434727386943476\\
401	0.00435278046216777\\
402	0.00435838682584821\\
403	0.00436409554012939\\
404	0.00436990936502242\\
405	0.0043758312591468\\
406	0.00438186439766819\\
407	0.00438801218860048\\
408	0.00439427828525927\\
409	0.00440066659110231\\
410	0.0044071812515267\\
411	0.00441382662958582\\
412	0.00442060725501341\\
413	0.00442752773369903\\
414	0.00443459259727504\\
415	0.00444180604812211\\
416	0.00444917192835531\\
417	0.00445669404834207\\
418	0.00446437634750615\\
419	0.00447222290003713\\
420	0.00448023792066377\\
421	0.00448842577068774\\
422	0.00449679096427237\\
423	0.00450533817498008\\
424	0.00451407224216589\\
425	0.00452299817835337\\
426	0.00453212118052739\\
427	0.0045414466429674\\
428	0.00455098016755669\\
429	0.00456072757490198\\
430	0.00457069491622738\\
431	0.00458088848628549\\
432	0.00459131483644244\\
433	0.00460198078562155\\
434	0.00461289343244344\\
435	0.00462406017275493\\
436	0.00463548871752189\\
437	0.00464718710969808\\
438	0.00465916374184674\\
439	0.00467142737449136\\
440	0.00468398715518755\\
441	0.00469685263836591\\
442	0.00471003380649355\\
443	0.00472354109554875\\
444	0.00473738543486821\\
445	0.00475157824689911\\
446	0.00476613143619902\\
447	0.00478105743071025\\
448	0.00479636924656813\\
449	0.00481208052910316\\
450	0.00482820559756616\\
451	0.00484475949353333\\
452	0.00486175803227568\\
453	0.00487921785586886\\
454	0.00489715649787599\\
455	0.00491559266873435\\
456	0.00493454612292579\\
457	0.0049540373357115\\
458	0.00497408748771998\\
459	0.00499471871224096\\
460	0.00501595407012546\\
461	0.00503781727833482\\
462	0.00506033258423081\\
463	0.00508352468522577\\
464	0.00510741842122384\\
465	0.00513204033536679\\
466	0.00515741948366903\\
467	0.00518358660865367\\
468	0.00521056816764443\\
469	0.00523839073483765\\
470	0.00526708158535866\\
471	0.00529666956048661\\
472	0.00532718623681849\\
473	0.00535866769970322\\
474	0.00539116649826347\\
475	0.00542474181628269\\
476	0.00545945169037871\\
477	0.00549535093760036\\
478	0.00553248836020859\\
479	0.00557090284134059\\
480	0.00561061805473222\\
481	0.00565163534862653\\
482	0.00569392423879731\\
483	0.00573740976427773\\
484	0.00578195573058062\\
485	0.00582734256557108\\
486	0.00587302073186874\\
487	0.00591777929369159\\
488	0.00596147832584702\\
489	0.00600396924585436\\
490	0.00604509677476079\\
491	0.00608470203551141\\
492	0.00612262728640956\\
493	0.00615872287684319\\
494	0.0061928572469614\\
495	0.00622493098413339\\
496	0.00625489643300038\\
497	0.0062827844896127\\
498	0.0063089053613691\\
499	0.00633450440142987\\
500	0.00635957631827112\\
501	0.00638412492653279\\
502	0.00640816477749661\\
503	0.00643172275774873\\
504	0.00645483951784793\\
505	0.0064775705149099\\
506	0.00649998635097495\\
507	0.00652217194101431\\
508	0.00654422384304134\\
509	0.0065662448150298\\
510	0.00658831692692719\\
511	0.00661047543979157\\
512	0.00663274081961346\\
513	0.00665513543500258\\
514	0.00667768320198779\\
515	0.0067004090917962\\
516	0.00672333849461296\\
517	0.00674649644599174\\
518	0.00676990674623638\\
519	0.00679359104100118\\
520	0.00681756798909042\\
521	0.00684185273053656\\
522	0.00686645862083711\\
523	0.00689139898441964\\
524	0.00691668742247002\\
525	0.00694233775045284\\
526	0.0069683639461627\\
527	0.00699478011683277\\
528	0.00702160049519244\\
529	0.00704883947499101\\
530	0.0070765116956076\\
531	0.007104632181729\\
532	0.00713321653579037\\
533	0.00716228113862316\\
534	0.00719184326231583\\
535	0.00722192112766865\\
536	0.00725253396467986\\
537	0.00728370207453166\\
538	0.00731544689055641\\
539	0.00734779103435149\\
540	0.00738075836152634\\
541	0.00741437398947174\\
542	0.00744866429705612\\
543	0.0074836568833871\\
544	0.00751938046982297\\
545	0.00755586473097468\\
546	0.0075931400381968\\
547	0.00763123709119732\\
548	0.00767018640119527\\
549	0.00771001757935446\\
550	0.00775075870022047\\
551	0.00779242478641457\\
552	0.00783502989694319\\
553	0.00787858722644379\\
554	0.00792310886900782\\
555	0.00796860537723829\\
556	0.00801508542259554\\
557	0.00806255540390485\\
558	0.00811101877401344\\
559	0.00816046466610628\\
560	0.00821089105072984\\
561	0.00826228888233602\\
562	0.00831463976241251\\
563	0.00836783969989453\\
564	0.00842192152680974\\
565	0.0084769504865854\\
566	0.00853302421756685\\
567	0.0085903217031176\\
568	0.00864905932111027\\
569	0.00870766413426631\\
570	0.00876495225119506\\
571	0.00882046728958851\\
572	0.00887398586474071\\
573	0.00892717754532439\\
574	0.00897988555572072\\
575	0.00903197458790182\\
576	0.00908265151027188\\
577	0.00913183644158302\\
578	0.00917988739287592\\
579	0.00922711699783626\\
580	0.00927343806398176\\
581	0.00931913102044634\\
582	0.0093645049608711\\
583	0.00940961130043772\\
584	0.0094544183622864\\
585	0.00949887800402646\\
586	0.00954293238643295\\
587	0.00958652004546675\\
588	0.00962957886512476\\
589	0.00967204874598804\\
590	0.00971387347495566\\
591	0.00975500239981678\\
592	0.00979538909575469\\
593	0.00983498278816331\\
594	0.00987369853931868\\
595	0.00991118387968948\\
596	0.00994658651044256\\
597	0.00997788999445116\\
598	0.010000292044645\\
599	0\\
600	0\\
};
\addplot [color=blue!40!mycolor9,solid,forget plot]
  table[row sep=crcr]{%
1	0.00381944488791505\\
2	0.00381945382964767\\
3	0.00381946292077991\\
4	0.00381947216387073\\
5	0.00381948156152383\\
6	0.00381949111638838\\
7	0.00381950083115981\\
8	0.00381951070858063\\
9	0.00381952075144128\\
10	0.00381953096258094\\
11	0.00381954134488834\\
12	0.00381955190130264\\
13	0.00381956263481437\\
14	0.00381957354846626\\
15	0.00381958464535422\\
16	0.00381959592862812\\
17	0.00381960740149288\\
18	0.00381961906720937\\
19	0.00381963092909535\\
20	0.00381964299052653\\
21	0.00381965525493748\\
22	0.00381966772582277\\
23	0.00381968040673791\\
24	0.00381969330130045\\
25	0.0038197064131911\\
26	0.00381971974615478\\
27	0.00381973330400171\\
28	0.0038197470906086\\
29	0.0038197611099198\\
30	0.00381977536594841\\
31	0.0038197898627776\\
32	0.00381980460456168\\
33	0.00381981959552742\\
34	0.00381983483997533\\
35	0.00381985034228096\\
36	0.00381986610689601\\
37	0.00381988213834992\\
38	0.00381989844125108\\
39	0.00381991502028821\\
40	0.00381993188023176\\
41	0.0038199490259354\\
42	0.00381996646233736\\
43	0.00381998419446203\\
44	0.00382000222742135\\
45	0.00382002056641638\\
46	0.0038200392167389\\
47	0.00382005818377291\\
48	0.00382007747299627\\
49	0.0038200970899824\\
50	0.00382011704040181\\
51	0.00382013733002396\\
52	0.00382015796471891\\
53	0.00382017895045909\\
54	0.00382020029332103\\
55	0.00382022199948732\\
56	0.00382024407524837\\
57	0.00382026652700434\\
58	0.00382028936126701\\
59	0.00382031258466178\\
60	0.00382033620392975\\
61	0.00382036022592958\\
62	0.00382038465763972\\
63	0.00382040950616044\\
64	0.00382043477871595\\
65	0.00382046048265671\\
66	0.0038204866254615\\
67	0.00382051321473978\\
68	0.00382054025823403\\
69	0.00382056776382201\\
70	0.00382059573951924\\
71	0.00382062419348138\\
72	0.00382065313400674\\
73	0.00382068256953883\\
74	0.00382071250866895\\
75	0.00382074296013872\\
76	0.00382077393284292\\
77	0.00382080543583205\\
78	0.00382083747831521\\
79	0.00382087006966293\\
80	0.00382090321940996\\
81	0.00382093693725835\\
82	0.00382097123308031\\
83	0.00382100611692133\\
84	0.00382104159900323\\
85	0.00382107768972738\\
86	0.00382111439967794\\
87	0.00382115173962502\\
88	0.00382118972052815\\
89	0.00382122835353958\\
90	0.00382126765000783\\
91	0.00382130762148123\\
92	0.00382134827971144\\
93	0.00382138963665715\\
94	0.0038214317044878\\
95	0.00382147449558743\\
96	0.00382151802255859\\
97	0.00382156229822609\\
98	0.00382160733564117\\
99	0.00382165314808561\\
100	0.00382169974907573\\
101	0.00382174715236684\\
102	0.00382179537195737\\
103	0.00382184442209338\\
104	0.00382189431727298\\
105	0.00382194507225083\\
106	0.00382199670204285\\
107	0.00382204922193089\\
108	0.00382210264746757\\
109	0.00382215699448107\\
110	0.00382221227908013\\
111	0.00382226851765912\\
112	0.00382232572690309\\
113	0.00382238392379307\\
114	0.00382244312561133\\
115	0.00382250334994673\\
116	0.00382256461470026\\
117	0.00382262693809057\\
118	0.00382269033865958\\
119	0.00382275483527822\\
120	0.00382282044715228\\
121	0.00382288719382837\\
122	0.00382295509519975\\
123	0.00382302417151253\\
124	0.00382309444337185\\
125	0.00382316593174796\\
126	0.0038232386579827\\
127	0.00382331264379583\\
128	0.00382338791129142\\
129	0.00382346448296455\\
130	0.0038235423817078\\
131	0.00382362163081799\\
132	0.00382370225400282\\
133	0.00382378427538783\\
134	0.00382386771952312\\
135	0.00382395261139026\\
136	0.00382403897640941\\
137	0.00382412684044632\\
138	0.00382421622981948\\
139	0.00382430717130773\\
140	0.00382439969215842\\
141	0.00382449382009734\\
142	0.00382458958334283\\
143	0.00382468701063115\\
144	0.00382478613126695\\
145	0.00382488697523013\\
146	0.00382498957338099\\
147	0.00382509395772975\\
148	0.00382520016121877\\
149	0.00382530821570657\\
150	0.00382541815360922\\
151	0.00382553000791004\\
152	0.00382564381216948\\
153	0.00382575960053517\\
154	0.00382587740775211\\
155	0.003825997269173\\
156	0.00382611922076889\\
157	0.00382624329913982\\
158	0.00382636954152591\\
159	0.00382649798581826\\
160	0.00382662867057043\\
161	0.00382676163500982\\
162	0.00382689691904938\\
163	0.00382703456329954\\
164	0.00382717460908022\\
165	0.00382731709843319\\
166	0.0038274620741345\\
167	0.00382760957970729\\
168	0.00382775965943458\\
169	0.00382791235837252\\
170	0.00382806772236377\\
171	0.00382822579805092\\
172	0.00382838663289046\\
173	0.00382855027516681\\
174	0.00382871677400655\\
175	0.00382888617939297\\
176	0.0038290585421808\\
177	0.00382923391411128\\
178	0.00382941234782737\\
179	0.00382959389688925\\
180	0.00382977861579013\\
181	0.00382996655997224\\
182	0.00383015778584317\\
183	0.00383035235079235\\
184	0.00383055031320795\\
185	0.00383075173249401\\
186	0.00383095666908772\\
187	0.00383116518447723\\
188	0.00383137734121947\\
189	0.00383159320295851\\
190	0.00383181283444411\\
191	0.00383203630155047\\
192	0.00383226367129549\\
193	0.00383249501186013\\
194	0.00383273039260829\\
195	0.00383296988410682\\
196	0.00383321355814593\\
197	0.00383346148775993\\
198	0.00383371374724835\\
199	0.00383397041219723\\
200	0.00383423155950099\\
201	0.00383449726738433\\
202	0.00383476761542474\\
203	0.00383504268457528\\
204	0.00383532255718768\\
205	0.00383560731703575\\
206	0.00383589704933934\\
207	0.00383619184078845\\
208	0.00383649177956781\\
209	0.00383679695538181\\
210	0.00383710745947986\\
211	0.00383742338468209\\
212	0.00383774482540548\\
213	0.00383807187769018\\
214	0.00383840463922657\\
215	0.00383874320938237\\
216	0.00383908768923041\\
217	0.00383943818157668\\
218	0.00383979479098878\\
219	0.00384015762382479\\
220	0.00384052678826269\\
221	0.00384090239432995\\
222	0.00384128455393374\\
223	0.00384167338089152\\
224	0.00384206899096197\\
225	0.00384247150187654\\
226	0.00384288103337115\\
227	0.00384329770721868\\
228	0.00384372164726155\\
229	0.00384415297944502\\
230	0.00384459183185069\\
231	0.00384503833473076\\
232	0.00384549262054244\\
233	0.00384595482398295\\
234	0.003846425082025\\
235	0.0038469035339526\\
236	0.00384739032139756\\
237	0.00384788558837617\\
238	0.00384838948132651\\
239	0.00384890214914628\\
240	0.00384942374323097\\
241	0.0038499544175126\\
242	0.00385049432849878\\
243	0.00385104363531254\\
244	0.00385160249973235\\
245	0.00385217108623283\\
246	0.00385274956202587\\
247	0.00385333809710236\\
248	0.00385393686427437\\
249	0.00385454603921786\\
250	0.00385516580051607\\
251	0.00385579632970351\\
252	0.00385643781131035\\
253	0.00385709043290793\\
254	0.00385775438515451\\
255	0.00385842986184218\\
256	0.00385911705994454\\
257	0.00385981617966518\\
258	0.00386052742448745\\
259	0.00386125100122527\\
260	0.00386198712007523\\
261	0.00386273599467023\\
262	0.00386349784213462\\
263	0.00386427288314106\\
264	0.00386506134196932\\
265	0.00386586344656686\\
266	0.00386667942861144\\
267	0.00386750952357506\\
268	0.00386835397078955\\
269	0.0038692130135119\\
270	0.00387008689898778\\
271	0.00387097587850831\\
272	0.00387188020744911\\
273	0.00387280014526215\\
274	0.00387373595533601\\
275	0.00387468790447146\\
276	0.0038756562611982\\
277	0.00387664129061343\\
278	0.00387764323968271\\
279	0.00387866230829795\\
280	0.0038796987741781\\
281	0.00388075292388058\\
282	0.00388182504825039\\
283	0.00388291544246592\\
284	0.00388402440608404\\
285	0.00388515224308433\\
286	0.0038862992619126\\
287	0.0038874657755231\\
288	0.0038886521014196\\
289	0.00388985856169444\\
290	0.00389108548306613\\
291	0.00389233319691414\\
292	0.00389360203931105\\
293	0.00389489235105145\\
294	0.00389620447767661\\
295	0.00389753876949465\\
296	0.0038988955815951\\
297	0.00390027527385714\\
298	0.0039016782109505\\
299	0.00390310476232804\\
300	0.00390455530220914\\
301	0.00390603020955305\\
302	0.00390752986802195\\
303	0.00390905466593376\\
304	0.00391060499620635\\
305	0.00391218125629692\\
306	0.0039137838481452\\
307	0.00391541317813866\\
308	0.00391706965714266\\
309	0.0039187537006957\\
310	0.00392046572962547\\
311	0.00392220617174694\\
312	0.00392397546638137\\
313	0.00392577407619996\\
314	0.00392760251734056\\
315	0.00392946142885158\\
316	0.00393135167357656\\
317	0.00393327387039218\\
318	0.00393522852295292\\
319	0.00393721614132281\\
320	0.00393923724199673\\
321	0.00394129234791969\\
322	0.00394338198850431\\
323	0.00394550669964648\\
324	0.00394766702373917\\
325	0.00394986350968491\\
326	0.00395209671290709\\
327	0.00395436719536063\\
328	0.00395667552554228\\
329	0.00395902227850162\\
330	0.00396140803585341\\
331	0.00396383338579263\\
332	0.0039662989231134\\
333	0.00396880524923413\\
334	0.00397135297223093\\
335	0.00397394270688251\\
336	0.00397657507473022\\
337	0.00397925070415858\\
338	0.00398197023050168\\
339	0.00398473429618398\\
340	0.00398754355090492\\
341	0.00399039865187992\\
342	0.00399330026415337\\
343	0.00399624906100389\\
344	0.00399924572446666\\
345	0.00400229094600521\\
346	0.00400538542737336\\
347	0.00400852988171844\\
348	0.0040117250349921\\
349	0.00401497162775132\\
350	0.00401827041745474\\
351	0.00402162218138518\\
352	0.00402502772035511\\
353	0.00402848786336176\\
354	0.00403200347329119\\
355	0.00403557545339288\\
356	0.00403920475267724\\
357	0.00404289236201626\\
358	0.00404663926532477\\
359	0.00405044617696116\\
360	0.00405431209755343\\
361	0.00405823454676541\\
362	0.00406221390273528\\
363	0.00406625050416226\\
364	0.00407034464456845\\
365	0.00407449656600271\\
366	0.00407870645169283\\
367	0.00408297441481805\\
368	0.00408730047692172\\
369	0.00409168456088001\\
370	0.00409612647725224\\
371	0.00410062590479673\\
372	0.00410518236700039\\
373	0.00410979520407372\\
374	0.00411446354037165\\
375	0.00411918624930451\\
376	0.00412396192469406\\
377	0.00412878888933131\\
378	0.00413366534096513\\
379	0.00413858996594277\\
380	0.00414356417617088\\
381	0.00414860064463897\\
382	0.00415370519126221\\
383	0.00415887782354044\\
384	0.00416411846490067\\
385	0.00416942694647098\\
386	0.00417480299821334\\
387	0.00418024623940122\\
388	0.00418575616844292\\
389	0.00419133215207308\\
390	0.00419697341396355\\
391	0.00420267902284577\\
392	0.00420844788029291\\
393	0.00421427870838504\\
394	0.00422017003758266\\
395	0.00422612019527058\\
396	0.00423212729561477\\
397	0.00423818923161663\\
398	0.00424430367056677\\
399	0.00425046805452244\\
400	0.00425667960798663\\
401	0.00426293535569741\\
402	0.00426923215439836\\
403	0.00427556674372872\\
404	0.00428193582305656\\
405	0.00428833616336053\\
406	0.00429476476652256\\
407	0.00430121908978586\\
408	0.00430769736567081\\
409	0.00431419908071125\\
410	0.00432072567623176\\
411	0.00432728118476319\\
412	0.00433387326282209\\
413	0.00434051456486937\\
414	0.00434722456676876\\
415	0.00435403197492565\\
416	0.00436096030699787\\
417	0.0043680178596261\\
418	0.00437520729992339\\
419	0.00438253139700803\\
420	0.00438999303583359\\
421	0.00439759523439973\\
422	0.0044053411642879\\
423	0.00441323417190158\\
424	0.00442127778875179\\
425	0.00442947568622172\\
426	0.00443783142315854\\
427	0.00444634839223422\\
428	0.00445503009046111\\
429	0.00446388012457871\\
430	0.00447290221682272\\
431	0.00448210021101293\\
432	0.0044914780788647\\
433	0.00450103992690116\\
434	0.00451079000428179\\
435	0.00452073271115161\\
436	0.00453087260748145\\
437	0.00454121442262168\\
438	0.00455176306564887\\
439	0.00456252363660099\\
440	0.00457350143872782\\
441	0.00458470199196287\\
442	0.00459613104799112\\
443	0.00460779460714568\\
444	0.00461969893332353\\
445	0.00463185057046742\\
446	0.00464425636526101\\
447	0.00465692349307205\\
448	0.00466985948416286\\
449	0.00468307225273333\\
450	0.00469657012804901\\
451	0.00471036188452164\\
452	0.0047244567769889\\
453	0.00473886458248146\\
454	0.00475359565575727\\
455	0.0047686609505671\\
456	0.00478407203480342\\
457	0.00479984113684248\\
458	0.00481598120812612\\
459	0.00483250597144564\\
460	0.00484943004417186\\
461	0.00486676903423521\\
462	0.00488453962375556\\
463	0.00490275965605901\\
464	0.00492144834571443\\
465	0.00494062633330873\\
466	0.0049603155092367\\
467	0.00498053854270406\\
468	0.00500131922287815\\
469	0.005022682272008\\
470	0.00504465297027654\\
471	0.00506725646762278\\
472	0.00509051803742607\\
473	0.00511446380997644\\
474	0.00513912196386034\\
475	0.00516452336427147\\
476	0.00519070172399126\\
477	0.00521768674120825\\
478	0.00524550790757517\\
479	0.00527419459831302\\
480	0.00530377649605227\\
481	0.00533428372592657\\
482	0.00536574715474855\\
483	0.00539819893647581\\
484	0.00543167340888997\\
485	0.00546620850085091\\
486	0.00550185097014559\\
487	0.00553866543163327\\
488	0.00557671540533525\\
489	0.00561606113875089\\
490	0.00565675655137328\\
491	0.00569884505698706\\
492	0.00574235390404411\\
493	0.00578728657121308\\
494	0.00583361260813407\\
495	0.00588125412523459\\
496	0.00593006789083969\\
497	0.00597982167539138\\
498	0.00603000261139891\\
499	0.00607919001400096\\
500	0.0061272311955095\\
501	0.00617396403990961\\
502	0.00621921947190392\\
503	0.00626282523322078\\
504	0.00630461149639069\\
505	0.00634441913381876\\
506	0.00638211144982614\\
507	0.00641759075241061\\
508	0.00645082139107356\\
509	0.00648186133141168\\
510	0.00651137009306334\\
511	0.00654034185867726\\
512	0.00656877382473748\\
513	0.00659667381570486\\
514	0.00662406212808724\\
515	0.00665097331469975\\
516	0.00667745772009153\\
517	0.00670358249291407\\
518	0.00672943165361346\\
519	0.00675510460981876\\
520	0.00678071227161439\\
521	0.00680636956803016\\
522	0.0068321418354832\\
523	0.00685805883381921\\
524	0.00688414616269528\\
525	0.00691043166666499\\
526	0.0069369449982943\\
527	0.00696371702157117\\
528	0.006990779052795\\
529	0.00701816195599068\\
530	0.00704589514320308\\
531	0.00707400558272338\\
532	0.007102517000605\\
533	0.00713145022215655\\
534	0.00716082564304089\\
535	0.00719066452125459\\
536	0.00722098892233578\\
537	0.00725182166615787\\
538	0.00728318628224142\\
539	0.00731510698194115\\
540	0.00734760865663381\\
541	0.00738071691038189\\
542	0.00741445813238253\\
543	0.0074488596067575\\
544	0.0074839496420057\\
545	0.00751975758741009\\
546	0.00755631369103536\\
547	0.00759364885472858\\
548	0.00763179428830673\\
549	0.00767078103297187\\
550	0.00771063930619072\\
551	0.00775139804237961\\
552	0.00779307363174633\\
553	0.00783568124693249\\
554	0.00787923540543176\\
555	0.00792374989807655\\
556	0.00796923763533189\\
557	0.00801571046336702\\
558	0.00806317895231008\\
559	0.00811165226865771\\
560	0.00816113770274232\\
561	0.00821164035063354\\
562	0.00826316276313242\\
563	0.00831570535723852\\
564	0.0083692583124851\\
565	0.00842380278768651\\
566	0.00847932001064811\\
567	0.00853574542202786\\
568	0.00859302797587814\\
569	0.00865122838399295\\
570	0.00871044661767393\\
571	0.00877081641557857\\
572	0.00883206177374144\\
573	0.00889210788623348\\
574	0.00895060193483743\\
575	0.0090069743480842\\
576	0.00906164998966982\\
577	0.00911569254926328\\
578	0.00916848694084165\\
579	0.00921948532352991\\
580	0.00926878734421818\\
581	0.00931668587461442\\
582	0.00936337646786582\\
583	0.00940899867258259\\
584	0.00945409509992045\\
585	0.00949869896494848\\
586	0.00954283083080037\\
587	0.00958646243837779\\
588	0.00962954789259008\\
589	0.0096720338085118\\
590	0.00971386797226699\\
591	0.00975500106378656\\
592	0.00979538897701793\\
593	0.00983498278816331\\
594	0.00987369853931868\\
595	0.00991118387968948\\
596	0.00994658651044256\\
597	0.00997788999445116\\
598	0.010000292044645\\
599	0\\
600	0\\
};
\addplot [color=blue!75!mycolor7,solid,forget plot]
  table[row sep=crcr]{%
1	0.00303384705345048\\
2	0.003033865512802\\
3	0.00303388428662202\\
4	0.00303390338031515\\
5	0.00303392279937922\\
6	0.00303394254940679\\
7	0.00303396263608701\\
8	0.00303398306520715\\
9	0.00303400384265426\\
10	0.0030340249744169\\
11	0.00303404646658698\\
12	0.00303406832536142\\
13	0.00303409055704408\\
14	0.00303411316804746\\
15	0.00303413616489464\\
16	0.0030341595542212\\
17	0.00303418334277715\\
18	0.00303420753742885\\
19	0.00303423214516106\\
20	0.00303425717307896\\
21	0.00303428262841024\\
22	0.0030343085185072\\
23	0.00303433485084882\\
24	0.00303436163304314\\
25	0.00303438887282922\\
26	0.00303441657807957\\
27	0.00303444475680241\\
28	0.00303447341714398\\
29	0.00303450256739096\\
30	0.00303453221597281\\
31	0.00303456237146433\\
32	0.00303459304258804\\
33	0.00303462423821688\\
34	0.00303465596737664\\
35	0.00303468823924865\\
36	0.00303472106317255\\
37	0.00303475444864891\\
38	0.00303478840534193\\
39	0.0030348229430825\\
40	0.00303485807187084\\
41	0.00303489380187951\\
42	0.0030349301434564\\
43	0.00303496710712772\\
44	0.00303500470360105\\
45	0.00303504294376846\\
46	0.0030350818387098\\
47	0.00303512139969574\\
48	0.00303516163819128\\
49	0.00303520256585894\\
50	0.00303524419456222\\
51	0.00303528653636906\\
52	0.00303532960355534\\
53	0.00303537340860847\\
54	0.00303541796423115\\
55	0.00303546328334487\\
56	0.00303550937909384\\
57	0.00303555626484877\\
58	0.00303560395421081\\
59	0.00303565246101549\\
60	0.00303570179933677\\
61	0.00303575198349118\\
62	0.00303580302804193\\
63	0.00303585494780319\\
64	0.00303590775784455\\
65	0.00303596147349518\\
66	0.00303601611034853\\
67	0.00303607168426674\\
68	0.00303612821138534\\
69	0.00303618570811794\\
70	0.00303624419116103\\
71	0.00303630367749889\\
72	0.00303636418440846\\
73	0.00303642572946448\\
74	0.00303648833054459\\
75	0.00303655200583452\\
76	0.00303661677383344\\
77	0.00303668265335929\\
78	0.00303674966355437\\
79	0.00303681782389086\\
80	0.0030368871541765\\
81	0.00303695767456038\\
82	0.00303702940553884\\
83	0.00303710236796138\\
84	0.00303717658303682\\
85	0.0030372520723394\\
86	0.00303732885781506\\
87	0.00303740696178794\\
88	0.00303748640696674\\
89	0.00303756721645135\\
90	0.00303764941373972\\
91	0.00303773302273442\\
92	0.00303781806774981\\
93	0.00303790457351899\\
94	0.00303799256520105\\
95	0.00303808206838826\\
96	0.00303817310911356\\
97	0.00303826571385812\\
98	0.00303835990955895\\
99	0.00303845572361669\\
100	0.00303855318390362\\
101	0.0030386523187716\\
102	0.00303875315706035\\
103	0.00303885572810569\\
104	0.00303896006174804\\
105	0.00303906618834104\\
106	0.00303917413876022\\
107	0.00303928394441195\\
108	0.0030393956372424\\
109	0.00303950924974678\\
110	0.00303962481497866\\
111	0.00303974236655932\\
112	0.00303986193868759\\
113	0.00303998356614951\\
114	0.0030401072843283\\
115	0.00304023312921447\\
116	0.00304036113741615\\
117	0.00304049134616944\\
118	0.00304062379334912\\
119	0.00304075851747948\\
120	0.00304089555774519\\
121	0.00304103495400245\\
122	0.00304117674679044\\
123	0.00304132097734277\\
124	0.00304146768759921\\
125	0.00304161692021761\\
126	0.00304176871858596\\
127	0.0030419231268348\\
128	0.00304208018984966\\
129	0.00304223995328376\\
130	0.00304240246357105\\
131	0.00304256776793924\\
132	0.00304273591442327\\
133	0.0030429069518788\\
134	0.00304308092999609\\
135	0.00304325789931412\\
136	0.00304343791123465\\
137	0.00304362101803702\\
138	0.00304380727289281\\
139	0.00304399672988108\\
140	0.00304418944400408\\
141	0.00304438547120398\\
142	0.00304458486838166\\
143	0.00304478769341881\\
144	0.00304499400520721\\
145	0.00304520386368493\\
146	0.00304541732986664\\
147	0.00304563446581277\\
148	0.00304585533449473\\
149	0.00304607999995983\\
150	0.00304630852734943\\
151	0.00304654098291754\\
152	0.00304677743404971\\
153	0.00304701794928214\\
154	0.00304726259832124\\
155	0.00304751145206357\\
156	0.00304776458261577\\
157	0.00304802206331531\\
158	0.00304828396875115\\
159	0.00304855037478505\\
160	0.00304882135857308\\
161	0.00304909699858753\\
162	0.00304937737463915\\
163	0.00304966256789988\\
164	0.0030499526609258\\
165	0.00305024773768053\\
166	0.00305054788355909\\
167	0.00305085318541202\\
168	0.00305116373157007\\
169	0.00305147961186905\\
170	0.00305180091767536\\
171	0.00305212774191184\\
172	0.00305246017908392\\
173	0.00305279832530643\\
174	0.00305314227833069\\
175	0.00305349213757211\\
176	0.00305384800413824\\
177	0.00305420998085731\\
178	0.00305457817230707\\
179	0.00305495268484445\\
180	0.00305533362663533\\
181	0.00305572110768509\\
182	0.00305611523986942\\
183	0.00305651613696585\\
184	0.00305692391468572\\
185	0.00305733869070659\\
186	0.0030577605847053\\
187	0.00305818971839144\\
188	0.00305862621554155\\
189	0.00305907020203372\\
190	0.00305952180588283\\
191	0.00305998115727626\\
192	0.00306044838861037\\
193	0.00306092363452737\\
194	0.00306140703195296\\
195	0.00306189872013442\\
196	0.00306239884067943\\
197	0.00306290753759554\\
198	0.00306342495733007\\
199	0.00306395124881094\\
200	0.00306448656348788\\
201	0.00306503105537458\\
202	0.00306558488109133\\
203	0.00306614819990824\\
204	0.00306672117378955\\
205	0.00306730396743808\\
206	0.00306789674834098\\
207	0.00306849968681576\\
208	0.00306911295605723\\
209	0.00306973673218526\\
210	0.00307037119429307\\
211	0.00307101652449648\\
212	0.00307167290798388\\
213	0.0030723405330669\\
214	0.00307301959123206\\
215	0.00307371027719301\\
216	0.00307441278894365\\
217	0.00307512732781234\\
218	0.00307585409851649\\
219	0.00307659330921831\\
220	0.00307734517158135\\
221	0.00307810990082795\\
222	0.00307888771579747\\
223	0.00307967883900548\\
224	0.00308048349670392\\
225	0.00308130191894195\\
226	0.00308213433962811\\
227	0.00308298099659295\\
228	0.00308384213165301\\
229	0.0030847179906756\\
230	0.00308560882364454\\
231	0.00308651488472681\\
232	0.00308743643234043\\
233	0.00308837372922319\\
234	0.0030893270425023\\
235	0.00309029664376532\\
236	0.00309128280913192\\
237	0.00309228581932678\\
238	0.0030933059597536\\
239	0.00309434352057004\\
240	0.00309539879676386\\
241	0.00309647208823016\\
242	0.00309756369984966\\
243	0.00309867394156809\\
244	0.00309980312847685\\
245	0.00310095158089455\\
246	0.00310211962444996\\
247	0.00310330759016596\\
248	0.00310451581454468\\
249	0.00310574463965382\\
250	0.00310699441321429\\
251	0.00310826548868878\\
252	0.00310955822537173\\
253	0.00311087298848048\\
254	0.00311221014924765\\
255	0.00311357008501479\\
256	0.00311495317932709\\
257	0.00311635982202967\\
258	0.00311779040936487\\
259	0.00311924534407096\\
260	0.00312072503548212\\
261	0.0031222298996297\\
262	0.00312376035934481\\
263	0.00312531684436224\\
264	0.00312689979142575\\
265	0.00312850964439462\\
266	0.00313014685435141\\
267	0.00313181187971142\\
268	0.00313350518633285\\
269	0.00313522724762836\\
270	0.00313697854467666\\
271	0.00313875956633266\\
272	0.00314057080933009\\
273	0.00314241277836093\\
274	0.00314428598608756\\
275	0.00314619095297797\\
276	0.00314812820673057\\
277	0.00315009828105087\\
278	0.00315210171510535\\
279	0.00315413906353755\\
280	0.00315621089020438\\
281	0.00315831776787394\\
282	0.00316046027834203\\
283	0.00316263901254875\\
284	0.003164854570695\\
285	0.00316710756235895\\
286	0.00316939860661227\\
287	0.00317172833213598\\
288	0.00317409737733571\\
289	0.00317650639045645\\
290	0.00317895602969642\\
291	0.00318144696331995\\
292	0.00318397986976924\\
293	0.00318655543777469\\
294	0.0031891743664637\\
295	0.00319183736546763\\
296	0.00319454515502686\\
297	0.00319729846609331\\
298	0.00320009804043076\\
299	0.00320294463071217\\
300	0.00320583900061387\\
301	0.00320878192490665\\
302	0.0032117741895431\\
303	0.0032148165917414\\
304	0.00321790994006526\\
305	0.00322105505450091\\
306	0.00322425276653286\\
307	0.00322750391922457\\
308	0.0032308093673196\\
309	0.00323416997740608\\
310	0.00323758662825071\\
311	0.00324106021155859\\
312	0.00324459163371413\\
313	0.00324818181944748\\
314	0.00325183171778768\\
315	0.00325554230340428\\
316	0.00325931453812834\\
317	0.00326314938512363\\
318	0.00326704781990338\\
319	0.00327101083037703\\
320	0.00327503941689374\\
321	0.00327913459228231\\
322	0.00328329738188779\\
323	0.0032875288236051\\
324	0.00329182996790991\\
325	0.00329620187788686\\
326	0.0033006456292558\\
327	0.00330516231039655\\
328	0.00330975302237266\\
329	0.00331441887895505\\
330	0.00331916100664649\\
331	0.00332398054470761\\
332	0.00332887864518655\\
333	0.00333385647295279\\
334	0.00333891520573777\\
335	0.00334405603418376\\
336	0.00334928016190405\\
337	0.00335458880555669\\
338	0.00335998319493566\\
339	0.00336546457308269\\
340	0.00337103419642487\\
341	0.00337669333494218\\
342	0.0033824432723713\\
343	0.0033882853064516\\
344	0.00339422074922057\\
345	0.00340025092736631\\
346	0.00340637718264471\\
347	0.00341260087237047\\
348	0.00341892336998915\\
349	0.00342534606573737\\
350	0.00343187036739461\\
351	0.00343849770112212\\
352	0.00344522951236527\\
353	0.00345206726674163\\
354	0.00345901245067994\\
355	0.00346606657109623\\
356	0.00347323115190361\\
357	0.00348050772061386\\
358	0.00348789776607916\\
359	0.00349540263597865\\
360	0.00350302364655946\\
361	0.00351076243536027\\
362	0.00351862066977116\\
363	0.00352660004706458\\
364	0.00353470229387294\\
365	0.00354292916578727\\
366	0.00355128244940566\\
367	0.00355976397110082\\
368	0.00356837560093155\\
369	0.00357711908392077\\
370	0.00358599606644035\\
371	0.00359500817362262\\
372	0.00360415701181463\\
373	0.00361344416473827\\
374	0.00362287119070602\\
375	0.00363243962329641\\
376	0.0036421509819296\\
377	0.00365200680929468\\
378	0.00366200877770539\\
379	0.00367215894944063\\
380	0.00368246019276606\\
381	0.00369291448887022\\
382	0.00370352316447297\\
383	0.00371428743342848\\
384	0.0037252083752065\\
385	0.00373628690975085\\
386	0.00374752376805794\\
387	0.00375891945768785\\
388	0.00377047422226314\\
389	0.00378218799381759\\
390	0.00379406033662271\\
391	0.00380609038083258\\
392	0.00381827674393536\\
393	0.00383061743756721\\
394	0.00384310975671144\\
395	0.00385575014764735\\
396	0.00386853405019943\\
397	0.00388145570882703\\
398	0.00389450794583856\\
399	0.00390768188844423\\
400	0.00392096663938711\\
401	0.0039343488783687\\
402	0.00394781237813493\\
403	0.00396133741420826\\
404	0.00397490003859048\\
405	0.00398847116685206\\
406	0.00400201535859076\\
407	0.00401548887315432\\
408	0.00402883494268167\\
409	0.00404197666253574\\
410	0.00405483387971556\\
411	0.00406730920686715\\
412	0.00407928411461078\\
413	0.00409061413591738\\
414	0.00410112328338029\\
415	0.00411059866972351\\
416	0.00411976746527252\\
417	0.00412906916830723\\
418	0.00413850301769023\\
419	0.00414806780402478\\
420	0.0041577618081022\\
421	0.00416758279534187\\
422	0.00417752822733678\\
423	0.00418759622241075\\
424	0.004197789117264\\
425	0.0042081269899841\\
426	0.00421863205451258\\
427	0.00422930607143691\\
428	0.00424015075867423\\
429	0.004251167785805\\
430	0.00426235876789\\
431	0.00427372525871681\\
432	0.00428526874345597\\
433	0.00429699063068474\\
434	0.00430889224367596\\
435	0.00432097481087908\\
436	0.00433323945553472\\
437	0.00434568718434684\\
438	0.0043583188751327\\
439	0.00437113526337202\\
440	0.00438413692758075\\
441	0.00439732427343946\\
442	0.00441069751655755\\
443	0.00442425666353447\\
444	0.00443800149163091\\
445	0.00445193152736049\\
446	0.00446604602374304\\
447	0.00448034393626135\\
448	0.00449482389843095\\
449	0.00450948420924725\\
450	0.00452432288789258\\
451	0.00453933797452697\\
452	0.00455452740746064\\
453	0.00456988901507084\\
454	0.00458542062548439\\
455	0.00460112023002508\\
456	0.00461698622005927\\
457	0.0046330177100885\\
458	0.00464921493486472\\
459	0.00466557939521225\\
460	0.00468210867217519\\
461	0.00469879965952423\\
462	0.00471565019985468\\
463	0.00473265971222406\\
464	0.00474983003071213\\
465	0.00476716651370559\\
466	0.00478467953169976\\
467	0.00480238653623335\\
468	0.00482031477230763\\
469	0.00483850488659161\\
470	0.00485701572111196\\
471	0.004875929161333\\
472	0.00489529644610638\\
473	0.00491513298305584\\
474	0.00493545509421521\\
475	0.00495627994266942\\
476	0.00497762505705277\\
477	0.00499950880073145\\
478	0.0050219504705099\\
479	0.0050449704075158\\
480	0.00506859007336904\\
481	0.00509283211145518\\
482	0.0051177203005794\\
483	0.00514327950696849\\
484	0.00516953856622172\\
485	0.00519652882533619\\
486	0.00522428493328236\\
487	0.00525283976962182\\
488	0.00528222471205487\\
489	0.00531247156248563\\
490	0.00534361250795315\\
491	0.00537568038786467\\
492	0.00540870878097448\\
493	0.00544273219379265\\
494	0.00547778641436578\\
495	0.00551390909868753\\
496	0.0055511406830368\\
497	0.00558952584142584\\
498	0.00562911789773015\\
499	0.00566999128690472\\
500	0.00571221932201546\\
501	0.0057558715749304\\
502	0.00580101005582034\\
503	0.00584768388787208\\
504	0.00589592198047116\\
505	0.00594572303841862\\
506	0.00599704395404753\\
507	0.00604978581485652\\
508	0.00610377318904819\\
509	0.00615872670430141\\
510	0.00621377305779257\\
511	0.00626773342543394\\
512	0.00632044037470168\\
513	0.00637171706810509\\
514	0.00642138012600038\\
515	0.00646924415730192\\
516	0.00651512881061301\\
517	0.00655886878984116\\
518	0.00660032857703474\\
519	0.00663942309655355\\
520	0.00667614607675877\\
521	0.00671060900481495\\
522	0.00674415544607651\\
523	0.00677716675605977\\
524	0.00680964580317178\\
525	0.00684160813228221\\
526	0.00687308399823199\\
527	0.00690412025177207\\
528	0.00693478183651414\\
529	0.00696515253551898\\
530	0.00699533442800454\\
531	0.00702544528207324\\
532	0.00705561271722277\\
533	0.00708594791661066\\
534	0.00711650232276352\\
535	0.00714730727861487\\
536	0.00717839739551637\\
537	0.00720981020187562\\
538	0.00724158560986581\\
539	0.00727376518449118\\
540	0.00730639121578814\\
541	0.00733950562402815\\
542	0.00737314876893419\\
543	0.00740735831282291\\
544	0.00744216840464061\\
545	0.00747761164752185\\
546	0.00751372125221221\\
547	0.00755053132123895\\
548	0.00758807649293622\\
549	0.00762639148986282\\
550	0.0076655105538372\\
551	0.00770546673078169\\
552	0.00774629139584217\\
553	0.00778800482332902\\
554	0.00783062355809369\\
555	0.00787416316338176\\
556	0.00791863849542332\\
557	0.00796406368646087\\
558	0.00801045200277901\\
559	0.00805781568956767\\
560	0.00810616580577066\\
561	0.00815551204477823\\
562	0.00820586253849474\\
563	0.00825722363207021\\
564	0.00830959973629561\\
565	0.00836299312329599\\
566	0.00841740357329269\\
567	0.00847282849714414\\
568	0.0085292617624085\\
569	0.00858669084759215\\
570	0.00864508965557358\\
571	0.00870442086865101\\
572	0.00876459910238825\\
573	0.00882562379752669\\
574	0.00888756930913724\\
575	0.00895054169680455\\
576	0.00901356110634767\\
577	0.00907514900313189\\
578	0.00913502099859305\\
579	0.00919263117907495\\
580	0.0092479758283961\\
581	0.00930184486036383\\
582	0.00935368351991403\\
583	0.00940333326402388\\
584	0.00945082356531764\\
585	0.00949690757289053\\
586	0.00954180842209612\\
587	0.00958587549421746\\
588	0.00962920975611778\\
589	0.0096718487149668\\
590	0.00971377621437142\\
591	0.00975496638267559\\
592	0.00979538031248056\\
593	0.00983498200924467\\
594	0.00987369853931868\\
595	0.00991118387968948\\
596	0.00994658651044256\\
597	0.00997788999445116\\
598	0.010000292044645\\
599	0\\
600	0\\
};
\addplot [color=blue!80!mycolor9,solid,forget plot]
  table[row sep=crcr]{%
1	0.00247811649021119\\
2	0.00247812555694198\\
3	0.00247813477847231\\
4	0.00247814415746359\\
5	0.00247815369662314\\
6	0.00247816339870495\\
7	0.00247817326651042\\
8	0.00247818330288924\\
9	0.00247819351074017\\
10	0.00247820389301195\\
11	0.00247821445270408\\
12	0.00247822519286778\\
13	0.00247823611660675\\
14	0.00247824722707825\\
15	0.00247825852749384\\
16	0.00247827002112047\\
17	0.00247828171128131\\
18	0.00247829360135681\\
19	0.00247830569478558\\
20	0.00247831799506553\\
21	0.00247833050575474\\
22	0.00247834323047263\\
23	0.00247835617290096\\
24	0.00247836933678482\\
25	0.00247838272593385\\
26	0.00247839634422328\\
27	0.00247841019559507\\
28	0.00247842428405909\\
29	0.00247843861369422\\
30	0.00247845318864961\\
31	0.00247846801314581\\
32	0.00247848309147611\\
33	0.00247849842800767\\
34	0.00247851402718287\\
35	0.00247852989352059\\
36	0.00247854603161746\\
37	0.00247856244614932\\
38	0.0024785791418725\\
39	0.00247859612362519\\
40	0.00247861339632894\\
41	0.00247863096498999\\
42	0.00247864883470076\\
43	0.00247866701064138\\
44	0.00247868549808116\\
45	0.00247870430238013\\
46	0.00247872342899056\\
47	0.00247874288345863\\
48	0.00247876267142599\\
49	0.00247878279863142\\
50	0.00247880327091246\\
51	0.0024788240942072\\
52	0.0024788452745559\\
53	0.00247886681810285\\
54	0.00247888873109806\\
55	0.00247891101989917\\
56	0.00247893369097324\\
57	0.00247895675089867\\
58	0.00247898020636708\\
59	0.00247900406418525\\
60	0.00247902833127715\\
61	0.00247905301468592\\
62	0.00247907812157592\\
63	0.00247910365923484\\
64	0.00247912963507571\\
65	0.00247915605663926\\
66	0.00247918293159587\\
67	0.00247921026774801\\
68	0.00247923807303234\\
69	0.00247926635552212\\
70	0.00247929512342953\\
71	0.00247932438510803\\
72	0.00247935414905477\\
73	0.0024793844239131\\
74	0.00247941521847502\\
75	0.00247944654168382\\
76	0.00247947840263653\\
77	0.00247951081058671\\
78	0.00247954377494698\\
79	0.00247957730529186\\
80	0.00247961141136047\\
81	0.00247964610305934\\
82	0.00247968139046533\\
83	0.00247971728382847\\
84	0.00247975379357496\\
85	0.00247979093031016\\
86	0.00247982870482167\\
87	0.00247986712808238\\
88	0.00247990621125367\\
89	0.00247994596568866\\
90	0.00247998640293537\\
91	0.00248002753474013\\
92	0.0024800693730509\\
93	0.0024801119300207\\
94	0.00248015521801113\\
95	0.00248019924959592\\
96	0.00248024403756445\\
97	0.00248028959492547\\
98	0.00248033593491084\\
99	0.00248038307097927\\
100	0.00248043101682014\\
101	0.00248047978635751\\
102	0.00248052939375398\\
103	0.00248057985341478\\
104	0.0024806311799919\\
105	0.00248068338838818\\
106	0.00248073649376166\\
107	0.00248079051152982\\
108	0.00248084545737399\\
109	0.00248090134724376\\
110	0.00248095819736159\\
111	0.00248101602422742\\
112	0.00248107484462324\\
113	0.00248113467561801\\
114	0.00248119553457237\\
115	0.0024812574391437\\
116	0.00248132040729099\\
117	0.00248138445728006\\
118	0.00248144960768869\\
119	0.00248151587741186\\
120	0.0024815832856672\\
121	0.00248165185200038\\
122	0.00248172159629065\\
123	0.00248179253875655\\
124	0.00248186469996156\\
125	0.00248193810082007\\
126	0.00248201276260318\\
127	0.00248208870694486\\
128	0.00248216595584803\\
129	0.00248224453169084\\
130	0.00248232445723302\\
131	0.00248240575562236\\
132	0.0024824884504013\\
133	0.00248257256551368\\
134	0.0024826581253114\\
135	0.00248274515456154\\
136	0.00248283367845326\\
137	0.00248292372260504\\
138	0.00248301531307206\\
139	0.00248310847635362\\
140	0.00248320323940094\\
141	0.00248329962962534\\
142	0.00248339767490694\\
143	0.0024834974036043\\
144	0.00248359884456432\\
145	0.00248370202713047\\
146	0.0024838069811441\\
147	0.00248391373694049\\
148	0.00248402232537194\\
149	0.00248413277781647\\
150	0.00248424512618674\\
151	0.00248435940293909\\
152	0.00248447564108274\\
153	0.00248459387418913\\
154	0.00248471413640142\\
155	0.00248483646244414\\
156	0.00248496088763306\\
157	0.00248508744788513\\
158	0.00248521617972863\\
159	0.00248534712031352\\
160	0.00248548030742194\\
161	0.00248561577947882\\
162	0.0024857535755628\\
163	0.00248589373541721\\
164	0.00248603629946128\\
165	0.00248618130880154\\
166	0.0024863288052434\\
167	0.00248647883130295\\
168	0.00248663143021888\\
169	0.00248678664596469\\
170	0.00248694452326102\\
171	0.00248710510758828\\
172	0.00248726844519937\\
173	0.00248743458313271\\
174	0.00248760356922545\\
175	0.00248777545212685\\
176	0.00248795028131197\\
177	0.0024881281070955\\
178	0.00248830898064589\\
179	0.00248849295399968\\
180	0.00248868008007601\\
181	0.00248887041269149\\
182	0.00248906400657519\\
183	0.002489260917384\\
184	0.00248946120171807\\
185	0.00248966491713671\\
186	0.00248987212217436\\
187	0.00249008287635696\\
188	0.0024902972402185\\
189	0.0024905152753179\\
190	0.00249073704425609\\
191	0.00249096261069343\\
192	0.00249119203936747\\
193	0.00249142539611082\\
194	0.00249166274786942\\
195	0.00249190416272117\\
196	0.00249214970989475\\
197	0.00249239945978876\\
198	0.00249265348399125\\
199	0.00249291185529943\\
200	0.0024931746477399\\
201	0.00249344193658893\\
202	0.0024937137983933\\
203	0.00249399031099142\\
204	0.00249427155353465\\
205	0.00249455760650921\\
206	0.00249484855175818\\
207	0.002495144472504\\
208	0.00249544545337136\\
209	0.0024957515804103\\
210	0.00249606294111987\\
211	0.00249637962447197\\
212	0.00249670172093573\\
213	0.00249702932250214\\
214	0.00249736252270925\\
215	0.0024977014166675\\
216	0.00249804610108577\\
217	0.00249839667429747\\
218	0.00249875323628743\\
219	0.00249911588871892\\
220	0.00249948473496118\\
221	0.00249985988011741\\
222	0.00250024143105315\\
223	0.00250062949642516\\
224	0.00250102418671064\\
225	0.002501425614237\\
226	0.00250183389321195\\
227	0.00250224913975434\\
228	0.00250267147192506\\
229	0.00250310100975868\\
230	0.00250353787529553\\
231	0.0025039821926142\\
232	0.00250443408786451\\
233	0.00250489368930104\\
234	0.0025053611273171\\
235	0.0025058365344792\\
236	0.00250632004556201\\
237	0.00250681179758382\\
238	0.00250731192984247\\
239	0.00250782058395191\\
240	0.00250833790387907\\
241	0.00250886403598137\\
242	0.00250939912904469\\
243	0.00250994333432182\\
244	0.00251049680557143\\
245	0.00251105969909756\\
246	0.00251163217378957\\
247	0.00251221439116256\\
248	0.00251280651539834\\
249	0.00251340871338679\\
250	0.00251402115476774\\
251	0.0025146440119733\\
252	0.00251527746027064\\
253	0.00251592167780518\\
254	0.00251657684564421\\
255	0.00251724314782092\\
256	0.00251792077137885\\
257	0.00251860990641661\\
258	0.00251931074613306\\
259	0.00252002348687272\\
260	0.00252074832817157\\
261	0.00252148547280303\\
262	0.00252223512682423\\
263	0.00252299749962257\\
264	0.00252377280396222\\
265	0.00252456125603103\\
266	0.00252536307548745\\
267	0.00252617848550731\\
268	0.00252700771283079\\
269	0.00252785098780889\\
270	0.00252870854444927\\
271	0.00252958062045988\\
272	0.00253046745728757\\
273	0.00253136930014451\\
274	0.00253228639800854\\
275	0.00253321900358004\\
276	0.00253416737321523\\
277	0.00253513176705866\\
278	0.00253611244993631\\
279	0.00253710969076723\\
280	0.00253812376256445\\
281	0.00253915494247581\\
282	0.00254020351182379\\
283	0.00254126975614418\\
284	0.00254235396522361\\
285	0.00254345643313563\\
286	0.0025445774582754\\
287	0.00254571734339269\\
288	0.00254687639562325\\
289	0.00254805492651808\\
290	0.00254925325207081\\
291	0.00255047169274272\\
292	0.00255171057348544\\
293	0.00255297022376099\\
294	0.00255425097755892\\
295	0.00255555317341061\\
296	0.00255687715440021\\
297	0.00255822326817209\\
298	0.00255959186693452\\
299	0.0025609833074594\\
300	0.00256239795107767\\
301	0.00256383616367012\\
302	0.00256529831565325\\
303	0.00256678478196009\\
304	0.00256829594201556\\
305	0.00256983217970675\\
306	0.00257139388334885\\
307	0.00257298144564973\\
308	0.00257459526368051\\
309	0.00257623573886992\\
310	0.00257790327705853\\
311	0.00257959828867543\\
312	0.00258132118909213\\
313	0.00258307239899605\\
314	0.00258485234379022\\
315	0.00258666145005346\\
316	0.00258850014697706\\
317	0.0025903688672863\\
318	0.00259226804708924\\
319	0.0025941981257115\\
320	0.00259615954551571\\
321	0.00259815275170518\\
322	0.0026001781921106\\
323	0.00260223631695857\\
324	0.00260432757862129\\
325	0.00260645243134581\\
326	0.00260861133096183\\
327	0.00261080473456659\\
328	0.00261303310018548\\
329	0.00261529688640687\\
330	0.00261759655198955\\
331	0.00261993255544103\\
332	0.00262230535456489\\
333	0.00262471540597562\\
334	0.00262716316457854\\
335	0.00262964908301356\\
336	0.00263217361106048\\
337	0.00263473719500448\\
338	0.00263734027696049\\
339	0.00263998329415559\\
340	0.00264266667816952\\
341	0.00264539085413462\\
342	0.0026481562398981\\
343	0.00265096324515207\\
344	0.00265381227054041\\
345	0.00265670370675562\\
346	0.00265963793364637\\
347	0.00266261531936403\\
348	0.00266563621958899\\
349	0.00266870097689276\\
350	0.00267180992031164\\
351	0.00267496336523225\\
352	0.00267816161371457\\
353	0.00268140495539548\\
354	0.00268469366908969\\
355	0.00268802802506198\\
356	0.00269140828756429\\
357	0.00269483471685336\\
358	0.00269830757285886\\
359	0.00270182714513143\\
360	0.00270539380062159\\
361	0.00270900798273177\\
362	0.00271267024811181\\
363	0.00271638131646151\\
364	0.00272014213838986\\
365	0.00272395399022021\\
366	0.00272781861511963\\
367	0.00273173846273063\\
368	0.00273571718681733\\
369	0.00273976151590172\\
370	0.00274387685410735\\
371	0.0027480652175805\\
372	0.0027523284555114\\
373	0.00275666849663476\\
374	0.00276108735077459\\
375	0.00276558710992141\\
376	0.0027701699490338\\
377	0.00277483812421509\\
378	0.00277959394645428\\
379	0.00278443959488847\\
380	0.00278937597696776\\
381	0.00279440505860262\\
382	0.00279952987475586\\
383	0.00280475368908226\\
384	0.00281008002083447\\
385	0.0028155126757165\\
386	0.0028210557813721\\
387	0.00282671382833309\\
388	0.00283249171741898\\
389	0.00283839481478436\\
390	0.00284442901606047\\
391	0.00285060082134511\\
392	0.00285691742317362\\
393	0.00286338681006984\\
394	0.00287001788885343\\
395	0.00287682062959411\\
396	0.00288380623799019\\
397	0.00289098736105367\\
398	0.00289837833336593\\
399	0.00290599547291058\\
400	0.00291385743772171\\
401	0.00292198565753793\\
402	0.00293040485881307\\
403	0.00293914370801658\\
404	0.00294823561058515\\
405	0.00295771973216965\\
406	0.0029676423972256\\
407	0.00297805938066788\\
408	0.00298904182625361\\
409	0.00300066528391573\\
410	0.00301301628128928\\
411	0.00302619752181333\\
412	0.00304033140484477\\
413	0.00305556435770621\\
414	0.00307207207453714\\
415	0.00309006482955999\\
416	0.00310880661883055\\
417	0.00312785605903395\\
418	0.00314721714365967\\
419	0.00316689384652706\\
420	0.00318689014002555\\
421	0.00320721005795063\\
422	0.00322785787452478\\
423	0.00324883855072928\\
424	0.0032701585118829\\
425	0.003291823707487\\
426	0.0033138379259991\\
427	0.0033362048377587\\
428	0.00335892797542417\\
429	0.00338201071198679\\
430	0.00340545623601502\\
431	0.00342926752372865\\
432	0.00345344730742967\\
433	0.00347799803973121\\
434	0.00350292185292983\\
435	0.00352822051275047\\
436	0.00355389536554866\\
437	0.00357994727788222\\
438	0.00360637656715544\\
439	0.00363318292178348\\
440	0.0036603653090112\\
441	0.00368792186812286\\
442	0.00371584978624311\\
443	0.0037441451531795\\
444	0.00377280279012463\\
445	0.00380181604259669\\
446	0.00383117651323895\\
447	0.00386087365251626\\
448	0.00389089388583082\\
449	0.00392121690199829\\
450	0.00395179716637846\\
451	0.00398259838267485\\
452	0.00401358468378152\\
453	0.00404471265061179\\
454	0.00407592977923796\\
455	0.00410717277754148\\
456	0.00413836605779342\\
457	0.00416942179077619\\
458	0.00420024634456966\\
459	0.00423081211061245\\
460	0.00426140020028307\\
461	0.00429200101853043\\
462	0.00432253003295564\\
463	0.0043528838050062\\
464	0.00438293598073556\\
465	0.0044125325848898\\
466	0.0044414859553579\\
467	0.00446956698539481\\
468	0.00449649542059098\\
469	0.00452192790322189\\
470	0.00454544332458644\\
471	0.00456660753965114\\
472	0.00458814729311082\\
473	0.0046100682754802\\
474	0.00463237617641393\\
475	0.00465507664007141\\
476	0.00467817529335925\\
477	0.00470167773876455\\
478	0.00472558954445719\\
479	0.00474991623144585\\
480	0.00477466327102612\\
481	0.00479983614302007\\
482	0.00482544066736576\\
483	0.00485148331035823\\
484	0.00487797072227519\\
485	0.00490490972166369\\
486	0.00493230688573686\\
487	0.00496016866117437\\
488	0.00498850152122487\\
489	0.00501731171296575\\
490	0.00504660464559965\\
491	0.00507638424155821\\
492	0.0051066539743321\\
493	0.00513741754812156\\
494	0.00516867868920914\\
495	0.00520044483843773\\
496	0.00523272670992935\\
497	0.00526553942337987\\
498	0.00529890398854494\\
499	0.00533285017825539\\
500	0.00536740912600212\\
501	0.00540262192716388\\
502	0.00543854471967436\\
503	0.0054752551006337\\
504	0.00551286051987373\\
505	0.0055515092381846\\
506	0.00559130719092906\\
507	0.00563230368435302\\
508	0.00567455068067331\\
509	0.00571810435251307\\
510	0.00576302553152001\\
511	0.00580939070107368\\
512	0.00585727370026442\\
513	0.00590674250978723\\
514	0.005957854838526\\
515	0.00601065214427572\\
516	0.00606515167623523\\
517	0.00612133597063666\\
518	0.00617913868276773\\
519	0.00623842574325406\\
520	0.00629897040266387\\
521	0.00636042026026284\\
522	0.00642121846438805\\
523	0.00648079970694343\\
524	0.00653898423739954\\
525	0.00659558459818578\\
526	0.00665040955394938\\
527	0.00670326996925213\\
528	0.00675398737755338\\
529	0.00680240615739019\\
530	0.00684841085110631\\
531	0.00689195017212662\\
532	0.00693307314023976\\
533	0.00697237816780817\\
534	0.00701119533494809\\
535	0.00704952235122946\\
536	0.00708736978307209\\
537	0.00712476342145339\\
538	0.0071617466085178\\
539	0.00719838229666012\\
540	0.00723475449016275\\
541	0.00727096855127332\\
542	0.00730714997322877\\
543	0.00734344027894289\\
544	0.00737998840528387\\
545	0.00741688811617026\\
546	0.00745418951685885\\
547	0.00749193730271656\\
548	0.00753017973388053\\
549	0.00756896791481386\\
550	0.00760835477395233\\
551	0.00764839371723731\\
552	0.00768913694195016\\
553	0.00773063378523078\\
554	0.00777292424702521\\
555	0.00781603521840035\\
556	0.00785998739163325\\
557	0.00790479871128239\\
558	0.00795048664294056\\
559	0.00799706792415065\\
560	0.00804455829354392\\
561	0.00809297220498276\\
562	0.00814232253144814\\
563	0.00819262028262012\\
564	0.00824387435428926\\
565	0.00829609133116553\\
566	0.00834927536349354\\
567	0.00840342812126299\\
568	0.00845854884012727\\
569	0.00851463422662447\\
570	0.00857167829860371\\
571	0.00862967217526905\\
572	0.00868860424489086\\
573	0.00874845809866899\\
574	0.00880921036058856\\
575	0.00887082894270814\\
576	0.00893328022478691\\
577	0.00899649525040652\\
578	0.00906043870669162\\
579	0.00912517915203537\\
580	0.00919005656715616\\
581	0.00925329200342372\\
582	0.00931456036497874\\
583	0.00937352397491844\\
584	0.00942923087906576\\
585	0.00948220461794547\\
586	0.00953271440881462\\
587	0.00958044208054697\\
588	0.00962596473294314\\
589	0.00966992069233143\\
590	0.00971269563307853\\
591	0.00975441316854107\\
592	0.00979516557260156\\
593	0.00983492675369396\\
594	0.00987369348216993\\
595	0.00991118387968948\\
596	0.00994658651044256\\
597	0.00997788999445116\\
598	0.010000292044645\\
599	0\\
600	0\\
};
\addplot [color=blue,solid,forget plot]
  table[row sep=crcr]{%
1	0.000300501426608031\\
2	0.000300519631885623\\
3	0.000300538148804544\\
4	0.000300556982713254\\
5	0.000300576139051885\\
6	0.000300595623353805\\
7	0.000300615441247206\\
8	0.000300635598456722\\
9	0.000300656100805079\\
10	0.000300676954214768\\
11	0.000300698164709764\\
12	0.000300719738417195\\
13	0.0003007416815692\\
14	0.000300764000504607\\
15	0.000300786701670854\\
16	0.00030080979162578\\
17	0.000300833277039507\\
18	0.000300857164696399\\
19	0.000300881461496977\\
20	0.000300906174459875\\
21	0.000300931310723911\\
22	0.000300956877550082\\
23	0.00030098288232366\\
24	0.000301009332556328\\
25	0.000301036235888336\\
26	0.000301063600090647\\
27	0.000301091433067201\\
28	0.000301119742857146\\
29	0.000301148537637201\\
30	0.000301177825723917\\
31	0.000301207615576129\\
32	0.000301237915797319\\
33	0.000301268735138103\\
34	0.000301300082498759\\
35	0.000301331966931708\\
36	0.000301364397644179\\
37	0.000301397384000774\\
38	0.000301430935526176\\
39	0.000301465061907863\\
40	0.000301499772998864\\
41	0.00030153507882061\\
42	0.000301570989565755\\
43	0.000301607515601091\\
44	0.00030164466747051\\
45	0.000301682455898007\\
46	0.000301720891790759\\
47	0.000301759986242223\\
48	0.00030179975053527\\
49	0.000301840196145408\\
50	0.000301881334744117\\
51	0.000301923178202066\\
52	0.000301965738592561\\
53	0.000302009028194973\\
54	0.000302053059498194\\
55	0.000302097845204237\\
56	0.000302143398231815\\
57	0.000302189731720019\\
58	0.000302236859032043\\
59	0.000302284793759027\\
60	0.000302333549723835\\
61	0.000302383140985043\\
62	0.000302433581840894\\
63	0.000302484886833373\\
64	0.000302537070752326\\
65	0.000302590148639623\\
66	0.000302644135793482\\
67	0.00030269904777273\\
68	0.000302754900401292\\
69	0.000302811709772578\\
70	0.000302869492254094\\
71	0.000302928264492068\\
72	0.000302988043416155\\
73	0.000303048846244216\\
74	0.000303110690487172\\
75	0.000303173593953979\\
76	0.00030323757475667\\
77	0.00030330265131539\\
78	0.000303368842363687\\
79	0.000303436166953741\\
80	0.000303504644461784\\
81	0.000303574294593534\\
82	0.000303645137389719\\
83	0.000303717193231815\\
84	0.000303790482847713\\
85	0.000303865027317568\\
86	0.00030394084807977\\
87	0.000304017966936938\\
88	0.000304096406062103\\
89	0.000304176188004911\\
90	0.000304257335697963\\
91	0.000304339872463302\\
92	0.000304423822018948\\
93	0.000304509208485584\\
94	0.000304596056393289\\
95	0.000304684390688489\\
96	0.000304774236740935\\
97	0.000304865620350862\\
98	0.000304958567756184\\
99	0.000305053105639937\\
100	0.000305149261137713\\
101	0.000305247061845276\\
102	0.000305346535826389\\
103	0.000305447711620587\\
104	0.000305550618251278\\
105	0.000305655285233842\\
106	0.000305761742583948\\
107	0.00030587002082592\\
108	0.000305980151001394\\
109	0.000306092164677971\\
110	0.000306206093958081\\
111	0.000306321971488003\\
112	0.000306439830467069\\
113	0.000306559704656874\\
114	0.000306681628390895\\
115	0.000306805636583991\\
116	0.000306931764742322\\
117	0.000307060048973214\\
118	0.000307190525995378\\
119	0.000307323233149159\\
120	0.000307458208407031\\
121	0.000307595490384293\\
122	0.000307735118349858\\
123	0.000307877132237315\\
124	0.0003080215726561\\
125	0.000308168480902933\\
126	0.000308317898973392\\
127	0.000308469869573712\\
128	0.000308624436132747\\
129	0.000308781642814205\\
130	0.000308941534529002\\
131	0.000309104156947894\\
132	0.00030926955651428\\
133	0.00030943778045723\\
134	0.000309608876804769\\
135	0.000309782894397335\\
136	0.000309959882901501\\
137	0.000310139892823947\\
138	0.00031032297552562\\
139	0.000310509183236287\\
140	0.000310698569069238\\
141	0.000310891187036353\\
142	0.000311087092063406\\
143	0.000311286340005505\\
144	0.000311488987662527\\
145	0.000311695092794237\\
146	0.000311904714135457\\
147	0.000312117911414112\\
148	0.000312334745368019\\
149	0.000312555277761922\\
150	0.000312779571404839\\
151	0.000313007690167653\\
152	0.000313239699001119\\
153	0.000313475663954089\\
154	0.000313715652191992\\
155	0.000313959732015797\\
156	0.000314207972881154\\
157	0.000314460445417914\\
158	0.00031471722145005\\
159	0.000314978374015754\\
160	0.000315243977388053\\
161	0.000315514107095696\\
162	0.000315788839944392\\
163	0.000316068254038434\\
164	0.000316352428802739\\
165	0.000316641445005159\\
166	0.000316935384779305\\
167	0.000317234331647649\\
168	0.000317538370545133\\
169	0.000317847587843084\\
170	0.000318162071373661\\
171	0.000318481910454595\\
172	0.000318807195914479\\
173	0.000319138020118415\\
174	0.000319474476994141\\
175	0.000319816662058643\\
176	0.000320164672445176\\
177	0.000320518606930806\\
178	0.000320878565964381\\
179	0.000321244651695082\\
180	0.00032161696800138\\
181	0.000321995620520563\\
182	0.000322380716678798\\
183	0.000322772365721634\\
184	0.000323170678745169\\
185	0.000323575768727642\\
186	0.000323987750561753\\
187	0.000324406741087356\\
188	0.000324832859124922\\
189	0.000325266225509471\\
190	0.000325706963125168\\
191	0.000326155196940573\\
192	0.000326611054044423\\
193	0.000327074663682172\\
194	0.000327546157293139\\
195	0.000328025668548302\\
196	0.00032851333338885\\
197	0.000329009290065364\\
198	0.000329513679177751\\
199	0.000330026643715947\\
200	0.000330548329101246\\
201	0.00033107888322853\\
202	0.000331618456509183\\
203	0.000332167201914825\\
204	0.00033272527502188\\
205	0.00033329283405693\\
206	0.000333870039942939\\
207	0.000334457056346362\\
208	0.000335054049725074\\
209	0.00033566118937726\\
210	0.000336278647491166\\
211	0.000336906599195865\\
212	0.000337545222612936\\
213	0.00033819469890916\\
214	0.000338855212350104\\
215	0.000339526950354936\\
216	0.000340210103552097\\
217	0.000340904865836186\\
218	0.00034161143442579\\
219	0.000342330009922658\\
220	0.000343060796371776\\
221	0.000343804001322807\\
222	0.000344559835892605\\
223	0.000345328514829062\\
224	0.000346110256576102\\
225	0.000346905283340077\\
226	0.000347713821157424\\
227	0.000348536099963604\\
228	0.000349372353663567\\
229	0.000350222820203491\\
230	0.000351087741644082\\
231	0.000351967364235264\\
232	0.000352861938492422\\
233	0.000353771719274236\\
234	0.00035469696586207\\
235	0.000355637942040964\\
236	0.000356594916182395\\
237	0.000357568161328625\\
238	0.000358557955278963\\
239	0.000359564580677665\\
240	0.000360588325103814\\
241	0.00036162948116306\\
242	0.000362688346581259\\
243	0.000363765224300177\\
244	0.00036486042257567\\
245	0.000365974255075711\\
246	0.000367107040985112\\
247	0.000368259105107945\\
248	0.000369430777974456\\
249	0.000370622395950006\\
250	0.00037183430134628\\
251	0.000373066842535078\\
252	0.000374320374064625\\
253	0.000375595256778484\\
254	0.000376891857937245\\
255	0.000378210551342952\\
256	0.000379551717466371\\
257	0.000380915743577281\\
258	0.000382303023877824\\
259	0.000383713959638895\\
260	0.000385148959339871\\
261	0.000386608438811652\\
262	0.000388092821383163\\
263	0.000389602538031377\\
264	0.000391138027535063\\
265	0.0003926997366323\\
266	0.000394288120181847\\
267	0.000395903641328527\\
268	0.000397546771672606\\
269	0.000399217991443161\\
270	0.000400917789675084\\
271	0.00040264666438957\\
272	0.000404405122777409\\
273	0.000406193681385395\\
274	0.000408012866308006\\
275	0.000409863213390112\\
276	0.000411745268452753\\
277	0.000413659587586473\\
278	0.000415606737301452\\
279	0.000417587294749511\\
280	0.000419601847956675\\
281	0.000421650996062257\\
282	0.000423735349564715\\
283	0.000425855530574405\\
284	0.000428012173073514\\
285	0.000430205923183494\\
286	0.000432437439440065\\
287	0.000434707393076316\\
288	0.000437016468313859\\
289	0.000439365362662696\\
290	0.000441754787229672\\
291	0.00044418546703613\\
292	0.000446658141344895\\
293	0.000449173563996945\\
294	0.000451732503758152\\
295	0.000454335744676249\\
296	0.000456984086448427\\
297	0.000459678344800051\\
298	0.000462419351874562\\
299	0.000465207956635017\\
300	0.000468045025277779\\
301	0.000470931441658408\\
302	0.000473868107730537\\
303	0.000476855943997822\\
304	0.000479895889980021\\
305	0.000482988904693828\\
306	0.000486135967150064\\
307	0.000489338076869641\\
308	0.000492596254421141\\
309	0.00049591154198332\\
310	0.000499285003931393\\
311	0.000502717727435768\\
312	0.000506210823043187\\
313	0.000509765425183453\\
314	0.000513382692465421\\
315	0.000517063808503682\\
316	0.000520809982684537\\
317	0.000524622450834215\\
318	0.000528502475904416\\
319	0.000532451348675109\\
320	0.000536470388474993\\
321	0.000540560943919258\\
322	0.000544724393664747\\
323	0.000548962147182398\\
324	0.000553275645546648\\
325	0.000557666362241569\\
326	0.000562135803983405\\
327	0.000566685511558842\\
328	0.000571317060678727\\
329	0.000576032062846095\\
330	0.000580832166237965\\
331	0.000585719056599692\\
332	0.000590694458150562\\
333	0.000595760134499168\\
334	0.000600917889566737\\
335	0.000606169568516076\\
336	0.000611517058683465\\
337	0.000616962290510151\\
338	0.000622507238469203\\
339	0.000628153921982653\\
340	0.000633904406322138\\
341	0.000639760803484944\\
342	0.0006457252730346\\
343	0.000651800022892481\\
344	0.000657987310062489\\
345	0.000664289441266136\\
346	0.000670708773457938\\
347	0.000677247714182177\\
348	0.000683908721719769\\
349	0.000690694304957674\\
350	0.000697607022890975\\
351	0.000704649483637008\\
352	0.000711824342799276\\
353	0.00071913430096381\\
354	0.000726582100049746\\
355	0.000734170518191728\\
356	0.000741902362825378\\
357	0.000749780461643887\\
358	0.000757807652416314\\
359	0.00076598676939214\\
360	0.00077432061904643\\
361	0.000782811951648188\\
362	0.000791463422611622\\
363	0.00080027753796957\\
364	0.000809256571552262\\
365	0.000818402422494648\\
366	0.000827716324432034\\
367	0.000837198121641135\\
368	0.000846843886573104\\
369	0.00085621759755801\\
370	0.000865719835157762\\
371	0.000875384922099738\\
372	0.000885215919596194\\
373	0.000895215956711118\\
374	0.000905388232153993\\
375	0.000915736015490102\\
376	0.000926262646707834\\
377	0.000936971533698378\\
378	0.000947866152129432\\
379	0.000958949985040097\\
380	0.000970226771537842\\
381	0.000981700472945479\\
382	0.000993375175281812\\
383	0.00100525509678743\\
384	0.00101734459611838\\
385	0.00102964818126044\\
386	0.00104217051921655\\
387	0.00105491644651524\\
388	0.00106789098058222\\
389	0.00108109933200473\\
390	0.00109454691770115\\
391	0.00110823937498109\\
392	0.00112218257644393\\
393	0.00113638264561016\\
394	0.0011508459731072\\
395	0.00116557923313257\\
396	0.00118058939978738\\
397	0.00119588376270498\\
398	0.00121146994119191\\
399	0.00122735589585838\\
400	0.00124354993647432\\
401	0.00126006072461482\\
402	0.00127689726966284\\
403	0.00129406891689551\\
404	0.00131158532557222\\
405	0.00132945642701809\\
406	0.00134769232504865\\
407	0.00136630334920238\\
408	0.00138529863967356\\
409	0.0014046865214009\\
410	0.00142447447639833\\
411	0.00144466876411519\\
412	0.00146527380807004\\
413	0.00148629236136712\\
414	0.00150772242408549\\
415	0.00152955174845827\\
416	0.00155178053429491\\
417	0.00157441738488319\\
418	0.00159747117000727\\
419	0.0016209510473805\\
420	0.00164486649233153\\
421	0.00166922733669089\\
422	0.00169404381114076\\
423	0.00171932658913223\\
424	0.00174508660911482\\
425	0.00177133497272887\\
426	0.00179808317403233\\
427	0.00182534312284982\\
428	0.00185312717023215\\
429	0.00188144813627173\\
430	0.00191031934055164\\
431	0.00193975463554364\\
432	0.00196976844331324\\
433	0.00200037579593989\\
434	0.00203159238011775\\
435	0.00206343458646723\\
436	0.00209591956415803\\
437	0.00212906528151442\\
438	0.00216289059332055\\
439	0.00219741531550776\\
440	0.00223266030762541\\
441	0.00226864756255277\\
442	0.00230540030025111\\
443	0.00234294305536536\\
444	0.00238130173134972\\
445	0.00242050355843802\\
446	0.00246057688303989\\
447	0.00250155151506867\\
448	0.00254345254983823\\
449	0.0025863163109077\\
450	0.00263018519724896\\
451	0.00267510706601041\\
452	0.00272113538969605\\
453	0.00276833032719836\\
454	0.00281676004265518\\
455	0.00286650228071771\\
456	0.00291764598612343\\
457	0.00297029183945389\\
458	0.00302454769034921\\
459	0.00305055853264484\\
460	0.0030717054879797\\
461	0.00309373699915778\\
462	0.00311676704613136\\
463	0.00314094401885549\\
464	0.00316642522675929\\
465	0.00319338423266275\\
466	0.00322202684986058\\
467	0.00325259823185108\\
468	0.00328539262995512\\
469	0.00332076001564318\\
470	0.00335911688018843\\
471	0.00340087897834244\\
472	0.00344329284598584\\
473	0.00348636279786332\\
474	0.00353009234460264\\
475	0.00357448410288194\\
476	0.00361953969633514\\
477	0.00366525962840545\\
478	0.00371164300374218\\
479	0.00375868663529404\\
480	0.00380638141616894\\
481	0.00385469345016474\\
482	0.00390358045788171\\
483	0.00395302257859817\\
484	0.00400299667579476\\
485	0.00405347626436221\\
486	0.0041044320174268\\
487	0.00415583392408783\\
488	0.00420765852502683\\
489	0.00425991707445931\\
490	0.00431270042363151\\
491	0.00436601211529089\\
492	0.0044197932248762\\
493	0.00447397060726542\\
494	0.00452845408490524\\
495	0.00458313250642518\\
496	0.00463786904438652\\
497	0.00469249538226218\\
498	0.00474680451832123\\
499	0.00480054107464478\\
500	0.00485339074819056\\
501	0.00490496638135537\\
502	0.00495479087986692\\
503	0.00500227689936128\\
504	0.00504670127209525\\
505	0.00508717390834856\\
506	0.00512773714340334\\
507	0.00516850937848077\\
508	0.00520938472009274\\
509	0.00525054066415425\\
510	0.00529261159274347\\
511	0.00533562078544122\\
512	0.00537959235534939\\
513	0.00542455163858445\\
514	0.00547052605096188\\
515	0.00551754726203399\\
516	0.00556563622915616\\
517	0.00561481300511334\\
518	0.00566509525535443\\
519	0.00571650047162964\\
520	0.00576904879683756\\
521	0.00582276570887622\\
522	0.00587769973423755\\
523	0.00593390604542822\\
524	0.00599143941593479\\
525	0.00605035287192922\\
526	0.00611069585636212\\
527	0.00617251175254849\\
528	0.00623583452787127\\
529	0.00630068445836571\\
530	0.00636706283199385\\
531	0.00643494485661551\\
532	0.00650411822733752\\
533	0.00657373800204402\\
534	0.0066422614192598\\
535	0.00670949828702219\\
536	0.0067752479697538\\
537	0.00683930310358746\\
538	0.00690145550474471\\
539	0.00696150512625364\\
540	0.00701927321286701\\
541	0.00707462004096888\\
542	0.00712745146192194\\
543	0.0071777532813973\\
544	0.00722564490142717\\
545	0.00727260397683621\\
546	0.00731915191368715\\
547	0.00736529988918563\\
548	0.00741107522451367\\
549	0.007456523672545\\
550	0.00750171124054802\\
551	0.00754672506984011\\
552	0.00759167265990017\\
553	0.00763667928142201\\
554	0.00768188692804579\\
555	0.00772745204322683\\
556	0.00777352896986929\\
557	0.00782017594083521\\
558	0.00786741630291251\\
559	0.00791527588252588\\
560	0.00796378264304856\\
561	0.00801296618049495\\
562	0.00806285723861679\\
563	0.00811348655083227\\
564	0.00816488335329511\\
565	0.0082170736010332\\
566	0.00827007798995417\\
567	0.00832390999756632\\
568	0.00837857422847543\\
569	0.00843407061551391\\
570	0.0084903967595156\\
571	0.00854754764423447\\
572	0.00860551529537557\\
573	0.00866428858236032\\
574	0.00872385303281798\\
575	0.00878419052251876\\
576	0.00884527895369516\\
577	0.00890709236790587\\
578	0.00896959986638077\\
579	0.00903276347919527\\
580	0.00909654468053119\\
581	0.00916091558216332\\
582	0.00922584619394916\\
583	0.00929124381478611\\
584	0.00935712289323052\\
585	0.00942180913543368\\
586	0.00948442867380405\\
587	0.00954462375269818\\
588	0.00960141065634341\\
589	0.00965385408934247\\
590	0.00970260651805489\\
591	0.00974846666295701\\
592	0.00979195816415004\\
593	0.00983364285089848\\
594	0.00987334736606283\\
595	0.0099111519251561\\
596	0.00994658651044256\\
597	0.00997788999445116\\
598	0.010000292044645\\
599	0\\
600	0\\
};
\addplot [color=mycolor10,solid,forget plot]
  table[row sep=crcr]{%
1	4.64198342570775e-05\\
2	4.64200387921881e-05\\
3	4.64202468284378e-05\\
4	4.64204584259058e-05\\
5	4.64206736456894e-05\\
6	4.64208925499346e-05\\
7	4.6421115201856e-05\\
8	4.64213416657481e-05\\
9	4.64215720070079e-05\\
10	4.64218062921487e-05\\
11	4.6422044588823e-05\\
12	4.64222869658486e-05\\
13	4.64225334932136e-05\\
14	4.64227842421049e-05\\
15	4.64230392849374e-05\\
16	4.64232986953489e-05\\
17	4.64235625482508e-05\\
18	4.6423830919835e-05\\
19	4.64241038875909e-05\\
20	4.6424381530333e-05\\
21	4.64246639282335e-05\\
22	4.64249511628312e-05\\
23	4.64252433170587e-05\\
24	4.64255404752666e-05\\
25	4.64258427232428e-05\\
26	4.64261501482564e-05\\
27	4.64264628390569e-05\\
28	4.6426780885913e-05\\
29	4.64271043806428e-05\\
30	4.64274334166229e-05\\
31	4.64277680888393e-05\\
32	4.6428108493887e-05\\
33	4.64284547300268e-05\\
34	4.64288068971803e-05\\
35	4.64291650969916e-05\\
36	4.6429529432838e-05\\
37	4.64299000098569e-05\\
38	4.64302769349861e-05\\
39	4.64306603169893e-05\\
40	4.64310502664832e-05\\
41	4.64314468959798e-05\\
42	4.64318503199062e-05\\
43	4.64322606546443e-05\\
44	4.64326780185624e-05\\
45	4.64331025320574e-05\\
46	4.64335343175637e-05\\
47	4.64339734996173e-05\\
48	4.64344202048737e-05\\
49	4.6434874562151e-05\\
50	4.6435336702462e-05\\
51	4.64358067590574e-05\\
52	4.64362848674592e-05\\
53	4.6436771165494e-05\\
54	4.64372657933456e-05\\
55	4.64377688935835e-05\\
56	4.64382806112103e-05\\
57	4.64388010936927e-05\\
58	4.6439330491014e-05\\
59	4.64398689557109e-05\\
60	4.64404166429274e-05\\
61	4.6440973710437e-05\\
62	4.64415403187029e-05\\
63	4.64421166309248e-05\\
64	4.64427028130801e-05\\
65	4.64432990339679e-05\\
66	4.64439054652635e-05\\
67	4.64445222815589e-05\\
68	4.64451496604215e-05\\
69	4.64457877824375e-05\\
70	4.64464368312647e-05\\
71	4.64470969936825e-05\\
72	4.64477684596473e-05\\
73	4.64484514223477e-05\\
74	4.64491460782499e-05\\
75	4.64498526271697e-05\\
76	4.64505712723039e-05\\
77	4.64513022203158e-05\\
78	4.64520456813705e-05\\
79	4.64528018692056e-05\\
80	4.64535710011883e-05\\
81	4.64543532983768e-05\\
82	4.64551489855851e-05\\
83	4.645595829144e-05\\
84	4.64567814484457e-05\\
85	4.64576186930545e-05\\
86	4.64584702657316e-05\\
87	4.64593364110224e-05\\
88	4.64602173776148e-05\\
89	4.64611134184232e-05\\
90	4.64620247906381e-05\\
91	4.64629517558166e-05\\
92	4.64638945799521e-05\\
93	4.64648535335428e-05\\
94	4.64658288916688e-05\\
95	4.646682093407e-05\\
96	4.64678299452291e-05\\
97	4.64688562144476e-05\\
98	4.64699000359229e-05\\
99	4.64709617088292e-05\\
100	4.64720415374174e-05\\
101	4.64731398310753e-05\\
102	4.64742569044417e-05\\
103	4.64753930774618e-05\\
104	4.64765486755127e-05\\
105	4.64777240294628e-05\\
106	4.6478919475784e-05\\
107	4.64801353566316e-05\\
108	4.64813720199543e-05\\
109	4.64826298195808e-05\\
110	4.6483909115327e-05\\
111	4.64852102730836e-05\\
112	4.64865336649245e-05\\
113	4.6487879669222e-05\\
114	4.64892486707316e-05\\
115	4.6490641060719e-05\\
116	4.64920572370491e-05\\
117	4.64934976043213e-05\\
118	4.64949625739567e-05\\
119	4.64964525643282e-05\\
120	4.64979680008789e-05\\
121	4.64995093162281e-05\\
122	4.65010769503138e-05\\
123	4.65026713504791e-05\\
124	4.65042929716377e-05\\
125	4.650594227637e-05\\
126	4.65076197350724e-05\\
127	4.65093258260765e-05\\
128	4.65110610357785e-05\\
129	4.65128258587917e-05\\
130	4.65146207980738e-05\\
131	4.65164463650569e-05\\
132	4.6518303079815e-05\\
133	4.65201914711785e-05\\
134	4.65221120769123e-05\\
135	4.65240654438355e-05\\
136	4.65260521279929e-05\\
137	4.65280726947906e-05\\
138	4.65301277191657e-05\\
139	4.65322177857247e-05\\
140	4.65343434889061e-05\\
141	4.65365054331653e-05\\
142	4.65387042331824e-05\\
143	4.65409405141296e-05\\
144	4.65432149117916e-05\\
145	4.65455280724838e-05\\
146	4.65478806534146e-05\\
147	4.65502733228561e-05\\
148	4.65527067603531e-05\\
149	4.65551816568849e-05\\
150	4.65576987150701e-05\\
151	4.65602586493646e-05\\
152	4.65628621862613e-05\\
153	4.65655100644745e-05\\
154	4.65682030351674e-05\\
155	4.65709418621456e-05\\
156	4.65737273220795e-05\\
157	4.65765602047107e-05\\
158	4.65794413130808e-05\\
159	4.65823714637411e-05\\
160	4.65853514869981e-05\\
161	4.65883822271284e-05\\
162	4.65914645426221e-05\\
163	4.6594599306412e-05\\
164	4.65977874061337e-05\\
165	4.66010297443526e-05\\
166	4.66043272388252e-05\\
167	4.6607680822761e-05\\
168	4.66110914450668e-05\\
169	4.66145600706136e-05\\
170	4.66180876805248e-05\\
171	4.66216752724192e-05\\
172	4.66253238607178e-05\\
173	4.66290344769032e-05\\
174	4.66328081698201e-05\\
175	4.6636646005968e-05\\
176	4.66405490697888e-05\\
177	4.66445184639809e-05\\
178	4.66485553098001e-05\\
179	4.66526607473745e-05\\
180	4.66568359360315e-05\\
181	4.66610820545924e-05\\
182	4.66654003017481e-05\\
183	4.66697918963588e-05\\
184	4.66742580778049e-05\\
185	4.6678800106339e-05\\
186	4.6683419263433e-05\\
187	4.6688116852151e-05\\
188	4.66928941974884e-05\\
189	4.66977526467724e-05\\
190	4.67026935700293e-05\\
191	4.67077183603696e-05\\
192	4.67128284343681e-05\\
193	4.67180252324946e-05\\
194	4.67233102194869e-05\\
195	4.67286848847864e-05\\
196	4.67341507429446e-05\\
197	4.67397093340665e-05\\
198	4.67453622242354e-05\\
199	4.67511110059563e-05\\
200	4.67569572986157e-05\\
201	4.67629027489341e-05\\
202	4.6768949031438e-05\\
203	4.67750978489395e-05\\
204	4.67813509330188e-05\\
205	4.67877100445156e-05\\
206	4.67941769740326e-05\\
207	4.68007535424626e-05\\
208	4.68074416014848e-05\\
209	4.68142430341166e-05\\
210	4.68211597552507e-05\\
211	4.68281937122044e-05\\
212	4.68353468852803e-05\\
213	4.68426212883389e-05\\
214	4.68500189693852e-05\\
215	4.68575420111601e-05\\
216	4.68651925317421e-05\\
217	4.68729726851704e-05\\
218	4.68808846620662e-05\\
219	4.68889306902817e-05\\
220	4.6897113035541e-05\\
221	4.69054340021147e-05\\
222	4.69138959334946e-05\\
223	4.69225012130759e-05\\
224	4.69312522648865e-05\\
225	4.69401515542656e-05\\
226	4.69492015886304e-05\\
227	4.69584049182035e-05\\
228	4.69677641367905e-05\\
229	4.6977281882533e-05\\
230	4.69869608387094e-05\\
231	4.69968037345613e-05\\
232	4.70068133460858e-05\\
233	4.70169924968945e-05\\
234	4.70273440590636e-05\\
235	4.70378709540219e-05\\
236	4.70485761534177e-05\\
237	4.70594626800511e-05\\
238	4.70705336088008e-05\\
239	4.70817920675649e-05\\
240	4.70932412382305e-05\\
241	4.71048843576777e-05\\
242	4.71167247187618e-05\\
243	4.71287656713784e-05\\
244	4.71410106234908e-05\\
245	4.71534630422283e-05\\
246	4.71661264549561e-05\\
247	4.71790044504387e-05\\
248	4.71921006799539e-05\\
249	4.7205418858496e-05\\
250	4.72189627659455e-05\\
251	4.72327362483355e-05\\
252	4.72467432190677e-05\\
253	4.72609876602362e-05\\
254	4.7275473623911e-05\\
255	4.72902052335034e-05\\
256	4.73051866851324e-05\\
257	4.73204222490495e-05\\
258	4.73359162710681e-05\\
259	4.73516731740516e-05\\
260	4.73676974594389e-05\\
261	4.73839937087783e-05\\
262	4.74005665853531e-05\\
263	4.74174208357881e-05\\
264	4.7434561291748e-05\\
265	4.74519928716659e-05\\
266	4.74697205825169e-05\\
267	4.74877495216719e-05\\
268	4.75060848788384e-05\\
269	4.7524731938104e-05\\
270	4.75436960801877e-05\\
271	4.75629827848413e-05\\
272	4.75825976331039e-05\\
273	4.76025463083986e-05\\
274	4.76228345948215e-05\\
275	4.76434683760235e-05\\
276	4.76644536527703e-05\\
277	4.76857965380295e-05\\
278	4.77075032589524e-05\\
279	4.77295801595217e-05\\
280	4.77520337032116e-05\\
281	4.77748704757659e-05\\
282	4.77980971880835e-05\\
283	4.78217206792177e-05\\
284	4.78457479194409e-05\\
285	4.78701860134646e-05\\
286	4.78950422037711e-05\\
287	4.79203238740843e-05\\
288	4.79460385529402e-05\\
289	4.7972193917466e-05\\
290	4.79987977972544e-05\\
291	4.80258581784642e-05\\
292	4.80533832080408e-05\\
293	4.80813811981757e-05\\
294	4.81098606309409e-05\\
295	4.81388301631565e-05\\
296	4.81682986314961e-05\\
297	4.81982750578213e-05\\
298	4.82287686548244e-05\\
299	4.82597888319346e-05\\
300	4.82913452015431e-05\\
301	4.8323447585574e-05\\
302	4.83561060223944e-05\\
303	4.83893307740658e-05\\
304	4.8423132333964e-05\\
305	4.84575214346171e-05\\
306	4.84925090557011e-05\\
307	4.85281064320679e-05\\
308	4.85643250620445e-05\\
309	4.86011767177443e-05\\
310	4.86386734621415e-05\\
311	4.8676827678597e-05\\
312	4.87156520978664e-05\\
313	4.87551597564142e-05\\
314	4.8795364022418e-05\\
315	4.88362786247455e-05\\
316	4.88779176705513e-05\\
317	4.89202956642902e-05\\
318	4.89634275283708e-05\\
319	4.9007328625485e-05\\
320	4.90520147829779e-05\\
321	4.90975023193e-05\\
322	4.91438080728845e-05\\
323	4.91909494336969e-05\\
324	4.92389443776877e-05\\
325	4.92878115045817e-05\\
326	4.93375700792812e-05\\
327	4.93882400773528e-05\\
328	4.94398422350384e-05\\
329	4.94923981043484e-05\\
330	4.95459301137594e-05\\
331	4.96004616352609e-05\\
332	4.96560170583851e-05\\
333	4.97126218722017e-05\\
334	4.97703027560492e-05\\
335	4.98290876801908e-05\\
336	4.9889006017507e-05\\
337	4.99500886676085e-05\\
338	5.00123681948524e-05\\
339	5.00758789819625e-05\\
340	5.01406574012291e-05\\
341	5.02067420053495e-05\\
342	5.02741737404943e-05\\
343	5.034299618427e-05\\
344	5.04132558117271e-05\\
345	5.04850022930345e-05\\
346	5.05582888267819e-05\\
347	5.06331725135548e-05\\
348	5.07097147751425e-05\\
349	5.07879818256077e-05\\
350	5.08680452019124e-05\\
351	5.09499823634595e-05\\
352	5.10338773721552e-05\\
353	5.11198216630597e-05\\
354	5.1207914899369e-05\\
355	5.12982658542689e-05\\
356	5.13909932478607e-05\\
357	5.14862270201235e-05\\
358	5.15841102216357e-05\\
359	5.16848002389749e-05\\
360	5.17884705448819e-05\\
361	5.18953126650908e-05\\
362	5.20055383462529e-05\\
363	5.21193817847083e-05\\
364	5.22371013522779e-05\\
365	5.23589791632466e-05\\
366	5.24853177180695e-05\\
367	5.26164927355185e-05\\
368	5.2754106380934e-05\\
369	5.33348951666007e-05\\
370	5.3957204054167e-05\\
371	5.45896062072596e-05\\
372	5.52322612354649e-05\\
373	5.588533110313e-05\\
374	5.65489803489557e-05\\
375	5.72233768696586e-05\\
376	5.79086938147883e-05\\
377	5.86051095135397e-05\\
378	5.93127818377497e-05\\
379	6.00318790577998e-05\\
380	6.07625838000523e-05\\
381	6.15050821419064e-05\\
382	6.22595638379744e-05\\
383	6.30262225781473e-05\\
384	6.38052562815622e-05\\
385	6.4596867430693e-05\\
386	6.54012634503663e-05\\
387	6.62186571366786e-05\\
388	6.70492671414194e-05\\
389	6.78933185177992e-05\\
390	6.87510433338879e-05\\
391	6.96226813603081e-05\\
392	7.05084808391372e-05\\
393	7.14086993412281e-05\\
394	7.23236047190819e-05\\
395	7.32534761625014e-05\\
396	7.41986053638208e-05\\
397	7.51592977986433e-05\\
398	7.6135874125322e-05\\
399	7.71286716975121e-05\\
400	7.81380461538272e-05\\
401	7.91643729526588e-05\\
402	8.02080484589262e-05\\
403	8.12694897303985e-05\\
404	8.23491329950335e-05\\
405	8.3447444861849e-05\\
406	8.45650264339686e-05\\
407	8.57024394423497e-05\\
408	8.68602485026623e-05\\
409	8.80390433137111e-05\\
410	8.92394189516993e-05\\
411	9.04619516920421e-05\\
412	9.17075342590605e-05\\
413	9.29772832542274e-05\\
414	9.42715832514439e-05\\
415	9.5590917679188e-05\\
416	9.69359307017726e-05\\
417	9.83072918494114e-05\\
418	9.9705696704008e-05\\
419	0.000101131867124936\\
420	0.000102586552091485\\
421	0.000104070534828886\\
422	0.000105584656761338\\
423	0.000107129805491077\\
424	0.000108706888637646\\
425	0.000110316857657739\\
426	0.000111960711090713\\
427	0.000113639498139863\\
428	0.00011535432263472\\
429	0.000117106347427119\\
430	0.000118896799282379\\
431	0.000120726974337093\\
432	0.000122598244207171\\
433	0.000124512062844152\\
434	0.000126469974254772\\
435	0.000128473621219041\\
436	0.000130524755165883\\
437	0.000132625247394223\\
438	0.000134777101861739\\
439	0.000136982469809312\\
440	0.000139243666565943\\
441	0.000141563191083511\\
442	0.000143943749479445\\
443	0.000146388286722446\\
444	0.000148900041242269\\
445	0.00015148266725791\\
446	0.000154140434601715\\
447	0.000156876604403537\\
448	0.000159695618446798\\
449	0.000162602583136702\\
450	0.000165603269980838\\
451	0.000168704141583991\\
452	0.000171912461067029\\
453	0.000175236415477924\\
454	0.00017868524931195\\
455	0.000182269412078902\\
456	0.000186000824948488\\
457	0.000189893986741423\\
458	0.000193970883317604\\
459	0.000228095183085844\\
460	0.00026877496219362\\
461	0.000310340054739103\\
462	0.000352813735782594\\
463	0.000396216382073708\\
464	0.000440565637866915\\
465	0.000485877110274124\\
466	0.000532160774069255\\
467	0.000579418257293104\\
468	0.00062738419818233\\
469	0.00067622482309534\\
470	0.000726048322986633\\
471	0.000776839706170607\\
472	0.000828636037132636\\
473	0.00088147678700264\\
474	0.000935403965952333\\
475	0.000990462253691594\\
476	0.00104669910155119\\
477	0.00110416455205536\\
478	0.00116291308479596\\
479	0.00122300222748416\\
480	0.00128449561765596\\
481	0.00134744304775154\\
482	0.00141189995900876\\
483	0.00147792462610287\\
484	0.00154557809802488\\
485	0.00161492391398022\\
486	0.0016860270809891\\
487	0.00175895033040454\\
488	0.00183373919105755\\
489	0.00190806619835934\\
490	0.0019832894053654\\
491	0.00206057427645102\\
492	0.00214005154122694\\
493	0.00222186862771036\\
494	0.00230619258930748\\
495	0.00239321367185647\\
496	0.00248314968594651\\
497	0.00257625145724361\\
498	0.00267280965494731\\
499	0.00277316374541858\\
500	0.0028777121918144\\
501	0.00298693285794693\\
502	0.0031013825506269\\
503	0.00322166732056289\\
504	0.00334849896939119\\
505	0.00348271740882836\\
506	0.00362027029291665\\
507	0.00376105869656856\\
508	0.00388611688136062\\
509	0.0039572418151582\\
510	0.00402944410584187\\
511	0.00410270379462366\\
512	0.00417699316230488\\
513	0.00425227577011228\\
514	0.00432850697474614\\
515	0.00440563912522553\\
516	0.00448365029461639\\
517	0.0045626587564041\\
518	0.00464264237219054\\
519	0.00472351612283466\\
520	0.00480517248048174\\
521	0.00488747531542428\\
522	0.00497025317401618\\
523	0.00505329154666099\\
524	0.00513632313230388\\
525	0.00521901734880742\\
526	0.00530096376322706\\
527	0.00538166427589422\\
528	0.00546050799752127\\
529	0.00553673733990381\\
530	0.00560940948104321\\
531	0.00567736265017387\\
532	0.00574657938646294\\
533	0.00581728614637982\\
534	0.0058895628459596\\
535	0.0059634355660333\\
536	0.00603886694478586\\
537	0.00611577996857281\\
538	0.00619404396421993\\
539	0.006273456223647\\
540	0.0063537180019648\\
541	0.00643455409956856\\
542	0.00651658735225979\\
543	0.00660009219092987\\
544	0.00668464219506546\\
545	0.00676848387754154\\
546	0.00685086218497004\\
547	0.00693155004757897\\
548	0.00701032202772476\\
549	0.00708696541304999\\
550	0.00716129639164414\\
551	0.00723318316095625\\
552	0.00730257843968756\\
553	0.00736952647902213\\
554	0.00743402578181928\\
555	0.00749607700169981\\
556	0.00755586493724629\\
557	0.00761541016178557\\
558	0.00767469271572267\\
559	0.00773370315312266\\
560	0.00779244441339949\\
561	0.00785091703370655\\
562	0.00790914232129952\\
563	0.00796716616600809\\
564	0.00802506207440758\\
565	0.00808293312451273\\
566	0.00814091167253474\\
567	0.00819915577571846\\
568	0.00825784157242284\\
569	0.00831701598922605\\
570	0.0083766845918232\\
571	0.00843685157104802\\
572	0.00849751790386216\\
573	0.00855867710387677\\
574	0.00862031664618866\\
575	0.00868242246674999\\
576	0.0087449779716787\\
577	0.00880796255215469\\
578	0.0088713498714046\\
579	0.00893510609465251\\
580	0.00899918825703781\\
581	0.00906354428152085\\
582	0.00912811788932691\\
583	0.00919285217075625\\
584	0.00925768822972085\\
585	0.00932258151766213\\
586	0.00938749662428032\\
587	0.00945240446296449\\
588	0.00951729025111872\\
589	0.0095821249003219\\
590	0.00964561720054655\\
591	0.0097066206152574\\
592	0.00976430172597694\\
593	0.00981781039518062\\
594	0.00986668609051667\\
595	0.00990919367976287\\
596	0.009946404199063\\
597	0.00997788999445116\\
598	0.010000292044645\\
599	0\\
600	0\\
};
\addplot [color=mycolor11,solid,forget plot]
  table[row sep=crcr]{%
1	2.87074056288534e-05\\
2	2.87074219770342e-05\\
3	2.8707438605019e-05\\
4	2.87074555176098e-05\\
5	2.87074727196844e-05\\
6	2.87074902162143e-05\\
7	2.87075080122498e-05\\
8	2.87075261129244e-05\\
9	2.87075445234721e-05\\
10	2.87075632491999e-05\\
11	2.87075822955249e-05\\
12	2.87076016679376e-05\\
13	2.87076213720374e-05\\
14	2.87076414135115e-05\\
15	2.87076617981448e-05\\
16	2.87076825318254e-05\\
17	2.87077036205368e-05\\
18	2.87077250703705e-05\\
19	2.87077468875226e-05\\
20	2.8707769078286e-05\\
21	2.87077916490701e-05\\
22	2.87078146063905e-05\\
23	2.87078379568757e-05\\
24	2.87078617072681e-05\\
25	2.87078858644229e-05\\
26	2.87079104353101e-05\\
27	2.87079354270323e-05\\
28	2.87079608467935e-05\\
29	2.8707986701937e-05\\
30	2.87080129999283e-05\\
31	2.87080397483458e-05\\
32	2.87080669549237e-05\\
33	2.87080946275071e-05\\
34	2.87081227740844e-05\\
35	2.87081514027762e-05\\
36	2.87081805218467e-05\\
37	2.87082101397009e-05\\
38	2.87082402648758e-05\\
39	2.87082709060712e-05\\
40	2.87083020721246e-05\\
41	2.87083337720337e-05\\
42	2.87083660149329e-05\\
43	2.87083988101306e-05\\
44	2.87084321670868e-05\\
45	2.87084660954185e-05\\
46	2.87085006049105e-05\\
47	2.87085357055212e-05\\
48	2.87085714073691e-05\\
49	2.87086077207499e-05\\
50	2.87086446561366e-05\\
51	2.87086822241715e-05\\
52	2.87087204356876e-05\\
53	2.87087593017029e-05\\
54	2.87087988334197e-05\\
55	2.87088390422278e-05\\
56	2.87088799397196e-05\\
57	2.87089215376791e-05\\
58	2.87089638480933e-05\\
59	2.87090068831497e-05\\
60	2.87090506552488e-05\\
61	2.87090951769986e-05\\
62	2.87091404612286e-05\\
63	2.87091865209722e-05\\
64	2.87092333694985e-05\\
65	2.87092810202971e-05\\
66	2.87093294870827e-05\\
67	2.87093787838186e-05\\
68	2.87094289246864e-05\\
69	2.87094799241193e-05\\
70	2.87095317967987e-05\\
71	2.87095845576489e-05\\
72	2.87096382218437e-05\\
73	2.87096928048299e-05\\
74	2.87097483223065e-05\\
75	2.87098047902417e-05\\
76	2.8709862224873e-05\\
77	2.87099206427135e-05\\
78	2.87099800605614e-05\\
79	2.87100404954949e-05\\
80	2.87101019648874e-05\\
81	2.87101644864034e-05\\
82	2.87102280780061e-05\\
83	2.8710292757966e-05\\
84	2.87103585448645e-05\\
85	2.87104254575978e-05\\
86	2.87104935153779e-05\\
87	2.87105627377484e-05\\
88	2.87106331445835e-05\\
89	2.8710704756096e-05\\
90	2.87107775928283e-05\\
91	2.87108516756889e-05\\
92	2.87109270259289e-05\\
93	2.87110036651669e-05\\
94	2.87110816153827e-05\\
95	2.87111608989232e-05\\
96	2.87112415385331e-05\\
97	2.87113235573141e-05\\
98	2.87114069787841e-05\\
99	2.87114918268461e-05\\
100	2.87115781258095e-05\\
101	2.87116659003968e-05\\
102	2.87117551757434e-05\\
103	2.87118459774197e-05\\
104	2.87119383314177e-05\\
105	2.87120322641772e-05\\
106	2.871212780258e-05\\
107	2.87122249739635e-05\\
108	2.87123238061285e-05\\
109	2.87124243273415e-05\\
110	2.8712526566356e-05\\
111	2.87126305524002e-05\\
112	2.87127363151966e-05\\
113	2.8712843884981e-05\\
114	2.87129532924846e-05\\
115	2.87130645689699e-05\\
116	2.87131777462208e-05\\
117	2.87132928565585e-05\\
118	2.87134099328484e-05\\
119	2.87135290085129e-05\\
120	2.8713650117539e-05\\
121	2.87137732944825e-05\\
122	2.87138985744863e-05\\
123	2.87140259932864e-05\\
124	2.87141555872151e-05\\
125	2.87142873932249e-05\\
126	2.87144214488846e-05\\
127	2.87145577924034e-05\\
128	2.87146964626287e-05\\
129	2.87148374990611e-05\\
130	2.87149809418793e-05\\
131	2.87151268319188e-05\\
132	2.87152752107225e-05\\
133	2.87154261205211e-05\\
134	2.87155796042571e-05\\
135	2.87157357055965e-05\\
136	2.87158944689389e-05\\
137	2.87160559394325e-05\\
138	2.87162201629782e-05\\
139	2.87163871862413e-05\\
140	2.87165570566855e-05\\
141	2.87167298225847e-05\\
142	2.87169055330413e-05\\
143	2.87170842379845e-05\\
144	2.87172659881683e-05\\
145	2.87174508351859e-05\\
146	2.87176388315083e-05\\
147	2.87178300304862e-05\\
148	2.87180244863631e-05\\
149	2.87182222542978e-05\\
150	2.87184233903808e-05\\
151	2.87186279516347e-05\\
152	2.87188359960527e-05\\
153	2.8719047582593e-05\\
154	2.87192627712097e-05\\
155	2.87194816228568e-05\\
156	2.87197041995304e-05\\
157	2.87199305642401e-05\\
158	2.87201607810818e-05\\
159	2.87203949152054e-05\\
160	2.87206330328721e-05\\
161	2.87208752014378e-05\\
162	2.87211214894095e-05\\
163	2.87213719664297e-05\\
164	2.87216267033141e-05\\
165	2.87218857720663e-05\\
166	2.87221492458993e-05\\
167	2.87224171992527e-05\\
168	2.87226897078212e-05\\
169	2.87229668485622e-05\\
170	2.87232486997237e-05\\
171	2.87235353408754e-05\\
172	2.87238268529116e-05\\
173	2.87241233180891e-05\\
174	2.87244248200478e-05\\
175	2.87247314438186e-05\\
176	2.87250432758698e-05\\
177	2.87253604041133e-05\\
178	2.8725682917941e-05\\
179	2.8726010908238e-05\\
180	2.8726344467414e-05\\
181	2.87266836894331e-05\\
182	2.8727028669831e-05\\
183	2.87273795057451e-05\\
184	2.8727736295944e-05\\
185	2.87280991408507e-05\\
186	2.87284681425684e-05\\
187	2.87288434049218e-05\\
188	2.87292250334677e-05\\
189	2.87296131355383e-05\\
190	2.87300078202593e-05\\
191	2.87304091985894e-05\\
192	2.87308173833562e-05\\
193	2.87312324892674e-05\\
194	2.87316546329607e-05\\
195	2.87320839330218e-05\\
196	2.87325205100417e-05\\
197	2.87329644866219e-05\\
198	2.87334159874207e-05\\
199	2.87338751391964e-05\\
200	2.87343420708235e-05\\
201	2.87348169133468e-05\\
202	2.87352998000084e-05\\
203	2.87357908662871e-05\\
204	2.87362902499362e-05\\
205	2.87367980910208e-05\\
206	2.87373145319596e-05\\
207	2.87378397175658e-05\\
208	2.87383737950801e-05\\
209	2.87389169142191e-05\\
210	2.87394692272215e-05\\
211	2.87400308888777e-05\\
212	2.87406020565832e-05\\
213	2.87411828903803e-05\\
214	2.87417735530064e-05\\
215	2.87423742099366e-05\\
216	2.87429850294291e-05\\
217	2.87436061825764e-05\\
218	2.87442378433557e-05\\
219	2.87448801886679e-05\\
220	2.87455333984064e-05\\
221	2.87461976554881e-05\\
222	2.87468731459194e-05\\
223	2.87475600588475e-05\\
224	2.8748258586608e-05\\
225	2.87489689247892e-05\\
226	2.87496912722746e-05\\
227	2.8750425831325e-05\\
228	2.87511728076093e-05\\
229	2.87519324102845e-05\\
230	2.87527048520416e-05\\
231	2.8753490349177e-05\\
232	2.87542891216552e-05\\
233	2.87551013931665e-05\\
234	2.87559273912032e-05\\
235	2.87567673471142e-05\\
236	2.87576214961827e-05\\
237	2.87584900776872e-05\\
238	2.87593733349814e-05\\
239	2.87602715155676e-05\\
240	2.87611848711529e-05\\
241	2.87621136577445e-05\\
242	2.87630581357161e-05\\
243	2.87640185698889e-05\\
244	2.8764995229616e-05\\
245	2.87659883888463e-05\\
246	2.87669983262339e-05\\
247	2.87680253252047e-05\\
248	2.87690696740387e-05\\
249	2.87701316659861e-05\\
250	2.87712115993204e-05\\
251	2.87723097774585e-05\\
252	2.87734265090484e-05\\
253	2.87745621080573e-05\\
254	2.87757168938728e-05\\
255	2.87768911914122e-05\\
256	2.87780853312216e-05\\
257	2.8779299649569e-05\\
258	2.87805344885712e-05\\
259	2.87817901962847e-05\\
260	2.87830671268389e-05\\
261	2.87843656405262e-05\\
262	2.87856861039463e-05\\
263	2.87870288901108e-05\\
264	2.87883943785704e-05\\
265	2.87897829555578e-05\\
266	2.87911950141216e-05\\
267	2.87926309542748e-05\\
268	2.87940911831882e-05\\
269	2.87955761153534e-05\\
270	2.87970861727768e-05\\
271	2.8798621785011e-05\\
272	2.88001833889952e-05\\
273	2.88017714286501e-05\\
274	2.8803386355146e-05\\
275	2.88050286289825e-05\\
276	2.88066987190239e-05\\
277	2.88083971026034e-05\\
278	2.88101242656961e-05\\
279	2.88118807031026e-05\\
280	2.88136669186484e-05\\
281	2.88154834253669e-05\\
282	2.88173307457072e-05\\
283	2.88192094117412e-05\\
284	2.88211199653794e-05\\
285	2.8823062958591e-05\\
286	2.88250389536283e-05\\
287	2.8827048523281e-05\\
288	2.88290922510929e-05\\
289	2.88311707316444e-05\\
290	2.88332845708011e-05\\
291	2.88354343859876e-05\\
292	2.88376208064754e-05\\
293	2.88398444736696e-05\\
294	2.88421060414301e-05\\
295	2.88444061763612e-05\\
296	2.88467455581792e-05\\
297	2.88491248800287e-05\\
298	2.88515448488514e-05\\
299	2.88540061857641e-05\\
300	2.88565096264286e-05\\
301	2.88590559214596e-05\\
302	2.88616458368236e-05\\
303	2.88642801541959e-05\\
304	2.88669596713435e-05\\
305	2.88696852024063e-05\\
306	2.88724575782161e-05\\
307	2.88752776467553e-05\\
308	2.88781462742516e-05\\
309	2.88810643474518e-05\\
310	2.88840327767239e-05\\
311	2.88870524963344e-05\\
312	2.88901244576064e-05\\
313	2.88932496319469e-05\\
314	2.88964290135874e-05\\
315	2.88996636204524e-05\\
316	2.89029544951364e-05\\
317	2.89063027058867e-05\\
318	2.89097093477215e-05\\
319	2.8913175543578e-05\\
320	2.89167024455904e-05\\
321	2.89202912364288e-05\\
322	2.89239431307811e-05\\
323	2.89276593769228e-05\\
324	2.89314412584406e-05\\
325	2.89352900960943e-05\\
326	2.8939207249852e-05\\
327	2.89431941210691e-05\\
328	2.89472521549184e-05\\
329	2.89513828430158e-05\\
330	2.89555877262789e-05\\
331	2.89598683980968e-05\\
332	2.89642265077754e-05\\
333	2.89686637643585e-05\\
334	2.89731819407897e-05\\
335	2.89777828785489e-05\\
336	2.89824684927481e-05\\
337	2.89872407777614e-05\\
338	2.89921018134589e-05\\
339	2.8997053772138e-05\\
340	2.90020989261909e-05\\
341	2.9007239656636e-05\\
342	2.90124784626291e-05\\
343	2.90178179720329e-05\\
344	2.90232609532281e-05\\
345	2.90288103283296e-05\\
346	2.90344691880262e-05\\
347	2.90402408082604e-05\\
348	2.90461286691986e-05\\
349	2.90521364769399e-05\\
350	2.90582681884599e-05\\
351	2.90645280401054e-05\\
352	2.90709205783075e-05\\
353	2.90774506878393e-05\\
354	2.90841236102916e-05\\
355	2.90909449613259e-05\\
356	2.90979208100995e-05\\
357	2.91050578025798e-05\\
358	2.91123631980276e-05\\
359	2.91198450062145e-05\\
360	2.91275122628458e-05\\
361	2.91353756940605e-05\\
362	2.9143449519973e-05\\
363	2.91517566766702e-05\\
364	2.91603444727928e-05\\
365	2.91693323614963e-05\\
366	2.91790572625571e-05\\
367	2.91904883455599e-05\\
368	2.92059904748458e-05\\
369	2.92220118418757e-05\\
370	2.92382851160593e-05\\
371	2.92548141350258e-05\\
372	2.92716028900053e-05\\
373	2.92886556623828e-05\\
374	2.93059772348849e-05\\
375	2.93235729474886e-05\\
376	2.93414476655059e-05\\
377	2.93596027275639e-05\\
378	2.93780407253037e-05\\
379	2.93967659119906e-05\\
380	2.94157826108006e-05\\
381	2.943509521892e-05\\
382	2.94547082122986e-05\\
383	2.94746261510439e-05\\
384	2.94948536856192e-05\\
385	2.95153955639028e-05\\
386	2.95362566391953e-05\\
387	2.95574418793126e-05\\
388	2.95789563768745e-05\\
389	2.96008053608892e-05\\
390	2.96229942097429e-05\\
391	2.96455284656924e-05\\
392	2.96684138507856e-05\\
393	2.96916562839949e-05\\
394	2.97152618986561e-05\\
395	2.97392370579873e-05\\
396	2.97635883633745e-05\\
397	2.97883226435484e-05\\
398	2.98134469000132e-05\\
399	2.983896816252e-05\\
400	2.98648931854672e-05\\
401	2.98912279412757e-05\\
402	2.9917977129054e-05\\
403	2.99451449001304e-05\\
404	2.99727400478062e-05\\
405	3.00007863959452e-05\\
406	3.00292971289041e-05\\
407	3.00582798077221e-05\\
408	3.00877378599853e-05\\
409	3.01176675906268e-05\\
410	3.01480642028151e-05\\
411	3.01789779503984e-05\\
412	3.02104853764506e-05\\
413	3.02425629355838e-05\\
414	3.02752030331295e-05\\
415	3.03084177303424e-05\\
416	3.03422192657478e-05\\
417	3.03766199085939e-05\\
418	3.04116318696233e-05\\
419	3.04472676443895e-05\\
420	3.04835414641953e-05\\
421	3.0520471431376e-05\\
422	3.05580764393456e-05\\
423	3.05963730460801e-05\\
424	3.06353785031662e-05\\
425	3.0675110802949e-05\\
426	3.07155887302112e-05\\
427	3.0756831918944e-05\\
428	3.07988609148731e-05\\
429	3.08416972445709e-05\\
430	3.08853634919592e-05\\
431	3.09298833833985e-05\\
432	3.09752818826228e-05\\
433	3.1021585297479e-05\\
434	3.1068821401146e-05\\
435	3.11170195730472e-05\\
436	3.11662109697854e-05\\
437	3.12164287501284e-05\\
438	3.12677084112617e-05\\
439	3.1320088374597e-05\\
440	3.13736111458136e-05\\
441	3.14283257581362e-05\\
442	3.14842927968935e-05\\
443	3.15415932268637e-05\\
444	3.16003361893756e-05\\
445	3.16606287197654e-05\\
446	3.17223983966092e-05\\
447	3.17856802878807e-05\\
448	3.18505393092603e-05\\
449	3.19170570309623e-05\\
450	3.19853231436586e-05\\
451	3.20554371384342e-05\\
452	3.21275132520946e-05\\
453	3.22016981889936e-05\\
454	3.22782337643279e-05\\
455	3.23576621816601e-05\\
456	3.24414281777169e-05\\
457	3.2533347786978e-05\\
458	3.26421118733469e-05\\
459	3.27552039992171e-05\\
460	3.28698685584011e-05\\
461	3.29864062298302e-05\\
462	3.31050456097429e-05\\
463	3.32259930698725e-05\\
464	3.33497356061983e-05\\
465	3.34764470236495e-05\\
466	3.36068489184736e-05\\
467	3.37469988655559e-05\\
468	3.41714855216607e-05\\
469	3.47180892944729e-05\\
470	3.52757507581536e-05\\
471	3.58451339388148e-05\\
472	3.64269757434914e-05\\
473	3.70220954982187e-05\\
474	3.76314063928785e-05\\
475	3.82559312065372e-05\\
476	3.88968456767497e-05\\
477	3.95557883791894e-05\\
478	4.02206634430905e-05\\
479	4.09064513907733e-05\\
480	4.16150064117664e-05\\
481	4.23479713284909e-05\\
482	4.31071548316835e-05\\
483	4.38945480553534e-05\\
484	4.47123580236007e-05\\
485	4.55631273707271e-05\\
486	4.6450251439557e-05\\
487	4.73802971189456e-05\\
488	4.83744473987448e-05\\
489	5.18255247195229e-05\\
490	5.64432043508488e-05\\
491	6.11757869814177e-05\\
492	6.60287180328892e-05\\
493	7.10079443662274e-05\\
494	7.61199865734989e-05\\
495	8.13720243861874e-05\\
496	8.67719896607228e-05\\
497	9.23286159209255e-05\\
498	9.80514008825647e-05\\
499	0.000103950459488896\\
500	0.000110039573939583\\
501	0.000116338196138091\\
502	0.000122866088819676\\
503	0.000129603148917134\\
504	0.000136554914834019\\
505	0.000143792738973412\\
506	0.000151348309712374\\
507	0.000159250000101562\\
508	0.000186465740107389\\
509	0.000270913190465583\\
510	0.000357339833777734\\
511	0.000445850217445032\\
512	0.000536559442871119\\
513	0.000629595116395348\\
514	0.000725100692200193\\
515	0.000823235332269393\\
516	0.00092418800105013\\
517	0.00102815991107104\\
518	0.00113536914372073\\
519	0.00124605849399606\\
520	0.00136049122063145\\
521	0.00147897623912208\\
522	0.0016018593706772\\
523	0.00172952379390597\\
524	0.00186241913462318\\
525	0.00200099235793762\\
526	0.00214576766890253\\
527	0.00229735402594387\\
528	0.00245567253882486\\
529	0.00262203513585597\\
530	0.00279776832073401\\
531	0.0029840588322844\\
532	0.00317503465125662\\
533	0.00336753682105394\\
534	0.00356080563091648\\
535	0.00375884896364931\\
536	0.00396189643029828\\
537	0.00417020157699044\\
538	0.00438404649089458\\
539	0.00460374472348798\\
540	0.00482962825192461\\
541	0.00498742931533325\\
542	0.0051011079388929\\
543	0.00521618695727248\\
544	0.00533272468242733\\
545	0.00545049487325927\\
546	0.00556921337530202\\
547	0.00568850362809646\\
548	0.00580786947567521\\
549	0.00592666237003143\\
550	0.00604404005502914\\
551	0.00615891412564674\\
552	0.00626988461770694\\
553	0.00637687442962497\\
554	0.00648639688360578\\
555	0.00659810957107587\\
556	0.00671146170833344\\
557	0.00682409800646433\\
558	0.00693572396512232\\
559	0.00704601191043983\\
560	0.00715459986066793\\
561	0.00726109172587377\\
562	0.00736505911124072\\
563	0.00746604653227905\\
564	0.00756358087956845\\
565	0.00765718855990147\\
566	0.0077464333920976\\
567	0.00783089357803292\\
568	0.00791059914192856\\
569	0.00798962772058366\\
570	0.00806803480922161\\
571	0.00814585569413195\\
572	0.00822322735200265\\
573	0.00830037097742555\\
574	0.00837736890349071\\
575	0.00845418668081767\\
576	0.00853079440325115\\
577	0.00860717799344965\\
578	0.00868333907131728\\
579	0.00875929228624666\\
580	0.00883505938019689\\
581	0.00891063794895334\\
582	0.00898592552194447\\
583	0.00906078197573007\\
584	0.00913506393062985\\
585	0.00920862685199412\\
586	0.00928132795734763\\
587	0.00935303006810276\\
588	0.00942360690540113\\
589	0.00949294844638006\\
590	0.00956097836809741\\
591	0.00962766009652905\\
592	0.0096930024817262\\
593	0.00975706769624642\\
594	0.00981997518922992\\
595	0.00988192166735141\\
596	0.00993785230806569\\
597	0.00997762486405646\\
598	0.010000292044645\\
599	0\\
600	0\\
};
\addplot [color=mycolor12,solid,forget plot]
  table[row sep=crcr]{%
1	4.34936132690398e-07\\
2	4.34936809193483e-07\\
3	4.34937497278928e-07\\
4	4.34938197143251e-07\\
5	4.34938908987252e-07\\
6	4.34939633019321e-07\\
7	4.34940369445737e-07\\
8	4.34941118477938e-07\\
9	4.34941880336154e-07\\
10	4.34942655238677e-07\\
11	4.34943443407577e-07\\
12	4.3494424507319e-07\\
13	4.34945060464087e-07\\
14	4.34945889817803e-07\\
15	4.34946733373395e-07\\
16	4.34947591374897e-07\\
17	4.34948464068387e-07\\
18	4.34949351704745e-07\\
19	4.34950254543989e-07\\
20	4.34951172843841e-07\\
21	4.34952106870986e-07\\
22	4.34953056895178e-07\\
23	4.34954023186657e-07\\
24	4.34955006031013e-07\\
25	4.34956005704797e-07\\
26	4.34957022502341e-07\\
27	4.34958056714123e-07\\
28	4.34959108639579e-07\\
29	4.34960178582937e-07\\
30	4.34961266850286e-07\\
31	4.34962373756493e-07\\
32	4.34963499624852e-07\\
33	4.3496464477342e-07\\
34	4.34965809537505e-07\\
35	4.34966994252714e-07\\
36	4.34968199259269e-07\\
37	4.34969424904786e-07\\
38	4.34970671543397e-07\\
39	4.34971939534548e-07\\
40	4.34973229244549e-07\\
41	4.34974541045021e-07\\
42	4.34975875314098e-07\\
43	4.34977232435398e-07\\
44	4.34978612800955e-07\\
45	4.34980016810016e-07\\
46	4.34981444864525e-07\\
47	4.34982897379353e-07\\
48	4.34984374772243e-07\\
49	4.34985877466414e-07\\
50	4.34987405898337e-07\\
51	4.34988960505118e-07\\
52	4.34990541739016e-07\\
53	4.34992150052223e-07\\
54	4.34993785907593e-07\\
55	4.34995449777764e-07\\
56	4.34997142144305e-07\\
57	4.3499886349079e-07\\
58	4.35000614313864e-07\\
59	4.35002395115809e-07\\
60	4.35004206414401e-07\\
61	4.35006048726993e-07\\
62	4.35007922585395e-07\\
63	4.35009828529289e-07\\
64	4.35011767105724e-07\\
65	4.35013738873596e-07\\
66	4.35015744402102e-07\\
67	4.35017784262257e-07\\
68	4.35019859047303e-07\\
69	4.35021969349018e-07\\
70	4.35024115777591e-07\\
71	4.35026298947283e-07\\
72	4.35028519489035e-07\\
73	4.35030778037682e-07\\
74	4.35033075244735e-07\\
75	4.35035411769056e-07\\
76	4.35037788284799e-07\\
77	4.35040205472766e-07\\
78	4.35042664024561e-07\\
79	4.35045164650362e-07\\
80	4.35047708065609e-07\\
81	4.35050295002757e-07\\
82	4.35052926203153e-07\\
83	4.35055602421873e-07\\
84	4.35058324427544e-07\\
85	4.3506109300079e-07\\
86	4.3506390893648e-07\\
87	4.35066773040953e-07\\
88	4.35069686137898e-07\\
89	4.3507264906006e-07\\
90	4.35075662658212e-07\\
91	4.35078727795986e-07\\
92	4.35081845350373e-07\\
93	4.35085016219171e-07\\
94	4.35088241305231e-07\\
95	4.35091521535492e-07\\
96	4.35094857850591e-07\\
97	4.35098251201583e-07\\
98	4.3510170256187e-07\\
99	4.35105212923911e-07\\
100	4.35108783283132e-07\\
101	4.351124146682e-07\\
102	4.35116108116108e-07\\
103	4.35119864682721e-07\\
104	4.3512368544278e-07\\
105	4.35127571486957e-07\\
106	4.35131523925825e-07\\
107	4.35135543891247e-07\\
108	4.35139632531871e-07\\
109	4.35143791013824e-07\\
110	4.35148020525372e-07\\
111	4.3515232227381e-07\\
112	4.35156697488917e-07\\
113	4.35161147417243e-07\\
114	4.35165673328162e-07\\
115	4.35170276514907e-07\\
116	4.35174958289548e-07\\
117	4.35179719990259e-07\\
118	4.35184562968507e-07\\
119	4.35189488609126e-07\\
120	4.35194498313168e-07\\
121	4.35199593511938e-07\\
122	4.35204775653304e-07\\
123	4.35210046214503e-07\\
124	4.3521540669677e-07\\
125	4.35220858624137e-07\\
126	4.35226403551018e-07\\
127	4.35232043052886e-07\\
128	4.35237778735943e-07\\
129	4.35243612230372e-07\\
130	4.35249545194144e-07\\
131	4.35255579316124e-07\\
132	4.35261716310876e-07\\
133	4.3526795792039e-07\\
134	4.35274305919615e-07\\
135	4.35280762109189e-07\\
136	4.35287328329096e-07\\
137	4.35294006434969e-07\\
138	4.35300798329048e-07\\
139	4.35307705933702e-07\\
140	4.35314731213234e-07\\
141	4.35321876155703e-07\\
142	4.35329142791769e-07\\
143	4.35336533180348e-07\\
144	4.35344049414302e-07\\
145	4.35351693624241e-07\\
146	4.35359467977315e-07\\
147	4.3536737467287e-07\\
148	4.35375415953705e-07\\
149	4.3538359409498e-07\\
150	4.35391911409577e-07\\
151	4.35400370252256e-07\\
152	4.35408973016518e-07\\
153	4.35417722137904e-07\\
154	4.35426620087401e-07\\
155	4.35435669381134e-07\\
156	4.35444872576374e-07\\
157	4.35454232275347e-07\\
158	4.35463751121059e-07\\
159	4.35473431803189e-07\\
160	4.35483277054617e-07\\
161	4.35493289654351e-07\\
162	4.35503472429955e-07\\
163	4.35513828253031e-07\\
164	4.35524360045623e-07\\
165	4.35535070777784e-07\\
166	4.35545963471205e-07\\
167	4.35557041193836e-07\\
168	4.35568307070617e-07\\
169	4.35579764274289e-07\\
170	4.35591416033009e-07\\
171	4.35603265628087e-07\\
172	4.35615316395709e-07\\
173	4.3562757173091e-07\\
174	4.35640035078223e-07\\
175	4.35652709949485e-07\\
176	4.35665599906192e-07\\
177	4.35678708575397e-07\\
178	4.35692039642616e-07\\
179	4.35705596855099e-07\\
180	4.35719384024077e-07\\
181	4.35733405023703e-07\\
182	4.35747663792794e-07\\
183	4.35762164332894e-07\\
184	4.35776910721599e-07\\
185	4.35791907093346e-07\\
186	4.35807157659635e-07\\
187	4.35822666701067e-07\\
188	4.35838438566988e-07\\
189	4.35854477678761e-07\\
190	4.35870788540484e-07\\
191	4.35887375721681e-07\\
192	4.3590424387251e-07\\
193	4.35921397718915e-07\\
194	4.35938842069352e-07\\
195	4.35956581809419e-07\\
196	4.35974621911367e-07\\
197	4.35992967422306e-07\\
198	4.36011623482897e-07\\
199	4.3603059531262e-07\\
200	4.36049888221872e-07\\
201	4.36069507612482e-07\\
202	4.36089458969045e-07\\
203	4.36109747875333e-07\\
204	4.36130380006508e-07\\
205	4.36151361129285e-07\\
206	4.36172697114527e-07\\
207	4.36194393921511e-07\\
208	4.3621645761762e-07\\
209	4.36238894371234e-07\\
210	4.36261710447751e-07\\
211	4.36284912224091e-07\\
212	4.36308506181932e-07\\
213	4.36332498913064e-07\\
214	4.36356897116083e-07\\
215	4.3638170760641e-07\\
216	4.36406937311076e-07\\
217	4.36432593274428e-07\\
218	4.36458682660864e-07\\
219	4.36485212751887e-07\\
220	4.36512190952827e-07\\
221	4.36539624796284e-07\\
222	4.36567521936743e-07\\
223	4.36595890161637e-07\\
224	4.36624737388367e-07\\
225	4.36654071665517e-07\\
226	4.36683901182853e-07\\
227	4.36714234260227e-07\\
228	4.36745079367812e-07\\
229	4.36776445105821e-07\\
230	4.36808340234951e-07\\
231	4.36840773651435e-07\\
232	4.36873754409861e-07\\
233	4.36907291716056e-07\\
234	4.36941394925503e-07\\
235	4.36976073562528e-07\\
236	4.37011337307279e-07\\
237	4.37047196003859e-07\\
238	4.37083659664784e-07\\
239	4.37120738474413e-07\\
240	4.37158442787372e-07\\
241	4.37196783133892e-07\\
242	4.37235770229807e-07\\
243	4.37275414967875e-07\\
244	4.37315728427107e-07\\
245	4.37356721880501e-07\\
246	4.37398406788451e-07\\
247	4.37440794810819e-07\\
248	4.374838978038e-07\\
249	4.37527727835106e-07\\
250	4.37572297171646e-07\\
251	4.37617618291956e-07\\
252	4.37663703894814e-07\\
253	4.37710566894014e-07\\
254	4.37758220430087e-07\\
255	4.37806677865781e-07\\
256	4.37855952795697e-07\\
257	4.37906059056981e-07\\
258	4.37957010719085e-07\\
259	4.38008822104305e-07\\
260	4.38061507775458e-07\\
261	4.38115082560407e-07\\
262	4.38169561534547e-07\\
263	4.38224960050163e-07\\
264	4.38281293717537e-07\\
265	4.38338578430848e-07\\
266	4.38396830362579e-07\\
267	4.3845606596573e-07\\
268	4.3851630199262e-07\\
269	4.38577555491015e-07\\
270	4.38639843808431e-07\\
271	4.38703184608483e-07\\
272	4.38767595867724e-07\\
273	4.38833095886327e-07\\
274	4.38899703295951e-07\\
275	4.38967437064215e-07\\
276	4.39036316501856e-07\\
277	4.39106361278283e-07\\
278	4.39177591412809e-07\\
279	4.39250027300402e-07\\
280	4.39323689707259e-07\\
281	4.39398599786122e-07\\
282	4.39474779085917e-07\\
283	4.39552249555797e-07\\
284	4.39631033554414e-07\\
285	4.39711153868686e-07\\
286	4.39792633715257e-07\\
287	4.39875496753221e-07\\
288	4.39959767098527e-07\\
289	4.40045469331684e-07\\
290	4.40132628516862e-07\\
291	4.40221270199159e-07\\
292	4.40311420438062e-07\\
293	4.40403105805777e-07\\
294	4.40496353412509e-07\\
295	4.40591190912235e-07\\
296	4.40687646524333e-07\\
297	4.40785749051628e-07\\
298	4.40885527893176e-07\\
299	4.40987013069395e-07\\
300	4.4109023523483e-07\\
301	4.41195225703087e-07\\
302	4.41302016470131e-07\\
303	4.41410640237651e-07\\
304	4.41521130432157e-07\\
305	4.41633521237861e-07\\
306	4.41747847627492e-07\\
307	4.41864145380769e-07\\
308	4.41982451133916e-07\\
309	4.42102802396652e-07\\
310	4.4222523760222e-07\\
311	4.42349796141637e-07\\
312	4.42476518405553e-07\\
313	4.42605445829455e-07\\
314	4.42736620952122e-07\\
315	4.42870087460403e-07\\
316	4.43005890250698e-07\\
317	4.4314407549477e-07\\
318	4.432846907065e-07\\
319	4.43427784814887e-07\\
320	4.43573408251904e-07\\
321	4.43721613033377e-07\\
322	4.43872452867464e-07\\
323	4.4402598324785e-07\\
324	4.44182261581356e-07\\
325	4.44341347315447e-07\\
326	4.44503302074274e-07\\
327	4.44668189813898e-07\\
328	4.44836077005044e-07\\
329	4.45007032804416e-07\\
330	4.45181129277089e-07\\
331	4.45358441622988e-07\\
332	4.45539048424822e-07\\
333	4.45723031947661e-07\\
334	4.45910478435786e-07\\
335	4.46101478465686e-07\\
336	4.46296127338126e-07\\
337	4.46494525500603e-07\\
338	4.46696779035211e-07\\
339	4.46903000178788e-07\\
340	4.47113307935286e-07\\
341	4.47327828717876e-07\\
342	4.47546697106324e-07\\
343	4.47770056669999e-07\\
344	4.47998060887322e-07\\
345	4.48230874191427e-07\\
346	4.4846867312327e-07\\
347	4.48711647631946e-07\\
348	4.48960002536266e-07\\
349	4.49213959164141e-07\\
350	4.49473757180344e-07\\
351	4.49739656669636e-07\\
352	4.50011940460597e-07\\
353	4.50290916782777e-07\\
354	4.50576922303904e-07\\
355	4.50870325745966e-07\\
356	4.51171532379942e-07\\
357	4.51480990167528e-07\\
358	4.51799199628439e-07\\
359	4.5212673234318e-07\\
360	4.52464270487952e-07\\
361	4.52812697191808e-07\\
362	4.53173306591444e-07\\
363	4.53548280021433e-07\\
364	4.53941692239194e-07\\
365	4.54361331329531e-07\\
366	4.54820730812272e-07\\
367	4.55334917625691e-07\\
368	4.5586024961437e-07\\
369	4.56393569206599e-07\\
370	4.56934984276364e-07\\
371	4.57484603740934e-07\\
372	4.58042537619154e-07\\
373	4.58608896809749e-07\\
374	4.59183792054823e-07\\
375	4.59767331818493e-07\\
376	4.60359621601097e-07\\
377	4.60960772392783e-07\\
378	4.61570899231697e-07\\
379	4.62190117959462e-07\\
380	4.62818545260268e-07\\
381	4.63456298693708e-07\\
382	4.6410349676993e-07\\
383	4.64760259023138e-07\\
384	4.65426706120389e-07\\
385	4.66102959973883e-07\\
386	4.66789143907746e-07\\
387	4.67485382833585e-07\\
388	4.68191803483914e-07\\
389	4.6890853465892e-07\\
390	4.69635707550573e-07\\
391	4.70373456090364e-07\\
392	4.71121917374378e-07\\
393	4.71881232125928e-07\\
394	4.72651545220905e-07\\
395	4.73433006233936e-07\\
396	4.74225769959324e-07\\
397	4.75029996811312e-07\\
398	4.75845852856791e-07\\
399	4.76673509100651e-07\\
400	4.77513139254203e-07\\
401	4.78364914254846e-07\\
402	4.79228988678671e-07\\
403	4.80105463284348e-07\\
404	4.80994279056869e-07\\
405	4.81894963224822e-07\\
406	4.82806276198916e-07\\
407	4.83725750066658e-07\\
408	4.84649734057819e-07\\
409	4.85575883494642e-07\\
410	4.86509813026681e-07\\
411	4.87459809787173e-07\\
412	4.88426119947507e-07\\
413	4.89409045121629e-07\\
414	4.90408939179918e-07\\
415	4.91426167774683e-07\\
416	4.92461108933913e-07\\
417	4.93514153760769e-07\\
418	4.9458570726104e-07\\
419	4.95676189302587e-07\\
420	4.96786035413769e-07\\
421	4.97915696913407e-07\\
422	4.99065641521776e-07\\
423	5.00236354831127e-07\\
424	5.01428341285755e-07\\
425	5.02642125220279e-07\\
426	5.03878251956168e-07\\
427	5.05137288982068e-07\\
428	5.06419827203559e-07\\
429	5.07726482271197e-07\\
430	5.09057896007294e-07\\
431	5.10414737898169e-07\\
432	5.1179770672642e-07\\
433	5.13207532280221e-07\\
434	5.14644977210104e-07\\
435	5.16110839052974e-07\\
436	5.17605952488326e-07\\
437	5.19131191943217e-07\\
438	5.20687474859379e-07\\
439	5.22275765951502e-07\\
440	5.23897082869642e-07\\
441	5.25552502653173e-07\\
442	5.27243164114881e-07\\
443	5.2897025031219e-07\\
444	5.30734924846659e-07\\
445	5.32538268289753e-07\\
446	5.34381629665944e-07\\
447	5.36266440973359e-07\\
448	5.38194216928926e-07\\
449	5.4016657023308e-07\\
450	5.42185250479107e-07\\
451	5.44252239932341e-07\\
452	5.46369993147793e-07\\
453	5.48542023433573e-07\\
454	5.50774254206475e-07\\
455	5.53077726880314e-07\\
456	5.55472044605509e-07\\
457	5.57978676489192e-07\\
458	5.60537195289438e-07\\
459	5.6313925249674e-07\\
460	5.65794875088331e-07\\
461	5.68525814775929e-07\\
462	5.71376082141709e-07\\
463	5.74413697553009e-07\\
464	5.77673303769232e-07\\
465	5.81332842023018e-07\\
466	5.86155485931426e-07\\
467	5.93913121381586e-07\\
468	6.02796678308234e-07\\
469	6.11827278019675e-07\\
470	6.21011842535799e-07\\
471	6.30356316779779e-07\\
472	6.39863208224527e-07\\
473	6.49526178899239e-07\\
474	6.59319845506786e-07\\
475	6.69186724437135e-07\\
476	6.79045612325569e-07\\
477	6.88913043163912e-07\\
478	7.1191710836285e-07\\
479	7.36251307929028e-07\\
480	7.61283855274156e-07\\
481	7.8706759957787e-07\\
482	8.13671492430865e-07\\
483	8.41200630273418e-07\\
484	8.69849572101272e-07\\
485	9.00045155930217e-07\\
486	9.32825888050371e-07\\
487	9.70793912198952e-07\\
488	1.0197698550852e-06\\
489	1.07349631091925e-06\\
490	1.12819026207342e-06\\
491	1.1838815485207e-06\\
492	1.24059958651208e-06\\
493	1.29836989074042e-06\\
494	1.35720566489286e-06\\
495	1.41708990526517e-06\\
496	1.47794345427834e-06\\
497	1.53959193930114e-06\\
498	1.60183097725021e-06\\
499	1.66489561723599e-06\\
500	1.72946768205313e-06\\
501	1.79565925023842e-06\\
502	1.86364139327674e-06\\
503	1.97547071890742e-06\\
504	2.14963390653748e-06\\
505	2.33186913378658e-06\\
506	2.52587543034996e-06\\
507	2.74397810153313e-06\\
508	2.99255659583446e-06\\
509	3.24410594676526e-06\\
510	3.49866347970021e-06\\
511	3.7562276102568e-06\\
512	4.01675334510043e-06\\
513	4.28015214097894e-06\\
514	4.54550038981188e-06\\
515	4.81392577946725e-06\\
516	5.08616511233571e-06\\
517	5.36234433405714e-06\\
518	5.64253156912134e-06\\
519	5.92666385147519e-06\\
520	6.22197699225318e-06\\
521	6.52325092110364e-06\\
522	6.82768405190497e-06\\
523	7.14319826271199e-06\\
524	7.46858967594716e-06\\
525	7.80362340238037e-06\\
526	8.15494582598335e-06\\
527	8.54647610794914e-06\\
528	9.7814929141284e-06\\
529	1.14106440019494e-05\\
530	1.31283062583692e-05\\
531	1.49323076594289e-05\\
532	1.68675940995809e-05\\
533	2.20733874857024e-05\\
534	3.1423030944064e-05\\
535	4.11197137462416e-05\\
536	5.11971048988361e-05\\
537	6.16939429296373e-05\\
538	7.26557429743505e-05\\
539	8.4140481352317e-05\\
540	9.62439842851792e-05\\
541	0.000183017843515156\\
542	0.000320058431690405\\
543	0.000461979071212767\\
544	0.000609226141479381\\
545	0.000762310686206207\\
546	0.000921818775126619\\
547	0.00108842497921463\\
548	0.00126290890867717\\
549	0.00144617528986034\\
550	0.00163927812649895\\
551	0.00184344890560487\\
552	0.00206012562878466\\
553	0.00228932252299241\\
554	0.00252454541366836\\
555	0.00276610985074724\\
556	0.00301435835283926\\
557	0.00326969255277124\\
558	0.00353252020540563\\
559	0.00380323021433977\\
560	0.00408230479019987\\
561	0.00437023316844891\\
562	0.00466746777496262\\
563	0.00497427187802432\\
564	0.00529105479441378\\
565	0.00561646159330649\\
566	0.00594152902958734\\
567	0.00626327298138607\\
568	0.00643204595372145\\
569	0.00659656549703352\\
570	0.00675547050848819\\
571	0.00690723144372143\\
572	0.00704974781070761\\
573	0.00718309251742785\\
574	0.00731539000987618\\
575	0.00744700693395213\\
576	0.0075776877763659\\
577	0.00770719439458229\\
578	0.0078353307681382\\
579	0.00796197427776425\\
580	0.00808701048954674\\
581	0.00821087605745987\\
582	0.00833514879710221\\
583	0.00845957027804329\\
584	0.00858384582922777\\
585	0.00870763757218698\\
586	0.00883055685842396\\
587	0.00895211666486942\\
588	0.0090717816925381\\
589	0.0091889696268204\\
590	0.00930305497846415\\
591	0.00941337639759054\\
592	0.00951924639550553\\
593	0.00961996226627061\\
594	0.00971482541990144\\
595	0.00980316865229207\\
596	0.00988444283510058\\
597	0.00995832188366689\\
598	0.010000292044645\\
599	0\\
600	0\\
};
\addplot [color=mycolor13,solid,forget plot]
  table[row sep=crcr]{%
1	0.000229137842526684\\
2	0.000229137842526684\\
3	0.000229137842526684\\
4	0.000229137842526684\\
5	0.000229137842526684\\
6	0.000229137842526684\\
7	0.000229137842526684\\
8	0.000229137842526684\\
9	0.000229137842526684\\
10	0.000229137842526684\\
11	0.000229137842526684\\
12	0.000229137842526684\\
13	0.000229137842526684\\
14	0.000229137842526684\\
15	0.000229137842526684\\
16	0.000229137842526684\\
17	0.000229137842526684\\
18	0.000229137842526684\\
19	0.000229137842526684\\
20	0.000229137842526684\\
21	0.000229137842526684\\
22	0.000229137842526684\\
23	0.000229137842526684\\
24	0.000229137842526684\\
25	0.000229137842526684\\
26	0.000229137842526684\\
27	0.000229137842526684\\
28	0.000229137842526684\\
29	0.000229137842526684\\
30	0.000229137842526684\\
31	0.000229137842526684\\
32	0.000229137842526684\\
33	0.000229137842526684\\
34	0.000229137842526684\\
35	0.000229137842526684\\
36	0.000229137842526684\\
37	0.000229137842526684\\
38	0.000229137842526684\\
39	0.000229137842526684\\
40	0.000229137842526684\\
41	0.000229137842526684\\
42	0.000229137842526684\\
43	0.000229137842526684\\
44	0.000229137842526684\\
45	0.000229137842526684\\
46	0.000229137842526684\\
47	0.000229137842526684\\
48	0.000229137842526684\\
49	0.000229137842526684\\
50	0.000229137842526684\\
51	0.000229137842526684\\
52	0.000229137842526684\\
53	0.000229137842526684\\
54	0.000229137842526684\\
55	0.000229137842526684\\
56	0.000229137842526684\\
57	0.000229137842526684\\
58	0.000229137842526684\\
59	0.000229137842526684\\
60	0.000229137842526684\\
61	0.000229137842526684\\
62	0.000229137842526684\\
63	0.000229137842526684\\
64	0.000229137842526684\\
65	0.000229137842526684\\
66	0.000229137842526684\\
67	0.000229137842526684\\
68	0.000229137842526684\\
69	0.000229137842526684\\
70	0.000229137842526684\\
71	0.000229137842526684\\
72	0.000229137842526684\\
73	0.000229137842526684\\
74	0.000229137842526684\\
75	0.000229137842526684\\
76	0.000229137842526684\\
77	0.000229137842526684\\
78	0.000229137842526684\\
79	0.000229137842526684\\
80	0.000229137842526684\\
81	0.000229137842526684\\
82	0.000229137842526684\\
83	0.000229137842526684\\
84	0.000229137842526684\\
85	0.000229137842526684\\
86	0.000229137842526684\\
87	0.000229137842526684\\
88	0.000229137842526684\\
89	0.000229137842526684\\
90	0.000229137842526684\\
91	0.000229137842526684\\
92	0.000229137842526684\\
93	0.000229137842526684\\
94	0.000229137842526684\\
95	0.000229137842526684\\
96	0.000229137842526684\\
97	0.000229137842526684\\
98	0.000229137842526684\\
99	0.000229137842526684\\
100	0.000229137842526684\\
101	0.000229137842526684\\
102	0.000229137842526684\\
103	0.000229137842526684\\
104	0.000229137842526684\\
105	0.000229137842526684\\
106	0.000229137842526684\\
107	0.000229137842526684\\
108	0.000229137842526684\\
109	0.000229137842526684\\
110	0.000229137842526684\\
111	0.000229137842526684\\
112	0.000229137842526684\\
113	0.000229137842526684\\
114	0.000229137842526684\\
115	0.000229137842526684\\
116	0.000229137842526684\\
117	0.000229137842526684\\
118	0.000229137842526684\\
119	0.000229137842526684\\
120	0.000229137842526684\\
121	0.000229137842526684\\
122	0.000229137842526684\\
123	0.000229137842526684\\
124	0.000229137842526684\\
125	0.000229137842526684\\
126	0.000229137842526684\\
127	0.000229137842526684\\
128	0.000229137842526684\\
129	0.000229137842526684\\
130	0.000229137842526684\\
131	0.000229137842526684\\
132	0.000229137842526684\\
133	0.000229137842526684\\
134	0.000229137842526684\\
135	0.000229137842526684\\
136	0.000229137842526684\\
137	0.000229137842526684\\
138	0.000229137842526684\\
139	0.000229137842526684\\
140	0.000229137842526684\\
141	0.000229137842526684\\
142	0.000229137842526684\\
143	0.000229137842526684\\
144	0.000229137842526684\\
145	0.000229137842526684\\
146	0.000229137842526684\\
147	0.000229137842526684\\
148	0.000229137842526684\\
149	0.000229137842526684\\
150	0.000229137842526684\\
151	0.000229137842526684\\
152	0.000229137842526684\\
153	0.000229137842526684\\
154	0.000229137842526684\\
155	0.000229137842526684\\
156	0.000229137842526684\\
157	0.000229137842526684\\
158	0.000229137842526684\\
159	0.000229137842526684\\
160	0.000229137842526684\\
161	0.000229137842526684\\
162	0.000229137842526684\\
163	0.000229137842526684\\
164	0.000229137842526684\\
165	0.000229137842526684\\
166	0.000229137842526684\\
167	0.000229137842526684\\
168	0.000229137842526684\\
169	0.000229137842526684\\
170	0.000229137842526684\\
171	0.000229137842526684\\
172	0.000229137842526684\\
173	0.000229137842526684\\
174	0.000229137842526684\\
175	0.000229137842526684\\
176	0.000229137842526684\\
177	0.000229137842526684\\
178	0.000229137842526684\\
179	0.000229137842526684\\
180	0.000229137842526684\\
181	0.000229137842526684\\
182	0.000229137842526684\\
183	0.000229137842526684\\
184	0.000229137842526684\\
185	0.000229137842526684\\
186	0.000229137842526684\\
187	0.000229137842526684\\
188	0.000229137842526684\\
189	0.000229137842526684\\
190	0.000229137842526684\\
191	0.000229137842526684\\
192	0.000229137842526684\\
193	0.000229137842526684\\
194	0.000229137842526684\\
195	0.000229137842526684\\
196	0.000229137842526684\\
197	0.000229137842526684\\
198	0.000229137842526684\\
199	0.000229137842526684\\
200	0.000229137842526684\\
201	0.000229137842526684\\
202	0.000229137842526684\\
203	0.000229137842526684\\
204	0.000229137842526684\\
205	0.000229137842526684\\
206	0.000229137842526684\\
207	0.000229137842526684\\
208	0.000229137842526684\\
209	0.000229137842526684\\
210	0.000229137842526684\\
211	0.000229137842526684\\
212	0.000229137842526684\\
213	0.000229137842526684\\
214	0.000229137842526684\\
215	0.000229137842526684\\
216	0.000229137842526684\\
217	0.000229137842526684\\
218	0.000229137842526684\\
219	0.000229137842526684\\
220	0.000229137842526684\\
221	0.000229137842526684\\
222	0.000229137842526684\\
223	0.000229137842526684\\
224	0.000229137842526684\\
225	0.000229137842526684\\
226	0.000229137842526684\\
227	0.000229137842526684\\
228	0.000229137842526684\\
229	0.000229137842526684\\
230	0.000229137842526684\\
231	0.000229137842526684\\
232	0.000229137842526684\\
233	0.000229137842526684\\
234	0.000229137842526684\\
235	0.000229137842526684\\
236	0.000229137842526684\\
237	0.000229137842526684\\
238	0.000229137842526684\\
239	0.000229137842526684\\
240	0.000229137842526684\\
241	0.000229137842526684\\
242	0.000229137842526684\\
243	0.000229137842526684\\
244	0.000229137842526684\\
245	0.000229137842526684\\
246	0.000229137842526684\\
247	0.000229137842526684\\
248	0.000229137842526684\\
249	0.000229137842526684\\
250	0.000229137842526684\\
251	0.000229137842526684\\
252	0.000229137842526684\\
253	0.000229137842526684\\
254	0.000229137842526684\\
255	0.000229137842526684\\
256	0.000229137842526684\\
257	0.000229137842526684\\
258	0.000229137842526684\\
259	0.000229137842526684\\
260	0.000229137842526684\\
261	0.000229137842526684\\
262	0.000229137842526684\\
263	0.000229137842526684\\
264	0.000229137842526684\\
265	0.000229137842526684\\
266	0.000229137842526684\\
267	0.000229137842526684\\
268	0.000229137842526684\\
269	0.000229137842526684\\
270	0.000229137842526684\\
271	0.000229137842526684\\
272	0.000229137842526684\\
273	0.000229137842526684\\
274	0.000229137842526684\\
275	0.000229137842526684\\
276	0.000229137842526684\\
277	0.000229137842526684\\
278	0.000229137842526684\\
279	0.000229137842526684\\
280	0.000229137842526684\\
281	0.000229137842526684\\
282	0.000229137842526684\\
283	0.000229137842526684\\
284	0.000229137842526684\\
285	0.000229137842526684\\
286	0.000229137842526684\\
287	0.000229137842526684\\
288	0.000229137842526684\\
289	0.000229137842526684\\
290	0.000229137842526684\\
291	0.000229137842526684\\
292	0.000229137842526684\\
293	0.000229137842526684\\
294	0.000229137842526684\\
295	0.000229137842526684\\
296	0.000229137842526684\\
297	0.000229137842526684\\
298	0.000229137842526684\\
299	0.000229137842526684\\
300	0.000229137842526684\\
301	0.000229137842526684\\
302	0.000229137842526684\\
303	0.000229137842526684\\
304	0.000229137842526684\\
305	0.000229137842526684\\
306	0.000229137842526684\\
307	0.000229137842526684\\
308	0.000229137842526684\\
309	0.000229137842526684\\
310	0.000229137842526684\\
311	0.000229137842526684\\
312	0.000229137842526684\\
313	0.000229137842526684\\
314	0.000229137842526684\\
315	0.000229137842526684\\
316	0.000229137842526684\\
317	0.000229137842526684\\
318	0.000229137842526684\\
319	0.000229137842526684\\
320	0.000229137842526684\\
321	0.000229137842526684\\
322	0.000229137842526684\\
323	0.000229137842526684\\
324	0.000229137842526684\\
325	0.000229137842526684\\
326	0.000229137842526684\\
327	0.000229137842526684\\
328	0.000229137842526684\\
329	0.000229137842526684\\
330	0.000229137842526684\\
331	0.000229137842526684\\
332	0.000229137842526684\\
333	0.000229137842526684\\
334	0.000229137842526684\\
335	0.000229137842526684\\
336	0.000229137842526684\\
337	0.000229137842526684\\
338	0.000229137842526684\\
339	0.000229137842526684\\
340	0.000229137842526684\\
341	0.000229137842526684\\
342	0.000229137842526684\\
343	0.000229137842526684\\
344	0.000229137842526684\\
345	0.000229137842526684\\
346	0.000229137842526684\\
347	0.000229137842526684\\
348	0.000229137842526684\\
349	0.000229137842526684\\
350	0.000229137842526684\\
351	0.000229137842526684\\
352	0.000229137842526684\\
353	0.000229137842526684\\
354	0.000229137842526684\\
355	0.000229137842526684\\
356	0.000229137842526684\\
357	0.000229137842526684\\
358	0.000229137842526684\\
359	0.000229137842526684\\
360	0.000229137842526684\\
361	0.000229137842526684\\
362	0.000229137842526684\\
363	0.000229137842526684\\
364	0.000229137842526684\\
365	0.000229137842526684\\
366	0.000229137842526684\\
367	0.000229137842526684\\
368	0.000229137842526684\\
369	0.000229137842526684\\
370	0.000229137842526684\\
371	0.000229137842526684\\
372	0.000229137842526684\\
373	0.000229137842526684\\
374	0.000229137842526684\\
375	0.000229137842526684\\
376	0.000229137842526684\\
377	0.000229137842526684\\
378	0.000229137842526684\\
379	0.000229137842526684\\
380	0.000229137842526684\\
381	0.000229137842526684\\
382	0.000229137842526684\\
383	0.000229137842526684\\
384	0.000229137842526684\\
385	0.000229137842526684\\
386	0.000229137842526684\\
387	0.000229137842526684\\
388	0.000229137842526684\\
389	0.000229137842526684\\
390	0.000229137842526684\\
391	0.000229137842526684\\
392	0.000229137842526684\\
393	0.000229137842526684\\
394	0.000229137842526684\\
395	0.000229137842526684\\
396	0.000229137842526684\\
397	0.000229137842526684\\
398	0.000229137842526684\\
399	0.000229137842526684\\
400	0.000229137842526684\\
401	0.000229137842526684\\
402	0.000229137842526684\\
403	0.000229137842526684\\
404	0.000229137842526684\\
405	0.000229137842526684\\
406	0.000229137842526684\\
407	0.000229137842526684\\
408	0.000229137842526684\\
409	0.000229137842526684\\
410	0.000229137842526684\\
411	0.000229137842526684\\
412	0.000229137842526684\\
413	0.000229137842526684\\
414	0.000229137842526684\\
415	0.000229137842526684\\
416	0.000229137842526684\\
417	0.000229137842526684\\
418	0.000229137842526684\\
419	0.000229137842526684\\
420	0.000229137842526684\\
421	0.000229137842526684\\
422	0.000229137842526684\\
423	0.000229137842526684\\
424	0.000229137842526684\\
425	0.000229137842526684\\
426	0.000229137842526684\\
427	0.000229137842526684\\
428	0.000229137842526684\\
429	0.000229137842526684\\
430	0.000229137842526684\\
431	0.000229137842526684\\
432	0.000229137842526684\\
433	0.000229137842526684\\
434	0.000229137842526684\\
435	0.000229137842526684\\
436	0.000229137842526684\\
437	0.000229137842526684\\
438	0.000229137842526684\\
439	0.000229137842526684\\
440	0.000229137842526684\\
441	0.000229137842526684\\
442	0.000229137842526684\\
443	0.000229137842526684\\
444	0.000229137842526684\\
445	0.000229137842526684\\
446	0.000229137842526684\\
447	0.000229137842526684\\
448	0.000229137842526684\\
449	0.000229137842526684\\
450	0.000229137842526684\\
451	0.000229137842526684\\
452	0.000229137842526684\\
453	0.000229137842526684\\
454	0.000229137842526684\\
455	0.000229137842526684\\
456	0.000229137842526684\\
457	0.000229137842526684\\
458	0.000229137842526684\\
459	0.000229137842526684\\
460	0.000229137842526684\\
461	0.000229137842526684\\
462	0.000229137842526684\\
463	0.000229137842526684\\
464	0.000229137842526684\\
465	0.000229137842526684\\
466	0.000229137842526684\\
467	0.000229137842526684\\
468	0.000229137842526684\\
469	0.000229137842526684\\
470	0.000229137842526684\\
471	0.000229137842526684\\
472	0.000229137842526684\\
473	0.000229137842526684\\
474	0.000229137842526684\\
475	0.000229137842526684\\
476	0.000229137842526684\\
477	0.000229137842526684\\
478	0.000229137842526684\\
479	0.000229137842526684\\
480	0.000229137842526684\\
481	0.000229137842526684\\
482	0.000229137842526684\\
483	0.000229137842526684\\
484	0.000229137842526684\\
485	0.000229137842526684\\
486	0.000229137842526684\\
487	0.000229137842526684\\
488	0.000229137842526684\\
489	0.000229137842526684\\
490	0.000229137842526684\\
491	0.000229137842526684\\
492	0.000229137842526684\\
493	0.000229137842526684\\
494	0.000229137842526684\\
495	0.000229137842526684\\
496	0.000229137842526684\\
497	0.000229137842526684\\
498	0.000229137842526684\\
499	0.000229137842526684\\
500	0.000229137842526684\\
501	0.000229137842526684\\
502	0.000229137842526684\\
503	0.000229137842526684\\
504	0.000229137842526684\\
505	0.000229137842526684\\
506	0.000229137842526684\\
507	0.000229137842526684\\
508	0.000229137842526684\\
509	0.000229137842526684\\
510	0.000229137842526684\\
511	0.000229137842526684\\
512	0.000229137842526684\\
513	0.000229137842526684\\
514	0.000229137842526684\\
515	0.000229137842526684\\
516	0.000229137842526684\\
517	0.000229137842526684\\
518	0.000229137842526684\\
519	0.000229137842526684\\
520	0.000229137842526684\\
521	0.000229137842526684\\
522	0.000229137842526684\\
523	0.000229137842526684\\
524	0.000229137842526684\\
525	0.000229137842526684\\
526	0.000229137842526684\\
527	0.000229137842526684\\
528	0.000229137842526684\\
529	0.000229137842526684\\
530	0.000229137842526684\\
531	0.000229137842526684\\
532	0.000229137842526684\\
533	0.000229137842526684\\
534	0.000229137842526684\\
535	0.000229137842526684\\
536	0.000229137842526684\\
537	0.000229137842526684\\
538	0.000229137842526684\\
539	0.000229137842526684\\
540	0.000229137842526684\\
541	0.000229137842526684\\
542	0.000229137842526684\\
543	0.000229137842526684\\
544	0.000229137842526684\\
545	0.000229137842526684\\
546	0.000229137842526684\\
547	0.000229137842526684\\
548	0.000229137842526684\\
549	0.000229137842526684\\
550	0.000229137842526684\\
551	0.000229137842526684\\
552	0.000229137842526684\\
553	0.000229137842526684\\
554	0.000229137842526684\\
555	0.000229137842526684\\
556	0.000229137842526684\\
557	0.000229137842526684\\
558	0.000229137842526684\\
559	0.000229137842526684\\
560	0.000229137842526684\\
561	0.000229137842526684\\
562	0.000229137842526684\\
563	0.000229137842526684\\
564	0.000229137842526684\\
565	0.00023087401359618\\
566	0.000242990420949804\\
567	0.000269136136587107\\
568	0.000448812459300998\\
569	0.000637589915763916\\
570	0.000838128118940092\\
571	0.00105199896671285\\
572	0.00128105703203953\\
573	0.00152512637960484\\
574	0.00177692804255711\\
575	0.00203669514046483\\
576	0.00230473255962503\\
577	0.00258130913841595\\
578	0.00286642117413111\\
579	0.00316021118843213\\
580	0.00346264275586949\\
581	0.00377236863713507\\
582	0.00408776308700265\\
583	0.00440895297433541\\
584	0.00473600934600418\\
585	0.0050690485378831\\
586	0.00540818140277331\\
587	0.00575350248078163\\
588	0.00610506048892851\\
589	0.00646281347291639\\
590	0.00682657112268328\\
591	0.00719605229431493\\
592	0.0075711377940388\\
593	0.00795166435354403\\
594	0.00833736409264152\\
595	0.00872777975004381\\
596	0.00912205985501356\\
597	0.00951842928902401\\
598	0.00991325377008988\\
599	0\\
600	0\\
};
\addplot [color=mycolor14,solid,forget plot]
  table[row sep=crcr]{%
1	0.0100002859420401\\
2	0.0100002859420007\\
3	0.0100002859419605\\
4	0.0100002859419196\\
5	0.0100002859418779\\
6	0.0100002859418355\\
7	0.0100002859417923\\
8	0.0100002859417484\\
9	0.0100002859417036\\
10	0.010000285941658\\
11	0.0100002859416117\\
12	0.0100002859415644\\
13	0.0100002859415163\\
14	0.0100002859414674\\
15	0.0100002859414176\\
16	0.0100002859413668\\
17	0.0100002859413152\\
18	0.0100002859412626\\
19	0.010000285941209\\
20	0.0100002859411545\\
21	0.010000285941099\\
22	0.0100002859410425\\
23	0.010000285940985\\
24	0.0100002859409265\\
25	0.0100002859408668\\
26	0.0100002859408061\\
27	0.0100002859407443\\
28	0.0100002859406814\\
29	0.0100002859406174\\
30	0.0100002859405522\\
31	0.0100002859404858\\
32	0.0100002859404182\\
33	0.0100002859403494\\
34	0.0100002859402793\\
35	0.010000285940208\\
36	0.0100002859401354\\
37	0.0100002859400615\\
38	0.0100002859399862\\
39	0.0100002859399096\\
40	0.0100002859398316\\
41	0.0100002859397522\\
42	0.0100002859396714\\
43	0.0100002859395891\\
44	0.0100002859395053\\
45	0.01000028593942\\
46	0.0100002859393331\\
47	0.0100002859392447\\
48	0.0100002859391547\\
49	0.010000285939063\\
50	0.0100002859389697\\
51	0.0100002859388748\\
52	0.0100002859387781\\
53	0.0100002859386796\\
54	0.0100002859385794\\
55	0.0100002859384774\\
56	0.0100002859383735\\
57	0.0100002859382677\\
58	0.0100002859381601\\
59	0.0100002859380505\\
60	0.0100002859379389\\
61	0.0100002859378253\\
62	0.0100002859377097\\
63	0.0100002859375919\\
64	0.0100002859374721\\
65	0.0100002859373501\\
66	0.0100002859372258\\
67	0.0100002859370994\\
68	0.0100002859369706\\
69	0.0100002859368396\\
70	0.0100002859367061\\
71	0.0100002859365703\\
72	0.010000285936432\\
73	0.0100002859362912\\
74	0.0100002859361479\\
75	0.010000285936002\\
76	0.0100002859358534\\
77	0.0100002859357022\\
78	0.0100002859355483\\
79	0.0100002859353916\\
80	0.010000285935232\\
81	0.0100002859350696\\
82	0.0100002859349042\\
83	0.0100002859347359\\
84	0.0100002859345645\\
85	0.0100002859343901\\
86	0.0100002859342125\\
87	0.0100002859340317\\
88	0.0100002859338476\\
89	0.0100002859336602\\
90	0.0100002859334695\\
91	0.0100002859332753\\
92	0.0100002859330776\\
93	0.0100002859328764\\
94	0.0100002859326715\\
95	0.0100002859324629\\
96	0.0100002859322506\\
97	0.0100002859320344\\
98	0.0100002859318144\\
99	0.0100002859315904\\
100	0.0100002859313624\\
101	0.0100002859311302\\
102	0.0100002859308939\\
103	0.0100002859306533\\
104	0.0100002859304084\\
105	0.0100002859301591\\
106	0.0100002859299053\\
107	0.010000285929647\\
108	0.010000285929384\\
109	0.0100002859291162\\
110	0.0100002859288436\\
111	0.0100002859285662\\
112	0.0100002859282837\\
113	0.0100002859279962\\
114	0.0100002859277035\\
115	0.0100002859274055\\
116	0.0100002859271022\\
117	0.0100002859267934\\
118	0.010000285926479\\
119	0.0100002859261591\\
120	0.0100002859258333\\
121	0.0100002859255017\\
122	0.0100002859251642\\
123	0.0100002859248205\\
124	0.0100002859244707\\
125	0.0100002859241147\\
126	0.0100002859237522\\
127	0.0100002859233832\\
128	0.0100002859230075\\
129	0.0100002859226252\\
130	0.0100002859222359\\
131	0.0100002859218397\\
132	0.0100002859214364\\
133	0.0100002859210258\\
134	0.0100002859206078\\
135	0.0100002859201823\\
136	0.0100002859197492\\
137	0.0100002859193083\\
138	0.0100002859188596\\
139	0.0100002859184027\\
140	0.0100002859179377\\
141	0.0100002859174643\\
142	0.0100002859169824\\
143	0.0100002859164918\\
144	0.0100002859159925\\
145	0.0100002859154842\\
146	0.0100002859149668\\
147	0.0100002859144401\\
148	0.010000285913904\\
149	0.0100002859133583\\
150	0.0100002859128027\\
151	0.0100002859122372\\
152	0.0100002859116616\\
153	0.0100002859110757\\
154	0.0100002859104792\\
155	0.0100002859098721\\
156	0.010000285909254\\
157	0.0100002859086249\\
158	0.0100002859079846\\
159	0.0100002859073327\\
160	0.0100002859066692\\
161	0.0100002859059938\\
162	0.0100002859053063\\
163	0.0100002859046065\\
164	0.0100002859038942\\
165	0.0100002859031691\\
166	0.010000285902431\\
167	0.0100002859016797\\
168	0.0100002859009149\\
169	0.0100002859001365\\
170	0.0100002858993441\\
171	0.0100002858985376\\
172	0.0100002858977166\\
173	0.0100002858968809\\
174	0.0100002858960303\\
175	0.0100002858951645\\
176	0.0100002858942831\\
177	0.010000285893386\\
178	0.0100002858924729\\
179	0.0100002858915434\\
180	0.0100002858905973\\
181	0.0100002858896343\\
182	0.0100002858886541\\
183	0.0100002858876564\\
184	0.0100002858866408\\
185	0.010000285885607\\
186	0.0100002858845548\\
187	0.0100002858834838\\
188	0.0100002858823936\\
189	0.010000285881284\\
190	0.0100002858801545\\
191	0.0100002858790049\\
192	0.0100002858778347\\
193	0.0100002858766436\\
194	0.0100002858754313\\
195	0.0100002858741973\\
196	0.0100002858729412\\
197	0.0100002858716627\\
198	0.0100002858703614\\
199	0.0100002858690369\\
200	0.0100002858676887\\
201	0.0100002858663165\\
202	0.0100002858649197\\
203	0.010000285863498\\
204	0.010000285862051\\
205	0.0100002858605781\\
206	0.010000285859079\\
207	0.010000285857553\\
208	0.0100002858559999\\
209	0.0100002858544191\\
210	0.01000028585281\\
211	0.0100002858511723\\
212	0.0100002858495053\\
213	0.0100002858478086\\
214	0.0100002858460817\\
215	0.0100002858443239\\
216	0.0100002858425348\\
217	0.0100002858407138\\
218	0.0100002858388603\\
219	0.0100002858369738\\
220	0.0100002858350536\\
221	0.0100002858330993\\
222	0.01000028583111\\
223	0.0100002858290853\\
224	0.0100002858270246\\
225	0.0100002858249271\\
226	0.0100002858227922\\
227	0.0100002858206192\\
228	0.0100002858184075\\
229	0.0100002858161565\\
230	0.0100002858138653\\
231	0.0100002858115332\\
232	0.0100002858091597\\
233	0.0100002858067438\\
234	0.0100002858042849\\
235	0.0100002858017821\\
236	0.0100002857992348\\
237	0.0100002857966421\\
238	0.0100002857940032\\
239	0.0100002857913173\\
240	0.0100002857885836\\
241	0.0100002857858011\\
242	0.0100002857829691\\
243	0.0100002857800867\\
244	0.0100002857771528\\
245	0.0100002857741668\\
246	0.0100002857711275\\
247	0.0100002857680341\\
248	0.0100002857648856\\
249	0.010000285761681\\
250	0.0100002857584194\\
251	0.0100002857550996\\
252	0.0100002857517207\\
253	0.0100002857482816\\
254	0.0100002857447813\\
255	0.0100002857412186\\
256	0.0100002857375925\\
257	0.0100002857339017\\
258	0.0100002857301452\\
259	0.0100002857263218\\
260	0.0100002857224303\\
261	0.0100002857184694\\
262	0.010000285714438\\
263	0.0100002857103348\\
264	0.0100002857061585\\
265	0.0100002857019077\\
266	0.0100002856975812\\
267	0.0100002856931777\\
268	0.0100002856886956\\
269	0.0100002856841337\\
270	0.0100002856794905\\
271	0.0100002856747646\\
272	0.0100002856699544\\
273	0.0100002856650585\\
274	0.0100002856600753\\
275	0.0100002856550033\\
276	0.0100002856498409\\
277	0.0100002856445865\\
278	0.0100002856392384\\
279	0.0100002856337949\\
280	0.0100002856282544\\
281	0.0100002856226151\\
282	0.0100002856168752\\
283	0.0100002856110329\\
284	0.0100002856050865\\
285	0.010000285599034\\
286	0.0100002855928735\\
287	0.0100002855866031\\
288	0.0100002855802208\\
289	0.0100002855737247\\
290	0.0100002855671127\\
291	0.0100002855603826\\
292	0.0100002855535325\\
293	0.01000028554656\\
294	0.0100002855394632\\
295	0.0100002855322396\\
296	0.0100002855248871\\
297	0.0100002855174033\\
298	0.0100002855097859\\
299	0.0100002855020324\\
300	0.0100002854941405\\
301	0.0100002854861076\\
302	0.0100002854779313\\
303	0.0100002854696089\\
304	0.0100002854611379\\
305	0.0100002854525155\\
306	0.0100002854437391\\
307	0.0100002854348059\\
308	0.0100002854257131\\
309	0.0100002854164578\\
310	0.0100002854070372\\
311	0.0100002853974482\\
312	0.0100002853876879\\
313	0.0100002853777532\\
314	0.0100002853676411\\
315	0.0100002853573482\\
316	0.0100002853468715\\
317	0.0100002853362076\\
318	0.0100002853253532\\
319	0.010000285314305\\
320	0.0100002853030595\\
321	0.0100002852916132\\
322	0.0100002852799626\\
323	0.010000285268104\\
324	0.0100002852560339\\
325	0.0100002852437485\\
326	0.0100002852312441\\
327	0.0100002852185168\\
328	0.0100002852055627\\
329	0.0100002851923779\\
330	0.0100002851789585\\
331	0.0100002851653003\\
332	0.010000285151399\\
333	0.0100002851372501\\
334	0.0100002851228489\\
335	0.0100002851081895\\
336	0.0100002850932653\\
337	0.0100002850780708\\
338	0.010000285062604\\
339	0.0100002850468651\\
340	0.0100002850308494\\
341	0.0100002850145525\\
342	0.0100002849979691\\
343	0.0100002849810934\\
344	0.0100002849639176\\
345	0.0100002849464311\\
346	0.01000028492862\\
347	0.0100002849104738\\
348	0.0100002848920035\\
349	0.0100002848732209\\
350	0.0100002848541209\\
351	0.010000284834698\\
352	0.0100002848149469\\
353	0.0100002847948621\\
354	0.010000284774438\\
355	0.010000284753669\\
356	0.0100002847325493\\
357	0.0100002847110732\\
358	0.0100002846892346\\
359	0.0100002846670276\\
360	0.0100002846444462\\
361	0.010000284621484\\
362	0.0100002845981349\\
363	0.0100002845743925\\
364	0.0100002845502503\\
365	0.0100002845257017\\
366	0.0100002845007402\\
367	0.010000284475359\\
368	0.0100002844495512\\
369	0.01000028442331\\
370	0.0100002843966282\\
371	0.0100002843694987\\
372	0.0100002843419144\\
373	0.0100002843138678\\
374	0.0100002842853515\\
375	0.0100002842563579\\
376	0.0100002842268794\\
377	0.0100002841969083\\
378	0.0100002841664373\\
379	0.0100002841354591\\
380	0.0100002841039676\\
381	0.010000284071959\\
382	0.0100002840394324\\
383	0.010000284006391\\
384	0.0100002839728374\\
385	0.0100002839387595\\
386	0.0100002839041161\\
387	0.0100002838688902\\
388	0.0100002838330719\\
389	0.0100002837966509\\
390	0.0100002837596164\\
391	0.0100002837219574\\
392	0.0100002836836623\\
393	0.0100002836447191\\
394	0.0100002836051154\\
395	0.0100002835648387\\
396	0.010000283523877\\
397	0.0100002834822202\\
398	0.0100002834398633\\
399	0.0100002833968121\\
400	0.0100002833530918\\
401	0.0100002833087522\\
402	0.0100002832638447\\
403	0.0100002832183229\\
404	0.0100002831719307\\
405	0.0100002831246517\\
406	0.0100002830764681\\
407	0.0100002830273611\\
408	0.0100002829773097\\
409	0.010000282926292\\
410	0.0100002828742905\\
411	0.0100002828213006\\
412	0.0100002827672993\\
413	0.0100002827122627\\
414	0.0100002826561662\\
415	0.0100002825989842\\
416	0.0100002825406901\\
417	0.0100002824812564\\
418	0.0100002824206546\\
419	0.0100002823588548\\
420	0.010000282295826\\
421	0.010000282231536\\
422	0.0100002821659511\\
423	0.0100002820990362\\
424	0.0100002820307548\\
425	0.0100002819610685\\
426	0.0100002818899375\\
427	0.0100002818173197\\
428	0.0100002817431712\\
429	0.0100002816674453\\
430	0.0100002815900927\\
431	0.0100002815110593\\
432	0.0100002814302846\\
433	0.010000281347695\\
434	0.010000281263193\\
435	0.0100002811766399\\
436	0.0100002810878466\\
437	0.0100002809966367\\
438	0.0100002809030812\\
439	0.0100002808073389\\
440	0.0100002807093254\\
441	0.0100002806089503\\
442	0.0100002805061168\\
443	0.0100002804007212\\
444	0.0100002802926518\\
445	0.0100002801817893\\
446	0.0100002800680067\\
447	0.0100002799511711\\
448	0.0100002798311475\\
449	0.0100002797078077\\
450	0.0100002795810433\\
451	0.0100002794507622\\
452	0.0100002793167834\\
453	0.0100002791783705\\
454	0.0100002790329246\\
455	0.0100002788742218\\
456	0.0100002786912104\\
457	0.0100002784662658\\
458	0.0100002781385529\\
459	0.0100002777823409\\
460	0.0100002774191207\\
461	0.0100002770480306\\
462	0.0100002766683527\\
463	0.0100002762820748\\
464	0.0100002758890429\\
465	0.0100002754890043\\
466	0.0100002750816683\\
467	0.0100002746666901\\
468	0.0100002742436629\\
469	0.010000273812189\\
470	0.010000273372242\\
471	0.0100002729235318\\
472	0.0100002724657608\\
473	0.010000271998872\\
474	0.0100002715233001\\
475	0.0100002710401318\\
476	0.0100002705501455\\
477	0.0100002700492408\\
478	0.010000269532826\\
479	0.0100002689996552\\
480	0.0100002684479949\\
481	0.0100002678749656\\
482	0.0100002672747749\\
483	0.0100002666339708\\
484	0.0100002659191922\\
485	0.0100002650527385\\
486	0.0100002639818543\\
487	0.0100002628908232\\
488	0.0100002617790337\\
489	0.0100002606458778\\
490	0.0100002594907488\\
491	0.0100002583130897\\
492	0.0100002571124782\\
493	0.0100002556337237\\
494	0.0100002527280105\\
495	0.0100002497661292\\
496	0.0100002467509689\\
497	0.010000243677361\\
498	0.010000240524656\\
499	0.0100002372882573\\
500	0.0100002339625459\\
501	0.010000230539997\\
502	0.0100002270092879\\
503	0.0100002233519699\\
504	0.0100002195409957\\
505	0.0100002155711457\\
506	0.0100002114097909\\
507	0.0100002069276187\\
508	0.0100002017910705\\
509	0.0100001961855836\\
510	0.0100001905050262\\
511	0.010000184748613\\
512	0.0100001789157995\\
513	0.0100001730075422\\
514	0.0100001670608041\\
515	0.0100001610419521\\
516	0.0100001549299995\\
517	0.0100001487215882\\
518	0.0100001424139541\\
519	0.010000136005646\\
520	0.0100001294962251\\
521	0.0100001226945644\\
522	0.0100001156869292\\
523	0.010000108464679\\
524	0.0100001010177974\\
525	0.0100000931838069\\
526	0.010000084356173\\
527	0.0100000538977315\\
528	0.0100000193361654\\
529	0.00999998321785323\\
530	0.0099999452882706\\
531	0.00999990517493529\\
532	0.00999986185080286\\
533	0.00999973171419417\\
534	0.00999952840856016\\
535	0.00999931776418426\\
536	0.00999909913812807\\
537	0.00999887174402923\\
538	0.00999863468328567\\
539	0.00999838690736834\\
540	0.00999812704560558\\
541	0.00999779878085236\\
542	0.0099948011198057\\
543	0.009991775940215\\
544	0.00998872111588451\\
545	0.00998563430540921\\
546	0.00998251297075415\\
547	0.00997935439821524\\
548	0.00997615571946956\\
549	0.00997291395177169\\
550	0.00996962606791562\\
551	0.00996628911718749\\
552	0.00996290045236572\\
553	0.00995945824145994\\
554	0.00995607539716139\\
555	0.00995278083794119\\
556	0.00994957833519301\\
557	0.00994647191919197\\
558	0.0099434660064334\\
559	0.00994056475167603\\
560	0.00993777262166188\\
561	0.00993509473746594\\
562	0.00993253721631877\\
563	0.00993010821294715\\
564	0.00992781407422411\\
565	0.00992488752606354\\
566	0.00991049821720781\\
567	0.00989541754955392\\
568	0.00973893862780728\\
569	0.00954294358828082\\
570	0.00933480713277825\\
571	0.00911292764504251\\
572	0.00887542241305692\\
573	0.0086216917751517\\
574	0.008360185452806\\
575	0.00809066765029874\\
576	0.00781282768321296\\
577	0.00752645339503333\\
578	0.00723148144731463\\
579	0.00692771451397655\\
580	0.00661569752735087\\
581	0.00629589973465196\\
582	0.00597019483658525\\
583	0.00563888259460737\\
584	0.00530187182734894\\
585	0.00495904947440365\\
586	0.0046102952828595\\
587	0.00425551508307342\\
588	0.00389467562961874\\
589	0.00352786639485424\\
590	0.00315529010407331\\
591	0.00277712112335691\\
592	0.00239346664055391\\
593	0.00200447501126164\\
594	0.00161039185165808\\
595	0.00121163414212813\\
596	0.000808977057617982\\
597	0.000404030214303389\\
598	2.9204464504877e-07\\
599	0\\
600	0\\
};
\addplot [color=mycolor15,solid,forget plot]
  table[row sep=crcr]{%
1	0.0099993294716719\\
2	0.00999932946961846\\
3	0.00999932946752786\\
4	0.00999932946539942\\
5	0.00999932946323247\\
6	0.0099993294610263\\
7	0.00999932945878021\\
8	0.00999932945649348\\
9	0.00999932945416537\\
10	0.00999932945179514\\
11	0.00999932944938202\\
12	0.00999932944692525\\
13	0.00999932944442402\\
14	0.00999932944187754\\
15	0.009999329439285\\
16	0.00999932943664556\\
17	0.00999932943395837\\
18	0.00999932943122257\\
19	0.00999932942843729\\
20	0.00999932942560163\\
21	0.00999932942271469\\
22	0.00999932941977553\\
23	0.00999932941678321\\
24	0.00999932941373678\\
25	0.00999932941063526\\
26	0.00999932940747765\\
27	0.00999932940426295\\
28	0.00999932940099012\\
29	0.00999932939765812\\
30	0.00999932939426587\\
31	0.00999932939081229\\
32	0.00999932938729628\\
33	0.00999932938371671\\
34	0.00999932938007243\\
35	0.00999932937636227\\
36	0.00999932937258505\\
37	0.00999932936873956\\
38	0.00999932936482457\\
39	0.00999932936083882\\
40	0.00999932935678105\\
41	0.00999932935264994\\
42	0.00999932934844418\\
43	0.00999932934416243\\
44	0.00999932933980331\\
45	0.00999932933536544\\
46	0.00999932933084738\\
47	0.0099993293262477\\
48	0.00999932932156493\\
49	0.00999932931679757\\
50	0.0099993293119441\\
51	0.00999932930700296\\
52	0.00999932930197258\\
53	0.00999932929685135\\
54	0.00999932929163763\\
55	0.00999932928632976\\
56	0.00999932928092604\\
57	0.00999932927542475\\
58	0.00999932926982413\\
59	0.0099993292641224\\
60	0.00999932925831773\\
61	0.00999932925240827\\
62	0.00999932924639214\\
63	0.00999932924026742\\
64	0.00999932923403215\\
65	0.00999932922768434\\
66	0.00999932922122197\\
67	0.00999932921464298\\
68	0.00999932920794527\\
69	0.00999932920112671\\
70	0.00999932919418512\\
71	0.00999932918711829\\
72	0.00999932917992397\\
73	0.00999932917259987\\
74	0.00999932916514366\\
75	0.00999932915755295\\
76	0.00999932914982534\\
77	0.00999932914195835\\
78	0.0099993291339495\\
79	0.00999932912579623\\
80	0.00999932911749593\\
81	0.00999932910904599\\
82	0.00999932910044369\\
83	0.00999932909168632\\
84	0.00999932908277107\\
85	0.00999932907369513\\
86	0.00999932906445559\\
87	0.00999932905504954\\
88	0.00999932904547396\\
89	0.00999932903572584\\
90	0.00999932902580205\\
91	0.00999932901569946\\
92	0.00999932900541485\\
93	0.00999932899494496\\
94	0.00999932898428646\\
95	0.00999932897343597\\
96	0.00999932896239005\\
97	0.00999932895114518\\
98	0.00999932893969781\\
99	0.00999932892804429\\
100	0.00999932891618094\\
101	0.00999932890410399\\
102	0.00999932889180961\\
103	0.0099993288792939\\
104	0.0099993288665529\\
105	0.00999932885358256\\
106	0.00999932884037878\\
107	0.00999932882693737\\
108	0.00999932881325408\\
109	0.00999932879932457\\
110	0.00999932878514444\\
111	0.00999932877070918\\
112	0.00999932875601424\\
113	0.00999932874105496\\
114	0.0099993287258266\\
115	0.00999932871032436\\
116	0.00999932869454332\\
117	0.0099993286784785\\
118	0.00999932866212481\\
119	0.00999932864547709\\
120	0.00999932862853007\\
121	0.0099993286112784\\
122	0.00999932859371662\\
123	0.00999932857583919\\
124	0.00999932855764046\\
125	0.00999932853911468\\
126	0.00999932852025601\\
127	0.00999932850105849\\
128	0.00999932848151605\\
129	0.00999932846162254\\
130	0.00999932844137168\\
131	0.00999932842075708\\
132	0.00999932839977223\\
133	0.00999932837841052\\
134	0.00999932835666521\\
135	0.00999932833452945\\
136	0.00999932831199627\\
137	0.00999932828905855\\
138	0.00999932826570908\\
139	0.0099993282419405\\
140	0.00999932821774533\\
141	0.00999932819311594\\
142	0.00999932816804458\\
143	0.00999932814252336\\
144	0.00999932811654425\\
145	0.00999932809009907\\
146	0.00999932806317951\\
147	0.00999932803577709\\
148	0.00999932800788321\\
149	0.00999932797948908\\
150	0.00999932795058578\\
151	0.00999932792116424\\
152	0.00999932789121519\\
153	0.00999932786072924\\
154	0.0099993278296968\\
155	0.00999932779810814\\
156	0.00999932776595332\\
157	0.00999932773322226\\
158	0.00999932769990468\\
159	0.00999932766599012\\
160	0.00999932763146793\\
161	0.0099993275963273\\
162	0.0099993275605572\\
163	0.0099993275241464\\
164	0.00999932748708349\\
165	0.00999932744935685\\
166	0.00999932741095465\\
167	0.00999932737186486\\
168	0.00999932733207523\\
169	0.00999932729157329\\
170	0.00999932725034635\\
171	0.0099993272083815\\
172	0.0099993271656656\\
173	0.00999932712218528\\
174	0.00999932707792692\\
175	0.00999932703287667\\
176	0.00999932698702043\\
177	0.00999932694034386\\
178	0.00999932689283234\\
179	0.00999932684447103\\
180	0.00999932679524479\\
181	0.00999932674513823\\
182	0.00999932669413568\\
183	0.00999932664222121\\
184	0.00999932658937857\\
185	0.00999932653559127\\
186	0.00999932648084248\\
187	0.00999932642511511\\
188	0.00999932636839175\\
189	0.00999932631065468\\
190	0.00999932625188587\\
191	0.00999932619206697\\
192	0.00999932613117929\\
193	0.00999932606920384\\
194	0.00999932600612126\\
195	0.00999932594191187\\
196	0.00999932587655564\\
197	0.00999932581003216\\
198	0.00999932574232068\\
199	0.00999932567340008\\
200	0.00999932560324887\\
201	0.00999932553184515\\
202	0.00999932545916667\\
203	0.00999932538519075\\
204	0.00999932530989434\\
205	0.00999932523325396\\
206	0.00999932515524571\\
207	0.00999932507584528\\
208	0.00999932499502793\\
209	0.00999932491276845\\
210	0.00999932482904122\\
211	0.00999932474382014\\
212	0.00999932465707866\\
213	0.00999932456878976\\
214	0.00999932447892592\\
215	0.00999932438745914\\
216	0.00999932429436094\\
217	0.00999932419960231\\
218	0.00999932410315373\\
219	0.00999932400498516\\
220	0.00999932390506602\\
221	0.00999932380336519\\
222	0.00999932369985099\\
223	0.00999932359449118\\
224	0.00999932348725295\\
225	0.00999932337810289\\
226	0.00999932326700701\\
227	0.00999932315393072\\
228	0.00999932303883878\\
229	0.00999932292169536\\
230	0.00999932280246397\\
231	0.00999932268110747\\
232	0.00999932255758806\\
233	0.00999932243186726\\
234	0.00999932230390592\\
235	0.00999932217366417\\
236	0.00999932204110144\\
237	0.00999932190617643\\
238	0.00999932176884709\\
239	0.00999932162907065\\
240	0.00999932148680354\\
241	0.00999932134200143\\
242	0.00999932119461919\\
243	0.00999932104461088\\
244	0.00999932089192975\\
245	0.0099993207365282\\
246	0.00999932057835777\\
247	0.00999932041736915\\
248	0.00999932025351213\\
249	0.00999932008673561\\
250	0.00999931991698756\\
251	0.00999931974421503\\
252	0.00999931956836411\\
253	0.00999931938937993\\
254	0.00999931920720662\\
255	0.00999931902178732\\
256	0.00999931883306414\\
257	0.00999931864097814\\
258	0.00999931844546934\\
259	0.00999931824647667\\
260	0.00999931804393796\\
261	0.00999931783778991\\
262	0.00999931762796809\\
263	0.00999931741440691\\
264	0.0099993171970396\\
265	0.00999931697579818\\
266	0.00999931675061343\\
267	0.0099993165214149\\
268	0.00999931628813086\\
269	0.00999931605068828\\
270	0.0099993158090128\\
271	0.00999931556302874\\
272	0.00999931531265904\\
273	0.00999931505782528\\
274	0.0099993147984476\\
275	0.00999931453444469\\
276	0.00999931426573378\\
277	0.00999931399223057\\
278	0.00999931371384927\\
279	0.00999931343050251\\
280	0.00999931314210137\\
281	0.00999931284855528\\
282	0.00999931254977206\\
283	0.00999931224565783\\
284	0.00999931193611706\\
285	0.00999931162105244\\
286	0.00999931130036492\\
287	0.00999931097395366\\
288	0.00999931064171599\\
289	0.00999931030354737\\
290	0.00999930995934138\\
291	0.00999930960898969\\
292	0.00999930925238198\\
293	0.00999930888940597\\
294	0.00999930851994732\\
295	0.00999930814388966\\
296	0.00999930776111449\\
297	0.0099993073715012\\
298	0.009999306974927\\
299	0.00999930657126689\\
300	0.00999930616039365\\
301	0.00999930574217775\\
302	0.00999930531648737\\
303	0.00999930488318832\\
304	0.00999930444214405\\
305	0.00999930399321555\\
306	0.0099993035362614\\
307	0.00999930307113768\\
308	0.00999930259769793\\
309	0.00999930211579314\\
310	0.00999930162527162\\
311	0.00999930112597899\\
312	0.00999930061775813\\
313	0.00999930010044936\\
314	0.00999929957389025\\
315	0.00999929903791562\\
316	0.0099992984923575\\
317	0.00999929793704509\\
318	0.00999929737180476\\
319	0.00999929679646003\\
320	0.00999929621083155\\
321	0.00999929561473706\\
322	0.00999929500799143\\
323	0.00999929439040663\\
324	0.00999929376179172\\
325	0.00999929312195288\\
326	0.00999929247069341\\
327	0.00999929180781369\\
328	0.00999929113311117\\
329	0.00999929044638015\\
330	0.00999928974741139\\
331	0.00999928903599092\\
332	0.00999928831189758\\
333	0.00999928757489793\\
334	0.0099992868247374\\
335	0.00999928606112985\\
336	0.00999928528375961\\
337	0.00999928449233488\\
338	0.00999928368671541\\
339	0.00999928286686936\\
340	0.00999928203259543\\
341	0.00999928118365192\\
342	0.00999928031977998\\
343	0.00999927944068345\\
344	0.00999927854599073\\
345	0.00999927763520075\\
346	0.00999927670767158\\
347	0.00999927576287146\\
348	0.00999927480114783\\
349	0.00999927382301776\\
350	0.00999927282832036\\
351	0.00999927181677662\\
352	0.00999927078810297\\
353	0.00999926974201125\\
354	0.00999926867820868\\
355	0.0099992675963982\\
356	0.00999926649627893\\
357	0.00999926537754606\\
358	0.0099992642398888\\
359	0.00999926308299162\\
360	0.0099992619065342\\
361	0.0099992607101914\\
362	0.00999925949363315\\
363	0.00999925825652447\\
364	0.00999925699852534\\
365	0.00999925571929069\\
366	0.0099992544184703\\
367	0.00999925309570874\\
368	0.00999925175064532\\
369	0.009999250382914\\
370	0.00999924899214341\\
371	0.00999924757795682\\
372	0.00999924613997222\\
373	0.00999924467780235\\
374	0.0099992431910543\\
375	0.00999924167932856\\
376	0.00999924014221759\\
377	0.00999923857930842\\
378	0.00999923699019723\\
379	0.00999923537449653\\
380	0.00999923373186045\\
381	0.0099992320620289\\
382	0.0099992303648864\\
383	0.00999922864048714\\
384	0.00999922688889\\
385	0.0099992251095598\\
386	0.00999922330070009\\
387	0.00999922146127299\\
388	0.00999921959069207\\
389	0.00999921768840462\\
390	0.0099992157538371\\
391	0.00999921378639325\\
392	0.00999921178545247\\
393	0.00999920975036931\\
394	0.00999920768047584\\
395	0.00999920557509121\\
396	0.00999920343354766\\
397	0.00999920125525213\\
398	0.00999919903981771\\
399	0.00999919678731162\\
400	0.00999919449862671\\
401	0.00999919217573563\\
402	0.0099991898208095\\
403	0.00999918743212194\\
404	0.00999918499899439\\
405	0.00999918251891641\\
406	0.00999917999097951\\
407	0.00999917741414851\\
408	0.00999917478725977\\
409	0.00999917210902027\\
410	0.00999916937824074\\
411	0.00999916659449198\\
412	0.00999916375757198\\
413	0.00999916086623067\\
414	0.00999915791917385\\
415	0.00999915491506151\\
416	0.00999915185250656\\
417	0.00999914873007448\\
418	0.00999914554628377\\
419	0.00999914229960513\\
420	0.00999913898845258\\
421	0.00999913561115313\\
422	0.00999913216589674\\
423	0.00999912865076033\\
424	0.0099991250638269\\
425	0.00999912140309305\\
426	0.00999911766646265\\
427	0.00999911385173922\\
428	0.00999910995661611\\
429	0.00999910597866151\\
430	0.0099991019152918\\
431	0.00999909776371751\\
432	0.00999909352082707\\
433	0.00999908918293894\\
434	0.00999908474531591\\
435	0.00999908020140347\\
436	0.00999907554235725\\
437	0.00999907075949376\\
438	0.00999906585399951\\
439	0.00999906083218615\\
440	0.0099990556912567\\
441	0.00999905042646166\\
442	0.00999904503269721\\
443	0.00999903950445559\\
444	0.00999903383576376\\
445	0.00999902802012308\\
446	0.00999902205051696\\
447	0.00999901591968957\\
448	0.00999900962088153\\
449	0.00999900314709705\\
450	0.00999899649175576\\
451	0.00999898964872395\\
452	0.00999898260774513\\
453	0.00999897533405877\\
454	0.00999896770985755\\
455	0.00999895944454393\\
456	0.00999894998388565\\
457	0.00999893838990642\\
458	0.00999892236539694\\
459	0.00999889969768127\\
460	0.00999887523716839\\
461	0.00999885026190805\\
462	0.00999882473405357\\
463	0.00999879872460709\\
464	0.00999877223404172\\
465	0.0099987452434011\\
466	0.00999871773053609\\
467	0.00999868966956539\\
468	0.00999866102980771\\
469	0.00999863177775147\\
470	0.00999860189300072\\
471	0.00999857136223286\\
472	0.00999854015882764\\
473	0.00999850826004149\\
474	0.00999847566151176\\
475	0.00999844238425681\\
476	0.00999840843607569\\
477	0.00999837362278948\\
478	0.00999833769663121\\
479	0.0099983005414136\\
480	0.00999826205301255\\
481	0.00999822207094425\\
482	0.00999818030273864\\
483	0.00999813612354068\\
484	0.00999808804373537\\
485	0.00999803258721717\\
486	0.00999795214372719\\
487	0.00999781736381448\\
488	0.00999767961429178\\
489	0.00999753875507356\\
490	0.00999739466506454\\
491	0.00999724721376824\\
492	0.00999709624852324\\
493	0.00999693121355686\\
494	0.00999670530755466\\
495	0.00999647416658922\\
496	0.00999623775903836\\
497	0.00999599571349533\\
498	0.00999574697161254\\
499	0.0099954910077634\\
500	0.00999522733925143\\
501	0.00999495537165017\\
502	0.00999467431269238\\
503	0.00999438302117948\\
504	0.00999407991692828\\
505	0.0099937641613075\\
506	0.00999343394490521\\
507	0.00999308334337721\\
508	0.00999269762074172\\
509	0.0099914554365218\\
510	0.00998940124696233\\
511	0.00998732863778075\\
512	0.0099852370976057\\
513	0.00998312610223551\\
514	0.00998099650252101\\
515	0.00997884616187541\\
516	0.00997667304539808\\
517	0.00997447597048032\\
518	0.00997225383250046\\
519	0.00997000550783338\\
520	0.0099677298408233\\
521	0.00996541777538252\\
522	0.00996307154385438\\
523	0.00996068939531412\\
524	0.00995826957185022\\
525	0.00995580426317674\\
526	0.00995326696843847\\
527	0.00994980420231118\\
528	0.00994613560965982\\
529	0.00994236367068748\\
530	0.0099384752067908\\
531	0.0099344521415761\\
532	0.00993027162010349\\
533	0.00992252256538219\\
534	0.00991175654377404\\
535	0.00990070756465675\\
536	0.00988934850347928\\
537	0.00987764589693117\\
538	0.00986556114281452\\
539	0.00985304879076982\\
540	0.00984004906369896\\
541	0.00982416100283728\\
542	0.00969282387754647\\
543	0.00955665231647043\\
544	0.00941520952431028\\
545	0.00926799773951631\\
546	0.00911444673581855\\
547	0.00895390159476861\\
548	0.00878560714903328\\
549	0.00860868904960451\\
550	0.00842213090002012\\
551	0.00822474711779364\\
552	0.00801515303577693\\
553	0.00779174238648091\\
554	0.00756053130714489\\
555	0.0073227869414988\\
556	0.00707816704069707\\
557	0.00682631110018469\\
558	0.00656685138566235\\
559	0.00629933175157336\\
560	0.00602325745472997\\
561	0.00573812592479245\\
562	0.00544347453026945\\
563	0.00513904679861026\\
564	0.00482435547151584\\
565	0.00449969466636478\\
566	0.00417568949117384\\
567	0.00384096269838464\\
568	0.00363912200043192\\
569	0.00346881456343741\\
570	0.00330378970436501\\
571	0.00314579069162\\
572	0.0029969195629914\\
573	0.00285788076580331\\
574	0.0027194058618991\\
575	0.00258161828951404\\
576	0.00244481403871134\\
577	0.00230928115503963\\
578	0.00217527788184951\\
579	0.00204300485446921\\
580	0.00191264896058479\\
581	0.00178427582037178\\
582	0.00165614535808679\\
583	0.00152811363328425\\
584	0.00140051135640073\\
585	0.00127372219294322\\
586	0.00114818338399011\\
587	0.00102441882820826\\
588	0.000903028512395\\
589	0.000784662617985752\\
590	0.000670016384725282\\
591	0.00055982374888513\\
592	0.000454847943832255\\
593	0.000355865647533659\\
594	0.000263643164224137\\
595	0.000178902643795078\\
596	0.000102272945261959\\
597	3.41569359503316e-05\\
598	2.9204464504877e-07\\
599	0\\
600	0\\
};
\addplot [color=mycolor16,solid,forget plot]
  table[row sep=crcr]{%
1	0.00996936184596852\\
2	0.009969361812907\\
3	0.00996936177924712\\
4	0.00996936174497806\\
5	0.00996936171008882\\
6	0.0099693616745682\\
7	0.00996936163840478\\
8	0.00996936160158697\\
9	0.00996936156410294\\
10	0.00996936152594065\\
11	0.00996936148708786\\
12	0.0099693614475321\\
13	0.00996936140726067\\
14	0.00996936136626065\\
15	0.00996936132451889\\
16	0.00996936128202199\\
17	0.00996936123875631\\
18	0.00996936119470799\\
19	0.00996936114986289\\
20	0.00996936110420663\\
21	0.00996936105772457\\
22	0.0099693610104018\\
23	0.00996936096222315\\
24	0.00996936091317318\\
25	0.00996936086323616\\
26	0.00996936081239608\\
27	0.00996936076063665\\
28	0.00996936070794129\\
29	0.00996936065429311\\
30	0.00996936059967491\\
31	0.00996936054406921\\
32	0.00996936048745819\\
33	0.00996936042982372\\
34	0.00996936037114734\\
35	0.00996936031141026\\
36	0.00996936025059335\\
37	0.00996936018867715\\
38	0.00996936012564183\\
39	0.00996936006146721\\
40	0.00996935999613276\\
41	0.00996935992961757\\
42	0.00996935986190036\\
43	0.00996935979295946\\
44	0.00996935972277281\\
45	0.00996935965131798\\
46	0.0099693595785721\\
47	0.00996935950451191\\
48	0.00996935942911374\\
49	0.00996935935235347\\
50	0.00996935927420657\\
51	0.00996935919464807\\
52	0.00996935911365253\\
53	0.00996935903119407\\
54	0.00996935894724636\\
55	0.00996935886178256\\
56	0.00996935877477539\\
57	0.00996935868619705\\
58	0.00996935859601926\\
59	0.00996935850421322\\
60	0.00996935841074963\\
61	0.00996935831559864\\
62	0.0099693582187299\\
63	0.00996935812011248\\
64	0.00996935801971492\\
65	0.00996935791750519\\
66	0.00996935781345069\\
67	0.00996935770751822\\
68	0.009969357599674\\
69	0.00996935748988366\\
70	0.00996935737811217\\
71	0.00996935726432392\\
72	0.00996935714848263\\
73	0.00996935703055139\\
74	0.00996935691049262\\
75	0.00996935678826808\\
76	0.00996935666383882\\
77	0.00996935653716523\\
78	0.00996935640820695\\
79	0.00996935627692293\\
80	0.00996935614327138\\
81	0.00996935600720974\\
82	0.00996935586869473\\
83	0.00996935572768225\\
84	0.00996935558412745\\
85	0.00996935543798466\\
86	0.00996935528920739\\
87	0.00996935513774834\\
88	0.00996935498355933\\
89	0.00996935482659136\\
90	0.00996935466679452\\
91	0.00996935450411804\\
92	0.0099693543385102\\
93	0.00996935416991841\\
94	0.00996935399828909\\
95	0.00996935382356775\\
96	0.00996935364569888\\
97	0.00996935346462602\\
98	0.00996935328029167\\
99	0.00996935309263733\\
100	0.00996935290160345\\
101	0.00996935270712939\\
102	0.00996935250915347\\
103	0.00996935230761288\\
104	0.0099693521024437\\
105	0.00996935189358088\\
106	0.00996935168095817\\
107	0.0099693514645082\\
108	0.00996935124416234\\
109	0.00996935101985078\\
110	0.00996935079150243\\
111	0.00996935055904496\\
112	0.00996935032240473\\
113	0.00996935008150682\\
114	0.00996934983627493\\
115	0.00996934958663143\\
116	0.0099693493324973\\
117	0.00996934907379211\\
118	0.009969348810434\\
119	0.00996934854233965\\
120	0.00996934826942425\\
121	0.00996934799160149\\
122	0.00996934770878352\\
123	0.00996934742088092\\
124	0.00996934712780268\\
125	0.00996934682945618\\
126	0.00996934652574712\\
127	0.00996934621657956\\
128	0.00996934590185584\\
129	0.00996934558147654\\
130	0.00996934525534049\\
131	0.00996934492334473\\
132	0.00996934458538443\\
133	0.00996934424135295\\
134	0.00996934389114169\\
135	0.00996934353464017\\
136	0.00996934317173592\\
137	0.00996934280231446\\
138	0.0099693424262593\\
139	0.00996934204345186\\
140	0.00996934165377145\\
141	0.00996934125709524\\
142	0.00996934085329822\\
143	0.00996934044225315\\
144	0.00996934002383052\\
145	0.00996933959789853\\
146	0.00996933916432303\\
147	0.00996933872296749\\
148	0.00996933827369294\\
149	0.00996933781635796\\
150	0.00996933735081861\\
151	0.00996933687692837\\
152	0.00996933639453814\\
153	0.00996933590349615\\
154	0.00996933540364796\\
155	0.00996933489483636\\
156	0.00996933437690135\\
157	0.0099693338496801\\
158	0.00996933331300687\\
159	0.00996933276671299\\
160	0.00996933221062677\\
161	0.00996933164457348\\
162	0.00996933106837531\\
163	0.00996933048185124\\
164	0.00996932988481708\\
165	0.00996932927708533\\
166	0.00996932865846519\\
167	0.00996932802876245\\
168	0.00996932738777944\\
169	0.009969326735315\\
170	0.00996932607116438\\
171	0.0099693253951192\\
172	0.00996932470696737\\
173	0.00996932400649304\\
174	0.0099693232934765\\
175	0.00996932256769416\\
176	0.00996932182891845\\
177	0.00996932107691774\\
178	0.0099693203114563\\
179	0.00996931953229421\\
180	0.00996931873918727\\
181	0.00996931793188694\\
182	0.00996931711014028\\
183	0.00996931627368983\\
184	0.00996931542227356\\
185	0.00996931455562478\\
186	0.00996931367347207\\
187	0.00996931277553915\\
188	0.00996931186154485\\
189	0.009969310931203\\
190	0.00996930998422232\\
191	0.00996930902030636\\
192	0.0099693080391534\\
193	0.00996930704045633\\
194	0.00996930602390258\\
195	0.00996930498917402\\
196	0.00996930393594686\\
197	0.00996930286389154\\
198	0.00996930177267263\\
199	0.0099693006619487\\
200	0.00996929953137229\\
201	0.00996929838058969\\
202	0.00996929720924093\\
203	0.0099692960169596\\
204	0.00996929480337277\\
205	0.00996929356810084\\
206	0.00996929231075747\\
207	0.00996929103094939\\
208	0.00996928972827635\\
209	0.00996928840233092\\
210	0.00996928705269843\\
211	0.00996928567895679\\
212	0.00996928428067636\\
213	0.00996928285741984\\
214	0.00996928140874213\\
215	0.00996927993419013\\
216	0.00996927843330267\\
217	0.00996927690561034\\
218	0.00996927535063532\\
219	0.00996927376789122\\
220	0.00996927215688299\\
221	0.00996927051710667\\
222	0.0099692688480493\\
223	0.00996926714918873\\
224	0.00996926541999343\\
225	0.00996926365992238\\
226	0.00996926186842481\\
227	0.0099692600449401\\
228	0.00996925818889757\\
229	0.00996925629971628\\
230	0.00996925437680486\\
231	0.00996925241956132\\
232	0.00996925042737284\\
233	0.00996924839961561\\
234	0.00996924633565455\\
235	0.00996924423484321\\
236	0.00996924209652345\\
237	0.00996923992002532\\
238	0.00996923770466676\\
239	0.00996923544975345\\
240	0.00996923315457854\\
241	0.00996923081842241\\
242	0.00996922844055249\\
243	0.00996922602022294\\
244	0.00996922355667447\\
245	0.00996922104913407\\
246	0.00996921849681475\\
247	0.00996921589891527\\
248	0.00996921325461991\\
249	0.00996921056309817\\
250	0.0099692078235045\\
251	0.00996920503497803\\
252	0.00996920219664228\\
253	0.00996919930760488\\
254	0.00996919636695724\\
255	0.00996919337377428\\
256	0.00996919032711411\\
257	0.00996918722601773\\
258	0.00996918406950867\\
259	0.00996918085659272\\
260	0.00996917758625754\\
261	0.00996917425747237\\
262	0.00996917086918766\\
263	0.0099691674203347\\
264	0.0099691639098253\\
265	0.00996916033655136\\
266	0.00996915669938455\\
267	0.00996915299717585\\
268	0.00996914922875514\\
269	0.00996914539293081\\
270	0.00996914148848932\\
271	0.00996913751419497\\
272	0.00996913346878982\\
273	0.00996912935099353\\
274	0.00996912515950223\\
275	0.00996912089298756\\
276	0.00996911655009667\\
277	0.00996911212945185\\
278	0.0099691076296501\\
279	0.00996910304926263\\
280	0.00996909838683436\\
281	0.00996909364088347\\
282	0.00996908880990086\\
283	0.00996908389234966\\
284	0.00996907888666469\\
285	0.00996907379125197\\
286	0.00996906860448814\\
287	0.00996906332471995\\
288	0.00996905795026368\\
289	0.00996905247940459\\
290	0.00996904691039637\\
291	0.00996904124146049\\
292	0.00996903547078571\\
293	0.00996902959652741\\
294	0.00996902361680701\\
295	0.00996901752971138\\
296	0.0099690113332922\\
297	0.00996900502556532\\
298	0.00996899860451019\\
299	0.00996899206806916\\
300	0.00996898541414689\\
301	0.0099689786406097\\
302	0.00996897174528493\\
303	0.00996896472596036\\
304	0.00996895758038362\\
305	0.00996895030626168\\
306	0.00996894290126038\\
307	0.00996893536300399\\
308	0.00996892768907455\\
309	0.00996891987701054\\
310	0.00996891192430447\\
311	0.0099689038284003\\
312	0.0099688955866946\\
313	0.00996888719654264\\
314	0.00996887865525406\\
315	0.00996886996009152\\
316	0.00996886110827023\\
317	0.00996885209695749\\
318	0.00996884292327232\\
319	0.00996883358428505\\
320	0.00996882407701709\\
321	0.00996881439844069\\
322	0.00996880454547877\\
323	0.00996879451500491\\
324	0.0099687843038434\\
325	0.00996877390876939\\
326	0.00996876332650918\\
327	0.00996875255374065\\
328	0.00996874158709383\\
329	0.00996873042315152\\
330	0.00996871905844984\\
331	0.00996870748947838\\
332	0.00996869571267886\\
333	0.00996868372444004\\
334	0.00996867152108336\\
335	0.00996865909882866\\
336	0.00996864645372656\\
337	0.00996863358157793\\
338	0.00996862047806023\\
339	0.0099686071397144\\
340	0.00996859356415082\\
341	0.00996857974775333\\
342	0.00996856568664523\\
343	0.00996855137690936\\
344	0.0099685368145441\\
345	0.00996852199534319\\
346	0.00996850691460096\\
347	0.00996849156656939\\
348	0.00996847594453828\\
349	0.00996846004729916\\
350	0.00996844387572912\\
351	0.00996842742591287\\
352	0.00996841069300453\\
353	0.00996839367206885\\
354	0.00996837635808168\\
355	0.00996835874594157\\
356	0.00996834083049468\\
357	0.00996832260653619\\
358	0.00996830406872731\\
359	0.00996828521163068\\
360	0.00996826602971775\\
361	0.00996824651736727\\
362	0.00996822666886374\\
363	0.00996820647839577\\
364	0.0099681859400543\\
365	0.00996816504783079\\
366	0.00996814379561519\\
367	0.00996812217719412\\
368	0.00996810018624911\\
369	0.00996807781635557\\
370	0.00996805506098307\\
371	0.00996803191349778\\
372	0.00996800836716694\\
373	0.00996798441516115\\
374	0.00996796005054027\\
375	0.00996793526619519\\
376	0.00996791005473633\\
377	0.00996788440846369\\
378	0.00996785831972465\\
379	0.00996783178076111\\
380	0.00996780478369372\\
381	0.00996777732058907\\
382	0.0099677493836164\\
383	0.00996772096535449\\
384	0.00996769205922427\\
385	0.00996766265946431\\
386	0.00996763275824557\\
387	0.0099676023368896\\
388	0.00996757138155168\\
389	0.00996753988143256\\
390	0.00996750782569727\\
391	0.00996747520304785\\
392	0.00996744200167978\\
393	0.00996740820923558\\
394	0.00996737381275691\\
395	0.00996733879863829\\
396	0.00996730315259064\\
397	0.00996726685963096\\
398	0.00996722990415092\\
399	0.00996719227016634\\
400	0.00996715394200545\\
401	0.00996711490592282\\
402	0.00996707515313806\\
403	0.00996703468212015\\
404	0.00996699348243015\\
405	0.00996695146777249\\
406	0.00996690860947114\\
407	0.00996686488848824\\
408	0.0099668202829035\\
409	0.00996677476655935\\
410	0.00996672830918364\\
411	0.00996668088824669\\
412	0.00996663252183162\\
413	0.00996658319375107\\
414	0.0099665328798882\\
415	0.00996648155525311\\
416	0.00996642919398741\\
417	0.00996637576941329\\
418	0.00996632125413397\\
419	0.00996626562011802\\
420	0.00996620883848122\\
421	0.00996615087834981\\
422	0.00996609170467166\\
423	0.00996603127879021\\
424	0.00996596956358955\\
425	0.00996590652082387\\
426	0.00996584211038325\\
427	0.00996577629015359\\
428	0.00996570901586141\\
429	0.00996564024089977\\
430	0.00996556991612889\\
431	0.00996549798963701\\
432	0.00996542440642666\\
433	0.00996534910793691\\
434	0.00996527203116697\\
435	0.00996519310679865\\
436	0.00996511225496241\\
437	0.00996502937706096\\
438	0.00996494435245766\\
439	0.0099648571154515\\
440	0.00996476764919769\\
441	0.0099646758685781\\
442	0.00996458166758121\\
443	0.00996448493022034\\
444	0.00996438552820102\\
445	0.00996428331792217\\
446	0.00996417813876847\\
447	0.00996406982018708\\
448	0.00996395820526764\\
449	0.00996384310879453\\
450	0.00996372431229352\\
451	0.00996360156980491\\
452	0.00996347459852649\\
453	0.00996334303737268\\
454	0.00996320627394929\\
455	0.00996306276019777\\
456	0.0099629082738575\\
457	0.00996272160641869\\
458	0.00996183313692551\\
459	0.00996070904071886\\
460	0.00995951123324781\\
461	0.00995829345035272\\
462	0.00995705525002235\\
463	0.00995579604567714\\
464	0.00995451605156727\\
465	0.00995321492125422\\
466	0.00995189217067572\\
467	0.00995054729742794\\
468	0.00994917977205121\\
469	0.00994778901926564\\
470	0.0099463744863286\\
471	0.00994493653174981\\
472	0.00994347522703532\\
473	0.00994198965175815\\
474	0.00994047858563699\\
475	0.00993894072108857\\
476	0.00993737472282705\\
477	0.00993577929934507\\
478	0.00993415176582555\\
479	0.00993248904722541\\
480	0.00993078956095074\\
481	0.00992905159421222\\
482	0.00992727297102582\\
483	0.00992545098097248\\
484	0.00992358152455413\\
485	0.00992165486828391\\
486	0.00991903208414031\\
487	0.00991417478746761\\
488	0.00990921257372146\\
489	0.00990413980369272\\
490	0.00989895150439706\\
491	0.00989364231171935\\
492	0.00988820558811218\\
493	0.0098826437066687\\
494	0.00987700135821326\\
495	0.00987121387196477\\
496	0.00986527323551472\\
497	0.00985917101724499\\
498	0.00985289670902821\\
499	0.00984643355755357\\
500	0.00983976795828853\\
501	0.00983288524363733\\
502	0.00982576869275228\\
503	0.00981839896031053\\
504	0.00981075281302382\\
505	0.00980279932324318\\
506	0.00979451444131653\\
507	0.00978591827079131\\
508	0.00977698445580002\\
509	0.00973195803472746\\
510	0.00965105588747158\\
511	0.00956811607322863\\
512	0.0094830290924446\\
513	0.00939567396480019\\
514	0.00930591449011373\\
515	0.00921360188318442\\
516	0.00911855396417855\\
517	0.00902057789311013\\
518	0.00891946808231931\\
519	0.00881499712801225\\
520	0.00870691226604886\\
521	0.00859494030878527\\
522	0.00847876237623706\\
523	0.0083580246132408\\
524	0.00823233644397344\\
525	0.0081012817910472\\
526	0.00796438806893423\\
527	0.0078219666668102\\
528	0.00767259631689416\\
529	0.00751537094821622\\
530	0.00734937101066814\\
531	0.00717352887574939\\
532	0.00698838745773369\\
533	0.00680199976078023\\
534	0.00661369802893454\\
535	0.00642052120659462\\
536	0.00622223797007795\\
537	0.00601859692696998\\
538	0.00580931975662197\\
539	0.00559409486758147\\
540	0.00537257221695576\\
541	0.0051467254174597\\
542	0.00503134269814583\\
543	0.00491423520334139\\
544	0.00479541795300435\\
545	0.00467512378405048\\
546	0.00455364026061907\\
547	0.00443132099421412\\
548	0.00430862748349586\\
549	0.00418616509956596\\
550	0.00406472203417226\\
551	0.00394531883637027\\
552	0.00382927076407412\\
553	0.00371826159400786\\
554	0.00360615050474313\\
555	0.00349161200839405\\
556	0.00337507055368945\\
557	0.00325756846643711\\
558	0.00314106624760566\\
559	0.00302590863410745\\
560	0.00291248032569543\\
561	0.00280120628054289\\
562	0.00269255205987361\\
563	0.00258702102201481\\
564	0.00248514913993936\\
565	0.00238748317032211\\
566	0.00229440259723362\\
567	0.00220660880533274\\
568	0.0021222859857676\\
569	0.0020401299752841\\
570	0.00195865682970082\\
571	0.00187781943912992\\
572	0.00179745585614607\\
573	0.00171731732385599\\
574	0.00163747296345394\\
575	0.00155798189225075\\
576	0.00147888770982828\\
577	0.00140021641993085\\
578	0.00132197584547392\\
579	0.00124415743122899\\
580	0.00116673974565477\\
581	0.00108970039256183\\
582	0.00101315452783778\\
583	0.0009372510557956\\
584	0.000862139893158467\\
585	0.000787968575672158\\
586	0.000714878431535376\\
587	0.000642998962491715\\
588	0.000572441638192295\\
589	0.000503293309197684\\
590	0.000435609150453159\\
591	0.000369387784493793\\
592	0.00030456560169281\\
593	0.000241014280104653\\
594	0.000178532861776521\\
595	0.000116902108299214\\
596	6.17886640995154e-05\\
597	2.28062284332058e-05\\
598	2.9204464504877e-07\\
599	0\\
600	0\\
};
\addplot [color=mycolor17,solid,forget plot]
  table[row sep=crcr]{%
1	0.00993093460416607\\
2	0.00993093371141091\\
3	0.00993093280249765\\
4	0.00993093187713434\\
5	0.0099309309350238\\
6	0.00993092997586345\\
7	0.0099309289993453\\
8	0.0099309280051558\\
9	0.00993092699297573\\
10	0.00993092596248016\\
11	0.00993092491333828\\
12	0.00993092384521334\\
13	0.00993092275776252\\
14	0.00993092165063682\\
15	0.00993092052348097\\
16	0.0099309193759333\\
17	0.00993091820762563\\
18	0.00993091701818316\\
19	0.00993091580722433\\
20	0.00993091457436074\\
21	0.00993091331919698\\
22	0.00993091204133054\\
23	0.00993091074035165\\
24	0.00993090941584318\\
25	0.00993090806738052\\
26	0.00993090669453139\\
27	0.00993090529685576\\
28	0.00993090387390566\\
29	0.00993090242522511\\
30	0.00993090095034989\\
31	0.00993089944880746\\
32	0.00993089792011679\\
33	0.00993089636378817\\
34	0.00993089477932314\\
35	0.00993089316621425\\
36	0.00993089152394494\\
37	0.00993088985198937\\
38	0.00993088814981226\\
39	0.00993088641686871\\
40	0.00993088465260403\\
41	0.00993088285645359\\
42	0.00993088102784259\\
43	0.00993087916618593\\
44	0.00993087727088801\\
45	0.00993087534134252\\
46	0.00993087337693229\\
47	0.00993087137702905\\
48	0.00993086934099328\\
49	0.00993086726817395\\
50	0.0099308651579084\\
51	0.00993086300952203\\
52	0.00993086082232817\\
53	0.00993085859562783\\
54	0.00993085632870949\\
55	0.00993085402084884\\
56	0.00993085167130862\\
57	0.00993084927933834\\
58	0.00993084684417404\\
59	0.00993084436503808\\
60	0.00993084184113889\\
61	0.0099308392716707\\
62	0.0099308366558133\\
63	0.0099308339927318\\
64	0.00993083128157635\\
65	0.00993082852148184\\
66	0.00993082571156773\\
67	0.00993082285093764\\
68	0.00993081993867919\\
69	0.00993081697386364\\
70	0.00993081395554562\\
71	0.00993081088276285\\
72	0.0099308077545358\\
73	0.00993080456986743\\
74	0.00993080132774285\\
75	0.00993079802712901\\
76	0.00993079466697437\\
77	0.00993079124620859\\
78	0.00993078776374216\\
79	0.0099307842184661\\
80	0.00993078060925159\\
81	0.00993077693494961\\
82	0.00993077319439063\\
83	0.00993076938638418\\
84	0.0099307655097185\\
85	0.00993076156316018\\
86	0.00993075754545378\\
87	0.0099307534553214\\
88	0.00993074929146231\\
89	0.00993074505255253\\
90	0.00993074073724446\\
91	0.00993073634416636\\
92	0.00993073187192206\\
93	0.0099307273190904\\
94	0.00993072268422486\\
95	0.00993071796585309\\
96	0.00993071316247644\\
97	0.00993070827256952\\
98	0.00993070329457969\\
99	0.00993069822692661\\
100	0.00993069306800171\\
101	0.00993068781616775\\
102	0.00993068246975825\\
103	0.00993067702707699\\
104	0.0099306714863975\\
105	0.0099306658459625\\
106	0.00993066010398338\\
107	0.00993065425863961\\
108	0.00993064830807821\\
109	0.00993064225041314\\
110	0.00993063608372475\\
111	0.00993062980605913\\
112	0.00993062341542758\\
113	0.00993061690980591\\
114	0.00993061028713389\\
115	0.00993060354531454\\
116	0.00993059668221352\\
117	0.00993058969565846\\
118	0.00993058258343829\\
119	0.00993057534330254\\
120	0.00993056797296064\\
121	0.00993056047008123\\
122	0.00993055283229143\\
123	0.00993054505717609\\
124	0.00993053714227705\\
125	0.0099305290850924\\
126	0.00993052088307564\\
127	0.00993051253363497\\
128	0.00993050403413244\\
129	0.00993049538188314\\
130	0.00993048657415436\\
131	0.00993047760816477\\
132	0.00993046848108353\\
133	0.00993045919002943\\
134	0.00993044973206998\\
135	0.00993044010422053\\
136	0.00993043030344332\\
137	0.00993042032664654\\
138	0.00993041017068338\\
139	0.00993039983235107\\
140	0.00993038930838985\\
141	0.00993037859548201\\
142	0.00993036769025081\\
143	0.00993035658925947\\
144	0.00993034528901009\\
145	0.00993033378594258\\
146	0.00993032207643354\\
147	0.00993031015679516\\
148	0.00993029802327405\\
149	0.0099302856720501\\
150	0.00993027309923528\\
151	0.00993026030087247\\
152	0.00993024727293417\\
153	0.00993023401132132\\
154	0.00993022051186199\\
155	0.00993020677031011\\
156	0.00993019278234413\\
157	0.0099301785435657\\
158	0.0099301640494983\\
159	0.00993014929558586\\
160	0.00993013427719133\\
161	0.00993011898959527\\
162	0.00993010342799436\\
163	0.00993008758749992\\
164	0.0099300714631364\\
165	0.00993005504983984\\
166	0.00993003834245628\\
167	0.00993002133574018\\
168	0.00993000402435278\\
169	0.00992998640286044\\
170	0.00992996846573298\\
171	0.00992995020734192\\
172	0.00992993162195879\\
173	0.00992991270375326\\
174	0.00992989344679144\\
175	0.00992987384503395\\
176	0.00992985389233406\\
177	0.00992983358243582\\
178	0.00992981290897206\\
179	0.00992979186546243\\
180	0.00992977044531139\\
181	0.00992974864180616\\
182	0.0099297264481146\\
183	0.0099297038572831\\
184	0.00992968086223442\\
185	0.00992965745576546\\
186	0.00992963363054504\\
187	0.00992960937911158\\
188	0.0099295846938708\\
189	0.00992955956709332\\
190	0.00992953399091226\\
191	0.00992950795732077\\
192	0.00992948145816955\\
193	0.00992945448516424\\
194	0.00992942702986291\\
195	0.00992939908367331\\
196	0.00992937063785027\\
197	0.00992934168349288\\
198	0.00992931221154176\\
199	0.00992928221277616\\
200	0.00992925167781109\\
201	0.00992922059709437\\
202	0.00992918896090359\\
203	0.00992915675934309\\
204	0.00992912398234081\\
205	0.00992909061964512\\
206	0.00992905666082161\\
207	0.00992902209524973\\
208	0.00992898691211951\\
209	0.00992895110042806\\
210	0.00992891464897615\\
211	0.00992887754636463\\
212	0.00992883978099082\\
213	0.00992880134104482\\
214	0.00992876221450577\\
215	0.00992872238913799\\
216	0.00992868185248715\\
217	0.00992864059187624\\
218	0.00992859859440157\\
219	0.00992855584692861\\
220	0.00992851233608782\\
221	0.00992846804827039\\
222	0.00992842296962383\\
223	0.00992837708604758\\
224	0.00992833038318847\\
225	0.0099282828464361\\
226	0.00992823446091814\\
227	0.00992818521149559\\
228	0.00992813508275784\\
229	0.00992808405901774\\
230	0.00992803212430653\\
231	0.00992797926236865\\
232	0.00992792545665653\\
233	0.00992787069032519\\
234	0.0099278149462268\\
235	0.0099277582069051\\
236	0.00992770045458973\\
237	0.00992764167119046\\
238	0.00992758183829127\\
239	0.00992752093714438\\
240	0.0099274589486641\\
241	0.00992739585342058\\
242	0.00992733163163352\\
243	0.00992726626316561\\
244	0.00992719972751594\\
245	0.00992713200381332\\
246	0.00992706307080935\\
247	0.00992699290687146\\
248	0.00992692148997578\\
249	0.00992684879769986\\
250	0.00992677480721526\\
251	0.00992669949528001\\
252	0.0099266228382309\\
253	0.00992654481197564\\
254	0.00992646539198487\\
255	0.00992638455328397\\
256	0.00992630227044479\\
257	0.00992621851757716\\
258	0.00992613326832022\\
259	0.00992604649583368\\
260	0.00992595817278878\\
261	0.00992586827135918\\
262	0.0099257767632116\\
263	0.00992568361949633\\
264	0.00992558881083752\\
265	0.0099254923073233\\
266	0.0099253940784957\\
267	0.0099252940933403\\
268	0.00992519232027576\\
269	0.00992508872714296\\
270	0.00992498328119394\\
271	0.00992487594908067\\
272	0.00992476669684414\\
273	0.00992465548990483\\
274	0.00992454229305393\\
275	0.00992442707043759\\
276	0.00992430978553932\\
277	0.00992419040117137\\
278	0.0099240688794629\\
279	0.00992394518184714\\
280	0.00992381926904825\\
281	0.00992369110106802\\
282	0.00992356063717229\\
283	0.00992342783587714\\
284	0.00992329265493487\\
285	0.00992315505131967\\
286	0.00992301498121313\\
287	0.00992287239998948\\
288	0.00992272726220057\\
289	0.00992257952156063\\
290	0.00992242913093084\\
291	0.00992227604230362\\
292	0.00992212020678668\\
293	0.00992196157458697\\
294	0.00992180009499428\\
295	0.00992163571636469\\
296	0.00992146838610395\\
297	0.00992129805065047\\
298	0.00992112465545834\\
299	0.00992094814498012\\
300	0.00992076846264954\\
301	0.00992058555086411\\
302	0.00992039935096766\\
303	0.00992020980323288\\
304	0.00992001684684393\\
305	0.00991982041987912\\
306	0.00991962045929402\\
307	0.00991941690090489\\
308	0.00991920967937278\\
309	0.00991899872818766\\
310	0.00991878397965054\\
311	0.00991856536484889\\
312	0.00991834281362408\\
313	0.0099181162545567\\
314	0.00991788561501206\\
315	0.00991765082109772\\
316	0.00991741179764503\\
317	0.0099171684681984\\
318	0.00991692075500623\\
319	0.00991666857901346\\
320	0.00991641185985619\\
321	0.0099161505158586\\
322	0.00991588446403243\\
323	0.00991561362007958\\
324	0.00991533789839802\\
325	0.00991505721209155\\
326	0.009914771472984\\
327	0.0099144805916381\\
328	0.00991418447737987\\
329	0.00991388303832891\\
330	0.00991357618143524\\
331	0.00991326381252325\\
332	0.00991294583634334\\
333	0.00991262215663096\\
334	0.00991229267617106\\
335	0.00991195729685783\\
336	0.00991161591971723\\
337	0.00991126844480942\\
338	0.00991091477091414\\
339	0.00991055479563033\\
340	0.00991018842127026\\
341	0.00990981555678587\\
342	0.00990943610069714\\
343	0.00990904994924596\\
344	0.00990865699838769\\
345	0.00990825714382186\\
346	0.00990785028089197\\
347	0.00990743630395627\\
348	0.00990701510394328\\
349	0.00990658656299084\\
350	0.00990615059228333\\
351	0.00990570710632888\\
352	0.00990525597935747\\
353	0.00990479707769154\\
354	0.00990433026527094\\
355	0.00990385540359984\\
356	0.00990337235174144\\
357	0.00990288096645583\\
358	0.0099023811023399\\
359	0.00990187261105522\\
360	0.00990135534156717\\
361	0.00990082914015778\\
362	0.00990029385036582\\
363	0.00989974931292212\\
364	0.00989919536567946\\
365	0.00989863184353588\\
366	0.00989805857835061\\
367	0.00989747539885124\\
368	0.00989688213053113\\
369	0.00989627859553574\\
370	0.00989566461253761\\
371	0.0098950399966002\\
372	0.00989440455903375\\
373	0.00989375810724723\\
374	0.00989310044459479\\
375	0.0098924313701728\\
376	0.00989175067839197\\
377	0.00989105815796495\\
378	0.00989035359062928\\
379	0.00988963675395991\\
380	0.00988890741967059\\
381	0.00988816535283248\\
382	0.00988741031139915\\
383	0.00988664204585051\\
384	0.00988586029923622\\
385	0.00988506480802466\\
386	0.00988425530285592\\
387	0.00988343149816776\\
388	0.00988259302418796\\
389	0.00988173954720821\\
390	0.00988087074932679\\
391	0.00987998630199731\\
392	0.00987908586177459\\
393	0.00987816906908797\\
394	0.0098772355469642\\
395	0.0098762848997191\\
396	0.00987531671164377\\
397	0.00987433054571894\\
398	0.00987332594240293\\
399	0.00987230241856791\\
400	0.0098712594667651\\
401	0.00987019655538514\\
402	0.0098691131315321\\
403	0.00986800863167006\\
404	0.0098668825074114\\
405	0.00986573420438174\\
406	0.00986456268705081\\
407	0.00986336742889049\\
408	0.00986214796778733\\
409	0.00986090383596427\\
410	0.00985963455651705\\
411	0.00985833962749916\\
412	0.00985701855127223\\
413	0.00985567125008841\\
414	0.00985429741320922\\
415	0.00985289638097873\\
416	0.0098514674683948\\
417	0.00985000996371651\\
418	0.00984852312713555\\
419	0.00984700618967824\\
420	0.00984545835245872\\
421	0.00984387878557286\\
422	0.00984226662243417\\
423	0.00984062093994073\\
424	0.00983894075034871\\
425	0.00983722504644191\\
426	0.00983547277651676\\
427	0.00983368283629492\\
428	0.00983185406482189\\
429	0.0098299852399367\\
430	0.00982807507325984\\
431	0.00982612220464185\\
432	0.00982412519601059\\
433	0.00982208252454507\\
434	0.00981999257505293\\
435	0.00981785363117089\\
436	0.00981566386388784\\
437	0.00981342131164205\\
438	0.00981112383340522\\
439	0.00980876901406392\\
440	0.00980635468891059\\
441	0.00980387887236951\\
442	0.00980133881930461\\
443	0.00979873145935177\\
444	0.00979605343638203\\
445	0.0097933010641487\\
446	0.00979047026815278\\
447	0.00978755650962274\\
448	0.00978455472915414\\
449	0.00978145956802953\\
450	0.00977826493401206\\
451	0.00977496387266974\\
452	0.0097715485274978\\
453	0.00976801002983891\\
454	0.00976433841792072\\
455	0.00976052260274695\\
456	0.00975654908188087\\
457	0.00975187337510829\\
458	0.00971722508156654\\
459	0.00968202931715126\\
460	0.00964609828519901\\
461	0.00960935424366499\\
462	0.00957177471230639\\
463	0.00953333699360045\\
464	0.00949401797492805\\
465	0.00945379720402719\\
466	0.00941264977884392\\
467	0.0093705515138918\\
468	0.00932747989843\\
469	0.00928341523661596\\
470	0.00923834220907166\\
471	0.00919225152495966\\
472	0.00914514529375721\\
473	0.00909702209437552\\
474	0.00904784884878742\\
475	0.00899758273021712\\
476	0.00894617829931817\\
477	0.00889358731911783\\
478	0.0088397592487368\\
479	0.00878463786755986\\
480	0.00872817097275927\\
481	0.00867031628596089\\
482	0.0086110286693143\\
483	0.00855026144771242\\
484	0.00848797038646977\\
485	0.0084241215869838\\
486	0.00835933739897901\\
487	0.00829516852057658\\
488	0.00822936249567639\\
489	0.0081618265298163\\
490	0.00809245455317616\\
491	0.00802114020293189\\
492	0.00794776691462567\\
493	0.00787220486253284\\
494	0.00779430110599939\\
495	0.00771388558172433\\
496	0.00763076347031366\\
497	0.00754470957221899\\
498	0.00745546162674872\\
499	0.00736270831820382\\
500	0.00726607771570519\\
501	0.00716519142095032\\
502	0.00705961055299592\\
503	0.00694882431976277\\
504	0.00683223712188952\\
505	0.00670915556106427\\
506	0.00657876271387543\\
507	0.0064443686339057\\
508	0.00630665936193442\\
509	0.0062021269716531\\
510	0.00613081217928649\\
511	0.00605834931285408\\
512	0.0059847585457202\\
513	0.00591006788827003\\
514	0.00583431268416269\\
515	0.00575753028797165\\
516	0.00567973480983292\\
517	0.00560081025772098\\
518	0.00552075419464518\\
519	0.0054396348244347\\
520	0.00535754050306231\\
521	0.00527458362688283\\
522	0.00519090733902166\\
523	0.00510669232669511\\
524	0.00502216522608326\\
525	0.0049376084664734\\
526	0.00485337485200936\\
527	0.00476988823244741\\
528	0.0046876916705995\\
529	0.00460746011895404\\
530	0.00453002778502993\\
531	0.00445642443236886\\
532	0.00438606543318127\\
533	0.00431411991300258\\
534	0.0042405441121292\\
535	0.00416536101625191\\
536	0.00408859738768097\\
537	0.00401031416093105\\
538	0.00393061848213843\\
539	0.00384967953737994\\
540	0.00376774944707301\\
541	0.00368515450614076\\
542	0.00360059345075749\\
543	0.00351426419829046\\
544	0.00342647862209044\\
545	0.00333769703700908\\
546	0.00324979024912032\\
547	0.00316349059199877\\
548	0.00307903376118264\\
549	0.00299664663417148\\
550	0.00291653241851946\\
551	0.00283884930907833\\
552	0.00276368032716527\\
553	0.0026909913567225\\
554	0.00262089551155703\\
555	0.00255348050617069\\
556	0.00248871242420521\\
557	0.00242593103386591\\
558	0.00236349281322508\\
559	0.00230141454744795\\
560	0.00223970032677368\\
561	0.00217835547665538\\
562	0.00211736633243854\\
563	0.00205669315196962\\
564	0.00199626632061458\\
565	0.00193598367096274\\
566	0.00187571269089403\\
567	0.0018152830983602\\
568	0.00175454130039174\\
569	0.00169338982056584\\
570	0.00163182870759884\\
571	0.0015698585537607\\
572	0.00150748492972648\\
573	0.0014447228627418\\
574	0.00138158723222657\\
575	0.00131809357034907\\
576	0.00125425925685432\\
577	0.00119010506498894\\
578	0.00112565699842713\\
579	0.00106094824693509\\
580	0.000996021033245017\\
581	0.000930927885436012\\
582	0.000865723636385916\\
583	0.000800463003868688\\
584	0.000735200717684098\\
585	0.000669991619104265\\
586	0.000604865722874402\\
587	0.000539844839955322\\
588	0.000474935038415231\\
589	0.000410184885592955\\
590	0.000347278553854741\\
591	0.000287502621334158\\
592	0.000231419174092323\\
593	0.000179467143924426\\
594	0.000132079920923583\\
595	9.10093979249611e-05\\
596	5.42660945238511e-05\\
597	2.28062284332058e-05\\
598	2.9204464504877e-07\\
599	0\\
600	0\\
};
\addplot [color=mycolor18,solid,forget plot]
  table[row sep=crcr]{%
1	0.00935388246113559\\
2	0.00935387188018864\\
3	0.00935386110766325\\
4	0.00935385014009582\\
5	0.0093538389739603\\
6	0.00935382760566703\\
7	0.00935381603156166\\
8	0.00935380424792389\\
9	0.00935379225096639\\
10	0.00935378003683355\\
11	0.0093537676016002\\
12	0.00935375494127047\\
13	0.00935374205177643\\
14	0.00935372892897686\\
15	0.00935371556865589\\
16	0.00935370196652169\\
17	0.00935368811820508\\
18	0.00935367401925817\\
19	0.00935365966515292\\
20	0.00935364505127973\\
21	0.00935363017294598\\
22	0.00935361502537449\\
23	0.00935359960370209\\
24	0.00935358390297799\\
25	0.00935356791816226\\
26	0.00935355164412422\\
27	0.00935353507564083\\
28	0.009353518207395\\
29	0.00935350103397394\\
30	0.00935348354986742\\
31	0.00935346574946604\\
32	0.00935344762705944\\
33	0.00935342917683452\\
34	0.00935341039287356\\
35	0.00935339126915239\\
36	0.00935337179953844\\
37	0.00935335197778883\\
38	0.00935333179754839\\
39	0.00935331125234764\\
40	0.00935329033560077\\
41	0.00935326904060352\\
42	0.00935324736053109\\
43	0.00935322528843596\\
44	0.00935320281724575\\
45	0.00935317993976088\\
46	0.00935315664865243\\
47	0.00935313293645972\\
48	0.00935310879558799\\
49	0.00935308421830602\\
50	0.00935305919674368\\
51	0.00935303372288943\\
52	0.00935300778858781\\
53	0.00935298138553686\\
54	0.00935295450528552\\
55	0.00935292713923094\\
56	0.00935289927861579\\
57	0.00935287091452547\\
58	0.00935284203788535\\
59	0.00935281263945787\\
60	0.00935278270983964\\
61	0.00935275223945848\\
62	0.00935272121857043\\
63	0.00935268963725665\\
64	0.00935265748542032\\
65	0.00935262475278343\\
66	0.00935259142888359\\
67	0.00935255750307072\\
68	0.00935252296450371\\
69	0.00935248780214698\\
70	0.00935245200476705\\
71	0.00935241556092898\\
72	0.00935237845899281\\
73	0.00935234068710986\\
74	0.00935230223321906\\
75	0.0093522630850431\\
76	0.00935222323008464\\
77	0.00935218265562232\\
78	0.00935214134870684\\
79	0.00935209929615682\\
80	0.00935205648455474\\
81	0.00935201290024265\\
82	0.00935196852931797\\
83	0.00935192335762906\\
84	0.00935187737077081\\
85	0.00935183055408016\\
86	0.00935178289263141\\
87	0.00935173437123165\\
88	0.00935168497441593\\
89	0.00935163468644244\\
90	0.00935158349128757\\
91	0.00935153137264089\\
92	0.00935147831390006\\
93	0.00935142429816562\\
94	0.00935136930823569\\
95	0.00935131332660059\\
96	0.00935125633543739\\
97	0.00935119831660427\\
98	0.00935113925163492\\
99	0.0093510791217327\\
100	0.00935101790776482\\
101	0.00935095559025627\\
102	0.00935089214938385\\
103	0.00935082756496984\\
104	0.0093507618164758\\
105	0.00935069488299613\\
106	0.00935062674325145\\
107	0.00935055737558208\\
108	0.00935048675794118\\
109	0.0093504148678879\\
110	0.00935034168258037\\
111	0.00935026717876854\\
112	0.00935019133278695\\
113	0.00935011412054733\\
114	0.00935003551753107\\
115	0.0093499554987816\\
116	0.00934987403889654\\
117	0.00934979111201982\\
118	0.00934970669183359\\
119	0.00934962075155001\\
120	0.0093495332639029\\
121	0.00934944420113922\\
122	0.00934935353501042\\
123	0.00934926123676361\\
124	0.00934916727713261\\
125	0.00934907162632883\\
126	0.00934897425403195\\
127	0.00934887512938047\\
128	0.00934877422096211\\
129	0.00934867149680399\\
130	0.00934856692436265\\
131	0.00934846047051393\\
132	0.00934835210154262\\
133	0.00934824178313197\\
134	0.00934812948035296\\
135	0.00934801515765343\\
136	0.00934789877884699\\
137	0.00934778030710172\\
138	0.00934765970492871\\
139	0.00934753693417037\\
140	0.00934741195598852\\
141	0.00934728473085229\\
142	0.00934715521852576\\
143	0.00934702337805546\\
144	0.00934688916775761\\
145	0.00934675254520502\\
146	0.00934661346721399\\
147	0.00934647188983075\\
148	0.00934632776831779\\
149	0.00934618105713989\\
150	0.00934603170994993\\
151	0.00934587967957439\\
152	0.00934572491799871\\
153	0.00934556737635221\\
154	0.00934540700489291\\
155	0.00934524375299197\\
156	0.0093450775691179\\
157	0.00934490840082048\\
158	0.00934473619471431\\
159	0.00934456089646227\\
160	0.00934438245075841\\
161	0.00934420080131076\\
162	0.00934401589082371\\
163	0.00934382766098008\\
164	0.00934363605242294\\
165	0.00934344100473701\\
166	0.00934324245642976\\
167	0.00934304034491221\\
168	0.0093428346064793\\
169	0.00934262517628997\\
170	0.0093424119883469\\
171	0.00934219497547575\\
172	0.00934197406930419\\
173	0.00934174920024043\\
174	0.00934152029745143\\
175	0.00934128728884067\\
176	0.00934105010102554\\
177	0.0093408086593143\\
178	0.00934056288768263\\
179	0.00934031270874978\\
180	0.0093400580437542\\
181	0.00933979881252881\\
182	0.0093395349334758\\
183	0.00933926632354093\\
184	0.00933899289818741\\
185	0.00933871457136921\\
186	0.00933843125550403\\
187	0.00933814286144563\\
188	0.0093378492984557\\
189	0.00933755047417525\\
190	0.00933724629459542\\
191	0.00933693666402776\\
192	0.00933662148507399\\
193	0.00933630065859514\\
194	0.0093359740836802\\
195	0.00933564165761412\\
196	0.00933530327584523\\
197	0.00933495883195209\\
198	0.00933460821760966\\
199	0.00933425132255488\\
200	0.00933388803455162\\
201	0.00933351823935491\\
202	0.00933314182067458\\
203	0.00933275866013817\\
204	0.00933236863725313\\
205	0.00933197162936836\\
206	0.00933156751163497\\
207	0.00933115615696634\\
208	0.00933073743599738\\
209	0.00933031121704311\\
210	0.00932987736605633\\
211	0.00932943574658456\\
212	0.00932898621972619\\
213	0.0093285286440857\\
214	0.00932806287572813\\
215	0.00932758876813258\\
216	0.00932710617214492\\
217	0.00932661493592952\\
218	0.00932611490492008\\
219	0.00932560592176952\\
220	0.0093250878262989\\
221	0.00932456045544534\\
222	0.00932402364320896\\
223	0.0093234772205988\\
224	0.00932292101557764\\
225	0.00932235485300581\\
226	0.00932177855458387\\
227	0.0093211919387942\\
228	0.00932059482084144\\
229	0.00931998701259174\\
230	0.00931936832251089\\
231	0.00931873855560115\\
232	0.00931809751333696\\
233	0.00931744499359923\\
234	0.00931678079060848\\
235	0.00931610469485662\\
236	0.00931541649303729\\
237	0.00931471596797501\\
238	0.00931400289855272\\
239	0.00931327705963808\\
240	0.00931253822200812\\
241	0.00931178615227254\\
242	0.00931102061279544\\
243	0.00931024136161539\\
244	0.00930944815236406\\
245	0.00930864073418312\\
246	0.00930781885163949\\
247	0.0093069822446389\\
248	0.0093061306483377\\
249	0.0093052637930529\\
250	0.00930438140417041\\
251	0.0093034832020513\\
252	0.00930256890193639\\
253	0.00930163821384867\\
254	0.00930069084249383\\
255	0.0092997264871588\\
256	0.00929874484160814\\
257	0.00929774559397829\\
258	0.00929672842666973\\
259	0.00929569301623684\\
260	0.00929463903327553\\
261	0.00929356614230848\\
262	0.00929247400166804\\
263	0.00929136226337673\\
264	0.00929023057302517\\
265	0.00928907856964746\\
266	0.00928790588559415\\
267	0.00928671214640243\\
268	0.00928549697066392\\
269	0.00928425996989008\\
270	0.00928300074837557\\
271	0.00928171890306\\
272	0.00928041402338833\\
273	0.00927908569116927\\
274	0.00927773348042881\\
275	0.00927635695725176\\
276	0.0092749556795893\\
277	0.00927352919708048\\
278	0.00927207705092298\\
279	0.00927059877371125\\
280	0.00926909388926583\\
281	0.00926756191245897\\
282	0.00926600234903667\\
283	0.00926441469543684\\
284	0.00926279843860374\\
285	0.00926115305579856\\
286	0.00925947801440592\\
287	0.00925777277173657\\
288	0.00925603677482595\\
289	0.00925426946022866\\
290	0.0092524702538089\\
291	0.00925063857052676\\
292	0.00924877381422034\\
293	0.00924687537738379\\
294	0.00924494264094119\\
295	0.00924297497401639\\
296	0.00924097173369885\\
297	0.00923893226480545\\
298	0.00923685589963861\\
299	0.00923474195774062\\
300	0.00923258974564453\\
301	0.00923039855662175\\
302	0.00922816767042654\\
303	0.00922589635303781\\
304	0.0092235838563983\\
305	0.00922122941815141\\
306	0.00921883226137546\\
307	0.00921639159431492\\
308	0.00921390661010708\\
309	0.00921137648650256\\
310	0.0092088003855786\\
311	0.00920617745344888\\
312	0.00920350681998642\\
313	0.00920078759859547\\
314	0.00919801888610708\\
315	0.00919519976283647\\
316	0.00919232929209615\\
317	0.00918940651995461\\
318	0.00918643047506029\\
319	0.00918340016848936\\
320	0.00918031459362137\\
321	0.00917717272604791\\
322	0.00917397352351965\\
323	0.00917071592593803\\
324	0.00916739885539842\\
325	0.00916402121629215\\
326	0.00916058189547558\\
327	0.00915707976251482\\
328	0.00915351367001522\\
329	0.00914988245404487\\
330	0.00914618493466163\\
331	0.00914241991655318\\
332	0.00913858618980006\\
333	0.00913468253077488\\
334	0.00913070770319948\\
335	0.0091266604594075\\
336	0.00912253954191899\\
337	0.00911834368555332\\
338	0.00911407162045904\\
339	0.00910972207626186\\
340	0.00910529378589186\\
341	0.00910078549893887\\
342	0.00909619595261919\\
343	0.00909152377963481\\
344	0.00908676758765\\
345	0.00908192597161524\\
346	0.00907699751383327\\
347	0.00907198078462663\\
348	0.00906687434498613\\
349	0.00906167675267459\\
350	0.00905638655439896\\
351	0.00905100241924003\\
352	0.00904552286378997\\
353	0.00903994608758106\\
354	0.00903427025068874\\
355	0.00902849347266619\\
356	0.0090226138315067\\
357	0.00901662936258298\\
358	0.00901053805740051\\
359	0.00900433786148541\\
360	0.00899802666873709\\
361	0.00899160232238428\\
362	0.00898506261350247\\
363	0.0089784052788706\\
364	0.00897162799864498\\
365	0.008964728393827\\
366	0.00895770402349778\\
367	0.00895055238178909\\
368	0.00894327089455504\\
369	0.00893585691570293\\
370	0.00892830772313262\\
371	0.00892062051422059\\
372	0.00891279240076288\\
373	0.00890482040325693\\
374	0.00889670144435709\\
375	0.00888843234130537\\
376	0.00888000979720011\\
377	0.00887143039126845\\
378	0.00886269056886668\\
379	0.00885378663281679\\
380	0.00884471474983117\\
381	0.0088354709255763\\
382	0.00882605098805673\\
383	0.00881645057115677\\
384	0.00880666509483924\\
385	0.00879668973995517\\
386	0.00878651941543478\\
387	0.00877614872036761\\
388	0.00876557191870218\\
389	0.00875478279931898\\
390	0.00874377548020158\\
391	0.00873254406419085\\
392	0.00872108240461992\\
393	0.00870938406743709\\
394	0.00869744231546698\\
395	0.00868525009351515\\
396	0.00867280001506848\\
397	0.00866008435157889\\
398	0.0086470950255808\\
399	0.00863382360912698\\
400	0.00862026132908542\\
401	0.00860639908035345\\
402	0.00859222744626044\\
403	0.00857773672126252\\
404	0.00856291692610347\\
405	0.00854775783126497\\
406	0.00853224903854124\\
407	0.00851637874887452\\
408	0.00850014017392674\\
409	0.00848352729068936\\
410	0.00846653532930649\\
411	0.00844916134249677\\
412	0.00843140494628865\\
413	0.0084132692340222\\
414	0.00839476293802955\\
415	0.00837589427803379\\
416	0.00835665512812732\\
417	0.00833703712372604\\
418	0.00831703164985451\\
419	0.00829662982675444\\
420	0.00827582249080892\\
421	0.00825460016794972\\
422	0.0082329530375656\\
423	0.00821087088992473\\
424	0.00818834308085885\\
425	0.0081653586062401\\
426	0.00814190635765207\\
427	0.00811797489406353\\
428	0.00809355238571992\\
429	0.00806862659615527\\
430	0.00804318486362214\\
431	0.00801721408211817\\
432	0.00799070068231922\\
433	0.00796363061299112\\
434	0.0079359893240296\\
435	0.00790776175364678\\
436	0.00787893232556888\\
437	0.00784948497019549\\
438	0.00781940320103334\\
439	0.00778867029193811\\
440	0.00775726930641245\\
441	0.00772518524823265\\
442	0.0076924019746264\\
443	0.00765889704039598\\
444	0.00762464640987071\\
445	0.00758962423797925\\
446	0.0075538026267833\\
447	0.00751715137017973\\
448	0.00747963769494051\\
449	0.00744122593783304\\
450	0.0074018779348395\\
451	0.00736154953242777\\
452	0.00732018995426554\\
453	0.00727774139820778\\
454	0.00723413764358201\\
455	0.0071893023992448\\
456	0.00714314881497628\\
457	0.00709611890625548\\
458	0.00707814715998119\\
459	0.00705950064542732\\
460	0.007040111390837\\
461	0.00701990586931755\\
462	0.006998799285829\\
463	0.00697669296911973\\
464	0.0069534716361377\\
465	0.00692900000368419\\
466	0.00690311877158677\\
467	0.00687563960169342\\
468	0.00684633893625431\\
469	0.00681495038705406\\
470	0.00678115540694164\\
471	0.00674457210883074\\
472	0.00670474309112749\\
473	0.00666369476105502\\
474	0.00662197457892596\\
475	0.00657957827687717\\
476	0.00653650234801006\\
477	0.00649274409124263\\
478	0.00644830169130089\\
479	0.00640317456720802\\
480	0.00635736400836668\\
481	0.0063108771065416\\
482	0.00626375221462998\\
483	0.00621601161985001\\
484	0.00616766531824011\\
485	0.00611872205674312\\
486	0.00606917904683886\\
487	0.00601901680727206\\
488	0.0059682650676501\\
489	0.00591694912971216\\
490	0.00586505640756299\\
491	0.00581251825469725\\
492	0.00575937889418669\\
493	0.00570569420173042\\
494	0.00565153429849766\\
495	0.00559698631085357\\
496	0.00554215845861785\\
497	0.00548718488524454\\
498	0.00543223164892994\\
499	0.00537750445415099\\
500	0.0053232586911779\\
501	0.00526980852182763\\
502	0.00521754039160833\\
503	0.00516692999283385\\
504	0.00511856338139452\\
505	0.00507316311871932\\
506	0.0050316199649441\\
507	0.0049905546194071\\
508	0.00494928986390594\\
509	0.00490747668529065\\
510	0.00486469649171915\\
511	0.00482092246210445\\
512	0.00477612639420774\\
513	0.00473027858039279\\
514	0.00468334734606081\\
515	0.00463529756612277\\
516	0.00458609874967004\\
517	0.00453572889551623\\
518	0.0044841677197608\\
519	0.00443139563389439\\
520	0.00437739300423209\\
521	0.00432213901410709\\
522	0.00426560983196854\\
523	0.00420777542104398\\
524	0.00414859175016062\\
525	0.00408799773907985\\
526	0.00402593207117221\\
527	0.00396233451736068\\
528	0.00389714705872905\\
529	0.00383031591407272\\
530	0.00376179331226578\\
531	0.00369153993728676\\
532	0.00361959527846768\\
533	0.0035462583284732\\
534	0.00347208389533472\\
535	0.00339908980553376\\
536	0.00332748166785388\\
537	0.00325747574193593\\
538	0.00318929462662516\\
539	0.00312316045408918\\
540	0.00305928458020588\\
541	0.00299785326245797\\
542	0.00293903789899305\\
543	0.00288294885860575\\
544	0.00282960002502755\\
545	0.00277886000119549\\
546	0.00272909952961711\\
547	0.00267981308881545\\
548	0.0026309828134207\\
549	0.00258257253401098\\
550	0.00253452574506337\\
551	0.00248676442562614\\
552	0.00243918939882931\\
553	0.00239168320911\\
554	0.00234410597758729\\
555	0.00229629841183345\\
556	0.0022480992879957\\
557	0.0021993849561593\\
558	0.00215013390444389\\
559	0.0021003217604376\\
560	0.0020499215771393\\
561	0.00199890396187714\\
562	0.00194723786732274\\
563	0.00189489179927033\\
564	0.00184183539852713\\
565	0.00178804137997534\\
566	0.00173348768596366\\
567	0.00167815996825155\\
568	0.00162205285376998\\
569	0.00156516806544259\\
570	0.00150750949614769\\
571	0.00144908358804972\\
572	0.00138989961244892\\
573	0.00132996979794916\\
574	0.00126930973751084\\
575	0.00120793876053359\\
576	0.00114588022454223\\
577	0.00108316193301972\\
578	0.00101981685596092\\
579	0.000955885432697557\\
580	0.000891416745494425\\
581	0.000826449428908736\\
582	0.00076103325331982\\
583	0.000695249245061119\\
584	0.000629112420938282\\
585	0.000564944790052522\\
586	0.000503134290179678\\
587	0.000444772794001453\\
588	0.000390227629686837\\
589	0.000340264159084332\\
590	0.000293674396315564\\
591	0.00024937252729716\\
592	0.000206953440128856\\
593	0.000166169107244796\\
594	0.000127115113967414\\
595	8.96111722547531e-05\\
596	5.42660945238511e-05\\
597	2.28062284332059e-05\\
598	2.9204464504877e-07\\
599	0\\
600	0\\
};
\addplot [color=red!25!mycolor17,solid,forget plot]
  table[row sep=crcr]{%
1	0.00752975719577589\\
2	0.00752975205343657\\
3	0.00752974681777696\\
4	0.00752974148710355\\
5	0.00752973605969214\\
6	0.00752973053378722\\
7	0.00752972490760141\\
8	0.00752971917931491\\
9	0.00752971334707491\\
10	0.00752970740899495\\
11	0.00752970136315436\\
12	0.00752969520759761\\
13	0.00752968894033371\\
14	0.00752968255933553\\
15	0.00752967606253913\\
16	0.00752966944784316\\
17	0.00752966271310816\\
18	0.00752965585615583\\
19	0.00752964887476838\\
20	0.00752964176668774\\
21	0.00752963452961493\\
22	0.00752962716120926\\
23	0.00752961965908754\\
24	0.00752961202082343\\
25	0.00752960424394652\\
26	0.00752959632594162\\
27	0.00752958826424795\\
28	0.00752958005625826\\
29	0.00752957169931803\\
30	0.00752956319072461\\
31	0.00752955452772636\\
32	0.00752954570752172\\
33	0.00752953672725836\\
34	0.00752952758403223\\
35	0.00752951827488665\\
36	0.00752950879681133\\
37	0.00752949914674141\\
38	0.0075294893215565\\
39	0.00752947931807964\\
40	0.00752946913307629\\
41	0.00752945876325331\\
42	0.00752944820525786\\
43	0.00752943745567639\\
44	0.00752942651103344\\
45	0.00752941536779064\\
46	0.00752940402234548\\
47	0.0075293924710302\\
48	0.00752938071011062\\
49	0.0075293687357849\\
50	0.00752935654418234\\
51	0.00752934413136215\\
52	0.00752933149331217\\
53	0.0075293186259476\\
54	0.00752930552510964\\
55	0.00752929218656425\\
56	0.00752927860600069\\
57	0.00752926477903021\\
58	0.00752925070118459\\
59	0.00752923636791476\\
60	0.0075292217745893\\
61	0.007529206916493\\
62	0.0075291917888253\\
63	0.00752917638669876\\
64	0.00752916070513756\\
65	0.00752914473907581\\
66	0.00752912848335602\\
67	0.00752911193272739\\
68	0.00752909508184412\\
69	0.00752907792526378\\
70	0.00752906045744548\\
71	0.00752904267274817\\
72	0.00752902456542876\\
73	0.00752900612964038\\
74	0.00752898735943041\\
75	0.00752896824873868\\
76	0.00752894879139544\\
77	0.00752892898111949\\
78	0.00752890881151607\\
79	0.00752888827607488\\
80	0.00752886736816801\\
81	0.00752884608104782\\
82	0.00752882440784475\\
83	0.00752880234156517\\
84	0.00752877987508915\\
85	0.00752875700116815\\
86	0.00752873371242279\\
87	0.0075287100013404\\
88	0.00752868586027273\\
89	0.00752866128143343\\
90	0.0075286362568956\\
91	0.0075286107785893\\
92	0.00752858483829895\\
93	0.00752855842766072\\
94	0.00752853153815989\\
95	0.00752850416112809\\
96	0.00752847628774064\\
97	0.00752844790901365\\
98	0.00752841901580124\\
99	0.00752838959879258\\
100	0.00752835964850895\\
101	0.00752832915530077\\
102	0.00752829810934449\\
103	0.00752826650063949\\
104	0.00752823431900494\\
105	0.00752820155407651\\
106	0.00752816819530313\\
107	0.00752813423194364\\
108	0.00752809965306338\\
109	0.00752806444753073\\
110	0.00752802860401361\\
111	0.00752799211097584\\
112	0.00752795495667353\\
113	0.00752791712915137\\
114	0.00752787861623882\\
115	0.00752783940554629\\
116	0.00752779948446119\\
117	0.00752775884014398\\
118	0.00752771745952413\\
119	0.00752767532929593\\
120	0.00752763243591434\\
121	0.0075275887655907\\
122	0.00752754430428839\\
123	0.00752749903771837\\
124	0.00752745295133474\\
125	0.00752740603033007\\
126	0.00752735825963083\\
127	0.00752730962389257\\
128	0.00752726010749509\\
129	0.00752720969453758\\
130	0.00752715836883358\\
131	0.00752710611390588\\
132	0.00752705291298136\\
133	0.0075269987489857\\
134	0.00752694360453804\\
135	0.00752688746194548\\
136	0.00752683030319757\\
137	0.00752677210996064\\
138	0.00752671286357203\\
139	0.00752665254503426\\
140	0.00752659113500905\\
141	0.0075265286138113\\
142	0.00752646496140288\\
143	0.00752640015738638\\
144	0.0075263341809987\\
145	0.0075262670111046\\
146	0.00752619862619003\\
147	0.00752612900435545\\
148	0.00752605812330895\\
149	0.00752598596035932\\
150	0.00752591249240895\\
151	0.0075258376959466\\
152	0.00752576154704007\\
153	0.00752568402132878\\
154	0.00752560509401615\\
155	0.00752552473986181\\
156	0.00752544293317384\\
157	0.00752535964780071\\
158	0.00752527485712319\\
159	0.00752518853404599\\
160	0.00752510065098941\\
161	0.00752501117988075\\
162	0.00752492009214558\\
163	0.00752482735869887\\
164	0.00752473294993598\\
165	0.00752463683572344\\
166	0.00752453898538968\\
167	0.00752443936771548\\
168	0.0075243379509243\\
169	0.00752423470267244\\
170	0.00752412959003905\\
171	0.00752402257951598\\
172	0.00752391363699737\\
173	0.00752380272776916\\
174	0.00752368981649834\\
175	0.0075235748672221\\
176	0.00752345784333668\\
177	0.00752333870758614\\
178	0.00752321742205086\\
179	0.00752309394813589\\
180	0.00752296824655904\\
181	0.00752284027733886\\
182	0.0075227099997823\\
183	0.00752257737247225\\
184	0.00752244235325482\\
185	0.0075223048992264\\
186	0.00752216496672052\\
187	0.0075220225112945\\
188	0.00752187748771579\\
189	0.00752172984994817\\
190	0.00752157955113768\\
191	0.00752142654359825\\
192	0.00752127077879721\\
193	0.00752111220734044\\
194	0.0075209507789573\\
195	0.00752078644248533\\
196	0.00752061914585465\\
197	0.00752044883607212\\
198	0.00752027545920519\\
199	0.00752009896036555\\
200	0.00751991928369243\\
201	0.00751973637233562\\
202	0.00751955016843826\\
203	0.00751936061311927\\
204	0.00751916764645548\\
205	0.00751897120746353\\
206	0.00751877123408139\\
207	0.00751856766314955\\
208	0.007518360430392\\
209	0.00751814947039674\\
210	0.00751793471659604\\
211	0.0075177161012464\\
212	0.00751749355540804\\
213	0.00751726700892417\\
214	0.00751703639039983\\
215	0.00751680162718042\\
216	0.00751656264532977\\
217	0.00751631936960797\\
218	0.00751607172344869\\
219	0.00751581962893617\\
220	0.00751556300678189\\
221	0.00751530177630066\\
222	0.00751503585538647\\
223	0.00751476516048784\\
224	0.00751448960658274\\
225	0.00751420910715318\\
226	0.00751392357415917\\
227	0.00751363291801246\\
228	0.00751333704754961\\
229	0.00751303587000479\\
230	0.007512729290982\\
231	0.00751241721442677\\
232	0.00751209954259747\\
233	0.00751177617603611\\
234	0.0075114470135386\\
235	0.00751111195212445\\
236	0.00751077088700609\\
237	0.00751042371155747\\
238	0.00751007031728229\\
239	0.00750971059378157\\
240	0.00750934442872073\\
241	0.00750897170779599\\
242	0.0075085923147003\\
243	0.00750820613108868\\
244	0.00750781303654287\\
245	0.00750741290853539\\
246	0.0075070056223931\\
247	0.00750659105125999\\
248	0.00750616906605933\\
249	0.0075057395354553\\
250	0.00750530232581381\\
251	0.0075048573011628\\
252	0.00750440432315164\\
253	0.00750394325101007\\
254	0.00750347394150628\\
255	0.00750299624890437\\
256	0.00750251002492099\\
257	0.0075020151186813\\
258	0.00750151137667419\\
259	0.00750099864270668\\
260	0.00750047675785759\\
261	0.00749994556043041\\
262	0.00749940488590532\\
263	0.00749885456689047\\
264	0.00749829443307233\\
265	0.00749772431116531\\
266	0.00749714402486035\\
267	0.00749655339477289\\
268	0.00749595223838985\\
269	0.00749534037001628\\
270	0.00749471760072198\\
271	0.00749408373829001\\
272	0.00749343858717124\\
273	0.0074927819484534\\
274	0.00749211361985863\\
275	0.00749143339577178\\
276	0.00749074106721274\\
277	0.00749003642144281\\
278	0.00748931924166022\\
279	0.00748858930723845\\
280	0.00748784639371893\\
281	0.00748709027274426\\
282	0.00748632071199015\\
283	0.00748553747509582\\
284	0.00748474032159292\\
285	0.00748392900683273\\
286	0.00748310328191177\\
287	0.0074822628935955\\
288	0.00748140758424024\\
289	0.00748053709171292\\
290	0.00747965114930884\\
291	0.007478749485667\\
292	0.00747783182468312\\
293	0.00747689788541996\\
294	0.00747594738201499\\
295	0.0074749800235849\\
296	0.00747399551412713\\
297	0.0074729935524178\\
298	0.00747197383190605\\
299	0.00747093604060444\\
300	0.00746987986097498\\
301	0.00746880496981065\\
302	0.00746771103811184\\
303	0.00746659773095754\\
304	0.0074654647073704\\
305	0.00746431162017531\\
306	0.00746313811585005\\
307	0.00746194383436596\\
308	0.00746072840901369\\
309	0.0074594914662036\\
310	0.00745823262521787\\
311	0.00745695149787069\\
312	0.00745564768800999\\
313	0.00745432079083502\\
314	0.00745297039232933\\
315	0.0074515960700461\\
316	0.00745019739601793\\
317	0.00744877393411322\\
318	0.00744732523925988\\
319	0.00744585085715755\\
320	0.00744435032397979\\
321	0.00744282316606823\\
322	0.00744126889962242\\
323	0.00743968703038992\\
324	0.00743807705336373\\
325	0.00743643845249687\\
326	0.00743477070044707\\
327	0.00743307325836974\\
328	0.00743134557578306\\
329	0.007429587090537\\
330	0.00742779722892859\\
331	0.00742597540601873\\
332	0.00742412102622348\\
333	0.00742223348427516\\
334	0.00742031216667975\\
335	0.00741835645384029\\
336	0.00741636572308849\\
337	0.00741433935301452\\
338	0.00741227672988319\\
339	0.00741017725826789\\
340	0.00740804038339811\\
341	0.00740586565793154\\
342	0.007403653038201\\
343	0.00740140271340758\\
344	0.00739911405810567\\
345	0.00739678643373589\\
346	0.00739441918769577\\
347	0.00739201165231525\\
348	0.00738956314374368\\
349	0.00738707296073435\\
350	0.00738454038327247\\
351	0.00738196466808564\\
352	0.00737934504954275\\
353	0.00737668074415684\\
354	0.0073739709498462\\
355	0.00737121484528177\\
356	0.00736841158958132\\
357	0.00736556032300077\\
358	0.00736266016990002\\
359	0.00735971024460689\\
360	0.00735670964922029\\
361	0.00735365742967529\\
362	0.00735055259845563\\
363	0.00734739413805547\\
364	0.00734418099895487\\
365	0.00734091209734935\\
366	0.0073375863125907\\
367	0.00733420248428568\\
368	0.00733075940898739\\
369	0.00732725583639804\\
370	0.00732369046498075\\
371	0.00732006193684797\\
372	0.00731636883174691\\
373	0.00731260965987817\\
374	0.00730878285311568\\
375	0.00730488675383421\\
376	0.00730091959977377\\
377	0.00729687950186456\\
378	0.00729276441000695\\
379	0.00728857206466934\\
380	0.00728429996771987\\
381	0.00727994554468052\\
382	0.00727550599739177\\
383	0.00727097823098026\\
384	0.00726635879220191\\
385	0.00726164378880268\\
386	0.00725682877613685\\
387	0.00725190857554885\\
388	0.00724687691834787\\
389	0.00724172555728899\\
390	0.00723644137563214\\
391	0.00723101554785491\\
392	0.00722543988330375\\
393	0.00721970529499155\\
394	0.00721380166796882\\
395	0.00720771770449372\\
396	0.00720144074128882\\
397	0.00719495653310726\\
398	0.00718824899550521\\
399	0.00718129989806737\\
400	0.00717408849727336\\
401	0.00716659109563358\\
402	0.00715878051056588\\
403	0.0071506254326964\\
404	0.00714208964907079\\
405	0.00713313110317246\\
406	0.00712370076604672\\
407	0.00711374132651605\\
408	0.00710318559509768\\
409	0.0070919547364818\\
410	0.00707995392375266\\
411	0.00706706986772846\\
412	0.00705316692827304\\
413	0.00703808231943389\\
414	0.00702167943378768\\
415	0.00700498846860179\\
416	0.0069880052658891\\
417	0.00697072568106828\\
418	0.00695314558986649\\
419	0.00693526089418565\\
420	0.00691706752383685\\
421	0.00689856142557651\\
422	0.00687973851619553\\
423	0.0068605945394869\\
424	0.00684112469188092\\
425	0.00682132287849355\\
426	0.00680118231290836\\
427	0.00678069868600467\\
428	0.00675986817327228\\
429	0.0067386870870183\\
430	0.00671715189842607\\
431	0.00669525926230584\\
432	0.00667300604491383\\
433	0.00665038935527493\\
434	0.00662740658050536\\
435	0.00660405542568729\\
436	0.00658033395887662\\
437	0.00655624066174944\\
438	0.0065317744859354\\
439	0.00650693491354474\\
440	0.00648172201726202\\
441	0.00645613645862427\\
442	0.00643017960258967\\
443	0.00640385381117351\\
444	0.00637716262060013\\
445	0.00635011095562376\\
446	0.00632270539431041\\
447	0.00629495451382842\\
448	0.00626686941319138\\
449	0.00623846479105154\\
450	0.00620976236471191\\
451	0.00618080754270315\\
452	0.0061516398098084\\
453	0.00612229921115789\\
454	0.00609283475388691\\
455	0.00606330621852963\\
456	0.00603378593943971\\
457	0.00600435309360236\\
458	0.00597472622139851\\
459	0.005944936642734\\
460	0.00591497869761773\\
461	0.00588488049490957\\
462	0.00585470720823154\\
463	0.00582453831237067\\
464	0.00579447066612896\\
465	0.00576462228241092\\
466	0.0057351371544715\\
467	0.00570619111318079\\
468	0.0056779990950215\\
469	0.00565082417820276\\
470	0.00562498881388385\\
471	0.00560088871720802\\
472	0.00557900960011686\\
473	0.00555729142567498\\
474	0.00553516701095325\\
475	0.00551262943556047\\
476	0.00548967173836112\\
477	0.00546628689209586\\
478	0.00544246779344238\\
479	0.00541820726369776\\
480	0.00539349805074807\\
481	0.00536833279070289\\
482	0.00534270362687262\\
483	0.00531660233968035\\
484	0.00529002037339372\\
485	0.00526294907004596\\
486	0.00523538001615325\\
487	0.00520730590458239\\
488	0.00517871950737512\\
489	0.00514961353753942\\
490	0.00511998114587108\\
491	0.00508981678447065\\
492	0.00505911483619851\\
493	0.00502786944273571\\
494	0.00499607435605367\\
495	0.00496372307028727\\
496	0.00493080396722983\\
497	0.00489730203712122\\
498	0.00486319740704639\\
499	0.00482846311294119\\
500	0.00479306325290543\\
501	0.00475696059332692\\
502	0.00472010551273883\\
503	0.00468243136231268\\
504	0.0046438481817106\\
505	0.00460423429291239\\
506	0.0045634250990333\\
507	0.00452134962894445\\
508	0.00447795628043695\\
509	0.00443319610689204\\
510	0.00438702368694223\\
511	0.00433939085082558\\
512	0.00429024426436863\\
513	0.00423951819484293\\
514	0.00418714193967163\\
515	0.00413305427633992\\
516	0.00407720885976529\\
517	0.00401958143675479\\
518	0.00396017969591317\\
519	0.0038990565281705\\
520	0.00383632709739516\\
521	0.00377219079310824\\
522	0.00370696148339691\\
523	0.00364150103709476\\
524	0.00357730663765718\\
525	0.003514572456412\\
526	0.003453502088954\\
527	0.00339430491184796\\
528	0.00333719045874468\\
529	0.0032823599463531\\
530	0.00322999434560389\\
531	0.00318023771765325\\
532	0.00313317193087718\\
533	0.00308877696155659\\
534	0.00304677637041605\\
535	0.00300533620988421\\
536	0.00296445707726527\\
537	0.00292412483708432\\
538	0.00288430801836305\\
539	0.00284495531333978\\
540	0.002805993459618\\
541	0.00276732591444782\\
542	0.00272883235327698\\
543	0.00269037068913318\\
544	0.00265178253040087\\
545	0.00261290383274353\\
546	0.00257366270674616\\
547	0.00253401370536589\\
548	0.00249390732631952\\
549	0.00245329074706882\\
550	0.00241210889436304\\
551	0.00237030589512094\\
552	0.00232782673648705\\
553	0.00228461998623048\\
554	0.00224065046301698\\
555	0.0021958884073572\\
556	0.00215031064061035\\
557	0.00210389946353566\\
558	0.00205663759187896\\
559	0.00200850842755855\\
560	0.00195949635800379\\
561	0.00190958708395927\\
562	0.00185876795497196\\
563	0.00180702829309604\\
564	0.00175435968293312\\
565	0.00170075620342841\\
566	0.00164621457913793\\
567	0.0015907342281127\\
568	0.00153431721326273\\
569	0.00147696824692171\\
570	0.00141869494997049\\
571	0.00135950808434287\\
572	0.00129942189673999\\
573	0.00123845564645323\\
574	0.00117663625677103\\
575	0.00111400030455689\\
576	0.00105059644939085\\
577	0.000986530157595525\\
578	0.000921762352719769\\
579	0.000856219106191558\\
580	0.000791664481391484\\
581	0.000729004586915961\\
582	0.000668600899830575\\
583	0.000611598515788096\\
584	0.000558751325735728\\
585	0.000508153571256534\\
586	0.000460174865235934\\
587	0.000414142193949086\\
588	0.000370019893892401\\
589	0.000327223396935425\\
590	0.000285420585329276\\
591	0.000244524545387522\\
592	0.000204480456411576\\
593	0.000165271055315995\\
594	0.000126870159937007\\
595	8.96111722547517e-05\\
596	5.42660945238508e-05\\
597	2.28062284332056e-05\\
598	2.9204464504877e-07\\
599	0\\
600	0\\
};
\addplot [color=mycolor19,solid,forget plot]
  table[row sep=crcr]{%
1	0.00695390139793097\\
2	0.00695389015754209\\
3	0.00695387871314316\\
4	0.00695386706103132\\
5	0.00695385519743642\\
6	0.00695384311851986\\
7	0.00695383082037335\\
8	0.00695381829901761\\
9	0.00695380555040107\\
10	0.00695379257039861\\
11	0.0069537793548102\\
12	0.00695376589935953\\
13	0.00695375219969261\\
14	0.00695373825137641\\
15	0.00695372404989741\\
16	0.0069537095906601\\
17	0.00695369486898552\\
18	0.00695367988010977\\
19	0.00695366461918243\\
20	0.00695364908126501\\
21	0.00695363326132935\\
22	0.00695361715425597\\
23	0.00695360075483249\\
24	0.00695358405775181\\
25	0.00695356705761055\\
26	0.0069535497489072\\
27	0.0069535321260403\\
28	0.00695351418330684\\
29	0.00695349591490013\\
30	0.00695347731490815\\
31	0.00695345837731154\\
32	0.00695343909598167\\
33	0.00695341946467864\\
34	0.0069533994770493\\
35	0.00695337912662519\\
36	0.00695335840682044\\
37	0.00695333731092963\\
38	0.00695331583212566\\
39	0.0069532939634575\\
40	0.00695327169784803\\
41	0.00695324902809163\\
42	0.00695322594685197\\
43	0.00695320244665956\\
44	0.00695317851990943\\
45	0.00695315415885856\\
46	0.0069531293556235\\
47	0.00695310410217771\\
48	0.0069530783903491\\
49	0.00695305221181729\\
50	0.006953025558111\\
51	0.00695299842060523\\
52	0.0069529707905186\\
53	0.00695294265891042\\
54	0.00695291401667788\\
55	0.00695288485455306\\
56	0.00695285516309997\\
57	0.00695282493271153\\
58	0.00695279415360646\\
59	0.00695276281582613\\
60	0.00695273090923136\\
61	0.00695269842349913\\
62	0.00695266534811933\\
63	0.00695263167239128\\
64	0.0069525973854204\\
65	0.00695256247611462\\
66	0.00695252693318083\\
67	0.00695249074512126\\
68	0.00695245390022984\\
69	0.00695241638658834\\
70	0.0069523781920626\\
71	0.00695233930429862\\
72	0.00695229971071861\\
73	0.00695225939851692\\
74	0.00695221835465596\\
75	0.00695217656586201\\
76	0.00695213401862091\\
77	0.00695209069917379\\
78	0.00695204659351266\\
79	0.00695200168737586\\
80	0.00695195596624356\\
81	0.00695190941533302\\
82	0.00695186201959392\\
83	0.00695181376370349\\
84	0.0069517646320617\\
85	0.00695171460878611\\
86	0.0069516636777069\\
87	0.00695161182236169\\
88	0.00695155902599015\\
89	0.00695150527152884\\
90	0.00695145054160554\\
91	0.00695139481853385\\
92	0.00695133808430745\\
93	0.00695128032059439\\
94	0.00695122150873117\\
95	0.00695116162971682\\
96	0.00695110066420685\\
97	0.00695103859250703\\
98	0.00695097539456708\\
99	0.00695091104997433\\
100	0.0069508455379472\\
101	0.00695077883732847\\
102	0.00695071092657865\\
103	0.00695064178376904\\
104	0.00695057138657477\\
105	0.00695049971226762\\
106	0.00695042673770889\\
107	0.00695035243934188\\
108	0.00695027679318451\\
109	0.0069501997748216\\
110	0.00695012135939712\\
111	0.0069500415216063\\
112	0.00694996023568754\\
113	0.00694987747541421\\
114	0.00694979321408632\\
115	0.00694970742452203\\
116	0.006949620079049\\
117	0.00694953114949558\\
118	0.00694944060718188\\
119	0.00694934842291061\\
120	0.00694925456695786\\
121	0.00694915900906359\\
122	0.00694906171842212\\
123	0.00694896266367224\\
124	0.00694886181288735\\
125	0.00694875913356523\\
126	0.00694865459261785\\
127	0.00694854815636073\\
128	0.00694843979050242\\
129	0.00694832946013351\\
130	0.00694821712971561\\
131	0.00694810276307007\\
132	0.00694798632336657\\
133	0.00694786777311141\\
134	0.00694774707413566\\
135	0.00694762418758307\\
136	0.00694749907389779\\
137	0.00694737169281185\\
138	0.00694724200333245\\
139	0.00694710996372896\\
140	0.00694697553151976\\
141	0.00694683866345879\\
142	0.00694669931552195\\
143	0.00694655744289313\\
144	0.00694641299995012\\
145	0.00694626594025021\\
146	0.00694611621651549\\
147	0.00694596378061796\\
148	0.00694580858356443\\
149	0.00694565057548093\\
150	0.00694548970559717\\
151	0.00694532592223043\\
152	0.00694515917276932\\
153	0.00694498940365725\\
154	0.00694481656037552\\
155	0.00694464058742624\\
156	0.00694446142831484\\
157	0.00694427902553231\\
158	0.00694409332053714\\
159	0.00694390425373695\\
160	0.00694371176446977\\
161	0.006943515790985\\
162	0.006943316270424\\
163	0.00694311313880048\\
164	0.00694290633098026\\
165	0.00694269578066109\\
166	0.00694248142035172\\
167	0.00694226318135076\\
168	0.00694204099372528\\
169	0.00694181478628887\\
170	0.00694158448657936\\
171	0.00694135002083621\\
172	0.00694111131397747\\
173	0.00694086828957625\\
174	0.00694062086983692\\
175	0.00694036897557088\\
176	0.00694011252617177\\
177	0.00693985143959037\\
178	0.00693958563230906\\
179	0.00693931501931578\\
180	0.00693903951407757\\
181	0.00693875902851364\\
182	0.006938473472968\\
183	0.00693818275618153\\
184	0.0069378867852637\\
185	0.00693758546566369\\
186	0.00693727870114104\\
187	0.00693696639373578\\
188	0.00693664844373811\\
189	0.00693632474965749\\
190	0.00693599520819113\\
191	0.00693565971419217\\
192	0.00693531816063699\\
193	0.00693497043859225\\
194	0.00693461643718118\\
195	0.00693425604354933\\
196	0.00693388914282978\\
197	0.0069335156181077\\
198	0.00693313535038432\\
199	0.00693274821854031\\
200	0.00693235409929843\\
201	0.00693195286718566\\
202	0.00693154439449466\\
203	0.00693112855124446\\
204	0.00693070520514063\\
205	0.00693027422153472\\
206	0.00692983546338289\\
207	0.00692938879120407\\
208	0.00692893406303712\\
209	0.00692847113439755\\
210	0.00692799985823328\\
211	0.00692752008487976\\
212	0.00692703166201432\\
213	0.00692653443460973\\
214	0.00692602824488693\\
215	0.00692551293226704\\
216	0.00692498833332255\\
217	0.00692445428172762\\
218	0.00692391060820754\\
219	0.00692335714048752\\
220	0.0069227937032403\\
221	0.00692222011803323\\
222	0.00692163620327414\\
223	0.00692104177415652\\
224	0.00692043664260375\\
225	0.00691982061721227\\
226	0.00691919350319398\\
227	0.00691855510231751\\
228	0.00691790521284864\\
229	0.00691724362948962\\
230	0.00691657014331765\\
231	0.00691588454172209\\
232	0.00691518660834084\\
233	0.00691447612299557\\
234	0.00691375286162589\\
235	0.00691301659622249\\
236	0.00691226709475902\\
237	0.00691150412112304\\
238	0.00691072743504577\\
239	0.00690993679203053\\
240	0.0069091319432803\\
241	0.00690831263562384\\
242	0.00690747861144072\\
243	0.00690662960858509\\
244	0.00690576536030822\\
245	0.0069048855951798\\
246	0.00690399003700792\\
247	0.00690307840475774\\
248	0.00690215041246895\\
249	0.00690120576917171\\
250	0.00690024417880139\\
251	0.00689926534011188\\
252	0.00689826894658748\\
253	0.00689725468635337\\
254	0.00689622224208467\\
255	0.00689517129091393\\
256	0.00689410150433732\\
257	0.00689301254811908\\
258	0.00689190408219455\\
259	0.00689077576057163\\
260	0.00688962723123057\\
261	0.00688845813602224\\
262	0.00688726811056464\\
263	0.00688605678413775\\
264	0.00688482377957676\\
265	0.00688356871316349\\
266	0.00688229119451611\\
267	0.00688099082647716\\
268	0.00687966720499996\\
269	0.00687831991903377\\
270	0.00687694855040877\\
271	0.00687555267372398\\
272	0.00687413185624691\\
273	0.00687268565784921\\
274	0.00687121363104095\\
275	0.00686971532124631\\
276	0.00686819026753304\\
277	0.00686663800347691\\
278	0.00686505805409404\\
279	0.00686344993264916\\
280	0.0068618131435009\\
281	0.00686014718256823\\
282	0.00685845153720273\\
283	0.00685672568606032\\
284	0.00685496909897197\\
285	0.00685318123681425\\
286	0.00685136155137918\\
287	0.00684950948524381\\
288	0.00684762447163972\\
289	0.00684570593432244\\
290	0.00684375328744113\\
291	0.0068417659354086\\
292	0.00683974327277196\\
293	0.00683768468408408\\
294	0.00683558954377621\\
295	0.00683345721603199\\
296	0.00683128705466313\\
297	0.00682907840298717\\
298	0.00682683059370771\\
299	0.00682454294879728\\
300	0.00682221477938361\\
301	0.00681984538563942\\
302	0.00681743405667633\\
303	0.00681498007044334\\
304	0.00681248269363003\\
305	0.00680994118157487\\
306	0.00680735477817774\\
307	0.00680472271581441\\
308	0.00680204421524527\\
309	0.00679931848549755\\
310	0.00679654472366677\\
311	0.00679372211450116\\
312	0.00679084982945135\\
313	0.00678792702453238\\
314	0.00678495283606589\\
315	0.00678192637496652\\
316	0.00677884673122257\\
317	0.00677571301116319\\
318	0.00677252431404008\\
319	0.00676927972679045\\
320	0.00676597832402508\\
321	0.00676261916802415\\
322	0.00675920130874011\\
323	0.0067557237838069\\
324	0.00675218561855415\\
325	0.00674858582602453\\
326	0.00674492340699199\\
327	0.00674119734997746\\
328	0.00673740663125787\\
329	0.00673355021486284\\
330	0.00672962705255205\\
331	0.00672563608376393\\
332	0.00672157623552386\\
333	0.0067174464222968\\
334	0.00671324554576512\\
335	0.0067089724945067\\
336	0.00670462614354091\\
337	0.00670020535369889\\
338	0.00669570897075152\\
339	0.00669113582416517\\
340	0.006686484725114\\
341	0.00668175446227064\\
342	0.00667694378810846\\
343	0.0066720514101265\\
344	0.00666707601442083\\
345	0.0066620162650849\\
346	0.0066568708036065\\
347	0.00665163824824929\\
348	0.00664631719336097\\
349	0.00664090620841075\\
350	0.0066354038361421\\
351	0.00662980858811405\\
352	0.00662411894676957\\
353	0.00661833337075079\\
354	0.00661245029465419\\
355	0.00660646812903499\\
356	0.00660038526113502\\
357	0.00659420005779262\\
358	0.00658791087504414\\
359	0.0065815160877256\\
360	0.00657501416938438\\
361	0.00656840376917955\\
362	0.00656168322022079\\
363	0.00655485078322227\\
364	0.00654790469801547\\
365	0.00654084318433572\\
366	0.00653366444282823\\
367	0.00652636665631342\\
368	0.00651894799135598\\
369	0.00651140660018733\\
370	0.00650374062303331\\
371	0.00649594819089508\\
372	0.00648802742880782\\
373	0.00647997645952718\\
374	0.00647179340737348\\
375	0.00646347640134645\\
376	0.00645502357491394\\
377	0.00644643305523614\\
378	0.00643770292244578\\
379	0.0064288310908781\\
380	0.00641981501860316\\
381	0.00641065128683896\\
382	0.0064013378902591\\
383	0.00639187317032329\\
384	0.00638225559557552\\
385	0.0063724837922119\\
386	0.00636255658396266\\
387	0.0063524730422203\\
388	0.00634223255301183\\
389	0.00633183492038799\\
390	0.00632128057070477\\
391	0.00631057048172071\\
392	0.00629970623276913\\
393	0.00628869011538802\\
394	0.00627752526468924\\
395	0.00626621581572812\\
396	0.00625476709009516\\
397	0.00624318581914798\\
398	0.0062314804117944\\
399	0.00621966127660698\\
400	0.00620774121040289\\
401	0.00619573586840969\\
402	0.00618366433499069\\
403	0.00617154981904448\\
404	0.00615942050553408\\
405	0.00614731060645685\\
406	0.00613526167730345\\
407	0.00612332430667249\\
408	0.00611156033221292\\
409	0.0061000470048815\\
410	0.00608888767423589\\
411	0.00607819528708678\\
412	0.0060681075480346\\
413	0.00605879216119026\\
414	0.00605039235996826\\
415	0.00604185743285749\\
416	0.00603318689318356\\
417	0.00602438049187169\\
418	0.00601543826557051\\
419	0.00600636059380947\\
420	0.00599714826356753\\
421	0.00598780253140995\\
422	0.00597832514686596\\
423	0.00596871821590453\\
424	0.00595898350381705\\
425	0.0059491197902136\\
426	0.00593911287090259\\
427	0.00592893537021737\\
428	0.00591858525772538\\
429	0.00590806053171042\\
430	0.00589735922382137\\
431	0.00588647940413584\\
432	0.0058754191866799\\
433	0.00586417673545369\\
434	0.00585275027102884\\
435	0.00584113807777788\\
436	0.00582933851177035\\
437	0.00581735000938853\\
438	0.00580517109684997\\
439	0.00579280040080217\\
440	0.00578023666036211\\
441	0.00576747874230785\\
442	0.00575452565753874\\
443	0.0057413765754582\\
444	0.00572803083883235\\
445	0.00571448797890645\\
446	0.00570074773024775\\
447	0.00568681004396134\\
448	0.00567267509619579\\
449	0.00565834328446983\\
450	0.00564381517755388\\
451	0.00562909122899558\\
452	0.0056141718403278\\
453	0.00559905734025508\\
454	0.00558374783688781\\
455	0.00556824300784175\\
456	0.0055525418165271\\
457	0.00553664223866819\\
458	0.0055205460104428\\
459	0.0055042548798025\\
460	0.00548777105199359\\
461	0.00547109670729628\\
462	0.00545423336834274\\
463	0.00543718144625986\\
464	0.00541993961752241\\
465	0.00540250397815816\\
466	0.00538486693980148\\
467	0.00536701575891807\\
468	0.00534893050941499\\
469	0.00533058140256037\\
470	0.00531192525664044\\
471	0.00529290082854101\\
472	0.00527342265064941\\
473	0.00525345928920426\\
474	0.00523299400938363\\
475	0.00521200905849416\\
476	0.00519048576086233\\
477	0.00516840498273882\\
478	0.00514574679734546\\
479	0.00512249027630091\\
480	0.00509861340375058\\
481	0.00507409300201966\\
482	0.00504890468123225\\
483	0.00502302294423405\\
484	0.00499642067975817\\
485	0.00496906674974231\\
486	0.0049409272542987\\
487	0.00491196446639418\\
488	0.00488214330691491\\
489	0.00485142886448882\\
490	0.0048197850881542\\
491	0.00478717504516566\\
492	0.00475356100831737\\
493	0.00471890451952105\\
494	0.0046831664209815\\
495	0.00464630683692947\\
496	0.00460828516353634\\
497	0.0045690598648893\\
498	0.00452858806221636\\
499	0.00448682481535209\\
500	0.00444372186236416\\
501	0.00439922444155628\\
502	0.00435326183881953\\
503	0.0043057663241142\\
504	0.00425668069219755\\
505	0.00420596463645533\\
506	0.00415360351552566\\
507	0.00409961616185581\\
508	0.00404406752357827\\
509	0.00398708599350592\\
510	0.00392888654377625\\
511	0.00386979945639612\\
512	0.00381098264892001\\
513	0.00375343889528498\\
514	0.00369735913363088\\
515	0.00364294323764303\\
516	0.00359039584672001\\
517	0.00353991996381999\\
518	0.00349170750589745\\
519	0.00344592571380514\\
520	0.00340269812280261\\
521	0.00336207834154239\\
522	0.00332401368087333\\
523	0.00328787877285572\\
524	0.00325235286091564\\
525	0.00321743390702476\\
526	0.00318310583236878\\
527	0.00314933622795304\\
528	0.00311607420835653\\
529	0.00308324867862273\\
530	0.00305076739670388\\
531	0.00301851739566748\\
532	0.00298636765065977\\
533	0.0029541753336562\\
534	0.00292180350962232\\
535	0.00288922044366352\\
536	0.00285639051087333\\
537	0.00282327447655823\\
538	0.00278982996972125\\
539	0.00275601217596438\\
540	0.00272177476417683\\
541	0.002687071039504\\
542	0.00265185528873788\\
543	0.00261608420867369\\
544	0.00257971821736867\\
545	0.0025427223019672\\
546	0.00250506177064863\\
547	0.00246670110567965\\
548	0.00242760442670507\\
549	0.0023877360820317\\
550	0.00234706140132897\\
551	0.00230554737593107\\
552	0.00226317048307561\\
553	0.00221991333580777\\
554	0.00217575964277463\\
555	0.00213069413955516\\
556	0.00208470244627019\\
557	0.0020377709651064\\
558	0.00198988703819046\\
559	0.00194103911958854\\
560	0.00189121696355546\\
561	0.00184041183165241\\
562	0.00178861672312973\\
563	0.00173582662502646\\
564	0.00168203875846467\\
565	0.0016272528277323\\
566	0.00157147127416107\\
567	0.00151469961622109\\
568	0.00145694766974197\\
569	0.00139823255111959\\
570	0.00133858101940291\\
571	0.00127804978197904\\
572	0.00121673136249583\\
573	0.00115457311231883\\
574	0.00109149386651943\\
575	0.00102737860674039\\
576	0.000964660699241159\\
577	0.000903620067808924\\
578	0.000844693973416802\\
579	0.00078896063058065\\
580	0.000735683997274819\\
581	0.000684393506326076\\
582	0.000635262737748697\\
583	0.000587719065319528\\
584	0.000541596458167286\\
585	0.000496834403922398\\
586	0.00045297776209589\\
587	0.0004098793501683\\
588	0.000367452930850881\\
589	0.000325673544872563\\
590	0.00028455397376655\\
591	0.000244105317769071\\
592	0.000204333810489543\\
593	0.000165233217676712\\
594	0.000126870159937007\\
595	8.96111722547522e-05\\
596	5.4266094523851e-05\\
597	2.28062284332059e-05\\
598	2.9204464504877e-07\\
599	0\\
600	0\\
};
\addplot [color=red!50!mycolor17,solid,forget plot]
  table[row sep=crcr]{%
1	0.00628094651839509\\
2	0.00628094102369023\\
3	0.00628093542892778\\
4	0.00628092973228313\\
5	0.00628092393189816\\
6	0.00628091802588075\\
7	0.00628091201230395\\
8	0.00628090588920548\\
9	0.00628089965458708\\
10	0.0062808933064137\\
11	0.00628088684261299\\
12	0.00628088026107447\\
13	0.00628087355964898\\
14	0.00628086673614781\\
15	0.00628085978834202\\
16	0.00628085271396175\\
17	0.00628084551069534\\
18	0.00628083817618871\\
19	0.00628083070804443\\
20	0.00628082310382097\\
21	0.00628081536103192\\
22	0.00628080747714516\\
23	0.00628079944958188\\
24	0.00628079127571589\\
25	0.00628078295287261\\
26	0.00628077447832827\\
27	0.00628076584930893\\
28	0.00628075706298955\\
29	0.00628074811649311\\
30	0.00628073900688952\\
31	0.00628072973119478\\
32	0.00628072028636988\\
33	0.00628071066931981\\
34	0.00628070087689251\\
35	0.00628069090587783\\
36	0.00628068075300646\\
37	0.00628067041494876\\
38	0.00628065988831376\\
39	0.00628064916964789\\
40	0.00628063825543392\\
41	0.00628062714208971\\
42	0.00628061582596708\\
43	0.0062806043033505\\
44	0.00628059257045588\\
45	0.00628058062342928\\
46	0.00628056845834568\\
47	0.00628055607120759\\
48	0.00628054345794367\\
49	0.00628053061440746\\
50	0.00628051753637593\\
51	0.00628050421954806\\
52	0.00628049065954342\\
53	0.00628047685190062\\
54	0.00628046279207592\\
55	0.00628044847544159\\
56	0.00628043389728444\\
57	0.00628041905280421\\
58	0.00628040393711189\\
59	0.00628038854522819\\
60	0.00628037287208172\\
61	0.00628035691250748\\
62	0.00628034066124487\\
63	0.00628032411293615\\
64	0.00628030726212452\\
65	0.00628029010325225\\
66	0.00628027263065894\\
67	0.0062802548385795\\
68	0.00628023672114227\\
69	0.00628021827236698\\
70	0.0062801994861628\\
71	0.00628018035632629\\
72	0.00628016087653928\\
73	0.00628014104036676\\
74	0.00628012084125468\\
75	0.00628010027252782\\
76	0.0062800793273875\\
77	0.00628005799890928\\
78	0.00628003628004066\\
79	0.0062800141635987\\
80	0.0062799916422676\\
81	0.00627996870859623\\
82	0.00627994535499566\\
83	0.00627992157373659\\
84	0.0062798973569467\\
85	0.00627987269660813\\
86	0.00627984758455471\\
87	0.00627982201246914\\
88	0.00627979597188038\\
89	0.00627976945416064\\
90	0.00627974245052258\\
91	0.00627971495201631\\
92	0.0062796869495264\\
93	0.00627965843376886\\
94	0.00627962939528797\\
95	0.00627959982445314\\
96	0.00627956971145567\\
97	0.00627953904630546\\
98	0.0062795078188277\\
99	0.00627947601865933\\
100	0.00627944363524576\\
101	0.00627941065783719\\
102	0.00627937707548512\\
103	0.00627934287703853\\
104	0.00627930805114029\\
105	0.00627927258622331\\
106	0.00627923647050663\\
107	0.00627919969199157\\
108	0.00627916223845761\\
109	0.00627912409745839\\
110	0.00627908525631748\\
111	0.00627904570212421\\
112	0.00627900542172933\\
113	0.00627896440174055\\
114	0.00627892262851818\\
115	0.00627888008817048\\
116	0.00627883676654913\\
117	0.00627879264924439\\
118	0.00627874772158035\\
119	0.00627870196861005\\
120	0.00627865537511043\\
121	0.00627860792557735\\
122	0.00627855960422035\\
123	0.0062785103949574\\
124	0.00627846028140953\\
125	0.00627840924689548\\
126	0.00627835727442593\\
127	0.00627830434669812\\
128	0.00627825044608983\\
129	0.0062781955546537\\
130	0.00627813965411117\\
131	0.00627808272584647\\
132	0.00627802475090034\\
133	0.00627796570996379\\
134	0.00627790558337171\\
135	0.00627784435109629\\
136	0.00627778199274037\\
137	0.0062777184875307\\
138	0.00627765381431106\\
139	0.00627758795153517\\
140	0.00627752087725963\\
141	0.00627745256913659\\
142	0.00627738300440642\\
143	0.00627731215989005\\
144	0.00627724001198138\\
145	0.00627716653663946\\
146	0.00627709170938051\\
147	0.00627701550526985\\
148	0.00627693789891362\\
149	0.00627685886445043\\
150	0.00627677837554275\\
151	0.00627669640536831\\
152	0.0062766129266111\\
153	0.00627652791145255\\
154	0.00627644133156215\\
155	0.00627635315808825\\
156	0.00627626336164851\\
157	0.00627617191232016\\
158	0.00627607877963023\\
159	0.00627598393254549\\
160	0.00627588733946222\\
161	0.00627578896819584\\
162	0.00627568878597031\\
163	0.00627558675940736\\
164	0.0062754828545156\\
165	0.00627537703667921\\
166	0.00627526927064674\\
167	0.00627515952051947\\
168	0.00627504774973969\\
169	0.00627493392107862\\
170	0.00627481799662442\\
171	0.00627469993776958\\
172	0.00627457970519841\\
173	0.00627445725887422\\
174	0.00627433255802615\\
175	0.00627420556113595\\
176	0.00627407622592435\\
177	0.0062739445093374\\
178	0.00627381036753238\\
179	0.0062736737558635\\
180	0.00627353462886746\\
181	0.00627339294024866\\
182	0.00627324864286411\\
183	0.0062731016887082\\
184	0.00627295202889711\\
185	0.00627279961365297\\
186	0.00627264439228783\\
187	0.00627248631318715\\
188	0.00627232532379321\\
189	0.00627216137058811\\
190	0.00627199439907659\\
191	0.00627182435376837\\
192	0.00627165117816039\\
193	0.00627147481471861\\
194	0.0062712952048595\\
195	0.00627111228893133\\
196	0.00627092600619503\\
197	0.00627073629480474\\
198	0.00627054309178807\\
199	0.00627034633302594\\
200	0.00627014595323224\\
201	0.00626994188593301\\
202	0.00626973406344522\\
203	0.00626952241685541\\
204	0.00626930687599775\\
205	0.00626908736943188\\
206	0.00626886382442027\\
207	0.00626863616690529\\
208	0.00626840432148594\\
209	0.00626816821139397\\
210	0.00626792775846993\\
211	0.00626768288313855\\
212	0.00626743350438385\\
213	0.00626717953972388\\
214	0.00626692090518494\\
215	0.00626665751527545\\
216	0.00626638928295935\\
217	0.00626611611962919\\
218	0.00626583793507867\\
219	0.00626555463747476\\
220	0.00626526613332943\\
221	0.00626497232747089\\
222	0.00626467312301442\\
223	0.00626436842133268\\
224	0.00626405812202566\\
225	0.00626374212288999\\
226	0.00626342031988799\\
227	0.00626309260711611\\
228	0.00626275887677293\\
229	0.00626241901912664\\
230	0.00626207292248208\\
231	0.00626172047314722\\
232	0.00626136155539926\\
233	0.00626099605145005\\
234	0.00626062384141118\\
235	0.0062602448032584\\
236	0.00625985881279563\\
237	0.00625946574361841\\
238	0.00625906546707674\\
239	0.00625865785223759\\
240	0.00625824276584659\\
241	0.00625782007228937\\
242	0.00625738963355231\\
243	0.00625695130918266\\
244	0.00625650495624811\\
245	0.00625605042929581\\
246	0.00625558758031067\\
247	0.00625511625867318\\
248	0.0062546363111165\\
249	0.00625414758168291\\
250	0.00625364991167956\\
251	0.00625314313963339\\
252	0.00625262710124551\\
253	0.00625210162934443\\
254	0.0062515665538387\\
255	0.0062510217016683\\
256	0.00625046689675522\\
257	0.00624990195995269\\
258	0.00624932670899319\\
259	0.00624874095843501\\
260	0.0062481445196072\\
261	0.00624753720055269\\
262	0.00624691880596951\\
263	0.00624628913714975\\
264	0.006245647991916\\
265	0.00624499516455529\\
266	0.00624433044575023\\
267	0.00624365362250734\\
268	0.00624296447808276\\
269	0.00624226279190632\\
270	0.00624154833950568\\
271	0.00624082089243628\\
272	0.00624008021823186\\
273	0.00623932608041917\\
274	0.00623855823872793\\
275	0.0062377764498981\\
276	0.00623698047030328\\
277	0.0062361700637432\\
278	0.0062353450193773\\
279	0.00623450512588287\\
280	0.00623365013075074\\
281	0.00623277977116969\\
282	0.00623189378012479\\
283	0.00623099188634747\\
284	0.00623007381426606\\
285	0.00622913928395706\\
286	0.00622818801109727\\
287	0.00622721970691689\\
288	0.00622623407815404\\
289	0.00622523082701093\\
290	0.00622420965111174\\
291	0.00622317024346303\\
292	0.00622211229241692\\
293	0.00622103548163765\\
294	0.0062199394900721\\
295	0.00621882399192518\\
296	0.00621768865664089\\
297	0.00621653314889005\\
298	0.00621535712856596\\
299	0.0062141602507891\\
300	0.00621294216592259\\
301	0.00621170251959943\\
302	0.00621044095276355\\
303	0.00620915710172548\\
304	0.006207850598234\\
305	0.00620652106956347\\
306	0.00620516813861468\\
307	0.00620379142402274\\
308	0.00620239054025357\\
309	0.0062009650976434\\
310	0.00619951470226192\\
311	0.00619803895528635\\
312	0.00619653745105236\\
313	0.00619500977157134\\
314	0.00619345547183289\\
315	0.00619187404272906\\
316	0.00619026483260075\\
317	0.00618862701493381\\
318	0.00618696005667385\\
319	0.0061852634724058\\
320	0.00618353677000882\\
321	0.00618177945060518\\
322	0.00617999100850778\\
323	0.00617817093116535\\
324	0.00617631869910501\\
325	0.00617443378587163\\
326	0.00617251565796289\\
327	0.00617056377475961\\
328	0.00616857758845009\\
329	0.0061665565439479\\
330	0.00616450007880163\\
331	0.00616240762309565\\
332	0.00616027859934056\\
333	0.00615811242235246\\
334	0.0061559084991195\\
335	0.00615366622865523\\
336	0.00615138500183909\\
337	0.00614906420124558\\
338	0.00614670320096608\\
339	0.0061443013664249\\
340	0.00614185805416666\\
341	0.00613937261154164\\
342	0.00613684437664284\\
343	0.00613427267874974\\
344	0.00613165683802625\\
345	0.00612899616513374\\
346	0.00612628996073316\\
347	0.00612353751484298\\
348	0.00612073810601039\\
349	0.00611789100024825\\
350	0.0061149954497102\\
351	0.00611205069110537\\
352	0.00610905594355766\\
353	0.00610601040567376\\
354	0.0061029132518177\\
355	0.00609976362747239\\
356	0.00609656064378319\\
357	0.00609330337230913\\
358	0.00608999084475669\\
359	0.00608662207816113\\
360	0.00608319621868035\\
361	0.00607971329950062\\
362	0.00607617618023874\\
363	0.00607258442884062\\
364	0.00606893764994095\\
365	0.0060652354907542\\
366	0.00606147764795087\\
367	0.00605766387572508\\
368	0.00605379399530891\\
369	0.00604986790625016\\
370	0.00604588559984586\\
371	0.00604184717521095\\
372	0.00603775285855215\\
373	0.00603360302627036\\
374	0.00602939823240385\\
375	0.0060251392402592\\
376	0.0060208270556653\\
377	0.00601646295147004\\
378	0.00601204844818144\\
379	0.00600758513639789\\
380	0.00600307396102361\\
381	0.00599851261440092\\
382	0.00599388544244718\\
383	0.00598919227205787\\
384	0.00598443314402723\\
385	0.0059796081968888\\
386	0.00597471767623049\\
387	0.00596976194436436\\
388	0.00596474149005861\\
389	0.00595965693759411\\
390	0.0059545090529708\\
391	0.00594929875289887\\
392	0.00594402711429341\\
393	0.00593869538304248\\
394	0.00593330498149172\\
395	0.00592785751390258\\
396	0.00592235476888149\\
397	0.00591679871743948\\
398	0.00591119150489946\\
399	0.00590553543428755\\
400	0.0058998329380851\\
401	0.00589408653422213\\
402	0.00588829876088577\\
403	0.00588247208299798\\
404	0.00587660876093922\\
405	0.00587071066903032\\
406	0.00586477904703945\\
407	0.00585881416217398\\
408	0.00585281485363119\\
409	0.00584677790125402\\
410	0.0058406970847465\\
411	0.00583456223492706\\
412	0.0058283578137602\\
413	0.00582206104411124\\
414	0.00581564132950739\\
415	0.00580909611989684\\
416	0.00580242278769309\\
417	0.00579561862140608\\
418	0.00578868081793819\\
419	0.00578160647315369\\
420	0.00577439257025361\\
421	0.00576703596549714\\
422	0.00575953337116851\\
423	0.00575188133728517\\
424	0.00574407623924873\\
425	0.00573611429910769\\
426	0.00572799174826807\\
427	0.00571970501971221\\
428	0.0057112504374799\\
429	0.00570262421128045\\
430	0.00569382243073522\\
431	0.00568484105921746\\
432	0.005675675927249\\
433	0.00566632272540006\\
434	0.00565677699666341\\
435	0.00564703412839972\\
436	0.00563708934390052\\
437	0.00562693769319163\\
438	0.0056165740425122\\
439	0.00560599306326688\\
440	0.00559518922013013\\
441	0.00558415675807559\\
442	0.00557288968817381\\
443	0.00556138177202994\\
444	0.00554962650456028\\
445	0.00553761709472205\\
446	0.00552534644538259\\
447	0.00551280713336373\\
448	0.00549999138768484\\
449	0.0054868910588211\\
450	0.00547349758831802\\
451	0.00545980198065547\\
452	0.0054457947712163\\
453	0.00543146599114788\\
454	0.00541680513034979\\
455	0.00540180109703208\\
456	0.00538644216815536\\
457	0.00537071598176243\\
458	0.00535460946385435\\
459	0.00533810875409222\\
460	0.00532119911108202\\
461	0.00530386483378619\\
462	0.00528608918842895\\
463	0.00526785433637298\\
464	0.00524914126685343\\
465	0.00522992969894428\\
466	0.00521019785529346\\
467	0.00518992263385337\\
468	0.0051690801278424\\
469	0.00514764554730165\\
470	0.00512559328073365\\
471	0.00510289727727141\\
472	0.00507953175229053\\
473	0.005055469974674\\
474	0.00503068410002096\\
475	0.00500514388755872\\
476	0.00497881545852904\\
477	0.00495166159305947\\
478	0.00492364845257451\\
479	0.00489474256434685\\
480	0.00486491000694632\\
481	0.00483411653850486\\
482	0.00480232771466033\\
483	0.004769508966389\\
484	0.00473562561016074\\
485	0.00470064279906671\\
486	0.00466452525591064\\
487	0.00462723672678512\\
488	0.00458873886863503\\
489	0.00454898918719202\\
490	0.00450793510723609\\
491	0.00446551549812167\\
492	0.00442167582343966\\
493	0.0043763724822605\\
494	0.00432957844475902\\
495	0.00428129072677211\\
496	0.00423154041810466\\
497	0.0041804059763298\\
498	0.00412803078363405\\
499	0.0040746463120854\\
500	0.00402060285304561\\
501	0.00396743933357704\\
502	0.00391556399806644\\
503	0.00386515616431572\\
504	0.00381640428098737\\
505	0.00376949956777484\\
506	0.00372462926229314\\
507	0.00368196700638546\\
508	0.0036416591034706\\
509	0.00360380526237516\\
510	0.00356843183757421\\
511	0.00353545553400809\\
512	0.00350392253949669\\
513	0.00347300447686475\\
514	0.00344269685930694\\
515	0.00341298179059693\\
516	0.00338382592998904\\
517	0.00335517866997427\\
518	0.00332697080439069\\
519	0.00329911409782002\\
520	0.00327150234383665\\
521	0.0032440147440854\\
522	0.00321652281051031\\
523	0.00318892662393476\\
524	0.00316120027083114\\
525	0.00313331491948637\\
526	0.0031052391674743\\
527	0.00307693956129006\\
528	0.00304838130493532\\
529	0.00301952916079463\\
530	0.00299034852332022\\
531	0.00296080660805138\\
532	0.00293087363594622\\
533	0.00290052379454109\\
534	0.00286973531170303\\
535	0.00283848542659518\\
536	0.00280675046465217\\
537	0.00277450591734095\\
538	0.00274172652008376\\
539	0.00270838631990362\\
540	0.00267445872297293\\
541	0.00263991651194983\\
542	0.00260473182461684\\
543	0.00256887609111131\\
544	0.00253231993928337\\
545	0.00249503310058525\\
546	0.00245698458609395\\
547	0.00241814303104803\\
548	0.00237847717142211\\
549	0.002337956497622\\
550	0.00229655183905641\\
551	0.00225424597099964\\
552	0.00221102370153545\\
553	0.00216687040382646\\
554	0.00212177211836048\\
555	0.00207571568159297\\
556	0.002028688891195\\
557	0.00198068071385247\\
558	0.00193168153175534\\
559	0.00188168342228085\\
560	0.00183068044001795\\
561	0.00177866892288609\\
562	0.00172564781376172\\
563	0.00167161948272084\\
564	0.00161659226410366\\
565	0.00156058298271428\\
566	0.00150362990328009\\
567	0.00144583137164707\\
568	0.00138714982852563\\
569	0.00132751977693301\\
570	0.00126683277350473\\
571	0.00120492049445878\\
572	0.00114382947878174\\
573	0.00108426163518732\\
574	0.00102666342567682\\
575	0.000971846090115743\\
576	0.000918265947001419\\
577	0.000866422173777079\\
578	0.000816462067759592\\
579	0.000767915527104241\\
580	0.000720194215446356\\
581	0.000673645505827697\\
582	0.000628150962911244\\
583	0.000583436223402909\\
584	0.000539226991113701\\
585	0.000495475690639145\\
586	0.000452196148255144\\
587	0.000409418505768188\\
588	0.000367181587116058\\
589	0.000325525791425008\\
590	0.000284484611477045\\
591	0.000244081548911079\\
592	0.000204328087251261\\
593	0.000165233217676711\\
594	0.000126870159937007\\
595	8.96111722547522e-05\\
596	5.42660945238508e-05\\
597	2.28062284332056e-05\\
598	2.9204464504877e-07\\
599	0\\
600	0\\
};
\addplot [color=red!40!mycolor19,solid,forget plot]
  table[row sep=crcr]{%
1	0.00609949007723919\\
2	0.00609948663958327\\
3	0.00609948313907455\\
4	0.00609947957456534\\
5	0.00609947594488679\\
6	0.00609947224884837\\
7	0.00609946848523761\\
8	0.00609946465281958\\
9	0.00609946075033646\\
10	0.00609945677650724\\
11	0.00609945273002708\\
12	0.00609944860956702\\
13	0.00609944441377342\\
14	0.00609944014126752\\
15	0.00609943579064496\\
16	0.00609943136047536\\
17	0.00609942684930172\\
18	0.00609942225563992\\
19	0.00609941757797826\\
20	0.00609941281477694\\
21	0.00609940796446747\\
22	0.00609940302545211\\
23	0.00609939799610336\\
24	0.00609939287476341\\
25	0.00609938765974349\\
26	0.00609938234932327\\
27	0.00609937694175034\\
28	0.00609937143523951\\
29	0.00609936582797223\\
30	0.00609936011809597\\
31	0.00609935430372347\\
32	0.00609934838293216\\
33	0.00609934235376347\\
34	0.0060993362142221\\
35	0.00609932996227535\\
36	0.00609932359585234\\
37	0.00609931711284339\\
38	0.00609931051109918\\
39	0.00609930378842994\\
40	0.00609929694260481\\
41	0.00609928997135099\\
42	0.00609928287235284\\
43	0.00609927564325116\\
44	0.00609926828164235\\
45	0.00609926078507749\\
46	0.00609925315106148\\
47	0.00609924537705219\\
48	0.0060992374604595\\
49	0.00609922939864443\\
50	0.0060992211889181\\
51	0.0060992128285409\\
52	0.00609920431472136\\
53	0.00609919564461529\\
54	0.00609918681532465\\
55	0.00609917782389656\\
56	0.00609916866732222\\
57	0.00609915934253584\\
58	0.00609914984641353\\
59	0.00609914017577214\\
60	0.00609913032736824\\
61	0.00609912029789677\\
62	0.00609911008399001\\
63	0.00609909968221624\\
64	0.00609908908907858\\
65	0.0060990783010137\\
66	0.00609906731439054\\
67	0.00609905612550898\\
68	0.00609904473059852\\
69	0.00609903312581693\\
70	0.00609902130724886\\
71	0.00609900927090432\\
72	0.00609899701271745\\
73	0.00609898452854482\\
74	0.00609897181416409\\
75	0.00609895886527241\\
76	0.00609894567748484\\
77	0.00609893224633285\\
78	0.00609891856726264\\
79	0.00609890463563345\\
80	0.00609889044671596\\
81	0.00609887599569055\\
82	0.00609886127764551\\
83	0.00609884628757535\\
84	0.00609883102037887\\
85	0.00609881547085742\\
86	0.00609879963371292\\
87	0.00609878350354604\\
88	0.00609876707485416\\
89	0.00609875034202935\\
90	0.00609873329935646\\
91	0.00609871594101096\\
92	0.00609869826105684\\
93	0.00609868025344447\\
94	0.00609866191200839\\
95	0.00609864323046511\\
96	0.00609862420241078\\
97	0.00609860482131893\\
98	0.00609858508053802\\
99	0.00609856497328922\\
100	0.00609854449266368\\
101	0.00609852363162027\\
102	0.00609850238298286\\
103	0.00609848073943784\\
104	0.00609845869353143\\
105	0.00609843623766696\\
106	0.00609841336410217\\
107	0.00609839006494641\\
108	0.0060983663321577\\
109	0.00609834215753996\\
110	0.00609831753273993\\
111	0.00609829244924423\\
112	0.00609826689837624\\
113	0.00609824087129303\\
114	0.00609821435898207\\
115	0.0060981873522581\\
116	0.00609815984175969\\
117	0.00609813181794595\\
118	0.00609810327109312\\
119	0.00609807419129092\\
120	0.00609804456843922\\
121	0.00609801439224414\\
122	0.00609798365221457\\
123	0.00609795233765821\\
124	0.00609792043767789\\
125	0.00609788794116754\\
126	0.00609785483680825\\
127	0.00609782111306418\\
128	0.00609778675817847\\
129	0.00609775176016894\\
130	0.00609771610682384\\
131	0.00609767978569751\\
132	0.00609764278410584\\
133	0.00609760508912176\\
134	0.00609756668757063\\
135	0.00609752756602546\\
136	0.00609748771080222\\
137	0.00609744710795483\\
138	0.00609740574327021\\
139	0.00609736360226324\\
140	0.00609732067017151\\
141	0.00609727693195015\\
142	0.00609723237226632\\
143	0.00609718697549387\\
144	0.00609714072570772\\
145	0.00609709360667812\\
146	0.00609704560186496\\
147	0.00609699669441184\\
148	0.00609694686714002\\
149	0.00609689610254236\\
150	0.0060968443827771\\
151	0.00609679168966141\\
152	0.00609673800466514\\
153	0.00609668330890394\\
154	0.00609662758313276\\
155	0.00609657080773896\\
156	0.00609651296273535\\
157	0.00609645402775305\\
158	0.00609639398203434\\
159	0.00609633280442522\\
160	0.00609627047336795\\
161	0.00609620696689341\\
162	0.0060961422626133\\
163	0.00609607633771221\\
164	0.00609600916893961\\
165	0.0060959407326015\\
166	0.00609587100455211\\
167	0.00609579996018535\\
168	0.00609572757442607\\
169	0.0060956538217213\\
170	0.00609557867603111\\
171	0.00609550211081947\\
172	0.00609542409904487\\
173	0.00609534461315076\\
174	0.00609526362505591\\
175	0.00609518110614439\\
176	0.00609509702725557\\
177	0.00609501135867384\\
178	0.00609492407011807\\
179	0.00609483513073107\\
180	0.00609474450906861\\
181	0.00609465217308845\\
182	0.00609455809013899\\
183	0.00609446222694786\\
184	0.00609436454961022\\
185	0.00609426502357682\\
186	0.00609416361364184\\
187	0.00609406028393064\\
188	0.00609395499788708\\
189	0.0060938477182607\\
190	0.00609373840709372\\
191	0.00609362702570769\\
192	0.00609351353468993\\
193	0.00609339789387978\\
194	0.00609328006235448\\
195	0.00609315999841487\\
196	0.00609303765957081\\
197	0.00609291300252628\\
198	0.0060927859831643\\
199	0.00609265655653147\\
200	0.00609252467682226\\
201	0.00609239029736297\\
202	0.00609225337059552\\
203	0.00609211384806071\\
204	0.00609197168038146\\
205	0.00609182681724539\\
206	0.00609167920738737\\
207	0.00609152879857161\\
208	0.00609137553757333\\
209	0.00609121937016029\\
210	0.00609106024107379\\
211	0.00609089809400936\\
212	0.00609073287159712\\
213	0.0060905645153817\\
214	0.00609039296580175\\
215	0.00609021816216924\\
216	0.00609004004264809\\
217	0.00608985854423248\\
218	0.00608967360272485\\
219	0.00608948515271332\\
220	0.0060892931275486\\
221	0.00608909745932065\\
222	0.0060888980788347\\
223	0.00608869491558677\\
224	0.00608848789773869\\
225	0.00608827695209266\\
226	0.0060880620040652\\
227	0.00608784297766047\\
228	0.00608761979544311\\
229	0.00608739237851043\\
230	0.00608716064646387\\
231	0.0060869245173799\\
232	0.00608668390778015\\
233	0.00608643873260086\\
234	0.00608618890516139\\
235	0.00608593433713219\\
236	0.00608567493850159\\
237	0.00608541061754193\\
238	0.00608514128077454\\
239	0.00608486683293366\\
240	0.00608458717692941\\
241	0.00608430221380937\\
242	0.00608401184271884\\
243	0.0060837159608598\\
244	0.00608341446344822\\
245	0.00608310724366962\\
246	0.00608279419263279\\
247	0.00608247519932137\\
248	0.00608215015054305\\
249	0.00608181893087611\\
250	0.00608148142261296\\
251	0.00608113750570028\\
252	0.0060807870576751\\
253	0.00608042995359647\\
254	0.00608006606597174\\
255	0.00607969526467671\\
256	0.00607931741686841\\
257	0.00607893238688942\\
258	0.00607854003616171\\
259	0.00607814022306827\\
260	0.00607773280281994\\
261	0.00607731762730415\\
262	0.00607689454491185\\
263	0.00607646340033768\\
264	0.00607602403434751\\
265	0.00607557628350515\\
266	0.00607511997984932\\
267	0.00607465495050848\\
268	0.0060741810172383\\
269	0.00607369799586306\\
270	0.00607320569559741\\
271	0.00607270391822066\\
272	0.00607219245707472\\
273	0.00607167109586992\\
274	0.00607113960735399\\
275	0.00607059775219016\\
276	0.00607004527956652\\
277	0.00606948193612833\\
278	0.0060689075145937\\
279	0.00606832212225923\\
280	0.00606772596503167\\
281	0.00606711885765475\\
282	0.00606650061256708\\
283	0.00606587103995243\\
284	0.00606522994780081\\
285	0.00606457714198132\\
286	0.00606391242632902\\
287	0.00606323560274758\\
288	0.00606254647133033\\
289	0.00606184483050241\\
290	0.00606113047718779\\
291	0.00606040320700509\\
292	0.00605966281449736\\
293	0.00605890909340178\\
294	0.00605814183696667\\
295	0.0060573608383249\\
296	0.00605656589093458\\
297	0.0060557567891005\\
298	0.00605493332859318\\
299	0.00605409530738581\\
300	0.00605324252653497\\
301	0.0060523747912365\\
302	0.00605149191209669\\
303	0.0060505937066678\\
304	0.00604968000131025\\
305	0.00604875063345861\\
306	0.006047805454388\\
307	0.00604684433259849\\
308	0.00604586715795675\\
309	0.00604487384674168\\
310	0.00604386434769657\\
311	0.0060428386489718\\
312	0.00604179678508451\\
313	0.00604073884058972\\
314	0.00603966493946368\\
315	0.00603857518426247\\
316	0.00603746942339006\\
317	0.00603634639281748\\
318	0.00603520227378819\\
319	0.00603403668044389\\
320	0.00603284922021106\\
321	0.00603163949369277\\
322	0.00603040709455885\\
323	0.00602915160943468\\
324	0.00602787261778861\\
325	0.00602656969181786\\
326	0.00602524239633337\\
327	0.00602389028864327\\
328	0.00602251291843567\\
329	0.00602110982766015\\
330	0.00601968055040881\\
331	0.00601822461279687\\
332	0.00601674153284308\\
333	0.00601523082035019\\
334	0.00601369197678595\\
335	0.00601212449516497\\
336	0.006010527859932\\
337	0.00600890154684694\\
338	0.00600724502287205\\
339	0.00600555774606172\\
340	0.00600383916545539\\
341	0.00600208872097596\\
342	0.00600030584332892\\
343	0.0059984899538931\\
344	0.00599664046461557\\
345	0.00599475677791284\\
346	0.00599283828658081\\
347	0.00599088437371715\\
348	0.00598889441266128\\
349	0.00598686776695855\\
350	0.00598480379035758\\
351	0.00598270182685029\\
352	0.00598056121077298\\
353	0.00597838126699589\\
354	0.00597616131123422\\
355	0.00597390065052333\\
356	0.00597159858391212\\
357	0.00596925440342943\\
358	0.00596686739533017\\
359	0.00596443684136008\\
360	0.00596196201846727\\
361	0.00595944218832653\\
362	0.00595687655876742\\
363	0.00595426432285699\\
364	0.00595160465835615\\
365	0.00594889672710658\\
366	0.00594613967433101\\
367	0.00594333262782399\\
368	0.00594047469700265\\
369	0.00593756497177677\\
370	0.00593460252118352\\
371	0.00593158639171427\\
372	0.00592851560523585\\
373	0.00592538915637774\\
374	0.00592220600921807\\
375	0.00591896509306117\\
376	0.00591566529708073\\
377	0.00591230546367939\\
378	0.00590888438083942\\
379	0.00590540077532086\\
380	0.00590185331409043\\
381	0.0058982406413478\\
382	0.0058945615607665\\
383	0.00589081484917345\\
384	0.00588699925315538\\
385	0.00588311348689868\\
386	0.00587915622972221\\
387	0.00587512612325945\\
388	0.00587102176824432\\
389	0.00586684172085428\\
390	0.00586258448862514\\
391	0.00585824852581624\\
392	0.00585383222813431\\
393	0.0058493339267348\\
394	0.00584475188141529\\
395	0.00584008427291369\\
396	0.00583532919422693\\
397	0.00583048464087574\\
398	0.00582554850005957\\
399	0.00582051853867892\\
400	0.00581539239025816\\
401	0.00581016754088863\\
402	0.00580484131444731\\
403	0.00579941085753478\\
404	0.00579387312485534\\
405	0.0057882248661641\\
406	0.00578246261647293\\
407	0.00577658269202535\\
408	0.00577058119559946\\
409	0.0057644540365423\\
410	0.00575819697446965\\
411	0.0057518056932042\\
412	0.00574527592104245\\
413	0.00573860362103671\\
414	0.00573178525222442\\
415	0.00572481715987743\\
416	0.00571769557054584\\
417	0.00571041658686079\\
418	0.00570297618209032\\
419	0.00569537019445206\\
420	0.00568759432119414\\
421	0.00567964411248218\\
422	0.00567151496514972\\
423	0.0056632021163124\\
424	0.00565470063701057\\
425	0.00564600542546769\\
426	0.0056371111976312\\
427	0.00562801247359913\\
428	0.00561870356763885\\
429	0.00560917857772503\\
430	0.00559943137445791\\
431	0.00558945558930404\\
432	0.0055792446020997\\
433	0.00556879152770817\\
434	0.00555808920160435\\
435	0.00554713016404182\\
436	0.00553590664433304\\
437	0.00552441054698277\\
438	0.00551263343706816\\
439	0.00550056651711529\\
440	0.00548820060611104\\
441	0.00547552611821937\\
442	0.0054625330403664\\
443	0.00544921090865481\\
444	0.00543554878338468\\
445	0.00542153522134404\\
446	0.00540715823967644\\
447	0.00539240528487763\\
448	0.00537726322626625\\
449	0.00536171836902219\\
450	0.00534575638316665\\
451	0.00532936225790829\\
452	0.00531252026494597\\
453	0.00529521391674586\\
454	0.00527742591865204\\
455	0.00525913811865879\\
456	0.00524033143738105\\
457	0.00522098557186356\\
458	0.00520107913309478\\
459	0.00518059002488525\\
460	0.00515949530239541\\
461	0.00513777091349513\\
462	0.00511539163234859\\
463	0.00509233104181511\\
464	0.00506856155377677\\
465	0.00504405459075588\\
466	0.00501878070847056\\
467	0.00499270703806631\\
468	0.00496579830873763\\
469	0.00493802272386192\\
470	0.00490934976632691\\
471	0.00487974891411962\\
472	0.00484918962606995\\
473	0.0048176412409891\\
474	0.00478507269140209\\
475	0.00475145198857918\\
476	0.00471674534696897\\
477	0.00468091556270191\\
478	0.00464391755227902\\
479	0.00460569392592616\\
480	0.00456619201277347\\
481	0.00452536715465093\\
482	0.00448318728872223\\
483	0.00443963912314147\\
484	0.00439473634050376\\
485	0.00434853028216572\\
486	0.00430112419214191\\
487	0.00425269202561763\\
488	0.00420350323911374\\
489	0.00415437121824339\\
490	0.00410637070479002\\
491	0.00405966503355721\\
492	0.00401442487833185\\
493	0.00397082558574463\\
494	0.00392904397100763\\
495	0.00388925106606612\\
496	0.00385160133021408\\
497	0.00381621843275766\\
498	0.00378317542947194\\
499	0.00375246742237028\\
500	0.00372397408550459\\
501	0.00369632046858735\\
502	0.00366926832203856\\
503	0.00364280960574355\\
504	0.00361692350771421\\
505	0.00359157468192518\\
506	0.00356671175220235\\
507	0.00354226635502932\\
508	0.00351815313048391\\
509	0.00349427124709755\\
510	0.0034705082972028\\
511	0.00344674770856446\\
512	0.00342292178581892\\
513	0.00339900788620017\\
514	0.00337498108408554\\
515	0.00335081455714488\\
516	0.00332648013014735\\
517	0.00330194898684415\\
518	0.00327719254604736\\
519	0.00325218347362656\\
520	0.00322689676323357\\
521	0.00320131075848223\\
522	0.00317540789701119\\
523	0.00314917367059463\\
524	0.00312259320271672\\
525	0.00309565133254662\\
526	0.00306833269407408\\
527	0.00304062178171857\\
528	0.00301250299167785\\
529	0.0029839606266123\\
530	0.00295497885052935\\
531	0.00292554158190216\\
532	0.00289563231772567\\
533	0.00286523389183085\\
534	0.00283432820493941\\
535	0.00280289616912907\\
536	0.00277091764986146\\
537	0.00273837140739797\\
538	0.00270523504060233\\
539	0.00267148493771352\\
540	0.00263709624074912\\
541	0.00260204283283811\\
542	0.00256629736102423\\
543	0.0025298313108894\\
544	0.00249261515356304\\
545	0.00245461858992128\\
546	0.00241581091084629\\
547	0.00237616150442382\\
548	0.00233564055112879\\
549	0.00229421974472738\\
550	0.00225188413966668\\
551	0.00220861946225582\\
552	0.00216441225788334\\
553	0.00211925011293704\\
554	0.00207312192622846\\
555	0.00202601822576494\\
556	0.00197793150135645\\
557	0.00192885657953965\\
558	0.00187879103415134\\
559	0.00182773633444033\\
560	0.00177570121612501\\
561	0.00172270419382382\\
562	0.00166883078065186\\
563	0.00161408765059616\\
564	0.00155842221278116\\
565	0.00150176416055747\\
566	0.00144398692427377\\
567	0.00138484423759802\\
568	0.00132527813048257\\
569	0.00126704480548041\\
570	0.00121055312007744\\
571	0.00115653787994659\\
572	0.00110330586081578\\
573	0.00105101228541044\\
574	0.00100035963156781\\
575	0.000951144518913378\\
576	0.000902948852799658\\
577	0.000855365674177795\\
578	0.000808355188236387\\
579	0.000762216035417117\\
580	0.00071690457967\\
581	0.00067198153227095\\
582	0.000627351837523727\\
583	0.000583007319138113\\
584	0.000538986303744318\\
585	0.000495338379819224\\
586	0.000452116120303638\\
587	0.000409372268862694\\
588	0.000367156964357134\\
589	0.000325514548405105\\
590	0.000284480845955471\\
591	0.000244080684870838\\
592	0.000204328087251261\\
593	0.000165233217676712\\
594	0.000126870159937008\\
595	8.96111722547531e-05\\
596	5.42660945238511e-05\\
597	2.28062284332057e-05\\
598	2.9204464504877e-07\\
599	0\\
600	0\\
};
\addplot [color=red!75!mycolor17,solid,forget plot]
  table[row sep=crcr]{%
1	0.0060542548985145\\
2	0.00605425199370972\\
3	0.00605424903553986\\
4	0.00605424602303839\\
5	0.00605424295522111\\
6	0.006054239831086\\
7	0.00605423664961272\\
8	0.00605423340976236\\
9	0.00605423011047707\\
10	0.00605422675067964\\
11	0.00605422332927326\\
12	0.00605421984514103\\
13	0.00605421629714559\\
14	0.00605421268412883\\
15	0.00605420900491139\\
16	0.00605420525829226\\
17	0.00605420144304841\\
18	0.00605419755793437\\
19	0.00605419360168178\\
20	0.00605418957299889\\
21	0.00605418547057024\\
22	0.00605418129305609\\
23	0.00605417703909203\\
24	0.00605417270728841\\
25	0.00605416829622995\\
26	0.00605416380447523\\
27	0.00605415923055607\\
28	0.00605415457297718\\
29	0.0060541498302155\\
30	0.00605414500071974\\
31	0.00605414008290975\\
32	0.00605413507517608\\
33	0.00605412997587927\\
34	0.00605412478334934\\
35	0.00605411949588519\\
36	0.00605411411175401\\
37	0.00605410862919055\\
38	0.0060541030463966\\
39	0.00605409736154033\\
40	0.00605409157275556\\
41	0.00605408567814111\\
42	0.00605407967576014\\
43	0.00605407356363946\\
44	0.00605406733976871\\
45	0.00605406100209975\\
46	0.00605405454854584\\
47	0.00605404797698082\\
48	0.0060540412852385\\
49	0.00605403447111171\\
50	0.00605402753235154\\
51	0.0060540204666665\\
52	0.00605401327172166\\
53	0.00605400594513783\\
54	0.00605399848449061\\
55	0.00605399088730956\\
56	0.00605398315107722\\
57	0.0060539752732282\\
58	0.00605396725114822\\
59	0.00605395908217315\\
60	0.00605395076358791\\
61	0.00605394229262556\\
62	0.00605393366646624\\
63	0.00605392488223606\\
64	0.00605391593700604\\
65	0.00605390682779104\\
66	0.00605389755154856\\
67	0.00605388810517763\\
68	0.00605387848551762\\
69	0.00605386868934704\\
70	0.00605385871338232\\
71	0.00605384855427655\\
72	0.00605383820861819\\
73	0.00605382767292976\\
74	0.00605381694366659\\
75	0.00605380601721534\\
76	0.00605379488989271\\
77	0.00605378355794394\\
78	0.00605377201754145\\
79	0.00605376026478337\\
80	0.00605374829569191\\
81	0.00605373610621198\\
82	0.00605372369220956\\
83	0.00605371104947008\\
84	0.0060536981736968\\
85	0.00605368506050917\\
86	0.00605367170544116\\
87	0.00605365810393945\\
88	0.00605364425136173\\
89	0.00605363014297491\\
90	0.00605361577395319\\
91	0.00605360113937629\\
92	0.00605358623422746\\
93	0.00605357105339158\\
94	0.00605355559165315\\
95	0.00605353984369429\\
96	0.00605352380409261\\
97	0.00605350746731915\\
98	0.00605349082773621\\
99	0.00605347387959511\\
100	0.00605345661703403\\
101	0.00605343903407568\\
102	0.00605342112462494\\
103	0.0060534028824666\\
104	0.00605338430126275\\
105	0.00605336537455043\\
106	0.00605334609573917\\
107	0.00605332645810821\\
108	0.00605330645480415\\
109	0.00605328607883798\\
110	0.00605326532308261\\
111	0.00605324418026994\\
112	0.00605322264298808\\
113	0.00605320070367846\\
114	0.00605317835463289\\
115	0.00605315558799049\\
116	0.00605313239573474\\
117	0.00605310876969023\\
118	0.00605308470151957\\
119	0.00605306018272017\\
120	0.00605303520462078\\
121	0.00605300975837828\\
122	0.00605298383497411\\
123	0.00605295742521092\\
124	0.00605293051970883\\
125	0.00605290310890187\\
126	0.00605287518303431\\
127	0.00605284673215675\\
128	0.00605281774612234\\
129	0.00605278821458291\\
130	0.00605275812698484\\
131	0.00605272747256499\\
132	0.00605269624034661\\
133	0.00605266441913503\\
134	0.00605263199751339\\
135	0.00605259896383815\\
136	0.00605256530623462\\
137	0.00605253101259243\\
138	0.00605249607056078\\
139	0.00605246046754374\\
140	0.00605242419069537\\
141	0.00605238722691476\\
142	0.006052349562841\\
143	0.00605231118484809\\
144	0.00605227207903958\\
145	0.00605223223124333\\
146	0.00605219162700609\\
147	0.00605215025158783\\
148	0.00605210808995623\\
149	0.00605206512678079\\
150	0.00605202134642706\\
151	0.00605197673295059\\
152	0.00605193127009082\\
153	0.0060518849412649\\
154	0.00605183772956136\\
155	0.00605178961773358\\
156	0.00605174058819324\\
157	0.00605169062300365\\
158	0.00605163970387276\\
159	0.00605158781214639\\
160	0.00605153492880093\\
161	0.00605148103443623\\
162	0.00605142610926808\\
163	0.00605137013312077\\
164	0.00605131308541933\\
165	0.00605125494518182\\
166	0.00605119569101119\\
167	0.00605113530108726\\
168	0.00605107375315838\\
169	0.00605101102453295\\
170	0.00605094709207084\\
171	0.00605088193217454\\
172	0.00605081552078026\\
173	0.00605074783334872\\
174	0.00605067884485589\\
175	0.00605060852978345\\
176	0.00605053686210912\\
177	0.00605046381529679\\
178	0.00605038936228641\\
179	0.00605031347548376\\
180	0.00605023612674995\\
181	0.00605015728739076\\
182	0.00605007692814574\\
183	0.00604999501917712\\
184	0.0060499115300585\\
185	0.00604982642976324\\
186	0.00604973968665278\\
187	0.00604965126846458\\
188	0.00604956114229992\\
189	0.00604946927461135\\
190	0.00604937563119001\\
191	0.00604928017715268\\
192	0.00604918287692849\\
193	0.00604908369424541\\
194	0.0060489825921166\\
195	0.00604887953282627\\
196	0.00604877447791537\\
197	0.00604866738816707\\
198	0.0060485582235918\\
199	0.00604844694341212\\
200	0.00604833350604725\\
201	0.00604821786909728\\
202	0.00604809998932707\\
203	0.00604797982264992\\
204	0.00604785732411068\\
205	0.00604773244786887\\
206	0.0060476051471812\\
207	0.00604747537438382\\
208	0.00604734308087434\\
209	0.00604720821709327\\
210	0.00604707073250529\\
211	0.00604693057558015\\
212	0.006046787693773\\
213	0.00604664203350458\\
214	0.00604649354014091\\
215	0.0060463421579725\\
216	0.00604618783019338\\
217	0.00604603049887952\\
218	0.00604587010496691\\
219	0.00604570658822923\\
220	0.00604553988725512\\
221	0.00604536993942488\\
222	0.00604519668088687\\
223	0.0060450200465334\\
224	0.00604483996997613\\
225	0.00604465638352118\\
226	0.00604446921814338\\
227	0.00604427840346052\\
228	0.00604408386770679\\
229	0.00604388553770576\\
230	0.00604368333884301\\
231	0.00604347719503816\\
232	0.00604326702871631\\
233	0.00604305276077907\\
234	0.00604283431057514\\
235	0.00604261159587007\\
236	0.0060423845328158\\
237	0.00604215303591947\\
238	0.00604191701801181\\
239	0.00604167639021485\\
240	0.00604143106190916\\
241	0.00604118094070057\\
242	0.00604092593238629\\
243	0.00604066594092043\\
244	0.00604040086837903\\
245	0.00604013061492462\\
246	0.00603985507877001\\
247	0.00603957415614173\\
248	0.00603928774124293\\
249	0.00603899572621561\\
250	0.00603869800110246\\
251	0.00603839445380817\\
252	0.00603808497006022\\
253	0.00603776943336931\\
254	0.00603744772498924\\
255	0.00603711972387649\\
256	0.00603678530664946\\
257	0.00603644434754738\\
258	0.00603609671838921\\
259	0.00603574228853227\\
260	0.00603538092483121\\
261	0.00603501249159726\\
262	0.00603463685055806\\
263	0.00603425386081876\\
264	0.00603386337882462\\
265	0.00603346525832618\\
266	0.00603305935034801\\
267	0.00603264550316234\\
268	0.00603222356226987\\
269	0.00603179337039027\\
270	0.00603135476746624\\
271	0.0060309075906859\\
272	0.00603045167453047\\
273	0.00602998685085563\\
274	0.00602951294901667\\
275	0.00602902979604564\\
276	0.0060285372168711\\
277	0.00602803503448028\\
278	0.00602752306947831\\
279	0.00602700113593192\\
280	0.00602646903952961\\
281	0.0060259265821394\\
282	0.00602537356172687\\
283	0.00602480977227086\\
284	0.00602423500367617\\
285	0.00602364904168356\\
286	0.00602305166777651\\
287	0.00602244265908472\\
288	0.00602182178828396\\
289	0.00602118882349233\\
290	0.00602054352816206\\
291	0.00601988566096721\\
292	0.00601921497568633\\
293	0.00601853122107978\\
294	0.00601783414076147\\
295	0.00601712347306377\\
296	0.00601639895089535\\
297	0.00601566030159045\\
298	0.0060149072467486\\
299	0.00601413950206287\\
300	0.00601335677713433\\
301	0.00601255877527002\\
302	0.00601174519325995\\
303	0.00601091572112823\\
304	0.00601007004185066\\
305	0.00600920783102938\\
306	0.00600832875651105\\
307	0.00600743247793097\\
308	0.00600651864615927\\
309	0.00600558690261738\\
310	0.00600463687842416\\
311	0.0060036681933215\\
312	0.00600268045432796\\
313	0.00600167325409898\\
314	0.00600064616910949\\
315	0.00599959875827926\\
316	0.00599853056443055\\
317	0.00599744112762684\\
318	0.00599633002043226\\
319	0.00599519680657051\\
320	0.0059940410407279\\
321	0.005992862268351\\
322	0.00599166002543922\\
323	0.00599043383833153\\
324	0.00598918322348752\\
325	0.00598790768726258\\
326	0.00598660672567669\\
327	0.005985279824177\\
328	0.00598392645739348\\
329	0.00598254608888788\\
330	0.00598113817089524\\
331	0.0059797021440581\\
332	0.00597823743715262\\
333	0.00597674346680675\\
334	0.00597521963720975\\
335	0.00597366533981282\\
336	0.0059720799530204\\
337	0.00597046284187164\\
338	0.00596881335771161\\
339	0.00596713083785189\\
340	0.00596541460521975\\
341	0.00596366396799539\\
342	0.00596187821923701\\
343	0.00596005663649259\\
344	0.00595819848139854\\
345	0.00595630299926424\\
346	0.00595436941864167\\
347	0.00595239695087973\\
348	0.00595038478966199\\
349	0.00594833211052729\\
350	0.00594623807037185\\
351	0.00594410180693178\\
352	0.0059419224382442\\
353	0.00593969906208525\\
354	0.00593743075538152\\
355	0.00593511657359166\\
356	0.00593275555005242\\
357	0.00593034669528196\\
358	0.0059278889962312\\
359	0.0059253814154743\\
360	0.00592282289034598\\
361	0.00592021233215843\\
362	0.00591754862594789\\
363	0.00591483062968437\\
364	0.00591205717345138\\
365	0.00590922705859407\\
366	0.0059063390568347\\
367	0.00590339190935374\\
368	0.00590038432583566\\
369	0.00589731498347839\\
370	0.00589418252596617\\
371	0.0058909855624063\\
372	0.0058877226662318\\
373	0.0058843923740743\\
374	0.00588099318461349\\
375	0.0058775235574158\\
376	0.00587398191177944\\
377	0.00587036662561085\\
378	0.00586667603436179\\
379	0.00586290843004274\\
380	0.00585906206024255\\
381	0.00585513512668948\\
382	0.00585112578103346\\
383	0.00584703212292182\\
384	0.0058428521980019\\
385	0.00583858399582633\\
386	0.00583422544765775\\
387	0.00582977442416987\\
388	0.00582522873304119\\
389	0.00582058611644157\\
390	0.00581584424840936\\
391	0.00581100073211716\\
392	0.0058060530970288\\
393	0.00580099879595051\\
394	0.00579583520198189\\
395	0.00579055960537552\\
396	0.00578516921031676\\
397	0.00577966113164156\\
398	0.00577403239152266\\
399	0.0057682799161579\\
400	0.00576240053247438\\
401	0.00575639096487995\\
402	0.00575024783207291\\
403	0.00574396764398991\\
404	0.00573754679892206\\
405	0.00573098158081809\\
406	0.00572426815680069\\
407	0.00571740257477954\\
408	0.0057103807610439\\
409	0.00570319851758423\\
410	0.00569585151864958\\
411	0.00568833530593727\\
412	0.00568064528179326\\
413	0.00567277669913962\\
414	0.00566472464014436\\
415	0.00565648400799689\\
416	0.00564804951792851\\
417	0.00563941568793487\\
418	0.00563057682899589\\
419	0.0056215270346997\\
420	0.00561226017019854\\
421	0.00560276986037648\\
422	0.00559304947721987\\
423	0.00558309212656227\\
424	0.00557289063323799\\
425	0.00556243752687089\\
426	0.00555172502862437\\
427	0.00554074503770892\\
428	0.00552948911145671\\
429	0.00551794844589325\\
430	0.00550611385673614\\
431	0.00549397575973935\\
432	0.00548152415026784\\
433	0.00546874858210162\\
434	0.00545563814522781\\
435	0.00544218144097815\\
436	0.00542836654737445\\
437	0.00541418098913318\\
438	0.00539961173926262\\
439	0.00538464523869321\\
440	0.0053692673161546\\
441	0.00535346313975109\\
442	0.0053372171746004\\
443	0.00532051313740289\\
444	0.00530333394848065\\
445	0.00528566168259585\\
446	0.0052674775156372\\
447	0.00524876156771987\\
448	0.00522949276320377\\
449	0.00520964896449091\\
450	0.00518920757373021\\
451	0.00516814511954083\\
452	0.00514643720112049\\
453	0.00512405864723791\\
454	0.00510098378576996\\
455	0.00507718651561285\\
456	0.00505264043758917\\
457	0.00502731926123773\\
458	0.00500119338249417\\
459	0.00497423113914844\\
460	0.00494640367555496\\
461	0.00491768390259255\\
462	0.00488804416224272\\
463	0.00485745557444009\\
464	0.00482588695608326\\
465	0.00479330279722531\\
466	0.0047596576280196\\
467	0.00472489779996188\\
468	0.00468897354242522\\
469	0.0046518417927839\\
470	0.00461347006042861\\
471	0.00457384167076456\\
472	0.00453296284521914\\
473	0.0044908721285255\\
474	0.0044476528958311\\
475	0.00440344963230884\\
476	0.00435848950460793\\
477	0.00431330090536759\\
478	0.00426911901941018\\
479	0.00422609131441622\\
480	0.0041843734761192\\
481	0.00414412701115376\\
482	0.00410551496252874\\
483	0.00406869694891489\\
484	0.00403382195664205\\
485	0.00400101794806777\\
486	0.0039703749638837\\
487	0.00394192340717972\\
488	0.00391560420247279\\
489	0.00389078961723187\\
490	0.00386655392115217\\
491	0.00384289551839856\\
492	0.00381980222562067\\
493	0.00379724964136457\\
494	0.00377519960045295\\
495	0.00375359895964029\\
496	0.00373237900858465\\
497	0.0037114559002514\\
498	0.00369073269417876\\
499	0.00367010384999713\\
500	0.00364946334530463\\
501	0.00362877905153483\\
502	0.00360803028005762\\
503	0.00358719451586058\\
504	0.00356624782116794\\
505	0.00354516538132623\\
506	0.00352392219824332\\
507	0.00350249392209815\\
508	0.00348085778748331\\
509	0.00345899358143589\\
510	0.00343688451126113\\
511	0.00341451775218712\\
512	0.0033918824440211\\
513	0.0033689676683681\\
514	0.00334576254116211\\
515	0.00332225629935928\\
516	0.00329843837371602\\
517	0.00327429843792487\\
518	0.0032498264231834\\
519	0.00322501248710322\\
520	0.00319984692763712\\
521	0.00317432003781731\\
522	0.00314842190777655\\
523	0.00312214225160906\\
524	0.00309547037273244\\
525	0.00306839512133135\\
526	0.00304090484325028\\
527	0.00301298732007266\\
528	0.00298462970066008\\
529	0.00295581842514291\\
530	0.00292653914324031\\
531	0.00289677662978668\\
532	0.00286651470130723\\
533	0.00283573613813255\\
534	0.00280442261571909\\
535	0.00277255463848415\\
536	0.00274011147920116\\
537	0.00270707112800494\\
538	0.0026734102563127\\
539	0.00263910420253288\\
540	0.00260412698838064\\
541	0.00256845137703628\\
542	0.00253204898738828\\
543	0.00249489048234822\\
544	0.00245694585391358\\
545	0.00241818483357086\\
546	0.00237857746439831\\
547	0.00233809488657614\\
548	0.00229671045645981\\
549	0.00225441111083414\\
550	0.00221118503939151\\
551	0.00216702202657525\\
552	0.00212191382248228\\
553	0.0020758545648019\\
554	0.00202884126536382\\
555	0.00198087510715297\\
556	0.00193196503358525\\
557	0.00188213043144942\\
558	0.00183148136505412\\
559	0.00177997739204913\\
560	0.00172755807796478\\
561	0.00167414569304286\\
562	0.00161956559511978\\
563	0.00156366756377305\\
564	0.00150631068618296\\
565	0.00144923707526241\\
566	0.00139360112641226\\
567	0.00133997155536227\\
568	0.00128796297954385\\
569	0.00123625110336431\\
570	0.0011849967424863\\
571	0.00113500063362126\\
572	0.00108630229033245\\
573	0.00103861044929871\\
574	0.000991302384123051\\
575	0.000944334643980418\\
576	0.00089771161392153\\
577	0.000851824028850601\\
578	0.000806587975280099\\
579	0.000761530778505787\\
580	0.000716607563069663\\
581	0.000671842104011154\\
582	0.000627277405667915\\
583	0.000582965786062491\\
584	0.000538962689738965\\
585	0.000495324737689842\\
586	0.000452108368453091\\
587	0.000409368226051223\\
588	0.000367155166688129\\
589	0.000325513960353391\\
590	0.000284480716949123\\
591	0.000244080684870838\\
592	0.000204328087251262\\
593	0.000165233217676712\\
594	0.000126870159937009\\
595	8.96111722547531e-05\\
596	5.42660945238512e-05\\
597	2.28062284332062e-05\\
598	2.9204464504877e-07\\
599	0\\
600	0\\
};
\addplot [color=red!80!mycolor19,solid,forget plot]
  table[row sep=crcr]{%
1	0.00604379698173766\\
2	0.00604379373848888\\
3	0.0060437904349912\\
4	0.00604378707016116\\
5	0.00604378364289623\\
6	0.00604378015207447\\
7	0.0060437765965541\\
8	0.00604377297517326\\
9	0.0060437692867496\\
10	0.00604376553007986\\
11	0.00604376170393957\\
12	0.00604375780708261\\
13	0.0060437538382409\\
14	0.00604374979612386\\
15	0.00604374567941812\\
16	0.00604374148678706\\
17	0.00604373721687039\\
18	0.00604373286828375\\
19	0.00604372843961818\\
20	0.00604372392943977\\
21	0.00604371933628912\\
22	0.00604371465868094\\
23	0.00604370989510351\\
24	0.0060437050440182\\
25	0.00604370010385904\\
26	0.0060436950730321\\
27	0.00604368994991509\\
28	0.00604368473285671\\
29	0.0060436794201762\\
30	0.00604367401016273\\
31	0.00604366850107489\\
32	0.00604366289114001\\
33	0.00604365717855372\\
34	0.00604365136147922\\
35	0.00604364543804676\\
36	0.00604363940635292\\
37	0.00604363326446007\\
38	0.00604362701039563\\
39	0.00604362064215146\\
40	0.00604361415768315\\
41	0.00604360755490934\\
42	0.00604360083171102\\
43	0.00604359398593074\\
44	0.00604358701537194\\
45	0.00604357991779815\\
46	0.00604357269093224\\
47	0.00604356533245563\\
48	0.00604355784000749\\
49	0.00604355021118384\\
50	0.00604354244353681\\
51	0.00604353453457374\\
52	0.00604352648175627\\
53	0.00604351828249949\\
54	0.00604350993417095\\
55	0.00604350143408981\\
56	0.00604349277952583\\
57	0.00604348396769837\\
58	0.00604347499577549\\
59	0.00604346586087274\\
60	0.00604345656005232\\
61	0.0060434470903219\\
62	0.00604343744863352\\
63	0.00604342763188252\\
64	0.00604341763690638\\
65	0.00604340746048358\\
66	0.00604339709933237\\
67	0.00604338655010954\\
68	0.00604337580940924\\
69	0.00604336487376163\\
70	0.0060433537396317\\
71	0.00604334240341784\\
72	0.00604333086145045\\
73	0.00604331910999072\\
74	0.00604330714522906\\
75	0.00604329496328367\\
76	0.00604328256019913\\
77	0.00604326993194488\\
78	0.00604325707441361\\
79	0.00604324398341973\\
80	0.00604323065469773\\
81	0.00604321708390063\\
82	0.00604320326659809\\
83	0.00604318919827488\\
84	0.00604317487432911\\
85	0.0060431602900703\\
86	0.00604314544071769\\
87	0.00604313032139822\\
88	0.00604311492714479\\
89	0.00604309925289413\\
90	0.00604308329348493\\
91	0.00604306704365572\\
92	0.00604305049804286\\
93	0.00604303365117828\\
94	0.00604301649748742\\
95	0.00604299903128696\\
96	0.00604298124678256\\
97	0.0060429631380665\\
98	0.00604294469911541\\
99	0.00604292592378771\\
100	0.00604290680582129\\
101	0.00604288733883078\\
102	0.00604286751630521\\
103	0.00604284733160519\\
104	0.00604282677796039\\
105	0.00604280584846656\\
106	0.00604278453608295\\
107	0.00604276283362935\\
108	0.00604274073378318\\
109	0.0060427182290765\\
110	0.00604269531189297\\
111	0.00604267197446476\\
112	0.00604264820886932\\
113	0.00604262400702621\\
114	0.0060425993606937\\
115	0.00604257426146556\\
116	0.00604254870076742\\
117	0.00604252266985342\\
118	0.00604249615980252\\
119	0.00604246916151483\\
120	0.00604244166570795\\
121	0.00604241366291311\\
122	0.00604238514347133\\
123	0.00604235609752938\\
124	0.00604232651503577\\
125	0.00604229638573664\\
126	0.00604226569917148\\
127	0.00604223444466896\\
128	0.00604220261134241\\
129	0.0060421701880854\\
130	0.00604213716356718\\
131	0.00604210352622806\\
132	0.00604206926427466\\
133	0.00604203436567501\\
134	0.00604199881815365\\
135	0.0060419626091866\\
136	0.00604192572599635\\
137	0.00604188815554639\\
138	0.00604184988453604\\
139	0.00604181089939501\\
140	0.00604177118627776\\
141	0.00604173073105795\\
142	0.0060416895193226\\
143	0.00604164753636629\\
144	0.00604160476718508\\
145	0.0060415611964705\\
146	0.0060415168086032\\
147	0.00604147158764674\\
148	0.00604142551734103\\
149	0.00604137858109577\\
150	0.00604133076198368\\
151	0.00604128204273377\\
152	0.0060412324057242\\
153	0.00604118183297524\\
154	0.00604113030614201\\
155	0.00604107780650709\\
156	0.00604102431497294\\
157	0.00604096981205427\\
158	0.00604091427787014\\
159	0.00604085769213601\\
160	0.00604080003415563\\
161	0.00604074128281264\\
162	0.00604068141656227\\
163	0.00604062041342257\\
164	0.00604055825096566\\
165	0.00604049490630886\\
166	0.0060404303561054\\
167	0.00604036457653526\\
168	0.00604029754329556\\
169	0.00604022923159099\\
170	0.00604015961612381\\
171	0.00604008867108396\\
172	0.0060400163701386\\
173	0.00603994268642189\\
174	0.00603986759252411\\
175	0.00603979106048093\\
176	0.00603971306176223\\
177	0.00603963356726089\\
178	0.00603955254728118\\
179	0.00603946997152703\\
180	0.00603938580909002\\
181	0.00603930002843719\\
182	0.00603921259739859\\
183	0.00603912348315443\\
184	0.00603903265222221\\
185	0.00603894007044345\\
186	0.0060388457029702\\
187	0.0060387495142512\\
188	0.00603865146801789\\
189	0.00603855152727004\\
190	0.00603844965426113\\
191	0.00603834581048335\\
192	0.00603823995665241\\
193	0.00603813205269206\\
194	0.00603802205771807\\
195	0.00603790993002215\\
196	0.00603779562705542\\
197	0.00603767910541146\\
198	0.00603756032080915\\
199	0.00603743922807512\\
200	0.00603731578112576\\
201	0.00603718993294888\\
202	0.00603706163558518\\
203	0.00603693084010897\\
204	0.00603679749660884\\
205	0.00603666155416769\\
206	0.00603652296084248\\
207	0.00603638166364351\\
208	0.00603623760851322\\
209	0.00603609074030466\\
210	0.00603594100275937\\
211	0.00603578833848494\\
212	0.00603563268893194\\
213	0.00603547399437057\\
214	0.00603531219386657\\
215	0.00603514722525689\\
216	0.00603497902512458\\
217	0.00603480752877349\\
218	0.006034632670202\\
219	0.00603445438207663\\
220	0.00603427259570483\\
221	0.00603408724100724\\
222	0.00603389824648951\\
223	0.00603370553921332\\
224	0.00603350904476698\\
225	0.0060333086872352\\
226	0.00603310438916856\\
227	0.00603289607155193\\
228	0.00603268365377257\\
229	0.00603246705358736\\
230	0.00603224618708944\\
231	0.00603202096867407\\
232	0.00603179131100386\\
233	0.00603155712497328\\
234	0.00603131831967223\\
235	0.00603107480234912\\
236	0.00603082647837303\\
237	0.00603057325119501\\
238	0.00603031502230869\\
239	0.00603005169120995\\
240	0.00602978315535595\\
241	0.00602950931012283\\
242	0.00602923004876309\\
243	0.00602894526236156\\
244	0.00602865483979086\\
245	0.00602835866766542\\
246	0.00602805663029497\\
247	0.00602774860963676\\
248	0.00602743448524672\\
249	0.00602711413422966\\
250	0.00602678743118839\\
251	0.00602645424817156\\
252	0.00602611445462051\\
253	0.00602576791731476\\
254	0.00602541450031635\\
255	0.00602505406491296\\
256	0.00602468646955961\\
257	0.00602431156981905\\
258	0.00602392921830078\\
259	0.0060235392645986\\
260	0.00602314155522674\\
261	0.00602273593355424\\
262	0.0060223222397381\\
263	0.00602190031065444\\
264	0.00602146997982814\\
265	0.00602103107736065\\
266	0.00602058342985596\\
267	0.00602012686034462\\
268	0.0060196611882056\\
269	0.00601918622908599\\
270	0.00601870179481818\\
271	0.00601820769333418\\
272	0.0060177037285767\\
273	0.00601718970040628\\
274	0.00601666540450337\\
275	0.00601613063226448\\
276	0.00601558517069031\\
277	0.00601502880226538\\
278	0.00601446130483493\\
279	0.00601388245153339\\
280	0.00601329201074085\\
281	0.00601268974597786\\
282	0.00601207541579783\\
283	0.00601144877367671\\
284	0.00601080956790019\\
285	0.0060101575414481\\
286	0.00600949243187622\\
287	0.00600881397119506\\
288	0.006008121885746\\
289	0.00600741589607442\\
290	0.00600669571679988\\
291	0.00600596105648334\\
292	0.00600521161749125\\
293	0.00600444709585661\\
294	0.00600366718113684\\
295	0.00600287155626848\\
296	0.00600205989741867\\
297	0.00600123187383337\\
298	0.00600038714768226\\
299	0.00599952537390029\\
300	0.00599864620002601\\
301	0.00599774926603647\\
302	0.0059968342041787\\
303	0.00599590063879805\\
304	0.00599494818616345\\
305	0.00599397645428955\\
306	0.00599298504275645\\
307	0.00599197354252766\\
308	0.00599094153576698\\
309	0.00598988859565647\\
310	0.0059888142862175\\
311	0.00598771816213875\\
312	0.00598659976861655\\
313	0.005985458641214\\
314	0.00598429430574684\\
315	0.00598310627819779\\
316	0.00598189406463208\\
317	0.00598065716095028\\
318	0.00597939505211107\\
319	0.00597810721187351\\
320	0.00597679310253201\\
321	0.0059754521746443\\
322	0.00597408386675148\\
323	0.00597268760509062\\
324	0.0059712628032988\\
325	0.00596980886210919\\
326	0.00596832516903809\\
327	0.00596681109806301\\
328	0.00596526600929152\\
329	0.00596368924862003\\
330	0.00596208014738282\\
331	0.00596043802199029\\
332	0.00595876217355639\\
333	0.00595705188751468\\
334	0.0059553064332226\\
335	0.00595352506355342\\
336	0.00595170701447526\\
337	0.00594985150461702\\
338	0.00594795773482025\\
339	0.00594602488767673\\
340	0.00594405212705106\\
341	0.00594203859758758\\
342	0.00593998342420066\\
343	0.00593788571154831\\
344	0.00593574454348784\\
345	0.00593355898251308\\
346	0.00593132806917237\\
347	0.00592905082146658\\
348	0.00592672623422611\\
349	0.00592435327846604\\
350	0.0059219309007185\\
351	0.005919458022341\\
352	0.00591693353879988\\
353	0.00591435631892751\\
354	0.00591172520415271\\
355	0.00590903900770255\\
356	0.00590629651377466\\
357	0.00590349647667853\\
358	0.00590063761994472\\
359	0.00589771863540054\\
360	0.00589473818221175\\
361	0.00589169488589031\\
362	0.00588858733725277\\
363	0.00588541409133603\\
364	0.00588217366626758\\
365	0.00587886454208901\\
366	0.00587548515953178\\
367	0.00587203391874112\\
368	0.00586850917794364\\
369	0.00586490925205651\\
370	0.00586123241123398\\
371	0.00585747687934691\\
372	0.00585364083239288\\
373	0.00584972239683445\\
374	0.00584571964786044\\
375	0.00584163060756524\\
376	0.00583745324304039\\
377	0.00583318546437371\\
378	0.00582882512254872\\
379	0.00582437000723769\\
380	0.00581981784447777\\
381	0.00581516629421113\\
382	0.00581041294774655\\
383	0.0058055553250961\\
384	0.00580059087218987\\
385	0.0057955169579579\\
386	0.00579033087126014\\
387	0.00578502981765638\\
388	0.00577961091601375\\
389	0.00577407119492913\\
390	0.00576840758897508\\
391	0.00576261693475519\\
392	0.00575669596675555\\
393	0.00575064131297993\\
394	0.00574444949035536\\
395	0.00573811689989189\\
396	0.00573163982157519\\
397	0.0057250144089629\\
398	0.00571823668345897\\
399	0.00571130252832844\\
400	0.00570420768255797\\
401	0.00569694773434462\\
402	0.00568951811419927\\
403	0.00568191408721808\\
404	0.00567413074496403\\
405	0.00566616299687093\\
406	0.00565800556104346\\
407	0.00564965295490358\\
408	0.00564109948494235\\
409	0.00563233923572894\\
410	0.00562336605821633\\
411	0.00561417355720179\\
412	0.00560475507766134\\
413	0.00559510369100685\\
414	0.005585212183069\\
415	0.0055750730417266\\
416	0.00556467844256432\\
417	0.00555402023031086\\
418	0.00554308990211027\\
419	0.00553187859084382\\
420	0.00552037704796187\\
421	0.00550857562560085\\
422	0.00549646425690355\\
423	0.00548403243420698\\
424	0.00547126918874049\\
425	0.00545816305475796\\
426	0.00544470204193996\\
427	0.00543087362543778\\
428	0.00541666475211721\\
429	0.00540206177521736\\
430	0.00538705039846046\\
431	0.00537161563139494\\
432	0.00535574174438478\\
433	0.00533941222309604\\
434	0.00532260972527357\\
435	0.00530531604468355\\
436	0.0052875120838388\\
437	0.00526917772433\\
438	0.00525029172219893\\
439	0.00523083201498075\\
440	0.00521077665230321\\
441	0.00519010343163706\\
442	0.00516878988349136\\
443	0.00514681334450342\\
444	0.00512415101997684\\
445	0.00510078002476416\\
446	0.00507667740522985\\
447	0.00505182028960638\\
448	0.00502618457628976\\
449	0.00499974289798137\\
450	0.00497246427710687\\
451	0.00494432117897287\\
452	0.00491528257570121\\
453	0.00488530972768461\\
454	0.00485435170143604\\
455	0.00482235931248683\\
456	0.00478928719534402\\
457	0.00475509663674163\\
458	0.00471975949187953\\
459	0.00468326338633841\\
460	0.00464561844076083\\
461	0.00460686637231825\\
462	0.00456709256807714\\
463	0.00452644208904394\\
464	0.00448514072074695\\
465	0.00444389923794743\\
466	0.00440357687039676\\
467	0.00436430832544291\\
468	0.00432623602995577\\
469	0.00428950815751473\\
470	0.00425427567283625\\
471	0.00422068738185323\\
472	0.00418888285783103\\
473	0.00415898311876477\\
474	0.00413107750704409\\
475	0.00410520609121975\\
476	0.00408133284316304\\
477	0.00405911043870122\\
478	0.00403743665823675\\
479	0.0040163131829262\\
480	0.00399573266567172\\
481	0.00397567727603209\\
482	0.00395611732407331\\
483	0.00393701007397081\\
484	0.00391829896134864\\
485	0.00389991353540765\\
486	0.00388177063464587\\
487	0.00386377740380242\\
488	0.00384583705763168\\
489	0.00382788266290682\\
490	0.00380989663293803\\
491	0.00379185963450906\\
492	0.00377375088435137\\
493	0.00375554856466038\\
494	0.00373723036853259\\
495	0.00371877417539081\\
496	0.00370015884067846\\
497	0.00368136505947589\\
498	0.00366237622430955\\
499	0.00364317913778042\\
500	0.00362376435192982\\
501	0.0036041230010163\\
502	0.00358424630052364\\
503	0.00356412563595359\\
504	0.00354375264480726\\
505	0.00352311928443311\\
506	0.00350221787714692\\
507	0.00348104112326624\\
508	0.00345958207301436\\
509	0.00343783405051128\\
510	0.0034157905286329\\
511	0.00339344496435606\\
512	0.0033707907054899\\
513	0.0033478209780482\\
514	0.00332452886784109\\
515	0.00330090729577894\\
516	0.00327694898667039\\
517	0.00325264643170242\\
518	0.00322799184531334\\
519	0.00320297711777891\\
520	0.00317759376545378\\
521	0.00315183288108833\\
522	0.00312568508669098\\
523	0.00309914048826814\\
524	0.00307218862662699\\
525	0.00304481842431517\\
526	0.00301701812888583\\
527	0.00298877525281087\\
528	0.00296007651051954\\
529	0.00293090775321448\\
530	0.00290125390231508\\
531	0.00287109888260652\\
532	0.00284042555644818\\
533	0.00280921566075377\\
534	0.00277744974899561\\
535	0.002745107141692\\
536	0.00271216588991355\\
537	0.00267860275770603\\
538	0.00264439323105173\\
539	0.00260951156316653\\
540	0.00257393086867269\\
541	0.00253762328263359\\
542	0.00250056020476996\\
543	0.00246271265464424\\
544	0.00242405177078421\\
545	0.00238454949588862\\
546	0.00234417950976365\\
547	0.00230291826941214\\
548	0.00226075729913756\\
549	0.00221769056880834\\
550	0.00217371502577192\\
551	0.00212883183532615\\
552	0.00208304983956979\\
553	0.00203638997391316\\
554	0.00198896231752856\\
555	0.00194072070159704\\
556	0.00189160323636402\\
557	0.00184153124543727\\
558	0.0017902977245096\\
559	0.00173782298923212\\
560	0.00168399686198448\\
561	0.0016286614295943\\
562	0.00157372532456192\\
563	0.00152029620686089\\
564	0.00146903631266427\\
565	0.0014186652881098\\
566	0.00136848386464934\\
567	0.00131849638653166\\
568	0.00126901022566881\\
569	0.00122069256463649\\
570	0.0011736084075615\\
571	0.00112676679780215\\
572	0.00108013609015035\\
573	0.00103366236371688\\
574	0.000987377889550623\\
575	0.000941691633156011\\
576	0.000896566979252233\\
577	0.000851509644641342\\
578	0.000806470720006763\\
579	0.0007614808141167\\
580	0.000716583918777049\\
581	0.000671829456087438\\
582	0.000627270354366442\\
583	0.000582961784628753\\
584	0.000538960396115172\\
585	0.000495323454366403\\
586	0.000452107712405483\\
587	0.000409367941602314\\
588	0.000367155075774456\\
589	0.0003255139412327\\
590	0.000284480716949123\\
591	0.000244080684870837\\
592	0.00020432808725126\\
593	0.00016523321767671\\
594	0.000126870159937007\\
595	8.96111722547515e-05\\
596	5.42660945238509e-05\\
597	2.28062284332055e-05\\
598	2.9204464504877e-07\\
599	0\\
600	0\\
};
\addplot [color=red,solid,forget plot]
  table[row sep=crcr]{%
1	0.00604132329944266\\
2	0.00604131925264746\\
3	0.00604131512902119\\
4	0.00604131092717107\\
5	0.00604130664568032\\
6	0.00604130228310772\\
7	0.00604129783798725\\
8	0.00604129330882765\\
9	0.00604128869411197\\
10	0.00604128399229723\\
11	0.00604127920181386\\
12	0.00604127432106537\\
13	0.0060412693484278\\
14	0.00604126428224932\\
15	0.00604125912084976\\
16	0.00604125386252011\\
17	0.00604124850552209\\
18	0.0060412430480875\\
19	0.00604123748841791\\
20	0.00604123182468409\\
21	0.00604122605502535\\
22	0.0060412201775492\\
23	0.0060412141903307\\
24	0.00604120809141197\\
25	0.00604120187880152\\
26	0.00604119555047378\\
27	0.00604118910436851\\
28	0.00604118253839014\\
29	0.0060411758504072\\
30	0.00604116903825171\\
31	0.00604116209971853\\
32	0.00604115503256479\\
33	0.00604114783450912\\
34	0.00604114050323102\\
35	0.00604113303637021\\
36	0.00604112543152595\\
37	0.00604111768625627\\
38	0.0060411097980773\\
39	0.00604110176446248\\
40	0.00604109358284187\\
41	0.00604108525060133\\
42	0.00604107676508178\\
43	0.00604106812357833\\
44	0.00604105932333964\\
45	0.00604105036156682\\
46	0.00604104123541281\\
47	0.00604103194198142\\
48	0.00604102247832642\\
49	0.00604101284145067\\
50	0.00604100302830524\\
51	0.0060409930357884\\
52	0.00604098286074466\\
53	0.00604097249996382\\
54	0.00604096195017998\\
55	0.0060409512080705\\
56	0.00604094027025489\\
57	0.00604092913329385\\
58	0.00604091779368803\\
59	0.00604090624787712\\
60	0.00604089449223851\\
61	0.00604088252308624\\
62	0.00604087033666972\\
63	0.00604085792917265\\
64	0.00604084529671161\\
65	0.00604083243533483\\
66	0.00604081934102101\\
67	0.00604080600967785\\
68	0.00604079243714074\\
69	0.00604077861917138\\
70	0.00604076455145631\\
71	0.00604075022960549\\
72	0.00604073564915083\\
73	0.00604072080554458\\
74	0.00604070569415793\\
75	0.00604069031027922\\
76	0.00604067464911248\\
77	0.00604065870577569\\
78	0.00604064247529903\\
79	0.0060406259526232\\
80	0.0060406091325977\\
81	0.00604059200997878\\
82	0.00604057457942783\\
83	0.00604055683550932\\
84	0.00604053877268885\\
85	0.00604052038533122\\
86	0.00604050166769827\\
87	0.00604048261394694\\
88	0.00604046321812696\\
89	0.00604044347417879\\
90	0.0060404233759313\\
91	0.00604040291709949\\
92	0.00604038209128214\\
93	0.00604036089195942\\
94	0.00604033931249039\\
95	0.00604031734611052\\
96	0.00604029498592913\\
97	0.00604027222492669\\
98	0.00604024905595212\\
99	0.00604022547172015\\
100	0.0060402014648083\\
101	0.00604017702765421\\
102	0.0060401521525525\\
103	0.00604012683165182\\
104	0.00604010105695174\\
105	0.00604007482029962\\
106	0.00604004811338731\\
107	0.0060400209277479\\
108	0.00603999325475223\\
109	0.00603996508560554\\
110	0.00603993641134386\\
111	0.0060399072228304\\
112	0.00603987751075174\\
113	0.00603984726561425\\
114	0.00603981647773998\\
115	0.00603978513726282\\
116	0.00603975323412439\\
117	0.00603972075806994\\
118	0.00603968769864406\\
119	0.00603965404518634\\
120	0.00603961978682697\\
121	0.00603958491248219\\
122	0.00603954941084964\\
123	0.00603951327040361\\
124	0.00603947647939026\\
125	0.00603943902582262\\
126	0.00603940089747556\\
127	0.0060393620818806\\
128	0.00603932256632063\\
129	0.00603928233782457\\
130	0.00603924138316183\\
131	0.00603919968883666\\
132	0.00603915724108239\\
133	0.00603911402585568\\
134	0.00603907002883038\\
135	0.00603902523539151\\
136	0.00603897963062892\\
137	0.00603893319933103\\
138	0.00603888592597824\\
139	0.00603883779473625\\
140	0.0060387887894494\\
141	0.00603873889363357\\
142	0.00603868809046935\\
143	0.00603863636279456\\
144	0.0060385836930971\\
145	0.00603853006350736\\
146	0.00603847545579058\\
147	0.00603841985133902\\
148	0.00603836323116402\\
149	0.00603830557588792\\
150	0.00603824686573561\\
151	0.00603818708052626\\
152	0.00603812619966465\\
153	0.0060380642021324\\
154	0.00603800106647893\\
155	0.00603793677081245\\
156	0.00603787129279056\\
157	0.00603780460961084\\
158	0.00603773669800116\\
159	0.00603766753420982\\
160	0.00603759709399551\\
161	0.00603752535261709\\
162	0.00603745228482317\\
163	0.00603737786484147\\
164	0.00603730206636806\\
165	0.00603722486255624\\
166	0.00603714622600539\\
167	0.00603706612874949\\
168	0.00603698454224547\\
169	0.0060369014373613\\
170	0.00603681678436404\\
171	0.00603673055290728\\
172	0.0060366427120188\\
173	0.00603655323008767\\
174	0.00603646207485128\\
175	0.00603636921338198\\
176	0.00603627461207375\\
177	0.00603617823662822\\
178	0.00603608005204077\\
179	0.00603598002258623\\
180	0.00603587811180431\\
181	0.00603577428248475\\
182	0.00603566849665226\\
183	0.00603556071555106\\
184	0.00603545089962927\\
185	0.00603533900852289\\
186	0.00603522500103951\\
187	0.00603510883514169\\
188	0.00603499046793007\\
189	0.00603486985562615\\
190	0.00603474695355463\\
191	0.00603462171612562\\
192	0.00603449409681624\\
193	0.00603436404815203\\
194	0.00603423152168799\\
195	0.00603409646798919\\
196	0.00603395883661098\\
197	0.00603381857607889\\
198	0.00603367563386799\\
199	0.00603352995638198\\
200	0.00603338148893178\\
201	0.00603323017571368\\
202	0.006033075959787\\
203	0.00603291878305142\\
204	0.00603275858622366\\
205	0.00603259530881392\\
206	0.00603242888910147\\
207	0.00603225926411005\\
208	0.00603208636958272\\
209	0.00603191013995585\\
210	0.006031730508333\\
211	0.00603154740645792\\
212	0.00603136076468708\\
213	0.00603117051196166\\
214	0.00603097657577886\\
215	0.00603077888216255\\
216	0.0060305773556334\\
217	0.00603037191917825\\
218	0.00603016249421888\\
219	0.00602994900058007\\
220	0.0060297313564569\\
221	0.0060295094783815\\
222	0.00602928328118882\\
223	0.00602905267798188\\
224	0.00602881758009614\\
225	0.00602857789706305\\
226	0.00602833353657291\\
227	0.00602808440443683\\
228	0.00602783040454781\\
229	0.00602757143884117\\
230	0.0060273074072538\\
231	0.00602703820768281\\
232	0.00602676373594302\\
233	0.00602648388572377\\
234	0.00602619854854449\\
235	0.00602590761370964\\
236	0.00602561096826228\\
237	0.00602530849693701\\
238	0.00602500008211152\\
239	0.00602468560375743\\
240	0.00602436493938971\\
241	0.00602403796401531\\
242	0.0060237045500804\\
243	0.0060233645674167\\
244	0.0060230178831864\\
245	0.00602266436182607\\
246	0.00602230386498922\\
247	0.00602193625148769\\
248	0.00602156137723161\\
249	0.00602117909516823\\
250	0.00602078925521923\\
251	0.00602039170421677\\
252	0.00601998628583804\\
253	0.00601957284053832\\
254	0.00601915120548276\\
255	0.00601872121447631\\
256	0.00601828269789236\\
257	0.00601783548259966\\
258	0.00601737939188752\\
259	0.0060169142453895\\
260	0.00601643985900513\\
261	0.00601595604482015\\
262	0.00601546261102462\\
263	0.00601495936182939\\
264	0.00601444609738052\\
265	0.00601392261367178\\
266	0.00601338870245519\\
267	0.00601284415114934\\
268	0.00601228874274578\\
269	0.00601172225571316\\
270	0.0060111444638991\\
271	0.00601055513642994\\
272	0.00600995403760808\\
273	0.00600934092680692\\
274	0.00600871555836368\\
275	0.00600807768146947\\
276	0.00600742704005722\\
277	0.00600676337268714\\
278	0.0060060864124301\\
279	0.00600539588674823\\
280	0.00600469151737163\\
281	0.00600397302017226\\
282	0.00600324010503467\\
283	0.00600249247572393\\
284	0.00600172982975035\\
285	0.00600095185823096\\
286	0.00600015824574771\\
287	0.00599934867020256\\
288	0.00599852280266881\\
289	0.00599768030723909\\
290	0.00599682084086989\\
291	0.00599594405322201\\
292	0.00599504958649758\\
293	0.00599413707527296\\
294	0.00599320614632769\\
295	0.00599225641846931\\
296	0.0059912875023539\\
297	0.00599029900030216\\
298	0.00598929050611112\\
299	0.0059882616048609\\
300	0.0059872118727169\\
301	0.00598614087672656\\
302	0.00598504817461133\\
303	0.00598393331455281\\
304	0.00598279583497354\\
305	0.00598163526431188\\
306	0.00598045112079092\\
307	0.00597924291218078\\
308	0.00597801013555466\\
309	0.00597675227703763\\
310	0.00597546881154838\\
311	0.00597415920253324\\
312	0.00597282290169219\\
313	0.00597145934869616\\
314	0.00597006797089525\\
315	0.00596864818301673\\
316	0.00596719938685158\\
317	0.00596572097092812\\
318	0.00596421231018518\\
319	0.00596267276563555\\
320	0.00596110168401903\\
321	0.00595949839744503\\
322	0.00595786222302412\\
323	0.00595619246248798\\
324	0.00595448840179789\\
325	0.00595274931074066\\
326	0.00595097444251194\\
327	0.00594916303328601\\
328	0.00594731430177211\\
329	0.00594542744875613\\
330	0.00594350165662732\\
331	0.00594153608888943\\
332	0.00593952988965533\\
333	0.00593748218312489\\
334	0.00593539207304499\\
335	0.00593325864215104\\
336	0.00593108095158899\\
337	0.00592885804031686\\
338	0.0059265889244846\\
339	0.00592427259679165\\
340	0.00592190802582186\\
341	0.00591949415535425\\
342	0.00591702990364704\\
343	0.00591451416269395\\
344	0.00591194579745183\\
345	0.00590932364503768\\
346	0.00590664651389399\\
347	0.00590391318292038\\
348	0.00590112240057143\\
349	0.00589827288391688\\
350	0.005895363317663\\
351	0.00589239235313299\\
352	0.00588935860720373\\
353	0.00588626066119526\\
354	0.00588309705971205\\
355	0.00587986630943471\\
356	0.00587656687785905\\
357	0.00587319719197867\\
358	0.00586975563690877\\
359	0.0058662405544477\\
360	0.00586265024157304\\
361	0.00585898294886799\\
362	0.00585523687887468\\
363	0.00585141018436922\\
364	0.00584750096655261\\
365	0.00584350727315025\\
366	0.00583942709641713\\
367	0.00583525837107253\\
368	0.0058309989721399\\
369	0.0058266467126732\\
370	0.00582219934137166\\
371	0.00581765454007395\\
372	0.00581300992110441\\
373	0.00580826302448079\\
374	0.00580341131499454\\
375	0.00579845217914324\\
376	0.00579338292190432\\
377	0.00578820076334026\\
378	0.00578290283502511\\
379	0.00577748617628099\\
380	0.0057719477302107\\
381	0.0057662843395138\\
382	0.00576049274206254\\
383	0.00575456956620739\\
384	0.00574851132579433\\
385	0.00574231441499663\\
386	0.00573597510292741\\
387	0.00572948952787593\\
388	0.00572285369113555\\
389	0.00571606345047797\\
390	0.00570911451306406\\
391	0.00570200242799517\\
392	0.00569472257847336\\
393	0.0056872701735383\\
394	0.0056796402393677\\
395	0.00567182761012981\\
396	0.00566382691837387\\
397	0.00565563258491992\\
398	0.00564723880811962\\
399	0.00563863955222837\\
400	0.00562982853558676\\
401	0.00562079922056375\\
402	0.00561154480310268\\
403	0.0056020582030006\\
404	0.00559233204912305\\
405	0.00558235866513765\\
406	0.00557213005533737\\
407	0.00556163788835732\\
408	0.00555087348540174\\
409	0.00553982780368971\\
410	0.00552849141492641\\
411	0.00551685448156283\\
412	0.00550490672915224\\
413	0.00549263740829848\\
414	0.00548003525961385\\
415	0.00546708849749552\\
416	0.00545378479780484\\
417	0.0054401112754309\\
418	0.00542605441187016\\
419	0.0054116000085708\\
420	0.00539673314876577\\
421	0.00538143816631\\
422	0.00536569862809995\\
423	0.00534949732924722\\
424	0.00533281630784088\\
425	0.00531563700179651\\
426	0.00529794031994624\\
427	0.00527970666589112\\
428	0.00526091616858311\\
429	0.00524154930988612\\
430	0.00522158670897148\\
431	0.00520100901224537\\
432	0.00517979689563761\\
433	0.00515793102197964\\
434	0.00513539190146522\\
435	0.00511215960693481\\
436	0.00508821329996635\\
437	0.0050635306819359\\
438	0.00503808544981885\\
439	0.00501184283113749\\
440	0.00498475355696809\\
441	0.00495677028297857\\
442	0.00492784673408646\\
443	0.004897939422489\\
444	0.00486701000135425\\
445	0.00483502847171133\\
446	0.00480197750681774\\
447	0.00476785824140474\\
448	0.00473269801185335\\
449	0.00469656058548712\\
450	0.0046595596934835\\
451	0.00462187671951774\\
452	0.00458393393414347\\
453	0.00454674949141144\\
454	0.00451044051279913\\
455	0.00447513188695105\\
456	0.00444095508964119\\
457	0.00440804619863455\\
458	0.00437654283428951\\
459	0.00434657956098983\\
460	0.00431828136589076\\
461	0.00429175330593373\\
462	0.00426706771200028\\
463	0.00424424587992064\\
464	0.00422323385217898\\
465	0.0042034722512688\\
466	0.00418422708684037\\
467	0.00416550069813697\\
468	0.00414728742576752\\
469	0.00412957230571683\\
470	0.00411232980882289\\
471	0.00409552275599316\\
472	0.00407910158538559\\
473	0.00406300421735797\\
474	0.00404715689755966\\
475	0.00403147654769942\\
476	0.00401587544089208\\
477	0.00400027999628674\\
478	0.0039846749808583\\
479	0.00396904354869625\\
480	0.00395336747204813\\
481	0.0039376274716649\\
482	0.00392180365609093\\
483	0.00390587607449125\\
484	0.00388982537664086\\
485	0.00387363355521733\\
486	0.00385728471522756\\
487	0.00384076577288425\\
488	0.00382406692090728\\
489	0.00380718053226114\\
490	0.00379009904886561\\
491	0.0037728150640437\\
492	0.00375532140232142\\
493	0.00373761119101988\\
494	0.00371967791683657\\
495	0.00370151545967474\\
496	0.00368311809563851\\
497	0.00366448046189818\\
498	0.00364559747892885\\
499	0.0036264642317458\\
500	0.00360707582316718\\
501	0.00358742735026712\\
502	0.00356751390046808\\
503	0.00354733054335028\\
504	0.00352687231790604\\
505	0.00350613421524147\\
506	0.00348511115709701\\
507	0.00346379797101353\\
508	0.00344218936347826\\
509	0.00342027989286092\\
510	0.00339806394424488\\
511	0.00337553570809208\\
512	0.00335268916009062\\
513	0.00332951803879312\\
514	0.00330601582100793\\
515	0.00328217569492879\\
516	0.00325799053100626\\
517	0.00323345285057013\\
518	0.00320855479219949\\
519	0.00318328807580211\\
520	0.00315764396430225\\
521	0.00313161322275183\\
522	0.00310518607459123\\
523	0.00307835215483463\\
524	0.00305110046025353\\
525	0.00302341929672944\\
526	0.00299529622407598\\
527	0.00296671799879961\\
528	0.00293767051549207\\
529	0.00290813874783711\\
530	0.0028781066905927\\
531	0.0028475573044006\\
532	0.00281647246590783\\
533	0.00278483292649969\\
534	0.00275261828398187\\
535	0.0027198069728484\\
536	0.0026863762804127\\
537	0.00265230239814621\\
538	0.00261756052016224\\
539	0.00258212500404021\\
540	0.00254596961341247\\
541	0.00250906786742201\\
542	0.00247139352909162\\
543	0.00243292127202608\\
544	0.00239362755840972\\
545	0.00235349175704467\\
546	0.00231249727013801\\
547	0.00227064453612754\\
548	0.00222794134987418\\
549	0.00218440577001577\\
550	0.00214013908144317\\
551	0.00209511622696201\\
552	0.00204927801102705\\
553	0.00200254729293833\\
554	0.00195472088222525\\
555	0.00190573084441342\\
556	0.00185549613625779\\
557	0.00180390471392892\\
558	0.00175081620496203\\
559	0.00169770725359811\\
560	0.00164609814802067\\
561	0.0015966641520743\\
562	0.00154794430499767\\
563	0.00149930786152006\\
564	0.00145063724791586\\
565	0.00140192736334399\\
566	0.0013538859561472\\
567	0.00130694443089763\\
568	0.00126075754464001\\
569	0.00121470715974243\\
570	0.00116868405784354\\
571	0.00112272218384932\\
572	0.00107685136522041\\
573	0.00103136023930306\\
574	0.000986412336410125\\
575	0.000941498720255945\\
576	0.000896516120383694\\
577	0.000851490573360361\\
578	0.000806462517147885\\
579	0.000761476872772062\\
580	0.000716581800501493\\
581	0.00067182827412899\\
582	0.00062726968466177\\
583	0.000582961403638049\\
584	0.000538960186060454\\
585	0.000495323349035172\\
586	0.000452107667820891\\
587	0.000409367927672478\\
588	0.000367155072958957\\
589	0.000325513941232698\\
590	0.000284480716949121\\
591	0.000244080684870835\\
592	0.00020432808725126\\
593	0.000165233217676711\\
594	0.000126870159937007\\
595	8.9611172254752e-05\\
596	5.42660945238507e-05\\
597	2.28062284332057e-05\\
598	2.9204464504877e-07\\
599	0\\
600	0\\
};
\addplot [color=mycolor20,solid,forget plot]
  table[row sep=crcr]{%
1	0.00604070286126672\\
2	0.00604069768145596\\
3	0.00604069240011982\\
4	0.00604068701536704\\
5	0.00604068152527344\\
6	0.00604067592788133\\
7	0.00604067022119914\\
8	0.00604066440320073\\
9	0.00604065847182493\\
10	0.006040652424975\\
11	0.00604064626051793\\
12	0.00604063997628406\\
13	0.00604063357006636\\
14	0.00604062703961985\\
15	0.00604062038266108\\
16	0.00604061359686745\\
17	0.00604060667987656\\
18	0.00604059962928574\\
19	0.00604059244265129\\
20	0.00604058511748774\\
21	0.00604057765126749\\
22	0.00604057004141987\\
23	0.00604056228533052\\
24	0.0060405543803408\\
25	0.00604054632374706\\
26	0.00604053811279983\\
27	0.00604052974470326\\
28	0.00604052121661428\\
29	0.00604051252564191\\
30	0.0060405036688465\\
31	0.0060404946432389\\
32	0.00604048544577976\\
33	0.00604047607337873\\
34	0.00604046652289367\\
35	0.00604045679112977\\
36	0.00604044687483879\\
37	0.00604043677071818\\
38	0.00604042647541023\\
39	0.00604041598550124\\
40	0.00604040529752056\\
41	0.00604039440793977\\
42	0.0060403833131717\\
43	0.00604037200956956\\
44	0.00604036049342594\\
45	0.00604034876097188\\
46	0.0060403368083759\\
47	0.00604032463174296\\
48	0.00604031222711347\\
49	0.00604029959046228\\
50	0.00604028671769763\\
51	0.00604027360465999\\
52	0.00604026024712113\\
53	0.00604024664078291\\
54	0.00604023278127614\\
55	0.00604021866415943\\
56	0.00604020428491812\\
57	0.00604018963896296\\
58	0.00604017472162895\\
59	0.00604015952817414\\
60	0.00604014405377826\\
61	0.00604012829354153\\
62	0.00604011224248334\\
63	0.00604009589554082\\
64	0.00604007924756761\\
65	0.00604006229333234\\
66	0.00604004502751723\\
67	0.00604002744471677\\
68	0.00604000953943605\\
69	0.00603999130608937\\
70	0.00603997273899869\\
71	0.00603995383239199\\
72	0.00603993458040168\\
73	0.00603991497706302\\
74	0.00603989501631235\\
75	0.00603987469198551\\
76	0.00603985399781589\\
77	0.00603983292743276\\
78	0.00603981147435948\\
79	0.00603978963201152\\
80	0.00603976739369456\\
81	0.00603974475260268\\
82	0.00603972170181615\\
83	0.00603969823429948\\
84	0.00603967434289937\\
85	0.00603965002034245\\
86	0.00603962525923319\\
87	0.00603960005205163\\
88	0.00603957439115098\\
89	0.00603954826875544\\
90	0.00603952167695767\\
91	0.00603949460771633\\
92	0.00603946705285358\\
93	0.00603943900405253\\
94	0.0060394104528545\\
95	0.00603938139065637\\
96	0.00603935180870775\\
97	0.00603932169810817\\
98	0.00603929104980418\\
99	0.00603925985458628\\
100	0.00603922810308589\\
101	0.0060391957857722\\
102	0.00603916289294902\\
103	0.0060391294147513\\
104	0.00603909534114195\\
105	0.00603906066190825\\
106	0.00603902536665826\\
107	0.00603898944481726\\
108	0.00603895288562394\\
109	0.00603891567812664\\
110	0.00603887781117927\\
111	0.00603883927343742\\
112	0.00603880005335422\\
113	0.00603876013917593\\
114	0.00603871951893777\\
115	0.00603867818045937\\
116	0.00603863611134024\\
117	0.00603859329895501\\
118	0.00603854973044864\\
119	0.00603850539273156\\
120	0.00603846027247447\\
121	0.0060384143561033\\
122	0.00603836762979378\\
123	0.00603832007946602\\
124	0.00603827169077896\\
125	0.00603822244912454\\
126	0.00603817233962188\\
127	0.00603812134711125\\
128	0.00603806945614792\\
129	0.0060380166509958\\
130	0.00603796291562085\\
131	0.00603790823368464\\
132	0.00603785258853736\\
133	0.00603779596321088\\
134	0.0060377383404116\\
135	0.00603767970251324\\
136	0.00603762003154909\\
137	0.00603755930920449\\
138	0.0060374975168089\\
139	0.0060374346353279\\
140	0.0060373706453548\\
141	0.00603730552710236\\
142	0.00603723926039393\\
143	0.00603717182465477\\
144	0.00603710319890287\\
145	0.00603703336173978\\
146	0.00603696229134103\\
147	0.00603688996544644\\
148	0.00603681636135023\\
149	0.00603674145589082\\
150	0.00603666522544051\\
151	0.00603658764589482\\
152	0.0060365086926617\\
153	0.00603642834065032\\
154	0.00603634656425989\\
155	0.00603626333736804\\
156	0.00603617863331896\\
157	0.00603609242491136\\
158	0.00603600468438619\\
159	0.00603591538341399\\
160	0.0060358244930821\\
161	0.00603573198388151\\
162	0.00603563782569354\\
163	0.00603554198777613\\
164	0.00603544443874995\\
165	0.00603534514658427\\
166	0.00603524407858244\\
167	0.00603514120136712\\
168	0.00603503648086523\\
169	0.0060349298822928\\
170	0.00603482137013902\\
171	0.0060347109081507\\
172	0.00603459845931581\\
173	0.00603448398584715\\
174	0.00603436744916537\\
175	0.00603424880988208\\
176	0.00603412802778226\\
177	0.00603400506180656\\
178	0.00603387987003342\\
179	0.00603375240966048\\
180	0.00603362263698609\\
181	0.00603349050739019\\
182	0.00603335597531511\\
183	0.00603321899424582\\
184	0.00603307951668995\\
185	0.00603293749415748\\
186	0.00603279287714004\\
187	0.00603264561508996\\
188	0.00603249565639885\\
189	0.00603234294837584\\
190	0.00603218743722551\\
191	0.00603202906802543\\
192	0.00603186778470337\\
193	0.00603170353001399\\
194	0.00603153624551522\\
195	0.00603136587154434\\
196	0.0060311923471935\\
197	0.0060310156102849\\
198	0.0060308355973455\\
199	0.00603065224358139\\
200	0.00603046548285162\\
201	0.00603027524764155\\
202	0.00603008146903589\\
203	0.00602988407669103\\
204	0.00602968299880715\\
205	0.00602947816209949\\
206	0.00602926949176939\\
207	0.00602905691147463\\
208	0.00602884034329925\\
209	0.00602861970772277\\
210	0.00602839492358883\\
211	0.0060281659080733\\
212	0.00602793257665158\\
213	0.0060276948430654\\
214	0.00602745261928877\\
215	0.00602720581549342\\
216	0.00602695434001352\\
217	0.00602669809930934\\
218	0.00602643699793054\\
219	0.00602617093847832\\
220	0.00602589982156705\\
221	0.00602562354578477\\
222	0.00602534200765297\\
223	0.00602505510158548\\
224	0.00602476271984619\\
225	0.00602446475250616\\
226	0.00602416108739942\\
227	0.00602385161007798\\
228	0.00602353620376556\\
229	0.00602321474931045\\
230	0.00602288712513714\\
231	0.00602255320719681\\
232	0.00602221286891671\\
233	0.00602186598114831\\
234	0.00602151241211429\\
235	0.00602115202735406\\
236	0.00602078468966836\\
237	0.0060204102590622\\
238	0.00602002859268669\\
239	0.00601963954477947\\
240	0.00601924296660363\\
241	0.0060188387063855\\
242	0.00601842660925068\\
243	0.00601800651715886\\
244	0.00601757826883696\\
245	0.00601714169971099\\
246	0.0060166966418361\\
247	0.00601624292382536\\
248	0.0060157803707767\\
249	0.00601530880419837\\
250	0.00601482804193277\\
251	0.00601433789807843\\
252	0.00601383818291052\\
253	0.00601332870279949\\
254	0.00601280926012784\\
255	0.00601227965320519\\
256	0.00601173967618163\\
257	0.00601118911895872\\
258	0.00601062776709899\\
259	0.00601005540173316\\
260	0.00600947179946547\\
261	0.00600887673227664\\
262	0.00600826996742509\\
263	0.00600765126734565\\
264	0.0060070203895459\\
265	0.00600637708650021\\
266	0.00600572110554137\\
267	0.0060050521887495\\
268	0.00600437007283856\\
269	0.00600367448903991\\
270	0.00600296516298329\\
271	0.00600224181457464\\
272	0.00600150415787112\\
273	0.00600075190095296\\
274	0.00599998474579196\\
275	0.0059992023881168\\
276	0.0059984045172748\\
277	0.00599759081609012\\
278	0.00599676096071829\\
279	0.00599591462049686\\
280	0.005995051457792\\
281	0.00599417112784122\\
282	0.00599327327859168\\
283	0.0059923575505342\\
284	0.0059914235765328\\
285	0.00599047098164961\\
286	0.00598949938296499\\
287	0.00598850838939272\\
288	0.00598749760149035\\
289	0.00598646661126413\\
290	0.0059854150019687\\
291	0.00598434234790135\\
292	0.0059832482141907\\
293	0.00598213215657961\\
294	0.00598099372120207\\
295	0.00597983244435416\\
296	0.00597864785225872\\
297	0.00597743946082362\\
298	0.0059762067753936\\
299	0.00597494929049533\\
300	0.00597366648957539\\
301	0.00597235784473117\\
302	0.00597102281643418\\
303	0.00596966085324572\\
304	0.00596827139152416\\
305	0.00596685385512431\\
306	0.00596540765508857\\
307	0.00596393218932975\\
308	0.00596242684230389\\
309	0.00596089098467301\\
310	0.005959323972958\\
311	0.00595772514918041\\
312	0.00595609384049316\\
313	0.00595442935879954\\
314	0.00595273100035977\\
315	0.00595099804538495\\
316	0.00594922975761662\\
317	0.00594742538389277\\
318	0.00594558415369862\\
319	0.00594370527870151\\
320	0.00594178795226922\\
321	0.00593983134897069\\
322	0.00593783462405842\\
323	0.00593579691293132\\
324	0.0059337173305774\\
325	0.00593159497099554\\
326	0.00592942890659603\\
327	0.0059272181875775\\
328	0.0059249618412794\\
329	0.00592265887150968\\
330	0.00592030825784598\\
331	0.00591790895490918\\
332	0.00591545989160667\\
333	0.00591295997034463\\
334	0.00591040806620996\\
335	0.00590780302611918\\
336	0.00590514366793275\\
337	0.00590242877953272\\
338	0.00589965711786155\\
339	0.00589682740791893\\
340	0.00589393834171451\\
341	0.00589098857718418\\
342	0.00588797673707183\\
343	0.00588490140775219\\
344	0.0058817611380014\\
345	0.00587855443771407\\
346	0.00587527977656284\\
347	0.00587193558259683\\
348	0.00586852024077776\\
349	0.00586503209146127\\
350	0.00586146942880643\\
351	0.00585783049911073\\
352	0.00585411349906987\\
353	0.00585031657395378\\
354	0.00584643781567608\\
355	0.00584247526076415\\
356	0.00583842688824244\\
357	0.0058342906174133\\
358	0.00583006430552663\\
359	0.00582574574533156\\
360	0.00582133266250288\\
361	0.00581682271293385\\
362	0.0058122134798853\\
363	0.00580750247098027\\
364	0.00580268711502714\\
365	0.00579776475864244\\
366	0.00579273266261862\\
367	0.00578758799798392\\
368	0.00578232784208436\\
369	0.00577694917450169\\
370	0.00577144887262362\\
371	0.0057658237069727\\
372	0.00576007033631017\\
373	0.00575418530222471\\
374	0.00574816502335601\\
375	0.00574200578947321\\
376	0.00573570375528036\\
377	0.00572925493391668\\
378	0.00572265519015235\\
379	0.00571590023328521\\
380	0.00570898560975258\\
381	0.00570190669547667\\
382	0.00569465868796211\\
383	0.00568723659813321\\
384	0.00567963524178665\\
385	0.00567184923042796\\
386	0.00566387296297438\\
387	0.00565570061850434\\
388	0.00564732614852558\\
389	0.00563874326842201\\
390	0.00562994544885467\\
391	0.00562092590397597\\
392	0.00561167757684297\\
393	0.0056021931235798\\
394	0.00559246489594869\\
395	0.00558248492210143\\
396	0.00557224488527129\\
397	0.00556173610014804\\
398	0.00555094948656369\\
399	0.00553987553928657\\
400	0.00552850428898409\\
401	0.00551682525811876\\
402	0.00550482743946284\\
403	0.00549249927104814\\
404	0.00547982863506249\\
405	0.00546680280346584\\
406	0.00545340840135631\\
407	0.00543963139660914\\
408	0.00542545709129351\\
409	0.00541087024193215\\
410	0.00539585526736261\\
411	0.00538039629792995\\
412	0.00536447723384199\\
413	0.00534808180737729\\
414	0.00533119353783731\\
415	0.0053137956986983\\
416	0.00529587151388881\\
417	0.00527740443553554\\
418	0.00525837840941861\\
419	0.00523877743823224\\
420	0.00521858532440277\\
421	0.00519778529766863\\
422	0.00517635939137337\\
423	0.00515428747691237\\
424	0.00513154562729427\\
425	0.00510810224031989\\
426	0.00508391495677255\\
427	0.00505893916759164\\
428	0.00503312793904506\\
429	0.00500643350322129\\
430	0.00497881596042503\\
431	0.00495024230059195\\
432	0.00492068858544584\\
433	0.00489014442261511\\
434	0.00485861893677142\\
435	0.0048261487135882\\
436	0.00479280832293565\\
437	0.00475872421739817\\
438	0.00472409309455758\\
439	0.0046897229291746\\
440	0.00465601899576579\\
441	0.00462308586759372\\
442	0.00459103525379076\\
443	0.00455998509454473\\
444	0.00453005799718654\\
445	0.00450137864226732\\
446	0.00447406996077481\\
447	0.00444824758907592\\
448	0.00442401202173317\\
449	0.00440143778640491\\
450	0.00438055781551818\\
451	0.00436134261793029\\
452	0.00434351468438177\\
453	0.00432616324410694\\
454	0.00430929422423594\\
455	0.00429290781882491\\
456	0.00427699674321518\\
457	0.00426154508181158\\
458	0.00424652720145676\\
459	0.00423190685177035\\
460	0.00421763662973471\\
461	0.004203658089645\\
462	0.0041899028287776\\
463	0.00417629507945139\\
464	0.0041627565142112\\
465	0.004149235382823\\
466	0.00413571810586105\\
467	0.00412218965680862\\
468	0.00410863376027093\\
469	0.00409503317653824\\
470	0.00408137008153313\\
471	0.0040676265459852\\
472	0.00405378510905811\\
473	0.00403982942760372\\
474	0.00402574495899102\\
475	0.00401151960005543\\
476	0.00399714414979814\\
477	0.0039826119151884\\
478	0.00396791625508381\\
479	0.00395305065346735\\
480	0.00393800879180704\\
481	0.00392278461619124\\
482	0.00390737239387512\\
483	0.00389176675294497\\
484	0.00387596269825045\\
485	0.00385995559697531\\
486	0.00384374112887292\\
487	0.00382731520011974\\
488	0.00381067382738024\\
489	0.00379381305678002\\
490	0.00377672896545118\\
491	0.00375941765990687\\
492	0.00374187527095354\\
493	0.0037240979450396\\
494	0.00370608183220922\\
495	0.00368782307117653\\
496	0.00366931777244603\\
497	0.00365056200083808\\
498	0.00363155175914802\\
499	0.0036122829748172\\
500	0.00359275149114712\\
501	0.00357295305835337\\
502	0.00355288332358055\\
503	0.00353253781991185\\
504	0.00351191195442204\\
505	0.00349100099532988\\
506	0.00346980005829799\\
507	0.00344830409190142\\
508	0.00342650786223379\\
509	0.00340440593654245\\
510	0.00338199266568192\\
511	0.00335926216506952\\
512	0.00333620829390472\\
513	0.00331282463253544\\
514	0.00328910445785155\\
515	0.00326504071658356\\
516	0.00324062599638607\\
517	0.00321585249459072\\
518	0.00319071198452735\\
519	0.00316519577933604\\
520	0.00313929469323235\\
521	0.00311299900024939\\
522	0.00308629839056859\\
523	0.0030591819246678\\
524	0.00303163798565948\\
525	0.00300365423037989\\
526	0.00297521754003508\\
527	0.00294631397152773\\
528	0.00291692871099567\\
529	0.00288704603161653\\
530	0.00285664925839858\\
531	0.00282572074352561\\
532	0.00279424185689702\\
533	0.00276219299786336\\
534	0.00272955363586645\\
535	0.00269630238983699\\
536	0.00266241715889706\\
537	0.00262787532050961\\
538	0.00259265401715977\\
539	0.00255673055832753\\
540	0.00252008296455489\\
541	0.00248269066657626\\
542	0.00244453538964067\\
543	0.00240560232750984\\
544	0.00236588246753321\\
545	0.00232537716519942\\
546	0.00228415675127574\\
547	0.00224224875684687\\
548	0.00219959959602875\\
549	0.00215614081365568\\
550	0.0021116893143768\\
551	0.00206616683318269\\
552	0.00201950322840626\\
553	0.00197160885480873\\
554	0.00192240234639964\\
555	0.00187175949147671\\
556	0.00182016778867611\\
557	0.0017700079133193\\
558	0.00172189222910075\\
559	0.00167483274808183\\
560	0.00162776118290781\\
561	0.00158051982821157\\
562	0.0015330935121052\\
563	0.00148554422170036\\
564	0.00143866178250926\\
565	0.00139285578388573\\
566	0.0013474996131113\\
567	0.00130214936842766\\
568	0.00125674675529522\\
569	0.00121131505068596\\
570	0.00116589341138373\\
571	0.00112058409493486\\
572	0.00107579239048085\\
573	0.00103115420667899\\
574	0.000986382803600775\\
575	0.00094149076835946\\
576	0.000896513086086129\\
577	0.000851489245902153\\
578	0.000806461868510235\\
579	0.000761476522194667\\
580	0.000716581604601778\\
581	0.000671828163310488\\
582	0.00062726962209076\\
583	0.000582961369628161\\
584	0.000538960169322208\\
585	0.000495323342111254\\
586	0.000452107665705174\\
587	0.000409367927260633\\
588	0.000367155072958956\\
589	0.000325513941232701\\
590	0.000284480716949122\\
591	0.000244080684870839\\
592	0.000204328087251261\\
593	0.000165233217676712\\
594	0.000126870159937008\\
595	8.9611172254752e-05\\
596	5.42660945238509e-05\\
597	2.28062284332057e-05\\
598	2.9204464504877e-07\\
599	0\\
600	0\\
};
\addplot [color=mycolor21,solid,forget plot]
  table[row sep=crcr]{%
1	0.0060405115386396\\
2	0.00604050497580479\\
3	0.00604049827929662\\
4	0.0060404914465168\\
5	0.00604048447481996\\
6	0.0060404773615129\\
7	0.00604047010385364\\
8	0.00604046269905084\\
9	0.00604045514426286\\
10	0.00604044743659698\\
11	0.0060404395731087\\
12	0.00604043155080067\\
13	0.00604042336662207\\
14	0.00604041501746772\\
15	0.00604040650017708\\
16	0.00604039781153354\\
17	0.00604038894826346\\
18	0.00604037990703524\\
19	0.00604037068445846\\
20	0.00604036127708301\\
21	0.00604035168139799\\
22	0.00604034189383093\\
23	0.00604033191074684\\
24	0.00604032172844708\\
25	0.00604031134316853\\
26	0.00604030075108255\\
27	0.00604028994829395\\
28	0.00604027893083998\\
29	0.00604026769468939\\
30	0.00604025623574118\\
31	0.0060402445498238\\
32	0.0060402326326939\\
33	0.00604022048003529\\
34	0.00604020808745783\\
35	0.00604019545049642\\
36	0.00604018256460975\\
37	0.0060401694251792\\
38	0.00604015602750773\\
39	0.00604014236681863\\
40	0.0060401284382545\\
41	0.00604011423687585\\
42	0.00604009975766\\
43	0.00604008499549986\\
44	0.00604006994520267\\
45	0.00604005460148872\\
46	0.00604003895899013\\
47	0.00604002301224949\\
48	0.00604000675571866\\
49	0.00603999018375738\\
50	0.00603997329063187\\
51	0.00603995607051361\\
52	0.00603993851747789\\
53	0.00603992062550243\\
54	0.00603990238846597\\
55	0.00603988380014686\\
56	0.00603986485422158\\
57	0.00603984554426332\\
58	0.00603982586374054\\
59	0.00603980580601526\\
60	0.00603978536434184\\
61	0.00603976453186516\\
62	0.00603974330161922\\
63	0.00603972166652551\\
64	0.00603969961939137\\
65	0.00603967715290844\\
66	0.00603965425965089\\
67	0.00603963093207381\\
68	0.00603960716251152\\
69	0.00603958294317581\\
70	0.00603955826615418\\
71	0.00603953312340808\\
72	0.0060395075067711\\
73	0.00603948140794716\\
74	0.00603945481850856\\
75	0.00603942772989412\\
76	0.00603940013340739\\
77	0.00603937202021446\\
78	0.00603934338134211\\
79	0.00603931420767584\\
80	0.00603928448995767\\
81	0.00603925421878412\\
82	0.0060392233846041\\
83	0.00603919197771672\\
84	0.00603915998826905\\
85	0.00603912740625393\\
86	0.00603909422150771\\
87	0.00603906042370773\\
88	0.00603902600237024\\
89	0.00603899094684771\\
90	0.00603895524632647\\
91	0.00603891888982421\\
92	0.00603888186618738\\
93	0.00603884416408854\\
94	0.00603880577202371\\
95	0.00603876667830962\\
96	0.00603872687108093\\
97	0.00603868633828738\\
98	0.00603864506769076\\
99	0.00603860304686205\\
100	0.00603856026317835\\
101	0.0060385167038197\\
102	0.00603847235576585\\
103	0.00603842720579312\\
104	0.00603838124047081\\
105	0.00603833444615807\\
106	0.00603828680900013\\
107	0.00603823831492472\\
108	0.00603818894963851\\
109	0.00603813869862311\\
110	0.0060380875471314\\
111	0.00603803548018332\\
112	0.00603798248256189\\
113	0.00603792853880897\\
114	0.00603787363322096\\
115	0.0060378177498443\\
116	0.00603776087247091\\
117	0.00603770298463356\\
118	0.00603764406960102\\
119	0.00603758411037308\\
120	0.0060375230896755\\
121	0.00603746098995471\\
122	0.00603739779337252\\
123	0.00603733348180048\\
124	0.00603726803681418\\
125	0.00603720143968747\\
126	0.00603713367138634\\
127	0.00603706471256278\\
128	0.00603699454354828\\
129	0.00603692314434729\\
130	0.00603685049463051\\
131	0.0060367765737278\\
132	0.00603670136062105\\
133	0.00603662483393677\\
134	0.0060365469719385\\
135	0.00603646775251884\\
136	0.00603638715319152\\
137	0.006036305151083\\
138	0.00603622172292386\\
139	0.00603613684504009\\
140	0.00603605049334386\\
141	0.00603596264332425\\
142	0.00603587327003759\\
143	0.00603578234809756\\
144	0.00603568985166492\\
145	0.00603559575443699\\
146	0.00603550002963689\\
147	0.00603540265000233\\
148	0.00603530358777422\\
149	0.00603520281468479\\
150	0.00603510030194546\\
151	0.00603499602023439\\
152	0.00603488993968359\\
153	0.00603478202986581\\
154	0.00603467225978081\\
155	0.00603456059784149\\
156	0.00603444701185953\\
157	0.00603433146903064\\
158	0.0060342139359194\\
159	0.00603409437844369\\
160	0.00603397276185872\\
161	0.00603384905074067\\
162	0.00603372320896976\\
163	0.0060335951997131\\
164	0.00603346498540687\\
165	0.00603333252773816\\
166	0.00603319778762635\\
167	0.00603306072520401\\
168	0.00603292129979739\\
169	0.00603277946990622\\
170	0.00603263519318343\\
171	0.00603248842641388\\
172	0.00603233912549306\\
173	0.00603218724540499\\
174	0.0060320327401998\\
175	0.00603187556297068\\
176	0.00603171566583043\\
177	0.00603155299988751\\
178	0.0060313875152214\\
179	0.00603121916085778\\
180	0.00603104788474286\\
181	0.00603087363371748\\
182	0.00603069635349043\\
183	0.00603051598861155\\
184	0.00603033248244412\\
185	0.00603014577713679\\
186	0.00602995581359512\\
187	0.0060297625314524\\
188	0.00602956586904013\\
189	0.00602936576335804\\
190	0.00602916215004345\\
191	0.00602895496334029\\
192	0.00602874413606756\\
193	0.00602852959958735\\
194	0.00602831128377235\\
195	0.00602808911697287\\
196	0.00602786302598349\\
197	0.00602763293600906\\
198	0.0060273987706305\\
199	0.00602716045176988\\
200	0.00602691789965512\\
201	0.00602667103278441\\
202	0.0060264197678899\\
203	0.00602616401990116\\
204	0.00602590370190799\\
205	0.006025638725123\\
206	0.00602536899884355\\
207	0.00602509443041335\\
208	0.00602481492518334\\
209	0.00602453038647252\\
210	0.00602424071552795\\
211	0.00602394581148433\\
212	0.00602364557132305\\
213	0.00602333988983085\\
214	0.00602302865955784\\
215	0.00602271177077484\\
216	0.00602238911143023\\
217	0.00602206056710623\\
218	0.00602172602097441\\
219	0.00602138535375063\\
220	0.00602103844364905\\
221	0.00602068516633558\\
222	0.00602032539488046\\
223	0.00601995899970986\\
224	0.00601958584855677\\
225	0.00601920580641077\\
226	0.00601881873546681\\
227	0.00601842449507293\\
228	0.00601802294167679\\
229	0.00601761392877109\\
230	0.00601719730683757\\
231	0.00601677292328973\\
232	0.00601634062241428\\
233	0.0060159002453106\\
234	0.00601545162982939\\
235	0.00601499461050892\\
236	0.00601452901851008\\
237	0.00601405468154942\\
238	0.00601357142383042\\
239	0.00601307906597272\\
240	0.00601257742493947\\
241	0.00601206631396247\\
242	0.00601154554246535\\
243	0.00601101491598445\\
244	0.00601047423608748\\
245	0.00600992330028988\\
246	0.00600936190196889\\
247	0.00600878983027507\\
248	0.00600820687004161\\
249	0.00600761280169106\\
250	0.00600700740113942\\
251	0.00600639043969812\\
252	0.00600576168397293\\
253	0.00600512089576077\\
254	0.00600446783194364\\
255	0.00600380224438007\\
256	0.00600312387979381\\
257	0.00600243247966014\\
258	0.00600172778008923\\
259	0.00600100951170685\\
260	0.00600027739953232\\
261	0.00599953116285375\\
262	0.00599877051510059\\
263	0.00599799516371429\\
264	0.00599720481001554\\
265	0.00599639914906835\\
266	0.0059955778695411\\
267	0.0059947406535649\\
268	0.00599388717658837\\
269	0.00599301710722961\\
270	0.00599213010712454\\
271	0.00599122583077206\\
272	0.00599030392537543\\
273	0.00598936403067997\\
274	0.00598840577880696\\
275	0.00598742879408339\\
276	0.00598643269286741\\
277	0.00598541708336929\\
278	0.00598438156546767\\
279	0.00598332573052086\\
280	0.0059822491611728\\
281	0.00598115143115341\\
282	0.00598003210507315\\
283	0.00597889073821138\\
284	0.00597772687629799\\
285	0.00597654005528827\\
286	0.00597532980113045\\
287	0.00597409562952539\\
288	0.00597283704567818\\
289	0.00597155354404147\\
290	0.00597024460804948\\
291	0.00596890970984297\\
292	0.00596754830998413\\
293	0.00596615985716149\\
294	0.00596474378788404\\
295	0.00596329952616421\\
296	0.0059618264831893\\
297	0.00596032405698086\\
298	0.00595879163204187\\
299	0.00595722857899214\\
300	0.00595563425419103\\
301	0.00595400799934622\\
302	0.00595234914110841\\
303	0.00595065699065172\\
304	0.00594893084323893\\
305	0.00594716997776979\\
306	0.00594537365631309\\
307	0.0059435411236291\\
308	0.00594167160668233\\
309	0.00593976431413166\\
310	0.00593781843580287\\
311	0.00593583314214417\\
312	0.0059338075836631\\
313	0.00593174089034451\\
314	0.00592963217105018\\
315	0.00592748051289843\\
316	0.00592528498062269\\
317	0.00592304461590789\\
318	0.00592075843670291\\
319	0.00591842543650829\\
320	0.00591604458363759\\
321	0.0059136148204504\\
322	0.00591113506255556\\
323	0.00590860419798202\\
324	0.00590602108631484\\
325	0.0059033845577936\\
326	0.00590069341237401\\
327	0.0058979464187612\\
328	0.00589514231339425\\
329	0.0058922797993854\\
330	0.00588935754541443\\
331	0.0058863741845763\\
332	0.00588332831317656\\
333	0.00588021848945845\\
334	0.00587704323226054\\
335	0.0058738010196226\\
336	0.00587049028732595\\
337	0.00586710942736092\\
338	0.00586365678631363\\
339	0.00586013066366021\\
340	0.00585652930994459\\
341	0.0058528509248092\\
342	0.00584909365496575\\
343	0.0058452555921781\\
344	0.00584133477099838\\
345	0.00583732916634428\\
346	0.00583323669094712\\
347	0.00582905519265595\\
348	0.00582478245157597\\
349	0.00582041617703999\\
350	0.00581595400455628\\
351	0.00581139349257955\\
352	0.00580673211910165\\
353	0.00580196727811239\\
354	0.00579709627591544\\
355	0.00579211632706821\\
356	0.00578702455004549\\
357	0.00578181796285351\\
358	0.00577649347849695\\
359	0.00577104790028927\\
360	0.00576547791702843\\
361	0.00575978009807038\\
362	0.00575395088834157\\
363	0.00574798660334017\\
364	0.00574188342418147\\
365	0.00573563739272935\\
366	0.00572924440677059\\
367	0.00572270021484352\\
368	0.00571600040952584\\
369	0.005709140423154\\
370	0.0057021155240592\\
371	0.00569492080962159\\
372	0.0056875511982807\\
373	0.0056800014216376\\
374	0.00567226601295422\\
375	0.00566433929319607\\
376	0.00565621535729714\\
377	0.00564788805911819\\
378	0.00563935099452979\\
379	0.0056305974823533\\
380	0.00562162054289488\\
381	0.00561241287384588\\
382	0.00560296682340784\\
383	0.00559327436064272\\
384	0.00558332704312636\\
385	0.00557311598119243\\
386	0.00556263179377011\\
387	0.0055518645746237\\
388	0.0055408038787054\\
389	0.00552943871860183\\
390	0.00551775757784875\\
391	0.00550574847396325\\
392	0.00549339911978506\\
393	0.00548069695745734\\
394	0.00546762919783081\\
395	0.00545418286695475\\
396	0.00544034485903358\\
397	0.0054261019943971\\
398	0.00541144107993989\\
399	0.00539634896805736\\
400	0.00538081260476591\\
401	0.00536481898707886\\
402	0.00534835499273021\\
403	0.00533140743911178\\
404	0.00531396296337945\\
405	0.00529600816592463\\
406	0.00527752883261459\\
407	0.00525850891416183\\
408	0.00523892916154191\\
409	0.00521876400267892\\
410	0.00519797662168044\\
411	0.00517652869294469\\
412	0.00515438080539511\\
413	0.00513149313071235\\
414	0.0051078266056657\\
415	0.00508334309874557\\
416	0.00505800565036955\\
417	0.00503178177338743\\
418	0.00500464863693634\\
419	0.00497660150946273\\
420	0.00494765572762768\\
421	0.00491785471270213\\
422	0.00488728058215687\\
423	0.00485606816725119\\
424	0.00482455303040914\\
425	0.0047935031221998\\
426	0.00476300169002057\\
427	0.00473313885047253\\
428	0.00470401144068946\\
429	0.00467572247426099\\
430	0.00464837987732808\\
431	0.00462209462361826\\
432	0.00459697793237357\\
433	0.00457313715956891\\
434	0.00455066990763092\\
435	0.00452965565608938\\
436	0.00451014421183016\\
437	0.00449213987423766\\
438	0.00447557991696671\\
439	0.00445976160220178\\
440	0.00444438329017133\\
441	0.00442945124921817\\
442	0.00441496622615378\\
443	0.00440092238274243\\
444	0.00438730644245399\\
445	0.00437409738269767\\
446	0.00436126490607438\\
447	0.00434876889534301\\
448	0.00433655936263782\\
449	0.0043245771894617\\
450	0.00431275609427381\\
451	0.00430102640784474\\
452	0.00428932972198311\\
453	0.0042776548476847\\
454	0.00426598925603829\\
455	0.00425431918821024\\
456	0.00424262985238187\\
457	0.0042309056985038\\
458	0.00421913077775146\\
459	0.0042072891883445\\
460	0.00419536560038398\\
461	0.00418334583722357\\
462	0.00417121746949017\\
463	0.00415897034334175\\
464	0.0041465969142752\\
465	0.00413409127308022\\
466	0.00412144755919448\\
467	0.00410866002540348\\
468	0.00409572310204968\\
469	0.00408263145707985\\
470	0.0040693800473816\\
471	0.00405596415608319\\
472	0.00404237940998745\\
473	0.00402862177137887\\
474	0.00401468749958905\\
475	0.00400057308067138\\
476	0.00398627512949324\\
477	0.00397179029839382\\
478	0.00395711528042726\\
479	0.00394224681010434\\
480	0.00392718166136174\\
481	0.00391191664262769\\
482	0.00389644858905306\\
483	0.00388077435224436\\
484	0.00386489078816142\\
485	0.00384879474420444\\
486	0.00383248304685515\\
487	0.00381595249145306\\
488	0.0037991998355973\\
489	0.00378222179513981\\
490	0.00376501503962558\\
491	0.00374757618723257\\
492	0.00372990179928235\\
493	0.00371198837440514\\
494	0.00369383234244635\\
495	0.00367543005819135\\
496	0.00365677779495725\\
497	0.00363787173804987\\
498	0.00361870797801236\\
499	0.00359928250350329\\
500	0.00357959119355817\\
501	0.00355962980915254\\
502	0.00353939398401156\\
503	0.00351887921460329\\
504	0.00349808084924341\\
505	0.00347699407622836\\
506	0.00345561391090399\\
507	0.00343393518156551\\
508	0.00341195251407696\\
509	0.00338966031509233\\
510	0.00336705275376157\\
511	0.00334412374181056\\
512	0.00332086691189175\\
513	0.00329727559410764\\
514	0.00327334279062074\\
515	0.00324906114828117\\
516	0.00322442292923013\\
517	0.00319941997947536\\
518	0.00317404369548875\\
519	0.00314828498894659\\
520	0.00312213424983111\\
521	0.00309558130823694\\
522	0.00306861539539113\\
523	0.00304122510460897\\
524	0.00301339835318134\\
525	0.00298512234654203\\
526	0.00295638354651234\\
527	0.00292716764599154\\
528	0.00289745955318733\\
529	0.0028672433893974\\
530	0.0028365025055124\\
531	0.00280521952386659\\
532	0.00277337641388418\\
533	0.00274095461225293\\
534	0.00270793520146081\\
535	0.00267429916488165\\
536	0.0026400277413916\\
537	0.00260510289781278\\
538	0.0025695079186138\\
539	0.00253322814582862\\
540	0.00249625223342463\\
541	0.00245857508153023\\
542	0.00242020649288695\\
543	0.00238123570251701\\
544	0.00234160205815798\\
545	0.00230124504563887\\
546	0.00226002773265821\\
547	0.00221783170899319\\
548	0.00217459767527134\\
549	0.0021302495305581\\
550	0.00208471834938659\\
551	0.00203792665729891\\
552	0.00198977932691693\\
553	0.00194012959396351\\
554	0.00189105467690076\\
555	0.0018437709329238\\
556	0.00179836906095719\\
557	0.00175287820462187\\
558	0.00170717331469996\\
559	0.00166116212177324\\
560	0.00161488078334582\\
561	0.00156841503513133\\
562	0.0015224984313825\\
563	0.00147763480850627\\
564	0.00143310019390703\\
565	0.00138847730609411\\
566	0.00134373361196189\\
567	0.00129889564446826\\
568	0.00125400069337501\\
569	0.00120908421228601\\
570	0.00116444936949479\\
571	0.0011201870485504\\
572	0.00107576112986418\\
573	0.00103114982609361\\
574	0.000986381581866229\\
575	0.000941490291903521\\
576	0.000896512873534784\\
577	0.000851489140303056\\
578	0.000806461811104773\\
579	0.000761476490067704\\
580	0.000716581586462604\\
581	0.000671828153146611\\
582	0.000627269616642477\\
583	0.000582961366995044\\
584	0.000538960168256757\\
585	0.00049532334179279\\
586	0.00045210766564535\\
587	0.000409367927260643\\
588	0.000367155072958964\\
589	0.000325513941232703\\
590	0.000284480716949125\\
591	0.000244080684870838\\
592	0.000204328087251261\\
593	0.000165233217676711\\
594	0.000126870159937007\\
595	8.96111722547522e-05\\
596	5.42660945238511e-05\\
597	2.2806228433206e-05\\
598	2.9204464504877e-07\\
599	0\\
600	0\\
};
\addplot [color=black!20!mycolor21,solid,forget plot]
  table[row sep=crcr]{%
1	0.00604041973083569\\
2	0.00604041163451223\\
3	0.00604040336646474\\
4	0.00604039492318405\\
5	0.0060403863010932\\
6	0.00604037749654635\\
7	0.00604036850582758\\
8	0.00604035932514956\\
9	0.00604034995065246\\
10	0.00604034037840258\\
11	0.00604033060439107\\
12	0.00604032062453276\\
13	0.00604031043466472\\
14	0.00604030003054496\\
15	0.00604028940785117\\
16	0.0060402785621792\\
17	0.00604026748904179\\
18	0.00604025618386714\\
19	0.00604024464199752\\
20	0.0060402328586877\\
21	0.00604022082910366\\
22	0.00604020854832101\\
23	0.0060401960113235\\
24	0.00604018321300155\\
25	0.00604017014815062\\
26	0.00604015681146975\\
27	0.00604014319755995\\
28	0.00604012930092258\\
29	0.00604011511595774\\
30	0.0060401006369627\\
31	0.0060400858581301\\
32	0.00604007077354647\\
33	0.00604005537719037\\
34	0.00604003966293073\\
35	0.00604002362452516\\
36	0.00604000725561808\\
37	0.0060399905497391\\
38	0.00603997350030111\\
39	0.00603995610059853\\
40	0.00603993834380532\\
41	0.00603992022297334\\
42	0.00603990173103038\\
43	0.00603988286077825\\
44	0.00603986360489084\\
45	0.00603984395591222\\
46	0.0060398239062547\\
47	0.00603980344819681\\
48	0.00603978257388132\\
49	0.0060397612753132\\
50	0.00603973954435765\\
51	0.00603971737273794\\
52	0.00603969475203342\\
53	0.00603967167367738\\
54	0.00603964812895493\\
55	0.00603962410900086\\
56	0.00603959960479752\\
57	0.00603957460717259\\
58	0.00603954910679688\\
59	0.00603952309418217\\
60	0.00603949655967891\\
61	0.00603946949347404\\
62	0.0060394418855886\\
63	0.00603941372587553\\
64	0.00603938500401733\\
65	0.00603935570952367\\
66	0.00603932583172915\\
67	0.00603929535979078\\
68	0.00603926428268575\\
69	0.00603923258920885\\
70	0.00603920026797016\\
71	0.00603916730739258\\
72	0.00603913369570929\\
73	0.00603909942096132\\
74	0.00603906447099502\\
75	0.0060390288334596\\
76	0.00603899249580445\\
77	0.00603895544527667\\
78	0.00603891766891847\\
79	0.00603887915356454\\
80	0.00603883988583942\\
81	0.0060387998521549\\
82	0.0060387590387073\\
83	0.0060387174314748\\
84	0.00603867501621471\\
85	0.00603863177846084\\
86	0.00603858770352055\\
87	0.00603854277647222\\
88	0.00603849698216225\\
89	0.00603845030520236\\
90	0.00603840272996667\\
91	0.00603835424058895\\
92	0.00603830482095957\\
93	0.00603825445472271\\
94	0.00603820312527338\\
95	0.00603815081575444\\
96	0.00603809750905357\\
97	0.00603804318780031\\
98	0.00603798783436296\\
99	0.0060379314308455\\
100	0.00603787395908446\\
101	0.00603781540064582\\
102	0.00603775573682165\\
103	0.00603769494862711\\
104	0.00603763301679705\\
105	0.00603756992178262\\
106	0.00603750564374808\\
107	0.00603744016256737\\
108	0.0060373734578205\\
109	0.00603730550879026\\
110	0.00603723629445852\\
111	0.00603716579350268\\
112	0.00603709398429197\\
113	0.00603702084488379\\
114	0.00603694635301978\\
115	0.00603687048612213\\
116	0.00603679322128945\\
117	0.00603671453529285\\
118	0.00603663440457184\\
119	0.00603655280523012\\
120	0.00603646971303127\\
121	0.00603638510339442\\
122	0.00603629895138968\\
123	0.00603621123173367\\
124	0.00603612191878477\\
125	0.00603603098653822\\
126	0.00603593840862124\\
127	0.0060358441582879\\
128	0.0060357482084139\\
129	0.00603565053149122\\
130	0.00603555109962257\\
131	0.00603544988451557\\
132	0.00603534685747704\\
133	0.00603524198940678\\
134	0.00603513525079143\\
135	0.00603502661169791\\
136	0.00603491604176677\\
137	0.00603480351020528\\
138	0.00603468898578028\\
139	0.00603457243681071\\
140	0.00603445383116012\\
141	0.00603433313622849\\
142	0.0060342103189442\\
143	0.00603408534575543\\
144	0.00603395818262127\\
145	0.00603382879500259\\
146	0.00603369714785249\\
147	0.00603356320560641\\
148	0.0060334269321718\\
149	0.0060332882909176\\
150	0.00603314724466295\\
151	0.00603300375566584\\
152	0.00603285778561102\\
153	0.0060327092955976\\
154	0.00603255824612614\\
155	0.00603240459708521\\
156	0.00603224830773736\\
157	0.00603208933670466\\
158	0.00603192764195358\\
159	0.00603176318077928\\
160	0.00603159590978937\\
161	0.0060314257848869\\
162	0.00603125276125283\\
163	0.00603107679332769\\
164	0.00603089783479258\\
165	0.00603071583854953\\
166	0.00603053075670093\\
167	0.00603034254052844\\
168	0.00603015114047072\\
169	0.0060299565061008\\
170	0.00602975858610219\\
171	0.00602955732824444\\
172	0.00602935267935756\\
173	0.00602914458530571\\
174	0.00602893299095985\\
175	0.0060287178401695\\
176	0.00602849907573345\\
177	0.00602827663936958\\
178	0.00602805047168352\\
179	0.00602782051213645\\
180	0.00602758669901166\\
181	0.00602734896938018\\
182	0.00602710725906535\\
183	0.00602686150260615\\
184	0.00602661163321953\\
185	0.00602635758276164\\
186	0.00602609928168786\\
187	0.00602583665901176\\
188	0.00602556964226287\\
189	0.0060252981574434\\
190	0.00602502212898375\\
191	0.00602474147969693\\
192	0.00602445613073175\\
193	0.00602416600152512\\
194	0.00602387100975294\\
195	0.00602357107128012\\
196	0.00602326610010947\\
197	0.00602295600832944\\
198	0.00602264070606097\\
199	0.0060223201014031\\
200	0.00602199410037796\\
201	0.00602166260687438\\
202	0.00602132552259091\\
203	0.00602098274697774\\
204	0.00602063417717792\\
205	0.00602027970796776\\
206	0.00601991923169633\\
207	0.00601955263822441\\
208	0.00601917981486271\\
209	0.00601880064630949\\
210	0.00601841501458756\\
211	0.00601802279898083\\
212	0.00601762387597035\\
213	0.00601721811916998\\
214	0.00601680539926162\\
215	0.00601638558393026\\
216	0.0060159585377985\\
217	0.00601552412236119\\
218	0.00601508219591953\\
219	0.00601463261351521\\
220	0.00601417522686442\\
221	0.00601370988429164\\
222	0.00601323643066339\\
223	0.0060127547073219\\
224	0.00601226455201865\\
225	0.00601176579884772\\
226	0.00601125827817919\\
227	0.00601074181659219\\
228	0.00601021623680778\\
229	0.00600968135762151\\
230	0.00600913699383562\\
231	0.00600858295619075\\
232	0.00600801905129705\\
233	0.00600744508156469\\
234	0.0060068608451333\\
235	0.0060062661358006\\
236	0.00600566074294979\\
237	0.00600504445147547\\
238	0.00600441704170816\\
239	0.00600377828933684\\
240	0.00600312796532969\\
241	0.00600246583585235\\
242	0.00600179166218383\\
243	0.00600110520062961\\
244	0.00600040620243174\\
245	0.00599969441367572\\
246	0.00599896957519369\\
247	0.00599823142246401\\
248	0.00599747968550672\\
249	0.00599671408877461\\
250	0.00599593435104\\
251	0.00599514018527647\\
252	0.00599433129853582\\
253	0.00599350739181975\\
254	0.0059926681599462\\
255	0.00599181329141022\\
256	0.00599094246823893\\
257	0.00599005536584096\\
258	0.00598915165284961\\
259	0.00598823099095999\\
260	0.00598729303475947\\
261	0.00598633743155098\\
262	0.0059853638211689\\
263	0.0059843718357903\\
264	0.0059833610997498\\
265	0.00598233122934154\\
266	0.00598128183261407\\
267	0.0059802125091595\\
268	0.00597912284989707\\
269	0.00597801243685141\\
270	0.00597688084292488\\
271	0.00597572763166461\\
272	0.00597455235702391\\
273	0.0059733545631182\\
274	0.00597213378397527\\
275	0.00597088954328013\\
276	0.00596962135411411\\
277	0.00596832871868851\\
278	0.00596701112807231\\
279	0.00596566806191407\\
280	0.0059642989881579\\
281	0.00596290336275316\\
282	0.00596148062935757\\
283	0.00596003021903368\\
284	0.0059585515499382\\
285	0.00595704402700358\\
286	0.00595550704161166\\
287	0.00595393997125856\\
288	0.00595234217921039\\
289	0.00595071301414884\\
290	0.00594905180980607\\
291	0.00594735788458802\\
292	0.00594563054118526\\
293	0.00594386906617028\\
294	0.00594207272958046\\
295	0.00594024078448527\\
296	0.00593837246653645\\
297	0.00593646699349897\\
298	0.00593452356476108\\
299	0.00593254136082331\\
300	0.00593051954277351\\
301	0.00592845725174483\\
302	0.00592635360834158\\
303	0.00592420771203587\\
304	0.00592201864053301\\
305	0.0059197854490976\\
306	0.00591750716982406\\
307	0.00591518281083395\\
308	0.00591281135547901\\
309	0.00591039176156725\\
310	0.00590792296046258\\
311	0.00590540385611892\\
312	0.00590283332406944\\
313	0.00590021021036055\\
314	0.00589753333042427\\
315	0.00589480146791016\\
316	0.00589201337346598\\
317	0.0058891677634614\\
318	0.00588626331865267\\
319	0.00588329868278457\\
320	0.00588027246112666\\
321	0.0058771832189402\\
322	0.00587402947987144\\
323	0.0058708097242668\\
324	0.00586752238740279\\
325	0.00586416585761891\\
326	0.0058607384743371\\
327	0.00585723852597493\\
328	0.00585366424790664\\
329	0.00585001382025991\\
330	0.00584628536560832\\
331	0.00584247694659282\\
332	0.00583858656350006\\
333	0.00583461215180657\\
334	0.00583055157955604\\
335	0.00582640264457449\\
336	0.00582216307181674\\
337	0.0058178305107879\\
338	0.00581340253306314\\
339	0.0058088766299555\\
340	0.00580425021035672\\
341	0.00579952059865069\\
342	0.00579468503224592\\
343	0.00578974065955761\\
344	0.00578468453962854\\
345	0.00577951363976895\\
346	0.00577422483167486\\
347	0.00576881488723711\\
348	0.00576328047393433\\
349	0.00575761814956387\\
350	0.00575182435598938\\
351	0.00574589541383171\\
352	0.00573982751635916\\
353	0.00573361672235197\\
354	0.00572725894847346\\
355	0.00572074996134833\\
356	0.00571408536631671\\
357	0.00570726059326061\\
358	0.00570027088200122\\
359	0.00569311126585978\\
360	0.00568577655283809\\
361	0.00567826130420807\\
362	0.00567055981040699\\
363	0.00566266606433347\\
364	0.00565457373248421\\
365	0.00564627612493743\\
366	0.00563776616607646\\
367	0.00562903636909487\\
368	0.00562007881702173\\
369	0.00561088513657466\\
370	0.00560144652750161\\
371	0.00559175389375945\\
372	0.00558179789037141\\
373	0.00557156895084197\\
374	0.00556105733721312\\
375	0.00555025316917609\\
376	0.00553914644108116\\
377	0.00552772706280232\\
378	0.00551598490973305\\
379	0.00550390987707951\\
380	0.00549149193592538\\
381	0.00547872118685052\\
382	0.00546558790448464\\
383	0.0054520825630505\\
384	0.00543819582848082\\
385	0.00542391849660757\\
386	0.00540924134281841\\
387	0.00539415472574238\\
388	0.00537864806235072\\
389	0.00536270919272062\\
390	0.00534632333267171\\
391	0.00532947120568322\\
392	0.00531212333194729\\
393	0.00529424825184008\\
394	0.00527581284121551\\
395	0.00525678251559689\\
396	0.00523712156378698\\
397	0.00521679365642495\\
398	0.00519576258982919\\
399	0.00517399334701273\\
400	0.00515145359893942\\
401	0.00512811592022666\\
402	0.00510396008989104\\
403	0.00507897457342055\\
404	0.00505316281341298\\
405	0.00502654890918629\\
406	0.00499919242962357\\
407	0.00497119736069615\\
408	0.00494277929217888\\
409	0.0049146626275627\\
410	0.00488691175444974\\
411	0.00485959695242314\\
412	0.00483279448306534\\
413	0.00480658649718681\\
414	0.00478106067117138\\
415	0.00475630951499183\\
416	0.00473242920785781\\
417	0.0047095176753174\\
418	0.00468767160117613\\
419	0.00466698195385279\\
420	0.00464752785040454\\
421	0.00462936801802073\\
422	0.00461252901153922\\
423	0.00459698912504476\\
424	0.00458252040870373\\
425	0.00456843631081279\\
426	0.00455474838043038\\
427	0.00454146452366907\\
428	0.00452858822431064\\
429	0.00451611773544137\\
430	0.00450404487672025\\
431	0.00449235441662874\\
432	0.00448102327740238\\
433	0.00447001987603998\\
434	0.00445930405550695\\
435	0.0044488278667083\\
436	0.00443853618507812\\
437	0.0044283692003319\\
438	0.00441826682854435\\
439	0.00440820415116507\\
440	0.00439817136788632\\
441	0.00438815749405422\\
442	0.0043781504687486\\
443	0.00436813731957653\\
444	0.00435810438547989\\
445	0.00434803759671732\\
446	0.00433792285124789\\
447	0.00432774646157792\\
448	0.00431749565613897\\
449	0.0043071591014906\\
450	0.00429672738417001\\
451	0.00428619335106061\\
452	0.00427555179762931\\
453	0.00426479752868549\\
454	0.00425392541315694\\
455	0.00424293044073412\\
456	0.00423180777714498\\
457	0.00422055281468181\\
458	0.00420916121386276\\
459	0.00419762893149913\\
460	0.00418595223013638\\
461	0.00417412766416368\\
462	0.00416215203923854\\
463	0.0041500223447995\\
464	0.00413773566535698\\
465	0.00412528912332364\\
466	0.00411267988230371\\
467	0.00409990514827473\\
468	0.00408696216842916\\
469	0.00407384822755778\\
470	0.00406056064202319\\
471	0.00404709675159173\\
472	0.00403345390966303\\
473	0.00401962947274033\\
474	0.00400562079028181\\
475	0.00399142519628373\\
476	0.00397704000392844\\
477	0.00396246250337754\\
478	0.0039476899592021\\
479	0.00393271960750239\\
480	0.00391754865278553\\
481	0.00390217426468383\\
482	0.00388659357460285\\
483	0.00387080367238742\\
484	0.00385480160307617\\
485	0.00383858436378452\\
486	0.00382214890070395\\
487	0.00380549210614036\\
488	0.00378861081544281\\
489	0.00377150180369735\\
490	0.00375416178217399\\
491	0.00373658739451004\\
492	0.00371877521260873\\
493	0.00370072173222481\\
494	0.00368242336820189\\
495	0.00366387644931887\\
496	0.00364507721269443\\
497	0.0036260217976926\\
498	0.00360670623926831\\
499	0.0035871264606905\\
500	0.00356727826558349\\
501	0.00354715732922336\\
502	0.00352675918902258\\
503	0.00350607923413049\\
504	0.00348511269407485\\
505	0.00346385462636618\\
506	0.00344229990298504\\
507	0.00342044319567329\\
508	0.00339827895995153\\
509	0.00337580141779204\\
510	0.00335300453888381\\
511	0.00332988202044153\\
512	0.00330642726552966\\
513	0.00328263335990133\\
514	0.00325849304739117\\
515	0.00323399870395426\\
516	0.0032091423105132\\
517	0.00318391542486947\\
518	0.00315830915305462\\
519	0.00313231412065579\\
520	0.00310592044485016\\
521	0.00307911770814284\\
522	0.00305189493513307\\
523	0.00302424057405301\\
524	0.00299614248535714\\
525	0.002967587940314\\
526	0.0029385636334032\\
527	0.00290905571339415\\
528	0.00287904983933047\\
529	0.00284853126932706\\
530	0.0028174849921919\\
531	0.00278589591474815\\
532	0.00275374912178569\\
533	0.00272103023008121\\
534	0.00268772585133614\\
535	0.00265382415755588\\
536	0.00261931558188282\\
537	0.00258419417888923\\
538	0.00254846100558979\\
539	0.00251216052959282\\
540	0.00247532628282182\\
541	0.00243788997122817\\
542	0.00239976415995061\\
543	0.00236075191454028\\
544	0.00232080359268986\\
545	0.00227985780608986\\
546	0.00223785281570349\\
547	0.00219472782639527\\
548	0.00215041177863719\\
549	0.00210482397663955\\
550	0.00205785625152206\\
551	0.00200963534708887\\
552	0.00196294539021133\\
553	0.00191846763346604\\
554	0.00187458162861202\\
555	0.00183052002201858\\
556	0.00178605773398743\\
557	0.00174121417877548\\
558	0.00169604534204883\\
559	0.00165063743477819\\
560	0.00160552143434389\\
561	0.00156143381812764\\
562	0.00151771985037248\\
563	0.00147385304555012\\
564	0.00142980369459937\\
565	0.00138559882693199\\
566	0.00134127297237882\\
567	0.00129686098742586\\
568	0.00125239367639895\\
569	0.00120835592750733\\
570	0.00116438756191665\\
571	0.00112018262134879\\
572	0.00107576048997661\\
573	0.0010311496404989\\
574	0.00098638150776886\\
575	0.000941490258173559\\
576	0.000896512856511153\\
577	0.000851489130996976\\
578	0.000806461805889532\\
579	0.000761476487129886\\
580	0.000716581584829257\\
581	0.000671828152282903\\
582	0.00062726961623216\\
583	0.000582961366832659\\
584	0.000538960168209238\\
585	0.000495323341784078\\
586	0.000452107665645342\\
587	0.000409367927260636\\
588	0.000367155072958956\\
589	0.0003255139412327\\
590	0.000284480716949123\\
591	0.000244080684870839\\
592	0.000204328087251262\\
593	0.000165233217676712\\
594	0.000126870159937008\\
595	8.96111722547525e-05\\
596	5.4266094523851e-05\\
597	2.28062284332056e-05\\
598	2.9204464504877e-07\\
599	0\\
600	0\\
};
\addplot [color=black!50!mycolor20,solid,forget plot]
  table[row sep=crcr]{%
1	0.00604035455130753\\
2	0.00604034491250777\\
3	0.00604033506129769\\
4	0.00604032499311465\\
5	0.00604031470330184\\
6	0.00604030418710641\\
7	0.00604029343967774\\
8	0.00604028245606537\\
9	0.00604027123121725\\
10	0.00604025975997777\\
11	0.00604024803708565\\
12	0.00604023605717213\\
13	0.00604022381475875\\
14	0.00604021130425541\\
15	0.00604019851995815\\
16	0.00604018545604706\\
17	0.00604017210658414\\
18	0.006040158465511\\
19	0.00604014452664662\\
20	0.00604013028368517\\
21	0.00604011573019356\\
22	0.00604010085960915\\
23	0.00604008566523733\\
24	0.00604007014024908\\
25	0.00604005427767858\\
26	0.00604003807042058\\
27	0.00604002151122794\\
28	0.00604000459270902\\
29	0.00603998730732511\\
30	0.00603996964738761\\
31	0.00603995160505559\\
32	0.00603993317233276\\
33	0.00603991434106494\\
34	0.00603989510293712\\
35	0.00603987544947057\\
36	0.00603985537202005\\
37	0.0060398348617708\\
38	0.00603981390973551\\
39	0.00603979250675143\\
40	0.00603977064347726\\
41	0.00603974831039\\
42	0.00603972549778182\\
43	0.0060397021957569\\
44	0.0060396783942282\\
45	0.00603965408291424\\
46	0.00603962925133564\\
47	0.00603960388881196\\
48	0.00603957798445819\\
49	0.00603955152718133\\
50	0.00603952450567691\\
51	0.00603949690842561\\
52	0.00603946872368945\\
53	0.00603943993950841\\
54	0.0060394105436967\\
55	0.00603938052383908\\
56	0.00603934986728715\\
57	0.00603931856115561\\
58	0.00603928659231845\\
59	0.00603925394740509\\
60	0.00603922061279656\\
61	0.00603918657462144\\
62	0.00603915181875213\\
63	0.00603911633080062\\
64	0.00603908009611462\\
65	0.00603904309977344\\
66	0.00603900532658388\\
67	0.00603896676107607\\
68	0.00603892738749933\\
69	0.00603888718981799\\
70	0.00603884615170701\\
71	0.00603880425654783\\
72	0.00603876148742409\\
73	0.00603871782711716\\
74	0.00603867325810185\\
75	0.00603862776254197\\
76	0.00603858132228596\\
77	0.00603853391886239\\
78	0.00603848553347546\\
79	0.00603843614700055\\
80	0.00603838573997964\\
81	0.00603833429261675\\
82	0.00603828178477338\\
83	0.00603822819596396\\
84	0.00603817350535117\\
85	0.00603811769174128\\
86	0.00603806073357962\\
87	0.00603800260894578\\
88	0.00603794329554906\\
89	0.00603788277072365\\
90	0.00603782101142409\\
91	0.00603775799422036\\
92	0.00603769369529337\\
93	0.00603762809043015\\
94	0.0060375611550191\\
95	0.00603749286404532\\
96	0.00603742319208591\\
97	0.00603735211330515\\
98	0.0060372796014499\\
99	0.00603720562984483\\
100	0.00603713017138775\\
101	0.00603705319854488\\
102	0.00603697468334623\\
103	0.00603689459738088\\
104	0.00603681291179239\\
105	0.00603672959727415\\
106	0.00603664462406469\\
107	0.0060365579619432\\
108	0.00603646958022498\\
109	0.00603637944775674\\
110	0.00603628753291225\\
111	0.0060361938035878\\
112	0.0060360982271977\\
113	0.0060360007706699\\
114	0.00603590140044156\\
115	0.00603580008245474\\
116	0.00603569678215202\\
117	0.00603559146447224\\
118	0.00603548409384623\\
119	0.00603537463419263\\
120	0.00603526304891366\\
121	0.00603514930089098\\
122	0.00603503335248161\\
123	0.0060349151655138\\
124	0.00603479470128311\\
125	0.00603467192054829\\
126	0.00603454678352742\\
127	0.00603441924989387\\
128	0.00603428927877249\\
129	0.00603415682873569\\
130	0.00603402185779955\\
131	0.0060338843234201\\
132	0.00603374418248944\\
133	0.00603360139133191\\
134	0.00603345590570039\\
135	0.00603330768077241\\
136	0.00603315667114645\\
137	0.00603300283083806\\
138	0.00603284611327618\\
139	0.00603268647129906\\
140	0.00603252385715058\\
141	0.00603235822247626\\
142	0.00603218951831917\\
143	0.00603201769511605\\
144	0.00603184270269301\\
145	0.0060316644902614\\
146	0.00603148300641334\\
147	0.00603129819911738\\
148	0.00603111001571373\\
149	0.00603091840290947\\
150	0.00603072330677369\\
151	0.00603052467273213\\
152	0.00603032244556186\\
153	0.00603011656938548\\
154	0.00602990698766532\\
155	0.00602969364319684\\
156	0.00602947647810232\\
157	0.0060292554338236\\
158	0.00602903045111478\\
159	0.00602880147003427\\
160	0.0060285684299365\\
161	0.00602833126946292\\
162	0.00602808992653266\\
163	0.00602784433833239\\
164	0.00602759444130564\\
165	0.00602734017114131\\
166	0.00602708146276157\\
167	0.00602681825030869\\
168	0.00602655046713143\\
169	0.00602627804576997\\
170	0.00602600091794035\\
171	0.00602571901451754\\
172	0.00602543226551752\\
173	0.00602514060007818\\
174	0.006024843946439\\
175	0.00602454223191927\\
176	0.00602423538289514\\
177	0.00602392332477499\\
178	0.00602360598197339\\
179	0.00602328327788333\\
180	0.00602295513484693\\
181	0.00602262147412411\\
182	0.00602228221585955\\
183	0.00602193727904763\\
184	0.00602158658149546\\
185	0.00602123003978353\\
186	0.00602086756922437\\
187	0.00602049908381879\\
188	0.00602012449620984\\
189	0.006019743717634\\
190	0.00601935665787035\\
191	0.00601896322518649\\
192	0.00601856332628224\\
193	0.00601815686623016\\
194	0.00601774374841347\\
195	0.00601732387446088\\
196	0.00601689714417841\\
197	0.00601646345547823\\
198	0.00601602270430429\\
199	0.00601557478455476\\
200	0.00601511958800134\\
201	0.00601465700420523\\
202	0.00601418692042981\\
203	0.00601370922155019\\
204	0.00601322378995926\\
205	0.00601273050547047\\
206	0.0060122292452175\\
207	0.00601171988355051\\
208	0.00601120229192933\\
209	0.0060106763388133\\
210	0.00601014188954833\\
211	0.00600959880625064\\
212	0.00600904694768804\\
213	0.00600848616915798\\
214	0.00600791632236334\\
215	0.00600733725528554\\
216	0.00600674881205547\\
217	0.00600615083282223\\
218	0.00600554315362001\\
219	0.00600492560623318\\
220	0.00600429801805994\\
221	0.00600366021197484\\
222	0.00600301200619012\\
223	0.00600235321411653\\
224	0.00600168364422354\\
225	0.0060010030998996\\
226	0.00600031137931222\\
227	0.0059996082752687\\
228	0.0059988935750774\\
229	0.00599816706040992\\
230	0.0059974285071646\\
231	0.0059966776853314\\
232	0.00599591435885837\\
233	0.00599513828552009\\
234	0.00599434921678806\\
235	0.00599354689770343\\
236	0.00599273106675183\\
237	0.00599190145574067\\
238	0.00599105778967874\\
239	0.00599019978665826\\
240	0.00598932715773907\\
241	0.00598843960683508\\
242	0.00598753683060262\\
243	0.00598661851833054\\
244	0.00598568435183174\\
245	0.00598473400533584\\
246	0.00598376714538252\\
247	0.00598278343071531\\
248	0.00598178251217499\\
249	0.00598076403259272\\
250	0.00597972762668163\\
251	0.00597867292092701\\
252	0.00597759953347416\\
253	0.00597650707401343\\
254	0.00597539514366176\\
255	0.00597426333484019\\
256	0.00597311123114652\\
257	0.005971938407222\\
258	0.00597074442861153\\
259	0.00596952885161539\\
260	0.00596829122313056\\
261	0.00596703108047722\\
262	0.00596574795120211\\
263	0.00596444135284604\\
264	0.00596311079268821\\
265	0.00596175576758368\\
266	0.00596037576370285\\
267	0.00595897025622848\\
268	0.00595753870903681\\
269	0.00595608057436188\\
270	0.00595459529244277\\
271	0.00595308229115259\\
272	0.00595154098560904\\
273	0.00594997077776536\\
274	0.00594837105598171\\
275	0.00594674119457572\\
276	0.00594508055335192\\
277	0.00594338847710932\\
278	0.00594166429512658\\
279	0.00593990732062399\\
280	0.00593811685020162\\
281	0.00593629216325293\\
282	0.00593443252135306\\
283	0.00593253716762066\\
284	0.00593060532605261\\
285	0.00592863620083044\\
286	0.00592662897559697\\
287	0.00592458281270248\\
288	0.00592249685241824\\
289	0.00592037021211663\\
290	0.00591820198541585\\
291	0.00591599124128818\\
292	0.00591373702313046\\
293	0.00591143834779619\\
294	0.00590909420458879\\
295	0.00590670355421709\\
296	0.00590426532771432\\
297	0.00590177842532272\\
298	0.00589924171534324\\
299	0.00589665403294819\\
300	0.00589401417896816\\
301	0.00589132091877082\\
302	0.00588857298127079\\
303	0.00588576905794149\\
304	0.00588290780192623\\
305	0.0058799878273027\\
306	0.00587700770851719\\
307	0.00587396597990175\\
308	0.00587086113495447\\
309	0.00586769162621109\\
310	0.00586445586609552\\
311	0.00586115222654217\\
312	0.00585777903781928\\
313	0.0058543345873161\\
314	0.00585081711820609\\
315	0.00584722482787601\\
316	0.00584355586639754\\
317	0.00583980833493617\\
318	0.00583598028405877\\
319	0.00583206971192252\\
320	0.00582807456232503\\
321	0.0058239927225891\\
322	0.00581982202124891\\
323	0.00581556022549524\\
324	0.00581120503832473\\
325	0.0058067540953129\\
326	0.00580220496087143\\
327	0.00579755512370845\\
328	0.00579280199116947\\
329	0.00578794288452105\\
330	0.00578297503252372\\
331	0.00577789556368191\\
332	0.00577270149724791\\
333	0.00576738973298516\\
334	0.00576195703974413\\
335	0.00575640004094692\\
336	0.00575071519598767\\
337	0.00574489878061114\\
338	0.0057389468653358\\
339	0.00573285529206363\\
340	0.0057266196498375\\
341	0.00572023525148013\\
342	0.00571369711339678\\
343	0.00570699993620161\\
344	0.00570013810097834\\
345	0.00569310572673465\\
346	0.00568589673648015\\
347	0.0056785048609832\\
348	0.00567092364497663\\
349	0.00566314645599178\\
350	0.00565516649457982\\
351	0.00564697679917844\\
352	0.0056385702728284\\
353	0.00562993971370758\\
354	0.00562107784745955\\
355	0.00561197736864539\\
356	0.00560263100200876\\
357	0.00559303154919701\\
358	0.00558317192088877\\
359	0.00557304518645487\\
360	0.00556264462551128\\
361	0.00555196377299146\\
362	0.00554099644978443\\
363	0.00552973676717171\\
364	0.0055181790880087\\
365	0.00550631792034881\\
366	0.00549414770945814\\
367	0.00548166248157145\\
368	0.00546885527818469\\
369	0.00545571730669379\\
370	0.00544223620615092\\
371	0.00542839179521135\\
372	0.00541416081700759\\
373	0.00539951832034958\\
374	0.00538443766169766\\
375	0.00536889081424874\\
376	0.00535284846039011\\
377	0.00533627990800812\\
378	0.00531915330324797\\
379	0.00530143601499595\\
380	0.00528309520580291\\
381	0.005264098651944\\
382	0.00524441589473051\\
383	0.00522401983064083\\
384	0.00520288888166286\\
385	0.00518100993417347\\
386	0.00515838231517751\\
387	0.00513502332663846\\
388	0.0051109739445012\\
389	0.00508630711750994\\
390	0.00506114078397471\\
391	0.00503610410007305\\
392	0.00501125655682597\\
393	0.00498664792446473\\
394	0.00496233283192958\\
395	0.00493837095736671\\
396	0.00491482711891741\\
397	0.00489177121564937\\
398	0.00486927795068279\\
399	0.00484742624496875\\
400	0.00482629821873276\\
401	0.0048059775741019\\
402	0.00478654718902244\\
403	0.00476808569265558\\
404	0.00475066251264518\\
405	0.00473433095851641\\
406	0.00471911843610295\\
407	0.00470501322286141\\
408	0.00469189080838035\\
409	0.00467909168690972\\
410	0.00466662958433403\\
411	0.0046545159743818\\
412	0.00464275949432454\\
413	0.00463136529122577\\
414	0.00462033432580973\\
415	0.00460966263109742\\
416	0.00459934058267878\\
417	0.00458935216047305\\
418	0.00457967434280781\\
419	0.00457027675662039\\
420	0.00456112164386289\\
421	0.00455216435069586\\
422	0.00454335500797968\\
423	0.00453464140084274\\
424	0.00452598108331289\\
425	0.00451736707410352\\
426	0.00450879128376258\\
427	0.0045002445499172\\
428	0.00449171671181201\\
429	0.00448319673197135\\
430	0.00447467287969468\\
431	0.0044661329667574\\
432	0.00445756464614725\\
433	0.00444895577249972\\
434	0.00444029480843122\\
435	0.004431571252739\\
436	0.00442277607768038\\
437	0.00441390208889142\\
438	0.00440494409640422\\
439	0.00439589753527377\\
440	0.00438675784669873\\
441	0.00437752052518183\\
442	0.0043681811668333\\
443	0.00435873551675346\\
444	0.00434917951305033\\
445	0.0043395093245137\\
446	0.00432972137728854\\
447	0.00431981236625983\\
448	0.00430977924699373\\
449	0.00429961920502578\\
450	0.00428932960160009\\
451	0.00427890789942998\\
452	0.00426835159385037\\
453	0.00425765821680932\\
454	0.00424682533924926\\
455	0.00423585057161851\\
456	0.00422473156234556\\
457	0.00421346599421916\\
458	0.00420205157877209\\
459	0.00419048604896569\\
460	0.00417876715071148\\
461	0.00416689263402309\\
462	0.00415486024482622\\
463	0.00414266771859025\\
464	0.00413031277684553\\
465	0.00411779312564403\\
466	0.00410510645374804\\
467	0.0040922504305949\\
468	0.00407922270410051\\
469	0.00406602089837641\\
470	0.00405264261144239\\
471	0.00403908541301646\\
472	0.00402534684245244\\
473	0.00401142440687092\\
474	0.00399731557948974\\
475	0.00398301779810677\\
476	0.0039685284636279\\
477	0.00395384493851744\\
478	0.0039389645451738\\
479	0.00392388456423045\\
480	0.00390860223278089\\
481	0.00389311474252117\\
482	0.00387741923780042\\
483	0.00386151281356359\\
484	0.00384539251316722\\
485	0.00382905532604283\\
486	0.00381249818518186\\
487	0.00379571796441353\\
488	0.00377871147545074\\
489	0.00376147546468013\\
490	0.00374400660966948\\
491	0.00372630151536363\\
492	0.00370835670993593\\
493	0.0036901686402604\\
494	0.00367173366696607\\
495	0.00365304805903214\\
496	0.00363410798788008\\
497	0.00361490952091579\\
498	0.00359544861447277\\
499	0.00357572110610464\\
500	0.00355572270617314\\
501	0.0035354489886765\\
502	0.00351489538126131\\
503	0.00349405715436267\\
504	0.00347292940941807\\
505	0.00345150706610567\\
506	0.00342978484856397\\
507	0.00340775727056105\\
508	0.00338541861959726\\
509	0.00336276293994635\\
510	0.00333978401467161\\
511	0.00331647534669245\\
512	0.00329283013903121\\
513	0.00326884127444044\\
514	0.0032445012947022\\
515	0.00321980238000906\\
516	0.00319473632898957\\
517	0.00316929454013484\\
518	0.00314346799563162\\
519	0.00311724724892223\\
520	0.00309062241771182\\
521	0.00306358318464741\\
522	0.00303611880853064\\
523	0.00300821814972765\\
524	0.0029798697144472\\
525	0.00295106172382384\\
526	0.00292178221532365\\
527	0.00289201918595378\\
528	0.00286176078938083\\
529	0.00283099560279367\\
530	0.00279971298377082\\
531	0.00276790353115984\\
532	0.00273555964225991\\
533	0.00270267619243534\\
534	0.00266925192065413\\
535	0.00263529296920442\\
536	0.00260086606279785\\
537	0.00256597321413507\\
538	0.00253054433721722\\
539	0.00249444974095169\\
540	0.00245754414628867\\
541	0.00241976540072678\\
542	0.00238103965629058\\
543	0.0023413125141704\\
544	0.00230052829688574\\
545	0.00225862371766199\\
546	0.00221552624303288\\
547	0.00217115084894059\\
548	0.00212537826171806\\
549	0.00207910769893968\\
550	0.00203457920162987\\
551	0.00199229323501405\\
552	0.00194984963616151\\
553	0.00190703680818334\\
554	0.00186376888205866\\
555	0.00182006478878576\\
556	0.00177598546288734\\
557	0.00173161014088666\\
558	0.00168713765704981\\
559	0.00164367024345205\\
560	0.00160078372768793\\
561	0.00155771264955085\\
562	0.00151441038560274\\
563	0.00147090213189777\\
564	0.00142722036978999\\
565	0.00138339816260417\\
566	0.00133946689808092\\
567	0.00129553652878565\\
568	0.00125204062884596\\
569	0.00120834736434655\\
570	0.00116438695026368\\
571	0.00112018252877128\\
572	0.00107576046202357\\
573	0.00103114962906728\\
574	0.000986381502459647\\
575	0.000941490255454449\\
576	0.000896512855016981\\
577	0.000851489130158736\\
578	0.00080646180541855\\
579	0.000761476486870332\\
580	0.000716581584693804\\
581	0.000671828152219676\\
582	0.000627269616207832\\
583	0.000582961366825552\\
584	0.000538960168207982\\
585	0.000495323341784074\\
586	0.000452107665645344\\
587	0.000409367927260635\\
588	0.000367155072958959\\
589	0.0003255139412327\\
590	0.00028448071694912\\
591	0.000244080684870837\\
592	0.000204328087251259\\
593	0.00016523321767671\\
594	0.000126870159937007\\
595	8.96111722547517e-05\\
596	5.42660945238506e-05\\
597	2.28062284332055e-05\\
598	2.9204464504877e-07\\
599	0\\
600	0\\
};
\addplot [color=black!60!mycolor21,solid,forget plot]
  table[row sep=crcr]{%
1	0.00604030315615854\\
2	0.00604029213280765\\
3	0.00604028085850546\\
4	0.00604026932762131\\
5	0.00604025753440102\\
6	0.00604024547296424\\
7	0.00604023313730183\\
8	0.0060402205212733\\
9	0.00604020761860375\\
10	0.00604019442288126\\
11	0.00604018092755395\\
12	0.00604016712592705\\
13	0.00604015301115993\\
14	0.00604013857626299\\
15	0.00604012381409466\\
16	0.00604010871735813\\
17	0.0060400932785982\\
18	0.00604007749019795\\
19	0.00604006134437541\\
20	0.00604004483318011\\
21	0.00604002794848966\\
22	0.00604001068200617\\
23	0.00603999302525264\\
24	0.00603997496956928\\
25	0.00603995650610979\\
26	0.00603993762583747\\
27	0.00603991831952148\\
28	0.00603989857773269\\
29	0.00603987839083977\\
30	0.00603985774900511\\
31	0.00603983664218049\\
32	0.00603981506010304\\
33	0.00603979299229069\\
34	0.00603977042803793\\
35	0.00603974735641119\\
36	0.00603972376624438\\
37	0.00603969964613416\\
38	0.00603967498443524\\
39	0.0060396497692556\\
40	0.00603962398845151\\
41	0.00603959762962259\\
42	0.00603957068010675\\
43	0.00603954312697506\\
44	0.00603951495702644\\
45	0.00603948615678234\\
46	0.0060394567124814\\
47	0.0060394266100738\\
48	0.00603939583521576\\
49	0.00603936437326383\\
50	0.0060393322092691\\
51	0.00603929932797119\\
52	0.0060392657137925\\
53	0.00603923135083194\\
54	0.0060391962228589\\
55	0.00603916031330686\\
56	0.00603912360526713\\
57	0.0060390860814823\\
58	0.00603904772433975\\
59	0.00603900851586495\\
60	0.00603896843771466\\
61	0.00603892747117013\\
62	0.00603888559713016\\
63	0.00603884279610392\\
64	0.00603879904820394\\
65	0.00603875433313866\\
66	0.00603870863020521\\
67	0.00603866191828188\\
68	0.00603861417582054\\
69	0.00603856538083893\\
70	0.00603851551091298\\
71	0.00603846454316885\\
72	0.00603841245427489\\
73	0.0060383592204337\\
74	0.00603830481737387\\
75	0.00603824922034165\\
76	0.00603819240409257\\
77	0.00603813434288305\\
78	0.00603807501046174\\
79	0.00603801438006077\\
80	0.00603795242438711\\
81	0.00603788911561362\\
82	0.00603782442537004\\
83	0.00603775832473398\\
84	0.00603769078422181\\
85	0.0060376217737793\\
86	0.00603755126277238\\
87	0.00603747921997769\\
88	0.00603740561357308\\
89	0.00603733041112803\\
90	0.00603725357959399\\
91	0.00603717508529463\\
92	0.00603709489391608\\
93	0.00603701297049695\\
94	0.00603692927941852\\
95	0.00603684378439462\\
96	0.00603675644846161\\
97	0.00603666723396823\\
98	0.00603657610256547\\
99	0.00603648301519629\\
100	0.00603638793208539\\
101	0.00603629081272888\\
102	0.00603619161588401\\
103	0.00603609029955871\\
104	0.00603598682100124\\
105	0.00603588113668981\\
106	0.00603577320232221\\
107	0.00603566297280529\\
108	0.0060355504022446\\
109	0.00603543544393402\\
110	0.00603531805034534\\
111	0.00603519817311788\\
112	0.00603507576304822\\
113	0.00603495077007992\\
114	0.00603482314329318\\
115	0.00603469283089476\\
116	0.00603455978020791\\
117	0.00603442393766219\\
118	0.00603428524878369\\
119	0.00603414365818501\\
120	0.00603399910955568\\
121	0.0060338515456524\\
122	0.00603370090828966\\
123	0.00603354713833032\\
124	0.00603339017567638\\
125	0.00603322995925995\\
126	0.00603306642703442\\
127	0.00603289951596574\\
128	0.00603272916202395\\
129	0.00603255530017487\\
130	0.0060323778643721\\
131	0.00603219678754922\\
132	0.00603201200161214\\
133	0.00603182343743191\\
134	0.0060316310248376\\
135	0.0060314346926097\\
136	0.00603123436847357\\
137	0.0060310299790934\\
138	0.00603082145006639\\
139	0.0060306087059174\\
140	0.00603039167009382\\
141	0.00603017026496083\\
142	0.00602994441179719\\
143	0.00602971403079119\\
144	0.00602947904103725\\
145	0.00602923936053267\\
146	0.00602899490617513\\
147	0.00602874559376031\\
148	0.00602849133798023\\
149	0.0060282320524218\\
150	0.00602796764956598\\
151	0.00602769804078748\\
152	0.00602742313635467\\
153	0.00602714284543029\\
154	0.00602685707607232\\
155	0.00602656573523567\\
156	0.00602626872877395\\
157	0.00602596596144213\\
158	0.00602565733689928\\
159	0.00602534275771209\\
160	0.00602502212535866\\
161	0.00602469534023269\\
162	0.00602436230164816\\
163	0.00602402290784449\\
164	0.00602367705599188\\
165	0.00602332464219712\\
166	0.00602296556150962\\
167	0.00602259970792783\\
168	0.00602222697440578\\
169	0.00602184725285984\\
170	0.00602146043417563\\
171	0.00602106640821494\\
172	0.00602066506382275\\
173	0.00602025628883411\\
174	0.00601983997008088\\
175	0.00601941599339837\\
176	0.00601898424363137\\
177	0.00601854460464016\\
178	0.0060180969593057\\
179	0.00601764118953426\\
180	0.00601717717626128\\
181	0.00601670479945442\\
182	0.00601622393811539\\
183	0.00601573447028078\\
184	0.00601523627302126\\
185	0.0060147292224395\\
186	0.00601421319366627\\
187	0.00601368806085453\\
188	0.00601315369717146\\
189	0.00601260997478819\\
190	0.00601205676486678\\
191	0.00601149393754458\\
192	0.00601092136191525\\
193	0.00601033890600672\\
194	0.00600974643675521\\
195	0.00600914381997563\\
196	0.00600853092032748\\
197	0.00600790760127641\\
198	0.00600727372505088\\
199	0.00600662915259358\\
200	0.00600597374350736\\
201	0.00600530735599531\\
202	0.00600462984679469\\
203	0.00600394107110405\\
204	0.00600324088250367\\
205	0.00600252913286865\\
206	0.00600180567227416\\
207	0.00600107034889287\\
208	0.00600032300888391\\
209	0.00599956349627308\\
210	0.00599879165282408\\
211	0.00599800731790024\\
212	0.00599721032831651\\
213	0.00599640051818154\\
214	0.00599557771872936\\
215	0.00599474175814054\\
216	0.00599389246135252\\
217	0.00599302964985902\\
218	0.00599215314149819\\
219	0.00599126275022954\\
220	0.00599035828589953\\
221	0.00598943955399564\\
222	0.00598850635538901\\
223	0.00598755848606573\\
224	0.00598659573684676\\
225	0.00598561789309678\\
226	0.00598462473442191\\
227	0.00598361603435678\\
228	0.0059825915600411\\
229	0.00598155107188605\\
230	0.00598049432323093\\
231	0.00597942105999027\\
232	0.00597833102029219\\
233	0.00597722393410799\\
234	0.00597609952287374\\
235	0.00597495749910423\\
236	0.00597379756599957\\
237	0.005972619417045\\
238	0.00597142273560416\\
239	0.00597020719450606\\
240	0.00596897245562604\\
241	0.00596771816946078\\
242	0.00596644397469735\\
243	0.00596514949777626\\
244	0.00596383435244846\\
245	0.00596249813932576\\
246	0.00596114044542479\\
247	0.00595976084370408\\
248	0.00595835889259425\\
249	0.00595693413552139\\
250	0.00595548610042442\\
251	0.00595401429926767\\
252	0.00595251822755056\\
253	0.00595099736381837\\
254	0.00594945116917886\\
255	0.00594787908683215\\
256	0.00594628054162402\\
257	0.00594465493963533\\
258	0.00594300166782471\\
259	0.00594132009374463\\
260	0.00593960956535445\\
261	0.00593786941095441\\
262	0.005936098939254\\
263	0.00593429743954531\\
264	0.00593246418184334\\
265	0.00593059841689974\\
266	0.00592869937769454\\
267	0.00592676628006091\\
268	0.00592479832244917\\
269	0.00592279468568871\\
270	0.00592075453274704\\
271	0.00591867700848528\\
272	0.00591656123940989\\
273	0.00591440633341983\\
274	0.00591221137954805\\
275	0.00590997544769704\\
276	0.00590769758836703\\
277	0.00590537683237556\\
278	0.00590301219056673\\
279	0.00590060265350832\\
280	0.00589814719117372\\
281	0.00589564475260517\\
282	0.00589309426555345\\
283	0.00589049463608816\\
284	0.00588784474817003\\
285	0.00588514346317461\\
286	0.00588238961935384\\
287	0.00587958203121711\\
288	0.00587671948880924\\
289	0.00587380075685598\\
290	0.00587082457374034\\
291	0.00586778965026332\\
292	0.0058646946681321\\
293	0.00586153827810502\\
294	0.00585831909770823\\
295	0.00585503570842117\\
296	0.0058516866522094\\
297	0.00584827042726107\\
298	0.005844785482755\\
299	0.00584123021243587\\
300	0.00583760294666414\\
301	0.00583390194251364\\
302	0.00583012537284562\\
303	0.00582627131476884\\
304	0.00582233773560924\\
305	0.00581832247727876\\
306	0.00581422323984336\\
307	0.00581003756541961\\
308	0.00580576282366509\\
309	0.00580139619656889\\
310	0.00579693467624333\\
311	0.00579237510283586\\
312	0.00578771420092257\\
313	0.00578294857752554\\
314	0.00577807472093154\\
315	0.00577308899970213\\
316	0.00576798766021564\\
317	0.00576276682629832\\
318	0.00575742250012165\\
319	0.00575195056427633\\
320	0.0057463467852586\\
321	0.00574060681863975\\
322	0.00573472621622314\\
323	0.00572870043553005\\
324	0.00572252485198907\\
325	0.00571619477423348\\
326	0.0057097054628966\\
327	0.00570305215303784\\
328	0.00569623007880404\\
329	0.00568923449356538\\
330	0.00568206071621986\\
331	0.00567470417398735\\
332	0.00566716044728615\\
333	0.0056594253182353\\
334	0.00565149482309718\\
335	0.00564336531262988\\
336	0.00563503349789744\\
337	0.00562649645778915\\
338	0.00561775163546331\\
339	0.00560879679679947\\
340	0.00559962992549786\\
341	0.00559024902400292\\
342	0.00558065177867284\\
343	0.00557083503636195\\
344	0.00556079387954658\\
345	0.00555051953043704\\
346	0.00553999868927008\\
347	0.00552921699659402\\
348	0.00551815895815368\\
349	0.00550680787097888\\
350	0.00549514575425154\\
351	0.00548315328290356\\
352	0.00547080958467503\\
353	0.00545809225929875\\
354	0.00544497741749707\\
355	0.00543143971394483\\
356	0.00541745244928231\\
357	0.00540298796483707\\
358	0.00538801795719036\\
359	0.00537251368367673\\
360	0.00535644652525741\\
361	0.00533978882818451\\
362	0.00532251507603197\\
363	0.00530460349848151\\
364	0.00528603825579737\\
365	0.00526681238042775\\
366	0.00524693171268722\\
367	0.00522642014082995\\
368	0.00520532655642379\\
369	0.00518373409893584\\
370	0.00516206586647816\\
371	0.00514044460870371\\
372	0.00511889942685035\\
373	0.005097461684701\\
374	0.00507616541703685\\
375	0.0050550469394263\\
376	0.00503414834652415\\
377	0.00501351824451031\\
378	0.00499320950024386\\
379	0.0049732792138121\\
380	0.00495378852032667\\
381	0.00493480214886549\\
382	0.00491638764213609\\
383	0.00489861409876216\\
384	0.0048815502818452\\
385	0.00486526185955928\\
386	0.00484980747629532\\
387	0.0048352332527258\\
388	0.00482156524752625\\
389	0.0048087991710341\\
390	0.00479688634022394\\
391	0.00478524062734394\\
392	0.004773860368139\\
393	0.00476275899406963\\
394	0.00475194851101444\\
395	0.00474143907290504\\
396	0.00473123849955345\\
397	0.00472135174083783\\
398	0.00471178029418218\\
399	0.0047025215893457\\
400	0.0046935683648113\\
401	0.00468490807480182\\
402	0.0046765223856452\\
403	0.00466838684684614\\
404	0.00466047086612644\\
405	0.00465273816968375\\
406	0.00464514801062233\\
407	0.00463765747607518\\
408	0.00463022821741895\\
409	0.00462285557451509\\
410	0.00461553393591348\\
411	0.00460825674039417\\
412	0.00460101649287061\\
413	0.00459380480350022\\
414	0.00458661236496382\\
415	0.00457942916295125\\
416	0.00457224459312569\\
417	0.00456504808471705\\
418	0.00455782910114952\\
419	0.00455057744375803\\
420	0.00454328362249896\\
421	0.00453593924048389\\
422	0.00452853733886829\\
423	0.00452107263056984\\
424	0.00451354120769885\\
425	0.00450593911397278\\
426	0.00449826238226361\\
427	0.00449050707496815\\
428	0.0044826693259891\\
429	0.00447474538262651\\
430	0.00446673164499422\\
431	0.00445862470053646\\
432	0.00445042135051302\\
433	0.00444211862500887\\
434	0.00443371378317185\\
435	0.00442520429581949\\
436	0.00441658780808193\\
437	0.00440786208314939\\
438	0.00439902493336187\\
439	0.00439007420042411\\
440	0.00438100775895551\\
441	0.00437182351869122\\
442	0.00436251942513567\\
443	0.00435309345852253\\
444	0.00434354363100991\\
445	0.00433386798214939\\
446	0.00432406457285262\\
447	0.00431413147827598\\
448	0.00430406678026078\\
449	0.00429386856017406\\
450	0.00428353489313293\\
451	0.00427306384456706\\
452	0.0042624534691652\\
453	0.00425170180958279\\
454	0.00424080689494397\\
455	0.00422976673918413\\
456	0.00421857933929183\\
457	0.00420724267351705\\
458	0.00419575469961823\\
459	0.00418411335321687\\
460	0.00417231654631692\\
461	0.00416036216602138\\
462	0.00414824807344308\\
463	0.00413597210275987\\
464	0.00412353206031631\\
465	0.0041109257237\\
466	0.0040981508407988\\
467	0.00408520512884452\\
468	0.00407208627344699\\
469	0.00405879192761907\\
470	0.00404531971079127\\
471	0.00403166720781047\\
472	0.00401783196791326\\
473	0.00400381150366288\\
474	0.00398960328983551\\
475	0.0039752047622424\\
476	0.0039606133164768\\
477	0.00394582630657787\\
478	0.00393084104360233\\
479	0.00391565479409372\\
480	0.00390026477843745\\
481	0.00388466816908875\\
482	0.00386886208865883\\
483	0.00385284360784337\\
484	0.00383660974317631\\
485	0.00382015745459029\\
486	0.00380348364276359\\
487	0.00378658514623244\\
488	0.00376945873824518\\
489	0.0037521011233333\\
490	0.00373450893357197\\
491	0.00371667872450067\\
492	0.00369860697067291\\
493	0.0036802900608012\\
494	0.00366172429246222\\
495	0.00364290586632503\\
496	0.00362383087986406\\
497	0.00360449532051765\\
498	0.003584895058252\\
499	0.00356502583749236\\
500	0.003544883268384\\
501	0.0035244628173497\\
502	0.00350375979691629\\
503	0.00348276935479163\\
504	0.00346148646218569\\
505	0.0034399059013872\\
506	0.00341802225263116\\
507	0.00339582988032334\\
508	0.00337332291873124\\
509	0.0033504952573046\\
510	0.00332734052585997\\
511	0.00330385207995603\\
512	0.00328002298690368\\
513	0.00325584601300512\\
514	0.00323131361280798\\
515	0.00320641792140324\\
516	0.0031811507511027\\
517	0.00315550359421973\\
518	0.00312946763416657\\
519	0.00310303376769541\\
520	0.00307619264188358\\
521	0.00304893471043217\\
522	0.00302125031506175\\
523	0.00299312979930897\\
524	0.00296456366392906\\
525	0.00293554277548647\\
526	0.00290605864302008\\
527	0.00287610378238502\\
528	0.00284567218440825\\
529	0.00281475988321566\\
530	0.00278336563667194\\
531	0.00275149226689524\\
532	0.00271915008403821\\
533	0.00268641628363034\\
534	0.00265328435369643\\
535	0.00261968582869031\\
536	0.0025854688063535\\
537	0.00255052199012215\\
538	0.00251478365610303\\
539	0.00247819002717404\\
540	0.00244068417819549\\
541	0.00240220348343551\\
542	0.00236268161584073\\
543	0.00232205422193516\\
544	0.00228024789926472\\
545	0.00223717673401289\\
546	0.00219271866604341\\
547	0.00214821380497191\\
548	0.00210568170443776\\
549	0.00206477854581583\\
550	0.00202367834717305\\
551	0.00198207658025245\\
552	0.00193997595629076\\
553	0.00189740567257551\\
554	0.00185441906889823\\
555	0.00181108860982139\\
556	0.00176750323728597\\
557	0.00172451269574327\\
558	0.00168246236750703\\
559	0.00164022170205425\\
560	0.00159770829159438\\
561	0.00155494262970847\\
562	0.00151195463762704\\
563	0.00146877499999463\\
564	0.00142543401352537\\
565	0.00138195982240711\\
566	0.00133854778582265\\
567	0.00129544180006035\\
568	0.00125203948509083\\
569	0.00120834728092063\\
570	0.00116438693696481\\
571	0.0011201825245913\\
572	0.0010757604602725\\
573	0.00103114962823804\\
574	0.00098638150202912\\
575	0.000941490255217017\\
576	0.000896512854883693\\
577	0.000851489130084093\\
578	0.000806461805377778\\
579	0.00076147648684911\\
580	0.000716581584684075\\
581	0.000671828152215967\\
582	0.000627269616206792\\
583	0.000582961366825369\\
584	0.00053896016820798\\
585	0.00049532334178407\\
586	0.000452107665645336\\
587	0.000409367927260632\\
588	0.000367155072958954\\
589	0.000325513941232697\\
590	0.000284480716949123\\
591	0.000244080684870839\\
592	0.000204328087251263\\
593	0.000165233217676714\\
594	0.000126870159937009\\
595	8.96111722547525e-05\\
596	5.42660945238512e-05\\
597	2.28062284332059e-05\\
598	2.9204464504877e-07\\
599	0\\
600	0\\
};
\addplot [color=black!80!mycolor21,solid,forget plot]
  table[row sep=crcr]{%
1	0.00604026546700612\\
2	0.0060402533658538\\
3	0.00604024098238466\\
4	0.00604022831004924\\
5	0.00604021534214751\\
6	0.00604020207182548\\
7	0.00604018849207171\\
8	0.00604017459571358\\
9	0.00604016037541399\\
10	0.00604014582366729\\
11	0.0060401309327957\\
12	0.00604011569494533\\
13	0.00604010010208226\\
14	0.00604008414598843\\
15	0.00604006781825751\\
16	0.00604005111029071\\
17	0.00604003401329235\\
18	0.00604001651826554\\
19	0.0060399986160076\\
20	0.00603998029710551\\
21	0.00603996155193108\\
22	0.00603994237063621\\
23	0.00603992274314792\\
24	0.00603990265916339\\
25	0.00603988210814473\\
26	0.00603986107931386\\
27	0.00603983956164693\\
28	0.00603981754386914\\
29	0.0060397950144488\\
30	0.00603977196159192\\
31	0.00603974837323622\\
32	0.00603972423704519\\
33	0.006039699540402\\
34	0.0060396742704033\\
35	0.00603964841385282\\
36	0.00603962195725503\\
37	0.0060395948868083\\
38	0.00603956718839841\\
39	0.00603953884759133\\
40	0.00603950984962644\\
41	0.00603948017940923\\
42	0.00603944982150397\\
43	0.00603941876012617\\
44	0.00603938697913498\\
45	0.0060393544620254\\
46	0.00603932119192022\\
47	0.00603928715156196\\
48	0.00603925232330455\\
49	0.00603921668910482\\
50	0.00603918023051389\\
51	0.00603914292866831\\
52	0.00603910476428106\\
53	0.0060390657176325\\
54	0.0060390257685607\\
55	0.00603898489645214\\
56	0.00603894308023186\\
57	0.00603890029835354\\
58	0.00603885652878928\\
59	0.00603881174901931\\
60	0.00603876593602144\\
61	0.00603871906626033\\
62	0.00603867111567635\\
63	0.00603862205967457\\
64	0.00603857187311328\\
65	0.00603852053029232\\
66	0.00603846800494131\\
67	0.00603841427020753\\
68	0.00603835929864351\\
69	0.00603830306219463\\
70	0.00603824553218625\\
71	0.00603818667931065\\
72	0.00603812647361384\\
73	0.00603806488448191\\
74	0.00603800188062737\\
75	0.00603793743007505\\
76	0.00603787150014784\\
77	0.0060378040574521\\
78	0.00603773506786285\\
79	0.00603766449650872\\
80	0.00603759230775658\\
81	0.00603751846519583\\
82	0.00603744293162264\\
83	0.00603736566902364\\
84	0.00603728663855948\\
85	0.00603720580054809\\
86	0.00603712311444767\\
87	0.00603703853883924\\
88	0.00603695203140918\\
89	0.00603686354893113\\
90	0.00603677304724792\\
91	0.00603668048125302\\
92	0.00603658580487165\\
93	0.00603648897104176\\
94	0.00603638993169448\\
95	0.00603628863773451\\
96	0.0060361850390201\\
97	0.00603607908434256\\
98	0.0060359707214058\\
99	0.00603585989680519\\
100	0.00603574655600643\\
101	0.00603563064332395\\
102	0.00603551210189894\\
103	0.00603539087367727\\
104	0.00603526689938696\\
105	0.0060351401185154\\
106	0.00603501046928619\\
107	0.00603487788863583\\
108	0.00603474231218996\\
109	0.00603460367423942\\
110	0.00603446190771583\\
111	0.00603431694416723\\
112	0.0060341687137331\\
113	0.00603401714511918\\
114	0.00603386216557229\\
115	0.00603370370085445\\
116	0.00603354167521696\\
117	0.00603337601137441\\
118	0.00603320663047808\\
119	0.00603303345208935\\
120	0.0060328563941528\\
121	0.00603267537296921\\
122	0.00603249030316817\\
123	0.00603230109768058\\
124	0.00603210766771103\\
125	0.00603190992271004\\
126	0.00603170777034592\\
127	0.00603150111647674\\
128	0.00603128986512214\\
129	0.00603107391843494\\
130	0.00603085317667271\\
131	0.00603062753816934\\
132	0.00603039689930652\\
133	0.00603016115448515\\
134	0.00602992019609692\\
135	0.00602967391449568\\
136	0.00602942219796904\\
137	0.00602916493271007\\
138	0.006028902002789\\
139	0.00602863329012514\\
140	0.00602835867445888\\
141	0.00602807803332407\\
142	0.00602779124202049\\
143	0.0060274981735866\\
144	0.00602719869877273\\
145	0.00602689268601452\\
146	0.00602658000140678\\
147	0.00602626050867775\\
148	0.00602593406916396\\
149	0.00602560054178549\\
150	0.00602525978302194\\
151	0.00602491164688882\\
152	0.00602455598491491\\
153	0.00602419264611996\\
154	0.00602382147699359\\
155	0.00602344232147472\\
156	0.00602305502093198\\
157	0.00602265941414526\\
158	0.00602225533728789\\
159	0.00602184262391019\\
160	0.00602142110492413\\
161	0.00602099060858907\\
162	0.0060205509604988\\
163	0.00602010198356985\\
164	0.00601964349803142\\
165	0.00601917532141646\\
166	0.00601869726855445\\
167	0.00601820915156574\\
168	0.00601771077985748\\
169	0.00601720196012123\\
170	0.00601668249633258\\
171	0.00601615218975228\\
172	0.00601561083892954\\
173	0.00601505823970703\\
174	0.00601449418522819\\
175	0.00601391846594625\\
176	0.00601333086963577\\
177	0.00601273118140585\\
178	0.00601211918371597\\
179	0.00601149465639383\\
180	0.00601085737665546\\
181	0.00601020711912762\\
182	0.00600954365587254\\
183	0.00600886675641485\\
184	0.00600817618777094\\
185	0.00600747171448041\\
186	0.00600675309863981\\
187	0.00600602009993884\\
188	0.00600527247569823\\
189	0.0060045099809102\\
190	0.00600373236828059\\
191	0.00600293938827294\\
192	0.00600213078915456\\
193	0.00600130631704403\\
194	0.00600046571596032\\
195	0.00599960872787332\\
196	0.00599873509275525\\
197	0.00599784454863329\\
198	0.00599693683164267\\
199	0.00599601167608023\\
200	0.00599506881445815\\
201	0.00599410797755745\\
202	0.0059931288944809\\
203	0.00599213129270514\\
204	0.00599111489813133\\
205	0.00599007943513419\\
206	0.00598902462660888\\
207	0.00598795019401522\\
208	0.00598685585741859\\
209	0.0059857413355276\\
210	0.00598460634572723\\
211	0.00598345060410724\\
212	0.00598227382548555\\
213	0.00598107572342554\\
214	0.00597985601024703\\
215	0.00597861439703013\\
216	0.00597735059361167\\
217	0.00597606430857348\\
218	0.00597475524922178\\
219	0.00597342312155775\\
220	0.00597206763023817\\
221	0.00597068847852598\\
222	0.00596928536823059\\
223	0.00596785799963718\\
224	0.00596640607142497\\
225	0.00596492928057418\\
226	0.00596342732226176\\
227	0.00596189988974522\\
228	0.00596034667423549\\
229	0.00595876736475797\\
230	0.00595716164800227\\
231	0.00595552920816083\\
232	0.00595386972675607\\
233	0.00595218288245648\\
234	0.00595046835088112\\
235	0.00594872580439209\\
236	0.00594695491187441\\
237	0.00594515533850142\\
238	0.00594332674548392\\
239	0.00594146878979969\\
240	0.00593958112389889\\
241	0.00593766339537894\\
242	0.00593571524662091\\
243	0.00593373631437569\\
244	0.00593172622928573\\
245	0.00592968461532358\\
246	0.00592761108912331\\
247	0.00592550525917501\\
248	0.00592336672484548\\
249	0.0059211950751797\\
250	0.00591898988742878\\
251	0.0059167507252386\\
252	0.0059144771364235\\
253	0.00591216865023706\\
254	0.00590982477404137\\
255	0.00590744498926841\\
256	0.00590502874656275\\
257	0.00590257545999976\\
258	0.00590008450029143\\
259	0.00589755518693252\\
260	0.00589498677931358\\
261	0.00589237846695294\\
262	0.0058897293591971\\
263	0.00588703847501438\\
264	0.00588430473372601\\
265	0.00588152694691003\\
266	0.00587870381006663\\
267	0.00587583392198569\\
268	0.00587291582604199\\
269	0.00586994800842341\\
270	0.00586692889637673\\
271	0.00586385685648655\\
272	0.00586073019300609\\
273	0.0058575471462592\\
274	0.00585430589114777\\
275	0.00585100453578624\\
276	0.00584764112030902\\
277	0.00584421361589246\\
278	0.00584071992404575\\
279	0.005837157876233\\
280	0.00583352523390158\\
281	0.005829819689003\\
282	0.00582603886510844\\
283	0.0058221803192368\\
284	0.00581824154453196\\
285	0.00581421997394661\\
286	0.00581011298511246\\
287	0.00580591790660013\\
288	0.00580163202579745\\
289	0.00579725259865931\\
290	0.00579277686160511\\
291	0.00578820204585885\\
292	0.00578352539453689\\
293	0.00577874418278647\\
294	0.00577385574125485\\
295	0.00576885748311349\\
296	0.00576374693476257\\
297	0.00575852177017284\\
298	0.00575317984855776\\
299	0.00574771925464148\\
300	0.00574213834001177\\
301	0.00573643576217835\\
302	0.00573061051281166\\
303	0.00572466194125788\\
304	0.00571858977587763\\
305	0.00571239410950164\\
306	0.00570607534071958\\
307	0.00569963405213641\\
308	0.0056930707978446\\
309	0.00568638576265987\\
310	0.00567957813374577\\
311	0.00567264463828431\\
312	0.0056655794880873\\
313	0.00565837646431389\\
314	0.00565102888368784\\
315	0.00564352956717776\\
316	0.00563587080655857\\
317	0.00562804430077823\\
318	0.00562004110318492\\
319	0.005611851568863\\
320	0.00560346529909061\\
321	0.00559487108331325\\
322	0.00558605683930868\\
323	0.00557700955256866\\
324	0.00556771521640067\\
325	0.00555815877491492\\
326	0.00554832407201765\\
327	0.00553819381099456\\
328	0.00552774953117986\\
329	0.0055169716024414\\
330	0.00550583909829185\\
331	0.00549432998080299\\
332	0.00548242125738638\\
333	0.00547008920475404\\
334	0.0054573097112811\\
335	0.00544405879011807\\
336	0.00543031344118811\\
337	0.0054160527628837\\
338	0.00540125912742442\\
339	0.00538592000330503\\
340	0.00537003044851019\\
341	0.00535359644526481\\
342	0.00533663934030161\\
343	0.00531920174680257\\
344	0.00530144252339929\\
345	0.00528361270366819\\
346	0.00526572555361539\\
347	0.00524779599459207\\
348	0.00522984077687911\\
349	0.00521187866831396\\
350	0.00519393066326458\\
351	0.00517602024434208\\
352	0.00515817393827313\\
353	0.00514041964728142\\
354	0.00512278743406033\\
355	0.00510530997362699\\
356	0.00508802247112745\\
357	0.00507096222996936\\
358	0.00505417115077324\\
359	0.00503769643319219\\
360	0.00502158814721375\\
361	0.00500589866402464\\
362	0.00499068173694374\\
363	0.00497599110596318\\
364	0.0049618784526552\\
365	0.0049483904821479\\
366	0.00493556483546533\\
367	0.00492342444071178\\
368	0.00491196978455394\\
369	0.00490116839863309\\
370	0.0048906324354223\\
371	0.00488027870923257\\
372	0.00487012001402301\\
373	0.00486016882262852\\
374	0.00485043709822925\\
375	0.00484093607625654\\
376	0.00483167592810388\\
377	0.0048226653696756\\
378	0.00481391127770395\\
379	0.00480541826302987\\
380	0.0047971882060768\\
381	0.00478921976522046\\
382	0.00478150787669776\\
383	0.0047740432764957\\
384	0.00476681208989165\\
385	0.00475979555689918\\
386	0.00475296999263243\\
387	0.00474630712384782\\
388	0.00473977499796648\\
389	0.00473333973708328\\
390	0.00472696851258153\\
391	0.00472065829735722\\
392	0.00471440615456371\\
393	0.0047082083800895\\
394	0.00470206048044941\\
395	0.00469595716595798\\
396	0.00468989236351199\\
397	0.00468385925379096\\
398	0.00467785033801642\\
399	0.00467185753945659\\
400	0.00466587234443052\\
401	0.00465988598636044\\
402	0.00465388967405033\\
403	0.00464787486131385\\
404	0.00464183354823129\\
405	0.00463575859381497\\
406	0.00462964400391982\\
407	0.00462348513651544\\
408	0.00461727863610413\\
409	0.00461102107911178\\
410	0.00460470887362543\\
411	0.00459833843936909\\
412	0.00459190635956627\\
413	0.00458540926575913\\
414	0.00457884387749456\\
415	0.00457220703776355\\
416	0.0045654957449966\\
417	0.00455870716969015\\
418	0.00455183867148872\\
419	0.00454488780577619\\
420	0.00453785231597264\\
421	0.00453073010934476\\
422	0.00452351921611247\\
423	0.00451621773483804\\
424	0.0045088237842544\\
425	0.00450133550763517\\
426	0.00449375107627498\\
427	0.00448606869187626\\
428	0.00447828658765964\\
429	0.00447040302805675\\
430	0.00446241630691595\\
431	0.0044543247442354\\
432	0.00444612668155861\\
433	0.00443782047631773\\
434	0.00442940449557253\\
435	0.00442087710976413\\
436	0.0044122366872682\\
437	0.00440348159058331\\
438	0.00439461017483153\\
439	0.00438562078690467\\
440	0.00437651176441105\\
441	0.00436728143444992\\
442	0.0043579281122518\\
443	0.00434845009973193\\
444	0.00433884568401337\\
445	0.00432911313598069\\
446	0.00431925070892445\\
447	0.00430925663732789\\
448	0.00429912913582875\\
449	0.00428886639836054\\
450	0.00427846659744047\\
451	0.00426792788352919\\
452	0.00425724838437782\\
453	0.00424642620436951\\
454	0.00423545942386197\\
455	0.00422434609853724\\
456	0.00421308425876252\\
457	0.00420167190896466\\
458	0.00419010702701767\\
459	0.00417838756364069\\
460	0.00416651144180048\\
461	0.0041544765561112\\
462	0.00414228077222317\\
463	0.00412992192619294\\
464	0.00411739782383058\\
465	0.00410470624002143\\
466	0.00409184491801969\\
467	0.00407881156871013\\
468	0.00406560386983372\\
469	0.00405221946517284\\
470	0.00403865596368979\\
471	0.00402491093861322\\
472	0.00401098192646624\\
473	0.00399686642602873\\
474	0.00398256189722682\\
475	0.00396806575994199\\
476	0.00395337539273091\\
477	0.00393848813144712\\
478	0.00392340126775386\\
479	0.0039081120475174\\
480	0.00389261766906804\\
481	0.00387691528131579\\
482	0.00386100198170571\\
483	0.00384487481399766\\
484	0.00382853076585254\\
485	0.00381196676620695\\
486	0.00379517968241609\\
487	0.00377816631714323\\
488	0.00376092340497299\\
489	0.0037434476087237\\
490	0.00372573551543333\\
491	0.00370778363199189\\
492	0.00368958838039254\\
493	0.00367114609257297\\
494	0.00365245300481891\\
495	0.00363350525170209\\
496	0.00361429885952708\\
497	0.00359482973926421\\
498	0.00357509367895112\\
499	0.00355508633555229\\
500	0.00353480322627609\\
501	0.00351423971936296\\
502	0.00349339102437709\\
503	0.00347225218205858\\
504	0.00345081805382653\\
505	0.00342908331106559\\
506	0.00340704242438401\\
507	0.00338468965310275\\
508	0.00336201903532531\\
509	0.00333902437905489\\
510	0.00331569925497228\\
511	0.00329203699167503\\
512	0.00326803067441413\\
513	0.00324367314866261\\
514	0.00321895703022545\\
515	0.00319387472407181\\
516	0.00316841845466331\\
517	0.0031425803112958\\
518	0.00311635231290443\\
519	0.00308972649795068\\
520	0.00306269504647383\\
521	0.003035250443222\\
522	0.00300738569307181\\
523	0.00297909460281731\\
524	0.00295037214700139\\
525	0.00292121493997158\\
526	0.00289162182508419\\
527	0.00286159456307669\\
528	0.00283113903050781\\
529	0.00280026820797992\\
530	0.00276905767127757\\
531	0.00273751343323209\\
532	0.00270557045782752\\
533	0.00267307211406851\\
534	0.00263992242027967\\
535	0.00260606339991704\\
536	0.00257144120591396\\
537	0.00253600462960201\\
538	0.00249969764946177\\
539	0.0024624594910934\\
540	0.0024242224843366\\
541	0.00238491137467294\\
542	0.00234444454378673\\
543	0.00230273639459776\\
544	0.00225966796547375\\
545	0.00221668865662068\\
546	0.00217582521296366\\
547	0.0021362607008898\\
548	0.00209639682394114\\
549	0.00205600872796146\\
550	0.00201508651193405\\
551	0.00197366306424266\\
552	0.00193178370444651\\
553	0.00188950895448212\\
554	0.00184691508122861\\
555	0.0018042815234374\\
556	0.00176267919822897\\
557	0.00172129441691307\\
558	0.00167960212397915\\
559	0.0016376158968852\\
560	0.00159536277689962\\
561	0.00155287132132885\\
562	0.00151016995834333\\
563	0.0014672865789184\\
564	0.00142424694170975\\
565	0.00138130050605624\\
566	0.00133853402642386\\
567	0.00129544164993273\\
568	0.00125203947380946\\
569	0.00120834727902135\\
570	0.00116438693634367\\
571	0.00112018252432505\\
572	0.00107576046014404\\
573	0.00103114962817055\\
574	0.000986381501991801\\
575	0.000941490255195813\\
576	0.000896512854871852\\
577	0.000851489130077657\\
578	0.000806461805374505\\
579	0.000761476486847638\\
580	0.000716581584683521\\
581	0.000671828152215824\\
582	0.000627269616206769\\
583	0.00058296136682538\\
584	0.000538960168207986\\
585	0.000495323341784082\\
586	0.00045210766564535\\
587	0.000409367927260641\\
588	0.000367155072958961\\
589	0.000325513941232703\\
590	0.000284480716949125\\
591	0.000244080684870838\\
592	0.00020432808725126\\
593	0.000165233217676711\\
594	0.000126870159937008\\
595	8.9611172254752e-05\\
596	5.42660945238509e-05\\
597	2.28062284332057e-05\\
598	2.9204464504877e-07\\
599	0\\
600	0\\
};
\addplot [color=black,solid,forget plot]
  table[row sep=crcr]{%
1	0.00604024220115463\\
2	0.00604022941156559\\
3	0.00604021631894476\\
4	0.00604020291611037\\
5	0.00604018919571038\\
6	0.00604017515021865\\
7	0.0060401607719305\\
8	0.00604014605295886\\
9	0.00604013098522959\\
10	0.00604011556047734\\
11	0.00604009977024089\\
12	0.00604008360585844\\
13	0.00604006705846304\\
14	0.00604005011897763\\
15	0.00604003277811003\\
16	0.00604001502634798\\
17	0.0060399968539538\\
18	0.00603997825095909\\
19	0.00603995920715936\\
20	0.00603993971210836\\
21	0.00603991975511225\\
22	0.00603989932522407\\
23	0.00603987841123736\\
24	0.00603985700168037\\
25	0.00603983508480957\\
26	0.00603981264860322\\
27	0.00603978968075492\\
28	0.00603976616866678\\
29	0.0060397420994427\\
30	0.00603971745988097\\
31	0.00603969223646742\\
32	0.00603966641536785\\
33	0.00603963998242046\\
34	0.00603961292312817\\
35	0.00603958522265073\\
36	0.00603955686579649\\
37	0.00603952783701423\\
38	0.00603949812038457\\
39	0.00603946769961138\\
40	0.00603943655801282\\
41	0.00603940467851219\\
42	0.00603937204362877\\
43	0.00603933863546809\\
44	0.00603930443571237\\
45	0.00603926942561029\\
46	0.00603923358596707\\
47	0.00603919689713374\\
48	0.00603915933899659\\
49	0.00603912089096622\\
50	0.00603908153196619\\
51	0.00603904124042176\\
52	0.00603899999424794\\
53	0.00603895777083757\\
54	0.00603891454704906\\
55	0.00603887029919381\\
56	0.00603882500302325\\
57	0.00603877863371572\\
58	0.00603873116586302\\
59	0.00603868257345652\\
60	0.00603863282987321\\
61	0.00603858190786107\\
62	0.00603852977952445\\
63	0.00603847641630884\\
64	0.00603842178898546\\
65	0.00603836586763539\\
66	0.0060383086216335\\
67	0.00603825001963159\\
68	0.0060381900295418\\
69	0.00603812861851904\\
70	0.00603806575294322\\
71	0.00603800139840123\\
72	0.00603793551966834\\
73	0.00603786808068912\\
74	0.00603779904455802\\
75	0.00603772837349957\\
76	0.00603765602884802\\
77	0.00603758197102642\\
78	0.00603750615952539\\
79	0.00603742855288145\\
80	0.00603734910865462\\
81	0.00603726778340568\\
82	0.00603718453267291\\
83	0.00603709931094821\\
84	0.00603701207165275\\
85	0.00603692276711206\\
86	0.00603683134853047\\
87	0.00603673776596517\\
88	0.00603664196829936\\
89	0.0060365439032152\\
90	0.00603644351716582\\
91	0.0060363407553467\\
92	0.00603623556166671\\
93	0.0060361278787182\\
94	0.00603601764774663\\
95	0.00603590480861926\\
96	0.00603578929979333\\
97	0.00603567105828357\\
98	0.00603555001962885\\
99	0.00603542611785818\\
100	0.00603529928545591\\
101	0.00603516945332611\\
102	0.00603503655075626\\
103	0.00603490050538015\\
104	0.00603476124313985\\
105	0.00603461868824691\\
106	0.00603447276314273\\
107	0.00603432338845804\\
108	0.00603417048297137\\
109	0.00603401396356695\\
110	0.00603385374519139\\
111	0.00603368974080943\\
112	0.00603352186135891\\
113	0.0060333500157048\\
114	0.00603317411059178\\
115	0.00603299405059647\\
116	0.00603280973807822\\
117	0.00603262107312888\\
118	0.00603242795352169\\
119	0.00603223027465891\\
120	0.00603202792951842\\
121	0.00603182080859922\\
122	0.00603160879986567\\
123	0.00603139178869086\\
124	0.00603116965779838\\
125	0.00603094228720326\\
126	0.00603070955415151\\
127	0.00603047133305845\\
128	0.0060302274954458\\
129	0.00602997790987749\\
130	0.00602972244189418\\
131	0.00602946095394636\\
132	0.00602919330532638\\
133	0.00602891935209879\\
134	0.00602863894702952\\
135	0.00602835193951357\\
136	0.00602805817550143\\
137	0.00602775749742385\\
138	0.00602744974411521\\
139	0.00602713475073561\\
140	0.00602681234869119\\
141	0.00602648236555314\\
142	0.0060261446249751\\
143	0.00602579894660905\\
144	0.00602544514601946\\
145	0.00602508303459615\\
146	0.00602471241946532\\
147	0.00602433310339914\\
148	0.00602394488472337\\
149	0.00602354755722375\\
150	0.00602314091005041\\
151	0.00602272472762075\\
152	0.00602229878952048\\
153	0.00602186287040315\\
154	0.00602141673988773\\
155	0.00602096016245453\\
156	0.00602049289733946\\
157	0.00602001469842612\\
158	0.00601952531413678\\
159	0.0060190244873208\\
160	0.00601851195514175\\
161	0.00601798744896242\\
162	0.00601745069422821\\
163	0.00601690141034843\\
164	0.00601633931057578\\
165	0.00601576410188413\\
166	0.00601517548484403\\
167	0.00601457315349658\\
168	0.00601395679522537\\
169	0.00601332609062629\\
170	0.00601268071337537\\
171	0.00601202033009492\\
172	0.00601134460021717\\
173	0.0060106531758464\\
174	0.00600994570161868\\
175	0.00600922181455949\\
176	0.00600848114393944\\
177	0.00600772331112782\\
178	0.00600694792944373\\
179	0.00600615460400528\\
180	0.00600534293157628\\
181	0.00600451250041084\\
182	0.00600366289009542\\
183	0.00600279367138868\\
184	0.00600190440605857\\
185	0.00600099464671707\\
186	0.00600006393665222\\
187	0.00599911180965739\\
188	0.0059981377898579\\
189	0.00599714139153454\\
190	0.00599612211894405\\
191	0.00599507946613675\\
192	0.0059940129167705\\
193	0.00599292194392151\\
194	0.00599180600989156\\
195	0.00599066456601142\\
196	0.00598949705244036\\
197	0.00598830289796187\\
198	0.00598708151977458\\
199	0.00598583232327963\\
200	0.00598455470186254\\
201	0.00598324803667099\\
202	0.00598191169638715\\
203	0.00598054503699518\\
204	0.00597914740154348\\
205	0.00597771811990127\\
206	0.00597625650851\\
207	0.00597476187012911\\
208	0.0059732334935763\\
209	0.00597167065346244\\
210	0.00597007260992117\\
211	0.00596843860833389\\
212	0.00596676787905015\\
213	0.00596505963710449\\
214	0.00596331308193016\\
215	0.00596152739707178\\
216	0.00595970174989727\\
217	0.00595783529131195\\
218	0.00595592715547681\\
219	0.0059539764595338\\
220	0.00595198230334206\\
221	0.00594994376922937\\
222	0.0059478599217644\\
223	0.00594572980755597\\
224	0.0059435524550875\\
225	0.00594132687459575\\
226	0.00593905205800509\\
227	0.00593672697893076\\
228	0.00593435059276621\\
229	0.00593192183687351\\
230	0.00592943963089844\\
231	0.00592690287723542\\
232	0.00592431046167212\\
233	0.00592166125424854\\
234	0.00591895411037039\\
235	0.00591618787222353\\
236	0.00591336137054312\\
237	0.00591047342679857\\
238	0.00590752285586539\\
239	0.00590450846926347\\
240	0.00590142907905286\\
241	0.0058982835024888\\
242	0.00589507056754937\\
243	0.00589178911946169\\
244	0.00588843802836297\\
245	0.00588501619824405\\
246	0.00588152257732885\\
247	0.00587795617004913\\
248	0.00587431605076752\\
249	0.00587060137939107\\
250	0.00586681141898741\\
251	0.00586294555546768\\
252	0.00585900331931978\\
253	0.00585498440925583\\
254	0.00585088871745866\\
255	0.00584671635585565\\
256	0.00584246768248407\\
257	0.00583814332650342\\
258	0.00583374420970669\\
259	0.00582927156141854\\
260	0.0058247269223566\\
261	0.00582011213126507\\
262	0.00581542928577073\\
263	0.0058106806658248\\
264	0.00580586860408352\\
265	0.00580099528198818\\
266	0.00579606241162627\\
267	0.00579107060308566\\
268	0.00578601800834475\\
269	0.00578090264365087\\
270	0.00577572237782153\\
271	0.00577047491949016\\
272	0.00576515780320051\\
273	0.00575976837424582\\
274	0.00575430377214284\\
275	0.00574876091262352\\
276	0.00574313646802086\\
277	0.00573742684592163\\
278	0.00573162816595557\\
279	0.00572573623459187\\
280	0.00571974651781746\\
281	0.00571365411158153\\
282	0.00570745370991107\\
283	0.00570113957063665\\
284	0.00569470547871148\\
285	0.00568814470717387\\
286	0.00568144997589694\\
287	0.00567461340840066\\
288	0.00566762648717802\\
289	0.0056604800082285\\
290	0.00565316403581248\\
291	0.00564566785886966\\
292	0.00563797995111085\\
293	0.00563008793754007\\
294	0.005621978571147\\
295	0.00561363772479723\\
296	0.00560505040503344\\
297	0.00559620079670585\\
298	0.0055870723502374\\
299	0.00557764792713239\\
300	0.00556791002437553\\
301	0.00555784110485093\\
302	0.00554742406491736\\
303	0.00553664279888841\\
304	0.00552548302651021\\
305	0.00551393362608928\\
306	0.00550198829571939\\
307	0.00548964775149012\\
308	0.00547692265350177\\
309	0.00546383750077499\\
310	0.00545050411544554\\
311	0.00543706551203978\\
312	0.00542352426312607\\
313	0.00540988338268015\\
314	0.00539614638453567\\
315	0.00538231736054493\\
316	0.00536840113025545\\
317	0.00535440347191429\\
318	0.00534033101754344\\
319	0.00532619131263288\\
320	0.00531199291781422\\
321	0.00529774552004493\\
322	0.00528346005357307\\
323	0.00526914883070616\\
324	0.00525482568205746\\
325	0.0052405061054705\\
326	0.00522620742230711\\
327	0.00521194893974284\\
328	0.00519775212120448\\
329	0.00518364079415995\\
330	0.00516964162363341\\
331	0.00515578212083989\\
332	0.00514209199396442\\
333	0.00512860302430427\\
334	0.00511534880080009\\
335	0.00510236423489391\\
336	0.00508968465564633\\
337	0.00507734632258802\\
338	0.00506538684370882\\
339	0.00505384123288094\\
340	0.00504273935097975\\
341	0.00503210235176745\\
342	0.0050219378023539\\
343	0.00501223303726919\\
344	0.00500285468968795\\
345	0.00499356955559161\\
346	0.00498438739563515\\
347	0.00497531818451975\\
348	0.00496637204863509\\
349	0.00495755918753014\\
350	0.00494888977653276\\
351	0.00494037384742799\\
352	0.00493202114184904\\
353	0.00492384098152797\\
354	0.00491584209842258\\
355	0.00490803242980753\\
356	0.00490041888681241\\
357	0.00489300709639044\\
358	0.0048858010512431\\
359	0.00487880271345722\\
360	0.00487201164452336\\
361	0.00486542462797597\\
362	0.00485903530661304\\
363	0.00485283386790705\\
364	0.00484680682774371\\
365	0.00484093698516495\\
366	0.00483520365176605\\
367	0.00482958330142587\\
368	0.00482405084203799\\
369	0.00481858178443982\\
370	0.00481317112843914\\
371	0.00480781825833003\\
372	0.00480252205246886\\
373	0.00479728084542065\\
374	0.0047920923938142\\
375	0.00478695384774833\\
376	0.00478186173200704\\
377	0.00477681194038512\\
378	0.00477179974536357\\
379	0.00476681982690542\\
380	0.00476186632440995\\
381	0.00475693291591714\\
382	0.00475201292834093\\
383	0.00474709948159236\\
384	0.00474218566766209\\
385	0.00473726476257569\\
386	0.00473233046398241\\
387	0.00472737713907369\\
388	0.00472240005545754\\
389	0.00471739555086495\\
390	0.00471236108295356\\
391	0.00470729403270115\\
392	0.00470219168751057\\
393	0.00469705126007499\\
394	0.00469186991009814\\
395	0.0046866447686414\\
396	0.00468137296467694\\
397	0.00467605165319088\\
398	0.0046706780439002\\
399	0.00466524942932727\\
400	0.00465976321063176\\
401	0.00465421691925515\\
402	0.00464860823213952\\
403	0.00464293497810728\\
404	0.00463719513305938\\
405	0.00463138680214551\\
406	0.00462550818845853\\
407	0.00461955755337387\\
408	0.00461353313115817\\
409	0.00460743324585354\\
410	0.00460125630724931\\
411	0.00459500075370312\\
412	0.00458866504993294\\
413	0.00458224768832692\\
414	0.00457574718895802\\
415	0.00456916209832784\\
416	0.00456249098687029\\
417	0.00455573244556927\\
418	0.00454888508158583\\
419	0.00454194751318431\\
420	0.00453491836442718\\
421	0.00452779626024947\\
422	0.00452057982260584\\
423	0.00451326766834301\\
424	0.00450585840873635\\
425	0.00449835064883985\\
426	0.0044907429866612\\
427	0.00448303401218258\\
428	0.00447522230625659\\
429	0.00446730643941554\\
430	0.00445928497063901\\
431	0.00445115644612937\\
432	0.00444291939814595\\
433	0.00443457234394397\\
434	0.00442611378485224\\
435	0.00441754220550516\\
436	0.0044088560732158\\
437	0.00440005383744301\\
438	0.00439113392927468\\
439	0.00438209476090384\\
440	0.00437293472510363\\
441	0.00436365219470771\\
442	0.00435424552210145\\
443	0.00434471303872821\\
444	0.00433505305461357\\
445	0.00432526385790813\\
446	0.00431534371444779\\
447	0.00430529086732777\\
448	0.00429510353648548\\
449	0.00428477991828609\\
450	0.00427431818510563\\
451	0.00426371648490848\\
452	0.00425297294081893\\
453	0.00424208565068619\\
454	0.00423105268664214\\
455	0.00421987209465014\\
456	0.00420854189404392\\
457	0.00419706007705395\\
458	0.0041854246083197\\
459	0.00417363342438504\\
460	0.00416168443317458\\
461	0.00414957551344839\\
462	0.00413730451423243\\
463	0.00412486925422227\\
464	0.00411226752115713\\
465	0.00409949707116105\\
466	0.00408655562804788\\
467	0.00407344088258621\\
468	0.00406015049171967\\
469	0.00404668207773832\\
470	0.00403303322739574\\
471	0.00401920149096636\\
472	0.00400518438123652\\
473	0.00399097937242283\\
474	0.00397658389901009\\
475	0.00396199535450056\\
476	0.00394721109006594\\
477	0.00393222841309141\\
478	0.00391704458560201\\
479	0.00390165682255867\\
480	0.00388606229001174\\
481	0.00387025810309765\\
482	0.00385424132386436\\
483	0.00383800895890885\\
484	0.00382155795681001\\
485	0.00380488520533814\\
486	0.00378798752842205\\
487	0.00377086168285347\\
488	0.00375350435470764\\
489	0.00373591215545927\\
490	0.00371808161777201\\
491	0.00370000919094156\\
492	0.0036816912359725\\
493	0.00366312402027269\\
494	0.00364430371195221\\
495	0.00362522637371953\\
496	0.00360588795637567\\
497	0.00358628429191823\\
498	0.00356641108628137\\
499	0.0035462639117581\\
500	0.00352583819917636\\
501	0.00350512922993411\\
502	0.00348413212804223\\
503	0.00346284185237889\\
504	0.00344125318943081\\
505	0.00341936074688669\\
506	0.00339715894856318\\
507	0.00337464203128774\\
508	0.0033518040445461\\
509	0.00332863885393061\\
510	0.0033051401497136\\
511	0.00328130146222999\\
512	0.00325711618620332\\
513	0.00323257761671049\\
514	0.00320767900018014\\
515	0.00318241360468962\\
516	0.00315677481490683\\
517	0.00313075625836541\\
518	0.00310435197142525\\
519	0.00307755661533165\\
520	0.00305036575533494\\
521	0.00302277621898349\\
522	0.00299478655359284\\
523	0.00296639760769014\\
524	0.00293761326714388\\
525	0.00290844138313012\\
526	0.00287889567964912\\
527	0.00284903708034976\\
528	0.00281890494915272\\
529	0.00278843855151698\\
530	0.00275749196392408\\
531	0.00272596507989511\\
532	0.00269380524096619\\
533	0.00266096644932591\\
534	0.00262740435415402\\
535	0.00259307084256343\\
536	0.00255791304693705\\
537	0.00252187174740753\\
538	0.00248488045298894\\
539	0.00244686410993284\\
540	0.0024077377673102\\
541	0.00236740477819526\\
542	0.00232575612961217\\
543	0.00228403044480058\\
544	0.00224447891824068\\
545	0.00220617374851022\\
546	0.00216751041189044\\
547	0.00212829967778029\\
548	0.00208853590537212\\
549	0.00204824471904266\\
550	0.00200746452793853\\
551	0.00196624548713097\\
552	0.0019246531104606\\
553	0.00188276877233496\\
554	0.00184135597079297\\
555	0.00180079542484969\\
556	0.00175995385435818\\
557	0.00171878217399611\\
558	0.00167730425837276\\
559	0.0016355463416697\\
560	0.00159353515574625\\
561	0.00155129727986493\\
562	0.00150885875639163\\
563	0.0014662435534718\\
564	0.00142371991366068\\
565	0.00138129869913588\\
566	0.00133853400693808\\
567	0.00129544164841737\\
568	0.00125203947353977\\
569	0.0012083472789297\\
570	0.00116438693630348\\
571	0.0011201825243051\\
572	0.00107576046013345\\
573	0.00103114962816466\\
574	0.000986381501988496\\
575	0.000941490255193976\\
576	0.000896512854870878\\
577	0.000851489130077175\\
578	0.000806461805374287\\
579	0.000761476486847549\\
580	0.000716581584683501\\
581	0.000671828152215815\\
582	0.000627269616206765\\
583	0.00058296136682537\\
584	0.000538960168207977\\
585	0.00049532334178407\\
586	0.000452107665645345\\
587	0.000409367927260634\\
588	0.000367155072958957\\
589	0.0003255139412327\\
590	0.000284480716949121\\
591	0.000244080684870836\\
592	0.00020432808725126\\
593	0.00016523321767671\\
594	0.000126870159937007\\
595	8.96111722547526e-05\\
596	5.4266094523851e-05\\
597	2.2806228433206e-05\\
598	2.9204464504877e-07\\
599	0\\
600	0\\
};
\end{axis}
\end{tikzpicture}% 
%  \caption{Discrete Time w/ nFPC}
%\end{subfigure}\\
%
%\leavevmode\smash{\makebox[0pt]{\hspace{-7em}% HORIZONTAL POSITION           
%  \rotatebox[origin=l]{90}{\hspace{20em}% VERTICAL POSITION
%    Depth $\delta^-$}%
%}}\hspace{0pt plus 1filll}\null
%
%Time (s)
%
%\vspace{1cm}
%\begin{subfigure}{\linewidth}
%  \centering
%  \tikzsetnextfilename{deltalegend}
%  \definecolor{mycolor1}{rgb}{1.00000,0.00000,1.00000}%
\begin{tikzpicture}[framed]
    \begingroup
    % inits/clears the lists (which might be populated from previous
    % axes):
    \csname pgfplots@init@cleared@structures\endcsname
    \pgfplotsset{legend style={at={(0,1)},anchor=north west},legend columns=-1,legend style={draw=none,column sep=1ex},legend entries={$q=-4$,$q=-3$,$q=-2$,$q=-1$}}%
    
    \csname pgfplots@addlegendimage\endcsname{thick,green,dashed,sharp plot}
    \csname pgfplots@addlegendimage\endcsname{thick,mycolor1,dashed,sharp plot}
    \csname pgfplots@addlegendimage\endcsname{thick,red,dashed,sharp plot}
    \csname pgfplots@addlegendimage\endcsname{thick,blue,dashed,sharp plot}

    % draws the legend:
    \csname pgfplots@createlegend\endcsname
    \endgroup

    \begingroup
    % inits/clears the lists (which might be populated from previous
    % axes):
    \csname pgfplots@init@cleared@structures\endcsname
    \pgfplotsset{legend style={at={(3.75,0.5)},anchor=north west},legend columns=-1,legend style={draw=none,column sep=1ex},legend entries={$q=0$}}%

    \csname pgfplots@addlegendimage\endcsname{thick,black,sharp plot}

    % draws the legend:
    \csname pgfplots@createlegend\endcsname
    \endgroup

    \begingroup
    % inits/clears the lists (which might be populated from previous
    % axes):
    \csname pgfplots@init@cleared@structures\endcsname
    \pgfplotsset{legend style={at={(0,0)},anchor=north west},legend columns=-1,legend style={draw=none,column sep=1ex},legend entries={$q=+4$,$q=+3$,$q=+2$,$q=+1$}}%
    
    \csname pgfplots@addlegendimage\endcsname{thick,green,sharp plot}
    \csname pgfplots@addlegendimage\endcsname{thick,mycolor1,sharp plot}
    \csname pgfplots@addlegendimage\endcsname{thick,red,sharp plot}
    \csname pgfplots@addlegendimage\endcsname{thick,blue,sharp plot}

    % draws the legend:
    \csname pgfplots@createlegend\endcsname
    \endgroup
\end{tikzpicture} 
%\end{subfigure}%
%  \caption{Optimal sell depths $\delta^-$ for Markov state $Z=(\rho = 0, \Delta S = 0)$, implying neutral imbalance and no previous price change. We expect no change in midprice.}
%  \label{fig:comp_dm_z8}
%\end{figure}
%
%\begin{figure}
%\centering
%\begin{subfigure}{.45\linewidth}
%  \centering
%  \setlength\figureheight{\linewidth} 
%  \setlength\figurewidth{\linewidth}
%  \tikzsetnextfilename{dm_cts_z15}
%  % This file was created by matlab2tikz.
%
%The latest updates can be retrieved from
%  http://www.mathworks.com/matlabcentral/fileexchange/22022-matlab2tikz-matlab2tikz
%where you can also make suggestions and rate matlab2tikz.
%
\definecolor{mycolor1}{rgb}{1.00000,0.00000,1.00000}%
%
\begin{tikzpicture}[trim axis left, trim axis right]

\begin{axis}[%
width=\figurewidth,
height=\figureheight,
at={(0\figurewidth,0\figureheight)},
scale only axis,
every outer x axis line/.append style={black},
every x tick label/.append style={font=\color{black}},
xmin=0,
xmax=100,
xlabel={Time},
every outer y axis line/.append style={black},
every y tick label/.append style={font=\color{black}},
ymin=0,
ymax=0.015,
ylabel={Depth $\delta^-$},
axis background/.style={fill=white},
axis x line*=bottom,
axis y line*=left,
yticklabel style={
        /pgf/number format/fixed,
        /pgf/number format/precision=3
},
scaled y ticks=false,
legend style={legend cell align=left,align=left,draw=black,font=\footnotesize, at={(0.98,0.02)},anchor=south east},
every axis legend/.code={\renewcommand\addlegendentry[2][]{}}  %ignore legend locally
]
\addplot [color=green,dashed]
  table[row sep=crcr]{%
0.01	0\\
1.01	0\\
2.01	0\\
3.01	0\\
4.01	0\\
5.01	0\\
6.01	0\\
7.01	0\\
8.01	0\\
9.01	0\\
10.01	0\\
11.01	0\\
12.01	0\\
13.01	0\\
14.01	0\\
15.01	0\\
16.01	0\\
17.01	0\\
18.01	0\\
19.01	0\\
20.01	0\\
21.01	0\\
22.01	0\\
23.01	0\\
24.01	0\\
25.01	0\\
26.01	0\\
27.01	0\\
28.01	0\\
29.01	0\\
30.01	0\\
31.01	0\\
32.01	0\\
33.01	0\\
34.01	0\\
35.01	0\\
36.01	0\\
37.01	0\\
38.01	0\\
39.01	0\\
40.01	0\\
41.01	0\\
42.01	0\\
43.01	0\\
44.01	0\\
45.01	0\\
46.01	0\\
47.01	0\\
48.01	0\\
49.01	0\\
50.01	0\\
51.01	0\\
52.01	0\\
53.01	0\\
54.01	0\\
55.01	0\\
56.01	0\\
57.01	0\\
58.01	0\\
59.01	0\\
60.01	0\\
61.01	0\\
62.01	0\\
63.01	0\\
64.01	0\\
65.01	0\\
66.01	0\\
67.01	0\\
68.01	0\\
69.01	0\\
70.01	0\\
71.01	0\\
72.01	0\\
73.01	0\\
74.01	0\\
75.01	0\\
76.01	0\\
77.01	0\\
78.01	0\\
79.01	0\\
80.01	0\\
81.01	0\\
82.01	0\\
83.01	0\\
84.01	0\\
85.01	0\\
86.01	0\\
87.01	0\\
88.01	0\\
89.01	0\\
90.01	0\\
91.01	0\\
92.01	0\\
93.01	0\\
94.01	0\\
95.01	0\\
96.01	0\\
97.01	0\\
98.01	0\\
99.01	0.00331175016347651\\
99.02	0.00335383459458577\\
99.03	0.00339629425946782\\
99.04	0.00343913299375206\\
99.05	0.00348235448270438\\
99.06	0.00352596247377274\\
99.07	0.00356996078207136\\
99.08	0.0036143532932759\\
99.09	0.003659143966672\\
99.1	0.00370433683836476\\
99.11	0.00374993602465781\\
99.12	0.00379594572561081\\
99.13	0.00384237022878453\\
99.14	0.00388921386374265\\
99.15	0.00393648045349505\\
99.16	0.00398417385589169\\
99.17	0.00403229796388142\\
99.18	0.00408085670576604\\
99.19	0.00412985404544918\\
99.2	0.00417929398267937\\
99.21	0.00422918055328676\\
99.22	0.00427951782957623\\
99.23	0.00433030992114294\\
99.24	0.00438156097518348\\
99.25	0.00443327517680532\\
99.26	0.00448545674933452\\
99.27	0.00453810995462083\\
99.28	0.00459123909334014\\
99.29	0.00464484850529341\\
99.3	0.00469894256970183\\
99.31	0.00475352570549745\\
99.32	0.00480860237160862\\
99.33	0.00486417706723978\\
99.34	0.00492025433214455\\
99.35	0.00497683874689156\\
99.36	0.00503393493312218\\
99.37	0.00509154755379893\\
99.38	0.00514968131344361\\
99.39	0.0052083409583641\\
99.4	0.00526753127686883\\
99.41	0.00532725709946772\\
99.42	0.00538752330062317\\
99.43	0.00544833479915421\\
99.44	0.0055096965586415\\
99.45	0.00557161358783625\\
99.46	0.00563409094107324\\
99.47	0.00569713371868781\\
99.48	0.0057607470674359\\
99.49	0.00582493618091646\\
99.5	0.0058897062999978\\
99.51	0.00595506271324757\\
99.52	0.00602101075736671\\
99.53	0.00608755581762727\\
99.54	0.00615470332831403\\
99.55	0.00622245877317017\\
99.56	0.00629082768584681\\
99.57	0.00635981565035658\\
99.58	0.00642942830153121\\
99.59	0.00649967132548313\\
99.6	0.00657055046007118\\
99.61	0.00664207149537041\\
99.62	0.00671424027414604\\
99.63	0.00678706269233151\\
99.64	0.00686054469951087\\
99.65	0.0069346922994052\\
99.66	0.00700951155036349\\
99.67	0.00708500856585769\\
99.68	0.00716118951498209\\
99.69	0.00723806060688493\\
99.7	0.00731562810494498\\
99.71	0.00739389832935691\\
99.72	0.007472877657644\\
99.73	0.00755257252517513\\
99.74	0.0076329894256864\\
99.75	0.00771413491180722\\
99.76	0.00779601559559102\\
99.77	0.00787863814905056\\
99.78	0.00796200930469796\\
99.79	0.00804613585608947\\
99.8	0.00813102465837497\\
99.81	0.00821668262885248\\
99.82	0.00830311674752747\\
99.83	0.00839033405767729\\
99.84	0.00847834166642063\\
99.85	0.00856714674529227\\
99.86	0.00865675653082306\\
99.87	0.00874717832512533\\
99.88	0.00883841949648396\\
99.89	0.00893048747995297\\
99.9	0.00902338977795818\\
99.91	0.00911713396090572\\
99.92	0.0092117276677969\\
99.93	0.00930717860684955\\
99.94	0.009403494556126\\
99.95	0.00950068336416817\\
99.96	0.00959875295063989\\
99.97	0.009697711306977\\
99.98	0.00979756649704546\\
99.99	0.00989832665780802\\
100	0.01\\
};
\addlegendentry{$q=-4$};

\addplot [color=mycolor1,dashed]
  table[row sep=crcr]{%
0.01	0\\
1.01	0\\
2.01	0\\
3.01	0\\
4.01	0\\
5.01	0\\
6.01	0\\
7.01	0\\
8.01	0\\
9.01	0\\
10.01	0\\
11.01	0\\
12.01	0\\
13.01	0\\
14.01	0\\
15.01	0\\
16.01	0\\
17.01	0\\
18.01	0\\
19.01	0\\
20.01	0\\
21.01	0\\
22.01	0\\
23.01	0\\
24.01	0\\
25.01	0\\
26.01	0\\
27.01	0\\
28.01	0\\
29.01	0\\
30.01	0\\
31.01	0\\
32.01	0\\
33.01	0\\
34.01	0\\
35.01	0\\
36.01	0\\
37.01	0\\
38.01	0\\
39.01	0\\
40.01	0\\
41.01	0\\
42.01	0\\
43.01	0\\
44.01	0\\
45.01	0\\
46.01	0\\
47.01	0\\
48.01	0\\
49.01	0\\
50.01	0\\
51.01	0\\
52.01	0\\
53.01	0\\
54.01	0\\
55.01	0\\
56.01	0\\
57.01	0\\
58.01	0\\
59.01	0\\
60.01	0\\
61.01	0\\
62.01	0\\
63.01	0\\
64.01	0\\
65.01	0\\
66.01	0\\
67.01	0\\
68.01	0\\
69.01	0\\
70.01	0\\
71.01	0\\
72.01	0\\
73.01	0\\
74.01	0\\
75.01	0\\
76.01	0\\
77.01	0\\
78.01	0\\
79.01	0\\
80.01	0\\
81.01	0\\
82.01	0\\
83.01	0\\
84.01	0\\
85.01	0\\
86.01	0\\
87.01	0\\
88.01	0\\
89.01	0\\
90.01	0\\
91.01	0\\
92.01	0\\
93.01	0\\
94.01	0\\
95.01	0\\
96.01	0\\
97.01	0\\
98.01	0\\
99.01	0.00328845890704418\\
99.02	0.00333351020433462\\
99.03	0.00337872989622107\\
99.04	0.00342410781940597\\
99.05	0.00346969067490262\\
99.06	0.00351546976428445\\
99.07	0.00356143509901703\\
99.08	0.00360757607931048\\
99.09	0.00365388146525549\\
99.1	0.00370033934650487\\
99.11	0.00374693711037871\\
99.12	0.00379366140830112\\
99.13	0.00384049812047099\\
99.14	0.0038874580034141\\
99.15	0.00393483565423353\\
99.16	0.00398263494425564\\
99.17	0.00403085980130233\\
99.18	0.0040795142118173\\
99.19	0.00412860222311627\\
99.2	0.00417812794569628\\
99.21	0.00422809555566403\\
99.22	0.00427850924718199\\
99.23	0.00432937310681449\\
99.24	0.0043806912758079\\
99.25	0.00443246795197619\\
99.26	0.00448470739168515\\
99.27	0.00453741391194087\\
99.28	0.00459059189258777\\
99.29	0.0046442457786222\\
99.3	0.00469838008262783\\
99.31	0.00475299938733924\\
99.32	0.00480810834834083\\
99.33	0.00486371169690803\\
99.34	0.00491981424299882\\
99.35	0.0049764208784027\\
99.36	0.00503353658005642\\
99.37	0.00509116641355064\\
99.38	0.00514931553692431\\
99.39	0.00520798920455344\\
99.4	0.00526719277124689\\
99.41	0.00532693169656051\\
99.42	0.00538721084840616\\
99.43	0.00544803513735539\\
99.44	0.00550940951836646\\
99.45	0.00557133899114194\\
99.46	0.0056338286004846\\
99.47	0.00569688343665094\\
99.48	0.0057605086360643\\
99.49	0.00582470938215867\\
99.5	0.00588949090583306\\
99.51	0.00595485848591075\\
99.52	0.00602081744960348\\
99.53	0.00608737317298039\\
99.54	0.00615453108144196\\
99.55	0.00622229665019873\\
99.56	0.00629067540475496\\
99.57	0.00635967292139696\\
99.58	0.00642929482768636\\
99.59	0.0064995468029579\\
99.6	0.00657043457882205\\
99.61	0.00664196393967204\\
99.62	0.0067141407231954\\
99.63	0.00678697082088982\\
99.64	0.00686046017858321\\
99.65	0.00693461479695774\\
99.66	0.0070094407320778\\
99.67	0.00708494409592144\\
99.68	0.00716113105691529\\
99.69	0.00723800782441484\\
99.7	0.00731558066288144\\
99.71	0.0073938558945165\\
99.72	0.0074728398998023\\
99.73	0.0075525391180442\\
99.74	0.00763296004791393\\
99.75	0.0077141092479932\\
99.76	0.00779599333731718\\
99.77	0.00787861899591701\\
99.78	0.00796199296536071\\
99.79	0.00804612204929144\\
99.8	0.00813101311396239\\
99.81	0.00821667308876694\\
99.82	0.00830310896676325\\
99.83	0.00839032780519167\\
99.84	0.00847833672598379\\
99.85	0.00856714291626138\\
99.86	0.00865675362882351\\
99.87	0.00874717618262004\\
99.88	0.00883841796320915\\
99.89	0.00893048642319673\\
99.9	0.00902338908265498\\
99.91	0.00911713352951744\\
99.92	0.00921172741994723\\
99.93	0.00930717847867511\\
99.94	0.00940349449930366\\
99.95	0.00950068334457319\\
99.96	0.00959875294658504\\
99.97	0.009697711306977\\
99.98	0.00979756649704546\\
99.99	0.00989832665780802\\
100	0.01\\
};
\addlegendentry{$q=-3$};

\addplot [color=red,dashed]
  table[row sep=crcr]{%
0.01	0\\
1.01	0\\
2.01	0\\
3.01	0\\
4.01	0\\
5.01	0\\
6.01	0\\
7.01	0\\
8.01	0\\
9.01	0\\
10.01	0\\
11.01	0\\
12.01	0\\
13.01	0\\
14.01	0\\
15.01	0\\
16.01	0\\
17.01	0\\
18.01	0\\
19.01	0\\
20.01	0\\
21.01	0\\
22.01	0\\
23.01	0\\
24.01	0\\
25.01	0\\
26.01	0\\
27.01	0\\
28.01	0\\
29.01	0\\
30.01	0\\
31.01	0\\
32.01	0\\
33.01	0\\
34.01	0\\
35.01	0\\
36.01	0\\
37.01	0\\
38.01	0\\
39.01	0\\
40.01	0\\
41.01	0\\
42.01	0\\
43.01	0\\
44.01	0\\
45.01	0\\
46.01	0\\
47.01	0\\
48.01	0\\
49.01	0\\
50.01	0\\
51.01	0\\
52.01	0\\
53.01	0\\
54.01	0\\
55.01	0\\
56.01	0\\
57.01	0\\
58.01	0\\
59.01	0\\
60.01	0\\
61.01	0\\
62.01	0\\
63.01	0\\
64.01	0\\
65.01	0\\
66.01	0\\
67.01	0\\
68.01	0\\
69.01	0\\
70.01	0\\
71.01	0\\
72.01	0\\
73.01	0\\
74.01	0\\
75.01	0\\
76.01	0\\
77.01	0\\
78.01	0\\
79.01	0\\
80.01	0\\
81.01	0\\
82.01	0\\
83.01	0\\
84.01	0\\
85.01	0\\
86.01	0\\
87.01	0\\
88.01	0\\
89.01	0\\
90.01	0\\
91.01	0\\
92.01	0\\
93.01	0\\
94.01	0\\
95.01	0\\
96.01	0\\
97.01	0\\
98.01	0\\
99.01	0.00104946028175264\\
99.02	0.00125306395411695\\
99.03	0.00145818452931908\\
99.04	0.0016648453630583\\
99.05	0.00187301328746817\\
99.06	0.00208271075060248\\
99.07	0.00229396160842808\\
99.08	0.00250679044560442\\
99.09	0.00272122260386963\\
99.1	0.002937284219982\\
99.11	0.0031550022516486\\
99.12	0.00337440450742363\\
99.13	0.00359551968105016\\
99.14	0.00380493250998101\\
99.15	0.00385656934463024\\
99.16	0.00390855237475965\\
99.17	0.00396087846322035\\
99.18	0.00401354417880869\\
99.19	0.00406654578056501\\
99.2	0.00411987921666678\\
99.21	0.00417354011061869\\
99.22	0.00422753536402045\\
99.23	0.00428190027022628\\
99.24	0.00433663114616696\\
99.25	0.0043917239783746\\
99.26	0.00444717440792617\\
99.27	0.00450297771469116\\
99.28	0.0045591288008455\\
99.29	0.00461562217361109\\
99.3	0.00467245192717746\\
99.31	0.00472961172375959\\
99.32	0.0047870947737426\\
99.33	0.0048448938148608\\
99.34	0.00490300109035484\\
99.35	0.00496140832634903\\
99.36	0.00502010670781841\\
99.37	0.00507908684954088\\
99.38	0.0051383387442909\\
99.39	0.00519785176102771\\
99.4	0.00525761461537342\\
99.41	0.00531761533853545\\
99.42	0.00537815543682032\\
99.43	0.00543924041152297\\
99.44	0.00550087518071841\\
99.45	0.00556306471542116\\
99.46	0.0056258140408721\\
99.47	0.00568912823785012\\
99.48	0.00575301234814035\\
99.49	0.00581747134619418\\
99.5	0.00588251024853272\\
99.51	0.00594813411434026\\
99.52	0.00601434804608568\\
99.53	0.00608115719017392\\
99.54	0.00614856673762911\\
99.55	0.00621658192481169\\
99.56	0.00628520803417149\\
99.57	0.00635445039503925\\
99.58	0.00642431438445891\\
99.59	0.00649480542806346\\
99.6	0.00656592900099709\\
99.61	0.00663769062888668\\
99.62	0.00671009588886575\\
99.63	0.00678315041065447\\
99.64	0.00685685987769925\\
99.65	0.00693123002837576\\
99.66	0.00700626665725987\\
99.67	0.00708197561647047\\
99.68	0.00715836281708942\\
99.69	0.00723543421531813\\
99.7	0.00731319582612591\\
99.71	0.00739165372783235\\
99.72	0.00747081406389988\\
99.73	0.00755068304485562\\
99.74	0.00763126695028889\\
99.75	0.00771257213096943\\
99.76	0.00779460501109486\\
99.77	0.00787737209067585\\
99.78	0.00796087994806859\\
99.79	0.0080451352426644\\
99.8	0.00813014471774746\\
99.81	0.00821591520353212\\
99.82	0.00830245362039213\\
99.83	0.00838976698229511\\
99.84	0.00847786240045663\\
99.85	0.008566747087229\\
99.86	0.00865642836024134\\
99.87	0.00874691364680851\\
99.88	0.0088382104886279\\
99.89	0.00893032654678435\\
99.9	0.0090232696070851\\
99.91	0.00911704758574675\\
99.92	0.00921166853545841\\
99.93	0.0093071406518502\\
99.94	0.009403472280396\\
99.95	0.00950067192378203\\
99.96	0.00959874824977478\\
99.97	0.00969771009962486\\
99.98	0.00979756649704546\\
99.99	0.00989832665780802\\
100	0.01\\
};
\addlegendentry{$q=-2$};

\addplot [color=blue,dashed]
  table[row sep=crcr]{%
0.01	0\\
1.01	0\\
2.01	0\\
3.01	0\\
4.01	0\\
5.01	0\\
6.01	0\\
7.01	0\\
8.01	0\\
9.01	0\\
10.01	0\\
11.01	0\\
12.01	0\\
13.01	0\\
14.01	0\\
15.01	0\\
16.01	0\\
17.01	0\\
18.01	0\\
19.01	0\\
20.01	0\\
21.01	0\\
22.01	0\\
23.01	0\\
24.01	0\\
25.01	0\\
26.01	0\\
27.01	0\\
28.01	0\\
29.01	0\\
30.01	0\\
31.01	0\\
32.01	0\\
33.01	0\\
34.01	0\\
35.01	0\\
36.01	0\\
37.01	0\\
38.01	0\\
39.01	0\\
40.01	0\\
41.01	0\\
42.01	0\\
43.01	0\\
44.01	0\\
45.01	0\\
46.01	0\\
47.01	0\\
48.01	0\\
49.01	0\\
50.01	0\\
51.01	0\\
52.01	0\\
53.01	0\\
54.01	0\\
55.01	0\\
56.01	0\\
57.01	0\\
58.01	0\\
59.01	0\\
60.01	0\\
61.01	0\\
62.01	0\\
63.01	0\\
64.01	0\\
65.01	0\\
66.01	0\\
67.01	0\\
68.01	0\\
69.01	0\\
70.01	0\\
71.01	0\\
72.01	0\\
73.01	0\\
74.01	0\\
75.01	0\\
76.01	0\\
77.01	0\\
78.01	0\\
79.01	0\\
80.01	0\\
81.01	0\\
82.01	0\\
83.01	0\\
84.01	0\\
85.01	0\\
86.01	0\\
87.01	0\\
88.01	0\\
89.01	0\\
90.01	0\\
91.01	0\\
92.01	0\\
93.01	0\\
94.01	0\\
95.01	0\\
96.01	0\\
97.01	0\\
98.01	0\\
99.01	0\\
99.02	0\\
99.03	0\\
99.04	0\\
99.05	0\\
99.06	0\\
99.07	0\\
99.08	0\\
99.09	0\\
99.1	0\\
99.11	0\\
99.12	0\\
99.13	0\\
99.14	1.34192545933262e-05\\
99.15	0.000186052377303365\\
99.16	0.000359788996779725\\
99.17	0.000534644183888877\\
99.18	0.000710633127091898\\
99.19	0.000887771451375849\\
99.2	0.00106607520955938\\
99.21	0.00124556089630498\\
99.22	0.00142623390412353\\
99.23	0.00160807155888041\\
99.24	0.00179109029889443\\
99.25	0.00197530702975846\\
99.26	0.00216073914018688\\
99.27	0.00234740451849972\\
99.28	0.00253532156977716\\
99.29	0.00272450923372016\\
99.3	0.00291498700325548\\
99.31	0.00310677494392576\\
99.32	0.00329989371410826\\
99.33	0.00349436458610888\\
99.34	0.00369020946818107\\
99.35	0.00388745077047987\\
99.36	0.00408611158292441\\
99.37	0.00428621572535727\\
99.38	0.00448778793606267\\
99.39	0.00469085371780874\\
99.4	0.00489543938389655\\
99.41	0.00510157207241651\\
99.42	0.00516645903340951\\
99.43	0.00523160273235865\\
99.44	0.00529730533150437\\
99.45	0.0053635695660718\\
99.46	0.00543039808954691\\
99.47	0.00549779347615066\\
99.48	0.00556577769541155\\
99.49	0.0056343774394848\\
99.5	0.0057035963880993\\
99.51	0.00577343816246014\\
99.52	0.00584390632119317\\
99.53	0.00591500435610443\\
99.54	0.00598673568774476\\
99.55	0.00605910366076959\\
99.56	0.0061321115390831\\
99.57	0.00620576250075546\\
99.58	0.00628005963270098\\
99.59	0.00635500592510444\\
99.6	0.00643060426558196\\
99.61	0.00650685743306189\\
99.62	0.00658376809137045\\
99.63	0.00666133878250572\\
99.64	0.00673957191958258\\
99.65	0.00681846977943027\\
99.66	0.00689803449482267\\
99.67	0.00697826804632059\\
99.68	0.00705917225370364\\
99.69	0.00714074876711371\\
99.7	0.00722299905762269\\
99.71	0.00730592440728626\\
99.72	0.00738952589167773\\
99.73	0.00747380436075851\\
99.74	0.00755876044180094\\
99.75	0.00764439452717997\\
99.76	0.00773070676151409\\
99.77	0.00781769702811535\\
99.78	0.007905364934706\\
99.79	0.00799370979835622\\
99.8	0.00808273062959414\\
99.81	0.008172426115636\\
99.82	0.0082627946026807\\
99.83	0.00835383407720883\\
99.84	0.00844554214622215\\
99.85	0.00853791601635481\\
99.86	0.00863095247178254\\
99.87	0.00872464785085084\\
99.88	0.00881899802133705\\
99.89	0.00891399835425527\\
99.9	0.0090096436961058\\
99.91	0.00910592833996096\\
99.92	0.00920284599502165\\
99.93	0.0093003897536808\\
99.94	0.00939855205657116\\
99.95	0.00949732465545663\\
99.96	0.00959669857381528\\
99.97	0.00969666406494989\\
99.98	0.00979721056744917\\
99.99	0.00989832665780802\\
100	0.01\\
};
\addlegendentry{$q=-1$};

\addplot [color=black,solid]
  table[row sep=crcr]{%
0.01	0\\
1.01	0\\
2.01	0\\
3.01	0\\
4.01	0\\
5.01	0\\
6.01	0\\
7.01	0\\
8.01	0\\
9.01	0\\
10.01	0\\
11.01	0\\
12.01	0\\
13.01	0\\
14.01	0\\
15.01	0\\
16.01	0\\
17.01	0\\
18.01	0\\
19.01	0\\
20.01	0\\
21.01	0\\
22.01	0\\
23.01	0\\
24.01	0\\
25.01	0\\
26.01	0\\
27.01	0\\
28.01	0\\
29.01	0\\
30.01	0\\
31.01	0\\
32.01	0\\
33.01	0\\
34.01	0\\
35.01	0\\
36.01	0\\
37.01	0\\
38.01	0\\
39.01	0\\
40.01	0\\
41.01	0\\
42.01	0\\
43.01	0\\
44.01	0\\
45.01	0\\
46.01	0\\
47.01	0\\
48.01	0\\
49.01	0\\
50.01	0\\
51.01	0\\
52.01	0\\
53.01	0\\
54.01	0\\
55.01	0\\
56.01	0\\
57.01	0\\
58.01	0\\
59.01	0\\
60.01	0\\
61.01	0\\
62.01	0\\
63.01	0\\
64.01	0\\
65.01	0\\
66.01	0\\
67.01	0\\
68.01	0\\
69.01	0\\
70.01	0\\
71.01	0\\
72.01	0\\
73.01	0\\
74.01	0\\
75.01	0\\
76.01	0\\
77.01	0\\
78.01	0\\
79.01	0\\
80.01	0\\
81.01	0\\
82.01	0\\
83.01	0\\
84.01	0\\
85.01	0\\
86.01	0\\
87.01	0\\
88.01	0\\
89.01	0\\
90.01	0\\
91.01	0\\
92.01	0\\
93.01	0\\
94.01	0\\
95.01	0\\
96.01	0\\
97.01	0\\
98.01	0\\
99.01	0\\
99.02	0\\
99.03	0\\
99.04	0\\
99.05	0\\
99.06	0\\
99.07	0\\
99.08	0\\
99.09	0\\
99.1	0\\
99.11	0\\
99.12	0\\
99.13	0\\
99.14	0\\
99.15	0\\
99.16	0\\
99.17	0\\
99.18	0\\
99.19	0\\
99.2	0\\
99.21	0\\
99.22	0\\
99.23	0\\
99.24	0\\
99.25	0\\
99.26	0\\
99.27	0\\
99.28	0\\
99.29	0\\
99.3	0\\
99.31	0\\
99.32	0\\
99.33	0\\
99.34	0\\
99.35	0\\
99.36	0\\
99.37	0\\
99.38	0\\
99.39	0\\
99.4	0\\
99.41	0\\
99.42	0.000142507484711799\\
99.43	0.00028603094158923\\
99.44	0.000430279959184088\\
99.45	0.000575263732640401\\
99.46	0.000720991715020906\\
99.47	0.000867473608317394\\
99.48	0.00101470001362642\\
99.49	0.00116265714149537\\
99.5	0.00131135440874501\\
99.51	0.00146080148760354\\
99.52	0.00161100831282696\\
99.53	0.00176198508903214\\
99.54	0.00191374229825145\\
99.55	0.00206629070771789\\
99.56	0.00221964137789047\\
99.57	0.00237380567073002\\
99.58	0.00252879525823614\\
99.59	0.00268462213125676\\
99.6	0.00284129860858223\\
99.61	0.00299883734633685\\
99.62	0.00315725134768115\\
99.63	0.00331655397283938\\
99.64	0.00347675894946723\\
99.65	0.00363788038337595\\
99.66	0.00379993276962971\\
99.67	0.00396293100403452\\
99.68	0.0041268903950377\\
99.69	0.00429182667605687\\
99.7	0.00445775601826144\\
99.71	0.00462469504382956\\
99.72	0.00479266083977007\\
99.73	0.00496167097234289\\
99.74	0.00513174350189687\\
99.75	0.00530289699829066\\
99.76	0.00547515055692814\\
99.77	0.00564852381544173\\
99.78	0.00582303697105938\\
99.79	0.00599871079869321\\
99.8	0.00617556666979046\\
99.81	0.00635362657199012\\
99.82	0.00653291312963158\\
99.83	0.00671344962516493\\
99.84	0.00689526002151581\\
99.85	0.00707836898546157\\
99.86	0.00726280191207965\\
99.87	0.00744858495033297\\
99.88	0.00763574502986251\\
99.89	0.0078243098890616\\
99.9	0.00801430810451265\\
99.91	0.0082057689535157\\
99.92	0.00839872238447791\\
99.93	0.00859319927334343\\
99.94	0.00878923145914393\\
99.95	0.00898685178139854\\
99.96	0.0091860941194868\\
99.97	0.0093869934341281\\
99.98	0.00958958581111154\\
99.99	0.0097939085074319\\
100	0.01\\
};
\addlegendentry{$q=0$};

\addplot [color=blue,solid]
  table[row sep=crcr]{%
0.01	0.00922742732202166\\
1.01	0.0092274130399497\\
2.01	0.00922739792741873\\
3.01	0.00922738222131146\\
4.01	0.00922736589596059\\
5.01	0.00922734892435787\\
6.01	0.00922733127806274\\
7.01	0.00922731292710504\\
8.01	0.00922729383988779\\
9.01	0.00922727398311702\\
10.01	0.00922725332186595\\
11.01	0.00922723182015138\\
12.01	0.0092272094430556\\
13.01	0.00922718616149337\\
14.01	0.00922716195212826\\
15.01	0.00922713676520024\\
16.01	0.00922711052046818\\
17.01	0.0092270831603759\\
18.01	0.00922705462948393\\
19.01	0.00922702486907789\\
20.01	0.00922699381699539\\
21.01	0.00922696140742542\\
22.01	0.0092269275705294\\
23.01	0.00922689223118527\\
24.01	0.00922685530378333\\
25.01	0.00922681666996403\\
26.01	0.00922677608529397\\
27.01	0.00922673280502235\\
28.01	0.00922668419077714\\
29.01	0.00922662106269991\\
30.01	0.00922651492566772\\
31.01	0.00922633446516904\\
32.01	0.00922613612752173\\
33.01	0.00922592990882459\\
34.01	0.00922571540224445\\
35.01	0.0092254921680819\\
36.01	0.00922525972966405\\
37.01	0.00922501756860596\\
38.01	0.00922476511943845\\
39.01	0.00922450176378566\\
40.01	0.00922422682443228\\
41.01	0.00922393955799911\\
42.01	0.00922363913562648\\
43.01	0.00922332459722462\\
44.01	0.00922299485833425\\
45.01	0.00922264875532557\\
46.01	0.00922228502688403\\
47.01	0.00922190241754074\\
48.01	0.00922150004276925\\
49.01	0.00922107782147045\\
50.01	0.00922063429811563\\
51.01	0.009220163730359\\
52.01	0.00921966145111034\\
53.01	0.00921912045848829\\
54.01	0.00921852278350417\\
55.01	0.00921781276658482\\
56.01	0.00921683754706864\\
57.01	0.00921537198395175\\
58.01	0.00921366886788504\\
59.01	0.00921188506743619\\
60.01	0.00921001457423623\\
61.01	0.00920805067758686\\
62.01	0.00920598582982146\\
63.01	0.00920381142141993\\
64.01	0.00920151730993209\\
65.01	0.00919909069106331\\
66.01	0.00919651353506634\\
67.01	0.00919376018707084\\
68.01	0.0091908105038698\\
69.01	0.00918760132951691\\
70.01	0.00918388721130242\\
71.01	0.00917883389870093\\
72.01	0.00917123417438804\\
73.01	0.00916016398907375\\
74.01	0.00914478415822474\\
75.01	0.0091278394184736\\
76.01	0.00910898099118535\\
77.01	0.0090863842332633\\
78.01	0.00905202319190744\\
79.01	0.00901183623729243\\
80.01	0.00896862825047214\\
81.01	0.00891994859362619\\
82.01	0.00886281994865658\\
83.01	0.00879447242289795\\
84.01	0.00869084382124653\\
85.01	0.00857162949842379\\
86.01	0.00844580465448999\\
87.01	0.00831061982868391\\
88.01	0.00815379040255077\\
89.01	0.0079242568210914\\
90.01	0.00766057592530038\\
91.01	0.00738581406637025\\
92.01	0.00709892259758851\\
93.01	0.00679870494687675\\
94.01	0.00648388363909919\\
95.01	0.00615365911855247\\
96.01	0.00580990015028039\\
97.01	0.00544683142545227\\
98.01	0.00501480028686344\\
99.01	0.00417782896024826\\
99.02	0.00416297391082926\\
99.03	0.00414787847786715\\
99.04	0.00413253714368365\\
99.05	0.00411694425386603\\
99.06	0.00410109401350807\\
99.07	0.00408498048333106\\
99.08	0.00406859757568049\\
99.09	0.00405193905039347\\
99.1	0.0040349985105342\\
99.11	0.00401776939798825\\
99.12	0.00400024498891175\\
99.13	0.00398241838903006\\
99.14	0.0039642825287796\\
99.15	0.00394583015828669\\
99.16	0.00392705384217749\\
99.17	0.00390794595421406\\
99.18	0.00388849867174659\\
99.19	0.00386870396997163\\
99.2	0.00384855361599089\\
99.21	0.00382803916266256\\
99.22	0.003807151942236\\
99.23	0.00378588305976082\\
99.24	0.0037642233862604\\
99.25	0.0037421635516597\\
99.26	0.00371969393745665\\
99.27	0.0036968046691257\\
99.28	0.00367348560824153\\
99.29	0.0036497263443105\\
99.3	0.00362551618629647\\
99.31	0.00360084413655559\\
99.32	0.00357569888582327\\
99.33	0.00355006881594236\\
99.34	0.00352394198998781\\
99.35	0.00349730614200647\\
99.36	0.00347014866635337\\
99.37	0.00344245660660522\\
99.38	0.00341425888549942\\
99.39	0.0033855659980539\\
99.4	0.00335636581065441\\
99.41	0.00332664584999986\\
99.42	0.00329639329011699\\
99.43	0.00326559492261193\\
99.44	0.00323425675308237\\
99.45	0.00320238362600423\\
99.46	0.0031699622610237\\
99.47	0.00313697899702963\\
99.48	0.00310341977913769\\
99.49	0.00306927014512682\\
99.5	0.00303451521130009\\
99.51	0.00299913965773969\\
99.52	0.00296312771292437\\
99.53	0.00292646313767548\\
99.54	0.00288912920839561\\
99.55	0.00285110869956154\\
99.56	0.00281238386543081\\
99.57	0.00277293642091844\\
99.58	0.00273274752159758\\
99.59	0.00269179774277483\\
99.6	0.00265006705758736\\
99.61	0.00260753481406595\\
99.62	0.00256417971110353\\
99.63	0.0025199797732653\\
99.64	0.00247491232437151\\
99.65	0.0024289539597794\\
99.66	0.00238208051728548\\
99.67	0.00233426704656365\\
99.68	0.00228548777704836\\
99.69	0.00223571608416403\\
99.7	0.00218492445379856\\
99.71	0.00213308444490778\\
99.72	0.00208016665012936\\
99.73	0.00202614065427561\\
99.74	0.00197097499056455\\
99.75	0.00191463709443744\\
99.76	0.00185709325479893\\
99.77	0.00179830856250254\\
99.78	0.00173824685588996\\
99.79	0.00167687066317651\\
99.8	0.00161414114145789\\
99.81	0.00155001801209408\\
99.82	0.00148445949220548\\
99.83	0.00141742222199301\\
99.84	0.00134886118756874\\
99.85	0.00127872963895532\\
99.86	0.00120697900288187\\
99.87	0.00113355878996945\\
99.88	0.00105841649586151\\
99.89	0.000981497495812871\\
99.9	0.000902744932204142\\
99.91	0.000822099594396612\\
99.92	0.000739499790285361\\
99.93	0.000654881208843966\\
99.94	0.000568176772882604\\
99.95	0.000479316481161323\\
99.96	0.000388227238910565\\
99.97	0.000294832675710451\\
99.98	0.000199052949567459\\
99.99	0.000100804535899853\\
100	0\\
};
\addlegendentry{$q=1$};

\addplot [color=red,solid]
  table[row sep=crcr]{%
0.01	0.00402362825950001\\
1.01	0.00402351587962618\\
2.01	0.00402339029379895\\
3.01	0.00402325978601299\\
4.01	0.00402312414476495\\
5.01	0.00402298314765832\\
6.01	0.00402283656067248\\
7.01	0.00402268413737226\\
8.01	0.00402252561806117\\
9.01	0.00402236072891558\\
10.01	0.00402218918130391\\
11.01	0.00402201067232423\\
12.01	0.00402182489147851\\
13.01	0.00402163155271235\\
14.01	0.004021430481385\\
15.01	0.00402122150854012\\
16.01	0.00402100390779924\\
17.01	0.00402077711062517\\
18.01	0.00402054066519825\\
19.01	0.00402029409346165\\
20.01	0.00402003688971778\\
21.01	0.00401976851939623\\
22.01	0.004019488417831\\
23.01	0.00401919598837888\\
24.01	0.00401889059640013\\
25.01	0.00401857154232075\\
26.01	0.0040182379328601\\
27.01	0.00401788805657958\\
28.01	0.00401751641407862\\
29.01	0.00401710090792436\\
30.01	0.00401654043053287\\
31.01	0.00401524077524837\\
32.01	0.00401356328581714\\
33.01	0.00401181919838981\\
34.01	0.00401000510163408\\
35.01	0.00400811731348965\\
36.01	0.00400615184786791\\
37.01	0.00400410437593584\\
38.01	0.00400197018113882\\
39.01	0.00399974410753812\\
40.01	0.0039974205029865\\
41.01	0.00399499316308984\\
42.01	0.00399245526468203\\
43.01	0.00398979908676834\\
44.01	0.00398701559597362\\
45.01	0.00398409504346138\\
46.01	0.0039810266773342\\
47.01	0.00397779871373349\\
48.01	0.00397439986095259\\
49.01	0.0039708263726504\\
50.01	0.003967087031578\\
51.01	0.00396313903742749\\
52.01	0.00395893722306839\\
53.01	0.00395444771955571\\
54.01	0.00394961335081805\\
55.01	0.00394427121862288\\
56.01	0.0039377329305626\\
57.01	0.00392734508317033\\
58.01	0.00391272863316338\\
59.01	0.00389740460488283\\
60.01	0.00388131958531887\\
61.01	0.00386441396860172\\
62.01	0.00384662097911993\\
63.01	0.00382786546168491\\
64.01	0.00380806233451378\\
65.01	0.00378711476944264\\
66.01	0.00376491376922966\\
67.01	0.0037413440497164\\
68.01	0.00371626118832227\\
69.01	0.00368942935742867\\
70.01	0.00366029305354543\\
71.01	0.00362606065237299\\
72.01	0.00357017100527519\\
73.01	0.0035039506031637\\
74.01	0.00343495441141227\\
75.01	0.00336092299662772\\
76.01	0.00328298424162559\\
77.01	0.00320051274712849\\
78.01	0.00311921445800905\\
79.01	0.00304018641402235\\
80.01	0.00295782854570667\\
81.01	0.00286161099820042\\
82.01	0.00270468057325484\\
83.01	0.0025219402204106\\
84.01	0.00234611554577139\\
85.01	0.00217116590119847\\
86.01	0.00198846164238502\\
87.01	0.00179611854344243\\
88.01	0.00159085725273375\\
89.01	0.00140325190254498\\
90.01	0.00122620645439562\\
91.01	0.00104124992940481\\
92.01	0.00084792810317127\\
93.01	0.000645779928157975\\
94.01	0.000434474210155834\\
95.01	0.000215534575809176\\
96.01	1.96404939382718e-05\\
97.01	0\\
98.01	1.73472347597681e-18\\
99.01	0\\
99.02	0\\
99.03	0\\
99.04	0\\
99.05	1.73472347597681e-18\\
99.06	0\\
99.07	0\\
99.08	1.73472347597681e-18\\
99.09	0\\
99.1	0\\
99.11	1.73472347597681e-18\\
99.12	0\\
99.13	0\\
99.14	0\\
99.15	0\\
99.16	0\\
99.17	1.73472347597681e-18\\
99.18	0\\
99.19	0\\
99.2	0\\
99.21	0\\
99.22	0\\
99.23	0\\
99.24	1.73472347597681e-18\\
99.25	0\\
99.26	0\\
99.27	1.73472347597681e-18\\
99.28	0\\
99.29	0\\
99.3	0\\
99.31	0\\
99.32	0\\
99.33	0\\
99.34	0\\
99.35	0\\
99.36	0\\
99.37	0\\
99.38	0\\
99.39	0\\
99.4	0\\
99.41	0\\
99.42	0\\
99.43	0\\
99.44	0\\
99.45	0\\
99.46	0\\
99.47	0\\
99.48	0\\
99.49	0\\
99.5	0\\
99.51	1.73472347597681e-18\\
99.52	0\\
99.53	0\\
99.54	0\\
99.55	0\\
99.56	0\\
99.57	0\\
99.58	0\\
99.59	0\\
99.6	0\\
99.61	1.73472347597681e-18\\
99.62	0\\
99.63	0\\
99.64	0\\
99.65	1.73472347597681e-18\\
99.66	0\\
99.67	1.73472347597681e-18\\
99.68	0\\
99.69	0\\
99.7	0\\
99.71	0\\
99.72	0\\
99.73	0\\
99.74	0\\
99.75	0\\
99.76	0\\
99.77	1.73472347597681e-18\\
99.78	0\\
99.79	0\\
99.8	0\\
99.81	0\\
99.82	0\\
99.83	0\\
99.84	0\\
99.85	0\\
99.86	0\\
99.87	0\\
99.88	0\\
99.89	0\\
99.9	0\\
99.91	0\\
99.92	0\\
99.93	0\\
99.94	0\\
99.95	0\\
99.96	0\\
99.97	0\\
99.98	0\\
99.99	0\\
100	0\\
};
\addlegendentry{$q=2$};

\addplot [color=mycolor1,solid]
  table[row sep=crcr]{%
0.01	0.00180108844777369\\
1.01	0.00179984246580634\\
2.01	0.00179823173510353\\
3.01	0.00179655773666776\\
4.01	0.00179481775455851\\
5.01	0.00179300893492088\\
6.01	0.00179112827705123\\
7.01	0.00178917262374545\\
8.01	0.00178713865088356\\
9.01	0.00178502285626283\\
10.01	0.00178282154804389\\
11.01	0.00178053083528039\\
12.01	0.00177814663585204\\
13.01	0.00177566479313577\\
14.01	0.00177308174165706\\
15.01	0.00177039578766906\\
16.01	0.00176760016414396\\
17.01	0.00176468662391234\\
18.01	0.00176164945348996\\
19.01	0.0017584826310553\\
20.01	0.00175517980316267\\
21.01	0.00175173427327766\\
22.01	0.00174813899362994\\
23.01	0.00174438656115718\\
24.01	0.00174046921478758\\
25.01	0.00173637880981325\\
26.01	0.00173210661519279\\
27.01	0.00172764193619225\\
28.01	0.00172296286916978\\
29.01	0.00171798170291983\\
30.01	0.00171252229161471\\
31.01	0.00170711068742848\\
32.01	0.00170173956209494\\
33.01	0.00169615557190633\\
34.01	0.00169034803230301\\
35.01	0.001684305455391\\
36.01	0.00167801546038683\\
37.01	0.00167146467134716\\
38.01	0.00166463860039271\\
39.01	0.00165752151542814\\
40.01	0.00165009629585938\\
41.01	0.00164234429673888\\
42.01	0.00163424525838736\\
43.01	0.00162577666146905\\
44.01	0.00161690984190359\\
45.01	0.00160761456231607\\
46.01	0.00159785989871978\\
47.01	0.00158761256245439\\
48.01	0.00157684419635927\\
49.01	0.00156559129391649\\
50.01	0.00155426769843272\\
51.01	0.00154290317002742\\
52.01	0.00153084890945793\\
53.01	0.00151802570807007\\
54.01	0.0015043147273238\\
55.01	0.00148932950688072\\
56.01	0.00147010534762902\\
57.01	0.00143763784887765\\
58.01	0.00140509227545263\\
59.01	0.00137107252375266\\
60.01	0.00133548917503814\\
61.01	0.00129824502343478\\
62.01	0.00125923407963422\\
63.01	0.00121834037200655\\
64.01	0.00117543653588249\\
65.01	0.00113038264286854\\
66.01	0.00108302726747121\\
67.01	0.00103320416477684\\
68.01	0.000980687385888791\\
69.01	0.000925145846401404\\
70.01	0.000865895158193228\\
71.01	0.000801309291955115\\
72.01	0.000740779584327448\\
73.01	0.00068271907955952\\
74.01	0.000606375862709216\\
75.01	0.000482764482176146\\
76.01	0.000354261142763266\\
77.01	0.000221793341475427\\
78.01	9.0130111263095e-05\\
79.01	0\\
80.01	0\\
81.01	0\\
82.01	1.73472347597681e-18\\
83.01	0\\
84.01	0\\
85.01	0\\
86.01	1.73472347597681e-18\\
87.01	0\\
88.01	1.73472347597681e-18\\
89.01	1.73472347597681e-18\\
90.01	0\\
91.01	1.73472347597681e-18\\
92.01	0\\
93.01	0\\
94.01	1.73472347597681e-18\\
95.01	1.73472347597681e-18\\
96.01	1.73472347597681e-18\\
97.01	0\\
98.01	1.73472347597681e-18\\
99.01	0\\
99.02	0\\
99.03	0\\
99.04	0\\
99.05	1.73472347597681e-18\\
99.06	0\\
99.07	0\\
99.08	1.73472347597681e-18\\
99.09	0\\
99.1	0\\
99.11	1.73472347597681e-18\\
99.12	0\\
99.13	0\\
99.14	0\\
99.15	0\\
99.16	0\\
99.17	1.73472347597681e-18\\
99.18	0\\
99.19	0\\
99.2	0\\
99.21	0\\
99.22	0\\
99.23	0\\
99.24	1.73472347597681e-18\\
99.25	0\\
99.26	0\\
99.27	1.73472347597681e-18\\
99.28	0\\
99.29	0\\
99.3	0\\
99.31	0\\
99.32	0\\
99.33	0\\
99.34	0\\
99.35	0\\
99.36	0\\
99.37	0\\
99.38	0\\
99.39	0\\
99.4	0\\
99.41	0\\
99.42	0\\
99.43	0\\
99.44	0\\
99.45	0\\
99.46	0\\
99.47	0\\
99.48	0\\
99.49	0\\
99.5	0\\
99.51	1.73472347597681e-18\\
99.52	0\\
99.53	0\\
99.54	0\\
99.55	0\\
99.56	0\\
99.57	0\\
99.58	0\\
99.59	0\\
99.6	0\\
99.61	1.73472347597681e-18\\
99.62	0\\
99.63	0\\
99.64	0\\
99.65	1.73472347597681e-18\\
99.66	0\\
99.67	1.73472347597681e-18\\
99.68	0\\
99.69	0\\
99.7	0\\
99.71	0\\
99.72	0\\
99.73	0\\
99.74	0\\
99.75	0\\
99.76	0\\
99.77	1.73472347597681e-18\\
99.78	0\\
99.79	0\\
99.8	0\\
99.81	0\\
99.82	0\\
99.83	0\\
99.84	0\\
99.85	0\\
99.86	0\\
99.87	0\\
99.88	0\\
99.89	0\\
99.9	0\\
99.91	0\\
99.92	0\\
99.93	0\\
99.94	0\\
99.95	0\\
99.96	0\\
99.97	0\\
99.98	0\\
99.99	0\\
100	0\\
};
\addlegendentry{$q=3$};

\addplot [color=green,solid]
  table[row sep=crcr]{%
0.01	0.000436939440003583\\
1.01	0.000435249038411013\\
2.01	0.000433715076373102\\
3.01	0.000432121203667831\\
4.01	0.000430464765526696\\
5.01	0.000428742950372802\\
6.01	0.000426952775594999\\
7.01	0.000425091071466512\\
8.01	0.000423154462925453\\
9.01	0.000421139348972532\\
10.01	0.000419041879963573\\
11.01	0.000416857936936011\\
12.01	0.000414583145707113\\
13.01	0.000412213189092001\\
14.01	0.00040974666013685\\
15.01	0.000407204605007089\\
16.01	0.000404596272407359\\
17.01	0.000401868534377913\\
18.01	0.000399012668723412\\
19.01	0.000396019856755268\\
20.01	0.000392880264043667\\
21.01	0.000389582869429453\\
22.01	0.000386115257936448\\
23.01	0.000382463367585647\\
24.01	0.000378611172048396\\
25.01	0.000374540243915152\\
26.01	0.000370228904079811\\
27.01	0.000365648712250342\\
28.01	0.000360735918349649\\
29.01	0.000355043552395995\\
30.01	0.00034501410892479\\
31.01	0.000333219282760679\\
32.01	0.000321032933456837\\
33.01	0.000308373354441102\\
34.01	0.00029521944760502\\
35.01	0.000281548946667593\\
36.01	0.000267338333795478\\
37.01	0.000252562748476483\\
38.01	0.000237195887947015\\
39.01	0.000221209899548568\\
40.01	0.000204575270032792\\
41.01	0.000187260725853388\\
42.01	0.000169233069767932\\
43.01	0.000150455923702175\\
44.01	0.000130887770278712\\
45.01	0.000110486949943736\\
46.01	8.92105669107138e-05\\
47.01	6.70151802680437e-05\\
48.01	4.38790378111831e-05\\
49.01	2.00812055629711e-05\\
50.01	6.16418725146728e-07\\
51.01	0\\
52.01	0\\
53.01	0\\
54.01	1.73472347597681e-18\\
55.01	1.73472347597681e-18\\
56.01	0\\
57.01	0\\
58.01	0\\
59.01	0\\
60.01	0\\
61.01	0\\
62.01	0\\
63.01	0\\
64.01	0\\
65.01	0\\
66.01	0\\
67.01	0\\
68.01	0\\
69.01	0\\
70.01	1.73472347597681e-18\\
71.01	0\\
72.01	0\\
73.01	1.73472347597681e-18\\
74.01	0\\
75.01	0\\
76.01	0\\
77.01	0\\
78.01	0\\
79.01	0\\
80.01	0\\
81.01	0\\
82.01	1.73472347597681e-18\\
83.01	0\\
84.01	0\\
85.01	0\\
86.01	1.73472347597681e-18\\
87.01	0\\
88.01	1.73472347597681e-18\\
89.01	1.73472347597681e-18\\
90.01	0\\
91.01	1.73472347597681e-18\\
92.01	0\\
93.01	0\\
94.01	1.73472347597681e-18\\
95.01	1.73472347597681e-18\\
96.01	1.73472347597681e-18\\
97.01	0\\
98.01	1.73472347597681e-18\\
99.01	0\\
99.02	0\\
99.03	0\\
99.04	0\\
99.05	1.73472347597681e-18\\
99.06	0\\
99.07	0\\
99.08	1.73472347597681e-18\\
99.09	0\\
99.1	0\\
99.11	1.73472347597681e-18\\
99.12	0\\
99.13	0\\
99.14	0\\
99.15	0\\
99.16	0\\
99.17	1.73472347597681e-18\\
99.18	0\\
99.19	0\\
99.2	0\\
99.21	0\\
99.22	0\\
99.23	0\\
99.24	1.73472347597681e-18\\
99.25	0\\
99.26	0\\
99.27	1.73472347597681e-18\\
99.28	0\\
99.29	0\\
99.3	0\\
99.31	0\\
99.32	0\\
99.33	0\\
99.34	0\\
99.35	0\\
99.36	0\\
99.37	0\\
99.38	0\\
99.39	0\\
99.4	0\\
99.41	0\\
99.42	0\\
99.43	0\\
99.44	0\\
99.45	0\\
99.46	0\\
99.47	0\\
99.48	0\\
99.49	0\\
99.5	0\\
99.51	1.73472347597681e-18\\
99.52	0\\
99.53	0\\
99.54	0\\
99.55	0\\
99.56	0\\
99.57	0\\
99.58	0\\
99.59	0\\
99.6	0\\
99.61	1.73472347597681e-18\\
99.62	0\\
99.63	0\\
99.64	0\\
99.65	1.73472347597681e-18\\
99.66	0\\
99.67	1.73472347597681e-18\\
99.68	0\\
99.69	0\\
99.7	0\\
99.71	0\\
99.72	0\\
99.73	0\\
99.74	0\\
99.75	0\\
99.76	0\\
99.77	1.73472347597681e-18\\
99.78	0\\
99.79	0\\
99.8	0\\
99.81	0\\
99.82	0\\
99.83	0\\
99.84	0\\
99.85	0\\
99.86	0\\
99.87	0\\
99.88	0\\
99.89	0\\
99.9	0\\
99.91	0\\
99.92	0\\
99.93	0\\
99.94	0\\
99.95	0\\
99.96	0\\
99.97	0\\
99.98	0\\
99.99	0\\
100	0\\
};
\addlegendentry{$q=4$};

\end{axis}
\end{tikzpicture}%

%  \caption{Continuous Time}
%\end{subfigure}%
%\hfill%
%\begin{subfigure}{.45\linewidth}
%  \centering
%  \setlength\figureheight{\linewidth} 
%  \setlength\figurewidth{\linewidth}
%  \tikzsetnextfilename{dm_dscr_z15}
%  % This file was created by matlab2tikz.
%
%The latest updates can be retrieved from
%  http://www.mathworks.com/matlabcentral/fileexchange/22022-matlab2tikz-matlab2tikz
%where you can also make suggestions and rate matlab2tikz.
%
\definecolor{mycolor1}{rgb}{1.00000,0.00000,1.00000}%
%
\begin{tikzpicture}

\begin{axis}[%
width=4.564in,
height=3.803in,
at={(1.067in,0.513in)},
scale only axis,
every outer x axis line/.append style={black},
every x tick label/.append style={font=\color{black}},
xmin=0,
xmax=100,
xlabel={Time},
every outer y axis line/.append style={black},
every y tick label/.append style={font=\color{black}},
ymin=0,
ymax=0.015,
ylabel={Depth $\delta$},
axis background/.style={fill=white},
title={Z=15},
axis x line*=bottom,
axis y line*=left,
legend style={legend cell align=left,align=left,draw=black}
]
\addplot [color=green,dashed]
  table[row sep=crcr]{%
1	0.0122327606316414\\
2	0.012241522057822\\
3	0.0122505439155707\\
4	0.0122598297553432\\
5	0.0122693826350237\\
6	0.012279205027388\\
7	0.012289298715094\\
8	0.0122996646760778\\
9	0.0123103031544958\\
10	0.0123212138640972\\
11	0.0123323970708161\\
12	0.0123438601925528\\
13	0.0123555950386767\\
14	0.0123675832077272\\
15	0.0123798181383589\\
16	0.0123922919220944\\
17	0.0124049951859417\\
18	0.0124179168490719\\
19	0.0124310356266748\\
20	0.0124443197451163\\
21	0.0124577176870763\\
22	0.0124711589225573\\
23	0.0124846853127427\\
24	0.0124982474848527\\
25	0.012511787186506\\
26	0.0125252316719989\\
27	0.0125388468111412\\
28	0.0125527990068079\\
29	0.012567079642699\\
30	0.0125816785416316\\
31	0.0125965842095548\\
32	0.0126117831474384\\
33	0.0126272595804471\\
34	0.0126429948698342\\
35	0.0126589650740243\\
36	0.0126751355269413\\
37	0.0126905800261\\
38	0.0127064191489913\\
39	0.0127229337549721\\
40	0.0127401769159195\\
41	0.0127581650269329\\
42	0.0127769001634385\\
43	0.0127963650035619\\
44	0.0128165844624822\\
45	0.0128375798199826\\
46	0.0128593680795727\\
47	0.0128819614778757\\
48	0.0129053704793083\\
49	0.0129295498474715\\
50	0.0129539447314585\\
51	0.0129780726543019\\
52	0.0130018492520287\\
53	0.0130276145365274\\
54	0.0130531189295267\\
55	0.013077990109756\\
56	0.0131022486945868\\
57	0.0131260879584517\\
58	0.0131497615244743\\
59	0.0131732433489588\\
60	0.0131965475314924\\
61	0.0132196466585857\\
62	0.0132425450126211\\
63	0.0132652629027781\\
64	0.013287851666599\\
65	0.0133106623280383\\
66	0.0133337769836722\\
67	0.0133591616733824\\
68	0.0133850191726385\\
69	0.0134111173069747\\
70	0.0134369346682186\\
71	0.0134612352303047\\
72	0.0134856037193887\\
73	0.0135100647937358\\
74	0.013534594106904\\
75	0.013559157506947\\
76	0.0135835908959974\\
77	0.0136078314992819\\
78	0.0136300809313359\\
79	0.0136515665992234\\
80	0.0136722646348219\\
81	0.0136920171315763\\
82	0.0137106095932766\\
83	0.0137286654461089\\
84	0.0137462465757114\\
85	0.0137623201421256\\
86	0.0137775050281561\\
87	0.01379271403945\\
88	0.0138076796863335\\
89	0.013821319253111\\
90	0.0138344704942092\\
91	0.0138472674773515\\
92	0.0138598474407153\\
93	0.0138728061381877\\
94	0.013887029133783\\
95	0.0139037923926731\\
96	0.0139270290006153\\
97	0.0139715613125096\\
98	0.0140755501155958\\
99	0\\
100	0\\
};
\addlegendentry{$q=-4$};

\addplot [color=mycolor1,dashed]
  table[row sep=crcr]{%
1	0.0117188170935237\\
2	0.0117306652698127\\
3	0.0117429096743193\\
4	0.0117555606001355\\
5	0.0117686282489777\\
6	0.0117821226810193\\
7	0.0117960537675407\\
8	0.0118104311719426\\
9	0.0118252644577415\\
10	0.0118405637383399\\
11	0.0118563426599663\\
12	0.0118726324617226\\
13	0.0118894277179298\\
14	0.0119067012395361\\
15	0.0119244562546676\\
16	0.0119426944788827\\
17	0.0119614156482702\\
18	0.0119806165037037\\
19	0.0120002883132638\\
20	0.0120204092071249\\
21	0.0120409200608136\\
22	0.0120617367423536\\
23	0.012083053809787\\
24	0.012104861188102\\
25	0.0121271461351857\\
26	0.0121498932691813\\
27	0.0121730761831054\\
28	0.0121966589106519\\
29	0.0122206010777664\\
30	0.0122448284114289\\
31	0.0122692717523863\\
32	0.0122938567059575\\
33	0.0123184969082716\\
34	0.0123430916315677\\
35	0.0123675163962651\\
36	0.0123903374999865\\
37	0.0124115583881214\\
38	0.0124327012648518\\
39	0.0124543856686423\\
40	0.012476729429032\\
41	0.0124998238060895\\
42	0.0125237004378295\\
43	0.0125483175301616\\
44	0.0125736466452735\\
45	0.0125996705302306\\
46	0.0126263637123744\\
47	0.0126536900652924\\
48	0.0126815968999723\\
49	0.0127099775677733\\
50	0.0127386531047608\\
51	0.012767632638374\\
52	0.012796976678315\\
53	0.0128266090984458\\
54	0.0128564260304907\\
55	0.0128868309920126\\
56	0.0129179453392102\\
57	0.0129492283379129\\
58	0.0129800666081532\\
59	0.0130101120923736\\
60	0.0130377302212764\\
61	0.0130648578004952\\
62	0.0130914199231992\\
63	0.0131173464107898\\
64	0.0131425288367032\\
65	0.0131673222267289\\
66	0.0131918299726905\\
67	0.013216034277627\\
68	0.0132398830908501\\
69	0.0132633332051133\\
70	0.0132874667748315\\
71	0.0133138365869203\\
72	0.0133398407301026\\
73	0.0133656919237774\\
74	0.0133916256423548\\
75	0.0134176120152253\\
76	0.0134435975183044\\
77	0.0134696206927688\\
78	0.0134979872064833\\
79	0.013526295047492\\
80	0.0135542916460223\\
81	0.0135819116043021\\
82	0.0136090604613035\\
83	0.0136353875792836\\
84	0.0136608899324634\\
85	0.0136839201253191\\
86	0.0137052795969719\\
87	0.0137263229713489\\
88	0.0137471985247367\\
89	0.0137659838894507\\
90	0.0137843083005153\\
91	0.0138025843373601\\
92	0.0138209614507844\\
93	0.0138398695231636\\
94	0.0138613215606166\\
95	0.0138894431312715\\
96	0.013925543792143\\
97	0.0139715613125096\\
98	0.0140755501155958\\
99	0\\
100	0\\
};
\addlegendentry{$q=-3$};

\addplot [color=red,dashed]
  table[row sep=crcr]{%
1	0.0102709580499725\\
2	0.010281563369072\\
3	0.010292582102453\\
4	0.0103040316655793\\
5	0.0103159303038681\\
6	0.0103282971421568\\
7	0.0103411522457359\\
8	0.010354516708772\\
9	0.0103684128154397\\
10	0.0103828643933117\\
11	0.0103978973564456\\
12	0.0104135351939038\\
13	0.0104298014667672\\
14	0.0104467241755461\\
15	0.0104643326364947\\
16	0.0104826574700699\\
17	0.0105017305045356\\
18	0.0105215844908043\\
19	0.0105422524187287\\
20	0.0105637664610809\\
21	0.0105861622341779\\
22	0.0106094979982446\\
23	0.0106338139771202\\
24	0.0106591515116842\\
25	0.0106855528951657\\
26	0.0107130611511757\\
27	0.010741720000705\\
28	0.0107715742260739\\
29	0.010802673584565\\
30	0.0108350071096665\\
31	0.0108685994341274\\
32	0.0109034901001623\\
33	0.0109397278197828\\
34	0.0109773917256394\\
35	0.0110166459370303\\
36	0.0110552203897181\\
37	0.0110977486314485\\
38	0.0111416762584311\\
39	0.0111874662894916\\
40	0.0112367524444499\\
41	0.0112944140689771\\
42	0.0113542590386678\\
43	0.0114158262939671\\
44	0.0114790656123249\\
45	0.0115439585004534\\
46	0.0116104705923482\\
47	0.0116785437745188\\
48	0.0117480816096814\\
49	0.0118189186003087\\
50	0.0118906944978455\\
51	0.0119633576791547\\
52	0.0120346561095056\\
53	0.0121037888850088\\
54	0.0121698739737614\\
55	0.0122279467797018\\
56	0.0122814698727606\\
57	0.012336506268143\\
58	0.0123929905653289\\
59	0.0124505089611907\\
60	0.01250652929889\\
61	0.0125638994053487\\
62	0.0126225039614988\\
63	0.0126820544130006\\
64	0.0127422062361778\\
65	0.0128018722105154\\
66	0.0128555686963515\\
67	0.0129057704861722\\
68	0.0129545669867357\\
69	0.0130016072701518\\
70	0.0130449403196094\\
71	0.013084902895006\\
72	0.0131233286706024\\
73	0.0131600244581203\\
74	0.0131952549803496\\
75	0.0132296069523315\\
76	0.01326340025322\\
77	0.0132940287863121\\
78	0.0133240202672836\\
79	0.0133536487911245\\
80	0.0133828154380139\\
81	0.0134114322445559\\
82	0.0134394244153484\\
83	0.0134669745121238\\
84	0.0134940944783442\\
85	0.0135229775828559\\
86	0.013553300417067\\
87	0.0135826847719989\\
88	0.0136109729044239\\
89	0.0136400769814614\\
90	0.0136685968057691\\
91	0.013696341564945\\
92	0.0137232969189836\\
93	0.0137498611905214\\
94	0.0137779259554712\\
95	0.0138158749324622\\
96	0.0138757626800631\\
97	0.0139621673973833\\
98	0.0140755501155958\\
99	0\\
100	0\\
};
\addlegendentry{$q=-2$};

\addplot [color=blue,dashed]
  table[row sep=crcr]{%
1	0.00696266678300967\\
2	0.00697407602410268\\
3	0.0069859610560876\\
4	0.00699834474908146\\
5	0.00701125131491679\\
6	0.00702470641419099\\
7	0.00703873728159877\\
8	0.00705337286457887\\
9	0.00706864392778429\\
10	0.00708458295699298\\
11	0.00710122362787537\\
12	0.00711860182755134\\
13	0.00713675645756719\\
14	0.00715572885890603\\
15	0.00717556294846286\\
16	0.00719630533805207\\
17	0.0072180054519068\\
18	0.00724071576011226\\
19	0.00726449251394272\\
20	0.00728939763495815\\
21	0.00731549996724737\\
22	0.00734286844257857\\
23	0.00737157676614258\\
24	0.00740170378354605\\
25	0.00743333384065768\\
26	0.00746655703611329\\
27	0.00750146899616408\\
28	0.00753816929183361\\
29	0.00757675676447259\\
30	0.00761735020637924\\
31	0.00766008499747461\\
32	0.00770512001860023\\
33	0.00775266670123059\\
34	0.00780307646824082\\
35	0.00785710738072752\\
36	0.00791754763436833\\
37	0.0079802615944167\\
38	0.0080458099583471\\
39	0.00811586911773047\\
40	0.00819559079318031\\
41	0.00827853940493495\\
42	0.0083644074040341\\
43	0.00845330614277967\\
44	0.00854536255092997\\
45	0.00864070759235894\\
46	0.00873947339316896\\
47	0.008841788480229\\
48	0.0089477718710809\\
49	0.00905752947720104\\
50	0.00917117414942731\\
51	0.00928858526987464\\
52	0.00940427094424446\\
53	0.00951649651222525\\
54	0.00962298834456926\\
55	0.00970805341192905\\
56	0.00977984868148762\\
57	0.009854736189011\\
58	0.00993279098107374\\
59	0.0100146390509547\\
60	0.0101040286200362\\
61	0.0101953663496222\\
62	0.0102884759572755\\
63	0.0103827928711533\\
64	0.0104778404210339\\
65	0.0105732559399415\\
66	0.0106583464882014\\
67	0.0107408684014955\\
68	0.010822614385659\\
69	0.0109033451817943\\
70	0.0109788420309948\\
71	0.0110499045347061\\
72	0.0111207746787612\\
73	0.0111903355374338\\
74	0.0112582487723513\\
75	0.011326219736867\\
76	0.0113955543662148\\
77	0.0114604731860944\\
78	0.0115265590503869\\
79	0.0115946330502547\\
80	0.011664374843197\\
81	0.0117357782886367\\
82	0.0118088834229557\\
83	0.0118845317440119\\
84	0.0119639441718954\\
85	0.012050478640946\\
86	0.0121645462105752\\
87	0.012289336771521\\
88	0.0124146773866143\\
89	0.0125402773446041\\
90	0.0126655152963389\\
91	0.0127899080838297\\
92	0.0129132089901202\\
93	0.0130354663147911\\
94	0.0131576833482414\\
95	0.0132849489166765\\
96	0.0134453474499231\\
97	0.0137304419781218\\
98	0.0140755501155958\\
99	0\\
100	0\\
};
\addlegendentry{$q=-1$};

\addplot [color=black,solid]
  table[row sep=crcr]{%
1	0.0069535399396323\\
2	0.0069535399396323\\
3	0.0069535399396323\\
4	0.0069535399396323\\
5	0.0069535399396323\\
6	0.0069535399396323\\
7	0.0069535399396323\\
8	0.0069535399396323\\
9	0.0069535399396323\\
10	0.0069535399396323\\
11	0.0069535399396323\\
12	0.0069535399396323\\
13	0.0069535399396323\\
14	0.0069535399396323\\
15	0.0069535399396323\\
16	0.0069535399396323\\
17	0.0069535399396323\\
18	0.0069535399396323\\
19	0.0069535399396323\\
20	0.0069535399396323\\
21	0.0069535399396323\\
22	0.0069535399396323\\
23	0.0069535399396323\\
24	0.0069535399396323\\
25	0.0069535399396323\\
26	0.0069535399396323\\
27	0.0069535399396323\\
28	0.0069535399396323\\
29	0.0069535399396323\\
30	0.0069535399396323\\
31	0.0069535399396323\\
32	0.0069535399396323\\
33	0.0069535399396323\\
34	0.0069535399396323\\
35	0.0069535399396323\\
36	0.0069535399396323\\
37	0.0069535399396323\\
38	0.0069535399396323\\
39	0.0069535399396323\\
40	0.0069535399396323\\
41	0.0069535399396323\\
42	0.0069535399396323\\
43	0.0069535399396323\\
44	0.0069535399396323\\
45	0.0069535399396323\\
46	0.0069535399396323\\
47	0.0069535399396323\\
48	0.0069535399396323\\
49	0.0069535399396323\\
50	0.0069535399396323\\
51	0.0069535399396323\\
52	0.00696027783267133\\
53	0.00697299671308261\\
54	0.00698657675953214\\
55	0.0070011446971794\\
56	0.00701684118719758\\
57	0.00703382593193524\\
58	0.00705228738988878\\
59	0.00707244823104831\\
60	0.00709457244918762\\
61	0.00711897426419684\\
62	0.00714589746726174\\
63	0.00716869992745805\\
64	0.00719429039367552\\
65	0.00722315004847161\\
66	0.00728020871235676\\
67	0.00737330386823627\\
68	0.00747216693218319\\
69	0.00757828644269027\\
70	0.00769633490019724\\
71	0.0078219484968913\\
72	0.00795285098323012\\
73	0.00808949725914809\\
74	0.00823162513640673\\
75	0.00837397560283416\\
76	0.0085228596890992\\
77	0.00868283792378282\\
78	0.00884766761610357\\
79	0.00901754273325763\\
80	0.00918788597327563\\
81	0.00934803350457614\\
82	0.00951064346236894\\
83	0.00967590203938414\\
84	0.00984670278679874\\
85	0.0100302965634104\\
86	0.0101930953735268\\
87	0.0103551969237519\\
88	0.0105160183316585\\
89	0.0106749180557875\\
90	0.0108304902743305\\
91	0.0109806504953299\\
92	0.011123550218616\\
93	0.0112610305818502\\
94	0.011396274611093\\
95	0.011536591915825\\
96	0.0116985505525057\\
97	0.0119313279515455\\
98	0.0119846348539947\\
99	0\\
100	0\\
};
\addlegendentry{$q=0$};

\addplot [color=blue,solid]
  table[row sep=crcr]{%
1	0.0140747890152839\\
2	0.0140747853067208\\
3	0.0140747814898151\\
4	0.0140747775603974\\
5	0.0140747735155351\\
6	0.0140747693515032\\
7	0.0140747650607592\\
8	0.0140747606318368\\
9	0.0140747560483417\\
10	0.0140747513062544\\
11	0.0140747464221367\\
12	0.0140747413874906\\
13	0.014074736188711\\
14	0.0140747307957396\\
15	0.0140747251195047\\
16	0.0140747188363629\\
17	0.0140746892835858\\
18	0.0140746199071754\\
19	0.0140745481717254\\
20	0.0140744739493555\\
21	0.0140743970942218\\
22	0.0140743174259298\\
23	0.0140742346795541\\
24	0.0140741483594454\\
25	0.0140740574021\\
26	0.014073960557788\\
27	0.0140738599566251\\
28	0.0140737553611988\\
29	0.0140736465126963\\
30	0.014073533128553\\
31	0.0140734148996141\\
32	0.014073291486317\\
33	0.0140731625124396\\
34	0.0140730275517528\\
35	0.0140728860909299\\
36	0.0140727374003998\\
37	0.0140725799860309\\
38	0.0140720464440718\\
39	0.0140705294753545\\
40	0.0140689606425357\\
41	0.0140673364972055\\
42	0.014065653183951\\
43	0.0140639066111707\\
44	0.0140620920611754\\
45	0.014060204184613\\
46	0.0140582363747563\\
47	0.0140561809923816\\
48	0.014054029155822\\
49	0.0140517704777872\\
50	0.0140493927258234\\
51	0.0140468813584505\\
52	0.014044218817105\\
53	0.0140413832111486\\
54	0.0140383468965883\\
55	0.0140318474827155\\
56	0.0140242787915779\\
57	0.0140164057083802\\
58	0.0140082085829942\\
59	0.0139996600516814\\
60	0.0139907282707493\\
61	0.0139813769434149\\
62	0.01397156464403\\
63	0.0139612439060719\\
64	0.0139503604044524\\
65	0.0139388516775686\\
66	0.0139264519209255\\
67	0.01391300485925\\
68	0.0138983067396105\\
69	0.0138717190905066\\
70	0.0138402116152021\\
71	0.0138032183439128\\
72	0.0137642523093567\\
73	0.0137224605122615\\
74	0.0136767886098927\\
75	0.0136209545126161\\
76	0.0135620737592675\\
77	0.0134996559309287\\
78	0.013432882253133\\
79	0.0133479929115408\\
80	0.0132318936792259\\
81	0.013091439493059\\
82	0.0129442082898861\\
83	0.012789546522986\\
84	0.012626515584212\\
85	0.0124493165633761\\
86	0.0122243530586431\\
87	0.0119881118982065\\
88	0.011739777735739\\
89	0.0114778950676377\\
90	0.0112018141990376\\
91	0.010909072124832\\
92	0.0106022834277032\\
93	0.0102795724167244\\
94	0.00993137033858306\\
95	0.00955101288282799\\
96	0.00912465292637579\\
97	0.00824476209521964\\
98	0.00407555011559578\\
99	0\\
100	0\\
};
\addlegendentry{$q=1$};

\addplot [color=red,solid]
  table[row sep=crcr]{%
1	0.0140131419108131\\
2	0.0140124158162123\\
3	0.0140116665278782\\
4	0.014010892898994\\
5	0.0140100937203853\\
6	0.0140092680910095\\
7	0.014008415232017\\
8	0.0140075339143292\\
9	0.0140066228448268\\
10	0.0140056807209616\\
11	0.0140047062337551\\
12	0.0140036978124129\\
13	0.0140026535428182\\
14	0.0140015712954017\\
15	0.0140004485977575\\
16	0.0139992821208324\\
17	0.0139980372303798\\
18	0.0139966885177159\\
19	0.0139952851922683\\
20	0.013993822329178\\
21	0.0139922941220683\\
22	0.0139906936029604\\
23	0.013989012123546\\
24	0.0139872382053057\\
25	0.0139853546189215\\
26	0.0139830892787469\\
27	0.0139789092124666\\
28	0.0139745902387157\\
29	0.0139701257499622\\
30	0.0139655086758974\\
31	0.0139607314474885\\
32	0.0139557859601876\\
33	0.0139506635360103\\
34	0.0139453548806164\\
35	0.0139398500152443\\
36	0.0139341380835649\\
37	0.0139282064841823\\
38	0.0139215291127289\\
39	0.0139132494872816\\
40	0.0139046762880946\\
41	0.0138957887819354\\
42	0.0138865556893053\\
43	0.0138769251814139\\
44	0.0138669105576204\\
45	0.0138565183828406\\
46	0.013845722919349\\
47	0.0138344910794825\\
48	0.0138227850556276\\
49	0.0138105613879769\\
50	0.0137977697513191\\
51	0.0137843512349195\\
52	0.013770235441885\\
53	0.0137553349298749\\
54	0.0137395342039856\\
55	0.0137181695943548\\
56	0.0136945031875181\\
57	0.0136633017396435\\
58	0.0136254936003157\\
59	0.0135861628752301\\
60	0.0135451901743813\\
61	0.0135024438323053\\
62	0.0134577763201429\\
63	0.0134110078656356\\
64	0.0133619239504832\\
65	0.0133102894262138\\
66	0.0132562187548939\\
67	0.0131989041215644\\
68	0.0131375648918451\\
69	0.0130570273506728\\
70	0.0129672719789755\\
71	0.0128799991873934\\
72	0.0127883625333282\\
73	0.0126904654314209\\
74	0.0125734096485984\\
75	0.0124400919478998\\
76	0.012300672920036\\
77	0.0121542941251746\\
78	0.0119995487933104\\
79	0.0118172874393079\\
80	0.0116026918830551\\
81	0.0114120955547575\\
82	0.0112141523588551\\
83	0.0110083908932307\\
84	0.0107938588018927\\
85	0.010577379001518\\
86	0.0104177862729992\\
87	0.0102514642963198\\
88	0.0100797791511674\\
89	0.00990011712857725\\
90	0.00971281494860763\\
91	0.0095213768432415\\
92	0.00933091037119082\\
93	0.00913507579011389\\
94	0.0089299129425915\\
95	0.0087147000474069\\
96	0.00827718649022398\\
97	0.00632133460307733\\
98	0.00407555011559578\\
99	0\\
100	0\\
};
\addlegendentry{$q=2$};

\addplot [color=mycolor1,solid]
  table[row sep=crcr]{%
1	0.0137701680573566\\
2	0.0137665932044743\\
3	0.0137629117712479\\
4	0.0137591193277895\\
5	0.0137552107147427\\
6	0.0137511805544953\\
7	0.0137470259076089\\
8	0.0137427449137119\\
9	0.0137383327324756\\
10	0.013733784335973\\
11	0.0137290944426782\\
12	0.0137242571616614\\
13	0.0137192657136528\\
14	0.0137141128126003\\
15	0.0137087906469272\\
16	0.0137032909542647\\
17	0.0136976597292383\\
18	0.0136919287327977\\
19	0.0136859965191235\\
20	0.0136798515107889\\
21	0.0136734790267163\\
22	0.013666862227869\\
23	0.0136599813561021\\
24	0.0136528121777675\\
25	0.0136453219859478\\
26	0.0136371160966439\\
27	0.0136260117248677\\
28	0.0136145015052309\\
29	0.0136026114874057\\
30	0.0135903225938376\\
31	0.0135776141542072\\
32	0.0135644637256027\\
33	0.013550846896024\\
34	0.0135367370818913\\
35	0.0135221053614088\\
36	0.0135069205081391\\
37	0.0134911499430103\\
38	0.013475650444195\\
39	0.0134618684065498\\
40	0.0134475005382067\\
41	0.0134324903049592\\
42	0.0134167511270578\\
43	0.0134001188163241\\
44	0.0133771851941023\\
45	0.0133501134992671\\
46	0.0133221402773208\\
47	0.0132932135562792\\
48	0.0132632763447413\\
49	0.0132322658634036\\
50	0.0132001125181892\\
51	0.0131667384085978\\
52	0.0131320546488386\\
53	0.0130959548139344\\
54	0.0130583035157773\\
55	0.013026827467008\\
56	0.0129954251156187\\
57	0.0129533411439786\\
58	0.0129020323456521\\
59	0.0128487426063729\\
60	0.012793334759399\\
61	0.0127356578542819\\
62	0.0126755685252279\\
63	0.0126129190001478\\
64	0.0125474360908037\\
65	0.012478723329547\\
66	0.012406385462252\\
67	0.0123310224627413\\
68	0.0122524759856178\\
69	0.0121955659437535\\
70	0.0121223723366172\\
71	0.0120268963747487\\
72	0.0119268429872047\\
73	0.0118227406772296\\
74	0.0116945929420149\\
75	0.0115277077437531\\
76	0.0113552798374595\\
77	0.0111772918531823\\
78	0.0109944964154016\\
79	0.0108397659615434\\
80	0.0107346360348897\\
81	0.0106255650825685\\
82	0.0105131150129121\\
83	0.0103982252022044\\
84	0.0102806923492592\\
85	0.0101609525846018\\
86	0.0100348669550475\\
87	0.00990542750144545\\
88	0.00977254152256229\\
89	0.00963271907182141\\
90	0.00948542100562233\\
91	0.00933003221642036\\
92	0.00916544862748719\\
93	0.00899179295123485\\
94	0.00880899567312919\\
95	0.00847358171863326\\
96	0.00734600028981367\\
97	0.00586739649120498\\
98	0.00407555011559578\\
99	0\\
100	0\\
};
\addlegendentry{$q=3$};

\addplot [color=green,solid]
  table[row sep=crcr]{%
1	0.0134720254946476\\
2	0.0134667840002946\\
3	0.0134613917977598\\
4	0.0134558438388858\\
5	0.0134501331454569\\
6	0.0134442477718128\\
7	0.013438170999875\\
8	0.0134319055216595\\
9	0.013425467866156\\
10	0.0134188519181047\\
11	0.0134120510835525\\
12	0.0134050582473476\\
13	0.0133978658185527\\
14	0.0133904657010935\\
15	0.0133828493470008\\
16	0.0133750083507678\\
17	0.0133669322243053\\
18	0.0133586079072552\\
19	0.0133500267485475\\
20	0.0133411816188853\\
21	0.0133320616986511\\
22	0.0133226559561399\\
23	0.0133129534056457\\
24	0.0133029438011369\\
25	0.0132926201431537\\
26	0.0132825806884392\\
27	0.0132764831833719\\
28	0.0132689373947882\\
29	0.0132557368854051\\
30	0.0132421565350406\\
31	0.0132281782625461\\
32	0.0132137820880791\\
33	0.0131989457842485\\
34	0.0131836444452043\\
35	0.0131678499618763\\
36	0.013151530426545\\
37	0.0131346497480047\\
38	0.0131171522919514\\
39	0.01309894889047\\
40	0.0130799762917019\\
41	0.0130601621672877\\
42	0.0130394240624307\\
43	0.0130176430428208\\
44	0.0129871267568911\\
45	0.0129509067503821\\
46	0.0129135846315742\\
47	0.0128751159829357\\
48	0.0128354546454393\\
49	0.0127945527982676\\
50	0.0127523609580503\\
51	0.0127088279306244\\
52	0.0126639016558636\\
53	0.0126175324887667\\
54	0.0125696831148186\\
55	0.0125202074910236\\
56	0.0124691564891979\\
57	0.0124317672558142\\
58	0.0124056690887418\\
59	0.0123782087026684\\
60	0.0123492922121896\\
61	0.012318684009363\\
62	0.0122860583715728\\
63	0.012251256164234\\
64	0.0122143817935398\\
65	0.0121747691199453\\
66	0.0121262291201672\\
67	0.0120388755397677\\
68	0.0119483220600089\\
69	0.0118531751669201\\
70	0.0117235133944667\\
71	0.0115630221741897\\
72	0.0113978677821986\\
73	0.0112286425469846\\
74	0.0110897806425187\\
75	0.0110056500983793\\
76	0.0109197013311663\\
77	0.0108324349380534\\
78	0.0107446101229008\\
79	0.0106562374605092\\
80	0.0105640666156659\\
81	0.0104679168593401\\
82	0.010367590898843\\
83	0.0102629765679872\\
84	0.0101544321151487\\
85	0.0100434433798738\\
86	0.0099270810727797\\
87	0.00980475556964079\\
88	0.00967581506317478\\
89	0.00953962853680043\\
90	0.00939493691026074\\
91	0.00924339423504485\\
92	0.00908201703891517\\
93	0.00891275907712221\\
94	0.00861876478833807\\
95	0.0078199693350052\\
96	0.00674579869692106\\
97	0.00586739649120498\\
98	0.00407555011559578\\
99	0\\
100	0\\
};
\addlegendentry{$q=4$};

\end{axis}
\end{tikzpicture}% 
%  \caption{Discrete Time}
%\end{subfigure}\\
%\vspace{1cm}
%\begin{subfigure}{.45\linewidth}
%  \centering
%  \setlength\figureheight{\linewidth} 
%  \setlength\figurewidth{\linewidth}
%  \tikzsetnextfilename{dm_cts_nFPC_z15}
%  % This file was created by matlab2tikz.
%
%The latest updates can be retrieved from
%  http://www.mathworks.com/matlabcentral/fileexchange/22022-matlab2tikz-matlab2tikz
%where you can also make suggestions and rate matlab2tikz.
%
\definecolor{mycolor1}{rgb}{1.00000,0.00000,1.00000}%
%
\begin{tikzpicture}[trim axis left, trim axis right]

\begin{axis}[%
width=\figurewidth,
height=\figureheight,
at={(0\figurewidth,0\figureheight)},
scale only axis,
every outer x axis line/.append style={black},
every x tick label/.append style={font=\color{black}},
xmin=0,
xmax=100,
%xlabel={Time},
every outer y axis line/.append style={black},
every y tick label/.append style={font=\color{black}},
ymin=0,
ymax=0.015,
%ylabel={Depth $\delta^-$},
axis background/.style={fill=white},
axis x line*=bottom,
axis y line*=left,
yticklabel style={
        /pgf/number format/fixed,
        /pgf/number format/precision=3
},
scaled y ticks=false,
legend style={legend cell align=left,align=left,draw=black,font=\footnotesize, at={(0.98,0.02)},anchor=south east},
every axis legend/.code={\renewcommand\addlegendentry[2][]{}}  %ignore legend locally
]
\addplot [color=green,dashed]
  table[row sep=crcr]{%
0.01	0\\
1.01	0\\
2.01	0\\
3.01	0\\
4.01	0\\
5.01	0\\
6.01	0\\
7.01	0\\
8.01	0\\
9.01	0\\
10.01	0\\
11.01	0\\
12.01	0\\
13.01	0\\
14.01	0\\
15.01	0\\
16.01	0\\
17.01	0\\
18.01	0\\
19.01	0\\
20.01	0\\
21.01	0\\
22.01	0\\
23.01	0\\
24.01	0\\
25.01	0\\
26.01	0\\
27.01	0\\
28.01	0\\
29.01	0\\
30.01	0\\
31.01	0\\
32.01	0\\
33.01	0\\
34.01	0\\
35.01	0\\
36.01	0\\
37.01	0\\
38.01	0\\
39.01	0\\
40.01	0\\
41.01	0\\
42.01	0\\
43.01	0\\
44.01	0\\
45.01	0\\
46.01	0\\
47.01	0\\
48.01	0\\
49.01	0\\
50.01	0\\
51.01	0\\
52.01	0\\
53.01	0\\
54.01	0\\
55.01	0\\
56.01	0\\
57.01	0\\
58.01	0\\
59.01	0\\
60.01	0\\
61.01	0\\
62.01	0\\
63.01	0\\
64.01	0\\
65.01	0\\
66.01	0\\
67.01	0\\
68.01	0\\
69.01	0\\
70.01	0\\
71.01	0\\
72.01	0\\
73.01	0\\
74.01	0\\
75.01	0\\
76.01	0\\
77.01	0\\
78.01	0\\
79.01	0\\
80.01	0\\
81.01	0\\
82.01	0\\
83.01	0\\
84.01	0\\
85.01	0\\
86.01	0\\
87.01	0\\
88.01	0\\
89.01	0\\
90.01	0\\
91.01	0\\
92.01	0\\
93.01	0\\
94.01	0\\
95.01	0\\
96.01	0\\
97.01	0\\
98.01	0.000386588687883091\\
99.01	0.00337828656858661\\
99.02	0.00341954072443339\\
99.03	0.00346117385023638\\
99.04	0.00350318948458161\\
99.05	0.00354559119907409\\
99.06	0.00358838259864556\\
99.07	0.00363156732186504\\
99.08	0.00367514904125229\\
99.09	0.00371913146359422\\
99.1	0.0037635183302642\\
99.11	0.00380831341754438\\
99.12	0.00385352053695097\\
99.13	0.00389914353556257\\
99.14	0.00394518629635159\\
99.15	0.00399165273851862\\
99.16	0.00403854681783014\\
99.17	0.00408587252695911\\
99.18	0.00413363389582897\\
99.19	0.00418183495431657\\
99.2	0.00423047976694747\\
99.21	0.00427957243576882\\
99.22	0.00432911710069517\\
99.23	0.00437911793985739\\
99.24	0.00442957916995488\\
99.25	0.00448050504661093\\
99.26	0.00453189986473142\\
99.27	0.0045837679588668\\
99.28	0.00463611370357735\\
99.29	0.00468894151380188\\
99.3	0.00474225584522977\\
99.31	0.00479606119467643\\
99.32	0.00485036210046225\\
99.33	0.00490516314279497\\
99.34	0.00496046894415566\\
99.35	0.00501628416968814\\
99.36	0.00507261352759206\\
99.37	0.0051294617695196\\
99.38	0.00518683369097572\\
99.39	0.00524473413172222\\
99.4	0.00530316797618543\\
99.41	0.00536214015386766\\
99.42	0.00542165563976247\\
99.43	0.00548171945477372\\
99.44	0.00554233666613847\\
99.45	0.00560351238785381\\
99.46	0.00566525178110757\\
99.47	0.00572756005471303\\
99.48	0.00579044246554756\\
99.49	0.00585390431899538\\
99.5	0.00591795096939432\\
99.51	0.00598258782048674\\
99.52	0.00604782032587452\\
99.53	0.00611365398947834\\
99.54	0.00618009436600108\\
99.55	0.00624714706139553\\
99.56	0.00631481773333638\\
99.57	0.00638311209169663\\
99.58	0.00645203589902817\\
99.59	0.00652159497104708\\
99.6	0.00659179517712308\\
99.61	0.00666264244077372\\
99.62	0.00673414273681778\\
99.63	0.00680630208973471\\
99.64	0.00687912657950685\\
99.65	0.00695262234212994\\
99.66	0.0070267955701283\\
99.67	0.00710165251307478\\
99.68	0.00717719947811546\\
99.69	0.00725344283049917\\
99.7	0.00733038899411186\\
99.71	0.0074080444520159\\
99.72	0.00748641574699432\\
99.73	0.00756550948210005\\
99.74	0.00764533232121015\\
99.75	0.00772589098958526\\
99.76	0.00780719227443403\\
99.77	0.00788924302548281\\
99.78	0.00797205015555061\\
99.79	0.00805562064112923\\
99.8	0.00813996152296881\\
99.81	0.00822507990666871\\
99.82	0.00831098296327383\\
99.83	0.00839767792987635\\
99.84	0.00848517211022314\\
99.85	0.00857347287532854\\
99.86	0.00866258766409292\\
99.87	0.00875252398392687\\
99.88	0.00884328941138109\\
99.89	0.00893489159278211\\
99.9	0.00902733824487381\\
99.91	0.00912063715546484\\
99.92	0.00921479618408196\\
99.93	0.00930982326262945\\
99.94	0.00940572639605445\\
99.95	0.00950251366301854\\
99.96	0.00960019321657535\\
99.97	0.00969877328485451\\
99.98	0.00979826217175179\\
99.99	0.00989866825762563\\
100	0.01\\
};
\addlegendentry{$q=-4$};

\addplot [color=mycolor1,dashed]
  table[row sep=crcr]{%
0.01	0\\
1.01	0\\
2.01	0\\
3.01	0\\
4.01	0\\
5.01	0\\
6.01	0\\
7.01	0\\
8.01	0\\
9.01	0\\
10.01	0\\
11.01	0\\
12.01	0\\
13.01	0\\
14.01	0\\
15.01	0\\
16.01	0\\
17.01	0\\
18.01	0\\
19.01	0\\
20.01	0\\
21.01	0\\
22.01	0\\
23.01	0\\
24.01	0\\
25.01	0\\
26.01	0\\
27.01	0\\
28.01	0\\
29.01	0\\
30.01	0\\
31.01	0\\
32.01	0\\
33.01	0\\
34.01	0\\
35.01	0\\
36.01	0\\
37.01	0\\
38.01	0\\
39.01	0\\
40.01	0\\
41.01	0\\
42.01	0\\
43.01	0\\
44.01	0\\
45.01	0\\
46.01	0\\
47.01	0\\
48.01	0\\
49.01	0\\
50.01	0\\
51.01	0\\
52.01	0\\
53.01	0\\
54.01	0\\
55.01	0\\
56.01	0\\
57.01	0\\
58.01	0\\
59.01	0\\
60.01	0\\
61.01	0\\
62.01	0\\
63.01	0\\
64.01	0\\
65.01	0\\
66.01	0\\
67.01	0\\
68.01	0\\
69.01	0\\
70.01	0\\
71.01	0\\
72.01	0\\
73.01	0\\
74.01	0\\
75.01	0\\
76.01	0\\
77.01	0\\
78.01	0\\
79.01	0\\
80.01	0\\
81.01	0\\
82.01	0\\
83.01	0\\
84.01	0\\
85.01	0\\
86.01	0\\
87.01	0\\
88.01	0\\
89.01	0\\
90.01	0\\
91.01	0\\
92.01	0\\
93.01	0\\
94.01	0\\
95.01	0\\
96.01	0\\
97.01	0\\
98.01	0\\
99.01	0.00337828226131128\\
99.02	0.00341953654113725\\
99.03	0.00346116978864239\\
99.04	0.00350318554243863\\
99.05	0.00354558737415644\\
99.06	0.00358837888875248\\
99.07	0.00363156372482022\\
99.08	0.00367514555490337\\
99.09	0.00371912808581235\\
99.1	0.00376351505894353\\
99.11	0.00380831025060157\\
99.12	0.00385351747232472\\
99.13	0.00389914057121316\\
99.14	0.00394518343026037\\
99.15	0.00399164996868758\\
99.16	0.00403854414228137\\
99.17	0.00408586994373441\\
99.18	0.0041336314029893\\
99.19	0.00418183254997663\\
99.2	0.0042304774492322\\
99.21	0.00427957020281343\\
99.22	0.00432911495064515\\
99.23	0.00437911587086857\\
99.24	0.00442957718019346\\
99.25	0.0044805031342536\\
99.26	0.00453189802796542\\
99.27	0.00458376619589004\\
99.28	0.00463611201259857\\
99.29	0.00468893989304077\\
99.3	0.00474225429291715\\
99.31	0.00479605970905447\\
99.32	0.00485036067978466\\
99.33	0.00490516178532726\\
99.34	0.00496046764817539\\
99.35	0.00501628293348524\\
99.36	0.00507261234946913\\
99.37	0.00512946064779225\\
99.38	0.00518683262397297\\
99.39	0.0052447331177869\\
99.4	0.00530316701367457\\
99.41	0.005362139241153\\
99.42	0.00542165477523084\\
99.43	0.00548171863682759\\
99.44	0.00554233589319639\\
99.45	0.00560351165835098\\
99.46	0.00566525109349632\\
99.47	0.00572755940746341\\
99.48	0.00579044185714788\\
99.49	0.00585390374795281\\
99.5	0.0059179504342355\\
99.51	0.00598258731975837\\
99.52	0.00604781985814402\\
99.53	0.00611365355333448\\
99.54	0.00618009396005464\\
99.55	0.00624714668428\\
99.56	0.00631481738370866\\
99.57	0.00638311176823765\\
99.58	0.00645203560044371\\
99.59	0.0065215946960684\\
99.6	0.00659179492450768\\
99.61	0.00666264220930608\\
99.62	0.00673414252531063\\
99.63	0.00680630189702865\\
99.64	0.00687912640447119\\
99.65	0.00695262218366352\\
99.66	0.00702679542716034\\
99.67	0.00710165238456571\\
99.68	0.00717719936305772\\
99.69	0.00725344272791803\\
99.7	0.00733038890306619\\
99.71	0.00740804437159896\\
99.72	0.00748641567633449\\
99.73	0.00756550942036152\\
99.74	0.00764533226759364\\
99.75	0.00772589094332859\\
99.76	0.0078071922348127\\
99.77	0.00788924299181055\\
99.78	0.00797205012717976\\
99.79	0.00805562061745114\\
99.8	0.0081399615034141\\
99.81	0.00822507989070742\\
99.82	0.00831098295041551\\
99.83	0.00839767791967001\\
99.84	0.00848517210225699\\
99.85	0.0085734728692297\\
99.86	0.00866258765952688\\
99.87	0.00875252398059677\\
99.88	0.00884328940902682\\
99.89	0.00893489159117915\\
99.9	0.00902733824383187\\
99.91	0.00912063715482618\\
99.92	0.00921479618371945\\
99.93	0.00930982326244423\\
99.94	0.00940572639597333\\
99.95	0.0095025136629909\\
99.96	0.0096001932165697\\
99.97	0.00969877328485451\\
99.98	0.0097982621717518\\
99.99	0.00989866825762563\\
100	0.01\\
};
\addlegendentry{$q=-3$};

\addplot [color=red,dashed]
  table[row sep=crcr]{%
0.01	0\\
1.01	0\\
2.01	0\\
3.01	0\\
4.01	0\\
5.01	0\\
6.01	0\\
7.01	0\\
8.01	0\\
9.01	0\\
10.01	0\\
11.01	0\\
12.01	0\\
13.01	0\\
14.01	0\\
15.01	0\\
16.01	0\\
17.01	0\\
18.01	0\\
19.01	0\\
20.01	0\\
21.01	0\\
22.01	0\\
23.01	0\\
24.01	0\\
25.01	0\\
26.01	0\\
27.01	0\\
28.01	0\\
29.01	0\\
30.01	0\\
31.01	0\\
32.01	0\\
33.01	0\\
34.01	0\\
35.01	0\\
36.01	0\\
37.01	0\\
38.01	0\\
39.01	0\\
40.01	0\\
41.01	0\\
42.01	0\\
43.01	0\\
44.01	0\\
45.01	0\\
46.01	0\\
47.01	0\\
48.01	0\\
49.01	0\\
50.01	0\\
51.01	0\\
52.01	0\\
53.01	0\\
54.01	0\\
55.01	0\\
56.01	0\\
57.01	0\\
58.01	0\\
59.01	0\\
60.01	0\\
61.01	0\\
62.01	0\\
63.01	0\\
64.01	0\\
65.01	0\\
66.01	0\\
67.01	0\\
68.01	0\\
69.01	0\\
70.01	0\\
71.01	0\\
72.01	0\\
73.01	0\\
74.01	0\\
75.01	0\\
76.01	0\\
77.01	0\\
78.01	0\\
79.01	0\\
80.01	0\\
81.01	0\\
82.01	0\\
83.01	0\\
84.01	0\\
85.01	0\\
86.01	0\\
87.01	0\\
88.01	0\\
89.01	0\\
90.01	0\\
91.01	0\\
92.01	0\\
93.01	0\\
94.01	0\\
95.01	0\\
96.01	0\\
97.01	0\\
98.01	0\\
99.01	0.00337794512537842\\
99.02	0.00341920636951997\\
99.03	0.00346084650937926\\
99.04	0.00350286908385008\\
99.05	0.00354527766483987\\
99.06	0.00358807585757756\\
99.07	0.00363126730092422\\
99.08	0.00367485566768671\\
99.09	0.00371884466493414\\
99.1	0.00376323803431739\\
99.11	0.00380803955239153\\
99.12	0.00385325303094139\\
99.13	0.00389888231730995\\
99.14	0.00394493129473008\\
99.15	0.00399140388265912\\
99.16	0.00403830403711683\\
99.17	0.00408563575102626\\
99.18	0.00413340305455804\\
99.19	0.00418160998596475\\
99.2	0.00423026060940814\\
99.21	0.00427935902655147\\
99.22	0.00432890937690489\\
99.23	0.00437891583817394\\
99.24	0.00442938262661121\\
99.25	0.00448031399737134\\
99.26	0.00453171424486916\\
99.27	0.00458358770314124\\
99.28	0.00463593874621069\\
99.29	0.00468877178845529\\
99.3	0.00474209128497913\\
99.31	0.00479590173198754\\
99.32	0.0048502076671655\\
99.33	0.0049050136700596\\
99.34	0.00496032436246343\\
99.35	0.00501614440880655\\
99.36	0.00507247851654706\\
99.37	0.00512933143656773\\
99.38	0.00518670796357587\\
99.39	0.00524461293650739\\
99.4	0.00530305123896248\\
99.41	0.00536202779961808\\
99.42	0.00542154759264429\\
99.43	0.00548161563812457\\
99.44	0.00554223700248002\\
99.45	0.00560341679889745\\
99.46	0.00566516018776163\\
99.47	0.00572747237709148\\
99.48	0.0057903586229804\\
99.49	0.00585382423004071\\
99.5	0.00591787455185232\\
99.51	0.00598251499141552\\
99.52	0.00604775100160814\\
99.53	0.00611358808564699\\
99.54	0.00618003179755359\\
99.55	0.00624708774262441\\
99.56	0.00631476157790545\\
99.57	0.00638305901267143\\
99.58	0.00645198580890939\\
99.59	0.00652154778180702\\
99.6	0.00659175080024555\\
99.61	0.00666260078729736\\
99.62	0.00673410371788326\\
99.63	0.00680626561604403\\
99.64	0.006879092561375\\
99.65	0.00695259068954072\\
99.66	0.00702676619279462\\
99.67	0.00710162532050356\\
99.68	0.00717717437967758\\
99.69	0.00725341973550465\\
99.7	0.00733036781189061\\
99.71	0.00740802509200435\\
99.72	0.00748639811882837\\
99.73	0.00756549349571853\\
99.74	0.0076453178869692\\
99.75	0.00772587801838046\\
99.76	0.00780718067783092\\
99.77	0.00788923271585624\\
99.78	0.00797204104623336\\
99.79	0.00805561264657055\\
99.8	0.00813995455890341\\
99.81	0.00822507389029681\\
99.82	0.00831097781345295\\
99.83	0.00839767356732553\\
99.84	0.0084851684577402\\
99.85	0.00857346985802142\\
99.86	0.00866258520962567\\
99.87	0.00875252202278125\\
99.88	0.00884328787713478\\
99.89	0.00893489042240442\\
99.9	0.00902733737903997\\
99.91	0.00912063653888998\\
99.92	0.009214795765876\\
99.93	0.00930982299667403\\
99.94	0.00940572624140342\\
99.95	0.00950251358432325\\
99.96	0.00960019318453634\\
99.97	0.00969877327670119\\
99.98	0.00979826217175179\\
99.99	0.00989866825762563\\
100	0.01\\
};
\addlegendentry{$q=-2$};

\addplot [color=blue,dashed]
  table[row sep=crcr]{%
0.01	0\\
1.01	0\\
2.01	0\\
3.01	0\\
4.01	0\\
5.01	0\\
6.01	0\\
7.01	0\\
8.01	0\\
9.01	0\\
10.01	0\\
11.01	0\\
12.01	0\\
13.01	0\\
14.01	0\\
15.01	0\\
16.01	0\\
17.01	0\\
18.01	0\\
19.01	0\\
20.01	0\\
21.01	0\\
22.01	0\\
23.01	0\\
24.01	0\\
25.01	0\\
26.01	0\\
27.01	0\\
28.01	0\\
29.01	0\\
30.01	0\\
31.01	0\\
32.01	0\\
33.01	0\\
34.01	0\\
35.01	0\\
36.01	0\\
37.01	0\\
38.01	0\\
39.01	0\\
40.01	0\\
41.01	0\\
42.01	0\\
43.01	0\\
44.01	0\\
45.01	0\\
46.01	0\\
47.01	0\\
48.01	0\\
49.01	0\\
50.01	0\\
51.01	0\\
52.01	0\\
53.01	0\\
54.01	0\\
55.01	0\\
56.01	0\\
57.01	0\\
58.01	0\\
59.01	0\\
60.01	0\\
61.01	0\\
62.01	0\\
63.01	0\\
64.01	0\\
65.01	0\\
66.01	0\\
67.01	0\\
68.01	0\\
69.01	0\\
70.01	0\\
71.01	0\\
72.01	0\\
73.01	0\\
74.01	0\\
75.01	0\\
76.01	0\\
77.01	0\\
78.01	0\\
79.01	0\\
80.01	0\\
81.01	0\\
82.01	0\\
83.01	0\\
84.01	0\\
85.01	0\\
86.01	0\\
87.01	0\\
88.01	0\\
89.01	0\\
90.01	0\\
91.01	0\\
92.01	0\\
93.01	0\\
94.01	0\\
95.01	0\\
96.01	0\\
97.01	0\\
98.01	0\\
99.01	0.00335442711367033\\
99.02	0.00339602125744558\\
99.03	0.00343799324950865\\
99.04	0.0034803466119229\\
99.05	0.00352308489923335\\
99.06	0.0035662116987579\\
99.07	0.00360973063088092\\
99.08	0.00365364534934916\\
99.09	0.00369795954156998\\
99.1	0.00374267692891206\\
99.11	0.00378780126700835\\
99.12	0.00383333634606152\\
99.13	0.00387928599115167\\
99.14	0.00392565406254658\\
99.15	0.0039724444560142\\
99.16	0.00401966110313765\\
99.17	0.0040673079716326\\
99.18	0.00411538906566701\\
99.19	0.00416390842624637\\
99.2	0.0042128701315636\\
99.21	0.00426227829732716\\
99.22	0.00431213707709156\\
99.23	0.00436245066259037\\
99.24	0.00441322328407146\\
99.25	0.00446445921063472\\
99.26	0.00451616275057213\\
99.27	0.00456833825171019\\
99.28	0.00462099010175469\\
99.29	0.00467412272863787\\
99.3	0.00472774060086789\\
99.31	0.00478184822788059\\
99.32	0.00483645016039364\\
99.33	0.00489155099076289\\
99.34	0.00494715535334105\\
99.35	0.00500326792483865\\
99.36	0.00505989342468713\\
99.37	0.00511703661540424\\
99.38	0.00517470230296157\\
99.39	0.00523289533568309\\
99.4	0.00529162054313569\\
99.41	0.00535088279818377\\
99.42	0.00541068701734764\\
99.43	0.0054710381611638\\
99.44	0.00553194123454689\\
99.45	0.00559340128715346\\
99.46	0.00565542341374731\\
99.47	0.00571801275456649\\
99.48	0.00578117449569182\\
99.49	0.00584491386941689\\
99.5	0.00590923615461952\\
99.51	0.00597414667713447\\
99.52	0.00603965081012756\\
99.53	0.00610575397447076\\
99.54	0.00617246163911862\\
99.55	0.00623977932148552\\
99.56	0.00630771258782391\\
99.57	0.00637626705360334\\
99.58	0.00644544838389017\\
99.59	0.00651526229372785\\
99.6	0.00658571454851766\\
99.61	0.00665681096439979\\
99.62	0.00672855740863903\\
99.63	0.00680095980001549\\
99.64	0.00687402410920575\\
99.65	0.00694775635916377\\
99.66	0.00702216262550119\\
99.67	0.00709724903686717\\
99.68	0.00717302177532716\\
99.69	0.0072494870767408\\
99.7	0.00732665123113838\\
99.71	0.00740452058309589\\
99.72	0.00748310153210824\\
99.73	0.00756240052588584\\
99.74	0.00764242406140549\\
99.75	0.00772317869143584\\
99.76	0.00780467102489812\\
99.77	0.00788690772722355\\
99.78	0.00796989552070725\\
99.79	0.00805364118485809\\
99.8	0.00813815155674424\\
99.81	0.00822343353133399\\
99.82	0.00830949406183134\\
99.83	0.00839634016000594\\
99.84	0.00848397889651701\\
99.85	0.00857241740123047\\
99.86	0.00866166286352905\\
99.87	0.00875172253261461\\
99.88	0.00884260371780208\\
99.89	0.00893431378880451\\
99.9	0.0090268601760084\\
99.91	0.00912025037073873\\
99.92	0.00921449192551288\\
99.93	0.00930959245428256\\
99.94	0.00940555963266315\\
99.95	0.00950240119814921\\
99.96	0.00960012495031559\\
99.97	0.00969873875100284\\
99.98	0.00979825052448599\\
99.99	0.00989866825762563\\
100	0.01\\
};
\addlegendentry{$q=-1$};

\addplot [color=black,solid]
  table[row sep=crcr]{%
0.01	0\\
1.01	0\\
2.01	0\\
3.01	0\\
4.01	0\\
5.01	0\\
6.01	0\\
7.01	0\\
8.01	0\\
9.01	0\\
10.01	0\\
11.01	0\\
12.01	0\\
13.01	0\\
14.01	0\\
15.01	0\\
16.01	0\\
17.01	0\\
18.01	0\\
19.01	0\\
20.01	0\\
21.01	0\\
22.01	0\\
23.01	0\\
24.01	0\\
25.01	0\\
26.01	0\\
27.01	0\\
28.01	0\\
29.01	0\\
30.01	0\\
31.01	0\\
32.01	0\\
33.01	0\\
34.01	0\\
35.01	0\\
36.01	0\\
37.01	0\\
38.01	0\\
39.01	0\\
40.01	0\\
41.01	0\\
42.01	0\\
43.01	0\\
44.01	0\\
45.01	0\\
46.01	0\\
47.01	0\\
48.01	0\\
49.01	0\\
50.01	0\\
51.01	0\\
52.01	0\\
53.01	0\\
54.01	0\\
55.01	0\\
56.01	0\\
57.01	0\\
58.01	0\\
59.01	0\\
60.01	0\\
61.01	0\\
62.01	0\\
63.01	0\\
64.01	0\\
65.01	0\\
66.01	0\\
67.01	0\\
68.01	0\\
69.01	0\\
70.01	0\\
71.01	0\\
72.01	0\\
73.01	0\\
74.01	0\\
75.01	0\\
76.01	0\\
77.01	0\\
78.01	0\\
79.01	0\\
80.01	0\\
81.01	0\\
82.01	0\\
83.01	0\\
84.01	0\\
85.01	0\\
86.01	0\\
87.01	0\\
88.01	0\\
89.01	0\\
90.01	0\\
91.01	0\\
92.01	0\\
93.01	0\\
94.01	0\\
95.01	0\\
96.01	0\\
97.01	0\\
98.01	0\\
99.01	0.00188347420949276\\
99.02	0.00193952357968626\\
99.03	0.00199595186539415\\
99.04	0.00205276251364543\\
99.05	0.00210995900922383\\
99.06	0.0021675448751667\\
99.07	0.00222552367327132\\
99.08	0.00228389900460891\\
99.09	0.00234267451004639\\
99.1	0.002401853870776\\
99.11	0.00246144080885295\\
99.12	0.00252143908774134\\
99.13	0.00258185251286822\\
99.14	0.00264268493218622\\
99.15	0.00270394023674476\\
99.16	0.00276562236126998\\
99.17	0.00282773528475357\\
99.18	0.0028902830310507\\
99.19	0.00295326966948699\\
99.2	0.00301669931547491\\
99.21	0.00308057613113965\\
99.22	0.00314490432595475\\
99.23	0.00320968815738764\\
99.24	0.00327493193155525\\
99.25	0.0033406400038899\\
99.26	0.00340681677981562\\
99.27	0.00347346671543517\\
99.28	0.00354059431822787\\
99.29	0.00360820414775853\\
99.3	0.00367630081639761\\
99.31	0.00374488899005289\\
99.32	0.00381397338891286\\
99.33	0.003883558788202\\
99.34	0.00395365001894827\\
99.35	0.00402425196876301\\
99.36	0.00409536958263347\\
99.37	0.00416700786372826\\
99.38	0.00423917187421598\\
99.39	0.00431186673610018\\
99.4	0.00438509763218818\\
99.41	0.00445886980695187\\
99.42	0.00453318856740286\\
99.43	0.00460805928398217\\
99.44	0.00468348739146472\\
99.45	0.00475947838987908\\
99.46	0.00483603784544274\\
99.47	0.00491317139151319\\
99.48	0.00499088472955523\\
99.49	0.00506918363012486\\
99.5	0.00514807393387008\\
99.51	0.00522756155254905\\
99.52	0.00530765247006595\\
99.53	0.00538835274352498\\
99.54	0.00546966850430286\\
99.55	0.00555160595914044\\
99.56	0.00563417139125364\\
99.57	0.00571737116146438\\
99.58	0.00580121170935183\\
99.59	0.00588569955442464\\
99.6	0.00597084129731456\\
99.61	0.00605664362099196\\
99.62	0.00614311329200395\\
99.63	0.00623025716173545\\
99.64	0.00631808216769411\\
99.65	0.00640659533481932\\
99.66	0.00649580377681639\\
99.67	0.00658571469751614\\
99.68	0.00667633539226089\\
99.69	0.00676767324931748\\
99.7	0.00685973575131798\\
99.71	0.00695253047672899\\
99.72	0.0070460651013502\\
99.73	0.0071403473998536\\
99.74	0.00723538524735233\\
99.75	0.00733118662099236\\
99.76	0.00742775960157701\\
99.77	0.0075251123752253\\
99.78	0.00762325323506514\\
99.79	0.00772219058296238\\
99.8	0.00782193293128693\\
99.81	0.00792248890471694\\
99.82	0.00802386724208233\\
99.83	0.0081260767982489\\
99.84	0.00822912654604423\\
99.85	0.00833302557822688\\
99.86	0.00843778310950011\\
99.87	0.00854340847857167\\
99.88	0.00864991115026125\\
99.89	0.00875730071765702\\
99.9	0.00886558690432315\\
99.91	0.00897477956655976\\
99.92	0.00908488869571742\\
99.93	0.00919592442056783\\
99.94	0.00930789700973289\\
99.95	0.009420816874174\\
99.96	0.00953469456974397\\
99.97	0.00964954079980366\\
99.98	0.00976536641790577\\
99.99	0.00988218243054826\\
100	0.01\\
};
\addlegendentry{$q=0$};

\addplot [color=blue,solid]
  table[row sep=crcr]{%
0.01	0.00162143504819226\\
1.01	0.00162143504819226\\
2.01	0.00162143504819226\\
3.01	0.00162143504819226\\
4.01	0.00162143504819226\\
5.01	0.00162143504819226\\
6.01	0.00162143504819226\\
7.01	0.00162143504819226\\
8.01	0.00162143504819226\\
9.01	0.00162143504819226\\
10.01	0.00162143504819226\\
11.01	0.00162143504819226\\
12.01	0.00162143504819226\\
13.01	0.00162143504819226\\
14.01	0.00162143504819226\\
15.01	0.00162143504819226\\
16.01	0.00162143504819226\\
17.01	0.00162143504819226\\
18.01	0.00162143504819226\\
19.01	0.00162143504819226\\
20.01	0.00162143504819226\\
21.01	0.00162143504819226\\
22.01	0.00162143504819226\\
23.01	0.00162143504819226\\
24.01	0.00162143504819226\\
25.01	0.00162143504819226\\
26.01	0.00162143504819226\\
27.01	0.00162143504819226\\
28.01	0.00162143504819226\\
29.01	0.00162143504819226\\
30.01	0.00162143504819226\\
31.01	0.00162143504819226\\
32.01	0.00162143504819226\\
33.01	0.00162143504819226\\
34.01	0.00162143504819226\\
35.01	0.00162143504819226\\
36.01	0.00162143504819226\\
37.01	0.00162143504819226\\
38.01	0.00162143504819226\\
39.01	0.00162143504819225\\
40.01	0.00162143504819221\\
41.01	0.0016214350481921\\
42.01	0.00162143504819177\\
43.01	0.00162143504819085\\
44.01	0.00162143504818824\\
45.01	0.00162143504818086\\
46.01	0.00162143504816007\\
47.01	0.00162143504810191\\
48.01	0.00162143504794009\\
49.01	0.00162143504749311\\
50.01	0.00162143504626871\\
51.01	0.00162143504294763\\
52.01	0.00162143503404522\\
53.01	0.00162143501052086\\
54.01	0.0016214349494443\\
55.01	0.00162143479433139\\
56.01	0.00162143441135684\\
57.01	0.00162143350012686\\
58.01	0.00162143143786484\\
59.01	0.00162142708863956\\
60.01	0.00162141882862004\\
61.01	0.00162140552473795\\
62.01	0.00162138889599654\\
63.01	0.00162137148007616\\
64.01	0.00162135232211013\\
65.01	0.00162132985636778\\
66.01	0.00162130145470477\\
67.01	0.0016212628304976\\
68.01	0.0016212097158356\\
69.01	0.00162114423258658\\
70.01	0.00162107328571607\\
71.01	0.00162099704961776\\
72.01	0.00162091422619813\\
73.01	0.00162082301169986\\
74.01	0.0016207217752044\\
75.01	0.00162061329719376\\
76.01	0.00162051680403822\\
77.01	0.00162045008134704\\
78.01	0.00162038202707512\\
79.01	0.00162019636512219\\
80.01	0.00161963055478129\\
81.01	0.00161790949924456\\
82.01	0.00161268396694046\\
83.01	0.0016002582561518\\
84.01	0.00158386659166719\\
85.01	0.00155951436451762\\
86.01	0.00152327426365535\\
87.01	0.00146417119193138\\
88.01	0.0013733235365292\\
89.01	0.00121366773820042\\
90.01	0.000933806065604585\\
91.01	0.000545681375545037\\
92.01	0.000157353162382811\\
93.01	0\\
94.01	0\\
95.01	0\\
96.01	0\\
97.01	0\\
98.01	0\\
99.01	0\\
99.02	0\\
99.03	0\\
99.04	0\\
99.05	0\\
99.06	0\\
99.07	0\\
99.08	0\\
99.09	0\\
99.1	0\\
99.11	0\\
99.12	0\\
99.13	0\\
99.14	0\\
99.15	0\\
99.16	0\\
99.17	0\\
99.18	0\\
99.19	0\\
99.2	0\\
99.21	0\\
99.22	0\\
99.23	0\\
99.24	0\\
99.25	0\\
99.26	0\\
99.27	0\\
99.28	0\\
99.29	0\\
99.3	0\\
99.31	0\\
99.32	0\\
99.33	0\\
99.34	0\\
99.35	0\\
99.36	0\\
99.37	0\\
99.38	0\\
99.39	0\\
99.4	0\\
99.41	0\\
99.42	0\\
99.43	0\\
99.44	0\\
99.45	0\\
99.46	0\\
99.47	0\\
99.48	0\\
99.49	0\\
99.5	0\\
99.51	0\\
99.52	0\\
99.53	0\\
99.54	0\\
99.55	0\\
99.56	0\\
99.57	0\\
99.58	0\\
99.59	0\\
99.6	0\\
99.61	0\\
99.62	0\\
99.63	0\\
99.64	0\\
99.65	0\\
99.66	0\\
99.67	0\\
99.68	0\\
99.69	0\\
99.7	0\\
99.71	0\\
99.72	0\\
99.73	0\\
99.74	0\\
99.75	0\\
99.76	0\\
99.77	0\\
99.78	0\\
99.79	0\\
99.8	0\\
99.81	0\\
99.82	0\\
99.83	0\\
99.84	0\\
99.85	0\\
99.86	0\\
99.87	0\\
99.88	0\\
99.89	0\\
99.9	0\\
99.91	0\\
99.92	0\\
99.93	0\\
99.94	0\\
99.95	0\\
99.96	0\\
99.97	0\\
99.98	0\\
99.99	0\\
100	0\\
};
\addlegendentry{$q=1$};

\addplot [color=red,solid]
  table[row sep=crcr]{%
0.01	0\\
1.01	0\\
2.01	0\\
3.01	0\\
4.01	0\\
5.01	0\\
6.01	0\\
7.01	0\\
8.01	0\\
9.01	0\\
10.01	0\\
11.01	0\\
12.01	0\\
13.01	0\\
14.01	0\\
15.01	0\\
16.01	0\\
17.01	0\\
18.01	0\\
19.01	0\\
20.01	0\\
21.01	0\\
22.01	0\\
23.01	0\\
24.01	0\\
25.01	0\\
26.01	0\\
27.01	0\\
28.01	0\\
29.01	0\\
30.01	0\\
31.01	0\\
32.01	0\\
33.01	0\\
34.01	0\\
35.01	0\\
36.01	0\\
37.01	0\\
38.01	0\\
39.01	0\\
40.01	0\\
41.01	0\\
42.01	0\\
43.01	0\\
44.01	0\\
45.01	0\\
46.01	0\\
47.01	0\\
48.01	1.73472347597681e-18\\
49.01	0\\
50.01	0\\
51.01	0\\
52.01	0\\
53.01	0\\
54.01	0\\
55.01	0\\
56.01	0\\
57.01	0\\
58.01	0\\
59.01	0\\
60.01	0\\
61.01	0\\
62.01	0\\
63.01	0\\
64.01	0\\
65.01	0\\
66.01	0\\
67.01	0\\
68.01	0\\
69.01	0\\
70.01	0\\
71.01	0\\
72.01	0\\
73.01	0\\
74.01	0\\
75.01	0\\
76.01	0\\
77.01	1.73472347597681e-18\\
78.01	0\\
79.01	0\\
80.01	0\\
81.01	0\\
82.01	0\\
83.01	0\\
84.01	0\\
85.01	0\\
86.01	0\\
87.01	0\\
88.01	0\\
89.01	0\\
90.01	0\\
91.01	0\\
92.01	0\\
93.01	0\\
94.01	0\\
95.01	0\\
96.01	0\\
97.01	0\\
98.01	0\\
99.01	0\\
99.02	0\\
99.03	0\\
99.04	0\\
99.05	0\\
99.06	0\\
99.07	0\\
99.08	0\\
99.09	0\\
99.1	0\\
99.11	0\\
99.12	0\\
99.13	0\\
99.14	0\\
99.15	0\\
99.16	0\\
99.17	0\\
99.18	0\\
99.19	0\\
99.2	0\\
99.21	0\\
99.22	0\\
99.23	0\\
99.24	0\\
99.25	0\\
99.26	0\\
99.27	0\\
99.28	0\\
99.29	0\\
99.3	0\\
99.31	0\\
99.32	0\\
99.33	0\\
99.34	0\\
99.35	0\\
99.36	0\\
99.37	0\\
99.38	0\\
99.39	0\\
99.4	0\\
99.41	0\\
99.42	0\\
99.43	0\\
99.44	0\\
99.45	0\\
99.46	0\\
99.47	0\\
99.48	0\\
99.49	0\\
99.5	0\\
99.51	0\\
99.52	0\\
99.53	0\\
99.54	0\\
99.55	0\\
99.56	0\\
99.57	0\\
99.58	0\\
99.59	0\\
99.6	0\\
99.61	0\\
99.62	0\\
99.63	0\\
99.64	0\\
99.65	0\\
99.66	0\\
99.67	0\\
99.68	0\\
99.69	0\\
99.7	0\\
99.71	0\\
99.72	0\\
99.73	0\\
99.74	0\\
99.75	0\\
99.76	0\\
99.77	0\\
99.78	0\\
99.79	0\\
99.8	0\\
99.81	0\\
99.82	0\\
99.83	0\\
99.84	0\\
99.85	0\\
99.86	0\\
99.87	0\\
99.88	0\\
99.89	0\\
99.9	0\\
99.91	0\\
99.92	0\\
99.93	0\\
99.94	0\\
99.95	0\\
99.96	0\\
99.97	0\\
99.98	0\\
99.99	0\\
100	0\\
};
\addlegendentry{$q=2$};

\addplot [color=mycolor1,solid]
  table[row sep=crcr]{%
0.01	0\\
1.01	0\\
2.01	0\\
3.01	0\\
4.01	0\\
5.01	0\\
6.01	0\\
7.01	0\\
8.01	0\\
9.01	0\\
10.01	0\\
11.01	0\\
12.01	0\\
13.01	0\\
14.01	0\\
15.01	0\\
16.01	0\\
17.01	0\\
18.01	0\\
19.01	0\\
20.01	0\\
21.01	0\\
22.01	0\\
23.01	0\\
24.01	0\\
25.01	0\\
26.01	0\\
27.01	0\\
28.01	0\\
29.01	0\\
30.01	0\\
31.01	0\\
32.01	0\\
33.01	0\\
34.01	0\\
35.01	0\\
36.01	0\\
37.01	0\\
38.01	0\\
39.01	0\\
40.01	0\\
41.01	0\\
42.01	0\\
43.01	0\\
44.01	0\\
45.01	0\\
46.01	0\\
47.01	0\\
48.01	1.73472347597681e-18\\
49.01	0\\
50.01	0\\
51.01	0\\
52.01	0\\
53.01	0\\
54.01	0\\
55.01	0\\
56.01	0\\
57.01	0\\
58.01	0\\
59.01	0\\
60.01	0\\
61.01	0\\
62.01	0\\
63.01	0\\
64.01	0\\
65.01	0\\
66.01	0\\
67.01	0\\
68.01	0\\
69.01	0\\
70.01	0\\
71.01	0\\
72.01	0\\
73.01	0\\
74.01	0\\
75.01	0\\
76.01	0\\
77.01	1.73472347597681e-18\\
78.01	0\\
79.01	0\\
80.01	0\\
81.01	0\\
82.01	0\\
83.01	0\\
84.01	0\\
85.01	0\\
86.01	0\\
87.01	0\\
88.01	0\\
89.01	0\\
90.01	0\\
91.01	0\\
92.01	0\\
93.01	0\\
94.01	0\\
95.01	0\\
96.01	0\\
97.01	0\\
98.01	0\\
99.01	0\\
99.02	0\\
99.03	0\\
99.04	0\\
99.05	0\\
99.06	0\\
99.07	0\\
99.08	0\\
99.09	0\\
99.1	0\\
99.11	0\\
99.12	0\\
99.13	0\\
99.14	0\\
99.15	0\\
99.16	0\\
99.17	0\\
99.18	0\\
99.19	0\\
99.2	0\\
99.21	0\\
99.22	0\\
99.23	0\\
99.24	0\\
99.25	0\\
99.26	0\\
99.27	0\\
99.28	0\\
99.29	0\\
99.3	0\\
99.31	0\\
99.32	0\\
99.33	0\\
99.34	0\\
99.35	0\\
99.36	0\\
99.37	0\\
99.38	0\\
99.39	0\\
99.4	0\\
99.41	0\\
99.42	0\\
99.43	0\\
99.44	0\\
99.45	0\\
99.46	0\\
99.47	0\\
99.48	0\\
99.49	0\\
99.5	0\\
99.51	0\\
99.52	0\\
99.53	0\\
99.54	0\\
99.55	0\\
99.56	0\\
99.57	0\\
99.58	0\\
99.59	0\\
99.6	0\\
99.61	0\\
99.62	0\\
99.63	0\\
99.64	0\\
99.65	0\\
99.66	0\\
99.67	0\\
99.68	0\\
99.69	0\\
99.7	0\\
99.71	0\\
99.72	0\\
99.73	0\\
99.74	0\\
99.75	0\\
99.76	0\\
99.77	0\\
99.78	0\\
99.79	0\\
99.8	0\\
99.81	0\\
99.82	0\\
99.83	0\\
99.84	0\\
99.85	0\\
99.86	0\\
99.87	0\\
99.88	0\\
99.89	0\\
99.9	0\\
99.91	0\\
99.92	0\\
99.93	0\\
99.94	0\\
99.95	0\\
99.96	0\\
99.97	0\\
99.98	0\\
99.99	0\\
100	0\\
};
\addlegendentry{$q=3$};

\addplot [color=green,solid]
  table[row sep=crcr]{%
0.01	0\\
1.01	0\\
2.01	0\\
3.01	0\\
4.01	0\\
5.01	0\\
6.01	0\\
7.01	0\\
8.01	0\\
9.01	0\\
10.01	0\\
11.01	0\\
12.01	0\\
13.01	0\\
14.01	0\\
15.01	0\\
16.01	0\\
17.01	0\\
18.01	0\\
19.01	0\\
20.01	0\\
21.01	0\\
22.01	0\\
23.01	0\\
24.01	0\\
25.01	0\\
26.01	0\\
27.01	0\\
28.01	0\\
29.01	0\\
30.01	0\\
31.01	0\\
32.01	0\\
33.01	0\\
34.01	0\\
35.01	0\\
36.01	0\\
37.01	0\\
38.01	0\\
39.01	0\\
40.01	0\\
41.01	0\\
42.01	0\\
43.01	0\\
44.01	0\\
45.01	0\\
46.01	0\\
47.01	0\\
48.01	1.73472347597681e-18\\
49.01	0\\
50.01	0\\
51.01	0\\
52.01	0\\
53.01	0\\
54.01	0\\
55.01	0\\
56.01	0\\
57.01	0\\
58.01	0\\
59.01	0\\
60.01	0\\
61.01	0\\
62.01	0\\
63.01	0\\
64.01	0\\
65.01	0\\
66.01	0\\
67.01	0\\
68.01	0\\
69.01	0\\
70.01	0\\
71.01	0\\
72.01	0\\
73.01	0\\
74.01	0\\
75.01	0\\
76.01	0\\
77.01	1.73472347597681e-18\\
78.01	0\\
79.01	0\\
80.01	0\\
81.01	0\\
82.01	0\\
83.01	0\\
84.01	0\\
85.01	0\\
86.01	0\\
87.01	0\\
88.01	0\\
89.01	0\\
90.01	0\\
91.01	0\\
92.01	0\\
93.01	0\\
94.01	0\\
95.01	0\\
96.01	0\\
97.01	0\\
98.01	0\\
99.01	0\\
99.02	0\\
99.03	0\\
99.04	0\\
99.05	0\\
99.06	0\\
99.07	0\\
99.08	0\\
99.09	0\\
99.1	0\\
99.11	0\\
99.12	0\\
99.13	0\\
99.14	0\\
99.15	0\\
99.16	0\\
99.17	0\\
99.18	0\\
99.19	0\\
99.2	0\\
99.21	0\\
99.22	0\\
99.23	0\\
99.24	0\\
99.25	0\\
99.26	0\\
99.27	0\\
99.28	0\\
99.29	0\\
99.3	0\\
99.31	0\\
99.32	0\\
99.33	0\\
99.34	0\\
99.35	0\\
99.36	0\\
99.37	0\\
99.38	0\\
99.39	0\\
99.4	0\\
99.41	0\\
99.42	0\\
99.43	0\\
99.44	0\\
99.45	0\\
99.46	0\\
99.47	0\\
99.48	0\\
99.49	0\\
99.5	0\\
99.51	0\\
99.52	0\\
99.53	0\\
99.54	0\\
99.55	0\\
99.56	0\\
99.57	0\\
99.58	0\\
99.59	0\\
99.6	0\\
99.61	0\\
99.62	0\\
99.63	0\\
99.64	0\\
99.65	0\\
99.66	0\\
99.67	0\\
99.68	0\\
99.69	0\\
99.7	0\\
99.71	0\\
99.72	0\\
99.73	0\\
99.74	0\\
99.75	0\\
99.76	0\\
99.77	0\\
99.78	0\\
99.79	0\\
99.8	0\\
99.81	0\\
99.82	0\\
99.83	0\\
99.84	0\\
99.85	0\\
99.86	0\\
99.87	0\\
99.88	0\\
99.89	0\\
99.9	0\\
99.91	0\\
99.92	0\\
99.93	0\\
99.94	0\\
99.95	0\\
99.96	0\\
99.97	0\\
99.98	0\\
99.99	0\\
100	0\\
};
\addlegendentry{$q=4$};

\end{axis}
\end{tikzpicture}%
 
%  \caption{Continuous Time w/ nFPC}
%\end{subfigure}%
%\hfill%
%\begin{subfigure}{.45\linewidth}
%  \centering
%  \setlength\figureheight{\linewidth} 
%  \setlength\figurewidth{\linewidth}
%  \tikzsetnextfilename{dm_dscr_nFPC_z15}
%  % This file was created by matlab2tikz.
%
%The latest updates can be retrieved from
%  http://www.mathworks.com/matlabcentral/fileexchange/22022-matlab2tikz-matlab2tikz
%where you can also make suggestions and rate matlab2tikz.
%
\definecolor{mycolor1}{rgb}{1.00000,0.00000,1.00000}%
%
\begin{tikzpicture}[trim axis left, trim axis right]

\begin{axis}[%
width=\figurewidth,
height=\figureheight,
at={(0\figurewidth,0\figureheight)},
scale only axis,
every outer x axis line/.append style={black},
every x tick label/.append style={font=\color{black}},
xmin=0,
xmax=100,
%xlabel={Time},
every outer y axis line/.append style={black},
every y tick label/.append style={font=\color{black}},
ymin=0,
ymax=0.015,
%ylabel={Depth $\delta^+$},
axis background/.style={fill=white},
axis x line*=bottom,
axis y line*=left,
yticklabel style={
        /pgf/number format/fixed,
        /pgf/number format/precision=3
},
scaled y ticks=false,
legend style={legend cell align=left,align=left,draw=black,font=\footnotesize, at={(0.98,0.02)},anchor=south east},
every axis legend/.code={\renewcommand\addlegendentry[2][]{}}  %ignore legend locally
]
\addplot [color=green,dashed]
  table[row sep=crcr]{%
1	0.0110455900738313\\
2	0.0110580658419961\\
3	0.0110711002023015\\
4	0.0110847186127849\\
5	0.011098947011392\\
6	0.0111138115686288\\
7	0.0111293383533924\\
8	0.0111455528900724\\
9	0.0111624795806907\\
10	0.0111801409614611\\
11	0.0111985567600587\\
12	0.0112177427225301\\
13	0.011237709201904\\
14	0.0112584595946247\\
15	0.0112799890528045\\
16	0.0113022851250408\\
17	0.0113253364165301\\
18	0.0113490753902364\\
19	0.011373362331623\\
20	0.011398075956214\\
21	0.0114230352769138\\
22	0.0114479928477834\\
23	0.011471374969726\\
24	0.0114947398506091\\
25	0.0115189546331043\\
26	0.0115440435517189\\
27	0.0115700309248424\\
28	0.0115969413676922\\
29	0.0116248002340417\\
30	0.0116536398840924\\
31	0.0116834886735304\\
32	0.0117143689176055\\
33	0.0117463019183949\\
34	0.0117793081126445\\
35	0.0118134075452996\\
36	0.0118486212132247\\
37	0.0118849754575801\\
38	0.0119221351203645\\
39	0.0119603630217099\\
40	0.012000447587354\\
41	0.0120427263453424\\
42	0.0120891770130887\\
43	0.0121365777521125\\
44	0.0121848349365603\\
45	0.0122338121987253\\
46	0.0122833556034042\\
47	0.0123332665891129\\
48	0.0123833847464205\\
49	0.0124335044426007\\
50	0.0124833659630434\\
51	0.0125327020752989\\
52	0.0125812214612951\\
53	0.0126285171151238\\
54	0.0126739498597923\\
55	0.0127169436838526\\
56	0.0127550898468969\\
57	0.0127919097843038\\
58	0.0128267594141972\\
59	0.0128594959001294\\
60	0.0128926643343223\\
61	0.0129261447385129\\
62	0.0129755266230054\\
63	0.0130286155819347\\
64	0.01308106914823\\
65	0.0131327173915753\\
66	0.0131834776323277\\
67	0.013233398777312\\
68	0.0132714278428884\\
69	0.0133090211115002\\
70	0.0133464232949045\\
71	0.0133874495048413\\
72	0.0134268569837115\\
73	0.0134612592362577\\
74	0.0134943750174216\\
75	0.0135268935899649\\
76	0.0135590004382286\\
77	0.0135874994915551\\
78	0.0136129285920795\\
79	0.0136378832610413\\
80	0.0136617356552583\\
81	0.013683673960379\\
82	0.0137032409774594\\
83	0.0137208351231345\\
84	0.0137367768512176\\
85	0.0137511467620862\\
86	0.0137635406378348\\
87	0.0137746336380689\\
88	0.0137848781595571\\
89	0.0137944730921645\\
90	0.0138038563586677\\
91	0.0138132229079681\\
92	0.0138227275629359\\
93	0.0138326426430243\\
94	0.0138436332014765\\
95	0.0138574154315624\\
96	0.0138783790039005\\
97	0.0139179224337987\\
98	0.0140058180264177\\
99	0\\
100	0\\
};
\addlegendentry{$q=-4$};

\addplot [color=mycolor1,dashed]
  table[row sep=crcr]{%
1	0.00993725649741715\\
2	0.00994743908148045\\
3	0.00995810956404846\\
4	0.00996929519187683\\
5	0.00998102523150166\\
6	0.00999333122873422\\
7	0.0100062473246203\\
8	0.0100198106422231\\
9	0.010034061761987\\
10	0.0100490453075863\\
11	0.0100648106694385\\
12	0.0100814129003547\\
13	0.0100989138287825\\
14	0.0101173834499466\\
15	0.0101369016517615\\
16	0.0101575601408662\\
17	0.0101794630802216\\
18	0.0102027310663123\\
19	0.0102275077167328\\
20	0.010253960817155\\
21	0.0102822909185747\\
22	0.0103127551382908\\
23	0.0103469405010079\\
24	0.0103834219573124\\
25	0.0104214986876766\\
26	0.0104612225610747\\
27	0.0105026329818961\\
28	0.0105457418207953\\
29	0.0105904934466075\\
30	0.0106367925732236\\
31	0.010684886305683\\
32	0.0107349767371514\\
33	0.0107870782513646\\
34	0.0108411770875308\\
35	0.0108972207748731\\
36	0.0109551016116553\\
37	0.0110146171596997\\
38	0.0110679656416736\\
39	0.0111136633318416\\
40	0.0111611611350395\\
41	0.0112104653192837\\
42	0.0112615510488951\\
43	0.0113144945252057\\
44	0.0113693662707487\\
45	0.0114259491659851\\
46	0.0114840983088909\\
47	0.0115433802594881\\
48	0.0116047575583339\\
49	0.0116665806572984\\
50	0.0117303611817939\\
51	0.0117959660250449\\
52	0.011863128475386\\
53	0.011931355243533\\
54	0.0119942627636737\\
55	0.012060544277407\\
56	0.0121325121090892\\
57	0.0122056816426013\\
58	0.0122798155140526\\
59	0.0123546103874823\\
60	0.0124295605965819\\
61	0.0125040499180606\\
62	0.0125773193351345\\
63	0.0126484616186782\\
64	0.0127163888925465\\
65	0.0127800816693448\\
66	0.0128372911983206\\
67	0.0128819721583471\\
68	0.0129377934149165\\
69	0.0129931791207402\\
70	0.0130475756700926\\
71	0.0131005037865686\\
72	0.0131570590102849\\
73	0.0132174657469469\\
74	0.0132670203012515\\
75	0.0133135493019146\\
76	0.0133590018515661\\
77	0.0134065253902449\\
78	0.0134559425450279\\
79	0.0135042967524793\\
80	0.0135514186663528\\
81	0.0135969729039468\\
82	0.0136322586220696\\
83	0.0136635805896388\\
84	0.0136896446077703\\
85	0.0137137713083481\\
86	0.0137356710415909\\
87	0.0137550051541237\\
88	0.0137715958172534\\
89	0.0137865131110447\\
90	0.0137990331112737\\
91	0.0138101317004599\\
92	0.0138208052690158\\
93	0.0138315709036979\\
94	0.0138431850534444\\
95	0.0138573254549391\\
96	0.0138783790039005\\
97	0.0139179224337987\\
98	0.0140058180264177\\
99	0\\
100	0\\
};
\addlegendentry{$q=-3$};

\addplot [color=red,dashed]
  table[row sep=crcr]{%
1	0.00845092557785682\\
2	0.00845656823941849\\
3	0.00846249107576248\\
4	0.00846871059035578\\
5	0.00847524450565671\\
6	0.00848211186795421\\
7	0.00848933315827441\\
8	0.00849693040842963\\
9	0.00850492732109275\\
10	0.00851334939277234\\
11	0.00852222403875616\\
12	0.00853158071912726\\
13	0.00854145106361232\\
14	0.00855186898865415\\
15	0.00856287079869339\\
16	0.00857449531764242\\
17	0.00858678440978791\\
18	0.00859978386050781\\
19	0.00861354459996147\\
20	0.00862812540429216\\
21	0.00864359717802262\\
22	0.00866004806934649\\
23	0.00867748307320201\\
24	0.0086959640757178\\
25	0.00871558977544893\\
26	0.00873644734090162\\
27	0.0087585514122345\\
28	0.00878164749295886\\
29	0.00880459120053278\\
30	0.00882338650485376\\
31	0.0088404307082489\\
32	0.00885883486335252\\
33	0.00887888961031317\\
34	0.00890101799341021\\
35	0.00892590423691198\\
36	0.00895483132840351\\
37	0.0089906163796001\\
38	0.0090408605051884\\
39	0.00910378174186449\\
40	0.00916942222967661\\
41	0.00923795615880716\\
42	0.00930957852060985\\
43	0.00938449145442318\\
44	0.00946288840424228\\
45	0.00954494274252755\\
46	0.00963069465067402\\
47	0.00971976082099793\\
48	0.00980496696703615\\
49	0.00986937744349283\\
50	0.00993814592443868\\
51	0.0100119720976435\\
52	0.010091464182447\\
53	0.0101774151149637\\
54	0.0102763767387847\\
55	0.0103797217785801\\
56	0.0104873311297228\\
57	0.0105994090813687\\
58	0.0107161515061263\\
59	0.0108377369718138\\
60	0.010964323063165\\
61	0.0110960446215847\\
62	0.0112330516965564\\
63	0.0113745177345102\\
64	0.0115190868409274\\
65	0.0116689361195459\\
66	0.0118068149881645\\
67	0.011921020819691\\
68	0.0120361397725554\\
69	0.0121513860103602\\
70	0.0122653080138773\\
71	0.0123653408301589\\
72	0.0124650462912041\\
73	0.0125640957472384\\
74	0.0126715293894912\\
75	0.0127771856867971\\
76	0.0128766812873347\\
77	0.0129617370189626\\
78	0.0130429843392688\\
79	0.0131194818731913\\
80	0.0131910198144678\\
81	0.0132530081282395\\
82	0.0133214089066423\\
83	0.0133861696491701\\
84	0.0134477731984394\\
85	0.0135067464386294\\
86	0.0135631957202715\\
87	0.013616831142668\\
88	0.0136673924845396\\
89	0.0137135717499587\\
90	0.0137455637238906\\
91	0.0137691815452557\\
92	0.0137904700127987\\
93	0.0138105125141738\\
94	0.0138302301789764\\
95	0.0138515483058692\\
96	0.0138771841329783\\
97	0.0139179224337987\\
98	0.0140058180264177\\
99	0\\
100	0\\
};
\addlegendentry{$q=-2$};

\addplot [color=blue,dashed]
  table[row sep=crcr]{%
1	0.00681554850718535\\
2	0.00681588973092251\\
3	0.00681624790604493\\
4	0.00681662403229282\\
5	0.00681701918263336\\
6	0.00681743450916522\\
7	0.00681787124928804\\
8	0.00681833073211409\\
9	0.0068188143851154\\
10	0.00681932374100495\\
11	0.00681986044480288\\
12	0.00682042626094946\\
13	0.0068210230805929\\
14	0.00682165293148809\\
15	0.00682231800084379\\
16	0.00682302069294455\\
17	0.00682376371377654\\
18	0.00682455017083602\\
19	0.00682538367532876\\
20	0.00682626823522245\\
21	0.00682720746955256\\
22	0.0068282024376938\\
23	0.00682925825974799\\
24	0.0068303846220623\\
25	0.00683159975836174\\
26	0.00683294794645898\\
27	0.00683455738287878\\
28	0.00683683256702084\\
29	0.00684109416742677\\
30	0.00685170549342327\\
31	0.00686637214688786\\
32	0.006882117146803\\
33	0.00689901996077394\\
34	0.00691716712181518\\
35	0.00693666354110884\\
36	0.00695764583303007\\
37	0.00698024282228485\\
38	0.00700415259648881\\
39	0.00702896958052679\\
40	0.00705474132081956\\
41	0.00708151927823191\\
42	0.00710935815038134\\
43	0.0071383147682871\\
44	0.00716844468766515\\
45	0.00719978697145568\\
46	0.00723232259049564\\
47	0.00726586038409361\\
48	0.00730028131436559\\
49	0.00733651719634623\\
50	0.00737469959653813\\
51	0.00741498112945589\\
52	0.0074575578790814\\
53	0.00750268940781908\\
54	0.00755025072979743\\
55	0.00760052206817181\\
56	0.00765385351790057\\
57	0.00771066220684896\\
58	0.00777144807873026\\
59	0.00783681469444389\\
60	0.00790749797473147\\
61	0.00798441341844466\\
62	0.00806875988614439\\
63	0.00816244126907436\\
64	0.00826904811391383\\
65	0.00839598143236846\\
66	0.00856300219363242\\
67	0.00877525537066877\\
68	0.00899933696993162\\
69	0.00923504581303584\\
70	0.00947869461855486\\
71	0.00971000767935811\\
72	0.00990465437474219\\
73	0.0101095951680048\\
74	0.0103258548766747\\
75	0.0105546924270711\\
76	0.0107977370384361\\
77	0.0110620631704474\\
78	0.0113293632347916\\
79	0.0115421906689915\\
80	0.011763453473767\\
81	0.0119973082073932\\
82	0.0122334666990685\\
83	0.0124319372635047\\
84	0.0125612878259806\\
85	0.0126817069884855\\
86	0.0127970668994875\\
87	0.0129045895389597\\
88	0.0130013449527564\\
89	0.0131014145585904\\
90	0.0132126248377904\\
91	0.0133188642268457\\
92	0.0134158310019865\\
93	0.013511266137075\\
94	0.0136047815916573\\
95	0.0136968616847195\\
96	0.013792309010026\\
97	0.0138803621983305\\
98	0.0140058180264177\\
99	0\\
100	0\\
};
\addlegendentry{$q=-1$};

\addplot [color=black,solid]
  table[row sep=crcr]{%
1	0.00621792783154248\\
2	0.00621795363151064\\
3	0.006217980731081\\
4	0.00621800920860938\\
5	0.00621803914829472\\
6	0.00621807064061459\\
7	0.00621810378283594\\
8	0.0062181386795362\\
9	0.00621817544314337\\
10	0.00621821419451566\\
11	0.00621825506366069\\
12	0.0062182981909585\\
13	0.00621834372977078\\
14	0.00621839185148292\\
15	0.00621844275128388\\
16	0.00621849665212127\\
17	0.00621855381158176\\
18	0.0062186145199329\\
19	0.00621867908373063\\
20	0.00621874781355397\\
21	0.0062188211347773\\
22	0.00621890017299207\\
23	0.00621898736594281\\
24	0.00621908806271129\\
25	0.0062192143396067\\
26	0.00621939307734149\\
27	0.00621968087587299\\
28	0.00622018200515726\\
29	0.00622102489640633\\
30	0.006222065928412\\
31	0.00622318211632672\\
32	0.00622437914036729\\
33	0.00622566332427819\\
34	0.00622704153512148\\
35	0.00622851935407994\\
36	0.00623009471218396\\
37	0.00623174686285373\\
38	0.00623345815815957\\
39	0.00623523145880468\\
40	0.0062370697610722\\
41	0.00623897606923901\\
42	0.0062409531792925\\
43	0.00624300321551038\\
44	0.0062451266250654\\
45	0.00624732106274971\\
46	0.00624958270784438\\
47	0.00625192203351148\\
48	0.00625437951091605\\
49	0.0062569642653318\\
50	0.00625968667240331\\
51	0.00626255775449155\\
52	0.00626558564303955\\
53	0.00626876738880855\\
54	0.00627212092530033\\
55	0.00627566847937362\\
56	0.00627943664884499\\
57	0.00628345771370841\\
58	0.00628777189884787\\
59	0.0062924320428559\\
60	0.00629751206215357\\
61	0.00630312702680858\\
62	0.00630947542351325\\
63	0.00631689409173776\\
64	0.006325890501855\\
65	0.00633704058716358\\
66	0.00634993194303342\\
67	0.00636319983109054\\
68	0.00637669793633343\\
69	0.00639006327006573\\
70	0.00640259934075841\\
71	0.00641406232098269\\
72	0.00642600457558528\\
73	0.00643861422240916\\
74	0.00645234168823419\\
75	0.00646831096392316\\
76	0.00648942213072244\\
77	0.00652294593690508\\
78	0.0065899276595373\\
79	0.00672850623736217\\
80	0.00688408450938001\\
81	0.00706463096777866\\
82	0.00728639351234022\\
83	0.00758917216479368\\
84	0.00797753818783012\\
85	0.00838912380471636\\
86	0.00881866262566954\\
87	0.00926903541158117\\
88	0.00974152623661196\\
89	0.0102143184308963\\
90	0.0106757434510928\\
91	0.0110774318305254\\
92	0.0113693047466867\\
93	0.0116714488375805\\
94	0.0119840727133986\\
95	0.0123074839379101\\
96	0.0126303225374715\\
97	0.0130412667576705\\
98	0.0136048426333631\\
99	0\\
100	0\\
};
\addlegendentry{$q=0$};

\addplot [color=blue,solid]
  table[row sep=crcr]{%
1	0.0139983083805517\\
2	0.0139983082720273\\
3	0.0139983081574379\\
4	0.0139983080363954\\
5	0.0139983079090009\\
6	0.0139983077753098\\
7	0.0139983076341143\\
8	0.0139983074828703\\
9	0.0139983073165798\\
10	0.0139983071255936\\
11	0.0139983068924995\\
12	0.0139983065898676\\
13	0.0139983061849815\\
14	0.0139983056621921\\
15	0.013998305057157\\
16	0.0139983044295312\\
17	0.0139983037771064\\
18	0.0139983030960604\\
19	0.0139983023790402\\
20	0.0139983016112709\\
21	0.0139983007634654\\
22	0.0139982997814332\\
23	0.0139982985788664\\
24	0.013998297060026\\
25	0.0139982952283117\\
26	0.0139982933076538\\
27	0.0139982913154207\\
28	0.0139982892426073\\
29	0.0139982870772956\\
30	0.0139982848063752\\
31	0.0139982824215368\\
32	0.0139982799282731\\
33	0.0139982773437443\\
34	0.0139982746681239\\
35	0.0139982718935258\\
36	0.013998269010109\\
37	0.0139982660076548\\
38	0.0139982628758252\\
39	0.0139982596013694\\
40	0.0139982561655939\\
41	0.0139982525386186\\
42	0.0139982486663215\\
43	0.0139982444414905\\
44	0.0139982396436937\\
45	0.0139982338273148\\
46	0.0139982261626455\\
47	0.0139982153866421\\
48	0.0139982004346597\\
49	0.013998182158501\\
50	0.0139981625593857\\
51	0.0139981420252606\\
52	0.0139981206092228\\
53	0.013998098155511\\
54	0.0139980744000645\\
55	0.0139980488305364\\
56	0.0139980203327613\\
57	0.0139979865168737\\
58	0.0139979432389766\\
59	0.0139976318273502\\
60	0.0139972043922104\\
61	0.0139967529347835\\
62	0.0139962718635278\\
63	0.01399575394748\\
64	0.013995194357183\\
65	0.0139945963173215\\
66	0.0139939544103745\\
67	0.0139932487227211\\
68	0.0139924440877367\\
69	0.0139911937998497\\
70	0.0139898463883936\\
71	0.0139884163899688\\
72	0.0139868807363929\\
73	0.0139852082736702\\
74	0.013983356982005\\
75	0.0139812313196168\\
76	0.0139777517095134\\
77	0.013973616531471\\
78	0.0139691494821928\\
79	0.013964214158185\\
80	0.0139514536016275\\
81	0.0139344793329246\\
82	0.0139107111233564\\
83	0.0138838507138313\\
84	0.0138477349623043\\
85	0.0138081269900492\\
86	0.013736240631162\\
87	0.0136473980836543\\
88	0.0135470430361267\\
89	0.0133243650786069\\
90	0.0130820992896335\\
91	0.0126355529612964\\
92	0.0121478036924208\\
93	0.0116210555269654\\
94	0.0110409883808773\\
95	0.0103634447591698\\
96	0.00918448118066964\\
97	0.00722167411404017\\
98	0.00400581802641767\\
99	0\\
100	0\\
};
\addlegendentry{$q=1$};

\addplot [color=red,solid]
  table[row sep=crcr]{%
1	0.0139245392166005\\
2	0.0139245378203568\\
3	0.0139245363514915\\
4	0.0139245347992432\\
5	0.0139245331576809\\
6	0.0139245314317816\\
7	0.0139245296261191\\
8	0.0139245277306554\\
9	0.0139245257231864\\
10	0.0139245235575464\\
11	0.0139245211335468\\
12	0.0139245182386413\\
13	0.0139245144653401\\
14	0.0139245092097827\\
15	0.0139245021355708\\
16	0.0139244939094515\\
17	0.0139244853833913\\
18	0.0139244765389551\\
19	0.0139244673498546\\
20	0.0139244577696135\\
21	0.0139244476988419\\
22	0.013924436901818\\
23	0.0139244248103099\\
24	0.0139244101461545\\
25	0.0139243906702063\\
26	0.013924237896993\\
27	0.0139240662485454\\
28	0.0139238888741228\\
29	0.0139237054071376\\
30	0.0139235153981125\\
31	0.0139233183180689\\
32	0.0139231136539381\\
33	0.0139229011275037\\
34	0.0139226805548886\\
35	0.013922451429106\\
36	0.0139222130570585\\
37	0.0139219646638605\\
38	0.0139217054655661\\
39	0.0139214346687592\\
40	0.0139211513825356\\
41	0.0139208545924508\\
42	0.0139205431123277\\
43	0.0139202154806119\\
44	0.0139198697130677\\
45	0.0139195026790389\\
46	0.0139191085170946\\
47	0.0139186749789327\\
48	0.0139181780829358\\
49	0.0139175226027776\\
50	0.0139166661680981\\
51	0.0139157672933066\\
52	0.0139148310113012\\
53	0.0139138566709414\\
54	0.0139128405886136\\
55	0.0139117783846599\\
56	0.0139106646228975\\
57	0.0139094885576432\\
58	0.0139082219134894\\
59	0.0139035937526516\\
60	0.0138974969498809\\
61	0.0138910967847774\\
62	0.0138843455455878\\
63	0.0138771613390056\\
64	0.0138694092466516\\
65	0.0138603275814683\\
66	0.0138503406141517\\
67	0.0138397361302318\\
68	0.0138282880259625\\
69	0.0138112612958869\\
70	0.0137930111444582\\
71	0.0137737719994472\\
72	0.0137533280914355\\
73	0.0137313371446556\\
74	0.0137031626087889\\
75	0.0136697451461063\\
76	0.0136193502677612\\
77	0.0135604069971424\\
78	0.0134973267826613\\
79	0.0134292481808521\\
80	0.0133626213272515\\
81	0.0132526603301399\\
82	0.0130496754137029\\
83	0.0128320506032241\\
84	0.012603996387161\\
85	0.0123530187471663\\
86	0.0119599087030613\\
87	0.0114832829858986\\
88	0.010975074275643\\
89	0.0105572257988587\\
90	0.0101048122462807\\
91	0.00982588095647541\\
92	0.00954189884861001\\
93	0.00922397477561944\\
94	0.00880247522304508\\
95	0.00769651298244612\\
96	0.00673001832143179\\
97	0.00574637998611275\\
98	0.00400581802641767\\
99	0\\
100	0\\
};
\addlegendentry{$q=2$};

\addplot [color=mycolor1,solid]
  table[row sep=crcr]{%
1	0.0138659281825004\\
2	0.0138658303664556\\
3	0.0138657284708293\\
4	0.0138656221337992\\
5	0.0138655109202661\\
6	0.0138653944050993\\
7	0.0138652723292665\\
8	0.0138651444583232\\
9	0.0138650103605435\\
10	0.0138648694949254\\
11	0.0138647211728643\\
12	0.0138645643799941\\
13	0.0138643972884491\\
14	0.0138642160275985\\
15	0.0138640125094838\\
16	0.0138637564037386\\
17	0.0138634166197179\\
18	0.0138630640967847\\
19	0.0138626981450955\\
20	0.0138623180178758\\
21	0.0138619228880535\\
22	0.0138615117666275\\
23	0.0138610831948392\\
24	0.0138606340410762\\
25	0.0138601546144108\\
26	0.0138580041589181\\
27	0.0138556098224583\\
28	0.0138531361555401\\
29	0.0138505790042018\\
30	0.0138479335717341\\
31	0.0138451940841394\\
32	0.0138423533955504\\
33	0.013839403382497\\
34	0.0138363387669431\\
35	0.0138331568375606\\
36	0.0138298501579202\\
37	0.0138264085172796\\
38	0.0138228207826792\\
39	0.0138190758423261\\
40	0.0138151622390994\\
41	0.0138110672022985\\
42	0.0138067764237643\\
43	0.0138022737649553\\
44	0.0137975408243028\\
45	0.0137925561702599\\
46	0.0137872935733119\\
47	0.0137817157837313\\
48	0.0137757454786795\\
49	0.0137682035150454\\
50	0.0137582480484675\\
51	0.0137477773588493\\
52	0.0137362444823498\\
53	0.013724017890704\\
54	0.0137112819435204\\
55	0.013697989944474\\
56	0.0136840876981937\\
57	0.0136695166819895\\
58	0.0136542206517211\\
59	0.0136417400219856\\
60	0.0136299858319327\\
61	0.0136175957092921\\
62	0.0136044746383344\\
63	0.0135904563512442\\
64	0.0135751176689904\\
65	0.0135482951130647\\
66	0.0135139517849026\\
67	0.0134729206133208\\
68	0.0134299869659956\\
69	0.0133901687966344\\
70	0.0133483630254395\\
71	0.013303721695035\\
72	0.0132556063839647\\
73	0.0132029130739205\\
74	0.0130893343139838\\
75	0.0129335155393375\\
76	0.0127833506564155\\
77	0.0126277866300801\\
78	0.0124475946068981\\
79	0.0122503212283224\\
80	0.0120352698675127\\
81	0.0118382330606605\\
82	0.0115127760722277\\
83	0.0111610068162654\\
84	0.0107917217872007\\
85	0.0104026883932588\\
86	0.0101489457463948\\
87	0.00995666357602472\\
88	0.00976697868687867\\
89	0.00957643399279868\\
90	0.00939558684152857\\
91	0.00920822989858796\\
92	0.00897855980677334\\
93	0.00842245474939861\\
94	0.00748614997919531\\
95	0.00709933536237606\\
96	0.00659458310959697\\
97	0.00574637998611275\\
98	0.00400581802641767\\
99	0\\
100	0\\
};
\addlegendentry{$q=3$};

\addplot [color=green,solid]
  table[row sep=crcr]{%
1	0.0137934540964822\\
2	0.0137920822172213\\
3	0.0137906552941932\\
4	0.0137891692232524\\
5	0.0137876190349365\\
6	0.0137859986619449\\
7	0.0137843018531052\\
8	0.0137825244217185\\
9	0.0137806633430117\\
10	0.0137787127499051\\
11	0.0137766655642113\\
12	0.0137745137387683\\
13	0.0137722477079747\\
14	0.013769854326488\\
15	0.0137673076174875\\
16	0.0137642538154483\\
17	0.0137601599183347\\
18	0.0137559135221988\\
19	0.0137515064838808\\
20	0.0137469300826655\\
21	0.0137421750415046\\
22	0.0137372316698655\\
23	0.0137320903609994\\
24	0.0137267431809246\\
25	0.0137211891919787\\
26	0.0137172292745451\\
27	0.0137133172931002\\
28	0.0137092661503339\\
29	0.0137050702508529\\
30	0.0137007235066934\\
31	0.0136962183852499\\
32	0.0136915431315267\\
33	0.013686674884002\\
34	0.013681571007666\\
35	0.0136758920841648\\
36	0.0136699121098025\\
37	0.0136637467300657\\
38	0.0136573747939205\\
39	0.013650775046622\\
40	0.0136439353666763\\
41	0.0136368423034981\\
42	0.0136294807701638\\
43	0.0136218336291618\\
44	0.0136138811660326\\
45	0.0136056004152668\\
46	0.0135969649725736\\
47	0.0135879508647332\\
48	0.013578571937541\\
49	0.0135700573734081\\
50	0.0135632395167464\\
51	0.0135557162512302\\
52	0.0135400632882504\\
53	0.01352087067752\\
54	0.0135008948792161\\
55	0.0134800769465166\\
56	0.0134583318837203\\
57	0.0134355222927865\\
58	0.0134101765001236\\
59	0.0133800158311734\\
60	0.0133483586758046\\
61	0.0133150073984438\\
62	0.0132797599746472\\
63	0.01324241044225\\
64	0.0132028895590607\\
65	0.0131717196046901\\
66	0.013128624100755\\
67	0.0130202144267264\\
68	0.0129061707761707\\
69	0.0127855041042009\\
70	0.0126573033069973\\
71	0.0125204864924476\\
72	0.0123736180735568\\
73	0.0122151095257336\\
74	0.0121008584942676\\
75	0.0120087275463001\\
76	0.0119099762978608\\
77	0.0118014285062564\\
78	0.0115223760489935\\
79	0.0111965821677563\\
80	0.0108583570943391\\
81	0.0105063547064552\\
82	0.0103502268726332\\
83	0.0102060253919138\\
84	0.0100646509320887\\
85	0.00993094654733278\\
86	0.00979740506831434\\
87	0.00966105444861244\\
88	0.00952161592045265\\
89	0.00937462806720677\\
90	0.00921206089831327\\
91	0.00900704371056902\\
92	0.00838247713451574\\
93	0.00771699577961124\\
94	0.00743248223339164\\
95	0.00708638392997788\\
96	0.00659458310959697\\
97	0.00574637998611275\\
98	0.00400581802641767\\
99	0\\
100	0\\
};
\addlegendentry{$q=4$};

\end{axis}
\end{tikzpicture}%
 
%  \caption{Discrete Time w/ nFPC}
%\end{subfigure}\\
%
%\leavevmode\smash{\makebox[0pt]{\hspace{-7em}% HORIZONTAL POSITION           
%  \rotatebox[origin=l]{90}{\hspace{20em}% VERTICAL POSITION
%    Depth $\delta^-$}%
%}}\hspace{0pt plus 1filll}\null
%
%Time (s)
%
%\vspace{1cm}
%\begin{subfigure}{\linewidth}
%  \centering
%  \tikzsetnextfilename{deltalegend}
%  \definecolor{mycolor1}{rgb}{1.00000,0.00000,1.00000}%
\begin{tikzpicture}[framed]
    \begingroup
    % inits/clears the lists (which might be populated from previous
    % axes):
    \csname pgfplots@init@cleared@structures\endcsname
    \pgfplotsset{legend style={at={(0,1)},anchor=north west},legend columns=-1,legend style={draw=none,column sep=1ex},legend entries={$q=-4$,$q=-3$,$q=-2$,$q=-1$}}%
    
    \csname pgfplots@addlegendimage\endcsname{thick,green,dashed,sharp plot}
    \csname pgfplots@addlegendimage\endcsname{thick,mycolor1,dashed,sharp plot}
    \csname pgfplots@addlegendimage\endcsname{thick,red,dashed,sharp plot}
    \csname pgfplots@addlegendimage\endcsname{thick,blue,dashed,sharp plot}

    % draws the legend:
    \csname pgfplots@createlegend\endcsname
    \endgroup

    \begingroup
    % inits/clears the lists (which might be populated from previous
    % axes):
    \csname pgfplots@init@cleared@structures\endcsname
    \pgfplotsset{legend style={at={(3.75,0.5)},anchor=north west},legend columns=-1,legend style={draw=none,column sep=1ex},legend entries={$q=0$}}%

    \csname pgfplots@addlegendimage\endcsname{thick,black,sharp plot}

    % draws the legend:
    \csname pgfplots@createlegend\endcsname
    \endgroup

    \begingroup
    % inits/clears the lists (which might be populated from previous
    % axes):
    \csname pgfplots@init@cleared@structures\endcsname
    \pgfplotsset{legend style={at={(0,0)},anchor=north west},legend columns=-1,legend style={draw=none,column sep=1ex},legend entries={$q=+4$,$q=+3$,$q=+2$,$q=+1$}}%
    
    \csname pgfplots@addlegendimage\endcsname{thick,green,sharp plot}
    \csname pgfplots@addlegendimage\endcsname{thick,mycolor1,sharp plot}
    \csname pgfplots@addlegendimage\endcsname{thick,red,sharp plot}
    \csname pgfplots@addlegendimage\endcsname{thick,blue,sharp plot}

    % draws the legend:
    \csname pgfplots@createlegend\endcsname
    \endgroup
\end{tikzpicture} 
%\end{subfigure}%
%  \caption{Optimal sell depths $\delta^{-}$ for Markov state $Z=(\rho = +1, \Delta S = +1)$, implying heavy imbalance in favor of buy pressure, and having previously seen an upward price change. We expect the midprice to rise.}
%  \label{fig:comp_dm_z15}
%\end{figure}

\begin{figure}
\centering
\begin{subfigure}{.45\linewidth}
  \centering
  \setlength\figureheight{\linewidth} 
  \setlength\figurewidth{\linewidth}
  \tikzsetnextfilename{dp_colorbar/dm_cts_z1}
  % This file was created by matlab2tikz.
%
%The latest updates can be retrieved from
%  http://www.mathworks.com/matlabcentral/fileexchange/22022-matlab2tikz-matlab2tikz
%where you can also make suggestions and rate matlab2tikz.
%
\definecolor{mycolor1}{rgb}{1.00000,0.00000,1.00000}%
%
\begin{tikzpicture}[trim axis left, trim axis right]

\begin{axis}[%
width=\figurewidth,
height=\figureheight,
at={(0\figurewidth,0\figureheight)},
scale only axis,
every outer x axis line/.append style={black},
every x tick label/.append style={font=\color{black}},
xmin=0,
xmax=100,
%xlabel={Time},
every outer y axis line/.append style={black},
every y tick label/.append style={font=\color{black}},
ymin=0,
ymax=0.015,
%ylabel={Depth $\delta^-$},
axis background/.style={fill=white},
axis x line*=bottom,
axis y line*=left,
yticklabel style={
        /pgf/number format/fixed,
        /pgf/number format/precision=3
},
scaled y ticks=false,
legend style={legend cell align=left,align=left,draw=black,font=\footnotesize, at={(0.98,0.02)},anchor=south east},
every axis legend/.code={\renewcommand\addlegendentry[2][]{}}  %ignore legend locally
]
\addplot [color=green,dashed]
  table[row sep=crcr]{%
0.01	0.00986194196855773\\
1.01	0.00986341623340505\\
2.01	0.00986496661174401\\
3.01	0.00986659669123327\\
4.01	0.00986831003816703\\
5.01	0.00987011011999358\\
6.01	0.00987200019916482\\
7.01	0.00987398319805029\\
8.01	0.00987606169403044\\
9.01	0.00987823934078589\\
10.01	0.009880521604036\\
11.01	0.00988291449512232\\
12.01	0.00988542448547511\\
13.01	0.00988805855563265\\
14.01	0.0098908242511548\\
15.01	0.00989372974661671\\
16.01	0.00989678391910397\\
17.01	0.00989999643292112\\
18.01	0.00990337783757954\\
19.01	0.009906939681569\\
20.01	0.0099106946449578\\
21.01	0.00991465669453728\\
22.01	0.00991884126606046\\
23.01	0.00992326547916659\\
24.01	0.00992794839188696\\
25.01	0.00993291130326705\\
26.01	0.00993817811470555\\
27.01	0.0099437757632201\\
28.01	0.00994973474311438\\
29.01	0.00995608973624679\\
30.01	0.00996288037145739\\
31.01	0.00997015208471616\\
32.01	0.00997795635927097\\
33.01	0.00998633940302696\\
34.01	0.0099951270420302\\
35.01	0.01\\
36.01	0.01\\
37.01	0.01\\
38.01	0.01\\
39.01	0.01\\
40.01	0.01\\
41.01	0.01\\
42.01	0.01\\
43.01	0.01\\
44.01	0.01\\
45.01	0.01\\
46.01	0.01\\
47.01	0.01\\
48.01	0.01\\
49.01	0.01\\
50.01	0.01\\
51.01	0.01\\
52.01	0.01\\
53.01	0.01\\
54.01	0.01\\
55.01	0.01\\
56.01	0.01\\
57.01	0.01\\
58.01	0.01\\
59.01	0.01\\
60.01	0.01\\
61.01	0.01\\
62.01	0.01\\
63.01	0.01\\
64.01	0.01\\
65.01	0.01\\
66.01	0.01\\
67.01	0.01\\
68.01	0.01\\
69.01	0.01\\
70.01	0.01\\
71.01	0.01\\
72.01	0.01\\
73.01	0.01\\
74.01	0.01\\
75.01	0.01\\
76.01	0.01\\
77.01	0.01\\
78.01	0.01\\
79.01	0.01\\
80.01	0.01\\
81.01	0.01\\
82.01	0.01\\
83.01	0.01\\
84.01	0.01\\
85.01	0.01\\
86.01	0.01\\
87.01	0.01\\
88.01	0.01\\
89.01	0.01\\
90.01	0.01\\
91.01	0.01\\
92.01	0.01\\
93.01	0.01\\
94.01	0.01\\
95.01	0.01\\
96.01	0.01\\
97.01	0.01\\
98.01	0.01\\
99.01	0.01\\
99.02	0.01\\
99.03	0.01\\
99.04	0.01\\
99.05	0.01\\
99.06	0.01\\
99.07	0.01\\
99.08	0.01\\
99.09	0.01\\
99.1	0.01\\
99.11	0.01\\
99.12	0.01\\
99.13	0.01\\
99.14	0.01\\
99.15	0.01\\
99.16	0.01\\
99.17	0.01\\
99.18	0.01\\
99.19	0.01\\
99.2	0.01\\
99.21	0.01\\
99.22	0.01\\
99.23	0.01\\
99.24	0.01\\
99.25	0.01\\
99.26	0.01\\
99.27	0.01\\
99.28	0.01\\
99.29	0.01\\
99.3	0.01\\
99.31	0.01\\
99.32	0.01\\
99.33	0.01\\
99.34	0.01\\
99.35	0.01\\
99.36	0.01\\
99.37	0.01\\
99.38	0.01\\
99.39	0.01\\
99.4	0.01\\
99.41	0.01\\
99.42	0.01\\
99.43	0.01\\
99.44	0.01\\
99.45	0.01\\
99.46	0.01\\
99.47	0.01\\
99.48	0.01\\
99.49	0.01\\
99.5	0.01\\
99.51	0.01\\
99.52	0.01\\
99.53	0.01\\
99.54	0.01\\
99.55	0.01\\
99.56	0.01\\
99.57	0.01\\
99.58	0.01\\
99.59	0.01\\
99.6	0.01\\
99.61	0.01\\
99.62	0.01\\
99.63	0.01\\
99.64	0.01\\
99.65	0.01\\
99.66	0.01\\
99.67	0.01\\
99.68	0.01\\
99.69	0.01\\
99.7	0.01\\
99.71	0.01\\
99.72	0.01\\
99.73	0.01\\
99.74	0.01\\
99.75	0.01\\
99.76	0.01\\
99.77	0.01\\
99.78	0.01\\
99.79	0.01\\
99.8	0.01\\
99.81	0.01\\
99.82	0.01\\
99.83	0.01\\
99.84	0.01\\
99.85	0.01\\
99.86	0.01\\
99.87	0.01\\
99.88	0.01\\
99.89	0.01\\
99.9	0.01\\
99.91	0.01\\
99.92	0.01\\
99.93	0.01\\
99.94	0.01\\
99.95	0.01\\
99.96	0.01\\
99.97	0.01\\
99.98	0.01\\
99.99	0.01\\
100	0.01\\
};
\addlegendentry{$q=-4$};

\addplot [color=mycolor1,dashed]
  table[row sep=crcr]{%
0.01	0.00899553026425094\\
1.01	0.00899681616575985\\
2.01	0.00899816538225813\\
3.01	0.00899958107874248\\
4.01	0.00900106658403744\\
5.01	0.00900262540233493\\
6.01	0.00900426122739801\\
7.01	0.00900597796249557\\
8.01	0.00900777976385147\\
9.01	0.00900967112417916\\
10.01	0.00901165685622353\\
11.01	0.00901374206170057\\
12.01	0.00901593214839169\\
13.01	0.00901823285092495\\
14.01	0.00902065025312495\\
15.01	0.00902319081204287\\
16.01	0.00902586138377884\\
17.01	0.00902866925120362\\
18.01	0.00903162215367876\\
19.01	0.00903472831885562\\
20.01	0.00903799649660647\\
21.01	0.00904143599509583\\
22.01	0.00904505671893642\\
23.01	0.00904886920927959\\
24.01	0.00905288468555789\\
25.01	0.00905711508841163\\
26.01	0.00906157312307532\\
27.01	0.0090662723021453\\
28.01	0.00907122698614346\\
29.01	0.00907645241939374\\
30.01	0.00908196475559562\\
31.01	0.00908778104450535\\
32.01	0.0090939189241983\\
33.01	0.00910039335109809\\
34.01	0.00910718464044096\\
35.01	0.00911411520958012\\
36.01	0.00912130597047572\\
37.01	0.00912884125803057\\
38.01	0.00913678267528008\\
39.01	0.0091451622034404\\
40.01	0.00915401322426913\\
41.01	0.00916337289989351\\
42.01	0.00917328265695761\\
43.01	0.00918378818678301\\
44.01	0.00919493637913159\\
45.01	0.00920676382532887\\
46.01	0.00921932995469721\\
47.01	0.00923275864703257\\
48.01	0.00924714537333336\\
49.01	0.00926259594936277\\
50.01	0.00927924146141869\\
51.01	0.0092974388850447\\
52.01	0.00932168772055321\\
53.01	0.00937977735553907\\
54.01	0.00944582638911571\\
55.01	0.0095144883310851\\
56.01	0.00958588953326436\\
57.01	0.00966019282817128\\
58.01	0.00973764887814367\\
59.01	0.00981796544005801\\
60.01	0.00989919159200994\\
61.01	0.00997816795322349\\
62.01	0.01\\
63.01	0.01\\
64.01	0.01\\
65.01	0.01\\
66.01	0.01\\
67.01	0.01\\
68.01	0.01\\
69.01	0.01\\
70.01	0.01\\
71.01	0.01\\
72.01	0.01\\
73.01	0.01\\
74.01	0.01\\
75.01	0.01\\
76.01	0.01\\
77.01	0.01\\
78.01	0.01\\
79.01	0.01\\
80.01	0.01\\
81.01	0.01\\
82.01	0.01\\
83.01	0.01\\
84.01	0.01\\
85.01	0.01\\
86.01	0.01\\
87.01	0.01\\
88.01	0.01\\
89.01	0.01\\
90.01	0.01\\
91.01	0.01\\
92.01	0.01\\
93.01	0.01\\
94.01	0.01\\
95.01	0.01\\
96.01	0.01\\
97.01	0.01\\
98.01	0.01\\
99.01	0.01\\
99.02	0.01\\
99.03	0.01\\
99.04	0.01\\
99.05	0.01\\
99.06	0.01\\
99.07	0.01\\
99.08	0.01\\
99.09	0.01\\
99.1	0.01\\
99.11	0.01\\
99.12	0.01\\
99.13	0.01\\
99.14	0.01\\
99.15	0.01\\
99.16	0.01\\
99.17	0.01\\
99.18	0.01\\
99.19	0.01\\
99.2	0.01\\
99.21	0.01\\
99.22	0.01\\
99.23	0.01\\
99.24	0.01\\
99.25	0.01\\
99.26	0.01\\
99.27	0.01\\
99.28	0.01\\
99.29	0.01\\
99.3	0.01\\
99.31	0.01\\
99.32	0.01\\
99.33	0.01\\
99.34	0.01\\
99.35	0.01\\
99.36	0.01\\
99.37	0.01\\
99.38	0.01\\
99.39	0.01\\
99.4	0.01\\
99.41	0.01\\
99.42	0.01\\
99.43	0.01\\
99.44	0.01\\
99.45	0.01\\
99.46	0.01\\
99.47	0.01\\
99.48	0.01\\
99.49	0.01\\
99.5	0.01\\
99.51	0.01\\
99.52	0.01\\
99.53	0.01\\
99.54	0.01\\
99.55	0.01\\
99.56	0.01\\
99.57	0.01\\
99.58	0.01\\
99.59	0.01\\
99.6	0.01\\
99.61	0.01\\
99.62	0.01\\
99.63	0.01\\
99.64	0.01\\
99.65	0.01\\
99.66	0.01\\
99.67	0.01\\
99.68	0.01\\
99.69	0.01\\
99.7	0.01\\
99.71	0.01\\
99.72	0.01\\
99.73	0.01\\
99.74	0.01\\
99.75	0.01\\
99.76	0.01\\
99.77	0.01\\
99.78	0.01\\
99.79	0.01\\
99.8	0.01\\
99.81	0.01\\
99.82	0.01\\
99.83	0.01\\
99.84	0.01\\
99.85	0.01\\
99.86	0.01\\
99.87	0.01\\
99.88	0.01\\
99.89	0.01\\
99.9	0.01\\
99.91	0.01\\
99.92	0.01\\
99.93	0.01\\
99.94	0.01\\
99.95	0.01\\
99.96	0.01\\
99.97	0.01\\
99.98	0.01\\
99.99	0.01\\
100	0.01\\
};
\addlegendentry{$q=-3$};

\addplot [color=red,dashed]
  table[row sep=crcr]{%
0.01	0.00755870723885662\\
1.01	0.00755995005918438\\
2.01	0.00756125346743988\\
3.01	0.00756262048487351\\
4.01	0.00756405429013468\\
5.01	0.00756555822817135\\
6.01	0.00756713581970831\\
7.01	0.00756879077156534\\
8.01	0.00757052698867975\\
9.01	0.0075723485862971\\
10.01	0.00757425989849911\\
11.01	0.00757626548904896\\
12.01	0.00757837016438062\\
13.01	0.00758057898743496\\
14.01	0.00758289729236785\\
15.01	0.0075853307001909\\
16.01	0.00758788513541049\\
17.01	0.00759056684373631\\
18.01	0.00759338241093571\\
19.01	0.00759633878291826\\
20.01	0.00759944328714147\\
21.01	0.00760270365543958\\
22.01	0.00760612804838858\\
23.01	0.00760972508133611\\
24.01	0.00761350385224449\\
25.01	0.00761747397152153\\
26.01	0.00762164559404763\\
27.01	0.00762602945365452\\
28.01	0.00763063690036136\\
29.01	0.00763547994066308\\
30.01	0.00764057128052756\\
31.01	0.00764592436513198\\
32.01	0.00765155336632105\\
33.01	0.0076574727872834\\
34.01	0.00766369546365032\\
35.01	0.00767023623343137\\
36.01	0.00767712760083648\\
37.01	0.00768429575999159\\
38.01	0.00769182629889674\\
39.01	0.00769975577025217\\
40.01	0.00770810772349439\\
41.01	0.00771690722720528\\
42.01	0.00772618087361956\\
43.01	0.00773595632075487\\
44.01	0.00774625890726877\\
45.01	0.00775709205126277\\
46.01	0.00776843215176224\\
47.01	0.00778040704901406\\
48.01	0.00779307824137373\\
49.01	0.00780648966293305\\
50.01	0.00782068547000673\\
51.01	0.0078357159842505\\
52.01	0.00785196613522768\\
53.01	0.00787030265439305\\
54.01	0.00788976411747893\\
55.01	0.00791037679807219\\
56.01	0.00793224877608778\\
57.01	0.00795553195994027\\
58.01	0.00798054100001887\\
59.01	0.00800772956786817\\
60.01	0.00803269001771853\\
61.01	0.00805761703777974\\
62.01	0.00808196757295361\\
63.01	0.00810849652382792\\
64.01	0.00814444089726804\\
65.01	0.00822870901319658\\
66.01	0.00832471528271918\\
67.01	0.00842433066189096\\
68.01	0.00852774614992076\\
69.01	0.00863518389616826\\
70.01	0.00874690286396195\\
71.01	0.00886295416043401\\
72.01	0.00898271641094435\\
73.01	0.00910780635697492\\
74.01	0.00924150988703335\\
75.01	0.00937239286505567\\
76.01	0.00948105615150061\\
77.01	0.00959301356417407\\
78.01	0.00970838445792531\\
79.01	0.00982473370683794\\
80.01	0.00993766075272426\\
81.01	0.01\\
82.01	0.01\\
83.01	0.01\\
84.01	0.01\\
85.01	0.01\\
86.01	0.01\\
87.01	0.01\\
88.01	0.01\\
89.01	0.01\\
90.01	0.01\\
91.01	0.01\\
92.01	0.01\\
93.01	0.01\\
94.01	0.01\\
95.01	0.01\\
96.01	0.01\\
97.01	0.01\\
98.01	0.01\\
99.01	0.01\\
99.02	0.01\\
99.03	0.01\\
99.04	0.01\\
99.05	0.01\\
99.06	0.01\\
99.07	0.01\\
99.08	0.01\\
99.09	0.01\\
99.1	0.01\\
99.11	0.01\\
99.12	0.01\\
99.13	0.01\\
99.14	0.01\\
99.15	0.01\\
99.16	0.01\\
99.17	0.01\\
99.18	0.01\\
99.19	0.01\\
99.2	0.01\\
99.21	0.01\\
99.22	0.01\\
99.23	0.01\\
99.24	0.01\\
99.25	0.01\\
99.26	0.01\\
99.27	0.01\\
99.28	0.01\\
99.29	0.01\\
99.3	0.01\\
99.31	0.01\\
99.32	0.01\\
99.33	0.01\\
99.34	0.01\\
99.35	0.01\\
99.36	0.01\\
99.37	0.01\\
99.38	0.01\\
99.39	0.01\\
99.4	0.01\\
99.41	0.01\\
99.42	0.01\\
99.43	0.01\\
99.44	0.01\\
99.45	0.01\\
99.46	0.01\\
99.47	0.01\\
99.48	0.01\\
99.49	0.01\\
99.5	0.01\\
99.51	0.01\\
99.52	0.01\\
99.53	0.01\\
99.54	0.01\\
99.55	0.01\\
99.56	0.01\\
99.57	0.01\\
99.58	0.01\\
99.59	0.01\\
99.6	0.01\\
99.61	0.01\\
99.62	0.01\\
99.63	0.01\\
99.64	0.01\\
99.65	0.01\\
99.66	0.01\\
99.67	0.01\\
99.68	0.01\\
99.69	0.01\\
99.7	0.01\\
99.71	0.01\\
99.72	0.01\\
99.73	0.01\\
99.74	0.01\\
99.75	0.01\\
99.76	0.01\\
99.77	0.01\\
99.78	0.01\\
99.79	0.01\\
99.8	0.01\\
99.81	0.01\\
99.82	0.01\\
99.83	0.01\\
99.84	0.01\\
99.85	0.01\\
99.86	0.01\\
99.87	0.01\\
99.88	0.01\\
99.89	0.01\\
99.9	0.01\\
99.91	0.01\\
99.92	0.01\\
99.93	0.01\\
99.94	0.01\\
99.95	0.01\\
99.96	0.01\\
99.97	0.01\\
99.98	0.01\\
99.99	0.01\\
100	0.01\\
};
\addlegendentry{$q=-2$};

\addplot [color=blue,dashed]
  table[row sep=crcr]{%
0.01	0.0052336283583432\\
1.01	0.00523405952759689\\
2.01	0.00523451144426746\\
3.01	0.00523498512482119\\
4.01	0.00523548163769127\\
5.01	0.00523600210621855\\
6.01	0.00523654771180334\\
7.01	0.00523711969731229\\
8.01	0.00523771937077291\\
9.01	0.00523834810918792\\
10.01	0.00523900736248806\\
11.01	0.00523969865792891\\
12.01	0.00524042360486926\\
13.01	0.00524118389992261\\
14.01	0.00524198133251751\\
15.01	0.00524281779090816\\
16.01	0.00524369526868089\\
17.01	0.00524461587180792\\
18.01	0.00524558182630641\\
19.01	0.00524659548656777\\
20.01	0.00524765934443075\\
21.01	0.0052487760390819\\
22.01	0.00524994836787769\\
23.01	0.00525117929819646\\
24.01	0.00525247198044371\\
25.01	0.00525382976235229\\
26.01	0.00525525620474145\\
27.01	0.00525675509892372\\
28.01	0.0052583304859752\\
29.01	0.00525998667808767\\
30.01	0.00526172828204119\\
31.01	0.00526356022361065\\
32.01	0.00526548776646861\\
33.01	0.00526751651945328\\
34.01	0.00526965270930324\\
35.01	0.00527190665371612\\
36.01	0.00527432045133193\\
37.01	0.00527675072416128\\
38.01	0.00527921584008379\\
39.01	0.00528181769128329\\
40.01	0.00528456547657343\\
41.01	0.00528746920701615\\
42.01	0.0052905397557492\\
43.01	0.00529378874812398\\
44.01	0.00529722778568257\\
45.01	0.00530086665234674\\
46.01	0.00530472070108109\\
47.01	0.0053088157934421\\
48.01	0.00531317227190738\\
49.01	0.0053178116668415\\
50.01	0.00532275671269768\\
51.01	0.0053280330424496\\
52.01	0.00533370329093398\\
53.01	0.00533981801011714\\
54.01	0.00534637957896053\\
55.01	0.00535343653292726\\
56.01	0.0053610504256392\\
57.01	0.00536931687208603\\
58.01	0.00537849450791244\\
59.01	0.0053900998177927\\
60.01	0.00540984323441987\\
61.01	0.00543174270743712\\
62.01	0.00545487888937094\\
63.01	0.00547962068593745\\
64.01	0.00550706020210262\\
65.01	0.00553812969480962\\
66.01	0.00557050594163365\\
67.01	0.00560465959825306\\
68.01	0.00564079150733262\\
69.01	0.00567912382286856\\
70.01	0.00571997627079662\\
71.01	0.00576368439127989\\
72.01	0.00580831664102341\\
73.01	0.00585543815447048\\
74.01	0.00590998169975969\\
75.01	0.00600682255684936\\
76.01	0.00614685440466075\\
77.01	0.00629203145580564\\
78.01	0.00644315948253136\\
79.01	0.00659897529478316\\
80.01	0.00674910200436107\\
81.01	0.00690115119851525\\
82.01	0.00705774555742922\\
83.01	0.0072216010520313\\
84.01	0.00739598435369107\\
85.01	0.0075606539147745\\
86.01	0.00769813899918943\\
87.01	0.00784011427768724\\
88.01	0.00798655784400265\\
89.01	0.00813709028378619\\
90.01	0.00829175391790736\\
91.01	0.00845175835788679\\
92.01	0.00861728743823078\\
93.01	0.00878838632839203\\
94.01	0.00896528894135422\\
95.01	0.00914814562033832\\
96.01	0.00933768382380203\\
97.01	0.00953901623464867\\
98.01	0.00978114964483068\\
99.01	0.01\\
99.02	0.01\\
99.03	0.01\\
99.04	0.01\\
99.05	0.01\\
99.06	0.01\\
99.07	0.01\\
99.08	0.01\\
99.09	0.01\\
99.1	0.01\\
99.11	0.01\\
99.12	0.01\\
99.13	0.01\\
99.14	0.01\\
99.15	0.01\\
99.16	0.01\\
99.17	0.01\\
99.18	0.01\\
99.19	0.01\\
99.2	0.01\\
99.21	0.01\\
99.22	0.01\\
99.23	0.01\\
99.24	0.01\\
99.25	0.01\\
99.26	0.01\\
99.27	0.01\\
99.28	0.01\\
99.29	0.01\\
99.3	0.01\\
99.31	0.01\\
99.32	0.01\\
99.33	0.01\\
99.34	0.01\\
99.35	0.01\\
99.36	0.01\\
99.37	0.01\\
99.38	0.01\\
99.39	0.01\\
99.4	0.01\\
99.41	0.01\\
99.42	0.01\\
99.43	0.01\\
99.44	0.01\\
99.45	0.01\\
99.46	0.01\\
99.47	0.01\\
99.48	0.01\\
99.49	0.01\\
99.5	0.01\\
99.51	0.01\\
99.52	0.01\\
99.53	0.01\\
99.54	0.01\\
99.55	0.01\\
99.56	0.01\\
99.57	0.01\\
99.58	0.01\\
99.59	0.01\\
99.6	0.01\\
99.61	0.01\\
99.62	0.01\\
99.63	0.01\\
99.64	0.01\\
99.65	0.01\\
99.66	0.01\\
99.67	0.01\\
99.68	0.01\\
99.69	0.01\\
99.7	0.01\\
99.71	0.01\\
99.72	0.01\\
99.73	0.01\\
99.74	0.01\\
99.75	0.01\\
99.76	0.01\\
99.77	0.01\\
99.78	0.01\\
99.79	0.01\\
99.8	0.01\\
99.81	0.01\\
99.82	0.01\\
99.83	0.01\\
99.84	0.01\\
99.85	0.01\\
99.86	0.01\\
99.87	0.01\\
99.88	0.01\\
99.89	0.01\\
99.9	0.01\\
99.91	0.01\\
99.92	0.01\\
99.93	0.01\\
99.94	0.01\\
99.95	0.01\\
99.96	0.01\\
99.97	0.01\\
99.98	0.01\\
99.99	0.01\\
100	0.01\\
};
\addlegendentry{$q=-1$};

\addplot [color=black,solid]
  table[row sep=crcr]{%
0.01	0\\
1.01	0\\
2.01	0\\
3.01	0\\
4.01	0\\
5.01	0\\
6.01	0\\
7.01	0\\
8.01	0\\
9.01	0\\
10.01	0\\
11.01	0\\
12.01	0\\
13.01	0\\
14.01	0\\
15.01	0\\
16.01	0\\
17.01	0\\
18.01	0\\
19.01	0\\
20.01	0\\
21.01	0\\
22.01	0\\
23.01	0\\
24.01	0\\
25.01	0\\
26.01	0\\
27.01	0\\
28.01	0\\
29.01	0\\
30.01	0\\
31.01	0\\
32.01	0\\
33.01	0\\
34.01	0\\
35.01	0\\
36.01	0\\
37.01	2.7098314418425e-07\\
38.01	6.72601697098568e-07\\
39.01	1.09603785213019e-06\\
40.01	1.54272042145881e-06\\
41.01	2.01419497010934e-06\\
42.01	2.51210975899668e-06\\
43.01	3.03814885859767e-06\\
44.01	3.59391466852634e-06\\
45.01	4.18118877907291e-06\\
46.01	4.80313631412779e-06\\
47.01	5.46318199914041e-06\\
48.01	6.1643496488701e-06\\
49.01	6.90976409224893e-06\\
50.01	7.70261142136672e-06\\
51.01	8.54754165984829e-06\\
52.01	9.45360081083813e-06\\
53.01	1.04256579901065e-05\\
54.01	1.14689184704305e-05\\
55.01	1.25924347604209e-05\\
56.01	1.38107675656928e-05\\
57.01	1.5158325066213e-05\\
58.01	1.67443761224864e-05\\
59.01	1.88896805408314e-05\\
60.01	2.16804617222208e-05\\
61.01	2.46604273995161e-05\\
62.01	2.78498511532919e-05\\
63.01	3.13412241101391e-05\\
64.01	3.53395605633931e-05\\
65.01	4.01477618857358e-05\\
66.01	4.60670770527482e-05\\
67.01	5.2389342955924e-05\\
68.01	5.91313983273386e-05\\
69.01	6.63797301328692e-05\\
70.01	7.43735053918473e-05\\
71.01	8.40659909861242e-05\\
72.01	9.87569411535407e-05\\
73.01	0.000116219008803816\\
74.01	0.00013658469870431\\
75.01	0.00016268964669495\\
76.01	0.000191701514614633\\
77.01	0.000222542667619325\\
78.01	0.000256658377935573\\
79.01	0.000301126961267025\\
80.01	0.000364415228369714\\
81.01	0.00043130172413214\\
82.01	0.000502298045484141\\
83.01	0.000579330785094683\\
84.01	0.000671081018125567\\
85.01	0.000824801827111416\\
86.01	0.00103084607679419\\
87.01	0.00124341172800786\\
88.01	0.00146287110651892\\
89.01	0.0016896428967384\\
90.01	0.0019243436918427\\
91.01	0.00216766740583869\\
92.01	0.00242027370067842\\
93.01	0.00268293612335087\\
94.01	0.00295656462490684\\
95.01	0.00324228884436571\\
96.01	0.00354208727028447\\
97.01	0.00386305506827559\\
98.01	0.00425019667752173\\
99.01	0.00507764158572622\\
99.02	0.00509333154654971\\
99.03	0.00510930350582921\\
99.04	0.00512556415351287\\
99.05	0.00514212035096883\\
99.06	0.00515897913589884\\
99.07	0.00517614772741901\\
99.08	0.00519363353131232\\
99.09	0.00521144414546006\\
99.1	0.00522958736545719\\
99.11	0.00524807119042307\\
99.12	0.00526690382901496\\
99.13	0.0052860937056522\\
99.14	0.00530564946695966\\
99.15	0.00532557998843928\\
99.16	0.00534589438138126\\
99.17	0.00536660200002445\\
99.18	0.00538771244897647\\
99.19	0.00540923559090341\\
99.2	0.00543118155450547\\
99.21	0.00545356074278902\\
99.22	0.00547638384164783\\
99.23	0.00549966182876664\\
99.24	0.00552340598286617\\
99.25	0.005547627893303\\
99.26	0.00557233947004092\\
99.27	0.00559755295401146\\
99.28	0.0056232809278821\\
99.29	0.005649536327252\\
99.3	0.00567633245229586\\
99.31	0.00570368297987814\\
99.32	0.00573160197616093\\
99.33	0.00576010390973003\\
99.34	0.00578920366526573\\
99.35	0.00581891655778816\\
99.36	0.00584925834750219\\
99.37	0.00588024525895556\\
99.38	0.00591189402111133\\
99.39	0.00594422185780725\\
99.4	0.00597717154226581\\
99.41	0.00601075621332471\\
99.42	0.00604499237217719\\
99.43	0.00607989701668596\\
99.44	0.00611548765934936\\
99.45	0.006151768920206\\
99.46	0.00618875089111782\\
99.47	0.00622645214145022\\
99.48	0.00626487592887922\\
99.49	0.00630402223164896\\
99.5	0.00634391039886314\\
99.51	0.00638456038915832\\
99.52	0.00642599279407218\\
99.53	0.0064682288625287\\
99.54	0.00651129052650586\\
99.55	0.00655520042795615\\
99.56	0.00659998194705504\\
99.57	0.00664565923185813\\
99.58	0.00669225722945325\\
99.59	0.00673980171870044\\
99.6	0.00678831934465935\\
99.61	0.00683783765481112\\
99.62	0.00688838513718996\\
99.63	0.00693999126054839\\
99.64	0.00699268651668975\\
99.65	0.00704650246511194\\
99.66	0.00710147178011797\\
99.67	0.00715762830056103\\
99.68	0.00721500708240582\\
99.69	0.00727364445430238\\
99.7	0.00733357807638529\\
99.71	0.00739484700252884\\
99.72	0.00745749174630883\\
99.73	0.00752155435094301\\
99.74	0.00758707846350445\\
99.75	0.00765410941373053\\
99.76	0.00772269429777857\\
99.77	0.00779288206731036\\
99.78	0.00786472362432376\\
99.79	0.00793827192218827\\
99.8	0.00801358207338469\\
99.81	0.00809071146449738\\
99.82	0.00816971987906073\\
99.83	0.00825066962892118\\
99.84	0.00833362569484242\\
99.85	0.00841865587715554\\
99.86	0.00850583095733869\\
99.87	0.00859522487150397\\
99.88	0.00868691489687313\\
99.89	0.00878098185244132\\
99.9	0.00887751031515977\\
99.91	0.00897658884981403\\
99.92	0.00907831025960909\\
99.93	0.00918277186280922\\
99.94	0.00929007578758202\\
99.95	0.00940032929168484\\
99.96	0.00951364510957589\\
99.97	0.00963014182985264\\
99.98	0.00974994430628741\\
99.99	0.00987318410615103\\
100	0.01\\
};
\addlegendentry{$q=0$};

\addplot [color=blue,solid]
  table[row sep=crcr]{%
0.01	0.01\\
1.01	0.01\\
2.01	0.01\\
3.01	0.01\\
4.01	0.01\\
5.01	0.01\\
6.01	0.01\\
7.01	0.01\\
8.01	0.01\\
9.01	0.01\\
10.01	0.01\\
11.01	0.01\\
12.01	0.01\\
13.01	0.01\\
14.01	0.01\\
15.01	0.01\\
16.01	0.01\\
17.01	0.01\\
18.01	0.01\\
19.01	0.01\\
20.01	0.01\\
21.01	0.01\\
22.01	0.01\\
23.01	0.01\\
24.01	0.01\\
25.01	0.01\\
26.01	0.01\\
27.01	0.01\\
28.01	0.01\\
29.01	0.01\\
30.01	0.01\\
31.01	0.01\\
32.01	0.01\\
33.01	0.01\\
34.01	0.01\\
35.01	0.01\\
36.01	0.01\\
37.01	0.01\\
38.01	0.01\\
39.01	0.01\\
40.01	0.01\\
41.01	0.01\\
42.01	0.01\\
43.01	0.01\\
44.01	0.01\\
45.01	0.01\\
46.01	0.01\\
47.01	0.01\\
48.01	0.01\\
49.01	0.01\\
50.01	0.01\\
51.01	0.01\\
52.01	0.01\\
53.01	0.01\\
54.01	0.01\\
55.01	0.01\\
56.01	0.01\\
57.01	0.01\\
58.01	0.01\\
59.01	0.01\\
60.01	0.01\\
61.01	0.01\\
62.01	0.01\\
63.01	0.01\\
64.01	0.01\\
65.01	0.01\\
66.01	0.01\\
67.01	0.01\\
68.01	0.01\\
69.01	0.01\\
70.01	0.01\\
71.01	0.01\\
72.01	0.01\\
73.01	0.01\\
74.01	0.01\\
75.01	0.01\\
76.01	0.01\\
77.01	0.01\\
78.01	0.01\\
79.01	0.01\\
80.01	0.01\\
81.01	0.01\\
82.01	0.01\\
83.01	0.01\\
84.01	0.01\\
85.01	0.01\\
86.01	0.01\\
87.01	0.01\\
88.01	0.01\\
89.01	0.01\\
90.01	0.01\\
91.01	0.01\\
92.01	0.01\\
93.01	0.01\\
94.01	0.01\\
95.01	0.01\\
96.01	0.01\\
97.01	0.01\\
98.01	0.01\\
99.01	0.01\\
99.02	0.01\\
99.03	0.01\\
99.04	0.01\\
99.05	0.01\\
99.06	0.01\\
99.07	0.01\\
99.08	0.01\\
99.09	0.01\\
99.1	0.01\\
99.11	0.01\\
99.12	0.01\\
99.13	0.01\\
99.14	0.01\\
99.15	0.01\\
99.16	0.01\\
99.17	0.01\\
99.18	0.01\\
99.19	0.01\\
99.2	0.01\\
99.21	0.01\\
99.22	0.01\\
99.23	0.01\\
99.24	0.01\\
99.25	0.01\\
99.26	0.01\\
99.27	0.01\\
99.28	0.01\\
99.29	0.01\\
99.3	0.01\\
99.31	0.01\\
99.32	0.01\\
99.33	0.01\\
99.34	0.01\\
99.35	0.01\\
99.36	0.01\\
99.37	0.01\\
99.38	0.01\\
99.39	0.01\\
99.4	0.01\\
99.41	0.01\\
99.42	0.00987202571330398\\
99.43	0.00972725105747827\\
99.44	0.00958178279195355\\
99.45	0.00943561855610114\\
99.46	0.00928875095460674\\
99.47	0.00914117239480965\\
99.48	0.00899287508119766\\
99.49	0.00884385100975308\\
99.5	0.00869409196214463\\
99.51	0.00854358949975994\\
99.52	0.00839233495757306\\
99.53	0.00824031943784096\\
99.54	0.00808753380362287\\
99.55	0.00793396867211586\\
99.56	0.00777961440779982\\
99.57	0.00762446111538449\\
99.58	0.007468498632551\\
99.59	0.00731171652247971\\
99.6	0.0071541040661559\\
99.61	0.00699565025444443\\
99.62	0.00683634377992369\\
99.63	0.006676173028469\\
99.64	0.00651512607057483\\
99.65	0.00635319065240467\\
99.66	0.00619035418656071\\
99.67	0.00602660374255689\\
99.68	0.00586192603698214\\
99.69	0.00569630742301583\\
99.7	0.00552973387981608\\
99.71	0.00536219100166203\\
99.72	0.00519366398672537\\
99.73	0.00502413762545392\\
99.74	0.00485359628854843\\
99.75	0.00468202391451301\\
99.76	0.0045094039967583\\
99.77	0.00433571957023531\\
99.78	0.00416095319757633\\
99.79	0.00398508695471812\\
99.8	0.00380810241598083\\
99.81	0.00362998063857475\\
99.82	0.00345070214650499\\
99.83	0.0032702469138424\\
99.84	0.00308859434732733\\
99.85	0.00290572326827031\\
99.86	0.00272161189371181\\
99.87	0.0025362378168008\\
99.88	0.00234957798634912\\
99.89	0.00216160868551618\\
99.9	0.0019723055095755\\
99.91	0.00178164334271157\\
99.92	0.00158959633379209\\
99.93	0.00139613787105757\\
99.94	0.00120124055566599\\
99.95	0.00100487617402678\\
99.96	0.000807015668854001\\
99.97	0.000607629108864105\\
99.98	0.000406685657039194\\
99.99	0.00020415353737153\\
100	0\\
};
\addlegendentry{$q=1$};

\addplot [color=red,solid]
  table[row sep=crcr]{%
0.01	0.01\\
1.01	0.01\\
2.01	0.01\\
3.01	0.01\\
4.01	0.01\\
5.01	0.01\\
6.01	0.01\\
7.01	0.01\\
8.01	0.01\\
9.01	0.01\\
10.01	0.01\\
11.01	0.01\\
12.01	0.01\\
13.01	0.01\\
14.01	0.01\\
15.01	0.01\\
16.01	0.01\\
17.01	0.01\\
18.01	0.01\\
19.01	0.01\\
20.01	0.01\\
21.01	0.01\\
22.01	0.01\\
23.01	0.01\\
24.01	0.01\\
25.01	0.01\\
26.01	0.01\\
27.01	0.01\\
28.01	0.01\\
29.01	0.01\\
30.01	0.01\\
31.01	0.01\\
32.01	0.01\\
33.01	0.01\\
34.01	0.01\\
35.01	0.01\\
36.01	0.01\\
37.01	0.01\\
38.01	0.01\\
39.01	0.01\\
40.01	0.01\\
41.01	0.01\\
42.01	0.01\\
43.01	0.01\\
44.01	0.01\\
45.01	0.01\\
46.01	0.01\\
47.01	0.01\\
48.01	0.01\\
49.01	0.01\\
50.01	0.01\\
51.01	0.01\\
52.01	0.01\\
53.01	0.01\\
54.01	0.01\\
55.01	0.01\\
56.01	0.01\\
57.01	0.01\\
58.01	0.01\\
59.01	0.01\\
60.01	0.01\\
61.01	0.01\\
62.01	0.01\\
63.01	0.01\\
64.01	0.01\\
65.01	0.01\\
66.01	0.01\\
67.01	0.01\\
68.01	0.01\\
69.01	0.01\\
70.01	0.01\\
71.01	0.01\\
72.01	0.01\\
73.01	0.01\\
74.01	0.01\\
75.01	0.01\\
76.01	0.01\\
77.01	0.01\\
78.01	0.01\\
79.01	0.01\\
80.01	0.01\\
81.01	0.01\\
82.01	0.01\\
83.01	0.01\\
84.01	0.01\\
85.01	0.01\\
86.01	0.01\\
87.01	0.01\\
88.01	0.01\\
89.01	0.01\\
90.01	0.01\\
91.01	0.01\\
92.01	0.01\\
93.01	0.01\\
94.01	0.01\\
95.01	0.01\\
96.01	0.01\\
97.01	0.01\\
98.01	0.01\\
99.01	0.01\\
99.02	0.01\\
99.03	0.01\\
99.04	0.01\\
99.05	0.01\\
99.06	0.01\\
99.07	0.01\\
99.08	0.01\\
99.09	0.01\\
99.1	0.01\\
99.11	0.01\\
99.12	0.00996032578161788\\
99.13	0.00979197844179356\\
99.14	0.00962254325655415\\
99.15	0.00945200595910701\\
99.16	0.0092803538332374\\
99.17	0.00910759370775365\\
99.18	0.00893376227252234\\
99.19	0.00875884596546429\\
99.2	0.00858283085638335\\
99.21	0.00840570263481935\\
99.22	0.00822744659744947\\
99.23	0.00804804763501606\\
99.24	0.00786749021875827\\
99.25	0.00768575838632298\\
99.26	0.0075028358535259\\
99.27	0.00731870601091347\\
99.28	0.00713335177468211\\
99.29	0.00694675557035369\\
99.3	0.00675889931578866\\
99.31	0.00656976436843338\\
99.32	0.006379331514453\\
99.33	0.00618758097392389\\
99.34	0.00599449238068321\\
99.35	0.00580004476130922\\
99.36	0.00560421651318306\\
99.37	0.00540698538157952\\
99.38	0.00520832843223798\\
99.39	0.00500822202787693\\
99.4	0.00480664180388455\\
99.41	0.00460356264092079\\
99.42	0.00452719674959113\\
99.43	0.00446639176629747\\
99.44	0.00440503023173172\\
99.45	0.00434310362475017\\
99.46	0.00428060830868315\\
99.47	0.00421754069180432\\
99.48	0.00415389723063298\\
99.49	0.00408967443337529\\
99.5	0.00402486886351078\\
99.51	0.00395947714353091\\
99.52	0.00389349595883679\\
99.53	0.00382692206180365\\
99.54	0.00375975227601999\\
99.55	0.00369198350070989\\
99.56	0.00362361271534733\\
99.57	0.00355463698447212\\
99.58	0.00348505346271721\\
99.59	0.00341485940005822\\
99.6	0.00334405214729615\\
99.61	0.00327262916178542\\
99.62	0.00320058801341965\\
99.63	0.00312792639088859\\
99.64	0.00305464210822042\\
99.65	0.00298073311162465\\
99.66	0.0029061974862751\\
99.67	0.00283103346379373\\
99.68	0.00275523943008846\\
99.69	0.00267881396475747\\
99.7	0.00260175583002313\\
99.71	0.00252406396949578\\
99.72	0.00244573751764144\\
99.73	0.00236677580970948\\
99.74	0.00228717839214652\\
99.75	0.00220694503352383\\
99.76	0.00212607573600818\\
99.77	0.00204457074740711\\
99.78	0.00196243057382237\\
99.79	0.00187965599294699\\
99.8	0.00179624806804411\\
99.81	0.00171220816264801\\
99.82	0.00162753795603073\\
99.83	0.00154223945948034\\
99.84	0.00145631503344026\\
99.85	0.00136976740556229\\
99.86	0.00128259968972964\\
99.87	0.00119481540611014\\
99.88	0.00110641850230417\\
99.89	0.00101741337565592\\
99.9	0.000927804896802106\\
99.91	0.000837598434536804\\
99.92	0.000746799882077327\\
99.93	0.000655415684821711\\
99.94	0.000563452869695232\\
99.95	0.000470919076190374\\
99.96	0.000377822589212344\\
99.97	0.000284172373850622\\
99.98	0.000189978112205905\\
99.99	9.5250242411667e-05\\
100	0\\
};
\addlegendentry{$q=2$};

\addplot [color=mycolor1,solid]
  table[row sep=crcr]{%
0.01	0.01\\
1.01	0.01\\
2.01	0.01\\
3.01	0.01\\
4.01	0.01\\
5.01	0.01\\
6.01	0.01\\
7.01	0.01\\
8.01	0.01\\
9.01	0.01\\
10.01	0.01\\
11.01	0.01\\
12.01	0.01\\
13.01	0.01\\
14.01	0.01\\
15.01	0.01\\
16.01	0.01\\
17.01	0.01\\
18.01	0.01\\
19.01	0.01\\
20.01	0.01\\
21.01	0.01\\
22.01	0.01\\
23.01	0.01\\
24.01	0.01\\
25.01	0.01\\
26.01	0.01\\
27.01	0.01\\
28.01	0.01\\
29.01	0.01\\
30.01	0.01\\
31.01	0.01\\
32.01	0.01\\
33.01	0.01\\
34.01	0.01\\
35.01	0.01\\
36.01	0.01\\
37.01	0.01\\
38.01	0.01\\
39.01	0.01\\
40.01	0.01\\
41.01	0.01\\
42.01	0.01\\
43.01	0.01\\
44.01	0.01\\
45.01	0.01\\
46.01	0.01\\
47.01	0.01\\
48.01	0.01\\
49.01	0.01\\
50.01	0.01\\
51.01	0.01\\
52.01	0.01\\
53.01	0.01\\
54.01	0.01\\
55.01	0.01\\
56.01	0.01\\
57.01	0.01\\
58.01	0.01\\
59.01	0.01\\
60.01	0.01\\
61.01	0.01\\
62.01	0.01\\
63.01	0.01\\
64.01	0.01\\
65.01	0.01\\
66.01	0.01\\
67.01	0.01\\
68.01	0.01\\
69.01	0.01\\
70.01	0.01\\
71.01	0.01\\
72.01	0.01\\
73.01	0.01\\
74.01	0.01\\
75.01	0.01\\
76.01	0.01\\
77.01	0.01\\
78.01	0.01\\
79.01	0.01\\
80.01	0.01\\
81.01	0.01\\
82.01	0.01\\
83.01	0.01\\
84.01	0.01\\
85.01	0.01\\
86.01	0.01\\
87.01	0.01\\
88.01	0.01\\
89.01	0.01\\
90.01	0.01\\
91.01	0.01\\
92.01	0.01\\
93.01	0.01\\
94.01	0.01\\
95.01	0.01\\
96.01	0.01\\
97.01	0.01\\
98.01	0.01\\
99.01	0.00812091055252546\\
99.02	0.00792185855338365\\
99.03	0.00772136441716776\\
99.04	0.00751940674024627\\
99.05	0.00731596371466714\\
99.06	0.0071110129187867\\
99.07	0.0069045312524993\\
99.08	0.00669649491102331\\
99.09	0.00648687935746255\\
99.1	0.00627565929135879\\
99.11	0.00606280862076723\\
99.12	0.00588804584143462\\
99.13	0.00584057973645197\\
99.14	0.00579281442628695\\
99.15	0.00574475348166593\\
99.16	0.00569639881934934\\
99.17	0.00564773260275226\\
99.18	0.00559870678018064\\
99.19	0.00554932343408073\\
99.2	0.00549958489634066\\
99.21	0.00544949375960291\\
99.22	0.00539905288906679\\
99.23	0.00534826543480551\\
99.24	0.00529713484462432\\
99.25	0.00524566487748732\\
99.26	0.00519385961735011\\
99.27	0.00514172348758126\\
99.28	0.00508926126623092\\
99.29	0.00503647810199398\\
99.3	0.00498337953090549\\
99.31	0.00492997151127088\\
99.32	0.00487626043875326\\
99.33	0.00482225315370379\\
99.34	0.00476795696134481\\
99.35	0.00471337965291966\\
99.36	0.00465852952786457\\
99.37	0.00460341541706176\\
99.38	0.00454804670721336\\
99.39	0.0044924333664363\\
99.4	0.00443658597115357\\
99.41	0.00438051573434752\\
99.42	0.00432397001312195\\
99.43	0.00426691221325709\\
99.44	0.00420933777840974\\
99.45	0.00415124214002621\\
99.46	0.00409262069100927\\
99.47	0.00403346878523613\\
99.48	0.00397378173705597\\
99.49	0.00391355482076605\\
99.5	0.00385278327006468\\
99.51	0.00379146227748001\\
99.52	0.00372958699377298\\
99.53	0.00366715252731277\\
99.54	0.0036041539434234\\
99.55	0.00354058626369925\\
99.56	0.00347644446528808\\
99.57	0.00341172348013916\\
99.58	0.00334641819421466\\
99.59	0.0032805234466619\\
99.6	0.0032140340289441\\
99.61	0.00314694468392712\\
99.62	0.00307925010491939\\
99.63	0.00301094493466227\\
99.64	0.00294202376426769\\
99.65	0.00287248113208192\\
99.66	0.00280231155789073\\
99.67	0.00273150950733312\\
99.68	0.00266006939035517\\
99.69	0.00258798555977133\\
99.7	0.0025152523097948\\
99.71	0.00244186387451185\\
99.72	0.00236781442627067\\
99.73	0.00229309807397977\\
99.74	0.00221770886130978\\
99.75	0.0021416407647929\\
99.76	0.00206488769181328\\
99.77	0.00198744347848148\\
99.78	0.00190930188738566\\
99.79	0.00183045660521148\\
99.8	0.00175090124022237\\
99.81	0.00167062931959128\\
99.82	0.00158963428657405\\
99.83	0.00150790949751441\\
99.84	0.00142544821866952\\
99.85	0.00134224362284443\\
99.86	0.00125828878582284\\
99.87	0.00117357668258095\\
99.88	0.00108810018327\\
99.89	0.0010018520489523\\
99.9	0.000914824927074323\\
99.91	0.000827011346659487\\
99.92	0.000738403713201803\\
99.93	0.000648994303240492\\
99.94	0.000558775258594022\\
99.95	0.000467738580230706\\
99.96	0.000375876121751237\\
99.97	0.000283179582456843\\
99.98	0.000189640499974889\\
99.99	9.52502424116652e-05\\
100	0\\
};
\addlegendentry{$q=3$};

\addplot [color=green,solid]
  table[row sep=crcr]{%
0.01	0.01\\
1.01	0.01\\
2.01	0.01\\
3.01	0.01\\
4.01	0.01\\
5.01	0.01\\
6.01	0.01\\
7.01	0.01\\
8.01	0.01\\
9.01	0.01\\
10.01	0.01\\
11.01	0.01\\
12.01	0.01\\
13.01	0.01\\
14.01	0.01\\
15.01	0.01\\
16.01	0.01\\
17.01	0.01\\
18.01	0.01\\
19.01	0.01\\
20.01	0.01\\
21.01	0.01\\
22.01	0.01\\
23.01	0.01\\
24.01	0.01\\
25.01	0.01\\
26.01	0.01\\
27.01	0.01\\
28.01	0.01\\
29.01	0.01\\
30.01	0.01\\
31.01	0.01\\
32.01	0.01\\
33.01	0.01\\
34.01	0.01\\
35.01	0.01\\
36.01	0.01\\
37.01	0.01\\
38.01	0.01\\
39.01	0.01\\
40.01	0.01\\
41.01	0.01\\
42.01	0.01\\
43.01	0.01\\
44.01	0.01\\
45.01	0.01\\
46.01	0.01\\
47.01	0.01\\
48.01	0.01\\
49.01	0.01\\
50.01	0.01\\
51.01	0.01\\
52.01	0.01\\
53.01	0.01\\
54.01	0.01\\
55.01	0.01\\
56.01	0.01\\
57.01	0.01\\
58.01	0.01\\
59.01	0.01\\
60.01	0.01\\
61.01	0.01\\
62.01	0.01\\
63.01	0.01\\
64.01	0.01\\
65.01	0.01\\
66.01	0.01\\
67.01	0.01\\
68.01	0.01\\
69.01	0.01\\
70.01	0.01\\
71.01	0.01\\
72.01	0.01\\
73.01	0.01\\
74.01	0.01\\
75.01	0.01\\
76.01	0.01\\
77.01	0.01\\
78.01	0.01\\
79.01	0.01\\
80.01	0.01\\
81.01	0.01\\
82.01	0.01\\
83.01	0.01\\
84.01	0.01\\
85.01	0.01\\
86.01	0.01\\
87.01	0.01\\
88.01	0.01\\
89.01	0.01\\
90.01	0.01\\
91.01	0.01\\
92.01	0.01\\
93.01	0.01\\
94.01	0.01\\
95.01	0.01\\
96.01	0.01\\
97.01	0.01\\
98.01	0.01\\
99.01	0.00626944196033862\\
99.02	0.00622788811379813\\
99.03	0.00618614667768265\\
99.04	0.00614422623325547\\
99.05	0.00610213589073552\\
99.06	0.0060598853136017\\
99.07	0.00601748474414148\\
99.08	0.00597494503025625\\
99.09	0.00593227765359743\\
99.1	0.00588949475907492\\
99.11	0.00584660918588716\\
99.12	0.00580356314443589\\
99.13	0.00576013259711799\\
99.14	0.00571631383922182\\
99.15	0.00567210310583831\\
99.16	0.00562749657899848\\
99.17	0.00558249047377842\\
99.18	0.00553708115557056\\
99.19	0.00549126494616604\\
99.2	0.0054450381224421\\
99.21	0.00539839691498269\\
99.22	0.00535133750662894\\
99.23	0.00530385603095577\\
99.24	0.00525594857067084\\
99.25	0.00520761115593174\\
99.26	0.00515883976257758\\
99.27	0.00510963031027058\\
99.28	0.00505997866054249\\
99.29	0.00500988061474126\\
99.3	0.00495933191187222\\
99.31	0.0049083282262621\\
99.32	0.00485686516511586\\
99.33	0.00480493826599399\\
99.34	0.00475254299416004\\
99.35	0.00469967473979126\\
99.36	0.00464632881504553\\
99.37	0.00459250045097687\\
99.38	0.00453818479429186\\
99.39	0.00448337690393861\\
99.4	0.00442807174751941\\
99.41	0.00437226419751779\\
99.42	0.00431594958059887\\
99.43	0.00425912324609415\\
99.44	0.0042017805006923\\
99.45	0.00414391660792148\\
99.46	0.00408552678773567\\
99.47	0.00402660621609648\\
99.48	0.00396715002455066\\
99.49	0.00390715329980313\\
99.5	0.00384661108328551\\
99.51	0.00378551837072029\\
99.52	0.00372387011168045\\
99.53	0.00366166120914458\\
99.54	0.00359888651904769\\
99.55	0.00353554084982734\\
99.56	0.00347161896196557\\
99.57	0.00340711556752629\\
99.58	0.00334202532968828\\
99.59	0.00327634286227403\\
99.6	0.0032100627292742\\
99.61	0.0031431794443679\\
99.62	0.00307568747043893\\
99.63	0.00300758121908806\\
99.64	0.00293885505014128\\
99.65	0.00286950327283309\\
99.66	0.00279952017735991\\
99.67	0.00272890000131781\\
99.68	0.00265763692920609\\
99.69	0.00258572509192821\\
99.7	0.00251315856629011\\
99.71	0.0024399313744963\\
99.72	0.00236603748364389\\
99.73	0.002291470805215\\
99.74	0.00221622519456793\\
99.75	0.00214029445042761\\
99.76	0.0020636723143756\\
99.77	0.00198635247034051\\
99.78	0.00190832854408906\\
99.79	0.00182959410271886\\
99.8	0.00175014265415328\\
99.81	0.00166996764663939\\
99.82	0.00158906246824988\\
99.83	0.00150742044638987\\
99.84	0.00142503484730966\\
99.85	0.00134189887562472\\
99.86	0.00125800567384411\\
99.87	0.00117334832190876\\
99.88	0.00108791983674124\\
99.89	0.0010017131718087\\
99.9	0.000914721216700781\\
99.91	0.000826936796724752\\
99.92	0.000738352672519866\\
99.93	0.000648961539693646\\
99.94	0.00055875602848268\\
99.95	0.000467728703440914\\
99.96	0.000375872063158707\\
99.97	0.0002831785400162\\
99.98	0.000189640499974889\\
99.99	9.52502424116652e-05\\
100	0\\
};
\addlegendentry{$q=4$};

\end{axis}
\end{tikzpicture}%

  \caption{Continuous Time}
\end{subfigure}%
\hfill%
\begin{subfigure}{.45\linewidth}
  \centering
  \setlength\figureheight{\linewidth} 
  \setlength\figurewidth{\linewidth}
  \tikzsetnextfilename{dp_colorbar/dm_dscr_z1}
  % This file was created by matlab2tikz.
%
%The latest updates can be retrieved from
%  http://www.mathworks.com/matlabcentral/fileexchange/22022-matlab2tikz-matlab2tikz
%where you can also make suggestions and rate matlab2tikz.
%
\definecolor{mycolor1}{rgb}{0.00000,1.00000,0.14286}%
\definecolor{mycolor2}{rgb}{0.00000,1.00000,0.28571}%
\definecolor{mycolor3}{rgb}{0.00000,1.00000,0.42857}%
\definecolor{mycolor4}{rgb}{0.00000,1.00000,0.57143}%
\definecolor{mycolor5}{rgb}{0.00000,1.00000,0.71429}%
\definecolor{mycolor6}{rgb}{0.00000,1.00000,0.85714}%
\definecolor{mycolor7}{rgb}{0.00000,1.00000,1.00000}%
\definecolor{mycolor8}{rgb}{0.00000,0.87500,1.00000}%
\definecolor{mycolor9}{rgb}{0.00000,0.62500,1.00000}%
\definecolor{mycolor10}{rgb}{0.12500,0.00000,1.00000}%
\definecolor{mycolor11}{rgb}{0.25000,0.00000,1.00000}%
\definecolor{mycolor12}{rgb}{0.37500,0.00000,1.00000}%
\definecolor{mycolor13}{rgb}{0.50000,0.00000,1.00000}%
\definecolor{mycolor14}{rgb}{0.62500,0.00000,1.00000}%
\definecolor{mycolor15}{rgb}{0.75000,0.00000,1.00000}%
\definecolor{mycolor16}{rgb}{0.87500,0.00000,1.00000}%
\definecolor{mycolor17}{rgb}{1.00000,0.00000,1.00000}%
\definecolor{mycolor18}{rgb}{1.00000,0.00000,0.87500}%
\definecolor{mycolor19}{rgb}{1.00000,0.00000,0.62500}%
\definecolor{mycolor20}{rgb}{0.85714,0.00000,0.00000}%
\definecolor{mycolor21}{rgb}{0.71429,0.00000,0.00000}%
%
\begin{tikzpicture}

\begin{axis}[%
width=4.1in,
height=3.803in,
at={(0.809in,0.513in)},
scale only axis,
point meta min=0,
point meta max=1,
every outer x axis line/.append style={black},
every x tick label/.append style={font=\color{black}},
xmin=0,
xmax=600,
every outer y axis line/.append style={black},
every y tick label/.append style={font=\color{black}},
ymin=0,
ymax=0.007,
axis background/.style={fill=white},
axis x line*=bottom,
axis y line*=left,
colormap={mymap}{[1pt] rgb(0pt)=(0,1,0); rgb(7pt)=(0,1,1); rgb(15pt)=(0,0,1); rgb(23pt)=(1,0,1); rgb(31pt)=(1,0,0); rgb(38pt)=(0,0,0)},
colorbar,
colorbar style={separate axis lines,every outer x axis line/.append style={black},every x tick label/.append style={font=\color{black}},every outer y axis line/.append style={black},every y tick label/.append style={font=\color{black}},yticklabels={{-19},{-17},{-15},{-13},{-11},{-9},{-7},{-5},{-3},{-1},{1},{3},{5},{7},{9},{11},{13},{15},{17},{19}}}
]
\addplot [color=green,solid,forget plot]
  table[row sep=crcr]{%
1	0\\
2	0\\
3	0\\
4	0\\
5	0\\
6	0\\
7	0\\
8	0\\
9	0\\
10	0\\
11	0\\
12	0\\
13	0\\
14	0\\
15	0\\
16	0\\
17	0\\
18	0\\
19	0\\
20	0\\
21	0\\
22	0\\
23	0\\
24	0\\
25	0\\
26	0\\
27	0\\
28	0\\
29	0\\
30	0\\
31	0\\
32	0\\
33	0\\
34	0\\
35	0\\
36	0\\
37	0\\
38	0\\
39	0\\
40	0\\
41	0\\
42	0\\
43	0\\
44	0\\
45	0\\
46	0\\
47	0\\
48	0\\
49	0\\
50	0\\
51	0\\
52	0\\
53	0\\
54	0\\
55	0\\
56	0\\
57	0\\
58	0\\
59	0\\
60	0\\
61	0\\
62	0\\
63	0\\
64	0\\
65	0\\
66	0\\
67	0\\
68	0\\
69	0\\
70	0\\
71	0\\
72	0\\
73	0\\
74	0\\
75	0\\
76	0\\
77	0\\
78	0\\
79	0\\
80	0\\
81	0\\
82	0\\
83	0\\
84	0\\
85	0\\
86	0\\
87	0\\
88	0\\
89	0\\
90	0\\
91	0\\
92	0\\
93	0\\
94	0\\
95	0\\
96	0\\
97	0\\
98	0\\
99	0\\
100	0\\
101	0\\
102	0\\
103	0\\
104	0\\
105	0\\
106	0\\
107	0\\
108	0\\
109	0\\
110	0\\
111	0\\
112	0\\
113	0\\
114	0\\
115	0\\
116	0\\
117	0\\
118	0\\
119	0\\
120	0\\
121	0\\
122	0\\
123	0\\
124	0\\
125	0\\
126	0\\
127	0\\
128	0\\
129	0\\
130	0\\
131	0\\
132	0\\
133	0\\
134	0\\
135	0\\
136	0\\
137	0\\
138	0\\
139	0\\
140	0\\
141	0\\
142	0\\
143	0\\
144	0\\
145	0\\
146	0\\
147	0\\
148	0\\
149	0\\
150	0\\
151	0\\
152	0\\
153	0\\
154	0\\
155	0\\
156	0\\
157	0\\
158	0\\
159	0\\
160	0\\
161	0\\
162	0\\
163	0\\
164	0\\
165	0\\
166	0\\
167	0\\
168	0\\
169	0\\
170	0\\
171	0\\
172	0\\
173	0\\
174	0\\
175	0\\
176	0\\
177	0\\
178	0\\
179	0\\
180	0\\
181	0\\
182	0\\
183	0\\
184	0\\
185	0\\
186	0\\
187	0\\
188	0\\
189	0\\
190	0\\
191	0\\
192	0\\
193	0\\
194	0\\
195	0\\
196	0\\
197	0\\
198	0\\
199	0\\
200	0\\
201	0\\
202	0\\
203	0\\
204	0\\
205	0\\
206	0\\
207	0\\
208	0\\
209	0\\
210	0\\
211	0\\
212	0\\
213	0\\
214	0\\
215	0\\
216	0\\
217	0\\
218	0\\
219	0\\
220	0\\
221	0\\
222	0\\
223	0\\
224	0\\
225	0\\
226	0\\
227	0\\
228	0\\
229	0\\
230	0\\
231	0\\
232	0\\
233	0\\
234	0\\
235	0\\
236	0\\
237	0\\
238	0\\
239	0\\
240	0\\
241	0\\
242	0\\
243	0\\
244	0\\
245	0\\
246	0\\
247	0\\
248	0\\
249	0\\
250	0\\
251	0\\
252	0\\
253	0\\
254	0\\
255	0\\
256	0\\
257	0\\
258	0\\
259	0\\
260	0\\
261	0\\
262	0\\
263	0\\
264	0\\
265	0\\
266	0\\
267	0\\
268	0\\
269	0\\
270	0\\
271	0\\
272	0\\
273	0\\
274	0\\
275	0\\
276	0\\
277	0\\
278	0\\
279	0\\
280	0\\
281	0\\
282	0\\
283	0\\
284	0\\
285	0\\
286	0\\
287	0\\
288	0\\
289	0\\
290	0\\
291	0\\
292	0\\
293	0\\
294	0\\
295	0\\
296	0\\
297	0\\
298	0\\
299	0\\
300	0\\
301	0\\
302	0\\
303	0\\
304	0\\
305	0\\
306	0\\
307	0\\
308	0\\
309	0\\
310	0\\
311	0\\
312	0\\
313	0\\
314	0\\
315	0\\
316	0\\
317	0\\
318	0\\
319	0\\
320	0\\
321	0\\
322	0\\
323	0\\
324	0\\
325	0\\
326	0\\
327	0\\
328	0\\
329	0\\
330	0\\
331	0\\
332	0\\
333	0\\
334	0\\
335	0\\
336	0\\
337	0\\
338	0\\
339	0\\
340	0\\
341	0\\
342	0\\
343	0\\
344	0\\
345	0\\
346	0\\
347	0\\
348	0\\
349	0\\
350	0\\
351	0\\
352	0\\
353	0\\
354	0\\
355	0\\
356	0\\
357	0\\
358	0\\
359	0\\
360	0\\
361	0\\
362	0\\
363	0\\
364	0\\
365	0\\
366	0\\
367	0\\
368	0\\
369	0\\
370	0\\
371	0\\
372	0\\
373	0\\
374	0\\
375	0\\
376	0\\
377	0\\
378	0\\
379	0\\
380	0\\
381	0\\
382	0\\
383	0\\
384	0\\
385	0\\
386	0\\
387	0\\
388	0\\
389	0\\
390	0\\
391	0\\
392	0\\
393	0\\
394	0\\
395	0\\
396	0\\
397	0\\
398	0\\
399	0\\
400	0\\
401	0\\
402	0\\
403	0\\
404	0\\
405	0\\
406	0\\
407	0\\
408	0\\
409	0\\
410	0\\
411	0\\
412	0\\
413	0\\
414	0\\
415	0\\
416	0\\
417	0\\
418	0\\
419	0\\
420	0\\
421	0\\
422	0\\
423	0\\
424	0\\
425	0\\
426	0\\
427	0\\
428	0\\
429	0\\
430	0\\
431	0\\
432	0\\
433	0\\
434	0\\
435	0\\
436	0\\
437	0\\
438	0\\
439	0\\
440	0\\
441	0\\
442	0\\
443	0\\
444	0\\
445	0\\
446	0\\
447	0\\
448	0\\
449	0\\
450	0\\
451	0\\
452	0\\
453	0\\
454	0\\
455	0\\
456	0\\
457	0\\
458	0\\
459	0\\
460	0\\
461	0\\
462	0\\
463	0\\
464	0\\
465	0\\
466	0\\
467	0\\
468	0\\
469	0\\
470	0\\
471	0\\
472	0\\
473	0\\
474	0\\
475	0\\
476	0\\
477	0\\
478	0\\
479	0\\
480	0\\
481	0\\
482	0\\
483	0\\
484	0\\
485	0\\
486	0\\
487	0\\
488	0\\
489	0\\
490	0\\
491	0\\
492	0\\
493	0\\
494	0\\
495	0\\
496	0\\
497	0\\
498	0\\
499	0\\
500	0\\
501	0\\
502	0\\
503	0\\
504	0\\
505	0\\
506	0\\
507	0\\
508	0\\
509	0\\
510	0\\
511	0\\
512	0\\
513	0\\
514	0\\
515	0\\
516	0\\
517	0\\
518	0\\
519	0\\
520	0\\
521	0\\
522	0\\
523	0\\
524	0\\
525	0\\
526	0\\
527	0\\
528	0\\
529	0\\
530	0\\
531	0\\
532	0\\
533	0\\
534	0\\
535	0\\
536	0\\
537	0\\
538	0\\
539	0\\
540	2.05661671419882e-05\\
541	4.80458379165604e-05\\
542	7.61310814796848e-05\\
543	0.000104836069177093\\
544	0.000134173974218699\\
545	0.000164154212091505\\
546	0.000194774568566477\\
547	0.000225999060540562\\
548	0.000257871056294527\\
549	0.000290476767183044\\
550	0.00032384026529897\\
551	0.00036317906641787\\
552	0.000398454290577665\\
553	0.000431034512338747\\
554	0.000463811877737574\\
555	0.000497238740956424\\
556	0.000531333042093673\\
557	0.000566110431097209\\
558	0.000601587038550907\\
559	0.000637779508959063\\
560	0.000674705040954899\\
561	0.000712381474657641\\
562	0.000750827512580226\\
563	0.000790063315042241\\
564	0.000830112145911742\\
565	0.00139458170127532\\
566	0.00148687528982974\\
567	0.00154917163770828\\
568	0.00161250300491599\\
569	0.00167689171273661\\
570	0.00174236090733987\\
571	0.00180893460520923\\
572	0.00187663774080877\\
573	0.00194549621643102\\
574	0.00201553695411736\\
575	0.00208678794948849\\
576	0.0021592783272771\\
577	0.00223303839836233\\
578	0.00230809971827688\\
579	0.00238449514780737\\
580	0.00246225891824811\\
581	0.00254142670923474\\
582	0.00262203576145702\\
583	0.00270412508445197\\
584	0.00278773591889263\\
585	0.00287291287145548\\
586	0.00295970681313861\\
587	0.00304818237891642\\
588	0.00313843743693858\\
589	0.00323065362889276\\
590	0.00332522743498846\\
591	0.00342286644802628\\
592	0.00352518337500009\\
593	0.00363635235366165\\
594	0.00376728283869333\\
595	0.0039465900585172\\
596	0.00424941998802837\\
597	0.00487316246495455\\
598	0.00633625614039688\\
599	0\\
600	0\\
};
\addplot [color=mycolor1,solid,forget plot]
  table[row sep=crcr]{%
1	0\\
2	0\\
3	0\\
4	0\\
5	0\\
6	0\\
7	0\\
8	0\\
9	0\\
10	0\\
11	0\\
12	0\\
13	0\\
14	0\\
15	0\\
16	0\\
17	0\\
18	0\\
19	0\\
20	0\\
21	0\\
22	0\\
23	0\\
24	0\\
25	0\\
26	0\\
27	0\\
28	0\\
29	0\\
30	0\\
31	0\\
32	0\\
33	0\\
34	0\\
35	0\\
36	0\\
37	0\\
38	0\\
39	0\\
40	0\\
41	0\\
42	0\\
43	0\\
44	0\\
45	0\\
46	0\\
47	0\\
48	0\\
49	0\\
50	0\\
51	0\\
52	0\\
53	0\\
54	0\\
55	0\\
56	0\\
57	0\\
58	0\\
59	0\\
60	0\\
61	0\\
62	0\\
63	0\\
64	0\\
65	0\\
66	0\\
67	0\\
68	0\\
69	0\\
70	0\\
71	0\\
72	0\\
73	0\\
74	0\\
75	0\\
76	0\\
77	0\\
78	0\\
79	0\\
80	0\\
81	0\\
82	0\\
83	0\\
84	0\\
85	0\\
86	0\\
87	0\\
88	0\\
89	0\\
90	0\\
91	0\\
92	0\\
93	0\\
94	0\\
95	0\\
96	0\\
97	0\\
98	0\\
99	0\\
100	0\\
101	0\\
102	0\\
103	0\\
104	0\\
105	0\\
106	0\\
107	0\\
108	0\\
109	0\\
110	0\\
111	0\\
112	0\\
113	0\\
114	0\\
115	0\\
116	0\\
117	0\\
118	0\\
119	0\\
120	0\\
121	0\\
122	0\\
123	0\\
124	0\\
125	0\\
126	0\\
127	0\\
128	0\\
129	0\\
130	0\\
131	0\\
132	0\\
133	0\\
134	0\\
135	0\\
136	0\\
137	0\\
138	0\\
139	0\\
140	0\\
141	0\\
142	0\\
143	0\\
144	0\\
145	0\\
146	0\\
147	0\\
148	0\\
149	0\\
150	0\\
151	0\\
152	0\\
153	0\\
154	0\\
155	0\\
156	0\\
157	0\\
158	0\\
159	0\\
160	0\\
161	0\\
162	0\\
163	0\\
164	0\\
165	0\\
166	0\\
167	0\\
168	0\\
169	0\\
170	0\\
171	0\\
172	0\\
173	0\\
174	0\\
175	0\\
176	0\\
177	0\\
178	0\\
179	0\\
180	0\\
181	0\\
182	0\\
183	0\\
184	0\\
185	0\\
186	0\\
187	0\\
188	0\\
189	0\\
190	0\\
191	0\\
192	0\\
193	0\\
194	0\\
195	0\\
196	0\\
197	0\\
198	0\\
199	0\\
200	0\\
201	0\\
202	0\\
203	0\\
204	0\\
205	0\\
206	0\\
207	0\\
208	0\\
209	0\\
210	0\\
211	0\\
212	0\\
213	0\\
214	0\\
215	0\\
216	0\\
217	0\\
218	0\\
219	0\\
220	0\\
221	0\\
222	0\\
223	0\\
224	0\\
225	0\\
226	0\\
227	0\\
228	0\\
229	0\\
230	0\\
231	0\\
232	0\\
233	0\\
234	0\\
235	0\\
236	0\\
237	0\\
238	0\\
239	0\\
240	0\\
241	0\\
242	0\\
243	0\\
244	0\\
245	0\\
246	0\\
247	0\\
248	0\\
249	0\\
250	0\\
251	0\\
252	0\\
253	0\\
254	0\\
255	0\\
256	0\\
257	0\\
258	0\\
259	0\\
260	0\\
261	0\\
262	0\\
263	0\\
264	0\\
265	0\\
266	0\\
267	0\\
268	0\\
269	0\\
270	0\\
271	0\\
272	0\\
273	0\\
274	0\\
275	0\\
276	0\\
277	0\\
278	0\\
279	0\\
280	0\\
281	0\\
282	0\\
283	0\\
284	0\\
285	0\\
286	0\\
287	0\\
288	0\\
289	0\\
290	0\\
291	0\\
292	0\\
293	0\\
294	0\\
295	0\\
296	0\\
297	0\\
298	0\\
299	0\\
300	0\\
301	0\\
302	0\\
303	0\\
304	0\\
305	0\\
306	0\\
307	0\\
308	0\\
309	0\\
310	0\\
311	0\\
312	0\\
313	0\\
314	0\\
315	0\\
316	0\\
317	0\\
318	0\\
319	0\\
320	0\\
321	0\\
322	0\\
323	0\\
324	0\\
325	0\\
326	0\\
327	0\\
328	0\\
329	0\\
330	0\\
331	0\\
332	0\\
333	0\\
334	0\\
335	0\\
336	0\\
337	0\\
338	0\\
339	0\\
340	0\\
341	0\\
342	0\\
343	0\\
344	0\\
345	0\\
346	0\\
347	0\\
348	0\\
349	0\\
350	0\\
351	0\\
352	0\\
353	0\\
354	0\\
355	0\\
356	0\\
357	0\\
358	0\\
359	0\\
360	0\\
361	0\\
362	0\\
363	0\\
364	0\\
365	0\\
366	0\\
367	0\\
368	0\\
369	0\\
370	0\\
371	0\\
372	0\\
373	0\\
374	0\\
375	0\\
376	0\\
377	0\\
378	0\\
379	0\\
380	0\\
381	0\\
382	0\\
383	0\\
384	0\\
385	0\\
386	0\\
387	0\\
388	0\\
389	0\\
390	0\\
391	0\\
392	0\\
393	0\\
394	0\\
395	0\\
396	0\\
397	0\\
398	0\\
399	0\\
400	0\\
401	0\\
402	0\\
403	0\\
404	0\\
405	0\\
406	0\\
407	0\\
408	0\\
409	0\\
410	0\\
411	0\\
412	0\\
413	0\\
414	0\\
415	0\\
416	0\\
417	0\\
418	0\\
419	0\\
420	0\\
421	0\\
422	0\\
423	0\\
424	0\\
425	0\\
426	0\\
427	0\\
428	0\\
429	0\\
430	0\\
431	0\\
432	0\\
433	0\\
434	0\\
435	0\\
436	0\\
437	0\\
438	0\\
439	0\\
440	0\\
441	0\\
442	0\\
443	0\\
444	0\\
445	0\\
446	0\\
447	0\\
448	0\\
449	0\\
450	0\\
451	0\\
452	0\\
453	0\\
454	0\\
455	0\\
456	0\\
457	0\\
458	0\\
459	0\\
460	0\\
461	0\\
462	0\\
463	0\\
464	0\\
465	0\\
466	0\\
467	0\\
468	0\\
469	0\\
470	0\\
471	0\\
472	0\\
473	0\\
474	0\\
475	0\\
476	0\\
477	0\\
478	0\\
479	0\\
480	0\\
481	0\\
482	0\\
483	0\\
484	0\\
485	0\\
486	0\\
487	0\\
488	0\\
489	0\\
490	0\\
491	0\\
492	0\\
493	0\\
494	0\\
495	0\\
496	0\\
497	0\\
498	0\\
499	0\\
500	0\\
501	0\\
502	0\\
503	0\\
504	0\\
505	0\\
506	0\\
507	0\\
508	0\\
509	0\\
510	0\\
511	0\\
512	0\\
513	0\\
514	0\\
515	0\\
516	0\\
517	0\\
518	0\\
519	0\\
520	0\\
521	0\\
522	0\\
523	0\\
524	0\\
525	0\\
526	0\\
527	0\\
528	0\\
529	0\\
530	0\\
531	0\\
532	0\\
533	0\\
534	0\\
535	0\\
536	0\\
537	0\\
538	0\\
539	0\\
540	1.6102870266306e-05\\
541	4.34537338603329e-05\\
542	7.14046837864829e-05\\
543	9.99696745254745e-05\\
544	0.000129161874177511\\
545	0.000158991438199705\\
546	0.000189458996257847\\
547	0.000220536555820973\\
548	0.000252227608453156\\
549	0.000284640600573077\\
550	0.00031779515745435\\
551	0.000351711615590877\\
552	0.000390037246453371\\
553	0.00042835512582672\\
554	0.00046174624242412\\
555	0.000495191349166611\\
556	0.000529232833119001\\
557	0.000563955199341248\\
558	0.000599374606030668\\
559	0.000635507545177786\\
560	0.000672371027359062\\
561	0.000709982635300644\\
562	0.000748360637990346\\
563	0.000787524257000521\\
564	0.000827494274159606\\
565	0.00100020560599384\\
566	0.00146187228657998\\
567	0.00154917163766174\\
568	0.0016125030049125\\
569	0.00167689171273503\\
570	0.00174236090733913\\
571	0.00180893460520911\\
572	0.00187663774080867\\
573	0.00194549621643094\\
574	0.00201553695411732\\
575	0.00208678794948844\\
576	0.0021592783272771\\
577	0.00223303839836235\\
578	0.00230809971827686\\
579	0.00238449514780734\\
580	0.00246225891824808\\
581	0.00254142670923472\\
582	0.002622035761457\\
583	0.00270412508445196\\
584	0.00278773591889263\\
585	0.00287291287145546\\
586	0.0029597068131386\\
587	0.00304818237891641\\
588	0.00313843743693856\\
589	0.00323065362889272\\
590	0.00332522743498843\\
591	0.00342286644802624\\
592	0.00352518337500009\\
593	0.00363635235366164\\
594	0.00376728283869331\\
595	0.00394659005851719\\
596	0.00424941998802836\\
597	0.00487316246495454\\
598	0.00633625614039688\\
599	0\\
600	0\\
};
\addplot [color=mycolor2,solid,forget plot]
  table[row sep=crcr]{%
1	0\\
2	0\\
3	0\\
4	0\\
5	0\\
6	0\\
7	0\\
8	0\\
9	0\\
10	0\\
11	0\\
12	0\\
13	0\\
14	0\\
15	0\\
16	0\\
17	0\\
18	0\\
19	0\\
20	0\\
21	0\\
22	0\\
23	0\\
24	0\\
25	0\\
26	0\\
27	0\\
28	0\\
29	0\\
30	0\\
31	0\\
32	0\\
33	0\\
34	0\\
35	0\\
36	0\\
37	0\\
38	0\\
39	0\\
40	0\\
41	0\\
42	0\\
43	0\\
44	0\\
45	0\\
46	0\\
47	0\\
48	0\\
49	0\\
50	0\\
51	0\\
52	0\\
53	0\\
54	0\\
55	0\\
56	0\\
57	0\\
58	0\\
59	0\\
60	0\\
61	0\\
62	0\\
63	0\\
64	0\\
65	0\\
66	0\\
67	0\\
68	0\\
69	0\\
70	0\\
71	0\\
72	0\\
73	0\\
74	0\\
75	0\\
76	0\\
77	0\\
78	0\\
79	0\\
80	0\\
81	0\\
82	0\\
83	0\\
84	0\\
85	0\\
86	0\\
87	0\\
88	0\\
89	0\\
90	0\\
91	0\\
92	0\\
93	0\\
94	0\\
95	0\\
96	0\\
97	0\\
98	0\\
99	0\\
100	0\\
101	0\\
102	0\\
103	0\\
104	0\\
105	0\\
106	0\\
107	0\\
108	0\\
109	0\\
110	0\\
111	0\\
112	0\\
113	0\\
114	0\\
115	0\\
116	0\\
117	0\\
118	0\\
119	0\\
120	0\\
121	0\\
122	0\\
123	0\\
124	0\\
125	0\\
126	0\\
127	0\\
128	0\\
129	0\\
130	0\\
131	0\\
132	0\\
133	0\\
134	0\\
135	0\\
136	0\\
137	0\\
138	0\\
139	0\\
140	0\\
141	0\\
142	0\\
143	0\\
144	0\\
145	0\\
146	0\\
147	0\\
148	0\\
149	0\\
150	0\\
151	0\\
152	0\\
153	0\\
154	0\\
155	0\\
156	0\\
157	0\\
158	0\\
159	0\\
160	0\\
161	0\\
162	0\\
163	0\\
164	0\\
165	0\\
166	0\\
167	0\\
168	0\\
169	0\\
170	0\\
171	0\\
172	0\\
173	0\\
174	0\\
175	0\\
176	0\\
177	0\\
178	0\\
179	0\\
180	0\\
181	0\\
182	0\\
183	0\\
184	0\\
185	0\\
186	0\\
187	0\\
188	0\\
189	0\\
190	0\\
191	0\\
192	0\\
193	0\\
194	0\\
195	0\\
196	0\\
197	0\\
198	0\\
199	0\\
200	0\\
201	0\\
202	0\\
203	0\\
204	0\\
205	0\\
206	0\\
207	0\\
208	0\\
209	0\\
210	0\\
211	0\\
212	0\\
213	0\\
214	0\\
215	0\\
216	0\\
217	0\\
218	0\\
219	0\\
220	0\\
221	0\\
222	0\\
223	0\\
224	0\\
225	0\\
226	0\\
227	0\\
228	0\\
229	0\\
230	0\\
231	0\\
232	0\\
233	0\\
234	0\\
235	0\\
236	0\\
237	0\\
238	0\\
239	0\\
240	0\\
241	0\\
242	0\\
243	0\\
244	0\\
245	0\\
246	0\\
247	0\\
248	0\\
249	0\\
250	0\\
251	0\\
252	0\\
253	0\\
254	0\\
255	0\\
256	0\\
257	0\\
258	0\\
259	0\\
260	0\\
261	0\\
262	0\\
263	0\\
264	0\\
265	0\\
266	0\\
267	0\\
268	0\\
269	0\\
270	0\\
271	0\\
272	0\\
273	0\\
274	0\\
275	0\\
276	0\\
277	0\\
278	0\\
279	0\\
280	0\\
281	0\\
282	0\\
283	0\\
284	0\\
285	0\\
286	0\\
287	0\\
288	0\\
289	0\\
290	0\\
291	0\\
292	0\\
293	0\\
294	0\\
295	0\\
296	0\\
297	0\\
298	0\\
299	0\\
300	0\\
301	0\\
302	0\\
303	0\\
304	0\\
305	0\\
306	0\\
307	0\\
308	0\\
309	0\\
310	0\\
311	0\\
312	0\\
313	0\\
314	0\\
315	0\\
316	0\\
317	0\\
318	0\\
319	0\\
320	0\\
321	0\\
322	0\\
323	0\\
324	0\\
325	0\\
326	0\\
327	0\\
328	0\\
329	0\\
330	0\\
331	0\\
332	0\\
333	0\\
334	0\\
335	0\\
336	0\\
337	0\\
338	0\\
339	0\\
340	0\\
341	0\\
342	0\\
343	0\\
344	0\\
345	0\\
346	0\\
347	0\\
348	0\\
349	0\\
350	0\\
351	0\\
352	0\\
353	0\\
354	0\\
355	0\\
356	0\\
357	0\\
358	0\\
359	0\\
360	0\\
361	0\\
362	0\\
363	0\\
364	0\\
365	0\\
366	0\\
367	0\\
368	0\\
369	0\\
370	0\\
371	0\\
372	0\\
373	0\\
374	0\\
375	0\\
376	0\\
377	0\\
378	0\\
379	0\\
380	0\\
381	0\\
382	0\\
383	0\\
384	0\\
385	0\\
386	0\\
387	0\\
388	0\\
389	0\\
390	0\\
391	0\\
392	0\\
393	0\\
394	0\\
395	0\\
396	0\\
397	0\\
398	0\\
399	0\\
400	0\\
401	0\\
402	0\\
403	0\\
404	0\\
405	0\\
406	0\\
407	0\\
408	0\\
409	0\\
410	0\\
411	0\\
412	0\\
413	0\\
414	0\\
415	0\\
416	0\\
417	0\\
418	0\\
419	0\\
420	0\\
421	0\\
422	0\\
423	0\\
424	0\\
425	0\\
426	0\\
427	0\\
428	0\\
429	0\\
430	0\\
431	0\\
432	0\\
433	0\\
434	0\\
435	0\\
436	0\\
437	0\\
438	0\\
439	0\\
440	0\\
441	0\\
442	0\\
443	0\\
444	0\\
445	0\\
446	0\\
447	0\\
448	0\\
449	0\\
450	0\\
451	0\\
452	0\\
453	0\\
454	0\\
455	0\\
456	0\\
457	0\\
458	0\\
459	0\\
460	0\\
461	0\\
462	0\\
463	0\\
464	0\\
465	0\\
466	0\\
467	0\\
468	0\\
469	0\\
470	0\\
471	0\\
472	0\\
473	0\\
474	0\\
475	0\\
476	0\\
477	0\\
478	0\\
479	0\\
480	0\\
481	0\\
482	0\\
483	0\\
484	0\\
485	0\\
486	0\\
487	0\\
488	0\\
489	0\\
490	0\\
491	0\\
492	0\\
493	0\\
494	0\\
495	0\\
496	0\\
497	0\\
498	0\\
499	0\\
500	0\\
501	0\\
502	0\\
503	0\\
504	0\\
505	0\\
506	0\\
507	0\\
508	0\\
509	0\\
510	0\\
511	0\\
512	0\\
513	0\\
514	0\\
515	0\\
516	0\\
517	0\\
518	0\\
519	0\\
520	0\\
521	0\\
522	0\\
523	0\\
524	0\\
525	0\\
526	0\\
527	0\\
528	0\\
529	0\\
530	0\\
531	0\\
532	0\\
533	0\\
534	0\\
535	0\\
536	0\\
537	0\\
538	0\\
539	0\\
540	1.043575262142e-05\\
541	3.76730705218496e-05\\
542	6.54984399976953e-05\\
543	9.39344174334583e-05\\
544	0.000122995254933542\\
545	0.000152692851212342\\
546	0.000183032131229788\\
547	0.000213998059291118\\
548	0.000245546201215325\\
549	0.00027781218108877\\
550	0.000310816922285056\\
551	0.000344581191504035\\
552	0.00037912545903052\\
553	0.000416165495955848\\
554	0.000457424449126694\\
555	0.000492300281589155\\
556	0.000526643713568248\\
557	0.000561320305433605\\
558	0.000596690131270986\\
559	0.000632772800261207\\
560	0.000669585437892132\\
561	0.000707145693984105\\
562	0.000745471835422497\\
563	0.000784582869067791\\
564	0.000824498821323294\\
565	0.000865241434196189\\
566	0.00113827341020856\\
567	0.0015274863410712\\
568	0.00161250300428452\\
569	0.00167689171270879\\
570	0.00174236090732779\\
571	0.00180893460520347\\
572	0.00187663774080614\\
573	0.00194549621642988\\
574	0.00201553695411676\\
575	0.00208678794948843\\
576	0.00215927832727709\\
577	0.00223303839836236\\
578	0.00230809971827687\\
579	0.00238449514780738\\
580	0.00246225891824811\\
581	0.00254142670923472\\
582	0.002622035761457\\
583	0.00270412508445197\\
584	0.00278773591889263\\
585	0.00287291287145548\\
586	0.0029597068131386\\
587	0.00304818237891639\\
588	0.00313843743693856\\
589	0.00323065362889273\\
590	0.00332522743498844\\
591	0.00342286644802626\\
592	0.00352518337500008\\
593	0.00363635235366164\\
594	0.00376728283869331\\
595	0.00394659005851718\\
596	0.00424941998802835\\
597	0.00487316246495454\\
598	0.00633625614039688\\
599	0\\
600	0\\
};
\addplot [color=mycolor3,solid,forget plot]
  table[row sep=crcr]{%
1	0\\
2	0\\
3	0\\
4	0\\
5	0\\
6	0\\
7	0\\
8	0\\
9	0\\
10	0\\
11	0\\
12	0\\
13	0\\
14	0\\
15	0\\
16	0\\
17	0\\
18	0\\
19	0\\
20	0\\
21	0\\
22	0\\
23	0\\
24	0\\
25	0\\
26	0\\
27	0\\
28	0\\
29	0\\
30	0\\
31	0\\
32	0\\
33	0\\
34	0\\
35	0\\
36	0\\
37	0\\
38	0\\
39	0\\
40	0\\
41	0\\
42	0\\
43	0\\
44	0\\
45	0\\
46	0\\
47	0\\
48	0\\
49	0\\
50	0\\
51	0\\
52	0\\
53	0\\
54	0\\
55	0\\
56	0\\
57	0\\
58	0\\
59	0\\
60	0\\
61	0\\
62	0\\
63	0\\
64	0\\
65	0\\
66	0\\
67	0\\
68	0\\
69	0\\
70	0\\
71	0\\
72	0\\
73	0\\
74	0\\
75	0\\
76	0\\
77	0\\
78	0\\
79	0\\
80	0\\
81	0\\
82	0\\
83	0\\
84	0\\
85	0\\
86	0\\
87	0\\
88	0\\
89	0\\
90	0\\
91	0\\
92	0\\
93	0\\
94	0\\
95	0\\
96	0\\
97	0\\
98	0\\
99	0\\
100	0\\
101	0\\
102	0\\
103	0\\
104	0\\
105	0\\
106	0\\
107	0\\
108	0\\
109	0\\
110	0\\
111	0\\
112	0\\
113	0\\
114	0\\
115	0\\
116	0\\
117	0\\
118	0\\
119	0\\
120	0\\
121	0\\
122	0\\
123	0\\
124	0\\
125	0\\
126	0\\
127	0\\
128	0\\
129	0\\
130	0\\
131	0\\
132	0\\
133	0\\
134	0\\
135	0\\
136	0\\
137	0\\
138	0\\
139	0\\
140	0\\
141	0\\
142	0\\
143	0\\
144	0\\
145	0\\
146	0\\
147	0\\
148	0\\
149	0\\
150	0\\
151	0\\
152	0\\
153	0\\
154	0\\
155	0\\
156	0\\
157	0\\
158	0\\
159	0\\
160	0\\
161	0\\
162	0\\
163	0\\
164	0\\
165	0\\
166	0\\
167	0\\
168	0\\
169	0\\
170	0\\
171	0\\
172	0\\
173	0\\
174	0\\
175	0\\
176	0\\
177	0\\
178	0\\
179	0\\
180	0\\
181	0\\
182	0\\
183	0\\
184	0\\
185	0\\
186	0\\
187	0\\
188	0\\
189	0\\
190	0\\
191	0\\
192	0\\
193	0\\
194	0\\
195	0\\
196	0\\
197	0\\
198	0\\
199	0\\
200	0\\
201	0\\
202	0\\
203	0\\
204	0\\
205	0\\
206	0\\
207	0\\
208	0\\
209	0\\
210	0\\
211	0\\
212	0\\
213	0\\
214	0\\
215	0\\
216	0\\
217	0\\
218	0\\
219	0\\
220	0\\
221	0\\
222	0\\
223	0\\
224	0\\
225	0\\
226	0\\
227	0\\
228	0\\
229	0\\
230	0\\
231	0\\
232	0\\
233	0\\
234	0\\
235	0\\
236	0\\
237	0\\
238	0\\
239	0\\
240	0\\
241	0\\
242	0\\
243	0\\
244	0\\
245	0\\
246	0\\
247	0\\
248	0\\
249	0\\
250	0\\
251	0\\
252	0\\
253	0\\
254	0\\
255	0\\
256	0\\
257	0\\
258	0\\
259	0\\
260	0\\
261	0\\
262	0\\
263	0\\
264	0\\
265	0\\
266	0\\
267	0\\
268	0\\
269	0\\
270	0\\
271	0\\
272	0\\
273	0\\
274	0\\
275	0\\
276	0\\
277	0\\
278	0\\
279	0\\
280	0\\
281	0\\
282	0\\
283	0\\
284	0\\
285	0\\
286	0\\
287	0\\
288	0\\
289	0\\
290	0\\
291	0\\
292	0\\
293	0\\
294	0\\
295	0\\
296	0\\
297	0\\
298	0\\
299	0\\
300	0\\
301	0\\
302	0\\
303	0\\
304	0\\
305	0\\
306	0\\
307	0\\
308	0\\
309	0\\
310	0\\
311	0\\
312	0\\
313	0\\
314	0\\
315	0\\
316	0\\
317	0\\
318	0\\
319	0\\
320	0\\
321	0\\
322	0\\
323	0\\
324	0\\
325	0\\
326	0\\
327	0\\
328	0\\
329	0\\
330	0\\
331	0\\
332	0\\
333	0\\
334	0\\
335	0\\
336	0\\
337	0\\
338	0\\
339	0\\
340	0\\
341	0\\
342	0\\
343	0\\
344	0\\
345	0\\
346	0\\
347	0\\
348	0\\
349	0\\
350	0\\
351	0\\
352	0\\
353	0\\
354	0\\
355	0\\
356	0\\
357	0\\
358	0\\
359	0\\
360	0\\
361	0\\
362	0\\
363	0\\
364	0\\
365	0\\
366	0\\
367	0\\
368	0\\
369	0\\
370	0\\
371	0\\
372	0\\
373	0\\
374	0\\
375	0\\
376	0\\
377	0\\
378	0\\
379	0\\
380	0\\
381	0\\
382	0\\
383	0\\
384	0\\
385	0\\
386	0\\
387	0\\
388	0\\
389	0\\
390	0\\
391	0\\
392	0\\
393	0\\
394	0\\
395	0\\
396	0\\
397	0\\
398	0\\
399	0\\
400	0\\
401	0\\
402	0\\
403	0\\
404	0\\
405	0\\
406	0\\
407	0\\
408	0\\
409	0\\
410	0\\
411	0\\
412	0\\
413	0\\
414	0\\
415	0\\
416	0\\
417	0\\
418	0\\
419	0\\
420	0\\
421	0\\
422	0\\
423	0\\
424	0\\
425	0\\
426	0\\
427	0\\
428	0\\
429	0\\
430	0\\
431	0\\
432	0\\
433	0\\
434	0\\
435	0\\
436	0\\
437	0\\
438	0\\
439	0\\
440	0\\
441	0\\
442	0\\
443	0\\
444	0\\
445	0\\
446	0\\
447	0\\
448	0\\
449	0\\
450	0\\
451	0\\
452	0\\
453	0\\
454	0\\
455	0\\
456	0\\
457	0\\
458	0\\
459	0\\
460	0\\
461	0\\
462	0\\
463	0\\
464	0\\
465	0\\
466	0\\
467	0\\
468	0\\
469	0\\
470	0\\
471	0\\
472	0\\
473	0\\
474	0\\
475	0\\
476	0\\
477	0\\
478	0\\
479	0\\
480	0\\
481	0\\
482	0\\
483	0\\
484	0\\
485	0\\
486	0\\
487	0\\
488	0\\
489	0\\
490	0\\
491	0\\
492	0\\
493	0\\
494	0\\
495	0\\
496	0\\
497	0\\
498	0\\
499	0\\
500	0\\
501	0\\
502	0\\
503	0\\
504	0\\
505	0\\
506	0\\
507	0\\
508	0\\
509	0\\
510	0\\
511	0\\
512	0\\
513	0\\
514	0\\
515	0\\
516	0\\
517	0\\
518	0\\
519	0\\
520	0\\
521	0\\
522	0\\
523	0\\
524	0\\
525	0\\
526	0\\
527	0\\
528	0\\
529	0\\
530	0\\
531	0\\
532	0\\
533	0\\
534	0\\
535	0\\
536	0\\
537	0\\
538	0\\
539	0\\
540	3.48797025866081e-06\\
541	3.10361975630429e-05\\
542	5.89987522129768e-05\\
543	8.73188240418785e-05\\
544	0.000116246767166137\\
545	0.000145801218191624\\
546	0.0001759952136604\\
547	0.000206821116582927\\
548	0.000238226463510221\\
549	0.000270319567015469\\
550	0.000303143125697196\\
551	0.000336716195874316\\
552	0.000371058848398407\\
553	0.00040619113801879\\
554	0.000442149506879859\\
555	0.000484062239463527\\
556	0.000523027009035641\\
557	0.000558307319211228\\
558	0.000593697063588013\\
559	0.000629724921195129\\
560	0.000666480983474951\\
561	0.000703983518743129\\
562	0.000742250808159385\\
563	0.000781301765229793\\
564	0.000821156099512044\\
565	0.00086183464309347\\
566	0.000903359467058362\\
567	0.00124194869524459\\
568	0.0015912465992359\\
569	0.00167689170233861\\
570	0.00174236090712023\\
571	0.00180893460512034\\
572	0.00187663774076567\\
573	0.00194549621641082\\
574	0.00201553695410846\\
575	0.00208678794948483\\
576	0.00215927832727575\\
577	0.00223303839836177\\
578	0.00230809971827685\\
579	0.00238449514780736\\
580	0.00246225891824812\\
581	0.00254142670923474\\
582	0.00262203576145702\\
583	0.00270412508445197\\
584	0.00278773591889264\\
585	0.00287291287145547\\
586	0.00295970681313861\\
587	0.00304818237891642\\
588	0.00313843743693858\\
589	0.00323065362889276\\
590	0.00332522743498845\\
591	0.00342286644802627\\
592	0.00352518337500009\\
593	0.00363635235366166\\
594	0.00376728283869333\\
595	0.0039465900585172\\
596	0.00424941998802837\\
597	0.00487316246495455\\
598	0.00633625614039688\\
599	0\\
600	0\\
};
\addplot [color=mycolor4,solid,forget plot]
  table[row sep=crcr]{%
1	0\\
2	0\\
3	0\\
4	0\\
5	0\\
6	0\\
7	0\\
8	0\\
9	0\\
10	0\\
11	0\\
12	0\\
13	0\\
14	0\\
15	0\\
16	0\\
17	0\\
18	0\\
19	0\\
20	0\\
21	0\\
22	0\\
23	0\\
24	0\\
25	0\\
26	0\\
27	0\\
28	0\\
29	0\\
30	0\\
31	0\\
32	0\\
33	0\\
34	0\\
35	0\\
36	0\\
37	0\\
38	0\\
39	0\\
40	0\\
41	0\\
42	0\\
43	0\\
44	0\\
45	0\\
46	0\\
47	0\\
48	0\\
49	0\\
50	0\\
51	0\\
52	0\\
53	0\\
54	0\\
55	0\\
56	0\\
57	0\\
58	0\\
59	0\\
60	0\\
61	0\\
62	0\\
63	0\\
64	0\\
65	0\\
66	0\\
67	0\\
68	0\\
69	0\\
70	0\\
71	0\\
72	0\\
73	0\\
74	0\\
75	0\\
76	0\\
77	0\\
78	0\\
79	0\\
80	0\\
81	0\\
82	0\\
83	0\\
84	0\\
85	0\\
86	0\\
87	0\\
88	0\\
89	0\\
90	0\\
91	0\\
92	0\\
93	0\\
94	0\\
95	0\\
96	0\\
97	0\\
98	0\\
99	0\\
100	0\\
101	0\\
102	0\\
103	0\\
104	0\\
105	0\\
106	0\\
107	0\\
108	0\\
109	0\\
110	0\\
111	0\\
112	0\\
113	0\\
114	0\\
115	0\\
116	0\\
117	0\\
118	0\\
119	0\\
120	0\\
121	0\\
122	0\\
123	0\\
124	0\\
125	0\\
126	0\\
127	0\\
128	0\\
129	0\\
130	0\\
131	0\\
132	0\\
133	0\\
134	0\\
135	0\\
136	0\\
137	0\\
138	0\\
139	0\\
140	0\\
141	0\\
142	0\\
143	0\\
144	0\\
145	0\\
146	0\\
147	0\\
148	0\\
149	0\\
150	0\\
151	0\\
152	0\\
153	0\\
154	0\\
155	0\\
156	0\\
157	0\\
158	0\\
159	0\\
160	0\\
161	0\\
162	0\\
163	0\\
164	0\\
165	0\\
166	0\\
167	0\\
168	0\\
169	0\\
170	0\\
171	0\\
172	0\\
173	0\\
174	0\\
175	0\\
176	0\\
177	0\\
178	0\\
179	0\\
180	0\\
181	0\\
182	0\\
183	0\\
184	0\\
185	0\\
186	0\\
187	0\\
188	0\\
189	0\\
190	0\\
191	0\\
192	0\\
193	0\\
194	0\\
195	0\\
196	0\\
197	0\\
198	0\\
199	0\\
200	0\\
201	0\\
202	0\\
203	0\\
204	0\\
205	0\\
206	0\\
207	0\\
208	0\\
209	0\\
210	0\\
211	0\\
212	0\\
213	0\\
214	0\\
215	0\\
216	0\\
217	0\\
218	0\\
219	0\\
220	0\\
221	0\\
222	0\\
223	0\\
224	0\\
225	0\\
226	0\\
227	0\\
228	0\\
229	0\\
230	0\\
231	0\\
232	0\\
233	0\\
234	0\\
235	0\\
236	0\\
237	0\\
238	0\\
239	0\\
240	0\\
241	0\\
242	0\\
243	0\\
244	0\\
245	0\\
246	0\\
247	0\\
248	0\\
249	0\\
250	0\\
251	0\\
252	0\\
253	0\\
254	0\\
255	0\\
256	0\\
257	0\\
258	0\\
259	0\\
260	0\\
261	0\\
262	0\\
263	0\\
264	0\\
265	0\\
266	0\\
267	0\\
268	0\\
269	0\\
270	0\\
271	0\\
272	0\\
273	0\\
274	0\\
275	0\\
276	0\\
277	0\\
278	0\\
279	0\\
280	0\\
281	0\\
282	0\\
283	0\\
284	0\\
285	0\\
286	0\\
287	0\\
288	0\\
289	0\\
290	0\\
291	0\\
292	0\\
293	0\\
294	0\\
295	0\\
296	0\\
297	0\\
298	0\\
299	0\\
300	0\\
301	0\\
302	0\\
303	0\\
304	0\\
305	0\\
306	0\\
307	0\\
308	0\\
309	0\\
310	0\\
311	0\\
312	0\\
313	0\\
314	0\\
315	0\\
316	0\\
317	0\\
318	0\\
319	0\\
320	0\\
321	0\\
322	0\\
323	0\\
324	0\\
325	0\\
326	0\\
327	0\\
328	0\\
329	0\\
330	0\\
331	0\\
332	0\\
333	0\\
334	0\\
335	0\\
336	0\\
337	0\\
338	0\\
339	0\\
340	0\\
341	0\\
342	0\\
343	0\\
344	0\\
345	0\\
346	0\\
347	0\\
348	0\\
349	0\\
350	0\\
351	0\\
352	0\\
353	0\\
354	0\\
355	0\\
356	0\\
357	0\\
358	0\\
359	0\\
360	0\\
361	0\\
362	0\\
363	0\\
364	0\\
365	0\\
366	0\\
367	0\\
368	0\\
369	0\\
370	0\\
371	0\\
372	0\\
373	0\\
374	0\\
375	0\\
376	0\\
377	0\\
378	0\\
379	0\\
380	0\\
381	0\\
382	0\\
383	0\\
384	0\\
385	0\\
386	0\\
387	0\\
388	0\\
389	0\\
390	0\\
391	0\\
392	0\\
393	0\\
394	0\\
395	0\\
396	0\\
397	0\\
398	0\\
399	0\\
400	0\\
401	0\\
402	0\\
403	0\\
404	0\\
405	0\\
406	0\\
407	0\\
408	0\\
409	0\\
410	0\\
411	0\\
412	0\\
413	0\\
414	0\\
415	0\\
416	0\\
417	0\\
418	0\\
419	0\\
420	0\\
421	0\\
422	0\\
423	0\\
424	0\\
425	0\\
426	0\\
427	0\\
428	0\\
429	0\\
430	0\\
431	0\\
432	0\\
433	0\\
434	0\\
435	0\\
436	0\\
437	0\\
438	0\\
439	0\\
440	0\\
441	0\\
442	0\\
443	0\\
444	0\\
445	0\\
446	0\\
447	0\\
448	0\\
449	0\\
450	0\\
451	0\\
452	0\\
453	0\\
454	0\\
455	0\\
456	0\\
457	0\\
458	0\\
459	0\\
460	0\\
461	0\\
462	0\\
463	0\\
464	0\\
465	0\\
466	0\\
467	0\\
468	0\\
469	0\\
470	0\\
471	0\\
472	0\\
473	0\\
474	0\\
475	0\\
476	0\\
477	0\\
478	0\\
479	0\\
480	0\\
481	0\\
482	0\\
483	0\\
484	0\\
485	0\\
486	0\\
487	0\\
488	0\\
489	0\\
490	0\\
491	0\\
492	0\\
493	0\\
494	0\\
495	0\\
496	0\\
497	0\\
498	0\\
499	0\\
500	0\\
501	0\\
502	0\\
503	0\\
504	0\\
505	0\\
506	0\\
507	0\\
508	0\\
509	0\\
510	0\\
511	0\\
512	0\\
513	0\\
514	0\\
515	0\\
516	0\\
517	0\\
518	0\\
519	0\\
520	0\\
521	0\\
522	0\\
523	0\\
524	0\\
525	0\\
526	0\\
527	0\\
528	0\\
529	0\\
530	0\\
531	0\\
532	0\\
533	0\\
534	0\\
535	0\\
536	0\\
537	0\\
538	0\\
539	0\\
540	0\\
541	1.96468212964802e-05\\
542	5.09270591881137e-05\\
543	7.96678779271732e-05\\
544	0.000108857158611573\\
545	0.000138469423037538\\
546	0.000168547998253239\\
547	0.000199251293832798\\
548	0.000230546998687823\\
549	0.000262493736100757\\
550	0.000295169201149189\\
551	0.000328592928239676\\
552	0.000362785127969851\\
553	0.000397767066658495\\
554	0.000433561181336885\\
555	0.000470189197593691\\
556	0.000510120726160936\\
557	0.000553077463148308\\
558	0.00058994417211552\\
559	0.000626359984799716\\
560	0.000663065711723231\\
561	0.000700504825484549\\
562	0.00073870671292666\\
563	0.00077769066007184\\
564	0.000817476258749015\\
565	0.000858083918159465\\
566	0.000899535269842526\\
567	0.000941852816480089\\
568	0.00131032407389147\\
569	0.00165300818929824\\
570	0.00174236067392589\\
571	0.00180893460337395\\
572	0.00187663774017175\\
573	0.00194549621611769\\
574	0.00201553695396648\\
575	0.0020867879494197\\
576	0.0021592783272479\\
577	0.00223303839835099\\
578	0.00230809971827289\\
579	0.00238449514780614\\
580	0.00246225891824759\\
581	0.00254142670923473\\
582	0.00262203576145702\\
583	0.00270412508445197\\
584	0.00278773591889265\\
585	0.00287291287145549\\
586	0.00295970681313861\\
587	0.00304818237891641\\
588	0.00313843743693858\\
589	0.00323065362889274\\
590	0.00332522743498846\\
591	0.00342286644802626\\
592	0.0035251833750001\\
593	0.00363635235366166\\
594	0.00376728283869333\\
595	0.0039465900585172\\
596	0.00424941998802838\\
597	0.00487316246495455\\
598	0.00633625614039688\\
599	0\\
600	0\\
};
\addplot [color=mycolor5,solid,forget plot]
  table[row sep=crcr]{%
1	0\\
2	0\\
3	0\\
4	0\\
5	0\\
6	0\\
7	0\\
8	0\\
9	0\\
10	0\\
11	0\\
12	0\\
13	0\\
14	0\\
15	0\\
16	0\\
17	0\\
18	0\\
19	0\\
20	0\\
21	0\\
22	0\\
23	0\\
24	0\\
25	0\\
26	0\\
27	0\\
28	0\\
29	0\\
30	0\\
31	0\\
32	0\\
33	0\\
34	0\\
35	0\\
36	0\\
37	0\\
38	0\\
39	0\\
40	0\\
41	0\\
42	0\\
43	0\\
44	0\\
45	0\\
46	0\\
47	0\\
48	0\\
49	0\\
50	0\\
51	0\\
52	0\\
53	0\\
54	0\\
55	0\\
56	0\\
57	0\\
58	0\\
59	0\\
60	0\\
61	0\\
62	0\\
63	0\\
64	0\\
65	0\\
66	0\\
67	0\\
68	0\\
69	0\\
70	0\\
71	0\\
72	0\\
73	0\\
74	0\\
75	0\\
76	0\\
77	0\\
78	0\\
79	0\\
80	0\\
81	0\\
82	0\\
83	0\\
84	0\\
85	0\\
86	0\\
87	0\\
88	0\\
89	0\\
90	0\\
91	0\\
92	0\\
93	0\\
94	0\\
95	0\\
96	0\\
97	0\\
98	0\\
99	0\\
100	0\\
101	0\\
102	0\\
103	0\\
104	0\\
105	0\\
106	0\\
107	0\\
108	0\\
109	0\\
110	0\\
111	0\\
112	0\\
113	0\\
114	0\\
115	0\\
116	0\\
117	0\\
118	0\\
119	0\\
120	0\\
121	0\\
122	0\\
123	0\\
124	0\\
125	0\\
126	0\\
127	0\\
128	0\\
129	0\\
130	0\\
131	0\\
132	0\\
133	0\\
134	0\\
135	0\\
136	0\\
137	0\\
138	0\\
139	0\\
140	0\\
141	0\\
142	0\\
143	0\\
144	0\\
145	0\\
146	0\\
147	0\\
148	0\\
149	0\\
150	0\\
151	0\\
152	0\\
153	0\\
154	0\\
155	0\\
156	0\\
157	0\\
158	0\\
159	0\\
160	0\\
161	0\\
162	0\\
163	0\\
164	0\\
165	0\\
166	0\\
167	0\\
168	0\\
169	0\\
170	0\\
171	0\\
172	0\\
173	0\\
174	0\\
175	0\\
176	0\\
177	0\\
178	0\\
179	0\\
180	0\\
181	0\\
182	0\\
183	0\\
184	0\\
185	0\\
186	0\\
187	0\\
188	0\\
189	0\\
190	0\\
191	0\\
192	0\\
193	0\\
194	0\\
195	0\\
196	0\\
197	0\\
198	0\\
199	0\\
200	0\\
201	0\\
202	0\\
203	0\\
204	0\\
205	0\\
206	0\\
207	0\\
208	0\\
209	0\\
210	0\\
211	0\\
212	0\\
213	0\\
214	0\\
215	0\\
216	0\\
217	0\\
218	0\\
219	0\\
220	0\\
221	0\\
222	0\\
223	0\\
224	0\\
225	0\\
226	0\\
227	0\\
228	0\\
229	0\\
230	0\\
231	0\\
232	0\\
233	0\\
234	0\\
235	0\\
236	0\\
237	0\\
238	0\\
239	0\\
240	0\\
241	0\\
242	0\\
243	0\\
244	0\\
245	0\\
246	0\\
247	0\\
248	0\\
249	0\\
250	0\\
251	0\\
252	0\\
253	0\\
254	0\\
255	0\\
256	0\\
257	0\\
258	0\\
259	0\\
260	0\\
261	0\\
262	0\\
263	0\\
264	0\\
265	0\\
266	0\\
267	0\\
268	0\\
269	0\\
270	0\\
271	0\\
272	0\\
273	0\\
274	0\\
275	0\\
276	0\\
277	0\\
278	0\\
279	0\\
280	0\\
281	0\\
282	0\\
283	0\\
284	0\\
285	0\\
286	0\\
287	0\\
288	0\\
289	0\\
290	0\\
291	0\\
292	0\\
293	0\\
294	0\\
295	0\\
296	0\\
297	0\\
298	0\\
299	0\\
300	0\\
301	0\\
302	0\\
303	0\\
304	0\\
305	0\\
306	0\\
307	0\\
308	0\\
309	0\\
310	0\\
311	0\\
312	0\\
313	0\\
314	0\\
315	0\\
316	0\\
317	0\\
318	0\\
319	0\\
320	0\\
321	0\\
322	0\\
323	0\\
324	0\\
325	0\\
326	0\\
327	0\\
328	0\\
329	0\\
330	0\\
331	0\\
332	0\\
333	0\\
334	0\\
335	0\\
336	0\\
337	0\\
338	0\\
339	0\\
340	0\\
341	0\\
342	0\\
343	0\\
344	0\\
345	0\\
346	0\\
347	0\\
348	0\\
349	0\\
350	0\\
351	0\\
352	0\\
353	0\\
354	0\\
355	0\\
356	0\\
357	0\\
358	0\\
359	0\\
360	0\\
361	0\\
362	0\\
363	0\\
364	0\\
365	0\\
366	0\\
367	0\\
368	0\\
369	0\\
370	0\\
371	0\\
372	0\\
373	0\\
374	0\\
375	0\\
376	0\\
377	0\\
378	0\\
379	0\\
380	0\\
381	0\\
382	0\\
383	0\\
384	0\\
385	0\\
386	0\\
387	0\\
388	0\\
389	0\\
390	0\\
391	0\\
392	0\\
393	0\\
394	0\\
395	0\\
396	0\\
397	0\\
398	0\\
399	0\\
400	0\\
401	0\\
402	0\\
403	0\\
404	0\\
405	0\\
406	0\\
407	0\\
408	0\\
409	0\\
410	0\\
411	0\\
412	0\\
413	0\\
414	0\\
415	0\\
416	0\\
417	0\\
418	0\\
419	0\\
420	0\\
421	0\\
422	0\\
423	0\\
424	0\\
425	0\\
426	0\\
427	0\\
428	0\\
429	0\\
430	0\\
431	0\\
432	0\\
433	0\\
434	0\\
435	0\\
436	0\\
437	0\\
438	0\\
439	0\\
440	0\\
441	0\\
442	0\\
443	0\\
444	0\\
445	0\\
446	0\\
447	0\\
448	0\\
449	0\\
450	0\\
451	0\\
452	0\\
453	0\\
454	0\\
455	0\\
456	0\\
457	0\\
458	0\\
459	0\\
460	0\\
461	0\\
462	0\\
463	0\\
464	0\\
465	0\\
466	0\\
467	0\\
468	0\\
469	0\\
470	0\\
471	0\\
472	0\\
473	0\\
474	0\\
475	0\\
476	0\\
477	0\\
478	0\\
479	0\\
480	0\\
481	0\\
482	0\\
483	0\\
484	0\\
485	0\\
486	0\\
487	0\\
488	0\\
489	0\\
490	0\\
491	0\\
492	0\\
493	0\\
494	0\\
495	0\\
496	0\\
497	0\\
498	0\\
499	0\\
500	0\\
501	0\\
502	0\\
503	0\\
504	0\\
505	0\\
506	0\\
507	0\\
508	0\\
509	0\\
510	0\\
511	0\\
512	0\\
513	0\\
514	0\\
515	0\\
516	0\\
517	0\\
518	0\\
519	0\\
520	0\\
521	0\\
522	0\\
523	0\\
524	0\\
525	0\\
526	0\\
527	0\\
528	0\\
529	0\\
530	0\\
531	0\\
532	0\\
533	0\\
534	0\\
535	0\\
536	0\\
537	0\\
538	0\\
539	0\\
540	0\\
541	1.60435556964414e-06\\
542	3.14302111237603e-05\\
543	6.44583727469957e-05\\
544	9.74389278759319e-05\\
545	0.000129414111585044\\
546	0.000159861388438848\\
547	0.000190773274010441\\
548	0.000222067607061106\\
549	0.000253847438959652\\
550	0.000286336225299031\\
551	0.000319561371325126\\
552	0.00035354558130088\\
553	0.000388309508804606\\
554	0.000423874256441935\\
555	0.000460261953488801\\
556	0.000497494185660627\\
557	0.000535616231024054\\
558	0.000579553452092161\\
559	0.000621594147079938\\
560	0.000659031655417829\\
561	0.000696686984170281\\
562	0.000734828044981065\\
563	0.000773741639529971\\
564	0.000813454544992937\\
565	0.000853987307148437\\
566	0.000895361001406296\\
567	0.000937597950788636\\
568	0.000980721094541498\\
569	0.00134222542701616\\
570	0.00171269757187565\\
571	0.00180892766513185\\
572	0.00187663772269433\\
573	0.00194549621194972\\
574	0.00201553695190381\\
575	0.0020867879483933\\
576	0.00215927832676105\\
577	0.00223303839813513\\
578	0.0023080997181849\\
579	0.00238449514777378\\
580	0.00246225891823735\\
581	0.00254142670923183\\
582	0.00262203576145638\\
583	0.00270412508445196\\
584	0.00278773591889265\\
585	0.00287291287145547\\
586	0.00295970681313861\\
587	0.00304818237891641\\
588	0.00313843743693857\\
589	0.00323065362889274\\
590	0.00332522743498845\\
591	0.00342286644802626\\
592	0.00352518337500009\\
593	0.00363635235366165\\
594	0.00376728283869332\\
595	0.00394659005851719\\
596	0.00424941998802836\\
597	0.00487316246495455\\
598	0.00633625614039688\\
599	0\\
600	0\\
};
\addplot [color=mycolor6,solid,forget plot]
  table[row sep=crcr]{%
1	0\\
2	0\\
3	0\\
4	0\\
5	0\\
6	0\\
7	0\\
8	0\\
9	0\\
10	0\\
11	0\\
12	0\\
13	0\\
14	0\\
15	0\\
16	0\\
17	0\\
18	0\\
19	0\\
20	0\\
21	0\\
22	0\\
23	0\\
24	0\\
25	0\\
26	0\\
27	0\\
28	0\\
29	0\\
30	0\\
31	0\\
32	0\\
33	0\\
34	0\\
35	0\\
36	0\\
37	0\\
38	0\\
39	0\\
40	0\\
41	0\\
42	0\\
43	0\\
44	0\\
45	0\\
46	0\\
47	0\\
48	0\\
49	0\\
50	0\\
51	0\\
52	0\\
53	0\\
54	0\\
55	0\\
56	0\\
57	0\\
58	0\\
59	0\\
60	0\\
61	0\\
62	0\\
63	0\\
64	0\\
65	0\\
66	0\\
67	0\\
68	0\\
69	0\\
70	0\\
71	0\\
72	0\\
73	0\\
74	0\\
75	0\\
76	0\\
77	0\\
78	0\\
79	0\\
80	0\\
81	0\\
82	0\\
83	0\\
84	0\\
85	0\\
86	0\\
87	0\\
88	0\\
89	0\\
90	0\\
91	0\\
92	0\\
93	0\\
94	0\\
95	0\\
96	0\\
97	0\\
98	0\\
99	0\\
100	0\\
101	0\\
102	0\\
103	0\\
104	0\\
105	0\\
106	0\\
107	0\\
108	0\\
109	0\\
110	0\\
111	0\\
112	0\\
113	0\\
114	0\\
115	0\\
116	0\\
117	0\\
118	0\\
119	0\\
120	0\\
121	0\\
122	0\\
123	0\\
124	0\\
125	0\\
126	0\\
127	0\\
128	0\\
129	0\\
130	0\\
131	0\\
132	0\\
133	0\\
134	0\\
135	0\\
136	0\\
137	0\\
138	0\\
139	0\\
140	0\\
141	0\\
142	0\\
143	0\\
144	0\\
145	0\\
146	0\\
147	0\\
148	0\\
149	0\\
150	0\\
151	0\\
152	0\\
153	0\\
154	0\\
155	0\\
156	0\\
157	0\\
158	0\\
159	0\\
160	0\\
161	0\\
162	0\\
163	0\\
164	0\\
165	0\\
166	0\\
167	0\\
168	0\\
169	0\\
170	0\\
171	0\\
172	0\\
173	0\\
174	0\\
175	0\\
176	0\\
177	0\\
178	0\\
179	0\\
180	0\\
181	0\\
182	0\\
183	0\\
184	0\\
185	0\\
186	0\\
187	0\\
188	0\\
189	0\\
190	0\\
191	0\\
192	0\\
193	0\\
194	0\\
195	0\\
196	0\\
197	0\\
198	0\\
199	0\\
200	0\\
201	0\\
202	0\\
203	0\\
204	0\\
205	0\\
206	0\\
207	0\\
208	0\\
209	0\\
210	0\\
211	0\\
212	0\\
213	0\\
214	0\\
215	0\\
216	0\\
217	0\\
218	0\\
219	0\\
220	0\\
221	0\\
222	0\\
223	0\\
224	0\\
225	0\\
226	0\\
227	0\\
228	0\\
229	0\\
230	0\\
231	0\\
232	0\\
233	0\\
234	0\\
235	0\\
236	0\\
237	0\\
238	0\\
239	0\\
240	0\\
241	0\\
242	0\\
243	0\\
244	0\\
245	0\\
246	0\\
247	0\\
248	0\\
249	0\\
250	0\\
251	0\\
252	0\\
253	0\\
254	0\\
255	0\\
256	0\\
257	0\\
258	0\\
259	0\\
260	0\\
261	0\\
262	0\\
263	0\\
264	0\\
265	0\\
266	0\\
267	0\\
268	0\\
269	0\\
270	0\\
271	0\\
272	0\\
273	0\\
274	0\\
275	0\\
276	0\\
277	0\\
278	0\\
279	0\\
280	0\\
281	0\\
282	0\\
283	0\\
284	0\\
285	0\\
286	0\\
287	0\\
288	0\\
289	0\\
290	0\\
291	0\\
292	0\\
293	0\\
294	0\\
295	0\\
296	0\\
297	0\\
298	0\\
299	0\\
300	0\\
301	0\\
302	0\\
303	0\\
304	0\\
305	0\\
306	0\\
307	0\\
308	0\\
309	0\\
310	0\\
311	0\\
312	0\\
313	0\\
314	0\\
315	0\\
316	0\\
317	0\\
318	0\\
319	0\\
320	0\\
321	0\\
322	0\\
323	0\\
324	0\\
325	0\\
326	0\\
327	0\\
328	0\\
329	0\\
330	0\\
331	0\\
332	0\\
333	0\\
334	0\\
335	0\\
336	0\\
337	0\\
338	0\\
339	0\\
340	0\\
341	0\\
342	0\\
343	0\\
344	0\\
345	0\\
346	0\\
347	0\\
348	0\\
349	0\\
350	0\\
351	0\\
352	0\\
353	0\\
354	0\\
355	0\\
356	0\\
357	0\\
358	0\\
359	0\\
360	0\\
361	0\\
362	0\\
363	0\\
364	0\\
365	0\\
366	0\\
367	0\\
368	0\\
369	0\\
370	0\\
371	0\\
372	0\\
373	0\\
374	0\\
375	0\\
376	0\\
377	0\\
378	0\\
379	0\\
380	0\\
381	0\\
382	0\\
383	0\\
384	0\\
385	0\\
386	0\\
387	0\\
388	0\\
389	0\\
390	0\\
391	0\\
392	0\\
393	0\\
394	0\\
395	0\\
396	0\\
397	0\\
398	0\\
399	0\\
400	0\\
401	0\\
402	0\\
403	0\\
404	0\\
405	0\\
406	0\\
407	0\\
408	0\\
409	0\\
410	0\\
411	0\\
412	0\\
413	0\\
414	0\\
415	0\\
416	0\\
417	0\\
418	0\\
419	0\\
420	0\\
421	0\\
422	0\\
423	0\\
424	0\\
425	0\\
426	0\\
427	0\\
428	0\\
429	0\\
430	0\\
431	0\\
432	0\\
433	0\\
434	0\\
435	0\\
436	0\\
437	0\\
438	0\\
439	0\\
440	0\\
441	0\\
442	0\\
443	0\\
444	0\\
445	0\\
446	0\\
447	0\\
448	0\\
449	0\\
450	0\\
451	0\\
452	0\\
453	0\\
454	0\\
455	0\\
456	0\\
457	0\\
458	0\\
459	0\\
460	0\\
461	0\\
462	0\\
463	0\\
464	0\\
465	0\\
466	0\\
467	0\\
468	0\\
469	0\\
470	0\\
471	0\\
472	0\\
473	0\\
474	0\\
475	0\\
476	0\\
477	0\\
478	0\\
479	0\\
480	0\\
481	0\\
482	0\\
483	0\\
484	0\\
485	0\\
486	0\\
487	0\\
488	0\\
489	0\\
490	0\\
491	0\\
492	0\\
493	0\\
494	0\\
495	0\\
496	0\\
497	0\\
498	0\\
499	0\\
500	0\\
501	0\\
502	0\\
503	0\\
504	0\\
505	0\\
506	0\\
507	0\\
508	0\\
509	0\\
510	0\\
511	0\\
512	0\\
513	0\\
514	0\\
515	0\\
516	0\\
517	0\\
518	0\\
519	0\\
520	0\\
521	0\\
522	0\\
523	0\\
524	0\\
525	0\\
526	0\\
527	0\\
528	0\\
529	0\\
530	0\\
531	0\\
532	0\\
533	0\\
534	0\\
535	0\\
536	0\\
537	0\\
538	0\\
539	0\\
540	0\\
541	0\\
542	1.63690351718187e-05\\
543	4.64318723006276e-05\\
544	7.72064183891249e-05\\
545	0.000109681172498384\\
546	0.000144326319315444\\
547	0.000178940796521601\\
548	0.000211991301117044\\
549	0.00024412624604269\\
550	0.000276785482440831\\
551	0.000309938580594202\\
552	0.000343753073709177\\
553	0.000378330150709172\\
554	0.000413698699966639\\
555	0.000449883738139326\\
556	0.000486908539075665\\
557	0.000524796936544112\\
558	0.000563571508554637\\
559	0.00060474487388057\\
560	0.000650150252615914\\
561	0.000691569997482022\\
562	0.000730363569551948\\
563	0.000769387350818147\\
564	0.000809034584368206\\
565	0.000849493623747197\\
566	0.000890791349062815\\
567	0.000932949900077076\\
568	0.00097599221169323\\
569	0.00101994155476744\\
570	0.00133660911702074\\
571	0.00177027877590743\\
572	0.00187639737504145\\
573	0.00194549595575903\\
574	0.00201553692275095\\
575	0.0020867879342917\\
576	0.00215927831957446\\
577	0.00223303839460999\\
578	0.00230809971656243\\
579	0.00238449514708515\\
580	0.00246225891797304\\
581	0.00254142670914249\\
582	0.0026220357614308\\
583	0.00270412508444604\\
584	0.00278773591889168\\
585	0.00287291287145548\\
586	0.00295970681313861\\
587	0.0030481823789164\\
588	0.00313843743693857\\
589	0.00323065362889274\\
590	0.00332522743498845\\
591	0.00342286644802626\\
592	0.00352518337500009\\
593	0.00363635235366165\\
594	0.00376728283869332\\
595	0.00394659005851719\\
596	0.00424941998802836\\
597	0.00487316246495455\\
598	0.00633625614039688\\
599	0\\
600	0\\
};
\addplot [color=mycolor7,solid,forget plot]
  table[row sep=crcr]{%
1	0\\
2	0\\
3	0\\
4	0\\
5	0\\
6	0\\
7	0\\
8	0\\
9	0\\
10	0\\
11	0\\
12	0\\
13	0\\
14	0\\
15	0\\
16	0\\
17	0\\
18	0\\
19	0\\
20	0\\
21	0\\
22	0\\
23	0\\
24	0\\
25	0\\
26	0\\
27	0\\
28	0\\
29	0\\
30	0\\
31	0\\
32	0\\
33	0\\
34	0\\
35	0\\
36	0\\
37	0\\
38	0\\
39	0\\
40	0\\
41	0\\
42	0\\
43	0\\
44	0\\
45	0\\
46	0\\
47	0\\
48	0\\
49	0\\
50	0\\
51	0\\
52	0\\
53	0\\
54	0\\
55	0\\
56	0\\
57	0\\
58	0\\
59	0\\
60	0\\
61	0\\
62	0\\
63	0\\
64	0\\
65	0\\
66	0\\
67	0\\
68	0\\
69	0\\
70	0\\
71	0\\
72	0\\
73	0\\
74	0\\
75	0\\
76	0\\
77	0\\
78	0\\
79	0\\
80	0\\
81	0\\
82	0\\
83	0\\
84	0\\
85	0\\
86	0\\
87	0\\
88	0\\
89	0\\
90	0\\
91	0\\
92	0\\
93	0\\
94	0\\
95	0\\
96	0\\
97	0\\
98	0\\
99	0\\
100	0\\
101	0\\
102	0\\
103	0\\
104	0\\
105	0\\
106	0\\
107	0\\
108	0\\
109	0\\
110	0\\
111	0\\
112	0\\
113	0\\
114	0\\
115	0\\
116	0\\
117	0\\
118	0\\
119	0\\
120	0\\
121	0\\
122	0\\
123	0\\
124	0\\
125	0\\
126	0\\
127	0\\
128	0\\
129	0\\
130	0\\
131	0\\
132	0\\
133	0\\
134	0\\
135	0\\
136	0\\
137	0\\
138	0\\
139	0\\
140	0\\
141	0\\
142	0\\
143	0\\
144	0\\
145	0\\
146	0\\
147	0\\
148	0\\
149	0\\
150	0\\
151	0\\
152	0\\
153	0\\
154	0\\
155	0\\
156	0\\
157	0\\
158	0\\
159	0\\
160	0\\
161	0\\
162	0\\
163	0\\
164	0\\
165	0\\
166	0\\
167	0\\
168	0\\
169	0\\
170	0\\
171	0\\
172	0\\
173	0\\
174	0\\
175	0\\
176	0\\
177	0\\
178	0\\
179	0\\
180	0\\
181	0\\
182	0\\
183	0\\
184	0\\
185	0\\
186	0\\
187	0\\
188	0\\
189	0\\
190	0\\
191	0\\
192	0\\
193	0\\
194	0\\
195	0\\
196	0\\
197	0\\
198	0\\
199	0\\
200	0\\
201	0\\
202	0\\
203	0\\
204	0\\
205	0\\
206	0\\
207	0\\
208	0\\
209	0\\
210	0\\
211	0\\
212	0\\
213	0\\
214	0\\
215	0\\
216	0\\
217	0\\
218	0\\
219	0\\
220	0\\
221	0\\
222	0\\
223	0\\
224	0\\
225	0\\
226	0\\
227	0\\
228	0\\
229	0\\
230	0\\
231	0\\
232	0\\
233	0\\
234	0\\
235	0\\
236	0\\
237	0\\
238	0\\
239	0\\
240	0\\
241	0\\
242	0\\
243	0\\
244	0\\
245	0\\
246	0\\
247	0\\
248	0\\
249	0\\
250	0\\
251	0\\
252	0\\
253	0\\
254	0\\
255	0\\
256	0\\
257	0\\
258	0\\
259	0\\
260	0\\
261	0\\
262	0\\
263	0\\
264	0\\
265	0\\
266	0\\
267	0\\
268	0\\
269	0\\
270	0\\
271	0\\
272	0\\
273	0\\
274	0\\
275	0\\
276	0\\
277	0\\
278	0\\
279	0\\
280	0\\
281	0\\
282	0\\
283	0\\
284	0\\
285	0\\
286	0\\
287	0\\
288	0\\
289	0\\
290	0\\
291	0\\
292	0\\
293	0\\
294	0\\
295	0\\
296	0\\
297	0\\
298	0\\
299	0\\
300	0\\
301	0\\
302	0\\
303	0\\
304	0\\
305	0\\
306	0\\
307	0\\
308	0\\
309	0\\
310	0\\
311	0\\
312	0\\
313	0\\
314	0\\
315	0\\
316	0\\
317	0\\
318	0\\
319	0\\
320	0\\
321	0\\
322	0\\
323	0\\
324	0\\
325	0\\
326	0\\
327	0\\
328	0\\
329	0\\
330	0\\
331	0\\
332	0\\
333	0\\
334	0\\
335	0\\
336	0\\
337	0\\
338	0\\
339	0\\
340	0\\
341	0\\
342	0\\
343	0\\
344	0\\
345	0\\
346	0\\
347	0\\
348	0\\
349	0\\
350	0\\
351	0\\
352	0\\
353	0\\
354	0\\
355	0\\
356	0\\
357	0\\
358	0\\
359	0\\
360	0\\
361	0\\
362	0\\
363	0\\
364	0\\
365	0\\
366	0\\
367	0\\
368	0\\
369	0\\
370	0\\
371	0\\
372	0\\
373	0\\
374	0\\
375	0\\
376	0\\
377	0\\
378	0\\
379	0\\
380	0\\
381	0\\
382	0\\
383	0\\
384	0\\
385	0\\
386	0\\
387	0\\
388	0\\
389	0\\
390	0\\
391	0\\
392	0\\
393	0\\
394	0\\
395	0\\
396	0\\
397	0\\
398	0\\
399	0\\
400	0\\
401	0\\
402	0\\
403	0\\
404	0\\
405	0\\
406	0\\
407	0\\
408	0\\
409	0\\
410	0\\
411	0\\
412	0\\
413	0\\
414	0\\
415	0\\
416	0\\
417	0\\
418	0\\
419	0\\
420	0\\
421	0\\
422	0\\
423	0\\
424	0\\
425	0\\
426	0\\
427	0\\
428	0\\
429	0\\
430	0\\
431	0\\
432	0\\
433	0\\
434	0\\
435	0\\
436	0\\
437	0\\
438	0\\
439	0\\
440	0\\
441	0\\
442	0\\
443	0\\
444	0\\
445	0\\
446	0\\
447	0\\
448	0\\
449	0\\
450	0\\
451	0\\
452	0\\
453	0\\
454	0\\
455	0\\
456	0\\
457	0\\
458	0\\
459	0\\
460	0\\
461	0\\
462	0\\
463	0\\
464	0\\
465	0\\
466	0\\
467	0\\
468	0\\
469	0\\
470	0\\
471	0\\
472	0\\
473	0\\
474	0\\
475	0\\
476	0\\
477	0\\
478	0\\
479	0\\
480	0\\
481	0\\
482	0\\
483	0\\
484	0\\
485	0\\
486	0\\
487	0\\
488	0\\
489	0\\
490	0\\
491	0\\
492	0\\
493	0\\
494	0\\
495	0\\
496	0\\
497	0\\
498	0\\
499	0\\
500	0\\
501	0\\
502	0\\
503	0\\
504	0\\
505	0\\
506	0\\
507	0\\
508	0\\
509	0\\
510	0\\
511	0\\
512	0\\
513	0\\
514	0\\
515	0\\
516	0\\
517	0\\
518	0\\
519	0\\
520	0\\
521	0\\
522	0\\
523	0\\
524	0\\
525	0\\
526	0\\
527	0\\
528	0\\
529	0\\
530	0\\
531	0\\
532	0\\
533	0\\
534	0\\
535	0\\
536	0\\
537	0\\
538	0\\
539	0\\
540	0\\
541	0\\
542	3.01637999570976e-08\\
543	2.99658860932134e-05\\
544	6.05018750287523e-05\\
545	9.17088661460931e-05\\
546	0.000123619841455608\\
547	0.000156278193318321\\
548	0.000191232011937836\\
549	0.000227577015837924\\
550	0.000263857115249925\\
551	0.000298369543038874\\
552	0.000332509480255741\\
553	0.000367239183237181\\
554	0.00040251235853391\\
555	0.0004385007442285\\
556	0.00047530803845914\\
557	0.000512965239185088\\
558	0.000551501490389918\\
559	0.000590941911369099\\
560	0.000631324814085933\\
561	0.000675582160272487\\
562	0.000722005927583519\\
563	0.000763647302898438\\
564	0.000803913197393734\\
565	0.00084444157933309\\
566	0.000885669182482565\\
567	0.000927749067611719\\
568	0.000970711974220374\\
569	0.00101458305214579\\
570	0.00105938727463191\\
571	0.00129287225666447\\
572	0.00182349119818199\\
573	0.00193672148976217\\
574	0.00201553086403398\\
575	0.00208678772032082\\
576	0.00215927822556618\\
577	0.00223303834594697\\
578	0.00230809969188834\\
579	0.0023844951352833\\
580	0.00246225891275016\\
581	0.00254142670704607\\
582	0.0026220357606876\\
583	0.00270412508422211\\
584	0.00278773591883777\\
585	0.00287291287144614\\
586	0.00295970681313776\\
587	0.00304818237891641\\
588	0.00313843743693857\\
589	0.00323065362889274\\
590	0.00332522743498844\\
591	0.00342286644802627\\
592	0.00352518337500009\\
593	0.00363635235366165\\
594	0.00376728283869332\\
595	0.00394659005851719\\
596	0.00424941998802836\\
597	0.00487316246495454\\
598	0.00633625614039688\\
599	0\\
600	0\\
};
\addplot [color=mycolor8,solid,forget plot]
  table[row sep=crcr]{%
1	0\\
2	0\\
3	0\\
4	0\\
5	0\\
6	0\\
7	0\\
8	0\\
9	0\\
10	0\\
11	0\\
12	0\\
13	0\\
14	0\\
15	0\\
16	0\\
17	0\\
18	0\\
19	0\\
20	0\\
21	0\\
22	0\\
23	0\\
24	0\\
25	0\\
26	0\\
27	0\\
28	0\\
29	0\\
30	0\\
31	0\\
32	0\\
33	0\\
34	0\\
35	0\\
36	0\\
37	0\\
38	0\\
39	0\\
40	0\\
41	0\\
42	0\\
43	0\\
44	0\\
45	0\\
46	0\\
47	0\\
48	0\\
49	0\\
50	0\\
51	0\\
52	0\\
53	0\\
54	0\\
55	0\\
56	0\\
57	0\\
58	0\\
59	0\\
60	0\\
61	0\\
62	0\\
63	0\\
64	0\\
65	0\\
66	0\\
67	0\\
68	0\\
69	0\\
70	0\\
71	0\\
72	0\\
73	0\\
74	0\\
75	0\\
76	0\\
77	0\\
78	0\\
79	0\\
80	0\\
81	0\\
82	0\\
83	0\\
84	0\\
85	0\\
86	0\\
87	0\\
88	0\\
89	0\\
90	0\\
91	0\\
92	0\\
93	0\\
94	0\\
95	0\\
96	0\\
97	0\\
98	0\\
99	0\\
100	0\\
101	0\\
102	0\\
103	0\\
104	0\\
105	0\\
106	0\\
107	0\\
108	0\\
109	0\\
110	0\\
111	0\\
112	0\\
113	0\\
114	0\\
115	0\\
116	0\\
117	0\\
118	0\\
119	0\\
120	0\\
121	0\\
122	0\\
123	0\\
124	0\\
125	0\\
126	0\\
127	0\\
128	0\\
129	0\\
130	0\\
131	0\\
132	0\\
133	0\\
134	0\\
135	0\\
136	0\\
137	0\\
138	0\\
139	0\\
140	0\\
141	0\\
142	0\\
143	0\\
144	0\\
145	0\\
146	0\\
147	0\\
148	0\\
149	0\\
150	0\\
151	0\\
152	0\\
153	0\\
154	0\\
155	0\\
156	0\\
157	0\\
158	0\\
159	0\\
160	0\\
161	0\\
162	0\\
163	0\\
164	0\\
165	0\\
166	0\\
167	0\\
168	0\\
169	0\\
170	0\\
171	0\\
172	0\\
173	0\\
174	0\\
175	0\\
176	0\\
177	0\\
178	0\\
179	0\\
180	0\\
181	0\\
182	0\\
183	0\\
184	0\\
185	0\\
186	0\\
187	0\\
188	0\\
189	0\\
190	0\\
191	0\\
192	0\\
193	0\\
194	0\\
195	0\\
196	0\\
197	0\\
198	0\\
199	0\\
200	0\\
201	0\\
202	0\\
203	0\\
204	0\\
205	0\\
206	0\\
207	0\\
208	0\\
209	0\\
210	0\\
211	0\\
212	0\\
213	0\\
214	0\\
215	0\\
216	0\\
217	0\\
218	0\\
219	0\\
220	0\\
221	0\\
222	0\\
223	0\\
224	0\\
225	0\\
226	0\\
227	0\\
228	0\\
229	0\\
230	0\\
231	0\\
232	0\\
233	0\\
234	0\\
235	0\\
236	0\\
237	0\\
238	0\\
239	0\\
240	0\\
241	0\\
242	0\\
243	0\\
244	0\\
245	0\\
246	0\\
247	0\\
248	0\\
249	0\\
250	0\\
251	0\\
252	0\\
253	0\\
254	0\\
255	0\\
256	0\\
257	0\\
258	0\\
259	0\\
260	0\\
261	0\\
262	0\\
263	0\\
264	0\\
265	0\\
266	0\\
267	0\\
268	0\\
269	0\\
270	0\\
271	0\\
272	0\\
273	0\\
274	0\\
275	0\\
276	0\\
277	0\\
278	0\\
279	0\\
280	0\\
281	0\\
282	0\\
283	0\\
284	0\\
285	0\\
286	0\\
287	0\\
288	0\\
289	0\\
290	0\\
291	0\\
292	0\\
293	0\\
294	0\\
295	0\\
296	0\\
297	0\\
298	0\\
299	0\\
300	0\\
301	0\\
302	0\\
303	0\\
304	0\\
305	0\\
306	0\\
307	0\\
308	0\\
309	0\\
310	0\\
311	0\\
312	0\\
313	0\\
314	0\\
315	0\\
316	0\\
317	0\\
318	0\\
319	0\\
320	0\\
321	0\\
322	0\\
323	0\\
324	0\\
325	0\\
326	0\\
327	0\\
328	0\\
329	0\\
330	0\\
331	0\\
332	0\\
333	0\\
334	0\\
335	0\\
336	0\\
337	0\\
338	0\\
339	0\\
340	0\\
341	0\\
342	0\\
343	0\\
344	0\\
345	0\\
346	0\\
347	0\\
348	0\\
349	0\\
350	0\\
351	0\\
352	0\\
353	0\\
354	0\\
355	0\\
356	0\\
357	0\\
358	0\\
359	0\\
360	0\\
361	0\\
362	0\\
363	0\\
364	0\\
365	0\\
366	0\\
367	0\\
368	0\\
369	0\\
370	0\\
371	0\\
372	0\\
373	0\\
374	0\\
375	0\\
376	0\\
377	0\\
378	0\\
379	0\\
380	0\\
381	0\\
382	0\\
383	0\\
384	0\\
385	0\\
386	0\\
387	0\\
388	0\\
389	0\\
390	0\\
391	0\\
392	0\\
393	0\\
394	0\\
395	0\\
396	0\\
397	0\\
398	0\\
399	0\\
400	0\\
401	0\\
402	0\\
403	0\\
404	0\\
405	0\\
406	0\\
407	0\\
408	0\\
409	0\\
410	0\\
411	0\\
412	0\\
413	0\\
414	0\\
415	0\\
416	0\\
417	0\\
418	0\\
419	0\\
420	0\\
421	0\\
422	0\\
423	0\\
424	0\\
425	0\\
426	0\\
427	0\\
428	0\\
429	0\\
430	0\\
431	0\\
432	0\\
433	0\\
434	0\\
435	0\\
436	0\\
437	0\\
438	0\\
439	0\\
440	0\\
441	0\\
442	0\\
443	0\\
444	0\\
445	0\\
446	0\\
447	0\\
448	0\\
449	0\\
450	0\\
451	0\\
452	0\\
453	0\\
454	0\\
455	0\\
456	0\\
457	0\\
458	0\\
459	0\\
460	0\\
461	0\\
462	0\\
463	0\\
464	0\\
465	0\\
466	0\\
467	0\\
468	0\\
469	0\\
470	0\\
471	0\\
472	0\\
473	0\\
474	0\\
475	0\\
476	0\\
477	0\\
478	0\\
479	0\\
480	0\\
481	0\\
482	0\\
483	0\\
484	0\\
485	0\\
486	0\\
487	0\\
488	0\\
489	0\\
490	0\\
491	0\\
492	0\\
493	0\\
494	0\\
495	0\\
496	0\\
497	0\\
498	0\\
499	0\\
500	0\\
501	0\\
502	0\\
503	0\\
504	0\\
505	0\\
506	0\\
507	0\\
508	0\\
509	0\\
510	0\\
511	0\\
512	0\\
513	0\\
514	0\\
515	0\\
516	0\\
517	0\\
518	0\\
519	0\\
520	0\\
521	0\\
522	0\\
523	0\\
524	0\\
525	0\\
526	0\\
527	0\\
528	0\\
529	0\\
530	0\\
531	0\\
532	0\\
533	0\\
534	0\\
535	0\\
536	0\\
537	0\\
538	0\\
539	0\\
540	0\\
541	0\\
542	0\\
543	9.36511750306959e-06\\
544	4.0595496929887e-05\\
545	7.22025169697008e-05\\
546	0.000104277738136623\\
547	0.000136901546371204\\
548	0.000170100838277375\\
549	0.000203976958092589\\
550	0.000238526871729759\\
551	0.000275685308335015\\
552	0.000313969737058781\\
553	0.000352328408951315\\
554	0.000388967976204953\\
555	0.000425291340411316\\
556	0.000462255096958164\\
557	0.000499826147251082\\
558	0.000538135586387126\\
559	0.000577322551403436\\
560	0.000617419953092465\\
561	0.000658460371560621\\
562	0.000700497322431724\\
563	0.000747313738427327\\
564	0.000795025248370413\\
565	0.000837765651433914\\
566	0.000879617402495584\\
567	0.000921780386830207\\
568	0.000964645213591637\\
569	0.00100839774047892\\
570	0.0010530754965657\\
571	0.00109870524560156\\
572	0.00121220604724951\\
573	0.00178738153659761\\
574	0.00198895788961421\\
575	0.00208659558592422\\
576	0.00215927626069132\\
577	0.0022330377236014\\
578	0.0023080993732893\\
579	0.0023844949687054\\
580	0.00246225882988089\\
581	0.0025414266687457\\
582	0.00262203574457408\\
583	0.00270412507821453\\
584	0.00278773591692754\\
585	0.00287291287095885\\
586	0.00295970681304879\\
587	0.00304818237890766\\
588	0.00313843743693858\\
589	0.00323065362889274\\
590	0.00332522743498845\\
591	0.00342286644802627\\
592	0.0035251833750001\\
593	0.00363635235366165\\
594	0.00376728283869332\\
595	0.0039465900585172\\
596	0.00424941998802837\\
597	0.00487316246495455\\
598	0.00633625614039688\\
599	0\\
600	0\\
};
\addplot [color=blue!25!mycolor7,solid,forget plot]
  table[row sep=crcr]{%
1	0\\
2	0\\
3	0\\
4	0\\
5	0\\
6	0\\
7	0\\
8	0\\
9	0\\
10	0\\
11	0\\
12	0\\
13	0\\
14	0\\
15	0\\
16	0\\
17	0\\
18	0\\
19	0\\
20	0\\
21	0\\
22	0\\
23	0\\
24	0\\
25	0\\
26	0\\
27	0\\
28	0\\
29	0\\
30	0\\
31	0\\
32	0\\
33	0\\
34	0\\
35	0\\
36	0\\
37	0\\
38	0\\
39	0\\
40	0\\
41	0\\
42	0\\
43	0\\
44	0\\
45	0\\
46	0\\
47	0\\
48	0\\
49	0\\
50	0\\
51	0\\
52	0\\
53	0\\
54	0\\
55	0\\
56	0\\
57	0\\
58	0\\
59	0\\
60	0\\
61	0\\
62	0\\
63	0\\
64	0\\
65	0\\
66	0\\
67	0\\
68	0\\
69	0\\
70	0\\
71	0\\
72	0\\
73	0\\
74	0\\
75	0\\
76	0\\
77	0\\
78	0\\
79	0\\
80	0\\
81	0\\
82	0\\
83	0\\
84	0\\
85	0\\
86	0\\
87	0\\
88	0\\
89	0\\
90	0\\
91	0\\
92	0\\
93	0\\
94	0\\
95	0\\
96	0\\
97	0\\
98	0\\
99	0\\
100	0\\
101	0\\
102	0\\
103	0\\
104	0\\
105	0\\
106	0\\
107	0\\
108	0\\
109	0\\
110	0\\
111	0\\
112	0\\
113	0\\
114	0\\
115	0\\
116	0\\
117	0\\
118	0\\
119	0\\
120	0\\
121	0\\
122	0\\
123	0\\
124	0\\
125	0\\
126	0\\
127	0\\
128	0\\
129	0\\
130	0\\
131	0\\
132	0\\
133	0\\
134	0\\
135	0\\
136	0\\
137	0\\
138	0\\
139	0\\
140	0\\
141	0\\
142	0\\
143	0\\
144	0\\
145	0\\
146	0\\
147	0\\
148	0\\
149	0\\
150	0\\
151	0\\
152	0\\
153	0\\
154	0\\
155	0\\
156	0\\
157	0\\
158	0\\
159	0\\
160	0\\
161	0\\
162	0\\
163	0\\
164	0\\
165	0\\
166	0\\
167	0\\
168	0\\
169	0\\
170	0\\
171	0\\
172	0\\
173	0\\
174	0\\
175	0\\
176	0\\
177	0\\
178	0\\
179	0\\
180	0\\
181	0\\
182	0\\
183	0\\
184	0\\
185	0\\
186	0\\
187	0\\
188	0\\
189	0\\
190	0\\
191	0\\
192	0\\
193	0\\
194	0\\
195	0\\
196	0\\
197	0\\
198	0\\
199	0\\
200	0\\
201	0\\
202	0\\
203	0\\
204	0\\
205	0\\
206	0\\
207	0\\
208	0\\
209	0\\
210	0\\
211	0\\
212	0\\
213	0\\
214	0\\
215	0\\
216	0\\
217	0\\
218	0\\
219	0\\
220	0\\
221	0\\
222	0\\
223	0\\
224	0\\
225	0\\
226	0\\
227	0\\
228	0\\
229	0\\
230	0\\
231	0\\
232	0\\
233	0\\
234	0\\
235	0\\
236	0\\
237	0\\
238	0\\
239	0\\
240	0\\
241	0\\
242	0\\
243	0\\
244	0\\
245	0\\
246	0\\
247	0\\
248	0\\
249	0\\
250	0\\
251	0\\
252	0\\
253	0\\
254	0\\
255	0\\
256	0\\
257	0\\
258	0\\
259	0\\
260	0\\
261	0\\
262	0\\
263	0\\
264	0\\
265	0\\
266	0\\
267	0\\
268	0\\
269	0\\
270	0\\
271	0\\
272	0\\
273	0\\
274	0\\
275	0\\
276	0\\
277	0\\
278	0\\
279	0\\
280	0\\
281	0\\
282	0\\
283	0\\
284	0\\
285	0\\
286	0\\
287	0\\
288	0\\
289	0\\
290	0\\
291	0\\
292	0\\
293	0\\
294	0\\
295	0\\
296	0\\
297	0\\
298	0\\
299	0\\
300	0\\
301	0\\
302	0\\
303	0\\
304	0\\
305	0\\
306	0\\
307	0\\
308	0\\
309	0\\
310	0\\
311	0\\
312	0\\
313	0\\
314	0\\
315	0\\
316	0\\
317	0\\
318	0\\
319	0\\
320	0\\
321	0\\
322	0\\
323	0\\
324	0\\
325	0\\
326	0\\
327	0\\
328	0\\
329	0\\
330	0\\
331	0\\
332	0\\
333	0\\
334	0\\
335	0\\
336	0\\
337	0\\
338	0\\
339	0\\
340	0\\
341	0\\
342	0\\
343	0\\
344	0\\
345	0\\
346	0\\
347	0\\
348	0\\
349	0\\
350	0\\
351	0\\
352	0\\
353	0\\
354	0\\
355	0\\
356	0\\
357	0\\
358	0\\
359	0\\
360	0\\
361	0\\
362	0\\
363	0\\
364	0\\
365	0\\
366	0\\
367	0\\
368	0\\
369	0\\
370	0\\
371	0\\
372	0\\
373	0\\
374	0\\
375	0\\
376	0\\
377	0\\
378	0\\
379	0\\
380	0\\
381	0\\
382	0\\
383	0\\
384	0\\
385	0\\
386	0\\
387	0\\
388	0\\
389	0\\
390	0\\
391	0\\
392	0\\
393	0\\
394	0\\
395	0\\
396	0\\
397	0\\
398	0\\
399	0\\
400	0\\
401	0\\
402	0\\
403	0\\
404	0\\
405	0\\
406	0\\
407	0\\
408	0\\
409	0\\
410	0\\
411	0\\
412	0\\
413	0\\
414	0\\
415	0\\
416	0\\
417	0\\
418	0\\
419	0\\
420	0\\
421	0\\
422	0\\
423	0\\
424	0\\
425	0\\
426	0\\
427	0\\
428	0\\
429	0\\
430	0\\
431	0\\
432	0\\
433	0\\
434	0\\
435	0\\
436	0\\
437	0\\
438	0\\
439	0\\
440	0\\
441	0\\
442	0\\
443	0\\
444	0\\
445	0\\
446	0\\
447	0\\
448	0\\
449	0\\
450	0\\
451	0\\
452	0\\
453	0\\
454	0\\
455	0\\
456	0\\
457	0\\
458	0\\
459	0\\
460	0\\
461	0\\
462	0\\
463	0\\
464	0\\
465	0\\
466	0\\
467	0\\
468	0\\
469	0\\
470	0\\
471	0\\
472	0\\
473	0\\
474	0\\
475	0\\
476	0\\
477	0\\
478	0\\
479	0\\
480	0\\
481	0\\
482	0\\
483	0\\
484	0\\
485	0\\
486	0\\
487	0\\
488	0\\
489	0\\
490	0\\
491	0\\
492	0\\
493	0\\
494	0\\
495	0\\
496	0\\
497	0\\
498	0\\
499	0\\
500	0\\
501	0\\
502	0\\
503	0\\
504	0\\
505	0\\
506	0\\
507	0\\
508	0\\
509	0\\
510	0\\
511	0\\
512	0\\
513	0\\
514	0\\
515	0\\
516	0\\
517	0\\
518	0\\
519	0\\
520	0\\
521	0\\
522	0\\
523	0\\
524	0\\
525	0\\
526	0\\
527	0\\
528	0\\
529	0\\
530	0\\
531	0\\
532	0\\
533	0\\
534	0\\
535	0\\
536	0\\
537	0\\
538	0\\
539	0\\
540	0\\
541	0\\
542	0\\
543	0\\
544	7.49517579551879e-07\\
545	3.85447340164759e-05\\
546	7.52662412083253e-05\\
547	0.000111131029203797\\
548	0.000145643679797062\\
549	0.000180296982227337\\
550	0.00021535568667807\\
551	0.00025080449961763\\
552	0.000286874977847823\\
553	0.000323643662070736\\
554	0.000363096748878308\\
555	0.000403697300813226\\
556	0.000444411306028165\\
557	0.000483782483864593\\
558	0.000522495298114908\\
559	0.000561899913432818\\
560	0.000601990613295173\\
561	0.00064280724862691\\
562	0.000684562043987083\\
563	0.000727288108897307\\
564	0.000771057340961328\\
565	0.000819898537954155\\
566	0.000869220907437458\\
567	0.000914032068548085\\
568	0.000957627672776297\\
569	0.00100160271504627\\
570	0.00104620716841636\\
571	0.00109173284949489\\
572	0.00113822717707793\\
573	0.00118572633563006\\
574	0.0016486833706938\\
575	0.00203747445819607\\
576	0.00215253303533154\\
577	0.00223300678105388\\
578	0.00230809492244212\\
579	0.00238449294006508\\
580	0.00246225774710443\\
581	0.00254142610849541\\
582	0.00262203547379105\\
583	0.00270412495853163\\
584	0.00278773586986386\\
585	0.00287291285511267\\
586	0.00295970680874964\\
587	0.00304818237806893\\
588	0.00313843743684985\\
589	0.00323065362889274\\
590	0.00332522743498845\\
591	0.00342286644802627\\
592	0.00352518337500008\\
593	0.00363635235366165\\
594	0.00376728283869331\\
595	0.00394659005851719\\
596	0.00424941998802836\\
597	0.00487316246495454\\
598	0.00633625614039688\\
599	0\\
600	0\\
};
\addplot [color=mycolor9,solid,forget plot]
  table[row sep=crcr]{%
1	0\\
2	0\\
3	0\\
4	0\\
5	0\\
6	0\\
7	0\\
8	0\\
9	0\\
10	0\\
11	0\\
12	0\\
13	0\\
14	0\\
15	0\\
16	0\\
17	0\\
18	0\\
19	0\\
20	0\\
21	0\\
22	0\\
23	0\\
24	0\\
25	0\\
26	0\\
27	0\\
28	0\\
29	0\\
30	0\\
31	0\\
32	0\\
33	0\\
34	0\\
35	0\\
36	0\\
37	0\\
38	0\\
39	0\\
40	0\\
41	0\\
42	0\\
43	0\\
44	0\\
45	0\\
46	0\\
47	0\\
48	0\\
49	0\\
50	0\\
51	0\\
52	0\\
53	0\\
54	0\\
55	0\\
56	0\\
57	0\\
58	0\\
59	0\\
60	0\\
61	0\\
62	0\\
63	0\\
64	0\\
65	0\\
66	0\\
67	0\\
68	0\\
69	0\\
70	0\\
71	0\\
72	0\\
73	0\\
74	0\\
75	0\\
76	0\\
77	0\\
78	0\\
79	0\\
80	0\\
81	0\\
82	0\\
83	0\\
84	0\\
85	0\\
86	0\\
87	0\\
88	0\\
89	0\\
90	0\\
91	0\\
92	0\\
93	0\\
94	0\\
95	0\\
96	0\\
97	0\\
98	0\\
99	0\\
100	0\\
101	0\\
102	0\\
103	0\\
104	0\\
105	0\\
106	0\\
107	0\\
108	0\\
109	0\\
110	0\\
111	0\\
112	0\\
113	0\\
114	0\\
115	0\\
116	0\\
117	0\\
118	0\\
119	0\\
120	0\\
121	0\\
122	0\\
123	0\\
124	0\\
125	0\\
126	0\\
127	0\\
128	0\\
129	0\\
130	0\\
131	0\\
132	0\\
133	0\\
134	0\\
135	0\\
136	0\\
137	0\\
138	0\\
139	0\\
140	0\\
141	0\\
142	0\\
143	0\\
144	0\\
145	0\\
146	0\\
147	0\\
148	0\\
149	0\\
150	0\\
151	0\\
152	0\\
153	0\\
154	0\\
155	0\\
156	0\\
157	0\\
158	0\\
159	0\\
160	0\\
161	0\\
162	0\\
163	0\\
164	0\\
165	0\\
166	0\\
167	0\\
168	0\\
169	0\\
170	0\\
171	0\\
172	0\\
173	0\\
174	0\\
175	0\\
176	0\\
177	0\\
178	0\\
179	0\\
180	0\\
181	0\\
182	0\\
183	0\\
184	0\\
185	0\\
186	0\\
187	0\\
188	0\\
189	0\\
190	0\\
191	0\\
192	0\\
193	0\\
194	0\\
195	0\\
196	0\\
197	0\\
198	0\\
199	0\\
200	0\\
201	0\\
202	0\\
203	0\\
204	0\\
205	0\\
206	0\\
207	0\\
208	0\\
209	0\\
210	0\\
211	0\\
212	0\\
213	0\\
214	0\\
215	0\\
216	0\\
217	0\\
218	0\\
219	0\\
220	0\\
221	0\\
222	0\\
223	0\\
224	0\\
225	0\\
226	0\\
227	0\\
228	0\\
229	0\\
230	0\\
231	0\\
232	0\\
233	0\\
234	0\\
235	0\\
236	0\\
237	0\\
238	0\\
239	0\\
240	0\\
241	0\\
242	0\\
243	0\\
244	0\\
245	0\\
246	0\\
247	0\\
248	0\\
249	0\\
250	0\\
251	0\\
252	0\\
253	0\\
254	0\\
255	0\\
256	0\\
257	0\\
258	0\\
259	0\\
260	0\\
261	0\\
262	0\\
263	0\\
264	0\\
265	0\\
266	0\\
267	0\\
268	0\\
269	0\\
270	0\\
271	0\\
272	0\\
273	0\\
274	0\\
275	0\\
276	0\\
277	0\\
278	0\\
279	0\\
280	0\\
281	0\\
282	0\\
283	0\\
284	0\\
285	0\\
286	0\\
287	0\\
288	0\\
289	0\\
290	0\\
291	0\\
292	0\\
293	0\\
294	0\\
295	0\\
296	0\\
297	0\\
298	0\\
299	0\\
300	0\\
301	0\\
302	0\\
303	0\\
304	0\\
305	0\\
306	0\\
307	0\\
308	0\\
309	0\\
310	0\\
311	0\\
312	0\\
313	0\\
314	0\\
315	0\\
316	0\\
317	0\\
318	0\\
319	0\\
320	0\\
321	0\\
322	0\\
323	0\\
324	0\\
325	0\\
326	0\\
327	0\\
328	0\\
329	0\\
330	0\\
331	0\\
332	0\\
333	0\\
334	0\\
335	0\\
336	0\\
337	0\\
338	0\\
339	0\\
340	0\\
341	0\\
342	0\\
343	0\\
344	0\\
345	0\\
346	0\\
347	0\\
348	0\\
349	0\\
350	0\\
351	0\\
352	0\\
353	0\\
354	0\\
355	0\\
356	0\\
357	0\\
358	0\\
359	0\\
360	0\\
361	0\\
362	0\\
363	0\\
364	0\\
365	0\\
366	0\\
367	0\\
368	0\\
369	0\\
370	0\\
371	0\\
372	0\\
373	0\\
374	0\\
375	0\\
376	0\\
377	0\\
378	0\\
379	0\\
380	0\\
381	0\\
382	0\\
383	0\\
384	0\\
385	0\\
386	0\\
387	0\\
388	0\\
389	0\\
390	0\\
391	0\\
392	0\\
393	0\\
394	0\\
395	0\\
396	0\\
397	0\\
398	0\\
399	0\\
400	0\\
401	0\\
402	0\\
403	0\\
404	0\\
405	0\\
406	0\\
407	0\\
408	0\\
409	0\\
410	0\\
411	0\\
412	0\\
413	0\\
414	0\\
415	0\\
416	0\\
417	0\\
418	0\\
419	0\\
420	0\\
421	0\\
422	0\\
423	0\\
424	0\\
425	0\\
426	0\\
427	0\\
428	0\\
429	0\\
430	0\\
431	0\\
432	0\\
433	0\\
434	0\\
435	0\\
436	0\\
437	0\\
438	0\\
439	0\\
440	0\\
441	0\\
442	0\\
443	0\\
444	0\\
445	0\\
446	0\\
447	0\\
448	0\\
449	0\\
450	0\\
451	0\\
452	0\\
453	0\\
454	0\\
455	0\\
456	0\\
457	0\\
458	0\\
459	0\\
460	0\\
461	0\\
462	0\\
463	0\\
464	0\\
465	0\\
466	0\\
467	0\\
468	0\\
469	0\\
470	0\\
471	0\\
472	0\\
473	0\\
474	0\\
475	0\\
476	0\\
477	0\\
478	0\\
479	0\\
480	0\\
481	0\\
482	0\\
483	0\\
484	0\\
485	0\\
486	0\\
487	0\\
488	0\\
489	0\\
490	0\\
491	0\\
492	0\\
493	0\\
494	0\\
495	0\\
496	0\\
497	0\\
498	0\\
499	0\\
500	0\\
501	0\\
502	0\\
503	0\\
504	0\\
505	0\\
506	0\\
507	0\\
508	0\\
509	0\\
510	0\\
511	0\\
512	0\\
513	0\\
514	0\\
515	0\\
516	0\\
517	0\\
518	0\\
519	0\\
520	0\\
521	0\\
522	0\\
523	0\\
524	0\\
525	0\\
526	0\\
527	0\\
528	0\\
529	0\\
530	0\\
531	0\\
532	0\\
533	0\\
534	0\\
535	0\\
536	0\\
537	0\\
538	0\\
539	0\\
540	0\\
541	0\\
542	0\\
543	0\\
544	0\\
545	0\\
546	0\\
547	4.00690256151494e-05\\
548	9.18803600518231e-05\\
549	0.000134564232292414\\
550	0.000176112314276617\\
551	0.000216585928094907\\
552	0.000256053948070476\\
553	0.000294072813221761\\
554	0.000332498645173348\\
555	0.00037143411280242\\
556	0.000411017553589854\\
557	0.000452836567132838\\
558	0.000496149912580918\\
559	0.000539598870420913\\
560	0.000582438546231535\\
561	0.00062382315388814\\
562	0.000665941388989703\\
563	0.000708824054090575\\
564	0.000752423920588811\\
565	0.000796984050205826\\
566	0.000842614329043036\\
567	0.000892957740392164\\
568	0.000944269325554044\\
569	0.000992198948404602\\
570	0.00103774091284746\\
571	0.00108375253963842\\
572	0.0011302426310335\\
573	0.00117769251944014\\
574	0.00122615490603659\\
575	0.00146936455367717\\
576	0.00201911956940423\\
577	0.00219874416083084\\
578	0.0023072684395088\\
579	0.0023844472193452\\
580	0.00246224470739401\\
581	0.00254141932241292\\
582	0.00262203183295566\\
583	0.00270412311835047\\
584	0.00278773501347819\\
585	0.00287291249876727\\
586	0.00295970668121992\\
587	0.00304818234110382\\
588	0.00313843742910297\\
589	0.00323065362800676\\
590	0.00332522743498845\\
591	0.00342286644802625\\
592	0.00352518337500009\\
593	0.00363635235366165\\
594	0.00376728283869332\\
595	0.0039465900585172\\
596	0.00424941998802836\\
597	0.00487316246495455\\
598	0.00633625614039688\\
599	0\\
600	0\\
};
\addplot [color=blue!50!mycolor7,solid,forget plot]
  table[row sep=crcr]{%
1	0\\
2	0\\
3	0\\
4	0\\
5	0\\
6	0\\
7	0\\
8	0\\
9	0\\
10	0\\
11	0\\
12	0\\
13	0\\
14	0\\
15	0\\
16	0\\
17	0\\
18	0\\
19	0\\
20	0\\
21	0\\
22	0\\
23	0\\
24	0\\
25	0\\
26	0\\
27	0\\
28	0\\
29	0\\
30	0\\
31	0\\
32	0\\
33	0\\
34	0\\
35	0\\
36	0\\
37	0\\
38	0\\
39	0\\
40	0\\
41	0\\
42	0\\
43	0\\
44	0\\
45	0\\
46	0\\
47	0\\
48	0\\
49	0\\
50	0\\
51	0\\
52	0\\
53	0\\
54	0\\
55	0\\
56	0\\
57	0\\
58	0\\
59	0\\
60	0\\
61	0\\
62	0\\
63	0\\
64	0\\
65	0\\
66	0\\
67	0\\
68	0\\
69	0\\
70	0\\
71	0\\
72	0\\
73	0\\
74	0\\
75	0\\
76	0\\
77	0\\
78	0\\
79	0\\
80	0\\
81	0\\
82	0\\
83	0\\
84	0\\
85	0\\
86	0\\
87	0\\
88	0\\
89	0\\
90	0\\
91	0\\
92	0\\
93	0\\
94	0\\
95	0\\
96	0\\
97	0\\
98	0\\
99	0\\
100	0\\
101	0\\
102	0\\
103	0\\
104	0\\
105	0\\
106	0\\
107	0\\
108	0\\
109	0\\
110	0\\
111	0\\
112	0\\
113	0\\
114	0\\
115	0\\
116	0\\
117	0\\
118	0\\
119	0\\
120	0\\
121	0\\
122	0\\
123	0\\
124	0\\
125	0\\
126	0\\
127	0\\
128	0\\
129	0\\
130	0\\
131	0\\
132	0\\
133	0\\
134	0\\
135	0\\
136	0\\
137	0\\
138	0\\
139	0\\
140	0\\
141	0\\
142	0\\
143	0\\
144	0\\
145	0\\
146	0\\
147	0\\
148	0\\
149	0\\
150	0\\
151	0\\
152	0\\
153	0\\
154	0\\
155	0\\
156	0\\
157	0\\
158	0\\
159	0\\
160	0\\
161	0\\
162	0\\
163	0\\
164	0\\
165	0\\
166	0\\
167	0\\
168	0\\
169	0\\
170	0\\
171	0\\
172	0\\
173	0\\
174	0\\
175	0\\
176	0\\
177	0\\
178	0\\
179	0\\
180	0\\
181	0\\
182	0\\
183	0\\
184	0\\
185	0\\
186	0\\
187	0\\
188	0\\
189	0\\
190	0\\
191	0\\
192	0\\
193	0\\
194	0\\
195	0\\
196	0\\
197	0\\
198	0\\
199	0\\
200	0\\
201	0\\
202	0\\
203	0\\
204	0\\
205	0\\
206	0\\
207	0\\
208	0\\
209	0\\
210	0\\
211	0\\
212	0\\
213	0\\
214	0\\
215	0\\
216	0\\
217	0\\
218	0\\
219	0\\
220	0\\
221	0\\
222	0\\
223	0\\
224	0\\
225	0\\
226	0\\
227	0\\
228	0\\
229	0\\
230	0\\
231	0\\
232	0\\
233	0\\
234	0\\
235	0\\
236	0\\
237	0\\
238	0\\
239	0\\
240	0\\
241	0\\
242	0\\
243	0\\
244	0\\
245	0\\
246	0\\
247	0\\
248	0\\
249	0\\
250	0\\
251	0\\
252	0\\
253	0\\
254	0\\
255	0\\
256	0\\
257	0\\
258	0\\
259	0\\
260	0\\
261	0\\
262	0\\
263	0\\
264	0\\
265	0\\
266	0\\
267	0\\
268	0\\
269	0\\
270	0\\
271	0\\
272	0\\
273	0\\
274	0\\
275	0\\
276	0\\
277	0\\
278	0\\
279	0\\
280	0\\
281	0\\
282	0\\
283	0\\
284	0\\
285	0\\
286	0\\
287	0\\
288	0\\
289	0\\
290	0\\
291	0\\
292	0\\
293	0\\
294	0\\
295	0\\
296	0\\
297	0\\
298	0\\
299	0\\
300	0\\
301	0\\
302	0\\
303	0\\
304	0\\
305	0\\
306	0\\
307	0\\
308	0\\
309	0\\
310	0\\
311	0\\
312	0\\
313	0\\
314	0\\
315	0\\
316	0\\
317	0\\
318	0\\
319	0\\
320	0\\
321	0\\
322	0\\
323	0\\
324	0\\
325	0\\
326	0\\
327	0\\
328	0\\
329	0\\
330	0\\
331	0\\
332	0\\
333	0\\
334	0\\
335	0\\
336	0\\
337	0\\
338	0\\
339	0\\
340	0\\
341	0\\
342	0\\
343	0\\
344	0\\
345	0\\
346	0\\
347	0\\
348	0\\
349	0\\
350	0\\
351	0\\
352	0\\
353	0\\
354	0\\
355	0\\
356	0\\
357	0\\
358	0\\
359	0\\
360	0\\
361	0\\
362	0\\
363	0\\
364	0\\
365	0\\
366	0\\
367	0\\
368	0\\
369	0\\
370	0\\
371	0\\
372	0\\
373	0\\
374	0\\
375	0\\
376	0\\
377	0\\
378	0\\
379	0\\
380	0\\
381	0\\
382	0\\
383	0\\
384	0\\
385	0\\
386	0\\
387	0\\
388	0\\
389	0\\
390	0\\
391	0\\
392	0\\
393	0\\
394	0\\
395	0\\
396	0\\
397	0\\
398	0\\
399	0\\
400	0\\
401	0\\
402	0\\
403	0\\
404	0\\
405	0\\
406	0\\
407	0\\
408	0\\
409	0\\
410	0\\
411	0\\
412	0\\
413	0\\
414	0\\
415	0\\
416	0\\
417	0\\
418	0\\
419	0\\
420	0\\
421	0\\
422	0\\
423	0\\
424	0\\
425	0\\
426	0\\
427	0\\
428	0\\
429	0\\
430	0\\
431	0\\
432	0\\
433	0\\
434	0\\
435	0\\
436	0\\
437	0\\
438	0\\
439	0\\
440	0\\
441	0\\
442	0\\
443	0\\
444	0\\
445	0\\
446	0\\
447	0\\
448	0\\
449	0\\
450	0\\
451	0\\
452	0\\
453	0\\
454	0\\
455	0\\
456	0\\
457	0\\
458	0\\
459	0\\
460	0\\
461	0\\
462	0\\
463	0\\
464	0\\
465	0\\
466	0\\
467	0\\
468	0\\
469	0\\
470	0\\
471	0\\
472	0\\
473	0\\
474	0\\
475	0\\
476	0\\
477	0\\
478	0\\
479	0\\
480	0\\
481	0\\
482	0\\
483	0\\
484	0\\
485	0\\
486	0\\
487	0\\
488	0\\
489	0\\
490	0\\
491	0\\
492	0\\
493	0\\
494	0\\
495	0\\
496	0\\
497	0\\
498	0\\
499	0\\
500	0\\
501	0\\
502	0\\
503	0\\
504	0\\
505	0\\
506	0\\
507	0\\
508	0\\
509	0\\
510	0\\
511	0\\
512	0\\
513	0\\
514	0\\
515	0\\
516	0\\
517	0\\
518	0\\
519	0\\
520	0\\
521	0\\
522	0\\
523	0\\
524	0\\
525	0\\
526	0\\
527	0\\
528	0\\
529	0\\
530	0\\
531	0\\
532	0\\
533	0\\
534	0\\
535	0\\
536	0\\
537	0\\
538	0\\
539	0\\
540	0\\
541	0\\
542	0\\
543	0\\
544	0\\
545	0\\
546	0\\
547	0\\
548	0\\
549	0\\
550	4.20820499820094e-05\\
551	0.000111412008282726\\
552	0.000174637757067829\\
553	0.000230566648731088\\
554	0.000277788239069468\\
555	0.000323717941772093\\
556	0.000368646415826842\\
557	0.000412551862509468\\
558	0.000455197630167723\\
559	0.000498339909108939\\
560	0.000542855731364164\\
561	0.000589683232461812\\
562	0.000636530039514592\\
563	0.000683424746743966\\
564	0.000728326965189056\\
565	0.00077355236992366\\
566	0.000819571520899858\\
567	0.000866401369931615\\
568	0.000914114247341885\\
569	0.000965487206123931\\
570	0.00101923516298803\\
571	0.00107141704545175\\
572	0.00111913073277248\\
573	0.0011673538424836\\
574	0.00121604908991694\\
575	0.0012655229210271\\
576	0.00131604936497319\\
577	0.00177799418687613\\
578	0.0022361256288932\\
579	0.00235742919584368\\
580	0.00246133490814186\\
581	0.0025413184666642\\
582	0.00262199004833523\\
583	0.0027041002675783\\
584	0.00278772301988261\\
585	0.00287290661496934\\
586	0.00295970408133531\\
587	0.00304818134793304\\
588	0.00313843711990243\\
589	0.0032306535579562\\
590	0.00332522742625831\\
591	0.00342286644802626\\
592	0.00352518337500009\\
593	0.00363635235366165\\
594	0.00376728283869331\\
595	0.00394659005851718\\
596	0.00424941998802836\\
597	0.00487316246495455\\
598	0.00633625614039688\\
599	0\\
600	0\\
};
\addplot [color=blue!40!mycolor9,solid,forget plot]
  table[row sep=crcr]{%
1	0\\
2	0\\
3	0\\
4	0\\
5	0\\
6	0\\
7	0\\
8	0\\
9	0\\
10	0\\
11	0\\
12	0\\
13	0\\
14	0\\
15	0\\
16	0\\
17	0\\
18	0\\
19	0\\
20	0\\
21	0\\
22	0\\
23	0\\
24	0\\
25	0\\
26	0\\
27	0\\
28	0\\
29	0\\
30	0\\
31	0\\
32	0\\
33	0\\
34	0\\
35	0\\
36	0\\
37	0\\
38	0\\
39	0\\
40	0\\
41	0\\
42	0\\
43	0\\
44	0\\
45	0\\
46	0\\
47	0\\
48	0\\
49	0\\
50	0\\
51	0\\
52	0\\
53	0\\
54	0\\
55	0\\
56	0\\
57	0\\
58	0\\
59	0\\
60	0\\
61	0\\
62	0\\
63	0\\
64	0\\
65	0\\
66	0\\
67	0\\
68	0\\
69	0\\
70	0\\
71	0\\
72	0\\
73	0\\
74	0\\
75	0\\
76	0\\
77	0\\
78	0\\
79	0\\
80	0\\
81	0\\
82	0\\
83	0\\
84	0\\
85	0\\
86	0\\
87	0\\
88	0\\
89	0\\
90	0\\
91	0\\
92	0\\
93	0\\
94	0\\
95	0\\
96	0\\
97	0\\
98	0\\
99	0\\
100	0\\
101	0\\
102	0\\
103	0\\
104	0\\
105	0\\
106	0\\
107	0\\
108	0\\
109	0\\
110	0\\
111	0\\
112	0\\
113	0\\
114	0\\
115	0\\
116	0\\
117	0\\
118	0\\
119	0\\
120	0\\
121	0\\
122	0\\
123	0\\
124	0\\
125	0\\
126	0\\
127	0\\
128	0\\
129	0\\
130	0\\
131	0\\
132	0\\
133	0\\
134	0\\
135	0\\
136	0\\
137	0\\
138	0\\
139	0\\
140	0\\
141	0\\
142	0\\
143	0\\
144	0\\
145	0\\
146	0\\
147	0\\
148	0\\
149	0\\
150	0\\
151	0\\
152	0\\
153	0\\
154	0\\
155	0\\
156	0\\
157	0\\
158	0\\
159	0\\
160	0\\
161	0\\
162	0\\
163	0\\
164	0\\
165	0\\
166	0\\
167	0\\
168	0\\
169	0\\
170	0\\
171	0\\
172	0\\
173	0\\
174	0\\
175	0\\
176	0\\
177	0\\
178	0\\
179	0\\
180	0\\
181	0\\
182	0\\
183	0\\
184	0\\
185	0\\
186	0\\
187	0\\
188	0\\
189	0\\
190	0\\
191	0\\
192	0\\
193	0\\
194	0\\
195	0\\
196	0\\
197	0\\
198	0\\
199	0\\
200	0\\
201	0\\
202	0\\
203	0\\
204	0\\
205	0\\
206	0\\
207	0\\
208	0\\
209	0\\
210	0\\
211	0\\
212	0\\
213	0\\
214	0\\
215	0\\
216	0\\
217	0\\
218	0\\
219	0\\
220	0\\
221	0\\
222	0\\
223	0\\
224	0\\
225	0\\
226	0\\
227	0\\
228	0\\
229	0\\
230	0\\
231	0\\
232	0\\
233	0\\
234	0\\
235	0\\
236	0\\
237	0\\
238	0\\
239	0\\
240	0\\
241	0\\
242	0\\
243	0\\
244	0\\
245	0\\
246	0\\
247	0\\
248	0\\
249	0\\
250	0\\
251	0\\
252	0\\
253	0\\
254	0\\
255	0\\
256	0\\
257	0\\
258	0\\
259	0\\
260	0\\
261	0\\
262	0\\
263	0\\
264	0\\
265	0\\
266	0\\
267	0\\
268	0\\
269	0\\
270	0\\
271	0\\
272	0\\
273	0\\
274	0\\
275	0\\
276	0\\
277	0\\
278	0\\
279	0\\
280	0\\
281	0\\
282	0\\
283	0\\
284	0\\
285	0\\
286	0\\
287	0\\
288	0\\
289	0\\
290	0\\
291	0\\
292	0\\
293	0\\
294	0\\
295	0\\
296	0\\
297	0\\
298	0\\
299	0\\
300	0\\
301	0\\
302	0\\
303	0\\
304	0\\
305	0\\
306	0\\
307	0\\
308	0\\
309	0\\
310	0\\
311	0\\
312	0\\
313	0\\
314	0\\
315	0\\
316	0\\
317	0\\
318	0\\
319	0\\
320	0\\
321	0\\
322	0\\
323	0\\
324	0\\
325	0\\
326	0\\
327	0\\
328	0\\
329	0\\
330	0\\
331	0\\
332	0\\
333	0\\
334	0\\
335	0\\
336	0\\
337	0\\
338	0\\
339	0\\
340	0\\
341	0\\
342	0\\
343	0\\
344	0\\
345	0\\
346	0\\
347	0\\
348	0\\
349	0\\
350	0\\
351	0\\
352	0\\
353	0\\
354	0\\
355	0\\
356	0\\
357	0\\
358	0\\
359	0\\
360	0\\
361	0\\
362	0\\
363	0\\
364	0\\
365	0\\
366	0\\
367	0\\
368	0\\
369	0\\
370	0\\
371	0\\
372	0\\
373	0\\
374	0\\
375	0\\
376	0\\
377	0\\
378	0\\
379	0\\
380	0\\
381	0\\
382	0\\
383	0\\
384	0\\
385	0\\
386	0\\
387	0\\
388	0\\
389	0\\
390	0\\
391	0\\
392	0\\
393	0\\
394	0\\
395	0\\
396	0\\
397	0\\
398	0\\
399	0\\
400	0\\
401	0\\
402	0\\
403	0\\
404	0\\
405	0\\
406	0\\
407	0\\
408	0\\
409	0\\
410	0\\
411	0\\
412	0\\
413	0\\
414	0\\
415	0\\
416	0\\
417	0\\
418	0\\
419	0\\
420	0\\
421	0\\
422	0\\
423	0\\
424	0\\
425	0\\
426	0\\
427	0\\
428	0\\
429	0\\
430	0\\
431	0\\
432	0\\
433	0\\
434	0\\
435	0\\
436	0\\
437	0\\
438	0\\
439	0\\
440	0\\
441	0\\
442	0\\
443	0\\
444	0\\
445	0\\
446	0\\
447	0\\
448	0\\
449	0\\
450	0\\
451	0\\
452	0\\
453	0\\
454	0\\
455	0\\
456	0\\
457	0\\
458	0\\
459	0\\
460	0\\
461	0\\
462	0\\
463	0\\
464	0\\
465	0\\
466	0\\
467	0\\
468	0\\
469	0\\
470	0\\
471	0\\
472	0\\
473	0\\
474	0\\
475	0\\
476	0\\
477	0\\
478	0\\
479	0\\
480	0\\
481	0\\
482	0\\
483	0\\
484	0\\
485	0\\
486	0\\
487	0\\
488	0\\
489	0\\
490	0\\
491	0\\
492	0\\
493	0\\
494	0\\
495	0\\
496	0\\
497	0\\
498	0\\
499	0\\
500	0\\
501	0\\
502	0\\
503	0\\
504	0\\
505	0\\
506	0\\
507	0\\
508	0\\
509	0\\
510	0\\
511	0\\
512	0\\
513	0\\
514	0\\
515	0\\
516	0\\
517	0\\
518	0\\
519	0\\
520	0\\
521	0\\
522	0\\
523	0\\
524	0\\
525	0\\
526	0\\
527	0\\
528	0\\
529	0\\
530	0\\
531	0\\
532	0\\
533	0\\
534	0\\
535	0\\
536	0\\
537	0\\
538	0\\
539	0\\
540	0\\
541	0\\
542	0\\
543	0\\
544	0\\
545	0\\
546	0\\
547	0\\
548	0\\
549	0\\
550	0\\
551	0\\
552	0\\
553	5.16889552756766e-06\\
554	8.77826221569004e-05\\
555	0.000167631433400357\\
556	0.000243418760937493\\
557	0.000313414182083021\\
558	0.000375902069304456\\
559	0.00042874744998465\\
560	0.000480599081949562\\
561	0.000531451904771833\\
562	0.000581189982271827\\
563	0.000629635906988094\\
564	0.000680653412810273\\
565	0.00073201619529141\\
566	0.000783309644171608\\
567	0.000833969941358776\\
568	0.000882957822718271\\
569	0.000932718658926889\\
570	0.000983330016376749\\
571	0.00103559259416003\\
572	0.00109244755817989\\
573	0.0011485579817111\\
574	0.00120114230828459\\
575	0.00125198809055414\\
576	0.00130338498565377\\
577	0.00135527218410646\\
578	0.00149765035674765\\
579	0.00201460319803586\\
580	0.00239071852558201\\
581	0.00251424382398314\\
582	0.00262058365852551\\
583	0.00270381978473586\\
584	0.00278757879201644\\
585	0.00287283135283087\\
586	0.00295966537365807\\
587	0.00304816311789294\\
588	0.00313842965532306\\
589	0.00323065104580206\\
590	0.00332522680618373\\
591	0.00342286636301155\\
592	0.00352518337500009\\
593	0.00363635235366165\\
594	0.00376728283869333\\
595	0.0039465900585172\\
596	0.00424941998802837\\
597	0.00487316246495455\\
598	0.00633625614039688\\
599	0\\
600	0\\
};
\addplot [color=blue!75!mycolor7,solid,forget plot]
  table[row sep=crcr]{%
1	0\\
2	0\\
3	0\\
4	0\\
5	0\\
6	0\\
7	0\\
8	0\\
9	0\\
10	0\\
11	0\\
12	0\\
13	0\\
14	0\\
15	0\\
16	0\\
17	0\\
18	0\\
19	0\\
20	0\\
21	0\\
22	0\\
23	0\\
24	0\\
25	0\\
26	0\\
27	0\\
28	0\\
29	0\\
30	0\\
31	0\\
32	0\\
33	0\\
34	0\\
35	0\\
36	0\\
37	0\\
38	0\\
39	0\\
40	0\\
41	0\\
42	0\\
43	0\\
44	0\\
45	0\\
46	0\\
47	0\\
48	0\\
49	0\\
50	0\\
51	0\\
52	0\\
53	0\\
54	0\\
55	0\\
56	0\\
57	0\\
58	0\\
59	0\\
60	0\\
61	0\\
62	0\\
63	0\\
64	0\\
65	0\\
66	0\\
67	0\\
68	0\\
69	0\\
70	0\\
71	0\\
72	0\\
73	0\\
74	0\\
75	0\\
76	0\\
77	0\\
78	0\\
79	0\\
80	0\\
81	0\\
82	0\\
83	0\\
84	0\\
85	0\\
86	0\\
87	0\\
88	0\\
89	0\\
90	0\\
91	0\\
92	0\\
93	0\\
94	0\\
95	0\\
96	0\\
97	0\\
98	0\\
99	0\\
100	0\\
101	0\\
102	0\\
103	0\\
104	0\\
105	0\\
106	0\\
107	0\\
108	0\\
109	0\\
110	0\\
111	0\\
112	0\\
113	0\\
114	0\\
115	0\\
116	0\\
117	0\\
118	0\\
119	0\\
120	0\\
121	0\\
122	0\\
123	0\\
124	0\\
125	0\\
126	0\\
127	0\\
128	0\\
129	0\\
130	0\\
131	0\\
132	0\\
133	0\\
134	0\\
135	0\\
136	0\\
137	0\\
138	0\\
139	0\\
140	0\\
141	0\\
142	0\\
143	0\\
144	0\\
145	0\\
146	0\\
147	0\\
148	0\\
149	0\\
150	0\\
151	0\\
152	0\\
153	0\\
154	0\\
155	0\\
156	0\\
157	0\\
158	0\\
159	0\\
160	0\\
161	0\\
162	0\\
163	0\\
164	0\\
165	0\\
166	0\\
167	0\\
168	0\\
169	0\\
170	0\\
171	0\\
172	0\\
173	0\\
174	0\\
175	0\\
176	0\\
177	0\\
178	0\\
179	0\\
180	0\\
181	0\\
182	0\\
183	0\\
184	0\\
185	0\\
186	0\\
187	0\\
188	0\\
189	0\\
190	0\\
191	0\\
192	0\\
193	0\\
194	0\\
195	0\\
196	0\\
197	0\\
198	0\\
199	0\\
200	0\\
201	0\\
202	0\\
203	0\\
204	0\\
205	0\\
206	0\\
207	0\\
208	0\\
209	0\\
210	0\\
211	0\\
212	0\\
213	0\\
214	0\\
215	0\\
216	0\\
217	0\\
218	0\\
219	0\\
220	0\\
221	0\\
222	0\\
223	0\\
224	0\\
225	0\\
226	0\\
227	0\\
228	0\\
229	0\\
230	0\\
231	0\\
232	0\\
233	0\\
234	0\\
235	0\\
236	0\\
237	0\\
238	0\\
239	0\\
240	0\\
241	0\\
242	0\\
243	0\\
244	0\\
245	0\\
246	0\\
247	0\\
248	0\\
249	0\\
250	0\\
251	0\\
252	0\\
253	0\\
254	0\\
255	0\\
256	0\\
257	0\\
258	0\\
259	0\\
260	0\\
261	0\\
262	0\\
263	0\\
264	0\\
265	0\\
266	0\\
267	0\\
268	0\\
269	0\\
270	0\\
271	0\\
272	0\\
273	0\\
274	0\\
275	0\\
276	0\\
277	0\\
278	0\\
279	0\\
280	0\\
281	0\\
282	0\\
283	0\\
284	0\\
285	0\\
286	0\\
287	0\\
288	0\\
289	0\\
290	0\\
291	0\\
292	0\\
293	0\\
294	0\\
295	0\\
296	0\\
297	0\\
298	0\\
299	0\\
300	0\\
301	0\\
302	0\\
303	0\\
304	0\\
305	0\\
306	0\\
307	0\\
308	0\\
309	0\\
310	0\\
311	0\\
312	0\\
313	0\\
314	0\\
315	0\\
316	0\\
317	0\\
318	0\\
319	0\\
320	0\\
321	0\\
322	0\\
323	0\\
324	0\\
325	0\\
326	0\\
327	0\\
328	0\\
329	0\\
330	0\\
331	0\\
332	0\\
333	0\\
334	0\\
335	0\\
336	0\\
337	0\\
338	0\\
339	0\\
340	0\\
341	0\\
342	0\\
343	0\\
344	0\\
345	0\\
346	0\\
347	0\\
348	0\\
349	0\\
350	0\\
351	0\\
352	0\\
353	0\\
354	0\\
355	0\\
356	0\\
357	0\\
358	0\\
359	0\\
360	0\\
361	0\\
362	0\\
363	0\\
364	0\\
365	0\\
366	0\\
367	0\\
368	0\\
369	0\\
370	0\\
371	0\\
372	0\\
373	0\\
374	0\\
375	0\\
376	0\\
377	0\\
378	0\\
379	0\\
380	0\\
381	0\\
382	0\\
383	0\\
384	0\\
385	0\\
386	0\\
387	0\\
388	0\\
389	0\\
390	0\\
391	0\\
392	0\\
393	0\\
394	0\\
395	0\\
396	0\\
397	0\\
398	0\\
399	0\\
400	0\\
401	0\\
402	0\\
403	0\\
404	0\\
405	0\\
406	0\\
407	0\\
408	0\\
409	0\\
410	0\\
411	0\\
412	0\\
413	0\\
414	0\\
415	0\\
416	0\\
417	0\\
418	0\\
419	0\\
420	0\\
421	0\\
422	0\\
423	0\\
424	0\\
425	0\\
426	0\\
427	0\\
428	0\\
429	0\\
430	0\\
431	0\\
432	0\\
433	0\\
434	0\\
435	0\\
436	0\\
437	0\\
438	0\\
439	0\\
440	0\\
441	0\\
442	0\\
443	0\\
444	0\\
445	0\\
446	0\\
447	0\\
448	0\\
449	0\\
450	0\\
451	0\\
452	0\\
453	0\\
454	0\\
455	0\\
456	0\\
457	0\\
458	0\\
459	0\\
460	0\\
461	0\\
462	0\\
463	0\\
464	0\\
465	0\\
466	0\\
467	0\\
468	0\\
469	0\\
470	0\\
471	0\\
472	0\\
473	0\\
474	0\\
475	0\\
476	0\\
477	0\\
478	0\\
479	0\\
480	0\\
481	0\\
482	0\\
483	0\\
484	0\\
485	0\\
486	0\\
487	0\\
488	0\\
489	0\\
490	0\\
491	0\\
492	0\\
493	0\\
494	0\\
495	0\\
496	0\\
497	0\\
498	0\\
499	0\\
500	0\\
501	0\\
502	0\\
503	0\\
504	0\\
505	0\\
506	0\\
507	0\\
508	0\\
509	0\\
510	0\\
511	0\\
512	0\\
513	0\\
514	0\\
515	0\\
516	0\\
517	0\\
518	0\\
519	0\\
520	0\\
521	0\\
522	0\\
523	0\\
524	0\\
525	0\\
526	0\\
527	0\\
528	0\\
529	0\\
530	0\\
531	0\\
532	0\\
533	0\\
534	0\\
535	0\\
536	0\\
537	0\\
538	0\\
539	0\\
540	0\\
541	0\\
542	0\\
543	0\\
544	0\\
545	0\\
546	0\\
547	0\\
548	0\\
549	0\\
550	0\\
551	0\\
552	0\\
553	0\\
554	0\\
555	0\\
556	0\\
557	0\\
558	0.000110242963623734\\
559	0.00019992901139366\\
560	0.000287549546867388\\
561	0.000371614328687814\\
562	0.000450389472560112\\
563	0.000523067401835359\\
564	0.000586315516345723\\
565	0.000645586600008865\\
566	0.000703925763667472\\
567	0.000761855676457282\\
568	0.000819911089405896\\
569	0.000877192507218882\\
570	0.000934343502707047\\
571	0.000990382564388671\\
572	0.00104522036747947\\
573	0.00110068131805173\\
574	0.00115976594531527\\
575	0.0012203982545827\\
576	0.00128005598982308\\
577	0.00133517164128012\\
578	0.00138993979421614\\
579	0.00144525306241601\\
580	0.0016674712958016\\
581	0.00217281737767825\\
582	0.0025404360521811\\
583	0.00266811515796659\\
584	0.00278469419261029\\
585	0.00287171693370239\\
586	0.00295919258138545\\
587	0.00304791879573238\\
588	0.00313830670140735\\
589	0.00323059700557465\\
590	0.00332520698609584\\
591	0.00342286098677809\\
592	0.00352518255333602\\
593	0.00363635235366164\\
594	0.00376728283869332\\
595	0.00394659005851719\\
596	0.00424941998802836\\
597	0.00487316246495455\\
598	0.00633625614039688\\
599	0\\
600	0\\
};
\addplot [color=blue!80!mycolor9,solid,forget plot]
  table[row sep=crcr]{%
1	0\\
2	0\\
3	0\\
4	0\\
5	0\\
6	0\\
7	0\\
8	0\\
9	0\\
10	0\\
11	0\\
12	0\\
13	0\\
14	0\\
15	0\\
16	0\\
17	0\\
18	0\\
19	0\\
20	0\\
21	0\\
22	0\\
23	0\\
24	0\\
25	0\\
26	0\\
27	0\\
28	0\\
29	0\\
30	0\\
31	0\\
32	0\\
33	0\\
34	0\\
35	0\\
36	0\\
37	0\\
38	0\\
39	0\\
40	0\\
41	0\\
42	0\\
43	0\\
44	0\\
45	0\\
46	0\\
47	0\\
48	0\\
49	0\\
50	0\\
51	0\\
52	0\\
53	0\\
54	0\\
55	0\\
56	0\\
57	0\\
58	0\\
59	0\\
60	0\\
61	0\\
62	0\\
63	0\\
64	0\\
65	0\\
66	0\\
67	0\\
68	0\\
69	0\\
70	0\\
71	0\\
72	0\\
73	0\\
74	0\\
75	0\\
76	0\\
77	0\\
78	0\\
79	0\\
80	0\\
81	0\\
82	0\\
83	0\\
84	0\\
85	0\\
86	0\\
87	0\\
88	0\\
89	0\\
90	0\\
91	0\\
92	0\\
93	0\\
94	0\\
95	0\\
96	0\\
97	0\\
98	0\\
99	0\\
100	0\\
101	0\\
102	0\\
103	0\\
104	0\\
105	0\\
106	0\\
107	0\\
108	0\\
109	0\\
110	0\\
111	0\\
112	0\\
113	0\\
114	0\\
115	0\\
116	0\\
117	0\\
118	0\\
119	0\\
120	0\\
121	0\\
122	0\\
123	0\\
124	0\\
125	0\\
126	0\\
127	0\\
128	0\\
129	0\\
130	0\\
131	0\\
132	0\\
133	0\\
134	0\\
135	0\\
136	0\\
137	0\\
138	0\\
139	0\\
140	0\\
141	0\\
142	0\\
143	0\\
144	0\\
145	0\\
146	0\\
147	0\\
148	0\\
149	0\\
150	0\\
151	0\\
152	0\\
153	0\\
154	0\\
155	0\\
156	0\\
157	0\\
158	0\\
159	0\\
160	0\\
161	0\\
162	0\\
163	0\\
164	0\\
165	0\\
166	0\\
167	0\\
168	0\\
169	0\\
170	0\\
171	0\\
172	0\\
173	0\\
174	0\\
175	0\\
176	0\\
177	0\\
178	0\\
179	0\\
180	0\\
181	0\\
182	0\\
183	0\\
184	0\\
185	0\\
186	0\\
187	0\\
188	0\\
189	0\\
190	0\\
191	0\\
192	0\\
193	0\\
194	0\\
195	0\\
196	0\\
197	0\\
198	0\\
199	0\\
200	0\\
201	0\\
202	0\\
203	0\\
204	0\\
205	0\\
206	0\\
207	0\\
208	0\\
209	0\\
210	0\\
211	0\\
212	0\\
213	0\\
214	0\\
215	0\\
216	0\\
217	0\\
218	0\\
219	0\\
220	0\\
221	0\\
222	0\\
223	0\\
224	0\\
225	0\\
226	0\\
227	0\\
228	0\\
229	0\\
230	0\\
231	0\\
232	0\\
233	0\\
234	0\\
235	0\\
236	0\\
237	0\\
238	0\\
239	0\\
240	0\\
241	0\\
242	0\\
243	0\\
244	0\\
245	0\\
246	0\\
247	0\\
248	0\\
249	0\\
250	0\\
251	0\\
252	0\\
253	0\\
254	0\\
255	0\\
256	0\\
257	0\\
258	0\\
259	0\\
260	0\\
261	0\\
262	0\\
263	0\\
264	0\\
265	0\\
266	0\\
267	0\\
268	0\\
269	0\\
270	0\\
271	0\\
272	0\\
273	0\\
274	0\\
275	0\\
276	0\\
277	0\\
278	0\\
279	0\\
280	0\\
281	0\\
282	0\\
283	0\\
284	0\\
285	0\\
286	0\\
287	0\\
288	0\\
289	0\\
290	0\\
291	0\\
292	0\\
293	0\\
294	0\\
295	0\\
296	0\\
297	0\\
298	0\\
299	0\\
300	0\\
301	0\\
302	0\\
303	0\\
304	0\\
305	0\\
306	0\\
307	0\\
308	0\\
309	0\\
310	0\\
311	0\\
312	0\\
313	0\\
314	0\\
315	0\\
316	0\\
317	0\\
318	0\\
319	0\\
320	0\\
321	0\\
322	0\\
323	0\\
324	0\\
325	0\\
326	0\\
327	0\\
328	0\\
329	0\\
330	0\\
331	0\\
332	0\\
333	0\\
334	0\\
335	0\\
336	0\\
337	0\\
338	0\\
339	0\\
340	0\\
341	0\\
342	0\\
343	0\\
344	0\\
345	0\\
346	0\\
347	0\\
348	0\\
349	0\\
350	0\\
351	0\\
352	0\\
353	0\\
354	0\\
355	0\\
356	0\\
357	0\\
358	0\\
359	0\\
360	0\\
361	0\\
362	0\\
363	0\\
364	0\\
365	0\\
366	0\\
367	0\\
368	0\\
369	0\\
370	0\\
371	0\\
372	0\\
373	0\\
374	0\\
375	0\\
376	0\\
377	0\\
378	0\\
379	0\\
380	0\\
381	0\\
382	0\\
383	0\\
384	0\\
385	0\\
386	0\\
387	0\\
388	0\\
389	0\\
390	0\\
391	0\\
392	0\\
393	0\\
394	0\\
395	0\\
396	0\\
397	0\\
398	0\\
399	0\\
400	0\\
401	0\\
402	0\\
403	0\\
404	0\\
405	0\\
406	0\\
407	0\\
408	0\\
409	0\\
410	0\\
411	0\\
412	0\\
413	0\\
414	0\\
415	0\\
416	0\\
417	0\\
418	0\\
419	0\\
420	0\\
421	0\\
422	0\\
423	0\\
424	0\\
425	0\\
426	0\\
427	0\\
428	0\\
429	0\\
430	0\\
431	0\\
432	0\\
433	0\\
434	0\\
435	0\\
436	0\\
437	0\\
438	0\\
439	0\\
440	0\\
441	0\\
442	0\\
443	0\\
444	0\\
445	0\\
446	0\\
447	0\\
448	0\\
449	0\\
450	0\\
451	0\\
452	0\\
453	0\\
454	0\\
455	0\\
456	0\\
457	0\\
458	0\\
459	0\\
460	0\\
461	0\\
462	0\\
463	0\\
464	0\\
465	0\\
466	0\\
467	0\\
468	0\\
469	0\\
470	0\\
471	0\\
472	0\\
473	0\\
474	0\\
475	0\\
476	0\\
477	0\\
478	0\\
479	0\\
480	0\\
481	0\\
482	0\\
483	0\\
484	0\\
485	0\\
486	0\\
487	0\\
488	0\\
489	0\\
490	0\\
491	0\\
492	0\\
493	0\\
494	0\\
495	0\\
496	0\\
497	0\\
498	0\\
499	0\\
500	0\\
501	0\\
502	0\\
503	0\\
504	0\\
505	0\\
506	0\\
507	0\\
508	0\\
509	0\\
510	0\\
511	0\\
512	0\\
513	0\\
514	0\\
515	0\\
516	0\\
517	0\\
518	0\\
519	0\\
520	0\\
521	0\\
522	0\\
523	0\\
524	0\\
525	0\\
526	0\\
527	0\\
528	0\\
529	0\\
530	0\\
531	0\\
532	0\\
533	0\\
534	0\\
535	0\\
536	0\\
537	0\\
538	0\\
539	0\\
540	0\\
541	0\\
542	0\\
543	0\\
544	0\\
545	0\\
546	0\\
547	0\\
548	0\\
549	0\\
550	0\\
551	0\\
552	0\\
553	0\\
554	0\\
555	0\\
556	0\\
557	0\\
558	0\\
559	0\\
560	0\\
561	0\\
562	0\\
563	0.000159515699674089\\
564	0.000298612314061124\\
565	0.000395677995910523\\
566	0.000490056265484305\\
567	0.00058073290802797\\
568	0.000667074587023057\\
569	0.000745398135148641\\
570	0.000813989669797523\\
571	0.000882939921427012\\
572	0.000952276485464555\\
573	0.00101907934158752\\
574	0.00108458248882874\\
575	0.00114854952533438\\
576	0.00121114417608812\\
577	0.00127807087113535\\
578	0.00134469341072086\\
579	0.00141007143413669\\
580	0.00147057610994654\\
581	0.0015307007770952\\
582	0.00175408882179572\\
583	0.00224638378967943\\
584	0.00268185823056266\\
585	0.00281454711196381\\
586	0.00294329313072073\\
587	0.0030443100960092\\
588	0.00313678597680714\\
589	0.00322978330798573\\
590	0.00332482940415523\\
591	0.00342270866919969\\
592	0.00352513673636771\\
593	0.00363634439388895\\
594	0.00376728283869332\\
595	0.00394659005851719\\
596	0.00424941998802837\\
597	0.00487316246495455\\
598	0.00633625614039688\\
599	0\\
600	0\\
};
\addplot [color=blue,solid,forget plot]
  table[row sep=crcr]{%
1	0\\
2	0\\
3	0\\
4	0\\
5	0\\
6	0\\
7	0\\
8	0\\
9	0\\
10	0\\
11	0\\
12	0\\
13	0\\
14	0\\
15	0\\
16	0\\
17	0\\
18	0\\
19	0\\
20	0\\
21	0\\
22	0\\
23	0\\
24	0\\
25	0\\
26	0\\
27	0\\
28	0\\
29	0\\
30	0\\
31	0\\
32	0\\
33	0\\
34	0\\
35	0\\
36	0\\
37	0\\
38	0\\
39	0\\
40	0\\
41	0\\
42	0\\
43	0\\
44	0\\
45	0\\
46	0\\
47	0\\
48	0\\
49	0\\
50	0\\
51	0\\
52	0\\
53	0\\
54	0\\
55	0\\
56	0\\
57	0\\
58	0\\
59	0\\
60	0\\
61	0\\
62	0\\
63	0\\
64	0\\
65	0\\
66	0\\
67	0\\
68	0\\
69	0\\
70	0\\
71	0\\
72	0\\
73	0\\
74	0\\
75	0\\
76	0\\
77	0\\
78	0\\
79	0\\
80	0\\
81	0\\
82	0\\
83	0\\
84	0\\
85	0\\
86	0\\
87	0\\
88	0\\
89	0\\
90	0\\
91	0\\
92	0\\
93	0\\
94	0\\
95	0\\
96	0\\
97	0\\
98	0\\
99	0\\
100	0\\
101	0\\
102	0\\
103	0\\
104	0\\
105	0\\
106	0\\
107	0\\
108	0\\
109	0\\
110	0\\
111	0\\
112	0\\
113	0\\
114	0\\
115	0\\
116	0\\
117	0\\
118	0\\
119	0\\
120	0\\
121	0\\
122	0\\
123	0\\
124	0\\
125	0\\
126	0\\
127	0\\
128	0\\
129	0\\
130	0\\
131	0\\
132	0\\
133	0\\
134	0\\
135	0\\
136	0\\
137	0\\
138	0\\
139	0\\
140	0\\
141	0\\
142	0\\
143	0\\
144	0\\
145	0\\
146	0\\
147	0\\
148	0\\
149	0\\
150	0\\
151	0\\
152	0\\
153	0\\
154	0\\
155	0\\
156	0\\
157	0\\
158	0\\
159	0\\
160	0\\
161	0\\
162	0\\
163	0\\
164	0\\
165	0\\
166	0\\
167	0\\
168	0\\
169	0\\
170	0\\
171	0\\
172	0\\
173	0\\
174	0\\
175	0\\
176	0\\
177	0\\
178	0\\
179	0\\
180	0\\
181	0\\
182	0\\
183	0\\
184	0\\
185	0\\
186	0\\
187	0\\
188	0\\
189	0\\
190	0\\
191	0\\
192	0\\
193	0\\
194	0\\
195	0\\
196	0\\
197	0\\
198	0\\
199	0\\
200	0\\
201	0\\
202	0\\
203	0\\
204	0\\
205	0\\
206	0\\
207	0\\
208	0\\
209	0\\
210	0\\
211	0\\
212	0\\
213	0\\
214	0\\
215	0\\
216	0\\
217	0\\
218	0\\
219	0\\
220	0\\
221	0\\
222	0\\
223	0\\
224	0\\
225	0\\
226	0\\
227	0\\
228	0\\
229	0\\
230	0\\
231	0\\
232	0\\
233	0\\
234	0\\
235	0\\
236	0\\
237	0\\
238	0\\
239	0\\
240	0\\
241	0\\
242	0\\
243	0\\
244	0\\
245	0\\
246	0\\
247	0\\
248	0\\
249	0\\
250	0\\
251	0\\
252	0\\
253	0\\
254	0\\
255	0\\
256	0\\
257	0\\
258	0\\
259	0\\
260	0\\
261	0\\
262	0\\
263	0\\
264	0\\
265	0\\
266	0\\
267	0\\
268	0\\
269	0\\
270	0\\
271	0\\
272	0\\
273	0\\
274	0\\
275	0\\
276	0\\
277	0\\
278	0\\
279	0\\
280	0\\
281	0\\
282	0\\
283	0\\
284	0\\
285	0\\
286	0\\
287	0\\
288	0\\
289	0\\
290	0\\
291	0\\
292	0\\
293	0\\
294	0\\
295	0\\
296	0\\
297	0\\
298	0\\
299	0\\
300	0\\
301	0\\
302	0\\
303	0\\
304	0\\
305	0\\
306	0\\
307	0\\
308	0\\
309	0\\
310	0\\
311	0\\
312	0\\
313	0\\
314	0\\
315	0\\
316	0\\
317	0\\
318	0\\
319	0\\
320	0\\
321	0\\
322	0\\
323	0\\
324	0\\
325	0\\
326	0\\
327	0\\
328	0\\
329	0\\
330	0\\
331	0\\
332	0\\
333	0\\
334	0\\
335	0\\
336	0\\
337	0\\
338	0\\
339	0\\
340	0\\
341	0\\
342	0\\
343	0\\
344	0\\
345	0\\
346	0\\
347	0\\
348	0\\
349	0\\
350	0\\
351	0\\
352	0\\
353	0\\
354	0\\
355	0\\
356	0\\
357	0\\
358	0\\
359	0\\
360	0\\
361	0\\
362	0\\
363	0\\
364	0\\
365	0\\
366	0\\
367	0\\
368	0\\
369	0\\
370	0\\
371	0\\
372	0\\
373	0\\
374	0\\
375	0\\
376	0\\
377	0\\
378	0\\
379	0\\
380	0\\
381	0\\
382	0\\
383	0\\
384	0\\
385	0\\
386	0\\
387	0\\
388	0\\
389	0\\
390	0\\
391	0\\
392	0\\
393	0\\
394	0\\
395	0\\
396	0\\
397	0\\
398	0\\
399	0\\
400	0\\
401	0\\
402	0\\
403	0\\
404	0\\
405	0\\
406	0\\
407	0\\
408	0\\
409	0\\
410	0\\
411	0\\
412	0\\
413	0\\
414	0\\
415	0\\
416	0\\
417	0\\
418	0\\
419	0\\
420	0\\
421	0\\
422	0\\
423	0\\
424	0\\
425	0\\
426	0\\
427	0\\
428	0\\
429	0\\
430	0\\
431	0\\
432	0\\
433	0\\
434	0\\
435	0\\
436	0\\
437	0\\
438	0\\
439	0\\
440	0\\
441	0\\
442	0\\
443	0\\
444	0\\
445	0\\
446	0\\
447	0\\
448	0\\
449	0\\
450	0\\
451	0\\
452	0\\
453	0\\
454	0\\
455	0\\
456	0\\
457	0\\
458	0\\
459	0\\
460	0\\
461	0\\
462	0\\
463	0\\
464	0\\
465	0\\
466	0\\
467	0\\
468	0\\
469	0\\
470	0\\
471	0\\
472	0\\
473	0\\
474	0\\
475	0\\
476	0\\
477	0\\
478	0\\
479	0\\
480	0\\
481	0\\
482	0\\
483	0\\
484	0\\
485	0\\
486	0\\
487	0\\
488	0\\
489	0\\
490	0\\
491	0\\
492	0\\
493	0\\
494	0\\
495	0\\
496	0\\
497	0\\
498	0\\
499	0\\
500	0\\
501	0\\
502	0\\
503	0\\
504	0\\
505	0\\
506	0\\
507	0\\
508	0\\
509	0\\
510	0\\
511	0\\
512	0\\
513	0\\
514	0\\
515	0\\
516	0\\
517	0\\
518	0\\
519	0\\
520	0\\
521	0\\
522	0\\
523	0\\
524	0\\
525	0\\
526	0\\
527	0\\
528	0\\
529	0\\
530	0\\
531	0\\
532	0\\
533	0\\
534	0\\
535	0\\
536	0\\
537	0\\
538	0\\
539	0\\
540	0\\
541	0\\
542	0\\
543	0\\
544	0\\
545	0\\
546	0\\
547	0\\
548	0\\
549	0\\
550	0\\
551	0\\
552	0\\
553	0\\
554	0\\
555	0\\
556	0\\
557	0\\
558	0\\
559	0\\
560	0\\
561	0\\
562	0\\
563	0\\
564	0\\
565	0\\
566	0\\
567	0\\
568	0.000141273876810025\\
569	0.000370678299245233\\
570	0.000478798156568623\\
571	0.000585374459440977\\
572	0.000689800406590731\\
573	0.000792165578853567\\
574	0.000889405598368621\\
575	0.000979616328357942\\
576	0.00106208975066892\\
577	0.00114339711980862\\
578	0.00122278879802299\\
579	0.00130020813270453\\
580	0.00137791300105533\\
581	0.0014543721388116\\
582	0.00152928169195912\\
583	0.00160026462104487\\
584	0.00174508295417719\\
585	0.00222608820292352\\
586	0.00271976975471917\\
587	0.00294688553227509\\
588	0.00308669213329453\\
589	0.003219110126369\\
590	0.0033189542078428\\
591	0.00342012153578412\\
592	0.00352397614616107\\
593	0.00363595699515989\\
594	0.00376720386728385\\
595	0.00394659005851719\\
596	0.00424941998802836\\
597	0.00487316246495455\\
598	0.00633625614039688\\
599	0\\
600	0\\
};
\addplot [color=mycolor10,solid,forget plot]
  table[row sep=crcr]{%
1	0\\
2	0\\
3	0\\
4	0\\
5	0\\
6	0\\
7	0\\
8	0\\
9	0\\
10	0\\
11	0\\
12	0\\
13	0\\
14	0\\
15	0\\
16	0\\
17	0\\
18	0\\
19	0\\
20	0\\
21	0\\
22	0\\
23	0\\
24	0\\
25	0\\
26	0\\
27	0\\
28	0\\
29	0\\
30	0\\
31	0\\
32	0\\
33	0\\
34	0\\
35	0\\
36	0\\
37	0\\
38	0\\
39	0\\
40	0\\
41	0\\
42	0\\
43	0\\
44	0\\
45	0\\
46	0\\
47	0\\
48	0\\
49	0\\
50	0\\
51	0\\
52	0\\
53	0\\
54	0\\
55	0\\
56	0\\
57	0\\
58	0\\
59	0\\
60	0\\
61	0\\
62	0\\
63	0\\
64	0\\
65	0\\
66	0\\
67	0\\
68	0\\
69	0\\
70	0\\
71	0\\
72	0\\
73	0\\
74	0\\
75	0\\
76	0\\
77	0\\
78	0\\
79	0\\
80	0\\
81	0\\
82	0\\
83	0\\
84	0\\
85	0\\
86	0\\
87	0\\
88	0\\
89	0\\
90	0\\
91	0\\
92	0\\
93	0\\
94	0\\
95	0\\
96	0\\
97	0\\
98	0\\
99	0\\
100	0\\
101	0\\
102	0\\
103	0\\
104	0\\
105	0\\
106	0\\
107	0\\
108	0\\
109	0\\
110	0\\
111	0\\
112	0\\
113	0\\
114	0\\
115	0\\
116	0\\
117	0\\
118	0\\
119	0\\
120	0\\
121	0\\
122	0\\
123	0\\
124	0\\
125	0\\
126	0\\
127	0\\
128	0\\
129	0\\
130	0\\
131	0\\
132	0\\
133	0\\
134	0\\
135	0\\
136	0\\
137	0\\
138	0\\
139	0\\
140	0\\
141	0\\
142	0\\
143	0\\
144	0\\
145	0\\
146	0\\
147	0\\
148	0\\
149	0\\
150	0\\
151	0\\
152	0\\
153	0\\
154	0\\
155	0\\
156	0\\
157	0\\
158	0\\
159	0\\
160	0\\
161	0\\
162	0\\
163	0\\
164	0\\
165	0\\
166	0\\
167	0\\
168	0\\
169	0\\
170	0\\
171	0\\
172	0\\
173	0\\
174	0\\
175	0\\
176	0\\
177	0\\
178	0\\
179	0\\
180	0\\
181	0\\
182	0\\
183	0\\
184	0\\
185	0\\
186	0\\
187	0\\
188	0\\
189	0\\
190	0\\
191	0\\
192	0\\
193	0\\
194	0\\
195	0\\
196	0\\
197	0\\
198	0\\
199	0\\
200	0\\
201	0\\
202	0\\
203	0\\
204	0\\
205	0\\
206	0\\
207	0\\
208	0\\
209	0\\
210	0\\
211	0\\
212	0\\
213	0\\
214	0\\
215	0\\
216	0\\
217	0\\
218	0\\
219	0\\
220	0\\
221	0\\
222	0\\
223	0\\
224	0\\
225	0\\
226	0\\
227	0\\
228	0\\
229	0\\
230	0\\
231	0\\
232	0\\
233	0\\
234	0\\
235	0\\
236	0\\
237	0\\
238	0\\
239	0\\
240	0\\
241	0\\
242	0\\
243	0\\
244	0\\
245	0\\
246	0\\
247	0\\
248	0\\
249	0\\
250	0\\
251	0\\
252	0\\
253	0\\
254	0\\
255	0\\
256	0\\
257	0\\
258	0\\
259	0\\
260	0\\
261	0\\
262	0\\
263	0\\
264	0\\
265	0\\
266	0\\
267	0\\
268	0\\
269	0\\
270	0\\
271	0\\
272	0\\
273	0\\
274	0\\
275	0\\
276	0\\
277	0\\
278	0\\
279	0\\
280	0\\
281	0\\
282	0\\
283	0\\
284	0\\
285	0\\
286	0\\
287	0\\
288	0\\
289	0\\
290	0\\
291	0\\
292	0\\
293	0\\
294	0\\
295	0\\
296	0\\
297	0\\
298	0\\
299	0\\
300	0\\
301	0\\
302	0\\
303	0\\
304	0\\
305	0\\
306	0\\
307	0\\
308	0\\
309	0\\
310	0\\
311	0\\
312	0\\
313	0\\
314	0\\
315	0\\
316	0\\
317	0\\
318	0\\
319	0\\
320	0\\
321	0\\
322	0\\
323	0\\
324	0\\
325	0\\
326	0\\
327	0\\
328	0\\
329	0\\
330	0\\
331	0\\
332	0\\
333	0\\
334	0\\
335	0\\
336	0\\
337	0\\
338	0\\
339	0\\
340	0\\
341	0\\
342	0\\
343	0\\
344	0\\
345	0\\
346	0\\
347	0\\
348	0\\
349	0\\
350	0\\
351	0\\
352	0\\
353	0\\
354	0\\
355	0\\
356	0\\
357	0\\
358	0\\
359	0\\
360	0\\
361	0\\
362	0\\
363	0\\
364	0\\
365	0\\
366	0\\
367	0\\
368	0\\
369	0\\
370	0\\
371	0\\
372	0\\
373	0\\
374	0\\
375	0\\
376	0\\
377	0\\
378	0\\
379	0\\
380	0\\
381	0\\
382	0\\
383	0\\
384	0\\
385	0\\
386	0\\
387	0\\
388	0\\
389	0\\
390	0\\
391	0\\
392	0\\
393	0\\
394	0\\
395	0\\
396	0\\
397	0\\
398	0\\
399	0\\
400	0\\
401	0\\
402	0\\
403	0\\
404	0\\
405	0\\
406	0\\
407	0\\
408	0\\
409	0\\
410	0\\
411	0\\
412	0\\
413	0\\
414	0\\
415	0\\
416	0\\
417	0\\
418	0\\
419	0\\
420	0\\
421	0\\
422	0\\
423	0\\
424	0\\
425	0\\
426	0\\
427	0\\
428	0\\
429	0\\
430	0\\
431	0\\
432	0\\
433	0\\
434	0\\
435	0\\
436	0\\
437	0\\
438	0\\
439	0\\
440	0\\
441	0\\
442	0\\
443	0\\
444	0\\
445	0\\
446	0\\
447	0\\
448	0\\
449	0\\
450	0\\
451	0\\
452	0\\
453	0\\
454	0\\
455	0\\
456	0\\
457	0\\
458	0\\
459	0\\
460	0\\
461	0\\
462	0\\
463	0\\
464	0\\
465	0\\
466	0\\
467	0\\
468	0\\
469	0\\
470	0\\
471	0\\
472	0\\
473	0\\
474	0\\
475	0\\
476	0\\
477	0\\
478	0\\
479	0\\
480	0\\
481	0\\
482	0\\
483	0\\
484	0\\
485	0\\
486	0\\
487	0\\
488	0\\
489	0\\
490	0\\
491	0\\
492	0\\
493	0\\
494	0\\
495	0\\
496	0\\
497	0\\
498	0\\
499	0\\
500	0\\
501	0\\
502	0\\
503	0\\
504	0\\
505	0\\
506	0\\
507	0\\
508	0\\
509	0\\
510	0\\
511	0\\
512	0\\
513	0\\
514	0\\
515	0\\
516	0\\
517	0\\
518	0\\
519	0\\
520	0\\
521	0\\
522	0\\
523	0\\
524	0\\
525	0\\
526	0\\
527	0\\
528	0\\
529	0\\
530	0\\
531	0\\
532	0\\
533	0\\
534	0\\
535	0\\
536	0\\
537	0\\
538	0\\
539	0\\
540	0\\
541	0\\
542	0\\
543	0\\
544	0\\
545	0\\
546	0\\
547	0\\
548	0\\
549	0\\
550	0\\
551	0\\
552	0\\
553	0\\
554	0\\
555	0\\
556	0\\
557	0\\
558	0\\
559	0\\
560	0\\
561	0\\
562	0\\
563	0\\
564	0\\
565	0\\
566	0\\
567	0\\
568	0\\
569	0\\
570	0\\
571	0\\
572	0\\
573	3.91930466837155e-05\\
574	0.000276356766258129\\
575	0.000514435338481855\\
576	0.00063669964778276\\
577	0.000758573950088116\\
578	0.000879483741525221\\
579	0.000998840951933042\\
580	0.00111633816343543\\
581	0.00122629809196345\\
582	0.00132652874047018\\
583	0.00142603296361657\\
584	0.00152331143743122\\
585	0.00161638870950575\\
586	0.00170662226027931\\
587	0.00208852306255227\\
588	0.00257153432767004\\
589	0.00304912288981168\\
590	0.0032031038023609\\
591	0.00335726982309828\\
592	0.0035053300163716\\
593	0.00362658088892234\\
594	0.00376388624932705\\
595	0.00394575354532189\\
596	0.00424941998802836\\
597	0.00487316246495455\\
598	0.00633625614039688\\
599	0\\
600	0\\
};
\addplot [color=mycolor11,solid,forget plot]
  table[row sep=crcr]{%
1	0\\
2	0\\
3	0\\
4	0\\
5	0\\
6	0\\
7	0\\
8	0\\
9	0\\
10	0\\
11	0\\
12	0\\
13	0\\
14	0\\
15	0\\
16	0\\
17	0\\
18	0\\
19	0\\
20	0\\
21	0\\
22	0\\
23	0\\
24	0\\
25	0\\
26	0\\
27	0\\
28	0\\
29	0\\
30	0\\
31	0\\
32	0\\
33	0\\
34	0\\
35	0\\
36	0\\
37	0\\
38	0\\
39	0\\
40	0\\
41	0\\
42	0\\
43	0\\
44	0\\
45	0\\
46	0\\
47	0\\
48	0\\
49	0\\
50	0\\
51	0\\
52	0\\
53	0\\
54	0\\
55	0\\
56	0\\
57	0\\
58	0\\
59	0\\
60	0\\
61	0\\
62	0\\
63	0\\
64	0\\
65	0\\
66	0\\
67	0\\
68	0\\
69	0\\
70	0\\
71	0\\
72	0\\
73	0\\
74	0\\
75	0\\
76	0\\
77	0\\
78	0\\
79	0\\
80	0\\
81	0\\
82	0\\
83	0\\
84	0\\
85	0\\
86	0\\
87	0\\
88	0\\
89	0\\
90	0\\
91	0\\
92	0\\
93	0\\
94	0\\
95	0\\
96	0\\
97	0\\
98	0\\
99	0\\
100	0\\
101	0\\
102	0\\
103	0\\
104	0\\
105	0\\
106	0\\
107	0\\
108	0\\
109	0\\
110	0\\
111	0\\
112	0\\
113	0\\
114	0\\
115	0\\
116	0\\
117	0\\
118	0\\
119	0\\
120	0\\
121	0\\
122	0\\
123	0\\
124	0\\
125	0\\
126	0\\
127	0\\
128	0\\
129	0\\
130	0\\
131	0\\
132	0\\
133	0\\
134	0\\
135	0\\
136	0\\
137	0\\
138	0\\
139	0\\
140	0\\
141	0\\
142	0\\
143	0\\
144	0\\
145	0\\
146	0\\
147	0\\
148	0\\
149	0\\
150	0\\
151	0\\
152	0\\
153	0\\
154	0\\
155	0\\
156	0\\
157	0\\
158	0\\
159	0\\
160	0\\
161	0\\
162	0\\
163	0\\
164	0\\
165	0\\
166	0\\
167	0\\
168	0\\
169	0\\
170	0\\
171	0\\
172	0\\
173	0\\
174	0\\
175	0\\
176	0\\
177	0\\
178	0\\
179	0\\
180	0\\
181	0\\
182	0\\
183	0\\
184	0\\
185	0\\
186	0\\
187	0\\
188	0\\
189	0\\
190	0\\
191	0\\
192	0\\
193	0\\
194	0\\
195	0\\
196	0\\
197	0\\
198	0\\
199	0\\
200	0\\
201	0\\
202	0\\
203	0\\
204	0\\
205	0\\
206	0\\
207	0\\
208	0\\
209	0\\
210	0\\
211	0\\
212	0\\
213	0\\
214	0\\
215	0\\
216	0\\
217	0\\
218	0\\
219	0\\
220	0\\
221	0\\
222	0\\
223	0\\
224	0\\
225	0\\
226	0\\
227	0\\
228	0\\
229	0\\
230	0\\
231	0\\
232	0\\
233	0\\
234	0\\
235	0\\
236	0\\
237	0\\
238	0\\
239	0\\
240	0\\
241	0\\
242	0\\
243	0\\
244	0\\
245	0\\
246	0\\
247	0\\
248	0\\
249	0\\
250	0\\
251	0\\
252	0\\
253	0\\
254	0\\
255	0\\
256	0\\
257	0\\
258	0\\
259	0\\
260	0\\
261	0\\
262	0\\
263	0\\
264	0\\
265	0\\
266	0\\
267	0\\
268	0\\
269	0\\
270	0\\
271	0\\
272	0\\
273	0\\
274	0\\
275	0\\
276	0\\
277	0\\
278	0\\
279	0\\
280	0\\
281	0\\
282	0\\
283	0\\
284	0\\
285	0\\
286	0\\
287	0\\
288	0\\
289	0\\
290	0\\
291	0\\
292	0\\
293	0\\
294	0\\
295	0\\
296	0\\
297	0\\
298	0\\
299	0\\
300	0\\
301	0\\
302	0\\
303	0\\
304	0\\
305	0\\
306	0\\
307	0\\
308	0\\
309	0\\
310	0\\
311	0\\
312	0\\
313	0\\
314	0\\
315	0\\
316	0\\
317	0\\
318	0\\
319	0\\
320	0\\
321	0\\
322	0\\
323	0\\
324	0\\
325	0\\
326	0\\
327	0\\
328	0\\
329	0\\
330	0\\
331	0\\
332	0\\
333	0\\
334	0\\
335	0\\
336	0\\
337	0\\
338	0\\
339	0\\
340	0\\
341	0\\
342	0\\
343	0\\
344	0\\
345	0\\
346	0\\
347	0\\
348	0\\
349	0\\
350	0\\
351	0\\
352	0\\
353	0\\
354	0\\
355	0\\
356	0\\
357	0\\
358	0\\
359	0\\
360	0\\
361	0\\
362	0\\
363	0\\
364	0\\
365	0\\
366	0\\
367	0\\
368	0\\
369	0\\
370	0\\
371	0\\
372	0\\
373	0\\
374	0\\
375	0\\
376	0\\
377	0\\
378	0\\
379	0\\
380	0\\
381	0\\
382	0\\
383	0\\
384	0\\
385	0\\
386	0\\
387	0\\
388	0\\
389	0\\
390	0\\
391	0\\
392	0\\
393	0\\
394	0\\
395	0\\
396	0\\
397	0\\
398	0\\
399	0\\
400	0\\
401	0\\
402	0\\
403	0\\
404	0\\
405	0\\
406	0\\
407	0\\
408	0\\
409	0\\
410	0\\
411	0\\
412	0\\
413	0\\
414	0\\
415	0\\
416	0\\
417	0\\
418	0\\
419	0\\
420	0\\
421	0\\
422	0\\
423	0\\
424	0\\
425	0\\
426	0\\
427	0\\
428	0\\
429	0\\
430	0\\
431	0\\
432	0\\
433	0\\
434	0\\
435	0\\
436	0\\
437	0\\
438	0\\
439	0\\
440	0\\
441	0\\
442	0\\
443	0\\
444	0\\
445	0\\
446	0\\
447	0\\
448	0\\
449	0\\
450	0\\
451	0\\
452	0\\
453	0\\
454	0\\
455	0\\
456	0\\
457	0\\
458	0\\
459	0\\
460	0\\
461	0\\
462	0\\
463	0\\
464	0\\
465	0\\
466	0\\
467	0\\
468	0\\
469	0\\
470	0\\
471	0\\
472	0\\
473	0\\
474	0\\
475	0\\
476	0\\
477	0\\
478	0\\
479	0\\
480	0\\
481	0\\
482	0\\
483	0\\
484	0\\
485	0\\
486	0\\
487	0\\
488	0\\
489	0\\
490	0\\
491	0\\
492	0\\
493	0\\
494	0\\
495	0\\
496	0\\
497	0\\
498	0\\
499	0\\
500	0\\
501	0\\
502	0\\
503	0\\
504	0\\
505	0\\
506	0\\
507	0\\
508	0\\
509	0\\
510	0\\
511	0\\
512	0\\
513	0\\
514	0\\
515	0\\
516	0\\
517	0\\
518	0\\
519	0\\
520	0\\
521	0\\
522	0\\
523	0\\
524	0\\
525	0\\
526	0\\
527	0\\
528	0\\
529	0\\
530	0\\
531	0\\
532	0\\
533	0\\
534	0\\
535	0\\
536	0\\
537	0\\
538	0\\
539	0\\
540	0\\
541	0\\
542	0\\
543	0\\
544	0\\
545	0\\
546	0\\
547	0\\
548	0\\
549	0\\
550	0\\
551	0\\
552	0\\
553	0\\
554	0\\
555	0\\
556	0\\
557	0\\
558	0\\
559	0\\
560	0\\
561	0\\
562	0\\
563	0\\
564	0\\
565	0\\
566	0\\
567	0\\
568	0\\
569	0\\
570	0\\
571	0\\
572	0\\
573	0\\
574	0\\
575	0\\
576	0\\
577	0\\
578	0\\
579	3.6352204945255e-05\\
580	0.000296758473508131\\
581	0.000566134643332366\\
582	0.000744728105586424\\
583	0.000891594524082745\\
584	0.0010379859245772\\
585	0.00118567476607632\\
586	0.00133151060223953\\
587	0.00147653786463307\\
588	0.00161254310062852\\
589	0.00175780084121125\\
590	0.00225707969882223\\
591	0.00276065377804058\\
592	0.00326067925967635\\
593	0.00345740326951338\\
594	0.00366867553292786\\
595	0.00391598560027283\\
596	0.00423940434301512\\
597	0.00487316246495455\\
598	0.00633625614039688\\
599	0\\
600	0\\
};
\addplot [color=mycolor12,solid,forget plot]
  table[row sep=crcr]{%
1	0\\
2	0\\
3	0\\
4	0\\
5	0\\
6	0\\
7	0\\
8	0\\
9	0\\
10	0\\
11	0\\
12	0\\
13	0\\
14	0\\
15	0\\
16	0\\
17	0\\
18	0\\
19	0\\
20	0\\
21	0\\
22	0\\
23	0\\
24	0\\
25	0\\
26	0\\
27	0\\
28	0\\
29	0\\
30	0\\
31	0\\
32	0\\
33	0\\
34	0\\
35	0\\
36	0\\
37	0\\
38	0\\
39	0\\
40	0\\
41	0\\
42	0\\
43	0\\
44	0\\
45	0\\
46	0\\
47	0\\
48	0\\
49	0\\
50	0\\
51	0\\
52	0\\
53	0\\
54	0\\
55	0\\
56	0\\
57	0\\
58	0\\
59	0\\
60	0\\
61	0\\
62	0\\
63	0\\
64	0\\
65	0\\
66	0\\
67	0\\
68	0\\
69	0\\
70	0\\
71	0\\
72	0\\
73	0\\
74	0\\
75	0\\
76	0\\
77	0\\
78	0\\
79	0\\
80	0\\
81	0\\
82	0\\
83	0\\
84	0\\
85	0\\
86	0\\
87	0\\
88	0\\
89	0\\
90	0\\
91	0\\
92	0\\
93	0\\
94	0\\
95	0\\
96	0\\
97	0\\
98	0\\
99	0\\
100	0\\
101	0\\
102	0\\
103	0\\
104	0\\
105	0\\
106	0\\
107	0\\
108	0\\
109	0\\
110	0\\
111	0\\
112	0\\
113	0\\
114	0\\
115	0\\
116	0\\
117	0\\
118	0\\
119	0\\
120	0\\
121	0\\
122	0\\
123	0\\
124	0\\
125	0\\
126	0\\
127	0\\
128	0\\
129	0\\
130	0\\
131	0\\
132	0\\
133	0\\
134	0\\
135	0\\
136	0\\
137	0\\
138	0\\
139	0\\
140	0\\
141	0\\
142	0\\
143	0\\
144	0\\
145	0\\
146	0\\
147	0\\
148	0\\
149	0\\
150	0\\
151	0\\
152	0\\
153	0\\
154	0\\
155	0\\
156	0\\
157	0\\
158	0\\
159	0\\
160	0\\
161	0\\
162	0\\
163	0\\
164	0\\
165	0\\
166	0\\
167	0\\
168	0\\
169	0\\
170	0\\
171	0\\
172	0\\
173	0\\
174	0\\
175	0\\
176	0\\
177	0\\
178	0\\
179	0\\
180	0\\
181	0\\
182	0\\
183	0\\
184	0\\
185	0\\
186	0\\
187	0\\
188	0\\
189	0\\
190	0\\
191	0\\
192	0\\
193	0\\
194	0\\
195	0\\
196	0\\
197	0\\
198	0\\
199	0\\
200	0\\
201	0\\
202	0\\
203	0\\
204	0\\
205	0\\
206	0\\
207	0\\
208	0\\
209	0\\
210	0\\
211	0\\
212	0\\
213	0\\
214	0\\
215	0\\
216	0\\
217	0\\
218	0\\
219	0\\
220	0\\
221	0\\
222	0\\
223	0\\
224	0\\
225	0\\
226	0\\
227	0\\
228	0\\
229	0\\
230	0\\
231	0\\
232	0\\
233	0\\
234	0\\
235	0\\
236	0\\
237	0\\
238	0\\
239	0\\
240	0\\
241	0\\
242	0\\
243	0\\
244	0\\
245	0\\
246	0\\
247	0\\
248	0\\
249	0\\
250	0\\
251	0\\
252	0\\
253	0\\
254	0\\
255	0\\
256	0\\
257	0\\
258	0\\
259	0\\
260	0\\
261	0\\
262	0\\
263	0\\
264	0\\
265	0\\
266	0\\
267	0\\
268	0\\
269	0\\
270	0\\
271	0\\
272	0\\
273	0\\
274	0\\
275	0\\
276	0\\
277	0\\
278	0\\
279	0\\
280	0\\
281	0\\
282	0\\
283	0\\
284	0\\
285	0\\
286	0\\
287	0\\
288	0\\
289	0\\
290	0\\
291	0\\
292	0\\
293	0\\
294	0\\
295	0\\
296	0\\
297	0\\
298	0\\
299	0\\
300	0\\
301	0\\
302	0\\
303	0\\
304	0\\
305	0\\
306	0\\
307	0\\
308	0\\
309	0\\
310	0\\
311	0\\
312	0\\
313	0\\
314	0\\
315	0\\
316	0\\
317	0\\
318	0\\
319	0\\
320	0\\
321	0\\
322	0\\
323	0\\
324	0\\
325	0\\
326	0\\
327	0\\
328	0\\
329	0\\
330	0\\
331	0\\
332	0\\
333	0\\
334	0\\
335	0\\
336	0\\
337	0\\
338	0\\
339	0\\
340	0\\
341	0\\
342	0\\
343	0\\
344	0\\
345	0\\
346	0\\
347	0\\
348	0\\
349	0\\
350	0\\
351	0\\
352	0\\
353	0\\
354	0\\
355	0\\
356	0\\
357	0\\
358	0\\
359	0\\
360	0\\
361	0\\
362	0\\
363	0\\
364	0\\
365	0\\
366	0\\
367	0\\
368	0\\
369	0\\
370	0\\
371	0\\
372	0\\
373	0\\
374	0\\
375	0\\
376	0\\
377	0\\
378	0\\
379	0\\
380	0\\
381	0\\
382	0\\
383	0\\
384	0\\
385	0\\
386	0\\
387	0\\
388	0\\
389	0\\
390	0\\
391	0\\
392	0\\
393	0\\
394	0\\
395	0\\
396	0\\
397	0\\
398	0\\
399	0\\
400	0\\
401	0\\
402	0\\
403	0\\
404	0\\
405	0\\
406	0\\
407	0\\
408	0\\
409	0\\
410	0\\
411	0\\
412	0\\
413	0\\
414	0\\
415	0\\
416	0\\
417	0\\
418	0\\
419	0\\
420	0\\
421	0\\
422	0\\
423	0\\
424	0\\
425	0\\
426	0\\
427	0\\
428	0\\
429	0\\
430	0\\
431	0\\
432	0\\
433	0\\
434	0\\
435	0\\
436	0\\
437	0\\
438	0\\
439	0\\
440	0\\
441	0\\
442	0\\
443	0\\
444	0\\
445	0\\
446	0\\
447	0\\
448	0\\
449	0\\
450	0\\
451	0\\
452	0\\
453	0\\
454	0\\
455	0\\
456	0\\
457	0\\
458	0\\
459	0\\
460	0\\
461	0\\
462	0\\
463	0\\
464	0\\
465	0\\
466	0\\
467	0\\
468	0\\
469	0\\
470	0\\
471	0\\
472	0\\
473	0\\
474	0\\
475	0\\
476	0\\
477	0\\
478	0\\
479	0\\
480	0\\
481	0\\
482	0\\
483	0\\
484	0\\
485	0\\
486	0\\
487	0\\
488	0\\
489	0\\
490	0\\
491	0\\
492	0\\
493	0\\
494	0\\
495	0\\
496	0\\
497	0\\
498	0\\
499	0\\
500	0\\
501	0\\
502	0\\
503	0\\
504	0\\
505	0\\
506	0\\
507	0\\
508	0\\
509	0\\
510	0\\
511	0\\
512	0\\
513	0\\
514	0\\
515	0\\
516	0\\
517	0\\
518	0\\
519	0\\
520	0\\
521	0\\
522	0\\
523	0\\
524	0\\
525	0\\
526	0\\
527	0\\
528	0\\
529	0\\
530	0\\
531	0\\
532	0\\
533	0\\
534	0\\
535	0\\
536	0\\
537	0\\
538	0\\
539	0\\
540	0\\
541	0\\
542	0\\
543	0\\
544	0\\
545	0\\
546	0\\
547	0\\
548	0\\
549	0\\
550	0\\
551	0\\
552	0\\
553	0\\
554	0\\
555	0\\
556	0\\
557	0\\
558	0\\
559	0\\
560	0\\
561	0\\
562	0\\
563	0\\
564	0\\
565	0\\
566	0\\
567	0\\
568	0\\
569	0\\
570	0\\
571	0\\
572	0\\
573	0\\
574	0\\
575	0\\
576	0\\
577	0\\
578	0\\
579	0\\
580	0\\
581	0\\
582	0\\
583	0\\
584	0\\
585	0\\
586	8.12200422505001e-05\\
587	0.000377265340553168\\
588	0.000688475577631225\\
589	0.00088764595963235\\
590	0.00108832691824831\\
591	0.00129672894429491\\
592	0.00153220976797035\\
593	0.00212558857692427\\
594	0.00276084955258031\\
595	0.00348703611492466\\
596	0.00396130651742771\\
597	0.00473923450745123\\
598	0.00633625614039688\\
599	0\\
600	0\\
};
\addplot [color=mycolor13,solid,forget plot]
  table[row sep=crcr]{%
1	0\\
2	0\\
3	0\\
4	0\\
5	0\\
6	0\\
7	0\\
8	0\\
9	0\\
10	0\\
11	0\\
12	0\\
13	0\\
14	0\\
15	0\\
16	0\\
17	0\\
18	0\\
19	0\\
20	0\\
21	0\\
22	0\\
23	0\\
24	0\\
25	0\\
26	0\\
27	0\\
28	0\\
29	0\\
30	0\\
31	0\\
32	0\\
33	0\\
34	0\\
35	0\\
36	0\\
37	0\\
38	0\\
39	0\\
40	0\\
41	0\\
42	0\\
43	0\\
44	0\\
45	0\\
46	0\\
47	0\\
48	0\\
49	0\\
50	0\\
51	0\\
52	0\\
53	0\\
54	0\\
55	0\\
56	0\\
57	0\\
58	0\\
59	0\\
60	0\\
61	0\\
62	0\\
63	0\\
64	0\\
65	0\\
66	0\\
67	0\\
68	0\\
69	0\\
70	0\\
71	0\\
72	0\\
73	0\\
74	0\\
75	0\\
76	0\\
77	0\\
78	0\\
79	0\\
80	0\\
81	0\\
82	0\\
83	0\\
84	0\\
85	0\\
86	0\\
87	0\\
88	0\\
89	0\\
90	0\\
91	0\\
92	0\\
93	0\\
94	0\\
95	0\\
96	0\\
97	0\\
98	0\\
99	0\\
100	0\\
101	0\\
102	0\\
103	0\\
104	0\\
105	0\\
106	0\\
107	0\\
108	0\\
109	0\\
110	0\\
111	0\\
112	0\\
113	0\\
114	0\\
115	0\\
116	0\\
117	0\\
118	0\\
119	0\\
120	0\\
121	0\\
122	0\\
123	0\\
124	0\\
125	0\\
126	0\\
127	0\\
128	0\\
129	0\\
130	0\\
131	0\\
132	0\\
133	0\\
134	0\\
135	0\\
136	0\\
137	0\\
138	0\\
139	0\\
140	0\\
141	0\\
142	0\\
143	0\\
144	0\\
145	0\\
146	0\\
147	0\\
148	0\\
149	0\\
150	0\\
151	0\\
152	0\\
153	0\\
154	0\\
155	0\\
156	0\\
157	0\\
158	0\\
159	0\\
160	0\\
161	0\\
162	0\\
163	0\\
164	0\\
165	0\\
166	0\\
167	0\\
168	0\\
169	0\\
170	0\\
171	0\\
172	0\\
173	0\\
174	0\\
175	0\\
176	0\\
177	0\\
178	0\\
179	0\\
180	0\\
181	0\\
182	0\\
183	0\\
184	0\\
185	0\\
186	0\\
187	0\\
188	0\\
189	0\\
190	0\\
191	0\\
192	0\\
193	0\\
194	0\\
195	0\\
196	0\\
197	0\\
198	0\\
199	0\\
200	0\\
201	0\\
202	0\\
203	0\\
204	0\\
205	0\\
206	0\\
207	0\\
208	0\\
209	0\\
210	0\\
211	0\\
212	0\\
213	0\\
214	0\\
215	0\\
216	0\\
217	0\\
218	0\\
219	0\\
220	0\\
221	0\\
222	0\\
223	0\\
224	0\\
225	0\\
226	0\\
227	0\\
228	0\\
229	0\\
230	0\\
231	0\\
232	0\\
233	0\\
234	0\\
235	0\\
236	0\\
237	0\\
238	0\\
239	0\\
240	0\\
241	0\\
242	0\\
243	0\\
244	0\\
245	0\\
246	0\\
247	0\\
248	0\\
249	0\\
250	0\\
251	0\\
252	0\\
253	0\\
254	0\\
255	0\\
256	0\\
257	0\\
258	0\\
259	0\\
260	0\\
261	0\\
262	0\\
263	0\\
264	0\\
265	0\\
266	0\\
267	0\\
268	0\\
269	0\\
270	0\\
271	0\\
272	0\\
273	0\\
274	0\\
275	0\\
276	0\\
277	0\\
278	0\\
279	0\\
280	0\\
281	0\\
282	0\\
283	0\\
284	0\\
285	0\\
286	0\\
287	0\\
288	0\\
289	0\\
290	0\\
291	0\\
292	0\\
293	0\\
294	0\\
295	0\\
296	0\\
297	0\\
298	0\\
299	0\\
300	0\\
301	0\\
302	0\\
303	0\\
304	0\\
305	0\\
306	0\\
307	0\\
308	0\\
309	0\\
310	0\\
311	0\\
312	0\\
313	0\\
314	0\\
315	0\\
316	0\\
317	0\\
318	0\\
319	0\\
320	0\\
321	0\\
322	0\\
323	0\\
324	0\\
325	0\\
326	0\\
327	0\\
328	0\\
329	0\\
330	0\\
331	0\\
332	0\\
333	0\\
334	0\\
335	0\\
336	0\\
337	0\\
338	0\\
339	0\\
340	0\\
341	0\\
342	0\\
343	0\\
344	0\\
345	0\\
346	0\\
347	0\\
348	0\\
349	0\\
350	0\\
351	0\\
352	0\\
353	0\\
354	0\\
355	0\\
356	0\\
357	0\\
358	0\\
359	0\\
360	0\\
361	0\\
362	0\\
363	0\\
364	0\\
365	0\\
366	0\\
367	0\\
368	0\\
369	0\\
370	0\\
371	0\\
372	0\\
373	0\\
374	0\\
375	0\\
376	0\\
377	0\\
378	0\\
379	0\\
380	0\\
381	0\\
382	0\\
383	0\\
384	0\\
385	0\\
386	0\\
387	0\\
388	0\\
389	0\\
390	0\\
391	0\\
392	0\\
393	0\\
394	0\\
395	0\\
396	0\\
397	0\\
398	0\\
399	0\\
400	0\\
401	0\\
402	0\\
403	0\\
404	0\\
405	0\\
406	0\\
407	0\\
408	0\\
409	0\\
410	0\\
411	0\\
412	0\\
413	0\\
414	0\\
415	0\\
416	0\\
417	0\\
418	0\\
419	0\\
420	0\\
421	0\\
422	0\\
423	0\\
424	0\\
425	0\\
426	0\\
427	0\\
428	0\\
429	0\\
430	0\\
431	0\\
432	0\\
433	0\\
434	0\\
435	0\\
436	0\\
437	0\\
438	0\\
439	0\\
440	0\\
441	0\\
442	0\\
443	0\\
444	0\\
445	0\\
446	0\\
447	0\\
448	0\\
449	0\\
450	0\\
451	0\\
452	0\\
453	0\\
454	0\\
455	0\\
456	0\\
457	0\\
458	0\\
459	0\\
460	0\\
461	0\\
462	0\\
463	0\\
464	0\\
465	0\\
466	0\\
467	0\\
468	0\\
469	0\\
470	0\\
471	0\\
472	0\\
473	0\\
474	0\\
475	0\\
476	0\\
477	0\\
478	0\\
479	0\\
480	0\\
481	0\\
482	0\\
483	0\\
484	0\\
485	0\\
486	0\\
487	0\\
488	0\\
489	0\\
490	0\\
491	0\\
492	0\\
493	0\\
494	0\\
495	0\\
496	0\\
497	0\\
498	0\\
499	0\\
500	0\\
501	0\\
502	0\\
503	0\\
504	0\\
505	0\\
506	0\\
507	0\\
508	0\\
509	0\\
510	0\\
511	0\\
512	0\\
513	0\\
514	0\\
515	0\\
516	0\\
517	0\\
518	0\\
519	0\\
520	0\\
521	0\\
522	0\\
523	0\\
524	0\\
525	0\\
526	0\\
527	0\\
528	0\\
529	0\\
530	0\\
531	0\\
532	0\\
533	0\\
534	0\\
535	0\\
536	0\\
537	0\\
538	0\\
539	0\\
540	0\\
541	0\\
542	0\\
543	0\\
544	0\\
545	0\\
546	0\\
547	0\\
548	0\\
549	0\\
550	0\\
551	0\\
552	0\\
553	0\\
554	0\\
555	0\\
556	0\\
557	0\\
558	0\\
559	0\\
560	0\\
561	0\\
562	0\\
563	0\\
564	0\\
565	0\\
566	0\\
567	0\\
568	0\\
569	0\\
570	0\\
571	0\\
572	0\\
573	0\\
574	0\\
575	0\\
576	0\\
577	0\\
578	0\\
579	0\\
580	0\\
581	0\\
582	0\\
583	0\\
584	0\\
585	0\\
586	0\\
587	0\\
588	0\\
589	0\\
590	0\\
591	0\\
592	0\\
593	0\\
594	5.46220493361368e-05\\
595	0.000539858466370388\\
596	0.00161128070222677\\
597	0.00313428048457131\\
598	0.00582886603382478\\
599	0\\
600	0\\
};
\addplot [color=mycolor14,solid,forget plot]
  table[row sep=crcr]{%
1	0.00510935392454428\\
2	0.00510935392448638\\
3	0.00510935392442745\\
4	0.00510935392436747\\
5	0.00510935392430641\\
6	0.00510935392424427\\
7	0.00510935392418101\\
8	0.00510935392411662\\
9	0.00510935392405108\\
10	0.00510935392398437\\
11	0.00510935392391646\\
12	0.00510935392384734\\
13	0.00510935392377699\\
14	0.00510935392370538\\
15	0.00510935392363249\\
16	0.0051093539235583\\
17	0.00510935392348278\\
18	0.00510935392340591\\
19	0.00510935392332767\\
20	0.00510935392324802\\
21	0.00510935392316696\\
22	0.00510935392308445\\
23	0.00510935392300046\\
24	0.00510935392291497\\
25	0.00510935392282796\\
26	0.00510935392273939\\
27	0.00510935392264923\\
28	0.00510935392255747\\
29	0.00510935392246406\\
30	0.00510935392236899\\
31	0.00510935392227222\\
32	0.00510935392217372\\
33	0.00510935392207346\\
34	0.00510935392197141\\
35	0.00510935392186754\\
36	0.00510935392176181\\
37	0.00510935392165419\\
38	0.00510935392154465\\
39	0.00510935392143315\\
40	0.00510935392131966\\
41	0.00510935392120415\\
42	0.00510935392108657\\
43	0.00510935392096689\\
44	0.00510935392084508\\
45	0.00510935392072108\\
46	0.00510935392059488\\
47	0.00510935392046642\\
48	0.00510935392033566\\
49	0.00510935392020258\\
50	0.00510935392006711\\
51	0.00510935391992922\\
52	0.00510935391978888\\
53	0.00510935391964603\\
54	0.00510935391950062\\
55	0.00510935391935262\\
56	0.00510935391920198\\
57	0.00510935391904865\\
58	0.00510935391889258\\
59	0.00510935391873372\\
60	0.00510935391857203\\
61	0.00510935391840745\\
62	0.00510935391823994\\
63	0.00510935391806943\\
64	0.00510935391789588\\
65	0.00510935391771923\\
66	0.00510935391753943\\
67	0.00510935391735641\\
68	0.00510935391717013\\
69	0.00510935391698053\\
70	0.00510935391678754\\
71	0.00510935391659111\\
72	0.00510935391639117\\
73	0.00510935391618766\\
74	0.00510935391598053\\
75	0.00510935391576969\\
76	0.00510935391555509\\
77	0.00510935391533666\\
78	0.00510935391511433\\
79	0.00510935391488804\\
80	0.0051093539146577\\
81	0.00510935391442326\\
82	0.00510935391418463\\
83	0.00510935391394175\\
84	0.00510935391369453\\
85	0.0051093539134429\\
86	0.00510935391318678\\
87	0.00510935391292609\\
88	0.00510935391266074\\
89	0.00510935391239066\\
90	0.00510935391211577\\
91	0.00510935391183597\\
92	0.00510935391155117\\
93	0.0051093539112613\\
94	0.00510935391096625\\
95	0.00510935391066594\\
96	0.00510935391036027\\
97	0.00510935391004915\\
98	0.00510935390973248\\
99	0.00510935390941015\\
100	0.00510935390908208\\
101	0.00510935390874815\\
102	0.00510935390840827\\
103	0.00510935390806232\\
104	0.0051093539077102\\
105	0.0051093539073518\\
106	0.00510935390698701\\
107	0.00510935390661571\\
108	0.00510935390623778\\
109	0.00510935390585311\\
110	0.00510935390546158\\
111	0.00510935390506306\\
112	0.00510935390465744\\
113	0.00510935390424458\\
114	0.00510935390382436\\
115	0.00510935390339664\\
116	0.00510935390296128\\
117	0.00510935390251816\\
118	0.00510935390206714\\
119	0.00510935390160807\\
120	0.00510935390114081\\
121	0.00510935390066522\\
122	0.00510935390018114\\
123	0.00510935389968842\\
124	0.00510935389918692\\
125	0.00510935389867647\\
126	0.00510935389815691\\
127	0.00510935389762808\\
128	0.00510935389708982\\
129	0.00510935389654195\\
130	0.00510935389598431\\
131	0.00510935389541673\\
132	0.00510935389483901\\
133	0.00510935389425099\\
134	0.00510935389365248\\
135	0.00510935389304328\\
136	0.00510935389242322\\
137	0.00510935389179209\\
138	0.0051093538911497\\
139	0.00510935389049585\\
140	0.00510935388983034\\
141	0.00510935388915294\\
142	0.00510935388846346\\
143	0.00510935388776167\\
144	0.00510935388704736\\
145	0.0051093538863203\\
146	0.00510935388558027\\
147	0.00510935388482702\\
148	0.00510935388406033\\
149	0.00510935388327996\\
150	0.00510935388248565\\
151	0.00510935388167717\\
152	0.00510935388085426\\
153	0.00510935388001665\\
154	0.00510935387916409\\
155	0.0051093538782963\\
156	0.00510935387741303\\
157	0.00510935387651398\\
158	0.00510935387559888\\
159	0.00510935387466743\\
160	0.00510935387371936\\
161	0.00510935387275434\\
162	0.0051093538717721\\
163	0.0051093538707723\\
164	0.00510935386975465\\
165	0.00510935386871882\\
166	0.00510935386766448\\
167	0.00510935386659131\\
168	0.00510935386549896\\
169	0.00510935386438708\\
170	0.00510935386325535\\
171	0.00510935386210338\\
172	0.00510935386093082\\
173	0.0051093538597373\\
174	0.00510935385852245\\
175	0.00510935385728588\\
176	0.0051093538560272\\
177	0.005109353854746\\
178	0.0051093538534419\\
179	0.00510935385211448\\
180	0.00510935385076331\\
181	0.00510935384938797\\
182	0.00510935384798802\\
183	0.00510935384656303\\
184	0.00510935384511253\\
185	0.00510935384363608\\
186	0.0051093538421332\\
187	0.00510935384060341\\
188	0.00510935383904624\\
189	0.00510935383746119\\
190	0.00510935383584775\\
191	0.00510935383420542\\
192	0.00510935383253366\\
193	0.00510935383083196\\
194	0.00510935382909976\\
195	0.00510935382733653\\
196	0.00510935382554169\\
197	0.00510935382371468\\
198	0.00510935382185491\\
199	0.0051093538199618\\
200	0.00510935381803473\\
201	0.0051093538160731\\
202	0.00510935381407627\\
203	0.00510935381204362\\
204	0.00510935380997449\\
205	0.00510935380786822\\
206	0.00510935380572413\\
207	0.00510935380354154\\
208	0.00510935380131975\\
209	0.00510935379905806\\
210	0.00510935379675572\\
211	0.00510935379441202\\
212	0.00510935379202618\\
213	0.00510935378959746\\
214	0.00510935378712506\\
215	0.00510935378460819\\
216	0.00510935378204604\\
217	0.00510935377943779\\
218	0.00510935377678259\\
219	0.0051093537740796\\
220	0.00510935377132793\\
221	0.0051093537685267\\
222	0.00510935376567501\\
223	0.00510935376277193\\
224	0.00510935375981652\\
225	0.00510935375680782\\
226	0.00510935375374486\\
227	0.00510935375062666\\
228	0.00510935374745218\\
229	0.0051093537442204\\
230	0.00510935374093028\\
231	0.00510935373758074\\
232	0.00510935373417069\\
233	0.00510935373069901\\
234	0.00510935372716459\\
235	0.00510935372356625\\
236	0.00510935371990283\\
237	0.00510935371617313\\
238	0.00510935371237593\\
239	0.00510935370850997\\
240	0.00510935370457401\\
241	0.00510935370056673\\
242	0.00510935369648683\\
243	0.00510935369233297\\
244	0.00510935368810377\\
245	0.00510935368379785\\
246	0.00510935367941378\\
247	0.00510935367495012\\
248	0.00510935367040539\\
249	0.00510935366577809\\
250	0.00510935366106669\\
251	0.00510935365626963\\
252	0.0051093536513853\\
253	0.00510935364641211\\
254	0.00510935364134838\\
255	0.00510935363619244\\
256	0.00510935363094257\\
257	0.00510935362559702\\
258	0.005109353620154\\
259	0.00510935361461171\\
260	0.00510935360896827\\
261	0.00510935360322182\\
262	0.00510935359737041\\
263	0.0051093535914121\\
264	0.00510935358534488\\
265	0.0051093535791667\\
266	0.0051093535728755\\
267	0.00510935356646916\\
268	0.00510935355994551\\
269	0.00510935355330235\\
270	0.00510935354653745\\
271	0.00510935353964852\\
272	0.00510935353263322\\
273	0.00510935352548919\\
274	0.00510935351821399\\
275	0.00510935351080517\\
276	0.0051093535032602\\
277	0.00510935349557652\\
278	0.00510935348775153\\
279	0.00510935347978255\\
280	0.00510935347166687\\
281	0.00510935346340172\\
282	0.00510935345498429\\
283	0.00510935344641169\\
284	0.005109353437681\\
285	0.00510935342878924\\
286	0.00510935341973335\\
287	0.00510935341051024\\
288	0.00510935340111674\\
289	0.00510935339154964\\
290	0.00510935338180565\\
291	0.00510935337188143\\
292	0.00510935336177357\\
293	0.00510935335147859\\
294	0.00510935334099296\\
295	0.00510935333031307\\
296	0.00510935331943524\\
297	0.00510935330835574\\
298	0.00510935329707074\\
299	0.00510935328557636\\
300	0.00510935327386865\\
301	0.00510935326194356\\
302	0.005109353249797\\
303	0.00510935323742479\\
304	0.00510935322482265\\
305	0.00510935321198625\\
306	0.00510935319891117\\
307	0.0051093531855929\\
308	0.00510935317202687\\
309	0.00510935315820841\\
310	0.00510935314413276\\
311	0.00510935312979509\\
312	0.00510935311519047\\
313	0.00510935310031389\\
314	0.00510935308516024\\
315	0.00510935306972432\\
316	0.00510935305400085\\
317	0.00510935303798444\\
318	0.00510935302166961\\
319	0.0051093530050508\\
320	0.00510935298812232\\
321	0.0051093529708784\\
322	0.00510935295331317\\
323	0.00510935293542065\\
324	0.00510935291719477\\
325	0.00510935289862933\\
326	0.00510935287971805\\
327	0.00510935286045451\\
328	0.00510935284083221\\
329	0.00510935282084453\\
330	0.00510935280048472\\
331	0.00510935277974593\\
332	0.00510935275862119\\
333	0.00510935273710341\\
334	0.00510935271518538\\
335	0.00510935269285976\\
336	0.00510935267011908\\
337	0.00510935264695576\\
338	0.00510935262336208\\
339	0.00510935259933018\\
340	0.00510935257485208\\
341	0.00510935254991964\\
342	0.00510935252452461\\
343	0.00510935249865856\\
344	0.00510935247231295\\
345	0.00510935244547907\\
346	0.00510935241814805\\
347	0.00510935239031089\\
348	0.00510935236195841\\
349	0.00510935233308126\\
350	0.00510935230366995\\
351	0.00510935227371478\\
352	0.00510935224320592\\
353	0.0051093522121333\\
354	0.00510935218048672\\
355	0.00510935214825574\\
356	0.00510935211542975\\
357	0.00510935208199791\\
358	0.00510935204794918\\
359	0.00510935201327231\\
360	0.00510935197795578\\
361	0.00510935194198785\\
362	0.00510935190535655\\
363	0.00510935186804961\\
364	0.00510935183005451\\
365	0.00510935179135843\\
366	0.00510935175194826\\
367	0.00510935171181056\\
368	0.00510935167093157\\
369	0.00510935162929717\\
370	0.00510935158689288\\
371	0.00510935154370381\\
372	0.00510935149971467\\
373	0.00510935145490971\\
374	0.00510935140927275\\
375	0.00510935136278707\\
376	0.00510935131543545\\
377	0.00510935126720007\\
378	0.00510935121806255\\
379	0.00510935116800382\\
380	0.00510935111700412\\
381	0.00510935106504293\\
382	0.00510935101209893\\
383	0.00510935095814987\\
384	0.00510935090317253\\
385	0.00510935084714257\\
386	0.00510935079003445\\
387	0.00510935073182125\\
388	0.00510935067247457\\
389	0.0051093506119645\\
390	0.00510935055025962\\
391	0.00510935048732717\\
392	0.005109350423133\\
393	0.00510935035764106\\
394	0.00510935029081268\\
395	0.0051093502226066\\
396	0.00510935015297935\\
397	0.0051093500818852\\
398	0.00510935000927617\\
399	0.00510934993510209\\
400	0.00510934985931063\\
401	0.00510934978184713\\
402	0.00510934970265412\\
403	0.00510934962167004\\
404	0.00510934953882765\\
405	0.00510934945405288\\
406	0.00510934936726596\\
407	0.00510934927838459\\
408	0.0051093491873237\\
409	0.00510934909399121\\
410	0.00510934899828718\\
411	0.00510934890010274\\
412	0.0051093487993189\\
413	0.00510934869580513\\
414	0.00510934858941767\\
415	0.00510934847999748\\
416	0.00510934836736759\\
417	0.00510934825132947\\
418	0.00510934813165768\\
419	0.00510934800809096\\
420	0.00510934788031657\\
421	0.00510934774794105\\
422	0.00510934761043526\\
423	0.00510934746703316\\
424	0.00510934731655571\\
425	0.00510934715713394\\
426	0.00510934698585053\\
427	0.00510934679846615\\
428	0.00510934658971189\\
429	0.00510934635501702\\
430	0.00510934609428173\\
431	0.00510934581553934\\
432	0.00510934553045902\\
433	0.00510934523883727\\
434	0.00510934494045925\\
435	0.00510934463509702\\
436	0.00510934432250662\\
437	0.0051093440024216\\
438	0.00510934367453844\\
439	0.00510934333848197\\
440	0.00510934299372291\\
441	0.00510934263938256\\
442	0.00510934227377758\\
443	0.00510934189338404\\
444	0.00510934149055524\\
445	0.00510934104871201\\
446	0.0051093405328209\\
447	0.00510933987224999\\
448	0.00510933893456566\\
449	0.00510933749814763\\
450	0.00510933525909056\\
451	0.00510933195691012\\
452	0.0051093277027712\\
453	0.00510932322326447\\
454	0.00510931866061475\\
455	0.00510931401393708\\
456	0.00510930928223714\\
457	0.00510930446437545\\
458	0.00510929955904657\\
459	0.00510929456478831\\
460	0.00510928948001827\\
461	0.00510928430303812\\
462	0.00510927903189066\\
463	0.0051092736641144\\
464	0.00510926819697369\\
465	0.00510926262766088\\
466	0.00510925695330791\\
467	0.00510925117092717\\
468	0.00510924527735716\\
469	0.0051092392693063\\
470	0.00510923314323429\\
471	0.00510922689517286\\
472	0.00510922052064365\\
473	0.00510921401495943\\
474	0.00510920737376227\\
475	0.00510920059241034\\
476	0.00510919366595188\\
477	0.00510918658909689\\
478	0.00510917935618547\\
479	0.00510917196115216\\
480	0.0051091643974856\\
481	0.00510915665818281\\
482	0.00510914873569687\\
483	0.00510914062187698\\
484	0.00510913230789939\\
485	0.00510912378418719\\
486	0.00510911504031678\\
487	0.0051091060649076\\
488	0.00510909684549009\\
489	0.00510908736834342\\
490	0.00510907761828662\\
491	0.00510906757838935\\
492	0.00510905722953093\\
493	0.00510904654965426\\
494	0.00510903551238932\\
495	0.00510902408437497\\
496	0.00510901221997077\\
497	0.00510899985100423\\
498	0.00510898686788099\\
499	0.00510897308793375\\
500	0.00510895821124027\\
501	0.00510894178309003\\
502	0.00510892323212426\\
503	0.00510890213499336\\
504	0.00510887884367902\\
505	0.00510885487124891\\
506	0.00510883054776192\\
507	0.00510880582712782\\
508	0.00510878066473975\\
509	0.00510875503919166\\
510	0.00510872892463499\\
511	0.00510870228658319\\
512	0.00510867507159252\\
513	0.0051086471819434\\
514	0.00510861841406499\\
515	0.0051085883118057\\
516	0.00510855582784442\\
517	0.00510851857885204\\
518	0.00510847132449955\\
519	0.00510840323682491\\
520	0.00510829419554075\\
521	0.00510811366057947\\
522	0.00510783477615486\\
523	0.00510748286176653\\
524	0.00510712175014415\\
525	0.00510675072678062\\
526	0.00510636890987652\\
527	0.00510597524346\\
528	0.00510556861103784\\
529	0.00510514817682335\\
530	0.00510471377207103\\
531	0.00510426496847091\\
532	0.00510379755925117\\
533	0.0051033024711498\\
534	0.00510276429339391\\
535	0.00510216082069459\\
536	0.00510146360797495\\
537	0.00510065909327794\\
538	0.00509980153479282\\
539	0.00509893015267351\\
540	0.00509804393789524\\
541	0.00509714171435844\\
542	0.00509622203850178\\
543	0.00509528298072277\\
544	0.00509432166950426\\
545	0.00509333344427241\\
546	0.00509231047983481\\
547	0.0050912399784705\\
548	0.00509010318509986\\
549	0.00508887989661606\\
550	0.00508756425762667\\
551	0.00508615254028593\\
552	0.00508451716960274\\
553	0.00508228135524955\\
554	0.00507845919342308\\
555	0.00507114661782098\\
556	0.00505821458816492\\
557	0.00504408871369172\\
558	0.00502877945922603\\
559	0.00501144345438311\\
560	0.00499075507820406\\
561	0.0049661841610934\\
562	0.00494140742442814\\
563	0.00491629961178414\\
564	0.00489058312765606\\
565	0.00486379574942121\\
566	0.00483556723733559\\
567	0.00480693086998232\\
568	0.00477879006842645\\
569	0.004751115067223\\
570	0.00472391025364932\\
571	0.00469746530010506\\
572	0.00467293971960066\\
573	0.0046515084899547\\
574	0.00463167812481128\\
575	0.00461032365983135\\
576	0.0045839928512043\\
577	0.00454273192026105\\
578	0.00445789096708661\\
579	0.00424277237319832\\
580	0.00395007387994891\\
581	0.00361600322881859\\
582	0.00327003545696772\\
583	0.00290671034981918\\
584	0.002516381243237\\
585	0.00211762748756851\\
586	0.00170903265059097\\
587	0.00128770391395649\\
588	0.000846590459924926\\
589	0.000367058431095583\\
590	0\\
591	0\\
592	0\\
593	0\\
594	0\\
595	0\\
596	0\\
597	0\\
598	0\\
599	0\\
600	0\\
};
\addplot [color=mycolor15,solid,forget plot]
  table[row sep=crcr]{%
1	0.00405783128187049\\
2	0.00405783128068545\\
3	0.00405783127947922\\
4	0.00405783127825142\\
5	0.00405783127700167\\
6	0.00405783127572958\\
7	0.00405783127443475\\
8	0.00405783127311677\\
9	0.00405783127177523\\
10	0.0040578312704097\\
11	0.00405783126901977\\
12	0.00405783126760499\\
13	0.00405783126616492\\
14	0.00405783126469911\\
15	0.0040578312632071\\
16	0.00405783126168842\\
17	0.0040578312601426\\
18	0.00405783125856915\\
19	0.00405783125696759\\
20	0.00405783125533739\\
21	0.00405783125367807\\
22	0.00405783125198909\\
23	0.00405783125026993\\
24	0.00405783124852004\\
25	0.00405783124673889\\
26	0.00405783124492591\\
27	0.00405783124308054\\
28	0.00405783124120219\\
29	0.00405783123929028\\
30	0.00405783123734421\\
31	0.00405783123536337\\
32	0.00405783123334714\\
33	0.00405783123129489\\
34	0.00405783122920598\\
35	0.00405783122707974\\
36	0.00405783122491553\\
37	0.00405783122271265\\
38	0.00405783122047042\\
39	0.00405783121818814\\
40	0.00405783121586509\\
41	0.00405783121350055\\
42	0.00405783121109377\\
43	0.004057831208644\\
44	0.00405783120615048\\
45	0.00405783120361243\\
46	0.00405783120102904\\
47	0.00405783119839952\\
48	0.00405783119572304\\
49	0.00405783119299876\\
50	0.00405783119022583\\
51	0.00405783118740338\\
52	0.00405783118453052\\
53	0.00405783118160638\\
54	0.00405783117863001\\
55	0.0040578311756005\\
56	0.0040578311725169\\
57	0.00405783116937823\\
58	0.00405783116618353\\
59	0.00405783116293179\\
60	0.004057831159622\\
61	0.00405783115625311\\
62	0.00405783115282408\\
63	0.00405783114933383\\
64	0.00405783114578127\\
65	0.00405783114216529\\
66	0.00405783113848477\\
67	0.00405783113473854\\
68	0.00405783113092544\\
69	0.00405783112704428\\
70	0.00405783112309383\\
71	0.00405783111907288\\
72	0.00405783111498015\\
73	0.00405783111081437\\
74	0.00405783110657424\\
75	0.00405783110225844\\
76	0.0040578310978656\\
77	0.00405783109339437\\
78	0.00405783108884334\\
79	0.00405783108421109\\
80	0.00405783107949617\\
81	0.00405783107469711\\
82	0.0040578310698124\\
83	0.00405783106484053\\
84	0.00405783105977993\\
85	0.00405783105462903\\
86	0.00405783104938622\\
87	0.00405783104404985\\
88	0.00405783103861826\\
89	0.00405783103308975\\
90	0.0040578310274626\\
91	0.00405783102173504\\
92	0.00405783101590529\\
93	0.00405783100997152\\
94	0.00405783100393188\\
95	0.00405783099778448\\
96	0.0040578309915274\\
97	0.00405783098515868\\
98	0.00405783097867634\\
99	0.00405783097207834\\
100	0.00405783096536263\\
101	0.00405783095852711\\
102	0.00405783095156964\\
103	0.00405783094448805\\
104	0.00405783093728011\\
105	0.00405783092994359\\
106	0.00405783092247619\\
107	0.00405783091487556\\
108	0.00405783090713935\\
109	0.00405783089926512\\
110	0.00405783089125043\\
111	0.00405783088309275\\
112	0.00405783087478954\\
113	0.00405783086633822\\
114	0.00405783085773613\\
115	0.00405783084898058\\
116	0.00405783084006884\\
117	0.00405783083099812\\
118	0.00405783082176559\\
119	0.00405783081236836\\
120	0.00405783080280348\\
121	0.00405783079306798\\
122	0.00405783078315879\\
123	0.00405783077307284\\
124	0.00405783076280695\\
125	0.00405783075235793\\
126	0.00405783074172249\\
127	0.00405783073089732\\
128	0.00405783071987903\\
129	0.00405783070866416\\
130	0.00405783069724922\\
131	0.00405783068563062\\
132	0.00405783067380473\\
133	0.00405783066176786\\
134	0.00405783064951623\\
135	0.00405783063704601\\
136	0.00405783062435329\\
137	0.0040578306114341\\
138	0.00405783059828439\\
139	0.00405783058490005\\
140	0.00405783057127688\\
141	0.00405783055741062\\
142	0.00405783054329693\\
143	0.00405783052893137\\
144	0.00405783051430945\\
145	0.00405783049942658\\
146	0.00405783048427811\\
147	0.00405783046885928\\
148	0.00405783045316526\\
149	0.00405783043719112\\
150	0.00405783042093186\\
151	0.00405783040438238\\
152	0.00405783038753748\\
153	0.00405783037039187\\
154	0.00405783035294019\\
155	0.00405783033517694\\
156	0.00405783031709656\\
157	0.00405783029869336\\
158	0.00405783027996156\\
159	0.00405783026089528\\
160	0.00405783024148854\\
161	0.00405783022173522\\
162	0.00405783020162912\\
163	0.00405783018116392\\
164	0.00405783016033319\\
165	0.00405783013913036\\
166	0.00405783011754878\\
167	0.00405783009558164\\
168	0.00405783007322204\\
169	0.00405783005046293\\
170	0.00405783002729714\\
171	0.00405783000371738\\
172	0.00405782997971622\\
173	0.00405782995528609\\
174	0.00405782993041928\\
175	0.00405782990510796\\
176	0.00405782987934414\\
177	0.00405782985311969\\
178	0.00405782982642632\\
179	0.00405782979925562\\
180	0.00405782977159899\\
181	0.00405782974344771\\
182	0.00405782971479285\\
183	0.00405782968562538\\
184	0.00405782965593606\\
185	0.00405782962571548\\
186	0.0040578295949541\\
187	0.00405782956364216\\
188	0.00405782953176975\\
189	0.00405782949932677\\
190	0.00405782946630291\\
191	0.00405782943268772\\
192	0.00405782939847052\\
193	0.00405782936364045\\
194	0.00405782932818645\\
195	0.00405782929209725\\
196	0.00405782925536137\\
197	0.00405782921796714\\
198	0.00405782917990266\\
199	0.0040578291411558\\
200	0.00405782910171422\\
201	0.00405782906156536\\
202	0.00405782902069641\\
203	0.00405782897909433\\
204	0.00405782893674585\\
205	0.00405782889363745\\
206	0.00405782884975534\\
207	0.0040578288050855\\
208	0.00405782875961363\\
209	0.00405782871332519\\
210	0.00405782866620535\\
211	0.00405782861823901\\
212	0.00405782856941079\\
213	0.00405782851970501\\
214	0.00405782846910573\\
215	0.00405782841759669\\
216	0.00405782836516133\\
217	0.00405782831178278\\
218	0.00405782825744386\\
219	0.00405782820212706\\
220	0.00405782814581456\\
221	0.00405782808848819\\
222	0.00405782803012945\\
223	0.00405782797071948\\
224	0.00405782791023909\\
225	0.0040578278486687\\
226	0.00405782778598838\\
227	0.00405782772217783\\
228	0.00405782765721635\\
229	0.00405782759108288\\
230	0.00405782752375592\\
231	0.0040578274552136\\
232	0.00405782738543363\\
233	0.00405782731439329\\
234	0.00405782724206943\\
235	0.00405782716843847\\
236	0.00405782709347639\\
237	0.00405782701715869\\
238	0.00405782693946044\\
239	0.00405782686035619\\
240	0.00405782677982006\\
241	0.00405782669782565\\
242	0.00405782661434604\\
243	0.00405782652935384\\
244	0.0040578264428211\\
245	0.00405782635471936\\
246	0.00405782626501961\\
247	0.00405782617369228\\
248	0.00405782608070726\\
249	0.00405782598603382\\
250	0.00405782588964068\\
251	0.00405782579149595\\
252	0.00405782569156712\\
253	0.00405782558982108\\
254	0.00405782548622405\\
255	0.00405782538074163\\
256	0.00405782527333876\\
257	0.00405782516397968\\
258	0.00405782505262796\\
259	0.00405782493924648\\
260	0.00405782482379739\\
261	0.00405782470624211\\
262	0.00405782458654133\\
263	0.00405782446465498\\
264	0.00405782434054219\\
265	0.00405782421416135\\
266	0.00405782408547001\\
267	0.00405782395442492\\
268	0.00405782382098198\\
269	0.00405782368509626\\
270	0.00405782354672193\\
271	0.00405782340581233\\
272	0.00405782326231983\\
273	0.00405782311619594\\
274	0.0040578229673912\\
275	0.00405782281585521\\
276	0.0040578226615366\\
277	0.00405782250438299\\
278	0.00405782234434101\\
279	0.00405782218135626\\
280	0.00405782201537327\\
281	0.00405782184633554\\
282	0.00405782167418543\\
283	0.00405782149886425\\
284	0.00405782132031214\\
285	0.0040578211384681\\
286	0.00405782095326998\\
287	0.0040578207646544\\
288	0.0040578205725568\\
289	0.00405782037691137\\
290	0.00405782017765105\\
291	0.00405781997470748\\
292	0.00405781976801102\\
293	0.0040578195574907\\
294	0.0040578193430742\\
295	0.00405781912468782\\
296	0.00405781890225648\\
297	0.00405781867570366\\
298	0.00405781844495141\\
299	0.00405781820992032\\
300	0.00405781797052948\\
301	0.00405781772669645\\
302	0.00405781747833727\\
303	0.0040578172253664\\
304	0.00405781696769673\\
305	0.00405781670523949\\
306	0.00405781643790431\\
307	0.00405781616559914\\
308	0.00405781588823022\\
309	0.0040578156057021\\
310	0.00405781531791755\\
311	0.0040578150247776\\
312	0.00405781472618145\\
313	0.0040578144220265\\
314	0.00405781411220828\\
315	0.00405781379662045\\
316	0.00405781347515476\\
317	0.00405781314770103\\
318	0.00405781281414711\\
319	0.00405781247437886\\
320	0.00405781212828012\\
321	0.00405781177573269\\
322	0.00405781141661628\\
323	0.00405781105080849\\
324	0.0040578106781848\\
325	0.0040578102986185\\
326	0.00405780991198069\\
327	0.00405780951814023\\
328	0.00405780911696372\\
329	0.00405780870831544\\
330	0.00405780829205735\\
331	0.00405780786804904\\
332	0.00405780743614766\\
333	0.00405780699620794\\
334	0.0040578065480821\\
335	0.00405780609161985\\
336	0.0040578056266683\\
337	0.00405780515307194\\
338	0.00405780467067261\\
339	0.00405780417930944\\
340	0.00405780367881878\\
341	0.00405780316903416\\
342	0.00405780264978626\\
343	0.00405780212090283\\
344	0.00405780158220862\\
345	0.00405780103352538\\
346	0.0040578004746717\\
347	0.00405779990546304\\
348	0.00405779932571159\\
349	0.00405779873522624\\
350	0.00405779813381247\\
351	0.00405779752127228\\
352	0.00405779689740411\\
353	0.0040577962620027\\
354	0.00405779561485905\\
355	0.00405779495576026\\
356	0.00405779428448941\\
357	0.00405779360082548\\
358	0.00405779290454315\\
359	0.00405779219541268\\
360	0.00405779147319974\\
361	0.00405779073766525\\
362	0.00405778998856516\\
363	0.00405778922565027\\
364	0.004057788448666\\
365	0.00405778765735214\\
366	0.00405778685144261\\
367	0.00405778603066515\\
368	0.00405778519474103\\
369	0.00405778434338473\\
370	0.00405778347630352\\
371	0.00405778259319717\\
372	0.00405778169375742\\
373	0.00405778077766757\\
374	0.00405777984460199\\
375	0.00405777889422551\\
376	0.00405777792619289\\
377	0.00405777694014812\\
378	0.00405777593572373\\
379	0.00405777491253998\\
380	0.004057773870204\\
381	0.00405777280830881\\
382	0.00405777172643225\\
383	0.00405777062413573\\
384	0.00405776950096275\\
385	0.00405776835643724\\
386	0.00405776719006136\\
387	0.00405776600131297\\
388	0.00405776478964258\\
389	0.0040577635544702\\
390	0.00405776229518312\\
391	0.00405776101113627\\
392	0.00405775970165634\\
393	0.0040577583660461\\
394	0.00405775700357734\\
395	0.00405775561346582\\
396	0.00405775419486886\\
397	0.00405775274689764\\
398	0.00405775126861576\\
399	0.0040577497590391\\
400	0.00405774821713706\\
401	0.00405774664183517\\
402	0.00405774503201691\\
403	0.00405774338651914\\
404	0.00405774170411221\\
405	0.0040577399834581\\
406	0.004057738223062\\
407	0.00405773642127506\\
408	0.00405773457639635\\
409	0.00405773268671545\\
410	0.00405773075038478\\
411	0.00405772876540242\\
412	0.00405772672959232\\
413	0.00405772464058141\\
414	0.00405772249577291\\
415	0.00405772029231515\\
416	0.00405771802706464\\
417	0.00405771569654162\\
418	0.00405771329687435\\
419	0.00405771082372536\\
420	0.00405770827218407\\
421	0.00405770563659196\\
422	0.00405770291022529\\
423	0.00405770008467385\\
424	0.00405769714858433\\
425	0.00405769408513857\\
426	0.0040576908672278\\
427	0.00405768744906833\\
428	0.00405768375418857\\
429	0.00405767966545186\\
430	0.00405767503814818\\
431	0.00405766977979874\\
432	0.00405766401724685\\
433	0.00405765812442432\\
434	0.00405765209720578\\
435	0.00405764593124648\\
436	0.00405763962196297\\
437	0.0040576331645073\\
438	0.00405762655372528\\
439	0.00405761978407451\\
440	0.00405761284943778\\
441	0.00405760574266237\\
442	0.00405759845438355\\
443	0.00405759096999643\\
444	0.00405758326190385\\
445	0.00405757526995869\\
446	0.0040575668532547\\
447	0.00405755767532177\\
448	0.00405754694458593\\
449	0.00405753287461\\
450	0.00405751171979886\\
451	0.00405747658620918\\
452	0.00405741782337858\\
453	0.00405733094206556\\
454	0.00405723774177815\\
455	0.00405714280824585\\
456	0.00405704612435729\\
457	0.00405694767112828\\
458	0.00405684742678483\\
459	0.00405674536597795\\
460	0.00405664145963472\\
461	0.00405653567601179\\
462	0.00405642798236996\\
463	0.00405631834316966\\
464	0.00405620670991157\\
465	0.004056093026219\\
466	0.00405597723410947\\
467	0.00405585927502846\\
468	0.00405573908853998\\
469	0.0040556166100909\\
470	0.00405549177375176\\
471	0.00405536451037369\\
472	0.00405523474269021\\
473	0.00405510237930375\\
474	0.00405496731849525\\
475	0.00405482947240714\\
476	0.00405468874754487\\
477	0.00405454504425679\\
478	0.00405439825618186\\
479	0.0040542482696327\\
480	0.00405409496290346\\
481	0.00405393820548961\\
482	0.00405377785720437\\
483	0.0040536137671737\\
484	0.00405344577268737\\
485	0.00405327369787945\\
486	0.00405309735220562\\
487	0.00405291652867797\\
488	0.00405273100180775\\
489	0.0040525405251902\\
490	0.00405234482864563\\
491	0.00405214361476636\\
492	0.00405193655460589\\
493	0.00405172328194951\\
494	0.00405150338494586\\
495	0.00405127639226724\\
496	0.00405104174713352\\
497	0.00405079875400372\\
498	0.00405054646426666\\
499	0.00405028343048887\\
500	0.00405000719573569\\
501	0.00404971331113423\\
502	0.00404939372089459\\
503	0.00404903503027239\\
504	0.0040486199641645\\
505	0.00404814310693598\\
506	0.00404764866385833\\
507	0.00404714756740775\\
508	0.00404663883761316\\
509	0.00404612128916025\\
510	0.00404559453378168\\
511	0.00404505813822849\\
512	0.00404451160523361\\
513	0.00404395432618424\\
514	0.00404338545841346\\
515	0.0040428035917532\\
516	0.00404220582736261\\
517	0.00404158523602025\\
518	0.00404092393042789\\
519	0.00404017466478783\\
520	0.00403921428067007\\
521	0.00403773678695009\\
522	0.00403506010340366\\
523	0.00403005195149346\\
524	0.00402283760207031\\
525	0.00401545845225671\\
526	0.00400790316818706\\
527	0.00400015730478272\\
528	0.00399220196177449\\
529	0.00398401382847418\\
530	0.00397557198005718\\
531	0.00396687959477154\\
532	0.00395797017314288\\
533	0.00394881320521243\\
534	0.003939296200743\\
535	0.00392918736081158\\
536	0.00391811384245127\\
537	0.00390536367213929\\
538	0.00389006191836176\\
539	0.00387347099840221\\
540	0.00385674672730471\\
541	0.00383987972589676\\
542	0.00382285939851747\\
543	0.0038056733707417\\
544	0.00378830593945472\\
545	0.00377073386590955\\
546	0.00375291663947729\\
547	0.00373477595823748\\
548	0.00371615447430384\\
549	0.00369675261602705\\
550	0.00367616734655516\\
551	0.00365447847782785\\
552	0.00363268508514219\\
553	0.003610797550057\\
554	0.00358686376537867\\
555	0.00354932816061378\\
556	0.003458258037869\\
557	0.00335761249498769\\
558	0.00325023830819355\\
559	0.00313186109142807\\
560	0.00298967747584707\\
561	0.00279974352474002\\
562	0.00260239693723076\\
563	0.00239685833453858\\
564	0.00218168025772699\\
565	0.00195248625972547\\
566	0.00169649043125886\\
567	0.00141986044564115\\
568	0.00113508540913231\\
569	0.000841186746185829\\
570	0.000536281815051808\\
571	0.000215644234799086\\
572	0\\
573	0\\
574	0\\
575	0\\
576	0\\
577	0\\
578	0\\
579	0\\
580	0\\
581	0\\
582	0\\
583	0\\
584	0\\
585	0\\
586	0\\
587	0\\
588	0\\
589	0\\
590	0\\
591	0\\
592	0\\
593	0\\
594	0\\
595	0\\
596	0\\
597	0\\
598	0\\
599	0\\
600	0\\
};
\addplot [color=mycolor16,solid,forget plot]
  table[row sep=crcr]{%
1	0.00399857970101667\\
2	0.00399857967758191\\
3	0.00399857965372814\\
4	0.00399857962944786\\
5	0.00399857960473347\\
6	0.00399857957957721\\
7	0.0039985795539712\\
8	0.00399857952790741\\
9	0.00399857950137767\\
10	0.00399857947437365\\
11	0.00399857944688689\\
12	0.00399857941890878\\
13	0.00399857939043054\\
14	0.00399857936144325\\
15	0.00399857933193782\\
16	0.003998579301905\\
17	0.00399857927133538\\
18	0.00399857924021939\\
19	0.00399857920854726\\
20	0.00399857917630907\\
21	0.00399857914349473\\
22	0.00399857911009395\\
23	0.00399857907609627\\
24	0.00399857904149102\\
25	0.00399857900626738\\
26	0.00399857897041431\\
27	0.00399857893392057\\
28	0.00399857889677473\\
29	0.00399857885896517\\
30	0.00399857882048004\\
31	0.00399857878130728\\
32	0.00399857874143463\\
33	0.0039985787008496\\
34	0.00399857865953948\\
35	0.00399857861749134\\
36	0.00399857857469201\\
37	0.00399857853112809\\
38	0.00399857848678594\\
39	0.00399857844165167\\
40	0.00399857839571117\\
41	0.00399857834895004\\
42	0.00399857830135366\\
43	0.00399857825290712\\
44	0.00399857820359525\\
45	0.00399857815340263\\
46	0.00399857810231355\\
47	0.00399857805031201\\
48	0.00399857799738176\\
49	0.00399857794350621\\
50	0.00399857788866852\\
51	0.00399857783285152\\
52	0.00399857777603775\\
53	0.00399857771820944\\
54	0.00399857765934849\\
55	0.00399857759943649\\
56	0.00399857753845469\\
57	0.00399857747638403\\
58	0.00399857741320508\\
59	0.00399857734889808\\
60	0.00399857728344292\\
61	0.00399857721681913\\
62	0.00399857714900586\\
63	0.00399857707998192\\
64	0.00399857700972571\\
65	0.00399857693821527\\
66	0.00399857686542822\\
67	0.00399857679134181\\
68	0.00399857671593288\\
69	0.00399857663917784\\
70	0.0039985765610527\\
71	0.00399857648153303\\
72	0.00399857640059396\\
73	0.0039985763182102\\
74	0.00399857623435598\\
75	0.0039985761490051\\
76	0.00399857606213086\\
77	0.00399857597370611\\
78	0.00399857588370321\\
79	0.00399857579209403\\
80	0.00399857569884992\\
81	0.00399857560394175\\
82	0.00399857550733984\\
83	0.00399857540901401\\
84	0.00399857530893351\\
85	0.00399857520706708\\
86	0.00399857510338286\\
87	0.00399857499784846\\
88	0.00399857489043089\\
89	0.00399857478109657\\
90	0.00399857466981134\\
91	0.00399857455654041\\
92	0.00399857444124839\\
93	0.00399857432389925\\
94	0.0039985742044563\\
95	0.00399857408288223\\
96	0.00399857395913902\\
97	0.00399857383318803\\
98	0.00399857370498987\\
99	0.00399857357450448\\
100	0.0039985734416911\\
101	0.0039985733065082\\
102	0.00399857316891353\\
103	0.0039985730288641\\
104	0.00399857288631614\\
105	0.00399857274122509\\
106	0.00399857259354561\\
107	0.00399857244323154\\
108	0.00399857229023589\\
109	0.00399857213451086\\
110	0.00399857197600775\\
111	0.00399857181467703\\
112	0.00399857165046827\\
113	0.00399857148333015\\
114	0.0039985713132104\\
115	0.00399857114005584\\
116	0.00399857096381236\\
117	0.00399857078442485\\
118	0.00399857060183721\\
119	0.00399857041599237\\
120	0.00399857022683221\\
121	0.00399857003429759\\
122	0.0039985698383283\\
123	0.00399856963886306\\
124	0.00399856943583947\\
125	0.00399856922919405\\
126	0.00399856901886217\\
127	0.00399856880477803\\
128	0.00399856858687465\\
129	0.00399856836508388\\
130	0.00399856813933632\\
131	0.00399856790956134\\
132	0.00399856767568703\\
133	0.00399856743764022\\
134	0.0039985671953464\\
135	0.00399856694872972\\
136	0.00399856669771301\\
137	0.00399856644221767\\
138	0.00399856618216371\\
139	0.0039985659174697\\
140	0.00399856564805277\\
141	0.00399856537382852\\
142	0.00399856509471106\\
143	0.00399856481061297\\
144	0.00399856452144522\\
145	0.00399856422711723\\
146	0.00399856392753674\\
147	0.00399856362260986\\
148	0.00399856331224102\\
149	0.0039985629963329\\
150	0.00399856267478646\\
151	0.00399856234750085\\
152	0.00399856201437341\\
153	0.00399856167529966\\
154	0.00399856133017319\\
155	0.00399856097888571\\
156	0.00399856062132695\\
157	0.00399856025738468\\
158	0.00399855988694461\\
159	0.00399855950989042\\
160	0.00399855912610368\\
161	0.00399855873546381\\
162	0.00399855833784805\\
163	0.00399855793313144\\
164	0.00399855752118674\\
165	0.00399855710188444\\
166	0.00399855667509264\\
167	0.00399855624067709\\
168	0.0039985557985011\\
169	0.00399855534842549\\
170	0.00399855489030857\\
171	0.00399855442400608\\
172	0.00399855394937115\\
173	0.00399855346625422\\
174	0.00399855297450304\\
175	0.00399855247396258\\
176	0.00399855196447501\\
177	0.00399855144587961\\
178	0.00399855091801275\\
179	0.00399855038070782\\
180	0.00399854983379517\\
181	0.00399854927710207\\
182	0.00399854871045264\\
183	0.00399854813366778\\
184	0.00399854754656514\\
185	0.00399854694895904\\
186	0.00399854634066039\\
187	0.00399854572147668\\
188	0.00399854509121185\\
189	0.00399854444966626\\
190	0.00399854379663663\\
191	0.00399854313191594\\
192	0.00399854245529339\\
193	0.00399854176655431\\
194	0.00399854106548009\\
195	0.00399854035184811\\
196	0.00399853962543166\\
197	0.00399853888599988\\
198	0.00399853813331763\\
199	0.00399853736714547\\
200	0.00399853658723955\\
201	0.00399853579335153\\
202	0.00399853498522846\\
203	0.00399853416261277\\
204	0.00399853332524209\\
205	0.00399853247284926\\
206	0.00399853160516211\\
207	0.0039985307219035\\
208	0.00399852982279113\\
209	0.00399852890753747\\
210	0.00399852797584966\\
211	0.00399852702742942\\
212	0.00399852606197292\\
213	0.00399852507917069\\
214	0.00399852407870751\\
215	0.00399852306026229\\
216	0.00399852202350798\\
217	0.0039985209681114\\
218	0.0039985198937332\\
219	0.00399851880002766\\
220	0.00399851768664264\\
221	0.00399851655321938\\
222	0.00399851539939245\\
223	0.00399851422478955\\
224	0.00399851302903142\\
225	0.00399851181173169\\
226	0.00399851057249673\\
227	0.00399850931092554\\
228	0.00399850802660956\\
229	0.00399850671913257\\
230	0.00399850538807049\\
231	0.00399850403299129\\
232	0.00399850265345476\\
233	0.0039985012490124\\
234	0.00399849981920724\\
235	0.00399849836357368\\
236	0.00399849688163731\\
237	0.00399849537291475\\
238	0.00399849383691345\\
239	0.00399849227313153\\
240	0.00399849068105759\\
241	0.00399848906017053\\
242	0.00399848740993934\\
243	0.00399848572982292\\
244	0.00399848401926985\\
245	0.00399848227771825\\
246	0.00399848050459551\\
247	0.00399847869931808\\
248	0.00399847686129133\\
249	0.00399847498990921\\
250	0.00399847308455413\\
251	0.00399847114459668\\
252	0.0039984691693954\\
253	0.00399846715829655\\
254	0.00399846511063386\\
255	0.00399846302572829\\
256	0.00399846090288778\\
257	0.00399845874140696\\
258	0.00399845654056695\\
259	0.00399845429963502\\
260	0.00399845201786438\\
261	0.00399844969449387\\
262	0.00399844732874765\\
263	0.00399844491983499\\
264	0.00399844246694992\\
265	0.00399843996927091\\
266	0.00399843742596062\\
267	0.00399843483616557\\
268	0.00399843219901583\\
269	0.00399842951362465\\
270	0.00399842677908822\\
271	0.00399842399448528\\
272	0.00399842115887676\\
273	0.00399841827130551\\
274	0.00399841533079588\\
275	0.00399841233635343\\
276	0.00399840928696449\\
277	0.00399840618159587\\
278	0.00399840301919443\\
279	0.00399839979868675\\
280	0.00399839651897868\\
281	0.00399839317895503\\
282	0.00399838977747912\\
283	0.00399838631339237\\
284	0.00399838278551395\\
285	0.00399837919264029\\
286	0.00399837553354475\\
287	0.0039983718069771\\
288	0.00399836801166317\\
289	0.00399836414630436\\
290	0.00399836020957722\\
291	0.00399835620013303\\
292	0.0039983521165973\\
293	0.00399834795756934\\
294	0.0039983437216218\\
295	0.00399833940730019\\
296	0.00399833501312243\\
297	0.00399833053757836\\
298	0.00399832597912925\\
299	0.00399832133620734\\
300	0.00399831660721534\\
301	0.00399831179052593\\
302	0.00399830688448126\\
303	0.00399830188739248\\
304	0.00399829679753921\\
305	0.00399829161316902\\
306	0.00399828633249696\\
307	0.00399828095370499\\
308	0.00399827547494151\\
309	0.00399826989432083\\
310	0.00399826420992262\\
311	0.00399825841979139\\
312	0.00399825252193598\\
313	0.003998246514329\\
314	0.00399824039490631\\
315	0.00399823416156646\\
316	0.00399822781217013\\
317	0.00399822134453963\\
318	0.0039982147564583\\
319	0.00399820804566994\\
320	0.00399820120987828\\
321	0.00399819424674638\\
322	0.00399818715389604\\
323	0.00399817992890722\\
324	0.00399817256931745\\
325	0.0039981650726212\\
326	0.00399815743626926\\
327	0.00399814965766809\\
328	0.00399814173417922\\
329	0.00399813366311852\\
330	0.00399812544175554\\
331	0.0039981170673128\\
332	0.00399810853696509\\
333	0.00399809984783868\\
334	0.00399809099701057\\
335	0.00399808198150768\\
336	0.00399807279830603\\
337	0.00399806344432988\\
338	0.00399805391645086\\
339	0.00399804421148699\\
340	0.00399803432620183\\
341	0.00399802425730339\\
342	0.00399801400144318\\
343	0.00399800355521513\\
344	0.00399799291515446\\
345	0.00399798207773657\\
346	0.00399797103937582\\
347	0.00399795979642428\\
348	0.00399794834517043\\
349	0.00399793668183774\\
350	0.00399792480258323\\
351	0.00399791270349591\\
352	0.00399790038059512\\
353	0.00399788782982874\\
354	0.00399787504707133\\
355	0.00399786202812206\\
356	0.00399784876870255\\
357	0.00399783526445451\\
358	0.00399782151093725\\
359	0.00399780750362489\\
360	0.00399779323790351\\
361	0.00399777870906792\\
362	0.00399776391231822\\
363	0.00399774884275618\\
364	0.00399773349538113\\
365	0.00399771786508571\\
366	0.00399770194665113\\
367	0.00399768573474207\\
368	0.00399766922390119\\
369	0.00399765240854318\\
370	0.00399763528294823\\
371	0.00399761784125501\\
372	0.00399760007745305\\
373	0.00399758198537448\\
374	0.00399756355868495\\
375	0.00399754479087392\\
376	0.00399752567524393\\
377	0.00399750620489901\\
378	0.003997486372732\\
379	0.00399746617141064\\
380	0.00399744559336238\\
381	0.0039974246307576\\
382	0.00399740327549112\\
383	0.00399738151916151\\
384	0.00399735935304781\\
385	0.00399733676808274\\
386	0.00399731375482103\\
387	0.00399729030340091\\
388	0.00399726640349559\\
389	0.00399724204425232\\
390	0.00399721721422017\\
391	0.00399719190128107\\
392	0.00399716609262436\\
393	0.00399713977482936\\
394	0.00399711293405988\\
395	0.00399708555608252\\
396	0.00399705762554789\\
397	0.00399702912580888\\
398	0.00399700003929598\\
399	0.00399697034747901\\
400	0.00399694003084921\\
401	0.00399690906893527\\
402	0.00399687744036482\\
403	0.00399684512295967\\
404	0.00399681209378542\\
405	0.00399677832894323\\
406	0.00399674380277119\\
407	0.0039967084863855\\
408	0.00399667234683618\\
409	0.00399663534974364\\
410	0.00399659746172118\\
411	0.0039965586467608\\
412	0.00399651886591\\
413	0.00399647807689943\\
414	0.00399643623371287\\
415	0.00399639328608915\\
416	0.00399634917894311\\
417	0.00399630385168904\\
418	0.00399625723744344\\
419	0.00399620926207135\\
420	0.00399615984301195\\
421	0.00399610888775217\\
422	0.00399605629165382\\
423	0.00399600193444172\\
424	0.00399594567370153\\
425	0.00399588733149965\\
426	0.00399582666528058\\
427	0.00399576330413516\\
428	0.00399569661432779\\
429	0.00399562544011629\\
430	0.0039955476934609\\
431	0.003995460009989\\
432	0.00399535855802043\\
433	0.00399524364579588\\
434	0.00399512615294169\\
435	0.00399500599877776\\
436	0.00399488309840421\\
437	0.00399475736237948\\
438	0.00399462869635417\\
439	0.00399449700062248\\
440	0.00399436216948627\\
441	0.00399422409013686\\
442	0.00399408264022094\\
443	0.00399393768172687\\
444	0.0039937890444934\\
445	0.00399363648040612\\
446	0.00399347953505816\\
447	0.00399331718877714\\
448	0.00399314686202122\\
449	0.00399296171003658\\
450	0.00399274350038222\\
451	0.00399244494390756\\
452	0.00399195091084209\\
453	0.00399101810193024\\
454	0.00398931856600917\\
455	0.00398745069141274\\
456	0.00398554914584857\\
457	0.00398361367123451\\
458	0.00398164398276477\\
459	0.00397963974954667\\
460	0.0039776005732863\\
461	0.00397552597425255\\
462	0.00397341540561641\\
463	0.00397126831564775\\
464	0.00396908418410526\\
465	0.00396686219389411\\
466	0.00396460132303283\\
467	0.00396230051428018\\
468	0.00395995871721609\\
469	0.00395757486966028\\
470	0.00395514780957969\\
471	0.00395267637158739\\
472	0.00395015937667282\\
473	0.00394759554206912\\
474	0.00394498327324883\\
475	0.00394232057158767\\
476	0.00393960589832815\\
477	0.00393683761784221\\
478	0.00393401398855618\\
479	0.00393113315363734\\
480	0.00392819313062419\\
481	0.00392519179982424\\
482	0.00392212689127023\\
483	0.00391899596998522\\
484	0.00391579641925903\\
485	0.00391252542157723\\
486	0.00390917993676815\\
487	0.00390575667684612\\
488	0.00390225207693731\\
489	0.00389866226152793\\
490	0.00389498300505269\\
491	0.00389120968580481\\
492	0.00388733723143133\\
493	0.00388336005374112\\
494	0.00387927196828374\\
495	0.00387506609003099\\
496	0.00387073468499104\\
497	0.00386626892556339\\
498	0.00386165842002803\\
499	0.00385689018870264\\
500	0.00385194627216657\\
501	0.00384679801097461\\
502	0.00384139254926805\\
503	0.00383562256041093\\
504	0.00382926552272995\\
505	0.00382189839320506\\
506	0.0038130775740306\\
507	0.00380388731055353\\
508	0.0037946208062746\\
509	0.00378526264768088\\
510	0.00377578725115241\\
511	0.00376619051702243\\
512	0.00375646791087849\\
513	0.00374661438458215\\
514	0.00373662419741183\\
515	0.00372649051374451\\
516	0.00371620429625266\\
517	0.0037057509206061\\
518	0.00369509933494614\\
519	0.00368416622119525\\
520	0.0036726943705344\\
521	0.00365983031567574\\
522	0.00364262745854996\\
523	0.0036106483834146\\
524	0.00355903794502149\\
525	0.00350587983412955\\
526	0.0034510683455673\\
527	0.00339447509023707\\
528	0.00333593083808788\\
529	0.0032751899804821\\
530	0.00321188216533384\\
531	0.00314556287288249\\
532	0.00307628479627494\\
533	0.0030052031131649\\
534	0.00293255211022299\\
535	0.00285732301245893\\
536	0.00277659728885553\\
537	0.00268580699715906\\
538	0.00257049800872363\\
539	0.0024390095813954\\
540	0.00230364665895996\\
541	0.00216414020967414\\
542	0.0020201905312924\\
543	0.00187146238020834\\
544	0.00171757856361426\\
545	0.00155810744068666\\
546	0.00139252543776588\\
547	0.00122011107609609\\
548	0.00103965922116657\\
549	0.000848609447595465\\
550	0.000640075640293778\\
551	0.000405854935500931\\
552	0.000159970413471418\\
553	0\\
554	0\\
555	0\\
556	0\\
557	0\\
558	0\\
559	0\\
560	0\\
561	0\\
562	0\\
563	0\\
564	0\\
565	0\\
566	0\\
567	0\\
568	0\\
569	0\\
570	0\\
571	0\\
572	0\\
573	0\\
574	0\\
575	0\\
576	0\\
577	0\\
578	0\\
579	0\\
580	0\\
581	0\\
582	0\\
583	0\\
584	0\\
585	0\\
586	0\\
587	0\\
588	0\\
589	0\\
590	0\\
591	0\\
592	0\\
593	0\\
594	0\\
595	0\\
596	0\\
597	0\\
598	0\\
599	0\\
600	0\\
};
\addplot [color=mycolor17,solid,forget plot]
  table[row sep=crcr]{%
1	0.00386320061721537\\
2	0.00386320017775786\\
3	0.00386319973044255\\
4	0.00386319927512914\\
5	0.00386319881167482\\
6	0.00386319833993427\\
7	0.00386319785975954\\
8	0.00386319737100007\\
9	0.0038631968735026\\
10	0.00386319636711114\\
11	0.00386319585166693\\
12	0.00386319532700837\\
13	0.00386319479297098\\
14	0.00386319424938736\\
15	0.00386319369608712\\
16	0.00386319313289682\\
17	0.00386319255963996\\
18	0.00386319197613686\\
19	0.00386319138220467\\
20	0.00386319077765725\\
21	0.00386319016230516\\
22	0.0038631895359556\\
23	0.0038631888984123\\
24	0.00386318824947553\\
25	0.00386318758894196\\
26	0.00386318691660467\\
27	0.00386318623225304\\
28	0.0038631855356727\\
29	0.00386318482664547\\
30	0.00386318410494927\\
31	0.00386318337035805\\
32	0.00386318262264177\\
33	0.00386318186156625\\
34	0.00386318108689318\\
35	0.00386318029837998\\
36	0.00386317949577974\\
37	0.00386317867884117\\
38	0.00386317784730851\\
39	0.0038631770009214\\
40	0.0038631761394149\\
41	0.00386317526251932\\
42	0.00386317436996016\\
43	0.00386317346145804\\
44	0.0038631725367286\\
45	0.00386317159548243\\
46	0.00386317063742494\\
47	0.00386316966225631\\
48	0.00386316866967137\\
49	0.00386316765935952\\
50	0.00386316663100463\\
51	0.00386316558428493\\
52	0.00386316451887291\\
53	0.00386316343443525\\
54	0.00386316233063269\\
55	0.0038631612071199\\
56	0.00386316006354544\\
57	0.00386315889955156\\
58	0.00386315771477418\\
59	0.00386315650884271\\
60	0.00386315528137998\\
61	0.00386315403200208\\
62	0.00386315276031827\\
63	0.00386315146593086\\
64	0.00386315014843505\\
65	0.00386314880741885\\
66	0.00386314744246293\\
67	0.00386314605314048\\
68	0.00386314463901709\\
69	0.00386314319965062\\
70	0.00386314173459104\\
71	0.00386314024338031\\
72	0.00386313872555222\\
73	0.00386313718063228\\
74	0.00386313560813751\\
75	0.00386313400757635\\
76	0.00386313237844846\\
77	0.00386313072024462\\
78	0.0038631290324465\\
79	0.00386312731452656\\
80	0.00386312556594783\\
81	0.0038631237861638\\
82	0.00386312197461821\\
83	0.00386312013074489\\
84	0.00386311825396756\\
85	0.00386311634369969\\
86	0.00386311439934429\\
87	0.00386311242029373\\
88	0.00386311040592954\\
89	0.00386310835562224\\
90	0.00386310626873112\\
91	0.00386310414460403\\
92	0.00386310198257722\\
93	0.00386309978197509\\
94	0.00386309754211\\
95	0.00386309526228204\\
96	0.00386309294177882\\
97	0.00386309057987525\\
98	0.0038630881758333\\
99	0.00386308572890178\\
100	0.0038630832383161\\
101	0.00386308070329802\\
102	0.00386307812305543\\
103	0.00386307549678208\\
104	0.00386307282365733\\
105	0.00386307010284588\\
106	0.00386306733349756\\
107	0.00386306451474699\\
108	0.00386306164571337\\
109	0.00386305872550014\\
110	0.00386305575319476\\
111	0.00386305272786839\\
112	0.0038630496485756\\
113	0.00386304651435407\\
114	0.00386304332422429\\
115	0.00386304007718928\\
116	0.0038630367722342\\
117	0.00386303340832611\\
118	0.00386302998441359\\
119	0.00386302649942646\\
120	0.00386302295227537\\
121	0.00386301934185151\\
122	0.00386301566702626\\
123	0.00386301192665079\\
124	0.00386300811955575\\
125	0.00386300424455084\\
126	0.00386300030042449\\
127	0.00386299628594344\\
128	0.00386299219985236\\
129	0.00386298804087345\\
130	0.00386298380770604\\
131	0.00386297949902615\\
132	0.0038629751134861\\
133	0.00386297064971409\\
134	0.0038629661063137\\
135	0.00386296148186353\\
136	0.00386295677491667\\
137	0.0038629519840003\\
138	0.00386294710761518\\
139	0.00386294214423517\\
140	0.00386293709230677\\
141	0.0038629319502486\\
142	0.00386292671645092\\
143	0.00386292138927506\\
144	0.00386291596705297\\
145	0.00386291044808661\\
146	0.00386290483064746\\
147	0.00386289911297593\\
148	0.00386289329328082\\
149	0.00386288736973871\\
150	0.00386288134049342\\
151	0.00386287520365537\\
152	0.00386286895730098\\
153	0.00386286259947206\\
154	0.00386285612817516\\
155	0.00386284954138096\\
156	0.00386284283702356\\
157	0.00386283601299985\\
158	0.00386282906716882\\
159	0.00386282199735084\\
160	0.003862814801327\\
161	0.00386280747683834\\
162	0.00386280002158514\\
163	0.00386279243322616\\
164	0.00386278470937789\\
165	0.00386277684761375\\
166	0.0038627688454633\\
167	0.00386276070041146\\
168	0.00386275240989764\\
169	0.00386274397131495\\
170	0.0038627353820093\\
171	0.00386272663927853\\
172	0.00386271774037157\\
173	0.00386270868248748\\
174	0.00386269946277452\\
175	0.00386269007832928\\
176	0.00386268052619563\\
177	0.00386267080336379\\
178	0.00386266090676933\\
179	0.00386265083329214\\
180	0.00386264057975542\\
181	0.00386263014292456\\
182	0.00386261951950614\\
183	0.00386260870614676\\
184	0.00386259769943198\\
185	0.00386258649588514\\
186	0.00386257509196619\\
187	0.00386256348407054\\
188	0.00386255166852781\\
189	0.00386253964160061\\
190	0.00386252739948329\\
191	0.00386251493830064\\
192	0.00386250225410657\\
193	0.00386248934288281\\
194	0.00386247620053752\\
195	0.00386246282290392\\
196	0.00386244920573884\\
197	0.00386243534472132\\
198	0.0038624212354511\\
199	0.00386240687344714\\
200	0.00386239225414605\\
201	0.00386237737290059\\
202	0.00386236222497802\\
203	0.0038623468055585\\
204	0.00386233110973342\\
205	0.0038623151325037\\
206	0.00386229886877809\\
207	0.00386228231337139\\
208	0.00386226546100264\\
209	0.00386224830629334\\
210	0.00386223084376549\\
211	0.00386221306783978\\
212	0.0038621949728336\\
213	0.00386217655295902\\
214	0.00386215780232084\\
215	0.00386213871491445\\
216	0.00386211928462378\\
217	0.00386209950521912\\
218	0.00386207937035493\\
219	0.00386205887356759\\
220	0.00386203800827314\\
221	0.00386201676776494\\
222	0.00386199514521127\\
223	0.00386197313365295\\
224	0.00386195072600083\\
225	0.00386192791503325\\
226	0.00386190469339351\\
227	0.00386188105358721\\
228	0.00386185698797957\\
229	0.0038618324887927\\
230	0.00386180754810279\\
231	0.00386178215783728\\
232	0.00386175630977196\\
233	0.00386172999552797\\
234	0.00386170320656881\\
235	0.00386167593419721\\
236	0.00386164816955202\\
237	0.00386161990360499\\
238	0.00386159112715746\\
239	0.00386156183083705\\
240	0.00386153200509424\\
241	0.00386150164019887\\
242	0.00386147072623659\\
243	0.00386143925310527\\
244	0.00386140721051128\\
245	0.00386137458796571\\
246	0.00386134137478055\\
247	0.00386130756006476\\
248	0.00386127313272027\\
249	0.00386123808143789\\
250	0.0038612023946932\\
251	0.00386116606074224\\
252	0.00386112906761724\\
253	0.0038610914031222\\
254	0.00386105305482836\\
255	0.00386101401006971\\
256	0.00386097425593821\\
257	0.00386093377927913\\
258	0.00386089256668612\\
259	0.00386085060449634\\
260	0.00386080787878537\\
261	0.00386076437536212\\
262	0.00386072007976358\\
263	0.0038606749772495\\
264	0.00386062905279701\\
265	0.00386058229109504\\
266	0.00386053467653874\\
267	0.00386048619322377\\
268	0.00386043682494045\\
269	0.00386038655516783\\
270	0.00386033536706773\\
271	0.0038602832434785\\
272	0.00386023016690889\\
273	0.00386017611953164\\
274	0.00386012108317706\\
275	0.00386006503932647\\
276	0.00386000796910552\\
277	0.00385994985327745\\
278	0.00385989067223624\\
279	0.00385983040599954\\
280	0.00385976903420166\\
281	0.00385970653608637\\
282	0.00385964289049955\\
283	0.00385957807588184\\
284	0.00385951207026108\\
285	0.00385944485124474\\
286	0.00385937639601218\\
287	0.00385930668130682\\
288	0.00385923568342824\\
289	0.00385916337822416\\
290	0.0038590897410823\\
291	0.00385901474692218\\
292	0.00385893837018685\\
293	0.00385886058483442\\
294	0.00385878136432962\\
295	0.00385870068163524\\
296	0.00385861850920341\\
297	0.00385853481896692\\
298	0.00385844958233035\\
299	0.00385836277016122\\
300	0.00385827435278097\\
301	0.00385818429995595\\
302	0.0038580925808883\\
303	0.00385799916420676\\
304	0.00385790401795746\\
305	0.00385780710959461\\
306	0.00385770840597112\\
307	0.00385760787332925\\
308	0.0038575054772911\\
309	0.00385740118284909\\
310	0.00385729495435646\\
311	0.00385718675551761\\
312	0.00385707654937844\\
313	0.00385696429831669\\
314	0.00385684996403215\\
315	0.00385673350753685\\
316	0.00385661488914523\\
317	0.0038564940684642\\
318	0.00385637100438315\\
319	0.00385624565506394\\
320	0.00385611797793069\\
321	0.00385598792965965\\
322	0.0038558554661688\\
323	0.00385572054260746\\
324	0.0038555831133457\\
325	0.00385544313196363\\
326	0.00385530055124052\\
327	0.00385515532314371\\
328	0.00385500739881735\\
329	0.00385485672857082\\
330	0.00385470326186702\\
331	0.00385454694731019\\
332	0.00385438773263353\\
333	0.00385422556468642\\
334	0.00385406038942125\\
335	0.00385389215187984\\
336	0.00385372079617938\\
337	0.00385354626549792\\
338	0.00385336850205936\\
339	0.0038531874471179\\
340	0.00385300304094193\\
341	0.00385281522279737\\
342	0.0038526239309304\\
343	0.00385242910254955\\
344	0.00385223067380719\\
345	0.0038520285797802\\
346	0.00385182275444996\\
347	0.00385161313068141\\
348	0.00385139964020125\\
349	0.00385118221357495\\
350	0.0038509607801827\\
351	0.00385073526819387\\
352	0.00385050560453995\\
353	0.00385027171488567\\
354	0.00385003352359821\\
355	0.00384979095371409\\
356	0.0038495439269038\\
357	0.00384929236343364\\
358	0.00384903618212468\\
359	0.00384877530030865\\
360	0.00384850963378012\\
361	0.00384823909674503\\
362	0.00384796360176499\\
363	0.00384768305969697\\
364	0.00384739737962805\\
365	0.00384710646880478\\
366	0.00384681023255653\\
367	0.00384650857421251\\
368	0.00384620139501168\\
369	0.00384588859400512\\
370	0.00384557006795003\\
371	0.00384524571119469\\
372	0.00384491541555363\\
373	0.00384457907017194\\
374	0.00384423656137798\\
375	0.00384388777252315\\
376	0.00384353258380758\\
377	0.00384317087209046\\
378	0.00384280251068319\\
379	0.0038424273691237\\
380	0.00384204531292975\\
381	0.00384165620332866\\
382	0.00384125989696056\\
383	0.00384085624555107\\
384	0.00384044509554857\\
385	0.00384002628771845\\
386	0.0038395996566827\\
387	0.00383916503038492\\
388	0.00383872222944601\\
389	0.00383827106635165\\
390	0.00383781134438536\\
391	0.00383734285622546\\
392	0.0038368653822783\\
393	0.00383637868937519\\
394	0.00383588253162161\\
395	0.00383537665575838\\
396	0.00383486080689695\\
397	0.00383433471050525\\
398	0.00383379806491724\\
399	0.00383325055168658\\
400	0.0038326918347335\\
401	0.00383212155977938\\
402	0.00383153935435902\\
403	0.00383094482882212\\
404	0.0038303375786635\\
405	0.00382971718767689\\
406	0.00382908322864968\\
407	0.00382843525276533\\
408	0.00382777275585792\\
409	0.00382709513659304\\
410	0.00382640175099086\\
411	0.00382569200259897\\
412	0.00382496524903276\\
413	0.00382422079636952\\
414	0.00382345789269941\\
415	0.00382267572069011\\
416	0.00382187338899307\\
417	0.00382104992228452\\
418	0.00382020424968888\\
419	0.00381933519126889\\
420	0.00381844144216556\\
421	0.00381752155378253\\
422	0.00381657391099624\\
423	0.00381559670336089\\
424	0.00381458788562918\\
425	0.00381354511578655\\
426	0.0038124656395306\\
427	0.00381134603853017\\
428	0.00381018162513228\\
429	0.00380896493197543\\
430	0.00380768199599962\\
431	0.0038063038431679\\
432	0.00380477037323438\\
433	0.0038029775999075\\
434	0.00380086129815737\\
435	0.00379869980355397\\
436	0.00379649180033432\\
437	0.00379423590688405\\
438	0.0037919306709257\\
439	0.00378957456430551\\
440	0.00378716597728736\\
441	0.0037847032121264\\
442	0.00378218447528161\\
443	0.00377960786636622\\
444	0.003776971358026\\
445	0.00377427274861871\\
446	0.00377150953004803\\
447	0.00376867848383706\\
448	0.0037657743871938\\
449	0.0037627857423846\\
450	0.00375968034386319\\
451	0.00375635545412733\\
452	0.00375246236568108\\
453	0.00374677720045931\\
454	0.00373490643622551\\
455	0.00372160884622855\\
456	0.00370803741193669\\
457	0.00369418850366611\\
458	0.00368005842123568\\
459	0.00366564328300176\\
460	0.00365093884673678\\
461	0.00363594024617064\\
462	0.00362064172233534\\
463	0.00360503677434\\
464	0.00358911983299576\\
465	0.00357288930539948\\
466	0.00355633899536265\\
467	0.00353945868158071\\
468	0.00352223773711009\\
469	0.00350466642955815\\
470	0.00348673600253414\\
471	0.00346843579184816\\
472	0.00344975571601982\\
473	0.00343068666761139\\
474	0.00341121947784049\\
475	0.00339133947212614\\
476	0.00337101863885799\\
477	0.00335024169153026\\
478	0.00332899237954051\\
479	0.00330725338479783\\
480	0.00328500622541673\\
481	0.00326223114833906\\
482	0.00323890700911283\\
483	0.00321501113671962\\
484	0.00319051918094141\\
485	0.00316540493926736\\
486	0.00313964015970263\\
487	0.00311319431499142\\
488	0.00308603434280269\\
489	0.00305812434561894\\
490	0.00302942524228893\\
491	0.00299989435999793\\
492	0.00296948495800572\\
493	0.00293814566607575\\
494	0.0029058198204287\\
495	0.00287244466145977\\
496	0.00283795034066952\\
497	0.00280225863393555\\
498	0.00276528106212388\\
499	0.00272691564219204\\
500	0.00268704007475841\\
501	0.0026454950243867\\
502	0.00260203892819086\\
503	0.00255621963891283\\
504	0.0025070010893715\\
505	0.0024515571816792\\
506	0.0023804071624726\\
507	0.00230400224055517\\
508	0.00222627062227436\\
509	0.00214744806552538\\
510	0.00206730837867354\\
511	0.00198508682477377\\
512	0.00190067678391583\\
513	0.00181396187354246\\
514	0.00172481490632849\\
515	0.00163309649818443\\
516	0.00153865394445642\\
517	0.00144132161519882\\
518	0.00134092725442683\\
519	0.00123732123756069\\
520	0.00113049282112899\\
521	0.0010210130159867\\
522	0.000911700043803545\\
523	0.000813861606725609\\
524	0.00073313170074981\\
525	0.000649680795420767\\
526	0.000563311348217927\\
527	0.000473772576124755\\
528	0.00038068974940592\\
529	0.000283321274931334\\
530	0.000179753860783708\\
531	6.42317041917577e-05\\
532	0\\
533	0\\
534	0\\
535	0\\
536	0\\
537	0\\
538	0\\
539	0\\
540	0\\
541	0\\
542	0\\
543	0\\
544	0\\
545	0\\
546	0\\
547	0\\
548	0\\
549	0\\
550	0\\
551	0\\
552	0\\
553	0\\
554	0\\
555	0\\
556	0\\
557	0\\
558	0\\
559	0\\
560	0\\
561	0\\
562	0\\
563	0\\
564	0\\
565	0\\
566	0\\
567	0\\
568	0\\
569	0\\
570	0\\
571	0\\
572	0\\
573	0\\
574	0\\
575	0\\
576	0\\
577	0\\
578	0\\
579	0\\
580	0\\
581	0\\
582	0\\
583	0\\
584	0\\
585	0\\
586	0\\
587	0\\
588	0\\
589	0\\
590	0\\
591	0\\
592	0\\
593	0\\
594	0\\
595	0\\
596	0\\
597	0\\
598	0\\
599	0\\
600	0\\
};
\addplot [color=mycolor18,solid,forget plot]
  table[row sep=crcr]{%
1	0.00283984636363328\\
2	0.00283984271850578\\
3	0.00283983900819595\\
4	0.00283983523153989\\
5	0.00283983138735291\\
6	0.0028398274744292\\
7	0.00283982349154148\\
8	0.00283981943744058\\
9	0.00283981531085504\\
10	0.00283981111049076\\
11	0.00283980683503055\\
12	0.00283980248313376\\
13	0.00283979805343584\\
14	0.00283979354454789\\
15	0.00283978895505627\\
16	0.00283978428352215\\
17	0.00283977952848104\\
18	0.00283977468844236\\
19	0.00283976976188895\\
20	0.00283976474727663\\
21	0.00283975964303366\\
22	0.00283975444756032\\
23	0.00283974915922837\\
24	0.00283974377638055\\
25	0.00283973829733007\\
26	0.00283973272036007\\
27	0.0028397270437231\\
28	0.00283972126564055\\
29	0.00283971538430215\\
30	0.00283970939786535\\
31	0.00283970330445476\\
32	0.00283969710216157\\
33	0.00283969078904297\\
34	0.00283968436312153\\
35	0.00283967782238456\\
36	0.00283967116478354\\
37	0.00283966438823341\\
38	0.00283965749061199\\
39	0.00283965046975926\\
40	0.00283964332347674\\
41	0.00283963604952675\\
42	0.00283962864563174\\
43	0.0028396211094736\\
44	0.0028396134386929\\
45	0.00283960563088817\\
46	0.00283959768361516\\
47	0.00283958959438605\\
48	0.00283958136066871\\
49	0.00283957297988588\\
50	0.0028395644494144\\
51	0.00283955576658435\\
52	0.00283954692867825\\
53	0.00283953793293018\\
54	0.00283952877652497\\
55	0.00283951945659726\\
56	0.00283950997023065\\
57	0.00283950031445677\\
58	0.00283949048625435\\
59	0.0028394804825483\\
60	0.00283947030020873\\
61	0.00283945993604997\\
62	0.0028394493868296\\
63	0.00283943864924741\\
64	0.00283942771994438\\
65	0.00283941659550164\\
66	0.00283940527243939\\
67	0.00283939374721582\\
68	0.00283938201622599\\
69	0.00283937007580074\\
70	0.0028393579222055\\
71	0.00283934555163913\\
72	0.00283933296023278\\
73	0.0028393201440486\\
74	0.0028393070990786\\
75	0.00283929382124331\\
76	0.00283928030639056\\
77	0.00283926655029417\\
78	0.00283925254865262\\
79	0.00283923829708771\\
80	0.00283922379114319\\
81	0.00283920902628336\\
82	0.00283919399789164\\
83	0.00283917870126916\\
84	0.00283916313163326\\
85	0.002839147284116\\
86	0.00283913115376263\\
87	0.00283911473553003\\
88	0.00283909802428516\\
89	0.00283908101480343\\
90	0.00283906370176706\\
91	0.00283904607976341\\
92	0.00283902814328329\\
93	0.00283900988671924\\
94	0.00283899130436373\\
95	0.00283897239040744\\
96	0.00283895313893736\\
97	0.00283893354393498\\
98	0.00283891359927438\\
99	0.00283889329872031\\
100	0.00283887263592623\\
101	0.00283885160443232\\
102	0.00283883019766343\\
103	0.00283880840892703\\
104	0.00283878623141109\\
105	0.00283876365818194\\
106	0.00283874068218209\\
107	0.002838717296228\\
108	0.00283869349300782\\
109	0.00283866926507907\\
110	0.00283864460486633\\
111	0.0028386195046588\\
112	0.00283859395660789\\
113	0.00283856795272475\\
114	0.00283854148487772\\
115	0.0028385145447898\\
116	0.00283848712403599\\
117	0.00283845921404063\\
118	0.00283843080607472\\
119	0.00283840189125311\\
120	0.00283837246053174\\
121	0.00283834250470471\\
122	0.00283831201440138\\
123	0.00283828098008345\\
124	0.00283824939204186\\
125	0.00283821724039373\\
126	0.00283818451507922\\
127	0.00283815120585835\\
128	0.00283811730230773\\
129	0.00283808279381722\\
130	0.00283804766958658\\
131	0.00283801191862198\\
132	0.00283797552973258\\
133	0.00283793849152687\\
134	0.00283790079240907\\
135	0.00283786242057544\\
136	0.00283782336401046\\
137	0.00283778361048304\\
138	0.00283774314754256\\
139	0.00283770196251489\\
140	0.00283766004249836\\
141	0.00283761737435955\\
142	0.00283757394472915\\
143	0.00283752973999759\\
144	0.00283748474631073\\
145	0.00283743894956534\\
146	0.00283739233540461\\
147	0.00283734488921348\\
148	0.00283729659611395\\
149	0.00283724744096026\\
150	0.00283719740833397\\
151	0.00283714648253905\\
152	0.00283709464759671\\
153	0.00283704188724026\\
154	0.00283698818490984\\
155	0.00283693352374703\\
156	0.00283687788658937\\
157	0.00283682125596478\\
158	0.00283676361408586\\
159	0.00283670494284409\\
160	0.00283664522380397\\
161	0.00283658443819692\\
162	0.00283652256691517\\
163	0.00283645959050554\\
164	0.00283639548916303\\
165	0.00283633024272432\\
166	0.00283626383066115\\
167	0.00283619623207359\\
168	0.00283612742568315\\
169	0.00283605738982573\\
170	0.00283598610244452\\
171	0.00283591354108269\\
172	0.00283583968287596\\
173	0.00283576450454504\\
174	0.00283568798238789\\
175	0.00283561009227186\\
176	0.00283553080962565\\
177	0.00283545010943117\\
178	0.00283536796621515\\
179	0.00283528435404067\\
180	0.00283519924649847\\
181	0.00283511261669812\\
182	0.002835024437259\\
183	0.00283493468030112\\
184	0.00283484331743571\\
185	0.00283475031975571\\
186	0.002834655657826\\
187	0.00283455930167346\\
188	0.00283446122077682\\
189	0.00283436138405635\\
190	0.00283425975986331\\
191	0.00283415631596916\\
192	0.00283405101955467\\
193	0.00283394383719864\\
194	0.00283383473486656\\
195	0.00283372367789893\\
196	0.00283361063099944\\
197	0.00283349555822279\\
198	0.00283337842296239\\
199	0.00283325918793775\\
200	0.0028331378151816\\
201	0.00283301426602681\\
202	0.00283288850109298\\
203	0.00283276048027282\\
204	0.00283263016271819\\
205	0.00283249750682592\\
206	0.0028323624702233\\
207	0.00283222500975327\\
208	0.00283208508145939\\
209	0.00283194264057037\\
210	0.00283179764148436\\
211	0.00283165003775298\\
212	0.00283149978206489\\
213	0.00283134682622912\\
214	0.00283119112115802\\
215	0.00283103261684988\\
216	0.00283087126237117\\
217	0.00283070700583846\\
218	0.00283053979439989\\
219	0.00283036957421633\\
220	0.00283019629044213\\
221	0.00283001988720545\\
222	0.00282984030758825\\
223	0.00282965749360577\\
224	0.00282947138618563\\
225	0.00282928192514657\\
226	0.00282908904917662\\
227	0.0028288926958109\\
228	0.00282869280140898\\
229	0.00282848930113166\\
230	0.00282828212891742\\
231	0.00282807121745825\\
232	0.00282785649817504\\
233	0.0028276379011925\\
234	0.00282741535531345\\
235	0.00282718878799268\\
236	0.00282695812531021\\
237	0.00282672329194402\\
238	0.00282648421114222\\
239	0.00282624080469461\\
240	0.0028259929929037\\
241	0.00282574069455508\\
242	0.00282548382688723\\
243	0.00282522230556064\\
244	0.00282495604462639\\
245	0.00282468495649398\\
246	0.00282440895189856\\
247	0.00282412793986747\\
248	0.00282384182768609\\
249	0.00282355052086301\\
250	0.00282325392309446\\
251	0.00282295193622805\\
252	0.00282264446022572\\
253	0.00282233139312598\\
254	0.0028220126310054\\
255	0.00282168806793927\\
256	0.0028213575959615\\
257	0.00282102110502374\\
258	0.00282067848295363\\
259	0.00282032961541223\\
260	0.00281997438585066\\
261	0.00281961267546582\\
262	0.00281924436315529\\
263	0.00281886932547126\\
264	0.0028184874365737\\
265	0.00281809856818253\\
266	0.00281770258952883\\
267	0.00281729936730527\\
268	0.00281688876561541\\
269	0.00281647064592223\\
270	0.0028160448669955\\
271	0.00281561128485836\\
272	0.00281516975273278\\
273	0.00281472012098409\\
274	0.00281426223706446\\
275	0.0028137959454554\\
276	0.00281332108760923\\
277	0.00281283750188951\\
278	0.00281234502351041\\
279	0.00281184348447511\\
280	0.00281133271351306\\
281	0.00281081253601624\\
282	0.00281028277397436\\
283	0.002809743245909\\
284	0.00280919376680665\\
285	0.00280863414805074\\
286	0.00280806419735257\\
287	0.00280748371868121\\
288	0.00280689251219233\\
289	0.00280629037415594\\
290	0.00280567709688316\\
291	0.00280505246865183\\
292	0.00280441627363122\\
293	0.00280376829180558\\
294	0.00280310829889677\\
295	0.0028024360662858\\
296	0.00280175136093345\\
297	0.00280105394529983\\
298	0.00280034357726302\\
299	0.00279962001003669\\
300	0.00279888299208685\\
301	0.00279813226704765\\
302	0.00279736757363616\\
303	0.00279658864556642\\
304	0.00279579521146249\\
305	0.00279498699477061\\
306	0.00279416371367063\\
307	0.00279332508098644\\
308	0.00279247080409567\\
309	0.00279160058483853\\
310	0.00279071411942579\\
311	0.002789811098346\\
312	0.00278889120627186\\
313	0.00278795412196575\\
314	0.00278699951818442\\
315	0.00278602706158284\\
316	0.00278503641261714\\
317	0.00278402722544659\\
318	0.00278299914783472\\
319	0.00278195182104928\\
320	0.00278088487976116\\
321	0.00277979795194219\\
322	0.00277869065876157\\
323	0.00277756261448098\\
324	0.00277641342634813\\
325	0.00277524269448873\\
326	0.00277405001179652\\
327	0.00277283496382143\\
328	0.00277159712865538\\
329	0.00277033607681578\\
330	0.00276905137112628\\
331	0.00276774256659457\\
332	0.00276640921028704\\
333	0.00276505084119987\\
334	0.00276366699012633\\
335	0.00276225717952003\\
336	0.0027608209233537\\
337	0.00275935772697327\\
338	0.002757867086947\\
339	0.00275634849090932\\
340	0.00275480141739916\\
341	0.00275322533569259\\
342	0.00275161970562948\\
343	0.00274998397743406\\
344	0.00274831759152909\\
345	0.0027466199783432\\
346	0.00274489055811124\\
347	0.00274312874066685\\
348	0.00274133392522688\\
349	0.00273950550016684\\
350	0.0027376428427863\\
351	0.00273574531906328\\
352	0.0027338122833961\\
353	0.00273184307833093\\
354	0.00272983703427334\\
355	0.00272779346918177\\
356	0.00272571168824128\\
357	0.00272359098351554\\
358	0.0027214306335751\\
359	0.00271922990309963\\
360	0.00271698804245171\\
361	0.00271470428721956\\
362	0.00271237785772575\\
363	0.00271000795849902\\
364	0.00270759377770548\\
365	0.00270513448653603\\
366	0.00270262923854566\\
367	0.00270007716894068\\
368	0.00269747739380911\\
369	0.00269482900928932\\
370	0.00269213109067148\\
371	0.00268938269142588\\
372	0.00268658284215176\\
373	0.00268373054943951\\
374	0.00268082479463852\\
375	0.0026778645325221\\
376	0.00267484868983998\\
377	0.00267177616374766\\
378	0.00266864582010079\\
379	0.00266545649160092\\
380	0.00266220697577716\\
381	0.00265889603278586\\
382	0.0026555223830072\\
383	0.00265208470441345\\
384	0.00264858162967697\\
385	0.00264501174297499\\
386	0.002641373576426\\
387	0.00263766560604577\\
388	0.00263388624700499\\
389	0.002630033847735\\
390	0.00262610668192722\\
391	0.00262210293652132\\
392	0.00261802069239686\\
393	0.00261385789407911\\
394	0.00260961231188414\\
395	0.00260528153371685\\
396	0.00260086310772563\\
397	0.00259635492022998\\
398	0.00259175470647134\\
399	0.00258705971132202\\
400	0.00258226702724363\\
401	0.00257737358682149\\
402	0.00257237615923375\\
403	0.00256727135586422\\
404	0.00256205565503259\\
405	0.00255672546191674\\
406	0.00255127721934318\\
407	0.00254570754700848\\
408	0.00254001322769217\\
409	0.00253419048557407\\
410	0.002528233073708\\
411	0.00252213320223588\\
412	0.00251588545339777\\
413	0.0025094840052289\\
414	0.00250292258335434\\
415	0.00249619440555662\\
416	0.00248929211789261\\
417	0.00248220772088192\\
418	0.00247493248399702\\
419	0.00246745684632346\\
420	0.00245977030079602\\
421	0.00245186125879021\\
422	0.00244371689089691\\
423	0.00243532293799798\\
424	0.00242666348297729\\
425	0.00241772066349657\\
426	0.00240847427715346\\
427	0.00239890113913909\\
428	0.00238897375490957\\
429	0.00237865687037138\\
430	0.0023678970214388\\
431	0.00235658812589125\\
432	0.00234445301092195\\
433	0.00233062431083223\\
434	0.00231321595823006\\
435	0.00229538887872112\\
436	0.00227712911169738\\
437	0.00225842195679285\\
438	0.00223925191838529\\
439	0.00221960264533866\\
440	0.00219945686561942\\
441	0.00217879631548368\\
442	0.00215760166320238\\
443	0.00213585242824216\\
444	0.0021135269000313\\
445	0.00209060207126689\\
446	0.00206705363748975\\
447	0.00204285624100025\\
448	0.00201798457705556\\
449	0.0019924175329847\\
450	0.00196615308900204\\
451	0.0019392618814366\\
452	0.00191208150788773\\
453	0.00188592980328271\\
454	0.00186575499415459\\
455	0.00184634549182879\\
456	0.0018264179726812\\
457	0.00180595847354217\\
458	0.001784953701425\\
459	0.00176339107088575\\
460	0.00174125826616331\\
461	0.00171854154734407\\
462	0.00169522118118219\\
463	0.00167126166615424\\
464	0.00164659860986812\\
465	0.00162116075735014\\
466	0.0015950962240734\\
467	0.00156848250343061\\
468	0.00154128321357145\\
469	0.00151344825891532\\
470	0.00148496540590409\\
471	0.00145590469527504\\
472	0.00142626966023149\\
473	0.0013960911929633\\
474	0.00136545656688742\\
475	0.00133453732621733\\
476	0.00130348109058821\\
477	0.00127171877652943\\
478	0.00123922891148697\\
479	0.00120598968326226\\
480	0.00117197819339074\\
481	0.00113717038199236\\
482	0.00110154094938564\\
483	0.0010650632747847\\
484	0.00102770933251127\\
485	0.000989449606398602\\
486	0.000950253003461431\\
487	0.000910086768033306\\
488	0.000868916396119481\\
489	0.000826705547804407\\
490	0.000783415967386659\\
491	0.000739007424426463\\
492	0.000693437597504418\\
493	0.000646661989758019\\
494	0.000598633811270603\\
495	0.00054930392098585\\
496	0.000498620679010176\\
497	0.000446529504226808\\
498	0.000392972284849068\\
499	0.000337885764413177\\
500	0.000281196759940849\\
501	0.000222807664419292\\
502	0.000162552183188757\\
503	0.000100059625682899\\
504	3.43368831611899e-05\\
505	0\\
506	0\\
507	0\\
508	0\\
509	0\\
510	0\\
511	0\\
512	0\\
513	0\\
514	0\\
515	0\\
516	0\\
517	0\\
518	0\\
519	0\\
520	0\\
521	0\\
522	0\\
523	0\\
524	0\\
525	0\\
526	0\\
527	0\\
528	0\\
529	0\\
530	0\\
531	0\\
532	0\\
533	0\\
534	0\\
535	0\\
536	0\\
537	0\\
538	0\\
539	0\\
540	0\\
541	0\\
542	0\\
543	0\\
544	0\\
545	0\\
546	0\\
547	0\\
548	0\\
549	0\\
550	0\\
551	0\\
552	0\\
553	0\\
554	0\\
555	0\\
556	0\\
557	0\\
558	0\\
559	0\\
560	0\\
561	0\\
562	0\\
563	0\\
564	0\\
565	0\\
566	0\\
567	0\\
568	0\\
569	0\\
570	0\\
571	0\\
572	0\\
573	0\\
574	0\\
575	0\\
576	0\\
577	0\\
578	0\\
579	0\\
580	0\\
581	0\\
582	0\\
583	0\\
584	0\\
585	0\\
586	0\\
587	0\\
588	0\\
589	0\\
590	0\\
591	0\\
592	0\\
593	0\\
594	0\\
595	0\\
596	0\\
597	0\\
598	0\\
599	0\\
600	0\\
};
\addplot [color=red!25!mycolor17,solid,forget plot]
  table[row sep=crcr]{%
1	0.00135496690887878\\
2	0.00135496089430922\\
3	0.0013549547721517\\
4	0.00135494854048434\\
5	0.00135494219735098\\
6	0.00135493574076059\\
7	0.00135492916868658\\
8	0.00135492247906624\\
9	0.0013549156698001\\
10	0.00135490873875119\\
11	0.00135490168374449\\
12	0.00135489450256614\\
13	0.00135488719296282\\
14	0.00135487975264104\\
15	0.00135487217926638\\
16	0.00135486447046282\\
17	0.00135485662381197\\
18	0.00135484863685228\\
19	0.00135484050707835\\
20	0.00135483223194009\\
21	0.00135482380884195\\
22	0.00135481523514208\\
23	0.00135480650815155\\
24	0.00135479762513347\\
25	0.00135478858330216\\
26	0.00135477937982228\\
27	0.00135477001180794\\
28	0.0013547604763218\\
29	0.00135475077037414\\
30	0.00135474089092195\\
31	0.00135473083486798\\
32	0.00135472059905976\\
33	0.00135471018028861\\
34	0.00135469957528865\\
35	0.00135468878073582\\
36	0.00135467779324675\\
37	0.00135466660937781\\
38	0.00135465522562394\\
39	0.00135464363841763\\
40	0.00135463184412777\\
41	0.00135461983905852\\
42	0.00135460761944819\\
43	0.00135459518146799\\
44	0.00135458252122093\\
45	0.00135456963474054\\
46	0.00135455651798963\\
47	0.0013545431668591\\
48	0.00135452957716658\\
49	0.00135451574465513\\
50	0.00135450166499198\\
51	0.00135448733376711\\
52	0.00135447274649191\\
53	0.00135445789859774\\
54	0.00135444278543457\\
55	0.00135442740226945\\
56	0.00135441174428509\\
57	0.00135439580657834\\
58	0.00135437958415866\\
59	0.00135436307194654\\
60	0.00135434626477197\\
61	0.00135432915737274\\
62	0.0013543117443929\\
63	0.001354294020381\\
64	0.00135427597978841\\
65	0.00135425761696763\\
66	0.00135423892617049\\
67	0.00135421990154635\\
68	0.00135420053714027\\
69	0.00135418082689117\\
70	0.00135416076462995\\
71	0.00135414034407751\\
72	0.00135411955884283\\
73	0.00135409840242096\\
74	0.00135407686819098\\
75	0.00135405494941393\\
76	0.00135403263923073\\
77	0.00135400993066\\
78	0.0013539868165959\\
79	0.00135396328980589\\
80	0.00135393934292847\\
81	0.0013539149684709\\
82	0.00135389015880683\\
83	0.00135386490617392\\
84	0.00135383920267142\\
85	0.00135381304025768\\
86	0.00135378641074764\\
87	0.00135375930581028\\
88	0.00135373171696598\\
89	0.00135370363558388\\
90	0.0013536750528792\\
91	0.00135364595991044\\
92	0.00135361634757661\\
93	0.00135358620661437\\
94	0.00135355552759512\\
95	0.00135352430092204\\
96	0.00135349251682708\\
97	0.00135346016536791\\
98	0.00135342723642478\\
99	0.00135339371969735\\
100	0.00135335960470146\\
101	0.00135332488076582\\
102	0.00135328953702868\\
103	0.00135325356243441\\
104	0.00135321694573\\
105	0.00135317967546154\\
106	0.00135314173997065\\
107	0.00135310312739074\\
108	0.00135306382564332\\
109	0.00135302382243422\\
110	0.00135298310524965\\
111	0.0013529416613523\\
112	0.00135289947777735\\
113	0.00135285654132832\\
114	0.00135281283857295\\
115	0.00135276835583895\\
116	0.0013527230792097\\
117	0.00135267699451981\\
118	0.0013526300873507\\
119	0.001352582343026\\
120	0.00135253374660697\\
121	0.00135248428288769\\
122	0.0013524339363903\\
123	0.00135238269136011\\
124	0.00135233053176057\\
125	0.00135227744126824\\
126	0.00135222340326755\\
127	0.0013521684008456\\
128	0.00135211241678672\\
129	0.00135205543356708\\
130	0.00135199743334905\\
131	0.00135193839797559\\
132	0.00135187830896445\\
133	0.00135181714750229\\
134	0.00135175489443867\\
135	0.00135169153028\\
136	0.00135162703518328\\
137	0.00135156138894978\\
138	0.00135149457101863\\
139	0.00135142656046019\\
140	0.00135135733596943\\
141	0.00135128687585906\\
142	0.00135121515805263\\
143	0.00135114216007744\\
144	0.00135106785905737\\
145	0.00135099223170547\\
146	0.0013509152543166\\
147	0.0013508369027597\\
148	0.00135075715247013\\
149	0.00135067597844172\\
150	0.00135059335521874\\
151	0.00135050925688768\\
152	0.00135042365706891\\
153	0.00135033652890818\\
154	0.00135024784506793\\
155	0.00135015757771846\\
156	0.00135006569852893\\
157	0.00134997217865818\\
158	0.00134987698874539\\
159	0.00134978009890056\\
160	0.00134968147869478\\
161	0.00134958109715037\\
162	0.00134947892273082\\
163	0.00134937492333046\\
164	0.00134926906626402\\
165	0.00134916131825601\\
166	0.00134905164542979\\
167	0.0013489400132965\\
168	0.00134882638674382\\
169	0.0013487107300244\\
170	0.00134859300674417\\
171	0.00134847317985039\\
172	0.00134835121161942\\
173	0.00134822706364436\\
174	0.00134810069682235\\
175	0.00134797207134167\\
176	0.00134784114666862\\
177	0.00134770788153404\\
178	0.00134757223391971\\
179	0.00134743416104437\\
180	0.00134729361934952\\
181	0.00134715056448493\\
182	0.00134700495129389\\
183	0.00134685673379811\\
184	0.00134670586518244\\
185	0.00134655229777907\\
186	0.00134639598305176\\
187	0.00134623687157937\\
188	0.00134607491303939\\
189	0.00134591005619096\\
190	0.00134574224885759\\
191	0.00134557143790959\\
192	0.00134539756924606\\
193	0.00134522058777666\\
194	0.00134504043740283\\
195	0.00134485706099881\\
196	0.00134467040039217\\
197	0.00134448039634399\\
198	0.00134428698852867\\
199	0.00134409011551324\\
200	0.00134388971473637\\
201	0.00134368572248689\\
202	0.00134347807388187\\
203	0.00134326670284429\\
204	0.00134305154208025\\
205	0.00134283252305568\\
206	0.00134260957597265\\
207	0.0013423826297451\\
208	0.00134215161197416\\
209	0.00134191644892294\\
210	0.00134167706549077\\
211	0.00134143338518696\\
212	0.00134118533010397\\
213	0.00134093282089009\\
214	0.00134067577672151\\
215	0.0013404141152738\\
216	0.00134014775269288\\
217	0.00133987660356527\\
218	0.00133960058088785\\
219	0.00133931959603687\\
220	0.00133903355873639\\
221	0.00133874237702607\\
222	0.00133844595722824\\
223	0.00133814420391427\\
224	0.00133783701987031\\
225	0.00133752430606225\\
226	0.00133720596159994\\
227	0.0013368818837007\\
228	0.00133655196765203\\
229	0.00133621610677351\\
230	0.00133587419237795\\
231	0.00133552611373166\\
232	0.00133517175801394\\
233	0.0013348110102756\\
234	0.00133444375339676\\
235	0.00133406986804358\\
236	0.00133368923262419\\
237	0.00133330172324359\\
238	0.00133290721365768\\
239	0.00133250557522622\\
240	0.00133209667686478\\
241	0.00133168038499576\\
242	0.00133125656349823\\
243	0.00133082507365676\\
244	0.00133038577410916\\
245	0.00132993852079302\\
246	0.00132948316689118\\
247	0.00132901956277597\\
248	0.00132854755595224\\
249	0.00132806699099922\\
250	0.00132757770951103\\
251	0.00132707955003596\\
252	0.00132657234801445\\
253	0.00132605593571566\\
254	0.00132553014217272\\
255	0.00132499479311653\\
256	0.00132444971090822\\
257	0.00132389471446996\\
258	0.00132332961921451\\
259	0.001322754236973\\
260	0.00132216837592137\\
261	0.001321571840505\\
262	0.0013209644313619\\
263	0.00132034594524407\\
264	0.00131971617493732\\
265	0.00131907490917922\\
266	0.00131842193257536\\
267	0.00131775702551378\\
268	0.00131707996407755\\
269	0.00131639051995551\\
270	0.00131568846035101\\
271	0.00131497354788884\\
272	0.00131424554052002\\
273	0.00131350419142467\\
274	0.00131274924891278\\
275	0.00131198045632283\\
276	0.0013111975519184\\
277	0.00131040026878244\\
278	0.0013095883347095\\
279	0.00130876147209553\\
280	0.00130791939782557\\
281	0.00130706182315896\\
282	0.00130618845361234\\
283	0.00130529898884009\\
284	0.0013043931225125\\
285	0.00130347054219139\\
286	0.00130253092920325\\
287	0.00130157395850985\\
288	0.00130059929857632\\
289	0.00129960661123658\\
290	0.0012985955515563\\
291	0.00129756576769301\\
292	0.00129651690075369\\
293	0.0012954485846497\\
294	0.00129436044594889\\
295	0.00129325210372504\\
296	0.00129212316940462\\
297	0.00129097324661072\\
298	0.00128980193100424\\
299	0.00128860881012234\\
300	0.00128739346321414\\
301	0.0012861554610736\\
302	0.00128489436586965\\
303	0.00128360973097366\\
304	0.00128230110078401\\
305	0.00128096801054811\\
306	0.00127960998618158\\
307	0.00127822654408486\\
308	0.00127681719095712\\
309	0.00127538142360755\\
310	0.00127391872876414\\
311	0.00127242858287987\\
312	0.00127091045193644\\
313	0.00126936379124556\\
314	0.0012677880452479\\
315	0.00126618264730961\\
316	0.0012645470195167\\
317	0.00126288057246705\\
318	0.00126118270506038\\
319	0.00125945280428601\\
320	0.00125769024500862\\
321	0.00125589438975189\\
322	0.00125406458848017\\
323	0.00125220017837815\\
324	0.00125030048362847\\
325	0.00124836481518733\\
326	0.00124639247055803\\
327	0.0012443827335622\\
328	0.00124233487410888\\
329	0.00124024814796093\\
330	0.0012381217964989\\
331	0.00123595504648178\\
332	0.00123374710980439\\
333	0.00123149718325107\\
334	0.00122920444824498\\
335	0.00122686807059255\\
336	0.0012244872002224\\
337	0.00122206097091792\\
338	0.00121958850004269\\
339	0.00121706888825804\\
340	0.00121450121923165\\
341	0.00121188455933671\\
342	0.00120921795734093\\
343	0.00120650044408543\\
344	0.00120373103215371\\
345	0.00120090871553188\\
346	0.00119803246926022\\
347	0.00119510124907725\\
348	0.00119211399105691\\
349	0.00118906961124057\\
350	0.00118596700526513\\
351	0.00118280504798926\\
352	0.00117958259311903\\
353	0.00117629847283388\\
354	0.0011729514974118\\
355	0.00116954045485176\\
356	0.00116606411048735\\
357	0.00116252120659077\\
358	0.00115891046196604\\
359	0.00115523057153034\\
360	0.00115148020588229\\
361	0.00114765801085599\\
362	0.00114376260705909\\
363	0.00113979258939345\\
364	0.00113574652655665\\
365	0.00113162296052233\\
366	0.0011274204059974\\
367	0.00112313734985399\\
368	0.00111877225053367\\
369	0.00111432353742172\\
370	0.00110978961018889\\
371	0.0011051688380981\\
372	0.00110045955927375\\
373	0.00109566007993108\\
374	0.0010907686735636\\
375	0.00108578358008659\\
376	0.00108070300493549\\
377	0.00107552511811846\\
378	0.0010702480532234\\
379	0.00106486990638113\\
380	0.0010593887351879\\
381	0.00105380255759245\\
382	0.00104810935075489\\
383	0.00104230704988652\\
384	0.0010363935470779\\
385	0.00103036669011159\\
386	0.0010242242812134\\
387	0.00101796407556321\\
388	0.00101158377899479\\
389	0.0010050810431937\\
390	0.000998453453580481\\
391	0.000991698496614763\\
392	0.000984813471315704\\
393	0.000977795256818242\\
394	0.000970639737619394\\
395	0.000963340544719313\\
396	0.00095588706417462\\
397	0.000948265363656983\\
398	0.000940482377897597\\
399	0.000932549926432309\\
400	0.000924464723434965\\
401	0.000916223368217055\\
402	0.000907822366721184\\
403	0.000899258207615171\\
404	0.000890527579727133\\
405	0.000881627940607092\\
406	0.000872558913773815\\
407	0.000863325483724827\\
408	0.000853944449055149\\
409	0.000844453882714544\\
410	0.000834909614338404\\
411	0.000825286993102631\\
412	0.000815461872882012\\
413	0.000805430106370623\\
414	0.000795187588521162\\
415	0.000784730288121531\\
416	0.000774054285943402\\
417	0.000763155821140604\\
418	0.00075203134761258\\
419	0.000740677602422308\\
420	0.000729091688817819\\
421	0.000717271176972546\\
422	0.00070521422628937\\
423	0.000692919734044887\\
424	0.000680387516983011\\
425	0.00066761853716504\\
426	0.000654615200771179\\
427	0.000641381823586381\\
428	0.000627925606596906\\
429	0.000614259419503956\\
430	0.000600411310885943\\
431	0.000586459318011014\\
432	0.000572661427200527\\
433	0.000559942695223632\\
434	0.000550463121390471\\
435	0.000540793076180043\\
436	0.000530929798269977\\
437	0.000520870642980117\\
438	0.000510613104098667\\
439	0.00050015483806598\\
440	0.000489493690757346\\
441	0.000478627727235138\\
442	0.000467555265246462\\
443	0.00045627491450659\\
444	0.000444785627672718\\
445	0.000433086780486784\\
446	0.000421178332435776\\
447	0.000409061215383379\\
448	0.000396738357956686\\
449	0.00038421740466746\\
450	0.00037151756828031\\
451	0.00035868472472185\\
452	0.000345813460923171\\
453	0.000333015268270055\\
454	0.000319915237785431\\
455	0.000306408211838269\\
456	0.000292461935805177\\
457	0.000278039159059683\\
458	0.000263096498968872\\
459	0.00024758257884974\\
460	0.000231434215845809\\
461	0.000214566844856925\\
462	0.000196847069257896\\
463	0.000178008364435826\\
464	0.000157382959220837\\
465	0.000133019685236701\\
466	0.000106644759587974\\
467	7.9655848393468e-05\\
468	5.18777476358028e-05\\
469	2.2741055242951e-05\\
470	0\\
471	0\\
472	0\\
473	0\\
474	0\\
475	0\\
476	0\\
477	0\\
478	0\\
479	0\\
480	0\\
481	0\\
482	0\\
483	0\\
484	0\\
485	0\\
486	0\\
487	0\\
488	0\\
489	0\\
490	0\\
491	0\\
492	0\\
493	0\\
494	0\\
495	0\\
496	0\\
497	0\\
498	0\\
499	0\\
500	0\\
501	0\\
502	0\\
503	0\\
504	0\\
505	0\\
506	0\\
507	0\\
508	0\\
509	0\\
510	0\\
511	0\\
512	0\\
513	0\\
514	0\\
515	0\\
516	0\\
517	0\\
518	0\\
519	0\\
520	0\\
521	0\\
522	0\\
523	0\\
524	0\\
525	0\\
526	0\\
527	0\\
528	0\\
529	0\\
530	0\\
531	0\\
532	0\\
533	0\\
534	0\\
535	0\\
536	0\\
537	0\\
538	0\\
539	0\\
540	0\\
541	0\\
542	0\\
543	0\\
544	0\\
545	0\\
546	0\\
547	0\\
548	0\\
549	0\\
550	0\\
551	0\\
552	0\\
553	0\\
554	0\\
555	0\\
556	0\\
557	0\\
558	0\\
559	0\\
560	0\\
561	0\\
562	0\\
563	0\\
564	0\\
565	0\\
566	0\\
567	0\\
568	0\\
569	0\\
570	0\\
571	0\\
572	0\\
573	0\\
574	0\\
575	0\\
576	0\\
577	0\\
578	0\\
579	0\\
580	0\\
581	0\\
582	0\\
583	0\\
584	0\\
585	0\\
586	0\\
587	0\\
588	0\\
589	0\\
590	0\\
591	0\\
592	0\\
593	0\\
594	0\\
595	0\\
596	0\\
597	0\\
598	0\\
599	0\\
600	0\\
};
\addplot [color=mycolor19,solid,forget plot]
  table[row sep=crcr]{%
1	0.000420210774699427\\
2	0.00042020462908718\\
3	0.000420198373475441\\
4	0.000420192005897872\\
5	0.00042018552435305\\
6	0.000420178926803814\\
7	0.000420172211176661\\
8	0.000420165375361069\\
9	0.000420158417208842\\
10	0.000420151334533464\\
11	0.000420144125109368\\
12	0.000420136786671287\\
13	0.000420129316913536\\
14	0.000420121713489273\\
15	0.000420113974009767\\
16	0.000420106096043688\\
17	0.00042009807711629\\
18	0.000420089914708694\\
19	0.000420081606257064\\
20	0.00042007314915182\\
21	0.000420064540736825\\
22	0.000420055778308539\\
23	0.000420046859115208\\
24	0.000420037780355971\\
25	0.000420028539180003\\
26	0.000420019132685619\\
27	0.000420009557919356\\
28	0.000419999811875083\\
29	0.000419989891493038\\
30	0.000419979793658882\\
31	0.000419969515202706\\
32	0.00041995905289807\\
33	0.000419948403460986\\
34	0.000419937563548894\\
35	0.000419926529759596\\
36	0.000419915298630228\\
37	0.000419903866636155\\
38	0.000419892230189896\\
39	0.000419880385639962\\
40	0.00041986832926977\\
41	0.000419856057296442\\
42	0.000419843565869626\\
43	0.000419830851070321\\
44	0.000419817908909638\\
45	0.000419804735327548\\
46	0.000419791326191633\\
47	0.000419777677295765\\
48	0.000419763784358834\\
49	0.000419749643023389\\
50	0.000419735248854272\\
51	0.000419720597337248\\
52	0.000419705683877593\\
53	0.000419690503798645\\
54	0.000419675052340375\\
55	0.000419659324657884\\
56	0.000419643315819889\\
57	0.000419627020807169\\
58	0.00041961043451104\\
59	0.00041959355173171\\
60	0.000419576367176722\\
61	0.000419558875459217\\
62	0.00041954107109633\\
63	0.000419522948507448\\
64	0.000419504502012458\\
65	0.000419485725829991\\
66	0.000419466614075598\\
67	0.000419447160759941\\
68	0.0004194273597869\\
69	0.000419407204951699\\
70	0.000419386689938925\\
71	0.000419365808320598\\
72	0.000419344553554127\\
73	0.000419322918980322\\
74	0.000419300897821263\\
75	0.000419278483178203\\
76	0.000419255668029415\\
77	0.000419232445228003\\
78	0.000419208807499666\\
79	0.00041918474744043\\
80	0.000419160257514329\\
81	0.000419135330051058\\
82	0.000419109957243569\\
83	0.000419084131145653\\
84	0.000419057843669439\\
85	0.000419031086582873\\
86	0.000419003851507143\\
87	0.000418976129914079\\
88	0.000418947913123461\\
89	0.000418919192300326\\
90	0.000418889958452197\\
91	0.000418860202426298\\
92	0.000418829914906654\\
93	0.000418799086411201\\
94	0.000418767707288828\\
95	0.00041873576771634\\
96	0.000418703257695417\\
97	0.000418670167049447\\
98	0.00041863648542038\\
99	0.000418602202265472\\
100	0.000418567306853973\\
101	0.000418531788263796\\
102	0.0004184956353781\\
103	0.000418458836881776\\
104	0.000418421381257942\\
105	0.000418383256784331\\
106	0.000418344451529609\\
107	0.000418304953349629\\
108	0.000418264749883661\\
109	0.000418223828550504\\
110	0.000418182176544531\\
111	0.000418139780831689\\
112	0.000418096628145412\\
113	0.000418052704982474\\
114	0.000418007997598726\\
115	0.000417962492004816\\
116	0.000417916173961774\\
117	0.000417869028976577\\
118	0.000417821042297561\\
119	0.000417772198909828\\
120	0.0004177224835305\\
121	0.000417671880603947\\
122	0.000417620374296881\\
123	0.000417567948493395\\
124	0.000417514586789881\\
125	0.000417460272489879\\
126	0.000417404988598841\\
127	0.00041734871781878\\
128	0.000417291442542798\\
129	0.000417233144849582\\
130	0.000417173806497729\\
131	0.00041711340892004\\
132	0.00041705193321764\\
133	0.00041698936015401\\
134	0.000416925670148956\\
135	0.000416860843272422\\
136	0.000416794859238187\\
137	0.000416727697397464\\
138	0.000416659336732411\\
139	0.000416589755849435\\
140	0.000416518932972488\\
141	0.000416446845936148\\
142	0.000416373472178619\\
143	0.000416298788734591\\
144	0.000416222772227961\\
145	0.00041614539886445\\
146	0.000416066644424035\\
147	0.000415986484253294\\
148	0.000415904893257578\\
149	0.000415821845893031\\
150	0.000415737316158512\\
151	0.000415651277587309\\
152	0.000415563703238727\\
153	0.000415474565689533\\
154	0.00041538383702522\\
155	0.000415291488831118\\
156	0.000415197492183353\\
157	0.000415101817639604\\
158	0.000415004435229723\\
159	0.000414905314446171\\
160	0.000414804424234272\\
161	0.000414701732982281\\
162	0.000414597208511289\\
163	0.000414490818064882\\
164	0.0004143825282987\\
165	0.000414272305269728\\
166	0.00041416011442539\\
167	0.000414045920592484\\
168	0.000413929687965865\\
169	0.000413811380096935\\
170	0.000413690959881913\\
171	0.000413568389549891\\
172	0.00041344363065066\\
173	0.000413316644042281\\
174	0.000413187389878481\\
175	0.000413055827595743\\
176	0.00041292191590022\\
177	0.000412785612754324\\
178	0.00041264687536313\\
179	0.000412505660160458\\
180	0.000412361922794757\\
181	0.000412215618114693\\
182	0.000412066700154405\\
183	0.000411915122118589\\
184	0.000411760836367182\\
185	0.000411603794399866\\
186	0.000411443946840139\\
187	0.000411281243419234\\
188	0.000411115632959598\\
189	0.000410947063358121\\
190	0.000410775481569032\\
191	0.000410600833586431\\
192	0.000410423064426531\\
193	0.000410242118109514\\
194	0.000410057937641047\\
195	0.000409870464993467\\
196	0.000409679641086545\\
197	0.000409485405767922\\
198	0.000409287697793106\\
199	0.000409086454805152\\
200	0.000408881613313874\\
201	0.000408673108674678\\
202	0.000408460875066989\\
203	0.00040824484547221\\
204	0.000408024951651287\\
205	0.000407801124121793\\
206	0.000407573292134596\\
207	0.000407341383650007\\
208	0.000407105325313521\\
209	0.000406865042430991\\
210	0.00040662045894337\\
211	0.000406371497400927\\
212	0.000406118078936918\\
213	0.000405860123240738\\
214	0.000405597548530553\\
215	0.000405330271525351\\
216	0.000405058207416416\\
217	0.000404781269838246\\
218	0.000404499370838857\\
219	0.000404212420849496\\
220	0.000403920328653705\\
221	0.000403623001355766\\
222	0.000403320344348494\\
223	0.000403012261280353\\
224	0.000402698654021902\\
225	0.000402379422631531\\
226	0.000402054465320472\\
227	0.000401723678417106\\
228	0.000401386956330495\\
229	0.000401044191513181\\
230	0.00040069527442313\\
231	0.000400340093484939\\
232	0.000399978535050196\\
233	0.000399610483356988\\
234	0.000399235820488535\\
235	0.00039885442633097\\
236	0.000398466178530193\\
237	0.000398070952447773\\
238	0.00039766862111594\\
239	0.000397259055191547\\
240	0.000396842122909098\\
241	0.000396417690032664\\
242	0.000395985619806851\\
243	0.00039554577290657\\
244	0.000395098007385799\\
245	0.00039464217862516\\
246	0.000394178139278344\\
247	0.000393705739217331\\
248	0.000393224825476414\\
249	0.000392735242194934\\
250	0.000392236830558735\\
251	0.00039172942874031\\
252	0.000391212871837579\\
253	0.000390686991811258\\
254	0.000390151617420819\\
255	0.000389606574158995\\
256	0.000389051684184712\\
257	0.000388486766254549\\
258	0.000387911635652571\\
259	0.000387326104118527\\
260	0.000386729979774372\\
261	0.000386123067049086\\
262	0.000385505166601701\\
263	0.000384876075242515\\
264	0.000384235585852435\\
265	0.000383583487300444\\
266	0.000382919564359009\\
267	0.000382243597617537\\
268	0.000381555363393725\\
269	0.000380854633642751\\
270	0.000380141175864319\\
271	0.000379414753007368\\
272	0.000378675123372514\\
273	0.000377922040512071\\
274	0.000377155253127655\\
275	0.000376374504965213\\
276	0.000375579534707496\\
277	0.000374770075863866\\
278	0.000373945856657327\\
279	0.000373106599908788\\
280	0.000372252022918408\\
281	0.000371381837343941\\
282	0.000370495749076078\\
283	0.000369593458110635\\
284	0.000368674658417494\\
285	0.000367739037806256\\
286	0.000366786277788477\\
287	0.000365816053436458\\
288	0.000364828033238396\\
289	0.000363821878949912\\
290	0.000362797245441791\\
291	0.000361753780543842\\
292	0.000360691124884887\\
293	0.000359608911728576\\
294	0.000358506766805149\\
295	0.000357384308138916\\
296	0.000356241145871404\\
297	0.000355076882080079\\
298	0.000353891110592532\\
299	0.000352683416796083\\
300	0.000351453377442623\\
301	0.000350200560448717\\
302	0.00034892452469083\\
303	0.000347624819795581\\
304	0.000346300985925042\\
305	0.000344952553556883\\
306	0.000343579043259457\\
307	0.00034217996546167\\
308	0.000340754820217642\\
309	0.000339303096966149\\
310	0.000337824274284814\\
311	0.000336317819639054\\
312	0.000334783189125854\\
313	0.00033321982721231\\
314	0.000331627166469109\\
315	0.000330004627299008\\
316	0.0003283516176604\\
317	0.000326667532786106\\
318	0.000324951754897629\\
319	0.000323203652915015\\
320	0.000321422582162619\\
321	0.000319607884071005\\
322	0.000317758885875354\\
323	0.000315874900310746\\
324	0.000313955225304691\\
325	0.00031199914366742\\
326	0.000310005922780365\\
327	0.000307974814283448\\
328	0.000305905053761702\\
329	0.000303795860431847\\
330	0.000301646436829404\\
331	0.000299455968497052\\
332	0.000297223623674725\\
333	0.000294948552992024\\
334	0.000292629889163423\\
335	0.000290266746686432\\
336	0.000287858221542931\\
337	0.000285403390903285\\
338	0.000282901312832588\\
339	0.00028035102599768\\
340	0.000277751549372744\\
341	0.000275101881939878\\
342	0.000272401002379806\\
343	0.000269647868747095\\
344	0.000266841418124055\\
345	0.000263980566248142\\
346	0.000261064207119875\\
347	0.000258091212589031\\
348	0.000255060431917373\\
349	0.000251970691317588\\
350	0.000248820793470033\\
351	0.000245609517022436\\
352	0.000242335616083149\\
353	0.000238997819725432\\
354	0.000235594831523735\\
355	0.000232125329139045\\
356	0.000228587963984056\\
357	0.00022498136090938\\
358	0.000221304117908507\\
359	0.00021755480584361\\
360	0.000213731968194677\\
361	0.000209834120834403\\
362	0.0002058597518315\\
363	0.000201807321285005\\
364	0.000197675261192476\\
365	0.000193461975354694\\
366	0.000189165839319934\\
367	0.000184785200370424\\
368	0.000180318377553957\\
369	0.000175763661763264\\
370	0.000171119315865822\\
371	0.00016638357488653\\
372	0.00016155464624531\\
373	0.000156630710051668\\
374	0.000151609919457554\\
375	0.000146490401069541\\
376	0.000141270255420753\\
377	0.000135947557502195\\
378	0.000130520357352411\\
379	0.000124986680703435\\
380	0.000119344529679791\\
381	0.000113591883545952\\
382	0.000107726699495406\\
383	0.000101746913470269\\
384	9.56504409906719e-05\\
385	8.94351779450992e-05\\
386	8.30990012107413e-05\\
387	7.66397687185278e-05\\
388	7.00553177826701e-05\\
389	6.33434579958076e-05\\
390	5.65019469337396e-05\\
391	4.95284108677339e-05\\
392	4.24200875911155e-05\\
393	3.51729873734826e-05\\
394	2.77791287649591e-05\\
395	2.02173300938862e-05\\
396	1.24221703714644e-05\\
397	4.17810039301993e-06\\
398	0\\
399	0\\
400	0\\
401	0\\
402	0\\
403	0\\
404	0\\
405	0\\
406	0\\
407	0\\
408	0\\
409	0\\
410	0\\
411	0\\
412	0\\
413	0\\
414	0\\
415	0\\
416	0\\
417	0\\
418	0\\
419	0\\
420	0\\
421	0\\
422	0\\
423	0\\
424	0\\
425	0\\
426	0\\
427	0\\
428	0\\
429	0\\
430	0\\
431	0\\
432	0\\
433	0\\
434	0\\
435	0\\
436	0\\
437	0\\
438	0\\
439	0\\
440	0\\
441	0\\
442	0\\
443	0\\
444	0\\
445	0\\
446	0\\
447	0\\
448	0\\
449	0\\
450	0\\
451	0\\
452	0\\
453	0\\
454	0\\
455	0\\
456	0\\
457	0\\
458	0\\
459	0\\
460	0\\
461	0\\
462	0\\
463	0\\
464	0\\
465	0\\
466	0\\
467	0\\
468	0\\
469	0\\
470	0\\
471	0\\
472	0\\
473	0\\
474	0\\
475	0\\
476	0\\
477	0\\
478	0\\
479	0\\
480	0\\
481	0\\
482	0\\
483	0\\
484	0\\
485	0\\
486	0\\
487	0\\
488	0\\
489	0\\
490	0\\
491	0\\
492	0\\
493	0\\
494	0\\
495	0\\
496	0\\
497	0\\
498	0\\
499	0\\
500	0\\
501	0\\
502	0\\
503	0\\
504	0\\
505	0\\
506	0\\
507	0\\
508	0\\
509	0\\
510	0\\
511	0\\
512	0\\
513	0\\
514	0\\
515	0\\
516	0\\
517	0\\
518	0\\
519	0\\
520	0\\
521	0\\
522	0\\
523	0\\
524	0\\
525	0\\
526	0\\
527	0\\
528	0\\
529	0\\
530	0\\
531	0\\
532	0\\
533	0\\
534	0\\
535	0\\
536	0\\
537	0\\
538	0\\
539	0\\
540	0\\
541	0\\
542	0\\
543	0\\
544	0\\
545	0\\
546	0\\
547	0\\
548	0\\
549	0\\
550	0\\
551	0\\
552	0\\
553	0\\
554	0\\
555	0\\
556	0\\
557	0\\
558	0\\
559	0\\
560	0\\
561	0\\
562	0\\
563	0\\
564	0\\
565	0\\
566	0\\
567	0\\
568	0\\
569	0\\
570	0\\
571	0\\
572	0\\
573	0\\
574	0\\
575	0\\
576	0\\
577	0\\
578	0\\
579	0\\
580	0\\
581	0\\
582	0\\
583	0\\
584	0\\
585	0\\
586	0\\
587	0\\
588	0\\
589	0\\
590	0\\
591	0\\
592	0\\
593	0\\
594	0\\
595	0\\
596	0\\
597	0\\
598	0\\
599	0\\
600	0\\
};
\addplot [color=red!50!mycolor17,solid,forget plot]
  table[row sep=crcr]{%
1	0\\
2	0\\
3	0\\
4	0\\
5	0\\
6	0\\
7	0\\
8	0\\
9	0\\
10	0\\
11	0\\
12	0\\
13	0\\
14	0\\
15	0\\
16	0\\
17	0\\
18	0\\
19	0\\
20	0\\
21	0\\
22	0\\
23	0\\
24	0\\
25	0\\
26	0\\
27	0\\
28	0\\
29	0\\
30	0\\
31	0\\
32	0\\
33	0\\
34	0\\
35	0\\
36	0\\
37	0\\
38	0\\
39	0\\
40	0\\
41	0\\
42	0\\
43	0\\
44	0\\
45	0\\
46	0\\
47	0\\
48	0\\
49	0\\
50	0\\
51	0\\
52	0\\
53	0\\
54	0\\
55	0\\
56	0\\
57	0\\
58	0\\
59	0\\
60	0\\
61	0\\
62	0\\
63	0\\
64	0\\
65	0\\
66	0\\
67	0\\
68	0\\
69	0\\
70	0\\
71	0\\
72	0\\
73	0\\
74	0\\
75	0\\
76	0\\
77	0\\
78	0\\
79	0\\
80	0\\
81	0\\
82	0\\
83	0\\
84	0\\
85	0\\
86	0\\
87	0\\
88	0\\
89	0\\
90	0\\
91	0\\
92	0\\
93	0\\
94	0\\
95	0\\
96	0\\
97	0\\
98	0\\
99	0\\
100	0\\
101	0\\
102	0\\
103	0\\
104	0\\
105	0\\
106	0\\
107	0\\
108	0\\
109	0\\
110	0\\
111	0\\
112	0\\
113	0\\
114	0\\
115	0\\
116	0\\
117	0\\
118	0\\
119	0\\
120	0\\
121	0\\
122	0\\
123	0\\
124	0\\
125	0\\
126	0\\
127	0\\
128	0\\
129	0\\
130	0\\
131	0\\
132	0\\
133	0\\
134	0\\
135	0\\
136	0\\
137	0\\
138	0\\
139	0\\
140	0\\
141	0\\
142	0\\
143	0\\
144	0\\
145	0\\
146	0\\
147	0\\
148	0\\
149	0\\
150	0\\
151	0\\
152	0\\
153	0\\
154	0\\
155	0\\
156	0\\
157	0\\
158	0\\
159	0\\
160	0\\
161	0\\
162	0\\
163	0\\
164	0\\
165	0\\
166	0\\
167	0\\
168	0\\
169	0\\
170	0\\
171	0\\
172	0\\
173	0\\
174	0\\
175	0\\
176	0\\
177	0\\
178	0\\
179	0\\
180	0\\
181	0\\
182	0\\
183	0\\
184	0\\
185	0\\
186	0\\
187	0\\
188	0\\
189	0\\
190	0\\
191	0\\
192	0\\
193	0\\
194	0\\
195	0\\
196	0\\
197	0\\
198	0\\
199	0\\
200	0\\
201	0\\
202	0\\
203	0\\
204	0\\
205	0\\
206	0\\
207	0\\
208	0\\
209	0\\
210	0\\
211	0\\
212	0\\
213	0\\
214	0\\
215	0\\
216	0\\
217	0\\
218	0\\
219	0\\
220	0\\
221	0\\
222	0\\
223	0\\
224	0\\
225	0\\
226	0\\
227	0\\
228	0\\
229	0\\
230	0\\
231	0\\
232	0\\
233	0\\
234	0\\
235	0\\
236	0\\
237	0\\
238	0\\
239	0\\
240	0\\
241	0\\
242	0\\
243	0\\
244	0\\
245	0\\
246	0\\
247	0\\
248	0\\
249	0\\
250	0\\
251	0\\
252	0\\
253	0\\
254	0\\
255	0\\
256	0\\
257	0\\
258	0\\
259	0\\
260	0\\
261	0\\
262	0\\
263	0\\
264	0\\
265	0\\
266	0\\
267	0\\
268	0\\
269	0\\
270	0\\
271	0\\
272	0\\
273	0\\
274	0\\
275	0\\
276	0\\
277	0\\
278	0\\
279	0\\
280	0\\
281	0\\
282	0\\
283	0\\
284	0\\
285	0\\
286	0\\
287	0\\
288	0\\
289	0\\
290	0\\
291	0\\
292	0\\
293	0\\
294	0\\
295	0\\
296	0\\
297	0\\
298	0\\
299	0\\
300	0\\
301	0\\
302	0\\
303	0\\
304	0\\
305	0\\
306	0\\
307	0\\
308	0\\
309	0\\
310	0\\
311	0\\
312	0\\
313	0\\
314	0\\
315	0\\
316	0\\
317	0\\
318	0\\
319	0\\
320	0\\
321	0\\
322	0\\
323	0\\
324	0\\
325	0\\
326	0\\
327	0\\
328	0\\
329	0\\
330	0\\
331	0\\
332	0\\
333	0\\
334	0\\
335	0\\
336	0\\
337	0\\
338	0\\
339	0\\
340	0\\
341	0\\
342	0\\
343	0\\
344	0\\
345	0\\
346	0\\
347	0\\
348	0\\
349	0\\
350	0\\
351	0\\
352	0\\
353	0\\
354	0\\
355	0\\
356	0\\
357	0\\
358	0\\
359	0\\
360	0\\
361	0\\
362	0\\
363	0\\
364	0\\
365	0\\
366	0\\
367	0\\
368	0\\
369	0\\
370	0\\
371	0\\
372	0\\
373	0\\
374	0\\
375	0\\
376	0\\
377	0\\
378	0\\
379	0\\
380	0\\
381	0\\
382	0\\
383	0\\
384	0\\
385	0\\
386	0\\
387	0\\
388	0\\
389	0\\
390	0\\
391	0\\
392	0\\
393	0\\
394	0\\
395	0\\
396	0\\
397	0\\
398	0\\
399	0\\
400	0\\
401	0\\
402	0\\
403	0\\
404	0\\
405	0\\
406	0\\
407	0\\
408	0\\
409	0\\
410	0\\
411	0\\
412	0\\
413	0\\
414	0\\
415	0\\
416	0\\
417	0\\
418	0\\
419	0\\
420	0\\
421	0\\
422	0\\
423	0\\
424	0\\
425	0\\
426	0\\
427	0\\
428	0\\
429	0\\
430	0\\
431	0\\
432	0\\
433	0\\
434	0\\
435	0\\
436	0\\
437	0\\
438	0\\
439	0\\
440	0\\
441	0\\
442	0\\
443	0\\
444	0\\
445	0\\
446	0\\
447	0\\
448	0\\
449	0\\
450	0\\
451	0\\
452	0\\
453	0\\
454	0\\
455	0\\
456	0\\
457	0\\
458	0\\
459	0\\
460	0\\
461	0\\
462	0\\
463	0\\
464	0\\
465	0\\
466	0\\
467	0\\
468	0\\
469	0\\
470	0\\
471	0\\
472	0\\
473	0\\
474	0\\
475	0\\
476	0\\
477	0\\
478	0\\
479	0\\
480	0\\
481	0\\
482	0\\
483	0\\
484	0\\
485	0\\
486	0\\
487	0\\
488	0\\
489	0\\
490	0\\
491	0\\
492	0\\
493	0\\
494	0\\
495	0\\
496	0\\
497	0\\
498	0\\
499	0\\
500	0\\
501	0\\
502	0\\
503	0\\
504	0\\
505	0\\
506	0\\
507	0\\
508	0\\
509	0\\
510	0\\
511	0\\
512	0\\
513	0\\
514	0\\
515	0\\
516	0\\
517	0\\
518	0\\
519	0\\
520	0\\
521	0\\
522	0\\
523	0\\
524	0\\
525	0\\
526	0\\
527	0\\
528	0\\
529	0\\
530	0\\
531	0\\
532	0\\
533	0\\
534	0\\
535	0\\
536	0\\
537	0\\
538	0\\
539	0\\
540	0\\
541	0\\
542	0\\
543	0\\
544	0\\
545	0\\
546	0\\
547	0\\
548	0\\
549	0\\
550	0\\
551	0\\
552	0\\
553	0\\
554	0\\
555	0\\
556	0\\
557	0\\
558	0\\
559	0\\
560	0\\
561	0\\
562	0\\
563	0\\
564	0\\
565	0\\
566	0\\
567	0\\
568	0\\
569	0\\
570	0\\
571	0\\
572	0\\
573	0\\
574	0\\
575	0\\
576	0\\
577	0\\
578	0\\
579	0\\
580	0\\
581	0\\
582	0\\
583	0\\
584	0\\
585	0\\
586	0\\
587	0\\
588	0\\
589	0\\
590	0\\
591	0\\
592	0\\
593	0\\
594	0\\
595	0\\
596	0\\
597	0\\
598	0\\
599	0\\
600	0\\
};
\addplot [color=red!40!mycolor19,solid,forget plot]
  table[row sep=crcr]{%
1	0\\
2	0\\
3	0\\
4	0\\
5	0\\
6	0\\
7	0\\
8	0\\
9	0\\
10	0\\
11	0\\
12	0\\
13	0\\
14	0\\
15	0\\
16	0\\
17	0\\
18	0\\
19	0\\
20	0\\
21	0\\
22	0\\
23	0\\
24	0\\
25	0\\
26	0\\
27	0\\
28	0\\
29	0\\
30	0\\
31	0\\
32	0\\
33	0\\
34	0\\
35	0\\
36	0\\
37	0\\
38	0\\
39	0\\
40	0\\
41	0\\
42	0\\
43	0\\
44	0\\
45	0\\
46	0\\
47	0\\
48	0\\
49	0\\
50	0\\
51	0\\
52	0\\
53	0\\
54	0\\
55	0\\
56	0\\
57	0\\
58	0\\
59	0\\
60	0\\
61	0\\
62	0\\
63	0\\
64	0\\
65	0\\
66	0\\
67	0\\
68	0\\
69	0\\
70	0\\
71	0\\
72	0\\
73	0\\
74	0\\
75	0\\
76	0\\
77	0\\
78	0\\
79	0\\
80	0\\
81	0\\
82	0\\
83	0\\
84	0\\
85	0\\
86	0\\
87	0\\
88	0\\
89	0\\
90	0\\
91	0\\
92	0\\
93	0\\
94	0\\
95	0\\
96	0\\
97	0\\
98	0\\
99	0\\
100	0\\
101	0\\
102	0\\
103	0\\
104	0\\
105	0\\
106	0\\
107	0\\
108	0\\
109	0\\
110	0\\
111	0\\
112	0\\
113	0\\
114	0\\
115	0\\
116	0\\
117	0\\
118	0\\
119	0\\
120	0\\
121	0\\
122	0\\
123	0\\
124	0\\
125	0\\
126	0\\
127	0\\
128	0\\
129	0\\
130	0\\
131	0\\
132	0\\
133	0\\
134	0\\
135	0\\
136	0\\
137	0\\
138	0\\
139	0\\
140	0\\
141	0\\
142	0\\
143	0\\
144	0\\
145	0\\
146	0\\
147	0\\
148	0\\
149	0\\
150	0\\
151	0\\
152	0\\
153	0\\
154	0\\
155	0\\
156	0\\
157	0\\
158	0\\
159	0\\
160	0\\
161	0\\
162	0\\
163	0\\
164	0\\
165	0\\
166	0\\
167	0\\
168	0\\
169	0\\
170	0\\
171	0\\
172	0\\
173	0\\
174	0\\
175	0\\
176	0\\
177	0\\
178	0\\
179	0\\
180	0\\
181	0\\
182	0\\
183	0\\
184	0\\
185	0\\
186	0\\
187	0\\
188	0\\
189	0\\
190	0\\
191	0\\
192	0\\
193	0\\
194	0\\
195	0\\
196	0\\
197	0\\
198	0\\
199	0\\
200	0\\
201	0\\
202	0\\
203	0\\
204	0\\
205	0\\
206	0\\
207	0\\
208	0\\
209	0\\
210	0\\
211	0\\
212	0\\
213	0\\
214	0\\
215	0\\
216	0\\
217	0\\
218	0\\
219	0\\
220	0\\
221	0\\
222	0\\
223	0\\
224	0\\
225	0\\
226	0\\
227	0\\
228	0\\
229	0\\
230	0\\
231	0\\
232	0\\
233	0\\
234	0\\
235	0\\
236	0\\
237	0\\
238	0\\
239	0\\
240	0\\
241	0\\
242	0\\
243	0\\
244	0\\
245	0\\
246	0\\
247	0\\
248	0\\
249	0\\
250	0\\
251	0\\
252	0\\
253	0\\
254	0\\
255	0\\
256	0\\
257	0\\
258	0\\
259	0\\
260	0\\
261	0\\
262	0\\
263	0\\
264	0\\
265	0\\
266	0\\
267	0\\
268	0\\
269	0\\
270	0\\
271	0\\
272	0\\
273	0\\
274	0\\
275	0\\
276	0\\
277	0\\
278	0\\
279	0\\
280	0\\
281	0\\
282	0\\
283	0\\
284	0\\
285	0\\
286	0\\
287	0\\
288	0\\
289	0\\
290	0\\
291	0\\
292	0\\
293	0\\
294	0\\
295	0\\
296	0\\
297	0\\
298	0\\
299	0\\
300	0\\
301	0\\
302	0\\
303	0\\
304	0\\
305	0\\
306	0\\
307	0\\
308	0\\
309	0\\
310	0\\
311	0\\
312	0\\
313	0\\
314	0\\
315	0\\
316	0\\
317	0\\
318	0\\
319	0\\
320	0\\
321	0\\
322	0\\
323	0\\
324	0\\
325	0\\
326	0\\
327	0\\
328	0\\
329	0\\
330	0\\
331	0\\
332	0\\
333	0\\
334	0\\
335	0\\
336	0\\
337	0\\
338	0\\
339	0\\
340	0\\
341	0\\
342	0\\
343	0\\
344	0\\
345	0\\
346	0\\
347	0\\
348	0\\
349	0\\
350	0\\
351	0\\
352	0\\
353	0\\
354	0\\
355	0\\
356	0\\
357	0\\
358	0\\
359	0\\
360	0\\
361	0\\
362	0\\
363	0\\
364	0\\
365	0\\
366	0\\
367	0\\
368	0\\
369	0\\
370	0\\
371	0\\
372	0\\
373	0\\
374	0\\
375	0\\
376	0\\
377	0\\
378	0\\
379	0\\
380	0\\
381	0\\
382	0\\
383	0\\
384	0\\
385	0\\
386	0\\
387	0\\
388	0\\
389	0\\
390	0\\
391	0\\
392	0\\
393	0\\
394	0\\
395	0\\
396	0\\
397	0\\
398	0\\
399	0\\
400	0\\
401	0\\
402	0\\
403	0\\
404	0\\
405	0\\
406	0\\
407	0\\
408	0\\
409	0\\
410	0\\
411	0\\
412	0\\
413	0\\
414	0\\
415	0\\
416	0\\
417	0\\
418	0\\
419	0\\
420	0\\
421	0\\
422	0\\
423	0\\
424	0\\
425	0\\
426	0\\
427	0\\
428	0\\
429	0\\
430	0\\
431	0\\
432	0\\
433	0\\
434	0\\
435	0\\
436	0\\
437	0\\
438	0\\
439	0\\
440	0\\
441	0\\
442	0\\
443	0\\
444	0\\
445	0\\
446	0\\
447	0\\
448	0\\
449	0\\
450	0\\
451	0\\
452	0\\
453	0\\
454	0\\
455	0\\
456	0\\
457	0\\
458	0\\
459	0\\
460	0\\
461	0\\
462	0\\
463	0\\
464	0\\
465	0\\
466	0\\
467	0\\
468	0\\
469	0\\
470	0\\
471	0\\
472	0\\
473	0\\
474	0\\
475	0\\
476	0\\
477	0\\
478	0\\
479	0\\
480	0\\
481	0\\
482	0\\
483	0\\
484	0\\
485	0\\
486	0\\
487	0\\
488	0\\
489	0\\
490	0\\
491	0\\
492	0\\
493	0\\
494	0\\
495	0\\
496	0\\
497	0\\
498	0\\
499	0\\
500	0\\
501	0\\
502	0\\
503	0\\
504	0\\
505	0\\
506	0\\
507	0\\
508	0\\
509	0\\
510	0\\
511	0\\
512	0\\
513	0\\
514	0\\
515	0\\
516	0\\
517	0\\
518	0\\
519	0\\
520	0\\
521	0\\
522	0\\
523	0\\
524	0\\
525	0\\
526	0\\
527	0\\
528	0\\
529	0\\
530	0\\
531	0\\
532	0\\
533	0\\
534	0\\
535	0\\
536	0\\
537	0\\
538	0\\
539	0\\
540	0\\
541	0\\
542	0\\
543	0\\
544	0\\
545	0\\
546	0\\
547	0\\
548	0\\
549	0\\
550	0\\
551	0\\
552	0\\
553	0\\
554	0\\
555	0\\
556	0\\
557	0\\
558	0\\
559	0\\
560	0\\
561	0\\
562	0\\
563	0\\
564	0\\
565	0\\
566	0\\
567	0\\
568	0\\
569	0\\
570	0\\
571	0\\
572	0\\
573	0\\
574	0\\
575	0\\
576	0\\
577	0\\
578	0\\
579	0\\
580	0\\
581	0\\
582	0\\
583	0\\
584	0\\
585	0\\
586	0\\
587	0\\
588	0\\
589	0\\
590	0\\
591	0\\
592	0\\
593	0\\
594	0\\
595	0\\
596	0\\
597	0\\
598	0\\
599	0\\
600	0\\
};
\addplot [color=red!75!mycolor17,solid,forget plot]
  table[row sep=crcr]{%
1	0\\
2	0\\
3	0\\
4	0\\
5	0\\
6	0\\
7	0\\
8	0\\
9	0\\
10	0\\
11	0\\
12	0\\
13	0\\
14	0\\
15	0\\
16	0\\
17	0\\
18	0\\
19	0\\
20	0\\
21	0\\
22	0\\
23	0\\
24	0\\
25	0\\
26	0\\
27	0\\
28	0\\
29	0\\
30	0\\
31	0\\
32	0\\
33	0\\
34	0\\
35	0\\
36	0\\
37	0\\
38	0\\
39	0\\
40	0\\
41	0\\
42	0\\
43	0\\
44	0\\
45	0\\
46	0\\
47	0\\
48	0\\
49	0\\
50	0\\
51	0\\
52	0\\
53	0\\
54	0\\
55	0\\
56	0\\
57	0\\
58	0\\
59	0\\
60	0\\
61	0\\
62	0\\
63	0\\
64	0\\
65	0\\
66	0\\
67	0\\
68	0\\
69	0\\
70	0\\
71	0\\
72	0\\
73	0\\
74	0\\
75	0\\
76	0\\
77	0\\
78	0\\
79	0\\
80	0\\
81	0\\
82	0\\
83	0\\
84	0\\
85	0\\
86	0\\
87	0\\
88	0\\
89	0\\
90	0\\
91	0\\
92	0\\
93	0\\
94	0\\
95	0\\
96	0\\
97	0\\
98	0\\
99	0\\
100	0\\
101	0\\
102	0\\
103	0\\
104	0\\
105	0\\
106	0\\
107	0\\
108	0\\
109	0\\
110	0\\
111	0\\
112	0\\
113	0\\
114	0\\
115	0\\
116	0\\
117	0\\
118	0\\
119	0\\
120	0\\
121	0\\
122	0\\
123	0\\
124	0\\
125	0\\
126	0\\
127	0\\
128	0\\
129	0\\
130	0\\
131	0\\
132	0\\
133	0\\
134	0\\
135	0\\
136	0\\
137	0\\
138	0\\
139	0\\
140	0\\
141	0\\
142	0\\
143	0\\
144	0\\
145	0\\
146	0\\
147	0\\
148	0\\
149	0\\
150	0\\
151	0\\
152	0\\
153	0\\
154	0\\
155	0\\
156	0\\
157	0\\
158	0\\
159	0\\
160	0\\
161	0\\
162	0\\
163	0\\
164	0\\
165	0\\
166	0\\
167	0\\
168	0\\
169	0\\
170	0\\
171	0\\
172	0\\
173	0\\
174	0\\
175	0\\
176	0\\
177	0\\
178	0\\
179	0\\
180	0\\
181	0\\
182	0\\
183	0\\
184	0\\
185	0\\
186	0\\
187	0\\
188	0\\
189	0\\
190	0\\
191	0\\
192	0\\
193	0\\
194	0\\
195	0\\
196	0\\
197	0\\
198	0\\
199	0\\
200	0\\
201	0\\
202	0\\
203	0\\
204	0\\
205	0\\
206	0\\
207	0\\
208	0\\
209	0\\
210	0\\
211	0\\
212	0\\
213	0\\
214	0\\
215	0\\
216	0\\
217	0\\
218	0\\
219	0\\
220	0\\
221	0\\
222	0\\
223	0\\
224	0\\
225	0\\
226	0\\
227	0\\
228	0\\
229	0\\
230	0\\
231	0\\
232	0\\
233	0\\
234	0\\
235	0\\
236	0\\
237	0\\
238	0\\
239	0\\
240	0\\
241	0\\
242	0\\
243	0\\
244	0\\
245	0\\
246	0\\
247	0\\
248	0\\
249	0\\
250	0\\
251	0\\
252	0\\
253	0\\
254	0\\
255	0\\
256	0\\
257	0\\
258	0\\
259	0\\
260	0\\
261	0\\
262	0\\
263	0\\
264	0\\
265	0\\
266	0\\
267	0\\
268	0\\
269	0\\
270	0\\
271	0\\
272	0\\
273	0\\
274	0\\
275	0\\
276	0\\
277	0\\
278	0\\
279	0\\
280	0\\
281	0\\
282	0\\
283	0\\
284	0\\
285	0\\
286	0\\
287	0\\
288	0\\
289	0\\
290	0\\
291	0\\
292	0\\
293	0\\
294	0\\
295	0\\
296	0\\
297	0\\
298	0\\
299	0\\
300	0\\
301	0\\
302	0\\
303	0\\
304	0\\
305	0\\
306	0\\
307	0\\
308	0\\
309	0\\
310	0\\
311	0\\
312	0\\
313	0\\
314	0\\
315	0\\
316	0\\
317	0\\
318	0\\
319	0\\
320	0\\
321	0\\
322	0\\
323	0\\
324	0\\
325	0\\
326	0\\
327	0\\
328	0\\
329	0\\
330	0\\
331	0\\
332	0\\
333	0\\
334	0\\
335	0\\
336	0\\
337	0\\
338	0\\
339	0\\
340	0\\
341	0\\
342	0\\
343	0\\
344	0\\
345	0\\
346	0\\
347	0\\
348	0\\
349	0\\
350	0\\
351	0\\
352	0\\
353	0\\
354	0\\
355	0\\
356	0\\
357	0\\
358	0\\
359	0\\
360	0\\
361	0\\
362	0\\
363	0\\
364	0\\
365	0\\
366	0\\
367	0\\
368	0\\
369	0\\
370	0\\
371	0\\
372	0\\
373	0\\
374	0\\
375	0\\
376	0\\
377	0\\
378	0\\
379	0\\
380	0\\
381	0\\
382	0\\
383	0\\
384	0\\
385	0\\
386	0\\
387	0\\
388	0\\
389	0\\
390	0\\
391	0\\
392	0\\
393	0\\
394	0\\
395	0\\
396	0\\
397	0\\
398	0\\
399	0\\
400	0\\
401	0\\
402	0\\
403	0\\
404	0\\
405	0\\
406	0\\
407	0\\
408	0\\
409	0\\
410	0\\
411	0\\
412	0\\
413	0\\
414	0\\
415	0\\
416	0\\
417	0\\
418	0\\
419	0\\
420	0\\
421	0\\
422	0\\
423	0\\
424	0\\
425	0\\
426	0\\
427	0\\
428	0\\
429	0\\
430	0\\
431	0\\
432	0\\
433	0\\
434	0\\
435	0\\
436	0\\
437	0\\
438	0\\
439	0\\
440	0\\
441	0\\
442	0\\
443	0\\
444	0\\
445	0\\
446	0\\
447	0\\
448	0\\
449	0\\
450	0\\
451	0\\
452	0\\
453	0\\
454	0\\
455	0\\
456	0\\
457	0\\
458	0\\
459	0\\
460	0\\
461	0\\
462	0\\
463	0\\
464	0\\
465	0\\
466	0\\
467	0\\
468	0\\
469	0\\
470	0\\
471	0\\
472	0\\
473	0\\
474	0\\
475	0\\
476	0\\
477	0\\
478	0\\
479	0\\
480	0\\
481	0\\
482	0\\
483	0\\
484	0\\
485	0\\
486	0\\
487	0\\
488	0\\
489	0\\
490	0\\
491	0\\
492	0\\
493	0\\
494	0\\
495	0\\
496	0\\
497	0\\
498	0\\
499	0\\
500	0\\
501	0\\
502	0\\
503	0\\
504	0\\
505	0\\
506	0\\
507	0\\
508	0\\
509	0\\
510	0\\
511	0\\
512	0\\
513	0\\
514	0\\
515	0\\
516	0\\
517	0\\
518	0\\
519	0\\
520	0\\
521	0\\
522	0\\
523	0\\
524	0\\
525	0\\
526	0\\
527	0\\
528	0\\
529	0\\
530	0\\
531	0\\
532	0\\
533	0\\
534	0\\
535	0\\
536	0\\
537	0\\
538	0\\
539	0\\
540	0\\
541	0\\
542	0\\
543	0\\
544	0\\
545	0\\
546	0\\
547	0\\
548	0\\
549	0\\
550	0\\
551	0\\
552	0\\
553	0\\
554	0\\
555	0\\
556	0\\
557	0\\
558	0\\
559	0\\
560	0\\
561	0\\
562	0\\
563	0\\
564	0\\
565	0\\
566	0\\
567	0\\
568	0\\
569	0\\
570	0\\
571	0\\
572	0\\
573	0\\
574	0\\
575	0\\
576	0\\
577	0\\
578	0\\
579	0\\
580	0\\
581	0\\
582	0\\
583	0\\
584	0\\
585	0\\
586	0\\
587	0\\
588	0\\
589	0\\
590	0\\
591	0\\
592	0\\
593	0\\
594	0\\
595	0\\
596	0\\
597	0\\
598	0\\
599	0\\
600	0\\
};
\addplot [color=red!80!mycolor19,solid,forget plot]
  table[row sep=crcr]{%
1	0\\
2	0\\
3	0\\
4	0\\
5	0\\
6	0\\
7	0\\
8	0\\
9	0\\
10	0\\
11	0\\
12	0\\
13	0\\
14	0\\
15	0\\
16	0\\
17	0\\
18	0\\
19	0\\
20	0\\
21	0\\
22	0\\
23	0\\
24	0\\
25	0\\
26	0\\
27	0\\
28	0\\
29	0\\
30	0\\
31	0\\
32	0\\
33	0\\
34	0\\
35	0\\
36	0\\
37	0\\
38	0\\
39	0\\
40	0\\
41	0\\
42	0\\
43	0\\
44	0\\
45	0\\
46	0\\
47	0\\
48	0\\
49	0\\
50	0\\
51	0\\
52	0\\
53	0\\
54	0\\
55	0\\
56	0\\
57	0\\
58	0\\
59	0\\
60	0\\
61	0\\
62	0\\
63	0\\
64	0\\
65	0\\
66	0\\
67	0\\
68	0\\
69	0\\
70	0\\
71	0\\
72	0\\
73	0\\
74	0\\
75	0\\
76	0\\
77	0\\
78	0\\
79	0\\
80	0\\
81	0\\
82	0\\
83	0\\
84	0\\
85	0\\
86	0\\
87	0\\
88	0\\
89	0\\
90	0\\
91	0\\
92	0\\
93	0\\
94	0\\
95	0\\
96	0\\
97	0\\
98	0\\
99	0\\
100	0\\
101	0\\
102	0\\
103	0\\
104	0\\
105	0\\
106	0\\
107	0\\
108	0\\
109	0\\
110	0\\
111	0\\
112	0\\
113	0\\
114	0\\
115	0\\
116	0\\
117	0\\
118	0\\
119	0\\
120	0\\
121	0\\
122	0\\
123	0\\
124	0\\
125	0\\
126	0\\
127	0\\
128	0\\
129	0\\
130	0\\
131	0\\
132	0\\
133	0\\
134	0\\
135	0\\
136	0\\
137	0\\
138	0\\
139	0\\
140	0\\
141	0\\
142	0\\
143	0\\
144	0\\
145	0\\
146	0\\
147	0\\
148	0\\
149	0\\
150	0\\
151	0\\
152	0\\
153	0\\
154	0\\
155	0\\
156	0\\
157	0\\
158	0\\
159	0\\
160	0\\
161	0\\
162	0\\
163	0\\
164	0\\
165	0\\
166	0\\
167	0\\
168	0\\
169	0\\
170	0\\
171	0\\
172	0\\
173	0\\
174	0\\
175	0\\
176	0\\
177	0\\
178	0\\
179	0\\
180	0\\
181	0\\
182	0\\
183	0\\
184	0\\
185	0\\
186	0\\
187	0\\
188	0\\
189	0\\
190	0\\
191	0\\
192	0\\
193	0\\
194	0\\
195	0\\
196	0\\
197	0\\
198	0\\
199	0\\
200	0\\
201	0\\
202	0\\
203	0\\
204	0\\
205	0\\
206	0\\
207	0\\
208	0\\
209	0\\
210	0\\
211	0\\
212	0\\
213	0\\
214	0\\
215	0\\
216	0\\
217	0\\
218	0\\
219	0\\
220	0\\
221	0\\
222	0\\
223	0\\
224	0\\
225	0\\
226	0\\
227	0\\
228	0\\
229	0\\
230	0\\
231	0\\
232	0\\
233	0\\
234	0\\
235	0\\
236	0\\
237	0\\
238	0\\
239	0\\
240	0\\
241	0\\
242	0\\
243	0\\
244	0\\
245	0\\
246	0\\
247	0\\
248	0\\
249	0\\
250	0\\
251	0\\
252	0\\
253	0\\
254	0\\
255	0\\
256	0\\
257	0\\
258	0\\
259	0\\
260	0\\
261	0\\
262	0\\
263	0\\
264	0\\
265	0\\
266	0\\
267	0\\
268	0\\
269	0\\
270	0\\
271	0\\
272	0\\
273	0\\
274	0\\
275	0\\
276	0\\
277	0\\
278	0\\
279	0\\
280	0\\
281	0\\
282	0\\
283	0\\
284	0\\
285	0\\
286	0\\
287	0\\
288	0\\
289	0\\
290	0\\
291	0\\
292	0\\
293	0\\
294	0\\
295	0\\
296	0\\
297	0\\
298	0\\
299	0\\
300	0\\
301	0\\
302	0\\
303	0\\
304	0\\
305	0\\
306	0\\
307	0\\
308	0\\
309	0\\
310	0\\
311	0\\
312	0\\
313	0\\
314	0\\
315	0\\
316	0\\
317	0\\
318	0\\
319	0\\
320	0\\
321	0\\
322	0\\
323	0\\
324	0\\
325	0\\
326	0\\
327	0\\
328	0\\
329	0\\
330	0\\
331	0\\
332	0\\
333	0\\
334	0\\
335	0\\
336	0\\
337	0\\
338	0\\
339	0\\
340	0\\
341	0\\
342	0\\
343	0\\
344	0\\
345	0\\
346	0\\
347	0\\
348	0\\
349	0\\
350	0\\
351	0\\
352	0\\
353	0\\
354	0\\
355	0\\
356	0\\
357	0\\
358	0\\
359	0\\
360	0\\
361	0\\
362	0\\
363	0\\
364	0\\
365	0\\
366	0\\
367	0\\
368	0\\
369	0\\
370	0\\
371	0\\
372	0\\
373	0\\
374	0\\
375	0\\
376	0\\
377	0\\
378	0\\
379	0\\
380	0\\
381	0\\
382	0\\
383	0\\
384	0\\
385	0\\
386	0\\
387	0\\
388	0\\
389	0\\
390	0\\
391	0\\
392	0\\
393	0\\
394	0\\
395	0\\
396	0\\
397	0\\
398	0\\
399	0\\
400	0\\
401	0\\
402	0\\
403	0\\
404	0\\
405	0\\
406	0\\
407	0\\
408	0\\
409	0\\
410	0\\
411	0\\
412	0\\
413	0\\
414	0\\
415	0\\
416	0\\
417	0\\
418	0\\
419	0\\
420	0\\
421	0\\
422	0\\
423	0\\
424	0\\
425	0\\
426	0\\
427	0\\
428	0\\
429	0\\
430	0\\
431	0\\
432	0\\
433	0\\
434	0\\
435	0\\
436	0\\
437	0\\
438	0\\
439	0\\
440	0\\
441	0\\
442	0\\
443	0\\
444	0\\
445	0\\
446	0\\
447	0\\
448	0\\
449	0\\
450	0\\
451	0\\
452	0\\
453	0\\
454	0\\
455	0\\
456	0\\
457	0\\
458	0\\
459	0\\
460	0\\
461	0\\
462	0\\
463	0\\
464	0\\
465	0\\
466	0\\
467	0\\
468	0\\
469	0\\
470	0\\
471	0\\
472	0\\
473	0\\
474	0\\
475	0\\
476	0\\
477	0\\
478	0\\
479	0\\
480	0\\
481	0\\
482	0\\
483	0\\
484	0\\
485	0\\
486	0\\
487	0\\
488	0\\
489	0\\
490	0\\
491	0\\
492	0\\
493	0\\
494	0\\
495	0\\
496	0\\
497	0\\
498	0\\
499	0\\
500	0\\
501	0\\
502	0\\
503	0\\
504	0\\
505	0\\
506	0\\
507	0\\
508	0\\
509	0\\
510	0\\
511	0\\
512	0\\
513	0\\
514	0\\
515	0\\
516	0\\
517	0\\
518	0\\
519	0\\
520	0\\
521	0\\
522	0\\
523	0\\
524	0\\
525	0\\
526	0\\
527	0\\
528	0\\
529	0\\
530	0\\
531	0\\
532	0\\
533	0\\
534	0\\
535	0\\
536	0\\
537	0\\
538	0\\
539	0\\
540	0\\
541	0\\
542	0\\
543	0\\
544	0\\
545	0\\
546	0\\
547	0\\
548	0\\
549	0\\
550	0\\
551	0\\
552	0\\
553	0\\
554	0\\
555	0\\
556	0\\
557	0\\
558	0\\
559	0\\
560	0\\
561	0\\
562	0\\
563	0\\
564	0\\
565	0\\
566	0\\
567	0\\
568	0\\
569	0\\
570	0\\
571	0\\
572	0\\
573	0\\
574	0\\
575	0\\
576	0\\
577	0\\
578	0\\
579	0\\
580	0\\
581	0\\
582	0\\
583	0\\
584	0\\
585	0\\
586	0\\
587	0\\
588	0\\
589	0\\
590	0\\
591	0\\
592	0\\
593	0\\
594	0\\
595	0\\
596	0\\
597	0\\
598	0\\
599	0\\
600	0\\
};
\addplot [color=red,solid,forget plot]
  table[row sep=crcr]{%
1	0\\
2	0\\
3	0\\
4	0\\
5	0\\
6	0\\
7	0\\
8	0\\
9	0\\
10	0\\
11	0\\
12	0\\
13	0\\
14	0\\
15	0\\
16	0\\
17	0\\
18	0\\
19	0\\
20	0\\
21	0\\
22	0\\
23	0\\
24	0\\
25	0\\
26	0\\
27	0\\
28	0\\
29	0\\
30	0\\
31	0\\
32	0\\
33	0\\
34	0\\
35	0\\
36	0\\
37	0\\
38	0\\
39	0\\
40	0\\
41	0\\
42	0\\
43	0\\
44	0\\
45	0\\
46	0\\
47	0\\
48	0\\
49	0\\
50	0\\
51	0\\
52	0\\
53	0\\
54	0\\
55	0\\
56	0\\
57	0\\
58	0\\
59	0\\
60	0\\
61	0\\
62	0\\
63	0\\
64	0\\
65	0\\
66	0\\
67	0\\
68	0\\
69	0\\
70	0\\
71	0\\
72	0\\
73	0\\
74	0\\
75	0\\
76	0\\
77	0\\
78	0\\
79	0\\
80	0\\
81	0\\
82	0\\
83	0\\
84	0\\
85	0\\
86	0\\
87	0\\
88	0\\
89	0\\
90	0\\
91	0\\
92	0\\
93	0\\
94	0\\
95	0\\
96	0\\
97	0\\
98	0\\
99	0\\
100	0\\
101	0\\
102	0\\
103	0\\
104	0\\
105	0\\
106	0\\
107	0\\
108	0\\
109	0\\
110	0\\
111	0\\
112	0\\
113	0\\
114	0\\
115	0\\
116	0\\
117	0\\
118	0\\
119	0\\
120	0\\
121	0\\
122	0\\
123	0\\
124	0\\
125	0\\
126	0\\
127	0\\
128	0\\
129	0\\
130	0\\
131	0\\
132	0\\
133	0\\
134	0\\
135	0\\
136	0\\
137	0\\
138	0\\
139	0\\
140	0\\
141	0\\
142	0\\
143	0\\
144	0\\
145	0\\
146	0\\
147	0\\
148	0\\
149	0\\
150	0\\
151	0\\
152	0\\
153	0\\
154	0\\
155	0\\
156	0\\
157	0\\
158	0\\
159	0\\
160	0\\
161	0\\
162	0\\
163	0\\
164	0\\
165	0\\
166	0\\
167	0\\
168	0\\
169	0\\
170	0\\
171	0\\
172	0\\
173	0\\
174	0\\
175	0\\
176	0\\
177	0\\
178	0\\
179	0\\
180	0\\
181	0\\
182	0\\
183	0\\
184	0\\
185	0\\
186	0\\
187	0\\
188	0\\
189	0\\
190	0\\
191	0\\
192	0\\
193	0\\
194	0\\
195	0\\
196	0\\
197	0\\
198	0\\
199	0\\
200	0\\
201	0\\
202	0\\
203	0\\
204	0\\
205	0\\
206	0\\
207	0\\
208	0\\
209	0\\
210	0\\
211	0\\
212	0\\
213	0\\
214	0\\
215	0\\
216	0\\
217	0\\
218	0\\
219	0\\
220	0\\
221	0\\
222	0\\
223	0\\
224	0\\
225	0\\
226	0\\
227	0\\
228	0\\
229	0\\
230	0\\
231	0\\
232	0\\
233	0\\
234	0\\
235	0\\
236	0\\
237	0\\
238	0\\
239	0\\
240	0\\
241	0\\
242	0\\
243	0\\
244	0\\
245	0\\
246	0\\
247	0\\
248	0\\
249	0\\
250	0\\
251	0\\
252	0\\
253	0\\
254	0\\
255	0\\
256	0\\
257	0\\
258	0\\
259	0\\
260	0\\
261	0\\
262	0\\
263	0\\
264	0\\
265	0\\
266	0\\
267	0\\
268	0\\
269	0\\
270	0\\
271	0\\
272	0\\
273	0\\
274	0\\
275	0\\
276	0\\
277	0\\
278	0\\
279	0\\
280	0\\
281	0\\
282	0\\
283	0\\
284	0\\
285	0\\
286	0\\
287	0\\
288	0\\
289	0\\
290	0\\
291	0\\
292	0\\
293	0\\
294	0\\
295	0\\
296	0\\
297	0\\
298	0\\
299	0\\
300	0\\
301	0\\
302	0\\
303	0\\
304	0\\
305	0\\
306	0\\
307	0\\
308	0\\
309	0\\
310	0\\
311	0\\
312	0\\
313	0\\
314	0\\
315	0\\
316	0\\
317	0\\
318	0\\
319	0\\
320	0\\
321	0\\
322	0\\
323	0\\
324	0\\
325	0\\
326	0\\
327	0\\
328	0\\
329	0\\
330	0\\
331	0\\
332	0\\
333	0\\
334	0\\
335	0\\
336	0\\
337	0\\
338	0\\
339	0\\
340	0\\
341	0\\
342	0\\
343	0\\
344	0\\
345	0\\
346	0\\
347	0\\
348	0\\
349	0\\
350	0\\
351	0\\
352	0\\
353	0\\
354	0\\
355	0\\
356	0\\
357	0\\
358	0\\
359	0\\
360	0\\
361	0\\
362	0\\
363	0\\
364	0\\
365	0\\
366	0\\
367	0\\
368	0\\
369	0\\
370	0\\
371	0\\
372	0\\
373	0\\
374	0\\
375	0\\
376	0\\
377	0\\
378	0\\
379	0\\
380	0\\
381	0\\
382	0\\
383	0\\
384	0\\
385	0\\
386	0\\
387	0\\
388	0\\
389	0\\
390	0\\
391	0\\
392	0\\
393	0\\
394	0\\
395	0\\
396	0\\
397	0\\
398	0\\
399	0\\
400	0\\
401	0\\
402	0\\
403	0\\
404	0\\
405	0\\
406	0\\
407	0\\
408	0\\
409	0\\
410	0\\
411	0\\
412	0\\
413	0\\
414	0\\
415	0\\
416	0\\
417	0\\
418	0\\
419	0\\
420	0\\
421	0\\
422	0\\
423	0\\
424	0\\
425	0\\
426	0\\
427	0\\
428	0\\
429	0\\
430	0\\
431	0\\
432	0\\
433	0\\
434	0\\
435	0\\
436	0\\
437	0\\
438	0\\
439	0\\
440	0\\
441	0\\
442	0\\
443	0\\
444	0\\
445	0\\
446	0\\
447	0\\
448	0\\
449	0\\
450	0\\
451	0\\
452	0\\
453	0\\
454	0\\
455	0\\
456	0\\
457	0\\
458	0\\
459	0\\
460	0\\
461	0\\
462	0\\
463	0\\
464	0\\
465	0\\
466	0\\
467	0\\
468	0\\
469	0\\
470	0\\
471	0\\
472	0\\
473	0\\
474	0\\
475	0\\
476	0\\
477	0\\
478	0\\
479	0\\
480	0\\
481	0\\
482	0\\
483	0\\
484	0\\
485	0\\
486	0\\
487	0\\
488	0\\
489	0\\
490	0\\
491	0\\
492	0\\
493	0\\
494	0\\
495	0\\
496	0\\
497	0\\
498	0\\
499	0\\
500	0\\
501	0\\
502	0\\
503	0\\
504	0\\
505	0\\
506	0\\
507	0\\
508	0\\
509	0\\
510	0\\
511	0\\
512	0\\
513	0\\
514	0\\
515	0\\
516	0\\
517	0\\
518	0\\
519	0\\
520	0\\
521	0\\
522	0\\
523	0\\
524	0\\
525	0\\
526	0\\
527	0\\
528	0\\
529	0\\
530	0\\
531	0\\
532	0\\
533	0\\
534	0\\
535	0\\
536	0\\
537	0\\
538	0\\
539	0\\
540	0\\
541	0\\
542	0\\
543	0\\
544	0\\
545	0\\
546	0\\
547	0\\
548	0\\
549	0\\
550	0\\
551	0\\
552	0\\
553	0\\
554	0\\
555	0\\
556	0\\
557	0\\
558	0\\
559	0\\
560	0\\
561	0\\
562	0\\
563	0\\
564	0\\
565	0\\
566	0\\
567	0\\
568	0\\
569	0\\
570	0\\
571	0\\
572	0\\
573	0\\
574	0\\
575	0\\
576	0\\
577	0\\
578	0\\
579	0\\
580	0\\
581	0\\
582	0\\
583	0\\
584	0\\
585	0\\
586	0\\
587	0\\
588	0\\
589	0\\
590	0\\
591	0\\
592	0\\
593	0\\
594	0\\
595	0\\
596	0\\
597	0\\
598	0\\
599	0\\
600	0\\
};
\addplot [color=mycolor20,solid,forget plot]
  table[row sep=crcr]{%
1	0\\
2	0\\
3	0\\
4	0\\
5	0\\
6	0\\
7	0\\
8	0\\
9	0\\
10	0\\
11	0\\
12	0\\
13	0\\
14	0\\
15	0\\
16	0\\
17	0\\
18	0\\
19	0\\
20	0\\
21	0\\
22	0\\
23	0\\
24	0\\
25	0\\
26	0\\
27	0\\
28	0\\
29	0\\
30	0\\
31	0\\
32	0\\
33	0\\
34	0\\
35	0\\
36	0\\
37	0\\
38	0\\
39	0\\
40	0\\
41	0\\
42	0\\
43	0\\
44	0\\
45	0\\
46	0\\
47	0\\
48	0\\
49	0\\
50	0\\
51	0\\
52	0\\
53	0\\
54	0\\
55	0\\
56	0\\
57	0\\
58	0\\
59	0\\
60	0\\
61	0\\
62	0\\
63	0\\
64	0\\
65	0\\
66	0\\
67	0\\
68	0\\
69	0\\
70	0\\
71	0\\
72	0\\
73	0\\
74	0\\
75	0\\
76	0\\
77	0\\
78	0\\
79	0\\
80	0\\
81	0\\
82	0\\
83	0\\
84	0\\
85	0\\
86	0\\
87	0\\
88	0\\
89	0\\
90	0\\
91	0\\
92	0\\
93	0\\
94	0\\
95	0\\
96	0\\
97	0\\
98	0\\
99	0\\
100	0\\
101	0\\
102	0\\
103	0\\
104	0\\
105	0\\
106	0\\
107	0\\
108	0\\
109	0\\
110	0\\
111	0\\
112	0\\
113	0\\
114	0\\
115	0\\
116	0\\
117	0\\
118	0\\
119	0\\
120	0\\
121	0\\
122	0\\
123	0\\
124	0\\
125	0\\
126	0\\
127	0\\
128	0\\
129	0\\
130	0\\
131	0\\
132	0\\
133	0\\
134	0\\
135	0\\
136	0\\
137	0\\
138	0\\
139	0\\
140	0\\
141	0\\
142	0\\
143	0\\
144	0\\
145	0\\
146	0\\
147	0\\
148	0\\
149	0\\
150	0\\
151	0\\
152	0\\
153	0\\
154	0\\
155	0\\
156	0\\
157	0\\
158	0\\
159	0\\
160	0\\
161	0\\
162	0\\
163	0\\
164	0\\
165	0\\
166	0\\
167	0\\
168	0\\
169	0\\
170	0\\
171	0\\
172	0\\
173	0\\
174	0\\
175	0\\
176	0\\
177	0\\
178	0\\
179	0\\
180	0\\
181	0\\
182	0\\
183	0\\
184	0\\
185	0\\
186	0\\
187	0\\
188	0\\
189	0\\
190	0\\
191	0\\
192	0\\
193	0\\
194	0\\
195	0\\
196	0\\
197	0\\
198	0\\
199	0\\
200	0\\
201	0\\
202	0\\
203	0\\
204	0\\
205	0\\
206	0\\
207	0\\
208	0\\
209	0\\
210	0\\
211	0\\
212	0\\
213	0\\
214	0\\
215	0\\
216	0\\
217	0\\
218	0\\
219	0\\
220	0\\
221	0\\
222	0\\
223	0\\
224	0\\
225	0\\
226	0\\
227	0\\
228	0\\
229	0\\
230	0\\
231	0\\
232	0\\
233	0\\
234	0\\
235	0\\
236	0\\
237	0\\
238	0\\
239	0\\
240	0\\
241	0\\
242	0\\
243	0\\
244	0\\
245	0\\
246	0\\
247	0\\
248	0\\
249	0\\
250	0\\
251	0\\
252	0\\
253	0\\
254	0\\
255	0\\
256	0\\
257	0\\
258	0\\
259	0\\
260	0\\
261	0\\
262	0\\
263	0\\
264	0\\
265	0\\
266	0\\
267	0\\
268	0\\
269	0\\
270	0\\
271	0\\
272	0\\
273	0\\
274	0\\
275	0\\
276	0\\
277	0\\
278	0\\
279	0\\
280	0\\
281	0\\
282	0\\
283	0\\
284	0\\
285	0\\
286	0\\
287	0\\
288	0\\
289	0\\
290	0\\
291	0\\
292	0\\
293	0\\
294	0\\
295	0\\
296	0\\
297	0\\
298	0\\
299	0\\
300	0\\
301	0\\
302	0\\
303	0\\
304	0\\
305	0\\
306	0\\
307	0\\
308	0\\
309	0\\
310	0\\
311	0\\
312	0\\
313	0\\
314	0\\
315	0\\
316	0\\
317	0\\
318	0\\
319	0\\
320	0\\
321	0\\
322	0\\
323	0\\
324	0\\
325	0\\
326	0\\
327	0\\
328	0\\
329	0\\
330	0\\
331	0\\
332	0\\
333	0\\
334	0\\
335	0\\
336	0\\
337	0\\
338	0\\
339	0\\
340	0\\
341	0\\
342	0\\
343	0\\
344	0\\
345	0\\
346	0\\
347	0\\
348	0\\
349	0\\
350	0\\
351	0\\
352	0\\
353	0\\
354	0\\
355	0\\
356	0\\
357	0\\
358	0\\
359	0\\
360	0\\
361	0\\
362	0\\
363	0\\
364	0\\
365	0\\
366	0\\
367	0\\
368	0\\
369	0\\
370	0\\
371	0\\
372	0\\
373	0\\
374	0\\
375	0\\
376	0\\
377	0\\
378	0\\
379	0\\
380	0\\
381	0\\
382	0\\
383	0\\
384	0\\
385	0\\
386	0\\
387	0\\
388	0\\
389	0\\
390	0\\
391	0\\
392	0\\
393	0\\
394	0\\
395	0\\
396	0\\
397	0\\
398	0\\
399	0\\
400	0\\
401	0\\
402	0\\
403	0\\
404	0\\
405	0\\
406	0\\
407	0\\
408	0\\
409	0\\
410	0\\
411	0\\
412	0\\
413	0\\
414	0\\
415	0\\
416	0\\
417	0\\
418	0\\
419	0\\
420	0\\
421	0\\
422	0\\
423	0\\
424	0\\
425	0\\
426	0\\
427	0\\
428	0\\
429	0\\
430	0\\
431	0\\
432	0\\
433	0\\
434	0\\
435	0\\
436	0\\
437	0\\
438	0\\
439	0\\
440	0\\
441	0\\
442	0\\
443	0\\
444	0\\
445	0\\
446	0\\
447	0\\
448	0\\
449	0\\
450	0\\
451	0\\
452	0\\
453	0\\
454	0\\
455	0\\
456	0\\
457	0\\
458	0\\
459	0\\
460	0\\
461	0\\
462	0\\
463	0\\
464	0\\
465	0\\
466	0\\
467	0\\
468	0\\
469	0\\
470	0\\
471	0\\
472	0\\
473	0\\
474	0\\
475	0\\
476	0\\
477	0\\
478	0\\
479	0\\
480	0\\
481	0\\
482	0\\
483	0\\
484	0\\
485	0\\
486	0\\
487	0\\
488	0\\
489	0\\
490	0\\
491	0\\
492	0\\
493	0\\
494	0\\
495	0\\
496	0\\
497	0\\
498	0\\
499	0\\
500	0\\
501	0\\
502	0\\
503	0\\
504	0\\
505	0\\
506	0\\
507	0\\
508	0\\
509	0\\
510	0\\
511	0\\
512	0\\
513	0\\
514	0\\
515	0\\
516	0\\
517	0\\
518	0\\
519	0\\
520	0\\
521	0\\
522	0\\
523	0\\
524	0\\
525	0\\
526	0\\
527	0\\
528	0\\
529	0\\
530	0\\
531	0\\
532	0\\
533	0\\
534	0\\
535	0\\
536	0\\
537	0\\
538	0\\
539	0\\
540	0\\
541	0\\
542	0\\
543	0\\
544	0\\
545	0\\
546	0\\
547	0\\
548	0\\
549	0\\
550	0\\
551	0\\
552	0\\
553	0\\
554	0\\
555	0\\
556	0\\
557	0\\
558	0\\
559	0\\
560	0\\
561	0\\
562	0\\
563	0\\
564	0\\
565	0\\
566	0\\
567	0\\
568	0\\
569	0\\
570	0\\
571	0\\
572	0\\
573	0\\
574	0\\
575	0\\
576	0\\
577	0\\
578	0\\
579	0\\
580	0\\
581	0\\
582	0\\
583	0\\
584	0\\
585	0\\
586	0\\
587	0\\
588	0\\
589	0\\
590	0\\
591	0\\
592	0\\
593	0\\
594	0\\
595	0\\
596	0\\
597	0\\
598	0\\
599	0\\
600	0\\
};
\addplot [color=mycolor21,solid,forget plot]
  table[row sep=crcr]{%
1	0\\
2	0\\
3	0\\
4	0\\
5	0\\
6	0\\
7	0\\
8	0\\
9	0\\
10	0\\
11	0\\
12	0\\
13	0\\
14	0\\
15	0\\
16	0\\
17	0\\
18	0\\
19	0\\
20	0\\
21	0\\
22	0\\
23	0\\
24	0\\
25	0\\
26	0\\
27	0\\
28	0\\
29	0\\
30	0\\
31	0\\
32	0\\
33	0\\
34	0\\
35	0\\
36	0\\
37	0\\
38	0\\
39	0\\
40	0\\
41	0\\
42	0\\
43	0\\
44	0\\
45	0\\
46	0\\
47	0\\
48	0\\
49	0\\
50	0\\
51	0\\
52	0\\
53	0\\
54	0\\
55	0\\
56	0\\
57	0\\
58	0\\
59	0\\
60	0\\
61	0\\
62	0\\
63	0\\
64	0\\
65	0\\
66	0\\
67	0\\
68	0\\
69	0\\
70	0\\
71	0\\
72	0\\
73	0\\
74	0\\
75	0\\
76	0\\
77	0\\
78	0\\
79	0\\
80	0\\
81	0\\
82	0\\
83	0\\
84	0\\
85	0\\
86	0\\
87	0\\
88	0\\
89	0\\
90	0\\
91	0\\
92	0\\
93	0\\
94	0\\
95	0\\
96	0\\
97	0\\
98	0\\
99	0\\
100	0\\
101	0\\
102	0\\
103	0\\
104	0\\
105	0\\
106	0\\
107	0\\
108	0\\
109	0\\
110	0\\
111	0\\
112	0\\
113	0\\
114	0\\
115	0\\
116	0\\
117	0\\
118	0\\
119	0\\
120	0\\
121	0\\
122	0\\
123	0\\
124	0\\
125	0\\
126	0\\
127	0\\
128	0\\
129	0\\
130	0\\
131	0\\
132	0\\
133	0\\
134	0\\
135	0\\
136	0\\
137	0\\
138	0\\
139	0\\
140	0\\
141	0\\
142	0\\
143	0\\
144	0\\
145	0\\
146	0\\
147	0\\
148	0\\
149	0\\
150	0\\
151	0\\
152	0\\
153	0\\
154	0\\
155	0\\
156	0\\
157	0\\
158	0\\
159	0\\
160	0\\
161	0\\
162	0\\
163	0\\
164	0\\
165	0\\
166	0\\
167	0\\
168	0\\
169	0\\
170	0\\
171	0\\
172	0\\
173	0\\
174	0\\
175	0\\
176	0\\
177	0\\
178	0\\
179	0\\
180	0\\
181	0\\
182	0\\
183	0\\
184	0\\
185	0\\
186	0\\
187	0\\
188	0\\
189	0\\
190	0\\
191	0\\
192	0\\
193	0\\
194	0\\
195	0\\
196	0\\
197	0\\
198	0\\
199	0\\
200	0\\
201	0\\
202	0\\
203	0\\
204	0\\
205	0\\
206	0\\
207	0\\
208	0\\
209	0\\
210	0\\
211	0\\
212	0\\
213	0\\
214	0\\
215	0\\
216	0\\
217	0\\
218	0\\
219	0\\
220	0\\
221	0\\
222	0\\
223	0\\
224	0\\
225	0\\
226	0\\
227	0\\
228	0\\
229	0\\
230	0\\
231	0\\
232	0\\
233	0\\
234	0\\
235	0\\
236	0\\
237	0\\
238	0\\
239	0\\
240	0\\
241	0\\
242	0\\
243	0\\
244	0\\
245	0\\
246	0\\
247	0\\
248	0\\
249	0\\
250	0\\
251	0\\
252	0\\
253	0\\
254	0\\
255	0\\
256	0\\
257	0\\
258	0\\
259	0\\
260	0\\
261	0\\
262	0\\
263	0\\
264	0\\
265	0\\
266	0\\
267	0\\
268	0\\
269	0\\
270	0\\
271	0\\
272	0\\
273	0\\
274	0\\
275	0\\
276	0\\
277	0\\
278	0\\
279	0\\
280	0\\
281	0\\
282	0\\
283	0\\
284	0\\
285	0\\
286	0\\
287	0\\
288	0\\
289	0\\
290	0\\
291	0\\
292	0\\
293	0\\
294	0\\
295	0\\
296	0\\
297	0\\
298	0\\
299	0\\
300	0\\
301	0\\
302	0\\
303	0\\
304	0\\
305	0\\
306	0\\
307	0\\
308	0\\
309	0\\
310	0\\
311	0\\
312	0\\
313	0\\
314	0\\
315	0\\
316	0\\
317	0\\
318	0\\
319	0\\
320	0\\
321	0\\
322	0\\
323	0\\
324	0\\
325	0\\
326	0\\
327	0\\
328	0\\
329	0\\
330	0\\
331	0\\
332	0\\
333	0\\
334	0\\
335	0\\
336	0\\
337	0\\
338	0\\
339	0\\
340	0\\
341	0\\
342	0\\
343	0\\
344	0\\
345	0\\
346	0\\
347	0\\
348	0\\
349	0\\
350	0\\
351	0\\
352	0\\
353	0\\
354	0\\
355	0\\
356	0\\
357	0\\
358	0\\
359	0\\
360	0\\
361	0\\
362	0\\
363	0\\
364	0\\
365	0\\
366	0\\
367	0\\
368	0\\
369	0\\
370	0\\
371	0\\
372	0\\
373	0\\
374	0\\
375	0\\
376	0\\
377	0\\
378	0\\
379	0\\
380	0\\
381	0\\
382	0\\
383	0\\
384	0\\
385	0\\
386	0\\
387	0\\
388	0\\
389	0\\
390	0\\
391	0\\
392	0\\
393	0\\
394	0\\
395	0\\
396	0\\
397	0\\
398	0\\
399	0\\
400	0\\
401	0\\
402	0\\
403	0\\
404	0\\
405	0\\
406	0\\
407	0\\
408	0\\
409	0\\
410	0\\
411	0\\
412	0\\
413	0\\
414	0\\
415	0\\
416	0\\
417	0\\
418	0\\
419	0\\
420	0\\
421	0\\
422	0\\
423	0\\
424	0\\
425	0\\
426	0\\
427	0\\
428	0\\
429	0\\
430	0\\
431	0\\
432	0\\
433	0\\
434	0\\
435	0\\
436	0\\
437	0\\
438	0\\
439	0\\
440	0\\
441	0\\
442	0\\
443	0\\
444	0\\
445	0\\
446	0\\
447	0\\
448	0\\
449	0\\
450	0\\
451	0\\
452	0\\
453	0\\
454	0\\
455	0\\
456	0\\
457	0\\
458	0\\
459	0\\
460	0\\
461	0\\
462	0\\
463	0\\
464	0\\
465	0\\
466	0\\
467	0\\
468	0\\
469	0\\
470	0\\
471	0\\
472	0\\
473	0\\
474	0\\
475	0\\
476	0\\
477	0\\
478	0\\
479	0\\
480	0\\
481	0\\
482	0\\
483	0\\
484	0\\
485	0\\
486	0\\
487	0\\
488	0\\
489	0\\
490	0\\
491	0\\
492	0\\
493	0\\
494	0\\
495	0\\
496	0\\
497	0\\
498	0\\
499	0\\
500	0\\
501	0\\
502	0\\
503	0\\
504	0\\
505	0\\
506	0\\
507	0\\
508	0\\
509	0\\
510	0\\
511	0\\
512	0\\
513	0\\
514	0\\
515	0\\
516	0\\
517	0\\
518	0\\
519	0\\
520	0\\
521	0\\
522	0\\
523	0\\
524	0\\
525	0\\
526	0\\
527	0\\
528	0\\
529	0\\
530	0\\
531	0\\
532	0\\
533	0\\
534	0\\
535	0\\
536	0\\
537	0\\
538	0\\
539	0\\
540	0\\
541	0\\
542	0\\
543	0\\
544	0\\
545	0\\
546	0\\
547	0\\
548	0\\
549	0\\
550	0\\
551	0\\
552	0\\
553	0\\
554	0\\
555	0\\
556	0\\
557	0\\
558	0\\
559	0\\
560	0\\
561	0\\
562	0\\
563	0\\
564	0\\
565	0\\
566	0\\
567	0\\
568	0\\
569	0\\
570	0\\
571	0\\
572	0\\
573	0\\
574	0\\
575	0\\
576	0\\
577	0\\
578	0\\
579	0\\
580	0\\
581	0\\
582	0\\
583	0\\
584	0\\
585	0\\
586	0\\
587	0\\
588	0\\
589	0\\
590	0\\
591	0\\
592	0\\
593	0\\
594	0\\
595	0\\
596	0\\
597	0\\
598	0\\
599	0\\
600	0\\
};
\addplot [color=black!20!mycolor21,solid,forget plot]
  table[row sep=crcr]{%
1	0\\
2	0\\
3	0\\
4	0\\
5	0\\
6	0\\
7	0\\
8	0\\
9	0\\
10	0\\
11	0\\
12	0\\
13	0\\
14	0\\
15	0\\
16	0\\
17	0\\
18	0\\
19	0\\
20	0\\
21	0\\
22	0\\
23	0\\
24	0\\
25	0\\
26	0\\
27	0\\
28	0\\
29	0\\
30	0\\
31	0\\
32	0\\
33	0\\
34	0\\
35	0\\
36	0\\
37	0\\
38	0\\
39	0\\
40	0\\
41	0\\
42	0\\
43	0\\
44	0\\
45	0\\
46	0\\
47	0\\
48	0\\
49	0\\
50	0\\
51	0\\
52	0\\
53	0\\
54	0\\
55	0\\
56	0\\
57	0\\
58	0\\
59	0\\
60	0\\
61	0\\
62	0\\
63	0\\
64	0\\
65	0\\
66	0\\
67	0\\
68	0\\
69	0\\
70	0\\
71	0\\
72	0\\
73	0\\
74	0\\
75	0\\
76	0\\
77	0\\
78	0\\
79	0\\
80	0\\
81	0\\
82	0\\
83	0\\
84	0\\
85	0\\
86	0\\
87	0\\
88	0\\
89	0\\
90	0\\
91	0\\
92	0\\
93	0\\
94	0\\
95	0\\
96	0\\
97	0\\
98	0\\
99	0\\
100	0\\
101	0\\
102	0\\
103	0\\
104	0\\
105	0\\
106	0\\
107	0\\
108	0\\
109	0\\
110	0\\
111	0\\
112	0\\
113	0\\
114	0\\
115	0\\
116	0\\
117	0\\
118	0\\
119	0\\
120	0\\
121	0\\
122	0\\
123	0\\
124	0\\
125	0\\
126	0\\
127	0\\
128	0\\
129	0\\
130	0\\
131	0\\
132	0\\
133	0\\
134	0\\
135	0\\
136	0\\
137	0\\
138	0\\
139	0\\
140	0\\
141	0\\
142	0\\
143	0\\
144	0\\
145	0\\
146	0\\
147	0\\
148	0\\
149	0\\
150	0\\
151	0\\
152	0\\
153	0\\
154	0\\
155	0\\
156	0\\
157	0\\
158	0\\
159	0\\
160	0\\
161	0\\
162	0\\
163	0\\
164	0\\
165	0\\
166	0\\
167	0\\
168	0\\
169	0\\
170	0\\
171	0\\
172	0\\
173	0\\
174	0\\
175	0\\
176	0\\
177	0\\
178	0\\
179	0\\
180	0\\
181	0\\
182	0\\
183	0\\
184	0\\
185	0\\
186	0\\
187	0\\
188	0\\
189	0\\
190	0\\
191	0\\
192	0\\
193	0\\
194	0\\
195	0\\
196	0\\
197	0\\
198	0\\
199	0\\
200	0\\
201	0\\
202	0\\
203	0\\
204	0\\
205	0\\
206	0\\
207	0\\
208	0\\
209	0\\
210	0\\
211	0\\
212	0\\
213	0\\
214	0\\
215	0\\
216	0\\
217	0\\
218	0\\
219	0\\
220	0\\
221	0\\
222	0\\
223	0\\
224	0\\
225	0\\
226	0\\
227	0\\
228	0\\
229	0\\
230	0\\
231	0\\
232	0\\
233	0\\
234	0\\
235	0\\
236	0\\
237	0\\
238	0\\
239	0\\
240	0\\
241	0\\
242	0\\
243	0\\
244	0\\
245	0\\
246	0\\
247	0\\
248	0\\
249	0\\
250	0\\
251	0\\
252	0\\
253	0\\
254	0\\
255	0\\
256	0\\
257	0\\
258	0\\
259	0\\
260	0\\
261	0\\
262	0\\
263	0\\
264	0\\
265	0\\
266	0\\
267	0\\
268	0\\
269	0\\
270	0\\
271	0\\
272	0\\
273	0\\
274	0\\
275	0\\
276	0\\
277	0\\
278	0\\
279	0\\
280	0\\
281	0\\
282	0\\
283	0\\
284	0\\
285	0\\
286	0\\
287	0\\
288	0\\
289	0\\
290	0\\
291	0\\
292	0\\
293	0\\
294	0\\
295	0\\
296	0\\
297	0\\
298	0\\
299	0\\
300	0\\
301	0\\
302	0\\
303	0\\
304	0\\
305	0\\
306	0\\
307	0\\
308	0\\
309	0\\
310	0\\
311	0\\
312	0\\
313	0\\
314	0\\
315	0\\
316	0\\
317	0\\
318	0\\
319	0\\
320	0\\
321	0\\
322	0\\
323	0\\
324	0\\
325	0\\
326	0\\
327	0\\
328	0\\
329	0\\
330	0\\
331	0\\
332	0\\
333	0\\
334	0\\
335	0\\
336	0\\
337	0\\
338	0\\
339	0\\
340	0\\
341	0\\
342	0\\
343	0\\
344	0\\
345	0\\
346	0\\
347	0\\
348	0\\
349	0\\
350	0\\
351	0\\
352	0\\
353	0\\
354	0\\
355	0\\
356	0\\
357	0\\
358	0\\
359	0\\
360	0\\
361	0\\
362	0\\
363	0\\
364	0\\
365	0\\
366	0\\
367	0\\
368	0\\
369	0\\
370	0\\
371	0\\
372	0\\
373	0\\
374	0\\
375	0\\
376	0\\
377	0\\
378	0\\
379	0\\
380	0\\
381	0\\
382	0\\
383	0\\
384	0\\
385	0\\
386	0\\
387	0\\
388	0\\
389	0\\
390	0\\
391	0\\
392	0\\
393	0\\
394	0\\
395	0\\
396	0\\
397	0\\
398	0\\
399	0\\
400	0\\
401	0\\
402	0\\
403	0\\
404	0\\
405	0\\
406	0\\
407	0\\
408	0\\
409	0\\
410	0\\
411	0\\
412	0\\
413	0\\
414	0\\
415	0\\
416	0\\
417	0\\
418	0\\
419	0\\
420	0\\
421	0\\
422	0\\
423	0\\
424	0\\
425	0\\
426	0\\
427	0\\
428	0\\
429	0\\
430	0\\
431	0\\
432	0\\
433	0\\
434	0\\
435	0\\
436	0\\
437	0\\
438	0\\
439	0\\
440	0\\
441	0\\
442	0\\
443	0\\
444	0\\
445	0\\
446	0\\
447	0\\
448	0\\
449	0\\
450	0\\
451	0\\
452	0\\
453	0\\
454	0\\
455	0\\
456	0\\
457	0\\
458	0\\
459	0\\
460	0\\
461	0\\
462	0\\
463	0\\
464	0\\
465	0\\
466	0\\
467	0\\
468	0\\
469	0\\
470	0\\
471	0\\
472	0\\
473	0\\
474	0\\
475	0\\
476	0\\
477	0\\
478	0\\
479	0\\
480	0\\
481	0\\
482	0\\
483	0\\
484	0\\
485	0\\
486	0\\
487	0\\
488	0\\
489	0\\
490	0\\
491	0\\
492	0\\
493	0\\
494	0\\
495	0\\
496	0\\
497	0\\
498	0\\
499	0\\
500	0\\
501	0\\
502	0\\
503	0\\
504	0\\
505	0\\
506	0\\
507	0\\
508	0\\
509	0\\
510	0\\
511	0\\
512	0\\
513	0\\
514	0\\
515	0\\
516	0\\
517	0\\
518	0\\
519	0\\
520	0\\
521	0\\
522	0\\
523	0\\
524	0\\
525	0\\
526	0\\
527	0\\
528	0\\
529	0\\
530	0\\
531	0\\
532	0\\
533	0\\
534	0\\
535	0\\
536	0\\
537	0\\
538	0\\
539	0\\
540	0\\
541	0\\
542	0\\
543	0\\
544	0\\
545	0\\
546	0\\
547	0\\
548	0\\
549	0\\
550	0\\
551	0\\
552	0\\
553	0\\
554	0\\
555	0\\
556	0\\
557	0\\
558	0\\
559	0\\
560	0\\
561	0\\
562	0\\
563	0\\
564	0\\
565	0\\
566	0\\
567	0\\
568	0\\
569	0\\
570	0\\
571	0\\
572	0\\
573	0\\
574	0\\
575	0\\
576	0\\
577	0\\
578	0\\
579	0\\
580	0\\
581	0\\
582	0\\
583	0\\
584	0\\
585	0\\
586	0\\
587	0\\
588	0\\
589	0\\
590	0\\
591	0\\
592	0\\
593	0\\
594	0\\
595	0\\
596	0\\
597	0\\
598	0\\
599	0\\
600	0\\
};
\addplot [color=black!50!mycolor20,solid,forget plot]
  table[row sep=crcr]{%
1	0\\
2	0\\
3	0\\
4	0\\
5	0\\
6	0\\
7	0\\
8	0\\
9	0\\
10	0\\
11	0\\
12	0\\
13	0\\
14	0\\
15	0\\
16	0\\
17	0\\
18	0\\
19	0\\
20	0\\
21	0\\
22	0\\
23	0\\
24	0\\
25	0\\
26	0\\
27	0\\
28	0\\
29	0\\
30	0\\
31	0\\
32	0\\
33	0\\
34	0\\
35	0\\
36	0\\
37	0\\
38	0\\
39	0\\
40	0\\
41	0\\
42	0\\
43	0\\
44	0\\
45	0\\
46	0\\
47	0\\
48	0\\
49	0\\
50	0\\
51	0\\
52	0\\
53	0\\
54	0\\
55	0\\
56	0\\
57	0\\
58	0\\
59	0\\
60	0\\
61	0\\
62	0\\
63	0\\
64	0\\
65	0\\
66	0\\
67	0\\
68	0\\
69	0\\
70	0\\
71	0\\
72	0\\
73	0\\
74	0\\
75	0\\
76	0\\
77	0\\
78	0\\
79	0\\
80	0\\
81	0\\
82	0\\
83	0\\
84	0\\
85	0\\
86	0\\
87	0\\
88	0\\
89	0\\
90	0\\
91	0\\
92	0\\
93	0\\
94	0\\
95	0\\
96	0\\
97	0\\
98	0\\
99	0\\
100	0\\
101	0\\
102	0\\
103	0\\
104	0\\
105	0\\
106	0\\
107	0\\
108	0\\
109	0\\
110	0\\
111	0\\
112	0\\
113	0\\
114	0\\
115	0\\
116	0\\
117	0\\
118	0\\
119	0\\
120	0\\
121	0\\
122	0\\
123	0\\
124	0\\
125	0\\
126	0\\
127	0\\
128	0\\
129	0\\
130	0\\
131	0\\
132	0\\
133	0\\
134	0\\
135	0\\
136	0\\
137	0\\
138	0\\
139	0\\
140	0\\
141	0\\
142	0\\
143	0\\
144	0\\
145	0\\
146	0\\
147	0\\
148	0\\
149	0\\
150	0\\
151	0\\
152	0\\
153	0\\
154	0\\
155	0\\
156	0\\
157	0\\
158	0\\
159	0\\
160	0\\
161	0\\
162	0\\
163	0\\
164	0\\
165	0\\
166	0\\
167	0\\
168	0\\
169	0\\
170	0\\
171	0\\
172	0\\
173	0\\
174	0\\
175	0\\
176	0\\
177	0\\
178	0\\
179	0\\
180	0\\
181	0\\
182	0\\
183	0\\
184	0\\
185	0\\
186	0\\
187	0\\
188	0\\
189	0\\
190	0\\
191	0\\
192	0\\
193	0\\
194	0\\
195	0\\
196	0\\
197	0\\
198	0\\
199	0\\
200	0\\
201	0\\
202	0\\
203	0\\
204	0\\
205	0\\
206	0\\
207	0\\
208	0\\
209	0\\
210	0\\
211	0\\
212	0\\
213	0\\
214	0\\
215	0\\
216	0\\
217	0\\
218	0\\
219	0\\
220	0\\
221	0\\
222	0\\
223	0\\
224	0\\
225	0\\
226	0\\
227	0\\
228	0\\
229	0\\
230	0\\
231	0\\
232	0\\
233	0\\
234	0\\
235	0\\
236	0\\
237	0\\
238	0\\
239	0\\
240	0\\
241	0\\
242	0\\
243	0\\
244	0\\
245	0\\
246	0\\
247	0\\
248	0\\
249	0\\
250	0\\
251	0\\
252	0\\
253	0\\
254	0\\
255	0\\
256	0\\
257	0\\
258	0\\
259	0\\
260	0\\
261	0\\
262	0\\
263	0\\
264	0\\
265	0\\
266	0\\
267	0\\
268	0\\
269	0\\
270	0\\
271	0\\
272	0\\
273	0\\
274	0\\
275	0\\
276	0\\
277	0\\
278	0\\
279	0\\
280	0\\
281	0\\
282	0\\
283	0\\
284	0\\
285	0\\
286	0\\
287	0\\
288	0\\
289	0\\
290	0\\
291	0\\
292	0\\
293	0\\
294	0\\
295	0\\
296	0\\
297	0\\
298	0\\
299	0\\
300	0\\
301	0\\
302	0\\
303	0\\
304	0\\
305	0\\
306	0\\
307	0\\
308	0\\
309	0\\
310	0\\
311	0\\
312	0\\
313	0\\
314	0\\
315	0\\
316	0\\
317	0\\
318	0\\
319	0\\
320	0\\
321	0\\
322	0\\
323	0\\
324	0\\
325	0\\
326	0\\
327	0\\
328	0\\
329	0\\
330	0\\
331	0\\
332	0\\
333	0\\
334	0\\
335	0\\
336	0\\
337	0\\
338	0\\
339	0\\
340	0\\
341	0\\
342	0\\
343	0\\
344	0\\
345	0\\
346	0\\
347	0\\
348	0\\
349	0\\
350	0\\
351	0\\
352	0\\
353	0\\
354	0\\
355	0\\
356	0\\
357	0\\
358	0\\
359	0\\
360	0\\
361	0\\
362	0\\
363	0\\
364	0\\
365	0\\
366	0\\
367	0\\
368	0\\
369	0\\
370	0\\
371	0\\
372	0\\
373	0\\
374	0\\
375	0\\
376	0\\
377	0\\
378	0\\
379	0\\
380	0\\
381	0\\
382	0\\
383	0\\
384	0\\
385	0\\
386	0\\
387	0\\
388	0\\
389	0\\
390	0\\
391	0\\
392	0\\
393	0\\
394	0\\
395	0\\
396	0\\
397	0\\
398	0\\
399	0\\
400	0\\
401	0\\
402	0\\
403	0\\
404	0\\
405	0\\
406	0\\
407	0\\
408	0\\
409	0\\
410	0\\
411	0\\
412	0\\
413	0\\
414	0\\
415	0\\
416	0\\
417	0\\
418	0\\
419	0\\
420	0\\
421	0\\
422	0\\
423	0\\
424	0\\
425	0\\
426	0\\
427	0\\
428	0\\
429	0\\
430	0\\
431	0\\
432	0\\
433	0\\
434	0\\
435	0\\
436	0\\
437	0\\
438	0\\
439	0\\
440	0\\
441	0\\
442	0\\
443	0\\
444	0\\
445	0\\
446	0\\
447	0\\
448	0\\
449	0\\
450	0\\
451	0\\
452	0\\
453	0\\
454	0\\
455	0\\
456	0\\
457	0\\
458	0\\
459	0\\
460	0\\
461	0\\
462	0\\
463	0\\
464	0\\
465	0\\
466	0\\
467	0\\
468	0\\
469	0\\
470	0\\
471	0\\
472	0\\
473	0\\
474	0\\
475	0\\
476	0\\
477	0\\
478	0\\
479	0\\
480	0\\
481	0\\
482	0\\
483	0\\
484	0\\
485	0\\
486	0\\
487	0\\
488	0\\
489	0\\
490	0\\
491	0\\
492	0\\
493	0\\
494	0\\
495	0\\
496	0\\
497	0\\
498	0\\
499	0\\
500	0\\
501	0\\
502	0\\
503	0\\
504	0\\
505	0\\
506	0\\
507	0\\
508	0\\
509	0\\
510	0\\
511	0\\
512	0\\
513	0\\
514	0\\
515	0\\
516	0\\
517	0\\
518	0\\
519	0\\
520	0\\
521	0\\
522	0\\
523	0\\
524	0\\
525	0\\
526	0\\
527	0\\
528	0\\
529	0\\
530	0\\
531	0\\
532	0\\
533	0\\
534	0\\
535	0\\
536	0\\
537	0\\
538	0\\
539	0\\
540	0\\
541	0\\
542	0\\
543	0\\
544	0\\
545	0\\
546	0\\
547	0\\
548	0\\
549	0\\
550	0\\
551	0\\
552	0\\
553	0\\
554	0\\
555	0\\
556	0\\
557	0\\
558	0\\
559	0\\
560	0\\
561	0\\
562	0\\
563	0\\
564	0\\
565	0\\
566	0\\
567	0\\
568	0\\
569	0\\
570	0\\
571	0\\
572	0\\
573	0\\
574	0\\
575	0\\
576	0\\
577	0\\
578	0\\
579	0\\
580	0\\
581	0\\
582	0\\
583	0\\
584	0\\
585	0\\
586	0\\
587	0\\
588	0\\
589	0\\
590	0\\
591	0\\
592	0\\
593	0\\
594	0\\
595	0\\
596	0\\
597	0\\
598	0\\
599	0\\
600	0\\
};
\addplot [color=black!60!mycolor21,solid,forget plot]
  table[row sep=crcr]{%
1	0\\
2	0\\
3	0\\
4	0\\
5	0\\
6	0\\
7	0\\
8	0\\
9	0\\
10	0\\
11	0\\
12	0\\
13	0\\
14	0\\
15	0\\
16	0\\
17	0\\
18	0\\
19	0\\
20	0\\
21	0\\
22	0\\
23	0\\
24	0\\
25	0\\
26	0\\
27	0\\
28	0\\
29	0\\
30	0\\
31	0\\
32	0\\
33	0\\
34	0\\
35	0\\
36	0\\
37	0\\
38	0\\
39	0\\
40	0\\
41	0\\
42	0\\
43	0\\
44	0\\
45	0\\
46	0\\
47	0\\
48	0\\
49	0\\
50	0\\
51	0\\
52	0\\
53	0\\
54	0\\
55	0\\
56	0\\
57	0\\
58	0\\
59	0\\
60	0\\
61	0\\
62	0\\
63	0\\
64	0\\
65	0\\
66	0\\
67	0\\
68	0\\
69	0\\
70	0\\
71	0\\
72	0\\
73	0\\
74	0\\
75	0\\
76	0\\
77	0\\
78	0\\
79	0\\
80	0\\
81	0\\
82	0\\
83	0\\
84	0\\
85	0\\
86	0\\
87	0\\
88	0\\
89	0\\
90	0\\
91	0\\
92	0\\
93	0\\
94	0\\
95	0\\
96	0\\
97	0\\
98	0\\
99	0\\
100	0\\
101	0\\
102	0\\
103	0\\
104	0\\
105	0\\
106	0\\
107	0\\
108	0\\
109	0\\
110	0\\
111	0\\
112	0\\
113	0\\
114	0\\
115	0\\
116	0\\
117	0\\
118	0\\
119	0\\
120	0\\
121	0\\
122	0\\
123	0\\
124	0\\
125	0\\
126	0\\
127	0\\
128	0\\
129	0\\
130	0\\
131	0\\
132	0\\
133	0\\
134	0\\
135	0\\
136	0\\
137	0\\
138	0\\
139	0\\
140	0\\
141	0\\
142	0\\
143	0\\
144	0\\
145	0\\
146	0\\
147	0\\
148	0\\
149	0\\
150	0\\
151	0\\
152	0\\
153	0\\
154	0\\
155	0\\
156	0\\
157	0\\
158	0\\
159	0\\
160	0\\
161	0\\
162	0\\
163	0\\
164	0\\
165	0\\
166	0\\
167	0\\
168	0\\
169	0\\
170	0\\
171	0\\
172	0\\
173	0\\
174	0\\
175	0\\
176	0\\
177	0\\
178	0\\
179	0\\
180	0\\
181	0\\
182	0\\
183	0\\
184	0\\
185	0\\
186	0\\
187	0\\
188	0\\
189	0\\
190	0\\
191	0\\
192	0\\
193	0\\
194	0\\
195	0\\
196	0\\
197	0\\
198	0\\
199	0\\
200	0\\
201	0\\
202	0\\
203	0\\
204	0\\
205	0\\
206	0\\
207	0\\
208	0\\
209	0\\
210	0\\
211	0\\
212	0\\
213	0\\
214	0\\
215	0\\
216	0\\
217	0\\
218	0\\
219	0\\
220	0\\
221	0\\
222	0\\
223	0\\
224	0\\
225	0\\
226	0\\
227	0\\
228	0\\
229	0\\
230	0\\
231	0\\
232	0\\
233	0\\
234	0\\
235	0\\
236	0\\
237	0\\
238	0\\
239	0\\
240	0\\
241	0\\
242	0\\
243	0\\
244	0\\
245	0\\
246	0\\
247	0\\
248	0\\
249	0\\
250	0\\
251	0\\
252	0\\
253	0\\
254	0\\
255	0\\
256	0\\
257	0\\
258	0\\
259	0\\
260	0\\
261	0\\
262	0\\
263	0\\
264	0\\
265	0\\
266	0\\
267	0\\
268	0\\
269	0\\
270	0\\
271	0\\
272	0\\
273	0\\
274	0\\
275	0\\
276	0\\
277	0\\
278	0\\
279	0\\
280	0\\
281	0\\
282	0\\
283	0\\
284	0\\
285	0\\
286	0\\
287	0\\
288	0\\
289	0\\
290	0\\
291	0\\
292	0\\
293	0\\
294	0\\
295	0\\
296	0\\
297	0\\
298	0\\
299	0\\
300	0\\
301	0\\
302	0\\
303	0\\
304	0\\
305	0\\
306	0\\
307	0\\
308	0\\
309	0\\
310	0\\
311	0\\
312	0\\
313	0\\
314	0\\
315	0\\
316	0\\
317	0\\
318	0\\
319	0\\
320	0\\
321	0\\
322	0\\
323	0\\
324	0\\
325	0\\
326	0\\
327	0\\
328	0\\
329	0\\
330	0\\
331	0\\
332	0\\
333	0\\
334	0\\
335	0\\
336	0\\
337	0\\
338	0\\
339	0\\
340	0\\
341	0\\
342	0\\
343	0\\
344	0\\
345	0\\
346	0\\
347	0\\
348	0\\
349	0\\
350	0\\
351	0\\
352	0\\
353	0\\
354	0\\
355	0\\
356	0\\
357	0\\
358	0\\
359	0\\
360	0\\
361	0\\
362	0\\
363	0\\
364	0\\
365	0\\
366	0\\
367	0\\
368	0\\
369	0\\
370	0\\
371	0\\
372	0\\
373	0\\
374	0\\
375	0\\
376	0\\
377	0\\
378	0\\
379	0\\
380	0\\
381	0\\
382	0\\
383	0\\
384	0\\
385	0\\
386	0\\
387	0\\
388	0\\
389	0\\
390	0\\
391	0\\
392	0\\
393	0\\
394	0\\
395	0\\
396	0\\
397	0\\
398	0\\
399	0\\
400	0\\
401	0\\
402	0\\
403	0\\
404	0\\
405	0\\
406	0\\
407	0\\
408	0\\
409	0\\
410	0\\
411	0\\
412	0\\
413	0\\
414	0\\
415	0\\
416	0\\
417	0\\
418	0\\
419	0\\
420	0\\
421	0\\
422	0\\
423	0\\
424	0\\
425	0\\
426	0\\
427	0\\
428	0\\
429	0\\
430	0\\
431	0\\
432	0\\
433	0\\
434	0\\
435	0\\
436	0\\
437	0\\
438	0\\
439	0\\
440	0\\
441	0\\
442	0\\
443	0\\
444	0\\
445	0\\
446	0\\
447	0\\
448	0\\
449	0\\
450	0\\
451	0\\
452	0\\
453	0\\
454	0\\
455	0\\
456	0\\
457	0\\
458	0\\
459	0\\
460	0\\
461	0\\
462	0\\
463	0\\
464	0\\
465	0\\
466	0\\
467	0\\
468	0\\
469	0\\
470	0\\
471	0\\
472	0\\
473	0\\
474	0\\
475	0\\
476	0\\
477	0\\
478	0\\
479	0\\
480	0\\
481	0\\
482	0\\
483	0\\
484	0\\
485	0\\
486	0\\
487	0\\
488	0\\
489	0\\
490	0\\
491	0\\
492	0\\
493	0\\
494	0\\
495	0\\
496	0\\
497	0\\
498	0\\
499	0\\
500	0\\
501	0\\
502	0\\
503	0\\
504	0\\
505	0\\
506	0\\
507	0\\
508	0\\
509	0\\
510	0\\
511	0\\
512	0\\
513	0\\
514	0\\
515	0\\
516	0\\
517	0\\
518	0\\
519	0\\
520	0\\
521	0\\
522	0\\
523	0\\
524	0\\
525	0\\
526	0\\
527	0\\
528	0\\
529	0\\
530	0\\
531	0\\
532	0\\
533	0\\
534	0\\
535	0\\
536	0\\
537	0\\
538	0\\
539	0\\
540	0\\
541	0\\
542	0\\
543	0\\
544	0\\
545	0\\
546	0\\
547	0\\
548	0\\
549	0\\
550	0\\
551	0\\
552	0\\
553	0\\
554	0\\
555	0\\
556	0\\
557	0\\
558	0\\
559	0\\
560	0\\
561	0\\
562	0\\
563	0\\
564	0\\
565	0\\
566	0\\
567	0\\
568	0\\
569	0\\
570	0\\
571	0\\
572	0\\
573	0\\
574	0\\
575	0\\
576	0\\
577	0\\
578	0\\
579	0\\
580	0\\
581	0\\
582	0\\
583	0\\
584	0\\
585	0\\
586	0\\
587	0\\
588	0\\
589	0\\
590	0\\
591	0\\
592	0\\
593	0\\
594	0\\
595	0\\
596	0\\
597	0\\
598	0\\
599	0\\
600	0\\
};
\addplot [color=black!80!mycolor21,solid,forget plot]
  table[row sep=crcr]{%
1	0\\
2	0\\
3	0\\
4	0\\
5	0\\
6	0\\
7	0\\
8	0\\
9	0\\
10	0\\
11	0\\
12	0\\
13	0\\
14	0\\
15	0\\
16	0\\
17	0\\
18	0\\
19	0\\
20	0\\
21	0\\
22	0\\
23	0\\
24	0\\
25	0\\
26	0\\
27	0\\
28	0\\
29	0\\
30	0\\
31	0\\
32	0\\
33	0\\
34	0\\
35	0\\
36	0\\
37	0\\
38	0\\
39	0\\
40	0\\
41	0\\
42	0\\
43	0\\
44	0\\
45	0\\
46	0\\
47	0\\
48	0\\
49	0\\
50	0\\
51	0\\
52	0\\
53	0\\
54	0\\
55	0\\
56	0\\
57	0\\
58	0\\
59	0\\
60	0\\
61	0\\
62	0\\
63	0\\
64	0\\
65	0\\
66	0\\
67	0\\
68	0\\
69	0\\
70	0\\
71	0\\
72	0\\
73	0\\
74	0\\
75	0\\
76	0\\
77	0\\
78	0\\
79	0\\
80	0\\
81	0\\
82	0\\
83	0\\
84	0\\
85	0\\
86	0\\
87	0\\
88	0\\
89	0\\
90	0\\
91	0\\
92	0\\
93	0\\
94	0\\
95	0\\
96	0\\
97	0\\
98	0\\
99	0\\
100	0\\
101	0\\
102	0\\
103	0\\
104	0\\
105	0\\
106	0\\
107	0\\
108	0\\
109	0\\
110	0\\
111	0\\
112	0\\
113	0\\
114	0\\
115	0\\
116	0\\
117	0\\
118	0\\
119	0\\
120	0\\
121	0\\
122	0\\
123	0\\
124	0\\
125	0\\
126	0\\
127	0\\
128	0\\
129	0\\
130	0\\
131	0\\
132	0\\
133	0\\
134	0\\
135	0\\
136	0\\
137	0\\
138	0\\
139	0\\
140	0\\
141	0\\
142	0\\
143	0\\
144	0\\
145	0\\
146	0\\
147	0\\
148	0\\
149	0\\
150	0\\
151	0\\
152	0\\
153	0\\
154	0\\
155	0\\
156	0\\
157	0\\
158	0\\
159	0\\
160	0\\
161	0\\
162	0\\
163	0\\
164	0\\
165	0\\
166	0\\
167	0\\
168	0\\
169	0\\
170	0\\
171	0\\
172	0\\
173	0\\
174	0\\
175	0\\
176	0\\
177	0\\
178	0\\
179	0\\
180	0\\
181	0\\
182	0\\
183	0\\
184	0\\
185	0\\
186	0\\
187	0\\
188	0\\
189	0\\
190	0\\
191	0\\
192	0\\
193	0\\
194	0\\
195	0\\
196	0\\
197	0\\
198	0\\
199	0\\
200	0\\
201	0\\
202	0\\
203	0\\
204	0\\
205	0\\
206	0\\
207	0\\
208	0\\
209	0\\
210	0\\
211	0\\
212	0\\
213	0\\
214	0\\
215	0\\
216	0\\
217	0\\
218	0\\
219	0\\
220	0\\
221	0\\
222	0\\
223	0\\
224	0\\
225	0\\
226	0\\
227	0\\
228	0\\
229	0\\
230	0\\
231	0\\
232	0\\
233	0\\
234	0\\
235	0\\
236	0\\
237	0\\
238	0\\
239	0\\
240	0\\
241	0\\
242	0\\
243	0\\
244	0\\
245	0\\
246	0\\
247	0\\
248	0\\
249	0\\
250	0\\
251	0\\
252	0\\
253	0\\
254	0\\
255	0\\
256	0\\
257	0\\
258	0\\
259	0\\
260	0\\
261	0\\
262	0\\
263	0\\
264	0\\
265	0\\
266	0\\
267	0\\
268	0\\
269	0\\
270	0\\
271	0\\
272	0\\
273	0\\
274	0\\
275	0\\
276	0\\
277	0\\
278	0\\
279	0\\
280	0\\
281	0\\
282	0\\
283	0\\
284	0\\
285	0\\
286	0\\
287	0\\
288	0\\
289	0\\
290	0\\
291	0\\
292	0\\
293	0\\
294	0\\
295	0\\
296	0\\
297	0\\
298	0\\
299	0\\
300	0\\
301	0\\
302	0\\
303	0\\
304	0\\
305	0\\
306	0\\
307	0\\
308	0\\
309	0\\
310	0\\
311	0\\
312	0\\
313	0\\
314	0\\
315	0\\
316	0\\
317	0\\
318	0\\
319	0\\
320	0\\
321	0\\
322	0\\
323	0\\
324	0\\
325	0\\
326	0\\
327	0\\
328	0\\
329	0\\
330	0\\
331	0\\
332	0\\
333	0\\
334	0\\
335	0\\
336	0\\
337	0\\
338	0\\
339	0\\
340	0\\
341	0\\
342	0\\
343	0\\
344	0\\
345	0\\
346	0\\
347	0\\
348	0\\
349	0\\
350	0\\
351	0\\
352	0\\
353	0\\
354	0\\
355	0\\
356	0\\
357	0\\
358	0\\
359	0\\
360	0\\
361	0\\
362	0\\
363	0\\
364	0\\
365	0\\
366	0\\
367	0\\
368	0\\
369	0\\
370	0\\
371	0\\
372	0\\
373	0\\
374	0\\
375	0\\
376	0\\
377	0\\
378	0\\
379	0\\
380	0\\
381	0\\
382	0\\
383	0\\
384	0\\
385	0\\
386	0\\
387	0\\
388	0\\
389	0\\
390	0\\
391	0\\
392	0\\
393	0\\
394	0\\
395	0\\
396	0\\
397	0\\
398	0\\
399	0\\
400	0\\
401	0\\
402	0\\
403	0\\
404	0\\
405	0\\
406	0\\
407	0\\
408	0\\
409	0\\
410	0\\
411	0\\
412	0\\
413	0\\
414	0\\
415	0\\
416	0\\
417	0\\
418	0\\
419	0\\
420	0\\
421	0\\
422	0\\
423	0\\
424	0\\
425	0\\
426	0\\
427	0\\
428	0\\
429	0\\
430	0\\
431	0\\
432	0\\
433	0\\
434	0\\
435	0\\
436	0\\
437	0\\
438	0\\
439	0\\
440	0\\
441	0\\
442	0\\
443	0\\
444	0\\
445	0\\
446	0\\
447	0\\
448	0\\
449	0\\
450	0\\
451	0\\
452	0\\
453	0\\
454	0\\
455	0\\
456	0\\
457	0\\
458	0\\
459	0\\
460	0\\
461	0\\
462	0\\
463	0\\
464	0\\
465	0\\
466	0\\
467	0\\
468	0\\
469	0\\
470	0\\
471	0\\
472	0\\
473	0\\
474	0\\
475	0\\
476	0\\
477	0\\
478	0\\
479	0\\
480	0\\
481	0\\
482	0\\
483	0\\
484	0\\
485	0\\
486	0\\
487	0\\
488	0\\
489	0\\
490	0\\
491	0\\
492	0\\
493	0\\
494	0\\
495	0\\
496	0\\
497	0\\
498	0\\
499	0\\
500	0\\
501	0\\
502	0\\
503	0\\
504	0\\
505	0\\
506	0\\
507	0\\
508	0\\
509	0\\
510	0\\
511	0\\
512	0\\
513	0\\
514	0\\
515	0\\
516	0\\
517	0\\
518	0\\
519	0\\
520	0\\
521	0\\
522	0\\
523	0\\
524	0\\
525	0\\
526	0\\
527	0\\
528	0\\
529	0\\
530	0\\
531	0\\
532	0\\
533	0\\
534	0\\
535	0\\
536	0\\
537	0\\
538	0\\
539	0\\
540	0\\
541	0\\
542	0\\
543	0\\
544	0\\
545	0\\
546	0\\
547	0\\
548	0\\
549	0\\
550	0\\
551	0\\
552	0\\
553	0\\
554	0\\
555	0\\
556	0\\
557	0\\
558	0\\
559	0\\
560	0\\
561	0\\
562	0\\
563	0\\
564	0\\
565	0\\
566	0\\
567	0\\
568	0\\
569	0\\
570	0\\
571	0\\
572	0\\
573	0\\
574	0\\
575	0\\
576	0\\
577	0\\
578	0\\
579	0\\
580	0\\
581	0\\
582	0\\
583	0\\
584	0\\
585	0\\
586	0\\
587	0\\
588	0\\
589	0\\
590	0\\
591	0\\
592	0\\
593	0\\
594	0\\
595	0\\
596	0\\
597	0\\
598	0\\
599	0\\
600	0\\
};
\addplot [color=black,solid,forget plot]
  table[row sep=crcr]{%
1	0\\
2	0\\
3	0\\
4	0\\
5	0\\
6	0\\
7	0\\
8	0\\
9	0\\
10	0\\
11	0\\
12	0\\
13	0\\
14	0\\
15	0\\
16	0\\
17	0\\
18	0\\
19	0\\
20	0\\
21	0\\
22	0\\
23	0\\
24	0\\
25	0\\
26	0\\
27	0\\
28	0\\
29	0\\
30	0\\
31	0\\
32	0\\
33	0\\
34	0\\
35	0\\
36	0\\
37	0\\
38	0\\
39	0\\
40	0\\
41	0\\
42	0\\
43	0\\
44	0\\
45	0\\
46	0\\
47	0\\
48	0\\
49	0\\
50	0\\
51	0\\
52	0\\
53	0\\
54	0\\
55	0\\
56	0\\
57	0\\
58	0\\
59	0\\
60	0\\
61	0\\
62	0\\
63	0\\
64	0\\
65	0\\
66	0\\
67	0\\
68	0\\
69	0\\
70	0\\
71	0\\
72	0\\
73	0\\
74	0\\
75	0\\
76	0\\
77	0\\
78	0\\
79	0\\
80	0\\
81	0\\
82	0\\
83	0\\
84	0\\
85	0\\
86	0\\
87	0\\
88	0\\
89	0\\
90	0\\
91	0\\
92	0\\
93	0\\
94	0\\
95	0\\
96	0\\
97	0\\
98	0\\
99	0\\
100	0\\
101	0\\
102	0\\
103	0\\
104	0\\
105	0\\
106	0\\
107	0\\
108	0\\
109	0\\
110	0\\
111	0\\
112	0\\
113	0\\
114	0\\
115	0\\
116	0\\
117	0\\
118	0\\
119	0\\
120	0\\
121	0\\
122	0\\
123	0\\
124	0\\
125	0\\
126	0\\
127	0\\
128	0\\
129	0\\
130	0\\
131	0\\
132	0\\
133	0\\
134	0\\
135	0\\
136	0\\
137	0\\
138	0\\
139	0\\
140	0\\
141	0\\
142	0\\
143	0\\
144	0\\
145	0\\
146	0\\
147	0\\
148	0\\
149	0\\
150	0\\
151	0\\
152	0\\
153	0\\
154	0\\
155	0\\
156	0\\
157	0\\
158	0\\
159	0\\
160	0\\
161	0\\
162	0\\
163	0\\
164	0\\
165	0\\
166	0\\
167	0\\
168	0\\
169	0\\
170	0\\
171	0\\
172	0\\
173	0\\
174	0\\
175	0\\
176	0\\
177	0\\
178	0\\
179	0\\
180	0\\
181	0\\
182	0\\
183	0\\
184	0\\
185	0\\
186	0\\
187	0\\
188	0\\
189	0\\
190	0\\
191	0\\
192	0\\
193	0\\
194	0\\
195	0\\
196	0\\
197	0\\
198	0\\
199	0\\
200	0\\
201	0\\
202	0\\
203	0\\
204	0\\
205	0\\
206	0\\
207	0\\
208	0\\
209	0\\
210	0\\
211	0\\
212	0\\
213	0\\
214	0\\
215	0\\
216	0\\
217	0\\
218	0\\
219	0\\
220	0\\
221	0\\
222	0\\
223	0\\
224	0\\
225	0\\
226	0\\
227	0\\
228	0\\
229	0\\
230	0\\
231	0\\
232	0\\
233	0\\
234	0\\
235	0\\
236	0\\
237	0\\
238	0\\
239	0\\
240	0\\
241	0\\
242	0\\
243	0\\
244	0\\
245	0\\
246	0\\
247	0\\
248	0\\
249	0\\
250	0\\
251	0\\
252	0\\
253	0\\
254	0\\
255	0\\
256	0\\
257	0\\
258	0\\
259	0\\
260	0\\
261	0\\
262	0\\
263	0\\
264	0\\
265	0\\
266	0\\
267	0\\
268	0\\
269	0\\
270	0\\
271	0\\
272	0\\
273	0\\
274	0\\
275	0\\
276	0\\
277	0\\
278	0\\
279	0\\
280	0\\
281	0\\
282	0\\
283	0\\
284	0\\
285	0\\
286	0\\
287	0\\
288	0\\
289	0\\
290	0\\
291	0\\
292	0\\
293	0\\
294	0\\
295	0\\
296	0\\
297	0\\
298	0\\
299	0\\
300	0\\
301	0\\
302	0\\
303	0\\
304	0\\
305	0\\
306	0\\
307	0\\
308	0\\
309	0\\
310	0\\
311	0\\
312	0\\
313	0\\
314	0\\
315	0\\
316	0\\
317	0\\
318	0\\
319	0\\
320	0\\
321	0\\
322	0\\
323	0\\
324	0\\
325	0\\
326	0\\
327	0\\
328	0\\
329	0\\
330	0\\
331	0\\
332	0\\
333	0\\
334	0\\
335	0\\
336	0\\
337	0\\
338	0\\
339	0\\
340	0\\
341	0\\
342	0\\
343	0\\
344	0\\
345	0\\
346	0\\
347	0\\
348	0\\
349	0\\
350	0\\
351	0\\
352	0\\
353	0\\
354	0\\
355	0\\
356	0\\
357	0\\
358	0\\
359	0\\
360	0\\
361	0\\
362	0\\
363	0\\
364	0\\
365	0\\
366	0\\
367	0\\
368	0\\
369	0\\
370	0\\
371	0\\
372	0\\
373	0\\
374	0\\
375	0\\
376	0\\
377	0\\
378	0\\
379	0\\
380	0\\
381	0\\
382	0\\
383	0\\
384	0\\
385	0\\
386	0\\
387	0\\
388	0\\
389	0\\
390	0\\
391	0\\
392	0\\
393	0\\
394	0\\
395	0\\
396	0\\
397	0\\
398	0\\
399	0\\
400	0\\
401	0\\
402	0\\
403	0\\
404	0\\
405	0\\
406	0\\
407	0\\
408	0\\
409	0\\
410	0\\
411	0\\
412	0\\
413	0\\
414	0\\
415	0\\
416	0\\
417	0\\
418	0\\
419	0\\
420	0\\
421	0\\
422	0\\
423	0\\
424	0\\
425	0\\
426	0\\
427	0\\
428	0\\
429	0\\
430	0\\
431	0\\
432	0\\
433	0\\
434	0\\
435	0\\
436	0\\
437	0\\
438	0\\
439	0\\
440	0\\
441	0\\
442	0\\
443	0\\
444	0\\
445	0\\
446	0\\
447	0\\
448	0\\
449	0\\
450	0\\
451	0\\
452	0\\
453	0\\
454	0\\
455	0\\
456	0\\
457	0\\
458	0\\
459	0\\
460	0\\
461	0\\
462	0\\
463	0\\
464	0\\
465	0\\
466	0\\
467	0\\
468	0\\
469	0\\
470	0\\
471	0\\
472	0\\
473	0\\
474	0\\
475	0\\
476	0\\
477	0\\
478	0\\
479	0\\
480	0\\
481	0\\
482	0\\
483	0\\
484	0\\
485	0\\
486	0\\
487	0\\
488	0\\
489	0\\
490	0\\
491	0\\
492	0\\
493	0\\
494	0\\
495	0\\
496	0\\
497	0\\
498	0\\
499	0\\
500	0\\
501	0\\
502	0\\
503	0\\
504	0\\
505	0\\
506	0\\
507	0\\
508	0\\
509	0\\
510	0\\
511	0\\
512	0\\
513	0\\
514	0\\
515	0\\
516	0\\
517	0\\
518	0\\
519	0\\
520	0\\
521	0\\
522	0\\
523	0\\
524	0\\
525	0\\
526	0\\
527	0\\
528	0\\
529	0\\
530	0\\
531	0\\
532	0\\
533	0\\
534	0\\
535	0\\
536	0\\
537	0\\
538	0\\
539	0\\
540	0\\
541	0\\
542	0\\
543	0\\
544	0\\
545	0\\
546	0\\
547	0\\
548	0\\
549	0\\
550	0\\
551	0\\
552	0\\
553	0\\
554	0\\
555	0\\
556	0\\
557	0\\
558	0\\
559	0\\
560	0\\
561	0\\
562	0\\
563	0\\
564	0\\
565	0\\
566	0\\
567	0\\
568	0\\
569	0\\
570	0\\
571	0\\
572	0\\
573	0\\
574	0\\
575	0\\
576	0\\
577	0\\
578	0\\
579	0\\
580	0\\
581	0\\
582	0\\
583	0\\
584	0\\
585	0\\
586	0\\
587	0\\
588	0\\
589	0\\
590	0\\
591	0\\
592	0\\
593	0\\
594	0\\
595	0\\
596	0\\
597	0\\
598	0\\
599	0\\
600	0\\
};
\end{axis}
\end{tikzpicture}% 
  \caption{Discrete Time}
\end{subfigure}\\
\vspace{1cm}
\begin{subfigure}{.45\linewidth}
  \centering
  \setlength\figureheight{\linewidth} 
  \setlength\figurewidth{\linewidth}
  \tikzsetnextfilename{dp_colorbar/dm_cts_nFPC_z1}
  % This file was created by matlab2tikz.
%
%The latest updates can be retrieved from
%  http://www.mathworks.com/matlabcentral/fileexchange/22022-matlab2tikz-matlab2tikz
%where you can also make suggestions and rate matlab2tikz.
%
\definecolor{mycolor1}{rgb}{0.00000,1.00000,0.14286}%
\definecolor{mycolor2}{rgb}{0.00000,1.00000,0.28571}%
\definecolor{mycolor3}{rgb}{0.00000,1.00000,0.42857}%
\definecolor{mycolor4}{rgb}{0.00000,1.00000,0.57143}%
\definecolor{mycolor5}{rgb}{0.00000,1.00000,0.71429}%
\definecolor{mycolor6}{rgb}{0.00000,1.00000,0.85714}%
\definecolor{mycolor7}{rgb}{0.00000,1.00000,1.00000}%
\definecolor{mycolor8}{rgb}{0.00000,0.87500,1.00000}%
\definecolor{mycolor9}{rgb}{0.00000,0.62500,1.00000}%
\definecolor{mycolor10}{rgb}{0.12500,0.00000,1.00000}%
\definecolor{mycolor11}{rgb}{0.25000,0.00000,1.00000}%
\definecolor{mycolor12}{rgb}{0.37500,0.00000,1.00000}%
\definecolor{mycolor13}{rgb}{0.50000,0.00000,1.00000}%
\definecolor{mycolor14}{rgb}{0.62500,0.00000,1.00000}%
\definecolor{mycolor15}{rgb}{0.75000,0.00000,1.00000}%
\definecolor{mycolor16}{rgb}{0.87500,0.00000,1.00000}%
\definecolor{mycolor17}{rgb}{1.00000,0.00000,1.00000}%
\definecolor{mycolor18}{rgb}{1.00000,0.00000,0.87500}%
\definecolor{mycolor19}{rgb}{1.00000,0.00000,0.62500}%
\definecolor{mycolor20}{rgb}{0.85714,0.00000,0.00000}%
\definecolor{mycolor21}{rgb}{0.71429,0.00000,0.00000}%
%
\begin{tikzpicture}

\begin{axis}[%
width=4.1in,
height=3.803in,
at={(0.809in,0.513in)},
scale only axis,
point meta min=0,
point meta max=1,
every outer x axis line/.append style={black},
every x tick label/.append style={font=\color{black}},
xmin=0,
xmax=600,
every outer y axis line/.append style={black},
every y tick label/.append style={font=\color{black}},
ymin=0,
ymax=0.012,
axis background/.style={fill=white},
axis x line*=bottom,
axis y line*=left,
colormap={mymap}{[1pt] rgb(0pt)=(0,1,0); rgb(7pt)=(0,1,1); rgb(15pt)=(0,0,1); rgb(23pt)=(1,0,1); rgb(31pt)=(1,0,0); rgb(38pt)=(0,0,0)},
colorbar,
colorbar style={separate axis lines,every outer x axis line/.append style={black},every x tick label/.append style={font=\color{black}},every outer y axis line/.append style={black},every y tick label/.append style={font=\color{black}},yticklabels={{-19},{-17},{-15},{-13},{-11},{-9},{-7},{-5},{-3},{-1},{1},{3},{5},{7},{9},{11},{13},{15},{17},{19}}}
]
\addplot [color=green,solid,forget plot]
  table[row sep=crcr]{%
0.01	0.01\\
1.01	0.01\\
2.01	0.01\\
3.01	0.01\\
4.01	0.01\\
5.01	0.01\\
6.01	0.01\\
7.01	0.01\\
8.01	0.01\\
9.01	0.01\\
10.01	0.01\\
11.01	0.01\\
12.01	0.01\\
13.01	0.01\\
14.01	0.01\\
15.01	0.01\\
16.01	0.01\\
17.01	0.01\\
18.01	0.01\\
19.01	0.01\\
20.01	0.01\\
21.01	0.01\\
22.01	0.01\\
23.01	0.01\\
24.01	0.01\\
25.01	0.01\\
26.01	0.01\\
27.01	0.01\\
28.01	0.01\\
29.01	0.01\\
30.01	0.01\\
31.01	0.01\\
32.01	0.01\\
33.01	0.01\\
34.01	0.01\\
35.01	0.01\\
36.01	0.01\\
37.01	0.01\\
38.01	0.01\\
39.01	0.01\\
40.01	0.01\\
41.01	0.01\\
42.01	0.01\\
43.01	0.01\\
44.01	0.01\\
45.01	0.01\\
46.01	0.01\\
47.01	0.01\\
48.01	0.01\\
49.01	0.01\\
50.01	0.01\\
51.01	0.01\\
52.01	0.01\\
53.01	0.01\\
54.01	0.01\\
55.01	0.01\\
56.01	0.01\\
57.01	0.01\\
58.01	0.01\\
59.01	0.01\\
60.01	0.01\\
61.01	0.01\\
62.01	0.01\\
63.01	0.01\\
64.01	0.01\\
65.01	0.01\\
66.01	0.01\\
67.01	0.01\\
68.01	0.01\\
69.01	0.01\\
70.01	0.01\\
71.01	0.01\\
72.01	0.01\\
73.01	0.01\\
74.01	0.01\\
75.01	0.01\\
76.01	0.01\\
77.01	0.01\\
78.01	0.01\\
79.01	0.01\\
80.01	0.01\\
81.01	0.01\\
82.01	0.01\\
83.01	0.01\\
84.01	0.01\\
85.01	0.01\\
86.01	0.01\\
87.01	0.01\\
88.01	0.01\\
89.01	0.01\\
90.01	0.01\\
91.01	0.01\\
92.01	0.01\\
93.01	0.01\\
94.01	0.01\\
95.01	0.01\\
96.01	0.01\\
97.01	0.01\\
98.01	0.01\\
99.01	0.01\\
100.01	0.01\\
101.01	0.01\\
102.01	0.01\\
103.01	0.01\\
104.01	0.01\\
105.01	0.01\\
106.01	0.01\\
107.01	0.01\\
108.01	0.01\\
109.01	0.01\\
110.01	0.01\\
111.01	0.01\\
112.01	0.01\\
113.01	0.01\\
114.01	0.01\\
115.01	0.01\\
116.01	0.01\\
117.01	0.01\\
118.01	0.01\\
119.01	0.01\\
120.01	0.01\\
121.01	0.01\\
122.01	0.01\\
123.01	0.01\\
124.01	0.01\\
125.01	0.01\\
126.01	0.01\\
127.01	0.01\\
128.01	0.01\\
129.01	0.01\\
130.01	0.01\\
131.01	0.01\\
132.01	0.01\\
133.01	0.01\\
134.01	0.01\\
135.01	0.01\\
136.01	0.01\\
137.01	0.01\\
138.01	0.01\\
139.01	0.01\\
140.01	0.01\\
141.01	0.01\\
142.01	0.01\\
143.01	0.01\\
144.01	0.01\\
145.01	0.01\\
146.01	0.01\\
147.01	0.01\\
148.01	0.01\\
149.01	0.01\\
150.01	0.01\\
151.01	0.01\\
152.01	0.01\\
153.01	0.01\\
154.01	0.01\\
155.01	0.01\\
156.01	0.01\\
157.01	0.01\\
158.01	0.01\\
159.01	0.01\\
160.01	0.01\\
161.01	0.01\\
162.01	0.01\\
163.01	0.01\\
164.01	0.01\\
165.01	0.01\\
166.01	0.01\\
167.01	0.01\\
168.01	0.01\\
169.01	0.01\\
170.01	0.01\\
171.01	0.01\\
172.01	0.01\\
173.01	0.01\\
174.01	0.01\\
175.01	0.01\\
176.01	0.01\\
177.01	0.01\\
178.01	0.01\\
179.01	0.01\\
180.01	0.01\\
181.01	0.01\\
182.01	0.01\\
183.01	0.01\\
184.01	0.01\\
185.01	0.01\\
186.01	0.01\\
187.01	0.01\\
188.01	0.01\\
189.01	0.01\\
190.01	0.01\\
191.01	0.01\\
192.01	0.01\\
193.01	0.01\\
194.01	0.01\\
195.01	0.01\\
196.01	0.01\\
197.01	0.01\\
198.01	0.01\\
199.01	0.01\\
200.01	0.01\\
201.01	0.01\\
202.01	0.01\\
203.01	0.01\\
204.01	0.01\\
205.01	0.01\\
206.01	0.01\\
207.01	0.01\\
208.01	0.01\\
209.01	0.01\\
210.01	0.01\\
211.01	0.01\\
212.01	0.01\\
213.01	0.01\\
214.01	0.01\\
215.01	0.01\\
216.01	0.01\\
217.01	0.01\\
218.01	0.01\\
219.01	0.01\\
220.01	0.01\\
221.01	0.01\\
222.01	0.01\\
223.01	0.01\\
224.01	0.01\\
225.01	0.01\\
226.01	0.01\\
227.01	0.01\\
228.01	0.01\\
229.01	0.01\\
230.01	0.01\\
231.01	0.01\\
232.01	0.01\\
233.01	0.01\\
234.01	0.01\\
235.01	0.01\\
236.01	0.01\\
237.01	0.01\\
238.01	0.01\\
239.01	0.01\\
240.01	0.01\\
241.01	0.01\\
242.01	0.01\\
243.01	0.01\\
244.01	0.01\\
245.01	0.01\\
246.01	0.01\\
247.01	0.01\\
248.01	0.01\\
249.01	0.01\\
250.01	0.01\\
251.01	0.01\\
252.01	0.01\\
253.01	0.01\\
254.01	0.01\\
255.01	0.01\\
256.01	0.01\\
257.01	0.01\\
258.01	0.01\\
259.01	0.01\\
260.01	0.01\\
261.01	0.01\\
262.01	0.01\\
263.01	0.01\\
264.01	0.01\\
265.01	0.01\\
266.01	0.01\\
267.01	0.01\\
268.01	0.01\\
269.01	0.01\\
270.01	0.01\\
271.01	0.01\\
272.01	0.01\\
273.01	0.01\\
274.01	0.01\\
275.01	0.01\\
276.01	0.01\\
277.01	0.01\\
278.01	0.01\\
279.01	0.01\\
280.01	0.01\\
281.01	0.01\\
282.01	0.01\\
283.01	0.01\\
284.01	0.01\\
285.01	0.01\\
286.01	0.01\\
287.01	0.01\\
288.01	0.01\\
289.01	0.01\\
290.01	0.01\\
291.01	0.01\\
292.01	0.01\\
293.01	0.01\\
294.01	0.01\\
295.01	0.01\\
296.01	0.01\\
297.01	0.01\\
298.01	0.01\\
299.01	0.01\\
300.01	0.01\\
301.01	0.01\\
302.01	0.01\\
303.01	0.01\\
304.01	0.01\\
305.01	0.01\\
306.01	0.01\\
307.01	0.01\\
308.01	0.01\\
309.01	0.01\\
310.01	0.01\\
311.01	0.01\\
312.01	0.01\\
313.01	0.01\\
314.01	0.01\\
315.01	0.01\\
316.01	0.01\\
317.01	0.01\\
318.01	0.01\\
319.01	0.01\\
320.01	0.01\\
321.01	0.01\\
322.01	0.01\\
323.01	0.01\\
324.01	0.01\\
325.01	0.01\\
326.01	0.01\\
327.01	0.01\\
328.01	0.01\\
329.01	0.01\\
330.01	0.01\\
331.01	0.01\\
332.01	0.01\\
333.01	0.01\\
334.01	0.01\\
335.01	0.01\\
336.01	0.01\\
337.01	0.01\\
338.01	0.01\\
339.01	0.01\\
340.01	0.01\\
341.01	0.01\\
342.01	0.01\\
343.01	0.01\\
344.01	0.01\\
345.01	0.01\\
346.01	0.01\\
347.01	0.01\\
348.01	0.01\\
349.01	0.01\\
350.01	0.01\\
351.01	0.01\\
352.01	0.01\\
353.01	0.01\\
354.01	0.01\\
355.01	0.01\\
356.01	0.01\\
357.01	0.01\\
358.01	0.01\\
359.01	0.01\\
360.01	0.01\\
361.01	0.01\\
362.01	0.01\\
363.01	0.01\\
364.01	0.01\\
365.01	0.01\\
366.01	0.01\\
367.01	0.01\\
368.01	0.01\\
369.01	0.01\\
370.01	0.01\\
371.01	0.01\\
372.01	0.01\\
373.01	0.01\\
374.01	0.01\\
375.01	0.01\\
376.01	0.01\\
377.01	0.01\\
378.01	0.01\\
379.01	0.01\\
380.01	0.01\\
381.01	0.01\\
382.01	0.01\\
383.01	0.01\\
384.01	0.01\\
385.01	0.01\\
386.01	0.01\\
387.01	0.01\\
388.01	0.01\\
389.01	0.01\\
390.01	0.01\\
391.01	0.01\\
392.01	0.01\\
393.01	0.01\\
394.01	0.01\\
395.01	0.01\\
396.01	0.01\\
397.01	0.01\\
398.01	0.01\\
399.01	0.01\\
400.01	0.01\\
401.01	0.01\\
402.01	0.01\\
403.01	0.01\\
404.01	0.01\\
405.01	0.01\\
406.01	0.01\\
407.01	0.01\\
408.01	0.01\\
409.01	0.01\\
410.01	0.01\\
411.01	0.01\\
412.01	0.01\\
413.01	0.01\\
414.01	0.01\\
415.01	0.01\\
416.01	0.01\\
417.01	0.01\\
418.01	0.01\\
419.01	0.01\\
420.01	0.01\\
421.01	0.01\\
422.01	0.01\\
423.01	0.01\\
424.01	0.01\\
425.01	0.01\\
426.01	0.01\\
427.01	0.01\\
428.01	0.01\\
429.01	0.01\\
430.01	0.01\\
431.01	0.01\\
432.01	0.01\\
433.01	0.01\\
434.01	0.01\\
435.01	0.01\\
436.01	0.01\\
437.01	0.01\\
438.01	0.01\\
439.01	0.01\\
440.01	0.01\\
441.01	0.01\\
442.01	0.01\\
443.01	0.01\\
444.01	0.01\\
445.01	0.01\\
446.01	0.01\\
447.01	0.01\\
448.01	0.01\\
449.01	0.01\\
450.01	0.01\\
451.01	0.01\\
452.01	0.01\\
453.01	0.01\\
454.01	0.01\\
455.01	0.01\\
456.01	0.01\\
457.01	0.01\\
458.01	0.01\\
459.01	0.01\\
460.01	0.01\\
461.01	0.01\\
462.01	0.01\\
463.01	0.01\\
464.01	0.01\\
465.01	0.01\\
466.01	0.01\\
467.01	0.01\\
468.01	0.01\\
469.01	0.01\\
470.01	0.01\\
471.01	0.01\\
472.01	0.01\\
473.01	0.01\\
474.01	0.01\\
475.01	0.01\\
476.01	0.01\\
477.01	0.01\\
478.01	0.01\\
479.01	0.01\\
480.01	0.01\\
481.01	0.01\\
482.01	0.01\\
483.01	0.01\\
484.01	0.01\\
485.01	0.01\\
486.01	0.01\\
487.01	0.01\\
488.01	0.01\\
489.01	0.01\\
490.01	0.01\\
491.01	0.01\\
492.01	0.01\\
493.01	0.01\\
494.01	0.01\\
495.01	0.01\\
496.01	0.01\\
497.01	0.01\\
498.01	0.01\\
499.01	0.01\\
500.01	0.01\\
501.01	0.01\\
502.01	0.01\\
503.01	0.01\\
504.01	0.01\\
505.01	0.01\\
506.01	0.01\\
507.01	0.01\\
508.01	0.01\\
509.01	0.01\\
510.01	0.01\\
511.01	0.01\\
512.01	0.01\\
513.01	0.01\\
514.01	0.01\\
515.01	0.01\\
516.01	0.01\\
517.01	0.01\\
518.01	0.01\\
519.01	0.01\\
520.01	0.01\\
521.01	0.01\\
522.01	0.01\\
523.01	0.01\\
524.01	0.01\\
525.01	0.01\\
526.01	0.01\\
527.01	0.01\\
528.01	0.01\\
529.01	0.01\\
530.01	0.01\\
531.01	0.01\\
532.01	0.01\\
533.01	0.01\\
534.01	0.01\\
535.01	0.01\\
536.01	0.01\\
537.01	0.01\\
538.01	0.01\\
539.01	0.01\\
540.01	0.01\\
541.01	0.01\\
542.01	0.01\\
543.01	0.01\\
544.01	0.01\\
545.01	0.01\\
546.01	0.01\\
547.01	0.01\\
548.01	0.01\\
549.01	0.01\\
550.01	0.01\\
551.01	0.01\\
552.01	0.01\\
553.01	0.01\\
554.01	0.01\\
555.01	0.01\\
556.01	0.01\\
557.01	0.01\\
558.01	0.01\\
559.01	0.01\\
560.01	0.01\\
561.01	0.01\\
562.01	0.01\\
563.01	0.01\\
564.01	0.01\\
565.01	0.01\\
566.01	0.01\\
567.01	0.01\\
568.01	0.01\\
569.01	0.01\\
570.01	0.01\\
571.01	0.01\\
572.01	0.01\\
573.01	0.01\\
574.01	0.01\\
575.01	0.01\\
576.01	0.01\\
577.01	0.01\\
578.01	0.01\\
579.01	0.01\\
580.01	0.01\\
581.01	0.01\\
582.01	0.01\\
583.01	0.01\\
584.01	0.01\\
585.01	0.01\\
586.01	0.01\\
587.01	0.01\\
588.01	0.01\\
589.01	0.01\\
590.01	0.01\\
591.01	0.01\\
592.01	0.01\\
593.01	0.01\\
594.01	0.01\\
595.01	0.01\\
596.01	0.01\\
597.01	0.01\\
598.01	0.01\\
599.01	0.01\\
599.02	0.01\\
599.03	0.01\\
599.04	0.01\\
599.05	0.01\\
599.06	0.01\\
599.07	0.01\\
599.08	0.01\\
599.09	0.01\\
599.1	0.01\\
599.11	0.01\\
599.12	0.01\\
599.13	0.01\\
599.14	0.01\\
599.15	0.01\\
599.16	0.01\\
599.17	0.01\\
599.18	0.01\\
599.19	0.01\\
599.2	0.01\\
599.21	0.01\\
599.22	0.01\\
599.23	0.01\\
599.24	0.01\\
599.25	0.01\\
599.26	0.01\\
599.27	0.01\\
599.28	0.01\\
599.29	0.01\\
599.3	0.01\\
599.31	0.01\\
599.32	0.01\\
599.33	0.01\\
599.34	0.01\\
599.35	0.01\\
599.36	0.01\\
599.37	0.01\\
599.38	0.01\\
599.39	0.01\\
599.4	0.01\\
599.41	0.01\\
599.42	0.01\\
599.43	0.01\\
599.44	0.01\\
599.45	0.01\\
599.46	0.01\\
599.47	0.01\\
599.48	0.01\\
599.49	0.01\\
599.5	0.01\\
599.51	0.01\\
599.52	0.01\\
599.53	0.01\\
599.54	0.01\\
599.55	0.01\\
599.56	0.01\\
599.57	0.01\\
599.58	0.01\\
599.59	0.01\\
599.6	0.01\\
599.61	0.01\\
599.62	0.01\\
599.63	0.01\\
599.64	0.01\\
599.65	0.01\\
599.66	0.01\\
599.67	0.01\\
599.68	0.01\\
599.69	0.01\\
599.7	0.01\\
599.71	0.01\\
599.72	0.01\\
599.73	0.01\\
599.74	0.01\\
599.75	0.01\\
599.76	0.01\\
599.77	0.01\\
599.78	0.01\\
599.79	0.01\\
599.8	0.01\\
599.81	0.01\\
599.82	0.01\\
599.83	0.01\\
599.84	0.01\\
599.85	0.01\\
599.86	0.01\\
599.87	0.01\\
599.88	0.01\\
599.89	0.01\\
599.9	0.01\\
599.91	0.01\\
599.92	0.01\\
599.93	0.01\\
599.94	0.01\\
599.95	0.01\\
599.96	0.01\\
599.97	0.01\\
599.98	0.01\\
599.99	0.01\\
600	0.01\\
};
\addplot [color=mycolor1,solid,forget plot]
  table[row sep=crcr]{%
0.01	0.01\\
1.01	0.01\\
2.01	0.01\\
3.01	0.01\\
4.01	0.01\\
5.01	0.01\\
6.01	0.01\\
7.01	0.01\\
8.01	0.01\\
9.01	0.01\\
10.01	0.01\\
11.01	0.01\\
12.01	0.01\\
13.01	0.01\\
14.01	0.01\\
15.01	0.01\\
16.01	0.01\\
17.01	0.01\\
18.01	0.01\\
19.01	0.01\\
20.01	0.01\\
21.01	0.01\\
22.01	0.01\\
23.01	0.01\\
24.01	0.01\\
25.01	0.01\\
26.01	0.01\\
27.01	0.01\\
28.01	0.01\\
29.01	0.01\\
30.01	0.01\\
31.01	0.01\\
32.01	0.01\\
33.01	0.01\\
34.01	0.01\\
35.01	0.01\\
36.01	0.01\\
37.01	0.01\\
38.01	0.01\\
39.01	0.01\\
40.01	0.01\\
41.01	0.01\\
42.01	0.01\\
43.01	0.01\\
44.01	0.01\\
45.01	0.01\\
46.01	0.01\\
47.01	0.01\\
48.01	0.01\\
49.01	0.01\\
50.01	0.01\\
51.01	0.01\\
52.01	0.01\\
53.01	0.01\\
54.01	0.01\\
55.01	0.01\\
56.01	0.01\\
57.01	0.01\\
58.01	0.01\\
59.01	0.01\\
60.01	0.01\\
61.01	0.01\\
62.01	0.01\\
63.01	0.01\\
64.01	0.01\\
65.01	0.01\\
66.01	0.01\\
67.01	0.01\\
68.01	0.01\\
69.01	0.01\\
70.01	0.01\\
71.01	0.01\\
72.01	0.01\\
73.01	0.01\\
74.01	0.01\\
75.01	0.01\\
76.01	0.01\\
77.01	0.01\\
78.01	0.01\\
79.01	0.01\\
80.01	0.01\\
81.01	0.01\\
82.01	0.01\\
83.01	0.01\\
84.01	0.01\\
85.01	0.01\\
86.01	0.01\\
87.01	0.01\\
88.01	0.01\\
89.01	0.01\\
90.01	0.01\\
91.01	0.01\\
92.01	0.01\\
93.01	0.01\\
94.01	0.01\\
95.01	0.01\\
96.01	0.01\\
97.01	0.01\\
98.01	0.01\\
99.01	0.01\\
100.01	0.01\\
101.01	0.01\\
102.01	0.01\\
103.01	0.01\\
104.01	0.01\\
105.01	0.01\\
106.01	0.01\\
107.01	0.01\\
108.01	0.01\\
109.01	0.01\\
110.01	0.01\\
111.01	0.01\\
112.01	0.01\\
113.01	0.01\\
114.01	0.01\\
115.01	0.01\\
116.01	0.01\\
117.01	0.01\\
118.01	0.01\\
119.01	0.01\\
120.01	0.01\\
121.01	0.01\\
122.01	0.01\\
123.01	0.01\\
124.01	0.01\\
125.01	0.01\\
126.01	0.01\\
127.01	0.01\\
128.01	0.01\\
129.01	0.01\\
130.01	0.01\\
131.01	0.01\\
132.01	0.01\\
133.01	0.01\\
134.01	0.01\\
135.01	0.01\\
136.01	0.01\\
137.01	0.01\\
138.01	0.01\\
139.01	0.01\\
140.01	0.01\\
141.01	0.01\\
142.01	0.01\\
143.01	0.01\\
144.01	0.01\\
145.01	0.01\\
146.01	0.01\\
147.01	0.01\\
148.01	0.01\\
149.01	0.01\\
150.01	0.01\\
151.01	0.01\\
152.01	0.01\\
153.01	0.01\\
154.01	0.01\\
155.01	0.01\\
156.01	0.01\\
157.01	0.01\\
158.01	0.01\\
159.01	0.01\\
160.01	0.01\\
161.01	0.01\\
162.01	0.01\\
163.01	0.01\\
164.01	0.01\\
165.01	0.01\\
166.01	0.01\\
167.01	0.01\\
168.01	0.01\\
169.01	0.01\\
170.01	0.01\\
171.01	0.01\\
172.01	0.01\\
173.01	0.01\\
174.01	0.01\\
175.01	0.01\\
176.01	0.01\\
177.01	0.01\\
178.01	0.01\\
179.01	0.01\\
180.01	0.01\\
181.01	0.01\\
182.01	0.01\\
183.01	0.01\\
184.01	0.01\\
185.01	0.01\\
186.01	0.01\\
187.01	0.01\\
188.01	0.01\\
189.01	0.01\\
190.01	0.01\\
191.01	0.01\\
192.01	0.01\\
193.01	0.01\\
194.01	0.01\\
195.01	0.01\\
196.01	0.01\\
197.01	0.01\\
198.01	0.01\\
199.01	0.01\\
200.01	0.01\\
201.01	0.01\\
202.01	0.01\\
203.01	0.01\\
204.01	0.01\\
205.01	0.01\\
206.01	0.01\\
207.01	0.01\\
208.01	0.01\\
209.01	0.01\\
210.01	0.01\\
211.01	0.01\\
212.01	0.01\\
213.01	0.01\\
214.01	0.01\\
215.01	0.01\\
216.01	0.01\\
217.01	0.01\\
218.01	0.01\\
219.01	0.01\\
220.01	0.01\\
221.01	0.01\\
222.01	0.01\\
223.01	0.01\\
224.01	0.01\\
225.01	0.01\\
226.01	0.01\\
227.01	0.01\\
228.01	0.01\\
229.01	0.01\\
230.01	0.01\\
231.01	0.01\\
232.01	0.01\\
233.01	0.01\\
234.01	0.01\\
235.01	0.01\\
236.01	0.01\\
237.01	0.01\\
238.01	0.01\\
239.01	0.01\\
240.01	0.01\\
241.01	0.01\\
242.01	0.01\\
243.01	0.01\\
244.01	0.01\\
245.01	0.01\\
246.01	0.01\\
247.01	0.01\\
248.01	0.01\\
249.01	0.01\\
250.01	0.01\\
251.01	0.01\\
252.01	0.01\\
253.01	0.01\\
254.01	0.01\\
255.01	0.01\\
256.01	0.01\\
257.01	0.01\\
258.01	0.01\\
259.01	0.01\\
260.01	0.01\\
261.01	0.01\\
262.01	0.01\\
263.01	0.01\\
264.01	0.01\\
265.01	0.01\\
266.01	0.01\\
267.01	0.01\\
268.01	0.01\\
269.01	0.01\\
270.01	0.01\\
271.01	0.01\\
272.01	0.01\\
273.01	0.01\\
274.01	0.01\\
275.01	0.01\\
276.01	0.01\\
277.01	0.01\\
278.01	0.01\\
279.01	0.01\\
280.01	0.01\\
281.01	0.01\\
282.01	0.01\\
283.01	0.01\\
284.01	0.01\\
285.01	0.01\\
286.01	0.01\\
287.01	0.01\\
288.01	0.01\\
289.01	0.01\\
290.01	0.01\\
291.01	0.01\\
292.01	0.01\\
293.01	0.01\\
294.01	0.01\\
295.01	0.01\\
296.01	0.01\\
297.01	0.01\\
298.01	0.01\\
299.01	0.01\\
300.01	0.01\\
301.01	0.01\\
302.01	0.01\\
303.01	0.01\\
304.01	0.01\\
305.01	0.01\\
306.01	0.01\\
307.01	0.01\\
308.01	0.01\\
309.01	0.01\\
310.01	0.01\\
311.01	0.01\\
312.01	0.01\\
313.01	0.01\\
314.01	0.01\\
315.01	0.01\\
316.01	0.01\\
317.01	0.01\\
318.01	0.01\\
319.01	0.01\\
320.01	0.01\\
321.01	0.01\\
322.01	0.01\\
323.01	0.01\\
324.01	0.01\\
325.01	0.01\\
326.01	0.01\\
327.01	0.01\\
328.01	0.01\\
329.01	0.01\\
330.01	0.01\\
331.01	0.01\\
332.01	0.01\\
333.01	0.01\\
334.01	0.01\\
335.01	0.01\\
336.01	0.01\\
337.01	0.01\\
338.01	0.01\\
339.01	0.01\\
340.01	0.01\\
341.01	0.01\\
342.01	0.01\\
343.01	0.01\\
344.01	0.01\\
345.01	0.01\\
346.01	0.01\\
347.01	0.01\\
348.01	0.01\\
349.01	0.01\\
350.01	0.01\\
351.01	0.01\\
352.01	0.01\\
353.01	0.01\\
354.01	0.01\\
355.01	0.01\\
356.01	0.01\\
357.01	0.01\\
358.01	0.01\\
359.01	0.01\\
360.01	0.01\\
361.01	0.01\\
362.01	0.01\\
363.01	0.01\\
364.01	0.01\\
365.01	0.01\\
366.01	0.01\\
367.01	0.01\\
368.01	0.01\\
369.01	0.01\\
370.01	0.01\\
371.01	0.01\\
372.01	0.01\\
373.01	0.01\\
374.01	0.01\\
375.01	0.01\\
376.01	0.01\\
377.01	0.01\\
378.01	0.01\\
379.01	0.01\\
380.01	0.01\\
381.01	0.01\\
382.01	0.01\\
383.01	0.01\\
384.01	0.01\\
385.01	0.01\\
386.01	0.01\\
387.01	0.01\\
388.01	0.01\\
389.01	0.01\\
390.01	0.01\\
391.01	0.01\\
392.01	0.01\\
393.01	0.01\\
394.01	0.01\\
395.01	0.01\\
396.01	0.01\\
397.01	0.01\\
398.01	0.01\\
399.01	0.01\\
400.01	0.01\\
401.01	0.01\\
402.01	0.01\\
403.01	0.01\\
404.01	0.01\\
405.01	0.01\\
406.01	0.01\\
407.01	0.01\\
408.01	0.01\\
409.01	0.01\\
410.01	0.01\\
411.01	0.01\\
412.01	0.01\\
413.01	0.01\\
414.01	0.01\\
415.01	0.01\\
416.01	0.01\\
417.01	0.01\\
418.01	0.01\\
419.01	0.01\\
420.01	0.01\\
421.01	0.01\\
422.01	0.01\\
423.01	0.01\\
424.01	0.01\\
425.01	0.01\\
426.01	0.01\\
427.01	0.01\\
428.01	0.01\\
429.01	0.01\\
430.01	0.01\\
431.01	0.01\\
432.01	0.01\\
433.01	0.01\\
434.01	0.01\\
435.01	0.01\\
436.01	0.01\\
437.01	0.01\\
438.01	0.01\\
439.01	0.01\\
440.01	0.01\\
441.01	0.01\\
442.01	0.01\\
443.01	0.01\\
444.01	0.01\\
445.01	0.01\\
446.01	0.01\\
447.01	0.01\\
448.01	0.01\\
449.01	0.01\\
450.01	0.01\\
451.01	0.01\\
452.01	0.01\\
453.01	0.01\\
454.01	0.01\\
455.01	0.01\\
456.01	0.01\\
457.01	0.01\\
458.01	0.01\\
459.01	0.01\\
460.01	0.01\\
461.01	0.01\\
462.01	0.01\\
463.01	0.01\\
464.01	0.01\\
465.01	0.01\\
466.01	0.01\\
467.01	0.01\\
468.01	0.01\\
469.01	0.01\\
470.01	0.01\\
471.01	0.01\\
472.01	0.01\\
473.01	0.01\\
474.01	0.01\\
475.01	0.01\\
476.01	0.01\\
477.01	0.01\\
478.01	0.01\\
479.01	0.01\\
480.01	0.01\\
481.01	0.01\\
482.01	0.01\\
483.01	0.01\\
484.01	0.01\\
485.01	0.01\\
486.01	0.01\\
487.01	0.01\\
488.01	0.01\\
489.01	0.01\\
490.01	0.01\\
491.01	0.01\\
492.01	0.01\\
493.01	0.01\\
494.01	0.01\\
495.01	0.01\\
496.01	0.01\\
497.01	0.01\\
498.01	0.01\\
499.01	0.01\\
500.01	0.01\\
501.01	0.01\\
502.01	0.01\\
503.01	0.01\\
504.01	0.01\\
505.01	0.01\\
506.01	0.01\\
507.01	0.01\\
508.01	0.01\\
509.01	0.01\\
510.01	0.01\\
511.01	0.01\\
512.01	0.01\\
513.01	0.01\\
514.01	0.01\\
515.01	0.01\\
516.01	0.01\\
517.01	0.01\\
518.01	0.01\\
519.01	0.01\\
520.01	0.01\\
521.01	0.01\\
522.01	0.01\\
523.01	0.01\\
524.01	0.01\\
525.01	0.01\\
526.01	0.01\\
527.01	0.01\\
528.01	0.01\\
529.01	0.01\\
530.01	0.01\\
531.01	0.01\\
532.01	0.01\\
533.01	0.01\\
534.01	0.01\\
535.01	0.01\\
536.01	0.01\\
537.01	0.01\\
538.01	0.01\\
539.01	0.01\\
540.01	0.01\\
541.01	0.01\\
542.01	0.01\\
543.01	0.01\\
544.01	0.01\\
545.01	0.01\\
546.01	0.01\\
547.01	0.01\\
548.01	0.01\\
549.01	0.01\\
550.01	0.01\\
551.01	0.01\\
552.01	0.01\\
553.01	0.01\\
554.01	0.01\\
555.01	0.01\\
556.01	0.01\\
557.01	0.01\\
558.01	0.01\\
559.01	0.01\\
560.01	0.01\\
561.01	0.01\\
562.01	0.01\\
563.01	0.01\\
564.01	0.01\\
565.01	0.01\\
566.01	0.01\\
567.01	0.01\\
568.01	0.01\\
569.01	0.01\\
570.01	0.01\\
571.01	0.01\\
572.01	0.01\\
573.01	0.01\\
574.01	0.01\\
575.01	0.01\\
576.01	0.01\\
577.01	0.01\\
578.01	0.01\\
579.01	0.01\\
580.01	0.01\\
581.01	0.01\\
582.01	0.01\\
583.01	0.01\\
584.01	0.01\\
585.01	0.01\\
586.01	0.01\\
587.01	0.01\\
588.01	0.01\\
589.01	0.01\\
590.01	0.01\\
591.01	0.01\\
592.01	0.01\\
593.01	0.01\\
594.01	0.01\\
595.01	0.01\\
596.01	0.01\\
597.01	0.01\\
598.01	0.01\\
599.01	0.01\\
599.02	0.01\\
599.03	0.01\\
599.04	0.01\\
599.05	0.01\\
599.06	0.01\\
599.07	0.01\\
599.08	0.01\\
599.09	0.01\\
599.1	0.01\\
599.11	0.01\\
599.12	0.01\\
599.13	0.01\\
599.14	0.01\\
599.15	0.01\\
599.16	0.01\\
599.17	0.01\\
599.18	0.01\\
599.19	0.01\\
599.2	0.01\\
599.21	0.01\\
599.22	0.01\\
599.23	0.01\\
599.24	0.01\\
599.25	0.01\\
599.26	0.01\\
599.27	0.01\\
599.28	0.01\\
599.29	0.01\\
599.3	0.01\\
599.31	0.01\\
599.32	0.01\\
599.33	0.01\\
599.34	0.01\\
599.35	0.01\\
599.36	0.01\\
599.37	0.01\\
599.38	0.01\\
599.39	0.01\\
599.4	0.01\\
599.41	0.01\\
599.42	0.01\\
599.43	0.01\\
599.44	0.01\\
599.45	0.01\\
599.46	0.01\\
599.47	0.01\\
599.48	0.01\\
599.49	0.01\\
599.5	0.01\\
599.51	0.01\\
599.52	0.01\\
599.53	0.01\\
599.54	0.01\\
599.55	0.01\\
599.56	0.01\\
599.57	0.01\\
599.58	0.01\\
599.59	0.01\\
599.6	0.01\\
599.61	0.01\\
599.62	0.01\\
599.63	0.01\\
599.64	0.01\\
599.65	0.01\\
599.66	0.01\\
599.67	0.01\\
599.68	0.01\\
599.69	0.01\\
599.7	0.01\\
599.71	0.01\\
599.72	0.01\\
599.73	0.01\\
599.74	0.01\\
599.75	0.01\\
599.76	0.01\\
599.77	0.01\\
599.78	0.01\\
599.79	0.01\\
599.8	0.01\\
599.81	0.01\\
599.82	0.01\\
599.83	0.01\\
599.84	0.01\\
599.85	0.01\\
599.86	0.01\\
599.87	0.01\\
599.88	0.01\\
599.89	0.01\\
599.9	0.01\\
599.91	0.01\\
599.92	0.01\\
599.93	0.01\\
599.94	0.01\\
599.95	0.01\\
599.96	0.01\\
599.97	0.01\\
599.98	0.01\\
599.99	0.01\\
600	0.01\\
};
\addplot [color=mycolor2,solid,forget plot]
  table[row sep=crcr]{%
0.01	0.01\\
1.01	0.01\\
2.01	0.01\\
3.01	0.01\\
4.01	0.01\\
5.01	0.01\\
6.01	0.01\\
7.01	0.01\\
8.01	0.01\\
9.01	0.01\\
10.01	0.01\\
11.01	0.01\\
12.01	0.01\\
13.01	0.01\\
14.01	0.01\\
15.01	0.01\\
16.01	0.01\\
17.01	0.01\\
18.01	0.01\\
19.01	0.01\\
20.01	0.01\\
21.01	0.01\\
22.01	0.01\\
23.01	0.01\\
24.01	0.01\\
25.01	0.01\\
26.01	0.01\\
27.01	0.01\\
28.01	0.01\\
29.01	0.01\\
30.01	0.01\\
31.01	0.01\\
32.01	0.01\\
33.01	0.01\\
34.01	0.01\\
35.01	0.01\\
36.01	0.01\\
37.01	0.01\\
38.01	0.01\\
39.01	0.01\\
40.01	0.01\\
41.01	0.01\\
42.01	0.01\\
43.01	0.01\\
44.01	0.01\\
45.01	0.01\\
46.01	0.01\\
47.01	0.01\\
48.01	0.01\\
49.01	0.01\\
50.01	0.01\\
51.01	0.01\\
52.01	0.01\\
53.01	0.01\\
54.01	0.01\\
55.01	0.01\\
56.01	0.01\\
57.01	0.01\\
58.01	0.01\\
59.01	0.01\\
60.01	0.01\\
61.01	0.01\\
62.01	0.01\\
63.01	0.01\\
64.01	0.01\\
65.01	0.01\\
66.01	0.01\\
67.01	0.01\\
68.01	0.01\\
69.01	0.01\\
70.01	0.01\\
71.01	0.01\\
72.01	0.01\\
73.01	0.01\\
74.01	0.01\\
75.01	0.01\\
76.01	0.01\\
77.01	0.01\\
78.01	0.01\\
79.01	0.01\\
80.01	0.01\\
81.01	0.01\\
82.01	0.01\\
83.01	0.01\\
84.01	0.01\\
85.01	0.01\\
86.01	0.01\\
87.01	0.01\\
88.01	0.01\\
89.01	0.01\\
90.01	0.01\\
91.01	0.01\\
92.01	0.01\\
93.01	0.01\\
94.01	0.01\\
95.01	0.01\\
96.01	0.01\\
97.01	0.01\\
98.01	0.01\\
99.01	0.01\\
100.01	0.01\\
101.01	0.01\\
102.01	0.01\\
103.01	0.01\\
104.01	0.01\\
105.01	0.01\\
106.01	0.01\\
107.01	0.01\\
108.01	0.01\\
109.01	0.01\\
110.01	0.01\\
111.01	0.01\\
112.01	0.01\\
113.01	0.01\\
114.01	0.01\\
115.01	0.01\\
116.01	0.01\\
117.01	0.01\\
118.01	0.01\\
119.01	0.01\\
120.01	0.01\\
121.01	0.01\\
122.01	0.01\\
123.01	0.01\\
124.01	0.01\\
125.01	0.01\\
126.01	0.01\\
127.01	0.01\\
128.01	0.01\\
129.01	0.01\\
130.01	0.01\\
131.01	0.01\\
132.01	0.01\\
133.01	0.01\\
134.01	0.01\\
135.01	0.01\\
136.01	0.01\\
137.01	0.01\\
138.01	0.01\\
139.01	0.01\\
140.01	0.01\\
141.01	0.01\\
142.01	0.01\\
143.01	0.01\\
144.01	0.01\\
145.01	0.01\\
146.01	0.01\\
147.01	0.01\\
148.01	0.01\\
149.01	0.01\\
150.01	0.01\\
151.01	0.01\\
152.01	0.01\\
153.01	0.01\\
154.01	0.01\\
155.01	0.01\\
156.01	0.01\\
157.01	0.01\\
158.01	0.01\\
159.01	0.01\\
160.01	0.01\\
161.01	0.01\\
162.01	0.01\\
163.01	0.01\\
164.01	0.01\\
165.01	0.01\\
166.01	0.01\\
167.01	0.01\\
168.01	0.01\\
169.01	0.01\\
170.01	0.01\\
171.01	0.01\\
172.01	0.01\\
173.01	0.01\\
174.01	0.01\\
175.01	0.01\\
176.01	0.01\\
177.01	0.01\\
178.01	0.01\\
179.01	0.01\\
180.01	0.01\\
181.01	0.01\\
182.01	0.01\\
183.01	0.01\\
184.01	0.01\\
185.01	0.01\\
186.01	0.01\\
187.01	0.01\\
188.01	0.01\\
189.01	0.01\\
190.01	0.01\\
191.01	0.01\\
192.01	0.01\\
193.01	0.01\\
194.01	0.01\\
195.01	0.01\\
196.01	0.01\\
197.01	0.01\\
198.01	0.01\\
199.01	0.01\\
200.01	0.01\\
201.01	0.01\\
202.01	0.01\\
203.01	0.01\\
204.01	0.01\\
205.01	0.01\\
206.01	0.01\\
207.01	0.01\\
208.01	0.01\\
209.01	0.01\\
210.01	0.01\\
211.01	0.01\\
212.01	0.01\\
213.01	0.01\\
214.01	0.01\\
215.01	0.01\\
216.01	0.01\\
217.01	0.01\\
218.01	0.01\\
219.01	0.01\\
220.01	0.01\\
221.01	0.01\\
222.01	0.01\\
223.01	0.01\\
224.01	0.01\\
225.01	0.01\\
226.01	0.01\\
227.01	0.01\\
228.01	0.01\\
229.01	0.01\\
230.01	0.01\\
231.01	0.01\\
232.01	0.01\\
233.01	0.01\\
234.01	0.01\\
235.01	0.01\\
236.01	0.01\\
237.01	0.01\\
238.01	0.01\\
239.01	0.01\\
240.01	0.01\\
241.01	0.01\\
242.01	0.01\\
243.01	0.01\\
244.01	0.01\\
245.01	0.01\\
246.01	0.01\\
247.01	0.01\\
248.01	0.01\\
249.01	0.01\\
250.01	0.01\\
251.01	0.01\\
252.01	0.01\\
253.01	0.01\\
254.01	0.01\\
255.01	0.01\\
256.01	0.01\\
257.01	0.01\\
258.01	0.01\\
259.01	0.01\\
260.01	0.01\\
261.01	0.01\\
262.01	0.01\\
263.01	0.01\\
264.01	0.01\\
265.01	0.01\\
266.01	0.01\\
267.01	0.01\\
268.01	0.01\\
269.01	0.01\\
270.01	0.01\\
271.01	0.01\\
272.01	0.01\\
273.01	0.01\\
274.01	0.01\\
275.01	0.01\\
276.01	0.01\\
277.01	0.01\\
278.01	0.01\\
279.01	0.01\\
280.01	0.01\\
281.01	0.01\\
282.01	0.01\\
283.01	0.01\\
284.01	0.01\\
285.01	0.01\\
286.01	0.01\\
287.01	0.01\\
288.01	0.01\\
289.01	0.01\\
290.01	0.01\\
291.01	0.01\\
292.01	0.01\\
293.01	0.01\\
294.01	0.01\\
295.01	0.01\\
296.01	0.01\\
297.01	0.01\\
298.01	0.01\\
299.01	0.01\\
300.01	0.01\\
301.01	0.01\\
302.01	0.01\\
303.01	0.01\\
304.01	0.01\\
305.01	0.01\\
306.01	0.01\\
307.01	0.01\\
308.01	0.01\\
309.01	0.01\\
310.01	0.01\\
311.01	0.01\\
312.01	0.01\\
313.01	0.01\\
314.01	0.01\\
315.01	0.01\\
316.01	0.01\\
317.01	0.01\\
318.01	0.01\\
319.01	0.01\\
320.01	0.01\\
321.01	0.01\\
322.01	0.01\\
323.01	0.01\\
324.01	0.01\\
325.01	0.01\\
326.01	0.01\\
327.01	0.01\\
328.01	0.01\\
329.01	0.01\\
330.01	0.01\\
331.01	0.01\\
332.01	0.01\\
333.01	0.01\\
334.01	0.01\\
335.01	0.01\\
336.01	0.01\\
337.01	0.01\\
338.01	0.01\\
339.01	0.01\\
340.01	0.01\\
341.01	0.01\\
342.01	0.01\\
343.01	0.01\\
344.01	0.01\\
345.01	0.01\\
346.01	0.01\\
347.01	0.01\\
348.01	0.01\\
349.01	0.01\\
350.01	0.01\\
351.01	0.01\\
352.01	0.01\\
353.01	0.01\\
354.01	0.01\\
355.01	0.01\\
356.01	0.01\\
357.01	0.01\\
358.01	0.01\\
359.01	0.01\\
360.01	0.01\\
361.01	0.01\\
362.01	0.01\\
363.01	0.01\\
364.01	0.01\\
365.01	0.01\\
366.01	0.01\\
367.01	0.01\\
368.01	0.01\\
369.01	0.01\\
370.01	0.01\\
371.01	0.01\\
372.01	0.01\\
373.01	0.01\\
374.01	0.01\\
375.01	0.01\\
376.01	0.01\\
377.01	0.01\\
378.01	0.01\\
379.01	0.01\\
380.01	0.01\\
381.01	0.01\\
382.01	0.01\\
383.01	0.01\\
384.01	0.01\\
385.01	0.01\\
386.01	0.01\\
387.01	0.01\\
388.01	0.01\\
389.01	0.01\\
390.01	0.01\\
391.01	0.01\\
392.01	0.01\\
393.01	0.01\\
394.01	0.01\\
395.01	0.01\\
396.01	0.01\\
397.01	0.01\\
398.01	0.01\\
399.01	0.01\\
400.01	0.01\\
401.01	0.01\\
402.01	0.01\\
403.01	0.01\\
404.01	0.01\\
405.01	0.01\\
406.01	0.01\\
407.01	0.01\\
408.01	0.01\\
409.01	0.01\\
410.01	0.01\\
411.01	0.01\\
412.01	0.01\\
413.01	0.01\\
414.01	0.01\\
415.01	0.01\\
416.01	0.01\\
417.01	0.01\\
418.01	0.01\\
419.01	0.01\\
420.01	0.01\\
421.01	0.01\\
422.01	0.01\\
423.01	0.01\\
424.01	0.01\\
425.01	0.01\\
426.01	0.01\\
427.01	0.01\\
428.01	0.01\\
429.01	0.01\\
430.01	0.01\\
431.01	0.01\\
432.01	0.01\\
433.01	0.01\\
434.01	0.01\\
435.01	0.01\\
436.01	0.01\\
437.01	0.01\\
438.01	0.01\\
439.01	0.01\\
440.01	0.01\\
441.01	0.01\\
442.01	0.01\\
443.01	0.01\\
444.01	0.01\\
445.01	0.01\\
446.01	0.01\\
447.01	0.01\\
448.01	0.01\\
449.01	0.01\\
450.01	0.01\\
451.01	0.01\\
452.01	0.01\\
453.01	0.01\\
454.01	0.01\\
455.01	0.01\\
456.01	0.01\\
457.01	0.01\\
458.01	0.01\\
459.01	0.01\\
460.01	0.01\\
461.01	0.01\\
462.01	0.01\\
463.01	0.01\\
464.01	0.01\\
465.01	0.01\\
466.01	0.01\\
467.01	0.01\\
468.01	0.01\\
469.01	0.01\\
470.01	0.01\\
471.01	0.01\\
472.01	0.01\\
473.01	0.01\\
474.01	0.01\\
475.01	0.01\\
476.01	0.01\\
477.01	0.01\\
478.01	0.01\\
479.01	0.01\\
480.01	0.01\\
481.01	0.01\\
482.01	0.01\\
483.01	0.01\\
484.01	0.01\\
485.01	0.01\\
486.01	0.01\\
487.01	0.01\\
488.01	0.01\\
489.01	0.01\\
490.01	0.01\\
491.01	0.01\\
492.01	0.01\\
493.01	0.01\\
494.01	0.01\\
495.01	0.01\\
496.01	0.01\\
497.01	0.01\\
498.01	0.01\\
499.01	0.01\\
500.01	0.01\\
501.01	0.01\\
502.01	0.01\\
503.01	0.01\\
504.01	0.01\\
505.01	0.01\\
506.01	0.01\\
507.01	0.01\\
508.01	0.01\\
509.01	0.01\\
510.01	0.01\\
511.01	0.01\\
512.01	0.01\\
513.01	0.01\\
514.01	0.01\\
515.01	0.01\\
516.01	0.01\\
517.01	0.01\\
518.01	0.01\\
519.01	0.01\\
520.01	0.01\\
521.01	0.01\\
522.01	0.01\\
523.01	0.01\\
524.01	0.01\\
525.01	0.01\\
526.01	0.01\\
527.01	0.01\\
528.01	0.01\\
529.01	0.01\\
530.01	0.01\\
531.01	0.01\\
532.01	0.01\\
533.01	0.01\\
534.01	0.01\\
535.01	0.01\\
536.01	0.01\\
537.01	0.01\\
538.01	0.01\\
539.01	0.01\\
540.01	0.01\\
541.01	0.01\\
542.01	0.01\\
543.01	0.01\\
544.01	0.01\\
545.01	0.01\\
546.01	0.01\\
547.01	0.01\\
548.01	0.01\\
549.01	0.01\\
550.01	0.01\\
551.01	0.01\\
552.01	0.01\\
553.01	0.01\\
554.01	0.01\\
555.01	0.01\\
556.01	0.01\\
557.01	0.01\\
558.01	0.01\\
559.01	0.01\\
560.01	0.01\\
561.01	0.01\\
562.01	0.01\\
563.01	0.01\\
564.01	0.01\\
565.01	0.01\\
566.01	0.01\\
567.01	0.01\\
568.01	0.01\\
569.01	0.01\\
570.01	0.01\\
571.01	0.01\\
572.01	0.01\\
573.01	0.01\\
574.01	0.01\\
575.01	0.01\\
576.01	0.01\\
577.01	0.01\\
578.01	0.01\\
579.01	0.01\\
580.01	0.01\\
581.01	0.01\\
582.01	0.01\\
583.01	0.01\\
584.01	0.01\\
585.01	0.01\\
586.01	0.01\\
587.01	0.01\\
588.01	0.01\\
589.01	0.01\\
590.01	0.01\\
591.01	0.01\\
592.01	0.01\\
593.01	0.01\\
594.01	0.01\\
595.01	0.01\\
596.01	0.01\\
597.01	0.01\\
598.01	0.01\\
599.01	0.01\\
599.02	0.01\\
599.03	0.01\\
599.04	0.01\\
599.05	0.01\\
599.06	0.01\\
599.07	0.01\\
599.08	0.01\\
599.09	0.01\\
599.1	0.01\\
599.11	0.01\\
599.12	0.01\\
599.13	0.01\\
599.14	0.01\\
599.15	0.01\\
599.16	0.01\\
599.17	0.01\\
599.18	0.01\\
599.19	0.01\\
599.2	0.01\\
599.21	0.01\\
599.22	0.01\\
599.23	0.01\\
599.24	0.01\\
599.25	0.01\\
599.26	0.01\\
599.27	0.01\\
599.28	0.01\\
599.29	0.01\\
599.3	0.01\\
599.31	0.01\\
599.32	0.01\\
599.33	0.01\\
599.34	0.01\\
599.35	0.01\\
599.36	0.01\\
599.37	0.01\\
599.38	0.01\\
599.39	0.01\\
599.4	0.01\\
599.41	0.01\\
599.42	0.01\\
599.43	0.01\\
599.44	0.01\\
599.45	0.01\\
599.46	0.01\\
599.47	0.01\\
599.48	0.01\\
599.49	0.01\\
599.5	0.01\\
599.51	0.01\\
599.52	0.01\\
599.53	0.01\\
599.54	0.01\\
599.55	0.01\\
599.56	0.01\\
599.57	0.01\\
599.58	0.01\\
599.59	0.01\\
599.6	0.01\\
599.61	0.01\\
599.62	0.01\\
599.63	0.01\\
599.64	0.01\\
599.65	0.01\\
599.66	0.01\\
599.67	0.01\\
599.68	0.01\\
599.69	0.01\\
599.7	0.01\\
599.71	0.01\\
599.72	0.01\\
599.73	0.01\\
599.74	0.01\\
599.75	0.01\\
599.76	0.01\\
599.77	0.01\\
599.78	0.01\\
599.79	0.01\\
599.8	0.01\\
599.81	0.01\\
599.82	0.01\\
599.83	0.01\\
599.84	0.01\\
599.85	0.01\\
599.86	0.01\\
599.87	0.01\\
599.88	0.01\\
599.89	0.01\\
599.9	0.01\\
599.91	0.01\\
599.92	0.01\\
599.93	0.01\\
599.94	0.01\\
599.95	0.01\\
599.96	0.01\\
599.97	0.01\\
599.98	0.01\\
599.99	0.01\\
600	0.01\\
};
\addplot [color=mycolor3,solid,forget plot]
  table[row sep=crcr]{%
0.01	0.01\\
1.01	0.01\\
2.01	0.01\\
3.01	0.01\\
4.01	0.01\\
5.01	0.01\\
6.01	0.01\\
7.01	0.01\\
8.01	0.01\\
9.01	0.01\\
10.01	0.01\\
11.01	0.01\\
12.01	0.01\\
13.01	0.01\\
14.01	0.01\\
15.01	0.01\\
16.01	0.01\\
17.01	0.01\\
18.01	0.01\\
19.01	0.01\\
20.01	0.01\\
21.01	0.01\\
22.01	0.01\\
23.01	0.01\\
24.01	0.01\\
25.01	0.01\\
26.01	0.01\\
27.01	0.01\\
28.01	0.01\\
29.01	0.01\\
30.01	0.01\\
31.01	0.01\\
32.01	0.01\\
33.01	0.01\\
34.01	0.01\\
35.01	0.01\\
36.01	0.01\\
37.01	0.01\\
38.01	0.01\\
39.01	0.01\\
40.01	0.01\\
41.01	0.01\\
42.01	0.01\\
43.01	0.01\\
44.01	0.01\\
45.01	0.01\\
46.01	0.01\\
47.01	0.01\\
48.01	0.01\\
49.01	0.01\\
50.01	0.01\\
51.01	0.01\\
52.01	0.01\\
53.01	0.01\\
54.01	0.01\\
55.01	0.01\\
56.01	0.01\\
57.01	0.01\\
58.01	0.01\\
59.01	0.01\\
60.01	0.01\\
61.01	0.01\\
62.01	0.01\\
63.01	0.01\\
64.01	0.01\\
65.01	0.01\\
66.01	0.01\\
67.01	0.01\\
68.01	0.01\\
69.01	0.01\\
70.01	0.01\\
71.01	0.01\\
72.01	0.01\\
73.01	0.01\\
74.01	0.01\\
75.01	0.01\\
76.01	0.01\\
77.01	0.01\\
78.01	0.01\\
79.01	0.01\\
80.01	0.01\\
81.01	0.01\\
82.01	0.01\\
83.01	0.01\\
84.01	0.01\\
85.01	0.01\\
86.01	0.01\\
87.01	0.01\\
88.01	0.01\\
89.01	0.01\\
90.01	0.01\\
91.01	0.01\\
92.01	0.01\\
93.01	0.01\\
94.01	0.01\\
95.01	0.01\\
96.01	0.01\\
97.01	0.01\\
98.01	0.01\\
99.01	0.01\\
100.01	0.01\\
101.01	0.01\\
102.01	0.01\\
103.01	0.01\\
104.01	0.01\\
105.01	0.01\\
106.01	0.01\\
107.01	0.01\\
108.01	0.01\\
109.01	0.01\\
110.01	0.01\\
111.01	0.01\\
112.01	0.01\\
113.01	0.01\\
114.01	0.01\\
115.01	0.01\\
116.01	0.01\\
117.01	0.01\\
118.01	0.01\\
119.01	0.01\\
120.01	0.01\\
121.01	0.01\\
122.01	0.01\\
123.01	0.01\\
124.01	0.01\\
125.01	0.01\\
126.01	0.01\\
127.01	0.01\\
128.01	0.01\\
129.01	0.01\\
130.01	0.01\\
131.01	0.01\\
132.01	0.01\\
133.01	0.01\\
134.01	0.01\\
135.01	0.01\\
136.01	0.01\\
137.01	0.01\\
138.01	0.01\\
139.01	0.01\\
140.01	0.01\\
141.01	0.01\\
142.01	0.01\\
143.01	0.01\\
144.01	0.01\\
145.01	0.01\\
146.01	0.01\\
147.01	0.01\\
148.01	0.01\\
149.01	0.01\\
150.01	0.01\\
151.01	0.01\\
152.01	0.01\\
153.01	0.01\\
154.01	0.01\\
155.01	0.01\\
156.01	0.01\\
157.01	0.01\\
158.01	0.01\\
159.01	0.01\\
160.01	0.01\\
161.01	0.01\\
162.01	0.01\\
163.01	0.01\\
164.01	0.01\\
165.01	0.01\\
166.01	0.01\\
167.01	0.01\\
168.01	0.01\\
169.01	0.01\\
170.01	0.01\\
171.01	0.01\\
172.01	0.01\\
173.01	0.01\\
174.01	0.01\\
175.01	0.01\\
176.01	0.01\\
177.01	0.01\\
178.01	0.01\\
179.01	0.01\\
180.01	0.01\\
181.01	0.01\\
182.01	0.01\\
183.01	0.01\\
184.01	0.01\\
185.01	0.01\\
186.01	0.01\\
187.01	0.01\\
188.01	0.01\\
189.01	0.01\\
190.01	0.01\\
191.01	0.01\\
192.01	0.01\\
193.01	0.01\\
194.01	0.01\\
195.01	0.01\\
196.01	0.01\\
197.01	0.01\\
198.01	0.01\\
199.01	0.01\\
200.01	0.01\\
201.01	0.01\\
202.01	0.01\\
203.01	0.01\\
204.01	0.01\\
205.01	0.01\\
206.01	0.01\\
207.01	0.01\\
208.01	0.01\\
209.01	0.01\\
210.01	0.01\\
211.01	0.01\\
212.01	0.01\\
213.01	0.01\\
214.01	0.01\\
215.01	0.01\\
216.01	0.01\\
217.01	0.01\\
218.01	0.01\\
219.01	0.01\\
220.01	0.01\\
221.01	0.01\\
222.01	0.01\\
223.01	0.01\\
224.01	0.01\\
225.01	0.01\\
226.01	0.01\\
227.01	0.01\\
228.01	0.01\\
229.01	0.01\\
230.01	0.01\\
231.01	0.01\\
232.01	0.01\\
233.01	0.01\\
234.01	0.01\\
235.01	0.01\\
236.01	0.01\\
237.01	0.01\\
238.01	0.01\\
239.01	0.01\\
240.01	0.01\\
241.01	0.01\\
242.01	0.01\\
243.01	0.01\\
244.01	0.01\\
245.01	0.01\\
246.01	0.01\\
247.01	0.01\\
248.01	0.01\\
249.01	0.01\\
250.01	0.01\\
251.01	0.01\\
252.01	0.01\\
253.01	0.01\\
254.01	0.01\\
255.01	0.01\\
256.01	0.01\\
257.01	0.01\\
258.01	0.01\\
259.01	0.01\\
260.01	0.01\\
261.01	0.01\\
262.01	0.01\\
263.01	0.01\\
264.01	0.01\\
265.01	0.01\\
266.01	0.01\\
267.01	0.01\\
268.01	0.01\\
269.01	0.01\\
270.01	0.01\\
271.01	0.01\\
272.01	0.01\\
273.01	0.01\\
274.01	0.01\\
275.01	0.01\\
276.01	0.01\\
277.01	0.01\\
278.01	0.01\\
279.01	0.01\\
280.01	0.01\\
281.01	0.01\\
282.01	0.01\\
283.01	0.01\\
284.01	0.01\\
285.01	0.01\\
286.01	0.01\\
287.01	0.01\\
288.01	0.01\\
289.01	0.01\\
290.01	0.01\\
291.01	0.01\\
292.01	0.01\\
293.01	0.01\\
294.01	0.01\\
295.01	0.01\\
296.01	0.01\\
297.01	0.01\\
298.01	0.01\\
299.01	0.01\\
300.01	0.01\\
301.01	0.01\\
302.01	0.01\\
303.01	0.01\\
304.01	0.01\\
305.01	0.01\\
306.01	0.01\\
307.01	0.01\\
308.01	0.01\\
309.01	0.01\\
310.01	0.01\\
311.01	0.01\\
312.01	0.01\\
313.01	0.01\\
314.01	0.01\\
315.01	0.01\\
316.01	0.01\\
317.01	0.01\\
318.01	0.01\\
319.01	0.01\\
320.01	0.01\\
321.01	0.01\\
322.01	0.01\\
323.01	0.01\\
324.01	0.01\\
325.01	0.01\\
326.01	0.01\\
327.01	0.01\\
328.01	0.01\\
329.01	0.01\\
330.01	0.01\\
331.01	0.01\\
332.01	0.01\\
333.01	0.01\\
334.01	0.01\\
335.01	0.01\\
336.01	0.01\\
337.01	0.01\\
338.01	0.01\\
339.01	0.01\\
340.01	0.01\\
341.01	0.01\\
342.01	0.01\\
343.01	0.01\\
344.01	0.01\\
345.01	0.01\\
346.01	0.01\\
347.01	0.01\\
348.01	0.01\\
349.01	0.01\\
350.01	0.01\\
351.01	0.01\\
352.01	0.01\\
353.01	0.01\\
354.01	0.01\\
355.01	0.01\\
356.01	0.01\\
357.01	0.01\\
358.01	0.01\\
359.01	0.01\\
360.01	0.01\\
361.01	0.01\\
362.01	0.01\\
363.01	0.01\\
364.01	0.01\\
365.01	0.01\\
366.01	0.01\\
367.01	0.01\\
368.01	0.01\\
369.01	0.01\\
370.01	0.01\\
371.01	0.01\\
372.01	0.01\\
373.01	0.01\\
374.01	0.01\\
375.01	0.01\\
376.01	0.01\\
377.01	0.01\\
378.01	0.01\\
379.01	0.01\\
380.01	0.01\\
381.01	0.01\\
382.01	0.01\\
383.01	0.01\\
384.01	0.01\\
385.01	0.01\\
386.01	0.01\\
387.01	0.01\\
388.01	0.01\\
389.01	0.01\\
390.01	0.01\\
391.01	0.01\\
392.01	0.01\\
393.01	0.01\\
394.01	0.01\\
395.01	0.01\\
396.01	0.01\\
397.01	0.01\\
398.01	0.01\\
399.01	0.01\\
400.01	0.01\\
401.01	0.01\\
402.01	0.01\\
403.01	0.01\\
404.01	0.01\\
405.01	0.01\\
406.01	0.01\\
407.01	0.01\\
408.01	0.01\\
409.01	0.01\\
410.01	0.01\\
411.01	0.01\\
412.01	0.01\\
413.01	0.01\\
414.01	0.01\\
415.01	0.01\\
416.01	0.01\\
417.01	0.01\\
418.01	0.01\\
419.01	0.01\\
420.01	0.01\\
421.01	0.01\\
422.01	0.01\\
423.01	0.01\\
424.01	0.01\\
425.01	0.01\\
426.01	0.01\\
427.01	0.01\\
428.01	0.01\\
429.01	0.01\\
430.01	0.01\\
431.01	0.01\\
432.01	0.01\\
433.01	0.01\\
434.01	0.01\\
435.01	0.01\\
436.01	0.01\\
437.01	0.01\\
438.01	0.01\\
439.01	0.01\\
440.01	0.01\\
441.01	0.01\\
442.01	0.01\\
443.01	0.01\\
444.01	0.01\\
445.01	0.01\\
446.01	0.01\\
447.01	0.01\\
448.01	0.01\\
449.01	0.01\\
450.01	0.01\\
451.01	0.01\\
452.01	0.01\\
453.01	0.01\\
454.01	0.01\\
455.01	0.01\\
456.01	0.01\\
457.01	0.01\\
458.01	0.01\\
459.01	0.01\\
460.01	0.01\\
461.01	0.01\\
462.01	0.01\\
463.01	0.01\\
464.01	0.01\\
465.01	0.01\\
466.01	0.01\\
467.01	0.01\\
468.01	0.01\\
469.01	0.01\\
470.01	0.01\\
471.01	0.01\\
472.01	0.01\\
473.01	0.01\\
474.01	0.01\\
475.01	0.01\\
476.01	0.01\\
477.01	0.01\\
478.01	0.01\\
479.01	0.01\\
480.01	0.01\\
481.01	0.01\\
482.01	0.01\\
483.01	0.01\\
484.01	0.01\\
485.01	0.01\\
486.01	0.01\\
487.01	0.01\\
488.01	0.01\\
489.01	0.01\\
490.01	0.01\\
491.01	0.01\\
492.01	0.01\\
493.01	0.01\\
494.01	0.01\\
495.01	0.01\\
496.01	0.01\\
497.01	0.01\\
498.01	0.01\\
499.01	0.01\\
500.01	0.01\\
501.01	0.01\\
502.01	0.01\\
503.01	0.01\\
504.01	0.01\\
505.01	0.01\\
506.01	0.01\\
507.01	0.01\\
508.01	0.01\\
509.01	0.01\\
510.01	0.01\\
511.01	0.01\\
512.01	0.01\\
513.01	0.01\\
514.01	0.01\\
515.01	0.01\\
516.01	0.01\\
517.01	0.01\\
518.01	0.01\\
519.01	0.01\\
520.01	0.01\\
521.01	0.01\\
522.01	0.01\\
523.01	0.01\\
524.01	0.01\\
525.01	0.01\\
526.01	0.01\\
527.01	0.01\\
528.01	0.01\\
529.01	0.01\\
530.01	0.01\\
531.01	0.01\\
532.01	0.01\\
533.01	0.01\\
534.01	0.01\\
535.01	0.01\\
536.01	0.01\\
537.01	0.01\\
538.01	0.01\\
539.01	0.01\\
540.01	0.01\\
541.01	0.01\\
542.01	0.01\\
543.01	0.01\\
544.01	0.01\\
545.01	0.01\\
546.01	0.01\\
547.01	0.01\\
548.01	0.01\\
549.01	0.01\\
550.01	0.01\\
551.01	0.01\\
552.01	0.01\\
553.01	0.01\\
554.01	0.01\\
555.01	0.01\\
556.01	0.01\\
557.01	0.01\\
558.01	0.01\\
559.01	0.01\\
560.01	0.01\\
561.01	0.01\\
562.01	0.01\\
563.01	0.01\\
564.01	0.01\\
565.01	0.01\\
566.01	0.01\\
567.01	0.01\\
568.01	0.01\\
569.01	0.01\\
570.01	0.01\\
571.01	0.01\\
572.01	0.01\\
573.01	0.01\\
574.01	0.01\\
575.01	0.01\\
576.01	0.01\\
577.01	0.01\\
578.01	0.01\\
579.01	0.01\\
580.01	0.01\\
581.01	0.01\\
582.01	0.01\\
583.01	0.01\\
584.01	0.01\\
585.01	0.01\\
586.01	0.01\\
587.01	0.01\\
588.01	0.01\\
589.01	0.01\\
590.01	0.01\\
591.01	0.01\\
592.01	0.01\\
593.01	0.01\\
594.01	0.01\\
595.01	0.01\\
596.01	0.01\\
597.01	0.01\\
598.01	0.01\\
599.01	0.01\\
599.02	0.01\\
599.03	0.01\\
599.04	0.01\\
599.05	0.01\\
599.06	0.01\\
599.07	0.01\\
599.08	0.01\\
599.09	0.01\\
599.1	0.01\\
599.11	0.01\\
599.12	0.01\\
599.13	0.01\\
599.14	0.01\\
599.15	0.01\\
599.16	0.01\\
599.17	0.01\\
599.18	0.01\\
599.19	0.01\\
599.2	0.01\\
599.21	0.01\\
599.22	0.01\\
599.23	0.01\\
599.24	0.01\\
599.25	0.01\\
599.26	0.01\\
599.27	0.01\\
599.28	0.01\\
599.29	0.01\\
599.3	0.01\\
599.31	0.01\\
599.32	0.01\\
599.33	0.01\\
599.34	0.01\\
599.35	0.01\\
599.36	0.01\\
599.37	0.01\\
599.38	0.01\\
599.39	0.01\\
599.4	0.01\\
599.41	0.01\\
599.42	0.01\\
599.43	0.01\\
599.44	0.01\\
599.45	0.01\\
599.46	0.01\\
599.47	0.01\\
599.48	0.01\\
599.49	0.01\\
599.5	0.01\\
599.51	0.01\\
599.52	0.01\\
599.53	0.01\\
599.54	0.01\\
599.55	0.01\\
599.56	0.01\\
599.57	0.01\\
599.58	0.01\\
599.59	0.01\\
599.6	0.01\\
599.61	0.01\\
599.62	0.01\\
599.63	0.01\\
599.64	0.01\\
599.65	0.01\\
599.66	0.01\\
599.67	0.01\\
599.68	0.01\\
599.69	0.01\\
599.7	0.01\\
599.71	0.01\\
599.72	0.01\\
599.73	0.01\\
599.74	0.01\\
599.75	0.01\\
599.76	0.01\\
599.77	0.01\\
599.78	0.01\\
599.79	0.01\\
599.8	0.01\\
599.81	0.01\\
599.82	0.01\\
599.83	0.01\\
599.84	0.01\\
599.85	0.01\\
599.86	0.01\\
599.87	0.01\\
599.88	0.01\\
599.89	0.01\\
599.9	0.01\\
599.91	0.01\\
599.92	0.01\\
599.93	0.01\\
599.94	0.01\\
599.95	0.01\\
599.96	0.01\\
599.97	0.01\\
599.98	0.01\\
599.99	0.01\\
600	0.01\\
};
\addplot [color=mycolor4,solid,forget plot]
  table[row sep=crcr]{%
0.01	0.01\\
1.01	0.01\\
2.01	0.01\\
3.01	0.01\\
4.01	0.01\\
5.01	0.01\\
6.01	0.01\\
7.01	0.01\\
8.01	0.01\\
9.01	0.01\\
10.01	0.01\\
11.01	0.01\\
12.01	0.01\\
13.01	0.01\\
14.01	0.01\\
15.01	0.01\\
16.01	0.01\\
17.01	0.01\\
18.01	0.01\\
19.01	0.01\\
20.01	0.01\\
21.01	0.01\\
22.01	0.01\\
23.01	0.01\\
24.01	0.01\\
25.01	0.01\\
26.01	0.01\\
27.01	0.01\\
28.01	0.01\\
29.01	0.01\\
30.01	0.01\\
31.01	0.01\\
32.01	0.01\\
33.01	0.01\\
34.01	0.01\\
35.01	0.01\\
36.01	0.01\\
37.01	0.01\\
38.01	0.01\\
39.01	0.01\\
40.01	0.01\\
41.01	0.01\\
42.01	0.01\\
43.01	0.01\\
44.01	0.01\\
45.01	0.01\\
46.01	0.01\\
47.01	0.01\\
48.01	0.01\\
49.01	0.01\\
50.01	0.01\\
51.01	0.01\\
52.01	0.01\\
53.01	0.01\\
54.01	0.01\\
55.01	0.01\\
56.01	0.01\\
57.01	0.01\\
58.01	0.01\\
59.01	0.01\\
60.01	0.01\\
61.01	0.01\\
62.01	0.01\\
63.01	0.01\\
64.01	0.01\\
65.01	0.01\\
66.01	0.01\\
67.01	0.01\\
68.01	0.01\\
69.01	0.01\\
70.01	0.01\\
71.01	0.01\\
72.01	0.01\\
73.01	0.01\\
74.01	0.01\\
75.01	0.01\\
76.01	0.01\\
77.01	0.01\\
78.01	0.01\\
79.01	0.01\\
80.01	0.01\\
81.01	0.01\\
82.01	0.01\\
83.01	0.01\\
84.01	0.01\\
85.01	0.01\\
86.01	0.01\\
87.01	0.01\\
88.01	0.01\\
89.01	0.01\\
90.01	0.01\\
91.01	0.01\\
92.01	0.01\\
93.01	0.01\\
94.01	0.01\\
95.01	0.01\\
96.01	0.01\\
97.01	0.01\\
98.01	0.01\\
99.01	0.01\\
100.01	0.01\\
101.01	0.01\\
102.01	0.01\\
103.01	0.01\\
104.01	0.01\\
105.01	0.01\\
106.01	0.01\\
107.01	0.01\\
108.01	0.01\\
109.01	0.01\\
110.01	0.01\\
111.01	0.01\\
112.01	0.01\\
113.01	0.01\\
114.01	0.01\\
115.01	0.01\\
116.01	0.01\\
117.01	0.01\\
118.01	0.01\\
119.01	0.01\\
120.01	0.01\\
121.01	0.01\\
122.01	0.01\\
123.01	0.01\\
124.01	0.01\\
125.01	0.01\\
126.01	0.01\\
127.01	0.01\\
128.01	0.01\\
129.01	0.01\\
130.01	0.01\\
131.01	0.01\\
132.01	0.01\\
133.01	0.01\\
134.01	0.01\\
135.01	0.01\\
136.01	0.01\\
137.01	0.01\\
138.01	0.01\\
139.01	0.01\\
140.01	0.01\\
141.01	0.01\\
142.01	0.01\\
143.01	0.01\\
144.01	0.01\\
145.01	0.01\\
146.01	0.01\\
147.01	0.01\\
148.01	0.01\\
149.01	0.01\\
150.01	0.01\\
151.01	0.01\\
152.01	0.01\\
153.01	0.01\\
154.01	0.01\\
155.01	0.01\\
156.01	0.01\\
157.01	0.01\\
158.01	0.01\\
159.01	0.01\\
160.01	0.01\\
161.01	0.01\\
162.01	0.01\\
163.01	0.01\\
164.01	0.01\\
165.01	0.01\\
166.01	0.01\\
167.01	0.01\\
168.01	0.01\\
169.01	0.01\\
170.01	0.01\\
171.01	0.01\\
172.01	0.01\\
173.01	0.01\\
174.01	0.01\\
175.01	0.01\\
176.01	0.01\\
177.01	0.01\\
178.01	0.01\\
179.01	0.01\\
180.01	0.01\\
181.01	0.01\\
182.01	0.01\\
183.01	0.01\\
184.01	0.01\\
185.01	0.01\\
186.01	0.01\\
187.01	0.01\\
188.01	0.01\\
189.01	0.01\\
190.01	0.01\\
191.01	0.01\\
192.01	0.01\\
193.01	0.01\\
194.01	0.01\\
195.01	0.01\\
196.01	0.01\\
197.01	0.01\\
198.01	0.01\\
199.01	0.01\\
200.01	0.01\\
201.01	0.01\\
202.01	0.01\\
203.01	0.01\\
204.01	0.01\\
205.01	0.01\\
206.01	0.01\\
207.01	0.01\\
208.01	0.01\\
209.01	0.01\\
210.01	0.01\\
211.01	0.01\\
212.01	0.01\\
213.01	0.01\\
214.01	0.01\\
215.01	0.01\\
216.01	0.01\\
217.01	0.01\\
218.01	0.01\\
219.01	0.01\\
220.01	0.01\\
221.01	0.01\\
222.01	0.01\\
223.01	0.01\\
224.01	0.01\\
225.01	0.01\\
226.01	0.01\\
227.01	0.01\\
228.01	0.01\\
229.01	0.01\\
230.01	0.01\\
231.01	0.01\\
232.01	0.01\\
233.01	0.01\\
234.01	0.01\\
235.01	0.01\\
236.01	0.01\\
237.01	0.01\\
238.01	0.01\\
239.01	0.01\\
240.01	0.01\\
241.01	0.01\\
242.01	0.01\\
243.01	0.01\\
244.01	0.01\\
245.01	0.01\\
246.01	0.01\\
247.01	0.01\\
248.01	0.01\\
249.01	0.01\\
250.01	0.01\\
251.01	0.01\\
252.01	0.01\\
253.01	0.01\\
254.01	0.01\\
255.01	0.01\\
256.01	0.01\\
257.01	0.01\\
258.01	0.01\\
259.01	0.01\\
260.01	0.01\\
261.01	0.01\\
262.01	0.01\\
263.01	0.01\\
264.01	0.01\\
265.01	0.01\\
266.01	0.01\\
267.01	0.01\\
268.01	0.01\\
269.01	0.01\\
270.01	0.01\\
271.01	0.01\\
272.01	0.01\\
273.01	0.01\\
274.01	0.01\\
275.01	0.01\\
276.01	0.01\\
277.01	0.01\\
278.01	0.01\\
279.01	0.01\\
280.01	0.01\\
281.01	0.01\\
282.01	0.01\\
283.01	0.01\\
284.01	0.01\\
285.01	0.01\\
286.01	0.01\\
287.01	0.01\\
288.01	0.01\\
289.01	0.01\\
290.01	0.01\\
291.01	0.01\\
292.01	0.01\\
293.01	0.01\\
294.01	0.01\\
295.01	0.01\\
296.01	0.01\\
297.01	0.01\\
298.01	0.01\\
299.01	0.01\\
300.01	0.01\\
301.01	0.01\\
302.01	0.01\\
303.01	0.01\\
304.01	0.01\\
305.01	0.01\\
306.01	0.01\\
307.01	0.01\\
308.01	0.01\\
309.01	0.01\\
310.01	0.01\\
311.01	0.01\\
312.01	0.01\\
313.01	0.01\\
314.01	0.01\\
315.01	0.01\\
316.01	0.01\\
317.01	0.01\\
318.01	0.01\\
319.01	0.01\\
320.01	0.01\\
321.01	0.01\\
322.01	0.01\\
323.01	0.01\\
324.01	0.01\\
325.01	0.01\\
326.01	0.01\\
327.01	0.01\\
328.01	0.01\\
329.01	0.01\\
330.01	0.01\\
331.01	0.01\\
332.01	0.01\\
333.01	0.01\\
334.01	0.01\\
335.01	0.01\\
336.01	0.01\\
337.01	0.01\\
338.01	0.01\\
339.01	0.01\\
340.01	0.01\\
341.01	0.01\\
342.01	0.01\\
343.01	0.01\\
344.01	0.01\\
345.01	0.01\\
346.01	0.01\\
347.01	0.01\\
348.01	0.01\\
349.01	0.01\\
350.01	0.01\\
351.01	0.01\\
352.01	0.01\\
353.01	0.01\\
354.01	0.01\\
355.01	0.01\\
356.01	0.01\\
357.01	0.01\\
358.01	0.01\\
359.01	0.01\\
360.01	0.01\\
361.01	0.01\\
362.01	0.01\\
363.01	0.01\\
364.01	0.01\\
365.01	0.01\\
366.01	0.01\\
367.01	0.01\\
368.01	0.01\\
369.01	0.01\\
370.01	0.01\\
371.01	0.01\\
372.01	0.01\\
373.01	0.01\\
374.01	0.01\\
375.01	0.01\\
376.01	0.01\\
377.01	0.01\\
378.01	0.01\\
379.01	0.01\\
380.01	0.01\\
381.01	0.01\\
382.01	0.01\\
383.01	0.01\\
384.01	0.01\\
385.01	0.01\\
386.01	0.01\\
387.01	0.01\\
388.01	0.01\\
389.01	0.01\\
390.01	0.01\\
391.01	0.01\\
392.01	0.01\\
393.01	0.01\\
394.01	0.01\\
395.01	0.01\\
396.01	0.01\\
397.01	0.01\\
398.01	0.01\\
399.01	0.01\\
400.01	0.01\\
401.01	0.01\\
402.01	0.01\\
403.01	0.01\\
404.01	0.01\\
405.01	0.01\\
406.01	0.01\\
407.01	0.01\\
408.01	0.01\\
409.01	0.01\\
410.01	0.01\\
411.01	0.01\\
412.01	0.01\\
413.01	0.01\\
414.01	0.01\\
415.01	0.01\\
416.01	0.01\\
417.01	0.01\\
418.01	0.01\\
419.01	0.01\\
420.01	0.01\\
421.01	0.01\\
422.01	0.01\\
423.01	0.01\\
424.01	0.01\\
425.01	0.01\\
426.01	0.01\\
427.01	0.01\\
428.01	0.01\\
429.01	0.01\\
430.01	0.01\\
431.01	0.01\\
432.01	0.01\\
433.01	0.01\\
434.01	0.01\\
435.01	0.01\\
436.01	0.01\\
437.01	0.01\\
438.01	0.01\\
439.01	0.01\\
440.01	0.01\\
441.01	0.01\\
442.01	0.01\\
443.01	0.01\\
444.01	0.01\\
445.01	0.01\\
446.01	0.01\\
447.01	0.01\\
448.01	0.01\\
449.01	0.01\\
450.01	0.01\\
451.01	0.01\\
452.01	0.01\\
453.01	0.01\\
454.01	0.01\\
455.01	0.01\\
456.01	0.01\\
457.01	0.01\\
458.01	0.01\\
459.01	0.01\\
460.01	0.01\\
461.01	0.01\\
462.01	0.01\\
463.01	0.01\\
464.01	0.01\\
465.01	0.01\\
466.01	0.01\\
467.01	0.01\\
468.01	0.01\\
469.01	0.01\\
470.01	0.01\\
471.01	0.01\\
472.01	0.01\\
473.01	0.01\\
474.01	0.01\\
475.01	0.01\\
476.01	0.01\\
477.01	0.01\\
478.01	0.01\\
479.01	0.01\\
480.01	0.01\\
481.01	0.01\\
482.01	0.01\\
483.01	0.01\\
484.01	0.01\\
485.01	0.01\\
486.01	0.01\\
487.01	0.01\\
488.01	0.01\\
489.01	0.01\\
490.01	0.01\\
491.01	0.01\\
492.01	0.01\\
493.01	0.01\\
494.01	0.01\\
495.01	0.01\\
496.01	0.01\\
497.01	0.01\\
498.01	0.01\\
499.01	0.01\\
500.01	0.01\\
501.01	0.01\\
502.01	0.01\\
503.01	0.01\\
504.01	0.01\\
505.01	0.01\\
506.01	0.01\\
507.01	0.01\\
508.01	0.01\\
509.01	0.01\\
510.01	0.01\\
511.01	0.01\\
512.01	0.01\\
513.01	0.01\\
514.01	0.01\\
515.01	0.01\\
516.01	0.01\\
517.01	0.01\\
518.01	0.01\\
519.01	0.01\\
520.01	0.01\\
521.01	0.01\\
522.01	0.01\\
523.01	0.01\\
524.01	0.01\\
525.01	0.01\\
526.01	0.01\\
527.01	0.01\\
528.01	0.01\\
529.01	0.01\\
530.01	0.01\\
531.01	0.01\\
532.01	0.01\\
533.01	0.01\\
534.01	0.01\\
535.01	0.01\\
536.01	0.01\\
537.01	0.01\\
538.01	0.01\\
539.01	0.01\\
540.01	0.01\\
541.01	0.01\\
542.01	0.01\\
543.01	0.01\\
544.01	0.01\\
545.01	0.01\\
546.01	0.01\\
547.01	0.01\\
548.01	0.01\\
549.01	0.01\\
550.01	0.01\\
551.01	0.01\\
552.01	0.01\\
553.01	0.01\\
554.01	0.01\\
555.01	0.01\\
556.01	0.01\\
557.01	0.01\\
558.01	0.01\\
559.01	0.01\\
560.01	0.01\\
561.01	0.01\\
562.01	0.01\\
563.01	0.01\\
564.01	0.01\\
565.01	0.01\\
566.01	0.01\\
567.01	0.01\\
568.01	0.01\\
569.01	0.01\\
570.01	0.01\\
571.01	0.01\\
572.01	0.01\\
573.01	0.01\\
574.01	0.01\\
575.01	0.01\\
576.01	0.01\\
577.01	0.01\\
578.01	0.01\\
579.01	0.01\\
580.01	0.01\\
581.01	0.01\\
582.01	0.01\\
583.01	0.01\\
584.01	0.01\\
585.01	0.01\\
586.01	0.01\\
587.01	0.01\\
588.01	0.01\\
589.01	0.01\\
590.01	0.01\\
591.01	0.01\\
592.01	0.01\\
593.01	0.01\\
594.01	0.01\\
595.01	0.01\\
596.01	0.01\\
597.01	0.01\\
598.01	0.01\\
599.01	0.01\\
599.02	0.01\\
599.03	0.01\\
599.04	0.01\\
599.05	0.01\\
599.06	0.01\\
599.07	0.01\\
599.08	0.01\\
599.09	0.01\\
599.1	0.01\\
599.11	0.01\\
599.12	0.01\\
599.13	0.01\\
599.14	0.01\\
599.15	0.01\\
599.16	0.01\\
599.17	0.01\\
599.18	0.01\\
599.19	0.01\\
599.2	0.01\\
599.21	0.01\\
599.22	0.01\\
599.23	0.01\\
599.24	0.01\\
599.25	0.01\\
599.26	0.01\\
599.27	0.01\\
599.28	0.01\\
599.29	0.01\\
599.3	0.01\\
599.31	0.01\\
599.32	0.01\\
599.33	0.01\\
599.34	0.01\\
599.35	0.01\\
599.36	0.01\\
599.37	0.01\\
599.38	0.01\\
599.39	0.01\\
599.4	0.01\\
599.41	0.01\\
599.42	0.01\\
599.43	0.01\\
599.44	0.01\\
599.45	0.01\\
599.46	0.01\\
599.47	0.01\\
599.48	0.01\\
599.49	0.01\\
599.5	0.01\\
599.51	0.01\\
599.52	0.01\\
599.53	0.01\\
599.54	0.01\\
599.55	0.01\\
599.56	0.01\\
599.57	0.01\\
599.58	0.01\\
599.59	0.01\\
599.6	0.01\\
599.61	0.01\\
599.62	0.01\\
599.63	0.01\\
599.64	0.01\\
599.65	0.01\\
599.66	0.01\\
599.67	0.01\\
599.68	0.01\\
599.69	0.01\\
599.7	0.01\\
599.71	0.01\\
599.72	0.01\\
599.73	0.01\\
599.74	0.01\\
599.75	0.01\\
599.76	0.01\\
599.77	0.01\\
599.78	0.01\\
599.79	0.01\\
599.8	0.01\\
599.81	0.01\\
599.82	0.01\\
599.83	0.01\\
599.84	0.01\\
599.85	0.01\\
599.86	0.01\\
599.87	0.01\\
599.88	0.01\\
599.89	0.01\\
599.9	0.01\\
599.91	0.01\\
599.92	0.01\\
599.93	0.01\\
599.94	0.01\\
599.95	0.01\\
599.96	0.01\\
599.97	0.01\\
599.98	0.01\\
599.99	0.01\\
600	0.01\\
};
\addplot [color=mycolor5,solid,forget plot]
  table[row sep=crcr]{%
0.01	0.01\\
1.01	0.01\\
2.01	0.01\\
3.01	0.01\\
4.01	0.01\\
5.01	0.01\\
6.01	0.01\\
7.01	0.01\\
8.01	0.01\\
9.01	0.01\\
10.01	0.01\\
11.01	0.01\\
12.01	0.01\\
13.01	0.01\\
14.01	0.01\\
15.01	0.01\\
16.01	0.01\\
17.01	0.01\\
18.01	0.01\\
19.01	0.01\\
20.01	0.01\\
21.01	0.01\\
22.01	0.01\\
23.01	0.01\\
24.01	0.01\\
25.01	0.01\\
26.01	0.01\\
27.01	0.01\\
28.01	0.01\\
29.01	0.01\\
30.01	0.01\\
31.01	0.01\\
32.01	0.01\\
33.01	0.01\\
34.01	0.01\\
35.01	0.01\\
36.01	0.01\\
37.01	0.01\\
38.01	0.01\\
39.01	0.01\\
40.01	0.01\\
41.01	0.01\\
42.01	0.01\\
43.01	0.01\\
44.01	0.01\\
45.01	0.01\\
46.01	0.01\\
47.01	0.01\\
48.01	0.01\\
49.01	0.01\\
50.01	0.01\\
51.01	0.01\\
52.01	0.01\\
53.01	0.01\\
54.01	0.01\\
55.01	0.01\\
56.01	0.01\\
57.01	0.01\\
58.01	0.01\\
59.01	0.01\\
60.01	0.01\\
61.01	0.01\\
62.01	0.01\\
63.01	0.01\\
64.01	0.01\\
65.01	0.01\\
66.01	0.01\\
67.01	0.01\\
68.01	0.01\\
69.01	0.01\\
70.01	0.01\\
71.01	0.01\\
72.01	0.01\\
73.01	0.01\\
74.01	0.01\\
75.01	0.01\\
76.01	0.01\\
77.01	0.01\\
78.01	0.01\\
79.01	0.01\\
80.01	0.01\\
81.01	0.01\\
82.01	0.01\\
83.01	0.01\\
84.01	0.01\\
85.01	0.01\\
86.01	0.01\\
87.01	0.01\\
88.01	0.01\\
89.01	0.01\\
90.01	0.01\\
91.01	0.01\\
92.01	0.01\\
93.01	0.01\\
94.01	0.01\\
95.01	0.01\\
96.01	0.01\\
97.01	0.01\\
98.01	0.01\\
99.01	0.01\\
100.01	0.01\\
101.01	0.01\\
102.01	0.01\\
103.01	0.01\\
104.01	0.01\\
105.01	0.01\\
106.01	0.01\\
107.01	0.01\\
108.01	0.01\\
109.01	0.01\\
110.01	0.01\\
111.01	0.01\\
112.01	0.01\\
113.01	0.01\\
114.01	0.01\\
115.01	0.01\\
116.01	0.01\\
117.01	0.01\\
118.01	0.01\\
119.01	0.01\\
120.01	0.01\\
121.01	0.01\\
122.01	0.01\\
123.01	0.01\\
124.01	0.01\\
125.01	0.01\\
126.01	0.01\\
127.01	0.01\\
128.01	0.01\\
129.01	0.01\\
130.01	0.01\\
131.01	0.01\\
132.01	0.01\\
133.01	0.01\\
134.01	0.01\\
135.01	0.01\\
136.01	0.01\\
137.01	0.01\\
138.01	0.01\\
139.01	0.01\\
140.01	0.01\\
141.01	0.01\\
142.01	0.01\\
143.01	0.01\\
144.01	0.01\\
145.01	0.01\\
146.01	0.01\\
147.01	0.01\\
148.01	0.01\\
149.01	0.01\\
150.01	0.01\\
151.01	0.01\\
152.01	0.01\\
153.01	0.01\\
154.01	0.01\\
155.01	0.01\\
156.01	0.01\\
157.01	0.01\\
158.01	0.01\\
159.01	0.01\\
160.01	0.01\\
161.01	0.01\\
162.01	0.01\\
163.01	0.01\\
164.01	0.01\\
165.01	0.01\\
166.01	0.01\\
167.01	0.01\\
168.01	0.01\\
169.01	0.01\\
170.01	0.01\\
171.01	0.01\\
172.01	0.01\\
173.01	0.01\\
174.01	0.01\\
175.01	0.01\\
176.01	0.01\\
177.01	0.01\\
178.01	0.01\\
179.01	0.01\\
180.01	0.01\\
181.01	0.01\\
182.01	0.01\\
183.01	0.01\\
184.01	0.01\\
185.01	0.01\\
186.01	0.01\\
187.01	0.01\\
188.01	0.01\\
189.01	0.01\\
190.01	0.01\\
191.01	0.01\\
192.01	0.01\\
193.01	0.01\\
194.01	0.01\\
195.01	0.01\\
196.01	0.01\\
197.01	0.01\\
198.01	0.01\\
199.01	0.01\\
200.01	0.01\\
201.01	0.01\\
202.01	0.01\\
203.01	0.01\\
204.01	0.01\\
205.01	0.01\\
206.01	0.01\\
207.01	0.01\\
208.01	0.01\\
209.01	0.01\\
210.01	0.01\\
211.01	0.01\\
212.01	0.01\\
213.01	0.01\\
214.01	0.01\\
215.01	0.01\\
216.01	0.01\\
217.01	0.01\\
218.01	0.01\\
219.01	0.01\\
220.01	0.01\\
221.01	0.01\\
222.01	0.01\\
223.01	0.01\\
224.01	0.01\\
225.01	0.01\\
226.01	0.01\\
227.01	0.01\\
228.01	0.01\\
229.01	0.01\\
230.01	0.01\\
231.01	0.01\\
232.01	0.01\\
233.01	0.01\\
234.01	0.01\\
235.01	0.01\\
236.01	0.01\\
237.01	0.01\\
238.01	0.01\\
239.01	0.01\\
240.01	0.01\\
241.01	0.01\\
242.01	0.01\\
243.01	0.01\\
244.01	0.01\\
245.01	0.01\\
246.01	0.01\\
247.01	0.01\\
248.01	0.01\\
249.01	0.01\\
250.01	0.01\\
251.01	0.01\\
252.01	0.01\\
253.01	0.01\\
254.01	0.01\\
255.01	0.01\\
256.01	0.01\\
257.01	0.01\\
258.01	0.01\\
259.01	0.01\\
260.01	0.01\\
261.01	0.01\\
262.01	0.01\\
263.01	0.01\\
264.01	0.01\\
265.01	0.01\\
266.01	0.01\\
267.01	0.01\\
268.01	0.01\\
269.01	0.01\\
270.01	0.01\\
271.01	0.01\\
272.01	0.01\\
273.01	0.01\\
274.01	0.01\\
275.01	0.01\\
276.01	0.01\\
277.01	0.01\\
278.01	0.01\\
279.01	0.01\\
280.01	0.01\\
281.01	0.01\\
282.01	0.01\\
283.01	0.01\\
284.01	0.01\\
285.01	0.01\\
286.01	0.01\\
287.01	0.01\\
288.01	0.01\\
289.01	0.01\\
290.01	0.01\\
291.01	0.01\\
292.01	0.01\\
293.01	0.01\\
294.01	0.01\\
295.01	0.01\\
296.01	0.01\\
297.01	0.01\\
298.01	0.01\\
299.01	0.01\\
300.01	0.01\\
301.01	0.01\\
302.01	0.01\\
303.01	0.01\\
304.01	0.01\\
305.01	0.01\\
306.01	0.01\\
307.01	0.01\\
308.01	0.01\\
309.01	0.01\\
310.01	0.01\\
311.01	0.01\\
312.01	0.01\\
313.01	0.01\\
314.01	0.01\\
315.01	0.01\\
316.01	0.01\\
317.01	0.01\\
318.01	0.01\\
319.01	0.01\\
320.01	0.01\\
321.01	0.01\\
322.01	0.01\\
323.01	0.01\\
324.01	0.01\\
325.01	0.01\\
326.01	0.01\\
327.01	0.01\\
328.01	0.01\\
329.01	0.01\\
330.01	0.01\\
331.01	0.01\\
332.01	0.01\\
333.01	0.01\\
334.01	0.01\\
335.01	0.01\\
336.01	0.01\\
337.01	0.01\\
338.01	0.01\\
339.01	0.01\\
340.01	0.01\\
341.01	0.01\\
342.01	0.01\\
343.01	0.01\\
344.01	0.01\\
345.01	0.01\\
346.01	0.01\\
347.01	0.01\\
348.01	0.01\\
349.01	0.01\\
350.01	0.01\\
351.01	0.01\\
352.01	0.01\\
353.01	0.01\\
354.01	0.01\\
355.01	0.01\\
356.01	0.01\\
357.01	0.01\\
358.01	0.01\\
359.01	0.01\\
360.01	0.01\\
361.01	0.01\\
362.01	0.01\\
363.01	0.01\\
364.01	0.01\\
365.01	0.01\\
366.01	0.01\\
367.01	0.01\\
368.01	0.01\\
369.01	0.01\\
370.01	0.01\\
371.01	0.01\\
372.01	0.01\\
373.01	0.01\\
374.01	0.01\\
375.01	0.01\\
376.01	0.01\\
377.01	0.01\\
378.01	0.01\\
379.01	0.01\\
380.01	0.01\\
381.01	0.01\\
382.01	0.01\\
383.01	0.01\\
384.01	0.01\\
385.01	0.01\\
386.01	0.01\\
387.01	0.01\\
388.01	0.01\\
389.01	0.01\\
390.01	0.01\\
391.01	0.01\\
392.01	0.01\\
393.01	0.01\\
394.01	0.01\\
395.01	0.01\\
396.01	0.01\\
397.01	0.01\\
398.01	0.01\\
399.01	0.01\\
400.01	0.01\\
401.01	0.01\\
402.01	0.01\\
403.01	0.01\\
404.01	0.01\\
405.01	0.01\\
406.01	0.01\\
407.01	0.01\\
408.01	0.01\\
409.01	0.01\\
410.01	0.01\\
411.01	0.01\\
412.01	0.01\\
413.01	0.01\\
414.01	0.01\\
415.01	0.01\\
416.01	0.01\\
417.01	0.01\\
418.01	0.01\\
419.01	0.01\\
420.01	0.01\\
421.01	0.01\\
422.01	0.01\\
423.01	0.01\\
424.01	0.01\\
425.01	0.01\\
426.01	0.01\\
427.01	0.01\\
428.01	0.01\\
429.01	0.01\\
430.01	0.01\\
431.01	0.01\\
432.01	0.01\\
433.01	0.01\\
434.01	0.01\\
435.01	0.01\\
436.01	0.01\\
437.01	0.01\\
438.01	0.01\\
439.01	0.01\\
440.01	0.01\\
441.01	0.01\\
442.01	0.01\\
443.01	0.01\\
444.01	0.01\\
445.01	0.01\\
446.01	0.01\\
447.01	0.01\\
448.01	0.01\\
449.01	0.01\\
450.01	0.01\\
451.01	0.01\\
452.01	0.01\\
453.01	0.01\\
454.01	0.01\\
455.01	0.01\\
456.01	0.01\\
457.01	0.01\\
458.01	0.01\\
459.01	0.01\\
460.01	0.01\\
461.01	0.01\\
462.01	0.01\\
463.01	0.01\\
464.01	0.01\\
465.01	0.01\\
466.01	0.01\\
467.01	0.01\\
468.01	0.01\\
469.01	0.01\\
470.01	0.01\\
471.01	0.01\\
472.01	0.01\\
473.01	0.01\\
474.01	0.01\\
475.01	0.01\\
476.01	0.01\\
477.01	0.01\\
478.01	0.01\\
479.01	0.01\\
480.01	0.01\\
481.01	0.01\\
482.01	0.01\\
483.01	0.01\\
484.01	0.01\\
485.01	0.01\\
486.01	0.01\\
487.01	0.01\\
488.01	0.01\\
489.01	0.01\\
490.01	0.01\\
491.01	0.01\\
492.01	0.01\\
493.01	0.01\\
494.01	0.01\\
495.01	0.01\\
496.01	0.01\\
497.01	0.01\\
498.01	0.01\\
499.01	0.01\\
500.01	0.01\\
501.01	0.01\\
502.01	0.01\\
503.01	0.01\\
504.01	0.01\\
505.01	0.01\\
506.01	0.01\\
507.01	0.01\\
508.01	0.01\\
509.01	0.01\\
510.01	0.01\\
511.01	0.01\\
512.01	0.01\\
513.01	0.01\\
514.01	0.01\\
515.01	0.01\\
516.01	0.01\\
517.01	0.01\\
518.01	0.01\\
519.01	0.01\\
520.01	0.01\\
521.01	0.01\\
522.01	0.01\\
523.01	0.01\\
524.01	0.01\\
525.01	0.01\\
526.01	0.01\\
527.01	0.01\\
528.01	0.01\\
529.01	0.01\\
530.01	0.01\\
531.01	0.01\\
532.01	0.01\\
533.01	0.01\\
534.01	0.01\\
535.01	0.01\\
536.01	0.01\\
537.01	0.01\\
538.01	0.01\\
539.01	0.01\\
540.01	0.01\\
541.01	0.01\\
542.01	0.01\\
543.01	0.01\\
544.01	0.01\\
545.01	0.01\\
546.01	0.01\\
547.01	0.01\\
548.01	0.01\\
549.01	0.01\\
550.01	0.01\\
551.01	0.01\\
552.01	0.01\\
553.01	0.01\\
554.01	0.01\\
555.01	0.01\\
556.01	0.01\\
557.01	0.01\\
558.01	0.01\\
559.01	0.01\\
560.01	0.01\\
561.01	0.01\\
562.01	0.01\\
563.01	0.01\\
564.01	0.01\\
565.01	0.01\\
566.01	0.01\\
567.01	0.01\\
568.01	0.01\\
569.01	0.01\\
570.01	0.01\\
571.01	0.01\\
572.01	0.01\\
573.01	0.01\\
574.01	0.01\\
575.01	0.01\\
576.01	0.01\\
577.01	0.01\\
578.01	0.01\\
579.01	0.01\\
580.01	0.01\\
581.01	0.01\\
582.01	0.01\\
583.01	0.01\\
584.01	0.01\\
585.01	0.01\\
586.01	0.01\\
587.01	0.01\\
588.01	0.01\\
589.01	0.01\\
590.01	0.01\\
591.01	0.01\\
592.01	0.01\\
593.01	0.01\\
594.01	0.01\\
595.01	0.01\\
596.01	0.01\\
597.01	0.01\\
598.01	0.01\\
599.01	0.01\\
599.02	0.01\\
599.03	0.01\\
599.04	0.01\\
599.05	0.01\\
599.06	0.01\\
599.07	0.01\\
599.08	0.01\\
599.09	0.01\\
599.1	0.01\\
599.11	0.01\\
599.12	0.01\\
599.13	0.01\\
599.14	0.01\\
599.15	0.01\\
599.16	0.01\\
599.17	0.01\\
599.18	0.01\\
599.19	0.01\\
599.2	0.01\\
599.21	0.01\\
599.22	0.01\\
599.23	0.01\\
599.24	0.01\\
599.25	0.01\\
599.26	0.01\\
599.27	0.01\\
599.28	0.01\\
599.29	0.01\\
599.3	0.01\\
599.31	0.01\\
599.32	0.01\\
599.33	0.01\\
599.34	0.01\\
599.35	0.01\\
599.36	0.01\\
599.37	0.01\\
599.38	0.01\\
599.39	0.01\\
599.4	0.01\\
599.41	0.01\\
599.42	0.01\\
599.43	0.01\\
599.44	0.01\\
599.45	0.01\\
599.46	0.01\\
599.47	0.01\\
599.48	0.01\\
599.49	0.01\\
599.5	0.01\\
599.51	0.01\\
599.52	0.01\\
599.53	0.01\\
599.54	0.01\\
599.55	0.01\\
599.56	0.01\\
599.57	0.01\\
599.58	0.01\\
599.59	0.01\\
599.6	0.01\\
599.61	0.01\\
599.62	0.01\\
599.63	0.01\\
599.64	0.01\\
599.65	0.01\\
599.66	0.01\\
599.67	0.01\\
599.68	0.01\\
599.69	0.01\\
599.7	0.01\\
599.71	0.01\\
599.72	0.01\\
599.73	0.01\\
599.74	0.01\\
599.75	0.01\\
599.76	0.01\\
599.77	0.01\\
599.78	0.01\\
599.79	0.01\\
599.8	0.01\\
599.81	0.01\\
599.82	0.01\\
599.83	0.01\\
599.84	0.01\\
599.85	0.01\\
599.86	0.01\\
599.87	0.01\\
599.88	0.01\\
599.89	0.01\\
599.9	0.01\\
599.91	0.01\\
599.92	0.01\\
599.93	0.01\\
599.94	0.01\\
599.95	0.01\\
599.96	0.01\\
599.97	0.01\\
599.98	0.01\\
599.99	0.01\\
600	0.01\\
};
\addplot [color=mycolor6,solid,forget plot]
  table[row sep=crcr]{%
0.01	0.01\\
1.01	0.01\\
2.01	0.01\\
3.01	0.01\\
4.01	0.01\\
5.01	0.01\\
6.01	0.01\\
7.01	0.01\\
8.01	0.01\\
9.01	0.01\\
10.01	0.01\\
11.01	0.01\\
12.01	0.01\\
13.01	0.01\\
14.01	0.01\\
15.01	0.01\\
16.01	0.01\\
17.01	0.01\\
18.01	0.01\\
19.01	0.01\\
20.01	0.01\\
21.01	0.01\\
22.01	0.01\\
23.01	0.01\\
24.01	0.01\\
25.01	0.01\\
26.01	0.01\\
27.01	0.01\\
28.01	0.01\\
29.01	0.01\\
30.01	0.01\\
31.01	0.01\\
32.01	0.01\\
33.01	0.01\\
34.01	0.01\\
35.01	0.01\\
36.01	0.01\\
37.01	0.01\\
38.01	0.01\\
39.01	0.01\\
40.01	0.01\\
41.01	0.01\\
42.01	0.01\\
43.01	0.01\\
44.01	0.01\\
45.01	0.01\\
46.01	0.01\\
47.01	0.01\\
48.01	0.01\\
49.01	0.01\\
50.01	0.01\\
51.01	0.01\\
52.01	0.01\\
53.01	0.01\\
54.01	0.01\\
55.01	0.01\\
56.01	0.01\\
57.01	0.01\\
58.01	0.01\\
59.01	0.01\\
60.01	0.01\\
61.01	0.01\\
62.01	0.01\\
63.01	0.01\\
64.01	0.01\\
65.01	0.01\\
66.01	0.01\\
67.01	0.01\\
68.01	0.01\\
69.01	0.01\\
70.01	0.01\\
71.01	0.01\\
72.01	0.01\\
73.01	0.01\\
74.01	0.01\\
75.01	0.01\\
76.01	0.01\\
77.01	0.01\\
78.01	0.01\\
79.01	0.01\\
80.01	0.01\\
81.01	0.01\\
82.01	0.01\\
83.01	0.01\\
84.01	0.01\\
85.01	0.01\\
86.01	0.01\\
87.01	0.01\\
88.01	0.01\\
89.01	0.01\\
90.01	0.01\\
91.01	0.01\\
92.01	0.01\\
93.01	0.01\\
94.01	0.01\\
95.01	0.01\\
96.01	0.01\\
97.01	0.01\\
98.01	0.01\\
99.01	0.01\\
100.01	0.01\\
101.01	0.01\\
102.01	0.01\\
103.01	0.01\\
104.01	0.01\\
105.01	0.01\\
106.01	0.01\\
107.01	0.01\\
108.01	0.01\\
109.01	0.01\\
110.01	0.01\\
111.01	0.01\\
112.01	0.01\\
113.01	0.01\\
114.01	0.01\\
115.01	0.01\\
116.01	0.01\\
117.01	0.01\\
118.01	0.01\\
119.01	0.01\\
120.01	0.01\\
121.01	0.01\\
122.01	0.01\\
123.01	0.01\\
124.01	0.01\\
125.01	0.01\\
126.01	0.01\\
127.01	0.01\\
128.01	0.01\\
129.01	0.01\\
130.01	0.01\\
131.01	0.01\\
132.01	0.01\\
133.01	0.01\\
134.01	0.01\\
135.01	0.01\\
136.01	0.01\\
137.01	0.01\\
138.01	0.01\\
139.01	0.01\\
140.01	0.01\\
141.01	0.01\\
142.01	0.01\\
143.01	0.01\\
144.01	0.01\\
145.01	0.01\\
146.01	0.01\\
147.01	0.01\\
148.01	0.01\\
149.01	0.01\\
150.01	0.01\\
151.01	0.01\\
152.01	0.01\\
153.01	0.01\\
154.01	0.01\\
155.01	0.01\\
156.01	0.01\\
157.01	0.01\\
158.01	0.01\\
159.01	0.01\\
160.01	0.01\\
161.01	0.01\\
162.01	0.01\\
163.01	0.01\\
164.01	0.01\\
165.01	0.01\\
166.01	0.01\\
167.01	0.01\\
168.01	0.01\\
169.01	0.01\\
170.01	0.01\\
171.01	0.01\\
172.01	0.01\\
173.01	0.01\\
174.01	0.01\\
175.01	0.01\\
176.01	0.01\\
177.01	0.01\\
178.01	0.01\\
179.01	0.01\\
180.01	0.01\\
181.01	0.01\\
182.01	0.01\\
183.01	0.01\\
184.01	0.01\\
185.01	0.01\\
186.01	0.01\\
187.01	0.01\\
188.01	0.01\\
189.01	0.01\\
190.01	0.01\\
191.01	0.01\\
192.01	0.01\\
193.01	0.01\\
194.01	0.01\\
195.01	0.01\\
196.01	0.01\\
197.01	0.01\\
198.01	0.01\\
199.01	0.01\\
200.01	0.01\\
201.01	0.01\\
202.01	0.01\\
203.01	0.01\\
204.01	0.01\\
205.01	0.01\\
206.01	0.01\\
207.01	0.01\\
208.01	0.01\\
209.01	0.01\\
210.01	0.01\\
211.01	0.01\\
212.01	0.01\\
213.01	0.01\\
214.01	0.01\\
215.01	0.01\\
216.01	0.01\\
217.01	0.01\\
218.01	0.01\\
219.01	0.01\\
220.01	0.01\\
221.01	0.01\\
222.01	0.01\\
223.01	0.01\\
224.01	0.01\\
225.01	0.01\\
226.01	0.01\\
227.01	0.01\\
228.01	0.01\\
229.01	0.01\\
230.01	0.01\\
231.01	0.01\\
232.01	0.01\\
233.01	0.01\\
234.01	0.01\\
235.01	0.01\\
236.01	0.01\\
237.01	0.01\\
238.01	0.01\\
239.01	0.01\\
240.01	0.01\\
241.01	0.01\\
242.01	0.01\\
243.01	0.01\\
244.01	0.01\\
245.01	0.01\\
246.01	0.01\\
247.01	0.01\\
248.01	0.01\\
249.01	0.01\\
250.01	0.01\\
251.01	0.01\\
252.01	0.01\\
253.01	0.01\\
254.01	0.01\\
255.01	0.01\\
256.01	0.01\\
257.01	0.01\\
258.01	0.01\\
259.01	0.01\\
260.01	0.01\\
261.01	0.01\\
262.01	0.01\\
263.01	0.01\\
264.01	0.01\\
265.01	0.01\\
266.01	0.01\\
267.01	0.01\\
268.01	0.01\\
269.01	0.01\\
270.01	0.01\\
271.01	0.01\\
272.01	0.01\\
273.01	0.01\\
274.01	0.01\\
275.01	0.01\\
276.01	0.01\\
277.01	0.01\\
278.01	0.01\\
279.01	0.01\\
280.01	0.01\\
281.01	0.01\\
282.01	0.01\\
283.01	0.01\\
284.01	0.01\\
285.01	0.01\\
286.01	0.01\\
287.01	0.01\\
288.01	0.01\\
289.01	0.01\\
290.01	0.01\\
291.01	0.01\\
292.01	0.01\\
293.01	0.01\\
294.01	0.01\\
295.01	0.01\\
296.01	0.01\\
297.01	0.01\\
298.01	0.01\\
299.01	0.01\\
300.01	0.01\\
301.01	0.01\\
302.01	0.01\\
303.01	0.01\\
304.01	0.01\\
305.01	0.01\\
306.01	0.01\\
307.01	0.01\\
308.01	0.01\\
309.01	0.01\\
310.01	0.01\\
311.01	0.01\\
312.01	0.01\\
313.01	0.01\\
314.01	0.01\\
315.01	0.01\\
316.01	0.01\\
317.01	0.01\\
318.01	0.01\\
319.01	0.01\\
320.01	0.01\\
321.01	0.01\\
322.01	0.01\\
323.01	0.01\\
324.01	0.01\\
325.01	0.01\\
326.01	0.01\\
327.01	0.01\\
328.01	0.01\\
329.01	0.01\\
330.01	0.01\\
331.01	0.01\\
332.01	0.01\\
333.01	0.01\\
334.01	0.01\\
335.01	0.01\\
336.01	0.01\\
337.01	0.01\\
338.01	0.01\\
339.01	0.01\\
340.01	0.01\\
341.01	0.01\\
342.01	0.01\\
343.01	0.01\\
344.01	0.01\\
345.01	0.01\\
346.01	0.01\\
347.01	0.01\\
348.01	0.01\\
349.01	0.01\\
350.01	0.01\\
351.01	0.01\\
352.01	0.01\\
353.01	0.01\\
354.01	0.01\\
355.01	0.01\\
356.01	0.01\\
357.01	0.01\\
358.01	0.01\\
359.01	0.01\\
360.01	0.01\\
361.01	0.01\\
362.01	0.01\\
363.01	0.01\\
364.01	0.01\\
365.01	0.01\\
366.01	0.01\\
367.01	0.01\\
368.01	0.01\\
369.01	0.01\\
370.01	0.01\\
371.01	0.01\\
372.01	0.01\\
373.01	0.01\\
374.01	0.01\\
375.01	0.01\\
376.01	0.01\\
377.01	0.01\\
378.01	0.01\\
379.01	0.01\\
380.01	0.01\\
381.01	0.01\\
382.01	0.01\\
383.01	0.01\\
384.01	0.01\\
385.01	0.01\\
386.01	0.01\\
387.01	0.01\\
388.01	0.01\\
389.01	0.01\\
390.01	0.01\\
391.01	0.01\\
392.01	0.01\\
393.01	0.01\\
394.01	0.01\\
395.01	0.01\\
396.01	0.01\\
397.01	0.01\\
398.01	0.01\\
399.01	0.01\\
400.01	0.01\\
401.01	0.01\\
402.01	0.01\\
403.01	0.01\\
404.01	0.01\\
405.01	0.01\\
406.01	0.01\\
407.01	0.01\\
408.01	0.01\\
409.01	0.01\\
410.01	0.01\\
411.01	0.01\\
412.01	0.01\\
413.01	0.01\\
414.01	0.01\\
415.01	0.01\\
416.01	0.01\\
417.01	0.01\\
418.01	0.01\\
419.01	0.01\\
420.01	0.01\\
421.01	0.01\\
422.01	0.01\\
423.01	0.01\\
424.01	0.01\\
425.01	0.01\\
426.01	0.01\\
427.01	0.01\\
428.01	0.01\\
429.01	0.01\\
430.01	0.01\\
431.01	0.01\\
432.01	0.01\\
433.01	0.01\\
434.01	0.01\\
435.01	0.01\\
436.01	0.01\\
437.01	0.01\\
438.01	0.01\\
439.01	0.01\\
440.01	0.01\\
441.01	0.01\\
442.01	0.01\\
443.01	0.01\\
444.01	0.01\\
445.01	0.01\\
446.01	0.01\\
447.01	0.01\\
448.01	0.01\\
449.01	0.01\\
450.01	0.01\\
451.01	0.01\\
452.01	0.01\\
453.01	0.01\\
454.01	0.01\\
455.01	0.01\\
456.01	0.01\\
457.01	0.01\\
458.01	0.01\\
459.01	0.01\\
460.01	0.01\\
461.01	0.01\\
462.01	0.01\\
463.01	0.01\\
464.01	0.01\\
465.01	0.01\\
466.01	0.01\\
467.01	0.01\\
468.01	0.01\\
469.01	0.01\\
470.01	0.01\\
471.01	0.01\\
472.01	0.01\\
473.01	0.01\\
474.01	0.01\\
475.01	0.01\\
476.01	0.01\\
477.01	0.01\\
478.01	0.01\\
479.01	0.01\\
480.01	0.01\\
481.01	0.01\\
482.01	0.01\\
483.01	0.01\\
484.01	0.01\\
485.01	0.01\\
486.01	0.01\\
487.01	0.01\\
488.01	0.01\\
489.01	0.01\\
490.01	0.01\\
491.01	0.01\\
492.01	0.01\\
493.01	0.01\\
494.01	0.01\\
495.01	0.01\\
496.01	0.01\\
497.01	0.01\\
498.01	0.01\\
499.01	0.01\\
500.01	0.01\\
501.01	0.01\\
502.01	0.01\\
503.01	0.01\\
504.01	0.01\\
505.01	0.01\\
506.01	0.01\\
507.01	0.01\\
508.01	0.01\\
509.01	0.01\\
510.01	0.01\\
511.01	0.01\\
512.01	0.01\\
513.01	0.01\\
514.01	0.01\\
515.01	0.01\\
516.01	0.01\\
517.01	0.01\\
518.01	0.01\\
519.01	0.01\\
520.01	0.01\\
521.01	0.01\\
522.01	0.01\\
523.01	0.01\\
524.01	0.01\\
525.01	0.01\\
526.01	0.01\\
527.01	0.01\\
528.01	0.01\\
529.01	0.01\\
530.01	0.01\\
531.01	0.01\\
532.01	0.01\\
533.01	0.01\\
534.01	0.01\\
535.01	0.01\\
536.01	0.01\\
537.01	0.01\\
538.01	0.01\\
539.01	0.01\\
540.01	0.01\\
541.01	0.01\\
542.01	0.01\\
543.01	0.01\\
544.01	0.01\\
545.01	0.01\\
546.01	0.01\\
547.01	0.01\\
548.01	0.01\\
549.01	0.01\\
550.01	0.01\\
551.01	0.01\\
552.01	0.01\\
553.01	0.01\\
554.01	0.01\\
555.01	0.01\\
556.01	0.01\\
557.01	0.01\\
558.01	0.01\\
559.01	0.01\\
560.01	0.01\\
561.01	0.01\\
562.01	0.01\\
563.01	0.01\\
564.01	0.01\\
565.01	0.01\\
566.01	0.01\\
567.01	0.01\\
568.01	0.01\\
569.01	0.01\\
570.01	0.01\\
571.01	0.01\\
572.01	0.01\\
573.01	0.01\\
574.01	0.01\\
575.01	0.01\\
576.01	0.01\\
577.01	0.01\\
578.01	0.01\\
579.01	0.01\\
580.01	0.01\\
581.01	0.01\\
582.01	0.01\\
583.01	0.01\\
584.01	0.01\\
585.01	0.01\\
586.01	0.01\\
587.01	0.01\\
588.01	0.01\\
589.01	0.01\\
590.01	0.01\\
591.01	0.01\\
592.01	0.01\\
593.01	0.01\\
594.01	0.01\\
595.01	0.01\\
596.01	0.01\\
597.01	0.01\\
598.01	0.01\\
599.01	0.01\\
599.02	0.01\\
599.03	0.01\\
599.04	0.01\\
599.05	0.01\\
599.06	0.01\\
599.07	0.01\\
599.08	0.01\\
599.09	0.01\\
599.1	0.01\\
599.11	0.01\\
599.12	0.01\\
599.13	0.01\\
599.14	0.01\\
599.15	0.01\\
599.16	0.01\\
599.17	0.01\\
599.18	0.01\\
599.19	0.01\\
599.2	0.01\\
599.21	0.01\\
599.22	0.01\\
599.23	0.01\\
599.24	0.01\\
599.25	0.01\\
599.26	0.01\\
599.27	0.01\\
599.28	0.01\\
599.29	0.01\\
599.3	0.01\\
599.31	0.01\\
599.32	0.01\\
599.33	0.01\\
599.34	0.01\\
599.35	0.01\\
599.36	0.01\\
599.37	0.01\\
599.38	0.01\\
599.39	0.01\\
599.4	0.01\\
599.41	0.01\\
599.42	0.01\\
599.43	0.01\\
599.44	0.01\\
599.45	0.01\\
599.46	0.01\\
599.47	0.01\\
599.48	0.01\\
599.49	0.01\\
599.5	0.01\\
599.51	0.01\\
599.52	0.01\\
599.53	0.01\\
599.54	0.01\\
599.55	0.01\\
599.56	0.01\\
599.57	0.01\\
599.58	0.01\\
599.59	0.01\\
599.6	0.01\\
599.61	0.01\\
599.62	0.01\\
599.63	0.01\\
599.64	0.01\\
599.65	0.01\\
599.66	0.01\\
599.67	0.01\\
599.68	0.01\\
599.69	0.01\\
599.7	0.01\\
599.71	0.01\\
599.72	0.01\\
599.73	0.01\\
599.74	0.01\\
599.75	0.01\\
599.76	0.01\\
599.77	0.01\\
599.78	0.01\\
599.79	0.01\\
599.8	0.01\\
599.81	0.01\\
599.82	0.01\\
599.83	0.01\\
599.84	0.01\\
599.85	0.01\\
599.86	0.01\\
599.87	0.01\\
599.88	0.01\\
599.89	0.01\\
599.9	0.01\\
599.91	0.01\\
599.92	0.01\\
599.93	0.01\\
599.94	0.01\\
599.95	0.01\\
599.96	0.01\\
599.97	0.01\\
599.98	0.01\\
599.99	0.01\\
600	0.01\\
};
\addplot [color=mycolor7,solid,forget plot]
  table[row sep=crcr]{%
0.01	0.01\\
1.01	0.01\\
2.01	0.01\\
3.01	0.01\\
4.01	0.01\\
5.01	0.01\\
6.01	0.01\\
7.01	0.01\\
8.01	0.01\\
9.01	0.01\\
10.01	0.01\\
11.01	0.01\\
12.01	0.01\\
13.01	0.01\\
14.01	0.01\\
15.01	0.01\\
16.01	0.01\\
17.01	0.01\\
18.01	0.01\\
19.01	0.01\\
20.01	0.01\\
21.01	0.01\\
22.01	0.01\\
23.01	0.01\\
24.01	0.01\\
25.01	0.01\\
26.01	0.01\\
27.01	0.01\\
28.01	0.01\\
29.01	0.01\\
30.01	0.01\\
31.01	0.01\\
32.01	0.01\\
33.01	0.01\\
34.01	0.01\\
35.01	0.01\\
36.01	0.01\\
37.01	0.01\\
38.01	0.01\\
39.01	0.01\\
40.01	0.01\\
41.01	0.01\\
42.01	0.01\\
43.01	0.01\\
44.01	0.01\\
45.01	0.01\\
46.01	0.01\\
47.01	0.01\\
48.01	0.01\\
49.01	0.01\\
50.01	0.01\\
51.01	0.01\\
52.01	0.01\\
53.01	0.01\\
54.01	0.01\\
55.01	0.01\\
56.01	0.01\\
57.01	0.01\\
58.01	0.01\\
59.01	0.01\\
60.01	0.01\\
61.01	0.01\\
62.01	0.01\\
63.01	0.01\\
64.01	0.01\\
65.01	0.01\\
66.01	0.01\\
67.01	0.01\\
68.01	0.01\\
69.01	0.01\\
70.01	0.01\\
71.01	0.01\\
72.01	0.01\\
73.01	0.01\\
74.01	0.01\\
75.01	0.01\\
76.01	0.01\\
77.01	0.01\\
78.01	0.01\\
79.01	0.01\\
80.01	0.01\\
81.01	0.01\\
82.01	0.01\\
83.01	0.01\\
84.01	0.01\\
85.01	0.01\\
86.01	0.01\\
87.01	0.01\\
88.01	0.01\\
89.01	0.01\\
90.01	0.01\\
91.01	0.01\\
92.01	0.01\\
93.01	0.01\\
94.01	0.01\\
95.01	0.01\\
96.01	0.01\\
97.01	0.01\\
98.01	0.01\\
99.01	0.01\\
100.01	0.01\\
101.01	0.01\\
102.01	0.01\\
103.01	0.01\\
104.01	0.01\\
105.01	0.01\\
106.01	0.01\\
107.01	0.01\\
108.01	0.01\\
109.01	0.01\\
110.01	0.01\\
111.01	0.01\\
112.01	0.01\\
113.01	0.01\\
114.01	0.01\\
115.01	0.01\\
116.01	0.01\\
117.01	0.01\\
118.01	0.01\\
119.01	0.01\\
120.01	0.01\\
121.01	0.01\\
122.01	0.01\\
123.01	0.01\\
124.01	0.01\\
125.01	0.01\\
126.01	0.01\\
127.01	0.01\\
128.01	0.01\\
129.01	0.01\\
130.01	0.01\\
131.01	0.01\\
132.01	0.01\\
133.01	0.01\\
134.01	0.01\\
135.01	0.01\\
136.01	0.01\\
137.01	0.01\\
138.01	0.01\\
139.01	0.01\\
140.01	0.01\\
141.01	0.01\\
142.01	0.01\\
143.01	0.01\\
144.01	0.01\\
145.01	0.01\\
146.01	0.01\\
147.01	0.01\\
148.01	0.01\\
149.01	0.01\\
150.01	0.01\\
151.01	0.01\\
152.01	0.01\\
153.01	0.01\\
154.01	0.01\\
155.01	0.01\\
156.01	0.01\\
157.01	0.01\\
158.01	0.01\\
159.01	0.01\\
160.01	0.01\\
161.01	0.01\\
162.01	0.01\\
163.01	0.01\\
164.01	0.01\\
165.01	0.01\\
166.01	0.01\\
167.01	0.01\\
168.01	0.01\\
169.01	0.01\\
170.01	0.01\\
171.01	0.01\\
172.01	0.01\\
173.01	0.01\\
174.01	0.01\\
175.01	0.01\\
176.01	0.01\\
177.01	0.01\\
178.01	0.01\\
179.01	0.01\\
180.01	0.01\\
181.01	0.01\\
182.01	0.01\\
183.01	0.01\\
184.01	0.01\\
185.01	0.01\\
186.01	0.01\\
187.01	0.01\\
188.01	0.01\\
189.01	0.01\\
190.01	0.01\\
191.01	0.01\\
192.01	0.01\\
193.01	0.01\\
194.01	0.01\\
195.01	0.01\\
196.01	0.01\\
197.01	0.01\\
198.01	0.01\\
199.01	0.01\\
200.01	0.01\\
201.01	0.01\\
202.01	0.01\\
203.01	0.01\\
204.01	0.01\\
205.01	0.01\\
206.01	0.01\\
207.01	0.01\\
208.01	0.01\\
209.01	0.01\\
210.01	0.01\\
211.01	0.01\\
212.01	0.01\\
213.01	0.01\\
214.01	0.01\\
215.01	0.01\\
216.01	0.01\\
217.01	0.01\\
218.01	0.01\\
219.01	0.01\\
220.01	0.01\\
221.01	0.01\\
222.01	0.01\\
223.01	0.01\\
224.01	0.01\\
225.01	0.01\\
226.01	0.01\\
227.01	0.01\\
228.01	0.01\\
229.01	0.01\\
230.01	0.01\\
231.01	0.01\\
232.01	0.01\\
233.01	0.01\\
234.01	0.01\\
235.01	0.01\\
236.01	0.01\\
237.01	0.01\\
238.01	0.01\\
239.01	0.01\\
240.01	0.01\\
241.01	0.01\\
242.01	0.01\\
243.01	0.01\\
244.01	0.01\\
245.01	0.01\\
246.01	0.01\\
247.01	0.01\\
248.01	0.01\\
249.01	0.01\\
250.01	0.01\\
251.01	0.01\\
252.01	0.01\\
253.01	0.01\\
254.01	0.01\\
255.01	0.01\\
256.01	0.01\\
257.01	0.01\\
258.01	0.01\\
259.01	0.01\\
260.01	0.01\\
261.01	0.01\\
262.01	0.01\\
263.01	0.01\\
264.01	0.01\\
265.01	0.01\\
266.01	0.01\\
267.01	0.01\\
268.01	0.01\\
269.01	0.01\\
270.01	0.01\\
271.01	0.01\\
272.01	0.01\\
273.01	0.01\\
274.01	0.01\\
275.01	0.01\\
276.01	0.01\\
277.01	0.01\\
278.01	0.01\\
279.01	0.01\\
280.01	0.01\\
281.01	0.01\\
282.01	0.01\\
283.01	0.01\\
284.01	0.01\\
285.01	0.01\\
286.01	0.01\\
287.01	0.01\\
288.01	0.01\\
289.01	0.01\\
290.01	0.01\\
291.01	0.01\\
292.01	0.01\\
293.01	0.01\\
294.01	0.01\\
295.01	0.01\\
296.01	0.01\\
297.01	0.01\\
298.01	0.01\\
299.01	0.01\\
300.01	0.01\\
301.01	0.01\\
302.01	0.01\\
303.01	0.01\\
304.01	0.01\\
305.01	0.01\\
306.01	0.01\\
307.01	0.01\\
308.01	0.01\\
309.01	0.01\\
310.01	0.01\\
311.01	0.01\\
312.01	0.01\\
313.01	0.01\\
314.01	0.01\\
315.01	0.01\\
316.01	0.01\\
317.01	0.01\\
318.01	0.01\\
319.01	0.01\\
320.01	0.01\\
321.01	0.01\\
322.01	0.01\\
323.01	0.01\\
324.01	0.01\\
325.01	0.01\\
326.01	0.01\\
327.01	0.01\\
328.01	0.01\\
329.01	0.01\\
330.01	0.01\\
331.01	0.01\\
332.01	0.01\\
333.01	0.01\\
334.01	0.01\\
335.01	0.01\\
336.01	0.01\\
337.01	0.01\\
338.01	0.01\\
339.01	0.01\\
340.01	0.01\\
341.01	0.01\\
342.01	0.01\\
343.01	0.01\\
344.01	0.01\\
345.01	0.01\\
346.01	0.01\\
347.01	0.01\\
348.01	0.01\\
349.01	0.01\\
350.01	0.01\\
351.01	0.01\\
352.01	0.01\\
353.01	0.01\\
354.01	0.01\\
355.01	0.01\\
356.01	0.01\\
357.01	0.01\\
358.01	0.01\\
359.01	0.01\\
360.01	0.01\\
361.01	0.01\\
362.01	0.01\\
363.01	0.01\\
364.01	0.01\\
365.01	0.01\\
366.01	0.01\\
367.01	0.01\\
368.01	0.01\\
369.01	0.01\\
370.01	0.01\\
371.01	0.01\\
372.01	0.01\\
373.01	0.01\\
374.01	0.01\\
375.01	0.01\\
376.01	0.01\\
377.01	0.01\\
378.01	0.01\\
379.01	0.01\\
380.01	0.01\\
381.01	0.01\\
382.01	0.01\\
383.01	0.01\\
384.01	0.01\\
385.01	0.01\\
386.01	0.01\\
387.01	0.01\\
388.01	0.01\\
389.01	0.01\\
390.01	0.01\\
391.01	0.01\\
392.01	0.01\\
393.01	0.01\\
394.01	0.01\\
395.01	0.01\\
396.01	0.01\\
397.01	0.01\\
398.01	0.01\\
399.01	0.01\\
400.01	0.01\\
401.01	0.01\\
402.01	0.01\\
403.01	0.01\\
404.01	0.01\\
405.01	0.01\\
406.01	0.01\\
407.01	0.01\\
408.01	0.01\\
409.01	0.01\\
410.01	0.01\\
411.01	0.01\\
412.01	0.01\\
413.01	0.01\\
414.01	0.01\\
415.01	0.01\\
416.01	0.01\\
417.01	0.01\\
418.01	0.01\\
419.01	0.01\\
420.01	0.01\\
421.01	0.01\\
422.01	0.01\\
423.01	0.01\\
424.01	0.01\\
425.01	0.01\\
426.01	0.01\\
427.01	0.01\\
428.01	0.01\\
429.01	0.01\\
430.01	0.01\\
431.01	0.01\\
432.01	0.01\\
433.01	0.01\\
434.01	0.01\\
435.01	0.01\\
436.01	0.01\\
437.01	0.01\\
438.01	0.01\\
439.01	0.01\\
440.01	0.01\\
441.01	0.01\\
442.01	0.01\\
443.01	0.01\\
444.01	0.01\\
445.01	0.01\\
446.01	0.01\\
447.01	0.01\\
448.01	0.01\\
449.01	0.01\\
450.01	0.01\\
451.01	0.01\\
452.01	0.01\\
453.01	0.01\\
454.01	0.01\\
455.01	0.01\\
456.01	0.01\\
457.01	0.01\\
458.01	0.01\\
459.01	0.01\\
460.01	0.01\\
461.01	0.01\\
462.01	0.01\\
463.01	0.01\\
464.01	0.01\\
465.01	0.01\\
466.01	0.01\\
467.01	0.01\\
468.01	0.01\\
469.01	0.01\\
470.01	0.01\\
471.01	0.01\\
472.01	0.01\\
473.01	0.01\\
474.01	0.01\\
475.01	0.01\\
476.01	0.01\\
477.01	0.01\\
478.01	0.01\\
479.01	0.01\\
480.01	0.01\\
481.01	0.01\\
482.01	0.01\\
483.01	0.01\\
484.01	0.01\\
485.01	0.01\\
486.01	0.01\\
487.01	0.01\\
488.01	0.01\\
489.01	0.01\\
490.01	0.01\\
491.01	0.01\\
492.01	0.01\\
493.01	0.01\\
494.01	0.01\\
495.01	0.01\\
496.01	0.01\\
497.01	0.01\\
498.01	0.01\\
499.01	0.01\\
500.01	0.01\\
501.01	0.01\\
502.01	0.01\\
503.01	0.01\\
504.01	0.01\\
505.01	0.01\\
506.01	0.01\\
507.01	0.01\\
508.01	0.01\\
509.01	0.01\\
510.01	0.01\\
511.01	0.01\\
512.01	0.01\\
513.01	0.01\\
514.01	0.01\\
515.01	0.01\\
516.01	0.01\\
517.01	0.01\\
518.01	0.01\\
519.01	0.01\\
520.01	0.01\\
521.01	0.01\\
522.01	0.01\\
523.01	0.01\\
524.01	0.01\\
525.01	0.01\\
526.01	0.01\\
527.01	0.01\\
528.01	0.01\\
529.01	0.01\\
530.01	0.01\\
531.01	0.01\\
532.01	0.01\\
533.01	0.01\\
534.01	0.01\\
535.01	0.01\\
536.01	0.01\\
537.01	0.01\\
538.01	0.01\\
539.01	0.01\\
540.01	0.01\\
541.01	0.01\\
542.01	0.01\\
543.01	0.01\\
544.01	0.01\\
545.01	0.01\\
546.01	0.01\\
547.01	0.01\\
548.01	0.01\\
549.01	0.01\\
550.01	0.01\\
551.01	0.01\\
552.01	0.01\\
553.01	0.01\\
554.01	0.01\\
555.01	0.01\\
556.01	0.01\\
557.01	0.01\\
558.01	0.01\\
559.01	0.01\\
560.01	0.01\\
561.01	0.01\\
562.01	0.01\\
563.01	0.01\\
564.01	0.01\\
565.01	0.01\\
566.01	0.01\\
567.01	0.01\\
568.01	0.01\\
569.01	0.01\\
570.01	0.01\\
571.01	0.01\\
572.01	0.01\\
573.01	0.01\\
574.01	0.01\\
575.01	0.01\\
576.01	0.01\\
577.01	0.01\\
578.01	0.01\\
579.01	0.01\\
580.01	0.01\\
581.01	0.01\\
582.01	0.01\\
583.01	0.01\\
584.01	0.01\\
585.01	0.01\\
586.01	0.01\\
587.01	0.01\\
588.01	0.01\\
589.01	0.01\\
590.01	0.01\\
591.01	0.01\\
592.01	0.01\\
593.01	0.01\\
594.01	0.01\\
595.01	0.01\\
596.01	0.01\\
597.01	0.01\\
598.01	0.01\\
599.01	0.01\\
599.02	0.01\\
599.03	0.01\\
599.04	0.01\\
599.05	0.01\\
599.06	0.01\\
599.07	0.01\\
599.08	0.01\\
599.09	0.01\\
599.1	0.01\\
599.11	0.01\\
599.12	0.01\\
599.13	0.01\\
599.14	0.01\\
599.15	0.01\\
599.16	0.01\\
599.17	0.01\\
599.18	0.01\\
599.19	0.01\\
599.2	0.01\\
599.21	0.01\\
599.22	0.01\\
599.23	0.01\\
599.24	0.01\\
599.25	0.01\\
599.26	0.01\\
599.27	0.01\\
599.28	0.01\\
599.29	0.01\\
599.3	0.01\\
599.31	0.01\\
599.32	0.01\\
599.33	0.01\\
599.34	0.01\\
599.35	0.01\\
599.36	0.01\\
599.37	0.01\\
599.38	0.01\\
599.39	0.01\\
599.4	0.01\\
599.41	0.01\\
599.42	0.01\\
599.43	0.01\\
599.44	0.01\\
599.45	0.01\\
599.46	0.01\\
599.47	0.01\\
599.48	0.01\\
599.49	0.01\\
599.5	0.01\\
599.51	0.01\\
599.52	0.01\\
599.53	0.01\\
599.54	0.01\\
599.55	0.01\\
599.56	0.01\\
599.57	0.01\\
599.58	0.01\\
599.59	0.01\\
599.6	0.01\\
599.61	0.01\\
599.62	0.01\\
599.63	0.01\\
599.64	0.01\\
599.65	0.01\\
599.66	0.01\\
599.67	0.01\\
599.68	0.01\\
599.69	0.01\\
599.7	0.01\\
599.71	0.01\\
599.72	0.01\\
599.73	0.01\\
599.74	0.01\\
599.75	0.01\\
599.76	0.01\\
599.77	0.01\\
599.78	0.01\\
599.79	0.01\\
599.8	0.01\\
599.81	0.01\\
599.82	0.01\\
599.83	0.01\\
599.84	0.01\\
599.85	0.01\\
599.86	0.01\\
599.87	0.01\\
599.88	0.01\\
599.89	0.01\\
599.9	0.01\\
599.91	0.01\\
599.92	0.01\\
599.93	0.01\\
599.94	0.01\\
599.95	0.01\\
599.96	0.01\\
599.97	0.01\\
599.98	0.01\\
599.99	0.01\\
600	0.01\\
};
\addplot [color=mycolor8,solid,forget plot]
  table[row sep=crcr]{%
0.01	0.01\\
1.01	0.01\\
2.01	0.01\\
3.01	0.01\\
4.01	0.01\\
5.01	0.01\\
6.01	0.01\\
7.01	0.01\\
8.01	0.01\\
9.01	0.01\\
10.01	0.01\\
11.01	0.01\\
12.01	0.01\\
13.01	0.01\\
14.01	0.01\\
15.01	0.01\\
16.01	0.01\\
17.01	0.01\\
18.01	0.01\\
19.01	0.01\\
20.01	0.01\\
21.01	0.01\\
22.01	0.01\\
23.01	0.01\\
24.01	0.01\\
25.01	0.01\\
26.01	0.01\\
27.01	0.01\\
28.01	0.01\\
29.01	0.01\\
30.01	0.01\\
31.01	0.01\\
32.01	0.01\\
33.01	0.01\\
34.01	0.01\\
35.01	0.01\\
36.01	0.01\\
37.01	0.01\\
38.01	0.01\\
39.01	0.01\\
40.01	0.01\\
41.01	0.01\\
42.01	0.01\\
43.01	0.01\\
44.01	0.01\\
45.01	0.01\\
46.01	0.01\\
47.01	0.01\\
48.01	0.01\\
49.01	0.01\\
50.01	0.01\\
51.01	0.01\\
52.01	0.01\\
53.01	0.01\\
54.01	0.01\\
55.01	0.01\\
56.01	0.01\\
57.01	0.01\\
58.01	0.01\\
59.01	0.01\\
60.01	0.01\\
61.01	0.01\\
62.01	0.01\\
63.01	0.01\\
64.01	0.01\\
65.01	0.01\\
66.01	0.01\\
67.01	0.01\\
68.01	0.01\\
69.01	0.01\\
70.01	0.01\\
71.01	0.01\\
72.01	0.01\\
73.01	0.01\\
74.01	0.01\\
75.01	0.01\\
76.01	0.01\\
77.01	0.01\\
78.01	0.01\\
79.01	0.01\\
80.01	0.01\\
81.01	0.01\\
82.01	0.01\\
83.01	0.01\\
84.01	0.01\\
85.01	0.01\\
86.01	0.01\\
87.01	0.01\\
88.01	0.01\\
89.01	0.01\\
90.01	0.01\\
91.01	0.01\\
92.01	0.01\\
93.01	0.01\\
94.01	0.01\\
95.01	0.01\\
96.01	0.01\\
97.01	0.01\\
98.01	0.01\\
99.01	0.01\\
100.01	0.01\\
101.01	0.01\\
102.01	0.01\\
103.01	0.01\\
104.01	0.01\\
105.01	0.01\\
106.01	0.01\\
107.01	0.01\\
108.01	0.01\\
109.01	0.01\\
110.01	0.01\\
111.01	0.01\\
112.01	0.01\\
113.01	0.01\\
114.01	0.01\\
115.01	0.01\\
116.01	0.01\\
117.01	0.01\\
118.01	0.01\\
119.01	0.01\\
120.01	0.01\\
121.01	0.01\\
122.01	0.01\\
123.01	0.01\\
124.01	0.01\\
125.01	0.01\\
126.01	0.01\\
127.01	0.01\\
128.01	0.01\\
129.01	0.01\\
130.01	0.01\\
131.01	0.01\\
132.01	0.01\\
133.01	0.01\\
134.01	0.01\\
135.01	0.01\\
136.01	0.01\\
137.01	0.01\\
138.01	0.01\\
139.01	0.01\\
140.01	0.01\\
141.01	0.01\\
142.01	0.01\\
143.01	0.01\\
144.01	0.01\\
145.01	0.01\\
146.01	0.01\\
147.01	0.01\\
148.01	0.01\\
149.01	0.01\\
150.01	0.01\\
151.01	0.01\\
152.01	0.01\\
153.01	0.01\\
154.01	0.01\\
155.01	0.01\\
156.01	0.01\\
157.01	0.01\\
158.01	0.01\\
159.01	0.01\\
160.01	0.01\\
161.01	0.01\\
162.01	0.01\\
163.01	0.01\\
164.01	0.01\\
165.01	0.01\\
166.01	0.01\\
167.01	0.01\\
168.01	0.01\\
169.01	0.01\\
170.01	0.01\\
171.01	0.01\\
172.01	0.01\\
173.01	0.01\\
174.01	0.01\\
175.01	0.01\\
176.01	0.01\\
177.01	0.01\\
178.01	0.01\\
179.01	0.01\\
180.01	0.01\\
181.01	0.01\\
182.01	0.01\\
183.01	0.01\\
184.01	0.01\\
185.01	0.01\\
186.01	0.01\\
187.01	0.01\\
188.01	0.01\\
189.01	0.01\\
190.01	0.01\\
191.01	0.01\\
192.01	0.01\\
193.01	0.01\\
194.01	0.01\\
195.01	0.01\\
196.01	0.01\\
197.01	0.01\\
198.01	0.01\\
199.01	0.01\\
200.01	0.01\\
201.01	0.01\\
202.01	0.01\\
203.01	0.01\\
204.01	0.01\\
205.01	0.01\\
206.01	0.01\\
207.01	0.01\\
208.01	0.01\\
209.01	0.01\\
210.01	0.01\\
211.01	0.01\\
212.01	0.01\\
213.01	0.01\\
214.01	0.01\\
215.01	0.01\\
216.01	0.01\\
217.01	0.01\\
218.01	0.01\\
219.01	0.01\\
220.01	0.01\\
221.01	0.01\\
222.01	0.01\\
223.01	0.01\\
224.01	0.01\\
225.01	0.01\\
226.01	0.01\\
227.01	0.01\\
228.01	0.01\\
229.01	0.01\\
230.01	0.01\\
231.01	0.01\\
232.01	0.01\\
233.01	0.01\\
234.01	0.01\\
235.01	0.01\\
236.01	0.01\\
237.01	0.01\\
238.01	0.01\\
239.01	0.01\\
240.01	0.01\\
241.01	0.01\\
242.01	0.01\\
243.01	0.01\\
244.01	0.01\\
245.01	0.01\\
246.01	0.01\\
247.01	0.01\\
248.01	0.01\\
249.01	0.01\\
250.01	0.01\\
251.01	0.01\\
252.01	0.01\\
253.01	0.01\\
254.01	0.01\\
255.01	0.01\\
256.01	0.01\\
257.01	0.01\\
258.01	0.01\\
259.01	0.01\\
260.01	0.01\\
261.01	0.01\\
262.01	0.01\\
263.01	0.01\\
264.01	0.01\\
265.01	0.01\\
266.01	0.01\\
267.01	0.01\\
268.01	0.01\\
269.01	0.01\\
270.01	0.01\\
271.01	0.01\\
272.01	0.01\\
273.01	0.01\\
274.01	0.01\\
275.01	0.01\\
276.01	0.01\\
277.01	0.01\\
278.01	0.01\\
279.01	0.01\\
280.01	0.01\\
281.01	0.01\\
282.01	0.01\\
283.01	0.01\\
284.01	0.01\\
285.01	0.01\\
286.01	0.01\\
287.01	0.01\\
288.01	0.01\\
289.01	0.01\\
290.01	0.01\\
291.01	0.01\\
292.01	0.01\\
293.01	0.01\\
294.01	0.01\\
295.01	0.01\\
296.01	0.01\\
297.01	0.01\\
298.01	0.01\\
299.01	0.01\\
300.01	0.01\\
301.01	0.01\\
302.01	0.01\\
303.01	0.01\\
304.01	0.01\\
305.01	0.01\\
306.01	0.01\\
307.01	0.01\\
308.01	0.01\\
309.01	0.01\\
310.01	0.01\\
311.01	0.01\\
312.01	0.01\\
313.01	0.01\\
314.01	0.01\\
315.01	0.01\\
316.01	0.01\\
317.01	0.01\\
318.01	0.01\\
319.01	0.01\\
320.01	0.01\\
321.01	0.01\\
322.01	0.01\\
323.01	0.01\\
324.01	0.01\\
325.01	0.01\\
326.01	0.01\\
327.01	0.01\\
328.01	0.01\\
329.01	0.01\\
330.01	0.01\\
331.01	0.01\\
332.01	0.01\\
333.01	0.01\\
334.01	0.01\\
335.01	0.01\\
336.01	0.01\\
337.01	0.01\\
338.01	0.01\\
339.01	0.01\\
340.01	0.01\\
341.01	0.01\\
342.01	0.01\\
343.01	0.01\\
344.01	0.01\\
345.01	0.01\\
346.01	0.01\\
347.01	0.01\\
348.01	0.01\\
349.01	0.01\\
350.01	0.01\\
351.01	0.01\\
352.01	0.01\\
353.01	0.01\\
354.01	0.01\\
355.01	0.01\\
356.01	0.01\\
357.01	0.01\\
358.01	0.01\\
359.01	0.01\\
360.01	0.01\\
361.01	0.01\\
362.01	0.01\\
363.01	0.01\\
364.01	0.01\\
365.01	0.01\\
366.01	0.01\\
367.01	0.01\\
368.01	0.01\\
369.01	0.01\\
370.01	0.01\\
371.01	0.01\\
372.01	0.01\\
373.01	0.01\\
374.01	0.01\\
375.01	0.01\\
376.01	0.01\\
377.01	0.01\\
378.01	0.01\\
379.01	0.01\\
380.01	0.01\\
381.01	0.01\\
382.01	0.01\\
383.01	0.01\\
384.01	0.01\\
385.01	0.01\\
386.01	0.01\\
387.01	0.01\\
388.01	0.01\\
389.01	0.01\\
390.01	0.01\\
391.01	0.01\\
392.01	0.01\\
393.01	0.01\\
394.01	0.01\\
395.01	0.01\\
396.01	0.01\\
397.01	0.01\\
398.01	0.01\\
399.01	0.01\\
400.01	0.01\\
401.01	0.01\\
402.01	0.01\\
403.01	0.01\\
404.01	0.01\\
405.01	0.01\\
406.01	0.01\\
407.01	0.01\\
408.01	0.01\\
409.01	0.01\\
410.01	0.01\\
411.01	0.01\\
412.01	0.01\\
413.01	0.01\\
414.01	0.01\\
415.01	0.01\\
416.01	0.01\\
417.01	0.01\\
418.01	0.01\\
419.01	0.01\\
420.01	0.01\\
421.01	0.01\\
422.01	0.01\\
423.01	0.01\\
424.01	0.01\\
425.01	0.01\\
426.01	0.01\\
427.01	0.01\\
428.01	0.01\\
429.01	0.01\\
430.01	0.01\\
431.01	0.01\\
432.01	0.01\\
433.01	0.01\\
434.01	0.01\\
435.01	0.01\\
436.01	0.01\\
437.01	0.01\\
438.01	0.01\\
439.01	0.01\\
440.01	0.01\\
441.01	0.01\\
442.01	0.01\\
443.01	0.01\\
444.01	0.01\\
445.01	0.01\\
446.01	0.01\\
447.01	0.01\\
448.01	0.01\\
449.01	0.01\\
450.01	0.01\\
451.01	0.01\\
452.01	0.01\\
453.01	0.01\\
454.01	0.01\\
455.01	0.01\\
456.01	0.01\\
457.01	0.01\\
458.01	0.01\\
459.01	0.01\\
460.01	0.01\\
461.01	0.01\\
462.01	0.01\\
463.01	0.01\\
464.01	0.01\\
465.01	0.01\\
466.01	0.01\\
467.01	0.01\\
468.01	0.01\\
469.01	0.01\\
470.01	0.01\\
471.01	0.01\\
472.01	0.01\\
473.01	0.01\\
474.01	0.01\\
475.01	0.01\\
476.01	0.01\\
477.01	0.01\\
478.01	0.01\\
479.01	0.01\\
480.01	0.01\\
481.01	0.01\\
482.01	0.01\\
483.01	0.01\\
484.01	0.01\\
485.01	0.01\\
486.01	0.01\\
487.01	0.01\\
488.01	0.01\\
489.01	0.01\\
490.01	0.01\\
491.01	0.01\\
492.01	0.01\\
493.01	0.01\\
494.01	0.01\\
495.01	0.01\\
496.01	0.01\\
497.01	0.01\\
498.01	0.01\\
499.01	0.01\\
500.01	0.01\\
501.01	0.01\\
502.01	0.01\\
503.01	0.01\\
504.01	0.01\\
505.01	0.01\\
506.01	0.01\\
507.01	0.01\\
508.01	0.01\\
509.01	0.01\\
510.01	0.01\\
511.01	0.01\\
512.01	0.01\\
513.01	0.01\\
514.01	0.01\\
515.01	0.01\\
516.01	0.01\\
517.01	0.01\\
518.01	0.01\\
519.01	0.01\\
520.01	0.01\\
521.01	0.01\\
522.01	0.01\\
523.01	0.01\\
524.01	0.01\\
525.01	0.01\\
526.01	0.01\\
527.01	0.01\\
528.01	0.01\\
529.01	0.01\\
530.01	0.01\\
531.01	0.01\\
532.01	0.01\\
533.01	0.01\\
534.01	0.01\\
535.01	0.01\\
536.01	0.01\\
537.01	0.01\\
538.01	0.01\\
539.01	0.01\\
540.01	0.01\\
541.01	0.01\\
542.01	0.01\\
543.01	0.01\\
544.01	0.01\\
545.01	0.01\\
546.01	0.01\\
547.01	0.01\\
548.01	0.01\\
549.01	0.01\\
550.01	0.01\\
551.01	0.01\\
552.01	0.01\\
553.01	0.01\\
554.01	0.01\\
555.01	0.01\\
556.01	0.01\\
557.01	0.01\\
558.01	0.01\\
559.01	0.01\\
560.01	0.01\\
561.01	0.01\\
562.01	0.01\\
563.01	0.01\\
564.01	0.01\\
565.01	0.01\\
566.01	0.01\\
567.01	0.01\\
568.01	0.01\\
569.01	0.01\\
570.01	0.01\\
571.01	0.01\\
572.01	0.01\\
573.01	0.01\\
574.01	0.01\\
575.01	0.01\\
576.01	0.01\\
577.01	0.01\\
578.01	0.01\\
579.01	0.01\\
580.01	0.01\\
581.01	0.01\\
582.01	0.01\\
583.01	0.01\\
584.01	0.01\\
585.01	0.01\\
586.01	0.01\\
587.01	0.01\\
588.01	0.01\\
589.01	0.01\\
590.01	0.01\\
591.01	0.01\\
592.01	0.01\\
593.01	0.01\\
594.01	0.01\\
595.01	0.01\\
596.01	0.01\\
597.01	0.01\\
598.01	0.01\\
599.01	0.01\\
599.02	0.01\\
599.03	0.01\\
599.04	0.01\\
599.05	0.01\\
599.06	0.01\\
599.07	0.01\\
599.08	0.01\\
599.09	0.01\\
599.1	0.01\\
599.11	0.01\\
599.12	0.01\\
599.13	0.01\\
599.14	0.01\\
599.15	0.01\\
599.16	0.01\\
599.17	0.01\\
599.18	0.01\\
599.19	0.01\\
599.2	0.01\\
599.21	0.01\\
599.22	0.01\\
599.23	0.01\\
599.24	0.01\\
599.25	0.01\\
599.26	0.01\\
599.27	0.01\\
599.28	0.01\\
599.29	0.01\\
599.3	0.01\\
599.31	0.01\\
599.32	0.01\\
599.33	0.01\\
599.34	0.01\\
599.35	0.01\\
599.36	0.01\\
599.37	0.01\\
599.38	0.01\\
599.39	0.01\\
599.4	0.01\\
599.41	0.01\\
599.42	0.01\\
599.43	0.01\\
599.44	0.01\\
599.45	0.01\\
599.46	0.01\\
599.47	0.01\\
599.48	0.01\\
599.49	0.01\\
599.5	0.01\\
599.51	0.01\\
599.52	0.01\\
599.53	0.01\\
599.54	0.01\\
599.55	0.01\\
599.56	0.01\\
599.57	0.01\\
599.58	0.01\\
599.59	0.01\\
599.6	0.01\\
599.61	0.01\\
599.62	0.01\\
599.63	0.01\\
599.64	0.01\\
599.65	0.01\\
599.66	0.01\\
599.67	0.01\\
599.68	0.01\\
599.69	0.01\\
599.7	0.01\\
599.71	0.01\\
599.72	0.01\\
599.73	0.01\\
599.74	0.01\\
599.75	0.01\\
599.76	0.01\\
599.77	0.01\\
599.78	0.01\\
599.79	0.01\\
599.8	0.01\\
599.81	0.01\\
599.82	0.01\\
599.83	0.01\\
599.84	0.01\\
599.85	0.01\\
599.86	0.01\\
599.87	0.01\\
599.88	0.01\\
599.89	0.01\\
599.9	0.01\\
599.91	0.01\\
599.92	0.01\\
599.93	0.01\\
599.94	0.01\\
599.95	0.01\\
599.96	0.01\\
599.97	0.01\\
599.98	0.01\\
599.99	0.01\\
600	0.01\\
};
\addplot [color=blue!25!mycolor7,solid,forget plot]
  table[row sep=crcr]{%
0.01	0.01\\
1.01	0.01\\
2.01	0.01\\
3.01	0.01\\
4.01	0.01\\
5.01	0.01\\
6.01	0.01\\
7.01	0.01\\
8.01	0.01\\
9.01	0.01\\
10.01	0.01\\
11.01	0.01\\
12.01	0.01\\
13.01	0.01\\
14.01	0.01\\
15.01	0.01\\
16.01	0.01\\
17.01	0.01\\
18.01	0.01\\
19.01	0.01\\
20.01	0.01\\
21.01	0.01\\
22.01	0.01\\
23.01	0.01\\
24.01	0.01\\
25.01	0.01\\
26.01	0.01\\
27.01	0.01\\
28.01	0.01\\
29.01	0.01\\
30.01	0.01\\
31.01	0.01\\
32.01	0.01\\
33.01	0.01\\
34.01	0.01\\
35.01	0.01\\
36.01	0.01\\
37.01	0.01\\
38.01	0.01\\
39.01	0.01\\
40.01	0.01\\
41.01	0.01\\
42.01	0.01\\
43.01	0.01\\
44.01	0.01\\
45.01	0.01\\
46.01	0.01\\
47.01	0.01\\
48.01	0.01\\
49.01	0.01\\
50.01	0.01\\
51.01	0.01\\
52.01	0.01\\
53.01	0.01\\
54.01	0.01\\
55.01	0.01\\
56.01	0.01\\
57.01	0.01\\
58.01	0.01\\
59.01	0.01\\
60.01	0.01\\
61.01	0.01\\
62.01	0.01\\
63.01	0.01\\
64.01	0.01\\
65.01	0.01\\
66.01	0.01\\
67.01	0.01\\
68.01	0.01\\
69.01	0.01\\
70.01	0.01\\
71.01	0.01\\
72.01	0.01\\
73.01	0.01\\
74.01	0.01\\
75.01	0.01\\
76.01	0.01\\
77.01	0.01\\
78.01	0.01\\
79.01	0.01\\
80.01	0.01\\
81.01	0.01\\
82.01	0.01\\
83.01	0.01\\
84.01	0.01\\
85.01	0.01\\
86.01	0.01\\
87.01	0.01\\
88.01	0.01\\
89.01	0.01\\
90.01	0.01\\
91.01	0.01\\
92.01	0.01\\
93.01	0.01\\
94.01	0.01\\
95.01	0.01\\
96.01	0.01\\
97.01	0.01\\
98.01	0.01\\
99.01	0.01\\
100.01	0.01\\
101.01	0.01\\
102.01	0.01\\
103.01	0.01\\
104.01	0.01\\
105.01	0.01\\
106.01	0.01\\
107.01	0.01\\
108.01	0.01\\
109.01	0.01\\
110.01	0.01\\
111.01	0.01\\
112.01	0.01\\
113.01	0.01\\
114.01	0.01\\
115.01	0.01\\
116.01	0.01\\
117.01	0.01\\
118.01	0.01\\
119.01	0.01\\
120.01	0.01\\
121.01	0.01\\
122.01	0.01\\
123.01	0.01\\
124.01	0.01\\
125.01	0.01\\
126.01	0.01\\
127.01	0.01\\
128.01	0.01\\
129.01	0.01\\
130.01	0.01\\
131.01	0.01\\
132.01	0.01\\
133.01	0.01\\
134.01	0.01\\
135.01	0.01\\
136.01	0.01\\
137.01	0.01\\
138.01	0.01\\
139.01	0.01\\
140.01	0.01\\
141.01	0.01\\
142.01	0.01\\
143.01	0.01\\
144.01	0.01\\
145.01	0.01\\
146.01	0.01\\
147.01	0.01\\
148.01	0.01\\
149.01	0.01\\
150.01	0.01\\
151.01	0.01\\
152.01	0.01\\
153.01	0.01\\
154.01	0.01\\
155.01	0.01\\
156.01	0.01\\
157.01	0.01\\
158.01	0.01\\
159.01	0.01\\
160.01	0.01\\
161.01	0.01\\
162.01	0.01\\
163.01	0.01\\
164.01	0.01\\
165.01	0.01\\
166.01	0.01\\
167.01	0.01\\
168.01	0.01\\
169.01	0.01\\
170.01	0.01\\
171.01	0.01\\
172.01	0.01\\
173.01	0.01\\
174.01	0.01\\
175.01	0.01\\
176.01	0.01\\
177.01	0.01\\
178.01	0.01\\
179.01	0.01\\
180.01	0.01\\
181.01	0.01\\
182.01	0.01\\
183.01	0.01\\
184.01	0.01\\
185.01	0.01\\
186.01	0.01\\
187.01	0.01\\
188.01	0.01\\
189.01	0.01\\
190.01	0.01\\
191.01	0.01\\
192.01	0.01\\
193.01	0.01\\
194.01	0.01\\
195.01	0.01\\
196.01	0.01\\
197.01	0.01\\
198.01	0.01\\
199.01	0.01\\
200.01	0.01\\
201.01	0.01\\
202.01	0.01\\
203.01	0.01\\
204.01	0.01\\
205.01	0.01\\
206.01	0.01\\
207.01	0.01\\
208.01	0.01\\
209.01	0.01\\
210.01	0.01\\
211.01	0.01\\
212.01	0.01\\
213.01	0.01\\
214.01	0.01\\
215.01	0.01\\
216.01	0.01\\
217.01	0.01\\
218.01	0.01\\
219.01	0.01\\
220.01	0.01\\
221.01	0.01\\
222.01	0.01\\
223.01	0.01\\
224.01	0.01\\
225.01	0.01\\
226.01	0.01\\
227.01	0.01\\
228.01	0.01\\
229.01	0.01\\
230.01	0.01\\
231.01	0.01\\
232.01	0.01\\
233.01	0.01\\
234.01	0.01\\
235.01	0.01\\
236.01	0.01\\
237.01	0.01\\
238.01	0.01\\
239.01	0.01\\
240.01	0.01\\
241.01	0.01\\
242.01	0.01\\
243.01	0.01\\
244.01	0.01\\
245.01	0.01\\
246.01	0.01\\
247.01	0.01\\
248.01	0.01\\
249.01	0.01\\
250.01	0.01\\
251.01	0.01\\
252.01	0.01\\
253.01	0.01\\
254.01	0.01\\
255.01	0.01\\
256.01	0.01\\
257.01	0.01\\
258.01	0.01\\
259.01	0.01\\
260.01	0.01\\
261.01	0.01\\
262.01	0.01\\
263.01	0.01\\
264.01	0.01\\
265.01	0.01\\
266.01	0.01\\
267.01	0.01\\
268.01	0.01\\
269.01	0.01\\
270.01	0.01\\
271.01	0.01\\
272.01	0.01\\
273.01	0.01\\
274.01	0.01\\
275.01	0.01\\
276.01	0.01\\
277.01	0.01\\
278.01	0.01\\
279.01	0.01\\
280.01	0.01\\
281.01	0.01\\
282.01	0.01\\
283.01	0.01\\
284.01	0.01\\
285.01	0.01\\
286.01	0.01\\
287.01	0.01\\
288.01	0.01\\
289.01	0.01\\
290.01	0.01\\
291.01	0.01\\
292.01	0.01\\
293.01	0.01\\
294.01	0.01\\
295.01	0.01\\
296.01	0.01\\
297.01	0.01\\
298.01	0.01\\
299.01	0.01\\
300.01	0.01\\
301.01	0.01\\
302.01	0.01\\
303.01	0.01\\
304.01	0.01\\
305.01	0.01\\
306.01	0.01\\
307.01	0.01\\
308.01	0.01\\
309.01	0.01\\
310.01	0.01\\
311.01	0.01\\
312.01	0.01\\
313.01	0.01\\
314.01	0.01\\
315.01	0.01\\
316.01	0.01\\
317.01	0.01\\
318.01	0.01\\
319.01	0.01\\
320.01	0.01\\
321.01	0.01\\
322.01	0.01\\
323.01	0.01\\
324.01	0.01\\
325.01	0.01\\
326.01	0.01\\
327.01	0.01\\
328.01	0.01\\
329.01	0.01\\
330.01	0.01\\
331.01	0.01\\
332.01	0.01\\
333.01	0.01\\
334.01	0.01\\
335.01	0.01\\
336.01	0.01\\
337.01	0.01\\
338.01	0.01\\
339.01	0.01\\
340.01	0.01\\
341.01	0.01\\
342.01	0.01\\
343.01	0.01\\
344.01	0.01\\
345.01	0.01\\
346.01	0.01\\
347.01	0.01\\
348.01	0.01\\
349.01	0.01\\
350.01	0.01\\
351.01	0.01\\
352.01	0.01\\
353.01	0.01\\
354.01	0.01\\
355.01	0.01\\
356.01	0.01\\
357.01	0.01\\
358.01	0.01\\
359.01	0.01\\
360.01	0.01\\
361.01	0.01\\
362.01	0.01\\
363.01	0.01\\
364.01	0.01\\
365.01	0.01\\
366.01	0.01\\
367.01	0.01\\
368.01	0.01\\
369.01	0.01\\
370.01	0.01\\
371.01	0.01\\
372.01	0.01\\
373.01	0.01\\
374.01	0.01\\
375.01	0.01\\
376.01	0.01\\
377.01	0.01\\
378.01	0.01\\
379.01	0.01\\
380.01	0.01\\
381.01	0.01\\
382.01	0.01\\
383.01	0.01\\
384.01	0.01\\
385.01	0.01\\
386.01	0.01\\
387.01	0.01\\
388.01	0.01\\
389.01	0.01\\
390.01	0.01\\
391.01	0.01\\
392.01	0.01\\
393.01	0.01\\
394.01	0.01\\
395.01	0.01\\
396.01	0.01\\
397.01	0.01\\
398.01	0.01\\
399.01	0.01\\
400.01	0.01\\
401.01	0.01\\
402.01	0.01\\
403.01	0.01\\
404.01	0.01\\
405.01	0.01\\
406.01	0.01\\
407.01	0.01\\
408.01	0.01\\
409.01	0.01\\
410.01	0.01\\
411.01	0.01\\
412.01	0.01\\
413.01	0.01\\
414.01	0.01\\
415.01	0.01\\
416.01	0.01\\
417.01	0.01\\
418.01	0.01\\
419.01	0.01\\
420.01	0.01\\
421.01	0.01\\
422.01	0.01\\
423.01	0.01\\
424.01	0.01\\
425.01	0.01\\
426.01	0.01\\
427.01	0.01\\
428.01	0.01\\
429.01	0.01\\
430.01	0.01\\
431.01	0.01\\
432.01	0.01\\
433.01	0.01\\
434.01	0.01\\
435.01	0.01\\
436.01	0.01\\
437.01	0.01\\
438.01	0.01\\
439.01	0.01\\
440.01	0.01\\
441.01	0.01\\
442.01	0.01\\
443.01	0.01\\
444.01	0.01\\
445.01	0.01\\
446.01	0.01\\
447.01	0.01\\
448.01	0.01\\
449.01	0.01\\
450.01	0.01\\
451.01	0.01\\
452.01	0.01\\
453.01	0.01\\
454.01	0.01\\
455.01	0.01\\
456.01	0.01\\
457.01	0.01\\
458.01	0.01\\
459.01	0.01\\
460.01	0.01\\
461.01	0.01\\
462.01	0.01\\
463.01	0.01\\
464.01	0.01\\
465.01	0.01\\
466.01	0.01\\
467.01	0.01\\
468.01	0.01\\
469.01	0.01\\
470.01	0.01\\
471.01	0.01\\
472.01	0.01\\
473.01	0.01\\
474.01	0.01\\
475.01	0.01\\
476.01	0.01\\
477.01	0.01\\
478.01	0.01\\
479.01	0.01\\
480.01	0.01\\
481.01	0.01\\
482.01	0.01\\
483.01	0.01\\
484.01	0.01\\
485.01	0.01\\
486.01	0.01\\
487.01	0.01\\
488.01	0.01\\
489.01	0.01\\
490.01	0.01\\
491.01	0.01\\
492.01	0.01\\
493.01	0.01\\
494.01	0.01\\
495.01	0.01\\
496.01	0.01\\
497.01	0.01\\
498.01	0.01\\
499.01	0.01\\
500.01	0.01\\
501.01	0.01\\
502.01	0.01\\
503.01	0.01\\
504.01	0.01\\
505.01	0.01\\
506.01	0.01\\
507.01	0.01\\
508.01	0.01\\
509.01	0.01\\
510.01	0.01\\
511.01	0.01\\
512.01	0.01\\
513.01	0.01\\
514.01	0.01\\
515.01	0.01\\
516.01	0.01\\
517.01	0.01\\
518.01	0.01\\
519.01	0.01\\
520.01	0.01\\
521.01	0.01\\
522.01	0.01\\
523.01	0.01\\
524.01	0.01\\
525.01	0.01\\
526.01	0.01\\
527.01	0.01\\
528.01	0.01\\
529.01	0.01\\
530.01	0.01\\
531.01	0.01\\
532.01	0.01\\
533.01	0.01\\
534.01	0.01\\
535.01	0.01\\
536.01	0.01\\
537.01	0.01\\
538.01	0.01\\
539.01	0.01\\
540.01	0.01\\
541.01	0.01\\
542.01	0.01\\
543.01	0.01\\
544.01	0.01\\
545.01	0.01\\
546.01	0.01\\
547.01	0.01\\
548.01	0.01\\
549.01	0.01\\
550.01	0.01\\
551.01	0.01\\
552.01	0.01\\
553.01	0.01\\
554.01	0.01\\
555.01	0.01\\
556.01	0.01\\
557.01	0.01\\
558.01	0.01\\
559.01	0.01\\
560.01	0.01\\
561.01	0.01\\
562.01	0.01\\
563.01	0.01\\
564.01	0.01\\
565.01	0.01\\
566.01	0.01\\
567.01	0.01\\
568.01	0.01\\
569.01	0.01\\
570.01	0.01\\
571.01	0.01\\
572.01	0.01\\
573.01	0.01\\
574.01	0.01\\
575.01	0.01\\
576.01	0.01\\
577.01	0.01\\
578.01	0.01\\
579.01	0.01\\
580.01	0.01\\
581.01	0.01\\
582.01	0.01\\
583.01	0.01\\
584.01	0.01\\
585.01	0.01\\
586.01	0.01\\
587.01	0.01\\
588.01	0.01\\
589.01	0.01\\
590.01	0.01\\
591.01	0.01\\
592.01	0.01\\
593.01	0.01\\
594.01	0.01\\
595.01	0.01\\
596.01	0.01\\
597.01	0.01\\
598.01	0.01\\
599.01	0.01\\
599.02	0.01\\
599.03	0.01\\
599.04	0.01\\
599.05	0.01\\
599.06	0.01\\
599.07	0.01\\
599.08	0.01\\
599.09	0.01\\
599.1	0.01\\
599.11	0.01\\
599.12	0.01\\
599.13	0.01\\
599.14	0.01\\
599.15	0.01\\
599.16	0.01\\
599.17	0.01\\
599.18	0.01\\
599.19	0.01\\
599.2	0.01\\
599.21	0.01\\
599.22	0.01\\
599.23	0.01\\
599.24	0.01\\
599.25	0.01\\
599.26	0.01\\
599.27	0.01\\
599.28	0.01\\
599.29	0.01\\
599.3	0.01\\
599.31	0.01\\
599.32	0.01\\
599.33	0.01\\
599.34	0.01\\
599.35	0.01\\
599.36	0.01\\
599.37	0.01\\
599.38	0.01\\
599.39	0.01\\
599.4	0.01\\
599.41	0.01\\
599.42	0.01\\
599.43	0.01\\
599.44	0.01\\
599.45	0.01\\
599.46	0.01\\
599.47	0.01\\
599.48	0.01\\
599.49	0.01\\
599.5	0.01\\
599.51	0.01\\
599.52	0.01\\
599.53	0.01\\
599.54	0.01\\
599.55	0.01\\
599.56	0.01\\
599.57	0.01\\
599.58	0.01\\
599.59	0.01\\
599.6	0.01\\
599.61	0.01\\
599.62	0.01\\
599.63	0.01\\
599.64	0.01\\
599.65	0.01\\
599.66	0.01\\
599.67	0.01\\
599.68	0.01\\
599.69	0.01\\
599.7	0.01\\
599.71	0.01\\
599.72	0.01\\
599.73	0.01\\
599.74	0.01\\
599.75	0.01\\
599.76	0.01\\
599.77	0.01\\
599.78	0.01\\
599.79	0.01\\
599.8	0.01\\
599.81	0.01\\
599.82	0.01\\
599.83	0.01\\
599.84	0.01\\
599.85	0.01\\
599.86	0.01\\
599.87	0.01\\
599.88	0.01\\
599.89	0.01\\
599.9	0.01\\
599.91	0.01\\
599.92	0.01\\
599.93	0.01\\
599.94	0.01\\
599.95	0.01\\
599.96	0.01\\
599.97	0.01\\
599.98	0.01\\
599.99	0.01\\
600	0.01\\
};
\addplot [color=mycolor9,solid,forget plot]
  table[row sep=crcr]{%
0.01	0.01\\
1.01	0.01\\
2.01	0.01\\
3.01	0.01\\
4.01	0.01\\
5.01	0.01\\
6.01	0.01\\
7.01	0.01\\
8.01	0.01\\
9.01	0.01\\
10.01	0.01\\
11.01	0.01\\
12.01	0.01\\
13.01	0.01\\
14.01	0.01\\
15.01	0.01\\
16.01	0.01\\
17.01	0.01\\
18.01	0.01\\
19.01	0.01\\
20.01	0.01\\
21.01	0.01\\
22.01	0.01\\
23.01	0.01\\
24.01	0.01\\
25.01	0.01\\
26.01	0.01\\
27.01	0.01\\
28.01	0.01\\
29.01	0.01\\
30.01	0.01\\
31.01	0.01\\
32.01	0.01\\
33.01	0.01\\
34.01	0.01\\
35.01	0.01\\
36.01	0.01\\
37.01	0.01\\
38.01	0.01\\
39.01	0.01\\
40.01	0.01\\
41.01	0.01\\
42.01	0.01\\
43.01	0.01\\
44.01	0.01\\
45.01	0.01\\
46.01	0.01\\
47.01	0.01\\
48.01	0.01\\
49.01	0.01\\
50.01	0.01\\
51.01	0.01\\
52.01	0.01\\
53.01	0.01\\
54.01	0.01\\
55.01	0.01\\
56.01	0.01\\
57.01	0.01\\
58.01	0.01\\
59.01	0.01\\
60.01	0.01\\
61.01	0.01\\
62.01	0.01\\
63.01	0.01\\
64.01	0.01\\
65.01	0.01\\
66.01	0.01\\
67.01	0.01\\
68.01	0.01\\
69.01	0.01\\
70.01	0.01\\
71.01	0.01\\
72.01	0.01\\
73.01	0.01\\
74.01	0.01\\
75.01	0.01\\
76.01	0.01\\
77.01	0.01\\
78.01	0.01\\
79.01	0.01\\
80.01	0.01\\
81.01	0.01\\
82.01	0.01\\
83.01	0.01\\
84.01	0.01\\
85.01	0.01\\
86.01	0.01\\
87.01	0.01\\
88.01	0.01\\
89.01	0.01\\
90.01	0.01\\
91.01	0.01\\
92.01	0.01\\
93.01	0.01\\
94.01	0.01\\
95.01	0.01\\
96.01	0.01\\
97.01	0.01\\
98.01	0.01\\
99.01	0.01\\
100.01	0.01\\
101.01	0.01\\
102.01	0.01\\
103.01	0.01\\
104.01	0.01\\
105.01	0.01\\
106.01	0.01\\
107.01	0.01\\
108.01	0.01\\
109.01	0.01\\
110.01	0.01\\
111.01	0.01\\
112.01	0.01\\
113.01	0.01\\
114.01	0.01\\
115.01	0.01\\
116.01	0.01\\
117.01	0.01\\
118.01	0.01\\
119.01	0.01\\
120.01	0.01\\
121.01	0.01\\
122.01	0.01\\
123.01	0.01\\
124.01	0.01\\
125.01	0.01\\
126.01	0.01\\
127.01	0.01\\
128.01	0.01\\
129.01	0.01\\
130.01	0.01\\
131.01	0.01\\
132.01	0.01\\
133.01	0.01\\
134.01	0.01\\
135.01	0.01\\
136.01	0.01\\
137.01	0.01\\
138.01	0.01\\
139.01	0.01\\
140.01	0.01\\
141.01	0.01\\
142.01	0.01\\
143.01	0.01\\
144.01	0.01\\
145.01	0.01\\
146.01	0.01\\
147.01	0.01\\
148.01	0.01\\
149.01	0.01\\
150.01	0.01\\
151.01	0.01\\
152.01	0.01\\
153.01	0.01\\
154.01	0.01\\
155.01	0.01\\
156.01	0.01\\
157.01	0.01\\
158.01	0.01\\
159.01	0.01\\
160.01	0.01\\
161.01	0.01\\
162.01	0.01\\
163.01	0.01\\
164.01	0.01\\
165.01	0.01\\
166.01	0.01\\
167.01	0.01\\
168.01	0.01\\
169.01	0.01\\
170.01	0.01\\
171.01	0.01\\
172.01	0.01\\
173.01	0.01\\
174.01	0.01\\
175.01	0.01\\
176.01	0.01\\
177.01	0.01\\
178.01	0.01\\
179.01	0.01\\
180.01	0.01\\
181.01	0.01\\
182.01	0.01\\
183.01	0.01\\
184.01	0.01\\
185.01	0.01\\
186.01	0.01\\
187.01	0.01\\
188.01	0.01\\
189.01	0.01\\
190.01	0.01\\
191.01	0.01\\
192.01	0.01\\
193.01	0.01\\
194.01	0.01\\
195.01	0.01\\
196.01	0.01\\
197.01	0.01\\
198.01	0.01\\
199.01	0.01\\
200.01	0.01\\
201.01	0.01\\
202.01	0.01\\
203.01	0.01\\
204.01	0.01\\
205.01	0.01\\
206.01	0.01\\
207.01	0.01\\
208.01	0.01\\
209.01	0.01\\
210.01	0.01\\
211.01	0.01\\
212.01	0.01\\
213.01	0.01\\
214.01	0.01\\
215.01	0.01\\
216.01	0.01\\
217.01	0.01\\
218.01	0.01\\
219.01	0.01\\
220.01	0.01\\
221.01	0.01\\
222.01	0.01\\
223.01	0.01\\
224.01	0.01\\
225.01	0.01\\
226.01	0.01\\
227.01	0.01\\
228.01	0.01\\
229.01	0.01\\
230.01	0.01\\
231.01	0.01\\
232.01	0.01\\
233.01	0.01\\
234.01	0.01\\
235.01	0.01\\
236.01	0.01\\
237.01	0.01\\
238.01	0.01\\
239.01	0.01\\
240.01	0.01\\
241.01	0.01\\
242.01	0.01\\
243.01	0.01\\
244.01	0.01\\
245.01	0.01\\
246.01	0.01\\
247.01	0.01\\
248.01	0.01\\
249.01	0.01\\
250.01	0.01\\
251.01	0.01\\
252.01	0.01\\
253.01	0.01\\
254.01	0.01\\
255.01	0.01\\
256.01	0.01\\
257.01	0.01\\
258.01	0.01\\
259.01	0.01\\
260.01	0.01\\
261.01	0.01\\
262.01	0.01\\
263.01	0.01\\
264.01	0.01\\
265.01	0.01\\
266.01	0.01\\
267.01	0.01\\
268.01	0.01\\
269.01	0.01\\
270.01	0.01\\
271.01	0.01\\
272.01	0.01\\
273.01	0.01\\
274.01	0.01\\
275.01	0.01\\
276.01	0.01\\
277.01	0.01\\
278.01	0.01\\
279.01	0.01\\
280.01	0.01\\
281.01	0.01\\
282.01	0.01\\
283.01	0.01\\
284.01	0.01\\
285.01	0.01\\
286.01	0.01\\
287.01	0.01\\
288.01	0.01\\
289.01	0.01\\
290.01	0.01\\
291.01	0.01\\
292.01	0.01\\
293.01	0.01\\
294.01	0.01\\
295.01	0.01\\
296.01	0.01\\
297.01	0.01\\
298.01	0.01\\
299.01	0.01\\
300.01	0.01\\
301.01	0.01\\
302.01	0.01\\
303.01	0.01\\
304.01	0.01\\
305.01	0.01\\
306.01	0.01\\
307.01	0.01\\
308.01	0.01\\
309.01	0.01\\
310.01	0.01\\
311.01	0.01\\
312.01	0.01\\
313.01	0.01\\
314.01	0.01\\
315.01	0.01\\
316.01	0.01\\
317.01	0.01\\
318.01	0.01\\
319.01	0.01\\
320.01	0.01\\
321.01	0.01\\
322.01	0.01\\
323.01	0.01\\
324.01	0.01\\
325.01	0.01\\
326.01	0.01\\
327.01	0.01\\
328.01	0.01\\
329.01	0.01\\
330.01	0.01\\
331.01	0.01\\
332.01	0.01\\
333.01	0.01\\
334.01	0.01\\
335.01	0.01\\
336.01	0.01\\
337.01	0.01\\
338.01	0.01\\
339.01	0.01\\
340.01	0.01\\
341.01	0.01\\
342.01	0.01\\
343.01	0.01\\
344.01	0.01\\
345.01	0.01\\
346.01	0.01\\
347.01	0.01\\
348.01	0.01\\
349.01	0.01\\
350.01	0.01\\
351.01	0.01\\
352.01	0.01\\
353.01	0.01\\
354.01	0.01\\
355.01	0.01\\
356.01	0.01\\
357.01	0.01\\
358.01	0.01\\
359.01	0.01\\
360.01	0.01\\
361.01	0.01\\
362.01	0.01\\
363.01	0.01\\
364.01	0.01\\
365.01	0.01\\
366.01	0.01\\
367.01	0.01\\
368.01	0.01\\
369.01	0.01\\
370.01	0.01\\
371.01	0.01\\
372.01	0.01\\
373.01	0.01\\
374.01	0.01\\
375.01	0.01\\
376.01	0.01\\
377.01	0.01\\
378.01	0.01\\
379.01	0.01\\
380.01	0.01\\
381.01	0.01\\
382.01	0.01\\
383.01	0.01\\
384.01	0.01\\
385.01	0.01\\
386.01	0.01\\
387.01	0.01\\
388.01	0.01\\
389.01	0.01\\
390.01	0.01\\
391.01	0.01\\
392.01	0.01\\
393.01	0.01\\
394.01	0.01\\
395.01	0.01\\
396.01	0.01\\
397.01	0.01\\
398.01	0.01\\
399.01	0.01\\
400.01	0.01\\
401.01	0.01\\
402.01	0.01\\
403.01	0.01\\
404.01	0.01\\
405.01	0.01\\
406.01	0.01\\
407.01	0.01\\
408.01	0.01\\
409.01	0.01\\
410.01	0.01\\
411.01	0.01\\
412.01	0.01\\
413.01	0.01\\
414.01	0.01\\
415.01	0.01\\
416.01	0.01\\
417.01	0.01\\
418.01	0.01\\
419.01	0.01\\
420.01	0.01\\
421.01	0.01\\
422.01	0.01\\
423.01	0.01\\
424.01	0.01\\
425.01	0.01\\
426.01	0.01\\
427.01	0.01\\
428.01	0.01\\
429.01	0.01\\
430.01	0.01\\
431.01	0.01\\
432.01	0.01\\
433.01	0.01\\
434.01	0.01\\
435.01	0.01\\
436.01	0.01\\
437.01	0.01\\
438.01	0.01\\
439.01	0.01\\
440.01	0.01\\
441.01	0.01\\
442.01	0.01\\
443.01	0.01\\
444.01	0.01\\
445.01	0.01\\
446.01	0.01\\
447.01	0.01\\
448.01	0.01\\
449.01	0.01\\
450.01	0.01\\
451.01	0.01\\
452.01	0.01\\
453.01	0.01\\
454.01	0.01\\
455.01	0.01\\
456.01	0.01\\
457.01	0.01\\
458.01	0.01\\
459.01	0.01\\
460.01	0.01\\
461.01	0.01\\
462.01	0.01\\
463.01	0.01\\
464.01	0.01\\
465.01	0.01\\
466.01	0.01\\
467.01	0.01\\
468.01	0.01\\
469.01	0.01\\
470.01	0.01\\
471.01	0.01\\
472.01	0.01\\
473.01	0.01\\
474.01	0.01\\
475.01	0.01\\
476.01	0.01\\
477.01	0.01\\
478.01	0.01\\
479.01	0.01\\
480.01	0.01\\
481.01	0.01\\
482.01	0.01\\
483.01	0.01\\
484.01	0.01\\
485.01	0.01\\
486.01	0.01\\
487.01	0.01\\
488.01	0.01\\
489.01	0.01\\
490.01	0.01\\
491.01	0.01\\
492.01	0.01\\
493.01	0.01\\
494.01	0.01\\
495.01	0.01\\
496.01	0.01\\
497.01	0.01\\
498.01	0.01\\
499.01	0.01\\
500.01	0.01\\
501.01	0.01\\
502.01	0.01\\
503.01	0.01\\
504.01	0.01\\
505.01	0.01\\
506.01	0.01\\
507.01	0.01\\
508.01	0.01\\
509.01	0.01\\
510.01	0.01\\
511.01	0.01\\
512.01	0.01\\
513.01	0.01\\
514.01	0.01\\
515.01	0.01\\
516.01	0.01\\
517.01	0.01\\
518.01	0.01\\
519.01	0.01\\
520.01	0.01\\
521.01	0.01\\
522.01	0.01\\
523.01	0.01\\
524.01	0.01\\
525.01	0.01\\
526.01	0.01\\
527.01	0.01\\
528.01	0.01\\
529.01	0.01\\
530.01	0.01\\
531.01	0.01\\
532.01	0.01\\
533.01	0.01\\
534.01	0.01\\
535.01	0.01\\
536.01	0.01\\
537.01	0.01\\
538.01	0.01\\
539.01	0.01\\
540.01	0.01\\
541.01	0.01\\
542.01	0.01\\
543.01	0.01\\
544.01	0.01\\
545.01	0.01\\
546.01	0.01\\
547.01	0.01\\
548.01	0.01\\
549.01	0.01\\
550.01	0.01\\
551.01	0.01\\
552.01	0.01\\
553.01	0.01\\
554.01	0.01\\
555.01	0.01\\
556.01	0.01\\
557.01	0.01\\
558.01	0.01\\
559.01	0.01\\
560.01	0.01\\
561.01	0.01\\
562.01	0.01\\
563.01	0.01\\
564.01	0.01\\
565.01	0.01\\
566.01	0.01\\
567.01	0.01\\
568.01	0.01\\
569.01	0.01\\
570.01	0.01\\
571.01	0.01\\
572.01	0.01\\
573.01	0.01\\
574.01	0.01\\
575.01	0.01\\
576.01	0.01\\
577.01	0.01\\
578.01	0.01\\
579.01	0.01\\
580.01	0.01\\
581.01	0.01\\
582.01	0.01\\
583.01	0.01\\
584.01	0.01\\
585.01	0.01\\
586.01	0.01\\
587.01	0.01\\
588.01	0.01\\
589.01	0.01\\
590.01	0.01\\
591.01	0.01\\
592.01	0.01\\
593.01	0.01\\
594.01	0.01\\
595.01	0.01\\
596.01	0.01\\
597.01	0.01\\
598.01	0.01\\
599.01	0.01\\
599.02	0.01\\
599.03	0.01\\
599.04	0.01\\
599.05	0.01\\
599.06	0.01\\
599.07	0.01\\
599.08	0.01\\
599.09	0.01\\
599.1	0.01\\
599.11	0.01\\
599.12	0.01\\
599.13	0.01\\
599.14	0.01\\
599.15	0.01\\
599.16	0.01\\
599.17	0.01\\
599.18	0.01\\
599.19	0.01\\
599.2	0.01\\
599.21	0.01\\
599.22	0.01\\
599.23	0.01\\
599.24	0.01\\
599.25	0.01\\
599.26	0.01\\
599.27	0.01\\
599.28	0.01\\
599.29	0.01\\
599.3	0.01\\
599.31	0.01\\
599.32	0.01\\
599.33	0.01\\
599.34	0.01\\
599.35	0.01\\
599.36	0.01\\
599.37	0.01\\
599.38	0.01\\
599.39	0.01\\
599.4	0.01\\
599.41	0.01\\
599.42	0.01\\
599.43	0.01\\
599.44	0.01\\
599.45	0.01\\
599.46	0.01\\
599.47	0.01\\
599.48	0.01\\
599.49	0.01\\
599.5	0.01\\
599.51	0.01\\
599.52	0.01\\
599.53	0.01\\
599.54	0.01\\
599.55	0.01\\
599.56	0.01\\
599.57	0.01\\
599.58	0.01\\
599.59	0.01\\
599.6	0.01\\
599.61	0.01\\
599.62	0.01\\
599.63	0.01\\
599.64	0.01\\
599.65	0.01\\
599.66	0.01\\
599.67	0.01\\
599.68	0.01\\
599.69	0.01\\
599.7	0.01\\
599.71	0.01\\
599.72	0.01\\
599.73	0.01\\
599.74	0.01\\
599.75	0.01\\
599.76	0.01\\
599.77	0.01\\
599.78	0.01\\
599.79	0.01\\
599.8	0.01\\
599.81	0.01\\
599.82	0.01\\
599.83	0.01\\
599.84	0.01\\
599.85	0.01\\
599.86	0.01\\
599.87	0.01\\
599.88	0.01\\
599.89	0.01\\
599.9	0.01\\
599.91	0.01\\
599.92	0.01\\
599.93	0.01\\
599.94	0.01\\
599.95	0.01\\
599.96	0.01\\
599.97	0.01\\
599.98	0.01\\
599.99	0.01\\
600	0.01\\
};
\addplot [color=blue!50!mycolor7,solid,forget plot]
  table[row sep=crcr]{%
0.01	0.01\\
1.01	0.01\\
2.01	0.01\\
3.01	0.01\\
4.01	0.01\\
5.01	0.01\\
6.01	0.01\\
7.01	0.01\\
8.01	0.01\\
9.01	0.01\\
10.01	0.01\\
11.01	0.01\\
12.01	0.01\\
13.01	0.01\\
14.01	0.01\\
15.01	0.01\\
16.01	0.01\\
17.01	0.01\\
18.01	0.01\\
19.01	0.01\\
20.01	0.01\\
21.01	0.01\\
22.01	0.01\\
23.01	0.01\\
24.01	0.01\\
25.01	0.01\\
26.01	0.01\\
27.01	0.01\\
28.01	0.01\\
29.01	0.01\\
30.01	0.01\\
31.01	0.01\\
32.01	0.01\\
33.01	0.01\\
34.01	0.01\\
35.01	0.01\\
36.01	0.01\\
37.01	0.01\\
38.01	0.01\\
39.01	0.01\\
40.01	0.01\\
41.01	0.01\\
42.01	0.01\\
43.01	0.01\\
44.01	0.01\\
45.01	0.01\\
46.01	0.01\\
47.01	0.01\\
48.01	0.01\\
49.01	0.01\\
50.01	0.01\\
51.01	0.01\\
52.01	0.01\\
53.01	0.01\\
54.01	0.01\\
55.01	0.01\\
56.01	0.01\\
57.01	0.01\\
58.01	0.01\\
59.01	0.01\\
60.01	0.01\\
61.01	0.01\\
62.01	0.01\\
63.01	0.01\\
64.01	0.01\\
65.01	0.01\\
66.01	0.01\\
67.01	0.01\\
68.01	0.01\\
69.01	0.01\\
70.01	0.01\\
71.01	0.01\\
72.01	0.01\\
73.01	0.01\\
74.01	0.01\\
75.01	0.01\\
76.01	0.01\\
77.01	0.01\\
78.01	0.01\\
79.01	0.01\\
80.01	0.01\\
81.01	0.01\\
82.01	0.01\\
83.01	0.01\\
84.01	0.01\\
85.01	0.01\\
86.01	0.01\\
87.01	0.01\\
88.01	0.01\\
89.01	0.01\\
90.01	0.01\\
91.01	0.01\\
92.01	0.01\\
93.01	0.01\\
94.01	0.01\\
95.01	0.01\\
96.01	0.01\\
97.01	0.01\\
98.01	0.01\\
99.01	0.01\\
100.01	0.01\\
101.01	0.01\\
102.01	0.01\\
103.01	0.01\\
104.01	0.01\\
105.01	0.01\\
106.01	0.01\\
107.01	0.01\\
108.01	0.01\\
109.01	0.01\\
110.01	0.01\\
111.01	0.01\\
112.01	0.01\\
113.01	0.01\\
114.01	0.01\\
115.01	0.01\\
116.01	0.01\\
117.01	0.01\\
118.01	0.01\\
119.01	0.01\\
120.01	0.01\\
121.01	0.01\\
122.01	0.01\\
123.01	0.01\\
124.01	0.01\\
125.01	0.01\\
126.01	0.01\\
127.01	0.01\\
128.01	0.01\\
129.01	0.01\\
130.01	0.01\\
131.01	0.01\\
132.01	0.01\\
133.01	0.01\\
134.01	0.01\\
135.01	0.01\\
136.01	0.01\\
137.01	0.01\\
138.01	0.01\\
139.01	0.01\\
140.01	0.01\\
141.01	0.01\\
142.01	0.01\\
143.01	0.01\\
144.01	0.01\\
145.01	0.01\\
146.01	0.01\\
147.01	0.01\\
148.01	0.01\\
149.01	0.01\\
150.01	0.01\\
151.01	0.01\\
152.01	0.01\\
153.01	0.01\\
154.01	0.01\\
155.01	0.01\\
156.01	0.01\\
157.01	0.01\\
158.01	0.01\\
159.01	0.01\\
160.01	0.01\\
161.01	0.01\\
162.01	0.01\\
163.01	0.01\\
164.01	0.01\\
165.01	0.01\\
166.01	0.01\\
167.01	0.01\\
168.01	0.01\\
169.01	0.01\\
170.01	0.01\\
171.01	0.01\\
172.01	0.01\\
173.01	0.01\\
174.01	0.01\\
175.01	0.01\\
176.01	0.01\\
177.01	0.01\\
178.01	0.01\\
179.01	0.01\\
180.01	0.01\\
181.01	0.01\\
182.01	0.01\\
183.01	0.01\\
184.01	0.01\\
185.01	0.01\\
186.01	0.01\\
187.01	0.01\\
188.01	0.01\\
189.01	0.01\\
190.01	0.01\\
191.01	0.01\\
192.01	0.01\\
193.01	0.01\\
194.01	0.01\\
195.01	0.01\\
196.01	0.01\\
197.01	0.01\\
198.01	0.01\\
199.01	0.01\\
200.01	0.01\\
201.01	0.01\\
202.01	0.01\\
203.01	0.01\\
204.01	0.01\\
205.01	0.01\\
206.01	0.01\\
207.01	0.01\\
208.01	0.01\\
209.01	0.01\\
210.01	0.01\\
211.01	0.01\\
212.01	0.01\\
213.01	0.01\\
214.01	0.01\\
215.01	0.01\\
216.01	0.01\\
217.01	0.01\\
218.01	0.01\\
219.01	0.01\\
220.01	0.01\\
221.01	0.01\\
222.01	0.01\\
223.01	0.01\\
224.01	0.01\\
225.01	0.01\\
226.01	0.01\\
227.01	0.01\\
228.01	0.01\\
229.01	0.01\\
230.01	0.01\\
231.01	0.01\\
232.01	0.01\\
233.01	0.01\\
234.01	0.01\\
235.01	0.01\\
236.01	0.01\\
237.01	0.01\\
238.01	0.01\\
239.01	0.01\\
240.01	0.01\\
241.01	0.01\\
242.01	0.01\\
243.01	0.01\\
244.01	0.01\\
245.01	0.01\\
246.01	0.01\\
247.01	0.01\\
248.01	0.01\\
249.01	0.01\\
250.01	0.01\\
251.01	0.01\\
252.01	0.01\\
253.01	0.01\\
254.01	0.01\\
255.01	0.01\\
256.01	0.01\\
257.01	0.01\\
258.01	0.01\\
259.01	0.01\\
260.01	0.01\\
261.01	0.01\\
262.01	0.01\\
263.01	0.01\\
264.01	0.01\\
265.01	0.01\\
266.01	0.01\\
267.01	0.01\\
268.01	0.01\\
269.01	0.01\\
270.01	0.01\\
271.01	0.01\\
272.01	0.01\\
273.01	0.01\\
274.01	0.01\\
275.01	0.01\\
276.01	0.01\\
277.01	0.01\\
278.01	0.01\\
279.01	0.01\\
280.01	0.01\\
281.01	0.01\\
282.01	0.01\\
283.01	0.01\\
284.01	0.01\\
285.01	0.01\\
286.01	0.01\\
287.01	0.01\\
288.01	0.01\\
289.01	0.01\\
290.01	0.01\\
291.01	0.01\\
292.01	0.01\\
293.01	0.01\\
294.01	0.01\\
295.01	0.01\\
296.01	0.01\\
297.01	0.01\\
298.01	0.01\\
299.01	0.01\\
300.01	0.01\\
301.01	0.01\\
302.01	0.01\\
303.01	0.01\\
304.01	0.01\\
305.01	0.01\\
306.01	0.01\\
307.01	0.01\\
308.01	0.01\\
309.01	0.01\\
310.01	0.01\\
311.01	0.01\\
312.01	0.01\\
313.01	0.01\\
314.01	0.01\\
315.01	0.01\\
316.01	0.01\\
317.01	0.01\\
318.01	0.01\\
319.01	0.01\\
320.01	0.01\\
321.01	0.01\\
322.01	0.01\\
323.01	0.01\\
324.01	0.01\\
325.01	0.01\\
326.01	0.01\\
327.01	0.01\\
328.01	0.01\\
329.01	0.01\\
330.01	0.01\\
331.01	0.01\\
332.01	0.01\\
333.01	0.01\\
334.01	0.01\\
335.01	0.01\\
336.01	0.01\\
337.01	0.01\\
338.01	0.01\\
339.01	0.01\\
340.01	0.01\\
341.01	0.01\\
342.01	0.01\\
343.01	0.01\\
344.01	0.01\\
345.01	0.01\\
346.01	0.01\\
347.01	0.01\\
348.01	0.01\\
349.01	0.01\\
350.01	0.01\\
351.01	0.01\\
352.01	0.01\\
353.01	0.01\\
354.01	0.01\\
355.01	0.01\\
356.01	0.01\\
357.01	0.01\\
358.01	0.01\\
359.01	0.01\\
360.01	0.01\\
361.01	0.01\\
362.01	0.01\\
363.01	0.01\\
364.01	0.01\\
365.01	0.01\\
366.01	0.01\\
367.01	0.01\\
368.01	0.01\\
369.01	0.01\\
370.01	0.01\\
371.01	0.01\\
372.01	0.01\\
373.01	0.01\\
374.01	0.01\\
375.01	0.01\\
376.01	0.01\\
377.01	0.01\\
378.01	0.01\\
379.01	0.01\\
380.01	0.01\\
381.01	0.01\\
382.01	0.01\\
383.01	0.01\\
384.01	0.01\\
385.01	0.01\\
386.01	0.01\\
387.01	0.01\\
388.01	0.01\\
389.01	0.01\\
390.01	0.01\\
391.01	0.01\\
392.01	0.01\\
393.01	0.01\\
394.01	0.01\\
395.01	0.01\\
396.01	0.01\\
397.01	0.01\\
398.01	0.01\\
399.01	0.01\\
400.01	0.01\\
401.01	0.01\\
402.01	0.01\\
403.01	0.01\\
404.01	0.01\\
405.01	0.01\\
406.01	0.01\\
407.01	0.01\\
408.01	0.01\\
409.01	0.01\\
410.01	0.01\\
411.01	0.01\\
412.01	0.01\\
413.01	0.01\\
414.01	0.01\\
415.01	0.01\\
416.01	0.01\\
417.01	0.01\\
418.01	0.01\\
419.01	0.01\\
420.01	0.01\\
421.01	0.01\\
422.01	0.01\\
423.01	0.01\\
424.01	0.01\\
425.01	0.01\\
426.01	0.01\\
427.01	0.01\\
428.01	0.01\\
429.01	0.01\\
430.01	0.01\\
431.01	0.01\\
432.01	0.01\\
433.01	0.01\\
434.01	0.01\\
435.01	0.01\\
436.01	0.01\\
437.01	0.01\\
438.01	0.01\\
439.01	0.01\\
440.01	0.01\\
441.01	0.01\\
442.01	0.01\\
443.01	0.01\\
444.01	0.01\\
445.01	0.01\\
446.01	0.01\\
447.01	0.01\\
448.01	0.01\\
449.01	0.01\\
450.01	0.01\\
451.01	0.01\\
452.01	0.01\\
453.01	0.01\\
454.01	0.01\\
455.01	0.01\\
456.01	0.01\\
457.01	0.01\\
458.01	0.01\\
459.01	0.01\\
460.01	0.01\\
461.01	0.01\\
462.01	0.01\\
463.01	0.01\\
464.01	0.01\\
465.01	0.01\\
466.01	0.01\\
467.01	0.01\\
468.01	0.01\\
469.01	0.01\\
470.01	0.01\\
471.01	0.01\\
472.01	0.01\\
473.01	0.01\\
474.01	0.01\\
475.01	0.01\\
476.01	0.01\\
477.01	0.01\\
478.01	0.01\\
479.01	0.01\\
480.01	0.01\\
481.01	0.01\\
482.01	0.01\\
483.01	0.01\\
484.01	0.01\\
485.01	0.01\\
486.01	0.01\\
487.01	0.01\\
488.01	0.01\\
489.01	0.01\\
490.01	0.01\\
491.01	0.01\\
492.01	0.01\\
493.01	0.01\\
494.01	0.01\\
495.01	0.01\\
496.01	0.01\\
497.01	0.01\\
498.01	0.01\\
499.01	0.01\\
500.01	0.01\\
501.01	0.01\\
502.01	0.01\\
503.01	0.01\\
504.01	0.01\\
505.01	0.01\\
506.01	0.01\\
507.01	0.01\\
508.01	0.01\\
509.01	0.01\\
510.01	0.01\\
511.01	0.01\\
512.01	0.01\\
513.01	0.01\\
514.01	0.01\\
515.01	0.01\\
516.01	0.01\\
517.01	0.01\\
518.01	0.01\\
519.01	0.01\\
520.01	0.01\\
521.01	0.01\\
522.01	0.01\\
523.01	0.01\\
524.01	0.01\\
525.01	0.01\\
526.01	0.01\\
527.01	0.01\\
528.01	0.01\\
529.01	0.01\\
530.01	0.01\\
531.01	0.01\\
532.01	0.01\\
533.01	0.01\\
534.01	0.01\\
535.01	0.01\\
536.01	0.01\\
537.01	0.01\\
538.01	0.01\\
539.01	0.01\\
540.01	0.01\\
541.01	0.01\\
542.01	0.01\\
543.01	0.01\\
544.01	0.01\\
545.01	0.01\\
546.01	0.01\\
547.01	0.01\\
548.01	0.01\\
549.01	0.01\\
550.01	0.01\\
551.01	0.01\\
552.01	0.01\\
553.01	0.01\\
554.01	0.01\\
555.01	0.01\\
556.01	0.01\\
557.01	0.01\\
558.01	0.01\\
559.01	0.01\\
560.01	0.01\\
561.01	0.01\\
562.01	0.01\\
563.01	0.01\\
564.01	0.01\\
565.01	0.01\\
566.01	0.01\\
567.01	0.01\\
568.01	0.01\\
569.01	0.01\\
570.01	0.01\\
571.01	0.01\\
572.01	0.01\\
573.01	0.01\\
574.01	0.01\\
575.01	0.01\\
576.01	0.01\\
577.01	0.01\\
578.01	0.01\\
579.01	0.01\\
580.01	0.01\\
581.01	0.01\\
582.01	0.01\\
583.01	0.01\\
584.01	0.01\\
585.01	0.01\\
586.01	0.01\\
587.01	0.01\\
588.01	0.01\\
589.01	0.01\\
590.01	0.01\\
591.01	0.01\\
592.01	0.01\\
593.01	0.01\\
594.01	0.01\\
595.01	0.01\\
596.01	0.01\\
597.01	0.01\\
598.01	0.01\\
599.01	0.01\\
599.02	0.01\\
599.03	0.01\\
599.04	0.01\\
599.05	0.01\\
599.06	0.01\\
599.07	0.01\\
599.08	0.01\\
599.09	0.01\\
599.1	0.01\\
599.11	0.01\\
599.12	0.01\\
599.13	0.01\\
599.14	0.01\\
599.15	0.01\\
599.16	0.01\\
599.17	0.01\\
599.18	0.01\\
599.19	0.01\\
599.2	0.01\\
599.21	0.01\\
599.22	0.01\\
599.23	0.01\\
599.24	0.01\\
599.25	0.01\\
599.26	0.01\\
599.27	0.01\\
599.28	0.01\\
599.29	0.01\\
599.3	0.01\\
599.31	0.01\\
599.32	0.01\\
599.33	0.01\\
599.34	0.01\\
599.35	0.01\\
599.36	0.01\\
599.37	0.01\\
599.38	0.01\\
599.39	0.01\\
599.4	0.01\\
599.41	0.01\\
599.42	0.01\\
599.43	0.01\\
599.44	0.01\\
599.45	0.01\\
599.46	0.01\\
599.47	0.01\\
599.48	0.01\\
599.49	0.01\\
599.5	0.01\\
599.51	0.01\\
599.52	0.01\\
599.53	0.01\\
599.54	0.01\\
599.55	0.01\\
599.56	0.01\\
599.57	0.01\\
599.58	0.01\\
599.59	0.01\\
599.6	0.01\\
599.61	0.01\\
599.62	0.01\\
599.63	0.01\\
599.64	0.01\\
599.65	0.01\\
599.66	0.01\\
599.67	0.01\\
599.68	0.01\\
599.69	0.01\\
599.7	0.01\\
599.71	0.01\\
599.72	0.01\\
599.73	0.01\\
599.74	0.01\\
599.75	0.01\\
599.76	0.01\\
599.77	0.01\\
599.78	0.01\\
599.79	0.01\\
599.8	0.01\\
599.81	0.01\\
599.82	0.01\\
599.83	0.01\\
599.84	0.01\\
599.85	0.01\\
599.86	0.01\\
599.87	0.01\\
599.88	0.01\\
599.89	0.01\\
599.9	0.01\\
599.91	0.01\\
599.92	0.01\\
599.93	0.01\\
599.94	0.01\\
599.95	0.01\\
599.96	0.01\\
599.97	0.01\\
599.98	0.01\\
599.99	0.01\\
600	0.01\\
};
\addplot [color=blue!40!mycolor9,solid,forget plot]
  table[row sep=crcr]{%
0.01	0.01\\
1.01	0.01\\
2.01	0.01\\
3.01	0.01\\
4.01	0.01\\
5.01	0.01\\
6.01	0.01\\
7.01	0.01\\
8.01	0.01\\
9.01	0.01\\
10.01	0.01\\
11.01	0.01\\
12.01	0.01\\
13.01	0.01\\
14.01	0.01\\
15.01	0.01\\
16.01	0.01\\
17.01	0.01\\
18.01	0.01\\
19.01	0.01\\
20.01	0.01\\
21.01	0.01\\
22.01	0.01\\
23.01	0.01\\
24.01	0.01\\
25.01	0.01\\
26.01	0.01\\
27.01	0.01\\
28.01	0.01\\
29.01	0.01\\
30.01	0.01\\
31.01	0.01\\
32.01	0.01\\
33.01	0.01\\
34.01	0.01\\
35.01	0.01\\
36.01	0.01\\
37.01	0.01\\
38.01	0.01\\
39.01	0.01\\
40.01	0.01\\
41.01	0.01\\
42.01	0.01\\
43.01	0.01\\
44.01	0.01\\
45.01	0.01\\
46.01	0.01\\
47.01	0.01\\
48.01	0.01\\
49.01	0.01\\
50.01	0.01\\
51.01	0.01\\
52.01	0.01\\
53.01	0.01\\
54.01	0.01\\
55.01	0.01\\
56.01	0.01\\
57.01	0.01\\
58.01	0.01\\
59.01	0.01\\
60.01	0.01\\
61.01	0.01\\
62.01	0.01\\
63.01	0.01\\
64.01	0.01\\
65.01	0.01\\
66.01	0.01\\
67.01	0.01\\
68.01	0.01\\
69.01	0.01\\
70.01	0.01\\
71.01	0.01\\
72.01	0.01\\
73.01	0.01\\
74.01	0.01\\
75.01	0.01\\
76.01	0.01\\
77.01	0.01\\
78.01	0.01\\
79.01	0.01\\
80.01	0.01\\
81.01	0.01\\
82.01	0.01\\
83.01	0.01\\
84.01	0.01\\
85.01	0.01\\
86.01	0.01\\
87.01	0.01\\
88.01	0.01\\
89.01	0.01\\
90.01	0.01\\
91.01	0.01\\
92.01	0.01\\
93.01	0.01\\
94.01	0.01\\
95.01	0.01\\
96.01	0.01\\
97.01	0.01\\
98.01	0.01\\
99.01	0.01\\
100.01	0.01\\
101.01	0.01\\
102.01	0.01\\
103.01	0.01\\
104.01	0.01\\
105.01	0.01\\
106.01	0.01\\
107.01	0.01\\
108.01	0.01\\
109.01	0.01\\
110.01	0.01\\
111.01	0.01\\
112.01	0.01\\
113.01	0.01\\
114.01	0.01\\
115.01	0.01\\
116.01	0.01\\
117.01	0.01\\
118.01	0.01\\
119.01	0.01\\
120.01	0.01\\
121.01	0.01\\
122.01	0.01\\
123.01	0.01\\
124.01	0.01\\
125.01	0.01\\
126.01	0.01\\
127.01	0.01\\
128.01	0.01\\
129.01	0.01\\
130.01	0.01\\
131.01	0.01\\
132.01	0.01\\
133.01	0.01\\
134.01	0.01\\
135.01	0.01\\
136.01	0.01\\
137.01	0.01\\
138.01	0.01\\
139.01	0.01\\
140.01	0.01\\
141.01	0.01\\
142.01	0.01\\
143.01	0.01\\
144.01	0.01\\
145.01	0.01\\
146.01	0.01\\
147.01	0.01\\
148.01	0.01\\
149.01	0.01\\
150.01	0.01\\
151.01	0.01\\
152.01	0.01\\
153.01	0.01\\
154.01	0.01\\
155.01	0.01\\
156.01	0.01\\
157.01	0.01\\
158.01	0.01\\
159.01	0.01\\
160.01	0.01\\
161.01	0.01\\
162.01	0.01\\
163.01	0.01\\
164.01	0.01\\
165.01	0.01\\
166.01	0.01\\
167.01	0.01\\
168.01	0.01\\
169.01	0.01\\
170.01	0.01\\
171.01	0.01\\
172.01	0.01\\
173.01	0.01\\
174.01	0.01\\
175.01	0.01\\
176.01	0.01\\
177.01	0.01\\
178.01	0.01\\
179.01	0.01\\
180.01	0.01\\
181.01	0.01\\
182.01	0.01\\
183.01	0.01\\
184.01	0.01\\
185.01	0.01\\
186.01	0.01\\
187.01	0.01\\
188.01	0.01\\
189.01	0.01\\
190.01	0.01\\
191.01	0.01\\
192.01	0.01\\
193.01	0.01\\
194.01	0.01\\
195.01	0.01\\
196.01	0.01\\
197.01	0.01\\
198.01	0.01\\
199.01	0.01\\
200.01	0.01\\
201.01	0.01\\
202.01	0.01\\
203.01	0.01\\
204.01	0.01\\
205.01	0.01\\
206.01	0.01\\
207.01	0.01\\
208.01	0.01\\
209.01	0.01\\
210.01	0.01\\
211.01	0.01\\
212.01	0.01\\
213.01	0.01\\
214.01	0.01\\
215.01	0.01\\
216.01	0.01\\
217.01	0.01\\
218.01	0.01\\
219.01	0.01\\
220.01	0.01\\
221.01	0.01\\
222.01	0.01\\
223.01	0.01\\
224.01	0.01\\
225.01	0.01\\
226.01	0.01\\
227.01	0.01\\
228.01	0.01\\
229.01	0.01\\
230.01	0.01\\
231.01	0.01\\
232.01	0.01\\
233.01	0.01\\
234.01	0.01\\
235.01	0.01\\
236.01	0.01\\
237.01	0.01\\
238.01	0.01\\
239.01	0.01\\
240.01	0.01\\
241.01	0.01\\
242.01	0.01\\
243.01	0.01\\
244.01	0.01\\
245.01	0.01\\
246.01	0.01\\
247.01	0.01\\
248.01	0.01\\
249.01	0.01\\
250.01	0.01\\
251.01	0.01\\
252.01	0.01\\
253.01	0.01\\
254.01	0.01\\
255.01	0.01\\
256.01	0.01\\
257.01	0.01\\
258.01	0.01\\
259.01	0.01\\
260.01	0.01\\
261.01	0.01\\
262.01	0.01\\
263.01	0.01\\
264.01	0.01\\
265.01	0.01\\
266.01	0.01\\
267.01	0.01\\
268.01	0.01\\
269.01	0.01\\
270.01	0.01\\
271.01	0.01\\
272.01	0.01\\
273.01	0.01\\
274.01	0.01\\
275.01	0.01\\
276.01	0.01\\
277.01	0.01\\
278.01	0.01\\
279.01	0.01\\
280.01	0.01\\
281.01	0.01\\
282.01	0.01\\
283.01	0.01\\
284.01	0.01\\
285.01	0.01\\
286.01	0.01\\
287.01	0.01\\
288.01	0.01\\
289.01	0.01\\
290.01	0.01\\
291.01	0.01\\
292.01	0.01\\
293.01	0.01\\
294.01	0.01\\
295.01	0.01\\
296.01	0.01\\
297.01	0.01\\
298.01	0.01\\
299.01	0.01\\
300.01	0.01\\
301.01	0.01\\
302.01	0.01\\
303.01	0.01\\
304.01	0.01\\
305.01	0.01\\
306.01	0.01\\
307.01	0.01\\
308.01	0.01\\
309.01	0.01\\
310.01	0.01\\
311.01	0.01\\
312.01	0.01\\
313.01	0.01\\
314.01	0.01\\
315.01	0.01\\
316.01	0.01\\
317.01	0.01\\
318.01	0.01\\
319.01	0.01\\
320.01	0.01\\
321.01	0.01\\
322.01	0.01\\
323.01	0.01\\
324.01	0.01\\
325.01	0.01\\
326.01	0.01\\
327.01	0.01\\
328.01	0.01\\
329.01	0.01\\
330.01	0.01\\
331.01	0.01\\
332.01	0.01\\
333.01	0.01\\
334.01	0.01\\
335.01	0.01\\
336.01	0.01\\
337.01	0.01\\
338.01	0.01\\
339.01	0.01\\
340.01	0.01\\
341.01	0.01\\
342.01	0.01\\
343.01	0.01\\
344.01	0.01\\
345.01	0.01\\
346.01	0.01\\
347.01	0.01\\
348.01	0.01\\
349.01	0.01\\
350.01	0.01\\
351.01	0.01\\
352.01	0.01\\
353.01	0.01\\
354.01	0.01\\
355.01	0.01\\
356.01	0.01\\
357.01	0.01\\
358.01	0.01\\
359.01	0.01\\
360.01	0.01\\
361.01	0.01\\
362.01	0.01\\
363.01	0.01\\
364.01	0.01\\
365.01	0.01\\
366.01	0.01\\
367.01	0.01\\
368.01	0.01\\
369.01	0.01\\
370.01	0.01\\
371.01	0.01\\
372.01	0.01\\
373.01	0.01\\
374.01	0.01\\
375.01	0.01\\
376.01	0.01\\
377.01	0.01\\
378.01	0.01\\
379.01	0.01\\
380.01	0.01\\
381.01	0.01\\
382.01	0.01\\
383.01	0.01\\
384.01	0.01\\
385.01	0.01\\
386.01	0.01\\
387.01	0.01\\
388.01	0.01\\
389.01	0.01\\
390.01	0.01\\
391.01	0.01\\
392.01	0.01\\
393.01	0.01\\
394.01	0.01\\
395.01	0.01\\
396.01	0.01\\
397.01	0.01\\
398.01	0.01\\
399.01	0.01\\
400.01	0.01\\
401.01	0.01\\
402.01	0.01\\
403.01	0.01\\
404.01	0.01\\
405.01	0.01\\
406.01	0.01\\
407.01	0.01\\
408.01	0.01\\
409.01	0.01\\
410.01	0.01\\
411.01	0.01\\
412.01	0.01\\
413.01	0.01\\
414.01	0.01\\
415.01	0.01\\
416.01	0.01\\
417.01	0.01\\
418.01	0.01\\
419.01	0.01\\
420.01	0.01\\
421.01	0.01\\
422.01	0.01\\
423.01	0.01\\
424.01	0.01\\
425.01	0.01\\
426.01	0.01\\
427.01	0.01\\
428.01	0.01\\
429.01	0.01\\
430.01	0.01\\
431.01	0.01\\
432.01	0.01\\
433.01	0.01\\
434.01	0.01\\
435.01	0.01\\
436.01	0.01\\
437.01	0.01\\
438.01	0.01\\
439.01	0.01\\
440.01	0.01\\
441.01	0.01\\
442.01	0.01\\
443.01	0.01\\
444.01	0.01\\
445.01	0.01\\
446.01	0.01\\
447.01	0.01\\
448.01	0.01\\
449.01	0.01\\
450.01	0.01\\
451.01	0.01\\
452.01	0.01\\
453.01	0.01\\
454.01	0.01\\
455.01	0.01\\
456.01	0.01\\
457.01	0.01\\
458.01	0.01\\
459.01	0.01\\
460.01	0.01\\
461.01	0.01\\
462.01	0.01\\
463.01	0.01\\
464.01	0.01\\
465.01	0.01\\
466.01	0.01\\
467.01	0.01\\
468.01	0.01\\
469.01	0.01\\
470.01	0.01\\
471.01	0.01\\
472.01	0.01\\
473.01	0.01\\
474.01	0.01\\
475.01	0.01\\
476.01	0.01\\
477.01	0.01\\
478.01	0.01\\
479.01	0.01\\
480.01	0.01\\
481.01	0.01\\
482.01	0.01\\
483.01	0.01\\
484.01	0.01\\
485.01	0.01\\
486.01	0.01\\
487.01	0.01\\
488.01	0.01\\
489.01	0.01\\
490.01	0.01\\
491.01	0.01\\
492.01	0.01\\
493.01	0.01\\
494.01	0.01\\
495.01	0.01\\
496.01	0.01\\
497.01	0.01\\
498.01	0.01\\
499.01	0.01\\
500.01	0.01\\
501.01	0.01\\
502.01	0.01\\
503.01	0.01\\
504.01	0.01\\
505.01	0.01\\
506.01	0.01\\
507.01	0.01\\
508.01	0.01\\
509.01	0.01\\
510.01	0.01\\
511.01	0.01\\
512.01	0.01\\
513.01	0.01\\
514.01	0.01\\
515.01	0.01\\
516.01	0.01\\
517.01	0.01\\
518.01	0.01\\
519.01	0.01\\
520.01	0.01\\
521.01	0.01\\
522.01	0.01\\
523.01	0.01\\
524.01	0.01\\
525.01	0.01\\
526.01	0.01\\
527.01	0.01\\
528.01	0.01\\
529.01	0.01\\
530.01	0.01\\
531.01	0.01\\
532.01	0.01\\
533.01	0.01\\
534.01	0.01\\
535.01	0.01\\
536.01	0.01\\
537.01	0.01\\
538.01	0.01\\
539.01	0.01\\
540.01	0.01\\
541.01	0.01\\
542.01	0.01\\
543.01	0.01\\
544.01	0.01\\
545.01	0.01\\
546.01	0.01\\
547.01	0.01\\
548.01	0.01\\
549.01	0.01\\
550.01	0.01\\
551.01	0.01\\
552.01	0.01\\
553.01	0.01\\
554.01	0.01\\
555.01	0.01\\
556.01	0.01\\
557.01	0.01\\
558.01	0.01\\
559.01	0.01\\
560.01	0.01\\
561.01	0.01\\
562.01	0.01\\
563.01	0.01\\
564.01	0.01\\
565.01	0.01\\
566.01	0.01\\
567.01	0.01\\
568.01	0.01\\
569.01	0.01\\
570.01	0.01\\
571.01	0.01\\
572.01	0.01\\
573.01	0.01\\
574.01	0.01\\
575.01	0.01\\
576.01	0.01\\
577.01	0.01\\
578.01	0.01\\
579.01	0.01\\
580.01	0.01\\
581.01	0.01\\
582.01	0.01\\
583.01	0.01\\
584.01	0.01\\
585.01	0.01\\
586.01	0.01\\
587.01	0.01\\
588.01	0.01\\
589.01	0.01\\
590.01	0.01\\
591.01	0.01\\
592.01	0.01\\
593.01	0.01\\
594.01	0.01\\
595.01	0.01\\
596.01	0.01\\
597.01	0.01\\
598.01	0.01\\
599.01	0.01\\
599.02	0.01\\
599.03	0.01\\
599.04	0.01\\
599.05	0.01\\
599.06	0.01\\
599.07	0.01\\
599.08	0.01\\
599.09	0.01\\
599.1	0.01\\
599.11	0.01\\
599.12	0.01\\
599.13	0.01\\
599.14	0.01\\
599.15	0.01\\
599.16	0.01\\
599.17	0.01\\
599.18	0.01\\
599.19	0.01\\
599.2	0.01\\
599.21	0.01\\
599.22	0.01\\
599.23	0.01\\
599.24	0.01\\
599.25	0.01\\
599.26	0.01\\
599.27	0.01\\
599.28	0.01\\
599.29	0.01\\
599.3	0.01\\
599.31	0.01\\
599.32	0.01\\
599.33	0.01\\
599.34	0.01\\
599.35	0.01\\
599.36	0.01\\
599.37	0.01\\
599.38	0.01\\
599.39	0.01\\
599.4	0.01\\
599.41	0.01\\
599.42	0.01\\
599.43	0.01\\
599.44	0.01\\
599.45	0.01\\
599.46	0.01\\
599.47	0.01\\
599.48	0.01\\
599.49	0.01\\
599.5	0.01\\
599.51	0.01\\
599.52	0.01\\
599.53	0.01\\
599.54	0.01\\
599.55	0.01\\
599.56	0.01\\
599.57	0.01\\
599.58	0.01\\
599.59	0.01\\
599.6	0.01\\
599.61	0.01\\
599.62	0.01\\
599.63	0.01\\
599.64	0.01\\
599.65	0.01\\
599.66	0.01\\
599.67	0.01\\
599.68	0.01\\
599.69	0.01\\
599.7	0.01\\
599.71	0.01\\
599.72	0.01\\
599.73	0.01\\
599.74	0.01\\
599.75	0.01\\
599.76	0.01\\
599.77	0.01\\
599.78	0.01\\
599.79	0.01\\
599.8	0.01\\
599.81	0.01\\
599.82	0.01\\
599.83	0.01\\
599.84	0.01\\
599.85	0.01\\
599.86	0.01\\
599.87	0.01\\
599.88	0.01\\
599.89	0.01\\
599.9	0.01\\
599.91	0.01\\
599.92	0.01\\
599.93	0.01\\
599.94	0.01\\
599.95	0.01\\
599.96	0.01\\
599.97	0.01\\
599.98	0.01\\
599.99	0.01\\
600	0.01\\
};
\addplot [color=blue!75!mycolor7,solid,forget plot]
  table[row sep=crcr]{%
0.01	0.00980832450319652\\
1.01	0.0098083246691072\\
2.01	0.00980832483848332\\
3.01	0.00980832501139764\\
4.01	0.00980832518792445\\
5.01	0.0098083253681396\\
6.01	0.00980832555212057\\
7.01	0.00980832573994645\\
8.01	0.00980832593169802\\
9.01	0.00980832612745776\\
10.01	0.00980832632730991\\
11.01	0.00980832653134047\\
12.01	0.00980832673963727\\
13.01	0.00980832695229001\\
14.01	0.00980832716939028\\
15.01	0.0098083273910316\\
16.01	0.00980832761730947\\
17.01	0.00980832784832146\\
18.01	0.00980832808416714\\
19.01	0.00980832832494821\\
20.01	0.00980832857076858\\
21.01	0.00980832882173431\\
22.01	0.00980832907795365\\
23.01	0.00980832933953731\\
24.01	0.00980832960659818\\
25.01	0.00980832987925165\\
26.01	0.00980833015761553\\
27.01	0.00980833044181012\\
28.01	0.00980833073195828\\
29.01	0.00980833102818547\\
30.01	0.00980833133061986\\
31.01	0.00980833163939228\\
32.01	0.00980833195463639\\
33.01	0.00980833227648865\\
34.01	0.00980833260508845\\
35.01	0.00980833294057813\\
36.01	0.00980833328310306\\
37.01	0.00980833363281172\\
38.01	0.00980833398985571\\
39.01	0.0098083343543899\\
40.01	0.00980833472657244\\
41.01	0.00980833510656483\\
42.01	0.00980833549453204\\
43.01	0.00980833589064253\\
44.01	0.00980833629506836\\
45.01	0.00980833670798528\\
46.01	0.00980833712957271\\
47.01	0.00980833756001398\\
48.01	0.0098083379994963\\
49.01	0.00980833844821089\\
50.01	0.00980833890635301\\
51.01	0.00980833937412208\\
52.01	0.00980833985172188\\
53.01	0.0098083403393604\\
54.01	0.0098083408372502\\
55.01	0.0098083413456083\\
56.01	0.00980834186465638\\
57.01	0.00980834239462087\\
58.01	0.00980834293573302\\
59.01	0.00980834348822905\\
60.01	0.00980834405235023\\
61.01	0.00980834462834299\\
62.01	0.009808345216459\\
63.01	0.00980834581695539\\
64.01	0.00980834643009474\\
65.01	0.00980834705614526\\
66.01	0.00980834769538095\\
67.01	0.00980834834808164\\
68.01	0.00980834901453318\\
69.01	0.00980834969502753\\
70.01	0.00980835038986295\\
71.01	0.00980835109934406\\
72.01	0.00980835182378205\\
73.01	0.00980835256349478\\
74.01	0.00980835331880693\\
75.01	0.00980835409005017\\
76.01	0.0098083548775633\\
77.01	0.00980835568169239\\
78.01	0.00980835650279094\\
79.01	0.00980835734122012\\
80.01	0.00980835819734878\\
81.01	0.00980835907155379\\
82.01	0.00980835996422004\\
83.01	0.00980836087574083\\
84.01	0.00980836180651786\\
85.01	0.00980836275696146\\
86.01	0.00980836372749088\\
87.01	0.00980836471853437\\
88.01	0.00980836573052946\\
89.01	0.00980836676392303\\
90.01	0.00980836781917171\\
91.01	0.00980836889674193\\
92.01	0.0098083699971102\\
93.01	0.00980837112076337\\
94.01	0.00980837226819877\\
95.01	0.00980837343992449\\
96.01	0.00980837463645962\\
97.01	0.0098083758583345\\
98.01	0.00980837710609093\\
99.01	0.00980837838028242\\
100.01	0.00980837968147452\\
101.01	0.009808381010245\\
102.01	0.00980838236718419\\
103.01	0.00980838375289516\\
104.01	0.00980838516799412\\
105.01	0.00980838661311064\\
106.01	0.00980838808888794\\
107.01	0.00980838959598325\\
108.01	0.00980839113506801\\
109.01	0.00980839270682833\\
110.01	0.00980839431196517\\
111.01	0.00980839595119481\\
112.01	0.00980839762524905\\
113.01	0.00980839933487567\\
114.01	0.00980840108083868\\
115.01	0.0098084028639188\\
116.01	0.00980840468491372\\
117.01	0.00980840654463853\\
118.01	0.0098084084439261\\
119.01	0.0098084103836275\\
120.01	0.00980841236461236\\
121.01	0.00980841438776927\\
122.01	0.00980841645400629\\
123.01	0.0098084185642513\\
124.01	0.00980842071945245\\
125.01	0.00980842292057869\\
126.01	0.00980842516862014\\
127.01	0.00980842746458859\\
128.01	0.00980842980951798\\
129.01	0.00980843220446498\\
130.01	0.00980843465050935\\
131.01	0.00980843714875457\\
132.01	0.00980843970032834\\
133.01	0.00980844230638307\\
134.01	0.00980844496809657\\
135.01	0.00980844768667243\\
136.01	0.00980845046334078\\
137.01	0.00980845329935875\\
138.01	0.00980845619601118\\
139.01	0.00980845915461115\\
140.01	0.00980846217650069\\
141.01	0.0098084652630514\\
142.01	0.00980846841566509\\
143.01	0.00980847163577453\\
144.01	0.0098084749248441\\
145.01	0.0098084782843705\\
146.01	0.0098084817158835\\
147.01	0.00980848522094664\\
148.01	0.00980848880115809\\
149.01	0.00980849245815135\\
150.01	0.00980849619359605\\
151.01	0.00980850000919885\\
152.01	0.00980850390670418\\
153.01	0.00980850788789515\\
154.01	0.00980851195459441\\
155.01	0.00980851610866507\\
156.01	0.00980852035201162\\
157.01	0.00980852468658077\\
158.01	0.00980852911436264\\
159.01	0.00980853363739147\\
160.01	0.00980853825774685\\
161.01	0.00980854297755463\\
162.01	0.00980854779898799\\
163.01	0.00980855272426859\\
164.01	0.00980855775566759\\
165.01	0.00980856289550684\\
166.01	0.00980856814616001\\
167.01	0.00980857351005378\\
168.01	0.0098085789896691\\
169.01	0.00980858458754231\\
170.01	0.00980859030626658\\
171.01	0.00980859614849313\\
172.01	0.00980860211693249\\
173.01	0.009808608214356\\
174.01	0.00980861444359713\\
175.01	0.00980862080755291\\
176.01	0.00980862730918543\\
177.01	0.00980863395152332\\
178.01	0.00980864073766329\\
179.01	0.00980864767077166\\
180.01	0.00980865475408604\\
181.01	0.00980866199091696\\
182.01	0.00980866938464949\\
183.01	0.00980867693874505\\
184.01	0.00980868465674313\\
185.01	0.00980869254226314\\
186.01	0.00980870059900621\\
187.01	0.00980870883075711\\
188.01	0.00980871724138624\\
189.01	0.00980872583485153\\
190.01	0.00980873461520058\\
191.01	0.00980874358657264\\
192.01	0.00980875275320085\\
193.01	0.00980876211941433\\
194.01	0.0098087716896405\\
195.01	0.00980878146840735\\
196.01	0.00980879146034575\\
197.01	0.0098088016701919\\
198.01	0.00980881210278984\\
199.01	0.0098088227630938\\
200.01	0.00980883365617103\\
201.01	0.00980884478720425\\
202.01	0.00980885616149441\\
203.01	0.00980886778446354\\
204.01	0.00980887966165752\\
205.01	0.009808891798749\\
206.01	0.00980890420154039\\
207.01	0.00980891687596697\\
208.01	0.00980892982809992\\
209.01	0.00980894306414961\\
210.01	0.00980895659046883\\
211.01	0.00980897041355615\\
212.01	0.00980898454005947\\
213.01	0.00980899897677934\\
214.01	0.00980901373067284\\
215.01	0.00980902880885704\\
216.01	0.0098090442186129\\
217.01	0.00980905996738919\\
218.01	0.00980907606280643\\
219.01	0.00980909251266091\\
220.01	0.00980910932492897\\
221.01	0.00980912650777118\\
222.01	0.00980914406953678\\
223.01	0.00980916201876819\\
224.01	0.00980918036420551\\
225.01	0.00980919911479132\\
226.01	0.00980921827967547\\
227.01	0.00980923786822002\\
228.01	0.00980925789000434\\
229.01	0.00980927835483018\\
230.01	0.00980929927272718\\
231.01	0.0098093206539581\\
232.01	0.00980934250902452\\
233.01	0.00980936484867253\\
234.01	0.00980938768389859\\
235.01	0.00980941102595547\\
236.01	0.00980943488635843\\
237.01	0.00980945927689151\\
238.01	0.00980948420961396\\
239.01	0.00980950969686683\\
240.01	0.00980953575127974\\
241.01	0.00980956238577781\\
242.01	0.00980958961358871\\
243.01	0.00980961744825\\
244.01	0.00980964590361649\\
245.01	0.00980967499386792\\
246.01	0.0098097047335167\\
247.01	0.009809735137416\\
248.01	0.00980976622076781\\
249.01	0.00980979799913146\\
250.01	0.00980983048843212\\
251.01	0.00980986370496962\\
252.01	0.00980989766542744\\
253.01	0.00980993238688199\\
254.01	0.00980996788681199\\
255.01	0.00981000418310818\\
256.01	0.00981004129408321\\
257.01	0.00981007923848179\\
258.01	0.00981011803549108\\
259.01	0.00981015770475127\\
260.01	0.00981019826636658\\
261.01	0.00981023974091627\\
262.01	0.00981028214946614\\
263.01	0.00981032551358015\\
264.01	0.00981036985533241\\
265.01	0.00981041519731943\\
266.01	0.0098104615626726\\
267.01	0.00981050897507098\\
268.01	0.00981055745875454\\
269.01	0.00981060703853742\\
270.01	0.00981065773982179\\
271.01	0.00981070958861183\\
272.01	0.00981076261152814\\
273.01	0.00981081683582245\\
274.01	0.00981087228939261\\
275.01	0.00981092900079803\\
276.01	0.00981098699927543\\
277.01	0.00981104631475486\\
278.01	0.00981110697787623\\
279.01	0.00981116902000615\\
280.01	0.00981123247325503\\
281.01	0.00981129737049476\\
282.01	0.0098113637453767\\
283.01	0.00981143163234997\\
284.01	0.00981150106668032\\
285.01	0.00981157208446925\\
286.01	0.00981164472267366\\
287.01	0.00981171901912586\\
288.01	0.00981179501255404\\
289.01	0.00981187274260315\\
290.01	0.00981195224985628\\
291.01	0.00981203357585641\\
292.01	0.00981211676312868\\
293.01	0.0098122018552031\\
294.01	0.00981228889663777\\
295.01	0.00981237793304244\\
296.01	0.00981246901110275\\
297.01	0.0098125621786049\\
298.01	0.00981265748446061\\
299.01	0.00981275497873291\\
300.01	0.00981285471266224\\
301.01	0.00981295673869308\\
302.01	0.00981306111050112\\
303.01	0.00981316788302106\\
304.01	0.00981327711247472\\
305.01	0.00981338885639989\\
306.01	0.00981350317367973\\
307.01	0.00981362012457249\\
308.01	0.00981373977074208\\
309.01	0.00981386217528906\\
310.01	0.00981398740278219\\
311.01	0.0098141155192906\\
312.01	0.0098142465924166\\
313.01	0.00981438069132897\\
314.01	0.00981451788679696\\
315.01	0.00981465825122476\\
316.01	0.00981480185868685\\
317.01	0.00981494878496367\\
318.01	0.00981509910757821\\
319.01	0.00981525290583298\\
320.01	0.00981541026084789\\
321.01	0.00981557125559867\\
322.01	0.009815735974956\\
323.01	0.00981590450572534\\
324.01	0.00981607693668747\\
325.01	0.00981625335863986\\
326.01	0.0098164338644386\\
327.01	0.00981661854904127\\
328.01	0.00981680750955056\\
329.01	0.0098170008452586\\
330.01	0.00981719865769233\\
331.01	0.00981740105065954\\
332.01	0.00981760813029585\\
333.01	0.00981782000511266\\
334.01	0.00981803678604609\\
335.01	0.00981825858650674\\
336.01	0.00981848552243061\\
337.01	0.00981871771233105\\
338.01	0.0098189552773517\\
339.01	0.00981919834132075\\
340.01	0.00981944703080621\\
341.01	0.00981970147517243\\
342.01	0.00981996180663798\\
343.01	0.00982022816033466\\
344.01	0.00982050067436802\\
345.01	0.00982077948987909\\
346.01	0.00982106475110773\\
347.01	0.00982135660545728\\
348.01	0.00982165520356078\\
349.01	0.00982196069934876\\
350.01	0.00982227325011843\\
351.01	0.00982259301660471\\
352.01	0.00982292016305264\\
353.01	0.00982325485729153\\
354.01	0.00982359727081075\\
355.01	0.0098239475788372\\
356.01	0.00982430596041443\\
357.01	0.00982467259848359\\
358.01	0.00982504767996608\\
359.01	0.00982543139584827\\
360.01	0.00982582394126826\\
361.01	0.00982622551560493\\
362.01	0.00982663632257002\\
363.01	0.00982705657030352\\
364.01	0.00982748647147349\\
365.01	0.00982792624338166\\
366.01	0.00982837610807682\\
367.01	0.00982883629247861\\
368.01	0.00982930702851594\\
369.01	0.00982978855328536\\
370.01	0.00983028110923704\\
371.01	0.00983078494439915\\
372.01	0.009831300312655\\
373.01	0.00983182747409331\\
374.01	0.0098323666954582\\
375.01	0.0098329182507305\\
376.01	0.0098334824218243\\
377.01	0.00983405949880053\\
378.01	0.00983464977742447\\
379.01	0.00983525357750229\\
380.01	0.0098358712362743\\
381.01	0.00983650309111121\\
382.01	0.0098371494883478\\
383.01	0.00983781078344905\\
384.01	0.00983848734127284\\
385.01	0.00983917953634092\\
386.01	0.00983988775311827\\
387.01	0.00984061238630143\\
388.01	0.00984135384111574\\
389.01	0.00984211253362219\\
390.01	0.00984288889103382\\
391.01	0.0098436833520423\\
392.01	0.00984449636715484\\
393.01	0.00984532839904187\\
394.01	0.00984617992289603\\
395.01	0.00984705142680236\\
396.01	0.00984794341212078\\
397.01	0.00984885639388064\\
398.01	0.00984979090118814\\
399.01	0.0098507474776471\\
400.01	0.00985172668179315\\
401.01	0.00985272908754247\\
402.01	0.00985375528465476\\
403.01	0.00985480587921179\\
404.01	0.00985588149411155\\
405.01	0.00985698276957858\\
406.01	0.00985811036369141\\
407.01	0.00985926495292725\\
408.01	0.009860447232725\\
409.01	0.00986165791806676\\
410.01	0.00986289774407896\\
411.01	0.00986416746665347\\
412.01	0.00986546786308951\\
413.01	0.00986679973275707\\
414.01	0.00986816389778261\\
415.01	0.00986956120375771\\
416.01	0.00987099252047151\\
417.01	0.00987245874266765\\
418.01	0.00987396079082651\\
419.01	0.00987549961197355\\
420.01	0.0098770761805143\\
421.01	0.00987869149909706\\
422.01	0.00988034659950356\\
423.01	0.00988204254356856\\
424.01	0.00988378042412868\\
425.01	0.00988556136600094\\
426.01	0.00988738652699127\\
427.01	0.00988925709893294\\
428.01	0.00989117430875495\\
429.01	0.0098931394195795\\
430.01	0.00989515373184786\\
431.01	0.00989721858447311\\
432.01	0.00989933535601752\\
433.01	0.00990150546589172\\
434.01	0.00990373037557172\\
435.01	0.00990601158982828\\
436.01	0.00990835065796195\\
437.01	0.00991074917503498\\
438.01	0.0099132087830883\\
439.01	0.00991573117232946\\
440.01	0.00991831808227295\\
441.01	0.00992097130280954\\
442.01	0.00992369267517551\\
443.01	0.00992648409278455\\
444.01	0.00992934750187626\\
445.01	0.00993228490192267\\
446.01	0.00993529834571985\\
447.01	0.00993838993907268\\
448.01	0.00994156183995766\\
449.01	0.00994481625701917\\
450.01	0.00994815544721714\\
451.01	0.00995158171239833\\
452.01	0.00995509739450326\\
453.01	0.00995870486904767\\
454.01	0.0099624065364222\\
455.01	0.00996620481043573\\
456.01	0.00997010210337537\\
457.01	0.00997410080665274\\
458.01	0.00997820326570706\\
459.01	0.00998241174530182\\
460.01	0.00998672834523047\\
461.01	0.00999115415772442\\
462.01	0.00999567150184493\\
463.01	0.00999971224540899\\
464.01	0.01\\
465.01	0.01\\
466.01	0.01\\
467.01	0.01\\
468.01	0.01\\
469.01	0.01\\
470.01	0.01\\
471.01	0.01\\
472.01	0.01\\
473.01	0.01\\
474.01	0.01\\
475.01	0.01\\
476.01	0.01\\
477.01	0.01\\
478.01	0.01\\
479.01	0.01\\
480.01	0.01\\
481.01	0.01\\
482.01	0.01\\
483.01	0.01\\
484.01	0.01\\
485.01	0.01\\
486.01	0.01\\
487.01	0.01\\
488.01	0.01\\
489.01	0.01\\
490.01	0.01\\
491.01	0.01\\
492.01	0.01\\
493.01	0.01\\
494.01	0.01\\
495.01	0.01\\
496.01	0.01\\
497.01	0.01\\
498.01	0.01\\
499.01	0.01\\
500.01	0.01\\
501.01	0.01\\
502.01	0.01\\
503.01	0.01\\
504.01	0.01\\
505.01	0.01\\
506.01	0.01\\
507.01	0.01\\
508.01	0.01\\
509.01	0.01\\
510.01	0.01\\
511.01	0.01\\
512.01	0.01\\
513.01	0.01\\
514.01	0.01\\
515.01	0.01\\
516.01	0.01\\
517.01	0.01\\
518.01	0.01\\
519.01	0.01\\
520.01	0.01\\
521.01	0.01\\
522.01	0.01\\
523.01	0.01\\
524.01	0.01\\
525.01	0.01\\
526.01	0.01\\
527.01	0.01\\
528.01	0.01\\
529.01	0.01\\
530.01	0.01\\
531.01	0.01\\
532.01	0.01\\
533.01	0.01\\
534.01	0.01\\
535.01	0.01\\
536.01	0.01\\
537.01	0.01\\
538.01	0.01\\
539.01	0.01\\
540.01	0.01\\
541.01	0.01\\
542.01	0.01\\
543.01	0.01\\
544.01	0.01\\
545.01	0.01\\
546.01	0.01\\
547.01	0.01\\
548.01	0.01\\
549.01	0.01\\
550.01	0.01\\
551.01	0.01\\
552.01	0.01\\
553.01	0.01\\
554.01	0.01\\
555.01	0.01\\
556.01	0.01\\
557.01	0.01\\
558.01	0.01\\
559.01	0.01\\
560.01	0.01\\
561.01	0.01\\
562.01	0.01\\
563.01	0.01\\
564.01	0.01\\
565.01	0.01\\
566.01	0.01\\
567.01	0.01\\
568.01	0.01\\
569.01	0.01\\
570.01	0.01\\
571.01	0.01\\
572.01	0.01\\
573.01	0.01\\
574.01	0.01\\
575.01	0.01\\
576.01	0.01\\
577.01	0.01\\
578.01	0.01\\
579.01	0.01\\
580.01	0.01\\
581.01	0.01\\
582.01	0.01\\
583.01	0.01\\
584.01	0.01\\
585.01	0.01\\
586.01	0.01\\
587.01	0.01\\
588.01	0.01\\
589.01	0.01\\
590.01	0.01\\
591.01	0.01\\
592.01	0.01\\
593.01	0.01\\
594.01	0.01\\
595.01	0.01\\
596.01	0.01\\
597.01	0.01\\
598.01	0.01\\
599.01	0.01\\
599.02	0.01\\
599.03	0.01\\
599.04	0.01\\
599.05	0.01\\
599.06	0.01\\
599.07	0.01\\
599.08	0.01\\
599.09	0.01\\
599.1	0.01\\
599.11	0.01\\
599.12	0.01\\
599.13	0.01\\
599.14	0.01\\
599.15	0.01\\
599.16	0.01\\
599.17	0.01\\
599.18	0.01\\
599.19	0.01\\
599.2	0.01\\
599.21	0.01\\
599.22	0.01\\
599.23	0.01\\
599.24	0.01\\
599.25	0.01\\
599.26	0.01\\
599.27	0.01\\
599.28	0.01\\
599.29	0.01\\
599.3	0.01\\
599.31	0.01\\
599.32	0.01\\
599.33	0.01\\
599.34	0.01\\
599.35	0.01\\
599.36	0.01\\
599.37	0.01\\
599.38	0.01\\
599.39	0.01\\
599.4	0.01\\
599.41	0.01\\
599.42	0.01\\
599.43	0.01\\
599.44	0.01\\
599.45	0.01\\
599.46	0.01\\
599.47	0.01\\
599.48	0.01\\
599.49	0.01\\
599.5	0.01\\
599.51	0.01\\
599.52	0.01\\
599.53	0.01\\
599.54	0.01\\
599.55	0.01\\
599.56	0.01\\
599.57	0.01\\
599.58	0.01\\
599.59	0.01\\
599.6	0.01\\
599.61	0.01\\
599.62	0.01\\
599.63	0.01\\
599.64	0.01\\
599.65	0.01\\
599.66	0.01\\
599.67	0.01\\
599.68	0.01\\
599.69	0.01\\
599.7	0.01\\
599.71	0.01\\
599.72	0.01\\
599.73	0.01\\
599.74	0.01\\
599.75	0.01\\
599.76	0.01\\
599.77	0.01\\
599.78	0.01\\
599.79	0.01\\
599.8	0.01\\
599.81	0.01\\
599.82	0.01\\
599.83	0.01\\
599.84	0.01\\
599.85	0.01\\
599.86	0.01\\
599.87	0.01\\
599.88	0.01\\
599.89	0.01\\
599.9	0.01\\
599.91	0.01\\
599.92	0.01\\
599.93	0.01\\
599.94	0.01\\
599.95	0.01\\
599.96	0.01\\
599.97	0.01\\
599.98	0.01\\
599.99	0.01\\
600	0.01\\
};
\addplot [color=blue!80!mycolor9,solid,forget plot]
  table[row sep=crcr]{%
0.01	0.00921108894345117\\
1.01	0.00921108909757844\\
2.01	0.00921108925493703\\
3.01	0.00921108941559513\\
4.01	0.00921108957962236\\
5.01	0.0092110897470898\\
6.01	0.00921108991807012\\
7.01	0.00921109009263744\\
8.01	0.00921109027086754\\
9.01	0.00921109045283774\\
10.01	0.00921109063862709\\
11.01	0.00921109082831624\\
12.01	0.00921109102198763\\
13.01	0.00921109121972539\\
14.01	0.00921109142161552\\
15.01	0.00921109162774578\\
16.01	0.00921109183820586\\
17.01	0.00921109205308733\\
18.01	0.00921109227248374\\
19.01	0.0092110924964906\\
20.01	0.00921109272520549\\
21.01	0.00921109295872806\\
22.01	0.00921109319716011\\
23.01	0.0092110934406056\\
24.01	0.00921109368917069\\
25.01	0.00921109394296387\\
26.01	0.00921109420209587\\
27.01	0.00921109446667987\\
28.01	0.00921109473683143\\
29.01	0.00921109501266855\\
30.01	0.00921109529431183\\
31.01	0.00921109558188442\\
32.01	0.00921109587551208\\
33.01	0.00921109617532331\\
34.01	0.00921109648144934\\
35.01	0.00921109679402422\\
36.01	0.00921109711318487\\
37.01	0.00921109743907115\\
38.01	0.00921109777182594\\
39.01	0.00921109811159515\\
40.01	0.00921109845852783\\
41.01	0.00921109881277628\\
42.01	0.00921109917449598\\
43.01	0.00921109954384587\\
44.01	0.0092110999209882\\
45.01	0.00921110030608877\\
46.01	0.00921110069931692\\
47.01	0.00921110110084566\\
48.01	0.0092111015108517\\
49.01	0.00921110192951552\\
50.01	0.00921110235702158\\
51.01	0.00921110279355824\\
52.01	0.00921110323931794\\
53.01	0.00921110369449727\\
54.01	0.00921110415929704\\
55.01	0.00921110463392246\\
56.01	0.00921110511858308\\
57.01	0.009211105613493\\
58.01	0.00921110611887097\\
59.01	0.00921110663494045\\
60.01	0.00921110716192973\\
61.01	0.009211107700072\\
62.01	0.00921110824960557\\
63.01	0.00921110881077383\\
64.01	0.00921110938382545\\
65.01	0.00921110996901451\\
66.01	0.00921111056660058\\
67.01	0.00921111117684884\\
68.01	0.00921111180003021\\
69.01	0.0092111124364215\\
70.01	0.00921111308630552\\
71.01	0.00921111374997123\\
72.01	0.00921111442771379\\
73.01	0.00921111511983486\\
74.01	0.00921111582664257\\
75.01	0.00921111654845179\\
76.01	0.00921111728558423\\
77.01	0.00921111803836857\\
78.01	0.00921111880714062\\
79.01	0.00921111959224359\\
80.01	0.00921112039402802\\
81.01	0.00921112121285218\\
82.01	0.00921112204908213\\
83.01	0.00921112290309188\\
84.01	0.00921112377526357\\
85.01	0.00921112466598775\\
86.01	0.00921112557566342\\
87.01	0.00921112650469828\\
88.01	0.00921112745350899\\
89.01	0.00921112842252129\\
90.01	0.00921112941217015\\
91.01	0.00921113042290014\\
92.01	0.0092111314551655\\
93.01	0.00921113250943041\\
94.01	0.00921113358616923\\
95.01	0.00921113468586669\\
96.01	0.00921113580901812\\
97.01	0.0092111369561297\\
98.01	0.00921113812771875\\
99.01	0.0092111393243139\\
100.01	0.00921114054645536\\
101.01	0.00921114179469524\\
102.01	0.00921114306959774\\
103.01	0.00921114437173944\\
104.01	0.00921114570170964\\
105.01	0.00921114706011054\\
106.01	0.00921114844755762\\
107.01	0.0092111498646799\\
108.01	0.00921115131212024\\
109.01	0.00921115279053564\\
110.01	0.00921115430059762\\
111.01	0.00921115584299246\\
112.01	0.00921115741842157\\
113.01	0.00921115902760188\\
114.01	0.00921116067126613\\
115.01	0.0092111623501632\\
116.01	0.00921116406505858\\
117.01	0.00921116581673461\\
118.01	0.009211167605991\\
119.01	0.00921116943364509\\
120.01	0.00921117130053235\\
121.01	0.00921117320750677\\
122.01	0.0092111751554412\\
123.01	0.00921117714522787\\
124.01	0.00921117917777883\\
125.01	0.00921118125402629\\
126.01	0.00921118337492323\\
127.01	0.00921118554144374\\
128.01	0.00921118775458359\\
129.01	0.00921119001536067\\
130.01	0.00921119232481549\\
131.01	0.00921119468401175\\
132.01	0.0092111970940368\\
133.01	0.00921119955600224\\
134.01	0.00921120207104442\\
135.01	0.00921120464032502\\
136.01	0.00921120726503166\\
137.01	0.00921120994637844\\
138.01	0.00921121268560657\\
139.01	0.00921121548398506\\
140.01	0.00921121834281115\\
141.01	0.0092112212634112\\
142.01	0.00921122424714123\\
143.01	0.00921122729538759\\
144.01	0.00921123040956767\\
145.01	0.00921123359113067\\
146.01	0.00921123684155824\\
147.01	0.00921124016236528\\
148.01	0.00921124355510069\\
149.01	0.00921124702134817\\
150.01	0.00921125056272696\\
151.01	0.00921125418089275\\
152.01	0.0092112578775384\\
153.01	0.00921126165439493\\
154.01	0.00921126551323228\\
155.01	0.0092112694558603\\
156.01	0.00921127348412959\\
157.01	0.00921127759993251\\
158.01	0.00921128180520409\\
159.01	0.00921128610192305\\
160.01	0.0092112904921128\\
161.01	0.00921129497784244\\
162.01	0.00921129956122791\\
163.01	0.00921130424443293\\
164.01	0.00921130902967027\\
165.01	0.00921131391920274\\
166.01	0.00921131891534446\\
167.01	0.00921132402046197\\
168.01	0.00921132923697549\\
169.01	0.00921133456736017\\
170.01	0.00921134001414731\\
171.01	0.00921134557992575\\
172.01	0.00921135126734318\\
173.01	0.00921135707910742\\
174.01	0.00921136301798795\\
175.01	0.00921136908681724\\
176.01	0.00921137528849233\\
177.01	0.0092113816259762\\
178.01	0.00921138810229944\\
179.01	0.00921139472056167\\
180.01	0.00921140148393334\\
181.01	0.00921140839565719\\
182.01	0.00921141545905011\\
183.01	0.00921142267750477\\
184.01	0.00921143005449139\\
185.01	0.00921143759355959\\
186.01	0.00921144529834025\\
187.01	0.00921145317254738\\
188.01	0.00921146121998007\\
189.01	0.0092114694445245\\
190.01	0.00921147785015599\\
191.01	0.00921148644094104\\
192.01	0.00921149522103949\\
193.01	0.00921150419470673\\
194.01	0.00921151336629594\\
195.01	0.00921152274026031\\
196.01	0.00921153232115551\\
197.01	0.00921154211364198\\
198.01	0.00921155212248745\\
199.01	0.00921156235256943\\
200.01	0.00921157280887784\\
201.01	0.00921158349651757\\
202.01	0.00921159442071124\\
203.01	0.00921160558680197\\
204.01	0.00921161700025613\\
205.01	0.0092116286666664\\
206.01	0.00921164059175454\\
207.01	0.00921165278137459\\
208.01	0.00921166524151585\\
209.01	0.00921167797830618\\
210.01	0.00921169099801517\\
211.01	0.00921170430705754\\
212.01	0.00921171791199648\\
213.01	0.00921173181954719\\
214.01	0.00921174603658049\\
215.01	0.00921176057012637\\
216.01	0.00921177542737787\\
217.01	0.00921179061569482\\
218.01	0.00921180614260777\\
219.01	0.00921182201582208\\
220.01	0.00921183824322191\\
221.01	0.00921185483287457\\
222.01	0.00921187179303469\\
223.01	0.00921188913214872\\
224.01	0.00921190685885937\\
225.01	0.00921192498201029\\
226.01	0.00921194351065076\\
227.01	0.00921196245404046\\
228.01	0.00921198182165454\\
229.01	0.00921200162318855\\
230.01	0.00921202186856373\\
231.01	0.00921204256793219\\
232.01	0.00921206373168246\\
233.01	0.00921208537044485\\
234.01	0.00921210749509733\\
235.01	0.00921213011677117\\
236.01	0.00921215324685694\\
237.01	0.0092121768970106\\
238.01	0.00921220107915963\\
239.01	0.00921222580550948\\
240.01	0.00921225108855001\\
241.01	0.00921227694106214\\
242.01	0.00921230337612464\\
243.01	0.00921233040712109\\
244.01	0.00921235804774701\\
245.01	0.00921238631201711\\
246.01	0.00921241521427272\\
247.01	0.00921244476918937\\
248.01	0.00921247499178461\\
249.01	0.00921250589742597\\
250.01	0.00921253750183899\\
251.01	0.00921256982111563\\
252.01	0.00921260287172274\\
253.01	0.0092126366705107\\
254.01	0.00921267123472234\\
255.01	0.00921270658200198\\
256.01	0.00921274273040478\\
257.01	0.00921277969840615\\
258.01	0.0092128175049114\\
259.01	0.00921285616926582\\
260.01	0.00921289571126458\\
261.01	0.00921293615116327\\
262.01	0.00921297750968832\\
263.01	0.00921301980804791\\
264.01	0.00921306306794294\\
265.01	0.00921310731157842\\
266.01	0.0092131525616748\\
267.01	0.00921319884147994\\
268.01	0.00921324617478105\\
269.01	0.00921329458591702\\
270.01	0.00921334409979092\\
271.01	0.0092133947418829\\
272.01	0.00921344653826326\\
273.01	0.00921349951560584\\
274.01	0.00921355370120173\\
275.01	0.00921360912297319\\
276.01	0.00921366580948799\\
277.01	0.00921372378997394\\
278.01	0.00921378309433382\\
279.01	0.00921384375316053\\
280.01	0.00921390579775274\\
281.01	0.00921396926013065\\
282.01	0.00921403417305232\\
283.01	0.00921410057003013\\
284.01	0.00921416848534777\\
285.01	0.00921423795407755\\
286.01	0.00921430901209799\\
287.01	0.00921438169611191\\
288.01	0.00921445604366491\\
289.01	0.00921453209316414\\
290.01	0.00921460988389755\\
291.01	0.00921468945605368\\
292.01	0.00921477085074166\\
293.01	0.00921485411001177\\
294.01	0.00921493927687652\\
295.01	0.00921502639533209\\
296.01	0.00921511551038031\\
297.01	0.00921520666805107\\
298.01	0.00921529991542541\\
299.01	0.00921539530065896\\
300.01	0.00921549287300591\\
301.01	0.0092155926828438\\
302.01	0.00921569478169855\\
303.01	0.00921579922227029\\
304.01	0.00921590605845982\\
305.01	0.00921601534539558\\
306.01	0.00921612713946142\\
307.01	0.00921624149832495\\
308.01	0.00921635848096661\\
309.01	0.00921647814770963\\
310.01	0.00921660056025051\\
311.01	0.00921672578169044\\
312.01	0.00921685387656756\\
313.01	0.00921698491089001\\
314.01	0.0092171189521699\\
315.01	0.00921725606945824\\
316.01	0.00921739633338067\\
317.01	0.00921753981617449\\
318.01	0.00921768659172643\\
319.01	0.0092178367356117\\
320.01	0.00921799032513416\\
321.01	0.00921814743936751\\
322.01	0.00921830815919799\\
323.01	0.00921847256736807\\
324.01	0.00921864074852167\\
325.01	0.00921881278925081\\
326.01	0.00921898877814351\\
327.01	0.00921916880583342\\
328.01	0.00921935296505091\\
329.01	0.00921954135067588\\
330.01	0.00921973405979224\\
331.01	0.00921993119174421\\
332.01	0.00922013284819449\\
333.01	0.00922033913318436\\
334.01	0.00922055015319579\\
335.01	0.00922076601721564\\
336.01	0.00922098683680215\\
337.01	0.00922121272615347\\
338.01	0.00922144380217881\\
339.01	0.00922168018457177\\
340.01	0.00922192199588644\\
341.01	0.00922216936161587\\
342.01	0.00922242241027346\\
343.01	0.00922268127347702\\
344.01	0.00922294608603565\\
345.01	0.00922321698603976\\
346.01	0.00922349411495412\\
347.01	0.00922377761771381\\
348.01	0.00922406764282382\\
349.01	0.00922436434246158\\
350.01	0.00922466787258323\\
351.01	0.00922497839303309\\
352.01	0.00922529606765697\\
353.01	0.00922562106441894\\
354.01	0.009225953555522\\
355.01	0.00922629371753241\\
356.01	0.00922664173150812\\
357.01	0.00922699778313105\\
358.01	0.00922736206284342\\
359.01	0.00922773476598845\\
360.01	0.00922811609295501\\
361.01	0.00922850624932687\\
362.01	0.00922890544603633\\
363.01	0.00922931389952239\\
364.01	0.00922973183189374\\
365.01	0.00923015947109653\\
366.01	0.00923059705108736\\
367.01	0.00923104481201135\\
368.01	0.00923150300038603\\
369.01	0.00923197186929067\\
370.01	0.00923245167856213\\
371.01	0.00923294269499689\\
372.01	0.00923344519256014\\
373.01	0.00923395945260188\\
374.01	0.00923448576408023\\
375.01	0.00923502442378896\\
376.01	0.00923557573655237\\
377.01	0.00923614001494795\\
378.01	0.00923671757396008\\
379.01	0.00923730874275894\\
380.01	0.00923791388387595\\
381.01	0.00923853334486768\\
382.01	0.00923916748183644\\
383.01	0.00923981666015873\\
384.01	0.00924048125473469\\
385.01	0.00924116165024438\\
386.01	0.0092418582414112\\
387.01	0.00924257143327242\\
388.01	0.00924330164145737\\
389.01	0.00924404929247327\\
390.01	0.00924481482399892\\
391.01	0.00924559868518673\\
392.01	0.00924640133697308\\
393.01	0.00924722325239746\\
394.01	0.00924806491693053\\
395.01	0.00924892682881168\\
396.01	0.00924980949939585\\
397.01	0.00925071345351055\\
398.01	0.00925163922982314\\
399.01	0.00925258738121843\\
400.01	0.00925355847518767\\
401.01	0.00925455309422862\\
402.01	0.00925557183625763\\
403.01	0.00925661531503378\\
404.01	0.00925768416059588\\
405.01	0.00925877901971252\\
406.01	0.00925990055634604\\
407.01	0.00926104945213043\\
408.01	0.00926222640686443\\
409.01	0.00926343213901974\\
410.01	0.00926466738626559\\
411.01	0.0092659329060099\\
412.01	0.00926722947595804\\
413.01	0.00926855789468987\\
414.01	0.00926991898225584\\
415.01	0.00927131358079306\\
416.01	0.00927274255516225\\
417.01	0.00927420679360662\\
418.01	0.00927570720843356\\
419.01	0.00927724473672032\\
420.01	0.00927882034104493\\
421.01	0.00928043501024345\\
422.01	0.0092820897601948\\
423.01	0.00928378563463494\\
424.01	0.00928552370600124\\
425.01	0.00928730507630909\\
426.01	0.00928913087806211\\
427.01	0.00929100227519775\\
428.01	0.0092929204640701\\
429.01	0.00929488667447159\\
430.01	0.00929690217069581\\
431.01	0.00929896825264339\\
432.01	0.00930108625697287\\
433.01	0.00930325755829929\\
434.01	0.00930548357044225\\
435.01	0.00930776574772598\\
436.01	0.00931010558633428\\
437.01	0.00931250462572198\\
438.01	0.00931496445008634\\
439.01	0.00931748668990021\\
440.01	0.00932007302351014\\
441.01	0.00932272517880156\\
442.01	0.00932544493493375\\
443.01	0.0093282341241473\\
444.01	0.00933109463364612\\
445.01	0.00933402840755682\\
446.01	0.00933703744896772\\
447.01	0.0093401238220498\\
448.01	0.00934328965426202\\
449.01	0.00934653713864381\\
450.01	0.00934986853619746\\
451.01	0.00935328617836443\\
452.01	0.0093567924695999\\
453.01	0.00936038989005259\\
454.01	0.00936408099835907\\
455.01	0.00936786843456627\\
456.01	0.00937175492320142\\
457.01	0.00937574327651184\\
458.01	0.00937983639783687\\
459.01	0.00938403728414399\\
460.01	0.00938834901341874\\
461.01	0.0093927744991172\\
462.01	0.00939731261417764\\
463.01	0.00940191165965253\\
464.01	0.00940651584863829\\
465.01	0.00941124578153715\\
466.01	0.00941612535678167\\
467.01	0.00942107866392531\\
468.01	0.00942604576696724\\
469.01	0.00943115651871189\\
470.01	0.00943640782485806\\
471.01	0.00944179890261241\\
472.01	0.00944735264085425\\
473.01	0.00945310306573079\\
474.01	0.00945925966041799\\
475.01	0.00946483081489598\\
476.01	0.00947024052540716\\
477.01	0.0094757919546411\\
478.01	0.00948148927711075\\
479.01	0.00948733679760845\\
480.01	0.0094933389529326\\
481.01	0.00949950031282027\\
482.01	0.00950582557982904\\
483.01	0.00951231958784148\\
484.01	0.0095189872987783\\
485.01	0.00952583379698963\\
486.01	0.00953286428057719\\
487.01	0.00954008404782149\\
488.01	0.00954749846484649\\
489.01	0.00955511273202057\\
490.01	0.00956292983508929\\
491.01	0.00957095547030021\\
492.01	0.00957919070781336\\
493.01	0.00958765658063882\\
494.01	0.00959636509544663\\
495.01	0.00960532279064326\\
496.01	0.00961453590847657\\
497.01	0.00962401039666919\\
498.01	0.00963375403002067\\
499.01	0.00964377876930783\\
500.01	0.00965410040047059\\
501.01	0.00966473935232702\\
502.01	0.0096756552260708\\
503.01	0.00968682625103773\\
504.01	0.00969835078995916\\
505.01	0.00971024869980506\\
506.01	0.00972254102155007\\
507.01	0.00973525087114764\\
508.01	0.00974840373137549\\
509.01	0.00976202779420764\\
510.01	0.00977615436467055\\
511.01	0.00979081834669956\\
512.01	0.00980605889893121\\
513.01	0.00982192094670078\\
514.01	0.00983846360566679\\
515.01	0.00985581757267285\\
516.01	0.00987421662991688\\
517.01	0.0098922816907823\\
518.01	0.00991636879998524\\
519.01	0.00996005438860847\\
520.01	0.00999814685474962\\
521.01	0.01\\
522.01	0.01\\
523.01	0.01\\
524.01	0.01\\
525.01	0.01\\
526.01	0.01\\
527.01	0.01\\
528.01	0.01\\
529.01	0.01\\
530.01	0.01\\
531.01	0.01\\
532.01	0.01\\
533.01	0.01\\
534.01	0.01\\
535.01	0.01\\
536.01	0.01\\
537.01	0.01\\
538.01	0.01\\
539.01	0.01\\
540.01	0.01\\
541.01	0.01\\
542.01	0.01\\
543.01	0.01\\
544.01	0.01\\
545.01	0.01\\
546.01	0.01\\
547.01	0.01\\
548.01	0.01\\
549.01	0.01\\
550.01	0.01\\
551.01	0.01\\
552.01	0.01\\
553.01	0.01\\
554.01	0.01\\
555.01	0.01\\
556.01	0.01\\
557.01	0.01\\
558.01	0.01\\
559.01	0.01\\
560.01	0.01\\
561.01	0.01\\
562.01	0.01\\
563.01	0.01\\
564.01	0.01\\
565.01	0.01\\
566.01	0.01\\
567.01	0.01\\
568.01	0.01\\
569.01	0.01\\
570.01	0.01\\
571.01	0.01\\
572.01	0.01\\
573.01	0.01\\
574.01	0.01\\
575.01	0.01\\
576.01	0.01\\
577.01	0.01\\
578.01	0.01\\
579.01	0.01\\
580.01	0.01\\
581.01	0.01\\
582.01	0.01\\
583.01	0.01\\
584.01	0.01\\
585.01	0.01\\
586.01	0.01\\
587.01	0.01\\
588.01	0.01\\
589.01	0.01\\
590.01	0.01\\
591.01	0.01\\
592.01	0.01\\
593.01	0.01\\
594.01	0.01\\
595.01	0.01\\
596.01	0.01\\
597.01	0.01\\
598.01	0.01\\
599.01	0.01\\
599.02	0.01\\
599.03	0.01\\
599.04	0.01\\
599.05	0.01\\
599.06	0.01\\
599.07	0.01\\
599.08	0.01\\
599.09	0.01\\
599.1	0.01\\
599.11	0.01\\
599.12	0.01\\
599.13	0.01\\
599.14	0.01\\
599.15	0.01\\
599.16	0.01\\
599.17	0.01\\
599.18	0.01\\
599.19	0.01\\
599.2	0.01\\
599.21	0.01\\
599.22	0.01\\
599.23	0.01\\
599.24	0.01\\
599.25	0.01\\
599.26	0.01\\
599.27	0.01\\
599.28	0.01\\
599.29	0.01\\
599.3	0.01\\
599.31	0.01\\
599.32	0.01\\
599.33	0.01\\
599.34	0.01\\
599.35	0.01\\
599.36	0.01\\
599.37	0.01\\
599.38	0.01\\
599.39	0.01\\
599.4	0.01\\
599.41	0.01\\
599.42	0.01\\
599.43	0.01\\
599.44	0.01\\
599.45	0.01\\
599.46	0.01\\
599.47	0.01\\
599.48	0.01\\
599.49	0.01\\
599.5	0.01\\
599.51	0.01\\
599.52	0.01\\
599.53	0.01\\
599.54	0.01\\
599.55	0.01\\
599.56	0.01\\
599.57	0.01\\
599.58	0.01\\
599.59	0.01\\
599.6	0.01\\
599.61	0.01\\
599.62	0.01\\
599.63	0.01\\
599.64	0.01\\
599.65	0.01\\
599.66	0.01\\
599.67	0.01\\
599.68	0.01\\
599.69	0.01\\
599.7	0.01\\
599.71	0.01\\
599.72	0.01\\
599.73	0.01\\
599.74	0.01\\
599.75	0.01\\
599.76	0.01\\
599.77	0.01\\
599.78	0.01\\
599.79	0.01\\
599.8	0.01\\
599.81	0.01\\
599.82	0.01\\
599.83	0.01\\
599.84	0.01\\
599.85	0.01\\
599.86	0.01\\
599.87	0.01\\
599.88	0.01\\
599.89	0.01\\
599.9	0.01\\
599.91	0.01\\
599.92	0.01\\
599.93	0.01\\
599.94	0.01\\
599.95	0.01\\
599.96	0.01\\
599.97	0.01\\
599.98	0.01\\
599.99	0.01\\
600	0.01\\
};
\addplot [color=blue,solid,forget plot]
  table[row sep=crcr]{%
0.01	0.00851332874111991\\
1.01	0.00851332879671018\\
2.01	0.00851332885346802\\
3.01	0.00851332891141812\\
4.01	0.00851332897058573\\
5.01	0.00851332903099658\\
6.01	0.00851332909267698\\
7.01	0.0085133291556538\\
8.01	0.00851332921995449\\
9.01	0.00851332928560708\\
10.01	0.0085133293526402\\
11.01	0.00851332942108309\\
12.01	0.00851332949096559\\
13.01	0.00851332956231827\\
14.01	0.00851332963517223\\
15.01	0.00851332970955932\\
16.01	0.00851332978551205\\
17.01	0.00851332986306359\\
18.01	0.0085133299422479\\
19.01	0.00851333002309961\\
20.01	0.00851333010565408\\
21.01	0.00851333018994745\\
22.01	0.00851333027601668\\
23.01	0.00851333036389944\\
24.01	0.00851333045363426\\
25.01	0.00851333054526046\\
26.01	0.00851333063881827\\
27.01	0.00851333073434872\\
28.01	0.00851333083189374\\
29.01	0.00851333093149618\\
30.01	0.00851333103319978\\
31.01	0.00851333113704924\\
32.01	0.00851333124309023\\
33.01	0.00851333135136937\\
34.01	0.00851333146193432\\
35.01	0.00851333157483374\\
36.01	0.00851333169011734\\
37.01	0.00851333180783592\\
38.01	0.00851333192804136\\
39.01	0.00851333205078666\\
40.01	0.00851333217612595\\
41.01	0.00851333230411454\\
42.01	0.00851333243480898\\
43.01	0.00851333256826693\\
44.01	0.0085133327045474\\
45.01	0.00851333284371061\\
46.01	0.00851333298581812\\
47.01	0.00851333313093279\\
48.01	0.00851333327911885\\
49.01	0.00851333343044196\\
50.01	0.00851333358496911\\
51.01	0.00851333374276883\\
52.01	0.00851333390391108\\
53.01	0.00851333406846735\\
54.01	0.00851333423651067\\
55.01	0.00851333440811568\\
56.01	0.00851333458335859\\
57.01	0.00851333476231732\\
58.01	0.00851333494507145\\
59.01	0.00851333513170226\\
60.01	0.00851333532229285\\
61.01	0.00851333551692808\\
62.01	0.00851333571569468\\
63.01	0.00851333591868126\\
64.01	0.00851333612597834\\
65.01	0.00851333633767842\\
66.01	0.00851333655387602\\
67.01	0.0085133367746677\\
68.01	0.00851333700015216\\
69.01	0.00851333723043017\\
70.01	0.00851333746560478\\
71.01	0.00851333770578126\\
72.01	0.00851333795106715\\
73.01	0.00851333820157235\\
74.01	0.00851333845740918\\
75.01	0.00851333871869239\\
76.01	0.00851333898553923\\
77.01	0.00851333925806952\\
78.01	0.00851333953640572\\
79.01	0.00851333982067295\\
80.01	0.00851334011099902\\
81.01	0.00851334040751464\\
82.01	0.00851334071035329\\
83.01	0.00851334101965143\\
84.01	0.00851334133554846\\
85.01	0.00851334165818688\\
86.01	0.00851334198771229\\
87.01	0.0085133423242735\\
88.01	0.00851334266802257\\
89.01	0.0085133430191149\\
90.01	0.00851334337770934\\
91.01	0.00851334374396817\\
92.01	0.00851334411805727\\
93.01	0.00851334450014617\\
94.01	0.00851334489040814\\
95.01	0.00851334528902023\\
96.01	0.00851334569616342\\
97.01	0.0085133461120227\\
98.01	0.00851334653678704\\
99.01	0.00851334697064969\\
100.01	0.00851334741380811\\
101.01	0.00851334786646411\\
102.01	0.00851334832882398\\
103.01	0.00851334880109856\\
104.01	0.00851334928350334\\
105.01	0.00851334977625859\\
106.01	0.00851335027958945\\
107.01	0.00851335079372603\\
108.01	0.00851335131890355\\
109.01	0.00851335185536242\\
110.01	0.00851335240334842\\
111.01	0.00851335296311274\\
112.01	0.00851335353491215\\
113.01	0.00851335411900915\\
114.01	0.008513354715672\\
115.01	0.00851335532517501\\
116.01	0.0085133559477985\\
117.01	0.0085133565838291\\
118.01	0.00851335723355974\\
119.01	0.00851335789728996\\
120.01	0.00851335857532587\\
121.01	0.00851335926798048\\
122.01	0.00851335997557375\\
123.01	0.00851336069843277\\
124.01	0.00851336143689191\\
125.01	0.00851336219129305\\
126.01	0.00851336296198568\\
127.01	0.00851336374932708\\
128.01	0.00851336455368256\\
129.01	0.00851336537542558\\
130.01	0.00851336621493795\\
131.01	0.00851336707261001\\
132.01	0.0085133679488409\\
133.01	0.00851336884403867\\
134.01	0.00851336975862051\\
135.01	0.00851337069301295\\
136.01	0.0085133716476522\\
137.01	0.00851337262298412\\
138.01	0.00851337361946472\\
139.01	0.00851337463756013\\
140.01	0.00851337567774708\\
141.01	0.00851337674051297\\
142.01	0.00851337782635615\\
143.01	0.00851337893578624\\
144.01	0.00851338006932434\\
145.01	0.00851338122750325\\
146.01	0.00851338241086785\\
147.01	0.00851338361997531\\
148.01	0.00851338485539531\\
149.01	0.00851338611771049\\
150.01	0.00851338740751661\\
151.01	0.0085133887254229\\
152.01	0.00851339007205238\\
153.01	0.00851339144804219\\
154.01	0.00851339285404388\\
155.01	0.00851339429072378\\
156.01	0.00851339575876329\\
157.01	0.00851339725885929\\
158.01	0.00851339879172449\\
159.01	0.00851340035808774\\
160.01	0.00851340195869447\\
161.01	0.00851340359430704\\
162.01	0.00851340526570516\\
163.01	0.00851340697368625\\
164.01	0.00851340871906588\\
165.01	0.00851341050267821\\
166.01	0.0085134123253764\\
167.01	0.00851341418803301\\
168.01	0.00851341609154058\\
169.01	0.0085134180368119\\
170.01	0.00851342002478066\\
171.01	0.00851342205640187\\
172.01	0.00851342413265237\\
173.01	0.00851342625453122\\
174.01	0.00851342842306048\\
175.01	0.00851343063928546\\
176.01	0.00851343290427545\\
177.01	0.00851343521912422\\
178.01	0.00851343758495054\\
179.01	0.00851344000289891\\
180.01	0.00851344247413997\\
181.01	0.00851344499987128\\
182.01	0.00851344758131784\\
183.01	0.00851345021973277\\
184.01	0.00851345291639799\\
185.01	0.00851345567262482\\
186.01	0.00851345848975477\\
187.01	0.00851346136916016\\
188.01	0.00851346431224488\\
189.01	0.00851346732044513\\
190.01	0.00851347039523016\\
191.01	0.00851347353810305\\
192.01	0.00851347675060151\\
193.01	0.00851348003429867\\
194.01	0.00851348339080395\\
195.01	0.00851348682176383\\
196.01	0.00851349032886285\\
197.01	0.00851349391382435\\
198.01	0.00851349757841146\\
199.01	0.00851350132442806\\
200.01	0.00851350515371966\\
201.01	0.00851350906817441\\
202.01	0.00851351306972413\\
203.01	0.00851351716034526\\
204.01	0.00851352134205997\\
205.01	0.00851352561693717\\
206.01	0.00851352998709372\\
207.01	0.00851353445469537\\
208.01	0.00851353902195804\\
209.01	0.008513543691149\\
210.01	0.00851354846458801\\
211.01	0.00851355334464848\\
212.01	0.00851355833375897\\
213.01	0.00851356343440416\\
214.01	0.00851356864912643\\
215.01	0.00851357398052706\\
216.01	0.00851357943126762\\
217.01	0.00851358500407145\\
218.01	0.00851359070172502\\
219.01	0.00851359652707942\\
220.01	0.00851360248305194\\
221.01	0.0085136085726275\\
222.01	0.00851361479886034\\
223.01	0.00851362116487549\\
224.01	0.00851362767387058\\
225.01	0.00851363432911742\\
226.01	0.00851364113396375\\
227.01	0.00851364809183506\\
228.01	0.00851365520623631\\
229.01	0.00851366248075382\\
230.01	0.00851366991905715\\
231.01	0.00851367752490108\\
232.01	0.00851368530212747\\
233.01	0.0085136932546674\\
234.01	0.00851370138654315\\
235.01	0.00851370970187036\\
236.01	0.00851371820486018\\
237.01	0.00851372689982144\\
238.01	0.00851373579116297\\
239.01	0.00851374488339583\\
240.01	0.00851375418113573\\
241.01	0.00851376368910539\\
242.01	0.00851377341213708\\
243.01	0.00851378335517503\\
244.01	0.0085137935232781\\
245.01	0.00851380392162236\\
246.01	0.00851381455550379\\
247.01	0.00851382543034103\\
248.01	0.00851383655167819\\
249.01	0.00851384792518772\\
250.01	0.00851385955667332\\
251.01	0.00851387145207302\\
252.01	0.00851388361746212\\
253.01	0.00851389605905641\\
254.01	0.00851390878321534\\
255.01	0.00851392179644532\\
256.01	0.00851393510540302\\
257.01	0.00851394871689877\\
258.01	0.00851396263790011\\
259.01	0.00851397687553531\\
260.01	0.00851399143709704\\
261.01	0.00851400633004601\\
262.01	0.00851402156201491\\
263.01	0.0085140371408122\\
264.01	0.00851405307442606\\
265.01	0.00851406937102851\\
266.01	0.00851408603897953\\
267.01	0.00851410308683128\\
268.01	0.00851412052333247\\
269.01	0.00851413835743269\\
270.01	0.00851415659828704\\
271.01	0.00851417525526069\\
272.01	0.00851419433793356\\
273.01	0.00851421385610522\\
274.01	0.00851423381979979\\
275.01	0.00851425423927092\\
276.01	0.00851427512500704\\
277.01	0.00851429648773654\\
278.01	0.00851431833843317\\
279.01	0.00851434068832159\\
280.01	0.00851436354888293\\
281.01	0.00851438693186056\\
282.01	0.00851441084926599\\
283.01	0.00851443531338489\\
284.01	0.00851446033678318\\
285.01	0.00851448593231346\\
286.01	0.00851451211312125\\
287.01	0.00851453889265184\\
288.01	0.0085145662846568\\
289.01	0.00851459430320103\\
290.01	0.00851462296266978\\
291.01	0.00851465227777591\\
292.01	0.0085146822635673\\
293.01	0.0085147129354345\\
294.01	0.00851474430911842\\
295.01	0.0085147764007184\\
296.01	0.00851480922670039\\
297.01	0.0085148428039053\\
298.01	0.00851487714955763\\
299.01	0.00851491228127432\\
300.01	0.00851494821707382\\
301.01	0.00851498497538532\\
302.01	0.00851502257505842\\
303.01	0.00851506103537288\\
304.01	0.00851510037604871\\
305.01	0.00851514061725652\\
306.01	0.00851518177962822\\
307.01	0.00851522388426793\\
308.01	0.00851526695276327\\
309.01	0.00851531100719694\\
310.01	0.00851535607015871\\
311.01	0.00851540216475766\\
312.01	0.00851544931463482\\
313.01	0.00851549754397626\\
314.01	0.00851554687752648\\
315.01	0.00851559734060232\\
316.01	0.00851564895910721\\
317.01	0.00851570175954583\\
318.01	0.00851575576903948\\
319.01	0.00851581101534154\\
320.01	0.00851586752685382\\
321.01	0.00851592533264323\\
322.01	0.00851598446245895\\
323.01	0.00851604494675034\\
324.01	0.00851610681668532\\
325.01	0.00851617010416926\\
326.01	0.00851623484186484\\
327.01	0.00851630106321214\\
328.01	0.00851636880244967\\
329.01	0.00851643809463606\\
330.01	0.00851650897567243\\
331.01	0.00851658148232552\\
332.01	0.00851665565225163\\
333.01	0.00851673152402126\\
334.01	0.00851680913714471\\
335.01	0.00851688853209844\\
336.01	0.00851696975035224\\
337.01	0.00851705283439752\\
338.01	0.00851713782777622\\
339.01	0.00851722477511091\\
340.01	0.00851731372213571\\
341.01	0.00851740471572835\\
342.01	0.00851749780394305\\
343.01	0.00851759303604459\\
344.01	0.00851769046254337\\
345.01	0.0085177901352316\\
346.01	0.00851789210722057\\
347.01	0.00851799643297901\\
348.01	0.00851810316837259\\
349.01	0.00851821237070473\\
350.01	0.00851832409875833\\
351.01	0.00851843841283893\\
352.01	0.00851855537481896\\
353.01	0.0085186750481833\\
354.01	0.00851879749807593\\
355.01	0.00851892279134808\\
356.01	0.00851905099660741\\
357.01	0.0085191821842687\\
358.01	0.00851931642660561\\
359.01	0.00851945379780399\\
360.01	0.00851959437401624\\
361.01	0.00851973823341729\\
362.01	0.00851988545626163\\
363.01	0.00852003612494187\\
364.01	0.00852019032404855\\
365.01	0.00852034814043119\\
366.01	0.00852050966326084\\
367.01	0.00852067498409377\\
368.01	0.00852084419693646\\
369.01	0.00852101739831183\\
370.01	0.0085211946873268\\
371.01	0.00852137616574072\\
372.01	0.0085215619380352\\
373.01	0.00852175211148479\\
374.01	0.00852194679622858\\
375.01	0.00852214610534118\\
376.01	0.00852235015489174\\
377.01	0.0085225590638877\\
378.01	0.00852277295377686\\
379.01	0.0085229919508181\\
380.01	0.00852321618480231\\
381.01	0.00852344578722066\\
382.01	0.00852368089303297\\
383.01	0.00852392164077436\\
384.01	0.00852416817265735\\
385.01	0.00852442063467709\\
386.01	0.00852467917672049\\
387.01	0.00852494395267876\\
388.01	0.00852521512056409\\
389.01	0.00852549284263029\\
390.01	0.00852577728549788\\
391.01	0.00852606862028348\\
392.01	0.00852636702273421\\
393.01	0.00852667267336677\\
394.01	0.00852698575761212\\
395.01	0.00852730646596539\\
396.01	0.00852763499414178\\
397.01	0.00852797154323861\\
398.01	0.0085283163199038\\
399.01	0.00852866953651131\\
400.01	0.00852903141134382\\
401.01	0.008529402168783\\
402.01	0.00852978203950815\\
403.01	0.00853017126070327\\
404.01	0.00853057007627347\\
405.01	0.00853097873707103\\
406.01	0.0085313975011319\\
407.01	0.00853182663392318\\
408.01	0.00853226640860232\\
409.01	0.00853271710628899\\
410.01	0.00853317901635003\\
411.01	0.00853365243669897\\
412.01	0.00853413767411058\\
413.01	0.00853463504455167\\
414.01	0.00853514487352953\\
415.01	0.00853566749645883\\
416.01	0.00853620325904867\\
417.01	0.00853675251771109\\
418.01	0.00853731563999265\\
419.01	0.00853789300503082\\
420.01	0.00853848500403716\\
421.01	0.00853909204080928\\
422.01	0.00853971453227379\\
423.01	0.00854035290906294\\
424.01	0.00854100761612752\\
425.01	0.008541679113389\\
426.01	0.00854236787643436\\
427.01	0.00854307439725726\\
428.01	0.00854379918504937\\
429.01	0.00854454276704671\\
430.01	0.00854530568943574\\
431.01	0.0085460885183248\\
432.01	0.00854689184078718\\
433.01	0.00854771626598274\\
434.01	0.00854856242636574\\
435.01	0.00854943097898797\\
436.01	0.00855032260690665\\
437.01	0.00855123802070853\\
438.01	0.00855217796016293\\
439.01	0.00855314319601784\\
440.01	0.00855413453195583\\
441.01	0.00855515280672833\\
442.01	0.00855619889648994\\
443.01	0.00855727371735749\\
444.01	0.00855837822822269\\
445.01	0.00855951343385113\\
446.01	0.00856068038830647\\
447.01	0.00856188019874435\\
448.01	0.00856311402962869\\
449.01	0.00856438310743165\\
450.01	0.00856568872588968\\
451.01	0.00856703225190094\\
452.01	0.00856841513216562\\
453.01	0.00856983890068926\\
454.01	0.00857130518729276\\
455.01	0.00857281572730137\\
456.01	0.00857437237261919\\
457.01	0.00857597710443567\\
458.01	0.00857763204783812\\
459.01	0.0085793394883786\\
460.01	0.00858110188738293\\
461.01	0.00858292185510583\\
462.01	0.00858480169043087\\
463.01	0.00858674128328256\\
464.01	0.00858874488580694\\
465.01	0.00859082224758488\\
466.01	0.00859299916213625\\
467.01	0.00859526614543668\\
468.01	0.00859733328976934\\
469.01	0.00859948299348634\\
470.01	0.00860171592593705\\
471.01	0.00860401063681875\\
472.01	0.00860640443248413\\
473.01	0.00860892106162595\\
474.01	0.00861180202011184\\
475.01	0.00861681143869944\\
476.01	0.00862234308761062\\
477.01	0.00862801603513246\\
478.01	0.00863383427611849\\
479.01	0.00863980194144616\\
480.01	0.00864592330437171\\
481.01	0.00865220278732148\\
482.01	0.00865864496916522\\
483.01	0.00866525459302367\\
484.01	0.00867203657467244\\
485.01	0.00867899601161019\\
486.01	0.00868613819283572\\
487.01	0.00869346860901762\\
488.01	0.00870099295893498\\
489.01	0.00870871711602004\\
490.01	0.00871664701723107\\
491.01	0.00872478910241052\\
492.01	0.00873314989142529\\
493.01	0.00874173766525284\\
494.01	0.0087505598241744\\
495.01	0.00875962387666015\\
496.01	0.00876893767204288\\
497.01	0.0087785094618282\\
498.01	0.00878834808788105\\
499.01	0.00879846332558306\\
500.01	0.00880886789421755\\
501.01	0.00881958508416636\\
502.01	0.0088305930665103\\
503.01	0.00884174477446774\\
504.01	0.00885321989685707\\
505.01	0.00886503719565277\\
506.01	0.00887721048991623\\
507.01	0.00888975442974024\\
508.01	0.00890268457076784\\
509.01	0.00891601745821034\\
510.01	0.00892977072267643\\
511.01	0.00894396319539225\\
512.01	0.00895861509225591\\
513.01	0.00897374869033576\\
514.01	0.00898939349871925\\
515.01	0.00900563383206323\\
516.01	0.00902291526562311\\
517.01	0.00903930867369387\\
518.01	0.00905529963267023\\
519.01	0.00907220801133853\\
520.01	0.00908897671067582\\
521.01	0.00910522407637488\\
522.01	0.00912157679396798\\
523.01	0.00913814928186252\\
524.01	0.00915529289708311\\
525.01	0.00917304475334421\\
526.01	0.00919144616047542\\
527.01	0.00921054326624265\\
528.01	0.00923038762439255\\
529.01	0.00925103644318186\\
530.01	0.00927255873337311\\
531.01	0.00929503074496483\\
532.01	0.00931854097522513\\
533.01	0.00934323123125701\\
534.01	0.009370135044619\\
535.01	0.00941740712267001\\
536.01	0.00948007000197791\\
537.01	0.00954397864109383\\
538.01	0.00960916764714868\\
539.01	0.00967580163460858\\
540.01	0.00974417307012064\\
541.01	0.0098134628691532\\
542.01	0.00988194439853754\\
543.01	0.00995110711589977\\
544.01	0.01\\
545.01	0.01\\
546.01	0.01\\
547.01	0.01\\
548.01	0.01\\
549.01	0.01\\
550.01	0.01\\
551.01	0.01\\
552.01	0.01\\
553.01	0.01\\
554.01	0.01\\
555.01	0.01\\
556.01	0.01\\
557.01	0.01\\
558.01	0.01\\
559.01	0.01\\
560.01	0.01\\
561.01	0.01\\
562.01	0.01\\
563.01	0.01\\
564.01	0.01\\
565.01	0.01\\
566.01	0.01\\
567.01	0.01\\
568.01	0.01\\
569.01	0.01\\
570.01	0.01\\
571.01	0.01\\
572.01	0.01\\
573.01	0.01\\
574.01	0.01\\
575.01	0.01\\
576.01	0.01\\
577.01	0.01\\
578.01	0.01\\
579.01	0.01\\
580.01	0.01\\
581.01	0.01\\
582.01	0.01\\
583.01	0.01\\
584.01	0.01\\
585.01	0.01\\
586.01	0.01\\
587.01	0.01\\
588.01	0.01\\
589.01	0.01\\
590.01	0.01\\
591.01	0.01\\
592.01	0.01\\
593.01	0.01\\
594.01	0.01\\
595.01	0.01\\
596.01	0.01\\
597.01	0.01\\
598.01	0.01\\
599.01	0.01\\
599.02	0.01\\
599.03	0.01\\
599.04	0.01\\
599.05	0.01\\
599.06	0.01\\
599.07	0.01\\
599.08	0.01\\
599.09	0.01\\
599.1	0.01\\
599.11	0.01\\
599.12	0.01\\
599.13	0.01\\
599.14	0.01\\
599.15	0.01\\
599.16	0.01\\
599.17	0.01\\
599.18	0.01\\
599.19	0.01\\
599.2	0.01\\
599.21	0.01\\
599.22	0.01\\
599.23	0.01\\
599.24	0.01\\
599.25	0.01\\
599.26	0.01\\
599.27	0.01\\
599.28	0.01\\
599.29	0.01\\
599.3	0.01\\
599.31	0.01\\
599.32	0.01\\
599.33	0.01\\
599.34	0.01\\
599.35	0.01\\
599.36	0.01\\
599.37	0.01\\
599.38	0.01\\
599.39	0.01\\
599.4	0.01\\
599.41	0.01\\
599.42	0.01\\
599.43	0.01\\
599.44	0.01\\
599.45	0.01\\
599.46	0.01\\
599.47	0.01\\
599.48	0.01\\
599.49	0.01\\
599.5	0.01\\
599.51	0.01\\
599.52	0.01\\
599.53	0.01\\
599.54	0.01\\
599.55	0.01\\
599.56	0.01\\
599.57	0.01\\
599.58	0.01\\
599.59	0.01\\
599.6	0.01\\
599.61	0.01\\
599.62	0.01\\
599.63	0.01\\
599.64	0.01\\
599.65	0.01\\
599.66	0.01\\
599.67	0.01\\
599.68	0.01\\
599.69	0.01\\
599.7	0.01\\
599.71	0.01\\
599.72	0.01\\
599.73	0.01\\
599.74	0.01\\
599.75	0.01\\
599.76	0.01\\
599.77	0.01\\
599.78	0.01\\
599.79	0.01\\
599.8	0.01\\
599.81	0.01\\
599.82	0.01\\
599.83	0.01\\
599.84	0.01\\
599.85	0.01\\
599.86	0.01\\
599.87	0.01\\
599.88	0.01\\
599.89	0.01\\
599.9	0.01\\
599.91	0.01\\
599.92	0.01\\
599.93	0.01\\
599.94	0.01\\
599.95	0.01\\
599.96	0.01\\
599.97	0.01\\
599.98	0.01\\
599.99	0.01\\
600	0.01\\
};
\addplot [color=mycolor10,solid,forget plot]
  table[row sep=crcr]{%
0.01	0.00759711428563548\\
1.01	0.00759711430341606\\
2.01	0.00759711432157064\\
3.01	0.00759711434010715\\
4.01	0.00759711435903364\\
5.01	0.00759711437835844\\
6.01	0.00759711439808997\\
7.01	0.00759711441823685\\
8.01	0.00759711443880789\\
9.01	0.00759711445981209\\
10.01	0.00759711448125864\\
11.01	0.00759711450315694\\
12.01	0.00759711452551657\\
13.01	0.00759711454834734\\
14.01	0.00759711457165924\\
15.01	0.0075971145954625\\
16.01	0.00759711461976756\\
17.01	0.00759711464458503\\
18.01	0.00759711466992591\\
19.01	0.00759711469580127\\
20.01	0.00759711472222247\\
21.01	0.00759711474920114\\
22.01	0.00759711477674912\\
23.01	0.00759711480487854\\
24.01	0.00759711483360177\\
25.01	0.00759711486293148\\
26.01	0.00759711489288056\\
27.01	0.00759711492346218\\
28.01	0.00759711495468987\\
29.01	0.00759711498657738\\
30.01	0.00759711501913875\\
31.01	0.00759711505238838\\
32.01	0.00759711508634094\\
33.01	0.00759711512101142\\
34.01	0.00759711515641514\\
35.01	0.00759711519256775\\
36.01	0.00759711522948524\\
37.01	0.00759711526718395\\
38.01	0.00759711530568055\\
39.01	0.00759711534499211\\
40.01	0.00759711538513603\\
41.01	0.00759711542613009\\
42.01	0.00759711546799251\\
43.01	0.00759711551074182\\
44.01	0.00759711555439702\\
45.01	0.00759711559897748\\
46.01	0.00759711564450302\\
47.01	0.00759711569099386\\
48.01	0.0075971157384707\\
49.01	0.00759711578695462\\
50.01	0.00759711583646721\\
51.01	0.00759711588703054\\
52.01	0.00759711593866713\\
53.01	0.00759711599140001\\
54.01	0.00759711604525268\\
55.01	0.00759711610024916\\
56.01	0.00759711615641406\\
57.01	0.00759711621377239\\
58.01	0.00759711627234983\\
59.01	0.00759711633217256\\
60.01	0.00759711639326734\\
61.01	0.00759711645566151\\
62.01	0.00759711651938302\\
63.01	0.00759711658446041\\
64.01	0.00759711665092289\\
65.01	0.00759711671880021\\
66.01	0.00759711678812288\\
67.01	0.00759711685892201\\
68.01	0.0075971169312294\\
69.01	0.00759711700507756\\
70.01	0.00759711708049972\\
71.01	0.00759711715752978\\
72.01	0.00759711723620247\\
73.01	0.00759711731655321\\
74.01	0.00759711739861821\\
75.01	0.0075971174824345\\
76.01	0.00759711756803991\\
77.01	0.00759711765547308\\
78.01	0.00759711774477351\\
79.01	0.00759711783598156\\
80.01	0.00759711792913853\\
81.01	0.00759711802428656\\
82.01	0.0075971181214687\\
83.01	0.00759711822072907\\
84.01	0.00759711832211264\\
85.01	0.00759711842566537\\
86.01	0.0075971185314343\\
87.01	0.00759711863946749\\
88.01	0.00759711874981403\\
89.01	0.0075971188625241\\
90.01	0.007597118977649\\
91.01	0.00759711909524118\\
92.01	0.00759711921535421\\
93.01	0.00759711933804283\\
94.01	0.00759711946336305\\
95.01	0.00759711959137209\\
96.01	0.0075971197221284\\
97.01	0.00759711985569178\\
98.01	0.00759711999212334\\
99.01	0.00759712013148551\\
100.01	0.00759712027384211\\
101.01	0.00759712041925845\\
102.01	0.00759712056780117\\
103.01	0.00759712071953847\\
104.01	0.00759712087454004\\
105.01	0.00759712103287714\\
106.01	0.00759712119462256\\
107.01	0.00759712135985078\\
108.01	0.00759712152863786\\
109.01	0.00759712170106162\\
110.01	0.00759712187720156\\
111.01	0.007597122057139\\
112.01	0.00759712224095703\\
113.01	0.00759712242874062\\
114.01	0.00759712262057662\\
115.01	0.0075971228165538\\
116.01	0.00759712301676297\\
117.01	0.00759712322129691\\
118.01	0.00759712343025052\\
119.01	0.00759712364372077\\
120.01	0.00759712386180685\\
121.01	0.00759712408461014\\
122.01	0.00759712431223429\\
123.01	0.00759712454478528\\
124.01	0.00759712478237151\\
125.01	0.0075971250251037\\
126.01	0.00759712527309516\\
127.01	0.00759712552646171\\
128.01	0.00759712578532174\\
129.01	0.00759712604979632\\
130.01	0.00759712632000929\\
131.01	0.00759712659608719\\
132.01	0.00759712687815947\\
133.01	0.00759712716635846\\
134.01	0.00759712746081951\\
135.01	0.00759712776168096\\
136.01	0.00759712806908432\\
137.01	0.00759712838317429\\
138.01	0.0075971287040988\\
139.01	0.00759712903200914\\
140.01	0.00759712936706004\\
141.01	0.00759712970940968\\
142.01	0.00759713005921985\\
143.01	0.00759713041665599\\
144.01	0.00759713078188731\\
145.01	0.00759713115508681\\
146.01	0.0075971315364314\\
147.01	0.00759713192610207\\
148.01	0.00759713232428385\\
149.01	0.007597132731166\\
150.01	0.00759713314694205\\
151.01	0.00759713357180995\\
152.01	0.00759713400597214\\
153.01	0.00759713444963565\\
154.01	0.00759713490301221\\
155.01	0.00759713536631841\\
156.01	0.00759713583977574\\
157.01	0.00759713632361072\\
158.01	0.00759713681805505\\
159.01	0.00759713732334571\\
160.01	0.0075971378397251\\
161.01	0.00759713836744111\\
162.01	0.0075971389067473\\
163.01	0.00759713945790305\\
164.01	0.00759714002117363\\
165.01	0.00759714059683038\\
166.01	0.00759714118515087\\
167.01	0.00759714178641895\\
168.01	0.00759714240092506\\
169.01	0.00759714302896624\\
170.01	0.0075971436708463\\
171.01	0.00759714432687604\\
172.01	0.00759714499737342\\
173.01	0.00759714568266363\\
174.01	0.00759714638307934\\
175.01	0.00759714709896089\\
176.01	0.00759714783065634\\
177.01	0.00759714857852186\\
178.01	0.00759714934292172\\
179.01	0.00759715012422858\\
180.01	0.00759715092282368\\
181.01	0.00759715173909701\\
182.01	0.00759715257344754\\
183.01	0.00759715342628343\\
184.01	0.00759715429802221\\
185.01	0.00759715518909107\\
186.01	0.00759715609992698\\
187.01	0.00759715703097702\\
188.01	0.00759715798269857\\
189.01	0.00759715895555955\\
190.01	0.00759715995003865\\
191.01	0.00759716096662565\\
192.01	0.0075971620058216\\
193.01	0.00759716306813912\\
194.01	0.00759716415410263\\
195.01	0.00759716526424875\\
196.01	0.0075971663991264\\
197.01	0.00759716755929726\\
198.01	0.00759716874533591\\
199.01	0.00759716995783028\\
200.01	0.00759717119738185\\
201.01	0.00759717246460604\\
202.01	0.00759717376013248\\
203.01	0.00759717508460537\\
204.01	0.00759717643868381\\
205.01	0.00759717782304216\\
206.01	0.00759717923837037\\
207.01	0.00759718068537439\\
208.01	0.00759718216477649\\
209.01	0.00759718367731565\\
210.01	0.00759718522374796\\
211.01	0.00759718680484706\\
212.01	0.00759718842140444\\
213.01	0.00759719007422999\\
214.01	0.00759719176415231\\
215.01	0.00759719349201915\\
216.01	0.00759719525869804\\
217.01	0.00759719706507643\\
218.01	0.00759719891206241\\
219.01	0.00759720080058515\\
220.01	0.00759720273159523\\
221.01	0.00759720470606533\\
222.01	0.00759720672499062\\
223.01	0.00759720878938931\\
224.01	0.00759721090030323\\
225.01	0.0075972130587983\\
226.01	0.00759721526596511\\
227.01	0.00759721752291954\\
228.01	0.00759721983080323\\
229.01	0.00759722219078431\\
230.01	0.0075972246040579\\
231.01	0.00759722707184678\\
232.01	0.00759722959540205\\
233.01	0.00759723217600367\\
234.01	0.00759723481496128\\
235.01	0.00759723751361473\\
236.01	0.0075972402733349\\
237.01	0.00759724309552432\\
238.01	0.0075972459816179\\
239.01	0.00759724893308371\\
240.01	0.00759725195142374\\
241.01	0.00759725503817462\\
242.01	0.00759725819490847\\
243.01	0.00759726142323365\\
244.01	0.00759726472479561\\
245.01	0.00759726810127777\\
246.01	0.00759727155440234\\
247.01	0.00759727508593118\\
248.01	0.00759727869766676\\
249.01	0.00759728239145301\\
250.01	0.00759728616917633\\
251.01	0.00759729003276654\\
252.01	0.00759729398419776\\
253.01	0.00759729802548958\\
254.01	0.00759730215870792\\
255.01	0.00759730638596622\\
256.01	0.00759731070942638\\
257.01	0.00759731513129995\\
258.01	0.00759731965384923\\
259.01	0.00759732427938837\\
260.01	0.00759732901028455\\
261.01	0.00759733384895924\\
262.01	0.00759733879788932\\
263.01	0.00759734385960838\\
264.01	0.00759734903670801\\
265.01	0.00759735433183906\\
266.01	0.00759735974771301\\
267.01	0.0075973652871033\\
268.01	0.00759737095284672\\
269.01	0.00759737674784489\\
270.01	0.00759738267506566\\
271.01	0.00759738873754458\\
272.01	0.00759739493838654\\
273.01	0.00759740128076716\\
274.01	0.00759740776793452\\
275.01	0.00759741440321073\\
276.01	0.00759742118999361\\
277.01	0.00759742813175841\\
278.01	0.00759743523205952\\
279.01	0.00759744249453233\\
280.01	0.00759744992289498\\
281.01	0.00759745752095033\\
282.01	0.00759746529258779\\
283.01	0.00759747324178531\\
284.01	0.0075974813726115\\
285.01	0.00759748968922757\\
286.01	0.00759749819588948\\
287.01	0.00759750689695017\\
288.01	0.00759751579686174\\
289.01	0.00759752490017776\\
290.01	0.00759753421155559\\
291.01	0.0075975437357588\\
292.01	0.00759755347765966\\
293.01	0.00759756344224168\\
294.01	0.00759757363460212\\
295.01	0.00759758405995484\\
296.01	0.00759759472363292\\
297.01	0.00759760563109147\\
298.01	0.00759761678791065\\
299.01	0.00759762819979853\\
300.01	0.0075976398725943\\
301.01	0.00759765181227124\\
302.01	0.00759766402494021\\
303.01	0.00759767651685277\\
304.01	0.00759768929440477\\
305.01	0.00759770236413984\\
306.01	0.00759771573275306\\
307.01	0.00759772940709473\\
308.01	0.00759774339417425\\
309.01	0.00759775770116406\\
310.01	0.00759777233540389\\
311.01	0.00759778730440486\\
312.01	0.00759780261585395\\
313.01	0.00759781827761858\\
314.01	0.00759783429775108\\
315.01	0.00759785068449371\\
316.01	0.00759786744628348\\
317.01	0.00759788459175736\\
318.01	0.00759790212975755\\
319.01	0.00759792006933688\\
320.01	0.0075979384197645\\
321.01	0.00759795719053168\\
322.01	0.00759797639135782\\
323.01	0.00759799603219666\\
324.01	0.0075980161232426\\
325.01	0.00759803667493738\\
326.01	0.00759805769797686\\
327.01	0.00759807920331804\\
328.01	0.00759810120218639\\
329.01	0.00759812370608319\\
330.01	0.00759814672679338\\
331.01	0.0075981702763935\\
332.01	0.00759819436725983\\
333.01	0.00759821901207697\\
334.01	0.00759824422384649\\
335.01	0.00759827001589595\\
336.01	0.00759829640188821\\
337.01	0.00759832339583089\\
338.01	0.00759835101208636\\
339.01	0.0075983792653817\\
340.01	0.00759840817081923\\
341.01	0.00759843774388726\\
342.01	0.00759846800047101\\
343.01	0.0075984989568641\\
344.01	0.0075985306297802\\
345.01	0.00759856303636495\\
346.01	0.00759859619420834\\
347.01	0.00759863012135745\\
348.01	0.00759866483632943\\
349.01	0.00759870035812481\\
350.01	0.00759873670624131\\
351.01	0.00759877390068791\\
352.01	0.0075988119619993\\
353.01	0.00759885091125074\\
354.01	0.00759889077007328\\
355.01	0.00759893156066941\\
356.01	0.00759897330582899\\
357.01	0.00759901602894575\\
358.01	0.00759905975403411\\
359.01	0.00759910450574636\\
360.01	0.00759915030939039\\
361.01	0.00759919719094773\\
362.01	0.00759924517709206\\
363.01	0.00759929429520817\\
364.01	0.00759934457341124\\
365.01	0.00759939604056666\\
366.01	0.00759944872631018\\
367.01	0.00759950266106848\\
368.01	0.00759955787608007\\
369.01	0.00759961440341674\\
370.01	0.00759967227600502\\
371.01	0.00759973152764839\\
372.01	0.00759979219304943\\
373.01	0.00759985430783259\\
374.01	0.00759991790856695\\
375.01	0.00759998303278911\\
376.01	0.0076000497190231\\
377.01	0.00760011800678446\\
378.01	0.00760018793659381\\
379.01	0.00760025955021629\\
380.01	0.00760033289043611\\
381.01	0.0076004080011099\\
382.01	0.00760048492729109\\
383.01	0.00760056371526734\\
384.01	0.00760064441259949\\
385.01	0.00760072706816217\\
386.01	0.00760081173218604\\
387.01	0.00760089845630183\\
388.01	0.00760098729358639\\
389.01	0.00760107829861057\\
390.01	0.0076011715274894\\
391.01	0.00760126703793447\\
392.01	0.00760136488930877\\
393.01	0.00760146514268394\\
394.01	0.00760156786090044\\
395.01	0.00760167310863041\\
396.01	0.00760178095244376\\
397.01	0.00760189146087739\\
398.01	0.00760200470450799\\
399.01	0.00760212075602848\\
400.01	0.00760223969032841\\
401.01	0.00760236158457857\\
402.01	0.00760248651832017\\
403.01	0.00760261457355861\\
404.01	0.00760274583486264\\
405.01	0.0076028803894688\\
406.01	0.00760301832739173\\
407.01	0.00760315974154093\\
408.01	0.00760330472784394\\
409.01	0.00760345338537697\\
410.01	0.0076036058165031\\
411.01	0.00760376212701876\\
412.01	0.00760392242630911\\
413.01	0.007604086827513\\
414.01	0.00760425544769812\\
415.01	0.00760442840804718\\
416.01	0.00760460583405607\\
417.01	0.00760478785574485\\
418.01	0.00760497460788239\\
419.01	0.0076051662302263\\
420.01	0.00760536286777884\\
421.01	0.00760556467106035\\
422.01	0.00760577179640201\\
423.01	0.00760598440625889\\
424.01	0.00760620266954593\\
425.01	0.0076064267619982\\
426.01	0.00760665686655819\\
427.01	0.0076068931737924\\
428.01	0.00760713588234016\\
429.01	0.00760738519939784\\
430.01	0.00760764134124227\\
431.01	0.00760790453379695\\
432.01	0.00760817501324634\\
433.01	0.00760845302670286\\
434.01	0.00760873883293303\\
435.01	0.00760903270314941\\
436.01	0.00760933492187631\\
437.01	0.00760964578789821\\
438.01	0.00760996561530154\\
439.01	0.00761029473462189\\
440.01	0.00761063349411057\\
441.01	0.00761098226113716\\
442.01	0.00761134142374703\\
443.01	0.00761171139239634\\
444.01	0.00761209260189073\\
445.01	0.00761248551355876\\
446.01	0.00761289061769659\\
447.01	0.00761330843632716\\
448.01	0.00761373952632497\\
449.01	0.00761418448296747\\
450.01	0.00761464394398552\\
451.01	0.00761511859419933\\
452.01	0.00761560917084363\\
453.01	0.00761611646970582\\
454.01	0.00761664135222632\\
455.01	0.00761718475374046\\
456.01	0.00761774769307792\\
457.01	0.00761833128377989\\
458.01	0.00761893674724141\\
459.01	0.00761956542807913\\
460.01	0.00762021881145276\\
461.01	0.00762089853711053\\
462.01	0.0076216063823837\\
463.01	0.00762234426991841\\
464.01	0.00762311478426318\\
465.01	0.00762392195354493\\
466.01	0.00762478319873316\\
467.01	0.00762593636630352\\
468.01	0.0076276003578341\\
469.01	0.00762931595635995\\
470.01	0.00763109162793824\\
471.01	0.007632864466257\\
472.01	0.0076346798115011\\
473.01	0.00763655067009834\\
474.01	0.00763851016701323\\
475.01	0.00764061064999437\\
476.01	0.00764277196042097\\
477.01	0.0076449900776459\\
478.01	0.00764726681690006\\
479.01	0.0076496040747383\\
480.01	0.00765200383421257\\
481.01	0.00765446817046258\\
482.01	0.00765699925676254\\
483.01	0.0076595993710684\\
484.01	0.00766227090311199\\
485.01	0.00766501636209198\\
486.01	0.00766783838500193\\
487.01	0.00767073974552987\\
488.01	0.0076737233627159\\
489.01	0.00767679230576994\\
490.01	0.00767994980575148\\
491.01	0.00768319928314602\\
492.01	0.00768654437221981\\
493.01	0.00768998898310824\\
494.01	0.00769353717777799\\
495.01	0.00769719325761285\\
496.01	0.00770096178927012\\
497.01	0.00770484763228593\\
498.01	0.00770885599703586\\
499.01	0.00771299269106102\\
500.01	0.00771726576956622\\
501.01	0.00772169639195388\\
502.01	0.00772633130952657\\
503.01	0.00773080003085001\\
504.01	0.00773537642440651\\
505.01	0.00774011049938724\\
506.01	0.00774501125892881\\
507.01	0.00775008859874314\\
508.01	0.00775535343463854\\
509.01	0.00776081785262583\\
510.01	0.00776649528661062\\
511.01	0.00777240073250528\\
512.01	0.00777855103115577\\
513.01	0.00778496544275355\\
514.01	0.00779166849852973\\
515.01	0.0077987164246254\\
516.01	0.0078065580776282\\
517.01	0.00781948919250816\\
518.01	0.00783469478529009\\
519.01	0.00785036114826689\\
520.01	0.0078664740808604\\
521.01	0.00788317313806947\\
522.01	0.00790009837544658\\
523.01	0.00791665118189264\\
524.01	0.00793370157798339\\
525.01	0.0079512706726707\\
526.01	0.00796938087882422\\
527.01	0.00798805601483179\\
528.01	0.00800732138455739\\
529.01	0.00802720396433126\\
530.01	0.00804773282009358\\
531.01	0.0080689388081184\\
532.01	0.00809085599779049\\
533.01	0.00811353415102963\\
534.01	0.00813718324366485\\
535.01	0.0081626618212072\\
536.01	0.00818931436024871\\
537.01	0.00821683851746443\\
538.01	0.00824516929339367\\
539.01	0.00827456816479533\\
540.01	0.00830541471736344\\
541.01	0.0083380882861966\\
542.01	0.00836680870452482\\
543.01	0.00839602967388001\\
544.01	0.00842467257447497\\
545.01	0.00845327850708192\\
546.01	0.00848323081843706\\
547.01	0.00851471674367666\\
548.01	0.00854877081933403\\
549.01	0.00860354157848882\\
550.01	0.00868426795184223\\
551.01	0.00876598105733376\\
552.01	0.00884943829693079\\
553.01	0.00893472459669881\\
554.01	0.00902193300020245\\
555.01	0.00911116602884714\\
556.01	0.00920253879627661\\
557.01	0.00929619350666944\\
558.01	0.00939237967236016\\
559.01	0.00949163304911767\\
560.01	0.00959232145567597\\
561.01	0.00969689070041613\\
562.01	0.009793005055321\\
563.01	0.00988051041000003\\
564.01	0.00996671501144665\\
565.01	0.01\\
566.01	0.01\\
567.01	0.01\\
568.01	0.01\\
569.01	0.01\\
570.01	0.01\\
571.01	0.01\\
572.01	0.01\\
573.01	0.01\\
574.01	0.01\\
575.01	0.01\\
576.01	0.01\\
577.01	0.01\\
578.01	0.01\\
579.01	0.01\\
580.01	0.01\\
581.01	0.01\\
582.01	0.01\\
583.01	0.01\\
584.01	0.01\\
585.01	0.01\\
586.01	0.01\\
587.01	0.01\\
588.01	0.01\\
589.01	0.01\\
590.01	0.01\\
591.01	0.01\\
592.01	0.01\\
593.01	0.01\\
594.01	0.01\\
595.01	0.01\\
596.01	0.01\\
597.01	0.01\\
598.01	0.01\\
599.01	0.01\\
599.02	0.01\\
599.03	0.01\\
599.04	0.01\\
599.05	0.01\\
599.06	0.01\\
599.07	0.01\\
599.08	0.01\\
599.09	0.01\\
599.1	0.01\\
599.11	0.01\\
599.12	0.01\\
599.13	0.01\\
599.14	0.01\\
599.15	0.01\\
599.16	0.01\\
599.17	0.01\\
599.18	0.01\\
599.19	0.01\\
599.2	0.01\\
599.21	0.01\\
599.22	0.01\\
599.23	0.01\\
599.24	0.01\\
599.25	0.01\\
599.26	0.01\\
599.27	0.01\\
599.28	0.01\\
599.29	0.01\\
599.3	0.01\\
599.31	0.01\\
599.32	0.01\\
599.33	0.01\\
599.34	0.01\\
599.35	0.01\\
599.36	0.01\\
599.37	0.01\\
599.38	0.01\\
599.39	0.01\\
599.4	0.01\\
599.41	0.01\\
599.42	0.01\\
599.43	0.01\\
599.44	0.01\\
599.45	0.01\\
599.46	0.01\\
599.47	0.01\\
599.48	0.01\\
599.49	0.01\\
599.5	0.01\\
599.51	0.01\\
599.52	0.01\\
599.53	0.01\\
599.54	0.01\\
599.55	0.01\\
599.56	0.01\\
599.57	0.01\\
599.58	0.01\\
599.59	0.01\\
599.6	0.01\\
599.61	0.01\\
599.62	0.01\\
599.63	0.01\\
599.64	0.01\\
599.65	0.01\\
599.66	0.01\\
599.67	0.01\\
599.68	0.01\\
599.69	0.01\\
599.7	0.01\\
599.71	0.01\\
599.72	0.01\\
599.73	0.01\\
599.74	0.01\\
599.75	0.01\\
599.76	0.01\\
599.77	0.01\\
599.78	0.01\\
599.79	0.01\\
599.8	0.01\\
599.81	0.01\\
599.82	0.01\\
599.83	0.01\\
599.84	0.01\\
599.85	0.01\\
599.86	0.01\\
599.87	0.01\\
599.88	0.01\\
599.89	0.01\\
599.9	0.01\\
599.91	0.01\\
599.92	0.01\\
599.93	0.01\\
599.94	0.01\\
599.95	0.01\\
599.96	0.01\\
599.97	0.01\\
599.98	0.01\\
599.99	0.01\\
600	0.01\\
};
\addplot [color=mycolor11,solid,forget plot]
  table[row sep=crcr]{%
0.01	0.00622874336967211\\
1.01	0.00622874337076519\\
2.01	0.00622874337188126\\
3.01	0.00622874337302081\\
4.01	0.00622874337418433\\
5.01	0.00622874337537234\\
6.01	0.00622874337658536\\
7.01	0.0062287433778239\\
8.01	0.00622874337908854\\
9.01	0.00622874338037979\\
10.01	0.00622874338169825\\
11.01	0.00622874338304448\\
12.01	0.00622874338441908\\
13.01	0.00622874338582262\\
14.01	0.00622874338725576\\
15.01	0.0062287433887191\\
16.01	0.00622874339021331\\
17.01	0.00622874339173904\\
18.01	0.00622874339329691\\
19.01	0.00622874339488765\\
20.01	0.00622874339651194\\
21.01	0.00622874339817051\\
22.01	0.00622874339986408\\
23.01	0.00622874340159342\\
24.01	0.00622874340335925\\
25.01	0.00622874340516236\\
26.01	0.00622874340700356\\
27.01	0.00622874340888365\\
28.01	0.00622874341080346\\
29.01	0.00622874341276384\\
30.01	0.00622874341476564\\
31.01	0.00622874341680977\\
32.01	0.0062287434188971\\
33.01	0.00622874342102859\\
34.01	0.00622874342320516\\
35.01	0.00622874342542776\\
36.01	0.0062287434276974\\
37.01	0.00622874343001506\\
38.01	0.00622874343238178\\
39.01	0.0062287434347986\\
40.01	0.00622874343726661\\
41.01	0.00622874343978689\\
42.01	0.00622874344236055\\
43.01	0.00622874344498875\\
44.01	0.00622874344767263\\
45.01	0.00622874345041341\\
46.01	0.00622874345321229\\
47.01	0.00622874345607054\\
48.01	0.00622874345898939\\
49.01	0.00622874346197016\\
50.01	0.00622874346501422\\
51.01	0.00622874346812285\\
52.01	0.00622874347129746\\
53.01	0.00622874347453949\\
54.01	0.00622874347785037\\
55.01	0.00622874348123158\\
56.01	0.00622874348468462\\
57.01	0.00622874348821105\\
58.01	0.00622874349181243\\
59.01	0.00622874349549038\\
60.01	0.00622874349924654\\
61.01	0.00622874350308261\\
62.01	0.00622874350700028\\
63.01	0.00622874351100132\\
64.01	0.00622874351508752\\
65.01	0.00622874351926072\\
66.01	0.00622874352352279\\
67.01	0.00622874352787564\\
68.01	0.00622874353232123\\
69.01	0.00622874353686158\\
70.01	0.00622874354149868\\
71.01	0.00622874354623466\\
72.01	0.00622874355107163\\
73.01	0.00622874355601179\\
74.01	0.00622874356105736\\
75.01	0.0062287435662106\\
76.01	0.00622874357147386\\
77.01	0.00622874357684951\\
78.01	0.00622874358233997\\
79.01	0.00622874358794772\\
80.01	0.00622874359367532\\
81.01	0.00622874359952534\\
82.01	0.00622874360550048\\
83.01	0.00622874361160334\\
84.01	0.00622874361783679\\
85.01	0.00622874362420362\\
86.01	0.00622874363070671\\
87.01	0.00622874363734905\\
88.01	0.00622874364413363\\
89.01	0.00622874365106355\\
90.01	0.00622874365814195\\
91.01	0.00622874366537207\\
92.01	0.0062287436727572\\
93.01	0.00622874368030072\\
94.01	0.00622874368800604\\
95.01	0.00622874369587671\\
96.01	0.00622874370391631\\
97.01	0.00622874371212854\\
98.01	0.00622874372051712\\
99.01	0.00622874372908591\\
100.01	0.00622874373783884\\
101.01	0.00622874374677993\\
102.01	0.00622874375591327\\
103.01	0.00622874376524304\\
104.01	0.00622874377477355\\
105.01	0.00622874378450918\\
106.01	0.0062287437944544\\
107.01	0.00622874380461378\\
108.01	0.00622874381499201\\
109.01	0.00622874382559389\\
110.01	0.00622874383642428\\
111.01	0.0062287438474882\\
112.01	0.00622874385879076\\
113.01	0.00622874387033717\\
114.01	0.0062287438821328\\
115.01	0.00622874389418309\\
116.01	0.00622874390649363\\
117.01	0.00622874391907011\\
118.01	0.0062287439319184\\
119.01	0.00622874394504444\\
120.01	0.00622874395845434\\
121.01	0.00622874397215433\\
122.01	0.00622874398615079\\
123.01	0.00622874400045023\\
124.01	0.00622874401505932\\
125.01	0.0062287440299849\\
126.01	0.00622874404523388\\
127.01	0.00622874406081344\\
128.01	0.00622874407673082\\
129.01	0.00622874409299351\\
130.01	0.00622874410960908\\
131.01	0.00622874412658535\\
132.01	0.00622874414393026\\
133.01	0.00622874416165197\\
134.01	0.00622874417975879\\
135.01	0.0062287441982592\\
136.01	0.00622874421716195\\
137.01	0.00622874423647596\\
138.01	0.00622874425621028\\
139.01	0.00622874427637422\\
140.01	0.00622874429697733\\
141.01	0.0062287443180293\\
142.01	0.00622874433954011\\
143.01	0.00622874436151992\\
144.01	0.00622874438397915\\
145.01	0.00622874440692843\\
146.01	0.00622874443037867\\
147.01	0.00622874445434096\\
148.01	0.00622874447882671\\
149.01	0.00622874450384755\\
150.01	0.00622874452941541\\
151.01	0.00622874455554244\\
152.01	0.00622874458224109\\
153.01	0.00622874460952411\\
154.01	0.00622874463740453\\
155.01	0.00622874466589564\\
156.01	0.0062287446950111\\
157.01	0.00622874472476483\\
158.01	0.00622874475517108\\
159.01	0.00622874478624445\\
160.01	0.0062287448179998\\
161.01	0.00622874485045245\\
162.01	0.00622874488361795\\
163.01	0.00622874491751226\\
164.01	0.0062287449521517\\
165.01	0.00622874498755297\\
166.01	0.00622874502373314\\
167.01	0.00622874506070968\\
168.01	0.00622874509850044\\
169.01	0.00622874513712371\\
170.01	0.00622874517659817\\
171.01	0.00622874521694295\\
172.01	0.00622874525817759\\
173.01	0.00622874530032211\\
174.01	0.00622874534339697\\
175.01	0.00622874538742312\\
176.01	0.00622874543242195\\
177.01	0.00622874547841538\\
178.01	0.00622874552542584\\
179.01	0.00622874557347622\\
180.01	0.00622874562258999\\
181.01	0.00622874567279114\\
182.01	0.00622874572410424\\
183.01	0.00622874577655436\\
184.01	0.0062287458301672\\
185.01	0.00622874588496906\\
186.01	0.00622874594098676\\
187.01	0.00622874599824788\\
188.01	0.00622874605678054\\
189.01	0.0062287461166135\\
190.01	0.00622874617777622\\
191.01	0.00622874624029885\\
192.01	0.00622874630421219\\
193.01	0.00622874636954777\\
194.01	0.0062287464363379\\
195.01	0.00622874650461555\\
196.01	0.0062287465744145\\
197.01	0.00622874664576929\\
198.01	0.00622874671871534\\
199.01	0.00622874679328871\\
200.01	0.00622874686952648\\
201.01	0.00622874694746649\\
202.01	0.00622874702714749\\
203.01	0.00622874710860908\\
204.01	0.00622874719189184\\
205.01	0.00622874727703724\\
206.01	0.00622874736408774\\
207.01	0.00622874745308676\\
208.01	0.00622874754407874\\
209.01	0.00622874763710914\\
210.01	0.00622874773222447\\
211.01	0.00622874782947234\\
212.01	0.00622874792890143\\
213.01	0.0062287480305616\\
214.01	0.00622874813450377\\
215.01	0.00622874824078015\\
216.01	0.00622874834944407\\
217.01	0.00622874846055013\\
218.01	0.00622874857415424\\
219.01	0.00622874869031352\\
220.01	0.00622874880908646\\
221.01	0.00622874893053291\\
222.01	0.0062287490547141\\
223.01	0.00622874918169265\\
224.01	0.00622874931153266\\
225.01	0.00622874944429972\\
226.01	0.00622874958006093\\
227.01	0.0062287497188849\\
228.01	0.00622874986084189\\
229.01	0.00622875000600378\\
230.01	0.00622875015444407\\
231.01	0.006228750306238\\
232.01	0.00622875046146251\\
233.01	0.00622875062019639\\
234.01	0.00622875078252015\\
235.01	0.00622875094851629\\
236.01	0.00622875111826908\\
237.01	0.00622875129186482\\
238.01	0.0062287514693918\\
239.01	0.00622875165094032\\
240.01	0.00622875183660278\\
241.01	0.00622875202647372\\
242.01	0.00622875222064985\\
243.01	0.00622875241923009\\
244.01	0.00622875262231571\\
245.01	0.00622875283001024\\
246.01	0.00622875304241962\\
247.01	0.00622875325965224\\
248.01	0.00622875348181899\\
249.01	0.00622875370903329\\
250.01	0.0062287539414112\\
251.01	0.00622875417907139\\
252.01	0.00622875442213535\\
253.01	0.0062287546707273\\
254.01	0.00622875492497431\\
255.01	0.00622875518500639\\
256.01	0.00622875545095655\\
257.01	0.00622875572296079\\
258.01	0.0062287560011583\\
259.01	0.00622875628569144\\
260.01	0.00622875657670578\\
261.01	0.00622875687435032\\
262.01	0.00622875717877736\\
263.01	0.00622875749014281\\
264.01	0.00622875780860603\\
265.01	0.00622875813433009\\
266.01	0.0062287584674818\\
267.01	0.00622875880823172\\
268.01	0.00622875915675435\\
269.01	0.00622875951322819\\
270.01	0.00622875987783575\\
271.01	0.0062287602507638\\
272.01	0.00622876063220325\\
273.01	0.00622876102234947\\
274.01	0.00622876142140221\\
275.01	0.00622876182956587\\
276.01	0.00622876224704936\\
277.01	0.0062287626740665\\
278.01	0.00622876311083586\\
279.01	0.00622876355758107\\
280.01	0.00622876401453081\\
281.01	0.00622876448191896\\
282.01	0.00622876495998476\\
283.01	0.0062287654489729\\
284.01	0.00622876594913358\\
285.01	0.00622876646072274\\
286.01	0.0062287669840022\\
287.01	0.00622876751923962\\
288.01	0.00622876806670885\\
289.01	0.00622876862668995\\
290.01	0.00622876919946939\\
291.01	0.00622876978534014\\
292.01	0.00622877038460187\\
293.01	0.00622877099756108\\
294.01	0.0062287716245313\\
295.01	0.00622877226583321\\
296.01	0.00622877292179477\\
297.01	0.00622877359275155\\
298.01	0.00622877427904675\\
299.01	0.00622877498103145\\
300.01	0.00622877569906479\\
301.01	0.00622877643351417\\
302.01	0.00622877718475544\\
303.01	0.00622877795317313\\
304.01	0.00622877873916063\\
305.01	0.00622877954312041\\
306.01	0.00622878036546429\\
307.01	0.00622878120661364\\
308.01	0.00622878206699957\\
309.01	0.00622878294706329\\
310.01	0.00622878384725626\\
311.01	0.00622878476804051\\
312.01	0.00622878570988889\\
313.01	0.00622878667328535\\
314.01	0.00622878765872523\\
315.01	0.00622878866671555\\
316.01	0.00622878969777532\\
317.01	0.00622879075243587\\
318.01	0.00622879183124113\\
319.01	0.00622879293474808\\
320.01	0.00622879406352692\\
321.01	0.00622879521816158\\
322.01	0.00622879639925002\\
323.01	0.00622879760740464\\
324.01	0.00622879884325266\\
325.01	0.00622880010743651\\
326.01	0.00622880140061428\\
327.01	0.00622880272346016\\
328.01	0.00622880407666484\\
329.01	0.00622880546093602\\
330.01	0.00622880687699887\\
331.01	0.00622880832559649\\
332.01	0.00622880980749047\\
333.01	0.00622881132346137\\
334.01	0.00622881287430931\\
335.01	0.00622881446085444\\
336.01	0.00622881608393754\\
337.01	0.00622881774442072\\
338.01	0.00622881944318785\\
339.01	0.0062288211811453\\
340.01	0.00622882295922254\\
341.01	0.0062288247783728\\
342.01	0.00622882663957379\\
343.01	0.00622882854382834\\
344.01	0.00622883049216513\\
345.01	0.00622883248563947\\
346.01	0.00622883452533406\\
347.01	0.00622883661235968\\
348.01	0.00622883874785609\\
349.01	0.00622884093299282\\
350.01	0.00622884316896999\\
351.01	0.0062288454570192\\
352.01	0.00622884779840442\\
353.01	0.0062288501944229\\
354.01	0.00622885264640609\\
355.01	0.00622885515572062\\
356.01	0.00622885772376928\\
357.01	0.00622886035199203\\
358.01	0.00622886304186699\\
359.01	0.00622886579491157\\
360.01	0.00622886861268352\\
361.01	0.00622887149678202\\
362.01	0.00622887444884886\\
363.01	0.00622887747056955\\
364.01	0.00622888056367452\\
365.01	0.0062288837299404\\
366.01	0.00622888697119115\\
367.01	0.00622889028929936\\
368.01	0.00622889368618758\\
369.01	0.00622889716382952\\
370.01	0.00622890072425154\\
371.01	0.00622890436953377\\
372.01	0.00622890810181167\\
373.01	0.00622891192327731\\
374.01	0.00622891583618078\\
375.01	0.00622891984283135\\
376.01	0.00622892394559809\\
377.01	0.00622892814690864\\
378.01	0.00622893244925577\\
379.01	0.00622893685521343\\
380.01	0.0062289413674131\\
381.01	0.00622894598855356\\
382.01	0.00622895072140726\\
383.01	0.00622895556882267\\
384.01	0.00622896053372661\\
385.01	0.00622896561912678\\
386.01	0.00622897082811439\\
387.01	0.00622897616386689\\
388.01	0.00622898162965073\\
389.01	0.00622898722882441\\
390.01	0.0062289929648415\\
391.01	0.006228998841254\\
392.01	0.00622900486171554\\
393.01	0.00622901102998509\\
394.01	0.00622901734993058\\
395.01	0.00622902382553277\\
396.01	0.00622903046088948\\
397.01	0.00622903726021968\\
398.01	0.00622904422786811\\
399.01	0.00622905136831001\\
400.01	0.006229058686156\\
401.01	0.00622906618615744\\
402.01	0.00622907387321184\\
403.01	0.00622908175236879\\
404.01	0.00622908982883595\\
405.01	0.00622909810798566\\
406.01	0.00622910659536166\\
407.01	0.00622911529668641\\
408.01	0.00622912421786871\\
409.01	0.00622913336501174\\
410.01	0.00622914274442171\\
411.01	0.00622915236261689\\
412.01	0.00622916222633727\\
413.01	0.00622917234255481\\
414.01	0.0062291827184843\\
415.01	0.00622919336159492\\
416.01	0.00622920427962254\\
417.01	0.00622921548058287\\
418.01	0.00622922697278535\\
419.01	0.00622923876484813\\
420.01	0.00622925086571396\\
421.01	0.0062292632846672\\
422.01	0.006229276031352\\
423.01	0.00622928911579191\\
424.01	0.0062293025484106\\
425.01	0.00622931634005455\\
426.01	0.00622933050201698\\
427.01	0.00622934504606398\\
428.01	0.00622935998446241\\
429.01	0.00622937533001021\\
430.01	0.00622939109606901\\
431.01	0.00622940729659966\\
432.01	0.00622942394620052\\
433.01	0.00622944106014928\\
434.01	0.00622945865444845\\
435.01	0.00622947674587496\\
436.01	0.00622949535203458\\
437.01	0.00622951449142147\\
438.01	0.0062295341834836\\
439.01	0.00622955444869505\\
440.01	0.00622957530863579\\
441.01	0.00622959678608016\\
442.01	0.00622961890509521\\
443.01	0.00622964169115047\\
444.01	0.00622966517124064\\
445.01	0.00622968937402342\\
446.01	0.00622971432997473\\
447.01	0.0062297400715641\\
448.01	0.00622976663345353\\
449.01	0.00622979405272399\\
450.01	0.00622982236913393\\
451.01	0.00622985162541575\\
452.01	0.00622988186761656\\
453.01	0.00622991314549177\\
454.01	0.00622994551296099\\
455.01	0.00622997902863769\\
456.01	0.00623001375644706\\
457.01	0.0062300497663487\\
458.01	0.00623008713518149\\
459.01	0.0062301259476367\\
460.01	0.00623016629726389\\
461.01	0.00623020828696017\\
462.01	0.00623025202854534\\
463.01	0.00623029765571453\\
464.01	0.00623034538591046\\
465.01	0.00623039585549172\\
466.01	0.00623045288955951\\
467.01	0.00623053630881283\\
468.01	0.00623063889289436\\
469.01	0.00623074647477932\\
470.01	0.00623087880683824\\
471.01	0.00623111863693082\\
472.01	0.00623137485948774\\
473.01	0.00623163880954006\\
474.01	0.00623191347255253\\
475.01	0.00623219912330979\\
476.01	0.00623249250456114\\
477.01	0.00623279382710131\\
478.01	0.00623310339606975\\
479.01	0.00623342153728351\\
480.01	0.0062337485993486\\
481.01	0.0062340849560285\\
482.01	0.00623443100890397\\
483.01	0.00623478719036257\\
484.01	0.00623515396696156\\
485.01	0.00623553184321373\\
486.01	0.00623592136584611\\
487.01	0.00623632312856245\\
488.01	0.00623673777723619\\
489.01	0.00623716601546134\\
490.01	0.00623760861225348\\
491.01	0.00623806641025245\\
492.01	0.00623854033874851\\
493.01	0.00623903142030608\\
494.01	0.00623954077702533\\
495.01	0.00624006964774182\\
496.01	0.00624061940331291\\
497.01	0.00624119156604566\\
498.01	0.00624178784668994\\
499.01	0.00624241029588527\\
500.01	0.00624306230901286\\
501.01	0.00624375695163478\\
502.01	0.00624461716589386\\
503.01	0.00624611619553652\\
504.01	0.00624771319754122\\
505.01	0.00624936308412463\\
506.01	0.00625106855843081\\
507.01	0.00625283256118287\\
508.01	0.00625465830308046\\
509.01	0.00625654930303459\\
510.01	0.00625850943368729\\
511.01	0.00626054297740323\\
512.01	0.00626265470710641\\
513.01	0.00626485008889032\\
514.01	0.00626713634837834\\
515.01	0.0062695304191542\\
516.01	0.00627211819892404\\
517.01	0.00627509256745018\\
518.01	0.00627823160609473\\
519.01	0.00628150390118513\\
520.01	0.00628492990926515\\
521.01	0.00628862902389006\\
522.01	0.00629387313213118\\
523.01	0.00630060424757553\\
524.01	0.00630754953469232\\
525.01	0.00631472012940907\\
526.01	0.00632212810859392\\
527.01	0.00632978659833269\\
528.01	0.00633770989679532\\
529.01	0.00634591363563067\\
530.01	0.00635441494768246\\
531.01	0.00636323266963891\\
532.01	0.00637238791548698\\
533.01	0.00638190690078944\\
534.01	0.00639183545566685\\
535.01	0.00640223102191707\\
536.01	0.0064130976871169\\
537.01	0.00642448974005286\\
538.01	0.00643606523913887\\
539.01	0.00644817761602978\\
540.01	0.0064611421623985\\
541.01	0.00647792950210211\\
542.01	0.00650441513295767\\
543.01	0.00653149862206026\\
544.01	0.00655873357067719\\
545.01	0.00658686779613701\\
546.01	0.00661598895446187\\
547.01	0.0066461769705681\\
548.01	0.00667773629947955\\
549.01	0.00671207064339874\\
550.01	0.00674787100487959\\
551.01	0.00678289469210539\\
552.01	0.00681912610093263\\
553.01	0.00685665739171707\\
554.01	0.00689559221571758\\
555.01	0.00693604783846996\\
556.01	0.00697815890035221\\
557.01	0.00702209126971914\\
558.01	0.00706812629165005\\
559.01	0.00711711053467372\\
560.01	0.00716820906240298\\
561.01	0.0072185056339763\\
562.01	0.00730871849609079\\
563.01	0.00741524301621832\\
564.01	0.00752344920588999\\
565.01	0.0076313687639263\\
566.01	0.00774115467975347\\
567.01	0.00785363528701536\\
568.01	0.00796898385122909\\
569.01	0.00808739615598924\\
570.01	0.00820910690812936\\
571.01	0.00833453511991283\\
572.01	0.00846526876419162\\
573.01	0.00860519158723727\\
574.01	0.0087215661705536\\
575.01	0.00883762282490467\\
576.01	0.00895594286371285\\
577.01	0.00907652599155817\\
578.01	0.00919953250302053\\
579.01	0.00932501892119195\\
580.01	0.00945299567345381\\
581.01	0.00958345369111326\\
582.01	0.00971634636051373\\
583.01	0.00985132789365488\\
584.01	0.00997897979146759\\
585.01	0.01\\
586.01	0.01\\
587.01	0.01\\
588.01	0.01\\
589.01	0.01\\
590.01	0.01\\
591.01	0.01\\
592.01	0.01\\
593.01	0.01\\
594.01	0.01\\
595.01	0.01\\
596.01	0.01\\
597.01	0.01\\
598.01	0.01\\
599.01	0.01\\
599.02	0.01\\
599.03	0.01\\
599.04	0.01\\
599.05	0.01\\
599.06	0.01\\
599.07	0.01\\
599.08	0.01\\
599.09	0.01\\
599.1	0.01\\
599.11	0.01\\
599.12	0.01\\
599.13	0.01\\
599.14	0.01\\
599.15	0.01\\
599.16	0.01\\
599.17	0.01\\
599.18	0.01\\
599.19	0.01\\
599.2	0.01\\
599.21	0.01\\
599.22	0.01\\
599.23	0.01\\
599.24	0.01\\
599.25	0.01\\
599.26	0.01\\
599.27	0.01\\
599.28	0.01\\
599.29	0.01\\
599.3	0.01\\
599.31	0.01\\
599.32	0.01\\
599.33	0.01\\
599.34	0.01\\
599.35	0.01\\
599.36	0.01\\
599.37	0.01\\
599.38	0.01\\
599.39	0.01\\
599.4	0.01\\
599.41	0.01\\
599.42	0.01\\
599.43	0.01\\
599.44	0.01\\
599.45	0.01\\
599.46	0.01\\
599.47	0.01\\
599.48	0.01\\
599.49	0.01\\
599.5	0.01\\
599.51	0.01\\
599.52	0.01\\
599.53	0.01\\
599.54	0.01\\
599.55	0.01\\
599.56	0.01\\
599.57	0.01\\
599.58	0.01\\
599.59	0.01\\
599.6	0.01\\
599.61	0.01\\
599.62	0.01\\
599.63	0.01\\
599.64	0.01\\
599.65	0.01\\
599.66	0.01\\
599.67	0.01\\
599.68	0.01\\
599.69	0.01\\
599.7	0.01\\
599.71	0.01\\
599.72	0.01\\
599.73	0.01\\
599.74	0.01\\
599.75	0.01\\
599.76	0.01\\
599.77	0.01\\
599.78	0.01\\
599.79	0.01\\
599.8	0.01\\
599.81	0.01\\
599.82	0.01\\
599.83	0.01\\
599.84	0.01\\
599.85	0.01\\
599.86	0.01\\
599.87	0.01\\
599.88	0.01\\
599.89	0.01\\
599.9	0.01\\
599.91	0.01\\
599.92	0.01\\
599.93	0.01\\
599.94	0.01\\
599.95	0.01\\
599.96	0.01\\
599.97	0.01\\
599.98	0.01\\
599.99	0.01\\
600	0.01\\
};
\addplot [color=mycolor12,solid,forget plot]
  table[row sep=crcr]{%
0.01	0.00371473620820593\\
1.01	0.00371473620829894\\
2.01	0.00371473620839391\\
3.01	0.00371473620849088\\
4.01	0.00371473620858988\\
5.01	0.00371473620869097\\
6.01	0.0037147362087942\\
7.01	0.00371473620889959\\
8.01	0.0037147362090072\\
9.01	0.00371473620911708\\
10.01	0.00371473620922927\\
11.01	0.00371473620934383\\
12.01	0.0037147362094608\\
13.01	0.00371473620958023\\
14.01	0.00371473620970218\\
15.01	0.0037147362098267\\
16.01	0.00371473620995385\\
17.01	0.00371473621008368\\
18.01	0.00371473621021625\\
19.01	0.00371473621035161\\
20.01	0.00371473621048983\\
21.01	0.00371473621063097\\
22.01	0.00371473621077508\\
23.01	0.00371473621092224\\
24.01	0.0037147362110725\\
25.01	0.00371473621122594\\
26.01	0.00371473621138262\\
27.01	0.00371473621154261\\
28.01	0.00371473621170598\\
29.01	0.0037147362118728\\
30.01	0.00371473621204314\\
31.01	0.00371473621221709\\
32.01	0.00371473621239472\\
33.01	0.0037147362125761\\
34.01	0.00371473621276132\\
35.01	0.00371473621295046\\
36.01	0.00371473621314359\\
37.01	0.00371473621334082\\
38.01	0.00371473621354222\\
39.01	0.00371473621374789\\
40.01	0.00371473621395791\\
41.01	0.00371473621417238\\
42.01	0.00371473621439139\\
43.01	0.00371473621461505\\
44.01	0.00371473621484344\\
45.01	0.00371473621507667\\
46.01	0.00371473621531486\\
47.01	0.00371473621555809\\
48.01	0.00371473621580648\\
49.01	0.00371473621606015\\
50.01	0.00371473621631919\\
51.01	0.00371473621658373\\
52.01	0.0037147362168539\\
53.01	0.00371473621712979\\
54.01	0.00371473621741155\\
55.01	0.00371473621769929\\
56.01	0.00371473621799314\\
57.01	0.00371473621829324\\
58.01	0.00371473621859972\\
59.01	0.00371473621891272\\
60.01	0.00371473621923237\\
61.01	0.00371473621955883\\
62.01	0.00371473621989223\\
63.01	0.00371473622023272\\
64.01	0.00371473622058046\\
65.01	0.00371473622093561\\
66.01	0.00371473622129832\\
67.01	0.00371473622166876\\
68.01	0.00371473622204709\\
69.01	0.00371473622243348\\
70.01	0.00371473622282811\\
71.01	0.00371473622323116\\
72.01	0.00371473622364281\\
73.01	0.00371473622406322\\
74.01	0.00371473622449262\\
75.01	0.00371473622493118\\
76.01	0.00371473622537911\\
77.01	0.00371473622583659\\
78.01	0.00371473622630385\\
79.01	0.0037147362267811\\
80.01	0.00371473622726855\\
81.01	0.00371473622776641\\
82.01	0.00371473622827493\\
83.01	0.00371473622879431\\
84.01	0.00371473622932481\\
85.01	0.00371473622986666\\
86.01	0.00371473623042012\\
87.01	0.00371473623098542\\
88.01	0.00371473623156284\\
89.01	0.00371473623215262\\
90.01	0.00371473623275504\\
91.01	0.00371473623337037\\
92.01	0.0037147362339989\\
93.01	0.0037147362346409\\
94.01	0.00371473623529669\\
95.01	0.00371473623596655\\
96.01	0.00371473623665079\\
97.01	0.00371473623734971\\
98.01	0.00371473623806365\\
99.01	0.00371473623879293\\
100.01	0.00371473623953789\\
101.01	0.00371473624029886\\
102.01	0.00371473624107619\\
103.01	0.00371473624187024\\
104.01	0.0037147362426814\\
105.01	0.00371473624350999\\
106.01	0.00371473624435643\\
107.01	0.0037147362452211\\
108.01	0.0037147362461044\\
109.01	0.00371473624700674\\
110.01	0.00371473624792853\\
111.01	0.0037147362488702\\
112.01	0.00371473624983218\\
113.01	0.00371473625081492\\
114.01	0.00371473625181887\\
115.01	0.00371473625284451\\
116.01	0.00371473625389229\\
117.01	0.00371473625496273\\
118.01	0.00371473625605628\\
119.01	0.00371473625717348\\
120.01	0.00371473625831486\\
121.01	0.00371473625948092\\
122.01	0.00371473626067222\\
123.01	0.00371473626188931\\
124.01	0.00371473626313277\\
125.01	0.00371473626440315\\
126.01	0.00371473626570108\\
127.01	0.00371473626702715\\
128.01	0.00371473626838197\\
129.01	0.00371473626976619\\
130.01	0.00371473627118046\\
131.01	0.00371473627262542\\
132.01	0.00371473627410177\\
133.01	0.00371473627561019\\
134.01	0.0037147362771514\\
135.01	0.00371473627872616\\
136.01	0.00371473628033513\\
137.01	0.00371473628197909\\
138.01	0.00371473628365884\\
139.01	0.00371473628537519\\
140.01	0.00371473628712891\\
141.01	0.00371473628892085\\
142.01	0.00371473629075184\\
143.01	0.00371473629262277\\
144.01	0.00371473629453451\\
145.01	0.00371473629648798\\
146.01	0.00371473629848409\\
147.01	0.0037147363005238\\
148.01	0.00371473630260807\\
149.01	0.0037147363047379\\
150.01	0.00371473630691429\\
151.01	0.00371473630913829\\
152.01	0.00371473631141097\\
153.01	0.00371473631373339\\
154.01	0.00371473631610667\\
155.01	0.00371473631853194\\
156.01	0.00371473632101037\\
157.01	0.00371473632354315\\
158.01	0.00371473632613148\\
159.01	0.00371473632877661\\
160.01	0.0037147363314798\\
161.01	0.00371473633424236\\
162.01	0.0037147363370656\\
163.01	0.00371473633995091\\
164.01	0.00371473634289965\\
165.01	0.00371473634591325\\
166.01	0.00371473634899318\\
167.01	0.00371473635214093\\
168.01	0.00371473635535798\\
169.01	0.0037147363586459\\
170.01	0.00371473636200632\\
171.01	0.00371473636544083\\
172.01	0.00371473636895111\\
173.01	0.00371473637253886\\
174.01	0.00371473637620583\\
175.01	0.00371473637995379\\
176.01	0.00371473638378457\\
177.01	0.00371473638770004\\
178.01	0.0037147363917021\\
179.01	0.00371473639579271\\
180.01	0.00371473639997387\\
181.01	0.00371473640424761\\
182.01	0.00371473640861603\\
183.01	0.00371473641308127\\
184.01	0.0037147364176455\\
185.01	0.00371473642231097\\
186.01	0.00371473642708001\\
187.01	0.0037147364319549\\
188.01	0.00371473643693806\\
189.01	0.00371473644203194\\
190.01	0.00371473644723905\\
191.01	0.00371473645256196\\
192.01	0.00371473645800328\\
193.01	0.00371473646356573\\
194.01	0.00371473646925201\\
195.01	0.00371473647506497\\
196.01	0.00371473648100747\\
197.01	0.00371473648708245\\
198.01	0.00371473649329293\\
199.01	0.003714736499642\\
200.01	0.00371473650613279\\
201.01	0.00371473651276853\\
202.01	0.00371473651955252\\
203.01	0.00371473652648813\\
204.01	0.00371473653357883\\
205.01	0.00371473654082815\\
206.01	0.00371473654823969\\
207.01	0.00371473655581715\\
208.01	0.00371473656356433\\
209.01	0.0037147365714851\\
210.01	0.00371473657958342\\
211.01	0.00371473658786332\\
212.01	0.00371473659632899\\
213.01	0.00371473660498463\\
214.01	0.00371473661383461\\
215.01	0.00371473662288336\\
216.01	0.00371473663213543\\
217.01	0.00371473664159549\\
218.01	0.00371473665126825\\
219.01	0.00371473666115862\\
220.01	0.00371473667127157\\
221.01	0.00371473668161218\\
222.01	0.00371473669218568\\
223.01	0.00371473670299742\\
224.01	0.00371473671405283\\
225.01	0.0037147367253575\\
226.01	0.00371473673691716\\
227.01	0.00371473674873764\\
228.01	0.00371473676082493\\
229.01	0.00371473677318516\\
230.01	0.00371473678582457\\
231.01	0.00371473679874958\\
232.01	0.00371473681196674\\
233.01	0.00371473682548277\\
234.01	0.00371473683930452\\
235.01	0.003714736853439\\
236.01	0.00371473686789341\\
237.01	0.00371473688267509\\
238.01	0.00371473689779156\\
239.01	0.00371473691325051\\
240.01	0.00371473692905982\\
241.01	0.00371473694522752\\
242.01	0.00371473696176186\\
243.01	0.00371473697867128\\
244.01	0.00371473699596437\\
245.01	0.00371473701364998\\
246.01	0.00371473703173709\\
247.01	0.00371473705023498\\
248.01	0.00371473706915306\\
249.01	0.003714737088501\\
250.01	0.00371473710828869\\
251.01	0.00371473712852623\\
252.01	0.00371473714922397\\
253.01	0.00371473717039247\\
254.01	0.00371473719204259\\
255.01	0.00371473721418537\\
256.01	0.00371473723683213\\
257.01	0.00371473725999448\\
258.01	0.00371473728368426\\
259.01	0.00371473730791359\\
260.01	0.00371473733269488\\
261.01	0.0037147373580408\\
262.01	0.00371473738396434\\
263.01	0.00371473741047875\\
264.01	0.00371473743759763\\
265.01	0.00371473746533485\\
266.01	0.00371473749370461\\
267.01	0.00371473752272144\\
268.01	0.00371473755240022\\
269.01	0.00371473758275611\\
270.01	0.00371473761380469\\
271.01	0.00371473764556183\\
272.01	0.00371473767804383\\
273.01	0.00371473771126729\\
274.01	0.00371473774524923\\
275.01	0.00371473778000706\\
276.01	0.00371473781555857\\
277.01	0.00371473785192195\\
278.01	0.00371473788911584\\
279.01	0.00371473792715927\\
280.01	0.00371473796607169\\
281.01	0.00371473800587306\\
282.01	0.00371473804658372\\
283.01	0.00371473808822451\\
284.01	0.00371473813081676\\
285.01	0.00371473817438223\\
286.01	0.00371473821894323\\
287.01	0.00371473826452256\\
288.01	0.00371473831114352\\
289.01	0.00371473835882998\\
290.01	0.00371473840760632\\
291.01	0.00371473845749749\\
292.01	0.00371473850852902\\
293.01	0.00371473856072698\\
294.01	0.00371473861411811\\
295.01	0.00371473866872968\\
296.01	0.00371473872458965\\
297.01	0.00371473878172657\\
298.01	0.00371473884016969\\
299.01	0.00371473889994889\\
300.01	0.00371473896109475\\
301.01	0.00371473902363859\\
302.01	0.00371473908761238\\
303.01	0.0037147391530489\\
304.01	0.00371473921998162\\
305.01	0.00371473928844485\\
306.01	0.00371473935847365\\
307.01	0.0037147394301039\\
308.01	0.00371473950337232\\
309.01	0.0037147395783165\\
310.01	0.00371473965497488\\
311.01	0.00371473973338682\\
312.01	0.0037147398135926\\
313.01	0.00371473989563342\\
314.01	0.00371473997955152\\
315.01	0.00371474006539003\\
316.01	0.00371474015319321\\
317.01	0.0037147402430063\\
318.01	0.00371474033487565\\
319.01	0.0037147404288487\\
320.01	0.00371474052497403\\
321.01	0.00371474062330138\\
322.01	0.00371474072388172\\
323.01	0.00371474082676721\\
324.01	0.00371474093201128\\
325.01	0.00371474103966871\\
326.01	0.00371474114979555\\
327.01	0.00371474126244925\\
328.01	0.00371474137768867\\
329.01	0.00371474149557411\\
330.01	0.00371474161616739\\
331.01	0.00371474173953182\\
332.01	0.00371474186573233\\
333.01	0.00371474199483544\\
334.01	0.00371474212690934\\
335.01	0.00371474226202393\\
336.01	0.00371474240025091\\
337.01	0.00371474254166373\\
338.01	0.00371474268633775\\
339.01	0.00371474283435023\\
340.01	0.00371474298578043\\
341.01	0.00371474314070958\\
342.01	0.00371474329922107\\
343.01	0.00371474346140036\\
344.01	0.00371474362733519\\
345.01	0.00371474379711553\\
346.01	0.00371474397083368\\
347.01	0.00371474414858437\\
348.01	0.00371474433046478\\
349.01	0.00371474451657463\\
350.01	0.00371474470701625\\
351.01	0.00371474490189468\\
352.01	0.0037147451013177\\
353.01	0.00371474530539594\\
354.01	0.00371474551424294\\
355.01	0.00371474572797523\\
356.01	0.00371474594671247\\
357.01	0.00371474617057745\\
358.01	0.00371474639969623\\
359.01	0.00371474663419821\\
360.01	0.00371474687421626\\
361.01	0.00371474711988675\\
362.01	0.00371474737134971\\
363.01	0.00371474762874888\\
364.01	0.00371474789223186\\
365.01	0.00371474816195014\\
366.01	0.00371474843805931\\
367.01	0.00371474872071906\\
368.01	0.00371474901009336\\
369.01	0.00371474930635054\\
370.01	0.00371474960966338\\
371.01	0.0037147499202093\\
372.01	0.00371475023817042\\
373.01	0.00371475056373365\\
374.01	0.00371475089709086\\
375.01	0.00371475123843888\\
376.01	0.00371475158797938\\
377.01	0.00371475194591887\\
378.01	0.00371475231247041\\
379.01	0.0037147526878539\\
380.01	0.00371475307229368\\
381.01	0.00371475346601998\\
382.01	0.00371475386926931\\
383.01	0.00371475428228464\\
384.01	0.00371475470531573\\
385.01	0.00371475513861914\\
386.01	0.00371475558245867\\
387.01	0.00371475603710542\\
388.01	0.00371475650283813\\
389.01	0.00371475697994341\\
390.01	0.00371475746871603\\
391.01	0.0037147579694591\\
392.01	0.00371475848248448\\
393.01	0.00371475900811306\\
394.01	0.00371475954667503\\
395.01	0.00371476009851026\\
396.01	0.00371476066396858\\
397.01	0.00371476124341027\\
398.01	0.00371476183720637\\
399.01	0.00371476244573904\\
400.01	0.00371476306940211\\
401.01	0.00371476370860144\\
402.01	0.00371476436375545\\
403.01	0.00371476503529553\\
404.01	0.0037147657236667\\
405.01	0.00371476642932807\\
406.01	0.00371476715275349\\
407.01	0.00371476789443212\\
408.01	0.00371476865486911\\
409.01	0.00371476943458633\\
410.01	0.0037147702341231\\
411.01	0.00371477105403691\\
412.01	0.00371477189490436\\
413.01	0.00371477275732197\\
414.01	0.00371477364190714\\
415.01	0.00371477454929916\\
416.01	0.0037147754801603\\
417.01	0.00371477643517684\\
418.01	0.00371477741506044\\
419.01	0.00371477842054929\\
420.01	0.0037147794524096\\
421.01	0.00371478051143698\\
422.01	0.0037147815984581\\
423.01	0.00371478271433233\\
424.01	0.00371478385995362\\
425.01	0.00371478503625234\\
426.01	0.00371478624419755\\
427.01	0.00371478748479906\\
428.01	0.00371478875911005\\
429.01	0.00371479006822958\\
430.01	0.00371479141330557\\
431.01	0.00371479279553772\\
432.01	0.00371479421618102\\
433.01	0.00371479567654933\\
434.01	0.00371479717801937\\
435.01	0.0037147987220351\\
436.01	0.00371480031011245\\
437.01	0.00371480194384455\\
438.01	0.00371480362490755\\
439.01	0.00371480535506691\\
440.01	0.00371480713618448\\
441.01	0.00371480897022626\\
442.01	0.00371481085927117\\
443.01	0.00371481280552069\\
444.01	0.0037148148113098\\
445.01	0.00371481687911917\\
446.01	0.00371481901158901\\
447.01	0.00371482121153461\\
448.01	0.00371482348196422\\
449.01	0.00371482582609918\\
450.01	0.00371482824739706\\
451.01	0.0037148307495783\\
452.01	0.00371483333665677\\
453.01	0.00371483601297517\\
454.01	0.00371483878324612\\
455.01	0.0037148416525999\\
456.01	0.00371484462664034\\
457.01	0.00371484771150997\\
458.01	0.00371485091396572\\
459.01	0.00371485424146246\\
460.01	0.00371485770222783\\
461.01	0.00371486130531297\\
462.01	0.0037148650610467\\
463.01	0.00371486898466294\\
464.01	0.0037148731137831\\
465.01	0.00371487760858888\\
466.01	0.00371488317942011\\
467.01	0.00371489108398733\\
468.01	0.00371489995627214\\
469.01	0.00371490988535324\\
470.01	0.00371492452648405\\
471.01	0.00371494563636893\\
472.01	0.00371496748996078\\
473.01	0.00371499007825305\\
474.01	0.00371501361808457\\
475.01	0.00371503794990053\\
476.01	0.0037150629365636\\
477.01	0.00371508860183391\\
478.01	0.00371511497205824\\
479.01	0.00371514207539233\\
480.01	0.00371516994198696\\
481.01	0.00371519860419688\\
482.01	0.00371522809681539\\
483.01	0.00371525845733818\\
484.01	0.00371528972626019\\
485.01	0.00371532194740921\\
486.01	0.00371535516831954\\
487.01	0.00371538944064158\\
488.01	0.00371542482057514\\
489.01	0.00371546136937689\\
490.01	0.00371549915411554\\
491.01	0.00371553824844404\\
492.01	0.00371557873381042\\
493.01	0.00371562069954055\\
494.01	0.00371566424386073\\
495.01	0.00371570947533162\\
496.01	0.00371575651433542\\
497.01	0.00371580549579813\\
498.01	0.00371585657969426\\
499.01	0.00371591001637779\\
500.01	0.00371596659823481\\
501.01	0.00371603065683166\\
502.01	0.00371612405325414\\
503.01	0.00371625525780549\\
504.01	0.00371639138932336\\
505.01	0.00371653203759087\\
506.01	0.00371667743572924\\
507.01	0.0037168278377098\\
508.01	0.00371698352124755\\
509.01	0.00371714479123643\\
510.01	0.00371731198394848\\
511.01	0.00371748547293173\\
512.01	0.00371766568240781\\
513.01	0.00371785314674762\\
514.01	0.00371804885784427\\
515.01	0.00371825612386256\\
516.01	0.00371848598834987\\
517.01	0.00371874213320585\\
518.01	0.00371900987173145\\
519.01	0.00371928968602028\\
520.01	0.0037195875617622\\
521.01	0.00371993718974971\\
522.01	0.0037204418181354\\
523.01	0.00372101226720697\\
524.01	0.00372160058198296\\
525.01	0.00372220768405245\\
526.01	0.00372283457293392\\
527.01	0.00372348233490491\\
528.01	0.00372415215365289\\
529.01	0.00372484532507435\\
530.01	0.00372556327076363\\
531.01	0.00372630756874893\\
532.01	0.00372708007265731\\
533.01	0.00372788341169996\\
534.01	0.00372872200576432\\
535.01	0.00372959997164683\\
536.01	0.0037305335956814\\
537.01	0.00373168245468962\\
538.01	0.00373350854393371\\
539.01	0.00373548258913944\\
540.01	0.0037376578577701\\
541.01	0.00374045005175643\\
542.01	0.00374392631121988\\
543.01	0.00374882512298112\\
544.01	0.00375488044447884\\
545.01	0.00376117865772368\\
546.01	0.0037677422988858\\
547.01	0.00377460456468436\\
548.01	0.00378185627663242\\
549.01	0.00378982490344149\\
550.01	0.00380080429530036\\
551.01	0.00381524220095096\\
552.01	0.00383022462797278\\
553.01	0.00384579394090635\\
554.01	0.00386199774815606\\
555.01	0.00387888985441291\\
556.01	0.00389653211926337\\
557.01	0.00391500268978535\\
558.01	0.00393445765712839\\
559.01	0.00395568688045589\\
560.01	0.00398598281189649\\
561.01	0.00402999950325096\\
562.01	0.00407816099412708\\
563.01	0.00412824187571079\\
564.01	0.00417993576366653\\
565.01	0.00423331976729216\\
566.01	0.0042886911189915\\
567.01	0.00434623726267961\\
568.01	0.00440615157857568\\
569.01	0.00446865264722467\\
570.01	0.00453398872948427\\
571.01	0.00460251852537614\\
572.01	0.00467574118070142\\
573.01	0.00476776327846847\\
574.01	0.00491794626528909\\
575.01	0.00507467824241099\\
576.01	0.00523528193918925\\
577.01	0.005399916221415\\
578.01	0.00556854809725973\\
579.01	0.00574142821143175\\
580.01	0.00591889087949194\\
581.01	0.00610131589583468\\
582.01	0.00628913280226145\\
583.01	0.00648277347046916\\
584.01	0.00668248777362501\\
585.01	0.00691013716772029\\
586.01	0.00707283502695298\\
587.01	0.00721101595543212\\
588.01	0.00735296580304815\\
589.01	0.00749881702190324\\
590.01	0.00764870538441459\\
591.01	0.0078027577931949\\
592.01	0.00796109156594893\\
593.01	0.0081238057137617\\
594.01	0.00829097637799254\\
595.01	0.00846268644848632\\
596.01	0.00863938267011223\\
597.01	0.00882513249561383\\
598.01	0.00905588119071764\\
599.01	0.00958729251118108\\
599.02	0.00959623629274237\\
599.03	0.00960526138104788\\
599.04	0.00961436620037015\\
599.05	0.00962354898352098\\
599.06	0.00963280776000137\\
599.07	0.00964214034350163\\
599.08	0.00965154431871596\\
599.09	0.00966101702743224\\
599.1	0.00967055555386263\\
599.11	0.00968015670917047\\
599.12	0.00968981701514841\\
599.13	0.00969953268700591\\
599.14	0.00970929961521607\\
599.15	0.0097191133463711\\
599.16	0.00972896906299274\\
599.17	0.00973886156224116\\
599.18	0.00974878523346277\\
599.19	0.0097587340345143\\
599.2	0.00976870146679697\\
599.21	0.00977867955130989\\
599.22	0.00978865949017655\\
599.23	0.00979863316102171\\
599.24	0.00980859186698365\\
599.25	0.00981852514805581\\
599.26	0.0098284189602883\\
599.27	0.00983826246151365\\
599.28	0.00984803656229261\\
599.29	0.00985772626247611\\
599.3	0.00986730605691319\\
599.31	0.00987675003622819\\
599.32	0.00988604182000195\\
599.33	0.00989516400617934\\
599.34	0.00990409812080403\\
599.35	0.00991282455939869\\
599.36	0.00992132252525996\\
599.37	0.00992956996451327\\
599.38	0.00993754349776543\\
599.39	0.00994521834818632\\
599.4	0.00995256826584351\\
599.41	0.00995956544810689\\
599.42	0.00996614822627669\\
599.43	0.00997226774932304\\
599.44	0.00997788884364375\\
599.45	0.00998297432653716\\
599.46	0.00998748490019112\\
599.47	0.00999137904040232\\
599.48	0.00999461287979794\\
599.49	0.00999714147250411\\
599.5	0.00999891661491674\\
599.51	0.00999988677912612\\
599.52	0.01\\
599.53	0.01\\
599.54	0.01\\
599.55	0.01\\
599.56	0.01\\
599.57	0.01\\
599.58	0.01\\
599.59	0.01\\
599.6	0.01\\
599.61	0.01\\
599.62	0.01\\
599.63	0.01\\
599.64	0.01\\
599.65	0.01\\
599.66	0.01\\
599.67	0.01\\
599.68	0.01\\
599.69	0.01\\
599.7	0.01\\
599.71	0.01\\
599.72	0.01\\
599.73	0.01\\
599.74	0.01\\
599.75	0.01\\
599.76	0.01\\
599.77	0.01\\
599.78	0.01\\
599.79	0.01\\
599.8	0.01\\
599.81	0.01\\
599.82	0.01\\
599.83	0.01\\
599.84	0.01\\
599.85	0.01\\
599.86	0.01\\
599.87	0.01\\
599.88	0.01\\
599.89	0.01\\
599.9	0.01\\
599.91	0.01\\
599.92	0.01\\
599.93	0.01\\
599.94	0.01\\
599.95	0.01\\
599.96	0.01\\
599.97	0.01\\
599.98	0.01\\
599.99	0.01\\
600	0.01\\
};
\addplot [color=mycolor13,solid,forget plot]
  table[row sep=crcr]{%
0.01	0\\
1.01	0\\
2.01	0\\
3.01	0\\
4.01	0\\
5.01	0\\
6.01	0\\
7.01	0\\
8.01	0\\
9.01	0\\
10.01	0\\
11.01	0\\
12.01	0\\
13.01	0\\
14.01	0\\
15.01	0\\
16.01	0\\
17.01	0\\
18.01	0\\
19.01	0\\
20.01	0\\
21.01	0\\
22.01	0\\
23.01	0\\
24.01	0\\
25.01	0\\
26.01	0\\
27.01	0\\
28.01	0\\
29.01	0\\
30.01	0\\
31.01	0\\
32.01	0\\
33.01	0\\
34.01	0\\
35.01	0\\
36.01	0\\
37.01	0\\
38.01	0\\
39.01	0\\
40.01	0\\
41.01	0\\
42.01	0\\
43.01	0\\
44.01	0\\
45.01	0\\
46.01	0\\
47.01	0\\
48.01	0\\
49.01	0\\
50.01	0\\
51.01	0\\
52.01	0\\
53.01	0\\
54.01	0\\
55.01	0\\
56.01	0\\
57.01	0\\
58.01	0\\
59.01	0\\
60.01	0\\
61.01	0\\
62.01	0\\
63.01	0\\
64.01	0\\
65.01	0\\
66.01	0\\
67.01	0\\
68.01	0\\
69.01	0\\
70.01	0\\
71.01	0\\
72.01	0\\
73.01	0\\
74.01	0\\
75.01	0\\
76.01	0\\
77.01	0\\
78.01	0\\
79.01	0\\
80.01	0\\
81.01	0\\
82.01	0\\
83.01	0\\
84.01	0\\
85.01	0\\
86.01	0\\
87.01	0\\
88.01	0\\
89.01	0\\
90.01	0\\
91.01	0\\
92.01	0\\
93.01	0\\
94.01	0\\
95.01	0\\
96.01	0\\
97.01	0\\
98.01	0\\
99.01	0\\
100.01	0\\
101.01	0\\
102.01	0\\
103.01	0\\
104.01	0\\
105.01	0\\
106.01	0\\
107.01	0\\
108.01	0\\
109.01	0\\
110.01	0\\
111.01	0\\
112.01	0\\
113.01	0\\
114.01	0\\
115.01	0\\
116.01	0\\
117.01	0\\
118.01	0\\
119.01	0\\
120.01	0\\
121.01	0\\
122.01	0\\
123.01	0\\
124.01	0\\
125.01	0\\
126.01	0\\
127.01	0\\
128.01	0\\
129.01	0\\
130.01	0\\
131.01	0\\
132.01	0\\
133.01	0\\
134.01	0\\
135.01	0\\
136.01	0\\
137.01	0\\
138.01	0\\
139.01	0\\
140.01	0\\
141.01	0\\
142.01	0\\
143.01	0\\
144.01	0\\
145.01	0\\
146.01	0\\
147.01	0\\
148.01	0\\
149.01	0\\
150.01	0\\
151.01	0\\
152.01	0\\
153.01	0\\
154.01	0\\
155.01	0\\
156.01	0\\
157.01	0\\
158.01	0\\
159.01	0\\
160.01	0\\
161.01	0\\
162.01	0\\
163.01	0\\
164.01	0\\
165.01	0\\
166.01	0\\
167.01	0\\
168.01	0\\
169.01	0\\
170.01	0\\
171.01	0\\
172.01	0\\
173.01	0\\
174.01	0\\
175.01	0\\
176.01	0\\
177.01	0\\
178.01	0\\
179.01	0\\
180.01	0\\
181.01	0\\
182.01	0\\
183.01	0\\
184.01	0\\
185.01	0\\
186.01	0\\
187.01	0\\
188.01	0\\
189.01	0\\
190.01	0\\
191.01	0\\
192.01	0\\
193.01	0\\
194.01	0\\
195.01	0\\
196.01	0\\
197.01	0\\
198.01	0\\
199.01	0\\
200.01	0\\
201.01	0\\
202.01	0\\
203.01	0\\
204.01	0\\
205.01	0\\
206.01	0\\
207.01	0\\
208.01	0\\
209.01	0\\
210.01	0\\
211.01	0\\
212.01	0\\
213.01	0\\
214.01	0\\
215.01	0\\
216.01	0\\
217.01	0\\
218.01	0\\
219.01	0\\
220.01	0\\
221.01	0\\
222.01	0\\
223.01	0\\
224.01	0\\
225.01	0\\
226.01	0\\
227.01	0\\
228.01	0\\
229.01	0\\
230.01	0\\
231.01	0\\
232.01	0\\
233.01	0\\
234.01	0\\
235.01	0\\
236.01	0\\
237.01	0\\
238.01	0\\
239.01	0\\
240.01	0\\
241.01	0\\
242.01	0\\
243.01	0\\
244.01	0\\
245.01	0\\
246.01	0\\
247.01	0\\
248.01	0\\
249.01	0\\
250.01	0\\
251.01	0\\
252.01	0\\
253.01	0\\
254.01	0\\
255.01	0\\
256.01	0\\
257.01	0\\
258.01	0\\
259.01	0\\
260.01	0\\
261.01	0\\
262.01	0\\
263.01	0\\
264.01	0\\
265.01	0\\
266.01	0\\
267.01	0\\
268.01	0\\
269.01	0\\
270.01	0\\
271.01	0\\
272.01	0\\
273.01	0\\
274.01	0\\
275.01	0\\
276.01	0\\
277.01	0\\
278.01	0\\
279.01	0\\
280.01	0\\
281.01	0\\
282.01	0\\
283.01	0\\
284.01	0\\
285.01	0\\
286.01	0\\
287.01	0\\
288.01	0\\
289.01	0\\
290.01	0\\
291.01	0\\
292.01	0\\
293.01	0\\
294.01	0\\
295.01	0\\
296.01	0\\
297.01	0\\
298.01	0\\
299.01	0\\
300.01	0\\
301.01	0\\
302.01	0\\
303.01	0\\
304.01	0\\
305.01	0\\
306.01	0\\
307.01	0\\
308.01	0\\
309.01	0\\
310.01	0\\
311.01	0\\
312.01	0\\
313.01	0\\
314.01	0\\
315.01	0\\
316.01	0\\
317.01	0\\
318.01	0\\
319.01	0\\
320.01	0\\
321.01	0\\
322.01	0\\
323.01	0\\
324.01	0\\
325.01	0\\
326.01	0\\
327.01	0\\
328.01	0\\
329.01	0\\
330.01	0\\
331.01	0\\
332.01	0\\
333.01	0\\
334.01	0\\
335.01	0\\
336.01	0\\
337.01	0\\
338.01	0\\
339.01	0\\
340.01	0\\
341.01	0\\
342.01	0\\
343.01	0\\
344.01	0\\
345.01	0\\
346.01	0\\
347.01	0\\
348.01	0\\
349.01	0\\
350.01	0\\
351.01	0\\
352.01	0\\
353.01	0\\
354.01	0\\
355.01	0\\
356.01	0\\
357.01	0\\
358.01	0\\
359.01	0\\
360.01	0\\
361.01	0\\
362.01	0\\
363.01	0\\
364.01	0\\
365.01	0\\
366.01	0\\
367.01	0\\
368.01	0\\
369.01	0\\
370.01	0\\
371.01	0\\
372.01	0\\
373.01	0\\
374.01	0\\
375.01	0\\
376.01	0\\
377.01	0\\
378.01	0\\
379.01	0\\
380.01	0\\
381.01	0\\
382.01	0\\
383.01	0\\
384.01	0\\
385.01	0\\
386.01	0\\
387.01	0\\
388.01	0\\
389.01	0\\
390.01	0\\
391.01	0\\
392.01	0\\
393.01	0\\
394.01	0\\
395.01	0\\
396.01	0\\
397.01	0\\
398.01	0\\
399.01	0\\
400.01	0\\
401.01	0\\
402.01	0\\
403.01	0\\
404.01	0\\
405.01	0\\
406.01	0\\
407.01	0\\
408.01	0\\
409.01	0\\
410.01	0\\
411.01	0\\
412.01	0\\
413.01	0\\
414.01	0\\
415.01	0\\
416.01	0\\
417.01	0\\
418.01	0\\
419.01	0\\
420.01	0\\
421.01	0\\
422.01	0\\
423.01	0\\
424.01	0\\
425.01	0\\
426.01	0\\
427.01	0\\
428.01	0\\
429.01	0\\
430.01	0\\
431.01	0\\
432.01	0\\
433.01	0\\
434.01	0\\
435.01	0\\
436.01	0\\
437.01	0\\
438.01	0\\
439.01	0\\
440.01	0\\
441.01	0\\
442.01	0\\
443.01	0\\
444.01	0\\
445.01	0\\
446.01	0\\
447.01	0\\
448.01	0\\
449.01	0\\
450.01	0\\
451.01	0\\
452.01	0\\
453.01	0\\
454.01	0\\
455.01	0\\
456.01	0\\
457.01	0\\
458.01	0\\
459.01	0\\
460.01	0\\
461.01	0\\
462.01	0\\
463.01	0\\
464.01	0\\
465.01	0\\
466.01	0\\
467.01	0\\
468.01	0\\
469.01	0\\
470.01	0\\
471.01	0\\
472.01	0\\
473.01	0\\
474.01	0\\
475.01	0\\
476.01	0\\
477.01	0\\
478.01	0\\
479.01	0\\
480.01	0\\
481.01	0\\
482.01	0\\
483.01	0\\
484.01	0\\
485.01	0\\
486.01	0\\
487.01	0\\
488.01	0\\
489.01	0\\
490.01	0\\
491.01	0\\
492.01	0\\
493.01	0\\
494.01	0\\
495.01	0\\
496.01	0\\
497.01	0\\
498.01	0\\
499.01	0\\
500.01	0\\
501.01	0\\
502.01	0\\
503.01	0\\
504.01	0\\
505.01	0\\
506.01	0\\
507.01	0\\
508.01	0\\
509.01	0\\
510.01	0\\
511.01	0\\
512.01	0\\
513.01	0\\
514.01	0\\
515.01	0\\
516.01	0\\
517.01	0\\
518.01	0\\
519.01	0\\
520.01	0\\
521.01	0\\
522.01	0\\
523.01	0\\
524.01	0\\
525.01	0\\
526.01	0\\
527.01	0\\
528.01	0\\
529.01	0\\
530.01	0\\
531.01	0\\
532.01	0\\
533.01	0\\
534.01	0\\
535.01	0\\
536.01	0\\
537.01	0\\
538.01	0\\
539.01	0\\
540.01	0\\
541.01	0\\
542.01	0\\
543.01	0\\
544.01	0\\
545.01	0\\
546.01	0\\
547.01	0\\
548.01	0\\
549.01	0\\
550.01	0\\
551.01	0\\
552.01	0\\
553.01	0\\
554.01	0\\
555.01	0\\
556.01	0\\
557.01	0\\
558.01	0\\
559.01	0\\
560.01	0\\
561.01	0\\
562.01	0\\
563.01	0\\
564.01	0\\
565.01	0\\
566.01	0\\
567.01	0\\
568.01	0\\
569.01	0\\
570.01	0\\
571.01	0\\
572.01	0\\
573.01	0\\
574.01	0\\
575.01	0\\
576.01	0\\
577.01	0\\
578.01	0\\
579.01	0\\
580.01	0\\
581.01	0\\
582.01	0\\
583.01	0\\
584.01	0\\
585.01	0\\
586.01	0.000175800090626798\\
587.01	0.000388923417000744\\
588.01	0.000607581801831777\\
589.01	0.000832227384632539\\
590.01	0.00106339227312889\\
591.01	0.00130168499733363\\
592.01	0.00154780821626582\\
593.01	0.00180257534930202\\
594.01	0.00206693981388568\\
595.01	0.00234207488850763\\
596.01	0.00262980748550698\\
597.01	0.00293601233450368\\
598.01	0.00330057550686641\\
599.01	0.00409089741288224\\
599.02	0.00410624158337643\\
599.03	0.00412187629047293\\
599.04	0.0041378089664602\\
599.05	0.00415404725346514\\
599.06	0.0041705990101973\\
599.07	0.00418747231894723\\
599.08	0.00420467549285039\\
599.09	0.00422221708342868\\
599.1	0.00424010588842206\\
599.11	0.0042583509599236\\
599.12	0.00427696161283197\\
599.13	0.00429594743363603\\
599.14	0.0043153182895472\\
599.15	0.00433508433799597\\
599.16	0.00435525603650987\\
599.17	0.00437584415299123\\
599.18	0.00439685977641415\\
599.19	0.00441831432796106\\
599.2	0.00444021957262047\\
599.21	0.00446258763268925\\
599.22	0.00448543100083088\\
599.23	0.00450876255151259\\
599.24	0.00453259555470182\\
599.25	0.00455694369179364\\
599.26	0.00458182107472543\\
599.27	0.00460724225617071\\
599.28	0.00463322225645973\\
599.29	0.00465977657394944\\
599.3	0.00468692121658179\\
599.31	0.0047146727202229\\
599.32	0.00474304815378429\\
599.33	0.0047720651397426\\
599.34	0.00480174187561087\\
599.35	0.00483209715642481\\
599.36	0.00486315039830252\\
599.37	0.00489492166313947\\
599.38	0.0049274316845055\\
599.39	0.00496070189481463\\
599.4	0.00499475445384406\\
599.41	0.00502961227868348\\
599.42	0.00506529912031843\\
599.43	0.00510183957745583\\
599.44	0.00513925911030819\\
599.45	0.00517758407750282\\
599.46	0.00521684177494776\\
599.47	0.00525706047678138\\
599.48	0.00529826947854203\\
599.49	0.00534049779916455\\
599.5	0.00538377625915345\\
599.51	0.00542813744455676\\
599.52	0.00547361280337367\\
599.53	0.0055202340564185\\
599.54	0.00556803008723294\\
599.55	0.00561703167916341\\
599.56	0.00566725869375703\\
599.57	0.00571874907924545\\
599.58	0.00577154219388792\\
599.59	0.00582567887207813\\
599.6	0.00588120149439699\\
599.61	0.0059381540619025\\
599.62	0.00599655232226517\\
599.63	0.00605643298747278\\
599.64	0.0061178440982782\\
599.65	0.00618083560576154\\
599.66	0.00624545946877155\\
599.67	0.00631176975769696\\
599.68	0.00637982276511079\\
599.69	0.00644967712385027\\
599.7	0.00652139393315005\\
599.71	0.00659503689350761\\
599.72	0.00667067245102825\\
599.73	0.00674836995207315\\
599.74	0.00682820180916178\\
599.75	0.00691024367906377\\
599.76	0.00699457465421154\\
599.77	0.00708127746867573\\
599.78	0.00717043872007276\\
599.79	0.00726214910894212\\
599.8	0.00735650369726444\\
599.81	0.00745360218802786\\
599.82	0.00755354922795706\\
599.83	0.00765645473577685\\
599.84	0.00776243425867489\\
599.85	0.0078716093599577\\
599.86	0.00798410804127475\\
599.87	0.00810006520322131\\
599.88	0.00821962314863375\\
599.89	0.00834293213347036\\
599.9	0.00847015097084234\\
599.91	0.00860144769453793\\
599.92	0.00873700028928946\\
599.93	0.00887699749609073\\
599.94	0.00902163970211132\\
599.95	0.00917113992620961\\
599.96	0.00932572491276224\\
599.97	0.00948563634855679\\
599.98	0.00965113221990432\\
599.99	0.00982248833000031\\
600	0.01\\
};
\addplot [color=mycolor14,solid,forget plot]
  table[row sep=crcr]{%
0.01	0.01\\
1.01	0.01\\
2.01	0.01\\
3.01	0.01\\
4.01	0.01\\
5.01	0.01\\
6.01	0.01\\
7.01	0.01\\
8.01	0.01\\
9.01	0.01\\
10.01	0.01\\
11.01	0.01\\
12.01	0.01\\
13.01	0.01\\
14.01	0.01\\
15.01	0.01\\
16.01	0.01\\
17.01	0.01\\
18.01	0.01\\
19.01	0.01\\
20.01	0.01\\
21.01	0.01\\
22.01	0.01\\
23.01	0.01\\
24.01	0.01\\
25.01	0.01\\
26.01	0.01\\
27.01	0.01\\
28.01	0.01\\
29.01	0.01\\
30.01	0.01\\
31.01	0.01\\
32.01	0.01\\
33.01	0.01\\
34.01	0.01\\
35.01	0.01\\
36.01	0.01\\
37.01	0.01\\
38.01	0.01\\
39.01	0.01\\
40.01	0.01\\
41.01	0.01\\
42.01	0.01\\
43.01	0.01\\
44.01	0.01\\
45.01	0.01\\
46.01	0.01\\
47.01	0.01\\
48.01	0.01\\
49.01	0.01\\
50.01	0.01\\
51.01	0.01\\
52.01	0.01\\
53.01	0.01\\
54.01	0.01\\
55.01	0.01\\
56.01	0.01\\
57.01	0.01\\
58.01	0.01\\
59.01	0.01\\
60.01	0.01\\
61.01	0.01\\
62.01	0.01\\
63.01	0.01\\
64.01	0.01\\
65.01	0.01\\
66.01	0.01\\
67.01	0.01\\
68.01	0.01\\
69.01	0.01\\
70.01	0.01\\
71.01	0.01\\
72.01	0.01\\
73.01	0.01\\
74.01	0.01\\
75.01	0.01\\
76.01	0.01\\
77.01	0.01\\
78.01	0.01\\
79.01	0.01\\
80.01	0.01\\
81.01	0.01\\
82.01	0.01\\
83.01	0.01\\
84.01	0.01\\
85.01	0.01\\
86.01	0.01\\
87.01	0.01\\
88.01	0.01\\
89.01	0.01\\
90.01	0.01\\
91.01	0.01\\
92.01	0.01\\
93.01	0.01\\
94.01	0.01\\
95.01	0.01\\
96.01	0.01\\
97.01	0.01\\
98.01	0.01\\
99.01	0.01\\
100.01	0.01\\
101.01	0.01\\
102.01	0.01\\
103.01	0.01\\
104.01	0.01\\
105.01	0.01\\
106.01	0.01\\
107.01	0.01\\
108.01	0.01\\
109.01	0.01\\
110.01	0.01\\
111.01	0.01\\
112.01	0.01\\
113.01	0.01\\
114.01	0.01\\
115.01	0.01\\
116.01	0.01\\
117.01	0.01\\
118.01	0.01\\
119.01	0.01\\
120.01	0.01\\
121.01	0.01\\
122.01	0.01\\
123.01	0.01\\
124.01	0.01\\
125.01	0.01\\
126.01	0.01\\
127.01	0.01\\
128.01	0.01\\
129.01	0.01\\
130.01	0.01\\
131.01	0.01\\
132.01	0.01\\
133.01	0.01\\
134.01	0.01\\
135.01	0.01\\
136.01	0.01\\
137.01	0.01\\
138.01	0.01\\
139.01	0.01\\
140.01	0.01\\
141.01	0.01\\
142.01	0.01\\
143.01	0.01\\
144.01	0.01\\
145.01	0.01\\
146.01	0.01\\
147.01	0.01\\
148.01	0.01\\
149.01	0.01\\
150.01	0.01\\
151.01	0.01\\
152.01	0.01\\
153.01	0.01\\
154.01	0.01\\
155.01	0.01\\
156.01	0.01\\
157.01	0.01\\
158.01	0.01\\
159.01	0.01\\
160.01	0.01\\
161.01	0.01\\
162.01	0.01\\
163.01	0.01\\
164.01	0.01\\
165.01	0.01\\
166.01	0.01\\
167.01	0.01\\
168.01	0.01\\
169.01	0.01\\
170.01	0.01\\
171.01	0.01\\
172.01	0.01\\
173.01	0.01\\
174.01	0.01\\
175.01	0.01\\
176.01	0.01\\
177.01	0.01\\
178.01	0.01\\
179.01	0.01\\
180.01	0.01\\
181.01	0.01\\
182.01	0.01\\
183.01	0.01\\
184.01	0.01\\
185.01	0.01\\
186.01	0.01\\
187.01	0.01\\
188.01	0.01\\
189.01	0.01\\
190.01	0.01\\
191.01	0.01\\
192.01	0.01\\
193.01	0.01\\
194.01	0.01\\
195.01	0.01\\
196.01	0.01\\
197.01	0.01\\
198.01	0.01\\
199.01	0.01\\
200.01	0.01\\
201.01	0.01\\
202.01	0.01\\
203.01	0.01\\
204.01	0.01\\
205.01	0.01\\
206.01	0.01\\
207.01	0.01\\
208.01	0.01\\
209.01	0.01\\
210.01	0.01\\
211.01	0.01\\
212.01	0.01\\
213.01	0.01\\
214.01	0.01\\
215.01	0.01\\
216.01	0.01\\
217.01	0.01\\
218.01	0.01\\
219.01	0.01\\
220.01	0.01\\
221.01	0.01\\
222.01	0.01\\
223.01	0.01\\
224.01	0.01\\
225.01	0.01\\
226.01	0.01\\
227.01	0.01\\
228.01	0.01\\
229.01	0.01\\
230.01	0.01\\
231.01	0.01\\
232.01	0.01\\
233.01	0.01\\
234.01	0.01\\
235.01	0.01\\
236.01	0.01\\
237.01	0.01\\
238.01	0.01\\
239.01	0.01\\
240.01	0.01\\
241.01	0.01\\
242.01	0.01\\
243.01	0.01\\
244.01	0.01\\
245.01	0.01\\
246.01	0.01\\
247.01	0.01\\
248.01	0.01\\
249.01	0.01\\
250.01	0.01\\
251.01	0.01\\
252.01	0.01\\
253.01	0.01\\
254.01	0.01\\
255.01	0.01\\
256.01	0.01\\
257.01	0.01\\
258.01	0.01\\
259.01	0.01\\
260.01	0.01\\
261.01	0.01\\
262.01	0.01\\
263.01	0.01\\
264.01	0.01\\
265.01	0.01\\
266.01	0.01\\
267.01	0.01\\
268.01	0.01\\
269.01	0.01\\
270.01	0.01\\
271.01	0.01\\
272.01	0.01\\
273.01	0.01\\
274.01	0.01\\
275.01	0.01\\
276.01	0.01\\
277.01	0.01\\
278.01	0.01\\
279.01	0.01\\
280.01	0.01\\
281.01	0.01\\
282.01	0.01\\
283.01	0.01\\
284.01	0.01\\
285.01	0.01\\
286.01	0.01\\
287.01	0.01\\
288.01	0.01\\
289.01	0.01\\
290.01	0.01\\
291.01	0.01\\
292.01	0.01\\
293.01	0.01\\
294.01	0.01\\
295.01	0.01\\
296.01	0.01\\
297.01	0.01\\
298.01	0.01\\
299.01	0.01\\
300.01	0.01\\
301.01	0.01\\
302.01	0.01\\
303.01	0.01\\
304.01	0.01\\
305.01	0.01\\
306.01	0.01\\
307.01	0.01\\
308.01	0.01\\
309.01	0.01\\
310.01	0.01\\
311.01	0.01\\
312.01	0.01\\
313.01	0.01\\
314.01	0.01\\
315.01	0.01\\
316.01	0.01\\
317.01	0.01\\
318.01	0.01\\
319.01	0.01\\
320.01	0.01\\
321.01	0.01\\
322.01	0.01\\
323.01	0.01\\
324.01	0.01\\
325.01	0.01\\
326.01	0.01\\
327.01	0.01\\
328.01	0.01\\
329.01	0.01\\
330.01	0.01\\
331.01	0.01\\
332.01	0.01\\
333.01	0.01\\
334.01	0.01\\
335.01	0.01\\
336.01	0.01\\
337.01	0.01\\
338.01	0.01\\
339.01	0.01\\
340.01	0.01\\
341.01	0.01\\
342.01	0.01\\
343.01	0.01\\
344.01	0.01\\
345.01	0.01\\
346.01	0.01\\
347.01	0.01\\
348.01	0.01\\
349.01	0.01\\
350.01	0.01\\
351.01	0.01\\
352.01	0.01\\
353.01	0.01\\
354.01	0.01\\
355.01	0.01\\
356.01	0.01\\
357.01	0.01\\
358.01	0.01\\
359.01	0.01\\
360.01	0.01\\
361.01	0.01\\
362.01	0.01\\
363.01	0.01\\
364.01	0.01\\
365.01	0.01\\
366.01	0.01\\
367.01	0.01\\
368.01	0.01\\
369.01	0.01\\
370.01	0.01\\
371.01	0.01\\
372.01	0.01\\
373.01	0.01\\
374.01	0.01\\
375.01	0.01\\
376.01	0.01\\
377.01	0.01\\
378.01	0.01\\
379.01	0.01\\
380.01	0.01\\
381.01	0.01\\
382.01	0.01\\
383.01	0.01\\
384.01	0.01\\
385.01	0.01\\
386.01	0.01\\
387.01	0.01\\
388.01	0.01\\
389.01	0.01\\
390.01	0.01\\
391.01	0.01\\
392.01	0.01\\
393.01	0.01\\
394.01	0.01\\
395.01	0.01\\
396.01	0.01\\
397.01	0.01\\
398.01	0.01\\
399.01	0.01\\
400.01	0.01\\
401.01	0.01\\
402.01	0.01\\
403.01	0.01\\
404.01	0.01\\
405.01	0.01\\
406.01	0.01\\
407.01	0.01\\
408.01	0.01\\
409.01	0.01\\
410.01	0.01\\
411.01	0.01\\
412.01	0.01\\
413.01	0.01\\
414.01	0.01\\
415.01	0.01\\
416.01	0.01\\
417.01	0.01\\
418.01	0.01\\
419.01	0.01\\
420.01	0.01\\
421.01	0.01\\
422.01	0.01\\
423.01	0.01\\
424.01	0.01\\
425.01	0.01\\
426.01	0.01\\
427.01	0.01\\
428.01	0.01\\
429.01	0.01\\
430.01	0.01\\
431.01	0.01\\
432.01	0.01\\
433.01	0.01\\
434.01	0.01\\
435.01	0.01\\
436.01	0.01\\
437.01	0.01\\
438.01	0.01\\
439.01	0.01\\
440.01	0.01\\
441.01	0.01\\
442.01	0.01\\
443.01	0.01\\
444.01	0.01\\
445.01	0.01\\
446.01	0.01\\
447.01	0.01\\
448.01	0.01\\
449.01	0.01\\
450.01	0.01\\
451.01	0.01\\
452.01	0.01\\
453.01	0.01\\
454.01	0.01\\
455.01	0.01\\
456.01	0.01\\
457.01	0.01\\
458.01	0.01\\
459.01	0.01\\
460.01	0.01\\
461.01	0.01\\
462.01	0.01\\
463.01	0.01\\
464.01	0.01\\
465.01	0.01\\
466.01	0.01\\
467.01	0.01\\
468.01	0.01\\
469.01	0.01\\
470.01	0.01\\
471.01	0.01\\
472.01	0.01\\
473.01	0.01\\
474.01	0.01\\
475.01	0.01\\
476.01	0.01\\
477.01	0.01\\
478.01	0.01\\
479.01	0.01\\
480.01	0.01\\
481.01	0.01\\
482.01	0.01\\
483.01	0.01\\
484.01	0.01\\
485.01	0.01\\
486.01	0.01\\
487.01	0.01\\
488.01	0.01\\
489.01	0.01\\
490.01	0.01\\
491.01	0.01\\
492.01	0.01\\
493.01	0.01\\
494.01	0.01\\
495.01	0.01\\
496.01	0.01\\
497.01	0.01\\
498.01	0.01\\
499.01	0.01\\
500.01	0.01\\
501.01	0.01\\
502.01	0.01\\
503.01	0.01\\
504.01	0.01\\
505.01	0.01\\
506.01	0.01\\
507.01	0.01\\
508.01	0.01\\
509.01	0.01\\
510.01	0.01\\
511.01	0.01\\
512.01	0.01\\
513.01	0.01\\
514.01	0.01\\
515.01	0.01\\
516.01	0.01\\
517.01	0.01\\
518.01	0.01\\
519.01	0.01\\
520.01	0.01\\
521.01	0.01\\
522.01	0.01\\
523.01	0.01\\
524.01	0.01\\
525.01	0.01\\
526.01	0.01\\
527.01	0.01\\
528.01	0.01\\
529.01	0.01\\
530.01	0.01\\
531.01	0.01\\
532.01	0.01\\
533.01	0.01\\
534.01	0.01\\
535.01	0.01\\
536.01	0.01\\
537.01	0.01\\
538.01	0.01\\
539.01	0.01\\
540.01	0.01\\
541.01	0.01\\
542.01	0.01\\
543.01	0.01\\
544.01	0.01\\
545.01	0.01\\
546.01	0.01\\
547.01	0.01\\
548.01	0.01\\
549.01	0.01\\
550.01	0.01\\
551.01	0.01\\
552.01	0.01\\
553.01	0.01\\
554.01	0.01\\
555.01	0.01\\
556.01	0.01\\
557.01	0.01\\
558.01	0.01\\
559.01	0.01\\
560.01	0.01\\
561.01	0.01\\
562.01	0.01\\
563.01	0.01\\
564.01	0.01\\
565.01	0.01\\
566.01	0.01\\
567.01	0.01\\
568.01	0.01\\
569.01	0.01\\
570.01	0.01\\
571.01	0.01\\
572.01	0.01\\
573.01	0.01\\
574.01	0.01\\
575.01	0.01\\
576.01	0.01\\
577.01	0.01\\
578.01	0.01\\
579.01	0.01\\
580.01	0.01\\
581.01	0.01\\
582.01	0.01\\
583.01	0.01\\
584.01	0.01\\
585.01	0.01\\
586.01	0.01\\
587.01	0.01\\
588.01	0.01\\
589.01	0.01\\
590.01	0.01\\
591.01	0.01\\
592.01	0.01\\
593.01	0.01\\
594.01	0.01\\
595.01	0.01\\
596.01	0.01\\
597.01	0.01\\
598.01	0.01\\
599.01	0.01\\
599.02	0.01\\
599.03	0.01\\
599.04	0.01\\
599.05	0.01\\
599.06	0.01\\
599.07	0.01\\
599.08	0.01\\
599.09	0.01\\
599.1	0.01\\
599.11	0.01\\
599.12	0.01\\
599.13	0.01\\
599.14	0.01\\
599.15	0.01\\
599.16	0.01\\
599.17	0.01\\
599.18	0.01\\
599.19	0.01\\
599.2	0.01\\
599.21	0.01\\
599.22	0.01\\
599.23	0.01\\
599.24	0.01\\
599.25	0.01\\
599.26	0.01\\
599.27	0.01\\
599.28	0.01\\
599.29	0.01\\
599.3	0.01\\
599.31	0.01\\
599.32	0.01\\
599.33	0.01\\
599.34	0.01\\
599.35	0.01\\
599.36	0.01\\
599.37	0.01\\
599.38	0.01\\
599.39	0.01\\
599.4	0.01\\
599.41	0.01\\
599.42	0.01\\
599.43	0.01\\
599.44	0.01\\
599.45	0.01\\
599.46	0.01\\
599.47	0.01\\
599.48	0.01\\
599.49	0.00989153793223401\\
599.5	0.00972504357100398\\
599.51	0.00955769558107442\\
599.52	0.00938948342288605\\
599.53	0.0092204021322092\\
599.54	0.00905044766612316\\
599.55	0.00887960877732616\\
599.56	0.00870787409334322\\
599.57	0.00853523207011476\\
599.58	0.0083616703641326\\
599.59	0.00818717640695089\\
599.6	0.00801173706697357\\
599.61	0.00783533878810403\\
599.62	0.0076579675834383\\
599.63	0.00747960902168377\\
599.64	0.00730024821313421\\
599.65	0.00711986979518157\\
599.66	0.00693845791734417\\
599.67	0.00675599622540386\\
599.68	0.00657246784521872\\
599.69	0.00638785536611911\\
599.7	0.0062021408237225\\
599.71	0.0060153056821398\\
599.72	0.00582733081357053\\
599.73	0.00563819647474965\\
599.74	0.00544788228619198\\
599.75	0.00525636721096807\\
599.76	0.00506362953138433\\
599.77	0.00486964682392524\\
599.78	0.00467439593512126\\
599.79	0.0044778529564409\\
599.8	0.00427999319815484\\
599.81	0.00408079116211723\\
599.82	0.00388022051340578\\
599.83	0.00367825405075887\\
599.84	0.00347486367574443\\
599.85	0.00327002036059148\\
599.86	0.00306369411461145\\
599.87	0.00285585394913248\\
599.88	0.00264646784086591\\
599.89	0.00243550269362013\\
599.9	0.00222292429827314\\
599.91	0.00200869729091123\\
599.92	0.00179278510903777\\
599.93	0.00157514994575309\\
599.94	0.00135575270180365\\
599.95	0.00113455293539763\\
599.96	0.000911508809683114\\
599.97	0.000686577037787083\\
599.98	0.000459712825316519\\
599.99	0.000230869810230147\\
600	0\\
};
\addplot [color=mycolor15,solid,forget plot]
  table[row sep=crcr]{%
0.01	0.01\\
1.01	0.01\\
2.01	0.01\\
3.01	0.01\\
4.01	0.01\\
5.01	0.01\\
6.01	0.01\\
7.01	0.01\\
8.01	0.01\\
9.01	0.01\\
10.01	0.01\\
11.01	0.01\\
12.01	0.01\\
13.01	0.01\\
14.01	0.01\\
15.01	0.01\\
16.01	0.01\\
17.01	0.01\\
18.01	0.01\\
19.01	0.01\\
20.01	0.01\\
21.01	0.01\\
22.01	0.01\\
23.01	0.01\\
24.01	0.01\\
25.01	0.01\\
26.01	0.01\\
27.01	0.01\\
28.01	0.01\\
29.01	0.01\\
30.01	0.01\\
31.01	0.01\\
32.01	0.01\\
33.01	0.01\\
34.01	0.01\\
35.01	0.01\\
36.01	0.01\\
37.01	0.01\\
38.01	0.01\\
39.01	0.01\\
40.01	0.01\\
41.01	0.01\\
42.01	0.01\\
43.01	0.01\\
44.01	0.01\\
45.01	0.01\\
46.01	0.01\\
47.01	0.01\\
48.01	0.01\\
49.01	0.01\\
50.01	0.01\\
51.01	0.01\\
52.01	0.01\\
53.01	0.01\\
54.01	0.01\\
55.01	0.01\\
56.01	0.01\\
57.01	0.01\\
58.01	0.01\\
59.01	0.01\\
60.01	0.01\\
61.01	0.01\\
62.01	0.01\\
63.01	0.01\\
64.01	0.01\\
65.01	0.01\\
66.01	0.01\\
67.01	0.01\\
68.01	0.01\\
69.01	0.01\\
70.01	0.01\\
71.01	0.01\\
72.01	0.01\\
73.01	0.01\\
74.01	0.01\\
75.01	0.01\\
76.01	0.01\\
77.01	0.01\\
78.01	0.01\\
79.01	0.01\\
80.01	0.01\\
81.01	0.01\\
82.01	0.01\\
83.01	0.01\\
84.01	0.01\\
85.01	0.01\\
86.01	0.01\\
87.01	0.01\\
88.01	0.01\\
89.01	0.01\\
90.01	0.01\\
91.01	0.01\\
92.01	0.01\\
93.01	0.01\\
94.01	0.01\\
95.01	0.01\\
96.01	0.01\\
97.01	0.01\\
98.01	0.01\\
99.01	0.01\\
100.01	0.01\\
101.01	0.01\\
102.01	0.01\\
103.01	0.01\\
104.01	0.01\\
105.01	0.01\\
106.01	0.01\\
107.01	0.01\\
108.01	0.01\\
109.01	0.01\\
110.01	0.01\\
111.01	0.01\\
112.01	0.01\\
113.01	0.01\\
114.01	0.01\\
115.01	0.01\\
116.01	0.01\\
117.01	0.01\\
118.01	0.01\\
119.01	0.01\\
120.01	0.01\\
121.01	0.01\\
122.01	0.01\\
123.01	0.01\\
124.01	0.01\\
125.01	0.01\\
126.01	0.01\\
127.01	0.01\\
128.01	0.01\\
129.01	0.01\\
130.01	0.01\\
131.01	0.01\\
132.01	0.01\\
133.01	0.01\\
134.01	0.01\\
135.01	0.01\\
136.01	0.01\\
137.01	0.01\\
138.01	0.01\\
139.01	0.01\\
140.01	0.01\\
141.01	0.01\\
142.01	0.01\\
143.01	0.01\\
144.01	0.01\\
145.01	0.01\\
146.01	0.01\\
147.01	0.01\\
148.01	0.01\\
149.01	0.01\\
150.01	0.01\\
151.01	0.01\\
152.01	0.01\\
153.01	0.01\\
154.01	0.01\\
155.01	0.01\\
156.01	0.01\\
157.01	0.01\\
158.01	0.01\\
159.01	0.01\\
160.01	0.01\\
161.01	0.01\\
162.01	0.01\\
163.01	0.01\\
164.01	0.01\\
165.01	0.01\\
166.01	0.01\\
167.01	0.01\\
168.01	0.01\\
169.01	0.01\\
170.01	0.01\\
171.01	0.01\\
172.01	0.01\\
173.01	0.01\\
174.01	0.01\\
175.01	0.01\\
176.01	0.01\\
177.01	0.01\\
178.01	0.01\\
179.01	0.01\\
180.01	0.01\\
181.01	0.01\\
182.01	0.01\\
183.01	0.01\\
184.01	0.01\\
185.01	0.01\\
186.01	0.01\\
187.01	0.01\\
188.01	0.01\\
189.01	0.01\\
190.01	0.01\\
191.01	0.01\\
192.01	0.01\\
193.01	0.01\\
194.01	0.01\\
195.01	0.01\\
196.01	0.01\\
197.01	0.01\\
198.01	0.01\\
199.01	0.01\\
200.01	0.01\\
201.01	0.01\\
202.01	0.01\\
203.01	0.01\\
204.01	0.01\\
205.01	0.01\\
206.01	0.01\\
207.01	0.01\\
208.01	0.01\\
209.01	0.01\\
210.01	0.01\\
211.01	0.01\\
212.01	0.01\\
213.01	0.01\\
214.01	0.01\\
215.01	0.01\\
216.01	0.01\\
217.01	0.01\\
218.01	0.01\\
219.01	0.01\\
220.01	0.01\\
221.01	0.01\\
222.01	0.01\\
223.01	0.01\\
224.01	0.01\\
225.01	0.01\\
226.01	0.01\\
227.01	0.01\\
228.01	0.01\\
229.01	0.01\\
230.01	0.01\\
231.01	0.01\\
232.01	0.01\\
233.01	0.01\\
234.01	0.01\\
235.01	0.01\\
236.01	0.01\\
237.01	0.01\\
238.01	0.01\\
239.01	0.01\\
240.01	0.01\\
241.01	0.01\\
242.01	0.01\\
243.01	0.01\\
244.01	0.01\\
245.01	0.01\\
246.01	0.01\\
247.01	0.01\\
248.01	0.01\\
249.01	0.01\\
250.01	0.01\\
251.01	0.01\\
252.01	0.01\\
253.01	0.01\\
254.01	0.01\\
255.01	0.01\\
256.01	0.01\\
257.01	0.01\\
258.01	0.01\\
259.01	0.01\\
260.01	0.01\\
261.01	0.01\\
262.01	0.01\\
263.01	0.01\\
264.01	0.01\\
265.01	0.01\\
266.01	0.01\\
267.01	0.01\\
268.01	0.01\\
269.01	0.01\\
270.01	0.01\\
271.01	0.01\\
272.01	0.01\\
273.01	0.01\\
274.01	0.01\\
275.01	0.01\\
276.01	0.01\\
277.01	0.01\\
278.01	0.01\\
279.01	0.01\\
280.01	0.01\\
281.01	0.01\\
282.01	0.01\\
283.01	0.01\\
284.01	0.01\\
285.01	0.01\\
286.01	0.01\\
287.01	0.01\\
288.01	0.01\\
289.01	0.01\\
290.01	0.01\\
291.01	0.01\\
292.01	0.01\\
293.01	0.01\\
294.01	0.01\\
295.01	0.01\\
296.01	0.01\\
297.01	0.01\\
298.01	0.01\\
299.01	0.01\\
300.01	0.01\\
301.01	0.01\\
302.01	0.01\\
303.01	0.01\\
304.01	0.01\\
305.01	0.01\\
306.01	0.01\\
307.01	0.01\\
308.01	0.01\\
309.01	0.01\\
310.01	0.01\\
311.01	0.01\\
312.01	0.01\\
313.01	0.01\\
314.01	0.01\\
315.01	0.01\\
316.01	0.01\\
317.01	0.01\\
318.01	0.01\\
319.01	0.01\\
320.01	0.01\\
321.01	0.01\\
322.01	0.01\\
323.01	0.01\\
324.01	0.01\\
325.01	0.01\\
326.01	0.01\\
327.01	0.01\\
328.01	0.01\\
329.01	0.01\\
330.01	0.01\\
331.01	0.01\\
332.01	0.01\\
333.01	0.01\\
334.01	0.01\\
335.01	0.01\\
336.01	0.01\\
337.01	0.01\\
338.01	0.01\\
339.01	0.01\\
340.01	0.01\\
341.01	0.01\\
342.01	0.01\\
343.01	0.01\\
344.01	0.01\\
345.01	0.01\\
346.01	0.01\\
347.01	0.01\\
348.01	0.01\\
349.01	0.01\\
350.01	0.01\\
351.01	0.01\\
352.01	0.01\\
353.01	0.01\\
354.01	0.01\\
355.01	0.01\\
356.01	0.01\\
357.01	0.01\\
358.01	0.01\\
359.01	0.01\\
360.01	0.01\\
361.01	0.01\\
362.01	0.01\\
363.01	0.01\\
364.01	0.01\\
365.01	0.01\\
366.01	0.01\\
367.01	0.01\\
368.01	0.01\\
369.01	0.01\\
370.01	0.01\\
371.01	0.01\\
372.01	0.01\\
373.01	0.01\\
374.01	0.01\\
375.01	0.01\\
376.01	0.01\\
377.01	0.01\\
378.01	0.01\\
379.01	0.01\\
380.01	0.01\\
381.01	0.01\\
382.01	0.01\\
383.01	0.01\\
384.01	0.01\\
385.01	0.01\\
386.01	0.01\\
387.01	0.01\\
388.01	0.01\\
389.01	0.01\\
390.01	0.01\\
391.01	0.01\\
392.01	0.01\\
393.01	0.01\\
394.01	0.01\\
395.01	0.01\\
396.01	0.01\\
397.01	0.01\\
398.01	0.01\\
399.01	0.01\\
400.01	0.01\\
401.01	0.01\\
402.01	0.01\\
403.01	0.01\\
404.01	0.01\\
405.01	0.01\\
406.01	0.01\\
407.01	0.01\\
408.01	0.01\\
409.01	0.01\\
410.01	0.01\\
411.01	0.01\\
412.01	0.01\\
413.01	0.01\\
414.01	0.01\\
415.01	0.01\\
416.01	0.01\\
417.01	0.01\\
418.01	0.01\\
419.01	0.01\\
420.01	0.01\\
421.01	0.01\\
422.01	0.01\\
423.01	0.01\\
424.01	0.01\\
425.01	0.01\\
426.01	0.01\\
427.01	0.01\\
428.01	0.01\\
429.01	0.01\\
430.01	0.01\\
431.01	0.01\\
432.01	0.01\\
433.01	0.01\\
434.01	0.01\\
435.01	0.01\\
436.01	0.01\\
437.01	0.01\\
438.01	0.01\\
439.01	0.01\\
440.01	0.01\\
441.01	0.01\\
442.01	0.01\\
443.01	0.01\\
444.01	0.01\\
445.01	0.01\\
446.01	0.01\\
447.01	0.01\\
448.01	0.01\\
449.01	0.01\\
450.01	0.01\\
451.01	0.01\\
452.01	0.01\\
453.01	0.01\\
454.01	0.01\\
455.01	0.01\\
456.01	0.01\\
457.01	0.01\\
458.01	0.01\\
459.01	0.01\\
460.01	0.01\\
461.01	0.01\\
462.01	0.01\\
463.01	0.01\\
464.01	0.01\\
465.01	0.01\\
466.01	0.01\\
467.01	0.01\\
468.01	0.01\\
469.01	0.01\\
470.01	0.01\\
471.01	0.01\\
472.01	0.01\\
473.01	0.01\\
474.01	0.01\\
475.01	0.01\\
476.01	0.01\\
477.01	0.01\\
478.01	0.01\\
479.01	0.01\\
480.01	0.01\\
481.01	0.01\\
482.01	0.01\\
483.01	0.01\\
484.01	0.01\\
485.01	0.01\\
486.01	0.01\\
487.01	0.01\\
488.01	0.01\\
489.01	0.01\\
490.01	0.01\\
491.01	0.01\\
492.01	0.01\\
493.01	0.01\\
494.01	0.01\\
495.01	0.01\\
496.01	0.01\\
497.01	0.01\\
498.01	0.01\\
499.01	0.01\\
500.01	0.01\\
501.01	0.01\\
502.01	0.01\\
503.01	0.01\\
504.01	0.01\\
505.01	0.01\\
506.01	0.01\\
507.01	0.01\\
508.01	0.01\\
509.01	0.01\\
510.01	0.01\\
511.01	0.01\\
512.01	0.01\\
513.01	0.01\\
514.01	0.01\\
515.01	0.01\\
516.01	0.01\\
517.01	0.01\\
518.01	0.01\\
519.01	0.01\\
520.01	0.01\\
521.01	0.01\\
522.01	0.01\\
523.01	0.01\\
524.01	0.01\\
525.01	0.01\\
526.01	0.01\\
527.01	0.01\\
528.01	0.01\\
529.01	0.01\\
530.01	0.01\\
531.01	0.01\\
532.01	0.01\\
533.01	0.01\\
534.01	0.01\\
535.01	0.01\\
536.01	0.01\\
537.01	0.01\\
538.01	0.01\\
539.01	0.01\\
540.01	0.01\\
541.01	0.01\\
542.01	0.01\\
543.01	0.01\\
544.01	0.01\\
545.01	0.01\\
546.01	0.01\\
547.01	0.01\\
548.01	0.01\\
549.01	0.01\\
550.01	0.01\\
551.01	0.01\\
552.01	0.01\\
553.01	0.01\\
554.01	0.01\\
555.01	0.01\\
556.01	0.01\\
557.01	0.01\\
558.01	0.01\\
559.01	0.01\\
560.01	0.01\\
561.01	0.01\\
562.01	0.01\\
563.01	0.01\\
564.01	0.01\\
565.01	0.01\\
566.01	0.01\\
567.01	0.01\\
568.01	0.01\\
569.01	0.01\\
570.01	0.01\\
571.01	0.01\\
572.01	0.01\\
573.01	0.01\\
574.01	0.01\\
575.01	0.01\\
576.01	0.01\\
577.01	0.01\\
578.01	0.01\\
579.01	0.01\\
580.01	0.01\\
581.01	0.01\\
582.01	0.01\\
583.01	0.01\\
584.01	0.01\\
585.01	0.01\\
586.01	0.01\\
587.01	0.01\\
588.01	0.01\\
589.01	0.01\\
590.01	0.01\\
591.01	0.01\\
592.01	0.01\\
593.01	0.01\\
594.01	0.01\\
595.01	0.01\\
596.01	0.01\\
597.01	0.01\\
598.01	0.01\\
599.01	0.01\\
599.02	0.01\\
599.03	0.01\\
599.04	0.01\\
599.05	0.01\\
599.06	0.01\\
599.07	0.01\\
599.08	0.01\\
599.09	0.01\\
599.1	0.01\\
599.11	0.01\\
599.12	0.01\\
599.13	0.01\\
599.14	0.01\\
599.15	0.01\\
599.16	0.01\\
599.17	0.01\\
599.18	0.01\\
599.19	0.01\\
599.2	0.01\\
599.21	0.00992567596961163\\
599.22	0.00973552951769219\\
599.23	0.0095441586337754\\
599.24	0.00935154653278383\\
599.25	0.00915767589482951\\
599.26	0.00896252884506647\\
599.27	0.00876609188285174\\
599.28	0.00856835603632599\\
599.29	0.0083693020586219\\
599.3	0.00816891006986826\\
599.31	0.00796715991787855\\
599.32	0.00776403075730489\\
599.33	0.00755950089682049\\
599.34	0.00735354792802892\\
599.35	0.00714614882670781\\
599.36	0.00693727969062358\\
599.37	0.00672691574027605\\
599.38	0.0065150313405578\\
599.39	0.00630159996621509\\
599.4	0.00608659416552601\\
599.41	0.00586998552217157\\
599.42	0.00565174461541249\\
599.43	0.00543184097827676\\
599.44	0.00521024305362824\\
599.45	0.00498691814797608\\
599.46	0.00476183238287344\\
599.47	0.00453495064374151\\
599.48	0.00430623652594129\\
599.49	0.0041843866201124\\
599.5	0.00411909531665794\\
599.51	0.00405317013116703\\
599.52	0.00398660758353459\\
599.53	0.00391939838601633\\
599.54	0.00385153203685256\\
599.55	0.00378300499736274\\
599.56	0.0037138136063696\\
599.57	0.00364395412032243\\
599.58	0.00357342333912867\\
599.59	0.00350221802491313\\
599.6	0.00343033523666813\\
599.61	0.00335777218644165\\
599.62	0.00328452624030823\\
599.63	0.00321059492650609\\
599.64	0.00313597594390693\\
599.65	0.00306066717083488\\
599.66	0.00298466667425192\\
599.67	0.00290797274578559\\
599.68	0.00283058389820869\\
599.69	0.00275249886666979\\
599.7	0.00267371661980331\\
599.71	0.00259423637132924\\
599.72	0.00251405772609498\\
599.73	0.00243318086727999\\
599.74	0.00235160633425846\\
599.75	0.00226933501860845\\
599.76	0.0021863682742743\\
599.77	0.00210270800264456\\
599.78	0.00201835654384337\\
599.79	0.00193331670079489\\
599.8	0.00184759176462326\\
599.81	0.00176118554147482\\
599.82	0.00167410238085579\\
599.83	0.00158634720558587\\
599.84	0.00149792554347528\\
599.85	0.00140884356084117\\
599.86	0.00131910809798803\\
599.87	0.00122872670678631\\
599.88	0.00113770769049358\\
599.89	0.00104606014597406\\
599.9	0.000953794008484178\\
599.91	0.000860920099205182\\
599.92	0.000767450175717955\\
599.93	0.000673396985630919\\
599.94	0.000578774323588566\\
599.95	0.000483597091906617\\
599.96	0.000387881365099733\\
599.97	0.000291644458589374\\
599.98	0.000194905001903006\\
599.99	9.76830167015649e-05\\
600	0\\
};
\addplot [color=mycolor16,solid,forget plot]
  table[row sep=crcr]{%
0.01	0.01\\
1.01	0.01\\
2.01	0.01\\
3.01	0.01\\
4.01	0.01\\
5.01	0.01\\
6.01	0.01\\
7.01	0.01\\
8.01	0.01\\
9.01	0.01\\
10.01	0.01\\
11.01	0.01\\
12.01	0.01\\
13.01	0.01\\
14.01	0.01\\
15.01	0.01\\
16.01	0.01\\
17.01	0.01\\
18.01	0.01\\
19.01	0.01\\
20.01	0.01\\
21.01	0.01\\
22.01	0.01\\
23.01	0.01\\
24.01	0.01\\
25.01	0.01\\
26.01	0.01\\
27.01	0.01\\
28.01	0.01\\
29.01	0.01\\
30.01	0.01\\
31.01	0.01\\
32.01	0.01\\
33.01	0.01\\
34.01	0.01\\
35.01	0.01\\
36.01	0.01\\
37.01	0.01\\
38.01	0.01\\
39.01	0.01\\
40.01	0.01\\
41.01	0.01\\
42.01	0.01\\
43.01	0.01\\
44.01	0.01\\
45.01	0.01\\
46.01	0.01\\
47.01	0.01\\
48.01	0.01\\
49.01	0.01\\
50.01	0.01\\
51.01	0.01\\
52.01	0.01\\
53.01	0.01\\
54.01	0.01\\
55.01	0.01\\
56.01	0.01\\
57.01	0.01\\
58.01	0.01\\
59.01	0.01\\
60.01	0.01\\
61.01	0.01\\
62.01	0.01\\
63.01	0.01\\
64.01	0.01\\
65.01	0.01\\
66.01	0.01\\
67.01	0.01\\
68.01	0.01\\
69.01	0.01\\
70.01	0.01\\
71.01	0.01\\
72.01	0.01\\
73.01	0.01\\
74.01	0.01\\
75.01	0.01\\
76.01	0.01\\
77.01	0.01\\
78.01	0.01\\
79.01	0.01\\
80.01	0.01\\
81.01	0.01\\
82.01	0.01\\
83.01	0.01\\
84.01	0.01\\
85.01	0.01\\
86.01	0.01\\
87.01	0.01\\
88.01	0.01\\
89.01	0.01\\
90.01	0.01\\
91.01	0.01\\
92.01	0.01\\
93.01	0.01\\
94.01	0.01\\
95.01	0.01\\
96.01	0.01\\
97.01	0.01\\
98.01	0.01\\
99.01	0.01\\
100.01	0.01\\
101.01	0.01\\
102.01	0.01\\
103.01	0.01\\
104.01	0.01\\
105.01	0.01\\
106.01	0.01\\
107.01	0.01\\
108.01	0.01\\
109.01	0.01\\
110.01	0.01\\
111.01	0.01\\
112.01	0.01\\
113.01	0.01\\
114.01	0.01\\
115.01	0.01\\
116.01	0.01\\
117.01	0.01\\
118.01	0.01\\
119.01	0.01\\
120.01	0.01\\
121.01	0.01\\
122.01	0.01\\
123.01	0.01\\
124.01	0.01\\
125.01	0.01\\
126.01	0.01\\
127.01	0.01\\
128.01	0.01\\
129.01	0.01\\
130.01	0.01\\
131.01	0.01\\
132.01	0.01\\
133.01	0.01\\
134.01	0.01\\
135.01	0.01\\
136.01	0.01\\
137.01	0.01\\
138.01	0.01\\
139.01	0.01\\
140.01	0.01\\
141.01	0.01\\
142.01	0.01\\
143.01	0.01\\
144.01	0.01\\
145.01	0.01\\
146.01	0.01\\
147.01	0.01\\
148.01	0.01\\
149.01	0.01\\
150.01	0.01\\
151.01	0.01\\
152.01	0.01\\
153.01	0.01\\
154.01	0.01\\
155.01	0.01\\
156.01	0.01\\
157.01	0.01\\
158.01	0.01\\
159.01	0.01\\
160.01	0.01\\
161.01	0.01\\
162.01	0.01\\
163.01	0.01\\
164.01	0.01\\
165.01	0.01\\
166.01	0.01\\
167.01	0.01\\
168.01	0.01\\
169.01	0.01\\
170.01	0.01\\
171.01	0.01\\
172.01	0.01\\
173.01	0.01\\
174.01	0.01\\
175.01	0.01\\
176.01	0.01\\
177.01	0.01\\
178.01	0.01\\
179.01	0.01\\
180.01	0.01\\
181.01	0.01\\
182.01	0.01\\
183.01	0.01\\
184.01	0.01\\
185.01	0.01\\
186.01	0.01\\
187.01	0.01\\
188.01	0.01\\
189.01	0.01\\
190.01	0.01\\
191.01	0.01\\
192.01	0.01\\
193.01	0.01\\
194.01	0.01\\
195.01	0.01\\
196.01	0.01\\
197.01	0.01\\
198.01	0.01\\
199.01	0.01\\
200.01	0.01\\
201.01	0.01\\
202.01	0.01\\
203.01	0.01\\
204.01	0.01\\
205.01	0.01\\
206.01	0.01\\
207.01	0.01\\
208.01	0.01\\
209.01	0.01\\
210.01	0.01\\
211.01	0.01\\
212.01	0.01\\
213.01	0.01\\
214.01	0.01\\
215.01	0.01\\
216.01	0.01\\
217.01	0.01\\
218.01	0.01\\
219.01	0.01\\
220.01	0.01\\
221.01	0.01\\
222.01	0.01\\
223.01	0.01\\
224.01	0.01\\
225.01	0.01\\
226.01	0.01\\
227.01	0.01\\
228.01	0.01\\
229.01	0.01\\
230.01	0.01\\
231.01	0.01\\
232.01	0.01\\
233.01	0.01\\
234.01	0.01\\
235.01	0.01\\
236.01	0.01\\
237.01	0.01\\
238.01	0.01\\
239.01	0.01\\
240.01	0.01\\
241.01	0.01\\
242.01	0.01\\
243.01	0.01\\
244.01	0.01\\
245.01	0.01\\
246.01	0.01\\
247.01	0.01\\
248.01	0.01\\
249.01	0.01\\
250.01	0.01\\
251.01	0.01\\
252.01	0.01\\
253.01	0.01\\
254.01	0.01\\
255.01	0.01\\
256.01	0.01\\
257.01	0.01\\
258.01	0.01\\
259.01	0.01\\
260.01	0.01\\
261.01	0.01\\
262.01	0.01\\
263.01	0.01\\
264.01	0.01\\
265.01	0.01\\
266.01	0.01\\
267.01	0.01\\
268.01	0.01\\
269.01	0.01\\
270.01	0.01\\
271.01	0.01\\
272.01	0.01\\
273.01	0.01\\
274.01	0.01\\
275.01	0.01\\
276.01	0.01\\
277.01	0.01\\
278.01	0.01\\
279.01	0.01\\
280.01	0.01\\
281.01	0.01\\
282.01	0.01\\
283.01	0.01\\
284.01	0.01\\
285.01	0.01\\
286.01	0.01\\
287.01	0.01\\
288.01	0.01\\
289.01	0.01\\
290.01	0.01\\
291.01	0.01\\
292.01	0.01\\
293.01	0.01\\
294.01	0.01\\
295.01	0.01\\
296.01	0.01\\
297.01	0.01\\
298.01	0.01\\
299.01	0.01\\
300.01	0.01\\
301.01	0.01\\
302.01	0.01\\
303.01	0.01\\
304.01	0.01\\
305.01	0.01\\
306.01	0.01\\
307.01	0.01\\
308.01	0.01\\
309.01	0.01\\
310.01	0.01\\
311.01	0.01\\
312.01	0.01\\
313.01	0.01\\
314.01	0.01\\
315.01	0.01\\
316.01	0.01\\
317.01	0.01\\
318.01	0.01\\
319.01	0.01\\
320.01	0.01\\
321.01	0.01\\
322.01	0.01\\
323.01	0.01\\
324.01	0.01\\
325.01	0.01\\
326.01	0.01\\
327.01	0.01\\
328.01	0.01\\
329.01	0.01\\
330.01	0.01\\
331.01	0.01\\
332.01	0.01\\
333.01	0.01\\
334.01	0.01\\
335.01	0.01\\
336.01	0.01\\
337.01	0.01\\
338.01	0.01\\
339.01	0.01\\
340.01	0.01\\
341.01	0.01\\
342.01	0.01\\
343.01	0.01\\
344.01	0.01\\
345.01	0.01\\
346.01	0.01\\
347.01	0.01\\
348.01	0.01\\
349.01	0.01\\
350.01	0.01\\
351.01	0.01\\
352.01	0.01\\
353.01	0.01\\
354.01	0.01\\
355.01	0.01\\
356.01	0.01\\
357.01	0.01\\
358.01	0.01\\
359.01	0.01\\
360.01	0.01\\
361.01	0.01\\
362.01	0.01\\
363.01	0.01\\
364.01	0.01\\
365.01	0.01\\
366.01	0.01\\
367.01	0.01\\
368.01	0.01\\
369.01	0.01\\
370.01	0.01\\
371.01	0.01\\
372.01	0.01\\
373.01	0.01\\
374.01	0.01\\
375.01	0.01\\
376.01	0.01\\
377.01	0.01\\
378.01	0.01\\
379.01	0.01\\
380.01	0.01\\
381.01	0.01\\
382.01	0.01\\
383.01	0.01\\
384.01	0.01\\
385.01	0.01\\
386.01	0.01\\
387.01	0.01\\
388.01	0.01\\
389.01	0.01\\
390.01	0.01\\
391.01	0.01\\
392.01	0.01\\
393.01	0.01\\
394.01	0.01\\
395.01	0.01\\
396.01	0.01\\
397.01	0.01\\
398.01	0.01\\
399.01	0.01\\
400.01	0.01\\
401.01	0.01\\
402.01	0.01\\
403.01	0.01\\
404.01	0.01\\
405.01	0.01\\
406.01	0.01\\
407.01	0.01\\
408.01	0.01\\
409.01	0.01\\
410.01	0.01\\
411.01	0.01\\
412.01	0.01\\
413.01	0.01\\
414.01	0.01\\
415.01	0.01\\
416.01	0.01\\
417.01	0.01\\
418.01	0.01\\
419.01	0.01\\
420.01	0.01\\
421.01	0.01\\
422.01	0.01\\
423.01	0.01\\
424.01	0.01\\
425.01	0.01\\
426.01	0.01\\
427.01	0.01\\
428.01	0.01\\
429.01	0.01\\
430.01	0.01\\
431.01	0.01\\
432.01	0.01\\
433.01	0.01\\
434.01	0.01\\
435.01	0.01\\
436.01	0.01\\
437.01	0.01\\
438.01	0.01\\
439.01	0.01\\
440.01	0.01\\
441.01	0.01\\
442.01	0.01\\
443.01	0.01\\
444.01	0.01\\
445.01	0.01\\
446.01	0.01\\
447.01	0.01\\
448.01	0.01\\
449.01	0.01\\
450.01	0.01\\
451.01	0.01\\
452.01	0.01\\
453.01	0.01\\
454.01	0.01\\
455.01	0.01\\
456.01	0.01\\
457.01	0.01\\
458.01	0.01\\
459.01	0.01\\
460.01	0.01\\
461.01	0.01\\
462.01	0.01\\
463.01	0.01\\
464.01	0.01\\
465.01	0.01\\
466.01	0.01\\
467.01	0.01\\
468.01	0.01\\
469.01	0.01\\
470.01	0.01\\
471.01	0.01\\
472.01	0.01\\
473.01	0.01\\
474.01	0.01\\
475.01	0.01\\
476.01	0.01\\
477.01	0.01\\
478.01	0.01\\
479.01	0.01\\
480.01	0.01\\
481.01	0.01\\
482.01	0.01\\
483.01	0.01\\
484.01	0.01\\
485.01	0.01\\
486.01	0.01\\
487.01	0.01\\
488.01	0.01\\
489.01	0.01\\
490.01	0.01\\
491.01	0.01\\
492.01	0.01\\
493.01	0.01\\
494.01	0.01\\
495.01	0.01\\
496.01	0.01\\
497.01	0.01\\
498.01	0.01\\
499.01	0.01\\
500.01	0.01\\
501.01	0.01\\
502.01	0.01\\
503.01	0.01\\
504.01	0.01\\
505.01	0.01\\
506.01	0.01\\
507.01	0.01\\
508.01	0.01\\
509.01	0.01\\
510.01	0.01\\
511.01	0.01\\
512.01	0.01\\
513.01	0.01\\
514.01	0.01\\
515.01	0.01\\
516.01	0.01\\
517.01	0.01\\
518.01	0.01\\
519.01	0.01\\
520.01	0.01\\
521.01	0.01\\
522.01	0.01\\
523.01	0.01\\
524.01	0.01\\
525.01	0.01\\
526.01	0.01\\
527.01	0.01\\
528.01	0.01\\
529.01	0.01\\
530.01	0.01\\
531.01	0.01\\
532.01	0.01\\
533.01	0.01\\
534.01	0.01\\
535.01	0.01\\
536.01	0.01\\
537.01	0.01\\
538.01	0.01\\
539.01	0.01\\
540.01	0.01\\
541.01	0.01\\
542.01	0.01\\
543.01	0.01\\
544.01	0.01\\
545.01	0.01\\
546.01	0.01\\
547.01	0.01\\
548.01	0.01\\
549.01	0.01\\
550.01	0.01\\
551.01	0.01\\
552.01	0.01\\
553.01	0.01\\
554.01	0.01\\
555.01	0.01\\
556.01	0.01\\
557.01	0.01\\
558.01	0.01\\
559.01	0.01\\
560.01	0.01\\
561.01	0.01\\
562.01	0.01\\
563.01	0.01\\
564.01	0.01\\
565.01	0.01\\
566.01	0.01\\
567.01	0.01\\
568.01	0.01\\
569.01	0.01\\
570.01	0.01\\
571.01	0.01\\
572.01	0.01\\
573.01	0.01\\
574.01	0.01\\
575.01	0.01\\
576.01	0.01\\
577.01	0.01\\
578.01	0.01\\
579.01	0.01\\
580.01	0.01\\
581.01	0.01\\
582.01	0.01\\
583.01	0.01\\
584.01	0.01\\
585.01	0.01\\
586.01	0.01\\
587.01	0.01\\
588.01	0.01\\
589.01	0.01\\
590.01	0.01\\
591.01	0.01\\
592.01	0.01\\
593.01	0.01\\
594.01	0.01\\
595.01	0.01\\
596.01	0.01\\
597.01	0.01\\
598.01	0.01\\
599.01	0.00988739368402132\\
599.02	0.00968074340707133\\
599.03	0.00947259332618321\\
599.04	0.00926292139085469\\
599.05	0.00905170483194615\\
599.06	0.00883892013221553\\
599.07	0.00862454299548213\\
599.08	0.00840855188409677\\
599.09	0.00819093287807366\\
599.1	0.00797166046229913\\
599.11	0.00775070766774859\\
599.12	0.00752804673107546\\
599.13	0.00730364946987773\\
599.14	0.00707748601739125\\
599.15	0.00684952544651139\\
599.16	0.00661973572330175\\
599.17	0.00638808372306731\\
599.18	0.00615453520876707\\
599.19	0.00591905452880597\\
599.2	0.00568160471301551\\
599.21	0.00551663477868118\\
599.22	0.00546587110992685\\
599.23	0.00541470224664812\\
599.24	0.00536313135198442\\
599.25	0.00531116197147127\\
599.26	0.00525879805221435\\
599.27	0.00520603898014746\\
599.28	0.00515287938995751\\
599.29	0.00509932402820601\\
599.3	0.00504537811029469\\
599.31	0.00499104695446077\\
599.32	0.00493633640166715\\
599.33	0.0048812531933729\\
599.34	0.00482580463929975\\
599.35	0.00476999849829594\\
599.36	0.00471384324138586\\
599.37	0.00465734805159418\\
599.38	0.00460052280219023\\
599.39	0.00454337809122961\\
599.4	0.00448592528632182\\
599.41	0.00442817656636471\\
599.42	0.00437014495065976\\
599.43	0.00431184434152103\\
599.44	0.00425328956922003\\
599.45	0.00419449643942142\\
599.46	0.00413548178327611\\
599.47	0.00407626351035228\\
599.48	0.0040168606645986\\
599.49	0.00395702039573175\\
599.5	0.00389659954551457\\
599.51	0.00383559266882257\\
599.52	0.00377399426818631\\
599.53	0.00371179882538318\\
599.54	0.00364900081011568\\
599.55	0.00358559464277508\\
599.56	0.00352157469499213\\
599.57	0.00345693529004798\\
599.58	0.00339167069943865\\
599.59	0.00332577514300587\\
599.6	0.00325924278682438\\
599.61	0.00319206774199901\\
599.62	0.0031242440634364\\
599.63	0.00305576574854174\\
599.64	0.00298662673583587\\
599.65	0.00291682090348755\\
599.66	0.00284634206775538\\
599.67	0.00277518398125874\\
599.68	0.00270334033118015\\
599.69	0.00263080473738273\\
599.7	0.00255757075041003\\
599.71	0.00248363184936088\\
599.72	0.0024089814392363\\
599.73	0.00233361284760038\\
599.74	0.00225751932173361\\
599.75	0.00218069402566163\\
599.76	0.00210313003670668\\
599.77	0.0020248203416071\\
599.78	0.00194575783277725\\
599.79	0.00186593530430994\\
599.8	0.00178534544770319\\
599.81	0.00170398084729193\\
599.82	0.0016218339753637\\
599.83	0.00153889718693601\\
599.84	0.00145516271417133\\
599.85	0.00137062266040393\\
599.86	0.00128526899375086\\
599.87	0.00119909354027743\\
599.88	0.00111208797668507\\
599.89	0.00102424382248756\\
599.9	0.000935552431638535\\
599.91	0.000846004983570773\\
599.92	0.000755592473604684\\
599.93	0.000664305702680239\\
599.94	0.000572135266363093\\
599.95	0.000479071543072036\\
599.96	0.000385104681470808\\
599.97	0.000290224586963043\\
599.98	0.000194420907224489\\
599.99	9.76830167015649e-05\\
600	0\\
};
\addplot [color=mycolor17,solid,forget plot]
  table[row sep=crcr]{%
0.01	0.01\\
1.01	0.01\\
2.01	0.01\\
3.01	0.01\\
4.01	0.01\\
5.01	0.01\\
6.01	0.01\\
7.01	0.01\\
8.01	0.01\\
9.01	0.01\\
10.01	0.01\\
11.01	0.01\\
12.01	0.01\\
13.01	0.01\\
14.01	0.01\\
15.01	0.01\\
16.01	0.01\\
17.01	0.01\\
18.01	0.01\\
19.01	0.01\\
20.01	0.01\\
21.01	0.01\\
22.01	0.01\\
23.01	0.01\\
24.01	0.01\\
25.01	0.01\\
26.01	0.01\\
27.01	0.01\\
28.01	0.01\\
29.01	0.01\\
30.01	0.01\\
31.01	0.01\\
32.01	0.01\\
33.01	0.01\\
34.01	0.01\\
35.01	0.01\\
36.01	0.01\\
37.01	0.01\\
38.01	0.01\\
39.01	0.01\\
40.01	0.01\\
41.01	0.01\\
42.01	0.01\\
43.01	0.01\\
44.01	0.01\\
45.01	0.01\\
46.01	0.01\\
47.01	0.01\\
48.01	0.01\\
49.01	0.01\\
50.01	0.01\\
51.01	0.01\\
52.01	0.01\\
53.01	0.01\\
54.01	0.01\\
55.01	0.01\\
56.01	0.01\\
57.01	0.01\\
58.01	0.01\\
59.01	0.01\\
60.01	0.01\\
61.01	0.01\\
62.01	0.01\\
63.01	0.01\\
64.01	0.01\\
65.01	0.01\\
66.01	0.01\\
67.01	0.01\\
68.01	0.01\\
69.01	0.01\\
70.01	0.01\\
71.01	0.01\\
72.01	0.01\\
73.01	0.01\\
74.01	0.01\\
75.01	0.01\\
76.01	0.01\\
77.01	0.01\\
78.01	0.01\\
79.01	0.01\\
80.01	0.01\\
81.01	0.01\\
82.01	0.01\\
83.01	0.01\\
84.01	0.01\\
85.01	0.01\\
86.01	0.01\\
87.01	0.01\\
88.01	0.01\\
89.01	0.01\\
90.01	0.01\\
91.01	0.01\\
92.01	0.01\\
93.01	0.01\\
94.01	0.01\\
95.01	0.01\\
96.01	0.01\\
97.01	0.01\\
98.01	0.01\\
99.01	0.01\\
100.01	0.01\\
101.01	0.01\\
102.01	0.01\\
103.01	0.01\\
104.01	0.01\\
105.01	0.01\\
106.01	0.01\\
107.01	0.01\\
108.01	0.01\\
109.01	0.01\\
110.01	0.01\\
111.01	0.01\\
112.01	0.01\\
113.01	0.01\\
114.01	0.01\\
115.01	0.01\\
116.01	0.01\\
117.01	0.01\\
118.01	0.01\\
119.01	0.01\\
120.01	0.01\\
121.01	0.01\\
122.01	0.01\\
123.01	0.01\\
124.01	0.01\\
125.01	0.01\\
126.01	0.01\\
127.01	0.01\\
128.01	0.01\\
129.01	0.01\\
130.01	0.01\\
131.01	0.01\\
132.01	0.01\\
133.01	0.01\\
134.01	0.01\\
135.01	0.01\\
136.01	0.01\\
137.01	0.01\\
138.01	0.01\\
139.01	0.01\\
140.01	0.01\\
141.01	0.01\\
142.01	0.01\\
143.01	0.01\\
144.01	0.01\\
145.01	0.01\\
146.01	0.01\\
147.01	0.01\\
148.01	0.01\\
149.01	0.01\\
150.01	0.01\\
151.01	0.01\\
152.01	0.01\\
153.01	0.01\\
154.01	0.01\\
155.01	0.01\\
156.01	0.01\\
157.01	0.01\\
158.01	0.01\\
159.01	0.01\\
160.01	0.01\\
161.01	0.01\\
162.01	0.01\\
163.01	0.01\\
164.01	0.01\\
165.01	0.01\\
166.01	0.01\\
167.01	0.01\\
168.01	0.01\\
169.01	0.01\\
170.01	0.01\\
171.01	0.01\\
172.01	0.01\\
173.01	0.01\\
174.01	0.01\\
175.01	0.01\\
176.01	0.01\\
177.01	0.01\\
178.01	0.01\\
179.01	0.01\\
180.01	0.01\\
181.01	0.01\\
182.01	0.01\\
183.01	0.01\\
184.01	0.01\\
185.01	0.01\\
186.01	0.01\\
187.01	0.01\\
188.01	0.01\\
189.01	0.01\\
190.01	0.01\\
191.01	0.01\\
192.01	0.01\\
193.01	0.01\\
194.01	0.01\\
195.01	0.01\\
196.01	0.01\\
197.01	0.01\\
198.01	0.01\\
199.01	0.01\\
200.01	0.01\\
201.01	0.01\\
202.01	0.01\\
203.01	0.01\\
204.01	0.01\\
205.01	0.01\\
206.01	0.01\\
207.01	0.01\\
208.01	0.01\\
209.01	0.01\\
210.01	0.01\\
211.01	0.01\\
212.01	0.01\\
213.01	0.01\\
214.01	0.01\\
215.01	0.01\\
216.01	0.01\\
217.01	0.01\\
218.01	0.01\\
219.01	0.01\\
220.01	0.01\\
221.01	0.01\\
222.01	0.01\\
223.01	0.01\\
224.01	0.01\\
225.01	0.01\\
226.01	0.01\\
227.01	0.01\\
228.01	0.01\\
229.01	0.01\\
230.01	0.01\\
231.01	0.01\\
232.01	0.01\\
233.01	0.01\\
234.01	0.01\\
235.01	0.01\\
236.01	0.01\\
237.01	0.01\\
238.01	0.01\\
239.01	0.01\\
240.01	0.01\\
241.01	0.01\\
242.01	0.01\\
243.01	0.01\\
244.01	0.01\\
245.01	0.01\\
246.01	0.01\\
247.01	0.01\\
248.01	0.01\\
249.01	0.01\\
250.01	0.01\\
251.01	0.01\\
252.01	0.01\\
253.01	0.01\\
254.01	0.01\\
255.01	0.01\\
256.01	0.01\\
257.01	0.01\\
258.01	0.01\\
259.01	0.01\\
260.01	0.01\\
261.01	0.01\\
262.01	0.01\\
263.01	0.01\\
264.01	0.01\\
265.01	0.01\\
266.01	0.01\\
267.01	0.01\\
268.01	0.01\\
269.01	0.01\\
270.01	0.01\\
271.01	0.01\\
272.01	0.01\\
273.01	0.01\\
274.01	0.01\\
275.01	0.01\\
276.01	0.01\\
277.01	0.01\\
278.01	0.01\\
279.01	0.01\\
280.01	0.01\\
281.01	0.01\\
282.01	0.01\\
283.01	0.01\\
284.01	0.01\\
285.01	0.01\\
286.01	0.01\\
287.01	0.01\\
288.01	0.01\\
289.01	0.01\\
290.01	0.01\\
291.01	0.01\\
292.01	0.01\\
293.01	0.01\\
294.01	0.01\\
295.01	0.01\\
296.01	0.01\\
297.01	0.01\\
298.01	0.01\\
299.01	0.01\\
300.01	0.01\\
301.01	0.01\\
302.01	0.01\\
303.01	0.01\\
304.01	0.01\\
305.01	0.01\\
306.01	0.01\\
307.01	0.01\\
308.01	0.01\\
309.01	0.01\\
310.01	0.01\\
311.01	0.01\\
312.01	0.01\\
313.01	0.01\\
314.01	0.01\\
315.01	0.01\\
316.01	0.01\\
317.01	0.01\\
318.01	0.01\\
319.01	0.01\\
320.01	0.01\\
321.01	0.01\\
322.01	0.01\\
323.01	0.01\\
324.01	0.01\\
325.01	0.01\\
326.01	0.01\\
327.01	0.01\\
328.01	0.01\\
329.01	0.01\\
330.01	0.01\\
331.01	0.01\\
332.01	0.01\\
333.01	0.01\\
334.01	0.01\\
335.01	0.01\\
336.01	0.01\\
337.01	0.01\\
338.01	0.01\\
339.01	0.01\\
340.01	0.01\\
341.01	0.01\\
342.01	0.01\\
343.01	0.01\\
344.01	0.01\\
345.01	0.01\\
346.01	0.01\\
347.01	0.01\\
348.01	0.01\\
349.01	0.01\\
350.01	0.01\\
351.01	0.01\\
352.01	0.01\\
353.01	0.01\\
354.01	0.01\\
355.01	0.01\\
356.01	0.01\\
357.01	0.01\\
358.01	0.01\\
359.01	0.01\\
360.01	0.01\\
361.01	0.01\\
362.01	0.01\\
363.01	0.01\\
364.01	0.01\\
365.01	0.01\\
366.01	0.01\\
367.01	0.01\\
368.01	0.01\\
369.01	0.01\\
370.01	0.01\\
371.01	0.01\\
372.01	0.01\\
373.01	0.01\\
374.01	0.01\\
375.01	0.01\\
376.01	0.01\\
377.01	0.01\\
378.01	0.01\\
379.01	0.01\\
380.01	0.01\\
381.01	0.01\\
382.01	0.01\\
383.01	0.01\\
384.01	0.01\\
385.01	0.01\\
386.01	0.01\\
387.01	0.01\\
388.01	0.01\\
389.01	0.01\\
390.01	0.01\\
391.01	0.01\\
392.01	0.01\\
393.01	0.01\\
394.01	0.01\\
395.01	0.01\\
396.01	0.01\\
397.01	0.01\\
398.01	0.01\\
399.01	0.01\\
400.01	0.01\\
401.01	0.01\\
402.01	0.01\\
403.01	0.01\\
404.01	0.01\\
405.01	0.01\\
406.01	0.01\\
407.01	0.01\\
408.01	0.01\\
409.01	0.01\\
410.01	0.01\\
411.01	0.01\\
412.01	0.01\\
413.01	0.01\\
414.01	0.01\\
415.01	0.01\\
416.01	0.01\\
417.01	0.01\\
418.01	0.01\\
419.01	0.01\\
420.01	0.01\\
421.01	0.01\\
422.01	0.01\\
423.01	0.01\\
424.01	0.01\\
425.01	0.01\\
426.01	0.01\\
427.01	0.01\\
428.01	0.01\\
429.01	0.01\\
430.01	0.01\\
431.01	0.01\\
432.01	0.01\\
433.01	0.01\\
434.01	0.01\\
435.01	0.01\\
436.01	0.01\\
437.01	0.01\\
438.01	0.01\\
439.01	0.01\\
440.01	0.01\\
441.01	0.01\\
442.01	0.01\\
443.01	0.01\\
444.01	0.01\\
445.01	0.01\\
446.01	0.01\\
447.01	0.01\\
448.01	0.01\\
449.01	0.01\\
450.01	0.01\\
451.01	0.01\\
452.01	0.01\\
453.01	0.01\\
454.01	0.01\\
455.01	0.01\\
456.01	0.01\\
457.01	0.01\\
458.01	0.01\\
459.01	0.01\\
460.01	0.01\\
461.01	0.01\\
462.01	0.01\\
463.01	0.01\\
464.01	0.01\\
465.01	0.01\\
466.01	0.01\\
467.01	0.01\\
468.01	0.01\\
469.01	0.01\\
470.01	0.01\\
471.01	0.01\\
472.01	0.01\\
473.01	0.01\\
474.01	0.01\\
475.01	0.01\\
476.01	0.01\\
477.01	0.01\\
478.01	0.01\\
479.01	0.01\\
480.01	0.01\\
481.01	0.01\\
482.01	0.01\\
483.01	0.01\\
484.01	0.01\\
485.01	0.01\\
486.01	0.01\\
487.01	0.01\\
488.01	0.01\\
489.01	0.01\\
490.01	0.01\\
491.01	0.01\\
492.01	0.01\\
493.01	0.01\\
494.01	0.01\\
495.01	0.01\\
496.01	0.01\\
497.01	0.01\\
498.01	0.01\\
499.01	0.01\\
500.01	0.01\\
501.01	0.01\\
502.01	0.01\\
503.01	0.01\\
504.01	0.01\\
505.01	0.01\\
506.01	0.01\\
507.01	0.01\\
508.01	0.01\\
509.01	0.01\\
510.01	0.01\\
511.01	0.01\\
512.01	0.01\\
513.01	0.01\\
514.01	0.01\\
515.01	0.01\\
516.01	0.01\\
517.01	0.01\\
518.01	0.01\\
519.01	0.01\\
520.01	0.01\\
521.01	0.01\\
522.01	0.01\\
523.01	0.01\\
524.01	0.01\\
525.01	0.01\\
526.01	0.01\\
527.01	0.01\\
528.01	0.01\\
529.01	0.01\\
530.01	0.01\\
531.01	0.01\\
532.01	0.01\\
533.01	0.01\\
534.01	0.01\\
535.01	0.01\\
536.01	0.01\\
537.01	0.01\\
538.01	0.01\\
539.01	0.01\\
540.01	0.01\\
541.01	0.01\\
542.01	0.01\\
543.01	0.01\\
544.01	0.01\\
545.01	0.01\\
546.01	0.01\\
547.01	0.01\\
548.01	0.01\\
549.01	0.01\\
550.01	0.01\\
551.01	0.01\\
552.01	0.01\\
553.01	0.01\\
554.01	0.01\\
555.01	0.01\\
556.01	0.01\\
557.01	0.01\\
558.01	0.01\\
559.01	0.01\\
560.01	0.01\\
561.01	0.01\\
562.01	0.01\\
563.01	0.01\\
564.01	0.01\\
565.01	0.01\\
566.01	0.01\\
567.01	0.01\\
568.01	0.01\\
569.01	0.01\\
570.01	0.01\\
571.01	0.01\\
572.01	0.01\\
573.01	0.01\\
574.01	0.01\\
575.01	0.01\\
576.01	0.01\\
577.01	0.01\\
578.01	0.01\\
579.01	0.01\\
580.01	0.01\\
581.01	0.01\\
582.01	0.01\\
583.01	0.01\\
584.01	0.01\\
585.01	0.01\\
586.01	0.01\\
587.01	0.01\\
588.01	0.01\\
589.01	0.01\\
590.01	0.01\\
591.01	0.01\\
592.01	0.01\\
593.01	0.01\\
594.01	0.01\\
595.01	0.01\\
596.01	0.01\\
597.01	0.01\\
598.01	0.01\\
599.01	0.00629791790588287\\
599.02	0.00625576917955417\\
599.03	0.00621335984443891\\
599.04	0.00617069785691429\\
599.05	0.00612779157677909\\
599.06	0.00608464998542295\\
599.07	0.0060412827172068\\
599.08	0.0059976965053873\\
599.09	0.00595389027958764\\
599.1	0.00590987443069723\\
599.11	0.00586566091945395\\
599.12	0.00582126238082717\\
599.13	0.00577669174281891\\
599.14	0.00573196349637267\\
599.15	0.00568709307297772\\
599.16	0.00564209689300802\\
599.17	0.00559699235136999\\
599.18	0.00555179784086047\\
599.19	0.00550653305777904\\
599.2	0.00546121890881882\\
599.21	0.00541571379251483\\
599.22	0.00536977225595453\\
599.23	0.00532339010397267\\
599.24	0.00527656308367495\\
599.25	0.00522928688212764\\
599.26	0.00518155712390623\\
599.27	0.00513336939271845\\
599.28	0.00508471925558006\\
599.29	0.00503560221536904\\
599.3	0.00498601370813244\\
599.31	0.00493594910233537\\
599.32	0.00488540369593917\\
599.33	0.00483437271134595\\
599.34	0.00478285129200968\\
599.35	0.00473083449973311\\
599.36	0.00467831731039358\\
599.37	0.00462529460959125\\
599.38	0.00457176118836954\\
599.39	0.00451771174234299\\
599.4	0.00446314087797853\\
599.41	0.00440804309309745\\
599.42	0.00435241277147536\\
599.43	0.0042962441771187\\
599.44	0.00423953144819901\\
599.45	0.00418226859062502\\
599.46	0.00412444947123163\\
599.47	0.00406606781056341\\
599.48	0.00400711717522909\\
599.49	0.00394759166761367\\
599.5	0.00388748567419746\\
599.51	0.00382679352639224\\
599.52	0.00376550949999173\\
599.53	0.00370362781448029\\
599.54	0.00364114263228638\\
599.55	0.00357804805817696\\
599.56	0.00351433813863804\\
599.57	0.00345000686124116\\
599.58	0.00338504815401388\\
599.59	0.00331945588479675\\
599.6	0.00325322386059824\\
599.61	0.00318634582694277\\
599.62	0.00311881546721176\\
599.63	0.00305062640197774\\
599.64	0.00298177218833173\\
599.65	0.00291224631920415\\
599.66	0.00284204222267936\\
599.67	0.00277115326130438\\
599.68	0.002699572731392\\
599.69	0.0026272938623186\\
599.7	0.00255430981581701\\
599.71	0.00248061368526496\\
599.72	0.00240619849497075\\
599.73	0.00233105719945854\\
599.74	0.00225518268275211\\
599.75	0.00217856775765777\\
599.76	0.00210120516504843\\
599.77	0.00202308757315053\\
599.78	0.00194420757683406\\
599.79	0.00186455769690723\\
599.8	0.0017841303794173\\
599.81	0.00170291799495944\\
599.82	0.00162091283799561\\
599.83	0.00153810712618569\\
599.84	0.00145449299973336\\
599.85	0.00137006252074928\\
599.86	0.00128480767263492\\
599.87	0.00119872035948991\\
599.88	0.00111179240554713\\
599.89	0.00102401555463903\\
599.9	0.000935381469700161\\
599.91	0.000845881732310544\\
599.92	0.000755507842285438\\
599.93	0.000664251217317542\\
599.94	0.000572103192678209\\
599.95	0.000479055020985011\\
599.96	0.000385097872043671\\
599.97	0.000290222832773275\\
599.98	0.000194420907224489\\
599.99	9.76830167015649e-05\\
600	0\\
};
\addplot [color=mycolor18,solid,forget plot]
  table[row sep=crcr]{%
0.01	0.01\\
1.01	0.01\\
2.01	0.01\\
3.01	0.01\\
4.01	0.01\\
5.01	0.01\\
6.01	0.01\\
7.01	0.01\\
8.01	0.01\\
9.01	0.01\\
10.01	0.01\\
11.01	0.01\\
12.01	0.01\\
13.01	0.01\\
14.01	0.01\\
15.01	0.01\\
16.01	0.01\\
17.01	0.01\\
18.01	0.01\\
19.01	0.01\\
20.01	0.01\\
21.01	0.01\\
22.01	0.01\\
23.01	0.01\\
24.01	0.01\\
25.01	0.01\\
26.01	0.01\\
27.01	0.01\\
28.01	0.01\\
29.01	0.01\\
30.01	0.01\\
31.01	0.01\\
32.01	0.01\\
33.01	0.01\\
34.01	0.01\\
35.01	0.01\\
36.01	0.01\\
37.01	0.01\\
38.01	0.01\\
39.01	0.01\\
40.01	0.01\\
41.01	0.01\\
42.01	0.01\\
43.01	0.01\\
44.01	0.01\\
45.01	0.01\\
46.01	0.01\\
47.01	0.01\\
48.01	0.01\\
49.01	0.01\\
50.01	0.01\\
51.01	0.01\\
52.01	0.01\\
53.01	0.01\\
54.01	0.01\\
55.01	0.01\\
56.01	0.01\\
57.01	0.01\\
58.01	0.01\\
59.01	0.01\\
60.01	0.01\\
61.01	0.01\\
62.01	0.01\\
63.01	0.01\\
64.01	0.01\\
65.01	0.01\\
66.01	0.01\\
67.01	0.01\\
68.01	0.01\\
69.01	0.01\\
70.01	0.01\\
71.01	0.01\\
72.01	0.01\\
73.01	0.01\\
74.01	0.01\\
75.01	0.01\\
76.01	0.01\\
77.01	0.01\\
78.01	0.01\\
79.01	0.01\\
80.01	0.01\\
81.01	0.01\\
82.01	0.01\\
83.01	0.01\\
84.01	0.01\\
85.01	0.01\\
86.01	0.01\\
87.01	0.01\\
88.01	0.01\\
89.01	0.01\\
90.01	0.01\\
91.01	0.01\\
92.01	0.01\\
93.01	0.01\\
94.01	0.01\\
95.01	0.01\\
96.01	0.01\\
97.01	0.01\\
98.01	0.01\\
99.01	0.01\\
100.01	0.01\\
101.01	0.01\\
102.01	0.01\\
103.01	0.01\\
104.01	0.01\\
105.01	0.01\\
106.01	0.01\\
107.01	0.01\\
108.01	0.01\\
109.01	0.01\\
110.01	0.01\\
111.01	0.01\\
112.01	0.01\\
113.01	0.01\\
114.01	0.01\\
115.01	0.01\\
116.01	0.01\\
117.01	0.01\\
118.01	0.01\\
119.01	0.01\\
120.01	0.01\\
121.01	0.01\\
122.01	0.01\\
123.01	0.01\\
124.01	0.01\\
125.01	0.01\\
126.01	0.01\\
127.01	0.01\\
128.01	0.01\\
129.01	0.01\\
130.01	0.01\\
131.01	0.01\\
132.01	0.01\\
133.01	0.01\\
134.01	0.01\\
135.01	0.01\\
136.01	0.01\\
137.01	0.01\\
138.01	0.01\\
139.01	0.01\\
140.01	0.01\\
141.01	0.01\\
142.01	0.01\\
143.01	0.01\\
144.01	0.01\\
145.01	0.01\\
146.01	0.01\\
147.01	0.01\\
148.01	0.01\\
149.01	0.01\\
150.01	0.01\\
151.01	0.01\\
152.01	0.01\\
153.01	0.01\\
154.01	0.01\\
155.01	0.01\\
156.01	0.01\\
157.01	0.01\\
158.01	0.01\\
159.01	0.01\\
160.01	0.01\\
161.01	0.01\\
162.01	0.01\\
163.01	0.01\\
164.01	0.01\\
165.01	0.01\\
166.01	0.01\\
167.01	0.01\\
168.01	0.01\\
169.01	0.01\\
170.01	0.01\\
171.01	0.01\\
172.01	0.01\\
173.01	0.01\\
174.01	0.01\\
175.01	0.01\\
176.01	0.01\\
177.01	0.01\\
178.01	0.01\\
179.01	0.01\\
180.01	0.01\\
181.01	0.01\\
182.01	0.01\\
183.01	0.01\\
184.01	0.01\\
185.01	0.01\\
186.01	0.01\\
187.01	0.01\\
188.01	0.01\\
189.01	0.01\\
190.01	0.01\\
191.01	0.01\\
192.01	0.01\\
193.01	0.01\\
194.01	0.01\\
195.01	0.01\\
196.01	0.01\\
197.01	0.01\\
198.01	0.01\\
199.01	0.01\\
200.01	0.01\\
201.01	0.01\\
202.01	0.01\\
203.01	0.01\\
204.01	0.01\\
205.01	0.01\\
206.01	0.01\\
207.01	0.01\\
208.01	0.01\\
209.01	0.01\\
210.01	0.01\\
211.01	0.01\\
212.01	0.01\\
213.01	0.01\\
214.01	0.01\\
215.01	0.01\\
216.01	0.01\\
217.01	0.01\\
218.01	0.01\\
219.01	0.01\\
220.01	0.01\\
221.01	0.01\\
222.01	0.01\\
223.01	0.01\\
224.01	0.01\\
225.01	0.01\\
226.01	0.01\\
227.01	0.01\\
228.01	0.01\\
229.01	0.01\\
230.01	0.01\\
231.01	0.01\\
232.01	0.01\\
233.01	0.01\\
234.01	0.01\\
235.01	0.01\\
236.01	0.01\\
237.01	0.01\\
238.01	0.01\\
239.01	0.01\\
240.01	0.01\\
241.01	0.01\\
242.01	0.01\\
243.01	0.01\\
244.01	0.01\\
245.01	0.01\\
246.01	0.01\\
247.01	0.01\\
248.01	0.01\\
249.01	0.01\\
250.01	0.01\\
251.01	0.01\\
252.01	0.01\\
253.01	0.01\\
254.01	0.01\\
255.01	0.01\\
256.01	0.01\\
257.01	0.01\\
258.01	0.01\\
259.01	0.01\\
260.01	0.01\\
261.01	0.01\\
262.01	0.01\\
263.01	0.01\\
264.01	0.01\\
265.01	0.01\\
266.01	0.01\\
267.01	0.01\\
268.01	0.01\\
269.01	0.01\\
270.01	0.01\\
271.01	0.01\\
272.01	0.01\\
273.01	0.01\\
274.01	0.01\\
275.01	0.01\\
276.01	0.01\\
277.01	0.01\\
278.01	0.01\\
279.01	0.01\\
280.01	0.01\\
281.01	0.01\\
282.01	0.01\\
283.01	0.01\\
284.01	0.01\\
285.01	0.01\\
286.01	0.01\\
287.01	0.01\\
288.01	0.01\\
289.01	0.01\\
290.01	0.01\\
291.01	0.01\\
292.01	0.01\\
293.01	0.01\\
294.01	0.01\\
295.01	0.01\\
296.01	0.01\\
297.01	0.01\\
298.01	0.01\\
299.01	0.01\\
300.01	0.01\\
301.01	0.01\\
302.01	0.01\\
303.01	0.01\\
304.01	0.01\\
305.01	0.01\\
306.01	0.01\\
307.01	0.01\\
308.01	0.01\\
309.01	0.01\\
310.01	0.01\\
311.01	0.01\\
312.01	0.01\\
313.01	0.01\\
314.01	0.01\\
315.01	0.01\\
316.01	0.01\\
317.01	0.01\\
318.01	0.01\\
319.01	0.01\\
320.01	0.01\\
321.01	0.01\\
322.01	0.01\\
323.01	0.01\\
324.01	0.01\\
325.01	0.01\\
326.01	0.01\\
327.01	0.01\\
328.01	0.01\\
329.01	0.01\\
330.01	0.01\\
331.01	0.01\\
332.01	0.01\\
333.01	0.01\\
334.01	0.01\\
335.01	0.01\\
336.01	0.01\\
337.01	0.01\\
338.01	0.01\\
339.01	0.01\\
340.01	0.01\\
341.01	0.01\\
342.01	0.01\\
343.01	0.01\\
344.01	0.01\\
345.01	0.01\\
346.01	0.01\\
347.01	0.01\\
348.01	0.01\\
349.01	0.01\\
350.01	0.01\\
351.01	0.01\\
352.01	0.01\\
353.01	0.01\\
354.01	0.01\\
355.01	0.01\\
356.01	0.01\\
357.01	0.01\\
358.01	0.01\\
359.01	0.01\\
360.01	0.01\\
361.01	0.01\\
362.01	0.01\\
363.01	0.01\\
364.01	0.01\\
365.01	0.01\\
366.01	0.01\\
367.01	0.01\\
368.01	0.01\\
369.01	0.01\\
370.01	0.01\\
371.01	0.01\\
372.01	0.01\\
373.01	0.01\\
374.01	0.01\\
375.01	0.01\\
376.01	0.01\\
377.01	0.01\\
378.01	0.01\\
379.01	0.01\\
380.01	0.01\\
381.01	0.01\\
382.01	0.01\\
383.01	0.01\\
384.01	0.01\\
385.01	0.01\\
386.01	0.01\\
387.01	0.01\\
388.01	0.01\\
389.01	0.01\\
390.01	0.01\\
391.01	0.01\\
392.01	0.01\\
393.01	0.01\\
394.01	0.01\\
395.01	0.01\\
396.01	0.01\\
397.01	0.01\\
398.01	0.01\\
399.01	0.01\\
400.01	0.01\\
401.01	0.01\\
402.01	0.01\\
403.01	0.01\\
404.01	0.01\\
405.01	0.01\\
406.01	0.01\\
407.01	0.01\\
408.01	0.01\\
409.01	0.01\\
410.01	0.01\\
411.01	0.01\\
412.01	0.01\\
413.01	0.01\\
414.01	0.01\\
415.01	0.01\\
416.01	0.01\\
417.01	0.01\\
418.01	0.01\\
419.01	0.01\\
420.01	0.01\\
421.01	0.01\\
422.01	0.01\\
423.01	0.01\\
424.01	0.01\\
425.01	0.01\\
426.01	0.01\\
427.01	0.01\\
428.01	0.01\\
429.01	0.01\\
430.01	0.01\\
431.01	0.01\\
432.01	0.01\\
433.01	0.01\\
434.01	0.01\\
435.01	0.01\\
436.01	0.01\\
437.01	0.01\\
438.01	0.01\\
439.01	0.01\\
440.01	0.01\\
441.01	0.01\\
442.01	0.01\\
443.01	0.01\\
444.01	0.01\\
445.01	0.01\\
446.01	0.01\\
447.01	0.01\\
448.01	0.01\\
449.01	0.01\\
450.01	0.01\\
451.01	0.01\\
452.01	0.01\\
453.01	0.01\\
454.01	0.01\\
455.01	0.01\\
456.01	0.01\\
457.01	0.01\\
458.01	0.01\\
459.01	0.01\\
460.01	0.01\\
461.01	0.01\\
462.01	0.01\\
463.01	0.01\\
464.01	0.01\\
465.01	0.01\\
466.01	0.01\\
467.01	0.01\\
468.01	0.01\\
469.01	0.01\\
470.01	0.01\\
471.01	0.01\\
472.01	0.01\\
473.01	0.01\\
474.01	0.01\\
475.01	0.01\\
476.01	0.01\\
477.01	0.01\\
478.01	0.01\\
479.01	0.01\\
480.01	0.01\\
481.01	0.01\\
482.01	0.01\\
483.01	0.01\\
484.01	0.01\\
485.01	0.01\\
486.01	0.01\\
487.01	0.01\\
488.01	0.01\\
489.01	0.01\\
490.01	0.01\\
491.01	0.01\\
492.01	0.01\\
493.01	0.01\\
494.01	0.01\\
495.01	0.01\\
496.01	0.01\\
497.01	0.01\\
498.01	0.01\\
499.01	0.01\\
500.01	0.01\\
501.01	0.01\\
502.01	0.01\\
503.01	0.01\\
504.01	0.01\\
505.01	0.01\\
506.01	0.01\\
507.01	0.01\\
508.01	0.01\\
509.01	0.01\\
510.01	0.01\\
511.01	0.01\\
512.01	0.01\\
513.01	0.01\\
514.01	0.01\\
515.01	0.01\\
516.01	0.01\\
517.01	0.01\\
518.01	0.01\\
519.01	0.01\\
520.01	0.01\\
521.01	0.01\\
522.01	0.01\\
523.01	0.01\\
524.01	0.01\\
525.01	0.01\\
526.01	0.01\\
527.01	0.01\\
528.01	0.01\\
529.01	0.01\\
530.01	0.01\\
531.01	0.01\\
532.01	0.01\\
533.01	0.01\\
534.01	0.01\\
535.01	0.01\\
536.01	0.01\\
537.01	0.01\\
538.01	0.01\\
539.01	0.01\\
540.01	0.01\\
541.01	0.01\\
542.01	0.01\\
543.01	0.01\\
544.01	0.01\\
545.01	0.01\\
546.01	0.01\\
547.01	0.01\\
548.01	0.01\\
549.01	0.01\\
550.01	0.01\\
551.01	0.01\\
552.01	0.01\\
553.01	0.01\\
554.01	0.01\\
555.01	0.01\\
556.01	0.01\\
557.01	0.01\\
558.01	0.01\\
559.01	0.01\\
560.01	0.01\\
561.01	0.01\\
562.01	0.01\\
563.01	0.01\\
564.01	0.01\\
565.01	0.01\\
566.01	0.01\\
567.01	0.01\\
568.01	0.01\\
569.01	0.01\\
570.01	0.01\\
571.01	0.01\\
572.01	0.01\\
573.01	0.01\\
574.01	0.01\\
575.01	0.01\\
576.01	0.01\\
577.01	0.01\\
578.01	0.01\\
579.01	0.01\\
580.01	0.01\\
581.01	0.01\\
582.01	0.01\\
583.01	0.01\\
584.01	0.01\\
585.01	0.01\\
586.01	0.01\\
587.01	0.01\\
588.01	0.01\\
589.01	0.01\\
590.01	0.01\\
591.01	0.01\\
592.01	0.01\\
593.01	0.01\\
594.01	0.01\\
595.01	0.01\\
596.01	0.01\\
597.01	0.01\\
598.01	0.01\\
599.01	0.0062426129370936\\
599.02	0.00620480443639245\\
599.03	0.00616663519666627\\
599.04	0.00612810141013912\\
599.05	0.00608919939176771\\
599.06	0.00604992538771588\\
599.07	0.00601027557193321\\
599.08	0.00597024605900537\\
599.09	0.00592983293920137\\
599.1	0.00588903222927545\\
599.11	0.00584783986513739\\
599.12	0.0058062516984085\\
599.13	0.00576426349491306\\
599.14	0.00572187092688378\\
599.15	0.00567906956807693\\
599.16	0.00563585488859709\\
599.17	0.00559222224977897\\
599.18	0.00554816689894846\\
599.19	0.00550368396247038\\
599.2	0.00545876843925004\\
599.21	0.00541341555627284\\
599.22	0.00536762101229125\\
599.23	0.00532138046333556\\
599.24	0.00527468952231843\\
599.25	0.00522754375864171\\
599.26	0.00517993869780592\\
599.27	0.00513186982093397\\
599.28	0.00508333256419477\\
599.29	0.00503432231838664\\
599.3	0.00498483442852312\\
599.31	0.00493486419341345\\
599.32	0.00488440686524595\\
599.33	0.00483345764918343\\
599.34	0.00478201170296413\\
599.35	0.00473006413650502\\
599.36	0.00467761001151456\\
599.37	0.00462464434111552\\
599.38	0.00457116208947773\\
599.39	0.00451715817542884\\
599.4	0.00446262748254593\\
599.41	0.00440756484478676\\
599.42	0.00435196504618212\\
599.43	0.00429582282054604\\
599.44	0.0042391328512059\\
599.45	0.00418188977075464\\
599.46	0.00412408816082742\\
599.47	0.00406572255190509\\
599.48	0.0040067874231475\\
599.49	0.00394727720047447\\
599.5	0.00388718625520624\\
599.51	0.00382650890352704\\
599.52	0.00376523940594317\\
599.53	0.00370337196673631\\
599.54	0.003640900733412\\
599.55	0.00357781979614291\\
599.56	0.00351412318720674\\
599.57	0.00344980488041878\\
599.58	0.00338485879055916\\
599.59	0.00331927877279455\\
599.6	0.00325305862209448\\
599.61	0.00318619207264212\\
599.62	0.00311867279723941\\
599.63	0.00305049440670675\\
599.64	0.00298165044927692\\
599.65	0.00291213440998337\\
599.66	0.00284193971004277\\
599.67	0.00277105970623188\\
599.68	0.00269948769025851\\
599.69	0.00262721688812665\\
599.7	0.00255424045949581\\
599.71	0.00248055149703432\\
599.72	0.00240614302576671\\
599.73	0.00233100800241499\\
599.74	0.00225513931473393\\
599.75	0.00217852978084007\\
599.76	0.0021011721485346\\
599.77	0.0020230590946199\\
599.78	0.00194418322420978\\
599.79	0.00186453707003317\\
599.8	0.00178411309173136\\
599.81	0.00170290367514858\\
599.82	0.00162090113161581\\
599.83	0.00153809769722781\\
599.84	0.001454485532113\\
599.85	0.00137005671969636\\
599.86	0.00128480326595493\\
599.87	0.00119871709866576\\
599.88	0.00111179006664627\\
599.89	0.00102401393898652\\
599.9	0.000935380404273357\\
599.91	0.000845881069805998\\
599.92	0.000755507460802737\\
599.93	0.000664251019598418\\
599.94	0.000572103104832229\\
599.95	0.000479054990625295\\
599.96	0.000385097865747556\\
599.97	0.000290222832773275\\
599.98	0.000194420907224489\\
599.99	9.76830167015649e-05\\
600	0\\
};
\addplot [color=red!25!mycolor17,solid,forget plot]
  table[row sep=crcr]{%
0.01	0.01\\
1.01	0.01\\
2.01	0.01\\
3.01	0.01\\
4.01	0.01\\
5.01	0.01\\
6.01	0.01\\
7.01	0.01\\
8.01	0.01\\
9.01	0.01\\
10.01	0.01\\
11.01	0.01\\
12.01	0.01\\
13.01	0.01\\
14.01	0.01\\
15.01	0.01\\
16.01	0.01\\
17.01	0.01\\
18.01	0.01\\
19.01	0.01\\
20.01	0.01\\
21.01	0.01\\
22.01	0.01\\
23.01	0.01\\
24.01	0.01\\
25.01	0.01\\
26.01	0.01\\
27.01	0.01\\
28.01	0.01\\
29.01	0.01\\
30.01	0.01\\
31.01	0.01\\
32.01	0.01\\
33.01	0.01\\
34.01	0.01\\
35.01	0.01\\
36.01	0.01\\
37.01	0.01\\
38.01	0.01\\
39.01	0.01\\
40.01	0.01\\
41.01	0.01\\
42.01	0.01\\
43.01	0.01\\
44.01	0.01\\
45.01	0.01\\
46.01	0.01\\
47.01	0.01\\
48.01	0.01\\
49.01	0.01\\
50.01	0.01\\
51.01	0.01\\
52.01	0.01\\
53.01	0.01\\
54.01	0.01\\
55.01	0.01\\
56.01	0.01\\
57.01	0.01\\
58.01	0.01\\
59.01	0.01\\
60.01	0.01\\
61.01	0.01\\
62.01	0.01\\
63.01	0.01\\
64.01	0.01\\
65.01	0.01\\
66.01	0.01\\
67.01	0.01\\
68.01	0.01\\
69.01	0.01\\
70.01	0.01\\
71.01	0.01\\
72.01	0.01\\
73.01	0.01\\
74.01	0.01\\
75.01	0.01\\
76.01	0.01\\
77.01	0.01\\
78.01	0.01\\
79.01	0.01\\
80.01	0.01\\
81.01	0.01\\
82.01	0.01\\
83.01	0.01\\
84.01	0.01\\
85.01	0.01\\
86.01	0.01\\
87.01	0.01\\
88.01	0.01\\
89.01	0.01\\
90.01	0.01\\
91.01	0.01\\
92.01	0.01\\
93.01	0.01\\
94.01	0.01\\
95.01	0.01\\
96.01	0.01\\
97.01	0.01\\
98.01	0.01\\
99.01	0.01\\
100.01	0.01\\
101.01	0.01\\
102.01	0.01\\
103.01	0.01\\
104.01	0.01\\
105.01	0.01\\
106.01	0.01\\
107.01	0.01\\
108.01	0.01\\
109.01	0.01\\
110.01	0.01\\
111.01	0.01\\
112.01	0.01\\
113.01	0.01\\
114.01	0.01\\
115.01	0.01\\
116.01	0.01\\
117.01	0.01\\
118.01	0.01\\
119.01	0.01\\
120.01	0.01\\
121.01	0.01\\
122.01	0.01\\
123.01	0.01\\
124.01	0.01\\
125.01	0.01\\
126.01	0.01\\
127.01	0.01\\
128.01	0.01\\
129.01	0.01\\
130.01	0.01\\
131.01	0.01\\
132.01	0.01\\
133.01	0.01\\
134.01	0.01\\
135.01	0.01\\
136.01	0.01\\
137.01	0.01\\
138.01	0.01\\
139.01	0.01\\
140.01	0.01\\
141.01	0.01\\
142.01	0.01\\
143.01	0.01\\
144.01	0.01\\
145.01	0.01\\
146.01	0.01\\
147.01	0.01\\
148.01	0.01\\
149.01	0.01\\
150.01	0.01\\
151.01	0.01\\
152.01	0.01\\
153.01	0.01\\
154.01	0.01\\
155.01	0.01\\
156.01	0.01\\
157.01	0.01\\
158.01	0.01\\
159.01	0.01\\
160.01	0.01\\
161.01	0.01\\
162.01	0.01\\
163.01	0.01\\
164.01	0.01\\
165.01	0.01\\
166.01	0.01\\
167.01	0.01\\
168.01	0.01\\
169.01	0.01\\
170.01	0.01\\
171.01	0.01\\
172.01	0.01\\
173.01	0.01\\
174.01	0.01\\
175.01	0.01\\
176.01	0.01\\
177.01	0.01\\
178.01	0.01\\
179.01	0.01\\
180.01	0.01\\
181.01	0.01\\
182.01	0.01\\
183.01	0.01\\
184.01	0.01\\
185.01	0.01\\
186.01	0.01\\
187.01	0.01\\
188.01	0.01\\
189.01	0.01\\
190.01	0.01\\
191.01	0.01\\
192.01	0.01\\
193.01	0.01\\
194.01	0.01\\
195.01	0.01\\
196.01	0.01\\
197.01	0.01\\
198.01	0.01\\
199.01	0.01\\
200.01	0.01\\
201.01	0.01\\
202.01	0.01\\
203.01	0.01\\
204.01	0.01\\
205.01	0.01\\
206.01	0.01\\
207.01	0.01\\
208.01	0.01\\
209.01	0.01\\
210.01	0.01\\
211.01	0.01\\
212.01	0.01\\
213.01	0.01\\
214.01	0.01\\
215.01	0.01\\
216.01	0.01\\
217.01	0.01\\
218.01	0.01\\
219.01	0.01\\
220.01	0.01\\
221.01	0.01\\
222.01	0.01\\
223.01	0.01\\
224.01	0.01\\
225.01	0.01\\
226.01	0.01\\
227.01	0.01\\
228.01	0.01\\
229.01	0.01\\
230.01	0.01\\
231.01	0.01\\
232.01	0.01\\
233.01	0.01\\
234.01	0.01\\
235.01	0.01\\
236.01	0.01\\
237.01	0.01\\
238.01	0.01\\
239.01	0.01\\
240.01	0.01\\
241.01	0.01\\
242.01	0.01\\
243.01	0.01\\
244.01	0.01\\
245.01	0.01\\
246.01	0.01\\
247.01	0.01\\
248.01	0.01\\
249.01	0.01\\
250.01	0.01\\
251.01	0.01\\
252.01	0.01\\
253.01	0.01\\
254.01	0.01\\
255.01	0.01\\
256.01	0.01\\
257.01	0.01\\
258.01	0.01\\
259.01	0.01\\
260.01	0.01\\
261.01	0.01\\
262.01	0.01\\
263.01	0.01\\
264.01	0.01\\
265.01	0.01\\
266.01	0.01\\
267.01	0.01\\
268.01	0.01\\
269.01	0.01\\
270.01	0.01\\
271.01	0.01\\
272.01	0.01\\
273.01	0.01\\
274.01	0.01\\
275.01	0.01\\
276.01	0.01\\
277.01	0.01\\
278.01	0.01\\
279.01	0.01\\
280.01	0.01\\
281.01	0.01\\
282.01	0.01\\
283.01	0.01\\
284.01	0.01\\
285.01	0.01\\
286.01	0.01\\
287.01	0.01\\
288.01	0.01\\
289.01	0.01\\
290.01	0.01\\
291.01	0.01\\
292.01	0.01\\
293.01	0.01\\
294.01	0.01\\
295.01	0.01\\
296.01	0.01\\
297.01	0.01\\
298.01	0.01\\
299.01	0.01\\
300.01	0.01\\
301.01	0.01\\
302.01	0.01\\
303.01	0.01\\
304.01	0.01\\
305.01	0.01\\
306.01	0.01\\
307.01	0.01\\
308.01	0.01\\
309.01	0.01\\
310.01	0.01\\
311.01	0.01\\
312.01	0.01\\
313.01	0.01\\
314.01	0.01\\
315.01	0.01\\
316.01	0.01\\
317.01	0.01\\
318.01	0.01\\
319.01	0.01\\
320.01	0.01\\
321.01	0.01\\
322.01	0.01\\
323.01	0.01\\
324.01	0.01\\
325.01	0.01\\
326.01	0.01\\
327.01	0.01\\
328.01	0.01\\
329.01	0.01\\
330.01	0.01\\
331.01	0.01\\
332.01	0.01\\
333.01	0.01\\
334.01	0.01\\
335.01	0.01\\
336.01	0.01\\
337.01	0.01\\
338.01	0.01\\
339.01	0.01\\
340.01	0.01\\
341.01	0.01\\
342.01	0.01\\
343.01	0.01\\
344.01	0.01\\
345.01	0.01\\
346.01	0.01\\
347.01	0.01\\
348.01	0.01\\
349.01	0.01\\
350.01	0.01\\
351.01	0.01\\
352.01	0.01\\
353.01	0.01\\
354.01	0.01\\
355.01	0.01\\
356.01	0.01\\
357.01	0.01\\
358.01	0.01\\
359.01	0.01\\
360.01	0.01\\
361.01	0.01\\
362.01	0.01\\
363.01	0.01\\
364.01	0.01\\
365.01	0.01\\
366.01	0.01\\
367.01	0.01\\
368.01	0.01\\
369.01	0.01\\
370.01	0.01\\
371.01	0.01\\
372.01	0.01\\
373.01	0.01\\
374.01	0.01\\
375.01	0.01\\
376.01	0.01\\
377.01	0.01\\
378.01	0.01\\
379.01	0.01\\
380.01	0.01\\
381.01	0.01\\
382.01	0.01\\
383.01	0.01\\
384.01	0.01\\
385.01	0.01\\
386.01	0.01\\
387.01	0.01\\
388.01	0.01\\
389.01	0.01\\
390.01	0.01\\
391.01	0.01\\
392.01	0.01\\
393.01	0.01\\
394.01	0.01\\
395.01	0.01\\
396.01	0.01\\
397.01	0.01\\
398.01	0.01\\
399.01	0.01\\
400.01	0.01\\
401.01	0.01\\
402.01	0.01\\
403.01	0.01\\
404.01	0.01\\
405.01	0.01\\
406.01	0.01\\
407.01	0.01\\
408.01	0.01\\
409.01	0.01\\
410.01	0.01\\
411.01	0.01\\
412.01	0.01\\
413.01	0.01\\
414.01	0.01\\
415.01	0.01\\
416.01	0.01\\
417.01	0.01\\
418.01	0.01\\
419.01	0.01\\
420.01	0.01\\
421.01	0.01\\
422.01	0.01\\
423.01	0.01\\
424.01	0.01\\
425.01	0.01\\
426.01	0.01\\
427.01	0.01\\
428.01	0.01\\
429.01	0.01\\
430.01	0.01\\
431.01	0.01\\
432.01	0.01\\
433.01	0.01\\
434.01	0.01\\
435.01	0.01\\
436.01	0.01\\
437.01	0.01\\
438.01	0.01\\
439.01	0.01\\
440.01	0.01\\
441.01	0.01\\
442.01	0.01\\
443.01	0.01\\
444.01	0.01\\
445.01	0.01\\
446.01	0.01\\
447.01	0.01\\
448.01	0.01\\
449.01	0.01\\
450.01	0.01\\
451.01	0.01\\
452.01	0.01\\
453.01	0.01\\
454.01	0.01\\
455.01	0.01\\
456.01	0.01\\
457.01	0.01\\
458.01	0.01\\
459.01	0.01\\
460.01	0.01\\
461.01	0.01\\
462.01	0.01\\
463.01	0.01\\
464.01	0.01\\
465.01	0.01\\
466.01	0.01\\
467.01	0.01\\
468.01	0.01\\
469.01	0.01\\
470.01	0.01\\
471.01	0.01\\
472.01	0.01\\
473.01	0.01\\
474.01	0.01\\
475.01	0.01\\
476.01	0.01\\
477.01	0.01\\
478.01	0.01\\
479.01	0.01\\
480.01	0.01\\
481.01	0.01\\
482.01	0.01\\
483.01	0.01\\
484.01	0.01\\
485.01	0.01\\
486.01	0.01\\
487.01	0.01\\
488.01	0.01\\
489.01	0.01\\
490.01	0.01\\
491.01	0.01\\
492.01	0.01\\
493.01	0.01\\
494.01	0.01\\
495.01	0.01\\
496.01	0.01\\
497.01	0.01\\
498.01	0.01\\
499.01	0.01\\
500.01	0.01\\
501.01	0.01\\
502.01	0.01\\
503.01	0.01\\
504.01	0.01\\
505.01	0.01\\
506.01	0.01\\
507.01	0.01\\
508.01	0.01\\
509.01	0.01\\
510.01	0.01\\
511.01	0.01\\
512.01	0.01\\
513.01	0.01\\
514.01	0.01\\
515.01	0.01\\
516.01	0.01\\
517.01	0.01\\
518.01	0.01\\
519.01	0.01\\
520.01	0.01\\
521.01	0.01\\
522.01	0.01\\
523.01	0.01\\
524.01	0.01\\
525.01	0.01\\
526.01	0.01\\
527.01	0.01\\
528.01	0.01\\
529.01	0.01\\
530.01	0.01\\
531.01	0.01\\
532.01	0.01\\
533.01	0.01\\
534.01	0.01\\
535.01	0.01\\
536.01	0.01\\
537.01	0.01\\
538.01	0.01\\
539.01	0.01\\
540.01	0.01\\
541.01	0.01\\
542.01	0.01\\
543.01	0.01\\
544.01	0.01\\
545.01	0.01\\
546.01	0.01\\
547.01	0.01\\
548.01	0.01\\
549.01	0.01\\
550.01	0.01\\
551.01	0.01\\
552.01	0.01\\
553.01	0.01\\
554.01	0.01\\
555.01	0.01\\
556.01	0.01\\
557.01	0.01\\
558.01	0.01\\
559.01	0.01\\
560.01	0.01\\
561.01	0.01\\
562.01	0.01\\
563.01	0.01\\
564.01	0.01\\
565.01	0.01\\
566.01	0.01\\
567.01	0.01\\
568.01	0.01\\
569.01	0.01\\
570.01	0.01\\
571.01	0.01\\
572.01	0.01\\
573.01	0.01\\
574.01	0.01\\
575.01	0.01\\
576.01	0.01\\
577.01	0.01\\
578.01	0.01\\
579.01	0.01\\
580.01	0.01\\
581.01	0.01\\
582.01	0.01\\
583.01	0.01\\
584.01	0.01\\
585.01	0.01\\
586.01	0.01\\
587.01	0.01\\
588.01	0.01\\
589.01	0.01\\
590.01	0.01\\
591.01	0.01\\
592.01	0.01\\
593.01	0.01\\
594.01	0.01\\
595.01	0.01\\
596.01	0.01\\
597.01	0.01\\
598.01	0.01\\
599.01	0.00624187794319273\\
599.02	0.00620415217318548\\
599.03	0.00616605955071492\\
599.04	0.00612759646350277\\
599.05	0.00608875926364123\\
599.06	0.00604954426731568\\
599.07	0.00600994775453456\\
599.08	0.00596996596881047\\
599.09	0.0059295951167174\\
599.1	0.00588883136760121\\
599.11	0.0058476708533095\\
599.12	0.00580610966792835\\
599.13	0.00576414386751836\\
599.14	0.00572176946987378\\
599.15	0.00567898245429428\\
599.16	0.00563577876137102\\
599.17	0.00559215429278665\\
599.18	0.0055481049111301\\
599.19	0.00550362643973364\\
599.2	0.00545871466253047\\
599.21	0.00541336532313308\\
599.22	0.00536757412333666\\
599.23	0.00532133672270926\\
599.24	0.00527464873817792\\
599.25	0.00522750574361058\\
599.26	0.00517990326939395\\
599.27	0.00513183680200728\\
599.28	0.00508330178359264\\
599.29	0.00503429361152088\\
599.3	0.00498480763795339\\
599.31	0.00493483916939956\\
599.32	0.00488438346626986\\
599.33	0.00483343574242454\\
599.34	0.00478199116471782\\
599.35	0.00473004485253745\\
599.36	0.0046775918773397\\
599.37	0.00462462726217956\\
599.38	0.00457114598123612\\
599.39	0.00451714296330743\\
599.4	0.00446261310174465\\
599.41	0.00440755123993145\\
599.42	0.00435195217079425\\
599.43	0.00429581063630734\\
599.44	0.00423912132699231\\
599.45	0.00418187888141208\\
599.46	0.00412407788565892\\
599.47	0.00406571287283668\\
599.48	0.00400677832253673\\
599.49	0.00394726866031217\\
599.5	0.00388717825714887\\
599.51	0.00382650142893142\\
599.52	0.00376523243590372\\
599.53	0.00370336548212449\\
599.54	0.00364089471491734\\
599.55	0.0035778142243154\\
599.56	0.00351411804250074\\
599.57	0.00344980014323806\\
599.58	0.00338485444130302\\
599.59	0.00331927479190491\\
599.6	0.00325305499010369\\
599.61	0.00318618877022133\\
599.62	0.00311866980524735\\
599.63	0.00305049170623868\\
599.64	0.00298164802171346\\
599.65	0.00291213223703912\\
599.66	0.00284193777381426\\
599.67	0.0027710579892447\\
599.68	0.00269948617551322\\
599.69	0.00262721555914333\\
599.7	0.00255423930035664\\
599.71	0.00248055049242407\\
599.72	0.00240614216101062\\
599.73	0.0023310072635138\\
599.74	0.00225513868839553\\
599.75	0.0021785292545075\\
599.76	0.00210117171040993\\
599.77	0.0020230587336837\\
599.78	0.00194418293023563\\
599.79	0.00186453683359705\\
599.8	0.00178411290421546\\
599.81	0.00170290352873919\\
599.82	0.00162090101929515\\
599.83	0.00153809761275938\\
599.84	0.00145448547002052\\
599.85	0.00137005667523597\\
599.86	0.00128480323508086\\
599.87	0.00119871707798955\\
599.88	0.00111179005338966\\
599.89	0.00102401393092871\\
599.9	0.000935380399693021\\
599.91	0.000845881067419046\\
599.92	0.000755507459696892\\
599.93	0.00066425101916609\\
599.94	0.000572103104703438\\
599.95	0.000479054990602912\\
599.96	0.000385097865747556\\
599.97	0.000290222832773275\\
599.98	0.000194420907224489\\
599.99	9.76830167015649e-05\\
600	0\\
};
\addplot [color=mycolor19,solid,forget plot]
  table[row sep=crcr]{%
0.01	0.01\\
1.01	0.01\\
2.01	0.01\\
3.01	0.01\\
4.01	0.01\\
5.01	0.01\\
6.01	0.01\\
7.01	0.01\\
8.01	0.01\\
9.01	0.01\\
10.01	0.01\\
11.01	0.01\\
12.01	0.01\\
13.01	0.01\\
14.01	0.01\\
15.01	0.01\\
16.01	0.01\\
17.01	0.01\\
18.01	0.01\\
19.01	0.01\\
20.01	0.01\\
21.01	0.01\\
22.01	0.01\\
23.01	0.01\\
24.01	0.01\\
25.01	0.01\\
26.01	0.01\\
27.01	0.01\\
28.01	0.01\\
29.01	0.01\\
30.01	0.01\\
31.01	0.01\\
32.01	0.01\\
33.01	0.01\\
34.01	0.01\\
35.01	0.01\\
36.01	0.01\\
37.01	0.01\\
38.01	0.01\\
39.01	0.01\\
40.01	0.01\\
41.01	0.01\\
42.01	0.01\\
43.01	0.01\\
44.01	0.01\\
45.01	0.01\\
46.01	0.01\\
47.01	0.01\\
48.01	0.01\\
49.01	0.01\\
50.01	0.01\\
51.01	0.01\\
52.01	0.01\\
53.01	0.01\\
54.01	0.01\\
55.01	0.01\\
56.01	0.01\\
57.01	0.01\\
58.01	0.01\\
59.01	0.01\\
60.01	0.01\\
61.01	0.01\\
62.01	0.01\\
63.01	0.01\\
64.01	0.01\\
65.01	0.01\\
66.01	0.01\\
67.01	0.01\\
68.01	0.01\\
69.01	0.01\\
70.01	0.01\\
71.01	0.01\\
72.01	0.01\\
73.01	0.01\\
74.01	0.01\\
75.01	0.01\\
76.01	0.01\\
77.01	0.01\\
78.01	0.01\\
79.01	0.01\\
80.01	0.01\\
81.01	0.01\\
82.01	0.01\\
83.01	0.01\\
84.01	0.01\\
85.01	0.01\\
86.01	0.01\\
87.01	0.01\\
88.01	0.01\\
89.01	0.01\\
90.01	0.01\\
91.01	0.01\\
92.01	0.01\\
93.01	0.01\\
94.01	0.01\\
95.01	0.01\\
96.01	0.01\\
97.01	0.01\\
98.01	0.01\\
99.01	0.01\\
100.01	0.01\\
101.01	0.01\\
102.01	0.01\\
103.01	0.01\\
104.01	0.01\\
105.01	0.01\\
106.01	0.01\\
107.01	0.01\\
108.01	0.01\\
109.01	0.01\\
110.01	0.01\\
111.01	0.01\\
112.01	0.01\\
113.01	0.01\\
114.01	0.01\\
115.01	0.01\\
116.01	0.01\\
117.01	0.01\\
118.01	0.01\\
119.01	0.01\\
120.01	0.01\\
121.01	0.01\\
122.01	0.01\\
123.01	0.01\\
124.01	0.01\\
125.01	0.01\\
126.01	0.01\\
127.01	0.01\\
128.01	0.01\\
129.01	0.01\\
130.01	0.01\\
131.01	0.01\\
132.01	0.01\\
133.01	0.01\\
134.01	0.01\\
135.01	0.01\\
136.01	0.01\\
137.01	0.01\\
138.01	0.01\\
139.01	0.01\\
140.01	0.01\\
141.01	0.01\\
142.01	0.01\\
143.01	0.01\\
144.01	0.01\\
145.01	0.01\\
146.01	0.01\\
147.01	0.01\\
148.01	0.01\\
149.01	0.01\\
150.01	0.01\\
151.01	0.01\\
152.01	0.01\\
153.01	0.01\\
154.01	0.01\\
155.01	0.01\\
156.01	0.01\\
157.01	0.01\\
158.01	0.01\\
159.01	0.01\\
160.01	0.01\\
161.01	0.01\\
162.01	0.01\\
163.01	0.01\\
164.01	0.01\\
165.01	0.01\\
166.01	0.01\\
167.01	0.01\\
168.01	0.01\\
169.01	0.01\\
170.01	0.01\\
171.01	0.01\\
172.01	0.01\\
173.01	0.01\\
174.01	0.01\\
175.01	0.01\\
176.01	0.01\\
177.01	0.01\\
178.01	0.01\\
179.01	0.01\\
180.01	0.01\\
181.01	0.01\\
182.01	0.01\\
183.01	0.01\\
184.01	0.01\\
185.01	0.01\\
186.01	0.01\\
187.01	0.01\\
188.01	0.01\\
189.01	0.01\\
190.01	0.01\\
191.01	0.01\\
192.01	0.01\\
193.01	0.01\\
194.01	0.01\\
195.01	0.01\\
196.01	0.01\\
197.01	0.01\\
198.01	0.01\\
199.01	0.01\\
200.01	0.01\\
201.01	0.01\\
202.01	0.01\\
203.01	0.01\\
204.01	0.01\\
205.01	0.01\\
206.01	0.01\\
207.01	0.01\\
208.01	0.01\\
209.01	0.01\\
210.01	0.01\\
211.01	0.01\\
212.01	0.01\\
213.01	0.01\\
214.01	0.01\\
215.01	0.01\\
216.01	0.01\\
217.01	0.01\\
218.01	0.01\\
219.01	0.01\\
220.01	0.01\\
221.01	0.01\\
222.01	0.01\\
223.01	0.01\\
224.01	0.01\\
225.01	0.01\\
226.01	0.01\\
227.01	0.01\\
228.01	0.01\\
229.01	0.01\\
230.01	0.01\\
231.01	0.01\\
232.01	0.01\\
233.01	0.01\\
234.01	0.01\\
235.01	0.01\\
236.01	0.01\\
237.01	0.01\\
238.01	0.01\\
239.01	0.01\\
240.01	0.01\\
241.01	0.01\\
242.01	0.01\\
243.01	0.01\\
244.01	0.01\\
245.01	0.01\\
246.01	0.01\\
247.01	0.01\\
248.01	0.01\\
249.01	0.01\\
250.01	0.01\\
251.01	0.01\\
252.01	0.01\\
253.01	0.01\\
254.01	0.01\\
255.01	0.01\\
256.01	0.01\\
257.01	0.01\\
258.01	0.01\\
259.01	0.01\\
260.01	0.01\\
261.01	0.01\\
262.01	0.01\\
263.01	0.01\\
264.01	0.01\\
265.01	0.01\\
266.01	0.01\\
267.01	0.01\\
268.01	0.01\\
269.01	0.01\\
270.01	0.01\\
271.01	0.01\\
272.01	0.01\\
273.01	0.01\\
274.01	0.01\\
275.01	0.01\\
276.01	0.01\\
277.01	0.01\\
278.01	0.01\\
279.01	0.01\\
280.01	0.01\\
281.01	0.01\\
282.01	0.01\\
283.01	0.01\\
284.01	0.01\\
285.01	0.01\\
286.01	0.01\\
287.01	0.01\\
288.01	0.01\\
289.01	0.01\\
290.01	0.01\\
291.01	0.01\\
292.01	0.01\\
293.01	0.01\\
294.01	0.01\\
295.01	0.01\\
296.01	0.01\\
297.01	0.01\\
298.01	0.01\\
299.01	0.01\\
300.01	0.01\\
301.01	0.01\\
302.01	0.01\\
303.01	0.01\\
304.01	0.01\\
305.01	0.01\\
306.01	0.01\\
307.01	0.01\\
308.01	0.01\\
309.01	0.01\\
310.01	0.01\\
311.01	0.01\\
312.01	0.01\\
313.01	0.01\\
314.01	0.01\\
315.01	0.01\\
316.01	0.01\\
317.01	0.01\\
318.01	0.01\\
319.01	0.01\\
320.01	0.01\\
321.01	0.01\\
322.01	0.01\\
323.01	0.01\\
324.01	0.01\\
325.01	0.01\\
326.01	0.01\\
327.01	0.01\\
328.01	0.01\\
329.01	0.01\\
330.01	0.01\\
331.01	0.01\\
332.01	0.01\\
333.01	0.01\\
334.01	0.01\\
335.01	0.01\\
336.01	0.01\\
337.01	0.01\\
338.01	0.01\\
339.01	0.01\\
340.01	0.01\\
341.01	0.01\\
342.01	0.01\\
343.01	0.01\\
344.01	0.01\\
345.01	0.01\\
346.01	0.01\\
347.01	0.01\\
348.01	0.01\\
349.01	0.01\\
350.01	0.01\\
351.01	0.01\\
352.01	0.01\\
353.01	0.01\\
354.01	0.01\\
355.01	0.01\\
356.01	0.01\\
357.01	0.01\\
358.01	0.01\\
359.01	0.01\\
360.01	0.01\\
361.01	0.01\\
362.01	0.01\\
363.01	0.01\\
364.01	0.01\\
365.01	0.01\\
366.01	0.01\\
367.01	0.01\\
368.01	0.01\\
369.01	0.01\\
370.01	0.01\\
371.01	0.01\\
372.01	0.01\\
373.01	0.01\\
374.01	0.01\\
375.01	0.01\\
376.01	0.01\\
377.01	0.01\\
378.01	0.01\\
379.01	0.01\\
380.01	0.01\\
381.01	0.01\\
382.01	0.01\\
383.01	0.01\\
384.01	0.01\\
385.01	0.01\\
386.01	0.01\\
387.01	0.01\\
388.01	0.01\\
389.01	0.01\\
390.01	0.01\\
391.01	0.01\\
392.01	0.01\\
393.01	0.01\\
394.01	0.01\\
395.01	0.01\\
396.01	0.01\\
397.01	0.01\\
398.01	0.01\\
399.01	0.01\\
400.01	0.01\\
401.01	0.01\\
402.01	0.01\\
403.01	0.01\\
404.01	0.01\\
405.01	0.01\\
406.01	0.01\\
407.01	0.01\\
408.01	0.01\\
409.01	0.01\\
410.01	0.01\\
411.01	0.01\\
412.01	0.01\\
413.01	0.01\\
414.01	0.01\\
415.01	0.01\\
416.01	0.01\\
417.01	0.01\\
418.01	0.01\\
419.01	0.01\\
420.01	0.01\\
421.01	0.01\\
422.01	0.01\\
423.01	0.01\\
424.01	0.01\\
425.01	0.01\\
426.01	0.01\\
427.01	0.01\\
428.01	0.01\\
429.01	0.01\\
430.01	0.01\\
431.01	0.01\\
432.01	0.01\\
433.01	0.01\\
434.01	0.01\\
435.01	0.01\\
436.01	0.01\\
437.01	0.01\\
438.01	0.01\\
439.01	0.01\\
440.01	0.01\\
441.01	0.01\\
442.01	0.01\\
443.01	0.01\\
444.01	0.01\\
445.01	0.01\\
446.01	0.01\\
447.01	0.01\\
448.01	0.01\\
449.01	0.01\\
450.01	0.01\\
451.01	0.01\\
452.01	0.01\\
453.01	0.01\\
454.01	0.01\\
455.01	0.01\\
456.01	0.01\\
457.01	0.01\\
458.01	0.01\\
459.01	0.01\\
460.01	0.01\\
461.01	0.01\\
462.01	0.01\\
463.01	0.01\\
464.01	0.01\\
465.01	0.01\\
466.01	0.01\\
467.01	0.01\\
468.01	0.01\\
469.01	0.01\\
470.01	0.01\\
471.01	0.01\\
472.01	0.01\\
473.01	0.01\\
474.01	0.01\\
475.01	0.01\\
476.01	0.01\\
477.01	0.01\\
478.01	0.01\\
479.01	0.01\\
480.01	0.01\\
481.01	0.01\\
482.01	0.01\\
483.01	0.01\\
484.01	0.01\\
485.01	0.01\\
486.01	0.01\\
487.01	0.01\\
488.01	0.01\\
489.01	0.01\\
490.01	0.01\\
491.01	0.01\\
492.01	0.01\\
493.01	0.01\\
494.01	0.01\\
495.01	0.01\\
496.01	0.01\\
497.01	0.01\\
498.01	0.01\\
499.01	0.01\\
500.01	0.01\\
501.01	0.01\\
502.01	0.01\\
503.01	0.01\\
504.01	0.01\\
505.01	0.01\\
506.01	0.01\\
507.01	0.01\\
508.01	0.01\\
509.01	0.01\\
510.01	0.01\\
511.01	0.01\\
512.01	0.01\\
513.01	0.01\\
514.01	0.01\\
515.01	0.01\\
516.01	0.01\\
517.01	0.01\\
518.01	0.01\\
519.01	0.01\\
520.01	0.01\\
521.01	0.01\\
522.01	0.01\\
523.01	0.01\\
524.01	0.01\\
525.01	0.01\\
526.01	0.01\\
527.01	0.01\\
528.01	0.01\\
529.01	0.01\\
530.01	0.01\\
531.01	0.01\\
532.01	0.01\\
533.01	0.01\\
534.01	0.01\\
535.01	0.01\\
536.01	0.01\\
537.01	0.01\\
538.01	0.01\\
539.01	0.01\\
540.01	0.01\\
541.01	0.01\\
542.01	0.01\\
543.01	0.01\\
544.01	0.01\\
545.01	0.01\\
546.01	0.01\\
547.01	0.01\\
548.01	0.01\\
549.01	0.01\\
550.01	0.01\\
551.01	0.01\\
552.01	0.01\\
553.01	0.01\\
554.01	0.01\\
555.01	0.01\\
556.01	0.01\\
557.01	0.01\\
558.01	0.01\\
559.01	0.01\\
560.01	0.01\\
561.01	0.01\\
562.01	0.01\\
563.01	0.01\\
564.01	0.01\\
565.01	0.01\\
566.01	0.01\\
567.01	0.01\\
568.01	0.01\\
569.01	0.01\\
570.01	0.01\\
571.01	0.01\\
572.01	0.01\\
573.01	0.01\\
574.01	0.01\\
575.01	0.01\\
576.01	0.01\\
577.01	0.01\\
578.01	0.01\\
579.01	0.01\\
580.01	0.01\\
581.01	0.01\\
582.01	0.01\\
583.01	0.01\\
584.01	0.01\\
585.01	0.01\\
586.01	0.01\\
587.01	0.01\\
588.01	0.01\\
589.01	0.01\\
590.01	0.01\\
591.01	0.01\\
592.01	0.01\\
593.01	0.01\\
594.01	0.01\\
595.01	0.01\\
596.01	0.01\\
597.01	0.01\\
598.01	0.01\\
599.01	0.00624186921776351\\
599.02	0.00620414452392774\\
599.03	0.00616605285708784\\
599.04	0.0061275906126133\\
599.05	0.00608875415047225\\
599.06	0.00604953979488413\\
599.07	0.00600994383396897\\
599.08	0.00596996251939311\\
599.09	0.00592959206601176\\
599.1	0.00588882865150792\\
599.11	0.00584766841602762\\
599.12	0.00580610746181134\\
599.13	0.00576414185282154\\
599.14	0.00572176761436633\\
599.15	0.00567898073271903\\
599.16	0.00563577715473362\\
599.17	0.00559215278745594\\
599.18	0.00554810349773059\\
599.19	0.00550362511180335\\
599.2	0.00545871341491899\\
599.21	0.00541336415091619\\
599.22	0.00536757302182078\\
599.23	0.005321335687435\\
599.24	0.00527464776492264\\
599.25	0.00522750482839026\\
599.26	0.00517990240846415\\
599.27	0.0051318359918632\\
599.28	0.00508330102096758\\
599.29	0.00503429289338311\\
599.3	0.00498480696150149\\
599.31	0.004934838532056\\
599.32	0.00488438286567301\\
599.33	0.00483343517641901\\
599.34	0.00478199063134318\\
599.35	0.00473004435001546\\
599.36	0.0046775914040601\\
599.37	0.00462462681668464\\
599.38	0.00457114556220417\\
599.39	0.00451714256953565\\
599.4	0.00446261273213111\\
599.41	0.0044075508934571\\
599.42	0.0043519518465054\\
599.43	0.00429581033329887\\
599.44	0.00423912104439258\\
599.45	0.00418187861836991\\
599.46	0.00412407764133385\\
599.47	0.00406571264639331\\
599.48	0.00400677811314435\\
599.49	0.00394726846714635\\
599.5	0.00388717807939316\\
599.51	0.00382650126577892\\
599.52	0.00376523228655877\\
599.53	0.00370336534580421\\
599.54	0.00364089459085318\\
599.55	0.00357781411175471\\
599.56	0.00351411794070817\\
599.57	0.00344980005149701\\
599.58	0.00338485435891695\\
599.59	0.00331927471819867\\
599.6	0.00325305492442463\\
599.61	0.00318618871194044\\
599.62	0.00311866975376029\\
599.63	0.00305049166096664\\
599.64	0.00298164798210398\\
599.65	0.00291213220256672\\
599.66	0.00284193774398107\\
599.67	0.00277105796358078\\
599.68	0.00269948615357693\\
599.69	0.00262721554052142\\
599.7	0.00255423928466429\\
599.71	0.00248055047930473\\
599.72	0.00240614215013571\\
599.73	0.00233100725458232\\
599.74	0.0022551386811334\\
599.75	0.0021785292486669\\
599.76	0.00210117170576842\\
599.77	0.00202305873004318\\
599.78	0.00194418292742126\\
599.79	0.00186453683145598\\
599.8	0.00178411290261548\\
599.81	0.00170290352756733\\
599.82	0.00162090101845609\\
599.83	0.00153809761217394\\
599.84	0.00145448546962399\\
599.85	0.00137005667497651\\
599.86	0.00128480323491787\\
599.87	0.00119871707789201\\
599.88	0.00111179005333465\\
599.89	0.0010240139308999\\
599.9	0.000935380399679301\\
599.91	0.000845881067413297\\
599.92	0.000755507459694884\\
599.93	0.000664251019165563\\
599.94	0.000572103104703356\\
599.95	0.000479054990602912\\
599.96	0.000385097865747556\\
599.97	0.000290222832773275\\
599.98	0.000194420907224489\\
599.99	9.76830167015649e-05\\
600	0\\
};
\addplot [color=red!50!mycolor17,solid,forget plot]
  table[row sep=crcr]{%
0.01	0.00999999999999999\\
1.01	0.00999999999999999\\
2.01	0.00999999999999999\\
3.01	0.00999999999999999\\
4.01	0.00999999999999999\\
5.01	0.00999999999999999\\
6.01	0.00999999999999999\\
7.01	0.00999999999999999\\
8.01	0.00999999999999999\\
9.01	0.00999999999999999\\
10.01	0.00999999999999999\\
11.01	0.00999999999999999\\
12.01	0.00999999999999999\\
13.01	0.00999999999999999\\
14.01	0.00999999999999999\\
15.01	0.00999999999999999\\
16.01	0.00999999999999999\\
17.01	0.00999999999999999\\
18.01	0.00999999999999999\\
19.01	0.00999999999999999\\
20.01	0.00999999999999999\\
21.01	0.00999999999999999\\
22.01	0.00999999999999999\\
23.01	0.00999999999999999\\
24.01	0.00999999999999999\\
25.01	0.00999999999999999\\
26.01	0.00999999999999999\\
27.01	0.00999999999999999\\
28.01	0.00999999999999999\\
29.01	0.00999999999999999\\
30.01	0.00999999999999999\\
31.01	0.00999999999999999\\
32.01	0.00999999999999999\\
33.01	0.00999999999999999\\
34.01	0.00999999999999999\\
35.01	0.00999999999999999\\
36.01	0.00999999999999999\\
37.01	0.00999999999999999\\
38.01	0.00999999999999999\\
39.01	0.00999999999999999\\
40.01	0.00999999999999999\\
41.01	0.00999999999999999\\
42.01	0.00999999999999999\\
43.01	0.00999999999999999\\
44.01	0.00999999999999999\\
45.01	0.00999999999999999\\
46.01	0.00999999999999999\\
47.01	0.00999999999999999\\
48.01	0.00999999999999999\\
49.01	0.00999999999999999\\
50.01	0.00999999999999999\\
51.01	0.00999999999999999\\
52.01	0.00999999999999999\\
53.01	0.00999999999999999\\
54.01	0.00999999999999999\\
55.01	0.00999999999999999\\
56.01	0.00999999999999999\\
57.01	0.00999999999999999\\
58.01	0.00999999999999999\\
59.01	0.00999999999999999\\
60.01	0.00999999999999999\\
61.01	0.00999999999999999\\
62.01	0.00999999999999999\\
63.01	0.00999999999999999\\
64.01	0.00999999999999999\\
65.01	0.00999999999999999\\
66.01	0.00999999999999999\\
67.01	0.00999999999999999\\
68.01	0.00999999999999999\\
69.01	0.00999999999999999\\
70.01	0.00999999999999999\\
71.01	0.00999999999999999\\
72.01	0.00999999999999999\\
73.01	0.00999999999999999\\
74.01	0.00999999999999999\\
75.01	0.00999999999999999\\
76.01	0.00999999999999999\\
77.01	0.00999999999999999\\
78.01	0.00999999999999999\\
79.01	0.00999999999999999\\
80.01	0.00999999999999999\\
81.01	0.00999999999999999\\
82.01	0.00999999999999999\\
83.01	0.00999999999999999\\
84.01	0.00999999999999999\\
85.01	0.00999999999999999\\
86.01	0.00999999999999999\\
87.01	0.00999999999999999\\
88.01	0.00999999999999999\\
89.01	0.00999999999999999\\
90.01	0.00999999999999999\\
91.01	0.00999999999999999\\
92.01	0.00999999999999999\\
93.01	0.00999999999999999\\
94.01	0.00999999999999999\\
95.01	0.00999999999999999\\
96.01	0.00999999999999999\\
97.01	0.00999999999999999\\
98.01	0.00999999999999999\\
99.01	0.00999999999999999\\
100.01	0.00999999999999999\\
101.01	0.00999999999999999\\
102.01	0.00999999999999999\\
103.01	0.00999999999999999\\
104.01	0.00999999999999999\\
105.01	0.00999999999999999\\
106.01	0.00999999999999999\\
107.01	0.00999999999999999\\
108.01	0.00999999999999999\\
109.01	0.00999999999999999\\
110.01	0.00999999999999999\\
111.01	0.00999999999999999\\
112.01	0.00999999999999999\\
113.01	0.00999999999999999\\
114.01	0.00999999999999999\\
115.01	0.00999999999999999\\
116.01	0.00999999999999999\\
117.01	0.00999999999999999\\
118.01	0.00999999999999999\\
119.01	0.00999999999999999\\
120.01	0.00999999999999999\\
121.01	0.00999999999999999\\
122.01	0.00999999999999999\\
123.01	0.00999999999999999\\
124.01	0.00999999999999999\\
125.01	0.00999999999999999\\
126.01	0.00999999999999999\\
127.01	0.00999999999999999\\
128.01	0.00999999999999999\\
129.01	0.00999999999999999\\
130.01	0.00999999999999999\\
131.01	0.00999999999999999\\
132.01	0.00999999999999999\\
133.01	0.00999999999999999\\
134.01	0.00999999999999999\\
135.01	0.00999999999999999\\
136.01	0.00999999999999999\\
137.01	0.00999999999999999\\
138.01	0.00999999999999999\\
139.01	0.00999999999999999\\
140.01	0.00999999999999999\\
141.01	0.00999999999999999\\
142.01	0.00999999999999999\\
143.01	0.00999999999999999\\
144.01	0.00999999999999999\\
145.01	0.00999999999999999\\
146.01	0.00999999999999999\\
147.01	0.00999999999999999\\
148.01	0.00999999999999999\\
149.01	0.00999999999999999\\
150.01	0.00999999999999999\\
151.01	0.00999999999999999\\
152.01	0.00999999999999999\\
153.01	0.00999999999999999\\
154.01	0.00999999999999999\\
155.01	0.00999999999999999\\
156.01	0.00999999999999999\\
157.01	0.00999999999999999\\
158.01	0.00999999999999999\\
159.01	0.00999999999999999\\
160.01	0.00999999999999999\\
161.01	0.00999999999999999\\
162.01	0.00999999999999999\\
163.01	0.00999999999999999\\
164.01	0.00999999999999999\\
165.01	0.00999999999999999\\
166.01	0.00999999999999999\\
167.01	0.00999999999999999\\
168.01	0.00999999999999999\\
169.01	0.00999999999999999\\
170.01	0.00999999999999999\\
171.01	0.00999999999999999\\
172.01	0.00999999999999999\\
173.01	0.00999999999999999\\
174.01	0.00999999999999999\\
175.01	0.00999999999999999\\
176.01	0.00999999999999999\\
177.01	0.00999999999999999\\
178.01	0.00999999999999999\\
179.01	0.00999999999999999\\
180.01	0.00999999999999999\\
181.01	0.00999999999999999\\
182.01	0.00999999999999999\\
183.01	0.00999999999999999\\
184.01	0.00999999999999999\\
185.01	0.00999999999999999\\
186.01	0.00999999999999999\\
187.01	0.00999999999999999\\
188.01	0.00999999999999999\\
189.01	0.00999999999999999\\
190.01	0.00999999999999999\\
191.01	0.00999999999999999\\
192.01	0.00999999999999999\\
193.01	0.00999999999999999\\
194.01	0.00999999999999999\\
195.01	0.00999999999999999\\
196.01	0.00999999999999999\\
197.01	0.00999999999999999\\
198.01	0.00999999999999999\\
199.01	0.00999999999999999\\
200.01	0.00999999999999999\\
201.01	0.00999999999999999\\
202.01	0.00999999999999999\\
203.01	0.00999999999999999\\
204.01	0.00999999999999999\\
205.01	0.00999999999999999\\
206.01	0.00999999999999999\\
207.01	0.00999999999999999\\
208.01	0.00999999999999999\\
209.01	0.00999999999999999\\
210.01	0.00999999999999999\\
211.01	0.00999999999999999\\
212.01	0.00999999999999999\\
213.01	0.00999999999999999\\
214.01	0.00999999999999999\\
215.01	0.00999999999999999\\
216.01	0.00999999999999999\\
217.01	0.00999999999999999\\
218.01	0.00999999999999999\\
219.01	0.00999999999999999\\
220.01	0.00999999999999999\\
221.01	0.00999999999999999\\
222.01	0.00999999999999999\\
223.01	0.00999999999999999\\
224.01	0.00999999999999999\\
225.01	0.00999999999999999\\
226.01	0.00999999999999999\\
227.01	0.00999999999999999\\
228.01	0.00999999999999999\\
229.01	0.00999999999999999\\
230.01	0.00999999999999999\\
231.01	0.00999999999999999\\
232.01	0.00999999999999999\\
233.01	0.00999999999999999\\
234.01	0.00999999999999999\\
235.01	0.00999999999999999\\
236.01	0.00999999999999999\\
237.01	0.00999999999999999\\
238.01	0.00999999999999999\\
239.01	0.00999999999999999\\
240.01	0.00999999999999999\\
241.01	0.00999999999999999\\
242.01	0.00999999999999999\\
243.01	0.00999999999999999\\
244.01	0.00999999999999999\\
245.01	0.00999999999999999\\
246.01	0.00999999999999999\\
247.01	0.00999999999999999\\
248.01	0.00999999999999999\\
249.01	0.00999999999999999\\
250.01	0.00999999999999999\\
251.01	0.00999999999999999\\
252.01	0.00999999999999999\\
253.01	0.00999999999999999\\
254.01	0.00999999999999999\\
255.01	0.00999999999999999\\
256.01	0.00999999999999999\\
257.01	0.00999999999999999\\
258.01	0.00999999999999999\\
259.01	0.00999999999999999\\
260.01	0.00999999999999999\\
261.01	0.00999999999999999\\
262.01	0.00999999999999999\\
263.01	0.00999999999999999\\
264.01	0.00999999999999999\\
265.01	0.00999999999999999\\
266.01	0.00999999999999999\\
267.01	0.00999999999999999\\
268.01	0.00999999999999999\\
269.01	0.00999999999999999\\
270.01	0.00999999999999999\\
271.01	0.00999999999999999\\
272.01	0.00999999999999999\\
273.01	0.00999999999999999\\
274.01	0.00999999999999999\\
275.01	0.00999999999999999\\
276.01	0.00999999999999999\\
277.01	0.00999999999999999\\
278.01	0.00999999999999999\\
279.01	0.00999999999999999\\
280.01	0.00999999999999999\\
281.01	0.00999999999999999\\
282.01	0.00999999999999999\\
283.01	0.00999999999999999\\
284.01	0.00999999999999999\\
285.01	0.00999999999999999\\
286.01	0.00999999999999999\\
287.01	0.00999999999999999\\
288.01	0.00999999999999999\\
289.01	0.00999999999999999\\
290.01	0.00999999999999999\\
291.01	0.00999999999999999\\
292.01	0.00999999999999999\\
293.01	0.00999999999999999\\
294.01	0.00999999999999999\\
295.01	0.00999999999999999\\
296.01	0.00999999999999999\\
297.01	0.00999999999999999\\
298.01	0.00999999999999999\\
299.01	0.00999999999999999\\
300.01	0.00999999999999999\\
301.01	0.00999999999999999\\
302.01	0.00999999999999999\\
303.01	0.00999999999999999\\
304.01	0.00999999999999999\\
305.01	0.00999999999999999\\
306.01	0.00999999999999999\\
307.01	0.00999999999999999\\
308.01	0.00999999999999999\\
309.01	0.00999999999999999\\
310.01	0.00999999999999999\\
311.01	0.00999999999999999\\
312.01	0.00999999999999999\\
313.01	0.00999999999999999\\
314.01	0.00999999999999999\\
315.01	0.00999999999999999\\
316.01	0.00999999999999999\\
317.01	0.00999999999999999\\
318.01	0.00999999999999999\\
319.01	0.00999999999999999\\
320.01	0.00999999999999999\\
321.01	0.00999999999999999\\
322.01	0.00999999999999999\\
323.01	0.00999999999999999\\
324.01	0.00999999999999999\\
325.01	0.00999999999999999\\
326.01	0.00999999999999999\\
327.01	0.00999999999999999\\
328.01	0.00999999999999999\\
329.01	0.00999999999999999\\
330.01	0.00999999999999999\\
331.01	0.00999999999999999\\
332.01	0.00999999999999999\\
333.01	0.00999999999999999\\
334.01	0.00999999999999999\\
335.01	0.00999999999999999\\
336.01	0.00999999999999999\\
337.01	0.00999999999999999\\
338.01	0.00999999999999999\\
339.01	0.00999999999999999\\
340.01	0.00999999999999999\\
341.01	0.00999999999999999\\
342.01	0.00999999999999999\\
343.01	0.00999999999999999\\
344.01	0.00999999999999999\\
345.01	0.00999999999999999\\
346.01	0.00999999999999999\\
347.01	0.00999999999999999\\
348.01	0.00999999999999999\\
349.01	0.00999999999999999\\
350.01	0.00999999999999999\\
351.01	0.00999999999999999\\
352.01	0.00999999999999999\\
353.01	0.00999999999999999\\
354.01	0.00999999999999999\\
355.01	0.00999999999999999\\
356.01	0.00999999999999999\\
357.01	0.00999999999999999\\
358.01	0.00999999999999999\\
359.01	0.00999999999999999\\
360.01	0.00999999999999999\\
361.01	0.00999999999999999\\
362.01	0.00999999999999999\\
363.01	0.00999999999999999\\
364.01	0.00999999999999999\\
365.01	0.00999999999999999\\
366.01	0.00999999999999999\\
367.01	0.00999999999999999\\
368.01	0.00999999999999999\\
369.01	0.00999999999999999\\
370.01	0.00999999999999999\\
371.01	0.00999999999999999\\
372.01	0.00999999999999999\\
373.01	0.00999999999999999\\
374.01	0.00999999999999999\\
375.01	0.00999999999999999\\
376.01	0.00999999999999999\\
377.01	0.00999999999999999\\
378.01	0.00999999999999999\\
379.01	0.00999999999999999\\
380.01	0.00999999999999999\\
381.01	0.00999999999999999\\
382.01	0.00999999999999999\\
383.01	0.00999999999999999\\
384.01	0.00999999999999999\\
385.01	0.00999999999999999\\
386.01	0.00999999999999999\\
387.01	0.00999999999999999\\
388.01	0.00999999999999999\\
389.01	0.00999999999999999\\
390.01	0.00999999999999999\\
391.01	0.00999999999999999\\
392.01	0.00999999999999999\\
393.01	0.00999999999999999\\
394.01	0.00999999999999999\\
395.01	0.00999999999999999\\
396.01	0.00999999999999999\\
397.01	0.00999999999999999\\
398.01	0.00999999999999999\\
399.01	0.00999999999999999\\
400.01	0.00999999999999999\\
401.01	0.00999999999999999\\
402.01	0.00999999999999999\\
403.01	0.00999999999999999\\
404.01	0.00999999999999999\\
405.01	0.00999999999999999\\
406.01	0.00999999999999999\\
407.01	0.00999999999999999\\
408.01	0.00999999999999999\\
409.01	0.00999999999999999\\
410.01	0.00999999999999999\\
411.01	0.00999999999999999\\
412.01	0.00999999999999999\\
413.01	0.00999999999999999\\
414.01	0.00999999999999999\\
415.01	0.00999999999999999\\
416.01	0.00999999999999999\\
417.01	0.00999999999999999\\
418.01	0.00999999999999999\\
419.01	0.00999999999999999\\
420.01	0.00999999999999999\\
421.01	0.00999999999999999\\
422.01	0.00999999999999999\\
423.01	0.00999999999999999\\
424.01	0.00999999999999999\\
425.01	0.00999999999999999\\
426.01	0.00999999999999999\\
427.01	0.00999999999999999\\
428.01	0.00999999999999999\\
429.01	0.00999999999999999\\
430.01	0.00999999999999999\\
431.01	0.00999999999999999\\
432.01	0.00999999999999999\\
433.01	0.00999999999999999\\
434.01	0.00999999999999999\\
435.01	0.00999999999999999\\
436.01	0.00999999999999999\\
437.01	0.00999999999999999\\
438.01	0.00999999999999999\\
439.01	0.00999999999999999\\
440.01	0.00999999999999999\\
441.01	0.00999999999999999\\
442.01	0.00999999999999999\\
443.01	0.00999999999999999\\
444.01	0.00999999999999999\\
445.01	0.00999999999999999\\
446.01	0.00999999999999999\\
447.01	0.00999999999999999\\
448.01	0.00999999999999999\\
449.01	0.00999999999999999\\
450.01	0.00999999999999999\\
451.01	0.00999999999999999\\
452.01	0.00999999999999999\\
453.01	0.00999999999999999\\
454.01	0.00999999999999999\\
455.01	0.00999999999999999\\
456.01	0.00999999999999999\\
457.01	0.00999999999999999\\
458.01	0.00999999999999999\\
459.01	0.00999999999999999\\
460.01	0.00999999999999999\\
461.01	0.00999999999999999\\
462.01	0.00999999999999999\\
463.01	0.00999999999999999\\
464.01	0.00999999999999999\\
465.01	0.00999999999999999\\
466.01	0.00999999999999999\\
467.01	0.00999999999999999\\
468.01	0.00999999999999999\\
469.01	0.00999999999999999\\
470.01	0.00999999999999999\\
471.01	0.00999999999999999\\
472.01	0.00999999999999999\\
473.01	0.00999999999999999\\
474.01	0.00999999999999999\\
475.01	0.00999999999999999\\
476.01	0.00999999999999999\\
477.01	0.00999999999999999\\
478.01	0.00999999999999999\\
479.01	0.00999999999999999\\
480.01	0.00999999999999999\\
481.01	0.00999999999999999\\
482.01	0.00999999999999999\\
483.01	0.00999999999999999\\
484.01	0.00999999999999999\\
485.01	0.00999999999999999\\
486.01	0.00999999999999999\\
487.01	0.00999999999999999\\
488.01	0.00999999999999999\\
489.01	0.00999999999999999\\
490.01	0.00999999999999999\\
491.01	0.00999999999999999\\
492.01	0.00999999999999999\\
493.01	0.00999999999999999\\
494.01	0.00999999999999999\\
495.01	0.00999999999999999\\
496.01	0.00999999999999999\\
497.01	0.00999999999999999\\
498.01	0.00999999999999999\\
499.01	0.00999999999999999\\
500.01	0.00999999999999999\\
501.01	0.00999999999999999\\
502.01	0.00999999999999999\\
503.01	0.00999999999999999\\
504.01	0.00999999999999999\\
505.01	0.00999999999999999\\
506.01	0.00999999999999999\\
507.01	0.00999999999999999\\
508.01	0.00999999999999999\\
509.01	0.00999999999999999\\
510.01	0.00999999999999999\\
511.01	0.00999999999999999\\
512.01	0.00999999999999999\\
513.01	0.00999999999999999\\
514.01	0.00999999999999999\\
515.01	0.00999999999999999\\
516.01	0.00999999999999999\\
517.01	0.00999999999999999\\
518.01	0.00999999999999999\\
519.01	0.00999999999999999\\
520.01	0.00999999999999999\\
521.01	0.00999999999999999\\
522.01	0.00999999999999999\\
523.01	0.00999999999999999\\
524.01	0.00999999999999999\\
525.01	0.00999999999999999\\
526.01	0.00999999999999999\\
527.01	0.00999999999999999\\
528.01	0.00999999999999999\\
529.01	0.00999999999999999\\
530.01	0.00999999999999999\\
531.01	0.00999999999999999\\
532.01	0.00999999999999999\\
533.01	0.00999999999999999\\
534.01	0.00999999999999999\\
535.01	0.00999999999999999\\
536.01	0.00999999999999999\\
537.01	0.00999999999999999\\
538.01	0.00999999999999999\\
539.01	0.00999999999999999\\
540.01	0.00999999999999999\\
541.01	0.00999999999999999\\
542.01	0.00999999999999999\\
543.01	0.00999999999999999\\
544.01	0.00999999999999999\\
545.01	0.00999999999999999\\
546.01	0.00999999999999999\\
547.01	0.00999999999999999\\
548.01	0.00999999999999999\\
549.01	0.00999999999999999\\
550.01	0.00999999999999999\\
551.01	0.00999999999999999\\
552.01	0.00999999999999999\\
553.01	0.00999999999999999\\
554.01	0.00999999999999999\\
555.01	0.00999999999999999\\
556.01	0.00999999999999999\\
557.01	0.00999999999999999\\
558.01	0.00999999999999999\\
559.01	0.00999999999999999\\
560.01	0.00999999999999999\\
561.01	0.00999999999999999\\
562.01	0.00999999999999999\\
563.01	0.00999999999999999\\
564.01	0.00999999999999999\\
565.01	0.00999999999999999\\
566.01	0.00999999999999999\\
567.01	0.00999999999999999\\
568.01	0.00999999999999999\\
569.01	0.00999999999999999\\
570.01	0.00999999999999999\\
571.01	0.00999999999999999\\
572.01	0.00999999999999999\\
573.01	0.00999999999999999\\
574.01	0.00999999999999999\\
575.01	0.00999999999999999\\
576.01	0.00999999999999999\\
577.01	0.00999999999999999\\
578.01	0.00999999999999999\\
579.01	0.00999999999999999\\
580.01	0.00999999999999999\\
581.01	0.00999999999999999\\
582.01	0.00999999999999999\\
583.01	0.00999999999999999\\
584.01	0.00999999999999999\\
585.01	0.00999999999999999\\
586.01	0.00999999999999999\\
587.01	0.00999999999999999\\
588.01	0.00999999999999999\\
589.01	0.00999999999999999\\
590.01	0.00999999999999999\\
591.01	0.00999999999999999\\
592.01	0.00999999999999999\\
593.01	0.00999999999999999\\
594.01	0.00999999999999999\\
595.01	0.00999999999999999\\
596.01	0.00999999999999999\\
597.01	0.00999999999999999\\
598.01	0.00999999999999999\\
599.01	0.00624186909636788\\
599.02	0.00620414441457609\\
599.03	0.00616605275823867\\
599.04	0.00612759052289781\\
599.05	0.00608875406868525\\
599.06	0.00604953971997439\\
599.07	0.00600994376502898\\
599.08	0.00596996245564835\\
599.09	0.0059295920068092\\
599.1	0.00588882859630377\\
599.11	0.00584766836437461\\
599.12	0.00580610741334552\\
599.13	0.00576414180724917\\
599.14	0.00572176757145079\\
599.15	0.00567898069226832\\
599.16	0.0056357771165888\\
599.17	0.00559215275148091\\
599.18	0.00554810346380382\\
599.19	0.00550362507981218\\
599.2	0.00545871338475718\\
599.21	0.0054133641224837\\
599.22	0.00536757299502359\\
599.23	0.00532133566218485\\
599.24	0.00527464774113687\\
599.25	0.00522750480599146\\
599.26	0.00517990238737996\\
599.27	0.00513183597202595\\
599.28	0.00508330100231403\\
599.29	0.00503429287585414\\
599.3	0.00498480694504173\\
599.31	0.00493483851661353\\
599.32	0.00488438285119905\\
599.33	0.00483343516286759\\
599.34	0.00478199061867081\\
599.35	0.00473004433818085\\
599.36	0.00467759139302389\\
599.37	0.00462462680640909\\
599.38	0.004571145552653\\
599.39	0.00451714256067378\\
599.4	0.0044626127239245\\
599.41	0.00440755088587263\\
599.42	0.00435195183951073\\
599.43	0.00429581032686245\\
599.44	0.00423912103848351\\
599.45	0.00418187861295803\\
599.46	0.00412407763638975\\
599.47	0.00406571264188831\\
599.48	0.00400677810905057\\
599.49	0.00394726846343678\\
599.5	0.00388717807604165\\
599.51	0.00382650126276023\\
599.52	0.0037652322838486\\
599.53	0.00370336534337923\\
599.54	0.00364089458869102\\
599.55	0.00357781410983402\\
599.56	0.00351411793900859\\
599.57	0.00344980004999922\\
599.58	0.00338485435760266\\
599.59	0.00331927471705059\\
599.6	0.00325305492342653\\
599.61	0.00318618871107708\\
599.62	0.00311866975301747\\
599.63	0.00305049166033112\\
599.64	0.00298164798156351\\
599.65	0.00291213220211002\\
599.66	0.00284193774359775\\
599.67	0.00277105796326139\\
599.68	0.00269948615331285\\
599.69	0.00262721554030489\\
599.7	0.00255423928448833\\
599.71	0.0024805504791631\\
599.72	0.0024061421500229\\
599.73	0.00233100725449347\\
599.74	0.00225513868106428\\
599.75	0.00217852924861384\\
599.76	0.00210117170572828\\
599.77	0.00202305873001331\\
599.78	0.00194418292739942\\
599.79	0.00186453683144033\\
599.8	0.00178411290260452\\
599.81	0.00170290352755984\\
599.82	0.00162090101845112\\
599.83	0.00153809761217074\\
599.84	0.00145448546962201\\
599.85	0.00137005667497535\\
599.86	0.00128480323491721\\
599.87	0.00119871707789166\\
599.88	0.00111179005333448\\
599.89	0.00102401393089982\\
599.9	0.000935380399679274\\
599.91	0.000845881067413288\\
599.92	0.00075550745969488\\
599.93	0.000664251019165561\\
599.94	0.000572103104703356\\
599.95	0.00047905499060291\\
599.96	0.000385097865747556\\
599.97	0.000290222832773275\\
599.98	0.000194420907224489\\
599.99	9.76830167015649e-05\\
600	0\\
};
\addplot [color=red!40!mycolor19,solid,forget plot]
  table[row sep=crcr]{%
0.01	0.00999999999999999\\
1.01	0.00999999999999999\\
2.01	0.00999999999999999\\
3.01	0.00999999999999999\\
4.01	0.00999999999999999\\
5.01	0.00999999999999999\\
6.01	0.00999999999999999\\
7.01	0.00999999999999999\\
8.01	0.00999999999999999\\
9.01	0.00999999999999999\\
10.01	0.00999999999999999\\
11.01	0.00999999999999999\\
12.01	0.00999999999999999\\
13.01	0.00999999999999999\\
14.01	0.00999999999999999\\
15.01	0.00999999999999999\\
16.01	0.00999999999999999\\
17.01	0.00999999999999999\\
18.01	0.00999999999999999\\
19.01	0.00999999999999999\\
20.01	0.00999999999999999\\
21.01	0.00999999999999999\\
22.01	0.00999999999999999\\
23.01	0.00999999999999999\\
24.01	0.00999999999999999\\
25.01	0.00999999999999999\\
26.01	0.00999999999999999\\
27.01	0.00999999999999999\\
28.01	0.00999999999999999\\
29.01	0.00999999999999999\\
30.01	0.00999999999999999\\
31.01	0.00999999999999999\\
32.01	0.00999999999999999\\
33.01	0.00999999999999999\\
34.01	0.00999999999999999\\
35.01	0.00999999999999999\\
36.01	0.00999999999999999\\
37.01	0.00999999999999999\\
38.01	0.00999999999999999\\
39.01	0.00999999999999999\\
40.01	0.00999999999999999\\
41.01	0.00999999999999999\\
42.01	0.00999999999999999\\
43.01	0.00999999999999999\\
44.01	0.00999999999999999\\
45.01	0.00999999999999999\\
46.01	0.00999999999999999\\
47.01	0.00999999999999999\\
48.01	0.00999999999999999\\
49.01	0.00999999999999999\\
50.01	0.00999999999999999\\
51.01	0.00999999999999999\\
52.01	0.00999999999999999\\
53.01	0.00999999999999999\\
54.01	0.00999999999999999\\
55.01	0.00999999999999999\\
56.01	0.00999999999999999\\
57.01	0.00999999999999999\\
58.01	0.00999999999999999\\
59.01	0.00999999999999999\\
60.01	0.00999999999999999\\
61.01	0.00999999999999999\\
62.01	0.00999999999999999\\
63.01	0.00999999999999999\\
64.01	0.00999999999999999\\
65.01	0.00999999999999999\\
66.01	0.00999999999999999\\
67.01	0.00999999999999999\\
68.01	0.00999999999999999\\
69.01	0.00999999999999999\\
70.01	0.00999999999999999\\
71.01	0.00999999999999999\\
72.01	0.00999999999999999\\
73.01	0.00999999999999999\\
74.01	0.00999999999999999\\
75.01	0.00999999999999999\\
76.01	0.00999999999999999\\
77.01	0.00999999999999999\\
78.01	0.00999999999999999\\
79.01	0.00999999999999999\\
80.01	0.00999999999999999\\
81.01	0.00999999999999999\\
82.01	0.00999999999999999\\
83.01	0.00999999999999999\\
84.01	0.00999999999999999\\
85.01	0.00999999999999999\\
86.01	0.00999999999999999\\
87.01	0.00999999999999999\\
88.01	0.00999999999999999\\
89.01	0.00999999999999999\\
90.01	0.00999999999999999\\
91.01	0.00999999999999999\\
92.01	0.00999999999999999\\
93.01	0.00999999999999999\\
94.01	0.00999999999999999\\
95.01	0.00999999999999999\\
96.01	0.00999999999999999\\
97.01	0.00999999999999999\\
98.01	0.00999999999999999\\
99.01	0.00999999999999999\\
100.01	0.00999999999999999\\
101.01	0.00999999999999999\\
102.01	0.00999999999999999\\
103.01	0.00999999999999999\\
104.01	0.00999999999999999\\
105.01	0.00999999999999999\\
106.01	0.00999999999999999\\
107.01	0.00999999999999999\\
108.01	0.00999999999999999\\
109.01	0.00999999999999999\\
110.01	0.00999999999999999\\
111.01	0.00999999999999999\\
112.01	0.00999999999999999\\
113.01	0.00999999999999999\\
114.01	0.00999999999999999\\
115.01	0.00999999999999999\\
116.01	0.00999999999999999\\
117.01	0.00999999999999999\\
118.01	0.00999999999999999\\
119.01	0.00999999999999999\\
120.01	0.00999999999999999\\
121.01	0.00999999999999999\\
122.01	0.00999999999999999\\
123.01	0.00999999999999999\\
124.01	0.00999999999999999\\
125.01	0.00999999999999999\\
126.01	0.00999999999999999\\
127.01	0.00999999999999999\\
128.01	0.00999999999999999\\
129.01	0.00999999999999999\\
130.01	0.00999999999999999\\
131.01	0.00999999999999999\\
132.01	0.00999999999999999\\
133.01	0.00999999999999999\\
134.01	0.00999999999999999\\
135.01	0.00999999999999999\\
136.01	0.00999999999999999\\
137.01	0.00999999999999999\\
138.01	0.00999999999999999\\
139.01	0.00999999999999999\\
140.01	0.00999999999999999\\
141.01	0.00999999999999999\\
142.01	0.00999999999999999\\
143.01	0.00999999999999999\\
144.01	0.00999999999999999\\
145.01	0.00999999999999999\\
146.01	0.00999999999999999\\
147.01	0.00999999999999999\\
148.01	0.00999999999999999\\
149.01	0.00999999999999999\\
150.01	0.00999999999999999\\
151.01	0.00999999999999999\\
152.01	0.00999999999999999\\
153.01	0.00999999999999999\\
154.01	0.00999999999999999\\
155.01	0.00999999999999999\\
156.01	0.00999999999999999\\
157.01	0.00999999999999999\\
158.01	0.00999999999999999\\
159.01	0.00999999999999999\\
160.01	0.00999999999999999\\
161.01	0.00999999999999999\\
162.01	0.00999999999999999\\
163.01	0.00999999999999999\\
164.01	0.00999999999999999\\
165.01	0.00999999999999999\\
166.01	0.00999999999999999\\
167.01	0.00999999999999999\\
168.01	0.00999999999999999\\
169.01	0.00999999999999999\\
170.01	0.00999999999999999\\
171.01	0.00999999999999999\\
172.01	0.00999999999999999\\
173.01	0.00999999999999999\\
174.01	0.00999999999999999\\
175.01	0.00999999999999999\\
176.01	0.00999999999999999\\
177.01	0.00999999999999999\\
178.01	0.00999999999999999\\
179.01	0.00999999999999999\\
180.01	0.00999999999999999\\
181.01	0.00999999999999999\\
182.01	0.00999999999999999\\
183.01	0.00999999999999999\\
184.01	0.00999999999999999\\
185.01	0.00999999999999999\\
186.01	0.00999999999999999\\
187.01	0.00999999999999999\\
188.01	0.00999999999999999\\
189.01	0.00999999999999999\\
190.01	0.00999999999999999\\
191.01	0.00999999999999999\\
192.01	0.00999999999999999\\
193.01	0.00999999999999999\\
194.01	0.00999999999999999\\
195.01	0.00999999999999999\\
196.01	0.00999999999999999\\
197.01	0.00999999999999999\\
198.01	0.00999999999999999\\
199.01	0.00999999999999999\\
200.01	0.00999999999999999\\
201.01	0.00999999999999999\\
202.01	0.00999999999999999\\
203.01	0.00999999999999999\\
204.01	0.00999999999999999\\
205.01	0.00999999999999999\\
206.01	0.00999999999999999\\
207.01	0.00999999999999999\\
208.01	0.00999999999999999\\
209.01	0.00999999999999999\\
210.01	0.00999999999999999\\
211.01	0.00999999999999999\\
212.01	0.00999999999999999\\
213.01	0.00999999999999999\\
214.01	0.00999999999999999\\
215.01	0.00999999999999999\\
216.01	0.00999999999999999\\
217.01	0.00999999999999999\\
218.01	0.00999999999999999\\
219.01	0.00999999999999999\\
220.01	0.00999999999999999\\
221.01	0.00999999999999999\\
222.01	0.00999999999999999\\
223.01	0.00999999999999999\\
224.01	0.00999999999999999\\
225.01	0.00999999999999999\\
226.01	0.00999999999999999\\
227.01	0.00999999999999999\\
228.01	0.00999999999999999\\
229.01	0.00999999999999999\\
230.01	0.00999999999999999\\
231.01	0.00999999999999999\\
232.01	0.00999999999999999\\
233.01	0.00999999999999999\\
234.01	0.00999999999999999\\
235.01	0.00999999999999999\\
236.01	0.00999999999999999\\
237.01	0.00999999999999999\\
238.01	0.00999999999999999\\
239.01	0.00999999999999999\\
240.01	0.00999999999999999\\
241.01	0.00999999999999999\\
242.01	0.00999999999999999\\
243.01	0.00999999999999999\\
244.01	0.00999999999999999\\
245.01	0.00999999999999999\\
246.01	0.00999999999999999\\
247.01	0.00999999999999999\\
248.01	0.00999999999999999\\
249.01	0.00999999999999999\\
250.01	0.00999999999999999\\
251.01	0.00999999999999999\\
252.01	0.00999999999999999\\
253.01	0.00999999999999999\\
254.01	0.00999999999999999\\
255.01	0.00999999999999999\\
256.01	0.00999999999999999\\
257.01	0.00999999999999999\\
258.01	0.00999999999999999\\
259.01	0.00999999999999999\\
260.01	0.00999999999999999\\
261.01	0.00999999999999999\\
262.01	0.00999999999999999\\
263.01	0.00999999999999999\\
264.01	0.00999999999999999\\
265.01	0.00999999999999999\\
266.01	0.00999999999999999\\
267.01	0.00999999999999999\\
268.01	0.00999999999999999\\
269.01	0.00999999999999999\\
270.01	0.00999999999999999\\
271.01	0.00999999999999999\\
272.01	0.00999999999999999\\
273.01	0.00999999999999999\\
274.01	0.00999999999999999\\
275.01	0.00999999999999999\\
276.01	0.00999999999999999\\
277.01	0.00999999999999999\\
278.01	0.00999999999999999\\
279.01	0.00999999999999999\\
280.01	0.00999999999999999\\
281.01	0.00999999999999999\\
282.01	0.00999999999999999\\
283.01	0.00999999999999999\\
284.01	0.00999999999999999\\
285.01	0.00999999999999999\\
286.01	0.00999999999999999\\
287.01	0.00999999999999999\\
288.01	0.00999999999999999\\
289.01	0.00999999999999999\\
290.01	0.00999999999999999\\
291.01	0.00999999999999999\\
292.01	0.00999999999999999\\
293.01	0.00999999999999999\\
294.01	0.00999999999999999\\
295.01	0.00999999999999999\\
296.01	0.00999999999999999\\
297.01	0.00999999999999999\\
298.01	0.00999999999999999\\
299.01	0.00999999999999999\\
300.01	0.00999999999999999\\
301.01	0.00999999999999999\\
302.01	0.00999999999999999\\
303.01	0.00999999999999999\\
304.01	0.00999999999999999\\
305.01	0.00999999999999999\\
306.01	0.00999999999999999\\
307.01	0.00999999999999999\\
308.01	0.00999999999999999\\
309.01	0.00999999999999999\\
310.01	0.00999999999999999\\
311.01	0.00999999999999999\\
312.01	0.00999999999999999\\
313.01	0.00999999999999999\\
314.01	0.00999999999999999\\
315.01	0.00999999999999999\\
316.01	0.00999999999999999\\
317.01	0.00999999999999999\\
318.01	0.00999999999999999\\
319.01	0.00999999999999999\\
320.01	0.00999999999999999\\
321.01	0.00999999999999999\\
322.01	0.00999999999999999\\
323.01	0.00999999999999999\\
324.01	0.00999999999999999\\
325.01	0.00999999999999999\\
326.01	0.00999999999999999\\
327.01	0.00999999999999999\\
328.01	0.00999999999999999\\
329.01	0.00999999999999999\\
330.01	0.00999999999999999\\
331.01	0.00999999999999999\\
332.01	0.00999999999999999\\
333.01	0.00999999999999999\\
334.01	0.00999999999999999\\
335.01	0.00999999999999999\\
336.01	0.00999999999999999\\
337.01	0.00999999999999999\\
338.01	0.00999999999999999\\
339.01	0.00999999999999999\\
340.01	0.00999999999999999\\
341.01	0.00999999999999999\\
342.01	0.00999999999999999\\
343.01	0.00999999999999999\\
344.01	0.00999999999999999\\
345.01	0.00999999999999999\\
346.01	0.00999999999999999\\
347.01	0.00999999999999999\\
348.01	0.00999999999999999\\
349.01	0.00999999999999999\\
350.01	0.00999999999999999\\
351.01	0.00999999999999999\\
352.01	0.00999999999999999\\
353.01	0.00999999999999999\\
354.01	0.00999999999999999\\
355.01	0.00999999999999999\\
356.01	0.00999999999999999\\
357.01	0.00999999999999999\\
358.01	0.00999999999999999\\
359.01	0.00999999999999999\\
360.01	0.00999999999999999\\
361.01	0.00999999999999999\\
362.01	0.00999999999999999\\
363.01	0.00999999999999999\\
364.01	0.00999999999999999\\
365.01	0.00999999999999999\\
366.01	0.00999999999999999\\
367.01	0.00999999999999999\\
368.01	0.00999999999999999\\
369.01	0.00999999999999999\\
370.01	0.00999999999999999\\
371.01	0.00999999999999999\\
372.01	0.00999999999999999\\
373.01	0.00999999999999999\\
374.01	0.00999999999999999\\
375.01	0.00999999999999999\\
376.01	0.00999999999999999\\
377.01	0.00999999999999999\\
378.01	0.00999999999999999\\
379.01	0.00999999999999999\\
380.01	0.00999999999999999\\
381.01	0.00999999999999999\\
382.01	0.00999999999999999\\
383.01	0.00999999999999999\\
384.01	0.00999999999999999\\
385.01	0.00999999999999999\\
386.01	0.00999999999999999\\
387.01	0.00999999999999999\\
388.01	0.00999999999999999\\
389.01	0.00999999999999999\\
390.01	0.00999999999999999\\
391.01	0.00999999999999999\\
392.01	0.00999999999999999\\
393.01	0.00999999999999999\\
394.01	0.00999999999999999\\
395.01	0.00999999999999999\\
396.01	0.00999999999999999\\
397.01	0.00999999999999999\\
398.01	0.00999999999999999\\
399.01	0.00999999999999999\\
400.01	0.00999999999999999\\
401.01	0.00999999999999999\\
402.01	0.00999999999999999\\
403.01	0.00999999999999999\\
404.01	0.00999999999999999\\
405.01	0.00999999999999999\\
406.01	0.00999999999999999\\
407.01	0.00999999999999999\\
408.01	0.00999999999999999\\
409.01	0.00999999999999999\\
410.01	0.00999999999999999\\
411.01	0.00999999999999999\\
412.01	0.00999999999999999\\
413.01	0.00999999999999999\\
414.01	0.00999999999999999\\
415.01	0.00999999999999999\\
416.01	0.00999999999999999\\
417.01	0.00999999999999999\\
418.01	0.00999999999999999\\
419.01	0.00999999999999999\\
420.01	0.00999999999999999\\
421.01	0.00999999999999999\\
422.01	0.00999999999999999\\
423.01	0.00999999999999999\\
424.01	0.00999999999999999\\
425.01	0.00999999999999999\\
426.01	0.00999999999999999\\
427.01	0.00999999999999999\\
428.01	0.00999999999999999\\
429.01	0.00999999999999999\\
430.01	0.00999999999999999\\
431.01	0.00999999999999999\\
432.01	0.00999999999999999\\
433.01	0.00999999999999999\\
434.01	0.00999999999999999\\
435.01	0.00999999999999999\\
436.01	0.00999999999999999\\
437.01	0.00999999999999999\\
438.01	0.00999999999999999\\
439.01	0.00999999999999999\\
440.01	0.00999999999999999\\
441.01	0.00999999999999999\\
442.01	0.00999999999999999\\
443.01	0.00999999999999999\\
444.01	0.00999999999999999\\
445.01	0.00999999999999999\\
446.01	0.00999999999999999\\
447.01	0.00999999999999999\\
448.01	0.00999999999999999\\
449.01	0.00999999999999999\\
450.01	0.00999999999999999\\
451.01	0.00999999999999999\\
452.01	0.00999999999999999\\
453.01	0.00999999999999999\\
454.01	0.00999999999999999\\
455.01	0.00999999999999999\\
456.01	0.00999999999999999\\
457.01	0.00999999999999999\\
458.01	0.00999999999999999\\
459.01	0.00999999999999999\\
460.01	0.00999999999999999\\
461.01	0.00999999999999999\\
462.01	0.00999999999999999\\
463.01	0.00999999999999999\\
464.01	0.00999999999999999\\
465.01	0.00999999999999999\\
466.01	0.00999999999999999\\
467.01	0.00999999999999999\\
468.01	0.00999999999999999\\
469.01	0.00999999999999999\\
470.01	0.00999999999999999\\
471.01	0.00999999999999999\\
472.01	0.00999999999999999\\
473.01	0.00999999999999999\\
474.01	0.00999999999999999\\
475.01	0.00999999999999999\\
476.01	0.00999999999999999\\
477.01	0.00999999999999999\\
478.01	0.00999999999999999\\
479.01	0.00999999999999999\\
480.01	0.00999999999999999\\
481.01	0.00999999999999999\\
482.01	0.00999999999999999\\
483.01	0.00999999999999999\\
484.01	0.00999999999999999\\
485.01	0.00999999999999999\\
486.01	0.00999999999999999\\
487.01	0.00999999999999999\\
488.01	0.00999999999999999\\
489.01	0.00999999999999999\\
490.01	0.00999999999999999\\
491.01	0.00999999999999999\\
492.01	0.00999999999999999\\
493.01	0.00999999999999999\\
494.01	0.00999999999999999\\
495.01	0.00999999999999999\\
496.01	0.00999999999999999\\
497.01	0.00999999999999999\\
498.01	0.00999999999999999\\
499.01	0.00999999999999999\\
500.01	0.00999999999999999\\
501.01	0.00999999999999999\\
502.01	0.00999999999999999\\
503.01	0.00999999999999999\\
504.01	0.00999999999999999\\
505.01	0.00999999999999999\\
506.01	0.00999999999999999\\
507.01	0.00999999999999999\\
508.01	0.00999999999999999\\
509.01	0.00999999999999999\\
510.01	0.00999999999999999\\
511.01	0.00999999999999999\\
512.01	0.00999999999999999\\
513.01	0.00999999999999999\\
514.01	0.00999999999999999\\
515.01	0.00999999999999999\\
516.01	0.00999999999999999\\
517.01	0.00999999999999999\\
518.01	0.00999999999999999\\
519.01	0.00999999999999999\\
520.01	0.00999999999999999\\
521.01	0.00999999999999999\\
522.01	0.00999999999999999\\
523.01	0.00999999999999999\\
524.01	0.00999999999999999\\
525.01	0.00999999999999999\\
526.01	0.00999999999999999\\
527.01	0.00999999999999999\\
528.01	0.00999999999999999\\
529.01	0.00999999999999999\\
530.01	0.00999999999999999\\
531.01	0.00999999999999999\\
532.01	0.00999999999999999\\
533.01	0.00999999999999999\\
534.01	0.00999999999999999\\
535.01	0.00999999999999999\\
536.01	0.00999999999999999\\
537.01	0.00999999999999999\\
538.01	0.00999999999999999\\
539.01	0.00999999999999999\\
540.01	0.00999999999999999\\
541.01	0.00999999999999999\\
542.01	0.00999999999999999\\
543.01	0.00999999999999999\\
544.01	0.00999999999999999\\
545.01	0.00999999999999999\\
546.01	0.00999999999999999\\
547.01	0.00999999999999999\\
548.01	0.00999999999999999\\
549.01	0.00999999999999999\\
550.01	0.00999999999999999\\
551.01	0.00999999999999999\\
552.01	0.00999999999999999\\
553.01	0.00999999999999999\\
554.01	0.00999999999999999\\
555.01	0.00999999999999999\\
556.01	0.00999999999999999\\
557.01	0.00999999999999999\\
558.01	0.00999999999999999\\
559.01	0.00999999999999999\\
560.01	0.00999999999999999\\
561.01	0.00999999999999999\\
562.01	0.00999999999999999\\
563.01	0.00999999999999999\\
564.01	0.00999999999999999\\
565.01	0.00999999999999999\\
566.01	0.00999999999999999\\
567.01	0.00999999999999999\\
568.01	0.00999999999999999\\
569.01	0.00999999999999999\\
570.01	0.00999999999999999\\
571.01	0.00999999999999999\\
572.01	0.00999999999999999\\
573.01	0.00999999999999999\\
574.01	0.00999999999999999\\
575.01	0.00999999999999999\\
576.01	0.00999999999999999\\
577.01	0.00999999999999999\\
578.01	0.00999999999999999\\
579.01	0.00999999999999999\\
580.01	0.00999999999999999\\
581.01	0.00999999999999999\\
582.01	0.00999999999999999\\
583.01	0.00999999999999999\\
584.01	0.00999999999999999\\
585.01	0.00999999999999999\\
586.01	0.00999999999999999\\
587.01	0.00999999999999999\\
588.01	0.00999999999999999\\
589.01	0.00999999999999999\\
590.01	0.00999999999999999\\
591.01	0.00999999999999999\\
592.01	0.00999999999999999\\
593.01	0.00999999999999999\\
594.01	0.00999999999999999\\
595.01	0.00999999999999999\\
596.01	0.00999999999999999\\
597.01	0.00999999999999999\\
598.01	0.00999999999999999\\
599.01	0.00624186909410445\\
599.02	0.00620414441246906\\
599.03	0.00616605275627208\\
599.04	0.00612759052105786\\
599.05	0.00608875406696011\\
599.06	0.00604953971835391\\
599.07	0.0060099437635045\\
599.08	0.0059699624542125\\
599.09	0.00592959200545564\\
599.1	0.00588882859502706\\
599.11	0.00584766836317001\\
599.12	0.00580610741220891\\
599.13	0.00576414180617682\\
599.14	0.00572176757043935\\
599.15	0.00567898069131471\\
599.16	0.00563577711569011\\
599.17	0.00559215275063443\\
599.18	0.005548103463007\\
599.19	0.00550362507906258\\
599.2	0.00545871338405247\\
599.21	0.0054133641218217\\
599.22	0.00536757299440222\\
599.23	0.00532133566160213\\
599.24	0.00527464774059092\\
599.25	0.00522750480548048\\
599.26	0.00517990238690222\\
599.27	0.00513183597157985\\
599.28	0.00508330100189798\\
599.29	0.00503429287546663\\
599.3	0.00498480694468132\\
599.31	0.00493483851627886\\
599.32	0.00488438285088877\\
599.33	0.0048334351625804\\
599.34	0.00478199061840547\\
599.35	0.00473004433793616\\
599.36	0.00467759139279867\\
599.37	0.00462462680620223\\
599.38	0.0045711455524634\\
599.39	0.00451714256050039\\
599.4	0.0044626127237663\\
599.41	0.00440755088572863\\
599.42	0.00435195183938\\
599.43	0.00429581032674405\\
599.44	0.00423912103837659\\
599.45	0.00418187861286175\\
599.46	0.00412407763630329\\
599.47	0.0040657126418109\\
599.48	0.00400677810898149\\
599.49	0.00394726846337534\\
599.5	0.00388717807598718\\
599.51	0.00382650126271212\\
599.52	0.00376523228380627\\
599.53	0.00370336534334213\\
599.54	0.00364089458865865\\
599.55	0.00357781410980589\\
599.56	0.00351411793898426\\
599.57	0.00344980004997828\\
599.58	0.00338485435758473\\
599.59	0.00331927471703531\\
599.6	0.00325305492341358\\
599.61	0.00318618871106618\\
599.62	0.00311866975300834\\
599.63	0.00305049166032353\\
599.64	0.00298164798155725\\
599.65	0.00291213220210488\\
599.66	0.00284193774359358\\
599.67	0.00277105796325802\\
599.68	0.00269948615331017\\
599.69	0.00262721554030278\\
599.7	0.00255423928448668\\
599.71	0.00248055047916183\\
599.72	0.00240614215002194\\
599.73	0.00233100725449274\\
599.74	0.00225513868106374\\
599.75	0.00217852924861345\\
599.76	0.00210117170572801\\
599.77	0.00202305873001312\\
599.78	0.00194418292739929\\
599.79	0.00186453683144024\\
599.8	0.00178411290260446\\
599.81	0.0017029035275598\\
599.82	0.0016209010184511\\
599.83	0.00153809761217073\\
599.84	0.001454485469622\\
599.85	0.00137005667497534\\
599.86	0.00128480323491721\\
599.87	0.00119871707789166\\
599.88	0.00111179005333448\\
599.89	0.00102401393089983\\
599.9	0.000935380399679275\\
599.91	0.000845881067413288\\
599.92	0.00075550745969488\\
599.93	0.000664251019165561\\
599.94	0.000572103104703356\\
599.95	0.000479054990602912\\
599.96	0.000385097865747556\\
599.97	0.000290222832773277\\
599.98	0.000194420907224489\\
599.99	9.76830167015632e-05\\
600	0\\
};
\addplot [color=red!75!mycolor17,solid,forget plot]
  table[row sep=crcr]{%
0.01	0.00999999999999999\\
1.01	0.00999999999999999\\
2.01	0.00999999999999999\\
3.01	0.00999999999999999\\
4.01	0.00999999999999999\\
5.01	0.00999999999999999\\
6.01	0.00999999999999999\\
7.01	0.00999999999999999\\
8.01	0.00999999999999999\\
9.01	0.00999999999999999\\
10.01	0.00999999999999999\\
11.01	0.00999999999999999\\
12.01	0.00999999999999999\\
13.01	0.00999999999999999\\
14.01	0.00999999999999999\\
15.01	0.00999999999999999\\
16.01	0.00999999999999999\\
17.01	0.00999999999999999\\
18.01	0.00999999999999999\\
19.01	0.00999999999999999\\
20.01	0.00999999999999999\\
21.01	0.00999999999999999\\
22.01	0.00999999999999999\\
23.01	0.00999999999999999\\
24.01	0.00999999999999999\\
25.01	0.00999999999999999\\
26.01	0.00999999999999999\\
27.01	0.00999999999999999\\
28.01	0.00999999999999999\\
29.01	0.00999999999999999\\
30.01	0.00999999999999999\\
31.01	0.00999999999999999\\
32.01	0.00999999999999999\\
33.01	0.00999999999999999\\
34.01	0.00999999999999999\\
35.01	0.00999999999999999\\
36.01	0.00999999999999999\\
37.01	0.00999999999999999\\
38.01	0.00999999999999999\\
39.01	0.00999999999999999\\
40.01	0.00999999999999999\\
41.01	0.00999999999999999\\
42.01	0.00999999999999999\\
43.01	0.00999999999999999\\
44.01	0.00999999999999999\\
45.01	0.00999999999999999\\
46.01	0.00999999999999999\\
47.01	0.00999999999999999\\
48.01	0.00999999999999999\\
49.01	0.00999999999999999\\
50.01	0.00999999999999999\\
51.01	0.00999999999999999\\
52.01	0.00999999999999999\\
53.01	0.00999999999999999\\
54.01	0.00999999999999999\\
55.01	0.00999999999999999\\
56.01	0.00999999999999999\\
57.01	0.00999999999999999\\
58.01	0.00999999999999999\\
59.01	0.00999999999999999\\
60.01	0.00999999999999999\\
61.01	0.00999999999999999\\
62.01	0.00999999999999999\\
63.01	0.00999999999999999\\
64.01	0.00999999999999999\\
65.01	0.00999999999999999\\
66.01	0.00999999999999999\\
67.01	0.00999999999999999\\
68.01	0.00999999999999999\\
69.01	0.00999999999999999\\
70.01	0.00999999999999999\\
71.01	0.00999999999999999\\
72.01	0.00999999999999999\\
73.01	0.00999999999999999\\
74.01	0.00999999999999999\\
75.01	0.00999999999999999\\
76.01	0.00999999999999999\\
77.01	0.00999999999999999\\
78.01	0.00999999999999999\\
79.01	0.00999999999999999\\
80.01	0.00999999999999999\\
81.01	0.00999999999999999\\
82.01	0.00999999999999999\\
83.01	0.00999999999999999\\
84.01	0.00999999999999999\\
85.01	0.00999999999999999\\
86.01	0.00999999999999999\\
87.01	0.00999999999999999\\
88.01	0.00999999999999999\\
89.01	0.00999999999999999\\
90.01	0.00999999999999999\\
91.01	0.00999999999999999\\
92.01	0.00999999999999999\\
93.01	0.00999999999999999\\
94.01	0.00999999999999999\\
95.01	0.00999999999999999\\
96.01	0.00999999999999999\\
97.01	0.00999999999999999\\
98.01	0.00999999999999999\\
99.01	0.00999999999999999\\
100.01	0.00999999999999999\\
101.01	0.00999999999999999\\
102.01	0.00999999999999999\\
103.01	0.00999999999999999\\
104.01	0.00999999999999999\\
105.01	0.00999999999999999\\
106.01	0.00999999999999999\\
107.01	0.00999999999999999\\
108.01	0.00999999999999999\\
109.01	0.00999999999999999\\
110.01	0.00999999999999999\\
111.01	0.00999999999999999\\
112.01	0.00999999999999999\\
113.01	0.00999999999999999\\
114.01	0.00999999999999999\\
115.01	0.00999999999999999\\
116.01	0.00999999999999999\\
117.01	0.00999999999999999\\
118.01	0.00999999999999999\\
119.01	0.00999999999999999\\
120.01	0.00999999999999999\\
121.01	0.00999999999999999\\
122.01	0.00999999999999999\\
123.01	0.00999999999999999\\
124.01	0.00999999999999999\\
125.01	0.00999999999999999\\
126.01	0.00999999999999999\\
127.01	0.00999999999999999\\
128.01	0.00999999999999999\\
129.01	0.00999999999999999\\
130.01	0.00999999999999999\\
131.01	0.00999999999999999\\
132.01	0.00999999999999999\\
133.01	0.00999999999999999\\
134.01	0.00999999999999999\\
135.01	0.00999999999999999\\
136.01	0.00999999999999999\\
137.01	0.00999999999999999\\
138.01	0.00999999999999999\\
139.01	0.00999999999999999\\
140.01	0.00999999999999999\\
141.01	0.00999999999999999\\
142.01	0.00999999999999999\\
143.01	0.00999999999999999\\
144.01	0.00999999999999999\\
145.01	0.00999999999999999\\
146.01	0.00999999999999999\\
147.01	0.00999999999999999\\
148.01	0.00999999999999999\\
149.01	0.00999999999999999\\
150.01	0.00999999999999999\\
151.01	0.00999999999999999\\
152.01	0.00999999999999999\\
153.01	0.00999999999999999\\
154.01	0.00999999999999999\\
155.01	0.00999999999999999\\
156.01	0.00999999999999999\\
157.01	0.00999999999999999\\
158.01	0.00999999999999999\\
159.01	0.00999999999999999\\
160.01	0.00999999999999999\\
161.01	0.00999999999999999\\
162.01	0.00999999999999999\\
163.01	0.00999999999999999\\
164.01	0.00999999999999999\\
165.01	0.00999999999999999\\
166.01	0.00999999999999999\\
167.01	0.00999999999999999\\
168.01	0.00999999999999999\\
169.01	0.00999999999999999\\
170.01	0.00999999999999999\\
171.01	0.00999999999999999\\
172.01	0.00999999999999999\\
173.01	0.00999999999999999\\
174.01	0.00999999999999999\\
175.01	0.00999999999999999\\
176.01	0.00999999999999999\\
177.01	0.00999999999999999\\
178.01	0.00999999999999999\\
179.01	0.00999999999999999\\
180.01	0.00999999999999999\\
181.01	0.00999999999999999\\
182.01	0.00999999999999999\\
183.01	0.00999999999999999\\
184.01	0.00999999999999999\\
185.01	0.00999999999999999\\
186.01	0.00999999999999999\\
187.01	0.00999999999999999\\
188.01	0.00999999999999999\\
189.01	0.00999999999999999\\
190.01	0.00999999999999999\\
191.01	0.00999999999999999\\
192.01	0.00999999999999999\\
193.01	0.00999999999999999\\
194.01	0.00999999999999999\\
195.01	0.00999999999999999\\
196.01	0.00999999999999999\\
197.01	0.00999999999999999\\
198.01	0.00999999999999999\\
199.01	0.00999999999999999\\
200.01	0.00999999999999999\\
201.01	0.00999999999999999\\
202.01	0.00999999999999999\\
203.01	0.00999999999999999\\
204.01	0.00999999999999999\\
205.01	0.00999999999999999\\
206.01	0.00999999999999999\\
207.01	0.00999999999999999\\
208.01	0.00999999999999999\\
209.01	0.00999999999999999\\
210.01	0.00999999999999999\\
211.01	0.00999999999999999\\
212.01	0.00999999999999999\\
213.01	0.00999999999999999\\
214.01	0.00999999999999999\\
215.01	0.00999999999999999\\
216.01	0.00999999999999999\\
217.01	0.00999999999999999\\
218.01	0.00999999999999999\\
219.01	0.00999999999999999\\
220.01	0.00999999999999999\\
221.01	0.00999999999999999\\
222.01	0.00999999999999999\\
223.01	0.00999999999999999\\
224.01	0.00999999999999999\\
225.01	0.00999999999999999\\
226.01	0.00999999999999999\\
227.01	0.00999999999999999\\
228.01	0.00999999999999999\\
229.01	0.00999999999999999\\
230.01	0.00999999999999999\\
231.01	0.00999999999999999\\
232.01	0.00999999999999999\\
233.01	0.00999999999999999\\
234.01	0.00999999999999999\\
235.01	0.00999999999999999\\
236.01	0.00999999999999999\\
237.01	0.00999999999999999\\
238.01	0.00999999999999999\\
239.01	0.00999999999999999\\
240.01	0.00999999999999999\\
241.01	0.00999999999999999\\
242.01	0.00999999999999999\\
243.01	0.00999999999999999\\
244.01	0.00999999999999999\\
245.01	0.00999999999999999\\
246.01	0.00999999999999999\\
247.01	0.00999999999999999\\
248.01	0.00999999999999999\\
249.01	0.00999999999999999\\
250.01	0.00999999999999999\\
251.01	0.00999999999999999\\
252.01	0.00999999999999999\\
253.01	0.00999999999999999\\
254.01	0.00999999999999999\\
255.01	0.00999999999999999\\
256.01	0.00999999999999999\\
257.01	0.00999999999999999\\
258.01	0.00999999999999999\\
259.01	0.00999999999999999\\
260.01	0.00999999999999999\\
261.01	0.00999999999999999\\
262.01	0.00999999999999999\\
263.01	0.00999999999999999\\
264.01	0.00999999999999999\\
265.01	0.00999999999999999\\
266.01	0.00999999999999999\\
267.01	0.00999999999999999\\
268.01	0.00999999999999999\\
269.01	0.00999999999999999\\
270.01	0.00999999999999999\\
271.01	0.00999999999999999\\
272.01	0.00999999999999999\\
273.01	0.00999999999999999\\
274.01	0.00999999999999999\\
275.01	0.00999999999999999\\
276.01	0.00999999999999999\\
277.01	0.00999999999999999\\
278.01	0.00999999999999999\\
279.01	0.00999999999999999\\
280.01	0.00999999999999999\\
281.01	0.00999999999999999\\
282.01	0.00999999999999999\\
283.01	0.00999999999999999\\
284.01	0.00999999999999999\\
285.01	0.00999999999999999\\
286.01	0.00999999999999999\\
287.01	0.00999999999999999\\
288.01	0.00999999999999999\\
289.01	0.00999999999999999\\
290.01	0.00999999999999999\\
291.01	0.00999999999999999\\
292.01	0.00999999999999999\\
293.01	0.00999999999999999\\
294.01	0.00999999999999999\\
295.01	0.00999999999999999\\
296.01	0.00999999999999999\\
297.01	0.00999999999999999\\
298.01	0.00999999999999999\\
299.01	0.00999999999999999\\
300.01	0.00999999999999999\\
301.01	0.00999999999999999\\
302.01	0.00999999999999999\\
303.01	0.00999999999999999\\
304.01	0.00999999999999999\\
305.01	0.00999999999999999\\
306.01	0.00999999999999999\\
307.01	0.00999999999999999\\
308.01	0.00999999999999999\\
309.01	0.00999999999999999\\
310.01	0.00999999999999999\\
311.01	0.00999999999999999\\
312.01	0.00999999999999999\\
313.01	0.00999999999999999\\
314.01	0.00999999999999999\\
315.01	0.00999999999999999\\
316.01	0.00999999999999999\\
317.01	0.00999999999999999\\
318.01	0.00999999999999999\\
319.01	0.00999999999999999\\
320.01	0.00999999999999999\\
321.01	0.00999999999999999\\
322.01	0.00999999999999999\\
323.01	0.00999999999999999\\
324.01	0.00999999999999999\\
325.01	0.00999999999999999\\
326.01	0.00999999999999999\\
327.01	0.00999999999999999\\
328.01	0.00999999999999999\\
329.01	0.00999999999999999\\
330.01	0.00999999999999999\\
331.01	0.00999999999999999\\
332.01	0.00999999999999999\\
333.01	0.00999999999999999\\
334.01	0.00999999999999999\\
335.01	0.00999999999999999\\
336.01	0.00999999999999999\\
337.01	0.00999999999999999\\
338.01	0.00999999999999999\\
339.01	0.00999999999999999\\
340.01	0.00999999999999999\\
341.01	0.00999999999999999\\
342.01	0.00999999999999999\\
343.01	0.00999999999999999\\
344.01	0.00999999999999999\\
345.01	0.00999999999999999\\
346.01	0.00999999999999999\\
347.01	0.00999999999999999\\
348.01	0.00999999999999999\\
349.01	0.00999999999999999\\
350.01	0.00999999999999999\\
351.01	0.00999999999999999\\
352.01	0.00999999999999999\\
353.01	0.00999999999999999\\
354.01	0.00999999999999999\\
355.01	0.00999999999999999\\
356.01	0.00999999999999999\\
357.01	0.00999999999999999\\
358.01	0.00999999999999999\\
359.01	0.00999999999999999\\
360.01	0.00999999999999999\\
361.01	0.00999999999999999\\
362.01	0.00999999999999999\\
363.01	0.00999999999999999\\
364.01	0.00999999999999999\\
365.01	0.00999999999999999\\
366.01	0.00999999999999999\\
367.01	0.00999999999999999\\
368.01	0.00999999999999999\\
369.01	0.00999999999999999\\
370.01	0.00999999999999999\\
371.01	0.00999999999999999\\
372.01	0.00999999999999999\\
373.01	0.00999999999999999\\
374.01	0.00999999999999999\\
375.01	0.00999999999999999\\
376.01	0.00999999999999999\\
377.01	0.00999999999999999\\
378.01	0.00999999999999999\\
379.01	0.00999999999999999\\
380.01	0.00999999999999999\\
381.01	0.00999999999999999\\
382.01	0.00999999999999999\\
383.01	0.00999999999999999\\
384.01	0.00999999999999999\\
385.01	0.00999999999999999\\
386.01	0.00999999999999999\\
387.01	0.00999999999999999\\
388.01	0.00999999999999999\\
389.01	0.00999999999999999\\
390.01	0.00999999999999999\\
391.01	0.00999999999999999\\
392.01	0.00999999999999999\\
393.01	0.00999999999999999\\
394.01	0.00999999999999999\\
395.01	0.00999999999999999\\
396.01	0.00999999999999999\\
397.01	0.00999999999999999\\
398.01	0.00999999999999999\\
399.01	0.00999999999999999\\
400.01	0.00999999999999999\\
401.01	0.00999999999999999\\
402.01	0.00999999999999999\\
403.01	0.00999999999999999\\
404.01	0.00999999999999999\\
405.01	0.00999999999999999\\
406.01	0.00999999999999999\\
407.01	0.00999999999999999\\
408.01	0.00999999999999999\\
409.01	0.00999999999999999\\
410.01	0.00999999999999999\\
411.01	0.00999999999999999\\
412.01	0.00999999999999999\\
413.01	0.00999999999999999\\
414.01	0.00999999999999999\\
415.01	0.00999999999999999\\
416.01	0.00999999999999999\\
417.01	0.00999999999999999\\
418.01	0.00999999999999999\\
419.01	0.00999999999999999\\
420.01	0.00999999999999999\\
421.01	0.00999999999999999\\
422.01	0.00999999999999999\\
423.01	0.00999999999999999\\
424.01	0.00999999999999999\\
425.01	0.00999999999999999\\
426.01	0.00999999999999999\\
427.01	0.00999999999999999\\
428.01	0.00999999999999999\\
429.01	0.00999999999999999\\
430.01	0.00999999999999999\\
431.01	0.00999999999999999\\
432.01	0.00999999999999999\\
433.01	0.00999999999999999\\
434.01	0.00999999999999999\\
435.01	0.00999999999999999\\
436.01	0.00999999999999999\\
437.01	0.00999999999999999\\
438.01	0.00999999999999999\\
439.01	0.00999999999999999\\
440.01	0.00999999999999999\\
441.01	0.00999999999999999\\
442.01	0.00999999999999999\\
443.01	0.00999999999999999\\
444.01	0.00999999999999999\\
445.01	0.00999999999999999\\
446.01	0.00999999999999999\\
447.01	0.00999999999999999\\
448.01	0.00999999999999999\\
449.01	0.00999999999999999\\
450.01	0.00999999999999999\\
451.01	0.00999999999999999\\
452.01	0.00999999999999999\\
453.01	0.00999999999999999\\
454.01	0.00999999999999999\\
455.01	0.00999999999999999\\
456.01	0.00999999999999999\\
457.01	0.00999999999999999\\
458.01	0.00999999999999999\\
459.01	0.00999999999999999\\
460.01	0.00999999999999999\\
461.01	0.00999999999999999\\
462.01	0.00999999999999999\\
463.01	0.00999999999999999\\
464.01	0.00999999999999999\\
465.01	0.00999999999999999\\
466.01	0.00999999999999999\\
467.01	0.00999999999999999\\
468.01	0.00999999999999999\\
469.01	0.00999999999999999\\
470.01	0.00999999999999999\\
471.01	0.00999999999999999\\
472.01	0.00999999999999999\\
473.01	0.00999999999999999\\
474.01	0.00999999999999999\\
475.01	0.00999999999999999\\
476.01	0.00999999999999999\\
477.01	0.00999999999999999\\
478.01	0.00999999999999999\\
479.01	0.00999999999999999\\
480.01	0.00999999999999999\\
481.01	0.00999999999999999\\
482.01	0.00999999999999999\\
483.01	0.00999999999999999\\
484.01	0.00999999999999999\\
485.01	0.00999999999999999\\
486.01	0.00999999999999999\\
487.01	0.00999999999999999\\
488.01	0.00999999999999999\\
489.01	0.00999999999999999\\
490.01	0.00999999999999999\\
491.01	0.00999999999999999\\
492.01	0.00999999999999999\\
493.01	0.00999999999999999\\
494.01	0.00999999999999999\\
495.01	0.00999999999999999\\
496.01	0.00999999999999999\\
497.01	0.00999999999999999\\
498.01	0.00999999999999999\\
499.01	0.00999999999999999\\
500.01	0.00999999999999999\\
501.01	0.00999999999999999\\
502.01	0.00999999999999999\\
503.01	0.00999999999999999\\
504.01	0.00999999999999999\\
505.01	0.00999999999999999\\
506.01	0.00999999999999999\\
507.01	0.00999999999999999\\
508.01	0.00999999999999999\\
509.01	0.00999999999999999\\
510.01	0.00999999999999999\\
511.01	0.00999999999999999\\
512.01	0.00999999999999999\\
513.01	0.00999999999999999\\
514.01	0.00999999999999999\\
515.01	0.00999999999999999\\
516.01	0.00999999999999999\\
517.01	0.00999999999999999\\
518.01	0.00999999999999999\\
519.01	0.00999999999999999\\
520.01	0.00999999999999999\\
521.01	0.00999999999999999\\
522.01	0.00999999999999999\\
523.01	0.00999999999999999\\
524.01	0.00999999999999999\\
525.01	0.00999999999999999\\
526.01	0.00999999999999999\\
527.01	0.00999999999999999\\
528.01	0.00999999999999999\\
529.01	0.00999999999999999\\
530.01	0.00999999999999999\\
531.01	0.00999999999999999\\
532.01	0.00999999999999999\\
533.01	0.00999999999999999\\
534.01	0.00999999999999999\\
535.01	0.00999999999999999\\
536.01	0.00999999999999999\\
537.01	0.00999999999999999\\
538.01	0.00999999999999999\\
539.01	0.00999999999999999\\
540.01	0.00999999999999999\\
541.01	0.00999999999999999\\
542.01	0.00999999999999999\\
543.01	0.00999999999999999\\
544.01	0.00999999999999999\\
545.01	0.00999999999999999\\
546.01	0.00999999999999999\\
547.01	0.00999999999999999\\
548.01	0.00999999999999999\\
549.01	0.00999999999999999\\
550.01	0.00999999999999999\\
551.01	0.00999999999999999\\
552.01	0.00999999999999999\\
553.01	0.00999999999999999\\
554.01	0.00999999999999999\\
555.01	0.00999999999999999\\
556.01	0.00999999999999999\\
557.01	0.00999999999999999\\
558.01	0.00999999999999999\\
559.01	0.00999999999999999\\
560.01	0.00999999999999999\\
561.01	0.00999999999999999\\
562.01	0.00999999999999999\\
563.01	0.00999999999999999\\
564.01	0.00999999999999999\\
565.01	0.00999999999999999\\
566.01	0.00999999999999999\\
567.01	0.00999999999999999\\
568.01	0.00999999999999999\\
569.01	0.00999999999999999\\
570.01	0.00999999999999999\\
571.01	0.00999999999999999\\
572.01	0.00999999999999999\\
573.01	0.00999999999999999\\
574.01	0.00999999999999999\\
575.01	0.00999999999999999\\
576.01	0.00999999999999999\\
577.01	0.00999999999999999\\
578.01	0.00999999999999999\\
579.01	0.00999999999999999\\
580.01	0.00999999999999999\\
581.01	0.00999999999999999\\
582.01	0.00999999999999999\\
583.01	0.00999999999999999\\
584.01	0.00999999999999999\\
585.01	0.00999999999999999\\
586.01	0.00999999999999999\\
587.01	0.00999999999999999\\
588.01	0.00999999999999999\\
589.01	0.00999999999999999\\
590.01	0.00999999999999999\\
591.01	0.00999999999999999\\
592.01	0.00999999999999999\\
593.01	0.00999999999999999\\
594.01	0.00999999999999999\\
595.01	0.00999999999999999\\
596.01	0.00999999999999999\\
597.01	0.00999999999999999\\
598.01	0.00999999999999999\\
599.01	0.00624186909405523\\
599.02	0.0062041444124227\\
599.03	0.00616605275622837\\
599.04	0.00612759052101666\\
599.05	0.00608875406692124\\
599.06	0.00604953971831725\\
599.07	0.00600994376346995\\
599.08	0.00596996245417992\\
599.09	0.00592959200542494\\
599.1	0.00588882859499817\\
599.11	0.00584766836314282\\
599.12	0.00580610741218332\\
599.13	0.00576414180615277\\
599.14	0.00572176757041677\\
599.15	0.00567898069129352\\
599.16	0.00563577711567027\\
599.17	0.00559215275061587\\
599.18	0.00554810346298963\\
599.19	0.00550362507904635\\
599.2	0.00545871338403733\\
599.21	0.00541336412180758\\
599.22	0.00536757299438907\\
599.23	0.00532133566158991\\
599.24	0.00527464774057957\\
599.25	0.00522750480546996\\
599.26	0.0051799023868925\\
599.27	0.00513183597157084\\
599.28	0.00508330100188967\\
599.29	0.00503429287545899\\
599.3	0.0049848069446743\\
599.31	0.0049348385162724\\
599.32	0.00488438285088287\\
599.33	0.00483343516257501\\
599.34	0.00478199061840055\\
599.35	0.00473004433793169\\
599.36	0.00467759139279461\\
599.37	0.00462462680619856\\
599.38	0.00457114555246008\\
599.39	0.0045171425604974\\
599.4	0.00446261272376361\\
599.41	0.00440755088572623\\
599.42	0.00435195183937785\\
599.43	0.00429581032674213\\
599.44	0.00423912103837488\\
599.45	0.00418187861286023\\
599.46	0.00412407763630195\\
599.47	0.00406571264180972\\
599.48	0.00400677810898045\\
599.49	0.00394726846337443\\
599.5	0.00388717807598639\\
599.51	0.00382650126271144\\
599.52	0.00376523228380568\\
599.53	0.00370336534334162\\
599.54	0.00364089458865821\\
599.55	0.00357781410980551\\
599.56	0.00351411793898395\\
599.57	0.00344980004997801\\
599.58	0.0033848543575845\\
599.59	0.00331927471703512\\
599.6	0.00325305492341342\\
599.61	0.00318618871106605\\
599.62	0.00311866975300824\\
599.63	0.00305049166032345\\
599.64	0.00298164798155718\\
599.65	0.00291213220210484\\
599.66	0.00284193774359354\\
599.67	0.00277105796325799\\
599.68	0.00269948615331015\\
599.69	0.00262721554030275\\
599.7	0.00255423928448666\\
599.71	0.00248055047916181\\
599.72	0.00240614215002192\\
599.73	0.00233100725449273\\
599.74	0.00225513868106373\\
599.75	0.00217852924861344\\
599.76	0.002101171705728\\
599.77	0.0020230587300131\\
599.78	0.00194418292739928\\
599.79	0.00186453683144023\\
599.8	0.00178411290260446\\
599.81	0.00170290352755979\\
599.82	0.00162090101845109\\
599.83	0.00153809761217072\\
599.84	0.001454485469622\\
599.85	0.00137005667497534\\
599.86	0.00128480323491721\\
599.87	0.00119871707789167\\
599.88	0.00111179005333448\\
599.89	0.00102401393089983\\
599.9	0.000935380399679274\\
599.91	0.000845881067413286\\
599.92	0.00075550745969488\\
599.93	0.000664251019165564\\
599.94	0.000572103104703356\\
599.95	0.000479054990602912\\
599.96	0.000385097865747556\\
599.97	0.000290222832773275\\
599.98	0.000194420907224489\\
599.99	9.76830167015649e-05\\
600	0\\
};
\addplot [color=red!80!mycolor19,solid,forget plot]
  table[row sep=crcr]{%
0.01	0.00999999999999999\\
1.01	0.00999999999999999\\
2.01	0.00999999999999999\\
3.01	0.00999999999999999\\
4.01	0.00999999999999999\\
5.01	0.00999999999999999\\
6.01	0.00999999999999999\\
7.01	0.00999999999999999\\
8.01	0.00999999999999999\\
9.01	0.00999999999999999\\
10.01	0.00999999999999999\\
11.01	0.00999999999999999\\
12.01	0.00999999999999999\\
13.01	0.00999999999999999\\
14.01	0.00999999999999999\\
15.01	0.00999999999999999\\
16.01	0.00999999999999999\\
17.01	0.00999999999999999\\
18.01	0.00999999999999999\\
19.01	0.00999999999999999\\
20.01	0.00999999999999999\\
21.01	0.00999999999999999\\
22.01	0.00999999999999999\\
23.01	0.00999999999999999\\
24.01	0.00999999999999999\\
25.01	0.00999999999999999\\
26.01	0.00999999999999999\\
27.01	0.00999999999999999\\
28.01	0.00999999999999999\\
29.01	0.00999999999999999\\
30.01	0.00999999999999999\\
31.01	0.00999999999999999\\
32.01	0.00999999999999999\\
33.01	0.00999999999999999\\
34.01	0.00999999999999999\\
35.01	0.00999999999999999\\
36.01	0.00999999999999999\\
37.01	0.00999999999999999\\
38.01	0.00999999999999999\\
39.01	0.00999999999999999\\
40.01	0.00999999999999999\\
41.01	0.00999999999999999\\
42.01	0.00999999999999999\\
43.01	0.00999999999999999\\
44.01	0.00999999999999999\\
45.01	0.00999999999999999\\
46.01	0.00999999999999999\\
47.01	0.00999999999999999\\
48.01	0.00999999999999999\\
49.01	0.00999999999999999\\
50.01	0.00999999999999999\\
51.01	0.00999999999999999\\
52.01	0.00999999999999999\\
53.01	0.00999999999999999\\
54.01	0.00999999999999999\\
55.01	0.00999999999999999\\
56.01	0.00999999999999999\\
57.01	0.00999999999999999\\
58.01	0.00999999999999999\\
59.01	0.00999999999999999\\
60.01	0.00999999999999999\\
61.01	0.00999999999999999\\
62.01	0.00999999999999999\\
63.01	0.00999999999999999\\
64.01	0.00999999999999999\\
65.01	0.00999999999999999\\
66.01	0.00999999999999999\\
67.01	0.00999999999999999\\
68.01	0.00999999999999999\\
69.01	0.00999999999999999\\
70.01	0.00999999999999999\\
71.01	0.00999999999999999\\
72.01	0.00999999999999999\\
73.01	0.00999999999999999\\
74.01	0.00999999999999999\\
75.01	0.00999999999999999\\
76.01	0.00999999999999999\\
77.01	0.00999999999999999\\
78.01	0.00999999999999999\\
79.01	0.00999999999999999\\
80.01	0.00999999999999999\\
81.01	0.00999999999999999\\
82.01	0.00999999999999999\\
83.01	0.00999999999999999\\
84.01	0.00999999999999999\\
85.01	0.00999999999999999\\
86.01	0.00999999999999999\\
87.01	0.00999999999999999\\
88.01	0.00999999999999999\\
89.01	0.00999999999999999\\
90.01	0.00999999999999999\\
91.01	0.00999999999999999\\
92.01	0.00999999999999999\\
93.01	0.00999999999999999\\
94.01	0.00999999999999999\\
95.01	0.00999999999999999\\
96.01	0.00999999999999999\\
97.01	0.00999999999999999\\
98.01	0.00999999999999999\\
99.01	0.00999999999999999\\
100.01	0.00999999999999999\\
101.01	0.00999999999999999\\
102.01	0.00999999999999999\\
103.01	0.00999999999999999\\
104.01	0.00999999999999999\\
105.01	0.00999999999999999\\
106.01	0.00999999999999999\\
107.01	0.00999999999999999\\
108.01	0.00999999999999999\\
109.01	0.00999999999999999\\
110.01	0.00999999999999999\\
111.01	0.00999999999999999\\
112.01	0.00999999999999999\\
113.01	0.00999999999999999\\
114.01	0.00999999999999999\\
115.01	0.00999999999999999\\
116.01	0.00999999999999999\\
117.01	0.00999999999999999\\
118.01	0.00999999999999999\\
119.01	0.00999999999999999\\
120.01	0.00999999999999999\\
121.01	0.00999999999999999\\
122.01	0.00999999999999999\\
123.01	0.00999999999999999\\
124.01	0.00999999999999999\\
125.01	0.00999999999999999\\
126.01	0.00999999999999999\\
127.01	0.00999999999999999\\
128.01	0.00999999999999999\\
129.01	0.00999999999999999\\
130.01	0.00999999999999999\\
131.01	0.00999999999999999\\
132.01	0.00999999999999999\\
133.01	0.00999999999999999\\
134.01	0.00999999999999999\\
135.01	0.00999999999999999\\
136.01	0.00999999999999999\\
137.01	0.00999999999999999\\
138.01	0.00999999999999999\\
139.01	0.00999999999999999\\
140.01	0.00999999999999999\\
141.01	0.00999999999999999\\
142.01	0.00999999999999999\\
143.01	0.00999999999999999\\
144.01	0.00999999999999999\\
145.01	0.00999999999999999\\
146.01	0.00999999999999999\\
147.01	0.00999999999999999\\
148.01	0.00999999999999999\\
149.01	0.00999999999999999\\
150.01	0.00999999999999999\\
151.01	0.00999999999999999\\
152.01	0.00999999999999999\\
153.01	0.00999999999999999\\
154.01	0.00999999999999999\\
155.01	0.00999999999999999\\
156.01	0.00999999999999999\\
157.01	0.00999999999999999\\
158.01	0.00999999999999999\\
159.01	0.00999999999999999\\
160.01	0.00999999999999999\\
161.01	0.00999999999999999\\
162.01	0.00999999999999999\\
163.01	0.00999999999999999\\
164.01	0.00999999999999999\\
165.01	0.00999999999999999\\
166.01	0.00999999999999999\\
167.01	0.00999999999999999\\
168.01	0.00999999999999999\\
169.01	0.00999999999999999\\
170.01	0.00999999999999999\\
171.01	0.00999999999999999\\
172.01	0.00999999999999999\\
173.01	0.00999999999999999\\
174.01	0.00999999999999999\\
175.01	0.00999999999999999\\
176.01	0.00999999999999999\\
177.01	0.00999999999999999\\
178.01	0.00999999999999999\\
179.01	0.00999999999999999\\
180.01	0.00999999999999999\\
181.01	0.00999999999999999\\
182.01	0.00999999999999999\\
183.01	0.00999999999999999\\
184.01	0.00999999999999999\\
185.01	0.00999999999999999\\
186.01	0.00999999999999999\\
187.01	0.00999999999999999\\
188.01	0.00999999999999999\\
189.01	0.00999999999999999\\
190.01	0.00999999999999999\\
191.01	0.00999999999999999\\
192.01	0.00999999999999999\\
193.01	0.00999999999999999\\
194.01	0.00999999999999999\\
195.01	0.00999999999999999\\
196.01	0.00999999999999999\\
197.01	0.00999999999999999\\
198.01	0.00999999999999999\\
199.01	0.00999999999999999\\
200.01	0.00999999999999999\\
201.01	0.00999999999999999\\
202.01	0.00999999999999999\\
203.01	0.00999999999999999\\
204.01	0.00999999999999999\\
205.01	0.00999999999999999\\
206.01	0.00999999999999999\\
207.01	0.00999999999999999\\
208.01	0.00999999999999999\\
209.01	0.00999999999999999\\
210.01	0.00999999999999999\\
211.01	0.00999999999999999\\
212.01	0.00999999999999999\\
213.01	0.00999999999999999\\
214.01	0.00999999999999999\\
215.01	0.00999999999999999\\
216.01	0.00999999999999999\\
217.01	0.00999999999999999\\
218.01	0.00999999999999999\\
219.01	0.00999999999999999\\
220.01	0.00999999999999999\\
221.01	0.00999999999999999\\
222.01	0.00999999999999999\\
223.01	0.00999999999999999\\
224.01	0.00999999999999999\\
225.01	0.00999999999999999\\
226.01	0.00999999999999999\\
227.01	0.00999999999999999\\
228.01	0.00999999999999999\\
229.01	0.00999999999999999\\
230.01	0.00999999999999999\\
231.01	0.00999999999999999\\
232.01	0.00999999999999999\\
233.01	0.00999999999999999\\
234.01	0.00999999999999999\\
235.01	0.00999999999999999\\
236.01	0.00999999999999999\\
237.01	0.00999999999999999\\
238.01	0.00999999999999999\\
239.01	0.00999999999999999\\
240.01	0.00999999999999999\\
241.01	0.00999999999999999\\
242.01	0.00999999999999999\\
243.01	0.00999999999999999\\
244.01	0.00999999999999999\\
245.01	0.00999999999999999\\
246.01	0.00999999999999999\\
247.01	0.00999999999999999\\
248.01	0.00999999999999999\\
249.01	0.00999999999999999\\
250.01	0.00999999999999999\\
251.01	0.00999999999999999\\
252.01	0.00999999999999999\\
253.01	0.00999999999999999\\
254.01	0.00999999999999999\\
255.01	0.00999999999999999\\
256.01	0.00999999999999999\\
257.01	0.00999999999999999\\
258.01	0.00999999999999999\\
259.01	0.00999999999999999\\
260.01	0.00999999999999999\\
261.01	0.00999999999999999\\
262.01	0.00999999999999999\\
263.01	0.00999999999999999\\
264.01	0.00999999999999999\\
265.01	0.00999999999999999\\
266.01	0.00999999999999999\\
267.01	0.00999999999999999\\
268.01	0.00999999999999999\\
269.01	0.00999999999999999\\
270.01	0.00999999999999999\\
271.01	0.00999999999999999\\
272.01	0.00999999999999999\\
273.01	0.00999999999999999\\
274.01	0.00999999999999999\\
275.01	0.00999999999999999\\
276.01	0.00999999999999999\\
277.01	0.00999999999999999\\
278.01	0.00999999999999999\\
279.01	0.00999999999999999\\
280.01	0.00999999999999999\\
281.01	0.00999999999999999\\
282.01	0.00999999999999999\\
283.01	0.00999999999999999\\
284.01	0.00999999999999999\\
285.01	0.00999999999999999\\
286.01	0.00999999999999999\\
287.01	0.00999999999999999\\
288.01	0.00999999999999999\\
289.01	0.00999999999999999\\
290.01	0.00999999999999999\\
291.01	0.00999999999999999\\
292.01	0.00999999999999999\\
293.01	0.00999999999999999\\
294.01	0.00999999999999999\\
295.01	0.00999999999999999\\
296.01	0.00999999999999999\\
297.01	0.00999999999999999\\
298.01	0.00999999999999999\\
299.01	0.00999999999999999\\
300.01	0.00999999999999999\\
301.01	0.00999999999999999\\
302.01	0.00999999999999999\\
303.01	0.00999999999999999\\
304.01	0.00999999999999999\\
305.01	0.00999999999999999\\
306.01	0.00999999999999999\\
307.01	0.00999999999999999\\
308.01	0.00999999999999999\\
309.01	0.00999999999999999\\
310.01	0.00999999999999999\\
311.01	0.00999999999999999\\
312.01	0.00999999999999999\\
313.01	0.00999999999999999\\
314.01	0.00999999999999999\\
315.01	0.00999999999999999\\
316.01	0.00999999999999999\\
317.01	0.00999999999999999\\
318.01	0.00999999999999999\\
319.01	0.00999999999999999\\
320.01	0.00999999999999999\\
321.01	0.00999999999999999\\
322.01	0.00999999999999999\\
323.01	0.00999999999999999\\
324.01	0.00999999999999999\\
325.01	0.00999999999999999\\
326.01	0.00999999999999999\\
327.01	0.00999999999999999\\
328.01	0.00999999999999999\\
329.01	0.00999999999999999\\
330.01	0.00999999999999999\\
331.01	0.00999999999999999\\
332.01	0.00999999999999999\\
333.01	0.00999999999999999\\
334.01	0.00999999999999999\\
335.01	0.00999999999999999\\
336.01	0.00999999999999999\\
337.01	0.00999999999999999\\
338.01	0.00999999999999999\\
339.01	0.00999999999999999\\
340.01	0.00999999999999999\\
341.01	0.00999999999999999\\
342.01	0.00999999999999999\\
343.01	0.00999999999999999\\
344.01	0.00999999999999999\\
345.01	0.00999999999999999\\
346.01	0.00999999999999999\\
347.01	0.00999999999999999\\
348.01	0.00999999999999999\\
349.01	0.00999999999999999\\
350.01	0.00999999999999999\\
351.01	0.00999999999999999\\
352.01	0.00999999999999999\\
353.01	0.00999999999999999\\
354.01	0.00999999999999999\\
355.01	0.00999999999999999\\
356.01	0.00999999999999999\\
357.01	0.00999999999999999\\
358.01	0.00999999999999999\\
359.01	0.00999999999999999\\
360.01	0.00999999999999999\\
361.01	0.00999999999999999\\
362.01	0.00999999999999999\\
363.01	0.00999999999999999\\
364.01	0.00999999999999999\\
365.01	0.00999999999999999\\
366.01	0.00999999999999999\\
367.01	0.00999999999999999\\
368.01	0.00999999999999999\\
369.01	0.00999999999999999\\
370.01	0.00999999999999999\\
371.01	0.00999999999999999\\
372.01	0.00999999999999999\\
373.01	0.00999999999999999\\
374.01	0.00999999999999999\\
375.01	0.00999999999999999\\
376.01	0.00999999999999999\\
377.01	0.00999999999999999\\
378.01	0.00999999999999999\\
379.01	0.00999999999999999\\
380.01	0.00999999999999999\\
381.01	0.00999999999999999\\
382.01	0.00999999999999999\\
383.01	0.00999999999999999\\
384.01	0.00999999999999999\\
385.01	0.00999999999999999\\
386.01	0.00999999999999999\\
387.01	0.00999999999999999\\
388.01	0.00999999999999999\\
389.01	0.00999999999999999\\
390.01	0.00999999999999999\\
391.01	0.00999999999999999\\
392.01	0.00999999999999999\\
393.01	0.00999999999999999\\
394.01	0.00999999999999999\\
395.01	0.00999999999999999\\
396.01	0.00999999999999999\\
397.01	0.00999999999999999\\
398.01	0.00999999999999999\\
399.01	0.00999999999999999\\
400.01	0.00999999999999999\\
401.01	0.00999999999999999\\
402.01	0.00999999999999999\\
403.01	0.00999999999999999\\
404.01	0.00999999999999999\\
405.01	0.00999999999999999\\
406.01	0.00999999999999999\\
407.01	0.00999999999999999\\
408.01	0.00999999999999999\\
409.01	0.00999999999999999\\
410.01	0.00999999999999999\\
411.01	0.00999999999999999\\
412.01	0.00999999999999999\\
413.01	0.00999999999999999\\
414.01	0.00999999999999999\\
415.01	0.00999999999999999\\
416.01	0.00999999999999999\\
417.01	0.00999999999999999\\
418.01	0.00999999999999999\\
419.01	0.00999999999999999\\
420.01	0.00999999999999999\\
421.01	0.00999999999999999\\
422.01	0.00999999999999999\\
423.01	0.00999999999999999\\
424.01	0.00999999999999999\\
425.01	0.00999999999999999\\
426.01	0.00999999999999999\\
427.01	0.00999999999999999\\
428.01	0.00999999999999999\\
429.01	0.00999999999999999\\
430.01	0.00999999999999999\\
431.01	0.00999999999999999\\
432.01	0.00999999999999999\\
433.01	0.00999999999999999\\
434.01	0.00999999999999999\\
435.01	0.00999999999999999\\
436.01	0.00999999999999999\\
437.01	0.00999999999999999\\
438.01	0.00999999999999999\\
439.01	0.00999999999999999\\
440.01	0.00999999999999999\\
441.01	0.00999999999999999\\
442.01	0.00999999999999999\\
443.01	0.00999999999999999\\
444.01	0.00999999999999999\\
445.01	0.00999999999999999\\
446.01	0.00999999999999999\\
447.01	0.00999999999999999\\
448.01	0.00999999999999999\\
449.01	0.00999999999999999\\
450.01	0.00999999999999999\\
451.01	0.00999999999999999\\
452.01	0.00999999999999999\\
453.01	0.00999999999999999\\
454.01	0.00999999999999999\\
455.01	0.00999999999999999\\
456.01	0.00999999999999999\\
457.01	0.00999999999999999\\
458.01	0.00999999999999999\\
459.01	0.00999999999999999\\
460.01	0.00999999999999999\\
461.01	0.00999999999999999\\
462.01	0.00999999999999999\\
463.01	0.00999999999999999\\
464.01	0.00999999999999999\\
465.01	0.00999999999999999\\
466.01	0.00999999999999999\\
467.01	0.00999999999999999\\
468.01	0.00999999999999999\\
469.01	0.00999999999999999\\
470.01	0.00999999999999999\\
471.01	0.00999999999999999\\
472.01	0.00999999999999999\\
473.01	0.00999999999999999\\
474.01	0.00999999999999999\\
475.01	0.00999999999999999\\
476.01	0.00999999999999999\\
477.01	0.00999999999999999\\
478.01	0.00999999999999999\\
479.01	0.00999999999999999\\
480.01	0.00999999999999999\\
481.01	0.00999999999999999\\
482.01	0.00999999999999999\\
483.01	0.00999999999999999\\
484.01	0.00999999999999999\\
485.01	0.00999999999999999\\
486.01	0.00999999999999999\\
487.01	0.00999999999999999\\
488.01	0.00999999999999999\\
489.01	0.00999999999999999\\
490.01	0.00999999999999999\\
491.01	0.00999999999999999\\
492.01	0.00999999999999999\\
493.01	0.00999999999999999\\
494.01	0.00999999999999999\\
495.01	0.00999999999999999\\
496.01	0.00999999999999999\\
497.01	0.00999999999999999\\
498.01	0.00999999999999999\\
499.01	0.00999999999999999\\
500.01	0.00999999999999999\\
501.01	0.00999999999999999\\
502.01	0.00999999999999999\\
503.01	0.00999999999999999\\
504.01	0.00999999999999999\\
505.01	0.00999999999999999\\
506.01	0.00999999999999999\\
507.01	0.00999999999999999\\
508.01	0.00999999999999999\\
509.01	0.00999999999999999\\
510.01	0.00999999999999999\\
511.01	0.00999999999999999\\
512.01	0.00999999999999999\\
513.01	0.00999999999999999\\
514.01	0.00999999999999999\\
515.01	0.00999999999999999\\
516.01	0.00999999999999999\\
517.01	0.00999999999999999\\
518.01	0.00999999999999999\\
519.01	0.00999999999999999\\
520.01	0.00999999999999999\\
521.01	0.00999999999999999\\
522.01	0.00999999999999999\\
523.01	0.00999999999999999\\
524.01	0.00999999999999999\\
525.01	0.00999999999999999\\
526.01	0.00999999999999999\\
527.01	0.00999999999999999\\
528.01	0.00999999999999999\\
529.01	0.00999999999999999\\
530.01	0.00999999999999999\\
531.01	0.00999999999999999\\
532.01	0.00999999999999999\\
533.01	0.00999999999999999\\
534.01	0.00999999999999999\\
535.01	0.00999999999999999\\
536.01	0.00999999999999999\\
537.01	0.00999999999999999\\
538.01	0.00999999999999999\\
539.01	0.00999999999999999\\
540.01	0.00999999999999999\\
541.01	0.00999999999999999\\
542.01	0.00999999999999999\\
543.01	0.00999999999999999\\
544.01	0.00999999999999999\\
545.01	0.00999999999999999\\
546.01	0.00999999999999999\\
547.01	0.00999999999999999\\
548.01	0.00999999999999999\\
549.01	0.00999999999999999\\
550.01	0.00999999999999999\\
551.01	0.00999999999999999\\
552.01	0.00999999999999999\\
553.01	0.00999999999999999\\
554.01	0.00999999999999999\\
555.01	0.00999999999999999\\
556.01	0.00999999999999999\\
557.01	0.00999999999999999\\
558.01	0.00999999999999999\\
559.01	0.00999999999999999\\
560.01	0.00999999999999999\\
561.01	0.00999999999999999\\
562.01	0.00999999999999999\\
563.01	0.00999999999999999\\
564.01	0.00999999999999999\\
565.01	0.00999999999999999\\
566.01	0.00999999999999999\\
567.01	0.00999999999999999\\
568.01	0.00999999999999999\\
569.01	0.00999999999999999\\
570.01	0.00999999999999999\\
571.01	0.00999999999999999\\
572.01	0.00999999999999999\\
573.01	0.00999999999999999\\
574.01	0.00999999999999999\\
575.01	0.00999999999999999\\
576.01	0.00999999999999999\\
577.01	0.00999999999999999\\
578.01	0.00999999999999999\\
579.01	0.00999999999999999\\
580.01	0.00999999999999999\\
581.01	0.00999999999999999\\
582.01	0.00999999999999999\\
583.01	0.00999999999999999\\
584.01	0.00999999999999999\\
585.01	0.00999999999999999\\
586.01	0.00999999999999999\\
587.01	0.00999999999999999\\
588.01	0.00999999999999999\\
589.01	0.00999999999999999\\
590.01	0.00999999999999999\\
591.01	0.00999999999999999\\
592.01	0.00999999999999999\\
593.01	0.00999999999999999\\
594.01	0.00999999999999999\\
595.01	0.00999999999999999\\
596.01	0.00999999999999999\\
597.01	0.00999999999999999\\
598.01	0.00999999999999999\\
599.01	0.00624186909405409\\
599.02	0.00620414441242163\\
599.03	0.00616605275622736\\
599.04	0.0061275905210157\\
599.05	0.00608875406692034\\
599.06	0.00604953971831642\\
599.07	0.00600994376346916\\
599.08	0.0059699624541792\\
599.09	0.00592959200542426\\
599.1	0.00588882859499752\\
599.11	0.00584766836314221\\
599.12	0.00580610741218275\\
599.13	0.00576414180615225\\
599.14	0.00572176757041629\\
599.15	0.00567898069129309\\
599.16	0.00563577711566985\\
599.17	0.00559215275061547\\
599.18	0.00554810346298928\\
599.19	0.00550362507904604\\
599.2	0.00545871338403705\\
599.21	0.00541336412180734\\
599.22	0.00536757299438886\\
599.23	0.00532133566158972\\
599.24	0.00527464774057939\\
599.25	0.0052275048054698\\
599.26	0.00517990238689235\\
599.27	0.00513183597157072\\
599.28	0.00508330100188957\\
599.29	0.00503429287545889\\
599.3	0.00498480694467421\\
599.31	0.00493483851627234\\
599.32	0.00488438285088279\\
599.33	0.00483343516257495\\
599.34	0.00478199061840051\\
599.35	0.00473004433793165\\
599.36	0.00467759139279459\\
599.37	0.00462462680619853\\
599.38	0.00457114555246006\\
599.39	0.00451714256049738\\
599.4	0.00446261272376361\\
599.41	0.00440755088572623\\
599.42	0.00435195183937785\\
599.43	0.00429581032674214\\
599.44	0.00423912103837489\\
599.45	0.00418187861286024\\
599.46	0.00412407763630196\\
599.47	0.00406571264180974\\
599.48	0.00400677810898047\\
599.49	0.00394726846337445\\
599.5	0.00388717807598641\\
599.51	0.00382650126271146\\
599.52	0.0037652322838057\\
599.53	0.00370336534334164\\
599.54	0.00364089458865823\\
599.55	0.00357781410980553\\
599.56	0.00351411793898395\\
599.57	0.00344980004997801\\
599.58	0.00338485435758451\\
599.59	0.00331927471703513\\
599.6	0.00325305492341343\\
599.61	0.00318618871106605\\
599.62	0.00311866975300824\\
599.63	0.00305049166032345\\
599.64	0.00298164798155718\\
599.65	0.00291213220210483\\
599.66	0.00284193774359354\\
599.67	0.002771057963258\\
599.68	0.00269948615331015\\
599.69	0.00262721554030276\\
599.7	0.00255423928448666\\
599.71	0.00248055047916181\\
599.72	0.00240614215002192\\
599.73	0.00233100725449272\\
599.74	0.00225513868106373\\
599.75	0.00217852924861344\\
599.76	0.00210117170572799\\
599.77	0.0020230587300131\\
599.78	0.00194418292739928\\
599.79	0.00186453683144023\\
599.8	0.00178411290260445\\
599.81	0.00170290352755979\\
599.82	0.00162090101845109\\
599.83	0.00153809761217072\\
599.84	0.001454485469622\\
599.85	0.00137005667497534\\
599.86	0.00128480323491721\\
599.87	0.00119871707789166\\
599.88	0.00111179005333448\\
599.89	0.00102401393089982\\
599.9	0.000935380399679275\\
599.91	0.000845881067413283\\
599.92	0.00075550745969488\\
599.93	0.000664251019165561\\
599.94	0.000572103104703356\\
599.95	0.00047905499060291\\
599.96	0.000385097865747556\\
599.97	0.000290222832773275\\
599.98	0.000194420907224489\\
599.99	9.76830167015649e-05\\
600	0\\
};
\addplot [color=red,solid,forget plot]
  table[row sep=crcr]{%
0.01	0.00999999999999999\\
1.01	0.00999999999999999\\
2.01	0.00999999999999999\\
3.01	0.00999999999999999\\
4.01	0.00999999999999999\\
5.01	0.00999999999999999\\
6.01	0.00999999999999999\\
7.01	0.00999999999999999\\
8.01	0.00999999999999999\\
9.01	0.00999999999999999\\
10.01	0.00999999999999999\\
11.01	0.00999999999999999\\
12.01	0.00999999999999999\\
13.01	0.00999999999999999\\
14.01	0.00999999999999999\\
15.01	0.00999999999999999\\
16.01	0.00999999999999999\\
17.01	0.00999999999999999\\
18.01	0.00999999999999999\\
19.01	0.00999999999999999\\
20.01	0.00999999999999999\\
21.01	0.00999999999999999\\
22.01	0.00999999999999999\\
23.01	0.00999999999999999\\
24.01	0.00999999999999999\\
25.01	0.00999999999999999\\
26.01	0.00999999999999999\\
27.01	0.00999999999999999\\
28.01	0.00999999999999999\\
29.01	0.00999999999999999\\
30.01	0.00999999999999999\\
31.01	0.00999999999999999\\
32.01	0.00999999999999999\\
33.01	0.00999999999999999\\
34.01	0.00999999999999999\\
35.01	0.00999999999999999\\
36.01	0.00999999999999999\\
37.01	0.00999999999999999\\
38.01	0.00999999999999999\\
39.01	0.00999999999999999\\
40.01	0.00999999999999999\\
41.01	0.00999999999999999\\
42.01	0.00999999999999999\\
43.01	0.00999999999999999\\
44.01	0.00999999999999999\\
45.01	0.00999999999999999\\
46.01	0.00999999999999999\\
47.01	0.00999999999999999\\
48.01	0.00999999999999999\\
49.01	0.00999999999999999\\
50.01	0.00999999999999999\\
51.01	0.00999999999999999\\
52.01	0.00999999999999999\\
53.01	0.00999999999999999\\
54.01	0.00999999999999999\\
55.01	0.00999999999999999\\
56.01	0.00999999999999999\\
57.01	0.00999999999999999\\
58.01	0.00999999999999999\\
59.01	0.00999999999999999\\
60.01	0.00999999999999999\\
61.01	0.00999999999999999\\
62.01	0.00999999999999999\\
63.01	0.00999999999999999\\
64.01	0.00999999999999999\\
65.01	0.00999999999999999\\
66.01	0.00999999999999999\\
67.01	0.00999999999999999\\
68.01	0.00999999999999999\\
69.01	0.00999999999999999\\
70.01	0.00999999999999999\\
71.01	0.00999999999999999\\
72.01	0.00999999999999999\\
73.01	0.00999999999999999\\
74.01	0.00999999999999999\\
75.01	0.00999999999999999\\
76.01	0.00999999999999999\\
77.01	0.00999999999999999\\
78.01	0.00999999999999999\\
79.01	0.00999999999999999\\
80.01	0.00999999999999999\\
81.01	0.00999999999999999\\
82.01	0.00999999999999999\\
83.01	0.00999999999999999\\
84.01	0.00999999999999999\\
85.01	0.00999999999999999\\
86.01	0.00999999999999999\\
87.01	0.00999999999999999\\
88.01	0.00999999999999999\\
89.01	0.00999999999999999\\
90.01	0.00999999999999999\\
91.01	0.00999999999999999\\
92.01	0.00999999999999999\\
93.01	0.00999999999999999\\
94.01	0.00999999999999999\\
95.01	0.00999999999999999\\
96.01	0.00999999999999999\\
97.01	0.00999999999999999\\
98.01	0.00999999999999999\\
99.01	0.00999999999999999\\
100.01	0.00999999999999999\\
101.01	0.00999999999999999\\
102.01	0.00999999999999999\\
103.01	0.00999999999999999\\
104.01	0.00999999999999999\\
105.01	0.00999999999999999\\
106.01	0.00999999999999999\\
107.01	0.00999999999999999\\
108.01	0.00999999999999999\\
109.01	0.00999999999999999\\
110.01	0.00999999999999999\\
111.01	0.00999999999999999\\
112.01	0.00999999999999999\\
113.01	0.00999999999999999\\
114.01	0.00999999999999999\\
115.01	0.00999999999999999\\
116.01	0.00999999999999999\\
117.01	0.00999999999999999\\
118.01	0.00999999999999999\\
119.01	0.00999999999999999\\
120.01	0.00999999999999999\\
121.01	0.00999999999999999\\
122.01	0.00999999999999999\\
123.01	0.00999999999999999\\
124.01	0.00999999999999999\\
125.01	0.00999999999999999\\
126.01	0.00999999999999999\\
127.01	0.00999999999999999\\
128.01	0.00999999999999999\\
129.01	0.00999999999999999\\
130.01	0.00999999999999999\\
131.01	0.00999999999999999\\
132.01	0.00999999999999999\\
133.01	0.00999999999999999\\
134.01	0.00999999999999999\\
135.01	0.00999999999999999\\
136.01	0.00999999999999999\\
137.01	0.00999999999999999\\
138.01	0.00999999999999999\\
139.01	0.00999999999999999\\
140.01	0.00999999999999999\\
141.01	0.00999999999999999\\
142.01	0.00999999999999999\\
143.01	0.00999999999999999\\
144.01	0.00999999999999999\\
145.01	0.00999999999999999\\
146.01	0.00999999999999999\\
147.01	0.00999999999999999\\
148.01	0.00999999999999999\\
149.01	0.00999999999999999\\
150.01	0.00999999999999999\\
151.01	0.00999999999999999\\
152.01	0.00999999999999999\\
153.01	0.00999999999999999\\
154.01	0.00999999999999999\\
155.01	0.00999999999999999\\
156.01	0.00999999999999999\\
157.01	0.00999999999999999\\
158.01	0.00999999999999999\\
159.01	0.00999999999999999\\
160.01	0.00999999999999999\\
161.01	0.00999999999999999\\
162.01	0.00999999999999999\\
163.01	0.00999999999999999\\
164.01	0.00999999999999999\\
165.01	0.00999999999999999\\
166.01	0.00999999999999999\\
167.01	0.00999999999999999\\
168.01	0.00999999999999999\\
169.01	0.00999999999999999\\
170.01	0.00999999999999999\\
171.01	0.00999999999999999\\
172.01	0.00999999999999999\\
173.01	0.00999999999999999\\
174.01	0.00999999999999999\\
175.01	0.00999999999999999\\
176.01	0.00999999999999999\\
177.01	0.00999999999999999\\
178.01	0.00999999999999999\\
179.01	0.00999999999999999\\
180.01	0.00999999999999999\\
181.01	0.00999999999999999\\
182.01	0.00999999999999999\\
183.01	0.00999999999999999\\
184.01	0.00999999999999999\\
185.01	0.00999999999999999\\
186.01	0.00999999999999999\\
187.01	0.00999999999999999\\
188.01	0.00999999999999999\\
189.01	0.00999999999999999\\
190.01	0.00999999999999999\\
191.01	0.00999999999999999\\
192.01	0.00999999999999999\\
193.01	0.00999999999999999\\
194.01	0.00999999999999999\\
195.01	0.00999999999999999\\
196.01	0.00999999999999999\\
197.01	0.00999999999999999\\
198.01	0.00999999999999999\\
199.01	0.00999999999999999\\
200.01	0.00999999999999999\\
201.01	0.00999999999999999\\
202.01	0.00999999999999999\\
203.01	0.00999999999999999\\
204.01	0.00999999999999999\\
205.01	0.00999999999999999\\
206.01	0.00999999999999999\\
207.01	0.00999999999999999\\
208.01	0.00999999999999999\\
209.01	0.00999999999999999\\
210.01	0.00999999999999999\\
211.01	0.00999999999999999\\
212.01	0.00999999999999999\\
213.01	0.00999999999999999\\
214.01	0.00999999999999999\\
215.01	0.00999999999999999\\
216.01	0.00999999999999999\\
217.01	0.00999999999999999\\
218.01	0.00999999999999999\\
219.01	0.00999999999999999\\
220.01	0.00999999999999999\\
221.01	0.00999999999999999\\
222.01	0.00999999999999999\\
223.01	0.00999999999999999\\
224.01	0.00999999999999999\\
225.01	0.00999999999999999\\
226.01	0.00999999999999999\\
227.01	0.00999999999999999\\
228.01	0.00999999999999999\\
229.01	0.00999999999999999\\
230.01	0.00999999999999999\\
231.01	0.00999999999999999\\
232.01	0.00999999999999999\\
233.01	0.00999999999999999\\
234.01	0.00999999999999999\\
235.01	0.00999999999999999\\
236.01	0.00999999999999999\\
237.01	0.00999999999999999\\
238.01	0.00999999999999999\\
239.01	0.00999999999999999\\
240.01	0.00999999999999999\\
241.01	0.00999999999999999\\
242.01	0.00999999999999999\\
243.01	0.00999999999999999\\
244.01	0.00999999999999999\\
245.01	0.00999999999999999\\
246.01	0.00999999999999999\\
247.01	0.00999999999999999\\
248.01	0.00999999999999999\\
249.01	0.00999999999999999\\
250.01	0.00999999999999999\\
251.01	0.00999999999999999\\
252.01	0.00999999999999999\\
253.01	0.00999999999999999\\
254.01	0.00999999999999999\\
255.01	0.00999999999999999\\
256.01	0.00999999999999999\\
257.01	0.00999999999999999\\
258.01	0.00999999999999999\\
259.01	0.00999999999999999\\
260.01	0.00999999999999999\\
261.01	0.00999999999999999\\
262.01	0.00999999999999999\\
263.01	0.00999999999999999\\
264.01	0.00999999999999999\\
265.01	0.00999999999999999\\
266.01	0.00999999999999999\\
267.01	0.00999999999999999\\
268.01	0.00999999999999999\\
269.01	0.00999999999999999\\
270.01	0.00999999999999999\\
271.01	0.00999999999999999\\
272.01	0.00999999999999999\\
273.01	0.00999999999999999\\
274.01	0.00999999999999999\\
275.01	0.00999999999999999\\
276.01	0.00999999999999999\\
277.01	0.00999999999999999\\
278.01	0.00999999999999999\\
279.01	0.00999999999999999\\
280.01	0.00999999999999999\\
281.01	0.00999999999999999\\
282.01	0.00999999999999999\\
283.01	0.00999999999999999\\
284.01	0.00999999999999999\\
285.01	0.00999999999999999\\
286.01	0.00999999999999999\\
287.01	0.00999999999999999\\
288.01	0.00999999999999999\\
289.01	0.00999999999999999\\
290.01	0.00999999999999999\\
291.01	0.00999999999999999\\
292.01	0.00999999999999999\\
293.01	0.00999999999999999\\
294.01	0.00999999999999999\\
295.01	0.00999999999999999\\
296.01	0.00999999999999999\\
297.01	0.00999999999999999\\
298.01	0.00999999999999999\\
299.01	0.00999999999999999\\
300.01	0.00999999999999999\\
301.01	0.00999999999999999\\
302.01	0.00999999999999999\\
303.01	0.00999999999999999\\
304.01	0.00999999999999999\\
305.01	0.00999999999999999\\
306.01	0.00999999999999999\\
307.01	0.00999999999999999\\
308.01	0.00999999999999999\\
309.01	0.00999999999999999\\
310.01	0.00999999999999999\\
311.01	0.00999999999999999\\
312.01	0.00999999999999999\\
313.01	0.00999999999999999\\
314.01	0.00999999999999999\\
315.01	0.00999999999999999\\
316.01	0.00999999999999999\\
317.01	0.00999999999999999\\
318.01	0.00999999999999999\\
319.01	0.00999999999999999\\
320.01	0.00999999999999999\\
321.01	0.00999999999999999\\
322.01	0.00999999999999999\\
323.01	0.00999999999999999\\
324.01	0.00999999999999999\\
325.01	0.00999999999999999\\
326.01	0.00999999999999999\\
327.01	0.00999999999999999\\
328.01	0.00999999999999999\\
329.01	0.00999999999999999\\
330.01	0.00999999999999999\\
331.01	0.00999999999999999\\
332.01	0.00999999999999999\\
333.01	0.00999999999999999\\
334.01	0.00999999999999999\\
335.01	0.00999999999999999\\
336.01	0.00999999999999999\\
337.01	0.00999999999999999\\
338.01	0.00999999999999999\\
339.01	0.00999999999999999\\
340.01	0.00999999999999999\\
341.01	0.00999999999999999\\
342.01	0.00999999999999999\\
343.01	0.00999999999999999\\
344.01	0.00999999999999999\\
345.01	0.00999999999999999\\
346.01	0.00999999999999999\\
347.01	0.00999999999999999\\
348.01	0.00999999999999999\\
349.01	0.00999999999999999\\
350.01	0.00999999999999999\\
351.01	0.00999999999999999\\
352.01	0.00999999999999999\\
353.01	0.00999999999999999\\
354.01	0.00999999999999999\\
355.01	0.00999999999999999\\
356.01	0.00999999999999999\\
357.01	0.00999999999999999\\
358.01	0.00999999999999999\\
359.01	0.00999999999999999\\
360.01	0.00999999999999999\\
361.01	0.00999999999999999\\
362.01	0.00999999999999999\\
363.01	0.00999999999999999\\
364.01	0.00999999999999999\\
365.01	0.00999999999999999\\
366.01	0.00999999999999999\\
367.01	0.00999999999999999\\
368.01	0.00999999999999999\\
369.01	0.00999999999999999\\
370.01	0.00999999999999999\\
371.01	0.00999999999999999\\
372.01	0.00999999999999999\\
373.01	0.00999999999999999\\
374.01	0.00999999999999999\\
375.01	0.00999999999999999\\
376.01	0.00999999999999999\\
377.01	0.00999999999999999\\
378.01	0.00999999999999999\\
379.01	0.00999999999999999\\
380.01	0.00999999999999999\\
381.01	0.00999999999999999\\
382.01	0.00999999999999999\\
383.01	0.00999999999999999\\
384.01	0.00999999999999999\\
385.01	0.00999999999999999\\
386.01	0.00999999999999999\\
387.01	0.00999999999999999\\
388.01	0.00999999999999999\\
389.01	0.00999999999999999\\
390.01	0.00999999999999999\\
391.01	0.00999999999999999\\
392.01	0.00999999999999999\\
393.01	0.00999999999999999\\
394.01	0.00999999999999999\\
395.01	0.00999999999999999\\
396.01	0.00999999999999999\\
397.01	0.00999999999999999\\
398.01	0.00999999999999999\\
399.01	0.00999999999999999\\
400.01	0.00999999999999999\\
401.01	0.00999999999999999\\
402.01	0.00999999999999999\\
403.01	0.00999999999999999\\
404.01	0.00999999999999999\\
405.01	0.00999999999999999\\
406.01	0.00999999999999999\\
407.01	0.00999999999999999\\
408.01	0.00999999999999999\\
409.01	0.00999999999999999\\
410.01	0.00999999999999999\\
411.01	0.00999999999999999\\
412.01	0.00999999999999999\\
413.01	0.00999999999999999\\
414.01	0.00999999999999999\\
415.01	0.00999999999999999\\
416.01	0.00999999999999999\\
417.01	0.00999999999999999\\
418.01	0.00999999999999999\\
419.01	0.00999999999999999\\
420.01	0.00999999999999999\\
421.01	0.00999999999999999\\
422.01	0.00999999999999999\\
423.01	0.00999999999999999\\
424.01	0.00999999999999999\\
425.01	0.00999999999999999\\
426.01	0.00999999999999999\\
427.01	0.00999999999999999\\
428.01	0.00999999999999999\\
429.01	0.00999999999999999\\
430.01	0.00999999999999999\\
431.01	0.00999999999999999\\
432.01	0.00999999999999999\\
433.01	0.00999999999999999\\
434.01	0.00999999999999999\\
435.01	0.00999999999999999\\
436.01	0.00999999999999999\\
437.01	0.00999999999999999\\
438.01	0.00999999999999999\\
439.01	0.00999999999999999\\
440.01	0.00999999999999999\\
441.01	0.00999999999999999\\
442.01	0.00999999999999999\\
443.01	0.00999999999999999\\
444.01	0.00999999999999999\\
445.01	0.00999999999999999\\
446.01	0.00999999999999999\\
447.01	0.00999999999999999\\
448.01	0.00999999999999999\\
449.01	0.00999999999999999\\
450.01	0.00999999999999999\\
451.01	0.00999999999999999\\
452.01	0.00999999999999999\\
453.01	0.00999999999999999\\
454.01	0.00999999999999999\\
455.01	0.00999999999999999\\
456.01	0.00999999999999999\\
457.01	0.00999999999999999\\
458.01	0.00999999999999999\\
459.01	0.00999999999999999\\
460.01	0.00999999999999999\\
461.01	0.00999999999999999\\
462.01	0.00999999999999999\\
463.01	0.00999999999999999\\
464.01	0.00999999999999999\\
465.01	0.00999999999999999\\
466.01	0.00999999999999999\\
467.01	0.00999999999999999\\
468.01	0.00999999999999999\\
469.01	0.00999999999999999\\
470.01	0.00999999999999999\\
471.01	0.00999999999999999\\
472.01	0.00999999999999999\\
473.01	0.00999999999999999\\
474.01	0.00999999999999999\\
475.01	0.00999999999999999\\
476.01	0.00999999999999999\\
477.01	0.00999999999999999\\
478.01	0.00999999999999999\\
479.01	0.00999999999999999\\
480.01	0.00999999999999999\\
481.01	0.00999999999999999\\
482.01	0.00999999999999999\\
483.01	0.00999999999999999\\
484.01	0.00999999999999999\\
485.01	0.00999999999999999\\
486.01	0.00999999999999999\\
487.01	0.00999999999999999\\
488.01	0.00999999999999999\\
489.01	0.00999999999999999\\
490.01	0.00999999999999999\\
491.01	0.00999999999999999\\
492.01	0.00999999999999999\\
493.01	0.00999999999999999\\
494.01	0.00999999999999999\\
495.01	0.00999999999999999\\
496.01	0.00999999999999999\\
497.01	0.00999999999999999\\
498.01	0.00999999999999999\\
499.01	0.00999999999999999\\
500.01	0.00999999999999999\\
501.01	0.00999999999999999\\
502.01	0.00999999999999999\\
503.01	0.00999999999999999\\
504.01	0.00999999999999999\\
505.01	0.00999999999999999\\
506.01	0.00999999999999999\\
507.01	0.00999999999999999\\
508.01	0.00999999999999999\\
509.01	0.00999999999999999\\
510.01	0.00999999999999999\\
511.01	0.00999999999999999\\
512.01	0.00999999999999999\\
513.01	0.00999999999999999\\
514.01	0.00999999999999999\\
515.01	0.00999999999999999\\
516.01	0.00999999999999999\\
517.01	0.00999999999999999\\
518.01	0.00999999999999999\\
519.01	0.00999999999999999\\
520.01	0.00999999999999999\\
521.01	0.00999999999999999\\
522.01	0.00999999999999999\\
523.01	0.00999999999999999\\
524.01	0.00999999999999999\\
525.01	0.00999999999999999\\
526.01	0.00999999999999999\\
527.01	0.00999999999999999\\
528.01	0.00999999999999999\\
529.01	0.00999999999999999\\
530.01	0.00999999999999999\\
531.01	0.00999999999999999\\
532.01	0.00999999999999999\\
533.01	0.00999999999999999\\
534.01	0.00999999999999999\\
535.01	0.00999999999999999\\
536.01	0.00999999999999999\\
537.01	0.00999999999999999\\
538.01	0.00999999999999999\\
539.01	0.00999999999999999\\
540.01	0.00999999999999999\\
541.01	0.00999999999999999\\
542.01	0.00999999999999999\\
543.01	0.00999999999999999\\
544.01	0.00999999999999999\\
545.01	0.00999999999999999\\
546.01	0.00999999999999999\\
547.01	0.00999999999999999\\
548.01	0.00999999999999999\\
549.01	0.00999999999999999\\
550.01	0.00999999999999999\\
551.01	0.00999999999999999\\
552.01	0.00999999999999999\\
553.01	0.00999999999999999\\
554.01	0.00999999999999999\\
555.01	0.00999999999999999\\
556.01	0.00999999999999999\\
557.01	0.00999999999999999\\
558.01	0.00999999999999999\\
559.01	0.00999999999999999\\
560.01	0.00999999999999999\\
561.01	0.00999999999999999\\
562.01	0.00999999999999999\\
563.01	0.00999999999999999\\
564.01	0.00999999999999999\\
565.01	0.00999999999999999\\
566.01	0.00999999999999999\\
567.01	0.00999999999999999\\
568.01	0.00999999999999999\\
569.01	0.00999999999999999\\
570.01	0.00999999999999999\\
571.01	0.00999999999999999\\
572.01	0.00999999999999999\\
573.01	0.00999999999999999\\
574.01	0.00999999999999999\\
575.01	0.00999999999999999\\
576.01	0.00999999999999999\\
577.01	0.00999999999999999\\
578.01	0.00999999999999999\\
579.01	0.00999999999999999\\
580.01	0.00999999999999999\\
581.01	0.00999999999999999\\
582.01	0.00999999999999999\\
583.01	0.00999999999999999\\
584.01	0.00999999999999999\\
585.01	0.00999999999999999\\
586.01	0.00999999999999999\\
587.01	0.00999999999999999\\
588.01	0.00999999999999999\\
589.01	0.00999999999999999\\
590.01	0.00999999999999999\\
591.01	0.00999999999999999\\
592.01	0.00999999999999999\\
593.01	0.00999999999999999\\
594.01	0.00999999999999999\\
595.01	0.00999999999999999\\
596.01	0.00999999999999999\\
597.01	0.00999999999999999\\
598.01	0.00999999999999999\\
599.01	0.00624186909405409\\
599.02	0.00620414441242161\\
599.03	0.00616605275622734\\
599.04	0.00612759052101569\\
599.05	0.00608875406692034\\
599.06	0.00604953971831641\\
599.07	0.00600994376346915\\
599.08	0.00596996245417917\\
599.09	0.00592959200542424\\
599.1	0.0058888285949975\\
599.11	0.00584766836314219\\
599.12	0.00580610741218273\\
599.13	0.00576414180615222\\
599.14	0.00572176757041625\\
599.15	0.00567898069129305\\
599.16	0.00563577711566982\\
599.17	0.00559215275061544\\
599.18	0.00554810346298924\\
599.19	0.00550362507904598\\
599.2	0.005458713384037\\
599.21	0.00541336412180727\\
599.22	0.00536757299438878\\
599.23	0.00532133566158964\\
599.24	0.00527464774057931\\
599.25	0.00522750480546972\\
599.26	0.00517990238689226\\
599.27	0.00513183597157064\\
599.28	0.0050833010018895\\
599.29	0.00503429287545883\\
599.3	0.00498480694467416\\
599.31	0.00493483851627227\\
599.32	0.00488438285088274\\
599.33	0.0048334351625749\\
599.34	0.00478199061840046\\
599.35	0.0047300443379316\\
599.36	0.00467759139279454\\
599.37	0.00462462680619849\\
599.38	0.00457114555246003\\
599.39	0.00451714256049735\\
599.4	0.00446261272376357\\
599.41	0.00440755088572618\\
599.42	0.00435195183937782\\
599.43	0.00429581032674211\\
599.44	0.00423912103837487\\
599.45	0.00418187861286022\\
599.46	0.00412407763630193\\
599.47	0.00406571264180971\\
599.48	0.00400677810898045\\
599.49	0.00394726846337441\\
599.5	0.00388717807598638\\
599.51	0.00382650126271143\\
599.52	0.00376523228380568\\
599.53	0.00370336534334161\\
599.54	0.00364089458865821\\
599.55	0.00357781410980551\\
599.56	0.00351411793898393\\
599.57	0.00344980004997799\\
599.58	0.00338485435758448\\
599.59	0.0033192747170351\\
599.6	0.00325305492341341\\
599.61	0.00318618871106604\\
599.62	0.00311866975300822\\
599.63	0.00305049166032343\\
599.64	0.00298164798155717\\
599.65	0.00291213220210483\\
599.66	0.00284193774359353\\
599.67	0.00277105796325798\\
599.68	0.00269948615331014\\
599.69	0.00262721554030275\\
599.7	0.00255423928448666\\
599.71	0.00248055047916181\\
599.72	0.00240614215002192\\
599.73	0.00233100725449273\\
599.74	0.00225513868106373\\
599.75	0.00217852924861344\\
599.76	0.00210117170572799\\
599.77	0.0020230587300131\\
599.78	0.00194418292739928\\
599.79	0.00186453683144023\\
599.8	0.00178411290260446\\
599.81	0.00170290352755979\\
599.82	0.00162090101845109\\
599.83	0.00153809761217072\\
599.84	0.001454485469622\\
599.85	0.00137005667497534\\
599.86	0.00128480323491721\\
599.87	0.00119871707789166\\
599.88	0.00111179005333448\\
599.89	0.00102401393089982\\
599.9	0.000935380399679274\\
599.91	0.000845881067413288\\
599.92	0.000755507459694882\\
599.93	0.000664251019165561\\
599.94	0.000572103104703355\\
599.95	0.00047905499060291\\
599.96	0.000385097865747554\\
599.97	0.000290222832773275\\
599.98	0.000194420907224489\\
599.99	9.76830167015649e-05\\
600	0\\
};
\addplot [color=mycolor20,solid,forget plot]
  table[row sep=crcr]{%
0.01	0.01\\
1.01	0.01\\
2.01	0.01\\
3.01	0.01\\
4.01	0.01\\
5.01	0.01\\
6.01	0.01\\
7.01	0.01\\
8.01	0.01\\
9.01	0.01\\
10.01	0.01\\
11.01	0.01\\
12.01	0.01\\
13.01	0.01\\
14.01	0.01\\
15.01	0.01\\
16.01	0.01\\
17.01	0.01\\
18.01	0.01\\
19.01	0.01\\
20.01	0.01\\
21.01	0.01\\
22.01	0.01\\
23.01	0.01\\
24.01	0.01\\
25.01	0.01\\
26.01	0.01\\
27.01	0.01\\
28.01	0.01\\
29.01	0.01\\
30.01	0.01\\
31.01	0.01\\
32.01	0.01\\
33.01	0.01\\
34.01	0.01\\
35.01	0.01\\
36.01	0.01\\
37.01	0.01\\
38.01	0.01\\
39.01	0.01\\
40.01	0.01\\
41.01	0.01\\
42.01	0.01\\
43.01	0.01\\
44.01	0.01\\
45.01	0.01\\
46.01	0.01\\
47.01	0.01\\
48.01	0.01\\
49.01	0.01\\
50.01	0.01\\
51.01	0.01\\
52.01	0.01\\
53.01	0.01\\
54.01	0.01\\
55.01	0.01\\
56.01	0.01\\
57.01	0.01\\
58.01	0.01\\
59.01	0.01\\
60.01	0.01\\
61.01	0.01\\
62.01	0.01\\
63.01	0.01\\
64.01	0.01\\
65.01	0.01\\
66.01	0.01\\
67.01	0.01\\
68.01	0.01\\
69.01	0.01\\
70.01	0.01\\
71.01	0.01\\
72.01	0.01\\
73.01	0.01\\
74.01	0.01\\
75.01	0.01\\
76.01	0.01\\
77.01	0.01\\
78.01	0.01\\
79.01	0.01\\
80.01	0.01\\
81.01	0.01\\
82.01	0.01\\
83.01	0.01\\
84.01	0.01\\
85.01	0.01\\
86.01	0.01\\
87.01	0.01\\
88.01	0.01\\
89.01	0.01\\
90.01	0.01\\
91.01	0.01\\
92.01	0.01\\
93.01	0.01\\
94.01	0.01\\
95.01	0.01\\
96.01	0.01\\
97.01	0.01\\
98.01	0.01\\
99.01	0.01\\
100.01	0.01\\
101.01	0.01\\
102.01	0.01\\
103.01	0.01\\
104.01	0.01\\
105.01	0.01\\
106.01	0.01\\
107.01	0.01\\
108.01	0.01\\
109.01	0.01\\
110.01	0.01\\
111.01	0.01\\
112.01	0.01\\
113.01	0.01\\
114.01	0.01\\
115.01	0.01\\
116.01	0.01\\
117.01	0.01\\
118.01	0.01\\
119.01	0.01\\
120.01	0.01\\
121.01	0.01\\
122.01	0.01\\
123.01	0.01\\
124.01	0.01\\
125.01	0.01\\
126.01	0.01\\
127.01	0.01\\
128.01	0.01\\
129.01	0.01\\
130.01	0.01\\
131.01	0.01\\
132.01	0.01\\
133.01	0.01\\
134.01	0.01\\
135.01	0.01\\
136.01	0.01\\
137.01	0.01\\
138.01	0.01\\
139.01	0.01\\
140.01	0.01\\
141.01	0.01\\
142.01	0.01\\
143.01	0.01\\
144.01	0.01\\
145.01	0.01\\
146.01	0.01\\
147.01	0.01\\
148.01	0.01\\
149.01	0.01\\
150.01	0.01\\
151.01	0.01\\
152.01	0.01\\
153.01	0.01\\
154.01	0.01\\
155.01	0.01\\
156.01	0.01\\
157.01	0.01\\
158.01	0.01\\
159.01	0.01\\
160.01	0.01\\
161.01	0.01\\
162.01	0.01\\
163.01	0.01\\
164.01	0.01\\
165.01	0.01\\
166.01	0.01\\
167.01	0.01\\
168.01	0.01\\
169.01	0.01\\
170.01	0.01\\
171.01	0.01\\
172.01	0.01\\
173.01	0.01\\
174.01	0.01\\
175.01	0.01\\
176.01	0.01\\
177.01	0.01\\
178.01	0.01\\
179.01	0.01\\
180.01	0.01\\
181.01	0.01\\
182.01	0.01\\
183.01	0.01\\
184.01	0.01\\
185.01	0.01\\
186.01	0.01\\
187.01	0.01\\
188.01	0.01\\
189.01	0.01\\
190.01	0.01\\
191.01	0.01\\
192.01	0.01\\
193.01	0.01\\
194.01	0.01\\
195.01	0.01\\
196.01	0.01\\
197.01	0.01\\
198.01	0.01\\
199.01	0.01\\
200.01	0.01\\
201.01	0.01\\
202.01	0.01\\
203.01	0.01\\
204.01	0.01\\
205.01	0.01\\
206.01	0.01\\
207.01	0.01\\
208.01	0.01\\
209.01	0.01\\
210.01	0.01\\
211.01	0.01\\
212.01	0.01\\
213.01	0.01\\
214.01	0.01\\
215.01	0.01\\
216.01	0.01\\
217.01	0.01\\
218.01	0.01\\
219.01	0.01\\
220.01	0.01\\
221.01	0.01\\
222.01	0.01\\
223.01	0.01\\
224.01	0.01\\
225.01	0.01\\
226.01	0.01\\
227.01	0.01\\
228.01	0.01\\
229.01	0.01\\
230.01	0.01\\
231.01	0.01\\
232.01	0.01\\
233.01	0.01\\
234.01	0.01\\
235.01	0.01\\
236.01	0.01\\
237.01	0.01\\
238.01	0.01\\
239.01	0.01\\
240.01	0.01\\
241.01	0.01\\
242.01	0.01\\
243.01	0.01\\
244.01	0.01\\
245.01	0.01\\
246.01	0.01\\
247.01	0.01\\
248.01	0.01\\
249.01	0.01\\
250.01	0.01\\
251.01	0.01\\
252.01	0.01\\
253.01	0.01\\
254.01	0.01\\
255.01	0.01\\
256.01	0.01\\
257.01	0.01\\
258.01	0.01\\
259.01	0.01\\
260.01	0.01\\
261.01	0.01\\
262.01	0.01\\
263.01	0.01\\
264.01	0.01\\
265.01	0.01\\
266.01	0.01\\
267.01	0.01\\
268.01	0.01\\
269.01	0.01\\
270.01	0.01\\
271.01	0.01\\
272.01	0.01\\
273.01	0.01\\
274.01	0.01\\
275.01	0.01\\
276.01	0.01\\
277.01	0.01\\
278.01	0.01\\
279.01	0.01\\
280.01	0.01\\
281.01	0.01\\
282.01	0.01\\
283.01	0.01\\
284.01	0.01\\
285.01	0.01\\
286.01	0.01\\
287.01	0.01\\
288.01	0.01\\
289.01	0.01\\
290.01	0.01\\
291.01	0.01\\
292.01	0.01\\
293.01	0.01\\
294.01	0.01\\
295.01	0.01\\
296.01	0.01\\
297.01	0.01\\
298.01	0.01\\
299.01	0.01\\
300.01	0.01\\
301.01	0.01\\
302.01	0.01\\
303.01	0.01\\
304.01	0.01\\
305.01	0.01\\
306.01	0.01\\
307.01	0.01\\
308.01	0.01\\
309.01	0.01\\
310.01	0.01\\
311.01	0.01\\
312.01	0.01\\
313.01	0.01\\
314.01	0.01\\
315.01	0.01\\
316.01	0.01\\
317.01	0.01\\
318.01	0.01\\
319.01	0.01\\
320.01	0.01\\
321.01	0.01\\
322.01	0.01\\
323.01	0.01\\
324.01	0.01\\
325.01	0.01\\
326.01	0.01\\
327.01	0.01\\
328.01	0.01\\
329.01	0.01\\
330.01	0.01\\
331.01	0.01\\
332.01	0.01\\
333.01	0.01\\
334.01	0.01\\
335.01	0.01\\
336.01	0.01\\
337.01	0.01\\
338.01	0.01\\
339.01	0.01\\
340.01	0.01\\
341.01	0.01\\
342.01	0.01\\
343.01	0.01\\
344.01	0.01\\
345.01	0.01\\
346.01	0.01\\
347.01	0.01\\
348.01	0.01\\
349.01	0.01\\
350.01	0.01\\
351.01	0.01\\
352.01	0.01\\
353.01	0.01\\
354.01	0.01\\
355.01	0.01\\
356.01	0.01\\
357.01	0.01\\
358.01	0.01\\
359.01	0.01\\
360.01	0.01\\
361.01	0.01\\
362.01	0.01\\
363.01	0.01\\
364.01	0.01\\
365.01	0.01\\
366.01	0.01\\
367.01	0.01\\
368.01	0.01\\
369.01	0.01\\
370.01	0.01\\
371.01	0.01\\
372.01	0.01\\
373.01	0.01\\
374.01	0.01\\
375.01	0.01\\
376.01	0.01\\
377.01	0.01\\
378.01	0.01\\
379.01	0.01\\
380.01	0.01\\
381.01	0.01\\
382.01	0.01\\
383.01	0.01\\
384.01	0.01\\
385.01	0.01\\
386.01	0.01\\
387.01	0.01\\
388.01	0.01\\
389.01	0.01\\
390.01	0.01\\
391.01	0.01\\
392.01	0.01\\
393.01	0.01\\
394.01	0.01\\
395.01	0.01\\
396.01	0.01\\
397.01	0.01\\
398.01	0.01\\
399.01	0.01\\
400.01	0.01\\
401.01	0.01\\
402.01	0.01\\
403.01	0.01\\
404.01	0.01\\
405.01	0.01\\
406.01	0.01\\
407.01	0.01\\
408.01	0.01\\
409.01	0.01\\
410.01	0.01\\
411.01	0.01\\
412.01	0.01\\
413.01	0.01\\
414.01	0.01\\
415.01	0.01\\
416.01	0.01\\
417.01	0.01\\
418.01	0.01\\
419.01	0.01\\
420.01	0.01\\
421.01	0.01\\
422.01	0.01\\
423.01	0.01\\
424.01	0.01\\
425.01	0.01\\
426.01	0.01\\
427.01	0.01\\
428.01	0.01\\
429.01	0.01\\
430.01	0.01\\
431.01	0.01\\
432.01	0.01\\
433.01	0.01\\
434.01	0.01\\
435.01	0.01\\
436.01	0.01\\
437.01	0.01\\
438.01	0.01\\
439.01	0.01\\
440.01	0.01\\
441.01	0.01\\
442.01	0.01\\
443.01	0.01\\
444.01	0.01\\
445.01	0.01\\
446.01	0.01\\
447.01	0.01\\
448.01	0.01\\
449.01	0.01\\
450.01	0.01\\
451.01	0.01\\
452.01	0.01\\
453.01	0.01\\
454.01	0.01\\
455.01	0.01\\
456.01	0.01\\
457.01	0.01\\
458.01	0.01\\
459.01	0.01\\
460.01	0.01\\
461.01	0.01\\
462.01	0.01\\
463.01	0.01\\
464.01	0.01\\
465.01	0.01\\
466.01	0.01\\
467.01	0.01\\
468.01	0.01\\
469.01	0.01\\
470.01	0.01\\
471.01	0.01\\
472.01	0.01\\
473.01	0.01\\
474.01	0.01\\
475.01	0.01\\
476.01	0.01\\
477.01	0.01\\
478.01	0.01\\
479.01	0.01\\
480.01	0.01\\
481.01	0.01\\
482.01	0.01\\
483.01	0.01\\
484.01	0.01\\
485.01	0.01\\
486.01	0.01\\
487.01	0.01\\
488.01	0.01\\
489.01	0.01\\
490.01	0.01\\
491.01	0.01\\
492.01	0.01\\
493.01	0.01\\
494.01	0.01\\
495.01	0.01\\
496.01	0.01\\
497.01	0.01\\
498.01	0.01\\
499.01	0.01\\
500.01	0.01\\
501.01	0.01\\
502.01	0.01\\
503.01	0.01\\
504.01	0.01\\
505.01	0.01\\
506.01	0.01\\
507.01	0.01\\
508.01	0.01\\
509.01	0.01\\
510.01	0.01\\
511.01	0.01\\
512.01	0.01\\
513.01	0.01\\
514.01	0.01\\
515.01	0.01\\
516.01	0.01\\
517.01	0.01\\
518.01	0.01\\
519.01	0.01\\
520.01	0.01\\
521.01	0.01\\
522.01	0.01\\
523.01	0.01\\
524.01	0.01\\
525.01	0.01\\
526.01	0.01\\
527.01	0.01\\
528.01	0.01\\
529.01	0.01\\
530.01	0.01\\
531.01	0.01\\
532.01	0.01\\
533.01	0.01\\
534.01	0.01\\
535.01	0.01\\
536.01	0.01\\
537.01	0.01\\
538.01	0.01\\
539.01	0.01\\
540.01	0.01\\
541.01	0.01\\
542.01	0.01\\
543.01	0.01\\
544.01	0.01\\
545.01	0.01\\
546.01	0.01\\
547.01	0.01\\
548.01	0.01\\
549.01	0.01\\
550.01	0.01\\
551.01	0.01\\
552.01	0.01\\
553.01	0.01\\
554.01	0.01\\
555.01	0.01\\
556.01	0.01\\
557.01	0.01\\
558.01	0.01\\
559.01	0.01\\
560.01	0.01\\
561.01	0.01\\
562.01	0.01\\
563.01	0.01\\
564.01	0.01\\
565.01	0.01\\
566.01	0.01\\
567.01	0.01\\
568.01	0.01\\
569.01	0.01\\
570.01	0.01\\
571.01	0.01\\
572.01	0.01\\
573.01	0.01\\
574.01	0.01\\
575.01	0.01\\
576.01	0.01\\
577.01	0.01\\
578.01	0.01\\
579.01	0.01\\
580.01	0.01\\
581.01	0.01\\
582.01	0.01\\
583.01	0.01\\
584.01	0.01\\
585.01	0.01\\
586.01	0.01\\
587.01	0.01\\
588.01	0.01\\
589.01	0.01\\
590.01	0.01\\
591.01	0.01\\
592.01	0.01\\
593.01	0.01\\
594.01	0.01\\
595.01	0.01\\
596.01	0.01\\
597.01	0.01\\
598.01	0.01\\
599.01	0.00624186909405406\\
599.02	0.00620414441242159\\
599.03	0.00616605275622732\\
599.04	0.00612759052101566\\
599.05	0.0060887540669203\\
599.06	0.00604953971831638\\
599.07	0.00600994376346912\\
599.08	0.00596996245417914\\
599.09	0.00592959200542422\\
599.1	0.00588882859499749\\
599.11	0.00584766836314218\\
599.12	0.00580610741218272\\
599.13	0.00576414180615221\\
599.14	0.00572176757041625\\
599.15	0.00567898069129305\\
599.16	0.00563577711566981\\
599.17	0.00559215275061543\\
599.18	0.00554810346298924\\
599.19	0.00550362507904598\\
599.2	0.005458713384037\\
599.21	0.00541336412180729\\
599.22	0.0053675729943888\\
599.23	0.00532133566158968\\
599.24	0.00527464774057935\\
599.25	0.00522750480546976\\
599.26	0.0051799023868923\\
599.27	0.00513183597157067\\
599.28	0.00508330100188951\\
599.29	0.00503429287545883\\
599.3	0.00498480694467416\\
599.31	0.00493483851627228\\
599.32	0.00488438285088274\\
599.33	0.00483343516257489\\
599.34	0.00478199061840044\\
599.35	0.00473004433793157\\
599.36	0.00467759139279451\\
599.37	0.00462462680619847\\
599.38	0.00457114555246001\\
599.39	0.00451714256049733\\
599.4	0.00446261272376355\\
599.41	0.00440755088572616\\
599.42	0.0043519518393778\\
599.43	0.00429581032674209\\
599.44	0.00423912103837485\\
599.45	0.00418187861286021\\
599.46	0.00412407763630192\\
599.47	0.0040657126418097\\
599.48	0.00400677810898044\\
599.49	0.00394726846337442\\
599.5	0.00388717807598638\\
599.51	0.00382650126271142\\
599.52	0.00376523228380568\\
599.53	0.00370336534334162\\
599.54	0.0036408945886582\\
599.55	0.00357781410980551\\
599.56	0.00351411793898395\\
599.57	0.00344980004997801\\
599.58	0.0033848543575845\\
599.59	0.00331927471703512\\
599.6	0.00325305492341343\\
599.61	0.00318618871106605\\
599.62	0.00311866975300824\\
599.63	0.00305049166032345\\
599.64	0.00298164798155719\\
599.65	0.00291213220210484\\
599.66	0.00284193774359355\\
599.67	0.002771057963258\\
599.68	0.00269948615331015\\
599.69	0.00262721554030276\\
599.7	0.00255423928448666\\
599.71	0.00248055047916181\\
599.72	0.00240614215002192\\
599.73	0.00233100725449273\\
599.74	0.00225513868106373\\
599.75	0.00217852924861344\\
599.76	0.00210117170572799\\
599.77	0.0020230587300131\\
599.78	0.00194418292739928\\
599.79	0.00186453683144024\\
599.8	0.00178411290260446\\
599.81	0.00170290352755979\\
599.82	0.00162090101845109\\
599.83	0.00153809761217073\\
599.84	0.001454485469622\\
599.85	0.00137005667497534\\
599.86	0.00128480323491721\\
599.87	0.00119871707789166\\
599.88	0.00111179005333449\\
599.89	0.00102401393089983\\
599.9	0.000935380399679279\\
599.91	0.000845881067413288\\
599.92	0.000755507459694884\\
599.93	0.000664251019165563\\
599.94	0.00057210310470336\\
599.95	0.00047905499060291\\
599.96	0.000385097865747556\\
599.97	0.000290222832773275\\
599.98	0.000194420907224487\\
599.99	9.76830167015649e-05\\
600	0\\
};
\addplot [color=mycolor21,solid,forget plot]
  table[row sep=crcr]{%
0.01	0.01\\
1.01	0.01\\
2.01	0.01\\
3.01	0.01\\
4.01	0.01\\
5.01	0.01\\
6.01	0.01\\
7.01	0.01\\
8.01	0.01\\
9.01	0.01\\
10.01	0.01\\
11.01	0.01\\
12.01	0.01\\
13.01	0.01\\
14.01	0.01\\
15.01	0.01\\
16.01	0.01\\
17.01	0.01\\
18.01	0.01\\
19.01	0.01\\
20.01	0.01\\
21.01	0.01\\
22.01	0.01\\
23.01	0.01\\
24.01	0.01\\
25.01	0.01\\
26.01	0.01\\
27.01	0.01\\
28.01	0.01\\
29.01	0.01\\
30.01	0.01\\
31.01	0.01\\
32.01	0.01\\
33.01	0.01\\
34.01	0.01\\
35.01	0.01\\
36.01	0.01\\
37.01	0.01\\
38.01	0.01\\
39.01	0.01\\
40.01	0.01\\
41.01	0.01\\
42.01	0.01\\
43.01	0.01\\
44.01	0.01\\
45.01	0.01\\
46.01	0.01\\
47.01	0.01\\
48.01	0.01\\
49.01	0.01\\
50.01	0.01\\
51.01	0.01\\
52.01	0.01\\
53.01	0.01\\
54.01	0.01\\
55.01	0.01\\
56.01	0.01\\
57.01	0.01\\
58.01	0.01\\
59.01	0.01\\
60.01	0.01\\
61.01	0.01\\
62.01	0.01\\
63.01	0.01\\
64.01	0.01\\
65.01	0.01\\
66.01	0.01\\
67.01	0.01\\
68.01	0.01\\
69.01	0.01\\
70.01	0.01\\
71.01	0.01\\
72.01	0.01\\
73.01	0.01\\
74.01	0.01\\
75.01	0.01\\
76.01	0.01\\
77.01	0.01\\
78.01	0.01\\
79.01	0.01\\
80.01	0.01\\
81.01	0.01\\
82.01	0.01\\
83.01	0.01\\
84.01	0.01\\
85.01	0.01\\
86.01	0.01\\
87.01	0.01\\
88.01	0.01\\
89.01	0.01\\
90.01	0.01\\
91.01	0.01\\
92.01	0.01\\
93.01	0.01\\
94.01	0.01\\
95.01	0.01\\
96.01	0.01\\
97.01	0.01\\
98.01	0.01\\
99.01	0.01\\
100.01	0.01\\
101.01	0.01\\
102.01	0.01\\
103.01	0.01\\
104.01	0.01\\
105.01	0.01\\
106.01	0.01\\
107.01	0.01\\
108.01	0.01\\
109.01	0.01\\
110.01	0.01\\
111.01	0.01\\
112.01	0.01\\
113.01	0.01\\
114.01	0.01\\
115.01	0.01\\
116.01	0.01\\
117.01	0.01\\
118.01	0.01\\
119.01	0.01\\
120.01	0.01\\
121.01	0.01\\
122.01	0.01\\
123.01	0.01\\
124.01	0.01\\
125.01	0.01\\
126.01	0.01\\
127.01	0.01\\
128.01	0.01\\
129.01	0.01\\
130.01	0.01\\
131.01	0.01\\
132.01	0.01\\
133.01	0.01\\
134.01	0.01\\
135.01	0.01\\
136.01	0.01\\
137.01	0.01\\
138.01	0.01\\
139.01	0.01\\
140.01	0.01\\
141.01	0.01\\
142.01	0.01\\
143.01	0.01\\
144.01	0.01\\
145.01	0.01\\
146.01	0.01\\
147.01	0.01\\
148.01	0.01\\
149.01	0.01\\
150.01	0.01\\
151.01	0.01\\
152.01	0.01\\
153.01	0.01\\
154.01	0.01\\
155.01	0.01\\
156.01	0.01\\
157.01	0.01\\
158.01	0.01\\
159.01	0.01\\
160.01	0.01\\
161.01	0.01\\
162.01	0.01\\
163.01	0.01\\
164.01	0.01\\
165.01	0.01\\
166.01	0.01\\
167.01	0.01\\
168.01	0.01\\
169.01	0.01\\
170.01	0.01\\
171.01	0.01\\
172.01	0.01\\
173.01	0.01\\
174.01	0.01\\
175.01	0.01\\
176.01	0.01\\
177.01	0.01\\
178.01	0.01\\
179.01	0.01\\
180.01	0.01\\
181.01	0.01\\
182.01	0.01\\
183.01	0.01\\
184.01	0.01\\
185.01	0.01\\
186.01	0.01\\
187.01	0.01\\
188.01	0.01\\
189.01	0.01\\
190.01	0.01\\
191.01	0.01\\
192.01	0.01\\
193.01	0.01\\
194.01	0.01\\
195.01	0.01\\
196.01	0.01\\
197.01	0.01\\
198.01	0.01\\
199.01	0.01\\
200.01	0.01\\
201.01	0.01\\
202.01	0.01\\
203.01	0.01\\
204.01	0.01\\
205.01	0.01\\
206.01	0.01\\
207.01	0.01\\
208.01	0.01\\
209.01	0.01\\
210.01	0.01\\
211.01	0.01\\
212.01	0.01\\
213.01	0.01\\
214.01	0.01\\
215.01	0.01\\
216.01	0.01\\
217.01	0.01\\
218.01	0.01\\
219.01	0.01\\
220.01	0.01\\
221.01	0.01\\
222.01	0.01\\
223.01	0.01\\
224.01	0.01\\
225.01	0.01\\
226.01	0.01\\
227.01	0.01\\
228.01	0.01\\
229.01	0.01\\
230.01	0.01\\
231.01	0.01\\
232.01	0.01\\
233.01	0.01\\
234.01	0.01\\
235.01	0.01\\
236.01	0.01\\
237.01	0.01\\
238.01	0.01\\
239.01	0.01\\
240.01	0.01\\
241.01	0.01\\
242.01	0.01\\
243.01	0.01\\
244.01	0.01\\
245.01	0.01\\
246.01	0.01\\
247.01	0.01\\
248.01	0.01\\
249.01	0.01\\
250.01	0.01\\
251.01	0.01\\
252.01	0.01\\
253.01	0.01\\
254.01	0.01\\
255.01	0.01\\
256.01	0.01\\
257.01	0.01\\
258.01	0.01\\
259.01	0.01\\
260.01	0.01\\
261.01	0.01\\
262.01	0.01\\
263.01	0.01\\
264.01	0.01\\
265.01	0.01\\
266.01	0.01\\
267.01	0.01\\
268.01	0.01\\
269.01	0.01\\
270.01	0.01\\
271.01	0.01\\
272.01	0.01\\
273.01	0.01\\
274.01	0.01\\
275.01	0.01\\
276.01	0.01\\
277.01	0.01\\
278.01	0.01\\
279.01	0.01\\
280.01	0.01\\
281.01	0.01\\
282.01	0.01\\
283.01	0.01\\
284.01	0.01\\
285.01	0.01\\
286.01	0.01\\
287.01	0.01\\
288.01	0.01\\
289.01	0.01\\
290.01	0.01\\
291.01	0.01\\
292.01	0.01\\
293.01	0.01\\
294.01	0.01\\
295.01	0.01\\
296.01	0.01\\
297.01	0.01\\
298.01	0.01\\
299.01	0.01\\
300.01	0.01\\
301.01	0.01\\
302.01	0.01\\
303.01	0.01\\
304.01	0.01\\
305.01	0.01\\
306.01	0.01\\
307.01	0.01\\
308.01	0.01\\
309.01	0.01\\
310.01	0.01\\
311.01	0.01\\
312.01	0.01\\
313.01	0.01\\
314.01	0.01\\
315.01	0.01\\
316.01	0.01\\
317.01	0.01\\
318.01	0.01\\
319.01	0.01\\
320.01	0.01\\
321.01	0.01\\
322.01	0.01\\
323.01	0.01\\
324.01	0.01\\
325.01	0.01\\
326.01	0.01\\
327.01	0.01\\
328.01	0.01\\
329.01	0.01\\
330.01	0.01\\
331.01	0.01\\
332.01	0.01\\
333.01	0.01\\
334.01	0.01\\
335.01	0.01\\
336.01	0.01\\
337.01	0.01\\
338.01	0.01\\
339.01	0.01\\
340.01	0.01\\
341.01	0.01\\
342.01	0.01\\
343.01	0.01\\
344.01	0.01\\
345.01	0.01\\
346.01	0.01\\
347.01	0.01\\
348.01	0.01\\
349.01	0.01\\
350.01	0.01\\
351.01	0.01\\
352.01	0.01\\
353.01	0.01\\
354.01	0.01\\
355.01	0.01\\
356.01	0.01\\
357.01	0.01\\
358.01	0.01\\
359.01	0.01\\
360.01	0.01\\
361.01	0.01\\
362.01	0.01\\
363.01	0.01\\
364.01	0.01\\
365.01	0.01\\
366.01	0.01\\
367.01	0.01\\
368.01	0.01\\
369.01	0.01\\
370.01	0.01\\
371.01	0.01\\
372.01	0.01\\
373.01	0.01\\
374.01	0.01\\
375.01	0.01\\
376.01	0.01\\
377.01	0.01\\
378.01	0.01\\
379.01	0.01\\
380.01	0.01\\
381.01	0.01\\
382.01	0.01\\
383.01	0.01\\
384.01	0.01\\
385.01	0.01\\
386.01	0.01\\
387.01	0.01\\
388.01	0.01\\
389.01	0.01\\
390.01	0.01\\
391.01	0.01\\
392.01	0.01\\
393.01	0.01\\
394.01	0.01\\
395.01	0.01\\
396.01	0.01\\
397.01	0.01\\
398.01	0.01\\
399.01	0.01\\
400.01	0.01\\
401.01	0.01\\
402.01	0.01\\
403.01	0.01\\
404.01	0.01\\
405.01	0.01\\
406.01	0.01\\
407.01	0.01\\
408.01	0.01\\
409.01	0.01\\
410.01	0.01\\
411.01	0.01\\
412.01	0.01\\
413.01	0.01\\
414.01	0.01\\
415.01	0.01\\
416.01	0.01\\
417.01	0.01\\
418.01	0.01\\
419.01	0.01\\
420.01	0.01\\
421.01	0.01\\
422.01	0.01\\
423.01	0.01\\
424.01	0.01\\
425.01	0.01\\
426.01	0.01\\
427.01	0.01\\
428.01	0.01\\
429.01	0.01\\
430.01	0.01\\
431.01	0.01\\
432.01	0.01\\
433.01	0.01\\
434.01	0.01\\
435.01	0.01\\
436.01	0.01\\
437.01	0.01\\
438.01	0.01\\
439.01	0.01\\
440.01	0.01\\
441.01	0.01\\
442.01	0.01\\
443.01	0.01\\
444.01	0.01\\
445.01	0.01\\
446.01	0.01\\
447.01	0.01\\
448.01	0.01\\
449.01	0.01\\
450.01	0.01\\
451.01	0.01\\
452.01	0.01\\
453.01	0.01\\
454.01	0.01\\
455.01	0.01\\
456.01	0.01\\
457.01	0.01\\
458.01	0.01\\
459.01	0.01\\
460.01	0.01\\
461.01	0.01\\
462.01	0.01\\
463.01	0.01\\
464.01	0.01\\
465.01	0.01\\
466.01	0.01\\
467.01	0.01\\
468.01	0.01\\
469.01	0.01\\
470.01	0.01\\
471.01	0.01\\
472.01	0.01\\
473.01	0.01\\
474.01	0.01\\
475.01	0.01\\
476.01	0.01\\
477.01	0.01\\
478.01	0.01\\
479.01	0.01\\
480.01	0.01\\
481.01	0.01\\
482.01	0.01\\
483.01	0.01\\
484.01	0.01\\
485.01	0.01\\
486.01	0.01\\
487.01	0.01\\
488.01	0.01\\
489.01	0.01\\
490.01	0.01\\
491.01	0.01\\
492.01	0.01\\
493.01	0.01\\
494.01	0.01\\
495.01	0.01\\
496.01	0.01\\
497.01	0.01\\
498.01	0.01\\
499.01	0.01\\
500.01	0.01\\
501.01	0.01\\
502.01	0.01\\
503.01	0.01\\
504.01	0.01\\
505.01	0.01\\
506.01	0.01\\
507.01	0.01\\
508.01	0.01\\
509.01	0.01\\
510.01	0.01\\
511.01	0.01\\
512.01	0.01\\
513.01	0.01\\
514.01	0.01\\
515.01	0.01\\
516.01	0.01\\
517.01	0.01\\
518.01	0.01\\
519.01	0.01\\
520.01	0.01\\
521.01	0.01\\
522.01	0.01\\
523.01	0.01\\
524.01	0.01\\
525.01	0.01\\
526.01	0.01\\
527.01	0.01\\
528.01	0.01\\
529.01	0.01\\
530.01	0.01\\
531.01	0.01\\
532.01	0.01\\
533.01	0.01\\
534.01	0.01\\
535.01	0.01\\
536.01	0.01\\
537.01	0.01\\
538.01	0.01\\
539.01	0.01\\
540.01	0.01\\
541.01	0.01\\
542.01	0.01\\
543.01	0.01\\
544.01	0.01\\
545.01	0.01\\
546.01	0.01\\
547.01	0.01\\
548.01	0.01\\
549.01	0.01\\
550.01	0.01\\
551.01	0.01\\
552.01	0.01\\
553.01	0.01\\
554.01	0.01\\
555.01	0.01\\
556.01	0.01\\
557.01	0.01\\
558.01	0.01\\
559.01	0.01\\
560.01	0.01\\
561.01	0.01\\
562.01	0.01\\
563.01	0.01\\
564.01	0.01\\
565.01	0.01\\
566.01	0.01\\
567.01	0.01\\
568.01	0.01\\
569.01	0.01\\
570.01	0.01\\
571.01	0.01\\
572.01	0.01\\
573.01	0.01\\
574.01	0.01\\
575.01	0.01\\
576.01	0.01\\
577.01	0.01\\
578.01	0.01\\
579.01	0.01\\
580.01	0.01\\
581.01	0.01\\
582.01	0.01\\
583.01	0.01\\
584.01	0.01\\
585.01	0.01\\
586.01	0.01\\
587.01	0.01\\
588.01	0.01\\
589.01	0.01\\
590.01	0.01\\
591.01	0.01\\
592.01	0.01\\
593.01	0.01\\
594.01	0.01\\
595.01	0.01\\
596.01	0.01\\
597.01	0.01\\
598.01	0.01\\
599.01	0.00624186909405409\\
599.02	0.00620414441242163\\
599.03	0.00616605275622737\\
599.04	0.00612759052101571\\
599.05	0.00608875406692037\\
599.06	0.00604953971831644\\
599.07	0.00600994376346918\\
599.08	0.0059699624541792\\
599.09	0.00592959200542426\\
599.1	0.00588882859499753\\
599.11	0.00584766836314222\\
599.12	0.00580610741218277\\
599.13	0.00576414180615227\\
599.14	0.00572176757041629\\
599.15	0.00567898069129309\\
599.16	0.00563577711566987\\
599.17	0.00559215275061548\\
599.18	0.00554810346298928\\
599.19	0.00550362507904603\\
599.2	0.00545871338403702\\
599.21	0.0054133641218073\\
599.22	0.00536757299438882\\
599.23	0.00532133566158968\\
599.24	0.00527464774057935\\
599.25	0.00522750480546976\\
599.26	0.0051799023868923\\
599.27	0.00513183597157068\\
599.28	0.00508330100188952\\
599.29	0.00503429287545886\\
599.3	0.00498480694467418\\
599.31	0.00493483851627229\\
599.32	0.00488438285088277\\
599.33	0.00483343516257492\\
599.34	0.00478199061840048\\
599.35	0.00473004433793162\\
599.36	0.00467759139279456\\
599.37	0.00462462680619851\\
599.38	0.00457114555246004\\
599.39	0.00451714256049737\\
599.4	0.00446261272376359\\
599.41	0.00440755088572621\\
599.42	0.00435195183937783\\
599.43	0.00429581032674213\\
599.44	0.00423912103837488\\
599.45	0.00418187861286022\\
599.46	0.00412407763630194\\
599.47	0.00406571264180972\\
599.48	0.00400677810898045\\
599.49	0.00394726846337443\\
599.5	0.00388717807598639\\
599.51	0.00382650126271144\\
599.52	0.00376523228380568\\
599.53	0.00370336534334162\\
599.54	0.00364089458865822\\
599.55	0.00357781410980552\\
599.56	0.00351411793898394\\
599.57	0.00344980004997801\\
599.58	0.00338485435758449\\
599.59	0.00331927471703512\\
599.6	0.00325305492341342\\
599.61	0.00318618871106605\\
599.62	0.00311866975300824\\
599.63	0.00305049166032345\\
599.64	0.00298164798155718\\
599.65	0.00291213220210483\\
599.66	0.00284193774359354\\
599.67	0.002771057963258\\
599.68	0.00269948615331015\\
599.69	0.00262721554030276\\
599.7	0.00255423928448666\\
599.71	0.00248055047916181\\
599.72	0.00240614215002193\\
599.73	0.00233100725449273\\
599.74	0.00225513868106374\\
599.75	0.00217852924861345\\
599.76	0.002101171705728\\
599.77	0.00202305873001311\\
599.78	0.00194418292739928\\
599.79	0.00186453683144024\\
599.8	0.00178411290260446\\
599.81	0.0017029035275598\\
599.82	0.00162090101845109\\
599.83	0.00153809761217073\\
599.84	0.001454485469622\\
599.85	0.00137005667497534\\
599.86	0.00128480323491721\\
599.87	0.00119871707789167\\
599.88	0.00111179005333449\\
599.89	0.00102401393089982\\
599.9	0.000935380399679274\\
599.91	0.000845881067413286\\
599.92	0.000755507459694882\\
599.93	0.000664251019165561\\
599.94	0.000572103104703356\\
599.95	0.00047905499060291\\
599.96	0.000385097865747558\\
599.97	0.000290222832773275\\
599.98	0.000194420907224489\\
599.99	9.76830167015649e-05\\
600	0\\
};
\addplot [color=black!20!mycolor21,solid,forget plot]
  table[row sep=crcr]{%
0.01	0.01\\
1.01	0.01\\
2.01	0.01\\
3.01	0.01\\
4.01	0.01\\
5.01	0.01\\
6.01	0.01\\
7.01	0.01\\
8.01	0.01\\
9.01	0.01\\
10.01	0.01\\
11.01	0.01\\
12.01	0.01\\
13.01	0.01\\
14.01	0.01\\
15.01	0.01\\
16.01	0.01\\
17.01	0.01\\
18.01	0.01\\
19.01	0.01\\
20.01	0.01\\
21.01	0.01\\
22.01	0.01\\
23.01	0.01\\
24.01	0.01\\
25.01	0.01\\
26.01	0.01\\
27.01	0.01\\
28.01	0.01\\
29.01	0.01\\
30.01	0.01\\
31.01	0.01\\
32.01	0.01\\
33.01	0.01\\
34.01	0.01\\
35.01	0.01\\
36.01	0.01\\
37.01	0.01\\
38.01	0.01\\
39.01	0.01\\
40.01	0.01\\
41.01	0.01\\
42.01	0.01\\
43.01	0.01\\
44.01	0.01\\
45.01	0.01\\
46.01	0.01\\
47.01	0.01\\
48.01	0.01\\
49.01	0.01\\
50.01	0.01\\
51.01	0.01\\
52.01	0.01\\
53.01	0.01\\
54.01	0.01\\
55.01	0.01\\
56.01	0.01\\
57.01	0.01\\
58.01	0.01\\
59.01	0.01\\
60.01	0.01\\
61.01	0.01\\
62.01	0.01\\
63.01	0.01\\
64.01	0.01\\
65.01	0.01\\
66.01	0.01\\
67.01	0.01\\
68.01	0.01\\
69.01	0.01\\
70.01	0.01\\
71.01	0.01\\
72.01	0.01\\
73.01	0.01\\
74.01	0.01\\
75.01	0.01\\
76.01	0.01\\
77.01	0.01\\
78.01	0.01\\
79.01	0.01\\
80.01	0.01\\
81.01	0.01\\
82.01	0.01\\
83.01	0.01\\
84.01	0.01\\
85.01	0.01\\
86.01	0.01\\
87.01	0.01\\
88.01	0.01\\
89.01	0.01\\
90.01	0.01\\
91.01	0.01\\
92.01	0.01\\
93.01	0.01\\
94.01	0.01\\
95.01	0.01\\
96.01	0.01\\
97.01	0.01\\
98.01	0.01\\
99.01	0.01\\
100.01	0.01\\
101.01	0.01\\
102.01	0.01\\
103.01	0.01\\
104.01	0.01\\
105.01	0.01\\
106.01	0.01\\
107.01	0.01\\
108.01	0.01\\
109.01	0.01\\
110.01	0.01\\
111.01	0.01\\
112.01	0.01\\
113.01	0.01\\
114.01	0.01\\
115.01	0.01\\
116.01	0.01\\
117.01	0.01\\
118.01	0.01\\
119.01	0.01\\
120.01	0.01\\
121.01	0.01\\
122.01	0.01\\
123.01	0.01\\
124.01	0.01\\
125.01	0.01\\
126.01	0.01\\
127.01	0.01\\
128.01	0.01\\
129.01	0.01\\
130.01	0.01\\
131.01	0.01\\
132.01	0.01\\
133.01	0.01\\
134.01	0.01\\
135.01	0.01\\
136.01	0.01\\
137.01	0.01\\
138.01	0.01\\
139.01	0.01\\
140.01	0.01\\
141.01	0.01\\
142.01	0.01\\
143.01	0.01\\
144.01	0.01\\
145.01	0.01\\
146.01	0.01\\
147.01	0.01\\
148.01	0.01\\
149.01	0.01\\
150.01	0.01\\
151.01	0.01\\
152.01	0.01\\
153.01	0.01\\
154.01	0.01\\
155.01	0.01\\
156.01	0.01\\
157.01	0.01\\
158.01	0.01\\
159.01	0.01\\
160.01	0.01\\
161.01	0.01\\
162.01	0.01\\
163.01	0.01\\
164.01	0.01\\
165.01	0.01\\
166.01	0.01\\
167.01	0.01\\
168.01	0.01\\
169.01	0.01\\
170.01	0.01\\
171.01	0.01\\
172.01	0.01\\
173.01	0.01\\
174.01	0.01\\
175.01	0.01\\
176.01	0.01\\
177.01	0.01\\
178.01	0.01\\
179.01	0.01\\
180.01	0.01\\
181.01	0.01\\
182.01	0.01\\
183.01	0.01\\
184.01	0.01\\
185.01	0.01\\
186.01	0.01\\
187.01	0.01\\
188.01	0.01\\
189.01	0.01\\
190.01	0.01\\
191.01	0.01\\
192.01	0.01\\
193.01	0.01\\
194.01	0.01\\
195.01	0.01\\
196.01	0.01\\
197.01	0.01\\
198.01	0.01\\
199.01	0.01\\
200.01	0.01\\
201.01	0.01\\
202.01	0.01\\
203.01	0.01\\
204.01	0.01\\
205.01	0.01\\
206.01	0.01\\
207.01	0.01\\
208.01	0.01\\
209.01	0.01\\
210.01	0.01\\
211.01	0.01\\
212.01	0.01\\
213.01	0.01\\
214.01	0.01\\
215.01	0.01\\
216.01	0.01\\
217.01	0.01\\
218.01	0.01\\
219.01	0.01\\
220.01	0.01\\
221.01	0.01\\
222.01	0.01\\
223.01	0.01\\
224.01	0.01\\
225.01	0.01\\
226.01	0.01\\
227.01	0.01\\
228.01	0.01\\
229.01	0.01\\
230.01	0.01\\
231.01	0.01\\
232.01	0.01\\
233.01	0.01\\
234.01	0.01\\
235.01	0.01\\
236.01	0.01\\
237.01	0.01\\
238.01	0.01\\
239.01	0.01\\
240.01	0.01\\
241.01	0.01\\
242.01	0.01\\
243.01	0.01\\
244.01	0.01\\
245.01	0.01\\
246.01	0.01\\
247.01	0.01\\
248.01	0.01\\
249.01	0.01\\
250.01	0.01\\
251.01	0.01\\
252.01	0.01\\
253.01	0.01\\
254.01	0.01\\
255.01	0.01\\
256.01	0.01\\
257.01	0.01\\
258.01	0.01\\
259.01	0.01\\
260.01	0.01\\
261.01	0.01\\
262.01	0.01\\
263.01	0.01\\
264.01	0.01\\
265.01	0.01\\
266.01	0.01\\
267.01	0.01\\
268.01	0.01\\
269.01	0.01\\
270.01	0.01\\
271.01	0.01\\
272.01	0.01\\
273.01	0.01\\
274.01	0.01\\
275.01	0.01\\
276.01	0.01\\
277.01	0.01\\
278.01	0.01\\
279.01	0.01\\
280.01	0.01\\
281.01	0.01\\
282.01	0.01\\
283.01	0.01\\
284.01	0.01\\
285.01	0.01\\
286.01	0.01\\
287.01	0.01\\
288.01	0.01\\
289.01	0.01\\
290.01	0.01\\
291.01	0.01\\
292.01	0.01\\
293.01	0.01\\
294.01	0.01\\
295.01	0.01\\
296.01	0.01\\
297.01	0.01\\
298.01	0.01\\
299.01	0.01\\
300.01	0.01\\
301.01	0.01\\
302.01	0.01\\
303.01	0.01\\
304.01	0.01\\
305.01	0.01\\
306.01	0.01\\
307.01	0.01\\
308.01	0.01\\
309.01	0.01\\
310.01	0.01\\
311.01	0.01\\
312.01	0.01\\
313.01	0.01\\
314.01	0.01\\
315.01	0.01\\
316.01	0.01\\
317.01	0.01\\
318.01	0.01\\
319.01	0.01\\
320.01	0.01\\
321.01	0.01\\
322.01	0.01\\
323.01	0.01\\
324.01	0.01\\
325.01	0.01\\
326.01	0.01\\
327.01	0.01\\
328.01	0.01\\
329.01	0.01\\
330.01	0.01\\
331.01	0.01\\
332.01	0.01\\
333.01	0.01\\
334.01	0.01\\
335.01	0.01\\
336.01	0.01\\
337.01	0.01\\
338.01	0.01\\
339.01	0.01\\
340.01	0.01\\
341.01	0.01\\
342.01	0.01\\
343.01	0.01\\
344.01	0.01\\
345.01	0.01\\
346.01	0.01\\
347.01	0.01\\
348.01	0.01\\
349.01	0.01\\
350.01	0.01\\
351.01	0.01\\
352.01	0.01\\
353.01	0.01\\
354.01	0.01\\
355.01	0.01\\
356.01	0.01\\
357.01	0.01\\
358.01	0.01\\
359.01	0.01\\
360.01	0.01\\
361.01	0.01\\
362.01	0.01\\
363.01	0.01\\
364.01	0.01\\
365.01	0.01\\
366.01	0.01\\
367.01	0.01\\
368.01	0.01\\
369.01	0.01\\
370.01	0.01\\
371.01	0.01\\
372.01	0.01\\
373.01	0.01\\
374.01	0.01\\
375.01	0.01\\
376.01	0.01\\
377.01	0.01\\
378.01	0.01\\
379.01	0.01\\
380.01	0.01\\
381.01	0.01\\
382.01	0.01\\
383.01	0.01\\
384.01	0.01\\
385.01	0.01\\
386.01	0.01\\
387.01	0.01\\
388.01	0.01\\
389.01	0.01\\
390.01	0.01\\
391.01	0.01\\
392.01	0.01\\
393.01	0.01\\
394.01	0.01\\
395.01	0.01\\
396.01	0.01\\
397.01	0.01\\
398.01	0.01\\
399.01	0.01\\
400.01	0.01\\
401.01	0.01\\
402.01	0.01\\
403.01	0.01\\
404.01	0.01\\
405.01	0.01\\
406.01	0.01\\
407.01	0.01\\
408.01	0.01\\
409.01	0.01\\
410.01	0.01\\
411.01	0.01\\
412.01	0.01\\
413.01	0.01\\
414.01	0.01\\
415.01	0.01\\
416.01	0.01\\
417.01	0.01\\
418.01	0.01\\
419.01	0.01\\
420.01	0.01\\
421.01	0.01\\
422.01	0.01\\
423.01	0.01\\
424.01	0.01\\
425.01	0.01\\
426.01	0.01\\
427.01	0.01\\
428.01	0.01\\
429.01	0.01\\
430.01	0.01\\
431.01	0.01\\
432.01	0.01\\
433.01	0.01\\
434.01	0.01\\
435.01	0.01\\
436.01	0.01\\
437.01	0.01\\
438.01	0.01\\
439.01	0.01\\
440.01	0.01\\
441.01	0.01\\
442.01	0.01\\
443.01	0.01\\
444.01	0.01\\
445.01	0.01\\
446.01	0.01\\
447.01	0.01\\
448.01	0.01\\
449.01	0.01\\
450.01	0.01\\
451.01	0.01\\
452.01	0.01\\
453.01	0.01\\
454.01	0.01\\
455.01	0.01\\
456.01	0.01\\
457.01	0.01\\
458.01	0.01\\
459.01	0.01\\
460.01	0.01\\
461.01	0.01\\
462.01	0.01\\
463.01	0.01\\
464.01	0.01\\
465.01	0.01\\
466.01	0.01\\
467.01	0.01\\
468.01	0.01\\
469.01	0.01\\
470.01	0.01\\
471.01	0.01\\
472.01	0.01\\
473.01	0.01\\
474.01	0.01\\
475.01	0.01\\
476.01	0.01\\
477.01	0.01\\
478.01	0.01\\
479.01	0.01\\
480.01	0.01\\
481.01	0.01\\
482.01	0.01\\
483.01	0.01\\
484.01	0.01\\
485.01	0.01\\
486.01	0.01\\
487.01	0.01\\
488.01	0.01\\
489.01	0.01\\
490.01	0.01\\
491.01	0.01\\
492.01	0.01\\
493.01	0.01\\
494.01	0.01\\
495.01	0.01\\
496.01	0.01\\
497.01	0.01\\
498.01	0.01\\
499.01	0.01\\
500.01	0.01\\
501.01	0.01\\
502.01	0.01\\
503.01	0.01\\
504.01	0.01\\
505.01	0.01\\
506.01	0.01\\
507.01	0.01\\
508.01	0.01\\
509.01	0.01\\
510.01	0.01\\
511.01	0.01\\
512.01	0.01\\
513.01	0.01\\
514.01	0.01\\
515.01	0.01\\
516.01	0.01\\
517.01	0.01\\
518.01	0.01\\
519.01	0.01\\
520.01	0.01\\
521.01	0.01\\
522.01	0.01\\
523.01	0.01\\
524.01	0.01\\
525.01	0.01\\
526.01	0.01\\
527.01	0.01\\
528.01	0.01\\
529.01	0.01\\
530.01	0.01\\
531.01	0.01\\
532.01	0.01\\
533.01	0.01\\
534.01	0.01\\
535.01	0.01\\
536.01	0.01\\
537.01	0.01\\
538.01	0.01\\
539.01	0.01\\
540.01	0.01\\
541.01	0.01\\
542.01	0.01\\
543.01	0.01\\
544.01	0.01\\
545.01	0.01\\
546.01	0.01\\
547.01	0.01\\
548.01	0.01\\
549.01	0.01\\
550.01	0.01\\
551.01	0.01\\
552.01	0.01\\
553.01	0.01\\
554.01	0.01\\
555.01	0.01\\
556.01	0.01\\
557.01	0.01\\
558.01	0.01\\
559.01	0.01\\
560.01	0.01\\
561.01	0.01\\
562.01	0.01\\
563.01	0.01\\
564.01	0.01\\
565.01	0.01\\
566.01	0.01\\
567.01	0.01\\
568.01	0.01\\
569.01	0.01\\
570.01	0.01\\
571.01	0.01\\
572.01	0.01\\
573.01	0.01\\
574.01	0.01\\
575.01	0.01\\
576.01	0.01\\
577.01	0.01\\
578.01	0.01\\
579.01	0.01\\
580.01	0.01\\
581.01	0.01\\
582.01	0.01\\
583.01	0.01\\
584.01	0.01\\
585.01	0.01\\
586.01	0.01\\
587.01	0.01\\
588.01	0.01\\
589.01	0.01\\
590.01	0.01\\
591.01	0.01\\
592.01	0.01\\
593.01	0.01\\
594.01	0.01\\
595.01	0.01\\
596.01	0.01\\
597.01	0.01\\
598.01	0.00929975107060225\\
599.01	0.00624186909405412\\
599.02	0.00620414441242164\\
599.03	0.00616605275622737\\
599.04	0.00612759052101571\\
599.05	0.00608875406692035\\
599.06	0.00604953971831644\\
599.07	0.00600994376346918\\
599.08	0.0059699624541792\\
599.09	0.00592959200542426\\
599.1	0.00588882859499752\\
599.11	0.00584766836314222\\
599.12	0.00580610741218276\\
599.13	0.00576414180615225\\
599.14	0.00572176757041629\\
599.15	0.00567898069129307\\
599.16	0.00563577711566984\\
599.17	0.00559215275061547\\
599.18	0.00554810346298928\\
599.19	0.00550362507904604\\
599.2	0.00545871338403705\\
599.21	0.00541336412180733\\
599.22	0.00536757299438884\\
599.23	0.0053213356615897\\
599.24	0.00527464774057938\\
599.25	0.00522750480546979\\
599.26	0.00517990238689233\\
599.27	0.00513183597157071\\
599.28	0.00508330100188955\\
599.29	0.00503429287545888\\
599.3	0.0049848069446742\\
599.31	0.00493483851627232\\
599.32	0.00488438285088278\\
599.33	0.00483343516257495\\
599.34	0.00478199061840049\\
599.35	0.00473004433793162\\
599.36	0.00467759139279456\\
599.37	0.00462462680619851\\
599.38	0.00457114555246004\\
599.39	0.00451714256049737\\
599.4	0.00446261272376359\\
599.41	0.00440755088572621\\
599.42	0.00435195183937783\\
599.43	0.00429581032674213\\
599.44	0.00423912103837487\\
599.45	0.00418187861286023\\
599.46	0.00412407763630194\\
599.47	0.00406571264180971\\
599.48	0.00400677810898045\\
599.49	0.00394726846337442\\
599.5	0.00388717807598638\\
599.51	0.00382650126271144\\
599.52	0.00376523228380569\\
599.53	0.00370336534334163\\
599.54	0.00364089458865821\\
599.55	0.00357781410980551\\
599.56	0.00351411793898394\\
599.57	0.003449800049978\\
599.58	0.00338485435758449\\
599.59	0.00331927471703512\\
599.6	0.00325305492341342\\
599.61	0.00318618871106605\\
599.62	0.00311866975300823\\
599.63	0.00305049166032344\\
599.64	0.00298164798155717\\
599.65	0.00291213220210483\\
599.66	0.00284193774359354\\
599.67	0.00277105796325799\\
599.68	0.00269948615331015\\
599.69	0.00262721554030275\\
599.7	0.00255423928448665\\
599.71	0.0024805504791618\\
599.72	0.00240614215002191\\
599.73	0.00233100725449272\\
599.74	0.00225513868106372\\
599.75	0.00217852924861344\\
599.76	0.00210117170572799\\
599.77	0.0020230587300131\\
599.78	0.00194418292739928\\
599.79	0.00186453683144023\\
599.8	0.00178411290260445\\
599.81	0.00170290352755979\\
599.82	0.00162090101845109\\
599.83	0.00153809761217072\\
599.84	0.001454485469622\\
599.85	0.00137005667497534\\
599.86	0.00128480323491721\\
599.87	0.00119871707789166\\
599.88	0.00111179005333448\\
599.89	0.00102401393089982\\
599.9	0.000935380399679275\\
599.91	0.000845881067413288\\
599.92	0.000755507459694884\\
599.93	0.000664251019165561\\
599.94	0.00057210310470336\\
599.95	0.00047905499060291\\
599.96	0.000385097865747556\\
599.97	0.000290222832773275\\
599.98	0.000194420907224489\\
599.99	9.76830167015649e-05\\
600	0\\
};
\addplot [color=black!50!mycolor20,solid,forget plot]
  table[row sep=crcr]{%
0.01	0.01\\
1.01	0.01\\
2.01	0.01\\
3.01	0.01\\
4.01	0.01\\
5.01	0.01\\
6.01	0.01\\
7.01	0.01\\
8.01	0.01\\
9.01	0.01\\
10.01	0.01\\
11.01	0.01\\
12.01	0.01\\
13.01	0.01\\
14.01	0.01\\
15.01	0.01\\
16.01	0.01\\
17.01	0.01\\
18.01	0.01\\
19.01	0.01\\
20.01	0.01\\
21.01	0.01\\
22.01	0.01\\
23.01	0.01\\
24.01	0.01\\
25.01	0.01\\
26.01	0.01\\
27.01	0.01\\
28.01	0.01\\
29.01	0.01\\
30.01	0.01\\
31.01	0.01\\
32.01	0.01\\
33.01	0.01\\
34.01	0.01\\
35.01	0.01\\
36.01	0.01\\
37.01	0.01\\
38.01	0.01\\
39.01	0.01\\
40.01	0.01\\
41.01	0.01\\
42.01	0.01\\
43.01	0.01\\
44.01	0.01\\
45.01	0.01\\
46.01	0.01\\
47.01	0.01\\
48.01	0.01\\
49.01	0.01\\
50.01	0.01\\
51.01	0.01\\
52.01	0.01\\
53.01	0.01\\
54.01	0.01\\
55.01	0.01\\
56.01	0.01\\
57.01	0.01\\
58.01	0.01\\
59.01	0.01\\
60.01	0.01\\
61.01	0.01\\
62.01	0.01\\
63.01	0.01\\
64.01	0.01\\
65.01	0.01\\
66.01	0.01\\
67.01	0.01\\
68.01	0.01\\
69.01	0.01\\
70.01	0.01\\
71.01	0.01\\
72.01	0.01\\
73.01	0.01\\
74.01	0.01\\
75.01	0.01\\
76.01	0.01\\
77.01	0.01\\
78.01	0.01\\
79.01	0.01\\
80.01	0.01\\
81.01	0.01\\
82.01	0.01\\
83.01	0.01\\
84.01	0.01\\
85.01	0.01\\
86.01	0.01\\
87.01	0.01\\
88.01	0.01\\
89.01	0.01\\
90.01	0.01\\
91.01	0.01\\
92.01	0.01\\
93.01	0.01\\
94.01	0.01\\
95.01	0.01\\
96.01	0.01\\
97.01	0.01\\
98.01	0.01\\
99.01	0.01\\
100.01	0.01\\
101.01	0.01\\
102.01	0.01\\
103.01	0.01\\
104.01	0.01\\
105.01	0.01\\
106.01	0.01\\
107.01	0.01\\
108.01	0.01\\
109.01	0.01\\
110.01	0.01\\
111.01	0.01\\
112.01	0.01\\
113.01	0.01\\
114.01	0.01\\
115.01	0.01\\
116.01	0.01\\
117.01	0.01\\
118.01	0.01\\
119.01	0.01\\
120.01	0.01\\
121.01	0.01\\
122.01	0.01\\
123.01	0.01\\
124.01	0.01\\
125.01	0.01\\
126.01	0.01\\
127.01	0.01\\
128.01	0.01\\
129.01	0.01\\
130.01	0.01\\
131.01	0.01\\
132.01	0.01\\
133.01	0.01\\
134.01	0.01\\
135.01	0.01\\
136.01	0.01\\
137.01	0.01\\
138.01	0.01\\
139.01	0.01\\
140.01	0.01\\
141.01	0.01\\
142.01	0.01\\
143.01	0.01\\
144.01	0.01\\
145.01	0.01\\
146.01	0.01\\
147.01	0.01\\
148.01	0.01\\
149.01	0.01\\
150.01	0.01\\
151.01	0.01\\
152.01	0.01\\
153.01	0.01\\
154.01	0.01\\
155.01	0.01\\
156.01	0.01\\
157.01	0.01\\
158.01	0.01\\
159.01	0.01\\
160.01	0.01\\
161.01	0.01\\
162.01	0.01\\
163.01	0.01\\
164.01	0.01\\
165.01	0.01\\
166.01	0.01\\
167.01	0.01\\
168.01	0.01\\
169.01	0.01\\
170.01	0.01\\
171.01	0.01\\
172.01	0.01\\
173.01	0.01\\
174.01	0.01\\
175.01	0.01\\
176.01	0.01\\
177.01	0.01\\
178.01	0.01\\
179.01	0.01\\
180.01	0.01\\
181.01	0.01\\
182.01	0.01\\
183.01	0.01\\
184.01	0.01\\
185.01	0.01\\
186.01	0.01\\
187.01	0.01\\
188.01	0.01\\
189.01	0.01\\
190.01	0.01\\
191.01	0.01\\
192.01	0.01\\
193.01	0.01\\
194.01	0.01\\
195.01	0.01\\
196.01	0.01\\
197.01	0.01\\
198.01	0.01\\
199.01	0.01\\
200.01	0.01\\
201.01	0.01\\
202.01	0.01\\
203.01	0.01\\
204.01	0.01\\
205.01	0.01\\
206.01	0.01\\
207.01	0.01\\
208.01	0.01\\
209.01	0.01\\
210.01	0.01\\
211.01	0.01\\
212.01	0.01\\
213.01	0.01\\
214.01	0.01\\
215.01	0.01\\
216.01	0.01\\
217.01	0.01\\
218.01	0.01\\
219.01	0.01\\
220.01	0.01\\
221.01	0.01\\
222.01	0.01\\
223.01	0.01\\
224.01	0.01\\
225.01	0.01\\
226.01	0.01\\
227.01	0.01\\
228.01	0.01\\
229.01	0.01\\
230.01	0.01\\
231.01	0.01\\
232.01	0.01\\
233.01	0.01\\
234.01	0.01\\
235.01	0.01\\
236.01	0.01\\
237.01	0.01\\
238.01	0.01\\
239.01	0.01\\
240.01	0.01\\
241.01	0.01\\
242.01	0.01\\
243.01	0.01\\
244.01	0.01\\
245.01	0.01\\
246.01	0.01\\
247.01	0.01\\
248.01	0.01\\
249.01	0.01\\
250.01	0.01\\
251.01	0.01\\
252.01	0.01\\
253.01	0.01\\
254.01	0.01\\
255.01	0.01\\
256.01	0.01\\
257.01	0.01\\
258.01	0.01\\
259.01	0.01\\
260.01	0.01\\
261.01	0.01\\
262.01	0.01\\
263.01	0.01\\
264.01	0.01\\
265.01	0.01\\
266.01	0.01\\
267.01	0.01\\
268.01	0.01\\
269.01	0.01\\
270.01	0.01\\
271.01	0.01\\
272.01	0.01\\
273.01	0.01\\
274.01	0.01\\
275.01	0.01\\
276.01	0.01\\
277.01	0.01\\
278.01	0.01\\
279.01	0.01\\
280.01	0.01\\
281.01	0.01\\
282.01	0.01\\
283.01	0.01\\
284.01	0.01\\
285.01	0.01\\
286.01	0.01\\
287.01	0.01\\
288.01	0.01\\
289.01	0.01\\
290.01	0.01\\
291.01	0.01\\
292.01	0.01\\
293.01	0.01\\
294.01	0.01\\
295.01	0.01\\
296.01	0.01\\
297.01	0.01\\
298.01	0.01\\
299.01	0.01\\
300.01	0.01\\
301.01	0.01\\
302.01	0.01\\
303.01	0.01\\
304.01	0.01\\
305.01	0.01\\
306.01	0.01\\
307.01	0.01\\
308.01	0.01\\
309.01	0.01\\
310.01	0.01\\
311.01	0.01\\
312.01	0.01\\
313.01	0.01\\
314.01	0.01\\
315.01	0.01\\
316.01	0.01\\
317.01	0.01\\
318.01	0.01\\
319.01	0.01\\
320.01	0.01\\
321.01	0.01\\
322.01	0.01\\
323.01	0.01\\
324.01	0.01\\
325.01	0.01\\
326.01	0.01\\
327.01	0.01\\
328.01	0.01\\
329.01	0.01\\
330.01	0.01\\
331.01	0.01\\
332.01	0.01\\
333.01	0.01\\
334.01	0.01\\
335.01	0.01\\
336.01	0.01\\
337.01	0.01\\
338.01	0.01\\
339.01	0.01\\
340.01	0.01\\
341.01	0.01\\
342.01	0.01\\
343.01	0.01\\
344.01	0.01\\
345.01	0.01\\
346.01	0.01\\
347.01	0.01\\
348.01	0.01\\
349.01	0.01\\
350.01	0.01\\
351.01	0.01\\
352.01	0.01\\
353.01	0.01\\
354.01	0.01\\
355.01	0.01\\
356.01	0.01\\
357.01	0.01\\
358.01	0.01\\
359.01	0.01\\
360.01	0.01\\
361.01	0.01\\
362.01	0.01\\
363.01	0.01\\
364.01	0.01\\
365.01	0.01\\
366.01	0.01\\
367.01	0.01\\
368.01	0.01\\
369.01	0.01\\
370.01	0.01\\
371.01	0.01\\
372.01	0.01\\
373.01	0.01\\
374.01	0.01\\
375.01	0.01\\
376.01	0.01\\
377.01	0.01\\
378.01	0.01\\
379.01	0.01\\
380.01	0.01\\
381.01	0.01\\
382.01	0.01\\
383.01	0.01\\
384.01	0.01\\
385.01	0.01\\
386.01	0.01\\
387.01	0.01\\
388.01	0.01\\
389.01	0.01\\
390.01	0.01\\
391.01	0.01\\
392.01	0.01\\
393.01	0.01\\
394.01	0.01\\
395.01	0.01\\
396.01	0.01\\
397.01	0.01\\
398.01	0.01\\
399.01	0.01\\
400.01	0.01\\
401.01	0.01\\
402.01	0.01\\
403.01	0.01\\
404.01	0.01\\
405.01	0.01\\
406.01	0.01\\
407.01	0.01\\
408.01	0.01\\
409.01	0.01\\
410.01	0.01\\
411.01	0.01\\
412.01	0.01\\
413.01	0.01\\
414.01	0.01\\
415.01	0.01\\
416.01	0.01\\
417.01	0.01\\
418.01	0.01\\
419.01	0.01\\
420.01	0.01\\
421.01	0.01\\
422.01	0.01\\
423.01	0.01\\
424.01	0.01\\
425.01	0.01\\
426.01	0.01\\
427.01	0.01\\
428.01	0.01\\
429.01	0.01\\
430.01	0.01\\
431.01	0.01\\
432.01	0.01\\
433.01	0.01\\
434.01	0.01\\
435.01	0.01\\
436.01	0.01\\
437.01	0.01\\
438.01	0.01\\
439.01	0.01\\
440.01	0.01\\
441.01	0.01\\
442.01	0.01\\
443.01	0.01\\
444.01	0.01\\
445.01	0.01\\
446.01	0.01\\
447.01	0.01\\
448.01	0.01\\
449.01	0.01\\
450.01	0.01\\
451.01	0.01\\
452.01	0.01\\
453.01	0.01\\
454.01	0.01\\
455.01	0.01\\
456.01	0.01\\
457.01	0.01\\
458.01	0.01\\
459.01	0.01\\
460.01	0.01\\
461.01	0.01\\
462.01	0.01\\
463.01	0.01\\
464.01	0.01\\
465.01	0.01\\
466.01	0.01\\
467.01	0.01\\
468.01	0.01\\
469.01	0.01\\
470.01	0.01\\
471.01	0.01\\
472.01	0.01\\
473.01	0.01\\
474.01	0.01\\
475.01	0.01\\
476.01	0.01\\
477.01	0.01\\
478.01	0.01\\
479.01	0.01\\
480.01	0.01\\
481.01	0.01\\
482.01	0.01\\
483.01	0.01\\
484.01	0.01\\
485.01	0.01\\
486.01	0.01\\
487.01	0.01\\
488.01	0.01\\
489.01	0.01\\
490.01	0.01\\
491.01	0.01\\
492.01	0.01\\
493.01	0.01\\
494.01	0.01\\
495.01	0.01\\
496.01	0.01\\
497.01	0.01\\
498.01	0.01\\
499.01	0.01\\
500.01	0.01\\
501.01	0.01\\
502.01	0.01\\
503.01	0.01\\
504.01	0.01\\
505.01	0.01\\
506.01	0.01\\
507.01	0.01\\
508.01	0.01\\
509.01	0.01\\
510.01	0.01\\
511.01	0.01\\
512.01	0.01\\
513.01	0.01\\
514.01	0.01\\
515.01	0.01\\
516.01	0.01\\
517.01	0.01\\
518.01	0.01\\
519.01	0.01\\
520.01	0.01\\
521.01	0.01\\
522.01	0.01\\
523.01	0.01\\
524.01	0.01\\
525.01	0.01\\
526.01	0.01\\
527.01	0.01\\
528.01	0.01\\
529.01	0.01\\
530.01	0.01\\
531.01	0.01\\
532.01	0.01\\
533.01	0.01\\
534.01	0.01\\
535.01	0.01\\
536.01	0.01\\
537.01	0.01\\
538.01	0.01\\
539.01	0.01\\
540.01	0.01\\
541.01	0.01\\
542.01	0.01\\
543.01	0.01\\
544.01	0.01\\
545.01	0.01\\
546.01	0.01\\
547.01	0.01\\
548.01	0.01\\
549.01	0.01\\
550.01	0.01\\
551.01	0.01\\
552.01	0.01\\
553.01	0.01\\
554.01	0.01\\
555.01	0.01\\
556.01	0.01\\
557.01	0.01\\
558.01	0.01\\
559.01	0.01\\
560.01	0.01\\
561.01	0.01\\
562.01	0.01\\
563.01	0.01\\
564.01	0.01\\
565.01	0.01\\
566.01	0.01\\
567.01	0.01\\
568.01	0.01\\
569.01	0.01\\
570.01	0.01\\
571.01	0.01\\
572.01	0.01\\
573.01	0.01\\
574.01	0.01\\
575.01	0.01\\
576.01	0.01\\
577.01	0.01\\
578.01	0.01\\
579.01	0.01\\
580.01	0.01\\
581.01	0.01\\
582.01	0.01\\
583.01	0.01\\
584.01	0.01\\
585.01	0.01\\
586.01	0.01\\
587.01	0.01\\
588.01	0.01\\
589.01	0.01\\
590.01	0.01\\
591.01	0.01\\
592.01	0.01\\
593.01	0.01\\
594.01	0.01\\
595.01	0.01\\
596.01	0.01\\
597.01	0.01\\
598.01	0.00865254016279302\\
599.01	0.00624186909405409\\
599.02	0.00620414441242163\\
599.03	0.00616605275622736\\
599.04	0.00612759052101569\\
599.05	0.00608875406692034\\
599.06	0.00604953971831641\\
599.07	0.00600994376346915\\
599.08	0.00596996245417918\\
599.09	0.00592959200542425\\
599.1	0.00588882859499752\\
599.11	0.00584766836314221\\
599.12	0.00580610741218275\\
599.13	0.00576414180615225\\
599.14	0.00572176757041628\\
599.15	0.00567898069129307\\
599.16	0.00563577711566984\\
599.17	0.00559215275061546\\
599.18	0.00554810346298925\\
599.19	0.005503625079046\\
599.2	0.00545871338403701\\
599.21	0.0054133641218073\\
599.22	0.0053675729943888\\
599.23	0.00532133566158966\\
599.24	0.00527464774057935\\
599.25	0.00522750480546976\\
599.26	0.0051799023868923\\
599.27	0.00513183597157067\\
599.28	0.00508330100188951\\
599.29	0.00503429287545885\\
599.3	0.00498480694467418\\
599.31	0.00493483851627228\\
599.32	0.00488438285088275\\
599.33	0.0048334351625749\\
599.34	0.00478199061840046\\
599.35	0.00473004433793161\\
599.36	0.00467759139279455\\
599.37	0.0046246268061985\\
599.38	0.00457114555246003\\
599.39	0.00451714256049735\\
599.4	0.00446261272376356\\
599.41	0.00440755088572619\\
599.42	0.00435195183937782\\
599.43	0.00429581032674211\\
599.44	0.00423912103837487\\
599.45	0.00418187861286022\\
599.46	0.00412407763630194\\
599.47	0.00406571264180972\\
599.48	0.00400677810898046\\
599.49	0.00394726846337443\\
599.5	0.0038871780759864\\
599.51	0.00382650126271144\\
599.52	0.00376523228380568\\
599.53	0.00370336534334161\\
599.54	0.00364089458865821\\
599.55	0.00357781410980551\\
599.56	0.00351411793898394\\
599.57	0.00344980004997799\\
599.58	0.00338485435758449\\
599.59	0.0033192747170351\\
599.6	0.00325305492341342\\
599.61	0.00318618871106605\\
599.62	0.00311866975300822\\
599.63	0.00305049166032344\\
599.64	0.00298164798155718\\
599.65	0.00291213220210483\\
599.66	0.00284193774359354\\
599.67	0.00277105796325799\\
599.68	0.00269948615331015\\
599.69	0.00262721554030276\\
599.7	0.00255423928448666\\
599.71	0.00248055047916181\\
599.72	0.00240614215002193\\
599.73	0.00233100725449273\\
599.74	0.00225513868106373\\
599.75	0.00217852924861345\\
599.76	0.002101171705728\\
599.77	0.00202305873001312\\
599.78	0.00194418292739929\\
599.79	0.00186453683144024\\
599.8	0.00178411290260446\\
599.81	0.00170290352755979\\
599.82	0.0016209010184511\\
599.83	0.00153809761217073\\
599.84	0.001454485469622\\
599.85	0.00137005667497534\\
599.86	0.00128480323491721\\
599.87	0.00119871707789166\\
599.88	0.00111179005333448\\
599.89	0.00102401393089982\\
599.9	0.000935380399679272\\
599.91	0.000845881067413286\\
599.92	0.000755507459694884\\
599.93	0.000664251019165561\\
599.94	0.000572103104703356\\
599.95	0.00047905499060291\\
599.96	0.000385097865747554\\
599.97	0.000290222832773275\\
599.98	0.000194420907224489\\
599.99	9.76830167015649e-05\\
600	0\\
};
\addplot [color=black!60!mycolor21,solid,forget plot]
  table[row sep=crcr]{%
0.01	0.01\\
1.01	0.01\\
2.01	0.01\\
3.01	0.01\\
4.01	0.01\\
5.01	0.01\\
6.01	0.01\\
7.01	0.01\\
8.01	0.01\\
9.01	0.01\\
10.01	0.01\\
11.01	0.01\\
12.01	0.01\\
13.01	0.01\\
14.01	0.01\\
15.01	0.01\\
16.01	0.01\\
17.01	0.01\\
18.01	0.01\\
19.01	0.01\\
20.01	0.01\\
21.01	0.01\\
22.01	0.01\\
23.01	0.01\\
24.01	0.01\\
25.01	0.01\\
26.01	0.01\\
27.01	0.01\\
28.01	0.01\\
29.01	0.01\\
30.01	0.01\\
31.01	0.01\\
32.01	0.01\\
33.01	0.01\\
34.01	0.01\\
35.01	0.01\\
36.01	0.01\\
37.01	0.01\\
38.01	0.01\\
39.01	0.01\\
40.01	0.01\\
41.01	0.01\\
42.01	0.01\\
43.01	0.01\\
44.01	0.01\\
45.01	0.01\\
46.01	0.01\\
47.01	0.01\\
48.01	0.01\\
49.01	0.01\\
50.01	0.01\\
51.01	0.01\\
52.01	0.01\\
53.01	0.01\\
54.01	0.01\\
55.01	0.01\\
56.01	0.01\\
57.01	0.01\\
58.01	0.01\\
59.01	0.01\\
60.01	0.01\\
61.01	0.01\\
62.01	0.01\\
63.01	0.01\\
64.01	0.01\\
65.01	0.01\\
66.01	0.01\\
67.01	0.01\\
68.01	0.01\\
69.01	0.01\\
70.01	0.01\\
71.01	0.01\\
72.01	0.01\\
73.01	0.01\\
74.01	0.01\\
75.01	0.01\\
76.01	0.01\\
77.01	0.01\\
78.01	0.01\\
79.01	0.01\\
80.01	0.01\\
81.01	0.01\\
82.01	0.01\\
83.01	0.01\\
84.01	0.01\\
85.01	0.01\\
86.01	0.01\\
87.01	0.01\\
88.01	0.01\\
89.01	0.01\\
90.01	0.01\\
91.01	0.01\\
92.01	0.01\\
93.01	0.01\\
94.01	0.01\\
95.01	0.01\\
96.01	0.01\\
97.01	0.01\\
98.01	0.01\\
99.01	0.01\\
100.01	0.01\\
101.01	0.01\\
102.01	0.01\\
103.01	0.01\\
104.01	0.01\\
105.01	0.01\\
106.01	0.01\\
107.01	0.01\\
108.01	0.01\\
109.01	0.01\\
110.01	0.01\\
111.01	0.01\\
112.01	0.01\\
113.01	0.01\\
114.01	0.01\\
115.01	0.01\\
116.01	0.01\\
117.01	0.01\\
118.01	0.01\\
119.01	0.01\\
120.01	0.01\\
121.01	0.01\\
122.01	0.01\\
123.01	0.01\\
124.01	0.01\\
125.01	0.01\\
126.01	0.01\\
127.01	0.01\\
128.01	0.01\\
129.01	0.01\\
130.01	0.01\\
131.01	0.01\\
132.01	0.01\\
133.01	0.01\\
134.01	0.01\\
135.01	0.01\\
136.01	0.01\\
137.01	0.01\\
138.01	0.01\\
139.01	0.01\\
140.01	0.01\\
141.01	0.01\\
142.01	0.01\\
143.01	0.01\\
144.01	0.01\\
145.01	0.01\\
146.01	0.01\\
147.01	0.01\\
148.01	0.01\\
149.01	0.01\\
150.01	0.01\\
151.01	0.01\\
152.01	0.01\\
153.01	0.01\\
154.01	0.01\\
155.01	0.01\\
156.01	0.01\\
157.01	0.01\\
158.01	0.01\\
159.01	0.01\\
160.01	0.01\\
161.01	0.01\\
162.01	0.01\\
163.01	0.01\\
164.01	0.01\\
165.01	0.01\\
166.01	0.01\\
167.01	0.01\\
168.01	0.01\\
169.01	0.01\\
170.01	0.01\\
171.01	0.01\\
172.01	0.01\\
173.01	0.01\\
174.01	0.01\\
175.01	0.01\\
176.01	0.01\\
177.01	0.01\\
178.01	0.01\\
179.01	0.01\\
180.01	0.01\\
181.01	0.01\\
182.01	0.01\\
183.01	0.01\\
184.01	0.01\\
185.01	0.01\\
186.01	0.01\\
187.01	0.01\\
188.01	0.01\\
189.01	0.01\\
190.01	0.01\\
191.01	0.01\\
192.01	0.01\\
193.01	0.01\\
194.01	0.01\\
195.01	0.01\\
196.01	0.01\\
197.01	0.01\\
198.01	0.01\\
199.01	0.01\\
200.01	0.01\\
201.01	0.01\\
202.01	0.01\\
203.01	0.01\\
204.01	0.01\\
205.01	0.01\\
206.01	0.01\\
207.01	0.01\\
208.01	0.01\\
209.01	0.01\\
210.01	0.01\\
211.01	0.01\\
212.01	0.01\\
213.01	0.01\\
214.01	0.01\\
215.01	0.01\\
216.01	0.01\\
217.01	0.01\\
218.01	0.01\\
219.01	0.01\\
220.01	0.01\\
221.01	0.01\\
222.01	0.01\\
223.01	0.01\\
224.01	0.01\\
225.01	0.01\\
226.01	0.01\\
227.01	0.01\\
228.01	0.01\\
229.01	0.01\\
230.01	0.01\\
231.01	0.01\\
232.01	0.01\\
233.01	0.01\\
234.01	0.01\\
235.01	0.01\\
236.01	0.01\\
237.01	0.01\\
238.01	0.01\\
239.01	0.01\\
240.01	0.01\\
241.01	0.01\\
242.01	0.01\\
243.01	0.01\\
244.01	0.01\\
245.01	0.01\\
246.01	0.01\\
247.01	0.01\\
248.01	0.01\\
249.01	0.01\\
250.01	0.01\\
251.01	0.01\\
252.01	0.01\\
253.01	0.01\\
254.01	0.01\\
255.01	0.01\\
256.01	0.01\\
257.01	0.01\\
258.01	0.01\\
259.01	0.01\\
260.01	0.01\\
261.01	0.01\\
262.01	0.01\\
263.01	0.01\\
264.01	0.01\\
265.01	0.01\\
266.01	0.01\\
267.01	0.01\\
268.01	0.01\\
269.01	0.01\\
270.01	0.01\\
271.01	0.01\\
272.01	0.01\\
273.01	0.01\\
274.01	0.01\\
275.01	0.01\\
276.01	0.01\\
277.01	0.01\\
278.01	0.01\\
279.01	0.01\\
280.01	0.01\\
281.01	0.01\\
282.01	0.01\\
283.01	0.01\\
284.01	0.01\\
285.01	0.01\\
286.01	0.01\\
287.01	0.01\\
288.01	0.01\\
289.01	0.01\\
290.01	0.01\\
291.01	0.01\\
292.01	0.01\\
293.01	0.01\\
294.01	0.01\\
295.01	0.01\\
296.01	0.01\\
297.01	0.01\\
298.01	0.01\\
299.01	0.01\\
300.01	0.01\\
301.01	0.01\\
302.01	0.01\\
303.01	0.01\\
304.01	0.01\\
305.01	0.01\\
306.01	0.01\\
307.01	0.01\\
308.01	0.01\\
309.01	0.01\\
310.01	0.01\\
311.01	0.01\\
312.01	0.01\\
313.01	0.01\\
314.01	0.01\\
315.01	0.01\\
316.01	0.01\\
317.01	0.01\\
318.01	0.01\\
319.01	0.01\\
320.01	0.01\\
321.01	0.01\\
322.01	0.01\\
323.01	0.01\\
324.01	0.01\\
325.01	0.01\\
326.01	0.01\\
327.01	0.01\\
328.01	0.01\\
329.01	0.01\\
330.01	0.01\\
331.01	0.01\\
332.01	0.01\\
333.01	0.01\\
334.01	0.01\\
335.01	0.01\\
336.01	0.01\\
337.01	0.01\\
338.01	0.01\\
339.01	0.01\\
340.01	0.01\\
341.01	0.01\\
342.01	0.01\\
343.01	0.01\\
344.01	0.01\\
345.01	0.01\\
346.01	0.01\\
347.01	0.01\\
348.01	0.01\\
349.01	0.01\\
350.01	0.01\\
351.01	0.01\\
352.01	0.01\\
353.01	0.01\\
354.01	0.01\\
355.01	0.01\\
356.01	0.01\\
357.01	0.01\\
358.01	0.01\\
359.01	0.01\\
360.01	0.01\\
361.01	0.01\\
362.01	0.01\\
363.01	0.01\\
364.01	0.01\\
365.01	0.01\\
366.01	0.01\\
367.01	0.01\\
368.01	0.01\\
369.01	0.01\\
370.01	0.01\\
371.01	0.01\\
372.01	0.01\\
373.01	0.01\\
374.01	0.01\\
375.01	0.01\\
376.01	0.01\\
377.01	0.01\\
378.01	0.01\\
379.01	0.01\\
380.01	0.01\\
381.01	0.01\\
382.01	0.01\\
383.01	0.01\\
384.01	0.01\\
385.01	0.01\\
386.01	0.01\\
387.01	0.01\\
388.01	0.01\\
389.01	0.01\\
390.01	0.01\\
391.01	0.01\\
392.01	0.01\\
393.01	0.01\\
394.01	0.01\\
395.01	0.01\\
396.01	0.01\\
397.01	0.01\\
398.01	0.01\\
399.01	0.01\\
400.01	0.01\\
401.01	0.01\\
402.01	0.01\\
403.01	0.01\\
404.01	0.01\\
405.01	0.01\\
406.01	0.01\\
407.01	0.01\\
408.01	0.01\\
409.01	0.01\\
410.01	0.01\\
411.01	0.01\\
412.01	0.01\\
413.01	0.01\\
414.01	0.01\\
415.01	0.01\\
416.01	0.01\\
417.01	0.01\\
418.01	0.01\\
419.01	0.01\\
420.01	0.01\\
421.01	0.01\\
422.01	0.01\\
423.01	0.01\\
424.01	0.01\\
425.01	0.01\\
426.01	0.01\\
427.01	0.01\\
428.01	0.01\\
429.01	0.01\\
430.01	0.01\\
431.01	0.01\\
432.01	0.01\\
433.01	0.01\\
434.01	0.01\\
435.01	0.01\\
436.01	0.01\\
437.01	0.01\\
438.01	0.01\\
439.01	0.01\\
440.01	0.01\\
441.01	0.01\\
442.01	0.01\\
443.01	0.01\\
444.01	0.01\\
445.01	0.01\\
446.01	0.01\\
447.01	0.01\\
448.01	0.01\\
449.01	0.01\\
450.01	0.01\\
451.01	0.01\\
452.01	0.01\\
453.01	0.01\\
454.01	0.01\\
455.01	0.01\\
456.01	0.01\\
457.01	0.01\\
458.01	0.01\\
459.01	0.01\\
460.01	0.01\\
461.01	0.01\\
462.01	0.01\\
463.01	0.01\\
464.01	0.01\\
465.01	0.01\\
466.01	0.01\\
467.01	0.01\\
468.01	0.01\\
469.01	0.01\\
470.01	0.01\\
471.01	0.01\\
472.01	0.01\\
473.01	0.01\\
474.01	0.01\\
475.01	0.01\\
476.01	0.01\\
477.01	0.01\\
478.01	0.01\\
479.01	0.01\\
480.01	0.01\\
481.01	0.01\\
482.01	0.01\\
483.01	0.01\\
484.01	0.01\\
485.01	0.01\\
486.01	0.01\\
487.01	0.01\\
488.01	0.01\\
489.01	0.01\\
490.01	0.01\\
491.01	0.01\\
492.01	0.01\\
493.01	0.01\\
494.01	0.01\\
495.01	0.01\\
496.01	0.01\\
497.01	0.01\\
498.01	0.01\\
499.01	0.01\\
500.01	0.01\\
501.01	0.01\\
502.01	0.01\\
503.01	0.01\\
504.01	0.01\\
505.01	0.01\\
506.01	0.01\\
507.01	0.01\\
508.01	0.01\\
509.01	0.01\\
510.01	0.01\\
511.01	0.01\\
512.01	0.01\\
513.01	0.01\\
514.01	0.01\\
515.01	0.01\\
516.01	0.01\\
517.01	0.01\\
518.01	0.01\\
519.01	0.01\\
520.01	0.01\\
521.01	0.01\\
522.01	0.01\\
523.01	0.01\\
524.01	0.01\\
525.01	0.01\\
526.01	0.01\\
527.01	0.01\\
528.01	0.01\\
529.01	0.01\\
530.01	0.01\\
531.01	0.01\\
532.01	0.01\\
533.01	0.01\\
534.01	0.01\\
535.01	0.01\\
536.01	0.01\\
537.01	0.01\\
538.01	0.01\\
539.01	0.01\\
540.01	0.01\\
541.01	0.01\\
542.01	0.01\\
543.01	0.01\\
544.01	0.01\\
545.01	0.01\\
546.01	0.01\\
547.01	0.01\\
548.01	0.01\\
549.01	0.01\\
550.01	0.01\\
551.01	0.01\\
552.01	0.01\\
553.01	0.01\\
554.01	0.01\\
555.01	0.01\\
556.01	0.01\\
557.01	0.01\\
558.01	0.01\\
559.01	0.01\\
560.01	0.01\\
561.01	0.01\\
562.01	0.01\\
563.01	0.01\\
564.01	0.01\\
565.01	0.01\\
566.01	0.01\\
567.01	0.01\\
568.01	0.01\\
569.01	0.01\\
570.01	0.01\\
571.01	0.01\\
572.01	0.01\\
573.01	0.01\\
574.01	0.01\\
575.01	0.01\\
576.01	0.01\\
577.01	0.01\\
578.01	0.01\\
579.01	0.01\\
580.01	0.01\\
581.01	0.01\\
582.01	0.01\\
583.01	0.01\\
584.01	0.01\\
585.01	0.01\\
586.01	0.01\\
587.01	0.01\\
588.01	0.01\\
589.01	0.01\\
590.01	0.01\\
591.01	0.01\\
592.01	0.01\\
593.01	0.01\\
594.01	0.01\\
595.01	0.01\\
596.01	0.01\\
597.01	0.01\\
598.01	0.00865150542787716\\
599.01	0.00624186909405409\\
599.02	0.00620414441242163\\
599.03	0.00616605275622734\\
599.04	0.00612759052101569\\
599.05	0.00608875406692033\\
599.06	0.00604953971831639\\
599.07	0.00600994376346915\\
599.08	0.00596996245417917\\
599.09	0.00592959200542425\\
599.1	0.00588882859499752\\
599.11	0.00584766836314221\\
599.12	0.00580610741218276\\
599.13	0.00576414180615225\\
599.14	0.00572176757041629\\
599.15	0.00567898069129309\\
599.16	0.00563577711566987\\
599.17	0.00559215275061548\\
599.18	0.00554810346298928\\
599.19	0.00550362507904603\\
599.2	0.00545871338403702\\
599.21	0.0054133641218073\\
599.22	0.00536757299438882\\
599.23	0.00532133566158968\\
599.24	0.00527464774057935\\
599.25	0.00522750480546976\\
599.26	0.0051799023868923\\
599.27	0.00513183597157068\\
599.28	0.00508330100188954\\
599.29	0.00503429287545886\\
599.3	0.00498480694467418\\
599.31	0.00493483851627229\\
599.32	0.00488438285088275\\
599.33	0.0048334351625749\\
599.34	0.00478199061840045\\
599.35	0.00473004433793158\\
599.36	0.00467759139279451\\
599.37	0.00462462680619847\\
599.38	0.00457114555246\\
599.39	0.00451714256049731\\
599.4	0.00446261272376353\\
599.41	0.00440755088572614\\
599.42	0.00435195183937776\\
599.43	0.00429581032674206\\
599.44	0.00423912103837481\\
599.45	0.00418187861286016\\
599.46	0.00412407763630187\\
599.47	0.00406571264180965\\
599.48	0.00400677810898039\\
599.49	0.00394726846337437\\
599.5	0.00388717807598633\\
599.51	0.00382650126271137\\
599.52	0.00376523228380562\\
599.53	0.00370336534334157\\
599.54	0.00364089458865816\\
599.55	0.00357781410980548\\
599.56	0.00351411793898392\\
599.57	0.00344980004997798\\
599.58	0.00338485435758447\\
599.59	0.0033192747170351\\
599.6	0.00325305492341341\\
599.61	0.00318618871106603\\
599.62	0.00311866975300821\\
599.63	0.00305049166032342\\
599.64	0.00298164798155715\\
599.65	0.0029121322021048\\
599.66	0.00284193774359352\\
599.67	0.00277105796325798\\
599.68	0.00269948615331014\\
599.69	0.00262721554030274\\
599.7	0.00255423928448665\\
599.71	0.0024805504791618\\
599.72	0.00240614215002191\\
599.73	0.00233100725449272\\
599.74	0.00225513868106372\\
599.75	0.00217852924861344\\
599.76	0.00210117170572799\\
599.77	0.0020230587300131\\
599.78	0.00194418292739927\\
599.79	0.00186453683144023\\
599.8	0.00178411290260445\\
599.81	0.00170290352755979\\
599.82	0.00162090101845109\\
599.83	0.00153809761217072\\
599.84	0.001454485469622\\
599.85	0.00137005667497534\\
599.86	0.00128480323491721\\
599.87	0.00119871707789166\\
599.88	0.00111179005333448\\
599.89	0.00102401393089982\\
599.9	0.000935380399679274\\
599.91	0.000845881067413283\\
599.92	0.000755507459694877\\
599.93	0.000664251019165561\\
599.94	0.000572103104703355\\
599.95	0.000479054990602912\\
599.96	0.000385097865747556\\
599.97	0.000290222832773275\\
599.98	0.000194420907224489\\
599.99	9.76830167015632e-05\\
600	0\\
};
\addplot [color=black!80!mycolor21,solid,forget plot]
  table[row sep=crcr]{%
0.01	0.01\\
1.01	0.01\\
2.01	0.01\\
3.01	0.01\\
4.01	0.01\\
5.01	0.01\\
6.01	0.01\\
7.01	0.01\\
8.01	0.01\\
9.01	0.01\\
10.01	0.01\\
11.01	0.01\\
12.01	0.01\\
13.01	0.01\\
14.01	0.01\\
15.01	0.01\\
16.01	0.01\\
17.01	0.01\\
18.01	0.01\\
19.01	0.01\\
20.01	0.01\\
21.01	0.01\\
22.01	0.01\\
23.01	0.01\\
24.01	0.01\\
25.01	0.01\\
26.01	0.01\\
27.01	0.01\\
28.01	0.01\\
29.01	0.01\\
30.01	0.01\\
31.01	0.01\\
32.01	0.01\\
33.01	0.01\\
34.01	0.01\\
35.01	0.01\\
36.01	0.01\\
37.01	0.01\\
38.01	0.01\\
39.01	0.01\\
40.01	0.01\\
41.01	0.01\\
42.01	0.01\\
43.01	0.01\\
44.01	0.01\\
45.01	0.01\\
46.01	0.01\\
47.01	0.01\\
48.01	0.01\\
49.01	0.01\\
50.01	0.01\\
51.01	0.01\\
52.01	0.01\\
53.01	0.01\\
54.01	0.01\\
55.01	0.01\\
56.01	0.01\\
57.01	0.01\\
58.01	0.01\\
59.01	0.01\\
60.01	0.01\\
61.01	0.01\\
62.01	0.01\\
63.01	0.01\\
64.01	0.01\\
65.01	0.01\\
66.01	0.01\\
67.01	0.01\\
68.01	0.01\\
69.01	0.01\\
70.01	0.01\\
71.01	0.01\\
72.01	0.01\\
73.01	0.01\\
74.01	0.01\\
75.01	0.01\\
76.01	0.01\\
77.01	0.01\\
78.01	0.01\\
79.01	0.01\\
80.01	0.01\\
81.01	0.01\\
82.01	0.01\\
83.01	0.01\\
84.01	0.01\\
85.01	0.01\\
86.01	0.01\\
87.01	0.01\\
88.01	0.01\\
89.01	0.01\\
90.01	0.01\\
91.01	0.01\\
92.01	0.01\\
93.01	0.01\\
94.01	0.01\\
95.01	0.01\\
96.01	0.01\\
97.01	0.01\\
98.01	0.01\\
99.01	0.01\\
100.01	0.01\\
101.01	0.01\\
102.01	0.01\\
103.01	0.01\\
104.01	0.01\\
105.01	0.01\\
106.01	0.01\\
107.01	0.01\\
108.01	0.01\\
109.01	0.01\\
110.01	0.01\\
111.01	0.01\\
112.01	0.01\\
113.01	0.01\\
114.01	0.01\\
115.01	0.01\\
116.01	0.01\\
117.01	0.01\\
118.01	0.01\\
119.01	0.01\\
120.01	0.01\\
121.01	0.01\\
122.01	0.01\\
123.01	0.01\\
124.01	0.01\\
125.01	0.01\\
126.01	0.01\\
127.01	0.01\\
128.01	0.01\\
129.01	0.01\\
130.01	0.01\\
131.01	0.01\\
132.01	0.01\\
133.01	0.01\\
134.01	0.01\\
135.01	0.01\\
136.01	0.01\\
137.01	0.01\\
138.01	0.01\\
139.01	0.01\\
140.01	0.01\\
141.01	0.01\\
142.01	0.01\\
143.01	0.01\\
144.01	0.01\\
145.01	0.01\\
146.01	0.01\\
147.01	0.01\\
148.01	0.01\\
149.01	0.01\\
150.01	0.01\\
151.01	0.01\\
152.01	0.01\\
153.01	0.01\\
154.01	0.01\\
155.01	0.01\\
156.01	0.01\\
157.01	0.01\\
158.01	0.01\\
159.01	0.01\\
160.01	0.01\\
161.01	0.01\\
162.01	0.01\\
163.01	0.01\\
164.01	0.01\\
165.01	0.01\\
166.01	0.01\\
167.01	0.01\\
168.01	0.01\\
169.01	0.01\\
170.01	0.01\\
171.01	0.01\\
172.01	0.01\\
173.01	0.01\\
174.01	0.01\\
175.01	0.01\\
176.01	0.01\\
177.01	0.01\\
178.01	0.01\\
179.01	0.01\\
180.01	0.01\\
181.01	0.01\\
182.01	0.01\\
183.01	0.01\\
184.01	0.01\\
185.01	0.01\\
186.01	0.01\\
187.01	0.01\\
188.01	0.01\\
189.01	0.01\\
190.01	0.01\\
191.01	0.01\\
192.01	0.01\\
193.01	0.01\\
194.01	0.01\\
195.01	0.01\\
196.01	0.01\\
197.01	0.01\\
198.01	0.01\\
199.01	0.01\\
200.01	0.01\\
201.01	0.01\\
202.01	0.01\\
203.01	0.01\\
204.01	0.01\\
205.01	0.01\\
206.01	0.01\\
207.01	0.01\\
208.01	0.01\\
209.01	0.01\\
210.01	0.01\\
211.01	0.01\\
212.01	0.01\\
213.01	0.01\\
214.01	0.01\\
215.01	0.01\\
216.01	0.01\\
217.01	0.01\\
218.01	0.01\\
219.01	0.01\\
220.01	0.01\\
221.01	0.01\\
222.01	0.01\\
223.01	0.01\\
224.01	0.01\\
225.01	0.01\\
226.01	0.01\\
227.01	0.01\\
228.01	0.01\\
229.01	0.01\\
230.01	0.01\\
231.01	0.01\\
232.01	0.01\\
233.01	0.01\\
234.01	0.01\\
235.01	0.01\\
236.01	0.01\\
237.01	0.01\\
238.01	0.01\\
239.01	0.01\\
240.01	0.01\\
241.01	0.01\\
242.01	0.01\\
243.01	0.01\\
244.01	0.01\\
245.01	0.01\\
246.01	0.01\\
247.01	0.01\\
248.01	0.01\\
249.01	0.01\\
250.01	0.01\\
251.01	0.01\\
252.01	0.01\\
253.01	0.01\\
254.01	0.01\\
255.01	0.01\\
256.01	0.01\\
257.01	0.01\\
258.01	0.01\\
259.01	0.01\\
260.01	0.01\\
261.01	0.01\\
262.01	0.01\\
263.01	0.01\\
264.01	0.01\\
265.01	0.01\\
266.01	0.01\\
267.01	0.01\\
268.01	0.01\\
269.01	0.01\\
270.01	0.01\\
271.01	0.01\\
272.01	0.01\\
273.01	0.01\\
274.01	0.01\\
275.01	0.01\\
276.01	0.01\\
277.01	0.01\\
278.01	0.01\\
279.01	0.01\\
280.01	0.01\\
281.01	0.01\\
282.01	0.01\\
283.01	0.01\\
284.01	0.01\\
285.01	0.01\\
286.01	0.01\\
287.01	0.01\\
288.01	0.01\\
289.01	0.01\\
290.01	0.01\\
291.01	0.01\\
292.01	0.01\\
293.01	0.01\\
294.01	0.01\\
295.01	0.01\\
296.01	0.01\\
297.01	0.01\\
298.01	0.01\\
299.01	0.01\\
300.01	0.01\\
301.01	0.01\\
302.01	0.01\\
303.01	0.01\\
304.01	0.01\\
305.01	0.01\\
306.01	0.01\\
307.01	0.01\\
308.01	0.01\\
309.01	0.01\\
310.01	0.01\\
311.01	0.01\\
312.01	0.01\\
313.01	0.01\\
314.01	0.01\\
315.01	0.01\\
316.01	0.01\\
317.01	0.01\\
318.01	0.01\\
319.01	0.01\\
320.01	0.01\\
321.01	0.01\\
322.01	0.01\\
323.01	0.01\\
324.01	0.01\\
325.01	0.01\\
326.01	0.01\\
327.01	0.01\\
328.01	0.01\\
329.01	0.01\\
330.01	0.01\\
331.01	0.01\\
332.01	0.01\\
333.01	0.01\\
334.01	0.01\\
335.01	0.01\\
336.01	0.01\\
337.01	0.01\\
338.01	0.01\\
339.01	0.01\\
340.01	0.01\\
341.01	0.01\\
342.01	0.01\\
343.01	0.01\\
344.01	0.01\\
345.01	0.01\\
346.01	0.01\\
347.01	0.01\\
348.01	0.01\\
349.01	0.01\\
350.01	0.01\\
351.01	0.01\\
352.01	0.01\\
353.01	0.01\\
354.01	0.01\\
355.01	0.01\\
356.01	0.01\\
357.01	0.01\\
358.01	0.01\\
359.01	0.01\\
360.01	0.01\\
361.01	0.01\\
362.01	0.01\\
363.01	0.01\\
364.01	0.01\\
365.01	0.01\\
366.01	0.01\\
367.01	0.01\\
368.01	0.01\\
369.01	0.01\\
370.01	0.01\\
371.01	0.01\\
372.01	0.01\\
373.01	0.01\\
374.01	0.01\\
375.01	0.01\\
376.01	0.01\\
377.01	0.01\\
378.01	0.01\\
379.01	0.01\\
380.01	0.01\\
381.01	0.01\\
382.01	0.01\\
383.01	0.01\\
384.01	0.01\\
385.01	0.01\\
386.01	0.01\\
387.01	0.01\\
388.01	0.01\\
389.01	0.01\\
390.01	0.01\\
391.01	0.01\\
392.01	0.01\\
393.01	0.01\\
394.01	0.01\\
395.01	0.01\\
396.01	0.01\\
397.01	0.01\\
398.01	0.01\\
399.01	0.01\\
400.01	0.01\\
401.01	0.01\\
402.01	0.01\\
403.01	0.01\\
404.01	0.01\\
405.01	0.01\\
406.01	0.01\\
407.01	0.01\\
408.01	0.01\\
409.01	0.01\\
410.01	0.01\\
411.01	0.01\\
412.01	0.01\\
413.01	0.01\\
414.01	0.01\\
415.01	0.01\\
416.01	0.01\\
417.01	0.01\\
418.01	0.01\\
419.01	0.01\\
420.01	0.01\\
421.01	0.01\\
422.01	0.01\\
423.01	0.01\\
424.01	0.01\\
425.01	0.01\\
426.01	0.01\\
427.01	0.01\\
428.01	0.01\\
429.01	0.01\\
430.01	0.01\\
431.01	0.01\\
432.01	0.01\\
433.01	0.01\\
434.01	0.01\\
435.01	0.01\\
436.01	0.01\\
437.01	0.01\\
438.01	0.01\\
439.01	0.01\\
440.01	0.01\\
441.01	0.01\\
442.01	0.01\\
443.01	0.01\\
444.01	0.01\\
445.01	0.01\\
446.01	0.01\\
447.01	0.01\\
448.01	0.01\\
449.01	0.01\\
450.01	0.01\\
451.01	0.01\\
452.01	0.01\\
453.01	0.01\\
454.01	0.01\\
455.01	0.01\\
456.01	0.01\\
457.01	0.01\\
458.01	0.01\\
459.01	0.01\\
460.01	0.01\\
461.01	0.01\\
462.01	0.01\\
463.01	0.01\\
464.01	0.01\\
465.01	0.01\\
466.01	0.01\\
467.01	0.01\\
468.01	0.01\\
469.01	0.01\\
470.01	0.01\\
471.01	0.01\\
472.01	0.01\\
473.01	0.01\\
474.01	0.01\\
475.01	0.01\\
476.01	0.01\\
477.01	0.01\\
478.01	0.01\\
479.01	0.01\\
480.01	0.01\\
481.01	0.01\\
482.01	0.01\\
483.01	0.01\\
484.01	0.01\\
485.01	0.01\\
486.01	0.01\\
487.01	0.01\\
488.01	0.01\\
489.01	0.01\\
490.01	0.01\\
491.01	0.01\\
492.01	0.01\\
493.01	0.01\\
494.01	0.01\\
495.01	0.01\\
496.01	0.01\\
497.01	0.01\\
498.01	0.01\\
499.01	0.01\\
500.01	0.01\\
501.01	0.01\\
502.01	0.01\\
503.01	0.01\\
504.01	0.01\\
505.01	0.01\\
506.01	0.01\\
507.01	0.01\\
508.01	0.01\\
509.01	0.01\\
510.01	0.01\\
511.01	0.01\\
512.01	0.01\\
513.01	0.01\\
514.01	0.01\\
515.01	0.01\\
516.01	0.01\\
517.01	0.01\\
518.01	0.01\\
519.01	0.01\\
520.01	0.01\\
521.01	0.01\\
522.01	0.01\\
523.01	0.01\\
524.01	0.01\\
525.01	0.01\\
526.01	0.01\\
527.01	0.01\\
528.01	0.01\\
529.01	0.01\\
530.01	0.01\\
531.01	0.01\\
532.01	0.01\\
533.01	0.01\\
534.01	0.01\\
535.01	0.01\\
536.01	0.01\\
537.01	0.01\\
538.01	0.01\\
539.01	0.01\\
540.01	0.01\\
541.01	0.01\\
542.01	0.01\\
543.01	0.01\\
544.01	0.01\\
545.01	0.01\\
546.01	0.01\\
547.01	0.01\\
548.01	0.01\\
549.01	0.01\\
550.01	0.01\\
551.01	0.01\\
552.01	0.01\\
553.01	0.01\\
554.01	0.01\\
555.01	0.01\\
556.01	0.01\\
557.01	0.01\\
558.01	0.01\\
559.01	0.01\\
560.01	0.01\\
561.01	0.01\\
562.01	0.01\\
563.01	0.01\\
564.01	0.01\\
565.01	0.01\\
566.01	0.01\\
567.01	0.01\\
568.01	0.01\\
569.01	0.01\\
570.01	0.01\\
571.01	0.01\\
572.01	0.01\\
573.01	0.01\\
574.01	0.01\\
575.01	0.01\\
576.01	0.01\\
577.01	0.01\\
578.01	0.01\\
579.01	0.01\\
580.01	0.01\\
581.01	0.01\\
582.01	0.01\\
583.01	0.01\\
584.01	0.01\\
585.01	0.01\\
586.01	0.01\\
587.01	0.01\\
588.01	0.01\\
589.01	0.01\\
590.01	0.01\\
591.01	0.01\\
592.01	0.01\\
593.01	0.01\\
594.01	0.01\\
595.01	0.01\\
596.01	0.01\\
597.01	0.01\\
598.01	0.00865127650878676\\
599.01	0.0062418690940541\\
599.02	0.00620414441242163\\
599.03	0.00616605275622739\\
599.04	0.00612759052101571\\
599.05	0.00608875406692037\\
599.06	0.00604953971831644\\
599.07	0.00600994376346918\\
599.08	0.0059699624541792\\
599.09	0.00592959200542428\\
599.1	0.00588882859499753\\
599.11	0.00584766836314222\\
599.12	0.00580610741218276\\
599.13	0.00576414180615225\\
599.14	0.00572176757041628\\
599.15	0.00567898069129307\\
599.16	0.00563577711566984\\
599.17	0.00559215275061546\\
599.18	0.00554810346298927\\
599.19	0.00550362507904601\\
599.2	0.00545871338403702\\
599.21	0.00541336412180731\\
599.22	0.00536757299438883\\
599.23	0.00532133566158969\\
599.24	0.00527464774057936\\
599.25	0.00522750480546979\\
599.26	0.00517990238689235\\
599.27	0.00513183597157071\\
599.28	0.00508330100188955\\
599.29	0.00503429287545888\\
599.3	0.0049848069446742\\
599.31	0.00493483851627232\\
599.32	0.00488438285088278\\
599.33	0.00483343516257493\\
599.34	0.00478199061840049\\
599.35	0.00473004433793164\\
599.36	0.00467759139279458\\
599.37	0.00462462680619853\\
599.38	0.00457114555246006\\
599.39	0.00451714256049738\\
599.4	0.0044626127237636\\
599.41	0.00440755088572623\\
599.42	0.00435195183937786\\
599.43	0.00429581032674215\\
599.44	0.00423912103837491\\
599.45	0.00418187861286026\\
599.46	0.00412407763630198\\
599.47	0.00406571264180975\\
599.48	0.00400677810898049\\
599.49	0.00394726846337446\\
599.5	0.00388717807598643\\
599.51	0.00382650126271148\\
599.52	0.00376523228380572\\
599.53	0.00370336534334166\\
599.54	0.00364089458865823\\
599.55	0.00357781410980553\\
599.56	0.00351411793898395\\
599.57	0.00344980004997801\\
599.58	0.00338485435758451\\
599.59	0.00331927471703512\\
599.6	0.00325305492341342\\
599.61	0.00318618871106606\\
599.62	0.00311866975300824\\
599.63	0.00305049166032345\\
599.64	0.00298164798155718\\
599.65	0.00291213220210484\\
599.66	0.00284193774359354\\
599.67	0.002771057963258\\
599.68	0.00269948615331015\\
599.69	0.00262721554030276\\
599.7	0.00255423928448666\\
599.71	0.00248055047916181\\
599.72	0.00240614215002192\\
599.73	0.00233100725449273\\
599.74	0.00225513868106373\\
599.75	0.00217852924861344\\
599.76	0.002101171705728\\
599.77	0.00202305873001311\\
599.78	0.00194418292739929\\
599.79	0.00186453683144024\\
599.8	0.00178411290260446\\
599.81	0.0017029035275598\\
599.82	0.0016209010184511\\
599.83	0.00153809761217073\\
599.84	0.001454485469622\\
599.85	0.00137005667497534\\
599.86	0.00128480323491721\\
599.87	0.00119871707789167\\
599.88	0.00111179005333449\\
599.89	0.00102401393089983\\
599.9	0.000935380399679277\\
599.91	0.00084588106741329\\
599.92	0.00075550745969488\\
599.93	0.000664251019165563\\
599.94	0.00057210310470336\\
599.95	0.000479054990602914\\
599.96	0.000385097865747556\\
599.97	0.000290222832773275\\
599.98	0.000194420907224489\\
599.99	9.76830167015632e-05\\
600	0\\
};
\addplot [color=black,solid,forget plot]
  table[row sep=crcr]{%
0.01	0.01\\
1.01	0.01\\
2.01	0.01\\
3.01	0.01\\
4.01	0.01\\
5.01	0.01\\
6.01	0.01\\
7.01	0.01\\
8.01	0.01\\
9.01	0.01\\
10.01	0.01\\
11.01	0.01\\
12.01	0.01\\
13.01	0.01\\
14.01	0.01\\
15.01	0.01\\
16.01	0.01\\
17.01	0.01\\
18.01	0.01\\
19.01	0.01\\
20.01	0.01\\
21.01	0.01\\
22.01	0.01\\
23.01	0.01\\
24.01	0.01\\
25.01	0.01\\
26.01	0.01\\
27.01	0.01\\
28.01	0.01\\
29.01	0.01\\
30.01	0.01\\
31.01	0.01\\
32.01	0.01\\
33.01	0.01\\
34.01	0.01\\
35.01	0.01\\
36.01	0.01\\
37.01	0.01\\
38.01	0.01\\
39.01	0.01\\
40.01	0.01\\
41.01	0.01\\
42.01	0.01\\
43.01	0.01\\
44.01	0.01\\
45.01	0.01\\
46.01	0.01\\
47.01	0.01\\
48.01	0.01\\
49.01	0.01\\
50.01	0.01\\
51.01	0.01\\
52.01	0.01\\
53.01	0.01\\
54.01	0.01\\
55.01	0.01\\
56.01	0.01\\
57.01	0.01\\
58.01	0.01\\
59.01	0.01\\
60.01	0.01\\
61.01	0.01\\
62.01	0.01\\
63.01	0.01\\
64.01	0.01\\
65.01	0.01\\
66.01	0.01\\
67.01	0.01\\
68.01	0.01\\
69.01	0.01\\
70.01	0.01\\
71.01	0.01\\
72.01	0.01\\
73.01	0.01\\
74.01	0.01\\
75.01	0.01\\
76.01	0.01\\
77.01	0.01\\
78.01	0.01\\
79.01	0.01\\
80.01	0.01\\
81.01	0.01\\
82.01	0.01\\
83.01	0.01\\
84.01	0.01\\
85.01	0.01\\
86.01	0.01\\
87.01	0.01\\
88.01	0.01\\
89.01	0.01\\
90.01	0.01\\
91.01	0.01\\
92.01	0.01\\
93.01	0.01\\
94.01	0.01\\
95.01	0.01\\
96.01	0.01\\
97.01	0.01\\
98.01	0.01\\
99.01	0.01\\
100.01	0.01\\
101.01	0.01\\
102.01	0.01\\
103.01	0.01\\
104.01	0.01\\
105.01	0.01\\
106.01	0.01\\
107.01	0.01\\
108.01	0.01\\
109.01	0.01\\
110.01	0.01\\
111.01	0.01\\
112.01	0.01\\
113.01	0.01\\
114.01	0.01\\
115.01	0.01\\
116.01	0.01\\
117.01	0.01\\
118.01	0.01\\
119.01	0.01\\
120.01	0.01\\
121.01	0.01\\
122.01	0.01\\
123.01	0.01\\
124.01	0.01\\
125.01	0.01\\
126.01	0.01\\
127.01	0.01\\
128.01	0.01\\
129.01	0.01\\
130.01	0.01\\
131.01	0.01\\
132.01	0.01\\
133.01	0.01\\
134.01	0.01\\
135.01	0.01\\
136.01	0.01\\
137.01	0.01\\
138.01	0.01\\
139.01	0.01\\
140.01	0.01\\
141.01	0.01\\
142.01	0.01\\
143.01	0.01\\
144.01	0.01\\
145.01	0.01\\
146.01	0.01\\
147.01	0.01\\
148.01	0.01\\
149.01	0.01\\
150.01	0.01\\
151.01	0.01\\
152.01	0.01\\
153.01	0.01\\
154.01	0.01\\
155.01	0.01\\
156.01	0.01\\
157.01	0.01\\
158.01	0.01\\
159.01	0.01\\
160.01	0.01\\
161.01	0.01\\
162.01	0.01\\
163.01	0.01\\
164.01	0.01\\
165.01	0.01\\
166.01	0.01\\
167.01	0.01\\
168.01	0.01\\
169.01	0.01\\
170.01	0.01\\
171.01	0.01\\
172.01	0.01\\
173.01	0.01\\
174.01	0.01\\
175.01	0.01\\
176.01	0.01\\
177.01	0.01\\
178.01	0.01\\
179.01	0.01\\
180.01	0.01\\
181.01	0.01\\
182.01	0.01\\
183.01	0.01\\
184.01	0.01\\
185.01	0.01\\
186.01	0.01\\
187.01	0.01\\
188.01	0.01\\
189.01	0.01\\
190.01	0.01\\
191.01	0.01\\
192.01	0.01\\
193.01	0.01\\
194.01	0.01\\
195.01	0.01\\
196.01	0.01\\
197.01	0.01\\
198.01	0.01\\
199.01	0.01\\
200.01	0.01\\
201.01	0.01\\
202.01	0.01\\
203.01	0.01\\
204.01	0.01\\
205.01	0.01\\
206.01	0.01\\
207.01	0.01\\
208.01	0.01\\
209.01	0.01\\
210.01	0.01\\
211.01	0.01\\
212.01	0.01\\
213.01	0.01\\
214.01	0.01\\
215.01	0.01\\
216.01	0.01\\
217.01	0.01\\
218.01	0.01\\
219.01	0.01\\
220.01	0.01\\
221.01	0.01\\
222.01	0.01\\
223.01	0.01\\
224.01	0.01\\
225.01	0.01\\
226.01	0.01\\
227.01	0.01\\
228.01	0.01\\
229.01	0.01\\
230.01	0.01\\
231.01	0.01\\
232.01	0.01\\
233.01	0.01\\
234.01	0.01\\
235.01	0.01\\
236.01	0.01\\
237.01	0.01\\
238.01	0.01\\
239.01	0.01\\
240.01	0.01\\
241.01	0.01\\
242.01	0.01\\
243.01	0.01\\
244.01	0.01\\
245.01	0.01\\
246.01	0.01\\
247.01	0.01\\
248.01	0.01\\
249.01	0.01\\
250.01	0.01\\
251.01	0.01\\
252.01	0.01\\
253.01	0.01\\
254.01	0.01\\
255.01	0.01\\
256.01	0.01\\
257.01	0.01\\
258.01	0.01\\
259.01	0.01\\
260.01	0.01\\
261.01	0.01\\
262.01	0.01\\
263.01	0.01\\
264.01	0.01\\
265.01	0.01\\
266.01	0.01\\
267.01	0.01\\
268.01	0.01\\
269.01	0.01\\
270.01	0.01\\
271.01	0.01\\
272.01	0.01\\
273.01	0.01\\
274.01	0.01\\
275.01	0.01\\
276.01	0.01\\
277.01	0.01\\
278.01	0.01\\
279.01	0.01\\
280.01	0.01\\
281.01	0.01\\
282.01	0.01\\
283.01	0.01\\
284.01	0.01\\
285.01	0.01\\
286.01	0.01\\
287.01	0.01\\
288.01	0.01\\
289.01	0.01\\
290.01	0.01\\
291.01	0.01\\
292.01	0.01\\
293.01	0.01\\
294.01	0.01\\
295.01	0.01\\
296.01	0.01\\
297.01	0.01\\
298.01	0.01\\
299.01	0.01\\
300.01	0.01\\
301.01	0.01\\
302.01	0.01\\
303.01	0.01\\
304.01	0.01\\
305.01	0.01\\
306.01	0.01\\
307.01	0.01\\
308.01	0.01\\
309.01	0.01\\
310.01	0.01\\
311.01	0.01\\
312.01	0.01\\
313.01	0.01\\
314.01	0.01\\
315.01	0.01\\
316.01	0.01\\
317.01	0.01\\
318.01	0.01\\
319.01	0.01\\
320.01	0.01\\
321.01	0.01\\
322.01	0.01\\
323.01	0.01\\
324.01	0.01\\
325.01	0.01\\
326.01	0.01\\
327.01	0.01\\
328.01	0.01\\
329.01	0.01\\
330.01	0.01\\
331.01	0.01\\
332.01	0.01\\
333.01	0.01\\
334.01	0.01\\
335.01	0.01\\
336.01	0.01\\
337.01	0.01\\
338.01	0.01\\
339.01	0.01\\
340.01	0.01\\
341.01	0.01\\
342.01	0.01\\
343.01	0.01\\
344.01	0.01\\
345.01	0.01\\
346.01	0.01\\
347.01	0.01\\
348.01	0.01\\
349.01	0.01\\
350.01	0.01\\
351.01	0.01\\
352.01	0.01\\
353.01	0.01\\
354.01	0.01\\
355.01	0.01\\
356.01	0.01\\
357.01	0.01\\
358.01	0.01\\
359.01	0.01\\
360.01	0.01\\
361.01	0.01\\
362.01	0.01\\
363.01	0.01\\
364.01	0.01\\
365.01	0.01\\
366.01	0.01\\
367.01	0.01\\
368.01	0.01\\
369.01	0.01\\
370.01	0.01\\
371.01	0.01\\
372.01	0.01\\
373.01	0.01\\
374.01	0.01\\
375.01	0.01\\
376.01	0.01\\
377.01	0.01\\
378.01	0.01\\
379.01	0.01\\
380.01	0.01\\
381.01	0.01\\
382.01	0.01\\
383.01	0.01\\
384.01	0.01\\
385.01	0.01\\
386.01	0.01\\
387.01	0.01\\
388.01	0.01\\
389.01	0.01\\
390.01	0.01\\
391.01	0.01\\
392.01	0.01\\
393.01	0.01\\
394.01	0.01\\
395.01	0.01\\
396.01	0.01\\
397.01	0.01\\
398.01	0.01\\
399.01	0.01\\
400.01	0.01\\
401.01	0.01\\
402.01	0.01\\
403.01	0.01\\
404.01	0.01\\
405.01	0.01\\
406.01	0.01\\
407.01	0.01\\
408.01	0.01\\
409.01	0.01\\
410.01	0.01\\
411.01	0.01\\
412.01	0.01\\
413.01	0.01\\
414.01	0.01\\
415.01	0.01\\
416.01	0.01\\
417.01	0.01\\
418.01	0.01\\
419.01	0.01\\
420.01	0.01\\
421.01	0.01\\
422.01	0.01\\
423.01	0.01\\
424.01	0.01\\
425.01	0.01\\
426.01	0.01\\
427.01	0.01\\
428.01	0.01\\
429.01	0.01\\
430.01	0.01\\
431.01	0.01\\
432.01	0.01\\
433.01	0.01\\
434.01	0.01\\
435.01	0.01\\
436.01	0.01\\
437.01	0.01\\
438.01	0.01\\
439.01	0.01\\
440.01	0.01\\
441.01	0.01\\
442.01	0.01\\
443.01	0.01\\
444.01	0.01\\
445.01	0.01\\
446.01	0.01\\
447.01	0.01\\
448.01	0.01\\
449.01	0.01\\
450.01	0.01\\
451.01	0.01\\
452.01	0.01\\
453.01	0.01\\
454.01	0.01\\
455.01	0.01\\
456.01	0.01\\
457.01	0.01\\
458.01	0.01\\
459.01	0.01\\
460.01	0.01\\
461.01	0.01\\
462.01	0.01\\
463.01	0.01\\
464.01	0.01\\
465.01	0.01\\
466.01	0.01\\
467.01	0.01\\
468.01	0.01\\
469.01	0.01\\
470.01	0.01\\
471.01	0.01\\
472.01	0.01\\
473.01	0.01\\
474.01	0.01\\
475.01	0.01\\
476.01	0.01\\
477.01	0.01\\
478.01	0.01\\
479.01	0.01\\
480.01	0.01\\
481.01	0.01\\
482.01	0.01\\
483.01	0.01\\
484.01	0.01\\
485.01	0.01\\
486.01	0.01\\
487.01	0.01\\
488.01	0.01\\
489.01	0.01\\
490.01	0.01\\
491.01	0.01\\
492.01	0.01\\
493.01	0.01\\
494.01	0.01\\
495.01	0.01\\
496.01	0.01\\
497.01	0.01\\
498.01	0.01\\
499.01	0.01\\
500.01	0.01\\
501.01	0.01\\
502.01	0.01\\
503.01	0.01\\
504.01	0.01\\
505.01	0.01\\
506.01	0.01\\
507.01	0.01\\
508.01	0.01\\
509.01	0.01\\
510.01	0.01\\
511.01	0.01\\
512.01	0.01\\
513.01	0.01\\
514.01	0.01\\
515.01	0.01\\
516.01	0.01\\
517.01	0.01\\
518.01	0.01\\
519.01	0.01\\
520.01	0.01\\
521.01	0.01\\
522.01	0.01\\
523.01	0.01\\
524.01	0.01\\
525.01	0.01\\
526.01	0.01\\
527.01	0.01\\
528.01	0.01\\
529.01	0.01\\
530.01	0.01\\
531.01	0.01\\
532.01	0.01\\
533.01	0.01\\
534.01	0.01\\
535.01	0.01\\
536.01	0.01\\
537.01	0.01\\
538.01	0.01\\
539.01	0.01\\
540.01	0.01\\
541.01	0.01\\
542.01	0.01\\
543.01	0.01\\
544.01	0.01\\
545.01	0.01\\
546.01	0.01\\
547.01	0.01\\
548.01	0.01\\
549.01	0.01\\
550.01	0.01\\
551.01	0.01\\
552.01	0.01\\
553.01	0.01\\
554.01	0.01\\
555.01	0.01\\
556.01	0.01\\
557.01	0.01\\
558.01	0.01\\
559.01	0.01\\
560.01	0.01\\
561.01	0.01\\
562.01	0.01\\
563.01	0.01\\
564.01	0.01\\
565.01	0.01\\
566.01	0.01\\
567.01	0.01\\
568.01	0.01\\
569.01	0.01\\
570.01	0.01\\
571.01	0.01\\
572.01	0.01\\
573.01	0.01\\
574.01	0.01\\
575.01	0.01\\
576.01	0.01\\
577.01	0.01\\
578.01	0.01\\
579.01	0.01\\
580.01	0.01\\
581.01	0.01\\
582.01	0.01\\
583.01	0.01\\
584.01	0.01\\
585.01	0.01\\
586.01	0.01\\
587.01	0.01\\
588.01	0.01\\
589.01	0.01\\
590.01	0.01\\
591.01	0.01\\
592.01	0.01\\
593.01	0.01\\
594.01	0.01\\
595.01	0.01\\
596.01	0.01\\
597.01	0.01\\
598.01	0.00865106781478654\\
599.01	0.00624186909405412\\
599.02	0.00620414441242164\\
599.03	0.00616605275622737\\
599.04	0.00612759052101571\\
599.05	0.00608875406692037\\
599.06	0.00604953971831645\\
599.07	0.00600994376346916\\
599.08	0.00596996245417918\\
599.09	0.00592959200542426\\
599.1	0.0058888285949975\\
599.11	0.00584766836314221\\
599.12	0.00580610741218275\\
599.13	0.00576414180615224\\
599.14	0.00572176757041627\\
599.15	0.00567898069129306\\
599.16	0.00563577711566984\\
599.17	0.00559215275061547\\
599.18	0.00554810346298925\\
599.19	0.00550362507904601\\
599.2	0.00545871338403702\\
599.21	0.0054133641218073\\
599.22	0.0053675729943888\\
599.23	0.00532133566158966\\
599.24	0.00527464774057934\\
599.25	0.00522750480546975\\
599.26	0.00517990238689229\\
599.27	0.00513183597157067\\
599.28	0.00508330100188951\\
599.29	0.00503429287545885\\
599.3	0.00498480694467416\\
599.31	0.00493483851627228\\
599.32	0.00488438285088275\\
599.33	0.00483343516257492\\
599.34	0.00478199061840046\\
599.35	0.0047300443379316\\
599.36	0.00467759139279454\\
599.37	0.00462462680619848\\
599.38	0.00457114555246002\\
599.39	0.00451714256049734\\
599.4	0.00446261272376355\\
599.41	0.00440755088572616\\
599.42	0.00435195183937779\\
599.43	0.00429581032674208\\
599.44	0.00423912103837484\\
599.45	0.00418187861286019\\
599.46	0.00412407763630191\\
599.47	0.0040657126418097\\
599.48	0.00400677810898042\\
599.49	0.00394726846337441\\
599.5	0.00388717807598638\\
599.51	0.00382650126271143\\
599.52	0.00376523228380568\\
599.53	0.00370336534334161\\
599.54	0.00364089458865821\\
599.55	0.00357781410980551\\
599.56	0.00351411793898394\\
599.57	0.00344980004997801\\
599.58	0.00338485435758449\\
599.59	0.00331927471703512\\
599.6	0.00325305492341342\\
599.61	0.00318618871106605\\
599.62	0.00311866975300823\\
599.63	0.00305049166032344\\
599.64	0.00298164798155719\\
599.65	0.00291213220210483\\
599.66	0.00284193774359355\\
599.67	0.002771057963258\\
599.68	0.00269948615331015\\
599.69	0.00262721554030275\\
599.7	0.00255423928448666\\
599.71	0.00248055047916181\\
599.72	0.00240614215002193\\
599.73	0.00233100725449273\\
599.74	0.00225513868106373\\
599.75	0.00217852924861344\\
599.76	0.002101171705728\\
599.77	0.00202305873001311\\
599.78	0.00194418292739928\\
599.79	0.00186453683144024\\
599.8	0.00178411290260446\\
599.81	0.00170290352755979\\
599.82	0.00162090101845109\\
599.83	0.00153809761217073\\
599.84	0.001454485469622\\
599.85	0.00137005667497534\\
599.86	0.00128480323491721\\
599.87	0.00119871707789166\\
599.88	0.00111179005333448\\
599.89	0.00102401393089982\\
599.9	0.000935380399679277\\
599.91	0.000845881067413285\\
599.92	0.00075550745969488\\
599.93	0.000664251019165563\\
599.94	0.000572103104703356\\
599.95	0.000479054990602914\\
599.96	0.000385097865747554\\
599.97	0.000290222832773275\\
599.98	0.000194420907224489\\
599.99	9.76830167015632e-05\\
600	0\\
};
\end{axis}
\end{tikzpicture}% 
  \caption{Continuous Time w/ nFPC}
\end{subfigure}%
\hfill%
\begin{subfigure}{.45\linewidth}
  \centering
  \setlength\figureheight{\linewidth} 
  \setlength\figurewidth{\linewidth}
  \tikzsetnextfilename{dp_colorbar/dm_dscr_nFPC_z1}
  % This file was created by matlab2tikz.
%
%The latest updates can be retrieved from
%  http://www.mathworks.com/matlabcentral/fileexchange/22022-matlab2tikz-matlab2tikz
%where you can also make suggestions and rate matlab2tikz.
%
\definecolor{mycolor1}{rgb}{0.00000,1.00000,0.14286}%
\definecolor{mycolor2}{rgb}{0.00000,1.00000,0.28571}%
\definecolor{mycolor3}{rgb}{0.00000,1.00000,0.42857}%
\definecolor{mycolor4}{rgb}{0.00000,1.00000,0.57143}%
\definecolor{mycolor5}{rgb}{0.00000,1.00000,0.71429}%
\definecolor{mycolor6}{rgb}{0.00000,1.00000,0.85714}%
\definecolor{mycolor7}{rgb}{0.00000,1.00000,1.00000}%
\definecolor{mycolor8}{rgb}{0.00000,0.87500,1.00000}%
\definecolor{mycolor9}{rgb}{0.00000,0.62500,1.00000}%
\definecolor{mycolor10}{rgb}{0.12500,0.00000,1.00000}%
\definecolor{mycolor11}{rgb}{0.25000,0.00000,1.00000}%
\definecolor{mycolor12}{rgb}{0.37500,0.00000,1.00000}%
\definecolor{mycolor13}{rgb}{0.50000,0.00000,1.00000}%
\definecolor{mycolor14}{rgb}{0.62500,0.00000,1.00000}%
\definecolor{mycolor15}{rgb}{0.75000,0.00000,1.00000}%
\definecolor{mycolor16}{rgb}{0.87500,0.00000,1.00000}%
\definecolor{mycolor17}{rgb}{1.00000,0.00000,1.00000}%
\definecolor{mycolor18}{rgb}{1.00000,0.00000,0.87500}%
\definecolor{mycolor19}{rgb}{1.00000,0.00000,0.62500}%
\definecolor{mycolor20}{rgb}{0.85714,0.00000,0.00000}%
\definecolor{mycolor21}{rgb}{0.71429,0.00000,0.00000}%
%
\begin{tikzpicture}

\begin{axis}[%
width=4.1in,
height=3.803in,
at={(0.809in,0.513in)},
scale only axis,
point meta min=0,
point meta max=1,
every outer x axis line/.append style={black},
every x tick label/.append style={font=\color{black}},
xmin=0,
xmax=600,
every outer y axis line/.append style={black},
every y tick label/.append style={font=\color{black}},
ymin=0,
ymax=0.007,
axis background/.style={fill=white},
axis x line*=bottom,
axis y line*=left,
colormap={mymap}{[1pt] rgb(0pt)=(0,1,0); rgb(7pt)=(0,1,1); rgb(15pt)=(0,0,1); rgb(23pt)=(1,0,1); rgb(31pt)=(1,0,0); rgb(38pt)=(0,0,0)},
colorbar,
colorbar style={separate axis lines,every outer x axis line/.append style={black},every x tick label/.append style={font=\color{black}},every outer y axis line/.append style={black},every y tick label/.append style={font=\color{black}},yticklabels={{-19},{-17},{-15},{-13},{-11},{-9},{-7},{-5},{-3},{-1},{1},{3},{5},{7},{9},{11},{13},{15},{17},{19}}}
]
\addplot [color=green,solid,forget plot]
  table[row sep=crcr]{%
1	0\\
2	0\\
3	0\\
4	0\\
5	0\\
6	0\\
7	0\\
8	0\\
9	0\\
10	0\\
11	0\\
12	0\\
13	0\\
14	0\\
15	0\\
16	0\\
17	0\\
18	0\\
19	0\\
20	0\\
21	0\\
22	0\\
23	0\\
24	0\\
25	0\\
26	0\\
27	0\\
28	0\\
29	0\\
30	0\\
31	0\\
32	0\\
33	0\\
34	0\\
35	0\\
36	0\\
37	0\\
38	0\\
39	0\\
40	0\\
41	0\\
42	0\\
43	0\\
44	0\\
45	0\\
46	0\\
47	0\\
48	0\\
49	0\\
50	0\\
51	0\\
52	0\\
53	0\\
54	0\\
55	0\\
56	0\\
57	0\\
58	0\\
59	0\\
60	0\\
61	0\\
62	0\\
63	0\\
64	0\\
65	0\\
66	0\\
67	0\\
68	0\\
69	0\\
70	0\\
71	0\\
72	0\\
73	0\\
74	0\\
75	0\\
76	0\\
77	0\\
78	0\\
79	0\\
80	0\\
81	0\\
82	0\\
83	0\\
84	0\\
85	0\\
86	0\\
87	0\\
88	0\\
89	0\\
90	0\\
91	0\\
92	0\\
93	0\\
94	0\\
95	0\\
96	0\\
97	0\\
98	0\\
99	0\\
100	0\\
101	0\\
102	0\\
103	0\\
104	0\\
105	0\\
106	0\\
107	0\\
108	0\\
109	0\\
110	0\\
111	0\\
112	0\\
113	0\\
114	0\\
115	0\\
116	0\\
117	0\\
118	0\\
119	0\\
120	0\\
121	0\\
122	0\\
123	0\\
124	0\\
125	0\\
126	0\\
127	0\\
128	0\\
129	0\\
130	0\\
131	0\\
132	0\\
133	0\\
134	0\\
135	0\\
136	0\\
137	0\\
138	0\\
139	0\\
140	0\\
141	0\\
142	0\\
143	0\\
144	0\\
145	0\\
146	0\\
147	0\\
148	0\\
149	0\\
150	0\\
151	0\\
152	0\\
153	0\\
154	0\\
155	0\\
156	0\\
157	0\\
158	0\\
159	0\\
160	0\\
161	0\\
162	0\\
163	0\\
164	0\\
165	0\\
166	0\\
167	0\\
168	0\\
169	0\\
170	0\\
171	0\\
172	0\\
173	0\\
174	0\\
175	0\\
176	0\\
177	0\\
178	0\\
179	0\\
180	0\\
181	0\\
182	0\\
183	0\\
184	0\\
185	0\\
186	0\\
187	0\\
188	0\\
189	0\\
190	0\\
191	0\\
192	0\\
193	0\\
194	0\\
195	0\\
196	0\\
197	0\\
198	0\\
199	0\\
200	0\\
201	0\\
202	0\\
203	0\\
204	0\\
205	0\\
206	0\\
207	0\\
208	0\\
209	0\\
210	0\\
211	0\\
212	0\\
213	0\\
214	0\\
215	0\\
216	0\\
217	0\\
218	0\\
219	0\\
220	0\\
221	0\\
222	0\\
223	0\\
224	0\\
225	0\\
226	0\\
227	0\\
228	0\\
229	0\\
230	0\\
231	0\\
232	0\\
233	0\\
234	0\\
235	0\\
236	0\\
237	0\\
238	0\\
239	0\\
240	0\\
241	0\\
242	0\\
243	0\\
244	0\\
245	0\\
246	0\\
247	0\\
248	0\\
249	0\\
250	0\\
251	0\\
252	0\\
253	0\\
254	0\\
255	0\\
256	0\\
257	0\\
258	0\\
259	0\\
260	0\\
261	0\\
262	0\\
263	0\\
264	0\\
265	0\\
266	0\\
267	0\\
268	0\\
269	0\\
270	0\\
271	0\\
272	0\\
273	0\\
274	0\\
275	0\\
276	0\\
277	0\\
278	0\\
279	0\\
280	0\\
281	0\\
282	0\\
283	0\\
284	0\\
285	0\\
286	0\\
287	0\\
288	0\\
289	0\\
290	0\\
291	0\\
292	0\\
293	0\\
294	0\\
295	0\\
296	0\\
297	0\\
298	0\\
299	0\\
300	0\\
301	0\\
302	0\\
303	0\\
304	0\\
305	0\\
306	0\\
307	0\\
308	0\\
309	0\\
310	0\\
311	0\\
312	0\\
313	0\\
314	0\\
315	0\\
316	0\\
317	0\\
318	0\\
319	0\\
320	0\\
321	0\\
322	0\\
323	0\\
324	0\\
325	0\\
326	0\\
327	0\\
328	0\\
329	0\\
330	0\\
331	0\\
332	0\\
333	0\\
334	0\\
335	0\\
336	0\\
337	0\\
338	0\\
339	0\\
340	0\\
341	0\\
342	0\\
343	0\\
344	0\\
345	0\\
346	0\\
347	0\\
348	0\\
349	0\\
350	0\\
351	0\\
352	0\\
353	0\\
354	0\\
355	0\\
356	0\\
357	0\\
358	0\\
359	0\\
360	0\\
361	0\\
362	0\\
363	0\\
364	0\\
365	0\\
366	0\\
367	0\\
368	0\\
369	0\\
370	0\\
371	0\\
372	0\\
373	0\\
374	0\\
375	0\\
376	0\\
377	0\\
378	0\\
379	0\\
380	0\\
381	0\\
382	0\\
383	0\\
384	0\\
385	0\\
386	0\\
387	0\\
388	0\\
389	0\\
390	0\\
391	0\\
392	0\\
393	0\\
394	0\\
395	0\\
396	0\\
397	0\\
398	0\\
399	0\\
400	0\\
401	0\\
402	0\\
403	0\\
404	0\\
405	0\\
406	0\\
407	0\\
408	0\\
409	0\\
410	0\\
411	0\\
412	0\\
413	0\\
414	0\\
415	0\\
416	0\\
417	0\\
418	0\\
419	0\\
420	0\\
421	0\\
422	0\\
423	0\\
424	0\\
425	0\\
426	0\\
427	0\\
428	0\\
429	0\\
430	0\\
431	0\\
432	0\\
433	0\\
434	0\\
435	0\\
436	0\\
437	0\\
438	0\\
439	0\\
440	0\\
441	0\\
442	0\\
443	0\\
444	0\\
445	0\\
446	0\\
447	0\\
448	0\\
449	0\\
450	0\\
451	0\\
452	0\\
453	0\\
454	0\\
455	0\\
456	0\\
457	0\\
458	0\\
459	0\\
460	0\\
461	0\\
462	0\\
463	0\\
464	0\\
465	0\\
466	0\\
467	0\\
468	0\\
469	0\\
470	0\\
471	0\\
472	0\\
473	0\\
474	0\\
475	0\\
476	0\\
477	0\\
478	0\\
479	0\\
480	0\\
481	0\\
482	0\\
483	0\\
484	0\\
485	0\\
486	0\\
487	0\\
488	0\\
489	0\\
490	0\\
491	0\\
492	0\\
493	0\\
494	0\\
495	0\\
496	0\\
497	0\\
498	0\\
499	0\\
500	0\\
501	0\\
502	0\\
503	0\\
504	0\\
505	0\\
506	0\\
507	0\\
508	0\\
509	0\\
510	0\\
511	0\\
512	0\\
513	0\\
514	0\\
515	0\\
516	0\\
517	0\\
518	0\\
519	0\\
520	0\\
521	0\\
522	0\\
523	0\\
524	0\\
525	0\\
526	0\\
527	0\\
528	0\\
529	0\\
530	0\\
531	0\\
532	0\\
533	0\\
534	0\\
535	0\\
536	0\\
537	0\\
538	0\\
539	1.95527641225376e-05\\
540	4.5298850109529e-05\\
541	7.16515787047177e-05\\
542	9.87260037163583e-05\\
543	0.000126597716204775\\
544	0.000154828213300624\\
545	0.000183516284188123\\
546	0.000212652451694723\\
547	0.000242445887004352\\
548	0.000272977208656028\\
549	0.000304273916678095\\
550	0.000336360067194731\\
551	0.000369260162993882\\
552	0.000402999655010358\\
553	0.000437605098067932\\
554	0.000473104294736734\\
555	0.000509526629431762\\
556	0.000547836300656453\\
557	0.00058668916353381\\
558	0.00062520296878696\\
559	0.000663720153911229\\
560	0.00070323559382204\\
561	0.000743809025123238\\
562	0.000785486013167821\\
563	0.000865379867363339\\
564	0.00126850553815116\\
565	0.00149693684286139\\
566	0.00155900187926996\\
567	0.00162212034359134\\
568	0.00168634680537145\\
569	0.00175171104531127\\
570	0.00181824424140386\\
571	0.00188597918426003\\
572	0.00195495039828782\\
573	0.00202519426953807\\
574	0.0020967491834911\\
575	0.00216965567387592\\
576	0.00224395658348052\\
577	0.00231969723802151\\
578	0.0023969256343423\\
579	0.00247569264466642\\
580	0.00255605223972909\\
581	0.00263806173637132\\
582	0.00272178208245877\\
583	0.00280727821124775\\
584	0.00289461954867252\\
585	0.00298388089415187\\
586	0.00307514426197734\\
587	0.00316850325005475\\
588	0.00326407412228819\\
589	0.00336202479534929\\
590	0.00346265165082637\\
591	0.00356658418066895\\
592	0.00367533141522135\\
593	0.00379274227372929\\
594	0.0039289098819834\\
595	0.00411043972693672\\
596	0.00440815817174734\\
597	0.00501118703192877\\
598	0.00642488516645657\\
599	0\\
600	0\\
};
\addplot [color=mycolor1,solid,forget plot]
  table[row sep=crcr]{%
1	0\\
2	0\\
3	0\\
4	0\\
5	0\\
6	0\\
7	0\\
8	0\\
9	0\\
10	0\\
11	0\\
12	0\\
13	0\\
14	0\\
15	0\\
16	0\\
17	0\\
18	0\\
19	0\\
20	0\\
21	0\\
22	0\\
23	0\\
24	0\\
25	0\\
26	0\\
27	0\\
28	0\\
29	0\\
30	0\\
31	0\\
32	0\\
33	0\\
34	0\\
35	0\\
36	0\\
37	0\\
38	0\\
39	0\\
40	0\\
41	0\\
42	0\\
43	0\\
44	0\\
45	0\\
46	0\\
47	0\\
48	0\\
49	0\\
50	0\\
51	0\\
52	0\\
53	0\\
54	0\\
55	0\\
56	0\\
57	0\\
58	0\\
59	0\\
60	0\\
61	0\\
62	0\\
63	0\\
64	0\\
65	0\\
66	0\\
67	0\\
68	0\\
69	0\\
70	0\\
71	0\\
72	0\\
73	0\\
74	0\\
75	0\\
76	0\\
77	0\\
78	0\\
79	0\\
80	0\\
81	0\\
82	0\\
83	0\\
84	0\\
85	0\\
86	0\\
87	0\\
88	0\\
89	0\\
90	0\\
91	0\\
92	0\\
93	0\\
94	0\\
95	0\\
96	0\\
97	0\\
98	0\\
99	0\\
100	0\\
101	0\\
102	0\\
103	0\\
104	0\\
105	0\\
106	0\\
107	0\\
108	0\\
109	0\\
110	0\\
111	0\\
112	0\\
113	0\\
114	0\\
115	0\\
116	0\\
117	0\\
118	0\\
119	0\\
120	0\\
121	0\\
122	0\\
123	0\\
124	0\\
125	0\\
126	0\\
127	0\\
128	0\\
129	0\\
130	0\\
131	0\\
132	0\\
133	0\\
134	0\\
135	0\\
136	0\\
137	0\\
138	0\\
139	0\\
140	0\\
141	0\\
142	0\\
143	0\\
144	0\\
145	0\\
146	0\\
147	0\\
148	0\\
149	0\\
150	0\\
151	0\\
152	0\\
153	0\\
154	0\\
155	0\\
156	0\\
157	0\\
158	0\\
159	0\\
160	0\\
161	0\\
162	0\\
163	0\\
164	0\\
165	0\\
166	0\\
167	0\\
168	0\\
169	0\\
170	0\\
171	0\\
172	0\\
173	0\\
174	0\\
175	0\\
176	0\\
177	0\\
178	0\\
179	0\\
180	0\\
181	0\\
182	0\\
183	0\\
184	0\\
185	0\\
186	0\\
187	0\\
188	0\\
189	0\\
190	0\\
191	0\\
192	0\\
193	0\\
194	0\\
195	0\\
196	0\\
197	0\\
198	0\\
199	0\\
200	0\\
201	0\\
202	0\\
203	0\\
204	0\\
205	0\\
206	0\\
207	0\\
208	0\\
209	0\\
210	0\\
211	0\\
212	0\\
213	0\\
214	0\\
215	0\\
216	0\\
217	0\\
218	0\\
219	0\\
220	0\\
221	0\\
222	0\\
223	0\\
224	0\\
225	0\\
226	0\\
227	0\\
228	0\\
229	0\\
230	0\\
231	0\\
232	0\\
233	0\\
234	0\\
235	0\\
236	0\\
237	0\\
238	0\\
239	0\\
240	0\\
241	0\\
242	0\\
243	0\\
244	0\\
245	0\\
246	0\\
247	0\\
248	0\\
249	0\\
250	0\\
251	0\\
252	0\\
253	0\\
254	0\\
255	0\\
256	0\\
257	0\\
258	0\\
259	0\\
260	0\\
261	0\\
262	0\\
263	0\\
264	0\\
265	0\\
266	0\\
267	0\\
268	0\\
269	0\\
270	0\\
271	0\\
272	0\\
273	0\\
274	0\\
275	0\\
276	0\\
277	0\\
278	0\\
279	0\\
280	0\\
281	0\\
282	0\\
283	0\\
284	0\\
285	0\\
286	0\\
287	0\\
288	0\\
289	0\\
290	0\\
291	0\\
292	0\\
293	0\\
294	0\\
295	0\\
296	0\\
297	0\\
298	0\\
299	0\\
300	0\\
301	0\\
302	0\\
303	0\\
304	0\\
305	0\\
306	0\\
307	0\\
308	0\\
309	0\\
310	0\\
311	0\\
312	0\\
313	0\\
314	0\\
315	0\\
316	0\\
317	0\\
318	0\\
319	0\\
320	0\\
321	0\\
322	0\\
323	0\\
324	0\\
325	0\\
326	0\\
327	0\\
328	0\\
329	0\\
330	0\\
331	0\\
332	0\\
333	0\\
334	0\\
335	0\\
336	0\\
337	0\\
338	0\\
339	0\\
340	0\\
341	0\\
342	0\\
343	0\\
344	0\\
345	0\\
346	0\\
347	0\\
348	0\\
349	0\\
350	0\\
351	0\\
352	0\\
353	0\\
354	0\\
355	0\\
356	0\\
357	0\\
358	0\\
359	0\\
360	0\\
361	0\\
362	0\\
363	0\\
364	0\\
365	0\\
366	0\\
367	0\\
368	0\\
369	0\\
370	0\\
371	0\\
372	0\\
373	0\\
374	0\\
375	0\\
376	0\\
377	0\\
378	0\\
379	0\\
380	0\\
381	0\\
382	0\\
383	0\\
384	0\\
385	0\\
386	0\\
387	0\\
388	0\\
389	0\\
390	0\\
391	0\\
392	0\\
393	0\\
394	0\\
395	0\\
396	0\\
397	0\\
398	0\\
399	0\\
400	0\\
401	0\\
402	0\\
403	0\\
404	0\\
405	0\\
406	0\\
407	0\\
408	0\\
409	0\\
410	0\\
411	0\\
412	0\\
413	0\\
414	0\\
415	0\\
416	0\\
417	0\\
418	0\\
419	0\\
420	0\\
421	0\\
422	0\\
423	0\\
424	0\\
425	0\\
426	0\\
427	0\\
428	0\\
429	0\\
430	0\\
431	0\\
432	0\\
433	0\\
434	0\\
435	0\\
436	0\\
437	0\\
438	0\\
439	0\\
440	0\\
441	0\\
442	0\\
443	0\\
444	0\\
445	0\\
446	0\\
447	0\\
448	0\\
449	0\\
450	0\\
451	0\\
452	0\\
453	0\\
454	0\\
455	0\\
456	0\\
457	0\\
458	0\\
459	0\\
460	0\\
461	0\\
462	0\\
463	0\\
464	0\\
465	0\\
466	0\\
467	0\\
468	0\\
469	0\\
470	0\\
471	0\\
472	0\\
473	0\\
474	0\\
475	0\\
476	0\\
477	0\\
478	0\\
479	0\\
480	0\\
481	0\\
482	0\\
483	0\\
484	0\\
485	0\\
486	0\\
487	0\\
488	0\\
489	0\\
490	0\\
491	0\\
492	0\\
493	0\\
494	0\\
495	0\\
496	0\\
497	0\\
498	0\\
499	0\\
500	0\\
501	0\\
502	0\\
503	0\\
504	0\\
505	0\\
506	0\\
507	0\\
508	0\\
509	0\\
510	0\\
511	0\\
512	0\\
513	0\\
514	0\\
515	0\\
516	0\\
517	0\\
518	0\\
519	0\\
520	0\\
521	0\\
522	0\\
523	0\\
524	0\\
525	0\\
526	0\\
527	0\\
528	0\\
529	0\\
530	0\\
531	0\\
532	0\\
533	0\\
534	0\\
535	0\\
536	0\\
537	0\\
538	0\\
539	1.37376150958274e-05\\
540	3.93767248397528e-05\\
541	6.56239748089781e-05\\
542	9.24933624265191e-05\\
543	0.000119999855829102\\
544	0.000148363699355699\\
545	0.000177377641200535\\
546	0.000206774145788462\\
547	0.000236720771405114\\
548	0.000267152374965129\\
549	0.00029829523360433\\
550	0.000330215002525051\\
551	0.000362941523011504\\
552	0.000396501057709032\\
553	0.000430920114469899\\
554	0.000466226213023714\\
555	0.000502448117088533\\
556	0.000539615657443634\\
557	0.000578457004892515\\
558	0.000618368790079295\\
559	0.000658014352548393\\
560	0.0006974686375873\\
561	0.000737812075177008\\
562	0.000779241486122308\\
563	0.000821805097695177\\
564	0.000946087951490863\\
565	0.00132524351514214\\
566	0.00155876358354603\\
567	0.0016221192151821\\
568	0.00168634675322425\\
569	0.00175171103773517\\
570	0.00181824423981386\\
571	0.00188597918379922\\
572	0.00195495039810385\\
573	0.00202519426944808\\
574	0.0020967491834472\\
575	0.00216965567385279\\
576	0.00224395658346852\\
577	0.00231969723801546\\
578	0.00239692563433929\\
579	0.00247569264466521\\
580	0.00255605223972864\\
581	0.00263806173637133\\
582	0.00272178208245879\\
583	0.00280727821124775\\
584	0.00289461954867253\\
585	0.00298388089415188\\
586	0.00307514426197735\\
587	0.00316850325005475\\
588	0.00326407412228818\\
589	0.0033620247953493\\
590	0.00346265165082638\\
591	0.00356658418066896\\
592	0.00367533141522136\\
593	0.00379274227372929\\
594	0.00392890988198341\\
595	0.00411043972693672\\
596	0.00440815817174735\\
597	0.00501118703192878\\
598	0.00642488516645657\\
599	0\\
600	0\\
};
\addplot [color=mycolor2,solid,forget plot]
  table[row sep=crcr]{%
1	0\\
2	0\\
3	0\\
4	0\\
5	0\\
6	0\\
7	0\\
8	0\\
9	0\\
10	0\\
11	0\\
12	0\\
13	0\\
14	0\\
15	0\\
16	0\\
17	0\\
18	0\\
19	0\\
20	0\\
21	0\\
22	0\\
23	0\\
24	0\\
25	0\\
26	0\\
27	0\\
28	0\\
29	0\\
30	0\\
31	0\\
32	0\\
33	0\\
34	0\\
35	0\\
36	0\\
37	0\\
38	0\\
39	0\\
40	0\\
41	0\\
42	0\\
43	0\\
44	0\\
45	0\\
46	0\\
47	0\\
48	0\\
49	0\\
50	0\\
51	0\\
52	0\\
53	0\\
54	0\\
55	0\\
56	0\\
57	0\\
58	0\\
59	0\\
60	0\\
61	0\\
62	0\\
63	0\\
64	0\\
65	0\\
66	0\\
67	0\\
68	0\\
69	0\\
70	0\\
71	0\\
72	0\\
73	0\\
74	0\\
75	0\\
76	0\\
77	0\\
78	0\\
79	0\\
80	0\\
81	0\\
82	0\\
83	0\\
84	0\\
85	0\\
86	0\\
87	0\\
88	0\\
89	0\\
90	0\\
91	0\\
92	0\\
93	0\\
94	0\\
95	0\\
96	0\\
97	0\\
98	0\\
99	0\\
100	0\\
101	0\\
102	0\\
103	0\\
104	0\\
105	0\\
106	0\\
107	0\\
108	0\\
109	0\\
110	0\\
111	0\\
112	0\\
113	0\\
114	0\\
115	0\\
116	0\\
117	0\\
118	0\\
119	0\\
120	0\\
121	0\\
122	0\\
123	0\\
124	0\\
125	0\\
126	0\\
127	0\\
128	0\\
129	0\\
130	0\\
131	0\\
132	0\\
133	0\\
134	0\\
135	0\\
136	0\\
137	0\\
138	0\\
139	0\\
140	0\\
141	0\\
142	0\\
143	0\\
144	0\\
145	0\\
146	0\\
147	0\\
148	0\\
149	0\\
150	0\\
151	0\\
152	0\\
153	0\\
154	0\\
155	0\\
156	0\\
157	0\\
158	0\\
159	0\\
160	0\\
161	0\\
162	0\\
163	0\\
164	0\\
165	0\\
166	0\\
167	0\\
168	0\\
169	0\\
170	0\\
171	0\\
172	0\\
173	0\\
174	0\\
175	0\\
176	0\\
177	0\\
178	0\\
179	0\\
180	0\\
181	0\\
182	0\\
183	0\\
184	0\\
185	0\\
186	0\\
187	0\\
188	0\\
189	0\\
190	0\\
191	0\\
192	0\\
193	0\\
194	0\\
195	0\\
196	0\\
197	0\\
198	0\\
199	0\\
200	0\\
201	0\\
202	0\\
203	0\\
204	0\\
205	0\\
206	0\\
207	0\\
208	0\\
209	0\\
210	0\\
211	0\\
212	0\\
213	0\\
214	0\\
215	0\\
216	0\\
217	0\\
218	0\\
219	0\\
220	0\\
221	0\\
222	0\\
223	0\\
224	0\\
225	0\\
226	0\\
227	0\\
228	0\\
229	0\\
230	0\\
231	0\\
232	0\\
233	0\\
234	0\\
235	0\\
236	0\\
237	0\\
238	0\\
239	0\\
240	0\\
241	0\\
242	0\\
243	0\\
244	0\\
245	0\\
246	0\\
247	0\\
248	0\\
249	0\\
250	0\\
251	0\\
252	0\\
253	0\\
254	0\\
255	0\\
256	0\\
257	0\\
258	0\\
259	0\\
260	0\\
261	0\\
262	0\\
263	0\\
264	0\\
265	0\\
266	0\\
267	0\\
268	0\\
269	0\\
270	0\\
271	0\\
272	0\\
273	0\\
274	0\\
275	0\\
276	0\\
277	0\\
278	0\\
279	0\\
280	0\\
281	0\\
282	0\\
283	0\\
284	0\\
285	0\\
286	0\\
287	0\\
288	0\\
289	0\\
290	0\\
291	0\\
292	0\\
293	0\\
294	0\\
295	0\\
296	0\\
297	0\\
298	0\\
299	0\\
300	0\\
301	0\\
302	0\\
303	0\\
304	0\\
305	0\\
306	0\\
307	0\\
308	0\\
309	0\\
310	0\\
311	0\\
312	0\\
313	0\\
314	0\\
315	0\\
316	0\\
317	0\\
318	0\\
319	0\\
320	0\\
321	0\\
322	0\\
323	0\\
324	0\\
325	0\\
326	0\\
327	0\\
328	0\\
329	0\\
330	0\\
331	0\\
332	0\\
333	0\\
334	0\\
335	0\\
336	0\\
337	0\\
338	0\\
339	0\\
340	0\\
341	0\\
342	0\\
343	0\\
344	0\\
345	0\\
346	0\\
347	0\\
348	0\\
349	0\\
350	0\\
351	0\\
352	0\\
353	0\\
354	0\\
355	0\\
356	0\\
357	0\\
358	0\\
359	0\\
360	0\\
361	0\\
362	0\\
363	0\\
364	0\\
365	0\\
366	0\\
367	0\\
368	0\\
369	0\\
370	0\\
371	0\\
372	0\\
373	0\\
374	0\\
375	0\\
376	0\\
377	0\\
378	0\\
379	0\\
380	0\\
381	0\\
382	0\\
383	0\\
384	0\\
385	0\\
386	0\\
387	0\\
388	0\\
389	0\\
390	0\\
391	0\\
392	0\\
393	0\\
394	0\\
395	0\\
396	0\\
397	0\\
398	0\\
399	0\\
400	0\\
401	0\\
402	0\\
403	0\\
404	0\\
405	0\\
406	0\\
407	0\\
408	0\\
409	0\\
410	0\\
411	0\\
412	0\\
413	0\\
414	0\\
415	0\\
416	0\\
417	0\\
418	0\\
419	0\\
420	0\\
421	0\\
422	0\\
423	0\\
424	0\\
425	0\\
426	0\\
427	0\\
428	0\\
429	0\\
430	0\\
431	0\\
432	0\\
433	0\\
434	0\\
435	0\\
436	0\\
437	0\\
438	0\\
439	0\\
440	0\\
441	0\\
442	0\\
443	0\\
444	0\\
445	0\\
446	0\\
447	0\\
448	0\\
449	0\\
450	0\\
451	0\\
452	0\\
453	0\\
454	0\\
455	0\\
456	0\\
457	0\\
458	0\\
459	0\\
460	0\\
461	0\\
462	0\\
463	0\\
464	0\\
465	0\\
466	0\\
467	0\\
468	0\\
469	0\\
470	0\\
471	0\\
472	0\\
473	0\\
474	0\\
475	0\\
476	0\\
477	0\\
478	0\\
479	0\\
480	0\\
481	0\\
482	0\\
483	0\\
484	0\\
485	0\\
486	0\\
487	0\\
488	0\\
489	0\\
490	0\\
491	0\\
492	0\\
493	0\\
494	0\\
495	0\\
496	0\\
497	0\\
498	0\\
499	0\\
500	0\\
501	0\\
502	0\\
503	0\\
504	0\\
505	0\\
506	0\\
507	0\\
508	0\\
509	0\\
510	0\\
511	0\\
512	0\\
513	0\\
514	0\\
515	0\\
516	0\\
517	0\\
518	0\\
519	0\\
520	0\\
521	0\\
522	0\\
523	0\\
524	0\\
525	0\\
526	0\\
527	0\\
528	0\\
529	0\\
530	0\\
531	0\\
532	0\\
533	0\\
534	0\\
535	0\\
536	0\\
537	0\\
538	0\\
539	6.8109836946266e-06\\
540	3.2333550628703e-05\\
541	5.84639163104344e-05\\
542	8.52179610591799e-05\\
543	0.000112610529629426\\
544	0.000140656593656372\\
545	0.000169372996211992\\
546	0.000199020774807513\\
547	0.000229285591716688\\
548	0.00025996464680747\\
549	0.000291246962245206\\
550	0.000323061482549782\\
551	0.000355620274831104\\
552	0.000388995178994957\\
553	0.000423221419650544\\
554	0.000458328095444309\\
555	0.000494344032598358\\
556	0.000531299105939698\\
557	0.000569224327827854\\
558	0.000608484400997936\\
559	0.000649558790110594\\
560	0.000690283174206512\\
561	0.000731090493962589\\
562	0.000772286971587725\\
563	0.000814588754116958\\
564	0.000858056127894556\\
565	0.000992351393135649\\
566	0.00136836372751276\\
567	0.00162000548846045\\
568	0.00168633694558227\\
569	0.00175171060027825\\
570	0.001818244178077\\
571	0.00188597917131468\\
572	0.00195495039465754\\
573	0.00202519426812732\\
574	0.00209674918280833\\
575	0.00216965567354768\\
576	0.00224395658330853\\
577	0.00231969723793263\\
578	0.00239692563429661\\
579	0.00247569264464508\\
580	0.00255605223972036\\
581	0.00263806173636886\\
582	0.00272178208245834\\
583	0.00280727821124774\\
584	0.00289461954867251\\
585	0.00298388089415188\\
586	0.00307514426197732\\
587	0.00316850325005475\\
588	0.00326407412228818\\
589	0.0033620247953493\\
590	0.00346265165082637\\
591	0.00356658418066895\\
592	0.00367533141522134\\
593	0.00379274227372929\\
594	0.00392890988198339\\
595	0.00411043972693671\\
596	0.00440815817174734\\
597	0.00501118703192877\\
598	0.00642488516645657\\
599	0\\
600	0\\
};
\addplot [color=mycolor3,solid,forget plot]
  table[row sep=crcr]{%
1	0\\
2	0\\
3	0\\
4	0\\
5	0\\
6	0\\
7	0\\
8	0\\
9	0\\
10	0\\
11	0\\
12	0\\
13	0\\
14	0\\
15	0\\
16	0\\
17	0\\
18	0\\
19	0\\
20	0\\
21	0\\
22	0\\
23	0\\
24	0\\
25	0\\
26	0\\
27	0\\
28	0\\
29	0\\
30	0\\
31	0\\
32	0\\
33	0\\
34	0\\
35	0\\
36	0\\
37	0\\
38	0\\
39	0\\
40	0\\
41	0\\
42	0\\
43	0\\
44	0\\
45	0\\
46	0\\
47	0\\
48	0\\
49	0\\
50	0\\
51	0\\
52	0\\
53	0\\
54	0\\
55	0\\
56	0\\
57	0\\
58	0\\
59	0\\
60	0\\
61	0\\
62	0\\
63	0\\
64	0\\
65	0\\
66	0\\
67	0\\
68	0\\
69	0\\
70	0\\
71	0\\
72	0\\
73	0\\
74	0\\
75	0\\
76	0\\
77	0\\
78	0\\
79	0\\
80	0\\
81	0\\
82	0\\
83	0\\
84	0\\
85	0\\
86	0\\
87	0\\
88	0\\
89	0\\
90	0\\
91	0\\
92	0\\
93	0\\
94	0\\
95	0\\
96	0\\
97	0\\
98	0\\
99	0\\
100	0\\
101	0\\
102	0\\
103	0\\
104	0\\
105	0\\
106	0\\
107	0\\
108	0\\
109	0\\
110	0\\
111	0\\
112	0\\
113	0\\
114	0\\
115	0\\
116	0\\
117	0\\
118	0\\
119	0\\
120	0\\
121	0\\
122	0\\
123	0\\
124	0\\
125	0\\
126	0\\
127	0\\
128	0\\
129	0\\
130	0\\
131	0\\
132	0\\
133	0\\
134	0\\
135	0\\
136	0\\
137	0\\
138	0\\
139	0\\
140	0\\
141	0\\
142	0\\
143	0\\
144	0\\
145	0\\
146	0\\
147	0\\
148	0\\
149	0\\
150	0\\
151	0\\
152	0\\
153	0\\
154	0\\
155	0\\
156	0\\
157	0\\
158	0\\
159	0\\
160	0\\
161	0\\
162	0\\
163	0\\
164	0\\
165	0\\
166	0\\
167	0\\
168	0\\
169	0\\
170	0\\
171	0\\
172	0\\
173	0\\
174	0\\
175	0\\
176	0\\
177	0\\
178	0\\
179	0\\
180	0\\
181	0\\
182	0\\
183	0\\
184	0\\
185	0\\
186	0\\
187	0\\
188	0\\
189	0\\
190	0\\
191	0\\
192	0\\
193	0\\
194	0\\
195	0\\
196	0\\
197	0\\
198	0\\
199	0\\
200	0\\
201	0\\
202	0\\
203	0\\
204	0\\
205	0\\
206	0\\
207	0\\
208	0\\
209	0\\
210	0\\
211	0\\
212	0\\
213	0\\
214	0\\
215	0\\
216	0\\
217	0\\
218	0\\
219	0\\
220	0\\
221	0\\
222	0\\
223	0\\
224	0\\
225	0\\
226	0\\
227	0\\
228	0\\
229	0\\
230	0\\
231	0\\
232	0\\
233	0\\
234	0\\
235	0\\
236	0\\
237	0\\
238	0\\
239	0\\
240	0\\
241	0\\
242	0\\
243	0\\
244	0\\
245	0\\
246	0\\
247	0\\
248	0\\
249	0\\
250	0\\
251	0\\
252	0\\
253	0\\
254	0\\
255	0\\
256	0\\
257	0\\
258	0\\
259	0\\
260	0\\
261	0\\
262	0\\
263	0\\
264	0\\
265	0\\
266	0\\
267	0\\
268	0\\
269	0\\
270	0\\
271	0\\
272	0\\
273	0\\
274	0\\
275	0\\
276	0\\
277	0\\
278	0\\
279	0\\
280	0\\
281	0\\
282	0\\
283	0\\
284	0\\
285	0\\
286	0\\
287	0\\
288	0\\
289	0\\
290	0\\
291	0\\
292	0\\
293	0\\
294	0\\
295	0\\
296	0\\
297	0\\
298	0\\
299	0\\
300	0\\
301	0\\
302	0\\
303	0\\
304	0\\
305	0\\
306	0\\
307	0\\
308	0\\
309	0\\
310	0\\
311	0\\
312	0\\
313	0\\
314	0\\
315	0\\
316	0\\
317	0\\
318	0\\
319	0\\
320	0\\
321	0\\
322	0\\
323	0\\
324	0\\
325	0\\
326	0\\
327	0\\
328	0\\
329	0\\
330	0\\
331	0\\
332	0\\
333	0\\
334	0\\
335	0\\
336	0\\
337	0\\
338	0\\
339	0\\
340	0\\
341	0\\
342	0\\
343	0\\
344	0\\
345	0\\
346	0\\
347	0\\
348	0\\
349	0\\
350	0\\
351	0\\
352	0\\
353	0\\
354	0\\
355	0\\
356	0\\
357	0\\
358	0\\
359	0\\
360	0\\
361	0\\
362	0\\
363	0\\
364	0\\
365	0\\
366	0\\
367	0\\
368	0\\
369	0\\
370	0\\
371	0\\
372	0\\
373	0\\
374	0\\
375	0\\
376	0\\
377	0\\
378	0\\
379	0\\
380	0\\
381	0\\
382	0\\
383	0\\
384	0\\
385	0\\
386	0\\
387	0\\
388	0\\
389	0\\
390	0\\
391	0\\
392	0\\
393	0\\
394	0\\
395	0\\
396	0\\
397	0\\
398	0\\
399	0\\
400	0\\
401	0\\
402	0\\
403	0\\
404	0\\
405	0\\
406	0\\
407	0\\
408	0\\
409	0\\
410	0\\
411	0\\
412	0\\
413	0\\
414	0\\
415	0\\
416	0\\
417	0\\
418	0\\
419	0\\
420	0\\
421	0\\
422	0\\
423	0\\
424	0\\
425	0\\
426	0\\
427	0\\
428	0\\
429	0\\
430	0\\
431	0\\
432	0\\
433	0\\
434	0\\
435	0\\
436	0\\
437	0\\
438	0\\
439	0\\
440	0\\
441	0\\
442	0\\
443	0\\
444	0\\
445	0\\
446	0\\
447	0\\
448	0\\
449	0\\
450	0\\
451	0\\
452	0\\
453	0\\
454	0\\
455	0\\
456	0\\
457	0\\
458	0\\
459	0\\
460	0\\
461	0\\
462	0\\
463	0\\
464	0\\
465	0\\
466	0\\
467	0\\
468	0\\
469	0\\
470	0\\
471	0\\
472	0\\
473	0\\
474	0\\
475	0\\
476	0\\
477	0\\
478	0\\
479	0\\
480	0\\
481	0\\
482	0\\
483	0\\
484	0\\
485	0\\
486	0\\
487	0\\
488	0\\
489	0\\
490	0\\
491	0\\
492	0\\
493	0\\
494	0\\
495	0\\
496	0\\
497	0\\
498	0\\
499	0\\
500	0\\
501	0\\
502	0\\
503	0\\
504	0\\
505	0\\
506	0\\
507	0\\
508	0\\
509	0\\
510	0\\
511	0\\
512	0\\
513	0\\
514	0\\
515	0\\
516	0\\
517	0\\
518	0\\
519	0\\
520	0\\
521	0\\
522	0\\
523	0\\
524	0\\
525	0\\
526	0\\
527	0\\
528	0\\
529	0\\
530	0\\
531	0\\
532	0\\
533	0\\
534	0\\
535	0\\
536	0\\
537	0\\
538	0\\
539	0\\
540	2.45283203031442e-05\\
541	5.05305641840116e-05\\
542	7.7151130188513e-05\\
543	0.000104409775999626\\
544	0.000132322707189817\\
545	0.000160906048590909\\
546	0.000190176201071463\\
547	0.000220151473717069\\
548	0.000251106602530764\\
549	0.000282689628183228\\
550	0.00031473882969767\\
551	0.000347426881058723\\
552	0.000380710692019464\\
553	0.000414748901432071\\
554	0.000449641734149903\\
555	0.000485433046687176\\
556	0.000522155524764944\\
557	0.000559840336988405\\
558	0.000598520034534356\\
559	0.00063823058415102\\
560	0.000680175666195175\\
561	0.000722313760005713\\
562	0.000764531447846155\\
563	0.000806672231281403\\
564	0.000849854681485079\\
565	0.000894229023457877\\
566	0.00101511639260571\\
567	0.00139640472273508\\
568	0.00166773442770165\\
569	0.00175162603362615\\
570	0.00181824053534768\\
571	0.00188597867140269\\
572	0.00195495029706772\\
573	0.0020251942424614\\
574	0.00209674917340268\\
575	0.00216965566904425\\
576	0.00224395658121081\\
577	0.00231969723683898\\
578	0.0023969256337334\\
579	0.00247569264435266\\
580	0.00255605223958102\\
581	0.00263806173630904\\
582	0.00272178208244085\\
583	0.0028072782112451\\
584	0.00289461954867251\\
585	0.00298388089415188\\
586	0.00307514426197733\\
587	0.00316850325005475\\
588	0.00326407412228818\\
589	0.00336202479534929\\
590	0.00346265165082637\\
591	0.00356658418066895\\
592	0.00367533141522135\\
593	0.00379274227372929\\
594	0.0039289098819834\\
595	0.00411043972693672\\
596	0.00440815817174734\\
597	0.00501118703192877\\
598	0.00642488516645657\\
599	0\\
600	0\\
};
\addplot [color=mycolor4,solid,forget plot]
  table[row sep=crcr]{%
1	0\\
2	0\\
3	0\\
4	0\\
5	0\\
6	0\\
7	0\\
8	0\\
9	0\\
10	0\\
11	0\\
12	0\\
13	0\\
14	0\\
15	0\\
16	0\\
17	0\\
18	0\\
19	0\\
20	0\\
21	0\\
22	0\\
23	0\\
24	0\\
25	0\\
26	0\\
27	0\\
28	0\\
29	0\\
30	0\\
31	0\\
32	0\\
33	0\\
34	0\\
35	0\\
36	0\\
37	0\\
38	0\\
39	0\\
40	0\\
41	0\\
42	0\\
43	0\\
44	0\\
45	0\\
46	0\\
47	0\\
48	0\\
49	0\\
50	0\\
51	0\\
52	0\\
53	0\\
54	0\\
55	0\\
56	0\\
57	0\\
58	0\\
59	0\\
60	0\\
61	0\\
62	0\\
63	0\\
64	0\\
65	0\\
66	0\\
67	0\\
68	0\\
69	0\\
70	0\\
71	0\\
72	0\\
73	0\\
74	0\\
75	0\\
76	0\\
77	0\\
78	0\\
79	0\\
80	0\\
81	0\\
82	0\\
83	0\\
84	0\\
85	0\\
86	0\\
87	0\\
88	0\\
89	0\\
90	0\\
91	0\\
92	0\\
93	0\\
94	0\\
95	0\\
96	0\\
97	0\\
98	0\\
99	0\\
100	0\\
101	0\\
102	0\\
103	0\\
104	0\\
105	0\\
106	0\\
107	0\\
108	0\\
109	0\\
110	0\\
111	0\\
112	0\\
113	0\\
114	0\\
115	0\\
116	0\\
117	0\\
118	0\\
119	0\\
120	0\\
121	0\\
122	0\\
123	0\\
124	0\\
125	0\\
126	0\\
127	0\\
128	0\\
129	0\\
130	0\\
131	0\\
132	0\\
133	0\\
134	0\\
135	0\\
136	0\\
137	0\\
138	0\\
139	0\\
140	0\\
141	0\\
142	0\\
143	0\\
144	0\\
145	0\\
146	0\\
147	0\\
148	0\\
149	0\\
150	0\\
151	0\\
152	0\\
153	0\\
154	0\\
155	0\\
156	0\\
157	0\\
158	0\\
159	0\\
160	0\\
161	0\\
162	0\\
163	0\\
164	0\\
165	0\\
166	0\\
167	0\\
168	0\\
169	0\\
170	0\\
171	0\\
172	0\\
173	0\\
174	0\\
175	0\\
176	0\\
177	0\\
178	0\\
179	0\\
180	0\\
181	0\\
182	0\\
183	0\\
184	0\\
185	0\\
186	0\\
187	0\\
188	0\\
189	0\\
190	0\\
191	0\\
192	0\\
193	0\\
194	0\\
195	0\\
196	0\\
197	0\\
198	0\\
199	0\\
200	0\\
201	0\\
202	0\\
203	0\\
204	0\\
205	0\\
206	0\\
207	0\\
208	0\\
209	0\\
210	0\\
211	0\\
212	0\\
213	0\\
214	0\\
215	0\\
216	0\\
217	0\\
218	0\\
219	0\\
220	0\\
221	0\\
222	0\\
223	0\\
224	0\\
225	0\\
226	0\\
227	0\\
228	0\\
229	0\\
230	0\\
231	0\\
232	0\\
233	0\\
234	0\\
235	0\\
236	0\\
237	0\\
238	0\\
239	0\\
240	0\\
241	0\\
242	0\\
243	0\\
244	0\\
245	0\\
246	0\\
247	0\\
248	0\\
249	0\\
250	0\\
251	0\\
252	0\\
253	0\\
254	0\\
255	0\\
256	0\\
257	0\\
258	0\\
259	0\\
260	0\\
261	0\\
262	0\\
263	0\\
264	0\\
265	0\\
266	0\\
267	0\\
268	0\\
269	0\\
270	0\\
271	0\\
272	0\\
273	0\\
274	0\\
275	0\\
276	0\\
277	0\\
278	0\\
279	0\\
280	0\\
281	0\\
282	0\\
283	0\\
284	0\\
285	0\\
286	0\\
287	0\\
288	0\\
289	0\\
290	0\\
291	0\\
292	0\\
293	0\\
294	0\\
295	0\\
296	0\\
297	0\\
298	0\\
299	0\\
300	0\\
301	0\\
302	0\\
303	0\\
304	0\\
305	0\\
306	0\\
307	0\\
308	0\\
309	0\\
310	0\\
311	0\\
312	0\\
313	0\\
314	0\\
315	0\\
316	0\\
317	0\\
318	0\\
319	0\\
320	0\\
321	0\\
322	0\\
323	0\\
324	0\\
325	0\\
326	0\\
327	0\\
328	0\\
329	0\\
330	0\\
331	0\\
332	0\\
333	0\\
334	0\\
335	0\\
336	0\\
337	0\\
338	0\\
339	0\\
340	0\\
341	0\\
342	0\\
343	0\\
344	0\\
345	0\\
346	0\\
347	0\\
348	0\\
349	0\\
350	0\\
351	0\\
352	0\\
353	0\\
354	0\\
355	0\\
356	0\\
357	0\\
358	0\\
359	0\\
360	0\\
361	0\\
362	0\\
363	0\\
364	0\\
365	0\\
366	0\\
367	0\\
368	0\\
369	0\\
370	0\\
371	0\\
372	0\\
373	0\\
374	0\\
375	0\\
376	0\\
377	0\\
378	0\\
379	0\\
380	0\\
381	0\\
382	0\\
383	0\\
384	0\\
385	0\\
386	0\\
387	0\\
388	0\\
389	0\\
390	0\\
391	0\\
392	0\\
393	0\\
394	0\\
395	0\\
396	0\\
397	0\\
398	0\\
399	0\\
400	0\\
401	0\\
402	0\\
403	0\\
404	0\\
405	0\\
406	0\\
407	0\\
408	0\\
409	0\\
410	0\\
411	0\\
412	0\\
413	0\\
414	0\\
415	0\\
416	0\\
417	0\\
418	0\\
419	0\\
420	0\\
421	0\\
422	0\\
423	0\\
424	0\\
425	0\\
426	0\\
427	0\\
428	0\\
429	0\\
430	0\\
431	0\\
432	0\\
433	0\\
434	0\\
435	0\\
436	0\\
437	0\\
438	0\\
439	0\\
440	0\\
441	0\\
442	0\\
443	0\\
444	0\\
445	0\\
446	0\\
447	0\\
448	0\\
449	0\\
450	0\\
451	0\\
452	0\\
453	0\\
454	0\\
455	0\\
456	0\\
457	0\\
458	0\\
459	0\\
460	0\\
461	0\\
462	0\\
463	0\\
464	0\\
465	0\\
466	0\\
467	0\\
468	0\\
469	0\\
470	0\\
471	0\\
472	0\\
473	0\\
474	0\\
475	0\\
476	0\\
477	0\\
478	0\\
479	0\\
480	0\\
481	0\\
482	0\\
483	0\\
484	0\\
485	0\\
486	0\\
487	0\\
488	0\\
489	0\\
490	0\\
491	0\\
492	0\\
493	0\\
494	0\\
495	0\\
496	0\\
497	0\\
498	0\\
499	0\\
500	0\\
501	0\\
502	0\\
503	0\\
504	0\\
505	0\\
506	0\\
507	0\\
508	0\\
509	0\\
510	0\\
511	0\\
512	0\\
513	0\\
514	0\\
515	0\\
516	0\\
517	0\\
518	0\\
519	0\\
520	0\\
521	0\\
522	0\\
523	0\\
524	0\\
525	0\\
526	0\\
527	0\\
528	0\\
529	0\\
530	0\\
531	0\\
532	0\\
533	0\\
534	0\\
535	0\\
536	0\\
537	0\\
538	0\\
539	0\\
540	1.57801742067035e-05\\
541	4.17209330634838e-05\\
542	6.82521651627544e-05\\
543	9.53742265120166e-05\\
544	0.000123140289849838\\
545	0.000151571903397981\\
546	0.000180687612467819\\
547	0.000210507578298351\\
548	0.000241049124398555\\
549	0.000272332494289059\\
550	0.000304617640894825\\
551	0.000337587086677401\\
552	0.000371095663930957\\
553	0.000405260843325515\\
554	0.000440102586296729\\
555	0.000475683908134913\\
556	0.000512157386629192\\
557	0.000549580513042214\\
558	0.000587986730254416\\
559	0.000627411471795976\\
560	0.00066788979602979\\
561	0.000710048071235101\\
562	0.0007536191055055\\
563	0.000796982169508077\\
564	0.000840690466772013\\
565	0.000884782514078633\\
566	0.000930038853850304\\
567	0.00101562515945223\\
568	0.00140105236470372\\
569	0.00167444769533238\\
570	0.00181751715607354\\
571	0.00188594856162508\\
572	0.00195494627428676\\
573	0.00202519348264098\\
574	0.00209674898273335\\
575	0.00216965560246406\\
576	0.00224395654960764\\
577	0.00231969722252966\\
578	0.0023969256263201\\
579	0.00247569264056459\\
580	0.00255605223759985\\
581	0.00263806173535728\\
582	0.00272178208202001\\
583	0.00280727821111812\\
584	0.00289461954865299\\
585	0.00298388089415187\\
586	0.00307514426197734\\
587	0.00316850325005476\\
588	0.00326407412228819\\
589	0.0033620247953493\\
590	0.00346265165082637\\
591	0.00356658418066895\\
592	0.00367533141522135\\
593	0.0037927422737293\\
594	0.0039289098819834\\
595	0.00411043972693671\\
596	0.00440815817174734\\
597	0.00501118703192877\\
598	0.00642488516645657\\
599	0\\
600	0\\
};
\addplot [color=mycolor5,solid,forget plot]
  table[row sep=crcr]{%
1	0\\
2	0\\
3	0\\
4	0\\
5	0\\
6	0\\
7	0\\
8	0\\
9	0\\
10	0\\
11	0\\
12	0\\
13	0\\
14	0\\
15	0\\
16	0\\
17	0\\
18	0\\
19	0\\
20	0\\
21	0\\
22	0\\
23	0\\
24	0\\
25	0\\
26	0\\
27	0\\
28	0\\
29	0\\
30	0\\
31	0\\
32	0\\
33	0\\
34	0\\
35	0\\
36	0\\
37	0\\
38	0\\
39	0\\
40	0\\
41	0\\
42	0\\
43	0\\
44	0\\
45	0\\
46	0\\
47	0\\
48	0\\
49	0\\
50	0\\
51	0\\
52	0\\
53	0\\
54	0\\
55	0\\
56	0\\
57	0\\
58	0\\
59	0\\
60	0\\
61	0\\
62	0\\
63	0\\
64	0\\
65	0\\
66	0\\
67	0\\
68	0\\
69	0\\
70	0\\
71	0\\
72	0\\
73	0\\
74	0\\
75	0\\
76	0\\
77	0\\
78	0\\
79	0\\
80	0\\
81	0\\
82	0\\
83	0\\
84	0\\
85	0\\
86	0\\
87	0\\
88	0\\
89	0\\
90	0\\
91	0\\
92	0\\
93	0\\
94	0\\
95	0\\
96	0\\
97	0\\
98	0\\
99	0\\
100	0\\
101	0\\
102	0\\
103	0\\
104	0\\
105	0\\
106	0\\
107	0\\
108	0\\
109	0\\
110	0\\
111	0\\
112	0\\
113	0\\
114	0\\
115	0\\
116	0\\
117	0\\
118	0\\
119	0\\
120	0\\
121	0\\
122	0\\
123	0\\
124	0\\
125	0\\
126	0\\
127	0\\
128	0\\
129	0\\
130	0\\
131	0\\
132	0\\
133	0\\
134	0\\
135	0\\
136	0\\
137	0\\
138	0\\
139	0\\
140	0\\
141	0\\
142	0\\
143	0\\
144	0\\
145	0\\
146	0\\
147	0\\
148	0\\
149	0\\
150	0\\
151	0\\
152	0\\
153	0\\
154	0\\
155	0\\
156	0\\
157	0\\
158	0\\
159	0\\
160	0\\
161	0\\
162	0\\
163	0\\
164	0\\
165	0\\
166	0\\
167	0\\
168	0\\
169	0\\
170	0\\
171	0\\
172	0\\
173	0\\
174	0\\
175	0\\
176	0\\
177	0\\
178	0\\
179	0\\
180	0\\
181	0\\
182	0\\
183	0\\
184	0\\
185	0\\
186	0\\
187	0\\
188	0\\
189	0\\
190	0\\
191	0\\
192	0\\
193	0\\
194	0\\
195	0\\
196	0\\
197	0\\
198	0\\
199	0\\
200	0\\
201	0\\
202	0\\
203	0\\
204	0\\
205	0\\
206	0\\
207	0\\
208	0\\
209	0\\
210	0\\
211	0\\
212	0\\
213	0\\
214	0\\
215	0\\
216	0\\
217	0\\
218	0\\
219	0\\
220	0\\
221	0\\
222	0\\
223	0\\
224	0\\
225	0\\
226	0\\
227	0\\
228	0\\
229	0\\
230	0\\
231	0\\
232	0\\
233	0\\
234	0\\
235	0\\
236	0\\
237	0\\
238	0\\
239	0\\
240	0\\
241	0\\
242	0\\
243	0\\
244	0\\
245	0\\
246	0\\
247	0\\
248	0\\
249	0\\
250	0\\
251	0\\
252	0\\
253	0\\
254	0\\
255	0\\
256	0\\
257	0\\
258	0\\
259	0\\
260	0\\
261	0\\
262	0\\
263	0\\
264	0\\
265	0\\
266	0\\
267	0\\
268	0\\
269	0\\
270	0\\
271	0\\
272	0\\
273	0\\
274	0\\
275	0\\
276	0\\
277	0\\
278	0\\
279	0\\
280	0\\
281	0\\
282	0\\
283	0\\
284	0\\
285	0\\
286	0\\
287	0\\
288	0\\
289	0\\
290	0\\
291	0\\
292	0\\
293	0\\
294	0\\
295	0\\
296	0\\
297	0\\
298	0\\
299	0\\
300	0\\
301	0\\
302	0\\
303	0\\
304	0\\
305	0\\
306	0\\
307	0\\
308	0\\
309	0\\
310	0\\
311	0\\
312	0\\
313	0\\
314	0\\
315	0\\
316	0\\
317	0\\
318	0\\
319	0\\
320	0\\
321	0\\
322	0\\
323	0\\
324	0\\
325	0\\
326	0\\
327	0\\
328	0\\
329	0\\
330	0\\
331	0\\
332	0\\
333	0\\
334	0\\
335	0\\
336	0\\
337	0\\
338	0\\
339	0\\
340	0\\
341	0\\
342	0\\
343	0\\
344	0\\
345	0\\
346	0\\
347	0\\
348	0\\
349	0\\
350	0\\
351	0\\
352	0\\
353	0\\
354	0\\
355	0\\
356	0\\
357	0\\
358	0\\
359	0\\
360	0\\
361	0\\
362	0\\
363	0\\
364	0\\
365	0\\
366	0\\
367	0\\
368	0\\
369	0\\
370	0\\
371	0\\
372	0\\
373	0\\
374	0\\
375	0\\
376	0\\
377	0\\
378	0\\
379	0\\
380	0\\
381	0\\
382	0\\
383	0\\
384	0\\
385	0\\
386	0\\
387	0\\
388	0\\
389	0\\
390	0\\
391	0\\
392	0\\
393	0\\
394	0\\
395	0\\
396	0\\
397	0\\
398	0\\
399	0\\
400	0\\
401	0\\
402	0\\
403	0\\
404	0\\
405	0\\
406	0\\
407	0\\
408	0\\
409	0\\
410	0\\
411	0\\
412	0\\
413	0\\
414	0\\
415	0\\
416	0\\
417	0\\
418	0\\
419	0\\
420	0\\
421	0\\
422	0\\
423	0\\
424	0\\
425	0\\
426	0\\
427	0\\
428	0\\
429	0\\
430	0\\
431	0\\
432	0\\
433	0\\
434	0\\
435	0\\
436	0\\
437	0\\
438	0\\
439	0\\
440	0\\
441	0\\
442	0\\
443	0\\
444	0\\
445	0\\
446	0\\
447	0\\
448	0\\
449	0\\
450	0\\
451	0\\
452	0\\
453	0\\
454	0\\
455	0\\
456	0\\
457	0\\
458	0\\
459	0\\
460	0\\
461	0\\
462	0\\
463	0\\
464	0\\
465	0\\
466	0\\
467	0\\
468	0\\
469	0\\
470	0\\
471	0\\
472	0\\
473	0\\
474	0\\
475	0\\
476	0\\
477	0\\
478	0\\
479	0\\
480	0\\
481	0\\
482	0\\
483	0\\
484	0\\
485	0\\
486	0\\
487	0\\
488	0\\
489	0\\
490	0\\
491	0\\
492	0\\
493	0\\
494	0\\
495	0\\
496	0\\
497	0\\
498	0\\
499	0\\
500	0\\
501	0\\
502	0\\
503	0\\
504	0\\
505	0\\
506	0\\
507	0\\
508	0\\
509	0\\
510	0\\
511	0\\
512	0\\
513	0\\
514	0\\
515	0\\
516	0\\
517	0\\
518	0\\
519	0\\
520	0\\
521	0\\
522	0\\
523	0\\
524	0\\
525	0\\
526	0\\
527	0\\
528	0\\
529	0\\
530	0\\
531	0\\
532	0\\
533	0\\
534	0\\
535	0\\
536	0\\
537	0\\
538	0\\
539	0\\
540	5.67779163272291e-06\\
541	3.16144865046584e-05\\
542	5.81200036704712e-05\\
543	8.51921790854403e-05\\
544	0.00011288255189914\\
545	0.000141212350434799\\
546	0.000170189245831659\\
547	0.000199839716824939\\
548	0.000230207356742342\\
549	0.000261309781838766\\
550	0.000293170189040735\\
551	0.000325810513721937\\
552	0.000359445923410564\\
553	0.000393869735973008\\
554	0.000428926673279003\\
555	0.000464639289161119\\
556	0.000501125425574296\\
557	0.00053830708211444\\
558	0.000576434759629047\\
559	0.000615550561680942\\
560	0.000655704173597092\\
561	0.000696934033709092\\
562	0.000739282894716805\\
563	0.000783943930247253\\
564	0.000828931810503418\\
565	0.000874071000879365\\
566	0.000919506942758544\\
567	0.000965676871722533\\
568	0.00101311411328194\\
569	0.00136536274941772\\
570	0.00166568078770398\\
571	0.00187981036938081\\
572	0.00195469927586354\\
573	0.00202516132606539\\
574	0.00209674309368475\\
575	0.00216965418970599\\
576	0.00224395608129885\\
577	0.00231969700168746\\
578	0.00239692552947569\\
579	0.00247569259072984\\
580	0.00255605221240535\\
581	0.00263806172208846\\
582	0.00272178207560973\\
583	0.00280727820819447\\
584	0.00289461954774638\\
585	0.00298388089400896\\
586	0.00307514426197733\\
587	0.00316850325005475\\
588	0.00326407412228819\\
589	0.0033620247953493\\
590	0.00346265165082638\\
591	0.00356658418066895\\
592	0.00367533141522135\\
593	0.00379274227372929\\
594	0.00392890988198341\\
595	0.00411043972693671\\
596	0.00440815817174734\\
597	0.00501118703192877\\
598	0.00642488516645657\\
599	0\\
600	0\\
};
\addplot [color=mycolor6,solid,forget plot]
  table[row sep=crcr]{%
1	0\\
2	0\\
3	0\\
4	0\\
5	0\\
6	0\\
7	0\\
8	0\\
9	0\\
10	0\\
11	0\\
12	0\\
13	0\\
14	0\\
15	0\\
16	0\\
17	0\\
18	0\\
19	0\\
20	0\\
21	0\\
22	0\\
23	0\\
24	0\\
25	0\\
26	0\\
27	0\\
28	0\\
29	0\\
30	0\\
31	0\\
32	0\\
33	0\\
34	0\\
35	0\\
36	0\\
37	0\\
38	0\\
39	0\\
40	0\\
41	0\\
42	0\\
43	0\\
44	0\\
45	0\\
46	0\\
47	0\\
48	0\\
49	0\\
50	0\\
51	0\\
52	0\\
53	0\\
54	0\\
55	0\\
56	0\\
57	0\\
58	0\\
59	0\\
60	0\\
61	0\\
62	0\\
63	0\\
64	0\\
65	0\\
66	0\\
67	0\\
68	0\\
69	0\\
70	0\\
71	0\\
72	0\\
73	0\\
74	0\\
75	0\\
76	0\\
77	0\\
78	0\\
79	0\\
80	0\\
81	0\\
82	0\\
83	0\\
84	0\\
85	0\\
86	0\\
87	0\\
88	0\\
89	0\\
90	0\\
91	0\\
92	0\\
93	0\\
94	0\\
95	0\\
96	0\\
97	0\\
98	0\\
99	0\\
100	0\\
101	0\\
102	0\\
103	0\\
104	0\\
105	0\\
106	0\\
107	0\\
108	0\\
109	0\\
110	0\\
111	0\\
112	0\\
113	0\\
114	0\\
115	0\\
116	0\\
117	0\\
118	0\\
119	0\\
120	0\\
121	0\\
122	0\\
123	0\\
124	0\\
125	0\\
126	0\\
127	0\\
128	0\\
129	0\\
130	0\\
131	0\\
132	0\\
133	0\\
134	0\\
135	0\\
136	0\\
137	0\\
138	0\\
139	0\\
140	0\\
141	0\\
142	0\\
143	0\\
144	0\\
145	0\\
146	0\\
147	0\\
148	0\\
149	0\\
150	0\\
151	0\\
152	0\\
153	0\\
154	0\\
155	0\\
156	0\\
157	0\\
158	0\\
159	0\\
160	0\\
161	0\\
162	0\\
163	0\\
164	0\\
165	0\\
166	0\\
167	0\\
168	0\\
169	0\\
170	0\\
171	0\\
172	0\\
173	0\\
174	0\\
175	0\\
176	0\\
177	0\\
178	0\\
179	0\\
180	0\\
181	0\\
182	0\\
183	0\\
184	0\\
185	0\\
186	0\\
187	0\\
188	0\\
189	0\\
190	0\\
191	0\\
192	0\\
193	0\\
194	0\\
195	0\\
196	0\\
197	0\\
198	0\\
199	0\\
200	0\\
201	0\\
202	0\\
203	0\\
204	0\\
205	0\\
206	0\\
207	0\\
208	0\\
209	0\\
210	0\\
211	0\\
212	0\\
213	0\\
214	0\\
215	0\\
216	0\\
217	0\\
218	0\\
219	0\\
220	0\\
221	0\\
222	0\\
223	0\\
224	0\\
225	0\\
226	0\\
227	0\\
228	0\\
229	0\\
230	0\\
231	0\\
232	0\\
233	0\\
234	0\\
235	0\\
236	0\\
237	0\\
238	0\\
239	0\\
240	0\\
241	0\\
242	0\\
243	0\\
244	0\\
245	0\\
246	0\\
247	0\\
248	0\\
249	0\\
250	0\\
251	0\\
252	0\\
253	0\\
254	0\\
255	0\\
256	0\\
257	0\\
258	0\\
259	0\\
260	0\\
261	0\\
262	0\\
263	0\\
264	0\\
265	0\\
266	0\\
267	0\\
268	0\\
269	0\\
270	0\\
271	0\\
272	0\\
273	0\\
274	0\\
275	0\\
276	0\\
277	0\\
278	0\\
279	0\\
280	0\\
281	0\\
282	0\\
283	0\\
284	0\\
285	0\\
286	0\\
287	0\\
288	0\\
289	0\\
290	0\\
291	0\\
292	0\\
293	0\\
294	0\\
295	0\\
296	0\\
297	0\\
298	0\\
299	0\\
300	0\\
301	0\\
302	0\\
303	0\\
304	0\\
305	0\\
306	0\\
307	0\\
308	0\\
309	0\\
310	0\\
311	0\\
312	0\\
313	0\\
314	0\\
315	0\\
316	0\\
317	0\\
318	0\\
319	0\\
320	0\\
321	0\\
322	0\\
323	0\\
324	0\\
325	0\\
326	0\\
327	0\\
328	0\\
329	0\\
330	0\\
331	0\\
332	0\\
333	0\\
334	0\\
335	0\\
336	0\\
337	0\\
338	0\\
339	0\\
340	0\\
341	0\\
342	0\\
343	0\\
344	0\\
345	0\\
346	0\\
347	0\\
348	0\\
349	0\\
350	0\\
351	0\\
352	0\\
353	0\\
354	0\\
355	0\\
356	0\\
357	0\\
358	0\\
359	0\\
360	0\\
361	0\\
362	0\\
363	0\\
364	0\\
365	0\\
366	0\\
367	0\\
368	0\\
369	0\\
370	0\\
371	0\\
372	0\\
373	0\\
374	0\\
375	0\\
376	0\\
377	0\\
378	0\\
379	0\\
380	0\\
381	0\\
382	0\\
383	0\\
384	0\\
385	0\\
386	0\\
387	0\\
388	0\\
389	0\\
390	0\\
391	0\\
392	0\\
393	0\\
394	0\\
395	0\\
396	0\\
397	0\\
398	0\\
399	0\\
400	0\\
401	0\\
402	0\\
403	0\\
404	0\\
405	0\\
406	0\\
407	0\\
408	0\\
409	0\\
410	0\\
411	0\\
412	0\\
413	0\\
414	0\\
415	0\\
416	0\\
417	0\\
418	0\\
419	0\\
420	0\\
421	0\\
422	0\\
423	0\\
424	0\\
425	0\\
426	0\\
427	0\\
428	0\\
429	0\\
430	0\\
431	0\\
432	0\\
433	0\\
434	0\\
435	0\\
436	0\\
437	0\\
438	0\\
439	0\\
440	0\\
441	0\\
442	0\\
443	0\\
444	0\\
445	0\\
446	0\\
447	0\\
448	0\\
449	0\\
450	0\\
451	0\\
452	0\\
453	0\\
454	0\\
455	0\\
456	0\\
457	0\\
458	0\\
459	0\\
460	0\\
461	0\\
462	0\\
463	0\\
464	0\\
465	0\\
466	0\\
467	0\\
468	0\\
469	0\\
470	0\\
471	0\\
472	0\\
473	0\\
474	0\\
475	0\\
476	0\\
477	0\\
478	0\\
479	0\\
480	0\\
481	0\\
482	0\\
483	0\\
484	0\\
485	0\\
486	0\\
487	0\\
488	0\\
489	0\\
490	0\\
491	0\\
492	0\\
493	0\\
494	0\\
495	0\\
496	0\\
497	0\\
498	0\\
499	0\\
500	0\\
501	0\\
502	0\\
503	0\\
504	0\\
505	0\\
506	0\\
507	0\\
508	0\\
509	0\\
510	0\\
511	0\\
512	0\\
513	0\\
514	0\\
515	0\\
516	0\\
517	0\\
518	0\\
519	0\\
520	0\\
521	0\\
522	0\\
523	0\\
524	0\\
525	0\\
526	0\\
527	0\\
528	0\\
529	0\\
530	0\\
531	0\\
532	0\\
533	0\\
534	0\\
535	0\\
536	0\\
537	0\\
538	0\\
539	0\\
540	0\\
541	1.99792072081089e-05\\
542	4.63727525165937e-05\\
543	7.34212262876605e-05\\
544	0.000101084696376327\\
545	0.000129373228443587\\
546	0.000158292503572227\\
547	0.000187861500603815\\
548	0.000218122519240035\\
549	0.00024909286214484\\
550	0.000280775360288263\\
551	0.000313225176815619\\
552	0.000346470143581298\\
553	0.00038053214301298\\
554	0.000415556128944413\\
555	0.000451524249748653\\
556	0.00048824208518114\\
557	0.000525592789678638\\
558	0.000563780851805092\\
559	0.000602776718153804\\
560	0.00064265699652178\\
561	0.000683574413147921\\
562	0.000725585257827588\\
563	0.000768736465274667\\
564	0.000813472406506441\\
565	0.000859930031446736\\
566	0.000906633225702796\\
567	0.000953771917621697\\
568	0.00100124266870638\\
569	0.00104967338643609\\
570	0.0013043068138501\\
571	0.00163930374758022\\
572	0.00190309861278883\\
573	0.00202315071336391\\
574	0.00209648783523968\\
575	0.00216960877787299\\
576	0.00224394564255947\\
577	0.00231969373006616\\
578	0.00239692399237447\\
579	0.00247569194043164\\
580	0.00255605188006998\\
581	0.00263806155640494\\
582	0.0027217819877466\\
583	0.0028072781656434\\
584	0.00289461952769629\\
585	0.00298388088761337\\
586	0.00307514426094347\\
587	0.00316850325005476\\
588	0.00326407412228819\\
589	0.00336202479534929\\
590	0.00346265165082637\\
591	0.00356658418066895\\
592	0.00367533141522134\\
593	0.00379274227372929\\
594	0.0039289098819834\\
595	0.00411043972693671\\
596	0.00440815817174734\\
597	0.00501118703192877\\
598	0.00642488516645657\\
599	0\\
600	0\\
};
\addplot [color=mycolor7,solid,forget plot]
  table[row sep=crcr]{%
1	0\\
2	0\\
3	0\\
4	0\\
5	0\\
6	0\\
7	0\\
8	0\\
9	0\\
10	0\\
11	0\\
12	0\\
13	0\\
14	0\\
15	0\\
16	0\\
17	0\\
18	0\\
19	0\\
20	0\\
21	0\\
22	0\\
23	0\\
24	0\\
25	0\\
26	0\\
27	0\\
28	0\\
29	0\\
30	0\\
31	0\\
32	0\\
33	0\\
34	0\\
35	0\\
36	0\\
37	0\\
38	0\\
39	0\\
40	0\\
41	0\\
42	0\\
43	0\\
44	0\\
45	0\\
46	0\\
47	0\\
48	0\\
49	0\\
50	0\\
51	0\\
52	0\\
53	0\\
54	0\\
55	0\\
56	0\\
57	0\\
58	0\\
59	0\\
60	0\\
61	0\\
62	0\\
63	0\\
64	0\\
65	0\\
66	0\\
67	0\\
68	0\\
69	0\\
70	0\\
71	0\\
72	0\\
73	0\\
74	0\\
75	0\\
76	0\\
77	0\\
78	0\\
79	0\\
80	0\\
81	0\\
82	0\\
83	0\\
84	0\\
85	0\\
86	0\\
87	0\\
88	0\\
89	0\\
90	0\\
91	0\\
92	0\\
93	0\\
94	0\\
95	0\\
96	0\\
97	0\\
98	0\\
99	0\\
100	0\\
101	0\\
102	0\\
103	0\\
104	0\\
105	0\\
106	0\\
107	0\\
108	0\\
109	0\\
110	0\\
111	0\\
112	0\\
113	0\\
114	0\\
115	0\\
116	0\\
117	0\\
118	0\\
119	0\\
120	0\\
121	0\\
122	0\\
123	0\\
124	0\\
125	0\\
126	0\\
127	0\\
128	0\\
129	0\\
130	0\\
131	0\\
132	0\\
133	0\\
134	0\\
135	0\\
136	0\\
137	0\\
138	0\\
139	0\\
140	0\\
141	0\\
142	0\\
143	0\\
144	0\\
145	0\\
146	0\\
147	0\\
148	0\\
149	0\\
150	0\\
151	0\\
152	0\\
153	0\\
154	0\\
155	0\\
156	0\\
157	0\\
158	0\\
159	0\\
160	0\\
161	0\\
162	0\\
163	0\\
164	0\\
165	0\\
166	0\\
167	0\\
168	0\\
169	0\\
170	0\\
171	0\\
172	0\\
173	0\\
174	0\\
175	0\\
176	0\\
177	0\\
178	0\\
179	0\\
180	0\\
181	0\\
182	0\\
183	0\\
184	0\\
185	0\\
186	0\\
187	0\\
188	0\\
189	0\\
190	0\\
191	0\\
192	0\\
193	0\\
194	0\\
195	0\\
196	0\\
197	0\\
198	0\\
199	0\\
200	0\\
201	0\\
202	0\\
203	0\\
204	0\\
205	0\\
206	0\\
207	0\\
208	0\\
209	0\\
210	0\\
211	0\\
212	0\\
213	0\\
214	0\\
215	0\\
216	0\\
217	0\\
218	0\\
219	0\\
220	0\\
221	0\\
222	0\\
223	0\\
224	0\\
225	0\\
226	0\\
227	0\\
228	0\\
229	0\\
230	0\\
231	0\\
232	0\\
233	0\\
234	0\\
235	0\\
236	0\\
237	0\\
238	0\\
239	0\\
240	0\\
241	0\\
242	0\\
243	0\\
244	0\\
245	0\\
246	0\\
247	0\\
248	0\\
249	0\\
250	0\\
251	0\\
252	0\\
253	0\\
254	0\\
255	0\\
256	0\\
257	0\\
258	0\\
259	0\\
260	0\\
261	0\\
262	0\\
263	0\\
264	0\\
265	0\\
266	0\\
267	0\\
268	0\\
269	0\\
270	0\\
271	0\\
272	0\\
273	0\\
274	0\\
275	0\\
276	0\\
277	0\\
278	0\\
279	0\\
280	0\\
281	0\\
282	0\\
283	0\\
284	0\\
285	0\\
286	0\\
287	0\\
288	0\\
289	0\\
290	0\\
291	0\\
292	0\\
293	0\\
294	0\\
295	0\\
296	0\\
297	0\\
298	0\\
299	0\\
300	0\\
301	0\\
302	0\\
303	0\\
304	0\\
305	0\\
306	0\\
307	0\\
308	0\\
309	0\\
310	0\\
311	0\\
312	0\\
313	0\\
314	0\\
315	0\\
316	0\\
317	0\\
318	0\\
319	0\\
320	0\\
321	0\\
322	0\\
323	0\\
324	0\\
325	0\\
326	0\\
327	0\\
328	0\\
329	0\\
330	0\\
331	0\\
332	0\\
333	0\\
334	0\\
335	0\\
336	0\\
337	0\\
338	0\\
339	0\\
340	0\\
341	0\\
342	0\\
343	0\\
344	0\\
345	0\\
346	0\\
347	0\\
348	0\\
349	0\\
350	0\\
351	0\\
352	0\\
353	0\\
354	0\\
355	0\\
356	0\\
357	0\\
358	0\\
359	0\\
360	0\\
361	0\\
362	0\\
363	0\\
364	0\\
365	0\\
366	0\\
367	0\\
368	0\\
369	0\\
370	0\\
371	0\\
372	0\\
373	0\\
374	0\\
375	0\\
376	0\\
377	0\\
378	0\\
379	0\\
380	0\\
381	0\\
382	0\\
383	0\\
384	0\\
385	0\\
386	0\\
387	0\\
388	0\\
389	0\\
390	0\\
391	0\\
392	0\\
393	0\\
394	0\\
395	0\\
396	0\\
397	0\\
398	0\\
399	0\\
400	0\\
401	0\\
402	0\\
403	0\\
404	0\\
405	0\\
406	0\\
407	0\\
408	0\\
409	0\\
410	0\\
411	0\\
412	0\\
413	0\\
414	0\\
415	0\\
416	0\\
417	0\\
418	0\\
419	0\\
420	0\\
421	0\\
422	0\\
423	0\\
424	0\\
425	0\\
426	0\\
427	0\\
428	0\\
429	0\\
430	0\\
431	0\\
432	0\\
433	0\\
434	0\\
435	0\\
436	0\\
437	0\\
438	0\\
439	0\\
440	0\\
441	0\\
442	0\\
443	0\\
444	0\\
445	0\\
446	0\\
447	0\\
448	0\\
449	0\\
450	0\\
451	0\\
452	0\\
453	0\\
454	0\\
455	0\\
456	0\\
457	0\\
458	0\\
459	0\\
460	0\\
461	0\\
462	0\\
463	0\\
464	0\\
465	0\\
466	0\\
467	0\\
468	0\\
469	0\\
470	0\\
471	0\\
472	0\\
473	0\\
474	0\\
475	0\\
476	0\\
477	0\\
478	0\\
479	0\\
480	0\\
481	0\\
482	0\\
483	0\\
484	0\\
485	0\\
486	0\\
487	0\\
488	0\\
489	0\\
490	0\\
491	0\\
492	0\\
493	0\\
494	0\\
495	0\\
496	0\\
497	0\\
498	0\\
499	0\\
500	0\\
501	0\\
502	0\\
503	0\\
504	0\\
505	0\\
506	0\\
507	0\\
508	0\\
509	0\\
510	0\\
511	0\\
512	0\\
513	0\\
514	0\\
515	0\\
516	0\\
517	0\\
518	0\\
519	0\\
520	0\\
521	0\\
522	0\\
523	0\\
524	0\\
525	0\\
526	0\\
527	0\\
528	0\\
529	0\\
530	0\\
531	0\\
532	0\\
533	0\\
534	0\\
535	0\\
536	0\\
537	0\\
538	0\\
539	0\\
540	0\\
541	6.76881387420085e-06\\
542	3.30164558502364e-05\\
543	5.98611904804831e-05\\
544	8.7334079349987e-05\\
545	0.000115457856912956\\
546	0.000144277166012184\\
547	0.000173793755482025\\
548	0.000203981423139992\\
549	0.000234866214895428\\
550	0.000266443042311468\\
551	0.00029876073672643\\
552	0.000331847287909978\\
553	0.000365721898928622\\
554	0.000400391322970895\\
555	0.000435922616984166\\
556	0.000472350252675263\\
557	0.000509932211154596\\
558	0.000548310927633405\\
559	0.000587456131965322\\
560	0.000627403331006158\\
561	0.000668274334551636\\
562	0.000710017142892241\\
563	0.000752802951629782\\
564	0.000796728665476221\\
565	0.000841860817331624\\
566	0.000888906921322659\\
567	0.000937293589642778\\
568	0.000985960275563987\\
569	0.0010352558317612\\
570	0.00108500923535093\\
571	0.00122235714586162\\
572	0.00157519138430146\\
573	0.0018518308403065\\
574	0.00208025282367438\\
575	0.00216759714565874\\
576	0.00224359741223466\\
577	0.0023196168261704\\
578	0.00239690130315676\\
579	0.00247568128212639\\
580	0.00255604754780296\\
581	0.00263805935698486\\
582	0.00272178091061339\\
583	0.00280727759029156\\
584	0.00289461924949013\\
585	0.00298388075192783\\
586	0.00307514421637755\\
587	0.00316850324266257\\
588	0.00326407412228818\\
589	0.00336202479534929\\
590	0.00346265165082637\\
591	0.00356658418066895\\
592	0.00367533141522135\\
593	0.00379274227372929\\
594	0.0039289098819834\\
595	0.00411043972693672\\
596	0.00440815817174734\\
597	0.00501118703192877\\
598	0.00642488516645657\\
599	0\\
600	0\\
};
\addplot [color=mycolor8,solid,forget plot]
  table[row sep=crcr]{%
1	0\\
2	0\\
3	0\\
4	0\\
5	0\\
6	0\\
7	0\\
8	0\\
9	0\\
10	0\\
11	0\\
12	0\\
13	0\\
14	0\\
15	0\\
16	0\\
17	0\\
18	0\\
19	0\\
20	0\\
21	0\\
22	0\\
23	0\\
24	0\\
25	0\\
26	0\\
27	0\\
28	0\\
29	0\\
30	0\\
31	0\\
32	0\\
33	0\\
34	0\\
35	0\\
36	0\\
37	0\\
38	0\\
39	0\\
40	0\\
41	0\\
42	0\\
43	0\\
44	0\\
45	0\\
46	0\\
47	0\\
48	0\\
49	0\\
50	0\\
51	0\\
52	0\\
53	0\\
54	0\\
55	0\\
56	0\\
57	0\\
58	0\\
59	0\\
60	0\\
61	0\\
62	0\\
63	0\\
64	0\\
65	0\\
66	0\\
67	0\\
68	0\\
69	0\\
70	0\\
71	0\\
72	0\\
73	0\\
74	0\\
75	0\\
76	0\\
77	0\\
78	0\\
79	0\\
80	0\\
81	0\\
82	0\\
83	0\\
84	0\\
85	0\\
86	0\\
87	0\\
88	0\\
89	0\\
90	0\\
91	0\\
92	0\\
93	0\\
94	0\\
95	0\\
96	0\\
97	0\\
98	0\\
99	0\\
100	0\\
101	0\\
102	0\\
103	0\\
104	0\\
105	0\\
106	0\\
107	0\\
108	0\\
109	0\\
110	0\\
111	0\\
112	0\\
113	0\\
114	0\\
115	0\\
116	0\\
117	0\\
118	0\\
119	0\\
120	0\\
121	0\\
122	0\\
123	0\\
124	0\\
125	0\\
126	0\\
127	0\\
128	0\\
129	0\\
130	0\\
131	0\\
132	0\\
133	0\\
134	0\\
135	0\\
136	0\\
137	0\\
138	0\\
139	0\\
140	0\\
141	0\\
142	0\\
143	0\\
144	0\\
145	0\\
146	0\\
147	0\\
148	0\\
149	0\\
150	0\\
151	0\\
152	0\\
153	0\\
154	0\\
155	0\\
156	0\\
157	0\\
158	0\\
159	0\\
160	0\\
161	0\\
162	0\\
163	0\\
164	0\\
165	0\\
166	0\\
167	0\\
168	0\\
169	0\\
170	0\\
171	0\\
172	0\\
173	0\\
174	0\\
175	0\\
176	0\\
177	0\\
178	0\\
179	0\\
180	0\\
181	0\\
182	0\\
183	0\\
184	0\\
185	0\\
186	0\\
187	0\\
188	0\\
189	0\\
190	0\\
191	0\\
192	0\\
193	0\\
194	0\\
195	0\\
196	0\\
197	0\\
198	0\\
199	0\\
200	0\\
201	0\\
202	0\\
203	0\\
204	0\\
205	0\\
206	0\\
207	0\\
208	0\\
209	0\\
210	0\\
211	0\\
212	0\\
213	0\\
214	0\\
215	0\\
216	0\\
217	0\\
218	0\\
219	0\\
220	0\\
221	0\\
222	0\\
223	0\\
224	0\\
225	0\\
226	0\\
227	0\\
228	0\\
229	0\\
230	0\\
231	0\\
232	0\\
233	0\\
234	0\\
235	0\\
236	0\\
237	0\\
238	0\\
239	0\\
240	0\\
241	0\\
242	0\\
243	0\\
244	0\\
245	0\\
246	0\\
247	0\\
248	0\\
249	0\\
250	0\\
251	0\\
252	0\\
253	0\\
254	0\\
255	0\\
256	0\\
257	0\\
258	0\\
259	0\\
260	0\\
261	0\\
262	0\\
263	0\\
264	0\\
265	0\\
266	0\\
267	0\\
268	0\\
269	0\\
270	0\\
271	0\\
272	0\\
273	0\\
274	0\\
275	0\\
276	0\\
277	0\\
278	0\\
279	0\\
280	0\\
281	0\\
282	0\\
283	0\\
284	0\\
285	0\\
286	0\\
287	0\\
288	0\\
289	0\\
290	0\\
291	0\\
292	0\\
293	0\\
294	0\\
295	0\\
296	0\\
297	0\\
298	0\\
299	0\\
300	0\\
301	0\\
302	0\\
303	0\\
304	0\\
305	0\\
306	0\\
307	0\\
308	0\\
309	0\\
310	0\\
311	0\\
312	0\\
313	0\\
314	0\\
315	0\\
316	0\\
317	0\\
318	0\\
319	0\\
320	0\\
321	0\\
322	0\\
323	0\\
324	0\\
325	0\\
326	0\\
327	0\\
328	0\\
329	0\\
330	0\\
331	0\\
332	0\\
333	0\\
334	0\\
335	0\\
336	0\\
337	0\\
338	0\\
339	0\\
340	0\\
341	0\\
342	0\\
343	0\\
344	0\\
345	0\\
346	0\\
347	0\\
348	0\\
349	0\\
350	0\\
351	0\\
352	0\\
353	0\\
354	0\\
355	0\\
356	0\\
357	0\\
358	0\\
359	0\\
360	0\\
361	0\\
362	0\\
363	0\\
364	0\\
365	0\\
366	0\\
367	0\\
368	0\\
369	0\\
370	0\\
371	0\\
372	0\\
373	0\\
374	0\\
375	0\\
376	0\\
377	0\\
378	0\\
379	0\\
380	0\\
381	0\\
382	0\\
383	0\\
384	0\\
385	0\\
386	0\\
387	0\\
388	0\\
389	0\\
390	0\\
391	0\\
392	0\\
393	0\\
394	0\\
395	0\\
396	0\\
397	0\\
398	0\\
399	0\\
400	0\\
401	0\\
402	0\\
403	0\\
404	0\\
405	0\\
406	0\\
407	0\\
408	0\\
409	0\\
410	0\\
411	0\\
412	0\\
413	0\\
414	0\\
415	0\\
416	0\\
417	0\\
418	0\\
419	0\\
420	0\\
421	0\\
422	0\\
423	0\\
424	0\\
425	0\\
426	0\\
427	0\\
428	0\\
429	0\\
430	0\\
431	0\\
432	0\\
433	0\\
434	0\\
435	0\\
436	0\\
437	0\\
438	0\\
439	0\\
440	0\\
441	0\\
442	0\\
443	0\\
444	0\\
445	0\\
446	0\\
447	0\\
448	0\\
449	0\\
450	0\\
451	0\\
452	0\\
453	0\\
454	0\\
455	0\\
456	0\\
457	0\\
458	0\\
459	0\\
460	0\\
461	0\\
462	0\\
463	0\\
464	0\\
465	0\\
466	0\\
467	0\\
468	0\\
469	0\\
470	0\\
471	0\\
472	0\\
473	0\\
474	0\\
475	0\\
476	0\\
477	0\\
478	0\\
479	0\\
480	0\\
481	0\\
482	0\\
483	0\\
484	0\\
485	0\\
486	0\\
487	0\\
488	0\\
489	0\\
490	0\\
491	0\\
492	0\\
493	0\\
494	0\\
495	0\\
496	0\\
497	0\\
498	0\\
499	0\\
500	0\\
501	0\\
502	0\\
503	0\\
504	0\\
505	0\\
506	0\\
507	0\\
508	0\\
509	0\\
510	0\\
511	0\\
512	0\\
513	0\\
514	0\\
515	0\\
516	0\\
517	0\\
518	0\\
519	0\\
520	0\\
521	0\\
522	0\\
523	0\\
524	0\\
525	0\\
526	0\\
527	0\\
528	0\\
529	0\\
530	0\\
531	0\\
532	0\\
533	0\\
534	0\\
535	0\\
536	0\\
537	0\\
538	0\\
539	0\\
540	0\\
541	0\\
542	1.77994563498494e-05\\
543	4.44690418034569e-05\\
544	7.17504882016815e-05\\
545	9.96653124370045e-05\\
546	0.000128247224210947\\
547	0.000157508764958893\\
548	0.000187477130071588\\
549	0.000218174077853842\\
550	0.00024965718467097\\
551	0.000281882535055856\\
552	0.000314854645582878\\
553	0.000348597040202349\\
554	0.000383112689282817\\
555	0.00041846178624356\\
556	0.000454668625667846\\
557	0.000491753884948208\\
558	0.000529732799945901\\
559	0.000568792410221848\\
560	0.000608906511575984\\
561	0.000649904672830707\\
562	0.000691724311605649\\
563	0.000734489579471478\\
564	0.000778277104886136\\
565	0.00082303473115782\\
566	0.000868959770003612\\
567	0.000916145326157647\\
568	0.000965414946849161\\
569	0.0010158882089616\\
570	0.00106677817567325\\
571	0.00111839899996879\\
572	0.00117070764637285\\
573	0.00145155287937048\\
574	0.00177963609047075\\
575	0.00203805474797816\\
576	0.00222786492014083\\
577	0.00231696253384747\\
578	0.00239633649011414\\
579	0.00247552515937604\\
580	0.00255597390481131\\
581	0.0026380307249917\\
582	0.00272176645872677\\
583	0.00280727066893297\\
584	0.00289461552253682\\
585	0.00298387896151858\\
586	0.00307514331064204\\
587	0.00316850293601066\\
588	0.00326407407006435\\
589	0.00336202479534929\\
590	0.00346265165082637\\
591	0.00356658418066895\\
592	0.00367533141522135\\
593	0.0037927422737293\\
594	0.00392890988198341\\
595	0.00411043972693672\\
596	0.00440815817174735\\
597	0.00501118703192877\\
598	0.00642488516645657\\
599	0\\
600	0\\
};
\addplot [color=blue!25!mycolor7,solid,forget plot]
  table[row sep=crcr]{%
1	0\\
2	0\\
3	0\\
4	0\\
5	0\\
6	0\\
7	0\\
8	0\\
9	0\\
10	0\\
11	0\\
12	0\\
13	0\\
14	0\\
15	0\\
16	0\\
17	0\\
18	0\\
19	0\\
20	0\\
21	0\\
22	0\\
23	0\\
24	0\\
25	0\\
26	0\\
27	0\\
28	0\\
29	0\\
30	0\\
31	0\\
32	0\\
33	0\\
34	0\\
35	0\\
36	0\\
37	0\\
38	0\\
39	0\\
40	0\\
41	0\\
42	0\\
43	0\\
44	0\\
45	0\\
46	0\\
47	0\\
48	0\\
49	0\\
50	0\\
51	0\\
52	0\\
53	0\\
54	0\\
55	0\\
56	0\\
57	0\\
58	0\\
59	0\\
60	0\\
61	0\\
62	0\\
63	0\\
64	0\\
65	0\\
66	0\\
67	0\\
68	0\\
69	0\\
70	0\\
71	0\\
72	0\\
73	0\\
74	0\\
75	0\\
76	0\\
77	0\\
78	0\\
79	0\\
80	0\\
81	0\\
82	0\\
83	0\\
84	0\\
85	0\\
86	0\\
87	0\\
88	0\\
89	0\\
90	0\\
91	0\\
92	0\\
93	0\\
94	0\\
95	0\\
96	0\\
97	0\\
98	0\\
99	0\\
100	0\\
101	0\\
102	0\\
103	0\\
104	0\\
105	0\\
106	0\\
107	0\\
108	0\\
109	0\\
110	0\\
111	0\\
112	0\\
113	0\\
114	0\\
115	0\\
116	0\\
117	0\\
118	0\\
119	0\\
120	0\\
121	0\\
122	0\\
123	0\\
124	0\\
125	0\\
126	0\\
127	0\\
128	0\\
129	0\\
130	0\\
131	0\\
132	0\\
133	0\\
134	0\\
135	0\\
136	0\\
137	0\\
138	0\\
139	0\\
140	0\\
141	0\\
142	0\\
143	0\\
144	0\\
145	0\\
146	0\\
147	0\\
148	0\\
149	0\\
150	0\\
151	0\\
152	0\\
153	0\\
154	0\\
155	0\\
156	0\\
157	0\\
158	0\\
159	0\\
160	0\\
161	0\\
162	0\\
163	0\\
164	0\\
165	0\\
166	0\\
167	0\\
168	0\\
169	0\\
170	0\\
171	0\\
172	0\\
173	0\\
174	0\\
175	0\\
176	0\\
177	0\\
178	0\\
179	0\\
180	0\\
181	0\\
182	0\\
183	0\\
184	0\\
185	0\\
186	0\\
187	0\\
188	0\\
189	0\\
190	0\\
191	0\\
192	0\\
193	0\\
194	0\\
195	0\\
196	0\\
197	0\\
198	0\\
199	0\\
200	0\\
201	0\\
202	0\\
203	0\\
204	0\\
205	0\\
206	0\\
207	0\\
208	0\\
209	0\\
210	0\\
211	0\\
212	0\\
213	0\\
214	0\\
215	0\\
216	0\\
217	0\\
218	0\\
219	0\\
220	0\\
221	0\\
222	0\\
223	0\\
224	0\\
225	0\\
226	0\\
227	0\\
228	0\\
229	0\\
230	0\\
231	0\\
232	0\\
233	0\\
234	0\\
235	0\\
236	0\\
237	0\\
238	0\\
239	0\\
240	0\\
241	0\\
242	0\\
243	0\\
244	0\\
245	0\\
246	0\\
247	0\\
248	0\\
249	0\\
250	0\\
251	0\\
252	0\\
253	0\\
254	0\\
255	0\\
256	0\\
257	0\\
258	0\\
259	0\\
260	0\\
261	0\\
262	0\\
263	0\\
264	0\\
265	0\\
266	0\\
267	0\\
268	0\\
269	0\\
270	0\\
271	0\\
272	0\\
273	0\\
274	0\\
275	0\\
276	0\\
277	0\\
278	0\\
279	0\\
280	0\\
281	0\\
282	0\\
283	0\\
284	0\\
285	0\\
286	0\\
287	0\\
288	0\\
289	0\\
290	0\\
291	0\\
292	0\\
293	0\\
294	0\\
295	0\\
296	0\\
297	0\\
298	0\\
299	0\\
300	0\\
301	0\\
302	0\\
303	0\\
304	0\\
305	0\\
306	0\\
307	0\\
308	0\\
309	0\\
310	0\\
311	0\\
312	0\\
313	0\\
314	0\\
315	0\\
316	0\\
317	0\\
318	0\\
319	0\\
320	0\\
321	0\\
322	0\\
323	0\\
324	0\\
325	0\\
326	0\\
327	0\\
328	0\\
329	0\\
330	0\\
331	0\\
332	0\\
333	0\\
334	0\\
335	0\\
336	0\\
337	0\\
338	0\\
339	0\\
340	0\\
341	0\\
342	0\\
343	0\\
344	0\\
345	0\\
346	0\\
347	0\\
348	0\\
349	0\\
350	0\\
351	0\\
352	0\\
353	0\\
354	0\\
355	0\\
356	0\\
357	0\\
358	0\\
359	0\\
360	0\\
361	0\\
362	0\\
363	0\\
364	0\\
365	0\\
366	0\\
367	0\\
368	0\\
369	0\\
370	0\\
371	0\\
372	0\\
373	0\\
374	0\\
375	0\\
376	0\\
377	0\\
378	0\\
379	0\\
380	0\\
381	0\\
382	0\\
383	0\\
384	0\\
385	0\\
386	0\\
387	0\\
388	0\\
389	0\\
390	0\\
391	0\\
392	0\\
393	0\\
394	0\\
395	0\\
396	0\\
397	0\\
398	0\\
399	0\\
400	0\\
401	0\\
402	0\\
403	0\\
404	0\\
405	0\\
406	0\\
407	0\\
408	0\\
409	0\\
410	0\\
411	0\\
412	0\\
413	0\\
414	0\\
415	0\\
416	0\\
417	0\\
418	0\\
419	0\\
420	0\\
421	0\\
422	0\\
423	0\\
424	0\\
425	0\\
426	0\\
427	0\\
428	0\\
429	0\\
430	0\\
431	0\\
432	0\\
433	0\\
434	0\\
435	0\\
436	0\\
437	0\\
438	0\\
439	0\\
440	0\\
441	0\\
442	0\\
443	0\\
444	0\\
445	0\\
446	0\\
447	0\\
448	0\\
449	0\\
450	0\\
451	0\\
452	0\\
453	0\\
454	0\\
455	0\\
456	0\\
457	0\\
458	0\\
459	0\\
460	0\\
461	0\\
462	0\\
463	0\\
464	0\\
465	0\\
466	0\\
467	0\\
468	0\\
469	0\\
470	0\\
471	0\\
472	0\\
473	0\\
474	0\\
475	0\\
476	0\\
477	0\\
478	0\\
479	0\\
480	0\\
481	0\\
482	0\\
483	0\\
484	0\\
485	0\\
486	0\\
487	0\\
488	0\\
489	0\\
490	0\\
491	0\\
492	0\\
493	0\\
494	0\\
495	0\\
496	0\\
497	0\\
498	0\\
499	0\\
500	0\\
501	0\\
502	0\\
503	0\\
504	0\\
505	0\\
506	0\\
507	0\\
508	0\\
509	0\\
510	0\\
511	0\\
512	0\\
513	0\\
514	0\\
515	0\\
516	0\\
517	0\\
518	0\\
519	0\\
520	0\\
521	0\\
522	0\\
523	0\\
524	0\\
525	0\\
526	0\\
527	0\\
528	0\\
529	0\\
530	0\\
531	0\\
532	0\\
533	0\\
534	0\\
535	0\\
536	0\\
537	0\\
538	0\\
539	0\\
540	0\\
541	0\\
542	6.11660056486632e-07\\
543	2.701048695734e-05\\
544	5.40246794947436e-05\\
545	8.16731202693553e-05\\
546	0.000109974832324423\\
547	0.000138958562454918\\
548	0.00016865085298941\\
549	0.000199069928818032\\
550	0.000230234589378527\\
551	0.00026216528174636\\
552	0.000294879489820562\\
553	0.000328399703159835\\
554	0.000362790215985211\\
555	0.00039799532627114\\
556	0.000434040398948903\\
557	0.000470950458242298\\
558	0.000508722127414138\\
559	0.00054742699063001\\
560	0.000587094676229419\\
561	0.000627754501768088\\
562	0.000669596686142599\\
563	0.000712473576967077\\
564	0.000756336153151856\\
565	0.000801114955820075\\
566	0.000846955852793247\\
567	0.000893934541161798\\
568	0.000942022580164056\\
569	0.00099137139308498\\
570	0.00104281851857678\\
571	0.00109557130842463\\
572	0.00114898652968873\\
573	0.00120314728583361\\
574	0.00131396473799518\\
575	0.00163850815209253\\
576	0.00193799204557724\\
577	0.00219526366421932\\
578	0.00237623984097034\\
579	0.00247139029517359\\
580	0.00255490860490003\\
581	0.00263752357935765\\
582	0.00272157874905703\\
583	0.00280717633476708\\
584	0.00289457157399351\\
585	0.00298385506818827\\
586	0.0030751319777398\\
587	0.00316849697477764\\
588	0.00326407198718924\\
589	0.00336202443091807\\
590	0.00346265165082637\\
591	0.00356658418066895\\
592	0.00367533141522134\\
593	0.00379274227372929\\
594	0.00392890988198339\\
595	0.00411043972693671\\
596	0.00440815817174734\\
597	0.00501118703192877\\
598	0.00642488516645657\\
599	0\\
600	0\\
};
\addplot [color=mycolor9,solid,forget plot]
  table[row sep=crcr]{%
1	0\\
2	0\\
3	0\\
4	0\\
5	0\\
6	0\\
7	0\\
8	0\\
9	0\\
10	0\\
11	0\\
12	0\\
13	0\\
14	0\\
15	0\\
16	0\\
17	0\\
18	0\\
19	0\\
20	0\\
21	0\\
22	0\\
23	0\\
24	0\\
25	0\\
26	0\\
27	0\\
28	0\\
29	0\\
30	0\\
31	0\\
32	0\\
33	0\\
34	0\\
35	0\\
36	0\\
37	0\\
38	0\\
39	0\\
40	0\\
41	0\\
42	0\\
43	0\\
44	0\\
45	0\\
46	0\\
47	0\\
48	0\\
49	0\\
50	0\\
51	0\\
52	0\\
53	0\\
54	0\\
55	0\\
56	0\\
57	0\\
58	0\\
59	0\\
60	0\\
61	0\\
62	0\\
63	0\\
64	0\\
65	0\\
66	0\\
67	0\\
68	0\\
69	0\\
70	0\\
71	0\\
72	0\\
73	0\\
74	0\\
75	0\\
76	0\\
77	0\\
78	0\\
79	0\\
80	0\\
81	0\\
82	0\\
83	0\\
84	0\\
85	0\\
86	0\\
87	0\\
88	0\\
89	0\\
90	0\\
91	0\\
92	0\\
93	0\\
94	0\\
95	0\\
96	0\\
97	0\\
98	0\\
99	0\\
100	0\\
101	0\\
102	0\\
103	0\\
104	0\\
105	0\\
106	0\\
107	0\\
108	0\\
109	0\\
110	0\\
111	0\\
112	0\\
113	0\\
114	0\\
115	0\\
116	0\\
117	0\\
118	0\\
119	0\\
120	0\\
121	0\\
122	0\\
123	0\\
124	0\\
125	0\\
126	0\\
127	0\\
128	0\\
129	0\\
130	0\\
131	0\\
132	0\\
133	0\\
134	0\\
135	0\\
136	0\\
137	0\\
138	0\\
139	0\\
140	0\\
141	0\\
142	0\\
143	0\\
144	0\\
145	0\\
146	0\\
147	0\\
148	0\\
149	0\\
150	0\\
151	0\\
152	0\\
153	0\\
154	0\\
155	0\\
156	0\\
157	0\\
158	0\\
159	0\\
160	0\\
161	0\\
162	0\\
163	0\\
164	0\\
165	0\\
166	0\\
167	0\\
168	0\\
169	0\\
170	0\\
171	0\\
172	0\\
173	0\\
174	0\\
175	0\\
176	0\\
177	0\\
178	0\\
179	0\\
180	0\\
181	0\\
182	0\\
183	0\\
184	0\\
185	0\\
186	0\\
187	0\\
188	0\\
189	0\\
190	0\\
191	0\\
192	0\\
193	0\\
194	0\\
195	0\\
196	0\\
197	0\\
198	0\\
199	0\\
200	0\\
201	0\\
202	0\\
203	0\\
204	0\\
205	0\\
206	0\\
207	0\\
208	0\\
209	0\\
210	0\\
211	0\\
212	0\\
213	0\\
214	0\\
215	0\\
216	0\\
217	0\\
218	0\\
219	0\\
220	0\\
221	0\\
222	0\\
223	0\\
224	0\\
225	0\\
226	0\\
227	0\\
228	0\\
229	0\\
230	0\\
231	0\\
232	0\\
233	0\\
234	0\\
235	0\\
236	0\\
237	0\\
238	0\\
239	0\\
240	0\\
241	0\\
242	0\\
243	0\\
244	0\\
245	0\\
246	0\\
247	0\\
248	0\\
249	0\\
250	0\\
251	0\\
252	0\\
253	0\\
254	0\\
255	0\\
256	0\\
257	0\\
258	0\\
259	0\\
260	0\\
261	0\\
262	0\\
263	0\\
264	0\\
265	0\\
266	0\\
267	0\\
268	0\\
269	0\\
270	0\\
271	0\\
272	0\\
273	0\\
274	0\\
275	0\\
276	0\\
277	0\\
278	0\\
279	0\\
280	0\\
281	0\\
282	0\\
283	0\\
284	0\\
285	0\\
286	0\\
287	0\\
288	0\\
289	0\\
290	0\\
291	0\\
292	0\\
293	0\\
294	0\\
295	0\\
296	0\\
297	0\\
298	0\\
299	0\\
300	0\\
301	0\\
302	0\\
303	0\\
304	0\\
305	0\\
306	0\\
307	0\\
308	0\\
309	0\\
310	0\\
311	0\\
312	0\\
313	0\\
314	0\\
315	0\\
316	0\\
317	0\\
318	0\\
319	0\\
320	0\\
321	0\\
322	0\\
323	0\\
324	0\\
325	0\\
326	0\\
327	0\\
328	0\\
329	0\\
330	0\\
331	0\\
332	0\\
333	0\\
334	0\\
335	0\\
336	0\\
337	0\\
338	0\\
339	0\\
340	0\\
341	0\\
342	0\\
343	0\\
344	0\\
345	0\\
346	0\\
347	0\\
348	0\\
349	0\\
350	0\\
351	0\\
352	0\\
353	0\\
354	0\\
355	0\\
356	0\\
357	0\\
358	0\\
359	0\\
360	0\\
361	0\\
362	0\\
363	0\\
364	0\\
365	0\\
366	0\\
367	0\\
368	0\\
369	0\\
370	0\\
371	0\\
372	0\\
373	0\\
374	0\\
375	0\\
376	0\\
377	0\\
378	0\\
379	0\\
380	0\\
381	0\\
382	0\\
383	0\\
384	0\\
385	0\\
386	0\\
387	0\\
388	0\\
389	0\\
390	0\\
391	0\\
392	0\\
393	0\\
394	0\\
395	0\\
396	0\\
397	0\\
398	0\\
399	0\\
400	0\\
401	0\\
402	0\\
403	0\\
404	0\\
405	0\\
406	0\\
407	0\\
408	0\\
409	0\\
410	0\\
411	0\\
412	0\\
413	0\\
414	0\\
415	0\\
416	0\\
417	0\\
418	0\\
419	0\\
420	0\\
421	0\\
422	0\\
423	0\\
424	0\\
425	0\\
426	0\\
427	0\\
428	0\\
429	0\\
430	0\\
431	0\\
432	0\\
433	0\\
434	0\\
435	0\\
436	0\\
437	0\\
438	0\\
439	0\\
440	0\\
441	0\\
442	0\\
443	0\\
444	0\\
445	0\\
446	0\\
447	0\\
448	0\\
449	0\\
450	0\\
451	0\\
452	0\\
453	0\\
454	0\\
455	0\\
456	0\\
457	0\\
458	0\\
459	0\\
460	0\\
461	0\\
462	0\\
463	0\\
464	0\\
465	0\\
466	0\\
467	0\\
468	0\\
469	0\\
470	0\\
471	0\\
472	0\\
473	0\\
474	0\\
475	0\\
476	0\\
477	0\\
478	0\\
479	0\\
480	0\\
481	0\\
482	0\\
483	0\\
484	0\\
485	0\\
486	0\\
487	0\\
488	0\\
489	0\\
490	0\\
491	0\\
492	0\\
493	0\\
494	0\\
495	0\\
496	0\\
497	0\\
498	0\\
499	0\\
500	0\\
501	0\\
502	0\\
503	0\\
504	0\\
505	0\\
506	0\\
507	0\\
508	0\\
509	0\\
510	0\\
511	0\\
512	0\\
513	0\\
514	0\\
515	0\\
516	0\\
517	0\\
518	0\\
519	0\\
520	0\\
521	0\\
522	0\\
523	0\\
524	0\\
525	0\\
526	0\\
527	0\\
528	0\\
529	0\\
530	0\\
531	0\\
532	0\\
533	0\\
534	0\\
535	0\\
536	0\\
537	0\\
538	0\\
539	0\\
540	0\\
541	0\\
542	0\\
543	7.19050061245648e-06\\
544	3.38811395008324e-05\\
545	6.11995138419879e-05\\
546	8.91653520785335e-05\\
547	0.000117798896383763\\
548	0.0001471207591222\\
549	0.000177163848134191\\
550	0.000207954234817611\\
551	0.000239511711651359\\
552	0.000271863001773644\\
553	0.000305030078629033\\
554	0.000339024023324224\\
555	0.00037386790771121\\
556	0.000409585805568485\\
557	0.000446207799016567\\
558	0.00048378405404937\\
559	0.000522274282432132\\
560	0.000561707740752937\\
561	0.000602117791849755\\
562	0.000643509855471186\\
563	0.000685944198495142\\
564	0.000729459648213524\\
565	0.000774285458757708\\
566	0.000820194696357515\\
567	0.000867170853535688\\
568	0.000915208070350155\\
569	0.00096440285599\\
570	0.00101486356606715\\
571	0.00106662268470533\\
572	0.00112017026618547\\
573	0.00117541917271621\\
574	0.00123165468208164\\
575	0.00128853771559882\\
576	0.00145855591256311\\
577	0.00177609907745312\\
578	0.00206426274766966\\
579	0.00232087238374528\\
580	0.00252475298378954\\
581	0.00263031863751007\\
582	0.00271809688623168\\
583	0.00280595565159673\\
584	0.00289395943431418\\
585	0.00298357938090699\\
586	0.00307498027389778\\
587	0.00316842647759219\\
588	0.00326403331960031\\
589	0.00336201047031448\\
590	0.00346264913972014\\
591	0.00356658418066895\\
592	0.00367533141522134\\
593	0.00379274227372928\\
594	0.00392890988198339\\
595	0.00411043972693671\\
596	0.00440815817174734\\
597	0.00501118703192877\\
598	0.00642488516645657\\
599	0\\
600	0\\
};
\addplot [color=blue!50!mycolor7,solid,forget plot]
  table[row sep=crcr]{%
1	0\\
2	0\\
3	0\\
4	0\\
5	0\\
6	0\\
7	0\\
8	0\\
9	0\\
10	0\\
11	0\\
12	0\\
13	0\\
14	0\\
15	0\\
16	0\\
17	0\\
18	0\\
19	0\\
20	0\\
21	0\\
22	0\\
23	0\\
24	0\\
25	0\\
26	0\\
27	0\\
28	0\\
29	0\\
30	0\\
31	0\\
32	0\\
33	0\\
34	0\\
35	0\\
36	0\\
37	0\\
38	0\\
39	0\\
40	0\\
41	0\\
42	0\\
43	0\\
44	0\\
45	0\\
46	0\\
47	0\\
48	0\\
49	0\\
50	0\\
51	0\\
52	0\\
53	0\\
54	0\\
55	0\\
56	0\\
57	0\\
58	0\\
59	0\\
60	0\\
61	0\\
62	0\\
63	0\\
64	0\\
65	0\\
66	0\\
67	0\\
68	0\\
69	0\\
70	0\\
71	0\\
72	0\\
73	0\\
74	0\\
75	0\\
76	0\\
77	0\\
78	0\\
79	0\\
80	0\\
81	0\\
82	0\\
83	0\\
84	0\\
85	0\\
86	0\\
87	0\\
88	0\\
89	0\\
90	0\\
91	0\\
92	0\\
93	0\\
94	0\\
95	0\\
96	0\\
97	0\\
98	0\\
99	0\\
100	0\\
101	0\\
102	0\\
103	0\\
104	0\\
105	0\\
106	0\\
107	0\\
108	0\\
109	0\\
110	0\\
111	0\\
112	0\\
113	0\\
114	0\\
115	0\\
116	0\\
117	0\\
118	0\\
119	0\\
120	0\\
121	0\\
122	0\\
123	0\\
124	0\\
125	0\\
126	0\\
127	0\\
128	0\\
129	0\\
130	0\\
131	0\\
132	0\\
133	0\\
134	0\\
135	0\\
136	0\\
137	0\\
138	0\\
139	0\\
140	0\\
141	0\\
142	0\\
143	0\\
144	0\\
145	0\\
146	0\\
147	0\\
148	0\\
149	0\\
150	0\\
151	0\\
152	0\\
153	0\\
154	0\\
155	0\\
156	0\\
157	0\\
158	0\\
159	0\\
160	0\\
161	0\\
162	0\\
163	0\\
164	0\\
165	0\\
166	0\\
167	0\\
168	0\\
169	0\\
170	0\\
171	0\\
172	0\\
173	0\\
174	0\\
175	0\\
176	0\\
177	0\\
178	0\\
179	0\\
180	0\\
181	0\\
182	0\\
183	0\\
184	0\\
185	0\\
186	0\\
187	0\\
188	0\\
189	0\\
190	0\\
191	0\\
192	0\\
193	0\\
194	0\\
195	0\\
196	0\\
197	0\\
198	0\\
199	0\\
200	0\\
201	0\\
202	0\\
203	0\\
204	0\\
205	0\\
206	0\\
207	0\\
208	0\\
209	0\\
210	0\\
211	0\\
212	0\\
213	0\\
214	0\\
215	0\\
216	0\\
217	0\\
218	0\\
219	0\\
220	0\\
221	0\\
222	0\\
223	0\\
224	0\\
225	0\\
226	0\\
227	0\\
228	0\\
229	0\\
230	0\\
231	0\\
232	0\\
233	0\\
234	0\\
235	0\\
236	0\\
237	0\\
238	0\\
239	0\\
240	0\\
241	0\\
242	0\\
243	0\\
244	0\\
245	0\\
246	0\\
247	0\\
248	0\\
249	0\\
250	0\\
251	0\\
252	0\\
253	0\\
254	0\\
255	0\\
256	0\\
257	0\\
258	0\\
259	0\\
260	0\\
261	0\\
262	0\\
263	0\\
264	0\\
265	0\\
266	0\\
267	0\\
268	0\\
269	0\\
270	0\\
271	0\\
272	0\\
273	0\\
274	0\\
275	0\\
276	0\\
277	0\\
278	0\\
279	0\\
280	0\\
281	0\\
282	0\\
283	0\\
284	0\\
285	0\\
286	0\\
287	0\\
288	0\\
289	0\\
290	0\\
291	0\\
292	0\\
293	0\\
294	0\\
295	0\\
296	0\\
297	0\\
298	0\\
299	0\\
300	0\\
301	0\\
302	0\\
303	0\\
304	0\\
305	0\\
306	0\\
307	0\\
308	0\\
309	0\\
310	0\\
311	0\\
312	0\\
313	0\\
314	0\\
315	0\\
316	0\\
317	0\\
318	0\\
319	0\\
320	0\\
321	0\\
322	0\\
323	0\\
324	0\\
325	0\\
326	0\\
327	0\\
328	0\\
329	0\\
330	0\\
331	0\\
332	0\\
333	0\\
334	0\\
335	0\\
336	0\\
337	0\\
338	0\\
339	0\\
340	0\\
341	0\\
342	0\\
343	0\\
344	0\\
345	0\\
346	0\\
347	0\\
348	0\\
349	0\\
350	0\\
351	0\\
352	0\\
353	0\\
354	0\\
355	0\\
356	0\\
357	0\\
358	0\\
359	0\\
360	0\\
361	0\\
362	0\\
363	0\\
364	0\\
365	0\\
366	0\\
367	0\\
368	0\\
369	0\\
370	0\\
371	0\\
372	0\\
373	0\\
374	0\\
375	0\\
376	0\\
377	0\\
378	0\\
379	0\\
380	0\\
381	0\\
382	0\\
383	0\\
384	0\\
385	0\\
386	0\\
387	0\\
388	0\\
389	0\\
390	0\\
391	0\\
392	0\\
393	0\\
394	0\\
395	0\\
396	0\\
397	0\\
398	0\\
399	0\\
400	0\\
401	0\\
402	0\\
403	0\\
404	0\\
405	0\\
406	0\\
407	0\\
408	0\\
409	0\\
410	0\\
411	0\\
412	0\\
413	0\\
414	0\\
415	0\\
416	0\\
417	0\\
418	0\\
419	0\\
420	0\\
421	0\\
422	0\\
423	0\\
424	0\\
425	0\\
426	0\\
427	0\\
428	0\\
429	0\\
430	0\\
431	0\\
432	0\\
433	0\\
434	0\\
435	0\\
436	0\\
437	0\\
438	0\\
439	0\\
440	0\\
441	0\\
442	0\\
443	0\\
444	0\\
445	0\\
446	0\\
447	0\\
448	0\\
449	0\\
450	0\\
451	0\\
452	0\\
453	0\\
454	0\\
455	0\\
456	0\\
457	0\\
458	0\\
459	0\\
460	0\\
461	0\\
462	0\\
463	0\\
464	0\\
465	0\\
466	0\\
467	0\\
468	0\\
469	0\\
470	0\\
471	0\\
472	0\\
473	0\\
474	0\\
475	0\\
476	0\\
477	0\\
478	0\\
479	0\\
480	0\\
481	0\\
482	0\\
483	0\\
484	0\\
485	0\\
486	0\\
487	0\\
488	0\\
489	0\\
490	0\\
491	0\\
492	0\\
493	0\\
494	0\\
495	0\\
496	0\\
497	0\\
498	0\\
499	0\\
500	0\\
501	0\\
502	0\\
503	0\\
504	0\\
505	0\\
506	0\\
507	0\\
508	0\\
509	0\\
510	0\\
511	0\\
512	0\\
513	0\\
514	0\\
515	0\\
516	0\\
517	0\\
518	0\\
519	0\\
520	0\\
521	0\\
522	0\\
523	0\\
524	0\\
525	0\\
526	0\\
527	0\\
528	0\\
529	0\\
530	0\\
531	0\\
532	0\\
533	0\\
534	0\\
535	0\\
536	0\\
537	0\\
538	0\\
539	0\\
540	0\\
541	0\\
542	0\\
543	0\\
544	1.05627776958363e-05\\
545	3.75304772690883e-05\\
546	6.51312718678484e-05\\
547	9.33863630317708e-05\\
548	0.000122317550313663\\
549	0.00015194702081486\\
550	0.000182308666181335\\
551	0.00021342974811385\\
552	0.000245331657737849\\
553	0.000278036116023763\\
554	0.000311565519271933\\
555	0.000345943015786943\\
556	0.000381192545748294\\
557	0.000417339269889016\\
558	0.000454422980542451\\
559	0.00049245289508931\\
560	0.00053145713940947\\
561	0.000571465034329317\\
562	0.000612547638993974\\
563	0.000654664794599318\\
564	0.000697840632010934\\
565	0.000742117586682551\\
566	0.000787524528073936\\
567	0.000834101450324763\\
568	0.000882052764535135\\
569	0.000931270793437822\\
570	0.000981668890123943\\
571	0.00103328290710104\\
572	0.00108611346496739\\
573	0.00114036010457888\\
574	0.00119618222241278\\
575	0.00125424290438048\\
576	0.00131331951945714\\
577	0.00137341947531424\\
578	0.00155650097528339\\
579	0.001863805897274\\
580	0.00215685965284502\\
581	0.00241246675897658\\
582	0.0026698348651085\\
583	0.00278212107437134\\
584	0.00288608491603381\\
585	0.00297962692672493\\
586	0.00307327221777276\\
587	0.00316747165964346\\
588	0.00326360274193222\\
589	0.00336176338818523\\
590	0.00346255683773199\\
591	0.00356656710209231\\
592	0.00367533141522134\\
593	0.00379274227372929\\
594	0.00392890988198339\\
595	0.00411043972693671\\
596	0.00440815817174734\\
597	0.00501118703192877\\
598	0.00642488516645657\\
599	0\\
600	0\\
};
\addplot [color=blue!40!mycolor9,solid,forget plot]
  table[row sep=crcr]{%
1	0\\
2	0\\
3	0\\
4	0\\
5	0\\
6	0\\
7	0\\
8	0\\
9	0\\
10	0\\
11	0\\
12	0\\
13	0\\
14	0\\
15	0\\
16	0\\
17	0\\
18	0\\
19	0\\
20	0\\
21	0\\
22	0\\
23	0\\
24	0\\
25	0\\
26	0\\
27	0\\
28	0\\
29	0\\
30	0\\
31	0\\
32	0\\
33	0\\
34	0\\
35	0\\
36	0\\
37	0\\
38	0\\
39	0\\
40	0\\
41	0\\
42	0\\
43	0\\
44	0\\
45	0\\
46	0\\
47	0\\
48	0\\
49	0\\
50	0\\
51	0\\
52	0\\
53	0\\
54	0\\
55	0\\
56	0\\
57	0\\
58	0\\
59	0\\
60	0\\
61	0\\
62	0\\
63	0\\
64	0\\
65	0\\
66	0\\
67	0\\
68	0\\
69	0\\
70	0\\
71	0\\
72	0\\
73	0\\
74	0\\
75	0\\
76	0\\
77	0\\
78	0\\
79	0\\
80	0\\
81	0\\
82	0\\
83	0\\
84	0\\
85	0\\
86	0\\
87	0\\
88	0\\
89	0\\
90	0\\
91	0\\
92	0\\
93	0\\
94	0\\
95	0\\
96	0\\
97	0\\
98	0\\
99	0\\
100	0\\
101	0\\
102	0\\
103	0\\
104	0\\
105	0\\
106	0\\
107	0\\
108	0\\
109	0\\
110	0\\
111	0\\
112	0\\
113	0\\
114	0\\
115	0\\
116	0\\
117	0\\
118	0\\
119	0\\
120	0\\
121	0\\
122	0\\
123	0\\
124	0\\
125	0\\
126	0\\
127	0\\
128	0\\
129	0\\
130	0\\
131	0\\
132	0\\
133	0\\
134	0\\
135	0\\
136	0\\
137	0\\
138	0\\
139	0\\
140	0\\
141	0\\
142	0\\
143	0\\
144	0\\
145	0\\
146	0\\
147	0\\
148	0\\
149	0\\
150	0\\
151	0\\
152	0\\
153	0\\
154	0\\
155	0\\
156	0\\
157	0\\
158	0\\
159	0\\
160	0\\
161	0\\
162	0\\
163	0\\
164	0\\
165	0\\
166	0\\
167	0\\
168	0\\
169	0\\
170	0\\
171	0\\
172	0\\
173	0\\
174	0\\
175	0\\
176	0\\
177	0\\
178	0\\
179	0\\
180	0\\
181	0\\
182	0\\
183	0\\
184	0\\
185	0\\
186	0\\
187	0\\
188	0\\
189	0\\
190	0\\
191	0\\
192	0\\
193	0\\
194	0\\
195	0\\
196	0\\
197	0\\
198	0\\
199	0\\
200	0\\
201	0\\
202	0\\
203	0\\
204	0\\
205	0\\
206	0\\
207	0\\
208	0\\
209	0\\
210	0\\
211	0\\
212	0\\
213	0\\
214	0\\
215	0\\
216	0\\
217	0\\
218	0\\
219	0\\
220	0\\
221	0\\
222	0\\
223	0\\
224	0\\
225	0\\
226	0\\
227	0\\
228	0\\
229	0\\
230	0\\
231	0\\
232	0\\
233	0\\
234	0\\
235	0\\
236	0\\
237	0\\
238	0\\
239	0\\
240	0\\
241	0\\
242	0\\
243	0\\
244	0\\
245	0\\
246	0\\
247	0\\
248	0\\
249	0\\
250	0\\
251	0\\
252	0\\
253	0\\
254	0\\
255	0\\
256	0\\
257	0\\
258	0\\
259	0\\
260	0\\
261	0\\
262	0\\
263	0\\
264	0\\
265	0\\
266	0\\
267	0\\
268	0\\
269	0\\
270	0\\
271	0\\
272	0\\
273	0\\
274	0\\
275	0\\
276	0\\
277	0\\
278	0\\
279	0\\
280	0\\
281	0\\
282	0\\
283	0\\
284	0\\
285	0\\
286	0\\
287	0\\
288	0\\
289	0\\
290	0\\
291	0\\
292	0\\
293	0\\
294	0\\
295	0\\
296	0\\
297	0\\
298	0\\
299	0\\
300	0\\
301	0\\
302	0\\
303	0\\
304	0\\
305	0\\
306	0\\
307	0\\
308	0\\
309	0\\
310	0\\
311	0\\
312	0\\
313	0\\
314	0\\
315	0\\
316	0\\
317	0\\
318	0\\
319	0\\
320	0\\
321	0\\
322	0\\
323	0\\
324	0\\
325	0\\
326	0\\
327	0\\
328	0\\
329	0\\
330	0\\
331	0\\
332	0\\
333	0\\
334	0\\
335	0\\
336	0\\
337	0\\
338	0\\
339	0\\
340	0\\
341	0\\
342	0\\
343	0\\
344	0\\
345	0\\
346	0\\
347	0\\
348	0\\
349	0\\
350	0\\
351	0\\
352	0\\
353	0\\
354	0\\
355	0\\
356	0\\
357	0\\
358	0\\
359	0\\
360	0\\
361	0\\
362	0\\
363	0\\
364	0\\
365	0\\
366	0\\
367	0\\
368	0\\
369	0\\
370	0\\
371	0\\
372	0\\
373	0\\
374	0\\
375	0\\
376	0\\
377	0\\
378	0\\
379	0\\
380	0\\
381	0\\
382	0\\
383	0\\
384	0\\
385	0\\
386	0\\
387	0\\
388	0\\
389	0\\
390	0\\
391	0\\
392	0\\
393	0\\
394	0\\
395	0\\
396	0\\
397	0\\
398	0\\
399	0\\
400	0\\
401	0\\
402	0\\
403	0\\
404	0\\
405	0\\
406	0\\
407	0\\
408	0\\
409	0\\
410	0\\
411	0\\
412	0\\
413	0\\
414	0\\
415	0\\
416	0\\
417	0\\
418	0\\
419	0\\
420	0\\
421	0\\
422	0\\
423	0\\
424	0\\
425	0\\
426	0\\
427	0\\
428	0\\
429	0\\
430	0\\
431	0\\
432	0\\
433	0\\
434	0\\
435	0\\
436	0\\
437	0\\
438	0\\
439	0\\
440	0\\
441	0\\
442	0\\
443	0\\
444	0\\
445	0\\
446	0\\
447	0\\
448	0\\
449	0\\
450	0\\
451	0\\
452	0\\
453	0\\
454	0\\
455	0\\
456	0\\
457	0\\
458	0\\
459	0\\
460	0\\
461	0\\
462	0\\
463	0\\
464	0\\
465	0\\
466	0\\
467	0\\
468	0\\
469	0\\
470	0\\
471	0\\
472	0\\
473	0\\
474	0\\
475	0\\
476	0\\
477	0\\
478	0\\
479	0\\
480	0\\
481	0\\
482	0\\
483	0\\
484	0\\
485	0\\
486	0\\
487	0\\
488	0\\
489	0\\
490	0\\
491	0\\
492	0\\
493	0\\
494	0\\
495	0\\
496	0\\
497	0\\
498	0\\
499	0\\
500	0\\
501	0\\
502	0\\
503	0\\
504	0\\
505	0\\
506	0\\
507	0\\
508	0\\
509	0\\
510	0\\
511	0\\
512	0\\
513	0\\
514	0\\
515	0\\
516	0\\
517	0\\
518	0\\
519	0\\
520	0\\
521	0\\
522	0\\
523	0\\
524	0\\
525	0\\
526	0\\
527	0\\
528	0\\
529	0\\
530	0\\
531	0\\
532	0\\
533	0\\
534	0\\
535	0\\
536	0\\
537	0\\
538	0\\
539	0\\
540	0\\
541	0\\
542	0\\
543	0\\
544	0\\
545	8.42502412765558e-06\\
546	3.5851947618167e-05\\
547	6.38863155815981e-05\\
548	9.25528110374975e-05\\
549	0.000121877183064925\\
550	0.000151885824432349\\
551	0.000182616850935656\\
552	0.000214101549367516\\
553	0.000246366661718656\\
554	0.000279437159386335\\
555	0.000313337079486906\\
556	0.000348091356271088\\
557	0.000383725850763514\\
558	0.000420267380630395\\
559	0.000457743782503094\\
560	0.000496183903776342\\
561	0.000535617637525919\\
562	0.000576075963802652\\
563	0.00061759716129537\\
564	0.000660217067372072\\
565	0.000703959760100343\\
566	0.000748880601024024\\
567	0.000795008079541074\\
568	0.000842337955796574\\
569	0.00089091219139234\\
570	0.000940779053632871\\
571	0.000992018378974306\\
572	0.00104480256110106\\
573	0.00109890666525564\\
574	0.00115440158344068\\
575	0.0012113083263446\\
576	0.0012696758029509\\
577	0.00133007165307071\\
578	0.00139239776289234\\
579	0.00145595886199769\\
580	0.00160714494923117\\
581	0.001899590420333\\
582	0.00221326571071451\\
583	0.00246366209819108\\
584	0.00272385949410461\\
585	0.00292924427885274\\
586	0.00304785416029217\\
587	0.00315702143525764\\
588	0.00325763804119556\\
589	0.00335918365979025\\
590	0.00346100219613368\\
591	0.0035659653520596\\
592	0.00367521681883594\\
593	0.00379274227372929\\
594	0.0039289098819834\\
595	0.00411043972693671\\
596	0.00440815817174734\\
597	0.00501118703192877\\
598	0.00642488516645657\\
599	0\\
600	0\\
};
\addplot [color=blue!75!mycolor7,solid,forget plot]
  table[row sep=crcr]{%
1	0\\
2	0\\
3	0\\
4	0\\
5	0\\
6	0\\
7	0\\
8	0\\
9	0\\
10	0\\
11	0\\
12	0\\
13	0\\
14	0\\
15	0\\
16	0\\
17	0\\
18	0\\
19	0\\
20	0\\
21	0\\
22	0\\
23	0\\
24	0\\
25	0\\
26	0\\
27	0\\
28	0\\
29	0\\
30	0\\
31	0\\
32	0\\
33	0\\
34	0\\
35	0\\
36	0\\
37	0\\
38	0\\
39	0\\
40	0\\
41	0\\
42	0\\
43	0\\
44	0\\
45	0\\
46	0\\
47	0\\
48	0\\
49	0\\
50	0\\
51	0\\
52	0\\
53	0\\
54	0\\
55	0\\
56	0\\
57	0\\
58	0\\
59	0\\
60	0\\
61	0\\
62	0\\
63	0\\
64	0\\
65	0\\
66	0\\
67	0\\
68	0\\
69	0\\
70	0\\
71	0\\
72	0\\
73	0\\
74	0\\
75	0\\
76	0\\
77	0\\
78	0\\
79	0\\
80	0\\
81	0\\
82	0\\
83	0\\
84	0\\
85	0\\
86	0\\
87	0\\
88	0\\
89	0\\
90	0\\
91	0\\
92	0\\
93	0\\
94	0\\
95	0\\
96	0\\
97	0\\
98	0\\
99	0\\
100	0\\
101	0\\
102	0\\
103	0\\
104	0\\
105	0\\
106	0\\
107	0\\
108	0\\
109	0\\
110	0\\
111	0\\
112	0\\
113	0\\
114	0\\
115	0\\
116	0\\
117	0\\
118	0\\
119	0\\
120	0\\
121	0\\
122	0\\
123	0\\
124	0\\
125	0\\
126	0\\
127	0\\
128	0\\
129	0\\
130	0\\
131	0\\
132	0\\
133	0\\
134	0\\
135	0\\
136	0\\
137	0\\
138	0\\
139	0\\
140	0\\
141	0\\
142	0\\
143	0\\
144	0\\
145	0\\
146	0\\
147	0\\
148	0\\
149	0\\
150	0\\
151	0\\
152	0\\
153	0\\
154	0\\
155	0\\
156	0\\
157	0\\
158	0\\
159	0\\
160	0\\
161	0\\
162	0\\
163	0\\
164	0\\
165	0\\
166	0\\
167	0\\
168	0\\
169	0\\
170	0\\
171	0\\
172	0\\
173	0\\
174	0\\
175	0\\
176	0\\
177	0\\
178	0\\
179	0\\
180	0\\
181	0\\
182	0\\
183	0\\
184	0\\
185	0\\
186	0\\
187	0\\
188	0\\
189	0\\
190	0\\
191	0\\
192	0\\
193	0\\
194	0\\
195	0\\
196	0\\
197	0\\
198	0\\
199	0\\
200	0\\
201	0\\
202	0\\
203	0\\
204	0\\
205	0\\
206	0\\
207	0\\
208	0\\
209	0\\
210	0\\
211	0\\
212	0\\
213	0\\
214	0\\
215	0\\
216	0\\
217	0\\
218	0\\
219	0\\
220	0\\
221	0\\
222	0\\
223	0\\
224	0\\
225	0\\
226	0\\
227	0\\
228	0\\
229	0\\
230	0\\
231	0\\
232	0\\
233	0\\
234	0\\
235	0\\
236	0\\
237	0\\
238	0\\
239	0\\
240	0\\
241	0\\
242	0\\
243	0\\
244	0\\
245	0\\
246	0\\
247	0\\
248	0\\
249	0\\
250	0\\
251	0\\
252	0\\
253	0\\
254	0\\
255	0\\
256	0\\
257	0\\
258	0\\
259	0\\
260	0\\
261	0\\
262	0\\
263	0\\
264	0\\
265	0\\
266	0\\
267	0\\
268	0\\
269	0\\
270	0\\
271	0\\
272	0\\
273	0\\
274	0\\
275	0\\
276	0\\
277	0\\
278	0\\
279	0\\
280	0\\
281	0\\
282	0\\
283	0\\
284	0\\
285	0\\
286	0\\
287	0\\
288	0\\
289	0\\
290	0\\
291	0\\
292	0\\
293	0\\
294	0\\
295	0\\
296	0\\
297	0\\
298	0\\
299	0\\
300	0\\
301	0\\
302	0\\
303	0\\
304	0\\
305	0\\
306	0\\
307	0\\
308	0\\
309	0\\
310	0\\
311	0\\
312	0\\
313	0\\
314	0\\
315	0\\
316	0\\
317	0\\
318	0\\
319	0\\
320	0\\
321	0\\
322	0\\
323	0\\
324	0\\
325	0\\
326	0\\
327	0\\
328	0\\
329	0\\
330	0\\
331	0\\
332	0\\
333	0\\
334	0\\
335	0\\
336	0\\
337	0\\
338	0\\
339	0\\
340	0\\
341	0\\
342	0\\
343	0\\
344	0\\
345	0\\
346	0\\
347	0\\
348	0\\
349	0\\
350	0\\
351	0\\
352	0\\
353	0\\
354	0\\
355	0\\
356	0\\
357	0\\
358	0\\
359	0\\
360	0\\
361	0\\
362	0\\
363	0\\
364	0\\
365	0\\
366	0\\
367	0\\
368	0\\
369	0\\
370	0\\
371	0\\
372	0\\
373	0\\
374	0\\
375	0\\
376	0\\
377	0\\
378	0\\
379	0\\
380	0\\
381	0\\
382	0\\
383	0\\
384	0\\
385	0\\
386	0\\
387	0\\
388	0\\
389	0\\
390	0\\
391	0\\
392	0\\
393	0\\
394	0\\
395	0\\
396	0\\
397	0\\
398	0\\
399	0\\
400	0\\
401	0\\
402	0\\
403	0\\
404	0\\
405	0\\
406	0\\
407	0\\
408	0\\
409	0\\
410	0\\
411	0\\
412	0\\
413	0\\
414	0\\
415	0\\
416	0\\
417	0\\
418	0\\
419	0\\
420	0\\
421	0\\
422	0\\
423	0\\
424	0\\
425	0\\
426	0\\
427	0\\
428	0\\
429	0\\
430	0\\
431	0\\
432	0\\
433	0\\
434	0\\
435	0\\
436	0\\
437	0\\
438	0\\
439	0\\
440	0\\
441	0\\
442	0\\
443	0\\
444	0\\
445	0\\
446	0\\
447	0\\
448	0\\
449	0\\
450	0\\
451	0\\
452	0\\
453	0\\
454	0\\
455	0\\
456	0\\
457	0\\
458	0\\
459	0\\
460	0\\
461	0\\
462	0\\
463	0\\
464	0\\
465	0\\
466	0\\
467	0\\
468	0\\
469	0\\
470	0\\
471	0\\
472	0\\
473	0\\
474	0\\
475	0\\
476	0\\
477	0\\
478	0\\
479	0\\
480	0\\
481	0\\
482	0\\
483	0\\
484	0\\
485	0\\
486	0\\
487	0\\
488	0\\
489	0\\
490	0\\
491	0\\
492	0\\
493	0\\
494	0\\
495	0\\
496	0\\
497	0\\
498	0\\
499	0\\
500	0\\
501	0\\
502	0\\
503	0\\
504	0\\
505	0\\
506	0\\
507	0\\
508	0\\
509	0\\
510	0\\
511	0\\
512	0\\
513	0\\
514	0\\
515	0\\
516	0\\
517	0\\
518	0\\
519	0\\
520	0\\
521	0\\
522	0\\
523	0\\
524	0\\
525	0\\
526	0\\
527	0\\
528	0\\
529	0\\
530	0\\
531	0\\
532	0\\
533	0\\
534	0\\
535	0\\
536	0\\
537	0\\
538	0\\
539	0\\
540	0\\
541	0\\
542	0\\
543	0\\
544	0\\
545	0\\
546	0\\
547	2.01299093932324e-05\\
548	4.97180538846728e-05\\
549	7.97809066500282e-05\\
550	0.000110346808772033\\
551	0.000141449544534373\\
552	0.000173138917481353\\
553	0.000205467689239195\\
554	0.000238497409334684\\
555	0.000272296809122876\\
556	0.000306891010061317\\
557	0.000342306630210377\\
558	0.000378571867476797\\
559	0.000415716567682697\\
560	0.000453772268996213\\
561	0.000492772207995115\\
562	0.000532751271474939\\
563	0.000573745863855526\\
564	0.000615793687390193\\
565	0.000658933366835831\\
566	0.000703203836142204\\
567	0.00074864369322039\\
568	0.000795293618467427\\
569	0.000843202487028982\\
570	0.000892415829948461\\
571	0.000942997007940762\\
572	0.000994962562184872\\
573	0.0010483381877448\\
574	0.00110318229396275\\
575	0.00115964812226496\\
576	0.00121775736770317\\
577	0.00127738546819808\\
578	0.00133863477863552\\
579	0.00140157076039025\\
580	0.001466687528102\\
581	0.00153378580448497\\
582	0.00160982745781871\\
583	0.00189508529007164\\
584	0.00219138160047944\\
585	0.00247449504394679\\
586	0.0027294908751432\\
587	0.00299438591461619\\
588	0.00319452135862246\\
589	0.0033221530258415\\
590	0.00344585440730691\\
591	0.00355633781678836\\
592	0.00367135035665401\\
593	0.00379198418312597\\
594	0.0039289098819834\\
595	0.00411043972693671\\
596	0.00440815817174734\\
597	0.00501118703192877\\
598	0.00642488516645657\\
599	0\\
600	0\\
};
\addplot [color=blue!80!mycolor9,solid,forget plot]
  table[row sep=crcr]{%
1	0\\
2	0\\
3	0\\
4	0\\
5	0\\
6	0\\
7	0\\
8	0\\
9	0\\
10	0\\
11	0\\
12	0\\
13	0\\
14	0\\
15	0\\
16	0\\
17	0\\
18	0\\
19	0\\
20	0\\
21	0\\
22	0\\
23	0\\
24	0\\
25	0\\
26	0\\
27	0\\
28	0\\
29	0\\
30	0\\
31	0\\
32	0\\
33	0\\
34	0\\
35	0\\
36	0\\
37	0\\
38	0\\
39	0\\
40	0\\
41	0\\
42	0\\
43	0\\
44	0\\
45	0\\
46	0\\
47	0\\
48	0\\
49	0\\
50	0\\
51	0\\
52	0\\
53	0\\
54	0\\
55	0\\
56	0\\
57	0\\
58	0\\
59	0\\
60	0\\
61	0\\
62	0\\
63	0\\
64	0\\
65	0\\
66	0\\
67	0\\
68	0\\
69	0\\
70	0\\
71	0\\
72	0\\
73	0\\
74	0\\
75	0\\
76	0\\
77	0\\
78	0\\
79	0\\
80	0\\
81	0\\
82	0\\
83	0\\
84	0\\
85	0\\
86	0\\
87	0\\
88	0\\
89	0\\
90	0\\
91	0\\
92	0\\
93	0\\
94	0\\
95	0\\
96	0\\
97	0\\
98	0\\
99	0\\
100	0\\
101	0\\
102	0\\
103	0\\
104	0\\
105	0\\
106	0\\
107	0\\
108	0\\
109	0\\
110	0\\
111	0\\
112	0\\
113	0\\
114	0\\
115	0\\
116	0\\
117	0\\
118	0\\
119	0\\
120	0\\
121	0\\
122	0\\
123	0\\
124	0\\
125	0\\
126	0\\
127	0\\
128	0\\
129	0\\
130	0\\
131	0\\
132	0\\
133	0\\
134	0\\
135	0\\
136	0\\
137	0\\
138	0\\
139	0\\
140	0\\
141	0\\
142	0\\
143	0\\
144	0\\
145	0\\
146	0\\
147	0\\
148	0\\
149	0\\
150	0\\
151	0\\
152	0\\
153	0\\
154	0\\
155	0\\
156	0\\
157	0\\
158	0\\
159	0\\
160	0\\
161	0\\
162	0\\
163	0\\
164	0\\
165	0\\
166	0\\
167	0\\
168	0\\
169	0\\
170	0\\
171	0\\
172	0\\
173	0\\
174	0\\
175	0\\
176	0\\
177	0\\
178	0\\
179	0\\
180	0\\
181	0\\
182	0\\
183	0\\
184	0\\
185	0\\
186	0\\
187	0\\
188	0\\
189	0\\
190	0\\
191	0\\
192	0\\
193	0\\
194	0\\
195	0\\
196	0\\
197	0\\
198	0\\
199	0\\
200	0\\
201	0\\
202	0\\
203	0\\
204	0\\
205	0\\
206	0\\
207	0\\
208	0\\
209	0\\
210	0\\
211	0\\
212	0\\
213	0\\
214	0\\
215	0\\
216	0\\
217	0\\
218	0\\
219	0\\
220	0\\
221	0\\
222	0\\
223	0\\
224	0\\
225	0\\
226	0\\
227	0\\
228	0\\
229	0\\
230	0\\
231	0\\
232	0\\
233	0\\
234	0\\
235	0\\
236	0\\
237	0\\
238	0\\
239	0\\
240	0\\
241	0\\
242	0\\
243	0\\
244	0\\
245	0\\
246	0\\
247	0\\
248	0\\
249	0\\
250	0\\
251	0\\
252	0\\
253	0\\
254	0\\
255	0\\
256	0\\
257	0\\
258	0\\
259	0\\
260	0\\
261	0\\
262	0\\
263	0\\
264	0\\
265	0\\
266	0\\
267	0\\
268	0\\
269	0\\
270	0\\
271	0\\
272	0\\
273	0\\
274	0\\
275	0\\
276	0\\
277	0\\
278	0\\
279	0\\
280	0\\
281	0\\
282	0\\
283	0\\
284	0\\
285	0\\
286	0\\
287	0\\
288	0\\
289	0\\
290	0\\
291	0\\
292	0\\
293	0\\
294	0\\
295	0\\
296	0\\
297	0\\
298	0\\
299	0\\
300	0\\
301	0\\
302	0\\
303	0\\
304	0\\
305	0\\
306	0\\
307	0\\
308	0\\
309	0\\
310	0\\
311	0\\
312	0\\
313	0\\
314	0\\
315	0\\
316	0\\
317	0\\
318	0\\
319	0\\
320	0\\
321	0\\
322	0\\
323	0\\
324	0\\
325	0\\
326	0\\
327	0\\
328	0\\
329	0\\
330	0\\
331	0\\
332	0\\
333	0\\
334	0\\
335	0\\
336	0\\
337	0\\
338	0\\
339	0\\
340	0\\
341	0\\
342	0\\
343	0\\
344	0\\
345	0\\
346	0\\
347	0\\
348	0\\
349	0\\
350	0\\
351	0\\
352	0\\
353	0\\
354	0\\
355	0\\
356	0\\
357	0\\
358	0\\
359	0\\
360	0\\
361	0\\
362	0\\
363	0\\
364	0\\
365	0\\
366	0\\
367	0\\
368	0\\
369	0\\
370	0\\
371	0\\
372	0\\
373	0\\
374	0\\
375	0\\
376	0\\
377	0\\
378	0\\
379	0\\
380	0\\
381	0\\
382	0\\
383	0\\
384	0\\
385	0\\
386	0\\
387	0\\
388	0\\
389	0\\
390	0\\
391	0\\
392	0\\
393	0\\
394	0\\
395	0\\
396	0\\
397	0\\
398	0\\
399	0\\
400	0\\
401	0\\
402	0\\
403	0\\
404	0\\
405	0\\
406	0\\
407	0\\
408	0\\
409	0\\
410	0\\
411	0\\
412	0\\
413	0\\
414	0\\
415	0\\
416	0\\
417	0\\
418	0\\
419	0\\
420	0\\
421	0\\
422	0\\
423	0\\
424	0\\
425	0\\
426	0\\
427	0\\
428	0\\
429	0\\
430	0\\
431	0\\
432	0\\
433	0\\
434	0\\
435	0\\
436	0\\
437	0\\
438	0\\
439	0\\
440	0\\
441	0\\
442	0\\
443	0\\
444	0\\
445	0\\
446	0\\
447	0\\
448	0\\
449	0\\
450	0\\
451	0\\
452	0\\
453	0\\
454	0\\
455	0\\
456	0\\
457	0\\
458	0\\
459	0\\
460	0\\
461	0\\
462	0\\
463	0\\
464	0\\
465	0\\
466	0\\
467	0\\
468	0\\
469	0\\
470	0\\
471	0\\
472	0\\
473	0\\
474	0\\
475	0\\
476	0\\
477	0\\
478	0\\
479	0\\
480	0\\
481	0\\
482	0\\
483	0\\
484	0\\
485	0\\
486	0\\
487	0\\
488	0\\
489	0\\
490	0\\
491	0\\
492	0\\
493	0\\
494	0\\
495	0\\
496	0\\
497	0\\
498	0\\
499	0\\
500	0\\
501	0\\
502	0\\
503	0\\
504	0\\
505	0\\
506	0\\
507	0\\
508	0\\
509	0\\
510	0\\
511	0\\
512	0\\
513	0\\
514	0\\
515	0\\
516	0\\
517	0\\
518	0\\
519	0\\
520	0\\
521	0\\
522	0\\
523	0\\
524	0\\
525	0\\
526	0\\
527	0\\
528	0\\
529	0\\
530	0\\
531	0\\
532	0\\
533	0\\
534	0\\
535	0\\
536	0\\
537	0\\
538	0\\
539	0\\
540	0\\
541	0\\
542	0\\
543	0\\
544	0\\
545	0\\
546	0\\
547	0\\
548	0\\
549	0\\
550	2.12039339737889e-05\\
551	5.84050361664144e-05\\
552	9.54010943067283e-05\\
553	0.000132096979318932\\
554	0.000168388025923539\\
555	0.000204174616590858\\
556	0.000240606685821581\\
557	0.000277698174390048\\
558	0.000315465232300687\\
559	0.000353926827843293\\
560	0.000393105503571735\\
561	0.000433028368811907\\
562	0.000473728260877721\\
563	0.000515245178121573\\
564	0.000557628044582771\\
565	0.000600936877891869\\
566	0.000645245449218651\\
567	0.000690644491509104\\
568	0.000737174475960431\\
569	0.000784875954349228\\
570	0.000833792095532848\\
571	0.000883968867581733\\
572	0.00093545555079639\\
573	0.000988305024199897\\
574	0.00104257376943079\\
575	0.00109832432258913\\
576	0.00115564793896351\\
577	0.00121458956000823\\
578	0.0012751791441012\\
579	0.00133755345032623\\
580	0.00140183211419833\\
581	0.00146787349040303\\
582	0.00153576033905729\\
583	0.00160587300607344\\
584	0.00167839160337079\\
585	0.00183894072237567\\
586	0.00211968813859839\\
587	0.00241319308767275\\
588	0.00268990374966985\\
589	0.0029472638427176\\
590	0.00321775581666242\\
591	0.00346925274976254\\
592	0.00361272863110318\\
593	0.00376752264049869\\
594	0.00392397211113342\\
595	0.00411043972693671\\
596	0.00440815817174734\\
597	0.00501118703192877\\
598	0.00642488516645657\\
599	0\\
600	0\\
};
\addplot [color=blue,solid,forget plot]
  table[row sep=crcr]{%
1	0\\
2	0\\
3	0\\
4	0\\
5	0\\
6	0\\
7	0\\
8	0\\
9	0\\
10	0\\
11	0\\
12	0\\
13	0\\
14	0\\
15	0\\
16	0\\
17	0\\
18	0\\
19	0\\
20	0\\
21	0\\
22	0\\
23	0\\
24	0\\
25	0\\
26	0\\
27	0\\
28	0\\
29	0\\
30	0\\
31	0\\
32	0\\
33	0\\
34	0\\
35	0\\
36	0\\
37	0\\
38	0\\
39	0\\
40	0\\
41	0\\
42	0\\
43	0\\
44	0\\
45	0\\
46	0\\
47	0\\
48	0\\
49	0\\
50	0\\
51	0\\
52	0\\
53	0\\
54	0\\
55	0\\
56	0\\
57	0\\
58	0\\
59	0\\
60	0\\
61	0\\
62	0\\
63	0\\
64	0\\
65	0\\
66	0\\
67	0\\
68	0\\
69	0\\
70	0\\
71	0\\
72	0\\
73	0\\
74	0\\
75	0\\
76	0\\
77	0\\
78	0\\
79	0\\
80	0\\
81	0\\
82	0\\
83	0\\
84	0\\
85	0\\
86	0\\
87	0\\
88	0\\
89	0\\
90	0\\
91	0\\
92	0\\
93	0\\
94	0\\
95	0\\
96	0\\
97	0\\
98	0\\
99	0\\
100	0\\
101	0\\
102	0\\
103	0\\
104	0\\
105	0\\
106	0\\
107	0\\
108	0\\
109	0\\
110	0\\
111	0\\
112	0\\
113	0\\
114	0\\
115	0\\
116	0\\
117	0\\
118	0\\
119	0\\
120	0\\
121	0\\
122	0\\
123	0\\
124	0\\
125	0\\
126	0\\
127	0\\
128	0\\
129	0\\
130	0\\
131	0\\
132	0\\
133	0\\
134	0\\
135	0\\
136	0\\
137	0\\
138	0\\
139	0\\
140	0\\
141	0\\
142	0\\
143	0\\
144	0\\
145	0\\
146	0\\
147	0\\
148	0\\
149	0\\
150	0\\
151	0\\
152	0\\
153	0\\
154	0\\
155	0\\
156	0\\
157	0\\
158	0\\
159	0\\
160	0\\
161	0\\
162	0\\
163	0\\
164	0\\
165	0\\
166	0\\
167	0\\
168	0\\
169	0\\
170	0\\
171	0\\
172	0\\
173	0\\
174	0\\
175	0\\
176	0\\
177	0\\
178	0\\
179	0\\
180	0\\
181	0\\
182	0\\
183	0\\
184	0\\
185	0\\
186	0\\
187	0\\
188	0\\
189	0\\
190	0\\
191	0\\
192	0\\
193	0\\
194	0\\
195	0\\
196	0\\
197	0\\
198	0\\
199	0\\
200	0\\
201	0\\
202	0\\
203	0\\
204	0\\
205	0\\
206	0\\
207	0\\
208	0\\
209	0\\
210	0\\
211	0\\
212	0\\
213	0\\
214	0\\
215	0\\
216	0\\
217	0\\
218	0\\
219	0\\
220	0\\
221	0\\
222	0\\
223	0\\
224	0\\
225	0\\
226	0\\
227	0\\
228	0\\
229	0\\
230	0\\
231	0\\
232	0\\
233	0\\
234	0\\
235	0\\
236	0\\
237	0\\
238	0\\
239	0\\
240	0\\
241	0\\
242	0\\
243	0\\
244	0\\
245	0\\
246	0\\
247	0\\
248	0\\
249	0\\
250	0\\
251	0\\
252	0\\
253	0\\
254	0\\
255	0\\
256	0\\
257	0\\
258	0\\
259	0\\
260	0\\
261	0\\
262	0\\
263	0\\
264	0\\
265	0\\
266	0\\
267	0\\
268	0\\
269	0\\
270	0\\
271	0\\
272	0\\
273	0\\
274	0\\
275	0\\
276	0\\
277	0\\
278	0\\
279	0\\
280	0\\
281	0\\
282	0\\
283	0\\
284	0\\
285	0\\
286	0\\
287	0\\
288	0\\
289	0\\
290	0\\
291	0\\
292	0\\
293	0\\
294	0\\
295	0\\
296	0\\
297	0\\
298	0\\
299	0\\
300	0\\
301	0\\
302	0\\
303	0\\
304	0\\
305	0\\
306	0\\
307	0\\
308	0\\
309	0\\
310	0\\
311	0\\
312	0\\
313	0\\
314	0\\
315	0\\
316	0\\
317	0\\
318	0\\
319	0\\
320	0\\
321	0\\
322	0\\
323	0\\
324	0\\
325	0\\
326	0\\
327	0\\
328	0\\
329	0\\
330	0\\
331	0\\
332	0\\
333	0\\
334	0\\
335	0\\
336	0\\
337	0\\
338	0\\
339	0\\
340	0\\
341	0\\
342	0\\
343	0\\
344	0\\
345	0\\
346	0\\
347	0\\
348	0\\
349	0\\
350	0\\
351	0\\
352	0\\
353	0\\
354	0\\
355	0\\
356	0\\
357	0\\
358	0\\
359	0\\
360	0\\
361	0\\
362	0\\
363	0\\
364	0\\
365	0\\
366	0\\
367	0\\
368	0\\
369	0\\
370	0\\
371	0\\
372	0\\
373	0\\
374	0\\
375	0\\
376	0\\
377	0\\
378	0\\
379	0\\
380	0\\
381	0\\
382	0\\
383	0\\
384	0\\
385	0\\
386	0\\
387	0\\
388	0\\
389	0\\
390	0\\
391	0\\
392	0\\
393	0\\
394	0\\
395	0\\
396	0\\
397	0\\
398	0\\
399	0\\
400	0\\
401	0\\
402	0\\
403	0\\
404	0\\
405	0\\
406	0\\
407	0\\
408	0\\
409	0\\
410	0\\
411	0\\
412	0\\
413	0\\
414	0\\
415	0\\
416	0\\
417	0\\
418	0\\
419	0\\
420	0\\
421	0\\
422	0\\
423	0\\
424	0\\
425	0\\
426	0\\
427	0\\
428	0\\
429	0\\
430	0\\
431	0\\
432	0\\
433	0\\
434	0\\
435	0\\
436	0\\
437	0\\
438	0\\
439	0\\
440	0\\
441	0\\
442	0\\
443	0\\
444	0\\
445	0\\
446	0\\
447	0\\
448	0\\
449	0\\
450	0\\
451	0\\
452	0\\
453	0\\
454	0\\
455	0\\
456	0\\
457	0\\
458	0\\
459	0\\
460	0\\
461	0\\
462	0\\
463	0\\
464	0\\
465	0\\
466	0\\
467	0\\
468	0\\
469	0\\
470	0\\
471	0\\
472	0\\
473	0\\
474	0\\
475	0\\
476	0\\
477	0\\
478	0\\
479	0\\
480	0\\
481	0\\
482	0\\
483	0\\
484	0\\
485	0\\
486	0\\
487	0\\
488	0\\
489	0\\
490	0\\
491	0\\
492	0\\
493	0\\
494	0\\
495	0\\
496	0\\
497	0\\
498	0\\
499	0\\
500	0\\
501	0\\
502	0\\
503	0\\
504	0\\
505	0\\
506	0\\
507	0\\
508	0\\
509	0\\
510	0\\
511	0\\
512	0\\
513	0\\
514	0\\
515	0\\
516	0\\
517	0\\
518	0\\
519	0\\
520	0\\
521	0\\
522	0\\
523	0\\
524	0\\
525	0\\
526	0\\
527	0\\
528	0\\
529	0\\
530	0\\
531	0\\
532	0\\
533	0\\
534	0\\
535	0\\
536	0\\
537	0\\
538	0\\
539	0\\
540	0\\
541	0\\
542	0\\
543	0\\
544	0\\
545	0\\
546	0\\
547	0\\
548	0\\
549	0\\
550	0\\
551	0\\
552	0\\
553	0\\
554	0\\
555	1.91719487111611e-05\\
556	6.51916416175343e-05\\
557	0.000111751777533538\\
558	0.000158812301352876\\
559	0.000206325699048006\\
560	0.000254245520571927\\
561	0.000302505267507824\\
562	0.000351027678648465\\
563	0.000399723539571551\\
564	0.000448490252455449\\
565	0.000497210249201547\\
566	0.000545749008894194\\
567	0.00059395425796242\\
568	0.000643093628485419\\
569	0.000693247352774227\\
570	0.000744444297273541\\
571	0.000796717437008355\\
572	0.000850102988926268\\
573	0.000904635303798376\\
574	0.00096035463702777\\
575	0.00101730890690584\\
576	0.0010755559393491\\
577	0.0011351665831607\\
578	0.00119622852090244\\
579	0.00125885100335975\\
580	0.00132315112991813\\
581	0.00138919553049845\\
582	0.00145709190305839\\
583	0.00152687282332745\\
584	0.00159873739035049\\
585	0.00167260810899154\\
586	0.00174851745901825\\
587	0.00182712492469281\\
588	0.0020052356741675\\
589	0.00228207394121861\\
590	0.0025666370940843\\
591	0.00286021103851328\\
592	0.00312453732154687\\
593	0.00341862421613178\\
594	0.00377220464640417\\
595	0.00407888616628598\\
596	0.00440815817174734\\
597	0.00501118703192877\\
598	0.00642488516645657\\
599	0\\
600	0\\
};
\addplot [color=mycolor10,solid,forget plot]
  table[row sep=crcr]{%
1	0\\
2	0\\
3	0\\
4	0\\
5	0\\
6	0\\
7	0\\
8	0\\
9	0\\
10	0\\
11	0\\
12	0\\
13	0\\
14	0\\
15	0\\
16	0\\
17	0\\
18	0\\
19	0\\
20	0\\
21	0\\
22	0\\
23	0\\
24	0\\
25	0\\
26	0\\
27	0\\
28	0\\
29	0\\
30	0\\
31	0\\
32	0\\
33	0\\
34	0\\
35	0\\
36	0\\
37	0\\
38	0\\
39	0\\
40	0\\
41	0\\
42	0\\
43	0\\
44	0\\
45	0\\
46	0\\
47	0\\
48	0\\
49	0\\
50	0\\
51	0\\
52	0\\
53	0\\
54	0\\
55	0\\
56	0\\
57	0\\
58	0\\
59	0\\
60	0\\
61	0\\
62	0\\
63	0\\
64	0\\
65	0\\
66	0\\
67	0\\
68	0\\
69	0\\
70	0\\
71	0\\
72	0\\
73	0\\
74	0\\
75	0\\
76	0\\
77	0\\
78	0\\
79	0\\
80	0\\
81	0\\
82	0\\
83	0\\
84	0\\
85	0\\
86	0\\
87	0\\
88	0\\
89	0\\
90	0\\
91	0\\
92	0\\
93	0\\
94	0\\
95	0\\
96	0\\
97	0\\
98	0\\
99	0\\
100	0\\
101	0\\
102	0\\
103	0\\
104	0\\
105	0\\
106	0\\
107	0\\
108	0\\
109	0\\
110	0\\
111	0\\
112	0\\
113	0\\
114	0\\
115	0\\
116	0\\
117	0\\
118	0\\
119	0\\
120	0\\
121	0\\
122	0\\
123	0\\
124	0\\
125	0\\
126	0\\
127	0\\
128	0\\
129	0\\
130	0\\
131	0\\
132	0\\
133	0\\
134	0\\
135	0\\
136	0\\
137	0\\
138	0\\
139	0\\
140	0\\
141	0\\
142	0\\
143	0\\
144	0\\
145	0\\
146	0\\
147	0\\
148	0\\
149	0\\
150	0\\
151	0\\
152	0\\
153	0\\
154	0\\
155	0\\
156	0\\
157	0\\
158	0\\
159	0\\
160	0\\
161	0\\
162	0\\
163	0\\
164	0\\
165	0\\
166	0\\
167	0\\
168	0\\
169	0\\
170	0\\
171	0\\
172	0\\
173	0\\
174	0\\
175	0\\
176	0\\
177	0\\
178	0\\
179	0\\
180	0\\
181	0\\
182	0\\
183	0\\
184	0\\
185	0\\
186	0\\
187	0\\
188	0\\
189	0\\
190	0\\
191	0\\
192	0\\
193	0\\
194	0\\
195	0\\
196	0\\
197	0\\
198	0\\
199	0\\
200	0\\
201	0\\
202	0\\
203	0\\
204	0\\
205	0\\
206	0\\
207	0\\
208	0\\
209	0\\
210	0\\
211	0\\
212	0\\
213	0\\
214	0\\
215	0\\
216	0\\
217	0\\
218	0\\
219	0\\
220	0\\
221	0\\
222	0\\
223	0\\
224	0\\
225	0\\
226	0\\
227	0\\
228	0\\
229	0\\
230	0\\
231	0\\
232	0\\
233	0\\
234	0\\
235	0\\
236	0\\
237	0\\
238	0\\
239	0\\
240	0\\
241	0\\
242	0\\
243	0\\
244	0\\
245	0\\
246	0\\
247	0\\
248	0\\
249	0\\
250	0\\
251	0\\
252	0\\
253	0\\
254	0\\
255	0\\
256	0\\
257	0\\
258	0\\
259	0\\
260	0\\
261	0\\
262	0\\
263	0\\
264	0\\
265	0\\
266	0\\
267	0\\
268	0\\
269	0\\
270	0\\
271	0\\
272	0\\
273	0\\
274	0\\
275	0\\
276	0\\
277	0\\
278	0\\
279	0\\
280	0\\
281	0\\
282	0\\
283	0\\
284	0\\
285	0\\
286	0\\
287	0\\
288	0\\
289	0\\
290	0\\
291	0\\
292	0\\
293	0\\
294	0\\
295	0\\
296	0\\
297	0\\
298	0\\
299	0\\
300	0\\
301	0\\
302	0\\
303	0\\
304	0\\
305	0\\
306	0\\
307	0\\
308	0\\
309	0\\
310	0\\
311	0\\
312	0\\
313	0\\
314	0\\
315	0\\
316	0\\
317	0\\
318	0\\
319	0\\
320	0\\
321	0\\
322	0\\
323	0\\
324	0\\
325	0\\
326	0\\
327	0\\
328	0\\
329	0\\
330	0\\
331	0\\
332	0\\
333	0\\
334	0\\
335	0\\
336	0\\
337	0\\
338	0\\
339	0\\
340	0\\
341	0\\
342	0\\
343	0\\
344	0\\
345	0\\
346	0\\
347	0\\
348	0\\
349	0\\
350	0\\
351	0\\
352	0\\
353	0\\
354	0\\
355	0\\
356	0\\
357	0\\
358	0\\
359	0\\
360	0\\
361	0\\
362	0\\
363	0\\
364	0\\
365	0\\
366	0\\
367	0\\
368	0\\
369	0\\
370	0\\
371	0\\
372	0\\
373	0\\
374	0\\
375	0\\
376	0\\
377	0\\
378	0\\
379	0\\
380	0\\
381	0\\
382	0\\
383	0\\
384	0\\
385	0\\
386	0\\
387	0\\
388	0\\
389	0\\
390	0\\
391	0\\
392	0\\
393	0\\
394	0\\
395	0\\
396	0\\
397	0\\
398	0\\
399	0\\
400	0\\
401	0\\
402	0\\
403	0\\
404	0\\
405	0\\
406	0\\
407	0\\
408	0\\
409	0\\
410	0\\
411	0\\
412	0\\
413	0\\
414	0\\
415	0\\
416	0\\
417	0\\
418	0\\
419	0\\
420	0\\
421	0\\
422	0\\
423	0\\
424	0\\
425	0\\
426	0\\
427	0\\
428	0\\
429	0\\
430	0\\
431	0\\
432	0\\
433	0\\
434	0\\
435	0\\
436	0\\
437	0\\
438	0\\
439	0\\
440	0\\
441	0\\
442	0\\
443	0\\
444	0\\
445	0\\
446	0\\
447	0\\
448	0\\
449	0\\
450	0\\
451	0\\
452	0\\
453	0\\
454	0\\
455	0\\
456	0\\
457	0\\
458	0\\
459	0\\
460	0\\
461	0\\
462	0\\
463	0\\
464	0\\
465	0\\
466	0\\
467	0\\
468	0\\
469	0\\
470	0\\
471	0\\
472	0\\
473	0\\
474	0\\
475	0\\
476	0\\
477	0\\
478	0\\
479	0\\
480	0\\
481	0\\
482	0\\
483	0\\
484	0\\
485	0\\
486	0\\
487	0\\
488	0\\
489	0\\
490	0\\
491	0\\
492	0\\
493	0\\
494	0\\
495	0\\
496	0\\
497	0\\
498	0\\
499	0\\
500	0\\
501	0\\
502	0\\
503	0\\
504	0\\
505	0\\
506	0\\
507	0\\
508	0\\
509	0\\
510	0\\
511	0\\
512	0\\
513	0\\
514	0\\
515	0\\
516	0\\
517	0\\
518	0\\
519	0\\
520	0\\
521	0\\
522	0\\
523	0\\
524	0\\
525	0\\
526	0\\
527	0\\
528	0\\
529	0\\
530	0\\
531	0\\
532	0\\
533	0\\
534	0\\
535	0\\
536	0\\
537	0\\
538	0\\
539	0\\
540	0\\
541	0\\
542	0\\
543	0\\
544	0\\
545	0\\
546	0\\
547	0\\
548	0\\
549	0\\
550	0\\
551	0\\
552	0\\
553	0\\
554	0\\
555	0\\
556	0\\
557	0\\
558	0\\
559	0\\
560	0\\
561	0\\
562	0\\
563	0\\
564	6.38099865833115e-05\\
565	0.000154253607646718\\
566	0.000245683738061437\\
567	0.000337871408199173\\
568	0.000399914179502145\\
569	0.000461561946766633\\
570	0.00052410013148521\\
571	0.000587469940331828\\
572	0.000651657865042651\\
573	0.000716829079787043\\
574	0.000782921384134708\\
575	0.000849856385345376\\
576	0.000917537151764115\\
577	0.000985844560883639\\
578	0.00105463204067166\\
579	0.00112371753012356\\
580	0.00119324941240428\\
581	0.00126444776909183\\
582	0.00133734372880463\\
583	0.00141196851593021\\
584	0.00148835333464273\\
585	0.00156653037063608\\
586	0.00164653267396629\\
587	0.00172839481985068\\
588	0.00181217273894298\\
589	0.00189801407111492\\
590	0.00198587444025641\\
591	0.00211599393384859\\
592	0.00239672135457705\\
593	0.00268923432027998\\
594	0.003042228405148\\
595	0.00347514822745281\\
596	0.00421265278177493\\
597	0.00501118703192877\\
598	0.00642488516645657\\
599	0\\
600	0\\
};
\addplot [color=mycolor11,solid,forget plot]
  table[row sep=crcr]{%
1	0\\
2	0\\
3	0\\
4	0\\
5	0\\
6	0\\
7	0\\
8	0\\
9	0\\
10	0\\
11	0\\
12	0\\
13	0\\
14	0\\
15	0\\
16	0\\
17	0\\
18	0\\
19	0\\
20	0\\
21	0\\
22	0\\
23	0\\
24	0\\
25	0\\
26	0\\
27	0\\
28	0\\
29	0\\
30	0\\
31	0\\
32	0\\
33	0\\
34	0\\
35	0\\
36	0\\
37	0\\
38	0\\
39	0\\
40	0\\
41	0\\
42	0\\
43	0\\
44	0\\
45	0\\
46	0\\
47	0\\
48	0\\
49	0\\
50	0\\
51	0\\
52	0\\
53	0\\
54	0\\
55	0\\
56	0\\
57	0\\
58	0\\
59	0\\
60	0\\
61	0\\
62	0\\
63	0\\
64	0\\
65	0\\
66	0\\
67	0\\
68	0\\
69	0\\
70	0\\
71	0\\
72	0\\
73	0\\
74	0\\
75	0\\
76	0\\
77	0\\
78	0\\
79	0\\
80	0\\
81	0\\
82	0\\
83	0\\
84	0\\
85	0\\
86	0\\
87	0\\
88	0\\
89	0\\
90	0\\
91	0\\
92	0\\
93	0\\
94	0\\
95	0\\
96	0\\
97	0\\
98	0\\
99	0\\
100	0\\
101	0\\
102	0\\
103	0\\
104	0\\
105	0\\
106	0\\
107	0\\
108	0\\
109	0\\
110	0\\
111	0\\
112	0\\
113	0\\
114	0\\
115	0\\
116	0\\
117	0\\
118	0\\
119	0\\
120	0\\
121	0\\
122	0\\
123	0\\
124	0\\
125	0\\
126	0\\
127	0\\
128	0\\
129	0\\
130	0\\
131	0\\
132	0\\
133	0\\
134	0\\
135	0\\
136	0\\
137	0\\
138	0\\
139	0\\
140	0\\
141	0\\
142	0\\
143	0\\
144	0\\
145	0\\
146	0\\
147	0\\
148	0\\
149	0\\
150	0\\
151	0\\
152	0\\
153	0\\
154	0\\
155	0\\
156	0\\
157	0\\
158	0\\
159	0\\
160	0\\
161	0\\
162	0\\
163	0\\
164	0\\
165	0\\
166	0\\
167	0\\
168	0\\
169	0\\
170	0\\
171	0\\
172	0\\
173	0\\
174	0\\
175	0\\
176	0\\
177	0\\
178	0\\
179	0\\
180	0\\
181	0\\
182	0\\
183	0\\
184	0\\
185	0\\
186	0\\
187	0\\
188	0\\
189	0\\
190	0\\
191	0\\
192	0\\
193	0\\
194	0\\
195	0\\
196	0\\
197	0\\
198	0\\
199	0\\
200	0\\
201	0\\
202	0\\
203	0\\
204	0\\
205	0\\
206	0\\
207	0\\
208	0\\
209	0\\
210	0\\
211	0\\
212	0\\
213	0\\
214	0\\
215	0\\
216	0\\
217	0\\
218	0\\
219	0\\
220	0\\
221	0\\
222	0\\
223	0\\
224	0\\
225	0\\
226	0\\
227	0\\
228	0\\
229	0\\
230	0\\
231	0\\
232	0\\
233	0\\
234	0\\
235	0\\
236	0\\
237	0\\
238	0\\
239	0\\
240	0\\
241	0\\
242	0\\
243	0\\
244	0\\
245	0\\
246	0\\
247	0\\
248	0\\
249	0\\
250	0\\
251	0\\
252	0\\
253	0\\
254	0\\
255	0\\
256	0\\
257	0\\
258	0\\
259	0\\
260	0\\
261	0\\
262	0\\
263	0\\
264	0\\
265	0\\
266	0\\
267	0\\
268	0\\
269	0\\
270	0\\
271	0\\
272	0\\
273	0\\
274	0\\
275	0\\
276	0\\
277	0\\
278	0\\
279	0\\
280	0\\
281	0\\
282	0\\
283	0\\
284	0\\
285	0\\
286	0\\
287	0\\
288	0\\
289	0\\
290	0\\
291	0\\
292	0\\
293	0\\
294	0\\
295	0\\
296	0\\
297	0\\
298	0\\
299	0\\
300	0\\
301	0\\
302	0\\
303	0\\
304	0\\
305	0\\
306	0\\
307	0\\
308	0\\
309	0\\
310	0\\
311	0\\
312	0\\
313	0\\
314	0\\
315	0\\
316	0\\
317	0\\
318	0\\
319	0\\
320	0\\
321	0\\
322	0\\
323	0\\
324	0\\
325	0\\
326	0\\
327	0\\
328	0\\
329	0\\
330	0\\
331	0\\
332	0\\
333	0\\
334	0\\
335	0\\
336	0\\
337	0\\
338	0\\
339	0\\
340	0\\
341	0\\
342	0\\
343	0\\
344	0\\
345	0\\
346	0\\
347	0\\
348	0\\
349	0\\
350	0\\
351	0\\
352	0\\
353	0\\
354	0\\
355	0\\
356	0\\
357	0\\
358	0\\
359	0\\
360	0\\
361	0\\
362	0\\
363	0\\
364	0\\
365	0\\
366	0\\
367	0\\
368	0\\
369	0\\
370	0\\
371	0\\
372	0\\
373	0\\
374	0\\
375	0\\
376	0\\
377	0\\
378	0\\
379	0\\
380	0\\
381	0\\
382	0\\
383	0\\
384	0\\
385	0\\
386	0\\
387	0\\
388	0\\
389	0\\
390	0\\
391	0\\
392	0\\
393	0\\
394	0\\
395	0\\
396	0\\
397	0\\
398	0\\
399	0\\
400	0\\
401	0\\
402	0\\
403	0\\
404	0\\
405	0\\
406	0\\
407	0\\
408	0\\
409	0\\
410	0\\
411	0\\
412	0\\
413	0\\
414	0\\
415	0\\
416	0\\
417	0\\
418	0\\
419	0\\
420	0\\
421	0\\
422	0\\
423	0\\
424	0\\
425	0\\
426	0\\
427	0\\
428	0\\
429	0\\
430	0\\
431	0\\
432	0\\
433	0\\
434	0\\
435	0\\
436	0\\
437	0\\
438	0\\
439	0\\
440	0\\
441	0\\
442	0\\
443	0\\
444	0\\
445	0\\
446	0\\
447	0\\
448	0\\
449	0\\
450	0\\
451	0\\
452	0\\
453	0\\
454	0\\
455	0\\
456	0\\
457	0\\
458	0\\
459	0\\
460	0\\
461	0\\
462	0\\
463	0\\
464	0\\
465	0\\
466	0\\
467	0\\
468	0\\
469	0\\
470	0\\
471	0\\
472	0\\
473	0\\
474	0\\
475	0\\
476	0\\
477	0\\
478	0\\
479	0\\
480	0\\
481	0\\
482	0\\
483	0\\
484	0\\
485	0\\
486	0\\
487	0\\
488	0\\
489	0\\
490	0\\
491	0\\
492	0\\
493	0\\
494	0\\
495	0\\
496	0\\
497	0\\
498	0\\
499	0\\
500	0\\
501	0\\
502	0\\
503	0\\
504	0\\
505	0\\
506	0\\
507	0\\
508	0\\
509	0\\
510	0\\
511	0\\
512	0\\
513	0\\
514	0\\
515	0\\
516	0\\
517	0\\
518	0\\
519	0\\
520	0\\
521	0\\
522	0\\
523	0\\
524	0\\
525	0\\
526	0\\
527	0\\
528	0\\
529	0\\
530	0\\
531	0\\
532	0\\
533	0\\
534	0\\
535	0\\
536	0\\
537	0\\
538	0\\
539	0\\
540	0\\
541	0\\
542	0\\
543	0\\
544	0\\
545	0\\
546	0\\
547	0\\
548	0\\
549	0\\
550	0\\
551	0\\
552	0\\
553	0\\
554	0\\
555	0\\
556	0\\
557	0\\
558	0\\
559	0\\
560	0\\
561	0\\
562	0\\
563	0\\
564	0\\
565	0\\
566	0\\
567	0\\
568	0\\
569	0\\
570	0\\
571	0\\
572	0\\
573	4.04709118961138e-05\\
574	0.000146135873699704\\
575	0.000254345046972783\\
576	0.000365036226188813\\
577	0.000477991922738407\\
578	0.000592630547932805\\
579	0.000709819060000769\\
580	0.000822303469911085\\
581	0.000910282263958814\\
582	0.00100058720699734\\
583	0.00109325009451827\\
584	0.00118829487178716\\
585	0.0012857341382992\\
586	0.00138559035948951\\
587	0.00148783487413243\\
588	0.00159241035075032\\
589	0.0016992221476438\\
590	0.00180812756853647\\
591	0.00191892108123612\\
592	0.00203131802967084\\
593	0.0021449343752569\\
594	0.00225924019373236\\
595	0.0025990765750797\\
596	0.00331669012914152\\
597	0.00466497804100969\\
598	0.00642488516645657\\
599	0\\
600	0\\
};
\addplot [color=mycolor12,solid,forget plot]
  table[row sep=crcr]{%
1	0\\
2	0\\
3	0\\
4	0\\
5	0\\
6	0\\
7	0\\
8	0\\
9	0\\
10	0\\
11	0\\
12	0\\
13	0\\
14	0\\
15	0\\
16	0\\
17	0\\
18	0\\
19	0\\
20	0\\
21	0\\
22	0\\
23	0\\
24	0\\
25	0\\
26	0\\
27	0\\
28	0\\
29	0\\
30	0\\
31	0\\
32	0\\
33	0\\
34	0\\
35	0\\
36	0\\
37	0\\
38	0\\
39	0\\
40	0\\
41	0\\
42	0\\
43	0\\
44	0\\
45	0\\
46	0\\
47	0\\
48	0\\
49	0\\
50	0\\
51	0\\
52	0\\
53	0\\
54	0\\
55	0\\
56	0\\
57	0\\
58	0\\
59	0\\
60	0\\
61	0\\
62	0\\
63	0\\
64	0\\
65	0\\
66	0\\
67	0\\
68	0\\
69	0\\
70	0\\
71	0\\
72	0\\
73	0\\
74	0\\
75	0\\
76	0\\
77	0\\
78	0\\
79	0\\
80	0\\
81	0\\
82	0\\
83	0\\
84	0\\
85	0\\
86	0\\
87	0\\
88	0\\
89	0\\
90	0\\
91	0\\
92	0\\
93	0\\
94	0\\
95	0\\
96	0\\
97	0\\
98	0\\
99	0\\
100	0\\
101	0\\
102	0\\
103	0\\
104	0\\
105	0\\
106	0\\
107	0\\
108	0\\
109	0\\
110	0\\
111	0\\
112	0\\
113	0\\
114	0\\
115	0\\
116	0\\
117	0\\
118	0\\
119	0\\
120	0\\
121	0\\
122	0\\
123	0\\
124	0\\
125	0\\
126	0\\
127	0\\
128	0\\
129	0\\
130	0\\
131	0\\
132	0\\
133	0\\
134	0\\
135	0\\
136	0\\
137	0\\
138	0\\
139	0\\
140	0\\
141	0\\
142	0\\
143	0\\
144	0\\
145	0\\
146	0\\
147	0\\
148	0\\
149	0\\
150	0\\
151	0\\
152	0\\
153	0\\
154	0\\
155	0\\
156	0\\
157	0\\
158	0\\
159	0\\
160	0\\
161	0\\
162	0\\
163	0\\
164	0\\
165	0\\
166	0\\
167	0\\
168	0\\
169	0\\
170	0\\
171	0\\
172	0\\
173	0\\
174	0\\
175	0\\
176	0\\
177	0\\
178	0\\
179	0\\
180	0\\
181	0\\
182	0\\
183	0\\
184	0\\
185	0\\
186	0\\
187	0\\
188	0\\
189	0\\
190	0\\
191	0\\
192	0\\
193	0\\
194	0\\
195	0\\
196	0\\
197	0\\
198	0\\
199	0\\
200	0\\
201	0\\
202	0\\
203	0\\
204	0\\
205	0\\
206	0\\
207	0\\
208	0\\
209	0\\
210	0\\
211	0\\
212	0\\
213	0\\
214	0\\
215	0\\
216	0\\
217	0\\
218	0\\
219	0\\
220	0\\
221	0\\
222	0\\
223	0\\
224	0\\
225	0\\
226	0\\
227	0\\
228	0\\
229	0\\
230	0\\
231	0\\
232	0\\
233	0\\
234	0\\
235	0\\
236	0\\
237	0\\
238	0\\
239	0\\
240	0\\
241	0\\
242	0\\
243	0\\
244	0\\
245	0\\
246	0\\
247	0\\
248	0\\
249	0\\
250	0\\
251	0\\
252	0\\
253	0\\
254	0\\
255	0\\
256	0\\
257	0\\
258	0\\
259	0\\
260	0\\
261	0\\
262	0\\
263	0\\
264	0\\
265	0\\
266	0\\
267	0\\
268	0\\
269	0\\
270	0\\
271	0\\
272	0\\
273	0\\
274	0\\
275	0\\
276	0\\
277	0\\
278	0\\
279	0\\
280	0\\
281	0\\
282	0\\
283	0\\
284	0\\
285	0\\
286	0\\
287	0\\
288	0\\
289	0\\
290	0\\
291	0\\
292	0\\
293	0\\
294	0\\
295	0\\
296	0\\
297	0\\
298	0\\
299	0\\
300	0\\
301	0\\
302	0\\
303	0\\
304	0\\
305	0\\
306	0\\
307	0\\
308	0\\
309	0\\
310	0\\
311	0\\
312	0\\
313	0\\
314	0\\
315	0\\
316	0\\
317	0\\
318	0\\
319	0\\
320	0\\
321	0\\
322	0\\
323	0\\
324	0\\
325	0\\
326	0\\
327	0\\
328	0\\
329	0\\
330	0\\
331	0\\
332	0\\
333	0\\
334	0\\
335	0\\
336	0\\
337	0\\
338	0\\
339	0\\
340	0\\
341	0\\
342	0\\
343	0\\
344	0\\
345	0\\
346	0\\
347	0\\
348	0\\
349	0\\
350	0\\
351	0\\
352	0\\
353	0\\
354	0\\
355	0\\
356	0\\
357	0\\
358	0\\
359	0\\
360	0\\
361	0\\
362	0\\
363	0\\
364	0\\
365	0\\
366	0\\
367	0\\
368	0\\
369	0\\
370	0\\
371	0\\
372	0\\
373	0\\
374	0\\
375	0\\
376	0\\
377	0\\
378	0\\
379	0\\
380	0\\
381	0\\
382	0\\
383	0\\
384	0\\
385	0\\
386	0\\
387	0\\
388	0\\
389	0\\
390	0\\
391	0\\
392	0\\
393	0\\
394	0\\
395	0\\
396	0\\
397	0\\
398	0\\
399	0\\
400	0\\
401	0\\
402	0\\
403	0\\
404	0\\
405	0\\
406	0\\
407	0\\
408	0\\
409	0\\
410	0\\
411	0\\
412	0\\
413	0\\
414	0\\
415	0\\
416	0\\
417	0\\
418	0\\
419	0\\
420	0\\
421	0\\
422	0\\
423	0\\
424	0\\
425	0\\
426	0\\
427	0\\
428	0\\
429	0\\
430	0\\
431	0\\
432	0\\
433	0\\
434	0\\
435	0\\
436	0\\
437	0\\
438	0\\
439	0\\
440	0\\
441	0\\
442	0\\
443	0\\
444	0\\
445	0\\
446	0\\
447	0\\
448	0\\
449	0\\
450	0\\
451	0\\
452	0\\
453	0\\
454	0\\
455	0\\
456	0\\
457	0\\
458	0\\
459	0\\
460	0\\
461	0\\
462	0\\
463	0\\
464	0\\
465	0\\
466	0\\
467	0\\
468	0\\
469	0\\
470	0\\
471	0\\
472	0\\
473	0\\
474	0\\
475	0\\
476	0\\
477	0\\
478	0\\
479	0\\
480	0\\
481	0\\
482	0\\
483	0\\
484	0\\
485	0\\
486	0\\
487	0\\
488	0\\
489	0\\
490	0\\
491	0\\
492	0\\
493	0\\
494	0\\
495	0\\
496	0\\
497	0\\
498	0\\
499	0\\
500	0\\
501	0\\
502	0\\
503	0\\
504	0\\
505	0\\
506	0\\
507	0\\
508	0\\
509	0\\
510	0\\
511	0\\
512	0\\
513	0\\
514	0\\
515	0\\
516	0\\
517	0\\
518	0\\
519	0\\
520	0\\
521	0\\
522	0\\
523	0\\
524	0\\
525	0\\
526	0\\
527	0\\
528	0\\
529	0\\
530	0\\
531	0\\
532	0\\
533	0\\
534	0\\
535	0\\
536	0\\
537	0\\
538	0\\
539	0\\
540	0\\
541	0\\
542	0\\
543	0\\
544	0\\
545	0\\
546	0\\
547	0\\
548	0\\
549	0\\
550	0\\
551	0\\
552	0\\
553	0\\
554	0\\
555	0\\
556	0\\
557	0\\
558	0\\
559	0\\
560	0\\
561	0\\
562	0\\
563	0\\
564	0\\
565	0\\
566	0\\
567	0\\
568	0\\
569	0\\
570	0\\
571	0\\
572	0\\
573	0\\
574	0\\
575	0\\
576	0\\
577	0\\
578	0\\
579	0\\
580	0\\
581	0\\
582	0\\
583	0\\
584	0\\
585	0.00013774192519883\\
586	0.000284032666100375\\
587	0.000435941232690743\\
588	0.000593865412635933\\
589	0.000758286829767887\\
590	0.000929880038087214\\
591	0.00110901579535917\\
592	0.0012962244624629\\
593	0.00149227987384335\\
594	0.00169817885362274\\
595	0.0019152966930724\\
596	0.00214579709295045\\
597	0.00338203886614602\\
598	0.00642488516645657\\
599	0\\
600	0\\
};
\addplot [color=mycolor13,solid,forget plot]
  table[row sep=crcr]{%
1	0.0013780462490059\\
2	0.0013780462490059\\
3	0.0013780462490059\\
4	0.0013780462490059\\
5	0.0013780462490059\\
6	0.0013780462490059\\
7	0.0013780462490059\\
8	0.0013780462490059\\
9	0.0013780462490059\\
10	0.0013780462490059\\
11	0.0013780462490059\\
12	0.0013780462490059\\
13	0.0013780462490059\\
14	0.0013780462490059\\
15	0.0013780462490059\\
16	0.0013780462490059\\
17	0.0013780462490059\\
18	0.0013780462490059\\
19	0.0013780462490059\\
20	0.0013780462490059\\
21	0.0013780462490059\\
22	0.0013780462490059\\
23	0.0013780462490059\\
24	0.0013780462490059\\
25	0.0013780462490059\\
26	0.0013780462490059\\
27	0.0013780462490059\\
28	0.0013780462490059\\
29	0.0013780462490059\\
30	0.0013780462490059\\
31	0.0013780462490059\\
32	0.0013780462490059\\
33	0.0013780462490059\\
34	0.0013780462490059\\
35	0.0013780462490059\\
36	0.0013780462490059\\
37	0.0013780462490059\\
38	0.0013780462490059\\
39	0.0013780462490059\\
40	0.0013780462490059\\
41	0.0013780462490059\\
42	0.0013780462490059\\
43	0.0013780462490059\\
44	0.0013780462490059\\
45	0.0013780462490059\\
46	0.0013780462490059\\
47	0.0013780462490059\\
48	0.0013780462490059\\
49	0.0013780462490059\\
50	0.0013780462490059\\
51	0.0013780462490059\\
52	0.0013780462490059\\
53	0.0013780462490059\\
54	0.0013780462490059\\
55	0.0013780462490059\\
56	0.0013780462490059\\
57	0.0013780462490059\\
58	0.0013780462490059\\
59	0.0013780462490059\\
60	0.0013780462490059\\
61	0.0013780462490059\\
62	0.0013780462490059\\
63	0.0013780462490059\\
64	0.0013780462490059\\
65	0.0013780462490059\\
66	0.0013780462490059\\
67	0.0013780462490059\\
68	0.0013780462490059\\
69	0.0013780462490059\\
70	0.0013780462490059\\
71	0.0013780462490059\\
72	0.0013780462490059\\
73	0.0013780462490059\\
74	0.0013780462490059\\
75	0.0013780462490059\\
76	0.0013780462490059\\
77	0.0013780462490059\\
78	0.0013780462490059\\
79	0.0013780462490059\\
80	0.0013780462490059\\
81	0.0013780462490059\\
82	0.0013780462490059\\
83	0.0013780462490059\\
84	0.0013780462490059\\
85	0.0013780462490059\\
86	0.0013780462490059\\
87	0.0013780462490059\\
88	0.0013780462490059\\
89	0.0013780462490059\\
90	0.0013780462490059\\
91	0.0013780462490059\\
92	0.0013780462490059\\
93	0.0013780462490059\\
94	0.0013780462490059\\
95	0.0013780462490059\\
96	0.0013780462490059\\
97	0.0013780462490059\\
98	0.0013780462490059\\
99	0.0013780462490059\\
100	0.0013780462490059\\
101	0.0013780462490059\\
102	0.0013780462490059\\
103	0.0013780462490059\\
104	0.0013780462490059\\
105	0.0013780462490059\\
106	0.0013780462490059\\
107	0.0013780462490059\\
108	0.0013780462490059\\
109	0.0013780462490059\\
110	0.0013780462490059\\
111	0.0013780462490059\\
112	0.0013780462490059\\
113	0.0013780462490059\\
114	0.0013780462490059\\
115	0.0013780462490059\\
116	0.0013780462490059\\
117	0.0013780462490059\\
118	0.0013780462490059\\
119	0.0013780462490059\\
120	0.0013780462490059\\
121	0.0013780462490059\\
122	0.0013780462490059\\
123	0.0013780462490059\\
124	0.0013780462490059\\
125	0.0013780462490059\\
126	0.0013780462490059\\
127	0.0013780462490059\\
128	0.0013780462490059\\
129	0.0013780462490059\\
130	0.0013780462490059\\
131	0.0013780462490059\\
132	0.0013780462490059\\
133	0.0013780462490059\\
134	0.0013780462490059\\
135	0.0013780462490059\\
136	0.0013780462490059\\
137	0.0013780462490059\\
138	0.0013780462490059\\
139	0.0013780462490059\\
140	0.0013780462490059\\
141	0.0013780462490059\\
142	0.0013780462490059\\
143	0.0013780462490059\\
144	0.0013780462490059\\
145	0.0013780462490059\\
146	0.0013780462490059\\
147	0.0013780462490059\\
148	0.0013780462490059\\
149	0.0013780462490059\\
150	0.0013780462490059\\
151	0.0013780462490059\\
152	0.0013780462490059\\
153	0.0013780462490059\\
154	0.0013780462490059\\
155	0.0013780462490059\\
156	0.0013780462490059\\
157	0.0013780462490059\\
158	0.0013780462490059\\
159	0.0013780462490059\\
160	0.0013780462490059\\
161	0.0013780462490059\\
162	0.0013780462490059\\
163	0.0013780462490059\\
164	0.0013780462490059\\
165	0.0013780462490059\\
166	0.0013780462490059\\
167	0.0013780462490059\\
168	0.0013780462490059\\
169	0.0013780462490059\\
170	0.0013780462490059\\
171	0.0013780462490059\\
172	0.0013780462490059\\
173	0.0013780462490059\\
174	0.0013780462490059\\
175	0.0013780462490059\\
176	0.0013780462490059\\
177	0.0013780462490059\\
178	0.0013780462490059\\
179	0.0013780462490059\\
180	0.0013780462490059\\
181	0.0013780462490059\\
182	0.0013780462490059\\
183	0.0013780462490059\\
184	0.0013780462490059\\
185	0.0013780462490059\\
186	0.0013780462490059\\
187	0.0013780462490059\\
188	0.0013780462490059\\
189	0.0013780462490059\\
190	0.0013780462490059\\
191	0.0013780462490059\\
192	0.0013780462490059\\
193	0.0013780462490059\\
194	0.0013780462490059\\
195	0.0013780462490059\\
196	0.0013780462490059\\
197	0.0013780462490059\\
198	0.0013780462490059\\
199	0.0013780462490059\\
200	0.0013780462490059\\
201	0.0013780462490059\\
202	0.0013780462490059\\
203	0.0013780462490059\\
204	0.0013780462490059\\
205	0.0013780462490059\\
206	0.0013780462490059\\
207	0.0013780462490059\\
208	0.0013780462490059\\
209	0.0013780462490059\\
210	0.0013780462490059\\
211	0.0013780462490059\\
212	0.0013780462490059\\
213	0.0013780462490059\\
214	0.0013780462490059\\
215	0.0013780462490059\\
216	0.0013780462490059\\
217	0.0013780462490059\\
218	0.0013780462490059\\
219	0.0013780462490059\\
220	0.0013780462490059\\
221	0.0013780462490059\\
222	0.0013780462490059\\
223	0.0013780462490059\\
224	0.0013780462490059\\
225	0.0013780462490059\\
226	0.0013780462490059\\
227	0.0013780462490059\\
228	0.0013780462490059\\
229	0.0013780462490059\\
230	0.0013780462490059\\
231	0.0013780462490059\\
232	0.0013780462490059\\
233	0.0013780462490059\\
234	0.0013780462490059\\
235	0.0013780462490059\\
236	0.0013780462490059\\
237	0.0013780462490059\\
238	0.0013780462490059\\
239	0.0013780462490059\\
240	0.0013780462490059\\
241	0.0013780462490059\\
242	0.0013780462490059\\
243	0.0013780462490059\\
244	0.0013780462490059\\
245	0.0013780462490059\\
246	0.0013780462490059\\
247	0.0013780462490059\\
248	0.0013780462490059\\
249	0.0013780462490059\\
250	0.0013780462490059\\
251	0.0013780462490059\\
252	0.0013780462490059\\
253	0.0013780462490059\\
254	0.0013780462490059\\
255	0.0013780462490059\\
256	0.0013780462490059\\
257	0.0013780462490059\\
258	0.0013780462490059\\
259	0.0013780462490059\\
260	0.0013780462490059\\
261	0.0013780462490059\\
262	0.0013780462490059\\
263	0.0013780462490059\\
264	0.0013780462490059\\
265	0.0013780462490059\\
266	0.0013780462490059\\
267	0.0013780462490059\\
268	0.0013780462490059\\
269	0.0013780462490059\\
270	0.0013780462490059\\
271	0.0013780462490059\\
272	0.0013780462490059\\
273	0.0013780462490059\\
274	0.0013780462490059\\
275	0.0013780462490059\\
276	0.0013780462490059\\
277	0.0013780462490059\\
278	0.0013780462490059\\
279	0.0013780462490059\\
280	0.0013780462490059\\
281	0.0013780462490059\\
282	0.0013780462490059\\
283	0.0013780462490059\\
284	0.0013780462490059\\
285	0.0013780462490059\\
286	0.0013780462490059\\
287	0.0013780462490059\\
288	0.0013780462490059\\
289	0.0013780462490059\\
290	0.0013780462490059\\
291	0.0013780462490059\\
292	0.0013780462490059\\
293	0.0013780462490059\\
294	0.0013780462490059\\
295	0.0013780462490059\\
296	0.0013780462490059\\
297	0.0013780462490059\\
298	0.0013780462490059\\
299	0.0013780462490059\\
300	0.0013780462490059\\
301	0.0013780462490059\\
302	0.0013780462490059\\
303	0.0013780462490059\\
304	0.0013780462490059\\
305	0.0013780462490059\\
306	0.0013780462490059\\
307	0.0013780462490059\\
308	0.0013780462490059\\
309	0.0013780462490059\\
310	0.0013780462490059\\
311	0.0013780462490059\\
312	0.0013780462490059\\
313	0.0013780462490059\\
314	0.0013780462490059\\
315	0.0013780462490059\\
316	0.0013780462490059\\
317	0.0013780462490059\\
318	0.0013780462490059\\
319	0.0013780462490059\\
320	0.0013780462490059\\
321	0.0013780462490059\\
322	0.0013780462490059\\
323	0.0013780462490059\\
324	0.0013780462490059\\
325	0.0013780462490059\\
326	0.0013780462490059\\
327	0.0013780462490059\\
328	0.0013780462490059\\
329	0.0013780462490059\\
330	0.0013780462490059\\
331	0.0013780462490059\\
332	0.0013780462490059\\
333	0.0013780462490059\\
334	0.0013780462490059\\
335	0.0013780462490059\\
336	0.0013780462490059\\
337	0.0013780462490059\\
338	0.0013780462490059\\
339	0.0013780462490059\\
340	0.0013780462490059\\
341	0.0013780462490059\\
342	0.0013780462490059\\
343	0.0013780462490059\\
344	0.0013780462490059\\
345	0.0013780462490059\\
346	0.0013780462490059\\
347	0.0013780462490059\\
348	0.0013780462490059\\
349	0.0013780462490059\\
350	0.0013780462490059\\
351	0.0013780462490059\\
352	0.0013780462490059\\
353	0.0013780462490059\\
354	0.0013780462490059\\
355	0.0013780462490059\\
356	0.0013780462490059\\
357	0.0013780462490059\\
358	0.0013780462490059\\
359	0.0013780462490059\\
360	0.0013780462490059\\
361	0.0013780462490059\\
362	0.0013780462490059\\
363	0.0013780462490059\\
364	0.0013780462490059\\
365	0.0013780462490059\\
366	0.0013780462490059\\
367	0.0013780462490059\\
368	0.0013780462490059\\
369	0.0013780462490059\\
370	0.0013780462490059\\
371	0.0013780462490059\\
372	0.0013780462490059\\
373	0.0013780462490059\\
374	0.0013780462490059\\
375	0.0013780462490059\\
376	0.0013780462490059\\
377	0.0013780462490059\\
378	0.0013780462490059\\
379	0.0013780462490059\\
380	0.0013780462490059\\
381	0.0013780462490059\\
382	0.0013780462490059\\
383	0.0013780462490059\\
384	0.0013780462490059\\
385	0.0013780462490059\\
386	0.0013780462490059\\
387	0.0013780462490059\\
388	0.0013780462490059\\
389	0.0013780462490059\\
390	0.0013780462490059\\
391	0.0013780462490059\\
392	0.0013780462490059\\
393	0.0013780462490059\\
394	0.0013780462490059\\
395	0.0013780462490059\\
396	0.0013780462490059\\
397	0.0013780462490059\\
398	0.0013780462490059\\
399	0.0013780462490059\\
400	0.0013780462490059\\
401	0.0013780462490059\\
402	0.0013780462490059\\
403	0.0013780462490059\\
404	0.0013780462490059\\
405	0.0013780462490059\\
406	0.0013780462490059\\
407	0.0013780462490059\\
408	0.0013780462490059\\
409	0.0013780462490059\\
410	0.0013780462490059\\
411	0.0013780462490059\\
412	0.0013780462490059\\
413	0.0013780462490059\\
414	0.0013780462490059\\
415	0.0013780462490059\\
416	0.0013780462490059\\
417	0.0013780462490059\\
418	0.0013780462490059\\
419	0.0013780462490059\\
420	0.0013780462490059\\
421	0.0013780462490059\\
422	0.0013780462490059\\
423	0.0013780462490059\\
424	0.0013780462490059\\
425	0.0013780462490059\\
426	0.0013780462490059\\
427	0.0013780462490059\\
428	0.0013780462490059\\
429	0.0013780462490059\\
430	0.0013780462490059\\
431	0.0013780462490059\\
432	0.0013780462490059\\
433	0.0013780462490059\\
434	0.0013780462490059\\
435	0.0013780462490059\\
436	0.0013780462490059\\
437	0.0013780462490059\\
438	0.0013780462490059\\
439	0.0013780462490059\\
440	0.0013780462490059\\
441	0.0013780462490059\\
442	0.0013780462490059\\
443	0.0013780462490059\\
444	0.0013780462490059\\
445	0.0013780462490059\\
446	0.0013780462490059\\
447	0.0013780462490059\\
448	0.0013780462490059\\
449	0.0013780462490059\\
450	0.0013780462490059\\
451	0.0013780462490059\\
452	0.0013780462490059\\
453	0.0013780462490059\\
454	0.0013780462490059\\
455	0.0013780462490059\\
456	0.0013780462490059\\
457	0.0013780462490059\\
458	0.0013780462490059\\
459	0.0013780462490059\\
460	0.0013780462490059\\
461	0.0013780462490059\\
462	0.0013780462490059\\
463	0.0013780462490059\\
464	0.0013780462490059\\
465	0.0013780462490059\\
466	0.0013780462490059\\
467	0.0013780462490059\\
468	0.0013780462490059\\
469	0.0013780462490059\\
470	0.0013780462490059\\
471	0.0013780462490059\\
472	0.0013780462490059\\
473	0.0013780462490059\\
474	0.0013780462490059\\
475	0.0013780462490059\\
476	0.0013780462490059\\
477	0.0013780462490059\\
478	0.0013780462490059\\
479	0.0013780462490059\\
480	0.0013780462490059\\
481	0.0013780462490059\\
482	0.0013780462490059\\
483	0.0013780462490059\\
484	0.0013780462490059\\
485	0.0013780462490059\\
486	0.0013780462490059\\
487	0.0013780462490059\\
488	0.0013780462490059\\
489	0.0013780462490059\\
490	0.0013780462490059\\
491	0.0013780462490059\\
492	0.0013780462490059\\
493	0.0013780462490059\\
494	0.0013780462490059\\
495	0.0013780462490059\\
496	0.0013780462490059\\
497	0.0013780462490059\\
498	0.0013780462490059\\
499	0.0013780462490059\\
500	0.0013780462490059\\
501	0.0013780462490059\\
502	0.0013780462490059\\
503	0.0013780462490059\\
504	0.0013780462490059\\
505	0.0013780462490059\\
506	0.0013780462490059\\
507	0.0013780462490059\\
508	0.0013780462490059\\
509	0.0013780462490059\\
510	0.0013780462490059\\
511	0.0013780462490059\\
512	0.0013780462490059\\
513	0.0013780462490059\\
514	0.0013780462490059\\
515	0.0013780462490059\\
516	0.0013780462490059\\
517	0.0013780462490059\\
518	0.0013780462490059\\
519	0.0013780462490059\\
520	0.0013780462490059\\
521	0.0013780462490059\\
522	0.0013780462490059\\
523	0.0013780462490059\\
524	0.0013780462490059\\
525	0.0013780462490059\\
526	0.0013780462490059\\
527	0.0013780462490059\\
528	0.0013780462490059\\
529	0.0013780462490059\\
530	0.0013780462490059\\
531	0.0013780462490059\\
532	0.0013780462490059\\
533	0.0013780462490059\\
534	0.0013780462490059\\
535	0.0013780462490059\\
536	0.0013780462490059\\
537	0.0013780462490059\\
538	0.0013780462490059\\
539	0.0013780462490059\\
540	0.0013780462490059\\
541	0.0013780462490059\\
542	0.0013780462490059\\
543	0.0013780462490059\\
544	0.0013780462490059\\
545	0.0013780462490059\\
546	0.0013780462490059\\
547	0.0013780462490059\\
548	0.0013780462490059\\
549	0.0013780462490059\\
550	0.0013780462490059\\
551	0.0013780462490059\\
552	0.0013780462490059\\
553	0.0013780462490059\\
554	0.0013780462490059\\
555	0.0013780462490059\\
556	0.0013780462490059\\
557	0.0013780462490059\\
558	0.0013780462490059\\
559	0.0013780462490059\\
560	0.0013780462490059\\
561	0.0013780462490059\\
562	0.0013780462490059\\
563	0.0013780462490059\\
564	0.0013780462490059\\
565	0.00136279015443986\\
566	0.00125827733001416\\
567	0.00115323348010358\\
568	0.00104696223543405\\
569	0.000939078664145309\\
570	0.000831870952691528\\
571	0.000725446290622915\\
572	0.000624556670773367\\
573	0.00053496420760559\\
574	0.000390445210304555\\
575	0.000253767653105659\\
576	0.000124813533253903\\
577	3.73689878182394e-06\\
578	0\\
579	0\\
580	0\\
581	0\\
582	0\\
583	0\\
584	0\\
585	0\\
586	0\\
587	0\\
588	0\\
589	0\\
590	0\\
591	0\\
592	6.50501568911646e-05\\
593	0.000171129827925065\\
594	0.000294096086145507\\
595	0.000433306443467276\\
596	0.000584745759468008\\
597	0.000739344758132349\\
598	0.00345717029705862\\
599	0\\
600	0\\
};
\addplot [color=mycolor14,solid,forget plot]
  table[row sep=crcr]{%
1	0.00634705736923244\\
2	0.00634705707591813\\
3	0.00634705677729562\\
4	0.00634705647326904\\
5	0.00634705616374076\\
6	0.00634705584861141\\
7	0.00634705552777983\\
8	0.00634705520114302\\
9	0.00634705486859616\\
10	0.00634705453003251\\
11	0.00634705418534343\\
12	0.00634705383441831\\
13	0.00634705347714455\\
14	0.00634705311340754\\
15	0.00634705274309059\\
16	0.00634705236607491\\
17	0.0063470519822396\\
18	0.00634705159146155\\
19	0.00634705119361545\\
20	0.00634705078857375\\
21	0.00634705037620657\\
22	0.00634704995638174\\
23	0.00634704952896467\\
24	0.00634704909381839\\
25	0.00634704865080343\\
26	0.00634704819977784\\
27	0.00634704774059711\\
28	0.00634704727311413\\
29	0.00634704679717915\\
30	0.00634704631263973\\
31	0.00634704581934069\\
32	0.00634704531712405\\
33	0.00634704480582902\\
34	0.00634704428529189\\
35	0.00634704375534603\\
36	0.00634704321582182\\
37	0.00634704266654658\\
38	0.00634704210734454\\
39	0.00634704153803677\\
40	0.00634704095844114\\
41	0.00634704036837222\\
42	0.00634703976764128\\
43	0.00634703915605621\\
44	0.00634703853342143\\
45	0.00634703789953786\\
46	0.00634703725420284\\
47	0.0063470365972101\\
48	0.00634703592834965\\
49	0.00634703524740774\\
50	0.00634703455416678\\
51	0.00634703384840529\\
52	0.00634703312989782\\
53	0.00634703239841486\\
54	0.00634703165372281\\
55	0.00634703089558388\\
56	0.006347030123756\\
57	0.00634702933799278\\
58	0.00634702853804341\\
59	0.00634702772365259\\
60	0.00634702689456044\\
61	0.00634702605050245\\
62	0.00634702519120935\\
63	0.00634702431640704\\
64	0.00634702342581654\\
65	0.00634702251915386\\
66	0.00634702159612992\\
67	0.0063470206564505\\
68	0.00634701969981608\\
69	0.00634701872592179\\
70	0.0063470177344573\\
71	0.00634701672510676\\
72	0.00634701569754863\\
73	0.00634701465145565\\
74	0.00634701358649468\\
75	0.00634701250232665\\
76	0.00634701139860642\\
77	0.00634701027498266\\
78	0.0063470091310978\\
79	0.00634700796658785\\
80	0.00634700678108232\\
81	0.00634700557420411\\
82	0.00634700434556936\\
83	0.00634700309478739\\
84	0.00634700182146052\\
85	0.00634700052518396\\
86	0.00634699920554571\\
87	0.0063469978621264\\
88	0.00634699649449918\\
89	0.00634699510222959\\
90	0.00634699368487539\\
91	0.00634699224198647\\
92	0.00634699077310468\\
93	0.0063469892777637\\
94	0.00634698775548889\\
95	0.00634698620579713\\
96	0.00634698462819671\\
97	0.00634698302218713\\
98	0.00634698138725896\\
99	0.00634697972289372\\
100	0.00634697802856366\\
101	0.00634697630373162\\
102	0.00634697454785088\\
103	0.00634697276036498\\
104	0.00634697094070752\\
105	0.00634696908830205\\
106	0.0063469672025618\\
107	0.00634696528288959\\
108	0.0063469633286776\\
109	0.00634696133930715\\
110	0.00634695931414859\\
111	0.00634695725256103\\
112	0.00634695515389217\\
113	0.00634695301747813\\
114	0.00634695084264318\\
115	0.00634694862869959\\
116	0.00634694637494736\\
117	0.00634694408067408\\
118	0.00634694174515463\\
119	0.006346939367651\\
120	0.00634693694741208\\
121	0.00634693448367337\\
122	0.00634693197565678\\
123	0.00634692942257041\\
124	0.00634692682360826\\
125	0.00634692417795001\\
126	0.00634692148476076\\
127	0.00634691874319076\\
128	0.00634691595237519\\
129	0.00634691311143384\\
130	0.00634691021947086\\
131	0.0063469072755745\\
132	0.00634690427881681\\
133	0.00634690122825337\\
134	0.00634689812292295\\
135	0.00634689496184731\\
136	0.00634689174403079\\
137	0.00634688846846009\\
138	0.00634688513410389\\
139	0.0063468817399126\\
140	0.00634687828481799\\
141	0.00634687476773284\\
142	0.0063468711875507\\
143	0.00634686754314543\\
144	0.00634686383337096\\
145	0.00634686005706085\\
146	0.00634685621302799\\
147	0.00634685230006422\\
148	0.00634684831693993\\
149	0.00634684426240374\\
150	0.00634684013518206\\
151	0.00634683593397872\\
152	0.00634683165747458\\
153	0.0063468273043271\\
154	0.00634682287316997\\
155	0.00634681836261263\\
156	0.0063468137712399\\
157	0.00634680909761152\\
158	0.00634680434026169\\
159	0.00634679949769865\\
160	0.00634679456840423\\
161	0.00634678955083332\\
162	0.00634678444341348\\
163	0.00634677924454438\\
164	0.00634677395259738\\
165	0.00634676856591495\\
166	0.00634676308281026\\
167	0.00634675750156654\\
168	0.00634675182043668\\
169	0.00634674603764258\\
170	0.00634674015137471\\
171	0.00634673415979143\\
172	0.00634672806101856\\
173	0.00634672185314869\\
174	0.00634671553424066\\
175	0.00634670910231894\\
176	0.00634670255537304\\
177	0.00634669589135686\\
178	0.00634668910818808\\
179	0.00634668220374752\\
180	0.0063466751758785\\
181	0.00634666802238616\\
182	0.00634666074103676\\
183	0.00634665332955706\\
184	0.00634664578563356\\
185	0.00634663810691181\\
186	0.00634663029099569\\
187	0.00634662233544665\\
188	0.006346614237783\\
189	0.00634660599547909\\
190	0.00634659760596459\\
191	0.00634658906662365\\
192	0.00634658037479413\\
193	0.00634657152776673\\
194	0.00634656252278421\\
195	0.00634655335704053\\
196	0.00634654402767992\\
197	0.00634653453179609\\
198	0.00634652486643127\\
199	0.0063465150285753\\
200	0.00634650501516476\\
201	0.00634649482308192\\
202	0.00634648444915386\\
203	0.00634647389015147\\
204	0.00634646314278842\\
205	0.00634645220372016\\
206	0.00634644106954291\\
207	0.00634642973679255\\
208	0.00634641820194359\\
209	0.00634640646140806\\
210	0.0063463945115344\\
211	0.00634638234860632\\
212	0.00634636996884166\\
213	0.00634635736839118\\
214	0.00634634454333739\\
215	0.00634633148969334\\
216	0.00634631820340137\\
217	0.00634630468033181\\
218	0.00634629091628174\\
219	0.00634627690697368\\
220	0.00634626264805419\\
221	0.0063462481350926\\
222	0.00634623336357957\\
223	0.00634621832892568\\
224	0.00634620302646004\\
225	0.00634618745142875\\
226	0.00634617159899349\\
227	0.00634615546422996\\
228	0.00634613904212636\\
229	0.00634612232758177\\
230	0.00634610531540462\\
231	0.006346088000311\\
232	0.00634607037692303\\
233	0.00634605243976716\\
234	0.00634603418327243\\
235	0.00634601560176877\\
236	0.00634599668948512\\
237	0.00634597744054773\\
238	0.00634595784897819\\
239	0.00634593790869163\\
240	0.00634591761349475\\
241	0.00634589695708388\\
242	0.00634587593304298\\
243	0.00634585453484162\\
244	0.00634583275583291\\
245	0.00634581058925138\\
246	0.00634578802821086\\
247	0.00634576506570227\\
248	0.00634574169459141\\
249	0.00634571790761672\\
250	0.00634569369738691\\
251	0.00634566905637868\\
252	0.00634564397693429\\
253	0.00634561845125914\\
254	0.00634559247141927\\
255	0.00634556602933886\\
256	0.00634553911679764\\
257	0.00634551172542829\\
258	0.00634548384671375\\
259	0.00634545547198453\\
260	0.00634542659241593\\
261	0.00634539719902521\\
262	0.00634536728266877\\
263	0.00634533683403918\\
264	0.00634530584366226\\
265	0.00634527430189401\\
266	0.00634524219891754\\
267	0.00634520952473997\\
268	0.00634517626918919\\
269	0.00634514242191067\\
270	0.00634510797236414\\
271	0.00634507290982028\\
272	0.00634503722335723\\
273	0.00634500090185709\\
274	0.00634496393400233\\
275	0.00634492630827227\\
276	0.00634488801293938\\
277	0.0063448490360656\\
278	0.00634480936549852\\
279	0.0063447689888675\\
280	0.00634472789357982\\
281	0.00634468606681663\\
282	0.00634464349552894\\
283	0.00634460016643353\\
284	0.00634455606600872\\
285	0.0063445111804902\\
286	0.00634446549586669\\
287	0.00634441899787558\\
288	0.00634437167199854\\
289	0.00634432350345699\\
290	0.00634427447720758\\
291	0.00634422457793762\\
292	0.00634417379006036\\
293	0.0063441220977103\\
294	0.00634406948473844\\
295	0.00634401593470744\\
296	0.00634396143088677\\
297	0.00634390595624778\\
298	0.00634384949345877\\
299	0.00634379202488\\
300	0.00634373353255865\\
301	0.00634367399822384\\
302	0.00634361340328147\\
303	0.00634355172880927\\
304	0.00634348895555162\\
305	0.00634342506391455\\
306	0.0063433600339606\\
307	0.00634329384540378\\
308	0.00634322647760438\\
309	0.00634315790956398\\
310	0.00634308811992051\\
311	0.00634301708694385\\
312	0.00634294478853137\\
313	0.00634287120220282\\
314	0.0063427963050959\\
315	0.00634272007396187\\
316	0.00634264248516149\\
317	0.00634256351466105\\
318	0.00634248313802878\\
319	0.0063424013304314\\
320	0.00634231806663113\\
321	0.00634223332098303\\
322	0.00634214706743279\\
323	0.00634205927951495\\
324	0.00634196993035181\\
325	0.00634187899265279\\
326	0.00634178643871461\\
327	0.00634169224042198\\
328	0.00634159636924884\\
329	0.0063414987962596\\
330	0.00634139949210948\\
331	0.00634129842704191\\
332	0.00634119557088033\\
333	0.00634109089301242\\
334	0.00634098436237369\\
335	0.00634087594746614\\
336	0.00634076561650549\\
337	0.00634065333780671\\
338	0.00634053908012841\\
339	0.00634042281148967\\
340	0.0063403044970337\\
341	0.00634018409871652\\
342	0.00634006157113215\\
343	0.00633993685233052\\
344	0.00633980984603337\\
345	0.00633968039571144\\
346	0.00633954827735552\\
347	0.00633941331734338\\
348	0.00633927576477506\\
349	0.00633913590350577\\
350	0.00633899369546465\\
351	0.00633884910201422\\
352	0.00633870208394732\\
353	0.00633855260148489\\
354	0.00633840061427464\\
355	0.00633824608138921\\
356	0.0063380889613211\\
357	0.00633792921197611\\
358	0.00633776679067756\\
359	0.00633760165416524\\
360	0.00633743375859476\\
361	0.00633726305953701\\
362	0.00633708951197813\\
363	0.0063369130703195\\
364	0.00633673368837812\\
365	0.00633655131938696\\
366	0.00633636591599548\\
367	0.0063361774302701\\
368	0.00633598581369455\\
369	0.00633579101716998\\
370	0.00633559299101447\\
371	0.00633539168496156\\
372	0.00633518704815668\\
373	0.00633497902915057\\
374	0.00633476757589027\\
375	0.00633455263571372\\
376	0.00633433415536172\\
377	0.0063341120810081\\
378	0.00633388635826129\\
379	0.00633365693222053\\
380	0.00633342374756986\\
381	0.00633318674867619\\
382	0.00633294587952931\\
383	0.00633270108308434\\
384	0.00633245229929575\\
385	0.00633219946197002\\
386	0.00633194249927014\\
387	0.00633168134762682\\
388	0.00633141594441352\\
389	0.00633114622526075\\
390	0.00633087212386279\\
391	0.00633059357183474\\
392	0.00633031049874965\\
393	0.00633002283268106\\
394	0.00632973050203992\\
395	0.00632943344055455\\
396	0.00632913159953022\\
397	0.00632882497602954\\
398	0.00632851367297699\\
399	0.00632819801383586\\
400	0.00632787871871675\\
401	0.00632755703509414\\
402	0.00632723434369704\\
403	0.00632691022043075\\
404	0.00632658003640803\\
405	0.00632624367882569\\
406	0.00632590102995405\\
407	0.00632555196672422\\
408	0.00632519636035737\\
409	0.00632483407667918\\
410	0.00632446497754873\\
411	0.00632408892036726\\
412	0.00632370574530547\\
413	0.0063233152870529\\
414	0.00632291737454029\\
415	0.00632251183064566\\
416	0.00632209847188187\\
417	0.00632167710805953\\
418	0.00632124754191151\\
419	0.00632080956865844\\
420	0.00632036297551085\\
421	0.00631990754118923\\
422	0.00631944303564629\\
423	0.00631896921980199\\
424	0.00631848584474056\\
425	0.00631799265114201\\
426	0.00631748936867251\\
427	0.00631697571532665\\
428	0.00631645139670927\\
429	0.00631591610523212\\
430	0.00631536951917585\\
431	0.00631481130152096\\
432	0.0063142410983804\\
433	0.00631365853681534\\
434	0.00631306322201186\\
435	0.00631245473496562\\
436	0.00631183263545229\\
437	0.00631119648081164\\
438	0.00631054585902489\\
439	0.00630988033286531\\
440	0.00630919937951012\\
441	0.00630850244862298\\
442	0.00630778896062646\\
443	0.00630705830485506\\
444	0.00630630983758596\\
445	0.00630554287987795\\
446	0.00630475671460947\\
447	0.00630395057978746\\
448	0.00630312364952706\\
449	0.00630227499286354\\
450	0.00630140346092567\\
451	0.00630050737731262\\
452	0.00629958369754549\\
453	0.00629862575552163\\
454	0.00629761732521094\\
455	0.0062965172609967\\
456	0.00629522043131399\\
457	0.00629346826565445\\
458	0.00629058504385814\\
459	0.0062873984227585\\
460	0.00628414460028995\\
461	0.00628080773056847\\
462	0.00627737393016521\\
463	0.00627388356313013\\
464	0.00627033595250894\\
465	0.00626672881184384\\
466	0.00626305953089114\\
467	0.00625932510476634\\
468	0.00625552214579273\\
469	0.0062516473709978\\
470	0.00624769936191294\\
471	0.00624367602190079\\
472	0.00623957699600573\\
473	0.00623540724679653\\
474	0.00623118227627135\\
475	0.00622693215559939\\
476	0.00622268415786011\\
477	0.00621837197795093\\
478	0.00621391858504133\\
479	0.00620931339029792\\
480	0.00620454353957697\\
481	0.00619959164428468\\
482	0.0061944298689289\\
483	0.00618900452079593\\
484	0.00618319741935313\\
485	0.00617674152536256\\
486	0.00616948027811247\\
487	0.00616210182564511\\
488	0.00615460347021238\\
489	0.00614698244999517\\
490	0.00613923586265543\\
491	0.00613136059652419\\
492	0.0061233530577195\\
493	0.00611520758579711\\
494	0.00610691552533023\\
495	0.00609848237957893\\
496	0.00608990449054164\\
497	0.00608117670253305\\
498	0.00607229238821892\\
499	0.00606324736646706\\
500	0.00605403705724364\\
501	0.00604465615743606\\
502	0.00603509786883715\\
503	0.00602535233006483\\
504	0.00601540563529222\\
505	0.00600525878536194\\
506	0.00599488587033331\\
507	0.00598422125494825\\
508	0.00597312631007669\\
509	0.00596172141658392\\
510	0.00595018659363723\\
511	0.0059384445843753\\
512	0.00592631108743226\\
513	0.00591346613950271\\
514	0.00590030505807963\\
515	0.00588707956134307\\
516	0.00587379765610018\\
517	0.00586048147386627\\
518	0.00584718801494137\\
519	0.00583405829195754\\
520	0.00582141659957693\\
521	0.00580989012708187\\
522	0.00580008189998949\\
523	0.00578994549122259\\
524	0.00577913384386583\\
525	0.00576657393904838\\
526	0.00574897062406128\\
527	0.00570949806036672\\
528	0.00566622002656595\\
529	0.00562168348095022\\
530	0.00557566290236694\\
531	0.00552774050188853\\
532	0.00547700589151219\\
533	0.00541874111612479\\
534	0.00535561486132491\\
535	0.00529235810299489\\
536	0.00522897636497744\\
537	0.00516547212723779\\
538	0.00510184062425884\\
539	0.00503805005930613\\
540	0.0049739785608551\\
541	0.00490919741497271\\
542	0.00484167426677888\\
543	0.0047750592880797\\
544	0.00470938112578479\\
545	0.00464466913499734\\
546	0.00458095349021259\\
547	0.00451826420357372\\
548	0.00445662654337949\\
549	0.00439604589693207\\
550	0.00433645820630629\\
551	0.00427756785872679\\
552	0.00421831745834273\\
553	0.00415492967898489\\
554	0.0040927650695408\\
555	0.00403290499091029\\
556	0.00397537262231718\\
557	0.00392025786722828\\
558	0.00386761667012469\\
559	0.00381749584867485\\
560	0.0037699710534416\\
561	0.00372520839465025\\
562	0.00368359691573897\\
563	0.00364607931640829\\
564	0.00361495875009759\\
565	0.00358407587802502\\
566	0.00350180811894708\\
567	0.00341627094766928\\
568	0.00332212922189586\\
569	0.00322378609530446\\
570	0.00312258660790876\\
571	0.00301842823208778\\
572	0.00291083880439416\\
573	0.00279199464559352\\
574	0.00262090950248387\\
575	0.00244700241101585\\
576	0.00227017807912024\\
577	0.00209042395100323\\
578	0.0019079951662922\\
579	0.00172275411076606\\
580	0.00153451315287464\\
581	0.00134245890765045\\
582	0.00114797364967888\\
583	0.000951303160335593\\
584	0.000752186592492726\\
585	0.000549776472909163\\
586	0.000341218417480763\\
587	0.000117616391847642\\
588	0\\
589	0\\
590	0\\
591	0\\
592	0\\
593	0\\
594	0\\
595	0\\
596	0\\
597	0\\
598	0\\
599	0\\
600	0\\
};
\addplot [color=mycolor15,solid,forget plot]
  table[row sep=crcr]{%
1	0.0050928946278291\\
2	0.00509289365214289\\
3	0.00509289265879869\\
4	0.00509289164747749\\
5	0.00509289061785456\\
6	0.00509288956959928\\
7	0.00509288850237511\\
8	0.00509288741583942\\
9	0.00509288630964345\\
10	0.00509288518343212\\
11	0.00509288403684399\\
12	0.00509288286951108\\
13	0.00509288168105882\\
14	0.00509288047110587\\
15	0.00509287923926405\\
16	0.00509287798513818\\
17	0.00509287670832597\\
18	0.00509287540841791\\
19	0.0050928740849971\\
20	0.00509287273763914\\
21	0.00509287136591202\\
22	0.00509286996937594\\
23	0.00509286854758319\\
24	0.00509286710007804\\
25	0.00509286562639651\\
26	0.00509286412606633\\
27	0.00509286259860671\\
28	0.00509286104352824\\
29	0.0050928594603327\\
30	0.00509285784851292\\
31	0.0050928562075526\\
32	0.0050928545369262\\
33	0.0050928528360987\\
34	0.00509285110452549\\
35	0.00509284934165218\\
36	0.00509284754691442\\
37	0.0050928457197377\\
38	0.00509284385953725\\
39	0.00509284196571773\\
40	0.00509284003767317\\
41	0.0050928380747867\\
42	0.00509283607643036\\
43	0.00509283404196496\\
44	0.0050928319707398\\
45	0.00509282986209254\\
46	0.00509282771534892\\
47	0.00509282552982261\\
48	0.00509282330481497\\
49	0.00509282103961481\\
50	0.0050928187334982\\
51	0.00509281638572822\\
52	0.00509281399555476\\
53	0.00509281156221422\\
54	0.00509280908492936\\
55	0.00509280656290897\\
56	0.00509280399534768\\
57	0.00509280138142569\\
58	0.0050927987203085\\
59	0.00509279601114666\\
60	0.00509279325307551\\
61	0.0050927904452149\\
62	0.00509278758666891\\
63	0.0050927846765256\\
64	0.00509278171385666\\
65	0.0050927786977172\\
66	0.00509277562714539\\
67	0.00509277250116219\\
68	0.00509276931877105\\
69	0.00509276607895757\\
70	0.0050927627806892\\
71	0.00509275942291493\\
72	0.00509275600456492\\
73	0.00509275252455024\\
74	0.00509274898176244\\
75	0.00509274537507326\\
76	0.00509274170333429\\
77	0.00509273796537655\\
78	0.00509273416001019\\
79	0.00509273028602408\\
80	0.00509272634218544\\
81	0.00509272232723947\\
82	0.00509271823990895\\
83	0.00509271407889383\\
84	0.00509270984287085\\
85	0.00509270553049308\\
86	0.00509270114038959\\
87	0.00509269667116491\\
88	0.0050926921213987\\
89	0.00509268748964523\\
90	0.00509268277443298\\
91	0.00509267797426414\\
92	0.00509267308761421\\
93	0.00509266811293145\\
94	0.00509266304863643\\
95	0.00509265789312156\\
96	0.00509265264475058\\
97	0.00509264730185801\\
98	0.0050926418627487\\
99	0.00509263632569725\\
100	0.00509263068894751\\
101	0.00509262495071201\\
102	0.00509261910917141\\
103	0.00509261316247397\\
104	0.00509260710873491\\
105	0.0050926009460359\\
106	0.00509259467242439\\
107	0.00509258828591309\\
108	0.00509258178447927\\
109	0.00509257516606419\\
110	0.00509256842857244\\
111	0.00509256156987129\\
112	0.00509255458779004\\
113	0.00509254748011936\\
114	0.00509254024461056\\
115	0.00509253287897496\\
116	0.00509252538088311\\
117	0.00509251774796413\\
118	0.00509250997780495\\
119	0.00509250206794957\\
120	0.0050924940158983\\
121	0.00509248581910697\\
122	0.00509247747498618\\
123	0.00509246898090049\\
124	0.00509246033416757\\
125	0.00509245153205742\\
126	0.0050924425717915\\
127	0.00509243345054189\\
128	0.00509242416543039\\
129	0.00509241471352766\\
130	0.0050924050918523\\
131	0.00509239529736995\\
132	0.00509238532699234\\
133	0.00509237517757631\\
134	0.00509236484592291\\
135	0.00509235432877635\\
136	0.00509234362282302\\
137	0.0050923327246905\\
138	0.00509232163094645\\
139	0.00509231033809762\\
140	0.00509229884258875\\
141	0.00509228714080147\\
142	0.00509227522905317\\
143	0.00509226310359592\\
144	0.00509225076061526\\
145	0.00509223819622906\\
146	0.00509222540648632\\
147	0.00509221238736592\\
148	0.00509219913477545\\
149	0.00509218564454988\\
150	0.00509217191245034\\
151	0.00509215793416276\\
152	0.00509214370529661\\
153	0.00509212922138346\\
154	0.00509211447787569\\
155	0.00509209947014505\\
156	0.00509208419348126\\
157	0.00509206864309053\\
158	0.0050920528140941\\
159	0.00509203670152675\\
160	0.00509202030033528\\
161	0.00509200360537694\\
162	0.00509198661141783\\
163	0.00509196931313133\\
164	0.00509195170509646\\
165	0.00509193378179618\\
166	0.00509191553761573\\
167	0.00509189696684087\\
168	0.00509187806365615\\
169	0.00509185882214313\\
170	0.00509183923627852\\
171	0.00509181929993237\\
172	0.00509179900686614\\
173	0.00509177835073085\\
174	0.00509175732506505\\
175	0.0050917359232929\\
176	0.00509171413872208\\
177	0.0050916919645418\\
178	0.00509166939382065\\
179	0.00509164641950451\\
180	0.00509162303441435\\
181	0.00509159923124401\\
182	0.005091575002558\\
183	0.00509155034078917\\
184	0.00509152523823638\\
185	0.00509149968706214\\
186	0.00509147367929021\\
187	0.00509144720680311\\
188	0.00509142026133964\\
189	0.00509139283449233\\
190	0.00509136491770483\\
191	0.00509133650226931\\
192	0.00509130757932373\\
193	0.00509127813984914\\
194	0.00509124817466685\\
195	0.00509121767443567\\
196	0.00509118662964895\\
197	0.00509115503063168\\
198	0.00509112286753751\\
199	0.00509109013034565\\
200	0.00509105680885785\\
201	0.00509102289269518\\
202	0.00509098837129484\\
203	0.00509095323390692\\
204	0.005090917469591\\
205	0.00509088106721283\\
206	0.00509084401544081\\
207	0.00509080630274256\\
208	0.00509076791738125\\
209	0.00509072884741202\\
210	0.00509068908067825\\
211	0.00509064860480777\\
212	0.00509060740720903\\
213	0.00509056547506718\\
214	0.00509052279534006\\
215	0.00509047935475418\\
216	0.00509043513980055\\
217	0.00509039013673047\\
218	0.00509034433155126\\
219	0.00509029771002189\\
220	0.0050902502576485\\
221	0.00509020195967992\\
222	0.00509015280110303\\
223	0.00509010276663806\\
224	0.00509005184073383\\
225	0.00509000000756285\\
226	0.00508994725101637\\
227	0.00508989355469935\\
228	0.00508983890192529\\
229	0.00508978327571099\\
230	0.00508972665877122\\
231	0.0050896690335133\\
232	0.00508961038203154\\
233	0.00508955068610163\\
234	0.00508948992717488\\
235	0.00508942808637236\\
236	0.00508936514447898\\
237	0.00508930108193742\\
238	0.00508923587884193\\
239	0.00508916951493205\\
240	0.00508910196958621\\
241	0.00508903322181522\\
242	0.00508896325025557\\
243	0.00508889203316274\\
244	0.00508881954840423\\
245	0.00508874577345261\\
246	0.00508867068537832\\
247	0.0050885942608424\\
248	0.0050885164760891\\
249	0.0050884373069383\\
250	0.00508835672877782\\
251	0.00508827471655561\\
252	0.00508819124477175\\
253	0.00508810628747034\\
254	0.00508801981823121\\
255	0.00508793181016152\\
256	0.00508784223588716\\
257	0.00508775106754401\\
258	0.00508765827676908\\
259	0.0050875638346914\\
260	0.00508746771192282\\
261	0.0050873698785486\\
262	0.00508727030411788\\
263	0.0050871689576339\\
264	0.00508706580754409\\
265	0.00508696082172997\\
266	0.00508685396749688\\
267	0.00508674521156349\\
268	0.00508663452005114\\
269	0.00508652185847301\\
270	0.00508640719172311\\
271	0.00508629048406506\\
272	0.00508617169912081\\
273	0.0050860507998589\\
274	0.00508592774858237\\
275	0.00508580250691664\\
276	0.00508567503579743\\
277	0.00508554529545836\\
278	0.00508541324541817\\
279	0.00508527884446789\\
280	0.00508514205065768\\
281	0.0050850028212835\\
282	0.00508486111287353\\
283	0.00508471688117443\\
284	0.00508457008113731\\
285	0.00508442066690353\\
286	0.00508426859179028\\
287	0.00508411380827591\\
288	0.00508395626798505\\
289	0.00508379592167355\\
290	0.00508363271921317\\
291	0.00508346660957611\\
292	0.00508329754081924\\
293	0.00508312546006827\\
294	0.00508295031350161\\
295	0.00508277204633408\\
296	0.00508259060280047\\
297	0.0050824059261389\\
298	0.00508221795857403\\
299	0.00508202664130014\\
300	0.00508183191446406\\
301	0.00508163371714798\\
302	0.00508143198735219\\
303	0.00508122666197778\\
304	0.00508101767680916\\
305	0.00508080496649673\\
306	0.00508058846453944\\
307	0.00508036810326731\\
308	0.005080143813824\\
309	0.00507991552614912\\
310	0.0050796831689607\\
311	0.00507944666973814\\
312	0.00507920595470679\\
313	0.00507896094882321\\
314	0.00507871157575707\\
315	0.00507845775787563\\
316	0.00507819941622867\\
317	0.00507793647053405\\
318	0.00507766883916395\\
319	0.00507739643913194\\
320	0.00507711918608097\\
321	0.00507683699427252\\
322	0.00507654977657687\\
323	0.00507625744446493\\
324	0.00507595990800155\\
325	0.00507565707584084\\
326	0.00507534885522337\\
327	0.00507503515197581\\
328	0.00507471587051285\\
329	0.00507439091384142\\
330	0.00507406018356655\\
331	0.00507372357989712\\
332	0.00507338100164791\\
333	0.00507303234623229\\
334	0.00507267750964002\\
335	0.00507231638640881\\
336	0.00507194886965547\\
337	0.00507157485138313\\
338	0.00507119422348189\\
339	0.00507080687923724\\
340	0.00507041271038931\\
341	0.00507001160023853\\
342	0.00506960342422547\\
343	0.00506918803930844\\
344	0.00506876525882263\\
345	0.0050683347975141\\
346	0.00506789616893734\\
347	0.00506744856867696\\
348	0.00506699109119338\\
349	0.00506652456440268\\
350	0.0050660501975954\\
351	0.00506556786113188\\
352	0.0050650774234284\\
353	0.00506457875094549\\
354	0.00506407170817921\\
355	0.00506355615765624\\
356	0.00506303195992941\\
357	0.00506249897356119\\
358	0.00506195705509287\\
359	0.00506140605906212\\
360	0.00506084583799855\\
361	0.0050602762424204\\
362	0.00505969712083228\\
363	0.00505910831972386\\
364	0.00505850968356943\\
365	0.00505790105482817\\
366	0.00505728227394488\\
367	0.00505665317935105\\
368	0.00505601360746598\\
369	0.00505536339269754\\
370	0.00505470236744231\\
371	0.00505403036208429\\
372	0.00505334720499054\\
373	0.00505265272250077\\
374	0.00505194673890596\\
375	0.00505122907641344\\
376	0.00505049955511283\\
377	0.00504975799299523\\
378	0.00504900420605909\\
379	0.00504823800823944\\
380	0.00504745921149303\\
381	0.00504666762595958\\
382	0.00504586306014246\\
383	0.00504504532078459\\
384	0.00504421421144329\\
385	0.00504336952770216\\
386	0.00504251104749455\\
387	0.00504163852689711\\
388	0.00504075174925823\\
389	0.0050398505012029\\
390	0.00503893456352778\\
391	0.00503800371050569\\
392	0.00503705770926875\\
393	0.00503609631957294\\
394	0.00503511929472342\\
395	0.00503412638559884\\
396	0.00503311735245732\\
397	0.00503209199545522\\
398	0.00503105022820599\\
399	0.00502999224452367\\
400	0.00502891886763283\\
401	0.00502783218799178\\
402	0.00502673639327816\\
403	0.00502563753529514\\
404	0.00502453721029399\\
405	0.00502341626063081\\
406	0.00502227430254632\\
407	0.00502111093574222\\
408	0.00501992574202896\\
409	0.00501871828365271\\
410	0.00501748810334104\\
411	0.00501623472918392\\
412	0.005014957679507\\
413	0.00501365640947008\\
414	0.00501233035555447\\
415	0.00501097893461764\\
416	0.00500960154289323\\
417	0.00500819755493185\\
418	0.00500676632246699\\
419	0.0050053071731649\\
420	0.00500381940917676\\
421	0.0050023023054178\\
422	0.00500075510777094\\
423	0.00499917703210561\\
424	0.00499756726373227\\
425	0.00499592495449141\\
426	0.0049942492208264\\
427	0.0049925391417214\\
428	0.00499079375648711\\
429	0.00498901206236689\\
430	0.00498719301191277\\
431	0.00498533551003465\\
432	0.00498343841053814\\
433	0.00498150051182483\\
434	0.00497952055128766\\
435	0.00497749719812845\\
436	0.00497542904612344\\
437	0.00497331461496192\\
438	0.00497115238815756\\
439	0.00496894092847353\\
440	0.00496667873037992\\
441	0.00496436401488461\\
442	0.00496199490964288\\
443	0.00495956944314037\\
444	0.00495708553846413\\
445	0.00495454100667428\\
446	0.00495193353985444\\
447	0.0049492607032568\\
448	0.00494651992026552\\
449	0.00494370842112155\\
450	0.00494082314810791\\
451	0.00493786049973932\\
452	0.00493481563215763\\
453	0.00493168056649731\\
454	0.00492843908689531\\
455	0.0049250531026715\\
456	0.00492142645084939\\
457	0.00491730494594139\\
458	0.00491208803876988\\
459	0.00490239900538334\\
460	0.00489149692452099\\
461	0.00488036978683903\\
462	0.0048689610936534\\
463	0.00485718523999083\\
464	0.00484520977544474\\
465	0.00483303324960802\\
466	0.0048206477514237\\
467	0.00480804428243009\\
468	0.0047952124548571\\
469	0.00478214021535745\\
470	0.00476881468701787\\
471	0.00475523106750443\\
472	0.00474138065763046\\
473	0.00472725851969782\\
474	0.00471287490790399\\
475	0.00469827434654054\\
476	0.00468356348793886\\
477	0.00466889699916233\\
478	0.00465405026924246\\
479	0.00463871239017449\\
480	0.00462284724770989\\
481	0.00460641219022524\\
482	0.00458935239600222\\
483	0.00457158616356049\\
484	0.0045529655160597\\
485	0.0045331696140612\\
486	0.00451120800814773\\
487	0.00448604223992208\\
488	0.00446045227420199\\
489	0.00443442838203219\\
490	0.00440796066997701\\
491	0.00438103880866148\\
492	0.00435365189049488\\
493	0.00432578813853256\\
494	0.00429743538016299\\
495	0.00426858429943873\\
496	0.00423922283061158\\
497	0.00420933758393212\\
498	0.00417891159500239\\
499	0.00414792434834379\\
500	0.00411636181189269\\
501	0.00408420885099331\\
502	0.0040514483152974\\
503	0.00401805870402856\\
504	0.00398400845011968\\
505	0.00394924445222225\\
506	0.00391378241291492\\
507	0.00387754851258355\\
508	0.00384034278626771\\
509	0.00380159692797307\\
510	0.00376171459281732\\
511	0.00372140524121531\\
512	0.00368042783019527\\
513	0.00363816404977489\\
514	0.00359328710604181\\
515	0.00354725849525978\\
516	0.00350104439854883\\
517	0.0034546796590398\\
518	0.0034082466826054\\
519	0.00336194673080687\\
520	0.00331627115305347\\
521	0.00327235601772224\\
522	0.00323245081899571\\
523	0.00319880639447882\\
524	0.00316409774624987\\
525	0.00312711704512529\\
526	0.00308409729865228\\
527	0.00302322049211327\\
528	0.00295799210260085\\
529	0.00289023238240467\\
530	0.00281957843518099\\
531	0.0027452515402818\\
532	0.00266566068754643\\
533	0.0025771460147355\\
534	0.00248201373084442\\
535	0.00238431533313198\\
536	0.00228393946136431\\
537	0.00218076254050936\\
538	0.00207463993340179\\
539	0.00196537226973244\\
540	0.00185259646353105\\
541	0.00173543479749105\\
542	0.00161171504284901\\
543	0.00148532245630597\\
544	0.0013561727032017\\
545	0.00122418802429166\\
546	0.00108928435702427\\
547	0.00095136984335789\\
548	0.000810335032611802\\
549	0.000666015326258075\\
550	0.00051806260568251\\
551	0.000365504845979632\\
552	0.000205227792137007\\
553	2.68864273520115e-05\\
554	0\\
555	0\\
556	0\\
557	0\\
558	0\\
559	0\\
560	0\\
561	0\\
562	0\\
563	0\\
564	0\\
565	0\\
566	0\\
567	0\\
568	0\\
569	0\\
570	0\\
571	0\\
572	0\\
573	0\\
574	0\\
575	0\\
576	0\\
577	0\\
578	0\\
579	0\\
580	0\\
581	0\\
582	0\\
583	0\\
584	0\\
585	0\\
586	0\\
587	0\\
588	0\\
589	0\\
590	0\\
591	0\\
592	0\\
593	0\\
594	0\\
595	0\\
596	0\\
597	0\\
598	0\\
599	0\\
600	0\\
};
\addplot [color=mycolor16,solid,forget plot]
  table[row sep=crcr]{%
1	0.00344286792547887\\
2	0.0034428627298262\\
3	0.00344285744013911\\
4	0.00344285205471872\\
5	0.00344284657183551\\
6	0.00344284098972885\\
7	0.00344283530660638\\
8	0.00344282952064345\\
9	0.00344282362998257\\
10	0.00344281763273278\\
11	0.00344281152696909\\
12	0.00344280531073186\\
13	0.00344279898202617\\
14	0.00344279253882116\\
15	0.00344278597904946\\
16	0.00344277930060647\\
17	0.00344277250134973\\
18	0.00344276557909821\\
19	0.00344275853163165\\
20	0.00344275135668987\\
21	0.00344274405197199\\
22	0.00344273661513577\\
23	0.00344272904379685\\
24	0.00344272133552797\\
25	0.00344271348785824\\
26	0.00344270549827233\\
27	0.00344269736420971\\
28	0.00344268908306379\\
29	0.00344268065218116\\
30	0.00344267206886071\\
31	0.00344266333035277\\
32	0.00344265443385828\\
33	0.00344264537652789\\
34	0.00344263615546104\\
35	0.00344262676770509\\
36	0.00344261721025434\\
37	0.00344260748004913\\
38	0.00344259757397484\\
39	0.00344258748886092\\
40	0.0034425772214799\\
41	0.00344256676854639\\
42	0.00344255612671598\\
43	0.00344254529258426\\
44	0.00344253426268573\\
45	0.00344252303349268\\
46	0.0034425116014141\\
47	0.00344249996279459\\
48	0.00344248811391314\\
49	0.00344247605098199\\
50	0.00344246377014548\\
51	0.00344245126747877\\
52	0.00344243853898668\\
53	0.00344242558060235\\
54	0.00344241238818607\\
55	0.00344239895752387\\
56	0.00344238528432629\\
57	0.00344237136422697\\
58	0.00344235719278133\\
59	0.00344234276546515\\
60	0.00344232807767315\\
61	0.00344231312471755\\
62	0.00344229790182662\\
63	0.00344228240414316\\
64	0.00344226662672301\\
65	0.00344225056453347\\
66	0.00344223421245173\\
67	0.00344221756526331\\
68	0.00344220061766035\\
69	0.00344218336424005\\
70	0.00344216579950289\\
71	0.00344214791785095\\
72	0.00344212971358619\\
73	0.00344211118090861\\
74	0.00344209231391449\\
75	0.00344207310659451\\
76	0.00344205355283191\\
77	0.00344203364640054\\
78	0.00344201338096296\\
79	0.00344199275006843\\
80	0.00344197174715092\\
81	0.00344195036552705\\
82	0.00344192859839402\\
83	0.00344190643882747\\
84	0.00344188387977934\\
85	0.00344186091407569\\
86	0.00344183753441441\\
87	0.00344181373336301\\
88	0.00344178950335629\\
89	0.00344176483669394\\
90	0.00344173972553823\\
91	0.00344171416191148\\
92	0.00344168813769367\\
93	0.00344166164461982\\
94	0.00344163467427752\\
95	0.00344160721810423\\
96	0.0034415792673847\\
97	0.0034415508132482\\
98	0.00344152184666578\\
99	0.00344149235844749\\
100	0.00344146233923953\\
101	0.00344143177952131\\
102	0.00344140066960253\\
103	0.00344136899962017\\
104	0.00344133675953543\\
105	0.00344130393913062\\
106	0.00344127052800598\\
107	0.00344123651557647\\
108	0.00344120189106851\\
109	0.00344116664351662\\
110	0.00344113076176001\\
111	0.00344109423443917\\
112	0.00344105704999233\\
113	0.00344101919665188\\
114	0.00344098066244072\\
115	0.00344094143516861\\
116	0.00344090150242832\\
117	0.00344086085159187\\
118	0.00344081946980659\\
119	0.00344077734399114\\
120	0.0034407344608315\\
121	0.00344069080677685\\
122	0.00344064636803533\\
123	0.00344060113056991\\
124	0.00344055508009391\\
125	0.00344050820206669\\
126	0.00344046048168914\\
127	0.0034404119038991\\
128	0.00344036245336672\\
129	0.00344031211448975\\
130	0.00344026087138871\\
131	0.003440208707902\\
132	0.00344015560758095\\
133	0.0034401015536847\\
134	0.00344004652917507\\
135	0.00343999051671133\\
136	0.00343993349864483\\
137	0.00343987545701357\\
138	0.0034398163735367\\
139	0.00343975622960885\\
140	0.00343969500629445\\
141	0.00343963268432187\\
142	0.00343956924407749\\
143	0.0034395046655997\\
144	0.00343943892857273\\
145	0.00343937201232041\\
146	0.00343930389579982\\
147	0.00343923455759482\\
148	0.00343916397590946\\
149	0.00343909212856127\\
150	0.0034390189929745\\
151	0.00343894454617309\\
152	0.00343886876477369\\
153	0.00343879162497845\\
154	0.00343871310256769\\
155	0.00343863317289249\\
156	0.00343855181086712\\
157	0.0034384689909613\\
158	0.00343838468719239\\
159	0.00343829887311741\\
160	0.00343821152182493\\
161	0.00343812260592678\\
162	0.00343803209754964\\
163	0.00343793996832654\\
164	0.00343784618938807\\
165	0.00343775073135359\\
166	0.00343765356432215\\
167	0.00343755465786337\\
168	0.00343745398100804\\
169	0.00343735150223867\\
170	0.00343724718947977\\
171	0.00343714101008803\\
172	0.0034370329308423\\
173	0.00343692291793336\\
174	0.00343681093695358\\
175	0.00343669695288634\\
176	0.00343658093009526\\
177	0.00343646283231331\\
178	0.00343634262263159\\
179	0.0034362202634881\\
180	0.0034360957166561\\
181	0.00343596894323245\\
182	0.00343583990362558\\
183	0.00343570855754343\\
184	0.00343557486398096\\
185	0.00343543878120763\\
186	0.00343530026675451\\
187	0.00343515927740128\\
188	0.00343501576916291\\
189	0.00343486969727613\\
190	0.00343472101618568\\
191	0.00343456967953029\\
192	0.0034344156401284\\
193	0.00343425884996365\\
194	0.0034340992601701\\
195	0.00343393682101718\\
196	0.00343377148189436\\
197	0.00343360319129557\\
198	0.00343343189680334\\
199	0.0034332575450726\\
200	0.00343308008181429\\
201	0.00343289945177855\\
202	0.00343271559873771\\
203	0.00343252846546893\\
204	0.00343233799373655\\
205	0.00343214412427403\\
206	0.00343194679676573\\
207	0.00343174594982822\\
208	0.00343154152099128\\
209	0.00343133344667862\\
210	0.00343112166218817\\
211	0.00343090610167205\\
212	0.00343068669811622\\
213	0.00343046338331963\\
214	0.00343023608787317\\
215	0.00343000474113806\\
216	0.00342976927122399\\
217	0.00342952960496678\\
218	0.00342928566790563\\
219	0.00342903738426005\\
220	0.00342878467690622\\
221	0.00342852746735306\\
222	0.00342826567571775\\
223	0.00342799922070092\\
224	0.00342772801956126\\
225	0.00342745198808977\\
226	0.00342717104058347\\
227	0.00342688508981867\\
228	0.00342659404702375\\
229	0.0034262978218514\\
230	0.00342599632235041\\
231	0.00342568945493688\\
232	0.00342537712436497\\
233	0.00342505923369709\\
234	0.00342473568427346\\
235	0.0034244063756813\\
236	0.00342407120572325\\
237	0.00342373007038532\\
238	0.0034233828638043\\
239	0.0034230294782344\\
240	0.00342266980401349\\
241	0.00342230372952856\\
242	0.0034219311411806\\
243	0.00342155192334889\\
244	0.00342116595835452\\
245	0.00342077312642337\\
246	0.00342037330564831\\
247	0.00341996637195078\\
248	0.0034195521990416\\
249	0.00341913065838114\\
250	0.00341870161913868\\
251	0.00341826494815112\\
252	0.00341782050988081\\
253	0.00341736816637278\\
254	0.00341690777721102\\
255	0.00341643919947406\\
256	0.00341596228768975\\
257	0.00341547689378915\\
258	0.00341498286705964\\
259	0.00341448005409716\\
260	0.00341396829875758\\
261	0.00341344744210718\\
262	0.00341291732237227\\
263	0.00341237777488788\\
264	0.00341182863204548\\
265	0.00341126972323985\\
266	0.00341070087481491\\
267	0.0034101219100086\\
268	0.00340953264889679\\
269	0.00340893290833617\\
270	0.00340832250190613\\
271	0.00340770123984972\\
272	0.00340706892901375\\
273	0.00340642537278796\\
274	0.00340577037104266\\
275	0.00340510372006404\\
276	0.00340442521248928\\
277	0.00340373463724212\\
278	0.00340303177946615\\
279	0.00340231642045696\\
280	0.00340158833759315\\
281	0.00340084730426624\\
282	0.00340009308980937\\
283	0.00339932545942495\\
284	0.00339854417411105\\
285	0.00339774899058674\\
286	0.00339693966121618\\
287	0.00339611593393218\\
288	0.0033952775521561\\
289	0.0033944242547214\\
290	0.00339355577579138\\
291	0.00339267184477818\\
292	0.00339177218626017\\
293	0.00339085651989832\\
294	0.00338992456035151\\
295	0.00338897601719089\\
296	0.00338801059481318\\
297	0.00338702799235318\\
298	0.00338602790359528\\
299	0.00338501001688428\\
300	0.00338397401503539\\
301	0.0033829195752436\\
302	0.00338184636899246\\
303	0.00338075406196239\\
304	0.00337964231393863\\
305	0.00337851077871895\\
306	0.00337735910402129\\
307	0.00337618693139139\\
308	0.00337499389611056\\
309	0.00337377962710332\\
310	0.00337254374684477\\
311	0.00337128587126779\\
312	0.00337000560967265\\
313	0.00336870256464419\\
314	0.00336737633197372\\
315	0.00336602650056444\\
316	0.00336465265234928\\
317	0.00336325436221154\\
318	0.00336183119790872\\
319	0.00336038272000004\\
320	0.0033589084817784\\
321	0.00335740802920719\\
322	0.00335588090086295\\
323	0.00335432662788449\\
324	0.00335274473392965\\
325	0.00335113473514055\\
326	0.00334949614011866\\
327	0.00334782844991109\\
328	0.0033461311580096\\
329	0.00334440375036411\\
330	0.00334264570541242\\
331	0.00334085649412775\\
332	0.00333903558008377\\
333	0.00333718241953204\\
334	0.00333529646147408\\
335	0.00333337714768695\\
336	0.00333142391265077\\
337	0.00332943618348289\\
338	0.00332741338083925\\
339	0.00332535492385863\\
340	0.00332326024002062\\
341	0.0033211287537288\\
342	0.00331895985646738\\
343	0.00331675292394826\\
344	0.00331450728973434\\
345	0.00331222216048866\\
346	0.00330989635828335\\
347	0.00330752761579761\\
348	0.00330511112844656\\
349	0.00330263941456339\\
350	0.00330011690336177\\
351	0.00329755216031542\\
352	0.00329494449277718\\
353	0.00329229319792663\\
354	0.00328959756271786\\
355	0.00328685686384107\\
356	0.00328407036770518\\
357	0.00328123733043031\\
358	0.00327835699778407\\
359	0.00327542860503736\\
360	0.00327245137706731\\
361	0.00326942452835106\\
362	0.00326634726296647\\
363	0.00326321877459954\\
364	0.00326003824655825\\
365	0.00325680485179212\\
366	0.00325351775291687\\
367	0.00325017610224328\\
368	0.00324677904180891\\
369	0.00324332570341137\\
370	0.00323981520864163\\
371	0.00323624666891518\\
372	0.0032326191854982\\
373	0.00322893184952281\\
374	0.00322518374197804\\
375	0.00322137393365025\\
376	0.00321750148498322\\
377	0.00321356544589632\\
378	0.00320956485581192\\
379	0.00320549874406713\\
380	0.00320136612929491\\
381	0.00319716601939614\\
382	0.00319289741177549\\
383	0.00318855929398756\\
384	0.00318415064427167\\
385	0.00317967042827671\\
386	0.00317511757946406\\
387	0.00317049094444681\\
388	0.00316578925304264\\
389	0.00316101137882746\\
390	0.00315615621444637\\
391	0.00315122262383991\\
392	0.00314620943821516\\
393	0.00314111545160758\\
394	0.00313593941641775\\
395	0.00313068004017766\\
396	0.00312533598732079\\
397	0.00311990589695682\\
398	0.00311438844816519\\
399	0.00310878256139036\\
400	0.00310308797710424\\
401	0.00309730682743635\\
402	0.00309144755358902\\
403	0.00308553294398344\\
404	0.00307960777443576\\
405	0.00307370305247482\\
406	0.00306768880373024\\
407	0.0030615630565163\\
408	0.00305532375840407\\
409	0.0030489687691679\\
410	0.0030424958507213\\
411	0.00303590266293422\\
412	0.00302918678566082\\
413	0.00302234573389074\\
414	0.0030153766995041\\
415	0.00300827678087732\\
416	0.00300104297822203\\
417	0.00299367218865368\\
418	0.00298616120097232\\
419	0.00297850669012333\\
420	0.00297070521120253\\
421	0.00296275319266422\\
422	0.0029546469282637\\
423	0.00294638256839115\\
424	0.0029379561147681\\
425	0.00292936341705382\\
426	0.0029206001594658\\
427	0.00291166185151239\\
428	0.00290254381812317\\
429	0.00289324118913661\\
430	0.00288374888809989\\
431	0.00287406162032386\\
432	0.00286417386009392\\
433	0.00285407983679185\\
434	0.00284377351921915\\
435	0.00283324859613197\\
436	0.00282249844842589\\
437	0.00281151610819758\\
438	0.00280029423099783\\
439	0.00278882526319504\\
440	0.0027771019541476\\
441	0.00276511668582589\\
442	0.00275286060858101\\
443	0.00274032442670874\\
444	0.00272749837285342\\
445	0.00271437218112794\\
446	0.00270093505952688\\
447	0.0026871756644159\\
448	0.00267308208307675\\
449	0.00265864181460575\\
450	0.00264384168309426\\
451	0.00262866779095149\\
452	0.00261310541108611\\
453	0.00259713867596304\\
454	0.00258074962229649\\
455	0.00256391505904901\\
456	0.00254659623316061\\
457	0.0025286955854366\\
458	0.00250987886118515\\
459	0.00248796942335234\\
460	0.00246493431717584\\
461	0.00244142489791835\\
462	0.00241733485852565\\
463	0.00239232361604092\\
464	0.00236569411982794\\
465	0.00233856268707768\\
466	0.00231095686283289\\
467	0.00228286699853765\\
468	0.0022542819120225\\
469	0.00222518696093254\\
470	0.00219555975230399\\
471	0.00216537201516941\\
472	0.00213465618073841\\
473	0.00210341748129568\\
474	0.0020716692733659\\
475	0.0020394910917054\\
476	0.00200716271405622\\
477	0.0019754970633491\\
478	0.00194601804071996\\
479	0.00191765058876644\\
480	0.00188845229302002\\
481	0.00185837562380705\\
482	0.0018273670959291\\
483	0.00179536361596924\\
484	0.00176227937297323\\
485	0.00172794340871963\\
486	0.00169145087460265\\
487	0.00165134325332256\\
488	0.00161026073528203\\
489	0.00156816972602933\\
490	0.00152503488596659\\
491	0.00148081997832478\\
492	0.0014354873087811\\
493	0.00138899963751852\\
494	0.00134132222049671\\
495	0.00129240247908461\\
496	0.00124219552545212\\
497	0.00119065577442196\\
498	0.00113773508033689\\
499	0.00108336920262801\\
500	0.00102747393336099\\
501	0.000969990569839575\\
502	0.000910857118539138\\
503	0.000850006359514929\\
504	0.000787355549633204\\
505	0.00072274295263537\\
506	0.000655701571956031\\
507	0.00058665174801009\\
508	0.000515530777339095\\
509	0.000441372331336617\\
510	0.000364445223296124\\
511	0.000285746035416068\\
512	0.000205074854842097\\
513	0.000121623393036926\\
514	3.08717202713595e-05\\
515	0\\
516	0\\
517	0\\
518	0\\
519	0\\
520	0\\
521	0\\
522	0\\
523	0\\
524	0\\
525	0\\
526	0\\
527	0\\
528	0\\
529	0\\
530	0\\
531	0\\
532	0\\
533	0\\
534	0\\
535	0\\
536	0\\
537	0\\
538	0\\
539	0\\
540	0\\
541	0\\
542	0\\
543	0\\
544	0\\
545	0\\
546	0\\
547	0\\
548	0\\
549	0\\
550	0\\
551	0\\
552	0\\
553	0\\
554	0\\
555	0\\
556	0\\
557	0\\
558	0\\
559	0\\
560	0\\
561	0\\
562	0\\
563	0\\
564	0\\
565	0\\
566	0\\
567	0\\
568	0\\
569	0\\
570	0\\
571	0\\
572	0\\
573	0\\
574	0\\
575	0\\
576	0\\
577	0\\
578	0\\
579	0\\
580	0\\
581	0\\
582	0\\
583	0\\
584	0\\
585	0\\
586	0\\
587	0\\
588	0\\
589	0\\
590	0\\
591	0\\
592	0\\
593	0\\
594	0\\
595	0\\
596	0\\
597	0\\
598	0\\
599	0\\
600	0\\
};
\addplot [color=mycolor17,solid,forget plot]
  table[row sep=crcr]{%
1	0.00189995238561724\\
2	0.0018999437136439\\
3	0.00189993488465496\\
4	0.00189992589581167\\
5	0.00189991674422404\\
6	0.00189990742695\\
7	0.0018998979409944\\
8	0.00189988828330809\\
9	0.00189987845078691\\
10	0.00189986844027078\\
11	0.00189985824854262\\
12	0.00189984787232735\\
13	0.00189983730829086\\
14	0.00189982655303896\\
15	0.00189981560311625\\
16	0.00189980445500509\\
17	0.00189979310512444\\
18	0.00189978154982872\\
19	0.00189976978540668\\
20	0.00189975780808019\\
21	0.0018997456140031\\
22	0.00189973319925991\\
23	0.00189972055986466\\
24	0.00189970769175955\\
25	0.00189969459081374\\
26	0.00189968125282197\\
27	0.00189966767350329\\
28	0.00189965384849965\\
29	0.00189963977337455\\
30	0.00189962544361162\\
31	0.00189961085461321\\
32	0.00189959600169889\\
33	0.00189958088010403\\
34	0.00189956548497822\\
35	0.00189954981138381\\
36	0.00189953385429427\\
37	0.00189951760859267\\
38	0.00189950106906999\\
39	0.00189948423042356\\
40	0.0018994670872553\\
41	0.00189944963407005\\
42	0.00189943186527383\\
43	0.00189941377517207\\
44	0.00189939535796783\\
45	0.00189937660775992\\
46	0.00189935751854108\\
47	0.00189933808419607\\
48	0.00189931829849972\\
49	0.001899298155115\\
50	0.00189927764759099\\
51	0.00189925676936085\\
52	0.00189923551373976\\
53	0.00189921387392279\\
54	0.00189919184298278\\
55	0.00189916941386813\\
56	0.0018991465794006\\
57	0.00189912333227302\\
58	0.00189909966504699\\
59	0.00189907557015058\\
60	0.0018990510398759\\
61	0.00189902606637667\\
62	0.00189900064166575\\
63	0.00189897475761267\\
64	0.00189894840594101\\
65	0.00189892157822584\\
66	0.00189889426589102\\
67	0.00189886646020656\\
68	0.00189883815228581\\
69	0.00189880933308273\\
70	0.00189877999338901\\
71	0.00189875012383114\\
72	0.00189871971486753\\
73	0.00189868875678549\\
74	0.00189865723969814\\
75	0.00189862515354133\\
76	0.00189859248807052\\
77	0.00189855923285746\\
78	0.00189852537728702\\
79	0.00189849091055379\\
80	0.00189845582165872\\
81	0.00189842009940564\\
82	0.00189838373239779\\
83	0.00189834670903417\\
84	0.00189830901750601\\
85	0.00189827064579293\\
86	0.00189823158165929\\
87	0.00189819181265027\\
88	0.00189815132608804\\
89	0.00189811010906769\\
90	0.00189806814845327\\
91	0.00189802543087364\\
92	0.00189798194271827\\
93	0.00189793767013299\\
94	0.00189789259901563\\
95	0.00189784671501165\\
96	0.00189780000350958\\
97	0.00189775244963649\\
98	0.00189770403825333\\
99	0.00189765475395014\\
100	0.00189760458104132\\
101	0.00189755350356063\\
102	0.00189750150525623\\
103	0.00189744856958562\\
104	0.00189739467971041\\
105	0.00189733981849109\\
106	0.00189728396848168\\
107	0.00189722711192423\\
108	0.0018971692307433\\
109	0.00189711030654029\\
110	0.00189705032058771\\
111	0.00189698925382329\\
112	0.00189692708684406\\
113	0.00189686379990026\\
114	0.00189679937288919\\
115	0.00189673378534888\\
116	0.00189666701645176\\
117	0.00189659904499809\\
118	0.00189652984940941\\
119	0.00189645940772171\\
120	0.00189638769757864\\
121	0.00189631469622449\\
122	0.0018962403804971\\
123	0.00189616472682061\\
124	0.00189608771119812\\
125	0.00189600930920417\\
126	0.00189592949597712\\
127	0.00189584824621138\\
128	0.00189576553414956\\
129	0.00189568133357432\\
130	0.00189559561780031\\
131	0.00189550835966573\\
132	0.00189541953152391\\
133	0.00189532910523467\\
134	0.00189523705215551\\
135	0.00189514334313268\\
136	0.00189504794849211\\
137	0.00189495083803007\\
138	0.0018948519810038\\
139	0.0018947513461219\\
140	0.00189464890153454\\
141	0.00189454461482352\\
142	0.00189443845299216\\
143	0.00189433038245497\\
144	0.00189422036902718\\
145	0.00189410837791406\\
146	0.00189399437370003\\
147	0.00189387832033764\\
148	0.00189376018113623\\
149	0.00189363991875057\\
150	0.00189351749516907\\
151	0.00189339287170204\\
152	0.00189326600896943\\
153	0.00189313686688867\\
154	0.00189300540466206\\
155	0.001892871580764\\
156	0.00189273535292805\\
157	0.00189259667813369\\
158	0.00189245551259283\\
159	0.00189231181173616\\
160	0.00189216553019915\\
161	0.00189201662180786\\
162	0.00189186503956452\\
163	0.00189171073563275\\
164	0.00189155366132262\\
165	0.00189139376707535\\
166	0.00189123100244783\\
167	0.00189106531609673\\
168	0.00189089665576248\\
169	0.00189072496825281\\
170	0.00189055019942608\\
171	0.00189037229417427\\
172	0.00189019119640573\\
173	0.00189000684902745\\
174	0.00188981919392722\\
175	0.00188962817195532\\
176	0.00188943372290589\\
177	0.00188923578549803\\
178	0.00188903429735649\\
179	0.00188882919499201\\
180	0.00188862041378136\\
181	0.00188840788794691\\
182	0.00188819155053594\\
183	0.00188797133339948\\
184	0.0018877471671708\\
185	0.00188751898124351\\
186	0.00188728670374925\\
187	0.00188705026153492\\
188	0.00188680958013959\\
189	0.00188656458377089\\
190	0.001886315195281\\
191	0.0018860613361422\\
192	0.00188580292642196\\
193	0.00188553988475757\\
194	0.00188527212833025\\
195	0.00188499957283889\\
196	0.00188472213247319\\
197	0.00188443971988633\\
198	0.00188415224616716\\
199	0.00188385962081185\\
200	0.00188356175169499\\
201	0.00188325854504014\\
202	0.00188294990538991\\
203	0.00188263573557535\\
204	0.00188231593668489\\
205	0.00188199040803256\\
206	0.00188165904712573\\
207	0.0018813217496322\\
208	0.00188097840934658\\
209	0.00188062891815617\\
210	0.00188027316600614\\
211	0.00187991104086394\\
212	0.00187954242868321\\
213	0.00187916721336689\\
214	0.00187878527672963\\
215	0.00187839649845949\\
216	0.00187800075607892\\
217	0.00187759792490499\\
218	0.0018771878780088\\
219	0.0018767704861742\\
220	0.00187634561785559\\
221	0.00187591313913506\\
222	0.0018754729136785\\
223	0.00187502480269105\\
224	0.00187456866487158\\
225	0.00187410435636625\\
226	0.00187363173072126\\
227	0.00187315063883462\\
228	0.00187266092890697\\
229	0.00187216244639142\\
230	0.0018716550339425\\
231	0.00187113853136392\\
232	0.00187061277555552\\
233	0.00187007760045899\\
234	0.0018695328370026\\
235	0.00186897831304481\\
236	0.0018684138533168\\
237	0.00186783927936376\\
238	0.00186725440948511\\
239	0.0018666590586735\\
240	0.00186605303855252\\
241	0.00186543615731325\\
242	0.0018648082196495\\
243	0.00186416902669172\\
244	0.00186351837593962\\
245	0.00186285606119344\\
246	0.00186218187248376\\
247	0.00186149559599996\\
248	0.0018607970140172\\
249	0.00186008590482191\\
250	0.00185936204263577\\
251	0.00185862519753818\\
252	0.00185787513538704\\
253	0.00185711161773808\\
254	0.00185633440176234\\
255	0.00185554324016213\\
256	0.00185473788108515\\
257	0.00185391806803691\\
258	0.00185308353979131\\
259	0.00185223403029937\\
260	0.00185136926859611\\
261	0.00185048897870546\\
262	0.00184959287954318\\
263	0.00184868068481791\\
264	0.00184775210292986\\
265	0.00184680683686772\\
266	0.00184584458410315\\
267	0.0018448650364832\\
268	0.00184386788012037\\
269	0.00184285279528038\\
270	0.00184181945626755\\
271	0.00184076753130771\\
272	0.00183969668242903\\
273	0.00183860656534083\\
274	0.0018374968293106\\
275	0.00183636711703653\\
276	0.00183521706451204\\
277	0.00183404630089088\\
278	0.00183285444835665\\
279	0.00183164112198385\\
280	0.00183040592959619\\
281	0.00182914847162169\\
282	0.00182786834094467\\
283	0.00182656512275454\\
284	0.0018252383943913\\
285	0.00182388772518768\\
286	0.00182251267630782\\
287	0.00182111280058288\\
288	0.0018196876423409\\
289	0.00181823673723737\\
290	0.00181675961207759\\
291	0.00181525578463748\\
292	0.00181372476348\\
293	0.00181216604776756\\
294	0.00181057912707035\\
295	0.00180896348117056\\
296	0.00180731857986229\\
297	0.00180564388274715\\
298	0.00180393883902548\\
299	0.00180220288728304\\
300	0.0018004354552731\\
301	0.00179863595969402\\
302	0.00179680380596191\\
303	0.00179493838797864\\
304	0.00179303908789491\\
305	0.00179110527586846\\
306	0.0017891363098173\\
307	0.00178713153516803\\
308	0.00178509028459898\\
309	0.00178301187777807\\
310	0.00178089562109412\\
311	0.00177874080737972\\
312	0.00177654671562439\\
313	0.00177431261068608\\
314	0.00177203774302466\\
315	0.00176972134844518\\
316	0.00176736264775351\\
317	0.00176496084645114\\
318	0.00176251513442429\\
319	0.00176002468562693\\
320	0.00175748865775794\\
321	0.00175490619193196\\
322	0.00175227641234405\\
323	0.00174959842592766\\
324	0.0017468713220059\\
325	0.00174409417193571\\
326	0.00174126602874465\\
327	0.00173838592675978\\
328	0.00173545288122827\\
329	0.00173246588792904\\
330	0.00172942392277472\\
331	0.0017263259414026\\
332	0.00172317087875161\\
333	0.00171995764861706\\
334	0.00171668514315773\\
335	0.00171335223228456\\
336	0.00170995776275204\\
337	0.0017065005565978\\
338	0.00170297940867129\\
339	0.00169939308558086\\
340	0.00169574034130069\\
341	0.00169201997073723\\
342	0.0016882307717239\\
343	0.00168437138920443\\
344	0.00168044040580117\\
345	0.00167643625454292\\
346	0.00167235691241596\\
347	0.00166819882434387\\
348	0.00166395318686893\\
349	0.00165959423041928\\
350	0.00165506096071799\\
351	0.00165037745262354\\
352	0.00164561111846711\\
353	0.00164076042838519\\
354	0.00163582382245267\\
355	0.00163079971010773\\
356	0.00162568646959972\\
357	0.0016204824475013\\
358	0.00161518595826011\\
359	0.00160979528345505\\
360	0.00160430867059299\\
361	0.00159872433309202\\
362	0.00159304044970079\\
363	0.00158725516392286\\
364	0.00158136658344821\\
365	0.00157537277959397\\
366	0.00156927178675731\\
367	0.00156306160188326\\
368	0.00155674018395112\\
369	0.00155030545348344\\
370	0.00154375529208251\\
371	0.00153708754200058\\
372	0.00153030000575121\\
373	0.00152339044577001\\
374	0.00151635658412768\\
375	0.00150919610227706\\
376	0.00150190664075992\\
377	0.00149448579874129\\
378	0.00148693113347432\\
379	0.00147924016103994\\
380	0.00147141035952659\\
381	0.00146343916733997\\
382	0.00145532398482506\\
383	0.00144706217760111\\
384	0.00143865108289257\\
385	0.00143008801900107\\
386	0.00142137028793421\\
387	0.00141249511799778\\
388	0.00140345938882302\\
389	0.00139425929207139\\
390	0.00138489183209175\\
391	0.00137535422876928\\
392	0.00136564368716252\\
393	0.00135575740034698\\
394	0.00134569255308525\\
395	0.00133544632776129\\
396	0.00132501591701025\\
397	0.00131439855660885\\
398	0.00130359161994592\\
399	0.00129259289934463\\
400	0.0012814014516963\\
401	0.0012700201346766\\
402	0.00125846313425795\\
403	0.00124677777455407\\
404	0.00123510367046939\\
405	0.00122379391174075\\
406	0.00121331495775664\\
407	0.00120261856048567\\
408	0.00119170054973482\\
409	0.001180556648175\\
410	0.00116918246115891\\
411	0.00115757344745319\\
412	0.00114572489073096\\
413	0.0011336319868634\\
414	0.00112128991524153\\
415	0.00110869242013178\\
416	0.00109583303037424\\
417	0.00108270504925331\\
418	0.00106930154381129\\
419	0.00105561533360234\\
420	0.00104163897886202\\
421	0.00102736476786103\\
422	0.0010127847024471\\
423	0.000997890479493102\\
424	0.000982673469571751\\
425	0.000967124709500084\\
426	0.000951234904115045\\
427	0.000934994389298189\\
428	0.000918393112486593\\
429	0.000901420611983254\\
430	0.000884065994998693\\
431	0.000866317914362073\\
432	0.000848164543847834\\
433	0.00082959355204378\\
434	0.000810592074483101\\
435	0.000791146682864857\\
436	0.000771243346814465\\
437	0.000750867373037908\\
438	0.000730003280363605\\
439	0.000708634566704641\\
440	0.000686744057229842\\
441	0.00066431680296066\\
442	0.000641337978190259\\
443	0.000617788359155828\\
444	0.000593647776015725\\
445	0.000568895056808599\\
446	0.000543507967089672\\
447	0.000517463145007269\\
448	0.000490736039535908\\
449	0.000463300891635387\\
450	0.000435130742918302\\
451	0.000406197158852362\\
452	0.000376470248027807\\
453	0.000345918755290155\\
454	0.000314510300852118\\
455	0.00028221207140335\\
456	0.000248992681573801\\
457	0.000214839035451101\\
458	0.000179918523768187\\
459	0.000151007207087082\\
460	0.000123087413694654\\
461	9.45158457683247e-05\\
462	6.51988540410677e-05\\
463	3.47459928719172e-05\\
464	1.0433240778689e-06\\
465	0\\
466	0\\
467	0\\
468	0\\
469	0\\
470	0\\
471	0\\
472	0\\
473	0\\
474	0\\
475	0\\
476	0\\
477	0\\
478	0\\
479	0\\
480	0\\
481	0\\
482	0\\
483	0\\
484	0\\
485	0\\
486	0\\
487	0\\
488	0\\
489	0\\
490	0\\
491	0\\
492	0\\
493	0\\
494	0\\
495	0\\
496	0\\
497	0\\
498	0\\
499	0\\
500	0\\
501	0\\
502	0\\
503	0\\
504	0\\
505	0\\
506	0\\
507	0\\
508	0\\
509	0\\
510	0\\
511	0\\
512	0\\
513	0\\
514	0\\
515	0\\
516	0\\
517	0\\
518	0\\
519	0\\
520	0\\
521	0\\
522	0\\
523	0\\
524	0\\
525	0\\
526	0\\
527	0\\
528	0\\
529	0\\
530	0\\
531	0\\
532	0\\
533	0\\
534	0\\
535	0\\
536	0\\
537	0\\
538	0\\
539	0\\
540	0\\
541	0\\
542	0\\
543	0\\
544	0\\
545	0\\
546	0\\
547	0\\
548	0\\
549	0\\
550	0\\
551	0\\
552	0\\
553	0\\
554	0\\
555	0\\
556	0\\
557	0\\
558	0\\
559	0\\
560	0\\
561	0\\
562	0\\
563	0\\
564	0\\
565	0\\
566	0\\
567	0\\
568	0\\
569	0\\
570	0\\
571	0\\
572	0\\
573	0\\
574	0\\
575	0\\
576	0\\
577	0\\
578	0\\
579	0\\
580	0\\
581	0\\
582	0\\
583	0\\
584	0\\
585	0\\
586	0\\
587	0\\
588	0\\
589	0\\
590	0\\
591	0\\
592	0\\
593	0\\
594	0\\
595	0\\
596	0\\
597	0\\
598	0\\
599	0\\
600	0\\
};
\addplot [color=mycolor18,solid,forget plot]
  table[row sep=crcr]{%
1	0.000185619121464617\\
2	0.000185612706249818\\
3	0.000185606174741565\\
4	0.00018559952483321\\
5	0.000185592754379972\\
6	0.000185585861198282\\
7	0.000185578843065035\\
8	0.000185571697716924\\
9	0.000185564422849684\\
10	0.000185557016117363\\
11	0.000185549475131572\\
12	0.000185541797460714\\
13	0.00018553398062921\\
14	0.000185526022116701\\
15	0.000185517919357245\\
16	0.00018550966973849\\
17	0.000185501270600826\\
18	0.000185492719236572\\
19	0.000185484012889063\\
20	0.000185475148751796\\
21	0.000185466123967508\\
22	0.0001854569356273\\
23	0.000185447580769662\\
24	0.000185438056379553\\
25	0.000185428359387416\\
26	0.000185418486668229\\
27	0.000185408435040469\\
28	0.000185398201265113\\
29	0.000185387782044588\\
30	0.00018537717402175\\
31	0.000185366373778767\\
32	0.000185355377836053\\
33	0.00018534418265115\\
34	0.000185332784617606\\
35	0.000185321180063799\\
36	0.000185309365251772\\
37	0.000185297336376068\\
38	0.000185285089562478\\
39	0.000185272620866811\\
40	0.000185259926273652\\
41	0.000185247001695066\\
42	0.000185233842969297\\
43	0.000185220445859436\\
44	0.000185206806052073\\
45	0.000185192919155937\\
46	0.000185178780700454\\
47	0.000185164386134368\\
48	0.000185149730824251\\
49	0.00018513481005306\\
50	0.000185119619018611\\
51	0.000185104152832056\\
52	0.000185088406516322\\
53	0.000185072375004529\\
54	0.000185056053138373\\
55	0.000185039435666491\\
56	0.000185022517242776\\
57	0.000185005292424688\\
58	0.000184987755671518\\
59	0.000184969901342615\\
60	0.000184951723695606\\
61	0.00018493321688456\\
62	0.000184914374958128\\
63	0.000184895191857661\\
64	0.000184875661415266\\
65	0.000184855777351865\\
66	0.000184835533275179\\
67	0.000184814922677716\\
68	0.000184793938934687\\
69	0.000184772575301912\\
70	0.000184750824913678\\
71	0.000184728680780546\\
72	0.000184706135787143\\
73	0.000184683182689907\\
74	0.000184659814114774\\
75	0.000184636022554853\\
76	0.000184611800368026\\
77	0.000184587139774541\\
78	0.00018456203285453\\
79	0.000184536471545518\\
80	0.000184510447639833\\
81	0.000184483952782041\\
82	0.000184456978466267\\
83	0.000184429516033519\\
84	0.000184401556668935\\
85	0.000184373091398991\\
86	0.00018434411108867\\
87	0.000184314606438536\\
88	0.000184284567981833\\
89	0.000184253986081458\\
90	0.000184222850926915\\
91	0.000184191152531218\\
92	0.000184158880727728\\
93	0.00018412602516692\\
94	0.000184092575313138\\
95	0.000184058520441248\\
96	0.000184023849633228\\
97	0.000183988551774742\\
98	0.000183952615551622\\
99	0.000183916029446278\\
100	0.000183878781734071\\
101	0.000183840860479602\\
102	0.00018380225353293\\
103	0.000183762948525753\\
104	0.000183722932867498\\
105	0.000183682193741331\\
106	0.000183640718100111\\
107	0.000183598492662304\\
108	0.000183555503907729\\
109	0.000183511738073346\\
110	0.000183467181148893\\
111	0.000183421818872449\\
112	0.000183375636725948\\
113	0.000183328619930616\\
114	0.000183280753442277\\
115	0.000183232021946637\\
116	0.000183182409854425\\
117	0.000183131901296501\\
118	0.000183080480118849\\
119	0.000183028129877477\\
120	0.000182974833833243\\
121	0.000182920574946577\\
122	0.000182865335872101\\
123	0.000182809098953183\\
124	0.000182751846216364\\
125	0.000182693559365692\\
126	0.000182634219776969\\
127	0.000182573808491879\\
128	0.000182512306212025\\
129	0.000182449693292846\\
130	0.000182385949737433\\
131	0.000182321055190257\\
132	0.000182254988930714\\
133	0.000182187729866642\\
134	0.000182119256527671\\
135	0.000182049547058475\\
136	0.000181978579211895\\
137	0.000181906330341912\\
138	0.000181832777396575\\
139	0.000181757896910689\\
140	0.000181681664998503\\
141	0.000181604057346123\\
142	0.000181525049203929\\
143	0.000181444615378757\\
144	0.000181362730225995\\
145	0.000181279367641508\\
146	0.00018119450105343\\
147	0.00018110810341383\\
148	0.000181020147190189\\
149	0.000180930604356744\\
150	0.000180839446385702\\
151	0.000180746644238246\\
152	0.000180652168355429\\
153	0.000180555988648863\\
154	0.000180458074491286\\
155	0.000180358394706937\\
156	0.000180256917561722\\
157	0.000180153610753286\\
158	0.000180048441400838\\
159	0.000179941376034829\\
160	0.000179832380586422\\
161	0.000179721420376791\\
162	0.00017960846010624\\
163	0.000179493463843082\\
164	0.000179376395012368\\
165	0.000179257216384383\\
166	0.000179135890062951\\
167	0.000179012377473527\\
168	0.000178886639351057\\
169	0.000178758635727662\\
170	0.00017862832592005\\
171	0.00017849566851675\\
172	0.000178360621365075\\
173	0.00017822314155788\\
174	0.000178083185420068\\
175	0.000177940708494852\\
176	0.000177795665529789\\
177	0.000177648010462527\\
178	0.000177497696406333\\
179	0.000177344675635338\\
180	0.000177188899569507\\
181	0.000177030318759379\\
182	0.000176868882870475\\
183	0.00017670454066747\\
184	0.000176537239998063\\
185	0.000176366927776539\\
186	0.000176193549967079\\
187	0.000176017051566696\\
188	0.000175837376587934\\
189	0.000175654468041207\\
190	0.000175468267916847\\
191	0.000175278717166785\\
192	0.000175085755685953\\
193	0.000174889322293305\\
194	0.000174689354712498\\
195	0.00017448578955225\\
196	0.000174278562286291\\
197	0.000174067607233022\\
198	0.00017385285753469\\
199	0.000173634245136292\\
200	0.000173411700764045\\
201	0.000173185153903429\\
202	0.000172954532776905\\
203	0.000172719764321141\\
204	0.000172480774163874\\
205	0.000172237486600319\\
206	0.000171989824569154\\
207	0.000171737709628052\\
208	0.000171481061928769\\
209	0.000171219800191751\\
210	0.000170953841680305\\
211	0.000170683102174234\\
212	0.000170407495942966\\
213	0.000170126935718529\\
214	0.00016984133266724\\
215	0.000169550596361666\\
216	0.000169254634751657\\
217	0.000168953354134901\\
218	0.000168646659126945\\
219	0.000168334452630624\\
220	0.00016801663580494\\
221	0.000167693108033341\\
222	0.000167363766891356\\
223	0.000167028508113683\\
224	0.000166687225560611\\
225	0.000166339811183783\\
226	0.000165986154991326\\
227	0.000165626145012311\\
228	0.000165259667260515\\
229	0.000164886605697482\\
230	0.000164506842194881\\
231	0.000164120256496109\\
232	0.000163726726177173\\
233	0.000163326126606785\\
234	0.00016291833090568\\
235	0.00016250320990514\\
236	0.000162080632104701\\
237	0.000161650463628998\\
238	0.000161212568183803\\
239	0.000160766807011131\\
240	0.000160313038843497\\
241	0.000159851119857239\\
242	0.000159380903624893\\
243	0.000158902241066628\\
244	0.000158414980400708\\
245	0.0001579189670929\\
246	0.000157414043804938\\
247	0.000156900050341865\\
248	0.000156376823598324\\
249	0.0001558441975038\\
250	0.000155302002966622\\
251	0.000154750067816939\\
252	0.000154188216748465\\
253	0.000153616271258996\\
254	0.000153034049589741\\
255	0.000152441366663364\\
256	0.000151838034020765\\
257	0.000151223859756513\\
258	0.000150598648452949\\
259	0.000149962201112916\\
260	0.000149314315091053\\
261	0.000148654784023675\\
262	0.000147983397757145\\
263	0.000147299942275301\\
264	0.000146604199622824\\
265	0.000145895947831796\\
266	0.00014517496084202\\
267	0.000144441008421891\\
268	0.000143693856087161\\
269	0.000142933265017849\\
270	0.000142158991973321\\
271	0.000141370789205363\\
272	0.000140568404369321\\
273	0.000139751580433838\\
274	0.000138920055590997\\
275	0.000138073563169448\\
276	0.000137211831543223\\
277	0.00013633458400787\\
278	0.000135441538663405\\
279	0.000134532408336163\\
280	0.000133606900471857\\
281	0.000132664717025979\\
282	0.000131705554351407\\
283	0.00013072910308324\\
284	0.000129735048020551\\
285	0.000128723068005081\\
286	0.000127692835796773\\
287	0.00012664401794577\\
288	0.000125576274661166\\
289	0.000124489259675891\\
290	0.000123382620107833\\
291	0.000122255996316973\\
292	0.000121109021758344\\
293	0.000119941322830555\\
294	0.000118752518719733\\
295	0.000117542221238646\\
296	0.000116310034660682\\
297	0.000115055555548544\\
298	0.00011377837257723\\
299	0.000112478066351131\\
300	0.000111154209214821\\
301	0.000109806365057201\\
302	0.000108434089108701\\
303	0.000107036927731037\\
304	0.000105614418199145\\
305	0.000104166088474863\\
306	0.000102691456971949\\
307	0.00010119003231192\\
308	9.96613130704809e-05\\
309	9.81047875139772e-05\\
310	9.65199333248541e-05\\
311	9.49062173128027e-05\\
312	9.32630951025622e-05\\
313	9.15900107841814e-05\\
314	8.98863965375841e-05\\
315	8.81516723557525e-05\\
316	8.63852459281013e-05\\
317	8.4586512040245e-05\\
318	8.27548521819326e-05\\
319	8.08896341350972e-05\\
320	7.8990211540252e-05\\
321	7.70559234394652e-05\\
322	7.50860937938349e-05\\
323	7.30800309730493e-05\\
324	7.10370272144234e-05\\
325	6.89563580484116e-05\\
326	6.68372816871838e-05\\
327	6.46790383722825e-05\\
328	6.24808496769509e-05\\
329	6.02419177578719e-05\\
330	5.79614245501665e-05\\
331	5.56385308980601e-05\\
332	5.32723756105729e-05\\
333	5.0862074424749e-05\\
334	4.84067188397502e-05\\
335	4.59053747312909e-05\\
336	4.33570805044987e-05\\
337	4.0760844132857e-05\\
338	3.81156374116781e-05\\
339	3.54203838004354e-05\\
340	3.26739372049016e-05\\
341	2.98751043135728e-05\\
342	2.70229872211389e-05\\
343	2.41169932287785e-05\\
344	2.11558559878758e-05\\
345	1.81381856318131e-05\\
346	1.50622635655824e-05\\
347	1.19252419144098e-05\\
348	8.71976743864639e-06\\
349	5.41806378954149e-06\\
350	1.88170861355089e-06\\
351	0\\
352	0\\
353	0\\
354	0\\
355	0\\
356	0\\
357	0\\
358	0\\
359	0\\
360	0\\
361	0\\
362	0\\
363	0\\
364	0\\
365	0\\
366	0\\
367	0\\
368	0\\
369	0\\
370	0\\
371	0\\
372	0\\
373	0\\
374	0\\
375	0\\
376	0\\
377	0\\
378	0\\
379	0\\
380	0\\
381	0\\
382	0\\
383	0\\
384	0\\
385	0\\
386	0\\
387	0\\
388	0\\
389	0\\
390	0\\
391	0\\
392	0\\
393	0\\
394	0\\
395	0\\
396	0\\
397	0\\
398	0\\
399	0\\
400	0\\
401	0\\
402	0\\
403	0\\
404	0\\
405	0\\
406	0\\
407	0\\
408	0\\
409	0\\
410	0\\
411	0\\
412	0\\
413	0\\
414	0\\
415	0\\
416	0\\
417	0\\
418	0\\
419	0\\
420	0\\
421	0\\
422	0\\
423	0\\
424	0\\
425	0\\
426	0\\
427	0\\
428	0\\
429	0\\
430	0\\
431	0\\
432	0\\
433	0\\
434	0\\
435	0\\
436	0\\
437	0\\
438	0\\
439	0\\
440	0\\
441	0\\
442	0\\
443	0\\
444	0\\
445	0\\
446	0\\
447	0\\
448	0\\
449	0\\
450	0\\
451	0\\
452	0\\
453	0\\
454	0\\
455	0\\
456	0\\
457	0\\
458	0\\
459	0\\
460	0\\
461	0\\
462	0\\
463	0\\
464	0\\
465	0\\
466	0\\
467	0\\
468	0\\
469	0\\
470	0\\
471	0\\
472	0\\
473	0\\
474	0\\
475	0\\
476	0\\
477	0\\
478	0\\
479	0\\
480	0\\
481	0\\
482	0\\
483	0\\
484	0\\
485	0\\
486	0\\
487	0\\
488	0\\
489	0\\
490	0\\
491	0\\
492	0\\
493	0\\
494	0\\
495	0\\
496	0\\
497	0\\
498	0\\
499	0\\
500	0\\
501	0\\
502	0\\
503	0\\
504	0\\
505	0\\
506	0\\
507	0\\
508	0\\
509	0\\
510	0\\
511	0\\
512	0\\
513	0\\
514	0\\
515	0\\
516	0\\
517	0\\
518	0\\
519	0\\
520	0\\
521	0\\
522	0\\
523	0\\
524	0\\
525	0\\
526	0\\
527	0\\
528	0\\
529	0\\
530	0\\
531	0\\
532	0\\
533	0\\
534	0\\
535	0\\
536	0\\
537	0\\
538	0\\
539	0\\
540	0\\
541	0\\
542	0\\
543	0\\
544	0\\
545	0\\
546	0\\
547	0\\
548	0\\
549	0\\
550	0\\
551	0\\
552	0\\
553	0\\
554	0\\
555	0\\
556	0\\
557	0\\
558	0\\
559	0\\
560	0\\
561	0\\
562	0\\
563	0\\
564	0\\
565	0\\
566	0\\
567	0\\
568	0\\
569	0\\
570	0\\
571	0\\
572	0\\
573	0\\
574	0\\
575	0\\
576	0\\
577	0\\
578	0\\
579	0\\
580	0\\
581	0\\
582	0\\
583	0\\
584	0\\
585	0\\
586	0\\
587	0\\
588	0\\
589	0\\
590	0\\
591	0\\
592	0\\
593	0\\
594	0\\
595	0\\
596	0\\
597	0\\
598	0\\
599	0\\
600	0\\
};
\addplot [color=red!25!mycolor17,solid,forget plot]
  table[row sep=crcr]{%
1	0\\
2	0\\
3	0\\
4	0\\
5	0\\
6	0\\
7	0\\
8	0\\
9	0\\
10	0\\
11	0\\
12	0\\
13	0\\
14	0\\
15	0\\
16	0\\
17	0\\
18	0\\
19	0\\
20	0\\
21	0\\
22	0\\
23	0\\
24	0\\
25	0\\
26	0\\
27	0\\
28	0\\
29	0\\
30	0\\
31	0\\
32	0\\
33	0\\
34	0\\
35	0\\
36	0\\
37	0\\
38	0\\
39	0\\
40	0\\
41	0\\
42	0\\
43	0\\
44	0\\
45	0\\
46	0\\
47	0\\
48	0\\
49	0\\
50	0\\
51	0\\
52	0\\
53	0\\
54	0\\
55	0\\
56	0\\
57	0\\
58	0\\
59	0\\
60	0\\
61	0\\
62	0\\
63	0\\
64	0\\
65	0\\
66	0\\
67	0\\
68	0\\
69	0\\
70	0\\
71	0\\
72	0\\
73	0\\
74	0\\
75	0\\
76	0\\
77	0\\
78	0\\
79	0\\
80	0\\
81	0\\
82	0\\
83	0\\
84	0\\
85	0\\
86	0\\
87	0\\
88	0\\
89	0\\
90	0\\
91	0\\
92	0\\
93	0\\
94	0\\
95	0\\
96	0\\
97	0\\
98	0\\
99	0\\
100	0\\
101	0\\
102	0\\
103	0\\
104	0\\
105	0\\
106	0\\
107	0\\
108	0\\
109	0\\
110	0\\
111	0\\
112	0\\
113	0\\
114	0\\
115	0\\
116	0\\
117	0\\
118	0\\
119	0\\
120	0\\
121	0\\
122	0\\
123	0\\
124	0\\
125	0\\
126	0\\
127	0\\
128	0\\
129	0\\
130	0\\
131	0\\
132	0\\
133	0\\
134	0\\
135	0\\
136	0\\
137	0\\
138	0\\
139	0\\
140	0\\
141	0\\
142	0\\
143	0\\
144	0\\
145	0\\
146	0\\
147	0\\
148	0\\
149	0\\
150	0\\
151	0\\
152	0\\
153	0\\
154	0\\
155	0\\
156	0\\
157	0\\
158	0\\
159	0\\
160	0\\
161	0\\
162	0\\
163	0\\
164	0\\
165	0\\
166	0\\
167	0\\
168	0\\
169	0\\
170	0\\
171	0\\
172	0\\
173	0\\
174	0\\
175	0\\
176	0\\
177	0\\
178	0\\
179	0\\
180	0\\
181	0\\
182	0\\
183	0\\
184	0\\
185	0\\
186	0\\
187	0\\
188	0\\
189	0\\
190	0\\
191	0\\
192	0\\
193	0\\
194	0\\
195	0\\
196	0\\
197	0\\
198	0\\
199	0\\
200	0\\
201	0\\
202	0\\
203	0\\
204	0\\
205	0\\
206	0\\
207	0\\
208	0\\
209	0\\
210	0\\
211	0\\
212	0\\
213	0\\
214	0\\
215	0\\
216	0\\
217	0\\
218	0\\
219	0\\
220	0\\
221	0\\
222	0\\
223	0\\
224	0\\
225	0\\
226	0\\
227	0\\
228	0\\
229	0\\
230	0\\
231	0\\
232	0\\
233	0\\
234	0\\
235	0\\
236	0\\
237	0\\
238	0\\
239	0\\
240	0\\
241	0\\
242	0\\
243	0\\
244	0\\
245	0\\
246	0\\
247	0\\
248	0\\
249	0\\
250	0\\
251	0\\
252	0\\
253	0\\
254	0\\
255	0\\
256	0\\
257	0\\
258	0\\
259	0\\
260	0\\
261	0\\
262	0\\
263	0\\
264	0\\
265	0\\
266	0\\
267	0\\
268	0\\
269	0\\
270	0\\
271	0\\
272	0\\
273	0\\
274	0\\
275	0\\
276	0\\
277	0\\
278	0\\
279	0\\
280	0\\
281	0\\
282	0\\
283	0\\
284	0\\
285	0\\
286	0\\
287	0\\
288	0\\
289	0\\
290	0\\
291	0\\
292	0\\
293	0\\
294	0\\
295	0\\
296	0\\
297	0\\
298	0\\
299	0\\
300	0\\
301	0\\
302	0\\
303	0\\
304	0\\
305	0\\
306	0\\
307	0\\
308	0\\
309	0\\
310	0\\
311	0\\
312	0\\
313	0\\
314	0\\
315	0\\
316	0\\
317	0\\
318	0\\
319	0\\
320	0\\
321	0\\
322	0\\
323	0\\
324	0\\
325	0\\
326	0\\
327	0\\
328	0\\
329	0\\
330	0\\
331	0\\
332	0\\
333	0\\
334	0\\
335	0\\
336	0\\
337	0\\
338	0\\
339	0\\
340	0\\
341	0\\
342	0\\
343	0\\
344	0\\
345	0\\
346	0\\
347	0\\
348	0\\
349	0\\
350	0\\
351	0\\
352	0\\
353	0\\
354	0\\
355	0\\
356	0\\
357	0\\
358	0\\
359	0\\
360	0\\
361	0\\
362	0\\
363	0\\
364	0\\
365	0\\
366	0\\
367	0\\
368	0\\
369	0\\
370	0\\
371	0\\
372	0\\
373	0\\
374	0\\
375	0\\
376	0\\
377	0\\
378	0\\
379	0\\
380	0\\
381	0\\
382	0\\
383	0\\
384	0\\
385	0\\
386	0\\
387	0\\
388	0\\
389	0\\
390	0\\
391	0\\
392	0\\
393	0\\
394	0\\
395	0\\
396	0\\
397	0\\
398	0\\
399	0\\
400	0\\
401	0\\
402	0\\
403	0\\
404	0\\
405	0\\
406	0\\
407	0\\
408	0\\
409	0\\
410	0\\
411	0\\
412	0\\
413	0\\
414	0\\
415	0\\
416	0\\
417	0\\
418	0\\
419	0\\
420	0\\
421	0\\
422	0\\
423	0\\
424	0\\
425	0\\
426	0\\
427	0\\
428	0\\
429	0\\
430	0\\
431	0\\
432	0\\
433	0\\
434	0\\
435	0\\
436	0\\
437	0\\
438	0\\
439	0\\
440	0\\
441	0\\
442	0\\
443	0\\
444	0\\
445	0\\
446	0\\
447	0\\
448	0\\
449	0\\
450	0\\
451	0\\
452	0\\
453	0\\
454	0\\
455	0\\
456	0\\
457	0\\
458	0\\
459	0\\
460	0\\
461	0\\
462	0\\
463	0\\
464	0\\
465	0\\
466	0\\
467	0\\
468	0\\
469	0\\
470	0\\
471	0\\
472	0\\
473	0\\
474	0\\
475	0\\
476	0\\
477	0\\
478	0\\
479	0\\
480	0\\
481	0\\
482	0\\
483	0\\
484	0\\
485	0\\
486	0\\
487	0\\
488	0\\
489	0\\
490	0\\
491	0\\
492	0\\
493	0\\
494	0\\
495	0\\
496	0\\
497	0\\
498	0\\
499	0\\
500	0\\
501	0\\
502	0\\
503	0\\
504	0\\
505	0\\
506	0\\
507	0\\
508	0\\
509	0\\
510	0\\
511	0\\
512	0\\
513	0\\
514	0\\
515	0\\
516	0\\
517	0\\
518	0\\
519	0\\
520	0\\
521	0\\
522	0\\
523	0\\
524	0\\
525	0\\
526	0\\
527	0\\
528	0\\
529	0\\
530	0\\
531	0\\
532	0\\
533	0\\
534	0\\
535	0\\
536	0\\
537	0\\
538	0\\
539	0\\
540	0\\
541	0\\
542	0\\
543	0\\
544	0\\
545	0\\
546	0\\
547	0\\
548	0\\
549	0\\
550	0\\
551	0\\
552	0\\
553	0\\
554	0\\
555	0\\
556	0\\
557	0\\
558	0\\
559	0\\
560	0\\
561	0\\
562	0\\
563	0\\
564	0\\
565	0\\
566	0\\
567	0\\
568	0\\
569	0\\
570	0\\
571	0\\
572	0\\
573	0\\
574	0\\
575	0\\
576	0\\
577	0\\
578	0\\
579	0\\
580	0\\
581	0\\
582	0\\
583	0\\
584	0\\
585	0\\
586	0\\
587	0\\
588	0\\
589	0\\
590	0\\
591	0\\
592	0\\
593	0\\
594	0\\
595	0\\
596	0\\
597	0\\
598	0\\
599	0\\
600	0\\
};
\addplot [color=mycolor19,solid,forget plot]
  table[row sep=crcr]{%
1	0\\
2	0\\
3	0\\
4	0\\
5	0\\
6	0\\
7	0\\
8	0\\
9	0\\
10	0\\
11	0\\
12	0\\
13	0\\
14	0\\
15	0\\
16	0\\
17	0\\
18	0\\
19	0\\
20	0\\
21	0\\
22	0\\
23	0\\
24	0\\
25	0\\
26	0\\
27	0\\
28	0\\
29	0\\
30	0\\
31	0\\
32	0\\
33	0\\
34	0\\
35	0\\
36	0\\
37	0\\
38	0\\
39	0\\
40	0\\
41	0\\
42	0\\
43	0\\
44	0\\
45	0\\
46	0\\
47	0\\
48	0\\
49	0\\
50	0\\
51	0\\
52	0\\
53	0\\
54	0\\
55	0\\
56	0\\
57	0\\
58	0\\
59	0\\
60	0\\
61	0\\
62	0\\
63	0\\
64	0\\
65	0\\
66	0\\
67	0\\
68	0\\
69	0\\
70	0\\
71	0\\
72	0\\
73	0\\
74	0\\
75	0\\
76	0\\
77	0\\
78	0\\
79	0\\
80	0\\
81	0\\
82	0\\
83	0\\
84	0\\
85	0\\
86	0\\
87	0\\
88	0\\
89	0\\
90	0\\
91	0\\
92	0\\
93	0\\
94	0\\
95	0\\
96	0\\
97	0\\
98	0\\
99	0\\
100	0\\
101	0\\
102	0\\
103	0\\
104	0\\
105	0\\
106	0\\
107	0\\
108	0\\
109	0\\
110	0\\
111	0\\
112	0\\
113	0\\
114	0\\
115	0\\
116	0\\
117	0\\
118	0\\
119	0\\
120	0\\
121	0\\
122	0\\
123	0\\
124	0\\
125	0\\
126	0\\
127	0\\
128	0\\
129	0\\
130	0\\
131	0\\
132	0\\
133	0\\
134	0\\
135	0\\
136	0\\
137	0\\
138	0\\
139	0\\
140	0\\
141	0\\
142	0\\
143	0\\
144	0\\
145	0\\
146	0\\
147	0\\
148	0\\
149	0\\
150	0\\
151	0\\
152	0\\
153	0\\
154	0\\
155	0\\
156	0\\
157	0\\
158	0\\
159	0\\
160	0\\
161	0\\
162	0\\
163	0\\
164	0\\
165	0\\
166	0\\
167	0\\
168	0\\
169	0\\
170	0\\
171	0\\
172	0\\
173	0\\
174	0\\
175	0\\
176	0\\
177	0\\
178	0\\
179	0\\
180	0\\
181	0\\
182	0\\
183	0\\
184	0\\
185	0\\
186	0\\
187	0\\
188	0\\
189	0\\
190	0\\
191	0\\
192	0\\
193	0\\
194	0\\
195	0\\
196	0\\
197	0\\
198	0\\
199	0\\
200	0\\
201	0\\
202	0\\
203	0\\
204	0\\
205	0\\
206	0\\
207	0\\
208	0\\
209	0\\
210	0\\
211	0\\
212	0\\
213	0\\
214	0\\
215	0\\
216	0\\
217	0\\
218	0\\
219	0\\
220	0\\
221	0\\
222	0\\
223	0\\
224	0\\
225	0\\
226	0\\
227	0\\
228	0\\
229	0\\
230	0\\
231	0\\
232	0\\
233	0\\
234	0\\
235	0\\
236	0\\
237	0\\
238	0\\
239	0\\
240	0\\
241	0\\
242	0\\
243	0\\
244	0\\
245	0\\
246	0\\
247	0\\
248	0\\
249	0\\
250	0\\
251	0\\
252	0\\
253	0\\
254	0\\
255	0\\
256	0\\
257	0\\
258	0\\
259	0\\
260	0\\
261	0\\
262	0\\
263	0\\
264	0\\
265	0\\
266	0\\
267	0\\
268	0\\
269	0\\
270	0\\
271	0\\
272	0\\
273	0\\
274	0\\
275	0\\
276	0\\
277	0\\
278	0\\
279	0\\
280	0\\
281	0\\
282	0\\
283	0\\
284	0\\
285	0\\
286	0\\
287	0\\
288	0\\
289	0\\
290	0\\
291	0\\
292	0\\
293	0\\
294	0\\
295	0\\
296	0\\
297	0\\
298	0\\
299	0\\
300	0\\
301	0\\
302	0\\
303	0\\
304	0\\
305	0\\
306	0\\
307	0\\
308	0\\
309	0\\
310	0\\
311	0\\
312	0\\
313	0\\
314	0\\
315	0\\
316	0\\
317	0\\
318	0\\
319	0\\
320	0\\
321	0\\
322	0\\
323	0\\
324	0\\
325	0\\
326	0\\
327	0\\
328	0\\
329	0\\
330	0\\
331	0\\
332	0\\
333	0\\
334	0\\
335	0\\
336	0\\
337	0\\
338	0\\
339	0\\
340	0\\
341	0\\
342	0\\
343	0\\
344	0\\
345	0\\
346	0\\
347	0\\
348	0\\
349	0\\
350	0\\
351	0\\
352	0\\
353	0\\
354	0\\
355	0\\
356	0\\
357	0\\
358	0\\
359	0\\
360	0\\
361	0\\
362	0\\
363	0\\
364	0\\
365	0\\
366	0\\
367	0\\
368	0\\
369	0\\
370	0\\
371	0\\
372	0\\
373	0\\
374	0\\
375	0\\
376	0\\
377	0\\
378	0\\
379	0\\
380	0\\
381	0\\
382	0\\
383	0\\
384	0\\
385	0\\
386	0\\
387	0\\
388	0\\
389	0\\
390	0\\
391	0\\
392	0\\
393	0\\
394	0\\
395	0\\
396	0\\
397	0\\
398	0\\
399	0\\
400	0\\
401	0\\
402	0\\
403	0\\
404	0\\
405	0\\
406	0\\
407	0\\
408	0\\
409	0\\
410	0\\
411	0\\
412	0\\
413	0\\
414	0\\
415	0\\
416	0\\
417	0\\
418	0\\
419	0\\
420	0\\
421	0\\
422	0\\
423	0\\
424	0\\
425	0\\
426	0\\
427	0\\
428	0\\
429	0\\
430	0\\
431	0\\
432	0\\
433	0\\
434	0\\
435	0\\
436	0\\
437	0\\
438	0\\
439	0\\
440	0\\
441	0\\
442	0\\
443	0\\
444	0\\
445	0\\
446	0\\
447	0\\
448	0\\
449	0\\
450	0\\
451	0\\
452	0\\
453	0\\
454	0\\
455	0\\
456	0\\
457	0\\
458	0\\
459	0\\
460	0\\
461	0\\
462	0\\
463	0\\
464	0\\
465	0\\
466	0\\
467	0\\
468	0\\
469	0\\
470	0\\
471	0\\
472	0\\
473	0\\
474	0\\
475	0\\
476	0\\
477	0\\
478	0\\
479	0\\
480	0\\
481	0\\
482	0\\
483	0\\
484	0\\
485	0\\
486	0\\
487	0\\
488	0\\
489	0\\
490	0\\
491	0\\
492	0\\
493	0\\
494	0\\
495	0\\
496	0\\
497	0\\
498	0\\
499	0\\
500	0\\
501	0\\
502	0\\
503	0\\
504	0\\
505	0\\
506	0\\
507	0\\
508	0\\
509	0\\
510	0\\
511	0\\
512	0\\
513	0\\
514	0\\
515	0\\
516	0\\
517	0\\
518	0\\
519	0\\
520	0\\
521	0\\
522	0\\
523	0\\
524	0\\
525	0\\
526	0\\
527	0\\
528	0\\
529	0\\
530	0\\
531	0\\
532	0\\
533	0\\
534	0\\
535	0\\
536	0\\
537	0\\
538	0\\
539	0\\
540	0\\
541	0\\
542	0\\
543	0\\
544	0\\
545	0\\
546	0\\
547	0\\
548	0\\
549	0\\
550	0\\
551	0\\
552	0\\
553	0\\
554	0\\
555	0\\
556	0\\
557	0\\
558	0\\
559	0\\
560	0\\
561	0\\
562	0\\
563	0\\
564	0\\
565	0\\
566	0\\
567	0\\
568	0\\
569	0\\
570	0\\
571	0\\
572	0\\
573	0\\
574	0\\
575	0\\
576	0\\
577	0\\
578	0\\
579	0\\
580	0\\
581	0\\
582	0\\
583	0\\
584	0\\
585	0\\
586	0\\
587	0\\
588	0\\
589	0\\
590	0\\
591	0\\
592	0\\
593	0\\
594	0\\
595	0\\
596	0\\
597	0\\
598	0\\
599	0\\
600	0\\
};
\addplot [color=red!50!mycolor17,solid,forget plot]
  table[row sep=crcr]{%
1	0\\
2	0\\
3	0\\
4	0\\
5	0\\
6	0\\
7	0\\
8	0\\
9	0\\
10	0\\
11	0\\
12	0\\
13	0\\
14	0\\
15	0\\
16	0\\
17	0\\
18	0\\
19	0\\
20	0\\
21	0\\
22	0\\
23	0\\
24	0\\
25	0\\
26	0\\
27	0\\
28	0\\
29	0\\
30	0\\
31	0\\
32	0\\
33	0\\
34	0\\
35	0\\
36	0\\
37	0\\
38	0\\
39	0\\
40	0\\
41	0\\
42	0\\
43	0\\
44	0\\
45	0\\
46	0\\
47	0\\
48	0\\
49	0\\
50	0\\
51	0\\
52	0\\
53	0\\
54	0\\
55	0\\
56	0\\
57	0\\
58	0\\
59	0\\
60	0\\
61	0\\
62	0\\
63	0\\
64	0\\
65	0\\
66	0\\
67	0\\
68	0\\
69	0\\
70	0\\
71	0\\
72	0\\
73	0\\
74	0\\
75	0\\
76	0\\
77	0\\
78	0\\
79	0\\
80	0\\
81	0\\
82	0\\
83	0\\
84	0\\
85	0\\
86	0\\
87	0\\
88	0\\
89	0\\
90	0\\
91	0\\
92	0\\
93	0\\
94	0\\
95	0\\
96	0\\
97	0\\
98	0\\
99	0\\
100	0\\
101	0\\
102	0\\
103	0\\
104	0\\
105	0\\
106	0\\
107	0\\
108	0\\
109	0\\
110	0\\
111	0\\
112	0\\
113	0\\
114	0\\
115	0\\
116	0\\
117	0\\
118	0\\
119	0\\
120	0\\
121	0\\
122	0\\
123	0\\
124	0\\
125	0\\
126	0\\
127	0\\
128	0\\
129	0\\
130	0\\
131	0\\
132	0\\
133	0\\
134	0\\
135	0\\
136	0\\
137	0\\
138	0\\
139	0\\
140	0\\
141	0\\
142	0\\
143	0\\
144	0\\
145	0\\
146	0\\
147	0\\
148	0\\
149	0\\
150	0\\
151	0\\
152	0\\
153	0\\
154	0\\
155	0\\
156	0\\
157	0\\
158	0\\
159	0\\
160	0\\
161	0\\
162	0\\
163	0\\
164	0\\
165	0\\
166	0\\
167	0\\
168	0\\
169	0\\
170	0\\
171	0\\
172	0\\
173	0\\
174	0\\
175	0\\
176	0\\
177	0\\
178	0\\
179	0\\
180	0\\
181	0\\
182	0\\
183	0\\
184	0\\
185	0\\
186	0\\
187	0\\
188	0\\
189	0\\
190	0\\
191	0\\
192	0\\
193	0\\
194	0\\
195	0\\
196	0\\
197	0\\
198	0\\
199	0\\
200	0\\
201	0\\
202	0\\
203	0\\
204	0\\
205	0\\
206	0\\
207	0\\
208	0\\
209	0\\
210	0\\
211	0\\
212	0\\
213	0\\
214	0\\
215	0\\
216	0\\
217	0\\
218	0\\
219	0\\
220	0\\
221	0\\
222	0\\
223	0\\
224	0\\
225	0\\
226	0\\
227	0\\
228	0\\
229	0\\
230	0\\
231	0\\
232	0\\
233	0\\
234	0\\
235	0\\
236	0\\
237	0\\
238	0\\
239	0\\
240	0\\
241	0\\
242	0\\
243	0\\
244	0\\
245	0\\
246	0\\
247	0\\
248	0\\
249	0\\
250	0\\
251	0\\
252	0\\
253	0\\
254	0\\
255	0\\
256	0\\
257	0\\
258	0\\
259	0\\
260	0\\
261	0\\
262	0\\
263	0\\
264	0\\
265	0\\
266	0\\
267	0\\
268	0\\
269	0\\
270	0\\
271	0\\
272	0\\
273	0\\
274	0\\
275	0\\
276	0\\
277	0\\
278	0\\
279	0\\
280	0\\
281	0\\
282	0\\
283	0\\
284	0\\
285	0\\
286	0\\
287	0\\
288	0\\
289	0\\
290	0\\
291	0\\
292	0\\
293	0\\
294	0\\
295	0\\
296	0\\
297	0\\
298	0\\
299	0\\
300	0\\
301	0\\
302	0\\
303	0\\
304	0\\
305	0\\
306	0\\
307	0\\
308	0\\
309	0\\
310	0\\
311	0\\
312	0\\
313	0\\
314	0\\
315	0\\
316	0\\
317	0\\
318	0\\
319	0\\
320	0\\
321	0\\
322	0\\
323	0\\
324	0\\
325	0\\
326	0\\
327	0\\
328	0\\
329	0\\
330	0\\
331	0\\
332	0\\
333	0\\
334	0\\
335	0\\
336	0\\
337	0\\
338	0\\
339	0\\
340	0\\
341	0\\
342	0\\
343	0\\
344	0\\
345	0\\
346	0\\
347	0\\
348	0\\
349	0\\
350	0\\
351	0\\
352	0\\
353	0\\
354	0\\
355	0\\
356	0\\
357	0\\
358	0\\
359	0\\
360	0\\
361	0\\
362	0\\
363	0\\
364	0\\
365	0\\
366	0\\
367	0\\
368	0\\
369	0\\
370	0\\
371	0\\
372	0\\
373	0\\
374	0\\
375	0\\
376	0\\
377	0\\
378	0\\
379	0\\
380	0\\
381	0\\
382	0\\
383	0\\
384	0\\
385	0\\
386	0\\
387	0\\
388	0\\
389	0\\
390	0\\
391	0\\
392	0\\
393	0\\
394	0\\
395	0\\
396	0\\
397	0\\
398	0\\
399	0\\
400	0\\
401	0\\
402	0\\
403	0\\
404	0\\
405	0\\
406	0\\
407	0\\
408	0\\
409	0\\
410	0\\
411	0\\
412	0\\
413	0\\
414	0\\
415	0\\
416	0\\
417	0\\
418	0\\
419	0\\
420	0\\
421	0\\
422	0\\
423	0\\
424	0\\
425	0\\
426	0\\
427	0\\
428	0\\
429	0\\
430	0\\
431	0\\
432	0\\
433	0\\
434	0\\
435	0\\
436	0\\
437	0\\
438	0\\
439	0\\
440	0\\
441	0\\
442	0\\
443	0\\
444	0\\
445	0\\
446	0\\
447	0\\
448	0\\
449	0\\
450	0\\
451	0\\
452	0\\
453	0\\
454	0\\
455	0\\
456	0\\
457	0\\
458	0\\
459	0\\
460	0\\
461	0\\
462	0\\
463	0\\
464	0\\
465	0\\
466	0\\
467	0\\
468	0\\
469	0\\
470	0\\
471	0\\
472	0\\
473	0\\
474	0\\
475	0\\
476	0\\
477	0\\
478	0\\
479	0\\
480	0\\
481	0\\
482	0\\
483	0\\
484	0\\
485	0\\
486	0\\
487	0\\
488	0\\
489	0\\
490	0\\
491	0\\
492	0\\
493	0\\
494	0\\
495	0\\
496	0\\
497	0\\
498	0\\
499	0\\
500	0\\
501	0\\
502	0\\
503	0\\
504	0\\
505	0\\
506	0\\
507	0\\
508	0\\
509	0\\
510	0\\
511	0\\
512	0\\
513	0\\
514	0\\
515	0\\
516	0\\
517	0\\
518	0\\
519	0\\
520	0\\
521	0\\
522	0\\
523	0\\
524	0\\
525	0\\
526	0\\
527	0\\
528	0\\
529	0\\
530	0\\
531	0\\
532	0\\
533	0\\
534	0\\
535	0\\
536	0\\
537	0\\
538	0\\
539	0\\
540	0\\
541	0\\
542	0\\
543	0\\
544	0\\
545	0\\
546	0\\
547	0\\
548	0\\
549	0\\
550	0\\
551	0\\
552	0\\
553	0\\
554	0\\
555	0\\
556	0\\
557	0\\
558	0\\
559	0\\
560	0\\
561	0\\
562	0\\
563	0\\
564	0\\
565	0\\
566	0\\
567	0\\
568	0\\
569	0\\
570	0\\
571	0\\
572	0\\
573	0\\
574	0\\
575	0\\
576	0\\
577	0\\
578	0\\
579	0\\
580	0\\
581	0\\
582	0\\
583	0\\
584	0\\
585	0\\
586	0\\
587	0\\
588	0\\
589	0\\
590	0\\
591	0\\
592	0\\
593	0\\
594	0\\
595	0\\
596	0\\
597	0\\
598	0\\
599	0\\
600	0\\
};
\addplot [color=red!40!mycolor19,solid,forget plot]
  table[row sep=crcr]{%
1	0\\
2	0\\
3	0\\
4	0\\
5	0\\
6	0\\
7	0\\
8	0\\
9	0\\
10	0\\
11	0\\
12	0\\
13	0\\
14	0\\
15	0\\
16	0\\
17	0\\
18	0\\
19	0\\
20	0\\
21	0\\
22	0\\
23	0\\
24	0\\
25	0\\
26	0\\
27	0\\
28	0\\
29	0\\
30	0\\
31	0\\
32	0\\
33	0\\
34	0\\
35	0\\
36	0\\
37	0\\
38	0\\
39	0\\
40	0\\
41	0\\
42	0\\
43	0\\
44	0\\
45	0\\
46	0\\
47	0\\
48	0\\
49	0\\
50	0\\
51	0\\
52	0\\
53	0\\
54	0\\
55	0\\
56	0\\
57	0\\
58	0\\
59	0\\
60	0\\
61	0\\
62	0\\
63	0\\
64	0\\
65	0\\
66	0\\
67	0\\
68	0\\
69	0\\
70	0\\
71	0\\
72	0\\
73	0\\
74	0\\
75	0\\
76	0\\
77	0\\
78	0\\
79	0\\
80	0\\
81	0\\
82	0\\
83	0\\
84	0\\
85	0\\
86	0\\
87	0\\
88	0\\
89	0\\
90	0\\
91	0\\
92	0\\
93	0\\
94	0\\
95	0\\
96	0\\
97	0\\
98	0\\
99	0\\
100	0\\
101	0\\
102	0\\
103	0\\
104	0\\
105	0\\
106	0\\
107	0\\
108	0\\
109	0\\
110	0\\
111	0\\
112	0\\
113	0\\
114	0\\
115	0\\
116	0\\
117	0\\
118	0\\
119	0\\
120	0\\
121	0\\
122	0\\
123	0\\
124	0\\
125	0\\
126	0\\
127	0\\
128	0\\
129	0\\
130	0\\
131	0\\
132	0\\
133	0\\
134	0\\
135	0\\
136	0\\
137	0\\
138	0\\
139	0\\
140	0\\
141	0\\
142	0\\
143	0\\
144	0\\
145	0\\
146	0\\
147	0\\
148	0\\
149	0\\
150	0\\
151	0\\
152	0\\
153	0\\
154	0\\
155	0\\
156	0\\
157	0\\
158	0\\
159	0\\
160	0\\
161	0\\
162	0\\
163	0\\
164	0\\
165	0\\
166	0\\
167	0\\
168	0\\
169	0\\
170	0\\
171	0\\
172	0\\
173	0\\
174	0\\
175	0\\
176	0\\
177	0\\
178	0\\
179	0\\
180	0\\
181	0\\
182	0\\
183	0\\
184	0\\
185	0\\
186	0\\
187	0\\
188	0\\
189	0\\
190	0\\
191	0\\
192	0\\
193	0\\
194	0\\
195	0\\
196	0\\
197	0\\
198	0\\
199	0\\
200	0\\
201	0\\
202	0\\
203	0\\
204	0\\
205	0\\
206	0\\
207	0\\
208	0\\
209	0\\
210	0\\
211	0\\
212	0\\
213	0\\
214	0\\
215	0\\
216	0\\
217	0\\
218	0\\
219	0\\
220	0\\
221	0\\
222	0\\
223	0\\
224	0\\
225	0\\
226	0\\
227	0\\
228	0\\
229	0\\
230	0\\
231	0\\
232	0\\
233	0\\
234	0\\
235	0\\
236	0\\
237	0\\
238	0\\
239	0\\
240	0\\
241	0\\
242	0\\
243	0\\
244	0\\
245	0\\
246	0\\
247	0\\
248	0\\
249	0\\
250	0\\
251	0\\
252	0\\
253	0\\
254	0\\
255	0\\
256	0\\
257	0\\
258	0\\
259	0\\
260	0\\
261	0\\
262	0\\
263	0\\
264	0\\
265	0\\
266	0\\
267	0\\
268	0\\
269	0\\
270	0\\
271	0\\
272	0\\
273	0\\
274	0\\
275	0\\
276	0\\
277	0\\
278	0\\
279	0\\
280	0\\
281	0\\
282	0\\
283	0\\
284	0\\
285	0\\
286	0\\
287	0\\
288	0\\
289	0\\
290	0\\
291	0\\
292	0\\
293	0\\
294	0\\
295	0\\
296	0\\
297	0\\
298	0\\
299	0\\
300	0\\
301	0\\
302	0\\
303	0\\
304	0\\
305	0\\
306	0\\
307	0\\
308	0\\
309	0\\
310	0\\
311	0\\
312	0\\
313	0\\
314	0\\
315	0\\
316	0\\
317	0\\
318	0\\
319	0\\
320	0\\
321	0\\
322	0\\
323	0\\
324	0\\
325	0\\
326	0\\
327	0\\
328	0\\
329	0\\
330	0\\
331	0\\
332	0\\
333	0\\
334	0\\
335	0\\
336	0\\
337	0\\
338	0\\
339	0\\
340	0\\
341	0\\
342	0\\
343	0\\
344	0\\
345	0\\
346	0\\
347	0\\
348	0\\
349	0\\
350	0\\
351	0\\
352	0\\
353	0\\
354	0\\
355	0\\
356	0\\
357	0\\
358	0\\
359	0\\
360	0\\
361	0\\
362	0\\
363	0\\
364	0\\
365	0\\
366	0\\
367	0\\
368	0\\
369	0\\
370	0\\
371	0\\
372	0\\
373	0\\
374	0\\
375	0\\
376	0\\
377	0\\
378	0\\
379	0\\
380	0\\
381	0\\
382	0\\
383	0\\
384	0\\
385	0\\
386	0\\
387	0\\
388	0\\
389	0\\
390	0\\
391	0\\
392	0\\
393	0\\
394	0\\
395	0\\
396	0\\
397	0\\
398	0\\
399	0\\
400	0\\
401	0\\
402	0\\
403	0\\
404	0\\
405	0\\
406	0\\
407	0\\
408	0\\
409	0\\
410	0\\
411	0\\
412	0\\
413	0\\
414	0\\
415	0\\
416	0\\
417	0\\
418	0\\
419	0\\
420	0\\
421	0\\
422	0\\
423	0\\
424	0\\
425	0\\
426	0\\
427	0\\
428	0\\
429	0\\
430	0\\
431	0\\
432	0\\
433	0\\
434	0\\
435	0\\
436	0\\
437	0\\
438	0\\
439	0\\
440	0\\
441	0\\
442	0\\
443	0\\
444	0\\
445	0\\
446	0\\
447	0\\
448	0\\
449	0\\
450	0\\
451	0\\
452	0\\
453	0\\
454	0\\
455	0\\
456	0\\
457	0\\
458	0\\
459	0\\
460	0\\
461	0\\
462	0\\
463	0\\
464	0\\
465	0\\
466	0\\
467	0\\
468	0\\
469	0\\
470	0\\
471	0\\
472	0\\
473	0\\
474	0\\
475	0\\
476	0\\
477	0\\
478	0\\
479	0\\
480	0\\
481	0\\
482	0\\
483	0\\
484	0\\
485	0\\
486	0\\
487	0\\
488	0\\
489	0\\
490	0\\
491	0\\
492	0\\
493	0\\
494	0\\
495	0\\
496	0\\
497	0\\
498	0\\
499	0\\
500	0\\
501	0\\
502	0\\
503	0\\
504	0\\
505	0\\
506	0\\
507	0\\
508	0\\
509	0\\
510	0\\
511	0\\
512	0\\
513	0\\
514	0\\
515	0\\
516	0\\
517	0\\
518	0\\
519	0\\
520	0\\
521	0\\
522	0\\
523	0\\
524	0\\
525	0\\
526	0\\
527	0\\
528	0\\
529	0\\
530	0\\
531	0\\
532	0\\
533	0\\
534	0\\
535	0\\
536	0\\
537	0\\
538	0\\
539	0\\
540	0\\
541	0\\
542	0\\
543	0\\
544	0\\
545	0\\
546	0\\
547	0\\
548	0\\
549	0\\
550	0\\
551	0\\
552	0\\
553	0\\
554	0\\
555	0\\
556	0\\
557	0\\
558	0\\
559	0\\
560	0\\
561	0\\
562	0\\
563	0\\
564	0\\
565	0\\
566	0\\
567	0\\
568	0\\
569	0\\
570	0\\
571	0\\
572	0\\
573	0\\
574	0\\
575	0\\
576	0\\
577	0\\
578	0\\
579	0\\
580	0\\
581	0\\
582	0\\
583	0\\
584	0\\
585	0\\
586	0\\
587	0\\
588	0\\
589	0\\
590	0\\
591	0\\
592	0\\
593	0\\
594	0\\
595	0\\
596	0\\
597	0\\
598	0\\
599	0\\
600	0\\
};
\addplot [color=red!75!mycolor17,solid,forget plot]
  table[row sep=crcr]{%
1	0\\
2	0\\
3	0\\
4	0\\
5	0\\
6	0\\
7	0\\
8	0\\
9	0\\
10	0\\
11	0\\
12	0\\
13	0\\
14	0\\
15	0\\
16	0\\
17	0\\
18	0\\
19	0\\
20	0\\
21	0\\
22	0\\
23	0\\
24	0\\
25	0\\
26	0\\
27	0\\
28	0\\
29	0\\
30	0\\
31	0\\
32	0\\
33	0\\
34	0\\
35	0\\
36	0\\
37	0\\
38	0\\
39	0\\
40	0\\
41	0\\
42	0\\
43	0\\
44	0\\
45	0\\
46	0\\
47	0\\
48	0\\
49	0\\
50	0\\
51	0\\
52	0\\
53	0\\
54	0\\
55	0\\
56	0\\
57	0\\
58	0\\
59	0\\
60	0\\
61	0\\
62	0\\
63	0\\
64	0\\
65	0\\
66	0\\
67	0\\
68	0\\
69	0\\
70	0\\
71	0\\
72	0\\
73	0\\
74	0\\
75	0\\
76	0\\
77	0\\
78	0\\
79	0\\
80	0\\
81	0\\
82	0\\
83	0\\
84	0\\
85	0\\
86	0\\
87	0\\
88	0\\
89	0\\
90	0\\
91	0\\
92	0\\
93	0\\
94	0\\
95	0\\
96	0\\
97	0\\
98	0\\
99	0\\
100	0\\
101	0\\
102	0\\
103	0\\
104	0\\
105	0\\
106	0\\
107	0\\
108	0\\
109	0\\
110	0\\
111	0\\
112	0\\
113	0\\
114	0\\
115	0\\
116	0\\
117	0\\
118	0\\
119	0\\
120	0\\
121	0\\
122	0\\
123	0\\
124	0\\
125	0\\
126	0\\
127	0\\
128	0\\
129	0\\
130	0\\
131	0\\
132	0\\
133	0\\
134	0\\
135	0\\
136	0\\
137	0\\
138	0\\
139	0\\
140	0\\
141	0\\
142	0\\
143	0\\
144	0\\
145	0\\
146	0\\
147	0\\
148	0\\
149	0\\
150	0\\
151	0\\
152	0\\
153	0\\
154	0\\
155	0\\
156	0\\
157	0\\
158	0\\
159	0\\
160	0\\
161	0\\
162	0\\
163	0\\
164	0\\
165	0\\
166	0\\
167	0\\
168	0\\
169	0\\
170	0\\
171	0\\
172	0\\
173	0\\
174	0\\
175	0\\
176	0\\
177	0\\
178	0\\
179	0\\
180	0\\
181	0\\
182	0\\
183	0\\
184	0\\
185	0\\
186	0\\
187	0\\
188	0\\
189	0\\
190	0\\
191	0\\
192	0\\
193	0\\
194	0\\
195	0\\
196	0\\
197	0\\
198	0\\
199	0\\
200	0\\
201	0\\
202	0\\
203	0\\
204	0\\
205	0\\
206	0\\
207	0\\
208	0\\
209	0\\
210	0\\
211	0\\
212	0\\
213	0\\
214	0\\
215	0\\
216	0\\
217	0\\
218	0\\
219	0\\
220	0\\
221	0\\
222	0\\
223	0\\
224	0\\
225	0\\
226	0\\
227	0\\
228	0\\
229	0\\
230	0\\
231	0\\
232	0\\
233	0\\
234	0\\
235	0\\
236	0\\
237	0\\
238	0\\
239	0\\
240	0\\
241	0\\
242	0\\
243	0\\
244	0\\
245	0\\
246	0\\
247	0\\
248	0\\
249	0\\
250	0\\
251	0\\
252	0\\
253	0\\
254	0\\
255	0\\
256	0\\
257	0\\
258	0\\
259	0\\
260	0\\
261	0\\
262	0\\
263	0\\
264	0\\
265	0\\
266	0\\
267	0\\
268	0\\
269	0\\
270	0\\
271	0\\
272	0\\
273	0\\
274	0\\
275	0\\
276	0\\
277	0\\
278	0\\
279	0\\
280	0\\
281	0\\
282	0\\
283	0\\
284	0\\
285	0\\
286	0\\
287	0\\
288	0\\
289	0\\
290	0\\
291	0\\
292	0\\
293	0\\
294	0\\
295	0\\
296	0\\
297	0\\
298	0\\
299	0\\
300	0\\
301	0\\
302	0\\
303	0\\
304	0\\
305	0\\
306	0\\
307	0\\
308	0\\
309	0\\
310	0\\
311	0\\
312	0\\
313	0\\
314	0\\
315	0\\
316	0\\
317	0\\
318	0\\
319	0\\
320	0\\
321	0\\
322	0\\
323	0\\
324	0\\
325	0\\
326	0\\
327	0\\
328	0\\
329	0\\
330	0\\
331	0\\
332	0\\
333	0\\
334	0\\
335	0\\
336	0\\
337	0\\
338	0\\
339	0\\
340	0\\
341	0\\
342	0\\
343	0\\
344	0\\
345	0\\
346	0\\
347	0\\
348	0\\
349	0\\
350	0\\
351	0\\
352	0\\
353	0\\
354	0\\
355	0\\
356	0\\
357	0\\
358	0\\
359	0\\
360	0\\
361	0\\
362	0\\
363	0\\
364	0\\
365	0\\
366	0\\
367	0\\
368	0\\
369	0\\
370	0\\
371	0\\
372	0\\
373	0\\
374	0\\
375	0\\
376	0\\
377	0\\
378	0\\
379	0\\
380	0\\
381	0\\
382	0\\
383	0\\
384	0\\
385	0\\
386	0\\
387	0\\
388	0\\
389	0\\
390	0\\
391	0\\
392	0\\
393	0\\
394	0\\
395	0\\
396	0\\
397	0\\
398	0\\
399	0\\
400	0\\
401	0\\
402	0\\
403	0\\
404	0\\
405	0\\
406	0\\
407	0\\
408	0\\
409	0\\
410	0\\
411	0\\
412	0\\
413	0\\
414	0\\
415	0\\
416	0\\
417	0\\
418	0\\
419	0\\
420	0\\
421	0\\
422	0\\
423	0\\
424	0\\
425	0\\
426	0\\
427	0\\
428	0\\
429	0\\
430	0\\
431	0\\
432	0\\
433	0\\
434	0\\
435	0\\
436	0\\
437	0\\
438	0\\
439	0\\
440	0\\
441	0\\
442	0\\
443	0\\
444	0\\
445	0\\
446	0\\
447	0\\
448	0\\
449	0\\
450	0\\
451	0\\
452	0\\
453	0\\
454	0\\
455	0\\
456	0\\
457	0\\
458	0\\
459	0\\
460	0\\
461	0\\
462	0\\
463	0\\
464	0\\
465	0\\
466	0\\
467	0\\
468	0\\
469	0\\
470	0\\
471	0\\
472	0\\
473	0\\
474	0\\
475	0\\
476	0\\
477	0\\
478	0\\
479	0\\
480	0\\
481	0\\
482	0\\
483	0\\
484	0\\
485	0\\
486	0\\
487	0\\
488	0\\
489	0\\
490	0\\
491	0\\
492	0\\
493	0\\
494	0\\
495	0\\
496	0\\
497	0\\
498	0\\
499	0\\
500	0\\
501	0\\
502	0\\
503	0\\
504	0\\
505	0\\
506	0\\
507	0\\
508	0\\
509	0\\
510	0\\
511	0\\
512	0\\
513	0\\
514	0\\
515	0\\
516	0\\
517	0\\
518	0\\
519	0\\
520	0\\
521	0\\
522	0\\
523	0\\
524	0\\
525	0\\
526	0\\
527	0\\
528	0\\
529	0\\
530	0\\
531	0\\
532	0\\
533	0\\
534	0\\
535	0\\
536	0\\
537	0\\
538	0\\
539	0\\
540	0\\
541	0\\
542	0\\
543	0\\
544	0\\
545	0\\
546	0\\
547	0\\
548	0\\
549	0\\
550	0\\
551	0\\
552	0\\
553	0\\
554	0\\
555	0\\
556	0\\
557	0\\
558	0\\
559	0\\
560	0\\
561	0\\
562	0\\
563	0\\
564	0\\
565	0\\
566	0\\
567	0\\
568	0\\
569	0\\
570	0\\
571	0\\
572	0\\
573	0\\
574	0\\
575	0\\
576	0\\
577	0\\
578	0\\
579	0\\
580	0\\
581	0\\
582	0\\
583	0\\
584	0\\
585	0\\
586	0\\
587	0\\
588	0\\
589	0\\
590	0\\
591	0\\
592	0\\
593	0\\
594	0\\
595	0\\
596	0\\
597	0\\
598	0\\
599	0\\
600	0\\
};
\addplot [color=red!80!mycolor19,solid,forget plot]
  table[row sep=crcr]{%
1	0\\
2	0\\
3	0\\
4	0\\
5	0\\
6	0\\
7	0\\
8	0\\
9	0\\
10	0\\
11	0\\
12	0\\
13	0\\
14	0\\
15	0\\
16	0\\
17	0\\
18	0\\
19	0\\
20	0\\
21	0\\
22	0\\
23	0\\
24	0\\
25	0\\
26	0\\
27	0\\
28	0\\
29	0\\
30	0\\
31	0\\
32	0\\
33	0\\
34	0\\
35	0\\
36	0\\
37	0\\
38	0\\
39	0\\
40	0\\
41	0\\
42	0\\
43	0\\
44	0\\
45	0\\
46	0\\
47	0\\
48	0\\
49	0\\
50	0\\
51	0\\
52	0\\
53	0\\
54	0\\
55	0\\
56	0\\
57	0\\
58	0\\
59	0\\
60	0\\
61	0\\
62	0\\
63	0\\
64	0\\
65	0\\
66	0\\
67	0\\
68	0\\
69	0\\
70	0\\
71	0\\
72	0\\
73	0\\
74	0\\
75	0\\
76	0\\
77	0\\
78	0\\
79	0\\
80	0\\
81	0\\
82	0\\
83	0\\
84	0\\
85	0\\
86	0\\
87	0\\
88	0\\
89	0\\
90	0\\
91	0\\
92	0\\
93	0\\
94	0\\
95	0\\
96	0\\
97	0\\
98	0\\
99	0\\
100	0\\
101	0\\
102	0\\
103	0\\
104	0\\
105	0\\
106	0\\
107	0\\
108	0\\
109	0\\
110	0\\
111	0\\
112	0\\
113	0\\
114	0\\
115	0\\
116	0\\
117	0\\
118	0\\
119	0\\
120	0\\
121	0\\
122	0\\
123	0\\
124	0\\
125	0\\
126	0\\
127	0\\
128	0\\
129	0\\
130	0\\
131	0\\
132	0\\
133	0\\
134	0\\
135	0\\
136	0\\
137	0\\
138	0\\
139	0\\
140	0\\
141	0\\
142	0\\
143	0\\
144	0\\
145	0\\
146	0\\
147	0\\
148	0\\
149	0\\
150	0\\
151	0\\
152	0\\
153	0\\
154	0\\
155	0\\
156	0\\
157	0\\
158	0\\
159	0\\
160	0\\
161	0\\
162	0\\
163	0\\
164	0\\
165	0\\
166	0\\
167	0\\
168	0\\
169	0\\
170	0\\
171	0\\
172	0\\
173	0\\
174	0\\
175	0\\
176	0\\
177	0\\
178	0\\
179	0\\
180	0\\
181	0\\
182	0\\
183	0\\
184	0\\
185	0\\
186	0\\
187	0\\
188	0\\
189	0\\
190	0\\
191	0\\
192	0\\
193	0\\
194	0\\
195	0\\
196	0\\
197	0\\
198	0\\
199	0\\
200	0\\
201	0\\
202	0\\
203	0\\
204	0\\
205	0\\
206	0\\
207	0\\
208	0\\
209	0\\
210	0\\
211	0\\
212	0\\
213	0\\
214	0\\
215	0\\
216	0\\
217	0\\
218	0\\
219	0\\
220	0\\
221	0\\
222	0\\
223	0\\
224	0\\
225	0\\
226	0\\
227	0\\
228	0\\
229	0\\
230	0\\
231	0\\
232	0\\
233	0\\
234	0\\
235	0\\
236	0\\
237	0\\
238	0\\
239	0\\
240	0\\
241	0\\
242	0\\
243	0\\
244	0\\
245	0\\
246	0\\
247	0\\
248	0\\
249	0\\
250	0\\
251	0\\
252	0\\
253	0\\
254	0\\
255	0\\
256	0\\
257	0\\
258	0\\
259	0\\
260	0\\
261	0\\
262	0\\
263	0\\
264	0\\
265	0\\
266	0\\
267	0\\
268	0\\
269	0\\
270	0\\
271	0\\
272	0\\
273	0\\
274	0\\
275	0\\
276	0\\
277	0\\
278	0\\
279	0\\
280	0\\
281	0\\
282	0\\
283	0\\
284	0\\
285	0\\
286	0\\
287	0\\
288	0\\
289	0\\
290	0\\
291	0\\
292	0\\
293	0\\
294	0\\
295	0\\
296	0\\
297	0\\
298	0\\
299	0\\
300	0\\
301	0\\
302	0\\
303	0\\
304	0\\
305	0\\
306	0\\
307	0\\
308	0\\
309	0\\
310	0\\
311	0\\
312	0\\
313	0\\
314	0\\
315	0\\
316	0\\
317	0\\
318	0\\
319	0\\
320	0\\
321	0\\
322	0\\
323	0\\
324	0\\
325	0\\
326	0\\
327	0\\
328	0\\
329	0\\
330	0\\
331	0\\
332	0\\
333	0\\
334	0\\
335	0\\
336	0\\
337	0\\
338	0\\
339	0\\
340	0\\
341	0\\
342	0\\
343	0\\
344	0\\
345	0\\
346	0\\
347	0\\
348	0\\
349	0\\
350	0\\
351	0\\
352	0\\
353	0\\
354	0\\
355	0\\
356	0\\
357	0\\
358	0\\
359	0\\
360	0\\
361	0\\
362	0\\
363	0\\
364	0\\
365	0\\
366	0\\
367	0\\
368	0\\
369	0\\
370	0\\
371	0\\
372	0\\
373	0\\
374	0\\
375	0\\
376	0\\
377	0\\
378	0\\
379	0\\
380	0\\
381	0\\
382	0\\
383	0\\
384	0\\
385	0\\
386	0\\
387	0\\
388	0\\
389	0\\
390	0\\
391	0\\
392	0\\
393	0\\
394	0\\
395	0\\
396	0\\
397	0\\
398	0\\
399	0\\
400	0\\
401	0\\
402	0\\
403	0\\
404	0\\
405	0\\
406	0\\
407	0\\
408	0\\
409	0\\
410	0\\
411	0\\
412	0\\
413	0\\
414	0\\
415	0\\
416	0\\
417	0\\
418	0\\
419	0\\
420	0\\
421	0\\
422	0\\
423	0\\
424	0\\
425	0\\
426	0\\
427	0\\
428	0\\
429	0\\
430	0\\
431	0\\
432	0\\
433	0\\
434	0\\
435	0\\
436	0\\
437	0\\
438	0\\
439	0\\
440	0\\
441	0\\
442	0\\
443	0\\
444	0\\
445	0\\
446	0\\
447	0\\
448	0\\
449	0\\
450	0\\
451	0\\
452	0\\
453	0\\
454	0\\
455	0\\
456	0\\
457	0\\
458	0\\
459	0\\
460	0\\
461	0\\
462	0\\
463	0\\
464	0\\
465	0\\
466	0\\
467	0\\
468	0\\
469	0\\
470	0\\
471	0\\
472	0\\
473	0\\
474	0\\
475	0\\
476	0\\
477	0\\
478	0\\
479	0\\
480	0\\
481	0\\
482	0\\
483	0\\
484	0\\
485	0\\
486	0\\
487	0\\
488	0\\
489	0\\
490	0\\
491	0\\
492	0\\
493	0\\
494	0\\
495	0\\
496	0\\
497	0\\
498	0\\
499	0\\
500	0\\
501	0\\
502	0\\
503	0\\
504	0\\
505	0\\
506	0\\
507	0\\
508	0\\
509	0\\
510	0\\
511	0\\
512	0\\
513	0\\
514	0\\
515	0\\
516	0\\
517	0\\
518	0\\
519	0\\
520	0\\
521	0\\
522	0\\
523	0\\
524	0\\
525	0\\
526	0\\
527	0\\
528	0\\
529	0\\
530	0\\
531	0\\
532	0\\
533	0\\
534	0\\
535	0\\
536	0\\
537	0\\
538	0\\
539	0\\
540	0\\
541	0\\
542	0\\
543	0\\
544	0\\
545	0\\
546	0\\
547	0\\
548	0\\
549	0\\
550	0\\
551	0\\
552	0\\
553	0\\
554	0\\
555	0\\
556	0\\
557	0\\
558	0\\
559	0\\
560	0\\
561	0\\
562	0\\
563	0\\
564	0\\
565	0\\
566	0\\
567	0\\
568	0\\
569	0\\
570	0\\
571	0\\
572	0\\
573	0\\
574	0\\
575	0\\
576	0\\
577	0\\
578	0\\
579	0\\
580	0\\
581	0\\
582	0\\
583	0\\
584	0\\
585	0\\
586	0\\
587	0\\
588	0\\
589	0\\
590	0\\
591	0\\
592	0\\
593	0\\
594	0\\
595	0\\
596	0\\
597	0\\
598	0\\
599	0\\
600	0\\
};
\addplot [color=red,solid,forget plot]
  table[row sep=crcr]{%
1	0\\
2	0\\
3	0\\
4	0\\
5	0\\
6	0\\
7	0\\
8	0\\
9	0\\
10	0\\
11	0\\
12	0\\
13	0\\
14	0\\
15	0\\
16	0\\
17	0\\
18	0\\
19	0\\
20	0\\
21	0\\
22	0\\
23	0\\
24	0\\
25	0\\
26	0\\
27	0\\
28	0\\
29	0\\
30	0\\
31	0\\
32	0\\
33	0\\
34	0\\
35	0\\
36	0\\
37	0\\
38	0\\
39	0\\
40	0\\
41	0\\
42	0\\
43	0\\
44	0\\
45	0\\
46	0\\
47	0\\
48	0\\
49	0\\
50	0\\
51	0\\
52	0\\
53	0\\
54	0\\
55	0\\
56	0\\
57	0\\
58	0\\
59	0\\
60	0\\
61	0\\
62	0\\
63	0\\
64	0\\
65	0\\
66	0\\
67	0\\
68	0\\
69	0\\
70	0\\
71	0\\
72	0\\
73	0\\
74	0\\
75	0\\
76	0\\
77	0\\
78	0\\
79	0\\
80	0\\
81	0\\
82	0\\
83	0\\
84	0\\
85	0\\
86	0\\
87	0\\
88	0\\
89	0\\
90	0\\
91	0\\
92	0\\
93	0\\
94	0\\
95	0\\
96	0\\
97	0\\
98	0\\
99	0\\
100	0\\
101	0\\
102	0\\
103	0\\
104	0\\
105	0\\
106	0\\
107	0\\
108	0\\
109	0\\
110	0\\
111	0\\
112	0\\
113	0\\
114	0\\
115	0\\
116	0\\
117	0\\
118	0\\
119	0\\
120	0\\
121	0\\
122	0\\
123	0\\
124	0\\
125	0\\
126	0\\
127	0\\
128	0\\
129	0\\
130	0\\
131	0\\
132	0\\
133	0\\
134	0\\
135	0\\
136	0\\
137	0\\
138	0\\
139	0\\
140	0\\
141	0\\
142	0\\
143	0\\
144	0\\
145	0\\
146	0\\
147	0\\
148	0\\
149	0\\
150	0\\
151	0\\
152	0\\
153	0\\
154	0\\
155	0\\
156	0\\
157	0\\
158	0\\
159	0\\
160	0\\
161	0\\
162	0\\
163	0\\
164	0\\
165	0\\
166	0\\
167	0\\
168	0\\
169	0\\
170	0\\
171	0\\
172	0\\
173	0\\
174	0\\
175	0\\
176	0\\
177	0\\
178	0\\
179	0\\
180	0\\
181	0\\
182	0\\
183	0\\
184	0\\
185	0\\
186	0\\
187	0\\
188	0\\
189	0\\
190	0\\
191	0\\
192	0\\
193	0\\
194	0\\
195	0\\
196	0\\
197	0\\
198	0\\
199	0\\
200	0\\
201	0\\
202	0\\
203	0\\
204	0\\
205	0\\
206	0\\
207	0\\
208	0\\
209	0\\
210	0\\
211	0\\
212	0\\
213	0\\
214	0\\
215	0\\
216	0\\
217	0\\
218	0\\
219	0\\
220	0\\
221	0\\
222	0\\
223	0\\
224	0\\
225	0\\
226	0\\
227	0\\
228	0\\
229	0\\
230	0\\
231	0\\
232	0\\
233	0\\
234	0\\
235	0\\
236	0\\
237	0\\
238	0\\
239	0\\
240	0\\
241	0\\
242	0\\
243	0\\
244	0\\
245	0\\
246	0\\
247	0\\
248	0\\
249	0\\
250	0\\
251	0\\
252	0\\
253	0\\
254	0\\
255	0\\
256	0\\
257	0\\
258	0\\
259	0\\
260	0\\
261	0\\
262	0\\
263	0\\
264	0\\
265	0\\
266	0\\
267	0\\
268	0\\
269	0\\
270	0\\
271	0\\
272	0\\
273	0\\
274	0\\
275	0\\
276	0\\
277	0\\
278	0\\
279	0\\
280	0\\
281	0\\
282	0\\
283	0\\
284	0\\
285	0\\
286	0\\
287	0\\
288	0\\
289	0\\
290	0\\
291	0\\
292	0\\
293	0\\
294	0\\
295	0\\
296	0\\
297	0\\
298	0\\
299	0\\
300	0\\
301	0\\
302	0\\
303	0\\
304	0\\
305	0\\
306	0\\
307	0\\
308	0\\
309	0\\
310	0\\
311	0\\
312	0\\
313	0\\
314	0\\
315	0\\
316	0\\
317	0\\
318	0\\
319	0\\
320	0\\
321	0\\
322	0\\
323	0\\
324	0\\
325	0\\
326	0\\
327	0\\
328	0\\
329	0\\
330	0\\
331	0\\
332	0\\
333	0\\
334	0\\
335	0\\
336	0\\
337	0\\
338	0\\
339	0\\
340	0\\
341	0\\
342	0\\
343	0\\
344	0\\
345	0\\
346	0\\
347	0\\
348	0\\
349	0\\
350	0\\
351	0\\
352	0\\
353	0\\
354	0\\
355	0\\
356	0\\
357	0\\
358	0\\
359	0\\
360	0\\
361	0\\
362	0\\
363	0\\
364	0\\
365	0\\
366	0\\
367	0\\
368	0\\
369	0\\
370	0\\
371	0\\
372	0\\
373	0\\
374	0\\
375	0\\
376	0\\
377	0\\
378	0\\
379	0\\
380	0\\
381	0\\
382	0\\
383	0\\
384	0\\
385	0\\
386	0\\
387	0\\
388	0\\
389	0\\
390	0\\
391	0\\
392	0\\
393	0\\
394	0\\
395	0\\
396	0\\
397	0\\
398	0\\
399	0\\
400	0\\
401	0\\
402	0\\
403	0\\
404	0\\
405	0\\
406	0\\
407	0\\
408	0\\
409	0\\
410	0\\
411	0\\
412	0\\
413	0\\
414	0\\
415	0\\
416	0\\
417	0\\
418	0\\
419	0\\
420	0\\
421	0\\
422	0\\
423	0\\
424	0\\
425	0\\
426	0\\
427	0\\
428	0\\
429	0\\
430	0\\
431	0\\
432	0\\
433	0\\
434	0\\
435	0\\
436	0\\
437	0\\
438	0\\
439	0\\
440	0\\
441	0\\
442	0\\
443	0\\
444	0\\
445	0\\
446	0\\
447	0\\
448	0\\
449	0\\
450	0\\
451	0\\
452	0\\
453	0\\
454	0\\
455	0\\
456	0\\
457	0\\
458	0\\
459	0\\
460	0\\
461	0\\
462	0\\
463	0\\
464	0\\
465	0\\
466	0\\
467	0\\
468	0\\
469	0\\
470	0\\
471	0\\
472	0\\
473	0\\
474	0\\
475	0\\
476	0\\
477	0\\
478	0\\
479	0\\
480	0\\
481	0\\
482	0\\
483	0\\
484	0\\
485	0\\
486	0\\
487	0\\
488	0\\
489	0\\
490	0\\
491	0\\
492	0\\
493	0\\
494	0\\
495	0\\
496	0\\
497	0\\
498	0\\
499	0\\
500	0\\
501	0\\
502	0\\
503	0\\
504	0\\
505	0\\
506	0\\
507	0\\
508	0\\
509	0\\
510	0\\
511	0\\
512	0\\
513	0\\
514	0\\
515	0\\
516	0\\
517	0\\
518	0\\
519	0\\
520	0\\
521	0\\
522	0\\
523	0\\
524	0\\
525	0\\
526	0\\
527	0\\
528	0\\
529	0\\
530	0\\
531	0\\
532	0\\
533	0\\
534	0\\
535	0\\
536	0\\
537	0\\
538	0\\
539	0\\
540	0\\
541	0\\
542	0\\
543	0\\
544	0\\
545	0\\
546	0\\
547	0\\
548	0\\
549	0\\
550	0\\
551	0\\
552	0\\
553	0\\
554	0\\
555	0\\
556	0\\
557	0\\
558	0\\
559	0\\
560	0\\
561	0\\
562	0\\
563	0\\
564	0\\
565	0\\
566	0\\
567	0\\
568	0\\
569	0\\
570	0\\
571	0\\
572	0\\
573	0\\
574	0\\
575	0\\
576	0\\
577	0\\
578	0\\
579	0\\
580	0\\
581	0\\
582	0\\
583	0\\
584	0\\
585	0\\
586	0\\
587	0\\
588	0\\
589	0\\
590	0\\
591	0\\
592	0\\
593	0\\
594	0\\
595	0\\
596	0\\
597	0\\
598	0\\
599	0\\
600	0\\
};
\addplot [color=mycolor20,solid,forget plot]
  table[row sep=crcr]{%
1	0\\
2	0\\
3	0\\
4	0\\
5	0\\
6	0\\
7	0\\
8	0\\
9	0\\
10	0\\
11	0\\
12	0\\
13	0\\
14	0\\
15	0\\
16	0\\
17	0\\
18	0\\
19	0\\
20	0\\
21	0\\
22	0\\
23	0\\
24	0\\
25	0\\
26	0\\
27	0\\
28	0\\
29	0\\
30	0\\
31	0\\
32	0\\
33	0\\
34	0\\
35	0\\
36	0\\
37	0\\
38	0\\
39	0\\
40	0\\
41	0\\
42	0\\
43	0\\
44	0\\
45	0\\
46	0\\
47	0\\
48	0\\
49	0\\
50	0\\
51	0\\
52	0\\
53	0\\
54	0\\
55	0\\
56	0\\
57	0\\
58	0\\
59	0\\
60	0\\
61	0\\
62	0\\
63	0\\
64	0\\
65	0\\
66	0\\
67	0\\
68	0\\
69	0\\
70	0\\
71	0\\
72	0\\
73	0\\
74	0\\
75	0\\
76	0\\
77	0\\
78	0\\
79	0\\
80	0\\
81	0\\
82	0\\
83	0\\
84	0\\
85	0\\
86	0\\
87	0\\
88	0\\
89	0\\
90	0\\
91	0\\
92	0\\
93	0\\
94	0\\
95	0\\
96	0\\
97	0\\
98	0\\
99	0\\
100	0\\
101	0\\
102	0\\
103	0\\
104	0\\
105	0\\
106	0\\
107	0\\
108	0\\
109	0\\
110	0\\
111	0\\
112	0\\
113	0\\
114	0\\
115	0\\
116	0\\
117	0\\
118	0\\
119	0\\
120	0\\
121	0\\
122	0\\
123	0\\
124	0\\
125	0\\
126	0\\
127	0\\
128	0\\
129	0\\
130	0\\
131	0\\
132	0\\
133	0\\
134	0\\
135	0\\
136	0\\
137	0\\
138	0\\
139	0\\
140	0\\
141	0\\
142	0\\
143	0\\
144	0\\
145	0\\
146	0\\
147	0\\
148	0\\
149	0\\
150	0\\
151	0\\
152	0\\
153	0\\
154	0\\
155	0\\
156	0\\
157	0\\
158	0\\
159	0\\
160	0\\
161	0\\
162	0\\
163	0\\
164	0\\
165	0\\
166	0\\
167	0\\
168	0\\
169	0\\
170	0\\
171	0\\
172	0\\
173	0\\
174	0\\
175	0\\
176	0\\
177	0\\
178	0\\
179	0\\
180	0\\
181	0\\
182	0\\
183	0\\
184	0\\
185	0\\
186	0\\
187	0\\
188	0\\
189	0\\
190	0\\
191	0\\
192	0\\
193	0\\
194	0\\
195	0\\
196	0\\
197	0\\
198	0\\
199	0\\
200	0\\
201	0\\
202	0\\
203	0\\
204	0\\
205	0\\
206	0\\
207	0\\
208	0\\
209	0\\
210	0\\
211	0\\
212	0\\
213	0\\
214	0\\
215	0\\
216	0\\
217	0\\
218	0\\
219	0\\
220	0\\
221	0\\
222	0\\
223	0\\
224	0\\
225	0\\
226	0\\
227	0\\
228	0\\
229	0\\
230	0\\
231	0\\
232	0\\
233	0\\
234	0\\
235	0\\
236	0\\
237	0\\
238	0\\
239	0\\
240	0\\
241	0\\
242	0\\
243	0\\
244	0\\
245	0\\
246	0\\
247	0\\
248	0\\
249	0\\
250	0\\
251	0\\
252	0\\
253	0\\
254	0\\
255	0\\
256	0\\
257	0\\
258	0\\
259	0\\
260	0\\
261	0\\
262	0\\
263	0\\
264	0\\
265	0\\
266	0\\
267	0\\
268	0\\
269	0\\
270	0\\
271	0\\
272	0\\
273	0\\
274	0\\
275	0\\
276	0\\
277	0\\
278	0\\
279	0\\
280	0\\
281	0\\
282	0\\
283	0\\
284	0\\
285	0\\
286	0\\
287	0\\
288	0\\
289	0\\
290	0\\
291	0\\
292	0\\
293	0\\
294	0\\
295	0\\
296	0\\
297	0\\
298	0\\
299	0\\
300	0\\
301	0\\
302	0\\
303	0\\
304	0\\
305	0\\
306	0\\
307	0\\
308	0\\
309	0\\
310	0\\
311	0\\
312	0\\
313	0\\
314	0\\
315	0\\
316	0\\
317	0\\
318	0\\
319	0\\
320	0\\
321	0\\
322	0\\
323	0\\
324	0\\
325	0\\
326	0\\
327	0\\
328	0\\
329	0\\
330	0\\
331	0\\
332	0\\
333	0\\
334	0\\
335	0\\
336	0\\
337	0\\
338	0\\
339	0\\
340	0\\
341	0\\
342	0\\
343	0\\
344	0\\
345	0\\
346	0\\
347	0\\
348	0\\
349	0\\
350	0\\
351	0\\
352	0\\
353	0\\
354	0\\
355	0\\
356	0\\
357	0\\
358	0\\
359	0\\
360	0\\
361	0\\
362	0\\
363	0\\
364	0\\
365	0\\
366	0\\
367	0\\
368	0\\
369	0\\
370	0\\
371	0\\
372	0\\
373	0\\
374	0\\
375	0\\
376	0\\
377	0\\
378	0\\
379	0\\
380	0\\
381	0\\
382	0\\
383	0\\
384	0\\
385	0\\
386	0\\
387	0\\
388	0\\
389	0\\
390	0\\
391	0\\
392	0\\
393	0\\
394	0\\
395	0\\
396	0\\
397	0\\
398	0\\
399	0\\
400	0\\
401	0\\
402	0\\
403	0\\
404	0\\
405	0\\
406	0\\
407	0\\
408	0\\
409	0\\
410	0\\
411	0\\
412	0\\
413	0\\
414	0\\
415	0\\
416	0\\
417	0\\
418	0\\
419	0\\
420	0\\
421	0\\
422	0\\
423	0\\
424	0\\
425	0\\
426	0\\
427	0\\
428	0\\
429	0\\
430	0\\
431	0\\
432	0\\
433	0\\
434	0\\
435	0\\
436	0\\
437	0\\
438	0\\
439	0\\
440	0\\
441	0\\
442	0\\
443	0\\
444	0\\
445	0\\
446	0\\
447	0\\
448	0\\
449	0\\
450	0\\
451	0\\
452	0\\
453	0\\
454	0\\
455	0\\
456	0\\
457	0\\
458	0\\
459	0\\
460	0\\
461	0\\
462	0\\
463	0\\
464	0\\
465	0\\
466	0\\
467	0\\
468	0\\
469	0\\
470	0\\
471	0\\
472	0\\
473	0\\
474	0\\
475	0\\
476	0\\
477	0\\
478	0\\
479	0\\
480	0\\
481	0\\
482	0\\
483	0\\
484	0\\
485	0\\
486	0\\
487	0\\
488	0\\
489	0\\
490	0\\
491	0\\
492	0\\
493	0\\
494	0\\
495	0\\
496	0\\
497	0\\
498	0\\
499	0\\
500	0\\
501	0\\
502	0\\
503	0\\
504	0\\
505	0\\
506	0\\
507	0\\
508	0\\
509	0\\
510	0\\
511	0\\
512	0\\
513	0\\
514	0\\
515	0\\
516	0\\
517	0\\
518	0\\
519	0\\
520	0\\
521	0\\
522	0\\
523	0\\
524	0\\
525	0\\
526	0\\
527	0\\
528	0\\
529	0\\
530	0\\
531	0\\
532	0\\
533	0\\
534	0\\
535	0\\
536	0\\
537	0\\
538	0\\
539	0\\
540	0\\
541	0\\
542	0\\
543	0\\
544	0\\
545	0\\
546	0\\
547	0\\
548	0\\
549	0\\
550	0\\
551	0\\
552	0\\
553	0\\
554	0\\
555	0\\
556	0\\
557	0\\
558	0\\
559	0\\
560	0\\
561	0\\
562	0\\
563	0\\
564	0\\
565	0\\
566	0\\
567	0\\
568	0\\
569	0\\
570	0\\
571	0\\
572	0\\
573	0\\
574	0\\
575	0\\
576	0\\
577	0\\
578	0\\
579	0\\
580	0\\
581	0\\
582	0\\
583	0\\
584	0\\
585	0\\
586	0\\
587	0\\
588	0\\
589	0\\
590	0\\
591	0\\
592	0\\
593	0\\
594	0\\
595	0\\
596	0\\
597	0\\
598	0\\
599	0\\
600	0\\
};
\addplot [color=mycolor21,solid,forget plot]
  table[row sep=crcr]{%
1	0\\
2	0\\
3	0\\
4	0\\
5	0\\
6	0\\
7	0\\
8	0\\
9	0\\
10	0\\
11	0\\
12	0\\
13	0\\
14	0\\
15	0\\
16	0\\
17	0\\
18	0\\
19	0\\
20	0\\
21	0\\
22	0\\
23	0\\
24	0\\
25	0\\
26	0\\
27	0\\
28	0\\
29	0\\
30	0\\
31	0\\
32	0\\
33	0\\
34	0\\
35	0\\
36	0\\
37	0\\
38	0\\
39	0\\
40	0\\
41	0\\
42	0\\
43	0\\
44	0\\
45	0\\
46	0\\
47	0\\
48	0\\
49	0\\
50	0\\
51	0\\
52	0\\
53	0\\
54	0\\
55	0\\
56	0\\
57	0\\
58	0\\
59	0\\
60	0\\
61	0\\
62	0\\
63	0\\
64	0\\
65	0\\
66	0\\
67	0\\
68	0\\
69	0\\
70	0\\
71	0\\
72	0\\
73	0\\
74	0\\
75	0\\
76	0\\
77	0\\
78	0\\
79	0\\
80	0\\
81	0\\
82	0\\
83	0\\
84	0\\
85	0\\
86	0\\
87	0\\
88	0\\
89	0\\
90	0\\
91	0\\
92	0\\
93	0\\
94	0\\
95	0\\
96	0\\
97	0\\
98	0\\
99	0\\
100	0\\
101	0\\
102	0\\
103	0\\
104	0\\
105	0\\
106	0\\
107	0\\
108	0\\
109	0\\
110	0\\
111	0\\
112	0\\
113	0\\
114	0\\
115	0\\
116	0\\
117	0\\
118	0\\
119	0\\
120	0\\
121	0\\
122	0\\
123	0\\
124	0\\
125	0\\
126	0\\
127	0\\
128	0\\
129	0\\
130	0\\
131	0\\
132	0\\
133	0\\
134	0\\
135	0\\
136	0\\
137	0\\
138	0\\
139	0\\
140	0\\
141	0\\
142	0\\
143	0\\
144	0\\
145	0\\
146	0\\
147	0\\
148	0\\
149	0\\
150	0\\
151	0\\
152	0\\
153	0\\
154	0\\
155	0\\
156	0\\
157	0\\
158	0\\
159	0\\
160	0\\
161	0\\
162	0\\
163	0\\
164	0\\
165	0\\
166	0\\
167	0\\
168	0\\
169	0\\
170	0\\
171	0\\
172	0\\
173	0\\
174	0\\
175	0\\
176	0\\
177	0\\
178	0\\
179	0\\
180	0\\
181	0\\
182	0\\
183	0\\
184	0\\
185	0\\
186	0\\
187	0\\
188	0\\
189	0\\
190	0\\
191	0\\
192	0\\
193	0\\
194	0\\
195	0\\
196	0\\
197	0\\
198	0\\
199	0\\
200	0\\
201	0\\
202	0\\
203	0\\
204	0\\
205	0\\
206	0\\
207	0\\
208	0\\
209	0\\
210	0\\
211	0\\
212	0\\
213	0\\
214	0\\
215	0\\
216	0\\
217	0\\
218	0\\
219	0\\
220	0\\
221	0\\
222	0\\
223	0\\
224	0\\
225	0\\
226	0\\
227	0\\
228	0\\
229	0\\
230	0\\
231	0\\
232	0\\
233	0\\
234	0\\
235	0\\
236	0\\
237	0\\
238	0\\
239	0\\
240	0\\
241	0\\
242	0\\
243	0\\
244	0\\
245	0\\
246	0\\
247	0\\
248	0\\
249	0\\
250	0\\
251	0\\
252	0\\
253	0\\
254	0\\
255	0\\
256	0\\
257	0\\
258	0\\
259	0\\
260	0\\
261	0\\
262	0\\
263	0\\
264	0\\
265	0\\
266	0\\
267	0\\
268	0\\
269	0\\
270	0\\
271	0\\
272	0\\
273	0\\
274	0\\
275	0\\
276	0\\
277	0\\
278	0\\
279	0\\
280	0\\
281	0\\
282	0\\
283	0\\
284	0\\
285	0\\
286	0\\
287	0\\
288	0\\
289	0\\
290	0\\
291	0\\
292	0\\
293	0\\
294	0\\
295	0\\
296	0\\
297	0\\
298	0\\
299	0\\
300	0\\
301	0\\
302	0\\
303	0\\
304	0\\
305	0\\
306	0\\
307	0\\
308	0\\
309	0\\
310	0\\
311	0\\
312	0\\
313	0\\
314	0\\
315	0\\
316	0\\
317	0\\
318	0\\
319	0\\
320	0\\
321	0\\
322	0\\
323	0\\
324	0\\
325	0\\
326	0\\
327	0\\
328	0\\
329	0\\
330	0\\
331	0\\
332	0\\
333	0\\
334	0\\
335	0\\
336	0\\
337	0\\
338	0\\
339	0\\
340	0\\
341	0\\
342	0\\
343	0\\
344	0\\
345	0\\
346	0\\
347	0\\
348	0\\
349	0\\
350	0\\
351	0\\
352	0\\
353	0\\
354	0\\
355	0\\
356	0\\
357	0\\
358	0\\
359	0\\
360	0\\
361	0\\
362	0\\
363	0\\
364	0\\
365	0\\
366	0\\
367	0\\
368	0\\
369	0\\
370	0\\
371	0\\
372	0\\
373	0\\
374	0\\
375	0\\
376	0\\
377	0\\
378	0\\
379	0\\
380	0\\
381	0\\
382	0\\
383	0\\
384	0\\
385	0\\
386	0\\
387	0\\
388	0\\
389	0\\
390	0\\
391	0\\
392	0\\
393	0\\
394	0\\
395	0\\
396	0\\
397	0\\
398	0\\
399	0\\
400	0\\
401	0\\
402	0\\
403	0\\
404	0\\
405	0\\
406	0\\
407	0\\
408	0\\
409	0\\
410	0\\
411	0\\
412	0\\
413	0\\
414	0\\
415	0\\
416	0\\
417	0\\
418	0\\
419	0\\
420	0\\
421	0\\
422	0\\
423	0\\
424	0\\
425	0\\
426	0\\
427	0\\
428	0\\
429	0\\
430	0\\
431	0\\
432	0\\
433	0\\
434	0\\
435	0\\
436	0\\
437	0\\
438	0\\
439	0\\
440	0\\
441	0\\
442	0\\
443	0\\
444	0\\
445	0\\
446	0\\
447	0\\
448	0\\
449	0\\
450	0\\
451	0\\
452	0\\
453	0\\
454	0\\
455	0\\
456	0\\
457	0\\
458	0\\
459	0\\
460	0\\
461	0\\
462	0\\
463	0\\
464	0\\
465	0\\
466	0\\
467	0\\
468	0\\
469	0\\
470	0\\
471	0\\
472	0\\
473	0\\
474	0\\
475	0\\
476	0\\
477	0\\
478	0\\
479	0\\
480	0\\
481	0\\
482	0\\
483	0\\
484	0\\
485	0\\
486	0\\
487	0\\
488	0\\
489	0\\
490	0\\
491	0\\
492	0\\
493	0\\
494	0\\
495	0\\
496	0\\
497	0\\
498	0\\
499	0\\
500	0\\
501	0\\
502	0\\
503	0\\
504	0\\
505	0\\
506	0\\
507	0\\
508	0\\
509	0\\
510	0\\
511	0\\
512	0\\
513	0\\
514	0\\
515	0\\
516	0\\
517	0\\
518	0\\
519	0\\
520	0\\
521	0\\
522	0\\
523	0\\
524	0\\
525	0\\
526	0\\
527	0\\
528	0\\
529	0\\
530	0\\
531	0\\
532	0\\
533	0\\
534	0\\
535	0\\
536	0\\
537	0\\
538	0\\
539	0\\
540	0\\
541	0\\
542	0\\
543	0\\
544	0\\
545	0\\
546	0\\
547	0\\
548	0\\
549	0\\
550	0\\
551	0\\
552	0\\
553	0\\
554	0\\
555	0\\
556	0\\
557	0\\
558	0\\
559	0\\
560	0\\
561	0\\
562	0\\
563	0\\
564	0\\
565	0\\
566	0\\
567	0\\
568	0\\
569	0\\
570	0\\
571	0\\
572	0\\
573	0\\
574	0\\
575	0\\
576	0\\
577	0\\
578	0\\
579	0\\
580	0\\
581	0\\
582	0\\
583	0\\
584	0\\
585	0\\
586	0\\
587	0\\
588	0\\
589	0\\
590	0\\
591	0\\
592	0\\
593	0\\
594	0\\
595	0\\
596	0\\
597	0\\
598	0\\
599	0\\
600	0\\
};
\addplot [color=black!20!mycolor21,solid,forget plot]
  table[row sep=crcr]{%
1	0\\
2	0\\
3	0\\
4	0\\
5	0\\
6	0\\
7	0\\
8	0\\
9	0\\
10	0\\
11	0\\
12	0\\
13	0\\
14	0\\
15	0\\
16	0\\
17	0\\
18	0\\
19	0\\
20	0\\
21	0\\
22	0\\
23	0\\
24	0\\
25	0\\
26	0\\
27	0\\
28	0\\
29	0\\
30	0\\
31	0\\
32	0\\
33	0\\
34	0\\
35	0\\
36	0\\
37	0\\
38	0\\
39	0\\
40	0\\
41	0\\
42	0\\
43	0\\
44	0\\
45	0\\
46	0\\
47	0\\
48	0\\
49	0\\
50	0\\
51	0\\
52	0\\
53	0\\
54	0\\
55	0\\
56	0\\
57	0\\
58	0\\
59	0\\
60	0\\
61	0\\
62	0\\
63	0\\
64	0\\
65	0\\
66	0\\
67	0\\
68	0\\
69	0\\
70	0\\
71	0\\
72	0\\
73	0\\
74	0\\
75	0\\
76	0\\
77	0\\
78	0\\
79	0\\
80	0\\
81	0\\
82	0\\
83	0\\
84	0\\
85	0\\
86	0\\
87	0\\
88	0\\
89	0\\
90	0\\
91	0\\
92	0\\
93	0\\
94	0\\
95	0\\
96	0\\
97	0\\
98	0\\
99	0\\
100	0\\
101	0\\
102	0\\
103	0\\
104	0\\
105	0\\
106	0\\
107	0\\
108	0\\
109	0\\
110	0\\
111	0\\
112	0\\
113	0\\
114	0\\
115	0\\
116	0\\
117	0\\
118	0\\
119	0\\
120	0\\
121	0\\
122	0\\
123	0\\
124	0\\
125	0\\
126	0\\
127	0\\
128	0\\
129	0\\
130	0\\
131	0\\
132	0\\
133	0\\
134	0\\
135	0\\
136	0\\
137	0\\
138	0\\
139	0\\
140	0\\
141	0\\
142	0\\
143	0\\
144	0\\
145	0\\
146	0\\
147	0\\
148	0\\
149	0\\
150	0\\
151	0\\
152	0\\
153	0\\
154	0\\
155	0\\
156	0\\
157	0\\
158	0\\
159	0\\
160	0\\
161	0\\
162	0\\
163	0\\
164	0\\
165	0\\
166	0\\
167	0\\
168	0\\
169	0\\
170	0\\
171	0\\
172	0\\
173	0\\
174	0\\
175	0\\
176	0\\
177	0\\
178	0\\
179	0\\
180	0\\
181	0\\
182	0\\
183	0\\
184	0\\
185	0\\
186	0\\
187	0\\
188	0\\
189	0\\
190	0\\
191	0\\
192	0\\
193	0\\
194	0\\
195	0\\
196	0\\
197	0\\
198	0\\
199	0\\
200	0\\
201	0\\
202	0\\
203	0\\
204	0\\
205	0\\
206	0\\
207	0\\
208	0\\
209	0\\
210	0\\
211	0\\
212	0\\
213	0\\
214	0\\
215	0\\
216	0\\
217	0\\
218	0\\
219	0\\
220	0\\
221	0\\
222	0\\
223	0\\
224	0\\
225	0\\
226	0\\
227	0\\
228	0\\
229	0\\
230	0\\
231	0\\
232	0\\
233	0\\
234	0\\
235	0\\
236	0\\
237	0\\
238	0\\
239	0\\
240	0\\
241	0\\
242	0\\
243	0\\
244	0\\
245	0\\
246	0\\
247	0\\
248	0\\
249	0\\
250	0\\
251	0\\
252	0\\
253	0\\
254	0\\
255	0\\
256	0\\
257	0\\
258	0\\
259	0\\
260	0\\
261	0\\
262	0\\
263	0\\
264	0\\
265	0\\
266	0\\
267	0\\
268	0\\
269	0\\
270	0\\
271	0\\
272	0\\
273	0\\
274	0\\
275	0\\
276	0\\
277	0\\
278	0\\
279	0\\
280	0\\
281	0\\
282	0\\
283	0\\
284	0\\
285	0\\
286	0\\
287	0\\
288	0\\
289	0\\
290	0\\
291	0\\
292	0\\
293	0\\
294	0\\
295	0\\
296	0\\
297	0\\
298	0\\
299	0\\
300	0\\
301	0\\
302	0\\
303	0\\
304	0\\
305	0\\
306	0\\
307	0\\
308	0\\
309	0\\
310	0\\
311	0\\
312	0\\
313	0\\
314	0\\
315	0\\
316	0\\
317	0\\
318	0\\
319	0\\
320	0\\
321	0\\
322	0\\
323	0\\
324	0\\
325	0\\
326	0\\
327	0\\
328	0\\
329	0\\
330	0\\
331	0\\
332	0\\
333	0\\
334	0\\
335	0\\
336	0\\
337	0\\
338	0\\
339	0\\
340	0\\
341	0\\
342	0\\
343	0\\
344	0\\
345	0\\
346	0\\
347	0\\
348	0\\
349	0\\
350	0\\
351	0\\
352	0\\
353	0\\
354	0\\
355	0\\
356	0\\
357	0\\
358	0\\
359	0\\
360	0\\
361	0\\
362	0\\
363	0\\
364	0\\
365	0\\
366	0\\
367	0\\
368	0\\
369	0\\
370	0\\
371	0\\
372	0\\
373	0\\
374	0\\
375	0\\
376	0\\
377	0\\
378	0\\
379	0\\
380	0\\
381	0\\
382	0\\
383	0\\
384	0\\
385	0\\
386	0\\
387	0\\
388	0\\
389	0\\
390	0\\
391	0\\
392	0\\
393	0\\
394	0\\
395	0\\
396	0\\
397	0\\
398	0\\
399	0\\
400	0\\
401	0\\
402	0\\
403	0\\
404	0\\
405	0\\
406	0\\
407	0\\
408	0\\
409	0\\
410	0\\
411	0\\
412	0\\
413	0\\
414	0\\
415	0\\
416	0\\
417	0\\
418	0\\
419	0\\
420	0\\
421	0\\
422	0\\
423	0\\
424	0\\
425	0\\
426	0\\
427	0\\
428	0\\
429	0\\
430	0\\
431	0\\
432	0\\
433	0\\
434	0\\
435	0\\
436	0\\
437	0\\
438	0\\
439	0\\
440	0\\
441	0\\
442	0\\
443	0\\
444	0\\
445	0\\
446	0\\
447	0\\
448	0\\
449	0\\
450	0\\
451	0\\
452	0\\
453	0\\
454	0\\
455	0\\
456	0\\
457	0\\
458	0\\
459	0\\
460	0\\
461	0\\
462	0\\
463	0\\
464	0\\
465	0\\
466	0\\
467	0\\
468	0\\
469	0\\
470	0\\
471	0\\
472	0\\
473	0\\
474	0\\
475	0\\
476	0\\
477	0\\
478	0\\
479	0\\
480	0\\
481	0\\
482	0\\
483	0\\
484	0\\
485	0\\
486	0\\
487	0\\
488	0\\
489	0\\
490	0\\
491	0\\
492	0\\
493	0\\
494	0\\
495	0\\
496	0\\
497	0\\
498	0\\
499	0\\
500	0\\
501	0\\
502	0\\
503	0\\
504	0\\
505	0\\
506	0\\
507	0\\
508	0\\
509	0\\
510	0\\
511	0\\
512	0\\
513	0\\
514	0\\
515	0\\
516	0\\
517	0\\
518	0\\
519	0\\
520	0\\
521	0\\
522	0\\
523	0\\
524	0\\
525	0\\
526	0\\
527	0\\
528	0\\
529	0\\
530	0\\
531	0\\
532	0\\
533	0\\
534	0\\
535	0\\
536	0\\
537	0\\
538	0\\
539	0\\
540	0\\
541	0\\
542	0\\
543	0\\
544	0\\
545	0\\
546	0\\
547	0\\
548	0\\
549	0\\
550	0\\
551	0\\
552	0\\
553	0\\
554	0\\
555	0\\
556	0\\
557	0\\
558	0\\
559	0\\
560	0\\
561	0\\
562	0\\
563	0\\
564	0\\
565	0\\
566	0\\
567	0\\
568	0\\
569	0\\
570	0\\
571	0\\
572	0\\
573	0\\
574	0\\
575	0\\
576	0\\
577	0\\
578	0\\
579	0\\
580	0\\
581	0\\
582	0\\
583	0\\
584	0\\
585	0\\
586	0\\
587	0\\
588	0\\
589	0\\
590	0\\
591	0\\
592	0\\
593	0\\
594	0\\
595	0\\
596	0\\
597	0\\
598	0\\
599	0\\
600	0\\
};
\addplot [color=black!50!mycolor20,solid,forget plot]
  table[row sep=crcr]{%
1	0\\
2	0\\
3	0\\
4	0\\
5	0\\
6	0\\
7	0\\
8	0\\
9	0\\
10	0\\
11	0\\
12	0\\
13	0\\
14	0\\
15	0\\
16	0\\
17	0\\
18	0\\
19	0\\
20	0\\
21	0\\
22	0\\
23	0\\
24	0\\
25	0\\
26	0\\
27	0\\
28	0\\
29	0\\
30	0\\
31	0\\
32	0\\
33	0\\
34	0\\
35	0\\
36	0\\
37	0\\
38	0\\
39	0\\
40	0\\
41	0\\
42	0\\
43	0\\
44	0\\
45	0\\
46	0\\
47	0\\
48	0\\
49	0\\
50	0\\
51	0\\
52	0\\
53	0\\
54	0\\
55	0\\
56	0\\
57	0\\
58	0\\
59	0\\
60	0\\
61	0\\
62	0\\
63	0\\
64	0\\
65	0\\
66	0\\
67	0\\
68	0\\
69	0\\
70	0\\
71	0\\
72	0\\
73	0\\
74	0\\
75	0\\
76	0\\
77	0\\
78	0\\
79	0\\
80	0\\
81	0\\
82	0\\
83	0\\
84	0\\
85	0\\
86	0\\
87	0\\
88	0\\
89	0\\
90	0\\
91	0\\
92	0\\
93	0\\
94	0\\
95	0\\
96	0\\
97	0\\
98	0\\
99	0\\
100	0\\
101	0\\
102	0\\
103	0\\
104	0\\
105	0\\
106	0\\
107	0\\
108	0\\
109	0\\
110	0\\
111	0\\
112	0\\
113	0\\
114	0\\
115	0\\
116	0\\
117	0\\
118	0\\
119	0\\
120	0\\
121	0\\
122	0\\
123	0\\
124	0\\
125	0\\
126	0\\
127	0\\
128	0\\
129	0\\
130	0\\
131	0\\
132	0\\
133	0\\
134	0\\
135	0\\
136	0\\
137	0\\
138	0\\
139	0\\
140	0\\
141	0\\
142	0\\
143	0\\
144	0\\
145	0\\
146	0\\
147	0\\
148	0\\
149	0\\
150	0\\
151	0\\
152	0\\
153	0\\
154	0\\
155	0\\
156	0\\
157	0\\
158	0\\
159	0\\
160	0\\
161	0\\
162	0\\
163	0\\
164	0\\
165	0\\
166	0\\
167	0\\
168	0\\
169	0\\
170	0\\
171	0\\
172	0\\
173	0\\
174	0\\
175	0\\
176	0\\
177	0\\
178	0\\
179	0\\
180	0\\
181	0\\
182	0\\
183	0\\
184	0\\
185	0\\
186	0\\
187	0\\
188	0\\
189	0\\
190	0\\
191	0\\
192	0\\
193	0\\
194	0\\
195	0\\
196	0\\
197	0\\
198	0\\
199	0\\
200	0\\
201	0\\
202	0\\
203	0\\
204	0\\
205	0\\
206	0\\
207	0\\
208	0\\
209	0\\
210	0\\
211	0\\
212	0\\
213	0\\
214	0\\
215	0\\
216	0\\
217	0\\
218	0\\
219	0\\
220	0\\
221	0\\
222	0\\
223	0\\
224	0\\
225	0\\
226	0\\
227	0\\
228	0\\
229	0\\
230	0\\
231	0\\
232	0\\
233	0\\
234	0\\
235	0\\
236	0\\
237	0\\
238	0\\
239	0\\
240	0\\
241	0\\
242	0\\
243	0\\
244	0\\
245	0\\
246	0\\
247	0\\
248	0\\
249	0\\
250	0\\
251	0\\
252	0\\
253	0\\
254	0\\
255	0\\
256	0\\
257	0\\
258	0\\
259	0\\
260	0\\
261	0\\
262	0\\
263	0\\
264	0\\
265	0\\
266	0\\
267	0\\
268	0\\
269	0\\
270	0\\
271	0\\
272	0\\
273	0\\
274	0\\
275	0\\
276	0\\
277	0\\
278	0\\
279	0\\
280	0\\
281	0\\
282	0\\
283	0\\
284	0\\
285	0\\
286	0\\
287	0\\
288	0\\
289	0\\
290	0\\
291	0\\
292	0\\
293	0\\
294	0\\
295	0\\
296	0\\
297	0\\
298	0\\
299	0\\
300	0\\
301	0\\
302	0\\
303	0\\
304	0\\
305	0\\
306	0\\
307	0\\
308	0\\
309	0\\
310	0\\
311	0\\
312	0\\
313	0\\
314	0\\
315	0\\
316	0\\
317	0\\
318	0\\
319	0\\
320	0\\
321	0\\
322	0\\
323	0\\
324	0\\
325	0\\
326	0\\
327	0\\
328	0\\
329	0\\
330	0\\
331	0\\
332	0\\
333	0\\
334	0\\
335	0\\
336	0\\
337	0\\
338	0\\
339	0\\
340	0\\
341	0\\
342	0\\
343	0\\
344	0\\
345	0\\
346	0\\
347	0\\
348	0\\
349	0\\
350	0\\
351	0\\
352	0\\
353	0\\
354	0\\
355	0\\
356	0\\
357	0\\
358	0\\
359	0\\
360	0\\
361	0\\
362	0\\
363	0\\
364	0\\
365	0\\
366	0\\
367	0\\
368	0\\
369	0\\
370	0\\
371	0\\
372	0\\
373	0\\
374	0\\
375	0\\
376	0\\
377	0\\
378	0\\
379	0\\
380	0\\
381	0\\
382	0\\
383	0\\
384	0\\
385	0\\
386	0\\
387	0\\
388	0\\
389	0\\
390	0\\
391	0\\
392	0\\
393	0\\
394	0\\
395	0\\
396	0\\
397	0\\
398	0\\
399	0\\
400	0\\
401	0\\
402	0\\
403	0\\
404	0\\
405	0\\
406	0\\
407	0\\
408	0\\
409	0\\
410	0\\
411	0\\
412	0\\
413	0\\
414	0\\
415	0\\
416	0\\
417	0\\
418	0\\
419	0\\
420	0\\
421	0\\
422	0\\
423	0\\
424	0\\
425	0\\
426	0\\
427	0\\
428	0\\
429	0\\
430	0\\
431	0\\
432	0\\
433	0\\
434	0\\
435	0\\
436	0\\
437	0\\
438	0\\
439	0\\
440	0\\
441	0\\
442	0\\
443	0\\
444	0\\
445	0\\
446	0\\
447	0\\
448	0\\
449	0\\
450	0\\
451	0\\
452	0\\
453	0\\
454	0\\
455	0\\
456	0\\
457	0\\
458	0\\
459	0\\
460	0\\
461	0\\
462	0\\
463	0\\
464	0\\
465	0\\
466	0\\
467	0\\
468	0\\
469	0\\
470	0\\
471	0\\
472	0\\
473	0\\
474	0\\
475	0\\
476	0\\
477	0\\
478	0\\
479	0\\
480	0\\
481	0\\
482	0\\
483	0\\
484	0\\
485	0\\
486	0\\
487	0\\
488	0\\
489	0\\
490	0\\
491	0\\
492	0\\
493	0\\
494	0\\
495	0\\
496	0\\
497	0\\
498	0\\
499	0\\
500	0\\
501	0\\
502	0\\
503	0\\
504	0\\
505	0\\
506	0\\
507	0\\
508	0\\
509	0\\
510	0\\
511	0\\
512	0\\
513	0\\
514	0\\
515	0\\
516	0\\
517	0\\
518	0\\
519	0\\
520	0\\
521	0\\
522	0\\
523	0\\
524	0\\
525	0\\
526	0\\
527	0\\
528	0\\
529	0\\
530	0\\
531	0\\
532	0\\
533	0\\
534	0\\
535	0\\
536	0\\
537	0\\
538	0\\
539	0\\
540	0\\
541	0\\
542	0\\
543	0\\
544	0\\
545	0\\
546	0\\
547	0\\
548	0\\
549	0\\
550	0\\
551	0\\
552	0\\
553	0\\
554	0\\
555	0\\
556	0\\
557	0\\
558	0\\
559	0\\
560	0\\
561	0\\
562	0\\
563	0\\
564	0\\
565	0\\
566	0\\
567	0\\
568	0\\
569	0\\
570	0\\
571	0\\
572	0\\
573	0\\
574	0\\
575	0\\
576	0\\
577	0\\
578	0\\
579	0\\
580	0\\
581	0\\
582	0\\
583	0\\
584	0\\
585	0\\
586	0\\
587	0\\
588	0\\
589	0\\
590	0\\
591	0\\
592	0\\
593	0\\
594	0\\
595	0\\
596	0\\
597	0\\
598	0\\
599	0\\
600	0\\
};
\addplot [color=black!60!mycolor21,solid,forget plot]
  table[row sep=crcr]{%
1	0\\
2	0\\
3	0\\
4	0\\
5	0\\
6	0\\
7	0\\
8	0\\
9	0\\
10	0\\
11	0\\
12	0\\
13	0\\
14	0\\
15	0\\
16	0\\
17	0\\
18	0\\
19	0\\
20	0\\
21	0\\
22	0\\
23	0\\
24	0\\
25	0\\
26	0\\
27	0\\
28	0\\
29	0\\
30	0\\
31	0\\
32	0\\
33	0\\
34	0\\
35	0\\
36	0\\
37	0\\
38	0\\
39	0\\
40	0\\
41	0\\
42	0\\
43	0\\
44	0\\
45	0\\
46	0\\
47	0\\
48	0\\
49	0\\
50	0\\
51	0\\
52	0\\
53	0\\
54	0\\
55	0\\
56	0\\
57	0\\
58	0\\
59	0\\
60	0\\
61	0\\
62	0\\
63	0\\
64	0\\
65	0\\
66	0\\
67	0\\
68	0\\
69	0\\
70	0\\
71	0\\
72	0\\
73	0\\
74	0\\
75	0\\
76	0\\
77	0\\
78	0\\
79	0\\
80	0\\
81	0\\
82	0\\
83	0\\
84	0\\
85	0\\
86	0\\
87	0\\
88	0\\
89	0\\
90	0\\
91	0\\
92	0\\
93	0\\
94	0\\
95	0\\
96	0\\
97	0\\
98	0\\
99	0\\
100	0\\
101	0\\
102	0\\
103	0\\
104	0\\
105	0\\
106	0\\
107	0\\
108	0\\
109	0\\
110	0\\
111	0\\
112	0\\
113	0\\
114	0\\
115	0\\
116	0\\
117	0\\
118	0\\
119	0\\
120	0\\
121	0\\
122	0\\
123	0\\
124	0\\
125	0\\
126	0\\
127	0\\
128	0\\
129	0\\
130	0\\
131	0\\
132	0\\
133	0\\
134	0\\
135	0\\
136	0\\
137	0\\
138	0\\
139	0\\
140	0\\
141	0\\
142	0\\
143	0\\
144	0\\
145	0\\
146	0\\
147	0\\
148	0\\
149	0\\
150	0\\
151	0\\
152	0\\
153	0\\
154	0\\
155	0\\
156	0\\
157	0\\
158	0\\
159	0\\
160	0\\
161	0\\
162	0\\
163	0\\
164	0\\
165	0\\
166	0\\
167	0\\
168	0\\
169	0\\
170	0\\
171	0\\
172	0\\
173	0\\
174	0\\
175	0\\
176	0\\
177	0\\
178	0\\
179	0\\
180	0\\
181	0\\
182	0\\
183	0\\
184	0\\
185	0\\
186	0\\
187	0\\
188	0\\
189	0\\
190	0\\
191	0\\
192	0\\
193	0\\
194	0\\
195	0\\
196	0\\
197	0\\
198	0\\
199	0\\
200	0\\
201	0\\
202	0\\
203	0\\
204	0\\
205	0\\
206	0\\
207	0\\
208	0\\
209	0\\
210	0\\
211	0\\
212	0\\
213	0\\
214	0\\
215	0\\
216	0\\
217	0\\
218	0\\
219	0\\
220	0\\
221	0\\
222	0\\
223	0\\
224	0\\
225	0\\
226	0\\
227	0\\
228	0\\
229	0\\
230	0\\
231	0\\
232	0\\
233	0\\
234	0\\
235	0\\
236	0\\
237	0\\
238	0\\
239	0\\
240	0\\
241	0\\
242	0\\
243	0\\
244	0\\
245	0\\
246	0\\
247	0\\
248	0\\
249	0\\
250	0\\
251	0\\
252	0\\
253	0\\
254	0\\
255	0\\
256	0\\
257	0\\
258	0\\
259	0\\
260	0\\
261	0\\
262	0\\
263	0\\
264	0\\
265	0\\
266	0\\
267	0\\
268	0\\
269	0\\
270	0\\
271	0\\
272	0\\
273	0\\
274	0\\
275	0\\
276	0\\
277	0\\
278	0\\
279	0\\
280	0\\
281	0\\
282	0\\
283	0\\
284	0\\
285	0\\
286	0\\
287	0\\
288	0\\
289	0\\
290	0\\
291	0\\
292	0\\
293	0\\
294	0\\
295	0\\
296	0\\
297	0\\
298	0\\
299	0\\
300	0\\
301	0\\
302	0\\
303	0\\
304	0\\
305	0\\
306	0\\
307	0\\
308	0\\
309	0\\
310	0\\
311	0\\
312	0\\
313	0\\
314	0\\
315	0\\
316	0\\
317	0\\
318	0\\
319	0\\
320	0\\
321	0\\
322	0\\
323	0\\
324	0\\
325	0\\
326	0\\
327	0\\
328	0\\
329	0\\
330	0\\
331	0\\
332	0\\
333	0\\
334	0\\
335	0\\
336	0\\
337	0\\
338	0\\
339	0\\
340	0\\
341	0\\
342	0\\
343	0\\
344	0\\
345	0\\
346	0\\
347	0\\
348	0\\
349	0\\
350	0\\
351	0\\
352	0\\
353	0\\
354	0\\
355	0\\
356	0\\
357	0\\
358	0\\
359	0\\
360	0\\
361	0\\
362	0\\
363	0\\
364	0\\
365	0\\
366	0\\
367	0\\
368	0\\
369	0\\
370	0\\
371	0\\
372	0\\
373	0\\
374	0\\
375	0\\
376	0\\
377	0\\
378	0\\
379	0\\
380	0\\
381	0\\
382	0\\
383	0\\
384	0\\
385	0\\
386	0\\
387	0\\
388	0\\
389	0\\
390	0\\
391	0\\
392	0\\
393	0\\
394	0\\
395	0\\
396	0\\
397	0\\
398	0\\
399	0\\
400	0\\
401	0\\
402	0\\
403	0\\
404	0\\
405	0\\
406	0\\
407	0\\
408	0\\
409	0\\
410	0\\
411	0\\
412	0\\
413	0\\
414	0\\
415	0\\
416	0\\
417	0\\
418	0\\
419	0\\
420	0\\
421	0\\
422	0\\
423	0\\
424	0\\
425	0\\
426	0\\
427	0\\
428	0\\
429	0\\
430	0\\
431	0\\
432	0\\
433	0\\
434	0\\
435	0\\
436	0\\
437	0\\
438	0\\
439	0\\
440	0\\
441	0\\
442	0\\
443	0\\
444	0\\
445	0\\
446	0\\
447	0\\
448	0\\
449	0\\
450	0\\
451	0\\
452	0\\
453	0\\
454	0\\
455	0\\
456	0\\
457	0\\
458	0\\
459	0\\
460	0\\
461	0\\
462	0\\
463	0\\
464	0\\
465	0\\
466	0\\
467	0\\
468	0\\
469	0\\
470	0\\
471	0\\
472	0\\
473	0\\
474	0\\
475	0\\
476	0\\
477	0\\
478	0\\
479	0\\
480	0\\
481	0\\
482	0\\
483	0\\
484	0\\
485	0\\
486	0\\
487	0\\
488	0\\
489	0\\
490	0\\
491	0\\
492	0\\
493	0\\
494	0\\
495	0\\
496	0\\
497	0\\
498	0\\
499	0\\
500	0\\
501	0\\
502	0\\
503	0\\
504	0\\
505	0\\
506	0\\
507	0\\
508	0\\
509	0\\
510	0\\
511	0\\
512	0\\
513	0\\
514	0\\
515	0\\
516	0\\
517	0\\
518	0\\
519	0\\
520	0\\
521	0\\
522	0\\
523	0\\
524	0\\
525	0\\
526	0\\
527	0\\
528	0\\
529	0\\
530	0\\
531	0\\
532	0\\
533	0\\
534	0\\
535	0\\
536	0\\
537	0\\
538	0\\
539	0\\
540	0\\
541	0\\
542	0\\
543	0\\
544	0\\
545	0\\
546	0\\
547	0\\
548	0\\
549	0\\
550	0\\
551	0\\
552	0\\
553	0\\
554	0\\
555	0\\
556	0\\
557	0\\
558	0\\
559	0\\
560	0\\
561	0\\
562	0\\
563	0\\
564	0\\
565	0\\
566	0\\
567	0\\
568	0\\
569	0\\
570	0\\
571	0\\
572	0\\
573	0\\
574	0\\
575	0\\
576	0\\
577	0\\
578	0\\
579	0\\
580	0\\
581	0\\
582	0\\
583	0\\
584	0\\
585	0\\
586	0\\
587	0\\
588	0\\
589	0\\
590	0\\
591	0\\
592	0\\
593	0\\
594	0\\
595	0\\
596	0\\
597	0\\
598	0\\
599	0\\
600	0\\
};
\addplot [color=black!80!mycolor21,solid,forget plot]
  table[row sep=crcr]{%
1	0\\
2	0\\
3	0\\
4	0\\
5	0\\
6	0\\
7	0\\
8	0\\
9	0\\
10	0\\
11	0\\
12	0\\
13	0\\
14	0\\
15	0\\
16	0\\
17	0\\
18	0\\
19	0\\
20	0\\
21	0\\
22	0\\
23	0\\
24	0\\
25	0\\
26	0\\
27	0\\
28	0\\
29	0\\
30	0\\
31	0\\
32	0\\
33	0\\
34	0\\
35	0\\
36	0\\
37	0\\
38	0\\
39	0\\
40	0\\
41	0\\
42	0\\
43	0\\
44	0\\
45	0\\
46	0\\
47	0\\
48	0\\
49	0\\
50	0\\
51	0\\
52	0\\
53	0\\
54	0\\
55	0\\
56	0\\
57	0\\
58	0\\
59	0\\
60	0\\
61	0\\
62	0\\
63	0\\
64	0\\
65	0\\
66	0\\
67	0\\
68	0\\
69	0\\
70	0\\
71	0\\
72	0\\
73	0\\
74	0\\
75	0\\
76	0\\
77	0\\
78	0\\
79	0\\
80	0\\
81	0\\
82	0\\
83	0\\
84	0\\
85	0\\
86	0\\
87	0\\
88	0\\
89	0\\
90	0\\
91	0\\
92	0\\
93	0\\
94	0\\
95	0\\
96	0\\
97	0\\
98	0\\
99	0\\
100	0\\
101	0\\
102	0\\
103	0\\
104	0\\
105	0\\
106	0\\
107	0\\
108	0\\
109	0\\
110	0\\
111	0\\
112	0\\
113	0\\
114	0\\
115	0\\
116	0\\
117	0\\
118	0\\
119	0\\
120	0\\
121	0\\
122	0\\
123	0\\
124	0\\
125	0\\
126	0\\
127	0\\
128	0\\
129	0\\
130	0\\
131	0\\
132	0\\
133	0\\
134	0\\
135	0\\
136	0\\
137	0\\
138	0\\
139	0\\
140	0\\
141	0\\
142	0\\
143	0\\
144	0\\
145	0\\
146	0\\
147	0\\
148	0\\
149	0\\
150	0\\
151	0\\
152	0\\
153	0\\
154	0\\
155	0\\
156	0\\
157	0\\
158	0\\
159	0\\
160	0\\
161	0\\
162	0\\
163	0\\
164	0\\
165	0\\
166	0\\
167	0\\
168	0\\
169	0\\
170	0\\
171	0\\
172	0\\
173	0\\
174	0\\
175	0\\
176	0\\
177	0\\
178	0\\
179	0\\
180	0\\
181	0\\
182	0\\
183	0\\
184	0\\
185	0\\
186	0\\
187	0\\
188	0\\
189	0\\
190	0\\
191	0\\
192	0\\
193	0\\
194	0\\
195	0\\
196	0\\
197	0\\
198	0\\
199	0\\
200	0\\
201	0\\
202	0\\
203	0\\
204	0\\
205	0\\
206	0\\
207	0\\
208	0\\
209	0\\
210	0\\
211	0\\
212	0\\
213	0\\
214	0\\
215	0\\
216	0\\
217	0\\
218	0\\
219	0\\
220	0\\
221	0\\
222	0\\
223	0\\
224	0\\
225	0\\
226	0\\
227	0\\
228	0\\
229	0\\
230	0\\
231	0\\
232	0\\
233	0\\
234	0\\
235	0\\
236	0\\
237	0\\
238	0\\
239	0\\
240	0\\
241	0\\
242	0\\
243	0\\
244	0\\
245	0\\
246	0\\
247	0\\
248	0\\
249	0\\
250	0\\
251	0\\
252	0\\
253	0\\
254	0\\
255	0\\
256	0\\
257	0\\
258	0\\
259	0\\
260	0\\
261	0\\
262	0\\
263	0\\
264	0\\
265	0\\
266	0\\
267	0\\
268	0\\
269	0\\
270	0\\
271	0\\
272	0\\
273	0\\
274	0\\
275	0\\
276	0\\
277	0\\
278	0\\
279	0\\
280	0\\
281	0\\
282	0\\
283	0\\
284	0\\
285	0\\
286	0\\
287	0\\
288	0\\
289	0\\
290	0\\
291	0\\
292	0\\
293	0\\
294	0\\
295	0\\
296	0\\
297	0\\
298	0\\
299	0\\
300	0\\
301	0\\
302	0\\
303	0\\
304	0\\
305	0\\
306	0\\
307	0\\
308	0\\
309	0\\
310	0\\
311	0\\
312	0\\
313	0\\
314	0\\
315	0\\
316	0\\
317	0\\
318	0\\
319	0\\
320	0\\
321	0\\
322	0\\
323	0\\
324	0\\
325	0\\
326	0\\
327	0\\
328	0\\
329	0\\
330	0\\
331	0\\
332	0\\
333	0\\
334	0\\
335	0\\
336	0\\
337	0\\
338	0\\
339	0\\
340	0\\
341	0\\
342	0\\
343	0\\
344	0\\
345	0\\
346	0\\
347	0\\
348	0\\
349	0\\
350	0\\
351	0\\
352	0\\
353	0\\
354	0\\
355	0\\
356	0\\
357	0\\
358	0\\
359	0\\
360	0\\
361	0\\
362	0\\
363	0\\
364	0\\
365	0\\
366	0\\
367	0\\
368	0\\
369	0\\
370	0\\
371	0\\
372	0\\
373	0\\
374	0\\
375	0\\
376	0\\
377	0\\
378	0\\
379	0\\
380	0\\
381	0\\
382	0\\
383	0\\
384	0\\
385	0\\
386	0\\
387	0\\
388	0\\
389	0\\
390	0\\
391	0\\
392	0\\
393	0\\
394	0\\
395	0\\
396	0\\
397	0\\
398	0\\
399	0\\
400	0\\
401	0\\
402	0\\
403	0\\
404	0\\
405	0\\
406	0\\
407	0\\
408	0\\
409	0\\
410	0\\
411	0\\
412	0\\
413	0\\
414	0\\
415	0\\
416	0\\
417	0\\
418	0\\
419	0\\
420	0\\
421	0\\
422	0\\
423	0\\
424	0\\
425	0\\
426	0\\
427	0\\
428	0\\
429	0\\
430	0\\
431	0\\
432	0\\
433	0\\
434	0\\
435	0\\
436	0\\
437	0\\
438	0\\
439	0\\
440	0\\
441	0\\
442	0\\
443	0\\
444	0\\
445	0\\
446	0\\
447	0\\
448	0\\
449	0\\
450	0\\
451	0\\
452	0\\
453	0\\
454	0\\
455	0\\
456	0\\
457	0\\
458	0\\
459	0\\
460	0\\
461	0\\
462	0\\
463	0\\
464	0\\
465	0\\
466	0\\
467	0\\
468	0\\
469	0\\
470	0\\
471	0\\
472	0\\
473	0\\
474	0\\
475	0\\
476	0\\
477	0\\
478	0\\
479	0\\
480	0\\
481	0\\
482	0\\
483	0\\
484	0\\
485	0\\
486	0\\
487	0\\
488	0\\
489	0\\
490	0\\
491	0\\
492	0\\
493	0\\
494	0\\
495	0\\
496	0\\
497	0\\
498	0\\
499	0\\
500	0\\
501	0\\
502	0\\
503	0\\
504	0\\
505	0\\
506	0\\
507	0\\
508	0\\
509	0\\
510	0\\
511	0\\
512	0\\
513	0\\
514	0\\
515	0\\
516	0\\
517	0\\
518	0\\
519	0\\
520	0\\
521	0\\
522	0\\
523	0\\
524	0\\
525	0\\
526	0\\
527	0\\
528	0\\
529	0\\
530	0\\
531	0\\
532	0\\
533	0\\
534	0\\
535	0\\
536	0\\
537	0\\
538	0\\
539	0\\
540	0\\
541	0\\
542	0\\
543	0\\
544	0\\
545	0\\
546	0\\
547	0\\
548	0\\
549	0\\
550	0\\
551	0\\
552	0\\
553	0\\
554	0\\
555	0\\
556	0\\
557	0\\
558	0\\
559	0\\
560	0\\
561	0\\
562	0\\
563	0\\
564	0\\
565	0\\
566	0\\
567	0\\
568	0\\
569	0\\
570	0\\
571	0\\
572	0\\
573	0\\
574	0\\
575	0\\
576	0\\
577	0\\
578	0\\
579	0\\
580	0\\
581	0\\
582	0\\
583	0\\
584	0\\
585	0\\
586	0\\
587	0\\
588	0\\
589	0\\
590	0\\
591	0\\
592	0\\
593	0\\
594	0\\
595	0\\
596	0\\
597	0\\
598	0\\
599	0\\
600	0\\
};
\addplot [color=black,solid,forget plot]
  table[row sep=crcr]{%
1	0\\
2	0\\
3	0\\
4	0\\
5	0\\
6	0\\
7	0\\
8	0\\
9	0\\
10	0\\
11	0\\
12	0\\
13	0\\
14	0\\
15	0\\
16	0\\
17	0\\
18	0\\
19	0\\
20	0\\
21	0\\
22	0\\
23	0\\
24	0\\
25	0\\
26	0\\
27	0\\
28	0\\
29	0\\
30	0\\
31	0\\
32	0\\
33	0\\
34	0\\
35	0\\
36	0\\
37	0\\
38	0\\
39	0\\
40	0\\
41	0\\
42	0\\
43	0\\
44	0\\
45	0\\
46	0\\
47	0\\
48	0\\
49	0\\
50	0\\
51	0\\
52	0\\
53	0\\
54	0\\
55	0\\
56	0\\
57	0\\
58	0\\
59	0\\
60	0\\
61	0\\
62	0\\
63	0\\
64	0\\
65	0\\
66	0\\
67	0\\
68	0\\
69	0\\
70	0\\
71	0\\
72	0\\
73	0\\
74	0\\
75	0\\
76	0\\
77	0\\
78	0\\
79	0\\
80	0\\
81	0\\
82	0\\
83	0\\
84	0\\
85	0\\
86	0\\
87	0\\
88	0\\
89	0\\
90	0\\
91	0\\
92	0\\
93	0\\
94	0\\
95	0\\
96	0\\
97	0\\
98	0\\
99	0\\
100	0\\
101	0\\
102	0\\
103	0\\
104	0\\
105	0\\
106	0\\
107	0\\
108	0\\
109	0\\
110	0\\
111	0\\
112	0\\
113	0\\
114	0\\
115	0\\
116	0\\
117	0\\
118	0\\
119	0\\
120	0\\
121	0\\
122	0\\
123	0\\
124	0\\
125	0\\
126	0\\
127	0\\
128	0\\
129	0\\
130	0\\
131	0\\
132	0\\
133	0\\
134	0\\
135	0\\
136	0\\
137	0\\
138	0\\
139	0\\
140	0\\
141	0\\
142	0\\
143	0\\
144	0\\
145	0\\
146	0\\
147	0\\
148	0\\
149	0\\
150	0\\
151	0\\
152	0\\
153	0\\
154	0\\
155	0\\
156	0\\
157	0\\
158	0\\
159	0\\
160	0\\
161	0\\
162	0\\
163	0\\
164	0\\
165	0\\
166	0\\
167	0\\
168	0\\
169	0\\
170	0\\
171	0\\
172	0\\
173	0\\
174	0\\
175	0\\
176	0\\
177	0\\
178	0\\
179	0\\
180	0\\
181	0\\
182	0\\
183	0\\
184	0\\
185	0\\
186	0\\
187	0\\
188	0\\
189	0\\
190	0\\
191	0\\
192	0\\
193	0\\
194	0\\
195	0\\
196	0\\
197	0\\
198	0\\
199	0\\
200	0\\
201	0\\
202	0\\
203	0\\
204	0\\
205	0\\
206	0\\
207	0\\
208	0\\
209	0\\
210	0\\
211	0\\
212	0\\
213	0\\
214	0\\
215	0\\
216	0\\
217	0\\
218	0\\
219	0\\
220	0\\
221	0\\
222	0\\
223	0\\
224	0\\
225	0\\
226	0\\
227	0\\
228	0\\
229	0\\
230	0\\
231	0\\
232	0\\
233	0\\
234	0\\
235	0\\
236	0\\
237	0\\
238	0\\
239	0\\
240	0\\
241	0\\
242	0\\
243	0\\
244	0\\
245	0\\
246	0\\
247	0\\
248	0\\
249	0\\
250	0\\
251	0\\
252	0\\
253	0\\
254	0\\
255	0\\
256	0\\
257	0\\
258	0\\
259	0\\
260	0\\
261	0\\
262	0\\
263	0\\
264	0\\
265	0\\
266	0\\
267	0\\
268	0\\
269	0\\
270	0\\
271	0\\
272	0\\
273	0\\
274	0\\
275	0\\
276	0\\
277	0\\
278	0\\
279	0\\
280	0\\
281	0\\
282	0\\
283	0\\
284	0\\
285	0\\
286	0\\
287	0\\
288	0\\
289	0\\
290	0\\
291	0\\
292	0\\
293	0\\
294	0\\
295	0\\
296	0\\
297	0\\
298	0\\
299	0\\
300	0\\
301	0\\
302	0\\
303	0\\
304	0\\
305	0\\
306	0\\
307	0\\
308	0\\
309	0\\
310	0\\
311	0\\
312	0\\
313	0\\
314	0\\
315	0\\
316	0\\
317	0\\
318	0\\
319	0\\
320	0\\
321	0\\
322	0\\
323	0\\
324	0\\
325	0\\
326	0\\
327	0\\
328	0\\
329	0\\
330	0\\
331	0\\
332	0\\
333	0\\
334	0\\
335	0\\
336	0\\
337	0\\
338	0\\
339	0\\
340	0\\
341	0\\
342	0\\
343	0\\
344	0\\
345	0\\
346	0\\
347	0\\
348	0\\
349	0\\
350	0\\
351	0\\
352	0\\
353	0\\
354	0\\
355	0\\
356	0\\
357	0\\
358	0\\
359	0\\
360	0\\
361	0\\
362	0\\
363	0\\
364	0\\
365	0\\
366	0\\
367	0\\
368	0\\
369	0\\
370	0\\
371	0\\
372	0\\
373	0\\
374	0\\
375	0\\
376	0\\
377	0\\
378	0\\
379	0\\
380	0\\
381	0\\
382	0\\
383	0\\
384	0\\
385	0\\
386	0\\
387	0\\
388	0\\
389	0\\
390	0\\
391	0\\
392	0\\
393	0\\
394	0\\
395	0\\
396	0\\
397	0\\
398	0\\
399	0\\
400	0\\
401	0\\
402	0\\
403	0\\
404	0\\
405	0\\
406	0\\
407	0\\
408	0\\
409	0\\
410	0\\
411	0\\
412	0\\
413	0\\
414	0\\
415	0\\
416	0\\
417	0\\
418	0\\
419	0\\
420	0\\
421	0\\
422	0\\
423	0\\
424	0\\
425	0\\
426	0\\
427	0\\
428	0\\
429	0\\
430	0\\
431	0\\
432	0\\
433	0\\
434	0\\
435	0\\
436	0\\
437	0\\
438	0\\
439	0\\
440	0\\
441	0\\
442	0\\
443	0\\
444	0\\
445	0\\
446	0\\
447	0\\
448	0\\
449	0\\
450	0\\
451	0\\
452	0\\
453	0\\
454	0\\
455	0\\
456	0\\
457	0\\
458	0\\
459	0\\
460	0\\
461	0\\
462	0\\
463	0\\
464	0\\
465	0\\
466	0\\
467	0\\
468	0\\
469	0\\
470	0\\
471	0\\
472	0\\
473	0\\
474	0\\
475	0\\
476	0\\
477	0\\
478	0\\
479	0\\
480	0\\
481	0\\
482	0\\
483	0\\
484	0\\
485	0\\
486	0\\
487	0\\
488	0\\
489	0\\
490	0\\
491	0\\
492	0\\
493	0\\
494	0\\
495	0\\
496	0\\
497	0\\
498	0\\
499	0\\
500	0\\
501	0\\
502	0\\
503	0\\
504	0\\
505	0\\
506	0\\
507	0\\
508	0\\
509	0\\
510	0\\
511	0\\
512	0\\
513	0\\
514	0\\
515	0\\
516	0\\
517	0\\
518	0\\
519	0\\
520	0\\
521	0\\
522	0\\
523	0\\
524	0\\
525	0\\
526	0\\
527	0\\
528	0\\
529	0\\
530	0\\
531	0\\
532	0\\
533	0\\
534	0\\
535	0\\
536	0\\
537	0\\
538	0\\
539	0\\
540	0\\
541	0\\
542	0\\
543	0\\
544	0\\
545	0\\
546	0\\
547	0\\
548	0\\
549	0\\
550	0\\
551	0\\
552	0\\
553	0\\
554	0\\
555	0\\
556	0\\
557	0\\
558	0\\
559	0\\
560	0\\
561	0\\
562	0\\
563	0\\
564	0\\
565	0\\
566	0\\
567	0\\
568	0\\
569	0\\
570	0\\
571	0\\
572	0\\
573	0\\
574	0\\
575	0\\
576	0\\
577	0\\
578	0\\
579	0\\
580	0\\
581	0\\
582	0\\
583	0\\
584	0\\
585	0\\
586	0\\
587	0\\
588	0\\
589	0\\
590	0\\
591	0\\
592	0\\
593	0\\
594	0\\
595	0\\
596	0\\
597	0\\
598	0\\
599	0\\
600	0\\
};
\end{axis}
\end{tikzpicture}% 
  \caption{Discrete Time w/ nFPC}
\end{subfigure}\\

\leavevmode\smash{\makebox[0pt]{\hspace{-7em}% HORIZONTAL POSITION           
  \rotatebox[origin=l]{90}{\hspace{20em}% VERTICAL POSITION
    Depth $\delta^-$}%
}}\hspace{0pt plus 1filll}\null

Time (s)

\vspace{1cm}
\begin{subfigure}{\linewidth}
  \centering
  \tikzsetnextfilename{altdeltalegend}
  \definecolor{mycolor1}{rgb}{0.00000,1.00000,0.14286}%
\definecolor{mycolor2}{rgb}{0.00000,1.00000,0.28571}%
\definecolor{mycolor3}{rgb}{0.00000,1.00000,0.42857}%
\definecolor{mycolor4}{rgb}{0.00000,1.00000,0.57143}%
\definecolor{mycolor5}{rgb}{0.00000,1.00000,0.71429}%
\definecolor{mycolor6}{rgb}{0.00000,1.00000,0.85714}%
\definecolor{mycolor7}{rgb}{0.00000,1.00000,1.00000}%
\definecolor{mycolor8}{rgb}{0.00000,0.87500,1.00000}%
\definecolor{mycolor9}{rgb}{0.00000,0.62500,1.00000}%
\definecolor{mycolor10}{rgb}{0.12500,0.00000,1.00000}%
\definecolor{mycolor11}{rgb}{0.25000,0.00000,1.00000}%
\definecolor{mycolor12}{rgb}{0.37500,0.00000,1.00000}%
\definecolor{mycolor13}{rgb}{0.50000,0.00000,1.00000}%
\definecolor{mycolor14}{rgb}{0.62500,0.00000,1.00000}%
\definecolor{mycolor15}{rgb}{0.75000,0.00000,1.00000}%
\definecolor{mycolor16}{rgb}{0.87500,0.00000,1.00000}%
\definecolor{mycolor17}{rgb}{1.00000,0.00000,1.00000}%
\definecolor{mycolor18}{rgb}{1.00000,0.00000,0.87500}%
\definecolor{mycolor19}{rgb}{1.00000,0.00000,0.62500}%
\definecolor{mycolor20}{rgb}{0.85714,0.00000,0.00000}%
\definecolor{mycolor21}{rgb}{0.71429,0.00000,0.00000}%
%[trim axis left, trim axis right]
\begin{tikzpicture}
\begin{axis}[%
    hide axis,
    scale only axis,
    height=0pt,
    width=0pt,
    point meta min=-19,
    point meta max=19,
    colormap={mymap}{[1pt] rgb(0pt)=(0,1,0); rgb(7pt)=(0,1,1); rgb(15pt)=(0,0,1); rgb(23pt)=(1,0,1); rgb(31pt)=(1,0,0); rgb(38pt)=(0,0,0)},
    colorbar horizontal,
    colorbar style={width=15cm,xtick={{-15},{-10},{-5},{0},{5},{10},{15}}}
    %colorbar style={separate axis lines,every outer x axis line/.append style={black},every x tick label/.append style={font=\color{black}},every outer y axis line/.append style={black},every y tick label/.append style={font=\color{black}},yticklabels={{-19},{-17},{-15},{-13},{-11},{-9},{-7},{-5},{-3},{-1},{1},{3},{5},{7},{9},{11},{13},{15},{17},{19}}}
]%
    \addplot [draw=none] coordinates {(0,0)};
\end{axis}
\end{tikzpicture}
 
\end{subfigure}%
  \caption{Optimal sell depths $\delta^{-}$ for Markov state $Z=(\rho = -1, \Delta S = -1)$, implying heavy imbalance in favor of sell pressure, and having previously seen a downward price change. We expect the midprice to fall.}
  \label{fig:comp_dm_z1}
\end{figure}

\begin{figure}
\centering
\begin{subfigure}{.45\linewidth}
  \centering
  \setlength\figureheight{\linewidth} 
  \setlength\figurewidth{\linewidth}
  \tikzsetnextfilename{dp_colorbar/dm_cts_z8}
  % This file was created by matlab2tikz.
%
%The latest updates can be retrieved from
%  http://www.mathworks.com/matlabcentral/fileexchange/22022-matlab2tikz-matlab2tikz
%where you can also make suggestions and rate matlab2tikz.
%
\definecolor{mycolor1}{rgb}{0.00000,1.00000,0.14286}%
\definecolor{mycolor2}{rgb}{0.00000,1.00000,0.28571}%
\definecolor{mycolor3}{rgb}{0.00000,1.00000,0.42857}%
\definecolor{mycolor4}{rgb}{0.00000,1.00000,0.57143}%
\definecolor{mycolor5}{rgb}{0.00000,1.00000,0.71429}%
\definecolor{mycolor6}{rgb}{0.00000,1.00000,0.85714}%
\definecolor{mycolor7}{rgb}{0.00000,1.00000,1.00000}%
\definecolor{mycolor8}{rgb}{0.00000,0.87500,1.00000}%
\definecolor{mycolor9}{rgb}{0.00000,0.62500,1.00000}%
\definecolor{mycolor10}{rgb}{0.12500,0.00000,1.00000}%
\definecolor{mycolor11}{rgb}{0.25000,0.00000,1.00000}%
\definecolor{mycolor12}{rgb}{0.37500,0.00000,1.00000}%
\definecolor{mycolor13}{rgb}{0.50000,0.00000,1.00000}%
\definecolor{mycolor14}{rgb}{0.62500,0.00000,1.00000}%
\definecolor{mycolor15}{rgb}{0.75000,0.00000,1.00000}%
\definecolor{mycolor16}{rgb}{0.87500,0.00000,1.00000}%
\definecolor{mycolor17}{rgb}{1.00000,0.00000,1.00000}%
\definecolor{mycolor18}{rgb}{1.00000,0.00000,0.87500}%
\definecolor{mycolor19}{rgb}{1.00000,0.00000,0.62500}%
\definecolor{mycolor20}{rgb}{0.85714,0.00000,0.00000}%
\definecolor{mycolor21}{rgb}{0.71429,0.00000,0.00000}%
%
\begin{tikzpicture}

\begin{axis}[%
width=4.1in,
height=3.803in,
at={(0.809in,0.513in)},
scale only axis,
point meta min=0,
point meta max=1,
every outer x axis line/.append style={black},
every x tick label/.append style={font=\color{black}},
xmin=0,
xmax=600,
every outer y axis line/.append style={black},
every y tick label/.append style={font=\color{black}},
ymin=0,
ymax=0.012,
axis background/.style={fill=white},
axis x line*=bottom,
axis y line*=left,
colormap={mymap}{[1pt] rgb(0pt)=(0,1,0); rgb(7pt)=(0,1,1); rgb(15pt)=(0,0,1); rgb(23pt)=(1,0,1); rgb(31pt)=(1,0,0); rgb(38pt)=(0,0,0)},
colorbar,
colorbar style={separate axis lines,every outer x axis line/.append style={black},every x tick label/.append style={font=\color{black}},every outer y axis line/.append style={black},every y tick label/.append style={font=\color{black}},yticklabels={{-19},{-17},{-15},{-13},{-11},{-9},{-7},{-5},{-3},{-1},{1},{3},{5},{7},{9},{11},{13},{15},{17},{19}}}
]
\addplot [color=green,solid,forget plot]
  table[row sep=crcr]{%
0.01	0.00503700863926798\\
1.01	0.0050370095564963\\
2.01	0.00503701049210395\\
3.01	0.0050370114464563\\
4.01	0.00503701241992682\\
5.01	0.00503701341289536\\
6.01	0.00503701442575027\\
7.01	0.00503701545888663\\
8.01	0.00503701651270766\\
9.01	0.00503701758762468\\
10.01	0.00503701868405695\\
11.01	0.00503701980243221\\
12.01	0.00503702094318642\\
13.01	0.00503702210676399\\
14.01	0.00503702329361844\\
15.01	0.00503702450421181\\
16.01	0.0050370257390156\\
17.01	0.00503702699851016\\
18.01	0.00503702828318539\\
19.01	0.00503702959354146\\
20.01	0.00503703093008762\\
21.01	0.00503703229334332\\
22.01	0.00503703368383834\\
23.01	0.00503703510211298\\
24.01	0.00503703654871751\\
25.01	0.00503703802421365\\
26.01	0.00503703952917398\\
27.01	0.00503704106418201\\
28.01	0.00503704262983323\\
29.01	0.00503704422673434\\
30.01	0.00503704585550413\\
31.01	0.00503704751677359\\
32.01	0.00503704921118597\\
33.01	0.00503705093939702\\
34.01	0.00503705270207538\\
35.01	0.00503705449990291\\
36.01	0.00503705633357451\\
37.01	0.0050370582037991\\
38.01	0.00503706011129942\\
39.01	0.00503706205681168\\
40.01	0.00503706404108728\\
41.01	0.00503706606489202\\
42.01	0.00503706812900628\\
43.01	0.00503707023422634\\
44.01	0.00503707238136358\\
45.01	0.0050370745712455\\
46.01	0.00503707680471538\\
47.01	0.00503707908263312\\
48.01	0.00503708140587569\\
49.01	0.00503708377533685\\
50.01	0.00503708619192778\\
51.01	0.00503708865657709\\
52.01	0.0050370911702326\\
53.01	0.00503709373385891\\
54.01	0.00503709634844085\\
55.01	0.00503709901498176\\
56.01	0.00503710173450458\\
57.01	0.00503710450805214\\
58.01	0.00503710733668758\\
59.01	0.00503711022149461\\
60.01	0.00503711316357836\\
61.01	0.00503711616406474\\
62.01	0.00503711922410222\\
63.01	0.00503712234486101\\
64.01	0.00503712552753433\\
65.01	0.0050371287733387\\
66.01	0.00503713208351358\\
67.01	0.00503713545932324\\
68.01	0.00503713890205601\\
69.01	0.00503714241302494\\
70.01	0.00503714599356911\\
71.01	0.00503714964505306\\
72.01	0.00503715336886795\\
73.01	0.00503715716643199\\
74.01	0.00503716103919033\\
75.01	0.00503716498861637\\
76.01	0.00503716901621196\\
77.01	0.00503717312350804\\
78.01	0.00503717731206504\\
79.01	0.00503718158347313\\
80.01	0.00503718593935376\\
81.01	0.00503719038135941\\
82.01	0.00503719491117427\\
83.01	0.00503719953051524\\
84.01	0.00503720424113244\\
85.01	0.00503720904480873\\
86.01	0.00503721394336243\\
87.01	0.00503721893864667\\
88.01	0.00503722403254947\\
89.01	0.00503722922699576\\
90.01	0.00503723452394766\\
91.01	0.00503723992540453\\
92.01	0.00503724543340422\\
93.01	0.00503725105002359\\
94.01	0.00503725677737982\\
95.01	0.00503726261763057\\
96.01	0.00503726857297443\\
97.01	0.0050372746456524\\
98.01	0.00503728083794855\\
99.01	0.00503728715219068\\
100.01	0.00503729359075072\\
101.01	0.00503730015604673\\
102.01	0.00503730685054238\\
103.01	0.00503731367674878\\
104.01	0.00503732063722473\\
105.01	0.00503732773457818\\
106.01	0.00503733497146674\\
107.01	0.00503734235059897\\
108.01	0.00503734987473498\\
109.01	0.00503735754668719\\
110.01	0.00503736536932195\\
111.01	0.00503737334556062\\
112.01	0.00503738147837934\\
113.01	0.00503738977081139\\
114.01	0.00503739822594776\\
115.01	0.00503740684693799\\
116.01	0.00503741563699175\\
117.01	0.00503742459937948\\
118.01	0.00503743373743401\\
119.01	0.00503744305455114\\
120.01	0.00503745255419133\\
121.01	0.00503746223988039\\
122.01	0.00503747211521105\\
123.01	0.00503748218384462\\
124.01	0.00503749244951107\\
125.01	0.00503750291601124\\
126.01	0.0050375135872181\\
127.01	0.00503752446707711\\
128.01	0.00503753555960904\\
129.01	0.00503754686891023\\
130.01	0.00503755839915435\\
131.01	0.00503757015459461\\
132.01	0.00503758213956329\\
133.01	0.00503759435847503\\
134.01	0.00503760681582774\\
135.01	0.00503761951620343\\
136.01	0.00503763246427077\\
137.01	0.00503764566478632\\
138.01	0.00503765912259596\\
139.01	0.00503767284263604\\
140.01	0.00503768682993702\\
141.01	0.00503770108962251\\
142.01	0.0050377156269131\\
143.01	0.00503773044712685\\
144.01	0.00503774555568146\\
145.01	0.00503776095809616\\
146.01	0.00503777665999397\\
147.01	0.00503779266710264\\
148.01	0.00503780898525674\\
149.01	0.00503782562040073\\
150.01	0.00503784257858947\\
151.01	0.00503785986599095\\
152.01	0.00503787748888843\\
153.01	0.00503789545368261\\
154.01	0.00503791376689258\\
155.01	0.00503793243515976\\
156.01	0.00503795146524905\\
157.01	0.005037970864051\\
158.01	0.00503799063858401\\
159.01	0.00503801079599758\\
160.01	0.00503803134357367\\
161.01	0.00503805228872931\\
162.01	0.00503807363901934\\
163.01	0.00503809540213851\\
164.01	0.00503811758592421\\
165.01	0.00503814019835902\\
166.01	0.00503816324757281\\
167.01	0.00503818674184675\\
168.01	0.00503821068961455\\
169.01	0.00503823509946524\\
170.01	0.00503825998014745\\
171.01	0.00503828534057066\\
172.01	0.0050383111898083\\
173.01	0.00503833753710173\\
174.01	0.00503836439186212\\
175.01	0.00503839176367404\\
176.01	0.00503841966229843\\
177.01	0.00503844809767484\\
178.01	0.00503847707992691\\
179.01	0.00503850661936314\\
180.01	0.00503853672648163\\
181.01	0.00503856741197301\\
182.01	0.00503859868672335\\
183.01	0.00503863056181852\\
184.01	0.00503866304854753\\
185.01	0.00503869615840566\\
186.01	0.00503872990309845\\
187.01	0.00503876429454511\\
188.01	0.00503879934488299\\
189.01	0.00503883506647076\\
190.01	0.00503887147189222\\
191.01	0.00503890857396075\\
192.01	0.00503894638572358\\
193.01	0.00503898492046473\\
194.01	0.00503902419171071\\
195.01	0.00503906421323363\\
196.01	0.00503910499905608\\
197.01	0.00503914656345553\\
198.01	0.00503918892096808\\
199.01	0.00503923208639412\\
200.01	0.0050392760748024\\
201.01	0.0050393209015345\\
202.01	0.00503936658221077\\
203.01	0.00503941313273334\\
204.01	0.00503946056929269\\
205.01	0.0050395089083721\\
206.01	0.00503955816675239\\
207.01	0.00503960836151857\\
208.01	0.00503965951006308\\
209.01	0.0050397116300932\\
210.01	0.00503976473963538\\
211.01	0.00503981885704116\\
212.01	0.00503987400099298\\
213.01	0.00503993019050977\\
214.01	0.00503998744495318\\
215.01	0.00504004578403339\\
216.01	0.00504010522781573\\
217.01	0.00504016579672581\\
218.01	0.00504022751155711\\
219.01	0.00504029039347662\\
220.01	0.00504035446403217\\
221.01	0.00504041974515843\\
222.01	0.00504048625918432\\
223.01	0.00504055402883963\\
224.01	0.00504062307726214\\
225.01	0.00504069342800531\\
226.01	0.005040765105045\\
227.01	0.0050408381327873\\
228.01	0.00504091253607636\\
229.01	0.00504098834020186\\
230.01	0.0050410655709073\\
231.01	0.00504114425439822\\
232.01	0.00504122441734989\\
233.01	0.00504130608691561\\
234.01	0.0050413892907368\\
235.01	0.00504147405694988\\
236.01	0.00504156041419651\\
237.01	0.00504164839163246\\
238.01	0.00504173801893631\\
239.01	0.00504182932631966\\
240.01	0.00504192234453665\\
241.01	0.00504201710489405\\
242.01	0.00504211363926055\\
243.01	0.00504221198007862\\
244.01	0.00504231216037326\\
245.01	0.00504241421376425\\
246.01	0.00504251817447671\\
247.01	0.00504262407735151\\
248.01	0.00504273195785787\\
249.01	0.00504284185210524\\
250.01	0.00504295379685413\\
251.01	0.00504306782952962\\
252.01	0.00504318398823272\\
253.01	0.00504330231175437\\
254.01	0.00504342283958797\\
255.01	0.00504354561194312\\
256.01	0.00504367066975919\\
257.01	0.00504379805471941\\
258.01	0.00504392780926499\\
259.01	0.00504405997661062\\
260.01	0.00504419460075892\\
261.01	0.00504433172651703\\
262.01	0.00504447139951025\\
263.01	0.00504461366620026\\
264.01	0.00504475857390158\\
265.01	0.0050449061707972\\
266.01	0.00504505650595788\\
267.01	0.00504520962935875\\
268.01	0.00504536559189853\\
269.01	0.00504552444541745\\
270.01	0.00504568624271717\\
271.01	0.0050458510375806\\
272.01	0.00504601888479148\\
273.01	0.00504618984015631\\
274.01	0.00504636396052505\\
275.01	0.00504654130381252\\
276.01	0.00504672192902139\\
277.01	0.00504690589626572\\
278.01	0.00504709326679303\\
279.01	0.00504728410301017\\
280.01	0.0050474784685072\\
281.01	0.00504767642808349\\
282.01	0.00504787804777293\\
283.01	0.00504808339487145\\
284.01	0.00504829253796438\\
285.01	0.00504850554695443\\
286.01	0.00504872249309057\\
287.01	0.00504894344899795\\
288.01	0.00504916848870818\\
289.01	0.00504939768769\\
290.01	0.00504963112288198\\
291.01	0.00504986887272421\\
292.01	0.00505011101719227\\
293.01	0.00505035763783143\\
294.01	0.00505060881779185\\
295.01	0.00505086464186458\\
296.01	0.00505112519651726\\
297.01	0.00505139056993306\\
298.01	0.00505166085204839\\
299.01	0.00505193613459221\\
300.01	0.00505221651112603\\
301.01	0.00505250207708521\\
302.01	0.00505279292981986\\
303.01	0.00505308916863922\\
304.01	0.00505339089485271\\
305.01	0.00505369821181673\\
306.01	0.00505401122497884\\
307.01	0.00505433004192291\\
308.01	0.00505465477241729\\
309.01	0.00505498552846184\\
310.01	0.00505532242433601\\
311.01	0.00505566557664891\\
312.01	0.00505601510438797\\
313.01	0.00505637112897089\\
314.01	0.00505673377429641\\
315.01	0.00505710316679639\\
316.01	0.00505747943548865\\
317.01	0.0050578627120301\\
318.01	0.0050582531307706\\
319.01	0.00505865082880838\\
320.01	0.00505905594604403\\
321.01	0.00505946862523698\\
322.01	0.00505988901206109\\
323.01	0.00506031725516179\\
324.01	0.0050607535062126\\
325.01	0.00506119791997345\\
326.01	0.00506165065434799\\
327.01	0.00506211187044248\\
328.01	0.00506258173262414\\
329.01	0.00506306040858084\\
330.01	0.00506354806938109\\
331.01	0.00506404488953358\\
332.01	0.00506455104704839\\
333.01	0.00506506672349805\\
334.01	0.0050655921040795\\
335.01	0.00506612737767675\\
336.01	0.00506667273692421\\
337.01	0.00506722837827064\\
338.01	0.00506779450204467\\
339.01	0.00506837131252053\\
340.01	0.00506895901798665\\
341.01	0.00506955783081389\\
342.01	0.00507016796752677\\
343.01	0.00507078964887561\\
344.01	0.00507142309991152\\
345.01	0.00507206855006312\\
346.01	0.00507272623321661\\
347.01	0.00507339638779827\\
348.01	0.00507407925685907\\
349.01	0.00507477508816512\\
350.01	0.00507548413428993\\
351.01	0.00507620665271169\\
352.01	0.00507694290591373\\
353.01	0.00507769316149151\\
354.01	0.00507845769226323\\
355.01	0.0050792367763855\\
356.01	0.00508003069747492\\
357.01	0.00508083974473519\\
358.01	0.0050816642130905\\
359.01	0.00508250440332336\\
360.01	0.00508336062221926\\
361.01	0.00508423318271767\\
362.01	0.00508512240406883\\
363.01	0.00508602861199517\\
364.01	0.00508695213885965\\
365.01	0.00508789332383966\\
366.01	0.00508885251310515\\
367.01	0.00508983006000338\\
368.01	0.00509082632524624\\
369.01	0.00509184167710397\\
370.01	0.00509287649160212\\
371.01	0.00509393115272302\\
372.01	0.00509500605261015\\
373.01	0.00509610159177717\\
374.01	0.00509721817931821\\
375.01	0.00509835623312491\\
376.01	0.00509951618010449\\
377.01	0.00510069845640224\\
378.01	0.00510190350762779\\
379.01	0.00510313178908584\\
380.01	0.0051043837660103\\
381.01	0.00510565991380521\\
382.01	0.00510696071828762\\
383.01	0.00510828667593788\\
384.01	0.00510963829415476\\
385.01	0.00511101609151532\\
386.01	0.00511242059804085\\
387.01	0.00511385235546691\\
388.01	0.00511531191752215\\
389.01	0.00511679985020833\\
390.01	0.00511831673208837\\
391.01	0.0051198631545798\\
392.01	0.00512143972225235\\
393.01	0.00512304705313218\\
394.01	0.00512468577901079\\
395.01	0.00512635654575985\\
396.01	0.00512806001365004\\
397.01	0.00512979685767662\\
398.01	0.00513156776788948\\
399.01	0.0051333734497285\\
400.01	0.00513521462436362\\
401.01	0.00513709202904053\\
402.01	0.00513900641743134\\
403.01	0.00514095855999003\\
404.01	0.0051429492443123\\
405.01	0.00514497927550131\\
406.01	0.00514704947653821\\
407.01	0.00514916068865695\\
408.01	0.0051513137717242\\
409.01	0.00515350960462505\\
410.01	0.00515574908565231\\
411.01	0.005158033132902\\
412.01	0.00516036268467382\\
413.01	0.00516273869987566\\
414.01	0.00516516215843577\\
415.01	0.00516763406171819\\
416.01	0.00517015543294619\\
417.01	0.00517272731762952\\
418.01	0.00517535078399994\\
419.01	0.0051780269234516\\
420.01	0.00518075685099006\\
421.01	0.00518354170568777\\
422.01	0.00518638265114635\\
423.01	0.00518928087596993\\
424.01	0.00519223759424342\\
425.01	0.00519525404602364\\
426.01	0.00519833149783658\\
427.01	0.00520147124318873\\
428.01	0.00520467460308546\\
429.01	0.00520794292656374\\
430.01	0.00521127759123499\\
431.01	0.00521468000384159\\
432.01	0.00521815160082545\\
433.01	0.00522169384891116\\
434.01	0.00522530824570214\\
435.01	0.00522899632029247\\
436.01	0.00523275963389283\\
437.01	0.00523659978047139\\
438.01	0.00524051838741119\\
439.01	0.0052445171161831\\
440.01	0.00524859766303384\\
441.01	0.00525276175969047\\
442.01	0.00525701117408137\\
443.01	0.00526134771107234\\
444.01	0.00526577321321897\\
445.01	0.00527028956153447\\
446.01	0.00527489867627398\\
447.01	0.00527960251773282\\
448.01	0.00528440308706105\\
449.01	0.0052893024270928\\
450.01	0.00529430262319144\\
451.01	0.00529940580410992\\
452.01	0.00530461414286562\\
453.01	0.00530992985763248\\
454.01	0.00531535521264853\\
455.01	0.00532089251913995\\
456.01	0.00532654413626477\\
457.01	0.00533231247207272\\
458.01	0.00533819998448709\\
459.01	0.00534420918230508\\
460.01	0.00535034262622169\\
461.01	0.00535660292987603\\
462.01	0.00536299276092225\\
463.01	0.00536951484212522\\
464.01	0.00537617195248202\\
465.01	0.00538296692837173\\
466.01	0.00538990266473194\\
467.01	0.00539698211626374\\
468.01	0.00540420829866611\\
469.01	0.00541158428989791\\
470.01	0.00541911323147179\\
471.01	0.00542679832977569\\
472.01	0.00543464285742695\\
473.01	0.00544265015465428\\
474.01	0.00545082363071245\\
475.01	0.00545916676532751\\
476.01	0.00546768311017262\\
477.01	0.00547637629037634\\
478.01	0.00548525000606218\\
479.01	0.00549430803392156\\
480.01	0.00550355422881765\\
481.01	0.00551299252542421\\
482.01	0.00552262693989639\\
483.01	0.005532461571577\\
484.01	0.0055425006047365\\
485.01	0.00555274831034856\\
486.01	0.00556320904790121\\
487.01	0.0055738872672441\\
488.01	0.00558478751047235\\
489.01	0.00559591441384842\\
490.01	0.00560727270975982\\
491.01	0.00561886722871592\\
492.01	0.00563070290138235\\
493.01	0.00564278476065359\\
494.01	0.00565511794376293\\
495.01	0.00566770769443155\\
496.01	0.00568055936505529\\
497.01	0.00569367841892915\\
498.01	0.00570707043250933\\
499.01	0.00572074109771233\\
500.01	0.00573469622425185\\
501.01	0.00574894174201072\\
502.01	0.0057634837034497\\
503.01	0.00577832828605053\\
504.01	0.00579348179479285\\
505.01	0.00580895066466464\\
506.01	0.00582474146320375\\
507.01	0.00584086089306883\\
508.01	0.00585731579463925\\
509.01	0.00587411314864009\\
510.01	0.00589126007879083\\
511.01	0.00590876385447447\\
512.01	0.00592663189342303\\
513.01	0.00594487176441871\\
514.01	0.00596349119000116\\
515.01	0.00598249804918133\\
516.01	0.00600190038015398\\
517.01	0.00602170638300233\\
518.01	0.00604192442238996\\
519.01	0.00606256303023013\\
520.01	0.006083630908326\\
521.01	0.0061051369309701\\
522.01	0.0061270901474948\\
523.01	0.00614949978475927\\
524.01	0.00617237524956166\\
525.01	0.0061957261309602\\
526.01	0.00621956220248883\\
527.01	0.0062438934242455\\
528.01	0.00626872994483725\\
529.01	0.00629408210315595\\
530.01	0.0063199604299614\\
531.01	0.00634637564924387\\
532.01	0.00637333867933546\\
533.01	0.00640086063373574\\
534.01	0.00642895282161518\\
535.01	0.00645762674795358\\
536.01	0.00648689411326941\\
537.01	0.00651676681288722\\
538.01	0.00654725693568947\\
539.01	0.00657837676228909\\
540.01	0.0066101387625562\\
541.01	0.00664255559242306\\
542.01	0.00667564008988554\\
543.01	0.00670940527010898\\
544.01	0.00674386431953959\\
545.01	0.00677903058890984\\
546.01	0.00681491758501747\\
547.01	0.00685153896114556\\
548.01	0.00688890850597589\\
549.01	0.00692704013083688\\
550.01	0.00696594785510919\\
551.01	0.00700564578959692\\
552.01	0.00704614811765319\\
553.01	0.00708746907382994\\
554.01	0.00712962291979972\\
555.01	0.00717262391727598\\
556.01	0.00721648629763107\\
557.01	0.00726122422788934\\
558.01	0.00730685177274028\\
559.01	0.00735338285219186\\
560.01	0.00740083119445041\\
561.01	0.00744921028358358\\
562.01	0.00749853330148946\\
563.01	0.00754881306366298\\
564.01	0.0076000619482173\\
565.01	0.00765229181758725\\
566.01	0.00770551393231232\\
567.01	0.00775973885627156\\
568.01	0.00781497635272441\\
569.01	0.00787123527049889\\
570.01	0.00792852341966897\\
571.01	0.00798684743608042\\
572.01	0.0080462126341193\\
573.01	0.00810662284718149\\
574.01	0.0081680802554003\\
575.01	0.00823058520033459\\
576.01	0.00829413598651979\\
577.01	0.00835872867006437\\
578.01	0.00842435683484077\\
579.01	0.0084910113573111\\
580.01	0.00855868016166438\\
581.01	0.00862734796776696\\
582.01	0.00869699603548486\\
583.01	0.0087676019102873\\
584.01	0.00883913917675417\\
585.01	0.00891157722877423\\
586.01	0.00898488106795022\\
587.01	0.00905901114515074\\
588.01	0.00913392326443695\\
589.01	0.00920956857395242\\
590.01	0.00928589367504629\\
591.01	0.00936284088921878\\
592.01	0.0094403487328199\\
593.01	0.00951835266226869\\
594.01	0.00959678616847739\\
595.01	0.00967558231888201\\
596.01	0.00975467586988015\\
597.01	0.00983387214752878\\
598.01	0.0099086620184807\\
599.01	0.00997087280416276\\
599.02	0.00997138072163725\\
599.03	0.00997188557625799\\
599.04	0.00997238733820211\\
599.05	0.00997288597735272\\
599.06	0.00997338146329604\\
599.07	0.00997387376531845\\
599.08	0.00997436285240349\\
599.09	0.00997484869322892\\
599.1	0.00997533125616361\\
599.11	0.00997581050926455\\
599.12	0.00997628642027372\\
599.13	0.00997675895661497\\
599.14	0.0099772280853909\\
599.15	0.00997769377337961\\
599.16	0.00997815598703157\\
599.17	0.00997861469246631\\
599.18	0.00997906985546915\\
599.19	0.00997952144148792\\
599.2	0.00997996941562957\\
599.21	0.00998041374265684\\
599.22	0.0099808543869848\\
599.23	0.00998129131267744\\
599.24	0.00998172448344418\\
599.25	0.00998215386263636\\
599.26	0.00998257941258354\\
599.27	0.00998300109279825\\
599.28	0.00998341886238981\\
599.29	0.00998383268006024\\
599.3	0.00998424250410032\\
599.31	0.00998464829238543\\
599.32	0.00998505000237151\\
599.33	0.00998544759109087\\
599.34	0.00998584101514799\\
599.35	0.00998623023071532\\
599.36	0.00998661519352895\\
599.37	0.00998699585888436\\
599.38	0.00998737218163197\\
599.39	0.00998774411617281\\
599.4	0.00998811161645403\\
599.41	0.00998847463596443\\
599.42	0.0099888331277299\\
599.43	0.00998918704430885\\
599.44	0.00998953633778757\\
599.45	0.00998988095977558\\
599.46	0.00999022086140088\\
599.47	0.00999055599330522\\
599.48	0.00999088630563925\\
599.49	0.00999121174805768\\
599.5	0.00999153226971435\\
599.51	0.0099918478192573\\
599.52	0.00999215834482374\\
599.53	0.00999246379403503\\
599.54	0.0099927641139915\\
599.55	0.00999305925126738\\
599.56	0.00999334915190553\\
599.57	0.0099936337614122\\
599.58	0.00999391302475173\\
599.59	0.00999418688634118\\
599.6	0.00999445529004489\\
599.61	0.00999471817916904\\
599.62	0.00999497549645613\\
599.63	0.00999522718407935\\
599.64	0.009995473183637\\
599.65	0.00999571343614678\\
599.66	0.00999594788204004\\
599.67	0.00999617646115598\\
599.68	0.0099963991127358\\
599.69	0.00999661577541675\\
599.7	0.00999682638722618\\
599.71	0.00999703088557551\\
599.72	0.00999722920725411\\
599.73	0.00999742128842315\\
599.74	0.00999760706460942\\
599.75	0.00999778647069897\\
599.76	0.00999795944093086\\
599.77	0.0099981259088907\\
599.78	0.0099982858075042\\
599.79	0.00999843906903064\\
599.8	0.00999858562505627\\
599.81	0.00999872540648766\\
599.82	0.00999885834354498\\
599.83	0.00999898436575515\\
599.84	0.00999910340194508\\
599.85	0.00999921538023465\\
599.86	0.00999932022802977\\
599.87	0.00999941787201528\\
599.88	0.00999950823814785\\
599.89	0.00999959125164875\\
599.9	0.00999966683699656\\
599.91	0.00999973491791987\\
599.92	0.0099997954173898\\
599.93	0.00999984825761255\\
599.94	0.00999989336002181\\
599.95	0.00999993064527112\\
599.96	0.00999996003322615\\
599.97	0.00999998144295691\\
599.98	0.00999999479272987\\
599.99	0.01\\
600	0.01\\
};
\addplot [color=mycolor1,solid,forget plot]
  table[row sep=crcr]{%
0.01	0.00503201950426191\\
1.01	0.00503202043624351\\
2.01	0.00503202138695444\\
3.01	0.00503202235676875\\
4.01	0.00503202334606778\\
5.01	0.00503202435524097\\
6.01	0.00503202538468486\\
7.01	0.00503202643480416\\
8.01	0.0050320275060117\\
9.01	0.00503202859872831\\
10.01	0.00503202971338315\\
11.01	0.00503203085041398\\
12.01	0.00503203201026672\\
13.01	0.00503203319339726\\
14.01	0.005032034400269\\
15.01	0.00503203563135604\\
16.01	0.00503203688714068\\
17.01	0.00503203816811504\\
18.01	0.00503203947478162\\
19.01	0.00503204080765182\\
20.01	0.00503204216724792\\
21.01	0.00503204355410197\\
22.01	0.00503204496875709\\
23.01	0.00503204641176672\\
24.01	0.00503204788369532\\
25.01	0.00503204938511864\\
26.01	0.00503205091662345\\
27.01	0.00503205247880859\\
28.01	0.00503205407228416\\
29.01	0.00503205569767265\\
30.01	0.00503205735560877\\
31.01	0.0050320590467394\\
32.01	0.00503206077172466\\
33.01	0.00503206253123722\\
34.01	0.00503206432596379\\
35.01	0.00503206615660365\\
36.01	0.00503206802387021\\
37.01	0.00503206992849109\\
38.01	0.0050320718712078\\
39.01	0.00503207385277674\\
40.01	0.00503207587396953\\
41.01	0.00503207793557234\\
42.01	0.0050320800383872\\
43.01	0.00503208218323159\\
44.01	0.00503208437093899\\
45.01	0.00503208660235979\\
46.01	0.00503208887836051\\
47.01	0.00503209119982492\\
48.01	0.0050320935676543\\
49.01	0.00503209598276713\\
50.01	0.00503209844610009\\
51.01	0.00503210095860906\\
52.01	0.00503210352126715\\
53.01	0.00503210613506777\\
54.01	0.00503210880102339\\
55.01	0.00503211152016583\\
56.01	0.00503211429354786\\
57.01	0.00503211712224252\\
58.01	0.00503212000734368\\
59.01	0.0050321229499671\\
60.01	0.00503212595124972\\
61.01	0.00503212901235128\\
62.01	0.00503213213445397\\
63.01	0.00503213531876315\\
64.01	0.00503213856650761\\
65.01	0.00503214187894002\\
66.01	0.00503214525733772\\
67.01	0.00503214870300306\\
68.01	0.00503215221726366\\
69.01	0.00503215580147326\\
70.01	0.00503215945701149\\
71.01	0.00503216318528557\\
72.01	0.00503216698772946\\
73.01	0.00503217086580512\\
74.01	0.00503217482100379\\
75.01	0.005032178854845\\
76.01	0.00503218296887792\\
77.01	0.00503218716468218\\
78.01	0.00503219144386768\\
79.01	0.00503219580807637\\
80.01	0.00503220025898121\\
81.01	0.00503220479828845\\
82.01	0.00503220942773705\\
83.01	0.00503221414910002\\
84.01	0.00503221896418446\\
85.01	0.00503222387483279\\
86.01	0.00503222888292315\\
87.01	0.00503223399036967\\
88.01	0.00503223919912457\\
89.01	0.00503224451117715\\
90.01	0.00503224992855547\\
91.01	0.00503225545332672\\
92.01	0.0050322610875983\\
93.01	0.00503226683351865\\
94.01	0.00503227269327697\\
95.01	0.00503227866910558\\
96.01	0.00503228476327977\\
97.01	0.00503229097811875\\
98.01	0.00503229731598653\\
99.01	0.00503230377929245\\
100.01	0.00503231037049318\\
101.01	0.00503231709209198\\
102.01	0.0050323239466408\\
103.01	0.00503233093674088\\
104.01	0.00503233806504341\\
105.01	0.00503234533425098\\
106.01	0.00503235274711802\\
107.01	0.00503236030645168\\
108.01	0.00503236801511339\\
109.01	0.00503237587602001\\
110.01	0.00503238389214397\\
111.01	0.00503239206651512\\
112.01	0.00503240040222153\\
113.01	0.00503240890241055\\
114.01	0.00503241757029019\\
115.01	0.00503242640912957\\
116.01	0.00503243542226084\\
117.01	0.00503244461308017\\
118.01	0.00503245398504823\\
119.01	0.00503246354169318\\
120.01	0.005032473286609\\
121.01	0.00503248322346042\\
122.01	0.00503249335598105\\
123.01	0.00503250368797589\\
124.01	0.00503251422332365\\
125.01	0.00503252496597636\\
126.01	0.00503253591996161\\
127.01	0.00503254708938419\\
128.01	0.00503255847842715\\
129.01	0.0050325700913533\\
130.01	0.00503258193250708\\
131.01	0.00503259400631453\\
132.01	0.00503260631728797\\
133.01	0.00503261887002382\\
134.01	0.00503263166920667\\
135.01	0.00503264471961051\\
136.01	0.00503265802610012\\
137.01	0.00503267159363238\\
138.01	0.00503268542725815\\
139.01	0.00503269953212504\\
140.01	0.00503271391347715\\
141.01	0.00503272857665937\\
142.01	0.00503274352711694\\
143.01	0.00503275877039866\\
144.01	0.00503277431215837\\
145.01	0.005032790158157\\
146.01	0.00503280631426428\\
147.01	0.00503282278646114\\
148.01	0.00503283958084161\\
149.01	0.00503285670361445\\
150.01	0.00503287416110612\\
151.01	0.00503289195976253\\
152.01	0.0050329101061502\\
153.01	0.00503292860696047\\
154.01	0.00503294746901059\\
155.01	0.00503296669924563\\
156.01	0.00503298630474189\\
157.01	0.0050330062927082\\
158.01	0.00503302667048941\\
159.01	0.0050330474455681\\
160.01	0.00503306862556732\\
161.01	0.00503309021825343\\
162.01	0.00503311223153793\\
163.01	0.00503313467348104\\
164.01	0.00503315755229414\\
165.01	0.00503318087634153\\
166.01	0.00503320465414483\\
167.01	0.00503322889438429\\
168.01	0.00503325360590273\\
169.01	0.00503327879770787\\
170.01	0.00503330447897571\\
171.01	0.0050333306590534\\
172.01	0.00503335734746257\\
173.01	0.00503338455390173\\
174.01	0.00503341228824994\\
175.01	0.00503344056057023\\
176.01	0.00503346938111257\\
177.01	0.00503349876031783\\
178.01	0.00503352870881993\\
179.01	0.00503355923745083\\
180.01	0.0050335903572428\\
181.01	0.00503362207943275\\
182.01	0.0050336544154656\\
183.01	0.00503368737699827\\
184.01	0.00503372097590274\\
185.01	0.00503375522427091\\
186.01	0.00503379013441749\\
187.01	0.00503382571888486\\
188.01	0.00503386199044611\\
189.01	0.00503389896211015\\
190.01	0.00503393664712548\\
191.01	0.00503397505898429\\
192.01	0.00503401421142635\\
193.01	0.00503405411844454\\
194.01	0.00503409479428832\\
195.01	0.00503413625346871\\
196.01	0.00503417851076279\\
197.01	0.00503422158121802\\
198.01	0.00503426548015751\\
199.01	0.00503431022318523\\
200.01	0.00503435582618919\\
201.01	0.00503440230534838\\
202.01	0.00503444967713718\\
203.01	0.00503449795832997\\
204.01	0.00503454716600721\\
205.01	0.00503459731756009\\
206.01	0.00503464843069654\\
207.01	0.00503470052344598\\
208.01	0.00503475361416586\\
209.01	0.00503480772154641\\
210.01	0.00503486286461741\\
211.01	0.00503491906275336\\
212.01	0.00503497633567945\\
213.01	0.00503503470347832\\
214.01	0.00503509418659531\\
215.01	0.0050351548058459\\
216.01	0.00503521658242043\\
217.01	0.00503527953789218\\
218.01	0.00503534369422357\\
219.01	0.00503540907377234\\
220.01	0.00503547569929882\\
221.01	0.00503554359397246\\
222.01	0.00503561278137921\\
223.01	0.00503568328552928\\
224.01	0.00503575513086363\\
225.01	0.00503582834226052\\
226.01	0.00503590294504541\\
227.01	0.00503597896499697\\
228.01	0.00503605642835475\\
229.01	0.00503613536182793\\
230.01	0.0050362157926031\\
231.01	0.00503629774835188\\
232.01	0.00503638125724035\\
233.01	0.00503646634793695\\
234.01	0.00503655304962039\\
235.01	0.00503664139199012\\
236.01	0.00503673140527389\\
237.01	0.00503682312023715\\
238.01	0.00503691656819238\\
239.01	0.00503701178100915\\
240.01	0.00503710879112251\\
241.01	0.00503720763154361\\
242.01	0.00503730833586963\\
243.01	0.00503741093829291\\
244.01	0.00503751547361275\\
245.01	0.00503762197724492\\
246.01	0.00503773048523254\\
247.01	0.00503784103425756\\
248.01	0.00503795366165097\\
249.01	0.0050380684054043\\
250.01	0.00503818530418168\\
251.01	0.00503830439733127\\
252.01	0.00503842572489746\\
253.01	0.00503854932763265\\
254.01	0.00503867524701027\\
255.01	0.00503880352523717\\
256.01	0.00503893420526672\\
257.01	0.00503906733081197\\
258.01	0.0050392029463599\\
259.01	0.00503934109718421\\
260.01	0.00503948182936034\\
261.01	0.00503962518977843\\
262.01	0.00503977122616076\\
263.01	0.00503991998707443\\
264.01	0.00504007152194746\\
265.01	0.0050402258810856\\
266.01	0.0050403831156868\\
267.01	0.0050405432778593\\
268.01	0.00504070642063711\\
269.01	0.00504087259799862\\
270.01	0.0050410418648838\\
271.01	0.00504121427721224\\
272.01	0.00504138989190245\\
273.01	0.0050415687668901\\
274.01	0.00504175096114795\\
275.01	0.00504193653470614\\
276.01	0.00504212554867273\\
277.01	0.00504231806525504\\
278.01	0.00504251414778109\\
279.01	0.00504271386072164\\
280.01	0.00504291726971426\\
281.01	0.00504312444158475\\
282.01	0.0050433354443741\\
283.01	0.00504355034736101\\
284.01	0.00504376922108885\\
285.01	0.00504399213739071\\
286.01	0.0050442191694183\\
287.01	0.00504445039166762\\
288.01	0.0050446858800087\\
289.01	0.00504492571171544\\
290.01	0.00504516996549426\\
291.01	0.00504541872151704\\
292.01	0.00504567206145204\\
293.01	0.00504593006849774\\
294.01	0.00504619282741553\\
295.01	0.00504646042456572\\
296.01	0.00504673294794333\\
297.01	0.00504701048721468\\
298.01	0.00504729313375593\\
299.01	0.00504758098069216\\
300.01	0.00504787412293708\\
301.01	0.00504817265723545\\
302.01	0.00504847668220459\\
303.01	0.00504878629837929\\
304.01	0.00504910160825628\\
305.01	0.0050494227163401\\
306.01	0.00504974972919078\\
307.01	0.00505008275547381\\
308.01	0.00505042190600799\\
309.01	0.00505076729381831\\
310.01	0.00505111903418812\\
311.01	0.0050514772447123\\
312.01	0.00505184204535378\\
313.01	0.00505221355849816\\
314.01	0.00505259190901264\\
315.01	0.00505297722430356\\
316.01	0.00505336963437772\\
317.01	0.00505376927190221\\
318.01	0.00505417627226815\\
319.01	0.00505459077365233\\
320.01	0.0050550129170826\\
321.01	0.00505544284650278\\
322.01	0.00505588070883858\\
323.01	0.00505632665406469\\
324.01	0.00505678083527231\\
325.01	0.00505724340873737\\
326.01	0.00505771453398907\\
327.01	0.00505819437387861\\
328.01	0.00505868309464988\\
329.01	0.00505918086600726\\
330.01	0.00505968786118643\\
331.01	0.00506020425702283\\
332.01	0.00506073023402123\\
333.01	0.00506126597642489\\
334.01	0.0050618116722832\\
335.01	0.00506236751351999\\
336.01	0.00506293369599944\\
337.01	0.0050635104195933\\
338.01	0.00506409788824524\\
339.01	0.00506469631003512\\
340.01	0.00506530589724219\\
341.01	0.00506592686640633\\
342.01	0.00506655943838927\\
343.01	0.00506720383843395\\
344.01	0.00506786029622303\\
345.01	0.00506852904593738\\
346.01	0.00506921032631205\\
347.01	0.00506990438069319\\
348.01	0.00507061145709439\\
349.01	0.00507133180825325\\
350.01	0.00507206569168783\\
351.01	0.0050728133697556\\
352.01	0.00507357510971221\\
353.01	0.0050743511837737\\
354.01	0.00507514186918129\\
355.01	0.0050759474482705\\
356.01	0.00507676820854269\\
357.01	0.00507760444274413\\
358.01	0.00507845644895008\\
359.01	0.00507932453065577\\
360.01	0.00508020899687583\\
361.01	0.00508111016225213\\
362.01	0.0050820283471692\\
363.01	0.0050829638778836\\
364.01	0.00508391708665953\\
365.01	0.00508488831191731\\
366.01	0.00508587789839304\\
367.01	0.00508688619730879\\
368.01	0.00508791356655353\\
369.01	0.00508896037087408\\
370.01	0.00509002698207614\\
371.01	0.00509111377923371\\
372.01	0.00509222114890623\\
373.01	0.00509334948536214\\
374.01	0.00509449919080782\\
375.01	0.00509567067562119\\
376.01	0.00509686435858675\\
377.01	0.00509808066713371\\
378.01	0.00509932003757677\\
379.01	0.00510058291535667\\
380.01	0.00510186975528346\\
381.01	0.00510318102178189\\
382.01	0.00510451718913998\\
383.01	0.00510587874176205\\
384.01	0.00510726617442703\\
385.01	0.00510867999255332\\
386.01	0.0051101207124698\\
387.01	0.00511158886169583\\
388.01	0.00511308497922627\\
389.01	0.00511460961582658\\
390.01	0.00511616333433346\\
391.01	0.00511774670996351\\
392.01	0.00511936033063096\\
393.01	0.00512100479727096\\
394.01	0.00512268072417241\\
395.01	0.00512438873931648\\
396.01	0.00512612948472518\\
397.01	0.0051279036168155\\
398.01	0.00512971180676239\\
399.01	0.00513155474086818\\
400.01	0.00513343312093995\\
401.01	0.00513534766467416\\
402.01	0.0051372991060481\\
403.01	0.00513928819571818\\
404.01	0.00514131570142495\\
405.01	0.00514338240840512\\
406.01	0.00514548911980908\\
407.01	0.00514763665712437\\
408.01	0.00514982586060625\\
409.01	0.00515205758971196\\
410.01	0.00515433272354165\\
411.01	0.00515665216128347\\
412.01	0.00515901682266326\\
413.01	0.00516142764839961\\
414.01	0.00516388560066147\\
415.01	0.00516639166353125\\
416.01	0.00516894684347027\\
417.01	0.00517155216978837\\
418.01	0.0051742086951167\\
419.01	0.00517691749588425\\
420.01	0.00517967967279609\\
421.01	0.00518249635131593\\
422.01	0.00518536868215205\\
423.01	0.00518829784174599\\
424.01	0.00519128503276489\\
425.01	0.00519433148459841\\
426.01	0.00519743845385992\\
427.01	0.00520060722489213\\
428.01	0.00520383911027941\\
429.01	0.00520713545136559\\
430.01	0.00521049761877979\\
431.01	0.0052139270129701\\
432.01	0.00521742506474721\\
433.01	0.00522099323583805\\
434.01	0.00522463301945253\\
435.01	0.00522834594086097\\
436.01	0.00523213355798921\\
437.01	0.00523599746202724\\
438.01	0.0052399392780561\\
439.01	0.0052439606656921\\
440.01	0.00524806331975258\\
441.01	0.00525224897094034\\
442.01	0.00525651938654981\\
443.01	0.00526087637119615\\
444.01	0.00526532176756633\\
445.01	0.00526985745719233\\
446.01	0.00527448536124653\\
447.01	0.00527920744135943\\
448.01	0.00528402570045861\\
449.01	0.00528894218362613\\
450.01	0.00529395897897561\\
451.01	0.00529907821854578\\
452.01	0.00530430207920979\\
453.01	0.00530963278359722\\
454.01	0.00531507260102934\\
455.01	0.00532062384846525\\
456.01	0.00532628889145623\\
457.01	0.00533207014511108\\
458.01	0.00533797007507009\\
459.01	0.00534399119848931\\
460.01	0.0053501360850358\\
461.01	0.0053564073578961\\
462.01	0.00536280769480051\\
463.01	0.00536933982906571\\
464.01	0.0053760065506588\\
465.01	0.00538281070728559\\
466.01	0.00538975520550605\\
467.01	0.00539684301187904\\
468.01	0.00540407715413795\\
469.01	0.00541146072239986\\
470.01	0.00541899687040678\\
471.01	0.00542668881680251\\
472.01	0.00543453984644367\\
473.01	0.0054425533117462\\
474.01	0.00545073263406792\\
475.01	0.00545908130512543\\
476.01	0.00546760288844794\\
477.01	0.00547630102086585\\
478.01	0.0054851794140342\\
479.01	0.00549424185599155\\
480.01	0.0055034922127532\\
481.01	0.00551293442993825\\
482.01	0.00552257253443218\\
483.01	0.00553241063608223\\
484.01	0.00554245292942729\\
485.01	0.00555270369546343\\
486.01	0.00556316730344353\\
487.01	0.00557384821271302\\
488.01	0.00558475097458304\\
489.01	0.00559588023423913\\
490.01	0.00560724073269073\\
491.01	0.00561883730875751\\
492.01	0.00563067490109745\\
493.01	0.00564275855027375\\
494.01	0.00565509340086368\\
495.01	0.00566768470360667\\
496.01	0.00568053781759384\\
497.01	0.00569365821249661\\
498.01	0.00570705147083497\\
499.01	0.00572072329028427\\
500.01	0.00573467948601941\\
501.01	0.00574892599309709\\
502.01	0.00576346886887259\\
503.01	0.00577831429545238\\
504.01	0.00579346858218174\\
505.01	0.0058089381681635\\
506.01	0.00582472962480907\\
507.01	0.00584084965842006\\
508.01	0.00585730511279649\\
509.01	0.00587410297187216\\
510.01	0.00589125036237267\\
511.01	0.00590875455649485\\
512.01	0.00592662297460268\\
513.01	0.00594486318793586\\
514.01	0.00596348292132978\\
515.01	0.0059824900559373\\
516.01	0.00600189263195081\\
517.01	0.00602169885131572\\
518.01	0.00604191708043086\\
519.01	0.00606255585282599\\
520.01	0.00608362387180911\\
521.01	0.00610513001307378\\
522.01	0.00612708332725578\\
523.01	0.00614949304242679\\
524.01	0.00617236856651236\\
525.01	0.006195719489619\\
526.01	0.00621955558625398\\
527.01	0.00624388681741989\\
528.01	0.00626872333256309\\
529.01	0.00629407547135422\\
530.01	0.00631995376527479\\
531.01	0.00634636893898359\\
532.01	0.00637333191143073\\
533.01	0.00640085379668688\\
534.01	0.00642894590444845\\
535.01	0.00645761974017953\\
536.01	0.00648688700484237\\
537.01	0.00651675959416793\\
538.01	0.00654724959740859\\
539.01	0.00657836929551299\\
540.01	0.00661013115865439\\
541.01	0.00664254784303731\\
542.01	0.0066756321869002\\
543.01	0.00670939720562297\\
544.01	0.00674385608583911\\
545.01	0.00677902217844316\\
546.01	0.00681490899037128\\
547.01	0.00685153017502185\\
548.01	0.00688889952117103\\
549.01	0.00692703094022179\\
550.01	0.00696593845161156\\
551.01	0.00700563616618453\\
552.01	0.00704613826731923\\
553.01	0.00708745898958019\\
554.01	0.00712961259464204\\
555.01	0.00717261334421132\\
556.01	0.0072164754696476\\
557.01	0.00726121313795847\\
558.01	0.00730684041381582\\
559.01	0.0073533712172122\\
560.01	0.00740081927634337\\
561.01	0.00744919807527555\\
562.01	0.00749852079591776\\
563.01	0.00754880025379286\\
564.01	0.00760004882706339\\
565.01	0.00765227837824005\\
566.01	0.0077055001679697\\
567.01	0.00775972476027739\\
568.01	0.00781496191861332\\
569.01	0.00787122049204771\\
570.01	0.00792850829095721\\
571.01	0.00798683195155923\\
572.01	0.00804619678868979\\
573.01	0.00810660663628399\\
574.01	0.00816806367511503\\
575.01	0.0082305682474943\\
576.01	0.00829411865883634\\
577.01	0.00835871096626884\\
578.01	0.00842433875483839\\
579.01	0.00849099290235135\\
580.01	0.00855866133452602\\
581.01	0.008627328772957\\
582.01	0.00869697647944997\\
583.01	0.00876758200163454\\
584.01	0.00883911892647557\\
585.01	0.00891155665046791\\
586.01	0.0089848601780259\\
587.01	0.00905898996300263\\
588.01	0.00913390181256045\\
589.01	0.00920954687797058\\
590.01	0.00928587176359865\\
591.01	0.0093628187936477\\
592.01	0.00944032648656475\\
593.01	0.00951833029984481\\
594.01	0.00959676372387219\\
595.01	0.00967555982313978\\
596.01	0.00975465334756447\\
597.01	0.00983385942577576\\
598.01	0.00990866201848071\\
599.01	0.00997087280416276\\
599.02	0.00997138072163725\\
599.03	0.009971885576258\\
599.04	0.00997238733820211\\
599.05	0.00997288597735272\\
599.06	0.00997338146329604\\
599.07	0.00997387376531845\\
599.08	0.00997436285240349\\
599.09	0.00997484869322892\\
599.1	0.00997533125616361\\
599.11	0.00997581050926455\\
599.12	0.00997628642027372\\
599.13	0.00997675895661498\\
599.14	0.0099772280853909\\
599.15	0.00997769377337961\\
599.16	0.00997815598703157\\
599.17	0.00997861469246631\\
599.18	0.00997906985546915\\
599.19	0.00997952144148792\\
599.2	0.00997996941562957\\
599.21	0.00998041374265684\\
599.22	0.0099808543869848\\
599.23	0.00998129131267744\\
599.24	0.00998172448344418\\
599.25	0.00998215386263636\\
599.26	0.00998257941258353\\
599.27	0.00998300109279825\\
599.28	0.00998341886238981\\
599.29	0.00998383268006024\\
599.3	0.00998424250410032\\
599.31	0.00998464829238543\\
599.32	0.00998505000237151\\
599.33	0.00998544759109087\\
599.34	0.00998584101514799\\
599.35	0.00998623023071532\\
599.36	0.00998661519352896\\
599.37	0.00998699585888436\\
599.38	0.00998737218163197\\
599.39	0.00998774411617281\\
599.4	0.00998811161645403\\
599.41	0.00998847463596443\\
599.42	0.0099888331277299\\
599.43	0.00998918704430885\\
599.44	0.00998953633778757\\
599.45	0.00998988095977558\\
599.46	0.00999022086140088\\
599.47	0.00999055599330522\\
599.48	0.00999088630563925\\
599.49	0.00999121174805768\\
599.5	0.00999153226971435\\
599.51	0.0099918478192573\\
599.52	0.00999215834482374\\
599.53	0.00999246379403503\\
599.54	0.0099927641139915\\
599.55	0.00999305925126738\\
599.56	0.00999334915190553\\
599.57	0.0099936337614122\\
599.58	0.00999391302475173\\
599.59	0.00999418688634118\\
599.6	0.00999445529004489\\
599.61	0.00999471817916904\\
599.62	0.00999497549645613\\
599.63	0.00999522718407934\\
599.64	0.009995473183637\\
599.65	0.00999571343614678\\
599.66	0.00999594788204004\\
599.67	0.00999617646115599\\
599.68	0.0099963991127358\\
599.69	0.00999661577541675\\
599.7	0.00999682638722618\\
599.71	0.00999703088557551\\
599.72	0.00999722920725411\\
599.73	0.00999742128842315\\
599.74	0.00999760706460942\\
599.75	0.00999778647069897\\
599.76	0.00999795944093086\\
599.77	0.0099981259088907\\
599.78	0.0099982858075042\\
599.79	0.00999843906903064\\
599.8	0.00999858562505627\\
599.81	0.00999872540648767\\
599.82	0.00999885834354498\\
599.83	0.00999898436575515\\
599.84	0.00999910340194508\\
599.85	0.00999921538023465\\
599.86	0.00999932022802977\\
599.87	0.00999941787201528\\
599.88	0.00999950823814785\\
599.89	0.00999959125164875\\
599.9	0.00999966683699656\\
599.91	0.00999973491791987\\
599.92	0.0099997954173898\\
599.93	0.00999984825761255\\
599.94	0.00999989336002181\\
599.95	0.00999993064527112\\
599.96	0.00999996003322615\\
599.97	0.00999998144295691\\
599.98	0.00999999479272987\\
599.99	0.01\\
600	0.01\\
};
\addplot [color=mycolor2,solid,forget plot]
  table[row sep=crcr]{%
0.01	0.00502120505551015\\
1.01	0.0050212060149824\\
2.01	0.00502120699383299\\
3.01	0.00502120799245165\\
4.01	0.00502120901123571\\
5.01	0.00502121005059042\\
6.01	0.00502121111092924\\
7.01	0.00502121219267379\\
8.01	0.00502121329625405\\
9.01	0.00502121442210887\\
10.01	0.00502121557068572\\
11.01	0.00502121674244047\\
12.01	0.00502121793783911\\
13.01	0.0050212191573559\\
14.01	0.00502122040147554\\
15.01	0.00502122167069138\\
16.01	0.00502122296550742\\
17.01	0.00502122428643781\\
18.01	0.00502122563400595\\
19.01	0.0050212270087466\\
20.01	0.00502122841120478\\
21.01	0.00502122984193676\\
22.01	0.00502123130150919\\
23.01	0.00502123279050046\\
24.01	0.00502123430950069\\
25.01	0.00502123585911135\\
26.01	0.00502123743994609\\
27.01	0.00502123905263077\\
28.01	0.00502124069780355\\
29.01	0.00502124237611543\\
30.01	0.00502124408823032\\
31.01	0.00502124583482541\\
32.01	0.00502124761659137\\
33.01	0.00502124943423254\\
34.01	0.0050212512884672\\
35.01	0.00502125318002777\\
36.01	0.00502125510966178\\
37.01	0.0050212570781312\\
38.01	0.00502125908621306\\
39.01	0.00502126113470028\\
40.01	0.00502126322440086\\
41.01	0.00502126535613939\\
42.01	0.00502126753075657\\
43.01	0.00502126974910991\\
44.01	0.00502127201207395\\
45.01	0.00502127432054039\\
46.01	0.00502127667541896\\
47.01	0.00502127907763728\\
48.01	0.00502128152814088\\
49.01	0.0050212840278949\\
50.01	0.00502128657788314\\
51.01	0.00502128917910871\\
52.01	0.00502129183259509\\
53.01	0.0050212945393856\\
54.01	0.00502129730054438\\
55.01	0.00502130011715682\\
56.01	0.00502130299032957\\
57.01	0.00502130592119115\\
58.01	0.00502130891089264\\
59.01	0.00502131196060768\\
60.01	0.00502131507153325\\
61.01	0.00502131824489013\\
62.01	0.00502132148192286\\
63.01	0.00502132478390088\\
64.01	0.00502132815211859\\
65.01	0.00502133158789611\\
66.01	0.00502133509257949\\
67.01	0.00502133866754131\\
68.01	0.00502134231418158\\
69.01	0.00502134603392737\\
70.01	0.0050213498282346\\
71.01	0.00502135369858719\\
72.01	0.00502135764649908\\
73.01	0.00502136167351383\\
74.01	0.00502136578120502\\
75.01	0.00502136997117758\\
76.01	0.00502137424506807\\
77.01	0.00502137860454521\\
78.01	0.00502138305131081\\
79.01	0.00502138758709988\\
80.01	0.00502139221368189\\
81.01	0.00502139693286069\\
82.01	0.00502140174647622\\
83.01	0.0050214066564041\\
84.01	0.00502141166455702\\
85.01	0.00502141677288556\\
86.01	0.00502142198337838\\
87.01	0.00502142729806312\\
88.01	0.00502143271900729\\
89.01	0.00502143824831927\\
90.01	0.00502144388814851\\
91.01	0.00502144964068685\\
92.01	0.00502145550816918\\
93.01	0.00502146149287397\\
94.01	0.00502146759712493\\
95.01	0.00502147382329066\\
96.01	0.00502148017378673\\
97.01	0.00502148665107566\\
98.01	0.00502149325766843\\
99.01	0.00502149999612531\\
100.01	0.00502150686905647\\
101.01	0.0050215138791232\\
102.01	0.00502152102903899\\
103.01	0.00502152832157038\\
104.01	0.00502153575953804\\
105.01	0.00502154334581734\\
106.01	0.00502155108334049\\
107.01	0.00502155897509676\\
108.01	0.00502156702413392\\
109.01	0.00502157523355901\\
110.01	0.00502158360653984\\
111.01	0.00502159214630588\\
112.01	0.00502160085614986\\
113.01	0.00502160973942875\\
114.01	0.00502161879956461\\
115.01	0.00502162804004638\\
116.01	0.0050216374644311\\
117.01	0.00502164707634492\\
118.01	0.00502165687948466\\
119.01	0.00502166687761887\\
120.01	0.00502167707459003\\
121.01	0.00502168747431439\\
122.01	0.00502169808078512\\
123.01	0.00502170889807277\\
124.01	0.00502171993032674\\
125.01	0.00502173118177726\\
126.01	0.00502174265673663\\
127.01	0.00502175435960078\\
128.01	0.00502176629485096\\
129.01	0.00502177846705515\\
130.01	0.00502179088086999\\
131.01	0.0050218035410424\\
132.01	0.00502181645241076\\
133.01	0.00502182961990796\\
134.01	0.00502184304856201\\
135.01	0.00502185674349784\\
136.01	0.00502187070993954\\
137.01	0.00502188495321224\\
138.01	0.00502189947874446\\
139.01	0.00502191429206877\\
140.01	0.00502192939882506\\
141.01	0.00502194480476151\\
142.01	0.00502196051573778\\
143.01	0.0050219765377259\\
144.01	0.00502199287681351\\
145.01	0.00502200953920504\\
146.01	0.00502202653122452\\
147.01	0.00502204385931768\\
148.01	0.00502206153005422\\
149.01	0.00502207955013015\\
150.01	0.00502209792636996\\
151.01	0.0050221166657293\\
152.01	0.0050221357752977\\
153.01	0.00502215526229993\\
154.01	0.00502217513409978\\
155.01	0.00502219539820223\\
156.01	0.00502221606225553\\
157.01	0.00502223713405503\\
158.01	0.0050222586215449\\
159.01	0.00502228053282122\\
160.01	0.00502230287613462\\
161.01	0.00502232565989372\\
162.01	0.00502234889266762\\
163.01	0.0050223725831888\\
164.01	0.00502239674035607\\
165.01	0.00502242137323838\\
166.01	0.005022446491077\\
167.01	0.00502247210328907\\
168.01	0.00502249821947123\\
169.01	0.00502252484940274\\
170.01	0.00502255200304808\\
171.01	0.00502257969056166\\
172.01	0.00502260792228945\\
173.01	0.00502263670877492\\
174.01	0.00502266606076066\\
175.01	0.00502269598919311\\
176.01	0.00502272650522542\\
177.01	0.00502275762022222\\
178.01	0.00502278934576256\\
179.01	0.00502282169364454\\
180.01	0.00502285467588862\\
181.01	0.00502288830474204\\
182.01	0.00502292259268314\\
183.01	0.00502295755242499\\
184.01	0.00502299319692021\\
185.01	0.00502302953936454\\
186.01	0.00502306659320215\\
187.01	0.00502310437212936\\
188.01	0.00502314289009939\\
189.01	0.00502318216132727\\
190.01	0.00502322220029433\\
191.01	0.00502326302175243\\
192.01	0.00502330464072964\\
193.01	0.00502334707253499\\
194.01	0.00502339033276278\\
195.01	0.00502343443729858\\
196.01	0.00502347940232345\\
197.01	0.00502352524432037\\
198.01	0.00502357198007852\\
199.01	0.00502361962669878\\
200.01	0.00502366820160027\\
201.01	0.0050237177225248\\
202.01	0.00502376820754208\\
203.01	0.00502381967505755\\
204.01	0.00502387214381636\\
205.01	0.00502392563291001\\
206.01	0.00502398016178238\\
207.01	0.00502403575023566\\
208.01	0.00502409241843702\\
209.01	0.00502415018692418\\
210.01	0.00502420907661233\\
211.01	0.00502426910880058\\
212.01	0.00502433030517863\\
213.01	0.00502439268783314\\
214.01	0.00502445627925474\\
215.01	0.005024521102345\\
216.01	0.00502458718042365\\
217.01	0.00502465453723511\\
218.01	0.00502472319695607\\
219.01	0.00502479318420265\\
220.01	0.00502486452403789\\
221.01	0.00502493724197978\\
222.01	0.0050250113640079\\
223.01	0.00502508691657197\\
224.01	0.00502516392659885\\
225.01	0.0050252424215021\\
226.01	0.00502532242918824\\
227.01	0.00502540397806611\\
228.01	0.00502548709705464\\
229.01	0.00502557181559166\\
230.01	0.00502565816364192\\
231.01	0.00502574617170654\\
232.01	0.005025835870831\\
233.01	0.00502592729261411\\
234.01	0.00502602046921832\\
235.01	0.0050261154333765\\
236.01	0.00502621221840324\\
237.01	0.00502631085820338\\
238.01	0.00502641138728205\\
239.01	0.00502651384075328\\
240.01	0.00502661825435088\\
241.01	0.00502672466443788\\
242.01	0.00502683310801658\\
243.01	0.00502694362273876\\
244.01	0.0050270562469161\\
245.01	0.00502717101953057\\
246.01	0.00502728798024471\\
247.01	0.00502740716941269\\
248.01	0.0050275286280913\\
249.01	0.005027652398051\\
250.01	0.00502777852178654\\
251.01	0.00502790704252892\\
252.01	0.00502803800425672\\
253.01	0.00502817145170778\\
254.01	0.00502830743039084\\
255.01	0.00502844598659744\\
256.01	0.00502858716741425\\
257.01	0.00502873102073589\\
258.01	0.00502887759527657\\
259.01	0.00502902694058344\\
260.01	0.005029179107049\\
261.01	0.00502933414592505\\
262.01	0.0050294921093347\\
263.01	0.00502965305028719\\
264.01	0.00502981702269076\\
265.01	0.0050299840813669\\
266.01	0.00503015428206463\\
267.01	0.00503032768147485\\
268.01	0.00503050433724536\\
269.01	0.00503068430799553\\
270.01	0.00503086765333136\\
271.01	0.00503105443386162\\
272.01	0.00503124471121294\\
273.01	0.00503143854804688\\
274.01	0.00503163600807607\\
275.01	0.00503183715608083\\
276.01	0.00503204205792718\\
277.01	0.00503225078058336\\
278.01	0.00503246339213911\\
279.01	0.00503267996182363\\
280.01	0.00503290056002414\\
281.01	0.00503312525830624\\
282.01	0.00503335412943298\\
283.01	0.0050335872473862\\
284.01	0.00503382468738714\\
285.01	0.00503406652591845\\
286.01	0.00503431284074593\\
287.01	0.00503456371094249\\
288.01	0.00503481921691118\\
289.01	0.00503507944040973\\
290.01	0.00503534446457649\\
291.01	0.00503561437395558\\
292.01	0.00503588925452533\\
293.01	0.00503616919372432\\
294.01	0.00503645428048247\\
295.01	0.00503674460524972\\
296.01	0.00503704026002744\\
297.01	0.00503734133840114\\
298.01	0.00503764793557371\\
299.01	0.00503796014839986\\
300.01	0.00503827807542342\\
301.01	0.00503860181691385\\
302.01	0.00503893147490631\\
303.01	0.00503926715324135\\
304.01	0.00503960895760799\\
305.01	0.00503995699558779\\
306.01	0.0050403113767003\\
307.01	0.00504067221245086\\
308.01	0.00504103961638069\\
309.01	0.00504141370411773\\
310.01	0.00504179459343115\\
311.01	0.00504218240428678\\
312.01	0.00504257725890531\\
313.01	0.00504297928182318\\
314.01	0.00504338859995435\\
315.01	0.00504380534265644\\
316.01	0.00504422964179752\\
317.01	0.00504466163182693\\
318.01	0.00504510144984702\\
319.01	0.00504554923568865\\
320.01	0.00504600513198888\\
321.01	0.00504646928427072\\
322.01	0.00504694184102603\\
323.01	0.00504742295380053\\
324.01	0.00504791277728073\\
325.01	0.00504841146938398\\
326.01	0.00504891919135017\\
327.01	0.00504943610783593\\
328.01	0.00504996238700892\\
329.01	0.00505049820064668\\
330.01	0.00505104372423423\\
331.01	0.00505159913706429\\
332.01	0.00505216462233776\\
333.01	0.0050527403672645\\
334.01	0.00505332656316482\\
335.01	0.00505392340556957\\
336.01	0.00505453109432047\\
337.01	0.00505514983366831\\
338.01	0.00505577983236974\\
339.01	0.00505642130378112\\
340.01	0.00505707446594919\\
341.01	0.00505773954169936\\
342.01	0.00505841675871721\\
343.01	0.00505910634962739\\
344.01	0.005059808552065\\
345.01	0.0050605236087411\\
346.01	0.00506125176750191\\
347.01	0.00506199328137963\\
348.01	0.00506274840863571\\
349.01	0.00506351741279459\\
350.01	0.00506430056267015\\
351.01	0.00506509813238215\\
352.01	0.00506591040136409\\
353.01	0.00506673765436222\\
354.01	0.00506758018142595\\
355.01	0.00506843827789061\\
356.01	0.00506931224435403\\
357.01	0.00507020238664743\\
358.01	0.00507110901580171\\
359.01	0.00507203244801305\\
360.01	0.00507297300460754\\
361.01	0.0050739310120094\\
362.01	0.00507490680171485\\
363.01	0.00507590071027472\\
364.01	0.00507691307928929\\
365.01	0.00507794425541877\\
366.01	0.00507899459041269\\
367.01	0.00508006444116119\\
368.01	0.00508115416977156\\
369.01	0.00508226414367324\\
370.01	0.00508339473575106\\
371.01	0.00508454632450858\\
372.01	0.00508571929426307\\
373.01	0.00508691403536712\\
374.01	0.00508813094445901\\
375.01	0.00508937042473226\\
376.01	0.00509063288622533\\
377.01	0.00509191874612255\\
378.01	0.00509322842905835\\
379.01	0.00509456236742253\\
380.01	0.0050959210016579\\
381.01	0.00509730478054768\\
382.01	0.00509871416149152\\
383.01	0.00510014961076941\\
384.01	0.00510161160379846\\
385.01	0.00510310062538774\\
386.01	0.00510461716999747\\
387.01	0.00510616174200523\\
388.01	0.00510773485598404\\
389.01	0.0051093370369904\\
390.01	0.00511096882086284\\
391.01	0.00511263075453339\\
392.01	0.00511432339634935\\
393.01	0.00511604731640772\\
394.01	0.00511780309690188\\
395.01	0.0051195913324808\\
396.01	0.00512141263062116\\
397.01	0.00512326761201211\\
398.01	0.00512515691095339\\
399.01	0.0051270811757666\\
400.01	0.00512904106921996\\
401.01	0.00513103726896684\\
402.01	0.00513307046799628\\
403.01	0.00513514137509836\\
404.01	0.00513725071534207\\
405.01	0.00513939923056583\\
406.01	0.00514158767988127\\
407.01	0.00514381684018901\\
408.01	0.00514608750670632\\
409.01	0.00514840049350604\\
410.01	0.00515075663406668\\
411.01	0.00515315678183237\\
412.01	0.00515560181078219\\
413.01	0.00515809261600762\\
414.01	0.00516063011429847\\
415.01	0.00516321524473404\\
416.01	0.00516584896928096\\
417.01	0.00516853227339409\\
418.01	0.00517126616662132\\
419.01	0.0051740516832099\\
420.01	0.00517688988271299\\
421.01	0.00517978185059545\\
422.01	0.0051827286988378\\
423.01	0.00518573156653616\\
424.01	0.00518879162049825\\
425.01	0.0051919100558333\\
426.01	0.00519508809653527\\
427.01	0.00519832699605861\\
428.01	0.005201628037886\\
429.01	0.00520499253608783\\
430.01	0.0052084218358726\\
431.01	0.00521191731412986\\
432.01	0.00521548037996471\\
433.01	0.00521911247522648\\
434.01	0.00522281507503175\\
435.01	0.00522658968828514\\
436.01	0.005230437858198\\
437.01	0.00523436116281045\\
438.01	0.00523836121551853\\
439.01	0.0052424396656111\\
440.01	0.00524659819881913\\
441.01	0.00525083853788445\\
442.01	0.00525516244315059\\
443.01	0.00525957171318059\\
444.01	0.00526406818540756\\
445.01	0.0052686537368216\\
446.01	0.0052733302846973\\
447.01	0.00527809978736513\\
448.01	0.00528296424502851\\
449.01	0.00528792570062933\\
450.01	0.00529298624076026\\
451.01	0.00529814799662302\\
452.01	0.00530341314502969\\
453.01	0.0053087839094425\\
454.01	0.00531426256104464\\
455.01	0.00531985141983556\\
456.01	0.00532555285574184\\
457.01	0.0053313692897333\\
458.01	0.00533730319493714\\
459.01	0.00534335709773871\\
460.01	0.00534953357886434\\
461.01	0.00535583527443989\\
462.01	0.00536226487702372\\
463.01	0.00536882513661484\\
464.01	0.00537551886164229\\
465.01	0.00538234891994208\\
466.01	0.0053893182397339\\
467.01	0.0053964298106058\\
468.01	0.00540368668452083\\
469.01	0.00541109197685086\\
470.01	0.0054186488674464\\
471.01	0.00542636060174487\\
472.01	0.00543423049192164\\
473.01	0.00544226191808676\\
474.01	0.00545045832952867\\
475.01	0.00545882324600859\\
476.01	0.00546736025910498\\
477.01	0.00547607303360939\\
478.01	0.00548496530897408\\
479.01	0.00549404090080845\\
480.01	0.00550330370242497\\
481.01	0.00551275768643106\\
482.01	0.00552240690636421\\
483.01	0.00553225549836771\\
484.01	0.00554230768290534\\
485.01	0.00555256776651032\\
486.01	0.00556304014356669\\
487.01	0.00557372929812386\\
488.01	0.00558463980574016\\
489.01	0.00559577633535974\\
490.01	0.00560714365122079\\
491.01	0.00561874661479977\\
492.01	0.00563059018679275\\
493.01	0.00564267942913979\\
494.01	0.00565501950709213\\
495.01	0.00566761569132816\\
496.01	0.00568047336011608\\
497.01	0.00569359800152755\\
498.01	0.00570699521569809\\
499.01	0.00572067071713506\\
500.01	0.0057346303370701\\
501.01	0.00574888002585278\\
502.01	0.00576342585538613\\
503.01	0.00577827402159808\\
504.01	0.00579343084694856\\
505.01	0.00580890278297078\\
506.01	0.00582469641284311\\
507.01	0.0058408184539908\\
508.01	0.00585727576071554\\
509.01	0.00587407532685036\\
510.01	0.00589122428843934\\
511.01	0.00590872992643619\\
512.01	0.00592659966942349\\
513.01	0.0059448410963448\\
514.01	0.00596346193924731\\
515.01	0.00598247008602941\\
516.01	0.00600187358318675\\
517.01	0.00602168063855298\\
518.01	0.0060418996240226\\
519.01	0.00606253907825346\\
520.01	0.00608360770933741\\
521.01	0.00610511439742895\\
522.01	0.00612706819732234\\
523.01	0.00614947834096546\\
524.01	0.00617235423989611\\
525.01	0.00619570548758692\\
526.01	0.00621954186168196\\
527.01	0.0062438733261071\\
528.01	0.00626871003303373\\
529.01	0.00629406232467292\\
530.01	0.00631994073487588\\
531.01	0.00634635599051178\\
532.01	0.00637331901259404\\
533.01	0.00640084091711891\\
534.01	0.00642893301558123\\
535.01	0.00645760681512446\\
536.01	0.00648687401827952\\
537.01	0.00651674652224228\\
538.01	0.00654723641763366\\
539.01	0.00657835598668036\\
540.01	0.00661011770074879\\
541.01	0.00664253421715752\\
542.01	0.00667561837518443\\
543.01	0.00670938319117929\\
544.01	0.00674384185268058\\
545.01	0.00677900771142645\\
546.01	0.00681489427513986\\
547.01	0.00685151519795303\\
548.01	0.00688888426932745\\
549.01	0.00692701540130676\\
550.01	0.00696592261392825\\
551.01	0.00700562001860003\\
552.01	0.00704612179923281\\
553.01	0.00708744219089684\\
554.01	0.00712959545574979\\
555.01	0.00717259585596489\\
556.01	0.00721645762335742\\
557.01	0.00726119492538552\\
558.01	0.00730682182717329\\
559.01	0.00735335224917305\\
560.01	0.00740079992005648\\
561.01	0.0074491783243876\\
562.01	0.0074985006446055\\
563.01	0.00754877969680245\\
564.01	0.00760002785975936\\
565.01	0.00765225699666378\\
566.01	0.00770547836890862\\
567.01	0.00775970254134385\\
568.01	0.00781493927833469\\
569.01	0.00787119742996866\\
570.01	0.00792848480775271\\
571.01	0.00798680804915989\\
572.01	0.00804617247041955\\
573.01	0.00810658190700895\\
574.01	0.00816803854140444\\
575.01	0.00823054271779196\\
576.01	0.00829409274364186\\
577.01	0.00835868467832643\\
578.01	0.00842431210933126\\
579.01	0.00849096591709788\\
580.01	0.0085586340301731\\
581.01	0.00862730117316442\\
582.01	0.00869694861105789\\
583.01	0.0087675538948034\\
584.01	0.00883909061478503\\
585.01	0.00891152817095808\\
586.01	0.00898483157115984\\
587.01	0.00905896127252333\\
588.01	0.00913387308520954\\
589.01	0.00920951816302855\\
590.01	0.00928584311219743\\
591.01	0.00936279025779631\\
592.01	0.00944029811781628\\
593.01	0.00951830214751989\\
594.01	0.00959673583273945\\
595.01	0.00967553223043691\\
596.01	0.00975462607922802\\
597.01	0.00983384409165019\\
598.01	0.00990866201848071\\
599.01	0.00997087280416276\\
599.02	0.00997138072163725\\
599.03	0.00997188557625799\\
599.04	0.00997238733820211\\
599.05	0.00997288597735272\\
599.06	0.00997338146329604\\
599.07	0.00997387376531845\\
599.08	0.00997436285240349\\
599.09	0.00997484869322892\\
599.1	0.00997533125616361\\
599.11	0.00997581050926455\\
599.12	0.00997628642027372\\
599.13	0.00997675895661497\\
599.14	0.0099772280853909\\
599.15	0.00997769377337961\\
599.16	0.00997815598703157\\
599.17	0.00997861469246631\\
599.18	0.00997906985546915\\
599.19	0.00997952144148792\\
599.2	0.00997996941562957\\
599.21	0.00998041374265684\\
599.22	0.0099808543869848\\
599.23	0.00998129131267744\\
599.24	0.00998172448344418\\
599.25	0.00998215386263636\\
599.26	0.00998257941258354\\
599.27	0.00998300109279825\\
599.28	0.00998341886238981\\
599.29	0.00998383268006024\\
599.3	0.00998424250410032\\
599.31	0.00998464829238543\\
599.32	0.00998505000237151\\
599.33	0.00998544759109087\\
599.34	0.00998584101514799\\
599.35	0.00998623023071532\\
599.36	0.00998661519352895\\
599.37	0.00998699585888436\\
599.38	0.00998737218163197\\
599.39	0.00998774411617281\\
599.4	0.00998811161645403\\
599.41	0.00998847463596443\\
599.42	0.0099888331277299\\
599.43	0.00998918704430885\\
599.44	0.00998953633778757\\
599.45	0.00998988095977558\\
599.46	0.00999022086140088\\
599.47	0.00999055599330522\\
599.48	0.00999088630563925\\
599.49	0.00999121174805768\\
599.5	0.00999153226971435\\
599.51	0.0099918478192573\\
599.52	0.00999215834482374\\
599.53	0.00999246379403503\\
599.54	0.0099927641139915\\
599.55	0.00999305925126738\\
599.56	0.00999334915190553\\
599.57	0.0099936337614122\\
599.58	0.00999391302475173\\
599.59	0.00999418688634118\\
599.6	0.00999445529004489\\
599.61	0.00999471817916904\\
599.62	0.00999497549645613\\
599.63	0.00999522718407935\\
599.64	0.009995473183637\\
599.65	0.00999571343614678\\
599.66	0.00999594788204004\\
599.67	0.00999617646115598\\
599.68	0.0099963991127358\\
599.69	0.00999661577541675\\
599.7	0.00999682638722618\\
599.71	0.00999703088557551\\
599.72	0.00999722920725411\\
599.73	0.00999742128842315\\
599.74	0.00999760706460942\\
599.75	0.00999778647069897\\
599.76	0.00999795944093086\\
599.77	0.0099981259088907\\
599.78	0.0099982858075042\\
599.79	0.00999843906903064\\
599.8	0.00999858562505627\\
599.81	0.00999872540648767\\
599.82	0.00999885834354498\\
599.83	0.00999898436575515\\
599.84	0.00999910340194508\\
599.85	0.00999921538023465\\
599.86	0.00999932022802977\\
599.87	0.00999941787201528\\
599.88	0.00999950823814785\\
599.89	0.00999959125164875\\
599.9	0.00999966683699656\\
599.91	0.00999973491791987\\
599.92	0.0099997954173898\\
599.93	0.00999984825761255\\
599.94	0.00999989336002181\\
599.95	0.00999993064527112\\
599.96	0.00999996003322615\\
599.97	0.00999998144295691\\
599.98	0.00999999479272987\\
599.99	0.01\\
600	0.01\\
};
\addplot [color=mycolor3,solid,forget plot]
  table[row sep=crcr]{%
0.01	0.0050016280518921\\
1.01	0.00500162904994192\\
2.01	0.00500163006827589\\
3.01	0.00500163110730512\\
4.01	0.00500163216744873\\
5.01	0.00500163324913459\\
6.01	0.00500163435279902\\
7.01	0.00500163547888721\\
8.01	0.00500163662785354\\
9.01	0.00500163780016109\\
10.01	0.00500163899628275\\
11.01	0.00500164021670073\\
12.01	0.0050016414619069\\
13.01	0.00500164273240315\\
14.01	0.00500164402870142\\
15.01	0.00500164535132418\\
16.01	0.00500164670080426\\
17.01	0.00500164807768509\\
18.01	0.00500164948252126\\
19.01	0.00500165091587854\\
20.01	0.00500165237833407\\
21.01	0.00500165387047644\\
22.01	0.00500165539290649\\
23.01	0.00500165694623676\\
24.01	0.00500165853109226\\
25.01	0.00500166014811082\\
26.01	0.00500166179794266\\
27.01	0.00500166348125161\\
28.01	0.0050016651987143\\
29.01	0.00500166695102154\\
30.01	0.00500166873887767\\
31.01	0.0050016705630014\\
32.01	0.00500167242412572\\
33.01	0.00500167432299854\\
34.01	0.0050016762603827\\
35.01	0.00500167823705644\\
36.01	0.00500168025381365\\
37.01	0.00500168231146401\\
38.01	0.00500168441083404\\
39.01	0.00500168655276619\\
40.01	0.00500168873812026\\
41.01	0.00500169096777317\\
42.01	0.00500169324261952\\
43.01	0.00500169556357194\\
44.01	0.00500169793156127\\
45.01	0.00500170034753701\\
46.01	0.00500170281246805\\
47.01	0.00500170532734243\\
48.01	0.00500170789316817\\
49.01	0.00500171051097331\\
50.01	0.00500171318180696\\
51.01	0.00500171590673891\\
52.01	0.00500171868686054\\
53.01	0.00500172152328528\\
54.01	0.00500172441714855\\
55.01	0.00500172736960895\\
56.01	0.00500173038184804\\
57.01	0.00500173345507127\\
58.01	0.00500173659050805\\
59.01	0.00500173978941269\\
60.01	0.00500174305306444\\
61.01	0.00500174638276826\\
62.01	0.00500174977985532\\
63.01	0.0050017532456833\\
64.01	0.00500175678163739\\
65.01	0.00500176038913015\\
66.01	0.00500176406960282\\
67.01	0.00500176782452535\\
68.01	0.00500177165539681\\
69.01	0.00500177556374673\\
70.01	0.00500177955113498\\
71.01	0.00500178361915304\\
72.01	0.00500178776942375\\
73.01	0.00500179200360257\\
74.01	0.00500179632337816\\
75.01	0.00500180073047306\\
76.01	0.00500180522664397\\
77.01	0.00500180981368298\\
78.01	0.00500181449341772\\
79.01	0.00500181926771266\\
80.01	0.0050018241384694\\
81.01	0.00500182910762757\\
82.01	0.00500183417716536\\
83.01	0.0050018393491007\\
84.01	0.00500184462549172\\
85.01	0.00500185000843775\\
86.01	0.00500185550007982\\
87.01	0.00500186110260194\\
88.01	0.00500186681823142\\
89.01	0.00500187264924018\\
90.01	0.00500187859794544\\
91.01	0.00500188466671074\\
92.01	0.00500189085794667\\
93.01	0.00500189717411179\\
94.01	0.00500190361771366\\
95.01	0.00500191019131004\\
96.01	0.00500191689750928\\
97.01	0.00500192373897206\\
98.01	0.00500193071841186\\
99.01	0.00500193783859608\\
100.01	0.00500194510234757\\
101.01	0.00500195251254507\\
102.01	0.00500196007212453\\
103.01	0.0050019677840805\\
104.01	0.00500197565146715\\
105.01	0.0050019836773994\\
106.01	0.00500199186505378\\
107.01	0.00500200021767026\\
108.01	0.00500200873855308\\
109.01	0.00500201743107212\\
110.01	0.00500202629866431\\
111.01	0.00500203534483475\\
112.01	0.00500204457315789\\
113.01	0.00500205398727957\\
114.01	0.00500206359091773\\
115.01	0.00500207338786419\\
116.01	0.00500208338198575\\
117.01	0.00500209357722599\\
118.01	0.00500210397760696\\
119.01	0.00500211458722999\\
120.01	0.00500212541027791\\
121.01	0.00500213645101627\\
122.01	0.00500214771379517\\
123.01	0.00500215920305093\\
124.01	0.00500217092330749\\
125.01	0.00500218287917836\\
126.01	0.00500219507536824\\
127.01	0.00500220751667511\\
128.01	0.00500222020799147\\
129.01	0.0050022331543068\\
130.01	0.00500224636070901\\
131.01	0.00500225983238653\\
132.01	0.00500227357463011\\
133.01	0.00500228759283523\\
134.01	0.00500230189250367\\
135.01	0.00500231647924548\\
136.01	0.00500233135878173\\
137.01	0.00500234653694627\\
138.01	0.00500236201968751\\
139.01	0.00500237781307142\\
140.01	0.00500239392328344\\
141.01	0.00500241035663061\\
142.01	0.00500242711954409\\
143.01	0.00500244421858152\\
144.01	0.00500246166042968\\
145.01	0.00500247945190674\\
146.01	0.00500249759996471\\
147.01	0.00500251611169228\\
148.01	0.00500253499431723\\
149.01	0.0050025542552094\\
150.01	0.00500257390188305\\
151.01	0.00500259394199984\\
152.01	0.0050026143833716\\
153.01	0.00500263523396326\\
154.01	0.00500265650189594\\
155.01	0.00500267819544939\\
156.01	0.00500270032306574\\
157.01	0.005002722893352\\
158.01	0.00500274591508375\\
159.01	0.0050027693972076\\
160.01	0.00500279334884582\\
161.01	0.0050028177792977\\
162.01	0.00500284269804473\\
163.01	0.00500286811475308\\
164.01	0.00500289403927738\\
165.01	0.00500292048166448\\
166.01	0.00500294745215671\\
167.01	0.00500297496119584\\
168.01	0.0050030030194267\\
169.01	0.00500303163770119\\
170.01	0.00500306082708216\\
171.01	0.00500309059884714\\
172.01	0.00500312096449286\\
173.01	0.00500315193573883\\
174.01	0.00500318352453204\\
175.01	0.00500321574305071\\
176.01	0.005003248603709\\
177.01	0.00500328211916144\\
178.01	0.00500331630230723\\
179.01	0.00500335116629466\\
180.01	0.00500338672452627\\
181.01	0.00500342299066324\\
182.01	0.00500345997863021\\
183.01	0.00500349770262021\\
184.01	0.00500353617709963\\
185.01	0.00500357541681333\\
186.01	0.00500361543678975\\
187.01	0.00500365625234608\\
188.01	0.00500369787909372\\
189.01	0.00500374033294362\\
190.01	0.00500378363011175\\
191.01	0.00500382778712488\\
192.01	0.00500387282082589\\
193.01	0.0050039187483798\\
194.01	0.00500396558727978\\
195.01	0.00500401335535278\\
196.01	0.00500406207076591\\
197.01	0.00500411175203229\\
198.01	0.00500416241801774\\
199.01	0.00500421408794673\\
200.01	0.00500426678140897\\
201.01	0.00500432051836609\\
202.01	0.00500437531915836\\
203.01	0.00500443120451104\\
204.01	0.00500448819554175\\
205.01	0.00500454631376729\\
206.01	0.00500460558111075\\
207.01	0.00500466601990832\\
208.01	0.00500472765291744\\
209.01	0.00500479050332339\\
210.01	0.00500485459474725\\
211.01	0.00500491995125339\\
212.01	0.00500498659735742\\
213.01	0.00500505455803364\\
214.01	0.0050051238587237\\
215.01	0.00500519452534379\\
216.01	0.0050052665842935\\
217.01	0.0050053400624642\\
218.01	0.0050054149872469\\
219.01	0.00500549138654141\\
220.01	0.00500556928876456\\
221.01	0.00500564872285925\\
222.01	0.00500572971830321\\
223.01	0.00500581230511794\\
224.01	0.00500589651387844\\
225.01	0.00500598237572125\\
226.01	0.00500606992235524\\
227.01	0.00500615918607009\\
228.01	0.00500625019974606\\
229.01	0.00500634299686419\\
230.01	0.00500643761151554\\
231.01	0.00500653407841152\\
232.01	0.00500663243289368\\
233.01	0.00500673271094423\\
234.01	0.00500683494919568\\
235.01	0.00500693918494234\\
236.01	0.00500704545614977\\
237.01	0.0050071538014658\\
238.01	0.00500726426023136\\
239.01	0.00500737687249117\\
240.01	0.00500749167900487\\
241.01	0.00500760872125791\\
242.01	0.00500772804147255\\
243.01	0.00500784968261975\\
244.01	0.00500797368842978\\
245.01	0.00500810010340398\\
246.01	0.0050082289728267\\
247.01	0.00500836034277632\\
248.01	0.00500849426013733\\
249.01	0.00500863077261224\\
250.01	0.00500876992873327\\
251.01	0.00500891177787448\\
252.01	0.00500905637026401\\
253.01	0.00500920375699584\\
254.01	0.00500935399004249\\
255.01	0.0050095071222674\\
256.01	0.00500966320743702\\
257.01	0.00500982230023315\\
258.01	0.00500998445626602\\
259.01	0.00501014973208669\\
260.01	0.00501031818519961\\
261.01	0.00501048987407515\\
262.01	0.00501066485816323\\
263.01	0.00501084319790523\\
264.01	0.00501102495474763\\
265.01	0.00501121019115461\\
266.01	0.00501139897062119\\
267.01	0.00501159135768638\\
268.01	0.00501178741794621\\
269.01	0.00501198721806694\\
270.01	0.00501219082579847\\
271.01	0.00501239830998744\\
272.01	0.00501260974059091\\
273.01	0.00501282518868935\\
274.01	0.00501304472650031\\
275.01	0.0050132684273924\\
276.01	0.00501349636589818\\
277.01	0.00501372861772856\\
278.01	0.00501396525978616\\
279.01	0.00501420637017932\\
280.01	0.00501445202823633\\
281.01	0.00501470231451923\\
282.01	0.00501495731083836\\
283.01	0.00501521710026688\\
284.01	0.00501548176715487\\
285.01	0.00501575139714465\\
286.01	0.00501602607718557\\
287.01	0.00501630589554915\\
288.01	0.00501659094184514\\
289.01	0.00501688130703674\\
290.01	0.00501717708345732\\
291.01	0.00501747836482667\\
292.01	0.00501778524626806\\
293.01	0.00501809782432616\\
294.01	0.00501841619698477\\
295.01	0.00501874046368529\\
296.01	0.00501907072534664\\
297.01	0.00501940708438503\\
298.01	0.00501974964473477\\
299.01	0.00502009851187034\\
300.01	0.00502045379282888\\
301.01	0.00502081559623435\\
302.01	0.00502118403232309\\
303.01	0.00502155921296964\\
304.01	0.00502194125171489\\
305.01	0.00502233026379545\\
306.01	0.00502272636617557\\
307.01	0.00502312967757916\\
308.01	0.00502354031852612\\
309.01	0.00502395841136923\\
310.01	0.00502438408033382\\
311.01	0.00502481745156044\\
312.01	0.00502525865315002\\
313.01	0.00502570781521222\\
314.01	0.00502616506991675\\
315.01	0.00502663055154874\\
316.01	0.00502710439656688\\
317.01	0.00502758674366617\\
318.01	0.00502807773384525\\
319.01	0.00502857751047717\\
320.01	0.00502908621938583\\
321.01	0.00502960400892682\\
322.01	0.00503013103007399\\
323.01	0.00503066743651082\\
324.01	0.0050312133847286\\
325.01	0.00503176903412964\\
326.01	0.00503233454713715\\
327.01	0.00503291008931162\\
328.01	0.00503349582947334\\
329.01	0.00503409193983205\\
330.01	0.00503469859612314\\
331.01	0.005035315977751\\
332.01	0.00503594426793894\\
333.01	0.00503658365388542\\
334.01	0.00503723432692794\\
335.01	0.00503789648271236\\
336.01	0.00503857032136812\\
337.01	0.00503925604768955\\
338.01	0.00503995387132164\\
339.01	0.00504066400694936\\
340.01	0.00504138667449157\\
341.01	0.00504212209929511\\
342.01	0.00504287051233135\\
343.01	0.00504363215039098\\
344.01	0.00504440725627697\\
345.01	0.00504519607899295\\
346.01	0.00504599887392596\\
347.01	0.00504681590301874\\
348.01	0.00504764743493299\\
349.01	0.00504849374519711\\
350.01	0.00504935511633733\\
351.01	0.00505023183798901\\
352.01	0.00505112420698506\\
353.01	0.00505203252741686\\
354.01	0.00505295711066694\\
355.01	0.0050538982754085\\
356.01	0.00505485634756882\\
357.01	0.00505583166025526\\
358.01	0.00505682455364158\\
359.01	0.00505783537481175\\
360.01	0.00505886447756317\\
361.01	0.00505991222216809\\
362.01	0.0050609789750954\\
363.01	0.00506206510869642\\
364.01	0.00506317100085968\\
365.01	0.00506429703464094\\
366.01	0.00506544359787774\\
367.01	0.00506661108279842\\
368.01	0.00506779988563986\\
369.01	0.00506901040628567\\
370.01	0.00507024304794354\\
371.01	0.00507149821687677\\
372.01	0.00507277632220685\\
373.01	0.00507407777580396\\
374.01	0.0050754029922775\\
375.01	0.00507675238907809\\
376.01	0.0050781263867148\\
377.01	0.00507952540908478\\
378.01	0.00508094988390589\\
379.01	0.00508240024323037\\
380.01	0.0050838769240124\\
381.01	0.00508538036869029\\
382.01	0.00508691102574624\\
383.01	0.00508846935020078\\
384.01	0.00509005580401857\\
385.01	0.00509167085641588\\
386.01	0.0050933149840912\\
387.01	0.0050949886714239\\
388.01	0.00509669241067484\\
389.01	0.00509842670220022\\
390.01	0.00510019205467784\\
391.01	0.00510198898534765\\
392.01	0.00510381802026706\\
393.01	0.00510567969458225\\
394.01	0.0051075745528162\\
395.01	0.0051095031491746\\
396.01	0.0051114660478695\\
397.01	0.00511346382346355\\
398.01	0.00511549706123428\\
399.01	0.00511756635755942\\
400.01	0.00511967232032567\\
401.01	0.00512181556935933\\
402.01	0.00512399673688282\\
403.01	0.00512621646799463\\
404.01	0.00512847542117596\\
405.01	0.00513077426882418\\
406.01	0.00513311369781239\\
407.01	0.00513549441007806\\
408.01	0.00513791712323855\\
409.01	0.00514038257123667\\
410.01	0.00514289150501269\\
411.01	0.00514544469320723\\
412.01	0.00514804292289033\\
413.01	0.0051506870003202\\
414.01	0.00515337775172785\\
415.01	0.00515611602412834\\
416.01	0.005158902686157\\
417.01	0.00516173862892902\\
418.01	0.00516462476692038\\
419.01	0.00516756203886763\\
420.01	0.00517055140868483\\
421.01	0.00517359386639368\\
422.01	0.00517669042906353\\
423.01	0.00517984214175886\\
424.01	0.00518305007848732\\
425.01	0.00518631534314723\\
426.01	0.0051896390704676\\
427.01	0.00519302242693516\\
428.01	0.00519646661170586\\
429.01	0.00519997285749194\\
430.01	0.0052035424314217\\
431.01	0.00520717663586574\\
432.01	0.00521087680922418\\
433.01	0.00521464432667026\\
434.01	0.00521848060084656\\
435.01	0.00522238708250908\\
436.01	0.00522636526111817\\
437.01	0.00523041666537323\\
438.01	0.00523454286369303\\
439.01	0.00523874546464077\\
440.01	0.00524302611730065\\
441.01	0.00524738651160832\\
442.01	0.00525182837864539\\
443.01	0.00525635349090743\\
444.01	0.00526096366255775\\
445.01	0.00526566074968343\\
446.01	0.00527044665056876\\
447.01	0.00527532330600824\\
448.01	0.00528029269967614\\
449.01	0.00528535685857795\\
450.01	0.00529051785360128\\
451.01	0.00529577780018857\\
452.01	0.00530113885914815\\
453.01	0.00530660323761675\\
454.01	0.00531217319018241\\
455.01	0.005317851020168\\
456.01	0.00532363908106857\\
457.01	0.0053295397781255\\
458.01	0.00533555557001072\\
459.01	0.005341688970586\\
460.01	0.00534794255069161\\
461.01	0.00535431893991593\\
462.01	0.00536082082829249\\
463.01	0.00536745096787895\\
464.01	0.00537421217417684\\
465.01	0.00538110732737072\\
466.01	0.00538813937338328\\
467.01	0.00539531132476963\\
468.01	0.00540262626148751\\
469.01	0.00541008733159785\\
470.01	0.00541769775194074\\
471.01	0.00542546080881408\\
472.01	0.0054333798586714\\
473.01	0.00544145832885058\\
474.01	0.00544969971834659\\
475.01	0.00545810759864\\
476.01	0.00546668561459478\\
477.01	0.00547543748543522\\
478.01	0.00548436700581247\\
479.01	0.00549347804696805\\
480.01	0.00550277455799925\\
481.01	0.00551226056722783\\
482.01	0.0055219401836714\\
483.01	0.00553181759861202\\
484.01	0.00554189708725276\\
485.01	0.00555218301045088\\
486.01	0.00556267981651445\\
487.01	0.00557339204304411\\
488.01	0.00558432431880738\\
489.01	0.00559548136562726\\
490.01	0.0056068680002765\\
491.01	0.00561848913636992\\
492.01	0.00563034978625274\\
493.01	0.0056424550628901\\
494.01	0.00565481018177132\\
495.01	0.00566742046284165\\
496.01	0.00568029133248303\\
497.01	0.00569342832555668\\
498.01	0.00570683708752279\\
499.01	0.0057205233766401\\
500.01	0.00573449306624481\\
501.01	0.00574875214710439\\
502.01	0.00576330672983821\\
503.01	0.00577816304739932\\
504.01	0.00579332745760815\\
505.01	0.00580880644572997\\
506.01	0.00582460662709058\\
507.01	0.00584073474972176\\
508.01	0.00585719769703225\\
509.01	0.00587400249050098\\
510.01	0.00589115629238746\\
511.01	0.00590866640846227\\
512.01	0.00592654029075282\\
513.01	0.00594478554030617\\
514.01	0.00596340990996545\\
515.01	0.0059824213071583\\
516.01	0.00600182779669088\\
517.01	0.00602163760354031\\
518.01	0.00604185911563766\\
519.01	0.00606250088662946\\
520.01	0.00608357163860813\\
521.01	0.00610508026480032\\
522.01	0.00612703583220095\\
523.01	0.00614944758414021\\
524.01	0.00617232494277072\\
525.01	0.00619567751146097\\
526.01	0.00621951507707789\\
527.01	0.00624384761214108\\
528.01	0.00626868527682911\\
529.01	0.00629403842081533\\
530.01	0.00631991758490741\\
531.01	0.00634633350246382\\
532.01	0.00637329710055433\\
533.01	0.00640081950083352\\
534.01	0.00642891202008611\\
535.01	0.00645758617040558\\
536.01	0.00648685365895875\\
537.01	0.00651672638728716\\
538.01	0.00654721645008825\\
539.01	0.00657833613341638\\
540.01	0.00661009791223496\\
541.01	0.00664251444724473\\
542.01	0.00667559858090686\\
543.01	0.00670936333256887\\
544.01	0.00674382189259384\\
545.01	0.00677898761538313\\
546.01	0.00681487401117106\\
547.01	0.00685149473646011\\
548.01	0.00688886358294855\\
549.01	0.00692699446479224\\
550.01	0.00696590140402426\\
551.01	0.00700559851393984\\
552.01	0.0070460999802358\\
553.01	0.00708742003967412\\
554.01	0.00712957295601932\\
555.01	0.00717257299297346\\
556.01	0.00721643438381055\\
557.01	0.00726117129738661\\
558.01	0.00730679780017085\\
559.01	0.00735332781391837\\
560.01	0.00740077506856995\\
561.01	0.00744915304993686\\
562.01	0.0074984749416927\\
563.01	0.00754875356116406\\
564.01	0.00760000128837737\\
565.01	0.00765222998778898\\
566.01	0.00770545092209638\\
567.01	0.00775967465750245\\
568.01	0.0078149109597867\\
569.01	0.00787116868052467\\
570.01	0.00792845563279784\\
571.01	0.0079867784557522\\
572.01	0.00804614246739978\\
573.01	0.00810655150512116\\
574.01	0.00816800775342491\\
575.01	0.00823051155866612\\
576.01	0.00829406123062584\\
577.01	0.00835865283113084\\
578.01	0.00842427995026335\\
579.01	0.00849093347119724\\
580.01	0.00855860132533547\\
581.01	0.00862726824024612\\
582.01	0.00869691548395165\\
583.01	0.00876752061047502\\
584.01	0.00883905721325699\\
585.01	0.00891149469522319\\
586.01	0.00898479806700421\\
587.01	0.00905892778823375\\
588.01	0.0091338396711344\\
589.01	0.00920948487095468\\
590.01	0.00928580999449757\\
591.01	0.00936275736629181\\
592.01	0.00944026550228819\\
593.01	0.00951826985378823\\
594.01	0.00959670390021289\\
595.01	0.00967550068901628\\
596.01	0.00975459494542325\\
597.01	0.00983382663319375\\
598.01	0.00990866201848071\\
599.01	0.00997087280416276\\
599.02	0.00997138072163725\\
599.03	0.009971885576258\\
599.04	0.00997238733820211\\
599.05	0.00997288597735272\\
599.06	0.00997338146329604\\
599.07	0.00997387376531845\\
599.08	0.00997436285240349\\
599.09	0.00997484869322892\\
599.1	0.00997533125616361\\
599.11	0.00997581050926455\\
599.12	0.00997628642027372\\
599.13	0.00997675895661498\\
599.14	0.0099772280853909\\
599.15	0.00997769377337961\\
599.16	0.00997815598703157\\
599.17	0.00997861469246631\\
599.18	0.00997906985546915\\
599.19	0.00997952144148792\\
599.2	0.00997996941562957\\
599.21	0.00998041374265684\\
599.22	0.0099808543869848\\
599.23	0.00998129131267744\\
599.24	0.00998172448344418\\
599.25	0.00998215386263636\\
599.26	0.00998257941258353\\
599.27	0.00998300109279825\\
599.28	0.00998341886238981\\
599.29	0.00998383268006024\\
599.3	0.00998424250410032\\
599.31	0.00998464829238543\\
599.32	0.00998505000237151\\
599.33	0.00998544759109087\\
599.34	0.00998584101514799\\
599.35	0.00998623023071532\\
599.36	0.00998661519352895\\
599.37	0.00998699585888436\\
599.38	0.00998737218163197\\
599.39	0.00998774411617281\\
599.4	0.00998811161645403\\
599.41	0.00998847463596443\\
599.42	0.0099888331277299\\
599.43	0.00998918704430885\\
599.44	0.00998953633778757\\
599.45	0.00998988095977558\\
599.46	0.00999022086140088\\
599.47	0.00999055599330522\\
599.48	0.00999088630563925\\
599.49	0.00999121174805768\\
599.5	0.00999153226971435\\
599.51	0.0099918478192573\\
599.52	0.00999215834482374\\
599.53	0.00999246379403503\\
599.54	0.0099927641139915\\
599.55	0.00999305925126738\\
599.56	0.00999334915190553\\
599.57	0.0099936337614122\\
599.58	0.00999391302475173\\
599.59	0.00999418688634118\\
599.6	0.00999445529004489\\
599.61	0.00999471817916904\\
599.62	0.00999497549645613\\
599.63	0.00999522718407934\\
599.64	0.009995473183637\\
599.65	0.00999571343614678\\
599.66	0.00999594788204004\\
599.67	0.00999617646115599\\
599.68	0.0099963991127358\\
599.69	0.00999661577541675\\
599.7	0.00999682638722618\\
599.71	0.00999703088557551\\
599.72	0.00999722920725411\\
599.73	0.00999742128842315\\
599.74	0.00999760706460942\\
599.75	0.00999778647069897\\
599.76	0.00999795944093086\\
599.77	0.0099981259088907\\
599.78	0.0099982858075042\\
599.79	0.00999843906903064\\
599.8	0.00999858562505627\\
599.81	0.00999872540648767\\
599.82	0.00999885834354498\\
599.83	0.00999898436575515\\
599.84	0.00999910340194508\\
599.85	0.00999921538023465\\
599.86	0.00999932022802977\\
599.87	0.00999941787201528\\
599.88	0.00999950823814785\\
599.89	0.00999959125164875\\
599.9	0.00999966683699656\\
599.91	0.00999973491791987\\
599.92	0.0099997954173898\\
599.93	0.00999984825761255\\
599.94	0.00999989336002181\\
599.95	0.00999993064527112\\
599.96	0.00999996003322615\\
599.97	0.00999998144295691\\
599.98	0.00999999479272987\\
599.99	0.01\\
600	0.01\\
};
\addplot [color=mycolor4,solid,forget plot]
  table[row sep=crcr]{%
0.01	0.00496755019417924\\
1.01	0.0049675512405473\\
2.01	0.00496755230832586\\
3.01	0.00496755339795235\\
4.01	0.00496755450987339\\
5.01	0.00496755564454414\\
6.01	0.00496755680242982\\
7.01	0.00496755798400442\\
8.01	0.00496755918975203\\
9.01	0.00496756042016645\\
10.01	0.0049675616757516\\
11.01	0.0049675629570216\\
12.01	0.00496756426450115\\
13.01	0.00496756559872562\\
14.01	0.00496756696024118\\
15.01	0.00496756834960534\\
16.01	0.00496756976738685\\
17.01	0.00496757121416612\\
18.01	0.0049675726905352\\
19.01	0.00496757419709846\\
20.01	0.00496757573447246\\
21.01	0.00496757730328634\\
22.01	0.00496757890418198\\
23.01	0.00496758053781456\\
24.01	0.00496758220485252\\
25.01	0.0049675839059775\\
26.01	0.00496758564188582\\
27.01	0.00496758741328719\\
28.01	0.00496758922090651\\
29.01	0.00496759106548292\\
30.01	0.0049675929477707\\
31.01	0.00496759486853966\\
32.01	0.00496759682857525\\
33.01	0.00496759882867888\\
34.01	0.00496760086966812\\
35.01	0.00496760295237748\\
36.01	0.00496760507765827\\
37.01	0.00496760724637924\\
38.01	0.00496760945942672\\
39.01	0.00496761171770513\\
40.01	0.00496761402213719\\
41.01	0.0049676163736647\\
42.01	0.00496761877324842\\
43.01	0.00496762122186858\\
44.01	0.0049676237205255\\
45.01	0.00496762627023999\\
46.01	0.00496762887205328\\
47.01	0.00496763152702795\\
48.01	0.0049676342362482\\
49.01	0.00496763700082062\\
50.01	0.00496763982187355\\
51.01	0.00496764270055901\\
52.01	0.00496764563805219\\
53.01	0.004967648635552\\
54.01	0.00496765169428197\\
55.01	0.00496765481549043\\
56.01	0.00496765800045121\\
57.01	0.00496766125046365\\
58.01	0.00496766456685403\\
59.01	0.00496766795097521\\
60.01	0.0049676714042078\\
61.01	0.0049676749279603\\
62.01	0.00496767852366987\\
63.01	0.00496768219280302\\
64.01	0.0049676859368558\\
65.01	0.00496768975735483\\
66.01	0.00496769365585771\\
67.01	0.00496769763395352\\
68.01	0.00496770169326417\\
69.01	0.00496770583544404\\
70.01	0.00496771006218102\\
71.01	0.00496771437519765\\
72.01	0.0049677187762512\\
73.01	0.00496772326713463\\
74.01	0.00496772784967736\\
75.01	0.00496773252574605\\
76.01	0.00496773729724513\\
77.01	0.00496774216611764\\
78.01	0.00496774713434632\\
79.01	0.00496775220395372\\
80.01	0.0049677573770037\\
81.01	0.004967762655602\\
82.01	0.00496776804189705\\
83.01	0.00496777353808089\\
84.01	0.00496777914638992\\
85.01	0.00496778486910593\\
86.01	0.00496779070855712\\
87.01	0.00496779666711872\\
88.01	0.0049678027472143\\
89.01	0.0049678089513164\\
90.01	0.00496781528194765\\
91.01	0.00496782174168196\\
92.01	0.00496782833314528\\
93.01	0.00496783505901684\\
94.01	0.00496784192203031\\
95.01	0.00496784892497434\\
96.01	0.00496785607069442\\
97.01	0.00496786336209352\\
98.01	0.00496787080213341\\
99.01	0.00496787839383603\\
100.01	0.00496788614028412\\
101.01	0.00496789404462306\\
102.01	0.00496790211006197\\
103.01	0.00496791033987477\\
104.01	0.00496791873740164\\
105.01	0.00496792730605047\\
106.01	0.00496793604929797\\
107.01	0.00496794497069106\\
108.01	0.0049679540738487\\
109.01	0.00496796336246286\\
110.01	0.00496797284030015\\
111.01	0.00496798251120333\\
112.01	0.00496799237909287\\
113.01	0.00496800244796837\\
114.01	0.0049680127219105\\
115.01	0.004968023205082\\
116.01	0.00496803390173005\\
117.01	0.00496804481618734\\
118.01	0.00496805595287435\\
119.01	0.00496806731630038\\
120.01	0.00496807891106613\\
121.01	0.00496809074186499\\
122.01	0.00496810281348509\\
123.01	0.00496811513081118\\
124.01	0.00496812769882653\\
125.01	0.00496814052261496\\
126.01	0.00496815360736277\\
127.01	0.00496816695836078\\
128.01	0.00496818058100656\\
129.01	0.00496819448080647\\
130.01	0.00496820866337745\\
131.01	0.00496822313445001\\
132.01	0.00496823789987005\\
133.01	0.00496825296560104\\
134.01	0.00496826833772641\\
135.01	0.00496828402245247\\
136.01	0.00496830002610989\\
137.01	0.00496831635515706\\
138.01	0.00496833301618216\\
139.01	0.0049683500159058\\
140.01	0.0049683673611837\\
141.01	0.0049683850590095\\
142.01	0.00496840311651728\\
143.01	0.00496842154098463\\
144.01	0.00496844033983493\\
145.01	0.00496845952064086\\
146.01	0.0049684790911274\\
147.01	0.00496849905917386\\
148.01	0.00496851943281836\\
149.01	0.00496854022025964\\
150.01	0.00496856142986134\\
151.01	0.00496858307015448\\
152.01	0.00496860514984107\\
153.01	0.00496862767779743\\
154.01	0.00496865066307772\\
155.01	0.00496867411491754\\
156.01	0.004968698042737\\
157.01	0.00496872245614495\\
158.01	0.00496874736494229\\
159.01	0.00496877277912618\\
160.01	0.0049687987088929\\
161.01	0.00496882516464336\\
162.01	0.00496885215698574\\
163.01	0.00496887969674001\\
164.01	0.00496890779494216\\
165.01	0.00496893646284843\\
166.01	0.00496896571193949\\
167.01	0.00496899555392501\\
168.01	0.00496902600074794\\
169.01	0.00496905706458917\\
170.01	0.00496908875787194\\
171.01	0.00496912109326691\\
172.01	0.0049691540836969\\
173.01	0.00496918774234145\\
174.01	0.00496922208264239\\
175.01	0.00496925711830809\\
176.01	0.00496929286331974\\
177.01	0.00496932933193543\\
178.01	0.00496936653869663\\
179.01	0.00496940449843285\\
180.01	0.00496944322626755\\
181.01	0.00496948273762356\\
182.01	0.0049695230482293\\
183.01	0.0049695641741242\\
184.01	0.00496960613166485\\
185.01	0.00496964893753113\\
186.01	0.00496969260873233\\
187.01	0.00496973716261352\\
188.01	0.00496978261686203\\
189.01	0.00496982898951351\\
190.01	0.0049698762989593\\
191.01	0.00496992456395258\\
192.01	0.00496997380361565\\
193.01	0.00497002403744673\\
194.01	0.00497007528532704\\
195.01	0.00497012756752824\\
196.01	0.00497018090471969\\
197.01	0.00497023531797575\\
198.01	0.00497029082878378\\
199.01	0.00497034745905163\\
200.01	0.00497040523111546\\
201.01	0.00497046416774798\\
202.01	0.00497052429216663\\
203.01	0.00497058562804158\\
204.01	0.00497064819950424\\
205.01	0.00497071203115605\\
206.01	0.00497077714807698\\
207.01	0.0049708435758344\\
208.01	0.0049709113404918\\
209.01	0.00497098046861857\\
210.01	0.00497105098729865\\
211.01	0.00497112292414026\\
212.01	0.00497119630728528\\
213.01	0.00497127116541907\\
214.01	0.00497134752778053\\
215.01	0.00497142542417174\\
216.01	0.00497150488496835\\
217.01	0.00497158594112979\\
218.01	0.00497166862420986\\
219.01	0.00497175296636723\\
220.01	0.00497183900037647\\
221.01	0.00497192675963833\\
222.01	0.00497201627819194\\
223.01	0.00497210759072489\\
224.01	0.00497220073258527\\
225.01	0.00497229573979346\\
226.01	0.00497239264905308\\
227.01	0.00497249149776331\\
228.01	0.00497259232403127\\
229.01	0.00497269516668333\\
230.01	0.00497280006527831\\
231.01	0.00497290706011921\\
232.01	0.00497301619226651\\
233.01	0.00497312750355068\\
234.01	0.00497324103658504\\
235.01	0.00497335683477877\\
236.01	0.00497347494235049\\
237.01	0.00497359540434152\\
238.01	0.00497371826662901\\
239.01	0.00497384357594036\\
240.01	0.00497397137986627\\
241.01	0.00497410172687499\\
242.01	0.0049742346663265\\
243.01	0.00497437024848646\\
244.01	0.00497450852454065\\
245.01	0.00497464954660931\\
246.01	0.00497479336776174\\
247.01	0.00497494004203085\\
248.01	0.00497508962442793\\
249.01	0.00497524217095781\\
250.01	0.00497539773863309\\
251.01	0.00497555638549001\\
252.01	0.00497571817060258\\
253.01	0.00497588315409864\\
254.01	0.0049760513971745\\
255.01	0.00497622296211021\\
256.01	0.00497639791228509\\
257.01	0.00497657631219313\\
258.01	0.00497675822745788\\
259.01	0.00497694372484804\\
260.01	0.00497713287229314\\
261.01	0.00497732573889832\\
262.01	0.00497752239496002\\
263.01	0.00497772291198105\\
264.01	0.00497792736268579\\
265.01	0.00497813582103557\\
266.01	0.0049783483622434\\
267.01	0.00497856506278914\\
268.01	0.00497878600043404\\
269.01	0.00497901125423583\\
270.01	0.00497924090456307\\
271.01	0.00497947503310968\\
272.01	0.00497971372290859\\
273.01	0.00497995705834638\\
274.01	0.00498020512517662\\
275.01	0.00498045801053344\\
276.01	0.00498071580294483\\
277.01	0.00498097859234527\\
278.01	0.00498124647008918\\
279.01	0.00498151952896214\\
280.01	0.00498179786319315\\
281.01	0.00498208156846677\\
282.01	0.00498237074193363\\
283.01	0.00498266548222114\\
284.01	0.00498296588944425\\
285.01	0.00498327206521498\\
286.01	0.00498358411265164\\
287.01	0.00498390213638834\\
288.01	0.00498422624258228\\
289.01	0.00498455653892249\\
290.01	0.00498489313463672\\
291.01	0.0049852361404982\\
292.01	0.00498558566883212\\
293.01	0.00498594183352066\\
294.01	0.00498630475000855\\
295.01	0.0049866745353076\\
296.01	0.0049870513080004\\
297.01	0.00498743518824385\\
298.01	0.00498782629777228\\
299.01	0.00498822475989989\\
300.01	0.00498863069952266\\
301.01	0.00498904424312037\\
302.01	0.00498946551875726\\
303.01	0.00498989465608378\\
304.01	0.00499033178633712\\
305.01	0.00499077704234251\\
306.01	0.00499123055851343\\
307.01	0.00499169247085336\\
308.01	0.00499216291695664\\
309.01	0.00499264203601017\\
310.01	0.00499312996879676\\
311.01	0.00499362685769801\\
312.01	0.00499413284669889\\
313.01	0.0049946480813938\\
314.01	0.0049951727089943\\
315.01	0.00499570687833882\\
316.01	0.00499625073990526\\
317.01	0.00499680444582594\\
318.01	0.00499736814990544\\
319.01	0.00499794200764365\\
320.01	0.00499852617626204\\
321.01	0.00499912081473511\\
322.01	0.00499972608382755\\
323.01	0.00500034214613818\\
324.01	0.00500096916614998\\
325.01	0.00500160731028902\\
326.01	0.00500225674699113\\
327.01	0.0050029176467794\\
328.01	0.00500359018235161\\
329.01	0.00500427452867935\\
330.01	0.00500497086312048\\
331.01	0.00500567936554518\\
332.01	0.00500640021847691\\
333.01	0.00500713360725059\\
334.01	0.00500787972018706\\
335.01	0.00500863874878736\\
336.01	0.00500941088794696\\
337.01	0.00501019633619076\\
338.01	0.00501099529593065\\
339.01	0.00501180797374665\\
340.01	0.0050126345806918\\
341.01	0.0050134753326232\\
342.01	0.00501433045055741\\
343.01	0.00501520016105224\\
344.01	0.00501608469661437\\
345.01	0.0050169842961323\\
346.01	0.0050178992053329\\
347.01	0.00501882967726136\\
348.01	0.00501977597278112\\
349.01	0.00502073836109178\\
350.01	0.00502171712025976\\
351.01	0.0050227125377582\\
352.01	0.00502372491100889\\
353.01	0.00502475454791997\\
354.01	0.00502580176740999\\
355.01	0.00502686689990904\\
356.01	0.00502795028782525\\
357.01	0.00502905228596373\\
358.01	0.00503017326188299\\
359.01	0.0050313135961752\\
360.01	0.00503247368265083\\
361.01	0.00503365392841168\\
362.01	0.00503485475379381\\
363.01	0.00503607659216244\\
364.01	0.00503731988954131\\
365.01	0.00503858510406292\\
366.01	0.00503987270522489\\
367.01	0.00504118317294898\\
368.01	0.005042516996438\\
369.01	0.00504387467284059\\
370.01	0.00504525670573876\\
371.01	0.00504666360348999\\
372.01	0.00504809587746541\\
373.01	0.00504955404024386\\
374.01	0.00505103860383685\\
375.01	0.0050525500780346\\
376.01	0.0050540889689753\\
377.01	0.0050556557780482\\
378.01	0.00505725100123884\\
379.01	0.0050588751290104\\
380.01	0.0050605286467818\\
381.01	0.00506221203601141\\
382.01	0.00506392577581308\\
383.01	0.00506567034493595\\
384.01	0.00506744622382201\\
385.01	0.00506925389636752\\
386.01	0.00507109385100737\\
387.01	0.00507296658105322\\
388.01	0.0050748725847341\\
389.01	0.00507681236521759\\
390.01	0.00507878643064001\\
391.01	0.00508079529414431\\
392.01	0.00508283947392883\\
393.01	0.00508491949330739\\
394.01	0.0050870358807824\\
395.01	0.00508918917013286\\
396.01	0.00509137990051904\\
397.01	0.00509360861660459\\
398.01	0.00509587586870029\\
399.01	0.00509818221292972\\
400.01	0.00510052821141903\\
401.01	0.00510291443251547\\
402.01	0.005105341451034\\
403.01	0.00510780984853788\\
404.01	0.00511032021365325\\
405.01	0.00511287314242206\\
406.01	0.00511546923869705\\
407.01	0.00511810911457931\\
408.01	0.00512079339090511\\
409.01	0.00512352269778132\\
410.01	0.00512629767517689\\
411.01	0.00512911897356907\\
412.01	0.00513198725465157\\
413.01	0.00513490319210574\\
414.01	0.00513786747243726\\
415.01	0.00514088079588267\\
416.01	0.00514394387738673\\
417.01	0.00514705744765431\\
418.01	0.00515022225427615\\
419.01	0.00515343906293231\\
420.01	0.00515670865867184\\
421.01	0.0051600318472699\\
422.01	0.00516340945666009\\
423.01	0.00516684233844038\\
424.01	0.00517033136945048\\
425.01	0.00517387745341429\\
426.01	0.00517748152264334\\
427.01	0.00518114453979394\\
428.01	0.00518486749966721\\
429.01	0.00518865143104338\\
430.01	0.00519249739853767\\
431.01	0.00519640650446129\\
432.01	0.00520037989067396\\
433.01	0.00520441874040711\\
434.01	0.00520852428003953\\
435.01	0.00521269778080247\\
436.01	0.00521694056039095\\
437.01	0.0052212539844582\\
438.01	0.00522563946796589\\
439.01	0.00523009847636763\\
440.01	0.00523463252660038\\
441.01	0.00523924318786104\\
442.01	0.00524393208215018\\
443.01	0.00524870088456564\\
444.01	0.00525355132333839\\
445.01	0.00525848517960635\\
446.01	0.00526350428693462\\
447.01	0.00526861053059645\\
448.01	0.00527380584664806\\
449.01	0.00527909222083719\\
450.01	0.00528447168740513\\
451.01	0.00528994632785534\\
452.01	0.00529551826977522\\
453.01	0.00530118968581521\\
454.01	0.0053069627929332\\
455.01	0.0053128398520253\\
456.01	0.00531882316805671\\
457.01	0.00532491509080059\\
458.01	0.00533111801626969\\
459.01	0.00533743438889314\\
460.01	0.00534386670444288\\
461.01	0.00535041751365456\\
462.01	0.00535708942641741\\
463.01	0.00536388511633094\\
464.01	0.0053708073253608\\
465.01	0.00537785886827462\\
466.01	0.0053850426365309\\
467.01	0.00539236160134585\\
468.01	0.00539981881579216\\
469.01	0.00540741741598878\\
470.01	0.00541516062166209\\
471.01	0.00542305173640308\\
472.01	0.00543109414774612\\
473.01	0.00543929132709937\\
474.01	0.00544764682954967\\
475.01	0.00545616429357375\\
476.01	0.00546484744069134\\
477.01	0.00547370007510466\\
478.01	0.00548272608336978\\
479.01	0.00549192943415226\\
480.01	0.00550131417811883\\
481.01	0.00551088444801663\\
482.01	0.00552064445898752\\
483.01	0.00553059850915716\\
484.01	0.00554075098052845\\
485.01	0.00555110634019344\\
486.01	0.00556166914186174\\
487.01	0.00557244402768026\\
488.01	0.0055834357303013\\
489.01	0.00559464907513395\\
490.01	0.00560608898269654\\
491.01	0.00561776047097949\\
492.01	0.0056296686577266\\
493.01	0.00564181876255559\\
494.01	0.00565421610886396\\
495.01	0.00566686612550699\\
496.01	0.00567977434828099\\
497.01	0.00569294642129253\\
498.01	0.00570638809832281\\
499.01	0.00572010524429577\\
500.01	0.0057341038369189\\
501.01	0.00574838996851797\\
502.01	0.00576296984806557\\
503.01	0.00577784980339512\\
504.01	0.00579303628358452\\
505.01	0.00580853586148709\\
506.01	0.00582435523638102\\
507.01	0.00584050123670836\\
508.01	0.0058569808228702\\
509.01	0.00587380109004865\\
510.01	0.00589096927103139\\
511.01	0.00590849273902176\\
512.01	0.00592637901042681\\
513.01	0.00594463574762613\\
514.01	0.00596327076173313\\
515.01	0.00598229201536413\\
516.01	0.00600170762542973\\
517.01	0.0060215258659571\\
518.01	0.00604175517093986\\
519.01	0.00606240413720248\\
520.01	0.00608348152725796\\
521.01	0.00610499627213941\\
522.01	0.00612695747418071\\
523.01	0.00614937440973107\\
524.01	0.0061722565317805\\
525.01	0.00619561347248153\\
526.01	0.00621945504555035\\
527.01	0.00624379124853081\\
528.01	0.00626863226490573\\
529.01	0.00629398846603425\\
530.01	0.0063198704128945\\
531.01	0.00634628885760222\\
532.01	0.00637325474467523\\
533.01	0.00640077921200628\\
534.01	0.00642887359150497\\
535.01	0.00645754940936542\\
536.01	0.00648681838591415\\
537.01	0.00651669243498613\\
538.01	0.00654718366277532\\
539.01	0.00657830436609863\\
540.01	0.00661006703000588\\
541.01	0.00664248432466176\\
542.01	0.00667556910141767\\
543.01	0.00670933438798234\\
544.01	0.00674379338259237\\
545.01	0.00677895944707115\\
546.01	0.00681484609865593\\
547.01	0.00685146700046133\\
548.01	0.00688883595043263\\
549.01	0.00692696686862976\\
550.01	0.00696587378266646\\
551.01	0.00700557081111293\\
552.01	0.00704607214465176\\
553.01	0.00708739202475621\\
554.01	0.0071295447196411\\
555.01	0.00717254449721021\\
556.01	0.00721640559470368\\
557.01	0.00726114218471995\\
558.01	0.00730676833725931\\
559.01	0.00735329797740891\\
560.01	0.00740074483825644\\
561.01	0.0074491224085882\\
562.01	0.00749844387489634\\
563.01	0.00754872205718511\\
564.01	0.00759996933803445\\
565.01	0.00765219758434847\\
566.01	0.00770541806118489\\
567.01	0.00775964133703997\\
568.01	0.00781487717993988\\
569.01	0.00787113444368163\\
570.01	0.00792842094356467\\
571.01	0.00798674332097159\\
572.01	0.00804610689619114\\
573.01	0.00810651550894171\\
574.01	0.00816797134615157\\
575.01	0.00823047475669562\\
576.01	0.00829402405299119\\
577.01	0.00835861529963226\\
578.01	0.00842424208960891\\
579.01	0.00849089530914933\\
580.01	0.00855856289285696\\
581.01	0.00862722957163953\\
582.01	0.00869687661698168\\
583.01	0.00876748158646345\\
584.01	0.00883901807713575\\
585.01	0.00891145549552764\\
586.01	0.00898475885578457\\
587.01	0.00905888862085642\\
588.01	0.00913380060593759\\
589.01	0.00920944596871416\\
590.01	0.00928577131764595\\
591.01	0.00936271897782057\\
592.01	0.00944022746424321\\
593.01	0.00951823222524592\\
594.01	0.00959666673459481\\
595.01	0.00967546403056262\\
596.01	0.00975455882459834\\
597.01	0.00983380640043464\\
598.01	0.00990866201848071\\
599.01	0.00997087280416276\\
599.02	0.00997138072163725\\
599.03	0.009971885576258\\
599.04	0.00997238733820211\\
599.05	0.00997288597735272\\
599.06	0.00997338146329604\\
599.07	0.00997387376531845\\
599.08	0.00997436285240349\\
599.09	0.00997484869322892\\
599.1	0.00997533125616361\\
599.11	0.00997581050926455\\
599.12	0.00997628642027372\\
599.13	0.00997675895661497\\
599.14	0.0099772280853909\\
599.15	0.00997769377337961\\
599.16	0.00997815598703157\\
599.17	0.00997861469246631\\
599.18	0.00997906985546915\\
599.19	0.00997952144148792\\
599.2	0.00997996941562957\\
599.21	0.00998041374265684\\
599.22	0.0099808543869848\\
599.23	0.00998129131267744\\
599.24	0.00998172448344418\\
599.25	0.00998215386263636\\
599.26	0.00998257941258354\\
599.27	0.00998300109279825\\
599.28	0.00998341886238981\\
599.29	0.00998383268006024\\
599.3	0.00998424250410032\\
599.31	0.00998464829238543\\
599.32	0.00998505000237151\\
599.33	0.00998544759109087\\
599.34	0.00998584101514799\\
599.35	0.00998623023071532\\
599.36	0.00998661519352896\\
599.37	0.00998699585888436\\
599.38	0.00998737218163197\\
599.39	0.00998774411617281\\
599.4	0.00998811161645403\\
599.41	0.00998847463596443\\
599.42	0.0099888331277299\\
599.43	0.00998918704430885\\
599.44	0.00998953633778757\\
599.45	0.00998988095977558\\
599.46	0.00999022086140088\\
599.47	0.00999055599330522\\
599.48	0.00999088630563925\\
599.49	0.00999121174805768\\
599.5	0.00999153226971435\\
599.51	0.0099918478192573\\
599.52	0.00999215834482374\\
599.53	0.00999246379403503\\
599.54	0.0099927641139915\\
599.55	0.00999305925126738\\
599.56	0.00999334915190553\\
599.57	0.0099936337614122\\
599.58	0.00999391302475173\\
599.59	0.00999418688634118\\
599.6	0.00999445529004489\\
599.61	0.00999471817916904\\
599.62	0.00999497549645613\\
599.63	0.00999522718407934\\
599.64	0.009995473183637\\
599.65	0.00999571343614678\\
599.66	0.00999594788204004\\
599.67	0.00999617646115598\\
599.68	0.0099963991127358\\
599.69	0.00999661577541675\\
599.7	0.00999682638722618\\
599.71	0.00999703088557551\\
599.72	0.00999722920725411\\
599.73	0.00999742128842315\\
599.74	0.00999760706460942\\
599.75	0.00999778647069897\\
599.76	0.00999795944093086\\
599.77	0.0099981259088907\\
599.78	0.0099982858075042\\
599.79	0.00999843906903064\\
599.8	0.00999858562505627\\
599.81	0.00999872540648767\\
599.82	0.00999885834354498\\
599.83	0.00999898436575515\\
599.84	0.00999910340194508\\
599.85	0.00999921538023465\\
599.86	0.00999932022802977\\
599.87	0.00999941787201528\\
599.88	0.00999950823814785\\
599.89	0.00999959125164875\\
599.9	0.00999966683699656\\
599.91	0.00999973491791987\\
599.92	0.0099997954173898\\
599.93	0.00999984825761255\\
599.94	0.00999989336002181\\
599.95	0.00999993064527112\\
599.96	0.00999996003322615\\
599.97	0.00999998144295691\\
599.98	0.00999999479272987\\
599.99	0.01\\
600	0.01\\
};
\addplot [color=mycolor5,solid,forget plot]
  table[row sep=crcr]{%
0.01	0.00490905749971206\\
1.01	0.00490905860197839\\
2.01	0.00490905972694872\\
3.01	0.00490906087509092\\
4.01	0.00490906204688202\\
5.01	0.00490906324280954\\
6.01	0.00490906446337047\\
7.01	0.00490906570907243\\
8.01	0.00490906698043314\\
9.01	0.00490906827798124\\
10.01	0.00490906960225622\\
11.01	0.00490907095380874\\
12.01	0.00490907233320062\\
13.01	0.00490907374100548\\
14.01	0.00490907517780848\\
15.01	0.00490907664420701\\
16.01	0.00490907814081074\\
17.01	0.00490907966824188\\
18.01	0.00490908122713578\\
19.01	0.0049090828181402\\
20.01	0.00490908444191679\\
21.01	0.00490908609914072\\
22.01	0.00490908779050088\\
23.01	0.00490908951670066\\
24.01	0.00490909127845751\\
25.01	0.00490909307650426\\
26.01	0.00490909491158817\\
27.01	0.00490909678447236\\
28.01	0.00490909869593543\\
29.01	0.00490910064677222\\
30.01	0.00490910263779389\\
31.01	0.00490910466982809\\
32.01	0.00490910674371986\\
33.01	0.00490910886033126\\
34.01	0.00490911102054242\\
35.01	0.00490911322525142\\
36.01	0.00490911547537501\\
37.01	0.0049091177718488\\
38.01	0.0049091201156274\\
39.01	0.00490912250768544\\
40.01	0.00490912494901743\\
41.01	0.00490912744063825\\
42.01	0.00490912998358396\\
43.01	0.0049091325789118\\
44.01	0.00490913522770087\\
45.01	0.00490913793105228\\
46.01	0.00490914069009012\\
47.01	0.00490914350596146\\
48.01	0.00490914637983708\\
49.01	0.00490914931291167\\
50.01	0.00490915230640494\\
51.01	0.0049091553615612\\
52.01	0.00490915847965061\\
53.01	0.00490916166196961\\
54.01	0.00490916490984129\\
55.01	0.00490916822461597\\
56.01	0.00490917160767151\\
57.01	0.00490917506041461\\
58.01	0.00490917858428062\\
59.01	0.00490918218073452\\
60.01	0.00490918585127144\\
61.01	0.00490918959741736\\
62.01	0.00490919342072964\\
63.01	0.00490919732279765\\
64.01	0.00490920130524378\\
65.01	0.00490920536972335\\
66.01	0.00490920951792605\\
67.01	0.00490921375157641\\
68.01	0.00490921807243412\\
69.01	0.00490922248229551\\
70.01	0.00490922698299386\\
71.01	0.00490923157639973\\
72.01	0.00490923626442272\\
73.01	0.00490924104901155\\
74.01	0.00490924593215501\\
75.01	0.00490925091588273\\
76.01	0.00490925600226642\\
77.01	0.00490926119342014\\
78.01	0.00490926649150164\\
79.01	0.0049092718987131\\
80.01	0.00490927741730177\\
81.01	0.00490928304956145\\
82.01	0.00490928879783287\\
83.01	0.00490929466450511\\
84.01	0.00490930065201656\\
85.01	0.00490930676285549\\
86.01	0.00490931299956143\\
87.01	0.00490931936472637\\
88.01	0.00490932586099546\\
89.01	0.0049093324910683\\
90.01	0.00490933925770003\\
91.01	0.00490934616370248\\
92.01	0.00490935321194527\\
93.01	0.00490936040535726\\
94.01	0.00490936774692727\\
95.01	0.00490937523970579\\
96.01	0.00490938288680609\\
97.01	0.00490939069140538\\
98.01	0.00490939865674618\\
99.01	0.00490940678613766\\
100.01	0.00490941508295725\\
101.01	0.00490942355065171\\
102.01	0.00490943219273871\\
103.01	0.00490944101280809\\
104.01	0.00490945001452382\\
105.01	0.00490945920162493\\
106.01	0.00490946857792736\\
107.01	0.00490947814732567\\
108.01	0.00490948791379411\\
109.01	0.00490949788138902\\
110.01	0.0049095080542499\\
111.01	0.0049095184366014\\
112.01	0.00490952903275502\\
113.01	0.00490953984711079\\
114.01	0.00490955088415899\\
115.01	0.00490956214848251\\
116.01	0.00490957364475825\\
117.01	0.00490958537775917\\
118.01	0.00490959735235603\\
119.01	0.00490960957352014\\
120.01	0.00490962204632456\\
121.01	0.00490963477594673\\
122.01	0.00490964776767029\\
123.01	0.00490966102688729\\
124.01	0.00490967455910074\\
125.01	0.00490968836992635\\
126.01	0.00490970246509535\\
127.01	0.00490971685045637\\
128.01	0.00490973153197836\\
129.01	0.00490974651575266\\
130.01	0.00490976180799554\\
131.01	0.00490977741505096\\
132.01	0.0049097933433928\\
133.01	0.00490980959962782\\
134.01	0.00490982619049835\\
135.01	0.00490984312288513\\
136.01	0.00490986040380985\\
137.01	0.00490987804043802\\
138.01	0.00490989604008256\\
139.01	0.00490991441020583\\
140.01	0.00490993315842334\\
141.01	0.00490995229250674\\
142.01	0.00490997182038666\\
143.01	0.00490999175015648\\
144.01	0.00491001209007534\\
145.01	0.00491003284857165\\
146.01	0.00491005403424595\\
147.01	0.00491007565587574\\
148.01	0.00491009772241746\\
149.01	0.0049101202430113\\
150.01	0.00491014322698433\\
151.01	0.00491016668385458\\
152.01	0.00491019062333502\\
153.01	0.00491021505533708\\
154.01	0.00491023998997492\\
155.01	0.00491026543756963\\
156.01	0.00491029140865341\\
157.01	0.00491031791397366\\
158.01	0.00491034496449737\\
159.01	0.00491037257141587\\
160.01	0.00491040074614905\\
161.01	0.00491042950034998\\
162.01	0.00491045884590979\\
163.01	0.00491048879496239\\
164.01	0.00491051935988952\\
165.01	0.0049105505533255\\
166.01	0.00491058238816239\\
167.01	0.00491061487755542\\
168.01	0.00491064803492815\\
169.01	0.00491068187397732\\
170.01	0.0049107164086796\\
171.01	0.00491075165329606\\
172.01	0.00491078762237854\\
173.01	0.00491082433077526\\
174.01	0.00491086179363659\\
175.01	0.00491090002642164\\
176.01	0.004910939044904\\
177.01	0.00491097886517806\\
178.01	0.00491101950366565\\
179.01	0.00491106097712254\\
180.01	0.00491110330264509\\
181.01	0.00491114649767722\\
182.01	0.00491119058001678\\
183.01	0.00491123556782333\\
184.01	0.00491128147962502\\
185.01	0.00491132833432602\\
186.01	0.00491137615121377\\
187.01	0.00491142494996694\\
188.01	0.00491147475066307\\
189.01	0.0049115255737865\\
190.01	0.00491157744023633\\
191.01	0.00491163037133473\\
192.01	0.0049116843888356\\
193.01	0.00491173951493254\\
194.01	0.00491179577226811\\
195.01	0.00491185318394217\\
196.01	0.00491191177352146\\
197.01	0.00491197156504844\\
198.01	0.00491203258305056\\
199.01	0.00491209485255008\\
200.01	0.00491215839907388\\
201.01	0.00491222324866275\\
202.01	0.00491228942788217\\
203.01	0.00491235696383208\\
204.01	0.00491242588415776\\
205.01	0.00491249621705986\\
206.01	0.00491256799130579\\
207.01	0.00491264123624047\\
208.01	0.0049127159817977\\
209.01	0.00491279225851149\\
210.01	0.00491287009752768\\
211.01	0.00491294953061565\\
212.01	0.00491303059018065\\
213.01	0.00491311330927587\\
214.01	0.00491319772161466\\
215.01	0.00491328386158379\\
216.01	0.00491337176425578\\
217.01	0.00491346146540234\\
218.01	0.00491355300150757\\
219.01	0.0049136464097818\\
220.01	0.00491374172817487\\
221.01	0.00491383899539088\\
222.01	0.00491393825090128\\
223.01	0.00491403953496078\\
224.01	0.00491414288862115\\
225.01	0.00491424835374625\\
226.01	0.00491435597302772\\
227.01	0.00491446578999996\\
228.01	0.00491457784905612\\
229.01	0.00491469219546373\\
230.01	0.00491480887538116\\
231.01	0.00491492793587396\\
232.01	0.00491504942493175\\
233.01	0.00491517339148445\\
234.01	0.0049152998854203\\
235.01	0.00491542895760302\\
236.01	0.00491556065988899\\
237.01	0.00491569504514588\\
238.01	0.00491583216727048\\
239.01	0.00491597208120721\\
240.01	0.00491611484296665\\
241.01	0.0049162605096445\\
242.01	0.00491640913944097\\
243.01	0.00491656079167989\\
244.01	0.00491671552682842\\
245.01	0.00491687340651699\\
246.01	0.00491703449355928\\
247.01	0.00491719885197298\\
248.01	0.00491736654699965\\
249.01	0.00491753764512574\\
250.01	0.00491771221410402\\
251.01	0.00491789032297398\\
252.01	0.00491807204208403\\
253.01	0.00491825744311238\\
254.01	0.00491844659908905\\
255.01	0.00491863958441796\\
256.01	0.00491883647489887\\
257.01	0.00491903734774932\\
258.01	0.00491924228162745\\
259.01	0.00491945135665417\\
260.01	0.00491966465443579\\
261.01	0.0049198822580869\\
262.01	0.00492010425225277\\
263.01	0.0049203307231323\\
264.01	0.004920561758501\\
265.01	0.00492079744773356\\
266.01	0.0049210378818268\\
267.01	0.00492128315342236\\
268.01	0.00492153335682981\\
269.01	0.00492178858804879\\
270.01	0.00492204894479158\\
271.01	0.00492231452650571\\
272.01	0.00492258543439587\\
273.01	0.00492286177144605\\
274.01	0.00492314364244112\\
275.01	0.00492343115398804\\
276.01	0.00492372441453698\\
277.01	0.00492402353440209\\
278.01	0.0049243286257809\\
279.01	0.00492463980277491\\
280.01	0.00492495718140823\\
281.01	0.00492528087964596\\
282.01	0.00492561101741196\\
283.01	0.00492594771660614\\
284.01	0.00492629110112015\\
285.01	0.0049266412968533\\
286.01	0.00492699843172646\\
287.01	0.00492736263569535\\
288.01	0.00492773404076295\\
289.01	0.00492811278099051\\
290.01	0.00492849899250662\\
291.01	0.00492889281351602\\
292.01	0.00492929438430578\\
293.01	0.00492970384725112\\
294.01	0.00493012134681801\\
295.01	0.0049305470295654\\
296.01	0.00493098104414412\\
297.01	0.00493142354129508\\
298.01	0.00493187467384421\\
299.01	0.00493233459669588\\
300.01	0.0049328034668234\\
301.01	0.004933281443257\\
302.01	0.00493376868706956\\
303.01	0.00493426536135884\\
304.01	0.00493477163122784\\
305.01	0.00493528766376086\\
306.01	0.00493581362799739\\
307.01	0.00493634969490228\\
308.01	0.00493689603733261\\
309.01	0.00493745283000145\\
310.01	0.00493802024943731\\
311.01	0.00493859847394061\\
312.01	0.00493918768353642\\
313.01	0.0049397880599234\\
314.01	0.00494039978641915\\
315.01	0.00494102304790169\\
316.01	0.0049416580307475\\
317.01	0.00494230492276611\\
318.01	0.00494296391313159\\
319.01	0.00494363519231056\\
320.01	0.00494431895198727\\
321.01	0.00494501538498655\\
322.01	0.00494572468519489\\
323.01	0.00494644704747903\\
324.01	0.00494718266760483\\
325.01	0.00494793174215515\\
326.01	0.00494869446844877\\
327.01	0.00494947104446108\\
328.01	0.00495026166874761\\
329.01	0.00495106654037195\\
330.01	0.00495188585883968\\
331.01	0.0049527198240398\\
332.01	0.00495356863619648\\
333.01	0.00495443249583227\\
334.01	0.0049553116037469\\
335.01	0.00495620616101282\\
336.01	0.00495711636899286\\
337.01	0.00495804242938172\\
338.01	0.00495898454427672\\
339.01	0.00495994291628156\\
340.01	0.00496091774864825\\
341.01	0.00496190924546254\\
342.01	0.00496291761187816\\
343.01	0.00496394305440709\\
344.01	0.00496498578127114\\
345.01	0.00496604600282269\\
346.01	0.0049671239320417\\
347.01	0.00496821978511656\\
348.01	0.00496933378211588\\
349.01	0.00497046614775908\\
350.01	0.00497161711229437\\
351.01	0.00497278691248819\\
352.01	0.00497397579273538\\
353.01	0.004975184006293\\
354.01	0.0049764118166411\\
355.01	0.00497765949897297\\
356.01	0.00497892734181093\\
357.01	0.00498021564874448\\
358.01	0.00498152474027903\\
359.01	0.00498285495578051\\
360.01	0.00498420665549278\\
361.01	0.00498558022259997\\
362.01	0.00498697606529111\\
363.01	0.00498839461877956\\
364.01	0.00498983634721632\\
365.01	0.00499130174541977\\
366.01	0.00499279134033873\\
367.01	0.00499430569214429\\
368.01	0.00499584539483784\\
369.01	0.00499741107625103\\
370.01	0.00499900339730523\\
371.01	0.00500062305039763\\
372.01	0.00500227075678872\\
373.01	0.0050039472628861\\
374.01	0.00500565333535642\\
375.01	0.00500738975505523\\
376.01	0.0050091573098522\\
377.01	0.00501095678654535\\
378.01	0.00501278896220887\\
379.01	0.00501465459550808\\
380.01	0.00501655441872212\\
381.01	0.00501848913142981\\
382.01	0.00502045939698054\\
383.01	0.00502246584291349\\
384.01	0.00502450906625633\\
385.01	0.00502658964389123\\
386.01	0.00502870814599819\\
387.01	0.00503086514465358\\
388.01	0.00503306121496463\\
389.01	0.00503529693482319\\
390.01	0.00503757288463709\\
391.01	0.00503988964705035\\
392.01	0.00504224780665428\\
393.01	0.0050446479496886\\
394.01	0.0050470906637351\\
395.01	0.00504957653740315\\
396.01	0.00505210616001035\\
397.01	0.00505468012125833\\
398.01	0.00505729901090658\\
399.01	0.00505996341844481\\
400.01	0.00506267393276891\\
401.01	0.00506543114186003\\
402.01	0.00506823563247182\\
403.01	0.00507108798982919\\
404.01	0.00507398879734204\\
405.01	0.00507693863633824\\
406.01	0.00507993808582029\\
407.01	0.00508298772225244\\
408.01	0.00508608811938221\\
409.01	0.00508923984810478\\
410.01	0.00509244347637487\\
411.01	0.00509569956917674\\
412.01	0.00509900868855837\\
413.01	0.00510237139373884\\
414.01	0.00510578824130067\\
415.01	0.00510925978547561\\
416.01	0.0051127865785352\\
417.01	0.00511636917129822\\
418.01	0.00512000811376698\\
419.01	0.00512370395590563\\
420.01	0.00512745724857272\\
421.01	0.00513126854462224\\
422.01	0.00513513840018684\\
423.01	0.00513906737615664\\
424.01	0.00514305603986628\\
425.01	0.00514710496700421\\
426.01	0.00515121474375491\\
427.01	0.00515538596918477\\
428.01	0.00515961925788036\\
429.01	0.00516391524284567\\
430.01	0.00516827457866102\\
431.01	0.00517269794490338\\
432.01	0.00517718604982302\\
433.01	0.00518173963426656\\
434.01	0.00518635947582922\\
435.01	0.00519104639321241\\
436.01	0.00519580125075539\\
437.01	0.00520062496309655\\
438.01	0.00520551849991628\\
439.01	0.0052104828906948\\
440.01	0.00521551922941084\\
441.01	0.00522062867909467\\
442.01	0.00522581247613339\\
443.01	0.00523107193421826\\
444.01	0.00523640844780934\\
445.01	0.00524182349498784\\
446.01	0.00524731863955634\\
447.01	0.0052528955322519\\
448.01	0.00525855591093484\\
449.01	0.00526430159963491\\
450.01	0.00527013450635724\\
451.01	0.00527605661958334\\
452.01	0.00528207000345424\\
453.01	0.00528817679168332\\
454.01	0.00529437918033213\\
455.01	0.00530067941967452\\
456.01	0.00530707980549043\\
457.01	0.00531358267025249\\
458.01	0.00532019037479467\\
459.01	0.00532690530116499\\
460.01	0.00533372984745722\\
461.01	0.00534066642544533\\
462.01	0.00534771746179938\\
463.01	0.00535488540348701\\
464.01	0.00536217272763402\\
465.01	0.00536958195559539\\
466.01	0.00537711567026543\\
467.01	0.00538477653479423\\
468.01	0.00539256731002991\\
469.01	0.00540049086756045\\
470.01	0.00540855019647486\\
471.01	0.00541674840612985\\
472.01	0.00542508872779501\\
473.01	0.00543357451562931\\
474.01	0.00544220924694998\\
475.01	0.00545099652176344\\
476.01	0.00545994006154736\\
477.01	0.00546904370729777\\
478.01	0.00547831141688372\\
479.01	0.00548774726178407\\
480.01	0.00549735542331861\\
481.01	0.005507140188521\\
482.01	0.00551710594584133\\
483.01	0.00552725718089843\\
484.01	0.00553759847253217\\
485.01	0.00554813448942038\\
486.01	0.0055588699875275\\
487.01	0.00556980980863281\\
488.01	0.00558095888013528\\
489.01	0.00559232221625795\\
490.01	0.00560390492066476\\
491.01	0.00561571219036122\\
492.01	0.00562774932059444\\
493.01	0.00564002171030427\\
494.01	0.0056525348675442\\
495.01	0.00566529441422003\\
496.01	0.00567830608954503\\
497.01	0.00569157575182161\\
498.01	0.00570510937855931\\
499.01	0.00571891306547014\\
500.01	0.0057329930251922\\
501.01	0.0057473555862806\\
502.01	0.00576200719262769\\
503.01	0.00577695440339838\\
504.01	0.00579220389354149\\
505.01	0.0058077624549108\\
506.01	0.00582363699799712\\
507.01	0.00583983455422826\\
508.01	0.00585636227876019\\
509.01	0.00587322745364077\\
510.01	0.00589043749120116\\
511.01	0.00590799993751283\\
512.01	0.00592592247575652\\
513.01	0.00594421292937826\\
514.01	0.00596287926496326\\
515.01	0.00598192959483183\\
516.01	0.00600137217944308\\
517.01	0.00602121542974826\\
518.01	0.00604146790964455\\
519.01	0.00606213833861916\\
520.01	0.00608323559458114\\
521.01	0.00610476871683385\\
522.01	0.0061267469091303\\
523.01	0.00614917954274147\\
524.01	0.00617207615947925\\
525.01	0.00619544647461083\\
526.01	0.00621930037961972\\
527.01	0.00624364794477872\\
528.01	0.0062684994215141\\
529.01	0.00629386524455719\\
530.01	0.00631975603388037\\
531.01	0.00634618259641703\\
532.01	0.00637315592754992\\
533.01	0.00640068721233276\\
534.01	0.00642878782640081\\
535.01	0.00645746933650783\\
536.01	0.00648674350063502\\
537.01	0.00651662226761109\\
538.01	0.00654711777618407\\
539.01	0.00657824235348439\\
540.01	0.00661000851281471\\
541.01	0.00664242895069685\\
542.01	0.00667551654309728\\
543.01	0.00670928434074414\\
544.01	0.00674374556343396\\
545.01	0.00677891359321937\\
546.01	0.00681480196635297\\
547.01	0.00685142436385382\\
548.01	0.00688879460055168\\
549.01	0.00692692661244959\\
550.01	0.00696583444223101\\
551.01	0.00700553222272127\\
552.01	0.00704603415809471\\
553.01	0.00708735450259801\\
554.01	0.00712950753653874\\
555.01	0.00717250753926676\\
556.01	0.00721636875885068\\
557.01	0.00726110537812294\\
558.01	0.00730673147674622\\
559.01	0.00735326098891716\\
560.01	0.00740070765629824\\
561.01	0.00744908497573459\\
562.01	0.00749840614127946\\
563.01	0.00754868398002041\\
564.01	0.00759993088116463\\
565.01	0.00765215871781027\\
566.01	0.00770537876080272\\
567.01	0.00775960158404742\\
568.01	0.00781483696063322\\
569.01	0.00787109374910713\\
570.01	0.00792837976924348\\
571.01	0.00798670166666285\\
572.01	0.00804606476569629\\
573.01	0.00810647290995056\\
574.01	0.00816792829013013\\
575.01	0.00823043125881488\\
576.01	0.00829398013209581\\
577.01	0.00835857097824435\\
578.01	0.00842419739396354\\
579.01	0.00849085026925531\\
580.01	0.00855851754257446\\
581.01	0.0086271839487638\\
582.01	0.00869683076332219\\
583.01	0.00876743554790117\\
584.01	0.00883897190364118\\
585.01	0.00891140924111708\\
586.01	0.00898471257838705\\
587.01	0.00905884238205851\\
588.01	0.0091337544705663\\
589.01	0.00920940000420826\\
590.01	0.00928572559315688\\
591.01	0.00936267356296942\\
592.01	0.00944018242744499\\
593.01	0.00951818763149205\\
594.01	0.00959662264256165\\
595.01	0.00967542048888486\\
596.01	0.00975451586711174\\
597.01	0.00983378252435979\\
598.01	0.00990866201848071\\
599.01	0.00997087280416276\\
599.02	0.00997138072163725\\
599.03	0.00997188557625799\\
599.04	0.00997238733820211\\
599.05	0.00997288597735272\\
599.06	0.00997338146329604\\
599.07	0.00997387376531845\\
599.08	0.00997436285240349\\
599.09	0.00997484869322892\\
599.1	0.00997533125616361\\
599.11	0.00997581050926455\\
599.12	0.00997628642027372\\
599.13	0.00997675895661497\\
599.14	0.0099772280853909\\
599.15	0.00997769377337961\\
599.16	0.00997815598703157\\
599.17	0.00997861469246631\\
599.18	0.00997906985546915\\
599.19	0.00997952144148792\\
599.2	0.00997996941562957\\
599.21	0.00998041374265684\\
599.22	0.0099808543869848\\
599.23	0.00998129131267744\\
599.24	0.00998172448344418\\
599.25	0.00998215386263636\\
599.26	0.00998257941258354\\
599.27	0.00998300109279825\\
599.28	0.00998341886238981\\
599.29	0.00998383268006024\\
599.3	0.00998424250410032\\
599.31	0.00998464829238543\\
599.32	0.00998505000237151\\
599.33	0.00998544759109087\\
599.34	0.00998584101514799\\
599.35	0.00998623023071532\\
599.36	0.00998661519352895\\
599.37	0.00998699585888436\\
599.38	0.00998737218163197\\
599.39	0.00998774411617281\\
599.4	0.00998811161645403\\
599.41	0.00998847463596443\\
599.42	0.0099888331277299\\
599.43	0.00998918704430885\\
599.44	0.00998953633778757\\
599.45	0.00998988095977558\\
599.46	0.00999022086140088\\
599.47	0.00999055599330522\\
599.48	0.00999088630563925\\
599.49	0.00999121174805768\\
599.5	0.00999153226971435\\
599.51	0.0099918478192573\\
599.52	0.00999215834482374\\
599.53	0.00999246379403503\\
599.54	0.0099927641139915\\
599.55	0.00999305925126738\\
599.56	0.00999334915190553\\
599.57	0.0099936337614122\\
599.58	0.00999391302475173\\
599.59	0.00999418688634118\\
599.6	0.00999445529004489\\
599.61	0.00999471817916904\\
599.62	0.00999497549645613\\
599.63	0.00999522718407935\\
599.64	0.009995473183637\\
599.65	0.00999571343614678\\
599.66	0.00999594788204004\\
599.67	0.00999617646115598\\
599.68	0.0099963991127358\\
599.69	0.00999661577541675\\
599.7	0.00999682638722618\\
599.71	0.00999703088557551\\
599.72	0.00999722920725411\\
599.73	0.00999742128842315\\
599.74	0.00999760706460942\\
599.75	0.00999778647069897\\
599.76	0.00999795944093086\\
599.77	0.0099981259088907\\
599.78	0.0099982858075042\\
599.79	0.00999843906903064\\
599.8	0.00999858562505627\\
599.81	0.00999872540648766\\
599.82	0.00999885834354498\\
599.83	0.00999898436575515\\
599.84	0.00999910340194508\\
599.85	0.00999921538023465\\
599.86	0.00999932022802977\\
599.87	0.00999941787201528\\
599.88	0.00999950823814785\\
599.89	0.00999959125164875\\
599.9	0.00999966683699656\\
599.91	0.00999973491791987\\
599.92	0.0099997954173898\\
599.93	0.00999984825761255\\
599.94	0.00999989336002181\\
599.95	0.00999993064527112\\
599.96	0.00999996003322615\\
599.97	0.00999998144295691\\
599.98	0.00999999479272987\\
599.99	0.01\\
600	0.01\\
};
\addplot [color=mycolor6,solid,forget plot]
  table[row sep=crcr]{%
0.01	0.00481014954134405\\
1.01	0.00481015070071254\\
2.01	0.00481015188410282\\
3.01	0.00481015309201312\\
4.01	0.0048101543249523\\
5.01	0.00481015558343964\\
6.01	0.0048101568680053\\
7.01	0.00481015817919046\\
8.01	0.00481015951754755\\
9.01	0.00481016088364039\\
10.01	0.00481016227804448\\
11.01	0.00481016370134746\\
12.01	0.00481016515414917\\
13.01	0.00481016663706167\\
14.01	0.0048101681507102\\
15.01	0.00481016969573249\\
16.01	0.0048101712727798\\
17.01	0.00481017288251699\\
18.01	0.00481017452562242\\
19.01	0.00481017620278913\\
20.01	0.00481017791472395\\
21.01	0.00481017966214872\\
22.01	0.00481018144580031\\
23.01	0.00481018326643066\\
24.01	0.00481018512480773\\
25.01	0.00481018702171504\\
26.01	0.00481018895795289\\
27.01	0.00481019093433795\\
28.01	0.00481019295170368\\
29.01	0.00481019501090106\\
30.01	0.00481019711279888\\
31.01	0.00481019925828392\\
32.01	0.00481020144826122\\
33.01	0.00481020368365494\\
34.01	0.0048102059654084\\
35.01	0.00481020829448453\\
36.01	0.00481021067186604\\
37.01	0.00481021309855627\\
38.01	0.00481021557557966\\
39.01	0.00481021810398161\\
40.01	0.0048102206848295\\
41.01	0.00481022331921278\\
42.01	0.00481022600824366\\
43.01	0.00481022875305748\\
44.01	0.00481023155481318\\
45.01	0.00481023441469378\\
46.01	0.00481023733390705\\
47.01	0.00481024031368568\\
48.01	0.00481024335528829\\
49.01	0.00481024645999943\\
50.01	0.00481024962913047\\
51.01	0.00481025286402029\\
52.01	0.00481025616603531\\
53.01	0.00481025953657056\\
54.01	0.00481026297705017\\
55.01	0.0048102664889277\\
56.01	0.00481027007368723\\
57.01	0.00481027373284361\\
58.01	0.0048102774679431\\
59.01	0.00481028128056438\\
60.01	0.00481028517231888\\
61.01	0.00481028914485161\\
62.01	0.0048102931998418\\
63.01	0.00481029733900368\\
64.01	0.00481030156408702\\
65.01	0.00481030587687843\\
66.01	0.00481031027920126\\
67.01	0.00481031477291716\\
68.01	0.00481031935992633\\
69.01	0.00481032404216857\\
70.01	0.00481032882162426\\
71.01	0.00481033370031486\\
72.01	0.00481033868030385\\
73.01	0.00481034376369797\\
74.01	0.00481034895264766\\
75.01	0.00481035424934825\\
76.01	0.00481035965604056\\
77.01	0.00481036517501233\\
78.01	0.00481037080859879\\
79.01	0.00481037655918383\\
80.01	0.00481038242920138\\
81.01	0.00481038842113546\\
82.01	0.00481039453752234\\
83.01	0.00481040078095088\\
84.01	0.00481040715406386\\
85.01	0.00481041365955931\\
86.01	0.00481042030019124\\
87.01	0.00481042707877116\\
88.01	0.00481043399816905\\
89.01	0.00481044106131481\\
90.01	0.00481044827119945\\
91.01	0.00481045563087631\\
92.01	0.0048104631434622\\
93.01	0.00481047081213884\\
94.01	0.00481047864015437\\
95.01	0.00481048663082471\\
96.01	0.00481049478753461\\
97.01	0.00481050311373948\\
98.01	0.00481051161296683\\
99.01	0.00481052028881776\\
100.01	0.00481052914496813\\
101.01	0.00481053818517066\\
102.01	0.00481054741325607\\
103.01	0.00481055683313504\\
104.01	0.00481056644879967\\
105.01	0.00481057626432532\\
106.01	0.00481058628387246\\
107.01	0.004810596511688\\
108.01	0.00481060695210768\\
109.01	0.00481061760955709\\
110.01	0.00481062848855444\\
111.01	0.00481063959371215\\
112.01	0.00481065092973859\\
113.01	0.00481066250144034\\
114.01	0.00481067431372413\\
115.01	0.0048106863715992\\
116.01	0.00481069868017873\\
117.01	0.00481071124468282\\
118.01	0.00481072407044042\\
119.01	0.00481073716289135\\
120.01	0.00481075052758896\\
121.01	0.00481076417020221\\
122.01	0.0048107780965182\\
123.01	0.00481079231244502\\
124.01	0.00481080682401336\\
125.01	0.0048108216373801\\
126.01	0.00481083675883011\\
127.01	0.0048108521947795\\
128.01	0.0048108679517779\\
129.01	0.00481088403651141\\
130.01	0.0048109004558058\\
131.01	0.00481091721662877\\
132.01	0.00481093432609353\\
133.01	0.00481095179146135\\
134.01	0.00481096962014495\\
135.01	0.00481098781971125\\
136.01	0.0048110063978853\\
137.01	0.0048110253625528\\
138.01	0.00481104472176394\\
139.01	0.00481106448373662\\
140.01	0.00481108465686003\\
141.01	0.00481110524969797\\
142.01	0.00481112627099319\\
143.01	0.00481114772967009\\
144.01	0.0048111696348393\\
145.01	0.00481119199580117\\
146.01	0.00481121482204984\\
147.01	0.00481123812327723\\
148.01	0.00481126190937731\\
149.01	0.00481128619045\\
150.01	0.00481131097680572\\
151.01	0.00481133627896967\\
152.01	0.00481136210768619\\
153.01	0.00481138847392362\\
154.01	0.0048114153888786\\
155.01	0.00481144286398083\\
156.01	0.00481147091089847\\
157.01	0.00481149954154214\\
158.01	0.00481152876807094\\
159.01	0.00481155860289674\\
160.01	0.00481158905869027\\
161.01	0.00481162014838588\\
162.01	0.00481165188518705\\
163.01	0.00481168428257234\\
164.01	0.00481171735430082\\
165.01	0.00481175111441771\\
166.01	0.00481178557726118\\
167.01	0.0048118207574671\\
168.01	0.00481185666997605\\
169.01	0.00481189333003998\\
170.01	0.00481193075322769\\
171.01	0.00481196895543208\\
172.01	0.00481200795287645\\
173.01	0.00481204776212178\\
174.01	0.00481208840007345\\
175.01	0.00481212988398846\\
176.01	0.00481217223148248\\
177.01	0.00481221546053767\\
178.01	0.00481225958950987\\
179.01	0.00481230463713658\\
180.01	0.00481235062254469\\
181.01	0.00481239756525864\\
182.01	0.00481244548520862\\
183.01	0.00481249440273894\\
184.01	0.0048125443386166\\
185.01	0.00481259531403974\\
186.01	0.00481264735064701\\
187.01	0.00481270047052633\\
188.01	0.00481275469622423\\
189.01	0.00481281005075523\\
190.01	0.0048128665576118\\
191.01	0.0048129242407738\\
192.01	0.00481298312471849\\
193.01	0.00481304323443138\\
194.01	0.00481310459541592\\
195.01	0.00481316723370447\\
196.01	0.00481323117586938\\
197.01	0.00481329644903362\\
198.01	0.00481336308088273\\
199.01	0.00481343109967566\\
200.01	0.00481350053425693\\
201.01	0.00481357141406868\\
202.01	0.00481364376916264\\
203.01	0.00481371763021285\\
204.01	0.0048137930285282\\
205.01	0.00481386999606585\\
206.01	0.0048139485654437\\
207.01	0.00481402876995489\\
208.01	0.00481411064358053\\
209.01	0.00481419422100451\\
210.01	0.00481427953762747\\
211.01	0.00481436662958172\\
212.01	0.00481445553374579\\
213.01	0.00481454628775976\\
214.01	0.00481463893004085\\
215.01	0.00481473349979893\\
216.01	0.00481483003705285\\
217.01	0.004814928582647\\
218.01	0.00481502917826748\\
219.01	0.00481513186645949\\
220.01	0.00481523669064492\\
221.01	0.00481534369513964\\
222.01	0.00481545292517209\\
223.01	0.00481556442690091\\
224.01	0.0048156782474344\\
225.01	0.00481579443484924\\
226.01	0.00481591303820997\\
227.01	0.0048160341075892\\
228.01	0.00481615769408726\\
229.01	0.00481628384985322\\
230.01	0.00481641262810552\\
231.01	0.0048165440831537\\
232.01	0.00481667827041972\\
233.01	0.00481681524646065\\
234.01	0.00481695506899074\\
235.01	0.00481709779690459\\
236.01	0.00481724349030065\\
237.01	0.00481739221050485\\
238.01	0.0048175440200949\\
239.01	0.00481769898292503\\
240.01	0.00481785716415088\\
241.01	0.00481801863025545\\
242.01	0.00481818344907474\\
243.01	0.00481835168982407\\
244.01	0.00481852342312557\\
245.01	0.00481869872103495\\
246.01	0.00481887765706959\\
247.01	0.00481906030623662\\
248.01	0.00481924674506191\\
249.01	0.0048194370516192\\
250.01	0.00481963130555982\\
251.01	0.00481982958814241\\
252.01	0.00482003198226424\\
253.01	0.00482023857249204\\
254.01	0.00482044944509324\\
255.01	0.00482066468806844\\
256.01	0.00482088439118374\\
257.01	0.00482110864600375\\
258.01	0.00482133754592508\\
259.01	0.00482157118621034\\
260.01	0.0048218096640221\\
261.01	0.0048220530784582\\
262.01	0.00482230153058678\\
263.01	0.00482255512348202\\
264.01	0.00482281396226006\\
265.01	0.00482307815411577\\
266.01	0.00482334780835965\\
267.01	0.00482362303645502\\
268.01	0.00482390395205565\\
269.01	0.00482419067104402\\
270.01	0.00482448331156941\\
271.01	0.00482478199408652\\
272.01	0.00482508684139468\\
273.01	0.00482539797867679\\
274.01	0.00482571553353858\\
275.01	0.0048260396360484\\
276.01	0.00482637041877684\\
277.01	0.00482670801683639\\
278.01	0.0048270525679215\\
279.01	0.0048274042123484\\
280.01	0.00482776309309454\\
281.01	0.00482812935583878\\
282.01	0.00482850314900027\\
283.01	0.00482888462377816\\
284.01	0.00482927393419044\\
285.01	0.00482967123711206\\
286.01	0.00483007669231331\\
287.01	0.00483049046249705\\
288.01	0.00483091271333523\\
289.01	0.00483134361350471\\
290.01	0.00483178333472237\\
291.01	0.00483223205177861\\
292.01	0.00483268994257\\
293.01	0.00483315718813013\\
294.01	0.00483363397265954\\
295.01	0.00483412048355308\\
296.01	0.00483461691142635\\
297.01	0.0048351234501388\\
298.01	0.00483564029681527\\
299.01	0.00483616765186451\\
300.01	0.00483670571899504\\
301.01	0.00483725470522769\\
302.01	0.00483781482090476\\
303.01	0.004838386279695\\
304.01	0.00483896929859511\\
305.01	0.00483956409792581\\
306.01	0.00484017090132424\\
307.01	0.00484078993572963\\
308.01	0.00484142143136446\\
309.01	0.00484206562170798\\
310.01	0.00484272274346456\\
311.01	0.00484339303652351\\
312.01	0.00484407674391166\\
313.01	0.00484477411173703\\
314.01	0.00484548538912395\\
315.01	0.00484621082813835\\
316.01	0.00484695068370171\\
317.01	0.00484770521349595\\
318.01	0.00484847467785354\\
319.01	0.00484925933963751\\
320.01	0.00485005946410659\\
321.01	0.00485087531876649\\
322.01	0.00485170717320614\\
323.01	0.00485255529891743\\
324.01	0.0048534199690986\\
325.01	0.0048543014584399\\
326.01	0.00485520004289085\\
327.01	0.00485611599940768\\
328.01	0.00485704960568156\\
329.01	0.00485800113984573\\
330.01	0.00485897088016164\\
331.01	0.00485995910468322\\
332.01	0.0048609660908997\\
333.01	0.00486199211535643\\
334.01	0.00486303745325403\\
335.01	0.00486410237802692\\
336.01	0.00486518716090109\\
337.01	0.00486629207043406\\
338.01	0.00486741737203751\\
339.01	0.00486856332748696\\
340.01	0.00486973019442055\\
341.01	0.00487091822583183\\
342.01	0.00487212766956283\\
343.01	0.00487335876780355\\
344.01	0.00487461175660618\\
345.01	0.00487588686542503\\
346.01	0.00487718431669312\\
347.01	0.00487850432545109\\
348.01	0.00487984709904424\\
349.01	0.00488121283690827\\
350.01	0.00488260173046647\\
351.01	0.00488401396316547\\
352.01	0.00488544971067974\\
353.01	0.00488690914131983\\
354.01	0.00488839241668426\\
355.01	0.00488989969259808\\
356.01	0.00489143112038896\\
357.01	0.00489298684855277\\
358.01	0.00489456702486797\\
359.01	0.00489617179902103\\
360.01	0.00489780132580704\\
361.01	0.00489945576897079\\
362.01	0.00490113530575392\\
363.01	0.00490284013220595\\
364.01	0.0049045704693079\\
365.01	0.00490632656994297\\
366.01	0.00490810872672136\\
367.01	0.00490991728063166\\
368.01	0.00491175263044519\\
369.01	0.00491361524273022\\
370.01	0.0049155056622537\\
371.01	0.00491742452243475\\
372.01	0.00491937255538659\\
373.01	0.00492135060091869\\
374.01	0.00492335961368574\\
375.01	0.00492540066746114\\
376.01	0.00492747495529421\\
377.01	0.00492958378410742\\
378.01	0.00493172856214356\\
379.01	0.00493391077765498\\
380.01	0.00493613196744747\\
381.01	0.00493839367453429\\
382.01	0.00494069739548804\\
383.01	0.00494304452051163\\
384.01	0.00494543627341122\\
385.01	0.0049478736656129\\
386.01	0.00495035751517274\\
387.01	0.00495288858523196\\
388.01	0.00495546764281225\\
389.01	0.00495809545977625\\
390.01	0.00496077281234379\\
391.01	0.00496350048057054\\
392.01	0.00496627924778534\\
393.01	0.00496910989998675\\
394.01	0.00497199322519443\\
395.01	0.00497493001275608\\
396.01	0.0049779210526074\\
397.01	0.00498096713448271\\
398.01	0.00498406904707539\\
399.01	0.00498722757714884\\
400.01	0.00499044350859192\\
401.01	0.00499371762142346\\
402.01	0.0049970506907422\\
403.01	0.00500044348562147\\
404.01	0.0050038967679506\\
405.01	0.00500741129122183\\
406.01	0.00501098779926511\\
407.01	0.0050146270249312\\
408.01	0.00501832968872651\\
409.01	0.00502209649740228\\
410.01	0.00502592814250234\\
411.01	0.00502982529887465\\
412.01	0.00503378862315343\\
413.01	0.00503781875221904\\
414.01	0.00504191630164617\\
415.01	0.00504608186414991\\
416.01	0.00505031600804466\\
417.01	0.00505461927572939\\
418.01	0.00505899218221903\\
419.01	0.0050634352137407\\
420.01	0.00506794882641948\\
421.01	0.00507253344507952\\
422.01	0.00507718946219067\\
423.01	0.00508191723699541\\
424.01	0.0050867170948518\\
425.01	0.00509158932683731\\
426.01	0.00509653418965687\\
427.01	0.00510155190590753\\
428.01	0.00510664266475552\\
429.01	0.00511180662308321\\
430.01	0.00511704390717192\\
431.01	0.0051223546149865\\
432.01	0.00512773881913381\\
433.01	0.00513319657056684\\
434.01	0.00513872790310789\\
435.01	0.00514433283886527\\
436.01	0.00515001139461045\\
437.01	0.00515576358918318\\
438.01	0.00516158945197483\\
439.01	0.00516748903253372\\
440.01	0.00517346241131204\\
441.01	0.00517950971155169\\
442.01	0.00518563111227167\\
443.01	0.0051918268622821\\
444.01	0.00519809729509522\\
445.01	0.00520444284454731\\
446.01	0.00521086406087209\\
447.01	0.00521736162688377\\
448.01	0.00522393637383599\\
449.01	0.00523058929641744\\
450.01	0.005237321566239\\
451.01	0.0052441345430564\\
452.01	0.00525102978286511\\
453.01	0.00525800904192067\\
454.01	0.00526507427567648\\
455.01	0.00527222763162986\\
456.01	0.00527947143513814\\
457.01	0.00528680816744723\\
458.01	0.00529424043550476\\
459.01	0.00530177093365197\\
460.01	0.00530940239803307\\
461.01	0.00531713755558978\\
462.01	0.00532497907080687\\
463.01	0.0053329294949338\\
464.01	0.0053409912240989\\
465.01	0.00534916647429395\\
466.01	0.00535745728213249\\
467.01	0.00536586553963554\\
468.01	0.00537439306754374\\
469.01	0.00538304172107079\\
470.01	0.00539181348436453\\
471.01	0.00540071050516695\\
472.01	0.00540973510763672\\
473.01	0.00541888980385066\\
474.01	0.00542817730434367\\
475.01	0.00543760052735374\\
476.01	0.00544716260641621\\
477.01	0.0054568668959398\\
478.01	0.00546671697440167\\
479.01	0.00547671664481987\\
480.01	0.00548686993220562\\
481.01	0.00549718107777489\\
482.01	0.00550765452980419\\
483.01	0.00551829493116719\\
484.01	0.00552910710377401\\
485.01	0.00554009603037293\\
486.01	0.00555126683443509\\
487.01	0.00556262475914229\\
488.01	0.00557417514679082\\
489.01	0.00558592342019337\\
490.01	0.00559787506784526\\
491.01	0.00561003563466669\\
492.01	0.00562241071994367\\
493.01	0.00563500598358412\\
494.01	0.00564782716088615\\
495.01	0.00566088008463324\\
496.01	0.0056741707115195\\
497.01	0.00568770514791241\\
498.01	0.00570148966840113\\
499.01	0.00571553072125539\\
500.01	0.0057298349224963\\
501.01	0.00574440904645443\\
502.01	0.00575926001579341\\
503.01	0.00577439489153707\\
504.01	0.00578982086358361\\
505.01	0.00580554524222467\\
506.01	0.00582157545117937\\
507.01	0.00583791902261407\\
508.01	0.00585458359451178\\
509.01	0.00587157691059846\\
510.01	0.00588890682281078\\
511.01	0.00590658129601832\\
512.01	0.00592460841441665\\
513.01	0.0059429963887318\\
514.01	0.00596175356318732\\
515.01	0.00598088842117465\\
516.01	0.00600040958882743\\
517.01	0.00602032583630079\\
518.01	0.00604064607744553\\
519.01	0.00606137936926249\\
520.01	0.0060825349121271\\
521.01	0.00610412205098902\\
522.01	0.00612615027750338\\
523.01	0.00614862923297689\\
524.01	0.00617156871194051\\
525.01	0.00619497866610692\\
526.01	0.00621886920843724\\
527.01	0.00624325061704201\\
528.01	0.00626813333868453\\
529.01	0.00629352799173694\\
530.01	0.00631944536855689\\
531.01	0.00634589643737332\\
532.01	0.00637289234385549\\
533.01	0.00640044441253146\\
534.01	0.00642856414810232\\
535.01	0.00645726323657654\\
536.01	0.00648655354609155\\
537.01	0.00651644712729382\\
538.01	0.0065469562131532\\
539.01	0.00657809321810346\\
540.01	0.00660987073641983\\
541.01	0.00664230153976614\\
542.01	0.00667539857385353\\
543.01	0.00670917495415527\\
544.01	0.0067436439606068\\
545.01	0.00677881903119029\\
546.01	0.00681471375427505\\
547.01	0.00685134185956257\\
548.01	0.00688871720747539\\
549.01	0.00692685377682434\\
550.01	0.00696576565057831\\
551.01	0.00700546699954922\\
552.01	0.00704597206379078\\
553.01	0.00708729513148984\\
554.01	0.00712945051510259\\
555.01	0.00717245252446584\\
556.01	0.00721631543658319\\
557.01	0.00726105346176332\\
558.01	0.00730668070575833\\
559.01	0.00735321112752477\\
560.01	0.0074006584921987\\
561.01	0.00744903631884569\\
562.01	0.00749835782251346\\
563.01	0.00754863585008057\\
564.01	0.00759988280936222\\
565.01	0.00765211059090243\\
566.01	0.00770533048185168\\
567.01	0.0077595530713051\\
568.01	0.00781478814645575\\
569.01	0.00787104457890665\\
570.01	0.00792833020048318\\
571.01	0.0079866516679058\\
572.01	0.00804601431571488\\
573.01	0.00810642199690455\\
574.01	0.0081678769108208\\
575.01	0.00823037941802071\\
576.01	0.00829392784199252\\
577.01	0.0083585182579118\\
578.01	0.00842414426897704\\
579.01	0.00849079677135603\\
580.01	0.00855846370941088\\
581.01	0.00862712982369113\\
582.01	0.0086967763952409\\
583.01	0.00876738099111419\\
584.01	0.00883891721770022\\
585.01	0.00891135449062501\\
586.01	0.00898465783271585\\
587.01	0.00905878771493402\\
588.01	0.00913369995946198\\
589.01	0.00920934572947927\\
590.01	0.00928567163683064\\
591.01	0.00936262000709249\\
592.01	0.00944012935186362\\
593.01	0.00951813511091667\\
594.01	0.00959657074273127\\
595.01	0.00967536926160508\\
596.01	0.00975446534388769\\
597.01	0.0098337539663526\\
598.01	0.00990866201848071\\
599.01	0.00997087280416276\\
599.02	0.00997138072163725\\
599.03	0.009971885576258\\
599.04	0.00997238733820211\\
599.05	0.00997288597735272\\
599.06	0.00997338146329604\\
599.07	0.00997387376531845\\
599.08	0.00997436285240349\\
599.09	0.00997484869322892\\
599.1	0.00997533125616361\\
599.11	0.00997581050926455\\
599.12	0.00997628642027372\\
599.13	0.00997675895661497\\
599.14	0.0099772280853909\\
599.15	0.00997769377337961\\
599.16	0.00997815598703157\\
599.17	0.00997861469246631\\
599.18	0.00997906985546915\\
599.19	0.00997952144148792\\
599.2	0.00997996941562957\\
599.21	0.00998041374265684\\
599.22	0.0099808543869848\\
599.23	0.00998129131267744\\
599.24	0.00998172448344418\\
599.25	0.00998215386263636\\
599.26	0.00998257941258353\\
599.27	0.00998300109279825\\
599.28	0.00998341886238981\\
599.29	0.00998383268006024\\
599.3	0.00998424250410032\\
599.31	0.00998464829238543\\
599.32	0.00998505000237151\\
599.33	0.00998544759109087\\
599.34	0.00998584101514799\\
599.35	0.00998623023071532\\
599.36	0.00998661519352895\\
599.37	0.00998699585888436\\
599.38	0.00998737218163197\\
599.39	0.00998774411617281\\
599.4	0.00998811161645403\\
599.41	0.00998847463596443\\
599.42	0.0099888331277299\\
599.43	0.00998918704430885\\
599.44	0.00998953633778757\\
599.45	0.00998988095977558\\
599.46	0.00999022086140088\\
599.47	0.00999055599330522\\
599.48	0.00999088630563925\\
599.49	0.00999121174805768\\
599.5	0.00999153226971435\\
599.51	0.0099918478192573\\
599.52	0.00999215834482374\\
599.53	0.00999246379403503\\
599.54	0.0099927641139915\\
599.55	0.00999305925126738\\
599.56	0.00999334915190553\\
599.57	0.0099936337614122\\
599.58	0.00999391302475173\\
599.59	0.00999418688634118\\
599.6	0.00999445529004489\\
599.61	0.00999471817916904\\
599.62	0.00999497549645613\\
599.63	0.00999522718407934\\
599.64	0.009995473183637\\
599.65	0.00999571343614678\\
599.66	0.00999594788204004\\
599.67	0.00999617646115599\\
599.68	0.0099963991127358\\
599.69	0.00999661577541675\\
599.7	0.00999682638722618\\
599.71	0.00999703088557551\\
599.72	0.00999722920725411\\
599.73	0.00999742128842315\\
599.74	0.00999760706460942\\
599.75	0.00999778647069897\\
599.76	0.00999795944093086\\
599.77	0.0099981259088907\\
599.78	0.0099982858075042\\
599.79	0.00999843906903064\\
599.8	0.00999858562505627\\
599.81	0.00999872540648767\\
599.82	0.00999885834354498\\
599.83	0.00999898436575515\\
599.84	0.00999910340194508\\
599.85	0.00999921538023465\\
599.86	0.00999932022802977\\
599.87	0.00999941787201528\\
599.88	0.00999950823814785\\
599.89	0.00999959125164875\\
599.9	0.00999966683699656\\
599.91	0.00999973491791987\\
599.92	0.0099997954173898\\
599.93	0.00999984825761255\\
599.94	0.00999989336002181\\
599.95	0.00999993064527112\\
599.96	0.00999996003322615\\
599.97	0.00999998144295691\\
599.98	0.00999999479272987\\
599.99	0.01\\
600	0.01\\
};
\addplot [color=mycolor7,solid,forget plot]
  table[row sep=crcr]{%
0.01	0.00464834445259417\\
1.01	0.00464834564822231\\
2.01	0.00464834686873988\\
3.01	0.0046483481146663\\
4.01	0.00464834938653171\\
5.01	0.00464835068487754\\
6.01	0.00464835201025615\\
7.01	0.00464835336323188\\
8.01	0.00464835474438058\\
9.01	0.00464835615429034\\
10.01	0.00464835759356162\\
11.01	0.00464835906280725\\
12.01	0.00464836056265287\\
13.01	0.00464836209373737\\
14.01	0.00464836365671296\\
15.01	0.00464836525224556\\
16.01	0.00464836688101485\\
17.01	0.00464836854371492\\
18.01	0.00464837024105429\\
19.01	0.00464837197375634\\
20.01	0.00464837374255971\\
21.01	0.00464837554821833\\
22.01	0.0046483773915021\\
23.01	0.00464837927319689\\
24.01	0.00464838119410517\\
25.01	0.00464838315504608\\
26.01	0.00464838515685599\\
27.01	0.00464838720038867\\
28.01	0.0046483892865162\\
29.01	0.00464839141612844\\
30.01	0.0046483935901341\\
31.01	0.00464839580946098\\
32.01	0.00464839807505615\\
33.01	0.00464840038788687\\
34.01	0.00464840274894025\\
35.01	0.00464840515922424\\
36.01	0.00464840761976822\\
37.01	0.00464841013162275\\
38.01	0.00464841269586058\\
39.01	0.00464841531357703\\
40.01	0.00464841798589027\\
41.01	0.00464842071394191\\
42.01	0.00464842349889752\\
43.01	0.00464842634194722\\
44.01	0.0046484292443059\\
45.01	0.00464843220721413\\
46.01	0.00464843523193825\\
47.01	0.00464843831977162\\
48.01	0.00464844147203429\\
49.01	0.00464844469007421\\
50.01	0.00464844797526748\\
51.01	0.0046484513290191\\
52.01	0.00464845475276383\\
53.01	0.00464845824796627\\
54.01	0.00464846181612165\\
55.01	0.00464846545875701\\
56.01	0.00464846917743099\\
57.01	0.00464847297373523\\
58.01	0.00464847684929467\\
59.01	0.00464848080576842\\
60.01	0.00464848484485038\\
61.01	0.00464848896826998\\
62.01	0.00464849317779312\\
63.01	0.00464849747522268\\
64.01	0.0046485018623994\\
65.01	0.00464850634120256\\
66.01	0.00464851091355128\\
67.01	0.00464851558140465\\
68.01	0.00464852034676305\\
69.01	0.00464852521166893\\
70.01	0.00464853017820752\\
71.01	0.00464853524850789\\
72.01	0.00464854042474395\\
73.01	0.00464854570913519\\
74.01	0.00464855110394777\\
75.01	0.00464855661149552\\
76.01	0.00464856223414092\\
77.01	0.00464856797429605\\
78.01	0.00464857383442366\\
79.01	0.00464857981703851\\
80.01	0.00464858592470777\\
81.01	0.00464859216005296\\
82.01	0.00464859852575081\\
83.01	0.00464860502453399\\
84.01	0.00464861165919278\\
85.01	0.00464861843257618\\
86.01	0.0046486253475931\\
87.01	0.0046486324072136\\
88.01	0.00464863961447035\\
89.01	0.00464864697245969\\
90.01	0.00464865448434311\\
91.01	0.00464866215334847\\
92.01	0.00464866998277179\\
93.01	0.00464867797597848\\
94.01	0.0046486861364045\\
95.01	0.00464869446755817\\
96.01	0.00464870297302199\\
97.01	0.00464871165645349\\
98.01	0.00464872052158731\\
99.01	0.00464872957223671\\
100.01	0.00464873881229512\\
101.01	0.00464874824573808\\
102.01	0.00464875787662486\\
103.01	0.00464876770910008\\
104.01	0.00464877774739562\\
105.01	0.00464878799583257\\
106.01	0.00464879845882279\\
107.01	0.00464880914087151\\
108.01	0.00464882004657829\\
109.01	0.00464883118064019\\
110.01	0.00464884254785258\\
111.01	0.00464885415311218\\
112.01	0.00464886600141887\\
113.01	0.00464887809787795\\
114.01	0.00464889044770213\\
115.01	0.00464890305621371\\
116.01	0.00464891592884785\\
117.01	0.00464892907115352\\
118.01	0.00464894248879703\\
119.01	0.00464895618756395\\
120.01	0.00464897017336184\\
121.01	0.00464898445222281\\
122.01	0.0046489990303062\\
123.01	0.00464901391390093\\
124.01	0.00464902910942883\\
125.01	0.00464904462344685\\
126.01	0.00464906046265044\\
127.01	0.00464907663387601\\
128.01	0.00464909314410422\\
129.01	0.00464911000046296\\
130.01	0.00464912721023041\\
131.01	0.00464914478083822\\
132.01	0.00464916271987477\\
133.01	0.00464918103508856\\
134.01	0.00464919973439162\\
135.01	0.00464921882586271\\
136.01	0.0046492383177511\\
137.01	0.00464925821848014\\
138.01	0.00464927853665087\\
139.01	0.00464929928104579\\
140.01	0.0046493204606327\\
141.01	0.00464934208456876\\
142.01	0.00464936416220404\\
143.01	0.00464938670308643\\
144.01	0.00464940971696495\\
145.01	0.00464943321379452\\
146.01	0.0046494572037403\\
147.01	0.00464948169718185\\
148.01	0.00464950670471789\\
149.01	0.00464953223717089\\
150.01	0.00464955830559181\\
151.01	0.00464958492126498\\
152.01	0.00464961209571292\\
153.01	0.00464963984070133\\
154.01	0.00464966816824472\\
155.01	0.004649697090611\\
156.01	0.00464972662032705\\
157.01	0.00464975677018478\\
158.01	0.00464978755324576\\
159.01	0.0046498189828479\\
160.01	0.00464985107261046\\
161.01	0.00464988383644051\\
162.01	0.00464991728853896\\
163.01	0.0046499514434067\\
164.01	0.00464998631585091\\
165.01	0.00465002192099168\\
166.01	0.00465005827426837\\
167.01	0.00465009539144685\\
168.01	0.00465013328862596\\
169.01	0.00465017198224481\\
170.01	0.00465021148908977\\
171.01	0.00465025182630202\\
172.01	0.0046502930113851\\
173.01	0.0046503350622123\\
174.01	0.0046503779970348\\
175.01	0.00465042183448953\\
176.01	0.00465046659360724\\
177.01	0.00465051229382124\\
178.01	0.00465055895497543\\
179.01	0.00465060659733316\\
180.01	0.00465065524158635\\
181.01	0.00465070490886434\\
182.01	0.0046507556207433\\
183.01	0.00465080739925518\\
184.01	0.00465086026689822\\
185.01	0.00465091424664639\\
186.01	0.00465096936195923\\
187.01	0.00465102563679255\\
188.01	0.00465108309560855\\
189.01	0.00465114176338727\\
190.01	0.00465120166563647\\
191.01	0.00465126282840389\\
192.01	0.00465132527828799\\
193.01	0.00465138904244983\\
194.01	0.00465145414862494\\
195.01	0.00465152062513614\\
196.01	0.00465158850090458\\
197.01	0.00465165780546387\\
198.01	0.00465172856897206\\
199.01	0.00465180082222541\\
200.01	0.0046518745966718\\
201.01	0.00465194992442446\\
202.01	0.00465202683827612\\
203.01	0.0046521053717135\\
204.01	0.00465218555893182\\
205.01	0.00465226743484987\\
206.01	0.00465235103512567\\
207.01	0.0046524363961717\\
208.01	0.0046525235551711\\
209.01	0.00465261255009412\\
210.01	0.00465270341971445\\
211.01	0.00465279620362667\\
212.01	0.00465289094226341\\
213.01	0.00465298767691322\\
214.01	0.00465308644973861\\
215.01	0.00465318730379469\\
216.01	0.00465329028304812\\
217.01	0.00465339543239622\\
218.01	0.00465350279768705\\
219.01	0.00465361242573939\\
220.01	0.00465372436436307\\
221.01	0.00465383866238059\\
222.01	0.00465395536964798\\
223.01	0.00465407453707718\\
224.01	0.00465419621665816\\
225.01	0.00465432046148176\\
226.01	0.00465444732576344\\
227.01	0.00465457686486646\\
228.01	0.00465470913532673\\
229.01	0.00465484419487774\\
230.01	0.00465498210247563\\
231.01	0.00465512291832523\\
232.01	0.00465526670390652\\
233.01	0.00465541352200186\\
234.01	0.00465556343672337\\
235.01	0.00465571651354121\\
236.01	0.00465587281931247\\
237.01	0.0046560324223103\\
238.01	0.00465619539225437\\
239.01	0.00465636180034099\\
240.01	0.00465653171927498\\
241.01	0.00465670522330118\\
242.01	0.00465688238823729\\
243.01	0.00465706329150724\\
244.01	0.00465724801217507\\
245.01	0.00465743663097969\\
246.01	0.00465762923037064\\
247.01	0.00465782589454374\\
248.01	0.00465802670947869\\
249.01	0.0046582317629764\\
250.01	0.00465844114469749\\
251.01	0.00465865494620205\\
252.01	0.0046588732609894\\
253.01	0.00465909618453894\\
254.01	0.00465932381435288\\
255.01	0.00465955624999767\\
256.01	0.00465979359314861\\
257.01	0.00466003594763367\\
258.01	0.00466028341947942\\
259.01	0.00466053611695691\\
260.01	0.00466079415062931\\
261.01	0.00466105763339973\\
262.01	0.00466132668056077\\
263.01	0.00466160140984465\\
264.01	0.00466188194147421\\
265.01	0.0046621683982155\\
266.01	0.00466246090543077\\
267.01	0.00466275959113274\\
268.01	0.00466306458604027\\
269.01	0.00466337602363437\\
270.01	0.00466369404021644\\
271.01	0.00466401877496626\\
272.01	0.00466435037000225\\
273.01	0.00466468897044188\\
274.01	0.00466503472446445\\
275.01	0.00466538778337351\\
276.01	0.00466574830166176\\
277.01	0.00466611643707615\\
278.01	0.00466649235068493\\
279.01	0.00466687620694507\\
280.01	0.00466726817377179\\
281.01	0.00466766842260811\\
282.01	0.00466807712849658\\
283.01	0.00466849447015159\\
284.01	0.00466892063003287\\
285.01	0.00466935579442034\\
286.01	0.00466980015348983\\
287.01	0.00467025390138999\\
288.01	0.00467071723632011\\
289.01	0.00467119036060909\\
290.01	0.00467167348079517\\
291.01	0.00467216680770687\\
292.01	0.00467267055654463\\
293.01	0.00467318494696309\\
294.01	0.00467371020315407\\
295.01	0.00467424655393085\\
296.01	0.0046747942328116\\
297.01	0.00467535347810466\\
298.01	0.00467592453299313\\
299.01	0.0046765076456199\\
300.01	0.00467710306917288\\
301.01	0.0046777110619695\\
302.01	0.00467833188754116\\
303.01	0.00467896581471694\\
304.01	0.00467961311770609\\
305.01	0.0046802740761803\\
306.01	0.00468094897535276\\
307.01	0.00468163810605712\\
308.01	0.0046823417648227\\
309.01	0.00468306025394839\\
310.01	0.00468379388157174\\
311.01	0.0046845429617358\\
312.01	0.00468530781445062\\
313.01	0.0046860887657502\\
314.01	0.00468688614774332\\
315.01	0.00468770029865807\\
316.01	0.004688531562879\\
317.01	0.00468938029097496\\
318.01	0.00469024683971839\\
319.01	0.0046911315720933\\
320.01	0.00469203485729067\\
321.01	0.00469295707069094\\
322.01	0.00469389859383011\\
323.01	0.00469485981435\\
324.01	0.00469584112592815\\
325.01	0.00469684292818687\\
326.01	0.00469786562657859\\
327.01	0.00469890963224451\\
328.01	0.00469997536184386\\
329.01	0.00470106323735046\\
330.01	0.00470217368581355\\
331.01	0.00470330713907838\\
332.01	0.00470446403346253\\
333.01	0.00470564480938383\\
334.01	0.0047068499109341\\
335.01	0.00470807978539418\\
336.01	0.00470933488268387\\
337.01	0.00471061565474067\\
338.01	0.00471192255482068\\
339.01	0.00471325603671385\\
340.01	0.00471461655386707\\
341.01	0.00471600455840571\\
342.01	0.00471742050004591\\
343.01	0.00471886482488885\\
344.01	0.0047203379740879\\
345.01	0.0047218403823789\\
346.01	0.0047233724764662\\
347.01	0.00472493467325462\\
348.01	0.00472652737792041\\
349.01	0.00472815098181462\\
350.01	0.00472980586019424\\
351.01	0.00473149236977929\\
352.01	0.00473321084613727\\
353.01	0.00473496160090216\\
354.01	0.00473674491883976\\
355.01	0.0047385610547811\\
356.01	0.00474041023045556\\
357.01	0.00474229263126748\\
358.01	0.00474420840307992\\
359.01	0.00474615764908549\\
360.01	0.00474814042687612\\
361.01	0.00475015674584822\\
362.01	0.00475220656512587\\
363.01	0.00475428979222349\\
364.01	0.00475640628273112\\
365.01	0.00475855584136699\\
366.01	0.00476073822481682\\
367.01	0.00476295314686802\\
368.01	0.00476520028643795\\
369.01	0.00476747929920025\\
370.01	0.00476978983361274\\
371.01	0.00477213155224933\\
372.01	0.00477450415940651\\
373.01	0.00477690743598608\\
374.01	0.00477934128259452\\
375.01	0.00478180577161051\\
376.01	0.00478430120855663\\
377.01	0.00478682820236268\\
378.01	0.00478938774284642\\
379.01	0.00479198128171347\\
380.01	0.00479461081023144\\
381.01	0.00479727892195163\\
382.01	0.00479998884170042\\
383.01	0.00480274439151386\\
384.01	0.00480554984876812\\
385.01	0.00480840960580881\\
386.01	0.00481132663638383\\
387.01	0.0048143021350969\\
388.01	0.00481733718293635\\
389.01	0.00482043287460097\\
390.01	0.00482359031817933\\
391.01	0.00482681063478051\\
392.01	0.00483009495811014\\
393.01	0.00483344443398835\\
394.01	0.00483686021980408\\
395.01	0.00484034348390203\\
396.01	0.00484389540489522\\
397.01	0.00484751717089908\\
398.01	0.00485120997867994\\
399.01	0.00485497503271119\\
400.01	0.00485881354413251\\
401.01	0.0048627267296011\\
402.01	0.00486671581003152\\
403.01	0.00487078200921302\\
404.01	0.00487492655229703\\
405.01	0.00487915066414647\\
406.01	0.00488345556753692\\
407.01	0.00488784248120079\\
408.01	0.00489231261770388\\
409.01	0.00489686718114493\\
410.01	0.00490150736466798\\
411.01	0.00490623434777614\\
412.01	0.00491104929343796\\
413.01	0.00491595334497498\\
414.01	0.0049209476227208\\
415.01	0.00492603322044267\\
416.01	0.00493121120151686\\
417.01	0.00493648259485009\\
418.01	0.00494184839054121\\
419.01	0.00494730953527837\\
420.01	0.00495286692747109\\
421.01	0.00495852141211699\\
422.01	0.00496427377540952\\
423.01	0.00497012473909408\\
424.01	0.00497607495459103\\
425.01	0.00498212499690281\\
426.01	0.00498827535833897\\
427.01	0.00499452644209535\\
428.01	0.00500087855573869\\
429.01	0.00500733190466011\\
430.01	0.00501388658557486\\
431.01	0.00502054258016595\\
432.01	0.00502729974898626\\
433.01	0.00503415782575883\\
434.01	0.00504111641224168\\
435.01	0.00504817497385036\\
436.01	0.00505533283626679\\
437.01	0.00506258918329626\\
438.01	0.00506994305627546\\
439.01	0.00507739335537467\\
440.01	0.00508493884317916\\
441.01	0.00509257815097973\\
442.01	0.00510030978824515\\
443.01	0.00510813215578212\\
444.01	0.00511604356312524\\
445.01	0.00512404225070982\\
446.01	0.00513212641738443\\
447.01	0.00514029425378412\\
448.01	0.00514854398202269\\
449.01	0.00515687390203937\\
450.01	0.00516528244474876\\
451.01	0.0051737682318643\\
452.01	0.00518233014188449\\
453.01	0.00519096738119998\\
454.01	0.00519967955859978\\
455.01	0.00520846676056924\\
456.01	0.00521732962368783\\
457.01	0.00522626939911519\\
458.01	0.00523528800263791\\
459.01	0.00524438804207301\\
460.01	0.00525357281213593\\
461.01	0.00526284624540482\\
462.01	0.00527221280719085\\
463.01	0.00528167732264719\\
464.01	0.00529124472744835\\
465.01	0.00530091974060239\\
466.01	0.0053107064721192\\
467.01	0.00532060800344458\\
468.01	0.00533062602087934\\
469.01	0.00534076072198664\\
470.01	0.00535101162228553\\
471.01	0.00536137827533253\\
472.01	0.0053718603558916\\
473.01	0.00538245769485676\\
474.01	0.00539317031688376\\
475.01	0.0054039984803502\\
476.01	0.00541494271909335\\
477.01	0.00542600388516891\\
478.01	0.00543718319164449\\
479.01	0.00544848225418493\\
480.01	0.00545990312991866\\
481.01	0.00547144835179354\\
482.01	0.00548312095635677\\
483.01	0.00549492450265347\\
484.01	0.00550686307978838\\
485.01	0.00551894130066258\\
486.01	0.00553116427957259\\
487.01	0.00554353759180509\\
488.01	0.00555606721418994\\
489.01	0.00556875944688499\\
490.01	0.00558162081857003\\
491.01	0.0055946579797886\\
492.01	0.00560787759240927\\
493.01	0.00562128622692798\\
494.01	0.00563489028319811\\
495.01	0.00564869595329159\\
496.01	0.00566270924591865\\
497.01	0.00567693608728224\\
498.01	0.00569138249822781\\
499.01	0.00570605479287023\\
500.01	0.00572095966238625\\
501.01	0.00573610416924741\\
502.01	0.00575149572772143\\
503.01	0.00576714207787485\\
504.01	0.00578305125354241\\
505.01	0.00579923154518509\\
506.01	0.00581569145909275\\
507.01	0.00583243967495409\\
508.01	0.00584948500437797\\
509.01	0.00586683635341503\\
510.01	0.00588450269239885\\
511.01	0.00590249303634946\\
512.01	0.00592081643858883\\
513.01	0.00593948199892681\\
514.01	0.00595849888564073\\
515.01	0.00597787636749476\\
516.01	0.00599762384849975\\
517.01	0.0060177508949207\\
518.01	0.00603826724368112\\
519.01	0.00605918279183401\\
520.01	0.00608050758068721\\
521.01	0.00610225178171825\\
522.01	0.006124425685514\\
523.01	0.00614703969441464\\
524.01	0.00617010431929348\\
525.01	0.00619363018054338\\
526.01	0.00621762801289089\\
527.01	0.00624210867316072\\
528.01	0.00626708314964169\\
529.01	0.00629256257138429\\
530.01	0.00631855821573893\\
531.01	0.00634508151289306\\
532.01	0.00637214404718316\\
533.01	0.00639975755639791\\
534.01	0.00642793393108233\\
535.01	0.00645668521483311\\
536.01	0.00648602360550507\\
537.01	0.00651596145698026\\
538.01	0.00654651128106563\\
539.01	0.00657768574904344\\
540.01	0.0066094976924182\\
541.01	0.0066419601024892\\
542.01	0.00667508612853107\\
543.01	0.00670888907454924\\
544.01	0.00674338239474022\\
545.01	0.00677857968783409\\
546.01	0.00681449469037443\\
547.01	0.00685114126877864\\
548.01	0.00688853340991637\\
549.01	0.00692668520993533\\
550.01	0.00696561086108091\\
551.01	0.00700532463627988\\
552.01	0.00704584087128132\\
553.01	0.00708717394416014\\
554.01	0.00712933825198563\\
555.01	0.00717234818441871\\
556.01	0.00721621809395735\\
557.01	0.00726096226249309\\
558.01	0.00730659486381002\\
559.01	0.00735312992163711\\
560.01	0.00740058126284496\\
561.01	0.00744896246535642\\
562.01	0.00749828680030934\\
563.01	0.00754856716798113\\
564.01	0.00759981602694357\\
565.01	0.00765204531588342\\
566.01	0.00770526636749114\\
567.01	0.00775948981379601\\
568.01	0.00781472548230801\\
569.01	0.00787098228231643\\
570.01	0.00792826808069656\\
571.01	0.00798658956658903\\
572.01	0.00804595210435032\\
573.01	0.00810635957423606\\
574.01	0.00816781420037203\\
575.01	0.0082303163657128\\
576.01	0.00829386441388749\\
577.01	0.00835845443810524\\
578.01	0.00842408005766179\\
579.01	0.00849073218307326\\
580.01	0.00855839877149774\\
581.01	0.00862706457492604\\
582.01	0.00869671088467823\\
583.01	0.00876731527708814\\
584.01	0.00883885136696868\\
585.01	0.00891128857761011\\
586.01	0.00898459193878753\\
587.01	0.00905872192767049\\
588.01	0.00913363437180803\\
589.01	0.00920928043871075\\
590.01	0.00928560674321885\\
591.01	0.00936255561214592\\
592.01	0.00944006555600452\\
593.01	0.00951807201042488\\
594.01	0.00959650842575684\\
595.01	0.00967530780301053\\
596.01	0.00975440479863077\\
597.01	0.00983371536751284\\
598.01	0.00990866201848071\\
599.01	0.00997087280416276\\
599.02	0.00997138072163725\\
599.03	0.009971885576258\\
599.04	0.00997238733820211\\
599.05	0.00997288597735272\\
599.06	0.00997338146329604\\
599.07	0.00997387376531845\\
599.08	0.00997436285240349\\
599.09	0.00997484869322892\\
599.1	0.00997533125616361\\
599.11	0.00997581050926455\\
599.12	0.00997628642027372\\
599.13	0.00997675895661497\\
599.14	0.0099772280853909\\
599.15	0.00997769377337961\\
599.16	0.00997815598703157\\
599.17	0.00997861469246631\\
599.18	0.00997906985546915\\
599.19	0.00997952144148792\\
599.2	0.00997996941562957\\
599.21	0.00998041374265684\\
599.22	0.0099808543869848\\
599.23	0.00998129131267744\\
599.24	0.00998172448344418\\
599.25	0.00998215386263636\\
599.26	0.00998257941258354\\
599.27	0.00998300109279825\\
599.28	0.00998341886238981\\
599.29	0.00998383268006024\\
599.3	0.00998424250410032\\
599.31	0.00998464829238543\\
599.32	0.00998505000237151\\
599.33	0.00998544759109087\\
599.34	0.00998584101514799\\
599.35	0.00998623023071532\\
599.36	0.00998661519352896\\
599.37	0.00998699585888436\\
599.38	0.00998737218163197\\
599.39	0.00998774411617281\\
599.4	0.00998811161645403\\
599.41	0.00998847463596443\\
599.42	0.0099888331277299\\
599.43	0.00998918704430885\\
599.44	0.00998953633778757\\
599.45	0.00998988095977558\\
599.46	0.00999022086140088\\
599.47	0.00999055599330522\\
599.48	0.00999088630563925\\
599.49	0.00999121174805768\\
599.5	0.00999153226971435\\
599.51	0.0099918478192573\\
599.52	0.00999215834482375\\
599.53	0.00999246379403503\\
599.54	0.0099927641139915\\
599.55	0.00999305925126738\\
599.56	0.00999334915190553\\
599.57	0.0099936337614122\\
599.58	0.00999391302475173\\
599.59	0.00999418688634118\\
599.6	0.00999445529004489\\
599.61	0.00999471817916904\\
599.62	0.00999497549645613\\
599.63	0.00999522718407935\\
599.64	0.009995473183637\\
599.65	0.00999571343614678\\
599.66	0.00999594788204004\\
599.67	0.00999617646115598\\
599.68	0.0099963991127358\\
599.69	0.00999661577541675\\
599.7	0.00999682638722618\\
599.71	0.00999703088557551\\
599.72	0.00999722920725411\\
599.73	0.00999742128842315\\
599.74	0.00999760706460942\\
599.75	0.00999778647069897\\
599.76	0.00999795944093086\\
599.77	0.0099981259088907\\
599.78	0.0099982858075042\\
599.79	0.00999843906903064\\
599.8	0.00999858562505627\\
599.81	0.00999872540648767\\
599.82	0.00999885834354498\\
599.83	0.00999898436575515\\
599.84	0.00999910340194508\\
599.85	0.00999921538023465\\
599.86	0.00999932022802977\\
599.87	0.00999941787201528\\
599.88	0.00999950823814785\\
599.89	0.00999959125164875\\
599.9	0.00999966683699656\\
599.91	0.00999973491791987\\
599.92	0.0099997954173898\\
599.93	0.00999984825761255\\
599.94	0.00999989336002181\\
599.95	0.00999993064527112\\
599.96	0.00999996003322615\\
599.97	0.00999998144295691\\
599.98	0.00999999479272987\\
599.99	0.01\\
600	0.01\\
};
\addplot [color=mycolor8,solid,forget plot]
  table[row sep=crcr]{%
0.01	0.00440626308531185\\
1.01	0.00440626422127448\\
2.01	0.00440626538095373\\
3.01	0.0044062665648462\\
4.01	0.00440626777345891\\
5.01	0.00440626900730949\\
6.01	0.00440627026692658\\
7.01	0.00440627155284969\\
8.01	0.00440627286562986\\
9.01	0.00440627420582962\\
10.01	0.0044062755740233\\
11.01	0.00440627697079723\\
12.01	0.00440627839675041\\
13.01	0.00440627985249417\\
14.01	0.00440628133865258\\
15.01	0.004406282855863\\
16.01	0.00440628440477632\\
17.01	0.00440628598605691\\
18.01	0.0044062876003831\\
19.01	0.00440628924844742\\
20.01	0.0044062909309573\\
21.01	0.00440629264863485\\
22.01	0.00440629440221726\\
23.01	0.00440629619245744\\
24.01	0.00440629802012393\\
25.01	0.00440629988600161\\
26.01	0.00440630179089169\\
27.01	0.0044063037356125\\
28.01	0.00440630572099936\\
29.01	0.0044063077479054\\
30.01	0.00440630981720147\\
31.01	0.00440631192977683\\
32.01	0.0044063140865396\\
33.01	0.00440631628841669\\
34.01	0.00440631853635493\\
35.01	0.00440632083132096\\
36.01	0.00440632317430162\\
37.01	0.00440632556630473\\
38.01	0.00440632800835908\\
39.01	0.00440633050151536\\
40.01	0.00440633304684623\\
41.01	0.00440633564544717\\
42.01	0.00440633829843672\\
43.01	0.00440634100695664\\
44.01	0.00440634377217311\\
45.01	0.0044063465952769\\
46.01	0.0044063494774837\\
47.01	0.00440635242003476\\
48.01	0.00440635542419768\\
49.01	0.00440635849126676\\
50.01	0.00440636162256351\\
51.01	0.0044063648194374\\
52.01	0.00440636808326623\\
53.01	0.00440637141545691\\
54.01	0.00440637481744614\\
55.01	0.00440637829070072\\
56.01	0.00440638183671864\\
57.01	0.00440638545702918\\
58.01	0.00440638915319415\\
59.01	0.00440639292680821\\
60.01	0.0044063967794996\\
61.01	0.00440640071293111\\
62.01	0.0044064047288004\\
63.01	0.00440640882884109\\
64.01	0.00440641301482347\\
65.01	0.00440641728855513\\
66.01	0.00440642165188182\\
67.01	0.00440642610668807\\
68.01	0.0044064306548985\\
69.01	0.0044064352984782\\
70.01	0.00440644003943368\\
71.01	0.00440644487981423\\
72.01	0.00440644982171214\\
73.01	0.00440645486726378\\
74.01	0.00440646001865092\\
75.01	0.00440646527810121\\
76.01	0.00440647064788945\\
77.01	0.00440647613033868\\
78.01	0.00440648172782072\\
79.01	0.00440648744275766\\
80.01	0.00440649327762303\\
81.01	0.00440649923494237\\
82.01	0.00440650531729445\\
83.01	0.004406511527313\\
84.01	0.00440651786768717\\
85.01	0.00440652434116291\\
86.01	0.0044065309505447\\
87.01	0.00440653769869574\\
88.01	0.00440654458854012\\
89.01	0.00440655162306375\\
90.01	0.00440655880531568\\
91.01	0.00440656613840951\\
92.01	0.00440657362552467\\
93.01	0.00440658126990784\\
94.01	0.00440658907487454\\
95.01	0.00440659704381034\\
96.01	0.00440660518017254\\
97.01	0.00440661348749162\\
98.01	0.00440662196937291\\
99.01	0.00440663062949775\\
100.01	0.00440663947162587\\
101.01	0.0044066484995962\\
102.01	0.00440665771732905\\
103.01	0.00440666712882812\\
104.01	0.00440667673818164\\
105.01	0.00440668654956447\\
106.01	0.00440669656723994\\
107.01	0.00440670679556165\\
108.01	0.00440671723897552\\
109.01	0.00440672790202165\\
110.01	0.00440673878933663\\
111.01	0.00440674990565476\\
112.01	0.00440676125581123\\
113.01	0.00440677284474333\\
114.01	0.00440678467749334\\
115.01	0.00440679675921023\\
116.01	0.0044068090951522\\
117.01	0.00440682169068929\\
118.01	0.00440683455130486\\
119.01	0.00440684768259891\\
120.01	0.00440686109029029\\
121.01	0.00440687478021913\\
122.01	0.00440688875834937\\
123.01	0.00440690303077154\\
124.01	0.00440691760370533\\
125.01	0.00440693248350259\\
126.01	0.00440694767664984\\
127.01	0.00440696318977146\\
128.01	0.00440697902963225\\
129.01	0.00440699520314074\\
130.01	0.00440701171735211\\
131.01	0.00440702857947142\\
132.01	0.00440704579685672\\
133.01	0.00440706337702228\\
134.01	0.00440708132764198\\
135.01	0.00440709965655289\\
136.01	0.00440711837175821\\
137.01	0.0044071374814314\\
138.01	0.00440715699391962\\
139.01	0.00440717691774739\\
140.01	0.00440719726162048\\
141.01	0.00440721803442939\\
142.01	0.0044072392452541\\
143.01	0.00440726090336699\\
144.01	0.00440728301823819\\
145.01	0.00440730559953875\\
146.01	0.00440732865714562\\
147.01	0.00440735220114562\\
148.01	0.00440737624184034\\
149.01	0.0044074007897504\\
150.01	0.00440742585562011\\
151.01	0.00440745145042237\\
152.01	0.00440747758536389\\
153.01	0.00440750427188961\\
154.01	0.00440753152168795\\
155.01	0.00440755934669618\\
156.01	0.00440758775910581\\
157.01	0.00440761677136776\\
158.01	0.00440764639619827\\
159.01	0.00440767664658397\\
160.01	0.00440770753578837\\
161.01	0.00440773907735721\\
162.01	0.00440777128512501\\
163.01	0.00440780417322095\\
164.01	0.00440783775607496\\
165.01	0.00440787204842495\\
166.01	0.00440790706532249\\
167.01	0.00440794282213992\\
168.01	0.00440797933457754\\
169.01	0.00440801661867002\\
170.01	0.00440805469079412\\
171.01	0.00440809356767564\\
172.01	0.0044081332663972\\
173.01	0.00440817380440565\\
174.01	0.00440821519951994\\
175.01	0.00440825746993932\\
176.01	0.00440830063425132\\
177.01	0.00440834471143978\\
178.01	0.00440838972089398\\
179.01	0.0044084356824169\\
180.01	0.0044084826162344\\
181.01	0.00440853054300401\\
182.01	0.00440857948382432\\
183.01	0.00440862946024473\\
184.01	0.00440868049427444\\
185.01	0.00440873260839318\\
186.01	0.00440878582556084\\
187.01	0.00440884016922796\\
188.01	0.00440889566334617\\
189.01	0.00440895233237887\\
190.01	0.00440901020131275\\
191.01	0.00440906929566838\\
192.01	0.00440912964151229\\
193.01	0.00440919126546833\\
194.01	0.00440925419473025\\
195.01	0.00440931845707281\\
196.01	0.00440938408086572\\
197.01	0.00440945109508563\\
198.01	0.00440951952932955\\
199.01	0.00440958941382805\\
200.01	0.00440966077945887\\
201.01	0.00440973365776112\\
202.01	0.00440980808094972\\
203.01	0.00440988408192954\\
204.01	0.00440996169431064\\
205.01	0.00441004095242328\\
206.01	0.00441012189133402\\
207.01	0.00441020454686088\\
208.01	0.00441028895559045\\
209.01	0.0044103751548937\\
210.01	0.00441046318294396\\
211.01	0.0044105530787333\\
212.01	0.00441064488209088\\
213.01	0.0044107386337011\\
214.01	0.00441083437512203\\
215.01	0.00441093214880439\\
216.01	0.00441103199811108\\
217.01	0.00441113396733675\\
218.01	0.00441123810172846\\
219.01	0.00441134444750611\\
220.01	0.00441145305188383\\
221.01	0.00441156396309121\\
222.01	0.00441167723039612\\
223.01	0.0044117929041269\\
224.01	0.00441191103569558\\
225.01	0.00441203167762177\\
226.01	0.00441215488355656\\
227.01	0.00441228070830745\\
228.01	0.00441240920786376\\
229.01	0.00441254043942216\\
230.01	0.00441267446141355\\
231.01	0.00441281133352994\\
232.01	0.00441295111675212\\
233.01	0.00441309387337799\\
234.01	0.00441323966705155\\
235.01	0.00441338856279257\\
236.01	0.00441354062702679\\
237.01	0.00441369592761702\\
238.01	0.00441385453389485\\
239.01	0.00441401651669299\\
240.01	0.00441418194837837\\
241.01	0.00441435090288646\\
242.01	0.00441452345575567\\
243.01	0.00441469968416305\\
244.01	0.00441487966696047\\
245.01	0.00441506348471231\\
246.01	0.00441525121973288\\
247.01	0.00441544295612602\\
248.01	0.0044156387798247\\
249.01	0.00441583877863176\\
250.01	0.00441604304226205\\
251.01	0.00441625166238495\\
252.01	0.00441646473266812\\
253.01	0.00441668234882274\\
254.01	0.00441690460864883\\
255.01	0.00441713161208299\\
256.01	0.00441736346124592\\
257.01	0.00441760026049218\\
258.01	0.00441784211646047\\
259.01	0.00441808913812528\\
260.01	0.00441834143685001\\
261.01	0.00441859912644132\\
262.01	0.00441886232320449\\
263.01	0.00441913114600051\\
264.01	0.00441940571630475\\
265.01	0.00441968615826624\\
266.01	0.00441997259876925\\
267.01	0.00442026516749622\\
268.01	0.00442056399699201\\
269.01	0.00442086922273013\\
270.01	0.0044211809831799\\
271.01	0.00442149941987666\\
272.01	0.00442182467749223\\
273.01	0.00442215690390849\\
274.01	0.00442249625029138\\
275.01	0.00442284287116884\\
276.01	0.00442319692450853\\
277.01	0.00442355857179931\\
278.01	0.00442392797813407\\
279.01	0.00442430531229437\\
280.01	0.00442469074683832\\
281.01	0.00442508445819004\\
282.01	0.0044254866267318\\
283.01	0.00442589743689811\\
284.01	0.0044263170772733\\
285.01	0.00442674574069093\\
286.01	0.00442718362433586\\
287.01	0.00442763092984971\\
288.01	0.004428087863439\\
289.01	0.00442855463598554\\
290.01	0.00442903146316111\\
291.01	0.00442951856554408\\
292.01	0.00443001616874006\\
293.01	0.00443052450350619\\
294.01	0.00443104380587758\\
295.01	0.00443157431729853\\
296.01	0.00443211628475747\\
297.01	0.00443266996092481\\
298.01	0.00443323560429575\\
299.01	0.00443381347933643\\
300.01	0.00443440385663482\\
301.01	0.00443500701305513\\
302.01	0.00443562323189807\\
303.01	0.00443625280306416\\
304.01	0.00443689602322331\\
305.01	0.00443755319598767\\
306.01	0.00443822463209094\\
307.01	0.00443891064957251\\
308.01	0.0044396115739662\\
309.01	0.00444032773849545\\
310.01	0.00444105948427355\\
311.01	0.00444180716050999\\
312.01	0.00444257112472224\\
313.01	0.00444335174295431\\
314.01	0.00444414939000094\\
315.01	0.00444496444963761\\
316.01	0.00444579731485817\\
317.01	0.00444664838811785\\
318.01	0.0044475180815835\\
319.01	0.00444840681738912\\
320.01	0.00444931502789954\\
321.01	0.00445024315597932\\
322.01	0.0044511916552685\\
323.01	0.00445216099046462\\
324.01	0.00445315163761006\\
325.01	0.0044541640843857\\
326.01	0.00445519883040953\\
327.01	0.00445625638753941\\
328.01	0.0044573372801805\\
329.01	0.00445844204559541\\
330.01	0.00445957123421578\\
331.01	0.00446072540995518\\
332.01	0.00446190515052064\\
333.01	0.00446311104772117\\
334.01	0.00446434370777075\\
335.01	0.00446560375158356\\
336.01	0.0044668918150572\\
337.01	0.00446820854933985\\
338.01	0.00446955462107711\\
339.01	0.00447093071263302\\
340.01	0.00447233752227598\\
341.01	0.00447377576432524\\
342.01	0.00447524616924487\\
343.01	0.00447674948367514\\
344.01	0.00447828647038827\\
345.01	0.00447985790815145\\
346.01	0.00448146459147909\\
347.01	0.0044831073302514\\
348.01	0.0044847869491738\\
349.01	0.00448650428704578\\
350.01	0.00448826019580327\\
351.01	0.00449005553929223\\
352.01	0.00449189119172355\\
353.01	0.00449376803575088\\
354.01	0.00449568696010501\\
355.01	0.00449764885670435\\
356.01	0.00449965461715125\\
357.01	0.00450170512850824\\
358.01	0.0045038012682342\\
359.01	0.00450594389814001\\
360.01	0.00450813385720904\\
361.01	0.00451037195310497\\
362.01	0.00451265895217097\\
363.01	0.00451499556770892\\
364.01	0.00451738244630782\\
365.01	0.00451982015198824\\
366.01	0.00452230914792892\\
367.01	0.00452484977556643\\
368.01	0.00452744223090766\\
369.01	0.00453008653798694\\
370.01	0.00453278251955661\\
371.01	0.00453552976534603\\
372.01	0.00453832759860198\\
373.01	0.00454117504219044\\
374.01	0.00454407078637301\\
375.01	0.00454701316158006\\
376.01	0.00455000012124152\\
377.01	0.00455302924221239\\
378.01	0.00455609775384222\\
379.01	0.00455920261167952\\
380.01	0.00456234063874891\\
381.01	0.00456550876703976\\
382.01	0.00456870442538744\\
383.01	0.00457192613875745\\
384.01	0.00457517443009159\\
385.01	0.00457845526305113\\
386.01	0.0045817974796559\\
387.01	0.00458520946202622\\
388.01	0.00458869264373901\\
389.01	0.00459224848572395\\
390.01	0.00459587847657204\\
391.01	0.00459958413282364\\
392.01	0.00460336699923236\\
393.01	0.00460722864900213\\
394.01	0.00461117068399346\\
395.01	0.00461519473489354\\
396.01	0.00461930246134645\\
397.01	0.00462349555203738\\
398.01	0.00462777572472496\\
399.01	0.0046321447262158\\
400.01	0.00463660433227225\\
401.01	0.00464115634744748\\
402.01	0.00464580260483666\\
403.01	0.00465054496573504\\
404.01	0.00465538531919338\\
405.01	0.00466032558145577\\
406.01	0.00466536769526826\\
407.01	0.00467051362904262\\
408.01	0.0046757653758587\\
409.01	0.00468112495228747\\
410.01	0.00468659439701416\\
411.01	0.00469217576923976\\
412.01	0.00469787114683555\\
413.01	0.00470368262422466\\
414.01	0.00470961230996053\\
415.01	0.00471566232396985\\
416.01	0.00472183479442334\\
417.01	0.00472813185419708\\
418.01	0.00473455563688004\\
419.01	0.00474110827228083\\
420.01	0.00474779188138451\\
421.01	0.00475460857070206\\
422.01	0.00476156042595463\\
423.01	0.0047686495050266\\
424.01	0.00477587783011849\\
425.01	0.00478324737902673\\
426.01	0.00479076007546984\\
427.01	0.00479841777838035\\
428.01	0.00480622227007421\\
429.01	0.00481417524321116\\
430.01	0.00482227828645251\\
431.01	0.00483053286872971\\
432.01	0.00483894032203472\\
433.01	0.00484750182265346\\
434.01	0.00485621837077113\\
435.01	0.00486509076839704\\
436.01	0.00487411959558005\\
437.01	0.00488330518491606\\
438.01	0.00489264759439711\\
439.01	0.00490214657870898\\
440.01	0.00491180155916121\\
441.01	0.00492161159253393\\
442.01	0.00493157533925231\\
443.01	0.00494169103145929\\
444.01	0.00495195644175643\\
445.01	0.00496236885363065\\
446.01	0.00497292503488503\\
447.01	0.00498362121575813\\
448.01	0.00499445307385384\\
449.01	0.00500541572852012\\
450.01	0.00501650374791437\\
451.01	0.00502771117267692\\
452.01	0.00503903156088574\\
453.01	0.00505045805977243\\
454.01	0.00506198351047833\\
455.01	0.00507360059285489\\
456.01	0.00508530201781257\\
457.01	0.00509708077479794\\
458.01	0.00510893044128637\\
459.01	0.00512084555923308\\
460.01	0.00513282207947717\\
461.01	0.00514485786805845\\
462.01	0.00515695325669842\\
463.01	0.00516911160101229\\
464.01	0.00518133978105872\\
465.01	0.00519364853487436\\
466.01	0.00520605244994091\\
467.01	0.00521856934055163\\
468.01	0.00523121859675642\\
469.01	0.00524401443556734\\
470.01	0.00525695595333581\\
471.01	0.00527003895708189\\
472.01	0.0052832589415233\\
473.01	0.00529661111701986\\
474.01	0.00531009044759288\\
475.01	0.00532369170046118\\
476.01	0.00533740950855351\\
477.01	0.0053512384473979\\
478.01	0.00536517312763309\\
479.01	0.00537920830408881\\
480.01	0.00539333900192615\\
481.01	0.00540756065978887\\
482.01	0.00542186928905644\\
483.01	0.00543626164696971\\
484.01	0.00545073541969959\\
485.01	0.00546528940926458\\
486.01	0.00547992371550203\\
487.01	0.00549463990104536\\
488.01	0.00550944112351877\\
489.01	0.00552433221510523\\
490.01	0.00553931968568339\\
491.01	0.00555441162261239\\
492.01	0.00556961745924792\\
493.01	0.00558494758736217\\
494.01	0.0056004127988741\\
495.01	0.00561602356490333\\
496.01	0.00563178920303264\\
497.01	0.00564771705855759\\
498.01	0.00566381198048466\\
499.01	0.00568007724959183\\
500.01	0.00569651658050933\\
501.01	0.00571313456604973\\
502.01	0.00572993672726638\\
503.01	0.00574692954501591\\
504.01	0.0057641204693081\\
505.01	0.00578151790174496\\
506.01	0.00579913114680775\\
507.01	0.00581697032881338\\
508.01	0.00583504627322513\\
509.01	0.00585337035387697\\
510.01	0.00587195431173997\\
511.01	0.0058908100562076\\
512.01	0.00590994946638894\\
513.01	0.00592938421704352\\
514.01	0.00594912566032621\\
515.01	0.005969184797927\\
516.01	0.00598957237400632\\
517.01	0.00601029909958091\\
518.01	0.00603137594750737\\
519.01	0.00605281429317925\\
520.01	0.00607462587650683\\
521.01	0.00609682273422721\\
522.01	0.00611941712827492\\
523.01	0.00614242147440214\\
524.01	0.00616584827601105\\
525.01	0.00618971006870538\\
526.01	0.00621401938102251\\
527.01	0.00623878871588064\\
528.01	0.00626403055516378\\
529.01	0.00628975738633661\\
530.01	0.00631598174502204\\
531.01	0.0063427162616497\\
532.01	0.00636997369532064\\
533.01	0.00639776693951445\\
534.01	0.00642610900545185\\
535.01	0.00645501300392987\\
536.01	0.00648449213262774\\
537.01	0.00651455966959223\\
538.01	0.00654522897277823\\
539.01	0.00657651348475956\\
540.01	0.00660842674092671\\
541.01	0.00664098237881587\\
542.01	0.00667419414588415\\
543.01	0.00670807590334808\\
544.01	0.00674264162486408\\
545.01	0.00677790539086053\\
546.01	0.00681388138114321\\
547.01	0.00685058386763962\\
548.01	0.00688802720720362\\
549.01	0.00692622583376501\\
550.01	0.00696519424901141\\
551.01	0.00700494701080015\\
552.01	0.00704549871862432\\
553.01	0.00708686399568258\\
554.01	0.00712905746738202\\
555.01	0.00717209373633373\\
556.01	0.00721598735393222\\
557.01	0.00726075278837836\\
558.01	0.00730640438870886\\
559.01	0.0073529563442945\\
560.01	0.0074004226392819\\
561.01	0.00744881700148302\\
562.01	0.0074981528452527\\
563.01	0.00754844320790874\\
564.01	0.00759970067923708\\
565.01	0.00765193732357281\\
566.01	0.00770516459387936\\
567.01	0.00775939323719052\\
568.01	0.0078146331907652\\
569.01	0.00787089346830959\\
570.01	0.00792818203563622\\
571.01	0.00798650567515498\\
572.01	0.00804586983862646\\
573.01	0.00810627848766415\\
574.01	0.00816773392156107\\
575.01	0.00823023659215375\\
576.01	0.00829378490563244\\
577.01	0.00835837501148391\\
578.01	0.00842400057911648\\
579.01	0.00849065256320018\\
580.01	0.00855831895938081\\
581.01	0.00862698455284182\\
582.01	0.00869663066323415\\
583.01	0.00876723489083508\\
584.01	0.00883877087050155\\
585.01	0.00891120804214121\\
586.01	0.00898451144914387\\
587.01	0.00905864157963132\\
588.01	0.00913355426966545\\
589.01	0.00920920069290074\\
590.01	0.0092855274678398\\
591.01	0.00936247692215228\\
592.01	0.00943998756383507\\
593.01	0.00951799482179795\\
594.01	0.00959643213433056\\
595.01	0.00967523248356592\\
596.01	0.00975433049838055\\
597.01	0.00983365065339071\\
598.01	0.00990866201848071\\
599.01	0.00997087280416276\\
599.02	0.00997138072163725\\
599.03	0.009971885576258\\
599.04	0.00997238733820211\\
599.05	0.00997288597735272\\
599.06	0.00997338146329604\\
599.07	0.00997387376531845\\
599.08	0.00997436285240349\\
599.09	0.00997484869322892\\
599.1	0.00997533125616361\\
599.11	0.00997581050926455\\
599.12	0.00997628642027372\\
599.13	0.00997675895661498\\
599.14	0.0099772280853909\\
599.15	0.00997769377337961\\
599.16	0.00997815598703157\\
599.17	0.00997861469246631\\
599.18	0.00997906985546915\\
599.19	0.00997952144148792\\
599.2	0.00997996941562957\\
599.21	0.00998041374265684\\
599.22	0.0099808543869848\\
599.23	0.00998129131267744\\
599.24	0.00998172448344418\\
599.25	0.00998215386263636\\
599.26	0.00998257941258353\\
599.27	0.00998300109279825\\
599.28	0.00998341886238981\\
599.29	0.00998383268006024\\
599.3	0.00998424250410032\\
599.31	0.00998464829238543\\
599.32	0.00998505000237151\\
599.33	0.00998544759109087\\
599.34	0.00998584101514799\\
599.35	0.00998623023071532\\
599.36	0.00998661519352895\\
599.37	0.00998699585888436\\
599.38	0.00998737218163197\\
599.39	0.00998774411617281\\
599.4	0.00998811161645403\\
599.41	0.00998847463596443\\
599.42	0.0099888331277299\\
599.43	0.00998918704430885\\
599.44	0.00998953633778757\\
599.45	0.00998988095977558\\
599.46	0.00999022086140088\\
599.47	0.00999055599330522\\
599.48	0.00999088630563925\\
599.49	0.00999121174805768\\
599.5	0.00999153226971435\\
599.51	0.0099918478192573\\
599.52	0.00999215834482374\\
599.53	0.00999246379403503\\
599.54	0.0099927641139915\\
599.55	0.00999305925126738\\
599.56	0.00999334915190553\\
599.57	0.0099936337614122\\
599.58	0.00999391302475173\\
599.59	0.00999418688634118\\
599.6	0.00999445529004489\\
599.61	0.00999471817916904\\
599.62	0.00999497549645613\\
599.63	0.00999522718407934\\
599.64	0.009995473183637\\
599.65	0.00999571343614678\\
599.66	0.00999594788204004\\
599.67	0.00999617646115599\\
599.68	0.0099963991127358\\
599.69	0.00999661577541675\\
599.7	0.00999682638722618\\
599.71	0.00999703088557551\\
599.72	0.00999722920725411\\
599.73	0.00999742128842315\\
599.74	0.00999760706460942\\
599.75	0.00999778647069897\\
599.76	0.00999795944093086\\
599.77	0.0099981259088907\\
599.78	0.0099982858075042\\
599.79	0.00999843906903064\\
599.8	0.00999858562505627\\
599.81	0.00999872540648767\\
599.82	0.00999885834354498\\
599.83	0.00999898436575515\\
599.84	0.00999910340194508\\
599.85	0.00999921538023465\\
599.86	0.00999932022802977\\
599.87	0.00999941787201528\\
599.88	0.00999950823814785\\
599.89	0.00999959125164875\\
599.9	0.00999966683699656\\
599.91	0.00999973491791987\\
599.92	0.0099997954173898\\
599.93	0.00999984825761255\\
599.94	0.00999989336002181\\
599.95	0.00999993064527112\\
599.96	0.00999996003322615\\
599.97	0.00999998144295691\\
599.98	0.00999999479272987\\
599.99	0.01\\
600	0.01\\
};
\addplot [color=blue!25!mycolor7,solid,forget plot]
  table[row sep=crcr]{%
0.01	0.00413645223902454\\
1.01	0.00413645300246634\\
2.01	0.00413645378183859\\
3.01	0.00413645457747443\\
4.01	0.0041364553897143\\
5.01	0.00413645621890542\\
6.01	0.00413645706540267\\
7.01	0.00413645792956811\\
8.01	0.00413645881177166\\
9.01	0.00413645971239082\\
10.01	0.00413646063181095\\
11.01	0.00413646157042567\\
12.01	0.00413646252863661\\
13.01	0.00413646350685418\\
14.01	0.00413646450549729\\
15.01	0.00413646552499354\\
16.01	0.00413646656577953\\
17.01	0.00413646762830092\\
18.01	0.00413646871301321\\
19.01	0.00413646982038111\\
20.01	0.00413647095087882\\
21.01	0.00413647210499092\\
22.01	0.00413647328321212\\
23.01	0.00413647448604743\\
24.01	0.0041364757140125\\
25.01	0.0041364769676339\\
26.01	0.00413647824744907\\
27.01	0.00413647955400687\\
28.01	0.00413648088786767\\
29.01	0.0041364822496036\\
30.01	0.00413648363979888\\
31.01	0.00413648505905004\\
32.01	0.00413648650796607\\
33.01	0.00413648798716877\\
34.01	0.00413648949729303\\
35.01	0.00413649103898717\\
36.01	0.00413649261291286\\
37.01	0.00413649421974589\\
38.01	0.00413649586017656\\
39.01	0.00413649753490919\\
40.01	0.0041364992446631\\
41.01	0.00413650099017301\\
42.01	0.00413650277218841\\
43.01	0.00413650459147534\\
44.01	0.00413650644881562\\
45.01	0.00413650834500721\\
46.01	0.00413651028086549\\
47.01	0.00413651225722261\\
48.01	0.00413651427492816\\
49.01	0.00413651633484986\\
50.01	0.0041365184378737\\
51.01	0.00413652058490413\\
52.01	0.00413652277686458\\
53.01	0.0041365250146983\\
54.01	0.00413652729936799\\
55.01	0.00413652963185687\\
56.01	0.00413653201316876\\
57.01	0.00413653444432874\\
58.01	0.00413653692638334\\
59.01	0.00413653946040121\\
60.01	0.00413654204747346\\
61.01	0.00413654468871429\\
62.01	0.00413654738526122\\
63.01	0.00413655013827585\\
64.01	0.00413655294894416\\
65.01	0.00413655581847719\\
66.01	0.00413655874811143\\
67.01	0.00413656173910959\\
68.01	0.00413656479276098\\
69.01	0.00413656791038176\\
70.01	0.00413657109331628\\
71.01	0.00413657434293679\\
72.01	0.00413657766064482\\
73.01	0.00413658104787133\\
74.01	0.00413658450607736\\
75.01	0.0041365880367549\\
76.01	0.00413659164142743\\
77.01	0.0041365953216502\\
78.01	0.0041365990790117\\
79.01	0.00413660291513383\\
80.01	0.00413660683167236\\
81.01	0.00413661083031852\\
82.01	0.00413661491279878\\
83.01	0.00413661908087617\\
84.01	0.00413662333635101\\
85.01	0.00413662768106166\\
86.01	0.00413663211688484\\
87.01	0.00413663664573743\\
88.01	0.00413664126957637\\
89.01	0.00413664599039996\\
90.01	0.00413665081024897\\
91.01	0.00413665573120693\\
92.01	0.00413666075540151\\
93.01	0.00413666588500525\\
94.01	0.00413667112223645\\
95.01	0.00413667646936043\\
96.01	0.00413668192869014\\
97.01	0.00413668750258765\\
98.01	0.00413669319346451\\
99.01	0.00413669900378365\\
100.01	0.00413670493605961\\
101.01	0.00413671099286049\\
102.01	0.00413671717680839\\
103.01	0.00413672349058052\\
104.01	0.00413672993691108\\
105.01	0.00413673651859182\\
106.01	0.0041367432384736\\
107.01	0.00413675009946735\\
108.01	0.00413675710454571\\
109.01	0.00413676425674391\\
110.01	0.00413677155916145\\
111.01	0.00413677901496344\\
112.01	0.0041367866273818\\
113.01	0.00413679439971668\\
114.01	0.00413680233533802\\
115.01	0.00413681043768732\\
116.01	0.00413681871027842\\
117.01	0.0041368271566995\\
118.01	0.00413683578061494\\
119.01	0.00413684458576649\\
120.01	0.00413685357597467\\
121.01	0.00413686275514111\\
122.01	0.00413687212725002\\
123.01	0.00413688169636965\\
124.01	0.00413689146665469\\
125.01	0.00413690144234731\\
126.01	0.00413691162777957\\
127.01	0.00413692202737526\\
128.01	0.00413693264565181\\
129.01	0.00413694348722218\\
130.01	0.00413695455679676\\
131.01	0.00413696585918608\\
132.01	0.00413697739930197\\
133.01	0.0041369891821606\\
134.01	0.00413700121288418\\
135.01	0.00413701349670324\\
136.01	0.00413702603895943\\
137.01	0.0041370388451072\\
138.01	0.00413705192071638\\
139.01	0.00413706527147512\\
140.01	0.00413707890319162\\
141.01	0.00413709282179756\\
142.01	0.00413710703334966\\
143.01	0.00413712154403365\\
144.01	0.00413713636016562\\
145.01	0.00413715148819595\\
146.01	0.00413716693471143\\
147.01	0.00413718270643845\\
148.01	0.00413719881024602\\
149.01	0.0041372152531487\\
150.01	0.00413723204230966\\
151.01	0.00413724918504406\\
152.01	0.00413726668882178\\
153.01	0.0041372845612714\\
154.01	0.00413730281018307\\
155.01	0.00413732144351218\\
156.01	0.00413734046938279\\
157.01	0.00413735989609126\\
158.01	0.00413737973210988\\
159.01	0.00413739998609051\\
160.01	0.00413742066686876\\
161.01	0.00413744178346762\\
162.01	0.0041374633451015\\
163.01	0.00413748536118023\\
164.01	0.0041375078413135\\
165.01	0.00413753079531483\\
166.01	0.00413755423320596\\
167.01	0.00413757816522158\\
168.01	0.00413760260181349\\
169.01	0.0041376275536554\\
170.01	0.0041376530316477\\
171.01	0.00413767904692233\\
172.01	0.00413770561084746\\
173.01	0.00413773273503304\\
174.01	0.00413776043133533\\
175.01	0.00413778871186272\\
176.01	0.00413781758898084\\
177.01	0.00413784707531809\\
178.01	0.0041378771837714\\
179.01	0.00413790792751159\\
180.01	0.00413793931998961\\
181.01	0.00413797137494241\\
182.01	0.00413800410639917\\
183.01	0.00413803752868733\\
184.01	0.0041380716564393\\
185.01	0.00413810650459848\\
186.01	0.00413814208842662\\
187.01	0.0041381784235099\\
188.01	0.00413821552576664\\
189.01	0.00413825341145374\\
190.01	0.00413829209717424\\
191.01	0.00413833159988498\\
192.01	0.00413837193690397\\
193.01	0.00413841312591806\\
194.01	0.00413845518499083\\
195.01	0.00413849813257113\\
196.01	0.00413854198750095\\
197.01	0.00413858676902401\\
198.01	0.0041386324967943\\
199.01	0.00413867919088515\\
200.01	0.0041387268717981\\
201.01	0.00413877556047228\\
202.01	0.00413882527829372\\
203.01	0.00413887604710475\\
204.01	0.0041389278892144\\
205.01	0.00413898082740843\\
206.01	0.00413903488495899\\
207.01	0.00413909008563577\\
208.01	0.0041391464537166\\
209.01	0.00413920401399849\\
210.01	0.00413926279180882\\
211.01	0.00413932281301681\\
212.01	0.00413938410404531\\
213.01	0.00413944669188301\\
214.01	0.00413951060409641\\
215.01	0.00413957586884263\\
216.01	0.00413964251488226\\
217.01	0.00413971057159257\\
218.01	0.00413978006898054\\
219.01	0.00413985103769716\\
220.01	0.00413992350905104\\
221.01	0.00413999751502343\\
222.01	0.00414007308828199\\
223.01	0.00414015026219657\\
224.01	0.00414022907085443\\
225.01	0.00414030954907614\\
226.01	0.0041403917324311\\
227.01	0.00414047565725467\\
228.01	0.00414056136066472\\
229.01	0.00414064888057912\\
230.01	0.0041407382557329\\
231.01	0.00414082952569686\\
232.01	0.00414092273089588\\
233.01	0.00414101791262771\\
234.01	0.00414111511308253\\
235.01	0.00414121437536263\\
236.01	0.00414131574350272\\
237.01	0.00414141926249077\\
238.01	0.00414152497828936\\
239.01	0.00414163293785735\\
240.01	0.00414174318917206\\
241.01	0.00414185578125228\\
242.01	0.00414197076418168\\
243.01	0.00414208818913275\\
244.01	0.00414220810839117\\
245.01	0.00414233057538117\\
246.01	0.00414245564469116\\
247.01	0.00414258337210011\\
248.01	0.00414271381460452\\
249.01	0.0041428470304464\\
250.01	0.00414298307914145\\
251.01	0.00414312202150835\\
252.01	0.00414326391969813\\
253.01	0.00414340883722551\\
254.01	0.00414355683899971\\
255.01	0.00414370799135663\\
256.01	0.0041438623620923\\
257.01	0.00414402002049635\\
258.01	0.00414418103738673\\
259.01	0.00414434548514565\\
260.01	0.00414451343775553\\
261.01	0.00414468497083743\\
262.01	0.00414486016168879\\
263.01	0.00414503908932352\\
264.01	0.0041452218345121\\
265.01	0.00414540847982364\\
266.01	0.00414559910966864\\
267.01	0.00414579381034308\\
268.01	0.00414599267007339\\
269.01	0.00414619577906319\\
270.01	0.00414640322954103\\
271.01	0.00414661511580915\\
272.01	0.00414683153429449\\
273.01	0.00414705258360024\\
274.01	0.00414727836455965\\
275.01	0.00414750898029047\\
276.01	0.00414774453625195\\
277.01	0.00414798514030318\\
278.01	0.00414823090276233\\
279.01	0.00414848193646948\\
280.01	0.0041487383568493\\
281.01	0.00414900028197704\\
282.01	0.00414926783264594\\
283.01	0.00414954113243737\\
284.01	0.00414982030779178\\
285.01	0.00415010548808338\\
286.01	0.00415039680569666\\
287.01	0.00415069439610491\\
288.01	0.00415099839795185\\
289.01	0.0041513089531358\\
290.01	0.00415162620689657\\
291.01	0.00415195030790523\\
292.01	0.00415228140835719\\
293.01	0.00415261966406762\\
294.01	0.00415296523457169\\
295.01	0.00415331828322656\\
296.01	0.00415367897731833\\
297.01	0.00415404748817211\\
298.01	0.0041544239912662\\
299.01	0.00415480866635073\\
300.01	0.00415520169757005\\
301.01	0.00415560327359081\\
302.01	0.00415601358773349\\
303.01	0.00415643283811012\\
304.01	0.00415686122776703\\
305.01	0.00415729896483309\\
306.01	0.00415774626267394\\
307.01	0.0041582033400524\\
308.01	0.00415867042129625\\
309.01	0.00415914773647154\\
310.01	0.00415963552156495\\
311.01	0.00416013401867235\\
312.01	0.00416064347619738\\
313.01	0.00416116414905673\\
314.01	0.0041616962988969\\
315.01	0.0041622401943196\\
316.01	0.00416279611111757\\
317.01	0.0041633643325221\\
318.01	0.00416394514946271\\
319.01	0.00416453886083896\\
320.01	0.00416514577380607\\
321.01	0.00416576620407566\\
322.01	0.00416640047623104\\
323.01	0.00416704892406029\\
324.01	0.00416771189090702\\
325.01	0.00416838973004008\\
326.01	0.00416908280504448\\
327.01	0.00416979149023516\\
328.01	0.0041705161710944\\
329.01	0.00417125724473598\\
330.01	0.00417201512039816\\
331.01	0.00417279021996729\\
332.01	0.00417358297853551\\
333.01	0.00417439384499517\\
334.01	0.00417522328267431\\
335.01	0.00417607177001537\\
336.01	0.00417693980130271\\
337.01	0.00417782788744424\\
338.01	0.00417873655681118\\
339.01	0.00417966635614245\\
340.01	0.00418061785152207\\
341.01	0.00418159162943539\\
342.01	0.00418258829791465\\
343.01	0.00418360848778307\\
344.01	0.00418465285400919\\
345.01	0.0041857220771854\\
346.01	0.00418681686514386\\
347.01	0.00418793795472863\\
348.01	0.00418908611374225\\
349.01	0.00419026214308968\\
350.01	0.00419146687914455\\
351.01	0.0041927011963669\\
352.01	0.00419396601020493\\
353.01	0.00419526228031948\\
354.01	0.00419659101417189\\
355.01	0.00419795327102597\\
356.01	0.00419935016641657\\
357.01	0.00420078287714831\\
358.01	0.00420225264688966\\
359.01	0.00420376079244129\\
360.01	0.00420530871075799\\
361.01	0.00420689788681412\\
362.01	0.0042085299024041\\
363.01	0.0042102064459706\\
364.01	0.0042119293235499\\
365.01	0.00421370047090788\\
366.01	0.00421552196691868\\
367.01	0.00421739604819074\\
368.01	0.00421932512487386\\
369.01	0.00422131179747372\\
370.01	0.0042233588743284\\
371.01	0.00422546938915921\\
372.01	0.00422764661774756\\
373.01	0.00422989409227113\\
374.01	0.00423221561109659\\
375.01	0.00423461524077492\\
376.01	0.00423709730550745\\
377.01	0.00423966635727248\\
378.01	0.00424232711689224\\
379.01	0.00424508437225969\\
380.01	0.00424794281429006\\
381.01	0.00425090678330682\\
382.01	0.0042539798876709\\
383.01	0.00425716444136141\\
384.01	0.00426046064632322\\
385.01	0.00426386332049204\\
386.01	0.0042673453438255\\
387.01	0.00427090121902333\\
388.01	0.0042745325387893\\
389.01	0.00427824093201318\\
390.01	0.00428202806470326\\
391.01	0.00428589564094661\\
392.01	0.0042898454039004\\
393.01	0.00429387913681263\\
394.01	0.0042979986640759\\
395.01	0.00430220585231304\\
396.01	0.00430650261149555\\
397.01	0.00431089089609782\\
398.01	0.00431537270628513\\
399.01	0.00431995008913716\\
400.01	0.00432462513990922\\
401.01	0.00432940000332797\\
402.01	0.0043342768749257\\
403.01	0.00433925800241129\\
404.01	0.00434434568707703\\
405.01	0.00434954228524249\\
406.01	0.00435485020973335\\
407.01	0.00436027193139435\\
408.01	0.00436580998063417\\
409.01	0.00437146694899987\\
410.01	0.0043772454907771\\
411.01	0.00438314832461265\\
412.01	0.00438917823515285\\
413.01	0.00439533807469185\\
414.01	0.00440163076482111\\
415.01	0.00440805929806949\\
416.01	0.00441462673952259\\
417.01	0.0044213362284047\\
418.01	0.00442819097960613\\
419.01	0.00443519428513427\\
420.01	0.00444234951546026\\
421.01	0.00444966012073209\\
422.01	0.00445712963181495\\
423.01	0.00446476166111412\\
424.01	0.00447255990312809\\
425.01	0.00448052813466672\\
426.01	0.00448867021466183\\
427.01	0.00449699008347793\\
428.01	0.00450549176162084\\
429.01	0.0045141793477156\\
430.01	0.00452305701561066\\
431.01	0.00453212901043024\\
432.01	0.00454139964337136\\
433.01	0.00455087328500416\\
434.01	0.00456055435678991\\
435.01	0.00457044732048396\\
436.01	0.0045805566650301\\
437.01	0.00459088689049059\\
438.01	0.00460144248847508\\
439.01	0.0046122279184453\\
440.01	0.00462324757917482\\
441.01	0.00463450577452604\\
442.01	0.00464600667258623\\
443.01	0.00465775425706393\\
444.01	0.00466975226970179\\
445.01	0.00468200414230838\\
446.01	0.00469451291685857\\
447.01	0.00470728115197526\\
448.01	0.00472031081399361\\
449.01	0.00473360315075769\\
450.01	0.00474715854634261\\
451.01	0.00476097635509615\\
452.01	0.00477505471382782\\
453.01	0.00478939033177071\\
454.01	0.00480397825926507\\
455.01	0.00481881163820742\\
456.01	0.00483388144051397\\
457.01	0.00484917620564145\\
458.01	0.00486468179525587\\
459.01	0.00488038119337847\\
460.01	0.00489625439505458\\
461.01	0.00491227844760229\\
462.01	0.00492842773829337\\
463.01	0.00494467466437917\\
464.01	0.00496099088050544\\
465.01	0.00497734940141575\\
466.01	0.00499372795367321\\
467.01	0.00501011413174371\\
468.01	0.0050265133947317\\
469.01	0.00504306504391844\\
470.01	0.0050598695421144\\
471.01	0.00507692297182101\\
472.01	0.00509422048189447\\
473.01	0.00511175620985842\\
474.01	0.00512952320483855\\
475.01	0.0051475133534548\\
476.01	0.00516571731177395\\
477.01	0.00518412444739152\\
478.01	0.00520272279690785\\
479.01	0.00522149904552283\\
480.01	0.00524043853601257\\
481.01	0.00525952531454729\\
482.01	0.00527874222508858\\
483.01	0.00529807106619583\\
484.01	0.00531749282573263\\
485.01	0.00533698801071057\\
486.01	0.0053565370906694\\
487.01	0.00537612107299407\\
488.01	0.00539572222640368\\
489.01	0.0054153249630382\\
490.01	0.00543491687789839\\
491.01	0.00545448992356011\\
492.01	0.00547404166447779\\
493.01	0.00549357650763554\\
494.01	0.00551310673116578\\
495.01	0.00553265301150467\\
496.01	0.00555224397090921\\
497.01	0.00557191400421153\\
498.01	0.00559169612135395\\
499.01	0.00561159653081639\\
500.01	0.0056316070469707\\
501.01	0.00565172015606437\\
502.01	0.00567192950010262\\
503.01	0.0056922301992143\\
504.01	0.00571261918965564\\
505.01	0.00573309556467543\\
506.01	0.00575366090006868\\
507.01	0.00577431953984645\\
508.01	0.00579507881024807\\
509.01	0.00581594912282832\\
510.01	0.00583694392057781\\
511.01	0.00585807941681996\\
512.01	0.00587937407805668\\
513.01	0.00590084781407138\\
514.01	0.00592252086944919\\
515.01	0.00594441247270583\\
516.01	0.00596653941150554\\
517.01	0.00598891490954435\\
518.01	0.00601154936829866\\
519.01	0.00603445364319628\\
520.01	0.00605764003084672\\
521.01	0.00608112221141253\\
522.01	0.00610491512470457\\
523.01	0.00612903478785888\\
524.01	0.00615349805588999\\
525.01	0.00617832233322134\\
526.01	0.00620352525356942\\
527.01	0.00622912435644384\\
528.01	0.0062551368000985\\
529.01	0.00628157916136718\\
530.01	0.00630846737794379\\
531.01	0.00633581688014802\\
532.01	0.00636364292158714\\
533.01	0.00639196096373634\\
534.01	0.00642078677220527\\
535.01	0.00645013631912694\\
536.01	0.00648002567432941\\
537.01	0.00651047090700989\\
538.01	0.00654148800648144\\
539.01	0.00657309283000611\\
540.01	0.00660530108354772\\
541.01	0.00663812833697643\\
542.01	0.00667159006859329\\
543.01	0.00670570172519962\\
544.01	0.00674047877514396\\
545.01	0.00677593672836852\\
546.01	0.00681209111868202\\
547.01	0.00684895747719921\\
548.01	0.00688655131297821\\
549.01	0.00692488810181778\\
550.01	0.00696398328188797\\
551.01	0.00700385225360844\\
552.01	0.00704451038011664\\
553.01	0.00708597298416507\\
554.01	0.00712825533781296\\
555.01	0.00717137264321357\\
556.01	0.00721534000599387\\
557.01	0.00726017240483853\\
558.01	0.00730588465886365\\
559.01	0.00735249139193321\\
560.01	0.0074000069925621\\
561.01	0.00744844556806246\\
562.01	0.00749782089178799\\
563.01	0.00754814634269311\\
564.01	0.00759943483685668\\
565.01	0.00765169875093423\\
566.01	0.00770494983745968\\
567.01	0.0077591991315086\\
568.01	0.0078144568479118\\
569.01	0.00787073226819176\\
570.01	0.00792803361649234\\
571.01	0.00798636792389737\\
572.01	0.00804574088066074\\
573.01	0.00810615667597409\\
574.01	0.00816761782497417\\
575.01	0.0082301249827724\\
576.01	0.00829367674544343\\
577.01	0.00835826943818673\\
578.01	0.00842389689126298\\
579.01	0.00849055020480459\\
580.01	0.00855821750423072\\
581.01	0.00862688368879396\\
582.01	0.00869653017680659\\
583.01	0.00876713465240274\\
584.01	0.00883867082037334\\
585.01	0.00891110817774947\\
586.01	0.00898441181351145\\
587.01	0.0090585422512016\\
588.01	0.00913345535348011\\
589.01	0.00920910231300336\\
590.01	0.00928542976066912\\
591.01	0.00936238003058418\\
592.01	0.00943989163144331\\
593.01	0.00951789998683473\\
594.01	0.00959633852288064\\
595.01	0.00967514020129159\\
596.01	0.00975423962023931\\
597.01	0.00983357023250315\\
598.01	0.00990866201848071\\
599.01	0.00997087280416276\\
599.02	0.00997138072163725\\
599.03	0.00997188557625799\\
599.04	0.00997238733820211\\
599.05	0.00997288597735272\\
599.06	0.00997338146329604\\
599.07	0.00997387376531845\\
599.08	0.00997436285240349\\
599.09	0.00997484869322892\\
599.1	0.00997533125616361\\
599.11	0.00997581050926455\\
599.12	0.00997628642027372\\
599.13	0.00997675895661497\\
599.14	0.0099772280853909\\
599.15	0.00997769377337961\\
599.16	0.00997815598703157\\
599.17	0.00997861469246631\\
599.18	0.00997906985546915\\
599.19	0.00997952144148792\\
599.2	0.00997996941562957\\
599.21	0.00998041374265684\\
599.22	0.0099808543869848\\
599.23	0.00998129131267744\\
599.24	0.00998172448344418\\
599.25	0.00998215386263636\\
599.26	0.00998257941258354\\
599.27	0.00998300109279825\\
599.28	0.00998341886238981\\
599.29	0.00998383268006024\\
599.3	0.00998424250410032\\
599.31	0.00998464829238543\\
599.32	0.00998505000237151\\
599.33	0.00998544759109087\\
599.34	0.00998584101514799\\
599.35	0.00998623023071532\\
599.36	0.00998661519352896\\
599.37	0.00998699585888436\\
599.38	0.00998737218163197\\
599.39	0.00998774411617281\\
599.4	0.00998811161645403\\
599.41	0.00998847463596443\\
599.42	0.0099888331277299\\
599.43	0.00998918704430885\\
599.44	0.00998953633778757\\
599.45	0.00998988095977558\\
599.46	0.00999022086140088\\
599.47	0.00999055599330522\\
599.48	0.00999088630563925\\
599.49	0.00999121174805768\\
599.5	0.00999153226971435\\
599.51	0.0099918478192573\\
599.52	0.00999215834482374\\
599.53	0.00999246379403503\\
599.54	0.0099927641139915\\
599.55	0.00999305925126738\\
599.56	0.00999334915190553\\
599.57	0.0099936337614122\\
599.58	0.00999391302475173\\
599.59	0.00999418688634118\\
599.6	0.00999445529004489\\
599.61	0.00999471817916904\\
599.62	0.00999497549645613\\
599.63	0.00999522718407934\\
599.64	0.009995473183637\\
599.65	0.00999571343614678\\
599.66	0.00999594788204004\\
599.67	0.00999617646115598\\
599.68	0.0099963991127358\\
599.69	0.00999661577541675\\
599.7	0.00999682638722618\\
599.71	0.00999703088557551\\
599.72	0.00999722920725411\\
599.73	0.00999742128842315\\
599.74	0.00999760706460942\\
599.75	0.00999778647069897\\
599.76	0.00999795944093086\\
599.77	0.0099981259088907\\
599.78	0.0099982858075042\\
599.79	0.00999843906903064\\
599.8	0.00999858562505627\\
599.81	0.00999872540648767\\
599.82	0.00999885834354498\\
599.83	0.00999898436575515\\
599.84	0.00999910340194508\\
599.85	0.00999921538023465\\
599.86	0.00999932022802977\\
599.87	0.00999941787201528\\
599.88	0.00999950823814785\\
599.89	0.00999959125164875\\
599.9	0.00999966683699656\\
599.91	0.00999973491791987\\
599.92	0.0099997954173898\\
599.93	0.00999984825761255\\
599.94	0.00999989336002181\\
599.95	0.00999993064527112\\
599.96	0.00999996003322615\\
599.97	0.00999998144295691\\
599.98	0.00999999479272987\\
599.99	0.01\\
600	0.01\\
};
\addplot [color=mycolor9,solid,forget plot]
  table[row sep=crcr]{%
0.01	0.00400079225488101\\
1.01	0.00400079282381871\\
2.01	0.00400079340463527\\
3.01	0.00400079399757922\\
4.01	0.00400079460290462\\
5.01	0.00400079522087064\\
6.01	0.00400079585174193\\
7.01	0.00400079649578867\\
8.01	0.00400079715328687\\
9.01	0.0040007978245182\\
10.01	0.00400079850977031\\
11.01	0.00400079920933689\\
12.01	0.00400079992351778\\
13.01	0.00400080065261907\\
14.01	0.00400080139695326\\
15.01	0.00400080215683971\\
16.01	0.00400080293260409\\
17.01	0.00400080372457927\\
18.01	0.00400080453310467\\
19.01	0.00400080535852729\\
20.01	0.00400080620120127\\
21.01	0.0040008070614881\\
22.01	0.00400080793975696\\
23.01	0.00400080883638475\\
24.01	0.00400080975175637\\
25.01	0.00400081068626465\\
26.01	0.00400081164031092\\
27.01	0.00400081261430475\\
28.01	0.00400081360866447\\
29.01	0.00400081462381706\\
30.01	0.00400081566019855\\
31.01	0.00400081671825407\\
32.01	0.00400081779843835\\
33.01	0.00400081890121556\\
34.01	0.00400082002705943\\
35.01	0.00400082117645384\\
36.01	0.00400082234989289\\
37.01	0.00400082354788113\\
38.01	0.00400082477093339\\
39.01	0.00400082601957574\\
40.01	0.0040008272943452\\
41.01	0.00400082859578976\\
42.01	0.00400082992446962\\
43.01	0.0040008312809561\\
44.01	0.00400083266583293\\
45.01	0.00400083407969611\\
46.01	0.00400083552315401\\
47.01	0.00400083699682792\\
48.01	0.00400083850135223\\
49.01	0.00400084003737461\\
50.01	0.00400084160555644\\
51.01	0.00400084320657304\\
52.01	0.00400084484111402\\
53.01	0.00400084650988335\\
54.01	0.00400084821359988\\
55.01	0.00400084995299756\\
56.01	0.00400085172882597\\
57.01	0.00400085354185044\\
58.01	0.0040008553928523\\
59.01	0.00400085728262939\\
60.01	0.00400085921199654\\
61.01	0.00400086118178546\\
62.01	0.00400086319284565\\
63.01	0.00400086524604443\\
64.01	0.00400086734226725\\
65.01	0.0040008694824184\\
66.01	0.00400087166742125\\
67.01	0.00400087389821861\\
68.01	0.00400087617577295\\
69.01	0.0040008785010673\\
70.01	0.0040008808751055\\
71.01	0.00400088329891232\\
72.01	0.00400088577353426\\
73.01	0.00400088830003978\\
74.01	0.00400089087951998\\
75.01	0.00400089351308901\\
76.01	0.0040008962018844\\
77.01	0.00400089894706777\\
78.01	0.00400090174982534\\
79.01	0.00400090461136811\\
80.01	0.00400090753293288\\
81.01	0.00400091051578234\\
82.01	0.00400091356120622\\
83.01	0.00400091667052092\\
84.01	0.00400091984507096\\
85.01	0.00400092308622907\\
86.01	0.00400092639539736\\
87.01	0.00400092977400694\\
88.01	0.00400093322351967\\
89.01	0.00400093674542806\\
90.01	0.00400094034125598\\
91.01	0.00400094401255949\\
92.01	0.00400094776092788\\
93.01	0.00400095158798371\\
94.01	0.00400095549538389\\
95.01	0.00400095948482035\\
96.01	0.00400096355802075\\
97.01	0.00400096771674907\\
98.01	0.00400097196280687\\
99.01	0.00400097629803352\\
100.01	0.00400098072430735\\
101.01	0.00400098524354638\\
102.01	0.00400098985770925\\
103.01	0.00400099456879582\\
104.01	0.00400099937884846\\
105.01	0.0040010042899525\\
106.01	0.00400100930423727\\
107.01	0.00400101442387746\\
108.01	0.00400101965109331\\
109.01	0.00400102498815241\\
110.01	0.00400103043737005\\
111.01	0.00400103600111069\\
112.01	0.00400104168178841\\
113.01	0.00400104748186875\\
114.01	0.00400105340386914\\
115.01	0.00400105945036039\\
116.01	0.00400106562396758\\
117.01	0.00400107192737163\\
118.01	0.00400107836330978\\
119.01	0.00400108493457729\\
120.01	0.00400109164402875\\
121.01	0.00400109849457905\\
122.01	0.00400110548920472\\
123.01	0.00400111263094542\\
124.01	0.00400111992290493\\
125.01	0.00400112736825299\\
126.01	0.00400113497022629\\
127.01	0.00400114273213006\\
128.01	0.00400115065733947\\
129.01	0.00400115874930135\\
130.01	0.00400116701153531\\
131.01	0.00400117544763559\\
132.01	0.00400118406127273\\
133.01	0.00400119285619467\\
134.01	0.00400120183622904\\
135.01	0.00400121100528446\\
136.01	0.0040012203673522\\
137.01	0.0040012299265086\\
138.01	0.00400123968691588\\
139.01	0.00400124965282467\\
140.01	0.00400125982857562\\
141.01	0.00400127021860153\\
142.01	0.00400128082742915\\
143.01	0.00400129165968113\\
144.01	0.00400130272007827\\
145.01	0.00400131401344138\\
146.01	0.00400132554469354\\
147.01	0.00400133731886239\\
148.01	0.00400134934108201\\
149.01	0.00400136161659541\\
150.01	0.00400137415075716\\
151.01	0.00400138694903507\\
152.01	0.00400140001701337\\
153.01	0.00400141336039447\\
154.01	0.00400142698500212\\
155.01	0.00400144089678368\\
156.01	0.00400145510181263\\
157.01	0.00400146960629152\\
158.01	0.00400148441655476\\
159.01	0.00400149953907127\\
160.01	0.00400151498044732\\
161.01	0.00400153074742948\\
162.01	0.00400154684690811\\
163.01	0.00400156328591966\\
164.01	0.00400158007165034\\
165.01	0.00400159721143896\\
166.01	0.00400161471278061\\
167.01	0.00400163258332972\\
168.01	0.00400165083090352\\
169.01	0.00400166946348571\\
170.01	0.00400168848922954\\
171.01	0.0040017079164623\\
172.01	0.00400172775368818\\
173.01	0.00400174800959244\\
174.01	0.00400176869304568\\
175.01	0.00400178981310693\\
176.01	0.00400181137902857\\
177.01	0.00400183340025986\\
178.01	0.00400185588645174\\
179.01	0.00400187884746085\\
180.01	0.004001902293354\\
181.01	0.00400192623441251\\
182.01	0.00400195068113719\\
183.01	0.00400197564425293\\
184.01	0.00400200113471347\\
185.01	0.00400202716370644\\
186.01	0.00400205374265833\\
187.01	0.00400208088323978\\
188.01	0.00400210859737063\\
189.01	0.00400213689722557\\
190.01	0.00400216579523964\\
191.01	0.00400219530411368\\
192.01	0.00400222543682022\\
193.01	0.00400225620660934\\
194.01	0.00400228762701485\\
195.01	0.00400231971186019\\
196.01	0.00400235247526493\\
197.01	0.00400238593165128\\
198.01	0.00400242009575025\\
199.01	0.00400245498260905\\
200.01	0.00400249060759751\\
201.01	0.00400252698641519\\
202.01	0.0040025641350986\\
203.01	0.00400260207002887\\
204.01	0.00400264080793888\\
205.01	0.00400268036592107\\
206.01	0.00400272076143546\\
207.01	0.00400276201231771\\
208.01	0.0040028041367873\\
209.01	0.00400284715345606\\
210.01	0.00400289108133653\\
211.01	0.0040029359398511\\
212.01	0.00400298174884117\\
213.01	0.00400302852857604\\
214.01	0.00400307629976253\\
215.01	0.00400312508355492\\
216.01	0.00400317490156429\\
217.01	0.00400322577586912\\
218.01	0.00400327772902583\\
219.01	0.00400333078407886\\
220.01	0.0040033849645722\\
221.01	0.00400344029455958\\
222.01	0.00400349679861733\\
223.01	0.0040035545018547\\
224.01	0.00400361342992643\\
225.01	0.00400367360904473\\
226.01	0.00400373506599244\\
227.01	0.00400379782813527\\
228.01	0.00400386192343491\\
229.01	0.00400392738046294\\
230.01	0.0040039942284144\\
231.01	0.00400406249712196\\
232.01	0.00400413221707023\\
233.01	0.00400420341941109\\
234.01	0.00400427613597826\\
235.01	0.00400435039930348\\
236.01	0.00400442624263209\\
237.01	0.00400450369993956\\
238.01	0.00400458280594816\\
239.01	0.00400466359614446\\
240.01	0.00400474610679673\\
241.01	0.00400483037497298\\
242.01	0.00400491643855967\\
243.01	0.00400500433628072\\
244.01	0.00400509410771696\\
245.01	0.00400518579332625\\
246.01	0.00400527943446398\\
247.01	0.00400537507340401\\
248.01	0.00400547275336068\\
249.01	0.00400557251851065\\
250.01	0.00400567441401559\\
251.01	0.00400577848604586\\
252.01	0.0040058847818048\\
253.01	0.00400599334955253\\
254.01	0.00400610423863214\\
255.01	0.00400621749949535\\
256.01	0.00400633318372889\\
257.01	0.00400645134408252\\
258.01	0.00400657203449714\\
259.01	0.0040066953101334\\
260.01	0.00400682122740204\\
261.01	0.00400694984399408\\
262.01	0.00400708121891265\\
263.01	0.00400721541250496\\
264.01	0.0040073524864961\\
265.01	0.00400749250402305\\
266.01	0.00400763552967005\\
267.01	0.00400778162950483\\
268.01	0.00400793087111605\\
269.01	0.0040080833236515\\
270.01	0.00400823905785821\\
271.01	0.00400839814612274\\
272.01	0.00400856066251321\\
273.01	0.00400872668282293\\
274.01	0.00400889628461461\\
275.01	0.00400906954726678\\
276.01	0.00400924655202069\\
277.01	0.00400942738202939\\
278.01	0.00400961212240829\\
279.01	0.00400980086028671\\
280.01	0.00400999368486213\\
281.01	0.00401019068745494\\
282.01	0.00401039196156595\\
283.01	0.00401059760293537\\
284.01	0.00401080770960394\\
285.01	0.00401102238197561\\
286.01	0.00401124172288256\\
287.01	0.00401146583765299\\
288.01	0.00401169483418017\\
289.01	0.00401192882299489\\
290.01	0.00401216791733948\\
291.01	0.00401241223324518\\
292.01	0.00401266188961152\\
293.01	0.00401291700828935\\
294.01	0.00401317771416633\\
295.01	0.00401344413525533\\
296.01	0.00401371640278654\\
297.01	0.00401399465130289\\
298.01	0.0040142790187586\\
299.01	0.00401456964662193\\
300.01	0.00401486667998169\\
301.01	0.00401517026765781\\
302.01	0.00401548056231579\\
303.01	0.00401579772058686\\
304.01	0.00401612190319132\\
305.01	0.00401645327506792\\
306.01	0.00401679200550804\\
307.01	0.00401713826829567\\
308.01	0.00401749224185236\\
309.01	0.00401785410938972\\
310.01	0.00401822405906641\\
311.01	0.0040186022841539\\
312.01	0.00401898898320749\\
313.01	0.00401938436024643\\
314.01	0.00401978862494078\\
315.01	0.00402020199280703\\
316.01	0.0040206246854131\\
317.01	0.00402105693059175\\
318.01	0.00402149896266428\\
319.01	0.00402195102267505\\
320.01	0.00402241335863672\\
321.01	0.00402288622578666\\
322.01	0.00402336988685704\\
323.01	0.00402386461235664\\
324.01	0.00402437068086765\\
325.01	0.00402488837935699\\
326.01	0.00402541800350273\\
327.01	0.00402595985803753\\
328.01	0.00402651425711005\\
329.01	0.00402708152466413\\
330.01	0.00402766199483845\\
331.01	0.0040282560123867\\
332.01	0.00402886393312019\\
333.01	0.00402948612437388\\
334.01	0.00403012296549693\\
335.01	0.00403077484837001\\
336.01	0.00403144217794987\\
337.01	0.00403212537284327\\
338.01	0.00403282486591065\\
339.01	0.00403354110490371\\
340.01	0.00403427455313465\\
341.01	0.00403502569018123\\
342.01	0.00403579501262745\\
343.01	0.00403658303484196\\
344.01	0.00403739028979434\\
345.01	0.00403821732990874\\
346.01	0.00403906472795776\\
347.01	0.00403993307799215\\
348.01	0.0040408229963084\\
349.01	0.00404173512244888\\
350.01	0.00404267012023247\\
351.01	0.00404362867880947\\
352.01	0.00404461151373312\\
353.01	0.00404561936803739\\
354.01	0.00404665301330805\\
355.01	0.00404771325072806\\
356.01	0.00404880091207617\\
357.01	0.00404991686064848\\
358.01	0.00405106199206755\\
359.01	0.00405223723493328\\
360.01	0.00405344355125863\\
361.01	0.00405468193662199\\
362.01	0.00405595341995024\\
363.01	0.00405725906282824\\
364.01	0.00405859995821199\\
365.01	0.00405997722839344\\
366.01	0.00406139202204257\\
367.01	0.00406284551011538\\
368.01	0.00406433888039251\\
369.01	0.00406587333037041\\
370.01	0.00406745005820723\\
371.01	0.00406907025139548\\
372.01	0.00407073507283152\\
373.01	0.00407244564396998\\
374.01	0.00407420302481734\\
375.01	0.00407600819065625\\
376.01	0.00407786200563846\\
377.01	0.00407976519380208\\
378.01	0.00408171830873572\\
379.01	0.00408372170414938\\
380.01	0.00408577550919545\\
381.01	0.00408787961475132\\
382.01	0.00409003368038524\\
383.01	0.00409223717687271\\
384.01	0.00409448948661441\\
385.01	0.00409679010368409\\
386.01	0.00409913926378894\\
387.01	0.00410153796860382\\
388.01	0.00410398729457805\\
389.01	0.00410648834409962\\
390.01	0.00410904224630156\\
391.01	0.00411165015790546\\
392.01	0.00411431326410397\\
393.01	0.00411703277948469\\
394.01	0.00411980994899869\\
395.01	0.00412264604897609\\
396.01	0.00412554238819288\\
397.01	0.00412850030899023\\
398.01	0.0041315211884525\\
399.01	0.00413460643964725\\
400.01	0.00413775751292993\\
401.01	0.00414097589732168\\
402.01	0.00414426312196102\\
403.01	0.00414762075763914\\
404.01	0.00415105041842263\\
405.01	0.00415455376337169\\
406.01	0.00415813249836025\\
407.01	0.00416178837800729\\
408.01	0.00416552320772841\\
409.01	0.0041693388459159\\
410.01	0.00417323720626121\\
411.01	0.00417722026022958\\
412.01	0.00418129003970098\\
413.01	0.00418544863979343\\
414.01	0.00418969822188282\\
415.01	0.00419404101684077\\
416.01	0.00419847932850825\\
417.01	0.00420301553742987\\
418.01	0.00420765210487349\\
419.01	0.00421239157716445\\
420.01	0.00421723659036616\\
421.01	0.00422218987534434\\
422.01	0.00422725426325424\\
423.01	0.00423243269149921\\
424.01	0.00423772821021023\\
425.01	0.00424314398930774\\
426.01	0.0042486833262106\\
427.01	0.00425434965427011\\
428.01	0.00426014655201326\\
429.01	0.00426607775329521\\
430.01	0.00427214715847105\\
431.01	0.00427835884671442\\
432.01	0.00428471708962851\\
433.01	0.00429122636631332\\
434.01	0.00429789138007754\\
435.01	0.00430471707700844\\
436.01	0.00431170866664313\\
437.01	0.00431887164501668\\
438.01	0.00432621182039702\\
439.01	0.00433373534206047\\
440.01	0.00434144873249774\\
441.01	0.00434935892349288\\
442.01	0.00435747329655405\\
443.01	0.00436579972822673\\
444.01	0.00437434664085377\\
445.01	0.00438312305937479\\
446.01	0.00439213867476621\\
447.01	0.00440140391469866\\
448.01	0.00441093002191358\\
449.01	0.00442072914067407\\
450.01	0.00443081441138396\\
451.01	0.00444120007304674\\
452.01	0.00445190157258024\\
453.01	0.00446293567901471\\
454.01	0.00447432059913423\\
455.01	0.00448607608899386\\
456.01	0.00449822355266942\\
457.01	0.00451078611521795\\
458.01	0.00452378865060674\\
459.01	0.00453725773660674\\
460.01	0.00455122149634399\\
461.01	0.00456570926899146\\
462.01	0.00458075102808025\\
463.01	0.00459637643250742\\
464.01	0.00461261334892561\\
465.01	0.00462948561986687\\
466.01	0.00464700976283944\\
467.01	0.00466519016233086\\
468.01	0.004684011890909\\
469.01	0.00470332858609318\\
470.01	0.00472304126243356\\
471.01	0.00474315610270316\\
472.01	0.00476367892864757\\
473.01	0.0047846150933752\\
474.01	0.00480596935236996\\
475.01	0.00482774570927307\\
476.01	0.00484994723197578\\
477.01	0.00487257583379748\\
478.01	0.00489563201320967\\
479.01	0.00491911454605194\\
480.01	0.00494302015020838\\
481.01	0.00496734311061718\\
482.01	0.00499207483835325\\
483.01	0.00501720336700964\\
484.01	0.00504271278708326\\
485.01	0.0050685826294028\\
486.01	0.00509478721452446\\
487.01	0.00512129499896691\\
488.01	0.00514806797106716\\
489.01	0.00517506118207071\\
490.01	0.00520222254669445\\
491.01	0.00522949311615405\\
492.01	0.00525680805092996\\
493.01	0.0052840985061533\\
494.01	0.00531129494555366\\
495.01	0.00533833263621925\\
496.01	0.005365160328844\\
497.01	0.00539175353915022\\
498.01	0.00541822952524772\\
499.01	0.00544489036454156\\
500.01	0.00547172941255036\\
501.01	0.00549871773255214\\
502.01	0.00552582449500173\\
503.01	0.00555301731921735\\
504.01	0.00558026276534436\\
505.01	0.00560752700751449\\
506.01	0.00563477672031228\\
507.01	0.00566198020904772\\
508.01	0.00568910880783696\\
509.01	0.00571613855510147\\
510.01	0.00574305212934537\\
511.01	0.00576984098227466\\
512.01	0.00579650753175162\\
513.01	0.00582306715959326\\
514.01	0.00584954957850267\\
515.01	0.00587599885929561\\
516.01	0.00590247100249808\\
517.01	0.00592902591337369\\
518.01	0.00595568984274648\\
519.01	0.00598245979477857\\
520.01	0.00600933477015154\\
521.01	0.00603631748495947\\
522.01	0.00606341483578303\\
523.01	0.00609063826637363\\
524.01	0.00611800397038684\\
525.01	0.00614553284666085\\
526.01	0.00617325011730924\\
527.01	0.00620118453538543\\
528.01	0.00622936712841125\\
529.01	0.00625782947214204\\
530.01	0.00628660159347257\\
531.01	0.00631570979295472\\
532.01	0.00634517511355617\\
533.01	0.00637501547558893\\
534.01	0.00640524991762986\\
535.01	0.00643589910486549\\
536.01	0.00646698502056707\\
537.01	0.00649853057532576\\
538.01	0.006530559152647\\
539.01	0.00656309412379254\\
540.01	0.00659615838242811\\
541.01	0.0066297739676066\\
542.01	0.0066639618567796\\
543.01	0.00669874200887199\\
544.01	0.00673413370293458\\
545.01	0.00677015607463429\\
546.01	0.00680682837909413\\
547.01	0.00684416988246932\\
548.01	0.00688219971110505\\
549.01	0.00692093672659515\\
550.01	0.00696039943911544\\
551.01	0.00700060596789915\\
552.01	0.00704157405109965\\
553.01	0.00708332109679305\\
554.01	0.00712586425334399\\
555.01	0.0071692204641202\\
556.01	0.00721340647046484\\
557.01	0.00725843876908058\\
558.01	0.00730433356515286\\
559.01	0.0073511067342187\\
560.01	0.00739877379108111\\
561.01	0.00744734986172339\\
562.01	0.00749684965260175\\
563.01	0.00754728741102208\\
564.01	0.00759867687142414\\
565.01	0.00765103118587222\\
566.01	0.00770436284199182\\
567.01	0.00775868357290257\\
568.01	0.007814004259616\\
569.01	0.00787033482401756\\
570.01	0.0079276841103895\\
571.01	0.0079860597537612\\
572.01	0.00804546803401966\\
573.01	0.00810591371553068\\
574.01	0.00816739987270856\\
575.01	0.00822992770213072\\
576.01	0.00829349632141887\\
577.01	0.00835810255497104\\
578.01	0.00842374070703372\\
579.01	0.00849040232326687\\
580.01	0.0085580759427603\\
581.01	0.00862674684338341\\
582.01	0.00869639678438916\\
583.01	0.00876700375140449\\
584.01	0.00883854171047707\\
585.01	0.0089109803798979\\
586.01	0.00898428503117805\\
587.01	0.00905841633390994\\
588.01	0.00913333026343812\\
589.01	0.00920897809551196\\
590.01	0.00928530651866069\\
591.01	0.00936225790326229\\
592.01	0.00943977077658321\\
593.01	0.00951778056592611\\
594.01	0.00959622068800083\\
595.01	0.00967502408241994\\
596.01	0.00975412531166755\\
597.01	0.00983346338244348\\
598.01	0.0099086620184807\\
599.01	0.00997087280416276\\
599.02	0.00997138072163725\\
599.03	0.009971885576258\\
599.04	0.00997238733820211\\
599.05	0.00997288597735272\\
599.06	0.00997338146329604\\
599.07	0.00997387376531845\\
599.08	0.00997436285240349\\
599.09	0.00997484869322892\\
599.1	0.00997533125616361\\
599.11	0.00997581050926455\\
599.12	0.00997628642027372\\
599.13	0.00997675895661497\\
599.14	0.0099772280853909\\
599.15	0.00997769377337961\\
599.16	0.00997815598703157\\
599.17	0.00997861469246631\\
599.18	0.00997906985546915\\
599.19	0.00997952144148792\\
599.2	0.00997996941562957\\
599.21	0.00998041374265684\\
599.22	0.0099808543869848\\
599.23	0.00998129131267744\\
599.24	0.00998172448344418\\
599.25	0.00998215386263636\\
599.26	0.00998257941258353\\
599.27	0.00998300109279825\\
599.28	0.00998341886238981\\
599.29	0.00998383268006024\\
599.3	0.00998424250410032\\
599.31	0.00998464829238543\\
599.32	0.00998505000237151\\
599.33	0.00998544759109087\\
599.34	0.00998584101514799\\
599.35	0.00998623023071532\\
599.36	0.00998661519352895\\
599.37	0.00998699585888436\\
599.38	0.00998737218163197\\
599.39	0.00998774411617281\\
599.4	0.00998811161645403\\
599.41	0.00998847463596443\\
599.42	0.0099888331277299\\
599.43	0.00998918704430885\\
599.44	0.00998953633778757\\
599.45	0.00998988095977558\\
599.46	0.00999022086140088\\
599.47	0.00999055599330522\\
599.48	0.00999088630563925\\
599.49	0.00999121174805768\\
599.5	0.00999153226971435\\
599.51	0.0099918478192573\\
599.52	0.00999215834482374\\
599.53	0.00999246379403503\\
599.54	0.0099927641139915\\
599.55	0.00999305925126738\\
599.56	0.00999334915190553\\
599.57	0.0099936337614122\\
599.58	0.00999391302475173\\
599.59	0.00999418688634118\\
599.6	0.00999445529004489\\
599.61	0.00999471817916904\\
599.62	0.00999497549645613\\
599.63	0.00999522718407935\\
599.64	0.009995473183637\\
599.65	0.00999571343614678\\
599.66	0.00999594788204004\\
599.67	0.00999617646115599\\
599.68	0.0099963991127358\\
599.69	0.00999661577541675\\
599.7	0.00999682638722618\\
599.71	0.00999703088557551\\
599.72	0.00999722920725411\\
599.73	0.00999742128842315\\
599.74	0.00999760706460942\\
599.75	0.00999778647069897\\
599.76	0.00999795944093086\\
599.77	0.0099981259088907\\
599.78	0.0099982858075042\\
599.79	0.00999843906903064\\
599.8	0.00999858562505627\\
599.81	0.00999872540648766\\
599.82	0.00999885834354498\\
599.83	0.00999898436575515\\
599.84	0.00999910340194508\\
599.85	0.00999921538023465\\
599.86	0.00999932022802977\\
599.87	0.00999941787201528\\
599.88	0.00999950823814785\\
599.89	0.00999959125164875\\
599.9	0.00999966683699656\\
599.91	0.00999973491791987\\
599.92	0.0099997954173898\\
599.93	0.00999984825761255\\
599.94	0.00999989336002181\\
599.95	0.00999993064527112\\
599.96	0.00999996003322615\\
599.97	0.00999998144295691\\
599.98	0.00999999479272987\\
599.99	0.01\\
600	0.01\\
};
\addplot [color=blue!50!mycolor7,solid,forget plot]
  table[row sep=crcr]{%
0.01	0.00386858074314488\\
1.01	0.00386858122147336\\
2.01	0.00386858170981677\\
3.01	0.00386858220838564\\
4.01	0.0038685827173946\\
5.01	0.00386858323706303\\
6.01	0.00386858376761477\\
7.01	0.00386858430927869\\
8.01	0.00386858486228814\\
9.01	0.0038685854268815\\
10.01	0.00386858600330241\\
11.01	0.00386858659179941\\
12.01	0.00386858719262637\\
13.01	0.00386858780604245\\
14.01	0.00386858843231234\\
15.01	0.00386858907170616\\
16.01	0.00386858972449997\\
17.01	0.00386859039097541\\
18.01	0.00386859107142044\\
19.01	0.00386859176612868\\
20.01	0.00386859247540022\\
21.01	0.0038685931995414\\
22.01	0.00386859393886488\\
23.01	0.00386859469369029\\
24.01	0.00386859546434348\\
25.01	0.00386859625115768\\
26.01	0.00386859705447302\\
27.01	0.00386859787463661\\
28.01	0.00386859871200307\\
29.01	0.00386859956693467\\
30.01	0.00386860043980092\\
31.01	0.00386860133097955\\
32.01	0.00386860224085585\\
33.01	0.00386860316982367\\
34.01	0.00386860411828495\\
35.01	0.0038686050866502\\
36.01	0.00386860607533868\\
37.01	0.00386860708477837\\
38.01	0.00386860811540639\\
39.01	0.00386860916766897\\
40.01	0.00386861024202195\\
41.01	0.00386861133893077\\
42.01	0.00386861245887059\\
43.01	0.00386861360232681\\
44.01	0.00386861476979494\\
45.01	0.00386861596178115\\
46.01	0.00386861717880201\\
47.01	0.00386861842138536\\
48.01	0.00386861969007007\\
49.01	0.00386862098540641\\
50.01	0.00386862230795627\\
51.01	0.00386862365829355\\
52.01	0.00386862503700421\\
53.01	0.00386862644468644\\
54.01	0.00386862788195155\\
55.01	0.00386862934942353\\
56.01	0.00386863084773952\\
57.01	0.0038686323775501\\
58.01	0.00386863393951985\\
59.01	0.0038686355343274\\
60.01	0.0038686371626656\\
61.01	0.00386863882524223\\
62.01	0.0038686405227797\\
63.01	0.00386864225601603\\
64.01	0.00386864402570479\\
65.01	0.00386864583261561\\
66.01	0.00386864767753425\\
67.01	0.00386864956126334\\
68.01	0.00386865148462237\\
69.01	0.00386865344844835\\
70.01	0.00386865545359587\\
71.01	0.00386865750093775\\
72.01	0.00386865959136522\\
73.01	0.00386866172578866\\
74.01	0.00386866390513758\\
75.01	0.00386866613036128\\
76.01	0.00386866840242901\\
77.01	0.00386867072233096\\
78.01	0.00386867309107809\\
79.01	0.0038686755097028\\
80.01	0.00386867797925949\\
81.01	0.00386868050082491\\
82.01	0.00386868307549844\\
83.01	0.00386868570440344\\
84.01	0.00386868838868656\\
85.01	0.0038686911295191\\
86.01	0.00386869392809693\\
87.01	0.00386869678564163\\
88.01	0.00386869970340065\\
89.01	0.00386870268264786\\
90.01	0.00386870572468433\\
91.01	0.00386870883083892\\
92.01	0.00386871200246827\\
93.01	0.00386871524095812\\
94.01	0.00386871854772387\\
95.01	0.00386872192421076\\
96.01	0.00386872537189478\\
97.01	0.00386872889228353\\
98.01	0.00386873248691645\\
99.01	0.00386873615736574\\
100.01	0.00386873990523717\\
101.01	0.00386874373217061\\
102.01	0.00386874763984076\\
103.01	0.00386875162995819\\
104.01	0.00386875570426959\\
105.01	0.00386875986455903\\
106.01	0.00386876411264855\\
107.01	0.00386876845039894\\
108.01	0.0038687728797106\\
109.01	0.00386877740252452\\
110.01	0.00386878202082291\\
111.01	0.00386878673663024\\
112.01	0.00386879155201419\\
113.01	0.00386879646908631\\
114.01	0.00386880149000345\\
115.01	0.00386880661696813\\
116.01	0.00386881185223024\\
117.01	0.00386881719808707\\
118.01	0.00386882265688533\\
119.01	0.00386882823102166\\
120.01	0.00386883392294382\\
121.01	0.00386883973515184\\
122.01	0.00386884567019892\\
123.01	0.00386885173069293\\
124.01	0.00386885791929734\\
125.01	0.00386886423873228\\
126.01	0.00386887069177644\\
127.01	0.00386887728126729\\
128.01	0.00386888401010324\\
129.01	0.00386889088124455\\
130.01	0.00386889789771479\\
131.01	0.00386890506260176\\
132.01	0.0038689123790597\\
133.01	0.00386891985031\\
134.01	0.00386892747964292\\
135.01	0.00386893527041925\\
136.01	0.00386894322607161\\
137.01	0.00386895135010566\\
138.01	0.00386895964610259\\
139.01	0.00386896811771983\\
140.01	0.00386897676869327\\
141.01	0.0038689856028387\\
142.01	0.0038689946240536\\
143.01	0.00386900383631873\\
144.01	0.00386901324370049\\
145.01	0.00386902285035202\\
146.01	0.00386903266051551\\
147.01	0.00386904267852417\\
148.01	0.00386905290880393\\
149.01	0.00386906335587582\\
150.01	0.0038690740243575\\
151.01	0.00386908491896602\\
152.01	0.00386909604451913\\
153.01	0.00386910740593839\\
154.01	0.00386911900825068\\
155.01	0.003869130856591\\
156.01	0.00386914295620451\\
157.01	0.00386915531244907\\
158.01	0.00386916793079733\\
159.01	0.00386918081684007\\
160.01	0.00386919397628767\\
161.01	0.00386920741497371\\
162.01	0.00386922113885681\\
163.01	0.00386923515402379\\
164.01	0.00386924946669284\\
165.01	0.00386926408321553\\
166.01	0.00386927901008027\\
167.01	0.00386929425391519\\
168.01	0.00386930982149108\\
169.01	0.0038693257197249\\
170.01	0.00386934195568241\\
171.01	0.00386935853658167\\
172.01	0.0038693754697964\\
173.01	0.00386939276285952\\
174.01	0.00386941042346642\\
175.01	0.00386942845947846\\
176.01	0.00386944687892686\\
177.01	0.00386946569001652\\
178.01	0.00386948490112914\\
179.01	0.00386950452082786\\
180.01	0.00386952455786106\\
181.01	0.00386954502116617\\
182.01	0.00386956591987418\\
183.01	0.00386958726331355\\
184.01	0.00386960906101484\\
185.01	0.00386963132271513\\
186.01	0.00386965405836236\\
187.01	0.00386967727812036\\
188.01	0.0038697009923734\\
189.01	0.0038697252117309\\
190.01	0.00386974994703293\\
191.01	0.00386977520935472\\
192.01	0.00386980101001223\\
193.01	0.00386982736056742\\
194.01	0.00386985427283396\\
195.01	0.00386988175888232\\
196.01	0.00386990983104603\\
197.01	0.00386993850192668\\
198.01	0.00386996778440132\\
199.01	0.00386999769162739\\
200.01	0.00387002823704935\\
201.01	0.00387005943440528\\
202.01	0.00387009129773321\\
203.01	0.003870123841378\\
204.01	0.00387015707999833\\
205.01	0.00387019102857358\\
206.01	0.00387022570241097\\
207.01	0.00387026111715313\\
208.01	0.00387029728878577\\
209.01	0.00387033423364494\\
210.01	0.00387037196842557\\
211.01	0.00387041051018915\\
212.01	0.00387044987637228\\
213.01	0.00387049008479498\\
214.01	0.0038705311536697\\
215.01	0.00387057310160979\\
216.01	0.00387061594763929\\
217.01	0.00387065971120169\\
218.01	0.00387070441216977\\
219.01	0.00387075007085552\\
220.01	0.00387079670802004\\
221.01	0.00387084434488412\\
222.01	0.0038708930031381\\
223.01	0.00387094270495389\\
224.01	0.00387099347299485\\
225.01	0.00387104533042792\\
226.01	0.0038710983009351\\
227.01	0.00387115240872488\\
228.01	0.00387120767854547\\
229.01	0.00387126413569633\\
230.01	0.00387132180604194\\
231.01	0.003871380716024\\
232.01	0.0038714408926759\\
233.01	0.00387150236363576\\
234.01	0.00387156515716117\\
235.01	0.00387162930214342\\
236.01	0.0038716948281229\\
237.01	0.00387176176530395\\
238.01	0.00387183014457057\\
239.01	0.00387189999750268\\
240.01	0.00387197135639282\\
241.01	0.00387204425426282\\
242.01	0.00387211872488141\\
243.01	0.00387219480278177\\
244.01	0.00387227252328017\\
245.01	0.00387235192249445\\
246.01	0.00387243303736348\\
247.01	0.0038725159056671\\
248.01	0.00387260056604614\\
249.01	0.00387268705802328\\
250.01	0.00387277542202481\\
251.01	0.00387286569940221\\
252.01	0.00387295793245487\\
253.01	0.00387305216445341\\
254.01	0.00387314843966292\\
255.01	0.00387324680336815\\
256.01	0.0038733473018983\\
257.01	0.00387344998265256\\
258.01	0.00387355489412662\\
259.01	0.00387366208594028\\
260.01	0.00387377160886529\\
261.01	0.00387388351485358\\
262.01	0.00387399785706723\\
263.01	0.00387411468990886\\
264.01	0.00387423406905245\\
265.01	0.00387435605147568\\
266.01	0.00387448069549264\\
267.01	0.00387460806078781\\
268.01	0.00387473820845088\\
269.01	0.00387487120101261\\
270.01	0.00387500710248137\\
271.01	0.00387514597838114\\
272.01	0.00387528789579067\\
273.01	0.00387543292338315\\
274.01	0.00387558113146752\\
275.01	0.00387573259203067\\
276.01	0.00387588737878133\\
277.01	0.0038760455671947\\
278.01	0.00387620723455863\\
279.01	0.00387637246002108\\
280.01	0.0038765413246388\\
281.01	0.0038767139114279\\
282.01	0.00387689030541528\\
283.01	0.00387707059369199\\
284.01	0.00387725486546811\\
285.01	0.00387744321212886\\
286.01	0.0038776357272933\\
287.01	0.00387783250687326\\
288.01	0.00387803364913558\\
289.01	0.00387823925476498\\
290.01	0.00387844942692965\\
291.01	0.00387866427134811\\
292.01	0.0038788838963589\\
293.01	0.00387910841299116\\
294.01	0.00387933793503815\\
295.01	0.00387957257913316\\
296.01	0.00387981246482685\\
297.01	0.00388005771466718\\
298.01	0.00388030845428254\\
299.01	0.00388056481246622\\
300.01	0.00388082692126358\\
301.01	0.00388109491606271\\
302.01	0.00388136893568684\\
303.01	0.00388164912248991\\
304.01	0.00388193562245431\\
305.01	0.00388222858529291\\
306.01	0.00388252816455287\\
307.01	0.00388283451772257\\
308.01	0.00388314780634237\\
309.01	0.00388346819611796\\
310.01	0.00388379585703746\\
311.01	0.00388413096349123\\
312.01	0.00388447369439597\\
313.01	0.00388482423332162\\
314.01	0.003885182768622\\
315.01	0.00388554949356952\\
316.01	0.0038859246064932\\
317.01	0.00388630831092017\\
318.01	0.00388670081572171\\
319.01	0.00388710233526245\\
320.01	0.00388751308955389\\
321.01	0.00388793330441166\\
322.01	0.00388836321161637\\
323.01	0.00388880304907896\\
324.01	0.0038892530610095\\
325.01	0.00388971349808938\\
326.01	0.0038901846176479\\
327.01	0.00389066668384264\\
328.01	0.00389115996784191\\
329.01	0.00389166474801204\\
330.01	0.00389218131010673\\
331.01	0.00389270994745958\\
332.01	0.00389325096117865\\
333.01	0.00389380466034321\\
334.01	0.00389437136220222\\
335.01	0.00389495139237302\\
336.01	0.00389554508504091\\
337.01	0.00389615278315711\\
338.01	0.00389677483863729\\
339.01	0.00389741161255426\\
340.01	0.00389806347532944\\
341.01	0.00389873080691741\\
342.01	0.00389941399698415\\
343.01	0.0039001134450754\\
344.01	0.00390082956077403\\
345.01	0.00390156276384374\\
346.01	0.00390231348435555\\
347.01	0.00390308216279489\\
348.01	0.00390386925014351\\
349.01	0.00390467520793431\\
350.01	0.00390550050827339\\
351.01	0.00390634563382347\\
352.01	0.00390721107774351\\
353.01	0.0039080973435775\\
354.01	0.00390900494508627\\
355.01	0.00390993440601353\\
356.01	0.00391088625977783\\
357.01	0.00391186104908239\\
358.01	0.00391285932543314\\
359.01	0.00391388164855558\\
360.01	0.00391492858570175\\
361.01	0.00391600071083822\\
362.01	0.00391709860370817\\
363.01	0.00391822284876306\\
364.01	0.0039193740339603\\
365.01	0.00392055274943226\\
366.01	0.00392175958603289\\
367.01	0.00392299513378343\\
368.01	0.00392425998024432\\
369.01	0.00392555470886186\\
370.01	0.0039268798973519\\
371.01	0.00392823611621257\\
372.01	0.00392962392748478\\
373.01	0.00393104388391531\\
374.01	0.00393249652871842\\
375.01	0.00393398239617307\\
376.01	0.00393550201333963\\
377.01	0.00393705590321207\\
378.01	0.0039386445896398\\
379.01	0.0039402686043306\\
380.01	0.00394192849615328\\
381.01	0.00394362484274391\\
382.01	0.0039453582640135\\
383.01	0.00394712943642735\\
384.01	0.00394893910572075\\
385.01	0.00395078809372049\\
386.01	0.00395267728833708\\
387.01	0.00395460761274427\\
388.01	0.00395658001442778\\
389.01	0.00395859546571165\\
390.01	0.00396065496457514\\
391.01	0.0039627595355054\\
392.01	0.0039649102303875\\
393.01	0.00396710812943429\\
394.01	0.00396935434215751\\
395.01	0.00397165000838327\\
396.01	0.00397399629931416\\
397.01	0.0039763944186407\\
398.01	0.00397884560370544\\
399.01	0.00398135112672141\\
400.01	0.00398391229605037\\
401.01	0.00398653045754237\\
402.01	0.00398920699594149\\
403.01	0.00399194333636183\\
404.01	0.00399474094583709\\
405.01	0.00399760133494935\\
406.01	0.00400052605954149\\
407.01	0.00400351672251856\\
408.01	0.00400657497574333\\
409.01	0.0040097025220328\\
410.01	0.0040129011172612\\
411.01	0.00401617257257576\\
412.01	0.00401951875673466\\
413.01	0.00402294159857138\\
414.01	0.00402644308959658\\
415.01	0.00403002528674379\\
416.01	0.00403369031526886\\
417.01	0.00403744037181229\\
418.01	0.00404127772763399\\
419.01	0.00404520473203011\\
420.01	0.00404922381594372\\
421.01	0.00405333749577814\\
422.01	0.00405754837742431\\
423.01	0.00406185916051362\\
424.01	0.00406627264290478\\
425.01	0.00407079172541597\\
426.01	0.00407541941681033\\
427.01	0.0040801588390427\\
428.01	0.00408501323277298\\
429.01	0.00408998596314955\\
430.01	0.00409508052586133\\
431.01	0.00410030055345542\\
432.01	0.00410564982190721\\
433.01	0.00411113225742545\\
434.01	0.00411675194346493\\
435.01	0.00412251312790421\\
436.01	0.00412842023033567\\
437.01	0.00413447784938988\\
438.01	0.00414069076999928\\
439.01	0.00414706397047156\\
440.01	0.00415360262921131\\
441.01	0.00416031213088211\\
442.01	0.00416719807175235\\
443.01	0.00417426626390027\\
444.01	0.00418152273788162\\
445.01	0.00418897374337146\\
446.01	0.00419662574718474\\
447.01	0.00420448542795937\\
448.01	0.00421255966664272\\
449.01	0.00422085553176207\\
450.01	0.00422938025828778\\
451.01	0.0042381412187074\\
452.01	0.00424714588475078\\
453.01	0.00425640177802644\\
454.01	0.00426591640771269\\
455.01	0.0042756971934038\\
456.01	0.00428575137133055\\
457.01	0.00429608588254402\\
458.01	0.004306707242431\\
459.01	0.00431762139232646\\
460.01	0.00432883353633232\\
461.01	0.00434034797020055\\
462.01	0.00435216791495188\\
463.01	0.00436429537673048\\
464.01	0.00437673106759208\\
465.01	0.00438947444140124\\
466.01	0.00440252392748826\\
467.01	0.00441587748602856\\
468.01	0.00442953366884713\\
469.01	0.00444349452215706\\
470.01	0.00445776866461103\\
471.01	0.00447236656830442\\
472.01	0.00448729958656169\\
473.01	0.0045025800846319\\
474.01	0.00451822158966254\\
475.01	0.0045342389610175\\
476.01	0.00455064858089547\\
477.01	0.00456746856324833\\
478.01	0.00458471897573415\\
479.01	0.00460242206412984\\
480.01	0.00462060245982847\\
481.01	0.00463928734521334\\
482.01	0.00465850675671346\\
483.01	0.00467829418219433\\
484.01	0.00469868696563419\\
485.01	0.00471972658808292\\
486.01	0.00474145882382038\\
487.01	0.00476393367448086\\
488.01	0.00478720493526354\\
489.01	0.00481132917779667\\
490.01	0.00483636383485271\\
491.01	0.00486236422364897\\
492.01	0.0048893823801086\\
493.01	0.00491746667181199\\
494.01	0.00494665806710051\\
495.01	0.00497698424535896\\
496.01	0.00500845104106382\\
497.01	0.00504103014834045\\
498.01	0.00507454796308061\\
499.01	0.00510865853264725\\
500.01	0.00514332840972172\\
501.01	0.00517853973412254\\
502.01	0.00521426915877894\\
503.01	0.00525048700255988\\
504.01	0.00528715633502748\\
505.01	0.00532423201429295\\
506.01	0.00536165971570869\\
507.01	0.00539937501559937\\
508.01	0.00543730263400269\\
509.01	0.00547535599460236\\
510.01	0.00551343733817398\\
511.01	0.00555143873823\\
512.01	0.00558924452729373\\
513.01	0.00562673586838863\\
514.01	0.00566379852542793\\
515.01	0.00570033533525394\\
516.01	0.00573628551453981\\
517.01	0.00577173599208902\\
518.01	0.00580717753185216\\
519.01	0.005842642439381\\
520.01	0.00587808362130181\\
521.01	0.00591345447457746\\
522.01	0.00594871041434022\\
523.01	0.0059838108718764\\
524.01	0.00601872188111489\\
525.01	0.00605341932537118\\
526.01	0.00608789247573951\\
527.01	0.0061221474258837\\
528.01	0.00615621015398192\\
529.01	0.00619012854416304\\
530.01	0.00622397222772657\\
531.01	0.00625782842397304\\
532.01	0.00629178452997938\\
533.01	0.00632587055864795\\
534.01	0.00636009328503807\\
535.01	0.00639446549593649\\
536.01	0.00642900656512509\\
537.01	0.00646374254740046\\
538.01	0.00649870585890901\\
539.01	0.00653393442389336\\
540.01	0.00656947018386905\\
541.01	0.00660535691740503\\
542.01	0.00664163743538566\\
543.01	0.00667835044132781\\
544.01	0.00671552776347704\\
545.01	0.00675319498961828\\
546.01	0.00679137757058428\\
547.01	0.00683010261535099\\
548.01	0.00686939836821782\\
549.01	0.00690929355948017\\
550.01	0.00694981670841965\\
551.01	0.0069909954519336\\
552.01	0.00703285600150236\\
553.01	0.0070754228525436\\
554.01	0.00711871886371888\\
555.01	0.0071627657603584\\
556.01	0.00720758481787483\\
557.01	0.00725319703853558\\
558.01	0.00729962296373621\\
559.01	0.00734688249195777\\
560.01	0.0073949947477996\\
561.01	0.00744397801310743\\
562.01	0.00749384972275357\\
563.01	0.00754462651135841\\
564.01	0.00759632427494854\\
565.01	0.00764895819286281\\
566.01	0.00770254266877708\\
567.01	0.00775709122588741\\
568.01	0.00781261640385305\\
569.01	0.00786912966267725\\
570.01	0.00792664128800444\\
571.01	0.00798516028951031\\
572.01	0.00804469428287664\\
573.01	0.0081052493475248\\
574.01	0.0081668298578033\\
575.01	0.00822943829317042\\
576.01	0.00829307503476616\\
577.01	0.00835773814990506\\
578.01	0.00842342316335163\\
579.01	0.00849012281477226\\
580.01	0.00855782680318764\\
581.01	0.00862652152130265\\
582.01	0.0086961897849244\\
583.01	0.0087668105646243\\
584.01	0.00883835872788432\\
585.01	0.00891080480112841\\
586.01	0.00898411476341371\\
587.01	0.0090582498869396\\
588.01	0.00913316664376106\\
589.01	0.00920881670326823\\
590.01	0.00928514705127742\\
591.01	0.00936210026927741\\
592.01	0.00943961502205398\\
593.01	0.00951762681432733\\
594.01	0.00959606909286117\\
595.01	0.0096748747904041\\
596.01	0.00975397843263973\\
597.01	0.00983331896004518\\
598.01	0.0099086620184803\\
599.01	0.00997087280416276\\
599.02	0.00997138072163725\\
599.03	0.00997188557625799\\
599.04	0.00997238733820211\\
599.05	0.00997288597735272\\
599.06	0.00997338146329604\\
599.07	0.00997387376531845\\
599.08	0.00997436285240349\\
599.09	0.00997484869322892\\
599.1	0.00997533125616361\\
599.11	0.00997581050926455\\
599.12	0.00997628642027372\\
599.13	0.00997675895661498\\
599.14	0.0099772280853909\\
599.15	0.00997769377337961\\
599.16	0.00997815598703157\\
599.17	0.00997861469246631\\
599.18	0.00997906985546915\\
599.19	0.00997952144148792\\
599.2	0.00997996941562957\\
599.21	0.00998041374265684\\
599.22	0.0099808543869848\\
599.23	0.00998129131267744\\
599.24	0.00998172448344418\\
599.25	0.00998215386263636\\
599.26	0.00998257941258354\\
599.27	0.00998300109279825\\
599.28	0.00998341886238981\\
599.29	0.00998383268006024\\
599.3	0.00998424250410032\\
599.31	0.00998464829238543\\
599.32	0.00998505000237151\\
599.33	0.00998544759109087\\
599.34	0.00998584101514799\\
599.35	0.00998623023071532\\
599.36	0.00998661519352895\\
599.37	0.00998699585888436\\
599.38	0.00998737218163197\\
599.39	0.00998774411617281\\
599.4	0.00998811161645403\\
599.41	0.00998847463596444\\
599.42	0.0099888331277299\\
599.43	0.00998918704430885\\
599.44	0.00998953633778757\\
599.45	0.00998988095977558\\
599.46	0.00999022086140088\\
599.47	0.00999055599330522\\
599.48	0.00999088630563925\\
599.49	0.00999121174805768\\
599.5	0.00999153226971435\\
599.51	0.0099918478192573\\
599.52	0.00999215834482374\\
599.53	0.00999246379403503\\
599.54	0.0099927641139915\\
599.55	0.00999305925126738\\
599.56	0.00999334915190553\\
599.57	0.0099936337614122\\
599.58	0.00999391302475173\\
599.59	0.00999418688634118\\
599.6	0.00999445529004489\\
599.61	0.00999471817916904\\
599.62	0.00999497549645613\\
599.63	0.00999522718407934\\
599.64	0.009995473183637\\
599.65	0.00999571343614678\\
599.66	0.00999594788204004\\
599.67	0.00999617646115598\\
599.68	0.0099963991127358\\
599.69	0.00999661577541675\\
599.7	0.00999682638722618\\
599.71	0.00999703088557551\\
599.72	0.00999722920725411\\
599.73	0.00999742128842315\\
599.74	0.00999760706460942\\
599.75	0.00999778647069897\\
599.76	0.00999795944093086\\
599.77	0.0099981259088907\\
599.78	0.0099982858075042\\
599.79	0.00999843906903064\\
599.8	0.00999858562505627\\
599.81	0.00999872540648767\\
599.82	0.00999885834354498\\
599.83	0.00999898436575515\\
599.84	0.00999910340194508\\
599.85	0.00999921538023465\\
599.86	0.00999932022802977\\
599.87	0.00999941787201528\\
599.88	0.00999950823814785\\
599.89	0.00999959125164875\\
599.9	0.00999966683699656\\
599.91	0.00999973491791987\\
599.92	0.0099997954173898\\
599.93	0.00999984825761255\\
599.94	0.00999989336002181\\
599.95	0.00999993064527112\\
599.96	0.00999996003322615\\
599.97	0.00999998144295691\\
599.98	0.00999999479272987\\
599.99	0.01\\
600	0.01\\
};
\addplot [color=blue!40!mycolor9,solid,forget plot]
  table[row sep=crcr]{%
0.01	0.00356935369650614\\
1.01	0.0035693541789565\\
2.01	0.00356935467156345\\
3.01	0.00356935517454188\\
4.01	0.003569355688111\\
5.01	0.00356935621249488\\
6.01	0.00356935674792234\\
7.01	0.00356935729462701\\
8.01	0.00356935785284748\\
9.01	0.00356935842282738\\
10.01	0.00356935900481559\\
11.01	0.00356935959906622\\
12.01	0.00356936020583877\\
13.01	0.00356936082539841\\
14.01	0.00356936145801577\\
15.01	0.00356936210396751\\
16.01	0.00356936276353592\\
17.01	0.00356936343700942\\
18.01	0.00356936412468239\\
19.01	0.00356936482685569\\
20.01	0.00356936554383659\\
21.01	0.00356936627593862\\
22.01	0.00356936702348254\\
23.01	0.00356936778679537\\
24.01	0.00356936856621158\\
25.01	0.00356936936207233\\
26.01	0.00356937017472626\\
27.01	0.00356937100452949\\
28.01	0.00356937185184578\\
29.01	0.00356937271704624\\
30.01	0.00356937360051053\\
31.01	0.00356937450262593\\
32.01	0.0035693754237882\\
33.01	0.00356937636440138\\
34.01	0.0035693773248783\\
35.01	0.00356937830564054\\
36.01	0.00356937930711877\\
37.01	0.00356938032975258\\
38.01	0.00356938137399115\\
39.01	0.00356938244029334\\
40.01	0.00356938352912765\\
41.01	0.00356938464097266\\
42.01	0.00356938577631713\\
43.01	0.00356938693566027\\
44.01	0.00356938811951205\\
45.01	0.00356938932839322\\
46.01	0.00356939056283574\\
47.01	0.00356939182338285\\
48.01	0.00356939311058957\\
49.01	0.0035693944250226\\
50.01	0.00356939576726092\\
51.01	0.00356939713789588\\
52.01	0.0035693985375314\\
53.01	0.0035693999667844\\
54.01	0.00356940142628485\\
55.01	0.00356940291667644\\
56.01	0.00356940443861649\\
57.01	0.00356940599277643\\
58.01	0.00356940757984218\\
59.01	0.00356940920051417\\
60.01	0.00356941085550791\\
61.01	0.00356941254555425\\
62.01	0.00356941427139984\\
63.01	0.00356941603380698\\
64.01	0.00356941783355474\\
65.01	0.00356941967143858\\
66.01	0.00356942154827111\\
67.01	0.00356942346488234\\
68.01	0.00356942542212033\\
69.01	0.00356942742085103\\
70.01	0.00356942946195912\\
71.01	0.0035694315463482\\
72.01	0.0035694336749415\\
73.01	0.00356943584868165\\
74.01	0.00356943806853179\\
75.01	0.00356944033547574\\
76.01	0.00356944265051855\\
77.01	0.00356944501468662\\
78.01	0.00356944742902841\\
79.01	0.00356944989461534\\
80.01	0.00356945241254138\\
81.01	0.00356945498392423\\
82.01	0.00356945760990569\\
83.01	0.00356946029165188\\
84.01	0.00356946303035427\\
85.01	0.00356946582722982\\
86.01	0.00356946868352175\\
87.01	0.0035694716005001\\
88.01	0.00356947457946204\\
89.01	0.00356947762173299\\
90.01	0.00356948072866653\\
91.01	0.0035694839016457\\
92.01	0.00356948714208323\\
93.01	0.00356949045142237\\
94.01	0.00356949383113733\\
95.01	0.0035694972827342\\
96.01	0.00356950080775159\\
97.01	0.00356950440776119\\
98.01	0.00356950808436861\\
99.01	0.00356951183921431\\
100.01	0.00356951567397391\\
101.01	0.00356951959035923\\
102.01	0.00356952359011911\\
103.01	0.00356952767504026\\
104.01	0.00356953184694772\\
105.01	0.00356953610770608\\
106.01	0.00356954045922046\\
107.01	0.00356954490343674\\
108.01	0.00356954944234325\\
109.01	0.0035695540779709\\
110.01	0.00356955881239492\\
111.01	0.00356956364773511\\
112.01	0.00356956858615737\\
113.01	0.00356957362987451\\
114.01	0.00356957878114701\\
115.01	0.00356958404228451\\
116.01	0.00356958941564639\\
117.01	0.00356959490364335\\
118.01	0.00356960050873852\\
119.01	0.00356960623344788\\
120.01	0.00356961208034247\\
121.01	0.00356961805204864\\
122.01	0.00356962415125012\\
123.01	0.00356963038068857\\
124.01	0.00356963674316516\\
125.01	0.00356964324154208\\
126.01	0.00356964987874328\\
127.01	0.00356965665775656\\
128.01	0.00356966358163448\\
129.01	0.00356967065349578\\
130.01	0.00356967787652725\\
131.01	0.00356968525398468\\
132.01	0.00356969278919485\\
133.01	0.0035697004855569\\
134.01	0.00356970834654367\\
135.01	0.00356971637570373\\
136.01	0.00356972457666296\\
137.01	0.00356973295312622\\
138.01	0.0035697415088789\\
139.01	0.00356975024778874\\
140.01	0.00356975917380811\\
141.01	0.00356976829097526\\
142.01	0.00356977760341658\\
143.01	0.00356978711534861\\
144.01	0.00356979683107955\\
145.01	0.00356980675501176\\
146.01	0.00356981689164382\\
147.01	0.00356982724557248\\
148.01	0.00356983782149475\\
149.01	0.00356984862421054\\
150.01	0.00356985965862446\\
151.01	0.00356987092974825\\
152.01	0.00356988244270362\\
153.01	0.00356989420272411\\
154.01	0.00356990621515779\\
155.01	0.00356991848546973\\
156.01	0.00356993101924495\\
157.01	0.00356994382219063\\
158.01	0.00356995690013892\\
159.01	0.00356997025904997\\
160.01	0.00356998390501463\\
161.01	0.00356999784425741\\
162.01	0.00357001208313922\\
163.01	0.00357002662816094\\
164.01	0.00357004148596586\\
165.01	0.00357005666334344\\
166.01	0.00357007216723233\\
167.01	0.00357008800472368\\
168.01	0.00357010418306469\\
169.01	0.00357012070966183\\
170.01	0.00357013759208466\\
171.01	0.00357015483806954\\
172.01	0.00357017245552338\\
173.01	0.003570190452527\\
174.01	0.00357020883733943\\
175.01	0.00357022761840199\\
176.01	0.00357024680434229\\
177.01	0.00357026640397823\\
178.01	0.00357028642632264\\
179.01	0.0035703068805874\\
180.01	0.00357032777618782\\
181.01	0.00357034912274765\\
182.01	0.00357037093010336\\
183.01	0.00357039320830949\\
184.01	0.00357041596764265\\
185.01	0.00357043921860747\\
186.01	0.00357046297194154\\
187.01	0.00357048723862011\\
188.01	0.00357051202986224\\
189.01	0.00357053735713599\\
190.01	0.00357056323216381\\
191.01	0.00357058966692904\\
192.01	0.00357061667368105\\
193.01	0.00357064426494189\\
194.01	0.00357067245351184\\
195.01	0.00357070125247674\\
196.01	0.00357073067521334\\
197.01	0.00357076073539701\\
198.01	0.0035707914470078\\
199.01	0.00357082282433764\\
200.01	0.00357085488199765\\
201.01	0.00357088763492534\\
202.01	0.0035709210983919\\
203.01	0.00357095528801027\\
204.01	0.00357099021974247\\
205.01	0.00357102590990781\\
206.01	0.00357106237519164\\
207.01	0.00357109963265299\\
208.01	0.0035711376997334\\
209.01	0.00357117659426605\\
210.01	0.0035712163344844\\
211.01	0.00357125693903206\\
212.01	0.00357129842697129\\
213.01	0.00357134081779364\\
214.01	0.00357138413142935\\
215.01	0.00357142838825765\\
216.01	0.0035714736091173\\
217.01	0.00357151981531727\\
218.01	0.0035715670286473\\
219.01	0.00357161527138948\\
220.01	0.00357166456632982\\
221.01	0.00357171493676962\\
222.01	0.00357176640653782\\
223.01	0.00357181900000292\\
224.01	0.00357187274208633\\
225.01	0.00357192765827472\\
226.01	0.00357198377463296\\
227.01	0.00357204111781907\\
228.01	0.00357209971509662\\
229.01	0.00357215959434987\\
230.01	0.00357222078409804\\
231.01	0.00357228331351042\\
232.01	0.00357234721242168\\
233.01	0.00357241251134764\\
234.01	0.00357247924150125\\
235.01	0.0035725474348093\\
236.01	0.00357261712392903\\
237.01	0.00357268834226578\\
238.01	0.00357276112399071\\
239.01	0.00357283550405872\\
240.01	0.00357291151822756\\
241.01	0.00357298920307673\\
242.01	0.00357306859602675\\
243.01	0.00357314973536013\\
244.01	0.00357323266024112\\
245.01	0.00357331741073738\\
246.01	0.0035734040278412\\
247.01	0.00357349255349197\\
248.01	0.00357358303059889\\
249.01	0.00357367550306435\\
250.01	0.00357377001580761\\
251.01	0.00357386661478948\\
252.01	0.00357396534703736\\
253.01	0.00357406626067107\\
254.01	0.00357416940492954\\
255.01	0.00357427483019705\\
256.01	0.0035743825880318\\
257.01	0.00357449273119416\\
258.01	0.00357460531367555\\
259.01	0.00357472039072872\\
260.01	0.00357483801889809\\
261.01	0.00357495825605129\\
262.01	0.00357508116141174\\
263.01	0.00357520679559105\\
264.01	0.0035753352206235\\
265.01	0.0035754665000003\\
266.01	0.00357560069870552\\
267.01	0.00357573788325241\\
268.01	0.00357587812172097\\
269.01	0.00357602148379602\\
270.01	0.00357616804080665\\
271.01	0.0035763178657669\\
272.01	0.00357647103341641\\
273.01	0.00357662762026308\\
274.01	0.00357678770462662\\
275.01	0.0035769513666829\\
276.01	0.00357711868850955\\
277.01	0.00357728975413245\\
278.01	0.0035774646495737\\
279.01	0.00357764346290101\\
280.01	0.0035778262842775\\
281.01	0.00357801320601334\\
282.01	0.0035782043226186\\
283.01	0.00357839973085708\\
284.01	0.00357859952980157\\
285.01	0.00357880382089096\\
286.01	0.00357901270798767\\
287.01	0.00357922629743751\\
288.01	0.00357944469813003\\
289.01	0.00357966802156129\\
290.01	0.003579896381897\\
291.01	0.00358012989603849\\
292.01	0.00358036868368844\\
293.01	0.00358061286742002\\
294.01	0.00358086257274595\\
295.01	0.0035811179281903\\
296.01	0.00358137906536138\\
297.01	0.00358164611902638\\
298.01	0.00358191922718731\\
299.01	0.00358219853115907\\
300.01	0.00358248417564937\\
301.01	0.00358277630883916\\
302.01	0.00358307508246659\\
303.01	0.0035833806519107\\
304.01	0.0035836931762786\\
305.01	0.00358401281849324\\
306.01	0.00358433974538321\\
307.01	0.00358467412777473\\
308.01	0.00358501614058483\\
309.01	0.00358536596291655\\
310.01	0.00358572377815553\\
311.01	0.00358608977406931\\
312.01	0.00358646414290688\\
313.01	0.0035868470815013\\
314.01	0.00358723879137322\\
315.01	0.00358763947883652\\
316.01	0.00358804935510501\\
317.01	0.00358846863640165\\
318.01	0.00358889754406856\\
319.01	0.00358933630467896\\
320.01	0.00358978515015054\\
321.01	0.00359024431786005\\
322.01	0.00359071405076002\\
323.01	0.00359119459749599\\
324.01	0.00359168621252533\\
325.01	0.00359218915623769\\
326.01	0.00359270369507596\\
327.01	0.00359323010165832\\
328.01	0.00359376865490199\\
329.01	0.00359431964014711\\
330.01	0.00359488334928176\\
331.01	0.0035954600808672\\
332.01	0.00359605014026447\\
333.01	0.00359665383976118\\
334.01	0.00359727149869771\\
335.01	0.00359790344359505\\
336.01	0.00359855000828187\\
337.01	0.00359921153402205\\
338.01	0.00359988836964127\\
339.01	0.00360058087165412\\
340.01	0.00360128940439082\\
341.01	0.00360201434012253\\
342.01	0.00360275605918707\\
343.01	0.00360351495011342\\
344.01	0.00360429140974572\\
345.01	0.00360508584336625\\
346.01	0.00360589866481749\\
347.01	0.00360673029662416\\
348.01	0.00360758117011349\\
349.01	0.0036084517255366\\
350.01	0.00360934241218824\\
351.01	0.00361025368852749\\
352.01	0.00361118602229971\\
353.01	0.0036121398906598\\
354.01	0.0036131157802978\\
355.01	0.0036141141875686\\
356.01	0.00361513561862792\\
357.01	0.00361618058957471\\
358.01	0.00361724962660467\\
359.01	0.00361834326617612\\
360.01	0.00361946205519366\\
361.01	0.0036206065512127\\
362.01	0.00362177732267134\\
363.01	0.00362297494915492\\
364.01	0.00362420002170184\\
365.01	0.00362545314315821\\
366.01	0.00362673492859153\\
367.01	0.00362804600577428\\
368.01	0.00362938701574894\\
369.01	0.00363075861348704\\
370.01	0.00363216146865467\\
371.01	0.00363359626649709\\
372.01	0.00363506370885205\\
373.01	0.0036365645152997\\
374.01	0.00363809942445012\\
375.01	0.00363966919536403\\
376.01	0.00364127460908717\\
377.01	0.00364291647026863\\
378.01	0.00364459560880954\\
379.01	0.0036463128814699\\
380.01	0.00364806917333388\\
381.01	0.00364986539901338\\
382.01	0.00365170250345209\\
383.01	0.00365358146220757\\
384.01	0.00365550328113723\\
385.01	0.00365746899556441\\
386.01	0.00365947966931873\\
387.01	0.00366153639456435\\
388.01	0.00366364029248849\\
389.01	0.00366579251415143\\
390.01	0.00366799424136459\\
391.01	0.00367024668759425\\
392.01	0.00367255109889375\\
393.01	0.00367490875486343\\
394.01	0.00367732096963948\\
395.01	0.00367978909291161\\
396.01	0.00368231451097226\\
397.01	0.00368489864779506\\
398.01	0.00368754296614511\\
399.01	0.00369024896872178\\
400.01	0.00369301819933206\\
401.01	0.00369585224409821\\
402.01	0.003698752732698\\
403.01	0.003701721339638\\
404.01	0.00370475978556105\\
405.01	0.00370786983858665\\
406.01	0.0037110533156855\\
407.01	0.00371431208408747\\
408.01	0.00371764806272219\\
409.01	0.00372106322369291\\
410.01	0.00372455959378086\\
411.01	0.00372813925598164\\
412.01	0.00373180435106996\\
413.01	0.00373555707919232\\
414.01	0.00373939970148496\\
415.01	0.00374333454171481\\
416.01	0.00374736398794043\\
417.01	0.00375149049418813\\
418.01	0.00375571658214024\\
419.01	0.00376004484283006\\
420.01	0.00376447793833572\\
421.01	0.00376901860346752\\
422.01	0.00377366964744026\\
423.01	0.00377843395551854\\
424.01	0.00378331449062667\\
425.01	0.00378831429490753\\
426.01	0.00379343649121612\\
427.01	0.0037986842845302\\
428.01	0.00380406096325764\\
429.01	0.00380956990041652\\
430.01	0.00381521455466308\\
431.01	0.00382099847113472\\
432.01	0.00382692528207579\\
433.01	0.00383299870720497\\
434.01	0.00383922255378029\\
435.01	0.00384560071631176\\
436.01	0.00385213717586352\\
437.01	0.00385883599888315\\
438.01	0.00386570133548747\\
439.01	0.00387273741712474\\
440.01	0.00387994855352876\\
441.01	0.00388733912886977\\
442.01	0.00389491359700014\\
443.01	0.00390267647568868\\
444.01	0.0039106323397283\\
445.01	0.00391878581280282\\
446.01	0.00392714155799803\\
447.01	0.00393570426684846\\
448.01	0.00394447864682365\\
449.01	0.003953469407182\\
450.01	0.00396268124315192\\
451.01	0.00397211881845402\\
452.01	0.00398178674624092\\
453.01	0.00399168956863305\\
454.01	0.0040018317351434\\
455.01	0.00401221758044108\\
456.01	0.00402285130209053\\
457.01	0.00403373693912723\\
458.01	0.00404487835258123\\
459.01	0.00405627920932505\\
460.01	0.00406794297087019\\
461.01	0.00407987288890881\\
462.01	0.00409207200940272\\
463.01	0.00410454318670068\\
464.01	0.00411728910826407\\
465.01	0.00413031232870952\\
466.01	0.00414361530839268\\
467.01	0.00415720044562973\\
468.01	0.00417107008107845\\
469.01	0.0041852264260813\\
470.01	0.0041996713398888\\
471.01	0.00421440617148017\\
472.01	0.00422943173661071\\
473.01	0.00424474833547754\\
474.01	0.00426035582131055\\
475.01	0.00427625374672212\\
476.01	0.00429244162620634\\
477.01	0.00430891936952346\\
478.01	0.00432568796411437\\
479.01	0.00434275051881251\\
480.01	0.00436011382294717\\
481.01	0.00437778925709713\\
482.01	0.00439578057004691\\
483.01	0.00441408517387566\\
484.01	0.00443270191090969\\
485.01	0.00445163272657016\\
486.01	0.00447088499173598\\
487.01	0.00449047488246039\\
488.01	0.00451043223607692\\
489.01	0.0045308074601669\\
490.01	0.00455168128593993\\
491.01	0.00457315422736754\\
492.01	0.00459526008907832\\
493.01	0.0046180086043931\\
494.01	0.00464140788574584\\
495.01	0.00466546479504304\\
496.01	0.00469018583076554\\
497.01	0.00471557883003723\\
498.01	0.00474165656460672\\
499.01	0.00476844518523685\\
500.01	0.00479597870150315\\
501.01	0.00482429399915098\\
502.01	0.00485343100202368\\
503.01	0.0048834328864565\\
504.01	0.00491434627623896\\
505.01	0.00494622140426403\\
506.01	0.00497911220262801\\
507.01	0.00501307619044987\\
508.01	0.00504817404249849\\
509.01	0.00508446872486107\\
510.01	0.00512202399523604\\
511.01	0.00516090197633048\\
512.01	0.00520115939160515\\
513.01	0.0052428418864781\\
514.01	0.0052859756268194\\
515.01	0.00533055504474222\\
516.01	0.00537652515399289\\
517.01	0.00542367506389353\\
518.01	0.00547140676547196\\
519.01	0.0055195941845584\\
520.01	0.00556818235367103\\
521.01	0.00561710597037564\\
522.01	0.00566628677968869\\
523.01	0.00571562983441296\\
524.01	0.00576501816609789\\
525.01	0.00581430985838814\\
526.01	0.00586334494528416\\
527.01	0.00591195183244157\\
528.01	0.00595995539606151\\
529.01	0.00600718933154067\\
530.01	0.00605351576115472\\
531.01	0.00609885554544969\\
532.01	0.00614347792173343\\
533.01	0.00618796876807136\\
534.01	0.00623229035747514\\
535.01	0.00627638553022832\\
536.01	0.00632020584718589\\
537.01	0.00636371544694091\\
538.01	0.00640689538419013\\
539.01	0.00644974820779094\\
540.01	0.00649230232892437\\
541.01	0.00653461539186144\\
542.01	0.00657677531915586\\
543.01	0.00661889690504216\\
544.01	0.00666110859425139\\
545.01	0.00670348372223335\\
546.01	0.00674604040021065\\
547.01	0.00678880258411917\\
548.01	0.00683180313316688\\
549.01	0.00687508335988836\\
550.01	0.00691869185821967\\
551.01	0.00696268244849074\\
552.01	0.00700711107313386\\
553.01	0.0070520317686205\\
554.01	0.0070974922108247\\
555.01	0.00714353011714465\\
556.01	0.00719017612575834\\
557.01	0.0072374611684107\\
558.01	0.00728541715086486\\
559.01	0.00733407609564801\\
560.01	0.00738346924975821\\
561.01	0.00743362623227398\\
562.01	0.00748457434096044\\
563.01	0.00753633822330531\\
564.01	0.00758894009010531\\
565.01	0.00764240049063996\\
566.01	0.00769673903354819\\
567.01	0.00775197432695984\\
568.01	0.00780812368455662\\
569.01	0.00786520288823177\\
570.01	0.00792322603603263\\
571.01	0.00798220548225692\\
572.01	0.00804215184843037\\
573.01	0.00810307405076856\\
574.01	0.00816497926209387\\
575.01	0.0082278727525132\\
576.01	0.00829175766618825\\
577.01	0.00835663479827209\\
578.01	0.00842250237713441\\
579.01	0.00848935584384751\\
580.01	0.00855718761791611\\
581.01	0.00862598683882235\\
582.01	0.0086957390795814\\
583.01	0.00876642604196961\\
584.01	0.00883802525422727\\
585.01	0.00891050978858627\\
586.01	0.00898384801195749\\
587.01	0.00905800338572644\\
588.01	0.00913293433529357\\
589.01	0.00920859421638682\\
590.01	0.00928493141303536\\
591.01	0.0093618896103455\\
592.01	0.00943940829294075\\
593.01	0.00951742352873278\\
594.01	0.00959586911059195\\
595.01	0.0096746781463672\\
596.01	0.00975378521137988\\
597.01	0.00983312920843281\\
598.01	0.00990866201846347\\
599.01	0.00997087280416267\\
599.02	0.00997138072163717\\
599.03	0.00997188557625792\\
599.04	0.00997238733820205\\
599.05	0.00997288597735266\\
599.06	0.00997338146329598\\
599.07	0.00997387376531839\\
599.08	0.00997436285240344\\
599.09	0.00997484869322887\\
599.1	0.00997533125616357\\
599.11	0.00997581050926451\\
599.12	0.00997628642027369\\
599.13	0.00997675895661494\\
599.14	0.00997722808539087\\
599.15	0.00997769377337959\\
599.16	0.00997815598703155\\
599.17	0.00997861469246629\\
599.18	0.00997906985546913\\
599.19	0.0099795214414879\\
599.2	0.00997996941562956\\
599.21	0.00998041374265682\\
599.22	0.00998085438698478\\
599.23	0.00998129131267743\\
599.24	0.00998172448344417\\
599.25	0.00998215386263635\\
599.26	0.00998257941258352\\
599.27	0.00998300109279824\\
599.28	0.0099834188623898\\
599.29	0.00998383268006024\\
599.3	0.00998424250410031\\
599.31	0.00998464829238542\\
599.32	0.0099850500023715\\
599.33	0.00998544759109086\\
599.34	0.00998584101514799\\
599.35	0.00998623023071531\\
599.36	0.00998661519352895\\
599.37	0.00998699585888435\\
599.38	0.00998737218163196\\
599.39	0.00998774411617281\\
599.4	0.00998811161645403\\
599.41	0.00998847463596443\\
599.42	0.0099888331277299\\
599.43	0.00998918704430885\\
599.44	0.00998953633778757\\
599.45	0.00998988095977558\\
599.46	0.00999022086140088\\
599.47	0.00999055599330522\\
599.48	0.00999088630563925\\
599.49	0.00999121174805768\\
599.5	0.00999153226971434\\
599.51	0.0099918478192573\\
599.52	0.00999215834482374\\
599.53	0.00999246379403503\\
599.54	0.0099927641139915\\
599.55	0.00999305925126738\\
599.56	0.00999334915190552\\
599.57	0.0099936337614122\\
599.58	0.00999391302475173\\
599.59	0.00999418688634118\\
599.6	0.00999445529004489\\
599.61	0.00999471817916904\\
599.62	0.00999497549645613\\
599.63	0.00999522718407935\\
599.64	0.009995473183637\\
599.65	0.00999571343614678\\
599.66	0.00999594788204004\\
599.67	0.00999617646115599\\
599.68	0.0099963991127358\\
599.69	0.00999661577541675\\
599.7	0.00999682638722618\\
599.71	0.00999703088557551\\
599.72	0.00999722920725411\\
599.73	0.00999742128842315\\
599.74	0.00999760706460942\\
599.75	0.00999778647069897\\
599.76	0.00999795944093086\\
599.77	0.0099981259088907\\
599.78	0.0099982858075042\\
599.79	0.00999843906903064\\
599.8	0.00999858562505627\\
599.81	0.00999872540648767\\
599.82	0.00999885834354498\\
599.83	0.00999898436575515\\
599.84	0.00999910340194508\\
599.85	0.00999921538023465\\
599.86	0.00999932022802977\\
599.87	0.00999941787201528\\
599.88	0.00999950823814785\\
599.89	0.00999959125164875\\
599.9	0.00999966683699656\\
599.91	0.00999973491791987\\
599.92	0.0099997954173898\\
599.93	0.00999984825761255\\
599.94	0.00999989336002181\\
599.95	0.00999993064527112\\
599.96	0.00999996003322615\\
599.97	0.00999998144295691\\
599.98	0.00999999479272987\\
599.99	0.01\\
600	0.01\\
};
\addplot [color=blue!75!mycolor7,solid,forget plot]
  table[row sep=crcr]{%
0.01	0.00270247467503103\\
1.01	0.00270247533429627\\
2.01	0.00270247600752455\\
3.01	0.00270247669501311\\
4.01	0.00270247739706596\\
5.01	0.00270247811399338\\
6.01	0.00270247884611237\\
7.01	0.00270247959374664\\
8.01	0.00270248035722696\\
9.01	0.00270248113689134\\
10.01	0.00270248193308464\\
11.01	0.00270248274615945\\
12.01	0.00270248357647591\\
13.01	0.00270248442440169\\
14.01	0.00270248529031265\\
15.01	0.0027024861745924\\
16.01	0.00270248707763303\\
17.01	0.00270248799983511\\
18.01	0.00270248894160763\\
19.01	0.00270248990336849\\
20.01	0.00270249088554445\\
21.01	0.00270249188857176\\
22.01	0.00270249291289565\\
23.01	0.00270249395897115\\
24.01	0.00270249502726301\\
25.01	0.0027024961182462\\
26.01	0.00270249723240559\\
27.01	0.00270249837023682\\
28.01	0.00270249953224588\\
29.01	0.00270250071895002\\
30.01	0.00270250193087732\\
31.01	0.00270250316856732\\
32.01	0.0027025044325714\\
33.01	0.00270250572345257\\
34.01	0.0027025070417861\\
35.01	0.00270250838815958\\
36.01	0.00270250976317325\\
37.01	0.00270251116744064\\
38.01	0.00270251260158821\\
39.01	0.00270251406625591\\
40.01	0.00270251556209768\\
41.01	0.00270251708978153\\
42.01	0.00270251864998984\\
43.01	0.0027025202434199\\
44.01	0.00270252187078381\\
45.01	0.00270252353280925\\
46.01	0.00270252523023966\\
47.01	0.00270252696383441\\
48.01	0.00270252873436937\\
49.01	0.00270253054263721\\
50.01	0.00270253238944777\\
51.01	0.00270253427562822\\
52.01	0.00270253620202393\\
53.01	0.00270253816949839\\
54.01	0.00270254017893395\\
55.01	0.00270254223123183\\
56.01	0.00270254432731296\\
57.01	0.00270254646811821\\
58.01	0.00270254865460874\\
59.01	0.00270255088776668\\
60.01	0.00270255316859545\\
61.01	0.00270255549812008\\
62.01	0.00270255787738798\\
63.01	0.00270256030746932\\
64.01	0.00270256278945737\\
65.01	0.00270256532446929\\
66.01	0.00270256791364634\\
67.01	0.00270257055815473\\
68.01	0.00270257325918585\\
69.01	0.00270257601795711\\
70.01	0.00270257883571216\\
71.01	0.00270258171372205\\
72.01	0.00270258465328506\\
73.01	0.00270258765572813\\
74.01	0.00270259072240689\\
75.01	0.00270259385470629\\
76.01	0.00270259705404162\\
77.01	0.00270260032185902\\
78.01	0.00270260365963614\\
79.01	0.00270260706888252\\
80.01	0.00270261055114097\\
81.01	0.00270261410798775\\
82.01	0.00270261774103349\\
83.01	0.00270262145192399\\
84.01	0.00270262524234094\\
85.01	0.00270262911400267\\
86.01	0.00270263306866504\\
87.01	0.00270263710812213\\
88.01	0.00270264123420753\\
89.01	0.00270264544879445\\
90.01	0.00270264975379733\\
91.01	0.0027026541511723\\
92.01	0.00270265864291827\\
93.01	0.0027026632310779\\
94.01	0.00270266791773846\\
95.01	0.00270267270503299\\
96.01	0.00270267759514119\\
97.01	0.00270268259029019\\
98.01	0.00270268769275631\\
99.01	0.00270269290486544\\
100.01	0.00270269822899444\\
101.01	0.00270270366757241\\
102.01	0.00270270922308156\\
103.01	0.00270271489805848\\
104.01	0.00270272069509551\\
105.01	0.00270272661684193\\
106.01	0.00270273266600483\\
107.01	0.00270273884535099\\
108.01	0.00270274515770798\\
109.01	0.00270275160596534\\
110.01	0.00270275819307605\\
111.01	0.00270276492205816\\
112.01	0.00270277179599596\\
113.01	0.00270277881804147\\
114.01	0.00270278599141625\\
115.01	0.00270279331941242\\
116.01	0.00270280080539494\\
117.01	0.00270280845280267\\
118.01	0.00270281626514995\\
119.01	0.00270282424602916\\
120.01	0.00270283239911125\\
121.01	0.00270284072814832\\
122.01	0.00270284923697524\\
123.01	0.00270285792951139\\
124.01	0.00270286680976269\\
125.01	0.00270287588182341\\
126.01	0.00270288514987828\\
127.01	0.00270289461820446\\
128.01	0.00270290429117342\\
129.01	0.00270291417325338\\
130.01	0.00270292426901132\\
131.01	0.00270293458311537\\
132.01	0.00270294512033654\\
133.01	0.00270295588555169\\
134.01	0.0027029668837457\\
135.01	0.00270297812001373\\
136.01	0.0027029895995636\\
137.01	0.00270300132771877\\
138.01	0.00270301330992052\\
139.01	0.0027030255517311\\
140.01	0.00270303805883587\\
141.01	0.00270305083704628\\
142.01	0.00270306389230295\\
143.01	0.00270307723067827\\
144.01	0.00270309085837955\\
145.01	0.00270310478175213\\
146.01	0.00270311900728221\\
147.01	0.00270313354160019\\
148.01	0.00270314839148419\\
149.01	0.0027031635638628\\
150.01	0.00270317906581906\\
151.01	0.00270319490459373\\
152.01	0.00270321108758879\\
153.01	0.00270322762237106\\
154.01	0.00270324451667618\\
155.01	0.00270326177841231\\
156.01	0.0027032794156638\\
157.01	0.00270329743669538\\
158.01	0.00270331584995673\\
159.01	0.00270333466408561\\
160.01	0.00270335388791323\\
161.01	0.00270337353046794\\
162.01	0.00270339360097988\\
163.01	0.00270341410888571\\
164.01	0.00270343506383317\\
165.01	0.00270345647568597\\
166.01	0.00270347835452865\\
167.01	0.00270350071067162\\
168.01	0.00270352355465635\\
169.01	0.00270354689726052\\
170.01	0.00270357074950371\\
171.01	0.00270359512265228\\
172.01	0.00270362002822581\\
173.01	0.00270364547800236\\
174.01	0.00270367148402458\\
175.01	0.00270369805860558\\
176.01	0.00270372521433532\\
177.01	0.00270375296408672\\
178.01	0.00270378132102239\\
179.01	0.0027038102986011\\
180.01	0.00270383991058468\\
181.01	0.00270387017104476\\
182.01	0.00270390109437012\\
183.01	0.00270393269527364\\
184.01	0.00270396498880028\\
185.01	0.00270399799033418\\
186.01	0.0027040317156065\\
187.01	0.00270406618070387\\
188.01	0.00270410140207589\\
189.01	0.00270413739654395\\
190.01	0.00270417418130976\\
191.01	0.00270421177396382\\
192.01	0.00270425019249451\\
193.01	0.00270428945529744\\
194.01	0.00270432958118458\\
195.01	0.00270437058939391\\
196.01	0.00270441249959952\\
197.01	0.00270445533192122\\
198.01	0.0027044991069352\\
199.01	0.00270454384568428\\
200.01	0.00270458956968893\\
201.01	0.00270463630095816\\
202.01	0.0027046840620009\\
203.01	0.00270473287583753\\
204.01	0.00270478276601178\\
205.01	0.00270483375660273\\
206.01	0.00270488587223702\\
207.01	0.00270493913810216\\
208.01	0.00270499357995892\\
209.01	0.00270504922415492\\
210.01	0.0027051060976382\\
211.01	0.00270516422797092\\
212.01	0.00270522364334395\\
213.01	0.00270528437259129\\
214.01	0.002705346445205\\
215.01	0.00270540989135065\\
216.01	0.00270547474188272\\
217.01	0.00270554102836068\\
218.01	0.00270560878306568\\
219.01	0.00270567803901699\\
220.01	0.00270574882998924\\
221.01	0.00270582119053019\\
222.01	0.00270589515597848\\
223.01	0.00270597076248241\\
224.01	0.00270604804701824\\
225.01	0.00270612704741006\\
226.01	0.00270620780234948\\
227.01	0.00270629035141534\\
228.01	0.00270637473509498\\
229.01	0.0027064609948053\\
230.01	0.00270654917291427\\
231.01	0.00270663931276335\\
232.01	0.00270673145869008\\
233.01	0.0027068256560514\\
234.01	0.00270692195124765\\
235.01	0.00270702039174651\\
236.01	0.00270712102610837\\
237.01	0.00270722390401164\\
238.01	0.00270732907627895\\
239.01	0.00270743659490411\\
240.01	0.0027075465130791\\
241.01	0.00270765888522241\\
242.01	0.00270777376700783\\
243.01	0.00270789121539326\\
244.01	0.00270801128865152\\
245.01	0.00270813404640037\\
246.01	0.00270825954963483\\
247.01	0.00270838786075876\\
248.01	0.00270851904361798\\
249.01	0.00270865316353411\\
250.01	0.00270879028733922\\
251.01	0.00270893048341113\\
252.01	0.00270907382170958\\
253.01	0.00270922037381315\\
254.01	0.00270937021295713\\
255.01	0.00270952341407281\\
256.01	0.00270968005382667\\
257.01	0.00270984021066105\\
258.01	0.00271000396483605\\
259.01	0.00271017139847174\\
260.01	0.00271034259559182\\
261.01	0.00271051764216831\\
262.01	0.00271069662616669\\
263.01	0.00271087963759313\\
264.01	0.00271106676854173\\
265.01	0.00271125811324362\\
266.01	0.00271145376811681\\
267.01	0.0027116538318173\\
268.01	0.00271185840529134\\
269.01	0.00271206759182897\\
270.01	0.00271228149711895\\
271.01	0.00271250022930422\\
272.01	0.00271272389903977\\
273.01	0.0027129526195508\\
274.01	0.00271318650669294\\
275.01	0.00271342567901324\\
276.01	0.00271367025781309\\
277.01	0.00271392036721234\\
278.01	0.00271417613421482\\
279.01	0.00271443768877547\\
280.01	0.00271470516386876\\
281.01	0.00271497869555914\\
282.01	0.0027152584230726\\
283.01	0.00271554448886996\\
284.01	0.00271583703872232\\
285.01	0.00271613622178692\\
286.01	0.00271644219068644\\
287.01	0.00271675510158866\\
288.01	0.00271707511428881\\
289.01	0.00271740239229304\\
290.01	0.00271773710290442\\
291.01	0.00271807941731017\\
292.01	0.00271842951067162\\
293.01	0.00271878756221533\\
294.01	0.00271915375532732\\
295.01	0.0027195282776481\\
296.01	0.00271991132117057\\
297.01	0.00272030308234024\\
298.01	0.00272070376215712\\
299.01	0.00272111356628033\\
300.01	0.00272153270513456\\
301.01	0.00272196139401958\\
302.01	0.00272239985322109\\
303.01	0.00272284830812539\\
304.01	0.00272330698933557\\
305.01	0.00272377613279026\\
306.01	0.00272425597988599\\
307.01	0.00272474677760093\\
308.01	0.00272524877862255\\
309.01	0.00272576224147726\\
310.01	0.00272628743066379\\
311.01	0.00272682461678848\\
312.01	0.00272737407670492\\
313.01	0.00272793609365547\\
314.01	0.00272851095741694\\
315.01	0.0027290989644492\\
316.01	0.00272970041804706\\
317.01	0.00273031562849621\\
318.01	0.00273094491323209\\
319.01	0.00273158859700301\\
320.01	0.00273224701203695\\
321.01	0.0027329204982125\\
322.01	0.00273360940323363\\
323.01	0.00273431408280884\\
324.01	0.00273503490083495\\
325.01	0.00273577222958521\\
326.01	0.00273652644990233\\
327.01	0.00273729795139636\\
328.01	0.00273808713264778\\
329.01	0.00273889440141571\\
330.01	0.0027397201748519\\
331.01	0.00274056487972089\\
332.01	0.00274142895262519\\
333.01	0.00274231284023753\\
334.01	0.00274321699953993\\
335.01	0.00274414189806882\\
336.01	0.00274508801416797\\
337.01	0.00274605583724873\\
338.01	0.00274704586805798\\
339.01	0.00274805861895508\\
340.01	0.00274909461419602\\
341.01	0.00275015439022826\\
342.01	0.00275123849599369\\
343.01	0.00275234749324232\\
344.01	0.00275348195685582\\
345.01	0.00275464247518248\\
346.01	0.00275582965038318\\
347.01	0.00275704409878906\\
348.01	0.00275828645127285\\
349.01	0.00275955735363188\\
350.01	0.00276085746698575\\
351.01	0.00276218746818835\\
352.01	0.00276354805025458\\
353.01	0.00276493992280309\\
354.01	0.00276636381251567\\
355.01	0.00276782046361405\\
356.01	0.00276931063835421\\
357.01	0.00277083511754045\\
358.01	0.00277239470105849\\
359.01	0.00277399020842996\\
360.01	0.0027756224793874\\
361.01	0.00277729237447215\\
362.01	0.00277900077565511\\
363.01	0.00278074858698127\\
364.01	0.0027825367352384\\
365.01	0.00278436617065053\\
366.01	0.00278623786759656\\
367.01	0.00278815282535375\\
368.01	0.00279011206886554\\
369.01	0.00279211664953301\\
370.01	0.00279416764602901\\
371.01	0.002796266165132\\
372.01	0.00279841334257751\\
373.01	0.00280061034392369\\
374.01	0.00280285836542661\\
375.01	0.00280515863492006\\
376.01	0.00280751241269528\\
377.01	0.00280992099237441\\
378.01	0.00281238570177332\\
379.01	0.00281490790375073\\
380.01	0.00281748899704232\\
381.01	0.00282013041708387\\
382.01	0.00282283363683311\\
383.01	0.00282560016760509\\
384.01	0.00282843155994579\\
385.01	0.00283132940456879\\
386.01	0.00283429533337751\\
387.01	0.00283733102056727\\
388.01	0.00284043818377587\\
389.01	0.00284361858526741\\
390.01	0.00284687403314821\\
391.01	0.00285020638261679\\
392.01	0.00285361753724704\\
393.01	0.00285710945030708\\
394.01	0.00286068412611297\\
395.01	0.00286434362142014\\
396.01	0.00286809004685051\\
397.01	0.00287192556835872\\
398.01	0.00287585240873706\\
399.01	0.00287987284915896\\
400.01	0.00288398923076455\\
401.01	0.0028882039562859\\
402.01	0.00289251949171476\\
403.01	0.00289693836801229\\
404.01	0.00290146318286257\\
405.01	0.00290609660246954\\
406.01	0.00291084136339893\\
407.01	0.00291570027446451\\
408.01	0.00292067621866084\\
409.01	0.00292577215514149\\
410.01	0.00293099112124419\\
411.01	0.00293633623456342\\
412.01	0.0029418106950693\\
413.01	0.00294741778727514\\
414.01	0.00295316088245261\\
415.01	0.00295904344089451\\
416.01	0.0029650690142259\\
417.01	0.00297124124776349\\
418.01	0.00297756388292232\\
419.01	0.0029840407596702\\
420.01	0.00299067581902952\\
421.01	0.00299747310562577\\
422.01	0.0030044367702815\\
423.01	0.00301157107265605\\
424.01	0.00301888038392892\\
425.01	0.00302636918952615\\
426.01	0.00303404209188808\\
427.01	0.00304190381327682\\
428.01	0.00304995919862031\\
429.01	0.00305821321839271\\
430.01	0.00306667097152531\\
431.01	0.00307533768834659\\
432.01	0.0030842187335459\\
433.01	0.00309331960915568\\
434.01	0.00310264595754579\\
435.01	0.00311220356442228\\
436.01	0.00312199836182031\\
437.01	0.00313203643107907\\
438.01	0.00314232400578267\\
439.01	0.00315286747464798\\
440.01	0.00316367338433383\\
441.01	0.00317474844213898\\
442.01	0.00318609951854902\\
443.01	0.00319773364957773\\
444.01	0.00320965803883519\\
445.01	0.00322188005923436\\
446.01	0.00323440725422265\\
447.01	0.00324724733839032\\
448.01	0.00326040819726636\\
449.01	0.0032738978860566\\
450.01	0.00328772462700695\\
451.01	0.00330189680498473\\
452.01	0.00331642296075368\\
453.01	0.00333131178126965\\
454.01	0.00334657208613542\\
455.01	0.00336221280911142\\
456.01	0.00337824297328074\\
457.01	0.00339467165808292\\
458.01	0.00341150795596004\\
459.01	0.0034287609157695\\
460.01	0.00344643946939566\\
461.01	0.00346455233711562\\
462.01	0.00348310790622741\\
463.01	0.00350211407624967\\
464.01	0.00352157806272528\\
465.01	0.00354150615061395\\
466.01	0.00356190338840462\\
467.01	0.00358277321850817\\
468.01	0.00360411705825751\\
469.01	0.00362593384889105\\
470.01	0.0036482193487216\\
471.01	0.00367096518644888\\
472.01	0.00369415773920015\\
473.01	0.00371777671173217\\
474.01	0.00374179332594245\\
475.01	0.00376616800608949\\
476.01	0.0037908474080284\\
477.01	0.00381576058788272\\
478.01	0.0038408140226982\\
479.01	0.00386588508688511\\
480.01	0.00389081376234303\\
481.01	0.00391560395586256\\
482.01	0.00394074112993927\\
483.01	0.00396620506208416\\
484.01	0.00399191528349777\\
485.01	0.00401776461077528\\
486.01	0.00404361142247187\\
487.01	0.00406926963312608\\
488.01	0.00409449564862648\\
489.01	0.00411897135558789\\
490.01	0.00414228188906971\\
491.01	0.00416507254868226\\
492.01	0.00418846898558721\\
493.01	0.00421248742566058\\
494.01	0.00423714381453705\\
495.01	0.00426245500524932\\
496.01	0.00428843889910915\\
497.01	0.00431511453216295\\
498.01	0.00434250203415325\\
499.01	0.00437062228851151\\
500.01	0.00439949642593\\
501.01	0.0044291454734143\\
502.01	0.00445958989481118\\
503.01	0.00449084888929291\\
504.01	0.00452293933632795\\
505.01	0.00455587427112269\\
506.01	0.00458966202379704\\
507.01	0.00462430698135259\\
508.01	0.00465980928779375\\
509.01	0.00469616417384544\\
510.01	0.0047333613215893\\
511.01	0.00477138434497626\\
512.01	0.0048102105109371\\
513.01	0.00484981089078331\\
514.01	0.00489015122506367\\
515.01	0.00493119391866482\\
516.01	0.00497290177298885\\
517.01	0.00501524473887385\\
518.01	0.00505821584207151\\
519.01	0.00510183294455671\\
520.01	0.00514613671083838\\
521.01	0.00519120211992206\\
522.01	0.00523715530414039\\
523.01	0.00528419723132036\\
524.01	0.0053326260424256\\
525.01	0.00538267931871982\\
526.01	0.00543447963477701\\
527.01	0.00548809594440868\\
528.01	0.00554356281551956\\
529.01	0.00560087933159319\\
530.01	0.0056599898495372\\
531.01	0.00572075662910118\\
532.01	0.00578268190428666\\
533.01	0.00584498022083387\\
534.01	0.00590749574688542\\
535.01	0.00597007948607768\\
536.01	0.00603256371658051\\
537.01	0.00609476253691515\\
538.01	0.00615647411242743\\
539.01	0.00621748539801371\\
540.01	0.00627758036634497\\
541.01	0.00633655347722414\\
542.01	0.0063942310381018\\
543.01	0.00645050344564633\\
544.01	0.00650549634327943\\
545.01	0.00656008411933888\\
546.01	0.00661435765076117\\
547.01	0.00666825438150137\\
548.01	0.00672172896272959\\
549.01	0.00677475834979131\\
550.01	0.00682734663152718\\
551.01	0.0068795317334653\\
552.01	0.00693139203259288\\
553.01	0.00698304779253108\\
554.01	0.00703465559071412\\
555.01	0.00708638066172824\\
556.01	0.00713829638688321\\
557.01	0.00719043503387839\\
558.01	0.00724283853598048\\
559.01	0.00729555771687146\\
560.01	0.00734865055334202\\
561.01	0.00740218005541401\\
562.01	0.0074562106142838\\
563.01	0.00751080248597193\\
564.01	0.00756600583168564\\
565.01	0.00762185735902811\\
566.01	0.00767838795473285\\
567.01	0.00773562904201511\\
568.01	0.00779361189922099\\
569.01	0.00785236643307945\\
570.01	0.00791191989642814\\
571.01	0.00797229587071889\\
572.01	0.00803351381482855\\
573.01	0.00809558940754691\\
574.01	0.00815853564761196\\
575.01	0.00822236373310811\\
576.01	0.00828708286240763\\
577.01	0.00835269977667169\\
578.01	0.00841921839411719\\
579.01	0.00848663956920399\\
580.01	0.00855496097497914\\
581.01	0.00862417706125201\\
582.01	0.00869427898579362\\
583.01	0.00876525439463777\\
584.01	0.00883708706023637\\
585.01	0.00890975649722161\\
586.01	0.00898323760124607\\
587.01	0.00905750032875147\\
588.01	0.00913250943662461\\
589.01	0.00920822430118101\\
590.01	0.00928459884902855\\
591.01	0.00936158165898825\\
592.01	0.00943911632123402\\
593.01	0.00951714214254039\\
594.01	0.00959559528261798\\
595.01	0.00967441041276702\\
596.01	0.00975352299930713\\
597.01	0.00983287233211095\\
598.01	0.00990866201774166\\
599.01	0.0099708728041569\\
599.02	0.00997138072163179\\
599.03	0.0099718855762529\\
599.04	0.00997238733819737\\
599.05	0.0099728859773483\\
599.06	0.00997338146329193\\
599.07	0.00997387376531463\\
599.08	0.00997436285239995\\
599.09	0.00997484869322563\\
599.1	0.00997533125616056\\
599.11	0.00997581050926173\\
599.12	0.00997628642027111\\
599.13	0.00997675895661256\\
599.14	0.00997722808538867\\
599.15	0.00997769377337755\\
599.16	0.00997815598702968\\
599.17	0.00997861469246456\\
599.18	0.00997906985546755\\
599.19	0.00997952144148644\\
599.2	0.00997996941562822\\
599.21	0.0099804137426556\\
599.22	0.00998085438698366\\
599.23	0.0099812913126764\\
599.24	0.00998172448344324\\
599.25	0.0099821538626355\\
599.26	0.00998257941258275\\
599.27	0.00998300109279754\\
599.28	0.00998341886238916\\
599.29	0.00998383268005965\\
599.3	0.00998424250409978\\
599.31	0.00998464829238495\\
599.32	0.00998505000237108\\
599.33	0.00998544759109047\\
599.34	0.00998584101514764\\
599.35	0.009986230230715\\
599.36	0.00998661519352867\\
599.37	0.0099869958588841\\
599.38	0.00998737218163174\\
599.39	0.00998774411617261\\
599.4	0.00998811161645385\\
599.41	0.00998847463596427\\
599.42	0.00998883312772976\\
599.43	0.00998918704430873\\
599.44	0.00998953633778746\\
599.45	0.00998988095977548\\
599.46	0.0099902208614008\\
599.47	0.00999055599330515\\
599.48	0.00999088630563919\\
599.49	0.00999121174805762\\
599.5	0.0099915322697143\\
599.51	0.00999184781925726\\
599.52	0.00999215834482371\\
599.53	0.009992463794035\\
599.54	0.00999276411399148\\
599.55	0.00999305925126736\\
599.56	0.00999334915190551\\
599.57	0.00999363376141218\\
599.58	0.00999391302475172\\
599.59	0.00999418688634117\\
599.6	0.00999445529004488\\
599.61	0.00999471817916904\\
599.62	0.00999497549645612\\
599.63	0.00999522718407934\\
599.64	0.00999547318363699\\
599.65	0.00999571343614677\\
599.66	0.00999594788204004\\
599.67	0.00999617646115598\\
599.68	0.0099963991127358\\
599.69	0.00999661577541675\\
599.7	0.00999682638722618\\
599.71	0.00999703088557551\\
599.72	0.00999722920725411\\
599.73	0.00999742128842315\\
599.74	0.00999760706460942\\
599.75	0.00999778647069897\\
599.76	0.00999795944093086\\
599.77	0.0099981259088907\\
599.78	0.0099982858075042\\
599.79	0.00999843906903064\\
599.8	0.00999858562505627\\
599.81	0.00999872540648767\\
599.82	0.00999885834354498\\
599.83	0.00999898436575515\\
599.84	0.00999910340194508\\
599.85	0.00999921538023465\\
599.86	0.00999932022802977\\
599.87	0.00999941787201528\\
599.88	0.00999950823814785\\
599.89	0.00999959125164875\\
599.9	0.00999966683699656\\
599.91	0.00999973491791987\\
599.92	0.0099997954173898\\
599.93	0.00999984825761255\\
599.94	0.00999989336002181\\
599.95	0.00999993064527112\\
599.96	0.00999996003322615\\
599.97	0.00999998144295691\\
599.98	0.00999999479272987\\
599.99	0.01\\
600	0.01\\
};
\addplot [color=blue!80!mycolor9,solid,forget plot]
  table[row sep=crcr]{%
0.01	0.000977852700539981\\
1.01	0.000977853634422134\\
2.01	0.000977854588141196\\
3.01	0.000977855562121314\\
4.01	0.000977856556795567\\
5.01	0.000977857572606509\\
6.01	0.000977858610006203\\
7.01	0.000977859669456425\\
8.01	0.000977860751428916\\
9.01	0.000977861856405533\\
10.01	0.000977862984878675\\
11.01	0.000977864137351257\\
12.01	0.000977865314337065\\
13.01	0.000977866516360955\\
14.01	0.000977867743959124\\
15.01	0.000977868997679321\\
16.01	0.000977870278081203\\
17.01	0.00097787158573636\\
18.01	0.000977872921228896\\
19.01	0.00097787428515541\\
20.01	0.000977875678125496\\
21.01	0.000977877100761824\\
22.01	0.000977878553700559\\
23.01	0.000977880037591639\\
24.01	0.000977881553099115\\
25.01	0.000977883100901272\\
26.01	0.000977884681691195\\
27.01	0.000977886296176872\\
28.01	0.000977887945081642\\
29.01	0.000977889629144538\\
30.01	0.000977891349120574\\
31.01	0.000977893105781148\\
32.01	0.000977894899914282\\
33.01	0.000977896732325137\\
34.01	0.000977898603836395\\
35.01	0.000977900515288496\\
36.01	0.000977902467540072\\
37.01	0.000977904461468414\\
38.01	0.00097790649796984\\
39.01	0.000977908577960193\\
40.01	0.000977910702375052\\
41.01	0.000977912872170437\\
42.01	0.00097791508832306\\
43.01	0.00097791735183083\\
44.01	0.000977919663713349\\
45.01	0.000977922025012408\\
46.01	0.000977924436792413\\
47.01	0.000977926900140917\\
48.01	0.000977929416169157\\
49.01	0.000977931986012488\\
50.01	0.000977934610830962\\
51.01	0.000977937291809956\\
52.01	0.000977940030160482\\
53.01	0.000977942827120125\\
54.01	0.000977945683953263\\
55.01	0.000977948601951856\\
56.01	0.000977951582436049\\
57.01	0.000977954626754755\\
58.01	0.00097795773628627\\
59.01	0.000977960912438923\\
60.01	0.000977964156651745\\
61.01	0.000977967470395219\\
62.01	0.000977970855171879\\
63.01	0.000977974312517095\\
64.01	0.000977977843999757\\
65.01	0.000977981451222902\\
66.01	0.000977985135824741\\
67.01	0.000977988899479122\\
68.01	0.000977992743896573\\
69.01	0.000977996670825036\\
70.01	0.000978000682050673\\
71.01	0.000978004779398585\\
72.01	0.000978008964733969\\
73.01	0.000978013239962737\\
74.01	0.000978017607032526\\
75.01	0.000978022067933695\\
76.01	0.000978026624700125\\
77.01	0.000978031279410277\\
78.01	0.00097803603418814\\
79.01	0.000978040891204311\\
80.01	0.000978045852676905\\
81.01	0.000978050920872761\\
82.01	0.000978056098108464\\
83.01	0.000978061386751328\\
84.01	0.000978066789220709\\
85.01	0.00097807230798911\\
86.01	0.000978077945583244\\
87.01	0.000978083704585513\\
88.01	0.000978089587634796\\
89.01	0.000978095597428223\\
90.01	0.000978101736722178\\
91.01	0.000978108008333575\\
92.01	0.000978114415141458\\
93.01	0.000978120960088147\\
94.01	0.000978127646180729\\
95.01	0.000978134476492525\\
96.01	0.000978141454164527\\
97.01	0.000978148582407093\\
98.01	0.000978155864501125\\
99.01	0.000978163303799975\\
100.01	0.000978170903731014\\
101.01	0.000978178667797105\\
102.01	0.000978186599578532\\
103.01	0.000978194702734539\\
104.01	0.000978202981005198\\
105.01	0.000978211438213146\\
106.01	0.000978220078265618\\
107.01	0.000978228905156135\\
108.01	0.000978237922966425\\
109.01	0.000978247135868679\\
110.01	0.000978256548127249\\
111.01	0.000978266164100912\\
112.01	0.000978275988244924\\
113.01	0.000978286025113302\\
114.01	0.000978296279360795\\
115.01	0.000978306755745513\\
116.01	0.000978317459130768\\
117.01	0.000978328394487936\\
118.01	0.00097833956689862\\
119.01	0.00097835098155704\\
120.01	0.000978362643772886\\
121.01	0.00097837455897367\\
122.01	0.000978386732707413\\
123.01	0.000978399170645523\\
124.01	0.00097841187858537\\
125.01	0.000978424862453239\\
126.01	0.000978438128307189\\
127.01	0.000978451682340092\\
128.01	0.000978465530882731\\
129.01	0.000978479680406708\\
130.01	0.000978494137527749\\
131.01	0.000978508909009045\\
132.01	0.000978524001764527\\
133.01	0.000978539422862211\\
134.01	0.000978555179527725\\
135.01	0.000978571279147918\\
136.01	0.000978587729274553\\
137.01	0.000978604537627846\\
138.01	0.0009786217121006\\
139.01	0.00097863926076177\\
140.01	0.000978657191860723\\
141.01	0.000978675513831254\\
142.01	0.00097869423529575\\
143.01	0.000978713365069467\\
144.01	0.000978732912164923\\
145.01	0.000978752885796436\\
146.01	0.000978773295384684\\
147.01	0.000978794150561374\\
148.01	0.000978815461174058\\
149.01	0.0009788372372911\\
150.01	0.00097885948920659\\
151.01	0.000978882227445584\\
152.01	0.000978905462769339\\
153.01	0.000978929206180701\\
154.01	0.000978953468929587\\
155.01	0.000978978262518707\\
156.01	0.000979003598709241\\
157.01	0.000979029489526842\\
158.01	0.000979055947267465\\
159.01	0.000979082984503894\\
160.01	0.000979110614091706\\
161.01	0.000979138849175894\\
162.01	0.000979167703197628\\
163.01	0.00097919718990077\\
164.01	0.000979227323338959\\
165.01	0.000979258117882749\\
166.01	0.00097928958822667\\
167.01	0.000979321749396848\\
168.01	0.000979354616758486\\
169.01	0.000979388206023696\\
170.01	0.000979422533259416\\
171.01	0.000979457614895658\\
172.01	0.000979493467733579\\
173.01	0.000979530108954289\\
174.01	0.00097956755612745\\
175.01	0.000979605827220183\\
176.01	0.000979644940606206\\
177.01	0.000979684915075218\\
178.01	0.000979725769842385\\
179.01	0.000979767524558271\\
180.01	0.000979810199318601\\
181.01	0.000979853814674782\\
182.01	0.000979898391644148\\
183.01	0.000979943951720866\\
184.01	0.000979990516886818\\
185.01	0.000980038109622841\\
186.01	0.000980086752920183\\
187.01	0.000980136470292346\\
188.01	0.000980187285787068\\
189.01	0.00098023922399873\\
190.01	0.000980292310080719\\
191.01	0.000980346569758483\\
192.01	0.000980402029342894\\
193.01	0.000980458715743308\\
194.01	0.000980516656481835\\
195.01	0.000980575879707124\\
196.01	0.000980636414208966\\
197.01	0.000980698289433195\\
198.01	0.000980761535496581\\
199.01	0.000980826183202589\\
200.01	0.000980892264057037\\
201.01	0.000980959810284399\\
202.01	0.000981028854844412\\
203.01	0.00098109943144898\\
204.01	0.000981171574579554\\
205.01	0.000981245319505095\\
206.01	0.000981320702300066\\
207.01	0.00098139775986307\\
208.01	0.000981476529935958\\
209.01	0.000981557051123218\\
210.01	0.000981639362912061\\
211.01	0.000981723505692672\\
212.01	0.000981809520779184\\
213.01	0.000981897450430896\\
214.01	0.00098198733787426\\
215.01	0.000982079227325149\\
216.01	0.000982173164011738\\
217.01	0.000982269194197881\\
218.01	0.000982367365207112\\
219.01	0.000982467725447043\\
220.01	0.00098257032443437\\
221.01	0.000982675212820631\\
222.01	0.000982782442418388\\
223.01	0.00098289206622787\\
224.01	0.000983004138464708\\
225.01	0.000983118714587679\\
226.01	0.000983235851327688\\
227.01	0.000983355606716904\\
228.01	0.000983478040118967\\
229.01	0.000983603212259584\\
230.01	0.000983731185257984\\
231.01	0.000983862022659251\\
232.01	0.000983995789466889\\
233.01	0.000984132552176664\\
234.01	0.000984272378810921\\
235.01	0.000984415338953833\\
236.01	0.000984561503787354\\
237.01	0.000984710946128009\\
238.01	0.000984863740464532\\
239.01	0.000985019962996402\\
240.01	0.000985179691673145\\
241.01	0.000985343006234672\\
242.01	0.00098550998825242\\
243.01	0.000985680721171445\\
244.01	0.000985855290353568\\
245.01	0.000986033783121336\\
246.01	0.000986216288803014\\
247.01	0.000986402898778722\\
248.01	0.000986593706527553\\
249.01	0.000986788807675575\\
250.01	0.000986988300045274\\
251.01	0.000987192283705654\\
252.01	0.000987400861023835\\
253.01	0.000987614136717815\\
254.01	0.00098783221791017\\
255.01	0.000988055214183013\\
256.01	0.000988283237634266\\
257.01	0.000988516402935354\\
258.01	0.000988754827389832\\
259.01	0.000988998630993609\\
260.01	0.000989247936496365\\
261.01	0.000989502869464469\\
262.01	0.000989763558345291\\
263.01	0.000990030134532631\\
264.01	0.000990302732434202\\
265.01	0.00099058148954017\\
266.01	0.000990866546493385\\
267.01	0.000991158047161086\\
268.01	0.000991456138708338\\
269.01	0.000991760971673181\\
270.01	0.000992072700043062\\
271.01	0.000992391481333565\\
272.01	0.000992717476668362\\
273.01	0.000993050850861391\\
274.01	0.000993391772500428\\
275.01	0.000993740414033099\\
276.01	0.000994096951854213\\
277.01	0.000994461566395627\\
278.01	0.000994834442217797\\
279.01	0.00099521576810329\\
280.01	0.000995605737152967\\
281.01	0.000996004546883714\\
282.01	0.000996412399328725\\
283.01	0.000996829501140058\\
284.01	0.000997256063693289\\
285.01	0.00099769230319513\\
286.01	0.000998138440792759\\
287.01	0.00099859470268631\\
288.01	0.000999061320243591\\
289.01	0.000999538530117548\\
290.01	0.00100002657436661\\
291.01	0.00100052570057756\\
292.01	0.00100103616199158\\
293.01	0.00100155821763315\\
294.01	0.00100209213244173\\
295.01	0.00100263817740722\\
296.01	0.00100319662970799\\
297.01	0.00100376777285256\\
298.01	0.00100435189682466\\
299.01	0.00100494929823174\\
300.01	0.00100556028045719\\
301.01	0.00100618515381621\\
302.01	0.00100682423571566\\
303.01	0.00100747785081769\\
304.01	0.00100814633120754\\
305.01	0.00100883001656578\\
306.01	0.00100952925434449\\
307.01	0.00101024439994826\\
308.01	0.00101097581691967\\
309.01	0.00101172387712957\\
310.01	0.00101248896097245\\
311.01	0.00101327145756668\\
312.01	0.00101407176496008\\
313.01	0.0010148902903412\\
314.01	0.00101572745025582\\
315.01	0.00101658367082933\\
316.01	0.00101745938799547\\
317.01	0.00101835504773071\\
318.01	0.00101927110629547\\
319.01	0.00102020803048173\\
320.01	0.00102116629786739\\
321.01	0.00102214639707782\\
322.01	0.00102314882805461\\
323.01	0.00102417410233171\\
324.01	0.00102522274331941\\
325.01	0.00102629528659612\\
326.01	0.00102739228020864\\
327.01	0.0010285142849805\\
328.01	0.00102966187482923\\
329.01	0.00103083563709257\\
330.01	0.00103203617286379\\
331.01	0.00103326409733641\\
332.01	0.001034520040159\\
333.01	0.00103580464579946\\
334.01	0.00103711857391993\\
335.01	0.00103846249976194\\
336.01	0.00103983711454256\\
337.01	0.00104124312586123\\
338.01	0.00104268125811837\\
339.01	0.00104415225294487\\
340.01	0.00104565686964395\\
341.01	0.00104719588564468\\
342.01	0.00104877009696788\\
343.01	0.00105038031870457\\
344.01	0.00105202738550707\\
345.01	0.00105371215209317\\
346.01	0.00105543549376341\\
347.01	0.00105719830693203\\
348.01	0.00105900150967126\\
349.01	0.00106084604226982\\
350.01	0.00106273286780535\\
351.01	0.00106466297273126\\
352.01	0.0010666373674783\\
353.01	0.00106865708707073\\
354.01	0.00107072319175775\\
355.01	0.00107283676766021\\
356.01	0.00107499892743283\\
357.01	0.00107721081094243\\
358.01	0.0010794735859621\\
359.01	0.00108178844888184\\
360.01	0.00108415662543624\\
361.01	0.00108657937144881\\
362.01	0.00108905797359376\\
363.01	0.00109159375017568\\
364.01	0.0010941880519273\\
365.01	0.00109684226282558\\
366.01	0.00109955780092683\\
367.01	0.00110233611922096\\
368.01	0.00110517870650565\\
369.01	0.00110808708828062\\
370.01	0.00111106282766278\\
371.01	0.00111410752632268\\
372.01	0.001117222825443\\
373.01	0.00112041040669993\\
374.01	0.00112367199326822\\
375.01	0.0011270093508514\\
376.01	0.00113042428873812\\
377.01	0.00113391866088624\\
378.01	0.00113749436703708\\
379.01	0.00114115335386133\\
380.01	0.00114489761613958\\
381.01	0.00114872919797993\\
382.01	0.00115265019407521\\
383.01	0.00115666275100261\\
384.01	0.00116076906856723\\
385.01	0.00116497140119103\\
386.01	0.00116927205934747\\
387.01	0.0011736734110416\\
388.01	0.00117817788333639\\
389.01	0.00118278796392682\\
390.01	0.00118750620276415\\
391.01	0.00119233521373234\\
392.01	0.00119727767637918\\
393.01	0.00120233633770445\\
394.01	0.00120751401400818\\
395.01	0.00121281359280129\\
396.01	0.00121823803478228\\
397.01	0.00122379037588293\\
398.01	0.00122947372938625\\
399.01	0.00123529128812105\\
400.01	0.00124124632673615\\
401.01	0.0012473422040592\\
402.01	0.00125358236554408\\
403.01	0.0012599703458121\\
404.01	0.00126650977129149\\
405.01	0.00127320436296136\\
406.01	0.00128005793920519\\
407.01	0.00128707441878058\\
408.01	0.00129425782391166\\
409.01	0.00130161228351112\\
410.01	0.00130914203653993\\
411.01	0.00131685143551185\\
412.01	0.00132474495015258\\
413.01	0.00133282717122183\\
414.01	0.00134110281450882\\
415.01	0.00134957672501127\\
416.01	0.00135825388130991\\
417.01	0.00136713940015008\\
418.01	0.00137623854124377\\
419.01	0.00138555671230634\\
420.01	0.00139509947434261\\
421.01	0.00140487254719901\\
422.01	0.00141488181539902\\
423.01	0.00142513333428104\\
424.01	0.00143563333645922\\
425.01	0.00144638823862921\\
426.01	0.00145740464874334\\
427.01	0.00146868937358098\\
428.01	0.00148024942674369\\
429.01	0.00149209203710532\\
430.01	0.00150422465775273\\
431.01	0.00151665497545458\\
432.01	0.00152939092070017\\
433.01	0.00154244067835579\\
434.01	0.00155581269899002\\
435.01	0.00156951571092695\\
436.01	0.00158355873309278\\
437.01	0.00159795108873086\\
438.01	0.00161270242006914\\
439.01	0.00162782270403704\\
440.01	0.00164332226914268\\
441.01	0.00165921181363873\\
442.01	0.00167550242512398\\
443.01	0.00169220560175436\\
444.01	0.00170933327526415\\
445.01	0.00172689783603421\\
446.01	0.00174491216048532\\
447.01	0.00176338964112644\\
448.01	0.00178234421964791\\
449.01	0.0018017904235238\\
450.01	0.00182174340667754\\
451.01	0.00184221899487168\\
452.01	0.0018632337366148\\
453.01	0.00188480496053472\\
454.01	0.00190695084035944\\
455.01	0.0019296904688766\\
456.01	0.00195304394251991\\
457.01	0.00197703245856441\\
458.01	0.0020016784273107\\
459.01	0.00202700560210914\\
460.01	0.00205303923061743\\
461.01	0.0020798062312703\\
462.01	0.00210733539945335\\
463.01	0.00213565764792342\\
464.01	0.00216480628436238\\
465.01	0.00219481732160913\\
466.01	0.00222572979017696\\
467.01	0.00225758593534509\\
468.01	0.0022904309084886\\
469.01	0.00232431367757898\\
470.01	0.00235929070988804\\
471.01	0.00239542522304157\\
472.01	0.00243278801778327\\
473.01	0.00247145889961402\\
474.01	0.00251152843611212\\
475.01	0.00255310014955491\\
476.01	0.00259629329740276\\
477.01	0.00264124651736053\\
478.01	0.00268812296161684\\
479.01	0.00273711620747841\\
480.01	0.00278845274894908\\
481.01	0.00281809485375913\\
482.01	0.0028376904564289\\
483.01	0.00285824238600901\\
484.01	0.00287986986848995\\
485.01	0.00290271791097438\\
486.01	0.00292696401033721\\
487.01	0.00295282676371047\\
488.01	0.00298057695173238\\
489.01	0.0030105518446611\\
490.01	0.00304317371901346\\
491.01	0.00307779714308335\\
492.01	0.00311332206216349\\
493.01	0.00314976706439995\\
494.01	0.00318715176354267\\
495.01	0.00322549583333774\\
496.01	0.00326481914899174\\
497.01	0.00330514204057174\\
498.01	0.00334648571336444\\
499.01	0.00338887291518974\\
500.01	0.0034323289608935\\
501.01	0.00347688324818993\\
502.01	0.00352257144897241\\
503.01	0.00356943863488352\\
504.01	0.00361754369033495\\
505.01	0.00366696072314491\\
506.01	0.00371773565702078\\
507.01	0.00376988797378531\\
508.01	0.00382343127861638\\
509.01	0.00387837187091133\\
510.01	0.00393470653278658\\
511.01	0.00399241970744769\\
512.01	0.00405147988646246\\
513.01	0.00411183496672934\\
514.01	0.0041734062597363\\
515.01	0.00423608073040822\\
516.01	0.00429970090066552\\
517.01	0.00436405165869855\\
518.01	0.00442884288366734\\
519.01	0.00449368678115327\\
520.01	0.00455806900599307\\
521.01	0.00462131128482143\\
522.01	0.0046825227041285\\
523.01	0.00474053607862413\\
524.01	0.00479468739552544\\
525.01	0.00484775780185083\\
526.01	0.00490099915561815\\
527.01	0.00495569460839533\\
528.01	0.005011842622378\\
529.01	0.00506943446390541\\
530.01	0.00512845593205903\\
531.01	0.00518889007741343\\
532.01	0.00525072401889197\\
533.01	0.00531396417467579\\
534.01	0.00537862213181257\\
535.01	0.00544470528652228\\
536.01	0.00551221652046107\\
537.01	0.00558115353795019\\
538.01	0.00565150795059695\\
539.01	0.00572326408073132\\
540.01	0.00579639733595701\\
541.01	0.0058708719582017\\
542.01	0.00594663896412132\\
543.01	0.006023640817064\\
544.01	0.00610169629838397\\
545.01	0.0061799446459423\\
546.01	0.00625806301414588\\
547.01	0.0063358244001688\\
548.01	0.00641299880601097\\
549.01	0.00648937082796436\\
550.01	0.00656473392672981\\
551.01	0.00663879945807873\\
552.01	0.00671127018401746\\
553.01	0.00678189713821906\\
554.01	0.00685053104426765\\
555.01	0.00691756867300738\\
556.01	0.00698399436934487\\
557.01	0.00704980690070247\\
558.01	0.0071149853986964\\
559.01	0.00717953865275502\\
560.01	0.00724349912485966\\
561.01	0.00730692267095094\\
562.01	0.00736991220277727\\
563.01	0.00743262416484024\\
564.01	0.00749525849673979\\
565.01	0.00755799012829877\\
566.01	0.00762087397561231\\
567.01	0.00768395414595888\\
568.01	0.00774728354746372\\
569.01	0.00781092427126301\\
570.01	0.0078749452850801\\
571.01	0.00793941585126919\\
572.01	0.00800439739752404\\
573.01	0.00806993562399001\\
574.01	0.00813605795685841\\
575.01	0.00820278543061315\\
576.01	0.00827013997487067\\
577.01	0.00833814309414947\\
578.01	0.0084068139649348\\
579.01	0.00847616764322697\\
580.01	0.00854621376178146\\
581.01	0.00861695616792148\\
582.01	0.00868839387545263\\
583.01	0.00876052282070629\\
584.01	0.00883333648987411\\
585.01	0.0089068255990963\\
586.01	0.00898097767697374\\
587.01	0.00905577659279556\\
588.01	0.00913120213809157\\
589.01	0.00920722970013294\\
590.01	0.00928382994806625\\
591.01	0.00936096842126078\\
592.01	0.00943860510469645\\
593.01	0.00951669429991475\\
594.01	0.00959518502752448\\
595.01	0.00967402215932641\\
596.01	0.00975314847738075\\
597.01	0.00983250783722213\\
598.01	0.00990866197942267\\
599.01	0.00997087280370268\\
599.02	0.00997138072120452\\
599.03	0.00997188557585126\\
599.04	0.00997238733782006\\
599.05	0.0099728859769941\\
599.06	0.00997338146295965\\
599.07	0.00997387376500313\\
599.08	0.00997436285210814\\
599.09	0.00997484869295248\\
599.1	0.00997533125590507\\
599.11	0.00997581050902294\\
599.12	0.00997628642004811\\
599.13	0.00997675895640448\\
599.14	0.00997722808519467\\
599.15	0.00997769377319684\\
599.16	0.00997815598686149\\
599.17	0.00997861469230817\\
599.18	0.00997906985532226\\
599.19	0.0099795214413516\\
599.2	0.0099799694155032\\
599.21	0.0099804137425398\\
599.22	0.00998085438687651\\
599.23	0.00998129131257736\\
599.24	0.00998172448335179\\
599.25	0.00998215386255116\\
599.26	0.00998257941250506\\
599.27	0.00998300109272606\\
599.28	0.00998341886232347\\
599.29	0.00998383267999937\\
599.3	0.00998424250404453\\
599.31	0.00998464829233438\\
599.32	0.00998505000232485\\
599.33	0.00998544759104829\\
599.34	0.00998584101510919\\
599.35	0.00998623023068002\\
599.36	0.00998661519349689\\
599.37	0.00998699585885528\\
599.38	0.00998737218160564\\
599.39	0.00998774411614902\\
599.4	0.00998811161643258\\
599.41	0.00998847463594512\\
599.42	0.00998883312771255\\
599.43	0.00998918704429328\\
599.44	0.00998953633777364\\
599.45	0.00998988095976314\\
599.46	0.00999022086138981\\
599.47	0.00999055599329538\\
599.48	0.00999088630563053\\
599.49	0.00999121174804997\\
599.5	0.00999153226970755\\
599.51	0.00999184781925132\\
599.52	0.0099921583448185\\
599.53	0.00999246379403045\\
599.54	0.00999276411398751\\
599.55	0.00999305925126392\\
599.56	0.00999334915190253\\
599.57	0.00999363376140962\\
599.58	0.00999391302474952\\
599.59	0.00999418688633929\\
599.6	0.00999445529004328\\
599.61	0.00999471817916768\\
599.62	0.00999497549645498\\
599.63	0.00999522718407838\\
599.64	0.00999547318363619\\
599.65	0.00999571343614611\\
599.66	0.00999594788203949\\
599.67	0.00999617646115553\\
599.68	0.00999639911273543\\
599.69	0.00999661577541645\\
599.7	0.00999682638722594\\
599.71	0.00999703088557532\\
599.72	0.00999722920725396\\
599.73	0.00999742128842304\\
599.74	0.00999760706460933\\
599.75	0.0099977864706989\\
599.76	0.00999795944093081\\
599.77	0.00999812590889066\\
599.78	0.00999828580750417\\
599.79	0.00999843906903062\\
599.8	0.00999858562505626\\
599.81	0.00999872540648766\\
599.82	0.00999885834354497\\
599.83	0.00999898436575515\\
599.84	0.00999910340194508\\
599.85	0.00999921538023465\\
599.86	0.00999932022802977\\
599.87	0.00999941787201528\\
599.88	0.00999950823814785\\
599.89	0.00999959125164875\\
599.9	0.00999966683699656\\
599.91	0.00999973491791987\\
599.92	0.0099997954173898\\
599.93	0.00999984825761255\\
599.94	0.00999989336002181\\
599.95	0.00999993064527112\\
599.96	0.00999996003322615\\
599.97	0.00999998144295691\\
599.98	0.00999999479272987\\
599.99	0.01\\
600	0.01\\
};
\addplot [color=blue,solid,forget plot]
  table[row sep=crcr]{%
0.01	0\\
1.01	0\\
2.01	0\\
3.01	0\\
4.01	0\\
5.01	0\\
6.01	0\\
7.01	0\\
8.01	0\\
9.01	0\\
10.01	0\\
11.01	0\\
12.01	0\\
13.01	0\\
14.01	0\\
15.01	0\\
16.01	0\\
17.01	0\\
18.01	0\\
19.01	0\\
20.01	0\\
21.01	0\\
22.01	0\\
23.01	0\\
24.01	0\\
25.01	0\\
26.01	0\\
27.01	0\\
28.01	0\\
29.01	0\\
30.01	0\\
31.01	0\\
32.01	0\\
33.01	0\\
34.01	0\\
35.01	0\\
36.01	0\\
37.01	0\\
38.01	0\\
39.01	0\\
40.01	0\\
41.01	0\\
42.01	0\\
43.01	0\\
44.01	0\\
45.01	0\\
46.01	0\\
47.01	0\\
48.01	0\\
49.01	0\\
50.01	0\\
51.01	0\\
52.01	0\\
53.01	0\\
54.01	0\\
55.01	0\\
56.01	0\\
57.01	0\\
58.01	0\\
59.01	0\\
60.01	0\\
61.01	0\\
62.01	0\\
63.01	0\\
64.01	0\\
65.01	0\\
66.01	0\\
67.01	0\\
68.01	0\\
69.01	0\\
70.01	0\\
71.01	0\\
72.01	0\\
73.01	0\\
74.01	0\\
75.01	0\\
76.01	0\\
77.01	0\\
78.01	0\\
79.01	0\\
80.01	0\\
81.01	0\\
82.01	0\\
83.01	0\\
84.01	0\\
85.01	0\\
86.01	0\\
87.01	0\\
88.01	0\\
89.01	0\\
90.01	0\\
91.01	0\\
92.01	0\\
93.01	0\\
94.01	0\\
95.01	0\\
96.01	0\\
97.01	0\\
98.01	0\\
99.01	0\\
100.01	0\\
101.01	0\\
102.01	0\\
103.01	0\\
104.01	0\\
105.01	0\\
106.01	0\\
107.01	0\\
108.01	0\\
109.01	0\\
110.01	0\\
111.01	0\\
112.01	0\\
113.01	0\\
114.01	0\\
115.01	0\\
116.01	0\\
117.01	0\\
118.01	0\\
119.01	0\\
120.01	0\\
121.01	0\\
122.01	0\\
123.01	0\\
124.01	0\\
125.01	0\\
126.01	0\\
127.01	0\\
128.01	0\\
129.01	0\\
130.01	0\\
131.01	0\\
132.01	0\\
133.01	0\\
134.01	0\\
135.01	0\\
136.01	0\\
137.01	0\\
138.01	0\\
139.01	0\\
140.01	0\\
141.01	0\\
142.01	0\\
143.01	0\\
144.01	0\\
145.01	0\\
146.01	0\\
147.01	0\\
148.01	0\\
149.01	0\\
150.01	0\\
151.01	0\\
152.01	0\\
153.01	0\\
154.01	0\\
155.01	0\\
156.01	0\\
157.01	0\\
158.01	0\\
159.01	0\\
160.01	0\\
161.01	0\\
162.01	0\\
163.01	0\\
164.01	0\\
165.01	0\\
166.01	0\\
167.01	0\\
168.01	0\\
169.01	0\\
170.01	0\\
171.01	0\\
172.01	0\\
173.01	0\\
174.01	0\\
175.01	0\\
176.01	0\\
177.01	0\\
178.01	0\\
179.01	0\\
180.01	0\\
181.01	0\\
182.01	0\\
183.01	0\\
184.01	0\\
185.01	0\\
186.01	0\\
187.01	0\\
188.01	0\\
189.01	0\\
190.01	0\\
191.01	0\\
192.01	0\\
193.01	0\\
194.01	0\\
195.01	0\\
196.01	0\\
197.01	0\\
198.01	0\\
199.01	0\\
200.01	0\\
201.01	0\\
202.01	0\\
203.01	0\\
204.01	0\\
205.01	0\\
206.01	0\\
207.01	0\\
208.01	0\\
209.01	0\\
210.01	0\\
211.01	0\\
212.01	0\\
213.01	0\\
214.01	0\\
215.01	0\\
216.01	0\\
217.01	0\\
218.01	0\\
219.01	0\\
220.01	0\\
221.01	0\\
222.01	0\\
223.01	0\\
224.01	0\\
225.01	0\\
226.01	0\\
227.01	0\\
228.01	0\\
229.01	0\\
230.01	0\\
231.01	0\\
232.01	0\\
233.01	0\\
234.01	0\\
235.01	0\\
236.01	0\\
237.01	0\\
238.01	0\\
239.01	0\\
240.01	0\\
241.01	0\\
242.01	0\\
243.01	0\\
244.01	0\\
245.01	0\\
246.01	0\\
247.01	0\\
248.01	0\\
249.01	0\\
250.01	0\\
251.01	0\\
252.01	0\\
253.01	0\\
254.01	0\\
255.01	0\\
256.01	0\\
257.01	0\\
258.01	0\\
259.01	0\\
260.01	0\\
261.01	0\\
262.01	0\\
263.01	0\\
264.01	0\\
265.01	0\\
266.01	0\\
267.01	0\\
268.01	0\\
269.01	0\\
270.01	0\\
271.01	0\\
272.01	0\\
273.01	0\\
274.01	0\\
275.01	0\\
276.01	0\\
277.01	0\\
278.01	0\\
279.01	0\\
280.01	0\\
281.01	0\\
282.01	0\\
283.01	0\\
284.01	0\\
285.01	0\\
286.01	0\\
287.01	0\\
288.01	0\\
289.01	0\\
290.01	0\\
291.01	0\\
292.01	0\\
293.01	0\\
294.01	0\\
295.01	0\\
296.01	0\\
297.01	0\\
298.01	0\\
299.01	0\\
300.01	0\\
301.01	0\\
302.01	0\\
303.01	0\\
304.01	0\\
305.01	0\\
306.01	0\\
307.01	0\\
308.01	0\\
309.01	0\\
310.01	0\\
311.01	0\\
312.01	0\\
313.01	0\\
314.01	0\\
315.01	0\\
316.01	0\\
317.01	0\\
318.01	0\\
319.01	0\\
320.01	0\\
321.01	0\\
322.01	0\\
323.01	0\\
324.01	0\\
325.01	0\\
326.01	0\\
327.01	0\\
328.01	0\\
329.01	0\\
330.01	0\\
331.01	0\\
332.01	0\\
333.01	0\\
334.01	0\\
335.01	0\\
336.01	0\\
337.01	0\\
338.01	0\\
339.01	0\\
340.01	0\\
341.01	0\\
342.01	0\\
343.01	0\\
344.01	0\\
345.01	0\\
346.01	0\\
347.01	0\\
348.01	0\\
349.01	0\\
350.01	0\\
351.01	0\\
352.01	0\\
353.01	0\\
354.01	0\\
355.01	0\\
356.01	0\\
357.01	0\\
358.01	0\\
359.01	0\\
360.01	0\\
361.01	0\\
362.01	0\\
363.01	0\\
364.01	0\\
365.01	0\\
366.01	0\\
367.01	0\\
368.01	0\\
369.01	0\\
370.01	0\\
371.01	0\\
372.01	0\\
373.01	0\\
374.01	0\\
375.01	0\\
376.01	0\\
377.01	0\\
378.01	0\\
379.01	0\\
380.01	0\\
381.01	0\\
382.01	0\\
383.01	0\\
384.01	0\\
385.01	0\\
386.01	0\\
387.01	0\\
388.01	0\\
389.01	0\\
390.01	0\\
391.01	0\\
392.01	0\\
393.01	0\\
394.01	0\\
395.01	0\\
396.01	0\\
397.01	0\\
398.01	0\\
399.01	0\\
400.01	0\\
401.01	0\\
402.01	0\\
403.01	0\\
404.01	0\\
405.01	0\\
406.01	0\\
407.01	0\\
408.01	0\\
409.01	0\\
410.01	0\\
411.01	0\\
412.01	0\\
413.01	0\\
414.01	0\\
415.01	0\\
416.01	0\\
417.01	0\\
418.01	0\\
419.01	0\\
420.01	0\\
421.01	0\\
422.01	0\\
423.01	0\\
424.01	0\\
425.01	0\\
426.01	0\\
427.01	0\\
428.01	0\\
429.01	0\\
430.01	0\\
431.01	0\\
432.01	0\\
433.01	0\\
434.01	0\\
435.01	0\\
436.01	0\\
437.01	0\\
438.01	0\\
439.01	0\\
440.01	0\\
441.01	0\\
442.01	0\\
443.01	0\\
444.01	0\\
445.01	0\\
446.01	0\\
447.01	0\\
448.01	0\\
449.01	0\\
450.01	0\\
451.01	0\\
452.01	0\\
453.01	0\\
454.01	0\\
455.01	0\\
456.01	0\\
457.01	0\\
458.01	0\\
459.01	0\\
460.01	0\\
461.01	0\\
462.01	0\\
463.01	0\\
464.01	0\\
465.01	0\\
466.01	0\\
467.01	0\\
468.01	0\\
469.01	0\\
470.01	0\\
471.01	0\\
472.01	0\\
473.01	0\\
474.01	0\\
475.01	0\\
476.01	0\\
477.01	0\\
478.01	0\\
479.01	0\\
480.01	0\\
481.01	2.40966981226928e-05\\
482.01	6.02289196438011e-05\\
483.01	9.7493952585008e-05\\
484.01	0.000135932639077595\\
485.01	0.000175584678312843\\
486.01	0.000216487278826116\\
487.01	0.000258673283014495\\
488.01	0.000302168582289146\\
489.01	0.000346988580984628\\
490.01	0.000393133387252401\\
491.01	0.000440595944165442\\
492.01	0.000489414579453555\\
493.01	0.000539644681450131\\
494.01	0.000591344781451793\\
495.01	0.000644576616767242\\
496.01	0.000699405125739842\\
497.01	0.000755898342664832\\
498.01	0.000814127149159516\\
499.01	0.000874164822961092\\
500.01	0.000936086304175951\\
501.01	0.000999967071387352\\
502.01	0.00106588148366777\\
503.01	0.00113390039593665\\
504.01	0.00120408779034123\\
505.01	0.00127650080732459\\
506.01	0.00135123384775706\\
507.01	0.0014284158805291\\
508.01	0.00150818998615476\\
509.01	0.00159071524288537\\
510.01	0.00167616935849231\\
511.01	0.00176475183676328\\
512.01	0.00185668780418655\\
513.01	0.00195223265526322\\
514.01	0.00205167771738155\\
515.01	0.00215535719120302\\
516.01	0.00226365669406223\\
517.01	0.00237702382739495\\
518.01	0.00249598131293208\\
519.01	0.00262114340450183\\
520.01	0.00275323647292278\\
521.01	0.00289312492573644\\
522.01	0.00304184399238899\\
523.01	0.00320064139004236\\
524.01	0.00337017847018786\\
525.01	0.0035477461113088\\
526.01	0.0036295125370641\\
527.01	0.0037108927160592\\
528.01	0.00379456895392187\\
529.01	0.00388056948962633\\
530.01	0.00396891275840736\\
531.01	0.0040596039183237\\
532.01	0.00415263008682738\\
533.01	0.00424795372319946\\
534.01	0.00434550378487202\\
535.01	0.00444515876139365\\
536.01	0.0045467365748054\\
537.01	0.0046499857123393\\
538.01	0.00475456809396017\\
539.01	0.00486003862294787\\
540.01	0.00496581932279654\\
541.01	0.00507116688167566\\
542.01	0.00517506758599043\\
543.01	0.00527604888611295\\
544.01	0.00537221070707498\\
545.01	0.00546672253319988\\
546.01	0.00556220625235845\\
547.01	0.0056580924038923\\
548.01	0.0057535950559224\\
549.01	0.00584763774075489\\
550.01	0.00594170309216324\\
551.01	0.00603796482129104\\
552.01	0.00613623750855436\\
553.01	0.00623623587041443\\
554.01	0.00633753842167594\\
555.01	0.00643917690572694\\
556.01	0.00653969205861012\\
557.01	0.00663870292660752\\
558.01	0.00673595224939041\\
559.01	0.00683132728671051\\
560.01	0.00692492472489956\\
561.01	0.00701661563739462\\
562.01	0.00710600980656644\\
563.01	0.00719278358715181\\
564.01	0.00727677208888639\\
565.01	0.00735899562298691\\
566.01	0.00744011141004657\\
567.01	0.00752017065907459\\
568.01	0.00759925750663107\\
569.01	0.00767738576415853\\
570.01	0.00775462864345971\\
571.01	0.00783113453106925\\
572.01	0.00790711791574717\\
573.01	0.00798282551971004\\
574.01	0.00805842551950888\\
575.01	0.00813393580891355\\
576.01	0.00820937321546196\\
577.01	0.00828477503550238\\
578.01	0.00836019599880814\\
579.01	0.00843570205622002\\
580.01	0.00851136051977954\\
581.01	0.0085872267444925\\
582.01	0.00866332969700664\\
583.01	0.00873967262081048\\
584.01	0.00881625188759261\\
585.01	0.00889306268112347\\
586.01	0.00897010042272349\\
587.01	0.00904736164167313\\
588.01	0.00912484327393588\\
589.01	0.00920254246997357\\
590.01	0.00928045754230262\\
591.01	0.00935858943802847\\
592.01	0.00943694120142937\\
593.01	0.00951551569543855\\
594.01	0.00959431296143754\\
595.01	0.00967332770823237\\
596.01	0.00975254751940146\\
597.01	0.00983195256120927\\
598.01	0.00990865761438377\\
599.01	0.00997087276439033\\
599.02	0.009971380683866\\
599.03	0.00997188554040684\\
599.04	0.00997238730419245\\
599.05	0.00997288594510838\\
599.06	0.00997338143274323\\
599.07	0.00997387373638571\\
599.08	0.00997436282502164\\
599.09	0.00997484866733099\\
599.1	0.00997533123168485\\
599.11	0.0099758104861423\\
599.12	0.00997628639844743\\
599.13	0.00997675893602614\\
599.14	0.00997722806598298\\
599.15	0.00997769375509804\\
599.16	0.00997815596982367\\
599.17	0.00997861467628126\\
599.18	0.00997906984025795\\
599.19	0.00997952142720333\\
599.2	0.00997996940222608\\
599.21	0.00998041373009062\\
599.22	0.00998085437521367\\
599.23	0.00998129130166082\\
599.24	0.00998172447314304\\
599.25	0.0099821538530132\\
599.26	0.00998257940360238\\
599.27	0.00998300108442457\\
599.28	0.00998341885459047\\
599.29	0.00998383267280348\\
599.3	0.00998424249735567\\
599.31	0.00998464828612373\\
599.32	0.00998504999656486\\
599.33	0.00998544758571256\\
599.34	0.00998584101017251\\
599.35	0.00998623022611829\\
599.36	0.00998661518928712\\
599.37	0.00998699585497554\\
599.38	0.00998737217803502\\
599.39	0.0099877441128676\\
599.4	0.00998811161342143\\
599.41	0.00998847463318623\\
599.42	0.00998883312518882\\
599.43	0.00998918704198849\\
599.44	0.00998953633567239\\
599.45	0.00998988095785088\\
599.46	0.00999022085965275\\
599.47	0.0099905559917205\\
599.48	0.00999088630420554\\
599.49	0.00999121174676329\\
599.5	0.00999153226854827\\
599.51	0.00999184781820919\\
599.52	0.00999215834388389\\
599.53	0.00999246379319433\\
599.54	0.00999276411324143\\
599.55	0.00999305925059996\\
599.56	0.00999334915131334\\
599.57	0.00999363376088831\\
599.58	0.0099939130242897\\
599.59	0.00999418688593504\\
599.6	0.0099944552896891\\
599.61	0.00999471817885849\\
599.62	0.00999497549618609\\
599.63	0.00999522718384549\\
599.64	0.00999547318343534\\
599.65	0.00999571343597366\\
599.66	0.00999594788189215\\
599.67	0.00999617646103028\\
599.68	0.00999639911262954\\
599.69	0.00999661577532744\\
599.7	0.00999682638715159\\
599.71	0.00999703088551362\\
599.72	0.00999722920720313\\
599.73	0.00999742128838149\\
599.74	0.00999760706457565\\
599.75	0.00999778647067185\\
599.76	0.00999795944090929\\
599.77	0.00999812590887373\\
599.78	0.00999828580749101\\
599.79	0.00999843906902052\\
599.8	0.00999858562504862\\
599.81	0.00999872540648197\\
599.82	0.00999885834354081\\
599.83	0.00999898436575217\\
599.84	0.009999103401943\\
599.85	0.00999921538023323\\
599.86	0.00999932022802883\\
599.87	0.00999941787201469\\
599.88	0.00999950823814749\\
599.89	0.00999959125164854\\
599.9	0.00999966683699645\\
599.91	0.00999973491791981\\
599.92	0.00999979541738977\\
599.93	0.00999984825761254\\
599.94	0.00999989336002181\\
599.95	0.00999993064527112\\
599.96	0.00999996003322615\\
599.97	0.00999998144295691\\
599.98	0.00999999479272987\\
599.99	0.01\\
600	0.01\\
};
\addplot [color=mycolor10,solid,forget plot]
  table[row sep=crcr]{%
0.01	0\\
1.01	0\\
2.01	0\\
3.01	0\\
4.01	0\\
5.01	0\\
6.01	0\\
7.01	0\\
8.01	0\\
9.01	0\\
10.01	0\\
11.01	0\\
12.01	0\\
13.01	0\\
14.01	0\\
15.01	0\\
16.01	0\\
17.01	0\\
18.01	0\\
19.01	0\\
20.01	0\\
21.01	0\\
22.01	0\\
23.01	0\\
24.01	0\\
25.01	0\\
26.01	0\\
27.01	0\\
28.01	0\\
29.01	0\\
30.01	0\\
31.01	0\\
32.01	0\\
33.01	0\\
34.01	0\\
35.01	0\\
36.01	0\\
37.01	0\\
38.01	0\\
39.01	0\\
40.01	0\\
41.01	0\\
42.01	0\\
43.01	0\\
44.01	0\\
45.01	0\\
46.01	0\\
47.01	0\\
48.01	0\\
49.01	0\\
50.01	0\\
51.01	0\\
52.01	0\\
53.01	0\\
54.01	0\\
55.01	0\\
56.01	0\\
57.01	0\\
58.01	0\\
59.01	0\\
60.01	0\\
61.01	0\\
62.01	0\\
63.01	0\\
64.01	0\\
65.01	0\\
66.01	0\\
67.01	0\\
68.01	0\\
69.01	0\\
70.01	0\\
71.01	0\\
72.01	0\\
73.01	0\\
74.01	0\\
75.01	0\\
76.01	0\\
77.01	0\\
78.01	0\\
79.01	0\\
80.01	0\\
81.01	0\\
82.01	0\\
83.01	0\\
84.01	0\\
85.01	0\\
86.01	0\\
87.01	0\\
88.01	0\\
89.01	0\\
90.01	0\\
91.01	0\\
92.01	0\\
93.01	0\\
94.01	0\\
95.01	0\\
96.01	0\\
97.01	0\\
98.01	0\\
99.01	0\\
100.01	0\\
101.01	0\\
102.01	0\\
103.01	0\\
104.01	0\\
105.01	0\\
106.01	0\\
107.01	0\\
108.01	0\\
109.01	0\\
110.01	0\\
111.01	0\\
112.01	0\\
113.01	0\\
114.01	0\\
115.01	0\\
116.01	0\\
117.01	0\\
118.01	0\\
119.01	0\\
120.01	0\\
121.01	0\\
122.01	0\\
123.01	0\\
124.01	0\\
125.01	0\\
126.01	0\\
127.01	0\\
128.01	0\\
129.01	0\\
130.01	0\\
131.01	0\\
132.01	0\\
133.01	0\\
134.01	0\\
135.01	0\\
136.01	0\\
137.01	0\\
138.01	0\\
139.01	0\\
140.01	0\\
141.01	0\\
142.01	0\\
143.01	0\\
144.01	0\\
145.01	0\\
146.01	0\\
147.01	0\\
148.01	0\\
149.01	0\\
150.01	0\\
151.01	0\\
152.01	0\\
153.01	0\\
154.01	0\\
155.01	0\\
156.01	0\\
157.01	0\\
158.01	0\\
159.01	0\\
160.01	0\\
161.01	0\\
162.01	0\\
163.01	0\\
164.01	0\\
165.01	0\\
166.01	0\\
167.01	0\\
168.01	0\\
169.01	0\\
170.01	0\\
171.01	0\\
172.01	0\\
173.01	0\\
174.01	0\\
175.01	0\\
176.01	0\\
177.01	0\\
178.01	0\\
179.01	0\\
180.01	0\\
181.01	0\\
182.01	0\\
183.01	0\\
184.01	0\\
185.01	0\\
186.01	0\\
187.01	0\\
188.01	0\\
189.01	0\\
190.01	0\\
191.01	0\\
192.01	0\\
193.01	0\\
194.01	0\\
195.01	0\\
196.01	0\\
197.01	0\\
198.01	0\\
199.01	0\\
200.01	0\\
201.01	0\\
202.01	0\\
203.01	0\\
204.01	0\\
205.01	0\\
206.01	0\\
207.01	0\\
208.01	0\\
209.01	0\\
210.01	0\\
211.01	0\\
212.01	0\\
213.01	0\\
214.01	0\\
215.01	0\\
216.01	0\\
217.01	0\\
218.01	0\\
219.01	0\\
220.01	0\\
221.01	0\\
222.01	0\\
223.01	0\\
224.01	0\\
225.01	0\\
226.01	0\\
227.01	0\\
228.01	0\\
229.01	0\\
230.01	0\\
231.01	0\\
232.01	0\\
233.01	0\\
234.01	0\\
235.01	0\\
236.01	0\\
237.01	0\\
238.01	0\\
239.01	0\\
240.01	0\\
241.01	0\\
242.01	0\\
243.01	0\\
244.01	0\\
245.01	0\\
246.01	0\\
247.01	0\\
248.01	0\\
249.01	0\\
250.01	0\\
251.01	0\\
252.01	0\\
253.01	0\\
254.01	0\\
255.01	0\\
256.01	0\\
257.01	0\\
258.01	0\\
259.01	0\\
260.01	0\\
261.01	0\\
262.01	0\\
263.01	0\\
264.01	0\\
265.01	0\\
266.01	0\\
267.01	0\\
268.01	0\\
269.01	0\\
270.01	0\\
271.01	0\\
272.01	0\\
273.01	0\\
274.01	0\\
275.01	0\\
276.01	0\\
277.01	0\\
278.01	0\\
279.01	0\\
280.01	0\\
281.01	0\\
282.01	0\\
283.01	0\\
284.01	0\\
285.01	0\\
286.01	0\\
287.01	0\\
288.01	0\\
289.01	0\\
290.01	0\\
291.01	0\\
292.01	0\\
293.01	0\\
294.01	0\\
295.01	0\\
296.01	0\\
297.01	0\\
298.01	0\\
299.01	0\\
300.01	0\\
301.01	0\\
302.01	0\\
303.01	0\\
304.01	0\\
305.01	0\\
306.01	0\\
307.01	0\\
308.01	0\\
309.01	0\\
310.01	0\\
311.01	0\\
312.01	0\\
313.01	0\\
314.01	0\\
315.01	0\\
316.01	0\\
317.01	0\\
318.01	0\\
319.01	0\\
320.01	0\\
321.01	0\\
322.01	0\\
323.01	0\\
324.01	0\\
325.01	0\\
326.01	0\\
327.01	0\\
328.01	0\\
329.01	0\\
330.01	0\\
331.01	0\\
332.01	0\\
333.01	0\\
334.01	0\\
335.01	0\\
336.01	0\\
337.01	0\\
338.01	0\\
339.01	0\\
340.01	0\\
341.01	0\\
342.01	0\\
343.01	0\\
344.01	0\\
345.01	0\\
346.01	0\\
347.01	0\\
348.01	0\\
349.01	0\\
350.01	0\\
351.01	0\\
352.01	0\\
353.01	0\\
354.01	0\\
355.01	0\\
356.01	0\\
357.01	0\\
358.01	0\\
359.01	0\\
360.01	0\\
361.01	0\\
362.01	0\\
363.01	0\\
364.01	0\\
365.01	0\\
366.01	0\\
367.01	0\\
368.01	0\\
369.01	0\\
370.01	0\\
371.01	0\\
372.01	0\\
373.01	0\\
374.01	0\\
375.01	0\\
376.01	0\\
377.01	0\\
378.01	0\\
379.01	0\\
380.01	0\\
381.01	0\\
382.01	0\\
383.01	0\\
384.01	0\\
385.01	0\\
386.01	0\\
387.01	0\\
388.01	0\\
389.01	0\\
390.01	0\\
391.01	0\\
392.01	0\\
393.01	0\\
394.01	0\\
395.01	0\\
396.01	0\\
397.01	0\\
398.01	0\\
399.01	0\\
400.01	0\\
401.01	0\\
402.01	0\\
403.01	0\\
404.01	0\\
405.01	0\\
406.01	0\\
407.01	0\\
408.01	0\\
409.01	0\\
410.01	0\\
411.01	0\\
412.01	0\\
413.01	0\\
414.01	0\\
415.01	0\\
416.01	0\\
417.01	0\\
418.01	0\\
419.01	0\\
420.01	0\\
421.01	0\\
422.01	0\\
423.01	0\\
424.01	0\\
425.01	0\\
426.01	0\\
427.01	0\\
428.01	0\\
429.01	0\\
430.01	0\\
431.01	0\\
432.01	0\\
433.01	0\\
434.01	0\\
435.01	0\\
436.01	0\\
437.01	0\\
438.01	0\\
439.01	0\\
440.01	0\\
441.01	0\\
442.01	0\\
443.01	0\\
444.01	0\\
445.01	0\\
446.01	0\\
447.01	0\\
448.01	0\\
449.01	0\\
450.01	0\\
451.01	0\\
452.01	0\\
453.01	0\\
454.01	0\\
455.01	0\\
456.01	0\\
457.01	0\\
458.01	0\\
459.01	0\\
460.01	0\\
461.01	0\\
462.01	0\\
463.01	0\\
464.01	0\\
465.01	0\\
466.01	0\\
467.01	0\\
468.01	0\\
469.01	0\\
470.01	0\\
471.01	0\\
472.01	0\\
473.01	0\\
474.01	0\\
475.01	0\\
476.01	0\\
477.01	0\\
478.01	0\\
479.01	0\\
480.01	0\\
481.01	0\\
482.01	0\\
483.01	0\\
484.01	0\\
485.01	0\\
486.01	0\\
487.01	0\\
488.01	0\\
489.01	0\\
490.01	0\\
491.01	0\\
492.01	0\\
493.01	0\\
494.01	0\\
495.01	0\\
496.01	0\\
497.01	0\\
498.01	0\\
499.01	0\\
500.01	0\\
501.01	0\\
502.01	0\\
503.01	0\\
504.01	0\\
505.01	0\\
506.01	0\\
507.01	0\\
508.01	0\\
509.01	0\\
510.01	0\\
511.01	0\\
512.01	0\\
513.01	0\\
514.01	0\\
515.01	0\\
516.01	0\\
517.01	0\\
518.01	0\\
519.01	0\\
520.01	0\\
521.01	0\\
522.01	0\\
523.01	0\\
524.01	0\\
525.01	0\\
526.01	0.000102771456501963\\
527.01	0.000211857262483357\\
528.01	0.000324838345335268\\
529.01	0.000441966255021686\\
530.01	0.000563519085941433\\
531.01	0.000689805325113445\\
532.01	0.000821168123154713\\
533.01	0.000957989226061683\\
534.01	0.00110069167556698\\
535.01	0.00124976707994646\\
536.01	0.00140579226223662\\
537.01	0.00156943036835827\\
538.01	0.00174142679592595\\
539.01	0.0019226250528362\\
540.01	0.00211398992820282\\
541.01	0.00231662768418949\\
542.01	0.00253187134003461\\
543.01	0.00276144006226553\\
544.01	0.00300736772181577\\
545.01	0.00326661567286447\\
546.01	0.00353699838186093\\
547.01	0.00381957633107237\\
548.01	0.00411560683043353\\
549.01	0.00442528142982412\\
550.01	0.00456342194511371\\
551.01	0.00470501977180571\\
552.01	0.00484979652204437\\
553.01	0.00499735440866724\\
554.01	0.00514714434222182\\
555.01	0.00529842819406245\\
556.01	0.00545023857511322\\
557.01	0.00560127475048913\\
558.01	0.00574977686249003\\
559.01	0.00589337759184622\\
560.01	0.00603144302138599\\
561.01	0.00617116532806554\\
562.01	0.00631279169818998\\
563.01	0.00645550940284293\\
564.01	0.00659814217849057\\
565.01	0.00673791067745923\\
566.01	0.0068719890954267\\
567.01	0.00699830368032526\\
568.01	0.00712095802857818\\
569.01	0.00724086556742704\\
570.01	0.00735745019262945\\
571.01	0.00747029237527268\\
572.01	0.00757928265660215\\
573.01	0.00768484450268323\\
574.01	0.00778859580472871\\
575.01	0.0078909719541174\\
576.01	0.00799171655542844\\
577.01	0.00809063074719604\\
578.01	0.00818760060298669\\
579.01	0.0082826282025498\\
580.01	0.00837586389043174\\
581.01	0.00846763456953067\\
582.01	0.00855842645664378\\
583.01	0.0086485452226052\\
584.01	0.0087380156008509\\
585.01	0.00882682224290697\\
586.01	0.00891491374940033\\
587.01	0.00900224063450917\\
588.01	0.00908876407379003\\
589.01	0.0091744495113924\\
590.01	0.00925926074743837\\
591.01	0.00934317099294246\\
592.01	0.0094261874329281\\
593.01	0.00950835739313356\\
594.01	0.0095897680639356\\
595.01	0.00967054251212487\\
596.01	0.00975083116190794\\
597.01	0.00983079836948479\\
598.01	0.00990857284332932\\
599.01	0.00997086936970872\\
599.02	0.0099713774254769\\
599.03	0.00997188241419166\\
599.04	0.00997238430611518\\
599.05	0.00997288307121482\\
599.06	0.00997337867916017\\
599.07	0.00997387109932015\\
599.08	0.00997436030076002\\
599.09	0.00997484625223838\\
599.1	0.00997532892220413\\
599.11	0.0099758082787934\\
599.12	0.00997628428982652\\
599.13	0.00997675692280479\\
599.14	0.00997722614490744\\
599.15	0.00997769192298835\\
599.16	0.0099781542235729\\
599.17	0.0099786130128547\\
599.18	0.00997906825669229\\
599.19	0.00997951992060585\\
599.2	0.00997996796977384\\
599.21	0.00998041236902966\\
599.22	0.00998085308285816\\
599.23	0.00998129007539227\\
599.24	0.00998172331040948\\
599.25	0.00998215275132834\\
599.26	0.00998257836054724\\
599.27	0.00998300009764374\\
599.28	0.00998341792179137\\
599.29	0.00998383179175552\\
599.3	0.00998424166588949\\
599.31	0.00998464750213039\\
599.32	0.00998504925799502\\
599.33	0.00998544689057571\\
599.34	0.00998584035653615\\
599.35	0.00998622961210713\\
599.36	0.00998661461308229\\
599.37	0.00998699531481375\\
599.38	0.00998737167220779\\
599.39	0.00998774363972043\\
599.4	0.00998811117135299\\
599.41	0.00998847422064756\\
599.42	0.0099888327406825\\
599.43	0.00998918668406783\\
599.44	0.00998953600294064\\
599.45	0.00998988064896035\\
599.46	0.00999022057330405\\
599.47	0.0099905557266617\\
599.48	0.00999088605923132\\
599.49	0.00999121152071412\\
599.5	0.0099915320603096\\
599.51	0.00999184762671058\\
599.52	0.0099921581680982\\
599.53	0.00999246363213685\\
599.54	0.00999276396596905\\
599.55	0.00999305911621034\\
599.56	0.00999334902894398\\
599.57	0.0099936336497158\\
599.58	0.00999391292352879\\
599.59	0.00999418679483778\\
599.6	0.009994455207544\\
599.61	0.00999471810498963\\
599.62	0.00999497542995225\\
599.63	0.00999522712463928\\
599.64	0.00999547313068229\\
599.65	0.00999571338913138\\
599.66	0.00999594784044938\\
599.67	0.00999617642450606\\
599.68	0.00999639908057226\\
599.69	0.00999661574731397\\
599.7	0.00999682636278636\\
599.71	0.00999703086442773\\
599.72	0.00999722918905342\\
599.73	0.00999742127284961\\
599.74	0.00999760705136717\\
599.75	0.00999778645951533\\
599.76	0.00999795943155535\\
599.77	0.00999812590109411\\
599.78	0.00999828580107764\\
599.79	0.0099984390637846\\
599.8	0.00999858562081968\\
599.81	0.00999872540310692\\
599.82	0.009998858340883\\
599.83	0.00999898436369044\\
599.84	0.00999910340037074\\
599.85	0.00999921537905747\\
599.86	0.00999932022716924\\
599.87	0.00999941787140266\\
599.88	0.0099995082377252\\
599.89	0.00999959125136801\\
599.9	0.00999966683681859\\
599.91	0.00999973491781351\\
599.92	0.00999979541733096\\
599.93	0.00999984825758326\\
599.94	0.00999989336000932\\
599.95	0.00999993064526697\\
599.96	0.00999996003322533\\
599.97	0.00999998144295691\\
599.98	0.00999999479272987\\
599.99	0.01\\
600	0.01\\
};
\addplot [color=mycolor11,solid,forget plot]
  table[row sep=crcr]{%
0.01	0\\
1.01	0\\
2.01	0\\
3.01	0\\
4.01	0\\
5.01	0\\
6.01	0\\
7.01	0\\
8.01	0\\
9.01	0\\
10.01	0\\
11.01	0\\
12.01	0\\
13.01	0\\
14.01	0\\
15.01	0\\
16.01	0\\
17.01	0\\
18.01	0\\
19.01	0\\
20.01	0\\
21.01	0\\
22.01	0\\
23.01	0\\
24.01	0\\
25.01	0\\
26.01	0\\
27.01	0\\
28.01	0\\
29.01	0\\
30.01	0\\
31.01	0\\
32.01	0\\
33.01	0\\
34.01	0\\
35.01	0\\
36.01	0\\
37.01	0\\
38.01	0\\
39.01	0\\
40.01	0\\
41.01	0\\
42.01	0\\
43.01	0\\
44.01	0\\
45.01	0\\
46.01	0\\
47.01	0\\
48.01	0\\
49.01	0\\
50.01	0\\
51.01	0\\
52.01	0\\
53.01	0\\
54.01	0\\
55.01	0\\
56.01	0\\
57.01	0\\
58.01	0\\
59.01	0\\
60.01	0\\
61.01	0\\
62.01	0\\
63.01	0\\
64.01	0\\
65.01	0\\
66.01	0\\
67.01	0\\
68.01	0\\
69.01	0\\
70.01	0\\
71.01	0\\
72.01	0\\
73.01	0\\
74.01	0\\
75.01	0\\
76.01	0\\
77.01	0\\
78.01	0\\
79.01	0\\
80.01	0\\
81.01	0\\
82.01	0\\
83.01	0\\
84.01	0\\
85.01	0\\
86.01	0\\
87.01	0\\
88.01	0\\
89.01	0\\
90.01	0\\
91.01	0\\
92.01	0\\
93.01	0\\
94.01	0\\
95.01	0\\
96.01	0\\
97.01	0\\
98.01	0\\
99.01	0\\
100.01	0\\
101.01	0\\
102.01	0\\
103.01	0\\
104.01	0\\
105.01	0\\
106.01	0\\
107.01	0\\
108.01	0\\
109.01	0\\
110.01	0\\
111.01	0\\
112.01	0\\
113.01	0\\
114.01	0\\
115.01	0\\
116.01	0\\
117.01	0\\
118.01	0\\
119.01	0\\
120.01	0\\
121.01	0\\
122.01	0\\
123.01	0\\
124.01	0\\
125.01	0\\
126.01	0\\
127.01	0\\
128.01	0\\
129.01	0\\
130.01	0\\
131.01	0\\
132.01	0\\
133.01	0\\
134.01	0\\
135.01	0\\
136.01	0\\
137.01	0\\
138.01	0\\
139.01	0\\
140.01	0\\
141.01	0\\
142.01	0\\
143.01	0\\
144.01	0\\
145.01	0\\
146.01	0\\
147.01	0\\
148.01	0\\
149.01	0\\
150.01	0\\
151.01	0\\
152.01	0\\
153.01	0\\
154.01	0\\
155.01	0\\
156.01	0\\
157.01	0\\
158.01	0\\
159.01	0\\
160.01	0\\
161.01	0\\
162.01	0\\
163.01	0\\
164.01	0\\
165.01	0\\
166.01	0\\
167.01	0\\
168.01	0\\
169.01	0\\
170.01	0\\
171.01	0\\
172.01	0\\
173.01	0\\
174.01	0\\
175.01	0\\
176.01	0\\
177.01	0\\
178.01	0\\
179.01	0\\
180.01	0\\
181.01	0\\
182.01	0\\
183.01	0\\
184.01	0\\
185.01	0\\
186.01	0\\
187.01	0\\
188.01	0\\
189.01	0\\
190.01	0\\
191.01	0\\
192.01	0\\
193.01	0\\
194.01	0\\
195.01	0\\
196.01	0\\
197.01	0\\
198.01	0\\
199.01	0\\
200.01	0\\
201.01	0\\
202.01	0\\
203.01	0\\
204.01	0\\
205.01	0\\
206.01	0\\
207.01	0\\
208.01	0\\
209.01	0\\
210.01	0\\
211.01	0\\
212.01	0\\
213.01	0\\
214.01	0\\
215.01	0\\
216.01	0\\
217.01	0\\
218.01	0\\
219.01	0\\
220.01	0\\
221.01	0\\
222.01	0\\
223.01	0\\
224.01	0\\
225.01	0\\
226.01	0\\
227.01	0\\
228.01	0\\
229.01	0\\
230.01	0\\
231.01	0\\
232.01	0\\
233.01	0\\
234.01	0\\
235.01	0\\
236.01	0\\
237.01	0\\
238.01	0\\
239.01	0\\
240.01	0\\
241.01	0\\
242.01	0\\
243.01	0\\
244.01	0\\
245.01	0\\
246.01	0\\
247.01	0\\
248.01	0\\
249.01	0\\
250.01	0\\
251.01	0\\
252.01	0\\
253.01	0\\
254.01	0\\
255.01	0\\
256.01	0\\
257.01	0\\
258.01	0\\
259.01	0\\
260.01	0\\
261.01	0\\
262.01	0\\
263.01	0\\
264.01	0\\
265.01	0\\
266.01	0\\
267.01	0\\
268.01	0\\
269.01	0\\
270.01	0\\
271.01	0\\
272.01	0\\
273.01	0\\
274.01	0\\
275.01	0\\
276.01	0\\
277.01	0\\
278.01	0\\
279.01	0\\
280.01	0\\
281.01	0\\
282.01	0\\
283.01	0\\
284.01	0\\
285.01	0\\
286.01	0\\
287.01	0\\
288.01	0\\
289.01	0\\
290.01	0\\
291.01	0\\
292.01	0\\
293.01	0\\
294.01	0\\
295.01	0\\
296.01	0\\
297.01	0\\
298.01	0\\
299.01	0\\
300.01	0\\
301.01	0\\
302.01	0\\
303.01	0\\
304.01	0\\
305.01	0\\
306.01	0\\
307.01	0\\
308.01	0\\
309.01	0\\
310.01	0\\
311.01	0\\
312.01	0\\
313.01	0\\
314.01	0\\
315.01	0\\
316.01	0\\
317.01	0\\
318.01	0\\
319.01	0\\
320.01	0\\
321.01	0\\
322.01	0\\
323.01	0\\
324.01	0\\
325.01	0\\
326.01	0\\
327.01	0\\
328.01	0\\
329.01	0\\
330.01	0\\
331.01	0\\
332.01	0\\
333.01	0\\
334.01	0\\
335.01	0\\
336.01	0\\
337.01	0\\
338.01	0\\
339.01	0\\
340.01	0\\
341.01	0\\
342.01	0\\
343.01	0\\
344.01	0\\
345.01	0\\
346.01	0\\
347.01	0\\
348.01	0\\
349.01	0\\
350.01	0\\
351.01	0\\
352.01	0\\
353.01	0\\
354.01	0\\
355.01	0\\
356.01	0\\
357.01	0\\
358.01	0\\
359.01	0\\
360.01	0\\
361.01	0\\
362.01	0\\
363.01	0\\
364.01	0\\
365.01	0\\
366.01	0\\
367.01	0\\
368.01	0\\
369.01	0\\
370.01	0\\
371.01	0\\
372.01	0\\
373.01	0\\
374.01	0\\
375.01	0\\
376.01	0\\
377.01	0\\
378.01	0\\
379.01	0\\
380.01	0\\
381.01	0\\
382.01	0\\
383.01	0\\
384.01	0\\
385.01	0\\
386.01	0\\
387.01	0\\
388.01	0\\
389.01	0\\
390.01	0\\
391.01	0\\
392.01	0\\
393.01	0\\
394.01	0\\
395.01	0\\
396.01	0\\
397.01	0\\
398.01	0\\
399.01	0\\
400.01	0\\
401.01	0\\
402.01	0\\
403.01	0\\
404.01	0\\
405.01	0\\
406.01	0\\
407.01	0\\
408.01	0\\
409.01	0\\
410.01	0\\
411.01	0\\
412.01	0\\
413.01	0\\
414.01	0\\
415.01	0\\
416.01	0\\
417.01	0\\
418.01	0\\
419.01	0\\
420.01	0\\
421.01	0\\
422.01	0\\
423.01	0\\
424.01	0\\
425.01	0\\
426.01	0\\
427.01	0\\
428.01	0\\
429.01	0\\
430.01	0\\
431.01	0\\
432.01	0\\
433.01	0\\
434.01	0\\
435.01	0\\
436.01	0\\
437.01	0\\
438.01	0\\
439.01	0\\
440.01	0\\
441.01	0\\
442.01	0\\
443.01	0\\
444.01	0\\
445.01	0\\
446.01	0\\
447.01	0\\
448.01	0\\
449.01	0\\
450.01	0\\
451.01	0\\
452.01	0\\
453.01	0\\
454.01	0\\
455.01	0\\
456.01	0\\
457.01	0\\
458.01	0\\
459.01	0\\
460.01	0\\
461.01	0\\
462.01	0\\
463.01	0\\
464.01	0\\
465.01	0\\
466.01	0\\
467.01	0\\
468.01	0\\
469.01	0\\
470.01	0\\
471.01	0\\
472.01	0\\
473.01	0\\
474.01	0\\
475.01	0\\
476.01	0\\
477.01	0\\
478.01	0\\
479.01	0\\
480.01	0\\
481.01	0\\
482.01	0\\
483.01	0\\
484.01	0\\
485.01	0\\
486.01	0\\
487.01	0\\
488.01	0\\
489.01	0\\
490.01	0\\
491.01	0\\
492.01	0\\
493.01	0\\
494.01	0\\
495.01	0\\
496.01	0\\
497.01	0\\
498.01	0\\
499.01	0\\
500.01	0\\
501.01	0\\
502.01	0\\
503.01	0\\
504.01	0\\
505.01	0\\
506.01	0\\
507.01	0\\
508.01	0\\
509.01	0\\
510.01	0\\
511.01	0\\
512.01	0\\
513.01	0\\
514.01	0\\
515.01	0\\
516.01	0\\
517.01	0\\
518.01	0\\
519.01	0\\
520.01	0\\
521.01	0\\
522.01	0\\
523.01	0\\
524.01	0\\
525.01	0\\
526.01	0\\
527.01	0\\
528.01	0\\
529.01	0\\
530.01	0\\
531.01	0\\
532.01	0\\
533.01	0\\
534.01	0\\
535.01	0\\
536.01	0\\
537.01	0\\
538.01	0\\
539.01	0\\
540.01	0\\
541.01	0\\
542.01	0\\
543.01	0\\
544.01	0\\
545.01	0\\
546.01	0\\
547.01	0\\
548.01	0\\
549.01	1.3150276878017e-06\\
550.01	0.00018803212887869\\
551.01	0.000383425867251676\\
552.01	0.000588371123930863\\
553.01	0.000803879090139129\\
554.01	0.0010311236007458\\
555.01	0.00127147336229072\\
556.01	0.00152653133905252\\
557.01	0.00179818381624366\\
558.01	0.00208866277061284\\
559.01	0.00240062447287979\\
560.01	0.0027347493126044\\
561.01	0.00308407921609736\\
562.01	0.00344878121259429\\
563.01	0.0038298958305099\\
564.01	0.00422869759739618\\
565.01	0.00464705663062651\\
566.01	0.00508738754435159\\
567.01	0.00544973364055435\\
568.01	0.0056475991624296\\
569.01	0.00584743223013972\\
570.01	0.00604748900368386\\
571.01	0.00624531985607343\\
572.01	0.00643752645254089\\
573.01	0.0066221749281171\\
574.01	0.00680685556915087\\
575.01	0.00699128669887769\\
576.01	0.00717427665445123\\
577.01	0.00735438709921835\\
578.01	0.00752989876416336\\
579.01	0.00769877883743499\\
580.01	0.0078586552305761\\
581.01	0.00800680768344173\\
582.01	0.00814164657742384\\
583.01	0.00827128831769647\\
584.01	0.00839855560001231\\
585.01	0.0085240333202042\\
586.01	0.00864790941724483\\
587.01	0.0087698867552374\\
588.01	0.00888970311215465\\
589.01	0.00900714619236641\\
590.01	0.00912201300988473\\
591.01	0.00923390318299627\\
592.01	0.00934234304250483\\
593.01	0.00944692082238488\\
594.01	0.00954732441074327\\
595.01	0.00964338499452397\\
596.01	0.00973512605081985\\
597.01	0.00982281460498365\\
598.01	0.00990633521989775\\
599.01	0.00997061088345206\\
599.02	0.00997112665722618\\
599.03	0.00997163921121569\\
599.04	0.00997214851709544\\
599.05	0.00997265454624966\\
599.06	0.00997315726976898\\
599.07	0.00997365665844757\\
599.08	0.00997415268278013\\
599.09	0.0099746453129589\\
599.1	0.00997513451887065\\
599.11	0.0099756202700936\\
599.12	0.00997610253589435\\
599.13	0.00997658128522475\\
599.14	0.00997705648671877\\
599.15	0.0099775281086893\\
599.16	0.00997799611912494\\
599.17	0.00997846048568677\\
599.18	0.00997892117570508\\
599.19	0.00997937815617603\\
599.2	0.00997983139375833\\
599.21	0.00998028085476989\\
599.22	0.00998072650518434\\
599.23	0.00998116831062769\\
599.24	0.00998160623637473\\
599.25	0.00998204024734566\\
599.26	0.00998247030788024\\
599.27	0.00998289637903019\\
599.28	0.0099833184214462\\
599.29	0.00998373639537388\\
599.3	0.0099841502606498\\
599.31	0.00998455997669731\\
599.32	0.00998496550252246\\
599.33	0.00998536679670981\\
599.34	0.00998576381741821\\
599.35	0.00998615652237657\\
599.36	0.00998654486887954\\
599.37	0.00998692881378318\\
599.38	0.00998730831350056\\
599.39	0.00998768332399733\\
599.4	0.00998805380078729\\
599.41	0.00998841969892779\\
599.42	0.00998878097301526\\
599.43	0.00998913757718051\\
599.44	0.00998948946508413\\
599.45	0.00998983658991178\\
599.46	0.00999017890437104\\
599.47	0.00999051636068749\\
599.48	0.00999084891059986\\
599.49	0.00999117650535519\\
599.5	0.00999149909570389\\
599.51	0.00999181663189475\\
599.52	0.00999212906366997\\
599.53	0.00999243634026004\\
599.54	0.00999273841037864\\
599.55	0.00999303522221745\\
599.56	0.00999332672344092\\
599.57	0.00999361286118097\\
599.58	0.00999389358203168\\
599.59	0.00999416883204387\\
599.6	0.00999443855671962\\
599.61	0.00999470270100682\\
599.62	0.00999496120929356\\
599.63	0.00999521402540248\\
599.64	0.00999546109258514\\
599.65	0.00999570235351621\\
599.66	0.00999593775028769\\
599.67	0.00999616722440306\\
599.68	0.00999639071677128\\
599.69	0.00999660816770083\\
599.7	0.00999681951689364\\
599.71	0.00999702470343895\\
599.72	0.00999722366580711\\
599.73	0.0099974163418433\\
599.74	0.00999760266876119\\
599.75	0.00999778258313656\\
599.76	0.00999795602090076\\
599.77	0.00999812291733421\\
599.78	0.00999828320705971\\
599.79	0.00999843682403578\\
599.8	0.00999858370154984\\
599.81	0.00999872377221136\\
599.82	0.00999885696794489\\
599.83	0.00999898321998304\\
599.84	0.00999910245885937\\
599.85	0.00999921461440118\\
599.86	0.00999931961572222\\
599.87	0.00999941739121527\\
599.88	0.00999950786854473\\
599.89	0.00999959097463901\\
599.9	0.00999966663568284\\
599.91	0.00999973477710955\\
599.92	0.00999979532359316\\
599.93	0.00999984819904042\\
599.94	0.00999989332658271\\
599.95	0.00999993062856786\\
599.96	0.00999996002655181\\
599.97	0.00999998144129022\\
599.98	0.00999999479272987\\
599.99	0.01\\
600	0.01\\
};
\addplot [color=mycolor12,solid,forget plot]
  table[row sep=crcr]{%
0.01	0\\
1.01	0\\
2.01	0\\
3.01	0\\
4.01	0\\
5.01	0\\
6.01	0\\
7.01	0\\
8.01	0\\
9.01	0\\
10.01	0\\
11.01	0\\
12.01	0\\
13.01	0\\
14.01	0\\
15.01	0\\
16.01	0\\
17.01	0\\
18.01	0\\
19.01	0\\
20.01	0\\
21.01	0\\
22.01	0\\
23.01	0\\
24.01	0\\
25.01	0\\
26.01	0\\
27.01	0\\
28.01	0\\
29.01	0\\
30.01	0\\
31.01	0\\
32.01	0\\
33.01	0\\
34.01	0\\
35.01	0\\
36.01	0\\
37.01	0\\
38.01	0\\
39.01	0\\
40.01	0\\
41.01	0\\
42.01	0\\
43.01	0\\
44.01	0\\
45.01	0\\
46.01	0\\
47.01	0\\
48.01	0\\
49.01	0\\
50.01	0\\
51.01	0\\
52.01	0\\
53.01	0\\
54.01	0\\
55.01	0\\
56.01	0\\
57.01	0\\
58.01	0\\
59.01	0\\
60.01	0\\
61.01	0\\
62.01	0\\
63.01	0\\
64.01	0\\
65.01	0\\
66.01	0\\
67.01	0\\
68.01	0\\
69.01	0\\
70.01	0\\
71.01	0\\
72.01	0\\
73.01	0\\
74.01	0\\
75.01	0\\
76.01	0\\
77.01	0\\
78.01	0\\
79.01	0\\
80.01	0\\
81.01	0\\
82.01	0\\
83.01	0\\
84.01	0\\
85.01	0\\
86.01	0\\
87.01	0\\
88.01	0\\
89.01	0\\
90.01	0\\
91.01	0\\
92.01	0\\
93.01	0\\
94.01	0\\
95.01	0\\
96.01	0\\
97.01	0\\
98.01	0\\
99.01	0\\
100.01	0\\
101.01	0\\
102.01	0\\
103.01	0\\
104.01	0\\
105.01	0\\
106.01	0\\
107.01	0\\
108.01	0\\
109.01	0\\
110.01	0\\
111.01	0\\
112.01	0\\
113.01	0\\
114.01	0\\
115.01	0\\
116.01	0\\
117.01	0\\
118.01	0\\
119.01	0\\
120.01	0\\
121.01	0\\
122.01	0\\
123.01	0\\
124.01	0\\
125.01	0\\
126.01	0\\
127.01	0\\
128.01	0\\
129.01	0\\
130.01	0\\
131.01	0\\
132.01	0\\
133.01	0\\
134.01	0\\
135.01	0\\
136.01	0\\
137.01	0\\
138.01	0\\
139.01	0\\
140.01	0\\
141.01	0\\
142.01	0\\
143.01	0\\
144.01	0\\
145.01	0\\
146.01	0\\
147.01	0\\
148.01	0\\
149.01	0\\
150.01	0\\
151.01	0\\
152.01	0\\
153.01	0\\
154.01	0\\
155.01	0\\
156.01	0\\
157.01	0\\
158.01	0\\
159.01	0\\
160.01	0\\
161.01	0\\
162.01	0\\
163.01	0\\
164.01	0\\
165.01	0\\
166.01	0\\
167.01	0\\
168.01	0\\
169.01	0\\
170.01	0\\
171.01	0\\
172.01	0\\
173.01	0\\
174.01	0\\
175.01	0\\
176.01	0\\
177.01	0\\
178.01	0\\
179.01	0\\
180.01	0\\
181.01	0\\
182.01	0\\
183.01	0\\
184.01	0\\
185.01	0\\
186.01	0\\
187.01	0\\
188.01	0\\
189.01	0\\
190.01	0\\
191.01	0\\
192.01	0\\
193.01	0\\
194.01	0\\
195.01	0\\
196.01	0\\
197.01	0\\
198.01	0\\
199.01	0\\
200.01	0\\
201.01	0\\
202.01	0\\
203.01	0\\
204.01	0\\
205.01	0\\
206.01	0\\
207.01	0\\
208.01	0\\
209.01	0\\
210.01	0\\
211.01	0\\
212.01	0\\
213.01	0\\
214.01	0\\
215.01	0\\
216.01	0\\
217.01	0\\
218.01	0\\
219.01	0\\
220.01	0\\
221.01	0\\
222.01	0\\
223.01	0\\
224.01	0\\
225.01	0\\
226.01	0\\
227.01	0\\
228.01	0\\
229.01	0\\
230.01	0\\
231.01	0\\
232.01	0\\
233.01	0\\
234.01	0\\
235.01	0\\
236.01	0\\
237.01	0\\
238.01	0\\
239.01	0\\
240.01	0\\
241.01	0\\
242.01	0\\
243.01	0\\
244.01	0\\
245.01	0\\
246.01	0\\
247.01	0\\
248.01	0\\
249.01	0\\
250.01	0\\
251.01	0\\
252.01	0\\
253.01	0\\
254.01	0\\
255.01	0\\
256.01	0\\
257.01	0\\
258.01	0\\
259.01	0\\
260.01	0\\
261.01	0\\
262.01	0\\
263.01	0\\
264.01	0\\
265.01	0\\
266.01	0\\
267.01	0\\
268.01	0\\
269.01	0\\
270.01	0\\
271.01	0\\
272.01	0\\
273.01	0\\
274.01	0\\
275.01	0\\
276.01	0\\
277.01	0\\
278.01	0\\
279.01	0\\
280.01	0\\
281.01	0\\
282.01	0\\
283.01	0\\
284.01	0\\
285.01	0\\
286.01	0\\
287.01	0\\
288.01	0\\
289.01	0\\
290.01	0\\
291.01	0\\
292.01	0\\
293.01	0\\
294.01	0\\
295.01	0\\
296.01	0\\
297.01	0\\
298.01	0\\
299.01	0\\
300.01	0\\
301.01	0\\
302.01	0\\
303.01	0\\
304.01	0\\
305.01	0\\
306.01	0\\
307.01	0\\
308.01	0\\
309.01	0\\
310.01	0\\
311.01	0\\
312.01	0\\
313.01	0\\
314.01	0\\
315.01	0\\
316.01	0\\
317.01	0\\
318.01	0\\
319.01	0\\
320.01	0\\
321.01	0\\
322.01	0\\
323.01	0\\
324.01	0\\
325.01	0\\
326.01	0\\
327.01	0\\
328.01	0\\
329.01	0\\
330.01	0\\
331.01	0\\
332.01	0\\
333.01	0\\
334.01	0\\
335.01	0\\
336.01	0\\
337.01	0\\
338.01	0\\
339.01	0\\
340.01	0\\
341.01	0\\
342.01	0\\
343.01	0\\
344.01	0\\
345.01	0\\
346.01	0\\
347.01	0\\
348.01	0\\
349.01	0\\
350.01	0\\
351.01	0\\
352.01	0\\
353.01	0\\
354.01	0\\
355.01	0\\
356.01	0\\
357.01	0\\
358.01	0\\
359.01	0\\
360.01	0\\
361.01	0\\
362.01	0\\
363.01	0\\
364.01	0\\
365.01	0\\
366.01	0\\
367.01	0\\
368.01	0\\
369.01	0\\
370.01	0\\
371.01	0\\
372.01	0\\
373.01	0\\
374.01	0\\
375.01	0\\
376.01	0\\
377.01	0\\
378.01	0\\
379.01	0\\
380.01	0\\
381.01	0\\
382.01	0\\
383.01	0\\
384.01	0\\
385.01	0\\
386.01	0\\
387.01	0\\
388.01	0\\
389.01	0\\
390.01	0\\
391.01	0\\
392.01	0\\
393.01	0\\
394.01	0\\
395.01	0\\
396.01	0\\
397.01	0\\
398.01	0\\
399.01	0\\
400.01	0\\
401.01	0\\
402.01	0\\
403.01	0\\
404.01	0\\
405.01	0\\
406.01	0\\
407.01	0\\
408.01	0\\
409.01	0\\
410.01	0\\
411.01	0\\
412.01	0\\
413.01	0\\
414.01	0\\
415.01	0\\
416.01	0\\
417.01	0\\
418.01	0\\
419.01	0\\
420.01	0\\
421.01	0\\
422.01	0\\
423.01	0\\
424.01	0\\
425.01	0\\
426.01	0\\
427.01	0\\
428.01	0\\
429.01	0\\
430.01	0\\
431.01	0\\
432.01	0\\
433.01	0\\
434.01	0\\
435.01	0\\
436.01	0\\
437.01	0\\
438.01	0\\
439.01	0\\
440.01	0\\
441.01	0\\
442.01	0\\
443.01	0\\
444.01	0\\
445.01	0\\
446.01	0\\
447.01	0\\
448.01	0\\
449.01	0\\
450.01	0\\
451.01	0\\
452.01	0\\
453.01	0\\
454.01	0\\
455.01	0\\
456.01	0\\
457.01	0\\
458.01	0\\
459.01	0\\
460.01	0\\
461.01	0\\
462.01	0\\
463.01	0\\
464.01	0\\
465.01	0\\
466.01	0\\
467.01	0\\
468.01	0\\
469.01	0\\
470.01	0\\
471.01	0\\
472.01	0\\
473.01	0\\
474.01	0\\
475.01	0\\
476.01	0\\
477.01	0\\
478.01	0\\
479.01	0\\
480.01	0\\
481.01	0\\
482.01	0\\
483.01	0\\
484.01	0\\
485.01	0\\
486.01	0\\
487.01	0\\
488.01	0\\
489.01	0\\
490.01	0\\
491.01	0\\
492.01	0\\
493.01	0\\
494.01	0\\
495.01	0\\
496.01	0\\
497.01	0\\
498.01	0\\
499.01	0\\
500.01	0\\
501.01	0\\
502.01	0\\
503.01	0\\
504.01	0\\
505.01	0\\
506.01	0\\
507.01	0\\
508.01	0\\
509.01	0\\
510.01	0\\
511.01	0\\
512.01	0\\
513.01	0\\
514.01	0\\
515.01	0\\
516.01	0\\
517.01	0\\
518.01	0\\
519.01	0\\
520.01	0\\
521.01	0\\
522.01	0\\
523.01	0\\
524.01	0\\
525.01	0\\
526.01	0\\
527.01	0\\
528.01	0\\
529.01	0\\
530.01	0\\
531.01	0\\
532.01	0\\
533.01	0\\
534.01	0\\
535.01	0\\
536.01	0\\
537.01	0\\
538.01	0\\
539.01	0\\
540.01	0\\
541.01	0\\
542.01	0\\
543.01	0\\
544.01	0\\
545.01	0\\
546.01	0\\
547.01	0\\
548.01	0\\
549.01	0\\
550.01	0\\
551.01	0\\
552.01	0\\
553.01	0\\
554.01	0\\
555.01	0\\
556.01	0\\
557.01	0\\
558.01	0\\
559.01	0\\
560.01	0\\
561.01	0\\
562.01	0\\
563.01	0\\
564.01	0\\
565.01	0\\
566.01	0\\
567.01	0.000102186357564371\\
568.01	0.000389004268560401\\
569.01	0.000693210272731084\\
570.01	0.00101718055313735\\
571.01	0.00136374365253529\\
572.01	0.0017362802601048\\
573.01	0.00213614104658714\\
574.01	0.00255403495792057\\
575.01	0.00298985795617562\\
576.01	0.00344508022503881\\
577.01	0.00392130871497483\\
578.01	0.00442029093212064\\
579.01	0.00494390684300543\\
580.01	0.00549412735162226\\
581.01	0.00607287639058549\\
582.01	0.00655902377829562\\
583.01	0.00679815156285074\\
584.01	0.00702626828205187\\
585.01	0.00724284999880616\\
586.01	0.00745955259200264\\
587.01	0.00767677270493367\\
588.01	0.00789364530608335\\
589.01	0.00810926959771742\\
590.01	0.00832353557737281\\
591.01	0.00853710703215436\\
592.01	0.00874855389543367\\
593.01	0.00895606871113526\\
594.01	0.00915750548245555\\
595.01	0.0093503362862595\\
596.01	0.0095316075842035\\
597.01	0.00969789970826579\\
598.01	0.00984529451390937\\
599.01	0.00995515823496999\\
599.02	0.00995597699786755\\
599.03	0.00995678973044666\\
599.04	0.00995759639776036\\
599.05	0.0099583969645674\\
599.06	0.00995919139532951\\
599.07	0.00995997965420866\\
599.08	0.00996076170506433\\
599.09	0.00996153751145075\\
599.1	0.00996230703661406\\
599.11	0.00996307024348954\\
599.12	0.00996382709469877\\
599.13	0.00996457755254676\\
599.14	0.00996532157901905\\
599.15	0.00996605913577888\\
599.16	0.00996679018416421\\
599.17	0.00996751468518479\\
599.18	0.00996823259951922\\
599.19	0.00996894388751196\\
599.2	0.00996964850917033\\
599.21	0.00997034642416144\\
599.22	0.00997103759180925\\
599.23	0.00997172197109139\\
599.24	0.00997239952063618\\
599.25	0.00997307019871944\\
599.26	0.00997373396326153\\
599.27	0.0099743907718255\\
599.28	0.00997504058161395\\
599.29	0.00997568334946589\\
599.3	0.00997631903185352\\
599.31	0.00997694758487909\\
599.32	0.00997756896427161\\
599.33	0.00997818312538369\\
599.34	0.00997879002318826\\
599.35	0.0099793896122753\\
599.36	0.00997998184684856\\
599.37	0.00998056668072227\\
599.38	0.00998114406731784\\
599.39	0.00998171395966052\\
599.4	0.00998227631037606\\
599.41	0.00998283107168736\\
599.42	0.00998337819541113\\
599.43	0.00998391763295447\\
599.44	0.00998444933531151\\
599.45	0.00998497325306004\\
599.46	0.00998548933306155\\
599.47	0.00998599752004539\\
599.48	0.00998649775826485\\
599.49	0.00998698999149323\\
599.5	0.00998747416301998\\
599.51	0.0099879502156468\\
599.52	0.00998841809168377\\
599.53	0.00998887773294543\\
599.54	0.00998932908074684\\
599.55	0.00998977207589977\\
599.56	0.00999020665870868\\
599.57	0.00999063276896691\\
599.58	0.00999105034595272\\
599.59	0.00999145932842542\\
599.6	0.00999185965462149\\
599.61	0.00999225126225067\\
599.62	0.00999263408849213\\
599.63	0.00999300806999059\\
599.64	0.0099933731428525\\
599.65	0.00999372924264221\\
599.66	0.00999407630437819\\
599.67	0.00999441426252925\\
599.68	0.00999474305101081\\
599.69	0.00999506260318117\\
599.7	0.00999537285183787\\
599.71	0.009995673729214\\
599.72	0.00999596516697468\\
599.73	0.00999624709621344\\
599.74	0.00999651944744879\\
599.75	0.00999678215062071\\
599.76	0.00999703513508736\\
599.77	0.0099972783296217\\
599.78	0.00999751166240827\\
599.79	0.00999773506104004\\
599.8	0.00999794845251533\\
599.81	0.00999815176323479\\
599.82	0.00999834491899857\\
599.83	0.00999852784500348\\
599.84	0.00999870046584035\\
599.85	0.00999886270549145\\
599.86	0.00999901448732809\\
599.87	0.00999915573410834\\
599.88	0.00999928636797488\\
599.89	0.00999940631045301\\
599.9	0.00999951548244887\\
599.91	0.00999961380424782\\
599.92	0.00999970119551297\\
599.93	0.00999977757528401\\
599.94	0.00999984286197615\\
599.95	0.00999989697337941\\
599.96	0.00999993982665806\\
599.97	0.0099999713383504\\
599.98	0.00999999142436879\\
599.99	0.01\\
600	0.01\\
};
\addplot [color=mycolor13,solid,forget plot]
  table[row sep=crcr]{%
0.01	0\\
1.01	0\\
2.01	0\\
3.01	0\\
4.01	0\\
5.01	0\\
6.01	0\\
7.01	0\\
8.01	0\\
9.01	0\\
10.01	0\\
11.01	0\\
12.01	0\\
13.01	0\\
14.01	0\\
15.01	0\\
16.01	0\\
17.01	0\\
18.01	0\\
19.01	0\\
20.01	0\\
21.01	0\\
22.01	0\\
23.01	0\\
24.01	0\\
25.01	0\\
26.01	0\\
27.01	0\\
28.01	0\\
29.01	0\\
30.01	0\\
31.01	0\\
32.01	0\\
33.01	0\\
34.01	0\\
35.01	0\\
36.01	0\\
37.01	0\\
38.01	0\\
39.01	0\\
40.01	0\\
41.01	0\\
42.01	0\\
43.01	0\\
44.01	0\\
45.01	0\\
46.01	0\\
47.01	0\\
48.01	0\\
49.01	0\\
50.01	0\\
51.01	0\\
52.01	0\\
53.01	0\\
54.01	0\\
55.01	0\\
56.01	0\\
57.01	0\\
58.01	0\\
59.01	0\\
60.01	0\\
61.01	0\\
62.01	0\\
63.01	0\\
64.01	0\\
65.01	0\\
66.01	0\\
67.01	0\\
68.01	0\\
69.01	0\\
70.01	0\\
71.01	0\\
72.01	0\\
73.01	0\\
74.01	0\\
75.01	0\\
76.01	0\\
77.01	0\\
78.01	0\\
79.01	0\\
80.01	0\\
81.01	0\\
82.01	0\\
83.01	0\\
84.01	0\\
85.01	0\\
86.01	0\\
87.01	0\\
88.01	0\\
89.01	0\\
90.01	0\\
91.01	0\\
92.01	0\\
93.01	0\\
94.01	0\\
95.01	0\\
96.01	0\\
97.01	0\\
98.01	0\\
99.01	0\\
100.01	0\\
101.01	0\\
102.01	0\\
103.01	0\\
104.01	0\\
105.01	0\\
106.01	0\\
107.01	0\\
108.01	0\\
109.01	0\\
110.01	0\\
111.01	0\\
112.01	0\\
113.01	0\\
114.01	0\\
115.01	0\\
116.01	0\\
117.01	0\\
118.01	0\\
119.01	0\\
120.01	0\\
121.01	0\\
122.01	0\\
123.01	0\\
124.01	0\\
125.01	0\\
126.01	0\\
127.01	0\\
128.01	0\\
129.01	0\\
130.01	0\\
131.01	0\\
132.01	0\\
133.01	0\\
134.01	0\\
135.01	0\\
136.01	0\\
137.01	0\\
138.01	0\\
139.01	0\\
140.01	0\\
141.01	0\\
142.01	0\\
143.01	0\\
144.01	0\\
145.01	0\\
146.01	0\\
147.01	0\\
148.01	0\\
149.01	0\\
150.01	0\\
151.01	0\\
152.01	0\\
153.01	0\\
154.01	0\\
155.01	0\\
156.01	0\\
157.01	0\\
158.01	0\\
159.01	0\\
160.01	0\\
161.01	0\\
162.01	0\\
163.01	0\\
164.01	0\\
165.01	0\\
166.01	0\\
167.01	0\\
168.01	0\\
169.01	0\\
170.01	0\\
171.01	0\\
172.01	0\\
173.01	0\\
174.01	0\\
175.01	0\\
176.01	0\\
177.01	0\\
178.01	0\\
179.01	0\\
180.01	0\\
181.01	0\\
182.01	0\\
183.01	0\\
184.01	0\\
185.01	0\\
186.01	0\\
187.01	0\\
188.01	0\\
189.01	0\\
190.01	0\\
191.01	0\\
192.01	0\\
193.01	0\\
194.01	0\\
195.01	0\\
196.01	0\\
197.01	0\\
198.01	0\\
199.01	0\\
200.01	0\\
201.01	0\\
202.01	0\\
203.01	0\\
204.01	0\\
205.01	0\\
206.01	0\\
207.01	0\\
208.01	0\\
209.01	0\\
210.01	0\\
211.01	0\\
212.01	0\\
213.01	0\\
214.01	0\\
215.01	0\\
216.01	0\\
217.01	0\\
218.01	0\\
219.01	0\\
220.01	0\\
221.01	0\\
222.01	0\\
223.01	0\\
224.01	0\\
225.01	0\\
226.01	0\\
227.01	0\\
228.01	0\\
229.01	0\\
230.01	0\\
231.01	0\\
232.01	0\\
233.01	0\\
234.01	0\\
235.01	0\\
236.01	0\\
237.01	0\\
238.01	0\\
239.01	0\\
240.01	0\\
241.01	0\\
242.01	0\\
243.01	0\\
244.01	0\\
245.01	0\\
246.01	0\\
247.01	0\\
248.01	0\\
249.01	0\\
250.01	0\\
251.01	0\\
252.01	0\\
253.01	0\\
254.01	0\\
255.01	0\\
256.01	0\\
257.01	0\\
258.01	0\\
259.01	0\\
260.01	0\\
261.01	0\\
262.01	0\\
263.01	0\\
264.01	0\\
265.01	0\\
266.01	0\\
267.01	0\\
268.01	0\\
269.01	0\\
270.01	0\\
271.01	0\\
272.01	0\\
273.01	0\\
274.01	0\\
275.01	0\\
276.01	0\\
277.01	0\\
278.01	0\\
279.01	0\\
280.01	0\\
281.01	0\\
282.01	0\\
283.01	0\\
284.01	0\\
285.01	0\\
286.01	0\\
287.01	0\\
288.01	0\\
289.01	0\\
290.01	0\\
291.01	0\\
292.01	0\\
293.01	0\\
294.01	0\\
295.01	0\\
296.01	0\\
297.01	0\\
298.01	0\\
299.01	0\\
300.01	0\\
301.01	0\\
302.01	0\\
303.01	0\\
304.01	0\\
305.01	0\\
306.01	0\\
307.01	0\\
308.01	0\\
309.01	0\\
310.01	0\\
311.01	0\\
312.01	0\\
313.01	0\\
314.01	0\\
315.01	0\\
316.01	0\\
317.01	0\\
318.01	0\\
319.01	0\\
320.01	0\\
321.01	0\\
322.01	0\\
323.01	0\\
324.01	0\\
325.01	0\\
326.01	0\\
327.01	0\\
328.01	0\\
329.01	0\\
330.01	0\\
331.01	0\\
332.01	0\\
333.01	0\\
334.01	0\\
335.01	0\\
336.01	0\\
337.01	0\\
338.01	0\\
339.01	0\\
340.01	0\\
341.01	0\\
342.01	0\\
343.01	0\\
344.01	0\\
345.01	0\\
346.01	0\\
347.01	0\\
348.01	0\\
349.01	0\\
350.01	0\\
351.01	0\\
352.01	0\\
353.01	0\\
354.01	0\\
355.01	0\\
356.01	0\\
357.01	0\\
358.01	0\\
359.01	0\\
360.01	0\\
361.01	0\\
362.01	0\\
363.01	0\\
364.01	0\\
365.01	0\\
366.01	0\\
367.01	0\\
368.01	0\\
369.01	0\\
370.01	0\\
371.01	0\\
372.01	0\\
373.01	0\\
374.01	0\\
375.01	0\\
376.01	0\\
377.01	0\\
378.01	0\\
379.01	0\\
380.01	0\\
381.01	0\\
382.01	0\\
383.01	0\\
384.01	0\\
385.01	0\\
386.01	0\\
387.01	0\\
388.01	0\\
389.01	0\\
390.01	0\\
391.01	0\\
392.01	0\\
393.01	0\\
394.01	0\\
395.01	0\\
396.01	0\\
397.01	0\\
398.01	0\\
399.01	0\\
400.01	0\\
401.01	0\\
402.01	0\\
403.01	0\\
404.01	0\\
405.01	0\\
406.01	0\\
407.01	0\\
408.01	0\\
409.01	0\\
410.01	0\\
411.01	0\\
412.01	0\\
413.01	0\\
414.01	0\\
415.01	0\\
416.01	0\\
417.01	0\\
418.01	0\\
419.01	0\\
420.01	0\\
421.01	0\\
422.01	0\\
423.01	0\\
424.01	0\\
425.01	0\\
426.01	0\\
427.01	0\\
428.01	0\\
429.01	0\\
430.01	0\\
431.01	0\\
432.01	0\\
433.01	0\\
434.01	0\\
435.01	0\\
436.01	0\\
437.01	0\\
438.01	0\\
439.01	0\\
440.01	0\\
441.01	0\\
442.01	0\\
443.01	0\\
444.01	0\\
445.01	0\\
446.01	0\\
447.01	0\\
448.01	0\\
449.01	0\\
450.01	0\\
451.01	0\\
452.01	0\\
453.01	0\\
454.01	0\\
455.01	0\\
456.01	0\\
457.01	0\\
458.01	0\\
459.01	0\\
460.01	0\\
461.01	0\\
462.01	0\\
463.01	0\\
464.01	0\\
465.01	0\\
466.01	0\\
467.01	0\\
468.01	0\\
469.01	0\\
470.01	0\\
471.01	0\\
472.01	0\\
473.01	0\\
474.01	0\\
475.01	0\\
476.01	0\\
477.01	0\\
478.01	0\\
479.01	0\\
480.01	0\\
481.01	0\\
482.01	0\\
483.01	0\\
484.01	0\\
485.01	0\\
486.01	0\\
487.01	0\\
488.01	0\\
489.01	0\\
490.01	0\\
491.01	0\\
492.01	0\\
493.01	0\\
494.01	0\\
495.01	0\\
496.01	0\\
497.01	0\\
498.01	0\\
499.01	0\\
500.01	0\\
501.01	0\\
502.01	0\\
503.01	0\\
504.01	0\\
505.01	0\\
506.01	0\\
507.01	0\\
508.01	0\\
509.01	0\\
510.01	0\\
511.01	0\\
512.01	0\\
513.01	0\\
514.01	0\\
515.01	0\\
516.01	0\\
517.01	0\\
518.01	0\\
519.01	0\\
520.01	0\\
521.01	0\\
522.01	0\\
523.01	0\\
524.01	0\\
525.01	0\\
526.01	0\\
527.01	0\\
528.01	0\\
529.01	0\\
530.01	0\\
531.01	0\\
532.01	0\\
533.01	0\\
534.01	0\\
535.01	0\\
536.01	0\\
537.01	0\\
538.01	0\\
539.01	0\\
540.01	0\\
541.01	0\\
542.01	0\\
543.01	0\\
544.01	0\\
545.01	0\\
546.01	0\\
547.01	0\\
548.01	0\\
549.01	0\\
550.01	0\\
551.01	0\\
552.01	0\\
553.01	0\\
554.01	0\\
555.01	0\\
556.01	0\\
557.01	0\\
558.01	0\\
559.01	0\\
560.01	0\\
561.01	0\\
562.01	0\\
563.01	0\\
564.01	0\\
565.01	0\\
566.01	0\\
567.01	0\\
568.01	0\\
569.01	0\\
570.01	0\\
571.01	0\\
572.01	0\\
573.01	0\\
574.01	0\\
575.01	0\\
576.01	0\\
577.01	0\\
578.01	0\\
579.01	0\\
580.01	0\\
581.01	0\\
582.01	0.000121361606057675\\
583.01	0.00050917585157843\\
584.01	0.000923898505224239\\
585.01	0.00136483889801735\\
586.01	0.00181970423639984\\
587.01	0.00228837891239878\\
588.01	0.00277241611548145\\
589.01	0.00327301277390237\\
590.01	0.00379000974525\\
591.01	0.00432249455715031\\
592.01	0.00487168325353377\\
593.01	0.0054388733982834\\
594.01	0.00602544747888225\\
595.01	0.00663295416224377\\
596.01	0.00726313661607713\\
597.01	0.00791796813678225\\
598.01	0.00859969802437893\\
599.01	0.00930685224832755\\
599.02	0.00931395529900888\\
599.03	0.00932105763771179\\
599.04	0.00932815923365039\\
599.05	0.00933526005568012\\
599.06	0.0093423600722933\\
599.07	0.00934945925161451\\
599.08	0.00935655756139602\\
599.09	0.00936365496901308\\
599.1	0.00937075144145913\\
599.11	0.009377846945341\\
599.12	0.00938494144687396\\
599.13	0.00939203491187677\\
599.14	0.00939912730576658\\
599.15	0.00940621859355381\\
599.16	0.00941330873983692\\
599.17	0.00942039770879713\\
599.18	0.00942748546419297\\
599.19	0.00943457196935488\\
599.2	0.00944165718717962\\
599.21	0.00944874108012463\\
599.22	0.00945582361020228\\
599.23	0.00946290473897406\\
599.24	0.00946998442754468\\
599.25	0.00947706263655601\\
599.26	0.00948413932618099\\
599.27	0.00949121445611743\\
599.28	0.00949828798558168\\
599.29	0.00950535987330221\\
599.3	0.0095124300775131\\
599.31	0.00951949855594741\\
599.32	0.00952656526583043\\
599.33	0.00953363016387283\\
599.34	0.00954069320626369\\
599.35	0.00954775434866342\\
599.36	0.00955481354619657\\
599.37	0.00956187075344444\\
599.38	0.0095689259244377\\
599.39	0.00957597901264877\\
599.4	0.00958302997098412\\
599.41	0.0095900787517764\\
599.42	0.00959712530677647\\
599.43	0.0096041695871453\\
599.44	0.00961121154344564\\
599.45	0.00961825112563368\\
599.46	0.0096252882830519\\
599.47	0.00963232296442116\\
599.48	0.00963935511783186\\
599.49	0.00964638469073488\\
599.5	0.00965341162993244\\
599.51	0.00966043588156874\\
599.52	0.00966745739112043\\
599.53	0.0096744761033869\\
599.54	0.00968149196248044\\
599.55	0.00968850491181611\\
599.56	0.00969551489410154\\
599.57	0.00970252185132643\\
599.58	0.00970952572475191\\
599.59	0.00971652645489968\\
599.6	0.0097235239815409\\
599.61	0.00973051824368496\\
599.62	0.00973750917956788\\
599.63	0.00974449672664063\\
599.64	0.00975148082155711\\
599.65	0.00975846140016193\\
599.66	0.00976543839747794\\
599.67	0.00977241174769349\\
599.68	0.00977938138414944\\
599.69	0.0097863472393259\\
599.7	0.00979330924482867\\
599.71	0.00980026733137542\\
599.72	0.00980722142878157\\
599.73	0.00981417146594583\\
599.74	0.00982111737083552\\
599.75	0.00982805907047141\\
599.76	0.00983499649091242\\
599.77	0.00984192955723982\\
599.78	0.00984885819354115\\
599.79	0.00985578232289379\\
599.8	0.0098627018673481\\
599.81	0.00986961674791019\\
599.82	0.00987652688452436\\
599.83	0.009883432196055\\
599.84	0.00989033260026818\\
599.85	0.00989722801381275\\
599.86	0.00990411835220096\\
599.87	0.00991100352978864\\
599.88	0.00991788345975494\\
599.89	0.00992475805408144\\
599.9	0.0099316272235309\\
599.91	0.00993849087762531\\
599.92	0.0099453489246235\\
599.93	0.00995220127149813\\
599.94	0.00995904782391208\\
599.95	0.00996588848619419\\
599.96	0.00997272316131441\\
599.97	0.00997955175085828\\
599.98	0.00998637415500062\\
599.99	0.00999319027247863\\
600	0.01\\
};
\addplot [color=mycolor14,solid,forget plot]
  table[row sep=crcr]{%
0.01	0.01\\
1.01	0.01\\
2.01	0.01\\
3.01	0.01\\
4.01	0.01\\
5.01	0.01\\
6.01	0.01\\
7.01	0.01\\
8.01	0.01\\
9.01	0.01\\
10.01	0.01\\
11.01	0.01\\
12.01	0.01\\
13.01	0.01\\
14.01	0.01\\
15.01	0.01\\
16.01	0.01\\
17.01	0.01\\
18.01	0.01\\
19.01	0.01\\
20.01	0.01\\
21.01	0.01\\
22.01	0.01\\
23.01	0.01\\
24.01	0.01\\
25.01	0.01\\
26.01	0.01\\
27.01	0.01\\
28.01	0.01\\
29.01	0.01\\
30.01	0.01\\
31.01	0.01\\
32.01	0.01\\
33.01	0.01\\
34.01	0.01\\
35.01	0.01\\
36.01	0.01\\
37.01	0.01\\
38.01	0.01\\
39.01	0.01\\
40.01	0.01\\
41.01	0.01\\
42.01	0.01\\
43.01	0.01\\
44.01	0.01\\
45.01	0.01\\
46.01	0.01\\
47.01	0.01\\
48.01	0.01\\
49.01	0.01\\
50.01	0.01\\
51.01	0.01\\
52.01	0.01\\
53.01	0.01\\
54.01	0.01\\
55.01	0.01\\
56.01	0.01\\
57.01	0.01\\
58.01	0.01\\
59.01	0.01\\
60.01	0.01\\
61.01	0.01\\
62.01	0.01\\
63.01	0.01\\
64.01	0.01\\
65.01	0.01\\
66.01	0.01\\
67.01	0.01\\
68.01	0.01\\
69.01	0.01\\
70.01	0.01\\
71.01	0.01\\
72.01	0.01\\
73.01	0.01\\
74.01	0.01\\
75.01	0.01\\
76.01	0.01\\
77.01	0.01\\
78.01	0.01\\
79.01	0.01\\
80.01	0.01\\
81.01	0.01\\
82.01	0.01\\
83.01	0.01\\
84.01	0.01\\
85.01	0.01\\
86.01	0.01\\
87.01	0.01\\
88.01	0.01\\
89.01	0.01\\
90.01	0.01\\
91.01	0.01\\
92.01	0.01\\
93.01	0.01\\
94.01	0.01\\
95.01	0.01\\
96.01	0.01\\
97.01	0.01\\
98.01	0.01\\
99.01	0.01\\
100.01	0.01\\
101.01	0.01\\
102.01	0.01\\
103.01	0.01\\
104.01	0.01\\
105.01	0.01\\
106.01	0.01\\
107.01	0.01\\
108.01	0.01\\
109.01	0.01\\
110.01	0.01\\
111.01	0.01\\
112.01	0.01\\
113.01	0.01\\
114.01	0.01\\
115.01	0.01\\
116.01	0.01\\
117.01	0.01\\
118.01	0.01\\
119.01	0.01\\
120.01	0.01\\
121.01	0.01\\
122.01	0.01\\
123.01	0.01\\
124.01	0.01\\
125.01	0.01\\
126.01	0.01\\
127.01	0.01\\
128.01	0.01\\
129.01	0.01\\
130.01	0.01\\
131.01	0.01\\
132.01	0.01\\
133.01	0.01\\
134.01	0.01\\
135.01	0.01\\
136.01	0.01\\
137.01	0.01\\
138.01	0.01\\
139.01	0.01\\
140.01	0.01\\
141.01	0.01\\
142.01	0.01\\
143.01	0.01\\
144.01	0.01\\
145.01	0.01\\
146.01	0.01\\
147.01	0.01\\
148.01	0.01\\
149.01	0.01\\
150.01	0.01\\
151.01	0.01\\
152.01	0.01\\
153.01	0.01\\
154.01	0.01\\
155.01	0.01\\
156.01	0.01\\
157.01	0.01\\
158.01	0.01\\
159.01	0.01\\
160.01	0.01\\
161.01	0.01\\
162.01	0.01\\
163.01	0.01\\
164.01	0.01\\
165.01	0.01\\
166.01	0.01\\
167.01	0.01\\
168.01	0.01\\
169.01	0.01\\
170.01	0.01\\
171.01	0.01\\
172.01	0.01\\
173.01	0.01\\
174.01	0.01\\
175.01	0.01\\
176.01	0.01\\
177.01	0.01\\
178.01	0.01\\
179.01	0.01\\
180.01	0.01\\
181.01	0.01\\
182.01	0.01\\
183.01	0.01\\
184.01	0.01\\
185.01	0.01\\
186.01	0.01\\
187.01	0.01\\
188.01	0.01\\
189.01	0.01\\
190.01	0.01\\
191.01	0.01\\
192.01	0.01\\
193.01	0.01\\
194.01	0.01\\
195.01	0.01\\
196.01	0.01\\
197.01	0.01\\
198.01	0.01\\
199.01	0.01\\
200.01	0.01\\
201.01	0.01\\
202.01	0.01\\
203.01	0.01\\
204.01	0.01\\
205.01	0.01\\
206.01	0.01\\
207.01	0.01\\
208.01	0.01\\
209.01	0.01\\
210.01	0.01\\
211.01	0.01\\
212.01	0.01\\
213.01	0.01\\
214.01	0.01\\
215.01	0.01\\
216.01	0.01\\
217.01	0.01\\
218.01	0.01\\
219.01	0.01\\
220.01	0.01\\
221.01	0.01\\
222.01	0.01\\
223.01	0.01\\
224.01	0.01\\
225.01	0.01\\
226.01	0.01\\
227.01	0.01\\
228.01	0.01\\
229.01	0.01\\
230.01	0.01\\
231.01	0.01\\
232.01	0.01\\
233.01	0.01\\
234.01	0.01\\
235.01	0.01\\
236.01	0.01\\
237.01	0.01\\
238.01	0.01\\
239.01	0.01\\
240.01	0.01\\
241.01	0.01\\
242.01	0.01\\
243.01	0.01\\
244.01	0.01\\
245.01	0.01\\
246.01	0.01\\
247.01	0.01\\
248.01	0.01\\
249.01	0.01\\
250.01	0.01\\
251.01	0.01\\
252.01	0.01\\
253.01	0.01\\
254.01	0.01\\
255.01	0.01\\
256.01	0.01\\
257.01	0.01\\
258.01	0.01\\
259.01	0.01\\
260.01	0.01\\
261.01	0.01\\
262.01	0.01\\
263.01	0.01\\
264.01	0.01\\
265.01	0.01\\
266.01	0.01\\
267.01	0.01\\
268.01	0.01\\
269.01	0.01\\
270.01	0.01\\
271.01	0.01\\
272.01	0.01\\
273.01	0.01\\
274.01	0.01\\
275.01	0.01\\
276.01	0.01\\
277.01	0.01\\
278.01	0.01\\
279.01	0.01\\
280.01	0.01\\
281.01	0.01\\
282.01	0.01\\
283.01	0.01\\
284.01	0.01\\
285.01	0.01\\
286.01	0.01\\
287.01	0.01\\
288.01	0.01\\
289.01	0.01\\
290.01	0.01\\
291.01	0.01\\
292.01	0.01\\
293.01	0.01\\
294.01	0.01\\
295.01	0.01\\
296.01	0.01\\
297.01	0.01\\
298.01	0.01\\
299.01	0.01\\
300.01	0.01\\
301.01	0.01\\
302.01	0.01\\
303.01	0.01\\
304.01	0.01\\
305.01	0.01\\
306.01	0.01\\
307.01	0.01\\
308.01	0.01\\
309.01	0.01\\
310.01	0.01\\
311.01	0.01\\
312.01	0.01\\
313.01	0.01\\
314.01	0.01\\
315.01	0.01\\
316.01	0.01\\
317.01	0.01\\
318.01	0.01\\
319.01	0.01\\
320.01	0.01\\
321.01	0.01\\
322.01	0.01\\
323.01	0.01\\
324.01	0.01\\
325.01	0.01\\
326.01	0.01\\
327.01	0.01\\
328.01	0.01\\
329.01	0.01\\
330.01	0.01\\
331.01	0.01\\
332.01	0.01\\
333.01	0.01\\
334.01	0.01\\
335.01	0.01\\
336.01	0.01\\
337.01	0.01\\
338.01	0.01\\
339.01	0.01\\
340.01	0.01\\
341.01	0.01\\
342.01	0.01\\
343.01	0.01\\
344.01	0.01\\
345.01	0.01\\
346.01	0.01\\
347.01	0.01\\
348.01	0.01\\
349.01	0.01\\
350.01	0.01\\
351.01	0.01\\
352.01	0.01\\
353.01	0.01\\
354.01	0.01\\
355.01	0.01\\
356.01	0.01\\
357.01	0.01\\
358.01	0.01\\
359.01	0.01\\
360.01	0.01\\
361.01	0.01\\
362.01	0.01\\
363.01	0.01\\
364.01	0.01\\
365.01	0.01\\
366.01	0.01\\
367.01	0.01\\
368.01	0.01\\
369.01	0.01\\
370.01	0.01\\
371.01	0.01\\
372.01	0.01\\
373.01	0.01\\
374.01	0.01\\
375.01	0.01\\
376.01	0.01\\
377.01	0.01\\
378.01	0.01\\
379.01	0.01\\
380.01	0.01\\
381.01	0.01\\
382.01	0.01\\
383.01	0.01\\
384.01	0.01\\
385.01	0.01\\
386.01	0.01\\
387.01	0.01\\
388.01	0.01\\
389.01	0.01\\
390.01	0.01\\
391.01	0.01\\
392.01	0.01\\
393.01	0.01\\
394.01	0.01\\
395.01	0.01\\
396.01	0.01\\
397.01	0.01\\
398.01	0.01\\
399.01	0.01\\
400.01	0.01\\
401.01	0.01\\
402.01	0.01\\
403.01	0.01\\
404.01	0.01\\
405.01	0.01\\
406.01	0.01\\
407.01	0.01\\
408.01	0.01\\
409.01	0.01\\
410.01	0.01\\
411.01	0.01\\
412.01	0.01\\
413.01	0.01\\
414.01	0.01\\
415.01	0.01\\
416.01	0.01\\
417.01	0.01\\
418.01	0.01\\
419.01	0.01\\
420.01	0.01\\
421.01	0.01\\
422.01	0.01\\
423.01	0.01\\
424.01	0.01\\
425.01	0.01\\
426.01	0.01\\
427.01	0.01\\
428.01	0.01\\
429.01	0.01\\
430.01	0.01\\
431.01	0.01\\
432.01	0.01\\
433.01	0.01\\
434.01	0.01\\
435.01	0.01\\
436.01	0.01\\
437.01	0.01\\
438.01	0.01\\
439.01	0.01\\
440.01	0.01\\
441.01	0.01\\
442.01	0.01\\
443.01	0.01\\
444.01	0.01\\
445.01	0.01\\
446.01	0.01\\
447.01	0.01\\
448.01	0.01\\
449.01	0.01\\
450.01	0.01\\
451.01	0.01\\
452.01	0.01\\
453.01	0.01\\
454.01	0.01\\
455.01	0.01\\
456.01	0.01\\
457.01	0.01\\
458.01	0.01\\
459.01	0.01\\
460.01	0.01\\
461.01	0.01\\
462.01	0.01\\
463.01	0.01\\
464.01	0.01\\
465.01	0.01\\
466.01	0.01\\
467.01	0.01\\
468.01	0.01\\
469.01	0.01\\
470.01	0.01\\
471.01	0.01\\
472.01	0.01\\
473.01	0.01\\
474.01	0.01\\
475.01	0.01\\
476.01	0.01\\
477.01	0.01\\
478.01	0.01\\
479.01	0.01\\
480.01	0.01\\
481.01	0.01\\
482.01	0.01\\
483.01	0.01\\
484.01	0.01\\
485.01	0.01\\
486.01	0.01\\
487.01	0.01\\
488.01	0.01\\
489.01	0.01\\
490.01	0.01\\
491.01	0.01\\
492.01	0.01\\
493.01	0.01\\
494.01	0.01\\
495.01	0.01\\
496.01	0.01\\
497.01	0.01\\
498.01	0.01\\
499.01	0.01\\
500.01	0.01\\
501.01	0.01\\
502.01	0.01\\
503.01	0.01\\
504.01	0.01\\
505.01	0.01\\
506.01	0.01\\
507.01	0.01\\
508.01	0.01\\
509.01	0.01\\
510.01	0.01\\
511.01	0.01\\
512.01	0.01\\
513.01	0.01\\
514.01	0.01\\
515.01	0.01\\
516.01	0.01\\
517.01	0.01\\
518.01	0.01\\
519.01	0.01\\
520.01	0.01\\
521.01	0.01\\
522.01	0.01\\
523.01	0.01\\
524.01	0.01\\
525.01	0.01\\
526.01	0.01\\
527.01	0.01\\
528.01	0.01\\
529.01	0.01\\
530.01	0.01\\
531.01	0.01\\
532.01	0.01\\
533.01	0.01\\
534.01	0.01\\
535.01	0.01\\
536.01	0.01\\
537.01	0.01\\
538.01	0.01\\
539.01	0.01\\
540.01	0.01\\
541.01	0.01\\
542.01	0.01\\
543.01	0.01\\
544.01	0.01\\
545.01	0.01\\
546.01	0.01\\
547.01	0.01\\
548.01	0.01\\
549.01	0.01\\
550.01	0.01\\
551.01	0.01\\
552.01	0.01\\
553.01	0.01\\
554.01	0.01\\
555.01	0.01\\
556.01	0.01\\
557.01	0.01\\
558.01	0.01\\
559.01	0.01\\
560.01	0.01\\
561.01	0.01\\
562.01	0.01\\
563.01	0.01\\
564.01	0.01\\
565.01	0.01\\
566.01	0.01\\
567.01	0.01\\
568.01	0.01\\
569.01	0.01\\
570.01	0.01\\
571.01	0.01\\
572.01	0.01\\
573.01	0.01\\
574.01	0.01\\
575.01	0.01\\
576.01	0.01\\
577.01	0.01\\
578.01	0.01\\
579.01	0.01\\
580.01	0.01\\
581.01	0.01\\
582.01	0.00992989647703613\\
583.01	0.00954177372982102\\
584.01	0.00912547726397663\\
585.01	0.00868439512189853\\
586.01	0.00822984681208113\\
587.01	0.00776098456227372\\
588.01	0.00727618774155096\\
589.01	0.00677424947721351\\
590.01	0.00625539153484212\\
591.01	0.00572055713570469\\
592.01	0.00516858725663456\\
593.01	0.00459821690406403\\
594.01	0.00400803771374159\\
595.01	0.00339646929396734\\
596.01	0.00276172792797194\\
597.01	0.00210178615661351\\
598.01	0.00141431937259587\\
599.01	0.000700643040783771\\
599.02	0.000693469184667535\\
599.03	0.00068629596350653\\
599.04	0.000679123407609256\\
599.05	0.000671951547635673\\
599.06	0.000664780414601591\\
599.07	0.000657610039883112\\
599.08	0.00065044045522116\\
599.09	0.00064327169272606\\
599.1	0.000636103784882198\\
599.11	0.000628936764552753\\
599.12	0.000621770664984485\\
599.13	0.000614605519812619\\
599.14	0.000607441363065791\\
599.15	0.000600278229171067\\
599.16	0.000593116152959046\\
599.17	0.000585955169669047\\
599.18	0.000578795314954366\\
599.19	0.000571636624887621\\
599.2	0.000564479135966182\\
599.21	0.000557322885117686\\
599.22	0.000550167909705638\\
599.23	0.000543014247535096\\
599.24	0.000535861936858464\\
599.25	0.000528711016381361\\
599.26	0.000521561525268591\\
599.27	0.000514413503150207\\
599.28	0.000507266990127684\\
599.29	0.000500122026780177\\
599.3	0.0004929786541709\\
599.31	0.000485836913853588\\
599.32	0.000478696847879087\\
599.33	0.000471558498802045\\
599.34	0.000464421909687714\\
599.35	0.00045728712411886\\
599.36	0.000450154186202811\\
599.37	0.000443023140578602\\
599.38	0.000435894032424248\\
599.39	0.000428766907464151\\
599.4	0.000421641811976625\\
599.41	0.000414518792801556\\
599.42	0.000407397897348187\\
599.43	0.000400279173603059\\
599.44	0.000393162670138064\\
599.45	0.000386048436118664\\
599.46	0.000378936521312195\\
599.47	0.000371826976094117\\
599.48	0.000364719851456648\\
599.49	0.000357615199017553\\
599.5	0.000350513071029101\\
599.51	0.000343413520387173\\
599.52	0.000336316600640549\\
599.53	0.000329222366000361\\
599.54	0.000322130871349731\\
599.55	0.000315042172253572\\
599.56	0.000307956324968602\\
599.57	0.000300873386453522\\
599.58	0.000293793414379399\\
599.59	0.000286716467140258\\
599.6	0.000279642603863864\\
599.61	0.000272571884422727\\
599.62	0.000265504369445314\\
599.63	0.000258440120327493\\
599.64	0.000251379199244197\\
599.65	0.000244321669161319\\
599.66	0.00023726759384787\\
599.67	0.000230217037888347\\
599.68	0.000223170066695397\\
599.69	0.000216126746522712\\
599.7	0.000209087144478199\\
599.71	0.00020205132853744\\
599.72	0.000195019367557427\\
599.73	0.000187991331290582\\
599.74	0.000180967290399096\\
599.75	0.000173947316469567\\
599.76	0.000166931482027961\\
599.77	0.000159919860554901\\
599.78	0.000152912526501302\\
599.79	0.000145909555304351\\
599.8	0.00013891102340384\\
599.81	0.000131917008258898\\
599.82	0.000124927588365074\\
599.83	0.000117942843271846\\
599.84	0.000110962853600519\\
599.85	0.000103987701062559\\
599.86	9.70174684783599e-05\\
599.87	9.00522397964624e-05\\
599.88	8.30921001132405e-05\\
599.89	7.61371356930683e-05\\
599.9	6.91874339889959e-05\\
599.91	6.22430836639334e-05\\
599.92	5.53041746123722e-05\\
599.93	4.83707979826668e-05\\
599.94	4.14430461998863e-05\\
599.95	3.45210129892633e-05\\
599.96	2.76047934002557e-05\\
599.97	2.06944838312597e-05\\
599.98	1.37901820549801e-05\\
599.99	6.891987244486e-06\\
600	0\\
};
\addplot [color=mycolor15,solid,forget plot]
  table[row sep=crcr]{%
0.01	0.01\\
1.01	0.01\\
2.01	0.01\\
3.01	0.01\\
4.01	0.01\\
5.01	0.01\\
6.01	0.01\\
7.01	0.01\\
8.01	0.01\\
9.01	0.01\\
10.01	0.01\\
11.01	0.01\\
12.01	0.01\\
13.01	0.01\\
14.01	0.01\\
15.01	0.01\\
16.01	0.01\\
17.01	0.01\\
18.01	0.01\\
19.01	0.01\\
20.01	0.01\\
21.01	0.01\\
22.01	0.01\\
23.01	0.01\\
24.01	0.01\\
25.01	0.01\\
26.01	0.01\\
27.01	0.01\\
28.01	0.01\\
29.01	0.01\\
30.01	0.01\\
31.01	0.01\\
32.01	0.01\\
33.01	0.01\\
34.01	0.01\\
35.01	0.01\\
36.01	0.01\\
37.01	0.01\\
38.01	0.01\\
39.01	0.01\\
40.01	0.01\\
41.01	0.01\\
42.01	0.01\\
43.01	0.01\\
44.01	0.01\\
45.01	0.01\\
46.01	0.01\\
47.01	0.01\\
48.01	0.01\\
49.01	0.01\\
50.01	0.01\\
51.01	0.01\\
52.01	0.01\\
53.01	0.01\\
54.01	0.01\\
55.01	0.01\\
56.01	0.01\\
57.01	0.01\\
58.01	0.01\\
59.01	0.01\\
60.01	0.01\\
61.01	0.01\\
62.01	0.01\\
63.01	0.01\\
64.01	0.01\\
65.01	0.01\\
66.01	0.01\\
67.01	0.01\\
68.01	0.01\\
69.01	0.01\\
70.01	0.01\\
71.01	0.01\\
72.01	0.01\\
73.01	0.01\\
74.01	0.01\\
75.01	0.01\\
76.01	0.01\\
77.01	0.01\\
78.01	0.01\\
79.01	0.01\\
80.01	0.01\\
81.01	0.01\\
82.01	0.01\\
83.01	0.01\\
84.01	0.01\\
85.01	0.01\\
86.01	0.01\\
87.01	0.01\\
88.01	0.01\\
89.01	0.01\\
90.01	0.01\\
91.01	0.01\\
92.01	0.01\\
93.01	0.01\\
94.01	0.01\\
95.01	0.01\\
96.01	0.01\\
97.01	0.01\\
98.01	0.01\\
99.01	0.01\\
100.01	0.01\\
101.01	0.01\\
102.01	0.01\\
103.01	0.01\\
104.01	0.01\\
105.01	0.01\\
106.01	0.01\\
107.01	0.01\\
108.01	0.01\\
109.01	0.01\\
110.01	0.01\\
111.01	0.01\\
112.01	0.01\\
113.01	0.01\\
114.01	0.01\\
115.01	0.01\\
116.01	0.01\\
117.01	0.01\\
118.01	0.01\\
119.01	0.01\\
120.01	0.01\\
121.01	0.01\\
122.01	0.01\\
123.01	0.01\\
124.01	0.01\\
125.01	0.01\\
126.01	0.01\\
127.01	0.01\\
128.01	0.01\\
129.01	0.01\\
130.01	0.01\\
131.01	0.01\\
132.01	0.01\\
133.01	0.01\\
134.01	0.01\\
135.01	0.01\\
136.01	0.01\\
137.01	0.01\\
138.01	0.01\\
139.01	0.01\\
140.01	0.01\\
141.01	0.01\\
142.01	0.01\\
143.01	0.01\\
144.01	0.01\\
145.01	0.01\\
146.01	0.01\\
147.01	0.01\\
148.01	0.01\\
149.01	0.01\\
150.01	0.01\\
151.01	0.01\\
152.01	0.01\\
153.01	0.01\\
154.01	0.01\\
155.01	0.01\\
156.01	0.01\\
157.01	0.01\\
158.01	0.01\\
159.01	0.01\\
160.01	0.01\\
161.01	0.01\\
162.01	0.01\\
163.01	0.01\\
164.01	0.01\\
165.01	0.01\\
166.01	0.01\\
167.01	0.01\\
168.01	0.01\\
169.01	0.01\\
170.01	0.01\\
171.01	0.01\\
172.01	0.01\\
173.01	0.01\\
174.01	0.01\\
175.01	0.01\\
176.01	0.01\\
177.01	0.01\\
178.01	0.01\\
179.01	0.01\\
180.01	0.01\\
181.01	0.01\\
182.01	0.01\\
183.01	0.01\\
184.01	0.01\\
185.01	0.01\\
186.01	0.01\\
187.01	0.01\\
188.01	0.01\\
189.01	0.01\\
190.01	0.01\\
191.01	0.01\\
192.01	0.01\\
193.01	0.01\\
194.01	0.01\\
195.01	0.01\\
196.01	0.01\\
197.01	0.01\\
198.01	0.01\\
199.01	0.01\\
200.01	0.01\\
201.01	0.01\\
202.01	0.01\\
203.01	0.01\\
204.01	0.01\\
205.01	0.01\\
206.01	0.01\\
207.01	0.01\\
208.01	0.01\\
209.01	0.01\\
210.01	0.01\\
211.01	0.01\\
212.01	0.01\\
213.01	0.01\\
214.01	0.01\\
215.01	0.01\\
216.01	0.01\\
217.01	0.01\\
218.01	0.01\\
219.01	0.01\\
220.01	0.01\\
221.01	0.01\\
222.01	0.01\\
223.01	0.01\\
224.01	0.01\\
225.01	0.01\\
226.01	0.01\\
227.01	0.01\\
228.01	0.01\\
229.01	0.01\\
230.01	0.01\\
231.01	0.01\\
232.01	0.01\\
233.01	0.01\\
234.01	0.01\\
235.01	0.01\\
236.01	0.01\\
237.01	0.01\\
238.01	0.01\\
239.01	0.01\\
240.01	0.01\\
241.01	0.01\\
242.01	0.01\\
243.01	0.01\\
244.01	0.01\\
245.01	0.01\\
246.01	0.01\\
247.01	0.01\\
248.01	0.01\\
249.01	0.01\\
250.01	0.01\\
251.01	0.01\\
252.01	0.01\\
253.01	0.01\\
254.01	0.01\\
255.01	0.01\\
256.01	0.01\\
257.01	0.01\\
258.01	0.01\\
259.01	0.01\\
260.01	0.01\\
261.01	0.01\\
262.01	0.01\\
263.01	0.01\\
264.01	0.01\\
265.01	0.01\\
266.01	0.01\\
267.01	0.01\\
268.01	0.01\\
269.01	0.01\\
270.01	0.01\\
271.01	0.01\\
272.01	0.01\\
273.01	0.01\\
274.01	0.01\\
275.01	0.01\\
276.01	0.01\\
277.01	0.01\\
278.01	0.01\\
279.01	0.01\\
280.01	0.01\\
281.01	0.01\\
282.01	0.01\\
283.01	0.01\\
284.01	0.01\\
285.01	0.01\\
286.01	0.01\\
287.01	0.01\\
288.01	0.01\\
289.01	0.01\\
290.01	0.01\\
291.01	0.01\\
292.01	0.01\\
293.01	0.01\\
294.01	0.01\\
295.01	0.01\\
296.01	0.01\\
297.01	0.01\\
298.01	0.01\\
299.01	0.01\\
300.01	0.01\\
301.01	0.01\\
302.01	0.01\\
303.01	0.01\\
304.01	0.01\\
305.01	0.01\\
306.01	0.01\\
307.01	0.01\\
308.01	0.01\\
309.01	0.01\\
310.01	0.01\\
311.01	0.01\\
312.01	0.01\\
313.01	0.01\\
314.01	0.01\\
315.01	0.01\\
316.01	0.01\\
317.01	0.01\\
318.01	0.01\\
319.01	0.01\\
320.01	0.01\\
321.01	0.01\\
322.01	0.01\\
323.01	0.01\\
324.01	0.01\\
325.01	0.01\\
326.01	0.01\\
327.01	0.01\\
328.01	0.01\\
329.01	0.01\\
330.01	0.01\\
331.01	0.01\\
332.01	0.01\\
333.01	0.01\\
334.01	0.01\\
335.01	0.01\\
336.01	0.01\\
337.01	0.01\\
338.01	0.01\\
339.01	0.01\\
340.01	0.01\\
341.01	0.01\\
342.01	0.01\\
343.01	0.01\\
344.01	0.01\\
345.01	0.01\\
346.01	0.01\\
347.01	0.01\\
348.01	0.01\\
349.01	0.01\\
350.01	0.01\\
351.01	0.01\\
352.01	0.01\\
353.01	0.01\\
354.01	0.01\\
355.01	0.01\\
356.01	0.01\\
357.01	0.01\\
358.01	0.01\\
359.01	0.01\\
360.01	0.01\\
361.01	0.01\\
362.01	0.01\\
363.01	0.01\\
364.01	0.01\\
365.01	0.01\\
366.01	0.01\\
367.01	0.01\\
368.01	0.01\\
369.01	0.01\\
370.01	0.01\\
371.01	0.01\\
372.01	0.01\\
373.01	0.01\\
374.01	0.01\\
375.01	0.01\\
376.01	0.01\\
377.01	0.01\\
378.01	0.01\\
379.01	0.01\\
380.01	0.01\\
381.01	0.01\\
382.01	0.01\\
383.01	0.01\\
384.01	0.01\\
385.01	0.01\\
386.01	0.01\\
387.01	0.01\\
388.01	0.01\\
389.01	0.01\\
390.01	0.01\\
391.01	0.01\\
392.01	0.01\\
393.01	0.01\\
394.01	0.01\\
395.01	0.01\\
396.01	0.01\\
397.01	0.01\\
398.01	0.01\\
399.01	0.01\\
400.01	0.01\\
401.01	0.01\\
402.01	0.01\\
403.01	0.01\\
404.01	0.01\\
405.01	0.01\\
406.01	0.01\\
407.01	0.01\\
408.01	0.01\\
409.01	0.01\\
410.01	0.01\\
411.01	0.01\\
412.01	0.01\\
413.01	0.01\\
414.01	0.01\\
415.01	0.01\\
416.01	0.01\\
417.01	0.01\\
418.01	0.01\\
419.01	0.01\\
420.01	0.01\\
421.01	0.01\\
422.01	0.01\\
423.01	0.01\\
424.01	0.01\\
425.01	0.01\\
426.01	0.01\\
427.01	0.01\\
428.01	0.01\\
429.01	0.01\\
430.01	0.01\\
431.01	0.01\\
432.01	0.01\\
433.01	0.01\\
434.01	0.01\\
435.01	0.01\\
436.01	0.01\\
437.01	0.01\\
438.01	0.01\\
439.01	0.01\\
440.01	0.01\\
441.01	0.01\\
442.01	0.01\\
443.01	0.01\\
444.01	0.01\\
445.01	0.01\\
446.01	0.01\\
447.01	0.01\\
448.01	0.01\\
449.01	0.01\\
450.01	0.01\\
451.01	0.01\\
452.01	0.01\\
453.01	0.01\\
454.01	0.01\\
455.01	0.01\\
456.01	0.01\\
457.01	0.01\\
458.01	0.01\\
459.01	0.01\\
460.01	0.01\\
461.01	0.01\\
462.01	0.01\\
463.01	0.01\\
464.01	0.01\\
465.01	0.01\\
466.01	0.01\\
467.01	0.01\\
468.01	0.01\\
469.01	0.01\\
470.01	0.01\\
471.01	0.01\\
472.01	0.01\\
473.01	0.01\\
474.01	0.01\\
475.01	0.01\\
476.01	0.01\\
477.01	0.01\\
478.01	0.01\\
479.01	0.01\\
480.01	0.01\\
481.01	0.01\\
482.01	0.01\\
483.01	0.01\\
484.01	0.01\\
485.01	0.01\\
486.01	0.01\\
487.01	0.01\\
488.01	0.01\\
489.01	0.01\\
490.01	0.01\\
491.01	0.01\\
492.01	0.01\\
493.01	0.01\\
494.01	0.01\\
495.01	0.01\\
496.01	0.01\\
497.01	0.01\\
498.01	0.01\\
499.01	0.01\\
500.01	0.01\\
501.01	0.01\\
502.01	0.01\\
503.01	0.01\\
504.01	0.01\\
505.01	0.01\\
506.01	0.01\\
507.01	0.01\\
508.01	0.01\\
509.01	0.01\\
510.01	0.01\\
511.01	0.01\\
512.01	0.01\\
513.01	0.01\\
514.01	0.01\\
515.01	0.01\\
516.01	0.01\\
517.01	0.01\\
518.01	0.01\\
519.01	0.01\\
520.01	0.01\\
521.01	0.01\\
522.01	0.01\\
523.01	0.01\\
524.01	0.01\\
525.01	0.01\\
526.01	0.01\\
527.01	0.01\\
528.01	0.01\\
529.01	0.01\\
530.01	0.01\\
531.01	0.01\\
532.01	0.01\\
533.01	0.01\\
534.01	0.01\\
535.01	0.01\\
536.01	0.01\\
537.01	0.01\\
538.01	0.01\\
539.01	0.01\\
540.01	0.01\\
541.01	0.01\\
542.01	0.01\\
543.01	0.01\\
544.01	0.01\\
545.01	0.01\\
546.01	0.01\\
547.01	0.01\\
548.01	0.01\\
549.01	0.01\\
550.01	0.01\\
551.01	0.01\\
552.01	0.01\\
553.01	0.01\\
554.01	0.01\\
555.01	0.01\\
556.01	0.01\\
557.01	0.01\\
558.01	0.01\\
559.01	0.01\\
560.01	0.01\\
561.01	0.01\\
562.01	0.01\\
563.01	0.01\\
564.01	0.01\\
565.01	0.01\\
566.01	0.01\\
567.01	0.01\\
568.01	0.00972744916310699\\
569.01	0.00942656800190368\\
570.01	0.00910533817373198\\
571.01	0.00876083290411074\\
572.01	0.00838953411780288\\
573.01	0.00799085123339294\\
574.01	0.00757422759961413\\
575.01	0.00713929514049568\\
576.01	0.00668456449300758\\
577.01	0.0062084103816373\\
578.01	0.00570906766945888\\
579.01	0.00518463758754371\\
580.01	0.00463312160520521\\
581.01	0.00405253293841158\\
582.01	0.00351244908859151\\
583.01	0.00326931276710382\\
584.01	0.00303822528933792\\
585.01	0.00281707885798722\\
586.01	0.00259498529778048\\
587.01	0.00237242508948119\\
588.01	0.00215029934370932\\
589.01	0.00192954919884637\\
590.01	0.00171025695550095\\
591.01	0.00149175973833009\\
592.01	0.00127554651852854\\
593.01	0.00106347889233958\\
594.01	0.000857764094569244\\
595.01	0.000660999733626294\\
596.01	0.000476218771598161\\
597.01	0.000306932213913688\\
598.01	0.000157169919638113\\
599.01	4.57135485354369e-05\\
599.02	4.48827722135618e-05\\
599.03	4.40580970898427e-05\\
599.04	4.32395580149871e-05\\
599.05	4.2427190131129e-05\\
599.06	4.16210288745004e-05\\
599.07	4.08211099781359e-05\\
599.08	4.00274694745805e-05\\
599.09	3.92401436986305e-05\\
599.1	3.84591692900969e-05\\
599.11	3.76845831965792e-05\\
599.12	3.69164226762652e-05\\
599.13	3.61547253007622e-05\\
599.14	3.53995289579228e-05\\
599.15	3.46508718547297e-05\\
599.16	3.39087925201615e-05\\
599.17	3.31733298081088e-05\\
599.18	3.24445229002955e-05\\
599.19	3.17224113092294e-05\\
599.2	3.100703488116e-05\\
599.21	3.02984337990763e-05\\
599.22	2.95966485857075e-05\\
599.23	2.89017201065538e-05\\
599.24	2.82136895729344e-05\\
599.25	2.75325985450488e-05\\
599.26	2.68584889344392e-05\\
599.27	2.61914030062933e-05\\
599.28	2.55313833825432e-05\\
599.29	2.48784730449945e-05\\
599.3	2.4232715338468e-05\\
599.31	2.35941539739638e-05\\
599.32	2.29628330318411e-05\\
599.33	2.23387969650098e-05\\
599.34	2.17220906021491e-05\\
599.35	2.11127591509385e-05\\
599.36	2.05108482012935e-05\\
599.37	1.99164037286426e-05\\
599.38	1.93294720971914e-05\\
599.39	1.87501000632195e-05\\
599.4	1.81783347783862e-05\\
599.41	1.76142237930405e-05\\
599.42	1.70578150595693e-05\\
599.43	1.65091569357192e-05\\
599.44	1.59682981879673e-05\\
599.45	1.54352879948774e-05\\
599.46	1.49101760638821e-05\\
599.47	1.43930175409875e-05\\
599.48	1.38838680450361e-05\\
599.49	1.33827836715731e-05\\
599.5	1.28898209967015e-05\\
599.51	1.24050370809564e-05\\
599.52	1.1928489473189e-05\\
599.53	1.14602362144486e-05\\
599.54	1.10003358418636e-05\\
599.55	1.05488473925511e-05\\
599.56	1.01058304074926e-05\\
599.57	9.67134493544235e-06\\
599.58	9.24545153681115e-06\\
599.59	8.82821128755755e-06\\
599.6	8.41968578306831e-06\\
599.61	8.01993714203558e-06\\
599.62	7.62902801032875e-06\\
599.63	7.24702156483861e-06\\
599.64	6.87398151732153e-06\\
599.65	6.50997211822796e-06\\
599.66	6.15505816049279e-06\\
599.67	5.80930498333791e-06\\
599.68	5.47277847600704e-06\\
599.69	5.14554508150751e-06\\
599.7	4.82767180030874e-06\\
599.71	4.51922619398859e-06\\
599.72	4.22027638887629e-06\\
599.73	3.93089107961382e-06\\
599.74	3.65113953270692e-06\\
599.75	3.38109158999275e-06\\
599.76	3.12081767206776e-06\\
599.77	2.87038878166342e-06\\
599.78	2.62987650693179e-06\\
599.79	2.39935302467388e-06\\
599.8	2.17889110350374e-06\\
599.81	1.96856410689117e-06\\
599.82	1.76844599615936e-06\\
599.83	1.57861133334887e-06\\
599.84	1.3991352839967e-06\\
599.85	1.23009361979905e-06\\
599.86	1.07156272114578e-06\\
599.87	9.23619579542082e-07\\
599.88	7.8634179987401e-07\\
599.89	6.59807602540474e-07\\
599.9	5.44095825409999e-07\\
599.91	4.39285925628308e-07\\
599.92	3.45457981228148e-07\\
599.93	2.62692692548291e-07\\
599.94	1.91071383456518e-07\\
599.95	1.30676002336669e-07\\
599.96	8.158912285193e-08\\
599.97	4.38939444496328e-08\\
599.98	1.76742926006473e-08\\
599.99	3.01461875948372e-09\\
600	0\\
};
\addplot [color=mycolor16,solid,forget plot]
  table[row sep=crcr]{%
0.01	0.01\\
1.01	0.01\\
2.01	0.01\\
3.01	0.01\\
4.01	0.01\\
5.01	0.01\\
6.01	0.01\\
7.01	0.01\\
8.01	0.01\\
9.01	0.01\\
10.01	0.01\\
11.01	0.01\\
12.01	0.01\\
13.01	0.01\\
14.01	0.01\\
15.01	0.01\\
16.01	0.01\\
17.01	0.01\\
18.01	0.01\\
19.01	0.01\\
20.01	0.01\\
21.01	0.01\\
22.01	0.01\\
23.01	0.01\\
24.01	0.01\\
25.01	0.01\\
26.01	0.01\\
27.01	0.01\\
28.01	0.01\\
29.01	0.01\\
30.01	0.01\\
31.01	0.01\\
32.01	0.01\\
33.01	0.01\\
34.01	0.01\\
35.01	0.01\\
36.01	0.01\\
37.01	0.01\\
38.01	0.01\\
39.01	0.01\\
40.01	0.01\\
41.01	0.01\\
42.01	0.01\\
43.01	0.01\\
44.01	0.01\\
45.01	0.01\\
46.01	0.01\\
47.01	0.01\\
48.01	0.01\\
49.01	0.01\\
50.01	0.01\\
51.01	0.01\\
52.01	0.01\\
53.01	0.01\\
54.01	0.01\\
55.01	0.01\\
56.01	0.01\\
57.01	0.01\\
58.01	0.01\\
59.01	0.01\\
60.01	0.01\\
61.01	0.01\\
62.01	0.01\\
63.01	0.01\\
64.01	0.01\\
65.01	0.01\\
66.01	0.01\\
67.01	0.01\\
68.01	0.01\\
69.01	0.01\\
70.01	0.01\\
71.01	0.01\\
72.01	0.01\\
73.01	0.01\\
74.01	0.01\\
75.01	0.01\\
76.01	0.01\\
77.01	0.01\\
78.01	0.01\\
79.01	0.01\\
80.01	0.01\\
81.01	0.01\\
82.01	0.01\\
83.01	0.01\\
84.01	0.01\\
85.01	0.01\\
86.01	0.01\\
87.01	0.01\\
88.01	0.01\\
89.01	0.01\\
90.01	0.01\\
91.01	0.01\\
92.01	0.01\\
93.01	0.01\\
94.01	0.01\\
95.01	0.01\\
96.01	0.01\\
97.01	0.01\\
98.01	0.01\\
99.01	0.01\\
100.01	0.01\\
101.01	0.01\\
102.01	0.01\\
103.01	0.01\\
104.01	0.01\\
105.01	0.01\\
106.01	0.01\\
107.01	0.01\\
108.01	0.01\\
109.01	0.01\\
110.01	0.01\\
111.01	0.01\\
112.01	0.01\\
113.01	0.01\\
114.01	0.01\\
115.01	0.01\\
116.01	0.01\\
117.01	0.01\\
118.01	0.01\\
119.01	0.01\\
120.01	0.01\\
121.01	0.01\\
122.01	0.01\\
123.01	0.01\\
124.01	0.01\\
125.01	0.01\\
126.01	0.01\\
127.01	0.01\\
128.01	0.01\\
129.01	0.01\\
130.01	0.01\\
131.01	0.01\\
132.01	0.01\\
133.01	0.01\\
134.01	0.01\\
135.01	0.01\\
136.01	0.01\\
137.01	0.01\\
138.01	0.01\\
139.01	0.01\\
140.01	0.01\\
141.01	0.01\\
142.01	0.01\\
143.01	0.01\\
144.01	0.01\\
145.01	0.01\\
146.01	0.01\\
147.01	0.01\\
148.01	0.01\\
149.01	0.01\\
150.01	0.01\\
151.01	0.01\\
152.01	0.01\\
153.01	0.01\\
154.01	0.01\\
155.01	0.01\\
156.01	0.01\\
157.01	0.01\\
158.01	0.01\\
159.01	0.01\\
160.01	0.01\\
161.01	0.01\\
162.01	0.01\\
163.01	0.01\\
164.01	0.01\\
165.01	0.01\\
166.01	0.01\\
167.01	0.01\\
168.01	0.01\\
169.01	0.01\\
170.01	0.01\\
171.01	0.01\\
172.01	0.01\\
173.01	0.01\\
174.01	0.01\\
175.01	0.01\\
176.01	0.01\\
177.01	0.01\\
178.01	0.01\\
179.01	0.01\\
180.01	0.01\\
181.01	0.01\\
182.01	0.01\\
183.01	0.01\\
184.01	0.01\\
185.01	0.01\\
186.01	0.01\\
187.01	0.01\\
188.01	0.01\\
189.01	0.01\\
190.01	0.01\\
191.01	0.01\\
192.01	0.01\\
193.01	0.01\\
194.01	0.01\\
195.01	0.01\\
196.01	0.01\\
197.01	0.01\\
198.01	0.01\\
199.01	0.01\\
200.01	0.01\\
201.01	0.01\\
202.01	0.01\\
203.01	0.01\\
204.01	0.01\\
205.01	0.01\\
206.01	0.01\\
207.01	0.01\\
208.01	0.01\\
209.01	0.01\\
210.01	0.01\\
211.01	0.01\\
212.01	0.01\\
213.01	0.01\\
214.01	0.01\\
215.01	0.01\\
216.01	0.01\\
217.01	0.01\\
218.01	0.01\\
219.01	0.01\\
220.01	0.01\\
221.01	0.01\\
222.01	0.01\\
223.01	0.01\\
224.01	0.01\\
225.01	0.01\\
226.01	0.01\\
227.01	0.01\\
228.01	0.01\\
229.01	0.01\\
230.01	0.01\\
231.01	0.01\\
232.01	0.01\\
233.01	0.01\\
234.01	0.01\\
235.01	0.01\\
236.01	0.01\\
237.01	0.01\\
238.01	0.01\\
239.01	0.01\\
240.01	0.01\\
241.01	0.01\\
242.01	0.01\\
243.01	0.01\\
244.01	0.01\\
245.01	0.01\\
246.01	0.01\\
247.01	0.01\\
248.01	0.01\\
249.01	0.01\\
250.01	0.01\\
251.01	0.01\\
252.01	0.01\\
253.01	0.01\\
254.01	0.01\\
255.01	0.01\\
256.01	0.01\\
257.01	0.01\\
258.01	0.01\\
259.01	0.01\\
260.01	0.01\\
261.01	0.01\\
262.01	0.01\\
263.01	0.01\\
264.01	0.01\\
265.01	0.01\\
266.01	0.01\\
267.01	0.01\\
268.01	0.01\\
269.01	0.01\\
270.01	0.01\\
271.01	0.01\\
272.01	0.01\\
273.01	0.01\\
274.01	0.01\\
275.01	0.01\\
276.01	0.01\\
277.01	0.01\\
278.01	0.01\\
279.01	0.01\\
280.01	0.01\\
281.01	0.01\\
282.01	0.01\\
283.01	0.01\\
284.01	0.01\\
285.01	0.01\\
286.01	0.01\\
287.01	0.01\\
288.01	0.01\\
289.01	0.01\\
290.01	0.01\\
291.01	0.01\\
292.01	0.01\\
293.01	0.01\\
294.01	0.01\\
295.01	0.01\\
296.01	0.01\\
297.01	0.01\\
298.01	0.01\\
299.01	0.01\\
300.01	0.01\\
301.01	0.01\\
302.01	0.01\\
303.01	0.01\\
304.01	0.01\\
305.01	0.01\\
306.01	0.01\\
307.01	0.01\\
308.01	0.01\\
309.01	0.01\\
310.01	0.01\\
311.01	0.01\\
312.01	0.01\\
313.01	0.01\\
314.01	0.01\\
315.01	0.01\\
316.01	0.01\\
317.01	0.01\\
318.01	0.01\\
319.01	0.01\\
320.01	0.01\\
321.01	0.01\\
322.01	0.01\\
323.01	0.01\\
324.01	0.01\\
325.01	0.01\\
326.01	0.01\\
327.01	0.01\\
328.01	0.01\\
329.01	0.01\\
330.01	0.01\\
331.01	0.01\\
332.01	0.01\\
333.01	0.01\\
334.01	0.01\\
335.01	0.01\\
336.01	0.01\\
337.01	0.01\\
338.01	0.01\\
339.01	0.01\\
340.01	0.01\\
341.01	0.01\\
342.01	0.01\\
343.01	0.01\\
344.01	0.01\\
345.01	0.01\\
346.01	0.01\\
347.01	0.01\\
348.01	0.01\\
349.01	0.01\\
350.01	0.01\\
351.01	0.01\\
352.01	0.01\\
353.01	0.01\\
354.01	0.01\\
355.01	0.01\\
356.01	0.01\\
357.01	0.01\\
358.01	0.01\\
359.01	0.01\\
360.01	0.01\\
361.01	0.01\\
362.01	0.01\\
363.01	0.01\\
364.01	0.01\\
365.01	0.01\\
366.01	0.01\\
367.01	0.01\\
368.01	0.01\\
369.01	0.01\\
370.01	0.01\\
371.01	0.01\\
372.01	0.01\\
373.01	0.01\\
374.01	0.01\\
375.01	0.01\\
376.01	0.01\\
377.01	0.01\\
378.01	0.01\\
379.01	0.01\\
380.01	0.01\\
381.01	0.01\\
382.01	0.01\\
383.01	0.01\\
384.01	0.01\\
385.01	0.01\\
386.01	0.01\\
387.01	0.01\\
388.01	0.01\\
389.01	0.01\\
390.01	0.01\\
391.01	0.01\\
392.01	0.01\\
393.01	0.01\\
394.01	0.01\\
395.01	0.01\\
396.01	0.01\\
397.01	0.01\\
398.01	0.01\\
399.01	0.01\\
400.01	0.01\\
401.01	0.01\\
402.01	0.01\\
403.01	0.01\\
404.01	0.01\\
405.01	0.01\\
406.01	0.01\\
407.01	0.01\\
408.01	0.01\\
409.01	0.01\\
410.01	0.01\\
411.01	0.01\\
412.01	0.01\\
413.01	0.01\\
414.01	0.01\\
415.01	0.01\\
416.01	0.01\\
417.01	0.01\\
418.01	0.01\\
419.01	0.01\\
420.01	0.01\\
421.01	0.01\\
422.01	0.01\\
423.01	0.01\\
424.01	0.01\\
425.01	0.01\\
426.01	0.01\\
427.01	0.01\\
428.01	0.01\\
429.01	0.01\\
430.01	0.01\\
431.01	0.01\\
432.01	0.01\\
433.01	0.01\\
434.01	0.01\\
435.01	0.01\\
436.01	0.01\\
437.01	0.01\\
438.01	0.01\\
439.01	0.01\\
440.01	0.01\\
441.01	0.01\\
442.01	0.01\\
443.01	0.01\\
444.01	0.01\\
445.01	0.01\\
446.01	0.01\\
447.01	0.01\\
448.01	0.01\\
449.01	0.01\\
450.01	0.01\\
451.01	0.01\\
452.01	0.01\\
453.01	0.01\\
454.01	0.01\\
455.01	0.01\\
456.01	0.01\\
457.01	0.01\\
458.01	0.01\\
459.01	0.01\\
460.01	0.01\\
461.01	0.01\\
462.01	0.01\\
463.01	0.01\\
464.01	0.01\\
465.01	0.01\\
466.01	0.01\\
467.01	0.01\\
468.01	0.01\\
469.01	0.01\\
470.01	0.01\\
471.01	0.01\\
472.01	0.01\\
473.01	0.01\\
474.01	0.01\\
475.01	0.01\\
476.01	0.01\\
477.01	0.01\\
478.01	0.01\\
479.01	0.01\\
480.01	0.01\\
481.01	0.01\\
482.01	0.01\\
483.01	0.01\\
484.01	0.01\\
485.01	0.01\\
486.01	0.01\\
487.01	0.01\\
488.01	0.01\\
489.01	0.01\\
490.01	0.01\\
491.01	0.01\\
492.01	0.01\\
493.01	0.01\\
494.01	0.01\\
495.01	0.01\\
496.01	0.01\\
497.01	0.01\\
498.01	0.01\\
499.01	0.01\\
500.01	0.01\\
501.01	0.01\\
502.01	0.01\\
503.01	0.01\\
504.01	0.01\\
505.01	0.01\\
506.01	0.01\\
507.01	0.01\\
508.01	0.01\\
509.01	0.01\\
510.01	0.01\\
511.01	0.01\\
512.01	0.01\\
513.01	0.01\\
514.01	0.01\\
515.01	0.01\\
516.01	0.01\\
517.01	0.01\\
518.01	0.01\\
519.01	0.01\\
520.01	0.01\\
521.01	0.01\\
522.01	0.01\\
523.01	0.01\\
524.01	0.01\\
525.01	0.01\\
526.01	0.01\\
527.01	0.01\\
528.01	0.01\\
529.01	0.01\\
530.01	0.01\\
531.01	0.01\\
532.01	0.01\\
533.01	0.01\\
534.01	0.01\\
535.01	0.01\\
536.01	0.01\\
537.01	0.01\\
538.01	0.01\\
539.01	0.01\\
540.01	0.01\\
541.01	0.01\\
542.01	0.01\\
543.01	0.01\\
544.01	0.01\\
545.01	0.01\\
546.01	0.01\\
547.01	0.01\\
548.01	0.01\\
549.01	0.01\\
550.01	0.00995817214987554\\
551.01	0.00976920720650903\\
552.01	0.00957068161494685\\
553.01	0.00936157648640411\\
554.01	0.00914070877388854\\
555.01	0.00890669871536501\\
556.01	0.00865793000953965\\
557.01	0.00839250036389308\\
558.01	0.00810815851763091\\
559.01	0.00780222450397486\\
560.01	0.00747301733874219\\
561.01	0.00712763658868991\\
562.01	0.00676663990071911\\
563.01	0.00638887747094233\\
564.01	0.00599300805243266\\
565.01	0.00557739361006792\\
566.01	0.00513978932111589\\
567.01	0.0046775026391738\\
568.01	0.00446437522935085\\
569.01	0.00425959342731657\\
570.01	0.00405456668876184\\
571.01	0.00385186976766412\\
572.01	0.00365507220738019\\
573.01	0.00346541136082133\\
574.01	0.00327522104056992\\
575.01	0.00308519632578407\\
576.01	0.00289654874182385\\
577.01	0.00271074026193477\\
578.01	0.00252951755813305\\
579.01	0.00235494432006656\\
580.01	0.00218942634918826\\
581.01	0.00203571962313973\\
582.01	0.00189569368076647\\
583.01	0.00176221935349623\\
584.01	0.00163117395006303\\
585.01	0.00150194649748184\\
586.01	0.00137451254708096\\
587.01	0.00124919829444105\\
588.01	0.001126282592439\\
589.01	0.00100599016945428\\
590.01	0.000888534916031865\\
591.01	0.00077433810196983\\
592.01	0.000663891166351075\\
593.01	0.000557617301030397\\
594.01	0.000455831854103557\\
595.01	0.000358696458436331\\
596.01	0.000266167733632852\\
597.01	0.000177943768206052\\
598.01	9.42984683149437e-05\\
599.01	2.97433029915143e-05\\
599.02	2.92255074340608e-05\\
599.03	2.87109129547599e-05\\
599.04	2.81995475120318e-05\\
599.05	2.76914393515584e-05\\
599.06	2.71866170091662e-05\\
599.07	2.66851093137094e-05\\
599.08	2.61869453900155e-05\\
599.09	2.56921546618414e-05\\
599.1	2.52007668548517e-05\\
599.11	2.47128119996477e-05\\
599.12	2.42283204348152e-05\\
599.13	2.37473228099933e-05\\
599.14	2.32698500889934e-05\\
599.15	2.2795933552925e-05\\
599.16	2.23256048033898e-05\\
599.17	2.18588957656557e-05\\
599.18	2.13958386919115e-05\\
599.19	2.09364661645194e-05\\
599.2	2.04808110993264e-05\\
599.21	2.00289067489832e-05\\
599.22	1.95807867063164e-05\\
599.23	1.91364849077228e-05\\
599.24	1.86960356366034e-05\\
599.25	1.82594735268233e-05\\
599.26	1.78268349628249e-05\\
599.27	1.73981584822864e-05\\
599.28	1.69734830200441e-05\\
599.29	1.65528479120682e-05\\
599.3	1.61362928994564e-05\\
599.31	1.57238581324892e-05\\
599.32	1.53155841747174e-05\\
599.33	1.49115120070802e-05\\
599.34	1.45116830320754e-05\\
599.35	1.41161390779781e-05\\
599.36	1.37249224030859e-05\\
599.37	1.33380757000104e-05\\
599.38	1.29556421000227e-05\\
599.39	1.2577665177432e-05\\
599.4	1.22041889540053e-05\\
599.41	1.18352579034451e-05\\
599.42	1.14709169558962e-05\\
599.43	1.11112115025128e-05\\
599.44	1.07561874000597e-05\\
599.45	1.04058909755692e-05\\
599.46	1.00603690309786e-05\\
599.47	9.7196688453368e-06\\
599.48	9.38383817956912e-06\\
599.49	9.05292528129154e-06\\
599.5	8.72697888967475e-06\\
599.51	8.40604824036036e-06\\
599.52	8.09018307042046e-06\\
599.53	7.77943362337274e-06\\
599.54	7.47385065425279e-06\\
599.55	7.17348543472461e-06\\
599.56	6.87838975826048e-06\\
599.57	6.58861594535556e-06\\
599.58	6.30421684882053e-06\\
599.59	6.02524585910542e-06\\
599.6	5.75175690969466e-06\\
599.61	5.4838044825558e-06\\
599.62	5.22144361363858e-06\\
599.63	4.96472989843823e-06\\
599.64	4.71371949762288e-06\\
599.65	4.46846914270785e-06\\
599.66	4.22903614179931e-06\\
599.67	3.9954783853987e-06\\
599.68	3.76785435226429e-06\\
599.69	3.54622311534575e-06\\
599.7	3.33064434777409e-06\\
599.71	3.12117832892804e-06\\
599.72	2.91788595054021e-06\\
599.73	2.72082872291787e-06\\
599.74	2.53006878117752e-06\\
599.75	2.34566889159918e-06\\
599.76	2.16769245801711e-06\\
599.77	1.99620352829381e-06\\
599.78	1.83126680087728e-06\\
599.79	1.67294763142416e-06\\
599.8	1.52131203949059e-06\\
599.81	1.37642671532712e-06\\
599.82	1.23835902672391e-06\\
599.83	1.10717702595832e-06\\
599.84	9.82949456815319e-07\\
599.85	8.65745761698122e-07\\
599.86	7.55636088823827e-07\\
599.87	6.52691299497105e-07\\
599.88	5.56982975498388e-07\\
599.89	4.6858342654145e-07\\
599.9	3.87565697841999e-07\\
599.91	3.14003577771282e-07\\
599.92	2.47971605622441e-07\\
599.93	1.89545079472275e-07\\
599.94	1.38800064147099e-07\\
599.95	9.58133993030769e-08\\
599.96	6.06627076158578e-08\\
599.97	3.34264030846937e-08\\
599.98	1.4183699454523e-08\\
599.99	3.01461875948372e-09\\
600	0\\
};
\addplot [color=mycolor17,solid,forget plot]
  table[row sep=crcr]{%
0.01	0.01\\
1.01	0.01\\
2.01	0.01\\
3.01	0.01\\
4.01	0.01\\
5.01	0.01\\
6.01	0.01\\
7.01	0.01\\
8.01	0.01\\
9.01	0.01\\
10.01	0.01\\
11.01	0.01\\
12.01	0.01\\
13.01	0.01\\
14.01	0.01\\
15.01	0.01\\
16.01	0.01\\
17.01	0.01\\
18.01	0.01\\
19.01	0.01\\
20.01	0.01\\
21.01	0.01\\
22.01	0.01\\
23.01	0.01\\
24.01	0.01\\
25.01	0.01\\
26.01	0.01\\
27.01	0.01\\
28.01	0.01\\
29.01	0.01\\
30.01	0.01\\
31.01	0.01\\
32.01	0.01\\
33.01	0.01\\
34.01	0.01\\
35.01	0.01\\
36.01	0.01\\
37.01	0.01\\
38.01	0.01\\
39.01	0.01\\
40.01	0.01\\
41.01	0.01\\
42.01	0.01\\
43.01	0.01\\
44.01	0.01\\
45.01	0.01\\
46.01	0.01\\
47.01	0.01\\
48.01	0.01\\
49.01	0.01\\
50.01	0.01\\
51.01	0.01\\
52.01	0.01\\
53.01	0.01\\
54.01	0.01\\
55.01	0.01\\
56.01	0.01\\
57.01	0.01\\
58.01	0.01\\
59.01	0.01\\
60.01	0.01\\
61.01	0.01\\
62.01	0.01\\
63.01	0.01\\
64.01	0.01\\
65.01	0.01\\
66.01	0.01\\
67.01	0.01\\
68.01	0.01\\
69.01	0.01\\
70.01	0.01\\
71.01	0.01\\
72.01	0.01\\
73.01	0.01\\
74.01	0.01\\
75.01	0.01\\
76.01	0.01\\
77.01	0.01\\
78.01	0.01\\
79.01	0.01\\
80.01	0.01\\
81.01	0.01\\
82.01	0.01\\
83.01	0.01\\
84.01	0.01\\
85.01	0.01\\
86.01	0.01\\
87.01	0.01\\
88.01	0.01\\
89.01	0.01\\
90.01	0.01\\
91.01	0.01\\
92.01	0.01\\
93.01	0.01\\
94.01	0.01\\
95.01	0.01\\
96.01	0.01\\
97.01	0.01\\
98.01	0.01\\
99.01	0.01\\
100.01	0.01\\
101.01	0.01\\
102.01	0.01\\
103.01	0.01\\
104.01	0.01\\
105.01	0.01\\
106.01	0.01\\
107.01	0.01\\
108.01	0.01\\
109.01	0.01\\
110.01	0.01\\
111.01	0.01\\
112.01	0.01\\
113.01	0.01\\
114.01	0.01\\
115.01	0.01\\
116.01	0.01\\
117.01	0.01\\
118.01	0.01\\
119.01	0.01\\
120.01	0.01\\
121.01	0.01\\
122.01	0.01\\
123.01	0.01\\
124.01	0.01\\
125.01	0.01\\
126.01	0.01\\
127.01	0.01\\
128.01	0.01\\
129.01	0.01\\
130.01	0.01\\
131.01	0.01\\
132.01	0.01\\
133.01	0.01\\
134.01	0.01\\
135.01	0.01\\
136.01	0.01\\
137.01	0.01\\
138.01	0.01\\
139.01	0.01\\
140.01	0.01\\
141.01	0.01\\
142.01	0.01\\
143.01	0.01\\
144.01	0.01\\
145.01	0.01\\
146.01	0.01\\
147.01	0.01\\
148.01	0.01\\
149.01	0.01\\
150.01	0.01\\
151.01	0.01\\
152.01	0.01\\
153.01	0.01\\
154.01	0.01\\
155.01	0.01\\
156.01	0.01\\
157.01	0.01\\
158.01	0.01\\
159.01	0.01\\
160.01	0.01\\
161.01	0.01\\
162.01	0.01\\
163.01	0.01\\
164.01	0.01\\
165.01	0.01\\
166.01	0.01\\
167.01	0.01\\
168.01	0.01\\
169.01	0.01\\
170.01	0.01\\
171.01	0.01\\
172.01	0.01\\
173.01	0.01\\
174.01	0.01\\
175.01	0.01\\
176.01	0.01\\
177.01	0.01\\
178.01	0.01\\
179.01	0.01\\
180.01	0.01\\
181.01	0.01\\
182.01	0.01\\
183.01	0.01\\
184.01	0.01\\
185.01	0.01\\
186.01	0.01\\
187.01	0.01\\
188.01	0.01\\
189.01	0.01\\
190.01	0.01\\
191.01	0.01\\
192.01	0.01\\
193.01	0.01\\
194.01	0.01\\
195.01	0.01\\
196.01	0.01\\
197.01	0.01\\
198.01	0.01\\
199.01	0.01\\
200.01	0.01\\
201.01	0.01\\
202.01	0.01\\
203.01	0.01\\
204.01	0.01\\
205.01	0.01\\
206.01	0.01\\
207.01	0.01\\
208.01	0.01\\
209.01	0.01\\
210.01	0.01\\
211.01	0.01\\
212.01	0.01\\
213.01	0.01\\
214.01	0.01\\
215.01	0.01\\
216.01	0.01\\
217.01	0.01\\
218.01	0.01\\
219.01	0.01\\
220.01	0.01\\
221.01	0.01\\
222.01	0.01\\
223.01	0.01\\
224.01	0.01\\
225.01	0.01\\
226.01	0.01\\
227.01	0.01\\
228.01	0.01\\
229.01	0.01\\
230.01	0.01\\
231.01	0.01\\
232.01	0.01\\
233.01	0.01\\
234.01	0.01\\
235.01	0.01\\
236.01	0.01\\
237.01	0.01\\
238.01	0.01\\
239.01	0.01\\
240.01	0.01\\
241.01	0.01\\
242.01	0.01\\
243.01	0.01\\
244.01	0.01\\
245.01	0.01\\
246.01	0.01\\
247.01	0.01\\
248.01	0.01\\
249.01	0.01\\
250.01	0.01\\
251.01	0.01\\
252.01	0.01\\
253.01	0.01\\
254.01	0.01\\
255.01	0.01\\
256.01	0.01\\
257.01	0.01\\
258.01	0.01\\
259.01	0.01\\
260.01	0.01\\
261.01	0.01\\
262.01	0.01\\
263.01	0.01\\
264.01	0.01\\
265.01	0.01\\
266.01	0.01\\
267.01	0.01\\
268.01	0.01\\
269.01	0.01\\
270.01	0.01\\
271.01	0.01\\
272.01	0.01\\
273.01	0.01\\
274.01	0.01\\
275.01	0.01\\
276.01	0.01\\
277.01	0.01\\
278.01	0.01\\
279.01	0.01\\
280.01	0.01\\
281.01	0.01\\
282.01	0.01\\
283.01	0.01\\
284.01	0.01\\
285.01	0.01\\
286.01	0.01\\
287.01	0.01\\
288.01	0.01\\
289.01	0.01\\
290.01	0.01\\
291.01	0.01\\
292.01	0.01\\
293.01	0.01\\
294.01	0.01\\
295.01	0.01\\
296.01	0.01\\
297.01	0.01\\
298.01	0.01\\
299.01	0.01\\
300.01	0.01\\
301.01	0.01\\
302.01	0.01\\
303.01	0.01\\
304.01	0.01\\
305.01	0.01\\
306.01	0.01\\
307.01	0.01\\
308.01	0.01\\
309.01	0.01\\
310.01	0.01\\
311.01	0.01\\
312.01	0.01\\
313.01	0.01\\
314.01	0.01\\
315.01	0.01\\
316.01	0.01\\
317.01	0.01\\
318.01	0.01\\
319.01	0.01\\
320.01	0.01\\
321.01	0.01\\
322.01	0.01\\
323.01	0.01\\
324.01	0.01\\
325.01	0.01\\
326.01	0.01\\
327.01	0.01\\
328.01	0.01\\
329.01	0.01\\
330.01	0.01\\
331.01	0.01\\
332.01	0.01\\
333.01	0.01\\
334.01	0.01\\
335.01	0.01\\
336.01	0.01\\
337.01	0.01\\
338.01	0.01\\
339.01	0.01\\
340.01	0.01\\
341.01	0.01\\
342.01	0.01\\
343.01	0.01\\
344.01	0.01\\
345.01	0.01\\
346.01	0.01\\
347.01	0.01\\
348.01	0.01\\
349.01	0.01\\
350.01	0.01\\
351.01	0.01\\
352.01	0.01\\
353.01	0.01\\
354.01	0.01\\
355.01	0.01\\
356.01	0.01\\
357.01	0.01\\
358.01	0.01\\
359.01	0.01\\
360.01	0.01\\
361.01	0.01\\
362.01	0.01\\
363.01	0.01\\
364.01	0.01\\
365.01	0.01\\
366.01	0.01\\
367.01	0.01\\
368.01	0.01\\
369.01	0.01\\
370.01	0.01\\
371.01	0.01\\
372.01	0.01\\
373.01	0.01\\
374.01	0.01\\
375.01	0.01\\
376.01	0.01\\
377.01	0.01\\
378.01	0.01\\
379.01	0.01\\
380.01	0.01\\
381.01	0.01\\
382.01	0.01\\
383.01	0.01\\
384.01	0.01\\
385.01	0.01\\
386.01	0.01\\
387.01	0.01\\
388.01	0.01\\
389.01	0.01\\
390.01	0.01\\
391.01	0.01\\
392.01	0.01\\
393.01	0.01\\
394.01	0.01\\
395.01	0.01\\
396.01	0.01\\
397.01	0.01\\
398.01	0.01\\
399.01	0.01\\
400.01	0.01\\
401.01	0.01\\
402.01	0.01\\
403.01	0.01\\
404.01	0.01\\
405.01	0.01\\
406.01	0.01\\
407.01	0.01\\
408.01	0.01\\
409.01	0.01\\
410.01	0.01\\
411.01	0.01\\
412.01	0.01\\
413.01	0.01\\
414.01	0.01\\
415.01	0.01\\
416.01	0.01\\
417.01	0.01\\
418.01	0.01\\
419.01	0.01\\
420.01	0.01\\
421.01	0.01\\
422.01	0.01\\
423.01	0.01\\
424.01	0.01\\
425.01	0.01\\
426.01	0.01\\
427.01	0.01\\
428.01	0.01\\
429.01	0.01\\
430.01	0.01\\
431.01	0.01\\
432.01	0.01\\
433.01	0.01\\
434.01	0.01\\
435.01	0.01\\
436.01	0.01\\
437.01	0.01\\
438.01	0.01\\
439.01	0.01\\
440.01	0.01\\
441.01	0.01\\
442.01	0.01\\
443.01	0.01\\
444.01	0.01\\
445.01	0.01\\
446.01	0.01\\
447.01	0.01\\
448.01	0.01\\
449.01	0.01\\
450.01	0.01\\
451.01	0.01\\
452.01	0.01\\
453.01	0.01\\
454.01	0.01\\
455.01	0.01\\
456.01	0.01\\
457.01	0.01\\
458.01	0.01\\
459.01	0.01\\
460.01	0.01\\
461.01	0.01\\
462.01	0.01\\
463.01	0.01\\
464.01	0.01\\
465.01	0.01\\
466.01	0.01\\
467.01	0.01\\
468.01	0.01\\
469.01	0.01\\
470.01	0.01\\
471.01	0.01\\
472.01	0.01\\
473.01	0.01\\
474.01	0.01\\
475.01	0.01\\
476.01	0.01\\
477.01	0.01\\
478.01	0.01\\
479.01	0.01\\
480.01	0.01\\
481.01	0.01\\
482.01	0.01\\
483.01	0.01\\
484.01	0.01\\
485.01	0.01\\
486.01	0.01\\
487.01	0.01\\
488.01	0.01\\
489.01	0.01\\
490.01	0.01\\
491.01	0.01\\
492.01	0.01\\
493.01	0.01\\
494.01	0.01\\
495.01	0.01\\
496.01	0.01\\
497.01	0.01\\
498.01	0.01\\
499.01	0.01\\
500.01	0.01\\
501.01	0.01\\
502.01	0.01\\
503.01	0.01\\
504.01	0.01\\
505.01	0.01\\
506.01	0.01\\
507.01	0.01\\
508.01	0.01\\
509.01	0.01\\
510.01	0.01\\
511.01	0.01\\
512.01	0.01\\
513.01	0.01\\
514.01	0.01\\
515.01	0.01\\
516.01	0.01\\
517.01	0.01\\
518.01	0.01\\
519.01	0.01\\
520.01	0.01\\
521.01	0.01\\
522.01	0.01\\
523.01	0.01\\
524.01	0.01\\
525.01	0.01\\
526.01	0.01\\
527.01	0.0099241290970491\\
528.01	0.0098175255165026\\
529.01	0.0097068930893769\\
530.01	0.00959196270396107\\
531.01	0.00947243515181103\\
532.01	0.00934797596358879\\
533.01	0.00921820807741072\\
534.01	0.00908269772191878\\
535.01	0.00894095105251903\\
536.01	0.00879240950820449\\
537.01	0.00863643967948476\\
538.01	0.00847232140769048\\
539.01	0.00829923383611116\\
540.01	0.00811623904417037\\
541.01	0.00792226282616251\\
542.01	0.00771606180810179\\
543.01	0.00749606450831156\\
544.01	0.00726030556795162\\
545.01	0.0070079519472063\\
546.01	0.00674388486960165\\
547.01	0.00646787322976985\\
548.01	0.00617876222036522\\
549.01	0.00587518142076962\\
550.01	0.00559746032953064\\
551.01	0.00545298715323179\\
552.01	0.00530500792545318\\
553.01	0.00515390011393403\\
554.01	0.00500019249536979\\
555.01	0.0048446042748905\\
556.01	0.00468808397530697\\
557.01	0.00453190880521706\\
558.01	0.00437781468838097\\
559.01	0.00422814498614577\\
560.01	0.00408450061202286\\
561.01	0.00393959397258884\\
562.01	0.0037924112109357\\
563.01	0.00364377063441716\\
564.01	0.00349484266244423\\
565.01	0.00334828400578513\\
566.01	0.0032068365243211\\
567.01	0.00307305339271171\\
568.01	0.00294563612557173\\
569.01	0.00282111130465503\\
570.01	0.00270007470687016\\
571.01	0.00258295637848159\\
572.01	0.00246985659109791\\
573.01	0.00236031264984525\\
574.01	0.00225264487338503\\
575.01	0.00214649048949318\\
576.01	0.00204212467871798\\
577.01	0.00193976321903908\\
578.01	0.00183953410827614\\
579.01	0.00174144548087779\\
580.01	0.00164535244959273\\
581.01	0.00155092827496037\\
582.01	0.00145768949973365\\
583.01	0.0013652994855999\\
584.01	0.00127372346543224\\
585.01	0.0011829917503887\\
586.01	0.00109316523826044\\
587.01	0.00100429568118757\\
588.01	0.00091642184966745\\
589.01	0.000829576198374293\\
590.01	0.000743791048770606\\
591.01	0.000659086383332572\\
592.01	0.000575443770475836\\
593.01	0.000492800020771381\\
594.01	0.000411047558901598\\
595.01	0.000330038786353392\\
596.01	0.000249595277727267\\
597.01	0.000169522162970183\\
598.01	9.18966740472912e-05\\
599.01	2.94706190942413e-05\\
599.02	2.89609613509639e-05\\
599.03	2.84543437073619e-05\\
599.04	2.79507956110585e-05\\
599.05	2.74503468013707e-05\\
599.06	2.69530273121939e-05\\
599.07	2.64588674749062e-05\\
599.08	2.59678979212998e-05\\
599.09	2.54801495865476e-05\\
599.1	2.49956537121988e-05\\
599.11	2.45144418491942e-05\\
599.12	2.40365458609258e-05\\
599.13	2.35619979263146e-05\\
599.14	2.30908305429295e-05\\
599.15	2.26230765301254e-05\\
599.16	2.21587690322159e-05\\
599.17	2.1697941521695e-05\\
599.18	2.12406278024529e-05\\
599.19	2.07868620130686e-05\\
599.2	2.03366786300958e-05\\
599.21	1.98901124714036e-05\\
599.22	1.9447198699547e-05\\
599.23	1.90079728251708e-05\\
599.24	1.85724707104321e-05\\
599.25	1.81407285724869e-05\\
599.26	1.77127848684563e-05\\
599.27	1.72886796869724e-05\\
599.28	1.68684535171004e-05\\
599.29	1.64521472522886e-05\\
599.3	1.60398021943637e-05\\
599.31	1.56314600575445e-05\\
599.32	1.52271629725205e-05\\
599.33	1.48269534905581e-05\\
599.34	1.44308745876499e-05\\
599.35	1.40389696686987e-05\\
599.36	1.36512825717559e-05\\
599.37	1.32678575722866e-05\\
599.38	1.28887393874896e-05\\
599.39	1.25139731806528e-05\\
599.4	1.21436045655531e-05\\
599.41	1.17776796109033e-05\\
599.42	1.14162448448338e-05\\
599.43	1.10593472594303e-05\\
599.44	1.07070343153062e-05\\
599.45	1.03593539462262e-05\\
599.46	1.00163545637792e-05\\
599.47	9.67808506207625e-06\\
599.48	9.34459482253812e-06\\
599.49	9.01593371867813e-06\\
599.5	8.69215212097318e-06\\
599.51	8.37330090176433e-06\\
599.52	8.05943144020597e-06\\
599.53	7.75059562726881e-06\\
599.54	7.44684587079478e-06\\
599.55	7.14823510058853e-06\\
599.56	6.85481677357644e-06\\
599.57	6.56664487900041e-06\\
599.58	6.28377394367753e-06\\
599.59	6.00625903730313e-06\\
599.6	5.73415577780415e-06\\
599.61	5.46752033675491e-06\\
599.62	5.20640944483286e-06\\
599.63	4.9508803973454e-06\\
599.64	4.70099105979137e-06\\
599.65	4.45679987349373e-06\\
599.66	4.21836586128252e-06\\
599.67	3.98574863322981e-06\\
599.68	3.75900839244374e-06\\
599.69	3.53820594092488e-06\\
599.7	3.32340268546956e-06\\
599.71	3.11466064364246e-06\\
599.72	2.9120424498031e-06\\
599.73	2.71561136118426e-06\\
599.74	2.52543126405373e-06\\
599.75	2.34156667990038e-06\\
599.76	2.16408277171551e-06\\
599.77	1.99304535031947e-06\\
599.78	1.82852088075058e-06\\
599.79	1.67057648871489e-06\\
599.8	1.51927996711911e-06\\
599.81	1.37469978262958e-06\\
599.82	1.23690508234062e-06\\
599.83	1.10596570046875e-06\\
599.84	9.81952165138994e-07\\
599.85	8.64935705228304e-07\\
599.86	7.54988257262862e-07\\
599.87	6.5218247242288e-07\\
599.88	5.56591723556085e-07\\
599.89	4.6829011231958e-07\\
599.9	3.87352476345984e-07\\
599.91	3.13854396508453e-07\\
599.92	2.47872204241217e-07\\
599.93	1.89482988934703e-07\\
599.94	1.38764605403519e-07\\
599.95	9.57956814255645e-08\\
599.96	6.06556253574669e-08\\
599.97	3.34246338211386e-08\\
599.98	1.4183699454523e-08\\
599.99	3.01461875948372e-09\\
600	0\\
};
\addplot [color=mycolor18,solid,forget plot]
  table[row sep=crcr]{%
0.01	0.01\\
1.01	0.01\\
2.01	0.01\\
3.01	0.01\\
4.01	0.01\\
5.01	0.01\\
6.01	0.01\\
7.01	0.01\\
8.01	0.01\\
9.01	0.01\\
10.01	0.01\\
11.01	0.01\\
12.01	0.01\\
13.01	0.01\\
14.01	0.01\\
15.01	0.01\\
16.01	0.01\\
17.01	0.01\\
18.01	0.01\\
19.01	0.01\\
20.01	0.01\\
21.01	0.01\\
22.01	0.01\\
23.01	0.01\\
24.01	0.01\\
25.01	0.01\\
26.01	0.01\\
27.01	0.01\\
28.01	0.01\\
29.01	0.01\\
30.01	0.01\\
31.01	0.01\\
32.01	0.01\\
33.01	0.01\\
34.01	0.01\\
35.01	0.01\\
36.01	0.01\\
37.01	0.01\\
38.01	0.01\\
39.01	0.01\\
40.01	0.01\\
41.01	0.01\\
42.01	0.01\\
43.01	0.01\\
44.01	0.01\\
45.01	0.01\\
46.01	0.01\\
47.01	0.01\\
48.01	0.01\\
49.01	0.01\\
50.01	0.01\\
51.01	0.01\\
52.01	0.01\\
53.01	0.01\\
54.01	0.01\\
55.01	0.01\\
56.01	0.01\\
57.01	0.01\\
58.01	0.01\\
59.01	0.01\\
60.01	0.01\\
61.01	0.01\\
62.01	0.01\\
63.01	0.01\\
64.01	0.01\\
65.01	0.01\\
66.01	0.01\\
67.01	0.01\\
68.01	0.01\\
69.01	0.01\\
70.01	0.01\\
71.01	0.01\\
72.01	0.01\\
73.01	0.01\\
74.01	0.01\\
75.01	0.01\\
76.01	0.01\\
77.01	0.01\\
78.01	0.01\\
79.01	0.01\\
80.01	0.01\\
81.01	0.01\\
82.01	0.01\\
83.01	0.01\\
84.01	0.01\\
85.01	0.01\\
86.01	0.01\\
87.01	0.01\\
88.01	0.01\\
89.01	0.01\\
90.01	0.01\\
91.01	0.01\\
92.01	0.01\\
93.01	0.01\\
94.01	0.01\\
95.01	0.01\\
96.01	0.01\\
97.01	0.01\\
98.01	0.01\\
99.01	0.01\\
100.01	0.01\\
101.01	0.01\\
102.01	0.01\\
103.01	0.01\\
104.01	0.01\\
105.01	0.01\\
106.01	0.01\\
107.01	0.01\\
108.01	0.01\\
109.01	0.01\\
110.01	0.01\\
111.01	0.01\\
112.01	0.01\\
113.01	0.01\\
114.01	0.01\\
115.01	0.01\\
116.01	0.01\\
117.01	0.01\\
118.01	0.01\\
119.01	0.01\\
120.01	0.01\\
121.01	0.01\\
122.01	0.01\\
123.01	0.01\\
124.01	0.01\\
125.01	0.01\\
126.01	0.01\\
127.01	0.01\\
128.01	0.01\\
129.01	0.01\\
130.01	0.01\\
131.01	0.01\\
132.01	0.01\\
133.01	0.01\\
134.01	0.01\\
135.01	0.01\\
136.01	0.01\\
137.01	0.01\\
138.01	0.01\\
139.01	0.01\\
140.01	0.01\\
141.01	0.01\\
142.01	0.01\\
143.01	0.01\\
144.01	0.01\\
145.01	0.01\\
146.01	0.01\\
147.01	0.01\\
148.01	0.01\\
149.01	0.01\\
150.01	0.01\\
151.01	0.01\\
152.01	0.01\\
153.01	0.01\\
154.01	0.01\\
155.01	0.01\\
156.01	0.01\\
157.01	0.01\\
158.01	0.01\\
159.01	0.01\\
160.01	0.01\\
161.01	0.01\\
162.01	0.01\\
163.01	0.01\\
164.01	0.01\\
165.01	0.01\\
166.01	0.01\\
167.01	0.01\\
168.01	0.01\\
169.01	0.01\\
170.01	0.01\\
171.01	0.01\\
172.01	0.01\\
173.01	0.01\\
174.01	0.01\\
175.01	0.01\\
176.01	0.01\\
177.01	0.01\\
178.01	0.01\\
179.01	0.01\\
180.01	0.01\\
181.01	0.01\\
182.01	0.01\\
183.01	0.01\\
184.01	0.01\\
185.01	0.01\\
186.01	0.01\\
187.01	0.01\\
188.01	0.01\\
189.01	0.01\\
190.01	0.01\\
191.01	0.01\\
192.01	0.01\\
193.01	0.01\\
194.01	0.01\\
195.01	0.01\\
196.01	0.01\\
197.01	0.01\\
198.01	0.01\\
199.01	0.01\\
200.01	0.01\\
201.01	0.01\\
202.01	0.01\\
203.01	0.01\\
204.01	0.01\\
205.01	0.01\\
206.01	0.01\\
207.01	0.01\\
208.01	0.01\\
209.01	0.01\\
210.01	0.01\\
211.01	0.01\\
212.01	0.01\\
213.01	0.01\\
214.01	0.01\\
215.01	0.01\\
216.01	0.01\\
217.01	0.01\\
218.01	0.01\\
219.01	0.01\\
220.01	0.01\\
221.01	0.01\\
222.01	0.01\\
223.01	0.01\\
224.01	0.01\\
225.01	0.01\\
226.01	0.01\\
227.01	0.01\\
228.01	0.01\\
229.01	0.01\\
230.01	0.01\\
231.01	0.01\\
232.01	0.01\\
233.01	0.01\\
234.01	0.01\\
235.01	0.01\\
236.01	0.01\\
237.01	0.01\\
238.01	0.01\\
239.01	0.01\\
240.01	0.01\\
241.01	0.01\\
242.01	0.01\\
243.01	0.01\\
244.01	0.01\\
245.01	0.01\\
246.01	0.01\\
247.01	0.01\\
248.01	0.01\\
249.01	0.01\\
250.01	0.01\\
251.01	0.01\\
252.01	0.01\\
253.01	0.01\\
254.01	0.01\\
255.01	0.01\\
256.01	0.01\\
257.01	0.01\\
258.01	0.01\\
259.01	0.01\\
260.01	0.01\\
261.01	0.01\\
262.01	0.01\\
263.01	0.01\\
264.01	0.01\\
265.01	0.01\\
266.01	0.01\\
267.01	0.01\\
268.01	0.01\\
269.01	0.01\\
270.01	0.01\\
271.01	0.01\\
272.01	0.01\\
273.01	0.01\\
274.01	0.01\\
275.01	0.01\\
276.01	0.01\\
277.01	0.01\\
278.01	0.01\\
279.01	0.01\\
280.01	0.01\\
281.01	0.01\\
282.01	0.01\\
283.01	0.01\\
284.01	0.01\\
285.01	0.01\\
286.01	0.01\\
287.01	0.01\\
288.01	0.01\\
289.01	0.01\\
290.01	0.01\\
291.01	0.01\\
292.01	0.01\\
293.01	0.01\\
294.01	0.01\\
295.01	0.01\\
296.01	0.01\\
297.01	0.01\\
298.01	0.01\\
299.01	0.01\\
300.01	0.01\\
301.01	0.01\\
302.01	0.01\\
303.01	0.01\\
304.01	0.01\\
305.01	0.01\\
306.01	0.01\\
307.01	0.01\\
308.01	0.01\\
309.01	0.01\\
310.01	0.01\\
311.01	0.01\\
312.01	0.01\\
313.01	0.01\\
314.01	0.01\\
315.01	0.01\\
316.01	0.01\\
317.01	0.01\\
318.01	0.01\\
319.01	0.01\\
320.01	0.01\\
321.01	0.01\\
322.01	0.01\\
323.01	0.01\\
324.01	0.01\\
325.01	0.01\\
326.01	0.01\\
327.01	0.01\\
328.01	0.01\\
329.01	0.01\\
330.01	0.01\\
331.01	0.01\\
332.01	0.01\\
333.01	0.01\\
334.01	0.01\\
335.01	0.01\\
336.01	0.01\\
337.01	0.01\\
338.01	0.01\\
339.01	0.01\\
340.01	0.01\\
341.01	0.01\\
342.01	0.01\\
343.01	0.01\\
344.01	0.01\\
345.01	0.01\\
346.01	0.01\\
347.01	0.01\\
348.01	0.01\\
349.01	0.01\\
350.01	0.01\\
351.01	0.01\\
352.01	0.01\\
353.01	0.01\\
354.01	0.01\\
355.01	0.01\\
356.01	0.01\\
357.01	0.01\\
358.01	0.01\\
359.01	0.01\\
360.01	0.01\\
361.01	0.01\\
362.01	0.01\\
363.01	0.01\\
364.01	0.01\\
365.01	0.01\\
366.01	0.01\\
367.01	0.01\\
368.01	0.01\\
369.01	0.01\\
370.01	0.01\\
371.01	0.01\\
372.01	0.01\\
373.01	0.01\\
374.01	0.01\\
375.01	0.01\\
376.01	0.01\\
377.01	0.01\\
378.01	0.01\\
379.01	0.01\\
380.01	0.01\\
381.01	0.01\\
382.01	0.01\\
383.01	0.01\\
384.01	0.01\\
385.01	0.01\\
386.01	0.01\\
387.01	0.01\\
388.01	0.01\\
389.01	0.01\\
390.01	0.01\\
391.01	0.01\\
392.01	0.01\\
393.01	0.01\\
394.01	0.01\\
395.01	0.01\\
396.01	0.01\\
397.01	0.01\\
398.01	0.01\\
399.01	0.01\\
400.01	0.01\\
401.01	0.01\\
402.01	0.01\\
403.01	0.01\\
404.01	0.01\\
405.01	0.01\\
406.01	0.01\\
407.01	0.01\\
408.01	0.01\\
409.01	0.01\\
410.01	0.01\\
411.01	0.01\\
412.01	0.01\\
413.01	0.01\\
414.01	0.01\\
415.01	0.01\\
416.01	0.01\\
417.01	0.01\\
418.01	0.01\\
419.01	0.01\\
420.01	0.01\\
421.01	0.01\\
422.01	0.01\\
423.01	0.01\\
424.01	0.01\\
425.01	0.01\\
426.01	0.01\\
427.01	0.01\\
428.01	0.01\\
429.01	0.01\\
430.01	0.01\\
431.01	0.01\\
432.01	0.01\\
433.01	0.01\\
434.01	0.01\\
435.01	0.01\\
436.01	0.01\\
437.01	0.01\\
438.01	0.01\\
439.01	0.01\\
440.01	0.01\\
441.01	0.01\\
442.01	0.01\\
443.01	0.01\\
444.01	0.01\\
445.01	0.01\\
446.01	0.01\\
447.01	0.01\\
448.01	0.01\\
449.01	0.01\\
450.01	0.01\\
451.01	0.01\\
452.01	0.01\\
453.01	0.01\\
454.01	0.01\\
455.01	0.01\\
456.01	0.01\\
457.01	0.01\\
458.01	0.01\\
459.01	0.01\\
460.01	0.01\\
461.01	0.01\\
462.01	0.01\\
463.01	0.01\\
464.01	0.01\\
465.01	0.01\\
466.01	0.01\\
467.01	0.01\\
468.01	0.01\\
469.01	0.01\\
470.01	0.01\\
471.01	0.01\\
472.01	0.01\\
473.01	0.01\\
474.01	0.01\\
475.01	0.01\\
476.01	0.01\\
477.01	0.01\\
478.01	0.01\\
479.01	0.01\\
480.01	0.01\\
481.01	0.01\\
482.01	0.00997587649458391\\
483.01	0.00994281436826373\\
484.01	0.00990865576868689\\
485.01	0.00987335571535595\\
486.01	0.00983686864441155\\
487.01	0.00979914916206342\\
488.01	0.00976015312757781\\
489.01	0.0097198391823294\\
490.01	0.0096781708806394\\
491.01	0.0096351196302989\\
492.01	0.00959066872002735\\
493.01	0.00954481854877986\\
494.01	0.00949756230674207\\
495.01	0.00944884687152364\\
496.01	0.00939861087884548\\
497.01	0.00934679062273646\\
498.01	0.00929332052445587\\
499.01	0.0092381338739154\\
500.01	0.00918116394599725\\
501.01	0.00912234562935636\\
502.01	0.00906161775233871\\
503.01	0.00899892635353021\\
504.01	0.00893422402364727\\
505.01	0.00886742921998204\\
506.01	0.00879843256617422\\
507.01	0.00872711390830633\\
508.01	0.00865334091708237\\
509.01	0.00857696723504866\\
510.01	0.00849783027010523\\
511.01	0.00841574855662225\\
512.01	0.00833051858576653\\
513.01	0.00824191098148276\\
514.01	0.00814966586632992\\
515.01	0.00805348721986368\\
516.01	0.00795303597859557\\
517.01	0.00784792155686998\\
518.01	0.00773769137706917\\
519.01	0.00762181787816733\\
520.01	0.00749968231209421\\
521.01	0.00737055443836958\\
522.01	0.00723356698045552\\
523.01	0.00708768334921067\\
524.01	0.00693165666092587\\
525.01	0.00676423745727685\\
526.01	0.00658790083961235\\
527.01	0.00647971681111041\\
528.01	0.00639593571072038\\
529.01	0.00630965465279673\\
530.01	0.00622083657043316\\
531.01	0.00612945588072651\\
532.01	0.00603550315539981\\
533.01	0.00593899257231927\\
534.01	0.00583997738401568\\
535.01	0.00573855520496804\\
536.01	0.00563487457077089\\
537.01	0.00552914787384191\\
538.01	0.00542166736493526\\
539.01	0.00531282489136661\\
540.01	0.00520313623617997\\
541.01	0.00509327112131359\\
542.01	0.00498410056830288\\
543.01	0.00487687893575851\\
544.01	0.00477334731136774\\
545.01	0.00467422638434592\\
546.01	0.0045743314764834\\
547.01	0.00447337191168162\\
548.01	0.00437198627604522\\
549.01	0.00427105996371826\\
550.01	0.00417166426615998\\
551.01	0.00407132678206081\\
552.01	0.00396880225878627\\
553.01	0.00386438111845274\\
554.01	0.00375849222447324\\
555.01	0.00365203488305964\\
556.01	0.0035466544291015\\
557.01	0.00344282389172102\\
558.01	0.00334084018973639\\
559.01	0.00324086386752923\\
560.01	0.00314283492651795\\
561.01	0.00304683398660745\\
562.01	0.00295326966246949\\
563.01	0.00286249569317403\\
564.01	0.00277470624413143\\
565.01	0.00268886161041842\\
566.01	0.00260428765967366\\
567.01	0.00252097709498086\\
568.01	0.00243880615864478\\
569.01	0.00235772436845609\\
570.01	0.00227765771045678\\
571.01	0.00219845346857509\\
572.01	0.00211989051998918\\
573.01	0.0020417157060724\\
574.01	0.00196376477038438\\
575.01	0.0018860263484256\\
576.01	0.00180848634643038\\
577.01	0.00173110876709557\\
578.01	0.00165383891558031\\
579.01	0.00157660995977223\\
580.01	0.00149935332576446\\
581.01	0.00142201270423878\\
582.01	0.00134455890324395\\
583.01	0.00126698852963796\\
584.01	0.00118930562924654\\
585.01	0.00111151429061311\\
586.01	0.00103361667856622\\
587.01	0.000955612541206028\\
588.01	0.000877500028226702\\
589.01	0.000799275845239126\\
590.01	0.000720934182363075\\
591.01	0.000642465092883062\\
592.01	0.000563855076711632\\
593.01	0.00048508951620752\\
594.01	0.000406155478895408\\
595.01	0.000327044352021353\\
596.01	0.000247753662207873\\
597.01	0.000168287229632226\\
598.01	9.17536481397484e-05\\
599.01	2.94669794711974e-05\\
599.02	2.89574678339153e-05\\
599.03	2.84509918799507e-05\\
599.04	2.79475811456025e-05\\
599.05	2.74472654580235e-05\\
599.06	2.69500749380813e-05\\
599.07	2.64560400032848e-05\\
599.08	2.59651913706981e-05\\
599.09	2.54775600599076e-05\\
599.1	2.49931773960189e-05\\
599.11	2.45120750126723e-05\\
599.12	2.40342848550935e-05\\
599.13	2.35598391831755e-05\\
599.14	2.30887705745895e-05\\
599.15	2.26211119279309e-05\\
599.16	2.21568964658882e-05\\
599.17	2.16961577384437e-05\\
599.18	2.12389296261259e-05\\
599.19	2.07852463432537e-05\\
599.2	2.03351424412582e-05\\
599.21	1.98886528120085e-05\\
599.22	1.94458126911764e-05\\
599.23	1.90066576616355e-05\\
599.24	1.85712236569005e-05\\
599.25	1.81395469645886e-05\\
599.26	1.77116661135811e-05\\
599.27	1.72876212607056e-05\\
599.28	1.68674529623658e-05\\
599.29	1.64512021784951e-05\\
599.3	1.60389102765428e-05\\
599.31	1.56306190355025e-05\\
599.32	1.52263706499779e-05\\
599.33	1.48262077342837e-05\\
599.34	1.44301733266006e-05\\
599.35	1.40383108931594e-05\\
599.36	1.36506643324697e-05\\
599.37	1.3267277979595e-05\\
599.38	1.28881966104562e-05\\
599.39	1.25134654462011e-05\\
599.4	1.21431301575905e-05\\
599.41	1.17772368694487e-05\\
599.42	1.14158321651448e-05\\
599.43	1.10589630911238e-05\\
599.44	1.07066771614876e-05\\
599.45	1.03590223626028e-05\\
599.46	1.00160471577775e-05\\
599.47	9.67780049196745e-06\\
599.48	9.3443317965361e-06\\
599.49	9.01569099405995e-06\\
599.5	8.6919285031805e-06\\
599.51	8.37309524350481e-06\\
599.52	8.05924264055125e-06\\
599.53	7.75042263075933e-06\\
599.54	7.44668766651697e-06\\
599.55	7.14809072127447e-06\\
599.56	6.85468529468269e-06\\
599.57	6.56652541779733e-06\\
599.58	6.28366565832289e-06\\
599.59	6.00616112590886e-06\\
599.6	5.73406747752039e-06\\
599.61	5.46744092282461e-06\\
599.62	5.20633822966195e-06\\
599.63	4.95081672955734e-06\\
599.64	4.70093432328351e-06\\
599.65	4.45674948649534e-06\\
599.66	4.21832127539025e-06\\
599.67	3.98570933245884e-06\\
599.68	3.75897389226325e-06\\
599.69	3.53817578729702e-06\\
599.7	3.32337645388148e-06\\
599.71	3.11463793812965e-06\\
599.72	2.9120229019762e-06\\
599.73	2.71559462925938e-06\\
599.74	2.52541703184977e-06\\
599.75	2.34155465586756e-06\\
599.76	2.16407268794142e-06\\
599.77	1.99303696153501e-06\\
599.78	1.82851396333256e-06\\
599.79	1.67057083969718e-06\\
599.8	1.51927540317613e-06\\
599.81	1.37469613908231e-06\\
599.82	1.23690221214869e-06\\
599.83	1.10596347322606e-06\\
599.84	9.81950466062351e-07\\
599.85	8.64934434142636e-07\\
599.86	7.54987327612408e-07\\
599.87	6.52181810235561e-07\\
599.88	5.56591266461653e-07\\
599.89	4.68289808524397e-07\\
599.9	3.87352283644227e-07\\
599.91	3.13854281279446e-07\\
599.92	2.47872140451966e-07\\
599.93	1.89482957158038e-07\\
599.94	1.38764591836246e-07\\
599.95	9.57956769204876e-08\\
599.96	6.06556244623496e-08\\
599.97	3.34246338194039e-08\\
599.98	1.4183699454523e-08\\
599.99	3.014618757749e-09\\
600	0\\
};
\addplot [color=red!25!mycolor17,solid,forget plot]
  table[row sep=crcr]{%
0.01	0.00894151321774357\\
1.01	0.00894151261464888\\
2.01	0.00894151199851174\\
3.01	0.00894151136904826\\
4.01	0.00894151072596831\\
5.01	0.00894151006897544\\
6.01	0.00894150939776663\\
7.01	0.0089415087120323\\
8.01	0.00894150801145603\\
9.01	0.00894150729571435\\
10.01	0.00894150656447688\\
11.01	0.00894150581740579\\
12.01	0.00894150505415598\\
13.01	0.00894150427437465\\
14.01	0.00894150347770131\\
15.01	0.00894150266376757\\
16.01	0.00894150183219687\\
17.01	0.00894150098260437\\
18.01	0.00894150011459686\\
19.01	0.00894149922777238\\
20.01	0.00894149832172012\\
21.01	0.00894149739602027\\
22.01	0.00894149645024375\\
23.01	0.00894149548395206\\
24.01	0.00894149449669706\\
25.01	0.00894149348802066\\
26.01	0.0089414924574547\\
27.01	0.00894149140452075\\
28.01	0.00894149032872973\\
29.01	0.00894148922958186\\
30.01	0.00894148810656636\\
31.01	0.00894148695916115\\
32.01	0.00894148578683254\\
33.01	0.00894148458903512\\
34.01	0.00894148336521152\\
35.01	0.00894148211479195\\
36.01	0.00894148083719403\\
37.01	0.00894147953182262\\
38.01	0.00894147819806935\\
39.01	0.00894147683531239\\
40.01	0.00894147544291619\\
41.01	0.00894147402023112\\
42.01	0.00894147256659323\\
43.01	0.0089414710813238\\
44.01	0.00894146956372919\\
45.01	0.00894146801310022\\
46.01	0.00894146642871228\\
47.01	0.00894146480982446\\
48.01	0.00894146315567963\\
49.01	0.00894146146550371\\
50.01	0.00894145973850553\\
51.01	0.00894145797387646\\
52.01	0.00894145617078981\\
53.01	0.00894145432840061\\
54.01	0.00894145244584514\\
55.01	0.0089414505222405\\
56.01	0.00894144855668418\\
57.01	0.00894144654825368\\
58.01	0.00894144449600596\\
59.01	0.00894144239897699\\
60.01	0.00894144025618145\\
61.01	0.00894143806661196\\
62.01	0.00894143582923881\\
63.01	0.00894143354300944\\
64.01	0.00894143120684771\\
65.01	0.00894142881965376\\
66.01	0.00894142638030302\\
67.01	0.00894142388764601\\
68.01	0.00894142134050764\\
69.01	0.00894141873768662\\
70.01	0.00894141607795485\\
71.01	0.00894141336005689\\
72.01	0.0089414105827093\\
73.01	0.00894140774460003\\
74.01	0.00894140484438768\\
75.01	0.00894140188070104\\
76.01	0.00894139885213815\\
77.01	0.00894139575726586\\
78.01	0.00894139259461894\\
79.01	0.00894138936269947\\
80.01	0.00894138605997596\\
81.01	0.00894138268488271\\
82.01	0.00894137923581905\\
83.01	0.0089413757111485\\
84.01	0.00894137210919787\\
85.01	0.00894136842825662\\
86.01	0.00894136466657578\\
87.01	0.00894136082236734\\
88.01	0.0089413568938032\\
89.01	0.00894135287901419\\
90.01	0.00894134877608929\\
91.01	0.00894134458307466\\
92.01	0.00894134029797261\\
93.01	0.0089413359187405\\
94.01	0.00894133144328999\\
95.01	0.00894132686948578\\
96.01	0.00894132219514458\\
97.01	0.00894131741803402\\
98.01	0.00894131253587157\\
99.01	0.00894130754632342\\
100.01	0.00894130244700323\\
101.01	0.00894129723547087\\
102.01	0.00894129190923148\\
103.01	0.00894128646573383\\
104.01	0.00894128090236932\\
105.01	0.00894127521647054\\
106.01	0.00894126940530998\\
107.01	0.0089412634660986\\
108.01	0.00894125739598443\\
109.01	0.0089412511920512\\
110.01	0.00894124485131678\\
111.01	0.00894123837073165\\
112.01	0.00894123174717752\\
113.01	0.00894122497746555\\
114.01	0.0089412180583348\\
115.01	0.0089412109864507\\
116.01	0.00894120375840312\\
117.01	0.00894119637070482\\
118.01	0.00894118881978959\\
119.01	0.00894118110201047\\
120.01	0.00894117321363788\\
121.01	0.00894116515085766\\
122.01	0.00894115690976925\\
123.01	0.00894114848638355\\
124.01	0.00894113987662105\\
125.01	0.0089411310763095\\
126.01	0.00894112208118201\\
127.01	0.00894111288687472\\
128.01	0.00894110348892462\\
129.01	0.0089410938827672\\
130.01	0.0089410840637342\\
131.01	0.00894107402705106\\
132.01	0.0089410637678346\\
133.01	0.0089410532810905\\
134.01	0.00894104256171054\\
135.01	0.00894103160447012\\
136.01	0.00894102040402564\\
137.01	0.00894100895491155\\
138.01	0.00894099725153761\\
139.01	0.00894098528818587\\
140.01	0.00894097305900795\\
141.01	0.0089409605580218\\
142.01	0.00894094777910862\\
143.01	0.00894093471600977\\
144.01	0.00894092136232341\\
145.01	0.00894090771150117\\
146.01	0.00894089375684481\\
147.01	0.00894087949150265\\
148.01	0.00894086490846605\\
149.01	0.00894085000056567\\
150.01	0.00894083476046775\\
151.01	0.00894081918067024\\
152.01	0.00894080325349894\\
153.01	0.00894078697110343\\
154.01	0.00894077032545292\\
155.01	0.00894075330833198\\
156.01	0.00894073591133631\\
157.01	0.0089407181258683\\
158.01	0.00894069994313242\\
159.01	0.00894068135413064\\
160.01	0.00894066234965762\\
161.01	0.00894064292029593\\
162.01	0.00894062305641088\\
163.01	0.00894060274814567\\
164.01	0.00894058198541587\\
165.01	0.00894056075790433\\
166.01	0.00894053905505555\\
167.01	0.00894051686606997\\
168.01	0.00894049417989846\\
169.01	0.00894047098523631\\
170.01	0.00894044727051697\\
171.01	0.00894042302390632\\
172.01	0.00894039823329595\\
173.01	0.00894037288629684\\
174.01	0.0089403469702327\\
175.01	0.00894032047213323\\
176.01	0.00894029337872692\\
177.01	0.00894026567643424\\
178.01	0.00894023735136012\\
179.01	0.00894020838928645\\
180.01	0.00894017877566452\\
181.01	0.00894014849560717\\
182.01	0.00894011753388064\\
183.01	0.00894008587489632\\
184.01	0.0089400535027025\\
185.01	0.00894002040097549\\
186.01	0.00893998655301091\\
187.01	0.00893995194171456\\
188.01	0.00893991654959314\\
189.01	0.00893988035874466\\
190.01	0.00893984335084885\\
191.01	0.00893980550715699\\
192.01	0.00893976680848176\\
193.01	0.00893972723518686\\
194.01	0.00893968676717593\\
195.01	0.0089396453838819\\
196.01	0.00893960306425557\\
197.01	0.00893955978675409\\
198.01	0.00893951552932906\\
199.01	0.00893947026941452\\
200.01	0.00893942398391432\\
201.01	0.00893937664918967\\
202.01	0.00893932824104591\\
203.01	0.00893927873471925\\
204.01	0.00893922810486304\\
205.01	0.00893917632553366\\
206.01	0.00893912337017641\\
207.01	0.00893906921161051\\
208.01	0.00893901382201416\\
209.01	0.00893895717290917\\
210.01	0.0089388992351449\\
211.01	0.00893883997888221\\
212.01	0.00893877937357691\\
213.01	0.00893871738796242\\
214.01	0.00893865399003264\\
215.01	0.00893858914702397\\
216.01	0.00893852282539686\\
217.01	0.00893845499081734\\
218.01	0.0089383856081374\\
219.01	0.00893831464137563\\
220.01	0.00893824205369682\\
221.01	0.00893816780739137\\
222.01	0.0089380918638539\\
223.01	0.00893801418356186\\
224.01	0.00893793472605301\\
225.01	0.00893785344990261\\
226.01	0.00893777031270021\\
227.01	0.00893768527102556\\
228.01	0.00893759828042407\\
229.01	0.00893750929538178\\
230.01	0.00893741826929937\\
231.01	0.00893732515446611\\
232.01	0.00893722990203235\\
233.01	0.00893713246198218\\
234.01	0.00893703278310474\\
235.01	0.00893693081296528\\
236.01	0.00893682649787527\\
237.01	0.00893671978286185\\
238.01	0.00893661061163641\\
239.01	0.00893649892656259\\
240.01	0.0089363846686233\\
241.01	0.00893626777738715\\
242.01	0.00893614819097369\\
243.01	0.00893602584601828\\
244.01	0.00893590067763559\\
245.01	0.00893577261938262\\
246.01	0.00893564160322045\\
247.01	0.00893550755947528\\
248.01	0.00893537041679852\\
249.01	0.0089352301021258\\
250.01	0.00893508654063487\\
251.01	0.00893493965570274\\
252.01	0.00893478936886143\\
253.01	0.00893463559975303\\
254.01	0.00893447826608312\\
255.01	0.00893431728357363\\
256.01	0.00893415256591417\\
257.01	0.00893398402471211\\
258.01	0.00893381156944177\\
259.01	0.00893363510739197\\
260.01	0.00893345454361264\\
261.01	0.00893326978085985\\
262.01	0.00893308071953978\\
263.01	0.00893288725765067\\
264.01	0.00893268929072435\\
265.01	0.00893248671176548\\
266.01	0.00893227941118956\\
267.01	0.0089320672767597\\
268.01	0.00893185019352134\\
269.01	0.00893162804373585\\
270.01	0.00893140070681221\\
271.01	0.008931168059237\\
272.01	0.00893092997450288\\
273.01	0.00893068632303518\\
274.01	0.00893043697211676\\
275.01	0.00893018178581092\\
276.01	0.00892992062488261\\
277.01	0.00892965334671749\\
278.01	0.00892937980523929\\
279.01	0.00892909985082486\\
280.01	0.00892881333021744\\
281.01	0.00892852008643756\\
282.01	0.00892821995869189\\
283.01	0.00892791278227995\\
284.01	0.0089275983884984\\
285.01	0.00892727660454309\\
286.01	0.0089269472534087\\
287.01	0.0089266101537859\\
288.01	0.00892626511995626\\
289.01	0.00892591196168419\\
290.01	0.00892555048410659\\
291.01	0.00892518048761965\\
292.01	0.00892480176776308\\
293.01	0.00892441411510132\\
294.01	0.00892401731510211\\
295.01	0.00892361114801182\\
296.01	0.00892319538872798\\
297.01	0.00892276980666865\\
298.01	0.00892233416563886\\
299.01	0.00892188822369321\\
300.01	0.00892143173299576\\
301.01	0.00892096443967636\\
302.01	0.00892048608368312\\
303.01	0.00891999639863201\\
304.01	0.00891949511165199\\
305.01	0.00891898194322718\\
306.01	0.00891845660703449\\
307.01	0.00891791880977794\\
308.01	0.00891736825101831\\
309.01	0.00891680462299919\\
310.01	0.00891622761046853\\
311.01	0.00891563689049579\\
312.01	0.00891503213228483\\
313.01	0.00891441299698199\\
314.01	0.00891377913747952\\
315.01	0.0089131301982144\\
316.01	0.00891246581496189\\
317.01	0.00891178561462424\\
318.01	0.00891108921501403\\
319.01	0.00891037622463211\\
320.01	0.00890964624244024\\
321.01	0.00890889885762785\\
322.01	0.00890813364937315\\
323.01	0.008907350186598\\
324.01	0.00890654802771699\\
325.01	0.00890572672037978\\
326.01	0.00890488580120724\\
327.01	0.00890402479552079\\
328.01	0.00890314321706468\\
329.01	0.00890224056772141\\
330.01	0.00890131633721959\\
331.01	0.00890037000283452\\
332.01	0.00889940102908055\\
333.01	0.00889840886739597\\
334.01	0.0088973929558192\\
335.01	0.00889635271865654\\
336.01	0.00889528756614137\\
337.01	0.00889419689408379\\
338.01	0.00889308008351116\\
339.01	0.00889193650029912\\
340.01	0.00889076549479213\\
341.01	0.00888956640141393\\
342.01	0.00888833853826738\\
343.01	0.00888708120672304\\
344.01	0.00888579369099642\\
345.01	0.00888447525771335\\
346.01	0.00888312515546309\\
347.01	0.00888174261433902\\
348.01	0.0088803268454657\\
349.01	0.00887887704051307\\
350.01	0.00887739237119596\\
351.01	0.00887587198875959\\
352.01	0.00887431502344956\\
353.01	0.0088727205839668\\
354.01	0.00887108775690598\\
355.01	0.0088694156061774\\
356.01	0.00886770317241179\\
357.01	0.00886594947234695\\
358.01	0.00886415349819599\\
359.01	0.00886231421699673\\
360.01	0.00886043056994076\\
361.01	0.00885850147168265\\
362.01	0.00885652580962747\\
363.01	0.00885450244319672\\
364.01	0.00885243020307152\\
365.01	0.00885030789041235\\
366.01	0.00884813427605492\\
367.01	0.00884590809968062\\
368.01	0.00884362806896159\\
369.01	0.00884129285867894\\
370.01	0.00883890110981364\\
371.01	0.008836451428609\\
372.01	0.00883394238560412\\
373.01	0.00883137251463696\\
374.01	0.00882874031181643\\
375.01	0.00882604423446269\\
376.01	0.00882328270001422\\
377.01	0.00882045408490118\\
378.01	0.00881755672338373\\
379.01	0.00881458890635412\\
380.01	0.00881154888010224\\
381.01	0.00880843484504269\\
382.01	0.00880524495440249\\
383.01	0.00880197731286887\\
384.01	0.00879862997519489\\
385.01	0.00879520094476274\\
386.01	0.0087916881721026\\
387.01	0.00878808955336624\\
388.01	0.00878440292875404\\
389.01	0.0087806260808934\\
390.01	0.00877675673316796\\
391.01	0.00877279254799538\\
392.01	0.00876873112505233\\
393.01	0.00876456999944523\\
394.01	0.00876030663982492\\
395.01	0.00875593844644316\\
396.01	0.00875146274914971\\
397.01	0.00874687680532766\\
398.01	0.00874217779776539\\
399.01	0.0087373628324628\\
400.01	0.00873242893637046\\
401.01	0.00872737305505869\\
402.01	0.00872219205031533\\
403.01	0.00871688269766945\\
404.01	0.00871144168383938\\
405.01	0.0087058656041022\\
406.01	0.0087001509595828\\
407.01	0.00869429415446006\\
408.01	0.00868829149308723\\
409.01	0.00868213917702424\\
410.01	0.00867583330197849\\
411.01	0.00866936985465104\\
412.01	0.00866274470948433\\
413.01	0.0086559536253075\\
414.01	0.00864899224187483\\
415.01	0.00864185607629205\\
416.01	0.00863454051932611\\
417.01	0.00862704083159274\\
418.01	0.00861935213961748\\
419.01	0.00861146943176589\\
420.01	0.00860338755403901\\
421.01	0.00859510120573041\\
422.01	0.00858660493493794\\
423.01	0.00857789313391867\\
424.01	0.0085689600342695\\
425.01	0.00855979970191195\\
426.01	0.00855040603185129\\
427.01	0.00854077274267304\\
428.01	0.00853089337072811\\
429.01	0.00852076126394522\\
430.01	0.00851036957519363\\
431.01	0.00849971125510028\\
432.01	0.0084887790442035\\
433.01	0.00847756546430114\\
434.01	0.00846606280882325\\
435.01	0.00845426313203126\\
436.01	0.00844215823681791\\
437.01	0.00842973966085852\\
438.01	0.00841699866085121\\
439.01	0.00840392619458725\\
440.01	0.00839051290063101\\
441.01	0.00837674907547543\\
442.01	0.00836262464820484\\
443.01	0.00834812915298423\\
444.01	0.00833325170015707\\
445.01	0.00831798094745767\\
446.01	0.00830230507491702\\
447.01	0.00828621177179482\\
448.01	0.00826968821763505\\
449.01	0.00825272104924269\\
450.01	0.00823529632331078\\
451.01	0.00821739947509464\\
452.01	0.00819901527257593\\
453.01	0.00818012776546474\\
454.01	0.00816072022827056\\
455.01	0.00814077509653561\\
456.01	0.00812027389515372\\
457.01	0.00809919715749363\\
458.01	0.00807752433379312\\
459.01	0.00805523368698338\\
460.01	0.00803230217372145\\
461.01	0.00800870530793988\\
462.01	0.00798441700363633\\
463.01	0.00795940939289735\\
464.01	0.00793365261423636\\
465.01	0.00790711456517616\\
466.01	0.0078797606115569\\
467.01	0.00785155324421342\\
468.01	0.00782245167133338\\
469.01	0.00779241133183461\\
470.01	0.00776138331130186\\
471.01	0.0077293136371557\\
472.01	0.00769614242347331\\
473.01	0.00766180282783474\\
474.01	0.00762621977218363\\
475.01	0.00758930836627548\\
476.01	0.00755097195491648\\
477.01	0.00751109968769789\\
478.01	0.00746956348057245\\
479.01	0.00742621419992225\\
480.01	0.00738087684906441\\
481.01	0.00733334447066516\\
482.01	0.00730772838409598\\
483.01	0.00728921539375883\\
484.01	0.00726986604526977\\
485.01	0.0072495895299529\\
486.01	0.00722827672678798\\
487.01	0.00720579559218599\\
488.01	0.00718198527132034\\
489.01	0.00715664855337426\\
490.01	0.00712954217595951\\
491.01	0.00710036432905067\\
492.01	0.00706873850339288\\
493.01	0.00703426993795644\\
494.01	0.00699822049880224\\
495.01	0.006961198131097\\
496.01	0.00692317937685264\\
497.01	0.00688413949676074\\
498.01	0.00684405178854448\\
499.01	0.00680288658077105\\
500.01	0.00676060976968188\\
501.01	0.00671718071535282\\
502.01	0.00667254924184988\\
503.01	0.00662665139217403\\
504.01	0.0065794087358317\\
505.01	0.00653076848116818\\
506.01	0.00648069800907385\\
507.01	0.00642916762080523\\
508.01	0.00637615143522\\
509.01	0.00632162873443743\\
510.01	0.00626558567286279\\
511.01	0.00620801745221732\\
512.01	0.00614893109665368\\
513.01	0.00608834900380557\\
514.01	0.00602631350328169\\
515.01	0.00596289272854014\\
516.01	0.00589818820786404\\
517.01	0.00583234471434273\\
518.01	0.00576556309567665\\
519.01	0.00569811707126496\\
520.01	0.00563037536487652\\
521.01	0.00556283014420392\\
522.01	0.00549613352115219\\
523.01	0.0054311450531315\\
524.01	0.00536899375642301\\
525.01	0.00531089743934325\\
526.01	0.00525433400642232\\
527.01	0.00519790383802797\\
528.01	0.00513996246353109\\
529.01	0.00508042133421524\\
530.01	0.00501928811296119\\
531.01	0.0049565761873247\\
532.01	0.00489230154984199\\
533.01	0.00482647312224399\\
534.01	0.00475908001949976\\
535.01	0.00469011082236774\\
536.01	0.00461955953879067\\
537.01	0.0045474267640274\\
538.01	0.00447372112767133\\
539.01	0.00439846114084963\\
540.01	0.00432167761323271\\
541.01	0.00424341689448369\\
542.01	0.00416374521879202\\
543.01	0.00408275070957415\\
544.01	0.00400054871057547\\
545.01	0.00391793968658502\\
546.01	0.00383545025872607\\
547.01	0.00375332823407465\\
548.01	0.00367184062787143\\
549.01	0.0035912528713037\\
550.01	0.00351180585965046\\
551.01	0.0034337427002255\\
552.01	0.00335737485324567\\
553.01	0.00328298039254646\\
554.01	0.00321074134821009\\
555.01	0.00314037875271919\\
556.01	0.00307077499893151\\
557.01	0.00300188263613312\\
558.01	0.00293373027196458\\
559.01	0.00286631379211223\\
560.01	0.00279960126959131\\
561.01	0.00273353630284404\\
562.01	0.0026680150261463\\
563.01	0.0026028766296019\\
564.01	0.00253791314161076\\
565.01	0.00247294065430676\\
566.01	0.00240790287881855\\
567.01	0.00234275431907017\\
568.01	0.00227743915861326\\
569.01	0.00221189383001435\\
570.01	0.00214604795923127\\
571.01	0.00207983098516851\\
572.01	0.00201318079937052\\
573.01	0.00194605206981758\\
574.01	0.0018784185498138\\
575.01	0.00181025995554062\\
576.01	0.00174155473393376\\
577.01	0.0016722815447536\\
578.01	0.00160242126650577\\
579.01	0.00153195886904355\\
580.01	0.00146088474238344\\
581.01	0.00138919499581868\\
582.01	0.0013168903329978\\
583.01	0.00124397413150743\\
584.01	0.00117045174146673\\
585.01	0.00109633076541331\\
586.01	0.00102162149322053\\
587.01	0.000946337383085976\\
588.01	0.000870495457990504\\
589.01	0.000794116593283583\\
590.01	0.00071722578280283\\
591.01	0.000639852503519138\\
592.01	0.00056203108043296\\
593.01	0.000483800710259085\\
594.01	0.000405204886350921\\
595.01	0.000326290009844911\\
596.01	0.000247102975742894\\
597.01	0.000167687548540458\\
598.01	9.17367679171607e-05\\
599.01	2.94669369274633e-05\\
599.02	2.89574274290674e-05\\
599.03	2.84509535274195e-05\\
599.04	2.7947544761462e-05\\
599.05	2.7447230960936e-05\\
599.06	2.69500422492523e-05\\
599.07	2.64560090463949e-05\\
599.08	2.5965162071857e-05\\
599.09	2.54775323476081e-05\\
599.1	2.49931512010769e-05\\
599.11	2.45120502681759e-05\\
599.12	2.40342614963567e-05\\
599.13	2.35598171476892e-05\\
599.14	2.30887498019697e-05\\
599.15	2.26210923598685e-05\\
599.16	2.21568780461034e-05\\
599.17	2.16961404126415e-05\\
599.18	2.12389133419366e-05\\
599.19	2.07852310502073e-05\\
599.2	2.03351280907219e-05\\
599.21	1.98886393571462e-05\\
599.22	1.94458000869063e-05\\
599.23	1.90066458645843e-05\\
599.24	1.85712126253585e-05\\
599.25	1.8139536658468e-05\\
599.26	1.77116564944301e-05\\
599.27	1.72876122915989e-05\\
599.28	1.68674446078751e-05\\
599.29	1.64511944046493e-05\\
599.3	1.60389030507795e-05\\
599.31	1.56306123266399e-05\\
599.32	1.52263644281631e-05\\
599.33	1.48262019709685e-05\\
599.34	1.44301679944961e-05\\
599.35	1.40383059661995e-05\\
599.36	1.36506597857768e-05\\
599.37	1.32672737894416e-05\\
599.38	1.28881927542353e-05\\
599.39	1.25134619023814e-05\\
599.4	1.21431269056952e-05\\
599.41	1.17772338900073e-05\\
599.42	1.14158294396705e-05\\
599.43	1.10589606020818e-05\\
599.44	1.07066748922489e-05\\
599.45	1.03590202974301e-05\\
599.46	1.00160452817816e-05\\
599.47	9.67779879108323e-06\\
599.48	9.34433025748943e-06\\
599.49	9.01568960434002e-06\\
599.5	8.69192725100679e-06\\
599.51	8.37309411780285e-06\\
599.52	8.0592416309283e-06\\
599.53	7.75042172746279e-06\\
599.54	7.4466868604222e-06\\
599.55	7.14809000385182e-06\\
599.56	6.85468465797499e-06\\
599.57	6.5665248543869e-06\\
599.58	6.283665161309e-06\\
599.59	6.00616068889559e-06\\
599.6	5.73406709457672e-06\\
599.61	5.46744058846881e-06\\
599.62	5.20633793883903e-06\\
599.63	4.95081647761304e-06\\
599.64	4.70093410595215e-06\\
599.65	4.45674929985991e-06\\
599.66	4.21832111588243e-06\\
599.67	3.98570919682602e-06\\
599.68	3.75897377755986e-06\\
599.69	3.5381756908516e-06\\
599.7	3.32337637328622e-06\\
599.71	3.11463787123004e-06\\
599.72	2.91202284684322e-06\\
599.73	2.71559458416872e-06\\
599.74	2.52541699527833e-06\\
599.75	2.34155462647441e-06\\
599.76	2.16407266454867e-06\\
599.77	1.99303694311918e-06\\
599.78	1.82851394900548e-06\\
599.79	1.67057082869557e-06\\
599.8	1.51927539484599e-06\\
599.81	1.37469613288067e-06\\
599.82	1.23690220761412e-06\\
599.83	1.10596346997692e-06\\
599.84	9.81950463786394e-07\\
599.85	8.64934432593528e-07\\
599.86	7.54987326588921e-07\\
599.87	6.5218180958504e-07\\
599.88	5.56591266064402e-07\\
599.89	4.68289808293679e-07\\
599.9	3.87352283521061e-07\\
599.91	3.13854281218731e-07\\
599.92	2.47872140425945e-07\\
599.93	1.89482957149364e-07\\
599.94	1.38764591834512e-07\\
599.95	9.57956769222224e-08\\
599.96	6.06556244606149e-08\\
599.97	3.34246338211386e-08\\
599.98	1.4183699454523e-08\\
599.99	3.01461875948372e-09\\
600	0\\
};
\addplot [color=mycolor19,solid,forget plot]
  table[row sep=crcr]{%
0.01	0.00739460781638744\\
1.01	0.00739460735721567\\
2.01	0.00739460688814162\\
3.01	0.00739460640895054\\
4.01	0.00739460591942291\\
5.01	0.00739460541933461\\
6.01	0.00739460490845621\\
7.01	0.00739460438655362\\
8.01	0.00739460385338746\\
9.01	0.00739460330871333\\
10.01	0.00739460275228116\\
11.01	0.00739460218383558\\
12.01	0.00739460160311581\\
13.01	0.00739460100985508\\
14.01	0.00739460040378089\\
15.01	0.00739459978461482\\
16.01	0.00739459915207223\\
17.01	0.00739459850586236\\
18.01	0.00739459784568802\\
19.01	0.00739459717124554\\
20.01	0.00739459648222471\\
21.01	0.00739459577830832\\
22.01	0.00739459505917233\\
23.01	0.00739459432448543\\
24.01	0.00739459357390921\\
25.01	0.00739459280709796\\
26.01	0.00739459202369806\\
27.01	0.00739459122334836\\
28.01	0.00739459040567979\\
29.01	0.00739458957031496\\
30.01	0.00739458871686853\\
31.01	0.00739458784494637\\
32.01	0.00739458695414586\\
33.01	0.0073945860440556\\
34.01	0.00739458511425492\\
35.01	0.00739458416431424\\
36.01	0.00739458319379423\\
37.01	0.00739458220224598\\
38.01	0.00739458118921078\\
39.01	0.0073945801542198\\
40.01	0.00739457909679383\\
41.01	0.0073945780164433\\
42.01	0.00739457691266755\\
43.01	0.00739457578495535\\
44.01	0.00739457463278377\\
45.01	0.00739457345561862\\
46.01	0.00739457225291381\\
47.01	0.00739457102411131\\
48.01	0.00739456976864065\\
49.01	0.00739456848591905\\
50.01	0.00739456717535065\\
51.01	0.00739456583632645\\
52.01	0.0073945644682242\\
53.01	0.0073945630704077\\
54.01	0.00739456164222689\\
55.01	0.00739456018301725\\
56.01	0.00739455869209964\\
57.01	0.00739455716877984\\
58.01	0.00739455561234832\\
59.01	0.00739455402208016\\
60.01	0.00739455239723384\\
61.01	0.00739455073705207\\
62.01	0.0073945490407604\\
63.01	0.00739454730756729\\
64.01	0.00739454553666394\\
65.01	0.00739454372722328\\
66.01	0.00739454187840033\\
67.01	0.0073945399893311\\
68.01	0.00739453805913265\\
69.01	0.00739453608690237\\
70.01	0.0073945340717177\\
71.01	0.00739453201263567\\
72.01	0.00739452990869233\\
73.01	0.00739452775890244\\
74.01	0.00739452556225888\\
75.01	0.00739452331773208\\
76.01	0.00739452102426993\\
77.01	0.00739451868079665\\
78.01	0.00739451628621268\\
79.01	0.00739451383939399\\
80.01	0.00739451133919172\\
81.01	0.00739450878443146\\
82.01	0.00739450617391255\\
83.01	0.00739450350640777\\
84.01	0.00739450078066259\\
85.01	0.00739449799539457\\
86.01	0.00739449514929286\\
87.01	0.00739449224101712\\
88.01	0.00739448926919739\\
89.01	0.00739448623243329\\
90.01	0.00739448312929309\\
91.01	0.00739447995831325\\
92.01	0.00739447671799745\\
93.01	0.00739447340681626\\
94.01	0.00739447002320606\\
95.01	0.00739446656556814\\
96.01	0.00739446303226845\\
97.01	0.00739445942163625\\
98.01	0.00739445573196368\\
99.01	0.00739445196150452\\
100.01	0.00739444810847374\\
101.01	0.00739444417104638\\
102.01	0.00739444014735652\\
103.01	0.00739443603549671\\
104.01	0.00739443183351683\\
105.01	0.00739442753942302\\
106.01	0.00739442315117697\\
107.01	0.00739441866669458\\
108.01	0.00739441408384518\\
109.01	0.00739440940045021\\
110.01	0.00739440461428254\\
111.01	0.00739439972306501\\
112.01	0.00739439472446939\\
113.01	0.00739438961611521\\
114.01	0.00739438439556867\\
115.01	0.00739437906034121\\
116.01	0.00739437360788864\\
117.01	0.00739436803560947\\
118.01	0.00739436234084363\\
119.01	0.00739435652087173\\
120.01	0.00739435057291278\\
121.01	0.00739434449412337\\
122.01	0.0073943382815963\\
123.01	0.00739433193235886\\
124.01	0.00739432544337121\\
125.01	0.00739431881152551\\
126.01	0.00739431203364364\\
127.01	0.0073943051064758\\
128.01	0.00739429802669893\\
129.01	0.00739429079091534\\
130.01	0.00739428339565034\\
131.01	0.00739427583735108\\
132.01	0.0073942681123843\\
133.01	0.00739426021703493\\
134.01	0.00739425214750385\\
135.01	0.00739424389990599\\
136.01	0.00739423547026865\\
137.01	0.00739422685452913\\
138.01	0.00739421804853298\\
139.01	0.00739420904803167\\
140.01	0.00739419984868031\\
141.01	0.00739419044603592\\
142.01	0.00739418083555458\\
143.01	0.00739417101258945\\
144.01	0.00739416097238848\\
145.01	0.00739415071009162\\
146.01	0.00739414022072869\\
147.01	0.00739412949921663\\
148.01	0.00739411854035687\\
149.01	0.00739410733883286\\
150.01	0.00739409588920699\\
151.01	0.00739408418591852\\
152.01	0.00739407222327968\\
153.01	0.00739405999547342\\
154.01	0.00739404749655035\\
155.01	0.00739403472042565\\
156.01	0.00739402166087577\\
157.01	0.00739400831153534\\
158.01	0.00739399466589381\\
159.01	0.0073939807172921\\
160.01	0.00739396645891918\\
161.01	0.00739395188380848\\
162.01	0.00739393698483459\\
163.01	0.00739392175470884\\
164.01	0.00739390618597618\\
165.01	0.00739389027101107\\
166.01	0.00739387400201339\\
167.01	0.00739385737100459\\
168.01	0.00739384036982332\\
169.01	0.0073938229901212\\
170.01	0.00739380522335864\\
171.01	0.00739378706080012\\
172.01	0.00739376849350976\\
173.01	0.00739374951234659\\
174.01	0.00739373010795965\\
175.01	0.0073937102707832\\
176.01	0.00739368999103181\\
177.01	0.00739366925869482\\
178.01	0.00739364806353133\\
179.01	0.00739362639506489\\
180.01	0.00739360424257758\\
181.01	0.00739358159510467\\
182.01	0.00739355844142866\\
183.01	0.0073935347700736\\
184.01	0.00739351056929848\\
185.01	0.00739348582709131\\
186.01	0.00739346053116302\\
187.01	0.00739343466894026\\
188.01	0.00739340822755921\\
189.01	0.00739338119385852\\
190.01	0.00739335355437231\\
191.01	0.0073933252953229\\
192.01	0.00739329640261368\\
193.01	0.00739326686182099\\
194.01	0.00739323665818721\\
195.01	0.00739320577661214\\
196.01	0.00739317420164503\\
197.01	0.00739314191747661\\
198.01	0.00739310890793014\\
199.01	0.00739307515645301\\
200.01	0.00739304064610744\\
201.01	0.00739300535956169\\
202.01	0.00739296927908054\\
203.01	0.00739293238651537\\
204.01	0.00739289466329505\\
205.01	0.00739285609041505\\
206.01	0.00739281664842748\\
207.01	0.0073927763174307\\
208.01	0.00739273507705801\\
209.01	0.00739269290646691\\
210.01	0.00739264978432763\\
211.01	0.00739260568881128\\
212.01	0.00739256059757806\\
213.01	0.00739251448776509\\
214.01	0.00739246733597363\\
215.01	0.0073924191182565\\
216.01	0.00739236981010455\\
217.01	0.00739231938643358\\
218.01	0.00739226782157008\\
219.01	0.00739221508923747\\
220.01	0.00739216116254139\\
221.01	0.00739210601395458\\
222.01	0.00739204961530218\\
223.01	0.00739199193774552\\
224.01	0.00739193295176645\\
225.01	0.00739187262715094\\
226.01	0.00739181093297207\\
227.01	0.00739174783757285\\
228.01	0.00739168330854878\\
229.01	0.0073916173127293\\
230.01	0.00739154981615984\\
231.01	0.00739148078408217\\
232.01	0.00739141018091542\\
233.01	0.0073913379702357\\
234.01	0.00739126411475604\\
235.01	0.00739118857630508\\
236.01	0.00739111131580584\\
237.01	0.00739103229325327\\
238.01	0.00739095146769215\\
239.01	0.00739086879719365\\
240.01	0.00739078423883171\\
241.01	0.0073906977486585\\
242.01	0.00739060928167972\\
243.01	0.00739051879182908\\
244.01	0.00739042623194195\\
245.01	0.00739033155372833\\
246.01	0.00739023470774575\\
247.01	0.0073901356433708\\
248.01	0.00739003430877038\\
249.01	0.00738993065087156\\
250.01	0.00738982461533181\\
251.01	0.00738971614650737\\
252.01	0.0073896051874218\\
253.01	0.00738949167973259\\
254.01	0.00738937556369788\\
255.01	0.00738925677814244\\
256.01	0.00738913526042168\\
257.01	0.00738901094638614\\
258.01	0.00738888377034369\\
259.01	0.00738875366502219\\
260.01	0.00738862056153028\\
261.01	0.00738848438931714\\
262.01	0.00738834507613185\\
263.01	0.00738820254798125\\
264.01	0.00738805672908689\\
265.01	0.00738790754184095\\
266.01	0.00738775490676085\\
267.01	0.00738759874244298\\
268.01	0.00738743896551507\\
269.01	0.00738727549058717\\
270.01	0.0073871082302019\\
271.01	0.00738693709478314\\
272.01	0.00738676199258338\\
273.01	0.00738658282962967\\
274.01	0.00738639950966869\\
275.01	0.00738621193410975\\
276.01	0.00738602000196675\\
277.01	0.00738582360979866\\
278.01	0.00738562265164842\\
279.01	0.00738541701898025\\
280.01	0.00738520660061536\\
281.01	0.00738499128266623\\
282.01	0.00738477094846914\\
283.01	0.00738454547851474\\
284.01	0.00738431475037708\\
285.01	0.00738407863864118\\
286.01	0.00738383701482791\\
287.01	0.00738358974731779\\
288.01	0.00738333670127216\\
289.01	0.00738307773855345\\
290.01	0.0073828127176421\\
291.01	0.00738254149355236\\
292.01	0.0073822639177452\\
293.01	0.00738197983803984\\
294.01	0.00738168909852228\\
295.01	0.007381391539452\\
296.01	0.00738108699716632\\
297.01	0.00738077530398194\\
298.01	0.0073804562880943\\
299.01	0.00738012977347433\\
300.01	0.00737979557976289\\
301.01	0.0073794535221616\\
302.01	0.00737910341132239\\
303.01	0.00737874505323258\\
304.01	0.00737837824909876\\
305.01	0.00737800279522633\\
306.01	0.00737761848289708\\
307.01	0.00737722509824278\\
308.01	0.00737682242211633\\
309.01	0.00737641022995919\\
310.01	0.00737598829166552\\
311.01	0.00737555637144354\\
312.01	0.00737511422767252\\
313.01	0.00737466161275675\\
314.01	0.00737419827297554\\
315.01	0.00737372394832979\\
316.01	0.0073732383723844\\
317.01	0.00737274127210699\\
318.01	0.00737223236770241\\
319.01	0.00737171137244328\\
320.01	0.00737117799249619\\
321.01	0.00737063192674344\\
322.01	0.00737007286660059\\
323.01	0.00736950049582947\\
324.01	0.00736891449034603\\
325.01	0.00736831451802444\\
326.01	0.0073677002384949\\
327.01	0.00736707130293777\\
328.01	0.00736642735387172\\
329.01	0.00736576802493683\\
330.01	0.00736509294067259\\
331.01	0.00736440171628967\\
332.01	0.00736369395743714\\
333.01	0.00736296925996266\\
334.01	0.0073622272096674\\
335.01	0.00736146738205507\\
336.01	0.00736068934207347\\
337.01	0.00735989264385179\\
338.01	0.00735907683042887\\
339.01	0.00735824143347624\\
340.01	0.00735738597301394\\
341.01	0.00735650995711878\\
342.01	0.00735561288162585\\
343.01	0.00735469422982191\\
344.01	0.00735375347213177\\
345.01	0.00735279006579585\\
346.01	0.00735180345454027\\
347.01	0.00735079306823823\\
348.01	0.00734975832256253\\
349.01	0.00734869861862977\\
350.01	0.00734761334263451\\
351.01	0.00734650186547459\\
352.01	0.00734536354236683\\
353.01	0.00734419771245164\\
354.01	0.00734300369838831\\
355.01	0.00734178080593891\\
356.01	0.00734052832354109\\
357.01	0.00733924552186971\\
358.01	0.00733793165338621\\
359.01	0.00733658595187573\\
360.01	0.00733520763197159\\
361.01	0.00733379588866597\\
362.01	0.00733234989680765\\
363.01	0.00733086881058492\\
364.01	0.0073293517629934\\
365.01	0.00732779786528957\\
366.01	0.00732620620642631\\
367.01	0.00732457585247379\\
368.01	0.0073229058460219\\
369.01	0.00732119520556453\\
370.01	0.00731944292486566\\
371.01	0.00731764797230505\\
372.01	0.00731580929020395\\
373.01	0.00731392579412926\\
374.01	0.00731199637217588\\
375.01	0.00731001988422509\\
376.01	0.00730799516117935\\
377.01	0.00730592100417133\\
378.01	0.00730379618374717\\
379.01	0.00730161943902166\\
380.01	0.0072993894768053\\
381.01	0.00729710497070073\\
382.01	0.00729476456016879\\
383.01	0.00729236684956112\\
384.01	0.00728991040711958\\
385.01	0.00728739376393997\\
386.01	0.0072848154128984\\
387.01	0.00728217380754028\\
388.01	0.00727946736092822\\
389.01	0.00727669444444872\\
390.01	0.00727385338657547\\
391.01	0.00727094247158749\\
392.01	0.00726795993824026\\
393.01	0.00726490397838857\\
394.01	0.00726177273555814\\
395.01	0.00725856430346603\\
396.01	0.00725527672448628\\
397.01	0.00725190798806082\\
398.01	0.00724845602905235\\
399.01	0.00724491872604027\\
400.01	0.00724129389955609\\
401.01	0.00723757931025992\\
402.01	0.00723377265705657\\
403.01	0.00722987157515321\\
404.01	0.00722587363405743\\
405.01	0.00722177633552025\\
406.01	0.00721757711142594\\
407.01	0.00721327332163232\\
408.01	0.00720886225176726\\
409.01	0.00720434111098601\\
410.01	0.00719970702969591\\
411.01	0.00719495705725344\\
412.01	0.00719008815963761\\
413.01	0.00718509721710061\\
414.01	0.00717998102179087\\
415.01	0.00717473627533758\\
416.01	0.00716935958637324\\
417.01	0.00716384746795955\\
418.01	0.00715819633486804\\
419.01	0.00715240250065579\\
420.01	0.00714646217447976\\
421.01	0.00714037145762384\\
422.01	0.00713412633977827\\
423.01	0.00712772269510845\\
424.01	0.00712115627811266\\
425.01	0.00711442271926145\\
426.01	0.00710751752041582\\
427.01	0.00710043605002663\\
428.01	0.00709317353812675\\
429.01	0.00708572507113625\\
430.01	0.00707808558651451\\
431.01	0.00707024986731098\\
432.01	0.00706221253668515\\
433.01	0.00705396805249208\\
434.01	0.00704551070205724\\
435.01	0.00703683459729696\\
436.01	0.00702793367037132\\
437.01	0.0070188016700894\\
438.01	0.00700943215930772\\
439.01	0.00699981851357164\\
440.01	0.0069899539212232\\
441.01	0.00697983138512249\\
442.01	0.00696944372596999\\
443.01	0.00695878358691883\\
444.01	0.00694784343866876\\
445.01	0.00693661558342861\\
446.01	0.00692509215389056\\
447.01	0.00691326509820568\\
448.01	0.00690112616826287\\
449.01	0.00688866691980275\\
450.01	0.00687587871469878\\
451.01	0.00686275272483023\\
452.01	0.00684927993790207\\
453.01	0.0068354511656438\\
454.01	0.0068212570549198\\
455.01	0.00680668810240477\\
456.01	0.00679173467363303\\
457.01	0.00677638702742366\\
458.01	0.00676063534693086\\
459.01	0.00674446977887691\\
460.01	0.00672788048292408\\
461.01	0.00671085769364399\\
462.01	0.00669339179818848\\
463.01	0.00667547343359382\\
464.01	0.00665709360870891\\
465.01	0.00663824385710418\\
466.01	0.00661891642907578\\
467.01	0.00659910453312369\\
468.01	0.00657880264020722\\
469.01	0.00655800686785727\\
470.01	0.00653671546609859\\
471.01	0.00651492943343259\\
472.01	0.00649265329926033\\
473.01	0.00646989611961106\\
474.01	0.0064466727465601\\
475.01	0.00642300544911735\\
476.01	0.00639892598551904\\
477.01	0.00637447825521281\\
478.01	0.00634972170595724\\
479.01	0.00632473573363495\\
480.01	0.00629962538319443\\
481.01	0.00627452875737553\\
482.01	0.00624939051251928\\
483.01	0.00622376064740071\\
484.01	0.00619765015738636\\
485.01	0.0061711067485383\\
486.01	0.00614419621701035\\
487.01	0.00611700764458573\\
488.01	0.00608966011335451\\
489.01	0.00606231140357749\\
490.01	0.00603516928644871\\
491.01	0.00600850621982586\\
492.01	0.00598267851658965\\
493.01	0.00595807385714244\\
494.01	0.00593342118060778\\
495.01	0.00590807476816829\\
496.01	0.00588201498763912\\
497.01	0.00585522118092143\\
498.01	0.00582767155076743\\
499.01	0.00579934311057551\\
500.01	0.00577021176562901\\
501.01	0.00574025263987014\\
502.01	0.00570944089116093\\
503.01	0.00567775295885528\\
504.01	0.00564516795050089\\
505.01	0.00561166826048236\\
506.01	0.00557723879844062\\
507.01	0.00554186725854646\\
508.01	0.0055055447350111\\
509.01	0.00546826638667768\\
510.01	0.00543003212621563\\
511.01	0.00539084729183767\\
512.01	0.00535072323266179\\
513.01	0.00530967769962473\\
514.01	0.00526773487704865\\
515.01	0.00522492480832201\\
516.01	0.00518128185258883\\
517.01	0.00513684164376462\\
518.01	0.0050916357889881\\
519.01	0.00504568026289477\\
520.01	0.00499895988309242\\
521.01	0.00495143501123638\\
522.01	0.00490303435970988\\
523.01	0.00485363928552798\\
524.01	0.00480306166516005\\
525.01	0.0047510147258005\\
526.01	0.004697213408747\\
527.01	0.00464152182251868\\
528.01	0.0045838662559028\\
529.01	0.00452422283921419\\
530.01	0.00446261138137171\\
531.01	0.00439911741448504\\
532.01	0.00433392363083164\\
533.01	0.00426778874389639\\
534.01	0.00420137881425066\\
535.01	0.00413485118336642\\
536.01	0.00406838381103923\\
537.01	0.00400217499259854\\
538.01	0.00393644160379946\\
539.01	0.0038714150870976\\
540.01	0.00380733405547405\\
541.01	0.00374443191699418\\
542.01	0.00368291726258625\\
543.01	0.00362294390816676\\
544.01	0.00356455659307963\\
545.01	0.00350696477086615\\
546.01	0.00344975170545223\\
547.01	0.00339298951158971\\
548.01	0.00333673215405864\\
549.01	0.00328100951661619\\
550.01	0.00322582173290015\\
551.01	0.00317113412249031\\
552.01	0.00311687052003248\\
553.01	0.00306290970465843\\
554.01	0.00300908991410397\\
555.01	0.00295523201111278\\
556.01	0.00290124563021432\\
557.01	0.002847098072094\\
558.01	0.00279274702632913\\
559.01	0.00273814046510795\\
560.01	0.00268321858787583\\
561.01	0.00262791619871552\\
562.01	0.00257216649574095\\
563.01	0.00251590691454799\\
564.01	0.00245908556494755\\
565.01	0.00240166502778602\\
566.01	0.00234361427178739\\
567.01	0.0022849016921835\\
568.01	0.0022254958877592\\
569.01	0.00216536696107141\\
570.01	0.0021044878379189\\
571.01	0.00204283533477184\\
572.01	0.00198039060692598\\
573.01	0.00191713873061543\\
574.01	0.0018530674700027\\
575.01	0.00178816633815506\\
576.01	0.00172242682014106\\
577.01	0.00165584284131613\\
578.01	0.001588411129313\\
579.01	0.00152013143583996\\
580.01	0.00145100662154961\\
581.01	0.00138104265774608\\
582.01	0.00131024865991341\\
583.01	0.0012386370874182\\
584.01	0.00116622409072038\\
585.01	0.00109302987562263\\
586.01	0.001019079034026\\
587.01	0.000944400819187679\\
588.01	0.000869029345008453\\
589.01	0.000793003687887234\\
590.01	0.000716367854795577\\
591.01	0.000639170551898716\\
592.01	0.000561464658762556\\
593.01	0.000483306311701305\\
594.01	0.000404753505687408\\
595.01	0.000325864119007519\\
596.01	0.000246693254416338\\
597.01	0.000167289772990669\\
598.01	9.17366978681194e-05\\
599.01	2.94669364335823e-05\\
599.02	2.89574269646022e-05\\
599.03	2.84509530909024e-05\\
599.04	2.79475443514862e-05\\
599.05	2.74472305761501e-05\\
599.06	2.69500418883586e-05\\
599.07	2.64560087081516e-05\\
599.08	2.59651617550757e-05\\
599.09	2.54775320511456e-05\\
599.1	2.49931509238403e-05\\
599.11	2.45120500091244e-05\\
599.12	2.40342612544929e-05\\
599.13	2.35598169220554e-05\\
599.14	2.30887495916553e-05\\
599.15	2.26210921640043e-05\\
599.16	2.21568778638551e-05\\
599.17	2.16961402432179e-05\\
599.18	2.12389131845833e-05\\
599.19	2.07852309042009e-05\\
599.2	2.03351279553787e-05\\
599.21	1.98886392318159e-05\\
599.22	1.94457999709644e-05\\
599.23	1.90066457574443e-05\\
599.24	1.85712125264602e-05\\
599.25	1.81395365672823e-05\\
599.26	1.77116564104504e-05\\
599.27	1.72876122143499e-05\\
599.28	1.68674445369076e-05\\
599.29	1.64511943395312e-05\\
599.3	1.60389029911102e-05\\
599.31	1.56306122720377e-05\\
599.32	1.52263643782673e-05\\
599.33	1.48262019254389e-05\\
599.34	1.44301679530119e-05\\
599.35	1.40383059284606e-05\\
599.36	1.36506597515022e-05\\
599.37	1.32672737583623e-05\\
599.38	1.28881927261015e-05\\
599.39	1.25134618769607e-05\\
599.4	1.21431268827639e-05\\
599.41	1.17772338693641e-05\\
599.42	1.14158294211263e-05\\
599.43	1.10589605854527e-05\\
599.44	1.0706674877372e-05\\
599.45	1.03590202841473e-05\\
599.46	1.00160452699508e-05\\
599.47	9.6777987805708e-06\\
599.48	9.34433024817223e-06\\
599.49	9.01568959610181e-06\\
599.5	8.6919272437435e-06\\
599.51	8.37309411141907e-06\\
599.52	8.05924162533209e-06\\
599.53	7.75042172257087e-06\\
599.54	7.44668685616172e-06\\
599.55	7.14809000015512e-06\\
599.56	6.85468465477443e-06\\
599.57	6.56652485162869e-06\\
599.58	6.28366515894284e-06\\
599.59	6.00616068687464e-06\\
599.6	5.73406709285587e-06\\
599.61	5.4674405870099e-06\\
599.62	5.20633793760911e-06\\
599.63	4.95081647658262e-06\\
599.64	4.70093410509e-06\\
599.65	4.45674929914694e-06\\
599.66	4.21832111529435e-06\\
599.67	3.9857091963455e-06\\
599.68	3.75897377716608e-06\\
599.69	3.53817569053415e-06\\
599.7	3.32337637303469e-06\\
599.71	3.11463787103054e-06\\
599.72	2.91202284668536e-06\\
599.73	2.71559458404555e-06\\
599.74	2.52541699518466e-06\\
599.75	2.34155462640155e-06\\
599.76	2.16407266449489e-06\\
599.77	1.99303694307755e-06\\
599.78	1.82851394897598e-06\\
599.79	1.67057082867302e-06\\
599.8	1.51927539483211e-06\\
599.81	1.37469613287027e-06\\
599.82	1.23690220760718e-06\\
599.83	1.10596346996998e-06\\
599.84	9.81950463782924e-07\\
599.85	8.64934432591793e-07\\
599.86	7.54987326588921e-07\\
599.87	6.5218180958504e-07\\
599.88	5.56591266064402e-07\\
599.89	4.68289808295413e-07\\
599.9	3.87352283521061e-07\\
599.91	3.13854281216996e-07\\
599.92	2.4787214042421e-07\\
599.93	1.8948295714763e-07\\
599.94	1.38764591834512e-07\\
599.95	9.57956769204876e-08\\
599.96	6.06556244606149e-08\\
599.97	3.34246338194039e-08\\
599.98	1.41836994527883e-08\\
599.99	3.01461875948372e-09\\
600	0\\
};
\addplot [color=red!50!mycolor17,solid,forget plot]
  table[row sep=crcr]{%
0.01	0.00656440582368625\\
1.01	0.00656440546540022\\
2.01	0.00656440509943144\\
3.01	0.00656440472561442\\
4.01	0.00656440434378001\\
5.01	0.00656440395375523\\
6.01	0.00656440355536398\\
7.01	0.00656440314842576\\
8.01	0.00656440273275647\\
9.01	0.00656440230816771\\
10.01	0.00656440187446719\\
11.01	0.00656440143145865\\
12.01	0.00656440097894109\\
13.01	0.00656440051670957\\
14.01	0.00656440004455453\\
15.01	0.00656439956226184\\
16.01	0.00656439906961291\\
17.01	0.0065643985663842\\
18.01	0.00656439805234737\\
19.01	0.00656439752726918\\
20.01	0.00656439699091115\\
21.01	0.00656439644302987\\
22.01	0.00656439588337666\\
23.01	0.00656439531169723\\
24.01	0.00656439472773186\\
25.01	0.00656439413121509\\
26.01	0.00656439352187586\\
27.01	0.00656439289943711\\
28.01	0.00656439226361582\\
29.01	0.00656439161412292\\
30.01	0.00656439095066264\\
31.01	0.00656439027293327\\
32.01	0.0065643895806263\\
33.01	0.00656438887342651\\
34.01	0.00656438815101185\\
35.01	0.00656438741305317\\
36.01	0.00656438665921446\\
37.01	0.00656438588915204\\
38.01	0.00656438510251502\\
39.01	0.00656438429894477\\
40.01	0.00656438347807471\\
41.01	0.00656438263953042\\
42.01	0.00656438178292943\\
43.01	0.00656438090788055\\
44.01	0.00656438001398459\\
45.01	0.0065643791008333\\
46.01	0.00656437816800967\\
47.01	0.00656437721508744\\
48.01	0.00656437624163129\\
49.01	0.00656437524719617\\
50.01	0.00656437423132769\\
51.01	0.0065643731935612\\
52.01	0.00656437213342191\\
53.01	0.00656437105042498\\
54.01	0.00656436994407461\\
55.01	0.00656436881386433\\
56.01	0.00656436765927691\\
57.01	0.00656436647978339\\
58.01	0.00656436527484341\\
59.01	0.00656436404390475\\
60.01	0.00656436278640348\\
61.01	0.0065643615017627\\
62.01	0.0065643601893935\\
63.01	0.0065643588486938\\
64.01	0.00656435747904851\\
65.01	0.00656435607982905\\
66.01	0.0065643546503928\\
67.01	0.00656435319008354\\
68.01	0.00656435169823045\\
69.01	0.00656435017414805\\
70.01	0.00656434861713593\\
71.01	0.00656434702647833\\
72.01	0.00656434540144379\\
73.01	0.00656434374128471\\
74.01	0.00656434204523739\\
75.01	0.00656434031252119\\
76.01	0.00656433854233837\\
77.01	0.00656433673387389\\
78.01	0.00656433488629476\\
79.01	0.00656433299874963\\
80.01	0.00656433107036859\\
81.01	0.00656432910026258\\
82.01	0.00656432708752323\\
83.01	0.00656432503122192\\
84.01	0.00656432293040998\\
85.01	0.00656432078411769\\
86.01	0.0065643185913541\\
87.01	0.00656431635110668\\
88.01	0.00656431406234046\\
89.01	0.00656431172399769\\
90.01	0.00656430933499771\\
91.01	0.0065643068942357\\
92.01	0.00656430440058294\\
93.01	0.00656430185288571\\
94.01	0.00656429924996476\\
95.01	0.00656429659061535\\
96.01	0.0065642938736059\\
97.01	0.00656429109767786\\
98.01	0.00656428826154488\\
99.01	0.00656428536389257\\
100.01	0.00656428240337719\\
101.01	0.00656427937862584\\
102.01	0.0065642762882352\\
103.01	0.00656427313077095\\
104.01	0.00656426990476732\\
105.01	0.00656426660872626\\
106.01	0.00656426324111655\\
107.01	0.00656425980037325\\
108.01	0.00656425628489711\\
109.01	0.00656425269305348\\
110.01	0.00656424902317158\\
111.01	0.00656424527354389\\
112.01	0.00656424144242516\\
113.01	0.00656423752803153\\
114.01	0.00656423352854006\\
115.01	0.00656422944208717\\
116.01	0.00656422526676833\\
117.01	0.00656422100063694\\
118.01	0.00656421664170345\\
119.01	0.00656421218793391\\
120.01	0.00656420763725004\\
121.01	0.00656420298752725\\
122.01	0.00656419823659398\\
123.01	0.00656419338223057\\
124.01	0.0065641884221685\\
125.01	0.00656418335408863\\
126.01	0.00656417817562074\\
127.01	0.0065641728843422\\
128.01	0.00656416747777677\\
129.01	0.00656416195339289\\
130.01	0.00656415630860342\\
131.01	0.00656415054076357\\
132.01	0.00656414464717013\\
133.01	0.00656413862505968\\
134.01	0.00656413247160771\\
135.01	0.00656412618392702\\
136.01	0.00656411975906615\\
137.01	0.00656411319400827\\
138.01	0.00656410648566962\\
139.01	0.00656409963089772\\
140.01	0.00656409262647038\\
141.01	0.00656408546909342\\
142.01	0.00656407815539972\\
143.01	0.00656407068194717\\
144.01	0.00656406304521711\\
145.01	0.00656405524161271\\
146.01	0.00656404726745693\\
147.01	0.00656403911899101\\
148.01	0.00656403079237245\\
149.01	0.00656402228367297\\
150.01	0.00656401358887737\\
151.01	0.00656400470388014\\
152.01	0.00656399562448489\\
153.01	0.00656398634640142\\
154.01	0.00656397686524381\\
155.01	0.00656396717652843\\
156.01	0.00656395727567134\\
157.01	0.00656394715798651\\
158.01	0.00656393681868305\\
159.01	0.00656392625286333\\
160.01	0.00656391545552006\\
161.01	0.0065639044215341\\
162.01	0.0065638931456716\\
163.01	0.00656388162258207\\
164.01	0.00656386984679521\\
165.01	0.00656385781271792\\
166.01	0.00656384551463225\\
167.01	0.00656383294669207\\
168.01	0.0065638201029203\\
169.01	0.00656380697720591\\
170.01	0.0065637935633009\\
171.01	0.00656377985481698\\
172.01	0.00656376584522262\\
173.01	0.00656375152784004\\
174.01	0.00656373689584109\\
175.01	0.00656372194224441\\
176.01	0.00656370665991189\\
177.01	0.00656369104154491\\
178.01	0.00656367507968084\\
179.01	0.00656365876668904\\
180.01	0.00656364209476747\\
181.01	0.00656362505593814\\
182.01	0.00656360764204395\\
183.01	0.00656358984474339\\
184.01	0.00656357165550749\\
185.01	0.00656355306561492\\
186.01	0.00656353406614752\\
187.01	0.00656351464798604\\
188.01	0.00656349480180533\\
189.01	0.00656347451806996\\
190.01	0.00656345378702875\\
191.01	0.00656343259871038\\
192.01	0.00656341094291802\\
193.01	0.00656338880922434\\
194.01	0.00656336618696599\\
195.01	0.00656334306523831\\
196.01	0.00656331943288979\\
197.01	0.00656329527851595\\
198.01	0.00656327059045398\\
199.01	0.00656324535677658\\
200.01	0.00656321956528616\\
201.01	0.00656319320350774\\
202.01	0.00656316625868347\\
203.01	0.00656313871776556\\
204.01	0.00656311056740936\\
205.01	0.00656308179396725\\
206.01	0.00656305238348084\\
207.01	0.00656302232167389\\
208.01	0.00656299159394528\\
209.01	0.00656296018536091\\
210.01	0.00656292808064631\\
211.01	0.00656289526417888\\
212.01	0.00656286171997907\\
213.01	0.0065628274317028\\
214.01	0.00656279238263234\\
215.01	0.0065627565556678\\
216.01	0.0065627199333185\\
217.01	0.00656268249769313\\
218.01	0.00656264423049121\\
219.01	0.00656260511299261\\
220.01	0.00656256512604795\\
221.01	0.00656252425006915\\
222.01	0.00656248246501819\\
223.01	0.00656243975039708\\
224.01	0.00656239608523671\\
225.01	0.00656235144808599\\
226.01	0.00656230581700023\\
227.01	0.00656225916952966\\
228.01	0.00656221148270753\\
229.01	0.00656216273303747\\
230.01	0.00656211289648114\\
231.01	0.00656206194844563\\
232.01	0.00656200986377009\\
233.01	0.00656195661671203\\
234.01	0.00656190218093363\\
235.01	0.00656184652948781\\
236.01	0.00656178963480315\\
237.01	0.00656173146866952\\
238.01	0.00656167200222242\\
239.01	0.00656161120592764\\
240.01	0.00656154904956486\\
241.01	0.00656148550221161\\
242.01	0.00656142053222612\\
243.01	0.00656135410723006\\
244.01	0.0065612861940909\\
245.01	0.00656121675890383\\
246.01	0.00656114576697296\\
247.01	0.00656107318279228\\
248.01	0.00656099897002596\\
249.01	0.00656092309148856\\
250.01	0.00656084550912443\\
251.01	0.00656076618398621\\
252.01	0.00656068507621333\\
253.01	0.00656060214501012\\
254.01	0.00656051734862315\\
255.01	0.00656043064431713\\
256.01	0.00656034198835127\\
257.01	0.00656025133595515\\
258.01	0.00656015864130328\\
259.01	0.0065600638574892\\
260.01	0.00655996693649861\\
261.01	0.00655986782918266\\
262.01	0.00655976648523\\
263.01	0.00655966285313819\\
264.01	0.00655955688018397\\
265.01	0.00655944851239334\\
266.01	0.00655933769451097\\
267.01	0.00655922436996773\\
268.01	0.00655910848084881\\
269.01	0.00655898996786009\\
270.01	0.00655886877029374\\
271.01	0.00655874482599301\\
272.01	0.00655861807131657\\
273.01	0.00655848844110089\\
274.01	0.00655835586862248\\
275.01	0.00655822028555866\\
276.01	0.00655808162194776\\
277.01	0.00655793980614784\\
278.01	0.00655779476479435\\
279.01	0.00655764642275691\\
280.01	0.00655749470309495\\
281.01	0.00655733952701165\\
282.01	0.00655718081380714\\
283.01	0.00655701848083072\\
284.01	0.00655685244343093\\
285.01	0.00655668261490464\\
286.01	0.00655650890644569\\
287.01	0.00655633122709056\\
288.01	0.00655614948366405\\
289.01	0.0065559635807221\\
290.01	0.00655577342049454\\
291.01	0.00655557890282469\\
292.01	0.00655537992510951\\
293.01	0.00655517638223547\\
294.01	0.00655496816651483\\
295.01	0.00655475516761933\\
296.01	0.00655453727251183\\
297.01	0.00655431436537697\\
298.01	0.0065540863275489\\
299.01	0.00655385303743793\\
300.01	0.00655361437045451\\
301.01	0.00655337019893227\\
302.01	0.00655312039204702\\
303.01	0.00655286481573576\\
304.01	0.00655260333261192\\
305.01	0.00655233580187901\\
306.01	0.00655206207924174\\
307.01	0.00655178201681469\\
308.01	0.00655149546302884\\
309.01	0.00655120226253515\\
310.01	0.00655090225610571\\
311.01	0.00655059528053253\\
312.01	0.0065502811685226\\
313.01	0.00654995974859157\\
314.01	0.00654963084495366\\
315.01	0.00654929427740841\\
316.01	0.0065489498612254\\
317.01	0.00654859740702488\\
318.01	0.00654823672065572\\
319.01	0.00654786760307014\\
320.01	0.00654748985019475\\
321.01	0.00654710325279886\\
322.01	0.0065467075963584\\
323.01	0.00654630266091722\\
324.01	0.00654588822094412\\
325.01	0.00654546404518628\\
326.01	0.00654502989651933\\
327.01	0.00654458553179271\\
328.01	0.00654413070167154\\
329.01	0.0065436651504752\\
330.01	0.00654318861601012\\
331.01	0.00654270082939996\\
332.01	0.0065422015149103\\
333.01	0.00654169038976984\\
334.01	0.00654116716398685\\
335.01	0.00654063154016085\\
336.01	0.00654008321329051\\
337.01	0.00653952187057521\\
338.01	0.00653894719121438\\
339.01	0.00653835884619936\\
340.01	0.00653775649810243\\
341.01	0.00653713980086034\\
342.01	0.00653650839955177\\
343.01	0.00653586193017142\\
344.01	0.0065352000193979\\
345.01	0.00653452228435681\\
346.01	0.00653382833237883\\
347.01	0.00653311776075192\\
348.01	0.00653239015646936\\
349.01	0.00653164509597112\\
350.01	0.00653088214488131\\
351.01	0.00653010085773923\\
352.01	0.00652930077772563\\
353.01	0.00652848143638347\\
354.01	0.00652764235333345\\
355.01	0.0065267830359844\\
356.01	0.00652590297923764\\
357.01	0.00652500166518708\\
358.01	0.00652407856281347\\
359.01	0.00652313312767335\\
360.01	0.00652216480158361\\
361.01	0.00652117301230029\\
362.01	0.00652015717319252\\
363.01	0.00651911668291189\\
364.01	0.00651805092505683\\
365.01	0.0065169592678316\\
366.01	0.00651584106370219\\
367.01	0.0065146956490462\\
368.01	0.00651352234379845\\
369.01	0.00651232045109326\\
370.01	0.00651108925690145\\
371.01	0.00650982802966368\\
372.01	0.00650853601991949\\
373.01	0.00650721245993259\\
374.01	0.00650585656331152\\
375.01	0.0065044675246271\\
376.01	0.00650304451902536\\
377.01	0.00650158670183648\\
378.01	0.00650009320817941\\
379.01	0.00649856315256223\\
380.01	0.00649699562847678\\
381.01	0.00649538970798896\\
382.01	0.0064937444413224\\
383.01	0.00649205885643525\\
384.01	0.00649033195858945\\
385.01	0.00648856272991013\\
386.01	0.00648675012893443\\
387.01	0.00648489309014677\\
388.01	0.00648299052349833\\
389.01	0.00648104131390808\\
390.01	0.00647904432073992\\
391.01	0.00647699837725221\\
392.01	0.00647490229001441\\
393.01	0.00647275483828189\\
394.01	0.00647055477332394\\
395.01	0.00646830081769172\\
396.01	0.00646599166441805\\
397.01	0.00646362597613286\\
398.01	0.00646120238408305\\
399.01	0.00645871948703488\\
400.01	0.00645617585004426\\
401.01	0.00645357000307023\\
402.01	0.00645090043941078\\
403.01	0.00644816561393521\\
404.01	0.00644536394109067\\
405.01	0.00644249379265812\\
406.01	0.00643955349523729\\
407.01	0.00643654132744707\\
408.01	0.00643345551683574\\
409.01	0.00643029423650974\\
410.01	0.00642705560150948\\
411.01	0.00642373766498846\\
412.01	0.00642033841428345\\
413.01	0.00641685576700698\\
414.01	0.00641328756733783\\
415.01	0.00640963158272969\\
416.01	0.00640588550129285\\
417.01	0.00640204693010357\\
418.01	0.00639811339462549\\
419.01	0.00639408233923406\\
420.01	0.00638995112838307\\
421.01	0.00638571704692442\\
422.01	0.00638137729839664\\
423.01	0.00637692900261399\\
424.01	0.00637236919322113\\
425.01	0.00636769481525123\\
426.01	0.00636290272270502\\
427.01	0.00635798967617175\\
428.01	0.00635295234051363\\
429.01	0.00634778728264398\\
430.01	0.00634249096942608\\
431.01	0.006337059765727\\
432.01	0.00633148993266037\\
433.01	0.00632577762605561\\
434.01	0.00631991889518877\\
435.01	0.00631390968181004\\
436.01	0.00630774581949997\\
437.01	0.00630142303337773\\
438.01	0.00629493694017861\\
439.01	0.00628828304870329\\
440.01	0.00628145676063154\\
441.01	0.00627445337167707\\
442.01	0.00626726807305398\\
443.01	0.00625989595323021\\
444.01	0.00625233199997218\\
445.01	0.0062445711027621\\
446.01	0.00623660805584039\\
447.01	0.00622843756253448\\
448.01	0.00622005424148167\\
449.01	0.00621145263441838\\
450.01	0.00620262721545311\\
451.01	0.00619357240196595\\
452.01	0.00618428256728609\\
453.01	0.00617475205529724\\
454.01	0.00616497519710713\\
455.01	0.00615494632990128\\
456.01	0.00614465981806608\\
457.01	0.00613411007662122\\
458.01	0.00612329159692982\\
459.01	0.00611219897456385\\
460.01	0.00610082693907493\\
461.01	0.00608917038525529\\
462.01	0.00607722440526771\\
463.01	0.00606498432075847\\
464.01	0.00605244571375353\\
465.01	0.00603960445476142\\
466.01	0.00602645672608229\\
467.01	0.00601299903786675\\
468.01	0.00599922823402013\\
469.01	0.00598514148467885\\
470.01	0.00597073626182786\\
471.01	0.00595601029487695\\
472.01	0.00594096150400683\\
473.01	0.0059255879113317\\
474.01	0.00590988753420292\\
475.01	0.00589385827255953\\
476.01	0.00587749783778725\\
477.01	0.00586080373908369\\
478.01	0.00584377299093302\\
479.01	0.00582640150876262\\
480.01	0.0058086831876211\\
481.01	0.00579060853200369\\
482.01	0.00577216432972964\\
483.01	0.00575334582220765\\
484.01	0.00573415424939939\\
485.01	0.00571459079020263\\
486.01	0.00569465572021814\\
487.01	0.00567434718468358\\
488.01	0.00565365936006162\\
489.01	0.00563257975556609\\
490.01	0.00561108531160021\\
491.01	0.00558913682306404\\
492.01	0.00556667103921158\\
493.01	0.00554358997176287\\
494.01	0.00551979880560138\\
495.01	0.00549527325324063\\
496.01	0.00546999979739121\\
497.01	0.00544396646201648\\
498.01	0.00541716238647811\\
499.01	0.00538957684242895\\
500.01	0.00536119737711359\\
501.01	0.00533200587711989\\
502.01	0.00530196903533774\\
503.01	0.00527104449029588\\
504.01	0.0052391864232501\\
505.01	0.00520634535208319\\
506.01	0.00517246793809063\\
507.01	0.00513749692894901\\
508.01	0.00510137127455286\\
509.01	0.00506402650617743\\
510.01	0.00502539551200544\\
511.01	0.00498540989747207\\
512.01	0.0049440021960149\\
513.01	0.0049011093030496\\
514.01	0.00485667765489779\\
515.01	0.0048106708810639\\
516.01	0.00476308094499622\\
517.01	0.00471394418628692\\
518.01	0.00466336422930108\\
519.01	0.00461186793412728\\
520.01	0.0045599046924575\\
521.01	0.00450754211297316\\
522.01	0.00445485964823526\\
523.01	0.00440195105852912\\
524.01	0.00434892780644088\\
525.01	0.00429592382681327\\
526.01	0.00424309974104571\\
527.01	0.00419063659086737\\
528.01	0.00413872705740767\\
529.01	0.0040875654696395\\
530.01	0.00403733280501714\\
531.01	0.00398817333943129\\
532.01	0.00394015889021307\\
533.01	0.0038927890665121\\
534.01	0.0038456196551268\\
535.01	0.00379871775713688\\
536.01	0.00375214226384924\\
537.01	0.00370593984677847\\
538.01	0.00366014038052297\\
539.01	0.00361475198846775\\
540.01	0.0035697561020783\\
541.01	0.00352510323923424\\
542.01	0.00348071068926054\\
543.01	0.00343646401783639\\
544.01	0.00339222542361496\\
545.01	0.00334788219760175\\
546.01	0.00330340572108235\\
547.01	0.00325877291055184\\
548.01	0.00321395096245914\\
549.01	0.00316889777513852\\
550.01	0.0031235631575099\\
551.01	0.00307789096322181\\
552.01	0.00303182227862506\\
553.01	0.00298529967226709\\
554.01	0.00293827203692452\\
555.01	0.00289069883880613\\
556.01	0.00284254826966974\\
557.01	0.00279378882254609\\
558.01	0.0027443879232907\\
559.01	0.00269431286133482\\
560.01	0.00264353176750194\\
561.01	0.0025920145436866\\
562.01	0.00253973362071961\\
563.01	0.00248666432825441\\
564.01	0.00243278467556005\\
565.01	0.00237807450916724\\
566.01	0.00232251470725555\\
567.01	0.00226608722974788\\
568.01	0.00220877542917444\\
569.01	0.00215056429547898\\
570.01	0.0020914406062683\\
571.01	0.00203139297475291\\
572.01	0.00197041181888403\\
573.01	0.00190848931254016\\
574.01	0.00184561940995461\\
575.01	0.00178179800211486\\
576.01	0.00171702313729112\\
577.01	0.00165129523694475\\
578.01	0.00158461730238309\\
579.01	0.00151699512093703\\
580.01	0.00144843748360971\\
581.01	0.00137895642524441\\
582.01	0.00130856749049479\\
583.01	0.00123729001347867\\
584.01	0.00116514738699767\\
585.01	0.00109216730167909\\
586.01	0.00101838193976332\\
587.01	0.000943828105516345\\
588.01	0.000868547269223433\\
589.01	0.000792585494808212\\
590.01	0.000715993212731169\\
591.01	0.000638824791183846\\
592.01	0.000561137850788497\\
593.01	0.000482992259190443\\
594.01	0.000404448728692637\\
595.01	0.000325566921518283\\
596.01	0.000246402942558918\\
597.01	0.000167006059591517\\
598.01	9.17366970263709e-05\\
599.01	2.94669364272089e-05\\
599.02	2.89574269586573e-05\\
599.03	2.84509530853617e-05\\
599.04	2.79475443463271e-05\\
599.05	2.74472305713484e-05\\
599.06	2.69500418838952e-05\\
599.07	2.64560087040022e-05\\
599.08	2.59651617512246e-05\\
599.09	2.54775320475721e-05\\
599.1	2.49931509205322e-05\\
599.11	2.45120500060592e-05\\
599.12	2.40342612516566e-05\\
599.13	2.35598169194343e-05\\
599.14	2.30887495892354e-05\\
599.15	2.262109216177e-05\\
599.16	2.21568778617977e-05\\
599.17	2.16961402413236e-05\\
599.18	2.12389131828417e-05\\
599.19	2.07852309026015e-05\\
599.2	2.03351279539129e-05\\
599.21	1.98886392304715e-05\\
599.22	1.94457999697362e-05\\
599.23	1.90066457563202e-05\\
599.24	1.85712125254332e-05\\
599.25	1.81395365663438e-05\\
599.26	1.77116564095969e-05\\
599.27	1.72876122135745e-05\\
599.28	1.68674445362033e-05\\
599.29	1.64511943388929e-05\\
599.3	1.60389029905342e-05\\
599.31	1.56306122715139e-05\\
599.32	1.52263643777972e-05\\
599.33	1.48262019250139e-05\\
599.34	1.4430167952632e-05\\
599.35	1.40383059281188e-05\\
599.36	1.36506597511951e-05\\
599.37	1.32672737580882e-05\\
599.38	1.28881927258552e-05\\
599.39	1.25134618767422e-05\\
599.4	1.21431268825713e-05\\
599.41	1.17772338691941e-05\\
599.42	1.14158294209736e-05\\
599.43	1.10589605853192e-05\\
599.44	1.0706674877254e-05\\
599.45	1.03590202840433e-05\\
599.46	1.00160452698606e-05\\
599.47	9.67779878049101e-06\\
599.48	9.34433024810284e-06\\
599.49	9.01568959604283e-06\\
599.5	8.69192724369319e-06\\
599.51	8.37309411137396e-06\\
599.52	8.05924162529219e-06\\
599.53	7.75042172253965e-06\\
599.54	7.4466868561357e-06\\
599.55	7.14809000013084e-06\\
599.56	6.85468465475708e-06\\
599.57	6.56652485161308e-06\\
599.58	6.2836651589307e-06\\
599.59	6.00616068686249e-06\\
599.6	5.73406709284546e-06\\
599.61	5.46744058700296e-06\\
599.62	5.20633793760217e-06\\
599.63	4.95081647657741e-06\\
599.64	4.70093410508653e-06\\
599.65	4.45674929914174e-06\\
599.66	4.21832111529089e-06\\
599.67	3.98570919634203e-06\\
599.68	3.75897377716608e-06\\
599.69	3.53817569053415e-06\\
599.7	3.32337637303469e-06\\
599.71	3.11463787102881e-06\\
599.72	2.91202284668363e-06\\
599.73	2.71559458404382e-06\\
599.74	2.52541699518292e-06\\
599.75	2.34155462640155e-06\\
599.76	2.16407266449663e-06\\
599.77	1.99303694307928e-06\\
599.78	1.82851394897598e-06\\
599.79	1.67057082867302e-06\\
599.8	1.51927539483211e-06\\
599.81	1.37469613287027e-06\\
599.82	1.23690220760544e-06\\
599.83	1.10596346997172e-06\\
599.84	9.81950463782924e-07\\
599.85	8.64934432591793e-07\\
599.86	7.54987326587186e-07\\
599.87	6.5218180958504e-07\\
599.88	5.56591266062667e-07\\
599.89	4.68289808295413e-07\\
599.9	3.87352283521061e-07\\
599.91	3.13854281218731e-07\\
599.92	2.47872140425945e-07\\
599.93	1.89482957149364e-07\\
599.94	1.38764591834512e-07\\
599.95	9.57956769204876e-08\\
599.96	6.06556244606149e-08\\
599.97	3.34246338211386e-08\\
599.98	1.4183699454523e-08\\
599.99	3.014618757749e-09\\
600	0\\
};
\addplot [color=red!40!mycolor19,solid,forget plot]
  table[row sep=crcr]{%
0.01	0.00627501982079631\\
1.01	0.00627501944692668\\
2.01	0.00627501906508025\\
3.01	0.00627501867508635\\
4.01	0.00627501827677046\\
5.01	0.00627501786995453\\
6.01	0.00627501745445643\\
7.01	0.00627501703009011\\
8.01	0.00627501659666559\\
9.01	0.00627501615398896\\
10.01	0.00627501570186209\\
11.01	0.00627501524008224\\
12.01	0.00627501476844279\\
13.01	0.00627501428673242\\
14.01	0.0062750137947353\\
15.01	0.00627501329223091\\
16.01	0.00627501277899408\\
17.01	0.00627501225479479\\
18.01	0.00627501171939803\\
19.01	0.00627501117256362\\
20.01	0.00627501061404649\\
21.01	0.00627501004359601\\
22.01	0.00627500946095607\\
23.01	0.00627500886586557\\
24.01	0.00627500825805735\\
25.01	0.00627500763725846\\
26.01	0.00627500700319044\\
27.01	0.00627500635556846\\
28.01	0.00627500569410162\\
29.01	0.00627500501849288\\
30.01	0.00627500432843883\\
31.01	0.00627500362362923\\
32.01	0.00627500290374755\\
33.01	0.00627500216847012\\
34.01	0.00627500141746654\\
35.01	0.00627500065039911\\
36.01	0.00627499986692271\\
37.01	0.00627499906668511\\
38.01	0.00627499824932634\\
39.01	0.00627499741447863\\
40.01	0.00627499656176637\\
41.01	0.00627499569080558\\
42.01	0.0062749948012044\\
43.01	0.00627499389256226\\
44.01	0.00627499296446983\\
45.01	0.00627499201650924\\
46.01	0.00627499104825335\\
47.01	0.00627499005926611\\
48.01	0.00627498904910168\\
49.01	0.00627498801730491\\
50.01	0.00627498696341052\\
51.01	0.00627498588694327\\
52.01	0.0062749847874179\\
53.01	0.00627498366433837\\
54.01	0.00627498251719804\\
55.01	0.00627498134547947\\
56.01	0.00627498014865353\\
57.01	0.00627497892618032\\
58.01	0.00627497767750798\\
59.01	0.00627497640207243\\
60.01	0.00627497509929776\\
61.01	0.00627497376859569\\
62.01	0.00627497240936482\\
63.01	0.00627497102099114\\
64.01	0.00627496960284693\\
65.01	0.00627496815429119\\
66.01	0.00627496667466925\\
67.01	0.00627496516331188\\
68.01	0.00627496361953534\\
69.01	0.00627496204264142\\
70.01	0.00627496043191656\\
71.01	0.00627495878663203\\
72.01	0.00627495710604279\\
73.01	0.00627495538938831\\
74.01	0.00627495363589117\\
75.01	0.00627495184475727\\
76.01	0.00627495001517523\\
77.01	0.00627494814631603\\
78.01	0.00627494623733275\\
79.01	0.00627494428736043\\
80.01	0.00627494229551492\\
81.01	0.00627494026089314\\
82.01	0.00627493818257228\\
83.01	0.00627493605960982\\
84.01	0.00627493389104248\\
85.01	0.0062749316758862\\
86.01	0.00627492941313573\\
87.01	0.00627492710176376\\
88.01	0.00627492474072093\\
89.01	0.00627492232893507\\
90.01	0.00627491986531064\\
91.01	0.00627491734872842\\
92.01	0.00627491477804482\\
93.01	0.00627491215209159\\
94.01	0.006274909469675\\
95.01	0.00627490672957539\\
96.01	0.00627490393054669\\
97.01	0.00627490107131582\\
98.01	0.00627489815058196\\
99.01	0.00627489516701594\\
100.01	0.00627489211926003\\
101.01	0.00627488900592669\\
102.01	0.00627488582559843\\
103.01	0.00627488257682694\\
104.01	0.00627487925813229\\
105.01	0.00627487586800252\\
106.01	0.00627487240489285\\
107.01	0.00627486886722501\\
108.01	0.00627486525338602\\
109.01	0.00627486156172819\\
110.01	0.00627485779056785\\
111.01	0.00627485393818469\\
112.01	0.00627485000282107\\
113.01	0.00627484598268125\\
114.01	0.00627484187593009\\
115.01	0.00627483768069272\\
116.01	0.00627483339505341\\
117.01	0.00627482901705461\\
118.01	0.00627482454469642\\
119.01	0.00627481997593549\\
120.01	0.00627481530868351\\
121.01	0.00627481054080715\\
122.01	0.00627480567012622\\
123.01	0.00627480069441342\\
124.01	0.00627479561139276\\
125.01	0.00627479041873867\\
126.01	0.00627478511407482\\
127.01	0.00627477969497318\\
128.01	0.00627477415895298\\
129.01	0.00627476850347911\\
130.01	0.00627476272596112\\
131.01	0.00627475682375241\\
132.01	0.00627475079414839\\
133.01	0.00627474463438554\\
134.01	0.0062747383416401\\
135.01	0.00627473191302679\\
136.01	0.00627472534559735\\
137.01	0.00627471863633906\\
138.01	0.00627471178217341\\
139.01	0.00627470477995492\\
140.01	0.00627469762646923\\
141.01	0.00627469031843194\\
142.01	0.00627468285248695\\
143.01	0.00627467522520483\\
144.01	0.00627466743308101\\
145.01	0.00627465947253462\\
146.01	0.0062746513399064\\
147.01	0.00627464303145715\\
148.01	0.00627463454336603\\
149.01	0.00627462587172868\\
150.01	0.00627461701255539\\
151.01	0.00627460796176902\\
152.01	0.00627459871520338\\
153.01	0.00627458926860131\\
154.01	0.00627457961761231\\
155.01	0.00627456975779074\\
156.01	0.00627455968459374\\
157.01	0.00627454939337881\\
158.01	0.00627453887940217\\
159.01	0.0062745281378158\\
160.01	0.00627451716366581\\
161.01	0.0062745059518896\\
162.01	0.00627449449731388\\
163.01	0.00627448279465192\\
164.01	0.00627447083850095\\
165.01	0.00627445862334025\\
166.01	0.00627444614352765\\
167.01	0.0062744333932974\\
168.01	0.00627442036675739\\
169.01	0.00627440705788619\\
170.01	0.00627439346053031\\
171.01	0.00627437956840123\\
172.01	0.00627436537507274\\
173.01	0.0062743508739768\\
174.01	0.00627433605840198\\
175.01	0.00627432092148916\\
176.01	0.00627430545622857\\
177.01	0.00627428965545641\\
178.01	0.00627427351185153\\
179.01	0.00627425701793191\\
180.01	0.00627424016605103\\
181.01	0.00627422294839444\\
182.01	0.00627420535697528\\
183.01	0.00627418738363186\\
184.01	0.00627416902002222\\
185.01	0.00627415025762123\\
186.01	0.00627413108771593\\
187.01	0.00627411150140158\\
188.01	0.00627409148957742\\
189.01	0.00627407104294207\\
190.01	0.00627405015198938\\
191.01	0.00627402880700365\\
192.01	0.00627400699805497\\
193.01	0.00627398471499471\\
194.01	0.00627396194745011\\
195.01	0.00627393868481967\\
196.01	0.00627391491626796\\
197.01	0.00627389063072046\\
198.01	0.00627386581685818\\
199.01	0.00627384046311181\\
200.01	0.00627381455765685\\
201.01	0.00627378808840736\\
202.01	0.00627376104301\\
203.01	0.0062737334088386\\
204.01	0.00627370517298764\\
205.01	0.00627367632226584\\
206.01	0.00627364684319003\\
207.01	0.00627361672197865\\
208.01	0.00627358594454488\\
209.01	0.00627355449648979\\
210.01	0.00627352236309501\\
211.01	0.00627348952931607\\
212.01	0.00627345597977495\\
213.01	0.00627342169875202\\
214.01	0.00627338667017911\\
215.01	0.00627335087763107\\
216.01	0.00627331430431797\\
217.01	0.00627327693307661\\
218.01	0.0062732387463622\\
219.01	0.00627319972623976\\
220.01	0.00627315985437529\\
221.01	0.00627311911202647\\
222.01	0.00627307748003373\\
223.01	0.00627303493881053\\
224.01	0.0062729914683338\\
225.01	0.00627294704813376\\
226.01	0.0062729016572838\\
227.01	0.00627285527439051\\
228.01	0.00627280787758198\\
229.01	0.00627275944449813\\
230.01	0.00627270995227875\\
231.01	0.00627265937755205\\
232.01	0.00627260769642319\\
233.01	0.00627255488446222\\
234.01	0.00627250091669176\\
235.01	0.00627244576757419\\
236.01	0.00627238941099914\\
237.01	0.00627233182026998\\
238.01	0.00627227296809058\\
239.01	0.00627221282655082\\
240.01	0.00627215136711321\\
241.01	0.00627208856059781\\
242.01	0.00627202437716745\\
243.01	0.00627195878631262\\
244.01	0.00627189175683558\\
245.01	0.00627182325683449\\
246.01	0.00627175325368689\\
247.01	0.00627168171403308\\
248.01	0.00627160860375854\\
249.01	0.0062715338879766\\
250.01	0.00627145753100993\\
251.01	0.00627137949637256\\
252.01	0.00627129974675016\\
253.01	0.00627121824398126\\
254.01	0.00627113494903623\\
255.01	0.00627104982199781\\
256.01	0.00627096282203976\\
257.01	0.00627087390740491\\
258.01	0.00627078303538332\\
259.01	0.00627069016228966\\
260.01	0.00627059524344006\\
261.01	0.00627049823312814\\
262.01	0.00627039908460019\\
263.01	0.00627029775003053\\
264.01	0.00627019418049573\\
265.01	0.00627008832594792\\
266.01	0.00626998013518788\\
267.01	0.0062698695558371\\
268.01	0.00626975653430933\\
269.01	0.00626964101578122\\
270.01	0.00626952294416235\\
271.01	0.00626940226206434\\
272.01	0.00626927891076863\\
273.01	0.00626915283019496\\
274.01	0.00626902395886698\\
275.01	0.00626889223387859\\
276.01	0.00626875759085857\\
277.01	0.00626861996393404\\
278.01	0.00626847928569379\\
279.01	0.00626833548714984\\
280.01	0.00626818849769801\\
281.01	0.00626803824507829\\
282.01	0.00626788465533277\\
283.01	0.00626772765276327\\
284.01	0.00626756715988795\\
285.01	0.00626740309739626\\
286.01	0.00626723538410234\\
287.01	0.00626706393689815\\
288.01	0.00626688867070387\\
289.01	0.00626670949841873\\
290.01	0.00626652633086838\\
291.01	0.00626633907675284\\
292.01	0.0062661476425907\\
293.01	0.00626595193266438\\
294.01	0.00626575184896115\\
295.01	0.00626554729111439\\
296.01	0.00626533815634235\\
297.01	0.00626512433938498\\
298.01	0.00626490573243966\\
299.01	0.00626468222509425\\
300.01	0.00626445370425851\\
301.01	0.00626422005409342\\
302.01	0.0062639811559389\\
303.01	0.00626373688823883\\
304.01	0.00626348712646358\\
305.01	0.00626323174303124\\
306.01	0.00626297060722554\\
307.01	0.00626270358511182\\
308.01	0.00626243053945027\\
309.01	0.00626215132960651\\
310.01	0.00626186581146005\\
311.01	0.0062615738373084\\
312.01	0.0062612752557708\\
313.01	0.00626096991168605\\
314.01	0.00626065764600972\\
315.01	0.00626033829570683\\
316.01	0.0062600116936413\\
317.01	0.00625967766846247\\
318.01	0.00625933604448791\\
319.01	0.00625898664158239\\
320.01	0.00625862927503338\\
321.01	0.00625826375542209\\
322.01	0.0062578898884914\\
323.01	0.00625750747500859\\
324.01	0.00625711631062481\\
325.01	0.00625671618572884\\
326.01	0.00625630688529746\\
327.01	0.00625588818874049\\
328.01	0.00625545986974115\\
329.01	0.00625502169609042\\
330.01	0.00625457342951782\\
331.01	0.00625411482551534\\
332.01	0.00625364563315641\\
333.01	0.00625316559490887\\
334.01	0.00625267444644229\\
335.01	0.00625217191642857\\
336.01	0.00625165772633658\\
337.01	0.0062511315902205\\
338.01	0.00625059321450026\\
339.01	0.00625004229773692\\
340.01	0.00624947853039881\\
341.01	0.00624890159462121\\
342.01	0.00624831116395897\\
343.01	0.00624770690313027\\
344.01	0.00624708846775251\\
345.01	0.00624645550407053\\
346.01	0.00624580764867563\\
347.01	0.00624514452821613\\
348.01	0.00624446575909892\\
349.01	0.00624377094718198\\
350.01	0.00624305968745715\\
351.01	0.00624233156372373\\
352.01	0.00624158614825195\\
353.01	0.00624082300143695\\
354.01	0.00624004167144166\\
355.01	0.00623924169383069\\
356.01	0.00623842259119347\\
357.01	0.00623758387275602\\
358.01	0.00623672503398323\\
359.01	0.00623584555617051\\
360.01	0.00623494490602452\\
361.01	0.006234022535234\\
362.01	0.00623307788003002\\
363.01	0.00623211036073653\\
364.01	0.00623111938131073\\
365.01	0.00623010432887544\\
366.01	0.00622906457324128\\
367.01	0.00622799946642182\\
368.01	0.00622690834214188\\
369.01	0.00622579051533872\\
370.01	0.0062246452816595\\
371.01	0.00622347191695422\\
372.01	0.00622226967676832\\
373.01	0.00622103779583443\\
374.01	0.00621977548756801\\
375.01	0.00621848194356779\\
376.01	0.00621715633312541\\
377.01	0.00621579780274746\\
378.01	0.00621440547569387\\
379.01	0.00621297845153867\\
380.01	0.00621151580575779\\
381.01	0.00621001658934978\\
382.01	0.00620847982849738\\
383.01	0.00620690452427863\\
384.01	0.00620528965243403\\
385.01	0.00620363416320286\\
386.01	0.00620193698123763\\
387.01	0.00620019700560957\\
388.01	0.00619841310991871\\
389.01	0.00619658414252233\\
390.01	0.00619470892689802\\
391.01	0.00619278626215702\\
392.01	0.00619081492372418\\
393.01	0.00618879366420226\\
394.01	0.00618672121443493\\
395.01	0.00618459628478509\\
396.01	0.00618241756663875\\
397.01	0.00618018373414341\\
398.01	0.00617789344617994\\
399.01	0.00617554534856301\\
400.01	0.00617313807644601\\
401.01	0.00617067025689468\\
402.01	0.00616814051156823\\
403.01	0.00616554745941866\\
404.01	0.00616288971928154\\
405.01	0.00616016591218665\\
406.01	0.00615737466316107\\
407.01	0.00615451460222969\\
408.01	0.00615158436424575\\
409.01	0.00614858258710004\\
410.01	0.00614550790777582\\
411.01	0.00614235895564246\\
412.01	0.00613913434234199\\
413.01	0.00613583264763858\\
414.01	0.00613245240073966\\
415.01	0.00612899205692615\\
416.01	0.00612544996998135\\
417.01	0.00612182436207277\\
418.01	0.00611811329471075\\
419.01	0.00611431464759695\\
420.01	0.00611042611998717\\
421.01	0.00610644528345852\\
422.01	0.00610236962267547\\
423.01	0.00609819653509054\\
424.01	0.00609392332698215\\
425.01	0.00608954720927813\\
426.01	0.00608506529315244\\
427.01	0.00608047458538285\\
428.01	0.00607577198345746\\
429.01	0.00607095427041315\\
430.01	0.00606601810939559\\
431.01	0.0060609600379238\\
432.01	0.0060557764618456\\
433.01	0.00605046364896941\\
434.01	0.00604501772236019\\
435.01	0.00603943465328612\\
436.01	0.00603371025380779\\
437.01	0.00602784016900473\\
438.01	0.00602181986883974\\
439.01	0.00601564463967026\\
440.01	0.00600930957542628\\
441.01	0.00600280956849132\\
442.01	0.00599613930034045\\
443.01	0.00598929323201499\\
444.01	0.0059822655945433\\
445.01	0.0059750503794517\\
446.01	0.00596764132954422\\
447.01	0.0059600319301649\\
448.01	0.00595221540118198\\
449.01	0.00594418468999795\\
450.01	0.00593593246598146\\
451.01	0.00592745111681121\\
452.01	0.00591873274732945\\
453.01	0.00590976918162688\\
454.01	0.00590055196922702\\
455.01	0.00589107239639559\\
456.01	0.00588132150378528\\
457.01	0.00587129011181471\\
458.01	0.00586096885537762\\
459.01	0.00585034822966616\\
460.01	0.00583941864904041\\
461.01	0.00582817052095897\\
462.01	0.00581659433692928\\
463.01	0.00580468078216617\\
464.01	0.00579242086502429\\
465.01	0.00577980606610628\\
466.01	0.0057668285049653\\
467.01	0.00575348111911687\\
468.01	0.0057397578450755\\
469.01	0.00572565378352102\\
470.01	0.00571116531930866\\
471.01	0.00569629015022579\\
472.01	0.00568102715383578\\
473.01	0.00566537598617683\\
474.01	0.00564933625488107\\
475.01	0.00563290599850248\\
476.01	0.00561607703700144\\
477.01	0.00559883378959261\\
478.01	0.00558115863084415\\
479.01	0.00556303221632345\\
480.01	0.00554443329402765\\
481.01	0.00552533853042991\\
482.01	0.00550572236828859\\
483.01	0.00548555671834599\\
484.01	0.00546481037634793\\
485.01	0.0054434485964233\\
486.01	0.00542143271600303\\
487.01	0.00539871981429891\\
488.01	0.00537526245314433\\
489.01	0.0053510085754977\\
490.01	0.00532590167516816\\
491.01	0.00529988140633078\\
492.01	0.00527288488008312\\
493.01	0.00524484900592452\\
494.01	0.00521571354443454\\
495.01	0.00518542144114009\\
496.01	0.00515392022106027\\
497.01	0.00512116654246566\\
498.01	0.00508713295858186\\
499.01	0.00505181766150161\\
500.01	0.00501525838772721\\
501.01	0.00497766164855981\\
502.01	0.00493943528264369\\
503.01	0.00490062690988339\\
504.01	0.00486126761608939\\
505.01	0.00482139625878809\\
506.01	0.00478106053220624\\
507.01	0.00474031806455034\\
508.01	0.00469923749003392\\
509.01	0.00465789940415897\\
510.01	0.00461639706089578\\
511.01	0.00457483659870346\\
512.01	0.00453333647999093\\
513.01	0.0044920256832665\\
514.01	0.00445103998170064\\
515.01	0.00441051535217822\\
516.01	0.00437057715191148\\
517.01	0.00433132312909362\\
518.01	0.00429279753583172\\
519.01	0.00425462094943984\\
520.01	0.00421646388263337\\
521.01	0.00417838022721257\\
522.01	0.004140422850745\\
523.01	0.00410264178097458\\
524.01	0.00406508188635623\\
525.01	0.00402777995207816\\
526.01	0.00399076106794841\\
527.01	0.00395403462590486\\
528.01	0.00391759050822663\\
529.01	0.00388139584061086\\
530.01	0.00384539310379418\\
531.01	0.00380950096592413\\
532.01	0.00377361999606866\\
533.01	0.00373766221781281\\
534.01	0.00370160676146456\\
535.01	0.00366544339044638\\
536.01	0.00362915467612006\\
537.01	0.00359271576305016\\
538.01	0.00355609454479962\\
539.01	0.00351925238193505\\
540.01	0.00348214548655068\\
541.01	0.00344472705759396\\
542.01	0.00340695015404107\\
543.01	0.0033687711001213\\
544.01	0.0033301528691409\\
545.01	0.0032910665487296\\
546.01	0.00325148580905002\\
547.01	0.00321138269631842\\
548.01	0.0031707279562313\\
549.01	0.00312949174391565\\
550.01	0.00308764439343728\\
551.01	0.00304515716922659\\
552.01	0.00300200289208221\\
553.01	0.00295815630627339\\
554.01	0.00291359405328688\\
555.01	0.00286829417430089\\
556.01	0.00282223533915401\\
557.01	0.0027753965662475\\
558.01	0.00272775741296249\\
559.01	0.00267929817756782\\
560.01	0.00263000004395317\\
561.01	0.00257984515678967\\
562.01	0.00252881662416956\\
563.01	0.00247689846218247\\
564.01	0.00242407552049739\\
565.01	0.00237033344952677\\
566.01	0.00231565875634426\\
567.01	0.00226003891137813\\
568.01	0.0022034624518856\\
569.01	0.0021459190761402\\
570.01	0.0020873997344778\\
571.01	0.00202789672639428\\
572.01	0.00196740381416456\\
573.01	0.00190591636143106\\
574.01	0.00184343149881933\\
575.01	0.00177994830972369\\
576.01	0.00171546802767963\\
577.01	0.00164999424345148\\
578.01	0.00158353312275398\\
579.01	0.00151609363483733\\
580.01	0.00144768779049829\\
581.01	0.0013783308857227\\
582.01	0.00130804174453238\\
583.01	0.00123684295248754\\
584.01	0.00116476107120689\\
585.01	0.00109182682306618\\
586.01	0.00101807523268008\\
587.01	0.000943545708142529\\
588.01	0.000868282040487625\\
589.01	0.000792332294341591\\
590.01	0.000715748556112592\\
591.01	0.000638586497983857\\
592.01	0.00056090470582257\\
593.01	0.00048276370607613\\
594.01	0.000404224610087212\\
595.01	0.000325347273344155\\
596.01	0.00024618784114796\\
597.01	0.000166807993572935\\
598.01	9.17366970068743e-05\\
599.01	2.94669364271152e-05\\
599.02	2.89574269585705e-05\\
599.03	2.84509530852819e-05\\
599.04	2.79475443462525e-05\\
599.05	2.7447230571279e-05\\
599.06	2.69500418838293e-05\\
599.07	2.64560087039449e-05\\
599.08	2.59651617511691e-05\\
599.09	2.54775320475235e-05\\
599.1	2.49931509204836e-05\\
599.11	2.45120500060158e-05\\
599.12	2.40342612516167e-05\\
599.13	2.35598169193996e-05\\
599.14	2.30887495892024e-05\\
599.15	2.26210921617405e-05\\
599.16	2.215687786177e-05\\
599.17	2.16961402412976e-05\\
599.18	2.12389131828191e-05\\
599.19	2.07852309025824e-05\\
599.2	2.03351279538938e-05\\
599.21	1.98886392304524e-05\\
599.22	1.94457999697188e-05\\
599.23	1.90066457563063e-05\\
599.24	1.85712125254194e-05\\
599.25	1.81395365663334e-05\\
599.26	1.77116564095865e-05\\
599.27	1.72876122135658e-05\\
599.28	1.68674445361946e-05\\
599.29	1.64511943388859e-05\\
599.3	1.60389029905273e-05\\
599.31	1.56306122715104e-05\\
599.32	1.52263643777902e-05\\
599.33	1.48262019250105e-05\\
599.34	1.44301679526268e-05\\
599.35	1.40383059281154e-05\\
599.36	1.36506597511899e-05\\
599.37	1.32672737580847e-05\\
599.38	1.28881927258535e-05\\
599.39	1.25134618767404e-05\\
599.4	1.21431268825696e-05\\
599.41	1.17772338691924e-05\\
599.42	1.14158294209736e-05\\
599.43	1.10589605853174e-05\\
599.44	1.07066748772523e-05\\
599.45	1.03590202840433e-05\\
599.46	1.00160452698589e-05\\
599.47	9.67779878049101e-06\\
599.48	9.34433024810284e-06\\
599.49	9.0156895960411e-06\\
599.5	8.69192724369319e-06\\
599.51	8.37309411137396e-06\\
599.52	8.05924162529392e-06\\
599.53	7.75042172253965e-06\\
599.54	7.44668685613396e-06\\
599.55	7.14809000013084e-06\\
599.56	6.85468465475535e-06\\
599.57	6.56652485161134e-06\\
599.58	6.28366515892896e-06\\
599.59	6.00616068686249e-06\\
599.6	5.73406709284546e-06\\
599.61	5.46744058700296e-06\\
599.62	5.2063379376039e-06\\
599.63	4.95081647657741e-06\\
599.64	4.70093410508653e-06\\
599.65	4.45674929914347e-06\\
599.66	4.21832111529262e-06\\
599.67	3.98570919634376e-06\\
599.68	3.75897377716608e-06\\
599.69	3.53817569053241e-06\\
599.7	3.32337637303295e-06\\
599.71	3.11463787102881e-06\\
599.72	2.91202284668363e-06\\
599.73	2.71559458404555e-06\\
599.74	2.52541699518466e-06\\
599.75	2.34155462640329e-06\\
599.76	2.16407266449489e-06\\
599.77	1.99303694307755e-06\\
599.78	1.82851394897772e-06\\
599.79	1.67057082867475e-06\\
599.8	1.51927539483385e-06\\
599.81	1.37469613287027e-06\\
599.82	1.23690220760718e-06\\
599.83	1.10596346997172e-06\\
599.84	9.81950463784659e-07\\
599.85	8.64934432591793e-07\\
599.86	7.54987326587186e-07\\
599.87	6.52181809583305e-07\\
599.88	5.56591266064402e-07\\
599.89	4.68289808295413e-07\\
599.9	3.87352283521061e-07\\
599.91	3.13854281216996e-07\\
599.92	2.4787214042421e-07\\
599.93	1.8948295714763e-07\\
599.94	1.38764591834512e-07\\
599.95	9.57956769204876e-08\\
599.96	6.06556244606149e-08\\
599.97	3.34246338194039e-08\\
599.98	1.4183699454523e-08\\
599.99	3.01461875948372e-09\\
600	0\\
};
\addplot [color=red!75!mycolor17,solid,forget plot]
  table[row sep=crcr]{%
0.01	0.00615124513121968\\
1.01	0.00615124466954397\\
2.01	0.00615124419804395\\
3.01	0.00615124371650986\\
4.01	0.00615124322472784\\
5.01	0.00615124272247878\\
6.01	0.00615124220953934\\
7.01	0.00615124168568137\\
8.01	0.00615124115067172\\
9.01	0.00615124060427232\\
10.01	0.00615124004623988\\
11.01	0.00615123947632624\\
12.01	0.00615123889427741\\
13.01	0.00615123829983411\\
14.01	0.00615123769273202\\
15.01	0.00615123707270048\\
16.01	0.00615123643946319\\
17.01	0.00615123579273817\\
18.01	0.00615123513223719\\
19.01	0.00615123445766603\\
20.01	0.00615123376872386\\
21.01	0.00615123306510371\\
22.01	0.00615123234649195\\
23.01	0.00615123161256789\\
24.01	0.00615123086300448\\
25.01	0.00615123009746729\\
26.01	0.00615122931561487\\
27.01	0.00615122851709837\\
28.01	0.00615122770156156\\
29.01	0.00615122686864046\\
30.01	0.00615122601796326\\
31.01	0.00615122514915043\\
32.01	0.00615122426181394\\
33.01	0.00615122335555755\\
34.01	0.0061512224299766\\
35.01	0.00615122148465763\\
36.01	0.00615122051917858\\
37.01	0.00615121953310805\\
38.01	0.00615121852600525\\
39.01	0.00615121749742009\\
40.01	0.00615121644689298\\
41.01	0.00615121537395443\\
42.01	0.00615121427812442\\
43.01	0.00615121315891327\\
44.01	0.00615121201582032\\
45.01	0.00615121084833429\\
46.01	0.00615120965593289\\
47.01	0.00615120843808264\\
48.01	0.00615120719423868\\
49.01	0.00615120592384418\\
50.01	0.00615120462633052\\
51.01	0.00615120330111693\\
52.01	0.00615120194761009\\
53.01	0.0061512005652038\\
54.01	0.00615119915327905\\
55.01	0.00615119771120327\\
56.01	0.00615119623833049\\
57.01	0.00615119473400065\\
58.01	0.00615119319753965\\
59.01	0.00615119162825901\\
60.01	0.00615119002545504\\
61.01	0.00615118838840926\\
62.01	0.00615118671638754\\
63.01	0.00615118500864007\\
64.01	0.00615118326440093\\
65.01	0.00615118148288763\\
66.01	0.00615117966330086\\
67.01	0.00615117780482403\\
68.01	0.00615117590662313\\
69.01	0.00615117396784602\\
70.01	0.00615117198762241\\
71.01	0.00615116996506288\\
72.01	0.00615116789925952\\
73.01	0.00615116578928429\\
74.01	0.00615116363418927\\
75.01	0.00615116143300638\\
76.01	0.00615115918474637\\
77.01	0.00615115688839902\\
78.01	0.00615115454293195\\
79.01	0.00615115214729076\\
80.01	0.00615114970039842\\
81.01	0.00615114720115439\\
82.01	0.00615114464843474\\
83.01	0.0061511420410911\\
84.01	0.00615113937795039\\
85.01	0.00615113665781445\\
86.01	0.00615113387945888\\
87.01	0.0061511310416332\\
88.01	0.00615112814305988\\
89.01	0.00615112518243382\\
90.01	0.00615112215842168\\
91.01	0.00615111906966155\\
92.01	0.00615111591476198\\
93.01	0.00615111269230171\\
94.01	0.00615110940082874\\
95.01	0.00615110603885955\\
96.01	0.00615110260487881\\
97.01	0.0061510990973384\\
98.01	0.00615109551465685\\
99.01	0.00615109185521853\\
100.01	0.00615108811737304\\
101.01	0.00615108429943421\\
102.01	0.0061510803996794\\
103.01	0.00615107641634907\\
104.01	0.0061510723476455\\
105.01	0.00615106819173216\\
106.01	0.00615106394673287\\
107.01	0.00615105961073085\\
108.01	0.00615105518176808\\
109.01	0.00615105065784422\\
110.01	0.00615104603691575\\
111.01	0.00615104131689492\\
112.01	0.00615103649564875\\
113.01	0.0061510315709984\\
114.01	0.00615102654071772\\
115.01	0.00615102140253287\\
116.01	0.00615101615412047\\
117.01	0.00615101079310716\\
118.01	0.00615100531706819\\
119.01	0.00615099972352604\\
120.01	0.0061509940099504\\
121.01	0.00615098817375558\\
122.01	0.00615098221230042\\
123.01	0.00615097612288638\\
124.01	0.00615096990275661\\
125.01	0.00615096354909479\\
126.01	0.00615095705902356\\
127.01	0.00615095042960346\\
128.01	0.00615094365783104\\
129.01	0.00615093674063835\\
130.01	0.00615092967489107\\
131.01	0.00615092245738647\\
132.01	0.00615091508485339\\
133.01	0.00615090755394924\\
134.01	0.00615089986125927\\
135.01	0.00615089200329484\\
136.01	0.00615088397649178\\
137.01	0.00615087577720894\\
138.01	0.00615086740172604\\
139.01	0.0061508588462425\\
140.01	0.00615085010687546\\
141.01	0.00615084117965773\\
142.01	0.00615083206053641\\
143.01	0.00615082274537083\\
144.01	0.00615081322993062\\
145.01	0.00615080350989381\\
146.01	0.00615079358084492\\
147.01	0.00615078343827253\\
148.01	0.00615077307756771\\
149.01	0.00615076249402159\\
150.01	0.00615075168282305\\
151.01	0.00615074063905718\\
152.01	0.00615072935770238\\
153.01	0.00615071783362799\\
154.01	0.00615070606159245\\
155.01	0.0061506940362403\\
156.01	0.00615068175210023\\
157.01	0.00615066920358219\\
158.01	0.00615065638497509\\
159.01	0.00615064329044373\\
160.01	0.00615062991402655\\
161.01	0.00615061624963279\\
162.01	0.00615060229103937\\
163.01	0.00615058803188819\\
164.01	0.00615057346568362\\
165.01	0.00615055858578872\\
166.01	0.00615054338542289\\
167.01	0.00615052785765803\\
168.01	0.00615051199541592\\
169.01	0.00615049579146478\\
170.01	0.00615047923841574\\
171.01	0.00615046232871981\\
172.01	0.00615044505466359\\
173.01	0.00615042740836686\\
174.01	0.006150409381778\\
175.01	0.00615039096667045\\
176.01	0.00615037215463921\\
177.01	0.00615035293709682\\
178.01	0.00615033330526905\\
179.01	0.00615031325019131\\
180.01	0.00615029276270415\\
181.01	0.00615027183344897\\
182.01	0.00615025045286421\\
183.01	0.00615022861118025\\
184.01	0.00615020629841525\\
185.01	0.00615018350437032\\
186.01	0.00615016021862495\\
187.01	0.0061501364305322\\
188.01	0.00615011212921345\\
189.01	0.00615008730355363\\
190.01	0.00615006194219591\\
191.01	0.00615003603353649\\
192.01	0.00615000956571926\\
193.01	0.00614998252662989\\
194.01	0.00614995490389073\\
195.01	0.00614992668485472\\
196.01	0.00614989785659962\\
197.01	0.00614986840592174\\
198.01	0.00614983831933027\\
199.01	0.0061498075830404\\
200.01	0.00614977618296703\\
201.01	0.00614974410471874\\
202.01	0.0061497113335904\\
203.01	0.00614967785455659\\
204.01	0.00614964365226432\\
205.01	0.0061496087110263\\
206.01	0.00614957301481319\\
207.01	0.00614953654724642\\
208.01	0.00614949929158999\\
209.01	0.00614946123074312\\
210.01	0.00614942234723224\\
211.01	0.00614938262320241\\
212.01	0.00614934204040888\\
213.01	0.00614930058020941\\
214.01	0.00614925822355434\\
215.01	0.00614921495097818\\
216.01	0.0061491707425906\\
217.01	0.00614912557806666\\
218.01	0.00614907943663756\\
219.01	0.00614903229708059\\
220.01	0.00614898413770928\\
221.01	0.00614893493636313\\
222.01	0.00614888467039695\\
223.01	0.0061488333166703\\
224.01	0.00614878085153668\\
225.01	0.00614872725083177\\
226.01	0.00614867248986285\\
227.01	0.00614861654339601\\
228.01	0.00614855938564508\\
229.01	0.00614850099025888\\
230.01	0.00614844133030864\\
231.01	0.00614838037827548\\
232.01	0.0061483181060369\\
233.01	0.00614825448485373\\
234.01	0.00614818948535598\\
235.01	0.00614812307752912\\
236.01	0.00614805523069937\\
237.01	0.00614798591351937\\
238.01	0.00614791509395259\\
239.01	0.00614784273925873\\
240.01	0.00614776881597704\\
241.01	0.00614769328991072\\
242.01	0.00614761612611038\\
243.01	0.00614753728885672\\
244.01	0.00614745674164379\\
245.01	0.00614737444716088\\
246.01	0.00614729036727486\\
247.01	0.00614720446301071\\
248.01	0.00614711669453367\\
249.01	0.00614702702112888\\
250.01	0.00614693540118201\\
251.01	0.00614684179215854\\
252.01	0.00614674615058323\\
253.01	0.00614664843201824\\
254.01	0.00614654859104191\\
255.01	0.00614644658122558\\
256.01	0.00614634235511138\\
257.01	0.00614623586418795\\
258.01	0.00614612705886717\\
259.01	0.00614601588845865\\
260.01	0.00614590230114513\\
261.01	0.00614578624395606\\
262.01	0.00614566766274146\\
263.01	0.00614554650214442\\
264.01	0.0061454227055733\\
265.01	0.00614529621517315\\
266.01	0.00614516697179641\\
267.01	0.00614503491497296\\
268.01	0.00614489998287925\\
269.01	0.0061447621123068\\
270.01	0.00614462123862972\\
271.01	0.00614447729577154\\
272.01	0.00614433021617141\\
273.01	0.00614417993074869\\
274.01	0.00614402636886761\\
275.01	0.00614386945830026\\
276.01	0.00614370912518887\\
277.01	0.00614354529400727\\
278.01	0.00614337788752106\\
279.01	0.00614320682674676\\
280.01	0.00614303203091033\\
281.01	0.00614285341740365\\
282.01	0.00614267090174094\\
283.01	0.00614248439751313\\
284.01	0.00614229381634143\\
285.01	0.00614209906782946\\
286.01	0.00614190005951409\\
287.01	0.00614169669681516\\
288.01	0.00614148888298399\\
289.01	0.00614127651904899\\
290.01	0.00614105950376205\\
291.01	0.00614083773354134\\
292.01	0.00614061110241417\\
293.01	0.00614037950195624\\
294.01	0.00614014282123195\\
295.01	0.00613990094672996\\
296.01	0.00613965376229902\\
297.01	0.00613940114908108\\
298.01	0.00613914298544244\\
299.01	0.00613887914690307\\
300.01	0.0061386095060638\\
301.01	0.00613833393253115\\
302.01	0.00613805229283985\\
303.01	0.00613776445037347\\
304.01	0.00613747026528237\\
305.01	0.00613716959439855\\
306.01	0.00613686229114891\\
307.01	0.00613654820546506\\
308.01	0.00613622718369033\\
309.01	0.00613589906848424\\
310.01	0.00613556369872345\\
311.01	0.00613522090939985\\
312.01	0.00613487053151501\\
313.01	0.00613451239197168\\
314.01	0.00613414631346101\\
315.01	0.00613377211434705\\
316.01	0.00613338960854596\\
317.01	0.00613299860540252\\
318.01	0.00613259890956122\\
319.01	0.00613219032083414\\
320.01	0.00613177263406312\\
321.01	0.00613134563897781\\
322.01	0.00613090912004864\\
323.01	0.00613046285633399\\
324.01	0.00613000662132245\\
325.01	0.00612954018276931\\
326.01	0.00612906330252674\\
327.01	0.00612857573636803\\
328.01	0.00612807723380498\\
329.01	0.00612756753789875\\
330.01	0.00612704638506286\\
331.01	0.00612651350485966\\
332.01	0.00612596861978793\\
333.01	0.0061254114450629\\
334.01	0.0061248416883865\\
335.01	0.0061242590497101\\
336.01	0.00612366322098599\\
337.01	0.00612305388590973\\
338.01	0.00612243071965169\\
339.01	0.00612179338857729\\
340.01	0.00612114154995517\\
341.01	0.00612047485165395\\
342.01	0.00611979293182497\\
343.01	0.00611909541857141\\
344.01	0.00611838192960334\\
345.01	0.00611765207187742\\
346.01	0.00611690544121983\\
347.01	0.00611614162193266\\
348.01	0.00611536018638253\\
349.01	0.00611456069456946\\
350.01	0.00611374269367594\\
351.01	0.00611290571759487\\
352.01	0.00611204928643453\\
353.01	0.00611117290600028\\
354.01	0.00611027606725035\\
355.01	0.00610935824572522\\
356.01	0.00610841890094807\\
357.01	0.00610745747579621\\
358.01	0.0061064733958397\\
359.01	0.00610546606864639\\
360.01	0.00610443488305109\\
361.01	0.00610337920838679\\
362.01	0.00610229839367529\\
363.01	0.00610119176677524\\
364.01	0.00610005863348544\\
365.01	0.00609889827659873\\
366.01	0.00609770995490624\\
367.01	0.00609649290214711\\
368.01	0.00609524632590198\\
369.01	0.00609396940642553\\
370.01	0.00609266129541588\\
371.01	0.0060913211147181\\
372.01	0.00608994795495598\\
373.01	0.00608854087409203\\
374.01	0.00608709889590906\\
375.01	0.0060856210084131\\
376.01	0.00608410616215264\\
377.01	0.00608255326845319\\
378.01	0.00608096119756439\\
379.01	0.0060793287767183\\
380.01	0.00607765478809893\\
381.01	0.00607593796672505\\
382.01	0.00607417699824758\\
383.01	0.00607237051666689\\
384.01	0.00607051710197895\\
385.01	0.00606861527775943\\
386.01	0.00606666350870381\\
387.01	0.00606466019814232\\
388.01	0.0060626036855588\\
389.01	0.00606049224414927\\
390.01	0.00605832407846529\\
391.01	0.00605609732220219\\
392.01	0.00605381003620425\\
393.01	0.00605146020677961\\
394.01	0.00604904574443693\\
395.01	0.00604656448318412\\
396.01	0.00604401418055822\\
397.01	0.00604139251859296\\
398.01	0.00603869710597121\\
399.01	0.00603592548165764\\
400.01	0.00603307512036123\\
401.01	0.00603014344023608\\
402.01	0.00602712781329085\\
403.01	0.00602402557904241\\
404.01	0.00602083406200576\\
405.01	0.00601755059365846\\
406.01	0.00601417253953499\\
407.01	0.00601069733207518\\
408.01	0.00600712250973919\\
409.01	0.00600344576266342\\
410.01	0.0059996649846932\\
411.01	0.00599577833089774\\
412.01	0.00599178427849172\\
413.01	0.00598768168726121\\
414.01	0.00598346985280637\\
415.01	0.0059791485417527\\
416.01	0.00597471799193696\\
417.01	0.00597017885158271\\
418.01	0.00596553201843817\\
419.01	0.00596077832103965\\
420.01	0.00595591762523831\\
421.01	0.0059509476872939\\
422.01	0.00594586574767648\\
423.01	0.00594066895397797\\
424.01	0.00593535435606941\\
425.01	0.00592991890088308\\
426.01	0.00592435942678017\\
427.01	0.00591867265746136\\
428.01	0.00591285519537157\\
429.01	0.00590690351454442\\
430.01	0.00590081395282439\\
431.01	0.005894582703397\\
432.01	0.00588820580554942\\
433.01	0.00588167913457138\\
434.01	0.00587499839069569\\
435.01	0.00586815908696614\\
436.01	0.00586115653590091\\
437.01	0.0058539858348075\\
438.01	0.00584664184958114\\
439.01	0.00583911919679941\\
440.01	0.00583141222389821\\
441.01	0.00582351498718718\\
442.01	0.00581542122743015\\
443.01	0.00580712434267734\\
444.01	0.00579861735800195\\
445.01	0.00578989289174168\\
446.01	0.00578094311780526\\
447.01	0.00577175972354865\\
448.01	0.00576233386267786\\
449.01	0.00575265610258605\\
450.01	0.00574271636548881\\
451.01	0.00573250386268671\\
452.01	0.00572200702127258\\
453.01	0.00571121340261374\\
454.01	0.00570010961200627\\
455.01	0.00568868119903684\\
456.01	0.00567691254843265\\
457.01	0.00566478676158142\\
458.01	0.00565228552953682\\
459.01	0.00563938899926999\\
460.01	0.00562607563633821\\
461.01	0.00561232208917969\\
462.01	0.0055981030631865\\
463.01	0.00558339121688668\\
464.01	0.0055681570984796\\
465.01	0.0055523691492855\\
466.01	0.00553599381229492\\
467.01	0.00551899580022513\\
468.01	0.00550133860003272\\
469.01	0.00548298532209655\\
470.01	0.00546390004556607\\
471.01	0.00544404987119809\\
472.01	0.00542340797560858\\
473.01	0.00540195807482421\\
474.01	0.00537970086212681\\
475.01	0.00535667491186894\\
476.01	0.00533310316495461\\
477.01	0.00530905067600642\\
478.01	0.00528451324571092\\
479.01	0.00525948785542397\\
480.01	0.00523397292823664\\
481.01	0.00520796863302586\\
482.01	0.00518147723529276\\
483.01	0.00515450349988047\\
484.01	0.0051270551583203\\
485.01	0.00509914344624079\\
486.01	0.00507078370929517\\
487.01	0.00504199607500807\\
488.01	0.00501280618139569\\
489.01	0.00498324594232345\\
490.01	0.00495335431279067\\
491.01	0.00492317799198879\\
492.01	0.00489277196421238\\
493.01	0.00486219972203605\\
494.01	0.00483153294461179\\
495.01	0.00480085041618139\\
496.01	0.00477023590868901\\
497.01	0.0047397744102359\\
498.01	0.00470954587410876\\
499.01	0.00467961535984617\\
500.01	0.00465001797757927\\
501.01	0.00462062509242448\\
502.01	0.00459109249044627\\
503.01	0.00456143234565688\\
504.01	0.0045316820202934\\
505.01	0.00450188043189105\\
506.01	0.00447206739146352\\
507.01	0.00444228271748359\\
508.01	0.00441256508991101\\
509.01	0.00438295061273709\\
510.01	0.00435347106457543\\
511.01	0.00432415183879997\\
512.01	0.00429500961361187\\
513.01	0.00426604985698718\\
514.01	0.00423726437445556\\
515.01	0.00420862926745604\\
516.01	0.00418010391291735\\
517.01	0.00415163193823281\\
518.01	0.00412314570316703\\
519.01	0.00409458785638967\\
520.01	0.00406595094842425\\
521.01	0.00403723620157672\\
522.01	0.00400844050139541\\
523.01	0.00397955589621132\\
524.01	0.00395056922292003\\
525.01	0.00392146193374226\\
526.01	0.00389221021685914\\
527.01	0.00386278551309164\\
528.01	0.0038331555110655\\
529.01	0.00380328567100357\\
530.01	0.00377314126510747\\
531.01	0.0037426897917219\\
532.01	0.00371190338476699\\
533.01	0.00368076009306139\\
534.01	0.00364924023746064\\
535.01	0.00361732258221762\\
536.01	0.00358498429767557\\
537.01	0.00355220137212305\\
538.01	0.00351894910426884\\
539.01	0.00348520264598917\\
540.01	0.00345093754616732\\
541.01	0.00341613022608219\\
542.01	0.0033807582992128\\
543.01	0.00334480064213531\\
544.01	0.00330823714412511\\
545.01	0.00327104815352956\\
546.01	0.00323321401108179\\
547.01	0.00319471507490733\\
548.01	0.0031555318906373\\
549.01	0.00311564533753994\\
550.01	0.00307503673297587\\
551.01	0.00303368788459745\\
552.01	0.00299158108659081\\
553.01	0.00294869906736923\\
554.01	0.00290502491097292\\
555.01	0.00286054198984103\\
556.01	0.00281523395154636\\
557.01	0.00276908476086833\\
558.01	0.0027220787504204\\
559.01	0.00267420066112334\\
560.01	0.00262543567394102\\
561.01	0.00257576943734479\\
562.01	0.00252518809674998\\
563.01	0.00247367833295417\\
564.01	0.00242122741540764\\
565.01	0.0023678232722642\\
566.01	0.00231345457350701\\
567.01	0.00225811082177995\\
568.01	0.00220178245017577\\
569.01	0.00214446092893834\\
570.01	0.00208613888320133\\
571.01	0.00202681022346972\\
572.01	0.00196647028977617\\
573.01	0.00190511600947689\\
574.01	0.00184274606784671\\
575.01	0.00177936109045576\\
576.01	0.00171496383673893\\
577.01	0.00164955940429165\\
578.01	0.00158315544293594\\
579.01	0.00151576237682872\\
580.01	0.00144739363196437\\
581.01	0.00137806586538633\\
582.01	0.0013077991912756\\
583.01	0.00123661739776533\\
584.01	0.0011645481466601\\
585.01	0.00109162314601654\\
586.01	0.00101787828266926\\
587.01	0.000943353698183423\\
588.01	0.000868093787227074\\
589.01	0.000792147091763009\\
590.01	0.000715566057480832\\
591.01	0.000638406610165536\\
592.01	0.000560727498800357\\
593.01	0.000482589338632919\\
594.01	0.000404053270619676\\
595.01	0.000325179132900562\\
596.01	0.000246023014372898\\
597.01	0.000166666784466021\\
598.01	9.17366970063904e-05\\
599.01	2.94669364271135e-05\\
599.02	2.89574269585705e-05\\
599.03	2.84509530852801e-05\\
599.04	2.79475443462508e-05\\
599.05	2.7447230571279e-05\\
599.06	2.69500418838293e-05\\
599.07	2.64560087039432e-05\\
599.08	2.59651617511691e-05\\
599.09	2.54775320475235e-05\\
599.1	2.49931509204854e-05\\
599.11	2.45120500060175e-05\\
599.12	2.40342612516185e-05\\
599.13	2.35598169193978e-05\\
599.14	2.30887495892024e-05\\
599.15	2.26210921617422e-05\\
599.16	2.21568778617717e-05\\
599.17	2.16961402412993e-05\\
599.18	2.12389131828191e-05\\
599.19	2.07852309025824e-05\\
599.2	2.03351279538938e-05\\
599.21	1.98886392304542e-05\\
599.22	1.94457999697188e-05\\
599.23	1.90066457563046e-05\\
599.24	1.85712125254211e-05\\
599.25	1.81395365663351e-05\\
599.26	1.77116564095865e-05\\
599.27	1.72876122135658e-05\\
599.28	1.68674445361946e-05\\
599.29	1.64511943388859e-05\\
599.3	1.60389029905273e-05\\
599.31	1.56306122715087e-05\\
599.32	1.5226364377792e-05\\
599.33	1.48262019250105e-05\\
599.34	1.44301679526268e-05\\
599.35	1.40383059281154e-05\\
599.36	1.36506597511916e-05\\
599.37	1.32672737580847e-05\\
599.38	1.28881927258552e-05\\
599.39	1.25134618767404e-05\\
599.4	1.21431268825696e-05\\
599.41	1.17772338691924e-05\\
599.42	1.14158294209719e-05\\
599.43	1.10589605853192e-05\\
599.44	1.07066748772523e-05\\
599.45	1.03590202840433e-05\\
599.46	1.00160452698606e-05\\
599.47	9.67779878049274e-06\\
599.48	9.34433024810284e-06\\
599.49	9.01568959604283e-06\\
599.5	8.69192724369319e-06\\
599.51	8.3730941113757e-06\\
599.52	8.05924162529219e-06\\
599.53	7.75042172253965e-06\\
599.54	7.4466868561357e-06\\
599.55	7.14809000013084e-06\\
599.56	6.85468465475535e-06\\
599.57	6.56652485161308e-06\\
599.58	6.2836651589307e-06\\
599.59	6.00616068686076e-06\\
599.6	5.73406709284546e-06\\
599.61	5.46744058700123e-06\\
599.62	5.2063379376039e-06\\
599.63	4.95081647657741e-06\\
599.64	4.70093410508653e-06\\
599.65	4.45674929914347e-06\\
599.66	4.21832111529089e-06\\
599.67	3.98570919634376e-06\\
599.68	3.75897377716435e-06\\
599.69	3.53817569053415e-06\\
599.7	3.32337637303295e-06\\
599.71	3.11463787103054e-06\\
599.72	2.91202284668536e-06\\
599.73	2.71559458404555e-06\\
599.74	2.52541699518292e-06\\
599.75	2.34155462640155e-06\\
599.76	2.16407266449489e-06\\
599.77	1.99303694307928e-06\\
599.78	1.82851394897598e-06\\
599.79	1.67057082867302e-06\\
599.8	1.51927539483211e-06\\
599.81	1.37469613287027e-06\\
599.82	1.23690220760718e-06\\
599.83	1.10596346997172e-06\\
599.84	9.81950463782924e-07\\
599.85	8.64934432591793e-07\\
599.86	7.54987326587186e-07\\
599.87	6.5218180958504e-07\\
599.88	5.56591266064402e-07\\
599.89	4.68289808295413e-07\\
599.9	3.87352283521061e-07\\
599.91	3.13854281218731e-07\\
599.92	2.47872140425945e-07\\
599.93	1.89482957149364e-07\\
599.94	1.38764591834512e-07\\
599.95	9.57956769222224e-08\\
599.96	6.06556244606149e-08\\
599.97	3.34246338211386e-08\\
599.98	1.4183699454523e-08\\
599.99	3.01461875948372e-09\\
600	0\\
};
\addplot [color=red!80!mycolor19,solid,forget plot]
  table[row sep=crcr]{%
0.01	0.0060323088673989\\
1.01	0.00603230822979061\\
2.01	0.0060323075786237\\
3.01	0.00603230691360877\\
4.01	0.00603230623445046\\
5.01	0.00603230554084724\\
6.01	0.00603230483249098\\
7.01	0.00603230410906711\\
8.01	0.00603230337025409\\
9.01	0.00603230261572357\\
10.01	0.00603230184514019\\
11.01	0.00603230105816139\\
12.01	0.00603230025443744\\
13.01	0.00603229943361105\\
14.01	0.00603229859531677\\
15.01	0.00603229773918224\\
16.01	0.00603229686482659\\
17.01	0.00603229597186072\\
18.01	0.00603229505988737\\
19.01	0.00603229412850074\\
20.01	0.00603229317728638\\
21.01	0.00603229220582086\\
22.01	0.00603229121367175\\
23.01	0.0060322902003973\\
24.01	0.0060322891655463\\
25.01	0.00603228810865806\\
26.01	0.0060322870292616\\
27.01	0.00603228592687629\\
28.01	0.006032284801011\\
29.01	0.00603228365116394\\
30.01	0.00603228247682287\\
31.01	0.0060322812774641\\
32.01	0.00603228005255319\\
33.01	0.00603227880154409\\
34.01	0.00603227752387895\\
35.01	0.00603227621898803\\
36.01	0.00603227488628917\\
37.01	0.00603227352518803\\
38.01	0.00603227213507735\\
39.01	0.00603227071533685\\
40.01	0.00603226926533295\\
41.01	0.00603226778441839\\
42.01	0.00603226627193235\\
43.01	0.00603226472719929\\
44.01	0.0060322631495296\\
45.01	0.0060322615382187\\
46.01	0.00603225989254695\\
47.01	0.00603225821177918\\
48.01	0.00603225649516439\\
49.01	0.00603225474193552\\
50.01	0.00603225295130922\\
51.01	0.00603225112248514\\
52.01	0.00603224925464555\\
53.01	0.00603224734695528\\
54.01	0.00603224539856145\\
55.01	0.00603224340859254\\
56.01	0.00603224137615865\\
57.01	0.00603223930035039\\
58.01	0.00603223718023908\\
59.01	0.0060322350148757\\
60.01	0.00603223280329144\\
61.01	0.00603223054449612\\
62.01	0.00603222823747868\\
63.01	0.00603222588120604\\
64.01	0.00603222347462281\\
65.01	0.00603222101665137\\
66.01	0.00603221850619037\\
67.01	0.00603221594211521\\
68.01	0.00603221332327691\\
69.01	0.00603221064850169\\
70.01	0.00603220791659062\\
71.01	0.00603220512631906\\
72.01	0.00603220227643587\\
73.01	0.00603219936566315\\
74.01	0.00603219639269535\\
75.01	0.0060321933561987\\
76.01	0.00603219025481103\\
77.01	0.00603218708714056\\
78.01	0.00603218385176591\\
79.01	0.00603218054723475\\
80.01	0.0060321771720635\\
81.01	0.00603217372473684\\
82.01	0.00603217020370665\\
83.01	0.00603216660739147\\
84.01	0.00603216293417598\\
85.01	0.00603215918240966\\
86.01	0.00603215535040686\\
87.01	0.00603215143644542\\
88.01	0.00603214743876594\\
89.01	0.00603214335557143\\
90.01	0.0060321391850259\\
91.01	0.00603213492525388\\
92.01	0.0060321305743395\\
93.01	0.00603212613032547\\
94.01	0.00603212159121241\\
95.01	0.00603211695495789\\
96.01	0.00603211221947524\\
97.01	0.00603210738263305\\
98.01	0.00603210244225382\\
99.01	0.00603209739611303\\
100.01	0.00603209224193806\\
101.01	0.00603208697740756\\
102.01	0.00603208160015007\\
103.01	0.00603207610774268\\
104.01	0.00603207049771058\\
105.01	0.00603206476752528\\
106.01	0.00603205891460374\\
107.01	0.00603205293630724\\
108.01	0.00603204682994021\\
109.01	0.00603204059274872\\
110.01	0.00603203422191947\\
111.01	0.00603202771457836\\
112.01	0.00603202106778945\\
113.01	0.00603201427855305\\
114.01	0.0060320073438051\\
115.01	0.00603200026041496\\
116.01	0.00603199302518452\\
117.01	0.00603198563484681\\
118.01	0.00603197808606388\\
119.01	0.0060319703754262\\
120.01	0.00603196249944992\\
121.01	0.00603195445457652\\
122.01	0.00603194623717027\\
123.01	0.00603193784351689\\
124.01	0.00603192926982196\\
125.01	0.00603192051220891\\
126.01	0.00603191156671731\\
127.01	0.00603190242930123\\
128.01	0.0060318930958273\\
129.01	0.0060318835620728\\
130.01	0.00603187382372352\\
131.01	0.00603186387637232\\
132.01	0.00603185371551617\\
133.01	0.00603184333655526\\
134.01	0.00603183273478989\\
135.01	0.00603182190541884\\
136.01	0.00603181084353705\\
137.01	0.00603179954413329\\
138.01	0.0060317880020879\\
139.01	0.00603177621217043\\
140.01	0.00603176416903718\\
141.01	0.00603175186722916\\
142.01	0.00603173930116886\\
143.01	0.00603172646515809\\
144.01	0.00603171335337539\\
145.01	0.00603169995987326\\
146.01	0.00603168627857525\\
147.01	0.00603167230327385\\
148.01	0.0060316580276267\\
149.01	0.0060316434451543\\
150.01	0.00603162854923662\\
151.01	0.00603161333311041\\
152.01	0.0060315977898659\\
153.01	0.00603158191244361\\
154.01	0.0060315656936313\\
155.01	0.0060315491260604\\
156.01	0.00603153220220251\\
157.01	0.00603151491436623\\
158.01	0.00603149725469341\\
159.01	0.00603147921515594\\
160.01	0.00603146078755138\\
161.01	0.0060314419634995\\
162.01	0.00603142273443861\\
163.01	0.0060314030916214\\
164.01	0.00603138302611078\\
165.01	0.00603136252877606\\
166.01	0.00603134159028829\\
167.01	0.00603132020111665\\
168.01	0.00603129835152348\\
169.01	0.0060312760315597\\
170.01	0.00603125323106085\\
171.01	0.00603122993964158\\
172.01	0.0060312061466916\\
173.01	0.0060311818413703\\
174.01	0.00603115701260169\\
175.01	0.00603113164907001\\
176.01	0.00603110573921339\\
177.01	0.00603107927121927\\
178.01	0.00603105223301893\\
179.01	0.00603102461228137\\
180.01	0.00603099639640822\\
181.01	0.00603096757252764\\
182.01	0.00603093812748813\\
183.01	0.00603090804785276\\
184.01	0.00603087731989301\\
185.01	0.00603084592958219\\
186.01	0.00603081386258891\\
187.01	0.00603078110427054\\
188.01	0.00603074763966666\\
189.01	0.00603071345349168\\
190.01	0.00603067853012803\\
191.01	0.00603064285361903\\
192.01	0.00603060640766096\\
193.01	0.00603056917559613\\
194.01	0.00603053114040492\\
195.01	0.00603049228469773\\
196.01	0.00603045259070696\\
197.01	0.00603041204027912\\
198.01	0.00603037061486567\\
199.01	0.00603032829551533\\
200.01	0.00603028506286447\\
201.01	0.00603024089712847\\
202.01	0.00603019577809261\\
203.01	0.00603014968510258\\
204.01	0.0060301025970547\\
205.01	0.00603005449238637\\
206.01	0.00603000534906598\\
207.01	0.00602995514458247\\
208.01	0.006029903855935\\
209.01	0.00602985145962232\\
210.01	0.00602979793163174\\
211.01	0.00602974324742779\\
212.01	0.00602968738194122\\
213.01	0.00602963030955659\\
214.01	0.00602957200410113\\
215.01	0.00602951243883215\\
216.01	0.00602945158642428\\
217.01	0.00602938941895736\\
218.01	0.00602932590790274\\
219.01	0.00602926102411058\\
220.01	0.00602919473779578\\
221.01	0.00602912701852399\\
222.01	0.0060290578351978\\
223.01	0.00602898715604198\\
224.01	0.00602891494858824\\
225.01	0.0060288411796609\\
226.01	0.00602876581536028\\
227.01	0.00602868882104774\\
228.01	0.00602861016132892\\
229.01	0.00602852980003709\\
230.01	0.00602844770021664\\
231.01	0.00602836382410504\\
232.01	0.00602827813311566\\
233.01	0.00602819058781932\\
234.01	0.00602810114792604\\
235.01	0.00602800977226561\\
236.01	0.00602791641876861\\
237.01	0.0060278210444464\\
238.01	0.00602772360537084\\
239.01	0.00602762405665334\\
240.01	0.00602752235242419\\
241.01	0.00602741844581051\\
242.01	0.00602731228891381\\
243.01	0.00602720383278795\\
244.01	0.00602709302741537\\
245.01	0.00602697982168388\\
246.01	0.00602686416336153\\
247.01	0.00602674599907317\\
248.01	0.00602662527427365\\
249.01	0.00602650193322271\\
250.01	0.00602637591895818\\
251.01	0.00602624717326863\\
252.01	0.00602611563666591\\
253.01	0.00602598124835641\\
254.01	0.00602584394621202\\
255.01	0.00602570366674065\\
256.01	0.00602556034505534\\
257.01	0.00602541391484351\\
258.01	0.00602526430833456\\
259.01	0.00602511145626771\\
260.01	0.00602495528785824\\
261.01	0.00602479573076368\\
262.01	0.00602463271104836\\
263.01	0.00602446615314793\\
264.01	0.0060242959798326\\
265.01	0.00602412211216966\\
266.01	0.00602394446948496\\
267.01	0.0060237629693236\\
268.01	0.00602357752740974\\
269.01	0.00602338805760507\\
270.01	0.00602319447186705\\
271.01	0.00602299668020533\\
272.01	0.00602279459063738\\
273.01	0.00602258810914331\\
274.01	0.00602237713961935\\
275.01	0.00602216158383037\\
276.01	0.00602194134136069\\
277.01	0.00602171630956459\\
278.01	0.00602148638351458\\
279.01	0.00602125145594939\\
280.01	0.00602101141721989\\
281.01	0.00602076615523391\\
282.01	0.0060205155554001\\
283.01	0.00602025950056934\\
284.01	0.00601999787097592\\
285.01	0.0060197305441762\\
286.01	0.00601945739498642\\
287.01	0.00601917829541845\\
288.01	0.00601889311461384\\
289.01	0.00601860171877663\\
290.01	0.00601830397110407\\
291.01	0.00601799973171564\\
292.01	0.00601768885757962\\
293.01	0.00601737120243923\\
294.01	0.00601704661673481\\
295.01	0.00601671494752553\\
296.01	0.00601637603840791\\
297.01	0.00601602972943313\\
298.01	0.00601567585702081\\
299.01	0.0060153142538717\\
300.01	0.00601494474887713\\
301.01	0.00601456716702615\\
302.01	0.00601418132931047\\
303.01	0.00601378705262579\\
304.01	0.00601338414967121\\
305.01	0.00601297242884532\\
306.01	0.00601255169413933\\
307.01	0.00601212174502735\\
308.01	0.00601168237635293\\
309.01	0.00601123337821282\\
310.01	0.00601077453583678\\
311.01	0.00601030562946407\\
312.01	0.00600982643421604\\
313.01	0.00600933671996531\\
314.01	0.00600883625119968\\
315.01	0.00600832478688312\\
316.01	0.00600780208031201\\
317.01	0.0060072678789662\\
318.01	0.00600672192435638\\
319.01	0.00600616395186524\\
320.01	0.00600559369058467\\
321.01	0.00600501086314687\\
322.01	0.00600441518554961\\
323.01	0.00600380636697669\\
324.01	0.00600318410961098\\
325.01	0.00600254810844194\\
326.01	0.00600189805106614\\
327.01	0.00600123361748045\\
328.01	0.0060005544798687\\
329.01	0.00599986030237971\\
330.01	0.00599915074089787\\
331.01	0.00599842544280482\\
332.01	0.00599768404673316\\
333.01	0.00599692618230945\\
334.01	0.00599615146988878\\
335.01	0.00599535952027785\\
336.01	0.00599454993444854\\
337.01	0.00599372230323855\\
338.01	0.00599287620704164\\
339.01	0.00599201121548363\\
340.01	0.00599112688708665\\
341.01	0.00599022276891755\\
342.01	0.00598929839622296\\
343.01	0.00598835329204795\\
344.01	0.00598738696683767\\
345.01	0.00598639891802182\\
346.01	0.00598538862958029\\
347.01	0.00598435557158891\\
348.01	0.00598329919974391\\
349.01	0.00598221895486372\\
350.01	0.00598111426236792\\
351.01	0.00597998453172951\\
352.01	0.00597882915590072\\
353.01	0.00597764751070914\\
354.01	0.00597643895422407\\
355.01	0.00597520282608819\\
356.01	0.00597393844681501\\
357.01	0.005972645117047\\
358.01	0.00597132211677355\\
359.01	0.0059699687045044\\
360.01	0.00596858411639517\\
361.01	0.00596716756532282\\
362.01	0.00596571823990379\\
363.01	0.00596423530345384\\
364.01	0.00596271789288124\\
365.01	0.00596116511751003\\
366.01	0.00595957605782627\\
367.01	0.00595794976414015\\
368.01	0.0059562852551558\\
369.01	0.00595458151644177\\
370.01	0.00595283749879069\\
371.01	0.00595105211645827\\
372.01	0.00594922424526912\\
373.01	0.00594735272057594\\
374.01	0.00594543633505635\\
375.01	0.00594347383632999\\
376.01	0.00594146392437586\\
377.01	0.00593940524872865\\
378.01	0.00593729640542735\\
379.01	0.00593513593368828\\
380.01	0.00593292231227016\\
381.01	0.00593065395549255\\
382.01	0.00592832920886761\\
383.01	0.00592594634429556\\
384.01	0.00592350355476975\\
385.01	0.00592099894852966\\
386.01	0.00591843054258989\\
387.01	0.00591579625556634\\
388.01	0.00591309389970917\\
389.01	0.00591032117203818\\
390.01	0.00590747564446821\\
391.01	0.00590455475279379\\
392.01	0.00590155578439159\\
393.01	0.00589847586448267\\
394.01	0.00589531194078513\\
395.01	0.00589206076637255\\
396.01	0.00588871888055167\\
397.01	0.00588528258756502\\
398.01	0.00588174793293934\\
399.01	0.00587811067732352\\
400.01	0.00587436626771316\\
401.01	0.0058705098060435\\
402.01	0.0058665360152756\\
403.01	0.00586243920331443\\
404.01	0.00585821322542409\\
405.01	0.00585385144627785\\
406.01	0.005849346703474\\
407.01	0.00584469127533662\\
408.01	0.00583987685722575\\
409.01	0.00583489455256717\\
410.01	0.00582973488760384\\
411.01	0.00582438786276795\\
412.01	0.00581884305901457\\
413.01	0.00581308982501403\\
414.01	0.00580711758158385\\
415.01	0.00580091629425014\\
416.01	0.00579447718488472\\
417.01	0.00578779378104929\\
418.01	0.0057808634398562\\
419.01	0.00577368953577726\\
420.01	0.00576631687256892\\
421.01	0.00575878074051462\\
422.01	0.00575107735371502\\
423.01	0.0057432028327119\\
424.01	0.00573515320250625\\
425.01	0.00572692439066269\\
426.01	0.00571851222553454\\
427.01	0.00570991243464606\\
428.01	0.00570112064327694\\
429.01	0.00569213237330246\\
430.01	0.00568294304235371\\
431.01	0.00567354796337234\\
432.01	0.00566394234465059\\
433.01	0.00565412129046134\\
434.01	0.00564407980240533\\
435.01	0.00563381278162197\\
436.01	0.00562331503203997\\
437.01	0.0056125812648738\\
438.01	0.00560160610460731\\
439.01	0.00559038409675234\\
440.01	0.00557890971771511\\
441.01	0.00556717738716608\\
442.01	0.00555518148337079\\
443.01	0.00554291636202407\\
444.01	0.00553037637921097\\
445.01	0.00551755591922929\\
446.01	0.00550444942811777\\
447.01	0.00549105145387235\\
448.01	0.00547735669447731\\
449.01	0.00546336005504276\\
450.01	0.00544905671551638\\
451.01	0.00543444221062791\\
452.01	0.00541951252391299\\
453.01	0.0054042641978536\\
454.01	0.00538869446233392\\
455.01	0.0053728013837293\\
456.01	0.00535658403697551\\
457.01	0.00534004270286315\\
458.01	0.00532317909247237\\
459.01	0.00530599660000874\\
460.01	0.00528850058414319\\
461.01	0.00527069867609095\\
462.01	0.00525260110973188\\
463.01	0.00523422106466096\\
464.01	0.00521557500650392\\
465.01	0.00519668299926648\\
466.01	0.00517756895069433\\
467.01	0.00515826073188677\\
468.01	0.00513879008439024\\
469.01	0.00511919218843788\\
470.01	0.00509950471039199\\
471.01	0.00507976606957098\\
472.01	0.00506001255591459\\
473.01	0.00504027377851333\\
474.01	0.00502056571458162\\
475.01	0.00500086854867461\\
476.01	0.00498097279191445\\
477.01	0.0049608181976079\\
478.01	0.00494041518376095\\
479.01	0.00491977568273822\\
480.01	0.00489891321125773\\
481.01	0.00487784292242705\\
482.01	0.00485658163246677\\
483.01	0.00483514781308942\\
484.01	0.0048135615384342\\
485.01	0.00479184437305596\\
486.01	0.00477001918516127\\
487.01	0.00474810986696565\\
488.01	0.00472614094195711\\
489.01	0.00470413703747452\\
490.01	0.00468212220105582\\
491.01	0.00466011904158858\\
492.01	0.00463814768309418\\
493.01	0.00461622453253979\\
494.01	0.00459436088709398\\
495.01	0.00457256144371503\\
496.01	0.00455082282733694\\
497.01	0.00452913234196017\\
498.01	0.00450746728654395\\
499.01	0.00448579538001983\\
500.01	0.00446407713748063\\
501.01	0.00444227403134053\\
502.01	0.00442037839858218\\
503.01	0.00439839898302056\\
504.01	0.00437634322091555\\
505.01	0.00435421663797677\\
506.01	0.00433202245018829\\
507.01	0.00430976117040764\\
508.01	0.00428743024502981\\
509.01	0.00426502375278557\\
510.01	0.00424253220619599\\
511.01	0.00421994250452162\\
512.01	0.00419723809369644\\
513.01	0.00417439939107087\\
514.01	0.00415140452641962\\
515.01	0.00412823042857954\\
516.01	0.00410485423826421\\
517.01	0.00408125493514216\\
518.01	0.00405741490545617\\
519.01	0.00403332066448421\\
520.01	0.00400896013218617\\
521.01	0.00398431983134321\\
522.01	0.0039593847362626\\
523.01	0.00393413843157042\\
524.01	0.00390856333801112\\
525.01	0.0038826410022877\\
526.01	0.0038563524398284\\
527.01	0.0038296785084463\\
528.01	0.00380260027802109\\
529.01	0.00377509934822052\\
530.01	0.00374715805511664\\
531.01	0.00371875950392218\\
532.01	0.00368988737917599\\
533.01	0.0036605255388462\\
534.01	0.00363065763087537\\
535.01	0.00360026706979577\\
536.01	0.00356933716041013\\
537.01	0.00353785121829806\\
538.01	0.00350579267201484\\
539.01	0.0034731451374028\\
540.01	0.00343989245617953\\
541.01	0.0034060186949134\\
542.01	0.003371508107079\\
543.01	0.0033363450700003\\
544.01	0.00330051401884971\\
545.01	0.00326399940755064\\
546.01	0.00322678571573412\\
547.01	0.00318885747810126\\
548.01	0.00315019930833668\\
549.01	0.00311079591342006\\
550.01	0.00307063209961312\\
551.01	0.00302969277287801\\
552.01	0.00298796293775365\\
553.01	0.0029454276994024\\
554.01	0.00290207227312257\\
555.01	0.00285788200362993\\
556.01	0.0028128423928549\\
557.01	0.00276693913212747\\
558.01	0.00272015813645232\\
559.01	0.00267248558161509\\
560.01	0.00262390794559945\\
561.01	0.00257441205578024\\
562.01	0.00252398514312515\\
563.01	0.00247261490420945\\
564.01	0.00242028957133667\\
565.01	0.00236699799067106\\
566.01	0.00231272970830979\\
567.01	0.00225747506467842\\
568.01	0.0022012252979938\\
569.01	0.00214397265754098\\
570.01	0.0020857105273739\\
571.01	0.00202643356088241\\
572.01	0.00196613782650036\\
573.01	0.00190482096468999\\
574.01	0.00184248235623363\\
575.01	0.00177912330175362\\
576.01	0.00171474721218609\\
577.01	0.00164935980960222\\
578.01	0.00158296933731052\\
579.01	0.001515586777603\\
580.01	0.00144722607478906\\
581.01	0.0013779043602682\\
582.01	0.00130764217525753\\
583.01	0.00123646368534738\\
584.01	0.00116439687921647\\
585.01	0.00109147374150863\\
586.01	0.00101773038694345\\
587.01	0.000943207139070279\\
588.01	0.000867948532496969\\
589.01	0.000792003211719553\\
590.01	0.000715423692577445\\
591.01	0.000638265943538507\\
592.01	0.000560588733085159\\
593.01	0.000482452675945771\\
594.01	0.00040391889420434\\
595.01	0.000325047188695244\\
596.01	0.000245893590649514\\
597.01	0.000166581383471795\\
598.01	9.17366970063782e-05\\
599.01	2.94669364271152e-05\\
599.02	2.89574269585705e-05\\
599.03	2.84509530852819e-05\\
599.04	2.79475443462525e-05\\
599.05	2.74472305712807e-05\\
599.06	2.6950041883831e-05\\
599.07	2.64560087039449e-05\\
599.08	2.59651617511691e-05\\
599.09	2.54775320475218e-05\\
599.1	2.49931509204854e-05\\
599.11	2.45120500060175e-05\\
599.12	2.40342612516167e-05\\
599.13	2.35598169193996e-05\\
599.14	2.30887495892024e-05\\
599.15	2.26210921617405e-05\\
599.16	2.215687786177e-05\\
599.17	2.16961402412993e-05\\
599.18	2.12389131828191e-05\\
599.19	2.07852309025824e-05\\
599.2	2.03351279538938e-05\\
599.21	1.98886392304542e-05\\
599.22	1.94457999697206e-05\\
599.23	1.90066457563063e-05\\
599.24	1.85712125254211e-05\\
599.25	1.81395365663334e-05\\
599.26	1.77116564095865e-05\\
599.27	1.72876122135658e-05\\
599.28	1.68674445361946e-05\\
599.29	1.64511943388877e-05\\
599.3	1.60389029905273e-05\\
599.31	1.56306122715087e-05\\
599.32	1.52263643777902e-05\\
599.33	1.48262019250105e-05\\
599.34	1.44301679526268e-05\\
599.35	1.40383059281154e-05\\
599.36	1.36506597511916e-05\\
599.37	1.32672737580865e-05\\
599.38	1.28881927258535e-05\\
599.39	1.25134618767404e-05\\
599.4	1.21431268825679e-05\\
599.41	1.17772338691924e-05\\
599.42	1.14158294209736e-05\\
599.43	1.10589605853174e-05\\
599.44	1.07066748772523e-05\\
599.45	1.03590202840433e-05\\
599.46	1.00160452698589e-05\\
599.47	9.67779878049101e-06\\
599.48	9.34433024810284e-06\\
599.49	9.0156895960411e-06\\
599.5	8.69192724369146e-06\\
599.51	8.37309411137396e-06\\
599.52	8.05924162529392e-06\\
599.53	7.75042172253965e-06\\
599.54	7.44668685613396e-06\\
599.55	7.14809000013084e-06\\
599.56	6.85468465475535e-06\\
599.57	6.56652485161308e-06\\
599.58	6.28366515892896e-06\\
599.59	6.00616068686249e-06\\
599.6	5.73406709284546e-06\\
599.61	5.46744058700296e-06\\
599.62	5.20633793760217e-06\\
599.63	4.95081647657741e-06\\
599.64	4.70093410508653e-06\\
599.65	4.45674929914347e-06\\
599.66	4.21832111529262e-06\\
599.67	3.98570919634203e-06\\
599.68	3.75897377716608e-06\\
599.69	3.53817569053415e-06\\
599.7	3.32337637303469e-06\\
599.71	3.11463787102881e-06\\
599.72	2.91202284668363e-06\\
599.73	2.71559458404382e-06\\
599.74	2.52541699518466e-06\\
599.75	2.34155462640329e-06\\
599.76	2.16407266449489e-06\\
599.77	1.99303694307928e-06\\
599.78	1.82851394897598e-06\\
599.79	1.67057082867302e-06\\
599.8	1.51927539483211e-06\\
599.81	1.37469613287027e-06\\
599.82	1.23690220760544e-06\\
599.83	1.10596346997172e-06\\
599.84	9.81950463782924e-07\\
599.85	8.64934432591793e-07\\
599.86	7.54987326588921e-07\\
599.87	6.5218180958504e-07\\
599.88	5.56591266064402e-07\\
599.89	4.68289808295413e-07\\
599.9	3.87352283521061e-07\\
599.91	3.13854281216996e-07\\
599.92	2.47872140425945e-07\\
599.93	1.8948295714763e-07\\
599.94	1.38764591834512e-07\\
599.95	9.57956769204876e-08\\
599.96	6.06556244606149e-08\\
599.97	3.34246338194039e-08\\
599.98	1.4183699454523e-08\\
599.99	3.014618757749e-09\\
600	0\\
};
\addplot [color=red,solid,forget plot]
  table[row sep=crcr]{%
0.01	0.00579827411309351\\
1.01	0.00579827314537753\\
2.01	0.00579827215707644\\
3.01	0.00579827114775116\\
4.01	0.00579827011695319\\
5.01	0.00579826906422434\\
6.01	0.00579826798909636\\
7.01	0.00579826689109139\\
8.01	0.00579826576972118\\
9.01	0.0057982646244869\\
10.01	0.00579826345487936\\
11.01	0.00579826226037807\\
12.01	0.00579826104045152\\
13.01	0.00579825979455707\\
14.01	0.00579825852214016\\
15.01	0.0057982572226343\\
16.01	0.00579825589546109\\
17.01	0.00579825454002963\\
18.01	0.00579825315573628\\
19.01	0.00579825174196466\\
20.01	0.00579825029808493\\
21.01	0.00579824882345383\\
22.01	0.0057982473174144\\
23.01	0.00579824577929556\\
24.01	0.00579824420841174\\
25.01	0.00579824260406261\\
26.01	0.0057982409655332\\
27.01	0.00579823929209277\\
28.01	0.00579823758299505\\
29.01	0.00579823583747786\\
30.01	0.00579823405476249\\
31.01	0.00579823223405377\\
32.01	0.00579823037453908\\
33.01	0.00579822847538871\\
34.01	0.00579822653575484\\
35.01	0.00579822455477176\\
36.01	0.00579822253155499\\
37.01	0.00579822046520106\\
38.01	0.0057982183547872\\
39.01	0.00579821619937075\\
40.01	0.00579821399798874\\
41.01	0.00579821174965755\\
42.01	0.00579820945337251\\
43.01	0.00579820710810723\\
44.01	0.00579820471281346\\
45.01	0.00579820226642009\\
46.01	0.00579819976783343\\
47.01	0.00579819721593598\\
48.01	0.00579819460958627\\
49.01	0.0057981919476184\\
50.01	0.00579818922884112\\
51.01	0.00579818645203786\\
52.01	0.00579818361596585\\
53.01	0.00579818071935561\\
54.01	0.00579817776091003\\
55.01	0.00579817473930455\\
56.01	0.00579817165318576\\
57.01	0.00579816850117141\\
58.01	0.0057981652818493\\
59.01	0.00579816199377695\\
60.01	0.00579815863548088\\
61.01	0.00579815520545563\\
62.01	0.00579815170216373\\
63.01	0.00579814812403414\\
64.01	0.00579814446946259\\
65.01	0.00579814073680959\\
66.01	0.00579813692440098\\
67.01	0.0057981330305261\\
68.01	0.00579812905343762\\
69.01	0.00579812499135079\\
70.01	0.00579812084244191\\
71.01	0.00579811660484832\\
72.01	0.00579811227666712\\
73.01	0.00579810785595438\\
74.01	0.00579810334072441\\
75.01	0.0057980987289487\\
76.01	0.00579809401855495\\
77.01	0.00579808920742611\\
78.01	0.0057980842933996\\
79.01	0.00579807927426606\\
80.01	0.0057980741477687\\
81.01	0.00579806891160204\\
82.01	0.00579806356341071\\
83.01	0.0057980581007887\\
84.01	0.00579805252127776\\
85.01	0.00579804682236706\\
86.01	0.00579804100149125\\
87.01	0.00579803505602974\\
88.01	0.00579802898330541\\
89.01	0.00579802278058322\\
90.01	0.00579801644506926\\
91.01	0.00579800997390934\\
92.01	0.00579800336418742\\
93.01	0.00579799661292453\\
94.01	0.00579798971707748\\
95.01	0.00579798267353733\\
96.01	0.00579797547912801\\
97.01	0.00579796813060459\\
98.01	0.00579796062465232\\
99.01	0.00579795295788471\\
100.01	0.00579794512684214\\
101.01	0.0057979371279901\\
102.01	0.00579792895771785\\
103.01	0.00579792061233645\\
104.01	0.00579791208807725\\
105.01	0.00579790338109002\\
106.01	0.00579789448744148\\
107.01	0.00579788540311315\\
108.01	0.0057978761239998\\
109.01	0.00579786664590704\\
110.01	0.00579785696455001\\
111.01	0.00579784707555114\\
112.01	0.00579783697443797\\
113.01	0.00579782665664153\\
114.01	0.00579781611749366\\
115.01	0.00579780535222541\\
116.01	0.00579779435596459\\
117.01	0.00579778312373305\\
118.01	0.00579777165044552\\
119.01	0.00579775993090621\\
120.01	0.00579774795980698\\
121.01	0.00579773573172455\\
122.01	0.00579772324111822\\
123.01	0.00579771048232739\\
124.01	0.00579769744956855\\
125.01	0.00579768413693305\\
126.01	0.00579767053838438\\
127.01	0.0057976566477548\\
128.01	0.00579764245874324\\
129.01	0.00579762796491163\\
130.01	0.00579761315968266\\
131.01	0.00579759803633612\\
132.01	0.00579758258800648\\
133.01	0.0057975668076788\\
134.01	0.00579755068818649\\
135.01	0.00579753422220721\\
136.01	0.00579751740225975\\
137.01	0.00579750022070049\\
138.01	0.00579748266972034\\
139.01	0.00579746474134026\\
140.01	0.00579744642740845\\
141.01	0.0057974277195958\\
142.01	0.00579740860939252\\
143.01	0.00579738908810407\\
144.01	0.00579736914684723\\
145.01	0.00579734877654559\\
146.01	0.00579732796792604\\
147.01	0.00579730671151351\\
148.01	0.00579728499762742\\
149.01	0.00579726281637671\\
150.01	0.00579724015765566\\
151.01	0.00579721701113865\\
152.01	0.00579719336627584\\
153.01	0.00579716921228806\\
154.01	0.00579714453816153\\
155.01	0.00579711933264326\\
156.01	0.00579709358423587\\
157.01	0.00579706728119147\\
158.01	0.00579704041150698\\
159.01	0.00579701296291797\\
160.01	0.0057969849228934\\
161.01	0.00579695627862977\\
162.01	0.00579692701704473\\
163.01	0.00579689712477128\\
164.01	0.00579686658815173\\
165.01	0.00579683539323069\\
166.01	0.00579680352574962\\
167.01	0.00579677097113907\\
168.01	0.00579673771451243\\
169.01	0.00579670374065922\\
170.01	0.00579666903403737\\
171.01	0.00579663357876651\\
172.01	0.00579659735862035\\
173.01	0.0057965603570189\\
174.01	0.00579652255702124\\
175.01	0.00579648394131712\\
176.01	0.00579644449221892\\
177.01	0.00579640419165386\\
178.01	0.00579636302115499\\
179.01	0.0057963209618527\\
180.01	0.00579627799446601\\
181.01	0.00579623409929379\\
182.01	0.00579618925620505\\
183.01	0.00579614344462948\\
184.01	0.00579609664354852\\
185.01	0.00579604883148476\\
186.01	0.00579599998649227\\
187.01	0.00579595008614644\\
188.01	0.0057958991075329\\
189.01	0.00579584702723753\\
190.01	0.00579579382133497\\
191.01	0.00579573946537734\\
192.01	0.00579568393438316\\
193.01	0.00579562720282547\\
194.01	0.00579556924461956\\
195.01	0.00579551003311128\\
196.01	0.005795449541064\\
197.01	0.00579538774064614\\
198.01	0.00579532460341794\\
199.01	0.0057952601003182\\
200.01	0.00579519420165056\\
201.01	0.00579512687706953\\
202.01	0.00579505809556621\\
203.01	0.0057949878254534\\
204.01	0.00579491603435147\\
205.01	0.00579484268917235\\
206.01	0.00579476775610444\\
207.01	0.00579469120059632\\
208.01	0.00579461298734054\\
209.01	0.00579453308025713\\
210.01	0.00579445144247664\\
211.01	0.00579436803632249\\
212.01	0.00579428282329334\\
213.01	0.00579419576404477\\
214.01	0.00579410681837108\\
215.01	0.00579401594518572\\
216.01	0.00579392310250235\\
217.01	0.00579382824741506\\
218.01	0.00579373133607749\\
219.01	0.00579363232368251\\
220.01	0.00579353116444109\\
221.01	0.00579342781156036\\
222.01	0.00579332221722175\\
223.01	0.00579321433255784\\
224.01	0.00579310410762988\\
225.01	0.00579299149140342\\
226.01	0.00579287643172464\\
227.01	0.00579275887529527\\
228.01	0.00579263876764722\\
229.01	0.00579251605311728\\
230.01	0.00579239067481969\\
231.01	0.00579226257462006\\
232.01	0.00579213169310728\\
233.01	0.00579199796956499\\
234.01	0.00579186134194323\\
235.01	0.00579172174682847\\
236.01	0.00579157911941347\\
237.01	0.00579143339346638\\
238.01	0.00579128450129919\\
239.01	0.00579113237373528\\
240.01	0.00579097694007667\\
241.01	0.00579081812806965\\
242.01	0.00579065586387095\\
243.01	0.00579049007201172\\
244.01	0.00579032067536187\\
245.01	0.00579014759509287\\
246.01	0.00578997075063997\\
247.01	0.00578979005966341\\
248.01	0.00578960543800922\\
249.01	0.00578941679966862\\
250.01	0.0057892240567364\\
251.01	0.00578902711936903\\
252.01	0.00578882589574102\\
253.01	0.00578862029200088\\
254.01	0.00578841021222596\\
255.01	0.00578819555837566\\
256.01	0.00578797623024446\\
257.01	0.0057877521254132\\
258.01	0.00578752313919994\\
259.01	0.00578728916460873\\
260.01	0.0057870500922779\\
261.01	0.00578680581042684\\
262.01	0.00578655620480192\\
263.01	0.00578630115862098\\
264.01	0.0057860405525161\\
265.01	0.0057857742644753\\
266.01	0.00578550216978347\\
267.01	0.00578522414096144\\
268.01	0.00578494004770291\\
269.01	0.00578464975681151\\
270.01	0.00578435313213467\\
271.01	0.00578405003449716\\
272.01	0.00578374032163303\\
273.01	0.00578342384811487\\
274.01	0.00578310046528246\\
275.01	0.00578277002116907\\
276.01	0.0057824323604267\\
277.01	0.00578208732424903\\
278.01	0.00578173475029253\\
279.01	0.00578137447259588\\
280.01	0.00578100632149719\\
281.01	0.00578063012354972\\
282.01	0.0057802457014351\\
283.01	0.00577985287387483\\
284.01	0.00577945145553875\\
285.01	0.00577904125695298\\
286.01	0.00577862208440381\\
287.01	0.00577819373984031\\
288.01	0.00577775602077444\\
289.01	0.00577730872017823\\
290.01	0.00577685162637893\\
291.01	0.00577638452295131\\
292.01	0.00577590718860751\\
293.01	0.00577541939708381\\
294.01	0.00577492091702463\\
295.01	0.00577441151186419\\
296.01	0.00577389093970455\\
297.01	0.00577335895319051\\
298.01	0.00577281529938182\\
299.01	0.00577225971962204\\
300.01	0.00577169194940389\\
301.01	0.00577111171823091\\
302.01	0.00577051874947646\\
303.01	0.00576991276023777\\
304.01	0.00576929346118716\\
305.01	0.00576866055641959\\
306.01	0.00576801374329498\\
307.01	0.00576735271227751\\
308.01	0.00576667714677085\\
309.01	0.00576598672294736\\
310.01	0.00576528110957491\\
311.01	0.0057645599678377\\
312.01	0.00576382295115243\\
313.01	0.00576306970497961\\
314.01	0.00576229986663068\\
315.01	0.00576151306506771\\
316.01	0.00576070892069997\\
317.01	0.00575988704517318\\
318.01	0.00575904704115399\\
319.01	0.0057581885021081\\
320.01	0.00575731101207152\\
321.01	0.00575641414541665\\
322.01	0.00575549746661117\\
323.01	0.00575456052996957\\
324.01	0.00575360287939838\\
325.01	0.005752624048134\\
326.01	0.00575162355847247\\
327.01	0.00575060092149164\\
328.01	0.00574955563676572\\
329.01	0.00574848719207092\\
330.01	0.00574739506308279\\
331.01	0.00574627871306462\\
332.01	0.00574513759254621\\
333.01	0.00574397113899348\\
334.01	0.00574277877646861\\
335.01	0.0057415599152792\\
336.01	0.00574031395161721\\
337.01	0.0057390402671877\\
338.01	0.00573773822882488\\
339.01	0.00573640718809818\\
340.01	0.00573504648090548\\
341.01	0.00573365542705417\\
342.01	0.00573223332982955\\
343.01	0.00573077947555017\\
344.01	0.00572929313311008\\
345.01	0.00572777355350623\\
346.01	0.00572621996935286\\
347.01	0.00572463159438094\\
348.01	0.00572300762292222\\
349.01	0.00572134722937925\\
350.01	0.00571964956767855\\
351.01	0.00571791377070902\\
352.01	0.00571613894974415\\
353.01	0.00571432419384813\\
354.01	0.00571246856926574\\
355.01	0.00571057111879649\\
356.01	0.00570863086115212\\
357.01	0.00570664679029909\\
358.01	0.00570461787478514\\
359.01	0.00570254305705154\\
360.01	0.00570042125273152\\
361.01	0.00569825134993498\\
362.01	0.0056960322085237\\
363.01	0.00569376265937483\\
364.01	0.00569144150363884\\
365.01	0.0056890675119921\\
366.01	0.00568663942388841\\
367.01	0.00568415594681422\\
368.01	0.00568161575555128\\
369.01	0.00567901749145439\\
370.01	0.00567635976175055\\
371.01	0.00567364113886814\\
372.01	0.0056708601598084\\
373.01	0.00566801532556912\\
374.01	0.00566510510063793\\
375.01	0.00566212791257145\\
376.01	0.00565908215168273\\
377.01	0.00565596617085954\\
378.01	0.00565277828554576\\
379.01	0.00564951677391856\\
380.01	0.00564617987730409\\
381.01	0.00564276580087828\\
382.01	0.00563927271471126\\
383.01	0.00563569875522094\\
384.01	0.00563204202711578\\
385.01	0.00562830060591658\\
386.01	0.00562447254116693\\
387.01	0.00562055586045687\\
388.01	0.00561654857440534\\
389.01	0.00561244868277319\\
390.01	0.00560825418190138\\
391.01	0.00560396307370325\\
392.01	0.00559957337646969\\
393.01	0.00559508313778593\\
394.01	0.00559049044989649\\
395.01	0.00558579346789798\\
396.01	0.00558099043117951\\
397.01	0.00557607968857266\\
398.01	0.00557105972770409\\
399.01	0.00556592920906667\\
400.01	0.00556068700532121\\
401.01	0.00555533224630952\\
402.01	0.00554986437016528\\
403.01	0.00554428318074\\
404.01	0.0055385889112695\\
405.01	0.00553278229373854\\
406.01	0.00552686463268604\\
407.01	0.00552083788111343\\
408.01	0.00551470471457791\\
409.01	0.0055084685972551\\
410.01	0.00550213383045587\\
411.01	0.00549570556939092\\
412.01	0.00548918978734055\\
413.01	0.0054825931570487\\
414.01	0.00547592280609357\\
415.01	0.00546918588474558\\
416.01	0.00546238885944615\\
417.01	0.00545553640981832\\
418.01	0.00544862975833197\\
419.01	0.00544166419427558\\
420.01	0.00543459249843498\\
421.01	0.00542737374615082\\
422.01	0.00542000527256111\\
423.01	0.00541248440613606\\
424.01	0.00540480847372568\\
425.01	0.00539697480624451\\
426.01	0.00538898074505829\\
427.01	0.0053808236491421\\
428.01	0.00537250090308864\\
429.01	0.00536400992604847\\
430.01	0.00535534818169118\\
431.01	0.00534651318928491\\
432.01	0.00533750253599418\\
433.01	0.00532831389050632\\
434.01	0.00531894501809819\\
435.01	0.00530939379726229\\
436.01	0.00529965823801376\\
437.01	0.00528973650200004\\
438.01	0.00527962692453595\\
439.01	0.00526932803867838\\
440.01	0.00525883860145041\\
441.01	0.00524815762230244\\
442.01	0.00523728439388126\\
443.01	0.00522621852513802\\
444.01	0.00521495997676421\\
445.01	0.00520350909887938\\
446.01	0.00519186667081365\\
447.01	0.00518003394271962\\
448.01	0.00516801267861244\\
449.01	0.00515580520026193\\
450.01	0.00514341443114205\\
451.01	0.00513084393937215\\
452.01	0.00511809797825092\\
453.01	0.00510518152257694\\
454.01	0.0050921002984596\\
455.01	0.00507886080374135\\
456.01	0.00506547031546919\\
457.01	0.00505193688006244\\
458.01	0.00503826928093613\\
459.01	0.00502447697736356\\
460.01	0.00501057000734561\\
461.01	0.0049965588462518\\
462.01	0.00498245421214625\\
463.01	0.00496826680818446\\
464.01	0.00495400699256695\\
465.01	0.00493968436772454\\
466.01	0.00492530728334791\\
467.01	0.00491088225358604\\
468.01	0.00489641329867669\\
469.01	0.00488190123757946\\
470.01	0.00486734298392155\\
471.01	0.0048527309371367\\
472.01	0.00483805262034577\\
473.01	0.0048232908051748\\
474.01	0.00480842449390443\\
475.01	0.00479343143458618\\
476.01	0.00477830033022567\\
477.01	0.00476303646284201\\
478.01	0.00474764684725635\\
479.01	0.00473213867838251\\
480.01	0.00471651923707135\\
481.01	0.00470079577686061\\
482.01	0.00468497539039543\\
483.01	0.00466906485480435\\
484.01	0.00465307045607682\\
485.01	0.00463699779360229\\
486.01	0.00462085156753712\\
487.01	0.00460463535367401\\
488.01	0.00458835137305685\\
489.01	0.00457200026678467\\
490.01	0.00455558089029488\\
491.01	0.0045390901458345\\
492.01	0.00452252287660589\\
493.01	0.0045058718507412\\
494.01	0.00448912786693964\\
495.01	0.00447228001483836\\
496.01	0.00445531611978078\\
497.01	0.00443822338997197\\
498.01	0.00442098925806168\\
499.01	0.00440360236002616\\
500.01	0.00438605350842562\\
501.01	0.00436833633741155\\
502.01	0.00435044623385012\\
503.01	0.00433237797366445\\
504.01	0.00431412521693101\\
505.01	0.00429568048089654\\
506.01	0.00427703514314936\\
507.01	0.00425817947909612\\
508.01	0.00423910273824293\\
509.01	0.00421979326213203\\
510.01	0.00420023864409566\\
511.01	0.00418042592702429\\
512.01	0.00416034182994515\\
513.01	0.00413997298735496\\
514.01	0.00411930617726896\\
515.01	0.00409832850577516\\
516.01	0.00407702750949197\\
517.01	0.00405539113648135\\
518.01	0.00403340757746375\\
519.01	0.00401106495778801\\
520.01	0.00398835105619131\\
521.01	0.00396525327661902\\
522.01	0.00394175872708765\\
523.01	0.00391785430582586\\
524.01	0.00389352678603324\\
525.01	0.00386876289331736\\
526.01	0.00384354936936821\\
527.01	0.00381787301565537\\
528.01	0.00379172071226016\\
529.01	0.00376507940970956\\
530.01	0.00373793609609847\\
531.01	0.00371027774778276\\
532.01	0.00368209127865747\\
533.01	0.00365336350813047\\
534.01	0.00362408116246383\\
535.01	0.00359423089568912\\
536.01	0.0035637993083127\\
537.01	0.00353277295895014\\
538.01	0.00350113836867015\\
539.01	0.00346888201873439\\
540.01	0.00343599034334621\\
541.01	0.00340244971988049\\
542.01	0.0033682464596358\\
543.01	0.00333336680210739\\
544.01	0.00329779691482852\\
545.01	0.00326152289880476\\
546.01	0.00322453079706372\\
547.01	0.00318680660347261\\
548.01	0.00314833627127922\\
549.01	0.00310910572210278\\
550.01	0.0030691008562936\\
551.01	0.00302830756557809\\
552.01	0.00298671174877585\\
553.01	0.00294429933111708\\
554.01	0.00290105628736226\\
555.01	0.00285696866863559\\
556.01	0.002812022632798\\
557.01	0.00276620447842027\\
558.01	0.00271950068276585\\
559.01	0.0026718979443291\\
560.01	0.0026233832304573\\
561.01	0.00257394383053661\\
562.01	0.00252356741517104\\
563.01	0.00247224210175046\\
564.01	0.00241995652680611\\
565.01	0.00236669992559503\\
566.01	0.00231246221942173\\
567.01	0.00225723411125625\\
568.01	0.00220100719020869\\
569.01	0.00214377404539634\\
570.01	0.00208552838970178\\
571.01	0.00202626519387332\\
572.01	0.00196598083135666\\
573.01	0.00190467323416349\\
574.01	0.00184234205996039\\
575.01	0.00177898887038548\\
576.01	0.00171461732035613\\
577.01	0.00164923335779841\\
578.01	0.00158284543280214\\
579.01	0.00151546471464598\\
580.01	0.00144710531442435\\
581.01	0.00137778451008954\\
582.01	0.00130752296955477\\
583.01	0.00123634496602094\\
584.01	0.00116427857781718\\
585.01	0.00109135586269028\\
586.01	0.00101761299352925\\
587.01	0.000943090338822979\\
588.01	0.000867832466552738\\
589.01	0.000791888044500708\\
590.01	0.000715309602850512\\
591.01	0.00063815311614427\\
592.01	0.000560477350747159\\
593.01	0.000482342910469856\\
594.01	0.000403810896310433\\
595.01	0.000324941075665367\\
596.01	0.00024578943091586\\
597.01	0.000166525623744458\\
598.01	9.17366970063782e-05\\
599.01	2.94669364271135e-05\\
599.02	2.89574269585688e-05\\
599.03	2.84509530852801e-05\\
599.04	2.79475443462508e-05\\
599.05	2.7447230571279e-05\\
599.06	2.69500418838293e-05\\
599.07	2.64560087039432e-05\\
599.08	2.59651617511691e-05\\
599.09	2.54775320475235e-05\\
599.1	2.49931509204836e-05\\
599.11	2.45120500060175e-05\\
599.12	2.40342612516185e-05\\
599.13	2.35598169193996e-05\\
599.14	2.30887495892024e-05\\
599.15	2.26210921617405e-05\\
599.16	2.215687786177e-05\\
599.17	2.16961402412976e-05\\
599.18	2.12389131828191e-05\\
599.19	2.07852309025824e-05\\
599.2	2.03351279538938e-05\\
599.21	1.98886392304542e-05\\
599.22	1.94457999697188e-05\\
599.23	1.90066457563063e-05\\
599.24	1.85712125254211e-05\\
599.25	1.81395365663334e-05\\
599.26	1.77116564095865e-05\\
599.27	1.72876122135658e-05\\
599.28	1.68674445361946e-05\\
599.29	1.64511943388859e-05\\
599.3	1.60389029905273e-05\\
599.31	1.56306122715087e-05\\
599.32	1.5226364377792e-05\\
599.33	1.48262019250105e-05\\
599.34	1.44301679526268e-05\\
599.35	1.40383059281154e-05\\
599.36	1.36506597511916e-05\\
599.37	1.32672737580847e-05\\
599.38	1.28881927258535e-05\\
599.39	1.25134618767404e-05\\
599.4	1.21431268825696e-05\\
599.41	1.17772338691924e-05\\
599.42	1.14158294209719e-05\\
599.43	1.10589605853174e-05\\
599.44	1.07066748772523e-05\\
599.45	1.03590202840433e-05\\
599.46	1.00160452698606e-05\\
599.47	9.67779878049101e-06\\
599.48	9.34433024810111e-06\\
599.49	9.01568959604283e-06\\
599.5	8.69192724369319e-06\\
599.51	8.37309411137396e-06\\
599.52	8.05924162529219e-06\\
599.53	7.75042172253791e-06\\
599.54	7.4466868561357e-06\\
599.55	7.14809000013084e-06\\
599.56	6.85468465475535e-06\\
599.57	6.56652485161134e-06\\
599.58	6.28366515892896e-06\\
599.59	6.00616068686249e-06\\
599.6	5.73406709284546e-06\\
599.61	5.46744058700296e-06\\
599.62	5.2063379376039e-06\\
599.63	4.95081647657741e-06\\
599.64	4.70093410508479e-06\\
599.65	4.45674929914174e-06\\
599.66	4.21832111529089e-06\\
599.67	3.98570919634376e-06\\
599.68	3.75897377716608e-06\\
599.69	3.53817569053241e-06\\
599.7	3.32337637303295e-06\\
599.71	3.11463787102881e-06\\
599.72	2.91202284668363e-06\\
599.73	2.71559458404555e-06\\
599.74	2.52541699518292e-06\\
599.75	2.34155462640155e-06\\
599.76	2.16407266449663e-06\\
599.77	1.99303694307755e-06\\
599.78	1.82851394897598e-06\\
599.79	1.67057082867475e-06\\
599.8	1.51927539483211e-06\\
599.81	1.37469613287027e-06\\
599.82	1.23690220760718e-06\\
599.83	1.10596346997172e-06\\
599.84	9.81950463784659e-07\\
599.85	8.64934432590059e-07\\
599.86	7.54987326587186e-07\\
599.87	6.52181809583305e-07\\
599.88	5.56591266064402e-07\\
599.89	4.68289808295413e-07\\
599.9	3.87352283521061e-07\\
599.91	3.13854281218731e-07\\
599.92	2.4787214042421e-07\\
599.93	1.8948295714763e-07\\
599.94	1.38764591834512e-07\\
599.95	9.57956769204876e-08\\
599.96	6.06556244606149e-08\\
599.97	3.34246338194039e-08\\
599.98	1.4183699454523e-08\\
599.99	3.01461875948372e-09\\
600	0\\
};
\addplot [color=mycolor20,solid,forget plot]
  table[row sep=crcr]{%
0.01	0.00553720637056298\\
1.01	0.00553720532562888\\
2.01	0.00553720425853025\\
3.01	0.00553720316879582\\
4.01	0.0055372020559441\\
5.01	0.00553720091948356\\
6.01	0.00553719975891235\\
7.01	0.00553719857371746\\
8.01	0.00553719736337522\\
9.01	0.00553719612735087\\
10.01	0.00553719486509796\\
11.01	0.00553719357605868\\
12.01	0.0055371922596631\\
13.01	0.00553719091532918\\
14.01	0.00553718954246257\\
15.01	0.00553718814045607\\
16.01	0.00553718670868974\\
17.01	0.00553718524653015\\
18.01	0.00553718375333059\\
19.01	0.00553718222843034\\
20.01	0.00553718067115473\\
21.01	0.00553717908081458\\
22.01	0.0055371774567061\\
23.01	0.00553717579811048\\
24.01	0.00553717410429338\\
25.01	0.005537172374505\\
26.01	0.00553717060797924\\
27.01	0.00553716880393391\\
28.01	0.00553716696157009\\
29.01	0.0055371650800717\\
30.01	0.00553716315860519\\
31.01	0.0055371611963193\\
32.01	0.00553715919234463\\
33.01	0.00553715714579308\\
34.01	0.0055371550557576\\
35.01	0.00553715292131169\\
36.01	0.00553715074150919\\
37.01	0.00553714851538364\\
38.01	0.00553714624194788\\
39.01	0.00553714392019385\\
40.01	0.00553714154909165\\
41.01	0.00553713912758965\\
42.01	0.00553713665461338\\
43.01	0.00553713412906581\\
44.01	0.00553713154982601\\
45.01	0.00553712891574957\\
46.01	0.00553712622566704\\
47.01	0.00553712347838437\\
48.01	0.00553712067268193\\
49.01	0.00553711780731382\\
50.01	0.00553711488100765\\
51.01	0.00553711189246362\\
52.01	0.00553710884035421\\
53.01	0.0055371057233236\\
54.01	0.00553710253998694\\
55.01	0.00553709928892959\\
56.01	0.00553709596870648\\
57.01	0.00553709257784203\\
58.01	0.00553708911482873\\
59.01	0.0055370855781272\\
60.01	0.00553708196616446\\
61.01	0.0055370782773344\\
62.01	0.00553707450999615\\
63.01	0.00553707066247405\\
64.01	0.00553706673305628\\
65.01	0.00553706271999447\\
66.01	0.00553705862150265\\
67.01	0.00553705443575672\\
68.01	0.00553705016089349\\
69.01	0.00553704579500971\\
70.01	0.00553704133616168\\
71.01	0.00553703678236353\\
72.01	0.0055370321315874\\
73.01	0.00553702738176148\\
74.01	0.0055370225307697\\
75.01	0.00553701757645073\\
76.01	0.00553701251659681\\
77.01	0.00553700734895287\\
78.01	0.00553700207121549\\
79.01	0.00553699668103181\\
80.01	0.0055369911759986\\
81.01	0.00553698555366081\\
82.01	0.00553697981151105\\
83.01	0.0055369739469879\\
84.01	0.00553696795747521\\
85.01	0.00553696184030048\\
86.01	0.00553695559273409\\
87.01	0.00553694921198756\\
88.01	0.00553694269521272\\
89.01	0.00553693603950039\\
90.01	0.0055369292418787\\
91.01	0.00553692229931204\\
92.01	0.00553691520869987\\
93.01	0.00553690796687468\\
94.01	0.00553690057060118\\
95.01	0.00553689301657463\\
96.01	0.00553688530141914\\
97.01	0.00553687742168652\\
98.01	0.00553686937385437\\
99.01	0.00553686115432469\\
100.01	0.00553685275942206\\
101.01	0.00553684418539229\\
102.01	0.00553683542840027\\
103.01	0.00553682648452875\\
104.01	0.00553681734977601\\
105.01	0.00553680802005464\\
106.01	0.00553679849118908\\
107.01	0.0055367887589142\\
108.01	0.00553677881887289\\
109.01	0.00553676866661483\\
110.01	0.00553675829759354\\
111.01	0.00553674770716508\\
112.01	0.00553673689058559\\
113.01	0.00553672584300898\\
114.01	0.00553671455948508\\
115.01	0.00553670303495715\\
116.01	0.00553669126425974\\
117.01	0.00553667924211642\\
118.01	0.00553666696313701\\
119.01	0.00553665442181534\\
120.01	0.00553664161252666\\
121.01	0.00553662852952538\\
122.01	0.00553661516694214\\
123.01	0.00553660151878125\\
124.01	0.00553658757891789\\
125.01	0.00553657334109543\\
126.01	0.00553655879892246\\
127.01	0.00553654394587013\\
128.01	0.00553652877526866\\
129.01	0.00553651328030499\\
130.01	0.00553649745401897\\
131.01	0.00553648128930096\\
132.01	0.00553646477888751\\
133.01	0.00553644791535928\\
134.01	0.00553643069113668\\
135.01	0.00553641309847707\\
136.01	0.00553639512947085\\
137.01	0.00553637677603802\\
138.01	0.00553635802992419\\
139.01	0.00553633888269745\\
140.01	0.0055363193257438\\
141.01	0.0055362993502634\\
142.01	0.0055362789472671\\
143.01	0.00553625810757184\\
144.01	0.0055362368217962\\
145.01	0.0055362150803568\\
146.01	0.00553619287346342\\
147.01	0.00553617019111457\\
148.01	0.00553614702309314\\
149.01	0.00553612335896175\\
150.01	0.00553609918805758\\
151.01	0.00553607449948804\\
152.01	0.00553604928212504\\
153.01	0.00553602352460071\\
154.01	0.00553599721530188\\
155.01	0.00553597034236451\\
156.01	0.00553594289366837\\
157.01	0.00553591485683214\\
158.01	0.00553588621920657\\
159.01	0.00553585696786985\\
160.01	0.00553582708962081\\
161.01	0.0055357965709733\\
162.01	0.00553576539814999\\
163.01	0.00553573355707596\\
164.01	0.00553570103337233\\
165.01	0.0055356678123497\\
166.01	0.00553563387900109\\
167.01	0.00553559921799565\\
168.01	0.00553556381367109\\
169.01	0.00553552765002675\\
170.01	0.0055354907107165\\
171.01	0.00553545297904048\\
172.01	0.00553541443793807\\
173.01	0.00553537506998019\\
174.01	0.00553533485736078\\
175.01	0.00553529378188889\\
176.01	0.00553525182498051\\
177.01	0.00553520896764966\\
178.01	0.00553516519049994\\
179.01	0.00553512047371552\\
180.01	0.00553507479705218\\
181.01	0.00553502813982761\\
182.01	0.00553498048091255\\
183.01	0.00553493179872054\\
184.01	0.00553488207119826\\
185.01	0.00553483127581537\\
186.01	0.00553477938955405\\
187.01	0.00553472638889845\\
188.01	0.00553467224982427\\
189.01	0.00553461694778693\\
190.01	0.00553456045771075\\
191.01	0.00553450275397771\\
192.01	0.00553444381041517\\
193.01	0.0055343836002841\\
194.01	0.00553432209626664\\
195.01	0.00553425927045353\\
196.01	0.00553419509433156\\
197.01	0.00553412953876997\\
198.01	0.0055340625740075\\
199.01	0.00553399416963834\\
200.01	0.00553392429459836\\
201.01	0.0055338529171507\\
202.01	0.00553378000487124\\
203.01	0.00553370552463382\\
204.01	0.00553362944259458\\
205.01	0.00553355172417663\\
206.01	0.00553347233405399\\
207.01	0.0055333912361354\\
208.01	0.00553330839354808\\
209.01	0.00553322376862004\\
210.01	0.00553313732286324\\
211.01	0.00553304901695543\\
212.01	0.00553295881072252\\
213.01	0.00553286666311976\\
214.01	0.00553277253221284\\
215.01	0.00553267637515866\\
216.01	0.00553257814818546\\
217.01	0.00553247780657256\\
218.01	0.00553237530463007\\
219.01	0.00553227059567756\\
220.01	0.00553216363202278\\
221.01	0.00553205436493915\\
222.01	0.00553194274464385\\
223.01	0.00553182872027455\\
224.01	0.00553171223986592\\
225.01	0.0055315932503259\\
226.01	0.00553147169741097\\
227.01	0.00553134752570107\\
228.01	0.00553122067857429\\
229.01	0.00553109109818053\\
230.01	0.00553095872541492\\
231.01	0.00553082349989038\\
232.01	0.00553068535990977\\
233.01	0.00553054424243772\\
234.01	0.00553040008307083\\
235.01	0.00553025281600856\\
236.01	0.00553010237402275\\
237.01	0.00552994868842619\\
238.01	0.00552979168904128\\
239.01	0.00552963130416764\\
240.01	0.00552946746054834\\
241.01	0.00552930008333689\\
242.01	0.00552912909606222\\
243.01	0.00552895442059329\\
244.01	0.00552877597710296\\
245.01	0.0055285936840316\\
246.01	0.00552840745804875\\
247.01	0.00552821721401476\\
248.01	0.00552802286494169\\
249.01	0.00552782432195278\\
250.01	0.00552762149424138\\
251.01	0.00552741428902916\\
252.01	0.00552720261152324\\
253.01	0.00552698636487199\\
254.01	0.00552676545012045\\
255.01	0.00552653976616447\\
256.01	0.00552630920970405\\
257.01	0.00552607367519561\\
258.01	0.00552583305480285\\
259.01	0.00552558723834706\\
260.01	0.0055253361132561\\
261.01	0.00552507956451263\\
262.01	0.00552481747460019\\
263.01	0.00552454972344926\\
264.01	0.00552427618838158\\
265.01	0.00552399674405377\\
266.01	0.00552371126239857\\
267.01	0.00552341961256598\\
268.01	0.00552312166086285\\
269.01	0.00552281727069066\\
270.01	0.00552250630248259\\
271.01	0.00552218861363884\\
272.01	0.00552186405846033\\
273.01	0.00552153248808176\\
274.01	0.00552119375040224\\
275.01	0.00552084769001537\\
276.01	0.00552049414813723\\
277.01	0.00552013296253245\\
278.01	0.00551976396743991\\
279.01	0.00551938699349545\\
280.01	0.00551900186765388\\
281.01	0.0055186084131092\\
282.01	0.00551820644921202\\
283.01	0.00551779579138705\\
284.01	0.00551737625104743\\
285.01	0.00551694763550737\\
286.01	0.00551650974789376\\
287.01	0.00551606238705486\\
288.01	0.00551560534746743\\
289.01	0.00551513841914275\\
290.01	0.00551466138752897\\
291.01	0.00551417403341258\\
292.01	0.00551367613281776\\
293.01	0.00551316745690295\\
294.01	0.00551264777185604\\
295.01	0.00551211683878689\\
296.01	0.00551157441361747\\
297.01	0.00551102024697017\\
298.01	0.00551045408405385\\
299.01	0.00550987566454698\\
300.01	0.00550928472247866\\
301.01	0.00550868098610768\\
302.01	0.00550806417779831\\
303.01	0.00550743401389427\\
304.01	0.00550679020459034\\
305.01	0.00550613245380003\\
306.01	0.00550546045902275\\
307.01	0.00550477391120687\\
308.01	0.00550407249461015\\
309.01	0.00550335588665917\\
310.01	0.00550262375780388\\
311.01	0.00550187577137102\\
312.01	0.00550111158341431\\
313.01	0.00550033084256196\\
314.01	0.00549953318986141\\
315.01	0.00549871825862176\\
316.01	0.00549788567425313\\
317.01	0.00549703505410395\\
318.01	0.00549616600729461\\
319.01	0.00549527813454928\\
320.01	0.0054943710280256\\
321.01	0.00549344427114077\\
322.01	0.00549249743839581\\
323.01	0.0054915300951978\\
324.01	0.00549054179767959\\
325.01	0.0054895320925169\\
326.01	0.00548850051674458\\
327.01	0.00548744659757026\\
328.01	0.00548636985218627\\
329.01	0.00548526978758043\\
330.01	0.00548414590034619\\
331.01	0.00548299767649076\\
332.01	0.00548182459124306\\
333.01	0.00548062610886197\\
334.01	0.00547940168244379\\
335.01	0.0054781507537308\\
336.01	0.00547687275292085\\
337.01	0.00547556709847771\\
338.01	0.00547423319694501\\
339.01	0.00547287044276185\\
340.01	0.00547147821808205\\
341.01	0.00547005589259885\\
342.01	0.00546860282337389\\
343.01	0.00546711835467275\\
344.01	0.00546560181780855\\
345.01	0.00546405253099437\\
346.01	0.00546246979920531\\
347.01	0.0054608529140528\\
348.01	0.0054592011536728\\
349.01	0.00545751378262847\\
350.01	0.00545579005183191\\
351.01	0.00545402919848476\\
352.01	0.00545223044604209\\
353.01	0.00545039300420147\\
354.01	0.00544851606892138\\
355.01	0.00544659882247069\\
356.01	0.00544464043351565\\
357.01	0.00544264005724628\\
358.01	0.00544059683554877\\
359.01	0.0054385098972271\\
360.01	0.00543637835828238\\
361.01	0.00543420132225272\\
362.01	0.00543197788062396\\
363.01	0.00542970711331608\\
364.01	0.00542738808925562\\
365.01	0.00542501986704131\\
366.01	0.00542260149571503\\
367.01	0.0054201320156464\\
368.01	0.00541761045954511\\
369.01	0.00541503585361179\\
370.01	0.00541240721884241\\
371.01	0.00540972357250119\\
372.01	0.00540698392977664\\
373.01	0.00540418730563962\\
374.01	0.00540133271692067\\
375.01	0.00539841918462585\\
376.01	0.00539544573651215\\
377.01	0.00539241140994294\\
378.01	0.00538931525504518\\
379.01	0.0053861563381919\\
380.01	0.00538293374582873\\
381.01	0.00537964658867044\\
382.01	0.00537629400628247\\
383.01	0.00537287517206913\\
384.01	0.00536938929867904\\
385.01	0.0053658356438392\\
386.01	0.00536221351661703\\
387.01	0.0053585222841025\\
388.01	0.00535476137848928\\
389.01	0.00535093030451537\\
390.01	0.00534702864720414\\
391.01	0.00534305607981669\\
392.01	0.00533901237189306\\
393.01	0.00533489739721644\\
394.01	0.00533071114147749\\
395.01	0.00532645370935609\\
396.01	0.00532212533064996\\
397.01	0.00531772636498756\\
398.01	0.00531325730454725\\
399.01	0.00530871877406405\\
400.01	0.00530411152725457\\
401.01	0.00529943643860297\\
402.01	0.00529469448925469\\
403.01	0.00528988674554758\\
404.01	0.00528501432848217\\
405.01	0.00528007837222782\\
406.01	0.00527507996957568\\
407.01	0.00527002010216799\\
408.01	0.00526489955338293\\
409.01	0.00525971880207216\\
410.01	0.00525447789606453\\
411.01	0.0052491763057043\\
412.01	0.00524381275999656\\
413.01	0.00523838507168268\\
414.01	0.00523288996340713\\
415.01	0.00522732291605919\\
416.01	0.00522167807372655\\
417.01	0.00521594825942429\\
418.01	0.00521012518459015\\
419.01	0.00520419997708913\\
420.01	0.00519816522500772\\
421.01	0.00519201847717648\\
422.01	0.0051857584488522\\
423.01	0.00517938390285342\\
424.01	0.00517289365542751\\
425.01	0.00516628658248699\\
426.01	0.00515956162622112\\
427.01	0.00515271780208505\\
428.01	0.00514575420616415\\
429.01	0.00513867002290691\\
430.01	0.00513146453321292\\
431.01	0.00512413712285335\\
432.01	0.00511668729119532\\
433.01	0.00510911466018588\\
434.01	0.00510141898354174\\
435.01	0.0050936001560743\\
436.01	0.00508565822306024\\
437.01	0.00507759338954814\\
438.01	0.00506940602946684\\
439.01	0.00506109669437276\\
440.01	0.00505266612164202\\
441.01	0.00504411524187867\\
442.01	0.00503544518526552\\
443.01	0.00502665728654644\\
444.01	0.00501775308827338\\
445.01	0.00500873434190417\\
446.01	0.00499960300627798\\
447.01	0.0049903612429404\\
448.01	0.00498101140773014\\
449.01	0.00497155603798732\\
450.01	0.00496199783469326\\
451.01	0.00495233963881487\\
452.01	0.00494258440110989\\
453.01	0.00493273514465536\\
454.01	0.00492279491941333\\
455.01	0.00491276674824503\\
456.01	0.00490265356395906\\
457.01	0.00489245813724484\\
458.01	0.00488218299572643\\
459.01	0.00487183033490665\\
460.01	0.00486140192248767\\
461.01	0.00485089899848782\\
462.01	0.00484032217475886\\
463.01	0.00482967133896136\\
464.01	0.00481894556979449\\
465.01	0.00480814307225671\\
466.01	0.00479726114385588\\
467.01	0.00478629618479182\\
468.01	0.00477524376685169\\
469.01	0.00476409877648323\\
470.01	0.00475285564626067\\
471.01	0.00474150868417128\\
472.01	0.00473005249939949\\
473.01	0.00471848250288801\\
474.01	0.00470679542534797\\
475.01	0.00469498973555899\\
476.01	0.00468306554036529\\
477.01	0.00467102333681361\\
478.01	0.00465886335633616\\
479.01	0.00464658549313939\\
480.01	0.00463418925027865\\
481.01	0.00462167368557904\\
482.01	0.00460903735892477\\
483.01	0.00459627828277447\\
484.01	0.00458339387811139\\
485.01	0.00457038093836366\\
486.01	0.00455723560412103\\
487.01	0.00454395335166238\\
488.01	0.00453052899834742\\
489.01	0.00451695672772696\\
490.01	0.00450323013668929\\
491.01	0.00448934230598205\\
492.01	0.00447528589388215\\
493.01	0.00446105325052688\\
494.01	0.00444663654737016\\
495.01	0.00443202791238837\\
496.01	0.00441721955719538\\
497.01	0.0044022038776158\\
498.01	0.00438697350549951\\
499.01	0.00437152128856624\\
500.01	0.00435584018016472\\
501.01	0.00433992303804354\\
502.01	0.00432376239648625\\
503.01	0.00430735037702502\\
504.01	0.00429067870800565\\
505.01	0.00427373876057116\\
506.01	0.00425652159158788\\
507.01	0.00423901799212023\\
508.01	0.00422121853940893\\
509.01	0.00420311364962767\\
510.01	0.00418469362806572\\
511.01	0.00416594871292255\\
512.01	0.00414686910874527\\
513.01	0.00412744500587848\\
514.01	0.00410766658334731\\
515.01	0.0040875239945441\\
516.01	0.00406700733806003\\
517.01	0.00404610661987399\\
518.01	0.00402481171727932\\
519.01	0.00400311235781672\\
520.01	0.00398099812237176\\
521.01	0.00395845846277685\\
522.01	0.00393548271928023\\
523.01	0.0039120601342729\\
524.01	0.00388817986146693\\
525.01	0.00386383097008843\\
526.01	0.00383900244409569\\
527.01	0.00381368317698033\\
528.01	0.00378786196329684\\
529.01	0.0037615274885997\\
530.01	0.00373466831978622\\
531.01	0.00370727289777221\\
532.01	0.00367932953377338\\
533.01	0.00365082640913506\\
534.01	0.00362175157696972\\
535.01	0.00359209296338246\\
536.01	0.00356183836755163\\
537.01	0.00353097546093829\\
538.01	0.00349949178610511\\
539.01	0.00346737475569416\\
540.01	0.00343461165211991\\
541.01	0.00340118962845988\\
542.01	0.00336709571087169\\
543.01	0.00333231680265383\\
544.01	0.00329683968985111\\
545.01	0.0032606510481884\\
546.01	0.00322373745118108\\
547.01	0.00318608537950776\\
548.01	0.00314768123191225\\
549.01	0.00310851133792719\\
550.01	0.00306856197269586\\
551.01	0.0030278193741351\\
552.01	0.00298626976264715\\
553.01	0.00294389936356977\\
554.01	0.0029006944325494\\
555.01	0.0028566412840523\\
556.01	0.00281172632328039\\
557.01	0.00276593608181439\\
558.01	0.00271925725734165\\
559.01	0.0026716767578422\\
560.01	0.00262318175062\\
561.01	0.00257375971658235\\
562.01	0.0025233985101925\\
563.01	0.00247208642554797\\
564.01	0.0024198122690649\\
565.01	0.00236656543928021\\
566.01	0.00231233601430398\\
567.01	0.00225711484746698\\
568.01	0.0022008936717135\\
569.01	0.00214366521328238\\
570.01	0.00208542331519992\\
571.01	0.0020261630710739\\
572.01	0.0019658809696171\\
573.01	0.00190457505024007\\
574.01	0.00184224506991961\\
575.01	0.00177889268136865\\
576.01	0.00171452162228293\\
577.01	0.00164913791511233\\
578.01	0.00158275007636962\\
579.01	0.00151536933392877\\
580.01	0.00144700985004043\\
581.01	0.0013776889468657\\
582.01	0.0013074273301515\\
583.01	0.00123624930517962\\
584.01	0.00116418297724289\\
585.01	0.00109126042654351\\
586.01	0.00101751784445324\\
587.01	0.000942995614387319\\
588.01	0.000867738315945504\\
589.01	0.000791794625255108\\
590.01	0.00071521707734773\\
591.01	0.000638061647592614\\
592.01	0.000560387098297354\\
593.01	0.000482254023083268\\
594.01	0.000403723504947367\\
595.01	0.000324855283297511\\
596.01	0.000245705299785164\\
597.01	0.000166480802755067\\
598.01	9.17366970063765e-05\\
599.01	2.94669364271135e-05\\
599.02	2.89574269585705e-05\\
599.03	2.84509530852801e-05\\
599.04	2.79475443462508e-05\\
599.05	2.7447230571279e-05\\
599.06	2.69500418838293e-05\\
599.07	2.64560087039432e-05\\
599.08	2.59651617511673e-05\\
599.09	2.54775320475218e-05\\
599.1	2.49931509204836e-05\\
599.11	2.45120500060158e-05\\
599.12	2.40342612516167e-05\\
599.13	2.35598169193978e-05\\
599.14	2.30887495892024e-05\\
599.15	2.26210921617405e-05\\
599.16	2.215687786177e-05\\
599.17	2.16961402412993e-05\\
599.18	2.12389131828191e-05\\
599.19	2.07852309025806e-05\\
599.2	2.03351279538921e-05\\
599.21	1.98886392304542e-05\\
599.22	1.94457999697188e-05\\
599.23	1.90066457563046e-05\\
599.24	1.85712125254211e-05\\
599.25	1.81395365663334e-05\\
599.26	1.77116564095865e-05\\
599.27	1.72876122135658e-05\\
599.28	1.68674445361946e-05\\
599.29	1.64511943388859e-05\\
599.3	1.60389029905273e-05\\
599.31	1.56306122715087e-05\\
599.32	1.52263643777902e-05\\
599.33	1.48262019250087e-05\\
599.34	1.44301679526268e-05\\
599.35	1.40383059281154e-05\\
599.36	1.36506597511916e-05\\
599.37	1.32672737580847e-05\\
599.38	1.28881927258535e-05\\
599.39	1.25134618767404e-05\\
599.4	1.21431268825696e-05\\
599.41	1.17772338691924e-05\\
599.42	1.14158294209719e-05\\
599.43	1.10589605853174e-05\\
599.44	1.0706674877254e-05\\
599.45	1.03590202840433e-05\\
599.46	1.00160452698589e-05\\
599.47	9.67779878049101e-06\\
599.48	9.34433024810284e-06\\
599.49	9.0156895960411e-06\\
599.5	8.69192724369146e-06\\
599.51	8.37309411137396e-06\\
599.52	8.05924162529392e-06\\
599.53	7.75042172253965e-06\\
599.54	7.44668685613396e-06\\
599.55	7.14809000013084e-06\\
599.56	6.85468465475708e-06\\
599.57	6.56652485161308e-06\\
599.58	6.2836651589307e-06\\
599.59	6.00616068686249e-06\\
599.6	5.7340670928472e-06\\
599.61	5.46744058700296e-06\\
599.62	5.20633793760217e-06\\
599.63	4.95081647657741e-06\\
599.64	4.70093410508653e-06\\
599.65	4.45674929914347e-06\\
599.66	4.21832111529089e-06\\
599.67	3.98570919634376e-06\\
599.68	3.75897377716608e-06\\
599.69	3.53817569053415e-06\\
599.7	3.32337637303469e-06\\
599.71	3.11463787102881e-06\\
599.72	2.91202284668536e-06\\
599.73	2.71559458404555e-06\\
599.74	2.52541699518466e-06\\
599.75	2.34155462640155e-06\\
599.76	2.16407266449489e-06\\
599.77	1.99303694307928e-06\\
599.78	1.82851394897598e-06\\
599.79	1.67057082867302e-06\\
599.8	1.51927539483211e-06\\
599.81	1.37469613287027e-06\\
599.82	1.23690220760718e-06\\
599.83	1.10596346997172e-06\\
599.84	9.81950463782924e-07\\
599.85	8.64934432591793e-07\\
599.86	7.54987326587186e-07\\
599.87	6.5218180958504e-07\\
599.88	5.56591266062667e-07\\
599.89	4.68289808295413e-07\\
599.9	3.87352283521061e-07\\
599.91	3.13854281216996e-07\\
599.92	2.47872140425945e-07\\
599.93	1.89482957149364e-07\\
599.94	1.38764591834512e-07\\
599.95	9.57956769222224e-08\\
599.96	6.06556244623496e-08\\
599.97	3.34246338211386e-08\\
599.98	1.4183699454523e-08\\
599.99	3.01461875948372e-09\\
600	0\\
};
\addplot [color=mycolor21,solid,forget plot]
  table[row sep=crcr]{%
0.01	0.00535026209346621\\
1.01	0.00535026104829641\\
2.01	0.00535025998105595\\
3.01	0.00535025889127814\\
4.01	0.0053502577784865\\
5.01	0.00535025664219418\\
6.01	0.00535025548190425\\
7.01	0.00535025429710914\\
8.01	0.00535025308729064\\
9.01	0.00535025185191931\\
10.01	0.00535025059045471\\
11.01	0.0053502493023449\\
12.01	0.00535024798702625\\
13.01	0.00535024664392316\\
14.01	0.00535024527244783\\
15.01	0.0053502438720001\\
16.01	0.00535024244196682\\
17.01	0.00535024098172227\\
18.01	0.005350239490627\\
19.01	0.00535023796802848\\
20.01	0.00535023641325996\\
21.01	0.0053502348256408\\
22.01	0.00535023320447572\\
23.01	0.00535023154905479\\
24.01	0.00535022985865317\\
25.01	0.00535022813253049\\
26.01	0.00535022636993082\\
27.01	0.00535022457008205\\
28.01	0.00535022273219562\\
29.01	0.00535022085546657\\
30.01	0.00535021893907261\\
31.01	0.00535021698217404\\
32.01	0.0053502149839134\\
33.01	0.00535021294341511\\
34.01	0.005350210859785\\
35.01	0.00535020873210986\\
36.01	0.00535020655945719\\
37.01	0.00535020434087453\\
38.01	0.0053502020753894\\
39.01	0.00535019976200858\\
40.01	0.00535019739971781\\
41.01	0.00535019498748146\\
42.01	0.00535019252424177\\
43.01	0.00535019000891849\\
44.01	0.00535018744040847\\
45.01	0.00535018481758538\\
46.01	0.0053501821392988\\
47.01	0.00535017940437382\\
48.01	0.00535017661161067\\
49.01	0.0053501737597842\\
50.01	0.00535017084764323\\
51.01	0.00535016787390984\\
52.01	0.00535016483727916\\
53.01	0.00535016173641865\\
54.01	0.00535015856996733\\
55.01	0.00535015533653546\\
56.01	0.00535015203470388\\
57.01	0.00535014866302299\\
58.01	0.00535014522001257\\
59.01	0.00535014170416103\\
60.01	0.0053501381139247\\
61.01	0.00535013444772711\\
62.01	0.00535013070395823\\
63.01	0.00535012688097388\\
64.01	0.00535012297709494\\
65.01	0.00535011899060661\\
66.01	0.00535011491975781\\
67.01	0.00535011076276013\\
68.01	0.00535010651778722\\
69.01	0.00535010218297388\\
70.01	0.00535009775641527\\
71.01	0.00535009323616632\\
72.01	0.00535008862024022\\
73.01	0.00535008390660834\\
74.01	0.0053500790931987\\
75.01	0.00535007417789551\\
76.01	0.00535006915853775\\
77.01	0.00535006403291869\\
78.01	0.00535005879878457\\
79.01	0.00535005345383386\\
80.01	0.00535004799571588\\
81.01	0.00535004242203036\\
82.01	0.00535003673032574\\
83.01	0.00535003091809843\\
84.01	0.00535002498279168\\
85.01	0.00535001892179418\\
86.01	0.0053500127324394\\
87.01	0.00535000641200381\\
88.01	0.00534999995770629\\
89.01	0.00534999336670627\\
90.01	0.00534998663610307\\
91.01	0.0053499797629341\\
92.01	0.0053499727441738\\
93.01	0.00534996557673259\\
94.01	0.00534995825745473\\
95.01	0.00534995078311778\\
96.01	0.00534994315043042\\
97.01	0.00534993535603153\\
98.01	0.00534992739648842\\
99.01	0.00534991926829532\\
100.01	0.00534991096787215\\
101.01	0.00534990249156239\\
102.01	0.00534989383563186\\
103.01	0.00534988499626685\\
104.01	0.00534987596957271\\
105.01	0.00534986675157164\\
106.01	0.00534985733820153\\
107.01	0.00534984772531359\\
108.01	0.00534983790867085\\
109.01	0.00534982788394607\\
110.01	0.00534981764672022\\
111.01	0.0053498071924799\\
112.01	0.00534979651661583\\
113.01	0.00534978561442046\\
114.01	0.00534977448108632\\
115.01	0.00534976311170351\\
116.01	0.00534975150125743\\
117.01	0.00534973964462699\\
118.01	0.00534972753658188\\
119.01	0.00534971517178064\\
120.01	0.00534970254476804\\
121.01	0.00534968964997274\\
122.01	0.00534967648170465\\
123.01	0.00534966303415268\\
124.01	0.00534964930138204\\
125.01	0.00534963527733151\\
126.01	0.00534962095581067\\
127.01	0.00534960633049756\\
128.01	0.00534959139493537\\
129.01	0.00534957614252992\\
130.01	0.00534956056654668\\
131.01	0.00534954466010754\\
132.01	0.00534952841618806\\
133.01	0.00534951182761399\\
134.01	0.00534949488705868\\
135.01	0.00534947758703889\\
136.01	0.00534945991991232\\
137.01	0.00534944187787386\\
138.01	0.00534942345295226\\
139.01	0.00534940463700621\\
140.01	0.00534938542172121\\
141.01	0.00534936579860556\\
142.01	0.00534934575898676\\
143.01	0.00534932529400738\\
144.01	0.00534930439462162\\
145.01	0.00534928305159063\\
146.01	0.00534926125547865\\
147.01	0.0053492389966496\\
148.01	0.00534921626526133\\
149.01	0.00534919305126237\\
150.01	0.00534916934438704\\
151.01	0.00534914513415085\\
152.01	0.00534912040984623\\
153.01	0.00534909516053715\\
154.01	0.00534906937505456\\
155.01	0.00534904304199172\\
156.01	0.00534901614969872\\
157.01	0.00534898868627732\\
158.01	0.0053489606395759\\
159.01	0.00534893199718371\\
160.01	0.00534890274642584\\
161.01	0.00534887287435704\\
162.01	0.00534884236775645\\
163.01	0.00534881121312145\\
164.01	0.0053487793966615\\
165.01	0.00534874690429258\\
166.01	0.0053487137216302\\
167.01	0.00534867983398361\\
168.01	0.0053486452263491\\
169.01	0.00534860988340292\\
170.01	0.00534857378949483\\
171.01	0.00534853692864163\\
172.01	0.00534849928451919\\
173.01	0.00534846084045555\\
174.01	0.0053484215794237\\
175.01	0.00534838148403353\\
176.01	0.00534834053652487\\
177.01	0.00534829871875869\\
178.01	0.00534825601220972\\
179.01	0.00534821239795825\\
180.01	0.00534816785668118\\
181.01	0.00534812236864375\\
182.01	0.00534807591369094\\
183.01	0.00534802847123837\\
184.01	0.00534798002026301\\
185.01	0.00534793053929424\\
186.01	0.00534788000640404\\
187.01	0.00534782839919714\\
188.01	0.00534777569480136\\
189.01	0.00534772186985746\\
190.01	0.00534766690050877\\
191.01	0.00534761076239052\\
192.01	0.00534755343061893\\
193.01	0.00534749487978045\\
194.01	0.0053474350839205\\
195.01	0.0053473740165318\\
196.01	0.00534731165054265\\
197.01	0.00534724795830508\\
198.01	0.00534718291158238\\
199.01	0.00534711648153705\\
200.01	0.00534704863871773\\
201.01	0.00534697935304602\\
202.01	0.00534690859380354\\
203.01	0.00534683632961815\\
204.01	0.00534676252845023\\
205.01	0.00534668715757838\\
206.01	0.00534661018358502\\
207.01	0.00534653157234164\\
208.01	0.00534645128899402\\
209.01	0.0053463692979464\\
210.01	0.00534628556284588\\
211.01	0.00534620004656689\\
212.01	0.00534611271119413\\
213.01	0.00534602351800639\\
214.01	0.00534593242745903\\
215.01	0.00534583939916739\\
216.01	0.00534574439188814\\
217.01	0.00534564736350164\\
218.01	0.00534554827099344\\
219.01	0.00534544707043504\\
220.01	0.00534534371696496\\
221.01	0.00534523816476892\\
222.01	0.0053451303670598\\
223.01	0.00534502027605716\\
224.01	0.0053449078429661\\
225.01	0.00534479301795617\\
226.01	0.00534467575013941\\
227.01	0.00534455598754815\\
228.01	0.0053444336771125\\
229.01	0.00534430876463645\\
230.01	0.00534418119477529\\
231.01	0.00534405091101063\\
232.01	0.00534391785562617\\
233.01	0.00534378196968224\\
234.01	0.00534364319299068\\
235.01	0.00534350146408818\\
236.01	0.00534335672020978\\
237.01	0.00534320889726148\\
238.01	0.00534305792979283\\
239.01	0.00534290375096808\\
240.01	0.00534274629253751\\
241.01	0.00534258548480774\\
242.01	0.00534242125661182\\
243.01	0.00534225353527838\\
244.01	0.00534208224660047\\
245.01	0.00534190731480294\\
246.01	0.00534172866251073\\
247.01	0.00534154621071516\\
248.01	0.00534135987873995\\
249.01	0.00534116958420659\\
250.01	0.00534097524299939\\
251.01	0.00534077676922931\\
252.01	0.00534057407519673\\
253.01	0.00534036707135504\\
254.01	0.00534015566627153\\
255.01	0.0053399397665895\\
256.01	0.00533971927698778\\
257.01	0.00533949410014039\\
258.01	0.00533926413667565\\
259.01	0.00533902928513409\\
260.01	0.00533878944192549\\
261.01	0.00533854450128538\\
262.01	0.00533829435523067\\
263.01	0.00533803889351393\\
264.01	0.0053377780035775\\
265.01	0.00533751157050692\\
266.01	0.00533723947698208\\
267.01	0.00533696160322902\\
268.01	0.00533667782697017\\
269.01	0.0053363880233736\\
270.01	0.00533609206500142\\
271.01	0.00533578982175755\\
272.01	0.00533548116083401\\
273.01	0.00533516594665643\\
274.01	0.00533484404082944\\
275.01	0.00533451530207933\\
276.01	0.00533417958619711\\
277.01	0.00533383674598056\\
278.01	0.00533348663117434\\
279.01	0.00533312908841\\
280.01	0.00533276396114464\\
281.01	0.00533239108959815\\
282.01	0.00533201031069036\\
283.01	0.00533162145797615\\
284.01	0.00533122436158038\\
285.01	0.00533081884813113\\
286.01	0.00533040474069229\\
287.01	0.00532998185869494\\
288.01	0.00532955001786797\\
289.01	0.00532910903016721\\
290.01	0.0053286587037039\\
291.01	0.00532819884267224\\
292.01	0.00532772924727555\\
293.01	0.00532724971365193\\
294.01	0.0053267600337985\\
295.01	0.00532625999549482\\
296.01	0.00532574938222581\\
297.01	0.00532522797310312\\
298.01	0.00532469554278615\\
299.01	0.00532415186140171\\
300.01	0.00532359669446365\\
301.01	0.00532302980279103\\
302.01	0.00532245094242565\\
303.01	0.00532185986454963\\
304.01	0.00532125631540115\\
305.01	0.00532064003619124\\
306.01	0.00532001076301838\\
307.01	0.00531936822678417\\
308.01	0.005318712153108\\
309.01	0.00531804226224141\\
310.01	0.00531735826898274\\
311.01	0.00531665988259147\\
312.01	0.00531594680670288\\
313.01	0.00531521873924296\\
314.01	0.00531447537234359\\
315.01	0.0053137163922588\\
316.01	0.00531294147928049\\
317.01	0.00531215030765688\\
318.01	0.00531134254551053\\
319.01	0.00531051785475856\\
320.01	0.00530967589103442\\
321.01	0.00530881630361108\\
322.01	0.00530793873532671\\
323.01	0.00530704282251277\\
324.01	0.00530612819492473\\
325.01	0.00530519447567572\\
326.01	0.00530424128117427\\
327.01	0.00530326822106537\\
328.01	0.00530227489817605\\
329.01	0.00530126090846623\\
330.01	0.00530022584098391\\
331.01	0.00529916927782694\\
332.01	0.00529809079411157\\
333.01	0.00529698995794719\\
334.01	0.00529586633041939\\
335.01	0.00529471946558195\\
336.01	0.00529354891045676\\
337.01	0.00529235420504582\\
338.01	0.00529113488235237\\
339.01	0.00528989046841502\\
340.01	0.00528862048235526\\
341.01	0.00528732443643789\\
342.01	0.00528600183614752\\
343.01	0.00528465218028151\\
344.01	0.00528327496106036\\
345.01	0.0052818696642579\\
346.01	0.00528043576935145\\
347.01	0.00527897274969461\\
348.01	0.0052774800727138\\
349.01	0.00527595720013088\\
350.01	0.0052744035882121\\
351.01	0.00527281868804799\\
352.01	0.00527120194586318\\
353.01	0.00526955280336065\\
354.01	0.00526787069810039\\
355.01	0.00526615506391757\\
356.01	0.00526440533137949\\
357.01	0.0052626209282854\\
358.01	0.00526080128021128\\
359.01	0.00525894581110196\\
360.01	0.00525705394391194\\
361.01	0.00525512510129937\\
362.01	0.00525315870637325\\
363.01	0.0052511541834974\\
364.01	0.00524911095915187\\
365.01	0.00524702846285501\\
366.01	0.00524490612814607\\
367.01	0.00524274339362976\\
368.01	0.00524053970408339\\
369.01	0.00523829451162582\\
370.01	0.00523600727694703\\
371.01	0.00523367747059659\\
372.01	0.00523130457432675\\
373.01	0.00522888808248597\\
374.01	0.00522642750345643\\
375.01	0.0052239223611262\\
376.01	0.00522137219638459\\
377.01	0.00521877656862848\\
378.01	0.00521613505726135\\
379.01	0.00521344726316258\\
380.01	0.00521071281010516\\
381.01	0.0052079313460878\\
382.01	0.00520510254454846\\
383.01	0.00520222610541506\\
384.01	0.00519930175594483\\
385.01	0.00519632925129438\\
386.01	0.00519330837475483\\
387.01	0.00519023893757661\\
388.01	0.00518712077829755\\
389.01	0.00518395376147891\\
390.01	0.00518073777574154\\
391.01	0.00517747273098405\\
392.01	0.00517415855465568\\
393.01	0.00517079518694496\\
394.01	0.00516738257474272\\
395.01	0.00516392066422929\\
396.01	0.00516040939194155\\
397.01	0.00515684867418499\\
398.01	0.00515323839467018\\
399.01	0.00514957839029082\\
400.01	0.00514586843500483\\
401.01	0.0051421082218543\\
402.01	0.00513829734325826\\
403.01	0.0051344352698442\\
404.01	0.00513052132825987\\
405.01	0.00512655467862695\\
406.01	0.00512253429258049\\
407.01	0.00511845893316631\\
408.01	0.00511432713827346\\
409.01	0.00511013720971888\\
410.01	0.00510588721058433\\
411.01	0.00510157497386776\\
412.01	0.00509719812589269\\
413.01	0.00509275412807227\\
414.01	0.00508824034035754\\
415.01	0.00508365410866688\\
416.01	0.00507899287629741\\
417.01	0.00507425431498538\\
418.01	0.00506943646375203\\
419.01	0.00506453785125295\\
420.01	0.00505955753787001\\
421.01	0.00505449487564457\\
422.01	0.0050493492743097\\
423.01	0.00504412018361756\\
424.01	0.00503880709561506\\
425.01	0.00503340954685702\\
426.01	0.00502792712052649\\
427.01	0.00502235944842922\\
428.01	0.00501670621282421\\
429.01	0.00501096714804755\\
430.01	0.00500514204188019\\
431.01	0.00499923073660854\\
432.01	0.00499323312971387\\
433.01	0.00498714917412894\\
434.01	0.00498097887798744\\
435.01	0.00497472230378842\\
436.01	0.00496837956689141\\
437.01	0.00496195083325152\\
438.01	0.00495543631629968\\
439.01	0.0049488362728684\\
440.01	0.00494215099806004\\
441.01	0.00493538081895403\\
442.01	0.00492852608705271\\
443.01	0.0049215871693661\\
444.01	0.00491456443804755\\
445.01	0.0049074582585057\\
446.01	0.00490026897593585\\
447.01	0.00489299690024187\\
448.01	0.00488564228935582\\
449.01	0.00487820533100495\\
450.01	0.00487068612303634\\
451.01	0.00486308465247448\\
452.01	0.00485540077357311\\
453.01	0.00484763418522\\
454.01	0.00483978440816255\\
455.01	0.00483185076265124\\
456.01	0.00482383234723436\\
457.01	0.00481572801957892\\
458.01	0.00480753638033996\\
459.01	0.00479925576123245\\
460.01	0.00479088421856735\\
461.01	0.00478241953357406\\
462.01	0.00477385922082033\\
463.01	0.0047652005459163\\
464.01	0.00475644055341732\\
465.01	0.00474757610536369\\
466.01	0.00473860393016235\\
467.01	0.00472952068047446\\
468.01	0.00472032299737828\\
469.01	0.0047110075763275\\
470.01	0.00470157122838082\\
471.01	0.00469201092801985\\
472.01	0.00468232383700174\\
473.01	0.0046725072928645\\
474.01	0.00466255875221294\\
475.01	0.00465247568499771\\
476.01	0.00464225543447259\\
477.01	0.00463189511510558\\
478.01	0.0046213915849398\\
479.01	0.00461074143567458\\
480.01	0.00459994098549038\\
481.01	0.00458898627489505\\
482.01	0.00457787306603472\\
483.01	0.00456659684584823\\
484.01	0.00455515283335924\\
485.01	0.00454353599126477\\
486.01	0.00453174104179636\\
487.01	0.00451976248659717\\
488.01	0.0045075946300765\\
489.01	0.00449523160536918\\
490.01	0.00448266740167535\\
491.01	0.00446989589138234\\
492.01	0.00445691085505024\\
493.01	0.00444370600210405\\
494.01	0.00443027498501948\\
495.01	0.00441661140500323\\
496.01	0.00440270880777203\\
497.01	0.00438856066912667\\
498.01	0.00437416037167617\\
499.01	0.00435950117623876\\
500.01	0.0043445761938544\\
501.01	0.00432937836627371\\
502.01	0.00431390046205232\\
503.01	0.00429813508615042\\
504.01	0.00428207469322905\\
505.01	0.00426571160085169\\
506.01	0.00424903800183912\\
507.01	0.0042320459751404\\
508.01	0.00421472749460594\\
509.01	0.00419707443513539\\
510.01	0.00417907857582396\\
511.01	0.0041607315999603\\
512.01	0.0041420250920226\\
513.01	0.00412295053216629\\
514.01	0.00410349928904926\\
515.01	0.00408366261213326\\
516.01	0.00406343162474818\\
517.01	0.00404279731908349\\
518.01	0.00402175055378084\\
519.01	0.00400028205391089\\
520.01	0.00397838241203012\\
521.01	0.00395604208873859\\
522.01	0.00393325141211583\\
523.01	0.00391000057605093\\
524.01	0.00388627963762727\\
525.01	0.00386207851380362\\
526.01	0.00383738697769382\\
527.01	0.00381219465478078\\
528.01	0.00378649101939319\\
529.01	0.00376026539170914\\
530.01	0.00373350693544559\\
531.01	0.00370620465624647\\
532.01	0.00367834740064358\\
533.01	0.00364992385538304\\
534.01	0.00362092254694498\\
535.01	0.00359133184124472\\
536.01	0.00356113994363432\\
537.01	0.00353033489935747\\
538.01	0.00349890459460845\\
539.01	0.00346683675832488\\
540.01	0.00343411896482426\\
541.01	0.00340073863736593\\
542.01	0.00336668305270029\\
543.01	0.0033319393466588\\
544.01	0.00329649452084877\\
545.01	0.00326033545053969\\
546.01	0.00322344889386798\\
547.01	0.00318582150251138\\
548.01	0.00314743983399592\\
549.01	0.00310829036580649\\
550.01	0.00306835951147481\\
551.01	0.0030276336388282\\
552.01	0.00298609909059895\\
553.01	0.00294374220760762\\
554.01	0.00290054935475943\\
555.01	0.0028565069501179\\
556.01	0.00281160149734305\\
557.01	0.0027658196218078\\
558.01	0.0027191481107284\\
559.01	0.00267157395766957\\
560.01	0.00262308441180904\\
561.01	0.00257366703237543\\
562.01	0.00252330974869932\\
563.01	0.00247200092634584\\
564.01	0.00241972943982337\\
565.01	0.00236648475238619\\
566.01	0.00231225700346725\\
567.01	0.00225703710429318\\
568.01	0.00220081684223704\\
569.01	0.00214358899446181\\
570.01	0.00208534745138918\\
571.01	0.00202608735049004\\
572.01	0.00196580522083402\\
573.01	0.00190449913874023\\
574.01	0.0018421688947402\\
575.01	0.00177881617187809\\
576.01	0.00171444473512429\\
577.01	0.00164906063134602\\
578.01	0.00158267239884402\\
579.01	0.00151529128489964\\
580.01	0.00144693146904958\\
581.01	0.00137761028887722\\
582.01	0.00130734846393035\\
583.01	0.00123617031188133\\
584.01	0.00116410394916785\\
585.01	0.00109118146599105\\
586.01	0.00101743906259415\\
587.01	0.000942917130054623\\
588.01	0.00086766025422433\\
589.01	0.000791717115732183\\
590.01	0.000715140251857945\\
591.01	0.000637985637273675\\
592.01	0.00056031202973124\\
593.01	0.000482180013262905\\
594.01	0.000403650654755804\\
595.01	0.00032478366912282\\
596.01	0.000245634962812777\\
597.01	0.000166444281554918\\
598.01	9.17366970063782e-05\\
599.01	2.94669364271135e-05\\
599.02	2.89574269585688e-05\\
599.03	2.84509530852819e-05\\
599.04	2.79475443462525e-05\\
599.05	2.7447230571279e-05\\
599.06	2.69500418838293e-05\\
599.07	2.64560087039432e-05\\
599.08	2.59651617511691e-05\\
599.09	2.54775320475235e-05\\
599.1	2.49931509204854e-05\\
599.11	2.45120500060175e-05\\
599.12	2.40342612516167e-05\\
599.13	2.35598169193996e-05\\
599.14	2.30887495892042e-05\\
599.15	2.26210921617422e-05\\
599.16	2.215687786177e-05\\
599.17	2.16961402412976e-05\\
599.18	2.12389131828191e-05\\
599.19	2.07852309025824e-05\\
599.2	2.03351279538938e-05\\
599.21	1.98886392304542e-05\\
599.22	1.94457999697188e-05\\
599.23	1.90066457563063e-05\\
599.24	1.85712125254194e-05\\
599.25	1.81395365663334e-05\\
599.26	1.77116564095883e-05\\
599.27	1.72876122135658e-05\\
599.28	1.68674445361946e-05\\
599.29	1.64511943388859e-05\\
599.3	1.60389029905273e-05\\
599.31	1.56306122715104e-05\\
599.32	1.5226364377792e-05\\
599.33	1.48262019250105e-05\\
599.34	1.44301679526285e-05\\
599.35	1.40383059281154e-05\\
599.36	1.36506597511916e-05\\
599.37	1.32672737580847e-05\\
599.38	1.28881927258552e-05\\
599.39	1.25134618767404e-05\\
599.4	1.21431268825696e-05\\
599.41	1.17772338691924e-05\\
599.42	1.14158294209736e-05\\
599.43	1.10589605853174e-05\\
599.44	1.07066748772523e-05\\
599.45	1.03590202840433e-05\\
599.46	1.00160452698606e-05\\
599.47	9.67779878049101e-06\\
599.48	9.34433024810284e-06\\
599.49	9.01568959604283e-06\\
599.5	8.69192724369319e-06\\
599.51	8.37309411137396e-06\\
599.52	8.05924162529219e-06\\
599.53	7.75042172253965e-06\\
599.54	7.4466868561357e-06\\
599.55	7.14809000013084e-06\\
599.56	6.85468465475535e-06\\
599.57	6.56652485161308e-06\\
599.58	6.28366515892896e-06\\
599.59	6.00616068686249e-06\\
599.6	5.7340670928472e-06\\
599.61	5.46744058700296e-06\\
599.62	5.20633793760217e-06\\
599.63	4.95081647657741e-06\\
599.64	4.70093410508653e-06\\
599.65	4.45674929914347e-06\\
599.66	4.21832111529262e-06\\
599.67	3.98570919634203e-06\\
599.68	3.75897377716435e-06\\
599.69	3.53817569053415e-06\\
599.7	3.32337637303295e-06\\
599.71	3.11463787103054e-06\\
599.72	2.91202284668363e-06\\
599.73	2.71559458404382e-06\\
599.74	2.52541699518292e-06\\
599.75	2.34155462640329e-06\\
599.76	2.16407266449489e-06\\
599.77	1.99303694307928e-06\\
599.78	1.82851394897598e-06\\
599.79	1.67057082867302e-06\\
599.8	1.51927539483211e-06\\
599.81	1.37469613287027e-06\\
599.82	1.23690220760718e-06\\
599.83	1.10596346997172e-06\\
599.84	9.81950463782924e-07\\
599.85	8.64934432591793e-07\\
599.86	7.54987326588921e-07\\
599.87	6.5218180958504e-07\\
599.88	5.56591266064402e-07\\
599.89	4.68289808295413e-07\\
599.9	3.87352283521061e-07\\
599.91	3.13854281218731e-07\\
599.92	2.4787214042421e-07\\
599.93	1.8948295714763e-07\\
599.94	1.38764591834512e-07\\
599.95	9.57956769204876e-08\\
599.96	6.06556244606149e-08\\
599.97	3.34246338194039e-08\\
599.98	1.41836994527883e-08\\
599.99	3.01461875948372e-09\\
600	0\\
};
\addplot [color=black!20!mycolor21,solid,forget plot]
  table[row sep=crcr]{%
0.01	0.00523138383269897\\
1.01	0.005231382805886\\
2.01	0.00523138175751092\\
3.01	0.0052313806871207\\
4.01	0.00523137959425308\\
5.01	0.00523137847843602\\
6.01	0.0052313773391874\\
7.01	0.0052313761760151\\
8.01	0.00523137498841659\\
9.01	0.00523137377587905\\
10.01	0.00523137253787854\\
11.01	0.00523137127388031\\
12.01	0.00523136998333838\\
13.01	0.00523136866569534\\
14.01	0.00523136732038186\\
15.01	0.00523136594681675\\
16.01	0.00523136454440681\\
17.01	0.00523136311254593\\
18.01	0.00523136165061591\\
19.01	0.00523136015798483\\
20.01	0.00523135863400801\\
21.01	0.0052313570780272\\
22.01	0.00523135548937016\\
23.01	0.00523135386735057\\
24.01	0.00523135221126765\\
25.01	0.00523135052040608\\
26.01	0.00523134879403519\\
27.01	0.00523134703140926\\
28.01	0.00523134523176695\\
29.01	0.00523134339433056\\
30.01	0.00523134151830637\\
31.01	0.00523133960288403\\
32.01	0.00523133764723592\\
33.01	0.00523133565051707\\
34.01	0.00523133361186495\\
35.01	0.00523133153039868\\
36.01	0.00523132940521904\\
37.01	0.00523132723540796\\
38.01	0.00523132502002802\\
39.01	0.005231322758122\\
40.01	0.00523132044871281\\
41.01	0.0052313180908025\\
42.01	0.00523131568337245\\
43.01	0.0052313132253825\\
44.01	0.00523131071577083\\
45.01	0.00523130815345305\\
46.01	0.00523130553732208\\
47.01	0.0052313028662477\\
48.01	0.00523130013907582\\
49.01	0.00523129735462829\\
50.01	0.00523129451170188\\
51.01	0.0052312916090685\\
52.01	0.00523128864547392\\
53.01	0.00523128561963773\\
54.01	0.00523128253025265\\
55.01	0.00523127937598381\\
56.01	0.00523127615546846\\
57.01	0.00523127286731504\\
58.01	0.0052312695101031\\
59.01	0.00523126608238191\\
60.01	0.00523126258267048\\
61.01	0.00523125900945685\\
62.01	0.00523125536119704\\
63.01	0.00523125163631499\\
64.01	0.00523124783320103\\
65.01	0.00523124395021218\\
66.01	0.00523123998567065\\
67.01	0.00523123593786356\\
68.01	0.00523123180504193\\
69.01	0.00523122758542023\\
70.01	0.0052312232771752\\
71.01	0.00523121887844556\\
72.01	0.00523121438733066\\
73.01	0.00523120980189001\\
74.01	0.00523120512014266\\
75.01	0.00523120034006559\\
76.01	0.00523119545959362\\
77.01	0.00523119047661807\\
78.01	0.00523118538898599\\
79.01	0.00523118019449925\\
80.01	0.00523117489091362\\
81.01	0.0052311694759376\\
82.01	0.00523116394723152\\
83.01	0.0052311583024069\\
84.01	0.00523115253902459\\
85.01	0.00523114665459489\\
86.01	0.00523114064657517\\
87.01	0.00523113451236992\\
88.01	0.00523112824932879\\
89.01	0.00523112185474595\\
90.01	0.00523111532585859\\
91.01	0.00523110865984609\\
92.01	0.00523110185382852\\
93.01	0.00523109490486543\\
94.01	0.00523108780995487\\
95.01	0.00523108056603126\\
96.01	0.00523107316996528\\
97.01	0.00523106561856186\\
98.01	0.0052310579085586\\
99.01	0.00523105003662475\\
100.01	0.00523104199935942\\
101.01	0.00523103379329062\\
102.01	0.00523102541487337\\
103.01	0.00523101686048825\\
104.01	0.00523100812643983\\
105.01	0.00523099920895525\\
106.01	0.00523099010418216\\
107.01	0.00523098080818766\\
108.01	0.00523097131695629\\
109.01	0.00523096162638815\\
110.01	0.00523095173229713\\
111.01	0.00523094163040974\\
112.01	0.00523093131636236\\
113.01	0.00523092078570006\\
114.01	0.0052309100338744\\
115.01	0.00523089905624127\\
116.01	0.00523088784805948\\
117.01	0.00523087640448797\\
118.01	0.00523086472058439\\
119.01	0.00523085279130236\\
120.01	0.00523084061149009\\
121.01	0.00523082817588724\\
122.01	0.00523081547912343\\
123.01	0.00523080251571542\\
124.01	0.00523078928006506\\
125.01	0.00523077576645655\\
126.01	0.00523076196905461\\
127.01	0.00523074788190097\\
128.01	0.00523073349891317\\
129.01	0.00523071881388052\\
130.01	0.00523070382046229\\
131.01	0.00523068851218465\\
132.01	0.00523067288243826\\
133.01	0.00523065692447512\\
134.01	0.0052306406314054\\
135.01	0.00523062399619537\\
136.01	0.00523060701166355\\
137.01	0.00523058967047768\\
138.01	0.00523057196515207\\
139.01	0.00523055388804402\\
140.01	0.00523053543135064\\
141.01	0.0052305165871056\\
142.01	0.00523049734717545\\
143.01	0.00523047770325624\\
144.01	0.00523045764687\\
145.01	0.00523043716936136\\
146.01	0.00523041626189342\\
147.01	0.00523039491544394\\
148.01	0.00523037312080204\\
149.01	0.00523035086856357\\
150.01	0.00523032814912757\\
151.01	0.00523030495269195\\
152.01	0.00523028126924912\\
153.01	0.00523025708858244\\
154.01	0.00523023240026072\\
155.01	0.00523020719363465\\
156.01	0.00523018145783191\\
157.01	0.00523015518175259\\
158.01	0.00523012835406437\\
159.01	0.00523010096319786\\
160.01	0.00523007299734137\\
161.01	0.00523004444443611\\
162.01	0.00523001529217109\\
163.01	0.00522998552797743\\
164.01	0.00522995513902352\\
165.01	0.00522992411220928\\
166.01	0.00522989243416079\\
167.01	0.00522986009122422\\
168.01	0.00522982706946012\\
169.01	0.00522979335463813\\
170.01	0.00522975893222984\\
171.01	0.00522972378740324\\
172.01	0.00522968790501672\\
173.01	0.00522965126961176\\
174.01	0.00522961386540708\\
175.01	0.00522957567629173\\
176.01	0.00522953668581796\\
177.01	0.00522949687719485\\
178.01	0.00522945623328061\\
179.01	0.0052294147365757\\
180.01	0.00522937236921498\\
181.01	0.00522932911296079\\
182.01	0.00522928494919445\\
183.01	0.00522923985890923\\
184.01	0.00522919382270152\\
185.01	0.00522914682076305\\
186.01	0.00522909883287243\\
187.01	0.00522904983838685\\
188.01	0.00522899981623294\\
189.01	0.00522894874489822\\
190.01	0.00522889660242194\\
191.01	0.00522884336638555\\
192.01	0.0052287890139039\\
193.01	0.00522873352161505\\
194.01	0.00522867686567052\\
195.01	0.00522861902172563\\
196.01	0.00522855996492882\\
197.01	0.00522849966991163\\
198.01	0.00522843811077804\\
199.01	0.00522837526109304\\
200.01	0.00522831109387224\\
201.01	0.00522824558157061\\
202.01	0.00522817869607035\\
203.01	0.0052281104086697\\
204.01	0.00522804069007087\\
205.01	0.00522796951036762\\
206.01	0.00522789683903318\\
207.01	0.00522782264490732\\
208.01	0.00522774689618307\\
209.01	0.00522766956039422\\
210.01	0.00522759060440158\\
211.01	0.00522750999437894\\
212.01	0.00522742769579952\\
213.01	0.00522734367342148\\
214.01	0.00522725789127358\\
215.01	0.00522717031263984\\
216.01	0.00522708090004511\\
217.01	0.00522698961523909\\
218.01	0.00522689641918104\\
219.01	0.00522680127202353\\
220.01	0.0052267041330961\\
221.01	0.005226604960889\\
222.01	0.00522650371303547\\
223.01	0.00522640034629519\\
224.01	0.00522629481653646\\
225.01	0.00522618707871809\\
226.01	0.00522607708687117\\
227.01	0.00522596479408068\\
228.01	0.00522585015246619\\
229.01	0.00522573311316276\\
230.01	0.00522561362630088\\
231.01	0.00522549164098697\\
232.01	0.00522536710528238\\
233.01	0.00522523996618295\\
234.01	0.0052251101695975\\
235.01	0.00522497766032645\\
236.01	0.00522484238203957\\
237.01	0.0052247042772541\\
238.01	0.00522456328731119\\
239.01	0.00522441935235325\\
240.01	0.00522427241130017\\
241.01	0.00522412240182509\\
242.01	0.00522396926033021\\
243.01	0.0052238129219215\\
244.01	0.00522365332038351\\
245.01	0.00522349038815377\\
246.01	0.00522332405629596\\
247.01	0.00522315425447371\\
248.01	0.00522298091092315\\
249.01	0.0052228039524253\\
250.01	0.00522262330427802\\
251.01	0.00522243889026721\\
252.01	0.00522225063263816\\
253.01	0.00522205845206537\\
254.01	0.00522186226762313\\
255.01	0.00522166199675438\\
256.01	0.00522145755524\\
257.01	0.00522124885716726\\
258.01	0.00522103581489799\\
259.01	0.00522081833903531\\
260.01	0.00522059633839147\\
261.01	0.00522036971995348\\
262.01	0.00522013838884953\\
263.01	0.00521990224831446\\
264.01	0.00521966119965464\\
265.01	0.00521941514221146\\
266.01	0.00521916397332687\\
267.01	0.00521890758830544\\
268.01	0.0052186458803775\\
269.01	0.00521837874066163\\
270.01	0.00521810605812694\\
271.01	0.00521782771955339\\
272.01	0.00521754360949365\\
273.01	0.00521725361023293\\
274.01	0.00521695760174888\\
275.01	0.00521665546167114\\
276.01	0.00521634706524061\\
277.01	0.00521603228526704\\
278.01	0.00521571099208779\\
279.01	0.00521538305352528\\
280.01	0.00521504833484423\\
281.01	0.00521470669870827\\
282.01	0.00521435800513644\\
283.01	0.00521400211145921\\
284.01	0.00521363887227421\\
285.01	0.00521326813940134\\
286.01	0.00521288976183818\\
287.01	0.00521250358571453\\
288.01	0.00521210945424713\\
289.01	0.0052117072076936\\
290.01	0.00521129668330694\\
291.01	0.00521087771528946\\
292.01	0.0052104501347468\\
293.01	0.00521001376964111\\
294.01	0.00520956844474569\\
295.01	0.0052091139815982\\
296.01	0.00520865019845492\\
297.01	0.00520817691024477\\
298.01	0.00520769392852357\\
299.01	0.0052072010614285\\
300.01	0.00520669811363309\\
301.01	0.00520618488630217\\
302.01	0.00520566117704823\\
303.01	0.00520512677988722\\
304.01	0.00520458148519613\\
305.01	0.0052040250796701\\
306.01	0.0052034573462818\\
307.01	0.00520287806424025\\
308.01	0.00520228700895233\\
309.01	0.00520168395198414\\
310.01	0.00520106866102452\\
311.01	0.00520044089984959\\
312.01	0.00519980042828956\\
313.01	0.00519914700219645\\
314.01	0.00519848037341427\\
315.01	0.00519780028975126\\
316.01	0.00519710649495468\\
317.01	0.00519639872868714\\
318.01	0.0051956767265069\\
319.01	0.00519494021985052\\
320.01	0.00519418893601805\\
321.01	0.00519342259816257\\
322.01	0.00519264092528269\\
323.01	0.00519184363221911\\
324.01	0.00519103042965536\\
325.01	0.00519020102412334\\
326.01	0.00518935511801243\\
327.01	0.00518849240958488\\
328.01	0.00518761259299581\\
329.01	0.00518671535831935\\
330.01	0.00518580039158017\\
331.01	0.00518486737479209\\
332.01	0.00518391598600277\\
333.01	0.00518294589934636\\
334.01	0.0051819567851026\\
335.01	0.00518094830976426\\
336.01	0.00517992013611334\\
337.01	0.00517887192330526\\
338.01	0.00517780332696261\\
339.01	0.00517671399927806\\
340.01	0.00517560358912744\\
341.01	0.00517447174219251\\
342.01	0.00517331810109557\\
343.01	0.00517214230554414\\
344.01	0.00517094399248761\\
345.01	0.00516972279628462\\
346.01	0.005168478348884\\
347.01	0.00516721028001698\\
348.01	0.00516591821740183\\
349.01	0.00516460178696177\\
350.01	0.0051632606130549\\
351.01	0.00516189431871695\\
352.01	0.00516050252591668\\
353.01	0.00515908485582376\\
354.01	0.00515764092908807\\
355.01	0.00515617036613068\\
356.01	0.00515467278744542\\
357.01	0.0051531478139106\\
358.01	0.00515159506710816\\
359.01	0.00515001416965157\\
360.01	0.00514840474551805\\
361.01	0.00514676642038459\\
362.01	0.00514509882196492\\
363.01	0.00514340158034452\\
364.01	0.00514167432831077\\
365.01	0.00513991670167382\\
366.01	0.00513812833957383\\
367.01	0.00513630888477042\\
368.01	0.00513445798390796\\
369.01	0.00513257528774949\\
370.01	0.00513066045137389\\
371.01	0.00512871313432686\\
372.01	0.00512673300071721\\
373.01	0.00512471971924839\\
374.01	0.00512267296317452\\
375.01	0.00512059241016839\\
376.01	0.00511847774208999\\
377.01	0.00511632864463985\\
378.01	0.00511414480688257\\
379.01	0.00511192592062577\\
380.01	0.00510967167963675\\
381.01	0.00510738177867964\\
382.01	0.00510505591235568\\
383.01	0.0051026937737284\\
384.01	0.00510029505271672\\
385.01	0.00509785943423936\\
386.01	0.00509538659609526\\
387.01	0.00509287620656777\\
388.01	0.00509032792174346\\
389.01	0.00508774138254064\\
390.01	0.00508511621144959\\
391.01	0.00508245200899375\\
392.01	0.00507974834993066\\
393.01	0.00507700477922407\\
394.01	0.00507422080783227\\
395.01	0.00507139590837533\\
396.01	0.00506852951076469\\
397.01	0.00506562099789917\\
398.01	0.0050626697015592\\
399.01	0.00505967489865909\\
400.01	0.00505663580804531\\
401.01	0.00505355158806207\\
402.01	0.00505042133513194\\
403.01	0.00504724408362658\\
404.01	0.0050440188073223\\
405.01	0.0050407444227402\\
406.01	0.00503741979466024\\
407.01	0.00503404374406445\\
408.01	0.0050306150586897\\
409.01	0.00502713250625813\\
410.01	0.00502359485027725\\
411.01	0.00502000086806385\\
412.01	0.00501634937032668\\
413.01	0.00501263922124967\\
414.01	0.00500886935755939\\
415.01	0.00500503880457567\\
416.01	0.00500114668680975\\
417.01	0.00499719223044989\\
418.01	0.00499317475531407\\
419.01	0.004989093654993\\
420.01	0.00498494836696045\\
421.01	0.00498073834559584\\
422.01	0.00497646305411321\\
423.01	0.00497212196381056\\
424.01	0.00496771455341656\\
425.01	0.00496324030825039\\
426.01	0.00495869871917461\\
427.01	0.00495408928132304\\
428.01	0.00494941149258343\\
429.01	0.00494466485181496\\
430.01	0.00493984885678136\\
431.01	0.00493496300177866\\
432.01	0.00493000677494162\\
433.01	0.00492497965520942\\
434.01	0.00491988110893609\\
435.01	0.004914710586133\\
436.01	0.00490946751633421\\
437.01	0.00490415130407918\\
438.01	0.0048987613240141\\
439.01	0.00489329691561693\\
440.01	0.00488775737756222\\
441.01	0.00488214196174756\\
442.01	0.00487644986701617\\
443.01	0.00487068023262009\\
444.01	0.00486483213148382\\
445.01	0.00485890456333915\\
446.01	0.00485289644782183\\
447.01	0.0048468066176342\\
448.01	0.00484063381189665\\
449.01	0.00483437666982691\\
450.01	0.00482803372490231\\
451.01	0.00482160339967746\\
452.01	0.00481508400143666\\
453.01	0.00480847371887101\\
454.01	0.00480177061996946\\
455.01	0.00479497265130526\\
456.01	0.00478807763887849\\
457.01	0.00478108329064709\\
458.01	0.0047739872008241\\
459.01	0.00476678685595524\\
460.01	0.00475947964270048\\
461.01	0.00475206285713371\\
462.01	0.00474453371524032\\
463.01	0.00473688936414579\\
464.01	0.00472912689344517\\
465.01	0.00472124334584061\\
466.01	0.00471323572615556\\
467.01	0.0047051010076903\\
468.01	0.00469683613486702\\
469.01	0.00468843802121081\\
470.01	0.00467990354198637\\
471.01	0.00467122952129828\\
472.01	0.00466241271420965\\
473.01	0.00465344978541999\\
474.01	0.00464433728719487\\
475.01	0.00463507164027726\\
476.01	0.00462564912182952\\
477.01	0.00461606586177708\\
478.01	0.00460631784347362\\
479.01	0.00459640090568819\\
480.01	0.00458631074558279\\
481.01	0.00457604292263263\\
482.01	0.0045655928633934\\
483.01	0.00455495586698143\\
484.01	0.00454412711108051\\
485.01	0.00453310165824177\\
486.01	0.0045218744621983\\
487.01	0.00451044037388256\\
488.01	0.00449879414680735\\
489.01	0.00448693044147396\\
490.01	0.00447484382848845\\
491.01	0.00446252879012912\\
492.01	0.00444997972019801\\
493.01	0.00443719092212448\\
494.01	0.00442415660545863\\
495.01	0.00441087088108571\\
496.01	0.00439732775569151\\
497.01	0.00438352112617594\\
498.01	0.00436944477479478\\
499.01	0.00435509236575418\\
500.01	0.00434045744372663\\
501.01	0.00432553343426664\\
502.01	0.00431031364545572\\
503.01	0.00429479126971478\\
504.01	0.00427895938512484\\
505.01	0.00426281095607702\\
506.01	0.0042463388331861\\
507.01	0.00422953575244153\\
508.01	0.0042123943336166\\
509.01	0.00419490707800405\\
510.01	0.00417706636559825\\
511.01	0.00415886445188262\\
512.01	0.00414029346441456\\
513.01	0.00412134539940507\\
514.01	0.00410201211847112\\
515.01	0.0040822853456925\\
516.01	0.00406215666502128\\
517.01	0.00404161751800122\\
518.01	0.00402065920166672\\
519.01	0.00399927286644217\\
520.01	0.00397744951389662\\
521.01	0.00395517999431117\\
522.01	0.00393245500410627\\
523.01	0.00390926508320447\\
524.01	0.00388560061240984\\
525.01	0.00386145181088166\\
526.01	0.00383680873377442\\
527.01	0.0038116612700997\\
528.01	0.00378599914084991\\
529.01	0.00375981189740243\\
530.01	0.00373308892021277\\
531.01	0.00370581941779547\\
532.01	0.00367799242599846\\
533.01	0.00364959680759006\\
534.01	0.00362062125220449\\
535.01	0.00359105427670131\\
536.01	0.0035608842260056\\
537.01	0.00353009927449551\\
538.01	0.00349868742799898\\
539.01	0.00346663652646789\\
540.01	0.00343393424739245\\
541.01	0.00340056811002963\\
542.01	0.00336652548052344\\
543.01	0.00333179357800268\\
544.01	0.00329635948175882\\
545.01	0.00326021013961405\\
546.01	0.00322333237760449\\
547.01	0.00318571291111079\\
548.01	0.0031473383575855\\
549.01	0.00310819525103227\\
550.01	0.00306827005840969\\
551.01	0.00302754919814918\\
552.01	0.00298601906098984\\
553.01	0.00294366603335671\\
554.01	0.0029004765235269\\
555.01	0.00285643699084808\\
556.01	0.0028115339783004\\
557.01	0.00276575414871214\\
558.01	0.00271908432496719\\
559.01	0.00267151153456668\\
560.01	0.00262302305893529\\
561.01	0.00257360648788791\\
562.01	0.00252324977970384\\
563.01	0.00247194132727781\\
564.01	0.00241967003084647\\
565.01	0.00236642537780996\\
566.01	0.00231219753018783\\
567.01	0.0022569774202631\\
568.01	0.00220075685497212\\
569.01	0.00214352862959637\\
570.01	0.00208528665129069\\
571.01	0.00202602607294628\\
572.01	0.00196574343782424\\
573.01	0.00190443683530311\\
574.01	0.00184210606794752\\
575.01	0.00177875282992286\\
576.01	0.00171438089652806\\
577.01	0.00164899632428734\\
578.01	0.00158260766060568\\
579.01	0.00151522616142658\\
580.01	0.00144686601460507\\
581.01	0.00137754456577855\\
582.01	0.00130728254233795\\
583.01	0.00123610426960943\\
584.01	0.00116403787147582\\
585.01	0.00109111544530627\\
586.01	0.00101737319810725\\
587.01	0.00094285152711592\\
588.01	0.000867595023456734\\
589.01	0.000791652371757953\\
590.01	0.000715076111516234\\
591.01	0.000637922217178835\\
592.01	0.000560249442989076\\
593.01	0.000482118365120262\\
594.01	0.00040359003690792\\
595.01	0.000324724152338418\\
596.01	0.000245576587458076\\
597.01	0.000166416924230011\\
598.01	9.17366970063765e-05\\
599.01	2.94669364271135e-05\\
599.02	2.89574269585705e-05\\
599.03	2.84509530852801e-05\\
599.04	2.79475443462508e-05\\
599.05	2.7447230571279e-05\\
599.06	2.69500418838293e-05\\
599.07	2.64560087039449e-05\\
599.08	2.59651617511691e-05\\
599.09	2.54775320475218e-05\\
599.1	2.49931509204836e-05\\
599.11	2.45120500060175e-05\\
599.12	2.40342612516185e-05\\
599.13	2.35598169193978e-05\\
599.14	2.30887495892007e-05\\
599.15	2.26210921617405e-05\\
599.16	2.215687786177e-05\\
599.17	2.16961402412993e-05\\
599.18	2.12389131828174e-05\\
599.19	2.07852309025806e-05\\
599.2	2.03351279538938e-05\\
599.21	1.98886392304524e-05\\
599.22	1.94457999697188e-05\\
599.23	1.90066457563046e-05\\
599.24	1.85712125254211e-05\\
599.25	1.81395365663334e-05\\
599.26	1.77116564095865e-05\\
599.27	1.72876122135658e-05\\
599.28	1.68674445361946e-05\\
599.29	1.64511943388859e-05\\
599.3	1.60389029905273e-05\\
599.31	1.56306122715087e-05\\
599.32	1.52263643777902e-05\\
599.33	1.48262019250105e-05\\
599.34	1.44301679526268e-05\\
599.35	1.40383059281154e-05\\
599.36	1.36506597511916e-05\\
599.37	1.32672737580847e-05\\
599.38	1.28881927258535e-05\\
599.39	1.25134618767404e-05\\
599.4	1.21431268825696e-05\\
599.41	1.17772338691906e-05\\
599.42	1.14158294209719e-05\\
599.43	1.10589605853174e-05\\
599.44	1.07066748772523e-05\\
599.45	1.03590202840433e-05\\
599.46	1.00160452698589e-05\\
599.47	9.67779878049101e-06\\
599.48	9.34433024810284e-06\\
599.49	9.0156895960411e-06\\
599.5	8.69192724369319e-06\\
599.51	8.37309411137396e-06\\
599.52	8.05924162529392e-06\\
599.53	7.75042172253965e-06\\
599.54	7.44668685613396e-06\\
599.55	7.14809000013084e-06\\
599.56	6.85468465475535e-06\\
599.57	6.56652485161134e-06\\
599.58	6.2836651589307e-06\\
599.59	6.00616068686249e-06\\
599.6	5.73406709284546e-06\\
599.61	5.46744058700123e-06\\
599.62	5.2063379376039e-06\\
599.63	4.95081647657741e-06\\
599.64	4.70093410508653e-06\\
599.65	4.45674929914174e-06\\
599.66	4.21832111529089e-06\\
599.67	3.98570919634376e-06\\
599.68	3.75897377716608e-06\\
599.69	3.53817569053241e-06\\
599.7	3.32337637303295e-06\\
599.71	3.11463787102881e-06\\
599.72	2.91202284668363e-06\\
599.73	2.71559458404555e-06\\
599.74	2.52541699518466e-06\\
599.75	2.34155462640155e-06\\
599.76	2.16407266449663e-06\\
599.77	1.99303694307755e-06\\
599.78	1.82851394897772e-06\\
599.79	1.67057082867475e-06\\
599.8	1.51927539483211e-06\\
599.81	1.37469613287027e-06\\
599.82	1.23690220760544e-06\\
599.83	1.10596346997172e-06\\
599.84	9.81950463784659e-07\\
599.85	8.64934432591793e-07\\
599.86	7.54987326587186e-07\\
599.87	6.52181809583305e-07\\
599.88	5.56591266064402e-07\\
599.89	4.68289808293679e-07\\
599.9	3.87352283521061e-07\\
599.91	3.13854281216996e-07\\
599.92	2.47872140425945e-07\\
599.93	1.89482957149364e-07\\
599.94	1.38764591834512e-07\\
599.95	9.57956769204876e-08\\
599.96	6.06556244606149e-08\\
599.97	3.34246338211386e-08\\
599.98	1.4183699454523e-08\\
599.99	3.014618757749e-09\\
600	0\\
};
\addplot [color=black!50!mycolor20,solid,forget plot]
  table[row sep=crcr]{%
0.01	0.00515885702990981\\
1.01	0.0051588560241\\
2.01	0.00515885499729988\\
3.01	0.00515885394907215\\
4.01	0.00515885287897005\\
5.01	0.00515885178653789\\
6.01	0.00515885067131022\\
7.01	0.00515884953281223\\
8.01	0.00515884837055896\\
9.01	0.00515884718405548\\
10.01	0.0051588459727966\\
11.01	0.00515884473626641\\
12.01	0.00515884347393856\\
13.01	0.00515884218527537\\
14.01	0.00515884086972846\\
15.01	0.00515883952673772\\
16.01	0.00515883815573141\\
17.01	0.0051588367561263\\
18.01	0.0051588353273264\\
19.01	0.00515883386872376\\
20.01	0.00515883237969774\\
21.01	0.00515883085961459\\
22.01	0.00515882930782761\\
23.01	0.00515882772367658\\
24.01	0.00515882610648755\\
25.01	0.00515882445557254\\
26.01	0.00515882277022942\\
27.01	0.00515882104974142\\
28.01	0.00515881929337658\\
29.01	0.00515881750038814\\
30.01	0.00515881567001362\\
31.01	0.00515881380147459\\
32.01	0.00515881189397682\\
33.01	0.00515880994670914\\
34.01	0.00515880795884389\\
35.01	0.00515880592953614\\
36.01	0.00515880385792326\\
37.01	0.0051588017431248\\
38.01	0.00515879958424202\\
39.01	0.00515879738035744\\
40.01	0.0051587951305346\\
41.01	0.00515879283381773\\
42.01	0.00515879048923105\\
43.01	0.00515878809577853\\
44.01	0.00515878565244348\\
45.01	0.00515878315818808\\
46.01	0.00515878061195318\\
47.01	0.00515877801265732\\
48.01	0.00515877535919682\\
49.01	0.00515877265044496\\
50.01	0.00515876988525172\\
51.01	0.00515876706244326\\
52.01	0.00515876418082143\\
53.01	0.00515876123916288\\
54.01	0.00515875823621926\\
55.01	0.00515875517071615\\
56.01	0.00515875204135281\\
57.01	0.00515874884680145\\
58.01	0.00515874558570655\\
59.01	0.00515874225668481\\
60.01	0.00515873885832414\\
61.01	0.00515873538918293\\
62.01	0.00515873184779\\
63.01	0.0051587282326435\\
64.01	0.00515872454221057\\
65.01	0.00515872077492647\\
66.01	0.00515871692919399\\
67.01	0.00515871300338292\\
68.01	0.00515870899582923\\
69.01	0.00515870490483436\\
70.01	0.00515870072866464\\
71.01	0.00515869646555036\\
72.01	0.0051586921136853\\
73.01	0.00515868767122566\\
74.01	0.0051586831362895\\
75.01	0.00515867850695594\\
76.01	0.00515867378126429\\
77.01	0.00515866895721325\\
78.01	0.00515866403276003\\
79.01	0.00515865900581964\\
80.01	0.00515865387426385\\
81.01	0.00515864863592042\\
82.01	0.00515864328857214\\
83.01	0.00515863782995577\\
84.01	0.0051586322577615\\
85.01	0.0051586265696312\\
86.01	0.00515862076315848\\
87.01	0.005158614835887\\
88.01	0.00515860878530952\\
89.01	0.00515860260886694\\
90.01	0.00515859630394725\\
91.01	0.00515858986788435\\
92.01	0.00515858329795716\\
93.01	0.00515857659138814\\
94.01	0.00515856974534228\\
95.01	0.00515856275692627\\
96.01	0.00515855562318678\\
97.01	0.00515854834110911\\
98.01	0.00515854090761676\\
99.01	0.00515853331956932\\
100.01	0.00515852557376181\\
101.01	0.00515851766692258\\
102.01	0.00515850959571263\\
103.01	0.00515850135672406\\
104.01	0.00515849294647825\\
105.01	0.005158484361425\\
106.01	0.00515847559794085\\
107.01	0.00515846665232715\\
108.01	0.00515845752080916\\
109.01	0.0051584481995342\\
110.01	0.00515843868457001\\
111.01	0.00515842897190299\\
112.01	0.00515841905743692\\
113.01	0.00515840893699111\\
114.01	0.00515839860629824\\
115.01	0.00515838806100324\\
116.01	0.00515837729666118\\
117.01	0.00515836630873546\\
118.01	0.0051583550925958\\
119.01	0.00515834364351641\\
120.01	0.00515833195667411\\
121.01	0.00515832002714636\\
122.01	0.00515830784990911\\
123.01	0.00515829541983468\\
124.01	0.00515828273168972\\
125.01	0.0051582697801332\\
126.01	0.00515825655971399\\
127.01	0.00515824306486858\\
128.01	0.00515822928991865\\
129.01	0.00515821522906947\\
130.01	0.00515820087640657\\
131.01	0.00515818622589406\\
132.01	0.00515817127137142\\
133.01	0.00515815600655153\\
134.01	0.00515814042501787\\
135.01	0.005158124520222\\
136.01	0.00515810828548049\\
137.01	0.0051580917139727\\
138.01	0.0051580747987376\\
139.01	0.00515805753267124\\
140.01	0.00515803990852314\\
141.01	0.00515802191889429\\
142.01	0.00515800355623319\\
143.01	0.00515798481283354\\
144.01	0.00515796568083053\\
145.01	0.00515794615219766\\
146.01	0.00515792621874401\\
147.01	0.00515790587211008\\
148.01	0.00515788510376496\\
149.01	0.00515786390500255\\
150.01	0.00515784226693803\\
151.01	0.00515782018050432\\
152.01	0.00515779763644852\\
153.01	0.00515777462532752\\
154.01	0.00515775113750518\\
155.01	0.00515772716314745\\
156.01	0.00515770269221862\\
157.01	0.00515767771447749\\
158.01	0.00515765221947307\\
159.01	0.00515762619654012\\
160.01	0.00515759963479528\\
161.01	0.00515757252313203\\
162.01	0.00515754485021663\\
163.01	0.00515751660448361\\
164.01	0.00515748777413071\\
165.01	0.00515745834711416\\
166.01	0.00515742831114403\\
167.01	0.00515739765367906\\
168.01	0.00515736636192184\\
169.01	0.00515733442281314\\
170.01	0.00515730182302728\\
171.01	0.00515726854896632\\
172.01	0.00515723458675454\\
173.01	0.00515719992223326\\
174.01	0.00515716454095461\\
175.01	0.00515712842817631\\
176.01	0.00515709156885521\\
177.01	0.00515705394764147\\
178.01	0.00515701554887255\\
179.01	0.0051569763565666\\
180.01	0.00515693635441651\\
181.01	0.00515689552578291\\
182.01	0.00515685385368791\\
183.01	0.0051568113208079\\
184.01	0.00515676790946716\\
185.01	0.00515672360163037\\
186.01	0.00515667837889592\\
187.01	0.00515663222248803\\
188.01	0.00515658511324966\\
189.01	0.00515653703163458\\
190.01	0.00515648795770014\\
191.01	0.0051564378710991\\
192.01	0.00515638675107132\\
193.01	0.005156334576436\\
194.01	0.00515628132558318\\
195.01	0.00515622697646506\\
196.01	0.00515617150658773\\
197.01	0.00515611489300189\\
198.01	0.00515605711229397\\
199.01	0.00515599814057718\\
200.01	0.00515593795348198\\
201.01	0.00515587652614653\\
202.01	0.00515581383320716\\
203.01	0.00515574984878829\\
204.01	0.00515568454649235\\
205.01	0.00515561789939006\\
206.01	0.00515554988000918\\
207.01	0.00515548046032457\\
208.01	0.00515540961174691\\
209.01	0.00515533730511193\\
210.01	0.00515526351066906\\
211.01	0.00515518819807012\\
212.01	0.00515511133635769\\
213.01	0.00515503289395306\\
214.01	0.00515495283864426\\
215.01	0.00515487113757402\\
216.01	0.00515478775722689\\
217.01	0.0051547026634168\\
218.01	0.00515461582127406\\
219.01	0.00515452719523198\\
220.01	0.00515443674901382\\
221.01	0.00515434444561874\\
222.01	0.00515425024730838\\
223.01	0.00515415411559234\\
224.01	0.00515405601121396\\
225.01	0.00515395589413587\\
226.01	0.00515385372352475\\
227.01	0.00515374945773651\\
228.01	0.00515364305430075\\
229.01	0.00515353446990535\\
230.01	0.00515342366038035\\
231.01	0.00515331058068166\\
232.01	0.00515319518487507\\
233.01	0.00515307742611925\\
234.01	0.00515295725664875\\
235.01	0.005152834627757\\
236.01	0.0051527094897785\\
237.01	0.0051525817920708\\
238.01	0.00515245148299724\\
239.01	0.00515231850990778\\
240.01	0.00515218281912065\\
241.01	0.00515204435590355\\
242.01	0.00515190306445377\\
243.01	0.00515175888787945\\
244.01	0.00515161176817957\\
245.01	0.00515146164622308\\
246.01	0.0051513084617296\\
247.01	0.00515115215324801\\
248.01	0.00515099265813542\\
249.01	0.00515082991253633\\
250.01	0.00515066385136038\\
251.01	0.00515049440826115\\
252.01	0.00515032151561338\\
253.01	0.00515014510449054\\
254.01	0.00514996510464259\\
255.01	0.00514978144447204\\
256.01	0.00514959405101135\\
257.01	0.00514940284989878\\
258.01	0.00514920776535424\\
259.01	0.00514900872015569\\
260.01	0.00514880563561422\\
261.01	0.0051485984315492\\
262.01	0.00514838702626342\\
263.01	0.00514817133651743\\
264.01	0.00514795127750393\\
265.01	0.00514772676282249\\
266.01	0.00514749770445233\\
267.01	0.00514726401272705\\
268.01	0.00514702559630736\\
269.01	0.00514678236215444\\
270.01	0.00514653421550262\\
271.01	0.00514628105983266\\
272.01	0.00514602279684371\\
273.01	0.00514575932642596\\
274.01	0.0051454905466326\\
275.01	0.005145216353652\\
276.01	0.0051449366417792\\
277.01	0.00514465130338831\\
278.01	0.00514436022890354\\
279.01	0.0051440633067712\\
280.01	0.00514376042343024\\
281.01	0.0051434514632844\\
282.01	0.0051431363086736\\
283.01	0.0051428148398448\\
284.01	0.00514248693492345\\
285.01	0.00514215246988519\\
286.01	0.00514181131852681\\
287.01	0.00514146335243798\\
288.01	0.00514110844097311\\
289.01	0.00514074645122274\\
290.01	0.00514037724798586\\
291.01	0.0051400006937419\\
292.01	0.00513961664862325\\
293.01	0.00513922497038804\\
294.01	0.00513882551439329\\
295.01	0.00513841813356863\\
296.01	0.00513800267838955\\
297.01	0.00513757899685242\\
298.01	0.00513714693444902\\
299.01	0.00513670633414256\\
300.01	0.00513625703634312\\
301.01	0.0051357988788851\\
302.01	0.00513533169700429\\
303.01	0.00513485532331667\\
304.01	0.00513436958779732\\
305.01	0.00513387431776059\\
306.01	0.00513336933784176\\
307.01	0.00513285446997861\\
308.01	0.00513232953339542\\
309.01	0.0051317943445872\\
310.01	0.00513124871730558\\
311.01	0.0051306924625467\\
312.01	0.00513012538853882\\
313.01	0.00512954730073386\\
314.01	0.00512895800179832\\
315.01	0.00512835729160743\\
316.01	0.00512774496724011\\
317.01	0.00512712082297672\\
318.01	0.00512648465029781\\
319.01	0.00512583623788549\\
320.01	0.00512517537162689\\
321.01	0.00512450183461948\\
322.01	0.0051238154071795\\
323.01	0.00512311586685151\\
324.01	0.00512240298842167\\
325.01	0.0051216765439323\\
326.01	0.00512093630270024\\
327.01	0.00512018203133715\\
328.01	0.00511941349377248\\
329.01	0.00511863045127929\\
330.01	0.00511783266250308\\
331.01	0.00511701988349325\\
332.01	0.00511619186773709\\
333.01	0.00511534836619702\\
334.01	0.00511448912735015\\
335.01	0.00511361389723113\\
336.01	0.0051127224194771\\
337.01	0.00511181443537549\\
338.01	0.005110889683914\\
339.01	0.0051099479018334\\
340.01	0.00510898882368156\\
341.01	0.00510801218187047\\
342.01	0.00510701770673342\\
343.01	0.00510600512658526\\
344.01	0.00510497416778141\\
345.01	0.00510392455477939\\
346.01	0.00510285601019871\\
347.01	0.00510176825488083\\
348.01	0.00510066100794834\\
349.01	0.00509953398686104\\
350.01	0.00509838690747049\\
351.01	0.00509721948406981\\
352.01	0.0050960314294396\\
353.01	0.00509482245488743\\
354.01	0.00509359227028119\\
355.01	0.00509234058407305\\
356.01	0.00509106710331452\\
357.01	0.00508977153365951\\
358.01	0.00508845357935487\\
359.01	0.00508711294321496\\
360.01	0.00508574932658015\\
361.01	0.00508436242925602\\
362.01	0.00508295194943009\\
363.01	0.0050815175835661\\
364.01	0.00508005902627021\\
365.01	0.0050785759701294\\
366.01	0.00507706810551666\\
367.01	0.00507553512036194\\
368.01	0.00507397669988443\\
369.01	0.00507239252628455\\
370.01	0.0050707822783914\\
371.01	0.00506914563126312\\
372.01	0.00506748225573749\\
373.01	0.00506579181792998\\
374.01	0.00506407397867616\\
375.01	0.00506232839291728\\
376.01	0.00506055470902655\\
377.01	0.0050587525680755\\
378.01	0.00505692160303954\\
379.01	0.00505506143794429\\
380.01	0.00505317168695244\\
381.01	0.00505125195339577\\
382.01	0.00504930182875586\\
383.01	0.00504732089159889\\
384.01	0.00504530870647419\\
385.01	0.00504326482278501\\
386.01	0.00504118877364544\\
387.01	0.00503908007473788\\
388.01	0.00503693822319072\\
389.01	0.00503476269649633\\
390.01	0.00503255295149568\\
391.01	0.00503030842345703\\
392.01	0.00502802852527992\\
393.01	0.0050257126468598\\
394.01	0.00502336015465031\\
395.01	0.00502097039146361\\
396.01	0.00501854267654705\\
397.01	0.00501607630598036\\
398.01	0.00501357055342879\\
399.01	0.00501102467128963\\
400.01	0.00500843789225892\\
401.01	0.00500580943133779\\
402.01	0.00500313848828397\\
403.01	0.00500042425049423\\
404.01	0.00499766589628516\\
405.01	0.00499486259850927\\
406.01	0.0049920135284137\\
407.01	0.00498911785961365\\
408.01	0.00498617477201683\\
409.01	0.00498318345549585\\
410.01	0.00498014311307841\\
411.01	0.0049770529633992\\
412.01	0.00497391224215602\\
413.01	0.00497072020233393\\
414.01	0.00496747611302017\\
415.01	0.00496417925673757\\
416.01	0.00496082892538658\\
417.01	0.00495742441509799\\
418.01	0.00495396502055228\\
419.01	0.00495045002956129\\
420.01	0.00494687871883495\\
421.01	0.0049432503514938\\
422.01	0.00493956417562139\\
423.01	0.0049358194228934\\
424.01	0.00493201530710189\\
425.01	0.00492815102256774\\
426.01	0.00492422574244058\\
427.01	0.00492023861688482\\
428.01	0.0049161887711512\\
429.01	0.00491207530353777\\
430.01	0.00490789728324046\\
431.01	0.00490365374809945\\
432.01	0.00489934370224745\\
433.01	0.00489496611366758\\
434.01	0.00489051991167313\\
435.01	0.00488600398431918\\
436.01	0.00488141717576553\\
437.01	0.00487675828360596\\
438.01	0.00487202605618598\\
439.01	0.00486721918993338\\
440.01	0.00486233632672787\\
441.01	0.0048573760513388\\
442.01	0.00485233688896245\\
443.01	0.00484721730289378\\
444.01	0.00484201569236629\\
445.01	0.00483673039059846\\
446.01	0.00483135966308048\\
447.01	0.00482590170613808\\
448.01	0.00482035464580481\\
449.01	0.00481471653703252\\
450.01	0.00480898536326192\\
451.01	0.00480315903636792\\
452.01	0.00479723539698434\\
453.01	0.00479121221520126\\
454.01	0.0047850871916114\\
455.01	0.0047788579586666\\
456.01	0.00477252208228742\\
457.01	0.00476607706364531\\
458.01	0.00475952034102221\\
459.01	0.00475284929162807\\
460.01	0.00474606123324042\\
461.01	0.0047391534255208\\
462.01	0.00473212307085284\\
463.01	0.00472496731455358\\
464.01	0.0047176832443234\\
465.01	0.00471026788883426\\
466.01	0.00470271821540123\\
467.01	0.00469503112674995\\
468.01	0.00468720345697479\\
469.01	0.0046792319668779\\
470.01	0.00467111333897339\\
471.01	0.00466284417252799\\
472.01	0.00465442097905768\\
473.01	0.00464584017869519\\
474.01	0.00463709809774316\\
475.01	0.0046281909675301\\
476.01	0.00461911492439009\\
477.01	0.00460986601030792\\
478.01	0.00460044017382117\\
479.01	0.00459083327102632\\
480.01	0.00458104106663141\\
481.01	0.00457105923499075\\
482.01	0.00456088336105611\\
483.01	0.00455050894117474\\
484.01	0.00453993138366703\\
485.01	0.0045291460091181\\
486.01	0.00451814805033159\\
487.01	0.00450693265190246\\
488.01	0.00449549486938637\\
489.01	0.00448382966806533\\
490.01	0.00447193192133378\\
491.01	0.00445979640875732\\
492.01	0.00444741781388005\\
493.01	0.0044347907218801\\
494.01	0.00442190961718438\\
495.01	0.00440876888115547\\
496.01	0.00439536278994963\\
497.01	0.00438168551261193\\
498.01	0.00436773110942803\\
499.01	0.00435349353049303\\
500.01	0.00433896661440256\\
501.01	0.00432414408693396\\
502.01	0.00430901955959005\\
503.01	0.00429358652792935\\
504.01	0.00427783836966785\\
505.01	0.00426176834256679\\
506.01	0.0042453695821297\\
507.01	0.00422863509914185\\
508.01	0.00421155777708945\\
509.01	0.0041941303695008\\
510.01	0.00417634549724843\\
511.01	0.00415819564585181\\
512.01	0.0041396731628067\\
513.01	0.00412077025496171\\
514.01	0.00410147898594817\\
515.01	0.00408179127365571\\
516.01	0.00406169888774293\\
517.01	0.00404119344716381\\
518.01	0.00402026641769837\\
519.01	0.00399890910948619\\
520.01	0.00397711267457466\\
521.01	0.00395486810450582\\
522.01	0.00393216622796758\\
523.01	0.00390899770853628\\
524.01	0.00388535304253463\\
525.01	0.00386122255702903\\
526.01	0.00383659640798494\\
527.01	0.00381146457860206\\
528.01	0.00378581687784591\\
529.01	0.00375964293919767\\
530.01	0.00373293221964467\\
531.01	0.00370567399893881\\
532.01	0.0036778573791551\\
533.01	0.00364947128458843\\
534.01	0.00362050446202891\\
535.01	0.00359094548146376\\
536.01	0.00356078273725293\\
537.01	0.0035300044498321\\
538.01	0.0034985986679996\\
539.01	0.00346655327184927\\
540.01	0.00343385597641933\\
541.01	0.00340049433613169\\
542.01	0.00336645575010434\\
543.01	0.00333172746843132\\
544.01	0.00329629659952941\\
545.01	0.0032601501186634\\
546.01	0.00322327487777092\\
547.01	0.00318565761672175\\
548.01	0.00314728497615331\\
549.01	0.00310814351204539\\
550.01	0.00306821971220547\\
551.01	0.00302750001485491\\
552.01	0.00298597082952428\\
553.01	0.00294361856048271\\
554.01	0.0029004296329475\\
555.01	0.00285639052234093\\
556.01	0.00281148778688329\\
557.01	0.00276570810383654\\
558.01	0.00271903830973548\\
559.01	0.0026714654449714\\
560.01	0.00262297680311917\\
561.01	0.00257355998542694\\
562.01	0.00252320296091246\\
563.01	0.00247189413254027\\
564.01	0.00241962240997658\\
565.01	0.00236637728944275\\
566.01	0.00231214894120763\\
567.01	0.0022569283052712\\
568.01	0.00220070719580021\\
569.01	0.00214347841486824\\
570.01	0.0020852358760371\\
571.01	0.00202597473827579\\
572.01	0.00196569155065329\\
573.01	0.0019043844081454\\
574.01	0.00184205311876436\\
575.01	0.00177869938203216\\
576.01	0.0017143269785692\\
577.01	0.00164894197023678\\
578.01	0.00158255290983592\\
579.01	0.00151517105879892\\
580.01	0.00144681061058329\\
581.01	0.00137748891654737\\
582.01	0.00130722670990687\\
583.01	0.00123604832187718\\
584.01	0.0011639818822259\\
585.01	0.00109105949409744\\
586.01	0.0010173173700143\\
587.01	0.000942795912266653\\
588.01	0.000867539716298419\\
589.01	0.000791597469971103\\
590.01	0.000715021714474689\\
591.01	0.000637868423831121\\
592.01	0.000560196349006645\\
593.01	0.000482066059121876\\
594.01	0.000403538595524574\\
595.01	0.000324673633827242\\
596.01	0.000245527023505882\\
597.01	0.000166394005355891\\
598.01	9.17366970063799e-05\\
599.01	2.94669364271135e-05\\
599.02	2.89574269585688e-05\\
599.03	2.84509530852819e-05\\
599.04	2.79475443462508e-05\\
599.05	2.7447230571279e-05\\
599.06	2.69500418838293e-05\\
599.07	2.64560087039432e-05\\
599.08	2.59651617511691e-05\\
599.09	2.54775320475218e-05\\
599.1	2.49931509204854e-05\\
599.11	2.45120500060158e-05\\
599.12	2.40342612516167e-05\\
599.13	2.35598169193996e-05\\
599.14	2.30887495892042e-05\\
599.15	2.26210921617405e-05\\
599.16	2.21568778617717e-05\\
599.17	2.16961402412976e-05\\
599.18	2.12389131828191e-05\\
599.19	2.07852309025806e-05\\
599.2	2.03351279538938e-05\\
599.21	1.98886392304542e-05\\
599.22	1.94457999697206e-05\\
599.23	1.90066457563046e-05\\
599.24	1.85712125254211e-05\\
599.25	1.81395365663334e-05\\
599.26	1.77116564095865e-05\\
599.27	1.72876122135641e-05\\
599.28	1.68674445361946e-05\\
599.29	1.64511943388859e-05\\
599.3	1.60389029905273e-05\\
599.31	1.56306122715087e-05\\
599.32	1.52263643777902e-05\\
599.33	1.48262019250087e-05\\
599.34	1.44301679526268e-05\\
599.35	1.40383059281154e-05\\
599.36	1.36506597511899e-05\\
599.37	1.32672737580847e-05\\
599.38	1.28881927258535e-05\\
599.39	1.25134618767387e-05\\
599.4	1.21431268825679e-05\\
599.41	1.17772338691924e-05\\
599.42	1.14158294209736e-05\\
599.43	1.10589605853174e-05\\
599.44	1.07066748772523e-05\\
599.45	1.03590202840433e-05\\
599.46	1.00160452698606e-05\\
599.47	9.67779878049101e-06\\
599.48	9.34433024810284e-06\\
599.49	9.0156895960411e-06\\
599.5	8.69192724369146e-06\\
599.51	8.37309411137396e-06\\
599.52	8.05924162529219e-06\\
599.53	7.75042172253791e-06\\
599.54	7.4466868561357e-06\\
599.55	7.14809000013084e-06\\
599.56	6.85468465475535e-06\\
599.57	6.56652485161308e-06\\
599.58	6.28366515892896e-06\\
599.59	6.00616068686249e-06\\
599.6	5.73406709284546e-06\\
599.61	5.46744058700296e-06\\
599.62	5.20633793760217e-06\\
599.63	4.95081647657568e-06\\
599.64	4.70093410508653e-06\\
599.65	4.45674929914347e-06\\
599.66	4.21832111529262e-06\\
599.67	3.98570919634203e-06\\
599.68	3.75897377716608e-06\\
599.69	3.53817569053415e-06\\
599.7	3.32337637303469e-06\\
599.71	3.11463787102881e-06\\
599.72	2.91202284668363e-06\\
599.73	2.71559458404555e-06\\
599.74	2.52541699518292e-06\\
599.75	2.34155462640155e-06\\
599.76	2.16407266449489e-06\\
599.77	1.99303694307928e-06\\
599.78	1.82851394897598e-06\\
599.79	1.67057082867302e-06\\
599.8	1.51927539483211e-06\\
599.81	1.37469613287027e-06\\
599.82	1.23690220760718e-06\\
599.83	1.10596346997172e-06\\
599.84	9.81950463782924e-07\\
599.85	8.64934432591793e-07\\
599.86	7.54987326587186e-07\\
599.87	6.5218180958504e-07\\
599.88	5.56591266064402e-07\\
599.89	4.68289808295413e-07\\
599.9	3.87352283521061e-07\\
599.91	3.13854281218731e-07\\
599.92	2.4787214042421e-07\\
599.93	1.8948295714763e-07\\
599.94	1.38764591834512e-07\\
599.95	9.57956769222224e-08\\
599.96	6.06556244606149e-08\\
599.97	3.34246338194039e-08\\
599.98	1.4183699454523e-08\\
599.99	3.01461875948372e-09\\
600	0\\
};
\addplot [color=black!60!mycolor21,solid,forget plot]
  table[row sep=crcr]{%
0.01	0.00511537456508619\\
1.01	0.00511537357866577\\
2.01	0.00511537257178985\\
3.01	0.00511537154403555\\
4.01	0.00511537049497141\\
5.01	0.00511536942415678\\
6.01	0.00511536833114231\\
7.01	0.00511536721546909\\
8.01	0.00511536607666884\\
9.01	0.00511536491426357\\
10.01	0.00511536372776555\\
11.01	0.00511536251667717\\
12.01	0.00511536128049021\\
13.01	0.0051153600186863\\
14.01	0.00511535873073629\\
15.01	0.00511535741610018\\
16.01	0.00511535607422688\\
17.01	0.00511535470455369\\
18.01	0.00511535330650676\\
19.01	0.00511535187950028\\
20.01	0.0051153504229363\\
21.01	0.00511534893620466\\
22.01	0.00511534741868266\\
23.01	0.00511534586973479\\
24.01	0.00511534428871243\\
25.01	0.0051153426749539\\
26.01	0.00511534102778349\\
27.01	0.00511533934651185\\
28.01	0.00511533763043549\\
29.01	0.00511533587883644\\
30.01	0.00511533409098209\\
31.01	0.00511533226612451\\
32.01	0.00511533040350066\\
33.01	0.00511532850233169\\
34.01	0.00511532656182295\\
35.01	0.0051153245811633\\
36.01	0.00511532255952498\\
37.01	0.00511532049606332\\
38.01	0.00511531838991627\\
39.01	0.00511531624020418\\
40.01	0.00511531404602932\\
41.01	0.00511531180647548\\
42.01	0.00511530952060763\\
43.01	0.00511530718747187\\
44.01	0.00511530480609428\\
45.01	0.00511530237548142\\
46.01	0.00511529989461918\\
47.01	0.00511529736247276\\
48.01	0.00511529477798627\\
49.01	0.00511529214008206\\
50.01	0.00511528944766045\\
51.01	0.00511528669959924\\
52.01	0.00511528389475322\\
53.01	0.00511528103195392\\
54.01	0.00511527811000864\\
55.01	0.00511527512770054\\
56.01	0.0051152720837875\\
57.01	0.00511526897700234\\
58.01	0.00511526580605207\\
59.01	0.0051152625696169\\
60.01	0.00511525926635029\\
61.01	0.00511525589487797\\
62.01	0.00511525245379763\\
63.01	0.00511524894167829\\
64.01	0.00511524535705981\\
65.01	0.00511524169845221\\
66.01	0.00511523796433488\\
67.01	0.00511523415315624\\
68.01	0.00511523026333304\\
69.01	0.0051152262932497\\
70.01	0.00511522224125771\\
71.01	0.0051152181056747\\
72.01	0.00511521388478418\\
73.01	0.0051152095768344\\
74.01	0.00511520518003802\\
75.01	0.00511520069257119\\
76.01	0.00511519611257275\\
77.01	0.00511519143814377\\
78.01	0.0051151866673465\\
79.01	0.00511518179820347\\
80.01	0.00511517682869724\\
81.01	0.00511517175676903\\
82.01	0.00511516658031813\\
83.01	0.005115161297201\\
84.01	0.00511515590523046\\
85.01	0.00511515040217494\\
86.01	0.00511514478575718\\
87.01	0.00511513905365357\\
88.01	0.00511513320349328\\
89.01	0.00511512723285709\\
90.01	0.00511512113927664\\
91.01	0.00511511492023344\\
92.01	0.00511510857315768\\
93.01	0.00511510209542713\\
94.01	0.00511509548436636\\
95.01	0.00511508873724558\\
96.01	0.00511508185127942\\
97.01	0.00511507482362601\\
98.01	0.00511506765138546\\
99.01	0.00511506033159934\\
100.01	0.00511505286124885\\
101.01	0.00511504523725388\\
102.01	0.00511503745647194\\
103.01	0.00511502951569646\\
104.01	0.00511502141165627\\
105.01	0.00511501314101339\\
106.01	0.0051150047003622\\
107.01	0.00511499608622811\\
108.01	0.00511498729506608\\
109.01	0.00511497832325891\\
110.01	0.00511496916711637\\
111.01	0.00511495982287342\\
112.01	0.00511495028668861\\
113.01	0.00511494055464267\\
114.01	0.00511493062273714\\
115.01	0.00511492048689247\\
116.01	0.00511491014294623\\
117.01	0.00511489958665213\\
118.01	0.00511488881367781\\
119.01	0.00511487781960355\\
120.01	0.00511486659991981\\
121.01	0.00511485515002609\\
122.01	0.00511484346522868\\
123.01	0.00511483154073941\\
124.01	0.00511481937167308\\
125.01	0.0051148069530458\\
126.01	0.00511479427977302\\
127.01	0.00511478134666765\\
128.01	0.00511476814843793\\
129.01	0.00511475467968496\\
130.01	0.00511474093490148\\
131.01	0.00511472690846857\\
132.01	0.00511471259465452\\
133.01	0.00511469798761181\\
134.01	0.00511468308137513\\
135.01	0.00511466786985886\\
136.01	0.005114652346855\\
137.01	0.00511463650603059\\
138.01	0.00511462034092494\\
139.01	0.00511460384494737\\
140.01	0.00511458701137482\\
141.01	0.0051145698333489\\
142.01	0.0051145523038734\\
143.01	0.00511453441581127\\
144.01	0.00511451616188224\\
145.01	0.00511449753465995\\
146.01	0.0051144785265685\\
147.01	0.00511445912988006\\
148.01	0.00511443933671178\\
149.01	0.00511441913902264\\
150.01	0.00511439852861036\\
151.01	0.00511437749710821\\
152.01	0.00511435603598174\\
153.01	0.00511433413652551\\
154.01	0.0051143117898597\\
155.01	0.00511428898692678\\
156.01	0.00511426571848785\\
157.01	0.00511424197511929\\
158.01	0.00511421774720892\\
159.01	0.00511419302495233\\
160.01	0.00511416779834923\\
161.01	0.00511414205719964\\
162.01	0.00511411579109974\\
163.01	0.00511408898943806\\
164.01	0.0051140616413913\\
165.01	0.00511403373592053\\
166.01	0.00511400526176639\\
167.01	0.00511397620744537\\
168.01	0.00511394656124498\\
169.01	0.00511391631121967\\
170.01	0.00511388544518601\\
171.01	0.00511385395071814\\
172.01	0.00511382181514319\\
173.01	0.00511378902553609\\
174.01	0.00511375556871525\\
175.01	0.00511372143123701\\
176.01	0.00511368659939112\\
177.01	0.00511365105919513\\
178.01	0.00511361479638916\\
179.01	0.00511357779643073\\
180.01	0.00511354004448925\\
181.01	0.00511350152544029\\
182.01	0.00511346222386012\\
183.01	0.00511342212402013\\
184.01	0.00511338120988018\\
185.01	0.00511333946508353\\
186.01	0.0051132968729502\\
187.01	0.00511325341647084\\
188.01	0.00511320907830058\\
189.01	0.00511316384075266\\
190.01	0.00511311768579164\\
191.01	0.00511307059502656\\
192.01	0.00511302254970484\\
193.01	0.00511297353070479\\
194.01	0.00511292351852883\\
195.01	0.00511287249329615\\
196.01	0.0051128204347357\\
197.01	0.00511276732217853\\
198.01	0.00511271313455037\\
199.01	0.00511265785036407\\
200.01	0.00511260144771163\\
201.01	0.00511254390425591\\
202.01	0.00511248519722357\\
203.01	0.00511242530339583\\
204.01	0.00511236419910063\\
205.01	0.00511230186020375\\
206.01	0.00511223826210062\\
207.01	0.00511217337970698\\
208.01	0.00511210718745071\\
209.01	0.00511203965926171\\
210.01	0.00511197076856337\\
211.01	0.00511190048826304\\
212.01	0.00511182879074205\\
213.01	0.00511175564784649\\
214.01	0.00511168103087718\\
215.01	0.00511160491057922\\
216.01	0.00511152725713227\\
217.01	0.00511144804013977\\
218.01	0.00511136722861875\\
219.01	0.00511128479098874\\
220.01	0.00511120069506087\\
221.01	0.00511111490802715\\
222.01	0.00511102739644873\\
223.01	0.00511093812624481\\
224.01	0.005110847062681\\
225.01	0.00511075417035707\\
226.01	0.00511065941319563\\
227.01	0.00511056275442924\\
228.01	0.00511046415658882\\
229.01	0.00511036358149029\\
230.01	0.00511026099022245\\
231.01	0.00511015634313397\\
232.01	0.00511004959981986\\
233.01	0.00510994071910844\\
234.01	0.00510982965904792\\
235.01	0.00510971637689247\\
236.01	0.00510960082908849\\
237.01	0.00510948297126058\\
238.01	0.0051093627581968\\
239.01	0.00510924014383445\\
240.01	0.00510911508124556\\
241.01	0.00510898752262192\\
242.01	0.00510885741925982\\
243.01	0.00510872472154461\\
244.01	0.00510858937893547\\
245.01	0.00510845133994987\\
246.01	0.00510831055214733\\
247.01	0.00510816696211345\\
248.01	0.00510802051544365\\
249.01	0.00510787115672668\\
250.01	0.00510771882952772\\
251.01	0.00510756347637195\\
252.01	0.00510740503872715\\
253.01	0.0051072434569865\\
254.01	0.00510707867045102\\
255.01	0.00510691061731235\\
256.01	0.00510673923463431\\
257.01	0.00510656445833546\\
258.01	0.00510638622317047\\
259.01	0.00510620446271231\\
260.01	0.0051060191093331\\
261.01	0.00510583009418576\\
262.01	0.0051056373471849\\
263.01	0.00510544079698826\\
264.01	0.00510524037097728\\
265.01	0.00510503599523753\\
266.01	0.00510482759453972\\
267.01	0.00510461509231971\\
268.01	0.00510439841065887\\
269.01	0.0051041774702647\\
270.01	0.00510395219044992\\
271.01	0.00510372248911315\\
272.01	0.00510348828271854\\
273.01	0.00510324948627538\\
274.01	0.00510300601331802\\
275.01	0.00510275777588507\\
276.01	0.00510250468449915\\
277.01	0.00510224664814656\\
278.01	0.00510198357425603\\
279.01	0.00510171536867864\\
280.01	0.00510144193566733\\
281.01	0.0051011631778558\\
282.01	0.0051008789962381\\
283.01	0.00510058929014804\\
284.01	0.00510029395723858\\
285.01	0.00509999289346137\\
286.01	0.0050996859930462\\
287.01	0.00509937314848069\\
288.01	0.00509905425048967\\
289.01	0.00509872918801566\\
290.01	0.00509839784819816\\
291.01	0.00509806011635408\\
292.01	0.0050977158759578\\
293.01	0.00509736500862192\\
294.01	0.00509700739407746\\
295.01	0.00509664291015461\\
296.01	0.00509627143276457\\
297.01	0.00509589283587995\\
298.01	0.00509550699151661\\
299.01	0.00509511376971576\\
300.01	0.00509471303852548\\
301.01	0.0050943046639842\\
302.01	0.00509388851010228\\
303.01	0.00509346443884578\\
304.01	0.00509303231011988\\
305.01	0.00509259198175275\\
306.01	0.00509214330947942\\
307.01	0.00509168614692717\\
308.01	0.00509122034559952\\
309.01	0.00509074575486255\\
310.01	0.00509026222193018\\
311.01	0.00508976959185067\\
312.01	0.00508926770749355\\
313.01	0.00508875640953611\\
314.01	0.00508823553645157\\
315.01	0.00508770492449654\\
316.01	0.00508716440769957\\
317.01	0.00508661381784954\\
318.01	0.00508605298448518\\
319.01	0.00508548173488376\\
320.01	0.00508489989405098\\
321.01	0.0050843072847109\\
322.01	0.0050837037272957\\
323.01	0.0050830890399361\\
324.01	0.00508246303845125\\
325.01	0.00508182553633909\\
326.01	0.00508117634476631\\
327.01	0.00508051527255817\\
328.01	0.00507984212618831\\
329.01	0.00507915670976748\\
330.01	0.00507845882503223\\
331.01	0.00507774827133237\\
332.01	0.00507702484561849\\
333.01	0.00507628834242714\\
334.01	0.00507553855386577\\
335.01	0.00507477526959526\\
336.01	0.00507399827681169\\
337.01	0.00507320736022492\\
338.01	0.0050724023020362\\
339.01	0.00507158288191181\\
340.01	0.00507074887695489\\
341.01	0.00506990006167322\\
342.01	0.00506903620794389\\
343.01	0.00506815708497276\\
344.01	0.00506726245925113\\
345.01	0.00506635209450567\\
346.01	0.00506542575164405\\
347.01	0.00506448318869422\\
348.01	0.00506352416073646\\
349.01	0.00506254841982928\\
350.01	0.00506155571492676\\
351.01	0.00506054579178812\\
352.01	0.00505951839287799\\
353.01	0.00505847325725697\\
354.01	0.00505741012046191\\
355.01	0.00505632871437514\\
356.01	0.00505522876708182\\
357.01	0.0050541100027152\\
358.01	0.00505297214128824\\
359.01	0.00505181489851223\\
360.01	0.00505063798560022\\
361.01	0.00504944110905546\\
362.01	0.00504822397044529\\
363.01	0.00504698626615752\\
364.01	0.00504572768714187\\
365.01	0.00504444791863355\\
366.01	0.00504314663986067\\
367.01	0.00504182352373438\\
368.01	0.00504047823652271\\
369.01	0.00503911043750747\\
370.01	0.00503771977862624\\
371.01	0.0050363059040985\\
372.01	0.00503486845003876\\
373.01	0.00503340704405683\\
374.01	0.00503192130484766\\
375.01	0.00503041084177334\\
376.01	0.00502887525443856\\
377.01	0.00502731413226412\\
378.01	0.00502572705406087\\
379.01	0.00502411358760904\\
380.01	0.0050224732892465\\
381.01	0.00502080570347132\\
382.01	0.00501911036256501\\
383.01	0.00501738678624098\\
384.01	0.00501563448132577\\
385.01	0.00501385294147864\\
386.01	0.00501204164695771\\
387.01	0.00501020006443945\\
388.01	0.00500832764689743\\
389.01	0.00500642383354889\\
390.01	0.00500448804987441\\
391.01	0.00500251970771589\\
392.01	0.00500051820545903\\
393.01	0.00499848292830042\\
394.01	0.00499641324860389\\
395.01	0.00499430852634055\\
396.01	0.0049921681096128\\
397.01	0.00498999133525027\\
398.01	0.00498777752947023\\
399.01	0.0049855260085841\\
400.01	0.00498323607973127\\
401.01	0.00498090704161521\\
402.01	0.00497853818521132\\
403.01	0.00497612879441471\\
404.01	0.00497367814658854\\
405.01	0.00497118551297381\\
406.01	0.00496865015892014\\
407.01	0.00496607134389796\\
408.01	0.00496344832125691\\
409.01	0.00496078033770405\\
410.01	0.00495806663248456\\
411.01	0.00495530643626621\\
412.01	0.00495249896974442\\
413.01	0.00494964344200882\\
414.01	0.0049467390487322\\
415.01	0.00494378497026461\\
416.01	0.004940780369726\\
417.01	0.00493772439119356\\
418.01	0.0049346161580621\\
419.01	0.00493145477161677\\
420.01	0.0049282393097965\\
421.01	0.00492496882606195\\
422.01	0.00492164234826773\\
423.01	0.00491825887750382\\
424.01	0.00491481738690305\\
425.01	0.00491131682041853\\
426.01	0.00490775609157437\\
427.01	0.0049041340821939\\
428.01	0.00490044964111023\\
429.01	0.00489670158286241\\
430.01	0.0048928886863865\\
431.01	0.00488900969370451\\
432.01	0.00488506330862008\\
433.01	0.00488104819542758\\
434.01	0.00487696297764154\\
435.01	0.00487280623675731\\
436.01	0.0048685765110472\\
437.01	0.00486427229440418\\
438.01	0.00485989203523984\\
439.01	0.00485543413544438\\
440.01	0.00485089694941698\\
441.01	0.00484627878317314\\
442.01	0.00484157789353523\\
443.01	0.00483679248740917\\
444.01	0.00483192072115264\\
445.01	0.00482696070003325\\
446.01	0.00482191047777615\\
447.01	0.0048167680561974\\
448.01	0.00481153138491524\\
449.01	0.00480619836112931\\
450.01	0.00480076682945332\\
451.01	0.00479523458178626\\
452.01	0.0047895993571997\\
453.01	0.00478385884181872\\
454.01	0.00477801066867047\\
455.01	0.00477205241747283\\
456.01	0.00476598161433339\\
457.01	0.00475979573133239\\
458.01	0.00475349218596161\\
459.01	0.00474706834039562\\
460.01	0.00474052150058208\\
461.01	0.00473384891513581\\
462.01	0.00472704777404177\\
463.01	0.00472011520717495\\
464.01	0.0047130482826625\\
465.01	0.00470584400512422\\
466.01	0.00469849931383928\\
467.01	0.00469101108089903\\
468.01	0.00468337610940962\\
469.01	0.00467559113180777\\
470.01	0.00466765280834874\\
471.01	0.00465955772580652\\
472.01	0.00465130239640618\\
473.01	0.00464288325697488\\
474.01	0.00463429666827228\\
475.01	0.00462553891443265\\
476.01	0.00461660620244293\\
477.01	0.00460749466159679\\
478.01	0.00459820034288695\\
479.01	0.00458871921831872\\
480.01	0.00457904718012815\\
481.01	0.00456918003989079\\
482.01	0.00455911352751145\\
483.01	0.00454884329008707\\
484.01	0.00453836489063851\\
485.01	0.00452767380671381\\
486.01	0.00451676542886776\\
487.01	0.00450563505902704\\
488.01	0.00449427790875835\\
489.01	0.00448268909745497\\
490.01	0.00447086365046334\\
491.01	0.00445879649716923\\
492.01	0.00444648246906463\\
493.01	0.00443391629780925\\
494.01	0.00442109261329795\\
495.01	0.00440800594173683\\
496.01	0.00439465070372407\\
497.01	0.00438102121232231\\
498.01	0.00436711167110619\\
499.01	0.00435291617216462\\
500.01	0.00433842869404037\\
501.01	0.00432364309959411\\
502.01	0.00430855313379167\\
503.01	0.00429315242141763\\
504.01	0.00427743446472644\\
505.01	0.0042613926410404\\
506.01	0.0042450202003061\\
507.01	0.00422831026262157\\
508.01	0.00421125581574242\\
509.01	0.00419384971257737\\
510.01	0.00417608466867955\\
511.01	0.00415795325973936\\
512.01	0.00413944791908216\\
513.01	0.00412056093517416\\
514.01	0.00410128444913751\\
515.01	0.00408161045227904\\
516.01	0.0040615307836341\\
517.01	0.00404103712753402\\
518.01	0.00402012101120428\\
519.01	0.00399877380240454\\
520.01	0.00397698670712463\\
521.01	0.00395475076734716\\
522.01	0.00393205685889359\\
523.01	0.00390889568936628\\
524.01	0.00388525779620339\\
525.01	0.00386113354486115\\
526.01	0.00383651312714225\\
527.01	0.00381138655968965\\
528.01	0.00378574368266509\\
529.01	0.00375957415863976\\
530.01	0.00373286747171988\\
531.01	0.00370561292693941\\
532.01	0.00367779964995124\\
533.01	0.0036494165870535\\
534.01	0.00362045250559075\\
535.01	0.0035908959947719\\
536.01	0.0035607354669546\\
537.01	0.00352995915944617\\
538.01	0.00349855513688045\\
539.01	0.00346651129423165\\
540.01	0.00343381536053716\\
541.01	0.0034004549034037\\
542.01	0.00336641733438228\\
543.01	0.00333168991530363\\
544.01	0.00329625976567544\\
545.01	0.00326011387125328\\
546.01	0.00322323909390545\\
547.01	0.00318562218290602\\
548.01	0.00314724978780293\\
549.01	0.00310810847301851\\
550.01	0.00306818473435897\\
551.01	0.00302746501762232\\
552.01	0.00298593573951309\\
553.01	0.00294358331108954\\
554.01	0.00290039416398957\\
555.01	0.00285635477970408\\
556.01	0.00281145172218551\\
557.01	0.00276567167410694\\
558.01	0.00271900147710948\\
559.01	0.00267142817640317\\
560.01	0.00262293907011199\\
561.01	0.00257352176378272\\
562.01	0.00252316423050231\\
563.01	0.00247185487709762\\
564.01	0.00241958261691408\\
565.01	0.00236633694969541\\
566.01	0.00231210804910259\\
567.01	0.00225688685842675\\
568.01	0.00220066519505408\\
569.01	0.00214343586423708\\
570.01	0.00208519278270673\\
571.01	0.0020259311126231\\
572.01	0.00196564740629848\\
573.01	0.00190433976203401\\
574.01	0.00184200799127701\\
575.01	0.00177865379711897\\
576.01	0.00171428096390598\\
577.01	0.00164889555739739\\
578.01	0.00158250613447483\\
579.01	0.00151512396083625\\
580.01	0.00144676323438244\\
581.01	0.00137744131107293\\
582.01	0.00130717892884751\\
583.01	0.00123600042371503\\
584.01	0.00116393393022847\\
585.01	0.00109101155620331\\
586.01	0.00101726951857635\\
587.01	0.000942748223607881\\
588.01	0.00086749227002462\\
589.01	0.000791550347971492\\
590.01	0.000714974999524649\\
591.01	0.000637822197691413\\
592.01	0.000560150689888671\\
593.01	0.000482021038358199\\
594.01	0.000403494273246938\\
595.01	0.000324630053410709\\
596.01	0.000245484204483332\\
597.01	0.00016637427046296\\
598.01	9.17366970063765e-05\\
599.01	2.94669364271135e-05\\
599.02	2.89574269585705e-05\\
599.03	2.84509530852819e-05\\
599.04	2.79475443462508e-05\\
599.05	2.7447230571279e-05\\
599.06	2.69500418838293e-05\\
599.07	2.64560087039432e-05\\
599.08	2.59651617511691e-05\\
599.09	2.54775320475235e-05\\
599.1	2.49931509204836e-05\\
599.11	2.45120500060158e-05\\
599.12	2.40342612516167e-05\\
599.13	2.35598169193978e-05\\
599.14	2.30887495892024e-05\\
599.15	2.26210921617405e-05\\
599.16	2.215687786177e-05\\
599.17	2.16961402412993e-05\\
599.18	2.12389131828191e-05\\
599.19	2.07852309025824e-05\\
599.2	2.03351279538938e-05\\
599.21	1.98886392304542e-05\\
599.22	1.94457999697206e-05\\
599.23	1.90066457563063e-05\\
599.24	1.85712125254211e-05\\
599.25	1.81395365663334e-05\\
599.26	1.77116564095865e-05\\
599.27	1.72876122135658e-05\\
599.28	1.68674445361963e-05\\
599.29	1.64511943388859e-05\\
599.3	1.60389029905273e-05\\
599.31	1.56306122715087e-05\\
599.32	1.5226364377792e-05\\
599.33	1.48262019250105e-05\\
599.34	1.44301679526268e-05\\
599.35	1.40383059281154e-05\\
599.36	1.36506597511916e-05\\
599.37	1.32672737580847e-05\\
599.38	1.28881927258535e-05\\
599.39	1.25134618767404e-05\\
599.4	1.21431268825696e-05\\
599.41	1.17772338691924e-05\\
599.42	1.14158294209719e-05\\
599.43	1.10589605853174e-05\\
599.44	1.0706674877254e-05\\
599.45	1.03590202840433e-05\\
599.46	1.00160452698589e-05\\
599.47	9.67779878049101e-06\\
599.48	9.34433024810284e-06\\
599.49	9.01568959604283e-06\\
599.5	8.69192724369319e-06\\
599.51	8.37309411137396e-06\\
599.52	8.05924162529392e-06\\
599.53	7.75042172253791e-06\\
599.54	7.44668685613396e-06\\
599.55	7.14809000013084e-06\\
599.56	6.85468465475708e-06\\
599.57	6.56652485161308e-06\\
599.58	6.28366515892896e-06\\
599.59	6.00616068686076e-06\\
599.6	5.73406709284546e-06\\
599.61	5.46744058700296e-06\\
599.62	5.2063379376039e-06\\
599.63	4.95081647657741e-06\\
599.64	4.70093410508653e-06\\
599.65	4.45674929914347e-06\\
599.66	4.21832111529089e-06\\
599.67	3.98570919634376e-06\\
599.68	3.75897377716608e-06\\
599.69	3.53817569053415e-06\\
599.7	3.32337637303295e-06\\
599.71	3.11463787102881e-06\\
599.72	2.91202284668536e-06\\
599.73	2.71559458404382e-06\\
599.74	2.52541699518466e-06\\
599.75	2.34155462640329e-06\\
599.76	2.16407266449489e-06\\
599.77	1.99303694307755e-06\\
599.78	1.82851394897598e-06\\
599.79	1.67057082867302e-06\\
599.8	1.51927539483211e-06\\
599.81	1.37469613287027e-06\\
599.82	1.23690220760718e-06\\
599.83	1.10596346997172e-06\\
599.84	9.81950463782924e-07\\
599.85	8.64934432591793e-07\\
599.86	7.54987326587186e-07\\
599.87	6.5218180958504e-07\\
599.88	5.56591266064402e-07\\
599.89	4.68289808295413e-07\\
599.9	3.87352283521061e-07\\
599.91	3.13854281216996e-07\\
599.92	2.47872140425945e-07\\
599.93	1.89482957149364e-07\\
599.94	1.38764591834512e-07\\
599.95	9.57956769204876e-08\\
599.96	6.06556244606149e-08\\
599.97	3.34246338211386e-08\\
599.98	1.4183699454523e-08\\
599.99	3.01461875948372e-09\\
600	0\\
};
\addplot [color=black!80!mycolor21,solid,forget plot]
  table[row sep=crcr]{%
0.01	0.00508968489854863\\
1.01	0.00508968392840915\\
2.01	0.00508968293827033\\
3.01	0.00508968192772168\\
4.01	0.00508968089634424\\
5.01	0.00508967984371108\\
6.01	0.00508967876938601\\
7.01	0.00508967767292402\\
8.01	0.0050896765538712\\
9.01	0.00508967541176465\\
10.01	0.00508967424613145\\
11.01	0.00508967305648953\\
12.01	0.00508967184234672\\
13.01	0.00508967060320077\\
14.01	0.00508966933853946\\
15.01	0.00508966804784016\\
16.01	0.00508966673056913\\
17.01	0.00508966538618245\\
18.01	0.00508966401412461\\
19.01	0.00508966261382906\\
20.01	0.00508966118471765\\
21.01	0.00508965972620041\\
22.01	0.00508965823767534\\
23.01	0.00508965671852852\\
24.01	0.00508965516813317\\
25.01	0.00508965358584988\\
26.01	0.00508965197102632\\
27.01	0.00508965032299697\\
28.01	0.00508964864108255\\
29.01	0.00508964692459043\\
30.01	0.00508964517281322\\
31.01	0.0050896433850298\\
32.01	0.00508964156050412\\
33.01	0.00508963969848547\\
34.01	0.00508963779820734\\
35.01	0.00508963585888803\\
36.01	0.00508963387973035\\
37.01	0.00508963185992034\\
38.01	0.00508962979862791\\
39.01	0.00508962769500595\\
40.01	0.00508962554819032\\
41.01	0.00508962335729921\\
42.01	0.00508962112143281\\
43.01	0.00508961883967355\\
44.01	0.00508961651108512\\
45.01	0.00508961413471206\\
46.01	0.00508961170957968\\
47.01	0.00508960923469374\\
48.01	0.0050896067090395\\
49.01	0.005089604131582\\
50.01	0.00508960150126519\\
51.01	0.00508959881701179\\
52.01	0.00508959607772295\\
53.01	0.00508959328227709\\
54.01	0.00508959042953037\\
55.01	0.00508958751831546\\
56.01	0.00508958454744193\\
57.01	0.0050895815156951\\
58.01	0.00508957842183554\\
59.01	0.00508957526459921\\
60.01	0.00508957204269618\\
61.01	0.00508956875481088\\
62.01	0.00508956539960106\\
63.01	0.00508956197569726\\
64.01	0.00508955848170247\\
65.01	0.00508955491619142\\
66.01	0.00508955127771036\\
67.01	0.00508954756477626\\
68.01	0.00508954377587622\\
69.01	0.00508953990946661\\
70.01	0.00508953596397271\\
71.01	0.00508953193778857\\
72.01	0.00508952782927572\\
73.01	0.00508952363676241\\
74.01	0.0050895193585434\\
75.01	0.0050895149928794\\
76.01	0.00508951053799624\\
77.01	0.00508950599208363\\
78.01	0.00508950135329541\\
79.01	0.00508949661974818\\
80.01	0.00508949178952017\\
81.01	0.00508948686065187\\
82.01	0.00508948183114403\\
83.01	0.00508947669895704\\
84.01	0.00508947146201124\\
85.01	0.00508946611818433\\
86.01	0.00508946066531193\\
87.01	0.00508945510118605\\
88.01	0.00508944942355465\\
89.01	0.00508944363012057\\
90.01	0.00508943771854086\\
91.01	0.00508943168642519\\
92.01	0.00508942553133553\\
93.01	0.00508941925078526\\
94.01	0.005089412842238\\
95.01	0.00508940630310644\\
96.01	0.00508939963075168\\
97.01	0.0050893928224821\\
98.01	0.00508938587555247\\
99.01	0.00508937878716225\\
100.01	0.00508937155445535\\
101.01	0.0050893641745188\\
102.01	0.00508935664438125\\
103.01	0.0050893489610122\\
104.01	0.00508934112132043\\
105.01	0.00508933312215349\\
106.01	0.00508932496029615\\
107.01	0.00508931663246878\\
108.01	0.00508930813532626\\
109.01	0.00508929946545756\\
110.01	0.00508929061938363\\
111.01	0.00508928159355538\\
112.01	0.00508927238435405\\
113.01	0.00508926298808852\\
114.01	0.00508925340099421\\
115.01	0.00508924361923217\\
116.01	0.00508923363888714\\
117.01	0.00508922345596582\\
118.01	0.0050892130663955\\
119.01	0.00508920246602324\\
120.01	0.00508919165061311\\
121.01	0.00508918061584572\\
122.01	0.00508916935731579\\
123.01	0.00508915787053103\\
124.01	0.00508914615090978\\
125.01	0.00508913419377994\\
126.01	0.00508912199437698\\
127.01	0.00508910954784219\\
128.01	0.0050890968492203\\
129.01	0.00508908389345868\\
130.01	0.00508907067540444\\
131.01	0.00508905718980313\\
132.01	0.00508904343129645\\
133.01	0.00508902939442083\\
134.01	0.00508901507360411\\
135.01	0.00508900046316471\\
136.01	0.00508898555730928\\
137.01	0.00508897035012952\\
138.01	0.00508895483560167\\
139.01	0.00508893900758307\\
140.01	0.00508892285981006\\
141.01	0.00508890638589551\\
142.01	0.00508888957932736\\
143.01	0.00508887243346507\\
144.01	0.00508885494153837\\
145.01	0.00508883709664292\\
146.01	0.00508881889173978\\
147.01	0.00508880031965169\\
148.01	0.00508878137306073\\
149.01	0.00508876204450545\\
150.01	0.00508874232637809\\
151.01	0.00508872221092231\\
152.01	0.00508870169022961\\
153.01	0.00508868075623708\\
154.01	0.00508865940072392\\
155.01	0.00508863761530899\\
156.01	0.00508861539144725\\
157.01	0.00508859272042685\\
158.01	0.00508856959336592\\
159.01	0.00508854600120961\\
160.01	0.00508852193472651\\
161.01	0.00508849738450504\\
162.01	0.00508847234095076\\
163.01	0.00508844679428235\\
164.01	0.00508842073452835\\
165.01	0.00508839415152334\\
166.01	0.00508836703490458\\
167.01	0.00508833937410815\\
168.01	0.00508831115836474\\
169.01	0.00508828237669681\\
170.01	0.00508825301791326\\
171.01	0.00508822307060691\\
172.01	0.00508819252314936\\
173.01	0.00508816136368744\\
174.01	0.00508812958013865\\
175.01	0.00508809716018666\\
176.01	0.00508806409127808\\
177.01	0.00508803036061665\\
178.01	0.00508799595515917\\
179.01	0.00508796086161161\\
180.01	0.00508792506642342\\
181.01	0.00508788855578339\\
182.01	0.00508785131561455\\
183.01	0.00508781333156922\\
184.01	0.00508777458902433\\
185.01	0.00508773507307595\\
186.01	0.00508769476853422\\
187.01	0.00508765365991789\\
188.01	0.00508761173144972\\
189.01	0.00508756896704989\\
190.01	0.00508752535033114\\
191.01	0.0050874808645932\\
192.01	0.00508743549281676\\
193.01	0.00508738921765766\\
194.01	0.00508734202144088\\
195.01	0.00508729388615485\\
196.01	0.00508724479344496\\
197.01	0.00508719472460746\\
198.01	0.00508714366058295\\
199.01	0.00508709158195026\\
200.01	0.00508703846891954\\
201.01	0.00508698430132592\\
202.01	0.0050869290586222\\
203.01	0.00508687271987262\\
204.01	0.00508681526374536\\
205.01	0.00508675666850569\\
206.01	0.00508669691200885\\
207.01	0.0050866359716921\\
208.01	0.00508657382456766\\
209.01	0.0050865104472155\\
210.01	0.0050864458157747\\
211.01	0.00508637990593653\\
212.01	0.00508631269293606\\
213.01	0.00508624415154391\\
214.01	0.0050861742560585\\
215.01	0.00508610298029722\\
216.01	0.00508603029758829\\
217.01	0.00508595618076212\\
218.01	0.00508588060214226\\
219.01	0.00508580353353721\\
220.01	0.00508572494623056\\
221.01	0.00508564481097246\\
222.01	0.00508556309797024\\
223.01	0.00508547977687867\\
224.01	0.00508539481679076\\
225.01	0.00508530818622798\\
226.01	0.00508521985313024\\
227.01	0.00508512978484633\\
228.01	0.00508503794812291\\
229.01	0.00508494430909511\\
230.01	0.00508484883327607\\
231.01	0.00508475148554574\\
232.01	0.00508465223014012\\
233.01	0.00508455103064164\\
234.01	0.00508444784996679\\
235.01	0.00508434265035488\\
236.01	0.00508423539335794\\
237.01	0.00508412603982814\\
238.01	0.00508401454990652\\
239.01	0.00508390088301113\\
240.01	0.00508378499782545\\
241.01	0.00508366685228511\\
242.01	0.00508354640356693\\
243.01	0.00508342360807613\\
244.01	0.00508329842143316\\
245.01	0.00508317079846144\\
246.01	0.00508304069317468\\
247.01	0.00508290805876296\\
248.01	0.00508277284758022\\
249.01	0.00508263501113062\\
250.01	0.00508249450005546\\
251.01	0.00508235126411846\\
252.01	0.00508220525219239\\
253.01	0.00508205641224551\\
254.01	0.00508190469132674\\
255.01	0.00508175003555108\\
256.01	0.00508159239008613\\
257.01	0.00508143169913625\\
258.01	0.00508126790592883\\
259.01	0.005081100952698\\
260.01	0.0050809307806709\\
261.01	0.0050807573300511\\
262.01	0.00508058054000401\\
263.01	0.00508040034864073\\
264.01	0.00508021669300259\\
265.01	0.00508002950904491\\
266.01	0.00507983873162133\\
267.01	0.00507964429446718\\
268.01	0.00507944613018312\\
269.01	0.00507924417021882\\
270.01	0.00507903834485659\\
271.01	0.00507882858319344\\
272.01	0.00507861481312517\\
273.01	0.00507839696132857\\
274.01	0.00507817495324472\\
275.01	0.00507794871306106\\
276.01	0.00507771816369434\\
277.01	0.00507748322677191\\
278.01	0.00507724382261517\\
279.01	0.0050769998702209\\
280.01	0.00507675128724319\\
281.01	0.00507649798997494\\
282.01	0.00507623989332975\\
283.01	0.0050759769108233\\
284.01	0.00507570895455487\\
285.01	0.00507543593518795\\
286.01	0.00507515776193144\\
287.01	0.00507487434252075\\
288.01	0.0050745855831988\\
289.01	0.00507429138869547\\
290.01	0.00507399166220907\\
291.01	0.00507368630538611\\
292.01	0.00507337521830149\\
293.01	0.0050730582994385\\
294.01	0.00507273544566812\\
295.01	0.00507240655222904\\
296.01	0.00507207151270666\\
297.01	0.00507173021901234\\
298.01	0.0050713825613626\\
299.01	0.00507102842825736\\
300.01	0.00507066770645871\\
301.01	0.00507030028096876\\
302.01	0.00506992603500801\\
303.01	0.00506954484999268\\
304.01	0.00506915660551168\\
305.01	0.00506876117930481\\
306.01	0.0050683584472374\\
307.01	0.00506794828327829\\
308.01	0.00506753055947467\\
309.01	0.00506710514592742\\
310.01	0.00506667191076681\\
311.01	0.00506623072012522\\
312.01	0.00506578143811225\\
313.01	0.00506532392678706\\
314.01	0.00506485804613066\\
315.01	0.00506438365401823\\
316.01	0.00506390060618953\\
317.01	0.00506340875621869\\
318.01	0.00506290795548339\\
319.01	0.0050623980531334\\
320.01	0.0050618788960569\\
321.01	0.0050613503288471\\
322.01	0.00506081219376555\\
323.01	0.00506026433070663\\
324.01	0.00505970657715841\\
325.01	0.00505913876816366\\
326.01	0.00505856073627759\\
327.01	0.00505797231152495\\
328.01	0.00505737332135447\\
329.01	0.00505676359059195\\
330.01	0.00505614294139056\\
331.01	0.00505551119317966\\
332.01	0.00505486816260919\\
333.01	0.00505421366349356\\
334.01	0.0050535475067512\\
335.01	0.00505286950034228\\
336.01	0.00505217944920215\\
337.01	0.00505147715517194\\
338.01	0.00505076241692552\\
339.01	0.00505003502989245\\
340.01	0.00504929478617756\\
341.01	0.00504854147447524\\
342.01	0.00504777487998067\\
343.01	0.00504699478429536\\
344.01	0.00504620096532795\\
345.01	0.00504539319719129\\
346.01	0.00504457125009211\\
347.01	0.00504373489021737\\
348.01	0.00504288387961356\\
349.01	0.00504201797606057\\
350.01	0.00504113693294076\\
351.01	0.00504024049909947\\
352.01	0.00503932841870174\\
353.01	0.00503840043108143\\
354.01	0.00503745627058352\\
355.01	0.00503649566640135\\
356.01	0.00503551834240592\\
357.01	0.00503452401696963\\
358.01	0.00503351240278326\\
359.01	0.00503248320666641\\
360.01	0.00503143612937219\\
361.01	0.00503037086538597\\
362.01	0.00502928710271854\\
363.01	0.00502818452269397\\
364.01	0.00502706279973244\\
365.01	0.00502592160112882\\
366.01	0.00502476058682842\\
367.01	0.00502357940919909\\
368.01	0.00502237771280032\\
369.01	0.00502115513415367\\
370.01	0.00501991130150968\\
371.01	0.00501864583461723\\
372.01	0.00501735834449525\\
373.01	0.00501604843320617\\
374.01	0.00501471569363434\\
375.01	0.00501335970927069\\
376.01	0.0050119800540043\\
377.01	0.0050105762919218\\
378.01	0.00500914797711773\\
379.01	0.00500769465351591\\
380.01	0.00500621585470338\\
381.01	0.00500471110377882\\
382.01	0.00500317991321475\\
383.01	0.00500162178473713\\
384.01	0.00500003620922055\\
385.01	0.00499842266659988\\
386.01	0.00499678062579989\\
387.01	0.004995109544679\\
388.01	0.00499340886998947\\
389.01	0.00499167803735109\\
390.01	0.00498991647123458\\
391.01	0.00498812358495372\\
392.01	0.00498629878066159\\
393.01	0.00498444144934696\\
394.01	0.00498255097082289\\
395.01	0.00498062671370551\\
396.01	0.00497866803537291\\
397.01	0.00497667428189747\\
398.01	0.00497464478794445\\
399.01	0.00497257887662799\\
400.01	0.00497047585931557\\
401.01	0.00496833503537484\\
402.01	0.00496615569185451\\
403.01	0.00496393710309253\\
404.01	0.00496167853024835\\
405.01	0.00495937922075627\\
406.01	0.00495703840769956\\
407.01	0.00495465530910754\\
408.01	0.00495222912718346\\
409.01	0.00494975904747006\\
410.01	0.00494724423796623\\
411.01	0.00494468384820806\\
412.01	0.00494207700833176\\
413.01	0.00493942282813428\\
414.01	0.0049367203961457\\
415.01	0.00493396877872884\\
416.01	0.00493116701921221\\
417.01	0.00492831413705652\\
418.01	0.00492540912705394\\
419.01	0.00492245095854786\\
420.01	0.00491943857466022\\
421.01	0.00491637089151961\\
422.01	0.00491324679748207\\
423.01	0.00491006515234761\\
424.01	0.00490682478657479\\
425.01	0.00490352450049337\\
426.01	0.00490016306351852\\
427.01	0.00489673921336825\\
428.01	0.00489325165528674\\
429.01	0.00488969906127556\\
430.01	0.00488608006933446\\
431.01	0.00488239328271397\\
432.01	0.00487863726918218\\
433.01	0.00487481056030704\\
434.01	0.00487091165075614\\
435.01	0.00486693899761418\\
436.01	0.00486289101971994\\
437.01	0.00485876609702326\\
438.01	0.00485456256996054\\
439.01	0.00485027873885053\\
440.01	0.00484591286330686\\
441.01	0.00484146316166711\\
442.01	0.00483692781043501\\
443.01	0.00483230494373362\\
444.01	0.00482759265276427\\
445.01	0.004822788985267\\
446.01	0.0048178919449794\\
447.01	0.00481289949108397\\
448.01	0.00480780953764226\\
449.01	0.00480261995300564\\
450.01	0.00479732855919909\\
451.01	0.00479193313126877\\
452.01	0.00478643139658867\\
453.01	0.00478082103412083\\
454.01	0.0047750996736238\\
455.01	0.00476926489480607\\
456.01	0.004763314226423\\
457.01	0.00475724514531626\\
458.01	0.00475105507539817\\
459.01	0.00474474138658562\\
460.01	0.00473830139368701\\
461.01	0.00473173235525485\\
462.01	0.00472503147240865\\
463.01	0.00471819588764355\\
464.01	0.00471122268363353\\
465.01	0.00470410888204121\\
466.01	0.00469685144234548\\
467.01	0.00468944726069261\\
468.01	0.00468189316877687\\
469.01	0.00467418593275002\\
470.01	0.00466632225215585\\
471.01	0.00465829875888062\\
472.01	0.00465011201610793\\
473.01	0.00464175851726228\\
474.01	0.0046332346849264\\
475.01	0.00462453686972014\\
476.01	0.00461566134912937\\
477.01	0.00460660432628032\\
478.01	0.00459736192865624\\
479.01	0.0045879302067535\\
480.01	0.00457830513267719\\
481.01	0.00456848259867723\\
482.01	0.0045584584156257\\
483.01	0.00454822831143751\\
484.01	0.00453778792943787\\
485.01	0.00452713282668092\\
486.01	0.00451625847222222\\
487.01	0.00450516024535199\\
488.01	0.00449383343379027\\
489.01	0.00448227323184985\\
490.01	0.0044704747385684\\
491.01	0.00445843295581342\\
492.01	0.00444614278635746\\
493.01	0.00443359903192642\\
494.01	0.00442079639121603\\
495.01	0.00440772945787658\\
496.01	0.00439439271846087\\
497.01	0.00438078055033382\\
498.01	0.00436688721954141\\
499.01	0.00435270687863867\\
500.01	0.00433823356447654\\
501.01	0.00432346119595061\\
502.01	0.00430838357171449\\
503.01	0.0042929943678623\\
504.01	0.00427728713558209\\
505.01	0.00426125529878641\\
506.01	0.00424489215172238\\
507.01	0.00422819085656428\\
508.01	0.00421114444099322\\
509.01	0.00419374579576618\\
510.01	0.0041759876722773\\
511.01	0.00415786268011639\\
512.01	0.00413936328462638\\
513.01	0.00412048180446544\\
514.01	0.00410121040917822\\
515.01	0.00408154111678125\\
516.01	0.00406146579137078\\
517.01	0.0040409761407585\\
518.01	0.00402006371414559\\
519.01	0.0039987198998435\\
520.01	0.00397693592305065\\
521.01	0.00395470284369803\\
522.01	0.00393201155437433\\
523.01	0.00390885277834494\\
524.01	0.00388521706767742\\
525.01	0.00386109480149263\\
526.01	0.00383647618435695\\
527.01	0.00381135124483494\\
528.01	0.00378570983422737\\
529.01	0.00375954162551551\\
530.01	0.00373283611253975\\
531.01	0.00370558260944227\\
532.01	0.0036777702504053\\
533.01	0.0036493879897215\\
534.01	0.00362042460223506\\
535.01	0.00359086868419784\\
536.01	0.00356070865458678\\
537.01	0.0035299327569359\\
538.01	0.00349852906174065\\
539.01	0.00346648546949845\\
540.01	0.00343378971445346\\
541.01	0.00340042936912447\\
542.01	0.00336639184969894\\
543.01	0.00333166442238612\\
544.01	0.00329623421082995\\
545.01	0.00326008820469374\\
546.01	0.00322321326953876\\
547.01	0.00318559615812999\\
548.01	0.00314722352331424\\
549.01	0.00310808193263235\\
550.01	0.00306815788483875\\
551.01	0.00302743782851966\\
552.01	0.00298590818301772\\
553.01	0.00294355536188978\\
554.01	0.00290036579914364\\
555.01	0.00285632597852142\\
556.01	0.00281142246612037\\
557.01	0.00276564194666362\\
558.01	0.00271897126376054\\
559.01	0.00267139746452063\\
560.01	0.00262290784891359\\
561.01	0.0025734900242925\\
562.01	0.00252313196552769\\
563.01	0.00247182208122277\\
564.01	0.00241954928651093\\
565.01	0.00236630308295206\\
566.01	0.00231207364607109\\
567.01	0.00225685192108934\\
568.01	0.00220062972740853\\
569.01	0.0021433998724014\\
570.01	0.00208515627504275\\
571.01	0.0020258940998783\\
572.01	0.00196560990176544\\
573.01	0.00190430178172618\\
574.01	0.00184196955411858\\
575.01	0.00177861492514706\\
576.01	0.00171424168248111\\
577.01	0.00164885589541889\\
578.01	0.00158246612459554\\
579.01	0.00151508363966999\\
580.01	0.00144672264269536\\
581.01	0.00137740049394915\\
582.01	0.0013071379358148\\
583.01	0.00123595930881451\\
584.01	0.00116389275200805\\
585.01	0.00109097037760846\\
586.01	0.00101722840670688\\
587.01	0.000942707249300885\\
588.01	0.000867451507214028\\
589.01	0.000791509872763252\\
590.01	0.000714934888910885\\
591.01	0.000637782527808906\\
592.01	0.000560111533702402\\
593.01	0.000481982462622189\\
594.01	0.000403456334556148\\
595.01	0.000324592793109807\\
596.01	0.000245447642136467\\
597.01	0.00016636800413272\\
598.01	9.17366970063782e-05\\
599.01	2.94669364271135e-05\\
599.02	2.89574269585688e-05\\
599.03	2.84509530852801e-05\\
599.04	2.79475443462508e-05\\
599.05	2.7447230571279e-05\\
599.06	2.69500418838293e-05\\
599.07	2.64560087039432e-05\\
599.08	2.59651617511691e-05\\
599.09	2.54775320475218e-05\\
599.1	2.49931509204854e-05\\
599.11	2.45120500060175e-05\\
599.12	2.40342612516185e-05\\
599.13	2.35598169193996e-05\\
599.14	2.30887495892024e-05\\
599.15	2.26210921617405e-05\\
599.16	2.215687786177e-05\\
599.17	2.16961402412976e-05\\
599.18	2.12389131828191e-05\\
599.19	2.07852309025824e-05\\
599.2	2.03351279538938e-05\\
599.21	1.98886392304542e-05\\
599.22	1.94457999697188e-05\\
599.23	1.90066457563046e-05\\
599.24	1.85712125254194e-05\\
599.25	1.81395365663351e-05\\
599.26	1.77116564095865e-05\\
599.27	1.72876122135658e-05\\
599.28	1.68674445361946e-05\\
599.29	1.64511943388859e-05\\
599.3	1.60389029905273e-05\\
599.31	1.56306122715087e-05\\
599.32	1.52263643777902e-05\\
599.33	1.48262019250105e-05\\
599.34	1.44301679526268e-05\\
599.35	1.40383059281154e-05\\
599.36	1.36506597511916e-05\\
599.37	1.32672737580847e-05\\
599.38	1.28881927258552e-05\\
599.39	1.25134618767404e-05\\
599.4	1.21431268825696e-05\\
599.41	1.17772338691924e-05\\
599.42	1.14158294209719e-05\\
599.43	1.10589605853174e-05\\
599.44	1.07066748772523e-05\\
599.45	1.03590202840433e-05\\
599.46	1.00160452698606e-05\\
599.47	9.67779878049101e-06\\
599.48	9.34433024810284e-06\\
599.49	9.01568959604283e-06\\
599.5	8.69192724369146e-06\\
599.51	8.3730941113757e-06\\
599.52	8.05924162529219e-06\\
599.53	7.75042172253965e-06\\
599.54	7.4466868561357e-06\\
599.55	7.14809000013257e-06\\
599.56	6.85468465475535e-06\\
599.57	6.56652485161134e-06\\
599.58	6.2836651589307e-06\\
599.59	6.00616068686249e-06\\
599.6	5.73406709284546e-06\\
599.61	5.46744058700296e-06\\
599.62	5.20633793760217e-06\\
599.63	4.95081647657741e-06\\
599.64	4.70093410508653e-06\\
599.65	4.45674929914347e-06\\
599.66	4.21832111529262e-06\\
599.67	3.98570919634376e-06\\
599.68	3.75897377716435e-06\\
599.69	3.53817569053241e-06\\
599.7	3.32337637303295e-06\\
599.71	3.11463787103054e-06\\
599.72	2.91202284668363e-06\\
599.73	2.71559458404555e-06\\
599.74	2.52541699518292e-06\\
599.75	2.34155462640155e-06\\
599.76	2.16407266449489e-06\\
599.77	1.99303694307928e-06\\
599.78	1.82851394897598e-06\\
599.79	1.67057082867475e-06\\
599.8	1.51927539483211e-06\\
599.81	1.37469613287027e-06\\
599.82	1.23690220760544e-06\\
599.83	1.10596346997172e-06\\
599.84	9.81950463784659e-07\\
599.85	8.64934432591793e-07\\
599.86	7.54987326588921e-07\\
599.87	6.52181809583305e-07\\
599.88	5.56591266062667e-07\\
599.89	4.68289808295413e-07\\
599.9	3.87352283521061e-07\\
599.91	3.13854281218731e-07\\
599.92	2.47872140425945e-07\\
599.93	1.8948295714763e-07\\
599.94	1.38764591834512e-07\\
599.95	9.57956769204876e-08\\
599.96	6.06556244606149e-08\\
599.97	3.34246338194039e-08\\
599.98	1.4183699454523e-08\\
599.99	3.014618757749e-09\\
600	0\\
};
\addplot [color=black,solid,forget plot]
  table[row sep=crcr]{%
0.01	0.00507509332097565\\
1.01	0.0050750923628945\\
2.01	0.00507509138515603\\
3.01	0.00507509038735957\\
4.01	0.00507508936909632\\
5.01	0.00507508832994861\\
6.01	0.00507508726949108\\
7.01	0.00507508618728923\\
8.01	0.00507508508290031\\
9.01	0.00507508395587187\\
10.01	0.00507508280574299\\
11.01	0.00507508163204302\\
12.01	0.00507508043429207\\
13.01	0.0050750792120006\\
14.01	0.00507507796466875\\
15.01	0.00507507669178709\\
16.01	0.00507507539283559\\
17.01	0.00507507406728384\\
18.01	0.00507507271459088\\
19.01	0.00507507133420444\\
20.01	0.00507506992556135\\
21.01	0.00507506848808715\\
22.01	0.00507506702119588\\
23.01	0.0050750655242895\\
24.01	0.00507506399675818\\
25.01	0.00507506243797951\\
26.01	0.00507506084731916\\
27.01	0.00507505922412902\\
28.01	0.0050750575677491\\
29.01	0.00507505587750485\\
30.01	0.00507505415270949\\
31.01	0.0050750523926618\\
32.01	0.0050750505966461\\
33.01	0.00507504876393276\\
34.01	0.00507504689377778\\
35.01	0.0050750449854215\\
36.01	0.00507504303808951\\
37.01	0.00507504105099175\\
38.01	0.00507503902332194\\
39.01	0.00507503695425819\\
40.01	0.00507503484296174\\
41.01	0.00507503268857721\\
42.01	0.00507503049023215\\
43.01	0.00507502824703633\\
44.01	0.00507502595808177\\
45.01	0.0050750236224424\\
46.01	0.00507502123917342\\
47.01	0.00507501880731144\\
48.01	0.00507501632587355\\
49.01	0.00507501379385709\\
50.01	0.0050750112102396\\
51.01	0.00507500857397792\\
52.01	0.00507500588400792\\
53.01	0.00507500313924481\\
54.01	0.00507500033858133\\
55.01	0.00507499748088886\\
56.01	0.00507499456501574\\
57.01	0.00507499158978734\\
58.01	0.00507498855400582\\
59.01	0.00507498545644934\\
60.01	0.00507498229587151\\
61.01	0.0050749790710015\\
62.01	0.0050749757805424\\
63.01	0.00507497242317229\\
64.01	0.00507496899754251\\
65.01	0.00507496550227766\\
66.01	0.00507496193597477\\
67.01	0.00507495829720301\\
68.01	0.00507495458450303\\
69.01	0.00507495079638682\\
70.01	0.00507494693133646\\
71.01	0.00507494298780371\\
72.01	0.00507493896420967\\
73.01	0.00507493485894434\\
74.01	0.00507493067036567\\
75.01	0.00507492639679894\\
76.01	0.0050749220365359\\
77.01	0.00507491758783472\\
78.01	0.00507491304891886\\
79.01	0.00507490841797665\\
80.01	0.00507490369316095\\
81.01	0.00507489887258731\\
82.01	0.00507489395433428\\
83.01	0.00507488893644271\\
84.01	0.00507488381691369\\
85.01	0.00507487859370976\\
86.01	0.00507487326475279\\
87.01	0.00507486782792364\\
88.01	0.00507486228106111\\
89.01	0.00507485662196151\\
90.01	0.00507485084837759\\
91.01	0.00507484495801778\\
92.01	0.00507483894854511\\
93.01	0.00507483281757652\\
94.01	0.00507482656268228\\
95.01	0.00507482018138451\\
96.01	0.00507481367115676\\
97.01	0.00507480702942243\\
98.01	0.00507480025355429\\
99.01	0.005074793340874\\
100.01	0.00507478628864997\\
101.01	0.00507477909409691\\
102.01	0.00507477175437477\\
103.01	0.00507476426658802\\
104.01	0.00507475662778401\\
105.01	0.00507474883495236\\
106.01	0.0050747408850232\\
107.01	0.00507473277486687\\
108.01	0.00507472450129263\\
109.01	0.00507471606104641\\
110.01	0.00507470745081088\\
111.01	0.00507469866720385\\
112.01	0.00507468970677664\\
113.01	0.00507468056601352\\
114.01	0.00507467124132977\\
115.01	0.00507466172907038\\
116.01	0.00507465202550914\\
117.01	0.0050746421268475\\
118.01	0.00507463202921244\\
119.01	0.00507462172865515\\
120.01	0.00507461122115039\\
121.01	0.00507460050259413\\
122.01	0.00507458956880223\\
123.01	0.00507457841550933\\
124.01	0.00507456703836704\\
125.01	0.00507455543294281\\
126.01	0.00507454359471697\\
127.01	0.00507453151908264\\
128.01	0.00507451920134384\\
129.01	0.00507450663671286\\
130.01	0.0050744938203094\\
131.01	0.00507448074715839\\
132.01	0.00507446741218821\\
133.01	0.0050744538102292\\
134.01	0.00507443993601179\\
135.01	0.00507442578416418\\
136.01	0.00507441134921064\\
137.01	0.00507439662557017\\
138.01	0.00507438160755303\\
139.01	0.00507436628936035\\
140.01	0.00507435066508125\\
141.01	0.00507433472869107\\
142.01	0.00507431847404841\\
143.01	0.00507430189489389\\
144.01	0.00507428498484734\\
145.01	0.00507426773740678\\
146.01	0.00507425014594418\\
147.01	0.00507423220370468\\
148.01	0.00507421390380356\\
149.01	0.00507419523922374\\
150.01	0.00507417620281398\\
151.01	0.00507415678728538\\
152.01	0.00507413698520982\\
153.01	0.00507411678901688\\
154.01	0.00507409619099173\\
155.01	0.005074075183271\\
156.01	0.00507405375784154\\
157.01	0.0050740319065375\\
158.01	0.00507400962103689\\
159.01	0.00507398689285877\\
160.01	0.00507396371336092\\
161.01	0.0050739400737363\\
162.01	0.00507391596501045\\
163.01	0.00507389137803769\\
164.01	0.00507386630349933\\
165.01	0.00507384073189902\\
166.01	0.00507381465356017\\
167.01	0.00507378805862289\\
168.01	0.00507376093704049\\
169.01	0.00507373327857531\\
170.01	0.00507370507279723\\
171.01	0.00507367630907735\\
172.01	0.0050736469765867\\
173.01	0.00507361706429175\\
174.01	0.00507358656095062\\
175.01	0.00507355545510958\\
176.01	0.00507352373509888\\
177.01	0.00507349138902891\\
178.01	0.00507345840478705\\
179.01	0.00507342477003229\\
180.01	0.00507339047219217\\
181.01	0.00507335549845812\\
182.01	0.00507331983578156\\
183.01	0.00507328347086898\\
184.01	0.00507324639017846\\
185.01	0.00507320857991437\\
186.01	0.00507317002602343\\
187.01	0.00507313071418995\\
188.01	0.00507309062983087\\
189.01	0.00507304975809168\\
190.01	0.00507300808384057\\
191.01	0.00507296559166482\\
192.01	0.00507292226586437\\
193.01	0.00507287809044797\\
194.01	0.00507283304912774\\
195.01	0.00507278712531304\\
196.01	0.00507274030210645\\
197.01	0.00507269256229743\\
198.01	0.00507264388835758\\
199.01	0.00507259426243429\\
200.01	0.00507254366634564\\
201.01	0.00507249208157421\\
202.01	0.00507243948926193\\
203.01	0.00507238587020346\\
204.01	0.00507233120484011\\
205.01	0.00507227547325413\\
206.01	0.00507221865516198\\
207.01	0.00507216072990919\\
208.01	0.00507210167646214\\
209.01	0.00507204147340324\\
210.01	0.00507198009892319\\
211.01	0.00507191753081424\\
212.01	0.00507185374646416\\
213.01	0.00507178872284889\\
214.01	0.00507172243652521\\
215.01	0.00507165486362391\\
216.01	0.00507158597984246\\
217.01	0.0050715157604375\\
218.01	0.00507144418021761\\
219.01	0.00507137121353524\\
220.01	0.0050712968342801\\
221.01	0.00507122101586983\\
222.01	0.00507114373124269\\
223.01	0.00507106495284985\\
224.01	0.005070984652647\\
225.01	0.00507090280208587\\
226.01	0.00507081937210558\\
227.01	0.00507073433312447\\
228.01	0.00507064765503149\\
229.01	0.00507055930717735\\
230.01	0.00507046925836458\\
231.01	0.00507037747684004\\
232.01	0.00507028393028489\\
233.01	0.00507018858580489\\
234.01	0.00507009140992138\\
235.01	0.00506999236856199\\
236.01	0.00506989142705026\\
237.01	0.00506978855009589\\
238.01	0.00506968370178452\\
239.01	0.00506957684556842\\
240.01	0.00506946794425484\\
241.01	0.00506935695999656\\
242.01	0.00506924385428079\\
243.01	0.00506912858791837\\
244.01	0.0050690111210332\\
245.01	0.00506889141305054\\
246.01	0.00506876942268595\\
247.01	0.00506864510793435\\
248.01	0.005068518426058\\
249.01	0.00506838933357479\\
250.01	0.00506825778624604\\
251.01	0.00506812373906494\\
252.01	0.00506798714624405\\
253.01	0.00506784796120308\\
254.01	0.00506770613655556\\
255.01	0.00506756162409706\\
256.01	0.00506741437479153\\
257.01	0.00506726433875824\\
258.01	0.00506711146525844\\
259.01	0.00506695570268196\\
260.01	0.00506679699853297\\
261.01	0.00506663529941704\\
262.01	0.00506647055102603\\
263.01	0.00506630269812392\\
264.01	0.00506613168453275\\
265.01	0.00506595745311698\\
266.01	0.00506577994576957\\
267.01	0.00506559910339558\\
268.01	0.00506541486589755\\
269.01	0.00506522717215955\\
270.01	0.00506503596003048\\
271.01	0.00506484116630971\\
272.01	0.00506464272672869\\
273.01	0.00506444057593556\\
274.01	0.00506423464747703\\
275.01	0.00506402487378258\\
276.01	0.00506381118614522\\
277.01	0.00506359351470526\\
278.01	0.00506337178843148\\
279.01	0.00506314593510227\\
280.01	0.00506291588128722\\
281.01	0.00506268155232887\\
282.01	0.00506244287232206\\
283.01	0.00506219976409476\\
284.01	0.00506195214918757\\
285.01	0.00506169994783382\\
286.01	0.00506144307893834\\
287.01	0.00506118146005622\\
288.01	0.0050609150073707\\
289.01	0.00506064363567161\\
290.01	0.00506036725833246\\
291.01	0.00506008578728751\\
292.01	0.00505979913300801\\
293.01	0.00505950720447802\\
294.01	0.00505920990917059\\
295.01	0.00505890715302186\\
296.01	0.00505859884040531\\
297.01	0.00505828487410596\\
298.01	0.00505796515529287\\
299.01	0.00505763958349196\\
300.01	0.00505730805655781\\
301.01	0.00505697047064422\\
302.01	0.00505662672017483\\
303.01	0.00505627669781245\\
304.01	0.00505592029442809\\
305.01	0.00505555739906764\\
306.01	0.00505518789892074\\
307.01	0.00505481167928527\\
308.01	0.00505442862353273\\
309.01	0.00505403861307289\\
310.01	0.00505364152731531\\
311.01	0.00505323724363369\\
312.01	0.00505282563732405\\
313.01	0.00505240658156526\\
314.01	0.00505197994737795\\
315.01	0.00505154560358038\\
316.01	0.0050511034167442\\
317.01	0.00505065325114879\\
318.01	0.00505019496873414\\
319.01	0.00504972842905117\\
320.01	0.00504925348921163\\
321.01	0.00504877000383523\\
322.01	0.00504827782499646\\
323.01	0.00504777680216768\\
324.01	0.00504726678216167\\
325.01	0.00504674760907182\\
326.01	0.00504621912421047\\
327.01	0.00504568116604355\\
328.01	0.00504513357012616\\
329.01	0.00504457616903229\\
330.01	0.00504400879228462\\
331.01	0.00504343126628027\\
332.01	0.00504284341421566\\
333.01	0.00504224505600726\\
334.01	0.00504163600820968\\
335.01	0.00504101608393246\\
336.01	0.00504038509275196\\
337.01	0.00503974284062201\\
338.01	0.00503908912978047\\
339.01	0.00503842375865283\\
340.01	0.00503774652175355\\
341.01	0.00503705720958282\\
342.01	0.00503635560852107\\
343.01	0.00503564150071973\\
344.01	0.00503491466398973\\
345.01	0.00503417487168396\\
346.01	0.00503342189258006\\
347.01	0.0050326554907563\\
348.01	0.00503187542546633\\
349.01	0.00503108145101045\\
350.01	0.00503027331660137\\
351.01	0.0050294507662298\\
352.01	0.00502861353852478\\
353.01	0.00502776136661102\\
354.01	0.00502689397796417\\
355.01	0.00502601109426194\\
356.01	0.00502511243123333\\
357.01	0.00502419769850433\\
358.01	0.00502326659944203\\
359.01	0.00502231883099567\\
360.01	0.00502135408353597\\
361.01	0.00502037204069237\\
362.01	0.00501937237918879\\
363.01	0.00501835476867738\\
364.01	0.00501731887157184\\
365.01	0.00501626434288031\\
366.01	0.00501519083003552\\
367.01	0.00501409797272657\\
368.01	0.00501298540273072\\
369.01	0.00501185274374292\\
370.01	0.00501069961120957\\
371.01	0.0050095256121601\\
372.01	0.00500833034503979\\
373.01	0.00500711339954497\\
374.01	0.00500587435645806\\
375.01	0.00500461278748491\\
376.01	0.00500332825509143\\
377.01	0.00500202031234377\\
378.01	0.00500068850274819\\
379.01	0.00499933236009091\\
380.01	0.00499795140827909\\
381.01	0.00499654516118146\\
382.01	0.00499511312246765\\
383.01	0.00499365478544587\\
384.01	0.00499216963289657\\
385.01	0.00499065713690462\\
386.01	0.00498911675868314\\
387.01	0.00498754794839435\\
388.01	0.00498595014495849\\
389.01	0.0049843227758546\\
390.01	0.00498266525690864\\
391.01	0.00498097699207024\\
392.01	0.0049792573731698\\
393.01	0.00497750577966067\\
394.01	0.00497572157834173\\
395.01	0.0049739041230573\\
396.01	0.00497205275437444\\
397.01	0.00497016679923682\\
398.01	0.00496824557059137\\
399.01	0.0049662883669906\\
400.01	0.00496429447216763\\
401.01	0.00496226315458532\\
402.01	0.00496019366695966\\
403.01	0.0049580852457607\\
404.01	0.0049559371106904\\
405.01	0.00495374846414219\\
406.01	0.00495151849064386\\
407.01	0.00494924635628814\\
408.01	0.00494693120815443\\
409.01	0.00494457217372436\\
410.01	0.00494216836029669\\
411.01	0.0049397188544016\\
412.01	0.00493722272122053\\
413.01	0.00493467900400838\\
414.01	0.00493208672352417\\
415.01	0.0049294448774652\\
416.01	0.00492675243990482\\
417.01	0.00492400836073521\\
418.01	0.00492121156510803\\
419.01	0.00491836095287491\\
420.01	0.00491545539802683\\
421.01	0.00491249374812972\\
422.01	0.00490947482375937\\
423.01	0.00490639741793495\\
424.01	0.00490326029554925\\
425.01	0.00490006219279999\\
426.01	0.00489680181662181\\
427.01	0.0048934778441154\\
428.01	0.00489008892197846\\
429.01	0.00488663366593589\\
430.01	0.00488311066016935\\
431.01	0.00487951845674879\\
432.01	0.0048758555750596\\
433.01	0.00487212050123231\\
434.01	0.00486831168756726\\
435.01	0.00486442755195876\\
436.01	0.00486046647731391\\
437.01	0.00485642681096679\\
438.01	0.00485230686408699\\
439.01	0.00484810491107931\\
440.01	0.00484381918897482\\
441.01	0.00483944789680955\\
442.01	0.00483498919499099\\
443.01	0.00483044120464899\\
444.01	0.0048258020069697\\
445.01	0.00482106964251116\\
446.01	0.00481624211049723\\
447.01	0.00481131736808979\\
448.01	0.00480629332963679\\
449.01	0.00480116786589522\\
450.01	0.00479593880322778\\
451.01	0.00479060392277257\\
452.01	0.00478516095958617\\
453.01	0.00477960760175957\\
454.01	0.00477394148950731\\
455.01	0.00476816021423241\\
456.01	0.0047622613175661\\
457.01	0.00475624229038595\\
458.01	0.00475010057181329\\
459.01	0.00474383354819188\\
460.01	0.00473743855205174\\
461.01	0.00473091286105579\\
462.01	0.0047242536969366\\
463.01	0.00471745822441844\\
464.01	0.00471052355013044\\
465.01	0.00470344672150507\\
466.01	0.0046962247256663\\
467.01	0.00468885448830111\\
468.01	0.0046813328725142\\
469.01	0.00467365667766195\\
470.01	0.00466582263816132\\
471.01	0.00465782742227181\\
472.01	0.00464966763084477\\
473.01	0.0046413397960396\\
474.01	0.00463284038000248\\
475.01	0.00462416577350806\\
476.01	0.00461531229456245\\
477.01	0.00460627618696779\\
478.01	0.00459705361884733\\
479.01	0.00458764068113383\\
480.01	0.00457803338601985\\
481.01	0.00456822766537132\\
482.01	0.00455821936910527\\
483.01	0.00454800426353266\\
484.01	0.00453757802966714\\
485.01	0.00452693626150026\\
486.01	0.00451607446424404\\
487.01	0.00450498805254091\\
488.01	0.00449367234864202\\
489.01	0.00448212258055315\\
490.01	0.00447033388014911\\
491.01	0.00445830128125419\\
492.01	0.00444601971769131\\
493.01	0.00443348402129692\\
494.01	0.00442068891990277\\
495.01	0.00440762903528331\\
496.01	0.00439429888107034\\
497.01	0.0043806928606338\\
498.01	0.0043668052649308\\
499.01	0.00435263027032252\\
500.01	0.0043381619363625\\
501.01	0.00432339420355518\\
502.01	0.00430832089108876\\
503.01	0.00429293569454225\\
504.01	0.00427723218357085\\
505.01	0.0042612037995689\\
506.01	0.00424484385331487\\
507.01	0.00422814552259942\\
508.01	0.0042111018498394\\
509.01	0.004193705739681\\
510.01	0.00417594995659516\\
511.01	0.00415782712246773\\
512.01	0.00413932971419067\\
513.01	0.00412045006125712\\
514.01	0.0041011803433656\\
515.01	0.00408151258804065\\
516.01	0.00406143866827475\\
517.01	0.00404095030019947\\
518.01	0.00402003904079373\\
519.01	0.00399869628563757\\
520.01	0.00397691326672204\\
521.01	0.0039546810503256\\
522.01	0.00393199053496824\\
523.01	0.00390883244945772\\
524.01	0.00388519735104193\\
525.01	0.00386107562368351\\
526.01	0.00383645747647488\\
527.01	0.00381133294221396\\
528.01	0.00378569187616113\\
529.01	0.00375952395500273\\
530.01	0.00373281867604811\\
531.01	0.00370556535668826\\
532.01	0.00367775313414919\\
533.01	0.00364937096557665\\
534.01	0.00362040762848954\\
535.01	0.00359085172164746\\
536.01	0.00356069166637926\\
537.01	0.00352991570842443\\
538.01	0.00349851192034694\\
539.01	0.0034664682045829\\
540.01	0.00343377229719395\\
541.01	0.00340041177240193\\
542.01	0.00336637404798946\\
543.01	0.00333164639165858\\
544.01	0.00329621592845059\\
545.01	0.00326006964933632\\
546.01	0.00322319442110026\\
547.01	0.00318557699765106\\
548.01	0.0031472040329063\\
549.01	0.00310806209541009\\
550.01	0.00306813768485848\\
551.01	0.00302741725072389\\
552.01	0.0029858872131864\\
553.01	0.00294353398659807\\
554.01	0.00290034400572704\\
555.01	0.00285630375504905\\
556.01	0.00281139980137633\\
557.01	0.00276561883013688\\
558.01	0.00271894768564463\\
559.01	0.00267137341572391\\
560.01	0.00262288332107934\\
561.01	0.00257346500983139\\
562.01	0.00252310645766198\\
563.01	0.00247179607404328\\
564.01	0.00241952277504812\\
565.01	0.00236627606326161\\
566.01	0.0023120461153342\\
567.01	0.0022568238777289\\
568.01	0.00220060117122217\\
569.01	0.00214337080471112\\
570.01	0.00208512669886227\\
571.01	0.00202586402009816\\
572.01	0.00196557932535577\\
573.01	0.00190427071795702\\
574.01	0.00184193801479703\\
575.01	0.00177858292487004\\
576.01	0.00171420923890147\\
577.01	0.00164882302952239\\
578.01	0.00158243286098413\\
579.01	0.00151505000684564\\
580.01	0.00144668867333727\\
581.01	0.00137736622517327\\
582.01	0.00130710340940336\\
583.01	0.00123592457139783\\
584.01	0.00116385785517657\\
585.01	0.00109093537792825\\
586.01	0.0010171933656031\\
587.01	0.0009426722327669\\
588.01	0.000867416585292753\\
589.01	0.000791475118736057\\
590.01	0.000714900378115771\\
591.01	0.000637748335993645\\
592.01	0.000560077734801639\\
593.01	0.000481949125828402\\
594.01	0.000403423520535174\\
595.01	0.000324560549193819\\
596.01	0.000245415996312425\\
597.01	0.000166366053398785\\
598.01	9.17366970063782e-05\\
599.01	2.94669364271135e-05\\
599.02	2.89574269585705e-05\\
599.03	2.84509530852801e-05\\
599.04	2.79475443462508e-05\\
599.05	2.7447230571279e-05\\
599.06	2.69500418838293e-05\\
599.07	2.64560087039432e-05\\
599.08	2.59651617511691e-05\\
599.09	2.54775320475235e-05\\
599.1	2.49931509204836e-05\\
599.11	2.45120500060175e-05\\
599.12	2.40342612516167e-05\\
599.13	2.35598169193978e-05\\
599.14	2.30887495892024e-05\\
599.15	2.26210921617405e-05\\
599.16	2.215687786177e-05\\
599.17	2.16961402412976e-05\\
599.18	2.12389131828191e-05\\
599.19	2.07852309025824e-05\\
599.2	2.03351279538938e-05\\
599.21	1.98886392304524e-05\\
599.22	1.94457999697188e-05\\
599.23	1.90066457563063e-05\\
599.24	1.85712125254211e-05\\
599.25	1.81395365663334e-05\\
599.26	1.77116564095865e-05\\
599.27	1.72876122135658e-05\\
599.28	1.68674445361946e-05\\
599.29	1.64511943388859e-05\\
599.3	1.60389029905273e-05\\
599.31	1.56306122715087e-05\\
599.32	1.5226364377792e-05\\
599.33	1.48262019250105e-05\\
599.34	1.44301679526268e-05\\
599.35	1.40383059281154e-05\\
599.36	1.36506597511916e-05\\
599.37	1.32672737580847e-05\\
599.38	1.28881927258535e-05\\
599.39	1.25134618767404e-05\\
599.4	1.21431268825696e-05\\
599.41	1.17772338691924e-05\\
599.42	1.14158294209736e-05\\
599.43	1.10589605853174e-05\\
599.44	1.07066748772523e-05\\
599.45	1.03590202840433e-05\\
599.46	1.00160452698589e-05\\
599.47	9.67779878049101e-06\\
599.48	9.34433024810111e-06\\
599.49	9.0156895960411e-06\\
599.5	8.69192724369319e-06\\
599.51	8.37309411137396e-06\\
599.52	8.05924162529392e-06\\
599.53	7.75042172253965e-06\\
599.54	7.44668685613396e-06\\
599.55	7.14809000013084e-06\\
599.56	6.85468465475535e-06\\
599.57	6.56652485161308e-06\\
599.58	6.28366515892896e-06\\
599.59	6.00616068686249e-06\\
599.6	5.73406709284546e-06\\
599.61	5.46744058700296e-06\\
599.62	5.20633793760217e-06\\
599.63	4.95081647657741e-06\\
599.64	4.70093410508653e-06\\
599.65	4.45674929914174e-06\\
599.66	4.21832111529089e-06\\
599.67	3.98570919634203e-06\\
599.68	3.75897377716608e-06\\
599.69	3.53817569053415e-06\\
599.7	3.32337637303469e-06\\
599.71	3.11463787102881e-06\\
599.72	2.91202284668363e-06\\
599.73	2.71559458404382e-06\\
599.74	2.52541699518466e-06\\
599.75	2.34155462640329e-06\\
599.76	2.16407266449663e-06\\
599.77	1.99303694307928e-06\\
599.78	1.82851394897598e-06\\
599.79	1.67057082867302e-06\\
599.8	1.51927539483385e-06\\
599.81	1.37469613287027e-06\\
599.82	1.23690220760718e-06\\
599.83	1.10596346997172e-06\\
599.84	9.81950463782924e-07\\
599.85	8.64934432591793e-07\\
599.86	7.54987326587186e-07\\
599.87	6.5218180958504e-07\\
599.88	5.56591266064402e-07\\
599.89	4.68289808295413e-07\\
599.9	3.87352283521061e-07\\
599.91	3.13854281216996e-07\\
599.92	2.4787214042421e-07\\
599.93	1.89482957149364e-07\\
599.94	1.38764591832777e-07\\
599.95	9.57956769222224e-08\\
599.96	6.06556244606149e-08\\
599.97	3.34246338211386e-08\\
599.98	1.4183699454523e-08\\
599.99	3.01461875948372e-09\\
600	0\\
};
\end{axis}
\end{tikzpicture}%
  \caption{Continuous Time}
\end{subfigure}%
\hfill%
\begin{subfigure}{.45\linewidth}
  \centering
  \setlength\figureheight{\linewidth} 
  \setlength\figurewidth{\linewidth}
  \tikzsetnextfilename{dp_colorbar/dm_dscr_z8}
  % This file was created by matlab2tikz.
%
%The latest updates can be retrieved from
%  http://www.mathworks.com/matlabcentral/fileexchange/22022-matlab2tikz-matlab2tikz
%where you can also make suggestions and rate matlab2tikz.
%
\definecolor{mycolor1}{rgb}{1.00000,0.00000,1.00000}%
%
\begin{tikzpicture}

\begin{axis}[%
width=4.564in,
height=3.803in,
at={(1.067in,0.513in)},
scale only axis,
every outer x axis line/.append style={black},
every x tick label/.append style={font=\color{black}},
xmin=0,
xmax=100,
xlabel={Time},
every outer y axis line/.append style={black},
every y tick label/.append style={font=\color{black}},
ymin=0,
ymax=0.012,
ylabel={Depth $\delta$},
axis background/.style={fill=white},
title={Z=8},
axis x line*=bottom,
axis y line*=left,
legend style={legend cell align=left,align=left,draw=black}
]
\addplot [color=green,dashed]
  table[row sep=crcr]{%
1	0.00485945625675347\\
2	0.00487414302033052\\
3	0.00488936844963023\\
4	0.00490515306501651\\
5	0.00492151829663004\\
6	0.00493848654848153\\
7	0.00495608127189559\\
8	0.00497432704549034\\
9	0.0049932496508163\\
10	0.00501287614094745\\
11	0.00503323490651359\\
12	0.00505435575266095\\
13	0.00507627004656115\\
14	0.00509901110469301\\
15	0.00512261431087908\\
16	0.00514711692825322\\
17	0.00517255821158281\\
18	0.0051989795554708\\
19	0.00522642469838481\\
20	0.00525494031771952\\
21	0.00528457648587319\\
22	0.0053153870786072\\
23	0.00534742886513508\\
24	0.00538076028949751\\
25	0.0054154428220149\\
26	0.00545154020211779\\
27	0.00548911714246375\\
28	0.00552823129860305\\
29	0.00556894135895538\\
30	0.00561131173166151\\
31	0.00565541678573689\\
32	0.00570134233290593\\
33	0.00574918961030762\\
34	0.00579908018002111\\
35	0.005851164268531\\
36	0.00590565702760265\\
37	0.00596283833729376\\
38	0.00602291072182752\\
39	0.00608596820594332\\
40	0.00615208768624582\\
41	0.0062213147623831\\
42	0.00629362362321532\\
43	0.00636892654015613\\
44	0.00644722868781644\\
45	0.00652843920469626\\
46	0.00661232934652509\\
47	0.00669847046861788\\
48	0.00678615428450335\\
49	0.00687421974785775\\
50	0.00696067182424244\\
51	0.00704414543345934\\
52	0.00712427704869411\\
53	0.00720078458073156\\
54	0.00727326795226155\\
55	0.0073418697025762\\
56	0.00740757939773257\\
57	0.00747194638628693\\
58	0.00753616146270302\\
59	0.00760032443049048\\
60	0.00766456970083794\\
61	0.00772906707473294\\
62	0.00779402631784734\\
63	0.00785968044069715\\
64	0.00792626260473561\\
65	0.00799388326307111\\
66	0.00806258634939459\\
67	0.00813239846694426\\
68	0.00820281961516284\\
69	0.00827383024153404\\
70	0.00834539769073058\\
71	0.00841756757162911\\
72	0.00849054969104642\\
73	0.00856434016815509\\
74	0.00863893108773656\\
75	0.00871432309957891\\
76	0.00878982547481302\\
77	0.00886405134220572\\
78	0.00893671335213983\\
79	0.00900822354657033\\
80	0.00907847647648609\\
81	0.00914730827594116\\
82	0.00921440840949971\\
83	0.00928113934875132\\
84	0.00934779119356061\\
85	0.00941413354923462\\
86	0.0094798880781265\\
87	0.00954466555764447\\
88	0.00960799533874711\\
89	0.00966853872526029\\
90	0.00972565262017696\\
91	0.0097788671886327\\
92	0.0098278806934644\\
93	0.00987342441910115\\
94	0.00991480090004276\\
95	0.00995105460587982\\
96	0.00998065038037144\\
97	0.00999982980159223\\
98	0.01003324910555\\
99	0\\
100	0\\
};
\addlegendentry{$q=-4$};

\addplot [color=mycolor1,dashed]
  table[row sep=crcr]{%
1	0.00382394873317695\\
2	0.00384957111565333\\
3	0.00387611528866614\\
4	0.00390361249529857\\
5	0.0039320947298463\\
6	0.00396159469728134\\
7	0.00399214575585568\\
8	0.00402378183377548\\
9	0.00405653729889385\\
10	0.00409044670018183\\
11	0.0041255440385175\\
12	0.00416185989878204\\
13	0.00419940650464184\\
14	0.00423819597083052\\
15	0.00427825536483802\\
16	0.0043196096312672\\
17	0.00436228095842971\\
18	0.00440628812960201\\
19	0.00445164616064467\\
20	0.00449836727525809\\
21	0.00454646698192176\\
22	0.0045959897301589\\
23	0.00464704069773572\\
24	0.00469960624769317\\
25	0.00475365559312346\\
26	0.00480913602535157\\
27	0.00486596706169554\\
28	0.00492403367726455\\
29	0.00498317662968881\\
30	0.00504318022982533\\
31	0.00510375560939573\\
32	0.00516451886391788\\
33	0.00522496301408269\\
34	0.00528442237628088\\
35	0.00534202975236024\\
36	0.00539678989436052\\
37	0.00544837002297262\\
38	0.00550237551216994\\
39	0.0055589315185265\\
40	0.00561814043264873\\
41	0.00568009263451653\\
42	0.00574487272443725\\
43	0.00581257793596128\\
44	0.00588329827868223\\
45	0.00595711328056412\\
46	0.00603408787606269\\
47	0.00611426621538027\\
48	0.00619765173247439\\
49	0.0062841498977199\\
50	0.00637368789242256\\
51	0.00646622029331727\\
52	0.00656160117960447\\
53	0.00665954007715095\\
54	0.00675956807606328\\
55	0.0068609611061804\\
56	0.00696252346518636\\
57	0.00706227997466448\\
58	0.00715870268656942\\
59	0.00725154359529958\\
60	0.00734071115791681\\
61	0.00742600917742923\\
62	0.00750754664013129\\
63	0.00758595983147457\\
64	0.00766200587575343\\
65	0.00773753850067737\\
66	0.00781290984532616\\
67	0.0078881605406168\\
68	0.00796339019400955\\
69	0.00803874861910076\\
70	0.00811444087980434\\
71	0.00819062585220229\\
72	0.00826742321998385\\
73	0.00834501196542576\\
74	0.00842343896169291\\
75	0.0085027251122091\\
76	0.00858289961770601\\
77	0.00866399899362369\\
78	0.00874604952977821\\
79	0.00882844448104465\\
80	0.00891112708521404\\
81	0.00899409332830582\\
82	0.0090773568188397\\
83	0.00915945101398878\\
84	0.009239854675922\\
85	0.00931847594409736\\
86	0.00939551870782303\\
87	0.00947072995214786\\
88	0.00954351336420434\\
89	0.00961296437622768\\
90	0.00967974901988489\\
91	0.00974411148075818\\
92	0.0098051085698454\\
93	0.00986066688361775\\
94	0.00990963950258614\\
95	0.00995008790550502\\
96	0.00998057518199083\\
97	0.00999982980159223\\
98	0.01003324910555\\
99	0\\
100	0\\
};
\addlegendentry{$q=-3$};

\addplot [color=red,dashed]
  table[row sep=crcr]{%
1	0.00203625637388802\\
2	0.00205694682399506\\
3	0.00207847386511244\\
4	0.00210087662914663\\
5	0.00212419664426941\\
6	0.00214847803912337\\
7	0.00217376776201011\\
8	0.00220011580521964\\
9	0.00222757541446491\\
10	0.00225620326863204\\
11	0.00228605967015089\\
12	0.00231720809827092\\
13	0.00234971785761731\\
14	0.00238366585758567\\
15	0.00241913551019996\\
16	0.00245621760525557\\
17	0.00249501140621176\\
18	0.00253562607725969\\
19	0.00257818256996886\\
20	0.00262281598900947\\
21	0.00266967873293951\\
22	0.00271894350197721\\
23	0.00277079045151589\\
24	0.00282542156688777\\
25	0.00288306442435626\\
26	0.00294397666663905\\
27	0.00300845123391652\\
28	0.00307682238123111\\
29	0.00314947455760223\\
30	0.00322683619825089\\
31	0.0033094322528504\\
32	0.00339788772366867\\
33	0.003492941051719\\
34	0.00359546346540763\\
35	0.00370646241761729\\
36	0.00377583934464846\\
37	0.00385317360833428\\
38	0.00393324370364177\\
39	0.0040161639545401\\
40	0.00410210460056693\\
41	0.00419125142992333\\
42	0.00428368399359528\\
43	0.00437941286843043\\
44	0.0044784804383629\\
45	0.0045809385558749\\
46	0.004686820361371\\
47	0.00479613230001219\\
48	0.00490884668681802\\
49	0.00502491892166917\\
50	0.00514431096295905\\
51	0.00526709981183514\\
52	0.00539322086161395\\
53	0.00552239190252712\\
54	0.00565414961068421\\
55	0.00578780616934352\\
56	0.00592231962241305\\
57	0.00605633882844064\\
58	0.00618818471261372\\
59	0.0063156027855923\\
60	0.00643720690847549\\
61	0.00656262720770935\\
62	0.00669131339088365\\
63	0.0068223129141614\\
64	0.00695409983086691\\
65	0.00708409457700272\\
66	0.00721129009152353\\
67	0.00733495466027041\\
68	0.00745411623004709\\
69	0.00756773135676126\\
70	0.00767512187930786\\
71	0.00777734561538589\\
72	0.00787627928465611\\
73	0.00797309686954974\\
74	0.00806886123862369\\
75	0.00816351733955394\\
76	0.00825716402085118\\
77	0.00835021851031631\\
78	0.00844302389172513\\
79	0.00853581364536144\\
80	0.00862862608066721\\
81	0.00872151736171665\\
82	0.00881453042058036\\
83	0.00890764015363901\\
84	0.00900077402725062\\
85	0.00909385280462877\\
86	0.00918659398642789\\
87	0.00927874325742761\\
88	0.00937027399818891\\
89	0.00946124141899078\\
90	0.0095502828761426\\
91	0.00963604389549451\\
92	0.00971781643994538\\
93	0.00979475046813423\\
94	0.00986566787131446\\
95	0.00992847447705118\\
96	0.00997640299214361\\
97	0.00999961158897599\\
98	0.01003324910555\\
99	0\\
100	0\\
};
\addlegendentry{$q=-2$};

\addplot [color=blue,dashed]
  table[row sep=crcr]{%
1	0.000107277425565821\\
2	0.00010829576775512\\
3	0.000109357246700352\\
4	0.000110464014507547\\
5	0.000111618360836398\\
6	0.000112822727060759\\
7	0.000114079725526992\\
8	0.000115392166628401\\
9	0.000116763089083499\\
10	0.000118195743580936\\
11	0.000119693361035447\\
12	0.000121259375659868\\
13	0.000122897678162007\\
14	0.000124612447359982\\
15	0.000126408170924103\\
16	0.000128289659307511\\
17	0.000130262047800479\\
18	0.000132330798449739\\
19	0.000134501802044341\\
20	0.000136781921283393\\
21	0.000139179763237535\\
22	0.000141703525482586\\
23	0.000144362268124243\\
24	0.000147166064002637\\
25	0.000150126205431619\\
26	0.000153255505478384\\
27	0.000156568692226079\\
28	0.000160082524410796\\
29	0.000163812506886159\\
30	0.00016777896427553\\
31	0.000172006473451556\\
32	0.000176524358632362\\
33	0.000181369318582451\\
34	0.000186592626833443\\
35	0.000192285669283197\\
36	0.00024942253577002\\
37	0.000309145709392737\\
38	0.000371348848368564\\
39	0.00043620111965971\\
40	0.000503882269590036\\
41	0.00057457384528812\\
42	0.00064848171472686\\
43	0.000725843514025343\\
44	0.000806928536617441\\
45	0.000892039518267811\\
46	0.000981518305163587\\
47	0.00107575357524396\\
48	0.0011751923918723\\
49	0.00128035430398567\\
50	0.00139184649354964\\
51	0.00151034982277297\\
52	0.00163648204936279\\
53	0.00177110975924253\\
54	0.00191537073237776\\
55	0.00207046770012352\\
56	0.0022378391613308\\
57	0.00241919027646704\\
58	0.00261652910209623\\
59	0.00283222966545185\\
60	0.00306752599786927\\
61	0.00331283086515275\\
62	0.0035685996048372\\
63	0.00383519421471796\\
64	0.00411302621380756\\
65	0.00440274342922439\\
66	0.00469935368877752\\
67	0.00500583342817488\\
68	0.00532532718734629\\
69	0.00565876112690701\\
70	0.00593065199045151\\
71	0.00613615367870856\\
72	0.00634160416831506\\
73	0.00654370918544362\\
74	0.00673930839805359\\
75	0.00692640629703696\\
76	0.00710211369599329\\
77	0.00726192851247405\\
78	0.00741896330693372\\
79	0.00757405866803044\\
80	0.00772673960027738\\
81	0.0078768892415683\\
82	0.00802476112604144\\
83	0.00817170332587985\\
84	0.00831818108566537\\
85	0.00846462780503413\\
86	0.00861055071139941\\
87	0.00875558871699461\\
88	0.00889949761240524\\
89	0.00904149002310103\\
90	0.00918069225954357\\
91	0.00931615360006014\\
92	0.00944684615622117\\
93	0.00957168006435625\\
94	0.00968954914319653\\
95	0.00979940168610245\\
96	0.00989957393262904\\
97	0.00998069825602133\\
98	0.01003324910555\\
99	0\\
100	0\\
};
\addlegendentry{$q=-1$};

\addplot [color=black,solid]
  table[row sep=crcr]{%
1	0.000315132141961445\\
2	0.000315132141961445\\
3	0.000315132141961445\\
4	0.000315132141961445\\
5	0.000315132141961445\\
6	0.000315132141961445\\
7	0.000315132141961445\\
8	0.000315132141961445\\
9	0.000315132141961445\\
10	0.000315132141961445\\
11	0.000315132141961445\\
12	0.000315132141961445\\
13	0.000315132141961445\\
14	0.000315132141961445\\
15	0.000315132141961445\\
16	0.000315132141961445\\
17	0.000315132141961445\\
18	0.000315132141961445\\
19	0.000315132141961445\\
20	0.000315132141961445\\
21	0.000315132141961445\\
22	0.000315132141961445\\
23	0.000315132141961445\\
24	0.000315132141961445\\
25	0.000315132141961445\\
26	0.000315132141961445\\
27	0.000315132141961445\\
28	0.000315132141961445\\
29	0.000315132141961445\\
30	0.000315132141961445\\
31	0.000315132141961445\\
32	0.000315132141961445\\
33	0.000315132141961445\\
34	0.000315132141961445\\
35	0.000315132141961445\\
36	0.000315132141961445\\
37	0.000315132141961445\\
38	0.000315132141961445\\
39	0.000315132141961445\\
40	0.000315132141961445\\
41	0.000315132141961445\\
42	0.000315132141961445\\
43	0.000315132141961445\\
44	0.000315132141961445\\
45	0.000315132141961445\\
46	0.000315132141961445\\
47	0.000315132141961445\\
48	0.000315132141961445\\
49	0.000315132141961445\\
50	0.000315132141961445\\
51	0.000315132141961445\\
52	0.00031527825582075\\
53	0.00031555428017459\\
54	0.000315849297560266\\
55	0.000316166125854495\\
56	0.000316507903973797\\
57	0.000316878207644637\\
58	0.000317281267679093\\
59	0.000317722098240475\\
60	0.000318206664581177\\
61	0.000318742094533239\\
62	0.000319336814178377\\
63	0.000320000854997598\\
64	0.000320746086345837\\
65	0.000321586520854692\\
66	0.000327984279529535\\
67	0.000337550585480914\\
68	0.00034797886066513\\
69	0.000359407450804621\\
70	0.000446271320100876\\
71	0.000611671498222364\\
72	0.000790051572150104\\
73	0.000983252379655302\\
74	0.00119225648897744\\
75	0.00141097874546424\\
76	0.00165196551615618\\
77	0.00191908662026477\\
78	0.00219924705869884\\
79	0.00249135195420683\\
80	0.00279464276648611\\
81	0.00310906201374306\\
82	0.00343464900349301\\
83	0.0037701413982495\\
84	0.004115488203838\\
85	0.00447109773233466\\
86	0.00483655278955692\\
87	0.00521198560788567\\
88	0.0055971677510235\\
89	0.00599221072788071\\
90	0.00639689196057587\\
91	0.00681115249885264\\
92	0.00723501370613978\\
93	0.00766825212328673\\
94	0.00811008000553154\\
95	0.0085593774169704\\
96	0.00901438057914194\\
97	0.00947206379995425\\
98	0.00992590828135393\\
99	0\\
100	0\\
};
\addlegendentry{$q=0$};

\addplot [color=blue,solid]
  table[row sep=crcr]{%
1	0.0100329692748296\\
2	0.0100329679438438\\
3	0.0100329665737844\\
4	0.0100329651631603\\
5	0.0100329637110275\\
6	0.0100329622162635\\
7	0.0100329606765574\\
8	0.010032959088764\\
9	0.0100329574488552\\
10	0.0100329557554473\\
11	0.0100329540110098\\
12	0.0100329522124844\\
13	0.010032950355083\\
14	0.0100329484286329\\
15	0.0100329464042331\\
16	0.0100329441798771\\
17	0.010032937934459\\
18	0.0100329249582351\\
19	0.01003291153935\\
20	0.0100328976522816\\
21	0.0100328832665241\\
22	0.0100328683403523\\
23	0.0100328528016462\\
24	0.0100328364917738\\
25	0.0100328190387565\\
26	0.0100327999966034\\
27	0.0100327802429298\\
28	0.0100327597344465\\
29	0.0100327384240231\\
30	0.0100327162602481\\
31	0.0100326931868577\\
32	0.0100326691418516\\
33	0.0100326440557504\\
34	0.0100326178472242\\
35	0.0100325904097408\\
36	0.0100325615631244\\
37	0.0100325308450213\\
38	0.0100324459543204\\
39	0.0100322214910233\\
40	0.0100319901987928\\
41	0.0100317511363865\\
42	0.0100315037528158\\
43	0.0100312475296138\\
44	0.010030981844076\\
45	0.0100307060610967\\
46	0.0100304193007889\\
47	0.0100301205485006\\
48	0.0100298086290865\\
49	0.0100294821760859\\
50	0.010029139593409\\
51	0.0100287789973107\\
52	0.0100283980729242\\
53	0.0100279935061273\\
54	0.0100275602526793\\
55	0.0100241158619052\\
56	0.0100199123580354\\
57	0.0100156641671768\\
58	0.0100113729002276\\
59	0.0100070372259259\\
60	0.0100026557934925\\
61	0.00999822745059608\\
62	0.0099937513919967\\
63	0.00998922738064705\\
64	0.00998465609120485\\
65	0.00998003948099702\\
66	0.00997460734725493\\
67	0.00996882978578957\\
68	0.00996295928471298\\
69	0.00995700828182999\\
70	0.00995115051904177\\
71	0.00993198468323465\\
72	0.00991179204224949\\
73	0.00989059140574361\\
74	0.00984791950631454\\
75	0.00962443667249636\\
76	0.00938631826922074\\
77	0.00913157865020818\\
78	0.0088578625968306\\
79	0.0085623508134461\\
80	0.0082418646173091\\
81	0.00789585237346233\\
82	0.00753610471222871\\
83	0.00716308979915167\\
84	0.00677636210378066\\
85	0.00637565744842168\\
86	0.00596114399340499\\
87	0.00553417378650248\\
88	0.00509423606601865\\
89	0.00464085403357268\\
90	0.00417361846537467\\
91	0.00369226049843334\\
92	0.00319748084124855\\
93	0.00269022915655069\\
94	0.00217160322227635\\
95	0.00164283062467496\\
96	0.00110643888137445\\
97	0.000567408530202185\\
98	3.32491055499654e-05\\
99	0\\
100	0\\
};
\addlegendentry{$q=1$};

\addplot [color=red,solid]
  table[row sep=crcr]{%
1	0.0100187058662163\\
2	0.0100185634774708\\
3	0.0100184167401897\\
4	0.0100182654762648\\
5	0.0100181095187468\\
6	0.0100179486955663\\
7	0.0100177827906476\\
8	0.0100176115426449\\
9	0.0100174346532581\\
10	0.0100172519066996\\
11	0.010017063213596\\
12	0.0100168682805138\\
13	0.0100166667049456\\
14	0.01001645788312\\
15	0.0100162405664583\\
16	0.0100160110702597\\
17	0.0100156438207477\\
18	0.0100150496634009\\
19	0.0100144342467735\\
20	0.0100137961191358\\
21	0.0100131335578139\\
22	0.0100124443370166\\
23	0.0100117250547756\\
24	0.0100109692032847\\
25	0.0100101627608925\\
26	0.0100090618504057\\
27	0.010006307215976\\
28	0.0100034896443133\\
29	0.0100006071590769\\
30	0.0099976576951472\\
31	0.00999463910218179\\
32	0.00999154914522113\\
33	0.00998838548652868\\
34	0.00998514559313926\\
35	0.00998182636436354\\
36	0.00997842262206861\\
37	0.00997492033449082\\
38	0.00996959056129169\\
39	0.00995959564403076\\
40	0.00994930541761534\\
41	0.00993870269318898\\
42	0.00992776844428659\\
43	0.00991648522397518\\
44	0.00990483303003661\\
45	0.00989278905253359\\
46	0.00988032052939186\\
47	0.00986738869252152\\
48	0.00985394858649275\\
49	0.00983994787608937\\
50	0.0098253253659455\\
51	0.00981000880022007\\
52	0.00979390964062349\\
53	0.00977690293422598\\
54	0.00975880086933501\\
55	0.00963403039976769\\
56	0.00947791061205357\\
57	0.00931475232544556\\
58	0.00914409334018826\\
59	0.00896527233445575\\
60	0.00877745433557311\\
61	0.00857967616664015\\
62	0.00837082206220613\\
63	0.00814959444555788\\
64	0.00791448054645849\\
65	0.00766370763610572\\
66	0.00739599210340636\\
67	0.0071083790920695\\
68	0.00679782284173191\\
69	0.00646119601050876\\
70	0.00610733115161804\\
71	0.00575262518513171\\
72	0.00538348894161203\\
73	0.00499873019055509\\
74	0.00461820533023241\\
75	0.00440450791375652\\
76	0.00418881841356822\\
77	0.00397235882276841\\
78	0.00375714245292512\\
79	0.00354664900457645\\
80	0.00334590530539139\\
81	0.0031572889882801\\
82	0.00296807243696754\\
83	0.00277736910648468\\
84	0.00258547823036496\\
85	0.00239267130567504\\
86	0.00219898690736822\\
87	0.00200317573278665\\
88	0.00180613206651937\\
89	0.00160871409912125\\
90	0.00141197570640385\\
91	0.00121677147072099\\
92	0.00102360464204711\\
93	0.000833141672008094\\
94	0.000646658220394614\\
95	0.000465861915824526\\
96	0.000292511834892549\\
97	0.00012959792500039\\
98	3.32491055499654e-05\\
99	0\\
100	0\\
};
\addlegendentry{$q=2$};

\addplot [color=mycolor1,solid]
  table[row sep=crcr]{%
1	0.00981860164552879\\
2	0.00981423735354565\\
3	0.00980974584851086\\
4	0.00980512224211095\\
5	0.0098003613757449\\
6	0.00979545810592951\\
7	0.00979040713881327\\
8	0.0097852025101279\\
9	0.0097798378894794\\
10	0.00977430665705891\\
11	0.00976860199779076\\
12	0.00976271640359355\\
13	0.00975664051604021\\
14	0.00975036421452315\\
15	0.00974387684648715\\
16	0.00973716875959182\\
17	0.00973035749903079\\
18	0.00972352322598488\\
19	0.00971643447848896\\
20	0.00970906919908387\\
21	0.00970140117514127\\
22	0.00969339953279936\\
23	0.00968502776192527\\
24	0.00967624238960126\\
25	0.00966698842236478\\
26	0.00964914125260309\\
27	0.00957274674989259\\
28	0.00949373601077428\\
29	0.00941198094100749\\
30	0.00932734405865233\\
31	0.00923967947397971\\
32	0.00914883301142084\\
33	0.0090546427260522\\
34	0.00895694005674083\\
35	0.00885555211217539\\
36	0.00875030643872931\\
37	0.00864104329795993\\
38	0.00852939674446276\\
39	0.00841815502170258\\
40	0.00830257718119142\\
41	0.00818242267763374\\
42	0.00805744616901231\\
43	0.00792740735238815\\
44	0.00779210186077065\\
45	0.00765128427700737\\
46	0.00750457647174612\\
47	0.00735143723509152\\
48	0.00719123946373411\\
49	0.00702325178588119\\
50	0.00684661528406853\\
51	0.00666031387530546\\
52	0.00646313762077005\\
53	0.00625364552867741\\
54	0.00603010672697628\\
55	0.00589894414857172\\
56	0.00578182254651292\\
57	0.00565159408914177\\
58	0.00551322823895752\\
59	0.00537184143962644\\
60	0.00522766866812109\\
61	0.00508104792573206\\
62	0.00493244932777306\\
63	0.00478250303054305\\
64	0.00463204718550654\\
65	0.00448222704670039\\
66	0.0043345728255961\\
67	0.00419112611387241\\
68	0.00405446500596207\\
69	0.00392790552331875\\
70	0.00380235521941472\\
71	0.00367457296327706\\
72	0.00354499292623339\\
73	0.00341423479445978\\
74	0.0032833149303601\\
75	0.00315159803372358\\
76	0.00302107454113035\\
77	0.00289199695409881\\
78	0.00276442662866803\\
79	0.00263800975207846\\
80	0.00251120114192371\\
81	0.00238203468679405\\
82	0.00225060925979982\\
83	0.00211690448951394\\
84	0.00198066918796629\\
85	0.00184174376140796\\
86	0.00170018422525516\\
87	0.00155632507068512\\
88	0.00141057483317854\\
89	0.00126294383989656\\
90	0.00111371401515057\\
91	0.000963208188100662\\
92	0.000811803711308885\\
93	0.00066005319163681\\
94	0.000509922248582303\\
95	0.0003623900613677\\
96	0.000222877382777326\\
97	0.0001131259618254\\
98	3.32491055499654e-05\\
99	0\\
100	0\\
};
\addlegendentry{$q=3$};

\addplot [color=green,solid]
  table[row sep=crcr]{%
1	0.00840850314635316\\
2	0.00836390536715778\\
3	0.00831801231030496\\
4	0.00827078217763323\\
5	0.0082221715057822\\
6	0.00817213507356084\\
7	0.0081206261136224\\
8	0.00806759381786456\\
9	0.00801298003986253\\
10	0.00795672111441184\\
11	0.00789874790436701\\
12	0.00783898535219299\\
13	0.00777735299267128\\
14	0.00771376447120918\\
15	0.00764812709959048\\
16	0.00758034147920659\\
17	0.00751029825504354\\
18	0.00743787883048643\\
19	0.00736296469479454\\
20	0.00728543408216624\\
21	0.00720513501774509\\
22	0.00712190115643649\\
23	0.00703555046877898\\
24	0.00694588526332273\\
25	0.0068526981975587\\
26	0.00676406609247885\\
27	0.00673119653239127\\
28	0.00669706169314112\\
29	0.00666164813517396\\
30	0.0066249502220872\\
31	0.00658684636954535\\
32	0.00654719118990135\\
33	0.00650580937341389\\
34	0.00646248777838751\\
35	0.00641696515639261\\
36	0.00636891871664562\\
37	0.00631794634792286\\
38	0.00626352317128828\\
39	0.00620498797956653\\
40	0.00614165712404742\\
41	0.00607269254849635\\
42	0.00599706175588608\\
43	0.00591348891830688\\
44	0.0058204177575719\\
45	0.00572246012464679\\
46	0.00562288682616604\\
47	0.00552189755763095\\
48	0.00541976259189682\\
49	0.00531684291696741\\
50	0.00521361638470062\\
51	0.00511071176009554\\
52	0.00500895302494531\\
53	0.0049094164795163\\
54	0.00481350405759531\\
55	0.00472144456513887\\
56	0.00463472119686986\\
57	0.00455560038458885\\
58	0.00447917291126602\\
59	0.00440035548125985\\
60	0.00431910495801014\\
61	0.0042354110809133\\
62	0.00414929315612524\\
63	0.00406081832404118\\
64	0.00396998039615607\\
65	0.00387657656643398\\
66	0.00378054699318786\\
67	0.00368198916379013\\
68	0.0035808306845737\\
69	0.00347693953677897\\
70	0.00337084635289038\\
71	0.00326451824871775\\
72	0.00315937428408911\\
73	0.00305537219184628\\
74	0.00295213854074137\\
75	0.00284825403748608\\
76	0.00274209284132924\\
77	0.00263373892104426\\
78	0.00252305301400646\\
79	0.00240975409464889\\
80	0.00229343741535956\\
81	0.00217409690567752\\
82	0.00205173954490793\\
83	0.00192639870728597\\
84	0.00179814115984751\\
85	0.00166709702916354\\
86	0.001533442602948\\
87	0.00139862000009753\\
88	0.00126255262385255\\
89	0.00112557608806227\\
90	0.000988077718204886\\
91	0.000850509502407808\\
92	0.000713404021036185\\
93	0.00057879574612941\\
94	0.000447390496825131\\
95	0.000321946478778335\\
96	0.000210513689464598\\
97	0.0001131259618254\\
98	3.32491055499654e-05\\
99	0\\
100	0\\
};
\addlegendentry{$q=4$};

\end{axis}
\end{tikzpicture}% 
  \caption{Discrete Time}
\end{subfigure}\\
\vspace{1cm}
\begin{subfigure}{.45\linewidth}
  \centering
  \setlength\figureheight{\linewidth} 
  \setlength\figurewidth{\linewidth}
  \tikzsetnextfilename{dp_colorbar/dm_cts_nFPC_z8}
  % This file was created by matlab2tikz.
%
%The latest updates can be retrieved from
%  http://www.mathworks.com/matlabcentral/fileexchange/22022-matlab2tikz-matlab2tikz
%where you can also make suggestions and rate matlab2tikz.
%
\definecolor{mycolor1}{rgb}{0.00000,1.00000,0.14286}%
\definecolor{mycolor2}{rgb}{0.00000,1.00000,0.28571}%
\definecolor{mycolor3}{rgb}{0.00000,1.00000,0.42857}%
\definecolor{mycolor4}{rgb}{0.00000,1.00000,0.57143}%
\definecolor{mycolor5}{rgb}{0.00000,1.00000,0.71429}%
\definecolor{mycolor6}{rgb}{0.00000,1.00000,0.85714}%
\definecolor{mycolor7}{rgb}{0.00000,1.00000,1.00000}%
\definecolor{mycolor8}{rgb}{0.00000,0.87500,1.00000}%
\definecolor{mycolor9}{rgb}{0.00000,0.62500,1.00000}%
\definecolor{mycolor10}{rgb}{0.12500,0.00000,1.00000}%
\definecolor{mycolor11}{rgb}{0.25000,0.00000,1.00000}%
\definecolor{mycolor12}{rgb}{0.37500,0.00000,1.00000}%
\definecolor{mycolor13}{rgb}{0.50000,0.00000,1.00000}%
\definecolor{mycolor14}{rgb}{0.62500,0.00000,1.00000}%
\definecolor{mycolor15}{rgb}{0.75000,0.00000,1.00000}%
\definecolor{mycolor16}{rgb}{0.87500,0.00000,1.00000}%
\definecolor{mycolor17}{rgb}{1.00000,0.00000,1.00000}%
\definecolor{mycolor18}{rgb}{1.00000,0.00000,0.87500}%
\definecolor{mycolor19}{rgb}{1.00000,0.00000,0.62500}%
\definecolor{mycolor20}{rgb}{0.85714,0.00000,0.00000}%
\definecolor{mycolor21}{rgb}{0.71429,0.00000,0.00000}%
%
\begin{tikzpicture}

\begin{axis}[%
width=4.1in,
height=3.803in,
at={(0.809in,0.513in)},
scale only axis,
point meta min=0,
point meta max=1,
every outer x axis line/.append style={black},
every x tick label/.append style={font=\color{black}},
xmin=0,
xmax=600,
every outer y axis line/.append style={black},
every y tick label/.append style={font=\color{black}},
ymin=0,
ymax=0.012,
axis background/.style={fill=white},
axis x line*=bottom,
axis y line*=left,
colormap={mymap}{[1pt] rgb(0pt)=(0,1,0); rgb(7pt)=(0,1,1); rgb(15pt)=(0,0,1); rgb(23pt)=(1,0,1); rgb(31pt)=(1,0,0); rgb(38pt)=(0,0,0)},
colorbar,
colorbar style={separate axis lines,every outer x axis line/.append style={black},every x tick label/.append style={font=\color{black}},every outer y axis line/.append style={black},every y tick label/.append style={font=\color{black}},yticklabels={{-19},{-17},{-15},{-13},{-11},{-9},{-7},{-5},{-3},{-1},{1},{3},{5},{7},{9},{11},{13},{15},{17},{19}}}
]
\addplot [color=green,solid,forget plot]
  table[row sep=crcr]{%
0.01	0.0050253436469414\\
1.01	0.00502534430234539\\
2.01	0.00502534497054888\\
3.01	0.00502534565180165\\
4.01	0.0050253463463589\\
5.01	0.00502534705448057\\
6.01	0.00502534777643188\\
7.01	0.00502534851248307\\
8.01	0.00502534926290973\\
9.01	0.00502535002799284\\
10.01	0.00502535080801872\\
11.01	0.00502535160327991\\
12.01	0.00502535241407387\\
13.01	0.00502535324070441\\
14.01	0.00502535408348131\\
15.01	0.00502535494272008\\
16.01	0.00502535581874267\\
17.01	0.00502535671187737\\
18.01	0.00502535762245873\\
19.01	0.00502535855082814\\
20.01	0.00502535949733349\\
21.01	0.00502536046232946\\
22.01	0.00502536144617782\\
23.01	0.00502536244924762\\
24.01	0.00502536347191462\\
25.01	0.00502536451456237\\
26.01	0.00502536557758182\\
27.01	0.00502536666137174\\
28.01	0.00502536776633886\\
29.01	0.00502536889289751\\
30.01	0.00502537004147008\\
31.01	0.00502537121248772\\
32.01	0.00502537240638989\\
33.01	0.00502537362362432\\
34.01	0.00502537486464871\\
35.01	0.00502537612992798\\
36.01	0.00502537741993778\\
37.01	0.00502537873516229\\
38.01	0.00502538007609546\\
39.01	0.00502538144324083\\
40.01	0.00502538283711204\\
41.01	0.00502538425823262\\
42.01	0.00502538570713662\\
43.01	0.00502538718436846\\
44.01	0.00502538869048358\\
45.01	0.00502539022604787\\
46.01	0.00502539179163891\\
47.01	0.005025393387845\\
48.01	0.00502539501526664\\
49.01	0.00502539667451579\\
50.01	0.00502539836621682\\
51.01	0.00502540009100638\\
52.01	0.00502540184953321\\
53.01	0.00502540364245924\\
54.01	0.00502540547045979\\
55.01	0.00502540733422271\\
56.01	0.00502540923444972\\
57.01	0.00502541117185656\\
58.01	0.00502541314717336\\
59.01	0.00502541516114388\\
60.01	0.00502541721452765\\
61.01	0.00502541930809802\\
62.01	0.00502542144264437\\
63.01	0.00502542361897154\\
64.01	0.00502542583790042\\
65.01	0.00502542810026785\\
66.01	0.00502543040692743\\
67.01	0.00502543275874937\\
68.01	0.00502543515662152\\
69.01	0.00502543760144908\\
70.01	0.00502544009415481\\
71.01	0.00502544263568036\\
72.01	0.0050254452269857\\
73.01	0.00502544786904979\\
74.01	0.00502545056287121\\
75.01	0.00502545330946821\\
76.01	0.00502545610987915\\
77.01	0.00502545896516299\\
78.01	0.00502546187639957\\
79.01	0.00502546484469038\\
80.01	0.00502546787115871\\
81.01	0.00502547095695013\\
82.01	0.00502547410323306\\
83.01	0.00502547731119857\\
84.01	0.00502548058206204\\
85.01	0.00502548391706273\\
86.01	0.00502548731746463\\
87.01	0.0050254907845565\\
88.01	0.00502549431965296\\
89.01	0.00502549792409521\\
90.01	0.00502550159925003\\
91.01	0.00502550534651219\\
92.01	0.0050255091673041\\
93.01	0.00502551306307628\\
94.01	0.0050255170353086\\
95.01	0.00502552108550974\\
96.01	0.0050255252152184\\
97.01	0.00502552942600487\\
98.01	0.00502553371946982\\
99.01	0.00502553809724601\\
100.01	0.00502554256099869\\
101.01	0.00502554711242673\\
102.01	0.00502555175326247\\
103.01	0.00502555648527314\\
104.01	0.00502556131026061\\
105.01	0.00502556623006328\\
106.01	0.00502557124655656\\
107.01	0.00502557636165213\\
108.01	0.00502558157730099\\
109.01	0.00502558689549265\\
110.01	0.00502559231825647\\
111.01	0.00502559784766227\\
112.01	0.00502560348582148\\
113.01	0.00502560923488735\\
114.01	0.00502561509705665\\
115.01	0.00502562107456972\\
116.01	0.00502562716971211\\
117.01	0.00502563338481501\\
118.01	0.00502563972225592\\
119.01	0.0050256461844608\\
120.01	0.00502565277390346\\
121.01	0.00502565949310754\\
122.01	0.00502566634464752\\
123.01	0.00502567333114898\\
124.01	0.00502568045529087\\
125.01	0.00502568771980505\\
126.01	0.00502569512747891\\
127.01	0.00502570268115568\\
128.01	0.00502571038373564\\
129.01	0.00502571823817732\\
130.01	0.00502572624749856\\
131.01	0.00502573441477791\\
132.01	0.00502574274315543\\
133.01	0.00502575123583529\\
134.01	0.00502575989608575\\
135.01	0.00502576872724045\\
136.01	0.00502577773270074\\
137.01	0.00502578691593569\\
138.01	0.00502579628048477\\
139.01	0.00502580582995865\\
140.01	0.0050258155680408\\
141.01	0.0050258254984889\\
142.01	0.00502583562513637\\
143.01	0.00502584595189431\\
144.01	0.00502585648275183\\
145.01	0.00502586722177942\\
146.01	0.00502587817312949\\
147.01	0.00502588934103815\\
148.01	0.00502590072982748\\
149.01	0.00502591234390623\\
150.01	0.00502592418777278\\
151.01	0.00502593626601633\\
152.01	0.00502594858331887\\
153.01	0.00502596114445725\\
154.01	0.00502597395430432\\
155.01	0.00502598701783252\\
156.01	0.00502600034011449\\
157.01	0.00502601392632548\\
158.01	0.00502602778174588\\
159.01	0.00502604191176259\\
160.01	0.00502605632187225\\
161.01	0.00502607101768271\\
162.01	0.0050260860049155\\
163.01	0.00502610128940839\\
164.01	0.00502611687711763\\
165.01	0.00502613277412037\\
166.01	0.00502614898661696\\
167.01	0.00502616552093366\\
168.01	0.00502618238352512\\
169.01	0.00502619958097786\\
170.01	0.00502621712001197\\
171.01	0.00502623500748392\\
172.01	0.00502625325038977\\
173.01	0.00502627185586785\\
174.01	0.00502629083120179\\
175.01	0.0050263101838236\\
176.01	0.00502632992131696\\
177.01	0.00502635005141905\\
178.01	0.00502637058202549\\
179.01	0.00502639152119269\\
180.01	0.00502641287714084\\
181.01	0.00502643465825794\\
182.01	0.00502645687310335\\
183.01	0.00502647953041124\\
184.01	0.00502650263909296\\
185.01	0.00502652620824279\\
186.01	0.00502655024714027\\
187.01	0.00502657476525428\\
188.01	0.00502659977224754\\
189.01	0.00502662527797978\\
190.01	0.00502665129251281\\
191.01	0.00502667782611341\\
192.01	0.00502670488925874\\
193.01	0.00502673249264007\\
194.01	0.00502676064716748\\
195.01	0.00502678936397455\\
196.01	0.0050268186544222\\
197.01	0.00502684853010429\\
198.01	0.00502687900285209\\
199.01	0.00502691008473879\\
200.01	0.00502694178808552\\
201.01	0.00502697412546577\\
202.01	0.00502700710971057\\
203.01	0.00502704075391375\\
204.01	0.00502707507143804\\
205.01	0.00502711007592019\\
206.01	0.00502714578127678\\
207.01	0.00502718220170939\\
208.01	0.00502721935171177\\
209.01	0.00502725724607461\\
210.01	0.0050272958998929\\
211.01	0.00502733532857103\\
212.01	0.00502737554782998\\
213.01	0.00502741657371389\\
214.01	0.00502745842259641\\
215.01	0.00502750111118755\\
216.01	0.00502754465654134\\
217.01	0.0050275890760616\\
218.01	0.00502763438751083\\
219.01	0.00502768060901643\\
220.01	0.00502772775907867\\
221.01	0.00502777585657851\\
222.01	0.00502782492078555\\
223.01	0.00502787497136614\\
224.01	0.00502792602839108\\
225.01	0.00502797811234429\\
226.01	0.00502803124413168\\
227.01	0.00502808544508973\\
228.01	0.0050281407369944\\
229.01	0.00502819714206984\\
230.01	0.00502825468299856\\
231.01	0.00502831338293011\\
232.01	0.00502837326549152\\
233.01	0.0050284343547963\\
234.01	0.00502849667545505\\
235.01	0.00502856025258627\\
236.01	0.00502862511182556\\
237.01	0.00502869127933764\\
238.01	0.00502875878182693\\
239.01	0.00502882764654813\\
240.01	0.00502889790131839\\
241.01	0.00502896957452861\\
242.01	0.00502904269515575\\
243.01	0.0050291172927743\\
244.01	0.0050291933975688\\
245.01	0.00502927104034726\\
246.01	0.0050293502525528\\
247.01	0.00502943106627787\\
248.01	0.00502951351427761\\
249.01	0.00502959762998303\\
250.01	0.00502968344751517\\
251.01	0.00502977100169932\\
252.01	0.00502986032808012\\
253.01	0.00502995146293608\\
254.01	0.00503004444329483\\
255.01	0.00503013930694796\\
256.01	0.00503023609246817\\
257.01	0.00503033483922449\\
258.01	0.0050304355873991\\
259.01	0.00503053837800363\\
260.01	0.00503064325289665\\
261.01	0.00503075025480176\\
262.01	0.00503085942732462\\
263.01	0.00503097081497172\\
264.01	0.00503108446316855\\
265.01	0.00503120041827904\\
266.01	0.0050313187276243\\
267.01	0.00503143943950354\\
268.01	0.0050315626032135\\
269.01	0.00503168826906858\\
270.01	0.00503181648842269\\
271.01	0.00503194731369064\\
272.01	0.0050320807983697\\
273.01	0.00503221699706208\\
274.01	0.00503235596549806\\
275.01	0.00503249776055859\\
276.01	0.00503264244029958\\
277.01	0.00503279006397625\\
278.01	0.00503294069206728\\
279.01	0.0050330943863007\\
280.01	0.00503325120967907\\
281.01	0.0050334112265065\\
282.01	0.00503357450241435\\
283.01	0.00503374110439042\\
284.01	0.00503391110080476\\
285.01	0.00503408456144014\\
286.01	0.0050342615575199\\
287.01	0.00503444216173804\\
288.01	0.0050346264482895\\
289.01	0.00503481449290169\\
290.01	0.00503500637286478\\
291.01	0.00503520216706568\\
292.01	0.0050354019560189\\
293.01	0.00503560582190185\\
294.01	0.00503581384858895\\
295.01	0.00503602612168494\\
296.01	0.00503624272856248\\
297.01	0.00503646375839809\\
298.01	0.0050366893022086\\
299.01	0.00503691945289003\\
300.01	0.00503715430525531\\
301.01	0.00503739395607522\\
302.01	0.00503763850411706\\
303.01	0.00503788805018743\\
304.01	0.00503814269717304\\
305.01	0.0050384025500842\\
306.01	0.00503866771609884\\
307.01	0.00503893830460631\\
308.01	0.005039214427254\\
309.01	0.00503949619799313\\
310.01	0.00503978373312652\\
311.01	0.00504007715135721\\
312.01	0.00504037657383813\\
313.01	0.00504068212422235\\
314.01	0.0050409939287139\\
315.01	0.00504131211612241\\
316.01	0.00504163681791495\\
317.01	0.0050419681682714\\
318.01	0.00504230630414115\\
319.01	0.00504265136529913\\
320.01	0.00504300349440465\\
321.01	0.005043362837061\\
322.01	0.00504372954187635\\
323.01	0.00504410376052492\\
324.01	0.00504448564781135\\
325.01	0.00504487536173437\\
326.01	0.0050452730635529\\
327.01	0.00504567891785457\\
328.01	0.00504609309262244\\
329.01	0.00504651575930629\\
330.01	0.00504694709289338\\
331.01	0.00504738727198278\\
332.01	0.00504783647885829\\
333.01	0.00504829489956601\\
334.01	0.00504876272398996\\
335.01	0.00504924014593339\\
336.01	0.00504972736319814\\
337.01	0.00505022457766829\\
338.01	0.00505073199539322\\
339.01	0.00505124982667407\\
340.01	0.00505177828615133\\
341.01	0.00505231759289476\\
342.01	0.00505286797049365\\
343.01	0.0050534296471508\\
344.01	0.00505400285577759\\
345.01	0.00505458783409077\\
346.01	0.00505518482471143\\
347.01	0.00505579407526607\\
348.01	0.00505641583849006\\
349.01	0.00505705037233223\\
350.01	0.00505769794006256\\
351.01	0.00505835881038222\\
352.01	0.00505903325753451\\
353.01	0.00505972156141962\\
354.01	0.00506042400771045\\
355.01	0.00506114088797194\\
356.01	0.00506187249978191\\
357.01	0.00506261914685557\\
358.01	0.00506338113917071\\
359.01	0.00506415879309688\\
360.01	0.00506495243152727\\
361.01	0.00506576238401224\\
362.01	0.00506658898689749\\
363.01	0.00506743258346209\\
364.01	0.00506829352406188\\
365.01	0.00506917216627517\\
366.01	0.00507006887505074\\
367.01	0.00507098402286013\\
368.01	0.00507191798985146\\
369.01	0.00507287116400743\\
370.01	0.00507384394130747\\
371.01	0.00507483672589013\\
372.01	0.00507584993022273\\
373.01	0.00507688397527178\\
374.01	0.00507793929067735\\
375.01	0.00507901631493216\\
376.01	0.00508011549556292\\
377.01	0.00508123728931636\\
378.01	0.00508238216234905\\
379.01	0.00508355059042037\\
380.01	0.00508474305909051\\
381.01	0.00508596006392134\\
382.01	0.00508720211068256\\
383.01	0.00508846971556157\\
384.01	0.00508976340537737\\
385.01	0.00509108371779993\\
386.01	0.00509243120157252\\
387.01	0.00509380641673987\\
388.01	0.00509520993488072\\
389.01	0.0050966423393447\\
390.01	0.00509810422549539\\
391.01	0.00509959620095607\\
392.01	0.00510111888586348\\
393.01	0.00510267291312402\\
394.01	0.00510425892867744\\
395.01	0.00510587759176476\\
396.01	0.00510752957520158\\
397.01	0.00510921556565833\\
398.01	0.00511093626394493\\
399.01	0.00511269238530185\\
400.01	0.00511448465969741\\
401.01	0.00511631383213107\\
402.01	0.00511818066294276\\
403.01	0.0051200859281289\\
404.01	0.00512203041966497\\
405.01	0.00512401494583434\\
406.01	0.00512604033156455\\
407.01	0.00512810741876957\\
408.01	0.00513021706670055\\
409.01	0.00513237015230281\\
410.01	0.00513456757058009\\
411.01	0.00513681023496659\\
412.01	0.00513909907770765\\
413.01	0.00514143505024678\\
414.01	0.00514381912362155\\
415.01	0.0051462522888675\\
416.01	0.0051487355574302\\
417.01	0.00515126996158641\\
418.01	0.00515385655487299\\
419.01	0.00515649641252641\\
420.01	0.00515919063192849\\
421.01	0.005161940333065\\
422.01	0.00516474665899053\\
423.01	0.00516761077630448\\
424.01	0.00517053387563707\\
425.01	0.00517351717214504\\
426.01	0.00517656190601712\\
427.01	0.00517966934299052\\
428.01	0.00518284077487896\\
429.01	0.00518607752010913\\
430.01	0.00518938092427124\\
431.01	0.00519275236067859\\
432.01	0.00519619323094003\\
433.01	0.00519970496554423\\
434.01	0.00520328902445524\\
435.01	0.00520694689772078\\
436.01	0.00521068010609343\\
437.01	0.00521449020166515\\
438.01	0.00521837876851267\\
439.01	0.00522234742335928\\
440.01	0.00522639781624823\\
441.01	0.00523053163123074\\
442.01	0.00523475058706865\\
443.01	0.00523905643795085\\
444.01	0.00524345097422511\\
445.01	0.00524793602314479\\
446.01	0.00525251344963172\\
447.01	0.00525718515705385\\
448.01	0.00526195308801961\\
449.01	0.00526681922518889\\
450.01	0.0052717855921007\\
451.01	0.00527685425401767\\
452.01	0.00528202731878902\\
453.01	0.00528730693772994\\
454.01	0.00529269530652099\\
455.01	0.00529819466612468\\
456.01	0.00530380730372196\\
457.01	0.00530953555366792\\
458.01	0.00531538179846745\\
459.01	0.0053213484697711\\
460.01	0.00532743804939127\\
461.01	0.00533365307034027\\
462.01	0.00533999611788943\\
463.01	0.00534646983065033\\
464.01	0.00535307690167858\\
465.01	0.00535982007960048\\
466.01	0.0053667021697633\\
467.01	0.00537372603540943\\
468.01	0.00538089459887411\\
469.01	0.00538821084280963\\
470.01	0.00539567781143413\\
471.01	0.00540329861180545\\
472.01	0.00541107641512373\\
473.01	0.00541901445805865\\
474.01	0.00542711604410569\\
475.01	0.00543538454497058\\
476.01	0.00544382340198156\\
477.01	0.00545243612753163\\
478.01	0.00546122630655084\\
479.01	0.00547019759800861\\
480.01	0.00547935373644722\\
481.01	0.00548869853354755\\
482.01	0.00549823587972666\\
483.01	0.00550796974576846\\
484.01	0.00551790418448807\\
485.01	0.00552804333243061\\
486.01	0.00553839141160435\\
487.01	0.00554895273125022\\
488.01	0.00555973168964703\\
489.01	0.00557073277595338\\
490.01	0.00558196057208785\\
491.01	0.00559341975464639\\
492.01	0.00560511509685933\\
493.01	0.00561705147058765\\
494.01	0.00562923384835898\\
495.01	0.00564166730544473\\
496.01	0.00565435702197845\\
497.01	0.00566730828511622\\
498.01	0.00568052649123867\\
499.01	0.00569401714819741\\
500.01	0.00570778587760391\\
501.01	0.00572183841716205\\
502.01	0.0057361806230467\\
503.01	0.00575081847232464\\
504.01	0.0057657580654227\\
505.01	0.00578100562863958\\
506.01	0.00579656751670357\\
507.01	0.00581245021537605\\
508.01	0.00582866034409943\\
509.01	0.00584520465869111\\
510.01	0.00586209005408149\\
511.01	0.00587932356709678\\
512.01	0.00589691237928515\\
513.01	0.00591486381978583\\
514.01	0.00593318536823928\\
515.01	0.00595188465773874\\
516.01	0.00597096947781858\\
517.01	0.0059904477774809\\
518.01	0.00601032766825516\\
519.01	0.00603061742729019\\
520.01	0.0060513255004731\\
521.01	0.00607246050557363\\
522.01	0.0060940312354081\\
523.01	0.00611604666101788\\
524.01	0.00613851593485767\\
525.01	0.00616144839398609\\
526.01	0.00618485356325117\\
527.01	0.006208741158463\\
528.01	0.00623312108954262\\
529.01	0.00625800346363684\\
530.01	0.0062833985881868\\
531.01	0.00630931697393645\\
532.01	0.00633576933786364\\
533.01	0.00636276660601857\\
534.01	0.00639031991624708\\
535.01	0.00641844062077823\\
536.01	0.00644714028865036\\
537.01	0.00647643070794703\\
538.01	0.00650632388781151\\
539.01	0.00653683206020528\\
540.01	0.00656796768136911\\
541.01	0.00659974343294424\\
542.01	0.00663217222270325\\
543.01	0.00666526718483568\\
544.01	0.00669904167972642\\
545.01	0.00673350929315926\\
546.01	0.00676868383486742\\
547.01	0.00680457933634701\\
548.01	0.00684121004783805\\
549.01	0.00687859043436699\\
550.01	0.00691673517073471\\
551.01	0.0069556591353172\\
552.01	0.00699537740253582\\
553.01	0.00703590523383653\\
554.01	0.0070772580669986\\
555.01	0.00711945150357646\\
556.01	0.00716250129425623\\
557.01	0.00720642332188554\\
558.01	0.00725123358191065\\
559.01	0.00729694815992683\\
560.01	0.00734358320602013\\
561.01	0.00739115490554365\\
562.01	0.00743967944594207\\
563.01	0.00748917297919549\\
564.01	0.00753965157942145\\
565.01	0.00759113119512742\\
566.01	0.00764362759556816\\
567.01	0.00769715631061607\\
568.01	0.00775173256351049\\
569.01	0.0078073711958089\\
570.01	0.00786408658382248\\
571.01	0.00792189254578262\\
572.01	0.00798080223895675\\
573.01	0.00804082804591287\\
574.01	0.00810198144913145\\
575.01	0.00816427289318073\\
576.01	0.00822771163372017\\
577.01	0.00829230557268554\\
578.01	0.00835806107914579\\
579.01	0.00842498279553232\\
580.01	0.00849307342923551\\
581.01	0.00856233352997421\\
582.01	0.00863276125390358\\
583.01	0.00870435211617224\\
584.01	0.00877709873462935\\
585.01	0.00885099056867839\\
586.01	0.00892601365895706\\
587.01	0.00900215037570352\\
588.01	0.00907937918646837\\
589.01	0.0091576744574184\\
590.01	0.00923700630705446\\
591.01	0.00931734053698447\\
592.01	0.00939863867177418\\
593.01	0.00948085814924172\\
594.01	0.00956395271435918\\
595.01	0.00964787308480385\\
596.01	0.00973256797493054\\
597.01	0.0098179855884858\\
598.01	0.00990285876644176\\
599.01	0.00996919203046377\\
599.02	0.00996973314335038\\
599.03	0.0099702709571694\\
599.04	0.00997080543920336\\
599.05	0.0099713365564138\\
599.06	0.00997186427543808\\
599.07	0.0099723885625862\\
599.08	0.00997290938383756\\
599.09	0.00997342670483771\\
599.1	0.00997394049089505\\
599.11	0.00997445070697751\\
599.12	0.00997495731770918\\
599.13	0.00997546028736696\\
599.14	0.00997595957987707\\
599.15	0.00997645515881165\\
599.16	0.00997694698738526\\
599.17	0.00997743502845131\\
599.18	0.00997791924449856\\
599.19	0.00997839959764748\\
599.2	0.00997887604964662\\
599.21	0.00997934856186899\\
599.22	0.00997981709530829\\
599.23	0.00998028161057521\\
599.24	0.00998074206789365\\
599.25	0.0099811984270969\\
599.26	0.00998165064762378\\
599.27	0.00998209868851477\\
599.28	0.00998254250840806\\
599.29	0.00998298206553559\\
599.3	0.00998341731771905\\
599.31	0.00998384822236584\\
599.32	0.00998427473646497\\
599.33	0.00998469681658292\\
599.34	0.00998511441885952\\
599.35	0.0099855274990037\\
599.36	0.00998593601228926\\
599.37	0.00998633991355056\\
599.38	0.00998673915717821\\
599.39	0.00998713369711469\\
599.4	0.00998752348684992\\
599.41	0.00998790847811157\\
599.42	0.00998828861974491\\
599.43	0.00998866386008804\\
599.44	0.00998903414696681\\
599.45	0.00998939942768985\\
599.46	0.00998975964904344\\
599.47	0.00999011475728636\\
599.48	0.00999046469814472\\
599.49	0.00999080941680669\\
599.5	0.00999114885791725\\
599.51	0.00999148296557278\\
599.52	0.0099918116833157\\
599.53	0.00999213495412901\\
599.54	0.00999245272043077\\
599.55	0.00999276492406855\\
599.56	0.00999307150631382\\
599.57	0.00999337240785624\\
599.58	0.00999366756879799\\
599.59	0.00999395692864795\\
599.6	0.00999424042631586\\
599.61	0.00999451800010642\\
599.62	0.00999478958771336\\
599.63	0.0099950551262134\\
599.64	0.00999531455206019\\
599.65	0.00999556780107816\\
599.66	0.00999581480845634\\
599.67	0.00999605550874211\\
599.68	0.00999628983583487\\
599.69	0.00999651772297967\\
599.7	0.00999673910276076\\
599.71	0.00999695390709509\\
599.72	0.00999716206722577\\
599.73	0.00999736351371538\\
599.74	0.00999755817643933\\
599.75	0.00999774598457904\\
599.76	0.00999792686661517\\
599.77	0.00999810075032068\\
599.78	0.00999826756275386\\
599.79	0.00999842723025133\\
599.8	0.00999857967842092\\
599.81	0.0099987248321345\\
599.82	0.00999886261552071\\
599.83	0.00999899295195769\\
599.84	0.00999911576406567\\
599.85	0.00999923097369951\\
599.86	0.00999933850194117\\
599.87	0.00999943826909211\\
599.88	0.00999953019466558\\
599.89	0.00999961419737891\\
599.9	0.00999969019514566\\
599.91	0.00999975810506767\\
599.92	0.00999981784342713\\
599.93	0.00999986932567848\\
599.94	0.00999991246644028\\
599.95	0.00999994717948697\\
599.96	0.00999997337774056\\
599.97	0.00999999097326228\\
599.98	0.00999999987724406\\
599.99	0.01\\
600	0.01\\
};
\addplot [color=mycolor1,solid,forget plot]
  table[row sep=crcr]{%
0.01	0.00502505708478154\\
1.01	0.00502505776987136\\
2.01	0.00502505846833001\\
3.01	0.00502505918041851\\
4.01	0.00502505990640237\\
5.01	0.00502506064655234\\
6.01	0.00502506140114448\\
7.01	0.00502506217046036\\
8.01	0.00502506295478708\\
9.01	0.00502506375441727\\
10.01	0.0050250645696491\\
11.01	0.00502506540078659\\
12.01	0.00502506624814011\\
13.01	0.00502506711202561\\
14.01	0.00502506799276525\\
15.01	0.00502506889068779\\
16.01	0.00502506980612805\\
17.01	0.00502507073942768\\
18.01	0.00502507169093472\\
19.01	0.00502507266100416\\
20.01	0.00502507364999782\\
21.01	0.0050250746582847\\
22.01	0.00502507568624078\\
23.01	0.00502507673424957\\
24.01	0.0050250778027022\\
25.01	0.0050250788919973\\
26.01	0.00502508000254118\\
27.01	0.00502508113474812\\
28.01	0.00502508228904022\\
29.01	0.00502508346584857\\
30.01	0.00502508466561202\\
31.01	0.00502508588877826\\
32.01	0.00502508713580357\\
33.01	0.00502508840715338\\
34.01	0.00502508970330179\\
35.01	0.00502509102473275\\
36.01	0.00502509237193918\\
37.01	0.00502509374542376\\
38.01	0.0050250951456991\\
39.01	0.00502509657328767\\
40.01	0.00502509802872239\\
41.01	0.00502509951254626\\
42.01	0.00502510102531317\\
43.01	0.0050251025675877\\
44.01	0.00502510413994553\\
45.01	0.00502510574297341\\
46.01	0.00502510737726991\\
47.01	0.00502510904344494\\
48.01	0.00502511074212078\\
49.01	0.00502511247393153\\
50.01	0.00502511423952365\\
51.01	0.00502511603955635\\
52.01	0.00502511787470195\\
53.01	0.00502511974564554\\
54.01	0.00502512165308583\\
55.01	0.0050251235977353\\
56.01	0.00502512558032045\\
57.01	0.00502512760158177\\
58.01	0.00502512966227386\\
59.01	0.00502513176316688\\
60.01	0.00502513390504561\\
61.01	0.00502513608871036\\
62.01	0.00502513831497725\\
63.01	0.00502514058467798\\
64.01	0.00502514289866096\\
65.01	0.0050251452577908\\
66.01	0.00502514766294929\\
67.01	0.00502515011503596\\
68.01	0.00502515261496719\\
69.01	0.00502515516367757\\
70.01	0.00502515776212048\\
71.01	0.00502516041126738\\
72.01	0.0050251631121091\\
73.01	0.00502516586565591\\
74.01	0.00502516867293762\\
75.01	0.0050251715350047\\
76.01	0.00502517445292788\\
77.01	0.00502517742779912\\
78.01	0.00502518046073192\\
79.01	0.00502518355286122\\
80.01	0.00502518670534463\\
81.01	0.00502518991936224\\
82.01	0.00502519319611752\\
83.01	0.00502519653683742\\
84.01	0.0050251999427732\\
85.01	0.00502520341520069\\
86.01	0.00502520695542052\\
87.01	0.00502521056475919\\
88.01	0.00502521424456948\\
89.01	0.00502521799622993\\
90.01	0.00502522182114742\\
91.01	0.00502522572075539\\
92.01	0.00502522969651603\\
93.01	0.00502523374992067\\
94.01	0.00502523788248893\\
95.01	0.00502524209577135\\
96.01	0.00502524639134888\\
97.01	0.00502525077083307\\
98.01	0.00502525523586747\\
99.01	0.00502525978812848\\
100.01	0.00502526442932494\\
101.01	0.00502526916119955\\
102.01	0.00502527398552965\\
103.01	0.00502527890412721\\
104.01	0.0050252839188403\\
105.01	0.00502528903155341\\
106.01	0.00502529424418774\\
107.01	0.00502529955870326\\
108.01	0.00502530497709808\\
109.01	0.00502531050140967\\
110.01	0.00502531613371601\\
111.01	0.00502532187613616\\
112.01	0.00502532773083089\\
113.01	0.00502533370000388\\
114.01	0.00502533978590211\\
115.01	0.00502534599081714\\
116.01	0.00502535231708595\\
117.01	0.00502535876709181\\
118.01	0.00502536534326498\\
119.01	0.00502537204808393\\
120.01	0.005025378884076\\
121.01	0.00502538585381879\\
122.01	0.00502539295994089\\
123.01	0.00502540020512304\\
124.01	0.00502540759209896\\
125.01	0.0050254151236567\\
126.01	0.00502542280263979\\
127.01	0.0050254306319476\\
128.01	0.00502543861453753\\
129.01	0.00502544675342539\\
130.01	0.00502545505168694\\
131.01	0.00502546351245933\\
132.01	0.00502547213894153\\
133.01	0.00502548093439648\\
134.01	0.00502548990215143\\
135.01	0.00502549904560015\\
136.01	0.00502550836820367\\
137.01	0.0050255178734924\\
138.01	0.00502552756506625\\
139.01	0.00502553744659737\\
140.01	0.00502554752183065\\
141.01	0.00502555779458538\\
142.01	0.00502556826875752\\
143.01	0.00502557894831987\\
144.01	0.0050255898373251\\
145.01	0.00502560093990664\\
146.01	0.00502561226027967\\
147.01	0.00502562380274427\\
148.01	0.00502563557168551\\
149.01	0.00502564757157675\\
150.01	0.00502565980698014\\
151.01	0.00502567228254919\\
152.01	0.00502568500303027\\
153.01	0.00502569797326444\\
154.01	0.00502571119819006\\
155.01	0.00502572468284337\\
156.01	0.00502573843236212\\
157.01	0.0050257524519868\\
158.01	0.00502576674706218\\
159.01	0.00502578132304063\\
160.01	0.00502579618548352\\
161.01	0.00502581134006361\\
162.01	0.00502582679256727\\
163.01	0.00502584254889701\\
164.01	0.00502585861507368\\
165.01	0.00502587499723875\\
166.01	0.00502589170165724\\
167.01	0.00502590873471953\\
168.01	0.00502592610294486\\
169.01	0.00502594381298288\\
170.01	0.00502596187161705\\
171.01	0.00502598028576712\\
172.01	0.0050259990624921\\
173.01	0.00502601820899304\\
174.01	0.00502603773261556\\
175.01	0.00502605764085359\\
176.01	0.00502607794135135\\
177.01	0.00502609864190751\\
178.01	0.00502611975047776\\
179.01	0.00502614127517792\\
180.01	0.00502616322428776\\
181.01	0.00502618560625406\\
182.01	0.00502620842969345\\
183.01	0.00502623170339703\\
184.01	0.00502625543633339\\
185.01	0.00502627963765204\\
186.01	0.00502630431668719\\
187.01	0.00502632948296225\\
188.01	0.0050263551461926\\
189.01	0.00502638131629046\\
190.01	0.00502640800336845\\
191.01	0.00502643521774387\\
192.01	0.00502646296994291\\
193.01	0.00502649127070517\\
194.01	0.00502652013098762\\
195.01	0.00502654956196918\\
196.01	0.00502657957505573\\
197.01	0.00502661018188422\\
198.01	0.00502664139432787\\
199.01	0.00502667322450087\\
200.01	0.00502670568476324\\
201.01	0.00502673878772611\\
202.01	0.00502677254625666\\
203.01	0.00502680697348418\\
204.01	0.00502684208280435\\
205.01	0.00502687788788525\\
206.01	0.00502691440267317\\
207.01	0.00502695164139846\\
208.01	0.00502698961858103\\
209.01	0.0050270283490364\\
210.01	0.00502706784788198\\
211.01	0.00502710813054341\\
212.01	0.00502714921276085\\
213.01	0.00502719111059502\\
214.01	0.0050272338404346\\
215.01	0.00502727741900305\\
216.01	0.00502732186336444\\
217.01	0.005027367190932\\
218.01	0.00502741341947435\\
219.01	0.00502746056712336\\
220.01	0.00502750865238133\\
221.01	0.00502755769412885\\
222.01	0.00502760771163299\\
223.01	0.00502765872455443\\
224.01	0.0050277107529568\\
225.01	0.00502776381731429\\
226.01	0.00502781793852039\\
227.01	0.00502787313789665\\
228.01	0.00502792943720123\\
229.01	0.00502798685863878\\
230.01	0.0050280454248687\\
231.01	0.00502810515901558\\
232.01	0.00502816608467785\\
233.01	0.00502822822593832\\
234.01	0.00502829160737461\\
235.01	0.00502835625406842\\
236.01	0.00502842219161706\\
237.01	0.00502848944614304\\
238.01	0.00502855804430619\\
239.01	0.00502862801331419\\
240.01	0.00502869938093389\\
241.01	0.00502877217550329\\
242.01	0.00502884642594303\\
243.01	0.00502892216176901\\
244.01	0.00502899941310453\\
245.01	0.00502907821069254\\
246.01	0.0050291585859096\\
247.01	0.00502924057077767\\
248.01	0.00502932419797852\\
249.01	0.00502940950086742\\
250.01	0.0050294965134865\\
251.01	0.00502958527058009\\
252.01	0.00502967580760794\\
253.01	0.00502976816076126\\
254.01	0.00502986236697753\\
255.01	0.00502995846395607\\
256.01	0.00503005649017382\\
257.01	0.00503015648490135\\
258.01	0.00503025848821978\\
259.01	0.0050303625410372\\
260.01	0.00503046868510645\\
261.01	0.00503057696304171\\
262.01	0.00503068741833726\\
263.01	0.00503080009538562\\
264.01	0.00503091503949554\\
265.01	0.00503103229691196\\
266.01	0.00503115191483511\\
267.01	0.00503127394143976\\
268.01	0.00503139842589623\\
269.01	0.00503152541839064\\
270.01	0.00503165497014652\\
271.01	0.00503178713344521\\
272.01	0.00503192196164903\\
273.01	0.00503205950922278\\
274.01	0.00503219983175696\\
275.01	0.00503234298599126\\
276.01	0.00503248902983819\\
277.01	0.0050326380224073\\
278.01	0.00503279002402979\\
279.01	0.00503294509628419\\
280.01	0.00503310330202217\\
281.01	0.00503326470539428\\
282.01	0.00503342937187773\\
283.01	0.00503359736830303\\
284.01	0.00503376876288231\\
285.01	0.00503394362523771\\
286.01	0.00503412202643089\\
287.01	0.0050343040389923\\
288.01	0.00503448973695184\\
289.01	0.00503467919586953\\
290.01	0.00503487249286755\\
291.01	0.00503506970666168\\
292.01	0.00503527091759529\\
293.01	0.00503547620767214\\
294.01	0.00503568566059007\\
295.01	0.0050358993617779\\
296.01	0.00503611739842902\\
297.01	0.00503633985953877\\
298.01	0.0050365668359417\\
299.01	0.00503679842034899\\
300.01	0.00503703470738753\\
301.01	0.00503727579363877\\
302.01	0.00503752177767999\\
303.01	0.00503777276012465\\
304.01	0.00503802884366436\\
305.01	0.00503829013311213\\
306.01	0.00503855673544537\\
307.01	0.00503882875985134\\
308.01	0.00503910631777216\\
309.01	0.00503938952295147\\
310.01	0.00503967849148132\\
311.01	0.00503997334185133\\
312.01	0.00504027419499739\\
313.01	0.00504058117435266\\
314.01	0.00504089440589889\\
315.01	0.00504121401821822\\
316.01	0.0050415401425481\\
317.01	0.005041872912835\\
318.01	0.00504221246579045\\
319.01	0.00504255894094829\\
320.01	0.00504291248072317\\
321.01	0.00504327323046944\\
322.01	0.0050436413385419\\
323.01	0.00504401695635803\\
324.01	0.00504440023846103\\
325.01	0.00504479134258454\\
326.01	0.00504519042971829\\
327.01	0.00504559766417539\\
328.01	0.00504601321366138\\
329.01	0.00504643724934425\\
330.01	0.00504686994592582\\
331.01	0.00504731148171481\\
332.01	0.00504776203870178\\
333.01	0.00504822180263492\\
334.01	0.00504869096309848\\
335.01	0.00504916971359129\\
336.01	0.00504965825160864\\
337.01	0.00505015677872389\\
338.01	0.00505066550067459\\
339.01	0.00505118462744747\\
340.01	0.00505171437336652\\
341.01	0.00505225495718289\\
342.01	0.00505280660216778\\
343.01	0.00505336953620408\\
344.01	0.00505394399188328\\
345.01	0.00505453020660221\\
346.01	0.00505512842266326\\
347.01	0.00505573888737543\\
348.01	0.00505636185315794\\
349.01	0.00505699757764599\\
350.01	0.00505764632379935\\
351.01	0.00505830836001134\\
352.01	0.00505898396022197\\
353.01	0.00505967340403249\\
354.01	0.00506037697682216\\
355.01	0.0050610949698679\\
356.01	0.0050618276804657\\
357.01	0.00506257541205518\\
358.01	0.00506333847434595\\
359.01	0.00506411718344768\\
360.01	0.00506491186200134\\
361.01	0.00506572283931435\\
362.01	0.00506655045149716\\
363.01	0.00506739504160418\\
364.01	0.00506825695977667\\
365.01	0.00506913656338806\\
366.01	0.00507003421719339\\
367.01	0.00507095029348086\\
368.01	0.00507188517222661\\
369.01	0.0050728392412534\\
370.01	0.00507381289639131\\
371.01	0.00507480654164263\\
372.01	0.00507582058934995\\
373.01	0.00507685546036712\\
374.01	0.0050779115842346\\
375.01	0.00507898939935726\\
376.01	0.0050800893531875\\
377.01	0.00508121190241012\\
378.01	0.00508235751313217\\
379.01	0.00508352666107692\\
380.01	0.00508471983178049\\
381.01	0.00508593752079405\\
382.01	0.0050871802338892\\
383.01	0.00508844848726744\\
384.01	0.0050897428077748\\
385.01	0.00509106373312024\\
386.01	0.00509241181209835\\
387.01	0.00509378760481723\\
388.01	0.00509519168293052\\
389.01	0.00509662462987447\\
390.01	0.00509808704111012\\
391.01	0.00509957952436935\\
392.01	0.00510110269990748\\
393.01	0.00510265720076039\\
394.01	0.00510424367300651\\
395.01	0.00510586277603481\\
396.01	0.00510751518281828\\
397.01	0.0051092015801926\\
398.01	0.00511092266914112\\
399.01	0.00511267916508518\\
400.01	0.00511447179818102\\
401.01	0.00511630131362208\\
402.01	0.00511816847194851\\
403.01	0.00512007404936213\\
404.01	0.00512201883804843\\
405.01	0.00512400364650505\\
406.01	0.00512602929987734\\
407.01	0.00512809664030063\\
408.01	0.00513020652724942\\
409.01	0.00513235983789442\\
410.01	0.00513455746746602\\
411.01	0.00513680032962638\\
412.01	0.00513908935684861\\
413.01	0.00514142550080382\\
414.01	0.00514380973275642\\
415.01	0.00514624304396753\\
416.01	0.00514872644610682\\
417.01	0.00515126097167247\\
418.01	0.00515384767442111\\
419.01	0.00515648762980477\\
420.01	0.00515918193541862\\
421.01	0.00516193171145674\\
422.01	0.00516473810117871\\
423.01	0.00516760227138448\\
424.01	0.0051705254129001\\
425.01	0.00517350874107243\\
426.01	0.00517655349627588\\
427.01	0.00517966094442791\\
428.01	0.00518283237751574\\
429.01	0.00518606911413511\\
430.01	0.00518937250003826\\
431.01	0.00519274390869479\\
432.01	0.00519618474186347\\
433.01	0.00519969643017593\\
434.01	0.00520328043373308\\
435.01	0.00520693824271272\\
436.01	0.00521067137799075\\
437.01	0.00521448139177509\\
438.01	0.00521836986825194\\
439.01	0.00522233842424669\\
440.01	0.0052263887098974\\
441.01	0.0052305224093428\\
442.01	0.00523474124142474\\
443.01	0.00523904696040461\\
444.01	0.00524344135669504\\
445.01	0.00524792625760664\\
446.01	0.00525250352811009\\
447.01	0.00525717507161445\\
448.01	0.00526194283076127\\
449.01	0.00526680878823532\\
450.01	0.005271774967592\\
451.01	0.00527684343410191\\
452.01	0.00528201629561358\\
453.01	0.00528729570343331\\
454.01	0.00529268385322357\\
455.01	0.00529818298592039\\
456.01	0.00530379538866937\\
457.01	0.00530952339578142\\
458.01	0.00531536938970828\\
459.01	0.00532133580203822\\
460.01	0.0053274251145128\\
461.01	0.00533363986006455\\
462.01	0.00533998262387593\\
463.01	0.00534645604446064\\
464.01	0.00535306281476762\\
465.01	0.00535980568330775\\
466.01	0.00536668745530369\\
467.01	0.00537371099386401\\
468.01	0.00538087922118197\\
469.01	0.00538819511975855\\
470.01	0.00539566173365116\\
471.01	0.00540328216974893\\
472.01	0.00541105959907338\\
473.01	0.00541899725810693\\
474.01	0.00542709845014895\\
475.01	0.00543536654669927\\
476.01	0.00544380498887161\\
477.01	0.005452417288835\\
478.01	0.00546120703128584\\
479.01	0.00547017787495054\\
480.01	0.00547933355411916\\
481.01	0.00548867788021037\\
482.01	0.00549821474336881\\
483.01	0.00550794811409631\\
484.01	0.00551788204491581\\
485.01	0.00552802067206939\\
486.01	0.00553836821725223\\
487.01	0.00554892898938124\\
488.01	0.00555970738639985\\
489.01	0.00557070789712073\\
490.01	0.00558193510310437\\
491.01	0.00559339368057722\\
492.01	0.00560508840238808\\
493.01	0.00561702414000388\\
494.01	0.00562920586554576\\
495.01	0.00564163865386569\\
496.01	0.00565432768466423\\
497.01	0.00566727824465042\\
498.01	0.00568049572974454\\
499.01	0.00569398564732272\\
500.01	0.00570775361850585\\
501.01	0.00572180538049252\\
502.01	0.00573614678893519\\
503.01	0.00575078382036302\\
504.01	0.00576572257464746\\
505.01	0.00578096927751429\\
506.01	0.00579653028310122\\
507.01	0.00581241207655984\\
508.01	0.0058286212767039\\
509.01	0.00584516463870154\\
510.01	0.0058620490568133\\
511.01	0.00587928156717432\\
512.01	0.0058968693506196\\
513.01	0.00591481973555158\\
514.01	0.00593314020085063\\
515.01	0.00595183837882469\\
516.01	0.00597092205819774\\
517.01	0.00599039918713468\\
518.01	0.00601027787630053\\
519.01	0.00603056640195079\\
520.01	0.00605127320905013\\
521.01	0.00607240691441507\\
522.01	0.00609397630987676\\
523.01	0.00611599036545887\\
524.01	0.00613845823256408\\
525.01	0.00616138924716375\\
526.01	0.00618479293298224\\
527.01	0.00620867900466769\\
528.01	0.0062330573709399\\
529.01	0.00625793813770395\\
530.01	0.00628333161111733\\
531.01	0.00630924830059647\\
532.01	0.00633569892174679\\
533.01	0.00636269439919962\\
534.01	0.00639024586933368\\
535.01	0.00641836468286175\\
536.01	0.0064470624072543\\
537.01	0.00647635082897466\\
538.01	0.00650624195549183\\
539.01	0.00653674801703709\\
540.01	0.00656788146806421\\
541.01	0.00659965498836909\\
542.01	0.0066320814838192\\
543.01	0.00666517408663819\\
544.01	0.0066989461551837\\
545.01	0.00673341127314891\\
546.01	0.0067685832481139\\
547.01	0.0068044761093578\\
548.01	0.00684110410483998\\
549.01	0.00687848169724375\\
550.01	0.0069166235589634\\
551.01	0.00695554456590763\\
552.01	0.00699525978997213\\
553.01	0.00703578449001997\\
554.01	0.00707713410119494\\
555.01	0.00711932422236754\\
556.01	0.00716237060149713\\
557.01	0.00720628911866866\\
558.01	0.00725109576653823\\
559.01	0.00729680662789368\\
560.01	0.00734343785000768\\
561.01	0.00739100561542945\\
562.01	0.00743952610882453\\
563.01	0.00748901547943973\\
564.01	0.00753948979872679\\
565.01	0.00759096501262174\\
566.01	0.00764345688793244\\
567.01	0.00769698095224389\\
568.01	0.0077515524267073\\
569.01	0.00780718615103681\\
570.01	0.00786389649999601\\
571.01	0.00792169729062253\\
572.01	0.00798060167940823\\
573.01	0.00804062204863615\\
574.01	0.00810176988107205\\
575.01	0.00816405562222705\\
576.01	0.00822748852945653\\
577.01	0.00829207650724693\\
578.01	0.00835782592818142\\
579.01	0.0084247414392816\\
580.01	0.00849282575371788\\
581.01	0.0085620794282911\\
582.01	0.00863250062764312\\
583.01	0.00870408487690098\\
584.01	0.00877682480544311\\
585.01	0.00885070988576821\\
586.01	0.00892572617312741\\
587.01	0.00900185605375053\\
588.01	0.00907907801228958\\
589.01	0.00915736643267762\\
590.01	0.00923669145116018\\
591.01	0.00931701888605708\\
592.01	0.00939831027616845\\
593.01	0.00948052306904656\\
594.01	0.00956361101211311\\
595.01	0.00964752481442215\\
596.01	0.00973221316552893\\
597.01	0.00981762422138093\\
598.01	0.00990285624732339\\
599.01	0.00996919203046377\\
599.02	0.00996973314335038\\
599.03	0.0099702709571694\\
599.04	0.00997080543920336\\
599.05	0.0099713365564138\\
599.06	0.00997186427543808\\
599.07	0.0099723885625862\\
599.08	0.00997290938383756\\
599.09	0.00997342670483771\\
599.1	0.00997394049089505\\
599.11	0.00997445070697751\\
599.12	0.00997495731770918\\
599.13	0.00997546028736696\\
599.14	0.00997595957987707\\
599.15	0.00997645515881165\\
599.16	0.00997694698738526\\
599.17	0.00997743502845131\\
599.18	0.00997791924449856\\
599.19	0.00997839959764748\\
599.2	0.00997887604964662\\
599.21	0.00997934856186899\\
599.22	0.00997981709530829\\
599.23	0.00998028161057521\\
599.24	0.00998074206789365\\
599.25	0.0099811984270969\\
599.26	0.00998165064762378\\
599.27	0.00998209868851477\\
599.28	0.00998254250840806\\
599.29	0.00998298206553559\\
599.3	0.00998341731771905\\
599.31	0.00998384822236584\\
599.32	0.00998427473646497\\
599.33	0.00998469681658292\\
599.34	0.00998511441885952\\
599.35	0.0099855274990037\\
599.36	0.00998593601228926\\
599.37	0.00998633991355056\\
599.38	0.00998673915717821\\
599.39	0.00998713369711469\\
599.4	0.00998752348684992\\
599.41	0.00998790847811156\\
599.42	0.00998828861974491\\
599.43	0.00998866386008803\\
599.44	0.00998903414696681\\
599.45	0.00998939942768985\\
599.46	0.00998975964904344\\
599.47	0.00999011475728636\\
599.48	0.00999046469814471\\
599.49	0.00999080941680669\\
599.5	0.00999114885791725\\
599.51	0.00999148296557278\\
599.52	0.0099918116833157\\
599.53	0.00999213495412901\\
599.54	0.00999245272043077\\
599.55	0.00999276492406855\\
599.56	0.00999307150631381\\
599.57	0.00999337240785624\\
599.58	0.00999366756879799\\
599.59	0.00999395692864795\\
599.6	0.00999424042631585\\
599.61	0.00999451800010642\\
599.62	0.00999478958771336\\
599.63	0.0099950551262134\\
599.64	0.00999531455206019\\
599.65	0.00999556780107816\\
599.66	0.00999581480845634\\
599.67	0.00999605550874211\\
599.68	0.00999628983583487\\
599.69	0.00999651772297967\\
599.7	0.00999673910276076\\
599.71	0.00999695390709509\\
599.72	0.00999716206722577\\
599.73	0.00999736351371538\\
599.74	0.00999755817643933\\
599.75	0.00999774598457904\\
599.76	0.00999792686661517\\
599.77	0.00999810075032068\\
599.78	0.00999826756275386\\
599.79	0.00999842723025133\\
599.8	0.00999857967842092\\
599.81	0.0099987248321345\\
599.82	0.00999886261552071\\
599.83	0.00999899295195769\\
599.84	0.00999911576406567\\
599.85	0.00999923097369951\\
599.86	0.00999933850194117\\
599.87	0.00999943826909211\\
599.88	0.00999953019466558\\
599.89	0.00999961419737891\\
599.9	0.00999969019514566\\
599.91	0.00999975810506767\\
599.92	0.00999981784342713\\
599.93	0.00999986932567848\\
599.94	0.00999991246644028\\
599.95	0.00999994717948697\\
599.96	0.00999997337774056\\
599.97	0.00999999097326228\\
599.98	0.00999999987724406\\
599.99	0.01\\
600	0.01\\
};
\addplot [color=mycolor2,solid,forget plot]
  table[row sep=crcr]{%
0.01	0.00502433004032531\\
1.01	0.00502433078290415\\
2.01	0.00502433153996987\\
3.01	0.00502433231180427\\
4.01	0.00502433309869423\\
5.01	0.00502433390093257\\
6.01	0.00502433471881732\\
7.01	0.00502433555265265\\
8.01	0.00502433640274838\\
9.01	0.00502433726942045\\
10.01	0.00502433815299091\\
11.01	0.00502433905378807\\
12.01	0.00502433997214664\\
13.01	0.0050243409084075\\
14.01	0.00502434186291843\\
15.01	0.00502434283603398\\
16.01	0.00502434382811538\\
17.01	0.00502434483953081\\
18.01	0.00502434587065569\\
19.01	0.00502434692187287\\
20.01	0.00502434799357236\\
21.01	0.00502434908615175\\
22.01	0.00502435020001639\\
23.01	0.00502435133557945\\
24.01	0.00502435249326209\\
25.01	0.00502435367349358\\
26.01	0.00502435487671149\\
27.01	0.0050243561033619\\
28.01	0.00502435735389965\\
29.01	0.00502435862878807\\
30.01	0.00502435992849977\\
31.01	0.00502436125351623\\
32.01	0.00502436260432844\\
33.01	0.00502436398143678\\
34.01	0.00502436538535125\\
35.01	0.00502436681659189\\
36.01	0.00502436827568868\\
37.01	0.00502436976318171\\
38.01	0.00502437127962201\\
39.01	0.00502437282557061\\
40.01	0.00502437440159964\\
41.01	0.00502437600829278\\
42.01	0.00502437764624425\\
43.01	0.00502437931606002\\
44.01	0.00502438101835799\\
45.01	0.00502438275376783\\
46.01	0.00502438452293126\\
47.01	0.00502438632650271\\
48.01	0.00502438816514894\\
49.01	0.00502439003954984\\
50.01	0.0050243919503983\\
51.01	0.00502439389840066\\
52.01	0.00502439588427681\\
53.01	0.00502439790876083\\
54.01	0.00502439997260064\\
55.01	0.00502440207655885\\
56.01	0.00502440422141273\\
57.01	0.00502440640795451\\
58.01	0.00502440863699193\\
59.01	0.00502441090934809\\
60.01	0.00502441322586211\\
61.01	0.00502441558738942\\
62.01	0.0050244179948018\\
63.01	0.00502442044898818\\
64.01	0.00502442295085417\\
65.01	0.00502442550132326\\
66.01	0.00502442810133673\\
67.01	0.00502443075185414\\
68.01	0.00502443345385328\\
69.01	0.00502443620833122\\
70.01	0.00502443901630395\\
71.01	0.00502444187880739\\
72.01	0.00502444479689727\\
73.01	0.00502444777164984\\
74.01	0.00502445080416218\\
75.01	0.00502445389555228\\
76.01	0.00502445704696021\\
77.01	0.00502446025954768\\
78.01	0.00502446353449897\\
79.01	0.00502446687302142\\
80.01	0.00502447027634557\\
81.01	0.00502447374572551\\
82.01	0.00502447728244004\\
83.01	0.00502448088779259\\
84.01	0.00502448456311146\\
85.01	0.00502448830975082\\
86.01	0.0050244921290912\\
87.01	0.00502449602253954\\
88.01	0.00502449999152989\\
89.01	0.00502450403752445\\
90.01	0.00502450816201346\\
91.01	0.00502451236651598\\
92.01	0.00502451665258054\\
93.01	0.0050245210217854\\
94.01	0.00502452547573952\\
95.01	0.00502453001608305\\
96.01	0.00502453464448767\\
97.01	0.0050245393626575\\
98.01	0.00502454417232982\\
99.01	0.00502454907527516\\
100.01	0.00502455407329852\\
101.01	0.00502455916823986\\
102.01	0.00502456436197452\\
103.01	0.00502456965641446\\
104.01	0.00502457505350838\\
105.01	0.00502458055524285\\
106.01	0.00502458616364285\\
107.01	0.00502459188077253\\
108.01	0.00502459770873598\\
109.01	0.00502460364967827\\
110.01	0.00502460970578573\\
111.01	0.00502461587928722\\
112.01	0.00502462217245472\\
113.01	0.00502462858760408\\
114.01	0.00502463512709636\\
115.01	0.00502464179333817\\
116.01	0.00502464858878258\\
117.01	0.0050246555159305\\
118.01	0.00502466257733118\\
119.01	0.00502466977558323\\
120.01	0.00502467711333579\\
121.01	0.00502468459328946\\
122.01	0.00502469221819687\\
123.01	0.00502469999086443\\
124.01	0.0050247079141528\\
125.01	0.0050247159909782\\
126.01	0.00502472422431341\\
127.01	0.0050247326171889\\
128.01	0.0050247411726939\\
129.01	0.00502474989397785\\
130.01	0.00502475878425128\\
131.01	0.00502476784678701\\
132.01	0.00502477708492131\\
133.01	0.00502478650205564\\
134.01	0.00502479610165719\\
135.01	0.00502480588726081\\
136.01	0.0050248158624699\\
137.01	0.00502482603095807\\
138.01	0.00502483639647029\\
139.01	0.00502484696282447\\
140.01	0.00502485773391259\\
141.01	0.00502486871370277\\
142.01	0.00502487990624001\\
143.01	0.00502489131564829\\
144.01	0.00502490294613174\\
145.01	0.00502491480197648\\
146.01	0.00502492688755225\\
147.01	0.00502493920731371\\
148.01	0.00502495176580255\\
149.01	0.00502496456764895\\
150.01	0.00502497761757341\\
151.01	0.00502499092038851\\
152.01	0.00502500448100068\\
153.01	0.00502501830441233\\
154.01	0.00502503239572324\\
155.01	0.00502504676013321\\
156.01	0.0050250614029432\\
157.01	0.00502507632955788\\
158.01	0.00502509154548772\\
159.01	0.0050251070563508\\
160.01	0.00502512286787522\\
161.01	0.0050251389859009\\
162.01	0.00502515541638252\\
163.01	0.00502517216539074\\
164.01	0.00502518923911544\\
165.01	0.0050252066438678\\
166.01	0.00502522438608252\\
167.01	0.00502524247232077\\
168.01	0.00502526090927197\\
169.01	0.00502527970375694\\
170.01	0.00502529886273037\\
171.01	0.00502531839328352\\
172.01	0.00502533830264698\\
173.01	0.00502535859819302\\
174.01	0.00502537928743922\\
175.01	0.00502540037805077\\
176.01	0.00502542187784366\\
177.01	0.0050254437947877\\
178.01	0.00502546613700938\\
179.01	0.0050254889127955\\
180.01	0.00502551213059595\\
181.01	0.00502553579902699\\
182.01	0.00502555992687535\\
183.01	0.00502558452310024\\
184.01	0.00502560959683822\\
185.01	0.00502563515740609\\
186.01	0.00502566121430463\\
187.01	0.00502568777722227\\
188.01	0.00502571485603907\\
189.01	0.00502574246083047\\
190.01	0.00502577060187115\\
191.01	0.00502579928963928\\
192.01	0.00502582853482043\\
193.01	0.00502585834831201\\
194.01	0.00502588874122717\\
195.01	0.00502591972489998\\
196.01	0.00502595131088869\\
197.01	0.00502598351098166\\
198.01	0.00502601633720089\\
199.01	0.00502604980180751\\
200.01	0.00502608391730629\\
201.01	0.00502611869645081\\
202.01	0.0050261541522485\\
203.01	0.00502619029796553\\
204.01	0.0050262271471328\\
205.01	0.00502626471355053\\
206.01	0.00502630301129413\\
207.01	0.00502634205472018\\
208.01	0.00502638185847148\\
209.01	0.00502642243748359\\
210.01	0.00502646380699059\\
211.01	0.00502650598253092\\
212.01	0.00502654897995404\\
213.01	0.00502659281542676\\
214.01	0.00502663750543959\\
215.01	0.0050266830668135\\
216.01	0.00502672951670692\\
217.01	0.00502677687262224\\
218.01	0.00502682515241343\\
219.01	0.00502687437429299\\
220.01	0.00502692455683951\\
221.01	0.00502697571900508\\
222.01	0.00502702788012314\\
223.01	0.00502708105991628\\
224.01	0.00502713527850461\\
225.01	0.00502719055641328\\
226.01	0.00502724691458187\\
227.01	0.00502730437437221\\
228.01	0.00502736295757766\\
229.01	0.00502742268643143\\
230.01	0.00502748358361636\\
231.01	0.00502754567227415\\
232.01	0.00502760897601477\\
233.01	0.00502767351892637\\
234.01	0.00502773932558503\\
235.01	0.005027806421065\\
236.01	0.00502787483094894\\
237.01	0.00502794458133929\\
238.01	0.00502801569886797\\
239.01	0.00502808821070845\\
240.01	0.00502816214458639\\
241.01	0.0050282375287917\\
242.01	0.00502831439218987\\
243.01	0.00502839276423442\\
244.01	0.00502847267497887\\
245.01	0.00502855415508958\\
246.01	0.00502863723585828\\
247.01	0.00502872194921569\\
248.01	0.00502880832774445\\
249.01	0.00502889640469281\\
250.01	0.00502898621398901\\
251.01	0.00502907779025501\\
252.01	0.00502917116882139\\
253.01	0.00502926638574215\\
254.01	0.00502936347780983\\
255.01	0.00502946248257113\\
256.01	0.00502956343834272\\
257.01	0.00502966638422728\\
258.01	0.00502977136013015\\
259.01	0.00502987840677609\\
260.01	0.00502998756572667\\
261.01	0.00503009887939768\\
262.01	0.00503021239107712\\
263.01	0.00503032814494354\\
264.01	0.00503044618608481\\
265.01	0.00503056656051706\\
266.01	0.00503068931520437\\
267.01	0.00503081449807827\\
268.01	0.00503094215805883\\
269.01	0.00503107234507448\\
270.01	0.00503120511008398\\
271.01	0.00503134050509758\\
272.01	0.00503147858319909\\
273.01	0.00503161939856864\\
274.01	0.00503176300650546\\
275.01	0.00503190946345133\\
276.01	0.00503205882701463\\
277.01	0.00503221115599456\\
278.01	0.00503236651040657\\
279.01	0.0050325249515072\\
280.01	0.00503268654182049\\
281.01	0.00503285134516392\\
282.01	0.00503301942667632\\
283.01	0.00503319085284439\\
284.01	0.005033365691532\\
285.01	0.00503354401200766\\
286.01	0.00503372588497485\\
287.01	0.00503391138260131\\
288.01	0.00503410057854997\\
289.01	0.00503429354800942\\
290.01	0.00503449036772634\\
291.01	0.00503469111603755\\
292.01	0.00503489587290304\\
293.01	0.0050351047199394\\
294.01	0.00503531774045483\\
295.01	0.00503553501948349\\
296.01	0.00503575664382178\\
297.01	0.00503598270206433\\
298.01	0.0050362132846417\\
299.01	0.00503644848385791\\
300.01	0.00503668839392943\\
301.01	0.00503693311102424\\
302.01	0.00503718273330268\\
303.01	0.0050374373609581\\
304.01	0.00503769709625872\\
305.01	0.00503796204359043\\
306.01	0.00503823230950071\\
307.01	0.00503850800274236\\
308.01	0.00503878923431933\\
309.01	0.00503907611753321\\
310.01	0.00503936876802981\\
311.01	0.00503966730384769\\
312.01	0.00503997184546696\\
313.01	0.00504028251585979\\
314.01	0.00504059944054119\\
315.01	0.00504092274762133\\
316.01	0.0050412525678584\\
317.01	0.00504158903471354\\
318.01	0.00504193228440589\\
319.01	0.00504228245596902\\
320.01	0.00504263969130883\\
321.01	0.00504300413526243\\
322.01	0.00504337593565841\\
323.01	0.00504375524337804\\
324.01	0.00504414221241793\\
325.01	0.00504453699995414\\
326.01	0.00504493976640799\\
327.01	0.00504535067551151\\
328.01	0.00504576989437699\\
329.01	0.00504619759356566\\
330.01	0.00504663394715922\\
331.01	0.00504707913283256\\
332.01	0.00504753333192737\\
333.01	0.0050479967295282\\
334.01	0.00504846951454033\\
335.01	0.00504895187976848\\
336.01	0.00504944402199728\\
337.01	0.00504994614207464\\
338.01	0.00505045844499537\\
339.01	0.00505098113998795\\
340.01	0.00505151444060255\\
341.01	0.00505205856480088\\
342.01	0.00505261373504817\\
343.01	0.00505318017840738\\
344.01	0.00505375812663522\\
345.01	0.00505434781628047\\
346.01	0.00505494948878352\\
347.01	0.00505556339057943\\
348.01	0.0050561897732026\\
349.01	0.00505682889339344\\
350.01	0.00505748101320719\\
351.01	0.00505814640012613\\
352.01	0.0050588253271726\\
353.01	0.00505951807302557\\
354.01	0.00506022492213949\\
355.01	0.0050609461648646\\
356.01	0.00506168209757112\\
357.01	0.00506243302277486\\
358.01	0.0050631992492661\\
359.01	0.00506398109224075\\
360.01	0.00506477887343372\\
361.01	0.00506559292125611\\
362.01	0.00506642357093377\\
363.01	0.0050672711646495\\
364.01	0.00506813605168775\\
365.01	0.00506901858858232\\
366.01	0.00506991913926619\\
367.01	0.00507083807522581\\
368.01	0.00507177577565693\\
369.01	0.00507273262762459\\
370.01	0.0050737090262252\\
371.01	0.00507470537475273\\
372.01	0.00507572208486796\\
373.01	0.0050767595767709\\
374.01	0.00507781827937674\\
375.01	0.00507889863049542\\
376.01	0.00508000107701456\\
377.01	0.00508112607508641\\
378.01	0.00508227409031825\\
379.01	0.00508344559796643\\
380.01	0.00508464108313535\\
381.01	0.00508586104097879\\
382.01	0.00508710597690695\\
383.01	0.00508837640679624\\
384.01	0.0050896728572046\\
385.01	0.00509099586558949\\
386.01	0.00509234598053222\\
387.01	0.00509372376196553\\
388.01	0.00509512978140565\\
389.01	0.00509656462218988\\
390.01	0.0050980288797183\\
391.01	0.00509952316170056\\
392.01	0.00510104808840786\\
393.01	0.00510260429292941\\
394.01	0.00510419242143478\\
395.01	0.00510581313344144\\
396.01	0.00510746710208676\\
397.01	0.00510915501440734\\
398.01	0.00511087757162237\\
399.01	0.00511263548942352\\
400.01	0.00511442949827068\\
401.01	0.00511626034369345\\
402.01	0.0051181287865992\\
403.01	0.00512003560358712\\
404.01	0.00512198158726905\\
405.01	0.00512396754659643\\
406.01	0.00512599430719415\\
407.01	0.00512806271170159\\
408.01	0.00513017362012046\\
409.01	0.00513232791016949\\
410.01	0.00513452647764723\\
411.01	0.00513677023680156\\
412.01	0.00513906012070751\\
413.01	0.0051413970816525\\
414.01	0.00514378209152984\\
415.01	0.00514621614224034\\
416.01	0.00514870024610219\\
417.01	0.00515123543627008\\
418.01	0.00515382276716167\\
419.01	0.00515646331489446\\
420.01	0.00515915817773094\\
421.01	0.00516190847653376\\
422.01	0.00516471535522936\\
423.01	0.0051675799812823\\
424.01	0.00517050354617916\\
425.01	0.00517348726592291\\
426.01	0.00517653238153705\\
427.01	0.00517964015958057\\
428.01	0.00518281189267433\\
429.01	0.00518604890003725\\
430.01	0.00518935252803483\\
431.01	0.00519272415073881\\
432.01	0.00519616517049813\\
433.01	0.00519967701852253\\
434.01	0.00520326115547738\\
435.01	0.00520691907209199\\
436.01	0.00521065228978022\\
437.01	0.00521446236127313\\
438.01	0.00521835087126643\\
439.01	0.00522231943707985\\
440.01	0.0052263697093313\\
441.01	0.00523050337262421\\
442.01	0.00523472214624948\\
443.01	0.00523902778490222\\
444.01	0.00524342207941315\\
445.01	0.00524790685749485\\
446.01	0.00525248398450374\\
447.01	0.00525715536421802\\
448.01	0.00526192293963146\\
449.01	0.00526678869376357\\
450.01	0.00527175465048671\\
451.01	0.00527682287537118\\
452.01	0.00528199547654581\\
453.01	0.0052872746055784\\
454.01	0.00529266245837408\\
455.01	0.00529816127609135\\
456.01	0.00530377334607839\\
457.01	0.00530950100282836\\
458.01	0.00531534662895433\\
459.01	0.00532131265618531\\
460.01	0.00532740156638227\\
461.01	0.00533361589257574\\
462.01	0.0053399582200252\\
463.01	0.00534643118730034\\
464.01	0.0053530374873846\\
465.01	0.00535977986880204\\
466.01	0.00536666113676787\\
467.01	0.00537368415436255\\
468.01	0.00538085184373044\\
469.01	0.00538816718730339\\
470.01	0.00539563322904991\\
471.01	0.00540325307574993\\
472.01	0.00541102989829671\\
473.01	0.00541896693302511\\
474.01	0.00542706748306775\\
475.01	0.00543533491973951\\
476.01	0.00544377268394998\\
477.01	0.00545238428764663\\
478.01	0.00546117331528623\\
479.01	0.0054701434253375\\
480.01	0.0054792983518152\\
481.01	0.00548864190584508\\
482.01	0.00549817797726253\\
483.01	0.00550791053624219\\
484.01	0.00551784363496273\\
485.01	0.0055279814093054\\
486.01	0.00553832808058766\\
487.01	0.00554888795733173\\
488.01	0.0055596654370699\\
489.01	0.00557066500818651\\
490.01	0.00558189125179769\\
491.01	0.00559334884366847\\
492.01	0.00560504255616977\\
493.01	0.00561697726027399\\
494.01	0.00562915792759081\\
495.01	0.00564158963244404\\
496.01	0.00565427755398963\\
497.01	0.00566722697837492\\
498.01	0.00568044330094133\\
499.01	0.00569393202846911\\
500.01	0.00570769878146611\\
501.01	0.00572174929650032\\
502.01	0.00573608942857665\\
503.01	0.00575072515355798\\
504.01	0.00576566257063221\\
505.01	0.00578090790482362\\
506.01	0.00579646750954958\\
507.01	0.00581234786922323\\
508.01	0.00582855560190046\\
509.01	0.00584509746197324\\
510.01	0.00586198034290671\\
511.01	0.00587921128002072\\
512.01	0.00589679745331509\\
513.01	0.00591474619033792\\
514.01	0.00593306496909482\\
515.01	0.00595176142099829\\
516.01	0.0059708433338565\\
517.01	0.00599031865489785\\
518.01	0.00601019549382988\\
519.01	0.00603048212592962\\
520.01	0.00605118699516193\\
521.01	0.00607231871732236\\
522.01	0.0060938860831997\\
523.01	0.00611589806175381\\
524.01	0.00613836380330246\\
525.01	0.00616129264271094\\
526.01	0.00618469410257619\\
527.01	0.00620857789639841\\
528.01	0.006232953931729\\
529.01	0.00625783231328467\\
530.01	0.00628322334601478\\
531.01	0.00630913753810876\\
532.01	0.0063355856039277\\
533.01	0.00636257846684055\\
534.01	0.00639012726194826\\
535.01	0.00641824333867082\\
536.01	0.00644693826317301\\
537.01	0.00647622382060102\\
538.01	0.00650611201709725\\
539.01	0.00653661508155973\\
540.01	0.00656774546710449\\
541.01	0.00659951585218799\\
542.01	0.00663193914134093\\
543.01	0.00666502846545589\\
544.01	0.00669879718156972\\
545.01	0.00673325887207067\\
546.01	0.00676842734325422\\
547.01	0.00680431662314229\\
548.01	0.00684094095847129\\
549.01	0.00687831481074262\\
550.01	0.00691645285122013\\
551.01	0.00695536995474208\\
552.01	0.00699508119220391\\
553.01	0.00703560182155209\\
554.01	0.00707694727710953\\
555.01	0.00711913315703655\\
556.01	0.00716217520870922\\
557.01	0.00720608931177373\\
558.01	0.00725089145861135\\
559.01	0.0072965977319211\\
560.01	0.00734322427909608\\
561.01	0.00739078728304045\\
562.01	0.00743930292903837\\
563.01	0.00748878736724826\\
564.01	0.00753925667036025\\
565.01	0.00759072678591111\\
566.01	0.0076432134827101\\
567.01	0.00769673229078503\\
568.01	0.00775129843421511\\
569.01	0.00780692675617325\\
570.01	0.00786363163546109\\
571.01	0.0079214268937827\\
572.01	0.00798032569297568\\
573.01	0.00804034042139869\\
574.01	0.00810148256867251\\
575.01	0.00816376258798929\\
576.01	0.00822718974525323\\
577.01	0.00829177195440151\\
578.01	0.00835751559839206\\
579.01	0.00842442533555191\\
580.01	0.00849250389127248\\
581.01	0.00856175183544577\\
582.01	0.00863216734659208\\
583.01	0.00870374596436989\\
584.01	0.00877648033314344\\
585.01	0.00885035994056966\\
586.01	0.00892537085684264\\
587.01	0.0090014954823987\\
588.01	0.00907871231467107\\
589.01	0.0091569957480491\\
590.01	0.00923631592574849\\
591.01	0.00931663866808596\\
592.01	0.00939792550899438\\
593.01	0.00948013388190589\\
594.01	0.00956321750786664\\
595.01	0.00964712705354462\\
596.01	0.00973181114542127\\
597.01	0.00981721784988034\\
598.01	0.00990284350259786\\
599.01	0.00996919203046377\\
599.02	0.00996973314335038\\
599.03	0.0099702709571694\\
599.04	0.00997080543920336\\
599.05	0.0099713365564138\\
599.06	0.00997186427543808\\
599.07	0.0099723885625862\\
599.08	0.00997290938383756\\
599.09	0.00997342670483771\\
599.1	0.00997394049089505\\
599.11	0.00997445070697751\\
599.12	0.00997495731770918\\
599.13	0.00997546028736696\\
599.14	0.00997595957987707\\
599.15	0.00997645515881166\\
599.16	0.00997694698738526\\
599.17	0.00997743502845131\\
599.18	0.00997791924449856\\
599.19	0.00997839959764748\\
599.2	0.00997887604964662\\
599.21	0.00997934856186899\\
599.22	0.00997981709530829\\
599.23	0.00998028161057521\\
599.24	0.00998074206789365\\
599.25	0.0099811984270969\\
599.26	0.00998165064762378\\
599.27	0.00998209868851477\\
599.28	0.00998254250840806\\
599.29	0.00998298206553559\\
599.3	0.00998341731771905\\
599.31	0.00998384822236584\\
599.32	0.00998427473646497\\
599.33	0.00998469681658292\\
599.34	0.00998511441885952\\
599.35	0.0099855274990037\\
599.36	0.00998593601228926\\
599.37	0.00998633991355056\\
599.38	0.00998673915717821\\
599.39	0.00998713369711469\\
599.4	0.00998752348684992\\
599.41	0.00998790847811157\\
599.42	0.00998828861974491\\
599.43	0.00998866386008804\\
599.44	0.00998903414696681\\
599.45	0.00998939942768986\\
599.46	0.00998975964904344\\
599.47	0.00999011475728636\\
599.48	0.00999046469814472\\
599.49	0.00999080941680669\\
599.5	0.00999114885791725\\
599.51	0.00999148296557278\\
599.52	0.0099918116833157\\
599.53	0.00999213495412901\\
599.54	0.00999245272043077\\
599.55	0.00999276492406855\\
599.56	0.00999307150631381\\
599.57	0.00999337240785624\\
599.58	0.00999366756879799\\
599.59	0.00999395692864795\\
599.6	0.00999424042631586\\
599.61	0.00999451800010642\\
599.62	0.00999478958771336\\
599.63	0.0099950551262134\\
599.64	0.00999531455206019\\
599.65	0.00999556780107816\\
599.66	0.00999581480845634\\
599.67	0.00999605550874211\\
599.68	0.00999628983583487\\
599.69	0.00999651772297967\\
599.7	0.00999673910276076\\
599.71	0.00999695390709509\\
599.72	0.00999716206722577\\
599.73	0.00999736351371538\\
599.74	0.00999755817643933\\
599.75	0.00999774598457904\\
599.76	0.00999792686661517\\
599.77	0.00999810075032068\\
599.78	0.00999826756275386\\
599.79	0.00999842723025133\\
599.8	0.00999857967842092\\
599.81	0.0099987248321345\\
599.82	0.00999886261552071\\
599.83	0.00999899295195769\\
599.84	0.00999911576406567\\
599.85	0.00999923097369951\\
599.86	0.00999933850194117\\
599.87	0.00999943826909211\\
599.88	0.00999953019466558\\
599.89	0.00999961419737891\\
599.9	0.00999969019514566\\
599.91	0.00999975810506767\\
599.92	0.00999981784342713\\
599.93	0.00999986932567848\\
599.94	0.00999991246644028\\
599.95	0.00999994717948697\\
599.96	0.00999997337774056\\
599.97	0.00999999097326229\\
599.98	0.00999999987724406\\
599.99	0.01\\
600	0.01\\
};
\addplot [color=mycolor3,solid,forget plot]
  table[row sep=crcr]{%
0.01	0.00502269192807735\\
1.01	0.00502269275816619\\
2.01	0.00502269360447714\\
3.01	0.00502269446732508\\
4.01	0.00502269534703129\\
5.01	0.00502269624392311\\
6.01	0.00502269715833426\\
7.01	0.00502269809060478\\
8.01	0.00502269904108132\\
9.01	0.00502270001011726\\
10.01	0.00502270099807273\\
11.01	0.00502270200531477\\
12.01	0.00502270303221747\\
13.01	0.00502270407916229\\
14.01	0.00502270514653784\\
15.01	0.00502270623474015\\
16.01	0.00502270734417315\\
17.01	0.00502270847524855\\
18.01	0.00502270962838561\\
19.01	0.00502271080401182\\
20.01	0.00502271200256298\\
21.01	0.00502271322448329\\
22.01	0.00502271447022537\\
23.01	0.00502271574025047\\
24.01	0.00502271703502895\\
25.01	0.00502271835504012\\
26.01	0.00502271970077223\\
27.01	0.00502272107272335\\
28.01	0.00502272247140077\\
29.01	0.00502272389732167\\
30.01	0.0050227253510132\\
31.01	0.00502272683301236\\
32.01	0.00502272834386676\\
33.01	0.00502272988413437\\
34.01	0.00502273145438391\\
35.01	0.00502273305519495\\
36.01	0.00502273468715808\\
37.01	0.00502273635087555\\
38.01	0.00502273804696077\\
39.01	0.00502273977603926\\
40.01	0.00502274153874836\\
41.01	0.00502274333573718\\
42.01	0.00502274516766807\\
43.01	0.00502274703521559\\
44.01	0.00502274893906715\\
45.01	0.00502275087992363\\
46.01	0.00502275285849908\\
47.01	0.00502275487552123\\
48.01	0.00502275693173183\\
49.01	0.00502275902788675\\
50.01	0.00502276116475632\\
51.01	0.00502276334312574\\
52.01	0.00502276556379504\\
53.01	0.00502276782757959\\
54.01	0.00502277013531031\\
55.01	0.00502277248783412\\
56.01	0.00502277488601388\\
57.01	0.00502277733072924\\
58.01	0.00502277982287644\\
59.01	0.00502278236336892\\
60.01	0.00502278495313754\\
61.01	0.00502278759313086\\
62.01	0.00502279028431567\\
63.01	0.00502279302767693\\
64.01	0.00502279582421873\\
65.01	0.00502279867496436\\
66.01	0.0050228015809561\\
67.01	0.00502280454325642\\
68.01	0.00502280756294829\\
69.01	0.00502281064113506\\
70.01	0.00502281377894102\\
71.01	0.0050228169775119\\
72.01	0.00502282023801544\\
73.01	0.00502282356164153\\
74.01	0.00502282694960268\\
75.01	0.00502283040313457\\
76.01	0.00502283392349639\\
77.01	0.00502283751197109\\
78.01	0.00502284116986636\\
79.01	0.00502284489851453\\
80.01	0.00502284869927325\\
81.01	0.00502285257352628\\
82.01	0.00502285652268339\\
83.01	0.00502286054818123\\
84.01	0.00502286465148379\\
85.01	0.00502286883408294\\
86.01	0.00502287309749881\\
87.01	0.00502287744328052\\
88.01	0.00502288187300651\\
89.01	0.00502288638828523\\
90.01	0.00502289099075573\\
91.01	0.00502289568208819\\
92.01	0.00502290046398435\\
93.01	0.00502290533817826\\
94.01	0.00502291030643741\\
95.01	0.00502291537056186\\
96.01	0.00502292053238662\\
97.01	0.00502292579378125\\
98.01	0.00502293115665067\\
99.01	0.00502293662293596\\
100.01	0.00502294219461531\\
101.01	0.00502294787370401\\
102.01	0.0050229536622559\\
103.01	0.00502295956236332\\
104.01	0.00502296557615868\\
105.01	0.00502297170581455\\
106.01	0.00502297795354477\\
107.01	0.00502298432160513\\
108.01	0.00502299081229411\\
109.01	0.00502299742795358\\
110.01	0.00502300417096997\\
111.01	0.00502301104377507\\
112.01	0.00502301804884617\\
113.01	0.00502302518870795\\
114.01	0.00502303246593248\\
115.01	0.00502303988314082\\
116.01	0.00502304744300355\\
117.01	0.00502305514824167\\
118.01	0.00502306300162776\\
119.01	0.0050230710059868\\
120.01	0.00502307916419732\\
121.01	0.00502308747919222\\
122.01	0.00502309595395981\\
123.01	0.00502310459154491\\
124.01	0.00502311339505002\\
125.01	0.00502312236763623\\
126.01	0.00502313151252447\\
127.01	0.00502314083299653\\
128.01	0.00502315033239616\\
129.01	0.00502316001413047\\
130.01	0.00502316988167072\\
131.01	0.00502317993855393\\
132.01	0.00502319018838419\\
133.01	0.00502320063483327\\
134.01	0.00502321128164261\\
135.01	0.00502322213262422\\
136.01	0.00502323319166223\\
137.01	0.00502324446271411\\
138.01	0.00502325594981212\\
139.01	0.00502326765706478\\
140.01	0.0050232795886581\\
141.01	0.0050232917488572\\
142.01	0.00502330414200786\\
143.01	0.00502331677253788\\
144.01	0.00502332964495891\\
145.01	0.00502334276386766\\
146.01	0.00502335613394771\\
147.01	0.00502336975997114\\
148.01	0.0050233836468003\\
149.01	0.0050233977993893\\
150.01	0.00502341222278592\\
151.01	0.0050234269221333\\
152.01	0.00502344190267173\\
153.01	0.00502345716974069\\
154.01	0.00502347272878047\\
155.01	0.00502348858533407\\
156.01	0.00502350474504953\\
157.01	0.00502352121368163\\
158.01	0.00502353799709374\\
159.01	0.00502355510126019\\
160.01	0.00502357253226837\\
161.01	0.0050235902963208\\
162.01	0.00502360839973709\\
163.01	0.00502362684895667\\
164.01	0.00502364565054074\\
165.01	0.00502366481117458\\
166.01	0.00502368433766998\\
167.01	0.00502370423696777\\
168.01	0.00502372451614012\\
169.01	0.00502374518239317\\
170.01	0.0050237662430692\\
171.01	0.00502378770564997\\
172.01	0.00502380957775894\\
173.01	0.00502383186716404\\
174.01	0.00502385458178017\\
175.01	0.0050238777296725\\
176.01	0.00502390131905903\\
177.01	0.00502392535831374\\
178.01	0.00502394985596955\\
179.01	0.00502397482072109\\
180.01	0.0050240002614282\\
181.01	0.00502402618711899\\
182.01	0.00502405260699289\\
183.01	0.00502407953042421\\
184.01	0.00502410696696538\\
185.01	0.0050241349263505\\
186.01	0.00502416341849879\\
187.01	0.00502419245351826\\
188.01	0.00502422204170916\\
189.01	0.00502425219356792\\
190.01	0.00502428291979097\\
191.01	0.00502431423127846\\
192.01	0.00502434613913834\\
193.01	0.00502437865469038\\
194.01	0.00502441178947051\\
195.01	0.00502444555523426\\
196.01	0.00502447996396256\\
197.01	0.0050245150278644\\
198.01	0.00502455075938266\\
199.01	0.00502458717119787\\
200.01	0.00502462427623343\\
201.01	0.00502466208765993\\
202.01	0.00502470061890026\\
203.01	0.00502473988363467\\
204.01	0.00502477989580546\\
205.01	0.00502482066962262\\
206.01	0.00502486221956912\\
207.01	0.00502490456040573\\
208.01	0.00502494770717721\\
209.01	0.00502499167521754\\
210.01	0.00502503648015548\\
211.01	0.00502508213792142\\
212.01	0.00502512866475206\\
213.01	0.00502517607719771\\
214.01	0.0050252243921276\\
215.01	0.00502527362673716\\
216.01	0.00502532379855358\\
217.01	0.00502537492544335\\
218.01	0.00502542702561851\\
219.01	0.00502548011764381\\
220.01	0.00502553422044383\\
221.01	0.00502558935331021\\
222.01	0.00502564553590917\\
223.01	0.00502570278828886\\
224.01	0.00502576113088737\\
225.01	0.00502582058454044\\
226.01	0.00502588117048948\\
227.01	0.00502594291039012\\
228.01	0.0050260058263203\\
229.01	0.00502606994078937\\
230.01	0.00502613527674647\\
231.01	0.00502620185758955\\
232.01	0.00502626970717483\\
233.01	0.00502633884982615\\
234.01	0.00502640931034432\\
235.01	0.0050264811140172\\
236.01	0.00502655428662993\\
237.01	0.00502662885447454\\
238.01	0.00502670484436112\\
239.01	0.00502678228362818\\
240.01	0.00502686120015372\\
241.01	0.00502694162236637\\
242.01	0.005027023579257\\
243.01	0.0050271071003903\\
244.01	0.00502719221591665\\
245.01	0.00502727895658453\\
246.01	0.00502736735375309\\
247.01	0.00502745743940455\\
248.01	0.00502754924615793\\
249.01	0.00502764280728168\\
250.01	0.00502773815670789\\
251.01	0.00502783532904608\\
252.01	0.00502793435959722\\
253.01	0.00502803528436875\\
254.01	0.00502813814008939\\
255.01	0.00502824296422411\\
256.01	0.0050283497949901\\
257.01	0.00502845867137251\\
258.01	0.00502856963314093\\
259.01	0.00502868272086569\\
260.01	0.00502879797593533\\
261.01	0.00502891544057366\\
262.01	0.00502903515785794\\
263.01	0.0050291571717366\\
264.01	0.00502928152704825\\
265.01	0.00502940826954061\\
266.01	0.00502953744588969\\
267.01	0.00502966910371992\\
268.01	0.00502980329162449\\
269.01	0.00502994005918601\\
270.01	0.00503007945699756\\
271.01	0.00503022153668439\\
272.01	0.00503036635092638\\
273.01	0.00503051395348038\\
274.01	0.00503066439920342\\
275.01	0.00503081774407646\\
276.01	0.00503097404522812\\
277.01	0.00503113336095971\\
278.01	0.00503129575077038\\
279.01	0.0050314612753828\\
280.01	0.00503162999676922\\
281.01	0.00503180197817885\\
282.01	0.00503197728416456\\
283.01	0.00503215598061148\\
284.01	0.00503233813476518\\
285.01	0.00503252381526131\\
286.01	0.00503271309215458\\
287.01	0.00503290603695017\\
288.01	0.00503310272263373\\
289.01	0.00503330322370403\\
290.01	0.00503350761620454\\
291.01	0.00503371597775681\\
292.01	0.00503392838759395\\
293.01	0.00503414492659505\\
294.01	0.00503436567732025\\
295.01	0.00503459072404615\\
296.01	0.00503482015280256\\
297.01	0.00503505405140962\\
298.01	0.00503529250951544\\
299.01	0.00503553561863494\\
300.01	0.005035783472189\\
301.01	0.005036036165545\\
302.01	0.00503629379605716\\
303.01	0.00503655646310849\\
304.01	0.00503682426815327\\
305.01	0.00503709731476035\\
306.01	0.00503737570865702\\
307.01	0.00503765955777422\\
308.01	0.005037948972292\\
309.01	0.00503824406468613\\
310.01	0.00503854494977575\\
311.01	0.00503885174477148\\
312.01	0.00503916456932479\\
313.01	0.00503948354557794\\
314.01	0.00503980879821511\\
315.01	0.00504014045451451\\
316.01	0.0050404786444011\\
317.01	0.00504082350050034\\
318.01	0.00504117515819326\\
319.01	0.00504153375567223\\
320.01	0.00504189943399799\\
321.01	0.00504227233715752\\
322.01	0.00504265261212303\\
323.01	0.00504304040891196\\
324.01	0.00504343588064847\\
325.01	0.00504383918362587\\
326.01	0.00504425047736964\\
327.01	0.00504466992470332\\
328.01	0.00504509769181376\\
329.01	0.00504553394831879\\
330.01	0.00504597886733602\\
331.01	0.00504643262555302\\
332.01	0.00504689540329863\\
333.01	0.00504736738461671\\
334.01	0.00504784875733993\\
335.01	0.00504833971316666\\
336.01	0.0050488404477388\\
337.01	0.00504935116072139\\
338.01	0.00504987205588395\\
339.01	0.00505040334118401\\
340.01	0.00505094522885204\\
341.01	0.00505149793547943\\
342.01	0.00505206168210733\\
343.01	0.00505263669431827\\
344.01	0.00505322320233027\\
345.01	0.00505382144109255\\
346.01	0.00505443165038472\\
347.01	0.00505505407491719\\
348.01	0.0050556889644352\\
349.01	0.00505633657382473\\
350.01	0.0050569971632219\\
351.01	0.00505767099812442\\
352.01	0.00505835834950605\\
353.01	0.00505905949393413\\
354.01	0.00505977471368994\\
355.01	0.00506050429689232\\
356.01	0.00506124853762378\\
357.01	0.0050620077360596\\
358.01	0.00506278219860108\\
359.01	0.00506357223801002\\
360.01	0.00506437817354806\\
361.01	0.00506520033111743\\
362.01	0.0050660390434064\\
363.01	0.00506689465003635\\
364.01	0.00506776749771209\\
365.01	0.00506865794037575\\
366.01	0.00506956633936299\\
367.01	0.00507049306356233\\
368.01	0.00507143848957758\\
369.01	0.00507240300189217\\
370.01	0.00507338699303843\\
371.01	0.00507439086376758\\
372.01	0.00507541502322463\\
373.01	0.00507645988912479\\
374.01	0.00507752588793453\\
375.01	0.00507861345505498\\
376.01	0.0050797230350091\\
377.01	0.00508085508163242\\
378.01	0.00508201005826769\\
379.01	0.00508318843796281\\
380.01	0.00508439070367284\\
381.01	0.00508561734846642\\
382.01	0.00508686887573492\\
383.01	0.00508814579940786\\
384.01	0.00508944864417014\\
385.01	0.00509077794568545\\
386.01	0.00509213425082232\\
387.01	0.00509351811788605\\
388.01	0.00509493011685437\\
389.01	0.0050963708296172\\
390.01	0.00509784085022219\\
391.01	0.00509934078512408\\
392.01	0.00510087125343914\\
393.01	0.00510243288720465\\
394.01	0.00510402633164277\\
395.01	0.00510565224543046\\
396.01	0.0051073113009738\\
397.01	0.00510900418468772\\
398.01	0.00511073159728134\\
399.01	0.00511249425404832\\
400.01	0.00511429288516371\\
401.01	0.0051161282359854\\
402.01	0.00511800106736243\\
403.01	0.00511991215594831\\
404.01	0.00512186229452122\\
405.01	0.00512385229231005\\
406.01	0.00512588297532704\\
407.01	0.00512795518670606\\
408.01	0.00513006978704923\\
409.01	0.00513222765477883\\
410.01	0.00513442968649672\\
411.01	0.00513667679735155\\
412.01	0.00513896992141193\\
413.01	0.00514131001204859\\
414.01	0.00514369804232345\\
415.01	0.00514613500538642\\
416.01	0.005148621914881\\
417.01	0.00515115980535756\\
418.01	0.00515374973269582\\
419.01	0.00515639277453533\\
420.01	0.00515909003071588\\
421.01	0.00516184262372579\\
422.01	0.00516465169916107\\
423.01	0.00516751842619395\\
424.01	0.00517044399805046\\
425.01	0.00517342963249932\\
426.01	0.00517647657235074\\
427.01	0.00517958608596648\\
428.01	0.0051827594677801\\
429.01	0.00518599803882874\\
430.01	0.00518930314729744\\
431.01	0.00519267616907353\\
432.01	0.00519611850831445\\
433.01	0.00519963159802705\\
434.01	0.00520321690065996\\
435.01	0.00520687590870869\\
436.01	0.00521061014533324\\
437.01	0.00521442116498987\\
438.01	0.00521831055407579\\
439.01	0.00522227993158805\\
440.01	0.00522633094979617\\
441.01	0.00523046529492927\\
442.01	0.00523468468787772\\
443.01	0.00523899088490927\\
444.01	0.00524338567839996\\
445.01	0.00524787089758146\\
446.01	0.00525244840930239\\
447.01	0.00525712011880645\\
448.01	0.0052618879705259\\
449.01	0.00526675394889162\\
450.01	0.00527172007916012\\
451.01	0.00527678842825666\\
452.01	0.0052819611056364\\
453.01	0.00528724026416325\\
454.01	0.00529262810100554\\
455.01	0.00529812685855215\\
456.01	0.00530373882534568\\
457.01	0.00530946633703643\\
458.01	0.00531531177735521\\
459.01	0.00532127757910709\\
460.01	0.00532736622518554\\
461.01	0.00533358024960859\\
462.01	0.00533992223857581\\
463.01	0.00534639483154867\\
464.01	0.00535300072235285\\
465.01	0.00535974266030467\\
466.01	0.00536662345136033\\
467.01	0.00537364595929006\\
468.01	0.00538081310687699\\
469.01	0.00538812787714035\\
470.01	0.00539559331458557\\
471.01	0.00540321252647991\\
472.01	0.00541098868415421\\
473.01	0.00541892502433317\\
474.01	0.00542702485049187\\
475.01	0.00543529153424156\\
476.01	0.00544372851674385\\
477.01	0.00545233931015376\\
478.01	0.00546112749909385\\
479.01	0.00547009674215762\\
480.01	0.00547925077344463\\
481.01	0.0054885934041268\\
482.01	0.00549812852404753\\
483.01	0.00550786010335333\\
484.01	0.00551779219415929\\
485.01	0.00552792893224861\\
486.01	0.0055382745388064\\
487.01	0.00554883332219024\\
488.01	0.00555960967973582\\
489.01	0.00557060809959947\\
490.01	0.00558183316263904\\
491.01	0.00559328954433222\\
492.01	0.00560498201673348\\
493.01	0.00561691545047095\\
494.01	0.00562909481678298\\
495.01	0.00564152518959594\\
496.01	0.00565421174764243\\
497.01	0.0056671597766221\\
498.01	0.00568037467140432\\
499.01	0.00569386193827418\\
500.01	0.00570762719722163\\
501.01	0.00572167618427445\\
502.01	0.00573601475387593\\
503.01	0.00575064888130682\\
504.01	0.00576558466515187\\
505.01	0.00578082832981236\\
506.01	0.0057963862280633\\
507.01	0.00581226484365609\\
508.01	0.00582847079396645\\
509.01	0.00584501083268776\\
510.01	0.00586189185256834\\
511.01	0.00587912088819392\\
512.01	0.0058967051188127\\
513.01	0.00591465187120406\\
514.01	0.00593296862258812\\
515.01	0.00595166300357612\\
516.01	0.0059707428011586\\
517.01	0.00599021596173088\\
518.01	0.00601009059415211\\
519.01	0.00603037497283576\\
520.01	0.00605107754086892\\
521.01	0.0060722069131553\\
522.01	0.00609377187957812\\
523.01	0.00611578140817863\\
524.01	0.00613824464834352\\
525.01	0.00616117093399438\\
526.01	0.00618456978677384\\
527.01	0.00620845091921686\\
528.01	0.00623282423789978\\
529.01	0.00625769984655496\\
530.01	0.00628308804914007\\
531.01	0.00630899935284569\\
532.01	0.00633544447102788\\
533.01	0.00636243432604691\\
534.01	0.00638998005199212\\
535.01	0.00641809299727018\\
536.01	0.00644678472703378\\
537.01	0.00647606702541922\\
538.01	0.00650595189756438\\
539.01	0.00653645157136962\\
540.01	0.00656757849896372\\
541.01	0.00659934535782969\\
542.01	0.00663176505154073\\
543.01	0.00666485071005252\\
544.01	0.00669861568948972\\
545.01	0.00673307357135754\\
546.01	0.00676823816110224\\
547.01	0.00680412348593676\\
548.01	0.00684074379183394\\
549.01	0.00687811353958502\\
550.01	0.00691624739980339\\
551.01	0.0069551602467454\\
552.01	0.0069948671508024\\
553.01	0.00703538336950402\\
554.01	0.00707672433685472\\
555.01	0.00711890565080693\\
556.01	0.00716194305865242\\
557.01	0.00720585244009135\\
558.01	0.00725064978771305\\
559.01	0.00729635118459537\\
560.01	0.00734297277870056\\
561.01	0.00739053075371283\\
562.01	0.00743904129592875\\
563.01	0.00748852055677669\\
564.01	0.00753898461049995\\
565.01	0.00759044940649993\\
566.01	0.00764293071579204\\
567.01	0.00769644407098446\\
568.01	0.00775100469914526\\
569.01	0.00780662744688079\\
570.01	0.0078633266969083\\
571.01	0.00792111627536951\\
572.01	0.00798000934910111\\
573.01	0.00804001831206291\\
574.01	0.00810115466011737\\
575.01	0.0081634288533764\\
576.01	0.00822685016537466\\
577.01	0.00829142651841632\\
578.01	0.00835716430458038\\
579.01	0.00842406819207157\\
580.01	0.00849214091689975\\
581.01	0.0085613830602753\\
582.01	0.00863179281266063\\
583.01	0.00870336572616018\\
584.01	0.00877609445790773\\
585.01	0.00884996850839709\\
586.01	0.00892497396037102\\
587.01	0.00900109322604318\\
588.01	0.00907830481320621\\
589.01	0.00915658312433641\\
590.01	0.00923589830734327\\
591.01	0.00931621618238686\\
592.01	0.00939749827650748\\
593.01	0.00947970200708216\\
594.01	0.00956278106682799\\
595.01	0.0096466860778335\\
596.01	0.00973136560068165\\
597.01	0.00981676760809251\\
598.01	0.00990268918671883\\
599.01	0.00996919203046377\\
599.02	0.00996973314335038\\
599.03	0.0099702709571694\\
599.04	0.00997080543920336\\
599.05	0.0099713365564138\\
599.06	0.00997186427543808\\
599.07	0.0099723885625862\\
599.08	0.00997290938383756\\
599.09	0.00997342670483771\\
599.1	0.00997394049089505\\
599.11	0.00997445070697751\\
599.12	0.00997495731770918\\
599.13	0.00997546028736696\\
599.14	0.00997595957987707\\
599.15	0.00997645515881165\\
599.16	0.00997694698738526\\
599.17	0.00997743502845131\\
599.18	0.00997791924449856\\
599.19	0.00997839959764748\\
599.2	0.00997887604964662\\
599.21	0.00997934856186899\\
599.22	0.00997981709530829\\
599.23	0.00998028161057521\\
599.24	0.00998074206789365\\
599.25	0.0099811984270969\\
599.26	0.00998165064762378\\
599.27	0.00998209868851477\\
599.28	0.00998254250840806\\
599.29	0.00998298206553559\\
599.3	0.00998341731771905\\
599.31	0.00998384822236584\\
599.32	0.00998427473646497\\
599.33	0.00998469681658292\\
599.34	0.00998511441885952\\
599.35	0.0099855274990037\\
599.36	0.00998593601228926\\
599.37	0.00998633991355056\\
599.38	0.00998673915717821\\
599.39	0.00998713369711469\\
599.4	0.00998752348684992\\
599.41	0.00998790847811157\\
599.42	0.00998828861974491\\
599.43	0.00998866386008804\\
599.44	0.00998903414696681\\
599.45	0.00998939942768985\\
599.46	0.00998975964904344\\
599.47	0.00999011475728636\\
599.48	0.00999046469814472\\
599.49	0.00999080941680669\\
599.5	0.00999114885791725\\
599.51	0.00999148296557278\\
599.52	0.0099918116833157\\
599.53	0.00999213495412901\\
599.54	0.00999245272043077\\
599.55	0.00999276492406855\\
599.56	0.00999307150631382\\
599.57	0.00999337240785624\\
599.58	0.00999366756879799\\
599.59	0.00999395692864795\\
599.6	0.00999424042631585\\
599.61	0.00999451800010642\\
599.62	0.00999478958771336\\
599.63	0.0099950551262134\\
599.64	0.00999531455206019\\
599.65	0.00999556780107816\\
599.66	0.00999581480845634\\
599.67	0.00999605550874211\\
599.68	0.00999628983583487\\
599.69	0.00999651772297967\\
599.7	0.00999673910276076\\
599.71	0.00999695390709509\\
599.72	0.00999716206722577\\
599.73	0.00999736351371538\\
599.74	0.00999755817643933\\
599.75	0.00999774598457904\\
599.76	0.00999792686661517\\
599.77	0.00999810075032068\\
599.78	0.00999826756275386\\
599.79	0.00999842723025133\\
599.8	0.00999857967842092\\
599.81	0.0099987248321345\\
599.82	0.00999886261552071\\
599.83	0.00999899295195769\\
599.84	0.00999911576406567\\
599.85	0.00999923097369951\\
599.86	0.00999933850194117\\
599.87	0.00999943826909211\\
599.88	0.00999953019466558\\
599.89	0.00999961419737891\\
599.9	0.00999969019514566\\
599.91	0.00999975810506767\\
599.92	0.00999981784342713\\
599.93	0.00999986932567848\\
599.94	0.00999991246644028\\
599.95	0.00999994717948697\\
599.96	0.00999997337774056\\
599.97	0.00999999097326228\\
599.98	0.00999999987724406\\
599.99	0.01\\
600	0.01\\
};
\addplot [color=mycolor4,solid,forget plot]
  table[row sep=crcr]{%
0.01	0.0050190636862812\\
1.01	0.00501906463782538\\
2.01	0.00501906560805189\\
3.01	0.00501906659732479\\
4.01	0.00501906760601505\\
5.01	0.00501906863450116\\
6.01	0.00501906968316851\\
7.01	0.00501907075241023\\
8.01	0.00501907184262679\\
9.01	0.00501907295422666\\
10.01	0.00501907408762585\\
11.01	0.00501907524324856\\
12.01	0.00501907642152723\\
13.01	0.00501907762290241\\
14.01	0.00501907884782346\\
15.01	0.00501908009674813\\
16.01	0.00501908137014302\\
17.01	0.00501908266848361\\
18.01	0.00501908399225469\\
19.01	0.00501908534195042\\
20.01	0.0050190867180743\\
21.01	0.00501908812113945\\
22.01	0.00501908955166902\\
23.01	0.00501909101019615\\
24.01	0.0050190924972642\\
25.01	0.0050190940134268\\
26.01	0.00501909555924866\\
27.01	0.00501909713530482\\
28.01	0.00501909874218165\\
29.01	0.00501910038047652\\
30.01	0.00501910205079835\\
31.01	0.00501910375376788\\
32.01	0.00501910549001755\\
33.01	0.005019107260192\\
34.01	0.00501910906494818\\
35.01	0.00501911090495559\\
36.01	0.00501911278089671\\
37.01	0.00501911469346675\\
38.01	0.00501911664337441\\
39.01	0.0050191186313419\\
40.01	0.0050191206581052\\
41.01	0.00501912272441454\\
42.01	0.00501912483103419\\
43.01	0.0050191269787432\\
44.01	0.0050191291683354\\
45.01	0.00501913140061969\\
46.01	0.00501913367642064\\
47.01	0.00501913599657836\\
48.01	0.00501913836194895\\
49.01	0.005019140773405\\
50.01	0.00501914323183551\\
51.01	0.00501914573814656\\
52.01	0.00501914829326133\\
53.01	0.00501915089812062\\
54.01	0.00501915355368324\\
55.01	0.00501915626092608\\
56.01	0.00501915902084474\\
57.01	0.0050191618344535\\
58.01	0.00501916470278627\\
59.01	0.00501916762689629\\
60.01	0.00501917060785684\\
61.01	0.00501917364676173\\
62.01	0.00501917674472523\\
63.01	0.00501917990288314\\
64.01	0.00501918312239248\\
65.01	0.00501918640443227\\
66.01	0.00501918975020393\\
67.01	0.00501919316093166\\
68.01	0.00501919663786256\\
69.01	0.00501920018226763\\
70.01	0.0050192037954418\\
71.01	0.00501920747870464\\
72.01	0.00501921123340041\\
73.01	0.00501921506089889\\
74.01	0.00501921896259575\\
75.01	0.00501922293991293\\
76.01	0.00501922699429926\\
77.01	0.00501923112723107\\
78.01	0.00501923534021226\\
79.01	0.00501923963477527\\
80.01	0.00501924401248134\\
81.01	0.0050192484749212\\
82.01	0.00501925302371536\\
83.01	0.00501925766051513\\
84.01	0.00501926238700277\\
85.01	0.00501926720489194\\
86.01	0.00501927211592888\\
87.01	0.00501927712189236\\
88.01	0.0050192822245949\\
89.01	0.00501928742588281\\
90.01	0.00501929272763717\\
91.01	0.00501929813177436\\
92.01	0.00501930364024664\\
93.01	0.00501930925504299\\
94.01	0.00501931497818932\\
95.01	0.00501932081175031\\
96.01	0.00501932675782853\\
97.01	0.00501933281856613\\
98.01	0.00501933899614557\\
99.01	0.00501934529278994\\
100.01	0.0050193517107639\\
101.01	0.00501935825237451\\
102.01	0.00501936491997196\\
103.01	0.00501937171595036\\
104.01	0.00501937864274822\\
105.01	0.00501938570285006\\
106.01	0.0050193928987864\\
107.01	0.00501940023313502\\
108.01	0.00501940770852174\\
109.01	0.0050194153276215\\
110.01	0.00501942309315881\\
111.01	0.00501943100790882\\
112.01	0.00501943907469866\\
113.01	0.00501944729640778\\
114.01	0.00501945567596943\\
115.01	0.00501946421637105\\
116.01	0.0050194729206558\\
117.01	0.00501948179192331\\
118.01	0.00501949083333078\\
119.01	0.0050195000480939\\
120.01	0.00501950943948809\\
121.01	0.00501951901084931\\
122.01	0.00501952876557563\\
123.01	0.00501953870712786\\
124.01	0.00501954883903089\\
125.01	0.00501955916487502\\
126.01	0.00501956968831682\\
127.01	0.00501958041308039\\
128.01	0.00501959134295896\\
129.01	0.00501960248181554\\
130.01	0.00501961383358478\\
131.01	0.0050196254022737\\
132.01	0.00501963719196331\\
133.01	0.00501964920680996\\
134.01	0.00501966145104675\\
135.01	0.00501967392898464\\
136.01	0.00501968664501412\\
137.01	0.00501969960360645\\
138.01	0.00501971280931506\\
139.01	0.00501972626677753\\
140.01	0.00501973998071634\\
141.01	0.00501975395594097\\
142.01	0.00501976819734937\\
143.01	0.00501978270992932\\
144.01	0.00501979749876019\\
145.01	0.00501981256901469\\
146.01	0.00501982792596037\\
147.01	0.00501984357496139\\
148.01	0.00501985952148028\\
149.01	0.00501987577107953\\
150.01	0.0050198923294236\\
151.01	0.0050199092022809\\
152.01	0.00501992639552516\\
153.01	0.00501994391513751\\
154.01	0.00501996176720869\\
155.01	0.00501997995794085\\
156.01	0.00501999849364934\\
157.01	0.00502001738076462\\
158.01	0.00502003662583522\\
159.01	0.0050200562355288\\
160.01	0.00502007621663457\\
161.01	0.00502009657606595\\
162.01	0.00502011732086234\\
163.01	0.00502013845819144\\
164.01	0.00502015999535146\\
165.01	0.00502018193977367\\
166.01	0.00502020429902479\\
167.01	0.00502022708080903\\
168.01	0.00502025029297101\\
169.01	0.00502027394349808\\
170.01	0.00502029804052302\\
171.01	0.00502032259232618\\
172.01	0.00502034760733859\\
173.01	0.00502037309414435\\
174.01	0.00502039906148381\\
175.01	0.00502042551825597\\
176.01	0.0050204524735213\\
177.01	0.00502047993650487\\
178.01	0.00502050791659914\\
179.01	0.00502053642336707\\
180.01	0.0050205654665449\\
181.01	0.00502059505604584\\
182.01	0.00502062520196246\\
183.01	0.00502065591457078\\
184.01	0.00502068720433252\\
185.01	0.00502071908189955\\
186.01	0.00502075155811632\\
187.01	0.0050207846440241\\
188.01	0.00502081835086399\\
189.01	0.00502085269008092\\
190.01	0.00502088767332684\\
191.01	0.00502092331246481\\
192.01	0.00502095961957291\\
193.01	0.00502099660694766\\
194.01	0.00502103428710834\\
195.01	0.00502107267280101\\
196.01	0.00502111177700215\\
197.01	0.00502115161292357\\
198.01	0.00502119219401589\\
199.01	0.0050212335339737\\
200.01	0.00502127564673925\\
201.01	0.00502131854650727\\
202.01	0.00502136224772966\\
203.01	0.00502140676511986\\
204.01	0.005021452113658\\
205.01	0.00502149830859555\\
206.01	0.00502154536546001\\
207.01	0.00502159330006058\\
208.01	0.00502164212849283\\
209.01	0.00502169186714428\\
210.01	0.0050217425326993\\
211.01	0.0050217941421451\\
212.01	0.00502184671277701\\
213.01	0.00502190026220412\\
214.01	0.00502195480835549\\
215.01	0.00502201036948524\\
216.01	0.0050220669641797\\
217.01	0.0050221246113627\\
218.01	0.00502218333030202\\
219.01	0.00502224314061629\\
220.01	0.00502230406228095\\
221.01	0.00502236611563531\\
222.01	0.00502242932138887\\
223.01	0.00502249370062913\\
224.01	0.00502255927482793\\
225.01	0.00502262606584909\\
226.01	0.00502269409595579\\
227.01	0.00502276338781804\\
228.01	0.00502283396452048\\
229.01	0.00502290584957027\\
230.01	0.00502297906690523\\
231.01	0.005023053640902\\
232.01	0.00502312959638448\\
233.01	0.00502320695863241\\
234.01	0.00502328575339044\\
235.01	0.00502336600687672\\
236.01	0.00502344774579229\\
237.01	0.00502353099733051\\
238.01	0.00502361578918678\\
239.01	0.00502370214956807\\
240.01	0.00502379010720296\\
241.01	0.00502387969135258\\
242.01	0.00502397093182022\\
243.01	0.00502406385896251\\
244.01	0.00502415850370061\\
245.01	0.00502425489753115\\
246.01	0.00502435307253787\\
247.01	0.00502445306140335\\
248.01	0.00502455489742119\\
249.01	0.00502465861450822\\
250.01	0.0050247642472171\\
251.01	0.00502487183074932\\
252.01	0.00502498140096876\\
253.01	0.00502509299441485\\
254.01	0.00502520664831621\\
255.01	0.00502532240060563\\
256.01	0.0050254402899341\\
257.01	0.00502556035568567\\
258.01	0.005025682637993\\
259.01	0.00502580717775299\\
260.01	0.00502593401664239\\
261.01	0.00502606319713503\\
262.01	0.00502619476251753\\
263.01	0.00502632875690788\\
264.01	0.00502646522527174\\
265.01	0.00502660421344144\\
266.01	0.00502674576813428\\
267.01	0.00502688993697173\\
268.01	0.00502703676849824\\
269.01	0.00502718631220181\\
270.01	0.00502733861853444\\
271.01	0.00502749373893292\\
272.01	0.00502765172584038\\
273.01	0.00502781263272795\\
274.01	0.0050279765141182\\
275.01	0.00502814342560737\\
276.01	0.00502831342388961\\
277.01	0.00502848656678119\\
278.01	0.00502866291324529\\
279.01	0.00502884252341747\\
280.01	0.00502902545863208\\
281.01	0.00502921178144883\\
282.01	0.00502940155568045\\
283.01	0.00502959484642063\\
284.01	0.00502979172007285\\
285.01	0.00502999224438008\\
286.01	0.00503019648845489\\
287.01	0.00503040452281049\\
288.01	0.00503061641939237\\
289.01	0.00503083225161066\\
290.01	0.00503105209437366\\
291.01	0.00503127602412171\\
292.01	0.00503150411886187\\
293.01	0.00503173645820396\\
294.01	0.00503197312339637\\
295.01	0.00503221419736403\\
296.01	0.00503245976474617\\
297.01	0.0050327099119353\\
298.01	0.0050329647271169\\
299.01	0.00503322430031067\\
300.01	0.00503348872341158\\
301.01	0.00503375809023245\\
302.01	0.00503403249654735\\
303.01	0.00503431204013617\\
304.01	0.00503459682082916\\
305.01	0.00503488694055343\\
306.01	0.00503518250337986\\
307.01	0.00503548361557084\\
308.01	0.00503579038562912\\
309.01	0.0050361029243475\\
310.01	0.00503642134485913\\
311.01	0.00503674576268956\\
312.01	0.00503707629580859\\
313.01	0.00503741306468343\\
314.01	0.00503775619233326\\
315.01	0.00503810580438374\\
316.01	0.00503846202912313\\
317.01	0.00503882499755876\\
318.01	0.00503919484347427\\
319.01	0.00503957170348862\\
320.01	0.00503995571711445\\
321.01	0.00504034702681803\\
322.01	0.00504074577808022\\
323.01	0.00504115211945731\\
324.01	0.0050415662026435\\
325.01	0.00504198818253367\\
326.01	0.00504241821728675\\
327.01	0.00504285646839001\\
328.01	0.00504330310072384\\
329.01	0.00504375828262761\\
330.01	0.00504422218596549\\
331.01	0.00504469498619389\\
332.01	0.00504517686242904\\
333.01	0.00504566799751503\\
334.01	0.00504616857809335\\
335.01	0.00504667879467241\\
336.01	0.0050471988416986\\
337.01	0.00504772891762756\\
338.01	0.00504826922499717\\
339.01	0.00504881997050075\\
340.01	0.0050493813650619\\
341.01	0.00504995362391013\\
342.01	0.00505053696665841\\
343.01	0.00505113161738205\\
344.01	0.00505173780469856\\
345.01	0.00505235576185045\\
346.01	0.00505298572678886\\
347.01	0.00505362794226062\\
348.01	0.00505428265589696\\
349.01	0.00505495012030505\\
350.01	0.00505563059316302\\
351.01	0.00505632433731748\\
352.01	0.00505703162088522\\
353.01	0.00505775271735856\\
354.01	0.00505848790571424\\
355.01	0.00505923747052754\\
356.01	0.00506000170209087\\
357.01	0.0050607808965366\\
358.01	0.00506157535596578\\
359.01	0.005062385388582\\
360.01	0.00506321130883047\\
361.01	0.00506405343754281\\
362.01	0.0050649121020872\\
363.01	0.00506578763652464\\
364.01	0.00506668038176997\\
365.01	0.00506759068575829\\
366.01	0.00506851890361704\\
367.01	0.00506946539784152\\
368.01	0.00507043053847556\\
369.01	0.00507141470329618\\
370.01	0.00507241827800047\\
371.01	0.00507344165639692\\
372.01	0.00507448524059727\\
373.01	0.00507554944121311\\
374.01	0.00507663467755196\\
375.01	0.00507774137781693\\
376.01	0.00507886997930695\\
377.01	0.0050800209286208\\
378.01	0.00508119468186213\\
379.01	0.00508239170484881\\
380.01	0.00508361247332564\\
381.01	0.00508485747318098\\
382.01	0.00508612720066827\\
383.01	0.00508742216263094\\
384.01	0.00508874287673265\\
385.01	0.00509008987169214\\
386.01	0.00509146368752236\\
387.01	0.00509286487577354\\
388.01	0.00509429399978219\\
389.01	0.00509575163492443\\
390.01	0.00509723836887327\\
391.01	0.00509875480186192\\
392.01	0.00510030154695075\\
393.01	0.00510187923030017\\
394.01	0.00510348849144823\\
395.01	0.00510512998359208\\
396.01	0.00510680437387648\\
397.01	0.00510851234368537\\
398.01	0.00511025458893963\\
399.01	0.00511203182040038\\
400.01	0.00511384476397572\\
401.01	0.00511569416103477\\
402.01	0.00511758076872551\\
403.01	0.00511950536029861\\
404.01	0.00512146872543671\\
405.01	0.00512347167058867\\
406.01	0.00512551501930966\\
407.01	0.00512759961260707\\
408.01	0.0051297263092914\\
409.01	0.00513189598633347\\
410.01	0.00513410953922723\\
411.01	0.00513636788235847\\
412.01	0.00513867194937997\\
413.01	0.00514102269359204\\
414.01	0.00514342108833014\\
415.01	0.005145868127359\\
416.01	0.00514836482527282\\
417.01	0.00515091221790266\\
418.01	0.00515351136273091\\
419.01	0.00515616333931325\\
420.01	0.0051588692497074\\
421.01	0.00516163021891089\\
422.01	0.00516444739530594\\
423.01	0.00516732195111302\\
424.01	0.00517025508285339\\
425.01	0.00517324801182083\\
426.01	0.00517630198456262\\
427.01	0.00517941827337016\\
428.01	0.00518259817678116\\
429.01	0.00518584302009149\\
430.01	0.00518915415587828\\
431.01	0.00519253296453566\\
432.01	0.00519598085482181\\
433.01	0.00519949926441959\\
434.01	0.00520308966050941\\
435.01	0.0052067535403565\\
436.01	0.00521049243191227\\
437.01	0.0052143078944302\\
438.01	0.00521820151909683\\
439.01	0.00522217492967812\\
440.01	0.00522622978318196\\
441.01	0.00523036777053668\\
442.01	0.00523459061728625\\
443.01	0.00523890008430183\\
444.01	0.00524329796851115\\
445.01	0.00524778610364351\\
446.01	0.00525236636099316\\
447.01	0.00525704065019965\\
448.01	0.00526181092004457\\
449.01	0.00526667915926614\\
450.01	0.00527164739739069\\
451.01	0.00527671770558057\\
452.01	0.00528189219749942\\
453.01	0.00528717303019413\\
454.01	0.00529256240499335\\
455.01	0.00529806256842333\\
456.01	0.00530367581314065\\
457.01	0.00530940447888219\\
458.01	0.00531525095343356\\
459.01	0.0053212176736156\\
460.01	0.00532730712629066\\
461.01	0.00533352184938808\\
462.01	0.00533986443295212\\
463.01	0.00534633752021037\\
464.01	0.00535294380866658\\
465.01	0.00535968605121613\\
466.01	0.00536656705728716\\
467.01	0.00537358969400614\\
468.01	0.00538075688738979\\
469.01	0.00538807162356364\\
470.01	0.00539553695000666\\
471.01	0.00540315597682403\\
472.01	0.00541093187804739\\
473.01	0.00541886789296336\\
474.01	0.00542696732747124\\
475.01	0.00543523355546909\\
476.01	0.00544367002027044\\
477.01	0.00545228023605022\\
478.01	0.00546106778932162\\
479.01	0.00547003634044396\\
480.01	0.00547918962516099\\
481.01	0.00548853145617276\\
482.01	0.00549806572473786\\
483.01	0.00550779640231014\\
484.01	0.00551772754220767\\
485.01	0.00552786328131574\\
486.01	0.00553820784182571\\
487.01	0.00554876553300686\\
488.01	0.00555954075301628\\
489.01	0.00557053799074402\\
490.01	0.00558176182769582\\
491.01	0.00559321693991399\\
492.01	0.00560490809993744\\
493.01	0.00561684017880044\\
494.01	0.0056290181480723\\
495.01	0.0056414470819366\\
496.01	0.00565413215931353\\
497.01	0.00566707866602291\\
498.01	0.00568029199698998\\
499.01	0.0056937776584946\\
500.01	0.00570754127046308\\
501.01	0.00572158856880466\\
502.01	0.00573592540779119\\
503.01	0.00575055776248193\\
504.01	0.00576549173119376\\
505.01	0.00578073353801487\\
506.01	0.00579628953536469\\
507.01	0.00581216620659866\\
508.01	0.0058283701686583\\
509.01	0.00584490817476563\\
510.01	0.00586178711716329\\
511.01	0.00587901402989748\\
512.01	0.00589659609164545\\
513.01	0.00591454062858528\\
514.01	0.0059328551173065\\
515.01	0.00595154718776236\\
516.01	0.00597062462625876\\
517.01	0.00599009537848122\\
518.01	0.00600996755255487\\
519.01	0.00603024942213667\\
520.01	0.00605094942953484\\
521.01	0.00607207618885311\\
522.01	0.00609363848915531\\
523.01	0.00611564529764388\\
524.01	0.00613810576284839\\
525.01	0.00616102921781665\\
526.01	0.00618442518330081\\
527.01	0.00620830337092997\\
528.01	0.00623267368636038\\
529.01	0.00625754623239091\\
530.01	0.00628293131203315\\
531.01	0.00630883943152065\\
532.01	0.00633528130324207\\
533.01	0.00636226784858186\\
534.01	0.00638981020064584\\
535.01	0.00641791970685204\\
536.01	0.00644660793135991\\
537.01	0.00647588665731036\\
538.01	0.00650576788884577\\
539.01	0.0065362638528734\\
540.01	0.00656738700053343\\
541.01	0.00659915000832749\\
542.01	0.00663156577885837\\
543.01	0.00666464744112576\\
544.01	0.00669840835031567\\
545.01	0.00673286208701748\\
546.01	0.00676802245578906\\
547.01	0.00680390348298683\\
548.01	0.00684051941376644\\
549.01	0.00687788470814671\\
550.01	0.0069160140360215\\
551.01	0.0069549222709883\\
552.01	0.00699462448284956\\
553.01	0.00703513592862633\\
554.01	0.00707647204190588\\
555.01	0.00711864842032717\\
556.01	0.00716168081098622\\
557.01	0.00720558509352023\\
558.01	0.00725037726060528\\
559.01	0.00729607339557415\\
560.01	0.0073426896468327\\
561.01	0.00739024219871955\\
562.01	0.00743874723842159\\
563.01	0.00748822091851965\\
564.01	0.00753867931470128\\
565.01	0.00759013837813551\\
566.01	0.00764261388196354\\
567.01	0.0076961213613144\\
568.01	0.00775067604621158\\
569.01	0.00780629278669433\\
570.01	0.00786298596943516\\
571.01	0.00792076942509971\\
572.01	0.0079796563256667\\
573.01	0.00803965907090454\\
574.01	0.00810078916320144\\
575.01	0.00816305706995899\\
576.01	0.00822647207281021\\
577.01	0.00829104210300508\\
578.01	0.00835677356244437\\
579.01	0.00842367113004633\\
580.01	0.00849173755342209\\
581.01	0.00856097342624057\\
582.01	0.00863137695221383\\
583.01	0.00870294369737224\\
584.01	0.00877566633327528\\
585.01	0.00884953437508313\\
586.01	0.00892453392008026\\
587.01	0.00900064739439612\\
588.01	0.00907785331843724\\
589.01	0.00915612610509321\\
590.01	0.00923543590930589\\
591.01	0.00931574855334877\\
592.01	0.00939702555946881\\
593.01	0.00947922433078829\\
594.01	0.00956229853304221\\
595.01	0.00964619874445287\\
596.01	0.00973087345958245\\
597.01	0.0098162705563152\\
598.01	0.00990228776786639\\
599.01	0.00996919203046377\\
599.02	0.00996973314335038\\
599.03	0.0099702709571694\\
599.04	0.00997080543920336\\
599.05	0.0099713365564138\\
599.06	0.00997186427543808\\
599.07	0.0099723885625862\\
599.08	0.00997290938383756\\
599.09	0.00997342670483771\\
599.1	0.00997394049089505\\
599.11	0.00997445070697751\\
599.12	0.00997495731770918\\
599.13	0.00997546028736696\\
599.14	0.00997595957987707\\
599.15	0.00997645515881166\\
599.16	0.00997694698738526\\
599.17	0.00997743502845132\\
599.18	0.00997791924449856\\
599.19	0.00997839959764748\\
599.2	0.00997887604964662\\
599.21	0.00997934856186899\\
599.22	0.00997981709530829\\
599.23	0.00998028161057521\\
599.24	0.00998074206789365\\
599.25	0.0099811984270969\\
599.26	0.00998165064762378\\
599.27	0.00998209868851477\\
599.28	0.00998254250840806\\
599.29	0.00998298206553559\\
599.3	0.00998341731771905\\
599.31	0.00998384822236584\\
599.32	0.00998427473646497\\
599.33	0.00998469681658292\\
599.34	0.00998511441885952\\
599.35	0.0099855274990037\\
599.36	0.00998593601228926\\
599.37	0.00998633991355056\\
599.38	0.00998673915717821\\
599.39	0.00998713369711469\\
599.4	0.00998752348684992\\
599.41	0.00998790847811157\\
599.42	0.00998828861974491\\
599.43	0.00998866386008804\\
599.44	0.00998903414696681\\
599.45	0.00998939942768985\\
599.46	0.00998975964904344\\
599.47	0.00999011475728636\\
599.48	0.00999046469814472\\
599.49	0.00999080941680669\\
599.5	0.00999114885791725\\
599.51	0.00999148296557278\\
599.52	0.0099918116833157\\
599.53	0.00999213495412901\\
599.54	0.00999245272043077\\
599.55	0.00999276492406855\\
599.56	0.00999307150631381\\
599.57	0.00999337240785624\\
599.58	0.00999366756879799\\
599.59	0.00999395692864795\\
599.6	0.00999424042631586\\
599.61	0.00999451800010642\\
599.62	0.00999478958771336\\
599.63	0.0099950551262134\\
599.64	0.00999531455206019\\
599.65	0.00999556780107816\\
599.66	0.00999581480845634\\
599.67	0.00999605550874211\\
599.68	0.00999628983583487\\
599.69	0.00999651772297967\\
599.7	0.00999673910276075\\
599.71	0.00999695390709509\\
599.72	0.00999716206722577\\
599.73	0.00999736351371538\\
599.74	0.00999755817643933\\
599.75	0.00999774598457904\\
599.76	0.00999792686661517\\
599.77	0.00999810075032068\\
599.78	0.00999826756275386\\
599.79	0.00999842723025133\\
599.8	0.00999857967842092\\
599.81	0.0099987248321345\\
599.82	0.00999886261552071\\
599.83	0.00999899295195769\\
599.84	0.00999911576406567\\
599.85	0.00999923097369951\\
599.86	0.00999933850194117\\
599.87	0.00999943826909211\\
599.88	0.00999953019466558\\
599.89	0.00999961419737891\\
599.9	0.00999969019514566\\
599.91	0.00999975810506767\\
599.92	0.00999981784342713\\
599.93	0.00999986932567848\\
599.94	0.00999991246644028\\
599.95	0.00999994717948697\\
599.96	0.00999997337774056\\
599.97	0.00999999097326228\\
599.98	0.00999999987724406\\
599.99	0.01\\
600	0.01\\
};
\addplot [color=mycolor5,solid,forget plot]
  table[row sep=crcr]{%
0.01	0.00501105323809203\\
1.01	0.00501105434915347\\
2.01	0.00501105548219559\\
3.01	0.00501105663765027\\
4.01	0.00501105781595748\\
5.01	0.00501105901756574\\
6.01	0.00501106024293257\\
7.01	0.00501106149252414\\
8.01	0.00501106276681591\\
9.01	0.00501106406629223\\
10.01	0.00501106539144701\\
11.01	0.00501106674278391\\
12.01	0.00501106812081607\\
13.01	0.00501106952606685\\
14.01	0.00501107095906948\\
15.01	0.0050110724203678\\
16.01	0.00501107391051601\\
17.01	0.00501107543007902\\
18.01	0.00501107697963291\\
19.01	0.0050110785597646\\
20.01	0.00501108017107262\\
21.01	0.00501108181416682\\
22.01	0.00501108348966919\\
23.01	0.00501108519821353\\
24.01	0.00501108694044591\\
25.01	0.00501108871702494\\
26.01	0.00501109052862167\\
27.01	0.00501109237592055\\
28.01	0.00501109425961895\\
29.01	0.00501109618042773\\
30.01	0.00501109813907175\\
31.01	0.00501110013628946\\
32.01	0.00501110217283358\\
33.01	0.00501110424947158\\
34.01	0.00501110636698546\\
35.01	0.00501110852617254\\
36.01	0.00501111072784531\\
37.01	0.00501111297283197\\
38.01	0.00501111526197658\\
39.01	0.00501111759613956\\
40.01	0.00501111997619762\\
41.01	0.00501112240304449\\
42.01	0.00501112487759112\\
43.01	0.00501112740076579\\
44.01	0.00501112997351484\\
45.01	0.00501113259680265\\
46.01	0.00501113527161187\\
47.01	0.00501113799894434\\
48.01	0.00501114077982097\\
49.01	0.00501114361528207\\
50.01	0.00501114650638813\\
51.01	0.00501114945421972\\
52.01	0.00501115245987828\\
53.01	0.00501115552448617\\
54.01	0.00501115864918734\\
55.01	0.00501116183514756\\
56.01	0.00501116508355486\\
57.01	0.0050111683956198\\
58.01	0.00501117177257621\\
59.01	0.00501117521568152\\
60.01	0.00501117872621715\\
61.01	0.00501118230548872\\
62.01	0.00501118595482709\\
63.01	0.00501118967558804\\
64.01	0.00501119346915346\\
65.01	0.00501119733693153\\
66.01	0.00501120128035717\\
67.01	0.00501120530089255\\
68.01	0.00501120940002762\\
69.01	0.00501121357928062\\
70.01	0.00501121784019883\\
71.01	0.00501122218435855\\
72.01	0.00501122661336638\\
73.01	0.00501123112885897\\
74.01	0.0050112357325043\\
75.01	0.00501124042600191\\
76.01	0.00501124521108336\\
77.01	0.00501125008951314\\
78.01	0.00501125506308905\\
79.01	0.005011260133643\\
80.01	0.0050112653030413\\
81.01	0.00501127057318572\\
82.01	0.00501127594601383\\
83.01	0.00501128142349977\\
84.01	0.0050112870076547\\
85.01	0.00501129270052825\\
86.01	0.0050112985042082\\
87.01	0.00501130442082177\\
88.01	0.00501131045253633\\
89.01	0.00501131660155973\\
90.01	0.0050113228701416\\
91.01	0.00501132926057374\\
92.01	0.00501133577519107\\
93.01	0.00501134241637217\\
94.01	0.00501134918654049\\
95.01	0.00501135608816462\\
96.01	0.00501136312375958\\
97.01	0.00501137029588755\\
98.01	0.00501137760715856\\
99.01	0.00501138506023138\\
100.01	0.00501139265781471\\
101.01	0.00501140040266766\\
102.01	0.00501140829760099\\
103.01	0.00501141634547772\\
104.01	0.00501142454921466\\
105.01	0.00501143291178256\\
106.01	0.0050114414362077\\
107.01	0.00501145012557279\\
108.01	0.00501145898301772\\
109.01	0.00501146801174081\\
110.01	0.00501147721499982\\
111.01	0.00501148659611307\\
112.01	0.00501149615846029\\
113.01	0.00501150590548408\\
114.01	0.00501151584069057\\
115.01	0.00501152596765125\\
116.01	0.00501153629000348\\
117.01	0.00501154681145168\\
118.01	0.00501155753576905\\
119.01	0.00501156846679843\\
120.01	0.0050115796084534\\
121.01	0.00501159096471993\\
122.01	0.00501160253965742\\
123.01	0.00501161433740002\\
124.01	0.00501162636215804\\
125.01	0.00501163861821899\\
126.01	0.00501165110994933\\
127.01	0.00501166384179594\\
128.01	0.0050116768182871\\
129.01	0.00501169004403393\\
130.01	0.00501170352373236\\
131.01	0.0050117172621643\\
132.01	0.0050117312641991\\
133.01	0.00501174553479515\\
134.01	0.00501176007900125\\
135.01	0.00501177490195867\\
136.01	0.00501179000890223\\
137.01	0.00501180540516227\\
138.01	0.0050118210961662\\
139.01	0.00501183708743999\\
140.01	0.00501185338461066\\
141.01	0.00501186999340709\\
142.01	0.00501188691966192\\
143.01	0.00501190416931392\\
144.01	0.00501192174840956\\
145.01	0.00501193966310458\\
146.01	0.00501195791966616\\
147.01	0.00501197652447485\\
148.01	0.00501199548402642\\
149.01	0.00501201480493388\\
150.01	0.00501203449392944\\
151.01	0.00501205455786659\\
152.01	0.00501207500372232\\
153.01	0.00501209583859905\\
154.01	0.00501211706972662\\
155.01	0.00501213870446485\\
156.01	0.00501216075030556\\
157.01	0.00501218321487495\\
158.01	0.00501220610593539\\
159.01	0.0050122294313884\\
160.01	0.00501225319927683\\
161.01	0.00501227741778669\\
162.01	0.00501230209525023\\
163.01	0.00501232724014815\\
164.01	0.00501235286111248\\
165.01	0.00501237896692812\\
166.01	0.00501240556653661\\
167.01	0.00501243266903807\\
168.01	0.00501246028369394\\
169.01	0.00501248841992995\\
170.01	0.00501251708733844\\
171.01	0.00501254629568146\\
172.01	0.00501257605489341\\
173.01	0.00501260637508434\\
174.01	0.00501263726654223\\
175.01	0.00501266873973628\\
176.01	0.00501270080532015\\
177.01	0.00501273347413449\\
178.01	0.00501276675721027\\
179.01	0.00501280066577221\\
180.01	0.00501283521124173\\
181.01	0.00501287040523979\\
182.01	0.00501290625959114\\
183.01	0.00501294278632643\\
184.01	0.00501297999768707\\
185.01	0.00501301790612706\\
186.01	0.00501305652431797\\
187.01	0.00501309586515127\\
188.01	0.00501313594174276\\
189.01	0.00501317676743585\\
190.01	0.00501321835580532\\
191.01	0.00501326072066127\\
192.01	0.00501330387605255\\
193.01	0.00501334783627098\\
194.01	0.00501339261585526\\
195.01	0.00501343822959488\\
196.01	0.00501348469253411\\
197.01	0.00501353201997619\\
198.01	0.00501358022748774\\
199.01	0.00501362933090251\\
200.01	0.00501367934632607\\
201.01	0.00501373029014007\\
202.01	0.00501378217900687\\
203.01	0.00501383502987365\\
204.01	0.00501388885997742\\
205.01	0.00501394368684922\\
206.01	0.0050139995283193\\
207.01	0.00501405640252163\\
208.01	0.00501411432789881\\
209.01	0.00501417332320697\\
210.01	0.00501423340752125\\
211.01	0.00501429460023999\\
212.01	0.00501435692109077\\
213.01	0.00501442039013514\\
214.01	0.00501448502777423\\
215.01	0.00501455085475419\\
216.01	0.00501461789217139\\
217.01	0.00501468616147842\\
218.01	0.00501475568448942\\
219.01	0.00501482648338628\\
220.01	0.00501489858072407\\
221.01	0.00501497199943747\\
222.01	0.00501504676284646\\
223.01	0.00501512289466254\\
224.01	0.00501520041899522\\
225.01	0.00501527936035853\\
226.01	0.00501535974367673\\
227.01	0.00501544159429182\\
228.01	0.00501552493796999\\
229.01	0.00501560980090822\\
230.01	0.0050156962097412\\
231.01	0.00501578419154891\\
232.01	0.00501587377386335\\
233.01	0.00501596498467613\\
234.01	0.00501605785244565\\
235.01	0.00501615240610501\\
236.01	0.00501624867506979\\
237.01	0.00501634668924585\\
238.01	0.00501644647903703\\
239.01	0.00501654807535393\\
240.01	0.00501665150962216\\
241.01	0.00501675681379021\\
242.01	0.00501686402033923\\
243.01	0.00501697316229131\\
244.01	0.0050170842732184\\
245.01	0.00501719738725173\\
246.01	0.00501731253909156\\
247.01	0.0050174297640166\\
248.01	0.00501754909789333\\
249.01	0.00501767057718719\\
250.01	0.00501779423897193\\
251.01	0.00501792012094072\\
252.01	0.00501804826141644\\
253.01	0.00501817869936318\\
254.01	0.00501831147439737\\
255.01	0.00501844662679896\\
256.01	0.00501858419752384\\
257.01	0.00501872422821566\\
258.01	0.00501886676121815\\
259.01	0.00501901183958807\\
260.01	0.0050191595071083\\
261.01	0.00501930980830093\\
262.01	0.00501946278844137\\
263.01	0.00501961849357192\\
264.01	0.00501977697051723\\
265.01	0.00501993826689813\\
266.01	0.00502010243114755\\
267.01	0.00502026951252586\\
268.01	0.0050204395611374\\
269.01	0.0050206126279468\\
270.01	0.00502078876479636\\
271.01	0.00502096802442337\\
272.01	0.0050211504604781\\
273.01	0.00502133612754316\\
274.01	0.00502152508115168\\
275.01	0.0050217173778079\\
276.01	0.00502191307500735\\
277.01	0.00502211223125749\\
278.01	0.00502231490609989\\
279.01	0.00502252116013228\\
280.01	0.0050227310550317\\
281.01	0.00502294465357837\\
282.01	0.00502316201967989\\
283.01	0.00502338321839679\\
284.01	0.00502360831596875\\
285.01	0.0050238373798413\\
286.01	0.00502407047869415\\
287.01	0.00502430768246917\\
288.01	0.00502454906240107\\
289.01	0.0050247946910473\\
290.01	0.00502504464232019\\
291.01	0.00502529899151921\\
292.01	0.00502555781536547\\
293.01	0.00502582119203565\\
294.01	0.00502608920119921\\
295.01	0.00502636192405458\\
296.01	0.00502663944336823\\
297.01	0.00502692184351397\\
298.01	0.00502720921051425\\
299.01	0.00502750163208222\\
300.01	0.00502779919766547\\
301.01	0.00502810199849096\\
302.01	0.00502841012761156\\
303.01	0.00502872367995369\\
304.01	0.00502904275236694\\
305.01	0.00502936744367477\\
306.01	0.00502969785472668\\
307.01	0.00503003408845215\\
308.01	0.00503037624991633\\
309.01	0.00503072444637663\\
310.01	0.00503107878734196\\
311.01	0.00503143938463199\\
312.01	0.00503180635243938\\
313.01	0.00503217980739329\\
314.01	0.00503255986862387\\
315.01	0.00503294665782869\\
316.01	0.00503334029934097\\
317.01	0.00503374092019886\\
318.01	0.00503414865021638\\
319.01	0.0050345636220554\\
320.01	0.0050349859712993\\
321.01	0.0050354158365284\\
322.01	0.00503585335939509\\
323.01	0.00503629868470189\\
324.01	0.00503675196047861\\
325.01	0.00503721333806179\\
326.01	0.00503768297217426\\
327.01	0.00503816102100541\\
328.01	0.00503864764629162\\
329.01	0.00503914301339753\\
330.01	0.00503964729139655\\
331.01	0.00504016065315194\\
332.01	0.00504068327539701\\
333.01	0.00504121533881544\\
334.01	0.00504175702811992\\
335.01	0.00504230853213031\\
336.01	0.00504287004385055\\
337.01	0.00504344176054386\\
338.01	0.00504402388380591\\
339.01	0.00504461661963666\\
340.01	0.00504522017850959\\
341.01	0.00504583477543853\\
342.01	0.00504646063004185\\
343.01	0.00504709796660381\\
344.01	0.00504774701413349\\
345.01	0.00504840800641998\\
346.01	0.00504908118208582\\
347.01	0.00504976678463623\\
348.01	0.00505046506250637\\
349.01	0.0050511762691063\\
350.01	0.0050519006628631\\
351.01	0.00505263850726183\\
352.01	0.00505339007088509\\
353.01	0.00505415562745272\\
354.01	0.00505493545586137\\
355.01	0.00505572984022609\\
356.01	0.00505653906992413\\
357.01	0.0050573634396435\\
358.01	0.00505820324943602\\
359.01	0.00505905880477907\\
360.01	0.00505993041664424\\
361.01	0.00506081840157829\\
362.01	0.00506172308179569\\
363.01	0.00506264478528499\\
364.01	0.00506358384593185\\
365.01	0.00506454060365845\\
366.01	0.00506551540458144\\
367.01	0.00506650860118923\\
368.01	0.00506752055253759\\
369.01	0.00506855162446482\\
370.01	0.00506960218982375\\
371.01	0.0050706726287294\\
372.01	0.00507176332881949\\
373.01	0.00507287468552395\\
374.01	0.00507400710233857\\
375.01	0.00507516099110038\\
376.01	0.00507633677225852\\
377.01	0.00507753487513728\\
378.01	0.00507875573819004\\
379.01	0.0050799998092432\\
380.01	0.00508126754573331\\
381.01	0.00508255941494219\\
382.01	0.00508387589423588\\
383.01	0.00508521747130959\\
384.01	0.00508658464443858\\
385.01	0.00508797792273444\\
386.01	0.0050893978264083\\
387.01	0.00509084488704023\\
388.01	0.00509231964785468\\
389.01	0.00509382266400238\\
390.01	0.00509535450284851\\
391.01	0.00509691574426789\\
392.01	0.00509850698094573\\
393.01	0.00510012881868571\\
394.01	0.00510178187672403\\
395.01	0.0051034667880506\\
396.01	0.00510518419973573\\
397.01	0.00510693477326443\\
398.01	0.00510871918487622\\
399.01	0.00511053812591178\\
400.01	0.00511239230316565\\
401.01	0.00511428243924573\\
402.01	0.00511620927293786\\
403.01	0.0051181735595774\\
404.01	0.00512017607142572\\
405.01	0.00512221759805365\\
406.01	0.0051242989467291\\
407.01	0.00512642094281104\\
408.01	0.00512858443014799\\
409.01	0.00513079027148199\\
410.01	0.00513303934885734\\
411.01	0.00513533256403319\\
412.01	0.00513767083890113\\
413.01	0.00514005511590683\\
414.01	0.00514248635847511\\
415.01	0.00514496555143869\\
416.01	0.00514749370147027\\
417.01	0.00515007183751852\\
418.01	0.00515270101124604\\
419.01	0.00515538229747044\\
420.01	0.00515811679460831\\
421.01	0.00516090562512184\\
422.01	0.00516374993596685\\
423.01	0.00516665089904494\\
424.01	0.0051696097116572\\
425.01	0.00517262759696037\\
426.01	0.00517570580442675\\
427.01	0.00517884561030678\\
428.01	0.0051820483180947\\
429.01	0.00518531525899951\\
430.01	0.00518864779242004\\
431.01	0.00519204730642486\\
432.01	0.0051955152182399\\
433.01	0.00519905297474199\\
434.01	0.00520266205296244\\
435.01	0.00520634396059898\\
436.01	0.00521010023654003\\
437.01	0.005213932451401\\
438.01	0.00521784220807515\\
439.01	0.00522183114230099\\
440.01	0.00522590092324672\\
441.01	0.00523005325411513\\
442.01	0.00523428987276905\\
443.01	0.00523861255238114\\
444.01	0.00524302310210695\\
445.01	0.00524752336778546\\
446.01	0.00525211523266651\\
447.01	0.00525680061816582\\
448.01	0.00526158148465021\\
449.01	0.00526645983225105\\
450.01	0.00527143770170575\\
451.01	0.00527651717522777\\
452.01	0.00528170037740337\\
453.01	0.00528698947611138\\
454.01	0.00529238668346678\\
455.01	0.00529789425678259\\
456.01	0.00530351449954898\\
457.01	0.00530924976242536\\
458.01	0.00531510244424384\\
459.01	0.00532107499302052\\
460.01	0.00532716990697359\\
461.01	0.00533338973554799\\
462.01	0.00533973708044658\\
463.01	0.00534621459667049\\
464.01	0.00535282499357081\\
465.01	0.00535957103591755\\
466.01	0.00536645554498819\\
467.01	0.00537348139968213\\
468.01	0.00538065153766244\\
469.01	0.00538796895652568\\
470.01	0.0053954367150031\\
471.01	0.00540305793419209\\
472.01	0.00541083579882057\\
473.01	0.00541877355854389\\
474.01	0.00542687452927569\\
475.01	0.00543514209455356\\
476.01	0.0054435797069395\\
477.01	0.0054521908894558\\
478.01	0.00546097923705703\\
479.01	0.0054699484181375\\
480.01	0.0054791021760752\\
481.01	0.00548844433081055\\
482.01	0.00549797878046185\\
483.01	0.00550770950297491\\
484.01	0.00551764055780818\\
485.01	0.00552777608765277\\
486.01	0.00553812032018593\\
487.01	0.00554867756985955\\
488.01	0.00555945223972305\\
489.01	0.00557044882328042\\
490.01	0.00558167190638299\\
491.01	0.00559312616915737\\
492.01	0.00560481638797075\\
493.01	0.00561674743743441\\
494.01	0.00562892429244649\\
495.01	0.00564135203027566\\
496.01	0.00565403583268666\\
497.01	0.00566698098810914\\
498.01	0.00568019289385005\\
499.01	0.00569367705835026\\
500.01	0.00570743910348621\\
501.01	0.00572148476691542\\
502.01	0.00573581990446716\\
503.01	0.00575045049257814\\
504.01	0.005765382630772\\
505.01	0.0057806225441852\\
506.01	0.00579617658613588\\
507.01	0.0058120512407385\\
508.01	0.00582825312556255\\
509.01	0.00584478899433546\\
510.01	0.0058616657396896\\
511.01	0.00587889039595234\\
512.01	0.00589647014198016\\
513.01	0.00591441230403314\\
514.01	0.00593272435869262\\
515.01	0.00595141393581679\\
516.01	0.0059704888215366\\
517.01	0.00598995696128612\\
518.01	0.00600982646286827\\
519.01	0.00603010559955078\\
520.01	0.00605080281319069\\
521.01	0.00607192671738277\\
522.01	0.00609348610062707\\
523.01	0.00611548992951208\\
524.01	0.00613794735190604\\
525.01	0.00616086770015153\\
526.01	0.0061842604942537\\
527.01	0.00620813544505668\\
528.01	0.00623250245739489\\
529.01	0.0062573716332129\\
530.01	0.00628275327463575\\
531.01	0.00630865788698202\\
532.01	0.00633509618169938\\
533.01	0.00636207907920586\\
534.01	0.00638961771161965\\
535.01	0.00641772342535149\\
536.01	0.00644640778353617\\
537.01	0.00647568256827581\\
538.01	0.00650555978266212\\
539.01	0.00653605165254259\\
540.01	0.00656717062799204\\
541.01	0.00659892938444444\\
542.01	0.00663134082343626\\
543.01	0.00666441807290583\\
544.01	0.00669817448698827\\
545.01	0.00673262364523612\\
546.01	0.00676777935119077\\
547.01	0.00680365563021905\\
548.01	0.00684026672651947\\
549.01	0.00687762709919433\\
550.01	0.00691575141726969\\
551.01	0.00695465455353335\\
552.01	0.00699435157704655\\
553.01	0.00703485774416845\\
554.01	0.00707618848791713\\
555.01	0.00711835940546861\\
556.01	0.00716138624357784\\
557.01	0.00720528488168006\\
558.01	0.00725007131240749\\
559.01	0.00729576161922826\\
560.01	0.00734237195088536\\
561.01	0.00738991849228227\\
562.01	0.00743841743142593\\
563.01	0.00748788492200297\\
564.01	0.00753833704112559\\
565.01	0.00758978974174271\\
566.01	0.0076422587991696\\
567.01	0.0076957597511456\\
568.01	0.00775030783078647\\
569.01	0.00780591789175387\\
570.01	0.007862604324924\\
571.01	0.00792038096580129\\
572.01	0.00797926099189406\\
573.01	0.00803925680924829\\
574.01	0.00810037992733435\\
575.01	0.00816264082149692\\
576.01	0.00822604878222494\\
577.01	0.00829061175058319\\
578.01	0.00835633613928231\\
579.01	0.0084232266390671\\
580.01	0.00849128601039259\\
581.01	0.00856051486076035\\
582.01	0.00863091140863674\\
583.01	0.00870247123560845\\
584.01	0.0087751870294043\\
585.01	0.00884904832168768\\
586.01	0.00892404122618365\\
587.01	0.00900014818485135\\
588.01	0.00907734773257349\\
589.01	0.00915561429437061\\
590.01	0.00923491803366146\\
591.01	0.00931522477583024\\
592.01	0.00939649603864537\\
593.01	0.00947868921029156\\
594.01	0.00956175792742154\\
595.01	0.00964565272031552\\
596.01	0.00973032201072283\\
597.01	0.00981571357120332\\
598.01	0.00990176516613374\\
599.01	0.00996919203046377\\
599.02	0.00996973314335038\\
599.03	0.0099702709571694\\
599.04	0.00997080543920336\\
599.05	0.0099713365564138\\
599.06	0.00997186427543808\\
599.07	0.0099723885625862\\
599.08	0.00997290938383756\\
599.09	0.00997342670483771\\
599.1	0.00997394049089505\\
599.11	0.00997445070697751\\
599.12	0.00997495731770918\\
599.13	0.00997546028736696\\
599.14	0.00997595957987707\\
599.15	0.00997645515881165\\
599.16	0.00997694698738526\\
599.17	0.00997743502845131\\
599.18	0.00997791924449856\\
599.19	0.00997839959764748\\
599.2	0.00997887604964662\\
599.21	0.00997934856186899\\
599.22	0.00997981709530829\\
599.23	0.00998028161057521\\
599.24	0.00998074206789365\\
599.25	0.0099811984270969\\
599.26	0.00998165064762378\\
599.27	0.00998209868851477\\
599.28	0.00998254250840806\\
599.29	0.00998298206553559\\
599.3	0.00998341731771905\\
599.31	0.00998384822236584\\
599.32	0.00998427473646497\\
599.33	0.00998469681658292\\
599.34	0.00998511441885952\\
599.35	0.0099855274990037\\
599.36	0.00998593601228926\\
599.37	0.00998633991355056\\
599.38	0.00998673915717821\\
599.39	0.00998713369711469\\
599.4	0.00998752348684992\\
599.41	0.00998790847811157\\
599.42	0.00998828861974491\\
599.43	0.00998866386008803\\
599.44	0.00998903414696681\\
599.45	0.00998939942768986\\
599.46	0.00998975964904344\\
599.47	0.00999011475728636\\
599.48	0.00999046469814472\\
599.49	0.00999080941680669\\
599.5	0.00999114885791725\\
599.51	0.00999148296557278\\
599.52	0.0099918116833157\\
599.53	0.00999213495412901\\
599.54	0.00999245272043077\\
599.55	0.00999276492406855\\
599.56	0.00999307150631381\\
599.57	0.00999337240785624\\
599.58	0.00999366756879799\\
599.59	0.00999395692864795\\
599.6	0.00999424042631585\\
599.61	0.00999451800010642\\
599.62	0.00999478958771336\\
599.63	0.0099950551262134\\
599.64	0.00999531455206019\\
599.65	0.00999556780107816\\
599.66	0.00999581480845634\\
599.67	0.00999605550874211\\
599.68	0.00999628983583487\\
599.69	0.00999651772297967\\
599.7	0.00999673910276076\\
599.71	0.00999695390709509\\
599.72	0.00999716206722577\\
599.73	0.00999736351371538\\
599.74	0.00999755817643933\\
599.75	0.00999774598457904\\
599.76	0.00999792686661517\\
599.77	0.00999810075032068\\
599.78	0.00999826756275386\\
599.79	0.00999842723025133\\
599.8	0.00999857967842092\\
599.81	0.0099987248321345\\
599.82	0.00999886261552071\\
599.83	0.00999899295195769\\
599.84	0.00999911576406567\\
599.85	0.00999923097369951\\
599.86	0.00999933850194117\\
599.87	0.00999943826909211\\
599.88	0.00999953019466558\\
599.89	0.00999961419737891\\
599.9	0.00999969019514566\\
599.91	0.00999975810506767\\
599.92	0.00999981784342713\\
599.93	0.00999986932567848\\
599.94	0.00999991246644028\\
599.95	0.00999994717948697\\
599.96	0.00999997337774056\\
599.97	0.00999999097326228\\
599.98	0.00999999987724406\\
599.99	0.01\\
600	0.01\\
};
\addplot [color=mycolor6,solid,forget plot]
  table[row sep=crcr]{%
0.01	0.00499338655419056\\
1.01	0.00499338786632048\\
2.01	0.00499338920466236\\
3.01	0.00499339056973667\\
4.01	0.00499339196207427\\
5.01	0.00499339338221642\\
6.01	0.00499339483071506\\
7.01	0.00499339630813272\\
8.01	0.00499339781504335\\
9.01	0.0049933993520319\\
10.01	0.00499340091969511\\
11.01	0.00499340251864137\\
12.01	0.00499340414949088\\
13.01	0.00499340581287615\\
14.01	0.00499340750944205\\
15.01	0.00499340923984605\\
16.01	0.00499341100475871\\
17.01	0.00499341280486372\\
18.01	0.00499341464085801\\
19.01	0.00499341651345235\\
20.01	0.00499341842337139\\
21.01	0.00499342037135406\\
22.01	0.00499342235815352\\
23.01	0.00499342438453793\\
24.01	0.00499342645129039\\
25.01	0.00499342855920965\\
26.01	0.00499343070910956\\
27.01	0.00499343290182029\\
28.01	0.00499343513818806\\
29.01	0.00499343741907576\\
30.01	0.00499343974536304\\
31.01	0.00499344211794697\\
32.01	0.00499344453774193\\
33.01	0.00499344700568019\\
34.01	0.00499344952271234\\
35.01	0.00499345208980734\\
36.01	0.00499345470795319\\
37.01	0.00499345737815723\\
38.01	0.00499346010144638\\
39.01	0.00499346287886737\\
40.01	0.0049934657114879\\
41.01	0.00499346860039604\\
42.01	0.00499347154670089\\
43.01	0.00499347455153364\\
44.01	0.00499347761604714\\
45.01	0.00499348074141685\\
46.01	0.00499348392884111\\
47.01	0.0049934871795414\\
48.01	0.00499349049476293\\
49.01	0.00499349387577538\\
50.01	0.00499349732387294\\
51.01	0.00499350084037486\\
52.01	0.00499350442662611\\
53.01	0.00499350808399769\\
54.01	0.00499351181388723\\
55.01	0.00499351561771942\\
56.01	0.00499351949694649\\
57.01	0.00499352345304914\\
58.01	0.00499352748753636\\
59.01	0.00499353160194656\\
60.01	0.00499353579784786\\
61.01	0.00499354007683879\\
62.01	0.00499354444054856\\
63.01	0.0049935488906382\\
64.01	0.0049935534288007\\
65.01	0.00499355805676154\\
66.01	0.00499356277627991\\
67.01	0.0049935675891484\\
68.01	0.0049935724971947\\
69.01	0.00499357750228167\\
70.01	0.00499358260630776\\
71.01	0.00499358781120819\\
72.01	0.00499359311895536\\
73.01	0.00499359853155977\\
74.01	0.00499360405107056\\
75.01	0.00499360967957618\\
76.01	0.00499361541920532\\
77.01	0.00499362127212724\\
78.01	0.0049936272405534\\
79.01	0.00499363332673711\\
80.01	0.00499363953297549\\
81.01	0.00499364586160916\\
82.01	0.00499365231502408\\
83.01	0.00499365889565152\\
84.01	0.00499366560596964\\
85.01	0.0049936724485036\\
86.01	0.00499367942582712\\
87.01	0.00499368654056336\\
88.01	0.00499369379538488\\
89.01	0.00499370119301604\\
90.01	0.00499370873623257\\
91.01	0.00499371642786353\\
92.01	0.00499372427079173\\
93.01	0.00499373226795513\\
94.01	0.00499374042234736\\
95.01	0.00499374873701904\\
96.01	0.00499375721507912\\
97.01	0.00499376585969533\\
98.01	0.00499377467409542\\
99.01	0.004993783661569\\
100.01	0.00499379282546757\\
101.01	0.00499380216920652\\
102.01	0.00499381169626573\\
103.01	0.00499382141019117\\
104.01	0.00499383131459584\\
105.01	0.00499384141316096\\
106.01	0.0049938517096375\\
107.01	0.00499386220784727\\
108.01	0.0049938729116842\\
109.01	0.00499388382511565\\
110.01	0.00499389495218394\\
111.01	0.00499390629700729\\
112.01	0.0049939178637817\\
113.01	0.00499392965678191\\
114.01	0.00499394168036323\\
115.01	0.00499395393896263\\
116.01	0.00499396643710051\\
117.01	0.00499397917938196\\
118.01	0.00499399217049852\\
119.01	0.00499400541522942\\
120.01	0.00499401891844364\\
121.01	0.00499403268510105\\
122.01	0.0049940467202543\\
123.01	0.00499406102905034\\
124.01	0.00499407561673199\\
125.01	0.00499409048864025\\
126.01	0.00499410565021534\\
127.01	0.00499412110699898\\
128.01	0.00499413686463544\\
129.01	0.00499415292887462\\
130.01	0.00499416930557263\\
131.01	0.00499418600069457\\
132.01	0.0049942030203159\\
133.01	0.00499422037062471\\
134.01	0.00499423805792357\\
135.01	0.00499425608863156\\
136.01	0.00499427446928641\\
137.01	0.00499429320654646\\
138.01	0.00499431230719284\\
139.01	0.00499433177813158\\
140.01	0.00499435162639551\\
141.01	0.00499437185914724\\
142.01	0.00499439248368068\\
143.01	0.00499441350742372\\
144.01	0.00499443493794008\\
145.01	0.00499445678293226\\
146.01	0.0049944790502436\\
147.01	0.00499450174786076\\
148.01	0.00499452488391619\\
149.01	0.00499454846669074\\
150.01	0.00499457250461605\\
151.01	0.00499459700627705\\
152.01	0.00499462198041467\\
153.01	0.00499464743592884\\
154.01	0.00499467338188067\\
155.01	0.00499469982749551\\
156.01	0.00499472678216545\\
157.01	0.00499475425545247\\
158.01	0.00499478225709107\\
159.01	0.00499481079699134\\
160.01	0.00499483988524153\\
161.01	0.00499486953211163\\
162.01	0.0049948997480561\\
163.01	0.00499493054371672\\
164.01	0.00499496192992582\\
165.01	0.00499499391771016\\
166.01	0.00499502651829281\\
167.01	0.00499505974309748\\
168.01	0.00499509360375137\\
169.01	0.00499512811208846\\
170.01	0.00499516328015329\\
171.01	0.00499519912020387\\
172.01	0.00499523564471568\\
173.01	0.00499527286638476\\
174.01	0.00499531079813151\\
175.01	0.00499534945310437\\
176.01	0.00499538884468316\\
177.01	0.00499542898648321\\
178.01	0.00499546989235899\\
179.01	0.00499551157640772\\
180.01	0.00499555405297349\\
181.01	0.00499559733665106\\
182.01	0.00499564144228974\\
183.01	0.00499568638499785\\
184.01	0.00499573218014604\\
185.01	0.00499577884337205\\
186.01	0.00499582639058445\\
187.01	0.00499587483796736\\
188.01	0.00499592420198405\\
189.01	0.00499597449938159\\
190.01	0.00499602574719532\\
191.01	0.00499607796275328\\
192.01	0.00499613116368033\\
193.01	0.00499618536790311\\
194.01	0.00499624059365412\\
195.01	0.00499629685947682\\
196.01	0.00499635418423008\\
197.01	0.00499641258709284\\
198.01	0.00499647208756888\\
199.01	0.00499653270549189\\
200.01	0.00499659446103007\\
201.01	0.00499665737469117\\
202.01	0.00499672146732727\\
203.01	0.00499678676014038\\
204.01	0.00499685327468653\\
205.01	0.00499692103288189\\
206.01	0.0049969900570075\\
207.01	0.00499706036971419\\
208.01	0.00499713199402853\\
209.01	0.00499720495335762\\
210.01	0.00499727927149465\\
211.01	0.0049973549726242\\
212.01	0.00499743208132786\\
213.01	0.0049975106225897\\
214.01	0.00499759062180158\\
215.01	0.00499767210476922\\
216.01	0.00499775509771747\\
217.01	0.00499783962729614\\
218.01	0.00499792572058577\\
219.01	0.00499801340510302\\
220.01	0.00499810270880725\\
221.01	0.00499819366010555\\
222.01	0.00499828628785939\\
223.01	0.00499838062139017\\
224.01	0.004998476690485\\
225.01	0.00499857452540325\\
226.01	0.00499867415688249\\
227.01	0.00499877561614422\\
228.01	0.00499887893490024\\
229.01	0.00499898414535897\\
230.01	0.0049990912802317\\
231.01	0.00499920037273871\\
232.01	0.00499931145661536\\
233.01	0.00499942456611892\\
234.01	0.00499953973603448\\
235.01	0.00499965700168177\\
236.01	0.00499977639892117\\
237.01	0.00499989796416045\\
238.01	0.00500002173436129\\
239.01	0.00500014774704567\\
240.01	0.00500027604030258\\
241.01	0.00500040665279465\\
242.01	0.00500053962376464\\
243.01	0.00500067499304205\\
244.01	0.0050008128010505\\
245.01	0.00500095308881363\\
246.01	0.00500109589796245\\
247.01	0.00500124127074171\\
248.01	0.00500138925001742\\
249.01	0.0050015398792831\\
250.01	0.00500169320266722\\
251.01	0.00500184926494003\\
252.01	0.00500200811152057\\
253.01	0.00500216978848361\\
254.01	0.00500233434256725\\
255.01	0.00500250182118003\\
256.01	0.00500267227240769\\
257.01	0.00500284574502116\\
258.01	0.00500302228848373\\
259.01	0.00500320195295877\\
260.01	0.00500338478931665\\
261.01	0.00500357084914305\\
262.01	0.0050037601847466\\
263.01	0.00500395284916681\\
264.01	0.00500414889618174\\
265.01	0.00500434838031662\\
266.01	0.00500455135685159\\
267.01	0.00500475788183073\\
268.01	0.00500496801207019\\
269.01	0.00500518180516701\\
270.01	0.0050053993195083\\
271.01	0.0050056206142802\\
272.01	0.00500584574947749\\
273.01	0.00500607478591301\\
274.01	0.00500630778522779\\
275.01	0.00500654480990131\\
276.01	0.0050067859232618\\
277.01	0.00500703118949763\\
278.01	0.00500728067366785\\
279.01	0.00500753444171483\\
280.01	0.0050077925604757\\
281.01	0.0050080550976951\\
282.01	0.00500832212203871\\
283.01	0.00500859370310652\\
284.01	0.00500886991144747\\
285.01	0.00500915081857438\\
286.01	0.00500943649697945\\
287.01	0.00500972702015111\\
288.01	0.00501002246259069\\
289.01	0.00501032289983132\\
290.01	0.00501062840845641\\
291.01	0.00501093906612013\\
292.01	0.0050112549515685\\
293.01	0.00501157614466213\\
294.01	0.00501190272639887\\
295.01	0.00501223477894006\\
296.01	0.00501257238563586\\
297.01	0.00501291563105405\\
298.01	0.00501326460100887\\
299.01	0.00501361938259264\\
300.01	0.00501398006420875\\
301.01	0.00501434673560681\\
302.01	0.00501471948791993\\
303.01	0.00501509841370372\\
304.01	0.00501548360697839\\
305.01	0.00501587516327254\\
306.01	0.00501627317967014\\
307.01	0.00501667775486028\\
308.01	0.00501708898918908\\
309.01	0.00501750698471545\\
310.01	0.00501793184526991\\
311.01	0.00501836367651623\\
312.01	0.00501880258601761\\
313.01	0.00501924868330537\\
314.01	0.00501970207995231\\
315.01	0.00502016288964997\\
316.01	0.00502063122828927\\
317.01	0.00502110721404658\\
318.01	0.00502159096747391\\
319.01	0.00502208261159244\\
320.01	0.00502258227199301\\
321.01	0.00502309007693881\\
322.01	0.00502360615747541\\
323.01	0.00502413064754381\\
324.01	0.0050246636840998\\
325.01	0.00502520540723725\\
326.01	0.00502575596031725\\
327.01	0.00502631549010182\\
328.01	0.00502688414689204\\
329.01	0.00502746208467171\\
330.01	0.00502804946125478\\
331.01	0.00502864643843685\\
332.01	0.00502925318215132\\
333.01	0.00502986986262773\\
334.01	0.00503049665455403\\
335.01	0.00503113373724073\\
336.01	0.00503178129478642\\
337.01	0.00503243951624402\\
338.01	0.00503310859578748\\
339.01	0.0050337887328762\\
340.01	0.00503448013241763\\
341.01	0.00503518300492596\\
342.01	0.00503589756667509\\
343.01	0.00503662403984562\\
344.01	0.0050373626526614\\
345.01	0.00503811363951756\\
346.01	0.00503887724109417\\
347.01	0.0050396537044564\\
348.01	0.00504044328313653\\
349.01	0.00504124623719773\\
350.01	0.00504206283327525\\
351.01	0.00504289334459476\\
352.01	0.00504373805096422\\
353.01	0.00504459723873756\\
354.01	0.00504547120074937\\
355.01	0.00504636023621806\\
356.01	0.00504726465061738\\
357.01	0.00504818475551667\\
358.01	0.0050491208683893\\
359.01	0.00505007331239125\\
360.01	0.00505104241611436\\
361.01	0.00505202851331574\\
362.01	0.00505303194263147\\
363.01	0.00505405304728081\\
364.01	0.00505509217477058\\
365.01	0.00505614967661133\\
366.01	0.00505722590805743\\
367.01	0.00505832122788587\\
368.01	0.00505943599823176\\
369.01	0.00506057058449409\\
370.01	0.00506172535533017\\
371.01	0.0050629006827536\\
372.01	0.00506409694234744\\
373.01	0.00506531451359821\\
374.01	0.0050665537803536\\
375.01	0.00506781513138806\\
376.01	0.00506909896105726\\
377.01	0.00507040567000097\\
378.01	0.0050717356658435\\
379.01	0.00507308936382878\\
380.01	0.00507446718732461\\
381.01	0.0050758695681664\\
382.01	0.00507729694690821\\
383.01	0.00507874977306559\\
384.01	0.00508022850536756\\
385.01	0.00508173361201872\\
386.01	0.00508326557097012\\
387.01	0.00508482487020107\\
388.01	0.0050864120080109\\
389.01	0.0050880274933212\\
390.01	0.0050896718459903\\
391.01	0.00509134559713759\\
392.01	0.00509304928948121\\
393.01	0.00509478347768592\\
394.01	0.00509654872872523\\
395.01	0.00509834562225397\\
396.01	0.00510017475099574\\
397.01	0.00510203672114171\\
398.01	0.00510393215276361\\
399.01	0.00510586168023992\\
400.01	0.00510782595269503\\
401.01	0.00510982563445289\\
402.01	0.00511186140550353\\
403.01	0.00511393396198424\\
404.01	0.00511604401667314\\
405.01	0.00511819229949688\\
406.01	0.00512037955805184\\
407.01	0.00512260655813837\\
408.01	0.00512487408430706\\
409.01	0.00512718294041817\\
410.01	0.00512953395021211\\
411.01	0.00513192795789152\\
412.01	0.00513436582871407\\
413.01	0.0051368484495937\\
414.01	0.00513937672971189\\
415.01	0.00514195160113549\\
416.01	0.00514457401944182\\
417.01	0.00514724496434761\\
418.01	0.00514996544034281\\
419.01	0.00515273647732563\\
420.01	0.00515555913123816\\
421.01	0.00515843448470016\\
422.01	0.00516136364764033\\
423.01	0.00516434775792068\\
424.01	0.00516738798195487\\
425.01	0.00517048551531558\\
426.01	0.00517364158333061\\
427.01	0.00517685744166467\\
428.01	0.00518013437688505\\
429.01	0.00518347370700885\\
430.01	0.00518687678202942\\
431.01	0.00519034498442225\\
432.01	0.0051938797296254\\
433.01	0.00519748246649712\\
434.01	0.00520115467774707\\
435.01	0.00520489788034244\\
436.01	0.00520871362588934\\
437.01	0.00521260350098952\\
438.01	0.00521656912757588\\
439.01	0.00522061216322811\\
440.01	0.00522473430147351\\
441.01	0.0052289372720771\\
442.01	0.00523322284132761\\
443.01	0.00523759281232565\\
444.01	0.00524204902528289\\
445.01	0.00524659335784246\\
446.01	0.00525122772542844\\
447.01	0.00525595408163848\\
448.01	0.00526077441868902\\
449.01	0.00526569076792553\\
450.01	0.00527070520040915\\
451.01	0.00527581982759093\\
452.01	0.0052810368020802\\
453.01	0.00528635831851634\\
454.01	0.00529178661454525\\
455.01	0.0052973239718996\\
456.01	0.00530297271757832\\
457.01	0.00530873522511274\\
458.01	0.00531461391590231\\
459.01	0.00532061126059696\\
460.01	0.00532672978049861\\
461.01	0.00533297204894941\\
462.01	0.00533934069267547\\
463.01	0.00534583839305854\\
464.01	0.00535246788731461\\
465.01	0.00535923196957572\\
466.01	0.00536613349188794\\
467.01	0.00537317536516092\\
468.01	0.00538036056011246\\
469.01	0.00538769210822991\\
470.01	0.00539517310275549\\
471.01	0.00540280669969907\\
472.01	0.00541059611888273\\
473.01	0.00541854464502098\\
474.01	0.00542665562884202\\
475.01	0.00543493248825357\\
476.01	0.00544337870955725\\
477.01	0.00545199784871682\\
478.01	0.00546079353268274\\
479.01	0.00546976946077612\\
480.01	0.0054789294061358\\
481.01	0.00548827721722787\\
482.01	0.00549781681941935\\
483.01	0.00550755221661568\\
484.01	0.00551748749295894\\
485.01	0.00552762681458464\\
486.01	0.00553797443143256\\
487.01	0.00554853467910639\\
488.01	0.00555931198077608\\
489.01	0.0055703108491171\\
490.01	0.00558153588827973\\
491.01	0.00559299179588486\\
492.01	0.00560468336504057\\
493.01	0.00561661548637882\\
494.01	0.00562879315011237\\
495.01	0.00564122144811656\\
496.01	0.0056539055760409\\
497.01	0.0056668508354585\\
498.01	0.00568006263606116\\
499.01	0.0056935464979061\\
500.01	0.00570730805371605\\
501.01	0.00572135305123445\\
502.01	0.00573568735563307\\
503.01	0.00575031695197263\\
504.01	0.00576524794771362\\
505.01	0.00578048657527726\\
506.01	0.0057960391946542\\
507.01	0.00581191229605887\\
508.01	0.00582811250262841\\
509.01	0.00584464657316448\\
510.01	0.00586152140491583\\
511.01	0.00587874403640189\\
512.01	0.0058963216502738\\
513.01	0.00591426157621598\\
514.01	0.00593257129388382\\
515.01	0.00595125843587969\\
516.01	0.00597033079076532\\
517.01	0.00598979630611013\\
518.01	0.00600966309157334\\
519.01	0.00602993942201802\\
520.01	0.00605063374065308\\
521.01	0.00607175466219925\\
522.01	0.00609331097607468\\
523.01	0.00611531164959501\\
524.01	0.00613776583118063\\
525.01	0.00616068285356548\\
526.01	0.00618407223699981\\
527.01	0.00620794369243782\\
528.01	0.00623230712470173\\
529.01	0.00625717263560928\\
530.01	0.00628255052705605\\
531.01	0.00630845130403507\\
532.01	0.00633488567758089\\
533.01	0.00636186456761978\\
534.01	0.00638939910570525\\
535.01	0.0064175006376183\\
536.01	0.0064461807258063\\
537.01	0.00647545115163257\\
538.01	0.0065053239174051\\
539.01	0.00653581124814943\\
540.01	0.00656692559308616\\
541.01	0.00659867962676944\\
542.01	0.00663108624983591\\
543.01	0.00666415858931164\\
544.01	0.00669790999841302\\
545.01	0.00673235405577483\\
546.01	0.00676750456402943\\
547.01	0.00680337554765068\\
548.01	0.00683998124996985\\
549.01	0.00687733612925698\\
550.01	0.00691545485375123\\
551.01	0.0069543522955103\\
552.01	0.00699404352293434\\
553.01	0.00703454379180462\\
554.01	0.00707586853465878\\
555.01	0.00711803334830747\\
556.01	0.0071610539792728\\
557.01	0.00720494630691036\\
558.01	0.00724972632394758\\
559.01	0.00729541011414746\\
560.01	0.00734201382677425\\
561.01	0.00738955364750814\\
562.01	0.0074380457654204\\
563.01	0.00748750633558497\\
564.01	0.00753795143686202\\
565.01	0.00758939702435083\\
566.01	0.00764185887596373\\
567.01	0.00769535253253253\\
568.01	0.00774989323081213\\
569.01	0.00780549582870473\\
570.01	0.00786217472198644\\
571.01	0.007919943751782\\
572.01	0.00797881610200229\\
573.01	0.00803880418594278\\
574.01	0.0080999195212333\\
575.01	0.00816217259234991\\
576.01	0.00822557269994251\\
577.01	0.00829012779631636\\
578.01	0.00835584430654082\\
579.01	0.00842272693485917\\
580.01	0.00849077845636217\\
581.01	0.00855999949428942\\
582.01	0.00863038828386762\\
583.01	0.00870194042432736\\
584.01	0.0087746486217084\\
585.01	0.00884850242633413\\
586.01	0.00892348797048993\\
587.01	0.00899958771397793\\
588.01	0.00907678020797329\\
589.01	0.00915503989113028\\
590.01	0.0092343369363852\\
591.01	0.00931463717262456\\
592.01	0.00939590211264666\\
593.01	0.00947808912803386\\
594.01	0.00956115182316124\\
595.01	0.00964504067520493\\
596.01	0.00972970402544469\\
597.01	0.0098150895303299\\
598.01	0.00990114621169971\\
599.01	0.00996919203046377\\
599.02	0.00996973314335038\\
599.03	0.0099702709571694\\
599.04	0.00997080543920336\\
599.05	0.0099713365564138\\
599.06	0.00997186427543808\\
599.07	0.0099723885625862\\
599.08	0.00997290938383756\\
599.09	0.00997342670483771\\
599.1	0.00997394049089505\\
599.11	0.00997445070697751\\
599.12	0.00997495731770918\\
599.13	0.00997546028736696\\
599.14	0.00997595957987707\\
599.15	0.00997645515881166\\
599.16	0.00997694698738526\\
599.17	0.00997743502845131\\
599.18	0.00997791924449856\\
599.19	0.00997839959764748\\
599.2	0.00997887604964662\\
599.21	0.00997934856186899\\
599.22	0.00997981709530829\\
599.23	0.00998028161057521\\
599.24	0.00998074206789365\\
599.25	0.0099811984270969\\
599.26	0.00998165064762378\\
599.27	0.00998209868851477\\
599.28	0.00998254250840806\\
599.29	0.00998298206553559\\
599.3	0.00998341731771905\\
599.31	0.00998384822236584\\
599.32	0.00998427473646497\\
599.33	0.00998469681658292\\
599.34	0.00998511441885952\\
599.35	0.0099855274990037\\
599.36	0.00998593601228926\\
599.37	0.00998633991355056\\
599.38	0.00998673915717821\\
599.39	0.00998713369711469\\
599.4	0.00998752348684992\\
599.41	0.00998790847811156\\
599.42	0.00998828861974491\\
599.43	0.00998866386008804\\
599.44	0.00998903414696681\\
599.45	0.00998939942768985\\
599.46	0.00998975964904344\\
599.47	0.00999011475728636\\
599.48	0.00999046469814472\\
599.49	0.00999080941680669\\
599.5	0.00999114885791725\\
599.51	0.00999148296557278\\
599.52	0.0099918116833157\\
599.53	0.00999213495412901\\
599.54	0.00999245272043077\\
599.55	0.00999276492406855\\
599.56	0.00999307150631382\\
599.57	0.00999337240785624\\
599.58	0.00999366756879799\\
599.59	0.00999395692864795\\
599.6	0.00999424042631585\\
599.61	0.00999451800010642\\
599.62	0.00999478958771336\\
599.63	0.0099950551262134\\
599.64	0.00999531455206019\\
599.65	0.00999556780107816\\
599.66	0.00999581480845634\\
599.67	0.00999605550874211\\
599.68	0.00999628983583487\\
599.69	0.00999651772297967\\
599.7	0.00999673910276076\\
599.71	0.00999695390709509\\
599.72	0.00999716206722577\\
599.73	0.00999736351371538\\
599.74	0.00999755817643933\\
599.75	0.00999774598457904\\
599.76	0.00999792686661517\\
599.77	0.00999810075032068\\
599.78	0.00999826756275386\\
599.79	0.00999842723025133\\
599.8	0.00999857967842092\\
599.81	0.0099987248321345\\
599.82	0.00999886261552071\\
599.83	0.00999899295195769\\
599.84	0.00999911576406567\\
599.85	0.00999923097369951\\
599.86	0.00999933850194117\\
599.87	0.00999943826909211\\
599.88	0.00999953019466558\\
599.89	0.00999961419737891\\
599.9	0.00999969019514566\\
599.91	0.00999975810506767\\
599.92	0.00999981784342713\\
599.93	0.00999986932567848\\
599.94	0.00999991246644028\\
599.95	0.00999994717948697\\
599.96	0.00999997337774056\\
599.97	0.00999999097326228\\
599.98	0.00999999987724406\\
599.99	0.01\\
600	0.01\\
};
\addplot [color=mycolor7,solid,forget plot]
  table[row sep=crcr]{%
0.01	0.00495449873497758\\
1.01	0.00495450029197152\\
2.01	0.00495450188039694\\
3.01	0.00495450350088551\\
4.01	0.00495450515408258\\
5.01	0.00495450684064533\\
6.01	0.0049545085612448\\
7.01	0.00495451031656516\\
8.01	0.0049545121073042\\
9.01	0.00495451393417384\\
10.01	0.00495451579790009\\
11.01	0.00495451769922321\\
12.01	0.00495451963889873\\
13.01	0.00495452161769698\\
14.01	0.00495452363640381\\
15.01	0.00495452569582075\\
16.01	0.00495452779676507\\
17.01	0.00495452994007062\\
18.01	0.00495453212658796\\
19.01	0.00495453435718417\\
20.01	0.00495453663274401\\
21.01	0.00495453895416986\\
22.01	0.00495454132238195\\
23.01	0.00495454373831894\\
24.01	0.0049545462029381\\
25.01	0.00495454871721564\\
26.01	0.00495455128214774\\
27.01	0.00495455389874973\\
28.01	0.00495455656805779\\
29.01	0.00495455929112838\\
30.01	0.00495456206903892\\
31.01	0.00495456490288834\\
32.01	0.00495456779379762\\
33.01	0.00495457074290986\\
34.01	0.0049545737513909\\
35.01	0.00495457682042971\\
36.01	0.00495457995123902\\
37.01	0.00495458314505552\\
38.01	0.00495458640314056\\
39.01	0.0049545897267807\\
40.01	0.00495459311728746\\
41.01	0.00495459657599899\\
42.01	0.00495460010427994\\
43.01	0.0049546037035217\\
44.01	0.00495460737514361\\
45.01	0.00495461112059283\\
46.01	0.0049546149413455\\
47.01	0.00495461883890681\\
48.01	0.0049546228148121\\
49.01	0.0049546268706267\\
50.01	0.00495463100794719\\
51.01	0.00495463522840174\\
52.01	0.00495463953365069\\
53.01	0.00495464392538732\\
54.01	0.00495464840533831\\
55.01	0.00495465297526451\\
56.01	0.00495465763696179\\
57.01	0.00495466239226126\\
58.01	0.00495466724303043\\
59.01	0.00495467219117345\\
60.01	0.00495467723863223\\
61.01	0.00495468238738704\\
62.01	0.00495468763945725\\
63.01	0.00495469299690213\\
64.01	0.00495469846182128\\
65.01	0.00495470403635614\\
66.01	0.00495470972268981\\
67.01	0.00495471552304907\\
68.01	0.00495472143970416\\
69.01	0.00495472747497004\\
70.01	0.00495473363120746\\
71.01	0.00495473991082342\\
72.01	0.00495474631627246\\
73.01	0.00495475285005728\\
74.01	0.00495475951472986\\
75.01	0.00495476631289217\\
76.01	0.00495477324719747\\
77.01	0.00495478032035129\\
78.01	0.00495478753511203\\
79.01	0.0049547948942924\\
80.01	0.00495480240076\\
81.01	0.00495481005743911\\
82.01	0.00495481786731085\\
83.01	0.00495482583341533\\
84.01	0.00495483395885171\\
85.01	0.00495484224678019\\
86.01	0.0049548507004227\\
87.01	0.00495485932306419\\
88.01	0.00495486811805396\\
89.01	0.0049548770888066\\
90.01	0.00495488623880375\\
91.01	0.00495489557159466\\
92.01	0.00495490509079834\\
93.01	0.00495491480010417\\
94.01	0.00495492470327344\\
95.01	0.00495493480414112\\
96.01	0.00495494510661651\\
97.01	0.00495495561468547\\
98.01	0.00495496633241142\\
99.01	0.00495497726393669\\
100.01	0.00495498841348437\\
101.01	0.00495499978535961\\
102.01	0.00495501138395121\\
103.01	0.00495502321373342\\
104.01	0.0049550352792672\\
105.01	0.00495504758520215\\
106.01	0.00495506013627819\\
107.01	0.00495507293732686\\
108.01	0.00495508599327349\\
109.01	0.00495509930913893\\
110.01	0.00495511289004101\\
111.01	0.00495512674119682\\
112.01	0.00495514086792431\\
113.01	0.00495515527564409\\
114.01	0.0049551699698817\\
115.01	0.00495518495626897\\
116.01	0.00495520024054676\\
117.01	0.00495521582856674\\
118.01	0.004955231726293\\
119.01	0.00495524793980481\\
120.01	0.00495526447529829\\
121.01	0.00495528133908892\\
122.01	0.00495529853761351\\
123.01	0.00495531607743251\\
124.01	0.00495533396523258\\
125.01	0.00495535220782854\\
126.01	0.00495537081216579\\
127.01	0.00495538978532296\\
128.01	0.00495540913451452\\
129.01	0.00495542886709247\\
130.01	0.00495544899054995\\
131.01	0.0049554695125229\\
132.01	0.00495549044079345\\
133.01	0.00495551178329211\\
134.01	0.00495553354810058\\
135.01	0.00495555574345463\\
136.01	0.00495557837774689\\
137.01	0.00495560145952961\\
138.01	0.00495562499751775\\
139.01	0.00495564900059182\\
140.01	0.00495567347780106\\
141.01	0.00495569843836585\\
142.01	0.00495572389168203\\
143.01	0.00495574984732285\\
144.01	0.0049557763150429\\
145.01	0.00495580330478111\\
146.01	0.0049558308266643\\
147.01	0.00495585889101022\\
148.01	0.00495588750833134\\
149.01	0.00495591668933791\\
150.01	0.00495594644494192\\
151.01	0.00495597678626046\\
152.01	0.00495600772461968\\
153.01	0.00495603927155806\\
154.01	0.00495607143883059\\
155.01	0.00495610423841204\\
156.01	0.0049561376825016\\
157.01	0.00495617178352639\\
158.01	0.00495620655414555\\
159.01	0.00495624200725424\\
160.01	0.00495627815598809\\
161.01	0.00495631501372726\\
162.01	0.00495635259410016\\
163.01	0.00495639091098891\\
164.01	0.00495642997853274\\
165.01	0.00495646981113275\\
166.01	0.00495651042345693\\
167.01	0.00495655183044394\\
168.01	0.00495659404730854\\
169.01	0.00495663708954593\\
170.01	0.00495668097293626\\
171.01	0.00495672571355051\\
172.01	0.00495677132775403\\
173.01	0.00495681783221298\\
174.01	0.00495686524389857\\
175.01	0.00495691358009234\\
176.01	0.00495696285839146\\
177.01	0.00495701309671413\\
178.01	0.00495706431330482\\
179.01	0.00495711652673982\\
180.01	0.0049571697559327\\
181.01	0.00495722402013991\\
182.01	0.00495727933896652\\
183.01	0.00495733573237175\\
184.01	0.00495739322067508\\
185.01	0.00495745182456218\\
186.01	0.00495751156509033\\
187.01	0.00495757246369501\\
188.01	0.00495763454219597\\
189.01	0.00495769782280309\\
190.01	0.00495776232812312\\
191.01	0.00495782808116539\\
192.01	0.00495789510534889\\
193.01	0.00495796342450822\\
194.01	0.00495803306290048\\
195.01	0.00495810404521171\\
196.01	0.00495817639656367\\
197.01	0.00495825014252073\\
198.01	0.00495832530909653\\
199.01	0.00495840192276086\\
200.01	0.00495848001044674\\
201.01	0.00495855959955767\\
202.01	0.00495864071797426\\
203.01	0.00495872339406181\\
204.01	0.00495880765667776\\
205.01	0.00495889353517825\\
206.01	0.00495898105942588\\
207.01	0.00495907025979767\\
208.01	0.00495916116719171\\
209.01	0.00495925381303512\\
210.01	0.00495934822929152\\
211.01	0.00495944444846912\\
212.01	0.00495954250362785\\
213.01	0.00495964242838731\\
214.01	0.00495974425693477\\
215.01	0.00495984802403256\\
216.01	0.00495995376502656\\
217.01	0.00496006151585361\\
218.01	0.00496017131304975\\
219.01	0.00496028319375823\\
220.01	0.00496039719573714\\
221.01	0.00496051335736789\\
222.01	0.00496063171766306\\
223.01	0.00496075231627453\\
224.01	0.00496087519350162\\
225.01	0.00496100039029876\\
226.01	0.00496112794828414\\
227.01	0.0049612579097475\\
228.01	0.00496139031765823\\
229.01	0.00496152521567356\\
230.01	0.00496166264814635\\
231.01	0.00496180266013309\\
232.01	0.00496194529740224\\
233.01	0.00496209060644175\\
234.01	0.00496223863446724\\
235.01	0.00496238942942935\\
236.01	0.00496254304002211\\
237.01	0.0049626995156903\\
238.01	0.00496285890663714\\
239.01	0.00496302126383162\\
240.01	0.00496318663901635\\
241.01	0.0049633550847142\\
242.01	0.00496352665423608\\
243.01	0.00496370140168802\\
244.01	0.00496387938197736\\
245.01	0.00496406065082051\\
246.01	0.00496424526474883\\
247.01	0.00496443328111536\\
248.01	0.00496462475810122\\
249.01	0.00496481975472149\\
250.01	0.00496501833083085\\
251.01	0.00496522054712963\\
252.01	0.00496542646516901\\
253.01	0.00496563614735633\\
254.01	0.00496584965695947\\
255.01	0.00496606705811198\\
256.01	0.0049662884158171\\
257.01	0.004966513795952\\
258.01	0.00496674326527102\\
259.01	0.00496697689140936\\
260.01	0.0049672147428861\\
261.01	0.00496745688910649\\
262.01	0.00496770340036452\\
263.01	0.00496795434784431\\
264.01	0.00496820980362171\\
265.01	0.00496846984066524\\
266.01	0.00496873453283674\\
267.01	0.00496900395489093\\
268.01	0.00496927818247525\\
269.01	0.00496955729212913\\
270.01	0.0049698413612816\\
271.01	0.00497013046825032\\
272.01	0.0049704246922382\\
273.01	0.0049707241133307\\
274.01	0.00497102881249172\\
275.01	0.00497133887155958\\
276.01	0.00497165437324192\\
277.01	0.00497197540111015\\
278.01	0.00497230203959358\\
279.01	0.00497263437397219\\
280.01	0.0049729724903692\\
281.01	0.00497331647574369\\
282.01	0.00497366641788095\\
283.01	0.00497402240538389\\
284.01	0.00497438452766298\\
285.01	0.00497475287492517\\
286.01	0.00497512753816343\\
287.01	0.00497550860914453\\
288.01	0.00497589618039711\\
289.01	0.00497629034519845\\
290.01	0.00497669119756142\\
291.01	0.00497709883222086\\
292.01	0.004977513344619\\
293.01	0.00497793483089126\\
294.01	0.00497836338785153\\
295.01	0.00497879911297684\\
296.01	0.0049792421043927\\
297.01	0.00497969246085737\\
298.01	0.00498015028174784\\
299.01	0.00498061566704373\\
300.01	0.00498108871731409\\
301.01	0.00498156953370295\\
302.01	0.00498205821791588\\
303.01	0.00498255487220821\\
304.01	0.00498305959937317\\
305.01	0.00498357250273236\\
306.01	0.00498409368612773\\
307.01	0.00498462325391459\\
308.01	0.00498516131095768\\
309.01	0.00498570796262973\\
310.01	0.00498626331481203\\
311.01	0.0049868274738999\\
312.01	0.00498740054681021\\
313.01	0.00498798264099452\\
314.01	0.0049885738644562\\
315.01	0.00498917432577321\\
316.01	0.00498978413412694\\
317.01	0.00499040339933726\\
318.01	0.00499103223190508\\
319.01	0.00499167074306354\\
320.01	0.00499231904483693\\
321.01	0.00499297725011064\\
322.01	0.00499364547271063\\
323.01	0.00499432382749534\\
324.01	0.00499501243046012\\
325.01	0.00499571139885524\\
326.01	0.00499642085131839\\
327.01	0.00499714090802421\\
328.01	0.00499787169085036\\
329.01	0.00499861332356202\\
330.01	0.00499936593201675\\
331.01	0.00500012964438984\\
332.01	0.00500090459142272\\
333.01	0.0050016909066939\\
334.01	0.00500248872691636\\
335.01	0.00500329819225946\\
336.01	0.00500411944669937\\
337.01	0.00500495263839654\\
338.01	0.00500579792010264\\
339.01	0.00500665544959673\\
340.01	0.00500752539015163\\
341.01	0.00500840791102919\\
342.01	0.0050093031880058\\
343.01	0.00501021140392467\\
344.01	0.00501113274927688\\
345.01	0.00501206742280441\\
346.01	0.0050130156321261\\
347.01	0.0050139775943794\\
348.01	0.00501495353687549\\
349.01	0.00501594369775796\\
350.01	0.00501694832666061\\
351.01	0.00501796768535226\\
352.01	0.00501900204835869\\
353.01	0.00502005170354786\\
354.01	0.00502111695266431\\
355.01	0.00502219811179455\\
356.01	0.00502329551174445\\
357.01	0.00502440949830703\\
358.01	0.00502554043239732\\
359.01	0.00502668869002935\\
360.01	0.00502785466210898\\
361.01	0.00502903875401514\\
362.01	0.00503024138494439\\
363.01	0.00503146298699349\\
364.01	0.0050327040039593\\
365.01	0.00503396488984265\\
366.01	0.00503524610704879\\
367.01	0.00503654812429304\\
368.01	0.00503787141423349\\
369.01	0.00503921645087724\\
370.01	0.00504058370683111\\
371.01	0.0050419736505011\\
372.01	0.00504338674337992\\
373.01	0.00504482343760372\\
374.01	0.00504628417399551\\
375.01	0.00504776938084786\\
376.01	0.00504927947371088\\
377.01	0.00505081485643616\\
378.01	0.00505237592364963\\
379.01	0.00505396306465568\\
380.01	0.00505557666840886\\
381.01	0.00505721712776508\\
382.01	0.00505888484080708\\
383.01	0.00506058021101743\\
384.01	0.00506230364741797\\
385.01	0.00506405556471838\\
386.01	0.00506583638347773\\
387.01	0.00506764653027661\\
388.01	0.00506948643790339\\
389.01	0.00507135654555459\\
390.01	0.00507325729904906\\
391.01	0.00507518915106002\\
392.01	0.00507715256136289\\
393.01	0.00507914799710259\\
394.01	0.00508117593307975\\
395.01	0.0050832368520585\\
396.01	0.00508533124509594\\
397.01	0.00508745961189566\\
398.01	0.00508962246118645\\
399.01	0.00509182031112708\\
400.01	0.00509405368973993\\
401.01	0.00509632313537381\\
402.01	0.00509862919719781\\
403.01	0.00510097243572839\\
404.01	0.00510335342339101\\
405.01	0.00510577274511721\\
406.01	0.00510823099897995\\
407.01	0.00511072879686754\\
408.01	0.0051132667651991\\
409.01	0.00511584554568112\\
410.01	0.00511846579610811\\
411.01	0.0051211281912082\\
412.01	0.0051238334235334\\
413.01	0.00512658220439763\\
414.01	0.00512937526486082\\
415.01	0.00513221335676202\\
416.01	0.00513509725379858\\
417.01	0.00513802775265401\\
418.01	0.00514100567417107\\
419.01	0.00514403186457146\\
420.01	0.00514710719671833\\
421.01	0.00515023257141975\\
422.01	0.00515340891877078\\
423.01	0.00515663719952925\\
424.01	0.00515991840652104\\
425.01	0.00516325356606923\\
426.01	0.00516664373944022\\
427.01	0.00517009002429979\\
428.01	0.00517359355616892\\
429.01	0.00517715550987126\\
430.01	0.00518077710095871\\
431.01	0.00518445958710489\\
432.01	0.00518820426945177\\
433.01	0.00519201249389373\\
434.01	0.00519588565228472\\
435.01	0.00519982518355067\\
436.01	0.00520383257468861\\
437.01	0.00520790936163687\\
438.01	0.00521205712999526\\
439.01	0.00521627751557836\\
440.01	0.00522057220478593\\
441.01	0.00522494293477301\\
442.01	0.00522939149340879\\
443.01	0.00523391971901313\\
444.01	0.00523852949986739\\
445.01	0.00524322277349935\\
446.01	0.00524800152575124\\
447.01	0.00525286778964769\\
448.01	0.0052578236440903\\
449.01	0.005262871212418\\
450.01	0.00526801266088528\\
451.01	0.00527325019712209\\
452.01	0.00527858606865597\\
453.01	0.00528402256158897\\
454.01	0.0052895619995329\\
455.01	0.00529520674291668\\
456.01	0.00530095918878035\\
457.01	0.00530682177116795\\
458.01	0.00531279696221322\\
459.01	0.00531888727398484\\
460.01	0.0053250952611094\\
461.01	0.00533142352412674\\
462.01	0.00533787471344524\\
463.01	0.00534445153366397\\
464.01	0.0053511567479194\\
465.01	0.00535799318182135\\
466.01	0.00536496372649997\\
467.01	0.00537207134042046\\
468.01	0.00537931905026122\\
469.01	0.00538670995152291\\
470.01	0.00539424720906119\\
471.01	0.00540193405755518\\
472.01	0.00540977380191732\\
473.01	0.00541776981765332\\
474.01	0.00542592555118307\\
475.01	0.00543424452013756\\
476.01	0.00544273031364788\\
477.01	0.00545138659264675\\
478.01	0.00546021709020519\\
479.01	0.00546922561192885\\
480.01	0.00547841603644079\\
481.01	0.00548779231597983\\
482.01	0.0054973584771407\\
483.01	0.00550711862178339\\
484.01	0.00551707692813627\\
485.01	0.00552723765211148\\
486.01	0.00553760512884571\\
487.01	0.005548183774471\\
488.01	0.0055589780881089\\
489.01	0.00556999265406935\\
490.01	0.00558123214422312\\
491.01	0.00559270132050174\\
492.01	0.00560440503747062\\
493.01	0.00561634824491018\\
494.01	0.00562853599034233\\
495.01	0.00564097342144393\\
496.01	0.00565366578831375\\
497.01	0.00566661844558945\\
498.01	0.00567983685445803\\
499.01	0.00569332658464414\\
500.01	0.00570709331646155\\
501.01	0.00572114284296089\\
502.01	0.00573548107217994\\
503.01	0.00575011402949795\\
504.01	0.00576504786009478\\
505.01	0.0057802888315108\\
506.01	0.00579584333630428\\
507.01	0.00581171789479883\\
508.01	0.00582791915791098\\
509.01	0.0058444539100473\\
510.01	0.00586132907205845\\
511.01	0.00587855170423747\\
512.01	0.00589612900934987\\
513.01	0.0059140683356845\\
514.01	0.00593237718011849\\
515.01	0.00595106319119111\\
516.01	0.00597013417218746\\
517.01	0.0059895980842368\\
518.01	0.00600946304943295\\
519.01	0.00602973735398422\\
520.01	0.00605042945139851\\
521.01	0.00607154796570211\\
522.01	0.00609310169468649\\
523.01	0.00611509961317347\\
524.01	0.00613755087629289\\
525.01	0.00616046482276039\\
526.01	0.00618385097814808\\
527.01	0.00620771905813523\\
528.01	0.00623207897172903\\
529.01	0.00625694082444449\\
530.01	0.00628231492143\\
531.01	0.00630821177052611\\
532.01	0.00633464208524313\\
533.01	0.00636161678764127\\
534.01	0.00638914701109523\\
535.01	0.00641724410292135\\
536.01	0.00644591962684281\\
537.01	0.00647518536526556\\
538.01	0.00650505332133199\\
539.01	0.00653553572071758\\
540.01	0.00656664501313042\\
541.01	0.00659839387346958\\
542.01	0.00663079520259363\\
543.01	0.00666386212764396\\
544.01	0.00669760800186221\\
545.01	0.00673204640383408\\
546.01	0.00676719113608257\\
547.01	0.00680305622292752\\
548.01	0.00683965590751582\\
549.01	0.00687700464791808\\
550.01	0.00691511711217481\\
551.01	0.00695400817216172\\
552.01	0.00699369289613081\\
553.01	0.00703418653976667\\
554.01	0.00707550453558119\\
555.01	0.00711766248044977\\
556.01	0.00716067612107247\\
557.01	0.00720456133711941\\
558.01	0.00724933412179551\\
559.01	0.0072950105595322\\
560.01	0.00734160680048446\\
561.01	0.00738913903147977\\
562.01	0.00743762344303022\\
563.01	0.00748707619198446\\
564.01	0.00753751335935612\\
565.01	0.00758895090282426\\
566.01	0.00764140460335994\\
567.01	0.00769489000538857\\
568.01	0.00774942234985408\\
569.01	0.00780501649950793\\
570.01	0.00786168685570438\\
571.01	0.00791944726594716\\
572.01	0.00797831092140276\\
573.01	0.00803829024357548\\
574.01	0.00809939675933612\\
575.01	0.0081616409635105\\
576.01	0.00822503216828076\\
577.01	0.00828957833873377\\
578.01	0.00835528591402531\\
579.01	0.00842215961382854\\
580.01	0.00849020223002198\\
581.01	0.00855941440397067\\
582.01	0.00862979439029825\\
583.01	0.00870133780877493\\
584.01	0.00877403738691154\\
585.01	0.00884788269711461\\
586.01	0.0089228598939045\\
587.01	0.00899895145882848\\
588.01	0.00907613596344161\\
589.01	0.00915438786423933\\
590.01	0.00923367734790629\\
591.01	0.00931397025094757\\
592.01	0.00939522808500194\\
593.01	0.00947740820829374\\
594.01	0.00956046419524734\\
595.01	0.00964434647087689\\
596.01	0.00972900329493185\\
597.01	0.0098143822038764\\
598.01	0.00990043204779049\\
599.01	0.00996919203046377\\
599.02	0.00996973314335038\\
599.03	0.0099702709571694\\
599.04	0.00997080543920336\\
599.05	0.0099713365564138\\
599.06	0.00997186427543808\\
599.07	0.0099723885625862\\
599.08	0.00997290938383756\\
599.09	0.00997342670483771\\
599.1	0.00997394049089505\\
599.11	0.00997445070697751\\
599.12	0.00997495731770918\\
599.13	0.00997546028736696\\
599.14	0.00997595957987707\\
599.15	0.00997645515881165\\
599.16	0.00997694698738526\\
599.17	0.00997743502845131\\
599.18	0.00997791924449856\\
599.19	0.00997839959764748\\
599.2	0.00997887604964662\\
599.21	0.00997934856186899\\
599.22	0.00997981709530829\\
599.23	0.00998028161057521\\
599.24	0.00998074206789365\\
599.25	0.0099811984270969\\
599.26	0.00998165064762378\\
599.27	0.00998209868851477\\
599.28	0.00998254250840806\\
599.29	0.00998298206553559\\
599.3	0.00998341731771905\\
599.31	0.00998384822236584\\
599.32	0.00998427473646497\\
599.33	0.00998469681658292\\
599.34	0.00998511441885952\\
599.35	0.0099855274990037\\
599.36	0.00998593601228926\\
599.37	0.00998633991355056\\
599.38	0.00998673915717821\\
599.39	0.0099871336971147\\
599.4	0.00998752348684992\\
599.41	0.00998790847811157\\
599.42	0.00998828861974491\\
599.43	0.00998866386008804\\
599.44	0.00998903414696681\\
599.45	0.00998939942768985\\
599.46	0.00998975964904344\\
599.47	0.00999011475728636\\
599.48	0.00999046469814472\\
599.49	0.00999080941680669\\
599.5	0.00999114885791725\\
599.51	0.00999148296557278\\
599.52	0.0099918116833157\\
599.53	0.00999213495412901\\
599.54	0.00999245272043077\\
599.55	0.00999276492406855\\
599.56	0.00999307150631381\\
599.57	0.00999337240785624\\
599.58	0.00999366756879799\\
599.59	0.00999395692864795\\
599.6	0.00999424042631586\\
599.61	0.00999451800010642\\
599.62	0.00999478958771336\\
599.63	0.0099950551262134\\
599.64	0.00999531455206019\\
599.65	0.00999556780107816\\
599.66	0.00999581480845634\\
599.67	0.00999605550874211\\
599.68	0.00999628983583487\\
599.69	0.00999651772297967\\
599.7	0.00999673910276076\\
599.71	0.00999695390709509\\
599.72	0.00999716206722577\\
599.73	0.00999736351371538\\
599.74	0.00999755817643933\\
599.75	0.00999774598457904\\
599.76	0.00999792686661517\\
599.77	0.00999810075032067\\
599.78	0.00999826756275386\\
599.79	0.00999842723025133\\
599.8	0.00999857967842092\\
599.81	0.0099987248321345\\
599.82	0.00999886261552071\\
599.83	0.00999899295195769\\
599.84	0.00999911576406567\\
599.85	0.00999923097369951\\
599.86	0.00999933850194117\\
599.87	0.00999943826909211\\
599.88	0.00999953019466558\\
599.89	0.00999961419737891\\
599.9	0.00999969019514566\\
599.91	0.00999975810506767\\
599.92	0.00999981784342713\\
599.93	0.00999986932567848\\
599.94	0.00999991246644028\\
599.95	0.00999994717948697\\
599.96	0.00999997337774056\\
599.97	0.00999999097326228\\
599.98	0.00999999987724406\\
599.99	0.01\\
600	0.01\\
};
\addplot [color=mycolor8,solid,forget plot]
  table[row sep=crcr]{%
0.01	0.00486947107861205\\
1.01	0.00486947292236736\\
2.01	0.00486947480371144\\
3.01	0.00486947672340973\\
4.01	0.00486947868224241\\
5.01	0.00486948068100652\\
6.01	0.00486948272051449\\
7.01	0.00486948480159557\\
8.01	0.00486948692509585\\
9.01	0.00486948909187865\\
10.01	0.00486949130282463\\
11.01	0.00486949355883243\\
12.01	0.00486949586081895\\
13.01	0.0048694982097197\\
14.01	0.00486950060648905\\
15.01	0.0048695030521007\\
16.01	0.00486950554754846\\
17.01	0.00486950809384586\\
18.01	0.00486951069202711\\
19.01	0.00486951334314763\\
20.01	0.0048695160482839\\
21.01	0.00486951880853456\\
22.01	0.00486952162502026\\
23.01	0.00486952449888436\\
24.01	0.00486952743129361\\
25.01	0.00486953042343835\\
26.01	0.00486953347653281\\
27.01	0.0048695365918162\\
28.01	0.00486953977055238\\
29.01	0.00486954301403122\\
30.01	0.0048695463235686\\
31.01	0.00486954970050693\\
32.01	0.00486955314621586\\
33.01	0.00486955666209285\\
34.01	0.00486956024956363\\
35.01	0.00486956391008279\\
36.01	0.00486956764513434\\
37.01	0.00486957145623229\\
38.01	0.00486957534492136\\
39.01	0.00486957931277746\\
40.01	0.00486958336140846\\
41.01	0.00486958749245478\\
42.01	0.00486959170758994\\
43.01	0.00486959600852142\\
44.01	0.00486960039699094\\
45.01	0.00486960487477578\\
46.01	0.00486960944368901\\
47.01	0.00486961410558036\\
48.01	0.00486961886233696\\
49.01	0.00486962371588401\\
50.01	0.00486962866818569\\
51.01	0.00486963372124574\\
52.01	0.00486963887710859\\
53.01	0.00486964413785981\\
54.01	0.00486964950562714\\
55.01	0.0048696549825813\\
56.01	0.00486966057093674\\
57.01	0.00486966627295266\\
58.01	0.00486967209093384\\
59.01	0.00486967802723172\\
60.01	0.004869684084245\\
61.01	0.0048696902644208\\
62.01	0.00486969657025571\\
63.01	0.00486970300429641\\
64.01	0.00486970956914139\\
65.01	0.00486971626744086\\
66.01	0.00486972310189916\\
67.01	0.00486973007527472\\
68.01	0.00486973719038161\\
69.01	0.00486974445009067\\
70.01	0.00486975185733038\\
71.01	0.00486975941508849\\
72.01	0.00486976712641271\\
73.01	0.00486977499441213\\
74.01	0.00486978302225829\\
75.01	0.00486979121318698\\
76.01	0.00486979957049866\\
77.01	0.00486980809756034\\
78.01	0.00486981679780672\\
79.01	0.00486982567474153\\
80.01	0.00486983473193921\\
81.01	0.00486984397304571\\
82.01	0.00486985340178046\\
83.01	0.0048698630219374\\
84.01	0.00486987283738692\\
85.01	0.00486988285207705\\
86.01	0.00486989307003499\\
87.01	0.00486990349536882\\
88.01	0.0048699141322692\\
89.01	0.00486992498501079\\
90.01	0.00486993605795404\\
91.01	0.00486994735554673\\
92.01	0.00486995888232605\\
93.01	0.00486997064291982\\
94.01	0.0048699826420489\\
95.01	0.00486999488452863\\
96.01	0.00487000737527089\\
97.01	0.00487002011928564\\
98.01	0.00487003312168349\\
99.01	0.00487004638767716\\
100.01	0.00487005992258352\\
101.01	0.00487007373182632\\
102.01	0.00487008782093738\\
103.01	0.00487010219555918\\
104.01	0.00487011686144692\\
105.01	0.00487013182447101\\
106.01	0.00487014709061895\\
107.01	0.00487016266599795\\
108.01	0.00487017855683707\\
109.01	0.00487019476948977\\
110.01	0.00487021131043615\\
111.01	0.0048702281862858\\
112.01	0.00487024540377978\\
113.01	0.00487026296979401\\
114.01	0.0048702808913408\\
115.01	0.00487029917557284\\
116.01	0.00487031782978466\\
117.01	0.00487033686141632\\
118.01	0.00487035627805595\\
119.01	0.00487037608744272\\
120.01	0.00487039629746952\\
121.01	0.00487041691618612\\
122.01	0.00487043795180245\\
123.01	0.0048704594126915\\
124.01	0.00487048130739251\\
125.01	0.0048705036446141\\
126.01	0.00487052643323813\\
127.01	0.00487054968232194\\
128.01	0.0048705734011032\\
129.01	0.00487059759900224\\
130.01	0.0048706222856261\\
131.01	0.00487064747077201\\
132.01	0.00487067316443125\\
133.01	0.00487069937679273\\
134.01	0.00487072611824707\\
135.01	0.00487075339939004\\
136.01	0.00487078123102694\\
137.01	0.00487080962417641\\
138.01	0.0048708385900747\\
139.01	0.00487086814017983\\
140.01	0.00487089828617572\\
141.01	0.00487092903997673\\
142.01	0.00487096041373185\\
143.01	0.00487099241982938\\
144.01	0.00487102507090169\\
145.01	0.00487105837982942\\
146.01	0.00487109235974664\\
147.01	0.00487112702404564\\
148.01	0.00487116238638149\\
149.01	0.00487119846067759\\
150.01	0.0048712352611306\\
151.01	0.00487127280221532\\
152.01	0.00487131109869036\\
153.01	0.00487135016560362\\
154.01	0.00487139001829714\\
155.01	0.00487143067241363\\
156.01	0.00487147214390136\\
157.01	0.00487151444902012\\
158.01	0.0048715576043474\\
159.01	0.00487160162678421\\
160.01	0.00487164653356111\\
161.01	0.0048716923422442\\
162.01	0.00487173907074211\\
163.01	0.00487178673731159\\
164.01	0.00487183536056479\\
165.01	0.00487188495947542\\
166.01	0.0048719355533856\\
167.01	0.00487198716201284\\
168.01	0.00487203980545693\\
169.01	0.00487209350420692\\
170.01	0.004872148279149\\
171.01	0.00487220415157283\\
172.01	0.00487226114318008\\
173.01	0.00487231927609102\\
174.01	0.00487237857285301\\
175.01	0.00487243905644807\\
176.01	0.00487250075030086\\
177.01	0.0048725636782868\\
178.01	0.00487262786474045\\
179.01	0.00487269333446377\\
180.01	0.00487276011273462\\
181.01	0.00487282822531551\\
182.01	0.00487289769846232\\
183.01	0.00487296855893347\\
184.01	0.00487304083399875\\
185.01	0.00487311455144837\\
186.01	0.00487318973960289\\
187.01	0.00487326642732225\\
188.01	0.00487334464401561\\
189.01	0.00487342441965137\\
190.01	0.00487350578476671\\
191.01	0.00487358877047799\\
192.01	0.00487367340849093\\
193.01	0.00487375973111112\\
194.01	0.00487384777125442\\
195.01	0.00487393756245795\\
196.01	0.0048740291388905\\
197.01	0.00487412253536421\\
198.01	0.00487421778734529\\
199.01	0.00487431493096541\\
200.01	0.00487441400303351\\
201.01	0.00487451504104718\\
202.01	0.00487461808320468\\
203.01	0.00487472316841675\\
204.01	0.00487483033631851\\
205.01	0.00487493962728282\\
206.01	0.00487505108243155\\
207.01	0.00487516474364866\\
208.01	0.00487528065359315\\
209.01	0.00487539885571175\\
210.01	0.00487551939425231\\
211.01	0.00487564231427645\\
212.01	0.00487576766167358\\
213.01	0.00487589548317438\\
214.01	0.00487602582636409\\
215.01	0.00487615873969677\\
216.01	0.004876294272509\\
217.01	0.00487643247503433\\
218.01	0.00487657339841745\\
219.01	0.00487671709472827\\
220.01	0.00487686361697712\\
221.01	0.00487701301912899\\
222.01	0.00487716535611812\\
223.01	0.00487732068386365\\
224.01	0.00487747905928396\\
225.01	0.00487764054031238\\
226.01	0.00487780518591223\\
227.01	0.00487797305609189\\
228.01	0.00487814421192095\\
229.01	0.004878318715545\\
230.01	0.00487849663020176\\
231.01	0.00487867802023668\\
232.01	0.00487886295111854\\
233.01	0.0048790514894554\\
234.01	0.00487924370301043\\
235.01	0.00487943966071792\\
236.01	0.00487963943269878\\
237.01	0.00487984309027684\\
238.01	0.00488005070599444\\
239.01	0.00488026235362868\\
240.01	0.00488047810820687\\
241.01	0.00488069804602276\\
242.01	0.00488092224465167\\
243.01	0.00488115078296653\\
244.01	0.00488138374115347\\
245.01	0.00488162120072699\\
246.01	0.00488186324454526\\
247.01	0.00488210995682524\\
248.01	0.00488236142315752\\
249.01	0.00488261773052098\\
250.01	0.00488287896729749\\
251.01	0.00488314522328534\\
252.01	0.00488341658971392\\
253.01	0.00488369315925632\\
254.01	0.0048839750260436\\
255.01	0.00488426228567626\\
256.01	0.00488455503523741\\
257.01	0.0048848533733041\\
258.01	0.00488515739995884\\
259.01	0.00488546721679987\\
260.01	0.00488578292695181\\
261.01	0.0048861046350747\\
262.01	0.00488643244737243\\
263.01	0.00488676647160112\\
264.01	0.00488710681707634\\
265.01	0.00488745359467868\\
266.01	0.00488780691685956\\
267.01	0.00488816689764514\\
268.01	0.00488853365263966\\
269.01	0.00488890729902719\\
270.01	0.00488928795557286\\
271.01	0.00488967574262189\\
272.01	0.00489007078209797\\
273.01	0.0048904731974995\\
274.01	0.00489088311389492\\
275.01	0.00489130065791585\\
276.01	0.00489172595774905\\
277.01	0.00489215914312559\\
278.01	0.004892600345309\\
279.01	0.00489304969708074\\
280.01	0.0048935073327239\\
281.01	0.00489397338800394\\
282.01	0.00489444800014771\\
283.01	0.00489493130781972\\
284.01	0.0048954234510946\\
285.01	0.00489592457142875\\
286.01	0.00489643481162716\\
287.01	0.00489695431580782\\
288.01	0.00489748322936308\\
289.01	0.00489802169891686\\
290.01	0.00489856987227907\\
291.01	0.004899127898395\\
292.01	0.00489969592729207\\
293.01	0.00490027411002165\\
294.01	0.00490086259859672\\
295.01	0.00490146154592501\\
296.01	0.0049020711057375\\
297.01	0.00490269143251219\\
298.01	0.00490332268139221\\
299.01	0.00490396500809982\\
300.01	0.00490461856884361\\
301.01	0.00490528352022102\\
302.01	0.00490596001911514\\
303.01	0.00490664822258498\\
304.01	0.00490734828775043\\
305.01	0.00490806037167041\\
306.01	0.00490878463121465\\
307.01	0.00490952122292969\\
308.01	0.0049102703028981\\
309.01	0.00491103202659038\\
310.01	0.00491180654871179\\
311.01	0.00491259402304131\\
312.01	0.00491339460226485\\
313.01	0.00491420843780183\\
314.01	0.0049150356796258\\
315.01	0.00491587647607902\\
316.01	0.00491673097368153\\
317.01	0.00491759931693518\\
318.01	0.00491848164812264\\
319.01	0.00491937810710343\\
320.01	0.00492028883110579\\
321.01	0.00492121395451707\\
322.01	0.00492215360867281\\
323.01	0.00492310792164599\\
324.01	0.0049240770180385\\
325.01	0.00492506101877592\\
326.01	0.00492606004090838\\
327.01	0.00492707419741913\\
328.01	0.0049281035970444\\
329.01	0.00492914834410706\\
330.01	0.00493020853836761\\
331.01	0.00493128427489808\\
332.01	0.00493237564397986\\
333.01	0.00493348273103555\\
334.01	0.00493460561659563\\
335.01	0.00493574437631042\\
336.01	0.00493689908101126\\
337.01	0.00493806979683226\\
338.01	0.00493925658539787\\
339.01	0.00494045950408954\\
340.01	0.00494167860639888\\
341.01	0.00494291394238186\\
342.01	0.00494416555922468\\
343.01	0.00494543350193688\\
344.01	0.00494671781418379\\
345.01	0.00494801853927719\\
346.01	0.00494933572133761\\
347.01	0.00495066940664869\\
348.01	0.0049520196452166\\
349.01	0.0049533864925576\\
350.01	0.00495477001172624\\
351.01	0.00495617027560363\\
352.01	0.00495758736945942\\
353.01	0.0049590213937989\\
354.01	0.00496047246750404\\
355.01	0.00496194073127067\\
356.01	0.00496342635133638\\
357.01	0.00496492952348586\\
358.01	0.00496645047730471\\
359.01	0.00496798948064016\\
360.01	0.00496954684420485\\
361.01	0.00497112292623784\\
362.01	0.00497271813710551\\
363.01	0.00497433294369125\\
364.01	0.00497596787338372\\
365.01	0.00497762351742256\\
366.01	0.00497930053331316\\
367.01	0.00498099964596406\\
368.01	0.00498272164714229\\
369.01	0.00498446739279117\\
370.01	0.00498623779770917\\
371.01	0.0049880338270704\\
372.01	0.00498985648428585\\
373.01	0.00499170679479151\\
374.01	0.00499358578553845\\
375.01	0.0049954944603099\\
376.01	0.00499743377157359\\
377.01	0.00499940459050881\\
378.01	0.00500140767827624\\
379.01	0.00500344366373888\\
380.01	0.00500551304083137\\
381.01	0.00500761623451497\\
382.01	0.00500975366440684\\
383.01	0.00501192575025035\\
384.01	0.00501413291170537\\
385.01	0.00501637556813499\\
386.01	0.00501865413838863\\
387.01	0.00502096904058337\\
388.01	0.00502332069188344\\
389.01	0.0050257095082792\\
390.01	0.00502813590436763\\
391.01	0.00503060029313336\\
392.01	0.00503310308573499\\
393.01	0.00503564469129482\\
394.01	0.00503822551669646\\
395.01	0.00504084596639151\\
396.01	0.00504350644221681\\
397.01	0.00504620734322635\\
398.01	0.00504894906553885\\
399.01	0.0050517320022056\\
400.01	0.00505455654310108\\
401.01	0.00505742307484021\\
402.01	0.00506033198072661\\
403.01	0.00506328364073562\\
404.01	0.00506627843153779\\
405.01	0.00506931672656746\\
406.01	0.00507239889614258\\
407.01	0.00507552530764104\\
408.01	0.00507869632574179\\
409.01	0.00508191231273589\\
410.01	0.00508517362891761\\
411.01	0.00508848063306101\\
412.01	0.00509183368299319\\
413.01	0.00509523313627251\\
414.01	0.00509867935098175\\
415.01	0.0051021726866462\\
416.01	0.00510571350528854\\
417.01	0.00510930217263104\\
418.01	0.00511293905945774\\
419.01	0.00511662454314692\\
420.01	0.00512035900938866\\
421.01	0.00512414285409804\\
422.01	0.00512797648553649\\
423.01	0.00513186032665466\\
424.01	0.00513579481766586\\
425.01	0.00513978041886369\\
426.01	0.00514381761369105\\
427.01	0.00514790691206936\\
428.01	0.00515204885399412\\
429.01	0.0051562440133987\\
430.01	0.00516049300228848\\
431.01	0.0051647964751378\\
432.01	0.0051691551335438\\
433.01	0.0051735697311199\\
434.01	0.00517804107860816\\
435.01	0.00518257004917931\\
436.01	0.00518715758388222\\
437.01	0.00519180469719159\\
438.01	0.00519651248259237\\
439.01	0.005201282118126\\
440.01	0.00520611487180661\\
441.01	0.00521101210680318\\
442.01	0.00521597528626405\\
443.01	0.00522100597764519\\
444.01	0.00522610585638446\\
445.01	0.0052312767087502\\
446.01	0.00523652043367955\\
447.01	0.00524183904340754\\
448.01	0.0052472346626893\\
449.01	0.00525270952641858\\
450.01	0.00525826597545891\\
451.01	0.00526390645053876\\
452.01	0.00526963348410613\\
453.01	0.00527544969011183\\
454.01	0.00528135775179404\\
455.01	0.00528736040766942\\
456.01	0.00529346043611076\\
457.01	0.00529966063910327\\
458.01	0.00530596382602569\\
459.01	0.00531237279858311\\
460.01	0.00531889033831225\\
461.01	0.00532551919835066\\
462.01	0.00533226210133951\\
463.01	0.00533912174532289\\
464.01	0.00534610081914363\\
465.01	0.00535320202787512\\
466.01	0.00536042812681975\\
467.01	0.00536778195505186\\
468.01	0.00537526645014033\\
469.01	0.0053828846510761\\
470.01	0.00539063970006425\\
471.01	0.00539853484395138\\
472.01	0.00540657343524228\\
473.01	0.00541475893266269\\
474.01	0.0054230949012279\\
475.01	0.00543158501178372\\
476.01	0.00544023303999507\\
477.01	0.00544904286476906\\
478.01	0.00545801846611518\\
479.01	0.00546716392246428\\
480.01	0.00547648340748907\\
481.01	0.00548598118649742\\
482.01	0.00549566161249708\\
483.01	0.00550552912206398\\
484.01	0.00551558823117855\\
485.01	0.00552584353122865\\
486.01	0.00553629968540818\\
487.01	0.00554696142576086\\
488.01	0.00555783355113311\\
489.01	0.00556892092629246\\
490.01	0.0055802284824364\\
491.01	0.00559176121925617\\
492.01	0.00560352420861528\\
493.01	0.0056155225997567\\
494.01	0.00562776162575973\\
495.01	0.00564024661073782\\
496.01	0.00565298297702609\\
497.01	0.00566597625140927\\
498.01	0.00567923206939053\\
499.01	0.00569275617702374\\
500.01	0.00570655443140132\\
501.01	0.00572063280103759\\
502.01	0.00573499736640269\\
503.01	0.00574965432066073\\
504.01	0.00576460997066185\\
505.01	0.00577987073823816\\
506.01	0.00579544316184305\\
507.01	0.00581133389856794\\
508.01	0.00582754972655403\\
509.01	0.00584409754780098\\
510.01	0.00586098439135369\\
511.01	0.00587821741682283\\
512.01	0.00589580391817397\\
513.01	0.00591375132769374\\
514.01	0.0059320672200273\\
515.01	0.00595075931617198\\
516.01	0.00596983548732184\\
517.01	0.00598930375848478\\
518.01	0.00600917231184658\\
519.01	0.00602944948993105\\
520.01	0.00605014379868112\\
521.01	0.00607126391059673\\
522.01	0.00609281866798034\\
523.01	0.00611481708628279\\
524.01	0.00613726835753446\\
525.01	0.00616018185383922\\
526.01	0.00618356713090683\\
527.01	0.00620743393159559\\
528.01	0.00623179218943586\\
529.01	0.00625665203210274\\
530.01	0.00628202378480992\\
531.01	0.0063079179735996\\
532.01	0.00633434532850766\\
533.01	0.00636131678658915\\
534.01	0.00638884349479479\\
535.01	0.00641693681269148\\
536.01	0.00644560831501638\\
537.01	0.00647486979404623\\
538.01	0.0065047332617518\\
539.01	0.00653521095169769\\
540.01	0.00656631532064435\\
541.01	0.00659805904980486\\
542.01	0.00663045504570371\\
543.01	0.00666351644058184\\
544.01	0.00669725659228586\\
545.01	0.00673168908357341\\
546.01	0.00676682772076062\\
547.01	0.00680268653162876\\
548.01	0.00683927976249817\\
549.01	0.00687662187436601\\
550.01	0.00691472753799263\\
551.01	0.0069536116278066\\
552.01	0.00699328921448385\\
553.01	0.00703377555604178\\
554.01	0.00707508608726937\\
555.01	0.00711723640729895\\
556.01	0.00716024226510179\\
557.01	0.00720411954266781\\
558.01	0.00724888423560591\\
559.01	0.00729455243087272\\
560.01	0.00734114028130855\\
561.01	0.00738866397662728\\
562.01	0.00743713971047418\\
563.01	0.00748658364312575\\
564.01	0.00753701185936998\\
565.01	0.00758844032106287\\
566.01	0.00764088481381529\\
567.01	0.00769436088721993\\
568.01	0.00774888378798493\\
569.01	0.00780446838529657\\
570.01	0.00786112908769292\\
571.01	0.00791887975069271\\
572.01	0.00797773357439382\\
573.01	0.00803770299023592\\
574.01	0.00809879953611645\\
575.01	0.00816103371906509\\
576.01	0.00822441486472502\\
577.01	0.00828895095297273\\
578.01	0.00835464843913835\\
579.01	0.00842151206048843\\
580.01	0.00848954462791702\\
581.01	0.00855874680318726\\
582.01	0.0086291168626057\\
583.01	0.00870065044873541\\
584.01	0.00877334031271359\\
585.01	0.00884717605099796\\
586.01	0.00892214384200484\\
587.01	0.0089982261902219\\
588.01	0.00907540168810686\\
589.01	0.00915364480957812\\
590.01	0.00923292575336488\\
591.01	0.0093132103601611\\
592.01	0.00939446013472989\\
593.01	0.00947663241322402\\
594.01	0.00955968072750807\\
595.01	0.00964355543279665\\
596.01	0.00972820468321698\\
597.01	0.00981357586291021\\
598.01	0.00989961760918075\\
599.01	0.00996919203046377\\
599.02	0.00996973314335038\\
599.03	0.0099702709571694\\
599.04	0.00997080543920336\\
599.05	0.0099713365564138\\
599.06	0.00997186427543808\\
599.07	0.0099723885625862\\
599.08	0.00997290938383756\\
599.09	0.00997342670483771\\
599.1	0.00997394049089505\\
599.11	0.00997445070697751\\
599.12	0.00997495731770918\\
599.13	0.00997546028736696\\
599.14	0.00997595957987707\\
599.15	0.00997645515881166\\
599.16	0.00997694698738526\\
599.17	0.00997743502845131\\
599.18	0.00997791924449856\\
599.19	0.00997839959764748\\
599.2	0.00997887604964662\\
599.21	0.00997934856186899\\
599.22	0.00997981709530829\\
599.23	0.00998028161057521\\
599.24	0.00998074206789365\\
599.25	0.0099811984270969\\
599.26	0.00998165064762378\\
599.27	0.00998209868851477\\
599.28	0.00998254250840806\\
599.29	0.00998298206553559\\
599.3	0.00998341731771905\\
599.31	0.00998384822236584\\
599.32	0.00998427473646497\\
599.33	0.00998469681658292\\
599.34	0.00998511441885952\\
599.35	0.0099855274990037\\
599.36	0.00998593601228926\\
599.37	0.00998633991355056\\
599.38	0.00998673915717821\\
599.39	0.00998713369711469\\
599.4	0.00998752348684992\\
599.41	0.00998790847811157\\
599.42	0.00998828861974491\\
599.43	0.00998866386008804\\
599.44	0.00998903414696681\\
599.45	0.00998939942768985\\
599.46	0.00998975964904344\\
599.47	0.00999011475728636\\
599.48	0.00999046469814472\\
599.49	0.00999080941680669\\
599.5	0.00999114885791725\\
599.51	0.00999148296557278\\
599.52	0.0099918116833157\\
599.53	0.00999213495412901\\
599.54	0.00999245272043077\\
599.55	0.00999276492406855\\
599.56	0.00999307150631381\\
599.57	0.00999337240785624\\
599.58	0.00999366756879799\\
599.59	0.00999395692864795\\
599.6	0.00999424042631585\\
599.61	0.00999451800010642\\
599.62	0.00999478958771336\\
599.63	0.0099950551262134\\
599.64	0.00999531455206019\\
599.65	0.00999556780107816\\
599.66	0.00999581480845634\\
599.67	0.00999605550874211\\
599.68	0.00999628983583487\\
599.69	0.00999651772297967\\
599.7	0.00999673910276076\\
599.71	0.00999695390709509\\
599.72	0.00999716206722577\\
599.73	0.00999736351371538\\
599.74	0.00999755817643933\\
599.75	0.00999774598457904\\
599.76	0.00999792686661517\\
599.77	0.00999810075032068\\
599.78	0.00999826756275386\\
599.79	0.00999842723025133\\
599.8	0.00999857967842092\\
599.81	0.0099987248321345\\
599.82	0.00999886261552071\\
599.83	0.00999899295195769\\
599.84	0.00999911576406567\\
599.85	0.00999923097369951\\
599.86	0.00999933850194117\\
599.87	0.00999943826909211\\
599.88	0.00999953019466558\\
599.89	0.00999961419737891\\
599.9	0.00999969019514566\\
599.91	0.00999975810506767\\
599.92	0.00999981784342713\\
599.93	0.00999986932567848\\
599.94	0.00999991246644028\\
599.95	0.00999994717948697\\
599.96	0.00999997337774056\\
599.97	0.00999999097326228\\
599.98	0.00999999987724406\\
599.99	0.01\\
600	0.01\\
};
\addplot [color=blue!25!mycolor7,solid,forget plot]
  table[row sep=crcr]{%
0.01	0.0046877913708233\\
1.01	0.00468779351620759\\
2.01	0.00468779570568027\\
3.01	0.00468779794014776\\
4.01	0.00468780022053526\\
5.01	0.00468780254778671\\
6.01	0.00468780492286568\\
7.01	0.00468780734675557\\
8.01	0.00468780982045992\\
9.01	0.00468781234500285\\
10.01	0.00468781492142956\\
11.01	0.00468781755080697\\
12.01	0.00468782023422366\\
13.01	0.00468782297279062\\
14.01	0.00468782576764179\\
15.01	0.0046878286199345\\
16.01	0.0046878315308496\\
17.01	0.00468783450159244\\
18.01	0.00468783753339311\\
19.01	0.00468784062750699\\
20.01	0.00468784378521535\\
21.01	0.00468784700782547\\
22.01	0.00468785029667193\\
23.01	0.00468785365311665\\
24.01	0.00468785707854939\\
25.01	0.00468786057438855\\
26.01	0.00468786414208172\\
27.01	0.00468786778310613\\
28.01	0.0046878714989697\\
29.01	0.00468787529121094\\
30.01	0.0046878791614004\\
31.01	0.0046878831111405\\
32.01	0.00468788714206692\\
33.01	0.0046878912558488\\
34.01	0.00468789545418967\\
35.01	0.00468789973882809\\
36.01	0.0046879041115383\\
37.01	0.00468790857413097\\
38.01	0.004687913128454\\
39.01	0.00468791777639346\\
40.01	0.00468792251987409\\
41.01	0.00468792736085995\\
42.01	0.00468793230135572\\
43.01	0.00468793734340721\\
44.01	0.00468794248910237\\
45.01	0.00468794774057203\\
46.01	0.00468795309999054\\
47.01	0.00468795856957747\\
48.01	0.00468796415159733\\
49.01	0.0046879698483618\\
50.01	0.00468797566222959\\
51.01	0.00468798159560826\\
52.01	0.00468798765095454\\
53.01	0.00468799383077587\\
54.01	0.00468800013763086\\
55.01	0.00468800657413113\\
56.01	0.00468801314294176\\
57.01	0.00468801984678247\\
58.01	0.00468802668842902\\
59.01	0.00468803367071403\\
60.01	0.00468804079652857\\
61.01	0.00468804806882284\\
62.01	0.0046880554906078\\
63.01	0.00468806306495617\\
64.01	0.00468807079500363\\
65.01	0.00468807868395074\\
66.01	0.00468808673506337\\
67.01	0.00468809495167452\\
68.01	0.00468810333718582\\
69.01	0.00468811189506857\\
70.01	0.00468812062886539\\
71.01	0.00468812954219165\\
72.01	0.00468813863873707\\
73.01	0.00468814792226696\\
74.01	0.00468815739662418\\
75.01	0.00468816706573004\\
76.01	0.00468817693358677\\
77.01	0.00468818700427871\\
78.01	0.0046881972819738\\
79.01	0.00468820777092574\\
80.01	0.00468821847547531\\
81.01	0.00468822940005259\\
82.01	0.00468824054917864\\
83.01	0.00468825192746695\\
84.01	0.00468826353962619\\
85.01	0.00468827539046124\\
86.01	0.00468828748487558\\
87.01	0.00468829982787353\\
88.01	0.00468831242456196\\
89.01	0.00468832528015235\\
90.01	0.00468833839996327\\
91.01	0.00468835178942247\\
92.01	0.00468836545406862\\
93.01	0.00468837939955452\\
94.01	0.00468839363164836\\
95.01	0.00468840815623696\\
96.01	0.00468842297932769\\
97.01	0.00468843810705127\\
98.01	0.0046884535456638\\
99.01	0.0046884693015498\\
100.01	0.00468848538122497\\
101.01	0.0046885017913378\\
102.01	0.00468851853867376\\
103.01	0.00468853563015703\\
104.01	0.00468855307285371\\
105.01	0.00468857087397483\\
106.01	0.00468858904087895\\
107.01	0.00468860758107535\\
108.01	0.00468862650222744\\
109.01	0.00468864581215518\\
110.01	0.00468866551883915\\
111.01	0.00468868563042284\\
112.01	0.00468870615521692\\
113.01	0.00468872710170185\\
114.01	0.00468874847853218\\
115.01	0.00468877029453905\\
116.01	0.00468879255873471\\
117.01	0.0046888152803158\\
118.01	0.00468883846866718\\
119.01	0.00468886213336556\\
120.01	0.00468888628418372\\
121.01	0.0046889109310944\\
122.01	0.00468893608427406\\
123.01	0.00468896175410764\\
124.01	0.00468898795119226\\
125.01	0.00468901468634156\\
126.01	0.00468904197059049\\
127.01	0.00468906981519962\\
128.01	0.00468909823165932\\
129.01	0.00468912723169509\\
130.01	0.00468915682727183\\
131.01	0.00468918703059898\\
132.01	0.00468921785413555\\
133.01	0.00468924931059461\\
134.01	0.00468928141294914\\
135.01	0.00468931417443692\\
136.01	0.00468934760856594\\
137.01	0.00468938172912008\\
138.01	0.00468941655016439\\
139.01	0.00468945208605075\\
140.01	0.00468948835142417\\
141.01	0.00468952536122824\\
142.01	0.00468956313071139\\
143.01	0.00468960167543311\\
144.01	0.00468964101126965\\
145.01	0.00468968115442142\\
146.01	0.00468972212141856\\
147.01	0.00468976392912843\\
148.01	0.00468980659476177\\
149.01	0.00468985013587997\\
150.01	0.00468989457040206\\
151.01	0.004689939916612\\
152.01	0.00468998619316581\\
153.01	0.0046900334190991\\
154.01	0.00469008161383513\\
155.01	0.00469013079719205\\
156.01	0.00469018098939114\\
157.01	0.00469023221106503\\
158.01	0.00469028448326561\\
159.01	0.00469033782747277\\
160.01	0.00469039226560272\\
161.01	0.0046904478200171\\
162.01	0.00469050451353166\\
163.01	0.00469056236942537\\
164.01	0.00469062141144987\\
165.01	0.00469068166383869\\
166.01	0.00469074315131712\\
167.01	0.00469080589911186\\
168.01	0.00469086993296136\\
169.01	0.00469093527912584\\
170.01	0.00469100196439738\\
171.01	0.00469107001611126\\
172.01	0.00469113946215627\\
173.01	0.00469121033098589\\
174.01	0.00469128265162952\\
175.01	0.00469135645370398\\
176.01	0.00469143176742538\\
177.01	0.00469150862362071\\
178.01	0.00469158705374005\\
179.01	0.00469166708986911\\
180.01	0.0046917487647417\\
181.01	0.00469183211175282\\
182.01	0.00469191716497146\\
183.01	0.00469200395915406\\
184.01	0.0046920925297582\\
185.01	0.00469218291295631\\
186.01	0.00469227514564995\\
187.01	0.00469236926548416\\
188.01	0.00469246531086198\\
189.01	0.00469256332095967\\
190.01	0.0046926633357415\\
191.01	0.00469276539597568\\
192.01	0.00469286954325\\
193.01	0.00469297581998768\\
194.01	0.00469308426946434\\
195.01	0.00469319493582397\\
196.01	0.00469330786409664\\
197.01	0.00469342310021517\\
198.01	0.00469354069103299\\
199.01	0.00469366068434245\\
200.01	0.00469378312889261\\
201.01	0.00469390807440817\\
202.01	0.00469403557160822\\
203.01	0.00469416567222568\\
204.01	0.00469429842902691\\
205.01	0.00469443389583178\\
206.01	0.00469457212753394\\
207.01	0.00469471318012162\\
208.01	0.00469485711069827\\
209.01	0.00469500397750496\\
210.01	0.00469515383994136\\
211.01	0.00469530675858849\\
212.01	0.00469546279523128\\
213.01	0.00469562201288101\\
214.01	0.00469578447579952\\
215.01	0.00469595024952263\\
216.01	0.00469611940088441\\
217.01	0.00469629199804179\\
218.01	0.00469646811049951\\
219.01	0.00469664780913606\\
220.01	0.00469683116622897\\
221.01	0.00469701825548141\\
222.01	0.0046972091520493\\
223.01	0.00469740393256773\\
224.01	0.00469760267517942\\
225.01	0.00469780545956214\\
226.01	0.00469801236695772\\
227.01	0.00469822348020087\\
228.01	0.00469843888374873\\
229.01	0.00469865866371036\\
230.01	0.00469888290787765\\
231.01	0.00469911170575583\\
232.01	0.00469934514859474\\
233.01	0.00469958332942075\\
234.01	0.00469982634306883\\
235.01	0.00470007428621501\\
236.01	0.00470032725741021\\
237.01	0.004700585357113\\
238.01	0.00470084868772419\\
239.01	0.00470111735362118\\
240.01	0.00470139146119281\\
241.01	0.00470167111887497\\
242.01	0.00470195643718639\\
243.01	0.00470224752876482\\
244.01	0.00470254450840389\\
245.01	0.00470284749309043\\
246.01	0.00470315660204168\\
247.01	0.00470347195674359\\
248.01	0.00470379368098893\\
249.01	0.00470412190091624\\
250.01	0.00470445674504887\\
251.01	0.00470479834433482\\
252.01	0.00470514683218584\\
253.01	0.00470550234451828\\
254.01	0.00470586501979306\\
255.01	0.00470623499905659\\
256.01	0.00470661242598161\\
257.01	0.00470699744690858\\
258.01	0.00470739021088667\\
259.01	0.00470779086971571\\
260.01	0.00470819957798718\\
261.01	0.00470861649312641\\
262.01	0.00470904177543439\\
263.01	0.00470947558812912\\
264.01	0.00470991809738729\\
265.01	0.00471036947238596\\
266.01	0.0047108298853437\\
267.01	0.00471129951156185\\
268.01	0.0047117785294653\\
269.01	0.00471226712064268\\
270.01	0.00471276546988608\\
271.01	0.00471327376523074\\
272.01	0.00471379219799338\\
273.01	0.00471432096281013\\
274.01	0.00471486025767372\\
275.01	0.00471541028396903\\
276.01	0.00471597124650817\\
277.01	0.00471654335356393\\
278.01	0.00471712681690171\\
279.01	0.00471772185181048\\
280.01	0.004718328677131\\
281.01	0.00471894751528298\\
282.01	0.00471957859228921\\
283.01	0.00472022213779801\\
284.01	0.00472087838510299\\
285.01	0.00472154757115958\\
286.01	0.00472222993659834\\
287.01	0.00472292572573532\\
288.01	0.00472363518657853\\
289.01	0.00472435857082978\\
290.01	0.00472509613388258\\
291.01	0.00472584813481511\\
292.01	0.00472661483637805\\
293.01	0.00472739650497579\\
294.01	0.00472819341064325\\
295.01	0.00472900582701362\\
296.01	0.00472983403128104\\
297.01	0.00473067830415359\\
298.01	0.00473153892979923\\
299.01	0.00473241619578142\\
300.01	0.00473331039298522\\
301.01	0.00473422181553319\\
302.01	0.00473515076068848\\
303.01	0.00473609752874667\\
304.01	0.00473706242291323\\
305.01	0.00473804574916711\\
306.01	0.00473904781610854\\
307.01	0.00474006893478995\\
308.01	0.00474110941852864\\
309.01	0.00474216958270086\\
310.01	0.00474324974451443\\
311.01	0.00474435022275863\\
312.01	0.0047454713375312\\
313.01	0.00474661340993866\\
314.01	0.00474777676176948\\
315.01	0.00474896171513786\\
316.01	0.00475016859209591\\
317.01	0.00475139771421119\\
318.01	0.00475264940211023\\
319.01	0.00475392397498133\\
320.01	0.00475522175003794\\
321.01	0.00475654304193771\\
322.01	0.0047578881621554\\
323.01	0.0047592574183065\\
324.01	0.00476065111341886\\
325.01	0.00476206954514856\\
326.01	0.0047635130049386\\
327.01	0.00476498177711507\\
328.01	0.00476647613791956\\
329.01	0.00476799635447413\\
330.01	0.00476954268367623\\
331.01	0.00477111537101982\\
332.01	0.00477271464934223\\
333.01	0.0047743407374924\\
334.01	0.00477599383892136\\
335.01	0.0047776741401921\\
336.01	0.00477938180941025\\
337.01	0.00478111699457616\\
338.01	0.00478287982186117\\
339.01	0.00478467039381171\\
340.01	0.00478648878748918\\
341.01	0.00478833505255249\\
342.01	0.00479020920929768\\
343.01	0.00479211124666848\\
344.01	0.00479404112026045\\
345.01	0.00479599875034337\\
346.01	0.00479798401993539\\
347.01	0.00479999677296886\\
348.01	0.00480203681259828\\
349.01	0.0048041038997108\\
350.01	0.00480619775171291\\
351.01	0.00480831804168072\\
352.01	0.00481046439798116\\
353.01	0.00481263640448791\\
354.01	0.00481483360154139\\
355.01	0.00481705548782523\\
356.01	0.00481930152336384\\
357.01	0.00482157113387317\\
358.01	0.00482386371673542\\
359.01	0.00482617864890168\\
360.01	0.00482851529706731\\
361.01	0.0048308730304962\\
362.01	0.00483325123690904\\
363.01	0.0048356493418704\\
364.01	0.00483806683212031\\
365.01	0.00484050328328288\\
366.01	0.00484295839233395\\
367.01	0.00484543201510167\\
368.01	0.00484792420889242\\
369.01	0.00485043528003107\\
370.01	0.00485296583564418\\
371.01	0.00485551683832332\\
372.01	0.00485808966130107\\
373.01	0.00486068614033032\\
374.01	0.00486330861641787\\
375.01	0.00486595996071022\\
376.01	0.00486864356886456\\
377.01	0.0048713633067735\\
378.01	0.00487412338201897\\
379.01	0.00487692810520714\\
380.01	0.0048797810365406\\
381.01	0.00488268339855548\\
382.01	0.00488563585314032\\
383.01	0.00488863906102592\\
384.01	0.00489169368111608\\
385.01	0.00489480036977825\\
386.01	0.00489795978009008\\
387.01	0.00490117256104255\\
388.01	0.00490443935669549\\
389.01	0.00490776080528592\\
390.01	0.00491113753828591\\
391.01	0.00491457017940904\\
392.01	0.00491805934356283\\
393.01	0.00492160563574746\\
394.01	0.00492520964989636\\
395.01	0.00492887196766008\\
396.01	0.00493259315713027\\
397.01	0.00493637377150284\\
398.01	0.00494021434768074\\
399.01	0.00494411540481332\\
400.01	0.00494807744277466\\
401.01	0.00495210094057746\\
402.01	0.00495618635472721\\
403.01	0.00496033411751279\\
404.01	0.00496454463523903\\
405.01	0.0049688182864005\\
406.01	0.00497315541980127\\
407.01	0.0049775563526234\\
408.01	0.00498202136844939\\
409.01	0.00498655071524534\\
410.01	0.00499114460331061\\
411.01	0.00499580320320435\\
412.01	0.00500052664365926\\
413.01	0.0050053150094949\\
414.01	0.00501016833954575\\
415.01	0.00501508662462151\\
416.01	0.00502006980552037\\
417.01	0.00502511777111758\\
418.01	0.00503023035655757\\
419.01	0.00503540734158045\\
420.01	0.00504064844901714\\
421.01	0.00504595334349552\\
422.01	0.00505132163040094\\
423.01	0.00505675285514366\\
424.01	0.00506224650279165\\
425.01	0.00506780199813149\\
426.01	0.00507341870623117\\
427.01	0.00507909593358397\\
428.01	0.00508483292992171\\
429.01	0.0050906288907933\\
430.01	0.00509648296101652\\
431.01	0.00510239423911465\\
432.01	0.00510836178286296\\
433.01	0.0051143846160757\\
434.01	0.00512046173677136\\
435.01	0.00512659212686058\\
436.01	0.00513277476350303\\
437.01	0.00513900863227999\\
438.01	0.00514529274232582\\
439.01	0.00515162614354973\\
440.01	0.00515800794606375\\
441.01	0.00516443734190444\\
442.01	0.00517091362909716\\
443.01	0.00517743623805843\\
444.01	0.00518400476026008\\
445.01	0.00519061897898486\\
446.01	0.00519727890188222\\
447.01	0.00520398479488479\\
448.01	0.00521073721685766\\
449.01	0.00521753705412668\\
450.01	0.00522438555376617\\
451.01	0.00523128435420685\\
452.01	0.00523823551137073\\
453.01	0.00524524151814018\\
454.01	0.00525230531454857\\
455.01	0.00525943028566056\\
456.01	0.00526662024373692\\
457.01	0.00527387939102032\\
458.01	0.00528121225944443\\
459.01	0.00528862362391745\\
460.01	0.00529611838680955\\
461.01	0.00530370143321333\\
462.01	0.00531137745996266\\
463.01	0.0053191507869801\\
464.01	0.00532702516825855\\
465.01	0.0053350036328947\\
466.01	0.00534308841881588\\
467.01	0.00535128129746901\\
468.01	0.00535958409596056\\
469.01	0.00536799879274197\\
470.01	0.00537652753053824\\
471.01	0.00538517262929297\\
472.01	0.00539393659903768\\
473.01	0.00540282215246299\\
474.01	0.00541183221693191\\
475.01	0.00542096994562932\\
476.01	0.00543023872749784\\
477.01	0.00543964219556525\\
478.01	0.00544918423322939\\
479.01	0.00545886897804068\\
480.01	0.00546870082249888\\
481.01	0.00547868441138057\\
482.01	0.00548882463513328\\
483.01	0.00549912661892601\\
484.01	0.00550959570704213\\
485.01	0.00552023744245141\\
486.01	0.00553105754161531\\
487.01	0.00554206186487474\\
488.01	0.00555325638315463\\
489.01	0.00556464714220292\\
490.01	0.00557624022615136\\
491.01	0.00558804172283474\\
492.01	0.00560005769397902\\
493.01	0.00561229415398543\\
494.01	0.00562475706143512\\
495.01	0.00563745232737498\\
496.01	0.00565038584350466\\
497.01	0.00566356353096031\\
498.01	0.00567699140473343\\
499.01	0.00569067562622107\\
500.01	0.00570462251382754\\
501.01	0.0057188385381076\\
502.01	0.00573333031557464\\
503.01	0.00574810460175816\\
504.01	0.0057631682837062\\
505.01	0.00577852837218618\\
506.01	0.00579419199390473\\
507.01	0.00581016638412089\\
508.01	0.00582645888008326\\
509.01	0.00584307691575363\\
510.01	0.00586002801828694\\
511.01	0.00587731980670701\\
512.01	0.00589495999313232\\
513.01	0.00591295638675328\\
514.01	0.00593131690053231\\
515.01	0.00595004956028283\\
516.01	0.00596916251539123\\
517.01	0.0059886640500192\\
518.01	0.00600856259324166\\
519.01	0.00602886672640598\\
520.01	0.00604958518667911\\
521.01	0.00607072686832795\\
522.01	0.00609230082395634\\
523.01	0.00611431626621421\\
524.01	0.00613678257003705\\
525.01	0.00615970927544816\\
526.01	0.00618310609092311\\
527.01	0.00620698289728304\\
528.01	0.00623134975204244\\
529.01	0.00625621689409586\\
530.01	0.00628159474858923\\
531.01	0.00630749393178962\\
532.01	0.00633392525575795\\
533.01	0.00636089973264133\\
534.01	0.00638842857845291\\
535.01	0.00641652321629788\\
536.01	0.00644519527912794\\
537.01	0.00647445661221022\\
538.01	0.00650431927544885\\
539.01	0.00653479554555353\\
540.01	0.0065658979179944\\
541.01	0.00659763910867128\\
542.01	0.00663003205521878\\
543.01	0.00666308991785967\\
544.01	0.00669682607971868\\
545.01	0.00673125414650544\\
546.01	0.00676638794547803\\
547.01	0.00680224152360036\\
548.01	0.00683882914480995\\
549.01	0.00687616528631111\\
550.01	0.00691426463379902\\
551.01	0.00695314207550284\\
552.01	0.00699281269490666\\
553.01	0.00703329176198434\\
554.01	0.00707459472276593\\
555.01	0.00711673718703461\\
556.01	0.00715973491393524\\
557.01	0.00720360379525376\\
558.01	0.00724835983610392\\
559.01	0.00729401913273179\\
560.01	0.00734059784712131\\
561.01	0.00738811217805081\\
562.01	0.00743657832821639\\
563.01	0.00748601246700066\\
564.01	0.0075364306884236\\
565.01	0.00758784896377284\\
566.01	0.00764028308836689\\
567.01	0.00769374862186187\\
568.01	0.00774826082146843\\
569.01	0.00780383456740263\\
570.01	0.00786048427985252\\
571.01	0.00791822382670581\\
572.01	0.00797706642125281\\
573.01	0.00803702450905831\\
574.01	0.00809810964319064\\
575.01	0.00816033234701034\\
576.01	0.00822370196376449\\
577.01	0.0082882264923125\\
578.01	0.00835391240844111\\
579.01	0.00842076447142201\\
580.01	0.00848878551574813\\
581.01	0.008557976228378\\
582.01	0.00862833491235338\\
583.01	0.0086998572383756\\
584.01	0.00877253598687894\\
585.01	0.00884636078439139\\
586.01	0.00892131783960339\\
587.01	0.00899738968667349\\
588.01	0.00907455494601469\\
589.01	0.00915278811628361\\
590.01	0.0092320594157346\\
591.01	0.00931233469675008\\
592.01	0.00939357546453003\\
593.01	0.00947573903999945\\
594.01	0.00955877891846273\\
595.01	0.00964264538999877\\
596.01	0.00972728650580666\\
597.01	0.00981264949761991\\
598.01	0.00989868278608211\\
599.01	0.00996919203046377\\
599.02	0.00996973314335038\\
599.03	0.0099702709571694\\
599.04	0.00997080543920336\\
599.05	0.0099713365564138\\
599.06	0.00997186427543808\\
599.07	0.0099723885625862\\
599.08	0.00997290938383756\\
599.09	0.00997342670483771\\
599.1	0.00997394049089505\\
599.11	0.00997445070697751\\
599.12	0.00997495731770918\\
599.13	0.00997546028736696\\
599.14	0.00997595957987707\\
599.15	0.00997645515881165\\
599.16	0.00997694698738526\\
599.17	0.00997743502845131\\
599.18	0.00997791924449856\\
599.19	0.00997839959764748\\
599.2	0.00997887604964662\\
599.21	0.00997934856186899\\
599.22	0.00997981709530829\\
599.23	0.00998028161057521\\
599.24	0.00998074206789365\\
599.25	0.0099811984270969\\
599.26	0.00998165064762378\\
599.27	0.00998209868851477\\
599.28	0.00998254250840806\\
599.29	0.00998298206553559\\
599.3	0.00998341731771905\\
599.31	0.00998384822236584\\
599.32	0.00998427473646497\\
599.33	0.00998469681658292\\
599.34	0.00998511441885952\\
599.35	0.0099855274990037\\
599.36	0.00998593601228926\\
599.37	0.00998633991355056\\
599.38	0.00998673915717821\\
599.39	0.00998713369711469\\
599.4	0.00998752348684992\\
599.41	0.00998790847811157\\
599.42	0.00998828861974491\\
599.43	0.00998866386008803\\
599.44	0.00998903414696681\\
599.45	0.00998939942768986\\
599.46	0.00998975964904344\\
599.47	0.00999011475728636\\
599.48	0.00999046469814472\\
599.49	0.00999080941680669\\
599.5	0.00999114885791725\\
599.51	0.00999148296557278\\
599.52	0.0099918116833157\\
599.53	0.00999213495412901\\
599.54	0.00999245272043077\\
599.55	0.00999276492406855\\
599.56	0.00999307150631382\\
599.57	0.00999337240785624\\
599.58	0.00999366756879799\\
599.59	0.00999395692864795\\
599.6	0.00999424042631586\\
599.61	0.00999451800010642\\
599.62	0.00999478958771336\\
599.63	0.0099950551262134\\
599.64	0.00999531455206019\\
599.65	0.00999556780107816\\
599.66	0.00999581480845634\\
599.67	0.00999605550874211\\
599.68	0.00999628983583487\\
599.69	0.00999651772297967\\
599.7	0.00999673910276075\\
599.71	0.00999695390709509\\
599.72	0.00999716206722577\\
599.73	0.00999736351371538\\
599.74	0.00999755817643933\\
599.75	0.00999774598457904\\
599.76	0.00999792686661517\\
599.77	0.00999810075032068\\
599.78	0.00999826756275386\\
599.79	0.00999842723025133\\
599.8	0.00999857967842092\\
599.81	0.0099987248321345\\
599.82	0.00999886261552071\\
599.83	0.00999899295195769\\
599.84	0.00999911576406567\\
599.85	0.00999923097369951\\
599.86	0.00999933850194117\\
599.87	0.00999943826909211\\
599.88	0.00999953019466558\\
599.89	0.00999961419737891\\
599.9	0.00999969019514566\\
599.91	0.00999975810506767\\
599.92	0.00999981784342713\\
599.93	0.00999986932567848\\
599.94	0.00999991246644028\\
599.95	0.00999994717948697\\
599.96	0.00999997337774056\\
599.97	0.00999999097326228\\
599.98	0.00999999987724406\\
599.99	0.01\\
600	0.01\\
};
\addplot [color=mycolor9,solid,forget plot]
  table[row sep=crcr]{%
0.01	0.00433017084612224\\
1.01	0.00433017312500381\\
2.01	0.00433017545095497\\
3.01	0.00433017782494978\\
4.01	0.00433018024798222\\
5.01	0.00433018272106704\\
6.01	0.00433018524523977\\
7.01	0.00433018782155786\\
8.01	0.00433019045110017\\
9.01	0.00433019313496824\\
10.01	0.00433019587428649\\
11.01	0.00433019867020241\\
12.01	0.00433020152388747\\
13.01	0.00433020443653742\\
14.01	0.00433020740937283\\
15.01	0.00433021044363945\\
16.01	0.00433021354060889\\
17.01	0.00433021670157932\\
18.01	0.00433021992787565\\
19.01	0.00433022322085035\\
20.01	0.00433022658188384\\
21.01	0.0043302300123854\\
22.01	0.00433023351379334\\
23.01	0.00433023708757577\\
24.01	0.00433024073523149\\
25.01	0.00433024445829021\\
26.01	0.00433024825831346\\
27.01	0.00433025213689534\\
28.01	0.00433025609566255\\
29.01	0.00433026013627608\\
30.01	0.0043302642604309\\
31.01	0.00433026846985764\\
32.01	0.0043302727663225\\
33.01	0.00433027715162843\\
34.01	0.00433028162761594\\
35.01	0.00433028619616342\\
36.01	0.00433029085918867\\
37.01	0.00433029561864896\\
38.01	0.00433030047654247\\
39.01	0.00433030543490852\\
40.01	0.0043303104958291\\
41.01	0.00433031566142937\\
42.01	0.00433032093387836\\
43.01	0.00433032631539047\\
44.01	0.00433033180822573\\
45.01	0.00433033741469135\\
46.01	0.0043303431371427\\
47.01	0.00433034897798339\\
48.01	0.00433035493966745\\
49.01	0.00433036102469984\\
50.01	0.00433036723563757\\
51.01	0.00433037357509045\\
52.01	0.00433038004572285\\
53.01	0.00433038665025423\\
54.01	0.00433039339146085\\
55.01	0.00433040027217621\\
56.01	0.00433040729529295\\
57.01	0.004330414463764\\
58.01	0.00433042178060344\\
59.01	0.00433042924888788\\
60.01	0.00433043687175783\\
61.01	0.00433044465241937\\
62.01	0.00433045259414487\\
63.01	0.00433046070027501\\
64.01	0.00433046897421975\\
65.01	0.00433047741945979\\
66.01	0.00433048603954836\\
67.01	0.00433049483811269\\
68.01	0.00433050381885497\\
69.01	0.00433051298555459\\
70.01	0.0043305223420696\\
71.01	0.00433053189233826\\
72.01	0.00433054164038044\\
73.01	0.00433055159029981\\
74.01	0.00433056174628528\\
75.01	0.00433057211261298\\
76.01	0.00433058269364797\\
77.01	0.00433059349384581\\
78.01	0.00433060451775481\\
79.01	0.00433061577001809\\
80.01	0.00433062725537522\\
81.01	0.00433063897866427\\
82.01	0.00433065094482382\\
83.01	0.0043306631588957\\
84.01	0.00433067562602585\\
85.01	0.00433068835146781\\
86.01	0.00433070134058434\\
87.01	0.00433071459884948\\
88.01	0.00433072813185141\\
89.01	0.00433074194529469\\
90.01	0.00433075604500216\\
91.01	0.00433077043691803\\
92.01	0.00433078512711026\\
93.01	0.00433080012177299\\
94.01	0.00433081542722946\\
95.01	0.00433083104993409\\
96.01	0.00433084699647591\\
97.01	0.004330863273581\\
98.01	0.00433087988811552\\
99.01	0.00433089684708847\\
100.01	0.00433091415765466\\
101.01	0.00433093182711798\\
102.01	0.00433094986293427\\
103.01	0.00433096827271437\\
104.01	0.00433098706422815\\
105.01	0.00433100624540667\\
106.01	0.00433102582434618\\
107.01	0.00433104580931194\\
108.01	0.00433106620874071\\
109.01	0.00433108703124521\\
110.01	0.00433110828561756\\
111.01	0.0043311299808329\\
112.01	0.00433115212605309\\
113.01	0.00433117473063106\\
114.01	0.00433119780411428\\
115.01	0.00433122135624922\\
116.01	0.00433124539698553\\
117.01	0.00433126993647953\\
118.01	0.00433129498509968\\
119.01	0.0043313205534299\\
120.01	0.00433134665227493\\
121.01	0.00433137329266459\\
122.01	0.00433140048585827\\
123.01	0.00433142824335001\\
124.01	0.0043314565768733\\
125.01	0.00433148549840623\\
126.01	0.00433151502017609\\
127.01	0.00433154515466544\\
128.01	0.0043315759146164\\
129.01	0.00433160731303699\\
130.01	0.00433163936320614\\
131.01	0.00433167207867974\\
132.01	0.00433170547329559\\
133.01	0.00433173956118046\\
134.01	0.00433177435675522\\
135.01	0.00433180987474144\\
136.01	0.00433184613016742\\
137.01	0.0043318831383747\\
138.01	0.00433192091502473\\
139.01	0.00433195947610543\\
140.01	0.00433199883793768\\
141.01	0.00433203901718285\\
142.01	0.00433208003084949\\
143.01	0.00433212189630079\\
144.01	0.00433216463126229\\
145.01	0.00433220825382846\\
146.01	0.00433225278247137\\
147.01	0.00433229823604812\\
148.01	0.00433234463380919\\
149.01	0.00433239199540626\\
150.01	0.00433244034090077\\
151.01	0.00433248969077268\\
152.01	0.00433254006592888\\
153.01	0.00433259148771226\\
154.01	0.00433264397791094\\
155.01	0.00433269755876733\\
156.01	0.00433275225298779\\
157.01	0.00433280808375204\\
158.01	0.00433286507472368\\
159.01	0.0043329232500595\\
160.01	0.00433298263442053\\
161.01	0.00433304325298182\\
162.01	0.00433310513144383\\
163.01	0.00433316829604315\\
164.01	0.00433323277356362\\
165.01	0.00433329859134778\\
166.01	0.00433336577730876\\
167.01	0.00433343435994199\\
168.01	0.00433350436833752\\
169.01	0.00433357583219236\\
170.01	0.00433364878182305\\
171.01	0.0043337232481787\\
172.01	0.00433379926285424\\
173.01	0.00433387685810409\\
174.01	0.00433395606685556\\
175.01	0.00433403692272322\\
176.01	0.00433411946002318\\
177.01	0.00433420371378762\\
178.01	0.00433428971978003\\
179.01	0.00433437751451023\\
180.01	0.00433446713525018\\
181.01	0.00433455862004985\\
182.01	0.00433465200775332\\
183.01	0.00433474733801562\\
184.01	0.00433484465131949\\
185.01	0.00433494398899309\\
186.01	0.00433504539322707\\
187.01	0.00433514890709322\\
188.01	0.00433525457456245\\
189.01	0.00433536244052353\\
190.01	0.00433547255080315\\
191.01	0.00433558495218429\\
192.01	0.00433569969242707\\
193.01	0.00433581682028934\\
194.01	0.00433593638554661\\
195.01	0.00433605843901434\\
196.01	0.00433618303256922\\
197.01	0.00433631021917127\\
198.01	0.00433644005288711\\
199.01	0.00433657258891254\\
200.01	0.00433670788359651\\
201.01	0.00433684599446493\\
202.01	0.00433698698024577\\
203.01	0.00433713090089369\\
204.01	0.00433727781761672\\
205.01	0.00433742779290112\\
206.01	0.00433758089053933\\
207.01	0.00433773717565676\\
208.01	0.00433789671474018\\
209.01	0.0043380595756658\\
210.01	0.00433822582772853\\
211.01	0.00433839554167183\\
212.01	0.00433856878971793\\
213.01	0.00433874564559936\\
214.01	0.00433892618459006\\
215.01	0.00433911048353805\\
216.01	0.00433929862089806\\
217.01	0.00433949067676585\\
218.01	0.00433968673291245\\
219.01	0.00433988687281911\\
220.01	0.00434009118171333\\
221.01	0.00434029974660607\\
222.01	0.00434051265632852\\
223.01	0.00434073000157076\\
224.01	0.00434095187492078\\
225.01	0.0043411783709045\\
226.01	0.00434140958602644\\
227.01	0.00434164561881134\\
228.01	0.00434188656984646\\
229.01	0.0043421325418261\\
230.01	0.0043423836395948\\
231.01	0.00434263997019303\\
232.01	0.00434290164290378\\
233.01	0.00434316876929976\\
234.01	0.00434344146329135\\
235.01	0.0043437198411766\\
236.01	0.00434400402169097\\
237.01	0.00434429412605936\\
238.01	0.00434459027804859\\
239.01	0.00434489260402099\\
240.01	0.00434520123298968\\
241.01	0.0043455162966741\\
242.01	0.00434583792955812\\
243.01	0.00434616626894793\\
244.01	0.00434650145503243\\
245.01	0.00434684363094361\\
246.01	0.00434719294281992\\
247.01	0.00434754953986957\\
248.01	0.00434791357443598\\
249.01	0.00434828520206429\\
250.01	0.00434866458156952\\
251.01	0.00434905187510615\\
252.01	0.00434944724823953\\
253.01	0.00434985087001818\\
254.01	0.0043502629130481\\
255.01	0.00435068355356909\\
256.01	0.00435111297153221\\
257.01	0.00435155135067847\\
258.01	0.00435199887862077\\
259.01	0.00435245574692603\\
260.01	0.00435292215120055\\
261.01	0.00435339829117608\\
262.01	0.00435388437079812\\
263.01	0.00435438059831655\\
264.01	0.00435488718637828\\
265.01	0.00435540435212144\\
266.01	0.00435593231727177\\
267.01	0.00435647130824146\\
268.01	0.00435702155622998\\
269.01	0.00435758329732734\\
270.01	0.00435815677261919\\
271.01	0.0043587422282943\\
272.01	0.00435933991575502\\
273.01	0.00435995009172968\\
274.01	0.00436057301838702\\
275.01	0.0043612089634542\\
276.01	0.00436185820033605\\
277.01	0.00436252100823858\\
278.01	0.0043631976722935\\
279.01	0.00436388848368592\\
280.01	0.00436459373978564\\
281.01	0.00436531374427969\\
282.01	0.00436604880730937\\
283.01	0.00436679924560854\\
284.01	0.00436756538264553\\
285.01	0.00436834754876796\\
286.01	0.00436914608135021\\
287.01	0.0043699613249442\\
288.01	0.00437079363143217\\
289.01	0.00437164336018346\\
290.01	0.00437251087821332\\
291.01	0.0043733965603453\\
292.01	0.00437430078937547\\
293.01	0.00437522395624056\\
294.01	0.00437616646018743\\
295.01	0.00437712870894692\\
296.01	0.00437811111890782\\
297.01	0.00437911411529571\\
298.01	0.00438013813235176\\
299.01	0.00438118361351485\\
300.01	0.00438225101160518\\
301.01	0.00438334078900816\\
302.01	0.00438445341786145\\
303.01	0.00438558938024082\\
304.01	0.00438674916834735\\
305.01	0.0043879332846942\\
306.01	0.0043891422422918\\
307.01	0.00439037656483271\\
308.01	0.0043916367868736\\
309.01	0.00439292345401367\\
310.01	0.00439423712306931\\
311.01	0.00439557836224365\\
312.01	0.00439694775128867\\
313.01	0.00439834588165999\\
314.01	0.00439977335666125\\
315.01	0.00440123079157653\\
316.01	0.00440271881378908\\
317.01	0.00440423806288427\\
318.01	0.0044057891907312\\
319.01	0.00440737286154438\\
320.01	0.00440898975191703\\
321.01	0.00441064055082571\\
322.01	0.00441232595959958\\
323.01	0.00441404669184965\\
324.01	0.00441580347335209\\
325.01	0.00441759704187901\\
326.01	0.00441942814696891\\
327.01	0.00442129754962872\\
328.01	0.00442320602195728\\
329.01	0.00442515434667984\\
330.01	0.00442714331658105\\
331.01	0.00442917373382191\\
332.01	0.004431246409127\\
333.01	0.00443336216082205\\
334.01	0.00443552181370352\\
335.01	0.00443772619771644\\
336.01	0.004439976146417\\
337.01	0.00444227249518929\\
338.01	0.00444461607918624\\
339.01	0.00444700773095674\\
340.01	0.00444944827772067\\
341.01	0.00445193853824481\\
342.01	0.00445447931927011\\
343.01	0.00445707141143288\\
344.01	0.00445971558461861\\
345.01	0.00446241258267614\\
346.01	0.00446516311741744\\
347.01	0.00446796786181537\\
348.01	0.00447082744230989\\
349.01	0.00447374243011707\\
350.01	0.00447671333143547\\
351.01	0.00447974057643231\\
352.01	0.00448282450688689\\
353.01	0.00448596536236742\\
354.01	0.0044891632648139\\
355.01	0.00449241820140781\\
356.01	0.00449573000562052\\
357.01	0.00449909833635545\\
358.01	0.00450252265513718\\
359.01	0.00450600220135811\\
360.01	0.00450953596567819\\
361.01	0.00451312266179822\\
362.01	0.00451676069699428\\
363.01	0.00452044814204629\\
364.01	0.00452418270151704\\
365.01	0.00452796168578788\\
366.01	0.00453178198685936\\
367.01	0.00453564006073842\\
368.01	0.00453953192031815\\
369.01	0.00454345314410495\\
370.01	0.00454739890806999\\
371.01	0.0045513640504516\\
372.01	0.00455534318270358\\
373.01	0.00455933086423309\\
374.01	0.00456332186443281\\
375.01	0.00456731154322568\\
376.01	0.00457129639147949\\
377.01	0.00457527478596882\\
378.01	0.00457924803103825\\
379.01	0.00458322178203278\\
380.01	0.00458723576922909\\
381.01	0.00459132580643362\\
382.01	0.00459549317978798\\
383.01	0.00459973918071012\\
384.01	0.00460406511206603\\
385.01	0.00460847228753523\\
386.01	0.00461296203090729\\
387.01	0.00461753567530385\\
388.01	0.00462219456232112\\
389.01	0.00462694004108527\\
390.01	0.00463177346721592\\
391.01	0.00463669620168919\\
392.01	0.00464170960959405\\
393.01	0.00464681505877265\\
394.01	0.00465201391833729\\
395.01	0.00465730755705253\\
396.01	0.00466269734157579\\
397.01	0.00466818463454245\\
398.01	0.00467377079248614\\
399.01	0.00467945716358194\\
400.01	0.00468524508519763\\
401.01	0.00469113588124237\\
402.01	0.00469713085929439\\
403.01	0.00470323130749577\\
404.01	0.00470943849119407\\
405.01	0.00471575364931583\\
406.01	0.0047221779904517\\
407.01	0.00472871268863441\\
408.01	0.00473535887878805\\
409.01	0.00474211765182792\\
410.01	0.00474899004938528\\
411.01	0.00475597705813798\\
412.01	0.00476307960371755\\
413.01	0.00477029854417019\\
414.01	0.00477763466294448\\
415.01	0.0047850886613799\\
416.01	0.00479266115066852\\
417.01	0.00480035264326593\\
418.01	0.00480816354372278\\
419.01	0.00481609413891522\\
420.01	0.00482414458765092\\
421.01	0.00483231490963051\\
422.01	0.00484060497375127\\
423.01	0.00484901448574133\\
424.01	0.00485754297512131\\
425.01	0.00486618978150076\\
426.01	0.00487495404022541\\
427.01	0.00488383466740644\\
428.01	0.00489283034438197\\
429.01	0.00490193950168054\\
430.01	0.00491116030258371\\
431.01	0.00492049062641793\\
432.01	0.00492992805174296\\
433.01	0.00493946983965368\\
434.01	0.00494911291746512\\
435.01	0.00495885386312005\\
436.01	0.00496868889073661\\
437.01	0.00497861383780683\\
438.01	0.0049886241546663\\
439.01	0.00499871489698278\\
440.01	0.00500888072215998\\
441.01	0.00501911589071902\\
442.01	0.00502941427391444\\
443.01	0.00503976936905594\\
444.01	0.00505017432424469\\
445.01	0.00506062197449125\\
446.01	0.00507110489145547\\
447.01	0.00508161544932473\\
448.01	0.00509214590961422\\
449.01	0.00510268852790492\\
450.01	0.00511323568569115\\
451.01	0.00512378005054773\\
452.01	0.00513431476765311\\
453.01	0.00514483368523116\\
454.01	0.00515533161552699\\
455.01	0.00516580463131715\\
456.01	0.00517625039536382\\
457.01	0.00518666851626712\\
458.01	0.00519706091828341\\
459.01	0.005207432204131\\
460.01	0.00521778997756235\\
461.01	0.00522814507518973\\
462.01	0.00523851163286017\\
463.01	0.00524890687837509\\
464.01	0.00525935049808428\\
465.01	0.00526986336656013\\
466.01	0.00528046390900781\\
467.01	0.00529115632719554\\
468.01	0.00530193845264243\\
469.01	0.00531280807546486\\
470.01	0.00532376301529709\\
471.01	0.00533480115650591\\
472.01	0.00534592048846137\\
473.01	0.00535711915115739\\
474.01	0.00536839548638489\\
475.01	0.00537974809454407\\
476.01	0.00539117589704932\\
477.01	0.00540267820407779\\
478.01	0.00541425478715471\\
479.01	0.00542590595576906\\
480.01	0.00543763263680076\\
481.01	0.00544943645503797\\
482.01	0.00546131981246499\\
483.01	0.00547328596328735\\
484.01	0.0054853390808436\\
485.01	0.00549748431163628\\
486.01	0.00550972781073364\\
487.01	0.00552207675180393\\
488.01	0.00553453930414448\\
489.01	0.00554712456842048\\
490.01	0.00555984246268859\\
491.01	0.00557270355102107\\
492.01	0.00558571880923359\\
493.01	0.00559889932668814\\
494.01	0.00561225595107349\\
495.01	0.00562579889621909\\
496.01	0.00563953735363371\\
497.01	0.00565347917996437\\
498.01	0.00566763086460868\\
499.01	0.00568199845586227\\
500.01	0.00569658835651467\\
501.01	0.00571140738657194\\
502.01	0.00572646278248946\\
503.01	0.00574176219000061\\
504.01	0.0057573136497279\\
505.01	0.0057731255748838\\
506.01	0.00578920672058626\\
507.01	0.00580556614463656\\
508.01	0.0058222131600648\\
509.01	0.00583915728036193\\
510.01	0.00585640815910021\\
511.01	0.00587397552660074\\
512.01	0.00589186912741288\\
513.01	0.00591009866356706\\
514.01	0.00592867374970717\\
515.01	0.00594760388707689\\
516.01	0.00596689846351392\\
517.01	0.00598656678545\\
518.01	0.00600661814441272\\
519.01	0.00602706191228984\\
520.01	0.00604790762593617\\
521.01	0.00606916500494155\\
522.01	0.00609084394021554\\
523.01	0.00611295448121869\\
524.01	0.006135506823636\\
525.01	0.00615851129817362\\
526.01	0.00618197836122347\\
527.01	0.00620591858815658\\
528.01	0.00623034266995729\\
529.01	0.00625526141377142\\
530.01	0.00628068574768584\\
531.01	0.00630662672966783\\
532.01	0.00633309556005505\\
533.01	0.00636010359633393\\
534.01	0.00638766236823543\\
535.01	0.00641578359059092\\
536.01	0.00644447917130381\\
537.01	0.00647376121386944\\
538.01	0.00650364201786405\\
539.01	0.00653413407973057\\
540.01	0.00656525009409123\\
541.01	0.00659700295554078\\
542.01	0.00662940576081271\\
543.01	0.00666247181114391\\
544.01	0.00669621461459394\\
545.01	0.00673064788801325\\
546.01	0.00676578555831429\\
547.01	0.0068016417626947\\
548.01	0.00683823084750839\\
549.01	0.00687556736559398\\
550.01	0.00691366607204055\\
551.01	0.00695254191854659\\
552.01	0.00699221004650432\\
553.01	0.0070326857787142\\
554.01	0.00707398460951927\\
555.01	0.00711612219311656\\
556.01	0.00715911432978177\\
557.01	0.00720297694972537\\
558.01	0.00724772609428095\\
559.01	0.00729337789411719\\
560.01	0.00733994854415049\\
561.01	0.00738745427482568\\
562.01	0.00743591131940887\\
563.01	0.00748533587690427\\
564.01	0.00753574407015443\\
565.01	0.00758715189862288\\
566.01	0.00763957518530753\\
567.01	0.00769302951718983\\
568.01	0.00774753017858247\\
569.01	0.00780309207669819\\
570.01	0.00785972965872284\\
571.01	0.00791745681964229\\
572.01	0.00797628680004186\\
573.01	0.00803623207307779\\
574.01	0.0080973042198116\\
575.01	0.00815951379211108\\
576.01	0.00822287016236105\\
577.01	0.00828738135930615\\
578.01	0.00835305388947689\\
579.01	0.0084198925438449\\
580.01	0.00848790018963276\\
581.01	0.00855707754759355\\
582.01	0.00862742295560682\\
583.01	0.00869893212015025\\
584.01	0.00877159785815303\\
585.01	0.00884540983297857\\
586.01	0.00892035428990249\\
587.01	0.00899641379854713\\
588.01	0.00907356701243149\\
589.01	0.00915178845925224\\
590.01	0.00923104837992613\\
591.01	0.00931131264004169\\
592.01	0.00939254274449733\\
593.01	0.00947469599512985\\
594.01	0.00955772584254327\\
595.01	0.00964158249773262\\
596.01	0.00972621388721412\\
597.01	0.00981156705815533\\
598.01	0.00989759016861823\\
599.01	0.00996919203046376\\
599.02	0.00996973314335037\\
599.03	0.00997027095716939\\
599.04	0.00997080543920335\\
599.05	0.00997133655641379\\
599.06	0.00997186427543808\\
599.07	0.0099723885625862\\
599.08	0.00997290938383756\\
599.09	0.00997342670483771\\
599.1	0.00997394049089505\\
599.11	0.00997445070697751\\
599.12	0.00997495731770918\\
599.13	0.00997546028736696\\
599.14	0.00997595957987706\\
599.15	0.00997645515881165\\
599.16	0.00997694698738526\\
599.17	0.00997743502845131\\
599.18	0.00997791924449856\\
599.19	0.00997839959764748\\
599.2	0.00997887604964662\\
599.21	0.00997934856186899\\
599.22	0.00997981709530829\\
599.23	0.00998028161057521\\
599.24	0.00998074206789365\\
599.25	0.0099811984270969\\
599.26	0.00998165064762378\\
599.27	0.00998209868851477\\
599.28	0.00998254250840806\\
599.29	0.00998298206553559\\
599.3	0.00998341731771905\\
599.31	0.00998384822236584\\
599.32	0.00998427473646497\\
599.33	0.00998469681658292\\
599.34	0.00998511441885952\\
599.35	0.0099855274990037\\
599.36	0.00998593601228926\\
599.37	0.00998633991355056\\
599.38	0.00998673915717821\\
599.39	0.00998713369711469\\
599.4	0.00998752348684992\\
599.41	0.00998790847811157\\
599.42	0.00998828861974491\\
599.43	0.00998866386008804\\
599.44	0.00998903414696681\\
599.45	0.00998939942768985\\
599.46	0.00998975964904344\\
599.47	0.00999011475728636\\
599.48	0.00999046469814472\\
599.49	0.00999080941680669\\
599.5	0.00999114885791725\\
599.51	0.00999148296557278\\
599.52	0.0099918116833157\\
599.53	0.00999213495412901\\
599.54	0.00999245272043077\\
599.55	0.00999276492406855\\
599.56	0.00999307150631381\\
599.57	0.00999337240785624\\
599.58	0.00999366756879799\\
599.59	0.00999395692864795\\
599.6	0.00999424042631585\\
599.61	0.00999451800010642\\
599.62	0.00999478958771336\\
599.63	0.0099950551262134\\
599.64	0.00999531455206019\\
599.65	0.00999556780107816\\
599.66	0.00999581480845634\\
599.67	0.00999605550874211\\
599.68	0.00999628983583487\\
599.69	0.00999651772297967\\
599.7	0.00999673910276076\\
599.71	0.00999695390709509\\
599.72	0.00999716206722577\\
599.73	0.00999736351371538\\
599.74	0.00999755817643933\\
599.75	0.00999774598457904\\
599.76	0.00999792686661517\\
599.77	0.00999810075032068\\
599.78	0.00999826756275386\\
599.79	0.00999842723025133\\
599.8	0.00999857967842092\\
599.81	0.0099987248321345\\
599.82	0.00999886261552071\\
599.83	0.00999899295195769\\
599.84	0.00999911576406567\\
599.85	0.00999923097369951\\
599.86	0.00999933850194117\\
599.87	0.00999943826909211\\
599.88	0.00999953019466558\\
599.89	0.00999961419737891\\
599.9	0.00999969019514566\\
599.91	0.00999975810506767\\
599.92	0.00999981784342713\\
599.93	0.00999986932567848\\
599.94	0.00999991246644028\\
599.95	0.00999994717948697\\
599.96	0.00999997337774056\\
599.97	0.00999999097326228\\
599.98	0.00999999987724406\\
599.99	0.01\\
600	0.01\\
};
\addplot [color=blue!50!mycolor7,solid,forget plot]
  table[row sep=crcr]{%
0.01	0.00384759528872361\\
1.01	0.00384759650035796\\
2.01	0.00384759773699071\\
3.01	0.00384759899913818\\
4.01	0.0038476002873275\\
5.01	0.00384760160209639\\
6.01	0.00384760294399437\\
7.01	0.00384760431358136\\
8.01	0.00384760571142964\\
9.01	0.003847607138123\\
10.01	0.00384760859425731\\
11.01	0.00384761008044081\\
12.01	0.00384761159729444\\
13.01	0.00384761314545166\\
14.01	0.00384761472555935\\
15.01	0.00384761633827756\\
16.01	0.00384761798428016\\
17.01	0.00384761966425487\\
18.01	0.00384762137890358\\
19.01	0.00384762312894275\\
20.01	0.0038476249151037\\
21.01	0.00384762673813283\\
22.01	0.00384762859879202\\
23.01	0.00384763049785899\\
24.01	0.00384763243612715\\
25.01	0.00384763441440672\\
26.01	0.00384763643352468\\
27.01	0.00384763849432476\\
28.01	0.00384764059766865\\
29.01	0.00384764274443537\\
30.01	0.00384764493552242\\
31.01	0.00384764717184573\\
32.01	0.00384764945434041\\
33.01	0.00384765178396063\\
34.01	0.00384765416168033\\
35.01	0.00384765658849384\\
36.01	0.00384765906541576\\
37.01	0.00384766159348192\\
38.01	0.00384766417374949\\
39.01	0.00384766680729738\\
40.01	0.00384766949522697\\
41.01	0.00384767223866228\\
42.01	0.0038476750387509\\
43.01	0.00384767789666381\\
44.01	0.00384768081359639\\
45.01	0.00384768379076875\\
46.01	0.00384768682942599\\
47.01	0.00384768993083938\\
48.01	0.00384769309630635\\
49.01	0.00384769632715087\\
50.01	0.00384769962472466\\
51.01	0.00384770299040733\\
52.01	0.00384770642560689\\
53.01	0.00384770993176066\\
54.01	0.0038477135103356\\
55.01	0.00384771716282899\\
56.01	0.00384772089076922\\
57.01	0.00384772469571612\\
58.01	0.00384772857926173\\
59.01	0.00384773254303135\\
60.01	0.00384773658868366\\
61.01	0.0038477407179118\\
62.01	0.00384774493244371\\
63.01	0.00384774923404323\\
64.01	0.00384775362451083\\
65.01	0.00384775810568398\\
66.01	0.0038477626794382\\
67.01	0.00384776734768782\\
68.01	0.00384777211238706\\
69.01	0.00384777697553007\\
70.01	0.00384778193915269\\
71.01	0.00384778700533263\\
72.01	0.00384779217619059\\
73.01	0.0038477974538913\\
74.01	0.00384780284064408\\
75.01	0.00384780833870411\\
76.01	0.00384781395037312\\
77.01	0.00384781967800038\\
78.01	0.00384782552398412\\
79.01	0.00384783149077164\\
80.01	0.00384783758086119\\
81.01	0.00384784379680263\\
82.01	0.00384785014119883\\
83.01	0.00384785661670597\\
84.01	0.00384786322603557\\
85.01	0.00384786997195508\\
86.01	0.00384787685728921\\
87.01	0.00384788388492122\\
88.01	0.00384789105779376\\
89.01	0.00384789837891045\\
90.01	0.00384790585133714\\
91.01	0.00384791347820311\\
92.01	0.0038479212627022\\
93.01	0.00384792920809417\\
94.01	0.00384793731770647\\
95.01	0.00384794559493538\\
96.01	0.0038479540432473\\
97.01	0.00384796266618023\\
98.01	0.00384797146734549\\
99.01	0.00384798045042906\\
100.01	0.00384798961919297\\
101.01	0.00384799897747733\\
102.01	0.0038480085292014\\
103.01	0.00384801827836574\\
104.01	0.00384802822905329\\
105.01	0.00384803838543156\\
106.01	0.00384804875175446\\
107.01	0.00384805933236342\\
108.01	0.00384807013168999\\
109.01	0.0038480811542573\\
110.01	0.00384809240468159\\
111.01	0.00384810388767497\\
112.01	0.00384811560804697\\
113.01	0.00384812757070631\\
114.01	0.00384813978066323\\
115.01	0.00384815224303175\\
116.01	0.0038481649630312\\
117.01	0.00384817794598918\\
118.01	0.0038481911973432\\
119.01	0.00384820472264338\\
120.01	0.00384821852755422\\
121.01	0.00384823261785741\\
122.01	0.00384824699945434\\
123.01	0.00384826167836812\\
124.01	0.00384827666074634\\
125.01	0.00384829195286389\\
126.01	0.00384830756112499\\
127.01	0.00384832349206626\\
128.01	0.00384833975235986\\
129.01	0.00384835634881517\\
130.01	0.00384837328838253\\
131.01	0.00384839057815589\\
132.01	0.00384840822537588\\
133.01	0.00384842623743248\\
134.01	0.00384844462186852\\
135.01	0.00384846338638264\\
136.01	0.00384848253883255\\
137.01	0.00384850208723819\\
138.01	0.00384852203978511\\
139.01	0.00384854240482807\\
140.01	0.00384856319089436\\
141.01	0.00384858440668753\\
142.01	0.00384860606109089\\
143.01	0.0038486281631708\\
144.01	0.00384865072218134\\
145.01	0.00384867374756762\\
146.01	0.00384869724896983\\
147.01	0.00384872123622711\\
148.01	0.00384874571938182\\
149.01	0.00384877070868388\\
150.01	0.00384879621459475\\
151.01	0.00384882224779173\\
152.01	0.00384884881917291\\
153.01	0.0038488759398613\\
154.01	0.00384890362120956\\
155.01	0.00384893187480476\\
156.01	0.00384896071247302\\
157.01	0.00384899014628526\\
158.01	0.00384902018856088\\
159.01	0.00384905085187411\\
160.01	0.00384908214905842\\
161.01	0.00384911409321263\\
162.01	0.00384914669770575\\
163.01	0.00384917997618305\\
164.01	0.00384921394257109\\
165.01	0.00384924861108442\\
166.01	0.00384928399623073\\
167.01	0.00384932011281745\\
168.01	0.00384935697595744\\
169.01	0.00384939460107557\\
170.01	0.00384943300391552\\
171.01	0.00384947220054573\\
172.01	0.00384951220736654\\
173.01	0.00384955304111667\\
174.01	0.00384959471888087\\
175.01	0.00384963725809606\\
176.01	0.00384968067655957\\
177.01	0.0038497249924361\\
178.01	0.00384977022426548\\
179.01	0.00384981639097029\\
180.01	0.00384986351186402\\
181.01	0.0038499116066588\\
182.01	0.00384996069547404\\
183.01	0.0038500107988446\\
184.01	0.00385006193772938\\
185.01	0.00385011413352032\\
186.01	0.00385016740805139\\
187.01	0.00385022178360732\\
188.01	0.00385027728293341\\
189.01	0.00385033392924493\\
190.01	0.0038503917462368\\
191.01	0.00385045075809383\\
192.01	0.00385051098950047\\
193.01	0.00385057246565138\\
194.01	0.00385063521226227\\
195.01	0.00385069925558031\\
196.01	0.00385076462239561\\
197.01	0.00385083134005224\\
198.01	0.00385089943645991\\
199.01	0.00385096894010575\\
200.01	0.00385103988006619\\
201.01	0.00385111228601959\\
202.01	0.00385118618825844\\
203.01	0.00385126161770273\\
204.01	0.00385133860591234\\
205.01	0.00385141718510092\\
206.01	0.00385149738814972\\
207.01	0.00385157924862121\\
208.01	0.00385166280077365\\
209.01	0.00385174807957529\\
210.01	0.00385183512072013\\
211.01	0.00385192396064234\\
212.01	0.0038520146365324\\
213.01	0.00385210718635255\\
214.01	0.00385220164885365\\
215.01	0.00385229806359161\\
216.01	0.00385239647094437\\
217.01	0.00385249691212945\\
218.01	0.00385259942922155\\
219.01	0.00385270406517098\\
220.01	0.00385281086382266\\
221.01	0.00385291986993412\\
222.01	0.00385303112919588\\
223.01	0.00385314468825143\\
224.01	0.00385326059471697\\
225.01	0.00385337889720312\\
226.01	0.00385349964533548\\
227.01	0.0038536228897771\\
228.01	0.00385374868225075\\
229.01	0.0038538770755614\\
230.01	0.00385400812362009\\
231.01	0.0038541418814681\\
232.01	0.00385427840530114\\
233.01	0.00385441775249454\\
234.01	0.00385455998162938\\
235.01	0.00385470515251875\\
236.01	0.0038548533262347\\
237.01	0.00385500456513629\\
238.01	0.00385515893289778\\
239.01	0.00385531649453771\\
240.01	0.00385547731644899\\
241.01	0.0038556414664296\\
242.01	0.00385580901371367\\
243.01	0.00385598002900422\\
244.01	0.00385615458450575\\
245.01	0.00385633275395872\\
246.01	0.00385651461267368\\
247.01	0.00385670023756761\\
248.01	0.00385688970720062\\
249.01	0.00385708310181328\\
250.01	0.00385728050336587\\
251.01	0.00385748199557781\\
252.01	0.00385768766396863\\
253.01	0.00385789759590026\\
254.01	0.00385811188062017\\
255.01	0.003858330609306\\
256.01	0.0038585538751109\\
257.01	0.00385878177321131\\
258.01	0.00385901440085491\\
259.01	0.00385925185741097\\
260.01	0.00385949424442153\\
261.01	0.00385974166565466\\
262.01	0.00385999422715901\\
263.01	0.00386025203732057\\
264.01	0.00386051520691995\\
265.01	0.00386078384919294\\
266.01	0.00386105807989293\\
267.01	0.00386133801735403\\
268.01	0.00386162378255764\\
269.01	0.00386191549920036\\
270.01	0.00386221329376521\\
271.01	0.00386251729559428\\
272.01	0.00386282763696461\\
273.01	0.00386314445316586\\
274.01	0.00386346788258228\\
275.01	0.00386379806677595\\
276.01	0.0038641351505745\\
277.01	0.00386447928216084\\
278.01	0.00386483061316766\\
279.01	0.0038651892987745\\
280.01	0.00386555549780873\\
281.01	0.00386592937285116\\
282.01	0.0038663110903453\\
283.01	0.00386670082071099\\
284.01	0.00386709873846357\\
285.01	0.00386750502233688\\
286.01	0.00386791985541253\\
287.01	0.00386834342525398\\
288.01	0.00386877592404709\\
289.01	0.00386921754874664\\
290.01	0.00386966850122977\\
291.01	0.00387012898845597\\
292.01	0.0038705992226349\\
293.01	0.00387107942140262\\
294.01	0.00387156980800543\\
295.01	0.003872070611493\\
296.01	0.00387258206692204\\
297.01	0.00387310441556857\\
298.01	0.00387363790515272\\
299.01	0.0038741827900744\\
300.01	0.00387473933166152\\
301.01	0.00387530779843209\\
302.01	0.00387588846636999\\
303.01	0.00387648161921647\\
304.01	0.00387708754877832\\
305.01	0.00387770655525267\\
306.01	0.00387833894757257\\
307.01	0.00387898504377129\\
308.01	0.00387964517136917\\
309.01	0.00388031966778415\\
310.01	0.00388100888076767\\
311.01	0.003881713168868\\
312.01	0.00388243290192311\\
313.01	0.00388316846158542\\
314.01	0.00388392024188122\\
315.01	0.00388468864980734\\
316.01	0.0038854741059684\\
317.01	0.00388627704525726\\
318.01	0.0038870979175842\\
319.01	0.00388793718865625\\
320.01	0.00388879534081335\\
321.01	0.00388967287392496\\
322.01	0.00389057030635291\\
323.01	0.00389148817598656\\
324.01	0.00389242704135562\\
325.01	0.00389338748282962\\
326.01	0.00389437010390933\\
327.01	0.00389537553262136\\
328.01	0.0038964044230234\\
329.01	0.003897457456831\\
330.01	0.00389853534517806\\
331.01	0.00389963883052295\\
332.01	0.00390076868871406\\
333.01	0.00390192573123038\\
334.01	0.00390311080761456\\
335.01	0.00390432480811607\\
336.01	0.00390556866656577\\
337.01	0.00390684336350376\\
338.01	0.0039081499295861\\
339.01	0.00390948944929785\\
340.01	0.00391086306500041\\
341.01	0.00391227198134919\\
342.01	0.00391371747011414\\
343.01	0.00391520087544454\\
344.01	0.00391672361961845\\
345.01	0.00391828720932369\\
346.01	0.00391989324251817\\
347.01	0.00392154341592235\\
348.01	0.00392323953319734\\
349.01	0.00392498351386607\\
350.01	0.00392677740303493\\
351.01	0.00392862338197279\\
352.01	0.00393052377960143\\
353.01	0.00393248108494643\\
354.01	0.00393449796058669\\
355.01	0.00393657725712421\\
356.01	0.00393872202867239\\
357.01	0.00394093554932545\\
358.01	0.00394322133052201\\
359.01	0.00394558313914724\\
360.01	0.00394802501612099\\
361.01	0.00395055129509342\\
362.01	0.00395316662069238\\
363.01	0.00395587596553301\\
364.01	0.00395868464488944\\
365.01	0.00396159832751057\\
366.01	0.00396462304051402\\
367.01	0.00396776516556191\\
368.01	0.00397103142256452\\
369.01	0.00397442883589473\\
370.01	0.00397796467643419\\
371.01	0.00398164637059396\\
372.01	0.00398548136459353\\
373.01	0.00398947692853466\\
374.01	0.0039936398799023\\
375.01	0.00399797619969865\\
376.01	0.00400249050601793\\
377.01	0.00400718533887115\\
378.01	0.0040120601956959\\
379.01	0.00401711023818462\\
380.01	0.00402229707812446\\
381.01	0.00402758779734362\\
382.01	0.00403298441935207\\
383.01	0.00403848901199207\\
384.01	0.00404410368126233\\
385.01	0.00404983057187891\\
386.01	0.00405567186783042\\
387.01	0.0040616297929237\\
388.01	0.00406770661131813\\
389.01	0.00407390462804669\\
390.01	0.00408022618951885\\
391.01	0.00408667368400339\\
392.01	0.00409324954208565\\
393.01	0.00409995623709576\\
394.01	0.00410679628550188\\
395.01	0.0041137722472635\\
396.01	0.00412088672613654\\
397.01	0.00412814236992487\\
398.01	0.00413554187066803\\
399.01	0.00414308796475642\\
400.01	0.00415078343296359\\
401.01	0.00415863110038205\\
402.01	0.00416663383625075\\
403.01	0.00417479455365705\\
404.01	0.00418311620909673\\
405.01	0.00419160180187202\\
406.01	0.00420025437330558\\
407.01	0.0042090770057451\\
408.01	0.00421807282133005\\
409.01	0.00422724498048821\\
410.01	0.00423659668012834\\
411.01	0.00424613115148417\\
412.01	0.00425585165756716\\
413.01	0.00426576149017367\\
414.01	0.00427586396639013\\
415.01	0.00428616242452759\\
416.01	0.00429666021941308\\
417.01	0.00430736071695087\\
418.01	0.00431826728786092\\
419.01	0.00432938330048456\\
420.01	0.00434071211253735\\
421.01	0.00435225706167172\\
422.01	0.00436402145469421\\
423.01	0.00437600855526338\\
424.01	0.00438822156987256\\
425.01	0.00440066363189556\\
426.01	0.00441333778344674\\
427.01	0.00442624695477689\\
428.01	0.00443939394089102\\
429.01	0.00445278137503738\\
430.01	0.00446641169867595\\
431.01	0.00448028712748935\\
432.01	0.00449440961295119\\
433.01	0.00450878079891455\\
434.01	0.00452340197262956\\
435.01	0.00453827400954186\\
436.01	0.00455339731116888\\
437.01	0.00456877173529571\\
438.01	0.00458439651768355\\
439.01	0.00460027018444709\\
440.01	0.00461639045423427\\
441.01	0.00463275412934851\\
442.01	0.00464935697499815\\
443.01	0.00466619358595628\\
444.01	0.00468325724009715\\
445.01	0.00470053973856403\\
446.01	0.00471803123276715\\
447.01	0.00473572003906111\\
448.01	0.00475359244288599\\
449.01	0.0047716324954655\\
450.01	0.00478982180798224\\
451.01	0.00480813935064878\\
452.01	0.00482656126750973\\
453.01	0.00484506072241868\\
454.01	0.00486360779785242\\
455.01	0.00488216947654946\\
456.01	0.00490070974708109\\
457.01	0.00491918988926615\\
458.01	0.00493756901500489\\
459.01	0.00495580496616116\\
460.01	0.00497385570559329\\
461.01	0.00499168138295437\\
462.01	0.00500924731669823\\
463.01	0.00502652820708054\\
464.01	0.0050435138947469\\
465.01	0.00506021730225566\\
466.01	0.00507679117790721\\
467.01	0.00509351189056388\\
468.01	0.00511038239909231\\
469.01	0.00512739407767977\\
470.01	0.00514453748828496\\
471.01	0.00516180235492443\\
472.01	0.00517917754523689\\
473.01	0.00519665106255998\\
474.01	0.00521421005259325\\
475.01	0.00523184082933595\\
476.01	0.0052495289236258\\
477.01	0.00526725915335834\\
478.01	0.00528501571822812\\
479.01	0.00530278232776542\\
480.01	0.00532054236939011\\
481.01	0.00533827912325906\\
482.01	0.00535597603111575\\
483.01	0.00537361702653526\\
484.01	0.00539118693371128\\
485.01	0.00540867194101191\\
486.01	0.00542606015360278\\
487.01	0.00544334222601611\\
488.01	0.00546051206996545\\
489.01	0.00547756762406072\\
490.01	0.00549451165910218\\
491.01	0.00551135257363967\\
492.01	0.00552810510700925\\
493.01	0.0055447908580748\\
494.01	0.00556143844260698\\
495.01	0.005578083045178\\
496.01	0.00559476501949797\\
497.01	0.00561152704485662\\
498.01	0.00562840221187111\\
499.01	0.00564539569062671\\
500.01	0.0056625077707454\\
501.01	0.00567973968740582\\
502.01	0.00569709375686213\\
503.01	0.00571457351239865\\
504.01	0.00573218383614435\\
505.01	0.00574993108078917\\
506.01	0.00576782317370089\\
507.01	0.00578586969427461\\
508.01	0.00580408191360225\\
509.01	0.00582247278388372\\
510.01	0.00584105686366401\\
511.01	0.00585985016435045\\
512.01	0.00587886990413575\\
513.01	0.0058981341583032\\
514.01	0.00591766140122166\\
515.01	0.00593746994702705\\
516.01	0.00595757731575038\\
517.01	0.00597799958337351\\
518.01	0.00599875082345979\\
519.01	0.00601984291558474\\
520.01	0.00604128680950413\\
521.01	0.00606309389102688\\
522.01	0.00608527608585403\\
523.01	0.00610784581405038\\
524.01	0.00613081593090994\\
525.01	0.00615419965464391\\
526.01	0.00617801048227137\\
527.01	0.00620226209636413\\
528.01	0.00622696826687023\\
529.01	0.006252142754069\\
530.01	0.00627779922067693\\
531.01	0.0063039511629907\\
532.01	0.00633061187228706\\
533.01	0.00635779443775675\\
534.01	0.00638551179984718\\
535.01	0.00641377685612876\\
536.01	0.00644260260165693\\
537.01	0.00647200221739751\\
538.01	0.00650198906876377\\
539.01	0.00653257668666739\\
540.01	0.00656377875008843\\
541.01	0.00659560907141205\\
542.01	0.0066280815855853\\
543.01	0.00666121034404836\\
544.01	0.00669500951413639\\
545.01	0.00672949338414971\\
546.01	0.00676467637354784\\
547.01	0.00680057304676548\\
548.01	0.00683719812803159\\
549.01	0.00687456651353929\\
550.01	0.00691269327682495\\
551.01	0.00695159366529483\\
552.01	0.00699128309192888\\
553.01	0.00703177712610352\\
554.01	0.00707309148393236\\
555.01	0.00711524201785119\\
556.01	0.00715824470504815\\
557.01	0.00720211563422044\\
558.01	0.0072468709900294\\
559.01	0.0072925270345521\\
560.01	0.00733910008501349\\
561.01	0.00738660648715271\\
562.01	0.00743506258375307\\
563.01	0.00748448467812454\\
564.01	0.0075348889925328\\
565.01	0.00758629162138968\\
566.01	0.00763870847868745\\
567.01	0.00769215523903229\\
568.01	0.00774664727157911\\
569.01	0.00780219956613101\\
570.01	0.00785882665063752\\
571.01	0.00791654249931098\\
572.01	0.00797536043057803\\
573.01	0.00803529299409064\\
574.01	0.0080963518460355\\
575.01	0.00815854761199431\\
576.01	0.00822188973662608\\
577.01	0.00828638631949463\\
578.01	0.00835204393648558\\
579.01	0.00841886744645257\\
580.01	0.00848685978301172\\
581.01	0.00855602173179519\\
582.01	0.00862635169400304\\
583.01	0.00869784543780366\\
584.01	0.00877049584006834\\
585.01	0.00884429262215488\\
586.01	0.00891922208505276\\
587.01	0.00899526685127645\\
588.01	0.00907240562356496\\
589.01	0.00915061297387856\\
590.01	0.00922985918056657\\
591.01	0.00931011013716259\\
592.01	0.00939132736335054\\
593.01	0.00947346815761936\\
594.01	0.00955648594246635\\
595.01	0.00964033086731408\\
596.01	0.0097249507523202\\
597.01	0.00981029247891265\\
598.01	0.00989630396133466\\
599.01	0.00996919203046332\\
599.02	0.00996973314335002\\
599.03	0.00997027095716909\\
599.04	0.00997080543920308\\
599.05	0.00997133655641355\\
599.06	0.00997186427543785\\
599.07	0.00997238856258599\\
599.08	0.00997290938383737\\
599.09	0.00997342670483753\\
599.1	0.00997394049089489\\
599.11	0.00997445070697736\\
599.12	0.00997495731770904\\
599.13	0.00997546028736683\\
599.14	0.00997595957987695\\
599.15	0.00997645515881154\\
599.16	0.00997694698738516\\
599.17	0.00997743502845122\\
599.18	0.00997791924449848\\
599.19	0.0099783995976474\\
599.2	0.00997887604964655\\
599.21	0.00997934856186892\\
599.22	0.00997981709530823\\
599.23	0.00998028161057515\\
599.24	0.0099807420678936\\
599.25	0.00998119842709685\\
599.26	0.00998165064762374\\
599.27	0.00998209868851473\\
599.28	0.00998254250840802\\
599.29	0.00998298206553556\\
599.3	0.00998341731771902\\
599.31	0.00998384822236582\\
599.32	0.00998427473646495\\
599.33	0.0099846968165829\\
599.34	0.00998511441885951\\
599.35	0.00998552749900369\\
599.36	0.00998593601228924\\
599.37	0.00998633991355054\\
599.38	0.0099867391571782\\
599.39	0.00998713369711469\\
599.4	0.00998752348684992\\
599.41	0.00998790847811156\\
599.42	0.0099882886197449\\
599.43	0.00998866386008803\\
599.44	0.00998903414696681\\
599.45	0.00998939942768985\\
599.46	0.00998975964904344\\
599.47	0.00999011475728636\\
599.48	0.00999046469814471\\
599.49	0.00999080941680669\\
599.5	0.00999114885791725\\
599.51	0.00999148296557278\\
599.52	0.0099918116833157\\
599.53	0.00999213495412901\\
599.54	0.00999245272043077\\
599.55	0.00999276492406855\\
599.56	0.00999307150631381\\
599.57	0.00999337240785624\\
599.58	0.00999366756879799\\
599.59	0.00999395692864795\\
599.6	0.00999424042631585\\
599.61	0.00999451800010642\\
599.62	0.00999478958771336\\
599.63	0.0099950551262134\\
599.64	0.00999531455206019\\
599.65	0.00999556780107816\\
599.66	0.00999581480845634\\
599.67	0.00999605550874211\\
599.68	0.00999628983583487\\
599.69	0.00999651772297967\\
599.7	0.00999673910276076\\
599.71	0.00999695390709509\\
599.72	0.00999716206722577\\
599.73	0.00999736351371538\\
599.74	0.00999755817643933\\
599.75	0.00999774598457904\\
599.76	0.00999792686661517\\
599.77	0.00999810075032068\\
599.78	0.00999826756275386\\
599.79	0.00999842723025133\\
599.8	0.00999857967842092\\
599.81	0.0099987248321345\\
599.82	0.00999886261552071\\
599.83	0.00999899295195769\\
599.84	0.00999911576406567\\
599.85	0.00999923097369951\\
599.86	0.00999933850194117\\
599.87	0.00999943826909211\\
599.88	0.00999953019466558\\
599.89	0.00999961419737891\\
599.9	0.00999969019514566\\
599.91	0.00999975810506767\\
599.92	0.00999981784342713\\
599.93	0.00999986932567848\\
599.94	0.00999991246644028\\
599.95	0.00999994717948697\\
599.96	0.00999997337774056\\
599.97	0.00999999097326228\\
599.98	0.00999999987724406\\
599.99	0.01\\
600	0.01\\
};
\addplot [color=blue!40!mycolor9,solid,forget plot]
  table[row sep=crcr]{%
0.01	0.00367405832583603\\
1.01	0.00367405912102549\\
2.01	0.00367405993263262\\
3.01	0.00367406076099672\\
4.01	0.00367406160646402\\
5.01	0.00367406246938859\\
6.01	0.00367406335013067\\
7.01	0.00367406424905903\\
8.01	0.00367406516654958\\
9.01	0.00367406610298597\\
10.01	0.00367406705875989\\
11.01	0.00367406803427119\\
12.01	0.00367406902992784\\
13.01	0.00367407004614641\\
14.01	0.00367407108335191\\
15.01	0.00367407214197835\\
16.01	0.00367407322246871\\
17.01	0.00367407432527495\\
18.01	0.0036740754508585\\
19.01	0.00367407659969039\\
20.01	0.00367407777225132\\
21.01	0.00367407896903182\\
22.01	0.00367408019053285\\
23.01	0.00367408143726555\\
24.01	0.00367408270975178\\
25.01	0.00367408400852423\\
26.01	0.00367408533412621\\
27.01	0.00367408668711273\\
28.01	0.00367408806805\\
29.01	0.00367408947751607\\
30.01	0.00367409091610117\\
31.01	0.00367409238440748\\
32.01	0.00367409388304958\\
33.01	0.00367409541265495\\
34.01	0.00367409697386401\\
35.01	0.00367409856733053\\
36.01	0.0036741001937215\\
37.01	0.00367410185371803\\
38.01	0.00367410354801519\\
39.01	0.00367410527732256\\
40.01	0.00367410704236429\\
41.01	0.00367410884387957\\
42.01	0.00367411068262281\\
43.01	0.00367411255936412\\
44.01	0.00367411447488969\\
45.01	0.00367411643000158\\
46.01	0.00367411842551899\\
47.01	0.00367412046227759\\
48.01	0.00367412254113066\\
49.01	0.00367412466294911\\
50.01	0.00367412682862167\\
51.01	0.0036741290390556\\
52.01	0.0036741312951771\\
53.01	0.00367413359793124\\
54.01	0.00367413594828284\\
55.01	0.00367413834721683\\
56.01	0.00367414079573823\\
57.01	0.00367414329487306\\
58.01	0.00367414584566859\\
59.01	0.00367414844919367\\
60.01	0.00367415110653928\\
61.01	0.0036741538188189\\
62.01	0.00367415658716955\\
63.01	0.00367415941275123\\
64.01	0.00367416229674802\\
65.01	0.00367416524036898\\
66.01	0.0036741682448477\\
67.01	0.0036741713114434\\
68.01	0.00367417444144151\\
69.01	0.00367417763615423\\
70.01	0.00367418089692056\\
71.01	0.00367418422510739\\
72.01	0.00367418762210989\\
73.01	0.00367419108935223\\
74.01	0.00367419462828794\\
75.01	0.00367419824040066\\
76.01	0.0036742019272047\\
77.01	0.00367420569024587\\
78.01	0.00367420953110184\\
79.01	0.00367421345138316\\
80.01	0.00367421745273361\\
81.01	0.00367422153683078\\
82.01	0.00367422570538732\\
83.01	0.00367422996015113\\
84.01	0.00367423430290657\\
85.01	0.00367423873547459\\
86.01	0.00367424325971415\\
87.01	0.00367424787752242\\
88.01	0.00367425259083611\\
89.01	0.00367425740163188\\
90.01	0.00367426231192744\\
91.01	0.00367426732378213\\
92.01	0.00367427243929801\\
93.01	0.0036742776606208\\
94.01	0.00367428298994044\\
95.01	0.00367428842949226\\
96.01	0.00367429398155808\\
97.01	0.00367429964846672\\
98.01	0.00367430543259524\\
99.01	0.0036743113363702\\
100.01	0.00367431736226802\\
101.01	0.00367432351281664\\
102.01	0.0036743297905963\\
103.01	0.00367433619824085\\
104.01	0.00367434273843835\\
105.01	0.00367434941393292\\
106.01	0.00367435622752513\\
107.01	0.00367436318207376\\
108.01	0.00367437028049679\\
109.01	0.00367437752577268\\
110.01	0.00367438492094152\\
111.01	0.00367439246910635\\
112.01	0.00367440017343455\\
113.01	0.00367440803715885\\
114.01	0.00367441606357934\\
115.01	0.00367442425606429\\
116.01	0.00367443261805184\\
117.01	0.00367444115305116\\
118.01	0.00367444986464413\\
119.01	0.0036744587564871\\
120.01	0.00367446783231197\\
121.01	0.00367447709592808\\
122.01	0.00367448655122356\\
123.01	0.00367449620216714\\
124.01	0.00367450605281003\\
125.01	0.00367451610728699\\
126.01	0.00367452636981899\\
127.01	0.00367453684471392\\
128.01	0.0036745475363691\\
129.01	0.00367455844927339\\
130.01	0.00367456958800837\\
131.01	0.00367458095725051\\
132.01	0.00367459256177353\\
133.01	0.00367460440644998\\
134.01	0.00367461649625347\\
135.01	0.00367462883626063\\
136.01	0.00367464143165344\\
137.01	0.00367465428772142\\
138.01	0.0036746674098637\\
139.01	0.00367468080359164\\
140.01	0.00367469447453068\\
141.01	0.00367470842842302\\
142.01	0.00367472267113002\\
143.01	0.00367473720863488\\
144.01	0.00367475204704466\\
145.01	0.00367476719259334\\
146.01	0.00367478265164431\\
147.01	0.0036747984306931\\
148.01	0.00367481453637007\\
149.01	0.00367483097544331\\
150.01	0.00367484775482131\\
151.01	0.00367486488155649\\
152.01	0.00367488236284745\\
153.01	0.0036749002060424\\
154.01	0.0036749184186423\\
155.01	0.00367493700830412\\
156.01	0.00367495598284417\\
157.01	0.00367497535024085\\
158.01	0.00367499511863893\\
159.01	0.00367501529635234\\
160.01	0.00367503589186823\\
161.01	0.00367505691385027\\
162.01	0.00367507837114245\\
163.01	0.0036751002727728\\
164.01	0.00367512262795758\\
165.01	0.003675145446105\\
166.01	0.00367516873681918\\
167.01	0.0036751925099046\\
168.01	0.00367521677536986\\
169.01	0.00367524154343279\\
170.01	0.00367526682452371\\
171.01	0.00367529262929117\\
172.01	0.00367531896860572\\
173.01	0.00367534585356497\\
174.01	0.00367537329549816\\
175.01	0.00367540130597163\\
176.01	0.00367542989679323\\
177.01	0.00367545908001773\\
178.01	0.00367548886795219\\
179.01	0.00367551927316123\\
180.01	0.00367555030847222\\
181.01	0.00367558198698174\\
182.01	0.00367561432206047\\
183.01	0.00367564732735948\\
184.01	0.00367568101681631\\
185.01	0.00367571540466079\\
186.01	0.00367575050542167\\
187.01	0.00367578633393296\\
188.01	0.00367582290534023\\
189.01	0.00367586023510806\\
190.01	0.00367589833902585\\
191.01	0.00367593723321591\\
192.01	0.00367597693414015\\
193.01	0.00367601745860749\\
194.01	0.00367605882378159\\
195.01	0.00367610104718842\\
196.01	0.00367614414672417\\
197.01	0.00367618814066356\\
198.01	0.00367623304766794\\
199.01	0.00367627888679362\\
200.01	0.00367632567750083\\
201.01	0.00367637343966273\\
202.01	0.00367642219357428\\
203.01	0.0036764719599615\\
204.01	0.00367652275999129\\
205.01	0.00367657461528113\\
206.01	0.00367662754790893\\
207.01	0.00367668158042372\\
208.01	0.00367673673585571\\
209.01	0.00367679303772729\\
210.01	0.00367685051006431\\
211.01	0.00367690917740703\\
212.01	0.00367696906482211\\
213.01	0.00367703019791429\\
214.01	0.00367709260283866\\
215.01	0.00367715630631309\\
216.01	0.00367722133563136\\
217.01	0.00367728771867616\\
218.01	0.00367735548393265\\
219.01	0.00367742466050223\\
220.01	0.00367749527811696\\
221.01	0.00367756736715395\\
222.01	0.00367764095865095\\
223.01	0.00367771608432077\\
224.01	0.00367779277656807\\
225.01	0.00367787106850474\\
226.01	0.00367795099396734\\
227.01	0.00367803258753387\\
228.01	0.00367811588454109\\
229.01	0.00367820092110299\\
230.01	0.00367828773412955\\
231.01	0.00367837636134524\\
232.01	0.00367846684130912\\
233.01	0.00367855921343473\\
234.01	0.00367865351801115\\
235.01	0.00367874979622426\\
236.01	0.0036788480901784\\
237.01	0.00367894844291927\\
238.01	0.00367905089845701\\
239.01	0.0036791555017904\\
240.01	0.00367926229893116\\
241.01	0.0036793713369296\\
242.01	0.00367948266390042\\
243.01	0.00367959632904982\\
244.01	0.00367971238270315\\
245.01	0.00367983087633339\\
246.01	0.00367995186259069\\
247.01	0.00368007539533263\\
248.01	0.00368020152965538\\
249.01	0.00368033032192623\\
250.01	0.00368046182981627\\
251.01	0.00368059611233512\\
252.01	0.00368073322986634\\
253.01	0.00368087324420359\\
254.01	0.0036810162185886\\
255.01	0.00368116221774978\\
256.01	0.00368131130794246\\
257.01	0.00368146355699048\\
258.01	0.00368161903432902\\
259.01	0.00368177781104863\\
260.01	0.00368193995994104\\
261.01	0.00368210555554632\\
262.01	0.00368227467420189\\
263.01	0.00368244739409282\\
264.01	0.00368262379530425\\
265.01	0.00368280395987521\\
266.01	0.00368298797185448\\
267.01	0.00368317591735849\\
268.01	0.00368336788463104\\
269.01	0.00368356396410541\\
270.01	0.00368376424846817\\
271.01	0.00368396883272569\\
272.01	0.00368417781427288\\
273.01	0.00368439129296444\\
274.01	0.00368460937118845\\
275.01	0.00368483215394291\\
276.01	0.003685059748915\\
277.01	0.00368529226656313\\
278.01	0.00368552982020205\\
279.01	0.00368577252609094\\
280.01	0.00368602050352533\\
281.01	0.00368627387493163\\
282.01	0.00368653276596563\\
283.01	0.003686797305615\\
284.01	0.00368706762630486\\
285.01	0.00368734386400814\\
286.01	0.00368762615835948\\
287.01	0.00368791465277396\\
288.01	0.00368820949456992\\
289.01	0.00368851083509693\\
290.01	0.00368881882986823\\
291.01	0.00368913363869844\\
292.01	0.003689455425847\\
293.01	0.00368978436016636\\
294.01	0.00369012061525605\\
295.01	0.00369046436962391\\
296.01	0.00369081580685172\\
297.01	0.00369117511576859\\
298.01	0.00369154249063074\\
299.01	0.00369191813130812\\
300.01	0.00369230224347841\\
301.01	0.00369269503882901\\
302.01	0.00369309673526556\\
303.01	0.00369350755713052\\
304.01	0.00369392773542858\\
305.01	0.00369435750806205\\
306.01	0.00369479712007419\\
307.01	0.00369524682390303\\
308.01	0.00369570687964417\\
309.01	0.00369617755532457\\
310.01	0.00369665912718591\\
311.01	0.00369715187997963\\
312.01	0.00369765610727224\\
313.01	0.0036981721117632\\
314.01	0.00369870020561382\\
315.01	0.00369924071078917\\
316.01	0.003699793959412\\
317.01	0.00370036029413032\\
318.01	0.0037009400684974\\
319.01	0.00370153364736611\\
320.01	0.0037021414072972\\
321.01	0.00370276373698146\\
322.01	0.00370340103767629\\
323.01	0.00370405372365683\\
324.01	0.00370472222268241\\
325.01	0.00370540697647703\\
326.01	0.00370610844122576\\
327.01	0.00370682708808467\\
328.01	0.00370756340370593\\
329.01	0.00370831789077751\\
330.01	0.00370909106857538\\
331.01	0.00370988347352987\\
332.01	0.00371069565980338\\
333.01	0.00371152819987965\\
334.01	0.0037123816851611\\
335.01	0.00371325672657484\\
336.01	0.00371415395518282\\
337.01	0.00371507402279454\\
338.01	0.00371601760257771\\
339.01	0.00371698538966353\\
340.01	0.00371797810174047\\
341.01	0.0037189964796306\\
342.01	0.00372004128784083\\
343.01	0.00372111331508027\\
344.01	0.00372221337473191\\
345.01	0.00372334230526716\\
346.01	0.00372450097058653\\
347.01	0.0037256902602693\\
348.01	0.00372691108970988\\
349.01	0.00372816440011696\\
350.01	0.00372945115834382\\
351.01	0.00373077235651685\\
352.01	0.00373212901142059\\
353.01	0.00373352216359135\\
354.01	0.00373495287606578\\
355.01	0.0037364222327198\\
356.01	0.00373793133612502\\
357.01	0.00373948130483995\\
358.01	0.00374107327004145\\
359.01	0.00374270837139101\\
360.01	0.00374438775202041\\
361.01	0.00374611255250908\\
362.01	0.00374788390372034\\
363.01	0.00374970291835895\\
364.01	0.00375157068111615\\
365.01	0.00375348823728412\\
366.01	0.00375545657975034\\
367.01	0.00375747663434159\\
368.01	0.00375954924357592\\
369.01	0.0037616751490212\\
370.01	0.00376385497267603\\
371.01	0.003766089198096\\
372.01	0.00376837815244509\\
373.01	0.00377072199129273\\
374.01	0.00377312068888773\\
375.01	0.00377557403791379\\
376.01	0.00377808166449803\\
377.01	0.00378064306669181\\
378.01	0.00378325768800604\\
379.01	0.00378592504219442\\
380.01	0.00378864505725178\\
381.01	0.00379141857738417\\
382.01	0.00379424667125867\\
383.01	0.00379713043234045\\
384.01	0.00380007097979201\\
385.01	0.00380306945942864\\
386.01	0.00380612704473182\\
387.01	0.00380924493792692\\
388.01	0.00381242437112994\\
389.01	0.00381566660756871\\
390.01	0.00381897294288394\\
391.01	0.00382234470651839\\
392.01	0.0038257832631996\\
393.01	0.00382929001452471\\
394.01	0.00383286640065623\\
395.01	0.00383651390213738\\
396.01	0.00384023404183742\\
397.01	0.0038440283870385\\
398.01	0.00384789855167558\\
399.01	0.00385184619874275\\
400.01	0.00385587304288089\\
401.01	0.00385998085316258\\
402.01	0.00386417145609101\\
403.01	0.00386844673883276\\
404.01	0.00387280865270531\\
405.01	0.00387725921694236\\
406.01	0.00388180052276163\\
407.01	0.00388643473776389\\
408.01	0.00389116411069319\\
409.01	0.00389599097659257\\
410.01	0.00390091776239113\\
411.01	0.00390594699296457\\
412.01	0.00391108129771316\\
413.01	0.0039163234177077\\
414.01	0.00392167621345645\\
415.01	0.00392714267335492\\
416.01	0.00393272592288359\\
417.01	0.0039384292346282\\
418.01	0.00394425603920165\\
419.01	0.00395020993715844\\
420.01	0.00395629471199882\\
421.01	0.00396251434437153\\
422.01	0.00396887302759433\\
423.01	0.00397537518462468\\
424.01	0.00398202548662442\\
425.01	0.00398882887327795\\
426.01	0.00399579057503845\\
427.01	0.00400291613749372\\
428.01	0.00401021144805891\\
429.01	0.00401768276522525\\
430.01	0.00402533675060855\\
431.01	0.00403318050406516\\
432.01	0.00404122160215854\\
433.01	0.00404946814027847\\
434.01	0.00405792877873196\\
435.01	0.00406661279313371\\
436.01	0.00407553012943204\\
437.01	0.0040846914639002\\
438.01	0.0040941082684095\\
439.01	0.00410379288126253\\
440.01	0.00411375858380633\\
441.01	0.00412401968295002\\
442.01	0.00413459159956747\\
443.01	0.00414549096256252\\
444.01	0.00415673570807933\\
445.01	0.00416834518293846\\
446.01	0.00418034025082298\\
447.01	0.00419274339898484\\
448.01	0.00420557884222234\\
449.01	0.00421887261951838\\
450.01	0.00423265267690792\\
451.01	0.00424694892772715\\
452.01	0.00426179327819508\\
453.01	0.00427721960205298\\
454.01	0.00429326364241243\\
455.01	0.00430996281163371\\
456.01	0.00432735585042339\\
457.01	0.00434548229469846\\
458.01	0.00436438168218982\\
459.01	0.00438409240903823\\
460.01	0.00440465011819025\\
461.01	0.00442608546416438\\
462.01	0.00444842105000842\\
463.01	0.00447166726820816\\
464.01	0.00449581668842845\\
465.01	0.00452083653810631\\
466.01	0.00454655444585295\\
467.01	0.00457268258452102\\
468.01	0.00459920934856123\\
469.01	0.00462613306159242\\
470.01	0.00465345098640395\\
471.01	0.0046811591223869\\
472.01	0.0047092519385362\\
473.01	0.00473772202655891\\
474.01	0.00476655965321753\\
475.01	0.00479575222489502\\
476.01	0.00482528394100766\\
477.01	0.00485513585982052\\
478.01	0.00488528566086753\\
479.01	0.0049157072618335\\
480.01	0.00494637041136256\\
481.01	0.00497724026453776\\
482.01	0.0050082769508859\\
483.01	0.00503943514996553\\
484.01	0.00507066369672383\\
485.01	0.00510190524850224\\
486.01	0.0051330960586651\\
487.01	0.00516416591944975\\
488.01	0.00519503836025188\\
489.01	0.00522563121910394\\
490.01	0.00525585774707416\\
491.01	0.00528562846197784\\
492.01	0.00531485405192902\\
493.01	0.00534344970529017\\
494.01	0.00537134142467356\\
495.01	0.00539847497274243\\
496.01	0.00542482833968094\\
497.01	0.00545042992074787\\
498.01	0.00547574300765019\\
499.01	0.00550109153729324\\
500.01	0.00552645060857317\\
501.01	0.00555179473582924\\
502.01	0.00557709816769944\\
503.01	0.00560233528928952\\
504.01	0.00562748112029061\\
505.01	0.00565251192163366\\
506.01	0.00567740592120446\\
507.01	0.00570214416678325\\
508.01	0.00572671150974177\\
509.01	0.0057510977150463\\
510.01	0.00577529868053916\\
511.01	0.00579931772940393\\
512.01	0.00582316691157837\\
513.01	0.0058468682091317\\
514.01	0.00587045448251295\\
515.01	0.00589396991263982\\
516.01	0.00591746957905275\\
517.01	0.00594101765388125\\
518.01	0.00596468346858382\\
519.01	0.00598852627082647\\
520.01	0.00601256288541119\\
521.01	0.00603680034606036\\
522.01	0.00606124780133964\\
523.01	0.00608591668244098\\
524.01	0.00611082082606813\\
525.01	0.00613597653886037\\
526.01	0.00616140258431862\\
527.01	0.00618712007040617\\
528.01	0.00621315221465982\\
529.01	0.00623952396483905\\
530.01	0.00626626145807431\\
531.01	0.00629339131242648\\
532.01	0.00632093976509019\\
533.01	0.00634893170585903\\
534.01	0.00637738970975967\\
535.01	0.00640633326085757\\
536.01	0.00643577895491486\\
537.01	0.00646574287119041\\
538.01	0.00649624165212671\\
539.01	0.00652729244507207\\
540.01	0.0065589127943315\\
541.01	0.0065911205186732\\
542.01	0.00662393357819652\\
543.01	0.00665736993624814\\
544.01	0.00669144742622071\\
545.01	0.00672618363750042\\
546.01	0.00676159583667395\\
547.01	0.00679770094139254\\
548.01	0.00683451556229975\\
549.01	0.00687205612044749\\
550.01	0.00691033902574197\\
551.01	0.00694938081321948\\
552.01	0.00698919814651896\\
553.01	0.00702980778368401\\
554.01	0.0070712265449191\\
555.01	0.00711347128465088\\
556.01	0.00715655886917984\\
557.01	0.00720050616072122\\
558.01	0.0072453300078212\\
559.01	0.00729104724095515\\
560.01	0.00733767467060229\\
561.01	0.00738522908340952\\
562.01	0.00743372723060418\\
563.01	0.00748318580258283\\
564.01	0.00753362138874788\\
565.01	0.00758505042934002\\
566.01	0.00763748916253181\\
567.01	0.00769095356644288\\
568.01	0.00774545929526543\\
569.01	0.00780102160845901\\
570.01	0.0078576552917938\\
571.01	0.00791537456890428\\
572.01	0.00797419300200297\\
573.01	0.00803412338054319\\
574.01	0.00809517759694714\\
575.01	0.00815736650899904\\
576.01	0.00822069978879914\\
577.01	0.00828518575791731\\
578.01	0.00835083120818599\\
579.01	0.00841764120771106\\
580.01	0.00848561889195915\\
581.01	0.00855476524018159\\
582.01	0.00862507883799715\\
583.01	0.00869655562770745\\
584.01	0.00876918864890651\\
585.01	0.0088429677732069\\
586.01	0.00891787943850125\\
587.01	0.00899390639018665\\
588.01	0.00907102743936527\\
589.01	0.00914921725139712\\
590.01	0.00922844618250661\\
591.01	0.0093086801876622\\
592.01	0.00938988082996722\\
593.01	0.00947200543070414\\
594.01	0.00955500741044542\\
595.01	0.00963883688587809\\
596.01	0.00972344160493188\\
597.01	0.00980876832537324\\
598.01	0.00989476477037853\\
599.01	0.00996919203040513\\
599.02	0.00996973314331371\\
599.03	0.00997027095714612\\
599.04	0.00997080543918755\\
599.05	0.00997133655640176\\
599.06	0.00997186427542786\\
599.07	0.00997238856257705\\
599.08	0.00997290938382928\\
599.09	0.00997342670483015\\
599.1	0.00997394049088811\\
599.11	0.0099744507069711\\
599.12	0.00997495731770325\\
599.13	0.00997546028736146\\
599.14	0.00997595957987197\\
599.15	0.00997645515880693\\
599.16	0.00997694698738088\\
599.17	0.00997743502844726\\
599.18	0.00997791924449481\\
599.19	0.00997839959764401\\
599.2	0.00997887604964342\\
599.21	0.00997934856186603\\
599.22	0.00997981709530556\\
599.23	0.0099802816105727\\
599.24	0.00998074206789134\\
599.25	0.00998119842709478\\
599.26	0.00998165064762184\\
599.27	0.00998209868851299\\
599.28	0.00998254250840643\\
599.29	0.00998298206553411\\
599.3	0.0099834173177177\\
599.31	0.00998384822236462\\
599.32	0.00998427473646385\\
599.33	0.00998469681658191\\
599.34	0.00998511441885861\\
599.35	0.00998552749900288\\
599.36	0.00998593601228852\\
599.37	0.00998633991354989\\
599.38	0.00998673915717762\\
599.39	0.00998713369711416\\
599.4	0.00998752348684945\\
599.41	0.00998790847811114\\
599.42	0.00998828861974453\\
599.43	0.0099886638600877\\
599.44	0.00998903414696652\\
599.45	0.0099893994276896\\
599.46	0.00998975964904322\\
599.47	0.00999011475728616\\
599.48	0.00999046469814455\\
599.49	0.00999080941680655\\
599.5	0.00999114885791712\\
599.51	0.00999148296557267\\
599.52	0.00999181168331561\\
599.53	0.00999213495412893\\
599.54	0.0099924527204307\\
599.55	0.00999276492406849\\
599.56	0.00999307150631377\\
599.57	0.0099933724078562\\
599.58	0.00999366756879796\\
599.59	0.00999395692864792\\
599.6	0.00999424042631583\\
599.61	0.0099945180001064\\
599.62	0.00999478958771334\\
599.63	0.00999505512621339\\
599.64	0.00999531455206018\\
599.65	0.00999556780107815\\
599.66	0.00999581480845634\\
599.67	0.00999605550874211\\
599.68	0.00999628983583487\\
599.69	0.00999651772297967\\
599.7	0.00999673910276075\\
599.71	0.00999695390709509\\
599.72	0.00999716206722577\\
599.73	0.00999736351371538\\
599.74	0.00999755817643933\\
599.75	0.00999774598457904\\
599.76	0.00999792686661517\\
599.77	0.00999810075032068\\
599.78	0.00999826756275386\\
599.79	0.00999842723025133\\
599.8	0.00999857967842092\\
599.81	0.0099987248321345\\
599.82	0.00999886261552071\\
599.83	0.00999899295195769\\
599.84	0.00999911576406567\\
599.85	0.00999923097369951\\
599.86	0.00999933850194117\\
599.87	0.00999943826909211\\
599.88	0.00999953019466558\\
599.89	0.00999961419737891\\
599.9	0.00999969019514566\\
599.91	0.00999975810506767\\
599.92	0.00999981784342713\\
599.93	0.00999986932567848\\
599.94	0.00999991246644028\\
599.95	0.00999994717948697\\
599.96	0.00999997337774056\\
599.97	0.00999999097326228\\
599.98	0.00999999987724406\\
599.99	0.01\\
600	0.01\\
};
\addplot [color=blue!75!mycolor7,solid,forget plot]
  table[row sep=crcr]{%
0.01	0.00346027296821295\\
1.01	0.00346027362420681\\
2.01	0.00346027429381217\\
3.01	0.0034602749773122\\
4.01	0.00346027567499618\\
5.01	0.0034602763871589\\
6.01	0.00346027711410207\\
7.01	0.00346027785613308\\
8.01	0.00346027861356599\\
9.01	0.00346027938672151\\
10.01	0.00346028017592695\\
11.01	0.00346028098151639\\
12.01	0.00346028180383087\\
13.01	0.00346028264321883\\
14.01	0.00346028350003563\\
15.01	0.00346028437464441\\
16.01	0.00346028526741551\\
17.01	0.00346028617872732\\
18.01	0.00346028710896605\\
19.01	0.0034602880585259\\
20.01	0.00346028902780944\\
21.01	0.00346029001722768\\
22.01	0.00346029102720004\\
23.01	0.00346029205815479\\
24.01	0.00346029311052924\\
25.01	0.00346029418476959\\
26.01	0.00346029528133186\\
27.01	0.00346029640068117\\
28.01	0.00346029754329262\\
29.01	0.00346029870965113\\
30.01	0.00346029990025172\\
31.01	0.00346030111559985\\
32.01	0.00346030235621163\\
33.01	0.00346030362261389\\
34.01	0.00346030491534453\\
35.01	0.00346030623495253\\
36.01	0.00346030758199867\\
37.01	0.00346030895705529\\
38.01	0.00346031036070667\\
39.01	0.00346031179354944\\
40.01	0.00346031325619285\\
41.01	0.00346031474925859\\
42.01	0.00346031627338167\\
43.01	0.00346031782921026\\
44.01	0.00346031941740603\\
45.01	0.00346032103864478\\
46.01	0.00346032269361626\\
47.01	0.00346032438302477\\
48.01	0.00346032610758935\\
49.01	0.00346032786804411\\
50.01	0.0034603296651386\\
51.01	0.00346033149963806\\
52.01	0.00346033337232378\\
53.01	0.00346033528399348\\
54.01	0.00346033723546163\\
55.01	0.00346033922755966\\
56.01	0.00346034126113662\\
57.01	0.00346034333705923\\
58.01	0.00346034545621254\\
59.01	0.00346034761950002\\
60.01	0.00346034982784436\\
61.01	0.00346035208218756\\
62.01	0.0034603543834911\\
63.01	0.00346035673273698\\
64.01	0.00346035913092776\\
65.01	0.00346036157908705\\
66.01	0.00346036407826\\
67.01	0.00346036662951375\\
68.01	0.00346036923393772\\
69.01	0.00346037189264434\\
70.01	0.00346037460676944\\
71.01	0.00346037737747276\\
72.01	0.00346038020593843\\
73.01	0.00346038309337537\\
74.01	0.00346038604101807\\
75.01	0.00346038905012697\\
76.01	0.00346039212198898\\
77.01	0.00346039525791822\\
78.01	0.00346039845925639\\
79.01	0.00346040172737334\\
80.01	0.00346040506366795\\
81.01	0.00346040846956863\\
82.01	0.0034604119465335\\
83.01	0.00346041549605182\\
84.01	0.00346041911964391\\
85.01	0.00346042281886236\\
86.01	0.00346042659529237\\
87.01	0.00346043045055275\\
88.01	0.00346043438629623\\
89.01	0.00346043840421047\\
90.01	0.00346044250601864\\
91.01	0.00346044669348052\\
92.01	0.00346045096839281\\
93.01	0.00346045533259015\\
94.01	0.00346045978794601\\
95.01	0.00346046433637333\\
96.01	0.00346046897982541\\
97.01	0.00346047372029699\\
98.01	0.0034604785598249\\
99.01	0.0034604835004888\\
100.01	0.00346048854441262\\
101.01	0.003460493693765\\
102.01	0.00346049895076038\\
103.01	0.00346050431766014\\
104.01	0.00346050979677362\\
105.01	0.00346051539045875\\
106.01	0.00346052110112362\\
107.01	0.0034605269312271\\
108.01	0.00346053288328019\\
109.01	0.00346053895984705\\
110.01	0.00346054516354626\\
111.01	0.00346055149705174\\
112.01	0.00346055796309411\\
113.01	0.00346056456446209\\
114.01	0.00346057130400326\\
115.01	0.00346057818462554\\
116.01	0.00346058520929884\\
117.01	0.00346059238105594\\
118.01	0.00346059970299396\\
119.01	0.00346060717827577\\
120.01	0.00346061481013159\\
121.01	0.00346062260185992\\
122.01	0.0034606305568297\\
123.01	0.00346063867848137\\
124.01	0.0034606469703283\\
125.01	0.00346065543595889\\
126.01	0.00346066407903767\\
127.01	0.00346067290330737\\
128.01	0.00346068191259023\\
129.01	0.0034606911107898\\
130.01	0.00346070050189307\\
131.01	0.00346071008997172\\
132.01	0.00346071987918425\\
133.01	0.00346072987377806\\
134.01	0.00346074007809076\\
135.01	0.00346075049655294\\
136.01	0.00346076113368944\\
137.01	0.00346077199412187\\
138.01	0.00346078308257053\\
139.01	0.00346079440385637\\
140.01	0.00346080596290387\\
141.01	0.00346081776474244\\
142.01	0.00346082981450914\\
143.01	0.00346084211745108\\
144.01	0.00346085467892764\\
145.01	0.00346086750441305\\
146.01	0.00346088059949883\\
147.01	0.00346089396989629\\
148.01	0.00346090762143927\\
149.01	0.0034609215600868\\
150.01	0.00346093579192595\\
151.01	0.00346095032317421\\
152.01	0.00346096516018295\\
153.01	0.00346098030944\\
154.01	0.00346099577757264\\
155.01	0.00346101157135083\\
156.01	0.00346102769769017\\
157.01	0.00346104416365536\\
158.01	0.0034610609764632\\
159.01	0.00346107814348626\\
160.01	0.003461095672256\\
161.01	0.00346111357046652\\
162.01	0.0034611318459782\\
163.01	0.00346115050682108\\
164.01	0.00346116956119898\\
165.01	0.00346118901749312\\
166.01	0.00346120888426622\\
167.01	0.00346122917026641\\
168.01	0.0034612498844316\\
169.01	0.00346127103589311\\
170.01	0.00346129263398104\\
171.01	0.00346131468822759\\
172.01	0.00346133720837241\\
173.01	0.00346136020436684\\
174.01	0.00346138368637866\\
175.01	0.00346140766479729\\
176.01	0.00346143215023823\\
177.01	0.00346145715354887\\
178.01	0.00346148268581301\\
179.01	0.00346150875835664\\
180.01	0.00346153538275346\\
181.01	0.00346156257082997\\
182.01	0.00346159033467176\\
183.01	0.00346161868662928\\
184.01	0.00346164763932366\\
185.01	0.00346167720565309\\
186.01	0.00346170739879893\\
187.01	0.00346173823223225\\
188.01	0.00346176971972081\\
189.01	0.00346180187533525\\
190.01	0.0034618347134566\\
191.01	0.00346186824878295\\
192.01	0.00346190249633714\\
193.01	0.00346193747147391\\
194.01	0.00346197318988783\\
195.01	0.00346200966762107\\
196.01	0.00346204692107138\\
197.01	0.00346208496700048\\
198.01	0.00346212382254231\\
199.01	0.00346216350521213\\
200.01	0.00346220403291492\\
201.01	0.00346224542395469\\
202.01	0.00346228769704397\\
203.01	0.00346233087131317\\
204.01	0.00346237496632039\\
205.01	0.00346242000206184\\
206.01	0.00346246599898185\\
207.01	0.00346251297798326\\
208.01	0.00346256096043894\\
209.01	0.00346260996820243\\
210.01	0.00346266002361927\\
211.01	0.00346271114953928\\
212.01	0.00346276336932792\\
213.01	0.00346281670687917\\
214.01	0.00346287118662797\\
215.01	0.00346292683356303\\
216.01	0.00346298367324049\\
217.01	0.00346304173179739\\
218.01	0.00346310103596568\\
219.01	0.00346316161308691\\
220.01	0.0034632234911264\\
221.01	0.00346328669868928\\
222.01	0.00346335126503503\\
223.01	0.00346341722009433\\
224.01	0.00346348459448502\\
225.01	0.00346355341952927\\
226.01	0.00346362372727028\\
227.01	0.00346369555049068\\
228.01	0.0034637689227305\\
229.01	0.00346384387830587\\
230.01	0.00346392045232814\\
231.01	0.00346399868072374\\
232.01	0.00346407860025449\\
233.01	0.00346416024853842\\
234.01	0.00346424366407096\\
235.01	0.00346432888624702\\
236.01	0.00346441595538365\\
237.01	0.00346450491274311\\
238.01	0.00346459580055654\\
239.01	0.00346468866204866\\
240.01	0.00346478354146264\\
241.01	0.00346488048408608\\
242.01	0.0034649795362774\\
243.01	0.00346508074549308\\
244.01	0.00346518416031558\\
245.01	0.00346528983048211\\
246.01	0.00346539780691429\\
247.01	0.00346550814174827\\
248.01	0.00346562088836583\\
249.01	0.00346573610142654\\
250.01	0.00346585383690075\\
251.01	0.00346597415210312\\
252.01	0.0034660971057273\\
253.01	0.0034662227578818\\
254.01	0.00346635117012642\\
255.01	0.00346648240550983\\
256.01	0.00346661652860854\\
257.01	0.00346675360556607\\
258.01	0.00346689370413411\\
259.01	0.00346703689371411\\
260.01	0.00346718324540039\\
261.01	0.0034673328320244\\
262.01	0.00346748572819977\\
263.01	0.00346764201036888\\
264.01	0.00346780175685081\\
265.01	0.00346796504789028\\
266.01	0.00346813196570774\\
267.01	0.00346830259455144\\
268.01	0.00346847702075015\\
269.01	0.00346865533276736\\
270.01	0.00346883762125744\\
271.01	0.00346902397912266\\
272.01	0.00346921450157171\\
273.01	0.00346940928617991\\
274.01	0.00346960843295105\\
275.01	0.00346981204438045\\
276.01	0.00347002022551944\\
277.01	0.00347023308404214\\
278.01	0.00347045073031277\\
279.01	0.00347067327745584\\
280.01	0.00347090084142684\\
281.01	0.00347113354108534\\
282.01	0.0034713714982694\\
283.01	0.00347161483787171\\
284.01	0.00347186368791751\\
285.01	0.0034721181796441\\
286.01	0.00347237844758251\\
287.01	0.00347264462964009\\
288.01	0.00347291686718541\\
289.01	0.00347319530513482\\
290.01	0.00347348009204033\\
291.01	0.00347377138017986\\
292.01	0.00347406932564839\\
293.01	0.0034743740884513\\
294.01	0.00347468583259949\\
295.01	0.00347500472620522\\
296.01	0.00347533094158058\\
297.01	0.00347566465533702\\
298.01	0.0034760060484862\\
299.01	0.00347635530654243\\
300.01	0.00347671261962672\\
301.01	0.00347707818257161\\
302.01	0.00347745219502794\\
303.01	0.00347783486157195\\
304.01	0.00347822639181405\\
305.01	0.00347862700050824\\
306.01	0.00347903690766271\\
307.01	0.00347945633865043\\
308.01	0.00347988552432155\\
309.01	0.00348032470111508\\
310.01	0.00348077411117149\\
311.01	0.00348123400244509\\
312.01	0.00348170462881735\\
313.01	0.00348218625020862\\
314.01	0.00348267913269065\\
315.01	0.00348318354859799\\
316.01	0.00348369977663887\\
317.01	0.00348422810200456\\
318.01	0.00348476881647847\\
319.01	0.00348532221854259\\
320.01	0.00348588861348323\\
321.01	0.00348646831349399\\
322.01	0.00348706163777694\\
323.01	0.00348766891264135\\
324.01	0.00348829047159897\\
325.01	0.00348892665545671\\
326.01	0.00348957781240554\\
327.01	0.00349024429810557\\
328.01	0.00349092647576742\\
329.01	0.00349162471622844\\
330.01	0.00349233939802451\\
331.01	0.00349307090745658\\
332.01	0.00349381963865134\\
333.01	0.00349458599361593\\
334.01	0.00349537038228661\\
335.01	0.00349617322257001\\
336.01	0.00349699494037756\\
337.01	0.00349783596965168\\
338.01	0.00349869675238413\\
339.01	0.00349957773862557\\
340.01	0.00350047938648575\\
341.01	0.00350140216212408\\
342.01	0.00350234653972973\\
343.01	0.00350331300149104\\
344.01	0.00350430203755365\\
345.01	0.00350531414596637\\
346.01	0.003506349832615\\
347.01	0.00350740961114285\\
348.01	0.00350849400285846\\
349.01	0.0035096035366293\\
350.01	0.00351073874876223\\
351.01	0.00351190018287018\\
352.01	0.0035130883897261\\
353.01	0.00351430392710535\\
354.01	0.0035155473596173\\
355.01	0.00351681925852976\\
356.01	0.00351812020158951\\
357.01	0.0035194507728439\\
358.01	0.00352081156247081\\
359.01	0.00352220316662644\\
360.01	0.00352362618732319\\
361.01	0.00352508123235327\\
362.01	0.00352656891527919\\
363.01	0.0035280898555162\\
364.01	0.00352964467853894\\
365.01	0.00353123401625111\\
366.01	0.00353285850756602\\
367.01	0.00353451879925318\\
368.01	0.00353621554711559\\
369.01	0.00353794941757132\\
370.01	0.003539721089716\\
371.01	0.00354153125794724\\
372.01	0.00354338063522171\\
373.01	0.00354526995699678\\
374.01	0.0035471999858638\\
375.01	0.00354917151680359\\
376.01	0.0035511853828645\\
377.01	0.00355324246085807\\
378.01	0.0035553436763458\\
379.01	0.00355749000667126\\
380.01	0.00355968247959745\\
381.01	0.00356192216159083\\
382.01	0.00356421014844351\\
383.01	0.00356654756533187\\
384.01	0.0035689355677666\\
385.01	0.00357137534257635\\
386.01	0.00357386810892851\\
387.01	0.00357641511938682\\
388.01	0.00357901766100757\\
389.01	0.00358167705647555\\
390.01	0.00358439466528152\\
391.01	0.00358717188494148\\
392.01	0.00359001015226043\\
393.01	0.00359291094464055\\
394.01	0.00359587578143637\\
395.01	0.00359890622535709\\
396.01	0.00360200388391832\\
397.01	0.00360517041094378\\
398.01	0.00360840750811844\\
399.01	0.00361171692659451\\
400.01	0.00361510046865083\\
401.01	0.00361855998940695\\
402.01	0.00362209739859295\\
403.01	0.00362571466237528\\
404.01	0.00362941380523974\\
405.01	0.00363319691193145\\
406.01	0.0036370661294524\\
407.01	0.00364102366911618\\
408.01	0.00364507180865957\\
409.01	0.00364921289440978\\
410.01	0.00365344934350697\\
411.01	0.00365778364617858\\
412.01	0.00366221836806465\\
413.01	0.00366675615258946\\
414.01	0.00367139972337641\\
415.01	0.00367615188669986\\
416.01	0.00368101553396871\\
417.01	0.00368599364423231\\
418.01	0.00369108928670067\\
419.01	0.00369630562326653\\
420.01	0.00370164591101524\\
421.01	0.00370711350470643\\
422.01	0.00371271185920789\\
423.01	0.00371844453185686\\
424.01	0.00372431518472269\\
425.01	0.00373032758673646\\
426.01	0.00373648561564981\\
427.01	0.00374279325977601\\
428.01	0.00374925461946055\\
429.01	0.00375587390821697\\
430.01	0.00376265545345388\\
431.01	0.00376960369670651\\
432.01	0.00377672319327018\\
433.01	0.00378401861111711\\
434.01	0.00379149472895806\\
435.01	0.00379915643328761\\
436.01	0.00380700871422563\\
437.01	0.00381505665993977\\
438.01	0.00382330544939905\\
439.01	0.00383176034317328\\
440.01	0.00384042667195154\\
441.01	0.00384930982241018\\
442.01	0.00385841522001223\\
443.01	0.00386774830827336\\
444.01	0.00387731452398092\\
445.01	0.00388711926780829\\
446.01	0.00389716786973208\\
447.01	0.00390746554863938\\
448.01	0.00391801736552\\
449.01	0.00392882816968527\\
450.01	0.00393990253756635\\
451.01	0.00395124470384521\\
452.01	0.00396285848500437\\
453.01	0.0039747471959013\\
454.01	0.00398691356075675\\
455.01	0.00399935962109656\\
456.01	0.00401208664484539\\
457.01	0.0040250950431314\\
458.01	0.00403838430467857\\
459.01	0.00405195296229161\\
460.01	0.00406579861235158\\
461.01	0.00407991801718984\\
462.01	0.00409430733516124\\
463.01	0.00410896262159766\\
464.01	0.00412388169678813\\
465.01	0.0041390635464146\\
466.01	0.00415450916369327\\
467.01	0.00417022703877926\\
468.01	0.00418623020788076\\
469.01	0.00420253300612251\\
470.01	0.00421915334458593\\
471.01	0.00423611450437305\\
472.01	0.00425344692776813\\
473.01	0.00427119058098699\\
474.01	0.0042893980666307\\
475.01	0.00430813595353832\\
476.01	0.00432745418754948\\
477.01	0.00434738642238951\\
478.01	0.00436796854694536\\
479.01	0.00438923923852827\\
480.01	0.00441124012149316\\
481.01	0.00443401590478815\\
482.01	0.00445761448463721\\
483.01	0.00448208699358519\\
484.01	0.00450748777056427\\
485.01	0.00453387421800094\\
486.01	0.00456130650062177\\
487.01	0.00458984702570931\\
488.01	0.0046195596249929\\
489.01	0.00465050833221185\\
490.01	0.00468275559761284\\
491.01	0.00471635938155702\\
492.01	0.0047513690821035\\
493.01	0.00478782101251431\\
494.01	0.0048257319804129\\
495.01	0.00486509058339598\\
496.01	0.00490584537563689\\
497.01	0.00494788813973441\\
498.01	0.00499067706248237\\
499.01	0.00503382368061261\\
500.01	0.00507728600037356\\
501.01	0.00512101589736558\\
502.01	0.00516495853105634\\
503.01	0.00520905175454348\\
504.01	0.00525322552146334\\
505.01	0.00529740134029413\\
506.01	0.00534149182733723\\
507.01	0.00538540041562128\\
508.01	0.00542902130718709\\
509.01	0.005472239789007\\
510.01	0.00551493307597316\\
511.01	0.00555697190167869\\
512.01	0.00559822315355755\\
513.01	0.00563855394918638\\
514.01	0.0056778376830278\\
515.01	0.00571596274960215\\
516.01	0.00575284489717085\\
517.01	0.00578844451789892\\
518.01	0.00582279122548661\\
519.01	0.00585647215374869\\
520.01	0.00589007821401159\\
521.01	0.00592357682065511\\
522.01	0.00595693559919004\\
523.01	0.00599012481542081\\
524.01	0.00602311876244421\\
525.01	0.00605589728395156\\
526.01	0.00608844743812646\\
527.01	0.00612076532603694\\
528.01	0.00615285802945342\\
529.01	0.00618474555417383\\
530.01	0.00621646260793039\\
531.01	0.00624805994354959\\
532.01	0.00627960485971661\\
533.01	0.00631118026364552\\
534.01	0.00634288143230977\\
535.01	0.00637480875100246\\
536.01	0.00640702911071041\\
537.01	0.00643956106745507\\
538.01	0.00647242193593152\\
539.01	0.00650563211117126\\
540.01	0.00653921506614063\\
541.01	0.00657319725365189\\
542.01	0.00660760790790732\\
543.01	0.00664247870742235\\
544.01	0.00667784317785525\\
545.01	0.00671373582867647\\
546.01	0.0067501910560746\\
547.01	0.00678724184788092\\
548.01	0.00682491841768709\\
549.01	0.00686324700395446\\
550.01	0.00690224961801737\\
551.01	0.00694194683434955\\
552.01	0.00698235970957576\\
553.01	0.00702350973072825\\
554.01	0.00706541862943193\\
555.01	0.00710810817972993\\
556.01	0.00715159999085044\\
557.01	0.00719591531061176\\
558.01	0.00724107485952969\\
559.01	0.00728709871898849\\
560.01	0.00733400629729445\\
561.01	0.00738181639206545\\
562.01	0.00743054735256967\\
563.01	0.00748021729887154\\
564.01	0.00753084421884847\\
565.01	0.00758244591252368\\
566.01	0.00763503991220713\\
567.01	0.00768864340185235\\
568.01	0.00774327313825653\\
569.01	0.00779894537555497\\
570.01	0.00785567579287152\\
571.01	0.00791347942325609\\
572.01	0.00797237057980226\\
573.01	0.00803236277225495\\
574.01	0.00809346860516428\\
575.01	0.00815569964939458\\
576.01	0.00821906628915308\\
577.01	0.00828357755347728\\
578.01	0.0083492409343018\\
579.01	0.00841606219068275\\
580.01	0.00848404513879882\\
581.01	0.00855319142743901\\
582.01	0.00862350029902764\\
583.01	0.00869496833695104\\
584.01	0.00876758920113668\\
585.01	0.00884135335563897\\
586.01	0.00891624779447431\\
587.01	0.00899225577465261\\
588.01	0.00906935656751973\\
589.01	0.00914752524228344\\
590.01	0.00922673249973243\\
591.01	0.00930694457949549\\
592.01	0.00938812327098054\\
593.01	0.00947022606676473\\
594.01	0.00955320650816933\\
595.01	0.00963701478667503\\
596.01	0.00972159868250433\\
597.01	0.00980690494409953\\
598.01	0.00989288124056736\\
599.01	0.00996919201787552\\
599.02	0.00996973313515437\\
599.03	0.00997027095217347\\
599.04	0.00997080543638652\\
599.05	0.00997133655493895\\
599.06	0.00997186427466602\\
599.07	0.00997238856209093\\
599.08	0.00997290938342301\\
599.09	0.00997342670448527\\
599.1	0.00997394049058981\\
599.11	0.00997445070670796\\
599.12	0.00997495731746685\\
599.13	0.00997546028714593\\
599.14	0.00997595957967355\\
599.15	0.00997645515862339\\
599.16	0.00997694698721088\\
599.17	0.00997743502828965\\
599.18	0.0099779192443486\\
599.19	0.00997839959750833\\
599.2	0.0099788760495175\\
599.21	0.00997934856174921\\
599.22	0.00997981709519721\\
599.23	0.00998028161047228\\
599.24	0.00998074206779834\\
599.25	0.00998119842700871\\
599.26	0.00998165064754226\\
599.27	0.00998209868843949\\
599.28	0.00998254250833861\\
599.29	0.00998298206547159\\
599.3	0.00998341731766015\\
599.31	0.00998384822231171\\
599.32	0.00998427473641528\\
599.33	0.00998469681653738\\
599.34	0.00998511441881785\\
599.35	0.00998552749896564\\
599.36	0.00998593601225455\\
599.37	0.00998633991351897\\
599.38	0.00998673915714952\\
599.39	0.00998713369708868\\
599.4	0.0099875234868264\\
599.41	0.00998790847809034\\
599.42	0.0099882886197258\\
599.43	0.00998866386007088\\
599.44	0.00998903414695145\\
599.45	0.00998939942767614\\
599.46	0.00998975964903123\\
599.47	0.00999011475727552\\
599.48	0.00999046469813513\\
599.49	0.00999080941679823\\
599.5	0.00999114885790982\\
599.51	0.00999148296556627\\
599.52	0.00999181168331002\\
599.53	0.00999213495412407\\
599.54	0.0099924527204265\\
599.55	0.00999276492406487\\
599.56	0.00999307150631066\\
599.57	0.00999337240785355\\
599.58	0.00999366756879571\\
599.59	0.00999395692864602\\
599.6	0.00999424042631423\\
599.61	0.00999451800010506\\
599.62	0.00999478958771224\\
599.63	0.00999505512621248\\
599.64	0.00999531455205943\\
599.65	0.00999556780107754\\
599.66	0.00999581480845585\\
599.67	0.00999605550874172\\
599.68	0.00999628983583456\\
599.69	0.00999651772297942\\
599.7	0.00999673910276056\\
599.71	0.00999695390709495\\
599.72	0.00999716206722566\\
599.73	0.0099973635137153\\
599.74	0.00999755817643927\\
599.75	0.009997745984579\\
599.76	0.00999792686661514\\
599.77	0.00999810075032065\\
599.78	0.00999826756275384\\
599.79	0.00999842723025132\\
599.8	0.00999857967842091\\
599.81	0.00999872483213449\\
599.82	0.0099988626155207\\
599.83	0.00999899295195769\\
599.84	0.00999911576406567\\
599.85	0.00999923097369951\\
599.86	0.00999933850194117\\
599.87	0.00999943826909211\\
599.88	0.00999953019466558\\
599.89	0.00999961419737891\\
599.9	0.00999969019514566\\
599.91	0.00999975810506767\\
599.92	0.00999981784342713\\
599.93	0.00999986932567848\\
599.94	0.00999991246644028\\
599.95	0.00999994717948697\\
599.96	0.00999997337774056\\
599.97	0.00999999097326228\\
599.98	0.00999999987724406\\
599.99	0.01\\
600	0.01\\
};
\addplot [color=blue!80!mycolor9,solid,forget plot]
  table[row sep=crcr]{%
0.01	0.00266457163794791\\
1.01	0.0026645724869681\\
2.01	0.00266457335376111\\
3.01	0.00266457423870125\\
4.01	0.00266457514217061\\
5.01	0.00266457606455949\\
6.01	0.00266457700626637\\
7.01	0.0026645779676983\\
8.01	0.00266457894927092\\
9.01	0.0026645799514086\\
10.01	0.00266458097454493\\
11.01	0.00266458201912257\\
12.01	0.00266458308559365\\
13.01	0.00266458417441982\\
14.01	0.00266458528607275\\
15.01	0.00266458642103385\\
16.01	0.00266458757979494\\
17.01	0.0026645887628583\\
18.01	0.00266458997073694\\
19.01	0.0026645912039547\\
20.01	0.00266459246304654\\
21.01	0.0026645937485589\\
22.01	0.00266459506104984\\
23.01	0.00266459640108933\\
24.01	0.00266459776925942\\
25.01	0.00266459916615464\\
26.01	0.00266460059238206\\
27.01	0.00266460204856192\\
28.01	0.00266460353532749\\
29.01	0.00266460505332568\\
30.01	0.00266460660321709\\
31.01	0.00266460818567651\\
32.01	0.00266460980139311\\
33.01	0.00266461145107075\\
34.01	0.00266461313542835\\
35.01	0.00266461485520027\\
36.01	0.00266461661113639\\
37.01	0.00266461840400285\\
38.01	0.00266462023458196\\
39.01	0.0026646221036729\\
40.01	0.0026646240120918\\
41.01	0.00266462596067241\\
42.01	0.00266462795026632\\
43.01	0.00266462998174324\\
44.01	0.0026646320559917\\
45.01	0.00266463417391909\\
46.01	0.00266463633645225\\
47.01	0.00266463854453793\\
48.01	0.00266464079914327\\
49.01	0.00266464310125601\\
50.01	0.00266464545188521\\
51.01	0.00266464785206155\\
52.01	0.00266465030283769\\
53.01	0.00266465280528916\\
54.01	0.00266465536051439\\
55.01	0.0026646579696355\\
56.01	0.00266466063379865\\
57.01	0.00266466335417471\\
58.01	0.0026646661319597\\
59.01	0.00266466896837545\\
60.01	0.00266467186467\\
61.01	0.00266467482211827\\
62.01	0.00266467784202264\\
63.01	0.00266468092571354\\
64.01	0.00266468407455017\\
65.01	0.00266468728992068\\
66.01	0.0026646905732436\\
67.01	0.00266469392596776\\
68.01	0.00266469734957341\\
69.01	0.00266470084557262\\
70.01	0.00266470441551018\\
71.01	0.00266470806096418\\
72.01	0.0026647117835469\\
73.01	0.00266471558490531\\
74.01	0.00266471946672216\\
75.01	0.00266472343071654\\
76.01	0.00266472747864462\\
77.01	0.00266473161230064\\
78.01	0.00266473583351774\\
79.01	0.00266474014416871\\
80.01	0.00266474454616691\\
81.01	0.00266474904146714\\
82.01	0.00266475363206663\\
83.01	0.00266475832000586\\
84.01	0.00266476310736967\\
85.01	0.00266476799628808\\
86.01	0.00266477298893737\\
87.01	0.00266477808754098\\
88.01	0.00266478329437085\\
89.01	0.00266478861174824\\
90.01	0.00266479404204483\\
91.01	0.00266479958768383\\
92.01	0.00266480525114132\\
93.01	0.00266481103494726\\
94.01	0.00266481694168662\\
95.01	0.00266482297400069\\
96.01	0.00266482913458842\\
97.01	0.00266483542620755\\
98.01	0.00266484185167591\\
99.01	0.00266484841387294\\
100.01	0.00266485511574084\\
101.01	0.00266486196028608\\
102.01	0.00266486895058091\\
103.01	0.0026648760897645\\
104.01	0.00266488338104495\\
105.01	0.00266489082770029\\
106.01	0.00266489843308033\\
107.01	0.00266490620060843\\
108.01	0.0026649141337827\\
109.01	0.0026649222361779\\
110.01	0.00266493051144728\\
111.01	0.00266493896332422\\
112.01	0.0026649475956238\\
113.01	0.00266495641224506\\
114.01	0.00266496541717257\\
115.01	0.00266497461447854\\
116.01	0.00266498400832446\\
117.01	0.00266499360296354\\
118.01	0.00266500340274252\\
119.01	0.00266501341210373\\
120.01	0.00266502363558734\\
121.01	0.00266503407783369\\
122.01	0.00266504474358513\\
123.01	0.00266505563768876\\
124.01	0.0026650667650986\\
125.01	0.00266507813087803\\
126.01	0.00266508974020212\\
127.01	0.00266510159836031\\
128.01	0.00266511371075902\\
129.01	0.00266512608292429\\
130.01	0.00266513872050413\\
131.01	0.00266515162927195\\
132.01	0.00266516481512874\\
133.01	0.00266517828410628\\
134.01	0.00266519204237022\\
135.01	0.00266520609622276\\
136.01	0.00266522045210615\\
137.01	0.0026652351166055\\
138.01	0.00266525009645228\\
139.01	0.00266526539852762\\
140.01	0.00266528102986538\\
141.01	0.00266529699765627\\
142.01	0.00266531330925086\\
143.01	0.00266532997216343\\
144.01	0.00266534699407573\\
145.01	0.00266536438284079\\
146.01	0.00266538214648667\\
147.01	0.00266540029322075\\
148.01	0.00266541883143365\\
149.01	0.00266543776970332\\
150.01	0.00266545711679963\\
151.01	0.00266547688168856\\
152.01	0.00266549707353662\\
153.01	0.00266551770171561\\
154.01	0.00266553877580733\\
155.01	0.00266556030560825\\
156.01	0.00266558230113468\\
157.01	0.00266560477262746\\
158.01	0.00266562773055756\\
159.01	0.00266565118563111\\
160.01	0.00266567514879483\\
161.01	0.00266569963124173\\
162.01	0.00266572464441652\\
163.01	0.00266575020002176\\
164.01	0.00266577631002354\\
165.01	0.0026658029866576\\
166.01	0.00266583024243576\\
167.01	0.0026658580901521\\
168.01	0.00266588654288963\\
169.01	0.0026659156140269\\
170.01	0.00266594531724485\\
171.01	0.00266597566653391\\
172.01	0.00266600667620102\\
173.01	0.00266603836087716\\
174.01	0.00266607073552483\\
175.01	0.00266610381544564\\
176.01	0.00266613761628846\\
177.01	0.00266617215405712\\
178.01	0.002666207445119\\
179.01	0.00266624350621338\\
180.01	0.00266628035446018\\
181.01	0.00266631800736872\\
182.01	0.00266635648284699\\
183.01	0.00266639579921062\\
184.01	0.00266643597519291\\
185.01	0.00266647702995419\\
186.01	0.002666518983092\\
187.01	0.0026665618546513\\
188.01	0.00266660566513491\\
189.01	0.00266665043551419\\
190.01	0.00266669618724028\\
191.01	0.00266674294225503\\
192.01	0.00266679072300276\\
193.01	0.00266683955244196\\
194.01	0.00266688945405739\\
195.01	0.00266694045187248\\
196.01	0.00266699257046196\\
197.01	0.0026670458349648\\
198.01	0.00266710027109773\\
199.01	0.00266715590516838\\
200.01	0.00266721276408991\\
201.01	0.00266727087539453\\
202.01	0.00266733026724861\\
203.01	0.00266739096846752\\
204.01	0.00266745300853088\\
205.01	0.00266751641759831\\
206.01	0.00266758122652549\\
207.01	0.00266764746688065\\
208.01	0.00266771517096133\\
209.01	0.00266778437181175\\
210.01	0.00266785510324043\\
211.01	0.00266792739983816\\
212.01	0.00266800129699683\\
213.01	0.00266807683092807\\
214.01	0.00266815403868285\\
215.01	0.00266823295817135\\
216.01	0.00266831362818346\\
217.01	0.00266839608840921\\
218.01	0.00266848037946069\\
219.01	0.00266856654289356\\
220.01	0.00266865462122943\\
221.01	0.00266874465797903\\
222.01	0.00266883669766536\\
223.01	0.00266893078584811\\
224.01	0.00266902696914791\\
225.01	0.00266912529527159\\
226.01	0.0026692258130382\\
227.01	0.00266932857240515\\
228.01	0.00266943362449527\\
229.01	0.00266954102162476\\
230.01	0.00266965081733107\\
231.01	0.00266976306640226\\
232.01	0.00266987782490644\\
233.01	0.00266999515022235\\
234.01	0.00267011510107036\\
235.01	0.00267023773754418\\
236.01	0.00267036312114371\\
237.01	0.00267049131480807\\
238.01	0.00267062238294997\\
239.01	0.0026707563914905\\
240.01	0.00267089340789491\\
241.01	0.00267103350120903\\
242.01	0.00267117674209699\\
243.01	0.00267132320287911\\
244.01	0.0026714729575714\\
245.01	0.00267162608192548\\
246.01	0.00267178265346955\\
247.01	0.00267194275155039\\
248.01	0.00267210645737619\\
249.01	0.0026722738540605\\
250.01	0.002672445026667\\
251.01	0.00267262006225551\\
252.01	0.00267279904992885\\
253.01	0.00267298208088099\\
254.01	0.00267316924844579\\
255.01	0.00267336064814749\\
256.01	0.00267355637775195\\
257.01	0.00267375653731905\\
258.01	0.00267396122925628\\
259.01	0.00267417055837368\\
260.01	0.00267438463193963\\
261.01	0.00267460355973817\\
262.01	0.0026748274541278\\
263.01	0.00267505643010083\\
264.01	0.00267529060534465\\
265.01	0.00267553010030422\\
266.01	0.00267577503824559\\
267.01	0.00267602554532119\\
268.01	0.00267628175063627\\
269.01	0.00267654378631698\\
270.01	0.0026768117875793\\
271.01	0.00267708589280041\\
272.01	0.00267736624359056\\
273.01	0.00267765298486741\\
274.01	0.00267794626493075\\
275.01	0.00267824623553987\\
276.01	0.00267855305199202\\
277.01	0.00267886687320239\\
278.01	0.00267918786178601\\
279.01	0.00267951618414114\\
280.01	0.00267985201053431\\
281.01	0.00268019551518725\\
282.01	0.00268054687636552\\
283.01	0.00268090627646873\\
284.01	0.00268127390212285\\
285.01	0.0026816499442741\\
286.01	0.00268203459828478\\
287.01	0.00268242806403111\\
288.01	0.00268283054600282\\
289.01	0.00268324225340482\\
290.01	0.00268366340026081\\
291.01	0.00268409420551895\\
292.01	0.00268453489315959\\
293.01	0.00268498569230494\\
294.01	0.00268544683733134\\
295.01	0.00268591856798311\\
296.01	0.00268640112948912\\
297.01	0.00268689477268145\\
298.01	0.00268739975411637\\
299.01	0.00268791633619792\\
300.01	0.00268844478730357\\
301.01	0.00268898538191259\\
302.01	0.00268953840073737\\
303.01	0.00269010413085666\\
304.01	0.00269068286585223\\
305.01	0.00269127490594787\\
306.01	0.0026918805581516\\
307.01	0.00269250013640087\\
308.01	0.00269313396171067\\
309.01	0.00269378236232487\\
310.01	0.00269444567387148\\
311.01	0.00269512423952074\\
312.01	0.00269581841014684\\
313.01	0.00269652854449395\\
314.01	0.00269725500934594\\
315.01	0.00269799817969983\\
316.01	0.00269875843894426\\
317.01	0.00269953617904219\\
318.01	0.0027003318007179\\
319.01	0.00270114571364983\\
320.01	0.00270197833666783\\
321.01	0.00270283009795648\\
322.01	0.00270370143526415\\
323.01	0.00270459279611743\\
324.01	0.0027055046380432\\
325.01	0.0027064374287965\\
326.01	0.00270739164659582\\
327.01	0.00270836778036614\\
328.01	0.00270936632998942\\
329.01	0.00271038780656391\\
330.01	0.00271143273267192\\
331.01	0.00271250164265704\\
332.01	0.0027135950829113\\
333.01	0.002714713612172\\
334.01	0.00271585780182998\\
335.01	0.00271702823624914\\
336.01	0.00271822551309752\\
337.01	0.00271945024369141\\
338.01	0.00272070305335274\\
339.01	0.0027219845817795\\
340.01	0.00272329548343155\\
341.01	0.00272463642793109\\
342.01	0.00272600810047945\\
343.01	0.00272741120228998\\
344.01	0.00272884645103871\\
345.01	0.00273031458133332\\
346.01	0.00273181634520065\\
347.01	0.0027333525125948\\
348.01	0.00273492387192541\\
349.01	0.00273653123060816\\
350.01	0.00273817541563807\\
351.01	0.00273985727418651\\
352.01	0.00274157767422373\\
353.01	0.00274333750516745\\
354.01	0.00274513767855961\\
355.01	0.00274697912877253\\
356.01	0.00274886281374583\\
357.01	0.00275078971575658\\
358.01	0.00275276084222435\\
359.01	0.00275477722655294\\
360.01	0.00275683992901134\\
361.01	0.00275895003765677\\
362.01	0.00276110866930049\\
363.01	0.00276331697052061\\
364.01	0.00276557611872193\\
365.01	0.00276788732324527\\
366.01	0.00277025182652645\\
367.01	0.00277267090530398\\
368.01	0.00277514587187318\\
369.01	0.00277767807538037\\
370.01	0.00278026890314976\\
371.01	0.00278291978202763\\
372.01	0.00278563217972543\\
373.01	0.00278840760613281\\
374.01	0.00279124761456527\\
375.01	0.00279415380289599\\
376.01	0.00279712781451333\\
377.01	0.00280017133904137\\
378.01	0.00280328611321068\\
379.01	0.00280647392864265\\
380.01	0.00280973660924248\\
381.01	0.00281307600841563\\
382.01	0.00281649402727172\\
383.01	0.00281999261600058\\
384.01	0.00282357377511652\\
385.01	0.00282723955672774\\
386.01	0.00283099206583052\\
387.01	0.00283483346162763\\
388.01	0.00283876595887161\\
389.01	0.00284279182923128\\
390.01	0.00284691340268196\\
391.01	0.00285113306891834\\
392.01	0.00285545327878948\\
393.01	0.00285987654575489\\
394.01	0.00286440544736059\\
395.01	0.00286904262673481\\
396.01	0.00287379079410072\\
397.01	0.00287865272830584\\
398.01	0.00288363127836563\\
399.01	0.00288872936501952\\
400.01	0.00289394998229737\\
401.01	0.00289929619909338\\
402.01	0.00290477116074526\\
403.01	0.00291037809061451\\
404.01	0.00291612029166492\\
405.01	0.002922001148035\\
406.01	0.00292802412659954\\
407.01	0.00293419277851558\\
408.01	0.00294051074074617\\
409.01	0.00294698173755663\\
410.01	0.00295360958197426\\
411.01	0.00296039817720486\\
412.01	0.00296735151799532\\
413.01	0.00297447369193275\\
414.01	0.00298176888066713\\
415.01	0.00298924136104482\\
416.01	0.00299689550613684\\
417.01	0.00300473578614513\\
418.01	0.00301276676916589\\
419.01	0.00302099312178881\\
420.01	0.00302941960950472\\
421.01	0.00303805109689367\\
422.01	0.0030468925475587\\
423.01	0.00305594902376694\\
424.01	0.00306522568575372\\
425.01	0.00307472779063846\\
426.01	0.00308446069089321\\
427.01	0.00309442983229672\\
428.01	0.00310464075129446\\
429.01	0.00311509907167435\\
430.01	0.00312581050045253\\
431.01	0.00313678082284618\\
432.01	0.00314801589619097\\
433.01	0.00315952164263663\\
434.01	0.0031713040404252\\
435.01	0.00318336911352479\\
436.01	0.00319572291935036\\
437.01	0.00320837153425779\\
438.01	0.00322132103643843\\
439.01	0.00323457748577748\\
440.01	0.00324814690015546\\
441.01	0.00326203522757654\\
442.01	0.003276248313389\\
443.01	0.00329079186172066\\
444.01	0.00330567139007952\\
445.01	0.00332089217585739\\
446.01	0.00333645919321636\\
447.01	0.00335237703852002\\
448.01	0.00336864984208066\\
449.01	0.00338528116350889\\
450.01	0.00340227386735189\\
451.01	0.00341962997495885\\
452.01	0.0034373504875775\\
453.01	0.0034554351745154\\
454.01	0.00347388231872536\\
455.01	0.00349268841031794\\
456.01	0.00351184777615066\\
457.01	0.00353135213066103\\
458.01	0.0035511900293104\\
459.01	0.00357134620116009\\
460.01	0.00359180073090039\\
461.01	0.00361252805274208\\
462.01	0.00363349570902957\\
463.01	0.00365466282199656\\
464.01	0.00367597820322375\\
465.01	0.00369737782856295\\
466.01	0.00371878167771662\\
467.01	0.00374009055359666\\
468.01	0.00376133808360597\\
469.01	0.00378254449084704\\
470.01	0.00380359765481982\\
471.01	0.00382435754725771\\
472.01	0.00384465007892143\\
473.01	0.00386425891196347\\
474.01	0.00388291464302406\\
475.01	0.00390069266259455\\
476.01	0.00391881844986454\\
477.01	0.00393741767035533\\
478.01	0.00395650237704295\\
479.01	0.00397608452442232\\
480.01	0.00399617587633663\\
481.01	0.00401678789648693\\
482.01	0.00403793161924702\\
483.01	0.0040596174982915\\
484.01	0.00408185523052651\\
485.01	0.00410465355293726\\
486.01	0.00412802001034002\\
487.01	0.00415196069297844\\
488.01	0.00417647994903544\\
489.01	0.00420158020761564\\
490.01	0.00422726721482038\\
491.01	0.00425360121734022\\
492.01	0.00428062631288843\\
493.01	0.00430837070158088\\
494.01	0.00433684816062326\\
495.01	0.00436607012923727\\
496.01	0.00439604936910906\\
497.01	0.00442680147837339\\
498.01	0.00445834901184761\\
499.01	0.00449072614764581\\
500.01	0.00452397116594763\\
501.01	0.00455812478068281\\
502.01	0.00459323015236275\\
503.01	0.00462933283209169\\
504.01	0.00466648068637506\\
505.01	0.0047047241659741\\
506.01	0.0047441163056711\\
507.01	0.00478471255337544\\
508.01	0.00482657046799717\\
509.01	0.00486974924138452\\
510.01	0.00491430898617083\\
511.01	0.00496030971408751\\
512.01	0.00500780990689396\\
513.01	0.0050568645528759\\
514.01	0.00510752248267552\\
515.01	0.00515982277813339\\
516.01	0.00521378994643276\\
517.01	0.00526942495828889\\
518.01	0.00532661479683448\\
519.01	0.00538461916995028\\
520.01	0.00544272241137332\\
521.01	0.00550084980280282\\
522.01	0.00555891123659179\\
523.01	0.00561679111518276\\
524.01	0.00567435363040023\\
525.01	0.0057314487426998\\
526.01	0.0057879137030987\\
527.01	0.00584357440185067\\
528.01	0.00589824790083561\\
529.01	0.0059517466270772\\
530.01	0.00600388487332472\\
531.01	0.00605448868008126\\
532.01	0.00610341014093586\\
533.01	0.00615054761026743\\
534.01	0.00619587398233221\\
535.01	0.0062395504608343\\
536.01	0.00628272466409035\\
537.01	0.00632570687442321\\
538.01	0.0063684708154812\\
539.01	0.00641099749854468\\
540.01	0.00645327643854359\\
541.01	0.00649530646326558\\
542.01	0.0065370960188697\\
543.01	0.0065786662053737\\
544.01	0.00662005583566263\\
545.01	0.00666132366051709\\
546.01	0.00670254971458035\\
547.01	0.00674383537412062\\
548.01	0.00678530020261345\\
549.01	0.00682707436024067\\
550.01	0.00686925950890378\\
551.01	0.00691188719112618\\
552.01	0.00695498218311738\\
553.01	0.00699857348386164\\
554.01	0.00704269416314167\\
555.01	0.00708738096435513\\
556.01	0.00713267362631148\\
557.01	0.007178613887444\\
558.01	0.0072252441598414\\
559.01	0.00727260590629534\\
560.01	0.00732073783306641\\
561.01	0.00736967414027251\\
562.01	0.007419443185512\\
563.01	0.00747006821770754\\
564.01	0.00752157146830121\\
565.01	0.00757397540609622\\
566.01	0.00762730254286453\\
567.01	0.00768157517579493\\
568.01	0.00773681508916095\\
569.01	0.00779304325438374\\
570.01	0.0078502795596901\\
571.01	0.00790854260152617\\
572.01	0.007967849572551\\
573.01	0.0080282162773056\\
574.01	0.00808965727958438\\
575.01	0.00815218607059426\\
576.01	0.00821581501954256\\
577.01	0.00828055518630606\\
578.01	0.00834641611642781\\
579.01	0.00841340562488602\\
580.01	0.00848152957296922\\
581.01	0.00855079164303971\\
582.01	0.00862119311141858\\
583.01	0.00869273261663716\\
584.01	0.00876540591708722\\
585.01	0.00883920562907762\\
586.01	0.00891412093710792\\
587.01	0.00899013728670779\\
588.01	0.00906723608772262\\
589.01	0.00914539444740412\\
590.01	0.00922458495535043\\
591.01	0.00930477554784545\\
592.01	0.00938592948552506\\
593.01	0.00946800548660199\\
594.01	0.0095509580681256\\
595.01	0.00963473816035478\\
596.01	0.00971929407498429\\
597.01	0.00980457292772622\\
598.01	0.00989052264111993\\
599.01	0.0099691898561019\\
599.02	0.00996973151746593\\
599.03	0.00997026979807466\\
599.04	0.00997080466579245\\
599.05	0.00997133608806853\\
599.06	0.0099718640319399\\
599.07	0.00997238846402112\\
599.08	0.00997290935048918\\
599.09	0.00997342667879889\\
599.1	0.00997394047065652\\
599.11	0.00997445069119892\\
599.12	0.00997495730522922\\
599.13	0.0099754602772142\\
599.14	0.00997595957128175\\
599.15	0.00997645515121832\\
599.16	0.00997694698046649\\
599.17	0.00997743502212201\\
599.18	0.00997791923868709\\
599.19	0.00997839959229404\\
599.2	0.00997887604470164\\
599.21	0.00997934855729153\\
599.22	0.00997981709106457\\
599.23	0.00998028160663717\\
599.24	0.0099807420642377\\
599.25	0.00998119842370277\\
599.26	0.00998165064447353\\
599.27	0.00998209868559199\\
599.28	0.00998254250569747\\
599.29	0.00998298206302299\\
599.3	0.00998341731539121\\
599.31	0.00998384822021048\\
599.32	0.00998427473447066\\
599.33	0.00998469681473907\\
599.34	0.00998511441715626\\
599.35	0.00998552749743186\\
599.36	0.00998593601084028\\
599.37	0.00998633991221646\\
599.38	0.00998673915595153\\
599.39	0.00998713369598844\\
599.4	0.00998752348581752\\
599.41	0.00998790847716685\\
599.42	0.00998828861888209\\
599.43	0.0099886638593016\\
599.44	0.00998903414625157\\
599.45	0.00998939942704085\\
599.46	0.00998975964845599\\
599.47	0.00999011475675598\\
599.48	0.00999046469766715\\
599.49	0.0099908094163779\\
599.5	0.00999114885753339\\
599.51	0.00999148296523019\\
599.52	0.00999181168301094\\
599.53	0.00999213495385881\\
599.54	0.00999245272019206\\
599.55	0.00999276492385844\\
599.56	0.0099930715061296\\
599.57	0.00999337240769538\\
599.58	0.00999366756865814\\
599.59	0.00999395692852691\\
599.6	0.00999424042621159\\
599.61	0.00999451800001706\\
599.62	0.00999478958763719\\
599.63	0.00999505512614884\\
599.64	0.00999531455200579\\
599.65	0.00999556780103262\\
599.66	0.00999581480841847\\
599.67	0.00999605550871085\\
599.68	0.00999628983580927\\
599.69	0.00999651772295887\\
599.7	0.00999673910274401\\
599.71	0.00999695390708174\\
599.72	0.00999716206721524\\
599.73	0.00999736351370716\\
599.74	0.009997558176433\\
599.75	0.00999774598457423\\
599.76	0.00999792686661157\\
599.77	0.00999810075031802\\
599.78	0.00999826756275193\\
599.79	0.00999842723024996\\
599.8	0.00999857967841997\\
599.81	0.00999872483213385\\
599.82	0.00999886261552029\\
599.83	0.00999899295195742\\
599.84	0.00999911576406551\\
599.85	0.00999923097369942\\
599.86	0.00999933850194112\\
599.87	0.00999943826909208\\
599.88	0.00999953019466556\\
599.89	0.00999961419737891\\
599.9	0.00999969019514565\\
599.91	0.00999975810506767\\
599.92	0.00999981784342713\\
599.93	0.00999986932567848\\
599.94	0.00999991246644028\\
599.95	0.00999994717948697\\
599.96	0.00999997337774056\\
599.97	0.00999999097326229\\
599.98	0.00999999987724406\\
599.99	0.01\\
600	0.01\\
};
\addplot [color=blue,solid,forget plot]
  table[row sep=crcr]{%
0.01	0.00125393423089431\\
1.01	0.00125393505353844\\
2.01	0.00125393589347361\\
3.01	0.00125393675106585\\
4.01	0.00125393762668902\\
5.01	0.00125393852072497\\
6.01	0.00125393943356374\\
7.01	0.00125394036560372\\
8.01	0.00125394131725177\\
9.01	0.00125394228892352\\
10.01	0.00125394328104338\\
11.01	0.001253944294045\\
12.01	0.00125394532837118\\
13.01	0.00125394638447431\\
14.01	0.00125394746281636\\
15.01	0.00125394856386934\\
16.01	0.00125394968811527\\
17.01	0.0012539508360466\\
18.01	0.00125395200816618\\
19.01	0.00125395320498783\\
20.01	0.00125395442703635\\
21.01	0.00125395567484773\\
22.01	0.00125395694896955\\
23.01	0.00125395824996109\\
24.01	0.00125395957839373\\
25.01	0.00125396093485108\\
26.01	0.0012539623199293\\
27.01	0.00125396373423735\\
28.01	0.00125396517839733\\
29.01	0.00125396665304475\\
30.01	0.00125396815882872\\
31.01	0.00125396969641239\\
32.01	0.00125397126647316\\
33.01	0.00125397286970307\\
34.01	0.00125397450680905\\
35.01	0.00125397617851321\\
36.01	0.00125397788555339\\
37.01	0.00125397962868316\\
38.01	0.00125398140867252\\
39.01	0.00125398322630799\\
40.01	0.0012539850823931\\
41.01	0.00125398697774879\\
42.01	0.00125398891321365\\
43.01	0.00125399088964447\\
44.01	0.00125399290791648\\
45.01	0.00125399496892396\\
46.01	0.00125399707358046\\
47.01	0.00125399922281941\\
48.01	0.00125400141759426\\
49.01	0.0012540036588793\\
50.01	0.00125400594766984\\
51.01	0.00125400828498275\\
52.01	0.00125401067185706\\
53.01	0.00125401310935421\\
54.01	0.0012540155985587\\
55.01	0.00125401814057864\\
56.01	0.00125402073654616\\
57.01	0.00125402338761801\\
58.01	0.00125402609497604\\
59.01	0.00125402885982782\\
60.01	0.00125403168340715\\
61.01	0.00125403456697468\\
62.01	0.00125403751181852\\
63.01	0.00125404051925487\\
64.01	0.00125404359062846\\
65.01	0.0012540467273135\\
66.01	0.00125404993071397\\
67.01	0.00125405320226462\\
68.01	0.00125405654343137\\
69.01	0.00125405995571222\\
70.01	0.00125406344063787\\
71.01	0.00125406699977242\\
72.01	0.00125407063471412\\
73.01	0.00125407434709629\\
74.01	0.00125407813858777\\
75.01	0.00125408201089396\\
76.01	0.00125408596575772\\
77.01	0.00125409000495993\\
78.01	0.00125409413032049\\
79.01	0.00125409834369923\\
80.01	0.0012541026469967\\
81.01	0.00125410704215514\\
82.01	0.00125411153115942\\
83.01	0.00125411611603796\\
84.01	0.00125412079886368\\
85.01	0.00125412558175515\\
86.01	0.00125413046687735\\
87.01	0.00125413545644301\\
88.01	0.00125414055271341\\
89.01	0.0012541457579996\\
90.01	0.0012541510746636\\
91.01	0.00125415650511938\\
92.01	0.00125416205183409\\
93.01	0.00125416771732924\\
94.01	0.00125417350418203\\
95.01	0.00125417941502655\\
96.01	0.00125418545255487\\
97.01	0.00125419161951863\\
98.01	0.00125419791873025\\
99.01	0.00125420435306429\\
100.01	0.0012542109254589\\
101.01	0.0012542176389172\\
102.01	0.00125422449650878\\
103.01	0.00125423150137134\\
104.01	0.0012542386567118\\
105.01	0.00125424596580841\\
106.01	0.00125425343201209\\
107.01	0.00125426105874794\\
108.01	0.00125426884951726\\
109.01	0.00125427680789904\\
110.01	0.00125428493755174\\
111.01	0.00125429324221514\\
112.01	0.00125430172571228\\
113.01	0.00125431039195117\\
114.01	0.00125431924492683\\
115.01	0.00125432828872321\\
116.01	0.00125433752751532\\
117.01	0.0012543469655712\\
118.01	0.00125435660725406\\
119.01	0.0012543664570245\\
120.01	0.00125437651944268\\
121.01	0.00125438679917059\\
122.01	0.00125439730097445\\
123.01	0.00125440802972698\\
124.01	0.00125441899040989\\
125.01	0.00125443018811639\\
126.01	0.00125444162805375\\
127.01	0.00125445331554588\\
128.01	0.00125446525603598\\
129.01	0.00125447745508923\\
130.01	0.00125448991839586\\
131.01	0.00125450265177357\\
132.01	0.00125451566117089\\
133.01	0.00125452895266993\\
134.01	0.00125454253248947\\
135.01	0.00125455640698822\\
136.01	0.00125457058266792\\
137.01	0.0012545850661767\\
138.01	0.00125459986431244\\
139.01	0.00125461498402614\\
140.01	0.00125463043242562\\
141.01	0.00125464621677888\\
142.01	0.00125466234451806\\
143.01	0.00125467882324309\\
144.01	0.00125469566072556\\
145.01	0.00125471286491268\\
146.01	0.00125473044393147\\
147.01	0.00125474840609271\\
148.01	0.00125476675989535\\
149.01	0.00125478551403082\\
150.01	0.00125480467738739\\
151.01	0.00125482425905484\\
152.01	0.00125484426832918\\
153.01	0.00125486471471729\\
154.01	0.00125488560794186\\
155.01	0.00125490695794642\\
156.01	0.00125492877490042\\
157.01	0.00125495106920464\\
158.01	0.00125497385149625\\
159.01	0.0012549971326546\\
160.01	0.00125502092380669\\
161.01	0.00125504523633294\\
162.01	0.00125507008187316\\
163.01	0.00125509547233247\\
164.01	0.00125512141988763\\
165.01	0.00125514793699322\\
166.01	0.0012551750363882\\
167.01	0.00125520273110249\\
168.01	0.00125523103446372\\
169.01	0.00125525996010425\\
170.01	0.00125528952196834\\
171.01	0.00125531973431916\\
172.01	0.00125535061174647\\
173.01	0.00125538216917418\\
174.01	0.00125541442186815\\
175.01	0.00125544738544411\\
176.01	0.0012554810758759\\
177.01	0.00125551550950389\\
178.01	0.00125555070304341\\
179.01	0.00125558667359361\\
180.01	0.00125562343864633\\
181.01	0.00125566101609541\\
182.01	0.00125569942424602\\
183.01	0.00125573868182436\\
184.01	0.00125577880798727\\
185.01	0.00125581982233263\\
186.01	0.00125586174490941\\
187.01	0.00125590459622835\\
188.01	0.00125594839727284\\
189.01	0.0012559931695098\\
190.01	0.00125603893490105\\
191.01	0.001256085715915\\
192.01	0.00125613353553838\\
193.01	0.00125618241728846\\
194.01	0.00125623238522533\\
195.01	0.00125628346396483\\
196.01	0.00125633567869134\\
197.01	0.00125638905517123\\
198.01	0.00125644361976628\\
199.01	0.00125649939944798\\
200.01	0.00125655642181124\\
201.01	0.00125661471508962\\
202.01	0.00125667430816975\\
203.01	0.00125673523060681\\
204.01	0.0012567975126402\\
205.01	0.00125686118520943\\
206.01	0.00125692627997055\\
207.01	0.00125699282931286\\
208.01	0.00125706086637609\\
209.01	0.00125713042506786\\
210.01	0.00125720154008163\\
211.01	0.00125727424691511\\
212.01	0.00125734858188885\\
213.01	0.00125742458216568\\
214.01	0.00125750228577012\\
215.01	0.00125758173160866\\
216.01	0.00125766295949011\\
217.01	0.00125774601014684\\
218.01	0.00125783092525605\\
219.01	0.00125791774746189\\
220.01	0.00125800652039799\\
221.01	0.00125809728871031\\
222.01	0.0012581900980809\\
223.01	0.0012582849952516\\
224.01	0.001258382028049\\
225.01	0.00125848124540939\\
226.01	0.00125858269740458\\
227.01	0.00125868643526809\\
228.01	0.00125879251142234\\
229.01	0.00125890097950571\\
230.01	0.00125901189440101\\
231.01	0.00125912531226417\\
232.01	0.00125924129055357\\
233.01	0.00125935988806013\\
234.01	0.00125948116493796\\
235.01	0.00125960518273595\\
236.01	0.00125973200442973\\
237.01	0.00125986169445445\\
238.01	0.00125999431873861\\
239.01	0.00126012994473805\\
240.01	0.0012602686414711\\
241.01	0.0012604104795545\\
242.01	0.00126055553123994\\
243.01	0.00126070387045153\\
244.01	0.00126085557282399\\
245.01	0.00126101071574184\\
246.01	0.00126116937837926\\
247.01	0.00126133164174099\\
248.01	0.00126149758870402\\
249.01	0.00126166730406024\\
250.01	0.00126184087456004\\
251.01	0.00126201838895677\\
252.01	0.00126219993805231\\
253.01	0.00126238561474343\\
254.01	0.00126257551406953\\
255.01	0.001262769733261\\
256.01	0.00126296837178885\\
257.01	0.00126317153141541\\
258.01	0.00126337931624609\\
259.01	0.00126359183278239\\
260.01	0.00126380918997596\\
261.01	0.00126403149928375\\
262.01	0.00126425887472449\\
263.01	0.00126449143293652\\
264.01	0.00126472929323666\\
265.01	0.00126497257768039\\
266.01	0.00126522141112361\\
267.01	0.00126547592128547\\
268.01	0.0012657362388127\\
269.01	0.00126600249734532\\
270.01	0.00126627483358403\\
271.01	0.00126655338735864\\
272.01	0.00126683830169855\\
273.01	0.00126712972290432\\
274.01	0.00126742780062136\\
275.01	0.00126773268791475\\
276.01	0.00126804454134626\\
277.01	0.00126836352105265\\
278.01	0.00126868979082625\\
279.01	0.00126902351819685\\
280.01	0.00126936487451616\\
281.01	0.00126971403504353\\
282.01	0.00127007117903408\\
283.01	0.00127043648982901\\
284.01	0.00127081015494774\\
285.01	0.00127119236618251\\
286.01	0.00127158331969509\\
287.01	0.00127198321611599\\
288.01	0.00127239226064606\\
289.01	0.00127281066316061\\
290.01	0.00127323863831603\\
291.01	0.00127367640565934\\
292.01	0.00127412418974026\\
293.01	0.00127458222022641\\
294.01	0.00127505073202128\\
295.01	0.00127552996538557\\
296.01	0.0012760201660615\\
297.01	0.00127652158540051\\
298.01	0.00127703448049461\\
299.01	0.00127755911431093\\
300.01	0.00127809575583051\\
301.01	0.00127864468019063\\
302.01	0.00127920616883109\\
303.01	0.00127978050964508\\
304.01	0.00128036799713387\\
305.01	0.00128096893256652\\
306.01	0.00128158362414391\\
307.01	0.00128221238716784\\
308.01	0.00128285554421517\\
309.01	0.00128351342531733\\
310.01	0.00128418636814499\\
311.01	0.00128487471819885\\
312.01	0.0012855788290061\\
313.01	0.00128629906232317\\
314.01	0.00128703578834468\\
315.01	0.00128778938591937\\
316.01	0.00128856024277257\\
317.01	0.001289348755736\\
318.01	0.00129015533098513\\
319.01	0.00129098038428395\\
320.01	0.00129182434123811\\
321.01	0.00129268763755599\\
322.01	0.00129357071931841\\
323.01	0.00129447404325767\\
324.01	0.00129539807704496\\
325.01	0.00129634329958798\\
326.01	0.00129731020133809\\
327.01	0.00129829928460753\\
328.01	0.00129931106389727\\
329.01	0.00130034606623538\\
330.01	0.00130140483152689\\
331.01	0.00130248791291458\\
332.01	0.00130359587715185\\
333.01	0.00130472930498773\\
334.01	0.00130588879156403\\
335.01	0.0013070749468253\\
336.01	0.00130828839594193\\
337.01	0.00130952977974665\\
338.01	0.00131079975518428\\
339.01	0.00131209899577613\\
340.01	0.00131342819209829\\
341.01	0.00131478805227443\\
342.01	0.00131617930248375\\
343.01	0.00131760268748405\\
344.01	0.0013190589711502\\
345.01	0.00132054893702836\\
346.01	0.00132207338890637\\
347.01	0.00132363315140022\\
348.01	0.00132522907055724\\
349.01	0.00132686201447599\\
350.01	0.00132853287394308\\
351.01	0.00133024256308745\\
352.01	0.00133199202005168\\
353.01	0.00133378220768134\\
354.01	0.00133561411423148\\
355.01	0.00133748875409111\\
356.01	0.00133940716852558\\
357.01	0.00134137042643621\\
358.01	0.00134337962513773\\
359.01	0.00134543589115287\\
360.01	0.00134754038102385\\
361.01	0.0013496942821399\\
362.01	0.00135189881358083\\
363.01	0.00135415522697461\\
364.01	0.00135646480736899\\
365.01	0.00135882887411507\\
366.01	0.00136124878176148\\
367.01	0.00136372592095738\\
368.01	0.00136626171936242\\
369.01	0.00136885764256187\\
370.01	0.00137151519498482\\
371.01	0.00137423592082448\\
372.01	0.00137702140495972\\
373.01	0.00137987327387897\\
374.01	0.00138279319660863\\
375.01	0.00138578288565305\\
376.01	0.00138884409795629\\
377.01	0.0013919786359096\\
378.01	0.00139518834852271\\
379.01	0.00139847513283203\\
380.01	0.00140184093410652\\
381.01	0.00140528774874939\\
382.01	0.00140881762657582\\
383.01	0.00141243267238211\\
384.01	0.00141613504756924\\
385.01	0.00141992697182572\\
386.01	0.00142381072487216\\
387.01	0.00142778864827148\\
388.01	0.00143186314730736\\
389.01	0.00143603669293546\\
390.01	0.00144031182381085\\
391.01	0.00144469114839608\\
392.01	0.00144917734715422\\
393.01	0.00145377317483215\\
394.01	0.00145848146283897\\
395.01	0.0014633051217252\\
396.01	0.00146824714376911\\
397.01	0.00147331060567626\\
398.01	0.00147849867139957\\
399.01	0.00148381459508746\\
400.01	0.00148926172416791\\
401.01	0.00149484350257767\\
402.01	0.00150056347414533\\
403.01	0.00150642528613958\\
404.01	0.00151243269299284\\
405.01	0.00151858956021261\\
406.01	0.00152489986849356\\
407.01	0.0015313677180441\\
408.01	0.00153799733314351\\
409.01	0.00154479306694518\\
410.01	0.00155175940654517\\
411.01	0.00155890097833498\\
412.01	0.00156622255366081\\
413.01	0.00157372905481189\\
414.01	0.00158142556136492\\
415.01	0.00158931731691189\\
416.01	0.00159740973620281\\
417.01	0.00160570841273723\\
418.01	0.00161421912684314\\
419.01	0.00162294785428316\\
420.01	0.00163190077543637\\
421.01	0.00164108428510503\\
422.01	0.00165050500300411\\
423.01	0.00166016978499704\\
424.01	0.00167008573514798\\
425.01	0.00168026021867062\\
426.01	0.00169070087586256\\
427.01	0.00170141563712507\\
428.01	0.00171241273918239\\
429.01	0.00172370074262813\\
430.01	0.00173528855094479\\
431.01	0.00174718543116167\\
432.01	0.00175940103634059\\
433.01	0.00177194543010548\\
434.01	0.00178482911346454\\
435.01	0.00179806305421074\\
436.01	0.0018116587192307\\
437.01	0.00182562811010398\\
438.01	0.00183998380243701\\
439.01	0.00185473898944803\\
440.01	0.00186990753040684\\
441.01	0.00188550400463645\\
442.01	0.00190154377190793\\
443.01	0.00191804304020712\\
444.01	0.0019350189420306\\
445.01	0.0019524896205829\\
446.01	0.00197047432750603\\
447.01	0.00198899353408612\\
448.01	0.00200806905826303\\
449.01	0.0020277242102326\\
450.01	0.00204798395999738\\
451.01	0.0020688751309135\\
452.01	0.00209042662413107\\
453.01	0.00211266967986786\\
454.01	0.00213563818274333\\
455.01	0.00215936901998632\\
456.01	0.00218390250329652\\
457.01	0.00220928286757602\\
458.01	0.00223555886277866\\
459.01	0.00226278445889772\\
460.01	0.0022910196888251\\
461.01	0.00232033165971263\\
462.01	0.00235079577092562\\
463.01	0.00238249718617517\\
464.01	0.00241553261370436\\
465.01	0.00245001243512748\\
466.01	0.0024860629897989\\
467.01	0.00252269340217711\\
468.01	0.00253901728385071\\
469.01	0.00255639392450184\\
470.01	0.00257496450888221\\
471.01	0.00259489690471044\\
472.01	0.00261639053285141\\
473.01	0.00263968364758491\\
474.01	0.00266506226992916\\
475.01	0.00269245505943368\\
476.01	0.00272065093610824\\
477.01	0.00274955271826105\\
478.01	0.00277917729609491\\
479.01	0.00280954178593348\\
480.01	0.00284066351017588\\
481.01	0.00287255997420967\\
482.01	0.00290524884014917\\
483.01	0.00293874789733586\\
484.01	0.00297307502964344\\
485.01	0.00300824817977751\\
486.01	0.00304428531097289\\
487.01	0.00308120436689807\\
488.01	0.00311902323296728\\
489.01	0.00315775973843944\\
490.01	0.00319743226223462\\
491.01	0.00323806066098094\\
492.01	0.00327966220576215\\
493.01	0.00332225262627554\\
494.01	0.00336584605081814\\
495.01	0.00341045552030027\\
496.01	0.00345609286794123\\
497.01	0.00350276851295833\\
498.01	0.00355049120673921\\
499.01	0.00359926768736594\\
500.01	0.00364910239797952\\
501.01	0.00369999726787668\\
502.01	0.0037519516255841\\
503.01	0.00380496318885829\\
504.01	0.00385900734789926\\
505.01	0.00391403660138614\\
506.01	0.00396998585551344\\
507.01	0.0040267684552152\\
508.01	0.00408427124200197\\
509.01	0.00414234847335277\\
510.01	0.00420081432960689\\
511.01	0.00425943366533543\\
512.01	0.00431791057508681\\
513.01	0.00437587423268418\\
514.01	0.00443286131603846\\
515.01	0.00448829405724582\\
516.01	0.00454145169655468\\
517.01	0.00459201346058562\\
518.01	0.0046429781323353\\
519.01	0.00469489782545895\\
520.01	0.00474771890165974\\
521.01	0.00480136321045815\\
522.01	0.00485574293994415\\
523.01	0.00491145818565206\\
524.01	0.00496885945429269\\
525.01	0.00502801076693878\\
526.01	0.00508897183480263\\
527.01	0.00515179490913153\\
528.01	0.00521652055630504\\
529.01	0.00528317329708926\\
530.01	0.00535175362761153\\
531.01	0.00542221561762005\\
532.01	0.00549445951575484\\
533.01	0.00556831438034086\\
534.01	0.00564352082639238\\
535.01	0.00571974964221054\\
536.01	0.0057957442445397\\
537.01	0.00587105093686107\\
538.01	0.00594551194541186\\
539.01	0.00601897841781172\\
540.01	0.00609132568218902\\
541.01	0.00616247150904156\\
542.01	0.00623236290012066\\
543.01	0.0063008176648035\\
544.01	0.00636760942669748\\
545.01	0.00643254230701142\\
546.01	0.00649542408219803\\
547.01	0.00655610162007356\\
548.01	0.006614507589475\\
549.01	0.0066707307889387\\
550.01	0.00672605603223849\\
551.01	0.00678112490934816\\
552.01	0.00683590420357794\\
553.01	0.0068903704202434\\
554.01	0.00694451627906765\\
555.01	0.00699835431625151\\
556.01	0.00705192041560494\\
557.01	0.00710527686155195\\
558.01	0.00715851427507727\\
559.01	0.00721175147058358\\
560.01	0.00726513178502448\\
561.01	0.00731881411340836\\
562.01	0.00737295989880207\\
563.01	0.00742766298174074\\
564.01	0.00748295075754973\\
565.01	0.00753885104528757\\
566.01	0.0075953952343396\\
567.01	0.00765261854464693\\
568.01	0.00771055980364574\\
569.01	0.0077692603192376\\
570.01	0.00782876227912249\\
571.01	0.00788910669409165\\
572.01	0.0079503309275909\\
573.01	0.00801246605430344\\
574.01	0.00807553530800077\\
575.01	0.00813955656820071\\
576.01	0.00820454654838254\\
577.01	0.00827052111535247\\
578.01	0.00833749503822894\\
579.01	0.00840548173779629\\
580.01	0.00847449285543066\\
581.01	0.00854453779037137\\
582.01	0.00861562329403154\\
583.01	0.00868775317389058\\
584.01	0.00876092815854303\\
585.01	0.00883514596130418\\
586.01	0.00891040144580925\\
587.01	0.00898668645072877\\
588.01	0.00906398946084543\\
589.01	0.00914229527513433\\
590.01	0.0092215846004733\\
591.01	0.00930183366668292\\
592.01	0.00938301391778946\\
593.01	0.00946509184244681\\
594.01	0.00954802902026272\\
595.01	0.00963178247925906\\
596.01	0.00971630547999516\\
597.01	0.00980154886224196\\
598.01	0.00988746310858009\\
599.01	0.00996918176309228\\
599.02	0.00996972426151389\\
599.03	0.00997026325652431\\
599.04	0.00997079875308772\\
599.05	0.00997133075863836\\
599.06	0.00997185927537436\\
599.07	0.00997238430589235\\
599.08	0.00997290585509465\\
599.09	0.00997342387048931\\
599.1	0.00997393826406157\\
599.11	0.00997444900322816\\
599.12	0.00997495605368425\\
599.13	0.00997545938076358\\
599.14	0.00997595894942923\\
599.15	0.0099764547242638\\
599.16	0.00997694666945895\\
599.17	0.00997743474895895\\
599.18	0.00997791899742809\\
599.19	0.00997839937780886\\
599.2	0.00997887585262738\\
599.21	0.00997934838398374\\
599.22	0.00997981693354265\\
599.23	0.00998028146252\\
599.24	0.00998074193166859\\
599.25	0.00998119830126832\\
599.26	0.00998165053112998\\
599.27	0.0099820985805808\\
599.28	0.00998254240840773\\
599.29	0.00998298197288523\\
599.3	0.00998341723187397\\
599.31	0.00998384814281939\\
599.32	0.0099842746627474\\
599.33	0.00998469674826017\\
599.34	0.00998511435553172\\
599.35	0.00998552744030362\\
599.36	0.00998593595788055\\
599.37	0.00998633986312586\\
599.38	0.00998673911045709\\
599.39	0.00998713365384143\\
599.4	0.00998752344679118\\
599.41	0.00998790844105529\\
599.42	0.00998828858549677\\
599.43	0.0099886638284696\\
599.44	0.00998903411781367\\
599.45	0.00998939940084989\\
599.46	0.0099897596243751\\
599.47	0.00999011473465703\\
599.48	0.00999046467742929\\
599.49	0.00999080939788633\\
599.5	0.00999114884067832\\
599.51	0.00999148294990612\\
599.52	0.0099918116691162\\
599.53	0.00999213494129551\\
599.54	0.00999245270886605\\
599.55	0.0099927649136794\\
599.56	0.009993071497011\\
599.57	0.00999337239955457\\
599.58	0.00999366756141627\\
599.59	0.00999395692210898\\
599.6	0.00999424042054647\\
599.61	0.00999451799503739\\
599.62	0.00999478958327942\\
599.63	0.00999505512235319\\
599.64	0.00999531454871618\\
599.65	0.00999556779819663\\
599.66	0.00999581480598729\\
599.67	0.00999605550663917\\
599.68	0.00999628983405521\\
599.69	0.0099965177214839\\
599.7	0.00999673910151283\\
599.71	0.00999695390606217\\
599.72	0.00999716206637809\\
599.73	0.00999736351302614\\
599.74	0.00999755817588451\\
599.75	0.00999774598413729\\
599.76	0.00999792686626763\\
599.77	0.00999810075005082\\
599.78	0.00999826756254734\\
599.79	0.0099984272300958\\
599.8	0.00999857967830587\\
599.81	0.00999872483205108\\
599.82	0.00999886261546159\\
599.83	0.00999899295191686\\
599.84	0.0099991157640383\\
599.85	0.00999923097368178\\
599.86	0.00999933850193015\\
599.87	0.00999943826908557\\
599.88	0.00999953019466193\\
599.89	0.00999961419737702\\
599.9	0.00999969019514477\\
599.91	0.0099997581050673\\
599.92	0.009999817843427\\
599.93	0.00999986932567845\\
599.94	0.00999991246644027\\
599.95	0.00999994717948697\\
599.96	0.00999997337774056\\
599.97	0.00999999097326228\\
599.98	0.00999999987724406\\
599.99	0.01\\
600	0.01\\
};
\addplot [color=mycolor10,solid,forget plot]
  table[row sep=crcr]{%
0.01	0\\
1.01	0\\
2.01	0\\
3.01	0\\
4.01	0\\
5.01	0\\
6.01	0\\
7.01	0\\
8.01	0\\
9.01	0\\
10.01	0\\
11.01	0\\
12.01	0\\
13.01	0\\
14.01	0\\
15.01	0\\
16.01	0\\
17.01	0\\
18.01	0\\
19.01	0\\
20.01	0\\
21.01	0\\
22.01	0\\
23.01	0\\
24.01	0\\
25.01	0\\
26.01	0\\
27.01	0\\
28.01	0\\
29.01	0\\
30.01	0\\
31.01	0\\
32.01	0\\
33.01	0\\
34.01	0\\
35.01	0\\
36.01	0\\
37.01	0\\
38.01	0\\
39.01	0\\
40.01	0\\
41.01	0\\
42.01	0\\
43.01	0\\
44.01	0\\
45.01	0\\
46.01	0\\
47.01	0\\
48.01	0\\
49.01	0\\
50.01	0\\
51.01	0\\
52.01	0\\
53.01	0\\
54.01	0\\
55.01	0\\
56.01	0\\
57.01	0\\
58.01	0\\
59.01	0\\
60.01	0\\
61.01	0\\
62.01	0\\
63.01	0\\
64.01	0\\
65.01	0\\
66.01	0\\
67.01	0\\
68.01	0\\
69.01	0\\
70.01	0\\
71.01	0\\
72.01	0\\
73.01	0\\
74.01	0\\
75.01	0\\
76.01	0\\
77.01	0\\
78.01	0\\
79.01	0\\
80.01	0\\
81.01	0\\
82.01	0\\
83.01	0\\
84.01	0\\
85.01	0\\
86.01	0\\
87.01	0\\
88.01	0\\
89.01	0\\
90.01	0\\
91.01	0\\
92.01	0\\
93.01	0\\
94.01	0\\
95.01	0\\
96.01	0\\
97.01	0\\
98.01	0\\
99.01	0\\
100.01	0\\
101.01	0\\
102.01	0\\
103.01	0\\
104.01	0\\
105.01	0\\
106.01	0\\
107.01	0\\
108.01	0\\
109.01	0\\
110.01	0\\
111.01	0\\
112.01	0\\
113.01	0\\
114.01	0\\
115.01	0\\
116.01	0\\
117.01	0\\
118.01	0\\
119.01	0\\
120.01	0\\
121.01	0\\
122.01	0\\
123.01	0\\
124.01	0\\
125.01	0\\
126.01	0\\
127.01	0\\
128.01	0\\
129.01	0\\
130.01	0\\
131.01	0\\
132.01	0\\
133.01	0\\
134.01	0\\
135.01	0\\
136.01	0\\
137.01	0\\
138.01	0\\
139.01	0\\
140.01	0\\
141.01	0\\
142.01	0\\
143.01	0\\
144.01	0\\
145.01	0\\
146.01	0\\
147.01	0\\
148.01	0\\
149.01	0\\
150.01	0\\
151.01	0\\
152.01	0\\
153.01	0\\
154.01	0\\
155.01	0\\
156.01	0\\
157.01	0\\
158.01	0\\
159.01	0\\
160.01	0\\
161.01	0\\
162.01	0\\
163.01	0\\
164.01	0\\
165.01	0\\
166.01	0\\
167.01	0\\
168.01	0\\
169.01	0\\
170.01	0\\
171.01	0\\
172.01	0\\
173.01	0\\
174.01	0\\
175.01	0\\
176.01	0\\
177.01	0\\
178.01	0\\
179.01	0\\
180.01	0\\
181.01	0\\
182.01	0\\
183.01	0\\
184.01	0\\
185.01	0\\
186.01	0\\
187.01	0\\
188.01	0\\
189.01	0\\
190.01	0\\
191.01	0\\
192.01	0\\
193.01	0\\
194.01	0\\
195.01	0\\
196.01	0\\
197.01	0\\
198.01	0\\
199.01	0\\
200.01	0\\
201.01	0\\
202.01	0\\
203.01	0\\
204.01	0\\
205.01	0\\
206.01	0\\
207.01	0\\
208.01	0\\
209.01	0\\
210.01	0\\
211.01	0\\
212.01	0\\
213.01	0\\
214.01	0\\
215.01	0\\
216.01	0\\
217.01	0\\
218.01	0\\
219.01	0\\
220.01	0\\
221.01	0\\
222.01	0\\
223.01	0\\
224.01	0\\
225.01	0\\
226.01	0\\
227.01	0\\
228.01	0\\
229.01	0\\
230.01	0\\
231.01	0\\
232.01	0\\
233.01	0\\
234.01	0\\
235.01	0\\
236.01	0\\
237.01	0\\
238.01	0\\
239.01	0\\
240.01	0\\
241.01	0\\
242.01	0\\
243.01	0\\
244.01	0\\
245.01	0\\
246.01	0\\
247.01	0\\
248.01	0\\
249.01	0\\
250.01	0\\
251.01	0\\
252.01	0\\
253.01	0\\
254.01	0\\
255.01	0\\
256.01	0\\
257.01	0\\
258.01	0\\
259.01	0\\
260.01	0\\
261.01	0\\
262.01	0\\
263.01	0\\
264.01	0\\
265.01	0\\
266.01	0\\
267.01	0\\
268.01	0\\
269.01	0\\
270.01	0\\
271.01	0\\
272.01	0\\
273.01	0\\
274.01	0\\
275.01	0\\
276.01	0\\
277.01	0\\
278.01	0\\
279.01	0\\
280.01	0\\
281.01	0\\
282.01	0\\
283.01	0\\
284.01	0\\
285.01	0\\
286.01	0\\
287.01	0\\
288.01	0\\
289.01	0\\
290.01	0\\
291.01	0\\
292.01	0\\
293.01	0\\
294.01	0\\
295.01	0\\
296.01	0\\
297.01	0\\
298.01	0\\
299.01	0\\
300.01	0\\
301.01	0\\
302.01	0\\
303.01	0\\
304.01	0\\
305.01	0\\
306.01	0\\
307.01	0\\
308.01	0\\
309.01	0\\
310.01	0\\
311.01	0\\
312.01	0\\
313.01	0\\
314.01	0\\
315.01	0\\
316.01	0\\
317.01	0\\
318.01	0\\
319.01	0\\
320.01	0\\
321.01	0\\
322.01	0\\
323.01	0\\
324.01	0\\
325.01	0\\
326.01	0\\
327.01	0\\
328.01	0\\
329.01	0\\
330.01	0\\
331.01	0\\
332.01	0\\
333.01	0\\
334.01	0\\
335.01	0\\
336.01	0\\
337.01	0\\
338.01	0\\
339.01	0\\
340.01	0\\
341.01	0\\
342.01	0\\
343.01	0\\
344.01	0\\
345.01	0\\
346.01	0\\
347.01	0\\
348.01	0\\
349.01	0\\
350.01	0\\
351.01	0\\
352.01	0\\
353.01	0\\
354.01	0\\
355.01	0\\
356.01	0\\
357.01	0\\
358.01	0\\
359.01	0\\
360.01	0\\
361.01	0\\
362.01	0\\
363.01	0\\
364.01	0\\
365.01	0\\
366.01	0\\
367.01	0\\
368.01	0\\
369.01	0\\
370.01	0\\
371.01	0\\
372.01	0\\
373.01	0\\
374.01	0\\
375.01	0\\
376.01	0\\
377.01	0\\
378.01	0\\
379.01	0\\
380.01	0\\
381.01	0\\
382.01	0\\
383.01	0\\
384.01	0\\
385.01	0\\
386.01	0\\
387.01	0\\
388.01	0\\
389.01	0\\
390.01	0\\
391.01	0\\
392.01	0\\
393.01	0\\
394.01	0\\
395.01	0\\
396.01	0\\
397.01	0\\
398.01	0\\
399.01	0\\
400.01	0\\
401.01	0\\
402.01	0\\
403.01	0\\
404.01	0\\
405.01	0\\
406.01	0\\
407.01	0\\
408.01	0\\
409.01	0\\
410.01	0\\
411.01	0\\
412.01	0\\
413.01	0\\
414.01	0\\
415.01	0\\
416.01	0\\
417.01	0\\
418.01	0\\
419.01	0\\
420.01	0\\
421.01	0\\
422.01	0\\
423.01	0\\
424.01	0\\
425.01	0\\
426.01	0\\
427.01	0\\
428.01	0\\
429.01	0\\
430.01	0\\
431.01	0\\
432.01	0\\
433.01	0\\
434.01	0\\
435.01	0\\
436.01	0\\
437.01	0\\
438.01	0\\
439.01	0\\
440.01	0\\
441.01	0\\
442.01	0\\
443.01	0\\
444.01	0\\
445.01	0\\
446.01	0\\
447.01	0\\
448.01	0\\
449.01	0\\
450.01	0\\
451.01	0\\
452.01	0\\
453.01	0\\
454.01	0\\
455.01	0\\
456.01	0\\
457.01	0\\
458.01	0\\
459.01	0\\
460.01	0\\
461.01	0\\
462.01	0\\
463.01	0\\
464.01	0\\
465.01	0\\
466.01	0\\
467.01	1.1329020402559e-06\\
468.01	2.42926371570376e-05\\
469.01	4.81479683581976e-05\\
470.01	7.27151256976269e-05\\
471.01	9.80079158148517e-05\\
472.01	0.000124038584063818\\
473.01	0.000150816227182535\\
474.01	0.000178344407502577\\
475.01	0.000206620404597957\\
476.01	0.000235655901361494\\
477.01	0.000265476405587259\\
478.01	0.000296108967598012\\
479.01	0.000327582024237015\\
480.01	0.000359925502228145\\
481.01	0.000393170931133675\\
482.01	0.000427351566845549\\
483.01	0.000462502526611258\\
484.01	0.000498660936648215\\
485.01	0.000535866093441591\\
486.01	0.000574159639844557\\
487.01	0.000613585757155623\\
488.01	0.000654191375036341\\
489.01	0.000696026407974198\\
490.01	0.000739144056248947\\
491.01	0.000783601041649637\\
492.01	0.000829457753426376\\
493.01	0.000876778638348154\\
494.01	0.000925632540756657\\
495.01	0.000976093074296254\\
496.01	0.00102823898237879\\
497.01	0.00108215450488519\\
498.01	0.00113792974319596\\
499.01	0.00119566100888369\\
500.01	0.00125545111606378\\
501.01	0.00131740941963204\\
502.01	0.00138165037152721\\
503.01	0.00144829225281667\\
504.01	0.00151748425710271\\
505.01	0.00158940412526166\\
506.01	0.00166425130362097\\
507.01	0.00174225050833373\\
508.01	0.00182365606282622\\
509.01	0.00190875711176604\\
510.01	0.00199788390795892\\
511.01	0.00209141541598479\\
512.01	0.002189788535845\\
513.01	0.00229350932384645\\
514.01	0.00240316667304261\\
515.01	0.0025194489415711\\
516.01	0.00264316302433761\\
517.01	0.00277466880452097\\
518.01	0.00291105160323856\\
519.01	0.00305194372531061\\
520.01	0.00319760921929432\\
521.01	0.00334834332533889\\
522.01	0.00348683251894904\\
523.01	0.00356545994298809\\
524.01	0.00364603069415331\\
525.01	0.00372855121145225\\
526.01	0.0038130173250395\\
527.01	0.00389941178465692\\
528.01	0.00398770129590559\\
529.01	0.00407783303432947\\
530.01	0.00416973012800455\\
531.01	0.00426328637259697\\
532.01	0.00435836060467695\\
533.01	0.00445476958820691\\
534.01	0.00455228061471653\\
535.01	0.00465060558407056\\
536.01	0.00474938505180781\\
537.01	0.00484816680571053\\
538.01	0.00494636987829088\\
539.01	0.00504324604238324\\
540.01	0.00513783824991967\\
541.01	0.00522892631659753\\
542.01	0.00531858003379235\\
543.01	0.00541077469880275\\
544.01	0.00550551210428055\\
545.01	0.0056026673440873\\
546.01	0.00570201569717175\\
547.01	0.00580320740074056\\
548.01	0.00590572513662101\\
549.01	0.0060087956203401\\
550.01	0.00611035611335997\\
551.01	0.00621030214124193\\
552.01	0.00630902143547821\\
553.01	0.00640619490681774\\
554.01	0.00650148779051777\\
555.01	0.00659455811486975\\
556.01	0.00668506994227237\\
557.01	0.00677271322077196\\
558.01	0.00685723252977279\\
559.01	0.00693846697486137\\
560.01	0.00701641341732511\\
561.01	0.00709125918268363\\
562.01	0.0071634980266294\\
563.01	0.00723508800698468\\
564.01	0.00730639425146508\\
565.01	0.00737738964046476\\
566.01	0.00744807960110319\\
567.01	0.00751845268206095\\
568.01	0.00758851847485299\\
569.01	0.00765831772946058\\
570.01	0.00772792707527122\\
571.01	0.00779746294572383\\
572.01	0.00786708620901358\\
573.01	0.00793698753582011\\
574.01	0.00800732822400679\\
575.01	0.00807816822687876\\
576.01	0.00814952977669954\\
577.01	0.00822143569195578\\
578.01	0.00829390508513462\\
579.01	0.00836695756900242\\
580.01	0.00844061618955115\\
581.01	0.00851490610843392\\
582.01	0.008589852586896\\
583.01	0.00866547846795157\\
584.01	0.00874180134816968\\
585.01	0.00881883011789336\\
586.01	0.00889656636788261\\
587.01	0.00897501123941693\\
588.01	0.00905415976584804\\
589.01	0.00913400639453102\\
590.01	0.00921454649402135\\
591.01	0.00929577647899807\\
592.01	0.00937769332881967\\
593.01	0.00946029382921912\\
594.01	0.00954357359616978\\
595.01	0.00962752597645235\\
596.01	0.00971214096674654\\
597.01	0.00979740439062092\\
598.01	0.00988329769391883\\
599.01	0.00996873201918203\\
599.02	0.00996938150051711\\
599.03	0.00997000982619995\\
599.04	0.00997061680203572\\
599.05	0.00997120222994154\\
599.06	0.00997176894407421\\
599.07	0.00997231748063729\\
599.08	0.00997284766231445\\
599.09	0.00997337282663754\\
599.1	0.00997389327698541\\
599.11	0.00997440918165476\\
599.12	0.00997492099195222\\
599.13	0.00997542869053242\\
599.14	0.00997593226134337\\
599.15	0.00997643168971941\\
599.16	0.00997692696247924\\
599.17	0.00997741806787515\\
599.18	0.00997790492496225\\
599.19	0.00997838752592332\\
599.2	0.00997886586487288\\
599.21	0.00997933993670913\\
599.22	0.00997980973680635\\
599.23	0.00998027526269526\\
599.24	0.00998073651420421\\
599.25	0.00998119349148986\\
599.26	0.00998164618964857\\
599.27	0.00998209460591699\\
599.28	0.00998253874521366\\
599.29	0.00998297859807655\\
599.3	0.00998341412419777\\
599.31	0.0099838452819738\\
599.32	0.00998427202941182\\
599.33	0.00998469432412354\\
599.34	0.00998511212331871\\
599.35	0.00998552538379823\\
599.36	0.00998593406194686\\
599.37	0.00998633811372554\\
599.38	0.00998673749466321\\
599.39	0.00998713215984817\\
599.4	0.00998752206391895\\
599.41	0.00998790715979874\\
599.42	0.00998828739732171\\
599.43	0.00998866272571617\\
599.44	0.00998903309366109\\
599.45	0.00998939844927219\\
599.46	0.00998975874008727\\
599.47	0.00999011391305059\\
599.48	0.00999046391449633\\
599.49	0.00999080869013592\\
599.5	0.00999114818504226\\
599.51	0.0099914823436306\\
599.52	0.0099918111096529\\
599.53	0.00999213442619165\\
599.54	0.00999245223572696\\
599.55	0.00999276448016512\\
599.56	0.00999307110083833\\
599.57	0.00999337203849892\\
599.58	0.00999366723331349\\
599.59	0.00999395662485697\\
599.6	0.00999424015210655\\
599.61	0.00999451775343565\\
599.62	0.00999478936660834\\
599.63	0.00999505492877341\\
599.64	0.00999531437645813\\
599.65	0.00999556764556198\\
599.66	0.00999581467135026\\
599.67	0.0099960553884477\\
599.68	0.00999628973083186\\
599.69	0.00999651763182663\\
599.7	0.00999673902409553\\
599.71	0.00999695383963498\\
599.72	0.00999716200976746\\
599.73	0.00999736346513468\\
599.74	0.00999755813569058\\
599.75	0.00999774595069428\\
599.76	0.00999792683870299\\
599.77	0.00999810072756478\\
599.78	0.00999826754441137\\
599.79	0.00999842721565074\\
599.8	0.00999857966695976\\
599.81	0.00999872482327668\\
599.82	0.00999886260879365\\
599.83	0.00999899294694907\\
599.84	0.00999911576041993\\
599.85	0.00999923097111411\\
599.86	0.00999933850016262\\
599.87	0.00999943826791175\\
599.88	0.00999953019391525\\
599.89	0.00999961419692642\\
599.9	0.0099996901948902\\
599.91	0.00999975810493522\\
599.92	0.00999981784336591\\
599.93	0.00999986932565446\\
599.94	0.009999912466433\\
599.95	0.00999994717948561\\
599.96	0.00999997337774052\\
599.97	0.00999999097326228\\
599.98	0.00999999987724406\\
599.99	0.01\\
600	0.01\\
};
\addplot [color=mycolor11,solid,forget plot]
  table[row sep=crcr]{%
0.01	0\\
1.01	0\\
2.01	0\\
3.01	0\\
4.01	0\\
5.01	0\\
6.01	0\\
7.01	0\\
8.01	0\\
9.01	0\\
10.01	0\\
11.01	0\\
12.01	0\\
13.01	0\\
14.01	0\\
15.01	0\\
16.01	0\\
17.01	0\\
18.01	0\\
19.01	0\\
20.01	0\\
21.01	0\\
22.01	0\\
23.01	0\\
24.01	0\\
25.01	0\\
26.01	0\\
27.01	0\\
28.01	0\\
29.01	0\\
30.01	0\\
31.01	0\\
32.01	0\\
33.01	0\\
34.01	0\\
35.01	0\\
36.01	0\\
37.01	0\\
38.01	0\\
39.01	0\\
40.01	0\\
41.01	0\\
42.01	0\\
43.01	0\\
44.01	0\\
45.01	0\\
46.01	0\\
47.01	0\\
48.01	0\\
49.01	0\\
50.01	0\\
51.01	0\\
52.01	0\\
53.01	0\\
54.01	0\\
55.01	0\\
56.01	0\\
57.01	0\\
58.01	0\\
59.01	0\\
60.01	0\\
61.01	0\\
62.01	0\\
63.01	0\\
64.01	0\\
65.01	0\\
66.01	0\\
67.01	0\\
68.01	0\\
69.01	0\\
70.01	0\\
71.01	0\\
72.01	0\\
73.01	0\\
74.01	0\\
75.01	0\\
76.01	0\\
77.01	0\\
78.01	0\\
79.01	0\\
80.01	0\\
81.01	0\\
82.01	0\\
83.01	0\\
84.01	0\\
85.01	0\\
86.01	0\\
87.01	0\\
88.01	0\\
89.01	0\\
90.01	0\\
91.01	0\\
92.01	0\\
93.01	0\\
94.01	0\\
95.01	0\\
96.01	0\\
97.01	0\\
98.01	0\\
99.01	0\\
100.01	0\\
101.01	0\\
102.01	0\\
103.01	0\\
104.01	0\\
105.01	0\\
106.01	0\\
107.01	0\\
108.01	0\\
109.01	0\\
110.01	0\\
111.01	0\\
112.01	0\\
113.01	0\\
114.01	0\\
115.01	0\\
116.01	0\\
117.01	0\\
118.01	0\\
119.01	0\\
120.01	0\\
121.01	0\\
122.01	0\\
123.01	0\\
124.01	0\\
125.01	0\\
126.01	0\\
127.01	0\\
128.01	0\\
129.01	0\\
130.01	0\\
131.01	0\\
132.01	0\\
133.01	0\\
134.01	0\\
135.01	0\\
136.01	0\\
137.01	0\\
138.01	0\\
139.01	0\\
140.01	0\\
141.01	0\\
142.01	0\\
143.01	0\\
144.01	0\\
145.01	0\\
146.01	0\\
147.01	0\\
148.01	0\\
149.01	0\\
150.01	0\\
151.01	0\\
152.01	0\\
153.01	0\\
154.01	0\\
155.01	0\\
156.01	0\\
157.01	0\\
158.01	0\\
159.01	0\\
160.01	0\\
161.01	0\\
162.01	0\\
163.01	0\\
164.01	0\\
165.01	0\\
166.01	0\\
167.01	0\\
168.01	0\\
169.01	0\\
170.01	0\\
171.01	0\\
172.01	0\\
173.01	0\\
174.01	0\\
175.01	0\\
176.01	0\\
177.01	0\\
178.01	0\\
179.01	0\\
180.01	0\\
181.01	0\\
182.01	0\\
183.01	0\\
184.01	0\\
185.01	0\\
186.01	0\\
187.01	0\\
188.01	0\\
189.01	0\\
190.01	0\\
191.01	0\\
192.01	0\\
193.01	0\\
194.01	0\\
195.01	0\\
196.01	0\\
197.01	0\\
198.01	0\\
199.01	0\\
200.01	0\\
201.01	0\\
202.01	0\\
203.01	0\\
204.01	0\\
205.01	0\\
206.01	0\\
207.01	0\\
208.01	0\\
209.01	0\\
210.01	0\\
211.01	0\\
212.01	0\\
213.01	0\\
214.01	0\\
215.01	0\\
216.01	0\\
217.01	0\\
218.01	0\\
219.01	0\\
220.01	0\\
221.01	0\\
222.01	0\\
223.01	0\\
224.01	0\\
225.01	0\\
226.01	0\\
227.01	0\\
228.01	0\\
229.01	0\\
230.01	0\\
231.01	0\\
232.01	0\\
233.01	0\\
234.01	0\\
235.01	0\\
236.01	0\\
237.01	0\\
238.01	0\\
239.01	0\\
240.01	0\\
241.01	0\\
242.01	0\\
243.01	0\\
244.01	0\\
245.01	0\\
246.01	0\\
247.01	0\\
248.01	0\\
249.01	0\\
250.01	0\\
251.01	0\\
252.01	0\\
253.01	0\\
254.01	0\\
255.01	0\\
256.01	0\\
257.01	0\\
258.01	0\\
259.01	0\\
260.01	0\\
261.01	0\\
262.01	0\\
263.01	0\\
264.01	0\\
265.01	0\\
266.01	0\\
267.01	0\\
268.01	0\\
269.01	0\\
270.01	0\\
271.01	0\\
272.01	0\\
273.01	0\\
274.01	0\\
275.01	0\\
276.01	0\\
277.01	0\\
278.01	0\\
279.01	0\\
280.01	0\\
281.01	0\\
282.01	0\\
283.01	0\\
284.01	0\\
285.01	0\\
286.01	0\\
287.01	0\\
288.01	0\\
289.01	0\\
290.01	0\\
291.01	0\\
292.01	0\\
293.01	0\\
294.01	0\\
295.01	0\\
296.01	0\\
297.01	0\\
298.01	0\\
299.01	0\\
300.01	0\\
301.01	0\\
302.01	0\\
303.01	0\\
304.01	0\\
305.01	0\\
306.01	0\\
307.01	0\\
308.01	0\\
309.01	0\\
310.01	0\\
311.01	0\\
312.01	0\\
313.01	0\\
314.01	0\\
315.01	0\\
316.01	0\\
317.01	0\\
318.01	0\\
319.01	0\\
320.01	0\\
321.01	0\\
322.01	0\\
323.01	0\\
324.01	0\\
325.01	0\\
326.01	0\\
327.01	0\\
328.01	0\\
329.01	0\\
330.01	0\\
331.01	0\\
332.01	0\\
333.01	0\\
334.01	0\\
335.01	0\\
336.01	0\\
337.01	0\\
338.01	0\\
339.01	0\\
340.01	0\\
341.01	0\\
342.01	0\\
343.01	0\\
344.01	0\\
345.01	0\\
346.01	0\\
347.01	0\\
348.01	0\\
349.01	0\\
350.01	0\\
351.01	0\\
352.01	0\\
353.01	0\\
354.01	0\\
355.01	0\\
356.01	0\\
357.01	0\\
358.01	0\\
359.01	0\\
360.01	0\\
361.01	0\\
362.01	0\\
363.01	0\\
364.01	0\\
365.01	0\\
366.01	0\\
367.01	0\\
368.01	0\\
369.01	0\\
370.01	0\\
371.01	0\\
372.01	0\\
373.01	0\\
374.01	0\\
375.01	0\\
376.01	0\\
377.01	0\\
378.01	0\\
379.01	0\\
380.01	0\\
381.01	0\\
382.01	0\\
383.01	0\\
384.01	0\\
385.01	0\\
386.01	0\\
387.01	0\\
388.01	0\\
389.01	0\\
390.01	0\\
391.01	0\\
392.01	0\\
393.01	0\\
394.01	0\\
395.01	0\\
396.01	0\\
397.01	0\\
398.01	0\\
399.01	0\\
400.01	0\\
401.01	0\\
402.01	0\\
403.01	0\\
404.01	0\\
405.01	0\\
406.01	0\\
407.01	0\\
408.01	0\\
409.01	0\\
410.01	0\\
411.01	0\\
412.01	0\\
413.01	0\\
414.01	0\\
415.01	0\\
416.01	0\\
417.01	0\\
418.01	0\\
419.01	0\\
420.01	0\\
421.01	0\\
422.01	0\\
423.01	0\\
424.01	0\\
425.01	0\\
426.01	0\\
427.01	0\\
428.01	0\\
429.01	0\\
430.01	0\\
431.01	0\\
432.01	0\\
433.01	0\\
434.01	0\\
435.01	0\\
436.01	0\\
437.01	0\\
438.01	0\\
439.01	0\\
440.01	0\\
441.01	0\\
442.01	0\\
443.01	0\\
444.01	0\\
445.01	0\\
446.01	0\\
447.01	0\\
448.01	0\\
449.01	0\\
450.01	0\\
451.01	0\\
452.01	0\\
453.01	0\\
454.01	0\\
455.01	0\\
456.01	0\\
457.01	0\\
458.01	0\\
459.01	0\\
460.01	0\\
461.01	0\\
462.01	0\\
463.01	0\\
464.01	0\\
465.01	0\\
466.01	0\\
467.01	0\\
468.01	0\\
469.01	0\\
470.01	0\\
471.01	0\\
472.01	0\\
473.01	0\\
474.01	0\\
475.01	0\\
476.01	0\\
477.01	0\\
478.01	0\\
479.01	0\\
480.01	0\\
481.01	0\\
482.01	0\\
483.01	0\\
484.01	0\\
485.01	0\\
486.01	0\\
487.01	0\\
488.01	0\\
489.01	0\\
490.01	0\\
491.01	0\\
492.01	0\\
493.01	0\\
494.01	0\\
495.01	0\\
496.01	0\\
497.01	0\\
498.01	0\\
499.01	0\\
500.01	0\\
501.01	0\\
502.01	0\\
503.01	0\\
504.01	0\\
505.01	0\\
506.01	0\\
507.01	0\\
508.01	0\\
509.01	0\\
510.01	0\\
511.01	0\\
512.01	0\\
513.01	0\\
514.01	0\\
515.01	0\\
516.01	0\\
517.01	0\\
518.01	0\\
519.01	0\\
520.01	0\\
521.01	0\\
522.01	1.76072062530383e-05\\
523.01	0.0001000531427582\\
524.01	0.000185428064959862\\
525.01	0.000273915626818395\\
526.01	0.000365718517235863\\
527.01	0.000461061166109798\\
528.01	0.000560192925065827\\
529.01	0.000663391817041978\\
530.01	0.000770968941636186\\
531.01	0.000883273738164617\\
532.01	0.00100070026104646\\
533.01	0.00112369466667597\\
534.01	0.00125276429462983\\
535.01	0.00138848841820502\\
536.01	0.00153153040644095\\
537.01	0.00168265200432605\\
538.01	0.00184273081035254\\
539.01	0.00201279319736091\\
540.01	0.00219403790053492\\
541.01	0.00238785532069443\\
542.01	0.00259226846870504\\
543.01	0.00280354376610866\\
544.01	0.0030219682635569\\
545.01	0.00324793355738788\\
546.01	0.00348188162292224\\
547.01	0.00372430787917213\\
548.01	0.00397577250198962\\
549.01	0.0042369103544857\\
550.01	0.00449075671486131\\
551.01	0.00462618461453768\\
552.01	0.00476421212870169\\
553.01	0.00490458243399326\\
554.01	0.00504693095053299\\
555.01	0.00519075452171885\\
556.01	0.00533537205239804\\
557.01	0.00547987416083386\\
558.01	0.00562305849481418\\
559.01	0.0057633450823129\\
560.01	0.00589868845148751\\
561.01	0.00603278110898117\\
562.01	0.00616897852510308\\
563.01	0.00630509330493087\\
564.01	0.00644045192015123\\
565.01	0.00657462204057506\\
566.01	0.00670711983291309\\
567.01	0.00683740812363565\\
568.01	0.0069648990353388\\
569.01	0.00708895843143271\\
570.01	0.00720891484742703\\
571.01	0.00732402511855551\\
572.01	0.00743343423629905\\
573.01	0.00753834307265663\\
574.01	0.00764066192485596\\
575.01	0.00774196963664896\\
576.01	0.00784232616627326\\
577.01	0.00794188335541293\\
578.01	0.00804077814675698\\
579.01	0.00813893755468454\\
580.01	0.00823628502515087\\
581.01	0.00833276801027717\\
582.01	0.00842836104623377\\
583.01	0.0085230664726219\\
584.01	0.00861691980849836\\
585.01	0.00871008730799935\\
586.01	0.00880256534465678\\
587.01	0.00889458764294722\\
588.01	0.00898632359220033\\
589.01	0.00907770521085289\\
590.01	0.00916865124237691\\
591.01	0.00925908767877544\\
592.01	0.00934895807582376\\
593.01	0.00943823031869177\\
594.01	0.00952690140896054\\
595.01	0.00961500159478332\\
596.01	0.00970259742227558\\
597.01	0.00978979309540728\\
598.01	0.00987673031330844\\
599.01	0.00996359465763311\\
599.02	0.00996446178033267\\
599.03	0.00996532763967337\\
599.04	0.00996619223192198\\
599.05	0.00996705555325088\\
599.06	0.00996791457106968\\
599.07	0.00996876854874041\\
599.08	0.00996961745928199\\
599.09	0.00997044779379745\\
599.1	0.00997125907535553\\
599.11	0.00997205095980902\\
599.12	0.00997282281992336\\
599.13	0.00997357449570154\\
599.14	0.00997430582442724\\
599.15	0.00997501664056067\\
599.16	0.00997570677562952\\
599.17	0.00997637605811486\\
599.18	0.00997702431333591\\
599.19	0.00997765136331818\\
599.2	0.00997825702666225\\
599.21	0.00997884164734969\\
599.22	0.00997940571276279\\
599.23	0.00997994904050891\\
599.24	0.00998047144445088\\
599.25	0.00998097350251616\\
599.26	0.00998145771758323\\
599.27	0.00998192391942239\\
599.28	0.00998238161802606\\
599.29	0.00998283374884875\\
599.3	0.00998328059264519\\
599.31	0.009983722405159\\
599.32	0.00998415915282055\\
599.33	0.0099845908027081\\
599.34	0.00998501732260721\\
599.35	0.00998543868107325\\
599.36	0.00998585484749724\\
599.37	0.00998626579217519\\
599.38	0.00998667148638112\\
599.39	0.00998707190244393\\
599.4	0.00998746701382841\\
599.41	0.00998785679522082\\
599.42	0.0099882412571037\\
599.43	0.00998862039174518\\
599.44	0.00998899417396288\\
599.45	0.00998936258018044\\
599.46	0.00998972558854081\\
599.47	0.00999008317902579\\
599.48	0.00999043533358202\\
599.49	0.00999078203429291\\
599.5	0.00999112326444559\\
599.51	0.00999145900953319\\
599.52	0.00999178925246696\\
599.53	0.00999211397373816\\
599.54	0.00999243312917881\\
599.55	0.00999274666306492\\
599.56	0.00999305451751694\\
599.57	0.00999335663412312\\
599.58	0.00999365295393464\\
599.59	0.00999394341746056\\
599.6	0.00999422796466266\\
599.61	0.00999450653495009\\
599.62	0.00999477906702091\\
599.63	0.00999504549894827\\
599.64	0.0099953057682215\\
599.65	0.00999555981174296\\
599.66	0.0099958075658224\\
599.67	0.00999604896616963\\
599.68	0.00999628394788682\\
599.69	0.00999651244546047\\
599.7	0.00999673439275298\\
599.71	0.0099969497229938\\
599.72	0.00999715836877016\\
599.73	0.00999736026201729\\
599.74	0.00999755533403083\\
599.75	0.00999774351546318\\
599.76	0.00999792473631449\\
599.77	0.00999809892592316\\
599.78	0.00999826601295585\\
599.79	0.00999842592539837\\
599.8	0.00999857859054469\\
599.81	0.00999872393498536\\
599.82	0.00999886188459496\\
599.83	0.00999899236451894\\
599.84	0.0099991152991595\\
599.85	0.00999923061216063\\
599.86	0.00999933822639223\\
599.87	0.00999943806393315\\
599.88	0.00999953004605316\\
599.89	0.00999961409319372\\
599.9	0.00999969012494744\\
599.91	0.00999975806003624\\
599.92	0.0099998178162879\\
599.93	0.00999986931061113\\
599.94	0.00999991245896887\\
599.95	0.00999994717634967\\
599.96	0.00999997337673722\\
599.97	0.00999999097307755\\
599.98	0.00999999987724406\\
599.99	0.01\\
600	0.01\\
};
\addplot [color=mycolor12,solid,forget plot]
  table[row sep=crcr]{%
0.01	0\\
1.01	0\\
2.01	0\\
3.01	0\\
4.01	0\\
5.01	0\\
6.01	0\\
7.01	0\\
8.01	0\\
9.01	0\\
10.01	0\\
11.01	0\\
12.01	0\\
13.01	0\\
14.01	0\\
15.01	0\\
16.01	0\\
17.01	0\\
18.01	0\\
19.01	0\\
20.01	0\\
21.01	0\\
22.01	0\\
23.01	0\\
24.01	0\\
25.01	0\\
26.01	0\\
27.01	0\\
28.01	0\\
29.01	0\\
30.01	0\\
31.01	0\\
32.01	0\\
33.01	0\\
34.01	0\\
35.01	0\\
36.01	0\\
37.01	0\\
38.01	0\\
39.01	0\\
40.01	0\\
41.01	0\\
42.01	0\\
43.01	0\\
44.01	0\\
45.01	0\\
46.01	0\\
47.01	0\\
48.01	0\\
49.01	0\\
50.01	0\\
51.01	0\\
52.01	0\\
53.01	0\\
54.01	0\\
55.01	0\\
56.01	0\\
57.01	0\\
58.01	0\\
59.01	0\\
60.01	0\\
61.01	0\\
62.01	0\\
63.01	0\\
64.01	0\\
65.01	0\\
66.01	0\\
67.01	0\\
68.01	0\\
69.01	0\\
70.01	0\\
71.01	0\\
72.01	0\\
73.01	0\\
74.01	0\\
75.01	0\\
76.01	0\\
77.01	0\\
78.01	0\\
79.01	0\\
80.01	0\\
81.01	0\\
82.01	0\\
83.01	0\\
84.01	0\\
85.01	0\\
86.01	0\\
87.01	0\\
88.01	0\\
89.01	0\\
90.01	0\\
91.01	0\\
92.01	0\\
93.01	0\\
94.01	0\\
95.01	0\\
96.01	0\\
97.01	0\\
98.01	0\\
99.01	0\\
100.01	0\\
101.01	0\\
102.01	0\\
103.01	0\\
104.01	0\\
105.01	0\\
106.01	0\\
107.01	0\\
108.01	0\\
109.01	0\\
110.01	0\\
111.01	0\\
112.01	0\\
113.01	0\\
114.01	0\\
115.01	0\\
116.01	0\\
117.01	0\\
118.01	0\\
119.01	0\\
120.01	0\\
121.01	0\\
122.01	0\\
123.01	0\\
124.01	0\\
125.01	0\\
126.01	0\\
127.01	0\\
128.01	0\\
129.01	0\\
130.01	0\\
131.01	0\\
132.01	0\\
133.01	0\\
134.01	0\\
135.01	0\\
136.01	0\\
137.01	0\\
138.01	0\\
139.01	0\\
140.01	0\\
141.01	0\\
142.01	0\\
143.01	0\\
144.01	0\\
145.01	0\\
146.01	0\\
147.01	0\\
148.01	0\\
149.01	0\\
150.01	0\\
151.01	0\\
152.01	0\\
153.01	0\\
154.01	0\\
155.01	0\\
156.01	0\\
157.01	0\\
158.01	0\\
159.01	0\\
160.01	0\\
161.01	0\\
162.01	0\\
163.01	0\\
164.01	0\\
165.01	0\\
166.01	0\\
167.01	0\\
168.01	0\\
169.01	0\\
170.01	0\\
171.01	0\\
172.01	0\\
173.01	0\\
174.01	0\\
175.01	0\\
176.01	0\\
177.01	0\\
178.01	0\\
179.01	0\\
180.01	0\\
181.01	0\\
182.01	0\\
183.01	0\\
184.01	0\\
185.01	0\\
186.01	0\\
187.01	0\\
188.01	0\\
189.01	0\\
190.01	0\\
191.01	0\\
192.01	0\\
193.01	0\\
194.01	0\\
195.01	0\\
196.01	0\\
197.01	0\\
198.01	0\\
199.01	0\\
200.01	0\\
201.01	0\\
202.01	0\\
203.01	0\\
204.01	0\\
205.01	0\\
206.01	0\\
207.01	0\\
208.01	0\\
209.01	0\\
210.01	0\\
211.01	0\\
212.01	0\\
213.01	0\\
214.01	0\\
215.01	0\\
216.01	0\\
217.01	0\\
218.01	0\\
219.01	0\\
220.01	0\\
221.01	0\\
222.01	0\\
223.01	0\\
224.01	0\\
225.01	0\\
226.01	0\\
227.01	0\\
228.01	0\\
229.01	0\\
230.01	0\\
231.01	0\\
232.01	0\\
233.01	0\\
234.01	0\\
235.01	0\\
236.01	0\\
237.01	0\\
238.01	0\\
239.01	0\\
240.01	0\\
241.01	0\\
242.01	0\\
243.01	0\\
244.01	0\\
245.01	0\\
246.01	0\\
247.01	0\\
248.01	0\\
249.01	0\\
250.01	0\\
251.01	0\\
252.01	0\\
253.01	0\\
254.01	0\\
255.01	0\\
256.01	0\\
257.01	0\\
258.01	0\\
259.01	0\\
260.01	0\\
261.01	0\\
262.01	0\\
263.01	0\\
264.01	0\\
265.01	0\\
266.01	0\\
267.01	0\\
268.01	0\\
269.01	0\\
270.01	0\\
271.01	0\\
272.01	0\\
273.01	0\\
274.01	0\\
275.01	0\\
276.01	0\\
277.01	0\\
278.01	0\\
279.01	0\\
280.01	0\\
281.01	0\\
282.01	0\\
283.01	0\\
284.01	0\\
285.01	0\\
286.01	0\\
287.01	0\\
288.01	0\\
289.01	0\\
290.01	0\\
291.01	0\\
292.01	0\\
293.01	0\\
294.01	0\\
295.01	0\\
296.01	0\\
297.01	0\\
298.01	0\\
299.01	0\\
300.01	0\\
301.01	0\\
302.01	0\\
303.01	0\\
304.01	0\\
305.01	0\\
306.01	0\\
307.01	0\\
308.01	0\\
309.01	0\\
310.01	0\\
311.01	0\\
312.01	0\\
313.01	0\\
314.01	0\\
315.01	0\\
316.01	0\\
317.01	0\\
318.01	0\\
319.01	0\\
320.01	0\\
321.01	0\\
322.01	0\\
323.01	0\\
324.01	0\\
325.01	0\\
326.01	0\\
327.01	0\\
328.01	0\\
329.01	0\\
330.01	0\\
331.01	0\\
332.01	0\\
333.01	0\\
334.01	0\\
335.01	0\\
336.01	0\\
337.01	0\\
338.01	0\\
339.01	0\\
340.01	0\\
341.01	0\\
342.01	0\\
343.01	0\\
344.01	0\\
345.01	0\\
346.01	0\\
347.01	0\\
348.01	0\\
349.01	0\\
350.01	0\\
351.01	0\\
352.01	0\\
353.01	0\\
354.01	0\\
355.01	0\\
356.01	0\\
357.01	0\\
358.01	0\\
359.01	0\\
360.01	0\\
361.01	0\\
362.01	0\\
363.01	0\\
364.01	0\\
365.01	0\\
366.01	0\\
367.01	0\\
368.01	0\\
369.01	0\\
370.01	0\\
371.01	0\\
372.01	0\\
373.01	0\\
374.01	0\\
375.01	0\\
376.01	0\\
377.01	0\\
378.01	0\\
379.01	0\\
380.01	0\\
381.01	0\\
382.01	0\\
383.01	0\\
384.01	0\\
385.01	0\\
386.01	0\\
387.01	0\\
388.01	0\\
389.01	0\\
390.01	0\\
391.01	0\\
392.01	0\\
393.01	0\\
394.01	0\\
395.01	0\\
396.01	0\\
397.01	0\\
398.01	0\\
399.01	0\\
400.01	0\\
401.01	0\\
402.01	0\\
403.01	0\\
404.01	0\\
405.01	0\\
406.01	0\\
407.01	0\\
408.01	0\\
409.01	0\\
410.01	0\\
411.01	0\\
412.01	0\\
413.01	0\\
414.01	0\\
415.01	0\\
416.01	0\\
417.01	0\\
418.01	0\\
419.01	0\\
420.01	0\\
421.01	0\\
422.01	0\\
423.01	0\\
424.01	0\\
425.01	0\\
426.01	0\\
427.01	0\\
428.01	0\\
429.01	0\\
430.01	0\\
431.01	0\\
432.01	0\\
433.01	0\\
434.01	0\\
435.01	0\\
436.01	0\\
437.01	0\\
438.01	0\\
439.01	0\\
440.01	0\\
441.01	0\\
442.01	0\\
443.01	0\\
444.01	0\\
445.01	0\\
446.01	0\\
447.01	0\\
448.01	0\\
449.01	0\\
450.01	0\\
451.01	0\\
452.01	0\\
453.01	0\\
454.01	0\\
455.01	0\\
456.01	0\\
457.01	0\\
458.01	0\\
459.01	0\\
460.01	0\\
461.01	0\\
462.01	0\\
463.01	0\\
464.01	0\\
465.01	0\\
466.01	0\\
467.01	0\\
468.01	0\\
469.01	0\\
470.01	0\\
471.01	0\\
472.01	0\\
473.01	0\\
474.01	0\\
475.01	0\\
476.01	0\\
477.01	0\\
478.01	0\\
479.01	0\\
480.01	0\\
481.01	0\\
482.01	0\\
483.01	0\\
484.01	0\\
485.01	0\\
486.01	0\\
487.01	0\\
488.01	0\\
489.01	0\\
490.01	0\\
491.01	0\\
492.01	0\\
493.01	0\\
494.01	0\\
495.01	0\\
496.01	0\\
497.01	0\\
498.01	0\\
499.01	0\\
500.01	0\\
501.01	0\\
502.01	0\\
503.01	0\\
504.01	0\\
505.01	0\\
506.01	0\\
507.01	0\\
508.01	0\\
509.01	0\\
510.01	0\\
511.01	0\\
512.01	0\\
513.01	0\\
514.01	0\\
515.01	0\\
516.01	0\\
517.01	0\\
518.01	0\\
519.01	0\\
520.01	0\\
521.01	0\\
522.01	0\\
523.01	0\\
524.01	0\\
525.01	0\\
526.01	0\\
527.01	0\\
528.01	0\\
529.01	0\\
530.01	0\\
531.01	0\\
532.01	0\\
533.01	0\\
534.01	0\\
535.01	0\\
536.01	0\\
537.01	0\\
538.01	0\\
539.01	0\\
540.01	0\\
541.01	0\\
542.01	0\\
543.01	0\\
544.01	0\\
545.01	0\\
546.01	0\\
547.01	0\\
548.01	0\\
549.01	0\\
550.01	1.76577693769484e-05\\
551.01	0.000163601594765743\\
552.01	0.000316552654936606\\
553.01	0.000477202157931674\\
554.01	0.000646343853199616\\
555.01	0.000824893090787756\\
556.01	0.00101390996811625\\
557.01	0.0012146275223706\\
558.01	0.00142848601941258\\
559.01	0.00165717309688124\\
560.01	0.00190263038585375\\
561.01	0.00216086094684066\\
562.01	0.00242801592023746\\
563.01	0.00270459001499955\\
564.01	0.00299112571883046\\
565.01	0.00328819303298395\\
566.01	0.00359638629634986\\
567.01	0.00391632018100313\\
568.01	0.00424862352259442\\
569.01	0.0045939301115959\\
570.01	0.00495286578552835\\
571.01	0.00532608000731284\\
572.01	0.00568646434170741\\
573.01	0.00588160416759255\\
574.01	0.00607494172253639\\
575.01	0.00626304604111949\\
576.01	0.00644372525962685\\
577.01	0.0066149467502195\\
578.01	0.00678349529972556\\
579.01	0.00695169599783461\\
580.01	0.00711915538111959\\
581.01	0.00728547230152028\\
582.01	0.00745025912556778\\
583.01	0.00761317424151823\\
584.01	0.00777397103082797\\
585.01	0.00793312457297383\\
586.01	0.00809207708383814\\
587.01	0.00825067850764279\\
588.01	0.0084088581555231\\
589.01	0.00856615818895188\\
590.01	0.00872204584445263\\
591.01	0.00887592796155797\\
592.01	0.00902715589244679\\
593.01	0.00917502985911509\\
594.01	0.00931880522180199\\
595.01	0.0094577020919197\\
596.01	0.00959091850522286\\
597.01	0.00971763773133404\\
598.01	0.00983693415442488\\
599.01	0.00994672806842535\\
599.02	0.00994775855856473\\
599.03	0.00994878730830944\\
599.04	0.00994981430099912\\
599.05	0.0099508395197916\\
599.06	0.00995186294794709\\
599.07	0.00995288456862204\\
599.08	0.00995390436481037\\
599.09	0.00995492232059677\\
599.1	0.00995593841996281\\
599.11	0.00995695264678405\\
599.12	0.00995796498486604\\
599.13	0.00995897541791202\\
599.14	0.00995998392953529\\
599.15	0.0099609905032724\\
599.16	0.00996199512259775\\
599.17	0.00996299777093928\\
599.18	0.00996399843169559\\
599.19	0.00996499708825443\\
599.2	0.00996599372401278\\
599.21	0.00996698779596926\\
599.22	0.00996797861736903\\
599.23	0.00996896616765423\\
599.24	0.00996995042631169\\
599.25	0.00997093060729154\\
599.26	0.00997190400413337\\
599.27	0.00997287058004875\\
599.28	0.00997382061001036\\
599.29	0.00997475095702614\\
599.3	0.00997566116920662\\
599.31	0.00997655082052863\\
599.32	0.0099774197729459\\
599.33	0.00997826788498202\\
599.34	0.00997909501386427\\
599.35	0.00997990101551993\\
599.36	0.00998068574457414\\
599.37	0.00998144905434973\\
599.38	0.00998219079686904\\
599.39	0.00998291082285798\\
599.4	0.0099836089817526\\
599.41	0.00998428512170819\\
599.42	0.00998493905288865\\
599.43	0.00998557060322691\\
599.44	0.00998617961841143\\
599.45	0.00998676594296101\\
599.46	0.0099873294202526\\
599.47	0.00998786989255357\\
599.48	0.00998838720105844\\
599.49	0.00998888190498132\\
599.5	0.00998935417965473\\
599.51	0.00998980387068191\\
599.52	0.0099902324022735\\
599.53	0.00999064107301789\\
599.54	0.00999103756789193\\
599.55	0.0099914252526318\\
599.56	0.0099918045563071\\
599.57	0.00999217544917753\\
599.58	0.00999253790543692\\
599.59	0.00999289190353504\\
599.6	0.00999323742652058\\
599.61	0.00999357446240663\\
599.62	0.00999390303937479\\
599.63	0.00999422316746411\\
599.64	0.00999453485133593\\
599.65	0.00999483808574109\\
599.66	0.00999513286499583\\
599.67	0.00999541918934847\\
599.68	0.00999569706543861\\
599.69	0.00999596650678687\\
599.7	0.009996227534317\\
599.71	0.00999648017691256\\
599.72	0.00999672447201024\\
599.73	0.0099969604662323\\
599.74	0.00999718812015448\\
599.75	0.00999740737458996\\
599.76	0.00999761817122877\\
599.77	0.00999782045273265\\
599.78	0.00999801416283588\\
599.79	0.00999819924009764\\
599.8	0.00999837562306064\\
599.81	0.00999854325056222\\
599.82	0.00999870206199949\\
599.83	0.00999885199763102\\
599.84	0.00999899299866258\\
599.85	0.00999912500733794\\
599.86	0.009999247967035\\
599.87	0.0099993618223675\\
599.88	0.00999946651929288\\
599.89	0.0099995620052265\\
599.9	0.00999964822916273\\
599.91	0.00999972514180333\\
599.92	0.00999979269569369\\
599.93	0.00999985084536732\\
599.94	0.00999989954749936\\
599.95	0.00999993876106953\\
599.96	0.00999996844753539\\
599.97	0.00999998857101644\\
599.98	0.00999999909849005\\
599.99	0.01\\
600	0.01\\
};
\addplot [color=mycolor13,solid,forget plot]
  table[row sep=crcr]{%
0.01	0\\
1.01	0\\
2.01	0\\
3.01	0\\
4.01	0\\
5.01	0\\
6.01	0\\
7.01	0\\
8.01	0\\
9.01	0\\
10.01	0\\
11.01	0\\
12.01	0\\
13.01	0\\
14.01	0\\
15.01	0\\
16.01	0\\
17.01	0\\
18.01	0\\
19.01	0\\
20.01	0\\
21.01	0\\
22.01	0\\
23.01	0\\
24.01	0\\
25.01	0\\
26.01	0\\
27.01	0\\
28.01	0\\
29.01	0\\
30.01	0\\
31.01	0\\
32.01	0\\
33.01	0\\
34.01	0\\
35.01	0\\
36.01	0\\
37.01	0\\
38.01	0\\
39.01	0\\
40.01	0\\
41.01	0\\
42.01	0\\
43.01	0\\
44.01	0\\
45.01	0\\
46.01	0\\
47.01	0\\
48.01	0\\
49.01	0\\
50.01	0\\
51.01	0\\
52.01	0\\
53.01	0\\
54.01	0\\
55.01	0\\
56.01	0\\
57.01	0\\
58.01	0\\
59.01	0\\
60.01	0\\
61.01	0\\
62.01	0\\
63.01	0\\
64.01	0\\
65.01	0\\
66.01	0\\
67.01	0\\
68.01	0\\
69.01	0\\
70.01	0\\
71.01	0\\
72.01	0\\
73.01	0\\
74.01	0\\
75.01	0\\
76.01	0\\
77.01	0\\
78.01	0\\
79.01	0\\
80.01	0\\
81.01	0\\
82.01	0\\
83.01	0\\
84.01	0\\
85.01	0\\
86.01	0\\
87.01	0\\
88.01	0\\
89.01	0\\
90.01	0\\
91.01	0\\
92.01	0\\
93.01	0\\
94.01	0\\
95.01	0\\
96.01	0\\
97.01	0\\
98.01	0\\
99.01	0\\
100.01	0\\
101.01	0\\
102.01	0\\
103.01	0\\
104.01	0\\
105.01	0\\
106.01	0\\
107.01	0\\
108.01	0\\
109.01	0\\
110.01	0\\
111.01	0\\
112.01	0\\
113.01	0\\
114.01	0\\
115.01	0\\
116.01	0\\
117.01	0\\
118.01	0\\
119.01	0\\
120.01	0\\
121.01	0\\
122.01	0\\
123.01	0\\
124.01	0\\
125.01	0\\
126.01	0\\
127.01	0\\
128.01	0\\
129.01	0\\
130.01	0\\
131.01	0\\
132.01	0\\
133.01	0\\
134.01	0\\
135.01	0\\
136.01	0\\
137.01	0\\
138.01	0\\
139.01	0\\
140.01	0\\
141.01	0\\
142.01	0\\
143.01	0\\
144.01	0\\
145.01	0\\
146.01	0\\
147.01	0\\
148.01	0\\
149.01	0\\
150.01	0\\
151.01	0\\
152.01	0\\
153.01	0\\
154.01	0\\
155.01	0\\
156.01	0\\
157.01	0\\
158.01	0\\
159.01	0\\
160.01	0\\
161.01	0\\
162.01	0\\
163.01	0\\
164.01	0\\
165.01	0\\
166.01	0\\
167.01	0\\
168.01	0\\
169.01	0\\
170.01	0\\
171.01	0\\
172.01	0\\
173.01	0\\
174.01	0\\
175.01	0\\
176.01	0\\
177.01	0\\
178.01	0\\
179.01	0\\
180.01	0\\
181.01	0\\
182.01	0\\
183.01	0\\
184.01	0\\
185.01	0\\
186.01	0\\
187.01	0\\
188.01	0\\
189.01	0\\
190.01	0\\
191.01	0\\
192.01	0\\
193.01	0\\
194.01	0\\
195.01	0\\
196.01	0\\
197.01	0\\
198.01	0\\
199.01	0\\
200.01	0\\
201.01	0\\
202.01	0\\
203.01	0\\
204.01	0\\
205.01	0\\
206.01	0\\
207.01	0\\
208.01	0\\
209.01	0\\
210.01	0\\
211.01	0\\
212.01	0\\
213.01	0\\
214.01	0\\
215.01	0\\
216.01	0\\
217.01	0\\
218.01	0\\
219.01	0\\
220.01	0\\
221.01	0\\
222.01	0\\
223.01	0\\
224.01	0\\
225.01	0\\
226.01	0\\
227.01	0\\
228.01	0\\
229.01	0\\
230.01	0\\
231.01	0\\
232.01	0\\
233.01	0\\
234.01	0\\
235.01	0\\
236.01	0\\
237.01	0\\
238.01	0\\
239.01	0\\
240.01	0\\
241.01	0\\
242.01	0\\
243.01	0\\
244.01	0\\
245.01	0\\
246.01	0\\
247.01	0\\
248.01	0\\
249.01	0\\
250.01	0\\
251.01	0\\
252.01	0\\
253.01	0\\
254.01	0\\
255.01	0\\
256.01	0\\
257.01	0\\
258.01	0\\
259.01	0\\
260.01	0\\
261.01	0\\
262.01	0\\
263.01	0\\
264.01	0\\
265.01	0\\
266.01	0\\
267.01	0\\
268.01	0\\
269.01	0\\
270.01	0\\
271.01	0\\
272.01	0\\
273.01	0\\
274.01	0\\
275.01	0\\
276.01	0\\
277.01	0\\
278.01	0\\
279.01	0\\
280.01	0\\
281.01	0\\
282.01	0\\
283.01	0\\
284.01	0\\
285.01	0\\
286.01	0\\
287.01	0\\
288.01	0\\
289.01	0\\
290.01	0\\
291.01	0\\
292.01	0\\
293.01	0\\
294.01	0\\
295.01	0\\
296.01	0\\
297.01	0\\
298.01	0\\
299.01	0\\
300.01	0\\
301.01	0\\
302.01	0\\
303.01	0\\
304.01	0\\
305.01	0\\
306.01	0\\
307.01	0\\
308.01	0\\
309.01	0\\
310.01	0\\
311.01	0\\
312.01	0\\
313.01	0\\
314.01	0\\
315.01	0\\
316.01	0\\
317.01	0\\
318.01	0\\
319.01	0\\
320.01	0\\
321.01	0\\
322.01	0\\
323.01	0\\
324.01	0\\
325.01	0\\
326.01	0\\
327.01	0\\
328.01	0\\
329.01	0\\
330.01	0\\
331.01	0\\
332.01	0\\
333.01	0\\
334.01	0\\
335.01	0\\
336.01	0\\
337.01	0\\
338.01	0\\
339.01	0\\
340.01	0\\
341.01	0\\
342.01	0\\
343.01	0\\
344.01	0\\
345.01	0\\
346.01	0\\
347.01	0\\
348.01	0\\
349.01	0\\
350.01	0\\
351.01	0\\
352.01	0\\
353.01	0\\
354.01	0\\
355.01	0\\
356.01	0\\
357.01	0\\
358.01	0\\
359.01	0\\
360.01	0\\
361.01	0\\
362.01	0\\
363.01	0\\
364.01	0\\
365.01	0\\
366.01	0\\
367.01	0\\
368.01	0\\
369.01	0\\
370.01	0\\
371.01	0\\
372.01	0\\
373.01	0\\
374.01	0\\
375.01	0\\
376.01	0\\
377.01	0\\
378.01	0\\
379.01	0\\
380.01	0\\
381.01	0\\
382.01	0\\
383.01	0\\
384.01	0\\
385.01	0\\
386.01	0\\
387.01	0\\
388.01	0\\
389.01	0\\
390.01	0\\
391.01	0\\
392.01	0\\
393.01	0\\
394.01	0\\
395.01	0\\
396.01	0\\
397.01	0\\
398.01	0\\
399.01	0\\
400.01	0\\
401.01	0\\
402.01	0\\
403.01	0\\
404.01	0\\
405.01	0\\
406.01	0\\
407.01	0\\
408.01	0\\
409.01	0\\
410.01	0\\
411.01	0\\
412.01	0\\
413.01	0\\
414.01	0\\
415.01	0\\
416.01	0\\
417.01	0\\
418.01	0\\
419.01	0\\
420.01	0\\
421.01	0\\
422.01	0\\
423.01	0\\
424.01	0\\
425.01	0\\
426.01	0\\
427.01	0\\
428.01	0\\
429.01	0\\
430.01	0\\
431.01	0\\
432.01	0\\
433.01	0\\
434.01	0\\
435.01	0\\
436.01	0\\
437.01	0\\
438.01	0\\
439.01	0\\
440.01	0\\
441.01	0\\
442.01	0\\
443.01	0\\
444.01	0\\
445.01	0\\
446.01	0\\
447.01	0\\
448.01	0\\
449.01	0\\
450.01	0\\
451.01	0\\
452.01	0\\
453.01	0\\
454.01	0\\
455.01	0\\
456.01	0\\
457.01	0\\
458.01	0\\
459.01	0\\
460.01	0\\
461.01	0\\
462.01	0\\
463.01	0\\
464.01	0\\
465.01	0\\
466.01	0\\
467.01	0\\
468.01	0\\
469.01	0\\
470.01	0\\
471.01	0\\
472.01	0\\
473.01	0\\
474.01	0\\
475.01	0\\
476.01	0\\
477.01	0\\
478.01	0\\
479.01	0\\
480.01	0\\
481.01	0\\
482.01	0\\
483.01	0\\
484.01	0\\
485.01	0\\
486.01	0\\
487.01	0\\
488.01	0\\
489.01	0\\
490.01	0\\
491.01	0\\
492.01	0\\
493.01	0\\
494.01	0\\
495.01	0\\
496.01	0\\
497.01	0\\
498.01	0\\
499.01	0\\
500.01	0\\
501.01	0\\
502.01	0\\
503.01	0\\
504.01	0\\
505.01	0\\
506.01	0\\
507.01	0\\
508.01	0\\
509.01	0\\
510.01	0\\
511.01	0\\
512.01	0\\
513.01	0\\
514.01	0\\
515.01	0\\
516.01	0\\
517.01	0\\
518.01	0\\
519.01	0\\
520.01	0\\
521.01	0\\
522.01	0\\
523.01	0\\
524.01	0\\
525.01	0\\
526.01	0\\
527.01	0\\
528.01	0\\
529.01	0\\
530.01	0\\
531.01	0\\
532.01	0\\
533.01	0\\
534.01	0\\
535.01	0\\
536.01	0\\
537.01	0\\
538.01	0\\
539.01	0\\
540.01	0\\
541.01	0\\
542.01	0\\
543.01	0\\
544.01	0\\
545.01	0\\
546.01	0\\
547.01	0\\
548.01	0\\
549.01	0\\
550.01	0\\
551.01	0\\
552.01	0\\
553.01	0\\
554.01	0\\
555.01	0\\
556.01	0\\
557.01	0\\
558.01	0\\
559.01	0\\
560.01	0\\
561.01	0\\
562.01	0\\
563.01	0\\
564.01	0\\
565.01	0\\
566.01	0\\
567.01	0\\
568.01	0\\
569.01	0\\
570.01	0\\
571.01	0\\
572.01	2.7816904777541e-05\\
573.01	0.000234469092053179\\
574.01	0.000454399221483506\\
575.01	0.000689399833623931\\
576.01	0.0009415921916204\\
577.01	0.00121282930990188\\
578.01	0.00149614830095804\\
579.01	0.00178923528638872\\
580.01	0.00209247197993115\\
581.01	0.00240619622796399\\
582.01	0.00273067721320507\\
583.01	0.00306608229716627\\
584.01	0.00341243284835259\\
585.01	0.00376898156286783\\
586.01	0.00413388772680092\\
587.01	0.00450662171783373\\
588.01	0.00488671348865318\\
589.01	0.00527426155529603\\
590.01	0.0056693963527141\\
591.01	0.00607223737509095\\
592.01	0.00648288259282408\\
593.01	0.00690139908379193\\
594.01	0.00732781618824651\\
595.01	0.00776211870274235\\
596.01	0.00820423876877381\\
597.01	0.00865403604580581\\
598.01	0.0091111651447944\\
599.01	0.00957383789362019\\
599.02	0.0095784699445533\\
599.03	0.00958310152356088\\
599.04	0.009587732601833\\
599.05	0.00959236314982192\\
599.06	0.00959699313722105\\
599.07	0.00960162253294322\\
599.08	0.00960625130509822\\
599.09	0.00961087942096949\\
599.1	0.00961550684699015\\
599.11	0.00962013354871818\\
599.12	0.00962475949081084\\
599.13	0.00962938463699814\\
599.14	0.00963400895005545\\
599.15	0.00963863239177514\\
599.16	0.00964325492293724\\
599.17	0.00964787650327905\\
599.18	0.00965249709146373\\
599.19	0.00965711664504773\\
599.2	0.00966173512044705\\
599.21	0.00966635247295013\\
599.22	0.00967096865669379\\
599.23	0.0096755836245642\\
599.24	0.00968019732815788\\
599.25	0.00968480971781115\\
599.26	0.00968942074273276\\
599.27	0.00969403035071384\\
599.28	0.00969863848897433\\
599.29	0.00970324510350724\\
599.3	0.00970785013878026\\
599.31	0.00971245353768087\\
599.32	0.0097170552414366\\
599.33	0.00972165518956062\\
599.34	0.00972625331979421\\
599.35	0.00973084956804691\\
599.36	0.00973544386833419\\
599.37	0.00974003615271259\\
599.38	0.00974462635121213\\
599.39	0.00974921439176584\\
599.4	0.00975380020013637\\
599.41	0.00975838369983937\\
599.42	0.00976296481206596\\
599.43	0.00976754345559703\\
599.44	0.00977211954671424\\
599.45	0.00977669299910894\\
599.46	0.00978126372378694\\
599.47	0.00978583162896891\\
599.48	0.00979039661998624\\
599.49	0.00979495788351871\\
599.5	0.00979951498770784\\
599.51	0.00980406782641168\\
599.52	0.00980861471572478\\
599.53	0.00981315408788168\\
599.54	0.00981767801023239\\
599.55	0.00982218287897037\\
599.56	0.00982666802718657\\
599.57	0.00983113324532572\\
599.58	0.00983557831854249\\
599.59	0.00984000302649006\\
599.6	0.00984440714309827\\
599.61	0.00984879043634082\\
599.62	0.00985315263350748\\
599.63	0.00985749347945079\\
599.64	0.00986181272388278\\
599.65	0.00986611010955735\\
599.66	0.00987038537198554\\
599.67	0.00987463823912518\\
599.68	0.00987886843105376\\
599.69	0.00988307565962328\\
599.7	0.00988725962809584\\
599.71	0.00989142003075855\\
599.72	0.00989555655251621\\
599.73	0.00989966886846011\\
599.74	0.00990375664357885\\
599.75	0.00990781953233169\\
599.76	0.00991185717815917\\
599.77	0.00991586921296373\\
599.78	0.00991985525655799\\
599.79	0.00992381491609016\\
599.8	0.00992774778542166\\
599.81	0.00993165344446322\\
599.82	0.00993553145846567\\
599.83	0.00993938137726134\\
599.84	0.00994320273445221\\
599.85	0.00994699504653957\\
599.86	0.00995075781198986\\
599.87	0.00995449051023017\\
599.88	0.00995819260056652\\
599.89	0.00996186352101685\\
599.9	0.00996550268704979\\
599.91	0.00996910949021901\\
599.92	0.00997268329668161\\
599.93	0.00997622344558731\\
599.94	0.00997972924732342\\
599.95	0.00998319998159831\\
599.96	0.00998663489534342\\
599.97	0.0099900332004109\\
599.98	0.00999339407104016\\
599.99	0.00999671664106234\\
600	0.01\\
};
\addplot [color=mycolor14,solid,forget plot]
  table[row sep=crcr]{%
0.01	0.01\\
1.01	0.01\\
2.01	0.01\\
3.01	0.01\\
4.01	0.01\\
5.01	0.01\\
6.01	0.01\\
7.01	0.01\\
8.01	0.01\\
9.01	0.01\\
10.01	0.01\\
11.01	0.01\\
12.01	0.01\\
13.01	0.01\\
14.01	0.01\\
15.01	0.01\\
16.01	0.01\\
17.01	0.01\\
18.01	0.01\\
19.01	0.01\\
20.01	0.01\\
21.01	0.01\\
22.01	0.01\\
23.01	0.01\\
24.01	0.01\\
25.01	0.01\\
26.01	0.01\\
27.01	0.01\\
28.01	0.01\\
29.01	0.01\\
30.01	0.01\\
31.01	0.01\\
32.01	0.01\\
33.01	0.01\\
34.01	0.01\\
35.01	0.01\\
36.01	0.01\\
37.01	0.01\\
38.01	0.01\\
39.01	0.01\\
40.01	0.01\\
41.01	0.01\\
42.01	0.01\\
43.01	0.01\\
44.01	0.01\\
45.01	0.01\\
46.01	0.01\\
47.01	0.01\\
48.01	0.01\\
49.01	0.01\\
50.01	0.01\\
51.01	0.01\\
52.01	0.01\\
53.01	0.01\\
54.01	0.01\\
55.01	0.01\\
56.01	0.01\\
57.01	0.01\\
58.01	0.01\\
59.01	0.01\\
60.01	0.01\\
61.01	0.01\\
62.01	0.01\\
63.01	0.01\\
64.01	0.01\\
65.01	0.01\\
66.01	0.01\\
67.01	0.01\\
68.01	0.01\\
69.01	0.01\\
70.01	0.01\\
71.01	0.01\\
72.01	0.01\\
73.01	0.01\\
74.01	0.01\\
75.01	0.01\\
76.01	0.01\\
77.01	0.01\\
78.01	0.01\\
79.01	0.01\\
80.01	0.01\\
81.01	0.01\\
82.01	0.01\\
83.01	0.01\\
84.01	0.01\\
85.01	0.01\\
86.01	0.01\\
87.01	0.01\\
88.01	0.01\\
89.01	0.01\\
90.01	0.01\\
91.01	0.01\\
92.01	0.01\\
93.01	0.01\\
94.01	0.01\\
95.01	0.01\\
96.01	0.01\\
97.01	0.01\\
98.01	0.01\\
99.01	0.01\\
100.01	0.01\\
101.01	0.01\\
102.01	0.01\\
103.01	0.01\\
104.01	0.01\\
105.01	0.01\\
106.01	0.01\\
107.01	0.01\\
108.01	0.01\\
109.01	0.01\\
110.01	0.01\\
111.01	0.01\\
112.01	0.01\\
113.01	0.01\\
114.01	0.01\\
115.01	0.01\\
116.01	0.01\\
117.01	0.01\\
118.01	0.01\\
119.01	0.01\\
120.01	0.01\\
121.01	0.01\\
122.01	0.01\\
123.01	0.01\\
124.01	0.01\\
125.01	0.01\\
126.01	0.01\\
127.01	0.01\\
128.01	0.01\\
129.01	0.01\\
130.01	0.01\\
131.01	0.01\\
132.01	0.01\\
133.01	0.01\\
134.01	0.01\\
135.01	0.01\\
136.01	0.01\\
137.01	0.01\\
138.01	0.01\\
139.01	0.01\\
140.01	0.01\\
141.01	0.01\\
142.01	0.01\\
143.01	0.01\\
144.01	0.01\\
145.01	0.01\\
146.01	0.01\\
147.01	0.01\\
148.01	0.01\\
149.01	0.01\\
150.01	0.01\\
151.01	0.01\\
152.01	0.01\\
153.01	0.01\\
154.01	0.01\\
155.01	0.01\\
156.01	0.01\\
157.01	0.01\\
158.01	0.01\\
159.01	0.01\\
160.01	0.01\\
161.01	0.01\\
162.01	0.01\\
163.01	0.01\\
164.01	0.01\\
165.01	0.01\\
166.01	0.01\\
167.01	0.01\\
168.01	0.01\\
169.01	0.01\\
170.01	0.01\\
171.01	0.01\\
172.01	0.01\\
173.01	0.01\\
174.01	0.01\\
175.01	0.01\\
176.01	0.01\\
177.01	0.01\\
178.01	0.01\\
179.01	0.01\\
180.01	0.01\\
181.01	0.01\\
182.01	0.01\\
183.01	0.01\\
184.01	0.01\\
185.01	0.01\\
186.01	0.01\\
187.01	0.01\\
188.01	0.01\\
189.01	0.01\\
190.01	0.01\\
191.01	0.01\\
192.01	0.01\\
193.01	0.01\\
194.01	0.01\\
195.01	0.01\\
196.01	0.01\\
197.01	0.01\\
198.01	0.01\\
199.01	0.01\\
200.01	0.01\\
201.01	0.01\\
202.01	0.01\\
203.01	0.01\\
204.01	0.01\\
205.01	0.01\\
206.01	0.01\\
207.01	0.01\\
208.01	0.01\\
209.01	0.01\\
210.01	0.01\\
211.01	0.01\\
212.01	0.01\\
213.01	0.01\\
214.01	0.01\\
215.01	0.01\\
216.01	0.01\\
217.01	0.01\\
218.01	0.01\\
219.01	0.01\\
220.01	0.01\\
221.01	0.01\\
222.01	0.01\\
223.01	0.01\\
224.01	0.01\\
225.01	0.01\\
226.01	0.01\\
227.01	0.01\\
228.01	0.01\\
229.01	0.01\\
230.01	0.01\\
231.01	0.01\\
232.01	0.01\\
233.01	0.01\\
234.01	0.01\\
235.01	0.01\\
236.01	0.01\\
237.01	0.01\\
238.01	0.01\\
239.01	0.01\\
240.01	0.01\\
241.01	0.01\\
242.01	0.01\\
243.01	0.01\\
244.01	0.01\\
245.01	0.01\\
246.01	0.01\\
247.01	0.01\\
248.01	0.01\\
249.01	0.01\\
250.01	0.01\\
251.01	0.01\\
252.01	0.01\\
253.01	0.01\\
254.01	0.01\\
255.01	0.01\\
256.01	0.01\\
257.01	0.01\\
258.01	0.01\\
259.01	0.01\\
260.01	0.01\\
261.01	0.01\\
262.01	0.01\\
263.01	0.01\\
264.01	0.01\\
265.01	0.01\\
266.01	0.01\\
267.01	0.01\\
268.01	0.01\\
269.01	0.01\\
270.01	0.01\\
271.01	0.01\\
272.01	0.01\\
273.01	0.01\\
274.01	0.01\\
275.01	0.01\\
276.01	0.01\\
277.01	0.01\\
278.01	0.01\\
279.01	0.01\\
280.01	0.01\\
281.01	0.01\\
282.01	0.01\\
283.01	0.01\\
284.01	0.01\\
285.01	0.01\\
286.01	0.01\\
287.01	0.01\\
288.01	0.01\\
289.01	0.01\\
290.01	0.01\\
291.01	0.01\\
292.01	0.01\\
293.01	0.01\\
294.01	0.01\\
295.01	0.01\\
296.01	0.01\\
297.01	0.01\\
298.01	0.01\\
299.01	0.01\\
300.01	0.01\\
301.01	0.01\\
302.01	0.01\\
303.01	0.01\\
304.01	0.01\\
305.01	0.01\\
306.01	0.01\\
307.01	0.01\\
308.01	0.01\\
309.01	0.01\\
310.01	0.01\\
311.01	0.01\\
312.01	0.01\\
313.01	0.01\\
314.01	0.01\\
315.01	0.01\\
316.01	0.01\\
317.01	0.01\\
318.01	0.01\\
319.01	0.01\\
320.01	0.01\\
321.01	0.01\\
322.01	0.01\\
323.01	0.01\\
324.01	0.01\\
325.01	0.01\\
326.01	0.01\\
327.01	0.01\\
328.01	0.01\\
329.01	0.01\\
330.01	0.01\\
331.01	0.01\\
332.01	0.01\\
333.01	0.01\\
334.01	0.01\\
335.01	0.01\\
336.01	0.01\\
337.01	0.01\\
338.01	0.01\\
339.01	0.01\\
340.01	0.01\\
341.01	0.01\\
342.01	0.01\\
343.01	0.01\\
344.01	0.01\\
345.01	0.01\\
346.01	0.01\\
347.01	0.01\\
348.01	0.01\\
349.01	0.01\\
350.01	0.01\\
351.01	0.01\\
352.01	0.01\\
353.01	0.01\\
354.01	0.01\\
355.01	0.01\\
356.01	0.01\\
357.01	0.01\\
358.01	0.01\\
359.01	0.01\\
360.01	0.01\\
361.01	0.01\\
362.01	0.01\\
363.01	0.01\\
364.01	0.01\\
365.01	0.01\\
366.01	0.01\\
367.01	0.01\\
368.01	0.01\\
369.01	0.01\\
370.01	0.01\\
371.01	0.01\\
372.01	0.01\\
373.01	0.01\\
374.01	0.01\\
375.01	0.01\\
376.01	0.01\\
377.01	0.01\\
378.01	0.01\\
379.01	0.01\\
380.01	0.01\\
381.01	0.01\\
382.01	0.01\\
383.01	0.01\\
384.01	0.01\\
385.01	0.01\\
386.01	0.01\\
387.01	0.01\\
388.01	0.01\\
389.01	0.01\\
390.01	0.01\\
391.01	0.01\\
392.01	0.01\\
393.01	0.01\\
394.01	0.01\\
395.01	0.01\\
396.01	0.01\\
397.01	0.01\\
398.01	0.01\\
399.01	0.01\\
400.01	0.01\\
401.01	0.01\\
402.01	0.01\\
403.01	0.01\\
404.01	0.01\\
405.01	0.01\\
406.01	0.01\\
407.01	0.01\\
408.01	0.01\\
409.01	0.01\\
410.01	0.01\\
411.01	0.01\\
412.01	0.01\\
413.01	0.01\\
414.01	0.01\\
415.01	0.01\\
416.01	0.01\\
417.01	0.01\\
418.01	0.01\\
419.01	0.01\\
420.01	0.01\\
421.01	0.01\\
422.01	0.01\\
423.01	0.01\\
424.01	0.01\\
425.01	0.01\\
426.01	0.01\\
427.01	0.01\\
428.01	0.01\\
429.01	0.01\\
430.01	0.01\\
431.01	0.01\\
432.01	0.01\\
433.01	0.01\\
434.01	0.01\\
435.01	0.01\\
436.01	0.01\\
437.01	0.01\\
438.01	0.01\\
439.01	0.01\\
440.01	0.01\\
441.01	0.01\\
442.01	0.01\\
443.01	0.01\\
444.01	0.01\\
445.01	0.01\\
446.01	0.01\\
447.01	0.01\\
448.01	0.01\\
449.01	0.01\\
450.01	0.01\\
451.01	0.01\\
452.01	0.01\\
453.01	0.01\\
454.01	0.01\\
455.01	0.01\\
456.01	0.01\\
457.01	0.01\\
458.01	0.01\\
459.01	0.01\\
460.01	0.01\\
461.01	0.01\\
462.01	0.01\\
463.01	0.01\\
464.01	0.01\\
465.01	0.01\\
466.01	0.01\\
467.01	0.01\\
468.01	0.01\\
469.01	0.01\\
470.01	0.01\\
471.01	0.01\\
472.01	0.01\\
473.01	0.01\\
474.01	0.01\\
475.01	0.01\\
476.01	0.01\\
477.01	0.01\\
478.01	0.01\\
479.01	0.01\\
480.01	0.01\\
481.01	0.01\\
482.01	0.01\\
483.01	0.01\\
484.01	0.01\\
485.01	0.01\\
486.01	0.01\\
487.01	0.01\\
488.01	0.01\\
489.01	0.01\\
490.01	0.01\\
491.01	0.01\\
492.01	0.01\\
493.01	0.01\\
494.01	0.01\\
495.01	0.01\\
496.01	0.01\\
497.01	0.01\\
498.01	0.01\\
499.01	0.01\\
500.01	0.01\\
501.01	0.01\\
502.01	0.01\\
503.01	0.01\\
504.01	0.01\\
505.01	0.01\\
506.01	0.01\\
507.01	0.01\\
508.01	0.01\\
509.01	0.01\\
510.01	0.01\\
511.01	0.01\\
512.01	0.01\\
513.01	0.01\\
514.01	0.01\\
515.01	0.01\\
516.01	0.01\\
517.01	0.01\\
518.01	0.01\\
519.01	0.01\\
520.01	0.01\\
521.01	0.01\\
522.01	0.01\\
523.01	0.01\\
524.01	0.01\\
525.01	0.01\\
526.01	0.01\\
527.01	0.01\\
528.01	0.01\\
529.01	0.01\\
530.01	0.01\\
531.01	0.01\\
532.01	0.01\\
533.01	0.01\\
534.01	0.01\\
535.01	0.01\\
536.01	0.01\\
537.01	0.01\\
538.01	0.01\\
539.01	0.01\\
540.01	0.01\\
541.01	0.01\\
542.01	0.01\\
543.01	0.01\\
544.01	0.01\\
545.01	0.01\\
546.01	0.01\\
547.01	0.01\\
548.01	0.01\\
549.01	0.01\\
550.01	0.01\\
551.01	0.01\\
552.01	0.01\\
553.01	0.01\\
554.01	0.01\\
555.01	0.01\\
556.01	0.01\\
557.01	0.01\\
558.01	0.01\\
559.01	0.01\\
560.01	0.01\\
561.01	0.01\\
562.01	0.01\\
563.01	0.01\\
564.01	0.01\\
565.01	0.01\\
566.01	0.01\\
567.01	0.01\\
568.01	0.01\\
569.01	0.01\\
570.01	0.01\\
571.01	0.01\\
572.01	0.01\\
573.01	0.00980894049358069\\
574.01	0.00959285206314057\\
575.01	0.00936137092261357\\
576.01	0.00911232543663533\\
577.01	0.00884392085869094\\
578.01	0.00856331316716572\\
579.01	0.00827261631885943\\
580.01	0.00797142148655063\\
581.01	0.00765935959990316\\
582.01	0.00733612494795841\\
583.01	0.00700150675312477\\
584.01	0.00665543123614698\\
585.01	0.00629825648550703\\
586.01	0.00593194486745375\\
587.01	0.0055575347412609\\
588.01	0.00517546743094326\\
589.01	0.00478557032423368\\
590.01	0.00438767896675604\\
591.01	0.00398164747908303\\
592.01	0.00356735318276342\\
593.01	0.00314470158774184\\
594.01	0.00271363155295263\\
595.01	0.00227412039702548\\
596.01	0.00182618962595358\\
597.01	0.00136992176993503\\
598.01	0.000905592002350849\\
599.01	0.00043492981576735\\
599.02	0.000430213974982772\\
599.03	0.000425498536119657\\
599.04	0.000420783528401717\\
599.05	0.000416068981802759\\
599.06	0.000411354927068031\\
599.07	0.000406641395736292\\
599.08	0.000401928420162703\\
599.09	0.000397216033542409\\
599.1	0.000392504269934873\\
599.11	0.000387793164289036\\
599.12	0.000383082752469323\\
599.13	0.000378373071282544\\
599.14	0.000373664158505704\\
599.15	0.000368956052914794\\
599.16	0.000364248794314553\\
599.17	0.000359542423569297\\
599.18	0.000354836982634823\\
599.19	0.000350132514591449\\
599.2	0.000345429063678236\\
599.21	0.000340726675289876\\
599.22	0.000336025395992838\\
599.23	0.000331325273624394\\
599.24	0.000326626357332171\\
599.25	0.000321928697615173\\
599.26	0.000317232346366351\\
599.27	0.000312537356328739\\
599.28	0.000307843780926497\\
599.29	0.000303151675113411\\
599.3	0.000298461095422742\\
599.31	0.000293772099994554\\
599.32	0.000289084748632946\\
599.33	0.000284399102886653\\
599.34	0.000279715226107495\\
599.35	0.000275033183508976\\
599.36	0.000270353042232248\\
599.37	0.000265674871413527\\
599.38	0.000260998742252962\\
599.39	0.00025632472808642\\
599.4	0.000251652904460315\\
599.41	0.000246983349209667\\
599.42	0.000242316142539528\\
599.43	0.000237651367110008\\
599.44	0.000232989108125029\\
599.45	0.000228329453425088\\
599.46	0.000223672493584178\\
599.47	0.000219018322011167\\
599.48	0.000214367035055838\\
599.49	0.000209719387470494\\
599.5	0.000205075830177353\\
599.51	0.00020043647066856\\
599.52	0.000195801419884652\\
599.53	0.000191173145721217\\
599.54	0.000186563422008259\\
599.55	0.000181972450344473\\
599.56	0.000177400620217833\\
599.57	0.00017284847721002\\
599.58	0.000168316234915521\\
599.59	0.000163804156364234\\
599.6	0.00015931246726116\\
599.61	0.000154841397385808\\
599.62	0.000150391182695489\\
599.63	0.000145962065581585\\
599.64	0.000141554295139023\\
599.65	0.000137168127449856\\
599.66	0.000132803825881878\\
599.67	0.000128461661373839\\
599.68	0.000124141912752868\\
599.69	0.000119844867080403\\
599.7	0.00011557082001749\\
599.71	0.00011132007621082\\
599.72	0.000107092949700736\\
599.73	0.000102889764201437\\
599.74	9.87108534851212e-05\\
599.75	9.45565618428821e-05\\
599.76	9.04272445685847e-05\\
599.77	8.6323268469345e-05\\
599.78	8.22450124151018e-05\\
599.79	7.81928679230454e-05\\
599.8	7.41672397797984e-05\\
599.81	7.01685467046336e-05\\
599.82	6.6197222057314e-05\\
599.83	6.22537145945345e-05\\
599.84	5.83384892794431e-05\\
599.85	5.44520281491948e-05\\
599.86	5.05948312460661e-05\\
599.87	4.67674176184013e-05\\
599.88	4.29703263983248e-05\\
599.89	3.92041179641221e-05\\
599.9	3.54693751961673e-05\\
599.91	3.17667048364539e-05\\
599.92	2.80967389631615e-05\\
599.93	2.44601365932736e-05\\
599.94	2.08575854280871e-05\\
599.95	1.72898037587021e-05\\
599.96	1.37575425510246e-05\\
599.97	1.02615877329605e-05\\
599.98	6.80276270998044e-06\\
599.99	3.38193113953895e-06\\
600	0\\
};
\addplot [color=mycolor15,solid,forget plot]
  table[row sep=crcr]{%
0.01	0.01\\
1.01	0.01\\
2.01	0.01\\
3.01	0.01\\
4.01	0.01\\
5.01	0.01\\
6.01	0.01\\
7.01	0.01\\
8.01	0.01\\
9.01	0.01\\
10.01	0.01\\
11.01	0.01\\
12.01	0.01\\
13.01	0.01\\
14.01	0.01\\
15.01	0.01\\
16.01	0.01\\
17.01	0.01\\
18.01	0.01\\
19.01	0.01\\
20.01	0.01\\
21.01	0.01\\
22.01	0.01\\
23.01	0.01\\
24.01	0.01\\
25.01	0.01\\
26.01	0.01\\
27.01	0.01\\
28.01	0.01\\
29.01	0.01\\
30.01	0.01\\
31.01	0.01\\
32.01	0.01\\
33.01	0.01\\
34.01	0.01\\
35.01	0.01\\
36.01	0.01\\
37.01	0.01\\
38.01	0.01\\
39.01	0.01\\
40.01	0.01\\
41.01	0.01\\
42.01	0.01\\
43.01	0.01\\
44.01	0.01\\
45.01	0.01\\
46.01	0.01\\
47.01	0.01\\
48.01	0.01\\
49.01	0.01\\
50.01	0.01\\
51.01	0.01\\
52.01	0.01\\
53.01	0.01\\
54.01	0.01\\
55.01	0.01\\
56.01	0.01\\
57.01	0.01\\
58.01	0.01\\
59.01	0.01\\
60.01	0.01\\
61.01	0.01\\
62.01	0.01\\
63.01	0.01\\
64.01	0.01\\
65.01	0.01\\
66.01	0.01\\
67.01	0.01\\
68.01	0.01\\
69.01	0.01\\
70.01	0.01\\
71.01	0.01\\
72.01	0.01\\
73.01	0.01\\
74.01	0.01\\
75.01	0.01\\
76.01	0.01\\
77.01	0.01\\
78.01	0.01\\
79.01	0.01\\
80.01	0.01\\
81.01	0.01\\
82.01	0.01\\
83.01	0.01\\
84.01	0.01\\
85.01	0.01\\
86.01	0.01\\
87.01	0.01\\
88.01	0.01\\
89.01	0.01\\
90.01	0.01\\
91.01	0.01\\
92.01	0.01\\
93.01	0.01\\
94.01	0.01\\
95.01	0.01\\
96.01	0.01\\
97.01	0.01\\
98.01	0.01\\
99.01	0.01\\
100.01	0.01\\
101.01	0.01\\
102.01	0.01\\
103.01	0.01\\
104.01	0.01\\
105.01	0.01\\
106.01	0.01\\
107.01	0.01\\
108.01	0.01\\
109.01	0.01\\
110.01	0.01\\
111.01	0.01\\
112.01	0.01\\
113.01	0.01\\
114.01	0.01\\
115.01	0.01\\
116.01	0.01\\
117.01	0.01\\
118.01	0.01\\
119.01	0.01\\
120.01	0.01\\
121.01	0.01\\
122.01	0.01\\
123.01	0.01\\
124.01	0.01\\
125.01	0.01\\
126.01	0.01\\
127.01	0.01\\
128.01	0.01\\
129.01	0.01\\
130.01	0.01\\
131.01	0.01\\
132.01	0.01\\
133.01	0.01\\
134.01	0.01\\
135.01	0.01\\
136.01	0.01\\
137.01	0.01\\
138.01	0.01\\
139.01	0.01\\
140.01	0.01\\
141.01	0.01\\
142.01	0.01\\
143.01	0.01\\
144.01	0.01\\
145.01	0.01\\
146.01	0.01\\
147.01	0.01\\
148.01	0.01\\
149.01	0.01\\
150.01	0.01\\
151.01	0.01\\
152.01	0.01\\
153.01	0.01\\
154.01	0.01\\
155.01	0.01\\
156.01	0.01\\
157.01	0.01\\
158.01	0.01\\
159.01	0.01\\
160.01	0.01\\
161.01	0.01\\
162.01	0.01\\
163.01	0.01\\
164.01	0.01\\
165.01	0.01\\
166.01	0.01\\
167.01	0.01\\
168.01	0.01\\
169.01	0.01\\
170.01	0.01\\
171.01	0.01\\
172.01	0.01\\
173.01	0.01\\
174.01	0.01\\
175.01	0.01\\
176.01	0.01\\
177.01	0.01\\
178.01	0.01\\
179.01	0.01\\
180.01	0.01\\
181.01	0.01\\
182.01	0.01\\
183.01	0.01\\
184.01	0.01\\
185.01	0.01\\
186.01	0.01\\
187.01	0.01\\
188.01	0.01\\
189.01	0.01\\
190.01	0.01\\
191.01	0.01\\
192.01	0.01\\
193.01	0.01\\
194.01	0.01\\
195.01	0.01\\
196.01	0.01\\
197.01	0.01\\
198.01	0.01\\
199.01	0.01\\
200.01	0.01\\
201.01	0.01\\
202.01	0.01\\
203.01	0.01\\
204.01	0.01\\
205.01	0.01\\
206.01	0.01\\
207.01	0.01\\
208.01	0.01\\
209.01	0.01\\
210.01	0.01\\
211.01	0.01\\
212.01	0.01\\
213.01	0.01\\
214.01	0.01\\
215.01	0.01\\
216.01	0.01\\
217.01	0.01\\
218.01	0.01\\
219.01	0.01\\
220.01	0.01\\
221.01	0.01\\
222.01	0.01\\
223.01	0.01\\
224.01	0.01\\
225.01	0.01\\
226.01	0.01\\
227.01	0.01\\
228.01	0.01\\
229.01	0.01\\
230.01	0.01\\
231.01	0.01\\
232.01	0.01\\
233.01	0.01\\
234.01	0.01\\
235.01	0.01\\
236.01	0.01\\
237.01	0.01\\
238.01	0.01\\
239.01	0.01\\
240.01	0.01\\
241.01	0.01\\
242.01	0.01\\
243.01	0.01\\
244.01	0.01\\
245.01	0.01\\
246.01	0.01\\
247.01	0.01\\
248.01	0.01\\
249.01	0.01\\
250.01	0.01\\
251.01	0.01\\
252.01	0.01\\
253.01	0.01\\
254.01	0.01\\
255.01	0.01\\
256.01	0.01\\
257.01	0.01\\
258.01	0.01\\
259.01	0.01\\
260.01	0.01\\
261.01	0.01\\
262.01	0.01\\
263.01	0.01\\
264.01	0.01\\
265.01	0.01\\
266.01	0.01\\
267.01	0.01\\
268.01	0.01\\
269.01	0.01\\
270.01	0.01\\
271.01	0.01\\
272.01	0.01\\
273.01	0.01\\
274.01	0.01\\
275.01	0.01\\
276.01	0.01\\
277.01	0.01\\
278.01	0.01\\
279.01	0.01\\
280.01	0.01\\
281.01	0.01\\
282.01	0.01\\
283.01	0.01\\
284.01	0.01\\
285.01	0.01\\
286.01	0.01\\
287.01	0.01\\
288.01	0.01\\
289.01	0.01\\
290.01	0.01\\
291.01	0.01\\
292.01	0.01\\
293.01	0.01\\
294.01	0.01\\
295.01	0.01\\
296.01	0.01\\
297.01	0.01\\
298.01	0.01\\
299.01	0.01\\
300.01	0.01\\
301.01	0.01\\
302.01	0.01\\
303.01	0.01\\
304.01	0.01\\
305.01	0.01\\
306.01	0.01\\
307.01	0.01\\
308.01	0.01\\
309.01	0.01\\
310.01	0.01\\
311.01	0.01\\
312.01	0.01\\
313.01	0.01\\
314.01	0.01\\
315.01	0.01\\
316.01	0.01\\
317.01	0.01\\
318.01	0.01\\
319.01	0.01\\
320.01	0.01\\
321.01	0.01\\
322.01	0.01\\
323.01	0.01\\
324.01	0.01\\
325.01	0.01\\
326.01	0.01\\
327.01	0.01\\
328.01	0.01\\
329.01	0.01\\
330.01	0.01\\
331.01	0.01\\
332.01	0.01\\
333.01	0.01\\
334.01	0.01\\
335.01	0.01\\
336.01	0.01\\
337.01	0.01\\
338.01	0.01\\
339.01	0.01\\
340.01	0.01\\
341.01	0.01\\
342.01	0.01\\
343.01	0.01\\
344.01	0.01\\
345.01	0.01\\
346.01	0.01\\
347.01	0.01\\
348.01	0.01\\
349.01	0.01\\
350.01	0.01\\
351.01	0.01\\
352.01	0.01\\
353.01	0.01\\
354.01	0.01\\
355.01	0.01\\
356.01	0.01\\
357.01	0.01\\
358.01	0.01\\
359.01	0.01\\
360.01	0.01\\
361.01	0.01\\
362.01	0.01\\
363.01	0.01\\
364.01	0.01\\
365.01	0.01\\
366.01	0.01\\
367.01	0.01\\
368.01	0.01\\
369.01	0.01\\
370.01	0.01\\
371.01	0.01\\
372.01	0.01\\
373.01	0.01\\
374.01	0.01\\
375.01	0.01\\
376.01	0.01\\
377.01	0.01\\
378.01	0.01\\
379.01	0.01\\
380.01	0.01\\
381.01	0.01\\
382.01	0.01\\
383.01	0.01\\
384.01	0.01\\
385.01	0.01\\
386.01	0.01\\
387.01	0.01\\
388.01	0.01\\
389.01	0.01\\
390.01	0.01\\
391.01	0.01\\
392.01	0.01\\
393.01	0.01\\
394.01	0.01\\
395.01	0.01\\
396.01	0.01\\
397.01	0.01\\
398.01	0.01\\
399.01	0.01\\
400.01	0.01\\
401.01	0.01\\
402.01	0.01\\
403.01	0.01\\
404.01	0.01\\
405.01	0.01\\
406.01	0.01\\
407.01	0.01\\
408.01	0.01\\
409.01	0.01\\
410.01	0.01\\
411.01	0.01\\
412.01	0.01\\
413.01	0.01\\
414.01	0.01\\
415.01	0.01\\
416.01	0.01\\
417.01	0.01\\
418.01	0.01\\
419.01	0.01\\
420.01	0.01\\
421.01	0.01\\
422.01	0.01\\
423.01	0.01\\
424.01	0.01\\
425.01	0.01\\
426.01	0.01\\
427.01	0.01\\
428.01	0.01\\
429.01	0.01\\
430.01	0.01\\
431.01	0.01\\
432.01	0.01\\
433.01	0.01\\
434.01	0.01\\
435.01	0.01\\
436.01	0.01\\
437.01	0.01\\
438.01	0.01\\
439.01	0.01\\
440.01	0.01\\
441.01	0.01\\
442.01	0.01\\
443.01	0.01\\
444.01	0.01\\
445.01	0.01\\
446.01	0.01\\
447.01	0.01\\
448.01	0.01\\
449.01	0.01\\
450.01	0.01\\
451.01	0.01\\
452.01	0.01\\
453.01	0.01\\
454.01	0.01\\
455.01	0.01\\
456.01	0.01\\
457.01	0.01\\
458.01	0.01\\
459.01	0.01\\
460.01	0.01\\
461.01	0.01\\
462.01	0.01\\
463.01	0.01\\
464.01	0.01\\
465.01	0.01\\
466.01	0.01\\
467.01	0.01\\
468.01	0.01\\
469.01	0.01\\
470.01	0.01\\
471.01	0.01\\
472.01	0.01\\
473.01	0.01\\
474.01	0.01\\
475.01	0.01\\
476.01	0.01\\
477.01	0.01\\
478.01	0.01\\
479.01	0.01\\
480.01	0.01\\
481.01	0.01\\
482.01	0.01\\
483.01	0.01\\
484.01	0.01\\
485.01	0.01\\
486.01	0.01\\
487.01	0.01\\
488.01	0.01\\
489.01	0.01\\
490.01	0.01\\
491.01	0.01\\
492.01	0.01\\
493.01	0.01\\
494.01	0.01\\
495.01	0.01\\
496.01	0.01\\
497.01	0.01\\
498.01	0.01\\
499.01	0.01\\
500.01	0.01\\
501.01	0.01\\
502.01	0.01\\
503.01	0.01\\
504.01	0.01\\
505.01	0.01\\
506.01	0.01\\
507.01	0.01\\
508.01	0.01\\
509.01	0.01\\
510.01	0.01\\
511.01	0.01\\
512.01	0.01\\
513.01	0.01\\
514.01	0.01\\
515.01	0.01\\
516.01	0.01\\
517.01	0.01\\
518.01	0.01\\
519.01	0.01\\
520.01	0.01\\
521.01	0.01\\
522.01	0.01\\
523.01	0.01\\
524.01	0.01\\
525.01	0.01\\
526.01	0.01\\
527.01	0.01\\
528.01	0.01\\
529.01	0.01\\
530.01	0.01\\
531.01	0.01\\
532.01	0.01\\
533.01	0.01\\
534.01	0.01\\
535.01	0.01\\
536.01	0.01\\
537.01	0.01\\
538.01	0.01\\
539.01	0.01\\
540.01	0.01\\
541.01	0.01\\
542.01	0.01\\
543.01	0.01\\
544.01	0.01\\
545.01	0.01\\
546.01	0.01\\
547.01	0.01\\
548.01	0.01\\
549.01	0.01\\
550.01	0.01\\
551.01	0.00990204942161797\\
552.01	0.00975551902864949\\
553.01	0.00960130073362049\\
554.01	0.0094386003805723\\
555.01	0.009266503152353\\
556.01	0.0090839507765319\\
557.01	0.00888971377982333\\
558.01	0.008682357705687\\
559.01	0.00846020294590994\\
560.01	0.00822128900639638\\
561.01	0.00796766659822521\\
562.01	0.0077048159941505\\
563.01	0.00743232512672041\\
564.01	0.0071496454824594\\
565.01	0.0068561899957042\\
566.01	0.0065513453559236\\
567.01	0.00623447560958786\\
568.01	0.00590492764055903\\
569.01	0.00556203943909303\\
570.01	0.00520515136201691\\
571.01	0.0048335653439925\\
572.01	0.00444648371089207\\
573.01	0.0042357046706426\\
574.01	0.00403720111108808\\
575.01	0.00384342391724351\\
576.01	0.00365706307610982\\
577.01	0.00348009927715622\\
578.01	0.00330555158339844\\
579.01	0.00313128876087222\\
580.01	0.00295772959046831\\
581.01	0.00278530784327323\\
582.01	0.0026144529462283\\
583.01	0.00244555985981434\\
584.01	0.00227894406209801\\
585.01	0.00211453903896056\\
586.01	0.00195076928643856\\
587.01	0.00178744637180153\\
588.01	0.00162464815568858\\
589.01	0.00146288912824371\\
590.01	0.00130274409803109\\
591.01	0.00114484388588358\\
592.01	0.000989873810182567\\
593.01	0.00083856959646709\\
594.01	0.00069170989414428\\
595.01	0.000550104571421454\\
596.01	0.000414578932401176\\
597.01	0.000285963596768515\\
598.01	0.000165188062198949\\
599.01	5.4345015548133e-05\\
599.02	5.33064624635531e-05\\
599.03	5.22697048745222e-05\\
599.04	5.12347597904136e-05\\
599.05	5.0201644409836e-05\\
599.06	4.91703761153605e-05\\
599.07	4.8140972467612e-05\\
599.08	4.71134500464859e-05\\
599.09	4.60878252401993e-05\\
599.1	4.50641145578815e-05\\
599.11	4.40423346413817e-05\\
599.12	4.30225022459152e-05\\
599.13	4.20046341915222e-05\\
599.14	4.0988747398104e-05\\
599.15	3.99748588722423e-05\\
599.16	3.89629856928746e-05\\
599.17	3.79531449942027e-05\\
599.18	3.69453539466515e-05\\
599.19	3.59396297421356e-05\\
599.2	3.49359895741794e-05\\
599.21	3.39348997364097e-05\\
599.22	3.29370757084346e-05\\
599.23	3.19425380165637e-05\\
599.24	3.09513071439539e-05\\
599.25	2.99634035031259e-05\\
599.26	2.89788474063021e-05\\
599.27	2.80044066980276e-05\\
599.28	2.70493587152636e-05\\
599.29	2.61138364157729e-05\\
599.3	2.51979740088139e-05\\
599.31	2.43022091112599e-05\\
599.32	2.34269645671102e-05\\
599.33	2.25723778285291e-05\\
599.34	2.17385877173907e-05\\
599.35	2.09257721965651e-05\\
599.36	2.01340862337751e-05\\
599.37	1.9363671100418e-05\\
599.38	1.86146691965421e-05\\
599.39	1.78872240385961e-05\\
599.4	1.71814802439572e-05\\
599.41	1.64975835122289e-05\\
599.42	1.58356806037336e-05\\
599.43	1.5195919314372e-05\\
599.44	1.457844844663e-05\\
599.45	1.39834177763486e-05\\
599.46	1.34109780150146e-05\\
599.47	1.28612807671867e-05\\
599.48	1.23344784827405e-05\\
599.49	1.18300658477049e-05\\
599.5	1.13478434777131e-05\\
599.51	1.08879592333778e-05\\
599.52	1.04505664172659e-05\\
599.53	1.00334667578663e-05\\
599.54	9.62511880027439e-06\\
599.55	9.2255549021085e-06\\
599.56	8.83462029344952e-06\\
599.57	8.45200412851778e-06\\
599.58	8.07772879304891e-06\\
599.59	7.71176793122087e-06\\
599.6	7.35413349658119e-06\\
599.61	7.00483423372172e-06\\
599.62	6.66387342551991e-06\\
599.63	6.33124847425164e-06\\
599.64	6.00695045552616e-06\\
599.65	5.69096364331188e-06\\
599.66	5.38326500425235e-06\\
599.67	5.08384138973167e-06\\
599.68	4.79267948952887e-06\\
599.69	4.50975886944101e-06\\
599.7	4.23505144031729e-06\\
599.71	3.96852089248369e-06\\
599.72	3.71012224967097e-06\\
599.73	3.45988538647847e-06\\
599.74	3.21786860717557e-06\\
599.75	2.98412945139333e-06\\
599.76	2.75872706875617e-06\\
599.77	2.54172382291654e-06\\
599.78	2.33318159911268e-06\\
599.79	2.13316174173424e-06\\
599.8	1.94172498835417e-06\\
599.81	1.75893139995643e-06\\
599.82	1.58484028716689e-06\\
599.83	1.41951013224277e-06\\
599.84	1.2629985065276e-06\\
599.85	1.11536198312695e-06\\
599.86	9.76656044501464e-07\\
599.87	8.46934984602421e-07\\
599.88	7.26251805284461e-07\\
599.89	6.14658106537491e-07\\
599.9	5.12203970199493e-07\\
599.91	4.18937836669728e-07\\
599.92	3.34906374173036e-07\\
599.93	2.60154340070429e-07\\
599.94	1.94724433657395e-07\\
599.95	1.38657139853171e-07\\
599.96	9.19905631651535e-08\\
599.97	5.47602512050716e-08\\
599.98	2.6999006997111e-08\\
599.99	8.73668930083393e-09\\
600	0\\
};
\addplot [color=mycolor16,solid,forget plot]
  table[row sep=crcr]{%
0.01	0.01\\
1.01	0.01\\
2.01	0.01\\
3.01	0.01\\
4.01	0.01\\
5.01	0.01\\
6.01	0.01\\
7.01	0.01\\
8.01	0.01\\
9.01	0.01\\
10.01	0.01\\
11.01	0.01\\
12.01	0.01\\
13.01	0.01\\
14.01	0.01\\
15.01	0.01\\
16.01	0.01\\
17.01	0.01\\
18.01	0.01\\
19.01	0.01\\
20.01	0.01\\
21.01	0.01\\
22.01	0.01\\
23.01	0.01\\
24.01	0.01\\
25.01	0.01\\
26.01	0.01\\
27.01	0.01\\
28.01	0.01\\
29.01	0.01\\
30.01	0.01\\
31.01	0.01\\
32.01	0.01\\
33.01	0.01\\
34.01	0.01\\
35.01	0.01\\
36.01	0.01\\
37.01	0.01\\
38.01	0.01\\
39.01	0.01\\
40.01	0.01\\
41.01	0.01\\
42.01	0.01\\
43.01	0.01\\
44.01	0.01\\
45.01	0.01\\
46.01	0.01\\
47.01	0.01\\
48.01	0.01\\
49.01	0.01\\
50.01	0.01\\
51.01	0.01\\
52.01	0.01\\
53.01	0.01\\
54.01	0.01\\
55.01	0.01\\
56.01	0.01\\
57.01	0.01\\
58.01	0.01\\
59.01	0.01\\
60.01	0.01\\
61.01	0.01\\
62.01	0.01\\
63.01	0.01\\
64.01	0.01\\
65.01	0.01\\
66.01	0.01\\
67.01	0.01\\
68.01	0.01\\
69.01	0.01\\
70.01	0.01\\
71.01	0.01\\
72.01	0.01\\
73.01	0.01\\
74.01	0.01\\
75.01	0.01\\
76.01	0.01\\
77.01	0.01\\
78.01	0.01\\
79.01	0.01\\
80.01	0.01\\
81.01	0.01\\
82.01	0.01\\
83.01	0.01\\
84.01	0.01\\
85.01	0.01\\
86.01	0.01\\
87.01	0.01\\
88.01	0.01\\
89.01	0.01\\
90.01	0.01\\
91.01	0.01\\
92.01	0.01\\
93.01	0.01\\
94.01	0.01\\
95.01	0.01\\
96.01	0.01\\
97.01	0.01\\
98.01	0.01\\
99.01	0.01\\
100.01	0.01\\
101.01	0.01\\
102.01	0.01\\
103.01	0.01\\
104.01	0.01\\
105.01	0.01\\
106.01	0.01\\
107.01	0.01\\
108.01	0.01\\
109.01	0.01\\
110.01	0.01\\
111.01	0.01\\
112.01	0.01\\
113.01	0.01\\
114.01	0.01\\
115.01	0.01\\
116.01	0.01\\
117.01	0.01\\
118.01	0.01\\
119.01	0.01\\
120.01	0.01\\
121.01	0.01\\
122.01	0.01\\
123.01	0.01\\
124.01	0.01\\
125.01	0.01\\
126.01	0.01\\
127.01	0.01\\
128.01	0.01\\
129.01	0.01\\
130.01	0.01\\
131.01	0.01\\
132.01	0.01\\
133.01	0.01\\
134.01	0.01\\
135.01	0.01\\
136.01	0.01\\
137.01	0.01\\
138.01	0.01\\
139.01	0.01\\
140.01	0.01\\
141.01	0.01\\
142.01	0.01\\
143.01	0.01\\
144.01	0.01\\
145.01	0.01\\
146.01	0.01\\
147.01	0.01\\
148.01	0.01\\
149.01	0.01\\
150.01	0.01\\
151.01	0.01\\
152.01	0.01\\
153.01	0.01\\
154.01	0.01\\
155.01	0.01\\
156.01	0.01\\
157.01	0.01\\
158.01	0.01\\
159.01	0.01\\
160.01	0.01\\
161.01	0.01\\
162.01	0.01\\
163.01	0.01\\
164.01	0.01\\
165.01	0.01\\
166.01	0.01\\
167.01	0.01\\
168.01	0.01\\
169.01	0.01\\
170.01	0.01\\
171.01	0.01\\
172.01	0.01\\
173.01	0.01\\
174.01	0.01\\
175.01	0.01\\
176.01	0.01\\
177.01	0.01\\
178.01	0.01\\
179.01	0.01\\
180.01	0.01\\
181.01	0.01\\
182.01	0.01\\
183.01	0.01\\
184.01	0.01\\
185.01	0.01\\
186.01	0.01\\
187.01	0.01\\
188.01	0.01\\
189.01	0.01\\
190.01	0.01\\
191.01	0.01\\
192.01	0.01\\
193.01	0.01\\
194.01	0.01\\
195.01	0.01\\
196.01	0.01\\
197.01	0.01\\
198.01	0.01\\
199.01	0.01\\
200.01	0.01\\
201.01	0.01\\
202.01	0.01\\
203.01	0.01\\
204.01	0.01\\
205.01	0.01\\
206.01	0.01\\
207.01	0.01\\
208.01	0.01\\
209.01	0.01\\
210.01	0.01\\
211.01	0.01\\
212.01	0.01\\
213.01	0.01\\
214.01	0.01\\
215.01	0.01\\
216.01	0.01\\
217.01	0.01\\
218.01	0.01\\
219.01	0.01\\
220.01	0.01\\
221.01	0.01\\
222.01	0.01\\
223.01	0.01\\
224.01	0.01\\
225.01	0.01\\
226.01	0.01\\
227.01	0.01\\
228.01	0.01\\
229.01	0.01\\
230.01	0.01\\
231.01	0.01\\
232.01	0.01\\
233.01	0.01\\
234.01	0.01\\
235.01	0.01\\
236.01	0.01\\
237.01	0.01\\
238.01	0.01\\
239.01	0.01\\
240.01	0.01\\
241.01	0.01\\
242.01	0.01\\
243.01	0.01\\
244.01	0.01\\
245.01	0.01\\
246.01	0.01\\
247.01	0.01\\
248.01	0.01\\
249.01	0.01\\
250.01	0.01\\
251.01	0.01\\
252.01	0.01\\
253.01	0.01\\
254.01	0.01\\
255.01	0.01\\
256.01	0.01\\
257.01	0.01\\
258.01	0.01\\
259.01	0.01\\
260.01	0.01\\
261.01	0.01\\
262.01	0.01\\
263.01	0.01\\
264.01	0.01\\
265.01	0.01\\
266.01	0.01\\
267.01	0.01\\
268.01	0.01\\
269.01	0.01\\
270.01	0.01\\
271.01	0.01\\
272.01	0.01\\
273.01	0.01\\
274.01	0.01\\
275.01	0.01\\
276.01	0.01\\
277.01	0.01\\
278.01	0.01\\
279.01	0.01\\
280.01	0.01\\
281.01	0.01\\
282.01	0.01\\
283.01	0.01\\
284.01	0.01\\
285.01	0.01\\
286.01	0.01\\
287.01	0.01\\
288.01	0.01\\
289.01	0.01\\
290.01	0.01\\
291.01	0.01\\
292.01	0.01\\
293.01	0.01\\
294.01	0.01\\
295.01	0.01\\
296.01	0.01\\
297.01	0.01\\
298.01	0.01\\
299.01	0.01\\
300.01	0.01\\
301.01	0.01\\
302.01	0.01\\
303.01	0.01\\
304.01	0.01\\
305.01	0.01\\
306.01	0.01\\
307.01	0.01\\
308.01	0.01\\
309.01	0.01\\
310.01	0.01\\
311.01	0.01\\
312.01	0.01\\
313.01	0.01\\
314.01	0.01\\
315.01	0.01\\
316.01	0.01\\
317.01	0.01\\
318.01	0.01\\
319.01	0.01\\
320.01	0.01\\
321.01	0.01\\
322.01	0.01\\
323.01	0.01\\
324.01	0.01\\
325.01	0.01\\
326.01	0.01\\
327.01	0.01\\
328.01	0.01\\
329.01	0.01\\
330.01	0.01\\
331.01	0.01\\
332.01	0.01\\
333.01	0.01\\
334.01	0.01\\
335.01	0.01\\
336.01	0.01\\
337.01	0.01\\
338.01	0.01\\
339.01	0.01\\
340.01	0.01\\
341.01	0.01\\
342.01	0.01\\
343.01	0.01\\
344.01	0.01\\
345.01	0.01\\
346.01	0.01\\
347.01	0.01\\
348.01	0.01\\
349.01	0.01\\
350.01	0.01\\
351.01	0.01\\
352.01	0.01\\
353.01	0.01\\
354.01	0.01\\
355.01	0.01\\
356.01	0.01\\
357.01	0.01\\
358.01	0.01\\
359.01	0.01\\
360.01	0.01\\
361.01	0.01\\
362.01	0.01\\
363.01	0.01\\
364.01	0.01\\
365.01	0.01\\
366.01	0.01\\
367.01	0.01\\
368.01	0.01\\
369.01	0.01\\
370.01	0.01\\
371.01	0.01\\
372.01	0.01\\
373.01	0.01\\
374.01	0.01\\
375.01	0.01\\
376.01	0.01\\
377.01	0.01\\
378.01	0.01\\
379.01	0.01\\
380.01	0.01\\
381.01	0.01\\
382.01	0.01\\
383.01	0.01\\
384.01	0.01\\
385.01	0.01\\
386.01	0.01\\
387.01	0.01\\
388.01	0.01\\
389.01	0.01\\
390.01	0.01\\
391.01	0.01\\
392.01	0.01\\
393.01	0.01\\
394.01	0.01\\
395.01	0.01\\
396.01	0.01\\
397.01	0.01\\
398.01	0.01\\
399.01	0.01\\
400.01	0.01\\
401.01	0.01\\
402.01	0.01\\
403.01	0.01\\
404.01	0.01\\
405.01	0.01\\
406.01	0.01\\
407.01	0.01\\
408.01	0.01\\
409.01	0.01\\
410.01	0.01\\
411.01	0.01\\
412.01	0.01\\
413.01	0.01\\
414.01	0.01\\
415.01	0.01\\
416.01	0.01\\
417.01	0.01\\
418.01	0.01\\
419.01	0.01\\
420.01	0.01\\
421.01	0.01\\
422.01	0.01\\
423.01	0.01\\
424.01	0.01\\
425.01	0.01\\
426.01	0.01\\
427.01	0.01\\
428.01	0.01\\
429.01	0.01\\
430.01	0.01\\
431.01	0.01\\
432.01	0.01\\
433.01	0.01\\
434.01	0.01\\
435.01	0.01\\
436.01	0.01\\
437.01	0.01\\
438.01	0.01\\
439.01	0.01\\
440.01	0.01\\
441.01	0.01\\
442.01	0.01\\
443.01	0.01\\
444.01	0.01\\
445.01	0.01\\
446.01	0.01\\
447.01	0.01\\
448.01	0.01\\
449.01	0.01\\
450.01	0.01\\
451.01	0.01\\
452.01	0.01\\
453.01	0.01\\
454.01	0.01\\
455.01	0.01\\
456.01	0.01\\
457.01	0.01\\
458.01	0.01\\
459.01	0.01\\
460.01	0.01\\
461.01	0.01\\
462.01	0.01\\
463.01	0.01\\
464.01	0.01\\
465.01	0.01\\
466.01	0.01\\
467.01	0.01\\
468.01	0.01\\
469.01	0.01\\
470.01	0.01\\
471.01	0.01\\
472.01	0.01\\
473.01	0.01\\
474.01	0.01\\
475.01	0.01\\
476.01	0.01\\
477.01	0.01\\
478.01	0.01\\
479.01	0.01\\
480.01	0.01\\
481.01	0.01\\
482.01	0.01\\
483.01	0.01\\
484.01	0.01\\
485.01	0.01\\
486.01	0.01\\
487.01	0.01\\
488.01	0.01\\
489.01	0.01\\
490.01	0.01\\
491.01	0.01\\
492.01	0.01\\
493.01	0.01\\
494.01	0.01\\
495.01	0.01\\
496.01	0.01\\
497.01	0.01\\
498.01	0.01\\
499.01	0.01\\
500.01	0.01\\
501.01	0.01\\
502.01	0.01\\
503.01	0.01\\
504.01	0.01\\
505.01	0.01\\
506.01	0.01\\
507.01	0.01\\
508.01	0.01\\
509.01	0.01\\
510.01	0.01\\
511.01	0.01\\
512.01	0.01\\
513.01	0.01\\
514.01	0.01\\
515.01	0.01\\
516.01	0.01\\
517.01	0.01\\
518.01	0.01\\
519.01	0.01\\
520.01	0.01\\
521.01	0.01\\
522.01	0.01\\
523.01	0.00992575245227274\\
524.01	0.00984652916315449\\
525.01	0.00976427383091153\\
526.01	0.00967878858387452\\
527.01	0.00958985481377644\\
528.01	0.00949723016296407\\
529.01	0.00940064498293353\\
530.01	0.00929979815948588\\
531.01	0.00919435215483429\\
532.01	0.0090839270942965\\
533.01	0.00896809366624845\\
534.01	0.00884636485433692\\
535.01	0.00871818833204663\\
536.01	0.00858293155538475\\
537.01	0.00843986048094458\\
538.01	0.00828812853295346\\
539.01	0.00812675686309659\\
540.01	0.00795460963928865\\
541.01	0.00777036785781529\\
542.01	0.00757342059949487\\
543.01	0.00736856329333439\\
544.01	0.00715647415153917\\
545.01	0.00693675136318322\\
546.01	0.00670895106381417\\
547.01	0.00647257909973894\\
548.01	0.00622708057036442\\
549.01	0.00597182819023422\\
550.01	0.00570611914029568\\
551.01	0.00552764479099582\\
552.01	0.00538785028227558\\
553.01	0.00524547405651935\\
554.01	0.00510085683299053\\
555.01	0.0049544724765762\\
556.01	0.00480696566205194\\
557.01	0.00465919998370261\\
558.01	0.00451231962311478\\
559.01	0.00436782949647694\\
560.01	0.00422770128462472\\
561.01	0.00409012220592474\\
562.01	0.0039497661840341\\
563.01	0.00380802511746821\\
564.01	0.00366686467040957\\
565.01	0.00352675473052659\\
566.01	0.00338818459516706\\
567.01	0.00325170044485083\\
568.01	0.00311790483298779\\
569.01	0.00298745287269209\\
570.01	0.00286104575543987\\
571.01	0.00273947549593472\\
572.01	0.00262369203361567\\
573.01	0.00251316420804345\\
574.01	0.00240612417636698\\
575.01	0.00230064181470498\\
576.01	0.00219622952468568\\
577.01	0.00209274219424811\\
578.01	0.00199006113334727\\
579.01	0.00188828358359086\\
580.01	0.00178750904497525\\
581.01	0.00168781344225515\\
582.01	0.00158924687363906\\
583.01	0.0014918307579822\\
584.01	0.00139556043494016\\
585.01	0.00130039603260936\\
586.01	0.00120615033170115\\
587.01	0.00111236845038909\\
588.01	0.00101905521743938\\
589.01	0.000926321771227706\\
590.01	0.000834264424821173\\
591.01	0.000742959206678368\\
592.01	0.000652455943327473\\
593.01	0.000562772011437012\\
594.01	0.000473886068201661\\
595.01	0.00038573221032095\\
596.01	0.000298195161474923\\
597.01	0.000211107141103397\\
598.01	0.000124246194188974\\
599.01	3.7327034981641e-05\\
599.02	3.64581259681349e-05\\
599.03	3.55904023650409e-05\\
599.04	3.47238671974722e-05\\
599.05	3.38585235776824e-05\\
599.06	3.29943747087724e-05\\
599.07	3.21314238885661e-05\\
599.08	3.12828146845964e-05\\
599.09	3.04526072935907e-05\\
599.1	2.96411999832092e-05\\
599.11	2.8848744456229e-05\\
599.12	2.80755012468988e-05\\
599.13	2.73221646003644e-05\\
599.14	2.65888933353459e-05\\
599.15	2.58758490175037e-05\\
599.16	2.51831960653721e-05\\
599.17	2.45111189409661e-05\\
599.18	2.38598295667809e-05\\
599.19	2.3229501472391e-05\\
599.2	2.26203115193009e-05\\
599.21	2.20319886654764e-05\\
599.22	2.14640126570079e-05\\
599.23	2.09165616798686e-05\\
599.24	2.03898175673723e-05\\
599.25	1.98839659503354e-05\\
599.26	1.93991964147905e-05\\
599.27	1.89289341987753e-05\\
599.28	1.8464068823297e-05\\
599.29	1.80046351070922e-05\\
599.3	1.75506675047163e-05\\
599.31	1.71018958724945e-05\\
599.32	1.66580635847045e-05\\
599.33	1.62192026103292e-05\\
599.34	1.5785344236359e-05\\
599.35	1.53564802825266e-05\\
599.36	1.49326256067388e-05\\
599.37	1.4513809013501e-05\\
599.38	1.41000583480724e-05\\
599.39	1.36914004212514e-05\\
599.4	1.32878626875211e-05\\
599.41	1.28894738098337e-05\\
599.42	1.24962611940758e-05\\
599.43	1.21082508957385e-05\\
599.44	1.17254675215759e-05\\
599.45	1.13479341259243e-05\\
599.46	1.09756721014351e-05\\
599.47	1.06087010638255e-05\\
599.48	1.02470387303398e-05\\
599.49	9.8907026131103e-06\\
599.5	9.53970915088496e-06\\
599.51	9.19407265268304e-06\\
599.52	8.85380515453817e-06\\
599.53	8.51892385799261e-06\\
599.54	8.18948273669656e-06\\
599.55	7.86553617346253e-06\\
599.56	7.54713965199667e-06\\
599.57	7.23435039426401e-06\\
599.58	6.92722614961672e-06\\
599.59	6.6258253988271e-06\\
599.6	6.33020718107669e-06\\
599.61	6.04043109214518e-06\\
599.62	5.75655729072831e-06\\
599.63	5.47864650500898e-06\\
599.64	5.2067600394886e-06\\
599.65	4.94095978209944e-06\\
599.66	4.68130821164141e-06\\
599.67	4.42786840111967e-06\\
599.68	4.18070402385777e-06\\
599.69	3.93987936155027e-06\\
599.7	3.70545931271775e-06\\
599.71	3.477509401573e-06\\
599.72	3.25609578728098e-06\\
599.73	3.04128525271952e-06\\
599.74	2.83314520516977e-06\\
599.75	2.63174368469489e-06\\
599.76	2.43714937234532e-06\\
599.77	2.24943159845246e-06\\
599.78	2.06866035230673e-06\\
599.79	1.89490629239931e-06\\
599.8	1.72824075721396e-06\\
599.81	1.56873577666743e-06\\
599.82	1.41646408419877e-06\\
599.83	1.27149912958896e-06\\
599.84	1.13391509257328e-06\\
599.85	1.00378689725684e-06\\
599.86	8.81190227465176e-07\\
599.87	7.66201543043674e-07\\
599.88	6.58898097204846e-07\\
599.89	5.59357955015258e-07\\
599.9	4.67660013056884e-07\\
599.91	3.8388402042247e-07\\
599.92	3.08110601103875e-07\\
599.93	2.40421277865333e-07\\
599.94	1.80898497769907e-07\\
599.95	1.29625659426799e-07\\
599.96	8.66871421312254e-08\\
599.97	5.21683370079129e-08\\
599.98	2.61556803542173e-08\\
599.99	8.73668930083393e-09\\
600	0\\
};
\addplot [color=mycolor17,solid,forget plot]
  table[row sep=crcr]{%
0.01	0.01\\
1.01	0.01\\
2.01	0.01\\
3.01	0.01\\
4.01	0.01\\
5.01	0.01\\
6.01	0.01\\
7.01	0.01\\
8.01	0.01\\
9.01	0.01\\
10.01	0.01\\
11.01	0.01\\
12.01	0.01\\
13.01	0.01\\
14.01	0.01\\
15.01	0.01\\
16.01	0.01\\
17.01	0.01\\
18.01	0.01\\
19.01	0.01\\
20.01	0.01\\
21.01	0.01\\
22.01	0.01\\
23.01	0.01\\
24.01	0.01\\
25.01	0.01\\
26.01	0.01\\
27.01	0.01\\
28.01	0.01\\
29.01	0.01\\
30.01	0.01\\
31.01	0.01\\
32.01	0.01\\
33.01	0.01\\
34.01	0.01\\
35.01	0.01\\
36.01	0.01\\
37.01	0.01\\
38.01	0.01\\
39.01	0.01\\
40.01	0.01\\
41.01	0.01\\
42.01	0.01\\
43.01	0.01\\
44.01	0.01\\
45.01	0.01\\
46.01	0.01\\
47.01	0.01\\
48.01	0.01\\
49.01	0.01\\
50.01	0.01\\
51.01	0.01\\
52.01	0.01\\
53.01	0.01\\
54.01	0.01\\
55.01	0.01\\
56.01	0.01\\
57.01	0.01\\
58.01	0.01\\
59.01	0.01\\
60.01	0.01\\
61.01	0.01\\
62.01	0.01\\
63.01	0.01\\
64.01	0.01\\
65.01	0.01\\
66.01	0.01\\
67.01	0.01\\
68.01	0.01\\
69.01	0.01\\
70.01	0.01\\
71.01	0.01\\
72.01	0.01\\
73.01	0.01\\
74.01	0.01\\
75.01	0.01\\
76.01	0.01\\
77.01	0.01\\
78.01	0.01\\
79.01	0.01\\
80.01	0.01\\
81.01	0.01\\
82.01	0.01\\
83.01	0.01\\
84.01	0.01\\
85.01	0.01\\
86.01	0.01\\
87.01	0.01\\
88.01	0.01\\
89.01	0.01\\
90.01	0.01\\
91.01	0.01\\
92.01	0.01\\
93.01	0.01\\
94.01	0.01\\
95.01	0.01\\
96.01	0.01\\
97.01	0.01\\
98.01	0.01\\
99.01	0.01\\
100.01	0.01\\
101.01	0.01\\
102.01	0.01\\
103.01	0.01\\
104.01	0.01\\
105.01	0.01\\
106.01	0.01\\
107.01	0.01\\
108.01	0.01\\
109.01	0.01\\
110.01	0.01\\
111.01	0.01\\
112.01	0.01\\
113.01	0.01\\
114.01	0.01\\
115.01	0.01\\
116.01	0.01\\
117.01	0.01\\
118.01	0.01\\
119.01	0.01\\
120.01	0.01\\
121.01	0.01\\
122.01	0.01\\
123.01	0.01\\
124.01	0.01\\
125.01	0.01\\
126.01	0.01\\
127.01	0.01\\
128.01	0.01\\
129.01	0.01\\
130.01	0.01\\
131.01	0.01\\
132.01	0.01\\
133.01	0.01\\
134.01	0.01\\
135.01	0.01\\
136.01	0.01\\
137.01	0.01\\
138.01	0.01\\
139.01	0.01\\
140.01	0.01\\
141.01	0.01\\
142.01	0.01\\
143.01	0.01\\
144.01	0.01\\
145.01	0.01\\
146.01	0.01\\
147.01	0.01\\
148.01	0.01\\
149.01	0.01\\
150.01	0.01\\
151.01	0.01\\
152.01	0.01\\
153.01	0.01\\
154.01	0.01\\
155.01	0.01\\
156.01	0.01\\
157.01	0.01\\
158.01	0.01\\
159.01	0.01\\
160.01	0.01\\
161.01	0.01\\
162.01	0.01\\
163.01	0.01\\
164.01	0.01\\
165.01	0.01\\
166.01	0.01\\
167.01	0.01\\
168.01	0.01\\
169.01	0.01\\
170.01	0.01\\
171.01	0.01\\
172.01	0.01\\
173.01	0.01\\
174.01	0.01\\
175.01	0.01\\
176.01	0.01\\
177.01	0.01\\
178.01	0.01\\
179.01	0.01\\
180.01	0.01\\
181.01	0.01\\
182.01	0.01\\
183.01	0.01\\
184.01	0.01\\
185.01	0.01\\
186.01	0.01\\
187.01	0.01\\
188.01	0.01\\
189.01	0.01\\
190.01	0.01\\
191.01	0.01\\
192.01	0.01\\
193.01	0.01\\
194.01	0.01\\
195.01	0.01\\
196.01	0.01\\
197.01	0.01\\
198.01	0.01\\
199.01	0.01\\
200.01	0.01\\
201.01	0.01\\
202.01	0.01\\
203.01	0.01\\
204.01	0.01\\
205.01	0.01\\
206.01	0.01\\
207.01	0.01\\
208.01	0.01\\
209.01	0.01\\
210.01	0.01\\
211.01	0.01\\
212.01	0.01\\
213.01	0.01\\
214.01	0.01\\
215.01	0.01\\
216.01	0.01\\
217.01	0.01\\
218.01	0.01\\
219.01	0.01\\
220.01	0.01\\
221.01	0.01\\
222.01	0.01\\
223.01	0.01\\
224.01	0.01\\
225.01	0.01\\
226.01	0.01\\
227.01	0.01\\
228.01	0.01\\
229.01	0.01\\
230.01	0.01\\
231.01	0.01\\
232.01	0.01\\
233.01	0.01\\
234.01	0.01\\
235.01	0.01\\
236.01	0.01\\
237.01	0.01\\
238.01	0.01\\
239.01	0.01\\
240.01	0.01\\
241.01	0.01\\
242.01	0.01\\
243.01	0.01\\
244.01	0.01\\
245.01	0.01\\
246.01	0.01\\
247.01	0.01\\
248.01	0.01\\
249.01	0.01\\
250.01	0.01\\
251.01	0.01\\
252.01	0.01\\
253.01	0.01\\
254.01	0.01\\
255.01	0.01\\
256.01	0.01\\
257.01	0.01\\
258.01	0.01\\
259.01	0.01\\
260.01	0.01\\
261.01	0.01\\
262.01	0.01\\
263.01	0.01\\
264.01	0.01\\
265.01	0.01\\
266.01	0.01\\
267.01	0.01\\
268.01	0.01\\
269.01	0.01\\
270.01	0.01\\
271.01	0.01\\
272.01	0.01\\
273.01	0.01\\
274.01	0.01\\
275.01	0.01\\
276.01	0.01\\
277.01	0.01\\
278.01	0.01\\
279.01	0.01\\
280.01	0.01\\
281.01	0.01\\
282.01	0.01\\
283.01	0.01\\
284.01	0.01\\
285.01	0.01\\
286.01	0.01\\
287.01	0.01\\
288.01	0.01\\
289.01	0.01\\
290.01	0.01\\
291.01	0.01\\
292.01	0.01\\
293.01	0.01\\
294.01	0.01\\
295.01	0.01\\
296.01	0.01\\
297.01	0.01\\
298.01	0.01\\
299.01	0.01\\
300.01	0.01\\
301.01	0.01\\
302.01	0.01\\
303.01	0.01\\
304.01	0.01\\
305.01	0.01\\
306.01	0.01\\
307.01	0.01\\
308.01	0.01\\
309.01	0.01\\
310.01	0.01\\
311.01	0.01\\
312.01	0.01\\
313.01	0.01\\
314.01	0.01\\
315.01	0.01\\
316.01	0.01\\
317.01	0.01\\
318.01	0.01\\
319.01	0.01\\
320.01	0.01\\
321.01	0.01\\
322.01	0.01\\
323.01	0.01\\
324.01	0.01\\
325.01	0.01\\
326.01	0.01\\
327.01	0.01\\
328.01	0.01\\
329.01	0.01\\
330.01	0.01\\
331.01	0.01\\
332.01	0.01\\
333.01	0.01\\
334.01	0.01\\
335.01	0.01\\
336.01	0.01\\
337.01	0.01\\
338.01	0.01\\
339.01	0.01\\
340.01	0.01\\
341.01	0.01\\
342.01	0.01\\
343.01	0.01\\
344.01	0.01\\
345.01	0.01\\
346.01	0.01\\
347.01	0.01\\
348.01	0.01\\
349.01	0.01\\
350.01	0.01\\
351.01	0.01\\
352.01	0.01\\
353.01	0.01\\
354.01	0.01\\
355.01	0.01\\
356.01	0.01\\
357.01	0.01\\
358.01	0.01\\
359.01	0.01\\
360.01	0.01\\
361.01	0.01\\
362.01	0.01\\
363.01	0.01\\
364.01	0.01\\
365.01	0.01\\
366.01	0.01\\
367.01	0.01\\
368.01	0.01\\
369.01	0.01\\
370.01	0.01\\
371.01	0.01\\
372.01	0.01\\
373.01	0.01\\
374.01	0.01\\
375.01	0.01\\
376.01	0.01\\
377.01	0.01\\
378.01	0.01\\
379.01	0.01\\
380.01	0.01\\
381.01	0.01\\
382.01	0.01\\
383.01	0.01\\
384.01	0.01\\
385.01	0.01\\
386.01	0.01\\
387.01	0.01\\
388.01	0.01\\
389.01	0.01\\
390.01	0.01\\
391.01	0.01\\
392.01	0.01\\
393.01	0.01\\
394.01	0.01\\
395.01	0.01\\
396.01	0.01\\
397.01	0.01\\
398.01	0.01\\
399.01	0.01\\
400.01	0.01\\
401.01	0.01\\
402.01	0.01\\
403.01	0.01\\
404.01	0.01\\
405.01	0.01\\
406.01	0.01\\
407.01	0.01\\
408.01	0.01\\
409.01	0.01\\
410.01	0.01\\
411.01	0.01\\
412.01	0.01\\
413.01	0.01\\
414.01	0.01\\
415.01	0.01\\
416.01	0.01\\
417.01	0.01\\
418.01	0.01\\
419.01	0.01\\
420.01	0.01\\
421.01	0.01\\
422.01	0.01\\
423.01	0.01\\
424.01	0.01\\
425.01	0.01\\
426.01	0.01\\
427.01	0.01\\
428.01	0.01\\
429.01	0.01\\
430.01	0.01\\
431.01	0.01\\
432.01	0.01\\
433.01	0.01\\
434.01	0.01\\
435.01	0.01\\
436.01	0.01\\
437.01	0.01\\
438.01	0.01\\
439.01	0.01\\
440.01	0.01\\
441.01	0.01\\
442.01	0.01\\
443.01	0.01\\
444.01	0.01\\
445.01	0.01\\
446.01	0.01\\
447.01	0.01\\
448.01	0.01\\
449.01	0.01\\
450.01	0.01\\
451.01	0.01\\
452.01	0.01\\
453.01	0.01\\
454.01	0.01\\
455.01	0.01\\
456.01	0.01\\
457.01	0.01\\
458.01	0.01\\
459.01	0.01\\
460.01	0.01\\
461.01	0.00999078095185203\\
462.01	0.00997419417939734\\
463.01	0.009957082611733\\
464.01	0.00993942808881208\\
465.01	0.00992121208632641\\
466.01	0.00990241583879067\\
467.01	0.00988302051352144\\
468.01	0.00986300745183245\\
469.01	0.00984235849873459\\
470.01	0.00982105644896362\\
471.01	0.0097990856459315\\
472.01	0.00977643278174111\\
473.01	0.00975308797251662\\
474.01	0.00972904630606034\\
475.01	0.00970430778655313\\
476.01	0.00967885773352897\\
477.01	0.00965267082451434\\
478.01	0.00962572037708996\\
479.01	0.00959797835263745\\
480.01	0.00956941526577682\\
481.01	0.00954000008832665\\
482.01	0.00950970014810938\\
483.01	0.00947848102321919\\
484.01	0.00944630643266237\\
485.01	0.00941313812349246\\
486.01	0.00937893574377376\\
487.01	0.00934365667464908\\
488.01	0.00930725604099166\\
489.01	0.00926968669288044\\
490.01	0.00923089904542123\\
491.01	0.00919084104334817\\
492.01	0.0091494581801977\\
493.01	0.00910669359017154\\
494.01	0.00906248826125746\\
495.01	0.00901678156923852\\
496.01	0.0089695132699148\\
497.01	0.00892062368141041\\
498.01	0.0088700333650678\\
499.01	0.00881764971181154\\
500.01	0.00876337149402897\\
501.01	0.0087070876953619\\
502.01	0.0086486761348062\\
503.01	0.00858800184576233\\
504.01	0.00852491515234896\\
505.01	0.00845924938584853\\
506.01	0.00839081825084369\\
507.01	0.00831941260721302\\
508.01	0.00824479661368092\\
509.01	0.00816670313252492\\
510.01	0.00808482820202401\\
511.01	0.00799882436587168\\
512.01	0.00790829259769038\\
513.01	0.00781277249428077\\
514.01	0.00771173033149313\\
515.01	0.00760454449104528\\
516.01	0.00749048781583137\\
517.01	0.00736870869453522\\
518.01	0.00723963818854821\\
519.01	0.00710597811992642\\
520.01	0.00696760229751159\\
521.01	0.00682422000099556\\
522.01	0.00667551254793456\\
523.01	0.00659587815347017\\
524.01	0.00651633098827587\\
525.01	0.00643473006207726\\
526.01	0.00635106660776621\\
527.01	0.00626534366701773\\
528.01	0.00617757882137513\\
529.01	0.00608780722930537\\
530.01	0.00599608567697558\\
531.01	0.00590249750060823\\
532.01	0.00580715853831556\\
533.01	0.00571022394240866\\
534.01	0.0056118929793264\\
535.01	0.00551242338902114\\
536.01	0.00541215254983005\\
537.01	0.00531151088970588\\
538.01	0.00521102795614884\\
539.01	0.00511137278512313\\
540.01	0.00501339837207893\\
541.01	0.00491819638302972\\
542.01	0.00482625485291065\\
543.01	0.0047326019711171\\
544.01	0.00463623227679512\\
545.01	0.00453723161982669\\
546.01	0.00443576164286419\\
547.01	0.00433208802327658\\
548.01	0.00422661498624084\\
549.01	0.00411993151685858\\
550.01	0.00401290011585353\\
551.01	0.00390721634275333\\
552.01	0.0038026967159869\\
553.01	0.00369958105289584\\
554.01	0.00359821419173106\\
555.01	0.00349895359268961\\
556.01	0.00340215802666711\\
557.01	0.00330817029183982\\
558.01	0.00321729180630797\\
559.01	0.00312974662100248\\
560.01	0.00304562319959933\\
561.01	0.0029648310383562\\
562.01	0.00288723971163634\\
563.01	0.00281168088604776\\
564.01	0.00273659112181641\\
565.01	0.00266192996599381\\
566.01	0.00258771427401347\\
567.01	0.00251394505248555\\
568.01	0.00244063027530275\\
569.01	0.00236774713459566\\
570.01	0.00229523575951096\\
571.01	0.00222299396948937\\
572.01	0.00215087054873885\\
573.01	0.00207867101495317\\
574.01	0.00200620911203144\\
575.01	0.00193337366729925\\
576.01	0.00186014255824003\\
577.01	0.00178649907178294\\
578.01	0.00171242998631992\\
579.01	0.0016379206162627\\
580.01	0.0015629515820556\\
581.01	0.00148749997079724\\
582.01	0.00141154117821388\\
583.01	0.00133505131726857\\
584.01	0.00125800997726646\\
585.01	0.00118040287913029\\
586.01	0.00110222304799643\\
587.01	0.00102347123973567\\
588.01	0.00094415206962192\\
589.01	0.000864266394493165\\
590.01	0.00078381051162838\\
591.01	0.00070277603982059\\
592.01	0.000621149948535656\\
593.01	0.000538914720361251\\
594.01	0.000456048589420978\\
595.01	0.000372525732669495\\
596.01	0.000288316195459062\\
597.01	0.00020338520208954\\
598.01	0.000117691340829394\\
599.01	3.2249382697221e-05\\
599.02	3.15980747054457e-05\\
599.03	3.09676859235018e-05\\
599.04	3.03584576151362e-05\\
599.05	2.97705837207663e-05\\
599.06	2.92042621379707e-05\\
599.07	2.86596948884162e-05\\
599.08	2.8123912998973e-05\\
599.09	2.75930227461897e-05\\
599.1	2.70667931478105e-05\\
599.11	2.65452441665006e-05\\
599.12	2.60282881461519e-05\\
599.13	2.55154035494164e-05\\
599.14	2.50066056569926e-05\\
599.15	2.45019084058242e-05\\
599.16	2.40013242945435e-05\\
599.17	2.35048471433803e-05\\
599.18	2.30124445947163e-05\\
599.19	2.25241241168985e-05\\
599.2	2.20398912888203e-05\\
599.21	2.15597507792053e-05\\
599.22	2.10837069131182e-05\\
599.23	2.0611761905133e-05\\
599.24	2.01439157217088e-05\\
599.25	1.96801659357645e-05\\
599.26	1.92205075730031e-05\\
599.27	1.87649527270909e-05\\
599.28	1.83135392298733e-05\\
599.29	1.78663051708071e-05\\
599.3	1.74232888957779e-05\\
599.31	1.6984530027039e-05\\
599.32	1.65500694928791e-05\\
599.33	1.61199486044834e-05\\
599.34	1.56942090620352e-05\\
599.35	1.52728931138019e-05\\
599.36	1.48560434799095e-05\\
599.37	1.44437033046136e-05\\
599.38	1.40359161648531e-05\\
599.39	1.36327268429586e-05\\
599.4	1.32341836384959e-05\\
599.41	1.28403353542913e-05\\
599.42	1.24512313076607e-05\\
599.43	1.2066921342313e-05\\
599.44	1.16874558409707e-05\\
599.45	1.13128857387526e-05\\
599.46	1.09432625373523e-05\\
599.47	1.05786383200971e-05\\
599.48	1.02190657679229e-05\\
599.49	9.86459817164075e-06\\
599.5	9.51528944691422e-06\\
599.51	9.17119415258169e-06\\
599.52	8.83236751010076e-06\\
599.53	8.4988654011544e-06\\
599.54	8.17074426504201e-06\\
599.55	7.84806110491927e-06\\
599.56	7.53087349160408e-06\\
599.57	7.21923956508233e-06\\
599.58	6.91321804026715e-06\\
599.59	6.6128682119828e-06\\
599.6	6.31824996075517e-06\\
599.61	6.02942375867339e-06\\
599.62	5.74645067535212e-06\\
599.63	5.46939238393711e-06\\
599.64	5.19831116718369e-06\\
599.65	4.93326992361681e-06\\
599.66	4.67433217375343e-06\\
599.67	4.42156206638573e-06\\
599.68	4.17502438496142e-06\\
599.69	3.93478455401261e-06\\
599.7	3.70090864566797e-06\\
599.71	3.47346338623079e-06\\
599.72	3.2525161628473e-06\\
599.73	3.03813503022173e-06\\
599.74	2.83038871743899e-06\\
599.75	2.62934663483247e-06\\
599.76	2.43507888095067e-06\\
599.77	2.2476562495672e-06\\
599.78	2.06715023678798e-06\\
599.79	1.89363304820347e-06\\
599.8	1.72717760612943e-06\\
599.81	1.56785755689968e-06\\
599.82	1.41574727823349e-06\\
599.83	1.27092188665495e-06\\
599.84	1.1334572449697e-06\\
599.85	1.00342996981438e-06\\
599.86	8.80917439232201e-07\\
599.87	7.65997800316123e-07\\
599.88	6.58749976879813e-07\\
599.89	5.59253677163279e-07\\
599.9	4.67589401585353e-07\\
599.91	3.83838450489227e-07\\
599.92	3.08082931920958e-07\\
599.93	2.40405769407967e-07\\
599.94	1.8089070970978e-07\\
599.95	1.2962233057745e-07\\
599.96	8.66860484539933e-08\\
599.97	5.21681261331924e-08\\
599.98	2.61556803542173e-08\\
599.99	8.73668930083393e-09\\
600	0\\
};
\addplot [color=mycolor18,solid,forget plot]
  table[row sep=crcr]{%
0.01	0.0085282693244888\\
1.01	0.00852826887287535\\
2.01	0.00852826841151541\\
3.01	0.00852826794019706\\
4.01	0.00852826745870373\\
5.01	0.00852826696681405\\
6.01	0.00852826646430188\\
7.01	0.00852826595093601\\
8.01	0.00852826542648029\\
9.01	0.00852826489069324\\
10.01	0.0085282643433281\\
11.01	0.00852826378413266\\
12.01	0.00852826321284936\\
13.01	0.0085282626292147\\
14.01	0.00852826203295951\\
15.01	0.00852826142380871\\
16.01	0.00852826080148116\\
17.01	0.00852826016568943\\
18.01	0.00852825951613987\\
19.01	0.00852825885253234\\
20.01	0.00852825817456001\\
21.01	0.0085282574819093\\
22.01	0.00852825677425974\\
23.01	0.00852825605128382\\
24.01	0.00852825531264676\\
25.01	0.00852825455800638\\
26.01	0.00852825378701293\\
27.01	0.00852825299930898\\
28.01	0.00852825219452916\\
29.01	0.00852825137229997\\
30.01	0.0085282505322398\\
31.01	0.00852824967395849\\
32.01	0.00852824879705723\\
33.01	0.00852824790112842\\
34.01	0.00852824698575541\\
35.01	0.00852824605051238\\
36.01	0.008528245094964\\
37.01	0.00852824411866531\\
38.01	0.00852824312116146\\
39.01	0.00852824210198756\\
40.01	0.00852824106066839\\
41.01	0.00852823999671815\\
42.01	0.00852823890964029\\
43.01	0.00852823779892717\\
44.01	0.00852823666405996\\
45.01	0.00852823550450822\\
46.01	0.00852823431972977\\
47.01	0.00852823310917022\\
48.01	0.00852823187226311\\
49.01	0.00852823060842911\\
50.01	0.00852822931707615\\
51.01	0.00852822799759891\\
52.01	0.00852822664937859\\
53.01	0.00852822527178259\\
54.01	0.00852822386416421\\
55.01	0.00852822242586225\\
56.01	0.00852822095620088\\
57.01	0.00852821945448912\\
58.01	0.00852821792002046\\
59.01	0.00852821635207279\\
60.01	0.00852821474990769\\
61.01	0.0085282131127703\\
62.01	0.00852821143988891\\
63.01	0.00852820973047442\\
64.01	0.0085282079837202\\
65.01	0.0085282061988014\\
66.01	0.00852820437487486\\
67.01	0.0085282025110784\\
68.01	0.00852820060653054\\
69.01	0.00852819866033\\
70.01	0.0085281966715553\\
71.01	0.00852819463926426\\
72.01	0.00852819256249349\\
73.01	0.00852819044025805\\
74.01	0.00852818827155076\\
75.01	0.00852818605534186\\
76.01	0.0085281837905783\\
77.01	0.00852818147618344\\
78.01	0.00852817911105634\\
79.01	0.0085281766940713\\
80.01	0.00852817422407719\\
81.01	0.00852817169989689\\
82.01	0.0085281691203268\\
83.01	0.00852816648413615\\
84.01	0.00852816379006628\\
85.01	0.00852816103683019\\
86.01	0.00852815822311177\\
87.01	0.00852815534756507\\
88.01	0.00852815240881381\\
89.01	0.00852814940545044\\
90.01	0.00852814633603559\\
91.01	0.00852814319909731\\
92.01	0.00852813999313021\\
93.01	0.00852813671659478\\
94.01	0.00852813336791666\\
95.01	0.00852812994548565\\
96.01	0.00852812644765504\\
97.01	0.00852812287274061\\
98.01	0.00852811921901989\\
99.01	0.00852811548473132\\
100.01	0.00852811166807319\\
101.01	0.00852810776720269\\
102.01	0.00852810378023512\\
103.01	0.00852809970524282\\
104.01	0.00852809554025416\\
105.01	0.00852809128325249\\
106.01	0.0085280869321752\\
107.01	0.00852808248491249\\
108.01	0.00852807793930634\\
109.01	0.00852807329314944\\
110.01	0.00852806854418401\\
111.01	0.0085280636901005\\
112.01	0.00852805872853654\\
113.01	0.00852805365707559\\
114.01	0.0085280484732458\\
115.01	0.0085280431745185\\
116.01	0.00852803775830707\\
117.01	0.00852803222196547\\
118.01	0.0085280265627868\\
119.01	0.008528020778002\\
120.01	0.00852801486477833\\
121.01	0.00852800882021782\\
122.01	0.00852800264135578\\
123.01	0.00852799632515918\\
124.01	0.00852798986852503\\
125.01	0.00852798326827892\\
126.01	0.00852797652117303\\
127.01	0.0085279696238846\\
128.01	0.00852796257301403\\
129.01	0.00852795536508316\\
130.01	0.00852794799653337\\
131.01	0.00852794046372366\\
132.01	0.0085279327629287\\
133.01	0.00852792489033672\\
134.01	0.0085279168420477\\
135.01	0.00852790861407104\\
136.01	0.00852790020232355\\
137.01	0.00852789160262709\\
138.01	0.00852788281070649\\
139.01	0.0085278738221871\\
140.01	0.00852786463259244\\
141.01	0.00852785523734174\\
142.01	0.00852784563174758\\
143.01	0.00852783581101312\\
144.01	0.00852782577022966\\
145.01	0.00852781550437395\\
146.01	0.00852780500830533\\
147.01	0.00852779427676297\\
148.01	0.00852778330436295\\
149.01	0.00852777208559542\\
150.01	0.00852776061482139\\
151.01	0.00852774888626977\\
152.01	0.00852773689403411\\
153.01	0.00852772463206939\\
154.01	0.00852771209418864\\
155.01	0.00852769927405947\\
156.01	0.00852768616520067\\
157.01	0.0085276727609785\\
158.01	0.00852765905460308\\
159.01	0.00852764503912454\\
160.01	0.0085276307074292\\
161.01	0.00852761605223562\\
162.01	0.00852760106609042\\
163.01	0.00852758574136424\\
164.01	0.00852757007024735\\
165.01	0.00852755404474531\\
166.01	0.00852753765667452\\
167.01	0.00852752089765755\\
168.01	0.00852750375911848\\
169.01	0.00852748623227795\\
170.01	0.00852746830814835\\
171.01	0.00852744997752857\\
172.01	0.008527431230999\\
173.01	0.00852741205891598\\
174.01	0.00852739245140631\\
175.01	0.00852737239836189\\
176.01	0.00852735188943368\\
177.01	0.00852733091402596\\
178.01	0.00852730946129019\\
179.01	0.0085272875201189\\
180.01	0.00852726507913922\\
181.01	0.00852724212670645\\
182.01	0.00852721865089733\\
183.01	0.00852719463950314\\
184.01	0.00852717008002282\\
185.01	0.00852714495965563\\
186.01	0.00852711926529373\\
187.01	0.00852709298351479\\
188.01	0.00852706610057399\\
189.01	0.00852703860239624\\
190.01	0.00852701047456802\\
191.01	0.00852698170232879\\
192.01	0.00852695227056261\\
193.01	0.00852692216378937\\
194.01	0.00852689136615564\\
195.01	0.00852685986142535\\
196.01	0.00852682763297052\\
197.01	0.00852679466376136\\
198.01	0.0085267609363564\\
199.01	0.00852672643289213\\
200.01	0.00852669113507269\\
201.01	0.00852665502415893\\
202.01	0.00852661808095746\\
203.01	0.00852658028580938\\
204.01	0.00852654161857869\\
205.01	0.00852650205864015\\
206.01	0.0085264615848675\\
207.01	0.00852642017562057\\
208.01	0.00852637780873278\\
209.01	0.00852633446149765\\
210.01	0.00852629011065562\\
211.01	0.00852624473238001\\
212.01	0.00852619830226295\\
213.01	0.00852615079530072\\
214.01	0.00852610218587904\\
215.01	0.0085260524477576\\
216.01	0.00852600155405443\\
217.01	0.00852594947722981\\
218.01	0.0085258961890697\\
219.01	0.00852584166066901\\
220.01	0.00852578586241413\\
221.01	0.00852572876396513\\
222.01	0.00852567033423764\\
223.01	0.00852561054138394\\
224.01	0.00852554935277405\\
225.01	0.00852548673497577\\
226.01	0.00852542265373473\\
227.01	0.00852535707395371\\
228.01	0.00852528995967124\\
229.01	0.00852522127404003\\
230.01	0.0085251509793045\\
231.01	0.00852507903677823\\
232.01	0.0085250054068201\\
233.01	0.00852493004881053\\
234.01	0.00852485292112682\\
235.01	0.00852477398111765\\
236.01	0.00852469318507734\\
237.01	0.00852461048821923\\
238.01	0.00852452584464841\\
239.01	0.00852443920733379\\
240.01	0.00852435052807936\\
241.01	0.00852425975749496\\
242.01	0.00852416684496602\\
243.01	0.00852407173862272\\
244.01	0.00852397438530831\\
245.01	0.0085238747305466\\
246.01	0.00852377271850874\\
247.01	0.00852366829197907\\
248.01	0.00852356139232001\\
249.01	0.0085234519594363\\
250.01	0.00852333993173823\\
251.01	0.00852322524610382\\
252.01	0.00852310783784022\\
253.01	0.00852298764064399\\
254.01	0.00852286458656052\\
255.01	0.00852273860594251\\
256.01	0.00852260962740696\\
257.01	0.00852247757779165\\
258.01	0.00852234238211008\\
259.01	0.00852220396350558\\
260.01	0.00852206224320412\\
261.01	0.00852191714046591\\
262.01	0.00852176857253592\\
263.01	0.00852161645459301\\
264.01	0.00852146069969794\\
265.01	0.00852130121873993\\
266.01	0.00852113792038201\\
267.01	0.00852097071100482\\
268.01	0.0085207994946494\\
269.01	0.00852062417295801\\
270.01	0.00852044464511388\\
271.01	0.00852026080777938\\
272.01	0.00852007255503248\\
273.01	0.00851987977830176\\
274.01	0.00851968236629997\\
275.01	0.00851948020495553\\
276.01	0.00851927317734264\\
277.01	0.00851906116360957\\
278.01	0.00851884404090515\\
279.01	0.00851862168330348\\
280.01	0.0085183939617267\\
281.01	0.00851816074386581\\
282.01	0.00851792189409971\\
283.01	0.00851767727341198\\
284.01	0.00851742673930586\\
285.01	0.00851717014571683\\
286.01	0.00851690734292317\\
287.01	0.00851663817745436\\
288.01	0.00851636249199709\\
289.01	0.00851608012529895\\
290.01	0.00851579091206974\\
291.01	0.00851549468288027\\
292.01	0.00851519126405875\\
293.01	0.00851488047758453\\
294.01	0.00851456214097902\\
295.01	0.00851423606719427\\
296.01	0.00851390206449821\\
297.01	0.00851355993635757\\
298.01	0.00851320948131761\\
299.01	0.00851285049287855\\
300.01	0.00851248275936947\\
301.01	0.00851210606381833\\
302.01	0.00851172018381939\\
303.01	0.00851132489139673\\
304.01	0.00851091995286449\\
305.01	0.00851050512868361\\
306.01	0.00851008017331476\\
307.01	0.00850964483506748\\
308.01	0.00850919885594564\\
309.01	0.00850874197148854\\
310.01	0.00850827391060823\\
311.01	0.00850779439542229\\
312.01	0.00850730314108252\\
313.01	0.0085067998555988\\
314.01	0.00850628423965839\\
315.01	0.00850575598644074\\
316.01	0.00850521478142667\\
317.01	0.00850466030220307\\
318.01	0.00850409221826213\\
319.01	0.00850351019079479\\
320.01	0.00850291387247883\\
321.01	0.00850230290726128\\
322.01	0.00850167693013432\\
323.01	0.0085010355669054\\
324.01	0.00850037843396062\\
325.01	0.00849970513802145\\
326.01	0.0084990152758946\\
327.01	0.00849830843421463\\
328.01	0.00849758418917936\\
329.01	0.00849684210627718\\
330.01	0.00849608174000655\\
331.01	0.00849530263358736\\
332.01	0.00849450431866355\\
333.01	0.00849368631499666\\
334.01	0.0084928481301504\\
335.01	0.00849198925916552\\
336.01	0.00849110918422448\\
337.01	0.00849020737430603\\
338.01	0.00848928328482888\\
339.01	0.00848833635728427\\
340.01	0.00848736601885668\\
341.01	0.00848637168203296\\
342.01	0.00848535274419822\\
343.01	0.00848430858721917\\
344.01	0.00848323857701382\\
345.01	0.00848214206310694\\
346.01	0.00848101837817107\\
347.01	0.00847986683755223\\
348.01	0.00847868673877964\\
349.01	0.00847747736105926\\
350.01	0.00847623796475034\\
351.01	0.00847496779082389\\
352.01	0.00847366606030291\\
353.01	0.00847233197368387\\
354.01	0.00847096471033751\\
355.01	0.00846956342789008\\
356.01	0.00846812726158228\\
357.01	0.00846665532360672\\
358.01	0.00846514670242193\\
359.01	0.00846360046204281\\
360.01	0.00846201564130635\\
361.01	0.00846039125311161\\
362.01	0.00845872628363337\\
363.01	0.00845701969150814\\
364.01	0.00845527040699201\\
365.01	0.00845347733108871\\
366.01	0.00845163933464759\\
367.01	0.00844975525743003\\
368.01	0.00844782390714294\\
369.01	0.00844584405843904\\
370.01	0.00844381445188201\\
371.01	0.00844173379287559\\
372.01	0.00843960075055563\\
373.01	0.00843741395664327\\
374.01	0.0084351720042585\\
375.01	0.00843287344669222\\
376.01	0.00843051679613575\\
377.01	0.00842810052236599\\
378.01	0.00842562305138452\\
379.01	0.00842308276400934\\
380.01	0.00842047799441703\\
381.01	0.00841780702863358\\
382.01	0.00841506810297179\\
383.01	0.00841225940241283\\
384.01	0.00840937905892985\\
385.01	0.00840642514975087\\
386.01	0.00840339569555807\\
387.01	0.00840028865862046\\
388.01	0.00839710194085653\\
389.01	0.00839383338182355\\
390.01	0.00839048075662889\\
391.01	0.00838704177375985\\
392.01	0.00838351407282648\\
393.01	0.00837989522221236\\
394.01	0.00837618271662836\\
395.01	0.00837237397456203\\
396.01	0.00836846633561692\\
397.01	0.0083644570577336\\
398.01	0.00836034331428519\\
399.01	0.00835612219103809\\
400.01	0.00835179068296904\\
401.01	0.00834734569092825\\
402.01	0.00834278401813774\\
403.01	0.0083381023665139\\
404.01	0.00833329733280185\\
405.01	0.00832836540450924\\
406.01	0.00832330295562662\\
407.01	0.00831810624212161\\
408.01	0.00831277139719408\\
409.01	0.00830729442628117\\
410.01	0.0083016712018026\\
411.01	0.00829589745764005\\
412.01	0.00828996878335004\\
413.01	0.00828388061811719\\
414.01	0.00827762824445846\\
415.01	0.00827120678153902\\
416.01	0.00826461117320351\\
417.01	0.0082578361846318\\
418.01	0.00825087643389198\\
419.01	0.00824372635047618\\
420.01	0.00823638016567071\\
421.01	0.00822883193369906\\
422.01	0.00822107557696015\\
423.01	0.00821310472571571\\
424.01	0.00820491269494183\\
425.01	0.00819649253760645\\
426.01	0.00818783703140281\\
427.01	0.00817893866291245\\
428.01	0.00816978961040823\\
429.01	0.00816038172513833\\
430.01	0.00815070651090978\\
431.01	0.00814075510176216\\
432.01	0.00813051823749057\\
433.01	0.00811998623674044\\
434.01	0.00810914896735115\\
435.01	0.00809799581357453\\
436.01	0.00808651563973082\\
437.01	0.00807469674979188\\
438.01	0.00806252684229236\\
439.01	0.00804999295986385\\
440.01	0.00803708143255964\\
441.01	0.00802377781398531\\
442.01	0.00801006680906401\\
443.01	0.00799593219204271\\
444.01	0.00798135671307224\\
445.01	0.00796632199136345\\
446.01	0.00795080839251616\\
447.01	0.00793479488712468\\
448.01	0.00791825888715257\\
449.01	0.00790117605582211\\
450.01	0.00788352008580184\\
451.01	0.00786526243873771\\
452.01	0.00784637203025313\\
453.01	0.00782681487280826\\
454.01	0.00780655371765629\\
455.01	0.00778554744100704\\
456.01	0.00776375037486257\\
457.01	0.00774111152438106\\
458.01	0.00771757362418356\\
459.01	0.007693072062799\\
460.01	0.00766753430450629\\
461.01	0.00765013319258523\\
462.01	0.00763908808592548\\
463.01	0.00762750964586218\\
464.01	0.00761534584623699\\
465.01	0.00760253654626921\\
466.01	0.00758901191196083\\
467.01	0.00757469048882546\\
468.01	0.0075594768419738\\
469.01	0.00754325865786523\\
470.01	0.00752590317422022\\
471.01	0.0075072527753302\\
472.01	0.0074871195413191\\
473.01	0.00746527851243589\\
474.01	0.00744145998050491\\
475.01	0.00741583934100435\\
476.01	0.00738948680289242\\
477.01	0.00736244307460696\\
478.01	0.00733469087315198\\
479.01	0.00730621265144998\\
480.01	0.0072769906088408\\
481.01	0.0072470067004919\\
482.01	0.00721624264423079\\
483.01	0.00718467992263655\\
484.01	0.00715229977706487\\
485.01	0.00711908318482394\\
486.01	0.00708501073440597\\
487.01	0.00705006155323819\\
488.01	0.00701421368066979\\
489.01	0.00697744713463668\\
490.01	0.00693974183472198\\
491.01	0.00690107722893671\\
492.01	0.00686143209391965\\
493.01	0.00682078424647268\\
494.01	0.00677911014682784\\
495.01	0.0067363844159549\\
496.01	0.00669257897406453\\
497.01	0.00664766209734195\\
498.01	0.00660161409848995\\
499.01	0.006554423046006\\
500.01	0.00650608067788641\\
501.01	0.00645658341378783\\
502.01	0.0064059336528664\\
503.01	0.00635414145503738\\
504.01	0.00630122645225831\\
505.01	0.00624721976066043\\
506.01	0.00619216778140031\\
507.01	0.00613613602116493\\
508.01	0.00607921259138708\\
509.01	0.00602151344421225\\
510.01	0.00596318885914\\
511.01	0.00590443146170561\\
512.01	0.0058454861538424\\
513.01	0.00578666243426488\\
514.01	0.00572834971386359\\
515.01	0.00567103640614396\\
516.01	0.00561533396415977\\
517.01	0.0055620088737059\\
518.01	0.00551059015308814\\
519.01	0.00545824927896969\\
520.01	0.0054049202581716\\
521.01	0.00535069205361565\\
522.01	0.00529567320000898\\
523.01	0.00523949329040571\\
524.01	0.00518161246188112\\
525.01	0.00512196036795267\\
526.01	0.00506046576669617\\
527.01	0.00499706214664321\\
528.01	0.00493169862185053\\
529.01	0.0048643375711972\\
530.01	0.0047949600813584\\
531.01	0.00472357368012475\\
532.01	0.00465022245915389\\
533.01	0.00457499852317799\\
534.01	0.00449799425098407\\
535.01	0.00441925191459439\\
536.01	0.00433906268995177\\
537.01	0.00425833640930586\\
538.01	0.00417838431595103\\
539.01	0.00409939265699864\\
540.01	0.00402143966493794\\
541.01	0.00394459238813641\\
542.01	0.00386891012544723\\
543.01	0.00379451669365658\\
544.01	0.00372165678674702\\
545.01	0.00365058060888114\\
546.01	0.00358150836090378\\
547.01	0.00351463948028555\\
548.01	0.00345014330083666\\
549.01	0.00338809864123182\\
550.01	0.00332842287084302\\
551.01	0.00326978705576702\\
552.01	0.0032115050237175\\
553.01	0.0031536159779063\\
554.01	0.00309614412897075\\
555.01	0.0030390947138821\\
556.01	0.00298245005115764\\
557.01	0.00292616583595391\\
558.01	0.00287016822054579\\
559.01	0.00281435251935063\\
560.01	0.00275858482690329\\
561.01	0.00270270824872089\\
562.01	0.00264655033766167\\
563.01	0.00258995010948459\\
564.01	0.00253284519234311\\
565.01	0.00247521006594279\\
566.01	0.00241701529277463\\
567.01	0.00235822741438245\\
568.01	0.00229880868214501\\
569.01	0.00223871784692923\\
570.01	0.00217791174383267\\
571.01	0.00211634742053402\\
572.01	0.0020539848162932\\
573.01	0.00199078985346574\\
574.01	0.00192673690423219\\
575.01	0.00186180846663631\\
576.01	0.00179598983847555\\
577.01	0.00172926700343405\\
578.01	0.00166162680595677\\
579.01	0.00159305719081834\\
580.01	0.00152354764606489\\
581.01	0.00145308967998356\\
582.01	0.0013816772350871\\
583.01	0.00130930697684359\\
584.01	0.00123597839125735\\
585.01	0.00116169364078332\\
586.01	0.00108645721274621\\
587.01	0.00101027566830084\\
588.01	0.000933157587584601\\
589.01	0.000855113653909664\\
590.01	0.000776156728517496\\
591.01	0.000696301791282395\\
592.01	0.000615565671439002\\
593.01	0.000533966471584386\\
594.01	0.000451522561386629\\
595.01	0.000368250987049191\\
596.01	0.000284165109520231\\
597.01	0.000199271250183436\\
598.01	0.000113564089302298\\
599.01	3.18261402467001e-05\\
599.02	3.12794051966058e-05\\
599.03	3.07361362371396e-05\\
599.04	3.01962768802927e-05\\
599.05	2.96598193405082e-05\\
599.06	2.91267532562291e-05\\
599.07	2.85970655242371e-05\\
599.08	2.80707764810702e-05\\
599.09	2.75479164762726e-05\\
599.1	2.70285167089327e-05\\
599.11	2.65126085881601e-05\\
599.12	2.60002240642203e-05\\
599.13	2.54913970200877e-05\\
599.14	2.49861616816692e-05\\
599.15	2.44845526258131e-05\\
599.16	2.39866047888303e-05\\
599.17	2.34923535378934e-05\\
599.18	2.30018347767025e-05\\
599.19	2.251508481242e-05\\
599.2	2.20321403675509e-05\\
599.21	2.15530385901663e-05\\
599.22	2.10778170631979e-05\\
599.23	2.06065138181385e-05\\
599.24	2.01391673495636e-05\\
599.25	1.96758166306349e-05\\
599.26	1.92165011295622e-05\\
599.27	1.87612607734713e-05\\
599.28	1.8310135886844e-05\\
599.29	1.78631671960919e-05\\
599.3	1.74203958342063e-05\\
599.31	1.69818633422789e-05\\
599.32	1.65476116708516e-05\\
599.33	1.61176831841251e-05\\
599.34	1.56921206641847e-05\\
599.35	1.52709673147209e-05\\
599.36	1.485426676507e-05\\
599.37	1.44420630744727e-05\\
599.38	1.40344007363812e-05\\
599.39	1.36313255098709e-05\\
599.4	1.3232886592919e-05\\
599.41	1.28391336731322e-05\\
599.42	1.24501169325771e-05\\
599.43	1.20658870526348e-05\\
599.44	1.16864952188697e-05\\
599.45	1.13119931259391e-05\\
599.46	1.09424329824918e-05\\
599.47	1.05778675161157e-05\\
599.48	1.02183499782645e-05\\
599.49	9.86393414922036e-06\\
599.5	9.51467434305575e-06\\
599.51	9.1706254125995e-06\\
599.52	8.83184275438965e-06\\
599.53	8.4983823136868e-06\\
599.54	8.17030058985514e-06\\
599.55	7.84765464179223e-06\\
599.56	7.53050209342451e-06\\
599.57	7.21890113926371e-06\\
599.58	6.91291055002208e-06\\
599.59	6.61258967827975e-06\\
599.6	6.31799846421283e-06\\
599.61	6.02919744138383e-06\\
599.62	5.74624774257564e-06\\
599.63	5.46921110568602e-06\\
599.64	5.19814987969512e-06\\
599.65	4.9331270306676e-06\\
599.66	4.67420614782936e-06\\
599.67	4.42145144970672e-06\\
599.68	4.17492779030207e-06\\
599.69	3.93470066536311e-06\\
599.7	3.70083621868862e-06\\
599.71	3.47340124850354e-06\\
599.72	3.25246321389479e-06\\
599.73	3.03809024131303e-06\\
599.74	2.83035113113513e-06\\
599.75	2.62931536429249e-06\\
599.76	2.43505310895328e-06\\
599.77	2.24763522728433e-06\\
599.78	2.06713328227207e-06\\
599.79	1.89361954460419e-06\\
599.8	1.72716699962938e-06\\
599.81	1.56784935437075e-06\\
599.82	1.41574104461396e-06\\
599.83	1.27091724206295e-06\\
599.84	1.13345386157197e-06\\
599.85	1.00342756843147e-06\\
599.86	8.80915785733682e-07\\
599.87	7.65996701814625e-07\\
599.88	6.58749277755027e-07\\
599.89	5.59253254961076e-07\\
599.9	4.67589162821483e-07\\
599.91	3.83838326433947e-07\\
599.92	3.08082874409671e-07\\
599.93	2.40405746740335e-07\\
599.94	1.80890702781294e-07\\
599.95	1.2962232925906e-07\\
599.96	8.66860484019516e-08\\
599.97	5.21681261349272e-08\\
599.98	2.61556803542173e-08\\
599.99	8.73668930083393e-09\\
600	0\\
};
\addplot [color=red!25!mycolor17,solid,forget plot]
  table[row sep=crcr]{%
0.01	0.00734126507339155\\
1.01	0.00734126455459817\\
2.01	0.00734126402465449\\
3.01	0.00734126348331942\\
4.01	0.00734126293034655\\
5.01	0.00734126236548402\\
6.01	0.0073412617884747\\
7.01	0.00734126119905562\\
8.01	0.00734126059695818\\
9.01	0.00734125998190793\\
10.01	0.00734125935362445\\
11.01	0.00734125871182122\\
12.01	0.00734125805620525\\
13.01	0.00734125738647739\\
14.01	0.00734125670233179\\
15.01	0.00734125600345608\\
16.01	0.00734125528953093\\
17.01	0.00734125456023013\\
18.01	0.00734125381522025\\
19.01	0.00734125305416054\\
20.01	0.00734125227670295\\
21.01	0.00734125148249162\\
22.01	0.00734125067116316\\
23.01	0.00734124984234594\\
24.01	0.00734124899566034\\
25.01	0.00734124813071843\\
26.01	0.00734124724712374\\
27.01	0.00734124634447111\\
28.01	0.0073412454223465\\
29.01	0.00734124448032689\\
30.01	0.00734124351797983\\
31.01	0.00734124253486344\\
32.01	0.00734124153052618\\
33.01	0.00734124050450658\\
34.01	0.00734123945633304\\
35.01	0.00734123838552357\\
36.01	0.00734123729158562\\
37.01	0.0073412361740158\\
38.01	0.00734123503229976\\
39.01	0.00734123386591169\\
40.01	0.0073412326743143\\
41.01	0.00734123145695845\\
42.01	0.0073412302132829\\
43.01	0.00734122894271422\\
44.01	0.00734122764466611\\
45.01	0.0073412263185395\\
46.01	0.0073412249637221\\
47.01	0.00734122357958816\\
48.01	0.00734122216549811\\
49.01	0.00734122072079827\\
50.01	0.00734121924482052\\
51.01	0.00734121773688207\\
52.01	0.00734121619628499\\
53.01	0.00734121462231603\\
54.01	0.00734121301424613\\
55.01	0.00734121137133014\\
56.01	0.00734120969280643\\
57.01	0.00734120797789652\\
58.01	0.00734120622580483\\
59.01	0.00734120443571808\\
60.01	0.00734120260680509\\
61.01	0.00734120073821619\\
62.01	0.00734119882908289\\
63.01	0.00734119687851762\\
64.01	0.00734119488561302\\
65.01	0.00734119284944171\\
66.01	0.0073411907690556\\
67.01	0.00734118864348575\\
68.01	0.00734118647174166\\
69.01	0.00734118425281084\\
70.01	0.00734118198565839\\
71.01	0.00734117966922635\\
72.01	0.00734117730243338\\
73.01	0.00734117488417396\\
74.01	0.0073411724133181\\
75.01	0.00734116988871071\\
76.01	0.00734116730917097\\
77.01	0.00734116467349179\\
78.01	0.00734116198043919\\
79.01	0.00734115922875178\\
80.01	0.00734115641713994\\
81.01	0.00734115354428552\\
82.01	0.00734115060884088\\
83.01	0.00734114760942827\\
84.01	0.00734114454463934\\
85.01	0.00734114141303422\\
86.01	0.00734113821314096\\
87.01	0.00734113494345461\\
88.01	0.00734113160243682\\
89.01	0.00734112818851467\\
90.01	0.00734112470008018\\
91.01	0.00734112113548928\\
92.01	0.00734111749306128\\
93.01	0.00734111377107779\\
94.01	0.00734110996778191\\
95.01	0.00734110608137747\\
96.01	0.00734110211002788\\
97.01	0.00734109805185562\\
98.01	0.00734109390494087\\
99.01	0.00734108966732074\\
100.01	0.00734108533698838\\
101.01	0.00734108091189177\\
102.01	0.00734107638993288\\
103.01	0.00734107176896641\\
104.01	0.00734106704679891\\
105.01	0.00734106222118752\\
106.01	0.00734105728983888\\
107.01	0.00734105225040815\\
108.01	0.00734104710049755\\
109.01	0.00734104183765527\\
110.01	0.0073410364593742\\
111.01	0.00734103096309071\\
112.01	0.00734102534618341\\
113.01	0.00734101960597159\\
114.01	0.00734101373971391\\
115.01	0.00734100774460734\\
116.01	0.00734100161778521\\
117.01	0.00734099535631623\\
118.01	0.00734098895720262\\
119.01	0.00734098241737884\\
120.01	0.00734097573370982\\
121.01	0.00734096890298946\\
122.01	0.00734096192193895\\
123.01	0.0073409547872052\\
124.01	0.00734094749535894\\
125.01	0.00734094004289301\\
126.01	0.00734093242622056\\
127.01	0.00734092464167328\\
128.01	0.0073409166854994\\
129.01	0.00734090855386181\\
130.01	0.00734090024283592\\
131.01	0.00734089174840787\\
132.01	0.00734088306647234\\
133.01	0.00734087419283038\\
134.01	0.00734086512318723\\
135.01	0.00734085585315009\\
136.01	0.00734084637822589\\
137.01	0.00734083669381879\\
138.01	0.00734082679522806\\
139.01	0.00734081667764527\\
140.01	0.00734080633615198\\
141.01	0.00734079576571723\\
142.01	0.00734078496119466\\
143.01	0.00734077391732002\\
144.01	0.00734076262870825\\
145.01	0.00734075108985067\\
146.01	0.00734073929511214\\
147.01	0.00734072723872807\\
148.01	0.00734071491480133\\
149.01	0.0073407023172991\\
150.01	0.00734068944004967\\
151.01	0.00734067627673933\\
152.01	0.00734066282090877\\
153.01	0.00734064906594976\\
154.01	0.00734063500510159\\
155.01	0.00734062063144758\\
156.01	0.00734060593791108\\
157.01	0.00734059091725212\\
158.01	0.00734057556206308\\
159.01	0.00734055986476504\\
160.01	0.0073405438176036\\
161.01	0.00734052741264453\\
162.01	0.00734051064176986\\
163.01	0.00734049349667319\\
164.01	0.00734047596885525\\
165.01	0.00734045804961941\\
166.01	0.00734043973006696\\
167.01	0.00734042100109207\\
168.01	0.00734040185337714\\
169.01	0.00734038227738751\\
170.01	0.00734036226336636\\
171.01	0.00734034180132939\\
172.01	0.00734032088105916\\
173.01	0.00734029949209986\\
174.01	0.00734027762375134\\
175.01	0.00734025526506323\\
176.01	0.00734023240482905\\
177.01	0.00734020903157987\\
178.01	0.00734018513357814\\
179.01	0.00734016069881111\\
180.01	0.00734013571498427\\
181.01	0.00734011016951444\\
182.01	0.00734008404952291\\
183.01	0.00734005734182818\\
184.01	0.00734003003293865\\
185.01	0.00734000210904514\\
186.01	0.00733997355601325\\
187.01	0.00733994435937539\\
188.01	0.00733991450432257\\
189.01	0.00733988397569632\\
190.01	0.00733985275797997\\
191.01	0.00733982083529023\\
192.01	0.00733978819136796\\
193.01	0.00733975480956899\\
194.01	0.00733972067285493\\
195.01	0.00733968576378361\\
196.01	0.00733965006449877\\
197.01	0.00733961355672036\\
198.01	0.00733957622173397\\
199.01	0.0073395380403803\\
200.01	0.00733949899304402\\
201.01	0.00733945905964283\\
202.01	0.00733941821961589\\
203.01	0.00733937645191195\\
204.01	0.00733933373497738\\
205.01	0.00733929004674399\\
206.01	0.00733924536461594\\
207.01	0.00733919966545706\\
208.01	0.00733915292557724\\
209.01	0.00733910512071911\\
210.01	0.00733905622604368\\
211.01	0.00733900621611606\\
212.01	0.00733895506489082\\
213.01	0.0073389027456968\\
214.01	0.00733884923122136\\
215.01	0.00733879449349471\\
216.01	0.00733873850387362\\
217.01	0.00733868123302454\\
218.01	0.00733862265090651\\
219.01	0.00733856272675349\\
220.01	0.00733850142905631\\
221.01	0.00733843872554425\\
222.01	0.00733837458316583\\
223.01	0.00733830896806951\\
224.01	0.00733824184558348\\
225.01	0.00733817318019561\\
226.01	0.0073381029355318\\
227.01	0.00733803107433481\\
228.01	0.00733795755844195\\
229.01	0.00733788234876259\\
230.01	0.00733780540525474\\
231.01	0.00733772668690099\\
232.01	0.00733764615168425\\
233.01	0.00733756375656269\\
234.01	0.00733747945744349\\
235.01	0.00733739320915682\\
236.01	0.00733730496542871\\
237.01	0.00733721467885293\\
238.01	0.00733712230086257\\
239.01	0.00733702778170085\\
240.01	0.00733693107039089\\
241.01	0.00733683211470503\\
242.01	0.00733673086113305\\
243.01	0.00733662725484983\\
244.01	0.00733652123968203\\
245.01	0.00733641275807399\\
246.01	0.00733630175105267\\
247.01	0.00733618815819157\\
248.01	0.00733607191757416\\
249.01	0.00733595296575578\\
250.01	0.00733583123772501\\
251.01	0.00733570666686371\\
252.01	0.00733557918490631\\
253.01	0.00733544872189793\\
254.01	0.00733531520615136\\
255.01	0.00733517856420289\\
256.01	0.00733503872076728\\
257.01	0.00733489559869112\\
258.01	0.00733474911890551\\
259.01	0.00733459920037698\\
260.01	0.00733444576005756\\
261.01	0.00733428871283346\\
262.01	0.00733412797147213\\
263.01	0.00733396344656839\\
264.01	0.00733379504648899\\
265.01	0.00733362267731562\\
266.01	0.00733344624278665\\
267.01	0.0073332656442373\\
268.01	0.00733308078053815\\
269.01	0.00733289154803233\\
270.01	0.0073326978404708\\
271.01	0.00733249954894618\\
272.01	0.00733229656182496\\
273.01	0.00733208876467764\\
274.01	0.00733187604020731\\
275.01	0.00733165826817652\\
276.01	0.00733143532533209\\
277.01	0.007331207085328\\
278.01	0.00733097341864639\\
279.01	0.00733073419251666\\
280.01	0.00733048927083215\\
281.01	0.00733023851406531\\
282.01	0.00732998177918005\\
283.01	0.00732971891954244\\
284.01	0.00732944978482864\\
285.01	0.00732917422093099\\
286.01	0.00732889206986159\\
287.01	0.00732860316965309\\
288.01	0.00732830735425762\\
289.01	0.00732800445344265\\
290.01	0.00732769429268461\\
291.01	0.00732737669305965\\
292.01	0.00732705147113174\\
293.01	0.00732671843883795\\
294.01	0.0073263774033711\\
295.01	0.0073260281670591\\
296.01	0.00732567052724178\\
297.01	0.00732530427614428\\
298.01	0.00732492920074749\\
299.01	0.0073245450826555\\
300.01	0.00732415169795939\\
301.01	0.00732374881709819\\
302.01	0.00732333620471588\\
303.01	0.00732291361951545\\
304.01	0.00732248081410905\\
305.01	0.00732203753486475\\
306.01	0.00732158352174942\\
307.01	0.00732111850816789\\
308.01	0.00732064222079828\\
309.01	0.00732015437942341\\
310.01	0.00731965469675795\\
311.01	0.00731914287827173\\
312.01	0.00731861862200856\\
313.01	0.007318081618401\\
314.01	0.00731753155008062\\
315.01	0.00731696809168352\\
316.01	0.00731639090965177\\
317.01	0.00731579966202974\\
318.01	0.00731519399825555\\
319.01	0.00731457355894801\\
320.01	0.00731393797568816\\
321.01	0.00731328687079602\\
322.01	0.00731261985710168\\
323.01	0.00731193653771122\\
324.01	0.00731123650576725\\
325.01	0.00731051934420369\\
326.01	0.00730978462549459\\
327.01	0.00730903191139745\\
328.01	0.00730826075269011\\
329.01	0.00730747068890161\\
330.01	0.00730666124803663\\
331.01	0.00730583194629368\\
332.01	0.00730498228777586\\
333.01	0.00730411176419571\\
334.01	0.00730321985457197\\
335.01	0.0073023060249201\\
336.01	0.00730136972793427\\
337.01	0.00730041040266253\\
338.01	0.00729942747417348\\
339.01	0.00729842035321495\\
340.01	0.00729738843586434\\
341.01	0.00729633110316978\\
342.01	0.00729524772078298\\
343.01	0.00729413763858231\\
344.01	0.00729300019028617\\
345.01	0.00729183469305672\\
346.01	0.00729064044709288\\
347.01	0.00728941673521282\\
348.01	0.00728816282242548\\
349.01	0.00728687795549005\\
350.01	0.00728556136246381\\
351.01	0.00728421225223741\\
352.01	0.00728282981405698\\
353.01	0.00728141321703262\\
354.01	0.00727996160963335\\
355.01	0.00727847411916644\\
356.01	0.00727694985124217\\
357.01	0.00727538788922212\\
358.01	0.00727378729365114\\
359.01	0.00727214710167144\\
360.01	0.00727046632641884\\
361.01	0.00726874395639984\\
362.01	0.00726697895484907\\
363.01	0.00726517025906573\\
364.01	0.00726331677972892\\
365.01	0.00726141740019008\\
366.01	0.00725947097574201\\
367.01	0.00725747633286347\\
368.01	0.00725543226843849\\
369.01	0.0072533375489489\\
370.01	0.00725119090963921\\
371.01	0.00724899105365316\\
372.01	0.00724673665113992\\
373.01	0.00724442633833004\\
374.01	0.00724205871657875\\
375.01	0.0072396323513764\\
376.01	0.00723714577132455\\
377.01	0.00723459746707628\\
378.01	0.0072319858902405\\
379.01	0.00722930945224778\\
380.01	0.00722656652317783\\
381.01	0.00722375543054717\\
382.01	0.00722087445805516\\
383.01	0.00721792184428871\\
384.01	0.00721489578138365\\
385.01	0.00721179441364176\\
386.01	0.00720861583610336\\
387.01	0.00720535809307382\\
388.01	0.00720201917660352\\
389.01	0.00719859702492024\\
390.01	0.00719508952081353\\
391.01	0.00719149448997062\\
392.01	0.00718780969926303\\
393.01	0.00718403285498466\\
394.01	0.00718016160104026\\
395.01	0.00717619351708595\\
396.01	0.00717212611662225\\
397.01	0.00716795684504191\\
398.01	0.00716368307763514\\
399.01	0.0071593021175567\\
400.01	0.00715481119376046\\
401.01	0.00715020745890973\\
402.01	0.00714548798727503\\
403.01	0.00714064977263219\\
404.01	0.00713568972618098\\
405.01	0.00713060467450701\\
406.01	0.00712539135761644\\
407.01	0.00712004642707825\\
408.01	0.00711456644431503\\
409.01	0.0071089478790868\\
410.01	0.00710318710821348\\
411.01	0.00709728041457593\\
412.01	0.00709122398641762\\
413.01	0.00708501391693231\\
414.01	0.0070786462040412\\
415.01	0.00707211675003738\\
416.01	0.00706542136092544\\
417.01	0.00705855574717348\\
418.01	0.00705151552320142\\
419.01	0.00704429620649164\\
420.01	0.00703689321959463\\
421.01	0.00702930188928812\\
422.01	0.00702151739972556\\
423.01	0.00701353482583495\\
424.01	0.00700534917614831\\
425.01	0.0069969553614563\\
426.01	0.00698834819560703\\
427.01	0.00697952239786452\\
428.01	0.00697047259597068\\
429.01	0.00696119333002084\\
430.01	0.00695167905728106\\
431.01	0.00694192415809635\\
432.01	0.00693192294306247\\
433.01	0.00692166966166322\\
434.01	0.00691115851260798\\
435.01	0.00690038365614526\\
436.01	0.00688933922867446\\
437.01	0.0068780193600339\\
438.01	0.00686641819391086\\
439.01	0.00685452991189808\\
440.01	0.00684234876181789\\
441.01	0.00682986909104756\\
442.01	0.00681708538572061\\
443.01	0.00680399231684287\\
444.01	0.00679058479456501\\
445.01	0.0067768580321002\\
446.01	0.00676280762107438\\
447.01	0.00674842962046204\\
448.01	0.00673372066171481\\
449.01	0.00671867807323846\\
450.01	0.00670330002799456\\
451.01	0.00668758571715203\\
452.01	0.00667153548351915\\
453.01	0.00665515057052235\\
454.01	0.00663843559358089\\
455.01	0.00662139865022594\\
456.01	0.00660405128096307\\
457.01	0.00658640926690019\\
458.01	0.00656849362613499\\
459.01	0.00655033190451136\\
460.01	0.00653195996748447\\
461.01	0.00651338992196601\\
462.01	0.00649451268265961\\
463.01	0.00647533204807501\\
464.01	0.00645587440892889\\
465.01	0.00643617364606535\\
466.01	0.00641627280342872\\
467.01	0.00639622615410085\\
468.01	0.0063761017581098\\
469.01	0.00635598463514874\\
470.01	0.00633598069084061\\
471.01	0.00631622165165349\\
472.01	0.00629687134529433\\
473.01	0.00627813341218427\\
474.01	0.0062602617677022\\
475.01	0.00624307687454638\\
476.01	0.00622549082442006\\
477.01	0.00620743337667525\\
478.01	0.00618889187776189\\
479.01	0.00616985371567746\\
480.01	0.00615030642372044\\
481.01	0.00613023781053281\\
482.01	0.00610963612259572\\
483.01	0.00608849024679421\\
484.01	0.00606678996198266\\
485.01	0.00604452623973263\\
486.01	0.00602169128206165\\
487.01	0.00599826555399007\\
488.01	0.00597418206558548\\
489.01	0.00594941305773239\\
490.01	0.00592394139900088\\
491.01	0.00589775033755464\\
492.01	0.00587082362165473\\
493.01	0.00584314561672445\\
494.01	0.00581470141027783\\
495.01	0.00578547689168251\\
496.01	0.00575545871032362\\
497.01	0.00572463408788436\\
498.01	0.00569299073822983\\
499.01	0.00566051586224345\\
500.01	0.00562719492563089\\
501.01	0.00559300990457205\\
502.01	0.0055579343979256\\
503.01	0.00552192924888774\\
504.01	0.00548494980648591\\
505.01	0.00544693740411984\\
506.01	0.0054078213643012\\
507.01	0.00536754783778978\\
508.01	0.005326061155258\\
509.01	0.00528330235990979\\
510.01	0.00523920912376\\
511.01	0.00519371569541818\\
512.01	0.00514675288793071\\
513.01	0.00509824811625097\\
514.01	0.00504812549505705\\
515.01	0.00499630601086775\\
516.01	0.00494270780002012\\
517.01	0.00488724652526266\\
518.01	0.00482984735845951\\
519.01	0.00477055288280926\\
520.01	0.00470952603271803\\
521.01	0.0046472138863196\\
522.01	0.00458470131205031\\
523.01	0.00452219569378065\\
524.01	0.00445980107877253\\
525.01	0.00439765771346164\\
526.01	0.00433592197309556\\
527.01	0.00427476476752623\\
528.01	0.00421437048237607\\
529.01	0.00415493489511919\\
530.01	0.00409666162626346\\
531.01	0.00403975646328572\\
532.01	0.00398441874170262\\
533.01	0.00393082869873044\\
534.01	0.00387912966662672\\
535.01	0.00382940493481397\\
536.01	0.00378164249179029\\
537.01	0.00373524000500358\\
538.01	0.00368912833644244\\
539.01	0.00364330189677589\\
540.01	0.00359778390829597\\
541.01	0.00355259025757311\\
542.01	0.00350772801655309\\
543.01	0.00346319391049507\\
544.01	0.00341896868234572\\
545.01	0.00337501297581907\\
546.01	0.00333126511313421\\
547.01	0.00328763977111955\\
548.01	0.00324402773980095\\
549.01	0.00320029915127878\\
550.01	0.00315631167800961\\
551.01	0.00311195285554549\\
552.01	0.0030671882490935\\
553.01	0.00302199127603092\\
554.01	0.00297633053814613\\
555.01	0.00293017008017734\\
556.01	0.00288346993778933\\
557.01	0.00283618702641363\\
558.01	0.00278827640582232\\
559.01	0.0027396929194969\\
560.01	0.00269039314200024\\
561.01	0.00264033745611435\\
562.01	0.00258949198921666\\
563.01	0.00253782990454083\\
564.01	0.0024853288845688\\
565.01	0.00243196691153569\\
566.01	0.00237772185534557\\
567.01	0.00232257175221729\\
568.01	0.00226649513730902\\
569.01	0.00220947140606029\\
570.01	0.00215148115625655\\
571.01	0.00209250647178937\\
572.01	0.00203253110496798\\
573.01	0.00197154051305623\\
574.01	0.00190952172343432\\
575.01	0.00184646307785447\\
576.01	0.00178235412443806\\
577.01	0.0017171857660873\\
578.01	0.00165095045995029\\
579.01	0.00158364242165109\\
580.01	0.00151525782603943\\
581.01	0.00144579499572334\\
582.01	0.00137525457316008\\
583.01	0.00130363967477813\\
584.01	0.00123095602868033\\
585.01	0.00115721210051214\\
586.01	0.00108241921223445\\
587.01	0.00100659164275123\\
588.01	0.000929746671129862\\
589.01	0.000851904520258075\\
590.01	0.000773088158107163\\
591.01	0.000693322906705438\\
592.01	0.000612635797368357\\
593.01	0.000531054596121653\\
594.01	0.000448606405509172\\
595.01	0.00036531572727916\\
596.01	0.000281201843626103\\
597.01	0.000196275341074087\\
598.01	0.000110533558374905\\
599.01	3.18230760274988e-05\\
599.02	3.12771225888381e-05\\
599.03	3.07343852024352e-05\\
599.04	3.01948954524109e-05\\
599.05	2.96586853198184e-05\\
599.06	2.91257871969947e-05\\
599.07	2.85962339028577e-05\\
599.08	2.8070058607518e-05\\
599.09	2.75472948133501e-05\\
599.1	2.70279763573446e-05\\
599.11	2.65121374149781e-05\\
599.12	2.59998125032215e-05\\
599.13	2.54910364795111e-05\\
599.14	2.49858445451751e-05\\
599.15	2.44842722489057e-05\\
599.16	2.39863554902346e-05\\
599.17	2.34921305228321e-05\\
599.18	2.30016339574767e-05\\
599.19	2.25149027655146e-05\\
599.2	2.20319742823254e-05\\
599.21	2.15528862108072e-05\\
599.22	2.10776766248668e-05\\
599.23	2.06063839729136e-05\\
599.24	2.01390470813568e-05\\
599.25	1.96757051580971e-05\\
599.26	1.92163977960017e-05\\
599.27	1.87611649765159e-05\\
599.28	1.83100470734845e-05\\
599.29	1.78630848569894e-05\\
599.3	1.74203194972418e-05\\
599.31	1.69817925685135e-05\\
599.32	1.65475460531112e-05\\
599.33	1.61176223453848e-05\\
599.34	1.56920642557928e-05\\
599.35	1.52709150149869e-05\\
599.36	1.48542182779494e-05\\
599.37	1.44420181281777e-05\\
599.38	1.40343590818905e-05\\
599.39	1.36312869210512e-05\\
599.4	1.32328508630885e-05\\
599.41	1.2839100613395e-05\\
599.42	1.24500863701339e-05\\
599.43	1.20658588290742e-05\\
599.44	1.16864691884749e-05\\
599.45	1.13119691540261e-05\\
599.46	1.09424109438309e-05\\
599.47	1.05778472934363e-05\\
599.48	1.02183314609188e-05\\
599.49	9.86391723201314e-06\\
599.5	9.51465892530708e-06\\
599.51	9.17061139746149e-06\\
599.52	8.83183004850643e-06\\
599.53	8.49837082718408e-06\\
599.54	8.17029023633675e-06\\
599.55	7.84764533835573e-06\\
599.56	7.53049376068203e-06\\
599.57	7.21889370135405e-06\\
599.58	6.91290393461941e-06\\
599.59	6.61258381660576e-06\\
599.6	6.31799329102282e-06\\
599.61	6.02919289495116e-06\\
599.62	5.74624376466562e-06\\
599.63	5.46920764152997e-06\\
599.64	5.19814687793482e-06\\
599.65	4.93312444331713e-06\\
599.66	4.67420393020224e-06\\
599.67	4.42144956034653e-06\\
599.68	4.17492619090958e-06\\
599.69	3.93469932070262e-06\\
599.7	3.70083509648733e-06\\
599.71	3.47340031935442e-06\\
599.72	3.25246245113688e-06\\
599.73	3.03808962091867e-06\\
599.74	2.8303506315764e-06\\
599.75	2.62931496640938e-06\\
599.76	2.43505279581835e-06\\
599.77	2.24763498406222e-06\\
599.78	2.06713309606858e-06\\
599.79	1.89361940432578e-06\\
599.8	1.72716689582353e-06\\
599.81	1.56784927908028e-06\\
599.82	1.41574099122785e-06\\
599.83	1.27091720517752e-06\\
599.84	1.13345383683135e-06\\
599.85	1.00342755240089e-06\\
599.86	8.80915775762492e-07\\
599.87	7.65996695906157e-07\\
599.88	6.58749274453849e-07\\
599.89	5.59253253248904e-07\\
599.9	4.67589162013102e-07\\
599.91	3.8383832610088e-07\\
599.92	3.0808287429171e-07\\
599.93	2.4040574671258e-07\\
599.94	1.80890702777825e-07\\
599.95	1.2962232925906e-07\\
599.96	8.66860484019516e-08\\
599.97	5.21681261331924e-08\\
599.98	2.61556803542173e-08\\
599.99	8.73668929909921e-09\\
600	0\\
};
\addplot [color=mycolor19,solid,forget plot]
  table[row sep=crcr]{%
0.01	0.00662800561349471\\
1.01	0.00662800516698634\\
2.01	0.00662800471096339\\
3.01	0.00662800424522229\\
4.01	0.00662800376955526\\
5.01	0.00662800328375014\\
6.01	0.00662800278758982\\
7.01	0.00662800228085294\\
8.01	0.00662800176331315\\
9.01	0.00662800123473928\\
10.01	0.00662800069489526\\
11.01	0.00662800014353982\\
12.01	0.00662799958042655\\
13.01	0.00662799900530394\\
14.01	0.00662799841791487\\
15.01	0.00662799781799666\\
16.01	0.00662799720528108\\
17.01	0.00662799657949403\\
18.01	0.00662799594035567\\
19.01	0.00662799528758004\\
20.01	0.006627994620875\\
21.01	0.0066279939399423\\
22.01	0.00662799324447699\\
23.01	0.00662799253416775\\
24.01	0.00662799180869655\\
25.01	0.00662799106773842\\
26.01	0.00662799031096156\\
27.01	0.00662798953802681\\
28.01	0.00662798874858787\\
29.01	0.00662798794229095\\
30.01	0.00662798711877457\\
31.01	0.0066279862776695\\
32.01	0.00662798541859861\\
33.01	0.00662798454117659\\
34.01	0.00662798364500984\\
35.01	0.00662798272969625\\
36.01	0.00662798179482509\\
37.01	0.00662798083997678\\
38.01	0.00662797986472263\\
39.01	0.00662797886862471\\
40.01	0.00662797785123572\\
41.01	0.00662797681209865\\
42.01	0.00662797575074674\\
43.01	0.00662797466670292\\
44.01	0.0066279735594801\\
45.01	0.00662797242858051\\
46.01	0.00662797127349571\\
47.01	0.0066279700937063\\
48.01	0.0066279688886815\\
49.01	0.00662796765787931\\
50.01	0.00662796640074588\\
51.01	0.0066279651167154\\
52.01	0.00662796380520997\\
53.01	0.00662796246563897\\
54.01	0.00662796109739918\\
55.01	0.00662795969987448\\
56.01	0.00662795827243515\\
57.01	0.00662795681443804\\
58.01	0.00662795532522601\\
59.01	0.0066279538041277\\
60.01	0.00662795225045732\\
61.01	0.00662795066351422\\
62.01	0.00662794904258259\\
63.01	0.00662794738693112\\
64.01	0.00662794569581267\\
65.01	0.00662794396846396\\
66.01	0.00662794220410519\\
67.01	0.00662794040193965\\
68.01	0.00662793856115338\\
69.01	0.00662793668091491\\
70.01	0.0066279347603746\\
71.01	0.00662793279866449\\
72.01	0.00662793079489781\\
73.01	0.0066279287481686\\
74.01	0.00662792665755128\\
75.01	0.00662792452210008\\
76.01	0.00662792234084891\\
77.01	0.00662792011281058\\
78.01	0.00662791783697666\\
79.01	0.00662791551231669\\
80.01	0.00662791313777803\\
81.01	0.00662791071228504\\
82.01	0.00662790823473874\\
83.01	0.00662790570401648\\
84.01	0.00662790311897112\\
85.01	0.00662790047843062\\
86.01	0.0066278977811975\\
87.01	0.00662789502604836\\
88.01	0.00662789221173306\\
89.01	0.00662788933697446\\
90.01	0.00662788640046762\\
91.01	0.00662788340087927\\
92.01	0.00662788033684711\\
93.01	0.00662787720697921\\
94.01	0.00662787400985347\\
95.01	0.00662787074401677\\
96.01	0.0066278674079845\\
97.01	0.0066278640002396\\
98.01	0.00662786051923218\\
99.01	0.00662785696337853\\
100.01	0.00662785333106043\\
101.01	0.0066278496206245\\
102.01	0.00662784583038131\\
103.01	0.0066278419586047\\
104.01	0.00662783800353083\\
105.01	0.00662783396335757\\
106.01	0.00662782983624339\\
107.01	0.00662782562030659\\
108.01	0.0066278213136246\\
109.01	0.00662781691423285\\
110.01	0.00662781242012386\\
111.01	0.00662780782924653\\
112.01	0.00662780313950477\\
113.01	0.00662779834875697\\
114.01	0.00662779345481474\\
115.01	0.00662778845544178\\
116.01	0.0066277833483531\\
117.01	0.00662777813121369\\
118.01	0.00662777280163762\\
119.01	0.00662776735718675\\
120.01	0.00662776179536972\\
121.01	0.00662775611364071\\
122.01	0.00662775030939814\\
123.01	0.00662774437998358\\
124.01	0.00662773832268047\\
125.01	0.00662773213471277\\
126.01	0.00662772581324374\\
127.01	0.00662771935537439\\
128.01	0.00662771275814245\\
129.01	0.00662770601852059\\
130.01	0.00662769913341532\\
131.01	0.00662769209966521\\
132.01	0.00662768491403954\\
133.01	0.00662767757323686\\
134.01	0.0066276700738832\\
135.01	0.00662766241253068\\
136.01	0.00662765458565565\\
137.01	0.00662764658965726\\
138.01	0.00662763842085535\\
139.01	0.0066276300754892\\
140.01	0.00662762154971541\\
141.01	0.00662761283960593\\
142.01	0.00662760394114659\\
143.01	0.00662759485023485\\
144.01	0.00662758556267798\\
145.01	0.00662757607419107\\
146.01	0.00662756638039477\\
147.01	0.0066275564768134\\
148.01	0.00662754635887267\\
149.01	0.00662753602189763\\
150.01	0.00662752546111022\\
151.01	0.00662751467162706\\
152.01	0.00662750364845716\\
153.01	0.00662749238649931\\
154.01	0.00662748088053993\\
155.01	0.00662746912525013\\
156.01	0.00662745711518362\\
157.01	0.00662744484477359\\
158.01	0.00662743230833038\\
159.01	0.00662741950003849\\
160.01	0.0066274064139538\\
161.01	0.00662739304400081\\
162.01	0.00662737938396948\\
163.01	0.00662736542751227\\
164.01	0.00662735116814124\\
165.01	0.00662733659922458\\
166.01	0.0066273217139835\\
167.01	0.00662730650548892\\
168.01	0.00662729096665803\\
169.01	0.00662727509025096\\
170.01	0.00662725886886697\\
171.01	0.00662724229494104\\
172.01	0.00662722536074013\\
173.01	0.00662720805835907\\
174.01	0.00662719037971722\\
175.01	0.0066271723165538\\
176.01	0.00662715386042436\\
177.01	0.00662713500269641\\
178.01	0.00662711573454498\\
179.01	0.00662709604694843\\
180.01	0.00662707593068403\\
181.01	0.00662705537632315\\
182.01	0.00662703437422661\\
183.01	0.00662701291454014\\
184.01	0.00662699098718905\\
185.01	0.00662696858187351\\
186.01	0.00662694568806321\\
187.01	0.00662692229499199\\
188.01	0.00662689839165287\\
189.01	0.006626873966792\\
190.01	0.0066268490089033\\
191.01	0.00662682350622247\\
192.01	0.0066267974467212\\
193.01	0.00662677081810105\\
194.01	0.00662674360778704\\
195.01	0.00662671580292151\\
196.01	0.00662668739035743\\
197.01	0.00662665835665177\\
198.01	0.00662662868805853\\
199.01	0.00662659837052181\\
200.01	0.00662656738966869\\
201.01	0.0066265357308016\\
202.01	0.00662650337889101\\
203.01	0.00662647031856767\\
204.01	0.00662643653411447\\
205.01	0.00662640200945852\\
206.01	0.00662636672816279\\
207.01	0.00662633067341761\\
208.01	0.00662629382803171\\
209.01	0.0066262561744237\\
210.01	0.00662621769461239\\
211.01	0.00662617837020773\\
212.01	0.00662613818240104\\
213.01	0.00662609711195519\\
214.01	0.00662605513919448\\
215.01	0.00662601224399421\\
216.01	0.00662596840576998\\
217.01	0.00662592360346703\\
218.01	0.0066258778155487\\
219.01	0.00662583101998533\\
220.01	0.00662578319424236\\
221.01	0.00662573431526805\\
222.01	0.00662568435948147\\
223.01	0.00662563330275989\\
224.01	0.00662558112042532\\
225.01	0.0066255277872315\\
226.01	0.00662547327735037\\
227.01	0.00662541756435768\\
228.01	0.00662536062121904\\
229.01	0.0066253024202747\\
230.01	0.00662524293322482\\
231.01	0.00662518213111368\\
232.01	0.00662511998431401\\
233.01	0.00662505646251033\\
234.01	0.00662499153468248\\
235.01	0.00662492516908842\\
236.01	0.00662485733324617\\
237.01	0.00662478799391618\\
238.01	0.00662471711708254\\
239.01	0.00662464466793375\\
240.01	0.00662457061084337\\
241.01	0.00662449490934975\\
242.01	0.00662441752613552\\
243.01	0.00662433842300622\\
244.01	0.0066242575608687\\
245.01	0.00662417489970877\\
246.01	0.00662409039856812\\
247.01	0.00662400401552091\\
248.01	0.00662391570764941\\
249.01	0.00662382543101951\\
250.01	0.00662373314065458\\
251.01	0.00662363879051004\\
252.01	0.00662354233344579\\
253.01	0.00662344372119895\\
254.01	0.00662334290435512\\
255.01	0.00662323983231958\\
256.01	0.00662313445328695\\
257.01	0.00662302671421067\\
258.01	0.00662291656077115\\
259.01	0.00662280393734327\\
260.01	0.00662268878696307\\
261.01	0.00662257105129317\\
262.01	0.00662245067058778\\
263.01	0.00662232758365616\\
264.01	0.00662220172782539\\
265.01	0.00662207303890209\\
266.01	0.00662194145113291\\
267.01	0.00662180689716418\\
268.01	0.00662166930800013\\
269.01	0.00662152861296002\\
270.01	0.00662138473963445\\
271.01	0.00662123761383974\\
272.01	0.00662108715957155\\
273.01	0.00662093329895719\\
274.01	0.00662077595220619\\
275.01	0.00662061503755995\\
276.01	0.00662045047123949\\
277.01	0.00662028216739223\\
278.01	0.00662011003803687\\
279.01	0.0066199339930068\\
280.01	0.00661975393989208\\
281.01	0.00661956978397959\\
282.01	0.00661938142819155\\
283.01	0.00661918877302219\\
284.01	0.00661899171647292\\
285.01	0.00661879015398522\\
286.01	0.00661858397837213\\
287.01	0.00661837307974748\\
288.01	0.00661815734545293\\
289.01	0.00661793665998358\\
290.01	0.00661771090491081\\
291.01	0.0066174799588034\\
292.01	0.00661724369714626\\
293.01	0.00661700199225676\\
294.01	0.00661675471319886\\
295.01	0.00661650172569491\\
296.01	0.0066162428920347\\
297.01	0.00661597807098249\\
298.01	0.00661570711768067\\
299.01	0.00661542988355159\\
300.01	0.00661514621619602\\
301.01	0.00661485595928926\\
302.01	0.00661455895247416\\
303.01	0.00661425503125132\\
304.01	0.00661394402686626\\
305.01	0.00661362576619343\\
306.01	0.0066133000716175\\
307.01	0.00661296676091094\\
308.01	0.00661262564710884\\
309.01	0.00661227653837986\\
310.01	0.00661191923789445\\
311.01	0.00661155354368929\\
312.01	0.00661117924852798\\
313.01	0.0066107961397587\\
314.01	0.00661040399916789\\
315.01	0.00661000260283024\\
316.01	0.00660959172095541\\
317.01	0.0066091711177301\\
318.01	0.0066087405511571\\
319.01	0.00660829977289013\\
320.01	0.00660784852806491\\
321.01	0.00660738655512567\\
322.01	0.00660691358564859\\
323.01	0.00660642934416056\\
324.01	0.00660593354795405\\
325.01	0.00660542590689797\\
326.01	0.00660490612324442\\
327.01	0.00660437389143127\\
328.01	0.00660382889788074\\
329.01	0.00660327082079387\\
330.01	0.00660269932994098\\
331.01	0.00660211408644781\\
332.01	0.00660151474257809\\
333.01	0.00660090094151163\\
334.01	0.0066002723171192\\
335.01	0.00659962849373248\\
336.01	0.00659896908591152\\
337.01	0.00659829369820749\\
338.01	0.00659760192492218\\
339.01	0.00659689334986413\\
340.01	0.00659616754610136\\
341.01	0.00659542407571067\\
342.01	0.00659466248952397\\
343.01	0.00659388232687187\\
344.01	0.00659308311532434\\
345.01	0.00659226437042884\\
346.01	0.00659142559544619\\
347.01	0.00659056628108438\\
348.01	0.00658968590523029\\
349.01	0.00658878393268019\\
350.01	0.0065878598148684\\
351.01	0.00658691298959533\\
352.01	0.00658594288075464\\
353.01	0.00658494889805978\\
354.01	0.00658393043677062\\
355.01	0.00658288687742027\\
356.01	0.00658181758554252\\
357.01	0.00658072191140011\\
358.01	0.00657959918971456\\
359.01	0.00657844873939757\\
360.01	0.00657726986328474\\
361.01	0.00657606184787175\\
362.01	0.00657482396305372\\
363.01	0.00657355546186785\\
364.01	0.00657225558024\\
365.01	0.006570923536736\\
366.01	0.00656955853231751\\
367.01	0.00656815975010318\\
368.01	0.00656672635513631\\
369.01	0.00656525749415812\\
370.01	0.00656375229538862\\
371.01	0.00656220986831459\\
372.01	0.00656062930348597\\
373.01	0.00655900967232032\\
374.01	0.00655735002691722\\
375.01	0.00655564939988173\\
376.01	0.00655390680415868\\
377.01	0.00655212123287765\\
378.01	0.00655029165920917\\
379.01	0.00654841703623332\\
380.01	0.00654649629681985\\
381.01	0.00654452835352096\\
382.01	0.00654251209847646\\
383.01	0.00654044640333092\\
384.01	0.00653833011916143\\
385.01	0.00653616207641647\\
386.01	0.0065339410848613\\
387.01	0.00653166593352889\\
388.01	0.0065293353906704\\
389.01	0.0065269482037002\\
390.01	0.00652450309912636\\
391.01	0.00652199878245545\\
392.01	0.0065194339380574\\
393.01	0.00651680722897081\\
394.01	0.00651411729662454\\
395.01	0.00651136276044442\\
396.01	0.00650854221730586\\
397.01	0.00650565424078453\\
398.01	0.00650269738014472\\
399.01	0.00649967015899487\\
400.01	0.00649657107352375\\
401.01	0.00649339859021859\\
402.01	0.00649015114294966\\
403.01	0.00648682712929694\\
404.01	0.00648342490598466\\
405.01	0.00647994278329268\\
406.01	0.00647637901833192\\
407.01	0.00647273180711472\\
408.01	0.00646899927543674\\
409.01	0.00646517946873548\\
410.01	0.00646127034133285\\
411.01	0.00645726974585314\\
412.01	0.00645317542419395\\
413.01	0.00644898500231479\\
414.01	0.00644469599438852\\
415.01	0.00644030582556887\\
416.01	0.00643581184422117\\
417.01	0.00643121132031199\\
418.01	0.00642650144298454\\
419.01	0.00642167931811403\\
420.01	0.00641674196582409\\
421.01	0.00641168631758996\\
422.01	0.0064065092131186\\
423.01	0.00640120740019603\\
424.01	0.00639577753187652\\
425.01	0.00639021616335015\\
426.01	0.00638451974956823\\
427.01	0.00637868464293872\\
428.01	0.00637270709110753\\
429.01	0.00636658323485393\\
430.01	0.00636030910613026\\
431.01	0.00635388062628287\\
432.01	0.00634729360449606\\
433.01	0.00634054373650708\\
434.01	0.00633362660364876\\
435.01	0.00632653767228254\\
436.01	0.00631927229369635\\
437.01	0.00631182570455096\\
438.01	0.00630419302797106\\
439.01	0.00629636927538985\\
440.01	0.00628834934927172\\
441.01	0.00628012804685388\\
442.01	0.00627170006506416\\
443.01	0.00626306000679248\\
444.01	0.00625420238871374\\
445.01	0.00624512165087826\\
446.01	0.00623581216830759\\
447.01	0.00622626826485076\\
448.01	0.00621648422956873\\
449.01	0.00620645433592308\\
450.01	0.00619617286403988\\
451.01	0.00618563412629655\\
452.01	0.00617483249659878\\
453.01	0.00616376244593114\\
454.01	0.00615241856387239\\
455.01	0.00614079555763997\\
456.01	0.00612888826101543\\
457.01	0.00611669163481851\\
458.01	0.0061042007499225\\
459.01	0.00609141074431236\\
460.01	0.00607831673480827\\
461.01	0.00606491372901676\\
462.01	0.00605119794332509\\
463.01	0.00603716701194948\\
464.01	0.00602281903691963\\
465.01	0.00600815244782415\\
466.01	0.00599316579941203\\
467.01	0.00597785743865201\\
468.01	0.00596222498751151\\
469.01	0.00594626455940978\\
470.01	0.00592996931183786\\
471.01	0.00591332644919716\\
472.01	0.0058963183014063\\
473.01	0.00587891840890463\\
474.01	0.00586108519619017\\
475.01	0.00584275945284479\\
476.01	0.0058239000907944\\
477.01	0.00580447904511014\\
478.01	0.00578446622209283\\
479.01	0.00576382917878065\\
480.01	0.00574253294140469\\
481.01	0.00572053981903716\\
482.01	0.00569780921596359\\
483.01	0.00567429744785275\\
484.01	0.00564995756883459\\
485.01	0.00562473921933519\\
486.01	0.00559858850938103\\
487.01	0.0055714480080374\\
488.01	0.00554325732314974\\
489.01	0.00551395347978441\\
490.01	0.00548347094837184\\
491.01	0.00545174230923784\\
492.01	0.00541869929752596\\
493.01	0.00538427432238069\\
494.01	0.00534840261886492\\
495.01	0.0053110252402656\\
496.01	0.00527209316299654\\
497.01	0.00523157286128748\\
498.01	0.00518945381409247\\
499.01	0.00514575855936362\\
500.01	0.00510055609813132\\
501.01	0.00505398507554129\\
502.01	0.00500669829198533\\
503.01	0.00495908387073525\\
504.01	0.00491119851801213\\
505.01	0.00486310693772908\\
506.01	0.00481488274983092\\
507.01	0.00476660888573337\\
508.01	0.0047183779636502\\
509.01	0.00467029270123617\\
510.01	0.00462246611531557\\
511.01	0.00457502139463206\\
512.01	0.00452809129427338\\
513.01	0.00448181684757209\\
514.01	0.00443634512148415\\
515.01	0.00439182564908723\\
516.01	0.0043484050477132\\
517.01	0.00430621914513456\\
518.01	0.00426538166528717\\
519.01	0.00422596707374778\\
520.01	0.00418798502878103\\
521.01	0.00415117282828407\\
522.01	0.004114607957536\\
523.01	0.00407821214291625\\
524.01	0.00404202278944255\\
525.01	0.00400607380377639\\
526.01	0.00397039386513776\\
527.01	0.00393500465022193\\
528.01	0.00389991884824442\\
529.01	0.00386513800886344\\
530.01	0.0038306503075186\\
531.01	0.00379642837598923\\
532.01	0.00376242743600688\\
533.01	0.00372858410164934\\
534.01	0.00369481639498225\\
535.01	0.00366102575010299\\
536.01	0.00362710213479891\\
537.01	0.00359294119869851\\
538.01	0.00355850046695407\\
539.01	0.00352376222467012\\
540.01	0.00348870536509431\\
541.01	0.00345330520045406\\
542.01	0.00341753352031443\\
543.01	0.00338135877420527\\
544.01	0.00334474647734787\\
545.01	0.00330765993147755\\
546.01	0.00327006123083538\\
547.01	0.00323191254095152\\
548.01	0.00319317761547406\\
549.01	0.0031538234145313\\
550.01	0.00311382155102045\\
551.01	0.0030731486693184\\
552.01	0.0030317832424425\\
553.01	0.00298970328206819\\
554.01	0.00294688640917473\\
555.01	0.0029033100905392\\
556.01	0.00285895189237939\\
557.01	0.00281378973570623\\
558.01	0.00276780213238578\\
559.01	0.00272096837574328\\
560.01	0.00267326865627233\\
561.01	0.00262468407428047\\
562.01	0.00257519653052143\\
563.01	0.00252478849624054\\
564.01	0.00247344276509564\\
565.01	0.00242114241434334\\
566.01	0.00236787089249708\\
567.01	0.0023136121155352\\
568.01	0.00225835056070582\\
569.01	0.00220207135471073\\
570.01	0.00214476035493563\\
571.01	0.00208640422450415\\
572.01	0.00202699050448087\\
573.01	0.00196650768972543\\
574.01	0.00190494531817138\\
575.01	0.00184229408482995\\
576.01	0.00177854598418229\\
577.01	0.00171369446888605\\
578.01	0.0016477346162388\\
579.01	0.0015806633005132\\
580.01	0.00151247936952008\\
581.01	0.00144318382328261\\
582.01	0.00137277999184947\\
583.01	0.00130127370788135\\
584.01	0.0012286734675554\\
585.01	0.00115499057031396\\
586.01	0.00108023922386147\\
587.01	0.00100443659586242\\
588.01	0.000927602789133309\\
589.01	0.000849760712020215\\
590.01	0.000770935808457237\\
591.01	0.00069115560281989\\
592.01	0.000610449002857873\\
593.01	0.000528845289120537\\
594.01	0.000446372700577078\\
595.01	0.000363056502572266\\
596.01	0.000278916393581779\\
597.01	0.000193963069867737\\
598.01	0.000108193720153206\\
599.01	3.18230447902152e-05\\
599.02	3.12770961984968e-05\\
599.03	3.07343624924919e-05\\
599.04	3.0194875613938e-05\\
599.05	2.96586678183493e-05\\
599.06	2.91257716778671e-05\\
599.07	2.85962200841493e-05\\
599.08	2.80700462514365e-05\\
599.09	2.75472837197206e-05\\
599.1	2.70279663579388e-05\\
599.11	2.65121283671912e-05\\
599.12	2.59998042840028e-05\\
599.13	2.54910289836183e-05\\
599.14	2.49858376833341e-05\\
599.15	2.44842659458625e-05\\
599.16	2.39863496827274e-05\\
599.17	2.34921251576915e-05\\
599.18	2.30016289902261e-05\\
599.19	2.2514898159005e-05\\
599.2	2.20319700054462e-05\\
599.21	2.15528822372704e-05\\
599.22	2.10776729321089e-05\\
599.23	2.06063805411449e-05\\
599.24	2.01390438927945e-05\\
599.25	1.96757021964072e-05\\
599.26	1.92163950460209e-05\\
599.27	1.87611624241565e-05\\
599.28	1.83100447056338e-05\\
599.29	1.78630826614314e-05\\
599.3	1.74203174626037e-05\\
599.31	1.69817906841944e-05\\
599.32	1.65475443092387e-05\\
599.33	1.61176207327651e-05\\
599.34	1.56920627658545e-05\\
599.35	1.52709136397415e-05\\
599.36	1.48542170099464e-05\\
599.37	1.44420169604503e-05\\
599.38	1.40343580079232e-05\\
599.39	1.36312859347863e-05\\
599.4	1.32328499588292e-05\\
599.41	1.28390997857844e-05\\
599.42	1.24500856141119e-05\\
599.43	1.20658581398547e-05\\
599.44	1.16864685615215e-05\\
599.45	1.13119685850316e-05\\
599.46	1.09424104286961e-05\\
599.47	1.05778468282615e-05\\
599.48	1.02183310419918e-05\\
599.49	9.86391685580193e-06\\
599.5	9.5146585884498e-06\\
599.51	9.17061109676973e-06\\
599.52	8.83182978096524e-06\\
599.53	8.49837058993982e-06\\
599.54	8.17029002670236e-06\\
599.55	7.84764515380544e-06\\
599.56	7.53049359884274e-06\\
599.57	7.21889356001225e-06\\
599.58	6.91290381171078e-06\\
599.59	6.61258371020476e-06\\
599.6	6.31799319935483e-06\\
599.61	6.02919281637339e-06\\
599.62	5.74624369766713e-06\\
599.63	5.46920758472819e-06\\
599.64	5.1981468300686e-06\\
599.65	4.93312440323461e-06\\
599.66	4.67420389686606e-06\\
599.67	4.42144953282167e-06\\
599.68	4.1749261683599e-06\\
599.69	3.93469930238047e-06\\
599.7	3.70083508173524e-06\\
599.71	3.47340030758779e-06\\
599.72	3.25246244185264e-06\\
599.73	3.03808961367273e-06\\
599.74	2.83035062599232e-06\\
599.75	2.62931496216104e-06\\
599.76	2.43505279263513e-06\\
599.77	2.24763498171514e-06\\
599.78	2.06713309437029e-06\\
599.79	1.89361940312015e-06\\
599.8	1.72716689498566e-06\\
599.81	1.56784927851129e-06\\
599.82	1.41574099085662e-06\\
599.83	1.27091720493987e-06\\
599.84	1.13345383668736e-06\\
599.85	1.00342755231589e-06\\
599.86	8.8091577571392e-07\\
599.87	7.65996695880136e-07\\
599.88	6.58749274441706e-07\\
599.89	5.59253253243699e-07\\
599.9	4.67589162013102e-07\\
599.91	3.8383832609741e-07\\
599.92	3.0808287429171e-07\\
599.93	2.40405746710845e-07\\
599.94	1.8089070277609e-07\\
599.95	1.2962232925906e-07\\
599.96	8.66860484002169e-08\\
599.97	5.21681261331924e-08\\
599.98	2.61556803542173e-08\\
599.99	8.73668930083393e-09\\
600	0\\
};
\addplot [color=red!50!mycolor17,solid,forget plot]
  table[row sep=crcr]{%
0.01	0.00643330023985397\\
1.01	0.00643329966361264\\
2.01	0.00643329907515522\\
3.01	0.00643329847422214\\
4.01	0.00643329786054826\\
5.01	0.00643329723386302\\
6.01	0.00643329659388985\\
7.01	0.00643329594034644\\
8.01	0.00643329527294438\\
9.01	0.00643329459138911\\
10.01	0.00643329389537988\\
11.01	0.00643329318460946\\
12.01	0.0064332924587642\\
13.01	0.00643329171752365\\
14.01	0.00643329096056051\\
15.01	0.0064332901875405\\
16.01	0.00643328939812239\\
17.01	0.00643328859195762\\
18.01	0.00643328776869002\\
19.01	0.00643328692795602\\
20.01	0.00643328606938434\\
21.01	0.00643328519259551\\
22.01	0.00643328429720226\\
23.01	0.00643328338280896\\
24.01	0.00643328244901161\\
25.01	0.0064332814953974\\
26.01	0.0064332805215449\\
27.01	0.0064332795270238\\
28.01	0.00643327851139429\\
29.01	0.00643327747420746\\
30.01	0.00643327641500483\\
31.01	0.006433275333318\\
32.01	0.00643327422866862\\
33.01	0.00643327310056819\\
34.01	0.00643327194851781\\
35.01	0.00643327077200789\\
36.01	0.0064332695705179\\
37.01	0.0064332683435164\\
38.01	0.00643326709046047\\
39.01	0.00643326581079568\\
40.01	0.00643326450395578\\
41.01	0.00643326316936238\\
42.01	0.00643326180642484\\
43.01	0.00643326041453989\\
44.01	0.00643325899309134\\
45.01	0.00643325754144988\\
46.01	0.00643325605897279\\
47.01	0.00643325454500375\\
48.01	0.00643325299887223\\
49.01	0.00643325141989364\\
50.01	0.00643324980736839\\
51.01	0.00643324816058252\\
52.01	0.00643324647880629\\
53.01	0.00643324476129471\\
54.01	0.00643324300728684\\
55.01	0.00643324121600541\\
56.01	0.00643323938665682\\
57.01	0.00643323751843024\\
58.01	0.00643323561049773\\
59.01	0.00643323366201344\\
60.01	0.00643323167211386\\
61.01	0.00643322963991663\\
62.01	0.00643322756452081\\
63.01	0.00643322544500617\\
64.01	0.00643322328043285\\
65.01	0.00643322106984085\\
66.01	0.00643321881224981\\
67.01	0.0064332165066584\\
68.01	0.00643321415204387\\
69.01	0.00643321174736173\\
70.01	0.00643320929154495\\
71.01	0.00643320678350395\\
72.01	0.0064332042221259\\
73.01	0.00643320160627412\\
74.01	0.00643319893478762\\
75.01	0.00643319620648064\\
76.01	0.00643319342014218\\
77.01	0.00643319057453537\\
78.01	0.00643318766839686\\
79.01	0.00643318470043639\\
80.01	0.00643318166933613\\
81.01	0.00643317857375005\\
82.01	0.00643317541230344\\
83.01	0.00643317218359219\\
84.01	0.00643316888618216\\
85.01	0.00643316551860874\\
86.01	0.00643316207937583\\
87.01	0.00643315856695557\\
88.01	0.0064331549797873\\
89.01	0.00643315131627714\\
90.01	0.00643314757479708\\
91.01	0.0064331437536844\\
92.01	0.00643313985124065\\
93.01	0.00643313586573155\\
94.01	0.00643313179538536\\
95.01	0.00643312763839263\\
96.01	0.00643312339290536\\
97.01	0.0064331190570357\\
98.01	0.0064331146288558\\
99.01	0.00643311010639661\\
100.01	0.00643310548764689\\
101.01	0.00643310077055248\\
102.01	0.00643309595301526\\
103.01	0.00643309103289239\\
104.01	0.00643308600799515\\
105.01	0.00643308087608805\\
106.01	0.00643307563488783\\
107.01	0.00643307028206251\\
108.01	0.0064330648152302\\
109.01	0.00643305923195813\\
110.01	0.00643305352976162\\
111.01	0.00643304770610254\\
112.01	0.00643304175838908\\
113.01	0.00643303568397349\\
114.01	0.00643302948015176\\
115.01	0.00643302314416188\\
116.01	0.00643301667318305\\
117.01	0.0064330100643339\\
118.01	0.0064330033146717\\
119.01	0.00643299642119072\\
120.01	0.00643298938082084\\
121.01	0.00643298219042644\\
122.01	0.00643297484680501\\
123.01	0.00643296734668524\\
124.01	0.00643295968672614\\
125.01	0.00643295186351517\\
126.01	0.00643294387356687\\
127.01	0.00643293571332134\\
128.01	0.00643292737914249\\
129.01	0.00643291886731646\\
130.01	0.00643291017404996\\
131.01	0.00643290129546863\\
132.01	0.00643289222761531\\
133.01	0.00643288296644819\\
134.01	0.0064328735078389\\
135.01	0.00643286384757091\\
136.01	0.00643285398133756\\
137.01	0.00643284390474001\\
138.01	0.00643283361328532\\
139.01	0.00643282310238452\\
140.01	0.00643281236735039\\
141.01	0.00643280140339574\\
142.01	0.00643279020563064\\
143.01	0.00643277876906069\\
144.01	0.0064327670885847\\
145.01	0.00643275515899224\\
146.01	0.0064327429749616\\
147.01	0.0064327305310571\\
148.01	0.0064327178217268\\
149.01	0.00643270484129986\\
150.01	0.0064326915839843\\
151.01	0.00643267804386418\\
152.01	0.00643266421489673\\
153.01	0.00643265009091029\\
154.01	0.00643263566560078\\
155.01	0.00643262093252944\\
156.01	0.0064326058851197\\
157.01	0.00643259051665394\\
158.01	0.0064325748202712\\
159.01	0.00643255878896333\\
160.01	0.00643254241557224\\
161.01	0.00643252569278664\\
162.01	0.00643250861313882\\
163.01	0.00643249116900114\\
164.01	0.00643247335258249\\
165.01	0.00643245515592519\\
166.01	0.00643243657090099\\
167.01	0.00643241758920761\\
168.01	0.00643239820236508\\
169.01	0.00643237840171157\\
170.01	0.00643235817839993\\
171.01	0.00643233752339339\\
172.01	0.00643231642746167\\
173.01	0.00643229488117671\\
174.01	0.0064322728749082\\
175.01	0.00643225039881996\\
176.01	0.00643222744286454\\
177.01	0.00643220399677939\\
178.01	0.00643218005008191\\
179.01	0.00643215559206491\\
180.01	0.00643213061179134\\
181.01	0.00643210509808991\\
182.01	0.00643207903954972\\
183.01	0.00643205242451492\\
184.01	0.00643202524108001\\
185.01	0.00643199747708377\\
186.01	0.00643196912010412\\
187.01	0.0064319401574526\\
188.01	0.00643191057616808\\
189.01	0.00643188036301151\\
190.01	0.00643184950445939\\
191.01	0.00643181798669802\\
192.01	0.00643178579561694\\
193.01	0.00643175291680245\\
194.01	0.00643171933553146\\
195.01	0.00643168503676419\\
196.01	0.00643165000513781\\
197.01	0.00643161422495921\\
198.01	0.00643157768019776\\
199.01	0.00643154035447815\\
200.01	0.00643150223107265\\
201.01	0.00643146329289373\\
202.01	0.00643142352248596\\
203.01	0.00643138290201833\\
204.01	0.00643134141327574\\
205.01	0.00643129903765055\\
206.01	0.00643125575613437\\
207.01	0.00643121154930904\\
208.01	0.00643116639733792\\
209.01	0.00643112027995636\\
210.01	0.00643107317646256\\
211.01	0.006431025065708\\
212.01	0.00643097592608755\\
213.01	0.00643092573552946\\
214.01	0.0064308744714851\\
215.01	0.00643082211091856\\
216.01	0.00643076863029572\\
217.01	0.00643071400557331\\
218.01	0.00643065821218798\\
219.01	0.00643060122504442\\
220.01	0.00643054301850361\\
221.01	0.00643048356637099\\
222.01	0.0064304228418841\\
223.01	0.00643036081769961\\
224.01	0.0064302974658809\\
225.01	0.00643023275788465\\
226.01	0.00643016666454723\\
227.01	0.00643009915607106\\
228.01	0.00643003020200987\\
229.01	0.00642995977125506\\
230.01	0.00642988783202002\\
231.01	0.00642981435182537\\
232.01	0.00642973929748326\\
233.01	0.00642966263508117\\
234.01	0.00642958432996606\\
235.01	0.00642950434672698\\
236.01	0.00642942264917838\\
237.01	0.00642933920034242\\
238.01	0.00642925396243062\\
239.01	0.00642916689682585\\
240.01	0.00642907796406301\\
241.01	0.00642898712380994\\
242.01	0.00642889433484729\\
243.01	0.0064287995550484\\
244.01	0.00642870274135785\\
245.01	0.00642860384977059\\
246.01	0.0064285028353096\\
247.01	0.00642839965200366\\
248.01	0.00642829425286382\\
249.01	0.00642818658985994\\
250.01	0.00642807661389634\\
251.01	0.00642796427478643\\
252.01	0.00642784952122762\\
253.01	0.00642773230077468\\
254.01	0.00642761255981288\\
255.01	0.0064274902435299\\
256.01	0.00642736529588781\\
257.01	0.00642723765959359\\
258.01	0.00642710727606899\\
259.01	0.00642697408541999\\
260.01	0.00642683802640458\\
261.01	0.0064266990364007\\
262.01	0.0064265570513725\\
263.01	0.00642641200583604\\
264.01	0.0064262638328237\\
265.01	0.0064261124638481\\
266.01	0.00642595782886436\\
267.01	0.00642579985623197\\
268.01	0.00642563847267495\\
269.01	0.00642547360324142\\
270.01	0.00642530517126144\\
271.01	0.00642513309830385\\
272.01	0.0064249573041322\\
273.01	0.00642477770665875\\
274.01	0.00642459422189782\\
275.01	0.0064244067639166\\
276.01	0.00642421524478602\\
277.01	0.0064240195745288\\
278.01	0.00642381966106671\\
279.01	0.00642361541016589\\
280.01	0.00642340672538038\\
281.01	0.00642319350799416\\
282.01	0.00642297565696161\\
283.01	0.0064227530688452\\
284.01	0.00642252563775196\\
285.01	0.00642229325526805\\
286.01	0.00642205581039044\\
287.01	0.00642181318945708\\
288.01	0.0064215652760747\\
289.01	0.00642131195104404\\
290.01	0.00642105309228254\\
291.01	0.00642078857474497\\
292.01	0.00642051827034081\\
293.01	0.00642024204784927\\
294.01	0.00641995977283155\\
295.01	0.00641967130753948\\
296.01	0.00641937651082184\\
297.01	0.00641907523802669\\
298.01	0.00641876734090096\\
299.01	0.00641845266748652\\
300.01	0.00641813106201194\\
301.01	0.00641780236478143\\
302.01	0.00641746641205873\\
303.01	0.00641712303594808\\
304.01	0.00641677206427006\\
305.01	0.00641641332043338\\
306.01	0.00641604662330142\\
307.01	0.00641567178705515\\
308.01	0.00641528862104976\\
309.01	0.00641489692966672\\
310.01	0.00641449651216027\\
311.01	0.00641408716249786\\
312.01	0.00641366866919556\\
313.01	0.00641324081514603\\
314.01	0.00641280337744097\\
315.01	0.00641235612718674\\
316.01	0.00641189882931247\\
317.01	0.00641143124237128\\
318.01	0.00641095311833422\\
319.01	0.00641046420237525\\
320.01	0.00640996423264893\\
321.01	0.0064094529400591\\
322.01	0.00640893004801868\\
323.01	0.00640839527219962\\
324.01	0.00640784832027409\\
325.01	0.00640728889164443\\
326.01	0.00640671667716335\\
327.01	0.00640613135884255\\
328.01	0.00640553260954994\\
329.01	0.00640492009269518\\
330.01	0.00640429346190234\\
331.01	0.00640365236066948\\
332.01	0.00640299642201571\\
333.01	0.00640232526811258\\
334.01	0.00640163850990212\\
335.01	0.00640093574669889\\
336.01	0.00640021656577682\\
337.01	0.0063994805419384\\
338.01	0.00639872723706807\\
339.01	0.00639795619966709\\
340.01	0.00639716696436969\\
341.01	0.00639635905144049\\
342.01	0.00639553196625191\\
343.01	0.00639468519874068\\
344.01	0.00639381822284263\\
345.01	0.00639293049590642\\
346.01	0.00639202145808291\\
347.01	0.00639109053169153\\
348.01	0.00639013712056225\\
349.01	0.00638916060935153\\
350.01	0.00638816036283289\\
351.01	0.0063871357251601\\
352.01	0.00638608601910334\\
353.01	0.00638501054525607\\
354.01	0.00638390858121396\\
355.01	0.0063827793807233\\
356.01	0.00638162217279975\\
357.01	0.00638043616081613\\
358.01	0.00637922052155926\\
359.01	0.00637797440425547\\
360.01	0.00637669692956446\\
361.01	0.00637538718854211\\
362.01	0.00637404424157169\\
363.01	0.00637266711726575\\
364.01	0.00637125481133701\\
365.01	0.00636980628544193\\
366.01	0.00636832046599739\\
367.01	0.00636679624297341\\
368.01	0.00636523246866418\\
369.01	0.00636362795644195\\
370.01	0.00636198147949785\\
371.01	0.00636029176957551\\
372.01	0.00635855751570492\\
373.01	0.00635677736294391\\
374.01	0.00635494991113921\\
375.01	0.00635307371371823\\
376.01	0.00635114727652669\\
377.01	0.00634916905673055\\
378.01	0.00634713746180239\\
379.01	0.0063450508486184\\
380.01	0.0063429075226946\\
381.01	0.00634070573759887\\
382.01	0.00633844369457892\\
383.01	0.00633611954245547\\
384.01	0.00633373137783847\\
385.01	0.00633127724573051\\
386.01	0.00632875514059788\\
387.01	0.00632616300799689\\
388.01	0.00632349874686038\\
389.01	0.00632076021256291\\
390.01	0.00631794522090061\\
391.01	0.0063150515531395\\
392.01	0.00631207696230492\\
393.01	0.00630901918090426\\
394.01	0.00630587593029035\\
395.01	0.00630264493189209\\
396.01	0.00629932392054452\\
397.01	0.00629591066015217\\
398.01	0.00629240296190524\\
399.01	0.0062887987052299\\
400.01	0.00628509586158552\\
401.01	0.0062812925210984\\
402.01	0.0062773869218394\\
403.01	0.00627337748125873\\
404.01	0.0062692628288728\\
405.01	0.00626504183867491\\
406.01	0.00626071365886577\\
407.01	0.00625627773524493\\
408.01	0.00625173382286064\\
409.01	0.0062470819780779\\
410.01	0.00624232251986272\\
411.01	0.00623745594444802\\
412.01	0.00623248277120422\\
413.01	0.00622740328765818\\
414.01	0.00622221673713735\\
415.01	0.00621692070537143\\
416.01	0.00621151246139671\\
417.01	0.00620598916715711\\
418.01	0.0062003478704892\\
419.01	0.00619458549751251\\
420.01	0.0061886988443706\\
421.01	0.00618268456826306\\
422.01	0.00617653917770893\\
423.01	0.00617025902193658\\
424.01	0.0061638402793036\\
425.01	0.00615727894469201\\
426.01	0.00615057081576944\\
427.01	0.00614371147799952\\
428.01	0.00613669628828092\\
429.01	0.00612952035707672\\
430.01	0.00612217852888811\\
431.01	0.00611466536090582\\
432.01	0.00610697509966166\\
433.01	0.00609910165548131\\
434.01	0.00609103857452474\\
435.01	0.00608277900817753\\
436.01	0.00607431567953827\\
437.01	0.00606564084672496\\
438.01	0.00605674626270089\\
439.01	0.00604762313130064\\
440.01	0.00603826205911502\\
441.01	0.00602865300287525\\
442.01	0.00601878521196404\\
443.01	0.00600864716567108\\
444.01	0.00599822650481147\\
445.01	0.00598750995734001\\
446.01	0.00597648325762419\\
447.01	0.00596513105909743\\
448.01	0.00595343684010651\\
449.01	0.00594138280290273\\
450.01	0.005928949765932\\
451.01	0.00591611704986275\\
452.01	0.0059028623581921\\
453.01	0.00588916165380416\\
454.01	0.00587498903373414\\
455.01	0.00586031660584085\\
456.01	0.00584511437208053\\
457.01	0.00582935012488772\\
458.01	0.00581298936587319\\
459.01	0.00579599525945634\\
460.01	0.00577832863859256\\
461.01	0.00575994808581527\\
462.01	0.00574081011055646\\
463.01	0.00572086944900231\\
464.01	0.00570007956420959\\
465.01	0.00567839341078409\\
466.01	0.00565576454955953\\
467.01	0.00563214873078398\\
468.01	0.00560750610165938\\
469.01	0.0055818042429304\\
470.01	0.00555502230411261\\
471.01	0.0055271565999938\\
472.01	0.00549822810063586\\
473.01	0.00546829475838497\\
474.01	0.00543770048875961\\
475.01	0.0054066488083951\\
476.01	0.00537514743510029\\
477.01	0.00534320637975664\\
478.01	0.00531083814295213\\
479.01	0.00527805803414741\\
480.01	0.00524488452308555\\
481.01	0.00521133962399908\\
482.01	0.00517744931233549\\
483.01	0.00514324397220679\\
484.01	0.00510875887058511\\
485.01	0.00507403465119368\\
486.01	0.00503911783676843\\
487.01	0.0050040613225054\\
488.01	0.00496892483227736\\
489.01	0.00493377529014779\\
490.01	0.00489868707071135\\
491.01	0.00486374205840157\\
492.01	0.00482902941053039\\
493.01	0.00479464488819632\\
494.01	0.00476068957112154\\
495.01	0.00472726770782663\\
496.01	0.00469448336651602\\
497.01	0.00466243544789561\\
498.01	0.00463121047938355\\
499.01	0.00460087238312644\\
500.01	0.00457144816890996\\
501.01	0.00454290277306364\\
502.01	0.00451468377798736\\
503.01	0.00448648666934954\\
504.01	0.00445834215488247\\
505.01	0.00443028109400982\\
506.01	0.00440233396611083\\
507.01	0.00437453021896771\\
508.01	0.00434689749021075\\
509.01	0.00431946068556158\\
510.01	0.00429224090559129\\
511.01	0.00426525422123244\\
512.01	0.00423851031014622\\
513.01	0.00421201098427128\\
514.01	0.0041857486660573\\
515.01	0.00415970491050224\\
516.01	0.00413384912698003\\
517.01	0.00410813773590309\\
518.01	0.00408251411026784\\
519.01	0.00405690981719585\\
520.01	0.00403124793559125\\
521.01	0.00400545145824772\\
522.01	0.00397948220161627\\
523.01	0.00395333152360722\\
524.01	0.00392698916657118\\
525.01	0.00390044210047198\\
526.01	0.00387367439700243\\
527.01	0.00384666718283707\\
528.01	0.00381939869609618\\
529.01	0.00379184447350555\\
530.01	0.00376397769607122\\
531.01	0.003735769718089\\
532.01	0.00370719079578369\\
533.01	0.00367821101462431\\
534.01	0.0036488013837574\\
535.01	0.00361893501550884\\
536.01	0.00358858822822191\\
537.01	0.00355774121106488\\
538.01	0.00352637650554267\\
539.01	0.00349447638476056\\
540.01	0.00346202253866048\\
541.01	0.00342899620714643\\
542.01	0.00339537833345592\\
543.01	0.00336114973292802\\
544.01	0.00332629126957986\\
545.01	0.0032907840270283\\
546.01	0.00325460945664299\\
547.01	0.00321774948331401\\
548.01	0.00318018654712834\\
549.01	0.00314190356106398\\
550.01	0.00310288377422265\\
551.01	0.00306311055643913\\
552.01	0.00302256723038261\\
553.01	0.00298123706917909\\
554.01	0.00293910334223729\\
555.01	0.00289614935795801\\
556.01	0.00285235850019123\\
557.01	0.00280771425681943\\
558.01	0.00276220023958686\\
559.01	0.00271580019548417\\
560.01	0.00266849801167274\\
561.01	0.00262027771800452\\
562.01	0.00257112349337822\\
563.01	0.00252101968387742\\
564.01	0.00246995083963625\\
565.01	0.00241790176743256\\
566.01	0.00236485758999453\\
567.01	0.00231080380985966\\
568.01	0.00225572637833465\\
569.01	0.00219961177044417\\
570.01	0.0021424470669859\\
571.01	0.00208422004494936\\
572.01	0.00202491927756344\\
573.01	0.00196453424505074\\
574.01	0.00190305545671095\\
575.01	0.00184047458421142\\
576.01	0.00177678460515771\\
577.01	0.00171197995585397\\
578.01	0.00164605669237123\\
579.01	0.0015790126587281\\
580.01	0.00151084766032477\\
581.01	0.00144156363987375\\
582.01	0.0013711648518907\\
583.01	0.00129965803028412\\
584.01	0.00122705254163708\\
585.01	0.0011533605143295\\
586.01	0.00107859693060129\\
587.01	0.00100277966487854\\
588.01	0.000925929446941498\\
589.01	0.000848069722467409\\
590.01	0.00076922637580672\\
591.01	0.000689427270172824\\
592.01	0.000608701548246961\\
593.01	0.000527078620894876\\
594.01	0.000444586752481062\\
595.01	0.000361251127183708\\
596.01	0.000277091250558454\\
597.01	0.000192117502885737\\
598.01	0.00010632661372072\\
599.01	3.1823044268001e-05\\
599.02	3.12770957282913e-05\\
599.03	3.07343620667162e-05\\
599.04	3.01948752267685e-05\\
599.05	2.96586674650972e-05\\
599.06	2.91257713545909e-05\\
599.07	2.85962197874873e-05\\
599.08	2.80700459785194e-05\\
599.09	2.75472834680903e-05\\
599.1	2.70279661254754e-05\\
599.11	2.65121281520647e-05\\
599.12	2.59998040846362e-05\\
599.13	2.54910287986413e-05\\
599.14	2.49858375115618e-05\\
599.15	2.44842657862541e-05\\
599.16	2.39863495343721e-05\\
599.17	2.34921250197827e-05\\
599.18	2.30016288620422e-05\\
599.19	2.25148980398954e-05\\
599.2	2.2031969894816e-05\\
599.21	2.15528821345731e-05\\
599.22	2.10776728368396e-05\\
599.23	2.06063804528353e-05\\
599.24	2.01390438110006e-05\\
599.25	1.96757021207125e-05\\
599.26	1.92163949760387e-05\\
599.27	1.8761162359519e-05\\
599.28	1.8310044645994e-05\\
599.29	1.78630826064737e-05\\
599.3	1.7420317412014e-05\\
599.31	1.69817906376935e-05\\
599.32	1.65475442665524e-05\\
599.33	1.61176206936401e-05\\
599.34	1.5692062730048e-05\\
599.35	1.52709136070281e-05\\
599.36	1.48542169801109e-05\\
599.37	1.44420169332915e-05\\
599.38	1.40343579832485e-05\\
599.39	1.36312859124171e-05\\
599.4	1.32328499385954e-05\\
599.41	1.2839099767523e-05\\
599.42	1.24500855976737e-05\\
599.43	1.20658581250939e-05\\
599.44	1.16864685483029e-05\\
599.45	1.13119685732268e-05\\
599.46	1.09424104181854e-05\\
599.47	1.05778468189321e-05\\
599.48	1.02183310337345e-05\\
599.49	9.86391684851956e-06\\
599.5	9.51465858204867e-06\\
599.51	9.17061109116483e-06\\
599.52	8.83182977607332e-06\\
599.53	8.49837058568975e-06\\
599.54	8.17029002302475e-06\\
599.55	7.84764515063437e-06\\
599.56	7.53049359612443e-06\\
599.57	7.21889355768945e-06\\
599.58	6.91290380973666e-06\\
599.59	6.61258370854116e-06\\
599.6	6.31799319795665e-06\\
599.61	6.02919281520418e-06\\
599.62	5.74624369669915e-06\\
599.63	5.46920758392848e-06\\
599.64	5.19814682941461e-06\\
599.65	4.93312440270205e-06\\
599.66	4.67420389643758e-06\\
599.67	4.42144953247993e-06\\
599.68	4.17492616808929e-06\\
599.69	3.93469930216883e-06\\
599.7	3.70083508157044e-06\\
599.71	3.47340030746289e-06\\
599.72	3.25246244175723e-06\\
599.73	3.03808961360334e-06\\
599.74	2.83035062593855e-06\\
599.75	2.62931496212288e-06\\
599.76	2.43505279260738e-06\\
599.77	2.24763498169432e-06\\
599.78	2.06713309435641e-06\\
599.79	1.89361940310974e-06\\
599.8	1.72716689497872e-06\\
599.81	1.56784927850956e-06\\
599.82	1.41574099085315e-06\\
599.83	1.27091720493813e-06\\
599.84	1.13345383668563e-06\\
599.85	1.00342755231589e-06\\
599.86	8.8091577571392e-07\\
599.87	7.65996695880136e-07\\
599.88	6.58749274441706e-07\\
599.89	5.59253253243699e-07\\
599.9	4.67589162011367e-07\\
599.91	3.83838326099145e-07\\
599.92	3.08082874293444e-07\\
599.93	2.40405746710845e-07\\
599.94	1.80890702777825e-07\\
599.95	1.2962232925906e-07\\
599.96	8.66860484019516e-08\\
599.97	5.21681261349272e-08\\
599.98	2.61556803542173e-08\\
599.99	8.73668930083393e-09\\
600	0\\
};
\addplot [color=red!40!mycolor19,solid,forget plot]
  table[row sep=crcr]{%
0.01	0.0062874344136244\\
1.01	0.00628743349800273\\
2.01	0.00628743256299474\\
3.01	0.00628743160818932\\
4.01	0.00628743063316646\\
5.01	0.00628742963749692\\
6.01	0.00628742862074276\\
7.01	0.0062874275824567\\
8.01	0.00628742652218161\\
9.01	0.00628742543945105\\
10.01	0.00628742433378832\\
11.01	0.0062874232047069\\
12.01	0.00628742205170972\\
13.01	0.00628742087428894\\
14.01	0.00628741967192625\\
15.01	0.00628741844409226\\
16.01	0.00628741719024629\\
17.01	0.00628741590983598\\
18.01	0.00628741460229756\\
19.01	0.00628741326705499\\
20.01	0.00628741190351998\\
21.01	0.00628741051109212\\
22.01	0.00628740908915761\\
23.01	0.00628740763709008\\
24.01	0.00628740615424945\\
25.01	0.00628740463998247\\
26.01	0.0062874030936216\\
27.01	0.00628740151448521\\
28.01	0.00628739990187734\\
29.01	0.00628739825508692\\
30.01	0.0062873965733878\\
31.01	0.00628739485603868\\
32.01	0.00628739310228218\\
33.01	0.00628739131134478\\
34.01	0.00628738948243671\\
35.01	0.00628738761475131\\
36.01	0.0062873857074647\\
37.01	0.00628738375973543\\
38.01	0.00628738177070424\\
39.01	0.00628737973949339\\
40.01	0.00628737766520673\\
41.01	0.00628737554692879\\
42.01	0.00628737338372471\\
43.01	0.0062873711746397\\
44.01	0.00628736891869874\\
45.01	0.00628736661490585\\
46.01	0.00628736426224395\\
47.01	0.00628736185967426\\
48.01	0.00628735940613593\\
49.01	0.0062873569005455\\
50.01	0.00628735434179671\\
51.01	0.006287351728759\\
52.01	0.00628734906027848\\
53.01	0.00628734633517628\\
54.01	0.00628734355224853\\
55.01	0.0062873407102658\\
56.01	0.00628733780797224\\
57.01	0.00628733484408533\\
58.01	0.00628733181729537\\
59.01	0.00628732872626476\\
60.01	0.00628732556962706\\
61.01	0.0062873223459872\\
62.01	0.00628731905392003\\
63.01	0.00628731569197016\\
64.01	0.00628731225865114\\
65.01	0.00628730875244501\\
66.01	0.00628730517180127\\
67.01	0.0062873015151364\\
68.01	0.00628729778083321\\
69.01	0.00628729396723982\\
70.01	0.00628729007266971\\
71.01	0.00628728609539976\\
72.01	0.00628728203367046\\
73.01	0.00628727788568475\\
74.01	0.00628727364960721\\
75.01	0.00628726932356354\\
76.01	0.00628726490563911\\
77.01	0.00628726039387883\\
78.01	0.00628725578628547\\
79.01	0.00628725108081969\\
80.01	0.00628724627539847\\
81.01	0.00628724136789457\\
82.01	0.00628723635613503\\
83.01	0.00628723123790091\\
84.01	0.00628722601092583\\
85.01	0.00628722067289502\\
86.01	0.00628721522144452\\
87.01	0.00628720965415978\\
88.01	0.0062872039685749\\
89.01	0.00628719816217146\\
90.01	0.00628719223237714\\
91.01	0.00628718617656494\\
92.01	0.00628717999205191\\
93.01	0.00628717367609755\\
94.01	0.00628716722590323\\
95.01	0.00628716063861062\\
96.01	0.00628715391130024\\
97.01	0.0062871470409906\\
98.01	0.00628714002463641\\
99.01	0.00628713285912764\\
100.01	0.0062871255412879\\
101.01	0.00628711806787282\\
102.01	0.00628711043556921\\
103.01	0.00628710264099285\\
104.01	0.00628709468068761\\
105.01	0.00628708655112356\\
106.01	0.0062870782486955\\
107.01	0.00628706976972144\\
108.01	0.00628706111044066\\
109.01	0.00628705226701221\\
110.01	0.00628704323551351\\
111.01	0.00628703401193834\\
112.01	0.00628702459219453\\
113.01	0.00628701497210318\\
114.01	0.00628700514739588\\
115.01	0.00628699511371355\\
116.01	0.00628698486660368\\
117.01	0.00628697440151922\\
118.01	0.00628696371381592\\
119.01	0.00628695279875048\\
120.01	0.00628694165147855\\
121.01	0.00628693026705234\\
122.01	0.00628691864041858\\
123.01	0.00628690676641635\\
124.01	0.00628689463977442\\
125.01	0.00628688225510939\\
126.01	0.00628686960692276\\
127.01	0.00628685668959899\\
128.01	0.00628684349740249\\
129.01	0.00628683002447583\\
130.01	0.00628681626483625\\
131.01	0.00628680221237351\\
132.01	0.00628678786084703\\
133.01	0.00628677320388323\\
134.01	0.00628675823497253\\
135.01	0.00628674294746665\\
136.01	0.00628672733457538\\
137.01	0.00628671138936365\\
138.01	0.00628669510474871\\
139.01	0.00628667847349666\\
140.01	0.00628666148821946\\
141.01	0.00628664414137117\\
142.01	0.00628662642524527\\
143.01	0.00628660833197088\\
144.01	0.0062865898535092\\
145.01	0.00628657098165012\\
146.01	0.00628655170800836\\
147.01	0.00628653202401986\\
148.01	0.00628651192093804\\
149.01	0.00628649138982988\\
150.01	0.00628647042157179\\
151.01	0.00628644900684564\\
152.01	0.0062864271361349\\
153.01	0.00628640479971971\\
154.01	0.00628638198767357\\
155.01	0.00628635868985792\\
156.01	0.00628633489591809\\
157.01	0.00628631059527905\\
158.01	0.00628628577713991\\
159.01	0.00628626043047003\\
160.01	0.00628623454400331\\
161.01	0.00628620810623378\\
162.01	0.00628618110540997\\
163.01	0.00628615352953008\\
164.01	0.00628612536633663\\
165.01	0.00628609660331089\\
166.01	0.00628606722766725\\
167.01	0.00628603722634769\\
168.01	0.00628600658601601\\
169.01	0.00628597529305184\\
170.01	0.00628594333354455\\
171.01	0.00628591069328737\\
172.01	0.0062858773577703\\
173.01	0.00628584331217474\\
174.01	0.00628580854136631\\
175.01	0.00628577302988811\\
176.01	0.00628573676195391\\
177.01	0.00628569972144138\\
178.01	0.00628566189188479\\
179.01	0.00628562325646768\\
180.01	0.00628558379801538\\
181.01	0.00628554349898735\\
182.01	0.00628550234146987\\
183.01	0.00628546030716749\\
184.01	0.006285417377395\\
185.01	0.00628537353306957\\
186.01	0.00628532875470208\\
187.01	0.00628528302238823\\
188.01	0.00628523631580016\\
189.01	0.00628518861417705\\
190.01	0.00628513989631629\\
191.01	0.00628509014056373\\
192.01	0.00628503932480434\\
193.01	0.00628498742645229\\
194.01	0.00628493442244111\\
195.01	0.0062848802892134\\
196.01	0.00628482500271029\\
197.01	0.00628476853836071\\
198.01	0.00628471087107083\\
199.01	0.00628465197521284\\
200.01	0.00628459182461342\\
201.01	0.0062845303925423\\
202.01	0.00628446765170066\\
203.01	0.00628440357420835\\
204.01	0.00628433813159232\\
205.01	0.00628427129477378\\
206.01	0.00628420303405488\\
207.01	0.00628413331910632\\
208.01	0.00628406211895315\\
209.01	0.0062839894019618\\
210.01	0.00628391513582533\\
211.01	0.00628383928754987\\
212.01	0.00628376182343942\\
213.01	0.00628368270908119\\
214.01	0.0062836019093302\\
215.01	0.00628351938829409\\
216.01	0.00628343510931667\\
217.01	0.00628334903496229\\
218.01	0.0062832611269986\\
219.01	0.00628317134637997\\
220.01	0.00628307965323034\\
221.01	0.0062829860068251\\
222.01	0.00628289036557343\\
223.01	0.00628279268699974\\
224.01	0.00628269292772447\\
225.01	0.0062825910434453\\
226.01	0.00628248698891729\\
227.01	0.0062823807179326\\
228.01	0.00628227218330053\\
229.01	0.00628216133682583\\
230.01	0.00628204812928764\\
231.01	0.00628193251041765\\
232.01	0.00628181442887756\\
233.01	0.00628169383223631\\
234.01	0.00628157066694637\\
235.01	0.00628144487832066\\
236.01	0.0062813164105073\\
237.01	0.00628118520646496\\
238.01	0.00628105120793778\\
239.01	0.00628091435542867\\
240.01	0.00628077458817327\\
241.01	0.00628063184411234\\
242.01	0.00628048605986412\\
243.01	0.00628033717069563\\
244.01	0.00628018511049424\\
245.01	0.0062800298117374\\
246.01	0.00627987120546236\\
247.01	0.00627970922123501\\
248.01	0.00627954378711856\\
249.01	0.00627937482964052\\
250.01	0.00627920227375994\\
251.01	0.00627902604283324\\
252.01	0.00627884605857934\\
253.01	0.00627866224104423\\
254.01	0.00627847450856465\\
255.01	0.00627828277773102\\
256.01	0.00627808696334902\\
257.01	0.00627788697840107\\
258.01	0.00627768273400627\\
259.01	0.0062774741393794\\
260.01	0.00627726110178995\\
261.01	0.00627704352651842\\
262.01	0.00627682131681328\\
263.01	0.00627659437384595\\
264.01	0.00627636259666503\\
265.01	0.00627612588214916\\
266.01	0.00627588412495935\\
267.01	0.00627563721748919\\
268.01	0.00627538504981479\\
269.01	0.00627512750964293\\
270.01	0.006274864482258\\
271.01	0.00627459585046796\\
272.01	0.00627432149454862\\
273.01	0.00627404129218614\\
274.01	0.00627375511841894\\
275.01	0.0062734628455778\\
276.01	0.00627316434322387\\
277.01	0.00627285947808595\\
278.01	0.00627254811399527\\
279.01	0.00627223011181933\\
280.01	0.0062719053293935\\
281.01	0.00627157362145101\\
282.01	0.00627123483955036\\
283.01	0.00627088883200205\\
284.01	0.00627053544379253\\
285.01	0.00627017451650581\\
286.01	0.00626980588824328\\
287.01	0.0062694293935412\\
288.01	0.00626904486328561\\
289.01	0.00626865212462511\\
290.01	0.00626825100088108\\
291.01	0.00626784131145485\\
292.01	0.00626742287173261\\
293.01	0.0062669954929873\\
294.01	0.0062665589822768\\
295.01	0.00626611314234064\\
296.01	0.00626565777149172\\
297.01	0.00626519266350579\\
298.01	0.00626471760750708\\
299.01	0.00626423238784969\\
300.01	0.00626373678399616\\
301.01	0.00626323057039097\\
302.01	0.0062627135163308\\
303.01	0.00626218538582935\\
304.01	0.00626164593747834\\
305.01	0.00626109492430317\\
306.01	0.00626053209361404\\
307.01	0.00625995718685075\\
308.01	0.00625936993942274\\
309.01	0.00625877008054312\\
310.01	0.00625815733305593\\
311.01	0.00625753141325752\\
312.01	0.00625689203071042\\
313.01	0.00625623888805078\\
314.01	0.00625557168078717\\
315.01	0.00625489009709179\\
316.01	0.00625419381758304\\
317.01	0.00625348251509964\\
318.01	0.00625275585446366\\
319.01	0.00625201349223544\\
320.01	0.00625125507645604\\
321.01	0.00625048024637951\\
322.01	0.00624968863219228\\
323.01	0.00624887985472068\\
324.01	0.00624805352512373\\
325.01	0.0062472092445725\\
326.01	0.00624634660391358\\
327.01	0.00624546518331611\\
328.01	0.00624456455190193\\
329.01	0.00624364426735621\\
330.01	0.0062427038755192\\
331.01	0.00624174290995664\\
332.01	0.00624076089150639\\
333.01	0.00623975732780229\\
334.01	0.00623873171277178\\
335.01	0.00623768352610487\\
336.01	0.00623661223269421\\
337.01	0.0062355172820435\\
338.01	0.0062343981076407\\
339.01	0.006233254126295\\
340.01	0.00623208473743404\\
341.01	0.00623088932235893\\
342.01	0.00622966724345284\\
343.01	0.00622841784334071\\
344.01	0.00622714044399542\\
345.01	0.00622583434578652\\
346.01	0.0062244988264667\\
347.01	0.00622313314009197\\
348.01	0.00622173651586795\\
349.01	0.0062203081569188\\
350.01	0.00621884723897052\\
351.01	0.00621735290894193\\
352.01	0.00621582428343583\\
353.01	0.00621426044712168\\
354.01	0.00621266045099962\\
355.01	0.00621102331053706\\
356.01	0.00620934800366509\\
357.01	0.00620763346862314\\
358.01	0.00620587860163797\\
359.01	0.00620408225442216\\
360.01	0.006202243231476\\
361.01	0.00620036028717386\\
362.01	0.00619843212261682\\
363.01	0.00619645738222726\\
364.01	0.006194434650064\\
365.01	0.00619236244583001\\
366.01	0.00619023922054431\\
367.01	0.0061880633518454\\
368.01	0.00618583313889235\\
369.01	0.00618354679682352\\
370.01	0.00618120245073172\\
371.01	0.00617879812910809\\
372.01	0.00617633175670501\\
373.01	0.00617380114676127\\
374.01	0.00617120399252893\\
375.01	0.00616853785803526\\
376.01	0.00616580016800745\\
377.01	0.0061629881968826\\
378.01	0.00616009905681787\\
379.01	0.00615712968461129\\
380.01	0.00615407682743772\\
381.01	0.00615093702729765\\
382.01	0.00614770660407561\\
383.01	0.00614438163709984\\
384.01	0.00614095794509658\\
385.01	0.00613743106443638\\
386.01	0.0061337962255768\\
387.01	0.00613004832762434\\
388.01	0.00612618191096159\\
389.01	0.00612219112792623\\
390.01	0.00611806971158247\\
391.01	0.00611381094270715\\
392.01	0.00610940761522144\\
393.01	0.00610485200045085\\
394.01	0.00610013581080529\\
395.01	0.00609525016374377\\
396.01	0.00609018554726357\\
397.01	0.00608493178864853\\
398.01	0.00607947802886572\\
399.01	0.00607381270586999\\
400.01	0.00606792355121726\\
401.01	0.00606179760589656\\
402.01	0.00605542126325994\\
403.01	0.00604878034952735\\
404.01	0.00604186025572631\\
405.01	0.00603464613937136\\
406.01	0.00602712321997983\\
407.01	0.00601927720010481\\
408.01	0.00601109485345431\\
409.01	0.00600256483459055\\
410.01	0.00599367878155951\\
411.01	0.00598443280481278\\
412.01	0.00597482948450378\\
413.01	0.00596488193307401\\
414.01	0.0059546831236043\\
415.01	0.00594427709668789\\
416.01	0.00593365990536041\\
417.01	0.00592282755153628\\
418.01	0.00591177598822489\\
419.01	0.00590050112219411\\
420.01	0.00588899881712395\\
421.01	0.00587726489722895\\
422.01	0.00586529515155599\\
423.01	0.00585308533949892\\
424.01	0.00584063119717485\\
425.01	0.00582792844452006\\
426.01	0.00581497279344729\\
427.01	0.00580175995723419\\
428.01	0.00578828566130119\\
429.01	0.00577454565556098\\
430.01	0.00576053572854219\\
431.01	0.0057462517235175\\
432.01	0.0057316895568933\\
433.01	0.00571684523915225\\
434.01	0.00570171489867311\\
435.01	0.00568629480879477\\
436.01	0.00567058141853382\\
437.01	0.00565457138741205\\
438.01	0.00563826162490599\\
439.01	0.00562164933508589\\
440.01	0.00560473206707327\\
441.01	0.00558750777201248\\
442.01	0.00556997486731927\\
443.01	0.00555213230904023\\
444.01	0.00553397967322594\\
445.01	0.00551551724728434\\
446.01	0.00549674613234149\\
447.01	0.00547766835767428\\
448.01	0.00545828700829854\\
449.01	0.00543860636677846\\
450.01	0.00541863207024607\\
451.01	0.00539837128347489\\
452.01	0.00537783288860757\\
453.01	0.00535702769195026\\
454.01	0.00533596864670675\\
455.01	0.00531467108993984\\
456.01	0.0052931529927509\\
457.01	0.0052714352185777\\
458.01	0.00524954178253281\\
459.01	0.00522750010109665\\
460.01	0.0052053412154063\\
461.01	0.00518309996732541\\
462.01	0.00516081511544892\\
463.01	0.00513852934235167\\
464.01	0.00511628908477032\\
465.01	0.00509414411971495\\
466.01	0.00507214680710524\\
467.01	0.00505035085397365\\
468.01	0.0050288094198075\\
469.01	0.00500757232323613\\
470.01	0.00498668204924411\\
471.01	0.00496616831028183\\
472.01	0.00494604015159451\\
473.01	0.00492627232442627\\
474.01	0.00490655094030001\\
475.01	0.00488669277603261\\
476.01	0.0048667116707059\\
477.01	0.00484662252716021\\
478.01	0.00482644130395952\\
479.01	0.00480618498903938\\
480.01	0.00478587155017494\\
481.01	0.00476551985709734\\
482.01	0.00474514956932396\\
483.01	0.00472478098298111\\
484.01	0.00470443482912806\\
485.01	0.00468413201539438\\
486.01	0.00466389330222162\\
487.01	0.00464373890478968\\
488.01	0.00462368801200742\\
489.01	0.0046037582152793\\
490.01	0.00458396484218238\\
491.01	0.00456432019387871\\
492.01	0.0045448326919775\\
493.01	0.00452550595109121\\
494.01	0.00450633780899151\\
495.01	0.00448731936920279\\
496.01	0.00446843414388941\\
497.01	0.00444965743157864\\
498.01	0.00443095612914953\\
499.01	0.00441228926890717\\
500.01	0.0043936096982971\\
501.01	0.00437486750573409\\
502.01	0.00435602565748428\\
503.01	0.0043370789330409\\
504.01	0.00431802629035681\\
505.01	0.00429886530309774\\
506.01	0.00427959199346607\\
507.01	0.00426020067367799\\
508.01	0.0042406838042687\\
509.01	0.00422103187942214\\
510.01	0.00420123335174307\\
511.01	0.00418127461102447\\
512.01	0.00416114003344601\\
513.01	0.00414081211889399\\
514.01	0.00412027173411878\\
515.01	0.00409949847735075\\
516.01	0.00407847117443859\\
517.01	0.00405716850560663\\
518.01	0.00403556974269061\\
519.01	0.00401365554521901\\
520.01	0.00399140871398024\\
521.01	0.00396881471001509\\
522.01	0.00394586099374016\\
523.01	0.00392253494191472\\
524.01	0.0038988233106298\\
525.01	0.00387471229518079\\
526.01	0.00385018761236385\\
527.01	0.00382523459925919\\
528.01	0.00379983832618786\\
529.01	0.00377398371974654\\
530.01	0.00374765568967166\\
531.01	0.00372083925084878\\
532.01	0.00369351962926699\\
533.01	0.00366568233851864\\
534.01	0.00363731321224822\\
535.01	0.00360839837891926\\
536.01	0.00357892417022308\\
537.01	0.00354887696689357\\
538.01	0.00351824303743295\\
539.01	0.00348700849082071\\
540.01	0.00345515929959017\\
541.01	0.00342268132576639\\
542.01	0.0033895603433927\\
543.01	0.00335578205619657\\
544.01	0.00332133210906197\\
545.01	0.00328619609232783\\
546.01	0.00325035953859222\\
547.01	0.00321380791264442\\
548.01	0.00317652659640092\\
549.01	0.00313850087220274\\
550.01	0.00309971590924588\\
551.01	0.00306015675865827\\
552.01	0.00301980836014317\\
553.01	0.00297865555498463\\
554.01	0.00293668310007627\\
555.01	0.00289387568239544\\
556.01	0.00285021793428497\\
557.01	0.00280569445007147\\
558.01	0.00276028980470851\\
559.01	0.00271398857526332\\
560.01	0.0026667753661384\\
561.01	0.0026186348388955\\
562.01	0.0025695517473923\\
563.01	0.00251951097863612\\
564.01	0.00246849759935086\\
565.01	0.00241649690806831\\
566.01	0.0023634944929314\\
567.01	0.0023094762957353\\
568.01	0.00225442868280603\\
569.01	0.00219833852332488\\
570.01	0.00214119327569023\\
571.01	0.00208298108246332\\
572.01	0.00202369087436417\\
573.01	0.00196331248367459\\
574.01	0.00190183676726109\\
575.01	0.0018392557392658\\
576.01	0.0017755627133218\\
577.01	0.00171075245390312\\
578.01	0.00164482133605403\\
579.01	0.00157776751222645\\
580.01	0.00150959108427172\\
581.01	0.00144029427774478\\
582.01	0.00136988161453246\\
583.01	0.001298360078346\\
584.01	0.00122573926573864\\
585.01	0.00115203151291554\\
586.01	0.0010772519855635\\
587.01	0.00100141871507007\\
588.01	0.000924552559616688\\
589.01	0.000846677062455354\\
590.01	0.000767818171894426\\
591.01	0.000688003777723883\\
592.01	0.000607263006503161\\
593.01	0.000525625202691551\\
594.01	0.000443118503243908\\
595.01	0.000359767889058513\\
596.01	0.000275592566341293\\
597.01	0.000190602493045661\\
598.01	0.000104793818189718\\
599.01	3.18230442566507e-05\\
599.02	3.12770957178118e-05\\
599.03	3.07343620570243e-05\\
599.04	3.01948752177914e-05\\
599.05	2.96586674567757e-05\\
599.06	2.9125771346868e-05\\
599.07	2.8596219780316e-05\\
599.08	2.80700459718581e-05\\
599.09	2.75472834618973e-05\\
599.1	2.70279661197196e-05\\
599.11	2.65121281467183e-05\\
599.12	2.59998040796645e-05\\
599.13	2.54910287940252e-05\\
599.14	2.49858375072735e-05\\
599.15	2.44842657822781e-05\\
599.16	2.39863495306858e-05\\
599.17	2.34921250163653e-05\\
599.18	2.30016288588781e-05\\
599.19	2.2514898036969e-05\\
599.2	2.20319698921133e-05\\
599.21	2.15528821320803e-05\\
599.22	2.10776728345428e-05\\
599.23	2.06063804507207e-05\\
599.24	2.01390438090542e-05\\
599.25	1.96757021189257e-05\\
599.26	1.92163949743994e-05\\
599.27	1.87611623580202e-05\\
599.28	1.83100446446254e-05\\
599.29	1.78630826052212e-05\\
599.3	1.7420317410876e-05\\
599.31	1.69817906366561e-05\\
599.32	1.65475442656122e-05\\
599.33	1.61176206927866e-05\\
599.34	1.56920627292778e-05\\
599.35	1.52709136063342e-05\\
599.36	1.48542169794846e-05\\
599.37	1.44420169327295e-05\\
599.38	1.40343579827472e-05\\
599.39	1.36312859119678e-05\\
599.4	1.32328499381947e-05\\
599.41	1.28390997671674e-05\\
599.42	1.24500855973562e-05\\
599.43	1.20658581248146e-05\\
599.44	1.16864685480583e-05\\
599.45	1.13119685730117e-05\\
599.46	1.09424104179964e-05\\
599.47	1.05778468187656e-05\\
599.48	1.02183310335905e-05\\
599.49	9.86391684839466e-06\\
599.5	9.51465858194285e-06\\
599.51	9.17061109107463e-06\\
599.52	8.83182977599699e-06\\
599.53	8.49837058562383e-06\\
599.54	8.1702900229675e-06\\
599.55	7.84764515058753e-06\\
599.56	7.53049359608279e-06\\
599.57	7.21889355765649e-06\\
599.58	6.91290380971064e-06\\
599.59	6.61258370851861e-06\\
599.6	6.31799319793756e-06\\
599.61	6.02919281519031e-06\\
599.62	5.74624369668701e-06\\
599.63	5.46920758391981e-06\\
599.64	5.19814682940593e-06\\
599.65	4.93312440269685e-06\\
599.66	4.67420389643411e-06\\
599.67	4.42144953247646e-06\\
599.68	4.17492616808582e-06\\
599.69	3.9346993021671e-06\\
599.7	3.70083508156871e-06\\
599.71	3.47340030746289e-06\\
599.72	3.25246244175549e-06\\
599.73	3.03808961359987e-06\\
599.74	2.83035062593855e-06\\
599.75	2.62931496212288e-06\\
599.76	2.43505279260738e-06\\
599.77	2.24763498169606e-06\\
599.78	2.06713309435641e-06\\
599.79	1.89361940310974e-06\\
599.8	1.72716689498045e-06\\
599.81	1.56784927850782e-06\\
599.82	1.41574099085315e-06\\
599.83	1.27091720493813e-06\\
599.84	1.13345383668736e-06\\
599.85	1.00342755231415e-06\\
599.86	8.8091577571392e-07\\
599.87	7.65996695881871e-07\\
599.88	6.58749274441706e-07\\
599.89	5.59253253241965e-07\\
599.9	4.67589162013102e-07\\
599.91	3.83838326099145e-07\\
599.92	3.0808287429171e-07\\
599.93	2.4040574671258e-07\\
599.94	1.80890702777825e-07\\
599.95	1.29622329257326e-07\\
599.96	8.66860484019516e-08\\
599.97	5.21681261331924e-08\\
599.98	2.61556803542173e-08\\
599.99	8.73668930083393e-09\\
600	0\\
};
\addplot [color=red!75!mycolor17,solid,forget plot]
  table[row sep=crcr]{%
0.01	0.00588381158434445\\
1.01	0.00588380981397887\\
2.01	0.00588380800610954\\
3.01	0.00588380615994001\\
4.01	0.0058838042746568\\
5.01	0.00588380234942946\\
6.01	0.0058838003834096\\
7.01	0.00588379837573055\\
8.01	0.00588379632550778\\
9.01	0.00588379423183732\\
10.01	0.00588379209379631\\
11.01	0.00588378991044209\\
12.01	0.00588378768081189\\
13.01	0.00588378540392255\\
14.01	0.00588378307877\\
15.01	0.00588378070432864\\
16.01	0.00588377827955094\\
17.01	0.00588377580336728\\
18.01	0.00588377327468511\\
19.01	0.0058837706923886\\
20.01	0.00588376805533812\\
21.01	0.00588376536236967\\
22.01	0.00588376261229484\\
23.01	0.00588375980389944\\
24.01	0.00588375693594367\\
25.01	0.00588375400716098\\
26.01	0.00588375101625819\\
27.01	0.0058837479619143\\
28.01	0.00588374484278038\\
29.01	0.00588374165747836\\
30.01	0.00588373840460112\\
31.01	0.00588373508271131\\
32.01	0.00588373169034102\\
33.01	0.00588372822599129\\
34.01	0.0058837246881306\\
35.01	0.00588372107519516\\
36.01	0.00588371738558779\\
37.01	0.00588371361767722\\
38.01	0.00588370976979719\\
39.01	0.00588370584024625\\
40.01	0.00588370182728623\\
41.01	0.00588369772914211\\
42.01	0.00588369354400089\\
43.01	0.00588368927001094\\
44.01	0.00588368490528103\\
45.01	0.00588368044787969\\
46.01	0.00588367589583411\\
47.01	0.00588367124712947\\
48.01	0.00588366649970772\\
49.01	0.00588366165146701\\
50.01	0.00588365670026041\\
51.01	0.00588365164389562\\
52.01	0.00588364648013299\\
53.01	0.00588364120668539\\
54.01	0.00588363582121674\\
55.01	0.00588363032134091\\
56.01	0.00588362470462086\\
57.01	0.00588361896856785\\
58.01	0.00588361311063934\\
59.01	0.00588360712823887\\
60.01	0.00588360101871454\\
61.01	0.00588359477935745\\
62.01	0.00588358840740108\\
63.01	0.00588358190001976\\
64.01	0.00588357525432742\\
65.01	0.00588356846737608\\
66.01	0.00588356153615494\\
67.01	0.00588355445758881\\
68.01	0.00588354722853688\\
69.01	0.00588353984579109\\
70.01	0.00588353230607453\\
71.01	0.0058835246060407\\
72.01	0.00588351674227109\\
73.01	0.00588350871127429\\
74.01	0.0058835005094842\\
75.01	0.00588349213325842\\
76.01	0.00588348357887669\\
77.01	0.00588347484253906\\
78.01	0.00588346592036446\\
79.01	0.00588345680838855\\
80.01	0.00588344750256232\\
81.01	0.00588343799874999\\
82.01	0.00588342829272758\\
83.01	0.00588341838018034\\
84.01	0.00588340825670129\\
85.01	0.00588339791778932\\
86.01	0.00588338735884676\\
87.01	0.00588337657517755\\
88.01	0.00588336556198514\\
89.01	0.00588335431437017\\
90.01	0.00588334282732862\\
91.01	0.00588333109574909\\
92.01	0.00588331911441085\\
93.01	0.00588330687798141\\
94.01	0.00588329438101373\\
95.01	0.00588328161794454\\
96.01	0.00588326858309107\\
97.01	0.00588325527064932\\
98.01	0.00588324167469059\\
99.01	0.00588322778915878\\
100.01	0.00588321360786858\\
101.01	0.00588319912450226\\
102.01	0.00588318433260617\\
103.01	0.00588316922558886\\
104.01	0.00588315379671724\\
105.01	0.00588313803911415\\
106.01	0.00588312194575496\\
107.01	0.00588310550946438\\
108.01	0.00588308872291343\\
109.01	0.00588307157861613\\
110.01	0.00588305406892553\\
111.01	0.00588303618603111\\
112.01	0.00588301792195478\\
113.01	0.00588299926854728\\
114.01	0.00588298021748495\\
115.01	0.00588296076026503\\
116.01	0.005882940888203\\
117.01	0.00588292059242756\\
118.01	0.00588289986387734\\
119.01	0.00588287869329652\\
120.01	0.00588285707123074\\
121.01	0.00588283498802288\\
122.01	0.00588281243380844\\
123.01	0.00588278939851127\\
124.01	0.00588276587183914\\
125.01	0.00588274184327876\\
126.01	0.0058827173020915\\
127.01	0.00588269223730786\\
128.01	0.00588266663772314\\
129.01	0.00588264049189196\\
130.01	0.00588261378812332\\
131.01	0.0058825865144752\\
132.01	0.00588255865874927\\
133.01	0.00588253020848514\\
134.01	0.00588250115095503\\
135.01	0.00588247147315758\\
136.01	0.00588244116181235\\
137.01	0.00588241020335382\\
138.01	0.00588237858392497\\
139.01	0.00588234628937106\\
140.01	0.00588231330523328\\
141.01	0.00588227961674249\\
142.01	0.00588224520881197\\
143.01	0.00588221006603104\\
144.01	0.00588217417265786\\
145.01	0.00588213751261204\\
146.01	0.00588210006946808\\
147.01	0.0058820618264473\\
148.01	0.00588202276641042\\
149.01	0.00588198287184971\\
150.01	0.00588194212488134\\
151.01	0.00588190050723683\\
152.01	0.0058818580002552\\
153.01	0.00588181458487425\\
154.01	0.00588177024162174\\
155.01	0.00588172495060694\\
156.01	0.00588167869151168\\
157.01	0.00588163144358043\\
158.01	0.00588158318561174\\
159.01	0.00588153389594783\\
160.01	0.00588148355246559\\
161.01	0.00588143213256567\\
162.01	0.00588137961316285\\
163.01	0.00588132597067528\\
164.01	0.00588127118101404\\
165.01	0.0058812152195715\\
166.01	0.0058811580612111\\
167.01	0.00588109968025551\\
168.01	0.00588104005047472\\
169.01	0.00588097914507433\\
170.01	0.00588091693668332\\
171.01	0.0058808533973417\\
172.01	0.00588078849848801\\
173.01	0.00588072221094589\\
174.01	0.00588065450491102\\
175.01	0.0058805853499375\\
176.01	0.00588051471492441\\
177.01	0.00588044256810102\\
178.01	0.00588036887701293\\
179.01	0.00588029360850673\\
180.01	0.00588021672871559\\
181.01	0.00588013820304342\\
182.01	0.00588005799614911\\
183.01	0.0058799760719306\\
184.01	0.00587989239350861\\
185.01	0.00587980692320948\\
186.01	0.00587971962254822\\
187.01	0.00587963045221116\\
188.01	0.00587953937203801\\
189.01	0.00587944634100323\\
190.01	0.00587935131719767\\
191.01	0.00587925425780959\\
192.01	0.0058791551191048\\
193.01	0.00587905385640724\\
194.01	0.00587895042407826\\
195.01	0.00587884477549585\\
196.01	0.00587873686303343\\
197.01	0.00587862663803878\\
198.01	0.00587851405081081\\
199.01	0.00587839905057767\\
200.01	0.00587828158547348\\
201.01	0.00587816160251482\\
202.01	0.0058780390475759\\
203.01	0.00587791386536484\\
204.01	0.00587778599939807\\
205.01	0.0058776553919744\\
206.01	0.00587752198414932\\
207.01	0.00587738571570717\\
208.01	0.0058772465251347\\
209.01	0.00587710434959241\\
210.01	0.00587695912488636\\
211.01	0.0058768107854383\\
212.01	0.00587665926425642\\
213.01	0.00587650449290441\\
214.01	0.00587634640147051\\
215.01	0.00587618491853521\\
216.01	0.00587601997113923\\
217.01	0.00587585148474994\\
218.01	0.00587567938322747\\
219.01	0.00587550358878984\\
220.01	0.00587532402197739\\
221.01	0.0058751406016171\\
222.01	0.00587495324478494\\
223.01	0.00587476186676826\\
224.01	0.00587456638102727\\
225.01	0.00587436669915539\\
226.01	0.00587416273083883\\
227.01	0.0058739543838156\\
228.01	0.00587374156383323\\
229.01	0.00587352417460554\\
230.01	0.00587330211776942\\
231.01	0.00587307529283898\\
232.01	0.00587284359716003\\
233.01	0.00587260692586363\\
234.01	0.00587236517181778\\
235.01	0.00587211822557813\\
236.01	0.00587186597533873\\
237.01	0.00587160830688041\\
238.01	0.00587134510351866\\
239.01	0.00587107624605066\\
240.01	0.00587080161270023\\
241.01	0.00587052107906241\\
242.01	0.00587023451804657\\
243.01	0.00586994179981858\\
244.01	0.00586964279174073\\
245.01	0.00586933735831147\\
246.01	0.00586902536110367\\
247.01	0.00586870665870073\\
248.01	0.00586838110663213\\
249.01	0.00586804855730707\\
250.01	0.00586770885994708\\
251.01	0.00586736186051688\\
252.01	0.00586700740165385\\
253.01	0.00586664532259585\\
254.01	0.00586627545910731\\
255.01	0.00586589764340432\\
256.01	0.00586551170407741\\
257.01	0.00586511746601293\\
258.01	0.00586471475031261\\
259.01	0.00586430337421192\\
260.01	0.00586388315099517\\
261.01	0.00586345388991064\\
262.01	0.00586301539608245\\
263.01	0.00586256747042118\\
264.01	0.00586210990953192\\
265.01	0.00586164250562143\\
266.01	0.00586116504640177\\
267.01	0.00586067731499283\\
268.01	0.00586017908982239\\
269.01	0.00585967014452416\\
270.01	0.00585915024783305\\
271.01	0.00585861916347889\\
272.01	0.00585807665007678\\
273.01	0.0058575224610167\\
274.01	0.00585695634434821\\
275.01	0.00585637804266499\\
276.01	0.00585578729298533\\
277.01	0.00585518382663051\\
278.01	0.00585456736910046\\
279.01	0.0058539376399463\\
280.01	0.00585329435264057\\
281.01	0.00585263721444388\\
282.01	0.00585196592626974\\
283.01	0.00585128018254474\\
284.01	0.00585057967106704\\
285.01	0.00584986407286065\\
286.01	0.00584913306202712\\
287.01	0.00584838630559376\\
288.01	0.00584762346335831\\
289.01	0.0058468441877298\\
290.01	0.00584604812356651\\
291.01	0.0058452349080099\\
292.01	0.00584440417031455\\
293.01	0.00584355553167466\\
294.01	0.00584268860504691\\
295.01	0.00584180299496801\\
296.01	0.00584089829736983\\
297.01	0.00583997409938889\\
298.01	0.00583902997917247\\
299.01	0.00583806550567998\\
300.01	0.00583708023847964\\
301.01	0.00583607372754137\\
302.01	0.0058350455130235\\
303.01	0.00583399512505619\\
304.01	0.00583292208351904\\
305.01	0.00583182589781418\\
306.01	0.00583070606663385\\
307.01	0.00582956207772283\\
308.01	0.00582839340763591\\
309.01	0.00582719952148923\\
310.01	0.00582597987270716\\
311.01	0.00582473390276236\\
312.01	0.00582346104091066\\
313.01	0.00582216070392023\\
314.01	0.00582083229579483\\
315.01	0.00581947520749061\\
316.01	0.00581808881662733\\
317.01	0.00581667248719307\\
318.01	0.00581522556924281\\
319.01	0.00581374739859083\\
320.01	0.00581223729649667\\
321.01	0.0058106945693446\\
322.01	0.00580911850831705\\
323.01	0.00580750838906135\\
324.01	0.00580586347135024\\
325.01	0.00580418299873598\\
326.01	0.00580246619819846\\
327.01	0.00580071227978684\\
328.01	0.00579892043625547\\
329.01	0.00579708984269393\\
330.01	0.00579521965615169\\
331.01	0.0057933090152569\\
332.01	0.00579135703983091\\
333.01	0.00578936283049848\\
334.01	0.00578732546829255\\
335.01	0.00578524401425749\\
336.01	0.00578311750904827\\
337.01	0.00578094497252816\\
338.01	0.00577872540336539\\
339.01	0.00577645777862977\\
340.01	0.00577414105339103\\
341.01	0.00577177416031952\\
342.01	0.00576935600929145\\
343.01	0.0057668854870002\\
344.01	0.00576436145657629\\
345.01	0.0057617827572184\\
346.01	0.00575914820383806\\
347.01	0.00575645658672056\\
348.01	0.00575370667120748\\
349.01	0.00575089719740307\\
350.01	0.0057480268799102\\
351.01	0.00574509440760137\\
352.01	0.00574209844342953\\
353.01	0.00573903762428736\\
354.01	0.00573591056092131\\
355.01	0.00573271583790994\\
356.01	0.0057294520137157\\
357.01	0.00572611762082254\\
358.01	0.00572271116597093\\
359.01	0.00571923113050552\\
360.01	0.00571567597085059\\
361.01	0.00571204411913296\\
362.01	0.00570833398397189\\
363.01	0.00570454395145984\\
364.01	0.00570067238636121\\
365.01	0.00569671763355712\\
366.01	0.0056926780197718\\
367.01	0.00568855185561727\\
368.01	0.00568433743799914\\
369.01	0.00568003305293321\\
370.01	0.00567563697882439\\
371.01	0.00567114749027247\\
372.01	0.00566656286246969\\
373.01	0.0056618813762707\\
374.01	0.0056571013240185\\
375.01	0.00565222101622497\\
376.01	0.0056472387892136\\
377.01	0.00564215301384461\\
378.01	0.00563696210545766\\
379.01	0.00563166453517971\\
380.01	0.00562625884276317\\
381.01	0.00562074365113432\\
382.01	0.0056151176828499\\
383.01	0.00560937977867574\\
384.01	0.00560352891851695\\
385.01	0.00559756424494493\\
386.01	0.00559148508957717\\
387.01	0.00558529100257136\\
388.01	0.00557898178549608\\
389.01	0.00557255752782516\\
390.01	0.00556601864727755\\
391.01	0.00555936593417019\\
392.01	0.00555260059987158\\
393.01	0.00554572432931722\\
394.01	0.0055387393373643\\
395.01	0.00553164842850102\\
396.01	0.00552445505905949\\
397.01	0.00551716340056851\\
398.01	0.00550977840219078\\
399.01	0.00550230584923837\\
400.01	0.00549475241348927\\
401.01	0.00548712568931336\\
402.01	0.0054794342073349\\
403.01	0.00547168741431669\\
404.01	0.00546389560392106\\
405.01	0.00545606977766861\\
406.01	0.00544822140836945\\
407.01	0.00544036206901984\\
408.01	0.00543250287793826\\
409.01	0.00542465369485019\\
410.01	0.00541682198154169\\
411.01	0.0054090112130263\\
412.01	0.00540121868888936\\
413.01	0.0053934311467888\\
414.01	0.00538555388893473\\
415.01	0.00537753794871848\\
416.01	0.00536938187750997\\
417.01	0.00536108428583244\\
418.01	0.00535264385052405\\
419.01	0.00534405932245016\\
420.01	0.00533532953480055\\
421.01	0.005326453412012\\
422.01	0.0053174299793551\\
423.01	0.00530825837321148\\
424.01	0.00529893785206253\\
425.01	0.00528946780823533\\
426.01	0.00527984778044697\\
427.01	0.0052700774671813\\
428.01	0.00526015674092809\\
429.01	0.00525008566331355\\
430.01	0.00523986450114147\\
431.01	0.00522949374336115\\
432.01	0.00521897411896488\\
433.01	0.00520830661580828\\
434.01	0.00519749250032951\\
435.01	0.00518653333812438\\
436.01	0.00517543101530906\\
437.01	0.00516418776057607\\
438.01	0.00515280616780604\\
439.01	0.00514128921905997\\
440.01	0.00512964030771935\\
441.01	0.00511786326147795\\
442.01	0.00510596236481325\\
443.01	0.005093942380472\\
444.01	0.00508180856939705\\
445.01	0.00506956670839642\\
446.01	0.00505722310470398\\
447.01	0.00504478460641066\\
448.01	0.00503225860754379\\
449.01	0.00501965304634434\\
450.01	0.00500697639503538\\
451.01	0.00499423763908425\\
452.01	0.00498144624364622\\
453.01	0.00496861210453071\\
454.01	0.00495574548067791\\
455.01	0.00494285690482716\\
456.01	0.00492995706867163\\
457.01	0.00491705667849878\\
458.01	0.00490416627718298\\
459.01	0.00489129602842898\\
460.01	0.00487845545954164\\
461.01	0.00486565315984728\\
462.01	0.00485289643286296\\
463.01	0.00484019090226389\\
464.01	0.00482754007636114\\
465.01	0.00481494488149143\\
466.01	0.00480240318289198\\
467.01	0.00478990932375952\\
468.01	0.00477745373048349\\
469.01	0.0047650226563763\\
470.01	0.00475259816988718\\
471.01	0.00474015853571868\\
472.01	0.00472767920367289\\
473.01	0.00471513474777203\\
474.01	0.00470250725175043\\
475.01	0.00468979689859931\\
476.01	0.00467700686081363\\
477.01	0.00466414014221827\\
478.01	0.00465119951291316\\
479.01	0.00463818743693199\\
480.01	0.00462510599264845\\
481.01	0.00461195678621757\\
482.01	0.00459874085866007\\
483.01	0.00458545858761119\\
484.01	0.00457210958527451\\
485.01	0.00455869259475305\\
486.01	0.00454520538770111\\
487.01	0.00453164466714042\\
488.01	0.00451800598032893\\
489.01	0.00450428364772914\\
490.01	0.00449047071535862\\
491.01	0.00447655893905665\\
492.01	0.00446253881033878\\
493.01	0.00444839963432259\\
494.01	0.00443412967037021\\
495.01	0.00441971634512523\\
496.01	0.00440514654481864\\
497.01	0.00439040698802298\\
498.01	0.00437548466997626\\
499.01	0.00436036735305318\\
500.01	0.00434504405182897\\
501.01	0.00432950542122603\\
502.01	0.00431374376253249\\
503.01	0.00429775181588751\\
504.01	0.00428152185964032\\
505.01	0.00426504566107352\\
506.01	0.00424831449026248\\
507.01	0.00423131914387567\\
508.01	0.00421404997969988\\
509.01	0.00419649696233378\\
510.01	0.00417864971998277\\
511.01	0.00416049761158845\\
512.01	0.00414202980261649\\
513.01	0.00412323534667832\\
514.01	0.00410410326881245\\
515.01	0.00408462264473276\\
516.01	0.00406478266880003\\
517.01	0.00404457270213406\\
518.01	0.00402398229155767\\
519.01	0.00400300115064997\\
520.01	0.00398161909713912\\
521.01	0.00395982594793969\\
522.01	0.00393761139778053\\
523.01	0.0039149649645009\\
524.01	0.00389187599995229\\
525.01	0.00386833370872558\\
526.01	0.00384432716607575\\
527.01	0.00381984533406512\\
528.01	0.00379487707489989\\
529.01	0.00376941116044047\\
530.01	0.00374343627697373\\
531.01	0.00371694102457945\\
532.01	0.00368991391083733\\
533.01	0.00366234333922554\\
534.01	0.00363421759335883\\
535.01	0.00360552481914565\\
536.01	0.00357625300787387\\
537.01	0.00354638998385786\\
538.01	0.0035159233995706\\
539.01	0.00348484073635985\\
540.01	0.00345312930589523\\
541.01	0.00342077625093994\\
542.01	0.0033877685454502\\
543.01	0.00335409299413567\\
544.01	0.00331973623170642\\
545.01	0.00328468472213207\\
546.01	0.00324892475832306\\
547.01	0.0032124424627048\\
548.01	0.00317522378916534\\
549.01	0.00313725452679205\\
550.01	0.00309852030566009\\
551.01	0.00305900660468921\\
552.01	0.00301869876133101\\
553.01	0.0029775819828656\\
554.01	0.00293564135940069\\
555.01	0.00289286187882892\\
556.01	0.00284922844403485\\
557.01	0.00280472589266526\\
558.01	0.00275933901978964\\
559.01	0.00271305260378651\\
560.01	0.00266585143579169\\
561.01	0.00261772035304431\\
562.01	0.00256864427646527\\
563.01	0.00251860825281521\\
564.01	0.00246759750180826\\
565.01	0.00241559746860962\\
566.01	0.00236259388219452\\
567.01	0.00230857282007239\\
568.01	0.00225352077989341\\
569.01	0.00219742475845609\\
570.01	0.00214027233862706\\
571.01	0.00208205178465884\\
572.01	0.00202275214634875\\
573.01	0.00196236337241117\\
574.01	0.00190087643333459\\
575.01	0.00183828345384417\\
576.01	0.00177457785488343\\
577.01	0.00170975450473966\\
578.01	0.00164380987854659\\
579.01	0.00157674222487883\\
580.01	0.00150855173747032\\
581.01	0.00143924072920082\\
582.01	0.00136881380434555\\
583.01	0.00129727802360513\\
584.01	0.00122464305453996\\
585.01	0.00115092129761855\\
586.01	0.00107612797501737\\
587.01	0.00100028116541664\\
588.01	0.000923401763108929\\
589.01	0.000845513333520894\\
590.01	0.000766641829415248\\
591.01	0.000686815122188048\\
592.01	0.000606062290302432\\
593.01	0.000524412591375313\\
594.01	0.000441894024979966\\
595.01	0.000358531368867966\\
596.01	0.000274343540840818\\
597.01	0.0001893401004031\\
598.01	0.000103516656727439\\
599.01	3.18230442563801e-05\\
599.02	3.1277095717562e-05\\
599.03	3.07343620567901e-05\\
599.04	3.01948752175762e-05\\
599.05	2.96586674565728e-05\\
599.06	2.91257713466823e-05\\
599.07	2.85962197801425e-05\\
599.08	2.8070045971695e-05\\
599.09	2.75472834617499e-05\\
599.1	2.70279661195826e-05\\
599.11	2.65121281465882e-05\\
599.12	2.59998040795482e-05\\
599.13	2.54910287939159e-05\\
599.14	2.49858375071747e-05\\
599.15	2.44842657821844e-05\\
599.16	2.39863495305973e-05\\
599.17	2.34921250162872e-05\\
599.18	2.30016288588052e-05\\
599.19	2.25148980369048e-05\\
599.2	2.20319698920526e-05\\
599.21	2.1552882132023e-05\\
599.22	2.10776728344908e-05\\
599.23	2.06063804506721e-05\\
599.24	2.01390438090126e-05\\
599.25	1.96757021188858e-05\\
599.26	1.92163949743647e-05\\
599.27	1.87611623579872e-05\\
599.28	1.83100446445959e-05\\
599.29	1.78630826051952e-05\\
599.3	1.74203174108534e-05\\
599.31	1.69817906366353e-05\\
599.32	1.65475442655931e-05\\
599.33	1.61176206927693e-05\\
599.34	1.56920627292639e-05\\
599.35	1.52709136063203e-05\\
599.36	1.48542169794742e-05\\
599.37	1.44420169327208e-05\\
599.38	1.40343579827368e-05\\
599.39	1.36312859119591e-05\\
599.4	1.32328499381877e-05\\
599.41	1.28390997671604e-05\\
599.42	1.24500855973528e-05\\
599.43	1.20658581248094e-05\\
599.44	1.16864685480531e-05\\
599.45	1.13119685730065e-05\\
599.46	1.09424104179946e-05\\
599.47	1.05778468187639e-05\\
599.48	1.02183310335888e-05\\
599.49	9.86391684839466e-06\\
599.5	9.51465858194112e-06\\
599.51	9.17061109107116e-06\\
599.52	8.83182977599525e-06\\
599.53	8.4983705856221e-06\\
599.54	8.1702900229675e-06\\
599.55	7.84764515058753e-06\\
599.56	7.53049359608453e-06\\
599.57	7.21889355765649e-06\\
599.58	6.91290380971064e-06\\
599.59	6.61258370851688e-06\\
599.6	6.31799319793756e-06\\
599.61	6.02919281519031e-06\\
599.62	5.74624369668701e-06\\
599.63	5.46920758391981e-06\\
599.64	5.19814682940767e-06\\
599.65	4.93312440269685e-06\\
599.66	4.67420389643411e-06\\
599.67	4.42144953247646e-06\\
599.68	4.17492616808582e-06\\
599.69	3.93469930216536e-06\\
599.7	3.70083508157044e-06\\
599.71	3.47340030746116e-06\\
599.72	3.25246244175549e-06\\
599.73	3.03808961360161e-06\\
599.74	2.83035062593855e-06\\
599.75	2.62931496212288e-06\\
599.76	2.43505279260738e-06\\
599.77	2.24763498169606e-06\\
599.78	2.06713309435641e-06\\
599.79	1.89361940311147e-06\\
599.8	1.72716689497872e-06\\
599.81	1.56784927850956e-06\\
599.82	1.41574099085315e-06\\
599.83	1.27091720493987e-06\\
599.84	1.13345383668563e-06\\
599.85	1.00342755231589e-06\\
599.86	8.8091577571392e-07\\
599.87	7.65996695880136e-07\\
599.88	6.5874927444344e-07\\
599.89	5.59253253243699e-07\\
599.9	4.67589162011367e-07\\
599.91	3.83838326099145e-07\\
599.92	3.0808287429171e-07\\
599.93	2.40405746710845e-07\\
599.94	1.8089070277609e-07\\
599.95	1.2962232925906e-07\\
599.96	8.66860484002169e-08\\
599.97	5.21681261331924e-08\\
599.98	2.61556803542173e-08\\
599.99	8.73668930083393e-09\\
600	0\\
};
\addplot [color=red!80!mycolor19,solid,forget plot]
  table[row sep=crcr]{%
0.01	0.00547444890358919\\
1.01	0.00547444712667968\\
2.01	0.00547444531233606\\
3.01	0.00547444345976878\\
4.01	0.0054744415681716\\
5.01	0.00547443963672114\\
6.01	0.00547443766457662\\
7.01	0.00547443565087979\\
8.01	0.00547443359475392\\
9.01	0.005474431495304\\
10.01	0.00547442935161593\\
11.01	0.00547442716275643\\
12.01	0.00547442492777244\\
13.01	0.005474422645691\\
14.01	0.00547442031551823\\
15.01	0.00547441793623944\\
16.01	0.00547441550681854\\
17.01	0.00547441302619754\\
18.01	0.00547441049329609\\
19.01	0.00547440790701098\\
20.01	0.00547440526621585\\
21.01	0.00547440256976035\\
22.01	0.00547439981646993\\
23.01	0.00547439700514498\\
24.01	0.00547439413456091\\
25.01	0.00547439120346722\\
26.01	0.0054743882105868\\
27.01	0.00547438515461574\\
28.01	0.00547438203422226\\
29.01	0.00547437884804688\\
30.01	0.00547437559470116\\
31.01	0.00547437227276731\\
32.01	0.00547436888079773\\
33.01	0.0054743654173139\\
34.01	0.00547436188080657\\
35.01	0.00547435826973421\\
36.01	0.00547435458252294\\
37.01	0.0054743508175653\\
38.01	0.00547434697322043\\
39.01	0.00547434304781212\\
40.01	0.00547433903962925\\
41.01	0.00547433494692446\\
42.01	0.00547433076791327\\
43.01	0.00547432650077363\\
44.01	0.00547432214364497\\
45.01	0.00547431769462733\\
46.01	0.00547431315178085\\
47.01	0.00547430851312424\\
48.01	0.00547430377663492\\
49.01	0.00547429894024707\\
50.01	0.00547429400185142\\
51.01	0.00547428895929409\\
52.01	0.00547428381037574\\
53.01	0.00547427855285056\\
54.01	0.00547427318442522\\
55.01	0.00547426770275792\\
56.01	0.00547426210545742\\
57.01	0.00547425639008182\\
58.01	0.00547425055413781\\
59.01	0.00547424459507934\\
60.01	0.00547423851030633\\
61.01	0.0054742322971641\\
62.01	0.0054742259529418\\
63.01	0.00547421947487096\\
64.01	0.00547421286012485\\
65.01	0.00547420610581714\\
66.01	0.00547419920900045\\
67.01	0.00547419216666483\\
68.01	0.0054741849757369\\
69.01	0.00547417763307836\\
70.01	0.00547417013548458\\
71.01	0.00547416247968332\\
72.01	0.00547415466233293\\
73.01	0.00547414668002145\\
74.01	0.00547413852926446\\
75.01	0.00547413020650412\\
76.01	0.00547412170810745\\
77.01	0.00547411303036458\\
78.01	0.0054741041694873\\
79.01	0.00547409512160737\\
80.01	0.00547408588277475\\
81.01	0.00547407644895593\\
82.01	0.00547406681603214\\
83.01	0.00547405697979773\\
84.01	0.00547404693595819\\
85.01	0.00547403668012787\\
86.01	0.00547402620782881\\
87.01	0.00547401551448836\\
88.01	0.00547400459543732\\
89.01	0.00547399344590764\\
90.01	0.00547398206103065\\
91.01	0.00547397043583482\\
92.01	0.00547395856524361\\
93.01	0.00547394644407307\\
94.01	0.00547393406702997\\
95.01	0.00547392142870883\\
96.01	0.00547390852359046\\
97.01	0.00547389534603869\\
98.01	0.00547388189029847\\
99.01	0.00547386815049334\\
100.01	0.00547385412062224\\
101.01	0.0054738397945576\\
102.01	0.0054738251660425\\
103.01	0.0054738102286877\\
104.01	0.00547379497596908\\
105.01	0.00547377940122467\\
106.01	0.00547376349765189\\
107.01	0.00547374725830422\\
108.01	0.00547373067608866\\
109.01	0.00547371374376237\\
110.01	0.00547369645392947\\
111.01	0.00547367879903787\\
112.01	0.00547366077137625\\
113.01	0.00547364236307018\\
114.01	0.00547362356607897\\
115.01	0.00547360437219226\\
116.01	0.00547358477302621\\
117.01	0.00547356476002007\\
118.01	0.00547354432443229\\
119.01	0.00547352345733684\\
120.01	0.00547350214961929\\
121.01	0.00547348039197256\\
122.01	0.00547345817489328\\
123.01	0.00547343548867756\\
124.01	0.00547341232341654\\
125.01	0.00547338866899223\\
126.01	0.00547336451507322\\
127.01	0.00547333985110985\\
128.01	0.0054733146663301\\
129.01	0.00547328894973431\\
130.01	0.00547326269009112\\
131.01	0.00547323587593198\\
132.01	0.00547320849554641\\
133.01	0.00547318053697676\\
134.01	0.00547315198801326\\
135.01	0.00547312283618873\\
136.01	0.0054730930687728\\
137.01	0.00547306267276675\\
138.01	0.00547303163489744\\
139.01	0.00547299994161216\\
140.01	0.00547296757907228\\
141.01	0.00547293453314739\\
142.01	0.00547290078940917\\
143.01	0.00547286633312511\\
144.01	0.00547283114925213\\
145.01	0.00547279522243025\\
146.01	0.00547275853697545\\
147.01	0.00547272107687336\\
148.01	0.00547268282577222\\
149.01	0.00547264376697546\\
150.01	0.00547260388343509\\
151.01	0.00547256315774372\\
152.01	0.00547252157212722\\
153.01	0.00547247910843723\\
154.01	0.00547243574814315\\
155.01	0.00547239147232418\\
156.01	0.00547234626166093\\
157.01	0.00547230009642765\\
158.01	0.00547225295648282\\
159.01	0.00547220482126146\\
160.01	0.00547215566976527\\
161.01	0.00547210548055443\\
162.01	0.00547205423173795\\
163.01	0.00547200190096412\\
164.01	0.005471948465411\\
165.01	0.00547189390177681\\
166.01	0.00547183818626952\\
167.01	0.00547178129459688\\
168.01	0.0054717232019556\\
169.01	0.00547166388302144\\
170.01	0.0054716033119376\\
171.01	0.00547154146230374\\
172.01	0.00547147830716507\\
173.01	0.00547141381900007\\
174.01	0.00547134796970932\\
175.01	0.00547128073060318\\
176.01	0.00547121207238956\\
177.01	0.00547114196516117\\
178.01	0.00547107037838286\\
179.01	0.00547099728087881\\
180.01	0.0054709226408186\\
181.01	0.00547084642570411\\
182.01	0.00547076860235531\\
183.01	0.00547068913689645\\
184.01	0.00547060799474112\\
185.01	0.00547052514057796\\
186.01	0.00547044053835526\\
187.01	0.00547035415126584\\
188.01	0.00547026594173106\\
189.01	0.00547017587138508\\
190.01	0.00547008390105858\\
191.01	0.00546998999076199\\
192.01	0.00546989409966854\\
193.01	0.00546979618609676\\
194.01	0.00546969620749292\\
195.01	0.00546959412041315\\
196.01	0.00546948988050503\\
197.01	0.0054693834424882\\
198.01	0.00546927476013626\\
199.01	0.00546916378625628\\
200.01	0.00546905047266936\\
201.01	0.00546893477018997\\
202.01	0.00546881662860582\\
203.01	0.00546869599665605\\
204.01	0.0054685728220099\\
205.01	0.0054684470512447\\
206.01	0.00546831862982342\\
207.01	0.00546818750207195\\
208.01	0.00546805361115537\\
209.01	0.00546791689905432\\
210.01	0.00546777730654086\\
211.01	0.00546763477315352\\
212.01	0.00546748923717205\\
213.01	0.00546734063559167\\
214.01	0.0054671889040968\\
215.01	0.00546703397703456\\
216.01	0.00546687578738701\\
217.01	0.0054667142667433\\
218.01	0.00546654934527175\\
219.01	0.00546638095169039\\
220.01	0.00546620901323807\\
221.01	0.00546603345564368\\
222.01	0.00546585420309565\\
223.01	0.00546567117821107\\
224.01	0.0054654843020036\\
225.01	0.00546529349385084\\
226.01	0.00546509867146126\\
227.01	0.00546489975084056\\
228.01	0.0054646966462573\\
229.01	0.00546448927020777\\
230.01	0.00546427753338016\\
231.01	0.0054640613446185\\
232.01	0.00546384061088514\\
233.01	0.00546361523722327\\
234.01	0.00546338512671819\\
235.01	0.00546315018045823\\
236.01	0.0054629102974949\\
237.01	0.00546266537480206\\
238.01	0.00546241530723413\\
239.01	0.00546215998748435\\
240.01	0.00546189930604151\\
241.01	0.00546163315114636\\
242.01	0.00546136140874677\\
243.01	0.00546108396245219\\
244.01	0.0054608006934879\\
245.01	0.00546051148064738\\
246.01	0.00546021620024458\\
247.01	0.00545991472606548\\
248.01	0.00545960692931768\\
249.01	0.0054592926785807\\
250.01	0.00545897183975395\\
251.01	0.00545864427600442\\
252.01	0.00545830984771375\\
253.01	0.00545796841242399\\
254.01	0.00545761982478237\\
255.01	0.00545726393648532\\
256.01	0.00545690059622154\\
257.01	0.00545652964961366\\
258.01	0.00545615093916022\\
259.01	0.00545576430417472\\
260.01	0.00545536958072562\\
261.01	0.00545496660157401\\
262.01	0.00545455519611091\\
263.01	0.00545413519029307\\
264.01	0.00545370640657917\\
265.01	0.00545326866386236\\
266.01	0.00545282177740459\\
267.01	0.0054523655587686\\
268.01	0.00545189981574863\\
269.01	0.00545142435230043\\
270.01	0.00545093896847087\\
271.01	0.00545044346032559\\
272.01	0.00544993761987627\\
273.01	0.00544942123500623\\
274.01	0.00544889408939605\\
275.01	0.00544835596244741\\
276.01	0.00544780662920622\\
277.01	0.00544724586028455\\
278.01	0.00544667342178206\\
279.01	0.00544608907520625\\
280.01	0.00544549257739157\\
281.01	0.00544488368041826\\
282.01	0.00544426213152959\\
283.01	0.00544362767304888\\
284.01	0.00544298004229537\\
285.01	0.00544231897149949\\
286.01	0.00544164418771732\\
287.01	0.0054409554127445\\
288.01	0.00544025236302916\\
289.01	0.00543953474958493\\
290.01	0.00543880227790276\\
291.01	0.00543805464786266\\
292.01	0.00543729155364504\\
293.01	0.00543651268364177\\
294.01	0.00543571772036664\\
295.01	0.00543490634036652\\
296.01	0.00543407821413152\\
297.01	0.00543323300600598\\
298.01	0.00543237037409957\\
299.01	0.00543148997019855\\
300.01	0.00543059143967702\\
301.01	0.00542967442140992\\
302.01	0.00542873854768631\\
303.01	0.00542778344412324\\
304.01	0.00542680872958117\\
305.01	0.00542581401608091\\
306.01	0.00542479890872185\\
307.01	0.0054237630056023\\
308.01	0.00542270589774171\\
309.01	0.00542162716900558\\
310.01	0.00542052639603238\\
311.01	0.00541940314816471\\
312.01	0.00541825698738258\\
313.01	0.00541708746824154\\
314.01	0.00541589413781383\\
315.01	0.00541467653563516\\
316.01	0.00541343419365607\\
317.01	0.00541216663619915\\
318.01	0.0054108733799222\\
319.01	0.00540955393378804\\
320.01	0.0054082077990423\\
321.01	0.0054068344691983\\
322.01	0.00540543343003169\\
323.01	0.00540400415958425\\
324.01	0.0054025461281779\\
325.01	0.0054010587984407\\
326.01	0.00539954162534438\\
327.01	0.00539799405625615\\
328.01	0.00539641553100435\\
329.01	0.00539480548195996\\
330.01	0.00539316333413541\\
331.01	0.00539148850530168\\
332.01	0.00538978040612542\\
333.01	0.0053880384403271\\
334.01	0.00538626200486351\\
335.01	0.00538445049013432\\
336.01	0.00538260328021608\\
337.01	0.00538071975312591\\
338.01	0.0053787992811161\\
339.01	0.00537684123100312\\
340.01	0.00537484496453274\\
341.01	0.00537280983878454\\
342.01	0.00537073520661919\\
343.01	0.00536862041717076\\
344.01	0.0053664648163883\\
345.01	0.00536426774762926\\
346.01	0.00536202855230991\\
347.01	0.00535974657061605\\
348.01	0.00535742114227824\\
349.01	0.00535505160741688\\
350.01	0.0053526373074612\\
351.01	0.00535017758614789\\
352.01	0.0053476717906051\\
353.01	0.00534511927252624\\
354.01	0.00534251938944185\\
355.01	0.00533987150609316\\
356.01	0.00533717499591671\\
357.01	0.0053344292426446\\
358.01	0.00533163364202873\\
359.01	0.00532878760369663\\
360.01	0.00532589055314491\\
361.01	0.00532294193387944\\
362.01	0.00531994120970843\\
363.01	0.00531688786719797\\
364.01	0.00531378141829418\\
365.01	0.00531062140312217\\
366.01	0.0053074073929661\\
367.01	0.00530413899343624\\
368.01	0.00530081584782815\\
369.01	0.00529743764067618\\
370.01	0.00529400410150313\\
371.01	0.00529051500876392\\
372.01	0.0052869701939804\\
373.01	0.00528336954605725\\
374.01	0.0052797130157684\\
375.01	0.00527600062039373\\
376.01	0.00527223244848314\\
377.01	0.00526840866471069\\
378.01	0.00526452951477912\\
379.01	0.00526059533031518\\
380.01	0.00525660653368822\\
381.01	0.00525256364266209\\
382.01	0.00524846727477379\\
383.01	0.00524431815130667\\
384.01	0.00524011710069707\\
385.01	0.00523586506118336\\
386.01	0.00523156308246611\\
387.01	0.00522721232610426\\
388.01	0.00522281406432562\\
389.01	0.00521836967687051\\
390.01	0.00521388064542686\\
391.01	0.00520934854514482\\
392.01	0.00520477503264535\\
393.01	0.00520016182985849\\
394.01	0.0051955107029452\\
395.01	0.0051908234354836\\
396.01	0.00518610179502812\\
397.01	0.00518134749210581\\
398.01	0.00517656213069732\\
399.01	0.00517174714929188\\
400.01	0.00516690375172526\\
401.01	0.00516203282725931\\
402.01	0.00515713485978329\\
403.01	0.00515220982670024\\
404.01	0.00514725708910238\\
405.01	0.00514227527638556\\
406.01	0.00513726217069819\\
407.01	0.00513221459982413\\
408.01	0.00512712835162468\\
409.01	0.00512199812947355\\
410.01	0.00511681757686981\\
411.01	0.00511157941144495\\
412.01	0.00510627572504719\\
413.01	0.00510089853303709\\
414.01	0.00509544215081351\\
415.01	0.00508990559050473\\
416.01	0.00508428859435006\\
417.01	0.00507859095793022\\
418.01	0.0050728125332189\\
419.01	0.00506695323167737\\
420.01	0.00506101302737739\\
421.01	0.00505499196013406\\
422.01	0.00504889013862831\\
423.01	0.00504270774349266\\
424.01	0.00503644503033259\\
425.01	0.00503010233265198\\
426.01	0.00502368006464182\\
427.01	0.0050171787237906\\
428.01	0.00501059889326493\\
429.01	0.00500394124400285\\
430.01	0.00499720653645617\\
431.01	0.00499039562190678\\
432.01	0.00498350944327549\\
433.01	0.00497654903533095\\
434.01	0.00496951552419494\\
435.01	0.00496241012603001\\
436.01	0.00495523414478367\\
437.01	0.00494798896884916\\
438.01	0.00494067606649338\\
439.01	0.0049332969798862\\
440.01	0.00492585331755582\\
441.01	0.00491834674508316\\
442.01	0.00491077897383633\\
443.01	0.00490315174754159\\
444.01	0.00489546682648113\\
445.01	0.00488772596910899\\
446.01	0.00487993091088409\\
447.01	0.00487208334013407\\
448.01	0.00486418487078903\\
449.01	0.00485623701186795\\
450.01	0.00484824113365184\\
451.01	0.00484019843056258\\
452.01	0.00483210988086774\\
453.01	0.00482397620346859\\
454.01	0.0048157978122043\\
455.01	0.00480757476831897\\
456.01	0.00479930673200534\\
457.01	0.00479099291426222\\
458.01	0.00478263203068424\\
459.01	0.00477422225924417\\
460.01	0.00476576120462847\\
461.01	0.00475724587222647\\
462.01	0.00474867265544567\\
463.01	0.00474003734060925\\
464.01	0.0047313351341798\\
465.01	0.00472256071731719\\
466.01	0.00471370833270468\\
467.01	0.00470477190790311\\
468.01	0.00469574521786123\\
469.01	0.00468662208610703\\
470.01	0.00467739661883009\\
471.01	0.00466806345757149\\
472.01	0.00465861802321931\\
473.01	0.00464905670308286\\
474.01	0.00463937683481917\\
475.01	0.00462957605184824\\
476.01	0.00461965177964405\\
477.01	0.00460960118805203\\
478.01	0.00459942117493768\\
479.01	0.00458910835096083\\
480.01	0.00457865902595065\\
481.01	0.00456806919740823\\
482.01	0.0045573345417218\\
483.01	0.00454645040872085\\
484.01	0.0045354118202245\\
485.01	0.00452421347324612\\
486.01	0.00451284974848832\\
487.01	0.00450131472469421\\
488.01	0.00448960219928953\\
489.01	0.00447770571555365\\
490.01	0.00446561859626462\\
491.01	0.00445333398337029\\
492.01	0.00444084488272215\\
493.01	0.00442814421226662\\
494.01	0.00441522485132207\\
495.01	0.00440207968771001\\
496.01	0.00438870165860932\\
497.01	0.00437508378018656\\
498.01	0.00436121916053661\\
499.01	0.00434710099056351\\
500.01	0.00433272250870228\\
501.01	0.00431807693864201\\
502.01	0.00430315740715636\\
503.01	0.00428795688086955\\
504.01	0.00427246815788195\\
505.01	0.00425668387727377\\
506.01	0.00424059652999392\\
507.01	0.00422419847041546\\
508.01	0.0042074819281634\\
509.01	0.00419043901973585\\
510.01	0.00417306175936144\\
511.01	0.00415534206847596\\
512.01	0.00413727178316525\\
513.01	0.00411884265894367\\
514.01	0.00410004637230826\\
515.01	0.00408087451867664\\
516.01	0.00406131860657965\\
517.01	0.00404137004835897\\
518.01	0.00402102014811284\\
519.01	0.00400026008820573\\
520.01	0.00397908091623087\\
521.01	0.00395747353471852\\
522.01	0.00393542869566511\\
523.01	0.00391293699939012\\
524.01	0.00388998889433544\\
525.01	0.00386657467628649\\
526.01	0.00384268448686651\\
527.01	0.00381830831127323\\
528.01	0.00379343597527271\\
529.01	0.00376805714151781\\
530.01	0.00374216130531905\\
531.01	0.0037157377900536\\
532.01	0.00368877574245122\\
533.01	0.00366126412803118\\
534.01	0.00363319172696691\\
535.01	0.00360454713061691\\
536.01	0.00357531873886363\\
537.01	0.00354549475825077\\
538.01	0.00351506320072506\\
539.01	0.0034840118827168\\
540.01	0.00345232842446228\\
541.01	0.00342000024963096\\
542.01	0.00338701458535218\\
543.01	0.00335335846274843\\
544.01	0.00331901871809061\\
545.01	0.00328398199469261\\
546.01	0.00324823474566422\\
547.01	0.00321176323763672\\
548.01	0.00317455355556667\\
549.01	0.00313659160871879\\
550.01	0.00309786313792477\\
551.01	0.00305835372422298\\
552.01	0.00301804879900678\\
553.01	0.00297693365584456\\
554.01	0.00293499346416344\\
555.01	0.00289221328500843\\
556.01	0.00284857808910646\\
557.01	0.00280407277748279\\
558.01	0.00275868220489704\\
559.01	0.00271239120638774\\
560.01	0.00266518462723621\\
561.01	0.00261704735668648\\
562.01	0.0025679643657854\\
563.01	0.00251792074973548\\
564.01	0.00246690177518275\\
565.01	0.00241489293289015\\
566.01	0.00236187999627078\\
567.01	0.00230784908627568\\
568.01	0.00225278674314507\\
569.01	0.00219668000553777\\
570.01	0.00213951649754955\\
571.01	0.00208128452410928\\
572.01	0.00202197317520257\\
573.01	0.00196157243930301\\
574.01	0.00190007332628658\\
575.01	0.00183746799995258\\
576.01	0.0017737499200606\\
577.01	0.00170891399350014\\
578.01	0.00164295673381588\\
579.01	0.00157587642778816\\
580.01	0.00150767330708272\\
581.01	0.00143834972208804\\
582.01	0.0013679103139034\\
583.01	0.00129636217895157\\
584.01	0.0012237150187842\\
585.01	0.00114998126521834\\
586.01	0.00107517616785102\\
587.01	0.000999317827082614\\
588.01	0.000922427150825461\\
589.01	0.000844527706822454\\
590.01	0.000765645434626115\\
591.01	0.000685808171383964\\
592.01	0.000605044933136151\\
593.01	0.000523384877725838\\
594.01	0.000440855855864791\\
595.01	0.00035748243240945\\
596.01	0.00027328322926679\\
597.01	0.000188267403047507\\
598.01	0.000102430022728519\\
599.01	3.18230442563731e-05\\
599.02	3.12770957175551e-05\\
599.03	3.07343620567866e-05\\
599.04	3.0194875217571e-05\\
599.05	2.96586674565693e-05\\
599.06	2.91257713466754e-05\\
599.07	2.85962197801391e-05\\
599.08	2.80700459716916e-05\\
599.09	2.75472834617447e-05\\
599.1	2.70279661195773e-05\\
599.11	2.65121281465865e-05\\
599.12	2.5999804079543e-05\\
599.13	2.54910287939124e-05\\
599.14	2.49858375071695e-05\\
599.15	2.4484265782181e-05\\
599.16	2.39863495305973e-05\\
599.17	2.34921250162837e-05\\
599.18	2.30016288588035e-05\\
599.19	2.25148980369013e-05\\
599.2	2.20319698920508e-05\\
599.21	2.1552882132023e-05\\
599.22	2.10776728344908e-05\\
599.23	2.06063804506721e-05\\
599.24	2.01390438090109e-05\\
599.25	1.96757021188858e-05\\
599.26	1.92163949743647e-05\\
599.27	1.87611623579855e-05\\
599.28	1.83100446445959e-05\\
599.29	1.78630826051952e-05\\
599.3	1.74203174108517e-05\\
599.31	1.69817906366353e-05\\
599.32	1.65475442655931e-05\\
599.33	1.61176206927693e-05\\
599.34	1.56920627292622e-05\\
599.35	1.52709136063186e-05\\
599.36	1.48542169794725e-05\\
599.37	1.4442016932719e-05\\
599.38	1.40343579827368e-05\\
599.39	1.36312859119591e-05\\
599.4	1.32328499381877e-05\\
599.41	1.28390997671621e-05\\
599.42	1.2450085597351e-05\\
599.43	1.20658581248094e-05\\
599.44	1.16864685480531e-05\\
599.45	1.13119685730082e-05\\
599.46	1.09424104179929e-05\\
599.47	1.05778468187639e-05\\
599.48	1.02183310335905e-05\\
599.49	9.86391684839293e-06\\
599.5	9.51465858193938e-06\\
599.51	9.17061109107289e-06\\
599.52	8.83182977599352e-06\\
599.53	8.4983705856221e-06\\
599.54	8.1702900229675e-06\\
599.55	7.84764515058753e-06\\
599.56	7.53049359608279e-06\\
599.57	7.21889355765649e-06\\
599.58	6.91290380971064e-06\\
599.59	6.61258370851688e-06\\
599.6	6.31799319793756e-06\\
599.61	6.02919281519031e-06\\
599.62	5.74624369668701e-06\\
599.63	5.46920758391981e-06\\
599.64	5.19814682940593e-06\\
599.65	4.93312440269685e-06\\
599.66	4.67420389643237e-06\\
599.67	4.42144953247646e-06\\
599.68	4.17492616808755e-06\\
599.69	3.9346993021671e-06\\
599.7	3.70083508156871e-06\\
599.71	3.47340030746116e-06\\
599.72	3.25246244175723e-06\\
599.73	3.03808961360161e-06\\
599.74	2.83035062593855e-06\\
599.75	2.62931496212288e-06\\
599.76	2.43505279260738e-06\\
599.77	2.24763498169606e-06\\
599.78	2.06713309435641e-06\\
599.79	1.89361940310974e-06\\
599.8	1.72716689498045e-06\\
599.81	1.56784927850782e-06\\
599.82	1.41574099085488e-06\\
599.83	1.27091720493813e-06\\
599.84	1.13345383668563e-06\\
599.85	1.00342755231589e-06\\
599.86	8.8091577571392e-07\\
599.87	7.65996695880136e-07\\
599.88	6.58749274441706e-07\\
599.89	5.59253253243699e-07\\
599.9	4.67589162013102e-07\\
599.91	3.83838326099145e-07\\
599.92	3.0808287429171e-07\\
599.93	2.40405746710845e-07\\
599.94	1.80890702777825e-07\\
599.95	1.2962232925906e-07\\
599.96	8.66860484019516e-08\\
599.97	5.21681261331924e-08\\
599.98	2.61556803542173e-08\\
599.99	8.73668930083393e-09\\
600	0\\
};
\addplot [color=red,solid,forget plot]
  table[row sep=crcr]{%
0.01	0.00524634702484204\\
1.01	0.00524634538167428\\
2.01	0.00524634370419065\\
3.01	0.00524634199167481\\
4.01	0.00524634024339566\\
5.01	0.00524633845860666\\
6.01	0.00524633663654615\\
7.01	0.00524633477643592\\
8.01	0.00524633287748189\\
9.01	0.0052463309388734\\
10.01	0.00524632895978309\\
11.01	0.00524632693936595\\
12.01	0.00524632487675974\\
13.01	0.00524632277108397\\
14.01	0.0052463206214401\\
15.01	0.00524631842691058\\
16.01	0.00524631618655892\\
17.01	0.00524631389942906\\
18.01	0.00524631156454481\\
19.01	0.00524630918091006\\
20.01	0.00524630674750763\\
21.01	0.00524630426329937\\
22.01	0.00524630172722527\\
23.01	0.00524629913820358\\
24.01	0.0052462964951298\\
25.01	0.00524629379687638\\
26.01	0.00524629104229229\\
27.01	0.00524628823020258\\
28.01	0.00524628535940793\\
29.01	0.00524628242868409\\
30.01	0.00524627943678111\\
31.01	0.00524627638242321\\
32.01	0.00524627326430783\\
33.01	0.00524627008110561\\
34.01	0.00524626683145922\\
35.01	0.00524626351398313\\
36.01	0.00524626012726317\\
37.01	0.00524625666985588\\
38.01	0.00524625314028707\\
39.01	0.00524624953705275\\
40.01	0.00524624585861699\\
41.01	0.00524624210341223\\
42.01	0.0052462382698383\\
43.01	0.0052462343562615\\
44.01	0.00524623036101445\\
45.01	0.00524622628239498\\
46.01	0.00524622211866539\\
47.01	0.00524621786805216\\
48.01	0.0052462135287446\\
49.01	0.0052462090988945\\
50.01	0.00524620457661531\\
51.01	0.00524619995998124\\
52.01	0.00524619524702633\\
53.01	0.00524619043574392\\
54.01	0.00524618552408546\\
55.01	0.00524618050996003\\
56.01	0.00524617539123337\\
57.01	0.00524617016572659\\
58.01	0.0052461648312158\\
59.01	0.00524615938543075\\
60.01	0.00524615382605415\\
61.01	0.00524614815072038\\
62.01	0.00524614235701494\\
63.01	0.0052461364424732\\
64.01	0.00524613040457932\\
65.01	0.0052461242407651\\
66.01	0.00524611794840926\\
67.01	0.00524611152483606\\
68.01	0.00524610496731396\\
69.01	0.00524609827305526\\
70.01	0.00524609143921402\\
71.01	0.0052460844628854\\
72.01	0.00524607734110442\\
73.01	0.00524607007084434\\
74.01	0.00524606264901603\\
75.01	0.00524605507246604\\
76.01	0.00524604733797553\\
77.01	0.00524603944225902\\
78.01	0.00524603138196313\\
79.01	0.00524602315366473\\
80.01	0.0052460147538698\\
81.01	0.00524600617901195\\
82.01	0.00524599742545109\\
83.01	0.00524598848947141\\
84.01	0.00524597936728023\\
85.01	0.00524597005500673\\
86.01	0.00524596054869953\\
87.01	0.00524595084432566\\
88.01	0.00524594093776854\\
89.01	0.00524593082482648\\
90.01	0.00524592050121083\\
91.01	0.0052459099625443\\
92.01	0.00524589920435879\\
93.01	0.00524588822209397\\
94.01	0.00524587701109538\\
95.01	0.00524586556661231\\
96.01	0.00524585388379547\\
97.01	0.00524584195769568\\
98.01	0.00524582978326137\\
99.01	0.00524581735533652\\
100.01	0.00524580466865875\\
101.01	0.00524579171785691\\
102.01	0.00524577849744897\\
103.01	0.00524576500183954\\
104.01	0.00524575122531796\\
105.01	0.00524573716205573\\
106.01	0.0052457228061036\\
107.01	0.00524570815139027\\
108.01	0.00524569319171874\\
109.01	0.00524567792076409\\
110.01	0.00524566233207136\\
111.01	0.00524564641905242\\
112.01	0.00524563017498314\\
113.01	0.00524561359300103\\
114.01	0.00524559666610192\\
115.01	0.00524557938713776\\
116.01	0.00524556174881299\\
117.01	0.00524554374368194\\
118.01	0.00524552536414564\\
119.01	0.00524550660244865\\
120.01	0.00524548745067598\\
121.01	0.00524546790074993\\
122.01	0.00524544794442652\\
123.01	0.00524542757329218\\
124.01	0.0052454067787606\\
125.01	0.00524538555206882\\
126.01	0.00524536388427362\\
127.01	0.0052453417662487\\
128.01	0.00524531918867973\\
129.01	0.00524529614206134\\
130.01	0.00524527261669292\\
131.01	0.00524524860267488\\
132.01	0.0052452240899045\\
133.01	0.00524519906807203\\
134.01	0.00524517352665601\\
135.01	0.0052451474549193\\
136.01	0.00524512084190489\\
137.01	0.00524509367643101\\
138.01	0.00524506594708726\\
139.01	0.00524503764222927\\
140.01	0.00524500874997448\\
141.01	0.00524497925819697\\
142.01	0.00524494915452292\\
143.01	0.00524491842632542\\
144.01	0.00524488706071916\\
145.01	0.00524485504455592\\
146.01	0.00524482236441842\\
147.01	0.00524478900661545\\
148.01	0.00524475495717632\\
149.01	0.00524472020184529\\
150.01	0.00524468472607508\\
151.01	0.00524464851502255\\
152.01	0.00524461155354148\\
153.01	0.0052445738261769\\
154.01	0.00524453531715894\\
155.01	0.00524449601039674\\
156.01	0.00524445588947163\\
157.01	0.0052444149376306\\
158.01	0.00524437313778031\\
159.01	0.00524433047247939\\
160.01	0.00524428692393208\\
161.01	0.00524424247398101\\
162.01	0.00524419710409988\\
163.01	0.00524415079538634\\
164.01	0.00524410352855446\\
165.01	0.00524405528392705\\
166.01	0.00524400604142793\\
167.01	0.00524395578057405\\
168.01	0.0052439044804675\\
169.01	0.00524385211978692\\
170.01	0.00524379867677972\\
171.01	0.00524374412925308\\
172.01	0.00524368845456543\\
173.01	0.00524363162961788\\
174.01	0.00524357363084466\\
175.01	0.00524351443420435\\
176.01	0.00524345401517028\\
177.01	0.00524339234872135\\
178.01	0.00524332940933197\\
179.01	0.00524326517096204\\
180.01	0.00524319960704773\\
181.01	0.00524313269049009\\
182.01	0.00524306439364537\\
183.01	0.00524299468831421\\
184.01	0.00524292354573063\\
185.01	0.00524285093655134\\
186.01	0.00524277683084407\\
187.01	0.0052427011980764\\
188.01	0.00524262400710397\\
189.01	0.00524254522615907\\
190.01	0.00524246482283786\\
191.01	0.00524238276408858\\
192.01	0.00524229901619896\\
193.01	0.00524221354478338\\
194.01	0.00524212631476993\\
195.01	0.00524203729038728\\
196.01	0.00524194643515123\\
197.01	0.00524185371185099\\
198.01	0.00524175908253533\\
199.01	0.00524166250849836\\
200.01	0.00524156395026513\\
201.01	0.00524146336757735\\
202.01	0.00524136071937776\\
203.01	0.00524125596379558\\
204.01	0.00524114905813088\\
205.01	0.00524103995883895\\
206.01	0.00524092862151417\\
207.01	0.00524081500087397\\
208.01	0.00524069905074223\\
209.01	0.00524058072403255\\
210.01	0.00524045997273106\\
211.01	0.00524033674787933\\
212.01	0.00524021099955651\\
213.01	0.0052400826768615\\
214.01	0.00523995172789479\\
215.01	0.00523981809973951\\
216.01	0.00523968173844319\\
217.01	0.00523954258899855\\
218.01	0.00523940059532382\\
219.01	0.00523925570024301\\
220.01	0.00523910784546591\\
221.01	0.00523895697156793\\
222.01	0.00523880301796943\\
223.01	0.00523864592291425\\
224.01	0.00523848562344873\\
225.01	0.00523832205540018\\
226.01	0.00523815515335461\\
227.01	0.00523798485063467\\
228.01	0.00523781107927687\\
229.01	0.00523763377000877\\
230.01	0.00523745285222539\\
231.01	0.00523726825396603\\
232.01	0.00523707990188999\\
233.01	0.00523688772125235\\
234.01	0.0052366916358793\\
235.01	0.00523649156814334\\
236.01	0.00523628743893765\\
237.01	0.00523607916765078\\
238.01	0.00523586667214054\\
239.01	0.00523564986870783\\
240.01	0.00523542867206966\\
241.01	0.00523520299533229\\
242.01	0.00523497274996426\\
243.01	0.00523473784576858\\
244.01	0.00523449819085457\\
245.01	0.00523425369160987\\
246.01	0.00523400425267175\\
247.01	0.00523374977689813\\
248.01	0.00523349016533854\\
249.01	0.00523322531720449\\
250.01	0.00523295512983983\\
251.01	0.00523267949869064\\
252.01	0.00523239831727478\\
253.01	0.0052321114771511\\
254.01	0.0052318188678893\\
255.01	0.00523152037703769\\
256.01	0.00523121589009264\\
257.01	0.00523090529046706\\
258.01	0.00523058845945782\\
259.01	0.00523026527621435\\
260.01	0.00522993561770634\\
261.01	0.00522959935869157\\
262.01	0.00522925637168265\\
263.01	0.00522890652691559\\
264.01	0.00522854969231577\\
265.01	0.00522818573346614\\
266.01	0.00522781451357373\\
267.01	0.0052274358934371\\
268.01	0.00522704973141289\\
269.01	0.00522665588338346\\
270.01	0.00522625420272365\\
271.01	0.00522584454026808\\
272.01	0.0052254267442784\\
273.01	0.00522500066041105\\
274.01	0.0052245661316846\\
275.01	0.00522412299844759\\
276.01	0.00522367109834689\\
277.01	0.00522321026629624\\
278.01	0.00522274033444507\\
279.01	0.00522226113214746\\
280.01	0.00522177248593254\\
281.01	0.00522127421947445\\
282.01	0.00522076615356294\\
283.01	0.00522024810607567\\
284.01	0.00521971989194984\\
285.01	0.00521918132315574\\
286.01	0.00521863220867095\\
287.01	0.0052180723544547\\
288.01	0.00521750156342459\\
289.01	0.00521691963543351\\
290.01	0.00521632636724819\\
291.01	0.00521572155252897\\
292.01	0.00521510498181107\\
293.01	0.00521447644248759\\
294.01	0.00521383571879398\\
295.01	0.00521318259179452\\
296.01	0.00521251683937033\\
297.01	0.00521183823621045\\
298.01	0.00521114655380373\\
299.01	0.00521044156043415\\
300.01	0.00520972302117921\\
301.01	0.00520899069790897\\
302.01	0.00520824434929016\\
303.01	0.00520748373079238\\
304.01	0.00520670859469767\\
305.01	0.00520591869011355\\
306.01	0.00520511376299\\
307.01	0.00520429355614022\\
308.01	0.00520345780926567\\
309.01	0.00520260625898526\\
310.01	0.00520173863886974\\
311.01	0.00520085467948036\\
312.01	0.00519995410841374\\
313.01	0.00519903665035063\\
314.01	0.00519810202711284\\
315.01	0.00519714995772436\\
316.01	0.00519618015847998\\
317.01	0.00519519234302001\\
318.01	0.00519418622241354\\
319.01	0.00519316150524767\\
320.01	0.00519211789772595\\
321.01	0.00519105510377478\\
322.01	0.00518997282515848\\
323.01	0.00518887076160365\\
324.01	0.00518774861093347\\
325.01	0.00518660606921215\\
326.01	0.00518544283089948\\
327.01	0.0051842585890171\\
328.01	0.00518305303532577\\
329.01	0.0051818258605159\\
330.01	0.00518057675440898\\
331.01	0.00517930540617347\\
332.01	0.00517801150455399\\
333.01	0.00517669473811432\\
334.01	0.00517535479549591\\
335.01	0.00517399136569088\\
336.01	0.00517260413833114\\
337.01	0.00517119280399358\\
338.01	0.00516975705452242\\
339.01	0.00516829658336771\\
340.01	0.00516681108594328\\
341.01	0.00516530026000083\\
342.01	0.00516376380602336\\
343.01	0.00516220142763736\\
344.01	0.00516061283204347\\
345.01	0.0051589977304664\\
346.01	0.00515735583862371\\
347.01	0.00515568687721378\\
348.01	0.00515399057242243\\
349.01	0.00515226665644796\\
350.01	0.00515051486804406\\
351.01	0.00514873495308003\\
352.01	0.00514692666511627\\
353.01	0.00514508976599623\\
354.01	0.0051432240264493\\
355.01	0.00514132922670752\\
356.01	0.00513940515712955\\
357.01	0.00513745161883129\\
358.01	0.00513546842431997\\
359.01	0.00513345539812616\\
360.01	0.00513141237743133\\
361.01	0.00512933921268395\\
362.01	0.00512723576819906\\
363.01	0.00512510192273381\\
364.01	0.00512293757003164\\
365.01	0.00512074261932499\\
366.01	0.00511851699578688\\
367.01	0.00511626064092056\\
368.01	0.00511397351287178\\
369.01	0.00511165558665115\\
370.01	0.00510930685424922\\
371.01	0.00510692732462547\\
372.01	0.00510451702355187\\
373.01	0.00510207599328724\\
374.01	0.00509960429205898\\
375.01	0.00509710199332412\\
376.01	0.00509456918477933\\
377.01	0.00509200596709101\\
378.01	0.00508941245230719\\
379.01	0.00508678876191842\\
380.01	0.00508413502452625\\
381.01	0.00508145137308095\\
382.01	0.00507873794164737\\
383.01	0.00507599486165573\\
384.01	0.00507322225759954\\
385.01	0.00507042024214\\
386.01	0.00506758891058257\\
387.01	0.00506472833469952\\
388.01	0.00506183855587652\\
389.01	0.00505891957757931\\
390.01	0.00505597135715017\\
391.01	0.00505299379696817\\
392.01	0.00504998673503601\\
393.01	0.00504694993509147\\
394.01	0.00504388307639118\\
395.01	0.00504078574336494\\
396.01	0.00503765741541047\\
397.01	0.00503449745717789\\
398.01	0.00503130510978448\\
399.01	0.00502807948350922\\
400.01	0.00502481955263314\\
401.01	0.00502152415321797\\
402.01	0.00501819198474338\\
403.01	0.00501482161663915\\
404.01	0.0050114115008428\\
405.01	0.00500795999154164\\
406.01	0.00500446537320313\\
407.01	0.00500092589777272\\
408.01	0.00499733983145012\\
409.01	0.00499370551062219\\
410.01	0.00499002140512592\\
411.01	0.00498628618482141\\
412.01	0.00498249878208409\\
413.01	0.00497865843779457\\
414.01	0.00497476469135668\\
415.01	0.00497081721153147\\
416.01	0.00496681568223637\\
417.01	0.00496275979613698\\
418.01	0.00495864925421746\\
419.01	0.00495448376522575\\
420.01	0.00495026304497878\\
421.01	0.00494598681551536\\
422.01	0.00494165480407954\\
423.01	0.00493726674192088\\
424.01	0.00493282236289352\\
425.01	0.00492832140183725\\
426.01	0.00492376359272367\\
427.01	0.00491914866654613\\
428.01	0.0049144763489378\\
429.01	0.00490974635749719\\
430.01	0.0049049583988014\\
431.01	0.00490011216509085\\
432.01	0.00489520733060504\\
433.01	0.00489024354755414\\
434.01	0.00488522044171163\\
435.01	0.00488013760761407\\
436.01	0.00487499460336023\\
437.01	0.00486979094500281\\
438.01	0.00486452610053278\\
439.01	0.00485919948346191\\
440.01	0.00485381044601682\\
441.01	0.00484835827196602\\
442.01	0.00484284216911223\\
443.01	0.00483726126149552\\
444.01	0.00483161458136402\\
445.01	0.00482590106098931\\
446.01	0.00482011952441736\\
447.01	0.0048142686792673\\
448.01	0.00480834710871583\\
449.01	0.00480235326382089\\
450.01	0.00479628545637138\\
451.01	0.00479014185246842\\
452.01	0.00478392046707218\\
453.01	0.00477761915976888\\
454.01	0.00477123563203365\\
455.01	0.00476476742627694\\
456.01	0.00475821192696775\\
457.01	0.00475156636411687\\
458.01	0.00474482781937998\\
459.01	0.00473799323499074\\
460.01	0.00473105942565689\\
461.01	0.00472402309344056\\
462.01	0.00471688084548702\\
463.01	0.00470962921426139\\
464.01	0.0047022646796878\\
465.01	0.00469478369226967\\
466.01	0.00468718269589198\\
467.01	0.00467945814860222\\
468.01	0.00467160653924801\\
469.01	0.00466362439749715\\
470.01	0.00465550829456611\\
471.01	0.00464725483211195\\
472.01	0.00463886061745229\\
473.01	0.00463032222497244\\
474.01	0.00462163614761239\\
475.01	0.00461279875818079\\
476.01	0.00460380629909003\\
477.01	0.00459465488160419\\
478.01	0.00458534048603782\\
479.01	0.00457585896272411\\
480.01	0.00456620603379292\\
481.01	0.00455637729577941\\
482.01	0.00454636822306456\\
483.01	0.00453617417212127\\
484.01	0.00452579038650785\\
485.01	0.00451521200251422\\
486.01	0.00450443405532398\\
487.01	0.0044934514855084\\
488.01	0.0044822591456219\\
489.01	0.00447085180662029\\
490.01	0.00445922416378062\\
491.01	0.00444737084176718\\
492.01	0.00443528639847168\\
493.01	0.00442296532726427\\
494.01	0.0044104020573332\\
495.01	0.00439759095188005\\
496.01	0.00438452630407865\\
497.01	0.00437120233091795\\
498.01	0.00435761316531334\\
499.01	0.00434375284719876\\
500.01	0.00432961531465032\\
501.01	0.00431519439636444\\
502.01	0.00430048380688169\\
503.01	0.00428547714512883\\
504.01	0.00427016789463693\\
505.01	0.00425454942377817\\
506.01	0.00423861498570822\\
507.01	0.00422235771793727\\
508.01	0.00420577064146486\\
509.01	0.00418884665942247\\
510.01	0.00417157855518518\\
511.01	0.00415395898993742\\
512.01	0.00413598049970358\\
513.01	0.00411763549188932\\
514.01	0.00409891624141305\\
515.01	0.00407981488654231\\
516.01	0.00406032342458033\\
517.01	0.00404043370756525\\
518.01	0.00402013743814512\\
519.01	0.00399942616576509\\
520.01	0.00397829128324327\\
521.01	0.00395672402371646\\
522.01	0.00393471545782475\\
523.01	0.00391225649092619\\
524.01	0.00388933786022006\\
525.01	0.00386595013177418\\
526.01	0.0038420836974821\\
527.01	0.003817728771985\\
528.01	0.00379287538959846\\
529.01	0.00376751340128798\\
530.01	0.00374163247173836\\
531.01	0.00371522207656307\\
532.01	0.00368827149969528\\
533.01	0.00366076983099453\\
534.01	0.00363270596409685\\
535.01	0.00360406859452609\\
536.01	0.00357484621807641\\
537.01	0.00354502712947658\\
538.01	0.00351459942135183\\
539.01	0.00348355098351638\\
540.01	0.00345186950264478\\
541.01	0.00341954246237786\\
542.01	0.00338655714392597\\
543.01	0.00335290062723492\\
544.01	0.00331855979278612\\
545.01	0.00328352132410759\\
546.01	0.00324777171107945\\
547.01	0.00321129725412398\\
548.01	0.00317408406938027\\
549.01	0.0031361180949736\\
550.01	0.00309738509850105\\
551.01	0.00305787068587207\\
552.01	0.00301756031165513\\
553.01	0.00297643929110013\\
554.01	0.00293449281402208\\
555.01	0.00289170596074975\\
556.01	0.00284806372036218\\
557.01	0.00280355101145738\\
558.01	0.00275815270571891\\
559.01	0.0027118536545707\\
560.01	0.00266463871923601\\
561.01	0.00261649280454222\\
562.01	0.00256740089684071\\
563.01	0.00251734810643931\\
564.01	0.00246631971497154\\
565.01	0.00241430122815276\\
566.01	0.00236127843439859\\
567.01	0.00230723746979893\\
568.01	0.00225216488995792\\
569.01	0.00219604774921472\\
570.01	0.00213887368775576\\
571.01	0.00208063102710875\\
572.01	0.00202130887446654\\
573.01	0.00196089723622006\\
574.01	0.00189938714097235\\
575.01	0.00183677077215304\\
576.01	0.00177304161013619\\
577.01	0.00170819458346989\\
578.01	0.00164222622842918\\
579.01	0.00157513485557815\\
580.01	0.0015069207213362\\
581.01	0.00143758620164401\\
582.01	0.0013671359636618\\
583.01	0.00129557712993561\\
584.01	0.0012229194275526\\
585.01	0.00114917531236232\\
586.01	0.00107436005523633\\
587.01	0.000998491773401186\\
588.01	0.000921591384903622\\
589.01	0.000843682457983908\\
590.01	0.00076479091922294\\
591.01	0.00068494457437703\\
592.01	0.000604172383317383\\
593.01	0.000522503414813216\\
594.01	0.000439965387250307\\
595.01	0.000356582676778022\\
596.01	0.000272373643602759\\
597.01	0.000187347088672941\\
598.01	0.000101497604931054\\
599.01	3.18230442563749e-05\\
599.02	3.12770957175551e-05\\
599.03	3.07343620567849e-05\\
599.04	3.0194875217571e-05\\
599.05	2.96586674565693e-05\\
599.06	2.91257713466771e-05\\
599.07	2.85962197801391e-05\\
599.08	2.80700459716916e-05\\
599.09	2.75472834617447e-05\\
599.1	2.70279661195791e-05\\
599.11	2.65121281465865e-05\\
599.12	2.59998040795448e-05\\
599.13	2.54910287939142e-05\\
599.14	2.49858375071712e-05\\
599.15	2.44842657821827e-05\\
599.16	2.39863495305973e-05\\
599.17	2.34921250162855e-05\\
599.18	2.30016288588035e-05\\
599.19	2.25148980369013e-05\\
599.2	2.20319698920526e-05\\
599.21	2.1552882132023e-05\\
599.22	2.10776728344891e-05\\
599.23	2.06063804506738e-05\\
599.24	2.01390438090109e-05\\
599.25	1.96757021188876e-05\\
599.26	1.92163949743647e-05\\
599.27	1.87611623579872e-05\\
599.28	1.83100446445959e-05\\
599.29	1.78630826051952e-05\\
599.3	1.74203174108517e-05\\
599.31	1.69817906366335e-05\\
599.32	1.65475442655914e-05\\
599.33	1.61176206927693e-05\\
599.34	1.56920627292639e-05\\
599.35	1.52709136063203e-05\\
599.36	1.48542169794742e-05\\
599.37	1.4442016932719e-05\\
599.38	1.40343579827385e-05\\
599.39	1.36312859119591e-05\\
599.4	1.32328499381877e-05\\
599.41	1.28390997671604e-05\\
599.42	1.24500855973528e-05\\
599.43	1.20658581248094e-05\\
599.44	1.16864685480531e-05\\
599.45	1.13119685730065e-05\\
599.46	1.09424104179929e-05\\
599.47	1.05778468187639e-05\\
599.48	1.02183310335888e-05\\
599.49	9.86391684839293e-06\\
599.5	9.51465858194112e-06\\
599.51	9.17061109107116e-06\\
599.52	8.83182977599525e-06\\
599.53	8.4983705856221e-06\\
599.54	8.1702900229675e-06\\
599.55	7.84764515058579e-06\\
599.56	7.53049359608453e-06\\
599.57	7.21889355765649e-06\\
599.58	6.91290380970891e-06\\
599.59	6.61258370851861e-06\\
599.6	6.31799319793756e-06\\
599.61	6.02919281519031e-06\\
599.62	5.74624369668701e-06\\
599.63	5.46920758391807e-06\\
599.64	5.19814682940593e-06\\
599.65	4.93312440269685e-06\\
599.66	4.67420389643411e-06\\
599.67	4.42144953247646e-06\\
599.68	4.17492616808582e-06\\
599.69	3.93469930216536e-06\\
599.7	3.70083508156871e-06\\
599.71	3.47340030746116e-06\\
599.72	3.25246244175549e-06\\
599.73	3.03808961360161e-06\\
599.74	2.83035062593855e-06\\
599.75	2.62931496212288e-06\\
599.76	2.43505279260911e-06\\
599.77	2.24763498169606e-06\\
599.78	2.06713309435641e-06\\
599.79	1.89361940310974e-06\\
599.8	1.72716689497872e-06\\
599.81	1.56784927850956e-06\\
599.82	1.41574099085315e-06\\
599.83	1.27091720493813e-06\\
599.84	1.13345383668736e-06\\
599.85	1.00342755231589e-06\\
599.86	8.8091577571392e-07\\
599.87	7.65996695880136e-07\\
599.88	6.58749274441706e-07\\
599.89	5.59253253243699e-07\\
599.9	4.67589162011367e-07\\
599.91	3.83838326099145e-07\\
599.92	3.08082874293444e-07\\
599.93	2.4040574671258e-07\\
599.94	1.80890702777825e-07\\
599.95	1.2962232925906e-07\\
599.96	8.66860484019516e-08\\
599.97	5.21681261349272e-08\\
599.98	2.61556803542173e-08\\
599.99	8.73668929909921e-09\\
600	0\\
};
\addplot [color=mycolor20,solid,forget plot]
  table[row sep=crcr]{%
0.01	0.00513241483564837\\
1.01	0.00513241334424749\\
2.01	0.00513241182204282\\
3.01	0.00513241026839975\\
4.01	0.00513240868267067\\
5.01	0.00513240706419493\\
6.01	0.00513240541229786\\
7.01	0.00513240372629167\\
8.01	0.00513240200547385\\
9.01	0.00513240024912786\\
10.01	0.00513239845652213\\
11.01	0.0051323966269105\\
12.01	0.00513239475953114\\
13.01	0.00513239285360692\\
14.01	0.00513239090834441\\
15.01	0.00513238892293442\\
16.01	0.00513238689655086\\
17.01	0.00513238482835078\\
18.01	0.00513238271747402\\
19.01	0.00513238056304285\\
20.01	0.00513237836416136\\
21.01	0.00513237611991541\\
22.01	0.0051323738293723\\
23.01	0.00513237149158016\\
24.01	0.00513236910556749\\
25.01	0.00513236667034322\\
26.01	0.00513236418489582\\
27.01	0.00513236164819297\\
28.01	0.0051323590591816\\
29.01	0.00513235641678669\\
30.01	0.00513235371991137\\
31.01	0.00513235096743661\\
32.01	0.0051323481582204\\
33.01	0.00513234529109728\\
34.01	0.00513234236487794\\
35.01	0.0051323393783491\\
36.01	0.00513233633027242\\
37.01	0.00513233321938416\\
38.01	0.00513233004439521\\
39.01	0.00513232680398992\\
40.01	0.00513232349682574\\
41.01	0.00513232012153271\\
42.01	0.00513231667671307\\
43.01	0.00513231316094059\\
44.01	0.00513230957275992\\
45.01	0.00513230591068593\\
46.01	0.00513230217320327\\
47.01	0.00513229835876573\\
48.01	0.00513229446579563\\
49.01	0.00513229049268308\\
50.01	0.00513228643778559\\
51.01	0.00513228229942684\\
52.01	0.00513227807589697\\
53.01	0.00513227376545062\\
54.01	0.00513226936630757\\
55.01	0.00513226487665105\\
56.01	0.00513226029462723\\
57.01	0.0051322556183449\\
58.01	0.00513225084587409\\
59.01	0.00513224597524579\\
60.01	0.00513224100445096\\
61.01	0.00513223593143984\\
62.01	0.0051322307541208\\
63.01	0.00513222547035982\\
64.01	0.00513222007797958\\
65.01	0.00513221457475855\\
66.01	0.00513220895843004\\
67.01	0.00513220322668146\\
68.01	0.00513219737715356\\
69.01	0.0051321914074386\\
70.01	0.00513218531508069\\
71.01	0.00513217909757361\\
72.01	0.00513217275236081\\
73.01	0.00513216627683364\\
74.01	0.00513215966833091\\
75.01	0.00513215292413726\\
76.01	0.00513214604148243\\
77.01	0.00513213901754015\\
78.01	0.00513213184942692\\
79.01	0.00513212453420093\\
80.01	0.00513211706886089\\
81.01	0.0051321094503449\\
82.01	0.00513210167552917\\
83.01	0.0051320937412265\\
84.01	0.00513208564418569\\
85.01	0.00513207738108971\\
86.01	0.00513206894855464\\
87.01	0.00513206034312811\\
88.01	0.00513205156128827\\
89.01	0.00513204259944226\\
90.01	0.00513203345392471\\
91.01	0.00513202412099615\\
92.01	0.00513201459684214\\
93.01	0.00513200487757133\\
94.01	0.0051319949592137\\
95.01	0.00513198483771948\\
96.01	0.00513197450895748\\
97.01	0.0051319639687133\\
98.01	0.00513195321268775\\
99.01	0.00513194223649546\\
100.01	0.00513193103566274\\
101.01	0.00513191960562597\\
102.01	0.00513190794172987\\
103.01	0.00513189603922598\\
104.01	0.00513188389327018\\
105.01	0.00513187149892124\\
106.01	0.00513185885113905\\
107.01	0.00513184594478222\\
108.01	0.00513183277460645\\
109.01	0.00513181933526249\\
110.01	0.00513180562129378\\
111.01	0.00513179162713463\\
112.01	0.00513177734710791\\
113.01	0.00513176277542302\\
114.01	0.00513174790617369\\
115.01	0.00513173273333546\\
116.01	0.00513171725076347\\
117.01	0.00513170145219042\\
118.01	0.00513168533122378\\
119.01	0.00513166888134339\\
120.01	0.00513165209589928\\
121.01	0.00513163496810875\\
122.01	0.00513161749105399\\
123.01	0.00513159965767963\\
124.01	0.00513158146078958\\
125.01	0.00513156289304461\\
126.01	0.00513154394695976\\
127.01	0.00513152461490098\\
128.01	0.00513150488908276\\
129.01	0.00513148476156499\\
130.01	0.00513146422424991\\
131.01	0.00513144326887912\\
132.01	0.00513142188703037\\
133.01	0.0051314000701145\\
134.01	0.00513137780937233\\
135.01	0.00513135509587135\\
136.01	0.00513133192050217\\
137.01	0.00513130827397524\\
138.01	0.00513128414681747\\
139.01	0.00513125952936845\\
140.01	0.0051312344117773\\
141.01	0.00513120878399871\\
142.01	0.00513118263578901\\
143.01	0.00513115595670288\\
144.01	0.00513112873608927\\
145.01	0.00513110096308722\\
146.01	0.00513107262662244\\
147.01	0.00513104371540271\\
148.01	0.00513101421791389\\
149.01	0.00513098412241576\\
150.01	0.00513095341693789\\
151.01	0.00513092208927484\\
152.01	0.0051308901269819\\
153.01	0.00513085751737094\\
154.01	0.00513082424750541\\
155.01	0.00513079030419562\\
156.01	0.00513075567399406\\
157.01	0.00513072034319089\\
158.01	0.00513068429780831\\
159.01	0.00513064752359607\\
160.01	0.00513061000602636\\
161.01	0.0051305717302882\\
162.01	0.00513053268128248\\
163.01	0.00513049284361671\\
164.01	0.00513045220159896\\
165.01	0.00513041073923299\\
166.01	0.00513036844021219\\
167.01	0.00513032528791378\\
168.01	0.00513028126539313\\
169.01	0.00513023635537786\\
170.01	0.00513019054026152\\
171.01	0.00513014380209775\\
172.01	0.00513009612259363\\
173.01	0.00513004748310344\\
174.01	0.00512999786462264\\
175.01	0.00512994724778056\\
176.01	0.00512989561283425\\
177.01	0.00512984293966133\\
178.01	0.00512978920775333\\
179.01	0.00512973439620871\\
180.01	0.00512967848372522\\
181.01	0.00512962144859345\\
182.01	0.00512956326868863\\
183.01	0.00512950392146356\\
184.01	0.00512944338394127\\
185.01	0.00512938163270644\\
186.01	0.00512931864389846\\
187.01	0.00512925439320302\\
188.01	0.00512918885584404\\
189.01	0.0051291220065753\\
190.01	0.00512905381967257\\
191.01	0.00512898426892433\\
192.01	0.00512891332762384\\
193.01	0.00512884096856025\\
194.01	0.00512876716400947\\
195.01	0.00512869188572549\\
196.01	0.00512861510493096\\
197.01	0.00512853679230802\\
198.01	0.00512845691798884\\
199.01	0.00512837545154612\\
200.01	0.00512829236198347\\
201.01	0.00512820761772528\\
202.01	0.00512812118660717\\
203.01	0.00512803303586553\\
204.01	0.00512794313212738\\
205.01	0.00512785144140001\\
206.01	0.00512775792906014\\
207.01	0.00512766255984398\\
208.01	0.00512756529783547\\
209.01	0.005127466106456\\
210.01	0.00512736494845286\\
211.01	0.00512726178588829\\
212.01	0.00512715658012785\\
213.01	0.00512704929182884\\
214.01	0.00512693988092886\\
215.01	0.00512682830663378\\
216.01	0.00512671452740574\\
217.01	0.00512659850095092\\
218.01	0.0051264801842074\\
219.01	0.00512635953333281\\
220.01	0.00512623650369184\\
221.01	0.00512611104984295\\
222.01	0.00512598312552607\\
223.01	0.00512585268364942\\
224.01	0.00512571967627659\\
225.01	0.00512558405461285\\
226.01	0.00512544576899208\\
227.01	0.00512530476886323\\
228.01	0.00512516100277649\\
229.01	0.00512501441836955\\
230.01	0.00512486496235384\\
231.01	0.0051247125804999\\
232.01	0.00512455721762403\\
233.01	0.00512439881757345\\
234.01	0.00512423732321212\\
235.01	0.00512407267640616\\
236.01	0.00512390481800906\\
237.01	0.00512373368784713\\
238.01	0.00512355922470507\\
239.01	0.00512338136630993\\
240.01	0.00512320004931743\\
241.01	0.00512301520929611\\
242.01	0.00512282678071231\\
243.01	0.00512263469691465\\
244.01	0.00512243889011949\\
245.01	0.00512223929139481\\
246.01	0.00512203583064516\\
247.01	0.00512182843659607\\
248.01	0.00512161703677903\\
249.01	0.00512140155751532\\
250.01	0.00512118192390098\\
251.01	0.00512095805979098\\
252.01	0.00512072988778389\\
253.01	0.00512049732920648\\
254.01	0.00512026030409788\\
255.01	0.00512001873119458\\
256.01	0.00511977252791487\\
257.01	0.00511952161034348\\
258.01	0.00511926589321649\\
259.01	0.00511900528990631\\
260.01	0.00511873971240635\\
261.01	0.00511846907131616\\
262.01	0.00511819327582711\\
263.01	0.00511791223370725\\
264.01	0.00511762585128748\\
265.01	0.00511733403344705\\
266.01	0.00511703668359962\\
267.01	0.00511673370367922\\
268.01	0.00511642499412766\\
269.01	0.00511611045388039\\
270.01	0.0051157899803543\\
271.01	0.00511546346943497\\
272.01	0.00511513081546445\\
273.01	0.00511479191122941\\
274.01	0.00511444664795006\\
275.01	0.00511409491526913\\
276.01	0.0051137366012415\\
277.01	0.00511337159232413\\
278.01	0.00511299977336665\\
279.01	0.00511262102760302\\
280.01	0.00511223523664279\\
281.01	0.00511184228046374\\
282.01	0.00511144203740521\\
283.01	0.00511103438416184\\
284.01	0.00511061919577816\\
285.01	0.00511019634564426\\
286.01	0.0051097657054915\\
287.01	0.0051093271453905\\
288.01	0.00510888053374866\\
289.01	0.00510842573730956\\
290.01	0.00510796262115269\\
291.01	0.0051074910486952\\
292.01	0.00510701088169404\\
293.01	0.00510652198024906\\
294.01	0.0051060242028085\\
295.01	0.00510551740617429\\
296.01	0.00510500144550995\\
297.01	0.00510447617434908\\
298.01	0.0051039414446056\\
299.01	0.00510339710658558\\
300.01	0.00510284300900028\\
301.01	0.00510227899898119\\
302.01	0.0051017049220967\\
303.01	0.00510112062237015\\
304.01	0.00510052594230007\\
305.01	0.00509992072288226\\
306.01	0.00509930480363314\\
307.01	0.00509867802261618\\
308.01	0.005098040216469\\
309.01	0.00509739122043389\\
310.01	0.00509673086838949\\
311.01	0.00509605899288485\\
312.01	0.00509537542517597\\
313.01	0.0050946799952644\\
314.01	0.00509397253193796\\
315.01	0.00509325286281404\\
316.01	0.00509252081438535\\
317.01	0.0050917762120676\\
318.01	0.00509101888024977\\
319.01	0.00509024864234706\\
320.01	0.00508946532085563\\
321.01	0.00508866873741051\\
322.01	0.00508785871284507\\
323.01	0.00508703506725312\\
324.01	0.00508619762005377\\
325.01	0.00508534619005734\\
326.01	0.00508448059553462\\
327.01	0.00508360065428733\\
328.01	0.00508270618372094\\
329.01	0.0050817970009187\\
330.01	0.00508087292271784\\
331.01	0.00507993376578699\\
332.01	0.00507897934670416\\
333.01	0.00507800948203693\\
334.01	0.00507702398842144\\
335.01	0.00507602268264328\\
336.01	0.00507500538171723\\
337.01	0.00507397190296693\\
338.01	0.00507292206410296\\
339.01	0.00507185568330056\\
340.01	0.00507077257927324\\
341.01	0.00506967257134569\\
342.01	0.00506855547952166\\
343.01	0.00506742112454792\\
344.01	0.00506626932797324\\
345.01	0.00506509991220067\\
346.01	0.00506391270053324\\
347.01	0.00506270751721066\\
348.01	0.0050614841874369\\
349.01	0.00506024253739589\\
350.01	0.00505898239425578\\
351.01	0.00505770358615814\\
352.01	0.00505640594219222\\
353.01	0.00505508929235078\\
354.01	0.00505375346746671\\
355.01	0.00505239829912762\\
356.01	0.00505102361956614\\
357.01	0.00504962926152432\\
358.01	0.00504821505808848\\
359.01	0.00504678084249199\\
360.01	0.00504532644788424\\
361.01	0.00504385170706125\\
362.01	0.00504235645215611\\
363.01	0.00504084051428484\\
364.01	0.00503930372314545\\
365.01	0.00503774590656626\\
366.01	0.00503616689000066\\
367.01	0.00503456649596352\\
368.01	0.00503294454340806\\
369.01	0.0050313008470378\\
370.01	0.00502963521655258\\
371.01	0.00502794745582441\\
372.01	0.00502623736200119\\
373.01	0.00502450472453784\\
374.01	0.00502274932415067\\
375.01	0.00502097093169802\\
376.01	0.00501916930698559\\
377.01	0.00501734419749892\\
378.01	0.00501549533706614\\
379.01	0.00501362244445593\\
380.01	0.00501172522191803\\
381.01	0.0050098033536758\\
382.01	0.00500785650438091\\
383.01	0.00500588431754847\\
384.01	0.00500388641398753\\
385.01	0.00500186239025184\\
386.01	0.00499981181713583\\
387.01	0.00499773423824771\\
388.01	0.00499562916869507\\
389.01	0.00499349609392439\\
390.01	0.00499133446875998\\
391.01	0.00498914371669356\\
392.01	0.00498692322948159\\
393.01	0.00498467236710779\\
394.01	0.0049823904581755\\
395.01	0.00498007680079112\\
396.01	0.00497773066400296\\
397.01	0.00497535128984978\\
398.01	0.00497293789606803\\
399.01	0.00497048967948952\\
400.01	0.00496800582013984\\
401.01	0.00496548548601524\\
402.01	0.00496292783847709\\
403.01	0.00496033203814928\\
404.01	0.00495769725113726\\
405.01	0.00495502265531409\\
406.01	0.00495230744632645\\
407.01	0.00494955084288241\\
408.01	0.00494675209079259\\
409.01	0.00494391046515843\\
410.01	0.00494102527007969\\
411.01	0.00493809583530152\\
412.01	0.0049351215094292\\
413.01	0.00493210164978179\\
414.01	0.00492903560998622\\
415.01	0.00492592273033172\\
416.01	0.00492276233481905\\
417.01	0.00491955373006865\\
418.01	0.00491629620420586\\
419.01	0.00491298902567613\\
420.01	0.00490963144199091\\
421.01	0.00490622267840105\\
422.01	0.00490276193649748\\
423.01	0.00489924839273843\\
424.01	0.00489568119690254\\
425.01	0.00489205947046875\\
426.01	0.0048883823049225\\
427.01	0.00488464875999325\\
428.01	0.00488085786182232\\
429.01	0.00487700860106707\\
430.01	0.00487309993094574\\
431.01	0.00486913076522753\\
432.01	0.00486509997617781\\
433.01	0.00486100639246517\\
434.01	0.00485684879704068\\
435.01	0.00485262592500365\\
436.01	0.00484833646146494\\
437.01	0.00484397903942548\\
438.01	0.00483955223768783\\
439.01	0.00483505457881895\\
440.01	0.00483048452718812\\
441.01	0.00482584048710082\\
442.01	0.00482112080105842\\
443.01	0.00481632374816491\\
444.01	0.0048114475427145\\
445.01	0.00480649033298484\\
446.01	0.00480145020026725\\
447.01	0.00479632515816302\\
448.01	0.00479111315217012\\
449.01	0.00478581205958852\\
450.01	0.00478041968976229\\
451.01	0.00477493378467433\\
452.01	0.00476935201990197\\
453.01	0.00476367200593159\\
454.01	0.00475789128981889\\
455.01	0.00475200735717024\\
456.01	0.00474601763439902\\
457.01	0.004739919491201\\
458.01	0.00473371024316399\\
459.01	0.0047273871544136\\
460.01	0.00472094744016993\\
461.01	0.00471438826907376\\
462.01	0.00470770676512423\\
463.01	0.00470090000905348\\
464.01	0.00469396503897045\\
465.01	0.00468689885010838\\
466.01	0.00467969839354542\\
467.01	0.00467236057381463\\
468.01	0.00466488224539955\\
469.01	0.00465726020821576\\
470.01	0.00464949120231721\\
471.01	0.00464157190222172\\
472.01	0.00463349891140976\\
473.01	0.00462526875768144\\
474.01	0.00461687789006565\\
475.01	0.00460832267757614\\
476.01	0.00459959940900615\\
477.01	0.00459070429294342\\
478.01	0.00458163345786526\\
479.01	0.00457238295229263\\
480.01	0.00456294874497674\\
481.01	0.00455332672509053\\
482.01	0.00454351270239028\\
483.01	0.00453350240731102\\
484.01	0.00452329149095487\\
485.01	0.0045128755249317\\
486.01	0.00450225000100838\\
487.01	0.00449141033052669\\
488.01	0.00448035184355369\\
489.01	0.00446906978773375\\
490.01	0.00445755932682209\\
491.01	0.00444581553889245\\
492.01	0.00443383341422656\\
493.01	0.00442160785291254\\
494.01	0.00440913366219645\\
495.01	0.00439640555365504\\
496.01	0.00438341814026985\\
497.01	0.00437016593349746\\
498.01	0.00435664334043017\\
499.01	0.00434284466113117\\
500.01	0.00432876408619353\\
501.01	0.00431439569452768\\
502.01	0.00429973345131089\\
503.01	0.00428477120597454\\
504.01	0.00426950269010674\\
505.01	0.00425392151523104\\
506.01	0.0042380211704564\\
507.01	0.00422179502000055\\
508.01	0.00420523630059349\\
509.01	0.00418833811877135\\
510.01	0.00417109344807316\\
511.01	0.00415349512615779\\
512.01	0.00413553585186063\\
513.01	0.00411720818221027\\
514.01	0.00409850452942619\\
515.01	0.00407941715791657\\
516.01	0.00405993818129196\\
517.01	0.00404005955940527\\
518.01	0.00401977309542275\\
519.01	0.0039990704329228\\
520.01	0.00397794305301479\\
521.01	0.00395638227146918\\
522.01	0.00393437923584688\\
523.01	0.00391192492263153\\
524.01	0.00388901013437162\\
525.01	0.00386562549684443\\
526.01	0.00384176145625882\\
527.01	0.00381740827651028\\
528.01	0.00379255603650427\\
529.01	0.00376719462756544\\
530.01	0.00374131375094899\\
531.01	0.00371490291547166\\
532.01	0.0036879514352818\\
533.01	0.00366044842778692\\
534.01	0.00363238281176114\\
535.01	0.00360374330565611\\
536.01	0.0035745184261425\\
537.01	0.00354469648691371\\
538.01	0.00351426559778857\\
539.01	0.00348321366415254\\
540.01	0.0034515283867838\\
541.01	0.00341919726211485\\
542.01	0.00338620758298481\\
543.01	0.00335254643994394\\
544.01	0.00331820072317836\\
545.01	0.00328315712513108\\
546.01	0.00324740214390159\\
547.01	0.0032109220875175\\
548.01	0.00317370307917983\\
549.01	0.00313573106359633\\
550.01	0.00309699181452752\\
551.01	0.00305747094368483\\
552.01	0.00301715391113351\\
553.01	0.00297602603736896\\
554.01	0.00293407251725179\\
555.01	0.00289127843600545\\
556.01	0.00284762878749973\\
557.01	0.0028031084950645\\
558.01	0.00275770243510144\\
559.01	0.00271139546378433\\
560.01	0.00266417244716499\\
561.01	0.0026160182950273\\
562.01	0.00256691799886021\\
563.01	0.00251685667434632\\
564.01	0.00246581960879106\\
565.01	0.00241379231394375\\
566.01	0.00236076058468414\\
567.01	0.00230671056407004\\
568.01	0.00225162881525418\\
569.01	0.00219550240078614\\
570.01	0.00213831896980923\\
571.01	0.00208006685364094\\
572.01	0.0020207351701836\\
573.01	0.00196031393754173\\
574.01	0.00189879419711535\\
575.01	0.0018361681462836\\
576.01	0.00177242928057545\\
577.01	0.00170757254492862\\
578.01	0.00164159449323789\\
579.01	0.00157449345486604\\
580.01	0.00150626970609533\\
581.01	0.00143692564359493\\
582.01	0.00136646595580921\\
583.01	0.00129489778666962\\
584.01	0.00122223088410795\\
585.01	0.00114847772339413\\
586.01	0.00107365359220303\\
587.01	0.000997776620360649\\
588.01	0.000920867732220132\\
589.01	0.000842950493310577\\
590.01	0.000764050814954753\\
591.01	0.000684196470559372\\
592.01	0.000603416364730805\\
593.01	0.000521739480625437\\
594.01	0.000439193411214457\\
595.01	0.000355802355441642\\
596.01	0.00027158442935076\\
597.01	0.000186548103627336\\
598.01	0.000100687530733697\\
599.01	3.18230442563749e-05\\
599.02	3.12770957175551e-05\\
599.03	3.07343620567866e-05\\
599.04	3.0194875217571e-05\\
599.05	2.96586674565693e-05\\
599.06	2.91257713466771e-05\\
599.07	2.85962197801391e-05\\
599.08	2.80700459716933e-05\\
599.09	2.75472834617464e-05\\
599.1	2.70279661195791e-05\\
599.11	2.65121281465865e-05\\
599.12	2.59998040795448e-05\\
599.13	2.54910287939124e-05\\
599.14	2.49858375071712e-05\\
599.15	2.44842657821827e-05\\
599.16	2.39863495305956e-05\\
599.17	2.34921250162837e-05\\
599.18	2.30016288588052e-05\\
599.19	2.25148980369013e-05\\
599.2	2.20319698920508e-05\\
599.21	2.1552882132023e-05\\
599.22	2.10776728344908e-05\\
599.23	2.06063804506721e-05\\
599.24	2.01390438090126e-05\\
599.25	1.96757021188858e-05\\
599.26	1.9216394974363e-05\\
599.27	1.87611623579872e-05\\
599.28	1.83100446445959e-05\\
599.29	1.78630826051952e-05\\
599.3	1.74203174108517e-05\\
599.31	1.69817906366353e-05\\
599.32	1.65475442655931e-05\\
599.33	1.61176206927693e-05\\
599.34	1.56920627292622e-05\\
599.35	1.52709136063203e-05\\
599.36	1.48542169794725e-05\\
599.37	1.44420169327208e-05\\
599.38	1.40343579827368e-05\\
599.39	1.36312859119591e-05\\
599.4	1.32328499381877e-05\\
599.41	1.28390997671604e-05\\
599.42	1.2450085597351e-05\\
599.43	1.20658581248111e-05\\
599.44	1.16864685480531e-05\\
599.45	1.13119685730082e-05\\
599.46	1.09424104179929e-05\\
599.47	1.05778468187639e-05\\
599.48	1.02183310335888e-05\\
599.49	9.86391684839293e-06\\
599.5	9.51465858193938e-06\\
599.51	9.17061109107289e-06\\
599.52	8.83182977599525e-06\\
599.53	8.4983705856221e-06\\
599.54	8.1702900229675e-06\\
599.55	7.84764515058753e-06\\
599.56	7.53049359608279e-06\\
599.57	7.21889355765649e-06\\
599.58	6.91290380971064e-06\\
599.59	6.61258370851688e-06\\
599.6	6.31799319793756e-06\\
599.61	6.02919281519031e-06\\
599.62	5.74624369668528e-06\\
599.63	5.46920758391981e-06\\
599.64	5.19814682940593e-06\\
599.65	4.93312440269685e-06\\
599.66	4.67420389643237e-06\\
599.67	4.42144953247819e-06\\
599.68	4.17492616808582e-06\\
599.69	3.9346993021671e-06\\
599.7	3.70083508156871e-06\\
599.71	3.47340030746289e-06\\
599.72	3.25246244175549e-06\\
599.73	3.03808961359987e-06\\
599.74	2.83035062593855e-06\\
599.75	2.62931496212288e-06\\
599.76	2.43505279260738e-06\\
599.77	2.24763498169606e-06\\
599.78	2.06713309435641e-06\\
599.79	1.89361940311147e-06\\
599.8	1.72716689497872e-06\\
599.81	1.56784927850956e-06\\
599.82	1.41574099085315e-06\\
599.83	1.27091720493813e-06\\
599.84	1.13345383668563e-06\\
599.85	1.00342755231415e-06\\
599.86	8.8091577571392e-07\\
599.87	7.65996695880136e-07\\
599.88	6.58749274441706e-07\\
599.89	5.59253253241965e-07\\
599.9	4.67589162013102e-07\\
599.91	3.83838326099145e-07\\
599.92	3.0808287429171e-07\\
599.93	2.40405746710845e-07\\
599.94	1.8089070277609e-07\\
599.95	1.29622329257326e-07\\
599.96	8.66860484002169e-08\\
599.97	5.21681261331924e-08\\
599.98	2.61556803542173e-08\\
599.99	8.73668930083393e-09\\
600	0\\
};
\addplot [color=mycolor21,solid,forget plot]
  table[row sep=crcr]{%
0.01	0.00507708887705905\\
1.01	0.00507708753621324\\
2.01	0.0050770861680176\\
3.01	0.00507708477191713\\
4.01	0.0050770833473457\\
5.01	0.00507708189372535\\
6.01	0.00507708041046706\\
7.01	0.00507707889696935\\
8.01	0.00507707735261923\\
9.01	0.00507707577679103\\
10.01	0.00507707416884656\\
11.01	0.00507707252813467\\
12.01	0.00507707085399132\\
13.01	0.00507706914573891\\
14.01	0.0050770674026864\\
15.01	0.00507706562412882\\
16.01	0.0050770638093468\\
17.01	0.00507706195760675\\
18.01	0.00507706006816042\\
19.01	0.00507705814024422\\
20.01	0.00507705617307949\\
21.01	0.00507705416587195\\
22.01	0.00507705211781113\\
23.01	0.00507705002807054\\
24.01	0.0050770478958069\\
25.01	0.00507704572016018\\
26.01	0.00507704350025313\\
27.01	0.00507704123519085\\
28.01	0.00507703892406036\\
29.01	0.00507703656593057\\
30.01	0.00507703415985188\\
31.01	0.0050770317048553\\
32.01	0.00507702919995268\\
33.01	0.00507702664413585\\
34.01	0.00507702403637687\\
35.01	0.00507702137562703\\
36.01	0.00507701866081639\\
37.01	0.00507701589085413\\
38.01	0.0050770130646273\\
39.01	0.0050770101810007\\
40.01	0.00507700723881643\\
41.01	0.00507700423689375\\
42.01	0.00507700117402806\\
43.01	0.00507699804899074\\
44.01	0.00507699486052866\\
45.01	0.00507699160736389\\
46.01	0.0050769882881928\\
47.01	0.00507698490168587\\
48.01	0.00507698144648692\\
49.01	0.00507697792121297\\
50.01	0.00507697432445339\\
51.01	0.00507697065476941\\
52.01	0.00507696691069357\\
53.01	0.00507696309072958\\
54.01	0.00507695919335097\\
55.01	0.00507695521700095\\
56.01	0.0050769511600919\\
57.01	0.00507694702100466\\
58.01	0.00507694279808788\\
59.01	0.00507693848965749\\
60.01	0.00507693409399573\\
61.01	0.00507692960935092\\
62.01	0.00507692503393644\\
63.01	0.00507692036593068\\
64.01	0.00507691560347542\\
65.01	0.00507691074467593\\
66.01	0.00507690578759975\\
67.01	0.0050769007302762\\
68.01	0.00507689557069561\\
69.01	0.00507689030680855\\
70.01	0.00507688493652499\\
71.01	0.00507687945771347\\
72.01	0.00507687386820064\\
73.01	0.00507686816577011\\
74.01	0.0050768623481615\\
75.01	0.00507685641306991\\
76.01	0.00507685035814514\\
77.01	0.00507684418099038\\
78.01	0.00507683787916161\\
79.01	0.00507683145016669\\
80.01	0.00507682489146434\\
81.01	0.0050768182004631\\
82.01	0.00507681137452044\\
83.01	0.00507680441094229\\
84.01	0.00507679730698098\\
85.01	0.00507679005983503\\
86.01	0.00507678266664797\\
87.01	0.00507677512450732\\
88.01	0.00507676743044301\\
89.01	0.00507675958142686\\
90.01	0.00507675157437126\\
91.01	0.00507674340612815\\
92.01	0.00507673507348772\\
93.01	0.00507672657317712\\
94.01	0.00507671790185935\\
95.01	0.00507670905613247\\
96.01	0.00507670003252765\\
97.01	0.00507669082750827\\
98.01	0.00507668143746858\\
99.01	0.00507667185873229\\
100.01	0.0050766620875516\\
101.01	0.00507665212010514\\
102.01	0.00507664195249747\\
103.01	0.00507663158075676\\
104.01	0.00507662100083391\\
105.01	0.0050766102086011\\
106.01	0.00507659919985008\\
107.01	0.00507658797029066\\
108.01	0.00507657651554915\\
109.01	0.00507656483116712\\
110.01	0.0050765529125992\\
111.01	0.00507654075521193\\
112.01	0.00507652835428203\\
113.01	0.00507651570499456\\
114.01	0.00507650280244127\\
115.01	0.00507648964161869\\
116.01	0.00507647621742685\\
117.01	0.00507646252466667\\
118.01	0.00507644855803879\\
119.01	0.00507643431214153\\
120.01	0.00507641978146884\\
121.01	0.00507640496040837\\
122.01	0.00507638984323938\\
123.01	0.00507637442413115\\
124.01	0.0050763586971405\\
125.01	0.00507634265621009\\
126.01	0.00507632629516564\\
127.01	0.0050763096077144\\
128.01	0.00507629258744297\\
129.01	0.00507627522781448\\
130.01	0.00507625752216674\\
131.01	0.00507623946371005\\
132.01	0.00507622104552452\\
133.01	0.00507620226055775\\
134.01	0.00507618310162244\\
135.01	0.00507616356139414\\
136.01	0.00507614363240815\\
137.01	0.00507612330705765\\
138.01	0.00507610257759048\\
139.01	0.005076081436107\\
140.01	0.00507605987455689\\
141.01	0.00507603788473684\\
142.01	0.00507601545828773\\
143.01	0.00507599258669128\\
144.01	0.00507596926126792\\
145.01	0.00507594547317327\\
146.01	0.00507592121339545\\
147.01	0.00507589647275183\\
148.01	0.00507587124188621\\
149.01	0.00507584551126559\\
150.01	0.00507581927117713\\
151.01	0.00507579251172424\\
152.01	0.00507576522282425\\
153.01	0.0050757373942047\\
154.01	0.00507570901539941\\
155.01	0.00507568007574582\\
156.01	0.00507565056438109\\
157.01	0.00507562047023848\\
158.01	0.00507558978204405\\
159.01	0.00507555848831241\\
160.01	0.00507552657734379\\
161.01	0.00507549403721958\\
162.01	0.00507546085579868\\
163.01	0.0050754270207135\\
164.01	0.00507539251936635\\
165.01	0.00507535733892502\\
166.01	0.00507532146631874\\
167.01	0.00507528488823392\\
168.01	0.00507524759111015\\
169.01	0.00507520956113589\\
170.01	0.00507517078424397\\
171.01	0.00507513124610713\\
172.01	0.00507509093213366\\
173.01	0.00507504982746286\\
174.01	0.00507500791695994\\
175.01	0.0050749651852123\\
176.01	0.00507492161652357\\
177.01	0.00507487719490985\\
178.01	0.005074831904094\\
179.01	0.0050747857275009\\
180.01	0.00507473864825254\\
181.01	0.00507469064916273\\
182.01	0.00507464171273208\\
183.01	0.00507459182114255\\
184.01	0.00507454095625181\\
185.01	0.00507448909958861\\
186.01	0.0050744362323465\\
187.01	0.00507438233537851\\
188.01	0.0050743273891917\\
189.01	0.005074271373941\\
190.01	0.00507421426942365\\
191.01	0.00507415605507348\\
192.01	0.0050740967099546\\
193.01	0.00507403621275545\\
194.01	0.00507397454178257\\
195.01	0.0050739116749546\\
196.01	0.00507384758979588\\
197.01	0.00507378226342988\\
198.01	0.00507371567257321\\
199.01	0.00507364779352859\\
200.01	0.00507357860217837\\
201.01	0.00507350807397796\\
202.01	0.00507343618394885\\
203.01	0.00507336290667201\\
204.01	0.00507328821628047\\
205.01	0.00507321208645276\\
206.01	0.00507313449040569\\
207.01	0.00507305540088659\\
208.01	0.00507297479016699\\
209.01	0.00507289263003433\\
210.01	0.00507280889178514\\
211.01	0.00507272354621716\\
212.01	0.0050726365636218\\
213.01	0.00507254791377652\\
214.01	0.00507245756593692\\
215.01	0.00507236548882902\\
216.01	0.00507227165064126\\
217.01	0.00507217601901641\\
218.01	0.00507207856104334\\
219.01	0.00507197924324943\\
220.01	0.00507187803159139\\
221.01	0.00507177489144776\\
222.01	0.0050716697876099\\
223.01	0.00507156268427412\\
224.01	0.00507145354503242\\
225.01	0.00507134233286413\\
226.01	0.00507122901012767\\
227.01	0.00507111353855088\\
228.01	0.00507099587922289\\
229.01	0.00507087599258482\\
230.01	0.00507075383842098\\
231.01	0.00507062937585008\\
232.01	0.00507050256331542\\
233.01	0.00507037335857621\\
234.01	0.00507024171869829\\
235.01	0.00507010760004478\\
236.01	0.00506997095826677\\
237.01	0.00506983174829409\\
238.01	0.00506968992432524\\
239.01	0.00506954543981895\\
240.01	0.00506939824748368\\
241.01	0.00506924829926839\\
242.01	0.00506909554635304\\
243.01	0.00506893993913896\\
244.01	0.00506878142723886\\
245.01	0.00506861995946761\\
246.01	0.00506845548383187\\
247.01	0.0050682879475211\\
248.01	0.00506811729689692\\
249.01	0.00506794347748404\\
250.01	0.00506776643395975\\
251.01	0.005067586110145\\
252.01	0.00506740244899392\\
253.01	0.00506721539258424\\
254.01	0.00506702488210727\\
255.01	0.00506683085785847\\
256.01	0.00506663325922733\\
257.01	0.00506643202468773\\
258.01	0.0050662270917882\\
259.01	0.00506601839714222\\
260.01	0.00506580587641847\\
261.01	0.00506558946433108\\
262.01	0.00506536909463067\\
263.01	0.00506514470009385\\
264.01	0.00506491621251443\\
265.01	0.00506468356269365\\
266.01	0.005064446680431\\
267.01	0.0050642054945148\\
268.01	0.00506395993271301\\
269.01	0.00506370992176399\\
270.01	0.00506345538736767\\
271.01	0.00506319625417625\\
272.01	0.00506293244578591\\
273.01	0.00506266388472717\\
274.01	0.00506239049245704\\
275.01	0.00506211218935009\\
276.01	0.00506182889469009\\
277.01	0.00506154052666153\\
278.01	0.0050612470023417\\
279.01	0.00506094823769231\\
280.01	0.00506064414755202\\
281.01	0.00506033464562809\\
282.01	0.00506001964448908\\
283.01	0.00505969905555732\\
284.01	0.00505937278910131\\
285.01	0.00505904075422845\\
286.01	0.00505870285887829\\
287.01	0.00505835900981521\\
288.01	0.00505800911262169\\
289.01	0.00505765307169174\\
290.01	0.00505729079022411\\
291.01	0.00505692217021622\\
292.01	0.00505654711245727\\
293.01	0.00505616551652256\\
294.01	0.00505577728076712\\
295.01	0.00505538230231984\\
296.01	0.00505498047707758\\
297.01	0.00505457169969901\\
298.01	0.0050541558635993\\
299.01	0.00505373286094394\\
300.01	0.00505330258264308\\
301.01	0.00505286491834604\\
302.01	0.00505241975643513\\
303.01	0.00505196698401995\\
304.01	0.00505150648693165\\
305.01	0.00505103814971642\\
306.01	0.00505056185562958\\
307.01	0.00505007748662879\\
308.01	0.00504958492336787\\
309.01	0.00504908404518902\\
310.01	0.00504857473011588\\
311.01	0.00504805685484564\\
312.01	0.00504753029474055\\
313.01	0.00504699492381921\\
314.01	0.0050464506147466\\
315.01	0.00504589723882434\\
316.01	0.00504533466597911\\
317.01	0.00504476276475078\\
318.01	0.00504418140227924\\
319.01	0.00504359044429042\\
320.01	0.00504298975508064\\
321.01	0.00504237919749954\\
322.01	0.00504175863293179\\
323.01	0.0050411279212771\\
324.01	0.00504048692092766\\
325.01	0.00503983548874477\\
326.01	0.00503917348003241\\
327.01	0.00503850074850792\\
328.01	0.00503781714627182\\
329.01	0.00503712252377266\\
330.01	0.00503641672977052\\
331.01	0.00503569961129585\\
332.01	0.00503497101360607\\
333.01	0.00503423078013689\\
334.01	0.00503347875245049\\
335.01	0.00503271477017883\\
336.01	0.00503193867096227\\
337.01	0.00503115029038325\\
338.01	0.00503034946189406\\
339.01	0.00502953601673909\\
340.01	0.00502870978387105\\
341.01	0.00502787058985985\\
342.01	0.00502701825879568\\
343.01	0.00502615261218312\\
344.01	0.00502527346882931\\
345.01	0.00502438064472175\\
346.01	0.00502347395289909\\
347.01	0.00502255320331151\\
348.01	0.005021618202672\\
349.01	0.00502066875429794\\
350.01	0.00501970465794117\\
351.01	0.0050187257096079\\
352.01	0.00501773170136676\\
353.01	0.00501672242114493\\
354.01	0.00501569765251243\\
355.01	0.00501465717445341\\
356.01	0.00501360076112476\\
357.01	0.00501252818160172\\
358.01	0.00501143919960925\\
359.01	0.00501033357324104\\
360.01	0.00500921105466373\\
361.01	0.00500807138980852\\
362.01	0.00500691431804838\\
363.01	0.00500573957186251\\
364.01	0.00500454687648826\\
365.01	0.0050033359495601\\
366.01	0.0050021065007374\\
367.01	0.00500085823132129\\
368.01	0.00499959083386242\\
369.01	0.00499830399176034\\
370.01	0.00499699737885614\\
371.01	0.00499567065902101\\
372.01	0.00499432348574205\\
373.01	0.00499295550170825\\
374.01	0.00499156633840018\\
375.01	0.00499015561568501\\
376.01	0.00498872294142227\\
377.01	0.00498726791108275\\
378.01	0.00498579010738665\\
379.01	0.00498428909996395\\
380.01	0.00498276444504269\\
381.01	0.00498121568517083\\
382.01	0.00497964234897718\\
383.01	0.00497804395097536\\
384.01	0.00497641999142021\\
385.01	0.00497476995621788\\
386.01	0.00497309331689844\\
387.01	0.00497138953065381\\
388.01	0.00496965804044674\\
389.01	0.00496789827519218\\
390.01	0.00496610965001477\\
391.01	0.00496429156658303\\
392.01	0.00496244341351421\\
393.01	0.00496056456685244\\
394.01	0.00495865439060511\\
395.01	0.00495671223733239\\
396.01	0.00495473744877186\\
397.01	0.00495272935648013\\
398.01	0.00495068728246698\\
399.01	0.00494861053979386\\
400.01	0.00494649843310342\\
401.01	0.00494435025904248\\
402.01	0.00494216530653852\\
403.01	0.00493994285688795\\
404.01	0.0049376821836139\\
405.01	0.00493538255205873\\
406.01	0.00493304321868069\\
407.01	0.0049306634300419\\
408.01	0.00492824242149082\\
409.01	0.00492577941557181\\
410.01	0.00492327362022339\\
411.01	0.00492072422686674\\
412.01	0.00491813040851883\\
413.01	0.00491549131808978\\
414.01	0.00491280608702397\\
415.01	0.00491007382433465\\
416.01	0.00490729361582734\\
417.01	0.00490446452333173\\
418.01	0.00490158558391598\\
419.01	0.00489865580908632\\
420.01	0.00489567418397143\\
421.01	0.00489263966649393\\
422.01	0.00488955118653039\\
423.01	0.00488640764506158\\
424.01	0.00488320791331547\\
425.01	0.00487995083190449\\
426.01	0.00487663520996096\\
427.01	0.00487325982427065\\
428.01	0.00486982341841193\\
429.01	0.00486632470189805\\
430.01	0.00486276234933108\\
431.01	0.00485913499956767\\
432.01	0.00485544125490201\\
433.01	0.00485167968026843\\
434.01	0.00484784880246914\\
435.01	0.00484394710942952\\
436.01	0.00483997304948575\\
437.01	0.00483592503070799\\
438.01	0.00483180142026234\\
439.01	0.0048276005438161\\
440.01	0.00482332068498685\\
441.01	0.00481896008484052\\
442.01	0.00481451694143632\\
443.01	0.00480998940942305\\
444.01	0.0048053755996833\\
445.01	0.00480067357902606\\
446.01	0.00479588136992385\\
447.01	0.00479099695029043\\
448.01	0.00478601825329393\\
449.01	0.00478094316719583\\
450.01	0.00477576953520854\\
451.01	0.00477049515535841\\
452.01	0.00476511778034158\\
453.01	0.0047596351173582\\
454.01	0.00475404482790781\\
455.01	0.00474834452752741\\
456.01	0.00474253178545619\\
457.01	0.00473660412420473\\
458.01	0.00473055901901403\\
459.01	0.00472439389718501\\
460.01	0.00471810613726825\\
461.01	0.0047116930681009\\
462.01	0.00470515196768929\\
463.01	0.00469848006193911\\
464.01	0.00469167452324305\\
465.01	0.00468473246894722\\
466.01	0.00467765095972418\\
467.01	0.00467042699789038\\
468.01	0.00466305752571401\\
469.01	0.00465553942376162\\
470.01	0.00464786950933107\\
471.01	0.00464004453501251\\
472.01	0.0046320611874001\\
473.01	0.00462391608595133\\
474.01	0.00461560578196254\\
475.01	0.00460712675759189\\
476.01	0.0045984754248689\\
477.01	0.00458964812466511\\
478.01	0.00458064112561716\\
479.01	0.00457145062299611\\
480.01	0.00456207273751556\\
481.01	0.00455250351407303\\
482.01	0.00454273892041895\\
483.01	0.00453277484574797\\
484.01	0.00452260709921037\\
485.01	0.00451223140833936\\
486.01	0.00450164341739547\\
487.01	0.00449083868562842\\
488.01	0.00447981268545985\\
489.01	0.00446856080059189\\
490.01	0.00445707832404862\\
491.01	0.00444536045615936\\
492.01	0.00443340230249307\\
493.01	0.00442119887175426\\
494.01	0.00440874507365103\\
495.01	0.00439603571674383\\
496.01	0.0043830655062815\\
497.01	0.00436982904202923\\
498.01	0.00435632081608763\\
499.01	0.00434253521069777\\
500.01	0.0043284664960249\\
501.01	0.00431410882791098\\
502.01	0.00429945624558325\\
503.01	0.00428450266931315\\
504.01	0.0042692418980233\\
505.01	0.00425366760684348\\
506.01	0.00423777334461912\\
507.01	0.004221552531376\\
508.01	0.00420499845574363\\
509.01	0.00418810427234193\\
510.01	0.00417086299913536\\
511.01	0.00415326751475782\\
512.01	0.0041353105558127\\
513.01	0.00411698471414996\\
514.01	0.00409828243412553\\
515.01	0.00407919600984361\\
516.01	0.00405971758238504\\
517.01	0.00403983913702328\\
518.01	0.00401955250043088\\
519.01	0.0039988493378783\\
520.01	0.00397772115042922\\
521.01	0.00395615927213539\\
522.01	0.00393415486723955\\
523.01	0.00391169892739118\\
524.01	0.00388878226888378\\
525.01	0.00386539552992501\\
526.01	0.00384152916794622\\
527.01	0.00381717345696531\\
528.01	0.00379231848501426\\
529.01	0.00376695415164407\\
530.01	0.00374107016552353\\
531.01	0.00371465604214729\\
532.01	0.00368770110167236\\
533.01	0.00366019446690445\\
534.01	0.00363212506145671\\
535.01	0.0036034816081067\\
536.01	0.00357425262738278\\
537.01	0.00354442643640999\\
538.01	0.00351399114805384\\
539.01	0.0034829346704015\\
540.01	0.00345124470662508\\
541.01	0.00341890875527855\\
542.01	0.00338591411108143\\
543.01	0.00335224786625271\\
544.01	0.00331789691246243\\
545.01	0.00328284794347674\\
546.01	0.00324708745858068\\
547.01	0.00321060176687146\\
548.01	0.00317337699252519\\
549.01	0.00313539908115085\\
550.01	0.00309665380735802\\
551.01	0.00305712678367617\\
552.01	0.00301680347097966\\
553.01	0.00297566919058672\\
554.01	0.00293370913821819\\
555.01	0.00289090840001998\\
556.01	0.00284725197087291\\
557.01	0.00280272477523499\\
558.01	0.00275731169078333\\
559.01	0.00271099757514768\\
560.01	0.0026637672960521\\
561.01	0.00261560576520825\\
562.01	0.00256649797633006\\
563.01	0.00251642904766814\\
564.01	0.00246538426948843\\
565.01	0.00241334915694547\\
566.01	0.0023603095088256\\
567.01	0.00230625147265386\\
568.01	0.00225116161667357\\
569.01	0.00219502700921337\\
570.01	0.00213783530595022\\
571.01	0.00207957484555631\\
572.01	0.00202023475417473\\
573.01	0.00195980505909759\\
574.01	0.00189827681191261\\
575.01	0.00183564222122825\\
576.01	0.00177189479486902\\
577.01	0.00170702949113364\\
578.01	0.00164104287830919\\
579.01	0.00157393330110091\\
580.01	0.00150570105194178\\
581.01	0.00143634854423568\\
582.01	0.0013658804834154\\
583.01	0.00129430403018606\\
584.01	0.00122162894839109\\
585.01	0.00114786772747427\\
586.01	0.00107303566637796\\
587.01	0.000997150901747018\\
588.01	0.000920234358289583\\
589.01	0.00084230959280966\\
590.01	0.000763402495449801\\
591.01	0.000683540801648747\\
592.01	0.000602753355717998\\
593.01	0.000521069051135661\\
594.01	0.000438515352848756\\
595.01	0.000355116282076462\\
596.01	0.000270889713084794\\
597.01	0.000185843792620275\\
598.01	9.99722432115345e-05\\
599.01	3.18230442563731e-05\\
599.02	3.12770957175551e-05\\
599.03	3.07343620567866e-05\\
599.04	3.0194875217571e-05\\
599.05	2.96586674565693e-05\\
599.06	2.91257713466771e-05\\
599.07	2.85962197801391e-05\\
599.08	2.80700459716916e-05\\
599.09	2.75472834617447e-05\\
599.1	2.70279661195773e-05\\
599.11	2.65121281465865e-05\\
599.12	2.59998040795448e-05\\
599.13	2.54910287939142e-05\\
599.14	2.49858375071729e-05\\
599.15	2.44842657821827e-05\\
599.16	2.39863495305973e-05\\
599.17	2.34921250162855e-05\\
599.18	2.30016288588035e-05\\
599.19	2.2514898036903e-05\\
599.2	2.20319698920508e-05\\
599.21	2.1552882132023e-05\\
599.22	2.10776728344908e-05\\
599.23	2.06063804506721e-05\\
599.24	2.01390438090109e-05\\
599.25	1.96757021188858e-05\\
599.26	1.92163949743647e-05\\
599.27	1.87611623579855e-05\\
599.28	1.83100446445959e-05\\
599.29	1.78630826051952e-05\\
599.3	1.74203174108517e-05\\
599.31	1.69817906366353e-05\\
599.32	1.65475442655931e-05\\
599.33	1.61176206927693e-05\\
599.34	1.56920627292639e-05\\
599.35	1.52709136063203e-05\\
599.36	1.48542169794742e-05\\
599.37	1.4442016932719e-05\\
599.38	1.40343579827368e-05\\
599.39	1.36312859119591e-05\\
599.4	1.32328499381877e-05\\
599.41	1.28390997671604e-05\\
599.42	1.2450085597351e-05\\
599.43	1.20658581248094e-05\\
599.44	1.16864685480531e-05\\
599.45	1.13119685730065e-05\\
599.46	1.09424104179929e-05\\
599.47	1.05778468187639e-05\\
599.48	1.02183310335888e-05\\
599.49	9.86391684839293e-06\\
599.5	9.51465858194112e-06\\
599.51	9.17061109107116e-06\\
599.52	8.83182977599352e-06\\
599.53	8.49837058562383e-06\\
599.54	8.17029002296576e-06\\
599.55	7.84764515058753e-06\\
599.56	7.53049359608453e-06\\
599.57	7.21889355765649e-06\\
599.58	6.91290380971064e-06\\
599.59	6.61258370851688e-06\\
599.6	6.31799319793756e-06\\
599.61	6.02919281518857e-06\\
599.62	5.74624369668701e-06\\
599.63	5.46920758391981e-06\\
599.64	5.19814682940767e-06\\
599.65	4.93312440269685e-06\\
599.66	4.67420389643411e-06\\
599.67	4.42144953247646e-06\\
599.68	4.17492616808755e-06\\
599.69	3.9346993021671e-06\\
599.7	3.70083508157044e-06\\
599.71	3.47340030746116e-06\\
599.72	3.25246244175723e-06\\
599.73	3.03808961360161e-06\\
599.74	2.83035062593855e-06\\
599.75	2.62931496212288e-06\\
599.76	2.43505279260738e-06\\
599.77	2.24763498169432e-06\\
599.78	2.06713309435815e-06\\
599.79	1.89361940310974e-06\\
599.8	1.72716689498045e-06\\
599.81	1.56784927850782e-06\\
599.82	1.41574099085315e-06\\
599.83	1.27091720493987e-06\\
599.84	1.13345383668563e-06\\
599.85	1.00342755231589e-06\\
599.86	8.8091577571392e-07\\
599.87	7.65996695880136e-07\\
599.88	6.58749274441706e-07\\
599.89	5.59253253243699e-07\\
599.9	4.67589162011367e-07\\
599.91	3.83838326099145e-07\\
599.92	3.0808287429171e-07\\
599.93	2.4040574671258e-07\\
599.94	1.80890702777825e-07\\
599.95	1.2962232925906e-07\\
599.96	8.66860484019516e-08\\
599.97	5.21681261331924e-08\\
599.98	2.61556803542173e-08\\
599.99	8.73668930083393e-09\\
600	0\\
};
\addplot [color=black!20!mycolor21,solid,forget plot]
  table[row sep=crcr]{%
0.01	0.00505041622436337\\
1.01	0.00505041502687961\\
2.01	0.00505041380528635\\
3.01	0.00505041255910172\\
4.01	0.00505041128783399\\
5.01	0.00505040999098198\\
6.01	0.00505040866803445\\
7.01	0.0050504073184702\\
8.01	0.00505040594175756\\
9.01	0.0050504045373543\\
10.01	0.00505040310470765\\
11.01	0.00505040164325379\\
12.01	0.00505040015241783\\
13.01	0.00505039863161348\\
14.01	0.00505039708024294\\
15.01	0.00505039549769652\\
16.01	0.00505039388335266\\
17.01	0.00505039223657732\\
18.01	0.00505039055672402\\
19.01	0.00505038884313369\\
20.01	0.00505038709513408\\
21.01	0.00505038531203984\\
22.01	0.00505038349315189\\
23.01	0.00505038163775768\\
24.01	0.00505037974513043\\
25.01	0.00505037781452916\\
26.01	0.00505037584519808\\
27.01	0.00505037383636694\\
28.01	0.00505037178725006\\
29.01	0.0050503696970464\\
30.01	0.00505036756493908\\
31.01	0.00505036539009538\\
32.01	0.00505036317166596\\
33.01	0.00505036090878514\\
34.01	0.00505035860056998\\
35.01	0.00505035624612024\\
36.01	0.00505035384451818\\
37.01	0.00505035139482785\\
38.01	0.0050503488960952\\
39.01	0.00505034634734725\\
40.01	0.00505034374759241\\
41.01	0.00505034109581914\\
42.01	0.00505033839099619\\
43.01	0.0050503356320726\\
44.01	0.00505033281797631\\
45.01	0.00505032994761456\\
46.01	0.00505032701987323\\
47.01	0.0050503240336164\\
48.01	0.00505032098768594\\
49.01	0.00505031788090126\\
50.01	0.00505031471205845\\
51.01	0.00505031147993047\\
52.01	0.00505030818326621\\
53.01	0.00505030482078999\\
54.01	0.0050503013912013\\
55.01	0.00505029789317477\\
56.01	0.00505029432535876\\
57.01	0.00505029068637558\\
58.01	0.00505028697482046\\
59.01	0.00505028318926152\\
60.01	0.00505027932823905\\
61.01	0.005050275390265\\
62.01	0.00505027137382239\\
63.01	0.00505026727736469\\
64.01	0.00505026309931556\\
65.01	0.00505025883806804\\
66.01	0.00505025449198377\\
67.01	0.00505025005939313\\
68.01	0.00505024553859374\\
69.01	0.0050502409278504\\
70.01	0.00505023622539427\\
71.01	0.00505023142942263\\
72.01	0.00505022653809739\\
73.01	0.00505022154954516\\
74.01	0.00505021646185647\\
75.01	0.00505021127308494\\
76.01	0.00505020598124666\\
77.01	0.00505020058431944\\
78.01	0.00505019508024193\\
79.01	0.00505018946691318\\
80.01	0.00505018374219198\\
81.01	0.00505017790389557\\
82.01	0.00505017194979935\\
83.01	0.00505016587763594\\
84.01	0.00505015968509456\\
85.01	0.00505015336981967\\
86.01	0.00505014692941063\\
87.01	0.00505014036142088\\
88.01	0.00505013366335681\\
89.01	0.00505012683267706\\
90.01	0.00505011986679147\\
91.01	0.00505011276306055\\
92.01	0.00505010551879388\\
93.01	0.00505009813124976\\
94.01	0.00505009059763433\\
95.01	0.0050500829150999\\
96.01	0.00505007508074471\\
97.01	0.00505006709161143\\
98.01	0.00505005894468656\\
99.01	0.00505005063689899\\
100.01	0.00505004216511883\\
101.01	0.00505003352615715\\
102.01	0.00505002471676396\\
103.01	0.00505001573362758\\
104.01	0.00505000657337335\\
105.01	0.00504999723256242\\
106.01	0.00504998770769082\\
107.01	0.00504997799518791\\
108.01	0.00504996809141548\\
109.01	0.00504995799266634\\
110.01	0.00504994769516324\\
111.01	0.00504993719505728\\
112.01	0.00504992648842675\\
113.01	0.00504991557127588\\
114.01	0.00504990443953338\\
115.01	0.00504989308905139\\
116.01	0.00504988151560328\\
117.01	0.0050498697148834\\
118.01	0.00504985768250443\\
119.01	0.00504984541399691\\
120.01	0.00504983290480681\\
121.01	0.00504982015029495\\
122.01	0.00504980714573495\\
123.01	0.00504979388631146\\
124.01	0.00504978036711917\\
125.01	0.0050497665831604\\
126.01	0.0050497525293443\\
127.01	0.00504973820048466\\
128.01	0.00504972359129796\\
129.01	0.00504970869640226\\
130.01	0.00504969351031514\\
131.01	0.00504967802745166\\
132.01	0.00504966224212308\\
133.01	0.00504964614853445\\
134.01	0.00504962974078308\\
135.01	0.00504961301285629\\
136.01	0.0050495959586299\\
137.01	0.00504957857186603\\
138.01	0.00504956084621088\\
139.01	0.00504954277519308\\
140.01	0.00504952435222127\\
141.01	0.00504950557058214\\
142.01	0.00504948642343829\\
143.01	0.00504946690382603\\
144.01	0.00504944700465291\\
145.01	0.00504942671869607\\
146.01	0.00504940603859908\\
147.01	0.00504938495687064\\
148.01	0.00504936346588136\\
149.01	0.00504934155786169\\
150.01	0.00504931922489933\\
151.01	0.00504929645893711\\
152.01	0.00504927325177016\\
153.01	0.00504924959504327\\
154.01	0.00504922548024874\\
155.01	0.00504920089872304\\
156.01	0.00504917584164474\\
157.01	0.00504915030003159\\
158.01	0.00504912426473755\\
159.01	0.00504909772645034\\
160.01	0.0050490706756881\\
161.01	0.00504904310279709\\
162.01	0.00504901499794829\\
163.01	0.00504898635113455\\
164.01	0.00504895715216749\\
165.01	0.00504892739067457\\
166.01	0.0050488970560957\\
167.01	0.00504886613768058\\
168.01	0.00504883462448481\\
169.01	0.00504880250536705\\
170.01	0.00504876976898552\\
171.01	0.0050487364037947\\
172.01	0.00504870239804215\\
173.01	0.00504866773976427\\
174.01	0.00504863241678364\\
175.01	0.00504859641670479\\
176.01	0.00504855972691134\\
177.01	0.00504852233456132\\
178.01	0.00504848422658404\\
179.01	0.00504844538967644\\
180.01	0.00504840581029883\\
181.01	0.00504836547467099\\
182.01	0.00504832436876866\\
183.01	0.00504828247831899\\
184.01	0.00504823978879698\\
185.01	0.0050481962854207\\
186.01	0.00504815195314776\\
187.01	0.00504810677667052\\
188.01	0.00504806074041221\\
189.01	0.00504801382852224\\
190.01	0.00504796602487202\\
191.01	0.00504791731305022\\
192.01	0.00504786767635818\\
193.01	0.00504781709780566\\
194.01	0.00504776556010587\\
195.01	0.00504771304567071\\
196.01	0.00504765953660593\\
197.01	0.00504760501470664\\
198.01	0.00504754946145155\\
199.01	0.005047492857999\\
200.01	0.0050474351851811\\
201.01	0.00504737642349863\\
202.01	0.00504731655311634\\
203.01	0.00504725555385704\\
204.01	0.00504719340519661\\
205.01	0.00504713008625858\\
206.01	0.00504706557580845\\
207.01	0.00504699985224826\\
208.01	0.00504693289361077\\
209.01	0.00504686467755405\\
210.01	0.00504679518135547\\
211.01	0.00504672438190567\\
212.01	0.00504665225570302\\
213.01	0.00504657877884728\\
214.01	0.00504650392703367\\
215.01	0.00504642767554666\\
216.01	0.00504634999925358\\
217.01	0.00504627087259842\\
218.01	0.00504619026959572\\
219.01	0.00504610816382335\\
220.01	0.00504602452841662\\
221.01	0.00504593933606141\\
222.01	0.00504585255898741\\
223.01	0.00504576416896103\\
224.01	0.00504567413727911\\
225.01	0.00504558243476158\\
226.01	0.00504548903174409\\
227.01	0.00504539389807151\\
228.01	0.00504529700309017\\
229.01	0.00504519831564083\\
230.01	0.00504509780405106\\
231.01	0.00504499543612776\\
232.01	0.00504489117914962\\
233.01	0.00504478499985944\\
234.01	0.0050446768644564\\
235.01	0.00504456673858798\\
236.01	0.00504445458734198\\
237.01	0.00504434037523866\\
238.01	0.00504422406622267\\
239.01	0.00504410562365426\\
240.01	0.0050439850103012\\
241.01	0.00504386218833065\\
242.01	0.00504373711929989\\
243.01	0.00504360976414827\\
244.01	0.00504348008318793\\
245.01	0.00504334803609521\\
246.01	0.00504321358190158\\
247.01	0.00504307667898427\\
248.01	0.00504293728505743\\
249.01	0.00504279535716248\\
250.01	0.00504265085165886\\
251.01	0.00504250372421412\\
252.01	0.00504235392979442\\
253.01	0.00504220142265456\\
254.01	0.00504204615632818\\
255.01	0.00504188808361737\\
256.01	0.00504172715658271\\
257.01	0.00504156332653268\\
258.01	0.00504139654401304\\
259.01	0.00504122675879636\\
260.01	0.00504105391987104\\
261.01	0.00504087797543049\\
262.01	0.00504069887286166\\
263.01	0.00504051655873399\\
264.01	0.00504033097878772\\
265.01	0.00504014207792223\\
266.01	0.00503994980018425\\
267.01	0.00503975408875577\\
268.01	0.00503955488594175\\
269.01	0.00503935213315789\\
270.01	0.00503914577091756\\
271.01	0.00503893573881922\\
272.01	0.00503872197553281\\
273.01	0.00503850441878714\\
274.01	0.00503828300535608\\
275.01	0.00503805767104399\\
276.01	0.0050378283506724\\
277.01	0.00503759497806512\\
278.01	0.0050373574860338\\
279.01	0.00503711580636279\\
280.01	0.0050368698697938\\
281.01	0.00503661960601027\\
282.01	0.00503636494362115\\
283.01	0.00503610581014482\\
284.01	0.00503584213199201\\
285.01	0.00503557383444909\\
286.01	0.00503530084165991\\
287.01	0.00503502307660814\\
288.01	0.00503474046109883\\
289.01	0.00503445291573879\\
290.01	0.00503416035991835\\
291.01	0.00503386271178994\\
292.01	0.00503355988824878\\
293.01	0.00503325180491076\\
294.01	0.00503293837609126\\
295.01	0.00503261951478264\\
296.01	0.00503229513263092\\
297.01	0.00503196513991238\\
298.01	0.0050316294455087\\
299.01	0.0050312879568819\\
300.01	0.00503094058004804\\
301.01	0.00503058721955057\\
302.01	0.00503022777843219\\
303.01	0.00502986215820591\\
304.01	0.00502949025882556\\
305.01	0.00502911197865478\\
306.01	0.00502872721443513\\
307.01	0.00502833586125309\\
308.01	0.00502793781250557\\
309.01	0.00502753295986485\\
310.01	0.00502712119324115\\
311.01	0.00502670240074511\\
312.01	0.0050262764686474\\
313.01	0.00502584328133831\\
314.01	0.00502540272128466\\
315.01	0.00502495466898543\\
316.01	0.00502449900292578\\
317.01	0.00502403559952918\\
318.01	0.00502356433310781\\
319.01	0.00502308507581059\\
320.01	0.00502259769756933\\
321.01	0.00502210206604349\\
322.01	0.00502159804656149\\
323.01	0.00502108550206058\\
324.01	0.00502056429302427\\
325.01	0.00502003427741649\\
326.01	0.00501949531061407\\
327.01	0.00501894724533653\\
328.01	0.00501838993157195\\
329.01	0.0050178232165013\\
330.01	0.005017246944419\\
331.01	0.00501666095665098\\
332.01	0.00501606509146876\\
333.01	0.00501545918400077\\
334.01	0.00501484306614038\\
335.01	0.00501421656644987\\
336.01	0.00501357951006183\\
337.01	0.00501293171857526\\
338.01	0.00501227300994974\\
339.01	0.00501160319839449\\
340.01	0.00501092209425414\\
341.01	0.00501022950389015\\
342.01	0.00500952522955797\\
343.01	0.00500880906928109\\
344.01	0.00500808081671953\\
345.01	0.00500734026103516\\
346.01	0.00500658718675241\\
347.01	0.00500582137361472\\
348.01	0.00500504259643691\\
349.01	0.00500425062495316\\
350.01	0.00500344522366115\\
351.01	0.00500262615166124\\
352.01	0.00500179316249306\\
353.01	0.00500094600396683\\
354.01	0.0050000844179919\\
355.01	0.00499920814040097\\
356.01	0.00499831690077129\\
357.01	0.00499741042224258\\
358.01	0.00499648842133201\\
359.01	0.0049955506077466\\
360.01	0.00499459668419335\\
361.01	0.00499362634618713\\
362.01	0.00499263928185728\\
363.01	0.00499163517175377\\
364.01	0.00499061368865077\\
365.01	0.00498957449735275\\
366.01	0.00498851725449865\\
367.01	0.00498744160836883\\
368.01	0.00498634719869235\\
369.01	0.00498523365645737\\
370.01	0.00498410060372413\\
371.01	0.00498294765344128\\
372.01	0.004981774409267\\
373.01	0.00498058046539411\\
374.01	0.00497936540638164\\
375.01	0.00497812880699148\\
376.01	0.00497687023203234\\
377.01	0.00497558923621056\\
378.01	0.00497428536398801\\
379.01	0.00497295814944752\\
380.01	0.00497160711616688\\
381.01	0.00497023177709904\\
382.01	0.00496883163446029\\
383.01	0.00496740617962532\\
384.01	0.00496595489302728\\
385.01	0.00496447724406343\\
386.01	0.00496297269100318\\
387.01	0.00496144068089814\\
388.01	0.00495988064949051\\
389.01	0.0049582920211187\\
390.01	0.00495667420861591\\
391.01	0.00495502661319822\\
392.01	0.00495334862434032\\
393.01	0.00495163961963088\\
394.01	0.0049498989646079\\
395.01	0.00494812601256582\\
396.01	0.00494632010433079\\
397.01	0.00494448056800078\\
398.01	0.00494260671864545\\
399.01	0.00494069785796225\\
400.01	0.00493875327388629\\
401.01	0.00493677224015327\\
402.01	0.00493475401581408\\
403.01	0.00493269784470363\\
404.01	0.00493060295486799\\
405.01	0.00492846855795484\\
406.01	0.00492629384857654\\
407.01	0.00492407800365421\\
408.01	0.00492182018175633\\
409.01	0.00491951952244241\\
410.01	0.00491717514562545\\
411.01	0.00491478615096051\\
412.01	0.0049123516172678\\
413.01	0.00490987060198812\\
414.01	0.0049073421406633\\
415.01	0.00490476524642877\\
416.01	0.00490213890950422\\
417.01	0.00489946209668025\\
418.01	0.00489673375080099\\
419.01	0.00489395279024194\\
420.01	0.00489111810838552\\
421.01	0.00488822857309285\\
422.01	0.0048852830261742\\
423.01	0.00488228028285686\\
424.01	0.00487921913125238\\
425.01	0.00487609833182316\\
426.01	0.0048729166168476\\
427.01	0.00486967268988787\\
428.01	0.00486636522525517\\
429.01	0.00486299286747915\\
430.01	0.00485955423077612\\
431.01	0.00485604789852002\\
432.01	0.00485247242271422\\
433.01	0.00484882632346511\\
434.01	0.00484510808845668\\
435.01	0.0048413161724264\\
436.01	0.00483744899664086\\
437.01	0.00483350494837175\\
438.01	0.00482948238037028\\
439.01	0.00482537961033915\\
440.01	0.00482119492040131\\
441.01	0.00481692655656285\\
442.01	0.00481257272816965\\
443.01	0.00480813160735409\\
444.01	0.00480360132847085\\
445.01	0.00479897998751858\\
446.01	0.00479426564154536\\
447.01	0.00478945630803472\\
448.01	0.00478454996426917\\
449.01	0.00477954454666905\\
450.01	0.00477443795010257\\
451.01	0.00476922802716536\\
452.01	0.00476391258742675\\
453.01	0.0047584893966391\\
454.01	0.00475295617590996\\
455.01	0.0047473106008352\\
456.01	0.00474155030059124\\
457.01	0.00473567285698744\\
458.01	0.00472967580347889\\
459.01	0.00472355662414201\\
460.01	0.00471731275261322\\
461.01	0.00471094157099698\\
462.01	0.00470444040874411\\
463.01	0.00469780654150765\\
464.01	0.00469103718997992\\
465.01	0.00468412951871555\\
466.01	0.00467708063494494\\
467.01	0.00466988758738365\\
468.01	0.00466254736503701\\
469.01	0.00465505689600287\\
470.01	0.0046474130462699\\
471.01	0.00463961261850789\\
472.01	0.00463165235084363\\
473.01	0.00462352891561738\\
474.01	0.00461523891810933\\
475.01	0.00460677889523471\\
476.01	0.00459814531420051\\
477.01	0.0045893345711252\\
478.01	0.00458034298961881\\
479.01	0.00457116681932295\\
480.01	0.00456180223441107\\
481.01	0.0045522453320476\\
482.01	0.00454249213080608\\
483.01	0.00453253856904761\\
484.01	0.00452238050325833\\
485.01	0.00451201370634854\\
486.01	0.00450143386591313\\
487.01	0.00449063658245445\\
488.01	0.0044796173675692\\
489.01	0.00446837164210053\\
490.01	0.00445689473425607\\
491.01	0.00444518187769261\\
492.01	0.00443322820956902\\
493.01	0.00442102876856632\\
494.01	0.00440857849287668\\
495.01	0.00439587221815883\\
496.01	0.00438290467546104\\
497.01	0.00436967048910916\\
498.01	0.00435616417455901\\
499.01	0.0043423801362124\\
500.01	0.00432831266519476\\
501.01	0.0043139559370941\\
502.01	0.00429930400966176\\
503.01	0.00428435082047391\\
504.01	0.0042690901845561\\
505.01	0.00425351579197043\\
506.01	0.00423762120536703\\
507.01	0.00422139985750059\\
508.01	0.00420484504871337\\
509.01	0.00418794994438539\\
510.01	0.00417070757235356\\
511.01	0.00415311082030088\\
512.01	0.00413515243311747\\
513.01	0.00411682501023545\\
514.01	0.00409812100293835\\
515.01	0.00407903271164887\\
516.01	0.00405955228319665\\
517.01	0.00403967170806943\\
518.01	0.00401938281765047\\
519.01	0.00399867728144732\\
520.01	0.00397754660431566\\
521.01	0.00395598212368432\\
522.01	0.00393397500678653\\
523.01	0.00391151624790538\\
524.01	0.0038885966656406\\
525.01	0.00386520690020487\\
526.01	0.0038413374107603\\
527.01	0.00381697847280496\\
528.01	0.0037921201756219\\
529.01	0.00376675241980425\\
530.01	0.00374086491487127\\
531.01	0.00371444717699319\\
532.01	0.00368748852684237\\
533.01	0.00365997808759294\\
534.01	0.00363190478309231\\
535.01	0.00360325733623105\\
536.01	0.00357402426753979\\
537.01	0.0035441938940475\\
538.01	0.00351375432843491\\
539.01	0.00348269347852603\\
540.01	0.00345099904716186\\
541.01	0.00341865853250594\\
542.01	0.00338565922883782\\
543.01	0.00335198822789567\\
544.01	0.00331763242083754\\
545.01	0.00328257850089516\\
546.01	0.00324681296680706\\
547.01	0.00321032212712155\\
548.01	0.00317309210547439\\
549.01	0.00313510884695503\\
550.01	0.00309635812568573\\
551.01	0.00305682555375531\\
552.01	0.00301649659165863\\
553.01	0.00297535656041113\\
554.01	0.0029333906555258\\
555.01	0.00289058396305458\\
556.01	0.00284692147791951\\
557.01	0.00280238812477814\\
558.01	0.00275696878169113\\
559.01	0.00271064830688349\\
560.01	0.00266341156891711\\
561.01	0.00261524348061712\\
562.01	0.00256612903712336\\
563.01	0.00251605335846367\\
564.01	0.00246500173707516\\
565.01	0.00241295969072369\\
566.01	0.00235991302129535\\
567.01	0.0023058478799553\\
568.01	0.00225075083918186\\
569.01	0.00219460897218979\\
570.01	0.0021374099402512\\
571.01	0.00207914208840051\\
572.01	0.00201979454996673\\
573.01	0.00195935736030474\\
574.01	0.00189782157998909\\
575.01	0.00183517942757606\\
576.01	0.00177142442182105\\
577.01	0.00170655153293824\\
578.01	0.00164055734208577\\
579.01	0.0015734402077281\\
580.01	0.00150520043682321\\
581.01	0.00143584045787427\\
582.01	0.00136536499170297\\
583.01	0.00129378121428826\\
584.01	0.00122109890407295\\
585.01	0.00114733056366669\\
586.01	0.0010724915027304\\
587.01	0.000996599864842329\\
588.01	0.000919676576107574\\
589.01	0.000841745186917533\\
590.01	0.000762831570259378\\
591.01	0.00068296342990758\\
592.01	0.00060216955918722\\
593.01	0.000520478775138188\\
594.01	0.000437918433036224\\
595.01	0.000354512401346254\\
596.01	0.000270278346057853\\
597.01	0.000185224134443608\\
598.01	9.93507029036552e-05\\
599.01	3.18230442563749e-05\\
599.02	3.12770957175551e-05\\
599.03	3.07343620567849e-05\\
599.04	3.01948752175693e-05\\
599.05	2.96586674565693e-05\\
599.06	2.91257713466771e-05\\
599.07	2.85962197801373e-05\\
599.08	2.80700459716933e-05\\
599.09	2.75472834617447e-05\\
599.1	2.70279661195791e-05\\
599.11	2.65121281465847e-05\\
599.12	2.59998040795448e-05\\
599.13	2.54910287939124e-05\\
599.14	2.49858375071695e-05\\
599.15	2.4484265782181e-05\\
599.16	2.39863495305973e-05\\
599.17	2.34921250162837e-05\\
599.18	2.30016288588035e-05\\
599.19	2.25148980369013e-05\\
599.2	2.20319698920508e-05\\
599.21	2.1552882132023e-05\\
599.22	2.10776728344891e-05\\
599.23	2.06063804506721e-05\\
599.24	2.01390438090109e-05\\
599.25	1.96757021188876e-05\\
599.26	1.92163949743647e-05\\
599.27	1.87611623579872e-05\\
599.28	1.83100446445941e-05\\
599.29	1.78630826051934e-05\\
599.3	1.74203174108517e-05\\
599.31	1.69817906366353e-05\\
599.32	1.65475442655914e-05\\
599.33	1.61176206927693e-05\\
599.34	1.56920627292622e-05\\
599.35	1.52709136063186e-05\\
599.36	1.48542169794725e-05\\
599.37	1.4442016932719e-05\\
599.38	1.40343579827368e-05\\
599.39	1.36312859119591e-05\\
599.4	1.32328499381877e-05\\
599.41	1.28390997671621e-05\\
599.42	1.2450085597351e-05\\
599.43	1.20658581248094e-05\\
599.44	1.16864685480531e-05\\
599.45	1.13119685730082e-05\\
599.46	1.09424104179929e-05\\
599.47	1.05778468187639e-05\\
599.48	1.02183310335888e-05\\
599.49	9.86391684839466e-06\\
599.5	9.51465858194112e-06\\
599.51	9.17061109107116e-06\\
599.52	8.83182977599352e-06\\
599.53	8.4983705856221e-06\\
599.54	8.1702900229675e-06\\
599.55	7.84764515058579e-06\\
599.56	7.53049359608279e-06\\
599.57	7.21889355765649e-06\\
599.58	6.91290380971064e-06\\
599.59	6.61258370851861e-06\\
599.6	6.31799319793756e-06\\
599.61	6.02919281519031e-06\\
599.62	5.74624369668701e-06\\
599.63	5.46920758391981e-06\\
599.64	5.19814682940593e-06\\
599.65	4.93312440269685e-06\\
599.66	4.67420389643411e-06\\
599.67	4.42144953247646e-06\\
599.68	4.17492616808582e-06\\
599.69	3.93469930216536e-06\\
599.7	3.70083508156871e-06\\
599.71	3.47340030746116e-06\\
599.72	3.25246244175549e-06\\
599.73	3.03808961360161e-06\\
599.74	2.83035062593855e-06\\
599.75	2.62931496212288e-06\\
599.76	2.43505279260738e-06\\
599.77	2.24763498169606e-06\\
599.78	2.06713309435641e-06\\
599.79	1.89361940310974e-06\\
599.8	1.72716689497872e-06\\
599.81	1.56784927850956e-06\\
599.82	1.41574099085488e-06\\
599.83	1.27091720493813e-06\\
599.84	1.13345383668563e-06\\
599.85	1.00342755231589e-06\\
599.86	8.8091577571392e-07\\
599.87	7.65996695881871e-07\\
599.88	6.58749274441706e-07\\
599.89	5.59253253243699e-07\\
599.9	4.67589162013102e-07\\
599.91	3.83838326099145e-07\\
599.92	3.08082874293444e-07\\
599.93	2.40405746710845e-07\\
599.94	1.8089070277609e-07\\
599.95	1.2962232925906e-07\\
599.96	8.66860484019516e-08\\
599.97	5.21681261349272e-08\\
599.98	2.61556803542173e-08\\
599.99	8.73668930083393e-09\\
600	0\\
};
\addplot [color=black!50!mycolor20,solid,forget plot]
  table[row sep=crcr]{%
0.01	0.00503757637221215\\
1.01	0.00503757530562437\\
2.01	0.00503757421782917\\
3.01	0.00503757310840824\\
4.01	0.00503757197693535\\
5.01	0.00503757082297613\\
6.01	0.00503756964608718\\
7.01	0.00503756844581672\\
8.01	0.00503756722170427\\
9.01	0.00503756597328036\\
10.01	0.00503756470006631\\
11.01	0.00503756340157425\\
12.01	0.00503756207730662\\
13.01	0.0050375607267563\\
14.01	0.00503755934940636\\
15.01	0.00503755794472965\\
16.01	0.0050375565121889\\
17.01	0.00503755505123641\\
18.01	0.00503755356131391\\
19.01	0.00503755204185207\\
20.01	0.0050375504922708\\
21.01	0.00503754891197831\\
22.01	0.00503754730037193\\
23.01	0.00503754565683676\\
24.01	0.00503754398074654\\
25.01	0.00503754227146226\\
26.01	0.00503754052833307\\
27.01	0.00503753875069508\\
28.01	0.00503753693787184\\
29.01	0.00503753508917366\\
30.01	0.0050375332038976\\
31.01	0.00503753128132708\\
32.01	0.00503752932073171\\
33.01	0.00503752732136696\\
34.01	0.00503752528247392\\
35.01	0.00503752320327906\\
36.01	0.00503752108299393\\
37.01	0.00503751892081476\\
38.01	0.00503751671592218\\
39.01	0.0050375144674814\\
40.01	0.00503751217464121\\
41.01	0.00503750983653403\\
42.01	0.00503750745227585\\
43.01	0.00503750502096533\\
44.01	0.00503750254168386\\
45.01	0.00503750001349514\\
46.01	0.00503749743544505\\
47.01	0.00503749480656079\\
48.01	0.00503749212585129\\
49.01	0.00503748939230592\\
50.01	0.00503748660489518\\
51.01	0.00503748376256947\\
52.01	0.00503748086425903\\
53.01	0.00503747790887379\\
54.01	0.00503747489530284\\
55.01	0.00503747182241355\\
56.01	0.00503746868905205\\
57.01	0.00503746549404199\\
58.01	0.00503746223618486\\
59.01	0.00503745891425895\\
60.01	0.00503745552701947\\
61.01	0.0050374520731974\\
62.01	0.00503744855150008\\
63.01	0.00503744496060934\\
64.01	0.00503744129918257\\
65.01	0.00503743756585114\\
66.01	0.00503743375922066\\
67.01	0.00503742987786966\\
68.01	0.00503742592034982\\
69.01	0.00503742188518542\\
70.01	0.00503741777087245\\
71.01	0.00503741357587815\\
72.01	0.00503740929864081\\
73.01	0.00503740493756893\\
74.01	0.00503740049104078\\
75.01	0.00503739595740386\\
76.01	0.00503739133497405\\
77.01	0.00503738662203546\\
78.01	0.00503738181683963\\
79.01	0.00503737691760501\\
80.01	0.00503737192251585\\
81.01	0.0050373668297227\\
82.01	0.00503736163734053\\
83.01	0.00503735634344888\\
84.01	0.00503735094609075\\
85.01	0.00503734544327243\\
86.01	0.00503733983296231\\
87.01	0.00503733411309024\\
88.01	0.00503732828154737\\
89.01	0.00503732233618485\\
90.01	0.00503731627481322\\
91.01	0.00503731009520176\\
92.01	0.00503730379507781\\
93.01	0.00503729737212582\\
94.01	0.00503729082398685\\
95.01	0.00503728414825724\\
96.01	0.00503727734248828\\
97.01	0.00503727040418533\\
98.01	0.00503726333080657\\
99.01	0.00503725611976281\\
100.01	0.00503724876841621\\
101.01	0.00503724127407932\\
102.01	0.00503723363401428\\
103.01	0.00503722584543207\\
104.01	0.00503721790549134\\
105.01	0.00503720981129765\\
106.01	0.00503720155990228\\
107.01	0.00503719314830158\\
108.01	0.0050371845734357\\
109.01	0.00503717583218735\\
110.01	0.00503716692138161\\
111.01	0.00503715783778399\\
112.01	0.00503714857809979\\
113.01	0.0050371391389728\\
114.01	0.00503712951698459\\
115.01	0.00503711970865275\\
116.01	0.0050371097104304\\
117.01	0.00503709951870469\\
118.01	0.00503708912979552\\
119.01	0.00503707853995445\\
120.01	0.00503706774536366\\
121.01	0.00503705674213456\\
122.01	0.00503704552630625\\
123.01	0.00503703409384467\\
124.01	0.00503702244064106\\
125.01	0.0050370105625105\\
126.01	0.00503699845519107\\
127.01	0.00503698611434173\\
128.01	0.00503697353554149\\
129.01	0.00503696071428767\\
130.01	0.00503694764599466\\
131.01	0.0050369343259925\\
132.01	0.00503692074952483\\
133.01	0.00503690691174793\\
134.01	0.00503689280772931\\
135.01	0.00503687843244535\\
136.01	0.00503686378078044\\
137.01	0.00503684884752489\\
138.01	0.00503683362737365\\
139.01	0.00503681811492427\\
140.01	0.0050368023046754\\
141.01	0.00503678619102482\\
142.01	0.0050367697682679\\
143.01	0.00503675303059582\\
144.01	0.00503673597209362\\
145.01	0.00503671858673818\\
146.01	0.0050367008683967\\
147.01	0.00503668281082442\\
148.01	0.0050366644076628\\
149.01	0.00503664565243758\\
150.01	0.00503662653855665\\
151.01	0.00503660705930822\\
152.01	0.0050365872078584\\
153.01	0.00503656697724912\\
154.01	0.00503654636039616\\
155.01	0.00503652535008687\\
156.01	0.00503650393897796\\
157.01	0.00503648211959321\\
158.01	0.00503645988432092\\
159.01	0.00503643722541213\\
160.01	0.00503641413497769\\
161.01	0.00503639060498585\\
162.01	0.00503636662726021\\
163.01	0.00503634219347696\\
164.01	0.00503631729516229\\
165.01	0.00503629192368973\\
166.01	0.00503626607027794\\
167.01	0.00503623972598729\\
168.01	0.00503621288171818\\
169.01	0.00503618552820714\\
170.01	0.00503615765602495\\
171.01	0.00503612925557314\\
172.01	0.00503610031708122\\
173.01	0.00503607083060432\\
174.01	0.00503604078601939\\
175.01	0.00503601017302279\\
176.01	0.00503597898112645\\
177.01	0.00503594719965542\\
178.01	0.00503591481774434\\
179.01	0.00503588182433441\\
180.01	0.00503584820816981\\
181.01	0.00503581395779453\\
182.01	0.00503577906154872\\
183.01	0.0050357435075658\\
184.01	0.00503570728376812\\
185.01	0.00503567037786405\\
186.01	0.00503563277734397\\
187.01	0.00503559446947682\\
188.01	0.00503555544130603\\
189.01	0.00503551567964594\\
190.01	0.0050354751710778\\
191.01	0.00503543390194573\\
192.01	0.00503539185835323\\
193.01	0.0050353490261584\\
194.01	0.00503530539097027\\
195.01	0.00503526093814411\\
196.01	0.00503521565277792\\
197.01	0.00503516951970722\\
198.01	0.00503512252350134\\
199.01	0.00503507464845814\\
200.01	0.00503502587860027\\
201.01	0.00503497619767009\\
202.01	0.00503492558912484\\
203.01	0.00503487403613193\\
204.01	0.00503482152156468\\
205.01	0.00503476802799605\\
206.01	0.00503471353769469\\
207.01	0.00503465803261952\\
208.01	0.00503460149441431\\
209.01	0.00503454390440248\\
210.01	0.00503448524358185\\
211.01	0.00503442549261899\\
212.01	0.00503436463184354\\
213.01	0.00503430264124295\\
214.01	0.00503423950045613\\
215.01	0.00503417518876822\\
216.01	0.00503410968510407\\
217.01	0.00503404296802266\\
218.01	0.00503397501571035\\
219.01	0.00503390580597508\\
220.01	0.00503383531623984\\
221.01	0.00503376352353603\\
222.01	0.00503369040449694\\
223.01	0.00503361593535149\\
224.01	0.00503354009191626\\
225.01	0.00503346284958959\\
226.01	0.00503338418334417\\
227.01	0.00503330406771952\\
228.01	0.00503322247681503\\
229.01	0.00503313938428232\\
230.01	0.00503305476331757\\
231.01	0.00503296858665383\\
232.01	0.00503288082655331\\
233.01	0.00503279145479899\\
234.01	0.00503270044268673\\
235.01	0.00503260776101671\\
236.01	0.00503251338008532\\
237.01	0.00503241726967597\\
238.01	0.00503231939905068\\
239.01	0.00503221973694097\\
240.01	0.00503211825153886\\
241.01	0.00503201491048714\\
242.01	0.00503190968087044\\
243.01	0.00503180252920504\\
244.01	0.00503169342142931\\
245.01	0.00503158232289353\\
246.01	0.00503146919834986\\
247.01	0.00503135401194118\\
248.01	0.0050312367271912\\
249.01	0.00503111730699308\\
250.01	0.00503099571359866\\
251.01	0.00503087190860672\\
252.01	0.00503074585295193\\
253.01	0.00503061750689229\\
254.01	0.00503048682999773\\
255.01	0.00503035378113773\\
256.01	0.00503021831846828\\
257.01	0.00503008039941939\\
258.01	0.00502993998068197\\
259.01	0.00502979701819428\\
260.01	0.0050296514671284\\
261.01	0.00502950328187615\\
262.01	0.00502935241603455\\
263.01	0.00502919882239165\\
264.01	0.00502904245291109\\
265.01	0.00502888325871732\\
266.01	0.00502872119007941\\
267.01	0.00502855619639554\\
268.01	0.00502838822617612\\
269.01	0.00502821722702749\\
270.01	0.00502804314563448\\
271.01	0.00502786592774291\\
272.01	0.00502768551814172\\
273.01	0.00502750186064476\\
274.01	0.00502731489807084\\
275.01	0.00502712457222613\\
276.01	0.00502693082388276\\
277.01	0.00502673359275983\\
278.01	0.0050265328175017\\
279.01	0.00502632843565699\\
280.01	0.0050261203836568\\
281.01	0.00502590859679258\\
282.01	0.00502569300919249\\
283.01	0.00502547355379831\\
284.01	0.00502525016234141\\
285.01	0.00502502276531742\\
286.01	0.00502479129196121\\
287.01	0.00502455567022076\\
288.01	0.00502431582673\\
289.01	0.00502407168678189\\
290.01	0.00502382317429955\\
291.01	0.00502357021180749\\
292.01	0.00502331272040208\\
293.01	0.0050230506197208\\
294.01	0.0050227838279105\\
295.01	0.00502251226159539\\
296.01	0.00502223583584404\\
297.01	0.00502195446413479\\
298.01	0.00502166805832143\\
299.01	0.0050213765285966\\
300.01	0.00502107978345517\\
301.01	0.00502077772965604\\
302.01	0.0050204702721828\\
303.01	0.0050201573142045\\
304.01	0.00501983875703335\\
305.01	0.00501951450008285\\
306.01	0.00501918444082401\\
307.01	0.00501884847474066\\
308.01	0.005018506495283\\
309.01	0.00501815839382062\\
310.01	0.00501780405959352\\
311.01	0.00501744337966192\\
312.01	0.00501707623885547\\
313.01	0.00501670251971957\\
314.01	0.00501632210246132\\
315.01	0.00501593486489365\\
316.01	0.00501554068237779\\
317.01	0.00501513942776458\\
318.01	0.00501473097133332\\
319.01	0.00501431518072992\\
320.01	0.00501389192090298\\
321.01	0.00501346105403746\\
322.01	0.00501302243948797\\
323.01	0.00501257593370905\\
324.01	0.00501212139018412\\
325.01	0.00501165865935276\\
326.01	0.00501118758853553\\
327.01	0.00501070802185731\\
328.01	0.00501021980016853\\
329.01	0.00500972276096439\\
330.01	0.0050092167383022\\
331.01	0.00500870156271605\\
332.01	0.00500817706113033\\
333.01	0.00500764305677077\\
334.01	0.00500709936907302\\
335.01	0.0050065458135899\\
336.01	0.00500598220189525\\
337.01	0.00500540834148718\\
338.01	0.0050048240356881\\
339.01	0.00500422908354254\\
340.01	0.00500362327971341\\
341.01	0.00500300641437589\\
342.01	0.0050023782731091\\
343.01	0.00500173863678528\\
344.01	0.00500108728145748\\
345.01	0.00500042397824491\\
346.01	0.00499974849321595\\
347.01	0.00499906058726931\\
348.01	0.00499836001601358\\
349.01	0.0049976465296438\\
350.01	0.00499691987281695\\
351.01	0.00499617978452547\\
352.01	0.00499542599796851\\
353.01	0.00499465824042179\\
354.01	0.00499387623310557\\
355.01	0.00499307969105064\\
356.01	0.00499226832296384\\
357.01	0.0049914418310903\\
358.01	0.00499059991107576\\
359.01	0.00498974225182652\\
360.01	0.00498886853536857\\
361.01	0.00498797843670493\\
362.01	0.00498707162367234\\
363.01	0.00498614775679577\\
364.01	0.00498520648914337\\
365.01	0.00498424746617824\\
366.01	0.00498327032561077\\
367.01	0.00498227469724855\\
368.01	0.00498126020284642\\
369.01	0.00498022645595387\\
370.01	0.00497917306176225\\
371.01	0.00497809961694975\\
372.01	0.00497700570952573\\
373.01	0.00497589091867262\\
374.01	0.00497475481458582\\
375.01	0.0049735969583125\\
376.01	0.00497241690158675\\
377.01	0.00497121418666233\\
378.01	0.00496998834614267\\
379.01	0.00496873890280737\\
380.01	0.00496746536943322\\
381.01	0.00496616724861297\\
382.01	0.00496484403256673\\
383.01	0.00496349520294922\\
384.01	0.0049621202306498\\
385.01	0.00496071857558527\\
386.01	0.00495928968648604\\
387.01	0.00495783300067191\\
388.01	0.00495634794382075\\
389.01	0.00495483392972596\\
390.01	0.00495329036004352\\
391.01	0.00495171662402862\\
392.01	0.00495011209825996\\
393.01	0.00494847614635249\\
394.01	0.00494680811865763\\
395.01	0.0049451073519513\\
396.01	0.00494337316910967\\
397.01	0.00494160487877243\\
398.01	0.00493980177499517\\
399.01	0.00493796313689028\\
400.01	0.00493608822825808\\
401.01	0.00493417629720918\\
402.01	0.00493222657577901\\
403.01	0.00493023827953653\\
404.01	0.00492821060718771\\
405.01	0.00492614274017615\\
406.01	0.00492403384228107\\
407.01	0.00492188305921548\\
408.01	0.00491968951822292\\
409.01	0.0049174523276755\\
410.01	0.00491517057667094\\
411.01	0.00491284333463032\\
412.01	0.00491046965089229\\
413.01	0.00490804855430651\\
414.01	0.0049055790528206\\
415.01	0.00490306013306335\\
416.01	0.00490049075992149\\
417.01	0.00489786987611105\\
418.01	0.00489519640174224\\
419.01	0.00489246923387906\\
420.01	0.0048896872460923\\
421.01	0.00488684928800671\\
422.01	0.00488395418484163\\
423.01	0.00488100073694514\\
424.01	0.00487798771932154\\
425.01	0.0048749138811516\\
426.01	0.00487177794530663\\
427.01	0.00486857860785368\\
428.01	0.00486531453755403\\
429.01	0.00486198437535299\\
430.01	0.00485858673386123\\
431.01	0.00485512019682707\\
432.01	0.00485158331859944\\
433.01	0.00484797462358041\\
434.01	0.00484429260566794\\
435.01	0.00484053572768602\\
436.01	0.00483670242080498\\
437.01	0.00483279108394687\\
438.01	0.00482880008317991\\
439.01	0.00482472775109747\\
440.01	0.00482057238618295\\
441.01	0.00481633225215964\\
442.01	0.00481200557732326\\
443.01	0.0048075905538585\\
444.01	0.00480308533713759\\
445.01	0.00479848804500103\\
446.01	0.00479379675701847\\
447.01	0.00478900951373037\\
448.01	0.00478412431586967\\
449.01	0.00477913912356268\\
450.01	0.00477405185550922\\
451.01	0.00476886038814144\\
452.01	0.00476356255476095\\
453.01	0.0047581561446548\\
454.01	0.00475263890219039\\
455.01	0.00474700852588799\\
456.01	0.00474126266747335\\
457.01	0.00473539893090912\\
458.01	0.00472941487140539\\
459.01	0.00472330799441088\\
460.01	0.00471707575458347\\
461.01	0.00471071555474243\\
462.01	0.00470422474480017\\
463.01	0.00469760062067582\\
464.01	0.00469084042318818\\
465.01	0.00468394133692999\\
466.01	0.00467690048912133\\
467.01	0.0046697149484417\\
468.01	0.00466238172384067\\
469.01	0.00465489776332429\\
470.01	0.00464725995271764\\
471.01	0.00463946511440073\\
472.01	0.00463151000601805\\
473.01	0.00462339131915957\\
474.01	0.00461510567801282\\
475.01	0.00460664963798493\\
476.01	0.00459801968429604\\
477.01	0.00458921223054078\\
478.01	0.00458022361722044\\
479.01	0.00457105011024327\\
480.01	0.00456168789939413\\
481.01	0.00455213309677256\\
482.01	0.00454238173519966\\
483.01	0.00453242976659276\\
484.01	0.00452227306030811\\
485.01	0.0045119074014521\\
486.01	0.00450132848915897\\
487.01	0.004490531934837\\
488.01	0.00447951326038137\\
489.01	0.00446826789635372\\
490.01	0.00445679118012794\\
491.01	0.00444507835400288\\
492.01	0.00443312456327982\\
493.01	0.00442092485430535\\
494.01	0.00440847417247915\\
495.01	0.00439576736022571\\
496.01	0.00438279915493058\\
497.01	0.0043695641868392\\
498.01	0.00435605697691972\\
499.01	0.00434227193468841\\
500.01	0.00432820335599812\\
501.01	0.00431384542079002\\
502.01	0.0042991921908077\\
503.01	0.00428423760727525\\
504.01	0.00426897548853848\\
505.01	0.00425339952767048\\
506.01	0.0042375032900411\\
507.01	0.00422128021085203\\
508.01	0.00420472359263741\\
509.01	0.00418782660273114\\
510.01	0.00417058227070207\\
511.01	0.00415298348575844\\
512.01	0.00413502299412217\\
513.01	0.00411669339637638\\
514.01	0.00409798714478656\\
515.01	0.0040788965405988\\
516.01	0.00405941373131771\\
517.01	0.00403953070796672\\
518.01	0.00401923930233447\\
519.01	0.00399853118421253\\
520.01	0.00397739785862717\\
521.01	0.00395583066307222\\
522.01	0.00393382076474854\\
523.01	0.00391135915781609\\
524.01	0.00388843666066737\\
525.01	0.00386504391323036\\
526.01	0.00384117137431015\\
527.01	0.00381680931898115\\
528.01	0.00379194783604039\\
529.01	0.00376657682553818\\
530.01	0.00374068599639765\\
531.01	0.0037142648641438\\
532.01	0.00368730274875859\\
533.01	0.00365978877268431\\
534.01	0.00363171185899915\\
535.01	0.00360306072979041\\
536.01	0.00357382390475556\\
537.01	0.00354398970006315\\
538.01	0.00351354622751146\\
539.01	0.0034824813940233\\
540.01	0.00345078290152427\\
541.01	0.00341843824725186\\
542.01	0.00338543472455447\\
543.01	0.00335175942423902\\
544.01	0.00331739923653709\\
545.01	0.00328234085376592\\
546.01	0.00324657077376695\\
547.01	0.00321007530421679\\
548.01	0.0031728405679133\\
549.01	0.00313485250914984\\
550.01	0.00309609690130523\\
551.01	0.00305655935578795\\
552.01	0.0030162253324876\\
553.01	0.00297508015190375\\
554.01	0.00293310900913688\\
555.01	0.00289029698994688\\
556.01	0.00284662908910202\\
557.01	0.00280209023126455\\
558.01	0.00275666529468013\\
559.01	0.00271033913796359\\
560.01	0.00266309663029763\\
561.01	0.00261492268538809\\
562.01	0.00256580229954686\\
563.01	0.0025157205942994\\
564.01	0.00246466286394252\\
565.01	0.00241261462850278\\
566.01	0.00235956169257031\\
567.01	0.00230549021050174\\
568.01	0.00225038675850025\\
569.01	0.00219423841408641\\
570.01	0.00213703284346752\\
571.01	0.00207875839729022\\
572.01	0.00201940421521853\\
573.01	0.00195896033970692\\
574.01	0.00189741783922914\\
575.01	0.00183476894106481\\
576.01	0.00177100717352635\\
577.01	0.00170612751720696\\
578.01	0.00164012656442551\\
579.01	0.0015730026855086\\
580.01	0.00150475619984723\\
581.01	0.00143538954874788\\
582.01	0.00136490746591783\\
583.01	0.00129331713990011\\
584.01	0.0012206283608285\\
585.01	0.00114685364138917\\
586.01	0.00107200829872149\\
587.01	0.000996110479992625\\
588.01	0.000919181109325396\\
589.01	0.000841243727382117\\
590.01	0.000762324186875217\\
591.01	0.000682450157175529\\
592.01	0.000601650378505039\\
593.01	0.000519953590291149\\
594.01	0.00043738703832388\\
595.01	0.00035397444039822\\
596.01	0.000269733258902364\\
597.01	0.000184671089782441\\
598.01	9.88225819852275e-05\\
599.01	3.18230442563731e-05\\
599.02	3.12770957175551e-05\\
599.03	3.07343620567849e-05\\
599.04	3.0194875217571e-05\\
599.05	2.96586674565693e-05\\
599.06	2.91257713466771e-05\\
599.07	2.85962197801391e-05\\
599.08	2.80700459716916e-05\\
599.09	2.75472834617447e-05\\
599.1	2.70279661195773e-05\\
599.11	2.65121281465865e-05\\
599.12	2.5999804079543e-05\\
599.13	2.54910287939142e-05\\
599.14	2.49858375071712e-05\\
599.15	2.44842657821827e-05\\
599.16	2.39863495305956e-05\\
599.17	2.34921250162855e-05\\
599.18	2.30016288588035e-05\\
599.19	2.25148980369013e-05\\
599.2	2.20319698920508e-05\\
599.21	2.15528821320213e-05\\
599.22	2.10776728344908e-05\\
599.23	2.06063804506721e-05\\
599.24	2.01390438090109e-05\\
599.25	1.96757021188858e-05\\
599.26	1.9216394974363e-05\\
599.27	1.87611623579855e-05\\
599.28	1.83100446445959e-05\\
599.29	1.78630826051952e-05\\
599.3	1.74203174108517e-05\\
599.31	1.69817906366353e-05\\
599.32	1.65475442655931e-05\\
599.33	1.61176206927693e-05\\
599.34	1.56920627292622e-05\\
599.35	1.52709136063203e-05\\
599.36	1.48542169794742e-05\\
599.37	1.4442016932719e-05\\
599.38	1.40343579827368e-05\\
599.39	1.36312859119591e-05\\
599.4	1.32328499381877e-05\\
599.41	1.28390997671604e-05\\
599.42	1.2450085597351e-05\\
599.43	1.20658581248094e-05\\
599.44	1.16864685480531e-05\\
599.45	1.13119685730065e-05\\
599.46	1.09424104179929e-05\\
599.47	1.05778468187639e-05\\
599.48	1.02183310335888e-05\\
599.49	9.86391684839293e-06\\
599.5	9.51465858193938e-06\\
599.51	9.17061109107289e-06\\
599.52	8.83182977599525e-06\\
599.53	8.4983705856221e-06\\
599.54	8.1702900229675e-06\\
599.55	7.84764515058753e-06\\
599.56	7.53049359608453e-06\\
599.57	7.21889355765649e-06\\
599.58	6.91290380970891e-06\\
599.59	6.61258370851688e-06\\
599.6	6.31799319793756e-06\\
599.61	6.02919281519031e-06\\
599.62	5.74624369668701e-06\\
599.63	5.46920758391807e-06\\
599.64	5.19814682940593e-06\\
599.65	4.93312440269685e-06\\
599.66	4.67420389643237e-06\\
599.67	4.42144953247646e-06\\
599.68	4.17492616808582e-06\\
599.69	3.9346993021671e-06\\
599.7	3.70083508156871e-06\\
599.71	3.47340030746289e-06\\
599.72	3.25246244175723e-06\\
599.73	3.03808961360161e-06\\
599.74	2.83035062593855e-06\\
599.75	2.62931496212288e-06\\
599.76	2.43505279260738e-06\\
599.77	2.24763498169606e-06\\
599.78	2.06713309435641e-06\\
599.79	1.89361940311147e-06\\
599.8	1.72716689498045e-06\\
599.81	1.56784927850782e-06\\
599.82	1.41574099085315e-06\\
599.83	1.27091720493813e-06\\
599.84	1.13345383668736e-06\\
599.85	1.00342755231415e-06\\
599.86	8.8091577571392e-07\\
599.87	7.65996695880136e-07\\
599.88	6.58749274441706e-07\\
599.89	5.59253253243699e-07\\
599.9	4.67589162011367e-07\\
599.91	3.83838326099145e-07\\
599.92	3.0808287429171e-07\\
599.93	2.40405746710845e-07\\
599.94	1.80890702777825e-07\\
599.95	1.2962232925906e-07\\
599.96	8.66860484002169e-08\\
599.97	5.21681261331924e-08\\
599.98	2.61556803542173e-08\\
599.99	8.73668930083393e-09\\
600	0\\
};
\addplot [color=black!60!mycolor21,solid,forget plot]
  table[row sep=crcr]{%
0.01	0.00503139859049536\\
1.01	0.00503139763734625\\
2.01	0.00503139666545196\\
3.01	0.00503139567444704\\
4.01	0.00503139466395896\\
5.01	0.00503139363360786\\
6.01	0.00503139258300674\\
7.01	0.00503139151176115\\
8.01	0.00503139041946888\\
9.01	0.00503138930572007\\
10.01	0.00503138817009698\\
11.01	0.00503138701217382\\
12.01	0.00503138583151661\\
13.01	0.005031384627683\\
14.01	0.00503138340022211\\
15.01	0.00503138214867432\\
16.01	0.00503138087257144\\
17.01	0.00503137957143602\\
18.01	0.00503137824478144\\
19.01	0.00503137689211191\\
20.01	0.00503137551292208\\
21.01	0.00503137410669669\\
22.01	0.00503137267291078\\
23.01	0.00503137121102927\\
24.01	0.00503136972050663\\
25.01	0.00503136820078729\\
26.01	0.0050313666513046\\
27.01	0.00503136507148146\\
28.01	0.00503136346072934\\
29.01	0.00503136181844857\\
30.01	0.00503136014402813\\
31.01	0.00503135843684492\\
32.01	0.00503135669626435\\
33.01	0.00503135492163941\\
34.01	0.0050313531123108\\
35.01	0.00503135126760659\\
36.01	0.0050313493868421\\
37.01	0.00503134746931938\\
38.01	0.00503134551432744\\
39.01	0.00503134352114138\\
40.01	0.00503134148902256\\
41.01	0.00503133941721832\\
42.01	0.00503133730496162\\
43.01	0.00503133515147055\\
44.01	0.00503133295594866\\
45.01	0.00503133071758422\\
46.01	0.00503132843554987\\
47.01	0.00503132610900244\\
48.01	0.00503132373708287\\
49.01	0.00503132131891577\\
50.01	0.00503131885360891\\
51.01	0.00503131634025312\\
52.01	0.00503131377792208\\
53.01	0.00503131116567176\\
54.01	0.0050313085025401\\
55.01	0.00503130578754686\\
56.01	0.00503130301969303\\
57.01	0.00503130019796089\\
58.01	0.00503129732131309\\
59.01	0.00503129438869273\\
60.01	0.0050312913990228\\
61.01	0.00503128835120597\\
62.01	0.00503128524412393\\
63.01	0.00503128207663747\\
64.01	0.0050312788475854\\
65.01	0.00503127555578462\\
66.01	0.00503127220002975\\
67.01	0.00503126877909265\\
68.01	0.0050312652917218\\
69.01	0.00503126173664195\\
70.01	0.00503125811255386\\
71.01	0.00503125441813366\\
72.01	0.0050312506520323\\
73.01	0.00503124681287535\\
74.01	0.00503124289926258\\
75.01	0.00503123890976706\\
76.01	0.0050312348429351\\
77.01	0.00503123069728553\\
78.01	0.00503122647130935\\
79.01	0.00503122216346869\\
80.01	0.00503121777219725\\
81.01	0.00503121329589863\\
82.01	0.00503120873294689\\
83.01	0.005031204081685\\
84.01	0.00503119934042499\\
85.01	0.00503119450744717\\
86.01	0.00503118958099935\\
87.01	0.0050311845592965\\
88.01	0.00503117944051983\\
89.01	0.00503117422281647\\
90.01	0.00503116890429895\\
91.01	0.00503116348304399\\
92.01	0.00503115795709251\\
93.01	0.00503115232444854\\
94.01	0.00503114658307843\\
95.01	0.00503114073091068\\
96.01	0.00503113476583492\\
97.01	0.0050311286857011\\
98.01	0.00503112248831905\\
99.01	0.00503111617145736\\
100.01	0.005031109732843\\
101.01	0.00503110317016036\\
102.01	0.00503109648105029\\
103.01	0.00503108966310976\\
104.01	0.00503108271389102\\
105.01	0.00503107563090009\\
106.01	0.00503106841159678\\
107.01	0.00503106105339315\\
108.01	0.005031053553653\\
109.01	0.00503104590969137\\
110.01	0.00503103811877243\\
111.01	0.00503103017810982\\
112.01	0.00503102208486511\\
113.01	0.00503101383614706\\
114.01	0.00503100542901042\\
115.01	0.00503099686045517\\
116.01	0.00503098812742551\\
117.01	0.00503097922680829\\
118.01	0.00503097015543318\\
119.01	0.00503096091007028\\
120.01	0.00503095148743003\\
121.01	0.00503094188416136\\
122.01	0.0050309320968513\\
123.01	0.00503092212202324\\
124.01	0.0050309119561359\\
125.01	0.00503090159558271\\
126.01	0.00503089103668979\\
127.01	0.00503088027571513\\
128.01	0.00503086930884764\\
129.01	0.00503085813220537\\
130.01	0.00503084674183433\\
131.01	0.00503083513370743\\
132.01	0.00503082330372289\\
133.01	0.00503081124770333\\
134.01	0.00503079896139353\\
135.01	0.00503078644045992\\
136.01	0.00503077368048869\\
137.01	0.00503076067698438\\
138.01	0.00503074742536853\\
139.01	0.00503073392097806\\
140.01	0.00503072015906363\\
141.01	0.00503070613478845\\
142.01	0.00503069184322624\\
143.01	0.00503067727935992\\
144.01	0.00503066243807978\\
145.01	0.00503064731418194\\
146.01	0.00503063190236643\\
147.01	0.00503061619723582\\
148.01	0.00503060019329304\\
149.01	0.00503058388493984\\
150.01	0.00503056726647493\\
151.01	0.00503055033209187\\
152.01	0.00503053307587752\\
153.01	0.00503051549180999\\
154.01	0.00503049757375628\\
155.01	0.00503047931547104\\
156.01	0.00503046071059376\\
157.01	0.00503044175264716\\
158.01	0.00503042243503509\\
159.01	0.00503040275103985\\
160.01	0.00503038269382059\\
161.01	0.00503036225641073\\
162.01	0.00503034143171598\\
163.01	0.00503032021251167\\
164.01	0.0050302985914408\\
165.01	0.00503027656101107\\
166.01	0.00503025411359278\\
167.01	0.00503023124141671\\
168.01	0.00503020793657069\\
169.01	0.00503018419099782\\
170.01	0.00503015999649337\\
171.01	0.00503013534470243\\
172.01	0.00503011022711691\\
173.01	0.00503008463507281\\
174.01	0.00503005855974766\\
175.01	0.00503003199215706\\
176.01	0.00503000492315242\\
177.01	0.00502997734341756\\
178.01	0.00502994924346589\\
179.01	0.00502992061363698\\
180.01	0.00502989144409368\\
181.01	0.0050298617248189\\
182.01	0.00502983144561235\\
183.01	0.00502980059608679\\
184.01	0.00502976916566513\\
185.01	0.00502973714357697\\
186.01	0.00502970451885467\\
187.01	0.00502967128032999\\
188.01	0.0050296374166304\\
189.01	0.00502960291617576\\
190.01	0.0050295677671735\\
191.01	0.00502953195761594\\
192.01	0.00502949547527527\\
193.01	0.00502945830770057\\
194.01	0.00502942044221267\\
195.01	0.00502938186590086\\
196.01	0.00502934256561803\\
197.01	0.00502930252797667\\
198.01	0.00502926173934435\\
199.01	0.0050292201858392\\
200.01	0.00502917785332523\\
201.01	0.00502913472740798\\
202.01	0.00502909079342936\\
203.01	0.00502904603646299\\
204.01	0.00502900044130903\\
205.01	0.0050289539924897\\
206.01	0.00502890667424345\\
207.01	0.00502885847051987\\
208.01	0.0050288093649748\\
209.01	0.00502875934096441\\
210.01	0.00502870838153951\\
211.01	0.00502865646944035\\
212.01	0.00502860358709064\\
213.01	0.00502854971659139\\
214.01	0.00502849483971531\\
215.01	0.0050284389379003\\
216.01	0.00502838199224336\\
217.01	0.00502832398349412\\
218.01	0.00502826489204852\\
219.01	0.00502820469794217\\
220.01	0.00502814338084295\\
221.01	0.00502808092004487\\
222.01	0.00502801729446075\\
223.01	0.0050279524826145\\
224.01	0.00502788646263481\\
225.01	0.00502781921224657\\
226.01	0.00502775070876387\\
227.01	0.00502768092908203\\
228.01	0.00502760984966948\\
229.01	0.00502753744655966\\
230.01	0.0050274636953428\\
231.01	0.00502738857115743\\
232.01	0.00502731204868156\\
233.01	0.005027234102124\\
234.01	0.00502715470521516\\
235.01	0.00502707383119783\\
236.01	0.00502699145281805\\
237.01	0.00502690754231499\\
238.01	0.00502682207141134\\
239.01	0.00502673501130347\\
240.01	0.00502664633265087\\
241.01	0.00502655600556584\\
242.01	0.0050264639996026\\
243.01	0.0050263702837467\\
244.01	0.00502627482640345\\
245.01	0.00502617759538634\\
246.01	0.00502607855790602\\
247.01	0.00502597768055802\\
248.01	0.00502587492931026\\
249.01	0.00502577026949106\\
250.01	0.00502566366577619\\
251.01	0.00502555508217566\\
252.01	0.00502544448202057\\
253.01	0.00502533182794971\\
254.01	0.0050252170818954\\
255.01	0.00502510020506895\\
256.01	0.00502498115794663\\
257.01	0.0050248599002545\\
258.01	0.00502473639095347\\
259.01	0.00502461058822324\\
260.01	0.00502448244944653\\
261.01	0.00502435193119314\\
262.01	0.00502421898920295\\
263.01	0.00502408357836889\\
264.01	0.00502394565271949\\
265.01	0.00502380516540097\\
266.01	0.0050236620686594\\
267.01	0.00502351631382134\\
268.01	0.00502336785127532\\
269.01	0.00502321663045207\\
270.01	0.00502306259980404\\
271.01	0.00502290570678556\\
272.01	0.00502274589783178\\
273.01	0.0050225831183366\\
274.01	0.00502241731263146\\
275.01	0.00502224842396228\\
276.01	0.00502207639446676\\
277.01	0.00502190116515079\\
278.01	0.0050217226758641\\
279.01	0.00502154086527589\\
280.01	0.00502135567084918\\
281.01	0.00502116702881495\\
282.01	0.00502097487414639\\
283.01	0.00502077914053067\\
284.01	0.00502057976034186\\
285.01	0.00502037666461219\\
286.01	0.00502016978300333\\
287.01	0.00501995904377588\\
288.01	0.00501974437375977\\
289.01	0.00501952569832223\\
290.01	0.00501930294133615\\
291.01	0.00501907602514742\\
292.01	0.00501884487054113\\
293.01	0.00501860939670745\\
294.01	0.00501836952120628\\
295.01	0.00501812515993157\\
296.01	0.00501787622707447\\
297.01	0.00501762263508545\\
298.01	0.00501736429463557\\
299.01	0.00501710111457737\\
300.01	0.00501683300190418\\
301.01	0.00501655986170874\\
302.01	0.00501628159714126\\
303.01	0.00501599810936545\\
304.01	0.00501570929751445\\
305.01	0.00501541505864538\\
306.01	0.00501511528769338\\
307.01	0.00501480987742318\\
308.01	0.00501449871838138\\
309.01	0.00501418169884616\\
310.01	0.00501385870477673\\
311.01	0.0050135296197612\\
312.01	0.0050131943249632\\
313.01	0.00501285269906782\\
314.01	0.00501250461822576\\
315.01	0.00501214995599656\\
316.01	0.00501178858329041\\
317.01	0.00501142036830881\\
318.01	0.00501104517648384\\
319.01	0.00501066287041591\\
320.01	0.00501027330981074\\
321.01	0.00500987635141417\\
322.01	0.00500947184894646\\
323.01	0.00500905965303438\\
324.01	0.00500863961114235\\
325.01	0.00500821156750197\\
326.01	0.0050077753630402\\
327.01	0.00500733083530581\\
328.01	0.0050068778183947\\
329.01	0.00500641614287336\\
330.01	0.00500594563570093\\
331.01	0.00500546612014977\\
332.01	0.00500497741572432\\
333.01	0.00500447933807829\\
334.01	0.00500397169893088\\
335.01	0.00500345430598014\\
336.01	0.00500292696281612\\
337.01	0.00500238946883131\\
338.01	0.00500184161912962\\
339.01	0.0050012832044345\\
340.01	0.00500071401099379\\
341.01	0.00500013382048451\\
342.01	0.00499954240991468\\
343.01	0.00499893955152418\\
344.01	0.00499832501268307\\
345.01	0.0049976985557892\\
346.01	0.00499705993816283\\
347.01	0.0049964089119405\\
348.01	0.0049957452239657\\
349.01	0.00499506861567949\\
350.01	0.00499437882300773\\
351.01	0.0049936755762471\\
352.01	0.00499295859994831\\
353.01	0.00499222761279911\\
354.01	0.00499148232750315\\
355.01	0.00499072245065837\\
356.01	0.00498994768263262\\
357.01	0.00498915771743696\\
358.01	0.0049883522425973\\
359.01	0.00498753093902332\\
360.01	0.00498669348087552\\
361.01	0.00498583953542931\\
362.01	0.00498496876293744\\
363.01	0.00498408081648879\\
364.01	0.00498317534186573\\
365.01	0.00498225197739787\\
366.01	0.00498131035381339\\
367.01	0.0049803500940877\\
368.01	0.00497937081328793\\
369.01	0.00497837211841585\\
370.01	0.0049773536082462\\
371.01	0.00497631487316221\\
372.01	0.0049752554949872\\
373.01	0.0049741750468126\\
374.01	0.00497307309282207\\
375.01	0.00497194918811139\\
376.01	0.00497080287850422\\
377.01	0.00496963370036354\\
378.01	0.00496844118039817\\
379.01	0.00496722483546439\\
380.01	0.00496598417236356\\
381.01	0.00496471868763301\\
382.01	0.00496342786733346\\
383.01	0.00496211118682948\\
384.01	0.00496076811056514\\
385.01	0.00495939809183382\\
386.01	0.0049580005725418\\
387.01	0.00495657498296681\\
388.01	0.0049551207415093\\
389.01	0.00495363725443867\\
390.01	0.0049521239156333\\
391.01	0.00495058010631375\\
392.01	0.0049490051947711\\
393.01	0.00494739853608841\\
394.01	0.00494575947185656\\
395.01	0.00494408732988409\\
396.01	0.00494238142390204\\
397.01	0.00494064105326258\\
398.01	0.0049388655026322\\
399.01	0.00493705404168015\\
400.01	0.00493520592476195\\
401.01	0.00493332039059669\\
402.01	0.00493139666194075\\
403.01	0.00492943394525598\\
404.01	0.00492743143037305\\
405.01	0.00492538829015008\\
406.01	0.00492330368012597\\
407.01	0.00492117673816835\\
408.01	0.00491900658411652\\
409.01	0.00491679231941834\\
410.01	0.00491453302676086\\
411.01	0.00491222776969481\\
412.01	0.00490987559225262\\
413.01	0.0049074755185588\\
414.01	0.00490502655243384\\
415.01	0.00490252767699035\\
416.01	0.00489997785422148\\
417.01	0.00489737602458166\\
418.01	0.00489472110655943\\
419.01	0.00489201199624185\\
420.01	0.00488924756687061\\
421.01	0.00488642666838923\\
422.01	0.00488354812698167\\
423.01	0.00488061074460138\\
424.01	0.00487761329849112\\
425.01	0.0048745545406932\\
426.01	0.00487143319754917\\
427.01	0.00486824796918974\\
428.01	0.00486499752901395\\
429.01	0.00486168052315687\\
430.01	0.00485829556994715\\
431.01	0.00485484125935185\\
432.01	0.00485131615240987\\
433.01	0.00484771878065343\\
434.01	0.00484404764551618\\
435.01	0.00484030121772918\\
436.01	0.00483647793670301\\
437.01	0.00483257620989627\\
438.01	0.00482859441217073\\
439.01	0.00482453088513136\\
440.01	0.00482038393645248\\
441.01	0.00481615183918814\\
442.01	0.00481183283106798\\
443.01	0.00480742511377696\\
444.01	0.00480292685222\\
445.01	0.00479833617376934\\
446.01	0.00479365116749608\\
447.01	0.00478886988338537\\
448.01	0.00478399033153439\\
449.01	0.00477901048133291\\
450.01	0.00477392826062676\\
451.01	0.00476874155486356\\
452.01	0.00476344820622057\\
453.01	0.0047580460127148\\
454.01	0.00475253272729457\\
455.01	0.00474690605691219\\
456.01	0.00474116366157828\\
457.01	0.00473530315339724\\
458.01	0.00472932209558291\\
459.01	0.00472321800145414\\
460.01	0.0047169883334113\\
461.01	0.00471063050189128\\
462.01	0.00470414186430254\\
463.01	0.0046975197239378\\
464.01	0.00469076132886535\\
465.01	0.00468386387079798\\
466.01	0.00467682448393921\\
467.01	0.00466964024380527\\
468.01	0.00466230816602322\\
469.01	0.00465482520510529\\
470.01	0.00464718825319733\\
471.01	0.00463939413880171\\
472.01	0.00463143962547456\\
473.01	0.00462332141049567\\
474.01	0.00461503612351281\\
475.01	0.0046065803251571\\
476.01	0.00459795050563165\\
477.01	0.00458914308327175\\
478.01	0.00458015440307578\\
479.01	0.00457098073520809\\
480.01	0.00456161827347155\\
481.01	0.00455206313375069\\
482.01	0.00454231135242416\\
483.01	0.00453235888474604\\
484.01	0.00452220160319721\\
485.01	0.00451183529580319\\
486.01	0.00450125566442095\\
487.01	0.00449045832299321\\
488.01	0.00447943879576871\\
489.01	0.00446819251548963\\
490.01	0.00445671482154523\\
491.01	0.00444500095808993\\
492.01	0.00443304607212781\\
493.01	0.00442084521156108\\
494.01	0.00440839332320334\\
495.01	0.00439568525075659\\
496.01	0.00438271573275181\\
497.01	0.0043694794004541\\
498.01	0.00435597077572967\\
499.01	0.0043421842688769\\
500.01	0.00432811417642031\\
501.01	0.0043137546788676\\
502.01	0.00429909983842902\\
503.01	0.0042841435967016\\
504.01	0.00426887977231503\\
505.01	0.00425330205854258\\
506.01	0.00423740402087563\\
507.01	0.00422117909456325\\
508.01	0.00420462058211687\\
509.01	0.00418772165078221\\
510.01	0.00417047532997789\\
511.01	0.00415287450870299\\
512.01	0.00413491193291534\\
513.01	0.00411658020288142\\
514.01	0.00409787177050073\\
515.01	0.00407877893660669\\
516.01	0.00405929384824654\\
517.01	0.00403940849594464\\
518.01	0.00401911471095102\\
519.01	0.00399840416248059\\
520.01	0.00397726835494781\\
521.01	0.00395569862520106\\
522.01	0.00393368613976365\\
523.01	0.00391122189208767\\
524.01	0.00388829669982948\\
525.01	0.00386490120215339\\
526.01	0.00384102585707547\\
527.01	0.00381666093885656\\
528.01	0.00379179653545806\\
529.01	0.00376642254607224\\
530.01	0.00374052867874502\\
531.01	0.003714104448105\\
532.01	0.00368713917322073\\
533.01	0.00365962197560618\\
534.01	0.00363154177739699\\
535.01	0.00360288729972611\\
536.01	0.00357364706132669\\
537.01	0.00354380937739529\\
538.01	0.00351336235875247\\
539.01	0.00348229391134084\\
540.01	0.00345059173610505\\
541.01	0.00341824332930561\\
542.01	0.00338523598332041\\
543.01	0.00335155678799634\\
544.01	0.00331719263262028\\
545.01	0.00328213020858442\\
546.01	0.00324635601283099\\
547.01	0.00320985635216906\\
548.01	0.00317261734856691\\
549.01	0.0031346249455342\\
550.01	0.00309586491572126\\
551.01	0.00305632286987125\\
552.01	0.00301598426728306\\
553.01	0.00297483442795171\\
554.01	0.00293285854657185\\
555.01	0.00289004170861091\\
556.01	0.00284636890867354\\
557.01	0.00280182507140488\\
558.01	0.00275639507519943\\
559.01	0.00271006377900744\\
560.01	0.0026628160525571\\
561.01	0.00261463681033538\\
562.01	0.00256551104969777\\
563.01	0.00251542389350631\\
564.01	0.00246436063771941\\
565.01	0.00241230680438499\\
566.01	0.00235924820051124\\
567.01	0.00230517098330829\\
568.01	0.00225006173230929\\
569.01	0.00219390752888354\\
570.01	0.00213669604364802\\
571.01	0.00207841563226292\\
572.01	0.00201905544004984\\
573.01	0.00195860551580188\\
574.01	0.00189705693504289\\
575.01	0.0018344019328357\\
576.01	0.00177063404601639\\
577.01	0.00170574826442997\\
578.01	0.0016397411903371\\
579.01	0.00157261120462172\\
580.01	0.00150435863772479\\
581.01	0.00143498594231021\\
582.01	0.00136449786348202\\
583.01	0.00129290160084457\\
584.01	0.00122020695474439\\
585.01	0.00114642644654167\\
586.01	0.00107157539959435\\
587.01	0.000995671963626618\\
588.01	0.000918737060083706\\
589.01	0.000840794219677079\\
590.01	0.000761869275267243\\
591.01	0.000681989863100538\\
592.01	0.000601184672692693\\
593.01	0.000519482369694076\\
594.01	0.000436910096074035\\
595.01	0.000353491426927423\\
596.01	0.000269243631889871\\
597.01	0.000184174050000235\\
598.01	9.85575539648349e-05\\
599.01	3.18230442563731e-05\\
599.02	3.12770957175568e-05\\
599.03	3.07343620567866e-05\\
599.04	3.0194875217571e-05\\
599.05	2.96586674565693e-05\\
599.06	2.91257713466771e-05\\
599.07	2.85962197801391e-05\\
599.08	2.80700459716933e-05\\
599.09	2.75472834617464e-05\\
599.1	2.70279661195791e-05\\
599.11	2.65121281465865e-05\\
599.12	2.59998040795448e-05\\
599.13	2.54910287939124e-05\\
599.14	2.49858375071712e-05\\
599.15	2.44842657821827e-05\\
599.16	2.39863495305973e-05\\
599.17	2.34921250162837e-05\\
599.18	2.30016288588035e-05\\
599.19	2.25148980369013e-05\\
599.2	2.20319698920526e-05\\
599.21	2.15528821320247e-05\\
599.22	2.10776728344908e-05\\
599.23	2.06063804506721e-05\\
599.24	2.01390438090126e-05\\
599.25	1.96757021188858e-05\\
599.26	1.92163949743647e-05\\
599.27	1.87611623579872e-05\\
599.28	1.83100446445959e-05\\
599.29	1.78630826051952e-05\\
599.3	1.74203174108517e-05\\
599.31	1.69817906366353e-05\\
599.32	1.65475442655931e-05\\
599.33	1.61176206927693e-05\\
599.34	1.56920627292639e-05\\
599.35	1.52709136063203e-05\\
599.36	1.48542169794725e-05\\
599.37	1.44420169327208e-05\\
599.38	1.40343579827368e-05\\
599.39	1.36312859119591e-05\\
599.4	1.32328499381877e-05\\
599.41	1.28390997671604e-05\\
599.42	1.24500855973528e-05\\
599.43	1.20658581248111e-05\\
599.44	1.16864685480531e-05\\
599.45	1.13119685730082e-05\\
599.46	1.09424104179929e-05\\
599.47	1.05778468187639e-05\\
599.48	1.02183310335888e-05\\
599.49	9.86391684839293e-06\\
599.5	9.51465858194112e-06\\
599.51	9.17061109107116e-06\\
599.52	8.83182977599525e-06\\
599.53	8.4983705856221e-06\\
599.54	8.1702900229675e-06\\
599.55	7.84764515058753e-06\\
599.56	7.53049359608279e-06\\
599.57	7.21889355765649e-06\\
599.58	6.91290380971064e-06\\
599.59	6.61258370851688e-06\\
599.6	6.31799319793583e-06\\
599.61	6.02919281519031e-06\\
599.62	5.74624369668701e-06\\
599.63	5.46920758391981e-06\\
599.64	5.19814682940593e-06\\
599.65	4.93312440269511e-06\\
599.66	4.67420389643411e-06\\
599.67	4.42144953247646e-06\\
599.68	4.17492616808755e-06\\
599.69	3.9346993021671e-06\\
599.7	3.70083508156871e-06\\
599.71	3.47340030746116e-06\\
599.72	3.25246244175549e-06\\
599.73	3.03808961359987e-06\\
599.74	2.83035062593855e-06\\
599.75	2.62931496212288e-06\\
599.76	2.43505279260738e-06\\
599.77	2.24763498169606e-06\\
599.78	2.06713309435641e-06\\
599.79	1.89361940310974e-06\\
599.8	1.72716689497872e-06\\
599.81	1.56784927850956e-06\\
599.82	1.41574099085315e-06\\
599.83	1.27091720493987e-06\\
599.84	1.13345383668563e-06\\
599.85	1.00342755231589e-06\\
599.86	8.80915775712185e-07\\
599.87	7.65996695880136e-07\\
599.88	6.5874927444344e-07\\
599.89	5.59253253243699e-07\\
599.9	4.67589162013102e-07\\
599.91	3.8383832609741e-07\\
599.92	3.0808287429171e-07\\
599.93	2.4040574671258e-07\\
599.94	1.80890702777825e-07\\
599.95	1.2962232925906e-07\\
599.96	8.66860484019516e-08\\
599.97	5.21681261331924e-08\\
599.98	2.61556803542173e-08\\
599.99	8.73668929909921e-09\\
600	0\\
};
\addplot [color=black!80!mycolor21,solid,forget plot]
  table[row sep=crcr]{%
0.01	0.00502843436055783\\
1.01	0.00502843349929418\\
2.01	0.00502843262123861\\
3.01	0.0050284317260663\\
4.01	0.00502843081344614\\
5.01	0.00502842988304058\\
6.01	0.00502842893450587\\
7.01	0.00502842796749136\\
8.01	0.00502842698163999\\
9.01	0.00502842597658766\\
10.01	0.00502842495196327\\
11.01	0.00502842390738872\\
12.01	0.0050284228424786\\
13.01	0.00502842175684009\\
14.01	0.00502842065007276\\
15.01	0.00502841952176892\\
16.01	0.0050284183715127\\
17.01	0.00502841719888041\\
18.01	0.00502841600344017\\
19.01	0.00502841478475189\\
20.01	0.00502841354236696\\
21.01	0.00502841227582853\\
22.01	0.00502841098467075\\
23.01	0.00502840966841888\\
24.01	0.00502840832658917\\
25.01	0.00502840695868844\\
26.01	0.00502840556421434\\
27.01	0.00502840414265467\\
28.01	0.00502840269348778\\
29.01	0.00502840121618183\\
30.01	0.0050283997101948\\
31.01	0.00502839817497453\\
32.01	0.00502839660995804\\
33.01	0.00502839501457191\\
34.01	0.00502839338823163\\
35.01	0.00502839173034153\\
36.01	0.00502839004029431\\
37.01	0.00502838831747177\\
38.01	0.00502838656124318\\
39.01	0.00502838477096642\\
40.01	0.00502838294598677\\
41.01	0.00502838108563712\\
42.01	0.00502837918923754\\
43.01	0.00502837725609535\\
44.01	0.0050283752855046\\
45.01	0.00502837327674568\\
46.01	0.00502837122908562\\
47.01	0.00502836914177741\\
48.01	0.00502836701405978\\
49.01	0.00502836484515692\\
50.01	0.00502836263427834\\
51.01	0.00502836038061843\\
52.01	0.00502835808335623\\
53.01	0.00502835574165529\\
54.01	0.00502835335466321\\
55.01	0.00502835092151147\\
56.01	0.00502834844131502\\
57.01	0.00502834591317168\\
58.01	0.00502834333616256\\
59.01	0.00502834070935102\\
60.01	0.00502833803178272\\
61.01	0.00502833530248539\\
62.01	0.00502833252046801\\
63.01	0.0050283296847209\\
64.01	0.0050283267942151\\
65.01	0.00502832384790237\\
66.01	0.0050283208447142\\
67.01	0.00502831778356208\\
68.01	0.00502831466333673\\
69.01	0.00502831148290791\\
70.01	0.00502830824112375\\
71.01	0.00502830493681093\\
72.01	0.00502830156877361\\
73.01	0.0050282981357933\\
74.01	0.00502829463662843\\
75.01	0.00502829107001377\\
76.01	0.00502828743466048\\
77.01	0.00502828372925505\\
78.01	0.00502827995245877\\
79.01	0.0050282761029083\\
80.01	0.00502827217921379\\
81.01	0.00502826817995968\\
82.01	0.00502826410370298\\
83.01	0.00502825994897383\\
84.01	0.0050282557142745\\
85.01	0.00502825139807869\\
86.01	0.00502824699883142\\
87.01	0.0050282425149483\\
88.01	0.00502823794481481\\
89.01	0.00502823328678589\\
90.01	0.00502822853918555\\
91.01	0.00502822370030597\\
92.01	0.00502821876840701\\
93.01	0.00502821374171555\\
94.01	0.00502820861842523\\
95.01	0.00502820339669533\\
96.01	0.00502819807465029\\
97.01	0.00502819265037929\\
98.01	0.00502818712193531\\
99.01	0.00502818148733451\\
100.01	0.00502817574455551\\
101.01	0.00502816989153885\\
102.01	0.00502816392618621\\
103.01	0.00502815784635948\\
104.01	0.00502815164988006\\
105.01	0.00502814533452847\\
106.01	0.00502813889804326\\
107.01	0.00502813233812017\\
108.01	0.00502812565241156\\
109.01	0.0050281188385253\\
110.01	0.00502811189402423\\
111.01	0.00502810481642512\\
112.01	0.00502809760319793\\
113.01	0.00502809025176476\\
114.01	0.00502808275949907\\
115.01	0.00502807512372475\\
116.01	0.00502806734171527\\
117.01	0.00502805941069259\\
118.01	0.00502805132782616\\
119.01	0.0050280430902323\\
120.01	0.00502803469497239\\
121.01	0.00502802613905277\\
122.01	0.00502801741942322\\
123.01	0.00502800853297598\\
124.01	0.00502799947654461\\
125.01	0.00502799024690289\\
126.01	0.00502798084076383\\
127.01	0.00502797125477838\\
128.01	0.00502796148553406\\
129.01	0.00502795152955438\\
130.01	0.00502794138329717\\
131.01	0.00502793104315328\\
132.01	0.00502792050544566\\
133.01	0.00502790976642754\\
134.01	0.00502789882228181\\
135.01	0.00502788766911928\\
136.01	0.00502787630297728\\
137.01	0.00502786471981833\\
138.01	0.00502785291552887\\
139.01	0.00502784088591756\\
140.01	0.00502782862671417\\
141.01	0.00502781613356767\\
142.01	0.00502780340204514\\
143.01	0.00502779042762982\\
144.01	0.00502777720571961\\
145.01	0.0050277637316257\\
146.01	0.00502775000057084\\
147.01	0.00502773600768727\\
148.01	0.00502772174801552\\
149.01	0.00502770721650234\\
150.01	0.00502769240799919\\
151.01	0.00502767731726028\\
152.01	0.00502766193894071\\
153.01	0.00502764626759434\\
154.01	0.00502763029767243\\
155.01	0.00502761402352129\\
156.01	0.00502759743938044\\
157.01	0.00502758053938022\\
158.01	0.00502756331754031\\
159.01	0.00502754576776728\\
160.01	0.00502752788385231\\
161.01	0.00502750965946929\\
162.01	0.00502749108817226\\
163.01	0.00502747216339341\\
164.01	0.00502745287844059\\
165.01	0.00502743322649504\\
166.01	0.00502741320060887\\
167.01	0.00502739279370249\\
168.01	0.00502737199856242\\
169.01	0.00502735080783846\\
170.01	0.00502732921404082\\
171.01	0.00502730720953808\\
172.01	0.00502728478655421\\
173.01	0.00502726193716557\\
174.01	0.00502723865329805\\
175.01	0.00502721492672447\\
176.01	0.00502719074906172\\
177.01	0.00502716611176715\\
178.01	0.00502714100613615\\
179.01	0.00502711542329865\\
180.01	0.00502708935421598\\
181.01	0.00502706278967793\\
182.01	0.00502703572029884\\
183.01	0.00502700813651456\\
184.01	0.00502698002857915\\
185.01	0.00502695138656097\\
186.01	0.00502692220033937\\
187.01	0.00502689245960104\\
188.01	0.00502686215383585\\
189.01	0.0050268312723334\\
190.01	0.00502679980417929\\
191.01	0.00502676773825054\\
192.01	0.00502673506321213\\
193.01	0.00502670176751231\\
194.01	0.00502666783937896\\
195.01	0.00502663326681498\\
196.01	0.00502659803759368\\
197.01	0.00502656213925446\\
198.01	0.00502652555909849\\
199.01	0.00502648828418351\\
200.01	0.00502645030131966\\
201.01	0.0050264115970639\\
202.01	0.00502637215771571\\
203.01	0.00502633196931158\\
204.01	0.0050262910176199\\
205.01	0.0050262492881358\\
206.01	0.00502620676607569\\
207.01	0.00502616343637166\\
208.01	0.0050261192836659\\
209.01	0.00502607429230486\\
210.01	0.00502602844633365\\
211.01	0.00502598172948986\\
212.01	0.0050259341251974\\
213.01	0.00502588561656033\\
214.01	0.00502583618635638\\
215.01	0.00502578581703054\\
216.01	0.00502573449068861\\
217.01	0.00502568218908995\\
218.01	0.00502562889364079\\
219.01	0.00502557458538716\\
220.01	0.00502551924500772\\
221.01	0.0050254628528063\\
222.01	0.0050254053887043\\
223.01	0.00502534683223293\\
224.01	0.00502528716252563\\
225.01	0.00502522635830983\\
226.01	0.00502516439789862\\
227.01	0.00502510125918259\\
228.01	0.00502503691962126\\
229.01	0.00502497135623418\\
230.01	0.00502490454559219\\
231.01	0.00502483646380783\\
232.01	0.00502476708652681\\
233.01	0.00502469638891764\\
234.01	0.00502462434566235\\
235.01	0.00502455093094678\\
236.01	0.00502447611844968\\
237.01	0.0050243998813329\\
238.01	0.005024322192231\\
239.01	0.00502424302323956\\
240.01	0.00502416234590481\\
241.01	0.00502408013121223\\
242.01	0.0050239963495747\\
243.01	0.00502391097082081\\
244.01	0.00502382396418274\\
245.01	0.00502373529828444\\
246.01	0.00502364494112798\\
247.01	0.00502355286008181\\
248.01	0.00502345902186676\\
249.01	0.005023363392543\\
250.01	0.00502326593749602\\
251.01	0.00502316662142288\\
252.01	0.00502306540831744\\
253.01	0.00502296226145618\\
254.01	0.00502285714338281\\
255.01	0.00502275001589302\\
256.01	0.00502264084001892\\
257.01	0.00502252957601304\\
258.01	0.00502241618333144\\
259.01	0.00502230062061778\\
260.01	0.00502218284568591\\
261.01	0.00502206281550182\\
262.01	0.00502194048616658\\
263.01	0.00502181581289769\\
264.01	0.00502168875001045\\
265.01	0.00502155925089892\\
266.01	0.005021427268016\\
267.01	0.00502129275285441\\
268.01	0.00502115565592534\\
269.01	0.00502101592673802\\
270.01	0.00502087351377883\\
271.01	0.00502072836448901\\
272.01	0.00502058042524221\\
273.01	0.00502042964132244\\
274.01	0.00502027595690047\\
275.01	0.00502011931500993\\
276.01	0.00501995965752344\\
277.01	0.00501979692512705\\
278.01	0.00501963105729606\\
279.01	0.00501946199226796\\
280.01	0.00501928966701663\\
281.01	0.00501911401722497\\
282.01	0.00501893497725712\\
283.01	0.00501875248013079\\
284.01	0.00501856645748749\\
285.01	0.00501837683956397\\
286.01	0.00501818355516119\\
287.01	0.00501798653161407\\
288.01	0.00501778569475948\\
289.01	0.00501758096890471\\
290.01	0.00501737227679403\\
291.01	0.00501715953957528\\
292.01	0.00501694267676547\\
293.01	0.00501672160621599\\
294.01	0.00501649624407649\\
295.01	0.00501626650475819\\
296.01	0.00501603230089676\\
297.01	0.00501579354331405\\
298.01	0.00501555014097906\\
299.01	0.005015302000968\\
300.01	0.00501504902842402\\
301.01	0.00501479112651473\\
302.01	0.00501452819639081\\
303.01	0.00501426013714194\\
304.01	0.00501398684575311\\
305.01	0.00501370821705909\\
306.01	0.00501342414369801\\
307.01	0.00501313451606501\\
308.01	0.00501283922226349\\
309.01	0.0050125381480563\\
310.01	0.00501223117681559\\
311.01	0.00501191818947187\\
312.01	0.00501159906446138\\
313.01	0.00501127367767312\\
314.01	0.00501094190239442\\
315.01	0.00501060360925566\\
316.01	0.00501025866617318\\
317.01	0.00500990693829178\\
318.01	0.00500954828792586\\
319.01	0.00500918257449919\\
320.01	0.00500880965448356\\
321.01	0.00500842938133637\\
322.01	0.00500804160543641\\
323.01	0.00500764617401919\\
324.01	0.00500724293111037\\
325.01	0.00500683171745786\\
326.01	0.005006412370463\\
327.01	0.00500598472411009\\
328.01	0.00500554860889441\\
329.01	0.00500510385174934\\
330.01	0.005004650275971\\
331.01	0.00500418770114295\\
332.01	0.00500371594305808\\
333.01	0.00500323481363959\\
334.01	0.00500274412086025\\
335.01	0.00500224366866026\\
336.01	0.0050017332568633\\
337.01	0.0050012126810911\\
338.01	0.0050006817326763\\
339.01	0.0050001401985734\\
340.01	0.00499958786126887\\
341.01	0.00499902449868812\\
342.01	0.00499844988410169\\
343.01	0.00499786378602948\\
344.01	0.00499726596814258\\
345.01	0.00499665618916393\\
346.01	0.00499603420276663\\
347.01	0.00499539975746997\\
348.01	0.00499475259653424\\
349.01	0.0049940924578527\\
350.01	0.00499341907384166\\
351.01	0.00499273217132882\\
352.01	0.00499203147143882\\
353.01	0.00499131668947634\\
354.01	0.00499058753480788\\
355.01	0.00498984371074027\\
356.01	0.0049890849143966\\
357.01	0.00498831083659057\\
358.01	0.00498752116169746\\
359.01	0.00498671556752294\\
360.01	0.00498589372516844\\
361.01	0.00498505529889489\\
362.01	0.00498419994598239\\
363.01	0.00498332731658825\\
364.01	0.00498243705360065\\
365.01	0.00498152879249033\\
366.01	0.00498060216115879\\
367.01	0.00497965677978277\\
368.01	0.00497869226065643\\
369.01	0.00497770820802893\\
370.01	0.00497670421793944\\
371.01	0.00497567987804857\\
372.01	0.00497463476746508\\
373.01	0.00497356845657062\\
374.01	0.0049724805068389\\
375.01	0.00497137047065233\\
376.01	0.00497023789111349\\
377.01	0.00496908230185381\\
378.01	0.00496790322683697\\
379.01	0.00496670018015875\\
380.01	0.00496547266584231\\
381.01	0.00496422017762947\\
382.01	0.00496294219876715\\
383.01	0.00496163820178908\\
384.01	0.00496030764829392\\
385.01	0.00495894998871806\\
386.01	0.00495756466210336\\
387.01	0.00495615109586128\\
388.01	0.00495470870553127\\
389.01	0.00495323689453462\\
390.01	0.0049517350539238\\
391.01	0.00495020256212605\\
392.01	0.00494863878468264\\
393.01	0.00494704307398262\\
394.01	0.00494541476899205\\
395.01	0.00494375319497745\\
396.01	0.00494205766322433\\
397.01	0.00494032747074996\\
398.01	0.00493856190001135\\
399.01	0.00493676021860735\\
400.01	0.00493492167897478\\
401.01	0.00493304551807963\\
402.01	0.00493113095710157\\
403.01	0.00492917720111273\\
404.01	0.00492718343875042\\
405.01	0.00492514884188331\\
406.01	0.00492307256527112\\
407.01	0.00492095374621756\\
408.01	0.00491879150421652\\
409.01	0.004916584940591\\
410.01	0.00491433313812519\\
411.01	0.00491203516068845\\
412.01	0.0049096900528525\\
413.01	0.00490729683949968\\
414.01	0.00490485452542453\\
415.01	0.0049023620949264\\
416.01	0.00489981851139374\\
417.01	0.00489722271688028\\
418.01	0.00489457363167176\\
419.01	0.00489187015384463\\
420.01	0.00488911115881449\\
421.01	0.00488629549887635\\
422.01	0.00488342200273419\\
423.01	0.00488048947502156\\
424.01	0.00487749669581191\\
425.01	0.00487444242011829\\
426.01	0.00487132537738289\\
427.01	0.00486814427095616\\
428.01	0.00486489777756404\\
429.01	0.00486158454676542\\
430.01	0.00485820320039651\\
431.01	0.0048547523320048\\
432.01	0.0048512305062708\\
433.01	0.00484763625841695\\
434.01	0.00484396809360503\\
435.01	0.00484022448632022\\
436.01	0.00483640387974204\\
437.01	0.00483250468510353\\
438.01	0.00482852528103479\\
439.01	0.00482446401289458\\
440.01	0.00482031919208718\\
441.01	0.00481608909536478\\
442.01	0.0048117719641162\\
443.01	0.00480736600364003\\
444.01	0.00480286938240266\\
445.01	0.0047982802312817\\
446.01	0.00479359664279346\\
447.01	0.00478881667030427\\
448.01	0.00478393832722535\\
449.01	0.00477895958619237\\
450.01	0.00477387837822654\\
451.01	0.00476869259188027\\
452.01	0.00476340007236439\\
453.01	0.00475799862065733\\
454.01	0.00475248599259701\\
455.01	0.00474685989795386\\
456.01	0.00474111799948513\\
457.01	0.00473525791196947\\
458.01	0.00472927720122272\\
459.01	0.00472317338309335\\
460.01	0.00471694392243755\\
461.01	0.00471058623207308\\
462.01	0.00470409767171277\\
463.01	0.00469747554687555\\
464.01	0.00469071710777627\\
465.01	0.00468381954819199\\
466.01	0.00467678000430614\\
467.01	0.00466959555352891\\
468.01	0.00466226321329406\\
469.01	0.00465477993983148\\
470.01	0.00464714262691423\\
471.01	0.00463934810458196\\
472.01	0.00463139313783721\\
473.01	0.00462327442531657\\
474.01	0.00461498859793493\\
475.01	0.00460653221750294\\
476.01	0.00459790177531674\\
477.01	0.00458909369072005\\
478.01	0.0045801043096378\\
479.01	0.00457092990308097\\
480.01	0.00456156666562179\\
481.01	0.00455201071383896\\
482.01	0.00454225808473339\\
483.01	0.00453230473411217\\
484.01	0.00452214653494158\\
485.01	0.00451177927566853\\
486.01	0.00450119865850974\\
487.01	0.00449040029770726\\
488.01	0.00447937971775215\\
489.01	0.00446813235157375\\
490.01	0.00445665353869464\\
491.01	0.00444493852335173\\
492.01	0.00443298245258193\\
493.01	0.00442078037427268\\
494.01	0.00440832723517685\\
495.01	0.00439561787889161\\
496.01	0.00438264704380088\\
497.01	0.00436940936098138\\
498.01	0.00435589935207233\\
499.01	0.00434211142710735\\
500.01	0.00432803988231031\\
501.01	0.00431367889785326\\
502.01	0.00429902253557849\\
503.01	0.00428406473668267\\
504.01	0.00426879931936507\\
505.01	0.00425321997643889\\
506.01	0.00423732027290725\\
507.01	0.00422109364350371\\
508.01	0.00420453339019825\\
509.01	0.00418763267966946\\
510.01	0.00417038454074419\\
511.01	0.0041527818618063\\
512.01	0.00413481738817535\\
513.01	0.0041164837194568\\
514.01	0.00409777330686782\\
515.01	0.00407867845053805\\
516.01	0.00405919129679085\\
517.01	0.00403930383540636\\
518.01	0.00401900789687119\\
519.01	0.00399829514961824\\
520.01	0.00397715709726165\\
521.01	0.00395558507583142\\
522.01	0.00393357025101552\\
523.01	0.00391110361541448\\
524.01	0.00388817598581671\\
525.01	0.0038647780005037\\
526.01	0.00384090011659382\\
527.01	0.00381653260743666\\
528.01	0.00379166556006887\\
529.01	0.00376628887274622\\
530.01	0.00374039225256658\\
531.01	0.00371396521320085\\
532.01	0.00368699707275148\\
533.01	0.00365947695175801\\
534.01	0.00363139377137604\\
535.01	0.00360273625175423\\
536.01	0.00357349291063878\\
537.01	0.00354365206223945\\
538.01	0.00351320181639204\\
539.01	0.00348213007805842\\
540.01	0.00345042454721088\\
541.01	0.00341807271914801\\
542.01	0.00338506188529999\\
543.01	0.00335137913458498\\
544.01	0.00331701135538371\\
545.01	0.00328194523821044\\
546.01	0.00324616727916235\\
547.01	0.00320966378424351\\
548.01	0.00317242087466379\\
549.01	0.00313442449322877\\
550.01	0.00309566041194589\\
551.01	0.00305611424098788\\
552.01	0.00301577143916516\\
553.01	0.0029746173260766\\
554.01	0.00293263709612681\\
555.01	0.00288981583461184\\
556.01	0.0028461385360993\\
557.01	0.00280159012534766\\
558.01	0.00275615548103306\\
559.01	0.00270981946257547\\
560.01	0.00266256694038211\\
561.01	0.0026143828298511\\
562.01	0.00256525212950647\\
563.01	0.00251515996366286\\
564.01	0.00246409163004402\\
565.01	0.00241203265280651\\
566.01	0.00235896884144259\\
567.01	0.00230488635605555\\
568.01	0.00224977177951561\\
569.01	0.00219361219700808\\
570.01	0.00213639528348043\\
571.01	0.00207810939947119\\
572.01	0.00201874369576008\\
573.01	0.00195828822720497\\
574.01	0.00189673407602165\\
575.01	0.00183407348460219\\
576.01	0.0017702999977462\\
577.01	0.00170540861387421\\
578.01	0.00163939594438615\\
579.01	0.00157226037978655\\
580.01	0.00150400226049134\\
581.01	0.00143462404930831\\
582.01	0.00136413050139218\\
583.01	0.00129252882594572\\
584.01	0.00121982883197683\\
585.01	0.00114604304792577\\
586.01	0.00107118680180153\\
587.01	0.000995278244445114\\
588.01	0.000918338293453008\\
589.01	0.0008403904688796\\
590.01	0.000761460583759315\\
591.01	0.000681576242330336\\
592.01	0.00060076608608719\\
593.01	0.000519058711787984\\
594.01	0.000436481165494431\\
595.01	0.000353056891624152\\
596.01	0.000268802984598084\\
597.01	0.000183726551418536\\
598.01	9.84661228899224e-05\\
599.01	3.18230442563749e-05\\
599.02	3.12770957175551e-05\\
599.03	3.07343620567866e-05\\
599.04	3.0194875217571e-05\\
599.05	2.9658667456571e-05\\
599.06	2.91257713466771e-05\\
599.07	2.85962197801391e-05\\
599.08	2.80700459716916e-05\\
599.09	2.75472834617447e-05\\
599.1	2.70279661195791e-05\\
599.11	2.65121281465865e-05\\
599.12	2.59998040795448e-05\\
599.13	2.54910287939142e-05\\
599.14	2.49858375071712e-05\\
599.15	2.4484265782181e-05\\
599.16	2.39863495305973e-05\\
599.17	2.34921250162855e-05\\
599.18	2.30016288588052e-05\\
599.19	2.25148980369013e-05\\
599.2	2.20319698920508e-05\\
599.21	2.1552882132023e-05\\
599.22	2.10776728344891e-05\\
599.23	2.06063804506721e-05\\
599.24	2.01390438090109e-05\\
599.25	1.96757021188858e-05\\
599.26	1.92163949743647e-05\\
599.27	1.87611623579872e-05\\
599.28	1.83100446445959e-05\\
599.29	1.78630826051952e-05\\
599.3	1.74203174108517e-05\\
599.31	1.69817906366335e-05\\
599.32	1.65475442655914e-05\\
599.33	1.61176206927693e-05\\
599.34	1.56920627292622e-05\\
599.35	1.52709136063186e-05\\
599.36	1.48542169794742e-05\\
599.37	1.4442016932719e-05\\
599.38	1.40343579827385e-05\\
599.39	1.36312859119608e-05\\
599.4	1.3232849938186e-05\\
599.41	1.28390997671621e-05\\
599.42	1.2450085597351e-05\\
599.43	1.20658581248094e-05\\
599.44	1.16864685480531e-05\\
599.45	1.13119685730065e-05\\
599.46	1.09424104179946e-05\\
599.47	1.05778468187639e-05\\
599.48	1.02183310335905e-05\\
599.49	9.86391684839293e-06\\
599.5	9.51465858193938e-06\\
599.51	9.17061109107289e-06\\
599.52	8.83182977599352e-06\\
599.53	8.4983705856221e-06\\
599.54	8.1702900229675e-06\\
599.55	7.84764515058579e-06\\
599.56	7.53049359608279e-06\\
599.57	7.21889355765649e-06\\
599.58	6.91290380971064e-06\\
599.59	6.61258370851688e-06\\
599.6	6.31799319793756e-06\\
599.61	6.02919281519031e-06\\
599.62	5.74624369668528e-06\\
599.63	5.46920758391981e-06\\
599.64	5.19814682940767e-06\\
599.65	4.93312440269685e-06\\
599.66	4.67420389643237e-06\\
599.67	4.42144953247646e-06\\
599.68	4.17492616808582e-06\\
599.69	3.93469930216536e-06\\
599.7	3.70083508156871e-06\\
599.71	3.47340030746116e-06\\
599.72	3.25246244175549e-06\\
599.73	3.03808961360161e-06\\
599.74	2.83035062593855e-06\\
599.75	2.62931496212288e-06\\
599.76	2.43505279260911e-06\\
599.77	2.24763498169606e-06\\
599.78	2.06713309435641e-06\\
599.79	1.89361940310974e-06\\
599.8	1.72716689498045e-06\\
599.81	1.56784927850782e-06\\
599.82	1.41574099085315e-06\\
599.83	1.27091720493813e-06\\
599.84	1.13345383668563e-06\\
599.85	1.00342755231589e-06\\
599.86	8.8091577571392e-07\\
599.87	7.65996695880136e-07\\
599.88	6.58749274441706e-07\\
599.89	5.59253253241965e-07\\
599.9	4.67589162011367e-07\\
599.91	3.83838326099145e-07\\
599.92	3.0808287429171e-07\\
599.93	2.40405746710845e-07\\
599.94	1.8089070277609e-07\\
599.95	1.29622329257326e-07\\
599.96	8.66860484019516e-08\\
599.97	5.21681261349272e-08\\
599.98	2.61556803542173e-08\\
599.99	8.73668930083393e-09\\
600	0\\
};
\addplot [color=black,solid,forget plot]
  table[row sep=crcr]{%
0.01	0.00502703576604918\\
1.01	0.00502703497172564\\
2.01	0.00502703416200814\\
3.01	0.00502703333659998\\
4.01	0.0050270324951992\\
5.01	0.00502703163749766\\
6.01	0.00502703076318172\\
7.01	0.00502702987193134\\
8.01	0.0050270289634203\\
9.01	0.00502702803731684\\
10.01	0.00502702709328209\\
11.01	0.00502702613097098\\
12.01	0.00502702515003172\\
13.01	0.00502702415010615\\
14.01	0.00502702313082892\\
15.01	0.00502702209182774\\
16.01	0.00502702103272314\\
17.01	0.00502701995312872\\
18.01	0.00502701885265022\\
19.01	0.00502701773088608\\
20.01	0.00502701658742719\\
21.01	0.00502701542185629\\
22.01	0.00502701423374847\\
23.01	0.00502701302267043\\
24.01	0.00502701178818072\\
25.01	0.00502701052982919\\
26.01	0.00502700924715755\\
27.01	0.00502700793969848\\
28.01	0.00502700660697566\\
29.01	0.00502700524850406\\
30.01	0.00502700386378847\\
31.01	0.00502700245232527\\
32.01	0.00502700101360067\\
33.01	0.00502699954709104\\
34.01	0.0050269980522628\\
35.01	0.00502699652857234\\
36.01	0.00502699497546558\\
37.01	0.00502699339237751\\
38.01	0.0050269917787328\\
39.01	0.00502699013394496\\
40.01	0.00502698845741587\\
41.01	0.00502698674853659\\
42.01	0.00502698500668635\\
43.01	0.00502698323123265\\
44.01	0.0050269814215304\\
45.01	0.0050269795769228\\
46.01	0.00502697769673993\\
47.01	0.00502697578029949\\
48.01	0.00502697382690603\\
49.01	0.00502697183585049\\
50.01	0.00502696980641097\\
51.01	0.00502696773785102\\
52.01	0.00502696562942034\\
53.01	0.00502696348035482\\
54.01	0.00502696128987477\\
55.01	0.00502695905718655\\
56.01	0.00502695678148061\\
57.01	0.00502695446193257\\
58.01	0.0050269520977019\\
59.01	0.00502694968793225\\
60.01	0.00502694723175042\\
61.01	0.00502694472826718\\
62.01	0.00502694217657556\\
63.01	0.0050269395757521\\
64.01	0.00502693692485474\\
65.01	0.00502693422292425\\
66.01	0.00502693146898289\\
67.01	0.00502692866203387\\
68.01	0.00502692580106153\\
69.01	0.0050269228850307\\
70.01	0.00502691991288679\\
71.01	0.00502691688355467\\
72.01	0.00502691379593868\\
73.01	0.00502691064892259\\
74.01	0.00502690744136828\\
75.01	0.00502690417211634\\
76.01	0.00502690083998477\\
77.01	0.00502689744376923\\
78.01	0.00502689398224234\\
79.01	0.00502689045415315\\
80.01	0.00502688685822671\\
81.01	0.00502688319316384\\
82.01	0.00502687945764056\\
83.01	0.00502687565030747\\
84.01	0.00502687176978926\\
85.01	0.00502686781468428\\
86.01	0.00502686378356412\\
87.01	0.00502685967497303\\
88.01	0.00502685548742783\\
89.01	0.00502685121941631\\
90.01	0.00502684686939733\\
91.01	0.00502684243580049\\
92.01	0.00502683791702561\\
93.01	0.00502683331144142\\
94.01	0.00502682861738518\\
95.01	0.00502682383316288\\
96.01	0.00502681895704779\\
97.01	0.00502681398728019\\
98.01	0.00502680892206636\\
99.01	0.00502680375957853\\
100.01	0.00502679849795363\\
101.01	0.00502679313529323\\
102.01	0.0050267876696622\\
103.01	0.00502678209908856\\
104.01	0.00502677642156221\\
105.01	0.00502677063503451\\
106.01	0.00502676473741767\\
107.01	0.00502675872658393\\
108.01	0.00502675260036436\\
109.01	0.00502674635654883\\
110.01	0.00502673999288452\\
111.01	0.0050267335070753\\
112.01	0.0050267268967811\\
113.01	0.00502672015961691\\
114.01	0.00502671329315179\\
115.01	0.00502670629490811\\
116.01	0.00502669916236056\\
117.01	0.00502669189293583\\
118.01	0.00502668448401019\\
119.01	0.00502667693291004\\
120.01	0.00502666923691057\\
121.01	0.00502666139323414\\
122.01	0.00502665339904985\\
123.01	0.00502664525147199\\
124.01	0.00502663694755967\\
125.01	0.00502662848431528\\
126.01	0.00502661985868368\\
127.01	0.00502661106755034\\
128.01	0.00502660210774116\\
129.01	0.00502659297602076\\
130.01	0.00502658366909147\\
131.01	0.00502657418359178\\
132.01	0.00502656451609557\\
133.01	0.00502655466311061\\
134.01	0.00502654462107741\\
135.01	0.00502653438636722\\
136.01	0.00502652395528167\\
137.01	0.00502651332405101\\
138.01	0.00502650248883249\\
139.01	0.00502649144570913\\
140.01	0.00502648019068817\\
141.01	0.00502646871969995\\
142.01	0.00502645702859572\\
143.01	0.00502644511314656\\
144.01	0.00502643296904195\\
145.01	0.00502642059188779\\
146.01	0.00502640797720459\\
147.01	0.00502639512042618\\
148.01	0.00502638201689851\\
149.01	0.00502636866187672\\
150.01	0.00502635505052398\\
151.01	0.00502634117790974\\
152.01	0.00502632703900747\\
153.01	0.00502631262869365\\
154.01	0.00502629794174505\\
155.01	0.00502628297283685\\
156.01	0.00502626771654075\\
157.01	0.00502625216732305\\
158.01	0.00502623631954242\\
159.01	0.00502622016744737\\
160.01	0.00502620370517524\\
161.01	0.00502618692674866\\
162.01	0.00502616982607423\\
163.01	0.00502615239693952\\
164.01	0.005026134633011\\
165.01	0.00502611652783171\\
166.01	0.00502609807481878\\
167.01	0.00502607926726106\\
168.01	0.00502606009831574\\
169.01	0.00502604056100712\\
170.01	0.00502602064822275\\
171.01	0.00502600035271122\\
172.01	0.00502597966707928\\
173.01	0.00502595858378888\\
174.01	0.00502593709515498\\
175.01	0.0050259151933413\\
176.01	0.00502589287035859\\
177.01	0.00502587011806067\\
178.01	0.00502584692814166\\
179.01	0.00502582329213277\\
180.01	0.00502579920139933\\
181.01	0.00502577464713643\\
182.01	0.0050257496203664\\
183.01	0.0050257241119352\\
184.01	0.00502569811250916\\
185.01	0.00502567161257017\\
186.01	0.00502564460241322\\
187.01	0.00502561707214224\\
188.01	0.00502558901166555\\
189.01	0.00502556041069319\\
190.01	0.00502553125873202\\
191.01	0.00502550154508167\\
192.01	0.00502547125883069\\
193.01	0.00502544038885155\\
194.01	0.00502540892379753\\
195.01	0.00502537685209701\\
196.01	0.00502534416194964\\
197.01	0.00502531084132103\\
198.01	0.005025276877939\\
199.01	0.00502524225928753\\
200.01	0.00502520697260286\\
201.01	0.00502517100486792\\
202.01	0.00502513434280697\\
203.01	0.00502509697288092\\
204.01	0.00502505888128121\\
205.01	0.00502502005392489\\
206.01	0.00502498047644921\\
207.01	0.00502494013420463\\
208.01	0.00502489901225038\\
209.01	0.00502485709534807\\
210.01	0.00502481436795496\\
211.01	0.00502477081421839\\
212.01	0.00502472641796924\\
213.01	0.00502468116271534\\
214.01	0.00502463503163499\\
215.01	0.00502458800756995\\
216.01	0.00502454007301841\\
217.01	0.00502449121012867\\
218.01	0.00502444140069091\\
219.01	0.00502439062613034\\
220.01	0.0050243388674997\\
221.01	0.00502428610547143\\
222.01	0.00502423232032982\\
223.01	0.00502417749196284\\
224.01	0.00502412159985483\\
225.01	0.005024064623076\\
226.01	0.00502400654027658\\
227.01	0.00502394732967589\\
228.01	0.00502388696905482\\
229.01	0.0050238254357458\\
230.01	0.00502376270662396\\
231.01	0.00502369875809768\\
232.01	0.00502363356609869\\
233.01	0.00502356710607221\\
234.01	0.00502349935296707\\
235.01	0.00502343028122427\\
236.01	0.0050233598647678\\
237.01	0.00502328807699373\\
238.01	0.00502321489075774\\
239.01	0.00502314027836531\\
240.01	0.00502306421156023\\
241.01	0.00502298666151143\\
242.01	0.00502290759880255\\
243.01	0.00502282699341911\\
244.01	0.00502274481473582\\
245.01	0.00502266103150399\\
246.01	0.00502257561183809\\
247.01	0.005022488523203\\
248.01	0.00502239973240027\\
249.01	0.00502230920555365\\
250.01	0.00502221690809563\\
251.01	0.00502212280475224\\
252.01	0.00502202685952898\\
253.01	0.00502192903569476\\
254.01	0.0050218292957669\\
255.01	0.00502172760149603\\
256.01	0.00502162391384869\\
257.01	0.00502151819299161\\
258.01	0.0050214103982748\\
259.01	0.00502130048821416\\
260.01	0.00502118842047428\\
261.01	0.00502107415184989\\
262.01	0.00502095763824812\\
263.01	0.00502083883466974\\
264.01	0.00502071769518958\\
265.01	0.00502059417293812\\
266.01	0.00502046822008034\\
267.01	0.00502033978779554\\
268.01	0.00502020882625775\\
269.01	0.00502007528461342\\
270.01	0.00501993911095965\\
271.01	0.0050198002523227\\
272.01	0.00501965865463546\\
273.01	0.0050195142627137\\
274.01	0.00501936702023292\\
275.01	0.00501921686970466\\
276.01	0.00501906375245166\\
277.01	0.00501890760858225\\
278.01	0.00501874837696522\\
279.01	0.00501858599520404\\
280.01	0.00501842039960989\\
281.01	0.00501825152517415\\
282.01	0.00501807930554093\\
283.01	0.00501790367297823\\
284.01	0.00501772455834978\\
285.01	0.0050175418910846\\
286.01	0.00501735559914695\\
287.01	0.00501716560900614\\
288.01	0.00501697184560448\\
289.01	0.00501677423232486\\
290.01	0.0050165726909586\\
291.01	0.00501636714167152\\
292.01	0.00501615750297024\\
293.01	0.00501594369166666\\
294.01	0.00501572562284251\\
295.01	0.00501550320981382\\
296.01	0.00501527636409247\\
297.01	0.00501504499534952\\
298.01	0.00501480901137572\\
299.01	0.0050145683180425\\
300.01	0.00501432281926165\\
301.01	0.00501407241694434\\
302.01	0.00501381701095937\\
303.01	0.00501355649908949\\
304.01	0.00501329077698911\\
305.01	0.00501301973813854\\
306.01	0.00501274327379982\\
307.01	0.00501246127296886\\
308.01	0.00501217362232905\\
309.01	0.00501188020620328\\
310.01	0.00501158090650419\\
311.01	0.00501127560268378\\
312.01	0.00501096417168221\\
313.01	0.00501064648787578\\
314.01	0.00501032242302336\\
315.01	0.00500999184621224\\
316.01	0.00500965462380213\\
317.01	0.00500931061936889\\
318.01	0.00500895969364617\\
319.01	0.00500860170446727\\
320.01	0.00500823650670423\\
321.01	0.00500786395220709\\
322.01	0.00500748388974067\\
323.01	0.0050070961649207\\
324.01	0.00500670062014944\\
325.01	0.00500629709454859\\
326.01	0.00500588542389132\\
327.01	0.00500546544053379\\
328.01	0.00500503697334403\\
329.01	0.00500459984762995\\
330.01	0.0050041538850663\\
331.01	0.00500369890361948\\
332.01	0.00500323471747099\\
333.01	0.00500276113694053\\
334.01	0.0050022779684046\\
335.01	0.00500178501421676\\
336.01	0.00500128207262463\\
337.01	0.00500076893768534\\
338.01	0.00500024539917948\\
339.01	0.00499971124252294\\
340.01	0.00499916624867749\\
341.01	0.004998610194059\\
342.01	0.00499804285044488\\
343.01	0.00499746398487761\\
344.01	0.00499687335956869\\
345.01	0.00499627073179867\\
346.01	0.00499565585381676\\
347.01	0.00499502847273668\\
348.01	0.0049943883304321\\
349.01	0.00499373516342843\\
350.01	0.00499306870279381\\
351.01	0.00499238867402622\\
352.01	0.00499169479694037\\
353.01	0.00499098678554984\\
354.01	0.00499026434794896\\
355.01	0.00498952718619072\\
356.01	0.00498877499616313\\
357.01	0.00498800746746252\\
358.01	0.004987224283264\\
359.01	0.00498642512019002\\
360.01	0.00498560964817559\\
361.01	0.00498477753033095\\
362.01	0.004983928422801\\
363.01	0.00498306197462264\\
364.01	0.00498217782757774\\
365.01	0.00498127561604488\\
366.01	0.00498035496684642\\
367.01	0.00497941549909322\\
368.01	0.00497845682402568\\
369.01	0.00497747854485262\\
370.01	0.0049764802565847\\
371.01	0.00497546154586643\\
372.01	0.00497442199080368\\
373.01	0.00497336116078744\\
374.01	0.00497227861631527\\
375.01	0.00497117390880699\\
376.01	0.00497004658041888\\
377.01	0.00496889616385164\\
378.01	0.00496772218215657\\
379.01	0.00496652414853671\\
380.01	0.0049653015661442\\
381.01	0.00496405392787294\\
382.01	0.00496278071614793\\
383.01	0.00496148140270978\\
384.01	0.00496015544839467\\
385.01	0.0049588023029107\\
386.01	0.00495742140460907\\
387.01	0.00495601218025018\\
388.01	0.00495457404476648\\
389.01	0.00495310640101909\\
390.01	0.0049516086395499\\
391.01	0.00495008013832966\\
392.01	0.00494852026249938\\
393.01	0.00494692836410812\\
394.01	0.00494530378184428\\
395.01	0.00494364584076202\\
396.01	0.00494195385200213\\
397.01	0.00494022711250749\\
398.01	0.00493846490473222\\
399.01	0.00493666649634533\\
400.01	0.00493483113992809\\
401.01	0.00493295807266601\\
402.01	0.00493104651603314\\
403.01	0.00492909567547131\\
404.01	0.00492710474006213\\
405.01	0.00492507288219215\\
406.01	0.0049229992572118\\
407.01	0.00492088300308632\\
408.01	0.0049187232400408\\
409.01	0.00491651907019639\\
410.01	0.0049142695772005\\
411.01	0.00491197382584832\\
412.01	0.00490963086169665\\
413.01	0.00490723971067068\\
414.01	0.00490479937866128\\
415.01	0.00490230885111476\\
416.01	0.0048997670926143\\
417.01	0.00489717304645162\\
418.01	0.00489452563419164\\
419.01	0.00489182375522598\\
420.01	0.00488906628631888\\
421.01	0.00488625208114297\\
422.01	0.00488337996980528\\
423.01	0.00488044875836417\\
424.01	0.00487745722833499\\
425.01	0.00487440413618674\\
426.01	0.00487128821282793\\
427.01	0.00486810816308112\\
428.01	0.00486486266514787\\
429.01	0.00486155037006109\\
430.01	0.00485816990112722\\
431.01	0.00485471985335676\\
432.01	0.00485119879288194\\
433.01	0.00484760525636404\\
434.01	0.00484393775038724\\
435.01	0.00484019475084038\\
436.01	0.00483637470228587\\
437.01	0.00483247601731603\\
438.01	0.00482849707589545\\
439.01	0.00482443622469048\\
440.01	0.00482029177638453\\
441.01	0.00481606200897971\\
442.01	0.00481174516508364\\
443.01	0.00480733945118229\\
444.01	0.00480284303689729\\
445.01	0.00479825405422835\\
446.01	0.00479357059678031\\
447.01	0.00478879071897395\\
448.01	0.00478391243524159\\
449.01	0.00477893371920487\\
450.01	0.0047738525028374\\
451.01	0.00476866667560869\\
452.01	0.0047633740836119\\
453.01	0.00475797252867356\\
454.01	0.00475245976744409\\
455.01	0.00474683351047193\\
456.01	0.00474109142125671\\
457.01	0.00473523111528438\\
458.01	0.00472925015904235\\
459.01	0.00472314606901494\\
460.01	0.00471691631065769\\
461.01	0.00471055829735238\\
462.01	0.00470406938933889\\
463.01	0.00469744689262706\\
464.01	0.00469068805788581\\
465.01	0.00468379007931023\\
466.01	0.00467675009346508\\
467.01	0.00466956517810593\\
468.01	0.00466223235097598\\
469.01	0.00465474856857865\\
470.01	0.00464711072492618\\
471.01	0.00463931565026219\\
472.01	0.0046313601097596\\
473.01	0.00462324080219216\\
474.01	0.00461495435857925\\
475.01	0.00460649734080428\\
476.01	0.00459786624020519\\
477.01	0.00458905747613741\\
478.01	0.00458006739450801\\
479.01	0.0045708922662815\\
480.01	0.00456152828595533\\
481.01	0.00455197157000686\\
482.01	0.00454221815530865\\
483.01	0.00453226399751377\\
484.01	0.00452210496940937\\
485.01	0.00451173685923857\\
486.01	0.00450115536898976\\
487.01	0.00449035611265384\\
488.01	0.00447933461444723\\
489.01	0.00446808630700198\\
490.01	0.00445660652952146\\
491.01	0.00444489052590145\\
492.01	0.00443293344281625\\
493.01	0.00442073032776978\\
494.01	0.0044082761271104\\
495.01	0.00439556568401039\\
496.01	0.00438259373640875\\
497.01	0.00436935491491723\\
498.01	0.00435584374068995\\
499.01	0.00434205462325613\\
500.01	0.00432798185831573\\
501.01	0.00431361962549771\\
502.01	0.00429896198608176\\
503.01	0.0042840028806827\\
504.01	0.00426873612689873\\
505.01	0.00425315541692276\\
506.01	0.00423725431511873\\
507.01	0.0042210262555614\\
508.01	0.00420446453954272\\
509.01	0.00418756233304362\\
510.01	0.00417031266417407\\
511.01	0.0041527084205811\\
512.01	0.00413474234682745\\
513.01	0.00411640704174263\\
514.01	0.00409769495574692\\
515.01	0.00407859838815333\\
516.01	0.00405910948444813\\
517.01	0.00403922023355392\\
518.01	0.00401892246507926\\
519.01	0.00399820784655814\\
520.01	0.00397706788068469\\
521.01	0.00395549390254862\\
522.01	0.00393347707687628\\
523.01	0.00391100839528585\\
524.01	0.00388807867356233\\
525.01	0.00386467854896303\\
526.01	0.00384079847756167\\
527.01	0.00381642873164236\\
528.01	0.00379155939715641\\
529.01	0.00376618037125456\\
530.01	0.00374028135991029\\
531.01	0.00371385187565177\\
532.01	0.00368688123541993\\
533.01	0.00365935855857643\\
534.01	0.00363127276508319\\
535.01	0.00360261257388014\\
536.01	0.00357336650149188\\
537.01	0.00354352286089486\\
538.01	0.00351306976068174\\
539.01	0.00348199510456496\\
540.01	0.00345028659126204\\
541.01	0.00341793171481548\\
542.01	0.00338491776540139\\
543.01	0.0033512318306892\\
544.01	0.00331686079782177\\
545.01	0.00328179135609021\\
546.01	0.00324601000039032\\
547.01	0.00320950303555117\\
548.01	0.00317225658164214\\
549.01	0.00313425658037082\\
550.01	0.00309548880269848\\
551.01	0.0030559388578128\\
552.01	0.0030155922036114\\
553.01	0.00297443415886602\\
554.01	0.0029324499172527\\
555.01	0.00288962456345301\\
556.01	0.00284594309155133\\
557.01	0.00280139042597268\\
558.01	0.00275595144523016\\
559.01	0.00270961100877377\\
560.01	0.0026623539872585\\
561.01	0.00261416529657479\\
562.01	0.0025650299360136\\
563.01	0.00251493303096204\\
564.01	0.00246385988055685\\
565.01	0.00241179601074453\\
566.01	0.00235872723322304\\
567.01	0.0023046397107587\\
568.01	0.00224952002938472\\
569.01	0.0021933552779938\\
570.01	0.00213613313583009\\
571.01	0.00207784196836236\\
572.01	0.00201847093197674\\
573.01	0.00195801008785242\\
574.01	0.00189645052527396\\
575.01	0.00183378449447307\\
576.01	0.00177000554886907\\
577.01	0.00170510869627301\\
578.01	0.00163909055821012\\
579.01	0.00157194953597514\\
580.01	0.00150368598132186\\
581.01	0.00143430236876668\\
582.01	0.00136380346528834\\
583.01	0.00129219649167196\\
584.01	0.00121949126778043\\
585.01	0.00114570033153021\\
586.01	0.00107083901816671\\
587.01	0.000994925482401572\\
588.01	0.000917980640875388\\
589.01	0.000840028005979048\\
590.01	0.000761093373966642\\
591.01	0.000681204320110056\\
592.01	0.000600389440856771\\
593.01	0.000518677266913736\\
594.01	0.000436094751083286\\
595.01	0.000352665209519579\\
596.01	0.000268405563604855\\
597.01	0.000183322690309106\\
598.01	9.83859705868326e-05\\
599.01	3.18230442563731e-05\\
599.02	3.12770957175551e-05\\
599.03	3.07343620567849e-05\\
599.04	3.0194875217571e-05\\
599.05	2.96586674565693e-05\\
599.06	2.91257713466771e-05\\
599.07	2.85962197801391e-05\\
599.08	2.80700459716933e-05\\
599.09	2.75472834617447e-05\\
599.1	2.70279661195773e-05\\
599.11	2.65121281465865e-05\\
599.12	2.59998040795448e-05\\
599.13	2.54910287939124e-05\\
599.14	2.49858375071712e-05\\
599.15	2.44842657821827e-05\\
599.16	2.39863495305956e-05\\
599.17	2.34921250162837e-05\\
599.18	2.30016288588035e-05\\
599.19	2.2514898036903e-05\\
599.2	2.20319698920508e-05\\
599.21	2.1552882132023e-05\\
599.22	2.10776728344908e-05\\
599.23	2.06063804506721e-05\\
599.24	2.01390438090109e-05\\
599.25	1.96757021188876e-05\\
599.26	1.9216394974363e-05\\
599.27	1.87611623579855e-05\\
599.28	1.83100446445959e-05\\
599.29	1.78630826051952e-05\\
599.3	1.74203174108517e-05\\
599.31	1.69817906366335e-05\\
599.32	1.65475442655914e-05\\
599.33	1.61176206927693e-05\\
599.34	1.56920627292622e-05\\
599.35	1.52709136063186e-05\\
599.36	1.48542169794725e-05\\
599.37	1.4442016932719e-05\\
599.38	1.40343579827368e-05\\
599.39	1.36312859119591e-05\\
599.4	1.3232849938186e-05\\
599.41	1.28390997671604e-05\\
599.42	1.24500855973528e-05\\
599.43	1.20658581248094e-05\\
599.44	1.16864685480531e-05\\
599.45	1.13119685730082e-05\\
599.46	1.09424104179929e-05\\
599.47	1.05778468187639e-05\\
599.48	1.02183310335888e-05\\
599.49	9.86391684839293e-06\\
599.5	9.51465858194112e-06\\
599.51	9.17061109107116e-06\\
599.52	8.83182977599525e-06\\
599.53	8.4983705856221e-06\\
599.54	8.1702900229675e-06\\
599.55	7.84764515058753e-06\\
599.56	7.53049359608453e-06\\
599.57	7.21889355765649e-06\\
599.58	6.91290380970891e-06\\
599.59	6.61258370851688e-06\\
599.6	6.31799319793756e-06\\
599.61	6.02919281519031e-06\\
599.62	5.74624369668701e-06\\
599.63	5.46920758391981e-06\\
599.64	5.19814682940593e-06\\
599.65	4.93312440269685e-06\\
599.66	4.67420389643411e-06\\
599.67	4.42144953247646e-06\\
599.68	4.17492616808582e-06\\
599.69	3.9346993021671e-06\\
599.7	3.70083508157044e-06\\
599.71	3.47340030746116e-06\\
599.72	3.25246244175723e-06\\
599.73	3.03808961360161e-06\\
599.74	2.83035062593855e-06\\
599.75	2.62931496212288e-06\\
599.76	2.43505279260738e-06\\
599.77	2.24763498169432e-06\\
599.78	2.06713309435641e-06\\
599.79	1.89361940311147e-06\\
599.8	1.72716689497872e-06\\
599.81	1.56784927850956e-06\\
599.82	1.41574099085488e-06\\
599.83	1.27091720493813e-06\\
599.84	1.13345383668736e-06\\
599.85	1.00342755231589e-06\\
599.86	8.8091577571392e-07\\
599.87	7.65996695880136e-07\\
599.88	6.58749274441706e-07\\
599.89	5.59253253243699e-07\\
599.9	4.67589162013102e-07\\
599.91	3.83838326099145e-07\\
599.92	3.08082874293444e-07\\
599.93	2.40405746710845e-07\\
599.94	1.80890702777825e-07\\
599.95	1.2962232925906e-07\\
599.96	8.66860484019516e-08\\
599.97	5.21681261331924e-08\\
599.98	2.61556803542173e-08\\
599.99	8.73668930083393e-09\\
600	0\\
};
\end{axis}
\end{tikzpicture}% 
  \caption{Continuous Time w/ nFPC}
\end{subfigure}%
\hfill%
\begin{subfigure}{.45\linewidth}
  \centering
  \setlength\figureheight{\linewidth} 
  \setlength\figurewidth{\linewidth}
  \tikzsetnextfilename{dp_colorbar/dm_dscr_nFPC_z8}
  % This file was created by matlab2tikz.
%
%The latest updates can be retrieved from
%  http://www.mathworks.com/matlabcentral/fileexchange/22022-matlab2tikz-matlab2tikz
%where you can also make suggestions and rate matlab2tikz.
%
\definecolor{mycolor1}{rgb}{0.00000,1.00000,0.14286}%
\definecolor{mycolor2}{rgb}{0.00000,1.00000,0.28571}%
\definecolor{mycolor3}{rgb}{0.00000,1.00000,0.42857}%
\definecolor{mycolor4}{rgb}{0.00000,1.00000,0.57143}%
\definecolor{mycolor5}{rgb}{0.00000,1.00000,0.71429}%
\definecolor{mycolor6}{rgb}{0.00000,1.00000,0.85714}%
\definecolor{mycolor7}{rgb}{0.00000,1.00000,1.00000}%
\definecolor{mycolor8}{rgb}{0.00000,0.87500,1.00000}%
\definecolor{mycolor9}{rgb}{0.00000,0.62500,1.00000}%
\definecolor{mycolor10}{rgb}{0.12500,0.00000,1.00000}%
\definecolor{mycolor11}{rgb}{0.25000,0.00000,1.00000}%
\definecolor{mycolor12}{rgb}{0.37500,0.00000,1.00000}%
\definecolor{mycolor13}{rgb}{0.50000,0.00000,1.00000}%
\definecolor{mycolor14}{rgb}{0.62500,0.00000,1.00000}%
\definecolor{mycolor15}{rgb}{0.75000,0.00000,1.00000}%
\definecolor{mycolor16}{rgb}{0.87500,0.00000,1.00000}%
\definecolor{mycolor17}{rgb}{1.00000,0.00000,1.00000}%
\definecolor{mycolor18}{rgb}{1.00000,0.00000,0.87500}%
\definecolor{mycolor19}{rgb}{1.00000,0.00000,0.62500}%
\definecolor{mycolor20}{rgb}{0.85714,0.00000,0.00000}%
\definecolor{mycolor21}{rgb}{0.71429,0.00000,0.00000}%
%
\begin{tikzpicture}

\begin{axis}[%
width=4.1in,
height=3.803in,
at={(0.809in,0.513in)},
scale only axis,
point meta min=0,
point meta max=1,
every outer x axis line/.append style={black},
every x tick label/.append style={font=\color{black}},
xmin=0,
xmax=600,
every outer y axis line/.append style={black},
every y tick label/.append style={font=\color{black}},
ymin=0,
ymax=0.012,
axis background/.style={fill=white},
axis x line*=bottom,
axis y line*=left,
colormap={mymap}{[1pt] rgb(0pt)=(0,1,0); rgb(7pt)=(0,1,1); rgb(15pt)=(0,0,1); rgb(23pt)=(1,0,1); rgb(31pt)=(1,0,0); rgb(38pt)=(0,0,0)},
colorbar,
colorbar style={separate axis lines,every outer x axis line/.append style={black},every x tick label/.append style={font=\color{black}},every outer y axis line/.append style={black},every y tick label/.append style={font=\color{black}},yticklabels={{-19},{-17},{-15},{-13},{-11},{-9},{-7},{-5},{-3},{-1},{1},{3},{5},{7},{9},{11},{13},{15},{17},{19}}}
]
\addplot [color=green,solid,forget plot]
  table[row sep=crcr]{%
1	0.00409784475759984\\
2	0.00409792030352774\\
3	0.00409799759515541\\
4	0.00409807667318205\\
5	0.00409815757926939\\
6	0.00409824035606549\\
7	0.00409832504722819\\
8	0.00409841169745005\\
9	0.00409850035248332\\
10	0.00409859105916587\\
11	0.00409868386544764\\
12	0.00409877882041777\\
13	0.00409887597433279\\
14	0.0040989753786449\\
15	0.0040990770860314\\
16	0.00409918115042499\\
17	0.00409928762704437\\
18	0.00409939657242646\\
19	0.00409950804445848\\
20	0.00409962210241159\\
21	0.00409973880697533\\
22	0.00409985822029268\\
23	0.00409998040599627\\
24	0.00410010542924576\\
25	0.00410023335676568\\
26	0.00410036425688493\\
27	0.00410049819957701\\
28	0.00410063525650136\\
29	0.00410077550104588\\
30	0.00410091900837079\\
31	0.00410106585545345\\
32	0.00410121612113476\\
33	0.00410136988616654\\
34	0.0041015272332605\\
35	0.00410168824713853\\
36	0.00410185301458445\\
37	0.00410202162449739\\
38	0.00410219416794652\\
39	0.00410237073822733\\
40	0.00410255143092025\\
41	0.00410273634395007\\
42	0.00410292557764804\\
43	0.00410311923481519\\
44	0.00410331742078817\\
45	0.00410352024350664\\
46	0.00410372781358327\\
47	0.00410394024437562\\
48	0.0041041576520604\\
49	0.00410438015571046\\
50	0.00410460787737368\\
51	0.00410484094215535\\
52	0.00410507947830251\\
53	0.00410532361729158\\
54	0.00410557349391918\\
55	0.00410582924639567\\
56	0.00410609101644245\\
57	0.00410635894939252\\
58	0.00410663319429459\\
59	0.00410691390402163\\
60	0.00410720123538273\\
61	0.00410749534923982\\
62	0.0041077964106286\\
63	0.00410810458888434\\
64	0.0041084200577726\\
65	0.00410874299562529\\
66	0.00410907358548255\\
67	0.00410941201524031\\
68	0.00410975847780406\\
69	0.0041101131712496\\
70	0.00411047629899072\\
71	0.00411084806995413\\
72	0.00411122869876268\\
73	0.00411161840592728\\
74	0.00411201741804687\\
75	0.00411242596801909\\
76	0.00411284429526073\\
77	0.00411327264593887\\
78	0.00411371127321411\\
79	0.00411416043749602\\
80	0.00411462040671191\\
81	0.00411509145658996\\
82	0.00411557387095742\\
83	0.00411606794205526\\
84	0.00411657397087042\\
85	0.00411709226748659\\
86	0.00411762315145538\\
87	0.00411816695218918\\
88	0.00411872400937669\\
89	0.00411929467342452\\
90	0.00411987930592472\\
91	0.00412047828015157\\
92	0.00412109198158957\\
93	0.00412172080849527\\
94	0.00412236517249433\\
95	0.0041230254992186\\
96	0.00412370222898441\\
97	0.00412439581751649\\
98	0.0041251067367204\\
99	0.00412583547550783\\
100	0.00412658254067848\\
101	0.00412734845786299\\
102	0.00412813377253208\\
103	0.00412893905107647\\
104	0.00412976488196353\\
105	0.00413061187697616\\
106	0.00413148067254064\\
107	0.00413237193114969\\
108	0.00413328634288785\\
109	0.00413422462706693\\
110	0.00413518753397913\\
111	0.00413617584677592\\
112	0.00413719038348127\\
113	0.00413823199914742\\
114	0.00413930158816231\\
115	0.00414040008671661\\
116	0.0041415284754392\\
117	0.00414268778220749\\
118	0.00414387908514009\\
119	0.00414510351577602\\
120	0.004146362262443\\
121	0.00414765657381456\\
122	0.00414898776265122\\
123	0.00415035720971535\\
124	0.00415176636784119\\
125	0.00415321676613401\\
126	0.00415471001425678\\
127	0.00415624780674915\\
128	0.00415783192730104\\
129	0.00415946425287718\\
130	0.00416114675755497\\
131	0.00416288151589485\\
132	0.00416467070560643\\
133	0.00416651660920656\\
134	0.0041684216142744\\
135	0.00417038821180032\\
136	0.00417241899198365\\
137	0.00417451663665705\\
138	0.00417668390729333\\
139	0.00417892362726813\\
140	0.00418123865669847\\
141	0.00418363185772974\\
142	0.00418610604758351\\
143	0.00418866393596887\\
144	0.00419130804257029\\
145	0.00419404058920214\\
146	0.00419686335980469\\
147	0.00419977751967251\\
148	0.00420278338304558\\
149	0.00420588011530566\\
150	0.0042090653522319\\
151	0.00421233471331403\\
152	0.00421568117537134\\
153	0.00421906581522593\\
154	0.00422248597450611\\
155	0.00422594201091482\\
156	0.00422943428503874\\
157	0.00423296316033096\\
158	0.00423652900308989\\
159	0.00424013218243538\\
160	0.00424377307028067\\
161	0.00424745204130091\\
162	0.00425116947289712\\
163	0.00425492574515596\\
164	0.00425872124080423\\
165	0.00426255634515837\\
166	0.0042664314460684\\
167	0.00427034693385525\\
168	0.00427430320124195\\
169	0.00427830064327726\\
170	0.00428233965725188\\
171	0.00428642064260612\\
172	0.00429054400082835\\
173	0.00429471013534397\\
174	0.00429891945139335\\
175	0.00430317235589869\\
176	0.00430746925731783\\
177	0.00431181056548482\\
178	0.00431619669143569\\
179	0.00432062804721811\\
180	0.00432510504568366\\
181	0.00432962810026109\\
182	0.00433419762470916\\
183	0.00433881403284646\\
184	0.00434347773825736\\
185	0.00434818915397096\\
186	0.00435294869211118\\
187	0.00435775676351291\\
188	0.00436261377730807\\
189	0.00436752014046877\\
190	0.00437247625731131\\
191	0.00437748252895453\\
192	0.00438253935272789\\
193	0.00438764712152498\\
194	0.00439280622309771\\
195	0.00439801703928436\\
196	0.00440327994516592\\
197	0.00440859530814329\\
198	0.00441396348692798\\
199	0.00441938483043717\\
200	0.00442485967658419\\
201	0.00443038835095385\\
202	0.00443597116535115\\
203	0.00444160841621043\\
204	0.00444730038285186\\
205	0.00445304732556862\\
206	0.00445884948352894\\
207	0.00446470707247404\\
208	0.00447062028219071\\
209	0.00447658927373774\\
210	0.0044826141763994\\
211	0.00448869508434091\\
212	0.00449483205293488\\
213	0.0045010250947277\\
214	0.00450727417500996\\
215	0.00451357920695347\\
216	0.00451994004627456\\
217	0.00452635648537871\\
218	0.00453282824694102\\
219	0.00453935497687283\\
220	0.00454593623662294\\
221	0.00455257149476086\\
222	0.00455926011778804\\
223	0.00456600136012463\\
224	0.00457279435322196\\
225	0.00457963809375516\\
226	0.00458653143086177\\
227	0.00459347305240365\\
228	0.00460046147025117\\
229	0.00460749500461732\\
230	0.00461457176750347\\
231	0.00462168964535071\\
232	0.00462884628093494\\
233	0.0046360390539445\\
234	0.00464326505527298\\
235	0.0046505210069576\\
236	0.00465780349876262\\
237	0.00466510876667863\\
238	0.00467243266248855\\
239	0.00467977062581579\\
240	0.00468711765648356\\
241	0.00469446828812083\\
242	0.00470181656430447\\
243	0.00470915601898387\\
244	0.00471647966353329\\
245	0.00472377998355318\\
246	0.00473104894955389\\
247	0.00473827804697045\\
248	0.0047454583326852\\
249	0.00475258052754034\\
250	0.00475963515751682\\
251	0.00476661276115319\\
252	0.00477350419045295\\
253	0.00478030099699438\\
254	0.00478699576098433\\
255	0.00479358310053279\\
256	0.00480006068343935\\
257	0.00480643056821392\\
258	0.00481270097541078\\
259	0.00481890168130836\\
260	0.00482513036488285\\
261	0.00483138572166826\\
262	0.00483766632757198\\
263	0.00484397062201996\\
264	0.00485029688657508\\
265	0.00485664321544166\\
266	0.00486300746246877\\
267	0.00486938705377548\\
268	0.0048757800140581\\
269	0.00488218440010652\\
270	0.00488859816855402\\
271	0.00489501917381586\\
272	0.00490144516664088\\
273	0.00490787379348034\\
274	0.00491430259692962\\
275	0.00492072901756244\\
276	0.00492715039755593\\
277	0.00493356398660151\\
278	0.00493996695070905\\
279	0.00494635638464548\\
280	0.00495272932888653\\
281	0.00495908279218016\\
282	0.00496541378042456\\
283	0.00497171933271478\\
284	0.00497799656666794\\
285	0.00498424273248062\\
286	0.00499045527173479\\
287	0.00499663188759957\\
288	0.00500276885455785\\
289	0.00500886179727862\\
290	0.00501490622294663\\
291	0.0050208975384141\\
292	0.00502683107286921\\
293	0.00503270210805545\\
294	0.00503850591519979\\
295	0.00504423779490658\\
296	0.00504989312489541\\
297	0.00505546741658601\\
298	0.00506095638156396\\
299	0.00506635600894974\\
300	0.00507166265461885\\
301	0.00507687314304434\\
302	0.00508198488219067\\
303	0.00508699599120746\\
304	0.00509190543908644\\
305	0.00509671318596088\\
306	0.00510142031598759\\
307	0.00510602917595886\\
308	0.00511054350191797\\
309	0.00511496851579151\\
310	0.00511931096968429\\
311	0.00512357910642704\\
312	0.00512778249307162\\
313	0.00513193166910957\\
314	0.00513603753568379\\
315	0.00514011038915024\\
316	0.00514415339170176\\
317	0.00514816558872805\\
318	0.00515214555600522\\
319	0.00515609192551052\\
320	0.00516000339726006\\
321	0.0051638787522915\\
322	0.00516771686684419\\
323	0.00517151672770696\\
324	0.00517527744867228\\
325	0.00517899828803098\\
326	0.00518267866697795\\
327	0.00518631818872863\\
328	0.00518991665801778\\
329	0.00519347410061639\\
330	0.00519699078257065\\
331	0.00520046722872537\\
332	0.00520390423989118\\
333	0.00520730290787344\\
334	0.00521066462749302\\
335	0.0052139911047977\\
336	0.00521728436050494\\
337	0.00522054672698069\\
338	0.00522378083724281\\
339	0.00522698960436026\\
340	0.00523017618951947\\
341	0.00523334395679822\\
342	0.00523649641238523\\
343	0.00523963712698979\\
344	0.00524276964205906\\
345	0.00524589736136628\\
346	0.00524902343209221\\
347	0.00525215062342403\\
348	0.00525528121650131\\
349	0.0052584169280298\\
350	0.00526155892451648\\
351	0.00526470821934065\\
352	0.0052678659099201\\
353	0.00527103317704476\\
354	0.0052742112832648\\
355	0.00527740157022393\\
356	0.00528060545483306\\
357	0.00528382442418496\\
358	0.00528706002911852\\
359	0.0052903138763541\\
360	0.00529358761915325\\
361	0.00529688294650302\\
362	0.0053002015708826\\
363	0.00530354521475302\\
364	0.00530691559602828\\
365	0.00531031441291814\\
366	0.00531374332865523\\
367	0.00531720395678034\\
368	0.00532069784783065\\
369	0.00532422647844399\\
370	0.00532779124403392\\
371	0.00533139345626187\\
372	0.00533503434648635\\
373	0.00533871507609726\\
374	0.00534243675401099\\
375	0.00534620046040236\\
376	0.00535000727230451\\
377	0.00535385827275344\\
378	0.0053577545480272\\
379	0.00536169718486027\\
380	0.00536568726769117\\
381	0.00536972587601087\\
382	0.00537381408188833\\
383	0.00537795294775574\\
384	0.00538214352454116\\
385	0.00538638685023603\\
386	0.00539068394898272\\
387	0.00539503583075542\\
388	0.00539944349168897\\
389	0.00540390791508142\\
390	0.00540843007305335\\
391	0.0054130109287946\\
392	0.00541765143926053\\
393	0.00542235255810581\\
394	0.00542711523856707\\
395	0.00543194043594388\\
396	0.00543682910931025\\
397	0.00544178222215611\\
398	0.00544680074245792\\
399	0.00545188564276676\\
400	0.00545703790040008\\
401	0.00546225849773876\\
402	0.00546754842262698\\
403	0.00547290866886857\\
404	0.00547834023680947\\
405	0.00548384413399068\\
406	0.00548942137585202\\
407	0.00549507298646269\\
408	0.00550079999925233\\
409	0.00550660345771517\\
410	0.00551248441606048\\
411	0.00551844393979057\\
412	0.00552448310619344\\
413	0.00553060300475403\\
414	0.00553680473750273\\
415	0.00554308941933771\\
416	0.00554945817835165\\
417	0.00555591215616768\\
418	0.00556245250828261\\
419	0.00556908040441308\\
420	0.00557579702884114\\
421	0.0055826035807562\\
422	0.00558950127459017\\
423	0.00559649134034323\\
424	0.00560357502389917\\
425	0.00561075358732925\\
426	0.00561802830918566\\
427	0.00562540048478577\\
428	0.00563287142648992\\
429	0.00564044246397542\\
430	0.00564811494450953\\
431	0.00565589023322228\\
432	0.00566376971337998\\
433	0.00567175478665855\\
434	0.00567984687341632\\
435	0.00568804741296627\\
436	0.00569635786384754\\
437	0.00570477970409607\\
438	0.00571331443151478\\
439	0.00572196356394295\\
440	0.00573072863952553\\
441	0.00573961121698208\\
442	0.00574861287587567\\
443	0.00575773521688192\\
444	0.00576697986205737\\
445	0.00577634845510774\\
446	0.00578584266165521\\
447	0.00579546416950444\\
448	0.00580521468890684\\
449	0.00581509595282272\\
450	0.00582510971718042\\
451	0.00583525776113227\\
452	0.00584554188730608\\
453	0.0058559639220517\\
454	0.00586652571568156\\
455	0.00587722914270388\\
456	0.00588807610204783\\
457	0.00589906851727852\\
458	0.00591020833680067\\
459	0.00592149753404935\\
460	0.00593293810766529\\
461	0.00594453208165292\\
462	0.00595628150551875\\
463	0.00596818845438693\\
464	0.00598025502908955\\
465	0.00599248335622785\\
466	0.00600487558820092\\
467	0.00601743390319778\\
468	0.00603016050514805\\
469	0.00604305762362689\\
470	0.00605612751370817\\
471	0.00606937245576052\\
472	0.00608279475517958\\
473	0.00609639674204979\\
474	0.00611018077072839\\
475	0.00612414921934346\\
476	0.00613830448919851\\
477	0.00615264900407409\\
478	0.00616718520941869\\
479	0.00618191557141915\\
480	0.00619684257594279\\
481	0.00621196872734266\\
482	0.00622729654711955\\
483	0.00624282857243494\\
484	0.00625856735447267\\
485	0.00627451545665015\\
486	0.00629067545268511\\
487	0.00630704992453122\\
488	0.00632364146020511\\
489	0.00634045265154001\\
490	0.00635748609191801\\
491	0.00637474437405498\\
492	0.00639223008794037\\
493	0.00640994581907178\\
494	0.0064278941471719\\
495	0.00644607764563728\\
496	0.00646449888204806\\
497	0.00648316042016958\\
498	0.00650206482400709\\
499	0.00652121466464175\\
500	0.00654061253078801\\
501	0.00656026104428372\\
502	0.00658016288206822\\
503	0.00660032080664221\\
504	0.00662073770756038\\
505	0.00664141665721423\\
506	0.00666236098506164\\
507	0.00668357437559948\\
508	0.00670506099682256\\
509	0.00672682566774802\\
510	0.00674886776502838\\
511	0.00677119585749369\\
512	0.00679382297813684\\
513	0.00681676657528694\\
514	0.00684004385018383\\
515	0.00686366344368807\\
516	0.00688763434733743\\
517	0.00691196385407253\\
518	0.00693665747087436\\
519	0.00696171964759408\\
520	0.00698715326698022\\
521	0.00701295919276243\\
522	0.00703913570363854\\
523	0.00706567763267601\\
524	0.00709257527891002\\
525	0.00711981302177701\\
526	0.0071473288879066\\
527	0.00717508630917037\\
528	0.00720315062282531\\
529	0.00723170825117882\\
530	0.00726080771265837\\
531	0.00729050452473134\\
532	0.00732083845906349\\
533	0.00735185058042227\\
534	0.00738358424139605\\
535	0.00741608727970591\\
536	0.0074494127711794\\
537	0.00748362000688747\\
538	0.0075187754227737\\
539	0.00755495371725533\\
540	0.00759223914403396\\
541	0.00763072064444217\\
542	0.00767050041992187\\
543	0.00771053336520066\\
544	0.00774814489342967\\
545	0.00778501922575503\\
546	0.00782234883417097\\
547	0.00786028395490776\\
548	0.00789883697694226\\
549	0.00793799584592659\\
550	0.00797774149567381\\
551	0.00801804850376936\\
552	0.00805888399757698\\
553	0.00810020597452262\\
554	0.00814196096787807\\
555	0.00818408040193254\\
556	0.00822647299553078\\
557	0.00826814713066404\\
558	0.00830933123458027\\
559	0.00835081763203065\\
560	0.00839260575580856\\
561	0.00843466894641207\\
562	0.00847698026063976\\
563	0.00851951301986419\\
564	0.00856218669075257\\
565	0.00860460466750246\\
566	0.00864738981450226\\
567	0.00869053926768215\\
568	0.00873403631435603\\
569	0.00877786229262354\\
570	0.00882199645485595\\
571	0.0088664158231443\\
572	0.00891109504152541\\
573	0.00895600622686392\\
574	0.0090011188209025\\
575	0.00904639944683076\\
576	0.00909181177477994\\
577	0.00913731640198181\\
578	0.00918287075500257\\
579	0.00922842902355833\\
580	0.0092739421380386\\
581	0.00931935780612553\\
582	0.00936462062793334\\
583	0.00940967231402828\\
584	0.00945445203657523\\
585	0.0094988969504773\\
586	0.00954294292772307\\
587	0.00958652555102227\\
588	0.00962958140299686\\
589	0.00967204963943729\\
590	0.0097138736812812\\
591	0.0097550024176802\\
592	0.00979538909575469\\
593	0.00983498278816331\\
594	0.00987369853931868\\
595	0.00991118387968948\\
596	0.00994658651044256\\
597	0.00997788999445116\\
598	0.010000292044645\\
599	0\\
600	0\\
};
\addplot [color=mycolor1,solid,forget plot]
  table[row sep=crcr]{%
1	0.00409772721480342\\
2	0.00409779894368203\\
3	0.00409787229004294\\
4	0.00409794729009925\\
5	0.00409802398086176\\
6	0.00409810240015601\\
7	0.00409818258663963\\
8	0.00409826457981992\\
9	0.00409834842007161\\
10	0.00409843414865529\\
11	0.00409852180773607\\
12	0.00409861144040237\\
13	0.00409870309068543\\
14	0.00409879680357872\\
15	0.00409889262505827\\
16	0.00409899060210289\\
17	0.00409909078271514\\
18	0.00409919321594221\\
19	0.00409929795189766\\
20	0.00409940504178319\\
21	0.00409951453791119\\
22	0.00409962649372711\\
23	0.00409974096383279\\
24	0.00409985800400979\\
25	0.00409997767124344\\
26	0.00410010002374688\\
27	0.00410022512098599\\
28	0.00410035302370438\\
29	0.00410048379394882\\
30	0.00410061749509548\\
31	0.0041007541918758\\
32	0.00410089395040352\\
33	0.00410103683820191\\
34	0.00410118292423112\\
35	0.0041013322789164\\
36	0.00410148497417633\\
37	0.00410164108345169\\
38	0.00410180068173495\\
39	0.00410196384559975\\
40	0.00410213065323089\\
41	0.00410230118445509\\
42	0.00410247552077149\\
43	0.00410265374538353\\
44	0.00410283594323015\\
45	0.00410302220101832\\
46	0.0041032126072553\\
47	0.00410340725228164\\
48	0.00410360622830462\\
49	0.00410380962943171\\
50	0.00410401755170488\\
51	0.00410423009313479\\
52	0.00410444735373589\\
53	0.00410466943556137\\
54	0.00410489644273873\\
55	0.00410512848150574\\
56	0.00410536566024641\\
57	0.00410560808952759\\
58	0.00410585588213612\\
59	0.00410610915311544\\
60	0.00410636801980324\\
61	0.00410663260186913\\
62	0.00410690302135278\\
63	0.00410717940270171\\
64	0.00410746187281014\\
65	0.00410775056105755\\
66	0.00410804559934759\\
67	0.00410834712214699\\
68	0.0041086552665253\\
69	0.00410897017219387\\
70	0.0041092919815457\\
71	0.00410962083969529\\
72	0.00410995689451837\\
73	0.00411030029669195\\
74	0.00411065119973452\\
75	0.00411100976004612\\
76	0.00411137613694869\\
77	0.00411175049272651\\
78	0.00411213299266626\\
79	0.00411252380509742\\
80	0.00411292310143299\\
81	0.00411333105620942\\
82	0.00411374784712704\\
83	0.00411417365509031\\
84	0.00411460866424798\\
85	0.00411505306203289\\
86	0.00411550703920199\\
87	0.00411597078987582\\
88	0.00411644451157802\\
89	0.00411692840527417\\
90	0.00411742267541044\\
91	0.00411792752995182\\
92	0.00411844318041973\\
93	0.00411896984192869\\
94	0.00411950773322249\\
95	0.0041200570767091\\
96	0.00412061809849419\\
97	0.00412119102841346\\
98	0.00412177610006287\\
99	0.00412237355082669\\
100	0.00412298362190276\\
101	0.0041236065583249\\
102	0.00412424260898072\\
103	0.00412489202662549\\
104	0.00412555506788985\\
105	0.00412623199328115\\
106	0.0041269230671765\\
107	0.00412762855780644\\
108	0.00412834873722706\\
109	0.00412908388127877\\
110	0.0041298342695286\\
111	0.00413060018519369\\
112	0.00413138191504216\\
113	0.00413217974926719\\
114	0.00413299398132975\\
115	0.00413382490776419\\
116	0.00413467282794\\
117	0.004135538043773\\
118	0.00413642085937625\\
119	0.00413732158064065\\
120	0.00413824051473449\\
121	0.00413917796950705\\
122	0.00414013425278189\\
123	0.00414110967152146\\
124	0.00414210453084355\\
125	0.00414311913286581\\
126	0.00414415377535373\\
127	0.00414520875014204\\
128	0.00414628434129883\\
129	0.0041473808229959\\
130	0.00414849845704804\\
131	0.00414963749007916\\
132	0.00415079815027464\\
133	0.00415198064367354\\
134	0.00415318514996139\\
135	0.00415441181772467\\
136	0.00415566075913799\\
137	0.00415693204407067\\
138	0.00415822569362178\\
139	0.00415954167312933\\
140	0.00416087988475102\\
141	0.00416224015978999\\
142	0.0041636222510447\\
143	0.00416502582561039\\
144	0.00416645045876467\\
145	0.00416789562984994\\
146	0.00416936072144634\\
147	0.00417084502364335\\
148	0.00417234774590553\\
149	0.00417386803993707\\
150	0.00417540503813451\\
151	0.0041769579138338\\
152	0.00417852597327299\\
153	0.00418010928885501\\
154	0.00418170799276501\\
155	0.00418332221827263\\
156	0.00418495209976837\\
157	0.0041865977728027\\
158	0.00418825937412943\\
159	0.00418993704175247\\
160	0.00419163091497726\\
161	0.00419334113446629\\
162	0.00419506784230025\\
163	0.00419681118204391\\
164	0.00419857129881815\\
165	0.00420034833937799\\
166	0.00420214245219747\\
167	0.00420395378756188\\
168	0.00420578249766757\\
169	0.00420762873673033\\
170	0.00420949266110283\\
171	0.00421137442940206\\
172	0.00421327420264693\\
173	0.00421519214440761\\
174	0.00421712842096694\\
175	0.00421908320149493\\
176	0.00422105665823793\\
177	0.00422304896672268\\
178	0.00422506030597716\\
179	0.00422709085876897\\
180	0.00422914081186331\\
181	0.00423121035630107\\
182	0.00423329968769973\\
183	0.00423540900657803\\
184	0.00423753851870656\\
185	0.00423968843548626\\
186	0.00424185897435704\\
187	0.00424405035923917\\
188	0.00424626282100948\\
189	0.00424849659801558\\
190	0.00425075193663102\\
191	0.00425302909185485\\
192	0.00425532832795867\\
193	0.00425764991918607\\
194	0.00425999415050744\\
195	0.00426236131843595\\
196	0.00426475173190895\\
197	0.00426716571324103\\
198	0.00426960359915399\\
199	0.00427206574189107\\
200	0.00427455251042226\\
201	0.00427706429174896\\
202	0.00427960149231631\\
203	0.00428216453954314\\
204	0.00428475388347959\\
205	0.00428736999860489\\
206	0.00429001338577685\\
207	0.00429268457434783\\
208	0.0042953841244628\\
209	0.00429811262955571\\
210	0.00430087071906375\\
211	0.00430365906137983\\
212	0.00430647836706625\\
213	0.0043093293923543\\
214	0.00431221294295856\\
215	0.0043151298782353\\
216	0.00431808111571905\\
217	0.00432106763607472\\
218	0.00432409048850475\\
219	0.00432715079665611\\
220	0.00433024976507524\\
221	0.0043333886862628\\
222	0.00433656894838491\\
223	0.00433979204370076\\
224	0.00434305957777054\\
225	0.00434637327951035\\
226	0.00434973501216295\\
227	0.00435314678525194\\
228	0.00435661076758406\\
229	0.00436012930135168\\
230	0.00436370491735915\\
231	0.00436734035132117\\
232	0.00437103856100468\\
233	0.00437480274356612\\
234	0.00437863635279915\\
235	0.00438254314485878\\
236	0.00438652718250342\\
237	0.00439059285877664\\
238	0.00439474492172882\\
239	0.00439898849948371\\
240	0.00440332912502909\\
241	0.00440777275985302\\
242	0.0044123258151979\\
243	0.00441699516925015\\
244	0.00442178817798137\\
245	0.00442671267656985\\
246	0.00443177696730284\\
247	0.00443698978851966\\
248	0.0044423602574157\\
249	0.00444789777725692\\
250	0.00445361189653253\\
251	0.00445951210304723\\
252	0.00446560752534391\\
253	0.00447190650305558\\
254	0.00447841604892367\\
255	0.00448514108665174\\
256	0.00449208342506626\\
257	0.00449924038604222\\
258	0.00450660297522872\\
259	0.00451414025698812\\
260	0.0045217525252719\\
261	0.00452944028791647\\
262	0.00453720402905694\\
263	0.00454504420549145\\
264	0.00455296124256425\\
265	0.00456095552955954\\
266	0.00456902741892628\\
267	0.00457717730623837\\
268	0.00458540556282764\\
269	0.00459371251732887\\
270	0.00460209845009204\\
271	0.00461056358702463\\
272	0.00461910809282056\\
273	0.00462773206353394\\
274	0.00463643551845733\\
275	0.00464521839126953\\
276	0.00465408052042462\\
277	0.00466302163876009\\
278	0.00467204136227265\\
279	0.00468113917773462\\
280	0.0046903144267936\\
281	0.00469956626631459\\
282	0.00470889371913389\\
283	0.00471829568279564\\
284	0.00472777082691935\\
285	0.00473731755503774\\
286	0.00474693394218696\\
287	0.00475661756059487\\
288	0.00476636552940042\\
289	0.00477617451557519\\
290	0.00478604067846271\\
291	0.0047959596220509\\
292	0.0048059263359506\\
293	0.00481593480188697\\
294	0.00482597820079109\\
295	0.0048360488286401\\
296	0.00484613800659609\\
297	0.00485623598621501\\
298	0.00486633185105257\\
299	0.00487641341673388\\
300	0.00488646713248571\\
301	0.00489647798800039\\
302	0.00490642942862883\\
303	0.00491630326797561\\
304	0.00492607946572709\\
305	0.00493573738047742\\
306	0.00494525536311096\\
307	0.00495461088938635\\
308	0.00496378103654142\\
309	0.00497274319842527\\
310	0.00498147612422024\\
311	0.0049899613955635\\
312	0.00499818550441449\\
313	0.00500614280671684\\
314	0.0050138383289735\\
315	0.00502129149402076\\
316	0.00502870260229165\\
317	0.00503610288391933\\
318	0.00504348774286446\\
319	0.00505085233759865\\
320	0.0050581915765222\\
321	0.00506550011410386\\
322	0.00507277234802151\\
323	0.00508000242071015\\
324	0.00508718422583651\\
325	0.00509431141980478\\
326	0.00510137743966654\\
327	0.00510837552893968\\
328	0.00511529877420358\\
329	0.00512214015415874\\
330	0.00512889259727722\\
331	0.00513554905037802\\
332	0.00514210256268315\\
333	0.00514854638708865\\
334	0.00515487409757838\\
335	0.00516107971488429\\
336	0.00516715784227585\\
337	0.00517310383639541\\
338	0.00517891399802226\\
339	0.00518458578150427\\
340	0.0051901180217842\\
341	0.00519551118469681\\
342	0.00520076765133354\\
343	0.00520589199788326\\
344	0.0052108912081264\\
345	0.00521577481157448\\
346	0.0052205548933601\\
347	0.00522524589882257\\
348	0.00522986412393199\\
349	0.0052344267388042\\
350	0.00523894945645597\\
351	0.00524343605940803\\
352	0.00524788560074577\\
353	0.00525229734303664\\
354	0.00525667078600199\\
355	0.00526100569488241\\
356	0.0052653021289482\\
357	0.00526956046970549\\
358	0.00527378144828387\\
359	0.00527796617157201\\
360	0.00528211614594839\\
361	0.00528623329710285\\
362	0.00529031998433898\\
363	0.00529437900711757\\
364	0.00529841360059086\\
365	0.00530242741703889\\
366	0.00530642449123779\\
367	0.00531040918682558\\
368	0.00531438612091678\\
369	0.00531836006481072\\
370	0.00532233581986472\\
371	0.00532631806978906\\
372	0.00533031121401844\\
373	0.00533431919234488\\
374	0.00533834531948073\\
375	0.00534239216076209\\
376	0.0053464615391883\\
377	0.00535055505978985\\
378	0.0053546744226983\\
379	0.00535882141702043\\
380	0.00536299791282706\\
381	0.00536720585112944\\
382	0.005371447231776\\
383	0.00537572409929912\\
384	0.00538003852682219\\
385	0.00538439259824991\\
386	0.00538878838911245\\
387	0.00539322794662035\\
388	0.0053977132697017\\
389	0.00540224629003136\\
390	0.00540682885531752\\
391	0.00541146271635344\\
392	0.00541614951952548\\
393	0.00542089080652173\\
394	0.00542568802278514\\
395	0.00543054253561985\\
396	0.00543545566152567\\
397	0.00544042869990902\\
398	0.00544546294960872\\
399	0.00545055970798066\\
400	0.00545572026769364\\
401	0.00546094591375229\\
402	0.00546623792085765\\
403	0.00547159755122089\\
404	0.00547702605294676\\
405	0.00548252465909634\\
406	0.00548809458752136\\
407	0.00549373704153303\\
408	0.00549945321142147\\
409	0.00550524427678036\\
410	0.00551111140950813\\
411	0.00551705577726307\\
412	0.00552307854704392\\
413	0.0055291808884754\\
414	0.00553536397632439\\
415	0.00554162899181428\\
416	0.00554797712303811\\
417	0.00555440956521041\\
418	0.00556092752104443\\
419	0.00556753220125155\\
420	0.00557422482515346\\
421	0.00558100662139226\\
422	0.00558787882871715\\
423	0.00559484269682172\\
424	0.00560189948720002\\
425	0.0056090504739877\\
426	0.00561629694475481\\
427	0.00562364020122091\\
428	0.00563108155987295\\
429	0.0056386223524815\\
430	0.00564626392653087\\
431	0.00565400764560148\\
432	0.00566185488974582\\
433	0.00566980705587261\\
434	0.00567786555813514\\
435	0.00568603182831905\\
436	0.00569430731622558\\
437	0.00570269349004649\\
438	0.00571119183672758\\
439	0.00571980386231872\\
440	0.00572853109230956\\
441	0.00573737507195137\\
442	0.00574633736656675\\
443	0.00575541956184987\\
444	0.00576462326416059\\
445	0.00577395010081486\\
446	0.00578340172037297\\
447	0.00579297979292529\\
448	0.00580268601037476\\
449	0.00581252208671591\\
450	0.00582248975830964\\
451	0.00583259078415376\\
452	0.00584282694614889\\
453	0.00585320004935955\\
454	0.00586371192227051\\
455	0.00587436441703796\\
456	0.00588515940973543\\
457	0.00589609880059389\\
458	0.0059071845142356\\
459	0.0059184184999007\\
460	0.00592980273166581\\
461	0.00594133920865384\\
462	0.00595302995523355\\
463	0.00596487702120814\\
464	0.00597688248199155\\
465	0.00598904843877122\\
466	0.00600137701865596\\
467	0.00601387037480744\\
468	0.00602653068655392\\
469	0.00603936015948441\\
470	0.00605236102552181\\
471	0.00606553554297312\\
472	0.00607888599655534\\
473	0.00609241469739513\\
474	0.00610612398300072\\
475	0.00612001621720504\\
476	0.00613409379007817\\
477	0.00614835911780911\\
478	0.00616281464255603\\
479	0.00617746283226529\\
480	0.00619230618046065\\
481	0.00620734720600465\\
482	0.00622258845283574\\
483	0.00623803248968672\\
484	0.00625368190979207\\
485	0.00626953933059434\\
486	0.00628560739346397\\
487	0.00630188876345045\\
488	0.00631838612908839\\
489	0.00633510220228862\\
490	0.00635203971835239\\
491	0.00636920143615627\\
492	0.00638659013856709\\
493	0.00640420863316077\\
494	0.00642205975333513\\
495	0.00644014635992797\\
496	0.00645847134347519\\
497	0.006477037627273\\
498	0.00649584817144181\\
499	0.0065149059782292\\
500	0.00653421409883596\\
501	0.00655377564210219\\
502	0.00657359378545241\\
503	0.00659367178856756\\
504	0.00661401301032988\\
505	0.00663462092967143\\
506	0.00665549917104909\\
507	0.00667665153536294\\
508	0.0066980820372294\\
509	0.00671979494960482\\
510	0.00674179493719375\\
511	0.0067640868956916\\
512	0.0067866759769133\\
513	0.00680956400369565\\
514	0.00683275491124096\\
515	0.00685626196973593\\
516	0.0068800986408156\\
517	0.00690428490325568\\
518	0.00692883527291861\\
519	0.00695375870382165\\
520	0.00697906425632247\\
521	0.00700475949871382\\
522	0.00703085045342747\\
523	0.00705734188569942\\
524	0.00708423681480593\\
525	0.00711153593740922\\
526	0.00713923746350955\\
527	0.00716733497321819\\
528	0.00719581077674005\\
529	0.00722455855703865\\
530	0.00725363372862162\\
531	0.00728313100693132\\
532	0.00731320423426021\\
533	0.00734390257698169\\
534	0.00737527918989368\\
535	0.00740737901823451\\
536	0.00744024906925833\\
537	0.00747393941753061\\
538	0.00750850490350621\\
539	0.00754400577180493\\
540	0.00758050847644705\\
541	0.00761808214679199\\
542	0.00765680105430943\\
543	0.00769676637028498\\
544	0.00773813039112718\\
545	0.00777868539256093\\
546	0.0078168880070114\\
547	0.00785492834535674\\
548	0.00789347294193416\\
549	0.00793261873474102\\
550	0.0079723685954058\\
551	0.00801270533222008\\
552	0.00805360340878692\\
553	0.00809502949182715\\
554	0.00813694106531913\\
555	0.00817928405928756\\
556	0.0082219892856467\\
557	0.00826498033629833\\
558	0.00830749515365131\\
559	0.00834919568350826\\
560	0.00839115862383602\\
561	0.00843340529491727\\
562	0.00847591105247397\\
563	0.00851864731722591\\
564	0.00856158658383183\\
565	0.0086045920915919\\
566	0.00864738978534756\\
567	0.00869053926739589\\
568	0.00873403631432268\\
569	0.00877786229261397\\
570	0.00882199645485205\\
571	0.00886641582314234\\
572	0.00891109504152458\\
573	0.00895600622686346\\
574	0.00900111882090225\\
575	0.00904639944683062\\
576	0.00909181177477987\\
577	0.00913731640198179\\
578	0.00918287075500256\\
579	0.00922842902355833\\
580	0.00927394213803861\\
581	0.00931935780612554\\
582	0.00936462062793335\\
583	0.00940967231402828\\
584	0.00945445203657523\\
585	0.0094988969504773\\
586	0.00954294292772307\\
587	0.00958652555102228\\
588	0.00962958140299686\\
589	0.0096720496394373\\
590	0.0097138736812812\\
591	0.0097550024176802\\
592	0.00979538909575469\\
593	0.00983498278816331\\
594	0.00987369853931868\\
595	0.00991118387968948\\
596	0.00994658651044256\\
597	0.00997788999445116\\
598	0.010000292044645\\
599	0\\
600	0\\
};
\addplot [color=mycolor2,solid,forget plot]
  table[row sep=crcr]{%
1	0.00409747161189642\\
2	0.00409753539813578\\
3	0.00409760055013061\\
4	0.00409766709594947\\
5	0.00409773506418996\\
6	0.00409780448398646\\
7	0.00409787538501809\\
8	0.00409794779751668\\
9	0.00409802175227469\\
10	0.00409809728065326\\
11	0.00409817441458998\\
12	0.00409825318660705\\
13	0.00409833362981889\\
14	0.00409841577794048\\
15	0.00409849966529485\\
16	0.00409858532682111\\
17	0.00409867279808217\\
18	0.00409876211527228\\
19	0.00409885331522501\\
20	0.00409894643542064\\
21	0.00409904151399345\\
22	0.0040991385897394\\
23	0.00409923770212313\\
24	0.00409933889128525\\
25	0.00409944219804905\\
26	0.00409954766392779\\
27	0.00409965533113078\\
28	0.00409976524257016\\
29	0.00409987744186714\\
30	0.00409999197335779\\
31	0.00410010888209922\\
32	0.00410022821387476\\
33	0.0041003500151994\\
34	0.00410047433332463\\
35	0.00410060121624315\\
36	0.00410073071269329\\
37	0.00410086287216279\\
38	0.00410099774489229\\
39	0.00410113538187862\\
40	0.00410127583487759\\
41	0.00410141915640617\\
42	0.00410156539974421\\
43	0.00410171461893575\\
44	0.00410186686878994\\
45	0.00410202220488092\\
46	0.00410218068354748\\
47	0.0041023423618923\\
48	0.00410250729777974\\
49	0.00410267554983383\\
50	0.00410284717743505\\
51	0.0041030222407164\\
52	0.00410320080055873\\
53	0.00410338291858535\\
54	0.00410356865715563\\
55	0.00410375807935773\\
56	0.00410395124900057\\
57	0.00410414823060455\\
58	0.00410434908939137\\
59	0.00410455389127309\\
60	0.00410476270283974\\
61	0.00410497559134594\\
62	0.00410519262469638\\
63	0.0041054138714302\\
64	0.00410563940070404\\
65	0.00410586928227383\\
66	0.00410610358647515\\
67	0.00410634238420284\\
68	0.00410658574688835\\
69	0.00410683374647658\\
70	0.0041070864554006\\
71	0.00410734394655529\\
72	0.00410760629326943\\
73	0.00410787356927631\\
74	0.00410814584868262\\
75	0.00410842320593601\\
76	0.00410870571579089\\
77	0.00410899345327286\\
78	0.00410928649364127\\
79	0.00410958491235057\\
80	0.00410988878500935\\
81	0.0041101981873385\\
82	0.00411051319512743\\
83	0.00411083388418873\\
84	0.00411116033031089\\
85	0.00411149260921013\\
86	0.00411183079648004\\
87	0.00411217496753997\\
88	0.00411252519758179\\
89	0.00411288156151534\\
90	0.00411324413391216\\
91	0.00411361298894819\\
92	0.00411398820034483\\
93	0.0041143698413086\\
94	0.00411475798447002\\
95	0.00411515270182075\\
96	0.00411555406465005\\
97	0.00411596214347976\\
98	0.00411637700799863\\
99	0.00411679872699537\\
100	0.00411722736829093\\
101	0.00411766299867002\\
102	0.0041181056838117\\
103	0.00411855548821906\\
104	0.00411901247514837\\
105	0.00411947670653741\\
106	0.0041199482429331\\
107	0.00412042714341815\\
108	0.0041209134655373\\
109	0.0041214072652224\\
110	0.00412190859671682\\
111	0.00412241751249883\\
112	0.00412293406320339\\
113	0.00412345829754346\\
114	0.00412399026222934\\
115	0.0041245300018866\\
116	0.00412507755897211\\
117	0.00412563297368823\\
118	0.00412619628389468\\
119	0.00412676752501859\\
120	0.00412734672996138\\
121	0.00412793392900395\\
122	0.00412852914970932\\
123	0.00412913241682325\\
124	0.00412974375217359\\
125	0.00413036317456882\\
126	0.00413099069969742\\
127	0.00413162634002976\\
128	0.00413227010472435\\
129	0.00413292199954316\\
130	0.00413358202677864\\
131	0.00413425018519948\\
132	0.00413492647002104\\
133	0.00413561087291054\\
134	0.00413630338203718\\
135	0.00413700398218096\\
136	0.00413771265491638\\
137	0.00413842937889044\\
138	0.00413915413021722\\
139	0.00413988688301409\\
140	0.00414062761010833\\
141	0.00414137628394292\\
142	0.00414213287771258\\
143	0.00414289736675549\\
144	0.00414366973022058\\
145	0.00414444995301399\\
146	0.0041452380280014\\
147	0.00414603395840073\\
148	0.00414683776023207\\
149	0.00414764946459766\\
150	0.00414846911946697\\
151	0.00414929679070874\\
152	0.00415013256280046\\
153	0.00415097652319167\\
154	0.00415182876115608\\
155	0.00415268936787204\\
156	0.00415355843650752\\
157	0.00415443606230929\\
158	0.00415532234269612\\
159	0.00415621737735695\\
160	0.00415712126835361\\
161	0.00415803412022869\\
162	0.00415895604011845\\
163	0.00415988713787103\\
164	0.00416082752617065\\
165	0.00416177732066735\\
166	0.00416273664011272\\
167	0.00416370560650229\\
168	0.00416468434522427\\
169	0.00416567298521502\\
170	0.00416667165912175\\
171	0.00416768050347225\\
172	0.00416869965885266\\
173	0.0041697292700925\\
174	0.00417076948645847\\
175	0.00417182046185595\\
176	0.00417288235503981\\
177	0.00417395532983361\\
178	0.00417503955535845\\
179	0.00417613520627111\\
180	0.00417724246301193\\
181	0.0041783615120632\\
182	0.00417949254621711\\
183	0.0041806357648553\\
184	0.00418179137423853\\
185	0.00418295958780796\\
186	0.00418414062649785\\
187	0.00418533471905979\\
188	0.00418654210239906\\
189	0.00418776302192296\\
190	0.00418899773190199\\
191	0.00419024649584326\\
192	0.00419150958687672\\
193	0.00419278728815461\\
194	0.00419407989326373\\
195	0.00419538770665124\\
196	0.00419671104406349\\
197	0.00419805023299822\\
198	0.00419940561317019\\
199	0.00420077753698996\\
200	0.00420216637005581\\
201	0.00420357249165847\\
202	0.00420499629529858\\
203	0.00420643818921634\\
204	0.0042078985969331\\
205	0.00420937795780344\\
206	0.00421087672757834\\
207	0.00421239537897701\\
208	0.00421393440226728\\
209	0.00421549430585238\\
210	0.0042170756168629\\
211	0.00421867888175132\\
212	0.00422030466688705\\
213	0.00422195355914864\\
214	0.0042236261665097\\
215	0.00422532311861412\\
216	0.00422704506733573\\
217	0.00422879268731582\\
218	0.00423056667647189\\
219	0.00423236775646867\\
220	0.00423419667314141\\
221	0.004236054196859\\
222	0.00423794112281342\\
223	0.0042398582712181\\
224	0.0042418064873953\\
225	0.00424378664172982\\
226	0.0042457996294605\\
227	0.00424784637027782\\
228	0.00424992780768809\\
229	0.00425204490809677\\
230	0.00425419865955228\\
231	0.00425639007007598\\
232	0.0042586201655153\\
233	0.00426088998707641\\
234	0.00426320058987421\\
235	0.00426555303748082\\
236	0.00426794839720525\\
237	0.00427038773444267\\
238	0.00427287210592968\\
239	0.0042754025517495\\
240	0.00427798008592344\\
241	0.00428060568541552\\
242	0.00428328027738017\\
243	0.00428600472449318\\
244	0.00428877980823606\\
245	0.0042916062100587\\
246	0.0042944844904334\\
247	0.00429741506595932\\
248	0.00430039818488352\\
249	0.00430343390169871\\
250	0.0043065220518256\\
251	0.00430966222770246\\
252	0.00431285375894432\\
253	0.00431609570450518\\
254	0.00431938685736227\\
255	0.00432272577187187\\
256	0.00432611082522832\\
257	0.0043295403286064\\
258	0.00433301270856113\\
259	0.0043365270143948\\
260	0.0043400842428449\\
261	0.00434368544098068\\
262	0.00434733170972222\\
263	0.00435102420769006\\
264	0.00435476415560266\\
265	0.00435855284221954\\
266	0.00436239163542775\\
267	0.00436628198165968\\
268	0.00437022541053829\\
269	0.00437422354125197\\
270	0.00437827808949937\\
271	0.0043823908750584\\
272	0.00438656383003684\\
273	0.00439079900786693\\
274	0.00439509859310615\\
275	0.00439946491211647\\
276	0.00440390044467257\\
277	0.00440840783653867\\
278	0.00441298991295949\\
279	0.00441764969279\\
280	0.00442239040296108\\
281	0.00442721550334059\\
282	0.00443212870843422\\
283	0.0044371340008761\\
284	0.00444223565292052\\
285	0.00444743824742085\\
286	0.00445274669560388\\
287	0.00445816627621191\\
288	0.00446370267254473\\
289	0.00446936200821571\\
290	0.00447515088662875\\
291	0.00448107642829197\\
292	0.00448714628446144\\
293	0.00449336870794162\\
294	0.00449975259858886\\
295	0.00450630754965302\\
296	0.00451304389387529\\
297	0.00451997274778971\\
298	0.004527106052044\\
299	0.00453445660470557\\
300	0.00454203808334817\\
301	0.00454986505006974\\
302	0.00455795293147298\\
303	0.00456631796775024\\
304	0.00457497720938437\\
305	0.00458394828581726\\
306	0.00459324913323066\\
307	0.00460289761933221\\
308	0.00461291100682783\\
309	0.00462330520184831\\
310	0.00463409371679639\\
311	0.00464528625435607\\
312	0.00465688678276178\\
313	0.00466889085781066\\
314	0.00468128210320214\\
315	0.00469402777432611\\
316	0.00470691739611307\\
317	0.00471991242935172\\
318	0.00473301002317151\\
319	0.00474620688294381\\
320	0.00475949917767092\\
321	0.00477288264150297\\
322	0.0047863524755296\\
323	0.00479990320100424\\
324	0.00481352857623578\\
325	0.00482722150450891\\
326	0.00484097393982608\\
327	0.0048547768328975\\
328	0.00486861970165124\\
329	0.00488249054713525\\
330	0.00489637589406776\\
331	0.00491026063622077\\
332	0.00492412785784585\\
333	0.00493795855675552\\
334	0.00495173168165891\\
335	0.00496542528834626\\
336	0.00497901542468126\\
337	0.00499247611671266\\
338	0.00500577942968237\\
339	0.00501889565646387\\
340	0.00503179379203509\\
341	0.00504444043675032\\
342	0.005056798614081\\
343	0.00506883063675311\\
344	0.00508049923158478\\
345	0.0050917691756241\\
346	0.00510260962124097\\
347	0.00511299734576308\\
348	0.00512292124026069\\
349	0.00513238847118869\\
350	0.00514145356279036\\
351	0.00515044292812221\\
352	0.00515934757340538\\
353	0.0051681582470874\\
354	0.00517686548853945\\
355	0.00518545969417062\\
356	0.00519393120286888\\
357	0.00520227039350545\\
358	0.00521046779163548\\
359	0.00521851417751173\\
360	0.00522640072029858\\
361	0.00523411914484358\\
362	0.00524166192587091\\
363	0.00524902252478635\\
364	0.00525619569649764\\
365	0.00526317784285559\\
366	0.00526996735421263\\
367	0.00527656497134352\\
368	0.00528297415389167\\
369	0.00528920143293488\\
370	0.0052952567130429\\
371	0.00530115347201255\\
372	0.00530690878823694\\
373	0.00531254308939831\\
374	0.00531807947203472\\
375	0.00532354238354427\\
376	0.00532895418314075\\
377	0.00533432142320478\\
378	0.0053396445335948\\
379	0.00534492439119021\\
380	0.00535016235681699\\
381	0.00535536030967686\\
382	0.0053605206769865\\
383	0.0053656464553484\\
384	0.00537074122179324\\
385	0.00537580913203551\\
386	0.00538085490254376\\
387	0.00538588377271432\\
388	0.00539090144328955\\
389	0.00539591398731883\\
390	0.00540092773045593\\
391	0.00540594909860276\\
392	0.00541098443331722\\
393	0.00541603977941393\\
394	0.00542112065567026\\
395	0.00542623182967747\\
396	0.00543137713273228\\
397	0.00543655937249408\\
398	0.00544178091217529\\
399	0.0054470441437416\\
400	0.00545235153771615\\
401	0.0054577056261782\\
402	0.00546310898322712\\
403	0.00546856420313171\\
404	0.0054740738765625\\
405	0.00547964056552351\\
406	0.0054852667778617\\
407	0.0054909549425352\\
408	0.00549670738714891\\
409	0.00550252631958955\\
410	0.00550841381585341\\
411	0.00551437181627652\\
412	0.00552040213222794\\
413	0.00552650646469684\\
414	0.00553268643478951\\
415	0.00553894362349785\\
416	0.00554527959897722\\
417	0.00555169592110127\\
418	0.00555819413806102\\
419	0.00556477578347433\\
420	0.00557144237415245\\
421	0.00557819540866512\\
422	0.00558503636682932\\
423	0.00559196671021332\\
424	0.00559898788369828\\
425	0.00560610131806688\\
426	0.00561330843349454\\
427	0.00562061064370411\\
428	0.0056280093604178\\
429	0.00563550599761747\\
430	0.00564310197503964\\
431	0.00565079872034391\\
432	0.00565859766997983\\
433	0.00566650026965315\\
434	0.00567450797493938\\
435	0.00568262225203333\\
436	0.00569084457861544\\
437	0.00569917644480915\\
438	0.00570761935419601\\
439	0.00571617482485011\\
440	0.00572484439035047\\
441	0.00573362960073054\\
442	0.0057425320233305\\
443	0.00575155324352999\\
444	0.00576069486535867\\
445	0.00576995851200706\\
446	0.00577934582628691\\
447	0.00578885847108561\\
448	0.00579849812982527\\
449	0.00580826650692118\\
450	0.00581816532823506\\
451	0.00582819634151819\\
452	0.00583836131684128\\
453	0.00584866204700868\\
454	0.00585910034795612\\
455	0.00586967805913298\\
456	0.00588039704387136\\
457	0.00589125918974586\\
458	0.0059022664089277\\
459	0.00591342063853709\\
460	0.00592472384099479\\
461	0.00593617800437307\\
462	0.00594778514274589\\
463	0.00595954729653819\\
464	0.00597146653287447\\
465	0.00598354494592692\\
466	0.0059957846572637\\
467	0.00600818781619825\\
468	0.00602075660013988\\
469	0.00603349321494683\\
470	0.00604639989528218\\
471	0.00605947890497346\\
472	0.00607273253737647\\
473	0.00608616311574416\\
474	0.00609977299360119\\
475	0.00611356455512523\\
476	0.00612754021553649\\
477	0.00614170242149662\\
478	0.00615605365151884\\
479	0.00617059641639156\\
480	0.00618533325961788\\
481	0.00620026675787385\\
482	0.00621539952148958\\
483	0.00623073419495676\\
484	0.00624627345746848\\
485	0.00626202002349724\\
486	0.00627797664341869\\
487	0.0062941461041901\\
488	0.0063105312300946\\
489	0.00632713488356401\\
490	0.00634395996609632\\
491	0.00636100941928663\\
492	0.00637828622599498\\
493	0.00639579341167863\\
494	0.00641353404592295\\
495	0.00643151124421204\\
496	0.00644972816998979\\
497	0.00646818803707225\\
498	0.00648689411248712\\
499	0.0065058497198329\\
500	0.00652505824327141\\
501	0.006544523132295\\
502	0.006564247907443\\
503	0.00658423616718477\\
504	0.00660449159624057\\
505	0.0066250179756797\\
506	0.0066458191952222\\
507	0.00666689926828206\\
508	0.00668826235043057\\
509	0.00670991276214244\\
510	0.00673185501589693\\
511	0.00675409385627773\\
512	0.00677663430915819\\
513	0.00679948178453945\\
514	0.00682264204388124\\
515	0.00684612107772275\\
516	0.00686992525165649\\
517	0.00689405601162315\\
518	0.00691852182093339\\
519	0.00694333742202316\\
520	0.00696851780686047\\
521	0.006994085086136\\
522	0.00702005457319023\\
523	0.00704643675525956\\
524	0.00707324204297694\\
525	0.0071004793565474\\
526	0.00712815559940787\\
527	0.00715627592819553\\
528	0.00718484326852532\\
529	0.00721385844412268\\
530	0.0072433155463926\\
531	0.00727317878533223\\
532	0.00730338078918359\\
533	0.0073339813864525\\
534	0.00736509082672553\\
535	0.00739685283269898\\
536	0.00742932163048537\\
537	0.00746255622633911\\
538	0.00749660705933109\\
539	0.00753152688765596\\
540	0.00756737104112477\\
541	0.00760419757173407\\
542	0.00764206275967574\\
543	0.0076810318607664\\
544	0.00772118083695201\\
545	0.00776262635070966\\
546	0.00780552562472109\\
547	0.00784719944470293\\
548	0.00788651753070341\\
549	0.00792583956011556\\
550	0.00796564674856095\\
551	0.00800603755448718\\
552	0.00804701590222305\\
553	0.00808856038361576\\
554	0.00813063840062174\\
555	0.00817320762573787\\
556	0.00821621465504425\\
557	0.00825959134358754\\
558	0.00830325988408064\\
559	0.00834680859180681\\
560	0.00838920854817912\\
561	0.00843169352148518\\
562	0.00847443197159942\\
563	0.00851741118575558\\
564	0.00856060116250899\\
565	0.0086039734318009\\
566	0.00864737435478978\\
567	0.00869053904167139\\
568	0.00873403631210821\\
569	0.00877786229236826\\
570	0.00882199645478541\\
571	0.00886641582311596\\
572	0.00891109504151141\\
573	0.00895600622685637\\
574	0.00900111882089835\\
575	0.00904639944682846\\
576	0.0090918117747785\\
577	0.00913731640198132\\
578	0.00918287075500235\\
579	0.00922842902355825\\
580	0.00927394213803858\\
581	0.00931935780612553\\
582	0.00936462062793333\\
583	0.00940967231402827\\
584	0.00945445203657523\\
585	0.00949889695047729\\
586	0.00954294292772306\\
587	0.00958652555102227\\
588	0.00962958140299686\\
589	0.00967204963943729\\
590	0.0097138736812812\\
591	0.00975500241768019\\
592	0.00979538909575469\\
593	0.00983498278816331\\
594	0.00987369853931868\\
595	0.00991118387968948\\
596	0.00994658651044256\\
597	0.00997788999445116\\
598	0.010000292044645\\
599	0\\
600	0\\
};
\addplot [color=mycolor3,solid,forget plot]
  table[row sep=crcr]{%
1	0.00409709505721719\\
2	0.00409714790196123\\
3	0.00409720179506604\\
4	0.00409725675550612\\
5	0.00409731280253287\\
6	0.00409736995567597\\
7	0.00409742823474466\\
8	0.00409748765982912\\
9	0.0040975482513015\\
10	0.0040976100298169\\
11	0.00409767301631428\\
12	0.00409773723201712\\
13	0.00409780269843404\\
14	0.00409786943735913\\
15	0.00409793747087224\\
16	0.00409800682133909\\
17	0.00409807751141115\\
18	0.0040981495640253\\
19	0.00409822300240346\\
20	0.00409829785005182\\
21	0.00409837413075997\\
22	0.00409845186859983\\
23	0.00409853108792437\\
24	0.0040986118133659\\
25	0.00409869406983447\\
26	0.00409877788251555\\
27	0.00409886327686802\\
28	0.00409895027862135\\
29	0.00409903891377286\\
30	0.0040991292085846\\
31	0.00409922118957973\\
32	0.00409931488353903\\
33	0.00409941031749665\\
34	0.00409950751873585\\
35	0.00409960651478426\\
36	0.00409970733340878\\
37	0.00409981000261029\\
38	0.00409991455061785\\
39	0.00410002100588264\\
40	0.00410012939707149\\
41	0.00410023975305994\\
42	0.00410035210292515\\
43	0.00410046647593821\\
44	0.00410058290155609\\
45	0.00410070140941337\\
46	0.00410082202931329\\
47	0.00410094479121861\\
48	0.00410106972524196\\
49	0.00410119686163573\\
50	0.00410132623078173\\
51	0.00410145786318031\\
52	0.00410159178943899\\
53	0.00410172804026081\\
54	0.00410186664643225\\
55	0.00410200763881073\\
56	0.00410215104831164\\
57	0.00410229690589511\\
58	0.00410244524255241\\
59	0.00410259608929171\\
60	0.0041027494771238\\
61	0.00410290543704742\\
62	0.00410306400003411\\
63	0.00410322519701274\\
64	0.0041033890588541\\
65	0.00410355561635478\\
66	0.00410372490022111\\
67	0.00410389694105278\\
68	0.00410407176932631\\
69	0.0041042494153785\\
70	0.00410442990938947\\
71	0.00410461328136599\\
72	0.00410479956112467\\
73	0.00410498877827502\\
74	0.00410518096220291\\
75	0.00410537614205391\\
76	0.00410557434671693\\
77	0.00410577560480814\\
78	0.00410597994465524\\
79	0.00410618739428196\\
80	0.00410639798139328\\
81	0.00410661173336094\\
82	0.00410682867720988\\
83	0.00410704883960502\\
84	0.00410727224683913\\
85	0.00410749892482147\\
86	0.00410772889906749\\
87	0.00410796219468941\\
88	0.00410819883638821\\
89	0.00410843884844677\\
90	0.0041086822547244\\
91	0.00410892907865275\\
92	0.00410917934323363\\
93	0.00410943307103807\\
94	0.00410969028420758\\
95	0.00410995100445693\\
96	0.00411021525307929\\
97	0.00411048305095331\\
98	0.00411075441855213\\
99	0.00411102937595528\\
100	0.00411130794286244\\
101	0.00411159013860977\\
102	0.00411187598218897\\
103	0.00411216549226877\\
104	0.00411245868721916\\
105	0.00411275558513829\\
106	0.00411305620388219\\
107	0.00411336056109718\\
108	0.00411366867425543\\
109	0.00411398056069289\\
110	0.00411429623765032\\
111	0.00411461572231712\\
112	0.00411493903187795\\
113	0.00411526618356198\\
114	0.00411559719469509\\
115	0.00411593208275438\\
116	0.00411627086542568\\
117	0.00411661356066322\\
118	0.00411696018675189\\
119	0.00411731076237185\\
120	0.00411766530666538\\
121	0.00411802383930604\\
122	0.00411838638056982\\
123	0.00411875295140849\\
124	0.00411912357352497\\
125	0.00411949826945069\\
126	0.00411987706262516\\
127	0.00412025997747728\\
128	0.00412064703950929\\
129	0.00412103827538233\\
130	0.0041214337130048\\
131	0.00412183338162291\\
132	0.0041222373119138\\
133	0.0041226455360812\\
134	0.00412305808795372\\
135	0.00412347500308553\\
136	0.00412389631885935\\
137	0.00412432207459104\\
138	0.00412475231163499\\
139	0.004125187073489\\
140	0.00412562640589676\\
141	0.00412607035694534\\
142	0.00412651897715428\\
143	0.00412697231955158\\
144	0.00412743043973162\\
145	0.00412789339588805\\
146	0.00412836124881516\\
147	0.0041288340618709\\
148	0.00412931190089873\\
149	0.00412979483411735\\
150	0.00413028293201533\\
151	0.00413077626729512\\
152	0.0041312749144199\\
153	0.0041317789496272\\
154	0.00413228845098237\\
155	0.00413280349843347\\
156	0.00413332417386732\\
157	0.0041338505611666\\
158	0.00413438274626842\\
159	0.00413492081722397\\
160	0.00413546486425925\\
161	0.00413601497983702\\
162	0.00413657125871991\\
163	0.00413713379803445\\
164	0.00413770269733617\\
165	0.00413827805867575\\
166	0.00413885998666593\\
167	0.00413944858854946\\
168	0.00414004397426772\\
169	0.00414064625652988\\
170	0.00414125555088306\\
171	0.00414187197578273\\
172	0.00414249565266374\\
173	0.00414312670601149\\
174	0.00414376526343341\\
175	0.00414441145573073\\
176	0.0041450654169698\\
177	0.00414572728455374\\
178	0.00414639719929339\\
179	0.00414707530547805\\
180	0.00414776175094538\\
181	0.00414845668715073\\
182	0.00414916026923534\\
183	0.0041498726560934\\
184	0.00415059401043789\\
185	0.00415132449886459\\
186	0.00415206429191454\\
187	0.00415281356413448\\
188	0.00415357249413494\\
189	0.0041543412646461\\
190	0.0041551200625707\\
191	0.00415590907903425\\
192	0.00415670850943206\\
193	0.00415751855347266\\
194	0.0041583394152178\\
195	0.00415917130311833\\
196	0.00416001443004602\\
197	0.00416086901332105\\
198	0.00416173527473464\\
199	0.00416261344056694\\
200	0.00416350374159974\\
201	0.00416440641312374\\
202	0.00416532169494035\\
203	0.00416624983135754\\
204	0.00416719107117978\\
205	0.00416814566769192\\
206	0.00416911387863618\\
207	0.00417009596618309\\
208	0.00417109219689547\\
209	0.00417210284168541\\
210	0.00417312817576454\\
211	0.00417416847858684\\
212	0.00417522403378432\\
213	0.0041762951290953\\
214	0.00417738205628507\\
215	0.00417848511105917\\
216	0.00417960459296875\\
217	0.0041807408053083\\
218	0.00418189405500533\\
219	0.00418306465250219\\
220	0.00418425291162984\\
221	0.00418545914947342\\
222	0.0041866836862296\\
223	0.00418792684505579\\
224	0.00418918895191122\\
225	0.00419047033538989\\
226	0.00419177132654538\\
227	0.00419309225870856\\
228	0.00419443346729708\\
229	0.00419579528961866\\
230	0.00419717806466829\\
231	0.00419858213292799\\
232	0.00420000783619921\\
233	0.00420145551752562\\
234	0.00420292552087986\\
235	0.00420441819099777\\
236	0.00420593387324132\\
237	0.00420747291350122\\
238	0.00420903565815524\\
239	0.00421062245410225\\
240	0.00421223364889825\\
241	0.00421386959102594\\
242	0.00421553063033765\\
243	0.00421721711871981\\
244	0.00421892941103672\\
245	0.00422066786642197\\
246	0.00422243284999639\\
247	0.00422422473510007\\
248	0.00422604390613041\\
249	0.0042278907620704\\
250	0.00422976572077321\\
251	0.00423166922412071\\
252	0.00423360174433245\\
253	0.00423556379097977\\
254	0.00423755591868842\\
255	0.00423957873513579\\
256	0.00424163290862752\\
257	0.00424371917420263\\
258	0.00424583833787788\\
259	0.00424799127431861\\
260	0.00425017888885837\\
261	0.00425240211861172\\
262	0.00425466193363533\\
263	0.00425695933817204\\
264	0.0042592953720731\\
265	0.00426167111248506\\
266	0.00426408767458076\\
267	0.00426654621277113\\
268	0.00426904792205369\\
269	0.0042715940393867\\
270	0.00427418584508419\\
271	0.00427682466422699\\
272	0.00427951186808263\\
273	0.00428224887552587\\
274	0.00428503715444896\\
275	0.00428787822314751\\
276	0.0042907736516625\\
277	0.00429372506305063\\
278	0.0042967341345539\\
279	0.00429980259872177\\
280	0.00430293224504238\\
281	0.00430612492086036\\
282	0.00430938253134836\\
283	0.00431270703968716\\
284	0.00431610046668569\\
285	0.00431956488984633\\
286	0.00432310244353248\\
287	0.00432671531807215\\
288	0.00433040575779906\\
289	0.00433417605820851\\
290	0.004338028561485\\
291	0.00434196564953624\\
292	0.00434598973981118\\
293	0.00435010327728252\\
294	0.00435430872458364\\
295	0.0043586085500224\\
296	0.00436300521317298\\
297	0.00436750114773264\\
298	0.00437209874132715\\
299	0.00437680031196629\\
300	0.00438160808091853\\
301	0.00438652414202639\\
302	0.00439155042855168\\
303	0.00439668868214047\\
304	0.00440194040551478\\
305	0.00440730681827399\\
306	0.00441278881462466\\
307	0.00441838692466119\\
308	0.00442410128403887\\
309	0.00442993161913903\\
310	0.00443587725775034\\
311	0.00444193717853756\\
312	0.00444811011538356\\
313	0.00445439475816206\\
314	0.00446079010033279\\
315	0.0044672959489268\\
316	0.00447391631848287\\
317	0.00448065625056135\\
318	0.00448752117149148\\
319	0.00449451692340825\\
320	0.00450164981628869\\
321	0.00450892666435019\\
322	0.0045163548226895\\
323	0.00452394223476899\\
324	0.00453169748434551\\
325	0.0045396298526327\\
326	0.0045477493794233\\
327	0.00455606689500467\\
328	0.0045645940897534\\
329	0.00457334359700911\\
330	0.00458232905924628\\
331	0.0045915651923621\\
332	0.00460106784652621\\
333	0.00461085409850468\\
334	0.00462094238577422\\
335	0.00463135242098504\\
336	0.00464210518561074\\
337	0.00465322288161534\\
338	0.00466472882292711\\
339	0.00467664723130864\\
340	0.00468900278185558\\
341	0.00470182005770181\\
342	0.0047151231142723\\
343	0.00472893456204501\\
344	0.00474327429492926\\
345	0.00475815774915058\\
346	0.00477359354148361\\
347	0.00478958028699191\\
348	0.00480610232394138\\
349	0.0048231239137538\\
350	0.00484056158046822\\
351	0.00485806520159102\\
352	0.00487562323831201\\
353	0.00489322270900446\\
354	0.0049108491103253\\
355	0.00492848572021073\\
356	0.00494611359557722\\
357	0.00496371126707528\\
358	0.00498125594075324\\
359	0.0049987233684562\\
360	0.00501608709894146\\
361	0.00503331849331765\\
362	0.00505038698214896\\
363	0.0050672585025896\\
364	0.00508389293770087\\
365	0.00510024698592675\\
366	0.00511627433362389\\
367	0.00513192602333867\\
368	0.00514715109622411\\
369	0.00516189761431189\\
370	0.00517611420720388\\
371	0.00518975233876229\\
372	0.00520276938265742\\
373	0.00521513302496895\\
374	0.00522682733997493\\
375	0.00523786108158864\\
376	0.00524831508627264\\
377	0.00525861736023971\\
378	0.00526875578457428\\
379	0.00527871832643526\\
380	0.00528849317371806\\
381	0.00529806890299139\\
382	0.00530743470690838\\
383	0.00531658071304657\\
384	0.00532549831821052\\
385	0.00533418054159597\\
386	0.00534262242059119\\
387	0.0053508214454658\\
388	0.00535877802430025\\
389	0.00536649596191278\\
390	0.00537398293026189\\
391	0.00538125089273225\\
392	0.00538831642273599\\
393	0.00539520083064398\\
394	0.00540192997276148\\
395	0.00540853355941611\\
396	0.00541504371526052\\
397	0.00542149244662168\\
398	0.00542789359962197\\
399	0.00543425068402017\\
400	0.00544056657450049\\
401	0.00544684484705187\\
402	0.00545308979767841\\
403	0.00545930644642226\\
404	0.00546550052258368\\
405	0.00547167842660904\\
406	0.0054778471637682\\
407	0.00548401424463447\\
408	0.00549018754787465\\
409	0.0054963751420991\\
410	0.00550258506634743\\
411	0.0055088250732971\\
412	0.00551510234635268\\
413	0.00552142321306698\\
414	0.00552779289425235\\
415	0.00553421535301886\\
416	0.00554069376033234\\
417	0.00554723114868104\\
418	0.00555383060849965\\
419	0.00556049526235636\\
420	0.0055672282365609\\
421	0.00557403263085659\\
422	0.00558091148717831\\
423	0.00558786775882935\\
424	0.00559490428182437\\
425	0.00560202375054027\\
426	0.00560922870016655\\
427	0.00561652149865555\\
428	0.00562390435080193\\
429	0.00563137931650108\\
430	0.00563894834380563\\
431	0.00564661331460629\\
432	0.00565437608506419\\
433	0.00566223849751214\\
434	0.00567020237707114\\
435	0.00567826952907412\\
436	0.00568644173746758\\
437	0.00569472076433976\\
438	0.0057031083506825\\
439	0.00571160621843063\\
440	0.00572021607373206\\
441	0.00572893961128431\\
442	0.00573777851943147\\
443	0.00574673448555957\\
444	0.00575580920118335\\
445	0.00576500436602482\\
446	0.00577432169042485\\
447	0.00578376289640306\\
448	0.00579332971842165\\
449	0.00580302390431187\\
450	0.00581284721633995\\
451	0.0058228014323806\\
452	0.00583288834715807\\
453	0.00584310977350766\\
454	0.0058534675436082\\
455	0.005863963510137\\
456	0.00587459954730718\\
457	0.00588537755176319\\
458	0.00589629944333533\\
459	0.00590736716568488\\
460	0.00591858268690471\\
461	0.00592994800012101\\
462	0.00594146512410113\\
463	0.00595313610386228\\
464	0.00596496301127626\\
465	0.00597694794566633\\
466	0.00598909303439411\\
467	0.00600140043343626\\
468	0.00601387232795212\\
469	0.00602651093284681\\
470	0.00603931849333446\\
471	0.00605229728550793\\
472	0.00606544961692059\\
473	0.0060787778271834\\
474	0.0060922842885795\\
475	0.00610597140669768\\
476	0.00611984162108701\\
477	0.00613389740593545\\
478	0.00614814127077531\\
479	0.00616257576121957\\
480	0.00617720345973286\\
481	0.006192026986442\\
482	0.00620704899999094\\
483	0.00622227219844596\\
484	0.00623769932025699\\
485	0.00625333314528213\\
486	0.00626917649588283\\
487	0.00628523223809936\\
488	0.00630150328291617\\
489	0.00631799258762984\\
490	0.00633470315733296\\
491	0.00635163804653024\\
492	0.00636880036090543\\
493	0.00638619325926091\\
494	0.00640381995565534\\
495	0.00642168372176941\\
496	0.00643978788953463\\
497	0.00645813585406661\\
498	0.00647673107695201\\
499	0.00649557708994683\\
500	0.00651467749915488\\
501	0.00653403598976808\\
502	0.00655365633146543\\
503	0.00657354238458662\\
504	0.00659369810721729\\
505	0.00661412756335069\\
506	0.00663483493232087\\
507	0.00665582451974051\\
508	0.00667710077022156\\
509	0.0066986682822085\\
510	0.00672053182533012\\
511	0.00674269636055962\\
512	0.00676516706373502\\
513	0.00678794935267998\\
514	0.00681104892293287\\
515	0.0068344717947585\\
516	0.00685822436532986\\
517	0.00688231353410591\\
518	0.00690674657611931\\
519	0.0069315311009031\\
520	0.00695667520078294\\
521	0.00698218171358456\\
522	0.00700806184724271\\
523	0.00703433235251492\\
524	0.0070610103146217\\
525	0.0070881189831278\\
526	0.00711567716246021\\
527	0.00714369741365174\\
528	0.00717219191737849\\
529	0.00720117163858217\\
530	0.00723064483866799\\
531	0.00726061777558896\\
532	0.00729109401196795\\
533	0.00732206999733117\\
534	0.00735350410895862\\
535	0.00738534794863182\\
536	0.00741766494491498\\
537	0.00745056597663264\\
538	0.00748420149227407\\
539	0.0075186282540824\\
540	0.00755391134287273\\
541	0.00759010429212257\\
542	0.00762725429437914\\
543	0.00766541139405159\\
544	0.00770463275832746\\
545	0.00774498450574268\\
546	0.00778654350476932\\
547	0.00782943284053057\\
548	0.007873812470974\\
549	0.00791685142016626\\
550	0.00795760885345326\\
551	0.00799826335611184\\
552	0.008039343319122\\
553	0.00808097441322759\\
554	0.00812317053076772\\
555	0.00816590702824844\\
556	0.00820914385485902\\
557	0.00825282989559372\\
558	0.00829690046744554\\
559	0.00834127797993642\\
560	0.00838587427252687\\
561	0.00842944747538469\\
562	0.00847248750936672\\
563	0.00851573631428289\\
564	0.00855920311781243\\
565	0.00860285881927415\\
566	0.00864667312030307\\
567	0.00869051061764692\\
568	0.00873403472335589\\
569	0.00877786227543177\\
570	0.00882199645300022\\
571	0.00886641582265344\\
572	0.00891109504133455\\
573	0.00895600622677124\\
574	0.00900111882085289\\
575	0.00904639944680352\\
576	0.00909181177476488\\
577	0.00913731640197366\\
578	0.00918287075499839\\
579	0.00922842902355639\\
580	0.00927394213803805\\
581	0.00931935780612538\\
582	0.00936462062793332\\
583	0.00940967231402828\\
584	0.00945445203657523\\
585	0.0094988969504773\\
586	0.00954294292772307\\
587	0.00958652555102228\\
588	0.00962958140299686\\
589	0.0096720496394373\\
590	0.0097138736812812\\
591	0.0097550024176802\\
592	0.00979538909575469\\
593	0.00983498278816331\\
594	0.00987369853931868\\
595	0.00991118387968948\\
596	0.00994658651044256\\
597	0.00997788999445116\\
598	0.010000292044645\\
599	0\\
600	0\\
};
\addplot [color=mycolor4,solid,forget plot]
  table[row sep=crcr]{%
1	0.00409664868257122\\
2	0.00409668962381598\\
3	0.00409673130488141\\
4	0.00409677373721799\\
5	0.00409681693239205\\
6	0.00409686090208509\\
7	0.00409690565809265\\
8	0.00409695121232338\\
9	0.00409699757679793\\
10	0.00409704476364767\\
11	0.00409709278511336\\
12	0.00409714165354387\\
13	0.00409719138139446\\
14	0.00409724198122542\\
15	0.00409729346570032\\
16	0.00409734584758411\\
17	0.00409739913974129\\
18	0.00409745335513396\\
19	0.0040975085068197\\
20	0.00409756460794935\\
21	0.0040976216717648\\
22	0.00409767971159646\\
23	0.00409773874086082\\
24	0.00409779877305787\\
25	0.00409785982176835\\
26	0.00409792190065095\\
27	0.00409798502343938\\
28	0.0040980492039394\\
29	0.0040981144560256\\
30	0.00409818079363827\\
31	0.00409824823078012\\
32	0.00409831678151281\\
33	0.00409838645995345\\
34	0.00409845728027101\\
35	0.00409852925668279\\
36	0.00409860240345046\\
37	0.00409867673487639\\
38	0.00409875226529966\\
39	0.0040988290090922\\
40	0.00409890698065467\\
41	0.00409898619441248\\
42	0.00409906666481163\\
43	0.00409914840631443\\
44	0.00409923143339559\\
45	0.00409931576053771\\
46	0.00409940140222727\\
47	0.00409948837295028\\
48	0.00409957668718817\\
49	0.00409966635941348\\
50	0.00409975740408583\\
51	0.00409984983564766\\
52	0.00409994366852032\\
53	0.00410003891709999\\
54	0.0041001355957539\\
55	0.0041002337188164\\
56	0.00410033330058548\\
57	0.00410043435531914\\
58	0.0041005368972321\\
59	0.00410064094049262\\
60	0.00410074649921956\\
61	0.00410085358747956\\
62	0.00410096221928472\\
63	0.00410107240859023\\
64	0.00410118416929251\\
65	0.00410129751522758\\
66	0.00410141246016985\\
67	0.00410152901783121\\
68	0.00410164720186052\\
69	0.00410176702584359\\
70	0.00410188850330351\\
71	0.00410201164770159\\
72	0.0041021364724387\\
73	0.00410226299085719\\
74	0.00410239121624346\\
75	0.00410252116183092\\
76	0.00410265284080381\\
77	0.00410278626630155\\
78	0.00410292145142377\\
79	0.00410305840923608\\
80	0.00410319715277667\\
81	0.00410333769506352\\
82	0.00410348004910242\\
83	0.00410362422789598\\
84	0.00410377024445325\\
85	0.00410391811180042\\
86	0.00410406784299215\\
87	0.00410421945112408\\
88	0.00410437294934605\\
89	0.00410452835087618\\
90	0.0041046856690162\\
91	0.00410484491716742\\
92	0.00410500610884766\\
93	0.00410516925770937\\
94	0.00410533437755831\\
95	0.00410550148237347\\
96	0.00410567058632778\\
97	0.00410584170380952\\
98	0.00410601484944506\\
99	0.0041061900381219\\
100	0.00410636728501282\\
101	0.00410654660560078\\
102	0.00410672801570444\\
103	0.00410691153150435\\
104	0.00410709716956986\\
105	0.00410728494688636\\
106	0.00410747488088329\\
107	0.00410766698946234\\
108	0.00410786129102603\\
109	0.00410805780450654\\
110	0.00410825654939485\\
111	0.00410845754576972\\
112	0.00410866081432693\\
113	0.00410886637640819\\
114	0.00410907425402994\\
115	0.00410928446991207\\
116	0.0041094970475059\\
117	0.00410971201102203\\
118	0.00410992938545747\\
119	0.0041101491966223\\
120	0.00411037147116538\\
121	0.00411059623659954\\
122	0.00411082352132581\\
123	0.00411105335465676\\
124	0.00411128576683904\\
125	0.00411152078907476\\
126	0.00411175845354187\\
127	0.00411199879341387\\
128	0.00411224184287793\\
129	0.00411248763715254\\
130	0.00411273621250376\\
131	0.00411298760626071\\
132	0.00411324185682997\\
133	0.00411349900370904\\
134	0.00411375908749892\\
135	0.00411402214991591\\
136	0.00411428823380235\\
137	0.00411455738313678\\
138	0.00411482964304308\\
139	0.00411510505979911\\
140	0.00411538368084452\\
141	0.00411566555478786\\
142	0.00411595073141336\\
143	0.00411623926168708\\
144	0.00411653119776312\\
145	0.00411682659298973\\
146	0.00411712550191647\\
147	0.00411742798030309\\
148	0.00411773408513234\\
149	0.00411804387462939\\
150	0.0041183574082853\\
151	0.00411867474685224\\
152	0.00411899595236577\\
153	0.00411932108816943\\
154	0.00411965021893952\\
155	0.00411998341070969\\
156	0.00412032073089582\\
157	0.00412066224832075\\
158	0.004121008033239\\
159	0.00412135815736145\\
160	0.0041217126938801\\
161	0.00412207171749242\\
162	0.00412243530442588\\
163	0.00412280353246209\\
164	0.004123176480961\\
165	0.00412355423088474\\
166	0.0041239368648212\\
167	0.00412432446700746\\
168	0.00412471712335291\\
169	0.00412511492146215\\
170	0.00412551795065742\\
171	0.00412592630200085\\
172	0.00412634006831619\\
173	0.0041267593442104\\
174	0.00412718422609469\\
175	0.00412761481220526\\
176	0.00412805120262364\\
177	0.00412849349929665\\
178	0.00412894180605595\\
179	0.00412939622863741\\
180	0.00412985687469984\\
181	0.00413032385384373\\
182	0.00413079727762944\\
183	0.00413127725959528\\
184	0.00413176391527536\\
185	0.00413225736221742\\
186	0.00413275772000041\\
187	0.00413326511025216\\
188	0.00413377965666715\\
189	0.00413430148502457\\
190	0.00413483072320659\\
191	0.00413536750121703\\
192	0.00413591195120078\\
193	0.00413646420746377\\
194	0.00413702440649402\\
195	0.00413759268698367\\
196	0.00413816918985233\\
197	0.00413875405827195\\
198	0.00413934743769338\\
199	0.00413994947587498\\
200	0.00414056032291342\\
201	0.00414118013127704\\
202	0.00414180905584204\\
203	0.00414244725393175\\
204	0.00414309488535929\\
205	0.00414375211247407\\
206	0.00414441910021256\\
207	0.00414509601615324\\
208	0.00414578303057657\\
209	0.0041464803165302\\
210	0.0041471880498995\\
211	0.0041479064094845\\
212	0.00414863557708264\\
213	0.00414937573757863\\
214	0.00415012707904113\\
215	0.00415088979282709\\
216	0.00415166407369351\\
217	0.00415245011991762\\
218	0.00415324813342542\\
219	0.00415405831992878\\
220	0.00415488088907169\\
221	0.00415571605458558\\
222	0.00415656403445428\\
223	0.00415742505108853\\
224	0.00415829933151019\\
225	0.00415918710754658\\
226	0.0041600886160349\\
227	0.00416100409903648\\
228	0.0041619338040616\\
229	0.00416287798430444\\
230	0.00416383689888962\\
231	0.00416481081313237\\
232	0.00416579999881265\\
233	0.00416680473444185\\
234	0.00416782530555101\\
235	0.00416886200499009\\
236	0.00416991513323826\\
237	0.00417098499872553\\
238	0.0041720719181655\\
239	0.00417317621689987\\
240	0.00417429822925369\\
241	0.00417543829890231\\
242	0.00417659677924825\\
243	0.00417777403380749\\
244	0.00417897043660341\\
245	0.00418018637256503\\
246	0.00418142223792556\\
247	0.00418267844061465\\
248	0.00418395540063496\\
249	0.0041852535504115\\
250	0.00418657333510657\\
251	0.00418791521289753\\
252	0.0041892796551678\\
253	0.00419066714660251\\
254	0.00419207818516755\\
255	0.00419351328196016\\
256	0.00419497296095737\\
257	0.00419645775879288\\
258	0.0041979682245596\\
259	0.00419950491873774\\
260	0.00420106841334051\\
261	0.004202659292063\\
262	0.00420427815043769\\
263	0.00420592559600409\\
264	0.00420760224848415\\
265	0.00420930873987836\\
266	0.00421104571460549\\
267	0.00421281382965351\\
268	0.0042146137547326\\
269	0.00421644617243015\\
270	0.00421831177836837\\
271	0.00422021128136477\\
272	0.0042221454035958\\
273	0.00422411488076409\\
274	0.00422612046226925\\
275	0.00422816291138236\\
276	0.00423024300542409\\
277	0.00423236153594855\\
278	0.00423451930894415\\
279	0.00423671714508118\\
280	0.00423895587990949\\
281	0.00424123636400916\\
282	0.00424355946318336\\
283	0.00424592605864222\\
284	0.00424833704720788\\
285	0.0042507933416628\\
286	0.00425329587106279\\
287	0.00425584558103343\\
288	0.00425844343407693\\
289	0.00426109040985437\\
290	0.00426378750546487\\
291	0.00426653573608827\\
292	0.00426933613552122\\
293	0.00427218975681514\\
294	0.00427509767304613\\
295	0.00427806097825504\\
296	0.00428108078860607\\
297	0.00428415824382355\\
298	0.00428729450898293\\
299	0.00429049077675482\\
300	0.00429374827024759\\
301	0.00429706824668784\\
302	0.00430045200220975\\
303	0.00430390087654047\\
304	0.00430741625931877\\
305	0.00431099959792094\\
306	0.00431465240683752\\
307	0.00431837627874854\\
308	0.0043221728973601\\
309	0.00432604405188818\\
310	0.00432999165274966\\
311	0.00433401774763531\\
312	0.00433812453795846\\
313	0.00434231439422532\\
314	0.00434658986551328\\
315	0.00435095368955819\\
316	0.00435540871945394\\
317	0.00435995791231748\\
318	0.00436460433143802\\
319	0.00436935114925647\\
320	0.00437420164829873\\
321	0.00437915922141272\\
322	0.00438422737192456\\
323	0.00438940971298996\\
324	0.00439470996591586\\
325	0.00440013195682813\\
326	0.00440567960975303\\
327	0.00441135694093633\\
328	0.00441716805169503\\
329	0.00442311711736991\\
330	0.00442920837349606\\
331	0.00443544609938092\\
332	0.00444183460149077\\
333	0.00444837819425593\\
334	0.00445508116082882\\
335	0.00446194771879704\\
336	0.00446898198085834\\
337	0.00447618791006925\\
338	0.00448356926800846\\
339	0.0044911295503311\\
340	0.00449887193144283\\
341	0.00450679923471691\\
342	0.00451491389170535\\
343	0.00452321791763498\\
344	0.00453171291502352\\
345	0.00454040012236478\\
346	0.00454928053170496\\
347	0.0045583551079548\\
348	0.00456762515584229\\
349	0.0045770929316558\\
350	0.00458676283704058\\
351	0.00459664640722626\\
352	0.00460675613519573\\
353	0.00461710555285139\\
354	0.00462770926379756\\
355	0.00463858304664908\\
356	0.00464974395497892\\
357	0.00466121052009476\\
358	0.0046730027990592\\
359	0.00468514236738451\\
360	0.00469765236112168\\
361	0.00471055747325224\\
362	0.00472388374758736\\
363	0.00473765833306152\\
364	0.00475190959924341\\
365	0.00476666685219704\\
366	0.00478195989826624\\
367	0.00479781840288387\\
368	0.00481427097438884\\
369	0.00483134387867026\\
370	0.00484905924751179\\
371	0.00486743258499649\\
372	0.00488646990496595\\
373	0.00490616315581023\\
374	0.00492648442273783\\
375	0.00494737823307882\\
376	0.00496871592348503\\
377	0.00499002714010541\\
378	0.00501128758581625\\
379	0.00503247096782956\\
380	0.0050535488223626\\
381	0.00507449094119751\\
382	0.00509526083085377\\
383	0.0051158164758645\\
384	0.00513611189221272\\
385	0.00515609703013776\\
386	0.0051757177862393\\
387	0.00519491617700118\\
388	0.00521363074694414\\
389	0.00523179732370554\\
390	0.00524935012226\\
391	0.00526622344716711\\
392	0.00528235423550805\\
393	0.00529768567804759\\
394	0.00531217237850484\\
395	0.00532578763345195\\
396	0.0053385332321474\\
397	0.00535045281232267\\
398	0.00536206547182922\\
399	0.00537344406308905\\
400	0.00538457529139072\\
401	0.00539544694147929\\
402	0.00540604824494743\\
403	0.0054163703009139\\
404	0.00542640655005739\\
405	0.0054361532980586\\
406	0.00544561028227294\\
407	0.00545478126853225\\
408	0.0054636746509502\\
409	0.00547230401387973\\
410	0.0054806885824226\\
411	0.00548885346400712\\
412	0.00549682954230167\\
413	0.00550465281782058\\
414	0.00551236291050277\\
415	0.00552000033082267\\
416	0.00552758999632444\\
417	0.00553514042149178\\
418	0.00554265655480703\\
419	0.00555014421767857\\
420	0.00555761009506642\\
421	0.00556506169942287\\
422	0.00557250730179434\\
423	0.00557995582371182\\
424	0.00558741668426262\\
425	0.00559489959813635\\
426	0.00560241432305374\\
427	0.00560997035970928\\
428	0.00561757661503771\\
429	0.00562524105176726\\
430	0.00563297036571746\\
431	0.00564076975981897\\
432	0.00564864322931405\\
433	0.00565659441994265\\
434	0.0056646270121865\\
435	0.00567274468719499\\
436	0.00568095109050692\\
437	0.00568924979480068\\
438	0.00569764426333909\\
439	0.00570613781625333\\
440	0.00571473360229106\\
441	0.00572343457906517\\
442	0.00573224350506446\\
443	0.00574116294654159\\
444	0.00575019530158278\\
445	0.00575934284175239\\
446	0.0057686077680368\\
447	0.00577799225250322\\
448	0.00578749844787233\\
449	0.00579712848460786\\
450	0.00580688446918496\\
451	0.00581676848371286\\
452	0.00582678258703619\\
453	0.00583692881735942\\
454	0.00584720919632644\\
455	0.00585762573434246\\
456	0.00586818043675091\\
457	0.0058788753102902\\
458	0.00588971236908199\\
459	0.00590069363930562\\
460	0.00591182116179189\\
461	0.00592309699327209\\
462	0.00593452320750874\\
463	0.00594610189660377\\
464	0.00595783517244888\\
465	0.00596972516827276\\
466	0.00598177404023063\\
467	0.00599398396897853\\
468	0.00600635716117462\\
469	0.00601889585085943\\
470	0.00603160230068508\\
471	0.00604447880299224\\
472	0.00605752768077079\\
473	0.00607075128858001\\
474	0.00608415201348621\\
475	0.00609773227602922\\
476	0.0061114945312139\\
477	0.00612544126952421\\
478	0.00613957501795931\\
479	0.00615389834109338\\
480	0.00616841384216404\\
481	0.00618312416419666\\
482	0.00619803199117528\\
483	0.00621314004927233\\
484	0.00622845110815102\\
485	0.00624396798235272\\
486	0.00625969353278066\\
487	0.00627563066829013\\
488	0.00629178234739776\\
489	0.00630815158012391\\
490	0.00632474142998397\\
491	0.00634155501614687\\
492	0.00635859551578096\\
493	0.00637586616661024\\
494	0.0063933702697064\\
495	0.00641111119254501\\
496	0.00642909237235806\\
497	0.0064473173198182\\
498	0.00646578962309558\\
499	0.00648451295233245\\
500	0.00650349106458752\\
501	0.00652272780930867\\
502	0.00654222713440037\\
503	0.00656199309296133\\
504	0.00658202985077848\\
505	0.00660234169467454\\
506	0.00662293304182132\\
507	0.00664380845014584\\
508	0.00666497262997551\\
509	0.00668643045708999\\
510	0.00670818698737221\\
511	0.00673024747328626\\
512	0.00675261738244383\\
513	0.00677530241856389\\
514	0.00679830854505638\\
515	0.00682164201148414\\
516	0.00684530938344261\\
517	0.0068693175757188\\
518	0.00689367389591976\\
519	0.00691838609654099\\
520	0.00694346243095817\\
521	0.00696891178565863\\
522	0.00699474352995971\\
523	0.00702096749344242\\
524	0.00704759410736137\\
525	0.00707462948156095\\
526	0.00710208597353999\\
527	0.00712998298566222\\
528	0.00715834050109882\\
529	0.00718718223334858\\
530	0.00721653322220688\\
531	0.00724640874563171\\
532	0.00727682321657789\\
533	0.00730779033496002\\
534	0.00733932054955617\\
535	0.00737142094251038\\
536	0.00740409155219721\\
537	0.00743729626273568\\
538	0.00747098540990441\\
539	0.00750522364908087\\
540	0.00754010540771196\\
541	0.00757580349715638\\
542	0.00761236890146435\\
543	0.00764985959963282\\
544	0.00768832507234716\\
545	0.00772781694935591\\
546	0.00776839367582466\\
547	0.00781012138020359\\
548	0.00785307687559534\\
549	0.00789738450562302\\
550	0.0079432034083031\\
551	0.00798788543240736\\
552	0.00803044603523655\\
553	0.00807251022837668\\
554	0.00811489868865496\\
555	0.00815779179765849\\
556	0.0082012227307077\\
557	0.00824515999795583\\
558	0.00828955575295806\\
559	0.00833434738316874\\
560	0.00837945454762876\\
561	0.00842479121386117\\
562	0.0084696778803419\\
563	0.00851341295005585\\
564	0.00855721779447572\\
565	0.00860119556072858\\
566	0.00864533797554444\\
567	0.00868961368428026\\
568	0.00873393789191277\\
569	0.00877785590949144\\
570	0.00882199632513923\\
571	0.0088664158097615\\
572	0.00891109503814012\\
573	0.00895600622559429\\
574	0.00900111882029819\\
575	0.00904639944650971\\
576	0.00909181177460439\\
577	0.00913731640188469\\
578	0.00918287075494969\\
579	0.00922842902353094\\
580	0.00927394213802573\\
581	0.00931935780612039\\
582	0.00936462062793178\\
583	0.00940967231402812\\
584	0.00945445203657523\\
585	0.00949889695047729\\
586	0.00954294292772307\\
587	0.00958652555102227\\
588	0.00962958140299686\\
589	0.00967204963943729\\
590	0.0097138736812812\\
591	0.0097550024176802\\
592	0.00979538909575469\\
593	0.00983498278816331\\
594	0.00987369853931868\\
595	0.00991118387968948\\
596	0.00994658651044256\\
597	0.00997788999445116\\
598	0.010000292044645\\
599	0\\
600	0\\
};
\addplot [color=mycolor5,solid,forget plot]
  table[row sep=crcr]{%
1	0.00409619160294684\\
2	0.00409622158226429\\
3	0.00409625205221089\\
4	0.00409628301926326\\
5	0.00409631448994264\\
6	0.00409634647081406\\
7	0.00409637896848576\\
8	0.00409641198960823\\
9	0.0040964455408736\\
10	0.0040964796290148\\
11	0.00409651426080469\\
12	0.00409654944305536\\
13	0.00409658518261726\\
14	0.0040966214863783\\
15	0.00409665836126297\\
16	0.00409669581423155\\
17	0.00409673385227931\\
18	0.00409677248243541\\
19	0.00409681171176222\\
20	0.00409685154735435\\
21	0.00409689199633775\\
22	0.00409693306586891\\
23	0.00409697476313393\\
24	0.00409701709534774\\
25	0.00409706006975311\\
26	0.00409710369362\\
27	0.00409714797424461\\
28	0.00409719291894871\\
29	0.00409723853507883\\
30	0.00409728483000557\\
31	0.00409733181112282\\
32	0.00409737948584722\\
33	0.00409742786161753\\
34	0.00409747694589413\\
35	0.00409752674615837\\
36	0.00409757726991232\\
37	0.00409762852467825\\
38	0.00409768051799848\\
39	0.00409773325743501\\
40	0.00409778675056946\\
41	0.00409784100500304\\
42	0.00409789602835649\\
43	0.0040979518282704\\
44	0.00409800841240532\\
45	0.00409806578844213\\
46	0.00409812396408264\\
47	0.00409818294705012\\
48	0.00409824274509012\\
49	0.00409830336597133\\
50	0.00409836481748665\\
51	0.00409842710745441\\
52	0.00409849024371973\\
53	0.00409855423415614\\
54	0.00409861908666724\\
55	0.00409868480918863\\
56	0.00409875140969016\\
57	0.00409881889617818\\
58	0.00409888727669801\\
59	0.00409895655933686\\
60	0.0040990267522268\\
61	0.00409909786354797\\
62	0.00409916990153199\\
63	0.00409924287446588\\
64	0.00409931679069591\\
65	0.00409939165863193\\
66	0.00409946748675191\\
67	0.00409954428360668\\
68	0.00409962205782512\\
69	0.00409970081811947\\
70	0.00409978057329107\\
71	0.00409986133223633\\
72	0.00409994310395293\\
73	0.00410002589754655\\
74	0.00410010972223754\\
75	0.00410019458736832\\
76	0.00410028050241081\\
77	0.00410036747697415\\
78	0.00410045552081292\\
79	0.00410054464383554\\
80	0.00410063485611294\\
81	0.00410072616788758\\
82	0.00410081858958281\\
83	0.00410091213181236\\
84	0.00410100680539028\\
85	0.00410110262134104\\
86	0.00410119959090989\\
87	0.00410129772557349\\
88	0.0041013970370508\\
89	0.00410149753731418\\
90	0.00410159923860063\\
91	0.00410170215342322\\
92	0.00410180629458288\\
93	0.00410191167518001\\
94	0.0041020183086265\\
95	0.00410212620865782\\
96	0.00410223538934498\\
97	0.00410234586510692\\
98	0.00410245765072262\\
99	0.00410257076134334\\
100	0.00410268521250494\\
101	0.00410280102014015\\
102	0.00410291820059057\\
103	0.00410303677061892\\
104	0.004103156747421\\
105	0.00410327814863759\\
106	0.00410340099236612\\
107	0.00410352529717227\\
108	0.00410365108210135\\
109	0.00410377836668947\\
110	0.00410390717097453\\
111	0.00410403751550685\\
112	0.00410416942135973\\
113	0.00410430291013961\\
114	0.00410443800399612\\
115	0.00410457472563163\\
116	0.00410471309831071\\
117	0.00410485314586941\\
118	0.00410499489272402\\
119	0.00410513836387979\\
120	0.00410528358493932\\
121	0.00410543058211083\\
122	0.00410557938221616\\
123	0.00410573001269859\\
124	0.00410588250163056\\
125	0.0041060368777213\\
126	0.00410619317032435\\
127	0.00410635140944498\\
128	0.00410651162574777\\
129	0.00410667385056405\\
130	0.00410683811589949\\
131	0.00410700445444173\\
132	0.00410717289956831\\
133	0.00410734348535462\\
134	0.00410751624658221\\
135	0.00410769121874715\\
136	0.00410786843806889\\
137	0.00410804794149926\\
138	0.00410822976673195\\
139	0.0041084139522122\\
140	0.00410860053714694\\
141	0.00410878956151536\\
142	0.00410898106607968\\
143	0.00410917509239668\\
144	0.00410937168282914\\
145	0.00410957088055834\\
146	0.00410977272959654\\
147	0.00410997727480027\\
148	0.00411018456188428\\
149	0.00411039463743527\\
150	0.0041106075489235\\
151	0.0041108233447165\\
152	0.00411104207409314\\
153	0.00411126378725806\\
154	0.00411148853535573\\
155	0.00411171637048515\\
156	0.00411194734571431\\
157	0.00411218151509491\\
158	0.00411241893367731\\
159	0.00411265965752546\\
160	0.00411290374373211\\
161	0.00411315125043418\\
162	0.00411340223682819\\
163	0.00411365676318623\\
164	0.00411391489087168\\
165	0.00411417668235555\\
166	0.00411444220123294\\
167	0.00411471151223977\\
168	0.00411498468126977\\
169	0.00411526177539199\\
170	0.0041155428628684\\
171	0.00411582801317217\\
172	0.00411611729700612\\
173	0.00411641078632185\\
174	0.0041167085543392\\
175	0.00411701067556636\\
176	0.00411731722582052\\
177	0.00411762828224914\\
178	0.00411794392335213\\
179	0.00411826422900421\\
180	0.00411858928047872\\
181	0.00411891916047184\\
182	0.00411925395312786\\
183	0.00411959374406541\\
184	0.00411993862040482\\
185	0.00412028867079633\\
186	0.00412064398544972\\
187	0.00412100465616507\\
188	0.00412137077636482\\
189	0.00412174244112731\\
190	0.00412211974722161\\
191	0.00412250279314403\\
192	0.00412289167915623\\
193	0.00412328650732495\\
194	0.00412368738156349\\
195	0.00412409440767505\\
196	0.00412450769339802\\
197	0.00412492734845314\\
198	0.00412535348459278\\
199	0.00412578621565219\\
200	0.00412622565760295\\
201	0.00412667192860869\\
202	0.00412712514908281\\
203	0.00412758544174877\\
204	0.00412805293170244\\
205	0.00412852774647686\\
206	0.00412901001610912\\
207	0.00412949987320979\\
208	0.00412999745303449\\
209	0.00413050289355773\\
210	0.00413101633554895\\
211	0.00413153792265059\\
212	0.00413206780145841\\
213	0.00413260612160365\\
214	0.00413315303583702\\
215	0.00413370870011443\\
216	0.00413427327368446\\
217	0.00413484691917709\\
218	0.00413542980269359\\
219	0.00413602209389782\\
220	0.004136623966108\\
221	0.00413723559638941\\
222	0.0041378571656474\\
223	0.00413848885872059\\
224	0.00413913086447404\\
225	0.00413978337589208\\
226	0.00414044659017064\\
227	0.00414112070880878\\
228	0.0041418059376992\\
229	0.00414250248721764\\
230	0.00414321057231111\\
231	0.00414393041258449\\
232	0.00414466223238378\\
233	0.00414540626087849\\
234	0.00414616273214168\\
235	0.00414693188522777\\
236	0.00414771396424824\\
237	0.00414850921844467\\
238	0.00414931790225951\\
239	0.00415014027540443\\
240	0.00415097660292645\\
241	0.00415182715527154\\
242	0.00415269220834638\\
243	0.0041535720435779\\
244	0.00415446694797091\\
245	0.00415537721416414\\
246	0.00415630314048469\\
247	0.00415724503100112\\
248	0.00415820319557572\\
249	0.00415917794991717\\
250	0.00416016961563463\\
251	0.00416117852029278\\
252	0.00416220499746934\\
253	0.0041632493868176\\
254	0.0041643120341376\\
255	0.00416539329146222\\
256	0.00416649351716763\\
257	0.00416761307609297\\
258	0.00416875233960254\\
259	0.00416991168572639\\
260	0.00417109149930953\\
261	0.00417229217217076\\
262	0.00417351410327181\\
263	0.00417475769889585\\
264	0.00417602337283\\
265	0.00417731154656293\\
266	0.00417862264949541\\
267	0.00417995711916321\\
268	0.00418131540147287\\
269	0.00418269795095136\\
270	0.00418410523100984\\
271	0.00418553771422237\\
272	0.00418699588261993\\
273	0.00418848022800048\\
274	0.00418999125225548\\
275	0.00419152946771326\\
276	0.00419309539750048\\
277	0.00419468957592257\\
278	0.0041963125488651\\
279	0.00419796487420833\\
280	0.00419964712225698\\
281	0.00420135987619224\\
282	0.00420310373254267\\
283	0.00420487930167866\\
284	0.00420668720833811\\
285	0.00420852809216799\\
286	0.00421040260828383\\
287	0.00421231142784932\\
288	0.00421425523867425\\
289	0.00421623474583657\\
290	0.00421825067235086\\
291	0.00422030375984715\\
292	0.00422239476927642\\
293	0.00422452448164367\\
294	0.00422669369876994\\
295	0.00422890324408457\\
296	0.0042311539634484\\
297	0.00423344672600954\\
298	0.00423578242509367\\
299	0.00423816197913298\\
300	0.00424058633263962\\
301	0.00424305645721312\\
302	0.00424557335247821\\
303	0.00424813804707692\\
304	0.00425075159967099\\
305	0.00425341509992398\\
306	0.00425612966943876\\
307	0.00425889646261729\\
308	0.00426171666739985\\
309	0.0042645915058305\\
310	0.00426752223441344\\
311	0.00427051014433791\\
312	0.00427355656159489\\
313	0.00427666284695063\\
314	0.00427983039637195\\
315	0.00428306063945929\\
316	0.00428635503992115\\
317	0.00428971509653306\\
318	0.0042931423442123\\
319	0.00429663835501139\\
320	0.00430020473916595\\
321	0.00430384314625455\\
322	0.00430755526643084\\
323	0.0043113428317262\\
324	0.0043152076173955\\
325	0.0043191514432586\\
326	0.00432317617543228\\
327	0.00432728372825828\\
328	0.00433147606632453\\
329	0.00433575520676757\\
330	0.00434012322199784\\
331	0.00434458224309119\\
332	0.00434913446355973\\
333	0.0043537821425117\\
334	0.00435852760938808\\
335	0.00436337326953708\\
336	0.00436832161071977\\
337	0.00437337521052818\\
338	0.00437853674471211\\
339	0.00438380899845113\\
340	0.00438919488129389\\
341	0.00439469744265264\\
342	0.00440031988971178\\
343	0.00440606560777859\\
344	0.00441193818286184\\
345	0.00441794142585865\\
346	0.00442407939712489\\
347	0.00443035642955957\\
348	0.0044367771487174\\
349	0.00444334648174883\\
350	0.00445006965806816\\
351	0.00445695208320719\\
352	0.00446399933593276\\
353	0.00447121716027577\\
354	0.00447861146097566\\
355	0.00448618829689002\\
356	0.00449395387742492\\
357	0.00450191454159263\\
358	0.00451007673028543\\
359	0.00451844695920616\\
360	0.00452703178309679\\
361	0.00453583774289785\\
362	0.00454487131592899\\
363	0.00455413889370712\\
364	0.00456364672977185\\
365	0.00457340088478789\\
366	0.00458340717185425\\
367	0.00459367110664716\\
368	0.00460419786932714\\
369	0.00461499228825761\\
370	0.00462605886356467\\
371	0.00463740188191018\\
372	0.00464902557505931\\
373	0.00466093443897203\\
374	0.00467313375313041\\
375	0.00468563037627639\\
376	0.00469843443609457\\
377	0.00471156587778469\\
378	0.00472504627848721\\
379	0.00473889890666229\\
380	0.00475314875338926\\
381	0.00476782220581069\\
382	0.0047829472136154\\
383	0.00479855342091387\\
384	0.00481467205339902\\
385	0.00483133570831505\\
386	0.00484857801096762\\
387	0.00486643308849613\\
388	0.004884934788385\\
389	0.0049041154916689\\
390	0.0049240048555595\\
391	0.00494462772721307\\
392	0.00496600122010846\\
393	0.00498813086022231\\
394	0.00501100534714339\\
395	0.00503459170974232\\
396	0.00505882507912041\\
397	0.00508359712951041\\
398	0.00510833073767857\\
399	0.00513289684048401\\
400	0.005157250316384\\
401	0.00518134153147043\\
402	0.005205116122047\\
403	0.00522851485600113\\
404	0.00525147361912384\\
405	0.00527392360208589\\
406	0.00529579166536875\\
407	0.00531700101044047\\
408	0.00533747237227751\\
409	0.00535712580524692\\
410	0.00537588365823375\\
411	0.00539367467105489\\
412	0.00541043974944722\\
413	0.00542614015003393\\
414	0.00544076868404582\\
415	0.00545436489319491\\
416	0.00546738929631827\\
417	0.00548011849802279\\
418	0.00549253944262353\\
419	0.00550464114701441\\
420	0.00551641524832511\\
421	0.00552785661896403\\
422	0.00553896404323548\\
423	0.00554974094209379\\
424	0.00556019610678231\\
425	0.00557034439458198\\
426	0.00558020731548305\\
427	0.00558981340369295\\
428	0.00559919822137929\\
429	0.00560840377312078\\
430	0.00561747701797148\\
431	0.00562646704137285\\
432	0.00563541158760834\\
433	0.00564432490145999\\
434	0.00565321393937883\\
435	0.00566208666998028\\
436	0.00567095202740168\\
437	0.00567981982455819\\
438	0.00568870061929596\\
439	0.00569760552686288\\
440	0.00570654597380089\\
441	0.00571553339180765\\
442	0.00572457885615251\\
443	0.00573369268309565\\
444	0.00574288401618527\\
445	0.00575216045473461\\
446	0.00576152781243046\\
447	0.00577099067349807\\
448	0.00578055331278669\\
449	0.00579022000283565\\
450	0.00579999497122763\\
451	0.00580988235693061\\
452	0.00581988616766082\\
453	0.0058300102408789\\
454	0.00584025821160249\\
455	0.00585063349069623\\
456	0.00586113925753365\\
457	0.00587177847068472\\
458	0.00588255389919582\\
459	0.00589346817452325\\
460	0.00590452385842724\\
461	0.00591572348402295\\
462	0.00592706956128874\\
463	0.00593856457489556\\
464	0.00595021098360879\\
465	0.00596201122142354\\
466	0.00597396770050443\\
467	0.00598608281587155\\
468	0.00599835895160588\\
469	0.00601079848814156\\
470	0.00602340380998847\\
471	0.00603617731301882\\
472	0.00604912141032395\\
473	0.00606223853571863\\
474	0.00607553114562047\\
475	0.00608900172073521\\
476	0.00610265276796723\\
477	0.00611648682251159\\
478	0.00613050645007231\\
479	0.00614471424914408\\
480	0.00615911285329214\\
481	0.00617370493337021\\
482	0.00618849319963286\\
483	0.00620348040372628\\
484	0.00621866934058261\\
485	0.00623406285029019\\
486	0.00624966382004183\\
487	0.00626547518621064\\
488	0.00628149993655961\\
489	0.0062977411125919\\
490	0.00631420181205383\\
491	0.00633088519160514\\
492	0.00634779446967697\\
493	0.00636493292954258\\
494	0.00638230392263143\\
495	0.0063999108721217\\
496	0.00641775727685009\\
497	0.00643584671557906\\
498	0.00645418285166201\\
499	0.00647276943814956\\
500	0.00649161032338459\\
501	0.0065107094571397\\
502	0.00653007089735626\\
503	0.00654969881755083\\
504	0.0065695975149618\\
505	0.00658977141951615\\
506	0.00661022510370496\\
507	0.00663096329346413\\
508	0.00665199088016667\\
509	0.0066733129338431\\
510	0.00669493471775744\\
511	0.00671686170447795\\
512	0.00673909959359459\\
513	0.00676165433124745\\
514	0.00678453213164819\\
515	0.00680773950079382\\
516	0.00683128326258385\\
517	0.00685517058756828\\
518	0.0068794090243893\\
519	0.00690400653407585\\
520	0.00692897152752586\\
521	0.00695431290571268\\
522	0.00698004010997143\\
523	0.00700616317875087\\
524	0.00703269280624844\\
525	0.00705964046356102\\
526	0.00708701828899484\\
527	0.00711483901941279\\
528	0.00714311610684903\\
529	0.00717186079668715\\
530	0.00720108516360538\\
531	0.00723081189144472\\
532	0.0072610644887466\\
533	0.00729186641857461\\
534	0.00732325074755106\\
535	0.00735523634621418\\
536	0.0073878402515663\\
537	0.00742107811239321\\
538	0.00745496230375221\\
539	0.0074894968293927\\
540	0.00752466168945731\\
541	0.00756038385752855\\
542	0.00759672502313432\\
543	0.00763374757337947\\
544	0.00767162181685216\\
545	0.00771041237843783\\
546	0.00775017685925997\\
547	0.00779097741675524\\
548	0.0078328729291118\\
549	0.00787593015589296\\
550	0.00792022484569437\\
551	0.00796587725701316\\
552	0.00801304218089454\\
553	0.00805960843508264\\
554	0.00810430047028959\\
555	0.00814779918778151\\
556	0.00819148112952877\\
557	0.00823562173607292\\
558	0.00828025250104444\\
559	0.00832535069346646\\
560	0.00837086503244462\\
561	0.00841672430456974\\
562	0.00846284579668691\\
563	0.00850912918133008\\
564	0.00855432580474309\\
565	0.00859878079862082\\
566	0.00864330641528182\\
567	0.00868795759871341\\
568	0.00873271555494383\\
569	0.00877755218112524\\
570	0.00882198003310067\\
571	0.00886641486864619\\
572	0.00891109494574295\\
573	0.00895600620365328\\
574	0.00900111881251587\\
575	0.00904639944292441\\
576	0.00909181177272146\\
577	0.0091373164008612\\
578	0.00918287075438226\\
579	0.00922842902321877\\
580	0.00927394213786092\\
581	0.00931935780604082\\
582	0.00936462062789891\\
583	0.00940967231401762\\
584	0.00945445203657308\\
585	0.0094988969504773\\
586	0.00954294292772307\\
587	0.00958652555102228\\
588	0.00962958140299686\\
589	0.00967204963943729\\
590	0.0097138736812812\\
591	0.00975500241768019\\
592	0.00979538909575469\\
593	0.00983498278816331\\
594	0.00987369853931868\\
595	0.00991118387968948\\
596	0.00994658651044256\\
597	0.00997788999445116\\
598	0.010000292044645\\
599	0\\
600	0\\
};
\addplot [color=mycolor6,solid,forget plot]
  table[row sep=crcr]{%
1	0.0040957575257232\\
2	0.00409577859341713\\
3	0.00409579997628435\\
4	0.004095821678038\\
5	0.00409584370241615\\
6	0.00409586605318159\\
7	0.00409588873412207\\
8	0.00409591174905029\\
9	0.00409593510180391\\
10	0.00409595879624581\\
11	0.00409598283626408\\
12	0.00409600722577218\\
13	0.00409603196870914\\
14	0.00409605706903975\\
15	0.00409608253075475\\
16	0.00409610835787113\\
17	0.00409613455443227\\
18	0.00409616112450847\\
19	0.00409618807219714\\
20	0.00409621540162322\\
21	0.00409624311693958\\
22	0.00409627122232759\\
23	0.00409629972199747\\
24	0.00409632862018893\\
25	0.00409635792117177\\
26	0.00409638762924648\\
27	0.00409641774874495\\
28	0.00409644828403125\\
29	0.00409647923950236\\
30	0.00409651061958913\\
31	0.00409654242875714\\
32	0.00409657467150769\\
33	0.00409660735237884\\
34	0.00409664047594657\\
35	0.00409667404682594\\
36	0.00409670806967228\\
37	0.00409674254918267\\
38	0.00409677749009724\\
39	0.00409681289720065\\
40	0.00409684877532375\\
41	0.00409688512934512\\
42	0.00409692196419291\\
43	0.00409695928484662\\
44	0.00409699709633897\\
45	0.00409703540375805\\
46	0.00409707421224927\\
47	0.00409711352701767\\
48	0.00409715335333018\\
49	0.00409719369651803\\
50	0.00409723456197922\\
51	0.00409727595518121\\
52	0.00409731788166357\\
53	0.00409736034704081\\
54	0.00409740335700531\\
55	0.00409744691733042\\
56	0.00409749103387354\\
57	0.00409753571257934\\
58	0.00409758095948331\\
59	0.00409762678071508\\
60	0.00409767318250206\\
61	0.00409772017117324\\
62	0.00409776775316286\\
63	0.00409781593501451\\
64	0.00409786472338505\\
65	0.00409791412504892\\
66	0.00409796414690219\\
67	0.00409801479596717\\
68	0.00409806607939678\\
69	0.00409811800447917\\
70	0.00409817057864237\\
71	0.00409822380945914\\
72	0.0040982777046519\\
73	0.00409833227209759\\
74	0.00409838751983293\\
75	0.00409844345605942\\
76	0.00409850008914866\\
77	0.00409855742764775\\
78	0.00409861548028459\\
79	0.00409867425597347\\
80	0.00409873376382058\\
81	0.00409879401312954\\
82	0.00409885501340726\\
83	0.00409891677436947\\
84	0.00409897930594667\\
85	0.00409904261828975\\
86	0.00409910672177596\\
87	0.0040991716270148\\
88	0.00409923734485389\\
89	0.00409930388638487\\
90	0.00409937126294934\\
91	0.00409943948614486\\
92	0.00409950856783085\\
93	0.00409957852013454\\
94	0.00409964935545692\\
95	0.00409972108647858\\
96	0.00409979372616584\\
97	0.00409986728777638\\
98	0.00409994178486525\\
99	0.00410001723129069\\
100	0.00410009364121998\\
101	0.0041001710291351\\
102	0.00410024940983868\\
103	0.00410032879845951\\
104	0.00410040921045839\\
105	0.00410049066163375\\
106	0.00410057316812724\\
107	0.0041006567464294\\
108	0.00410074141338527\\
109	0.00410082718619986\\
110	0.00410091408244383\\
111	0.00410100212005902\\
112	0.00410109131736399\\
113	0.00410118169305965\\
114	0.00410127326623474\\
115	0.0041013660563716\\
116	0.00410146008335181\\
117	0.00410155536746177\\
118	0.00410165192939864\\
119	0.00410174979027601\\
120	0.00410184897163003\\
121	0.00410194949542532\\
122	0.00410205138406099\\
123	0.00410215466037715\\
124	0.00410225934766104\\
125	0.00410236546965366\\
126	0.00410247305055639\\
127	0.00410258211503788\\
128	0.00410269268824098\\
129	0.00410280479578989\\
130	0.00410291846379755\\
131	0.00410303371887327\\
132	0.0041031505881303\\
133	0.00410326909919388\\
134	0.00410338928020944\\
135	0.00410351115985074\\
136	0.00410363476732871\\
137	0.00410376013240006\\
138	0.00410388728537628\\
139	0.00410401625713288\\
140	0.00410414707911871\\
141	0.0041042797833657\\
142	0.00410441440249856\\
143	0.00410455096974481\\
144	0.00410468951894524\\
145	0.00410483008456399\\
146	0.00410497270169956\\
147	0.00410511740609558\\
148	0.00410526423415193\\
149	0.00410541322293573\\
150	0.00410556441019279\\
151	0.00410571783435943\\
152	0.00410587353457443\\
153	0.0041060315506911\\
154	0.00410619192328988\\
155	0.00410635469369109\\
156	0.0041065199039677\\
157	0.00410668759695891\\
158	0.00410685781628351\\
159	0.004107030606354\\
160	0.00410720601239067\\
161	0.00410738408043628\\
162	0.00410756485737098\\
163	0.00410774839092757\\
164	0.00410793472970733\\
165	0.00410812392319595\\
166	0.00410831602178023\\
167	0.00410851107676491\\
168	0.00410870914039013\\
169	0.00410891026584936\\
170	0.00410911450730776\\
171	0.00410932191992095\\
172	0.00410953255985457\\
173	0.00410974648430407\\
174	0.00410996375151537\\
175	0.00411018442080578\\
176	0.0041104085525858\\
177	0.0041106362083814\\
178	0.00411086745085676\\
179	0.00411110234383806\\
180	0.00411134095233764\\
181	0.00411158334257883\\
182	0.00411182958202163\\
183	0.004112079739389\\
184	0.00411233388469389\\
185	0.004112592089267\\
186	0.00411285442578532\\
187	0.00411312096830136\\
188	0.00411339179227324\\
189	0.00411366697459539\\
190	0.00411394659363018\\
191	0.00411423072924033\\
192	0.0041145194628219\\
193	0.00411481287733823\\
194	0.00411511105735464\\
195	0.00411541408907388\\
196	0.00411572206037222\\
197	0.0041160350608364\\
198	0.00411635318180128\\
199	0.00411667651638819\\
200	0.004117005159544\\
201	0.00411733920808075\\
202	0.00411767876071612\\
203	0.00411802391811416\\
204	0.00411837478292706\\
205	0.00411873145983705\\
206	0.00411909405559909\\
207	0.00411946267908387\\
208	0.00411983744132134\\
209	0.00412021845554453\\
210	0.00412060583723384\\
211	0.00412099970416156\\
212	0.00412140017643675\\
213	0.00412180737655021\\
214	0.00412222142941977\\
215	0.00412264246243567\\
216	0.00412307060550606\\
217	0.00412350599110271\\
218	0.0041239487543066\\
219	0.00412439903285371\\
220	0.00412485696718078\\
221	0.00412532270047123\\
222	0.00412579637870083\\
223	0.00412627815068381\\
224	0.00412676816811873\\
225	0.00412726658563466\\
226	0.00412777356083737\\
227	0.00412828925435592\\
228	0.0041288138298894\\
229	0.00412934745425412\\
230	0.00412989029743132\\
231	0.00413044253261524\\
232	0.0041310043362622\\
233	0.00413157588814017\\
234	0.00413215737137989\\
235	0.00413274897252666\\
236	0.00413335088159374\\
237	0.00413396329211722\\
238	0.00413458640121268\\
239	0.00413522040963379\\
240	0.00413586552183297\\
241	0.00413652194602454\\
242	0.00413718989425032\\
243	0.00413786958244821\\
244	0.00413856123052352\\
245	0.00413926506242377\\
246	0.00413998130621683\\
247	0.00414071019417249\\
248	0.00414145196284843\\
249	0.00414220685317966\\
250	0.00414297511057278\\
251	0.00414375698500402\\
252	0.00414455273112235\\
253	0.00414536260835737\\
254	0.00414618688103258\\
255	0.00414702581848413\\
256	0.00414787969518288\\
257	0.00414874879085554\\
258	0.00414963339061607\\
259	0.00415053378510179\\
260	0.00415145027061411\\
261	0.00415238314926421\\
262	0.00415333272912317\\
263	0.00415429932437663\\
264	0.00415528325548441\\
265	0.00415628484934529\\
266	0.00415730443946647\\
267	0.00415834236613792\\
268	0.00415939897661162\\
269	0.00416047462528526\\
270	0.00416156967389093\\
271	0.00416268449168826\\
272	0.00416381945566199\\
273	0.00416497495072398\\
274	0.00416615136991966\\
275	0.00416734911463855\\
276	0.00416856859482894\\
277	0.00416981022921671\\
278	0.00417107444552737\\
279	0.00417236168071149\\
280	0.00417367238117411\\
281	0.00417500700300752\\
282	0.00417636601222807\\
283	0.00417774988501727\\
284	0.00417915910796586\\
285	0.00418059417832096\\
286	0.00418205560423672\\
287	0.00418354390502853\\
288	0.00418505961143148\\
289	0.00418660326586487\\
290	0.00418817542270023\\
291	0.00418977664853424\\
292	0.00419140752246745\\
293	0.00419306863638921\\
294	0.00419476059526954\\
295	0.00419648401745842\\
296	0.00419823953499375\\
297	0.00420002779391894\\
298	0.0042018494546114\\
299	0.0042037051921232\\
300	0.00420559569653311\\
301	0.00420752167330453\\
302	0.00420948384366157\\
303	0.00421148294498171\\
304	0.00421351973120494\\
305	0.00421559497326092\\
306	0.00421770945951509\\
307	0.0042198639962358\\
308	0.00422205940808414\\
309	0.0042242965386328\\
310	0.00422657625092935\\
311	0.00422889942811147\\
312	0.00423126697407714\\
313	0.00423367981422548\\
314	0.00423613889609186\\
315	0.00423864519013073\\
316	0.00424119969056351\\
317	0.00424380341627011\\
318	0.00424645741171143\\
319	0.0042491627478955\\
320	0.00425192052339256\\
321	0.00425473186539667\\
322	0.00425759793083365\\
323	0.0042605199075153\\
324	0.00426349901534109\\
325	0.00426653650757716\\
326	0.00426963367219471\\
327	0.0042727918332629\\
328	0.0042760123524123\\
329	0.00427929663038039\\
330	0.00428264610864861\\
331	0.00428606227112797\\
332	0.0042895466458308\\
333	0.00429310080669641\\
334	0.00429672637548867\\
335	0.00430042502375289\\
336	0.00430419847480939\\
337	0.00430804850578537\\
338	0.00431197694982626\\
339	0.0043159856984577\\
340	0.00432007670380599\\
341	0.0043242519807938\\
342	0.00432851360926847\\
343	0.00433286373601202\\
344	0.00433730457657591\\
345	0.00434183841689476\\
346	0.00434646761468339\\
347	0.00435119460075521\\
348	0.00435602188021876\\
349	0.00436095203435385\\
350	0.00436598771960515\\
351	0.00437113167052693\\
352	0.00437638670281178\\
353	0.00438175571708934\\
354	0.00438724170317217\\
355	0.00439284774500271\\
356	0.00439857702484328\\
357	0.00440443282788267\\
358	0.00441041854789761\\
359	0.00441653769327272\\
360	0.00442279389322759\\
361	0.0044291909064023\\
362	0.00443573263321883\\
363	0.00444242312760875\\
364	0.0044492666106289\\
365	0.0044562674862465\\
366	0.00446343035955969\\
367	0.00447076005766673\\
368	0.0044782616533322\\
369	0.0044859404917794\\
370	0.00449380222149397\\
371	0.00450185282282893\\
372	0.00451009863813425\\
373	0.00451854639854589\\
374	0.00452720324248943\\
375	0.00453607672148651\\
376	0.00454517480924487\\
377	0.00455450571499868\\
378	0.00456407786561144\\
379	0.00457389987876267\\
380	0.00458398051178957\\
381	0.00459432865007819\\
382	0.00460495328678225\\
383	0.00461586348087892\\
384	0.00462706830927145\\
385	0.0046385768137451\\
386	0.00465039794434209\\
387	0.0046625405016459\\
388	0.00467501308273088\\
389	0.00468782406537107\\
390	0.0047009815918066\\
391	0.00471449359032724\\
392	0.00472836787429916\\
393	0.00474261236282055\\
394	0.00475723558076776\\
395	0.00477224709037745\\
396	0.00478765835628073\\
397	0.00480348381186787\\
398	0.00481974914418876\\
399	0.00483648354993576\\
400	0.00485371815958732\\
401	0.00487148598747839\\
402	0.00488982180268676\\
403	0.00490876188850308\\
404	0.00492834364062747\\
405	0.00494860488779465\\
406	0.00496958322414589\\
407	0.00499131486987985\\
408	0.00501383256332424\\
409	0.00503716515917807\\
410	0.00506133397257322\\
411	0.00508634837559588\\
412	0.00511220029646462\\
413	0.00513884973395452\\
414	0.00516621948017444\\
415	0.00519418140474314\\
416	0.00522219113231119\\
417	0.00524989647293191\\
418	0.00527723538116368\\
419	0.00530414018582434\\
420	0.00533053757369377\\
421	0.00535634855725208\\
422	0.00538148875596453\\
423	0.00540586895196347\\
424	0.00542939656222645\\
425	0.00545197769857029\\
426	0.00547352034188589\\
427	0.00549393886548258\\
428	0.00551316036094833\\
429	0.00553113346425168\\
430	0.00554784052366552\\
431	0.00556331422444391\\
432	0.00557791451485326\\
433	0.00559216759613327\\
434	0.00560606131561664\\
435	0.00561958666055824\\
436	0.00563273850432294\\
437	0.0056455164115504\\
438	0.00565792548096981\\
439	0.00566997719468297\\
440	0.00568169021734538\\
441	0.00569309105650054\\
442	0.00570421445207726\\
443	0.00571510330308959\\
444	0.00572580785695178\\
445	0.00573638377343506\\
446	0.00574688852537332\\
447	0.00575736051133426\\
448	0.00576781473905054\\
449	0.00577826051033063\\
450	0.00578870823467299\\
451	0.00579916932351866\\
452	0.00580965602840855\\
453	0.00582018121534943\\
454	0.00583075806991991\\
455	0.00584139973202969\\
456	0.00585211886682199\\
457	0.00586292719038153\\
458	0.00587383498817624\\
459	0.0058848506935462\\
460	0.00589598063712908\\
461	0.00590722996532073\\
462	0.00591860360210382\\
463	0.00593010643016358\\
464	0.00594174324038562\\
465	0.0059535186821428\\
466	0.0059654372173562\\
467	0.00597750308200778\\
468	0.00598972025937734\\
469	0.00600209246959655\\
470	0.00601462317989138\\
471	0.00602731563870397\\
472	0.00604017293410104\\
473	0.00605319807152112\\
474	0.00606639402431047\\
475	0.00607976374298815\\
476	0.00609331015374547\\
477	0.00610703615890761\\
478	0.00612094463951541\\
479	0.00613503846005787\\
480	0.00614932047521305\\
481	0.00616379353823352\\
482	0.00617846051035483\\
483	0.00619332427033949\\
484	0.00620838772305117\\
485	0.00622365380589142\\
486	0.00623912549261596\\
487	0.00625480579621778\\
488	0.00627069777218987\\
489	0.00628680452212941\\
490	0.00630312919763236\\
491	0.00631967500441896\\
492	0.00633644520662642\\
493	0.00635344313121247\\
494	0.00637067217243157\\
495	0.00638813579637969\\
496	0.00640583754565251\\
497	0.00642378104422246\\
498	0.00644197000268395\\
499	0.00646040822396084\\
500	0.00647909960951374\\
501	0.006498048166091\\
502	0.00651725801307465\\
503	0.00653673339048063\\
504	0.00655647866768176\\
505	0.0065764983529322\\
506	0.00659679710378154\\
507	0.00661737973847604\\
508	0.00663825124845157\\
509	0.00665941681202778\\
510	0.00668088180941628\\
511	0.00670265183916187\\
512	0.00672473273614287\\
513	0.00674713059126367\\
514	0.0067698517729772\\
515	0.00679290295077906\\
516	0.00681629112081456\\
517	0.00684002363373771\\
518	0.00686410822495721\\
519	0.00688855304739165\\
520	0.00691336670683068\\
521	0.00693855829996578\\
522	0.00696413745491618\\
523	0.00699011437411914\\
524	0.00701649987951138\\
525	0.00704330545914029\\
526	0.00707054332066159\\
527	0.00709822644926038\\
528	0.00712636866359132\\
529	0.0071549847035114\\
530	0.00718409021224089\\
531	0.00721370155228759\\
532	0.00724383586893235\\
533	0.00727451118922645\\
534	0.00730573830354627\\
535	0.00733754289382939\\
536	0.00736995197449274\\
537	0.00740299261041146\\
538	0.00743669497365421\\
539	0.00747108869579545\\
540	0.00750619254437731\\
541	0.00754202239752112\\
542	0.00757858696594791\\
543	0.00761588370558736\\
544	0.00765381442335629\\
545	0.00769240422359834\\
546	0.00773170952485417\\
547	0.00777184860363677\\
548	0.0078129512996561\\
549	0.0078550743470806\\
550	0.00789829083241618\\
551	0.00794266950789453\\
552	0.00798828443682109\\
553	0.00803524805689781\\
554	0.00808370675822525\\
555	0.00813245686176951\\
556	0.00817966759564629\\
557	0.00822469162758563\\
558	0.00826987878715762\\
559	0.00831532757736804\\
560	0.00836118747187362\\
561	0.00840745876100177\\
562	0.00845408897122118\\
563	0.00850099755665664\\
564	0.00854809439039289\\
565	0.00859487531085683\\
566	0.00864051776728843\\
567	0.00868567121298489\\
568	0.00873085196361323\\
569	0.00877609289333616\\
570	0.00882137592721109\\
571	0.00886636048870486\\
572	0.00891108851230718\\
573	0.00895600554864665\\
574	0.00900111866282342\\
575	0.00904639939181835\\
576	0.00909181174974404\\
577	0.00913731638890149\\
578	0.00918287074791749\\
579	0.00922842901963624\\
580	0.00927394213588079\\
581	0.00931935780498493\\
582	0.00936462062738148\\
583	0.00940967231379966\\
584	0.00945445203650304\\
585	0.00949889695046246\\
586	0.00954294292772186\\
587	0.00958652555102227\\
588	0.00962958140299686\\
589	0.00967204963943729\\
590	0.0097138736812812\\
591	0.00975500241768019\\
592	0.00979538909575469\\
593	0.00983498278816331\\
594	0.00987369853931868\\
595	0.00991118387968948\\
596	0.00994658651044256\\
597	0.00997788999445116\\
598	0.010000292044645\\
599	0\\
600	0\\
};
\addplot [color=mycolor7,solid,forget plot]
  table[row sep=crcr]{%
1	0.00409529888158573\\
2	0.00409531335916626\\
3	0.00409532803996346\\
4	0.00409534292632739\\
5	0.00409535802063159\\
6	0.00409537332527361\\
7	0.00409538884267553\\
8	0.0040954045752844\\
9	0.00409542052557287\\
10	0.00409543669603968\\
11	0.00409545308921039\\
12	0.00409546970763792\\
13	0.00409548655390317\\
14	0.0040955036306158\\
15	0.00409552094041491\\
16	0.00409553848596972\\
17	0.00409555626998042\\
18	0.00409557429517898\\
19	0.00409559256432993\\
20	0.00409561108023117\\
21	0.00409562984571509\\
22	0.00409564886364924\\
23	0.00409566813693751\\
24	0.00409568766852107\\
25	0.00409570746137935\\
26	0.00409572751853122\\
27	0.00409574784303607\\
28	0.00409576843799497\\
29	0.00409578930655189\\
30	0.00409581045189494\\
31	0.00409583187725764\\
32	0.00409585358592025\\
33	0.00409587558121114\\
34	0.00409589786650818\\
35	0.0040959204452403\\
36	0.00409594332088887\\
37	0.00409596649698929\\
38	0.00409598997713262\\
39	0.00409601376496716\\
40	0.00409603786420016\\
41	0.00409606227859958\\
42	0.00409608701199582\\
43	0.00409611206828361\\
44	0.00409613745142382\\
45	0.00409616316544542\\
46	0.00409618921444747\\
47	0.00409621560260106\\
48	0.00409624233415152\\
49	0.00409626941342034\\
50	0.00409629684480756\\
51	0.00409632463279382\\
52	0.00409635278194263\\
53	0.00409638129690279\\
54	0.00409641018241064\\
55	0.00409643944329256\\
56	0.00409646908446722\\
57	0.00409649911094831\\
58	0.004096529527847\\
59	0.00409656034037446\\
60	0.00409659155384456\\
61	0.00409662317367653\\
62	0.00409665520539764\\
63	0.00409668765464605\\
64	0.0040967205271735\\
65	0.00409675382884819\\
66	0.0040967875656577\\
67	0.00409682174371186\\
68	0.00409685636924571\\
69	0.00409689144862251\\
70	0.00409692698833673\\
71	0.00409696299501717\\
72	0.00409699947543001\\
73	0.0040970364364819\\
74	0.00409707388522319\\
75	0.00409711182885107\\
76	0.00409715027471284\\
77	0.00409718923030907\\
78	0.00409722870329695\\
79	0.00409726870149347\\
80	0.00409730923287888\\
81	0.00409735030559997\\
82	0.00409739192797341\\
83	0.00409743410848921\\
84	0.00409747685581403\\
85	0.00409752017879478\\
86	0.00409756408646196\\
87	0.00409760858803316\\
88	0.00409765369291657\\
89	0.0040976994107146\\
90	0.00409774575122724\\
91	0.00409779272445585\\
92	0.00409784034060656\\
93	0.00409788861009412\\
94	0.00409793754354529\\
95	0.00409798715180282\\
96	0.00409803744592888\\
97	0.00409808843720899\\
98	0.00409814013715571\\
99	0.00409819255751257\\
100	0.00409824571025767\\
101	0.00409829960760789\\
102	0.00409835426202258\\
103	0.00409840968620775\\
104	0.00409846589311991\\
105	0.0040985228959703\\
106	0.00409858070822895\\
107	0.00409863934362897\\
108	0.00409869881617072\\
109	0.00409875914012631\\
110	0.00409882033004383\\
111	0.004098882400752\\
112	0.00409894536736461\\
113	0.00409900924528532\\
114	0.00409907405021224\\
115	0.00409913979814298\\
116	0.00409920650537936\\
117	0.00409927418853265\\
118	0.00409934286452856\\
119	0.00409941255061264\\
120	0.00409948326435549\\
121	0.00409955502365831\\
122	0.00409962784675861\\
123	0.00409970175223569\\
124	0.00409977675901671\\
125	0.00409985288638252\\
126	0.00409993015397387\\
127	0.00410000858179754\\
128	0.00410008819023275\\
129	0.00410016900003767\\
130	0.00410025103235615\\
131	0.0041003343087243\\
132	0.00410041885107761\\
133	0.00410050468175801\\
134	0.00410059182352093\\
135	0.00410068029954293\\
136	0.00410077013342902\\
137	0.00410086134922046\\
138	0.00410095397140261\\
139	0.00410104802491292\\
140	0.00410114353514911\\
141	0.00410124052797757\\
142	0.00410133902974192\\
143	0.00410143906727174\\
144	0.00410154066789134\\
145	0.00410164385942924\\
146	0.00410174867022732\\
147	0.00410185512915045\\
148	0.00410196326559641\\
149	0.00410207310950645\\
150	0.0041021846913758\\
151	0.00410229804226414\\
152	0.00410241319380625\\
153	0.00410253017822295\\
154	0.00410264902833234\\
155	0.00410276977756104\\
156	0.00410289245995604\\
157	0.00410301711019652\\
158	0.00410314376360608\\
159	0.00410327245616516\\
160	0.00410340322452384\\
161	0.00410353610601481\\
162	0.00410367113866676\\
163	0.00410380836121798\\
164	0.00410394781313014\\
165	0.00410408953460283\\
166	0.00410423356658778\\
167	0.00410437995080393\\
168	0.00410452872975262\\
169	0.00410467994673294\\
170	0.00410483364585779\\
171	0.00410498987207004\\
172	0.00410514867115905\\
173	0.0041053100897776\\
174	0.00410547417545914\\
175	0.00410564097663547\\
176	0.00410581054265464\\
177	0.00410598292379933\\
178	0.00410615817130577\\
179	0.00410633633738256\\
180	0.0041065174752303\\
181	0.0041067016390614\\
182	0.00410688888412046\\
183	0.0041070792667047\\
184	0.00410727284418501\\
185	0.0041074696750275\\
186	0.00410766981881499\\
187	0.00410787333626945\\
188	0.00410808028927429\\
189	0.00410829074089746\\
190	0.00410850475541469\\
191	0.0041087223983331\\
192	0.00410894373641531\\
193	0.00410916883770393\\
194	0.00410939777154627\\
195	0.00410963060861962\\
196	0.00410986742095669\\
197	0.00411010828197159\\
198	0.00411035326648619\\
199	0.00411060245075664\\
200	0.00411085591250051\\
201	0.00411111373092417\\
202	0.0041113759867505\\
203	0.00411164276224718\\
204	0.00411191414125503\\
205	0.004112190209217\\
206	0.00411247105320746\\
207	0.00411275676196188\\
208	0.00411304742590691\\
209	0.00411334313719079\\
210	0.00411364398971436\\
211	0.00411395007916245\\
212	0.00411426150303559\\
213	0.00411457836068234\\
214	0.00411490075333212\\
215	0.00411522878412847\\
216	0.00411556255816302\\
217	0.00411590218250975\\
218	0.0041162477662602\\
219	0.00411659942055925\\
220	0.00411695725864131\\
221	0.00411732139586763\\
222	0.00411769194976415\\
223	0.00411806904006028\\
224	0.00411845278872851\\
225	0.00411884332002493\\
226	0.00411924076053081\\
227	0.00411964523919514\\
228	0.00412005688737832\\
229	0.00412047583889703\\
230	0.00412090223007023\\
231	0.00412133619976646\\
232	0.00412177788945259\\
233	0.00412222744324388\\
234	0.00412268500795542\\
235	0.0041231507331553\\
236	0.00412362477121914\\
237	0.00412410727738639\\
238	0.00412459840981831\\
239	0.00412509832965749\\
240	0.00412560720108942\\
241	0.00412612519140568\\
242	0.004126652471069\\
243	0.00412718921378027\\
244	0.00412773559654744\\
245	0.00412829179975632\\
246	0.00412885800724327\\
247	0.0041294344063702\\
248	0.00413002118810105\\
249	0.00413061854708067\\
250	0.00413122668171561\\
251	0.00413184579425673\\
252	0.00413247609088421\\
253	0.00413311778179431\\
254	0.00413377108128832\\
255	0.00413443620786323\\
256	0.00413511338430401\\
257	0.00413580283777837\\
258	0.00413650479993368\\
259	0.00413721950699552\\
260	0.00413794719986878\\
261	0.00413868812424015\\
262	0.0041394425306831\\
263	0.00414021067476449\\
264	0.00414099281715355\\
265	0.00414178922373251\\
266	0.00414260016570967\\
267	0.00414342591973415\\
268	0.00414426676801282\\
269	0.00414512299842949\\
270	0.00414599490466596\\
271	0.00414688278632531\\
272	0.00414778694905744\\
273	0.00414870770468688\\
274	0.00414964537134285\\
275	0.00415060027359177\\
276	0.00415157274257249\\
277	0.00415256311613383\\
278	0.00415357173897492\\
279	0.00415459896278864\\
280	0.0041556451464081\\
281	0.00415671065595639\\
282	0.00415779586500009\\
283	0.00415890115470628\\
284	0.0041600269140035\\
285	0.00416117353974716\\
286	0.00416234143688919\\
287	0.00416353101865285\\
288	0.00416474270671286\\
289	0.00416597693138086\\
290	0.00416723413179686\\
291	0.00416851475612708\\
292	0.00416981926176857\\
293	0.00417114811556082\\
294	0.00417250179400487\\
295	0.00417388078349076\\
296	0.00417528558053294\\
297	0.0041767166920149\\
298	0.00417817463544284\\
299	0.00417965993920883\\
300	0.00418117314286336\\
301	0.00418271479739847\\
302	0.00418428546554185\\
303	0.00418588572206159\\
304	0.00418751615408255\\
305	0.00418917736141384\\
306	0.00419086995688843\\
307	0.00419259456671431\\
308	0.00419435183083869\\
309	0.00419614240332592\\
310	0.00419796695274931\\
311	0.00419982616259705\\
312	0.0042017207316897\\
313	0.00420365137459837\\
314	0.00420561882208432\\
315	0.00420762382155358\\
316	0.0042096671375247\\
317	0.00421174955210941\\
318	0.00421387186550712\\
319	0.00421603489651504\\
320	0.00421823948305327\\
321	0.00422048648270602\\
322	0.00422277677327956\\
323	0.00422511125337804\\
324	0.00422749084300021\\
325	0.00422991648415631\\
326	0.00423238914150635\\
327	0.00423490980302193\\
328	0.00423747948067404\\
329	0.00424009921114819\\
330	0.00424277005658351\\
331	0.00424549310533493\\
332	0.00424826947277318\\
333	0.00425110030211756\\
334	0.00425398676530208\\
335	0.0042569300638763\\
336	0.00425993142994667\\
337	0.00426299212717308\\
338	0.00426611345181838\\
339	0.00426929673383262\\
340	0.00427254333798882\\
341	0.00427585466507403\\
342	0.00427923215314009\\
343	0.00428267727882064\\
344	0.00428619155872267\\
345	0.00428977655090712\\
346	0.00429343385647985\\
347	0.00429716512129608\\
348	0.00430097203779398\\
349	0.00430485634670329\\
350	0.0043088198390626\\
351	0.00431286435833666\\
352	0.00431699180268599\\
353	0.00432120412736427\\
354	0.00432550334724409\\
355	0.00432989153936209\\
356	0.00433437084559166\\
357	0.00433894347548184\\
358	0.00434361170919735\\
359	0.00434837790057111\\
360	0.0043532444804495\\
361	0.00435821396036348\\
362	0.00436328893615132\\
363	0.00436847209172436\\
364	0.00437376620296244\\
365	0.00437917414171798\\
366	0.00438469887990066\\
367	0.00439034349361296\\
368	0.00439611116732586\\
369	0.00440200519807622\\
370	0.00440802899928185\\
371	0.00441418610453587\\
372	0.00442048017121563\\
373	0.00442691498401626\\
374	0.00443349445901006\\
375	0.00444022264981431\\
376	0.00444710374764763\\
377	0.00445414208774742\\
378	0.00446134215594629\\
379	0.00446870859513666\\
380	0.00447624621648438\\
381	0.00448396001123058\\
382	0.00449185516245293\\
383	0.00449993705836801\\
384	0.00450821130742561\\
385	0.00451668375545995\\
386	0.00452536050518122\\
387	0.00453424793849747\\
388	0.00454335274328614\\
389	0.00455268194107426\\
390	0.00456224291751835\\
391	0.00457204345661088\\
392	0.00458209177891305\\
393	0.00459239658516182\\
394	0.00460296707186981\\
395	0.00461381294564437\\
396	0.00462494440577857\\
397	0.00463637213268345\\
398	0.00464810711780087\\
399	0.00466016060587126\\
400	0.00467254406173833\\
401	0.00468526913030364\\
402	0.00469834758954639\\
403	0.00471179129669854\\
404	0.00472561212869725\\
405	0.004739821942918\\
406	0.0047544325287009\\
407	0.00476945555926563\\
408	0.00478490272671644\\
409	0.0048007856878958\\
410	0.00481711609581658\\
411	0.00483390569989827\\
412	0.00485116620336472\\
413	0.0048689103631968\\
414	0.00488715290178423\\
415	0.0049059118337603\\
416	0.00492521605889702\\
417	0.00494510181564877\\
418	0.0049656072975093\\
419	0.0049867723737793\\
420	0.00500863798590982\\
421	0.00503124585634181\\
422	0.0050546384992814\\
423	0.00507885959850083\\
424	0.00510395023558282\\
425	0.00512994484037258\\
426	0.00515686433680294\\
427	0.00518471516750045\\
428	0.00521348289821372\\
429	0.00524312368771077\\
430	0.00527355295684454\\
431	0.0053046303856445\\
432	0.00533589051850966\\
433	0.00536671291745671\\
434	0.00539701960726992\\
435	0.00542672580803933\\
436	0.00545573994399862\\
437	0.00548396439442113\\
438	0.00551129656975341\\
439	0.00553763057692947\\
440	0.00556285975532503\\
441	0.00558688044713349\\
442	0.00560959744130947\\
443	0.00563093167375537\\
444	0.00565083095726243\\
445	0.00566928477518812\\
446	0.00568634437505773\\
447	0.00570257952405059\\
448	0.00571841830284484\\
449	0.00573385115307405\\
450	0.00574887307694424\\
451	0.00576348456606036\\
452	0.0057776925779305\\
453	0.00579151151564601\\
454	0.00580496413961544\\
455	0.00581808230232539\\
456	0.00583090734542928\\
457	0.00584348992964762\\
458	0.00585588896168109\\
459	0.00586816914156737\\
460	0.00588039646581119\\
461	0.00589260790387998\\
462	0.00590481832838029\\
463	0.00591703939841473\\
464	0.0059292839423036\\
465	0.0059415657811149\\
466	0.00595389947914521\\
467	0.00596630001433183\\
468	0.00597878236640431\\
469	0.00599136102925191\\
470	0.00600404946788492\\
471	0.00601685956198455\\
472	0.00602980111100503\\
473	0.00604288152490197\\
474	0.00605610677287527\\
475	0.00606948250346558\\
476	0.00608301430127271\\
477	0.00609670762906814\\
478	0.00611056777217856\\
479	0.00612459978900502\\
480	0.00613880847232134\\
481	0.00615319832658378\\
482	0.00616777356663\\
483	0.0061825381424517\\
484	0.00619749579257747\\
485	0.00621265012407515\\
486	0.00622800469780967\\
487	0.00624356306178102\\
488	0.00625932874999612\\
489	0.00627530528369641\\
490	0.00629149617516352\\
491	0.00630790493418527\\
492	0.00632453507706829\\
493	0.00634139013782821\\
494	0.00635847368088586\\
495	0.00637578931427811\\
496	0.00639334070211891\\
497	0.00641113157493753\\
498	0.00642916573711289\\
499	0.00644744707338765\\
500	0.00646597955615283\\
501	0.00648476725348922\\
502	0.00650381433793958\\
503	0.00652312509597778\\
504	0.00654270393814206\\
505	0.00656255540981227\\
506	0.00658268420263934\\
507	0.00660309516668271\\
508	0.00662379332337688\\
509	0.00664478387952437\\
510	0.00666607224253939\\
511	0.00668766403709905\\
512	0.00670956512331254\\
513	0.00673178161652677\\
514	0.00675431990889423\\
515	0.00677718669283627\\
516	0.00680038898653935\\
517	0.00682393416162442\\
518	0.00684782997312543\\
519	0.0068720845919022\\
520	0.00689670663959\\
521	0.00692170522615219\\
522	0.00694708999005904\\
523	0.00697287114105492\\
524	0.00699905950538988\\
525	0.00702566657328029\\
526	0.00705270454805575\\
527	0.00708018639631153\\
528	0.00710812589829871\\
529	0.00713653769710684\\
530	0.00716543734818967\\
531	0.00719484137021395\\
532	0.007224767287208\\
533	0.00725523365595185\\
534	0.00728626016991656\\
535	0.00731786730671243\\
536	0.0073500762855338\\
537	0.00738290903487387\\
538	0.00741638488680999\\
539	0.00745052410967822\\
540	0.00748535622462389\\
541	0.00752091025166277\\
542	0.00755721342335093\\
543	0.00759429613161743\\
544	0.00763218088623283\\
545	0.00767086853324643\\
546	0.00771034715895772\\
547	0.00775055806174717\\
548	0.00779145697827102\\
549	0.00783309498128678\\
550	0.00787553646602525\\
551	0.00791895405846008\\
552	0.00796342795168289\\
553	0.00800902558206601\\
554	0.00805583223394632\\
555	0.00810394296610672\\
556	0.00815348917904342\\
557	0.00820452788108392\\
558	0.00825427676315796\\
559	0.00830217862287868\\
560	0.00834893731572075\\
561	0.00839579436605613\\
562	0.00844291676212203\\
563	0.00849036691366464\\
564	0.00853811179111934\\
565	0.00858607445145949\\
566	0.00863415405798469\\
567	0.00868157869679225\\
568	0.00872782524480766\\
569	0.00877366426417583\\
570	0.00881945432030142\\
571	0.00886522995351837\\
572	0.00891084129040892\\
573	0.00895597597182961\\
574	0.00900111411185122\\
575	0.00904639837989421\\
576	0.00909181141669918\\
577	0.00913731624296646\\
578	0.00918287067265086\\
579	0.00922842897919994\\
580	0.00927394211349327\\
581	0.00931935779255734\\
582	0.0093646206206887\\
583	0.00940967231047064\\
584	0.00945445203507202\\
585	0.00949889694999284\\
586	0.00954294292762137\\
587	0.00958652555101391\\
588	0.00962958140299686\\
589	0.00967204963943729\\
590	0.0097138736812812\\
591	0.00975500241768019\\
592	0.00979538909575469\\
593	0.00983498278816331\\
594	0.00987369853931868\\
595	0.00991118387968948\\
596	0.00994658651044256\\
597	0.00997788999445116\\
598	0.010000292044645\\
599	0\\
600	0\\
};
\addplot [color=mycolor8,solid,forget plot]
  table[row sep=crcr]{%
1	0.0040944850682663\\
2	0.00409449503326179\\
3	0.00409450513478612\\
4	0.00409451537454594\\
5	0.0040945257542724\\
6	0.00409453627572179\\
7	0.00409454694067606\\
8	0.00409455775094352\\
9	0.00409456870835943\\
10	0.00409457981478679\\
11	0.00409459107211679\\
12	0.00409460248226963\\
13	0.00409461404719534\\
14	0.0040946257688743\\
15	0.00409463764931816\\
16	0.00409464969057058\\
17	0.00409466189470794\\
18	0.00409467426384026\\
19	0.00409468680011199\\
20	0.00409469950570288\\
21	0.00409471238282884\\
22	0.00409472543374286\\
23	0.00409473866073591\\
24	0.00409475206613794\\
25	0.00409476565231875\\
26	0.00409477942168906\\
27	0.00409479337670149\\
28	0.00409480751985164\\
29	0.00409482185367903\\
30	0.00409483638076831\\
31	0.00409485110375029\\
32	0.00409486602530311\\
33	0.00409488114815337\\
34	0.00409489647507731\\
35	0.00409491200890193\\
36	0.00409492775250638\\
37	0.00409494370882305\\
38	0.00409495988083891\\
39	0.00409497627159686\\
40	0.0040949928841969\\
41	0.00409500972179762\\
42	0.00409502678761747\\
43	0.00409504408493619\\
44	0.00409506161709623\\
45	0.0040950793875042\\
46	0.00409509739963229\\
47	0.0040951156570198\\
48	0.00409513416327465\\
49	0.00409515292207488\\
50	0.00409517193717029\\
51	0.00409519121238387\\
52	0.00409521075161362\\
53	0.00409523055883402\\
54	0.00409525063809774\\
55	0.00409527099353731\\
56	0.00409529162936692\\
57	0.00409531254988402\\
58	0.00409533375947105\\
59	0.00409535526259738\\
60	0.00409537706382103\\
61	0.0040953991677904\\
62	0.00409542157924627\\
63	0.00409544430302357\\
64	0.00409546734405334\\
65	0.00409549070736458\\
66	0.0040955143980863\\
67	0.00409553842144936\\
68	0.00409556278278859\\
69	0.00409558748754472\\
70	0.0040956125412665\\
71	0.00409563794961268\\
72	0.00409566371835417\\
73	0.00409568985337613\\
74	0.0040957163606802\\
75	0.00409574324638658\\
76	0.00409577051673624\\
77	0.0040957981780932\\
78	0.00409582623694681\\
79	0.00409585469991399\\
80	0.00409588357374152\\
81	0.0040959128653085\\
82	0.00409594258162867\\
83	0.00409597272985288\\
84	0.00409600331727147\\
85	0.00409603435131681\\
86	0.00409606583956581\\
87	0.00409609778974255\\
88	0.00409613020972078\\
89	0.00409616310752666\\
90	0.00409619649134142\\
91	0.0040962303695041\\
92	0.00409626475051434\\
93	0.00409629964303513\\
94	0.00409633505589592\\
95	0.00409637099809522\\
96	0.00409640747880389\\
97	0.004096444507368\\
98	0.00409648209331201\\
99	0.00409652024634192\\
100	0.00409655897634848\\
101	0.00409659829341052\\
102	0.00409663820779827\\
103	0.00409667872997674\\
104	0.00409671987060928\\
105	0.00409676164056121\\
106	0.00409680405090336\\
107	0.00409684711291586\\
108	0.00409689083809193\\
109	0.00409693523814174\\
110	0.00409698032499651\\
111	0.00409702611081243\\
112	0.00409707260797493\\
113	0.00409711982910283\\
114	0.0040971677870529\\
115	0.00409721649492399\\
116	0.00409726596606184\\
117	0.00409731621406359\\
118	0.00409736725278263\\
119	0.00409741909633328\\
120	0.004097471759096\\
121	0.00409752525572223\\
122	0.00409757960113968\\
123	0.00409763481055758\\
124	0.00409769089947206\\
125	0.0040977478836717\\
126	0.00409780577924305\\
127	0.0040978646025764\\
128	0.00409792437037164\\
129	0.00409798509964415\\
130	0.00409804680773077\\
131	0.00409810951229613\\
132	0.00409817323133875\\
133	0.00409823798319739\\
134	0.0040983037865577\\
135	0.00409837066045853\\
136	0.00409843862429871\\
137	0.00409850769784378\\
138	0.00409857790123282\\
139	0.00409864925498518\\
140	0.00409872178000752\\
141	0.00409879549760041\\
142	0.00409887042946505\\
143	0.00409894659770956\\
144	0.00409902402485492\\
145	0.00409910273383964\\
146	0.00409918274802353\\
147	0.00409926409118955\\
148	0.00409934678754345\\
149	0.00409943086171322\\
150	0.00409951633875851\\
151	0.00409960324420857\\
152	0.00409969160407124\\
153	0.00409978144484217\\
154	0.00409987279351427\\
155	0.00409996567758716\\
156	0.00410006012507712\\
157	0.00410015616452687\\
158	0.00410025382501584\\
159	0.00410035313617052\\
160	0.00410045412817502\\
161	0.00410055683178177\\
162	0.00410066127832264\\
163	0.00410076749971998\\
164	0.00410087552849824\\
165	0.0041009853977953\\
166	0.00410109714137453\\
167	0.00410121079363689\\
168	0.00410132638963302\\
169	0.00410144396507606\\
170	0.00410156355635425\\
171	0.00410168520054384\\
172	0.00410180893542254\\
173	0.00410193479948294\\
174	0.00410206283194616\\
175	0.00410219307277598\\
176	0.00410232556269296\\
177	0.00410246034318903\\
178	0.00410259745654214\\
179	0.00410273694583135\\
180	0.00410287885495212\\
181	0.00410302322863191\\
182	0.00410317011244582\\
183	0.00410331955283289\\
184	0.00410347159711248\\
185	0.00410362629350077\\
186	0.00410378369112799\\
187	0.00410394384005544\\
188	0.00410410679129315\\
189	0.00410427259681782\\
190	0.00410444130959084\\
191	0.00410461298357691\\
192	0.00410478767376283\\
193	0.00410496543617664\\
194	0.00410514632790705\\
195	0.00410533040712331\\
196	0.00410551773309538\\
197	0.00410570836621455\\
198	0.00410590236801416\\
199	0.00410609980119111\\
200	0.00410630072962732\\
201	0.00410650521841196\\
202	0.0041067133338639\\
203	0.00410692514355454\\
204	0.00410714071633124\\
205	0.00410736012234113\\
206	0.00410758343305527\\
207	0.00410781072129342\\
208	0.00410804206124933\\
209	0.00410827752851641\\
210	0.00410851720011405\\
211	0.00410876115451436\\
212	0.00410900947166952\\
213	0.00410926223303989\\
214	0.00410951952162238\\
215	0.00410978142197975\\
216	0.00411004802027031\\
217	0.00411031940427848\\
218	0.00411059566344596\\
219	0.00411087688890343\\
220	0.0041111631735034\\
221	0.00411145461185331\\
222	0.00411175130034976\\
223	0.0041120533372133\\
224	0.00411236082252425\\
225	0.00411267385825921\\
226	0.00411299254832841\\
227	0.00411331699861403\\
228	0.00411364731700936\\
229	0.00411398361345889\\
230	0.00411432599999937\\
231	0.00411467459080176\\
232	0.00411502950221417\\
233	0.00411539085280594\\
234	0.00411575876341246\\
235	0.00411613335718126\\
236	0.00411651475961895\\
237	0.00411690309863952\\
238	0.00411729850461322\\
239	0.00411770111041709\\
240	0.00411811105148613\\
241	0.00411852846586585\\
242	0.00411895349426588\\
243	0.00411938628011473\\
244	0.00411982696961559\\
245	0.00412027571180354\\
246	0.00412073265860372\\
247	0.00412119796489075\\
248	0.00412167178854948\\
249	0.00412215429053686\\
250	0.004122645634945\\
251	0.00412314598906578\\
252	0.00412365552345631\\
253	0.00412417441200613\\
254	0.00412470283200538\\
255	0.00412524096421461\\
256	0.00412578899293563\\
257	0.0041263471060841\\
258	0.00412691549526344\\
259	0.00412749435584006\\
260	0.00412808388702039\\
261	0.00412868429192916\\
262	0.00412929577768935\\
263	0.00412991855550383\\
264	0.0041305528407385\\
265	0.00413119885300734\\
266	0.00413185681625898\\
267	0.00413252695886524\\
268	0.00413320951371146\\
269	0.0041339047182888\\
270	0.00413461281478836\\
271	0.00413533405019761\\
272	0.00413606867639873\\
273	0.0041368169502692\\
274	0.00413757913378469\\
275	0.00413835549412417\\
276	0.00413914630377745\\
277	0.00413995184065516\\
278	0.00414077238820099\\
279	0.00414160823550564\\
280	0.00414245967742329\\
281	0.00414332701469304\\
282	0.00414421055406293\\
283	0.00414511060841741\\
284	0.00414602749690825\\
285	0.00414696154508837\\
286	0.00414791308504986\\
287	0.00414888245556494\\
288	0.00414987000223126\\
289	0.00415087607762071\\
290	0.00415190104143217\\
291	0.0041529452606486\\
292	0.00415400910969802\\
293	0.00415509297061887\\
294	0.00415619723322993\\
295	0.00415732229530426\\
296	0.00415846856274837\\
297	0.00415963644978543\\
298	0.00416082637914369\\
299	0.00416203878224955\\
300	0.00416327409942557\\
301	0.00416453278009381\\
302	0.00416581528298421\\
303	0.00416712207634835\\
304	0.00416845363817869\\
305	0.00416981045643338\\
306	0.00417119302926692\\
307	0.00417260186526686\\
308	0.0041740374836967\\
309	0.00417550041474544\\
310	0.00417699119978404\\
311	0.00417851039162818\\
312	0.00418005855480817\\
313	0.00418163626584737\\
314	0.00418324411354955\\
315	0.00418488269929651\\
316	0.00418655263735967\\
317	0.0041882545552216\\
318	0.00418998909390109\\
319	0.00419175690828564\\
320	0.00419355866747476\\
321	0.00419539505513425\\
322	0.00419726676986236\\
323	0.00419917452556858\\
324	0.00420111905186532\\
325	0.00420310109447362\\
326	0.0042051214156433\\
327	0.0042071807945886\\
328	0.00420928002793997\\
329	0.00421141993021198\\
330	0.00421360133428882\\
331	0.00421582509192851\\
332	0.00421809207428652\\
333	0.0042204031724594\\
334	0.0042227592980493\\
335	0.00422516138375087\\
336	0.00422761038396227\\
337	0.00423010727542065\\
338	0.00423265305786158\\
339	0.00423524875470434\\
340	0.00423789541376437\\
341	0.00424059410799402\\
342	0.00424334593625302\\
343	0.00424615202410991\\
344	0.00424901352467674\\
345	0.00425193161947885\\
346	0.00425490751936006\\
347	0.00425794246542194\\
348	0.0042610377299823\\
349	0.00426419461758929\\
350	0.00426741446607708\\
351	0.00427069864766853\\
352	0.00427404857012499\\
353	0.00427746567794334\\
354	0.00428095145359514\\
355	0.00428450741882037\\
356	0.00428813513598078\\
357	0.00429183620947113\\
358	0.00429561228719527\\
359	0.00429946506212598\\
360	0.00430339627395013\\
361	0.00430740771077397\\
362	0.00431150121091045\\
363	0.00431567866475392\\
364	0.00431994201674703\\
365	0.00432429326744639\\
366	0.00432873447569465\\
367	0.00433326776090921\\
368	0.00433789530549518\\
369	0.00434261935737215\\
370	0.00434744223265894\\
371	0.00435236631852469\\
372	0.00435739407623897\\
373	0.00436252804447881\\
374	0.00436777084290759\\
375	0.00437312517544296\\
376	0.00437859383406985\\
377	0.00438417970281689\\
378	0.00438988576193537\\
379	0.00439571509262823\\
380	0.00440167088194655\\
381	0.00440775642781222\\
382	0.00441397514429105\\
383	0.00442033056711756\\
384	0.00442682635946839\\
385	0.00443346631798028\\
386	0.00444025437902488\\
387	0.00444719462527896\\
388	0.00445429129230295\\
389	0.0044615487752745\\
390	0.00446897163596021\\
391	0.00447656460997009\\
392	0.00448433261412566\\
393	0.00449228075180321\\
394	0.00450041431890918\\
395	0.0045087388093455\\
396	0.0045172599237471\\
397	0.00452598357799719\\
398	0.00453491591445068\\
399	0.00454406331532495\\
400	0.0045534324175252\\
401	0.00456303012911759\\
402	0.00457286364769764\\
403	0.00458294048112205\\
404	0.00459326847221748\\
405	0.00460385582532282\\
406	0.00461471113660434\\
407	0.00462584343692504\\
408	0.00463726221812296\\
409	0.00464897746180932\\
410	0.00466099966391952\\
411	0.00467333983998608\\
412	0.00468600960271405\\
413	0.00469902118314442\\
414	0.00471238743222883\\
415	0.00472612181762712\\
416	0.00474023824665073\\
417	0.00475475092648928\\
418	0.00476967431722962\\
419	0.00478502307559469\\
420	0.00480081204463977\\
421	0.0048170562845384\\
422	0.00483377108440233\\
423	0.00485097166490057\\
424	0.00486867295982357\\
425	0.00488688942746511\\
426	0.00490563551255536\\
427	0.00492492582990952\\
428	0.00494477550822037\\
429	0.00496520076900319\\
430	0.00498621983945126\\
431	0.00500785429668989\\
432	0.00503013499635789\\
433	0.00505310424657145\\
434	0.00507680680091776\\
435	0.0051012920114171\\
436	0.00512661182744845\\
437	0.00515281394145678\\
438	0.00517994117337858\\
439	0.00520803153272416\\
440	0.00523711506502273\\
441	0.00526720957702407\\
442	0.00529831488517551\\
443	0.00533040512244193\\
444	0.00536341849519898\\
445	0.00539724369300036\\
446	0.00543170189834466\\
447	0.00546610176344987\\
448	0.00549990610319642\\
449	0.00553301837589037\\
450	0.00556533487705187\\
451	0.00559674536991488\\
452	0.0056271342825192\\
453	0.00565638269071956\\
454	0.00568437138183165\\
455	0.00571098540078678\\
456	0.00573612057040539\\
457	0.00575969257333471\\
458	0.00578164973338271\\
459	0.00580199069287926\\
460	0.00582078861599557\\
461	0.0058388742305532\\
462	0.00585652106058518\\
463	0.00587372288773997\\
464	0.00589047957807101\\
465	0.00590679819241133\\
466	0.00592269410962667\\
467	0.00593819208772081\\
468	0.00595332714098071\\
469	0.00596814504838323\\
470	0.00598270223198533\\
471	0.00599706463399752\\
472	0.00601130506551497\\
473	0.00602549828895608\\
474	0.00603968883362203\\
475	0.00605389505768314\\
476	0.00606813068370529\\
477	0.00608241069479026\\
478	0.00609675109911606\\
479	0.00611116860606077\\
480	0.00612568020806276\\
481	0.00614030266958773\\
482	0.00615505193696179\\
483	0.00616994250272472\\
484	0.00618498678894013\\
485	0.0062001946602503\\
486	0.00621557354234828\\
487	0.00623112997508471\\
488	0.00624687045373967\\
489	0.006262801363535\\
490	0.00627892891613285\\
491	0.00629525909250054\\
492	0.00631179759747023\\
493	0.00632854983207803\\
494	0.00634552089004974\\
495	0.00636271558416575\\
496	0.00638013850598619\\
497	0.00639779411750585\\
498	0.00641568685551596\\
499	0.00643382117631433\\
500	0.00645220155775853\\
501	0.00647083250442214\\
502	0.00648971855613801\\
503	0.00650886430004107\\
504	0.0065282743859806\\
505	0.00654795354486264\\
506	0.00656790660911695\\
507	0.00658813853410985\\
508	0.00660865441902504\\
509	0.00662945952567692\\
510	0.00665055929520593\\
511	0.00667195936502683\\
512	0.00669366558756179\\
513	0.00671568405080497\\
514	0.00673802110075266\\
515	0.00676068336572467\\
516	0.00678367778260441\\
517	0.00680701162504017\\
518	0.00683069253368399\\
519	0.00685472854859688\\
520	0.00687912814401729\\
521	0.00690390026575907\\
522	0.00692905437143166\\
523	0.00695460047353777\\
524	0.00698054918542238\\
525	0.00700691176997008\\
526	0.00703370019084742\\
527	0.00706092716595188\\
528	0.00708860622254237\\
529	0.00711675175328499\\
530	0.00714537907203752\\
531	0.00717450446771347\\
532	0.00720414525424005\\
533	0.00723431981403947\\
534	0.00726504763054851\\
535	0.00729634931502445\\
536	0.00732824661246005\\
537	0.00736076237665303\\
538	0.0073939205406289\\
539	0.00742774592141068\\
540	0.00746226382161746\\
541	0.007497499781614\\
542	0.00753347926021568\\
543	0.00757022206150404\\
544	0.00760775241840647\\
545	0.00764609839092532\\
546	0.00768528247896232\\
547	0.00772530972612136\\
548	0.00776618996457804\\
549	0.00780792246057549\\
550	0.00785049159263132\\
551	0.00789380041828882\\
552	0.00793786967130199\\
553	0.00798275837360501\\
554	0.00802854609916383\\
555	0.00807542371984085\\
556	0.00812345795096185\\
557	0.00817273046972689\\
558	0.00822337605108118\\
559	0.0082755419680626\\
560	0.00832801029844381\\
561	0.00837909318940157\\
562	0.00842813555899623\\
563	0.00847661969636791\\
564	0.00852511945157862\\
565	0.00857380264738345\\
566	0.00862269013955515\\
567	0.00867172225479524\\
568	0.00872080160460923\\
569	0.00876904162934258\\
570	0.00881606796714063\\
571	0.00886258479454568\\
572	0.00890894303025856\\
573	0.00895519463224402\\
574	0.00900097900874467\\
575	0.00904636842012797\\
576	0.00909180468302204\\
577	0.00913731409301758\\
578	0.00918286975492669\\
579	0.0092284285101118\\
580	0.00927394186312772\\
581	0.00931935765412942\\
582	0.00936462054353618\\
583	0.00940967226850926\\
584	0.00945445201388313\\
585	0.00949889694069117\\
586	0.00954294292450232\\
587	0.00958652555033089\\
588	0.00962958140293929\\
589	0.0096720496394373\\
590	0.0097138736812812\\
591	0.0097550024176802\\
592	0.00979538909575469\\
593	0.00983498278816331\\
594	0.00987369853931868\\
595	0.00991118387968948\\
596	0.00994658651044256\\
597	0.00997788999445116\\
598	0.010000292044645\\
599	0\\
600	0\\
};
\addplot [color=blue!25!mycolor7,solid,forget plot]
  table[row sep=crcr]{%
1	0.00409179167663971\\
2	0.00409179884488707\\
3	0.00409180611331869\\
4	0.00409181348334759\\
5	0.00409182095641101\\
6	0.00409182853397098\\
7	0.00409183621751489\\
8	0.00409184400855593\\
9	0.00409185190863384\\
10	0.00409185991931529\\
11	0.00409186804219466\\
12	0.00409187627889456\\
13	0.00409188463106638\\
14	0.00409189310039107\\
15	0.00409190168857962\\
16	0.00409191039737383\\
17	0.004091919228547\\
18	0.00409192818390445\\
19	0.00409193726528435\\
20	0.0040919464745584\\
21	0.00409195581363252\\
22	0.00409196528444763\\
23	0.00409197488898028\\
24	0.00409198462924354\\
25	0.00409199450728775\\
26	0.00409200452520124\\
27	0.0040920146851111\\
28	0.00409202498918411\\
29	0.00409203543962756\\
30	0.00409204603868992\\
31	0.00409205678866186\\
32	0.00409206769187707\\
33	0.0040920787507131\\
34	0.00409208996759235\\
35	0.00409210134498289\\
36	0.00409211288539939\\
37	0.00409212459140411\\
38	0.00409213646560782\\
39	0.00409214851067072\\
40	0.00409216072930356\\
41	0.00409217312426847\\
42	0.00409218569838013\\
43	0.00409219845450664\\
44	0.00409221139557072\\
45	0.0040922245245506\\
46	0.00409223784448126\\
47	0.00409225135845542\\
48	0.00409226506962462\\
49	0.00409227898120043\\
50	0.00409229309645551\\
51	0.00409230741872489\\
52	0.00409232195140688\\
53	0.00409233669796452\\
54	0.00409235166192667\\
55	0.00409236684688922\\
56	0.00409238225651635\\
57	0.00409239789454178\\
58	0.00409241376477004\\
59	0.00409242987107773\\
60	0.00409244621741486\\
61	0.00409246280780622\\
62	0.00409247964635263\\
63	0.00409249673723235\\
64	0.00409251408470245\\
65	0.00409253169310027\\
66	0.00409254956684473\\
67	0.00409256771043788\\
68	0.00409258612846634\\
69	0.00409260482560274\\
70	0.00409262380660726\\
71	0.00409264307632917\\
72	0.00409266263970835\\
73	0.00409268250177698\\
74	0.00409270266766095\\
75	0.0040927231425816\\
76	0.00409274393185739\\
77	0.00409276504090553\\
78	0.00409278647524373\\
79	0.00409280824049184\\
80	0.00409283034237376\\
81	0.00409285278671906\\
82	0.00409287557946492\\
83	0.00409289872665787\\
84	0.00409292223445579\\
85	0.00409294610912974\\
86	0.004092970357066\\
87	0.00409299498476787\\
88	0.00409301999885785\\
89	0.0040930454060796\\
90	0.00409307121330013\\
91	0.00409309742751178\\
92	0.0040931240558345\\
93	0.00409315110551802\\
94	0.00409317858394403\\
95	0.00409320649862865\\
96	0.00409323485722457\\
97	0.00409326366752356\\
98	0.00409329293745885\\
99	0.00409332267510763\\
100	0.00409335288869359\\
101	0.00409338358658947\\
102	0.00409341477731978\\
103	0.00409344646956346\\
104	0.00409347867215667\\
105	0.00409351139409552\\
106	0.00409354464453912\\
107	0.00409357843281241\\
108	0.00409361276840935\\
109	0.00409364766099587\\
110	0.00409368312041317\\
111	0.00409371915668098\\
112	0.00409375578000092\\
113	0.00409379300076002\\
114	0.00409383082953415\\
115	0.00409386927709188\\
116	0.00409390835439814\\
117	0.00409394807261812\\
118	0.00409398844312139\\
119	0.00409402947748602\\
120	0.00409407118750284\\
121	0.00409411358518003\\
122	0.00409415668274764\\
123	0.00409420049266249\\
124	0.00409424502761317\\
125	0.0040942903005252\\
126	0.00409433632456659\\
127	0.00409438311315353\\
128	0.00409443067995644\\
129	0.00409447903890625\\
130	0.00409452820420132\\
131	0.00409457819031449\\
132	0.00409462901200102\\
133	0.00409468068430702\\
134	0.00409473322257879\\
135	0.00409478664247332\\
136	0.00409484095997023\\
137	0.00409489619138556\\
138	0.00409495235338789\\
139	0.00409500946301823\\
140	0.00409506753771362\\
141	0.00409512659533744\\
142	0.00409518665421704\\
143	0.00409524773319237\\
144	0.00409530985167809\\
145	0.0040953730297432\\
146	0.00409543728820971\\
147	0.00409550264876417\\
148	0.00409556913404742\\
149	0.00409563676758475\\
150	0.00409570557304781\\
151	0.00409577557197914\\
152	0.00409584678632554\\
153	0.00409591923844574\\
154	0.00409599295111822\\
155	0.00409606794754915\\
156	0.00409614425138048\\
157	0.00409622188669833\\
158	0.00409630087804125\\
159	0.0040963812504089\\
160	0.00409646302927072\\
161	0.00409654624057492\\
162	0.00409663091075744\\
163	0.00409671706675124\\
164	0.00409680473599569\\
165	0.00409689394644617\\
166	0.00409698472658383\\
167	0.00409707710542542\\
168	0.00409717111253366\\
169	0.00409726677802718\\
170	0.00409736413259134\\
171	0.0040974632074888\\
172	0.00409756403457039\\
173	0.00409766664628621\\
174	0.00409777107569697\\
175	0.0040978773564854\\
176	0.00409798552296808\\
177	0.00409809561010729\\
178	0.00409820765352321\\
179	0.00409832168950623\\
180	0.00409843775502965\\
181	0.00409855588776244\\
182	0.00409867612608244\\
183	0.00409879850908959\\
184	0.00409892307661945\\
185	0.00409904986925723\\
186	0.00409917892835166\\
187	0.00409931029602955\\
188	0.0040994440152103\\
189	0.00409958012962086\\
190	0.00409971868381087\\
191	0.00409985972316823\\
192	0.00410000329393484\\
193	0.00410014944322269\\
194	0.00410029821903024\\
195	0.00410044967025921\\
196	0.00410060384673164\\
197	0.00410076079920708\\
198	0.00410092057940066\\
199	0.00410108324000084\\
200	0.00410124883468806\\
201	0.00410141741815344\\
202	0.00410158904611804\\
203	0.00410176377535236\\
204	0.00410194166369637\\
205	0.00410212277007982\\
206	0.00410230715454314\\
207	0.00410249487825852\\
208	0.00410268600355158\\
209	0.00410288059392352\\
210	0.0041030787140736\\
211	0.00410328042992225\\
212	0.00410348580863442\\
213	0.00410369491864363\\
214	0.00410390782967637\\
215	0.00410412461277718\\
216	0.00410434534033399\\
217	0.00410457008610425\\
218	0.00410479892524139\\
219	0.004105031934322\\
220	0.00410526919137341\\
221	0.004105510775902\\
222	0.00410575676892189\\
223	0.00410600725298448\\
224	0.00410626231220832\\
225	0.00410652203230979\\
226	0.00410678650063431\\
227	0.00410705580618822\\
228	0.00410733003967132\\
229	0.00410760929350995\\
230	0.00410789366189082\\
231	0.00410818324079571\\
232	0.00410847812803645\\
233	0.00410877842329085\\
234	0.00410908422813944\\
235	0.00410939564610263\\
236	0.00410971278267897\\
237	0.00411003574538369\\
238	0.00411036464378859\\
239	0.00411069958956213\\
240	0.00411104069651084\\
241	0.00411138808062097\\
242	0.00411174186010148\\
243	0.00411210215542755\\
244	0.00411246908938508\\
245	0.00411284278711603\\
246	0.00411322337616464\\
247	0.0041136109865246\\
248	0.00411400575068721\\
249	0.00411440780369023\\
250	0.0041148172831682\\
251	0.00411523432940332\\
252	0.00411565908537769\\
253	0.00411609169682644\\
254	0.00411653231229203\\
255	0.00411698108317972\\
256	0.00411743816381432\\
257	0.00411790371149797\\
258	0.00411837788656926\\
259	0.00411886085246391\\
260	0.0041193527757763\\
261	0.00411985382632294\\
262	0.0041203641772071\\
263	0.00412088400488505\\
264	0.00412141348923386\\
265	0.00412195281362084\\
266	0.00412250216497463\\
267	0.00412306173385831\\
268	0.00412363171454399\\
269	0.00412421230508999\\
270	0.00412480370741964\\
271	0.00412540612740305\\
272	0.00412601977494126\\
273	0.00412664486405411\\
274	0.00412728161297198\\
275	0.00412793024423284\\
276	0.00412859098478615\\
277	0.0041292640661058\\
278	0.00412994972431493\\
279	0.00413064820032278\\
280	0.00413135973994347\\
281	0.00413208459388185\\
282	0.00413282301783096\\
283	0.00413357527257914\\
284	0.00413434162411919\\
285	0.00413512234376086\\
286	0.00413591770824562\\
287	0.00413672799986431\\
288	0.00413755350657742\\
289	0.00413839452213839\\
290	0.00413925134621974\\
291	0.00414012428454199\\
292	0.00414101364900559\\
293	0.00414191975782595\\
294	0.00414284293567111\\
295	0.00414378351380293\\
296	0.00414474183022088\\
297	0.00414571822980925\\
298	0.00414671306448721\\
299	0.00414772669336217\\
300	0.0041487594828863\\
301	0.00414981180701613\\
302	0.00415088404737535\\
303	0.00415197659342082\\
304	0.00415308984261149\\
305	0.00415422420058033\\
306	0.00415538008130855\\
307	0.00415655790730176\\
308	0.0041577581097668\\
309	0.00415898112878767\\
310	0.00416022741349748\\
311	0.00416149742224266\\
312	0.00416279162273294\\
313	0.00416411049216903\\
314	0.00416545451733653\\
315	0.00416682419465783\\
316	0.00416822003021616\\
317	0.00416964253990388\\
318	0.004171092250146\\
319	0.0041725696983368\\
320	0.00417407543309548\\
321	0.00417561001452936\\
322	0.00417717401450484\\
323	0.0041787680169263\\
324	0.00418039261802365\\
325	0.00418204842664817\\
326	0.00418373606457772\\
327	0.00418545616683093\\
328	0.00418720938199121\\
329	0.00418899637254059\\
330	0.00419081781520407\\
331	0.00419267440130464\\
332	0.0041945668371294\\
333	0.00419649584430726\\
334	0.00419846216019875\\
335	0.00420046653829827\\
336	0.00420250974864935\\
337	0.00420459257827343\\
338	0.00420671583161246\\
339	0.00420888033098623\\
340	0.00421108691706498\\
341	0.0042133364493579\\
342	0.0042156298067184\\
343	0.00421796788786684\\
344	0.00422035161193176\\
345	0.00422278191901008\\
346	0.00422525977074702\\
347	0.00422778615093572\\
348	0.00423036206614031\\
349	0.00423298854634276\\
350	0.00423566664561411\\
351	0.00423839744281184\\
352	0.00424118204230375\\
353	0.00424402157471958\\
354	0.00424691719773253\\
355	0.00424987009687203\\
356	0.00425288148636925\\
357	0.00425595261003717\\
358	0.00425908474218753\\
359	0.0042622791885842\\
360	0.004265537287431\\
361	0.00426886041040458\\
362	0.00427224996373771\\
363	0.00427570738934931\\
364	0.00427923416602433\\
365	0.00428283181064584\\
366	0.00428650187948255\\
367	0.00429024596953427\\
368	0.00429406571993667\\
369	0.00429796281343218\\
370	0.00430193897790976\\
371	0.00430599598801906\\
372	0.0043101356668646\\
373	0.00431435988777598\\
374	0.00431867057611981\\
375	0.0043230697112264\\
376	0.00432755932840648\\
377	0.00433214152107077\\
378	0.00433681844298267\\
379	0.00434159231061924\\
380	0.00434646540565764\\
381	0.00435144007758906\\
382	0.00435651874643855\\
383	0.00436170390561014\\
384	0.00436699812487807\\
385	0.00437240405353485\\
386	0.00437792442370775\\
387	0.00438356205383751\\
388	0.00438931985234851\\
389	0.00439520082153542\\
390	0.00440120806168268\\
391	0.00440734477539559\\
392	0.00441361427201974\\
393	0.00442001997239376\\
394	0.00442656541389017\\
395	0.00443325425600352\\
396	0.00444009028617354\\
397	0.00444707742590148\\
398	0.0044542197372455\\
399	0.00446152142965565\\
400	0.00446898686715969\\
401	0.00447662057591346\\
402	0.00448442725215078\\
403	0.00449241177061241\\
404	0.00450057919329857\\
405	0.00450893477873799\\
406	0.00451748399207861\\
407	0.00452623251381179\\
408	0.00453518624866284\\
409	0.00454435133425066\\
410	0.00455373414953379\\
411	0.00456334132928628\\
412	0.00457317977488908\\
413	0.00458325666578606\\
414	0.00459357947365316\\
415	0.00460415597434657\\
416	0.00461499426294936\\
417	0.00462610277355903\\
418	0.00463749030177741\\
419	0.00464916603393684\\
420	0.00466113958139602\\
421	0.00467342101085839\\
422	0.0046860208531977\\
423	0.00469895012246663\\
424	0.00471222034629779\\
425	0.00472584364112071\\
426	0.00473983275497101\\
427	0.00475420110916848\\
428	0.00476896283486819\\
429	0.00478413279786332\\
430	0.00479972660194332\\
431	0.0048157605760717\\
432	0.00483225164319992\\
433	0.00484921720001487\\
434	0.00486667521852141\\
435	0.00488464392529949\\
436	0.0049031412572039\\
437	0.00492218498644744\\
438	0.00494179291144521\\
439	0.00496198283679481\\
440	0.00498277260286462\\
441	0.00500418019598847\\
442	0.00502622398219888\\
443	0.00504892311777252\\
444	0.00507229816531374\\
445	0.00509637169284442\\
446	0.00512117354920067\\
447	0.00514674900909566\\
448	0.00517315033241187\\
449	0.00520042753936902\\
450	0.00522863063663961\\
451	0.00525780818888461\\
452	0.00528800531821086\\
453	0.00531926094482009\\
454	0.00535160402406627\\
455	0.00538504845803528\\
456	0.00541958625039071\\
457	0.00545517834741904\\
458	0.00549174252478596\\
459	0.00552913740957662\\
460	0.00556714135810725\\
461	0.005604787298736\\
462	0.0056416798000527\\
463	0.00567770459455718\\
464	0.00571274007328253\\
465	0.00574665857487867\\
466	0.00577932847560934\\
467	0.0058106174084599\\
468	0.00584039718465546\\
469	0.0058685510370182\\
470	0.00589498376403355\\
471	0.00591963580200271\\
472	0.00594250260714197\\
473	0.00596366113022463\\
474	0.00598397845686839\\
475	0.0060038223638901\\
476	0.00602318874475772\\
477	0.00604208086658422\\
478	0.00606051065279006\\
479	0.00607849994106377\\
480	0.00609608160931803\\
481	0.00611330040582075\\
482	0.00613021324010889\\
483	0.00614688858232525\\
484	0.00616340446907811\\
485	0.00617984441050601\\
486	0.00619628223156762\\
487	0.00621274872484579\\
488	0.00622925904839289\\
489	0.00624582983842187\\
490	0.006262478959408\\
491	0.00627922515411315\\
492	0.00629608758567791\\
493	0.00631308527169584\\
494	0.00633023642363993\\
495	0.00634755772678505\\
496	0.00636506362978485\\
497	0.00638276576432228\\
498	0.00640067291806857\\
499	0.00641879282478134\\
500	0.00643713319965338\\
501	0.00645570166760095\\
502	0.00647450569462391\\
503	0.00649355252754852\\
504	0.00651284914856596\\
505	0.00653240225181333\\
506	0.00655221824944383\\
507	0.00657230331361228\\
508	0.00659266345765685\\
509	0.00661330465311988\\
510	0.00663423294085632\\
511	0.00665545447750711\\
512	0.00667697554746031\\
513	0.00669880257938798\\
514	0.00672094216772658\\
515	0.00674340109923942\\
516	0.00676618638448121\\
517	0.00678930529357986\\
518	0.00681276539528721\\
519	0.00683657459779465\\
520	0.00686074118949649\\
521	0.00688527387795266\\
522	0.00691018182879633\\
523	0.00693547470735929\\
524	0.00696116272365154\\
525	0.0069872566806296\\
526	0.00701376802560303\\
527	0.00704070890452965\\
528	0.00706809221883977\\
529	0.00709593168430487\\
530	0.00712424189132716\\
531	0.00715303836586031\\
532	0.00718233762994827\\
533	0.00721215726049923\\
534	0.00724251594426931\\
535	0.00727343352600266\\
536	0.00730493104605926\\
537	0.00733703076285003\\
538	0.00736975615367599\\
539	0.00740313188943341\\
540	0.00743718377564411\\
541	0.00747193864038859\\
542	0.00750742415076012\\
543	0.00754366859468992\\
544	0.00758070031442702\\
545	0.00761854702852742\\
546	0.00765723521324297\\
547	0.00769678967550589\\
548	0.00773722468672789\\
549	0.00777854347761477\\
550	0.00782075245913859\\
551	0.00786385640142318\\
552	0.00790785477206057\\
553	0.0079527453250826\\
554	0.00799851308237961\\
555	0.00804505598680104\\
556	0.00809242117741512\\
557	0.00814067103610812\\
558	0.00818989521339174\\
559	0.00824028601165865\\
560	0.00829193340389175\\
561	0.00834496567632439\\
562	0.00839955265381671\\
563	0.00845374671421613\\
564	0.00850651646442516\\
565	0.00855718422615864\\
566	0.00860737522808454\\
567	0.00865747862960277\\
568	0.0087075839716849\\
569	0.00875773555704555\\
570	0.00880783885424744\\
571	0.00885708033400435\\
572	0.0089050682940259\\
573	0.00895224789310436\\
574	0.00899912084127209\\
575	0.00904569281043743\\
576	0.00909165223986215\\
577	0.00913727099022529\\
578	0.00918285607127283\\
579	0.00922842280592135\\
580	0.00927393897059496\\
581	0.00931935612056587\\
582	0.00936461969699102\\
583	0.00940967179489983\\
584	0.00945445175374296\\
585	0.00949889680731318\\
586	0.00954294286465648\\
587	0.00958652552981098\\
588	0.00962958139833628\\
589	0.00967204963904421\\
590	0.0097138736812812\\
591	0.00975500241768019\\
592	0.00979538909575469\\
593	0.00983498278816331\\
594	0.00987369853931868\\
595	0.00991118387968948\\
596	0.00994658651044256\\
597	0.00997788999445116\\
598	0.010000292044645\\
599	0\\
600	0\\
};
\addplot [color=mycolor9,solid,forget plot]
  table[row sep=crcr]{%
1	0.00408032568010838\\
2	0.0040803317214586\\
3	0.00408033785212888\\
4	0.00408034407351985\\
5	0.00408035038705741\\
6	0.00408035679419308\\
7	0.00408036329640468\\
8	0.00408036989519683\\
9	0.00408037659210129\\
10	0.00408038338867769\\
11	0.00408039028651398\\
12	0.00408039728722694\\
13	0.00408040439246285\\
14	0.00408041160389798\\
15	0.00408041892323913\\
16	0.00408042635222434\\
17	0.00408043389262328\\
18	0.00408044154623809\\
19	0.00408044931490382\\
20	0.00408045720048911\\
21	0.00408046520489688\\
22	0.00408047333006484\\
23	0.00408048157796627\\
24	0.00408048995061064\\
25	0.00408049845004424\\
26	0.00408050707835092\\
27	0.00408051583765277\\
28	0.00408052473011081\\
29	0.00408053375792565\\
30	0.00408054292333841\\
31	0.00408055222863123\\
32	0.00408056167612811\\
33	0.00408057126819572\\
34	0.00408058100724404\\
35	0.00408059089572735\\
36	0.00408060093614478\\
37	0.00408061113104128\\
38	0.0040806214830084\\
39	0.00408063199468506\\
40	0.00408064266875844\\
41	0.00408065350796486\\
42	0.00408066451509057\\
43	0.00408067569297266\\
44	0.00408068704449997\\
45	0.00408069857261397\\
46	0.00408071028030962\\
47	0.00408072217063644\\
48	0.00408073424669928\\
49	0.0040807465116594\\
50	0.00408075896873532\\
51	0.00408077162120389\\
52	0.00408078447240126\\
53	0.00408079752572385\\
54	0.00408081078462939\\
55	0.00408082425263794\\
56	0.00408083793333293\\
57	0.00408085183036224\\
58	0.00408086594743926\\
59	0.00408088028834398\\
60	0.00408089485692403\\
61	0.00408090965709584\\
62	0.00408092469284576\\
63	0.00408093996823122\\
64	0.00408095548738186\\
65	0.00408097125450065\\
66	0.00408098727386516\\
67	0.00408100354982865\\
68	0.00408102008682132\\
69	0.00408103688935157\\
70	0.00408105396200713\\
71	0.00408107130945636\\
72	0.00408108893644952\\
73	0.00408110684781996\\
74	0.00408112504848546\\
75	0.00408114354344953\\
76	0.0040811623378026\\
77	0.00408118143672343\\
78	0.0040812008454803\\
79	0.0040812205694325\\
80	0.00408124061403147\\
81	0.00408126098482225\\
82	0.00408128168744471\\
83	0.00408130272763498\\
84	0.00408132411122675\\
85	0.00408134584415252\\
86	0.00408136793244503\\
87	0.00408139038223856\\
88	0.00408141319977018\\
89	0.00408143639138112\\
90	0.00408145996351798\\
91	0.00408148392273409\\
92	0.00408150827569066\\
93	0.00408153302915806\\
94	0.00408155819001696\\
95	0.00408158376525952\\
96	0.00408160976199053\\
97	0.00408163618742834\\
98	0.00408166304890601\\
99	0.00408169035387211\\
100	0.00408171810989168\\
101	0.00408174632464702\\
102	0.00408177500593827\\
103	0.0040818041616841\\
104	0.00408183379992208\\
105	0.00408186392880907\\
106	0.00408189455662128\\
107	0.00408192569175447\\
108	0.00408195734272342\\
109	0.00408198951816175\\
110	0.00408202222682114\\
111	0.00408205547757034\\
112	0.00408208927939387\\
113	0.00408212364139036\\
114	0.00408215857277041\\
115	0.00408219408285411\\
116	0.00408223018106779\\
117	0.00408226687694039\\
118	0.0040823041800988\\
119	0.00408234210026281\\
120	0.00408238064723854\\
121	0.00408241983091128\\
122	0.00408245966123659\\
123	0.00408250014823023\\
124	0.00408254130195599\\
125	0.00408258313251164\\
126	0.00408262565001214\\
127	0.00408266886457015\\
128	0.00408271278627252\\
129	0.0040827574251529\\
130	0.00408280279115896\\
131	0.00408284889411325\\
132	0.00408289574366644\\
133	0.00408294334924135\\
134	0.00408299171996512\\
135	0.00408304086458768\\
136	0.00408309079138236\\
137	0.00408314150802506\\
138	0.00408319302144679\\
139	0.00408324533765281\\
140	0.00408329846150164\\
141	0.00408335239643531\\
142	0.00408340714415597\\
143	0.00408346270425434\\
144	0.00408351907383572\\
145	0.00408357624731534\\
146	0.00408363421695109\\
147	0.00408369297595483\\
148	0.00408375253033698\\
149	0.00408381294198024\\
150	0.00408387442250217\\
151	0.00408393699150774\\
152	0.0040840006689656\\
153	0.00408406547521487\\
154	0.004084131430972\\
155	0.0040841985573378\\
156	0.00408426687580454\\
157	0.00408433640826325\\
158	0.00408440717701115\\
159	0.0040844792047591\\
160	0.00408455251463939\\
161	0.00408462713021347\\
162	0.00408470307548001\\
163	0.00408478037488287\\
164	0.00408485905331948\\
165	0.00408493913614922\\
166	0.00408502064920193\\
167	0.00408510361878671\\
168	0.00408518807170073\\
169	0.00408527403523841\\
170	0.00408536153720045\\
171	0.00408545060590334\\
172	0.00408554127018892\\
173	0.00408563355943405\\
174	0.00408572750356054\\
175	0.00408582313304534\\
176	0.0040859204789307\\
177	0.00408601957283474\\
178	0.00408612044696201\\
179	0.00408622313411457\\
180	0.0040863276677028\\
181	0.00408643408175689\\
182	0.00408654241093819\\
183	0.00408665269055101\\
184	0.00408676495655458\\
185	0.00408687924557497\\
186	0.00408699559491781\\
187	0.00408711404258058\\
188	0.00408723462726558\\
189	0.00408735738839309\\
190	0.00408748236611461\\
191	0.00408760960132647\\
192	0.00408773913568356\\
193	0.00408787101161365\\
194	0.00408800527233164\\
195	0.00408814196185414\\
196	0.00408828112501439\\
197	0.00408842280747765\\
198	0.00408856705575633\\
199	0.00408871391722615\\
200	0.0040888634401419\\
201	0.00408901567365398\\
202	0.00408917066782506\\
203	0.00408932847364704\\
204	0.00408948914305838\\
205	0.00408965272896175\\
206	0.00408981928524199\\
207	0.00408998886678448\\
208	0.0040901615294938\\
209	0.00409033733031263\\
210	0.00409051632724121\\
211	0.00409069857935697\\
212	0.00409088414683478\\
213	0.00409107309096724\\
214	0.00409126547418565\\
215	0.00409146136008117\\
216	0.00409166081342647\\
217	0.0040918639001977\\
218	0.00409207068759704\\
219	0.0040922812440754\\
220	0.00409249563935562\\
221	0.00409271394445633\\
222	0.00409293623171591\\
223	0.00409316257481705\\
224	0.00409339304881169\\
225	0.00409362773014645\\
226	0.00409386669668852\\
227	0.0040941100277519\\
228	0.0040943578041243\\
229	0.00409461010809421\\
230	0.00409486702347883\\
231	0.00409512863565205\\
232	0.00409539503157325\\
233	0.00409566629981644\\
234	0.00409594253059986\\
235	0.00409622381581618\\
236	0.00409651024906302\\
237	0.00409680192567429\\
238	0.00409709894275164\\
239	0.00409740139919666\\
240	0.00409770939574349\\
241	0.00409802303499203\\
242	0.00409834242144152\\
243	0.00409866766152481\\
244	0.00409899886364285\\
245	0.00409933613819994\\
246	0.00409967959763939\\
247	0.00410002935647955\\
248	0.00410038553135043\\
249	0.00410074824103093\\
250	0.00410111760648614\\
251	0.00410149375090549\\
252	0.00410187679974104\\
253	0.00410226688074624\\
254	0.00410266412401525\\
255	0.00410306866202235\\
256	0.00410348062966169\\
257	0.00410390016428735\\
258	0.00410432740575361\\
259	0.0041047624964549\\
260	0.00410520558136627\\
261	0.00410565680808287\\
262	0.00410611632685941\\
263	0.0041065842906483\\
264	0.00410706085513649\\
265	0.00410754617878015\\
266	0.00410804042283614\\
267	0.00410854375138903\\
268	0.00410905633137184\\
269	0.00410957833257757\\
270	0.00411010992765883\\
271	0.00411065129211007\\
272	0.00411120260422655\\
273	0.00411176404503058\\
274	0.00411233579815451\\
275	0.00411291804966501\\
276	0.00411351098781273\\
277	0.0041141148026981\\
278	0.00411472968589211\\
279	0.00411535583028356\\
280	0.00411599343159979\\
281	0.00411664269946865\\
282	0.0041173038480168\\
283	0.00411797709534109\\
284	0.00411866266359079\\
285	0.00411936077905226\\
286	0.0041200716722368\\
287	0.00412079557797152\\
288	0.00412153273549352\\
289	0.0041222833885479\\
290	0.00412304778548947\\
291	0.00412382617938882\\
292	0.00412461882814302\\
293	0.00412542599459136\\
294	0.004126247946637\\
295	0.00412708495737457\\
296	0.00412793730522538\\
297	0.0041288052740808\\
298	0.00412968915345553\\
299	0.00413058923865261\\
300	0.0041315058309427\\
301	0.00413243923776124\\
302	0.00413338977292813\\
303	0.00413435775689608\\
304	0.00413534351703653\\
305	0.00413634738797513\\
306	0.00413736971199214\\
307	0.00413841083951022\\
308	0.00413947112969897\\
309	0.00414055095123541\\
310	0.00414165068327278\\
311	0.00414277071668319\\
312	0.00414391145564883\\
313	0.00414507331965988\\
314	0.0041462567458689\\
315	0.00414746219133093\\
316	0.00414869013318215\\
317	0.00414994105951313\\
318	0.00415121542405593\\
319	0.00415251367288903\\
320	0.00415383626087322\\
321	0.00415518365184186\\
322	0.00415655631879631\\
323	0.00415795474410643\\
324	0.00415937941971667\\
325	0.00416083084735821\\
326	0.00416230953876658\\
327	0.0041638160159058\\
328	0.0041653508111988\\
329	0.00416691446776466\\
330	0.00416850753966274\\
331	0.00417013059214414\\
332	0.00417178420191086\\
333	0.00417346895738277\\
334	0.00417518545897301\\
335	0.00417693431937199\\
336	0.0041787161638408\\
337	0.00418053163051362\\
338	0.00418238137071067\\
339	0.0041842660492613\\
340	0.00418618634483813\\
341	0.00418814295030285\\
342	0.00419013657306392\\
343	0.00419216793544708\\
344	0.00419423777507914\\
345	0.00419634684528539\\
346	0.00419849591550199\\
347	0.00420068577170344\\
348	0.0042029172168468\\
349	0.00420519107133266\\
350	0.00420750817348505\\
351	0.00420986938005095\\
352	0.00421227556672175\\
353	0.00421472762867913\\
354	0.00421722648116924\\
355	0.00421977306011006\\
356	0.00422236832273927\\
357	0.00422501324831298\\
358	0.00422770883886822\\
359	0.00423045612005951\\
360	0.00423325614202024\\
361	0.00423610997994662\\
362	0.00423901873440536\\
363	0.00424198353210752\\
364	0.00424500552671295\\
365	0.00424808589966584\\
366	0.00425122586106436\\
367	0.00425442665056831\\
368	0.00425768953834222\\
369	0.00426101582603336\\
370	0.0042644068477877\\
371	0.00426786397130313\\
372	0.004271388598917\\
373	0.00427498216872125\\
374	0.00427864615570287\\
375	0.00428238207289374\\
376	0.00428619147251039\\
377	0.00429007594705438\\
378	0.00429403713033155\\
379	0.00429807669835758\\
380	0.00430219637020581\\
381	0.00430639790936694\\
382	0.00431068312732309\\
383	0.00431505388634855\\
384	0.00431951210153059\\
385	0.00432405974289698\\
386	0.0043286988376544\\
387	0.00433343147254496\\
388	0.00433825979632698\\
389	0.00434318602238383\\
390	0.00434821243146024\\
391	0.00435334137452092\\
392	0.00435857527575316\\
393	0.00436391663571122\\
394	0.00436936803461629\\
395	0.00437493213578097\\
396	0.00438061168915676\\
397	0.0043864095350035\\
398	0.00439232860766343\\
399	0.00439837193942105\\
400	0.00440454266442583\\
401	0.00441084402264989\\
402	0.0044172793638488\\
403	0.00442385215147291\\
404	0.00443056596649915\\
405	0.00443742451113101\\
406	0.00444443161216516\\
407	0.00445159122412042\\
408	0.00445890743212007\\
409	0.00446638445468958\\
410	0.00447402664703967\\
411	0.00448183850433959\\
412	0.004489824665813\\
413	0.00449798992021728\\
414	0.00450633921284684\\
415	0.00451487766131778\\
416	0.00452361057509217\\
417	0.00453254346895203\\
418	0.00454168207405567\\
419	0.00455103234957194\\
420	0.00456060049389352\\
421	0.00457039295449526\\
422	0.00458041643914397\\
423	0.00459067792893476\\
424	0.00460118469484066\\
425	0.00461194431180844\\
426	0.00462296467320176\\
427	0.00463425400557573\\
428	0.00464582088389187\\
429	0.00465767424757893\\
430	0.00466982341915739\\
431	0.00468227812477799\\
432	0.0046950485232572\\
433	0.00470814523954862\\
434	0.00472157936550173\\
435	0.00473536245780382\\
436	0.00474950659521184\\
437	0.0047640244387787\\
438	0.00477892927470935\\
439	0.00479423506031151\\
440	0.00480995647212761\\
441	0.00482610895435044\\
442	0.00484270876391218\\
443	0.00485977300663505\\
444	0.00487731967260649\\
445	0.00489536790457525\\
446	0.00491393775780624\\
447	0.00493304950806499\\
448	0.00495272349413835\\
449	0.00497298037922986\\
450	0.00499384109334846\\
451	0.00501532678326504\\
452	0.00503745878020474\\
453	0.00506025859515393\\
454	0.0050837479230485\\
455	0.00510794836847699\\
456	0.0051328848498848\\
457	0.00515858542537479\\
458	0.00518507829680298\\
459	0.0052123916363027\\
460	0.00524055774871919\\
461	0.00526962519665851\\
462	0.0052996483229499\\
463	0.00533068136551119\\
464	0.00536277687710101\\
465	0.00539598350622364\\
466	0.00543034292173908\\
467	0.00546588564894863\\
468	0.00550262551694845\\
469	0.00554055226288177\\
470	0.00557962158440338\\
471	0.00561974192124584\\
472	0.00566075695097854\\
473	0.00570242247316536\\
474	0.00574371723299744\\
475	0.00578413666240843\\
476	0.0058235514348476\\
477	0.00586182452730329\\
478	0.00589881302973534\\
479	0.00593437122793735\\
480	0.00596835508548402\\
481	0.0060006288329282\\
482	0.00603107436035096\\
483	0.00605960441782868\\
484	0.00608618090950803\\
485	0.0061108399790797\\
486	0.00613394827269195\\
487	0.0061565572231749\\
488	0.00617866106963987\\
489	0.00620026206261332\\
490	0.00622137190575537\\
491	0.00624201318766271\\
492	0.00626222068822813\\
493	0.00628204238213333\\
494	0.00630153987527973\\
495	0.00632078789161669\\
496	0.00633987226080195\\
497	0.00635888563866321\\
498	0.00637791389053429\\
499	0.00639699321982514\\
500	0.00641614092023951\\
501	0.00643537595941995\\
502	0.00645471868676816\\
503	0.00647419042522651\\
504	0.00649381293908639\\
505	0.00651360777983297\\
506	0.00653359552902565\\
507	0.00655379498480331\\
508	0.00657422238156807\\
509	0.00659489079711179\\
510	0.00661581066893441\\
511	0.00663699164239693\\
512	0.00665844341738469\\
513	0.00668017567478128\\
514	0.00670219800864469\\
515	0.00672451987082082\\
516	0.00674715053602441\\
517	0.00677009909628874\\
518	0.0067933744936005\\
519	0.00681698559766519\\
520	0.00684094133084513\\
521	0.00686525083245867\\
522	0.00688992357169755\\
523	0.006914969393402\\
524	0.00694039855405188\\
525	0.00696622176460758\\
526	0.00699245024049316\\
527	0.00701909575863972\\
528	0.00704617072100075\\
529	0.00707368822332517\\
530	0.00710166212726932\\
531	0.00713010713325534\\
532	0.00715903885104884\\
533	0.00718847386631718\\
534	0.00721842980511202\\
535	0.00724892539692609\\
536	0.00727998053398537\\
537	0.00731161632367419\\
538	0.00734385513004147\\
539	0.00737672059906469\\
540	0.00741023766083625\\
541	0.00744443250022344\\
542	0.00747933248540618\\
543	0.00751496604033652\\
544	0.00755136244984742\\
545	0.00758855157424061\\
546	0.00762656343832719\\
547	0.00766542764552213\\
548	0.00770517274565856\\
549	0.00774582545318689\\
550	0.00778739842297575\\
551	0.00782990245782332\\
552	0.00787334632801186\\
553	0.00791773109971343\\
554	0.00796305422576511\\
555	0.00800931980616587\\
556	0.00805652647189953\\
557	0.0081046653681536\\
558	0.00815370949778185\\
559	0.00820357552247344\\
560	0.00825431025459932\\
561	0.00830597583091507\\
562	0.00835865057764933\\
563	0.00841255528549486\\
564	0.00846782554576636\\
565	0.00852461550995907\\
566	0.00858084970818208\\
567	0.00863568580905793\\
568	0.00868850380505972\\
569	0.00874038008329295\\
570	0.00879204607992877\\
571	0.00884343635142392\\
572	0.00889467572782135\\
573	0.00894516958355532\\
574	0.00899434902706773\\
575	0.00904222600225824\\
576	0.00908965295883001\\
577	0.00913649196733126\\
578	0.00918262170103313\\
579	0.00922833831212848\\
580	0.00927390407257964\\
581	0.00931933850652799\\
582	0.00936461041460092\\
583	0.00940966667844402\\
584	0.00945444888056296\\
585	0.00949889521326668\\
586	0.00954294203465611\\
587	0.00958652514879544\\
588	0.00962958126464585\\
589	0.00967204960829027\\
590	0.00971387367861981\\
591	0.0097550024176802\\
592	0.00979538909575469\\
593	0.00983498278816331\\
594	0.00987369853931868\\
595	0.00991118387968948\\
596	0.00994658651044256\\
597	0.00997788999445116\\
598	0.010000292044645\\
599	0\\
600	0\\
};
\addplot [color=blue!50!mycolor7,solid,forget plot]
  table[row sep=crcr]{%
1	0.00402964426831132\\
2	0.00402965034379505\\
3	0.00402965651562647\\
4	0.00402966278541615\\
5	0.0040296691548033\\
6	0.00402967562545648\\
7	0.00402968219907398\\
8	0.0040296888773844\\
9	0.00402969566214725\\
10	0.00402970255515347\\
11	0.00402970955822588\\
12	0.00402971667321998\\
13	0.00402972390202436\\
14	0.00402973124656131\\
15	0.00402973870878748\\
16	0.00402974629069442\\
17	0.00402975399430933\\
18	0.00402976182169553\\
19	0.00402976977495321\\
20	0.00402977785622004\\
21	0.00402978606767185\\
22	0.00402979441152328\\
23	0.00402980289002851\\
24	0.00402981150548188\\
25	0.00402982026021867\\
26	0.00402982915661576\\
27	0.00402983819709248\\
28	0.00402984738411119\\
29	0.00402985672017811\\
30	0.00402986620784412\\
31	0.00402987584970546\\
32	0.00402988564840465\\
33	0.0040298956066311\\
34	0.00402990572712215\\
35	0.0040299160126637\\
36	0.00402992646609121\\
37	0.00402993709029046\\
38	0.00402994788819844\\
39	0.00402995886280432\\
40	0.00402997001715026\\
41	0.0040299813543323\\
42	0.00402999287750143\\
43	0.00403000458986439\\
44	0.00403001649468467\\
45	0.00403002859528357\\
46	0.00403004089504111\\
47	0.00403005339739706\\
48	0.00403006610585196\\
49	0.00403007902396819\\
50	0.00403009215537106\\
51	0.00403010550374975\\
52	0.00403011907285858\\
53	0.00403013286651801\\
54	0.00403014688861583\\
55	0.00403016114310827\\
56	0.00403017563402121\\
57	0.00403019036545137\\
58	0.00403020534156747\\
59	0.00403022056661152\\
60	0.00403023604490003\\
61	0.00403025178082535\\
62	0.00403026777885687\\
63	0.00403028404354242\\
64	0.00403030057950953\\
65	0.00403031739146684\\
66	0.0040303344842055\\
67	0.00403035186260053\\
68	0.00403036953161226\\
69	0.00403038749628784\\
70	0.00403040576176262\\
71	0.00403042433326177\\
72	0.00403044321610175\\
73	0.00403046241569182\\
74	0.00403048193753568\\
75	0.0040305017872331\\
76	0.00403052197048147\\
77	0.0040305424930776\\
78	0.00403056336091931\\
79	0.00403058458000714\\
80	0.00403060615644616\\
81	0.00403062809644772\\
82	0.00403065040633129\\
83	0.0040306730925263\\
84	0.00403069616157396\\
85	0.00403071962012924\\
86	0.00403074347496271\\
87	0.00403076773296263\\
88	0.00403079240113689\\
89	0.00403081748661507\\
90	0.00403084299665048\\
91	0.00403086893862231\\
92	0.00403089532003773\\
93	0.00403092214853418\\
94	0.00403094943188149\\
95	0.00403097717798414\\
96	0.00403100539488353\\
97	0.00403103409076045\\
98	0.00403106327393735\\
99	0.00403109295288078\\
100	0.00403112313620386\\
101	0.00403115383266873\\
102	0.00403118505118914\\
103	0.00403121680083304\\
104	0.00403124909082514\\
105	0.00403128193054971\\
106	0.00403131532955316\\
107	0.00403134929754687\\
108	0.00403138384440994\\
109	0.00403141898019218\\
110	0.00403145471511682\\
111	0.00403149105958357\\
112	0.0040315280241716\\
113	0.00403156561964252\\
114	0.00403160385694349\\
115	0.0040316427472103\\
116	0.00403168230177054\\
117	0.00403172253214682\\
118	0.00403176345005995\\
119	0.00403180506743214\\
120	0.00403184739639037\\
121	0.00403189044926957\\
122	0.00403193423861598\\
123	0.00403197877719031\\
124	0.00403202407797098\\
125	0.00403207015415745\\
126	0.00403211701917317\\
127	0.00403216468666858\\
128	0.00403221317052421\\
129	0.00403226248485323\\
130	0.00403231264400393\\
131	0.00403236366256223\\
132	0.00403241555535348\\
133	0.00403246833744424\\
134	0.00403252202414366\\
135	0.00403257663100445\\
136	0.00403263217382357\\
137	0.00403268866864269\\
138	0.00403274613174887\\
139	0.00403280457967544\\
140	0.00403286402920493\\
141	0.00403292449737621\\
142	0.00403298600150241\\
143	0.00403304855921572\\
144	0.00403311218858156\\
145	0.00403317690839188\\
146	0.00403324273891722\\
147	0.00403330970375947\\
148	0.0040333778337388\\
149	0.004033447168939\\
150	0.00403351773105495\\
151	0.00403358954217357\\
152	0.00403366262478084\\
153	0.00403373700176913\\
154	0.0040338126964444\\
155	0.00403388973253367\\
156	0.0040339681341927\\
157	0.00403404792601353\\
158	0.00403412913303239\\
159	0.00403421178073775\\
160	0.00403429589507838\\
161	0.0040343815024716\\
162	0.00403446862981177\\
163	0.0040345573044788\\
164	0.00403464755434689\\
165	0.00403473940779341\\
166	0.0040348328937079\\
167	0.0040349280415013\\
168	0.00403502488111525\\
169	0.00403512344303169\\
170	0.00403522375828242\\
171	0.0040353258584591\\
172	0.0040354297757232\\
173	0.00403553554281623\\
174	0.00403564319307008\\
175	0.00403575276041768\\
176	0.00403586427940372\\
177	0.00403597778519553\\
178	0.00403609331359445\\
179	0.00403621090104681\\
180	0.00403633058465587\\
181	0.00403645240219323\\
182	0.00403657639211106\\
183	0.00403670259355399\\
184	0.0040368310463717\\
185	0.00403696179113145\\
186	0.00403709486913082\\
187	0.00403723032241092\\
188	0.00403736819376953\\
189	0.00403750852677466\\
190	0.00403765136577822\\
191	0.00403779675593019\\
192	0.00403794474319268\\
193	0.00403809537435453\\
194	0.00403824869704588\\
195	0.00403840475975344\\
196	0.00403856361183554\\
197	0.00403872530353779\\
198	0.00403888988600878\\
199	0.00403905741131628\\
200	0.00403922793246357\\
201	0.00403940150340611\\
202	0.00403957817906847\\
203	0.00403975801536169\\
204	0.00403994106920072\\
205	0.00404012739852243\\
206	0.00404031706230372\\
207	0.00404051012058006\\
208	0.00404070663446437\\
209	0.00404090666616626\\
210	0.00404111027901159\\
211	0.00404131753746227\\
212	0.00404152850713655\\
213	0.00404174325482977\\
214	0.00404196184853525\\
215	0.00404218435746575\\
216	0.00404241085207537\\
217	0.00404264140408154\\
218	0.0040428760864879\\
219	0.00404311497360726\\
220	0.00404335814108511\\
221	0.00404360566592364\\
222	0.00404385762650622\\
223	0.00404411410262221\\
224	0.00404437517549263\\
225	0.00404464092779601\\
226	0.00404491144369495\\
227	0.00404518680886314\\
228	0.00404546711051317\\
229	0.00404575243742481\\
230	0.00404604287997398\\
231	0.00404633853016232\\
232	0.00404663948164759\\
233	0.00404694582977477\\
234	0.00404725767160795\\
235	0.00404757510596315\\
236	0.00404789823344184\\
237	0.00404822715646565\\
238	0.00404856197931195\\
239	0.00404890280815061\\
240	0.00404924975108192\\
241	0.00404960291817575\\
242	0.0040499624215123\\
243	0.00405032837522407\\
244	0.00405070089553996\\
245	0.00405108010083074\\
246	0.00405146611165699\\
247	0.00405185905081918\\
248	0.00405225904341034\\
249	0.00405266621687171\\
250	0.00405308070105178\\
251	0.00405350262826892\\
252	0.00405393213337873\\
253	0.00405436935384623\\
254	0.004054814429824\\
255	0.00405526750423721\\
256	0.00405572872287694\\
257	0.00405619823450304\\
258	0.00405667619095848\\
259	0.00405716274729766\\
260	0.00405765806193113\\
261	0.00405816229679059\\
262	0.00405867561751827\\
263	0.00405919819368625\\
264	0.00405973019905237\\
265	0.00406027181186149\\
266	0.0040608232152023\\
267	0.00406138459743367\\
268	0.00406195615269679\\
269	0.0040625380815348\\
270	0.00406313059164622\\
271	0.00406373389880394\\
272	0.00406434822797799\\
273	0.00406497381469557\\
274	0.00406561090663889\\
275	0.00406625976532493\\
276	0.00406692066708692\\
277	0.0040675939000439\\
278	0.00406827974262886\\
279	0.00406897835323118\\
280	0.00406968914749888\\
281	0.00407041227445104\\
282	0.00407114793169407\\
283	0.0040718963188144\\
284	0.00407265763728611\\
285	0.0040734320903632\\
286	0.00407421988295385\\
287	0.00407502122147461\\
288	0.00407583631368072\\
289	0.00407666536846943\\
290	0.00407750859565094\\
291	0.00407836620568241\\
292	0.00407923840935802\\
293	0.00408012541744706\\
294	0.00408102744027068\\
295	0.00408194468720526\\
296	0.00408287736609797\\
297	0.00408382568257669\\
298	0.0040847898392321\\
299	0.00408577003464479\\
300	0.00408676646222342\\
301	0.00408777930881153\\
302	0.00408880875300992\\
303	0.00408985496314912\\
304	0.0040909180948289\\
305	0.00409199828792148\\
306	0.00409309566291238\\
307	0.00409421031642586\\
308	0.00409534231576585\\
309	0.00409649169232576\\
310	0.00409765843387609\\
311	0.00409884247631765\\
312	0.00410004369736373\\
313	0.00410126192051685\\
314	0.00410249695646168\\
315	0.0041037487710781\\
316	0.00410501809269451\\
317	0.00410630872083941\\
318	0.00410762228403794\\
319	0.00410895917827685\\
320	0.00411031980594528\\
321	0.00411170457592415\\
322	0.00411311390367631\\
323	0.0041145482113375\\
324	0.00411600792780806\\
325	0.00411749348884517\\
326	0.00411900533715594\\
327	0.00412054392249108\\
328	0.00412210970173916\\
329	0.00412370313902142\\
330	0.00412532470578721\\
331	0.00412697488090991\\
332	0.00412865415078315\\
333	0.00413036300941769\\
334	0.00413210195853834\\
335	0.00413387150768126\\
336	0.00413567217429118\\
337	0.00413750448381872\\
338	0.00413936896981707\\
339	0.00414126617403825\\
340	0.00414319664652806\\
341	0.004145160945719\\
342	0.00414715963852069\\
343	0.0041491933004056\\
344	0.00415126251548892\\
345	0.0041533678765995\\
346	0.00415550998533789\\
347	0.00415768945211674\\
348	0.00415990689617547\\
349	0.00416216294555975\\
350	0.00416445823705093\\
351	0.00416679341602682\\
352	0.00416916913622703\\
353	0.00417158605938683\\
354	0.00417404485469147\\
355	0.00417654619798789\\
356	0.00417909077067757\\
357	0.00418167925822084\\
358	0.00418431234829119\\
359	0.00418699072918654\\
360	0.00418971509198084\\
361	0.00419248615450633\\
362	0.00419530468493859\\
363	0.00419817146485468\\
364	0.00420108728974278\\
365	0.00420405296957705\\
366	0.00420706932946481\\
367	0.00421013721038834\\
368	0.00421325747015759\\
369	0.00421643098468567\\
370	0.00421965864940848\\
371	0.00422294138104568\\
372	0.00422628011982449\\
373	0.00422967583232177\\
374	0.00423312951512707\\
375	0.00423664219958411\\
376	0.00424021495790487\\
377	0.00424384891089465\\
378	0.00424754523712741\\
379	0.00425130518182945\\
380	0.00425513005820619\\
381	0.00425902121396172\\
382	0.0042629798657054\\
383	0.0042670071798537\\
384	0.00427110434913889\\
385	0.00427527259430024\\
386	0.00427951316601193\\
387	0.00428382734708664\\
388	0.00428821645499868\\
389	0.00429268184477737\\
390	0.00429722491232846\\
391	0.00430184709825241\\
392	0.00430654989223478\\
393	0.00431133483809401\\
394	0.00431620353957894\\
395	0.00432115766701866\\
396	0.00432619896493554\\
397	0.00433132926073578\\
398	0.00433655047459088\\
399	0.00434186463061658\\
400	0.00434727386943476\\
401	0.00435278046216777\\
402	0.00435838682584821\\
403	0.00436409554012939\\
404	0.00436990936502242\\
405	0.0043758312591468\\
406	0.00438186439766819\\
407	0.00438801218860048\\
408	0.00439427828525927\\
409	0.00440066659110231\\
410	0.0044071812515267\\
411	0.00441382662958582\\
412	0.00442060725501341\\
413	0.00442752773369903\\
414	0.00443459259727504\\
415	0.00444180604812211\\
416	0.00444917192835531\\
417	0.00445669404834207\\
418	0.00446437634750615\\
419	0.00447222290003713\\
420	0.00448023792066377\\
421	0.00448842577068774\\
422	0.00449679096427237\\
423	0.00450533817498008\\
424	0.00451407224216589\\
425	0.00452299817835337\\
426	0.00453212118052739\\
427	0.0045414466429674\\
428	0.00455098016755669\\
429	0.00456072757490198\\
430	0.00457069491622738\\
431	0.00458088848628549\\
432	0.00459131483644244\\
433	0.00460198078562155\\
434	0.00461289343244344\\
435	0.00462406017275493\\
436	0.00463548871752189\\
437	0.00464718710969808\\
438	0.00465916374184674\\
439	0.00467142737449136\\
440	0.00468398715518755\\
441	0.00469685263836591\\
442	0.00471003380649355\\
443	0.00472354109554875\\
444	0.00473738543486821\\
445	0.00475157824689911\\
446	0.00476613143619902\\
447	0.00478105743071025\\
448	0.00479636924656813\\
449	0.00481208052910316\\
450	0.00482820559756616\\
451	0.00484475949353333\\
452	0.00486175803227568\\
453	0.00487921785586886\\
454	0.00489715649787599\\
455	0.00491559266873435\\
456	0.00493454612292579\\
457	0.0049540373357115\\
458	0.00497408748771998\\
459	0.00499471871224096\\
460	0.00501595407012546\\
461	0.00503781727833482\\
462	0.00506033258423081\\
463	0.00508352468522577\\
464	0.00510741842122384\\
465	0.00513204033536679\\
466	0.00515741948366903\\
467	0.00518358660865367\\
468	0.00521056816764443\\
469	0.00523839073483765\\
470	0.00526708158535866\\
471	0.00529666956048661\\
472	0.00532718623681849\\
473	0.00535866769970322\\
474	0.00539116649826347\\
475	0.00542474181628269\\
476	0.00545945169037871\\
477	0.00549535093760036\\
478	0.00553248836020859\\
479	0.00557090284134059\\
480	0.00561061805473222\\
481	0.00565163534862653\\
482	0.00569392423879731\\
483	0.00573740976427773\\
484	0.00578195573058062\\
485	0.00582734256557108\\
486	0.00587302073186874\\
487	0.00591777929369159\\
488	0.00596147832584702\\
489	0.00600396924585436\\
490	0.00604509677476079\\
491	0.00608470203551141\\
492	0.00612262728640956\\
493	0.00615872287684319\\
494	0.0061928572469614\\
495	0.00622493098413339\\
496	0.00625489643300038\\
497	0.0062827844896127\\
498	0.0063089053613691\\
499	0.00633450440142987\\
500	0.00635957631827112\\
501	0.00638412492653279\\
502	0.00640816477749661\\
503	0.00643172275774873\\
504	0.00645483951784793\\
505	0.0064775705149099\\
506	0.00649998635097495\\
507	0.00652217194101431\\
508	0.00654422384304134\\
509	0.0065662448150298\\
510	0.00658831692692719\\
511	0.00661047543979157\\
512	0.00663274081961346\\
513	0.00665513543500258\\
514	0.00667768320198779\\
515	0.0067004090917962\\
516	0.00672333849461296\\
517	0.00674649644599174\\
518	0.00676990674623638\\
519	0.00679359104100118\\
520	0.00681756798909042\\
521	0.00684185273053656\\
522	0.00686645862083711\\
523	0.00689139898441964\\
524	0.00691668742247002\\
525	0.00694233775045284\\
526	0.0069683639461627\\
527	0.00699478011683277\\
528	0.00702160049519244\\
529	0.00704883947499101\\
530	0.0070765116956076\\
531	0.007104632181729\\
532	0.00713321653579037\\
533	0.00716228113862316\\
534	0.00719184326231583\\
535	0.00722192112766865\\
536	0.00725253396467986\\
537	0.00728370207453166\\
538	0.00731544689055641\\
539	0.00734779103435149\\
540	0.00738075836152634\\
541	0.00741437398947174\\
542	0.00744866429705612\\
543	0.0074836568833871\\
544	0.00751938046982297\\
545	0.00755586473097468\\
546	0.0075931400381968\\
547	0.00763123709119732\\
548	0.00767018640119527\\
549	0.00771001757935446\\
550	0.00775075870022047\\
551	0.00779242478641457\\
552	0.00783502989694319\\
553	0.00787858722644379\\
554	0.00792310886900782\\
555	0.00796860537723829\\
556	0.00801508542259554\\
557	0.00806255540390485\\
558	0.00811101877401344\\
559	0.00816046466610628\\
560	0.00821089105072984\\
561	0.00826228888233602\\
562	0.00831463976241251\\
563	0.00836783969989453\\
564	0.00842192152680974\\
565	0.0084769504865854\\
566	0.00853302421756685\\
567	0.0085903217031176\\
568	0.00864905932111027\\
569	0.00870766413426631\\
570	0.00876495225119506\\
571	0.00882046728958851\\
572	0.00887398586474071\\
573	0.00892717754532439\\
574	0.00897988555572072\\
575	0.00903197458790182\\
576	0.00908265151027188\\
577	0.00913183644158302\\
578	0.00917988739287592\\
579	0.00922711699783626\\
580	0.00927343806398176\\
581	0.00931913102044634\\
582	0.0093645049608711\\
583	0.00940961130043772\\
584	0.0094544183622864\\
585	0.00949887800402646\\
586	0.00954293238643295\\
587	0.00958652004546675\\
588	0.00962957886512476\\
589	0.00967204874598804\\
590	0.00971387347495566\\
591	0.00975500239981678\\
592	0.00979538909575469\\
593	0.00983498278816331\\
594	0.00987369853931868\\
595	0.00991118387968948\\
596	0.00994658651044256\\
597	0.00997788999445116\\
598	0.010000292044645\\
599	0\\
600	0\\
};
\addplot [color=blue!40!mycolor9,solid,forget plot]
  table[row sep=crcr]{%
1	0.00381944488791505\\
2	0.00381945382964767\\
3	0.00381946292077991\\
4	0.00381947216387073\\
5	0.00381948156152383\\
6	0.00381949111638838\\
7	0.00381950083115981\\
8	0.00381951070858063\\
9	0.00381952075144128\\
10	0.00381953096258094\\
11	0.00381954134488834\\
12	0.00381955190130264\\
13	0.00381956263481437\\
14	0.00381957354846626\\
15	0.00381958464535422\\
16	0.00381959592862812\\
17	0.00381960740149288\\
18	0.00381961906720937\\
19	0.00381963092909535\\
20	0.00381964299052653\\
21	0.00381965525493748\\
22	0.00381966772582277\\
23	0.00381968040673791\\
24	0.00381969330130045\\
25	0.0038197064131911\\
26	0.00381971974615478\\
27	0.00381973330400171\\
28	0.0038197470906086\\
29	0.0038197611099198\\
30	0.00381977536594841\\
31	0.0038197898627776\\
32	0.00381980460456168\\
33	0.00381981959552742\\
34	0.00381983483997533\\
35	0.00381985034228096\\
36	0.00381986610689601\\
37	0.00381988213834992\\
38	0.00381989844125108\\
39	0.00381991502028821\\
40	0.00381993188023176\\
41	0.0038199490259354\\
42	0.00381996646233736\\
43	0.00381998419446203\\
44	0.00382000222742135\\
45	0.00382002056641638\\
46	0.0038200392167389\\
47	0.00382005818377291\\
48	0.00382007747299627\\
49	0.0038200970899824\\
50	0.00382011704040181\\
51	0.00382013733002396\\
52	0.00382015796471891\\
53	0.00382017895045909\\
54	0.00382020029332103\\
55	0.00382022199948732\\
56	0.00382024407524837\\
57	0.00382026652700434\\
58	0.00382028936126701\\
59	0.00382031258466178\\
60	0.00382033620392975\\
61	0.00382036022592958\\
62	0.00382038465763972\\
63	0.00382040950616044\\
64	0.00382043477871595\\
65	0.00382046048265671\\
66	0.0038204866254615\\
67	0.00382051321473978\\
68	0.00382054025823403\\
69	0.00382056776382201\\
70	0.00382059573951924\\
71	0.00382062419348138\\
72	0.00382065313400674\\
73	0.00382068256953883\\
74	0.00382071250866895\\
75	0.00382074296013872\\
76	0.00382077393284292\\
77	0.00382080543583205\\
78	0.00382083747831521\\
79	0.00382087006966293\\
80	0.00382090321940996\\
81	0.00382093693725835\\
82	0.00382097123308031\\
83	0.00382100611692133\\
84	0.00382104159900323\\
85	0.00382107768972738\\
86	0.00382111439967794\\
87	0.00382115173962502\\
88	0.00382118972052815\\
89	0.00382122835353958\\
90	0.00382126765000783\\
91	0.00382130762148123\\
92	0.00382134827971144\\
93	0.00382138963665715\\
94	0.0038214317044878\\
95	0.00382147449558743\\
96	0.00382151802255859\\
97	0.00382156229822609\\
98	0.00382160733564117\\
99	0.00382165314808561\\
100	0.00382169974907573\\
101	0.00382174715236684\\
102	0.00382179537195737\\
103	0.00382184442209338\\
104	0.00382189431727298\\
105	0.00382194507225083\\
106	0.00382199670204285\\
107	0.00382204922193089\\
108	0.00382210264746757\\
109	0.00382215699448107\\
110	0.00382221227908013\\
111	0.00382226851765912\\
112	0.00382232572690309\\
113	0.00382238392379307\\
114	0.00382244312561133\\
115	0.00382250334994673\\
116	0.00382256461470026\\
117	0.00382262693809057\\
118	0.00382269033865958\\
119	0.00382275483527822\\
120	0.00382282044715228\\
121	0.00382288719382837\\
122	0.00382295509519975\\
123	0.00382302417151253\\
124	0.00382309444337185\\
125	0.00382316593174796\\
126	0.0038232386579827\\
127	0.00382331264379583\\
128	0.00382338791129142\\
129	0.00382346448296455\\
130	0.0038235423817078\\
131	0.00382362163081799\\
132	0.00382370225400282\\
133	0.00382378427538783\\
134	0.00382386771952312\\
135	0.00382395261139026\\
136	0.00382403897640941\\
137	0.00382412684044632\\
138	0.00382421622981948\\
139	0.00382430717130773\\
140	0.00382439969215842\\
141	0.00382449382009734\\
142	0.00382458958334283\\
143	0.00382468701063115\\
144	0.00382478613126695\\
145	0.00382488697523013\\
146	0.00382498957338099\\
147	0.00382509395772975\\
148	0.00382520016121877\\
149	0.00382530821570657\\
150	0.00382541815360922\\
151	0.00382553000791004\\
152	0.00382564381216948\\
153	0.00382575960053517\\
154	0.00382587740775211\\
155	0.003825997269173\\
156	0.00382611922076889\\
157	0.00382624329913982\\
158	0.00382636954152591\\
159	0.00382649798581826\\
160	0.00382662867057043\\
161	0.00382676163500982\\
162	0.00382689691904938\\
163	0.00382703456329954\\
164	0.00382717460908022\\
165	0.00382731709843319\\
166	0.0038274620741345\\
167	0.00382760957970729\\
168	0.00382775965943458\\
169	0.00382791235837252\\
170	0.00382806772236377\\
171	0.00382822579805092\\
172	0.00382838663289046\\
173	0.00382855027516681\\
174	0.00382871677400655\\
175	0.00382888617939297\\
176	0.0038290585421808\\
177	0.00382923391411128\\
178	0.00382941234782737\\
179	0.00382959389688925\\
180	0.00382977861579013\\
181	0.00382996655997224\\
182	0.00383015778584317\\
183	0.00383035235079235\\
184	0.00383055031320795\\
185	0.00383075173249401\\
186	0.00383095666908772\\
187	0.00383116518447723\\
188	0.00383137734121947\\
189	0.00383159320295851\\
190	0.00383181283444411\\
191	0.00383203630155047\\
192	0.00383226367129549\\
193	0.00383249501186013\\
194	0.00383273039260829\\
195	0.00383296988410682\\
196	0.00383321355814593\\
197	0.00383346148775993\\
198	0.00383371374724835\\
199	0.00383397041219723\\
200	0.00383423155950099\\
201	0.00383449726738433\\
202	0.00383476761542474\\
203	0.00383504268457528\\
204	0.00383532255718768\\
205	0.00383560731703575\\
206	0.00383589704933934\\
207	0.00383619184078845\\
208	0.00383649177956781\\
209	0.00383679695538181\\
210	0.00383710745947986\\
211	0.00383742338468209\\
212	0.00383774482540548\\
213	0.00383807187769018\\
214	0.00383840463922657\\
215	0.00383874320938237\\
216	0.00383908768923041\\
217	0.00383943818157668\\
218	0.00383979479098878\\
219	0.00384015762382479\\
220	0.00384052678826269\\
221	0.00384090239432995\\
222	0.00384128455393374\\
223	0.00384167338089152\\
224	0.00384206899096197\\
225	0.00384247150187654\\
226	0.00384288103337115\\
227	0.00384329770721868\\
228	0.00384372164726155\\
229	0.00384415297944502\\
230	0.00384459183185069\\
231	0.00384503833473076\\
232	0.00384549262054244\\
233	0.00384595482398295\\
234	0.003846425082025\\
235	0.0038469035339526\\
236	0.00384739032139756\\
237	0.00384788558837617\\
238	0.00384838948132651\\
239	0.00384890214914628\\
240	0.00384942374323097\\
241	0.0038499544175126\\
242	0.00385049432849878\\
243	0.00385104363531254\\
244	0.00385160249973235\\
245	0.00385217108623283\\
246	0.00385274956202587\\
247	0.00385333809710236\\
248	0.00385393686427437\\
249	0.00385454603921786\\
250	0.00385516580051607\\
251	0.00385579632970351\\
252	0.00385643781131035\\
253	0.00385709043290793\\
254	0.00385775438515451\\
255	0.00385842986184218\\
256	0.00385911705994454\\
257	0.00385981617966518\\
258	0.00386052742448745\\
259	0.00386125100122527\\
260	0.00386198712007523\\
261	0.00386273599467023\\
262	0.00386349784213462\\
263	0.00386427288314106\\
264	0.00386506134196932\\
265	0.00386586344656686\\
266	0.00386667942861144\\
267	0.00386750952357506\\
268	0.00386835397078955\\
269	0.0038692130135119\\
270	0.00387008689898778\\
271	0.00387097587850831\\
272	0.00387188020744911\\
273	0.00387280014526215\\
274	0.00387373595533601\\
275	0.00387468790447146\\
276	0.0038756562611982\\
277	0.00387664129061343\\
278	0.00387764323968271\\
279	0.00387866230829795\\
280	0.0038796987741781\\
281	0.00388075292388058\\
282	0.00388182504825039\\
283	0.00388291544246592\\
284	0.00388402440608404\\
285	0.00388515224308433\\
286	0.0038862992619126\\
287	0.0038874657755231\\
288	0.0038886521014196\\
289	0.00388985856169444\\
290	0.00389108548306613\\
291	0.00389233319691414\\
292	0.00389360203931105\\
293	0.00389489235105145\\
294	0.00389620447767661\\
295	0.00389753876949465\\
296	0.0038988955815951\\
297	0.00390027527385714\\
298	0.0039016782109505\\
299	0.00390310476232804\\
300	0.00390455530220914\\
301	0.00390603020955305\\
302	0.00390752986802195\\
303	0.00390905466593376\\
304	0.00391060499620635\\
305	0.00391218125629692\\
306	0.0039137838481452\\
307	0.00391541317813866\\
308	0.00391706965714266\\
309	0.0039187537006957\\
310	0.00392046572962547\\
311	0.00392220617174694\\
312	0.00392397546638137\\
313	0.00392577407619996\\
314	0.00392760251734056\\
315	0.00392946142885158\\
316	0.00393135167357656\\
317	0.00393327387039218\\
318	0.00393522852295292\\
319	0.00393721614132281\\
320	0.00393923724199673\\
321	0.00394129234791969\\
322	0.00394338198850431\\
323	0.00394550669964648\\
324	0.00394766702373917\\
325	0.00394986350968491\\
326	0.00395209671290709\\
327	0.00395436719536063\\
328	0.00395667552554228\\
329	0.00395902227850162\\
330	0.00396140803585341\\
331	0.00396383338579263\\
332	0.0039662989231134\\
333	0.00396880524923413\\
334	0.00397135297223093\\
335	0.00397394270688251\\
336	0.00397657507473022\\
337	0.00397925070415858\\
338	0.00398197023050168\\
339	0.00398473429618398\\
340	0.00398754355090492\\
341	0.00399039865187992\\
342	0.00399330026415337\\
343	0.00399624906100389\\
344	0.00399924572446666\\
345	0.00400229094600521\\
346	0.00400538542737336\\
347	0.00400852988171844\\
348	0.0040117250349921\\
349	0.00401497162775132\\
350	0.00401827041745474\\
351	0.00402162218138518\\
352	0.00402502772035511\\
353	0.00402848786336176\\
354	0.00403200347329119\\
355	0.00403557545339288\\
356	0.00403920475267724\\
357	0.00404289236201626\\
358	0.00404663926532477\\
359	0.00405044617696116\\
360	0.00405431209755343\\
361	0.00405823454676541\\
362	0.00406221390273528\\
363	0.00406625050416226\\
364	0.00407034464456845\\
365	0.00407449656600271\\
366	0.00407870645169283\\
367	0.00408297441481805\\
368	0.00408730047692172\\
369	0.00409168456088001\\
370	0.00409612647725224\\
371	0.00410062590479673\\
372	0.00410518236700039\\
373	0.00410979520407372\\
374	0.00411446354037165\\
375	0.00411918624930451\\
376	0.00412396192469406\\
377	0.00412878888933131\\
378	0.00413366534096513\\
379	0.00413858996594277\\
380	0.00414356417617088\\
381	0.00414860064463897\\
382	0.00415370519126221\\
383	0.00415887782354044\\
384	0.00416411846490067\\
385	0.00416942694647098\\
386	0.00417480299821334\\
387	0.00418024623940122\\
388	0.00418575616844292\\
389	0.00419133215207308\\
390	0.00419697341396355\\
391	0.00420267902284577\\
392	0.00420844788029291\\
393	0.00421427870838504\\
394	0.00422017003758266\\
395	0.00422612019527058\\
396	0.00423212729561477\\
397	0.00423818923161663\\
398	0.00424430367056677\\
399	0.00425046805452244\\
400	0.00425667960798663\\
401	0.00426293535569741\\
402	0.00426923215439836\\
403	0.00427556674372872\\
404	0.00428193582305656\\
405	0.00428833616336053\\
406	0.00429476476652256\\
407	0.00430121908978586\\
408	0.00430769736567081\\
409	0.00431419908071125\\
410	0.00432072567623176\\
411	0.00432728118476319\\
412	0.00433387326282209\\
413	0.00434051456486937\\
414	0.00434722456676876\\
415	0.00435403197492565\\
416	0.00436096030699787\\
417	0.0043680178596261\\
418	0.00437520729992339\\
419	0.00438253139700803\\
420	0.00438999303583359\\
421	0.00439759523439973\\
422	0.0044053411642879\\
423	0.00441323417190158\\
424	0.00442127778875179\\
425	0.00442947568622172\\
426	0.00443783142315854\\
427	0.00444634839223422\\
428	0.00445503009046111\\
429	0.00446388012457871\\
430	0.00447290221682272\\
431	0.00448210021101293\\
432	0.0044914780788647\\
433	0.00450103992690116\\
434	0.00451079000428179\\
435	0.00452073271115161\\
436	0.00453087260748145\\
437	0.00454121442262168\\
438	0.00455176306564887\\
439	0.00456252363660099\\
440	0.00457350143872782\\
441	0.00458470199196287\\
442	0.00459613104799112\\
443	0.00460779460714568\\
444	0.00461969893332353\\
445	0.00463185057046742\\
446	0.00464425636526101\\
447	0.00465692349307205\\
448	0.00466985948416286\\
449	0.00468307225273333\\
450	0.00469657012804901\\
451	0.00471036188452164\\
452	0.0047244567769889\\
453	0.00473886458248146\\
454	0.00475359565575727\\
455	0.0047686609505671\\
456	0.00478407203480342\\
457	0.00479984113684248\\
458	0.00481598120812612\\
459	0.00483250597144564\\
460	0.00484943004417186\\
461	0.00486676903423521\\
462	0.00488453962375556\\
463	0.00490275965605901\\
464	0.00492144834571443\\
465	0.00494062633330873\\
466	0.0049603155092367\\
467	0.00498053854270406\\
468	0.00500131922287815\\
469	0.005022682272008\\
470	0.00504465297027654\\
471	0.00506725646762278\\
472	0.00509051803742607\\
473	0.00511446380997644\\
474	0.00513912196386034\\
475	0.00516452336427147\\
476	0.00519070172399126\\
477	0.00521768674120825\\
478	0.00524550790757517\\
479	0.00527419459831302\\
480	0.00530377649605227\\
481	0.00533428372592657\\
482	0.00536574715474855\\
483	0.00539819893647581\\
484	0.00543167340888997\\
485	0.00546620850085091\\
486	0.00550185097014559\\
487	0.00553866543163327\\
488	0.00557671540533525\\
489	0.00561606113875089\\
490	0.00565675655137328\\
491	0.00569884505698706\\
492	0.00574235390404411\\
493	0.00578728657121308\\
494	0.00583361260813407\\
495	0.00588125412523459\\
496	0.00593006789083969\\
497	0.00597982167539138\\
498	0.00603000261139891\\
499	0.00607919001400096\\
500	0.0061272311955095\\
501	0.00617396403990961\\
502	0.00621921947190392\\
503	0.00626282523322078\\
504	0.00630461149639069\\
505	0.00634441913381876\\
506	0.00638211144982614\\
507	0.00641759075241061\\
508	0.00645082139107356\\
509	0.00648186133141168\\
510	0.00651137009306334\\
511	0.00654034185867726\\
512	0.00656877382473748\\
513	0.00659667381570486\\
514	0.00662406212808724\\
515	0.00665097331469975\\
516	0.00667745772009153\\
517	0.00670358249291407\\
518	0.00672943165361346\\
519	0.00675510460981876\\
520	0.00678071227161439\\
521	0.00680636956803016\\
522	0.0068321418354832\\
523	0.00685805883381921\\
524	0.00688414616269528\\
525	0.00691043166666499\\
526	0.0069369449982943\\
527	0.00696371702157117\\
528	0.006990779052795\\
529	0.00701816195599068\\
530	0.00704589514320308\\
531	0.00707400558272338\\
532	0.007102517000605\\
533	0.00713145022215655\\
534	0.00716082564304089\\
535	0.00719066452125459\\
536	0.00722098892233578\\
537	0.00725182166615787\\
538	0.00728318628224142\\
539	0.00731510698194115\\
540	0.00734760865663381\\
541	0.00738071691038189\\
542	0.00741445813238253\\
543	0.0074488596067575\\
544	0.0074839496420057\\
545	0.00751975758741009\\
546	0.00755631369103536\\
547	0.00759364885472858\\
548	0.00763179428830673\\
549	0.00767078103297187\\
550	0.00771063930619072\\
551	0.00775139804237961\\
552	0.00779307363174633\\
553	0.00783568124693249\\
554	0.00787923540543176\\
555	0.00792374989807655\\
556	0.00796923763533189\\
557	0.00801571046336702\\
558	0.00806317895231008\\
559	0.00811165226865771\\
560	0.00816113770274232\\
561	0.00821164035063354\\
562	0.00826316276313242\\
563	0.00831570535723852\\
564	0.0083692583124851\\
565	0.00842380278768651\\
566	0.00847932001064811\\
567	0.00853574542202786\\
568	0.00859302797587814\\
569	0.00865122838399295\\
570	0.00871044661767393\\
571	0.00877081641557857\\
572	0.00883206177374144\\
573	0.00889210788623348\\
574	0.00895060193483743\\
575	0.0090069743480842\\
576	0.00906164998966982\\
577	0.00911569254926328\\
578	0.00916848694084165\\
579	0.00921948532352991\\
580	0.00926878734421818\\
581	0.00931668587461442\\
582	0.00936337646786582\\
583	0.00940899867258259\\
584	0.00945409509992045\\
585	0.00949869896494848\\
586	0.00954283083080037\\
587	0.00958646243837779\\
588	0.00962954789259008\\
589	0.0096720338085118\\
590	0.00971386797226699\\
591	0.00975500106378656\\
592	0.00979538897701793\\
593	0.00983498278816331\\
594	0.00987369853931868\\
595	0.00991118387968948\\
596	0.00994658651044256\\
597	0.00997788999445116\\
598	0.010000292044645\\
599	0\\
600	0\\
};
\addplot [color=blue!75!mycolor7,solid,forget plot]
  table[row sep=crcr]{%
1	0.00303384705345048\\
2	0.003033865512802\\
3	0.00303388428662202\\
4	0.00303390338031515\\
5	0.00303392279937922\\
6	0.00303394254940679\\
7	0.00303396263608701\\
8	0.00303398306520715\\
9	0.00303400384265426\\
10	0.0030340249744169\\
11	0.00303404646658698\\
12	0.00303406832536142\\
13	0.00303409055704408\\
14	0.00303411316804746\\
15	0.00303413616489464\\
16	0.0030341595542212\\
17	0.00303418334277715\\
18	0.00303420753742885\\
19	0.00303423214516106\\
20	0.00303425717307896\\
21	0.00303428262841024\\
22	0.0030343085185072\\
23	0.00303433485084882\\
24	0.00303436163304314\\
25	0.00303438887282922\\
26	0.00303441657807957\\
27	0.00303444475680241\\
28	0.00303447341714398\\
29	0.00303450256739096\\
30	0.00303453221597281\\
31	0.00303456237146433\\
32	0.00303459304258804\\
33	0.00303462423821688\\
34	0.00303465596737664\\
35	0.00303468823924865\\
36	0.00303472106317255\\
37	0.00303475444864891\\
38	0.00303478840534193\\
39	0.0030348229430825\\
40	0.00303485807187084\\
41	0.00303489380187951\\
42	0.0030349301434564\\
43	0.00303496710712772\\
44	0.00303500470360105\\
45	0.00303504294376846\\
46	0.0030350818387098\\
47	0.00303512139969574\\
48	0.00303516163819128\\
49	0.00303520256585894\\
50	0.00303524419456222\\
51	0.00303528653636906\\
52	0.00303532960355534\\
53	0.00303537340860847\\
54	0.00303541796423115\\
55	0.00303546328334487\\
56	0.00303550937909384\\
57	0.00303555626484877\\
58	0.00303560395421081\\
59	0.00303565246101549\\
60	0.00303570179933677\\
61	0.00303575198349118\\
62	0.00303580302804193\\
63	0.00303585494780319\\
64	0.00303590775784455\\
65	0.00303596147349518\\
66	0.00303601611034853\\
67	0.00303607168426674\\
68	0.00303612821138534\\
69	0.00303618570811794\\
70	0.00303624419116103\\
71	0.00303630367749889\\
72	0.00303636418440846\\
73	0.00303642572946448\\
74	0.00303648833054459\\
75	0.00303655200583452\\
76	0.00303661677383344\\
77	0.00303668265335929\\
78	0.00303674966355437\\
79	0.00303681782389086\\
80	0.0030368871541765\\
81	0.00303695767456038\\
82	0.00303702940553884\\
83	0.00303710236796138\\
84	0.00303717658303682\\
85	0.0030372520723394\\
86	0.00303732885781506\\
87	0.00303740696178794\\
88	0.00303748640696674\\
89	0.00303756721645135\\
90	0.00303764941373972\\
91	0.00303773302273442\\
92	0.00303781806774981\\
93	0.00303790457351899\\
94	0.00303799256520105\\
95	0.00303808206838826\\
96	0.00303817310911356\\
97	0.00303826571385812\\
98	0.00303835990955895\\
99	0.00303845572361669\\
100	0.00303855318390362\\
101	0.0030386523187716\\
102	0.00303875315706035\\
103	0.00303885572810569\\
104	0.00303896006174804\\
105	0.00303906618834104\\
106	0.00303917413876022\\
107	0.00303928394441195\\
108	0.0030393956372424\\
109	0.00303950924974678\\
110	0.00303962481497866\\
111	0.00303974236655932\\
112	0.00303986193868759\\
113	0.00303998356614951\\
114	0.0030401072843283\\
115	0.00304023312921447\\
116	0.00304036113741615\\
117	0.00304049134616944\\
118	0.00304062379334912\\
119	0.00304075851747948\\
120	0.00304089555774519\\
121	0.00304103495400245\\
122	0.00304117674679044\\
123	0.00304132097734277\\
124	0.00304146768759921\\
125	0.00304161692021761\\
126	0.00304176871858596\\
127	0.0030419231268348\\
128	0.00304208018984966\\
129	0.00304223995328376\\
130	0.00304240246357105\\
131	0.00304256776793924\\
132	0.00304273591442327\\
133	0.0030429069518788\\
134	0.00304308092999609\\
135	0.00304325789931412\\
136	0.00304343791123465\\
137	0.00304362101803702\\
138	0.00304380727289281\\
139	0.00304399672988108\\
140	0.00304418944400408\\
141	0.00304438547120398\\
142	0.00304458486838166\\
143	0.00304478769341881\\
144	0.00304499400520721\\
145	0.00304520386368493\\
146	0.00304541732986664\\
147	0.00304563446581277\\
148	0.00304585533449473\\
149	0.00304607999995983\\
150	0.00304630852734943\\
151	0.00304654098291754\\
152	0.00304677743404971\\
153	0.00304701794928214\\
154	0.00304726259832124\\
155	0.00304751145206357\\
156	0.00304776458261577\\
157	0.00304802206331531\\
158	0.00304828396875115\\
159	0.00304855037478505\\
160	0.00304882135857308\\
161	0.00304909699858753\\
162	0.00304937737463915\\
163	0.00304966256789988\\
164	0.0030499526609258\\
165	0.00305024773768053\\
166	0.00305054788355909\\
167	0.00305085318541202\\
168	0.00305116373157007\\
169	0.00305147961186905\\
170	0.00305180091767536\\
171	0.00305212774191184\\
172	0.00305246017908392\\
173	0.00305279832530643\\
174	0.00305314227833069\\
175	0.00305349213757211\\
176	0.00305384800413824\\
177	0.00305420998085731\\
178	0.00305457817230707\\
179	0.00305495268484445\\
180	0.00305533362663533\\
181	0.00305572110768509\\
182	0.00305611523986942\\
183	0.00305651613696585\\
184	0.00305692391468572\\
185	0.00305733869070659\\
186	0.0030577605847053\\
187	0.00305818971839144\\
188	0.00305862621554155\\
189	0.00305907020203372\\
190	0.00305952180588283\\
191	0.00305998115727626\\
192	0.00306044838861037\\
193	0.00306092363452737\\
194	0.00306140703195296\\
195	0.00306189872013442\\
196	0.00306239884067943\\
197	0.00306290753759554\\
198	0.00306342495733007\\
199	0.00306395124881094\\
200	0.00306448656348788\\
201	0.00306503105537458\\
202	0.00306558488109133\\
203	0.00306614819990824\\
204	0.00306672117378955\\
205	0.00306730396743808\\
206	0.00306789674834098\\
207	0.00306849968681576\\
208	0.00306911295605723\\
209	0.00306973673218526\\
210	0.00307037119429307\\
211	0.00307101652449648\\
212	0.00307167290798388\\
213	0.0030723405330669\\
214	0.00307301959123206\\
215	0.00307371027719301\\
216	0.00307441278894365\\
217	0.00307512732781234\\
218	0.00307585409851649\\
219	0.00307659330921831\\
220	0.00307734517158135\\
221	0.00307810990082795\\
222	0.00307888771579747\\
223	0.00307967883900548\\
224	0.00308048349670392\\
225	0.00308130191894195\\
226	0.00308213433962811\\
227	0.00308298099659295\\
228	0.00308384213165301\\
229	0.0030847179906756\\
230	0.00308560882364454\\
231	0.00308651488472681\\
232	0.00308743643234043\\
233	0.00308837372922319\\
234	0.0030893270425023\\
235	0.00309029664376532\\
236	0.00309128280913192\\
237	0.00309228581932678\\
238	0.0030933059597536\\
239	0.00309434352057004\\
240	0.00309539879676386\\
241	0.00309647208823016\\
242	0.00309756369984966\\
243	0.00309867394156809\\
244	0.00309980312847685\\
245	0.00310095158089455\\
246	0.00310211962444996\\
247	0.00310330759016596\\
248	0.00310451581454468\\
249	0.00310574463965382\\
250	0.00310699441321429\\
251	0.00310826548868878\\
252	0.00310955822537173\\
253	0.00311087298848048\\
254	0.00311221014924765\\
255	0.00311357008501479\\
256	0.00311495317932709\\
257	0.00311635982202967\\
258	0.00311779040936487\\
259	0.00311924534407096\\
260	0.00312072503548212\\
261	0.0031222298996297\\
262	0.00312376035934481\\
263	0.00312531684436224\\
264	0.00312689979142575\\
265	0.00312850964439462\\
266	0.00313014685435141\\
267	0.00313181187971142\\
268	0.00313350518633285\\
269	0.00313522724762836\\
270	0.00313697854467666\\
271	0.00313875956633266\\
272	0.00314057080933009\\
273	0.00314241277836093\\
274	0.00314428598608756\\
275	0.00314619095297797\\
276	0.00314812820673057\\
277	0.00315009828105087\\
278	0.00315210171510535\\
279	0.00315413906353755\\
280	0.00315621089020438\\
281	0.00315831776787394\\
282	0.00316046027834203\\
283	0.00316263901254875\\
284	0.003164854570695\\
285	0.00316710756235895\\
286	0.00316939860661227\\
287	0.00317172833213598\\
288	0.00317409737733571\\
289	0.00317650639045645\\
290	0.00317895602969642\\
291	0.00318144696331995\\
292	0.00318397986976924\\
293	0.00318655543777469\\
294	0.0031891743664637\\
295	0.00319183736546763\\
296	0.00319454515502686\\
297	0.00319729846609331\\
298	0.00320009804043076\\
299	0.00320294463071217\\
300	0.00320583900061387\\
301	0.00320878192490665\\
302	0.0032117741895431\\
303	0.0032148165917414\\
304	0.00321790994006526\\
305	0.00322105505450091\\
306	0.00322425276653286\\
307	0.00322750391922457\\
308	0.0032308093673196\\
309	0.00323416997740608\\
310	0.00323758662825071\\
311	0.00324106021155859\\
312	0.00324459163371413\\
313	0.00324818181944748\\
314	0.00325183171778768\\
315	0.00325554230340428\\
316	0.00325931453812834\\
317	0.00326314938512363\\
318	0.00326704781990338\\
319	0.00327101083037703\\
320	0.00327503941689374\\
321	0.00327913459228231\\
322	0.00328329738188779\\
323	0.0032875288236051\\
324	0.00329182996790991\\
325	0.00329620187788686\\
326	0.0033006456292558\\
327	0.00330516231039655\\
328	0.00330975302237266\\
329	0.00331441887895505\\
330	0.00331916100664649\\
331	0.00332398054470761\\
332	0.00332887864518655\\
333	0.00333385647295279\\
334	0.00333891520573777\\
335	0.00334405603418376\\
336	0.00334928016190405\\
337	0.00335458880555669\\
338	0.00335998319493566\\
339	0.00336546457308269\\
340	0.00337103419642487\\
341	0.00337669333494218\\
342	0.0033824432723713\\
343	0.0033882853064516\\
344	0.00339422074922057\\
345	0.00340025092736631\\
346	0.00340637718264471\\
347	0.00341260087237047\\
348	0.00341892336998915\\
349	0.00342534606573737\\
350	0.00343187036739461\\
351	0.00343849770112212\\
352	0.00344522951236527\\
353	0.00345206726674163\\
354	0.00345901245067994\\
355	0.00346606657109623\\
356	0.00347323115190361\\
357	0.00348050772061386\\
358	0.00348789776607916\\
359	0.00349540263597865\\
360	0.00350302364655946\\
361	0.00351076243536027\\
362	0.00351862066977116\\
363	0.00352660004706458\\
364	0.00353470229387294\\
365	0.00354292916578727\\
366	0.00355128244940566\\
367	0.00355976397110082\\
368	0.00356837560093155\\
369	0.00357711908392077\\
370	0.00358599606644035\\
371	0.00359500817362262\\
372	0.00360415701181463\\
373	0.00361344416473827\\
374	0.00362287119070602\\
375	0.00363243962329641\\
376	0.0036421509819296\\
377	0.00365200680929468\\
378	0.00366200877770539\\
379	0.00367215894944063\\
380	0.00368246019276606\\
381	0.00369291448887022\\
382	0.00370352316447297\\
383	0.00371428743342848\\
384	0.0037252083752065\\
385	0.00373628690975085\\
386	0.00374752376805794\\
387	0.00375891945768785\\
388	0.00377047422226314\\
389	0.00378218799381759\\
390	0.00379406033662271\\
391	0.00380609038083258\\
392	0.00381827674393536\\
393	0.00383061743756721\\
394	0.00384310975671144\\
395	0.00385575014764735\\
396	0.00386853405019943\\
397	0.00388145570882703\\
398	0.00389450794583856\\
399	0.00390768188844423\\
400	0.00392096663938711\\
401	0.0039343488783687\\
402	0.00394781237813493\\
403	0.00396133741420826\\
404	0.00397490003859048\\
405	0.00398847116685206\\
406	0.00400201535859076\\
407	0.00401548887315432\\
408	0.00402883494268167\\
409	0.00404197666253574\\
410	0.00405483387971556\\
411	0.00406730920686715\\
412	0.00407928411461078\\
413	0.00409061413591738\\
414	0.00410112328338029\\
415	0.00411059866972351\\
416	0.00411976746527252\\
417	0.00412906916830723\\
418	0.00413850301769023\\
419	0.00414806780402478\\
420	0.0041577618081022\\
421	0.00416758279534187\\
422	0.00417752822733678\\
423	0.00418759622241075\\
424	0.004197789117264\\
425	0.0042081269899841\\
426	0.00421863205451258\\
427	0.00422930607143691\\
428	0.00424015075867423\\
429	0.004251167785805\\
430	0.00426235876789\\
431	0.00427372525871681\\
432	0.00428526874345597\\
433	0.00429699063068474\\
434	0.00430889224367596\\
435	0.00432097481087908\\
436	0.00433323945553472\\
437	0.00434568718434684\\
438	0.0043583188751327\\
439	0.00437113526337202\\
440	0.00438413692758075\\
441	0.00439732427343946\\
442	0.00441069751655755\\
443	0.00442425666353447\\
444	0.00443800149163091\\
445	0.00445193152736049\\
446	0.00446604602374304\\
447	0.00448034393626135\\
448	0.00449482389843095\\
449	0.00450948420924725\\
450	0.00452432288789258\\
451	0.00453933797452697\\
452	0.00455452740746064\\
453	0.00456988901507084\\
454	0.00458542062548439\\
455	0.00460112023002508\\
456	0.00461698622005927\\
457	0.0046330177100885\\
458	0.00464921493486472\\
459	0.00466557939521225\\
460	0.00468210867217519\\
461	0.00469879965952423\\
462	0.00471565019985468\\
463	0.00473265971222406\\
464	0.00474983003071213\\
465	0.00476716651370559\\
466	0.00478467953169976\\
467	0.00480238653623335\\
468	0.00482031477230763\\
469	0.00483850488659161\\
470	0.00485701572111196\\
471	0.004875929161333\\
472	0.00489529644610638\\
473	0.00491513298305584\\
474	0.00493545509421521\\
475	0.00495627994266942\\
476	0.00497762505705277\\
477	0.00499950880073145\\
478	0.0050219504705099\\
479	0.0050449704075158\\
480	0.00506859007336904\\
481	0.00509283211145518\\
482	0.0051177203005794\\
483	0.00514327950696849\\
484	0.00516953856622172\\
485	0.00519652882533619\\
486	0.00522428493328236\\
487	0.00525283976962182\\
488	0.00528222471205487\\
489	0.00531247156248563\\
490	0.00534361250795315\\
491	0.00537568038786467\\
492	0.00540870878097448\\
493	0.00544273219379265\\
494	0.00547778641436578\\
495	0.00551390909868753\\
496	0.0055511406830368\\
497	0.00558952584142584\\
498	0.00562911789773015\\
499	0.00566999128690472\\
500	0.00571221932201546\\
501	0.0057558715749304\\
502	0.00580101005582034\\
503	0.00584768388787208\\
504	0.00589592198047116\\
505	0.00594572303841862\\
506	0.00599704395404753\\
507	0.00604978581485652\\
508	0.00610377318904819\\
509	0.00615872670430141\\
510	0.00621377305779257\\
511	0.00626773342543394\\
512	0.00632044037470168\\
513	0.00637171706810509\\
514	0.00642138012600038\\
515	0.00646924415730192\\
516	0.00651512881061301\\
517	0.00655886878984116\\
518	0.00660032857703474\\
519	0.00663942309655355\\
520	0.00667614607675877\\
521	0.00671060900481495\\
522	0.00674415544607651\\
523	0.00677716675605977\\
524	0.00680964580317178\\
525	0.00684160813228221\\
526	0.00687308399823199\\
527	0.00690412025177207\\
528	0.00693478183651414\\
529	0.00696515253551898\\
530	0.00699533442800454\\
531	0.00702544528207324\\
532	0.00705561271722277\\
533	0.00708594791661066\\
534	0.00711650232276352\\
535	0.00714730727861487\\
536	0.00717839739551637\\
537	0.00720981020187562\\
538	0.00724158560986581\\
539	0.00727376518449118\\
540	0.00730639121578814\\
541	0.00733950562402815\\
542	0.00737314876893419\\
543	0.00740735831282291\\
544	0.00744216840464061\\
545	0.00747761164752185\\
546	0.00751372125221221\\
547	0.00755053132123895\\
548	0.00758807649293622\\
549	0.00762639148986282\\
550	0.0076655105538372\\
551	0.00770546673078169\\
552	0.00774629139584217\\
553	0.00778800482332902\\
554	0.00783062355809369\\
555	0.00787416316338176\\
556	0.00791863849542332\\
557	0.00796406368646087\\
558	0.00801045200277901\\
559	0.00805781568956767\\
560	0.00810616580577066\\
561	0.00815551204477823\\
562	0.00820586253849474\\
563	0.00825722363207021\\
564	0.00830959973629561\\
565	0.00836299312329599\\
566	0.00841740357329269\\
567	0.00847282849714414\\
568	0.0085292617624085\\
569	0.00858669084759215\\
570	0.00864508965557358\\
571	0.00870442086865101\\
572	0.00876459910238825\\
573	0.00882562379752669\\
574	0.00888756930913724\\
575	0.00895054169680455\\
576	0.00901356110634767\\
577	0.00907514900313189\\
578	0.00913502099859305\\
579	0.00919263117907495\\
580	0.0092479758283961\\
581	0.00930184486036383\\
582	0.00935368351991403\\
583	0.00940333326402388\\
584	0.00945082356531764\\
585	0.00949690757289053\\
586	0.00954180842209612\\
587	0.00958587549421746\\
588	0.00962920975611778\\
589	0.0096718487149668\\
590	0.00971377621437142\\
591	0.00975496638267559\\
592	0.00979538031248056\\
593	0.00983498200924467\\
594	0.00987369853931868\\
595	0.00991118387968948\\
596	0.00994658651044256\\
597	0.00997788999445116\\
598	0.010000292044645\\
599	0\\
600	0\\
};
\addplot [color=blue!80!mycolor9,solid,forget plot]
  table[row sep=crcr]{%
1	0.00247811649021119\\
2	0.00247812555694198\\
3	0.00247813477847231\\
4	0.00247814415746359\\
5	0.00247815369662314\\
6	0.00247816339870495\\
7	0.00247817326651042\\
8	0.00247818330288924\\
9	0.00247819351074017\\
10	0.00247820389301195\\
11	0.00247821445270408\\
12	0.00247822519286778\\
13	0.00247823611660675\\
14	0.00247824722707825\\
15	0.00247825852749384\\
16	0.00247827002112047\\
17	0.00247828171128131\\
18	0.00247829360135681\\
19	0.00247830569478558\\
20	0.00247831799506553\\
21	0.00247833050575474\\
22	0.00247834323047263\\
23	0.00247835617290096\\
24	0.00247836933678482\\
25	0.00247838272593385\\
26	0.00247839634422328\\
27	0.00247841019559507\\
28	0.00247842428405909\\
29	0.00247843861369422\\
30	0.00247845318864961\\
31	0.00247846801314581\\
32	0.00247848309147611\\
33	0.00247849842800767\\
34	0.00247851402718287\\
35	0.00247852989352059\\
36	0.00247854603161746\\
37	0.00247856244614932\\
38	0.0024785791418725\\
39	0.00247859612362519\\
40	0.00247861339632894\\
41	0.00247863096498999\\
42	0.00247864883470076\\
43	0.00247866701064138\\
44	0.00247868549808116\\
45	0.00247870430238013\\
46	0.00247872342899056\\
47	0.00247874288345863\\
48	0.00247876267142599\\
49	0.00247878279863142\\
50	0.00247880327091246\\
51	0.0024788240942072\\
52	0.0024788452745559\\
53	0.00247886681810285\\
54	0.00247888873109806\\
55	0.00247891101989917\\
56	0.00247893369097324\\
57	0.00247895675089867\\
58	0.00247898020636708\\
59	0.00247900406418525\\
60	0.00247902833127715\\
61	0.00247905301468592\\
62	0.00247907812157592\\
63	0.00247910365923484\\
64	0.00247912963507571\\
65	0.00247915605663926\\
66	0.00247918293159587\\
67	0.00247921026774801\\
68	0.00247923807303234\\
69	0.00247926635552212\\
70	0.00247929512342953\\
71	0.00247932438510803\\
72	0.00247935414905477\\
73	0.0024793844239131\\
74	0.00247941521847502\\
75	0.00247944654168382\\
76	0.00247947840263653\\
77	0.00247951081058671\\
78	0.00247954377494698\\
79	0.00247957730529186\\
80	0.00247961141136047\\
81	0.00247964610305934\\
82	0.00247968139046533\\
83	0.00247971728382847\\
84	0.00247975379357496\\
85	0.00247979093031016\\
86	0.00247982870482167\\
87	0.00247986712808238\\
88	0.00247990621125367\\
89	0.00247994596568866\\
90	0.00247998640293537\\
91	0.00248002753474013\\
92	0.0024800693730509\\
93	0.0024801119300207\\
94	0.00248015521801113\\
95	0.00248019924959592\\
96	0.00248024403756445\\
97	0.00248028959492547\\
98	0.00248033593491084\\
99	0.00248038307097927\\
100	0.00248043101682014\\
101	0.00248047978635751\\
102	0.00248052939375398\\
103	0.00248057985341478\\
104	0.0024806311799919\\
105	0.00248068338838818\\
106	0.00248073649376166\\
107	0.00248079051152982\\
108	0.00248084545737399\\
109	0.00248090134724376\\
110	0.00248095819736159\\
111	0.00248101602422742\\
112	0.00248107484462324\\
113	0.00248113467561801\\
114	0.00248119553457237\\
115	0.0024812574391437\\
116	0.00248132040729099\\
117	0.00248138445728006\\
118	0.00248144960768869\\
119	0.00248151587741186\\
120	0.0024815832856672\\
121	0.00248165185200038\\
122	0.00248172159629065\\
123	0.00248179253875655\\
124	0.00248186469996156\\
125	0.00248193810082007\\
126	0.00248201276260318\\
127	0.00248208870694486\\
128	0.00248216595584803\\
129	0.00248224453169084\\
130	0.00248232445723302\\
131	0.00248240575562236\\
132	0.0024824884504013\\
133	0.00248257256551368\\
134	0.0024826581253114\\
135	0.00248274515456154\\
136	0.00248283367845326\\
137	0.00248292372260504\\
138	0.00248301531307206\\
139	0.00248310847635362\\
140	0.00248320323940094\\
141	0.00248329962962534\\
142	0.00248339767490694\\
143	0.0024834974036043\\
144	0.00248359884456432\\
145	0.00248370202713047\\
146	0.0024838069811441\\
147	0.00248391373694049\\
148	0.00248402232537194\\
149	0.00248413277781647\\
150	0.00248424512618674\\
151	0.00248435940293909\\
152	0.00248447564108274\\
153	0.00248459387418913\\
154	0.00248471413640142\\
155	0.00248483646244414\\
156	0.00248496088763306\\
157	0.00248508744788513\\
158	0.00248521617972863\\
159	0.00248534712031352\\
160	0.00248548030742194\\
161	0.00248561577947882\\
162	0.0024857535755628\\
163	0.00248589373541721\\
164	0.00248603629946128\\
165	0.00248618130880154\\
166	0.0024863288052434\\
167	0.00248647883130295\\
168	0.00248663143021888\\
169	0.00248678664596469\\
170	0.00248694452326102\\
171	0.00248710510758828\\
172	0.00248726844519937\\
173	0.00248743458313271\\
174	0.00248760356922545\\
175	0.00248777545212685\\
176	0.00248795028131197\\
177	0.0024881281070955\\
178	0.00248830898064589\\
179	0.00248849295399968\\
180	0.00248868008007601\\
181	0.00248887041269149\\
182	0.00248906400657519\\
183	0.002489260917384\\
184	0.00248946120171807\\
185	0.00248966491713671\\
186	0.00248987212217436\\
187	0.00249008287635696\\
188	0.0024902972402185\\
189	0.0024905152753179\\
190	0.00249073704425609\\
191	0.00249096261069343\\
192	0.00249119203936747\\
193	0.00249142539611082\\
194	0.00249166274786942\\
195	0.00249190416272117\\
196	0.00249214970989475\\
197	0.00249239945978876\\
198	0.00249265348399125\\
199	0.00249291185529943\\
200	0.0024931746477399\\
201	0.00249344193658893\\
202	0.0024937137983933\\
203	0.00249399031099142\\
204	0.00249427155353465\\
205	0.00249455760650921\\
206	0.00249484855175818\\
207	0.002495144472504\\
208	0.00249544545337136\\
209	0.0024957515804103\\
210	0.00249606294111987\\
211	0.00249637962447197\\
212	0.00249670172093573\\
213	0.00249702932250214\\
214	0.00249736252270925\\
215	0.0024977014166675\\
216	0.00249804610108577\\
217	0.00249839667429747\\
218	0.00249875323628743\\
219	0.00249911588871892\\
220	0.00249948473496118\\
221	0.00249985988011741\\
222	0.00250024143105315\\
223	0.00250062949642516\\
224	0.00250102418671064\\
225	0.002501425614237\\
226	0.00250183389321195\\
227	0.00250224913975434\\
228	0.00250267147192506\\
229	0.00250310100975868\\
230	0.00250353787529553\\
231	0.0025039821926142\\
232	0.00250443408786451\\
233	0.00250489368930104\\
234	0.0025053611273171\\
235	0.0025058365344792\\
236	0.00250632004556201\\
237	0.00250681179758382\\
238	0.00250731192984247\\
239	0.00250782058395191\\
240	0.00250833790387907\\
241	0.00250886403598137\\
242	0.00250939912904469\\
243	0.00250994333432182\\
244	0.00251049680557143\\
245	0.00251105969909756\\
246	0.00251163217378957\\
247	0.00251221439116256\\
248	0.00251280651539834\\
249	0.00251340871338679\\
250	0.00251402115476774\\
251	0.0025146440119733\\
252	0.00251527746027064\\
253	0.00251592167780518\\
254	0.00251657684564421\\
255	0.00251724314782092\\
256	0.00251792077137885\\
257	0.00251860990641661\\
258	0.00251931074613306\\
259	0.00252002348687272\\
260	0.00252074832817157\\
261	0.00252148547280303\\
262	0.00252223512682423\\
263	0.00252299749962257\\
264	0.00252377280396222\\
265	0.00252456125603103\\
266	0.00252536307548745\\
267	0.00252617848550731\\
268	0.00252700771283079\\
269	0.00252785098780889\\
270	0.00252870854444927\\
271	0.00252958062045988\\
272	0.00253046745728757\\
273	0.00253136930014451\\
274	0.00253228639800854\\
275	0.00253321900358004\\
276	0.00253416737321523\\
277	0.00253513176705866\\
278	0.00253611244993631\\
279	0.00253710969076723\\
280	0.00253812376256445\\
281	0.00253915494247581\\
282	0.00254020351182379\\
283	0.00254126975614418\\
284	0.00254235396522361\\
285	0.00254345643313563\\
286	0.0025445774582754\\
287	0.00254571734339269\\
288	0.00254687639562325\\
289	0.00254805492651808\\
290	0.00254925325207081\\
291	0.00255047169274272\\
292	0.00255171057348544\\
293	0.00255297022376099\\
294	0.00255425097755892\\
295	0.00255555317341061\\
296	0.00255687715440021\\
297	0.00255822326817209\\
298	0.00255959186693452\\
299	0.0025609833074594\\
300	0.00256239795107767\\
301	0.00256383616367012\\
302	0.00256529831565325\\
303	0.00256678478196009\\
304	0.00256829594201556\\
305	0.00256983217970675\\
306	0.00257139388334885\\
307	0.00257298144564973\\
308	0.00257459526368051\\
309	0.00257623573886992\\
310	0.00257790327705853\\
311	0.00257959828867543\\
312	0.00258132118909213\\
313	0.00258307239899605\\
314	0.00258485234379022\\
315	0.00258666145005346\\
316	0.00258850014697706\\
317	0.0025903688672863\\
318	0.00259226804708924\\
319	0.0025941981257115\\
320	0.00259615954551571\\
321	0.00259815275170518\\
322	0.0026001781921106\\
323	0.00260223631695857\\
324	0.00260432757862129\\
325	0.00260645243134581\\
326	0.00260861133096183\\
327	0.00261080473456659\\
328	0.00261303310018548\\
329	0.00261529688640687\\
330	0.00261759655198955\\
331	0.00261993255544103\\
332	0.00262230535456489\\
333	0.00262471540597562\\
334	0.00262716316457854\\
335	0.00262964908301356\\
336	0.00263217361106048\\
337	0.00263473719500448\\
338	0.00263734027696049\\
339	0.00263998329415559\\
340	0.00264266667816952\\
341	0.00264539085413462\\
342	0.0026481562398981\\
343	0.00265096324515207\\
344	0.00265381227054041\\
345	0.00265670370675562\\
346	0.00265963793364637\\
347	0.00266261531936403\\
348	0.00266563621958899\\
349	0.00266870097689276\\
350	0.00267180992031164\\
351	0.00267496336523225\\
352	0.00267816161371457\\
353	0.00268140495539548\\
354	0.00268469366908969\\
355	0.00268802802506198\\
356	0.00269140828756429\\
357	0.00269483471685336\\
358	0.00269830757285886\\
359	0.00270182714513143\\
360	0.00270539380062159\\
361	0.00270900798273177\\
362	0.00271267024811181\\
363	0.00271638131646151\\
364	0.00272014213838986\\
365	0.00272395399022021\\
366	0.00272781861511963\\
367	0.00273173846273063\\
368	0.00273571718681733\\
369	0.00273976151590172\\
370	0.00274387685410735\\
371	0.0027480652175805\\
372	0.0027523284555114\\
373	0.00275666849663476\\
374	0.00276108735077459\\
375	0.00276558710992141\\
376	0.0027701699490338\\
377	0.00277483812421509\\
378	0.00277959394645428\\
379	0.00278443959488847\\
380	0.00278937597696776\\
381	0.00279440505860262\\
382	0.00279952987475586\\
383	0.00280475368908226\\
384	0.00281008002083447\\
385	0.0028155126757165\\
386	0.0028210557813721\\
387	0.00282671382833309\\
388	0.00283249171741898\\
389	0.00283839481478436\\
390	0.00284442901606047\\
391	0.00285060082134511\\
392	0.00285691742317362\\
393	0.00286338681006984\\
394	0.00287001788885343\\
395	0.00287682062959411\\
396	0.00288380623799019\\
397	0.00289098736105367\\
398	0.00289837833336593\\
399	0.00290599547291058\\
400	0.00291385743772171\\
401	0.00292198565753793\\
402	0.00293040485881307\\
403	0.00293914370801658\\
404	0.00294823561058515\\
405	0.00295771973216965\\
406	0.0029676423972256\\
407	0.00297805938066788\\
408	0.00298904182625361\\
409	0.00300066528391573\\
410	0.00301301628128928\\
411	0.00302619752181333\\
412	0.00304033140484477\\
413	0.00305556435770621\\
414	0.00307207207453714\\
415	0.00309006482955999\\
416	0.00310880661883055\\
417	0.00312785605903395\\
418	0.00314721714365967\\
419	0.00316689384652706\\
420	0.00318689014002555\\
421	0.00320721005795063\\
422	0.00322785787452478\\
423	0.00324883855072928\\
424	0.0032701585118829\\
425	0.003291823707487\\
426	0.0033138379259991\\
427	0.0033362048377587\\
428	0.00335892797542417\\
429	0.00338201071198679\\
430	0.00340545623601502\\
431	0.00342926752372865\\
432	0.00345344730742967\\
433	0.00347799803973121\\
434	0.00350292185292983\\
435	0.00352822051275047\\
436	0.00355389536554866\\
437	0.00357994727788222\\
438	0.00360637656715544\\
439	0.00363318292178348\\
440	0.0036603653090112\\
441	0.00368792186812286\\
442	0.00371584978624311\\
443	0.0037441451531795\\
444	0.00377280279012463\\
445	0.00380181604259669\\
446	0.00383117651323895\\
447	0.00386087365251626\\
448	0.00389089388583082\\
449	0.00392121690199829\\
450	0.00395179716637846\\
451	0.00398259838267485\\
452	0.00401358468378152\\
453	0.00404471265061179\\
454	0.00407592977923796\\
455	0.00410717277754148\\
456	0.00413836605779342\\
457	0.00416942179077619\\
458	0.00420024634456966\\
459	0.00423081211061245\\
460	0.00426140020028307\\
461	0.00429200101853043\\
462	0.00432253003295564\\
463	0.0043528838050062\\
464	0.00438293598073556\\
465	0.0044125325848898\\
466	0.0044414859553579\\
467	0.00446956698539481\\
468	0.00449649542059098\\
469	0.00452192790322189\\
470	0.00454544332458644\\
471	0.00456660753965114\\
472	0.00458814729311082\\
473	0.0046100682754802\\
474	0.00463237617641393\\
475	0.00465507664007141\\
476	0.00467817529335925\\
477	0.00470167773876455\\
478	0.00472558954445719\\
479	0.00474991623144585\\
480	0.00477466327102612\\
481	0.00479983614302007\\
482	0.00482544066736576\\
483	0.00485148331035823\\
484	0.00487797072227519\\
485	0.00490490972166369\\
486	0.00493230688573686\\
487	0.00496016866117437\\
488	0.00498850152122487\\
489	0.00501731171296575\\
490	0.00504660464559965\\
491	0.00507638424155821\\
492	0.0051066539743321\\
493	0.00513741754812156\\
494	0.00516867868920914\\
495	0.00520044483843773\\
496	0.00523272670992935\\
497	0.00526553942337987\\
498	0.00529890398854494\\
499	0.00533285017825539\\
500	0.00536740912600212\\
501	0.00540262192716388\\
502	0.00543854471967436\\
503	0.0054752551006337\\
504	0.00551286051987373\\
505	0.0055515092381846\\
506	0.00559130719092906\\
507	0.00563230368435302\\
508	0.00567455068067331\\
509	0.00571810435251307\\
510	0.00576302553152001\\
511	0.00580939070107368\\
512	0.00585727370026442\\
513	0.00590674250978723\\
514	0.005957854838526\\
515	0.00601065214427572\\
516	0.00606515167623523\\
517	0.00612133597063666\\
518	0.00617913868276773\\
519	0.00623842574325406\\
520	0.00629897040266387\\
521	0.00636042026026284\\
522	0.00642121846438805\\
523	0.00648079970694343\\
524	0.00653898423739954\\
525	0.00659558459818578\\
526	0.00665040955394938\\
527	0.00670326996925213\\
528	0.00675398737755338\\
529	0.00680240615739019\\
530	0.00684841085110631\\
531	0.00689195017212662\\
532	0.00693307314023976\\
533	0.00697237816780817\\
534	0.00701119533494809\\
535	0.00704952235122946\\
536	0.00708736978307209\\
537	0.00712476342145339\\
538	0.0071617466085178\\
539	0.00719838229666012\\
540	0.00723475449016275\\
541	0.00727096855127332\\
542	0.00730714997322877\\
543	0.00734344027894289\\
544	0.00737998840528387\\
545	0.00741688811617026\\
546	0.00745418951685885\\
547	0.00749193730271656\\
548	0.00753017973388053\\
549	0.00756896791481386\\
550	0.00760835477395233\\
551	0.00764839371723731\\
552	0.00768913694195016\\
553	0.00773063378523078\\
554	0.00777292424702521\\
555	0.00781603521840035\\
556	0.00785998739163325\\
557	0.00790479871128239\\
558	0.00795048664294056\\
559	0.00799706792415065\\
560	0.00804455829354392\\
561	0.00809297220498276\\
562	0.00814232253144814\\
563	0.00819262028262012\\
564	0.00824387435428926\\
565	0.00829609133116553\\
566	0.00834927536349354\\
567	0.00840342812126299\\
568	0.00845854884012727\\
569	0.00851463422662447\\
570	0.00857167829860371\\
571	0.00862967217526905\\
572	0.00868860424489086\\
573	0.00874845809866899\\
574	0.00880921036058856\\
575	0.00887082894270814\\
576	0.00893328022478691\\
577	0.00899649525040652\\
578	0.00906043870669162\\
579	0.00912517915203537\\
580	0.00919005656715616\\
581	0.00925329200342372\\
582	0.00931456036497874\\
583	0.00937352397491844\\
584	0.00942923087906576\\
585	0.00948220461794547\\
586	0.00953271440881462\\
587	0.00958044208054697\\
588	0.00962596473294314\\
589	0.00966992069233143\\
590	0.00971269563307853\\
591	0.00975441316854107\\
592	0.00979516557260156\\
593	0.00983492675369396\\
594	0.00987369348216993\\
595	0.00991118387968948\\
596	0.00994658651044256\\
597	0.00997788999445116\\
598	0.010000292044645\\
599	0\\
600	0\\
};
\addplot [color=blue,solid,forget plot]
  table[row sep=crcr]{%
1	0.000300501426608031\\
2	0.000300519631885623\\
3	0.000300538148804544\\
4	0.000300556982713254\\
5	0.000300576139051885\\
6	0.000300595623353805\\
7	0.000300615441247206\\
8	0.000300635598456722\\
9	0.000300656100805079\\
10	0.000300676954214768\\
11	0.000300698164709764\\
12	0.000300719738417195\\
13	0.0003007416815692\\
14	0.000300764000504607\\
15	0.000300786701670854\\
16	0.00030080979162578\\
17	0.000300833277039507\\
18	0.000300857164696399\\
19	0.000300881461496977\\
20	0.000300906174459875\\
21	0.000300931310723911\\
22	0.000300956877550082\\
23	0.00030098288232366\\
24	0.000301009332556328\\
25	0.000301036235888336\\
26	0.000301063600090647\\
27	0.000301091433067201\\
28	0.000301119742857146\\
29	0.000301148537637201\\
30	0.000301177825723917\\
31	0.000301207615576129\\
32	0.000301237915797319\\
33	0.000301268735138103\\
34	0.000301300082498759\\
35	0.000301331966931708\\
36	0.000301364397644179\\
37	0.000301397384000774\\
38	0.000301430935526176\\
39	0.000301465061907863\\
40	0.000301499772998864\\
41	0.00030153507882061\\
42	0.000301570989565755\\
43	0.000301607515601091\\
44	0.00030164466747051\\
45	0.000301682455898007\\
46	0.000301720891790759\\
47	0.000301759986242223\\
48	0.00030179975053527\\
49	0.000301840196145408\\
50	0.000301881334744117\\
51	0.000301923178202066\\
52	0.000301965738592561\\
53	0.000302009028194973\\
54	0.000302053059498194\\
55	0.000302097845204237\\
56	0.000302143398231815\\
57	0.000302189731720019\\
58	0.000302236859032043\\
59	0.000302284793759027\\
60	0.000302333549723835\\
61	0.000302383140985043\\
62	0.000302433581840894\\
63	0.000302484886833373\\
64	0.000302537070752326\\
65	0.000302590148639623\\
66	0.000302644135793482\\
67	0.00030269904777273\\
68	0.000302754900401292\\
69	0.000302811709772578\\
70	0.000302869492254094\\
71	0.000302928264492068\\
72	0.000302988043416155\\
73	0.000303048846244216\\
74	0.000303110690487172\\
75	0.000303173593953979\\
76	0.00030323757475667\\
77	0.00030330265131539\\
78	0.000303368842363687\\
79	0.000303436166953741\\
80	0.000303504644461784\\
81	0.000303574294593534\\
82	0.000303645137389719\\
83	0.000303717193231815\\
84	0.000303790482847713\\
85	0.000303865027317568\\
86	0.00030394084807977\\
87	0.000304017966936938\\
88	0.000304096406062103\\
89	0.000304176188004911\\
90	0.000304257335697963\\
91	0.000304339872463302\\
92	0.000304423822018948\\
93	0.000304509208485584\\
94	0.000304596056393289\\
95	0.000304684390688489\\
96	0.000304774236740935\\
97	0.000304865620350862\\
98	0.000304958567756184\\
99	0.000305053105639937\\
100	0.000305149261137713\\
101	0.000305247061845276\\
102	0.000305346535826389\\
103	0.000305447711620587\\
104	0.000305550618251278\\
105	0.000305655285233842\\
106	0.000305761742583948\\
107	0.00030587002082592\\
108	0.000305980151001394\\
109	0.000306092164677971\\
110	0.000306206093958081\\
111	0.000306321971488003\\
112	0.000306439830467069\\
113	0.000306559704656874\\
114	0.000306681628390895\\
115	0.000306805636583991\\
116	0.000306931764742322\\
117	0.000307060048973214\\
118	0.000307190525995378\\
119	0.000307323233149159\\
120	0.000307458208407031\\
121	0.000307595490384293\\
122	0.000307735118349858\\
123	0.000307877132237315\\
124	0.0003080215726561\\
125	0.000308168480902933\\
126	0.000308317898973392\\
127	0.000308469869573712\\
128	0.000308624436132747\\
129	0.000308781642814205\\
130	0.000308941534529002\\
131	0.000309104156947894\\
132	0.00030926955651428\\
133	0.00030943778045723\\
134	0.000309608876804769\\
135	0.000309782894397335\\
136	0.000309959882901501\\
137	0.000310139892823947\\
138	0.00031032297552562\\
139	0.000310509183236287\\
140	0.000310698569069238\\
141	0.000310891187036353\\
142	0.000311087092063406\\
143	0.000311286340005505\\
144	0.000311488987662527\\
145	0.000311695092794237\\
146	0.000311904714135457\\
147	0.000312117911414112\\
148	0.000312334745368019\\
149	0.000312555277761922\\
150	0.000312779571404839\\
151	0.000313007690167653\\
152	0.000313239699001119\\
153	0.000313475663954089\\
154	0.000313715652191992\\
155	0.000313959732015797\\
156	0.000314207972881154\\
157	0.000314460445417914\\
158	0.00031471722145005\\
159	0.000314978374015754\\
160	0.000315243977388053\\
161	0.000315514107095696\\
162	0.000315788839944392\\
163	0.000316068254038434\\
164	0.000316352428802739\\
165	0.000316641445005159\\
166	0.000316935384779305\\
167	0.000317234331647649\\
168	0.000317538370545133\\
169	0.000317847587843084\\
170	0.000318162071373661\\
171	0.000318481910454595\\
172	0.000318807195914479\\
173	0.000319138020118415\\
174	0.000319474476994141\\
175	0.000319816662058643\\
176	0.000320164672445176\\
177	0.000320518606930806\\
178	0.000320878565964381\\
179	0.000321244651695082\\
180	0.00032161696800138\\
181	0.000321995620520563\\
182	0.000322380716678798\\
183	0.000322772365721634\\
184	0.000323170678745169\\
185	0.000323575768727642\\
186	0.000323987750561753\\
187	0.000324406741087356\\
188	0.000324832859124922\\
189	0.000325266225509471\\
190	0.000325706963125168\\
191	0.000326155196940573\\
192	0.000326611054044423\\
193	0.000327074663682172\\
194	0.000327546157293139\\
195	0.000328025668548302\\
196	0.00032851333338885\\
197	0.000329009290065364\\
198	0.000329513679177751\\
199	0.000330026643715947\\
200	0.000330548329101246\\
201	0.00033107888322853\\
202	0.000331618456509183\\
203	0.000332167201914825\\
204	0.00033272527502188\\
205	0.00033329283405693\\
206	0.000333870039942939\\
207	0.000334457056346362\\
208	0.000335054049725074\\
209	0.00033566118937726\\
210	0.000336278647491166\\
211	0.000336906599195865\\
212	0.000337545222612936\\
213	0.00033819469890916\\
214	0.000338855212350104\\
215	0.000339526950354936\\
216	0.000340210103552097\\
217	0.000340904865836186\\
218	0.00034161143442579\\
219	0.000342330009922658\\
220	0.000343060796371776\\
221	0.000343804001322807\\
222	0.000344559835892605\\
223	0.000345328514829062\\
224	0.000346110256576102\\
225	0.000346905283340077\\
226	0.000347713821157424\\
227	0.000348536099963604\\
228	0.000349372353663567\\
229	0.000350222820203491\\
230	0.000351087741644082\\
231	0.000351967364235264\\
232	0.000352861938492422\\
233	0.000353771719274236\\
234	0.00035469696586207\\
235	0.000355637942040964\\
236	0.000356594916182395\\
237	0.000357568161328625\\
238	0.000358557955278963\\
239	0.000359564580677665\\
240	0.000360588325103814\\
241	0.00036162948116306\\
242	0.000362688346581259\\
243	0.000363765224300177\\
244	0.00036486042257567\\
245	0.000365974255075711\\
246	0.000367107040985112\\
247	0.000368259105107945\\
248	0.000369430777974456\\
249	0.000370622395950006\\
250	0.00037183430134628\\
251	0.000373066842535078\\
252	0.000374320374064625\\
253	0.000375595256778484\\
254	0.000376891857937245\\
255	0.000378210551342952\\
256	0.000379551717466371\\
257	0.000380915743577281\\
258	0.000382303023877824\\
259	0.000383713959638895\\
260	0.000385148959339871\\
261	0.000386608438811652\\
262	0.000388092821383163\\
263	0.000389602538031377\\
264	0.000391138027535063\\
265	0.0003926997366323\\
266	0.000394288120181847\\
267	0.000395903641328527\\
268	0.000397546771672606\\
269	0.000399217991443161\\
270	0.000400917789675084\\
271	0.00040264666438957\\
272	0.000404405122777409\\
273	0.000406193681385395\\
274	0.000408012866308006\\
275	0.000409863213390112\\
276	0.000411745268452753\\
277	0.000413659587586473\\
278	0.000415606737301452\\
279	0.000417587294749511\\
280	0.000419601847956675\\
281	0.000421650996062257\\
282	0.000423735349564715\\
283	0.000425855530574405\\
284	0.000428012173073514\\
285	0.000430205923183494\\
286	0.000432437439440065\\
287	0.000434707393076316\\
288	0.000437016468313859\\
289	0.000439365362662696\\
290	0.000441754787229672\\
291	0.00044418546703613\\
292	0.000446658141344895\\
293	0.000449173563996945\\
294	0.000451732503758152\\
295	0.000454335744676249\\
296	0.000456984086448427\\
297	0.000459678344800051\\
298	0.000462419351874562\\
299	0.000465207956635017\\
300	0.000468045025277779\\
301	0.000470931441658408\\
302	0.000473868107730537\\
303	0.000476855943997822\\
304	0.000479895889980021\\
305	0.000482988904693828\\
306	0.000486135967150064\\
307	0.000489338076869641\\
308	0.000492596254421141\\
309	0.00049591154198332\\
310	0.000499285003931393\\
311	0.000502717727435768\\
312	0.000506210823043187\\
313	0.000509765425183453\\
314	0.000513382692465421\\
315	0.000517063808503682\\
316	0.000520809982684537\\
317	0.000524622450834215\\
318	0.000528502475904416\\
319	0.000532451348675109\\
320	0.000536470388474993\\
321	0.000540560943919258\\
322	0.000544724393664747\\
323	0.000548962147182398\\
324	0.000553275645546648\\
325	0.000557666362241569\\
326	0.000562135803983405\\
327	0.000566685511558842\\
328	0.000571317060678727\\
329	0.000576032062846095\\
330	0.000580832166237965\\
331	0.000585719056599692\\
332	0.000590694458150562\\
333	0.000595760134499168\\
334	0.000600917889566737\\
335	0.000606169568516076\\
336	0.000611517058683465\\
337	0.000616962290510151\\
338	0.000622507238469203\\
339	0.000628153921982653\\
340	0.000633904406322138\\
341	0.000639760803484944\\
342	0.0006457252730346\\
343	0.000651800022892481\\
344	0.000657987310062489\\
345	0.000664289441266136\\
346	0.000670708773457938\\
347	0.000677247714182177\\
348	0.000683908721719769\\
349	0.000690694304957674\\
350	0.000697607022890975\\
351	0.000704649483637008\\
352	0.000711824342799276\\
353	0.00071913430096381\\
354	0.000726582100049746\\
355	0.000734170518191728\\
356	0.000741902362825378\\
357	0.000749780461643887\\
358	0.000757807652416314\\
359	0.00076598676939214\\
360	0.00077432061904643\\
361	0.000782811951648188\\
362	0.000791463422611622\\
363	0.00080027753796957\\
364	0.000809256571552262\\
365	0.000818402422494648\\
366	0.000827716324432034\\
367	0.000837198121641135\\
368	0.000846843886573104\\
369	0.00085621759755801\\
370	0.000865719835157762\\
371	0.000875384922099738\\
372	0.000885215919596194\\
373	0.000895215956711118\\
374	0.000905388232153993\\
375	0.000915736015490102\\
376	0.000926262646707834\\
377	0.000936971533698378\\
378	0.000947866152129432\\
379	0.000958949985040097\\
380	0.000970226771537842\\
381	0.000981700472945479\\
382	0.000993375175281812\\
383	0.00100525509678743\\
384	0.00101734459611838\\
385	0.00102964818126044\\
386	0.00104217051921655\\
387	0.00105491644651524\\
388	0.00106789098058222\\
389	0.00108109933200473\\
390	0.00109454691770115\\
391	0.00110823937498109\\
392	0.00112218257644393\\
393	0.00113638264561016\\
394	0.0011508459731072\\
395	0.00116557923313257\\
396	0.00118058939978738\\
397	0.00119588376270498\\
398	0.00121146994119191\\
399	0.00122735589585838\\
400	0.00124354993647432\\
401	0.00126006072461482\\
402	0.00127689726966284\\
403	0.00129406891689551\\
404	0.00131158532557222\\
405	0.00132945642701809\\
406	0.00134769232504865\\
407	0.00136630334920238\\
408	0.00138529863967356\\
409	0.0014046865214009\\
410	0.00142447447639833\\
411	0.00144466876411519\\
412	0.00146527380807004\\
413	0.00148629236136712\\
414	0.00150772242408549\\
415	0.00152955174845827\\
416	0.00155178053429491\\
417	0.00157441738488319\\
418	0.00159747117000727\\
419	0.0016209510473805\\
420	0.00164486649233153\\
421	0.00166922733669089\\
422	0.00169404381114076\\
423	0.00171932658913223\\
424	0.00174508660911482\\
425	0.00177133497272887\\
426	0.00179808317403233\\
427	0.00182534312284982\\
428	0.00185312717023215\\
429	0.00188144813627173\\
430	0.00191031934055164\\
431	0.00193975463554364\\
432	0.00196976844331324\\
433	0.00200037579593989\\
434	0.00203159238011775\\
435	0.00206343458646723\\
436	0.00209591956415803\\
437	0.00212906528151442\\
438	0.00216289059332055\\
439	0.00219741531550776\\
440	0.00223266030762541\\
441	0.00226864756255277\\
442	0.00230540030025111\\
443	0.00234294305536536\\
444	0.00238130173134972\\
445	0.00242050355843802\\
446	0.00246057688303989\\
447	0.00250155151506867\\
448	0.00254345254983823\\
449	0.0025863163109077\\
450	0.00263018519724896\\
451	0.00267510706601041\\
452	0.00272113538969605\\
453	0.00276833032719836\\
454	0.00281676004265518\\
455	0.00286650228071771\\
456	0.00291764598612343\\
457	0.00297029183945389\\
458	0.00302454769034921\\
459	0.00305055853264484\\
460	0.0030717054879797\\
461	0.00309373699915778\\
462	0.00311676704613136\\
463	0.00314094401885549\\
464	0.00316642522675929\\
465	0.00319338423266275\\
466	0.00322202684986058\\
467	0.00325259823185108\\
468	0.00328539262995512\\
469	0.00332076001564318\\
470	0.00335911688018843\\
471	0.00340087897834244\\
472	0.00344329284598584\\
473	0.00348636279786332\\
474	0.00353009234460264\\
475	0.00357448410288194\\
476	0.00361953969633514\\
477	0.00366525962840545\\
478	0.00371164300374218\\
479	0.00375868663529404\\
480	0.00380638141616894\\
481	0.00385469345016474\\
482	0.00390358045788171\\
483	0.00395302257859817\\
484	0.00400299667579476\\
485	0.00405347626436221\\
486	0.0041044320174268\\
487	0.00415583392408783\\
488	0.00420765852502683\\
489	0.00425991707445931\\
490	0.00431270042363151\\
491	0.00436601211529089\\
492	0.0044197932248762\\
493	0.00447397060726542\\
494	0.00452845408490524\\
495	0.00458313250642518\\
496	0.00463786904438652\\
497	0.00469249538226218\\
498	0.00474680451832123\\
499	0.00480054107464478\\
500	0.00485339074819056\\
501	0.00490496638135537\\
502	0.00495479087986692\\
503	0.00500227689936128\\
504	0.00504670127209525\\
505	0.00508717390834856\\
506	0.00512773714340334\\
507	0.00516850937848077\\
508	0.00520938472009274\\
509	0.00525054066415425\\
510	0.00529261159274347\\
511	0.00533562078544122\\
512	0.00537959235534939\\
513	0.00542455163858445\\
514	0.00547052605096188\\
515	0.00551754726203399\\
516	0.00556563622915616\\
517	0.00561481300511334\\
518	0.00566509525535443\\
519	0.00571650047162964\\
520	0.00576904879683756\\
521	0.00582276570887622\\
522	0.00587769973423755\\
523	0.00593390604542822\\
524	0.00599143941593479\\
525	0.00605035287192922\\
526	0.00611069585636212\\
527	0.00617251175254849\\
528	0.00623583452787127\\
529	0.00630068445836571\\
530	0.00636706283199385\\
531	0.00643494485661551\\
532	0.00650411822733752\\
533	0.00657373800204402\\
534	0.0066422614192598\\
535	0.00670949828702219\\
536	0.0067752479697538\\
537	0.00683930310358746\\
538	0.00690145550474471\\
539	0.00696150512625364\\
540	0.00701927321286701\\
541	0.00707462004096888\\
542	0.00712745146192194\\
543	0.0071777532813973\\
544	0.00722564490142717\\
545	0.00727260397683621\\
546	0.00731915191368715\\
547	0.00736529988918563\\
548	0.00741107522451367\\
549	0.007456523672545\\
550	0.00750171124054802\\
551	0.00754672506984011\\
552	0.00759167265990017\\
553	0.00763667928142201\\
554	0.00768188692804579\\
555	0.00772745204322683\\
556	0.00777352896986929\\
557	0.00782017594083521\\
558	0.00786741630291251\\
559	0.00791527588252588\\
560	0.00796378264304856\\
561	0.00801296618049495\\
562	0.00806285723861679\\
563	0.00811348655083227\\
564	0.00816488335329511\\
565	0.0082170736010332\\
566	0.00827007798995417\\
567	0.00832390999756632\\
568	0.00837857422847543\\
569	0.00843407061551391\\
570	0.0084903967595156\\
571	0.00854754764423447\\
572	0.00860551529537557\\
573	0.00866428858236032\\
574	0.00872385303281798\\
575	0.00878419052251876\\
576	0.00884527895369516\\
577	0.00890709236790587\\
578	0.00896959986638077\\
579	0.00903276347919527\\
580	0.00909654468053119\\
581	0.00916091558216332\\
582	0.00922584619394916\\
583	0.00929124381478611\\
584	0.00935712289323052\\
585	0.00942180913543368\\
586	0.00948442867380405\\
587	0.00954462375269818\\
588	0.00960141065634341\\
589	0.00965385408934247\\
590	0.00970260651805489\\
591	0.00974846666295701\\
592	0.00979195816415004\\
593	0.00983364285089848\\
594	0.00987334736606283\\
595	0.0099111519251561\\
596	0.00994658651044256\\
597	0.00997788999445116\\
598	0.010000292044645\\
599	0\\
600	0\\
};
\addplot [color=mycolor10,solid,forget plot]
  table[row sep=crcr]{%
1	4.64198342570775e-05\\
2	4.64200387921881e-05\\
3	4.64202468284378e-05\\
4	4.64204584259058e-05\\
5	4.64206736456894e-05\\
6	4.64208925499346e-05\\
7	4.6421115201856e-05\\
8	4.64213416657481e-05\\
9	4.64215720070079e-05\\
10	4.64218062921487e-05\\
11	4.6422044588823e-05\\
12	4.64222869658486e-05\\
13	4.64225334932136e-05\\
14	4.64227842421049e-05\\
15	4.64230392849374e-05\\
16	4.64232986953489e-05\\
17	4.64235625482508e-05\\
18	4.6423830919835e-05\\
19	4.64241038875909e-05\\
20	4.6424381530333e-05\\
21	4.64246639282335e-05\\
22	4.64249511628312e-05\\
23	4.64252433170587e-05\\
24	4.64255404752666e-05\\
25	4.64258427232428e-05\\
26	4.64261501482564e-05\\
27	4.64264628390569e-05\\
28	4.6426780885913e-05\\
29	4.64271043806428e-05\\
30	4.64274334166229e-05\\
31	4.64277680888393e-05\\
32	4.6428108493887e-05\\
33	4.64284547300268e-05\\
34	4.64288068971803e-05\\
35	4.64291650969916e-05\\
36	4.6429529432838e-05\\
37	4.64299000098569e-05\\
38	4.64302769349861e-05\\
39	4.64306603169893e-05\\
40	4.64310502664832e-05\\
41	4.64314468959798e-05\\
42	4.64318503199062e-05\\
43	4.64322606546443e-05\\
44	4.64326780185624e-05\\
45	4.64331025320574e-05\\
46	4.64335343175637e-05\\
47	4.64339734996173e-05\\
48	4.64344202048737e-05\\
49	4.6434874562151e-05\\
50	4.6435336702462e-05\\
51	4.64358067590574e-05\\
52	4.64362848674592e-05\\
53	4.6436771165494e-05\\
54	4.64372657933456e-05\\
55	4.64377688935835e-05\\
56	4.64382806112103e-05\\
57	4.64388010936927e-05\\
58	4.6439330491014e-05\\
59	4.64398689557109e-05\\
60	4.64404166429274e-05\\
61	4.6440973710437e-05\\
62	4.64415403187029e-05\\
63	4.64421166309248e-05\\
64	4.64427028130801e-05\\
65	4.64432990339679e-05\\
66	4.64439054652635e-05\\
67	4.64445222815589e-05\\
68	4.64451496604215e-05\\
69	4.64457877824375e-05\\
70	4.64464368312647e-05\\
71	4.64470969936825e-05\\
72	4.64477684596473e-05\\
73	4.64484514223477e-05\\
74	4.64491460782499e-05\\
75	4.64498526271697e-05\\
76	4.64505712723039e-05\\
77	4.64513022203158e-05\\
78	4.64520456813705e-05\\
79	4.64528018692056e-05\\
80	4.64535710011883e-05\\
81	4.64543532983768e-05\\
82	4.64551489855851e-05\\
83	4.645595829144e-05\\
84	4.64567814484457e-05\\
85	4.64576186930545e-05\\
86	4.64584702657316e-05\\
87	4.64593364110224e-05\\
88	4.64602173776148e-05\\
89	4.64611134184232e-05\\
90	4.64620247906381e-05\\
91	4.64629517558166e-05\\
92	4.64638945799521e-05\\
93	4.64648535335428e-05\\
94	4.64658288916688e-05\\
95	4.646682093407e-05\\
96	4.64678299452291e-05\\
97	4.64688562144476e-05\\
98	4.64699000359229e-05\\
99	4.64709617088292e-05\\
100	4.64720415374174e-05\\
101	4.64731398310753e-05\\
102	4.64742569044417e-05\\
103	4.64753930774618e-05\\
104	4.64765486755127e-05\\
105	4.64777240294628e-05\\
106	4.6478919475784e-05\\
107	4.64801353566316e-05\\
108	4.64813720199543e-05\\
109	4.64826298195808e-05\\
110	4.6483909115327e-05\\
111	4.64852102730836e-05\\
112	4.64865336649245e-05\\
113	4.6487879669222e-05\\
114	4.64892486707316e-05\\
115	4.6490641060719e-05\\
116	4.64920572370491e-05\\
117	4.64934976043213e-05\\
118	4.64949625739567e-05\\
119	4.64964525643282e-05\\
120	4.64979680008789e-05\\
121	4.64995093162281e-05\\
122	4.65010769503138e-05\\
123	4.65026713504791e-05\\
124	4.65042929716377e-05\\
125	4.650594227637e-05\\
126	4.65076197350724e-05\\
127	4.65093258260765e-05\\
128	4.65110610357785e-05\\
129	4.65128258587917e-05\\
130	4.65146207980738e-05\\
131	4.65164463650569e-05\\
132	4.6518303079815e-05\\
133	4.65201914711785e-05\\
134	4.65221120769123e-05\\
135	4.65240654438355e-05\\
136	4.65260521279929e-05\\
137	4.65280726947906e-05\\
138	4.65301277191657e-05\\
139	4.65322177857247e-05\\
140	4.65343434889061e-05\\
141	4.65365054331653e-05\\
142	4.65387042331824e-05\\
143	4.65409405141296e-05\\
144	4.65432149117916e-05\\
145	4.65455280724838e-05\\
146	4.65478806534146e-05\\
147	4.65502733228561e-05\\
148	4.65527067603531e-05\\
149	4.65551816568849e-05\\
150	4.65576987150701e-05\\
151	4.65602586493646e-05\\
152	4.65628621862613e-05\\
153	4.65655100644745e-05\\
154	4.65682030351674e-05\\
155	4.65709418621456e-05\\
156	4.65737273220795e-05\\
157	4.65765602047107e-05\\
158	4.65794413130808e-05\\
159	4.65823714637411e-05\\
160	4.65853514869981e-05\\
161	4.65883822271284e-05\\
162	4.65914645426221e-05\\
163	4.6594599306412e-05\\
164	4.65977874061337e-05\\
165	4.66010297443526e-05\\
166	4.66043272388252e-05\\
167	4.6607680822761e-05\\
168	4.66110914450668e-05\\
169	4.66145600706136e-05\\
170	4.66180876805248e-05\\
171	4.66216752724192e-05\\
172	4.66253238607178e-05\\
173	4.66290344769032e-05\\
174	4.66328081698201e-05\\
175	4.6636646005968e-05\\
176	4.66405490697888e-05\\
177	4.66445184639809e-05\\
178	4.66485553098001e-05\\
179	4.66526607473745e-05\\
180	4.66568359360315e-05\\
181	4.66610820545924e-05\\
182	4.66654003017481e-05\\
183	4.66697918963588e-05\\
184	4.66742580778049e-05\\
185	4.6678800106339e-05\\
186	4.6683419263433e-05\\
187	4.6688116852151e-05\\
188	4.66928941974884e-05\\
189	4.66977526467724e-05\\
190	4.67026935700293e-05\\
191	4.67077183603696e-05\\
192	4.67128284343681e-05\\
193	4.67180252324946e-05\\
194	4.67233102194869e-05\\
195	4.67286848847864e-05\\
196	4.67341507429446e-05\\
197	4.67397093340665e-05\\
198	4.67453622242354e-05\\
199	4.67511110059563e-05\\
200	4.67569572986157e-05\\
201	4.67629027489341e-05\\
202	4.6768949031438e-05\\
203	4.67750978489395e-05\\
204	4.67813509330188e-05\\
205	4.67877100445156e-05\\
206	4.67941769740326e-05\\
207	4.68007535424626e-05\\
208	4.68074416014848e-05\\
209	4.68142430341166e-05\\
210	4.68211597552507e-05\\
211	4.68281937122044e-05\\
212	4.68353468852803e-05\\
213	4.68426212883389e-05\\
214	4.68500189693852e-05\\
215	4.68575420111601e-05\\
216	4.68651925317421e-05\\
217	4.68729726851704e-05\\
218	4.68808846620662e-05\\
219	4.68889306902817e-05\\
220	4.6897113035541e-05\\
221	4.69054340021147e-05\\
222	4.69138959334946e-05\\
223	4.69225012130759e-05\\
224	4.69312522648865e-05\\
225	4.69401515542656e-05\\
226	4.69492015886304e-05\\
227	4.69584049182035e-05\\
228	4.69677641367905e-05\\
229	4.6977281882533e-05\\
230	4.69869608387094e-05\\
231	4.69968037345613e-05\\
232	4.70068133460858e-05\\
233	4.70169924968945e-05\\
234	4.70273440590636e-05\\
235	4.70378709540219e-05\\
236	4.70485761534177e-05\\
237	4.70594626800511e-05\\
238	4.70705336088008e-05\\
239	4.70817920675649e-05\\
240	4.70932412382305e-05\\
241	4.71048843576777e-05\\
242	4.71167247187618e-05\\
243	4.71287656713784e-05\\
244	4.71410106234908e-05\\
245	4.71534630422283e-05\\
246	4.71661264549561e-05\\
247	4.71790044504387e-05\\
248	4.71921006799539e-05\\
249	4.7205418858496e-05\\
250	4.72189627659455e-05\\
251	4.72327362483355e-05\\
252	4.72467432190677e-05\\
253	4.72609876602362e-05\\
254	4.7275473623911e-05\\
255	4.72902052335034e-05\\
256	4.73051866851324e-05\\
257	4.73204222490495e-05\\
258	4.73359162710681e-05\\
259	4.73516731740516e-05\\
260	4.73676974594389e-05\\
261	4.73839937087783e-05\\
262	4.74005665853531e-05\\
263	4.74174208357881e-05\\
264	4.7434561291748e-05\\
265	4.74519928716659e-05\\
266	4.74697205825169e-05\\
267	4.74877495216719e-05\\
268	4.75060848788384e-05\\
269	4.7524731938104e-05\\
270	4.75436960801877e-05\\
271	4.75629827848413e-05\\
272	4.75825976331039e-05\\
273	4.76025463083986e-05\\
274	4.76228345948215e-05\\
275	4.76434683760235e-05\\
276	4.76644536527703e-05\\
277	4.76857965380295e-05\\
278	4.77075032589524e-05\\
279	4.77295801595217e-05\\
280	4.77520337032116e-05\\
281	4.77748704757659e-05\\
282	4.77980971880835e-05\\
283	4.78217206792177e-05\\
284	4.78457479194409e-05\\
285	4.78701860134646e-05\\
286	4.78950422037711e-05\\
287	4.79203238740843e-05\\
288	4.79460385529402e-05\\
289	4.7972193917466e-05\\
290	4.79987977972544e-05\\
291	4.80258581784642e-05\\
292	4.80533832080408e-05\\
293	4.80813811981757e-05\\
294	4.81098606309409e-05\\
295	4.81388301631565e-05\\
296	4.81682986314961e-05\\
297	4.81982750578213e-05\\
298	4.82287686548244e-05\\
299	4.82597888319346e-05\\
300	4.82913452015431e-05\\
301	4.8323447585574e-05\\
302	4.83561060223944e-05\\
303	4.83893307740658e-05\\
304	4.8423132333964e-05\\
305	4.84575214346171e-05\\
306	4.84925090557011e-05\\
307	4.85281064320679e-05\\
308	4.85643250620445e-05\\
309	4.86011767177443e-05\\
310	4.86386734621415e-05\\
311	4.8676827678597e-05\\
312	4.87156520978664e-05\\
313	4.87551597564142e-05\\
314	4.8795364022418e-05\\
315	4.88362786247455e-05\\
316	4.88779176705513e-05\\
317	4.89202956642902e-05\\
318	4.89634275283708e-05\\
319	4.9007328625485e-05\\
320	4.90520147829779e-05\\
321	4.90975023193e-05\\
322	4.91438080728845e-05\\
323	4.91909494336969e-05\\
324	4.92389443776877e-05\\
325	4.92878115045817e-05\\
326	4.93375700792812e-05\\
327	4.93882400773528e-05\\
328	4.94398422350384e-05\\
329	4.94923981043484e-05\\
330	4.95459301137594e-05\\
331	4.96004616352609e-05\\
332	4.96560170583851e-05\\
333	4.97126218722017e-05\\
334	4.97703027560492e-05\\
335	4.98290876801908e-05\\
336	4.9889006017507e-05\\
337	4.99500886676085e-05\\
338	5.00123681948524e-05\\
339	5.00758789819625e-05\\
340	5.01406574012291e-05\\
341	5.02067420053495e-05\\
342	5.02741737404943e-05\\
343	5.034299618427e-05\\
344	5.04132558117271e-05\\
345	5.04850022930345e-05\\
346	5.05582888267819e-05\\
347	5.06331725135548e-05\\
348	5.07097147751425e-05\\
349	5.07879818256077e-05\\
350	5.08680452019124e-05\\
351	5.09499823634595e-05\\
352	5.10338773721552e-05\\
353	5.11198216630597e-05\\
354	5.1207914899369e-05\\
355	5.12982658542689e-05\\
356	5.13909932478607e-05\\
357	5.14862270201235e-05\\
358	5.15841102216357e-05\\
359	5.16848002389749e-05\\
360	5.17884705448819e-05\\
361	5.18953126650908e-05\\
362	5.20055383462529e-05\\
363	5.21193817847083e-05\\
364	5.22371013522779e-05\\
365	5.23589791632466e-05\\
366	5.24853177180695e-05\\
367	5.26164927355185e-05\\
368	5.2754106380934e-05\\
369	5.33348951666007e-05\\
370	5.3957204054167e-05\\
371	5.45896062072596e-05\\
372	5.52322612354649e-05\\
373	5.588533110313e-05\\
374	5.65489803489557e-05\\
375	5.72233768696586e-05\\
376	5.79086938147883e-05\\
377	5.86051095135397e-05\\
378	5.93127818377497e-05\\
379	6.00318790577998e-05\\
380	6.07625838000523e-05\\
381	6.15050821419064e-05\\
382	6.22595638379744e-05\\
383	6.30262225781473e-05\\
384	6.38052562815622e-05\\
385	6.4596867430693e-05\\
386	6.54012634503663e-05\\
387	6.62186571366786e-05\\
388	6.70492671414194e-05\\
389	6.78933185177992e-05\\
390	6.87510433338879e-05\\
391	6.96226813603081e-05\\
392	7.05084808391372e-05\\
393	7.14086993412281e-05\\
394	7.23236047190819e-05\\
395	7.32534761625014e-05\\
396	7.41986053638208e-05\\
397	7.51592977986433e-05\\
398	7.6135874125322e-05\\
399	7.71286716975121e-05\\
400	7.81380461538272e-05\\
401	7.91643729526588e-05\\
402	8.02080484589262e-05\\
403	8.12694897303985e-05\\
404	8.23491329950335e-05\\
405	8.3447444861849e-05\\
406	8.45650264339686e-05\\
407	8.57024394423497e-05\\
408	8.68602485026623e-05\\
409	8.80390433137111e-05\\
410	8.92394189516993e-05\\
411	9.04619516920421e-05\\
412	9.17075342590605e-05\\
413	9.29772832542274e-05\\
414	9.42715832514439e-05\\
415	9.5590917679188e-05\\
416	9.69359307017726e-05\\
417	9.83072918494114e-05\\
418	9.9705696704008e-05\\
419	0.000101131867124936\\
420	0.000102586552091485\\
421	0.000104070534828886\\
422	0.000105584656761338\\
423	0.000107129805491077\\
424	0.000108706888637646\\
425	0.000110316857657739\\
426	0.000111960711090713\\
427	0.000113639498139863\\
428	0.00011535432263472\\
429	0.000117106347427119\\
430	0.000118896799282379\\
431	0.000120726974337093\\
432	0.000122598244207171\\
433	0.000124512062844152\\
434	0.000126469974254772\\
435	0.000128473621219041\\
436	0.000130524755165883\\
437	0.000132625247394223\\
438	0.000134777101861739\\
439	0.000136982469809312\\
440	0.000139243666565943\\
441	0.000141563191083511\\
442	0.000143943749479445\\
443	0.000146388286722446\\
444	0.000148900041242269\\
445	0.00015148266725791\\
446	0.000154140434601715\\
447	0.000156876604403537\\
448	0.000159695618446798\\
449	0.000162602583136702\\
450	0.000165603269980838\\
451	0.000168704141583991\\
452	0.000171912461067029\\
453	0.000175236415477924\\
454	0.00017868524931195\\
455	0.000182269412078902\\
456	0.000186000824948488\\
457	0.000189893986741423\\
458	0.000193970883317604\\
459	0.000228095183085844\\
460	0.00026877496219362\\
461	0.000310340054739103\\
462	0.000352813735782594\\
463	0.000396216382073708\\
464	0.000440565637866915\\
465	0.000485877110274124\\
466	0.000532160774069255\\
467	0.000579418257293104\\
468	0.00062738419818233\\
469	0.00067622482309534\\
470	0.000726048322986633\\
471	0.000776839706170607\\
472	0.000828636037132636\\
473	0.00088147678700264\\
474	0.000935403965952333\\
475	0.000990462253691594\\
476	0.00104669910155119\\
477	0.00110416455205536\\
478	0.00116291308479596\\
479	0.00122300222748416\\
480	0.00128449561765596\\
481	0.00134744304775154\\
482	0.00141189995900876\\
483	0.00147792462610287\\
484	0.00154557809802488\\
485	0.00161492391398022\\
486	0.0016860270809891\\
487	0.00175895033040454\\
488	0.00183373919105755\\
489	0.00190806619835934\\
490	0.0019832894053654\\
491	0.00206057427645102\\
492	0.00214005154122694\\
493	0.00222186862771036\\
494	0.00230619258930748\\
495	0.00239321367185647\\
496	0.00248314968594651\\
497	0.00257625145724361\\
498	0.00267280965494731\\
499	0.00277316374541858\\
500	0.0028777121918144\\
501	0.00298693285794693\\
502	0.0031013825506269\\
503	0.00322166732056289\\
504	0.00334849896939119\\
505	0.00348271740882836\\
506	0.00362027029291665\\
507	0.00376105869656856\\
508	0.00388611688136062\\
509	0.0039572418151582\\
510	0.00402944410584187\\
511	0.00410270379462366\\
512	0.00417699316230488\\
513	0.00425227577011228\\
514	0.00432850697474614\\
515	0.00440563912522553\\
516	0.00448365029461639\\
517	0.0045626587564041\\
518	0.00464264237219054\\
519	0.00472351612283466\\
520	0.00480517248048174\\
521	0.00488747531542428\\
522	0.00497025317401618\\
523	0.00505329154666099\\
524	0.00513632313230388\\
525	0.00521901734880742\\
526	0.00530096376322706\\
527	0.00538166427589422\\
528	0.00546050799752127\\
529	0.00553673733990381\\
530	0.00560940948104321\\
531	0.00567736265017387\\
532	0.00574657938646294\\
533	0.00581728614637982\\
534	0.0058895628459596\\
535	0.0059634355660333\\
536	0.00603886694478586\\
537	0.00611577996857281\\
538	0.00619404396421993\\
539	0.006273456223647\\
540	0.0063537180019648\\
541	0.00643455409956856\\
542	0.00651658735225979\\
543	0.00660009219092987\\
544	0.00668464219506546\\
545	0.00676848387754154\\
546	0.00685086218497004\\
547	0.00693155004757897\\
548	0.00701032202772476\\
549	0.00708696541304999\\
550	0.00716129639164414\\
551	0.00723318316095625\\
552	0.00730257843968756\\
553	0.00736952647902213\\
554	0.00743402578181928\\
555	0.00749607700169981\\
556	0.00755586493724629\\
557	0.00761541016178557\\
558	0.00767469271572267\\
559	0.00773370315312266\\
560	0.00779244441339949\\
561	0.00785091703370655\\
562	0.00790914232129952\\
563	0.00796716616600809\\
564	0.00802506207440758\\
565	0.00808293312451273\\
566	0.00814091167253474\\
567	0.00819915577571846\\
568	0.00825784157242284\\
569	0.00831701598922605\\
570	0.0083766845918232\\
571	0.00843685157104802\\
572	0.00849751790386216\\
573	0.00855867710387677\\
574	0.00862031664618866\\
575	0.00868242246674999\\
576	0.0087449779716787\\
577	0.00880796255215469\\
578	0.0088713498714046\\
579	0.00893510609465251\\
580	0.00899918825703781\\
581	0.00906354428152085\\
582	0.00912811788932691\\
583	0.00919285217075625\\
584	0.00925768822972085\\
585	0.00932258151766213\\
586	0.00938749662428032\\
587	0.00945240446296449\\
588	0.00951729025111872\\
589	0.0095821249003219\\
590	0.00964561720054655\\
591	0.0097066206152574\\
592	0.00976430172597694\\
593	0.00981781039518062\\
594	0.00986668609051667\\
595	0.00990919367976287\\
596	0.009946404199063\\
597	0.00997788999445116\\
598	0.010000292044645\\
599	0\\
600	0\\
};
\addplot [color=mycolor11,solid,forget plot]
  table[row sep=crcr]{%
1	2.87074056288534e-05\\
2	2.87074219770342e-05\\
3	2.8707438605019e-05\\
4	2.87074555176098e-05\\
5	2.87074727196844e-05\\
6	2.87074902162143e-05\\
7	2.87075080122498e-05\\
8	2.87075261129244e-05\\
9	2.87075445234721e-05\\
10	2.87075632491999e-05\\
11	2.87075822955249e-05\\
12	2.87076016679376e-05\\
13	2.87076213720374e-05\\
14	2.87076414135115e-05\\
15	2.87076617981448e-05\\
16	2.87076825318254e-05\\
17	2.87077036205368e-05\\
18	2.87077250703705e-05\\
19	2.87077468875226e-05\\
20	2.8707769078286e-05\\
21	2.87077916490701e-05\\
22	2.87078146063905e-05\\
23	2.87078379568757e-05\\
24	2.87078617072681e-05\\
25	2.87078858644229e-05\\
26	2.87079104353101e-05\\
27	2.87079354270323e-05\\
28	2.87079608467935e-05\\
29	2.8707986701937e-05\\
30	2.87080129999283e-05\\
31	2.87080397483458e-05\\
32	2.87080669549237e-05\\
33	2.87080946275071e-05\\
34	2.87081227740844e-05\\
35	2.87081514027762e-05\\
36	2.87081805218467e-05\\
37	2.87082101397009e-05\\
38	2.87082402648758e-05\\
39	2.87082709060712e-05\\
40	2.87083020721246e-05\\
41	2.87083337720337e-05\\
42	2.87083660149329e-05\\
43	2.87083988101306e-05\\
44	2.87084321670868e-05\\
45	2.87084660954185e-05\\
46	2.87085006049105e-05\\
47	2.87085357055212e-05\\
48	2.87085714073691e-05\\
49	2.87086077207499e-05\\
50	2.87086446561366e-05\\
51	2.87086822241715e-05\\
52	2.87087204356876e-05\\
53	2.87087593017029e-05\\
54	2.87087988334197e-05\\
55	2.87088390422278e-05\\
56	2.87088799397196e-05\\
57	2.87089215376791e-05\\
58	2.87089638480933e-05\\
59	2.87090068831497e-05\\
60	2.87090506552488e-05\\
61	2.87090951769986e-05\\
62	2.87091404612286e-05\\
63	2.87091865209722e-05\\
64	2.87092333694985e-05\\
65	2.87092810202971e-05\\
66	2.87093294870827e-05\\
67	2.87093787838186e-05\\
68	2.87094289246864e-05\\
69	2.87094799241193e-05\\
70	2.87095317967987e-05\\
71	2.87095845576489e-05\\
72	2.87096382218437e-05\\
73	2.87096928048299e-05\\
74	2.87097483223065e-05\\
75	2.87098047902417e-05\\
76	2.8709862224873e-05\\
77	2.87099206427135e-05\\
78	2.87099800605614e-05\\
79	2.87100404954949e-05\\
80	2.87101019648874e-05\\
81	2.87101644864034e-05\\
82	2.87102280780061e-05\\
83	2.8710292757966e-05\\
84	2.87103585448645e-05\\
85	2.87104254575978e-05\\
86	2.87104935153779e-05\\
87	2.87105627377484e-05\\
88	2.87106331445835e-05\\
89	2.8710704756096e-05\\
90	2.87107775928283e-05\\
91	2.87108516756889e-05\\
92	2.87109270259289e-05\\
93	2.87110036651669e-05\\
94	2.87110816153827e-05\\
95	2.87111608989232e-05\\
96	2.87112415385331e-05\\
97	2.87113235573141e-05\\
98	2.87114069787841e-05\\
99	2.87114918268461e-05\\
100	2.87115781258095e-05\\
101	2.87116659003968e-05\\
102	2.87117551757434e-05\\
103	2.87118459774197e-05\\
104	2.87119383314177e-05\\
105	2.87120322641772e-05\\
106	2.871212780258e-05\\
107	2.87122249739635e-05\\
108	2.87123238061285e-05\\
109	2.87124243273415e-05\\
110	2.8712526566356e-05\\
111	2.87126305524002e-05\\
112	2.87127363151966e-05\\
113	2.8712843884981e-05\\
114	2.87129532924846e-05\\
115	2.87130645689699e-05\\
116	2.87131777462208e-05\\
117	2.87132928565585e-05\\
118	2.87134099328484e-05\\
119	2.87135290085129e-05\\
120	2.8713650117539e-05\\
121	2.87137732944825e-05\\
122	2.87138985744863e-05\\
123	2.87140259932864e-05\\
124	2.87141555872151e-05\\
125	2.87142873932249e-05\\
126	2.87144214488846e-05\\
127	2.87145577924034e-05\\
128	2.87146964626287e-05\\
129	2.87148374990611e-05\\
130	2.87149809418793e-05\\
131	2.87151268319188e-05\\
132	2.87152752107225e-05\\
133	2.87154261205211e-05\\
134	2.87155796042571e-05\\
135	2.87157357055965e-05\\
136	2.87158944689389e-05\\
137	2.87160559394325e-05\\
138	2.87162201629782e-05\\
139	2.87163871862413e-05\\
140	2.87165570566855e-05\\
141	2.87167298225847e-05\\
142	2.87169055330413e-05\\
143	2.87170842379845e-05\\
144	2.87172659881683e-05\\
145	2.87174508351859e-05\\
146	2.87176388315083e-05\\
147	2.87178300304862e-05\\
148	2.87180244863631e-05\\
149	2.87182222542978e-05\\
150	2.87184233903808e-05\\
151	2.87186279516347e-05\\
152	2.87188359960527e-05\\
153	2.8719047582593e-05\\
154	2.87192627712097e-05\\
155	2.87194816228568e-05\\
156	2.87197041995304e-05\\
157	2.87199305642401e-05\\
158	2.87201607810818e-05\\
159	2.87203949152054e-05\\
160	2.87206330328721e-05\\
161	2.87208752014378e-05\\
162	2.87211214894095e-05\\
163	2.87213719664297e-05\\
164	2.87216267033141e-05\\
165	2.87218857720663e-05\\
166	2.87221492458993e-05\\
167	2.87224171992527e-05\\
168	2.87226897078212e-05\\
169	2.87229668485622e-05\\
170	2.87232486997237e-05\\
171	2.87235353408754e-05\\
172	2.87238268529116e-05\\
173	2.87241233180891e-05\\
174	2.87244248200478e-05\\
175	2.87247314438186e-05\\
176	2.87250432758698e-05\\
177	2.87253604041133e-05\\
178	2.8725682917941e-05\\
179	2.8726010908238e-05\\
180	2.8726344467414e-05\\
181	2.87266836894331e-05\\
182	2.8727028669831e-05\\
183	2.87273795057451e-05\\
184	2.8727736295944e-05\\
185	2.87280991408507e-05\\
186	2.87284681425684e-05\\
187	2.87288434049218e-05\\
188	2.87292250334677e-05\\
189	2.87296131355383e-05\\
190	2.87300078202593e-05\\
191	2.87304091985894e-05\\
192	2.87308173833562e-05\\
193	2.87312324892674e-05\\
194	2.87316546329607e-05\\
195	2.87320839330218e-05\\
196	2.87325205100417e-05\\
197	2.87329644866219e-05\\
198	2.87334159874207e-05\\
199	2.87338751391964e-05\\
200	2.87343420708235e-05\\
201	2.87348169133468e-05\\
202	2.87352998000084e-05\\
203	2.87357908662871e-05\\
204	2.87362902499362e-05\\
205	2.87367980910208e-05\\
206	2.87373145319596e-05\\
207	2.87378397175658e-05\\
208	2.87383737950801e-05\\
209	2.87389169142191e-05\\
210	2.87394692272215e-05\\
211	2.87400308888777e-05\\
212	2.87406020565832e-05\\
213	2.87411828903803e-05\\
214	2.87417735530064e-05\\
215	2.87423742099366e-05\\
216	2.87429850294291e-05\\
217	2.87436061825764e-05\\
218	2.87442378433557e-05\\
219	2.87448801886679e-05\\
220	2.87455333984064e-05\\
221	2.87461976554881e-05\\
222	2.87468731459194e-05\\
223	2.87475600588475e-05\\
224	2.8748258586608e-05\\
225	2.87489689247892e-05\\
226	2.87496912722746e-05\\
227	2.8750425831325e-05\\
228	2.87511728076093e-05\\
229	2.87519324102845e-05\\
230	2.87527048520416e-05\\
231	2.8753490349177e-05\\
232	2.87542891216552e-05\\
233	2.87551013931665e-05\\
234	2.87559273912032e-05\\
235	2.87567673471142e-05\\
236	2.87576214961827e-05\\
237	2.87584900776872e-05\\
238	2.87593733349814e-05\\
239	2.87602715155676e-05\\
240	2.87611848711529e-05\\
241	2.87621136577445e-05\\
242	2.87630581357161e-05\\
243	2.87640185698889e-05\\
244	2.8764995229616e-05\\
245	2.87659883888463e-05\\
246	2.87669983262339e-05\\
247	2.87680253252047e-05\\
248	2.87690696740387e-05\\
249	2.87701316659861e-05\\
250	2.87712115993204e-05\\
251	2.87723097774585e-05\\
252	2.87734265090484e-05\\
253	2.87745621080573e-05\\
254	2.87757168938728e-05\\
255	2.87768911914122e-05\\
256	2.87780853312216e-05\\
257	2.8779299649569e-05\\
258	2.87805344885712e-05\\
259	2.87817901962847e-05\\
260	2.87830671268389e-05\\
261	2.87843656405262e-05\\
262	2.87856861039463e-05\\
263	2.87870288901108e-05\\
264	2.87883943785704e-05\\
265	2.87897829555578e-05\\
266	2.87911950141216e-05\\
267	2.87926309542748e-05\\
268	2.87940911831882e-05\\
269	2.87955761153534e-05\\
270	2.87970861727768e-05\\
271	2.8798621785011e-05\\
272	2.88001833889952e-05\\
273	2.88017714286501e-05\\
274	2.8803386355146e-05\\
275	2.88050286289825e-05\\
276	2.88066987190239e-05\\
277	2.88083971026034e-05\\
278	2.88101242656961e-05\\
279	2.88118807031026e-05\\
280	2.88136669186484e-05\\
281	2.88154834253669e-05\\
282	2.88173307457072e-05\\
283	2.88192094117412e-05\\
284	2.88211199653794e-05\\
285	2.8823062958591e-05\\
286	2.88250389536283e-05\\
287	2.8827048523281e-05\\
288	2.88290922510929e-05\\
289	2.88311707316444e-05\\
290	2.88332845708011e-05\\
291	2.88354343859876e-05\\
292	2.88376208064754e-05\\
293	2.88398444736696e-05\\
294	2.88421060414301e-05\\
295	2.88444061763612e-05\\
296	2.88467455581792e-05\\
297	2.88491248800287e-05\\
298	2.88515448488514e-05\\
299	2.88540061857641e-05\\
300	2.88565096264286e-05\\
301	2.88590559214596e-05\\
302	2.88616458368236e-05\\
303	2.88642801541959e-05\\
304	2.88669596713435e-05\\
305	2.88696852024063e-05\\
306	2.88724575782161e-05\\
307	2.88752776467553e-05\\
308	2.88781462742516e-05\\
309	2.88810643474518e-05\\
310	2.88840327767239e-05\\
311	2.88870524963344e-05\\
312	2.88901244576064e-05\\
313	2.88932496319469e-05\\
314	2.88964290135874e-05\\
315	2.88996636204524e-05\\
316	2.89029544951364e-05\\
317	2.89063027058867e-05\\
318	2.89097093477215e-05\\
319	2.8913175543578e-05\\
320	2.89167024455904e-05\\
321	2.89202912364288e-05\\
322	2.89239431307811e-05\\
323	2.89276593769228e-05\\
324	2.89314412584406e-05\\
325	2.89352900960943e-05\\
326	2.8939207249852e-05\\
327	2.89431941210691e-05\\
328	2.89472521549184e-05\\
329	2.89513828430158e-05\\
330	2.89555877262789e-05\\
331	2.89598683980968e-05\\
332	2.89642265077754e-05\\
333	2.89686637643585e-05\\
334	2.89731819407897e-05\\
335	2.89777828785489e-05\\
336	2.89824684927481e-05\\
337	2.89872407777614e-05\\
338	2.89921018134589e-05\\
339	2.8997053772138e-05\\
340	2.90020989261909e-05\\
341	2.9007239656636e-05\\
342	2.90124784626291e-05\\
343	2.90178179720329e-05\\
344	2.90232609532281e-05\\
345	2.90288103283296e-05\\
346	2.90344691880262e-05\\
347	2.90402408082604e-05\\
348	2.90461286691986e-05\\
349	2.90521364769399e-05\\
350	2.90582681884599e-05\\
351	2.90645280401054e-05\\
352	2.90709205783075e-05\\
353	2.90774506878393e-05\\
354	2.90841236102916e-05\\
355	2.90909449613259e-05\\
356	2.90979208100995e-05\\
357	2.91050578025798e-05\\
358	2.91123631980276e-05\\
359	2.91198450062145e-05\\
360	2.91275122628458e-05\\
361	2.91353756940605e-05\\
362	2.9143449519973e-05\\
363	2.91517566766702e-05\\
364	2.91603444727928e-05\\
365	2.91693323614963e-05\\
366	2.91790572625571e-05\\
367	2.91904883455599e-05\\
368	2.92059904748458e-05\\
369	2.92220118418757e-05\\
370	2.92382851160593e-05\\
371	2.92548141350258e-05\\
372	2.92716028900053e-05\\
373	2.92886556623828e-05\\
374	2.93059772348849e-05\\
375	2.93235729474886e-05\\
376	2.93414476655059e-05\\
377	2.93596027275639e-05\\
378	2.93780407253037e-05\\
379	2.93967659119906e-05\\
380	2.94157826108006e-05\\
381	2.943509521892e-05\\
382	2.94547082122986e-05\\
383	2.94746261510439e-05\\
384	2.94948536856192e-05\\
385	2.95153955639028e-05\\
386	2.95362566391953e-05\\
387	2.95574418793126e-05\\
388	2.95789563768745e-05\\
389	2.96008053608892e-05\\
390	2.96229942097429e-05\\
391	2.96455284656924e-05\\
392	2.96684138507856e-05\\
393	2.96916562839949e-05\\
394	2.97152618986561e-05\\
395	2.97392370579873e-05\\
396	2.97635883633745e-05\\
397	2.97883226435484e-05\\
398	2.98134469000132e-05\\
399	2.983896816252e-05\\
400	2.98648931854672e-05\\
401	2.98912279412757e-05\\
402	2.9917977129054e-05\\
403	2.99451449001304e-05\\
404	2.99727400478062e-05\\
405	3.00007863959452e-05\\
406	3.00292971289041e-05\\
407	3.00582798077221e-05\\
408	3.00877378599853e-05\\
409	3.01176675906268e-05\\
410	3.01480642028151e-05\\
411	3.01789779503984e-05\\
412	3.02104853764506e-05\\
413	3.02425629355838e-05\\
414	3.02752030331295e-05\\
415	3.03084177303424e-05\\
416	3.03422192657478e-05\\
417	3.03766199085939e-05\\
418	3.04116318696233e-05\\
419	3.04472676443895e-05\\
420	3.04835414641953e-05\\
421	3.0520471431376e-05\\
422	3.05580764393456e-05\\
423	3.05963730460801e-05\\
424	3.06353785031662e-05\\
425	3.0675110802949e-05\\
426	3.07155887302112e-05\\
427	3.0756831918944e-05\\
428	3.07988609148731e-05\\
429	3.08416972445709e-05\\
430	3.08853634919592e-05\\
431	3.09298833833985e-05\\
432	3.09752818826228e-05\\
433	3.1021585297479e-05\\
434	3.1068821401146e-05\\
435	3.11170195730472e-05\\
436	3.11662109697854e-05\\
437	3.12164287501284e-05\\
438	3.12677084112617e-05\\
439	3.1320088374597e-05\\
440	3.13736111458136e-05\\
441	3.14283257581362e-05\\
442	3.14842927968935e-05\\
443	3.15415932268637e-05\\
444	3.16003361893756e-05\\
445	3.16606287197654e-05\\
446	3.17223983966092e-05\\
447	3.17856802878807e-05\\
448	3.18505393092603e-05\\
449	3.19170570309623e-05\\
450	3.19853231436586e-05\\
451	3.20554371384342e-05\\
452	3.21275132520946e-05\\
453	3.22016981889936e-05\\
454	3.22782337643279e-05\\
455	3.23576621816601e-05\\
456	3.24414281777169e-05\\
457	3.2533347786978e-05\\
458	3.26421118733469e-05\\
459	3.27552039992171e-05\\
460	3.28698685584011e-05\\
461	3.29864062298302e-05\\
462	3.31050456097429e-05\\
463	3.32259930698725e-05\\
464	3.33497356061983e-05\\
465	3.34764470236495e-05\\
466	3.36068489184736e-05\\
467	3.37469988655559e-05\\
468	3.41714855216607e-05\\
469	3.47180892944729e-05\\
470	3.52757507581536e-05\\
471	3.58451339388148e-05\\
472	3.64269757434914e-05\\
473	3.70220954982187e-05\\
474	3.76314063928785e-05\\
475	3.82559312065372e-05\\
476	3.88968456767497e-05\\
477	3.95557883791894e-05\\
478	4.02206634430905e-05\\
479	4.09064513907733e-05\\
480	4.16150064117664e-05\\
481	4.23479713284909e-05\\
482	4.31071548316835e-05\\
483	4.38945480553534e-05\\
484	4.47123580236007e-05\\
485	4.55631273707271e-05\\
486	4.6450251439557e-05\\
487	4.73802971189456e-05\\
488	4.83744473987448e-05\\
489	5.18255247195229e-05\\
490	5.64432043508488e-05\\
491	6.11757869814177e-05\\
492	6.60287180328892e-05\\
493	7.10079443662274e-05\\
494	7.61199865734989e-05\\
495	8.13720243861874e-05\\
496	8.67719896607228e-05\\
497	9.23286159209255e-05\\
498	9.80514008825647e-05\\
499	0.000103950459488896\\
500	0.000110039573939583\\
501	0.000116338196138091\\
502	0.000122866088819676\\
503	0.000129603148917134\\
504	0.000136554914834019\\
505	0.000143792738973412\\
506	0.000151348309712374\\
507	0.000159250000101562\\
508	0.000186465740107389\\
509	0.000270913190465583\\
510	0.000357339833777734\\
511	0.000445850217445032\\
512	0.000536559442871119\\
513	0.000629595116395348\\
514	0.000725100692200193\\
515	0.000823235332269393\\
516	0.00092418800105013\\
517	0.00102815991107104\\
518	0.00113536914372073\\
519	0.00124605849399606\\
520	0.00136049122063145\\
521	0.00147897623912208\\
522	0.0016018593706772\\
523	0.00172952379390597\\
524	0.00186241913462318\\
525	0.00200099235793762\\
526	0.00214576766890253\\
527	0.00229735402594387\\
528	0.00245567253882486\\
529	0.00262203513585597\\
530	0.00279776832073401\\
531	0.0029840588322844\\
532	0.00317503465125662\\
533	0.00336753682105394\\
534	0.00356080563091648\\
535	0.00375884896364931\\
536	0.00396189643029828\\
537	0.00417020157699044\\
538	0.00438404649089458\\
539	0.00460374472348798\\
540	0.00482962825192461\\
541	0.00498742931533325\\
542	0.0051011079388929\\
543	0.00521618695727248\\
544	0.00533272468242733\\
545	0.00545049487325927\\
546	0.00556921337530202\\
547	0.00568850362809646\\
548	0.00580786947567521\\
549	0.00592666237003143\\
550	0.00604404005502914\\
551	0.00615891412564674\\
552	0.00626988461770694\\
553	0.00637687442962497\\
554	0.00648639688360578\\
555	0.00659810957107587\\
556	0.00671146170833344\\
557	0.00682409800646433\\
558	0.00693572396512232\\
559	0.00704601191043983\\
560	0.00715459986066793\\
561	0.00726109172587377\\
562	0.00736505911124072\\
563	0.00746604653227905\\
564	0.00756358087956845\\
565	0.00765718855990147\\
566	0.0077464333920976\\
567	0.00783089357803292\\
568	0.00791059914192856\\
569	0.00798962772058366\\
570	0.00806803480922161\\
571	0.00814585569413195\\
572	0.00822322735200265\\
573	0.00830037097742555\\
574	0.00837736890349071\\
575	0.00845418668081767\\
576	0.00853079440325115\\
577	0.00860717799344965\\
578	0.00868333907131728\\
579	0.00875929228624666\\
580	0.00883505938019689\\
581	0.00891063794895334\\
582	0.00898592552194447\\
583	0.00906078197573007\\
584	0.00913506393062985\\
585	0.00920862685199412\\
586	0.00928132795734763\\
587	0.00935303006810276\\
588	0.00942360690540113\\
589	0.00949294844638006\\
590	0.00956097836809741\\
591	0.00962766009652905\\
592	0.0096930024817262\\
593	0.00975706769624642\\
594	0.00981997518922992\\
595	0.00988192166735141\\
596	0.00993785230806569\\
597	0.00997762486405646\\
598	0.010000292044645\\
599	0\\
600	0\\
};
\addplot [color=mycolor12,solid,forget plot]
  table[row sep=crcr]{%
1	4.34936132690398e-07\\
2	4.34936809193483e-07\\
3	4.34937497278928e-07\\
4	4.34938197143251e-07\\
5	4.34938908987252e-07\\
6	4.34939633019321e-07\\
7	4.34940369445737e-07\\
8	4.34941118477938e-07\\
9	4.34941880336154e-07\\
10	4.34942655238677e-07\\
11	4.34943443407577e-07\\
12	4.3494424507319e-07\\
13	4.34945060464087e-07\\
14	4.34945889817803e-07\\
15	4.34946733373395e-07\\
16	4.34947591374897e-07\\
17	4.34948464068387e-07\\
18	4.34949351704745e-07\\
19	4.34950254543989e-07\\
20	4.34951172843841e-07\\
21	4.34952106870986e-07\\
22	4.34953056895178e-07\\
23	4.34954023186657e-07\\
24	4.34955006031013e-07\\
25	4.34956005704797e-07\\
26	4.34957022502341e-07\\
27	4.34958056714123e-07\\
28	4.34959108639579e-07\\
29	4.34960178582937e-07\\
30	4.34961266850286e-07\\
31	4.34962373756493e-07\\
32	4.34963499624852e-07\\
33	4.3496464477342e-07\\
34	4.34965809537505e-07\\
35	4.34966994252714e-07\\
36	4.34968199259269e-07\\
37	4.34969424904786e-07\\
38	4.34970671543397e-07\\
39	4.34971939534548e-07\\
40	4.34973229244549e-07\\
41	4.34974541045021e-07\\
42	4.34975875314098e-07\\
43	4.34977232435398e-07\\
44	4.34978612800955e-07\\
45	4.34980016810016e-07\\
46	4.34981444864525e-07\\
47	4.34982897379353e-07\\
48	4.34984374772243e-07\\
49	4.34985877466414e-07\\
50	4.34987405898337e-07\\
51	4.34988960505118e-07\\
52	4.34990541739016e-07\\
53	4.34992150052223e-07\\
54	4.34993785907593e-07\\
55	4.34995449777764e-07\\
56	4.34997142144305e-07\\
57	4.3499886349079e-07\\
58	4.35000614313864e-07\\
59	4.35002395115809e-07\\
60	4.35004206414401e-07\\
61	4.35006048726993e-07\\
62	4.35007922585395e-07\\
63	4.35009828529289e-07\\
64	4.35011767105724e-07\\
65	4.35013738873596e-07\\
66	4.35015744402102e-07\\
67	4.35017784262257e-07\\
68	4.35019859047303e-07\\
69	4.35021969349018e-07\\
70	4.35024115777591e-07\\
71	4.35026298947283e-07\\
72	4.35028519489035e-07\\
73	4.35030778037682e-07\\
74	4.35033075244735e-07\\
75	4.35035411769056e-07\\
76	4.35037788284799e-07\\
77	4.35040205472766e-07\\
78	4.35042664024561e-07\\
79	4.35045164650362e-07\\
80	4.35047708065609e-07\\
81	4.35050295002757e-07\\
82	4.35052926203153e-07\\
83	4.35055602421873e-07\\
84	4.35058324427544e-07\\
85	4.3506109300079e-07\\
86	4.3506390893648e-07\\
87	4.35066773040953e-07\\
88	4.35069686137898e-07\\
89	4.3507264906006e-07\\
90	4.35075662658212e-07\\
91	4.35078727795986e-07\\
92	4.35081845350373e-07\\
93	4.35085016219171e-07\\
94	4.35088241305231e-07\\
95	4.35091521535492e-07\\
96	4.35094857850591e-07\\
97	4.35098251201583e-07\\
98	4.3510170256187e-07\\
99	4.35105212923911e-07\\
100	4.35108783283132e-07\\
101	4.351124146682e-07\\
102	4.35116108116108e-07\\
103	4.35119864682721e-07\\
104	4.3512368544278e-07\\
105	4.35127571486957e-07\\
106	4.35131523925825e-07\\
107	4.35135543891247e-07\\
108	4.35139632531871e-07\\
109	4.35143791013824e-07\\
110	4.35148020525372e-07\\
111	4.3515232227381e-07\\
112	4.35156697488917e-07\\
113	4.35161147417243e-07\\
114	4.35165673328162e-07\\
115	4.35170276514907e-07\\
116	4.35174958289548e-07\\
117	4.35179719990259e-07\\
118	4.35184562968507e-07\\
119	4.35189488609126e-07\\
120	4.35194498313168e-07\\
121	4.35199593511938e-07\\
122	4.35204775653304e-07\\
123	4.35210046214503e-07\\
124	4.3521540669677e-07\\
125	4.35220858624137e-07\\
126	4.35226403551018e-07\\
127	4.35232043052886e-07\\
128	4.35237778735943e-07\\
129	4.35243612230372e-07\\
130	4.35249545194144e-07\\
131	4.35255579316124e-07\\
132	4.35261716310876e-07\\
133	4.3526795792039e-07\\
134	4.35274305919615e-07\\
135	4.35280762109189e-07\\
136	4.35287328329096e-07\\
137	4.35294006434969e-07\\
138	4.35300798329048e-07\\
139	4.35307705933702e-07\\
140	4.35314731213234e-07\\
141	4.35321876155703e-07\\
142	4.35329142791769e-07\\
143	4.35336533180348e-07\\
144	4.35344049414302e-07\\
145	4.35351693624241e-07\\
146	4.35359467977315e-07\\
147	4.3536737467287e-07\\
148	4.35375415953705e-07\\
149	4.3538359409498e-07\\
150	4.35391911409577e-07\\
151	4.35400370252256e-07\\
152	4.35408973016518e-07\\
153	4.35417722137904e-07\\
154	4.35426620087401e-07\\
155	4.35435669381134e-07\\
156	4.35444872576374e-07\\
157	4.35454232275347e-07\\
158	4.35463751121059e-07\\
159	4.35473431803189e-07\\
160	4.35483277054617e-07\\
161	4.35493289654351e-07\\
162	4.35503472429955e-07\\
163	4.35513828253031e-07\\
164	4.35524360045623e-07\\
165	4.35535070777784e-07\\
166	4.35545963471205e-07\\
167	4.35557041193836e-07\\
168	4.35568307070617e-07\\
169	4.35579764274289e-07\\
170	4.35591416033009e-07\\
171	4.35603265628087e-07\\
172	4.35615316395709e-07\\
173	4.3562757173091e-07\\
174	4.35640035078223e-07\\
175	4.35652709949485e-07\\
176	4.35665599906192e-07\\
177	4.35678708575397e-07\\
178	4.35692039642616e-07\\
179	4.35705596855099e-07\\
180	4.35719384024077e-07\\
181	4.35733405023703e-07\\
182	4.35747663792794e-07\\
183	4.35762164332894e-07\\
184	4.35776910721599e-07\\
185	4.35791907093346e-07\\
186	4.35807157659635e-07\\
187	4.35822666701067e-07\\
188	4.35838438566988e-07\\
189	4.35854477678761e-07\\
190	4.35870788540484e-07\\
191	4.35887375721681e-07\\
192	4.3590424387251e-07\\
193	4.35921397718915e-07\\
194	4.35938842069352e-07\\
195	4.35956581809419e-07\\
196	4.35974621911367e-07\\
197	4.35992967422306e-07\\
198	4.36011623482897e-07\\
199	4.3603059531262e-07\\
200	4.36049888221872e-07\\
201	4.36069507612482e-07\\
202	4.36089458969045e-07\\
203	4.36109747875333e-07\\
204	4.36130380006508e-07\\
205	4.36151361129285e-07\\
206	4.36172697114527e-07\\
207	4.36194393921511e-07\\
208	4.3621645761762e-07\\
209	4.36238894371234e-07\\
210	4.36261710447751e-07\\
211	4.36284912224091e-07\\
212	4.36308506181932e-07\\
213	4.36332498913064e-07\\
214	4.36356897116083e-07\\
215	4.3638170760641e-07\\
216	4.36406937311076e-07\\
217	4.36432593274428e-07\\
218	4.36458682660864e-07\\
219	4.36485212751887e-07\\
220	4.36512190952827e-07\\
221	4.36539624796284e-07\\
222	4.36567521936743e-07\\
223	4.36595890161637e-07\\
224	4.36624737388367e-07\\
225	4.36654071665517e-07\\
226	4.36683901182853e-07\\
227	4.36714234260227e-07\\
228	4.36745079367812e-07\\
229	4.36776445105821e-07\\
230	4.36808340234951e-07\\
231	4.36840773651435e-07\\
232	4.36873754409861e-07\\
233	4.36907291716056e-07\\
234	4.36941394925503e-07\\
235	4.36976073562528e-07\\
236	4.37011337307279e-07\\
237	4.37047196003859e-07\\
238	4.37083659664784e-07\\
239	4.37120738474413e-07\\
240	4.37158442787372e-07\\
241	4.37196783133892e-07\\
242	4.37235770229807e-07\\
243	4.37275414967875e-07\\
244	4.37315728427107e-07\\
245	4.37356721880501e-07\\
246	4.37398406788451e-07\\
247	4.37440794810819e-07\\
248	4.374838978038e-07\\
249	4.37527727835106e-07\\
250	4.37572297171646e-07\\
251	4.37617618291956e-07\\
252	4.37663703894814e-07\\
253	4.37710566894014e-07\\
254	4.37758220430087e-07\\
255	4.37806677865781e-07\\
256	4.37855952795697e-07\\
257	4.37906059056981e-07\\
258	4.37957010719085e-07\\
259	4.38008822104305e-07\\
260	4.38061507775458e-07\\
261	4.38115082560407e-07\\
262	4.38169561534547e-07\\
263	4.38224960050163e-07\\
264	4.38281293717537e-07\\
265	4.38338578430848e-07\\
266	4.38396830362579e-07\\
267	4.3845606596573e-07\\
268	4.3851630199262e-07\\
269	4.38577555491015e-07\\
270	4.38639843808431e-07\\
271	4.38703184608483e-07\\
272	4.38767595867724e-07\\
273	4.38833095886327e-07\\
274	4.38899703295951e-07\\
275	4.38967437064215e-07\\
276	4.39036316501856e-07\\
277	4.39106361278283e-07\\
278	4.39177591412809e-07\\
279	4.39250027300402e-07\\
280	4.39323689707259e-07\\
281	4.39398599786122e-07\\
282	4.39474779085917e-07\\
283	4.39552249555797e-07\\
284	4.39631033554414e-07\\
285	4.39711153868686e-07\\
286	4.39792633715257e-07\\
287	4.39875496753221e-07\\
288	4.39959767098527e-07\\
289	4.40045469331684e-07\\
290	4.40132628516862e-07\\
291	4.40221270199159e-07\\
292	4.40311420438062e-07\\
293	4.40403105805777e-07\\
294	4.40496353412509e-07\\
295	4.40591190912235e-07\\
296	4.40687646524333e-07\\
297	4.40785749051628e-07\\
298	4.40885527893176e-07\\
299	4.40987013069395e-07\\
300	4.4109023523483e-07\\
301	4.41195225703087e-07\\
302	4.41302016470131e-07\\
303	4.41410640237651e-07\\
304	4.41521130432157e-07\\
305	4.41633521237861e-07\\
306	4.41747847627492e-07\\
307	4.41864145380769e-07\\
308	4.41982451133916e-07\\
309	4.42102802396652e-07\\
310	4.4222523760222e-07\\
311	4.42349796141637e-07\\
312	4.42476518405553e-07\\
313	4.42605445829455e-07\\
314	4.42736620952122e-07\\
315	4.42870087460403e-07\\
316	4.43005890250698e-07\\
317	4.4314407549477e-07\\
318	4.432846907065e-07\\
319	4.43427784814887e-07\\
320	4.43573408251904e-07\\
321	4.43721613033377e-07\\
322	4.43872452867464e-07\\
323	4.4402598324785e-07\\
324	4.44182261581356e-07\\
325	4.44341347315447e-07\\
326	4.44503302074274e-07\\
327	4.44668189813898e-07\\
328	4.44836077005044e-07\\
329	4.45007032804416e-07\\
330	4.45181129277089e-07\\
331	4.45358441622988e-07\\
332	4.45539048424822e-07\\
333	4.45723031947661e-07\\
334	4.45910478435786e-07\\
335	4.46101478465686e-07\\
336	4.46296127338126e-07\\
337	4.46494525500603e-07\\
338	4.46696779035211e-07\\
339	4.46903000178788e-07\\
340	4.47113307935286e-07\\
341	4.47327828717876e-07\\
342	4.47546697106324e-07\\
343	4.47770056669999e-07\\
344	4.47998060887322e-07\\
345	4.48230874191427e-07\\
346	4.4846867312327e-07\\
347	4.48711647631946e-07\\
348	4.48960002536266e-07\\
349	4.49213959164141e-07\\
350	4.49473757180344e-07\\
351	4.49739656669636e-07\\
352	4.50011940460597e-07\\
353	4.50290916782777e-07\\
354	4.50576922303904e-07\\
355	4.50870325745966e-07\\
356	4.51171532379942e-07\\
357	4.51480990167528e-07\\
358	4.51799199628439e-07\\
359	4.5212673234318e-07\\
360	4.52464270487952e-07\\
361	4.52812697191808e-07\\
362	4.53173306591444e-07\\
363	4.53548280021433e-07\\
364	4.53941692239194e-07\\
365	4.54361331329531e-07\\
366	4.54820730812272e-07\\
367	4.55334917625691e-07\\
368	4.5586024961437e-07\\
369	4.56393569206599e-07\\
370	4.56934984276364e-07\\
371	4.57484603740934e-07\\
372	4.58042537619154e-07\\
373	4.58608896809749e-07\\
374	4.59183792054823e-07\\
375	4.59767331818493e-07\\
376	4.60359621601097e-07\\
377	4.60960772392783e-07\\
378	4.61570899231697e-07\\
379	4.62190117959462e-07\\
380	4.62818545260268e-07\\
381	4.63456298693708e-07\\
382	4.6410349676993e-07\\
383	4.64760259023138e-07\\
384	4.65426706120389e-07\\
385	4.66102959973883e-07\\
386	4.66789143907746e-07\\
387	4.67485382833585e-07\\
388	4.68191803483914e-07\\
389	4.6890853465892e-07\\
390	4.69635707550573e-07\\
391	4.70373456090364e-07\\
392	4.71121917374378e-07\\
393	4.71881232125928e-07\\
394	4.72651545220905e-07\\
395	4.73433006233936e-07\\
396	4.74225769959324e-07\\
397	4.75029996811312e-07\\
398	4.75845852856791e-07\\
399	4.76673509100651e-07\\
400	4.77513139254203e-07\\
401	4.78364914254846e-07\\
402	4.79228988678671e-07\\
403	4.80105463284348e-07\\
404	4.80994279056869e-07\\
405	4.81894963224822e-07\\
406	4.82806276198916e-07\\
407	4.83725750066658e-07\\
408	4.84649734057819e-07\\
409	4.85575883494642e-07\\
410	4.86509813026681e-07\\
411	4.87459809787173e-07\\
412	4.88426119947507e-07\\
413	4.89409045121629e-07\\
414	4.90408939179918e-07\\
415	4.91426167774683e-07\\
416	4.92461108933913e-07\\
417	4.93514153760769e-07\\
418	4.9458570726104e-07\\
419	4.95676189302587e-07\\
420	4.96786035413769e-07\\
421	4.97915696913407e-07\\
422	4.99065641521776e-07\\
423	5.00236354831127e-07\\
424	5.01428341285755e-07\\
425	5.02642125220279e-07\\
426	5.03878251956168e-07\\
427	5.05137288982068e-07\\
428	5.06419827203559e-07\\
429	5.07726482271197e-07\\
430	5.09057896007294e-07\\
431	5.10414737898169e-07\\
432	5.1179770672642e-07\\
433	5.13207532280221e-07\\
434	5.14644977210104e-07\\
435	5.16110839052974e-07\\
436	5.17605952488326e-07\\
437	5.19131191943217e-07\\
438	5.20687474859379e-07\\
439	5.22275765951502e-07\\
440	5.23897082869642e-07\\
441	5.25552502653173e-07\\
442	5.27243164114881e-07\\
443	5.2897025031219e-07\\
444	5.30734924846659e-07\\
445	5.32538268289753e-07\\
446	5.34381629665944e-07\\
447	5.36266440973359e-07\\
448	5.38194216928926e-07\\
449	5.4016657023308e-07\\
450	5.42185250479107e-07\\
451	5.44252239932341e-07\\
452	5.46369993147793e-07\\
453	5.48542023433573e-07\\
454	5.50774254206475e-07\\
455	5.53077726880314e-07\\
456	5.55472044605509e-07\\
457	5.57978676489192e-07\\
458	5.60537195289438e-07\\
459	5.6313925249674e-07\\
460	5.65794875088331e-07\\
461	5.68525814775929e-07\\
462	5.71376082141709e-07\\
463	5.74413697553009e-07\\
464	5.77673303769232e-07\\
465	5.81332842023018e-07\\
466	5.86155485931426e-07\\
467	5.93913121381586e-07\\
468	6.02796678308234e-07\\
469	6.11827278019675e-07\\
470	6.21011842535799e-07\\
471	6.30356316779779e-07\\
472	6.39863208224527e-07\\
473	6.49526178899239e-07\\
474	6.59319845506786e-07\\
475	6.69186724437135e-07\\
476	6.79045612325569e-07\\
477	6.88913043163912e-07\\
478	7.1191710836285e-07\\
479	7.36251307929028e-07\\
480	7.61283855274156e-07\\
481	7.8706759957787e-07\\
482	8.13671492430865e-07\\
483	8.41200630273418e-07\\
484	8.69849572101272e-07\\
485	9.00045155930217e-07\\
486	9.32825888050371e-07\\
487	9.70793912198952e-07\\
488	1.0197698550852e-06\\
489	1.07349631091925e-06\\
490	1.12819026207342e-06\\
491	1.1838815485207e-06\\
492	1.24059958651208e-06\\
493	1.29836989074042e-06\\
494	1.35720566489286e-06\\
495	1.41708990526517e-06\\
496	1.47794345427834e-06\\
497	1.53959193930114e-06\\
498	1.60183097725021e-06\\
499	1.66489561723599e-06\\
500	1.72946768205313e-06\\
501	1.79565925023842e-06\\
502	1.86364139327674e-06\\
503	1.97547071890742e-06\\
504	2.14963390653748e-06\\
505	2.33186913378658e-06\\
506	2.52587543034996e-06\\
507	2.74397810153313e-06\\
508	2.99255659583446e-06\\
509	3.24410594676526e-06\\
510	3.49866347970021e-06\\
511	3.7562276102568e-06\\
512	4.01675334510043e-06\\
513	4.28015214097894e-06\\
514	4.54550038981188e-06\\
515	4.81392577946725e-06\\
516	5.08616511233571e-06\\
517	5.36234433405714e-06\\
518	5.64253156912134e-06\\
519	5.92666385147519e-06\\
520	6.22197699225318e-06\\
521	6.52325092110364e-06\\
522	6.82768405190497e-06\\
523	7.14319826271199e-06\\
524	7.46858967594716e-06\\
525	7.80362340238037e-06\\
526	8.15494582598335e-06\\
527	8.54647610794914e-06\\
528	9.7814929141284e-06\\
529	1.14106440019494e-05\\
530	1.31283062583692e-05\\
531	1.49323076594289e-05\\
532	1.68675940995809e-05\\
533	2.20733874857024e-05\\
534	3.1423030944064e-05\\
535	4.11197137462416e-05\\
536	5.11971048988361e-05\\
537	6.16939429296373e-05\\
538	7.26557429743505e-05\\
539	8.4140481352317e-05\\
540	9.62439842851792e-05\\
541	0.000183017843515156\\
542	0.000320058431690405\\
543	0.000461979071212767\\
544	0.000609226141479381\\
545	0.000762310686206207\\
546	0.000921818775126619\\
547	0.00108842497921463\\
548	0.00126290890867717\\
549	0.00144617528986034\\
550	0.00163927812649895\\
551	0.00184344890560487\\
552	0.00206012562878466\\
553	0.00228932252299241\\
554	0.00252454541366836\\
555	0.00276610985074724\\
556	0.00301435835283926\\
557	0.00326969255277124\\
558	0.00353252020540563\\
559	0.00380323021433977\\
560	0.00408230479019987\\
561	0.00437023316844891\\
562	0.00466746777496262\\
563	0.00497427187802432\\
564	0.00529105479441378\\
565	0.00561646159330649\\
566	0.00594152902958734\\
567	0.00626327298138607\\
568	0.00643204595372145\\
569	0.00659656549703352\\
570	0.00675547050848819\\
571	0.00690723144372143\\
572	0.00704974781070761\\
573	0.00718309251742785\\
574	0.00731539000987618\\
575	0.00744700693395213\\
576	0.0075776877763659\\
577	0.00770719439458229\\
578	0.0078353307681382\\
579	0.00796197427776425\\
580	0.00808701048954674\\
581	0.00821087605745987\\
582	0.00833514879710221\\
583	0.00845957027804329\\
584	0.00858384582922777\\
585	0.00870763757218698\\
586	0.00883055685842396\\
587	0.00895211666486942\\
588	0.0090717816925381\\
589	0.0091889696268204\\
590	0.00930305497846415\\
591	0.00941337639759054\\
592	0.00951924639550553\\
593	0.00961996226627061\\
594	0.00971482541990144\\
595	0.00980316865229207\\
596	0.00988444283510058\\
597	0.00995832188366689\\
598	0.010000292044645\\
599	0\\
600	0\\
};
\addplot [color=mycolor13,solid,forget plot]
  table[row sep=crcr]{%
1	0.000229137842526684\\
2	0.000229137842526684\\
3	0.000229137842526684\\
4	0.000229137842526684\\
5	0.000229137842526684\\
6	0.000229137842526684\\
7	0.000229137842526684\\
8	0.000229137842526684\\
9	0.000229137842526684\\
10	0.000229137842526684\\
11	0.000229137842526684\\
12	0.000229137842526684\\
13	0.000229137842526684\\
14	0.000229137842526684\\
15	0.000229137842526684\\
16	0.000229137842526684\\
17	0.000229137842526684\\
18	0.000229137842526684\\
19	0.000229137842526684\\
20	0.000229137842526684\\
21	0.000229137842526684\\
22	0.000229137842526684\\
23	0.000229137842526684\\
24	0.000229137842526684\\
25	0.000229137842526684\\
26	0.000229137842526684\\
27	0.000229137842526684\\
28	0.000229137842526684\\
29	0.000229137842526684\\
30	0.000229137842526684\\
31	0.000229137842526684\\
32	0.000229137842526684\\
33	0.000229137842526684\\
34	0.000229137842526684\\
35	0.000229137842526684\\
36	0.000229137842526684\\
37	0.000229137842526684\\
38	0.000229137842526684\\
39	0.000229137842526684\\
40	0.000229137842526684\\
41	0.000229137842526684\\
42	0.000229137842526684\\
43	0.000229137842526684\\
44	0.000229137842526684\\
45	0.000229137842526684\\
46	0.000229137842526684\\
47	0.000229137842526684\\
48	0.000229137842526684\\
49	0.000229137842526684\\
50	0.000229137842526684\\
51	0.000229137842526684\\
52	0.000229137842526684\\
53	0.000229137842526684\\
54	0.000229137842526684\\
55	0.000229137842526684\\
56	0.000229137842526684\\
57	0.000229137842526684\\
58	0.000229137842526684\\
59	0.000229137842526684\\
60	0.000229137842526684\\
61	0.000229137842526684\\
62	0.000229137842526684\\
63	0.000229137842526684\\
64	0.000229137842526684\\
65	0.000229137842526684\\
66	0.000229137842526684\\
67	0.000229137842526684\\
68	0.000229137842526684\\
69	0.000229137842526684\\
70	0.000229137842526684\\
71	0.000229137842526684\\
72	0.000229137842526684\\
73	0.000229137842526684\\
74	0.000229137842526684\\
75	0.000229137842526684\\
76	0.000229137842526684\\
77	0.000229137842526684\\
78	0.000229137842526684\\
79	0.000229137842526684\\
80	0.000229137842526684\\
81	0.000229137842526684\\
82	0.000229137842526684\\
83	0.000229137842526684\\
84	0.000229137842526684\\
85	0.000229137842526684\\
86	0.000229137842526684\\
87	0.000229137842526684\\
88	0.000229137842526684\\
89	0.000229137842526684\\
90	0.000229137842526684\\
91	0.000229137842526684\\
92	0.000229137842526684\\
93	0.000229137842526684\\
94	0.000229137842526684\\
95	0.000229137842526684\\
96	0.000229137842526684\\
97	0.000229137842526684\\
98	0.000229137842526684\\
99	0.000229137842526684\\
100	0.000229137842526684\\
101	0.000229137842526684\\
102	0.000229137842526684\\
103	0.000229137842526684\\
104	0.000229137842526684\\
105	0.000229137842526684\\
106	0.000229137842526684\\
107	0.000229137842526684\\
108	0.000229137842526684\\
109	0.000229137842526684\\
110	0.000229137842526684\\
111	0.000229137842526684\\
112	0.000229137842526684\\
113	0.000229137842526684\\
114	0.000229137842526684\\
115	0.000229137842526684\\
116	0.000229137842526684\\
117	0.000229137842526684\\
118	0.000229137842526684\\
119	0.000229137842526684\\
120	0.000229137842526684\\
121	0.000229137842526684\\
122	0.000229137842526684\\
123	0.000229137842526684\\
124	0.000229137842526684\\
125	0.000229137842526684\\
126	0.000229137842526684\\
127	0.000229137842526684\\
128	0.000229137842526684\\
129	0.000229137842526684\\
130	0.000229137842526684\\
131	0.000229137842526684\\
132	0.000229137842526684\\
133	0.000229137842526684\\
134	0.000229137842526684\\
135	0.000229137842526684\\
136	0.000229137842526684\\
137	0.000229137842526684\\
138	0.000229137842526684\\
139	0.000229137842526684\\
140	0.000229137842526684\\
141	0.000229137842526684\\
142	0.000229137842526684\\
143	0.000229137842526684\\
144	0.000229137842526684\\
145	0.000229137842526684\\
146	0.000229137842526684\\
147	0.000229137842526684\\
148	0.000229137842526684\\
149	0.000229137842526684\\
150	0.000229137842526684\\
151	0.000229137842526684\\
152	0.000229137842526684\\
153	0.000229137842526684\\
154	0.000229137842526684\\
155	0.000229137842526684\\
156	0.000229137842526684\\
157	0.000229137842526684\\
158	0.000229137842526684\\
159	0.000229137842526684\\
160	0.000229137842526684\\
161	0.000229137842526684\\
162	0.000229137842526684\\
163	0.000229137842526684\\
164	0.000229137842526684\\
165	0.000229137842526684\\
166	0.000229137842526684\\
167	0.000229137842526684\\
168	0.000229137842526684\\
169	0.000229137842526684\\
170	0.000229137842526684\\
171	0.000229137842526684\\
172	0.000229137842526684\\
173	0.000229137842526684\\
174	0.000229137842526684\\
175	0.000229137842526684\\
176	0.000229137842526684\\
177	0.000229137842526684\\
178	0.000229137842526684\\
179	0.000229137842526684\\
180	0.000229137842526684\\
181	0.000229137842526684\\
182	0.000229137842526684\\
183	0.000229137842526684\\
184	0.000229137842526684\\
185	0.000229137842526684\\
186	0.000229137842526684\\
187	0.000229137842526684\\
188	0.000229137842526684\\
189	0.000229137842526684\\
190	0.000229137842526684\\
191	0.000229137842526684\\
192	0.000229137842526684\\
193	0.000229137842526684\\
194	0.000229137842526684\\
195	0.000229137842526684\\
196	0.000229137842526684\\
197	0.000229137842526684\\
198	0.000229137842526684\\
199	0.000229137842526684\\
200	0.000229137842526684\\
201	0.000229137842526684\\
202	0.000229137842526684\\
203	0.000229137842526684\\
204	0.000229137842526684\\
205	0.000229137842526684\\
206	0.000229137842526684\\
207	0.000229137842526684\\
208	0.000229137842526684\\
209	0.000229137842526684\\
210	0.000229137842526684\\
211	0.000229137842526684\\
212	0.000229137842526684\\
213	0.000229137842526684\\
214	0.000229137842526684\\
215	0.000229137842526684\\
216	0.000229137842526684\\
217	0.000229137842526684\\
218	0.000229137842526684\\
219	0.000229137842526684\\
220	0.000229137842526684\\
221	0.000229137842526684\\
222	0.000229137842526684\\
223	0.000229137842526684\\
224	0.000229137842526684\\
225	0.000229137842526684\\
226	0.000229137842526684\\
227	0.000229137842526684\\
228	0.000229137842526684\\
229	0.000229137842526684\\
230	0.000229137842526684\\
231	0.000229137842526684\\
232	0.000229137842526684\\
233	0.000229137842526684\\
234	0.000229137842526684\\
235	0.000229137842526684\\
236	0.000229137842526684\\
237	0.000229137842526684\\
238	0.000229137842526684\\
239	0.000229137842526684\\
240	0.000229137842526684\\
241	0.000229137842526684\\
242	0.000229137842526684\\
243	0.000229137842526684\\
244	0.000229137842526684\\
245	0.000229137842526684\\
246	0.000229137842526684\\
247	0.000229137842526684\\
248	0.000229137842526684\\
249	0.000229137842526684\\
250	0.000229137842526684\\
251	0.000229137842526684\\
252	0.000229137842526684\\
253	0.000229137842526684\\
254	0.000229137842526684\\
255	0.000229137842526684\\
256	0.000229137842526684\\
257	0.000229137842526684\\
258	0.000229137842526684\\
259	0.000229137842526684\\
260	0.000229137842526684\\
261	0.000229137842526684\\
262	0.000229137842526684\\
263	0.000229137842526684\\
264	0.000229137842526684\\
265	0.000229137842526684\\
266	0.000229137842526684\\
267	0.000229137842526684\\
268	0.000229137842526684\\
269	0.000229137842526684\\
270	0.000229137842526684\\
271	0.000229137842526684\\
272	0.000229137842526684\\
273	0.000229137842526684\\
274	0.000229137842526684\\
275	0.000229137842526684\\
276	0.000229137842526684\\
277	0.000229137842526684\\
278	0.000229137842526684\\
279	0.000229137842526684\\
280	0.000229137842526684\\
281	0.000229137842526684\\
282	0.000229137842526684\\
283	0.000229137842526684\\
284	0.000229137842526684\\
285	0.000229137842526684\\
286	0.000229137842526684\\
287	0.000229137842526684\\
288	0.000229137842526684\\
289	0.000229137842526684\\
290	0.000229137842526684\\
291	0.000229137842526684\\
292	0.000229137842526684\\
293	0.000229137842526684\\
294	0.000229137842526684\\
295	0.000229137842526684\\
296	0.000229137842526684\\
297	0.000229137842526684\\
298	0.000229137842526684\\
299	0.000229137842526684\\
300	0.000229137842526684\\
301	0.000229137842526684\\
302	0.000229137842526684\\
303	0.000229137842526684\\
304	0.000229137842526684\\
305	0.000229137842526684\\
306	0.000229137842526684\\
307	0.000229137842526684\\
308	0.000229137842526684\\
309	0.000229137842526684\\
310	0.000229137842526684\\
311	0.000229137842526684\\
312	0.000229137842526684\\
313	0.000229137842526684\\
314	0.000229137842526684\\
315	0.000229137842526684\\
316	0.000229137842526684\\
317	0.000229137842526684\\
318	0.000229137842526684\\
319	0.000229137842526684\\
320	0.000229137842526684\\
321	0.000229137842526684\\
322	0.000229137842526684\\
323	0.000229137842526684\\
324	0.000229137842526684\\
325	0.000229137842526684\\
326	0.000229137842526684\\
327	0.000229137842526684\\
328	0.000229137842526684\\
329	0.000229137842526684\\
330	0.000229137842526684\\
331	0.000229137842526684\\
332	0.000229137842526684\\
333	0.000229137842526684\\
334	0.000229137842526684\\
335	0.000229137842526684\\
336	0.000229137842526684\\
337	0.000229137842526684\\
338	0.000229137842526684\\
339	0.000229137842526684\\
340	0.000229137842526684\\
341	0.000229137842526684\\
342	0.000229137842526684\\
343	0.000229137842526684\\
344	0.000229137842526684\\
345	0.000229137842526684\\
346	0.000229137842526684\\
347	0.000229137842526684\\
348	0.000229137842526684\\
349	0.000229137842526684\\
350	0.000229137842526684\\
351	0.000229137842526684\\
352	0.000229137842526684\\
353	0.000229137842526684\\
354	0.000229137842526684\\
355	0.000229137842526684\\
356	0.000229137842526684\\
357	0.000229137842526684\\
358	0.000229137842526684\\
359	0.000229137842526684\\
360	0.000229137842526684\\
361	0.000229137842526684\\
362	0.000229137842526684\\
363	0.000229137842526684\\
364	0.000229137842526684\\
365	0.000229137842526684\\
366	0.000229137842526684\\
367	0.000229137842526684\\
368	0.000229137842526684\\
369	0.000229137842526684\\
370	0.000229137842526684\\
371	0.000229137842526684\\
372	0.000229137842526684\\
373	0.000229137842526684\\
374	0.000229137842526684\\
375	0.000229137842526684\\
376	0.000229137842526684\\
377	0.000229137842526684\\
378	0.000229137842526684\\
379	0.000229137842526684\\
380	0.000229137842526684\\
381	0.000229137842526684\\
382	0.000229137842526684\\
383	0.000229137842526684\\
384	0.000229137842526684\\
385	0.000229137842526684\\
386	0.000229137842526684\\
387	0.000229137842526684\\
388	0.000229137842526684\\
389	0.000229137842526684\\
390	0.000229137842526684\\
391	0.000229137842526684\\
392	0.000229137842526684\\
393	0.000229137842526684\\
394	0.000229137842526684\\
395	0.000229137842526684\\
396	0.000229137842526684\\
397	0.000229137842526684\\
398	0.000229137842526684\\
399	0.000229137842526684\\
400	0.000229137842526684\\
401	0.000229137842526684\\
402	0.000229137842526684\\
403	0.000229137842526684\\
404	0.000229137842526684\\
405	0.000229137842526684\\
406	0.000229137842526684\\
407	0.000229137842526684\\
408	0.000229137842526684\\
409	0.000229137842526684\\
410	0.000229137842526684\\
411	0.000229137842526684\\
412	0.000229137842526684\\
413	0.000229137842526684\\
414	0.000229137842526684\\
415	0.000229137842526684\\
416	0.000229137842526684\\
417	0.000229137842526684\\
418	0.000229137842526684\\
419	0.000229137842526684\\
420	0.000229137842526684\\
421	0.000229137842526684\\
422	0.000229137842526684\\
423	0.000229137842526684\\
424	0.000229137842526684\\
425	0.000229137842526684\\
426	0.000229137842526684\\
427	0.000229137842526684\\
428	0.000229137842526684\\
429	0.000229137842526684\\
430	0.000229137842526684\\
431	0.000229137842526684\\
432	0.000229137842526684\\
433	0.000229137842526684\\
434	0.000229137842526684\\
435	0.000229137842526684\\
436	0.000229137842526684\\
437	0.000229137842526684\\
438	0.000229137842526684\\
439	0.000229137842526684\\
440	0.000229137842526684\\
441	0.000229137842526684\\
442	0.000229137842526684\\
443	0.000229137842526684\\
444	0.000229137842526684\\
445	0.000229137842526684\\
446	0.000229137842526684\\
447	0.000229137842526684\\
448	0.000229137842526684\\
449	0.000229137842526684\\
450	0.000229137842526684\\
451	0.000229137842526684\\
452	0.000229137842526684\\
453	0.000229137842526684\\
454	0.000229137842526684\\
455	0.000229137842526684\\
456	0.000229137842526684\\
457	0.000229137842526684\\
458	0.000229137842526684\\
459	0.000229137842526684\\
460	0.000229137842526684\\
461	0.000229137842526684\\
462	0.000229137842526684\\
463	0.000229137842526684\\
464	0.000229137842526684\\
465	0.000229137842526684\\
466	0.000229137842526684\\
467	0.000229137842526684\\
468	0.000229137842526684\\
469	0.000229137842526684\\
470	0.000229137842526684\\
471	0.000229137842526684\\
472	0.000229137842526684\\
473	0.000229137842526684\\
474	0.000229137842526684\\
475	0.000229137842526684\\
476	0.000229137842526684\\
477	0.000229137842526684\\
478	0.000229137842526684\\
479	0.000229137842526684\\
480	0.000229137842526684\\
481	0.000229137842526684\\
482	0.000229137842526684\\
483	0.000229137842526684\\
484	0.000229137842526684\\
485	0.000229137842526684\\
486	0.000229137842526684\\
487	0.000229137842526684\\
488	0.000229137842526684\\
489	0.000229137842526684\\
490	0.000229137842526684\\
491	0.000229137842526684\\
492	0.000229137842526684\\
493	0.000229137842526684\\
494	0.000229137842526684\\
495	0.000229137842526684\\
496	0.000229137842526684\\
497	0.000229137842526684\\
498	0.000229137842526684\\
499	0.000229137842526684\\
500	0.000229137842526684\\
501	0.000229137842526684\\
502	0.000229137842526684\\
503	0.000229137842526684\\
504	0.000229137842526684\\
505	0.000229137842526684\\
506	0.000229137842526684\\
507	0.000229137842526684\\
508	0.000229137842526684\\
509	0.000229137842526684\\
510	0.000229137842526684\\
511	0.000229137842526684\\
512	0.000229137842526684\\
513	0.000229137842526684\\
514	0.000229137842526684\\
515	0.000229137842526684\\
516	0.000229137842526684\\
517	0.000229137842526684\\
518	0.000229137842526684\\
519	0.000229137842526684\\
520	0.000229137842526684\\
521	0.000229137842526684\\
522	0.000229137842526684\\
523	0.000229137842526684\\
524	0.000229137842526684\\
525	0.000229137842526684\\
526	0.000229137842526684\\
527	0.000229137842526684\\
528	0.000229137842526684\\
529	0.000229137842526684\\
530	0.000229137842526684\\
531	0.000229137842526684\\
532	0.000229137842526684\\
533	0.000229137842526684\\
534	0.000229137842526684\\
535	0.000229137842526684\\
536	0.000229137842526684\\
537	0.000229137842526684\\
538	0.000229137842526684\\
539	0.000229137842526684\\
540	0.000229137842526684\\
541	0.000229137842526684\\
542	0.000229137842526684\\
543	0.000229137842526684\\
544	0.000229137842526684\\
545	0.000229137842526684\\
546	0.000229137842526684\\
547	0.000229137842526684\\
548	0.000229137842526684\\
549	0.000229137842526684\\
550	0.000229137842526684\\
551	0.000229137842526684\\
552	0.000229137842526684\\
553	0.000229137842526684\\
554	0.000229137842526684\\
555	0.000229137842526684\\
556	0.000229137842526684\\
557	0.000229137842526684\\
558	0.000229137842526684\\
559	0.000229137842526684\\
560	0.000229137842526684\\
561	0.000229137842526684\\
562	0.000229137842526684\\
563	0.000229137842526684\\
564	0.000229137842526684\\
565	0.00023087401359618\\
566	0.000242990420949804\\
567	0.000269136136587107\\
568	0.000448812459300998\\
569	0.000637589915763916\\
570	0.000838128118940092\\
571	0.00105199896671285\\
572	0.00128105703203953\\
573	0.00152512637960484\\
574	0.00177692804255711\\
575	0.00203669514046483\\
576	0.00230473255962503\\
577	0.00258130913841595\\
578	0.00286642117413111\\
579	0.00316021118843213\\
580	0.00346264275586949\\
581	0.00377236863713507\\
582	0.00408776308700265\\
583	0.00440895297433541\\
584	0.00473600934600418\\
585	0.0050690485378831\\
586	0.00540818140277331\\
587	0.00575350248078163\\
588	0.00610506048892851\\
589	0.00646281347291639\\
590	0.00682657112268328\\
591	0.00719605229431493\\
592	0.0075711377940388\\
593	0.00795166435354403\\
594	0.00833736409264152\\
595	0.00872777975004381\\
596	0.00912205985501356\\
597	0.00951842928902401\\
598	0.00991325377008988\\
599	0\\
600	0\\
};
\addplot [color=mycolor14,solid,forget plot]
  table[row sep=crcr]{%
1	0.0100002859420401\\
2	0.0100002859420007\\
3	0.0100002859419605\\
4	0.0100002859419196\\
5	0.0100002859418779\\
6	0.0100002859418355\\
7	0.0100002859417923\\
8	0.0100002859417484\\
9	0.0100002859417036\\
10	0.010000285941658\\
11	0.0100002859416117\\
12	0.0100002859415644\\
13	0.0100002859415163\\
14	0.0100002859414674\\
15	0.0100002859414176\\
16	0.0100002859413668\\
17	0.0100002859413152\\
18	0.0100002859412626\\
19	0.010000285941209\\
20	0.0100002859411545\\
21	0.010000285941099\\
22	0.0100002859410425\\
23	0.010000285940985\\
24	0.0100002859409265\\
25	0.0100002859408668\\
26	0.0100002859408061\\
27	0.0100002859407443\\
28	0.0100002859406814\\
29	0.0100002859406174\\
30	0.0100002859405522\\
31	0.0100002859404858\\
32	0.0100002859404182\\
33	0.0100002859403494\\
34	0.0100002859402793\\
35	0.010000285940208\\
36	0.0100002859401354\\
37	0.0100002859400615\\
38	0.0100002859399862\\
39	0.0100002859399096\\
40	0.0100002859398316\\
41	0.0100002859397522\\
42	0.0100002859396714\\
43	0.0100002859395891\\
44	0.0100002859395053\\
45	0.01000028593942\\
46	0.0100002859393331\\
47	0.0100002859392447\\
48	0.0100002859391547\\
49	0.010000285939063\\
50	0.0100002859389697\\
51	0.0100002859388748\\
52	0.0100002859387781\\
53	0.0100002859386796\\
54	0.0100002859385794\\
55	0.0100002859384774\\
56	0.0100002859383735\\
57	0.0100002859382677\\
58	0.0100002859381601\\
59	0.0100002859380505\\
60	0.0100002859379389\\
61	0.0100002859378253\\
62	0.0100002859377097\\
63	0.0100002859375919\\
64	0.0100002859374721\\
65	0.0100002859373501\\
66	0.0100002859372258\\
67	0.0100002859370994\\
68	0.0100002859369706\\
69	0.0100002859368396\\
70	0.0100002859367061\\
71	0.0100002859365703\\
72	0.010000285936432\\
73	0.0100002859362912\\
74	0.0100002859361479\\
75	0.010000285936002\\
76	0.0100002859358534\\
77	0.0100002859357022\\
78	0.0100002859355483\\
79	0.0100002859353916\\
80	0.010000285935232\\
81	0.0100002859350696\\
82	0.0100002859349042\\
83	0.0100002859347359\\
84	0.0100002859345645\\
85	0.0100002859343901\\
86	0.0100002859342125\\
87	0.0100002859340317\\
88	0.0100002859338476\\
89	0.0100002859336602\\
90	0.0100002859334695\\
91	0.0100002859332753\\
92	0.0100002859330776\\
93	0.0100002859328764\\
94	0.0100002859326715\\
95	0.0100002859324629\\
96	0.0100002859322506\\
97	0.0100002859320344\\
98	0.0100002859318144\\
99	0.0100002859315904\\
100	0.0100002859313624\\
101	0.0100002859311302\\
102	0.0100002859308939\\
103	0.0100002859306533\\
104	0.0100002859304084\\
105	0.0100002859301591\\
106	0.0100002859299053\\
107	0.010000285929647\\
108	0.010000285929384\\
109	0.0100002859291162\\
110	0.0100002859288436\\
111	0.0100002859285662\\
112	0.0100002859282837\\
113	0.0100002859279962\\
114	0.0100002859277035\\
115	0.0100002859274055\\
116	0.0100002859271022\\
117	0.0100002859267934\\
118	0.010000285926479\\
119	0.0100002859261591\\
120	0.0100002859258333\\
121	0.0100002859255017\\
122	0.0100002859251642\\
123	0.0100002859248205\\
124	0.0100002859244707\\
125	0.0100002859241147\\
126	0.0100002859237522\\
127	0.0100002859233832\\
128	0.0100002859230075\\
129	0.0100002859226252\\
130	0.0100002859222359\\
131	0.0100002859218397\\
132	0.0100002859214364\\
133	0.0100002859210258\\
134	0.0100002859206078\\
135	0.0100002859201823\\
136	0.0100002859197492\\
137	0.0100002859193083\\
138	0.0100002859188596\\
139	0.0100002859184027\\
140	0.0100002859179377\\
141	0.0100002859174643\\
142	0.0100002859169824\\
143	0.0100002859164918\\
144	0.0100002859159925\\
145	0.0100002859154842\\
146	0.0100002859149668\\
147	0.0100002859144401\\
148	0.010000285913904\\
149	0.0100002859133583\\
150	0.0100002859128027\\
151	0.0100002859122372\\
152	0.0100002859116616\\
153	0.0100002859110757\\
154	0.0100002859104792\\
155	0.0100002859098721\\
156	0.010000285909254\\
157	0.0100002859086249\\
158	0.0100002859079846\\
159	0.0100002859073327\\
160	0.0100002859066692\\
161	0.0100002859059938\\
162	0.0100002859053063\\
163	0.0100002859046065\\
164	0.0100002859038942\\
165	0.0100002859031691\\
166	0.010000285902431\\
167	0.0100002859016797\\
168	0.0100002859009149\\
169	0.0100002859001365\\
170	0.0100002858993441\\
171	0.0100002858985376\\
172	0.0100002858977166\\
173	0.0100002858968809\\
174	0.0100002858960303\\
175	0.0100002858951645\\
176	0.0100002858942831\\
177	0.010000285893386\\
178	0.0100002858924729\\
179	0.0100002858915434\\
180	0.0100002858905973\\
181	0.0100002858896343\\
182	0.0100002858886541\\
183	0.0100002858876564\\
184	0.0100002858866408\\
185	0.010000285885607\\
186	0.0100002858845548\\
187	0.0100002858834838\\
188	0.0100002858823936\\
189	0.010000285881284\\
190	0.0100002858801545\\
191	0.0100002858790049\\
192	0.0100002858778347\\
193	0.0100002858766436\\
194	0.0100002858754313\\
195	0.0100002858741973\\
196	0.0100002858729412\\
197	0.0100002858716627\\
198	0.0100002858703614\\
199	0.0100002858690369\\
200	0.0100002858676887\\
201	0.0100002858663165\\
202	0.0100002858649197\\
203	0.010000285863498\\
204	0.010000285862051\\
205	0.0100002858605781\\
206	0.010000285859079\\
207	0.010000285857553\\
208	0.0100002858559999\\
209	0.0100002858544191\\
210	0.01000028585281\\
211	0.0100002858511723\\
212	0.0100002858495053\\
213	0.0100002858478086\\
214	0.0100002858460817\\
215	0.0100002858443239\\
216	0.0100002858425348\\
217	0.0100002858407138\\
218	0.0100002858388603\\
219	0.0100002858369738\\
220	0.0100002858350536\\
221	0.0100002858330993\\
222	0.01000028583111\\
223	0.0100002858290853\\
224	0.0100002858270246\\
225	0.0100002858249271\\
226	0.0100002858227922\\
227	0.0100002858206192\\
228	0.0100002858184075\\
229	0.0100002858161565\\
230	0.0100002858138653\\
231	0.0100002858115332\\
232	0.0100002858091597\\
233	0.0100002858067438\\
234	0.0100002858042849\\
235	0.0100002858017821\\
236	0.0100002857992348\\
237	0.0100002857966421\\
238	0.0100002857940032\\
239	0.0100002857913173\\
240	0.0100002857885836\\
241	0.0100002857858011\\
242	0.0100002857829691\\
243	0.0100002857800867\\
244	0.0100002857771528\\
245	0.0100002857741668\\
246	0.0100002857711275\\
247	0.0100002857680341\\
248	0.0100002857648856\\
249	0.010000285761681\\
250	0.0100002857584194\\
251	0.0100002857550996\\
252	0.0100002857517207\\
253	0.0100002857482816\\
254	0.0100002857447813\\
255	0.0100002857412186\\
256	0.0100002857375925\\
257	0.0100002857339017\\
258	0.0100002857301452\\
259	0.0100002857263218\\
260	0.0100002857224303\\
261	0.0100002857184694\\
262	0.010000285714438\\
263	0.0100002857103348\\
264	0.0100002857061585\\
265	0.0100002857019077\\
266	0.0100002856975812\\
267	0.0100002856931777\\
268	0.0100002856886956\\
269	0.0100002856841337\\
270	0.0100002856794905\\
271	0.0100002856747646\\
272	0.0100002856699544\\
273	0.0100002856650585\\
274	0.0100002856600753\\
275	0.0100002856550033\\
276	0.0100002856498409\\
277	0.0100002856445865\\
278	0.0100002856392384\\
279	0.0100002856337949\\
280	0.0100002856282544\\
281	0.0100002856226151\\
282	0.0100002856168752\\
283	0.0100002856110329\\
284	0.0100002856050865\\
285	0.010000285599034\\
286	0.0100002855928735\\
287	0.0100002855866031\\
288	0.0100002855802208\\
289	0.0100002855737247\\
290	0.0100002855671127\\
291	0.0100002855603826\\
292	0.0100002855535325\\
293	0.01000028554656\\
294	0.0100002855394632\\
295	0.0100002855322396\\
296	0.0100002855248871\\
297	0.0100002855174033\\
298	0.0100002855097859\\
299	0.0100002855020324\\
300	0.0100002854941405\\
301	0.0100002854861076\\
302	0.0100002854779313\\
303	0.0100002854696089\\
304	0.0100002854611379\\
305	0.0100002854525155\\
306	0.0100002854437391\\
307	0.0100002854348059\\
308	0.0100002854257131\\
309	0.0100002854164578\\
310	0.0100002854070372\\
311	0.0100002853974482\\
312	0.0100002853876879\\
313	0.0100002853777532\\
314	0.0100002853676411\\
315	0.0100002853573482\\
316	0.0100002853468715\\
317	0.0100002853362076\\
318	0.0100002853253532\\
319	0.010000285314305\\
320	0.0100002853030595\\
321	0.0100002852916132\\
322	0.0100002852799626\\
323	0.010000285268104\\
324	0.0100002852560339\\
325	0.0100002852437485\\
326	0.0100002852312441\\
327	0.0100002852185168\\
328	0.0100002852055627\\
329	0.0100002851923779\\
330	0.0100002851789585\\
331	0.0100002851653003\\
332	0.010000285151399\\
333	0.0100002851372501\\
334	0.0100002851228489\\
335	0.0100002851081895\\
336	0.0100002850932653\\
337	0.0100002850780708\\
338	0.010000285062604\\
339	0.0100002850468651\\
340	0.0100002850308494\\
341	0.0100002850145525\\
342	0.0100002849979691\\
343	0.0100002849810934\\
344	0.0100002849639176\\
345	0.0100002849464311\\
346	0.01000028492862\\
347	0.0100002849104738\\
348	0.0100002848920035\\
349	0.0100002848732209\\
350	0.0100002848541209\\
351	0.010000284834698\\
352	0.0100002848149469\\
353	0.0100002847948621\\
354	0.010000284774438\\
355	0.010000284753669\\
356	0.0100002847325493\\
357	0.0100002847110732\\
358	0.0100002846892346\\
359	0.0100002846670276\\
360	0.0100002846444462\\
361	0.010000284621484\\
362	0.0100002845981349\\
363	0.0100002845743925\\
364	0.0100002845502503\\
365	0.0100002845257017\\
366	0.0100002845007402\\
367	0.010000284475359\\
368	0.0100002844495512\\
369	0.01000028442331\\
370	0.0100002843966282\\
371	0.0100002843694987\\
372	0.0100002843419144\\
373	0.0100002843138678\\
374	0.0100002842853515\\
375	0.0100002842563579\\
376	0.0100002842268794\\
377	0.0100002841969083\\
378	0.0100002841664373\\
379	0.0100002841354591\\
380	0.0100002841039676\\
381	0.010000284071959\\
382	0.0100002840394324\\
383	0.010000284006391\\
384	0.0100002839728374\\
385	0.0100002839387595\\
386	0.0100002839041161\\
387	0.0100002838688902\\
388	0.0100002838330719\\
389	0.0100002837966509\\
390	0.0100002837596164\\
391	0.0100002837219574\\
392	0.0100002836836623\\
393	0.0100002836447191\\
394	0.0100002836051154\\
395	0.0100002835648387\\
396	0.010000283523877\\
397	0.0100002834822202\\
398	0.0100002834398633\\
399	0.0100002833968121\\
400	0.0100002833530918\\
401	0.0100002833087522\\
402	0.0100002832638447\\
403	0.0100002832183229\\
404	0.0100002831719307\\
405	0.0100002831246517\\
406	0.0100002830764681\\
407	0.0100002830273611\\
408	0.0100002829773097\\
409	0.010000282926292\\
410	0.0100002828742905\\
411	0.0100002828213006\\
412	0.0100002827672993\\
413	0.0100002827122627\\
414	0.0100002826561662\\
415	0.0100002825989842\\
416	0.0100002825406901\\
417	0.0100002824812564\\
418	0.0100002824206546\\
419	0.0100002823588548\\
420	0.010000282295826\\
421	0.010000282231536\\
422	0.0100002821659511\\
423	0.0100002820990362\\
424	0.0100002820307548\\
425	0.0100002819610685\\
426	0.0100002818899375\\
427	0.0100002818173197\\
428	0.0100002817431712\\
429	0.0100002816674453\\
430	0.0100002815900927\\
431	0.0100002815110593\\
432	0.0100002814302846\\
433	0.010000281347695\\
434	0.010000281263193\\
435	0.0100002811766399\\
436	0.0100002810878466\\
437	0.0100002809966367\\
438	0.0100002809030812\\
439	0.0100002808073389\\
440	0.0100002807093254\\
441	0.0100002806089503\\
442	0.0100002805061168\\
443	0.0100002804007212\\
444	0.0100002802926518\\
445	0.0100002801817893\\
446	0.0100002800680067\\
447	0.0100002799511711\\
448	0.0100002798311475\\
449	0.0100002797078077\\
450	0.0100002795810433\\
451	0.0100002794507622\\
452	0.0100002793167834\\
453	0.0100002791783705\\
454	0.0100002790329246\\
455	0.0100002788742218\\
456	0.0100002786912104\\
457	0.0100002784662658\\
458	0.0100002781385529\\
459	0.0100002777823409\\
460	0.0100002774191207\\
461	0.0100002770480306\\
462	0.0100002766683527\\
463	0.0100002762820748\\
464	0.0100002758890429\\
465	0.0100002754890043\\
466	0.0100002750816683\\
467	0.0100002746666901\\
468	0.0100002742436629\\
469	0.010000273812189\\
470	0.010000273372242\\
471	0.0100002729235318\\
472	0.0100002724657608\\
473	0.010000271998872\\
474	0.0100002715233001\\
475	0.0100002710401318\\
476	0.0100002705501455\\
477	0.0100002700492408\\
478	0.010000269532826\\
479	0.0100002689996552\\
480	0.0100002684479949\\
481	0.0100002678749656\\
482	0.0100002672747749\\
483	0.0100002666339708\\
484	0.0100002659191922\\
485	0.0100002650527385\\
486	0.0100002639818543\\
487	0.0100002628908232\\
488	0.0100002617790337\\
489	0.0100002606458778\\
490	0.0100002594907488\\
491	0.0100002583130897\\
492	0.0100002571124782\\
493	0.0100002556337237\\
494	0.0100002527280105\\
495	0.0100002497661292\\
496	0.0100002467509689\\
497	0.010000243677361\\
498	0.010000240524656\\
499	0.0100002372882573\\
500	0.0100002339625459\\
501	0.010000230539997\\
502	0.0100002270092879\\
503	0.0100002233519699\\
504	0.0100002195409957\\
505	0.0100002155711457\\
506	0.0100002114097909\\
507	0.0100002069276187\\
508	0.0100002017910705\\
509	0.0100001961855836\\
510	0.0100001905050262\\
511	0.010000184748613\\
512	0.0100001789157995\\
513	0.0100001730075422\\
514	0.0100001670608041\\
515	0.0100001610419521\\
516	0.0100001549299995\\
517	0.0100001487215882\\
518	0.0100001424139541\\
519	0.010000136005646\\
520	0.0100001294962251\\
521	0.0100001226945644\\
522	0.0100001156869292\\
523	0.010000108464679\\
524	0.0100001010177974\\
525	0.0100000931838069\\
526	0.010000084356173\\
527	0.0100000538977315\\
528	0.0100000193361654\\
529	0.00999998321785323\\
530	0.0099999452882706\\
531	0.00999990517493529\\
532	0.00999986185080286\\
533	0.00999973171419417\\
534	0.00999952840856016\\
535	0.00999931776418426\\
536	0.00999909913812807\\
537	0.00999887174402923\\
538	0.00999863468328567\\
539	0.00999838690736834\\
540	0.00999812704560558\\
541	0.00999779878085236\\
542	0.0099948011198057\\
543	0.009991775940215\\
544	0.00998872111588451\\
545	0.00998563430540921\\
546	0.00998251297075415\\
547	0.00997935439821524\\
548	0.00997615571946956\\
549	0.00997291395177169\\
550	0.00996962606791562\\
551	0.00996628911718749\\
552	0.00996290045236572\\
553	0.00995945824145994\\
554	0.00995607539716139\\
555	0.00995278083794119\\
556	0.00994957833519301\\
557	0.00994647191919197\\
558	0.0099434660064334\\
559	0.00994056475167603\\
560	0.00993777262166188\\
561	0.00993509473746594\\
562	0.00993253721631877\\
563	0.00993010821294715\\
564	0.00992781407422411\\
565	0.00992488752606354\\
566	0.00991049821720781\\
567	0.00989541754955392\\
568	0.00973893862780728\\
569	0.00954294358828082\\
570	0.00933480713277825\\
571	0.00911292764504251\\
572	0.00887542241305692\\
573	0.0086216917751517\\
574	0.008360185452806\\
575	0.00809066765029874\\
576	0.00781282768321296\\
577	0.00752645339503333\\
578	0.00723148144731463\\
579	0.00692771451397655\\
580	0.00661569752735087\\
581	0.00629589973465196\\
582	0.00597019483658525\\
583	0.00563888259460737\\
584	0.00530187182734894\\
585	0.00495904947440365\\
586	0.0046102952828595\\
587	0.00425551508307342\\
588	0.00389467562961874\\
589	0.00352786639485424\\
590	0.00315529010407331\\
591	0.00277712112335691\\
592	0.00239346664055391\\
593	0.00200447501126164\\
594	0.00161039185165808\\
595	0.00121163414212813\\
596	0.000808977057617982\\
597	0.000404030214303389\\
598	2.9204464504877e-07\\
599	0\\
600	0\\
};
\addplot [color=mycolor15,solid,forget plot]
  table[row sep=crcr]{%
1	0.0099993294716719\\
2	0.00999932946961846\\
3	0.00999932946752786\\
4	0.00999932946539942\\
5	0.00999932946323247\\
6	0.0099993294610263\\
7	0.00999932945878021\\
8	0.00999932945649348\\
9	0.00999932945416537\\
10	0.00999932945179514\\
11	0.00999932944938202\\
12	0.00999932944692525\\
13	0.00999932944442402\\
14	0.00999932944187754\\
15	0.009999329439285\\
16	0.00999932943664556\\
17	0.00999932943395837\\
18	0.00999932943122257\\
19	0.00999932942843729\\
20	0.00999932942560163\\
21	0.00999932942271469\\
22	0.00999932941977553\\
23	0.00999932941678321\\
24	0.00999932941373678\\
25	0.00999932941063526\\
26	0.00999932940747765\\
27	0.00999932940426295\\
28	0.00999932940099012\\
29	0.00999932939765812\\
30	0.00999932939426587\\
31	0.00999932939081229\\
32	0.00999932938729628\\
33	0.00999932938371671\\
34	0.00999932938007243\\
35	0.00999932937636227\\
36	0.00999932937258505\\
37	0.00999932936873956\\
38	0.00999932936482457\\
39	0.00999932936083882\\
40	0.00999932935678105\\
41	0.00999932935264994\\
42	0.00999932934844418\\
43	0.00999932934416243\\
44	0.00999932933980331\\
45	0.00999932933536544\\
46	0.00999932933084738\\
47	0.0099993293262477\\
48	0.00999932932156493\\
49	0.00999932931679757\\
50	0.0099993293119441\\
51	0.00999932930700296\\
52	0.00999932930197258\\
53	0.00999932929685135\\
54	0.00999932929163763\\
55	0.00999932928632976\\
56	0.00999932928092604\\
57	0.00999932927542475\\
58	0.00999932926982413\\
59	0.0099993292641224\\
60	0.00999932925831773\\
61	0.00999932925240827\\
62	0.00999932924639214\\
63	0.00999932924026742\\
64	0.00999932923403215\\
65	0.00999932922768434\\
66	0.00999932922122197\\
67	0.00999932921464298\\
68	0.00999932920794527\\
69	0.00999932920112671\\
70	0.00999932919418512\\
71	0.00999932918711829\\
72	0.00999932917992397\\
73	0.00999932917259987\\
74	0.00999932916514366\\
75	0.00999932915755295\\
76	0.00999932914982534\\
77	0.00999932914195835\\
78	0.0099993291339495\\
79	0.00999932912579623\\
80	0.00999932911749593\\
81	0.00999932910904599\\
82	0.00999932910044369\\
83	0.00999932909168632\\
84	0.00999932908277107\\
85	0.00999932907369513\\
86	0.00999932906445559\\
87	0.00999932905504954\\
88	0.00999932904547396\\
89	0.00999932903572584\\
90	0.00999932902580205\\
91	0.00999932901569946\\
92	0.00999932900541485\\
93	0.00999932899494496\\
94	0.00999932898428646\\
95	0.00999932897343597\\
96	0.00999932896239005\\
97	0.00999932895114518\\
98	0.00999932893969781\\
99	0.00999932892804429\\
100	0.00999932891618094\\
101	0.00999932890410399\\
102	0.00999932889180961\\
103	0.0099993288792939\\
104	0.0099993288665529\\
105	0.00999932885358256\\
106	0.00999932884037878\\
107	0.00999932882693737\\
108	0.00999932881325408\\
109	0.00999932879932457\\
110	0.00999932878514444\\
111	0.00999932877070918\\
112	0.00999932875601424\\
113	0.00999932874105496\\
114	0.0099993287258266\\
115	0.00999932871032436\\
116	0.00999932869454332\\
117	0.0099993286784785\\
118	0.00999932866212481\\
119	0.00999932864547709\\
120	0.00999932862853007\\
121	0.0099993286112784\\
122	0.00999932859371662\\
123	0.00999932857583919\\
124	0.00999932855764046\\
125	0.00999932853911468\\
126	0.00999932852025601\\
127	0.00999932850105849\\
128	0.00999932848151605\\
129	0.00999932846162254\\
130	0.00999932844137168\\
131	0.00999932842075708\\
132	0.00999932839977223\\
133	0.00999932837841052\\
134	0.00999932835666521\\
135	0.00999932833452945\\
136	0.00999932831199627\\
137	0.00999932828905855\\
138	0.00999932826570908\\
139	0.0099993282419405\\
140	0.00999932821774533\\
141	0.00999932819311594\\
142	0.00999932816804458\\
143	0.00999932814252336\\
144	0.00999932811654425\\
145	0.00999932809009907\\
146	0.00999932806317951\\
147	0.00999932803577709\\
148	0.00999932800788321\\
149	0.00999932797948908\\
150	0.00999932795058578\\
151	0.00999932792116424\\
152	0.00999932789121519\\
153	0.00999932786072924\\
154	0.0099993278296968\\
155	0.00999932779810814\\
156	0.00999932776595332\\
157	0.00999932773322226\\
158	0.00999932769990468\\
159	0.00999932766599012\\
160	0.00999932763146793\\
161	0.0099993275963273\\
162	0.0099993275605572\\
163	0.0099993275241464\\
164	0.00999932748708349\\
165	0.00999932744935685\\
166	0.00999932741095465\\
167	0.00999932737186486\\
168	0.00999932733207523\\
169	0.00999932729157329\\
170	0.00999932725034635\\
171	0.0099993272083815\\
172	0.0099993271656656\\
173	0.00999932712218528\\
174	0.00999932707792692\\
175	0.00999932703287667\\
176	0.00999932698702043\\
177	0.00999932694034386\\
178	0.00999932689283234\\
179	0.00999932684447103\\
180	0.00999932679524479\\
181	0.00999932674513823\\
182	0.00999932669413568\\
183	0.00999932664222121\\
184	0.00999932658937857\\
185	0.00999932653559127\\
186	0.00999932648084248\\
187	0.00999932642511511\\
188	0.00999932636839175\\
189	0.00999932631065468\\
190	0.00999932625188587\\
191	0.00999932619206697\\
192	0.00999932613117929\\
193	0.00999932606920384\\
194	0.00999932600612126\\
195	0.00999932594191187\\
196	0.00999932587655564\\
197	0.00999932581003216\\
198	0.00999932574232068\\
199	0.00999932567340008\\
200	0.00999932560324887\\
201	0.00999932553184515\\
202	0.00999932545916667\\
203	0.00999932538519075\\
204	0.00999932530989434\\
205	0.00999932523325396\\
206	0.00999932515524571\\
207	0.00999932507584528\\
208	0.00999932499502793\\
209	0.00999932491276845\\
210	0.00999932482904122\\
211	0.00999932474382014\\
212	0.00999932465707866\\
213	0.00999932456878976\\
214	0.00999932447892592\\
215	0.00999932438745914\\
216	0.00999932429436094\\
217	0.00999932419960231\\
218	0.00999932410315373\\
219	0.00999932400498516\\
220	0.00999932390506602\\
221	0.00999932380336519\\
222	0.00999932369985099\\
223	0.00999932359449118\\
224	0.00999932348725295\\
225	0.00999932337810289\\
226	0.00999932326700701\\
227	0.00999932315393072\\
228	0.00999932303883878\\
229	0.00999932292169536\\
230	0.00999932280246397\\
231	0.00999932268110747\\
232	0.00999932255758806\\
233	0.00999932243186726\\
234	0.00999932230390592\\
235	0.00999932217366417\\
236	0.00999932204110144\\
237	0.00999932190617643\\
238	0.00999932176884709\\
239	0.00999932162907065\\
240	0.00999932148680354\\
241	0.00999932134200143\\
242	0.00999932119461919\\
243	0.00999932104461088\\
244	0.00999932089192975\\
245	0.0099993207365282\\
246	0.00999932057835777\\
247	0.00999932041736915\\
248	0.00999932025351213\\
249	0.00999932008673561\\
250	0.00999931991698756\\
251	0.00999931974421503\\
252	0.00999931956836411\\
253	0.00999931938937993\\
254	0.00999931920720662\\
255	0.00999931902178732\\
256	0.00999931883306414\\
257	0.00999931864097814\\
258	0.00999931844546934\\
259	0.00999931824647667\\
260	0.00999931804393796\\
261	0.00999931783778991\\
262	0.00999931762796809\\
263	0.00999931741440691\\
264	0.0099993171970396\\
265	0.00999931697579818\\
266	0.00999931675061343\\
267	0.0099993165214149\\
268	0.00999931628813086\\
269	0.00999931605068828\\
270	0.0099993158090128\\
271	0.00999931556302874\\
272	0.00999931531265904\\
273	0.00999931505782528\\
274	0.0099993147984476\\
275	0.00999931453444469\\
276	0.00999931426573378\\
277	0.00999931399223057\\
278	0.00999931371384927\\
279	0.00999931343050251\\
280	0.00999931314210137\\
281	0.00999931284855528\\
282	0.00999931254977206\\
283	0.00999931224565783\\
284	0.00999931193611706\\
285	0.00999931162105244\\
286	0.00999931130036492\\
287	0.00999931097395366\\
288	0.00999931064171599\\
289	0.00999931030354737\\
290	0.00999930995934138\\
291	0.00999930960898969\\
292	0.00999930925238198\\
293	0.00999930888940597\\
294	0.00999930851994732\\
295	0.00999930814388966\\
296	0.00999930776111449\\
297	0.0099993073715012\\
298	0.009999306974927\\
299	0.00999930657126689\\
300	0.00999930616039365\\
301	0.00999930574217775\\
302	0.00999930531648737\\
303	0.00999930488318832\\
304	0.00999930444214405\\
305	0.00999930399321555\\
306	0.0099993035362614\\
307	0.00999930307113768\\
308	0.00999930259769793\\
309	0.00999930211579314\\
310	0.00999930162527162\\
311	0.00999930112597899\\
312	0.00999930061775813\\
313	0.00999930010044936\\
314	0.00999929957389025\\
315	0.00999929903791562\\
316	0.0099992984923575\\
317	0.00999929793704509\\
318	0.00999929737180476\\
319	0.00999929679646003\\
320	0.00999929621083155\\
321	0.00999929561473706\\
322	0.00999929500799143\\
323	0.00999929439040663\\
324	0.00999929376179172\\
325	0.00999929312195288\\
326	0.00999929247069341\\
327	0.00999929180781369\\
328	0.00999929113311117\\
329	0.00999929044638015\\
330	0.00999928974741139\\
331	0.00999928903599092\\
332	0.00999928831189758\\
333	0.00999928757489793\\
334	0.0099992868247374\\
335	0.00999928606112985\\
336	0.00999928528375961\\
337	0.00999928449233488\\
338	0.00999928368671541\\
339	0.00999928286686936\\
340	0.00999928203259543\\
341	0.00999928118365192\\
342	0.00999928031977998\\
343	0.00999927944068345\\
344	0.00999927854599073\\
345	0.00999927763520075\\
346	0.00999927670767158\\
347	0.00999927576287146\\
348	0.00999927480114783\\
349	0.00999927382301776\\
350	0.00999927282832036\\
351	0.00999927181677662\\
352	0.00999927078810297\\
353	0.00999926974201125\\
354	0.00999926867820868\\
355	0.0099992675963982\\
356	0.00999926649627893\\
357	0.00999926537754606\\
358	0.0099992642398888\\
359	0.00999926308299162\\
360	0.0099992619065342\\
361	0.0099992607101914\\
362	0.00999925949363315\\
363	0.00999925825652447\\
364	0.00999925699852534\\
365	0.00999925571929069\\
366	0.0099992544184703\\
367	0.00999925309570874\\
368	0.00999925175064532\\
369	0.009999250382914\\
370	0.00999924899214341\\
371	0.00999924757795682\\
372	0.00999924613997222\\
373	0.00999924467780235\\
374	0.0099992431910543\\
375	0.00999924167932856\\
376	0.00999924014221759\\
377	0.00999923857930842\\
378	0.00999923699019723\\
379	0.00999923537449653\\
380	0.00999923373186045\\
381	0.0099992320620289\\
382	0.0099992303648864\\
383	0.00999922864048714\\
384	0.00999922688889\\
385	0.0099992251095598\\
386	0.00999922330070009\\
387	0.00999922146127299\\
388	0.00999921959069207\\
389	0.00999921768840462\\
390	0.0099992157538371\\
391	0.00999921378639325\\
392	0.00999921178545247\\
393	0.00999920975036931\\
394	0.00999920768047584\\
395	0.00999920557509121\\
396	0.00999920343354766\\
397	0.00999920125525213\\
398	0.00999919903981771\\
399	0.00999919678731162\\
400	0.00999919449862671\\
401	0.00999919217573563\\
402	0.0099991898208095\\
403	0.00999918743212194\\
404	0.00999918499899439\\
405	0.00999918251891641\\
406	0.00999917999097951\\
407	0.00999917741414851\\
408	0.00999917478725977\\
409	0.00999917210902027\\
410	0.00999916937824074\\
411	0.00999916659449198\\
412	0.00999916375757198\\
413	0.00999916086623067\\
414	0.00999915791917385\\
415	0.00999915491506151\\
416	0.00999915185250656\\
417	0.00999914873007448\\
418	0.00999914554628377\\
419	0.00999914229960513\\
420	0.00999913898845258\\
421	0.00999913561115313\\
422	0.00999913216589674\\
423	0.00999912865076033\\
424	0.0099991250638269\\
425	0.00999912140309305\\
426	0.00999911766646265\\
427	0.00999911385173922\\
428	0.00999910995661611\\
429	0.00999910597866151\\
430	0.0099991019152918\\
431	0.00999909776371751\\
432	0.00999909352082707\\
433	0.00999908918293894\\
434	0.00999908474531591\\
435	0.00999908020140347\\
436	0.00999907554235725\\
437	0.00999907075949376\\
438	0.00999906585399951\\
439	0.00999906083218615\\
440	0.0099990556912567\\
441	0.00999905042646166\\
442	0.00999904503269721\\
443	0.00999903950445559\\
444	0.00999903383576376\\
445	0.00999902802012308\\
446	0.00999902205051696\\
447	0.00999901591968957\\
448	0.00999900962088153\\
449	0.00999900314709705\\
450	0.00999899649175576\\
451	0.00999898964872395\\
452	0.00999898260774513\\
453	0.00999897533405877\\
454	0.00999896770985755\\
455	0.00999895944454393\\
456	0.00999894998388565\\
457	0.00999893838990642\\
458	0.00999892236539694\\
459	0.00999889969768127\\
460	0.00999887523716839\\
461	0.00999885026190805\\
462	0.00999882473405357\\
463	0.00999879872460709\\
464	0.00999877223404172\\
465	0.0099987452434011\\
466	0.00999871773053609\\
467	0.00999868966956539\\
468	0.00999866102980771\\
469	0.00999863177775147\\
470	0.00999860189300072\\
471	0.00999857136223286\\
472	0.00999854015882764\\
473	0.00999850826004149\\
474	0.00999847566151176\\
475	0.00999844238425681\\
476	0.00999840843607569\\
477	0.00999837362278948\\
478	0.00999833769663121\\
479	0.0099983005414136\\
480	0.00999826205301255\\
481	0.00999822207094425\\
482	0.00999818030273864\\
483	0.00999813612354068\\
484	0.00999808804373537\\
485	0.00999803258721717\\
486	0.00999795214372719\\
487	0.00999781736381448\\
488	0.00999767961429178\\
489	0.00999753875507356\\
490	0.00999739466506454\\
491	0.00999724721376824\\
492	0.00999709624852324\\
493	0.00999693121355686\\
494	0.00999670530755466\\
495	0.00999647416658922\\
496	0.00999623775903836\\
497	0.00999599571349533\\
498	0.00999574697161254\\
499	0.0099954910077634\\
500	0.00999522733925143\\
501	0.00999495537165017\\
502	0.00999467431269238\\
503	0.00999438302117948\\
504	0.00999407991692828\\
505	0.0099937641613075\\
506	0.00999343394490521\\
507	0.00999308334337721\\
508	0.00999269762074172\\
509	0.0099914554365218\\
510	0.00998940124696233\\
511	0.00998732863778075\\
512	0.0099852370976057\\
513	0.00998312610223551\\
514	0.00998099650252101\\
515	0.00997884616187541\\
516	0.00997667304539808\\
517	0.00997447597048032\\
518	0.00997225383250046\\
519	0.00997000550783338\\
520	0.0099677298408233\\
521	0.00996541777538252\\
522	0.00996307154385438\\
523	0.00996068939531412\\
524	0.00995826957185022\\
525	0.00995580426317674\\
526	0.00995326696843847\\
527	0.00994980420231118\\
528	0.00994613560965982\\
529	0.00994236367068748\\
530	0.0099384752067908\\
531	0.0099344521415761\\
532	0.00993027162010349\\
533	0.00992252256538219\\
534	0.00991175654377404\\
535	0.00990070756465675\\
536	0.00988934850347928\\
537	0.00987764589693117\\
538	0.00986556114281452\\
539	0.00985304879076982\\
540	0.00984004906369896\\
541	0.00982416100283728\\
542	0.00969282387754647\\
543	0.00955665231647043\\
544	0.00941520952431028\\
545	0.00926799773951631\\
546	0.00911444673581855\\
547	0.00895390159476861\\
548	0.00878560714903328\\
549	0.00860868904960451\\
550	0.00842213090002012\\
551	0.00822474711779364\\
552	0.00801515303577693\\
553	0.00779174238648091\\
554	0.00756053130714489\\
555	0.0073227869414988\\
556	0.00707816704069707\\
557	0.00682631110018469\\
558	0.00656685138566235\\
559	0.00629933175157336\\
560	0.00602325745472997\\
561	0.00573812592479245\\
562	0.00544347453026945\\
563	0.00513904679861026\\
564	0.00482435547151584\\
565	0.00449969466636478\\
566	0.00417568949117384\\
567	0.00384096269838464\\
568	0.00363912200043192\\
569	0.00346881456343741\\
570	0.00330378970436501\\
571	0.00314579069162\\
572	0.0029969195629914\\
573	0.00285788076580331\\
574	0.0027194058618991\\
575	0.00258161828951404\\
576	0.00244481403871134\\
577	0.00230928115503963\\
578	0.00217527788184951\\
579	0.00204300485446921\\
580	0.00191264896058479\\
581	0.00178427582037178\\
582	0.00165614535808679\\
583	0.00152811363328425\\
584	0.00140051135640073\\
585	0.00127372219294322\\
586	0.00114818338399011\\
587	0.00102441882820826\\
588	0.000903028512395\\
589	0.000784662617985752\\
590	0.000670016384725282\\
591	0.00055982374888513\\
592	0.000454847943832255\\
593	0.000355865647533659\\
594	0.000263643164224137\\
595	0.000178902643795078\\
596	0.000102272945261959\\
597	3.41569359503316e-05\\
598	2.9204464504877e-07\\
599	0\\
600	0\\
};
\addplot [color=mycolor16,solid,forget plot]
  table[row sep=crcr]{%
1	0.00996936184596852\\
2	0.009969361812907\\
3	0.00996936177924712\\
4	0.00996936174497806\\
5	0.00996936171008882\\
6	0.0099693616745682\\
7	0.00996936163840478\\
8	0.00996936160158697\\
9	0.00996936156410294\\
10	0.00996936152594065\\
11	0.00996936148708786\\
12	0.0099693614475321\\
13	0.00996936140726067\\
14	0.00996936136626065\\
15	0.00996936132451889\\
16	0.00996936128202199\\
17	0.00996936123875631\\
18	0.00996936119470799\\
19	0.00996936114986289\\
20	0.00996936110420663\\
21	0.00996936105772457\\
22	0.0099693610104018\\
23	0.00996936096222315\\
24	0.00996936091317318\\
25	0.00996936086323616\\
26	0.00996936081239608\\
27	0.00996936076063665\\
28	0.00996936070794129\\
29	0.00996936065429311\\
30	0.00996936059967491\\
31	0.00996936054406921\\
32	0.00996936048745819\\
33	0.00996936042982372\\
34	0.00996936037114734\\
35	0.00996936031141026\\
36	0.00996936025059335\\
37	0.00996936018867715\\
38	0.00996936012564183\\
39	0.00996936006146721\\
40	0.00996935999613276\\
41	0.00996935992961757\\
42	0.00996935986190036\\
43	0.00996935979295946\\
44	0.00996935972277281\\
45	0.00996935965131798\\
46	0.0099693595785721\\
47	0.00996935950451191\\
48	0.00996935942911374\\
49	0.00996935935235347\\
50	0.00996935927420657\\
51	0.00996935919464807\\
52	0.00996935911365253\\
53	0.00996935903119407\\
54	0.00996935894724636\\
55	0.00996935886178256\\
56	0.00996935877477539\\
57	0.00996935868619705\\
58	0.00996935859601926\\
59	0.00996935850421322\\
60	0.00996935841074963\\
61	0.00996935831559864\\
62	0.0099693582187299\\
63	0.00996935812011248\\
64	0.00996935801971492\\
65	0.00996935791750519\\
66	0.00996935781345069\\
67	0.00996935770751822\\
68	0.009969357599674\\
69	0.00996935748988366\\
70	0.00996935737811217\\
71	0.00996935726432392\\
72	0.00996935714848263\\
73	0.00996935703055139\\
74	0.00996935691049262\\
75	0.00996935678826808\\
76	0.00996935666383882\\
77	0.00996935653716523\\
78	0.00996935640820695\\
79	0.00996935627692293\\
80	0.00996935614327138\\
81	0.00996935600720974\\
82	0.00996935586869473\\
83	0.00996935572768225\\
84	0.00996935558412745\\
85	0.00996935543798466\\
86	0.00996935528920739\\
87	0.00996935513774834\\
88	0.00996935498355933\\
89	0.00996935482659136\\
90	0.00996935466679452\\
91	0.00996935450411804\\
92	0.0099693543385102\\
93	0.00996935416991841\\
94	0.00996935399828909\\
95	0.00996935382356775\\
96	0.00996935364569888\\
97	0.00996935346462602\\
98	0.00996935328029167\\
99	0.00996935309263733\\
100	0.00996935290160345\\
101	0.00996935270712939\\
102	0.00996935250915347\\
103	0.00996935230761288\\
104	0.0099693521024437\\
105	0.00996935189358088\\
106	0.00996935168095817\\
107	0.0099693514645082\\
108	0.00996935124416234\\
109	0.00996935101985078\\
110	0.00996935079150243\\
111	0.00996935055904496\\
112	0.00996935032240473\\
113	0.00996935008150682\\
114	0.00996934983627493\\
115	0.00996934958663143\\
116	0.0099693493324973\\
117	0.00996934907379211\\
118	0.009969348810434\\
119	0.00996934854233965\\
120	0.00996934826942425\\
121	0.00996934799160149\\
122	0.00996934770878352\\
123	0.00996934742088092\\
124	0.00996934712780268\\
125	0.00996934682945618\\
126	0.00996934652574712\\
127	0.00996934621657956\\
128	0.00996934590185584\\
129	0.00996934558147654\\
130	0.00996934525534049\\
131	0.00996934492334473\\
132	0.00996934458538443\\
133	0.00996934424135295\\
134	0.00996934389114169\\
135	0.00996934353464017\\
136	0.00996934317173592\\
137	0.00996934280231446\\
138	0.0099693424262593\\
139	0.00996934204345186\\
140	0.00996934165377145\\
141	0.00996934125709524\\
142	0.00996934085329822\\
143	0.00996934044225315\\
144	0.00996934002383052\\
145	0.00996933959789853\\
146	0.00996933916432303\\
147	0.00996933872296749\\
148	0.00996933827369294\\
149	0.00996933781635796\\
150	0.00996933735081861\\
151	0.00996933687692837\\
152	0.00996933639453814\\
153	0.00996933590349615\\
154	0.00996933540364796\\
155	0.00996933489483636\\
156	0.00996933437690135\\
157	0.0099693338496801\\
158	0.00996933331300687\\
159	0.00996933276671299\\
160	0.00996933221062677\\
161	0.00996933164457348\\
162	0.00996933106837531\\
163	0.00996933048185124\\
164	0.00996932988481708\\
165	0.00996932927708533\\
166	0.00996932865846519\\
167	0.00996932802876245\\
168	0.00996932738777944\\
169	0.009969326735315\\
170	0.00996932607116438\\
171	0.0099693253951192\\
172	0.00996932470696737\\
173	0.00996932400649304\\
174	0.0099693232934765\\
175	0.00996932256769416\\
176	0.00996932182891845\\
177	0.00996932107691774\\
178	0.0099693203114563\\
179	0.00996931953229421\\
180	0.00996931873918727\\
181	0.00996931793188694\\
182	0.00996931711014028\\
183	0.00996931627368983\\
184	0.00996931542227356\\
185	0.00996931455562478\\
186	0.00996931367347207\\
187	0.00996931277553915\\
188	0.00996931186154485\\
189	0.009969310931203\\
190	0.00996930998422232\\
191	0.00996930902030636\\
192	0.0099693080391534\\
193	0.00996930704045633\\
194	0.00996930602390258\\
195	0.00996930498917402\\
196	0.00996930393594686\\
197	0.00996930286389154\\
198	0.00996930177267263\\
199	0.0099693006619487\\
200	0.00996929953137229\\
201	0.00996929838058969\\
202	0.00996929720924093\\
203	0.0099692960169596\\
204	0.00996929480337277\\
205	0.00996929356810084\\
206	0.00996929231075747\\
207	0.00996929103094939\\
208	0.00996928972827635\\
209	0.00996928840233092\\
210	0.00996928705269843\\
211	0.00996928567895679\\
212	0.00996928428067636\\
213	0.00996928285741984\\
214	0.00996928140874213\\
215	0.00996927993419013\\
216	0.00996927843330267\\
217	0.00996927690561034\\
218	0.00996927535063532\\
219	0.00996927376789122\\
220	0.00996927215688299\\
221	0.00996927051710667\\
222	0.0099692688480493\\
223	0.00996926714918873\\
224	0.00996926541999343\\
225	0.00996926365992238\\
226	0.00996926186842481\\
227	0.0099692600449401\\
228	0.00996925818889757\\
229	0.00996925629971628\\
230	0.00996925437680486\\
231	0.00996925241956132\\
232	0.00996925042737284\\
233	0.00996924839961561\\
234	0.00996924633565455\\
235	0.00996924423484321\\
236	0.00996924209652345\\
237	0.00996923992002532\\
238	0.00996923770466676\\
239	0.00996923544975345\\
240	0.00996923315457854\\
241	0.00996923081842241\\
242	0.00996922844055249\\
243	0.00996922602022294\\
244	0.00996922355667447\\
245	0.00996922104913407\\
246	0.00996921849681475\\
247	0.00996921589891527\\
248	0.00996921325461991\\
249	0.00996921056309817\\
250	0.0099692078235045\\
251	0.00996920503497803\\
252	0.00996920219664228\\
253	0.00996919930760488\\
254	0.00996919636695724\\
255	0.00996919337377428\\
256	0.00996919032711411\\
257	0.00996918722601773\\
258	0.00996918406950867\\
259	0.00996918085659272\\
260	0.00996917758625754\\
261	0.00996917425747237\\
262	0.00996917086918766\\
263	0.0099691674203347\\
264	0.0099691639098253\\
265	0.00996916033655136\\
266	0.00996915669938455\\
267	0.00996915299717585\\
268	0.00996914922875514\\
269	0.00996914539293081\\
270	0.00996914148848932\\
271	0.00996913751419497\\
272	0.00996913346878982\\
273	0.00996912935099353\\
274	0.00996912515950223\\
275	0.00996912089298756\\
276	0.00996911655009667\\
277	0.00996911212945185\\
278	0.0099691076296501\\
279	0.00996910304926263\\
280	0.00996909838683436\\
281	0.00996909364088347\\
282	0.00996908880990086\\
283	0.00996908389234966\\
284	0.00996907888666469\\
285	0.00996907379125197\\
286	0.00996906860448814\\
287	0.00996906332471995\\
288	0.00996905795026368\\
289	0.00996905247940459\\
290	0.00996904691039637\\
291	0.00996904124146049\\
292	0.00996903547078571\\
293	0.00996902959652741\\
294	0.00996902361680701\\
295	0.00996901752971138\\
296	0.0099690113332922\\
297	0.00996900502556532\\
298	0.00996899860451019\\
299	0.00996899206806916\\
300	0.00996898541414689\\
301	0.0099689786406097\\
302	0.00996897174528493\\
303	0.00996896472596036\\
304	0.00996895758038362\\
305	0.00996895030626168\\
306	0.00996894290126038\\
307	0.00996893536300399\\
308	0.00996892768907455\\
309	0.00996891987701054\\
310	0.00996891192430447\\
311	0.0099689038284003\\
312	0.0099688955866946\\
313	0.00996888719654264\\
314	0.00996887865525406\\
315	0.00996886996009152\\
316	0.00996886110827023\\
317	0.00996885209695749\\
318	0.00996884292327232\\
319	0.00996883358428505\\
320	0.00996882407701709\\
321	0.00996881439844069\\
322	0.00996880454547877\\
323	0.00996879451500491\\
324	0.0099687843038434\\
325	0.00996877390876939\\
326	0.00996876332650918\\
327	0.00996875255374065\\
328	0.00996874158709383\\
329	0.00996873042315152\\
330	0.00996871905844984\\
331	0.00996870748947838\\
332	0.00996869571267886\\
333	0.00996868372444004\\
334	0.00996867152108336\\
335	0.00996865909882866\\
336	0.00996864645372656\\
337	0.00996863358157793\\
338	0.00996862047806023\\
339	0.0099686071397144\\
340	0.00996859356415082\\
341	0.00996857974775333\\
342	0.00996856568664523\\
343	0.00996855137690936\\
344	0.0099685368145441\\
345	0.00996852199534319\\
346	0.00996850691460096\\
347	0.00996849156656939\\
348	0.00996847594453828\\
349	0.00996846004729916\\
350	0.00996844387572912\\
351	0.00996842742591287\\
352	0.00996841069300453\\
353	0.00996839367206885\\
354	0.00996837635808168\\
355	0.00996835874594157\\
356	0.00996834083049468\\
357	0.00996832260653619\\
358	0.00996830406872731\\
359	0.00996828521163068\\
360	0.00996826602971775\\
361	0.00996824651736727\\
362	0.00996822666886374\\
363	0.00996820647839577\\
364	0.0099681859400543\\
365	0.00996816504783079\\
366	0.00996814379561519\\
367	0.00996812217719412\\
368	0.00996810018624911\\
369	0.00996807781635557\\
370	0.00996805506098307\\
371	0.00996803191349778\\
372	0.00996800836716694\\
373	0.00996798441516115\\
374	0.00996796005054027\\
375	0.00996793526619519\\
376	0.00996791005473633\\
377	0.00996788440846369\\
378	0.00996785831972465\\
379	0.00996783178076111\\
380	0.00996780478369372\\
381	0.00996777732058907\\
382	0.0099677493836164\\
383	0.00996772096535449\\
384	0.00996769205922427\\
385	0.00996766265946431\\
386	0.00996763275824557\\
387	0.0099676023368896\\
388	0.00996757138155168\\
389	0.00996753988143256\\
390	0.00996750782569727\\
391	0.00996747520304785\\
392	0.00996744200167978\\
393	0.00996740820923558\\
394	0.00996737381275691\\
395	0.00996733879863829\\
396	0.00996730315259064\\
397	0.00996726685963096\\
398	0.00996722990415092\\
399	0.00996719227016634\\
400	0.00996715394200545\\
401	0.00996711490592282\\
402	0.00996707515313806\\
403	0.00996703468212015\\
404	0.00996699348243015\\
405	0.00996695146777249\\
406	0.00996690860947114\\
407	0.00996686488848824\\
408	0.0099668202829035\\
409	0.00996677476655935\\
410	0.00996672830918364\\
411	0.00996668088824669\\
412	0.00996663252183162\\
413	0.00996658319375107\\
414	0.0099665328798882\\
415	0.00996648155525311\\
416	0.00996642919398741\\
417	0.00996637576941329\\
418	0.00996632125413397\\
419	0.00996626562011802\\
420	0.00996620883848122\\
421	0.00996615087834981\\
422	0.00996609170467166\\
423	0.00996603127879021\\
424	0.00996596956358955\\
425	0.00996590652082387\\
426	0.00996584211038325\\
427	0.00996577629015359\\
428	0.00996570901586141\\
429	0.00996564024089977\\
430	0.00996556991612889\\
431	0.00996549798963701\\
432	0.00996542440642666\\
433	0.00996534910793691\\
434	0.00996527203116697\\
435	0.00996519310679865\\
436	0.00996511225496241\\
437	0.00996502937706096\\
438	0.00996494435245766\\
439	0.0099648571154515\\
440	0.00996476764919769\\
441	0.0099646758685781\\
442	0.00996458166758121\\
443	0.00996448493022034\\
444	0.00996438552820102\\
445	0.00996428331792217\\
446	0.00996417813876847\\
447	0.00996406982018708\\
448	0.00996395820526764\\
449	0.00996384310879453\\
450	0.00996372431229352\\
451	0.00996360156980491\\
452	0.00996347459852649\\
453	0.00996334303737268\\
454	0.00996320627394929\\
455	0.00996306276019777\\
456	0.0099629082738575\\
457	0.00996272160641869\\
458	0.00996183313692551\\
459	0.00996070904071886\\
460	0.00995951123324781\\
461	0.00995829345035272\\
462	0.00995705525002235\\
463	0.00995579604567714\\
464	0.00995451605156727\\
465	0.00995321492125422\\
466	0.00995189217067572\\
467	0.00995054729742794\\
468	0.00994917977205121\\
469	0.00994778901926564\\
470	0.0099463744863286\\
471	0.00994493653174981\\
472	0.00994347522703532\\
473	0.00994198965175815\\
474	0.00994047858563699\\
475	0.00993894072108857\\
476	0.00993737472282705\\
477	0.00993577929934507\\
478	0.00993415176582555\\
479	0.00993248904722541\\
480	0.00993078956095074\\
481	0.00992905159421222\\
482	0.00992727297102582\\
483	0.00992545098097248\\
484	0.00992358152455413\\
485	0.00992165486828391\\
486	0.00991903208414031\\
487	0.00991417478746761\\
488	0.00990921257372146\\
489	0.00990413980369272\\
490	0.00989895150439706\\
491	0.00989364231171935\\
492	0.00988820558811218\\
493	0.0098826437066687\\
494	0.00987700135821326\\
495	0.00987121387196477\\
496	0.00986527323551472\\
497	0.00985917101724499\\
498	0.00985289670902821\\
499	0.00984643355755357\\
500	0.00983976795828853\\
501	0.00983288524363733\\
502	0.00982576869275228\\
503	0.00981839896031053\\
504	0.00981075281302382\\
505	0.00980279932324318\\
506	0.00979451444131653\\
507	0.00978591827079131\\
508	0.00977698445580002\\
509	0.00973195803472746\\
510	0.00965105588747158\\
511	0.00956811607322863\\
512	0.0094830290924446\\
513	0.00939567396480019\\
514	0.00930591449011373\\
515	0.00921360188318442\\
516	0.00911855396417855\\
517	0.00902057789311013\\
518	0.00891946808231931\\
519	0.00881499712801225\\
520	0.00870691226604886\\
521	0.00859494030878527\\
522	0.00847876237623706\\
523	0.0083580246132408\\
524	0.00823233644397344\\
525	0.0081012817910472\\
526	0.00796438806893423\\
527	0.0078219666668102\\
528	0.00767259631689416\\
529	0.00751537094821622\\
530	0.00734937101066814\\
531	0.00717352887574939\\
532	0.00698838745773369\\
533	0.00680199976078023\\
534	0.00661369802893454\\
535	0.00642052120659462\\
536	0.00622223797007795\\
537	0.00601859692696998\\
538	0.00580931975662197\\
539	0.00559409486758147\\
540	0.00537257221695576\\
541	0.0051467254174597\\
542	0.00503134269814583\\
543	0.00491423520334139\\
544	0.00479541795300435\\
545	0.00467512378405048\\
546	0.00455364026061907\\
547	0.00443132099421412\\
548	0.00430862748349586\\
549	0.00418616509956596\\
550	0.00406472203417226\\
551	0.00394531883637027\\
552	0.00382927076407412\\
553	0.00371826159400786\\
554	0.00360615050474313\\
555	0.00349161200839405\\
556	0.00337507055368945\\
557	0.00325756846643711\\
558	0.00314106624760566\\
559	0.00302590863410745\\
560	0.00291248032569543\\
561	0.00280120628054289\\
562	0.00269255205987361\\
563	0.00258702102201481\\
564	0.00248514913993936\\
565	0.00238748317032211\\
566	0.00229440259723362\\
567	0.00220660880533274\\
568	0.0021222859857676\\
569	0.0020401299752841\\
570	0.00195865682970082\\
571	0.00187781943912992\\
572	0.00179745585614607\\
573	0.00171731732385599\\
574	0.00163747296345394\\
575	0.00155798189225075\\
576	0.00147888770982828\\
577	0.00140021641993085\\
578	0.00132197584547392\\
579	0.00124415743122899\\
580	0.00116673974565477\\
581	0.00108970039256183\\
582	0.00101315452783778\\
583	0.0009372510557956\\
584	0.000862139893158467\\
585	0.000787968575672158\\
586	0.000714878431535376\\
587	0.000642998962491715\\
588	0.000572441638192295\\
589	0.000503293309197684\\
590	0.000435609150453159\\
591	0.000369387784493793\\
592	0.00030456560169281\\
593	0.000241014280104653\\
594	0.000178532861776521\\
595	0.000116902108299214\\
596	6.17886640995154e-05\\
597	2.28062284332058e-05\\
598	2.9204464504877e-07\\
599	0\\
600	0\\
};
\addplot [color=mycolor17,solid,forget plot]
  table[row sep=crcr]{%
1	0.00993093460416607\\
2	0.00993093371141091\\
3	0.00993093280249765\\
4	0.00993093187713434\\
5	0.0099309309350238\\
6	0.00993092997586345\\
7	0.0099309289993453\\
8	0.0099309280051558\\
9	0.00993092699297573\\
10	0.00993092596248016\\
11	0.00993092491333828\\
12	0.00993092384521334\\
13	0.00993092275776252\\
14	0.00993092165063682\\
15	0.00993092052348097\\
16	0.0099309193759333\\
17	0.00993091820762563\\
18	0.00993091701818316\\
19	0.00993091580722433\\
20	0.00993091457436074\\
21	0.00993091331919698\\
22	0.00993091204133054\\
23	0.00993091074035165\\
24	0.00993090941584318\\
25	0.00993090806738052\\
26	0.00993090669453139\\
27	0.00993090529685576\\
28	0.00993090387390566\\
29	0.00993090242522511\\
30	0.00993090095034989\\
31	0.00993089944880746\\
32	0.00993089792011679\\
33	0.00993089636378817\\
34	0.00993089477932314\\
35	0.00993089316621425\\
36	0.00993089152394494\\
37	0.00993088985198937\\
38	0.00993088814981226\\
39	0.00993088641686871\\
40	0.00993088465260403\\
41	0.00993088285645359\\
42	0.00993088102784259\\
43	0.00993087916618593\\
44	0.00993087727088801\\
45	0.00993087534134252\\
46	0.00993087337693229\\
47	0.00993087137702905\\
48	0.00993086934099328\\
49	0.00993086726817395\\
50	0.0099308651579084\\
51	0.00993086300952203\\
52	0.00993086082232817\\
53	0.00993085859562783\\
54	0.00993085632870949\\
55	0.00993085402084884\\
56	0.00993085167130862\\
57	0.00993084927933834\\
58	0.00993084684417404\\
59	0.00993084436503808\\
60	0.00993084184113889\\
61	0.0099308392716707\\
62	0.0099308366558133\\
63	0.0099308339927318\\
64	0.00993083128157635\\
65	0.00993082852148184\\
66	0.00993082571156773\\
67	0.00993082285093764\\
68	0.00993081993867919\\
69	0.00993081697386364\\
70	0.00993081395554562\\
71	0.00993081088276285\\
72	0.0099308077545358\\
73	0.00993080456986743\\
74	0.00993080132774285\\
75	0.00993079802712901\\
76	0.00993079466697437\\
77	0.00993079124620859\\
78	0.00993078776374216\\
79	0.0099307842184661\\
80	0.00993078060925159\\
81	0.00993077693494961\\
82	0.00993077319439063\\
83	0.00993076938638418\\
84	0.0099307655097185\\
85	0.00993076156316018\\
86	0.00993075754545378\\
87	0.0099307534553214\\
88	0.00993074929146231\\
89	0.00993074505255253\\
90	0.00993074073724446\\
91	0.00993073634416636\\
92	0.00993073187192206\\
93	0.0099307273190904\\
94	0.00993072268422486\\
95	0.00993071796585309\\
96	0.00993071316247644\\
97	0.00993070827256952\\
98	0.00993070329457969\\
99	0.00993069822692661\\
100	0.00993069306800171\\
101	0.00993068781616775\\
102	0.00993068246975825\\
103	0.00993067702707699\\
104	0.0099306714863975\\
105	0.0099306658459625\\
106	0.00993066010398338\\
107	0.00993065425863961\\
108	0.00993064830807821\\
109	0.00993064225041314\\
110	0.00993063608372475\\
111	0.00993062980605913\\
112	0.00993062341542758\\
113	0.00993061690980591\\
114	0.00993061028713389\\
115	0.00993060354531454\\
116	0.00993059668221352\\
117	0.00993058969565846\\
118	0.00993058258343829\\
119	0.00993057534330254\\
120	0.00993056797296064\\
121	0.00993056047008123\\
122	0.00993055283229143\\
123	0.00993054505717609\\
124	0.00993053714227705\\
125	0.0099305290850924\\
126	0.00993052088307564\\
127	0.00993051253363497\\
128	0.00993050403413244\\
129	0.00993049538188314\\
130	0.00993048657415436\\
131	0.00993047760816477\\
132	0.00993046848108353\\
133	0.00993045919002943\\
134	0.00993044973206998\\
135	0.00993044010422053\\
136	0.00993043030344332\\
137	0.00993042032664654\\
138	0.00993041017068338\\
139	0.00993039983235107\\
140	0.00993038930838985\\
141	0.00993037859548201\\
142	0.00993036769025081\\
143	0.00993035658925947\\
144	0.00993034528901009\\
145	0.00993033378594258\\
146	0.00993032207643354\\
147	0.00993031015679516\\
148	0.00993029802327405\\
149	0.0099302856720501\\
150	0.00993027309923528\\
151	0.00993026030087247\\
152	0.00993024727293417\\
153	0.00993023401132132\\
154	0.00993022051186199\\
155	0.00993020677031011\\
156	0.00993019278234413\\
157	0.0099301785435657\\
158	0.0099301640494983\\
159	0.00993014929558586\\
160	0.00993013427719133\\
161	0.00993011898959527\\
162	0.00993010342799436\\
163	0.00993008758749992\\
164	0.0099300714631364\\
165	0.00993005504983984\\
166	0.00993003834245628\\
167	0.00993002133574018\\
168	0.00993000402435278\\
169	0.00992998640286044\\
170	0.00992996846573298\\
171	0.00992995020734192\\
172	0.00992993162195879\\
173	0.00992991270375326\\
174	0.00992989344679144\\
175	0.00992987384503395\\
176	0.00992985389233406\\
177	0.00992983358243582\\
178	0.00992981290897206\\
179	0.00992979186546243\\
180	0.00992977044531139\\
181	0.00992974864180616\\
182	0.0099297264481146\\
183	0.0099297038572831\\
184	0.00992968086223442\\
185	0.00992965745576546\\
186	0.00992963363054504\\
187	0.00992960937911158\\
188	0.0099295846938708\\
189	0.00992955956709332\\
190	0.00992953399091226\\
191	0.00992950795732077\\
192	0.00992948145816955\\
193	0.00992945448516424\\
194	0.00992942702986291\\
195	0.00992939908367331\\
196	0.00992937063785027\\
197	0.00992934168349288\\
198	0.00992931221154176\\
199	0.00992928221277616\\
200	0.00992925167781109\\
201	0.00992922059709437\\
202	0.00992918896090359\\
203	0.00992915675934309\\
204	0.00992912398234081\\
205	0.00992909061964512\\
206	0.00992905666082161\\
207	0.00992902209524973\\
208	0.00992898691211951\\
209	0.00992895110042806\\
210	0.00992891464897615\\
211	0.00992887754636463\\
212	0.00992883978099082\\
213	0.00992880134104482\\
214	0.00992876221450577\\
215	0.00992872238913799\\
216	0.00992868185248715\\
217	0.00992864059187624\\
218	0.00992859859440157\\
219	0.00992855584692861\\
220	0.00992851233608782\\
221	0.00992846804827039\\
222	0.00992842296962383\\
223	0.00992837708604758\\
224	0.00992833038318847\\
225	0.0099282828464361\\
226	0.00992823446091814\\
227	0.00992818521149559\\
228	0.00992813508275784\\
229	0.00992808405901774\\
230	0.00992803212430653\\
231	0.00992797926236865\\
232	0.00992792545665653\\
233	0.00992787069032519\\
234	0.0099278149462268\\
235	0.0099277582069051\\
236	0.00992770045458973\\
237	0.00992764167119046\\
238	0.00992758183829127\\
239	0.00992752093714438\\
240	0.0099274589486641\\
241	0.00992739585342058\\
242	0.00992733163163352\\
243	0.00992726626316561\\
244	0.00992719972751594\\
245	0.00992713200381332\\
246	0.00992706307080935\\
247	0.00992699290687146\\
248	0.00992692148997578\\
249	0.00992684879769986\\
250	0.00992677480721526\\
251	0.00992669949528001\\
252	0.0099266228382309\\
253	0.00992654481197564\\
254	0.00992646539198487\\
255	0.00992638455328397\\
256	0.00992630227044479\\
257	0.00992621851757716\\
258	0.00992613326832022\\
259	0.00992604649583368\\
260	0.00992595817278878\\
261	0.00992586827135918\\
262	0.0099257767632116\\
263	0.00992568361949633\\
264	0.00992558881083752\\
265	0.0099254923073233\\
266	0.0099253940784957\\
267	0.0099252940933403\\
268	0.00992519232027576\\
269	0.00992508872714296\\
270	0.00992498328119394\\
271	0.00992487594908067\\
272	0.00992476669684414\\
273	0.00992465548990483\\
274	0.00992454229305393\\
275	0.00992442707043759\\
276	0.00992430978553932\\
277	0.00992419040117137\\
278	0.0099240688794629\\
279	0.00992394518184714\\
280	0.00992381926904825\\
281	0.00992369110106802\\
282	0.00992356063717229\\
283	0.00992342783587714\\
284	0.00992329265493487\\
285	0.00992315505131967\\
286	0.00992301498121313\\
287	0.00992287239998948\\
288	0.00992272726220057\\
289	0.00992257952156063\\
290	0.00992242913093084\\
291	0.00992227604230362\\
292	0.00992212020678668\\
293	0.00992196157458697\\
294	0.00992180009499428\\
295	0.00992163571636469\\
296	0.00992146838610395\\
297	0.00992129805065047\\
298	0.00992112465545834\\
299	0.00992094814498012\\
300	0.00992076846264954\\
301	0.00992058555086411\\
302	0.00992039935096766\\
303	0.00992020980323288\\
304	0.00992001684684393\\
305	0.00991982041987912\\
306	0.00991962045929402\\
307	0.00991941690090489\\
308	0.00991920967937278\\
309	0.00991899872818766\\
310	0.00991878397965054\\
311	0.00991856536484889\\
312	0.00991834281362408\\
313	0.0099181162545567\\
314	0.00991788561501206\\
315	0.00991765082109772\\
316	0.00991741179764503\\
317	0.0099171684681984\\
318	0.00991692075500623\\
319	0.00991666857901346\\
320	0.00991641185985619\\
321	0.0099161505158586\\
322	0.00991588446403243\\
323	0.00991561362007958\\
324	0.00991533789839802\\
325	0.00991505721209155\\
326	0.009914771472984\\
327	0.0099144805916381\\
328	0.00991418447737987\\
329	0.00991388303832891\\
330	0.00991357618143524\\
331	0.00991326381252325\\
332	0.00991294583634334\\
333	0.00991262215663096\\
334	0.00991229267617106\\
335	0.00991195729685783\\
336	0.00991161591971723\\
337	0.00991126844480942\\
338	0.00991091477091414\\
339	0.00991055479563033\\
340	0.00991018842127026\\
341	0.00990981555678587\\
342	0.00990943610069714\\
343	0.00990904994924596\\
344	0.00990865699838769\\
345	0.00990825714382186\\
346	0.00990785028089197\\
347	0.00990743630395627\\
348	0.00990701510394328\\
349	0.00990658656299084\\
350	0.00990615059228333\\
351	0.00990570710632888\\
352	0.00990525597935747\\
353	0.00990479707769154\\
354	0.00990433026527094\\
355	0.00990385540359984\\
356	0.00990337235174144\\
357	0.00990288096645583\\
358	0.0099023811023399\\
359	0.00990187261105522\\
360	0.00990135534156717\\
361	0.00990082914015778\\
362	0.00990029385036582\\
363	0.00989974931292212\\
364	0.00989919536567946\\
365	0.00989863184353588\\
366	0.00989805857835061\\
367	0.00989747539885124\\
368	0.00989688213053113\\
369	0.00989627859553574\\
370	0.00989566461253761\\
371	0.0098950399966002\\
372	0.00989440455903375\\
373	0.00989375810724723\\
374	0.00989310044459479\\
375	0.0098924313701728\\
376	0.00989175067839197\\
377	0.00989105815796495\\
378	0.00989035359062928\\
379	0.00988963675395991\\
380	0.00988890741967059\\
381	0.00988816535283248\\
382	0.00988741031139915\\
383	0.00988664204585051\\
384	0.00988586029923622\\
385	0.00988506480802466\\
386	0.00988425530285592\\
387	0.00988343149816776\\
388	0.00988259302418796\\
389	0.00988173954720821\\
390	0.00988087074932679\\
391	0.00987998630199731\\
392	0.00987908586177459\\
393	0.00987816906908797\\
394	0.0098772355469642\\
395	0.0098762848997191\\
396	0.00987531671164377\\
397	0.00987433054571894\\
398	0.00987332594240293\\
399	0.00987230241856791\\
400	0.0098712594667651\\
401	0.00987019655538514\\
402	0.0098691131315321\\
403	0.00986800863167006\\
404	0.0098668825074114\\
405	0.00986573420438174\\
406	0.00986456268705081\\
407	0.00986336742889049\\
408	0.00986214796778733\\
409	0.00986090383596427\\
410	0.00985963455651705\\
411	0.00985833962749916\\
412	0.00985701855127223\\
413	0.00985567125008841\\
414	0.00985429741320922\\
415	0.00985289638097873\\
416	0.0098514674683948\\
417	0.00985000996371651\\
418	0.00984852312713555\\
419	0.00984700618967824\\
420	0.00984545835245872\\
421	0.00984387878557286\\
422	0.00984226662243417\\
423	0.00984062093994073\\
424	0.00983894075034871\\
425	0.00983722504644191\\
426	0.00983547277651676\\
427	0.00983368283629492\\
428	0.00983185406482189\\
429	0.0098299852399367\\
430	0.00982807507325984\\
431	0.00982612220464185\\
432	0.00982412519601059\\
433	0.00982208252454507\\
434	0.00981999257505293\\
435	0.00981785363117089\\
436	0.00981566386388784\\
437	0.00981342131164205\\
438	0.00981112383340522\\
439	0.00980876901406392\\
440	0.00980635468891059\\
441	0.00980387887236951\\
442	0.00980133881930461\\
443	0.00979873145935177\\
444	0.00979605343638203\\
445	0.0097933010641487\\
446	0.00979047026815278\\
447	0.00978755650962274\\
448	0.00978455472915414\\
449	0.00978145956802953\\
450	0.00977826493401206\\
451	0.00977496387266974\\
452	0.0097715485274978\\
453	0.00976801002983891\\
454	0.00976433841792072\\
455	0.00976052260274695\\
456	0.00975654908188087\\
457	0.00975187337510829\\
458	0.00971722508156654\\
459	0.00968202931715126\\
460	0.00964609828519901\\
461	0.00960935424366499\\
462	0.00957177471230639\\
463	0.00953333699360045\\
464	0.00949401797492805\\
465	0.00945379720402719\\
466	0.00941264977884392\\
467	0.0093705515138918\\
468	0.00932747989843\\
469	0.00928341523661596\\
470	0.00923834220907166\\
471	0.00919225152495966\\
472	0.00914514529375721\\
473	0.00909702209437552\\
474	0.00904784884878742\\
475	0.00899758273021712\\
476	0.00894617829931817\\
477	0.00889358731911783\\
478	0.0088397592487368\\
479	0.00878463786755986\\
480	0.00872817097275927\\
481	0.00867031628596089\\
482	0.0086110286693143\\
483	0.00855026144771242\\
484	0.00848797038646977\\
485	0.0084241215869838\\
486	0.00835933739897901\\
487	0.00829516852057658\\
488	0.00822936249567639\\
489	0.0081618265298163\\
490	0.00809245455317616\\
491	0.00802114020293189\\
492	0.00794776691462567\\
493	0.00787220486253284\\
494	0.00779430110599939\\
495	0.00771388558172433\\
496	0.00763076347031366\\
497	0.00754470957221899\\
498	0.00745546162674872\\
499	0.00736270831820382\\
500	0.00726607771570519\\
501	0.00716519142095032\\
502	0.00705961055299592\\
503	0.00694882431976277\\
504	0.00683223712188952\\
505	0.00670915556106427\\
506	0.00657876271387543\\
507	0.0064443686339057\\
508	0.00630665936193442\\
509	0.0062021269716531\\
510	0.00613081217928649\\
511	0.00605834931285408\\
512	0.0059847585457202\\
513	0.00591006788827003\\
514	0.00583431268416269\\
515	0.00575753028797165\\
516	0.00567973480983292\\
517	0.00560081025772098\\
518	0.00552075419464518\\
519	0.0054396348244347\\
520	0.00535754050306231\\
521	0.00527458362688283\\
522	0.00519090733902166\\
523	0.00510669232669511\\
524	0.00502216522608326\\
525	0.0049376084664734\\
526	0.00485337485200936\\
527	0.00476988823244741\\
528	0.0046876916705995\\
529	0.00460746011895404\\
530	0.00453002778502993\\
531	0.00445642443236886\\
532	0.00438606543318127\\
533	0.00431411991300258\\
534	0.0042405441121292\\
535	0.00416536101625191\\
536	0.00408859738768097\\
537	0.00401031416093105\\
538	0.00393061848213843\\
539	0.00384967953737994\\
540	0.00376774944707301\\
541	0.00368515450614076\\
542	0.00360059345075749\\
543	0.00351426419829046\\
544	0.00342647862209044\\
545	0.00333769703700908\\
546	0.00324979024912032\\
547	0.00316349059199877\\
548	0.00307903376118264\\
549	0.00299664663417148\\
550	0.00291653241851946\\
551	0.00283884930907833\\
552	0.00276368032716527\\
553	0.0026909913567225\\
554	0.00262089551155703\\
555	0.00255348050617069\\
556	0.00248871242420521\\
557	0.00242593103386591\\
558	0.00236349281322508\\
559	0.00230141454744795\\
560	0.00223970032677368\\
561	0.00217835547665538\\
562	0.00211736633243854\\
563	0.00205669315196962\\
564	0.00199626632061458\\
565	0.00193598367096274\\
566	0.00187571269089403\\
567	0.0018152830983602\\
568	0.00175454130039174\\
569	0.00169338982056584\\
570	0.00163182870759884\\
571	0.0015698585537607\\
572	0.00150748492972648\\
573	0.0014447228627418\\
574	0.00138158723222657\\
575	0.00131809357034907\\
576	0.00125425925685432\\
577	0.00119010506498894\\
578	0.00112565699842713\\
579	0.00106094824693509\\
580	0.000996021033245017\\
581	0.000930927885436012\\
582	0.000865723636385916\\
583	0.000800463003868688\\
584	0.000735200717684098\\
585	0.000669991619104265\\
586	0.000604865722874402\\
587	0.000539844839955322\\
588	0.000474935038415231\\
589	0.000410184885592955\\
590	0.000347278553854741\\
591	0.000287502621334158\\
592	0.000231419174092323\\
593	0.000179467143924426\\
594	0.000132079920923583\\
595	9.10093979249611e-05\\
596	5.42660945238511e-05\\
597	2.28062284332058e-05\\
598	2.9204464504877e-07\\
599	0\\
600	0\\
};
\addplot [color=mycolor18,solid,forget plot]
  table[row sep=crcr]{%
1	0.00935388246113559\\
2	0.00935387188018864\\
3	0.00935386110766325\\
4	0.00935385014009582\\
5	0.0093538389739603\\
6	0.00935382760566703\\
7	0.00935381603156166\\
8	0.00935380424792389\\
9	0.00935379225096639\\
10	0.00935378003683355\\
11	0.0093537676016002\\
12	0.00935375494127047\\
13	0.00935374205177643\\
14	0.00935372892897686\\
15	0.00935371556865589\\
16	0.00935370196652169\\
17	0.00935368811820508\\
18	0.00935367401925817\\
19	0.00935365966515292\\
20	0.00935364505127973\\
21	0.00935363017294598\\
22	0.00935361502537449\\
23	0.00935359960370209\\
24	0.00935358390297799\\
25	0.00935356791816226\\
26	0.00935355164412422\\
27	0.00935353507564083\\
28	0.009353518207395\\
29	0.00935350103397394\\
30	0.00935348354986742\\
31	0.00935346574946604\\
32	0.00935344762705944\\
33	0.00935342917683452\\
34	0.00935341039287356\\
35	0.00935339126915239\\
36	0.00935337179953844\\
37	0.00935335197778883\\
38	0.00935333179754839\\
39	0.00935331125234764\\
40	0.00935329033560077\\
41	0.00935326904060352\\
42	0.00935324736053109\\
43	0.00935322528843596\\
44	0.00935320281724575\\
45	0.00935317993976088\\
46	0.00935315664865243\\
47	0.00935313293645972\\
48	0.00935310879558799\\
49	0.00935308421830602\\
50	0.00935305919674368\\
51	0.00935303372288943\\
52	0.00935300778858781\\
53	0.00935298138553686\\
54	0.00935295450528552\\
55	0.00935292713923094\\
56	0.00935289927861579\\
57	0.00935287091452547\\
58	0.00935284203788535\\
59	0.00935281263945787\\
60	0.00935278270983964\\
61	0.00935275223945848\\
62	0.00935272121857043\\
63	0.00935268963725665\\
64	0.00935265748542032\\
65	0.00935262475278343\\
66	0.00935259142888359\\
67	0.00935255750307072\\
68	0.00935252296450371\\
69	0.00935248780214698\\
70	0.00935245200476705\\
71	0.00935241556092898\\
72	0.00935237845899281\\
73	0.00935234068710986\\
74	0.00935230223321906\\
75	0.0093522630850431\\
76	0.00935222323008464\\
77	0.00935218265562232\\
78	0.00935214134870684\\
79	0.00935209929615682\\
80	0.00935205648455474\\
81	0.00935201290024265\\
82	0.00935196852931797\\
83	0.00935192335762906\\
84	0.00935187737077081\\
85	0.00935183055408016\\
86	0.00935178289263141\\
87	0.00935173437123165\\
88	0.00935168497441593\\
89	0.00935163468644244\\
90	0.00935158349128757\\
91	0.00935153137264089\\
92	0.00935147831390006\\
93	0.00935142429816562\\
94	0.00935136930823569\\
95	0.00935131332660059\\
96	0.00935125633543739\\
97	0.00935119831660427\\
98	0.00935113925163492\\
99	0.0093510791217327\\
100	0.00935101790776482\\
101	0.00935095559025627\\
102	0.00935089214938385\\
103	0.00935082756496984\\
104	0.0093507618164758\\
105	0.00935069488299613\\
106	0.00935062674325145\\
107	0.00935055737558208\\
108	0.00935048675794118\\
109	0.0093504148678879\\
110	0.00935034168258037\\
111	0.00935026717876854\\
112	0.00935019133278695\\
113	0.00935011412054733\\
114	0.00935003551753107\\
115	0.0093499554987816\\
116	0.00934987403889654\\
117	0.00934979111201982\\
118	0.00934970669183359\\
119	0.00934962075155001\\
120	0.0093495332639029\\
121	0.00934944420113922\\
122	0.00934935353501042\\
123	0.00934926123676361\\
124	0.00934916727713261\\
125	0.00934907162632883\\
126	0.00934897425403195\\
127	0.00934887512938047\\
128	0.00934877422096211\\
129	0.00934867149680399\\
130	0.00934856692436265\\
131	0.00934846047051393\\
132	0.00934835210154262\\
133	0.00934824178313197\\
134	0.00934812948035296\\
135	0.00934801515765343\\
136	0.00934789877884699\\
137	0.00934778030710172\\
138	0.00934765970492871\\
139	0.00934753693417037\\
140	0.00934741195598852\\
141	0.00934728473085229\\
142	0.00934715521852576\\
143	0.00934702337805546\\
144	0.00934688916775761\\
145	0.00934675254520502\\
146	0.00934661346721399\\
147	0.00934647188983075\\
148	0.00934632776831779\\
149	0.00934618105713989\\
150	0.00934603170994993\\
151	0.00934587967957439\\
152	0.00934572491799871\\
153	0.00934556737635221\\
154	0.00934540700489291\\
155	0.00934524375299197\\
156	0.0093450775691179\\
157	0.00934490840082048\\
158	0.00934473619471431\\
159	0.00934456089646227\\
160	0.00934438245075841\\
161	0.00934420080131076\\
162	0.00934401589082371\\
163	0.00934382766098008\\
164	0.00934363605242294\\
165	0.00934344100473701\\
166	0.00934324245642976\\
167	0.00934304034491221\\
168	0.0093428346064793\\
169	0.00934262517628997\\
170	0.0093424119883469\\
171	0.00934219497547575\\
172	0.00934197406930419\\
173	0.00934174920024043\\
174	0.00934152029745143\\
175	0.00934128728884067\\
176	0.00934105010102554\\
177	0.0093408086593143\\
178	0.00934056288768263\\
179	0.00934031270874978\\
180	0.0093400580437542\\
181	0.00933979881252881\\
182	0.0093395349334758\\
183	0.00933926632354093\\
184	0.00933899289818741\\
185	0.00933871457136921\\
186	0.00933843125550403\\
187	0.00933814286144563\\
188	0.0093378492984557\\
189	0.00933755047417525\\
190	0.00933724629459542\\
191	0.00933693666402776\\
192	0.00933662148507399\\
193	0.00933630065859514\\
194	0.0093359740836802\\
195	0.00933564165761412\\
196	0.00933530327584523\\
197	0.00933495883195209\\
198	0.00933460821760966\\
199	0.00933425132255488\\
200	0.00933388803455162\\
201	0.00933351823935491\\
202	0.00933314182067458\\
203	0.00933275866013817\\
204	0.00933236863725313\\
205	0.00933197162936836\\
206	0.00933156751163497\\
207	0.00933115615696634\\
208	0.00933073743599738\\
209	0.00933031121704311\\
210	0.00932987736605633\\
211	0.00932943574658456\\
212	0.00932898621972619\\
213	0.0093285286440857\\
214	0.00932806287572813\\
215	0.00932758876813258\\
216	0.00932710617214492\\
217	0.00932661493592952\\
218	0.00932611490492008\\
219	0.00932560592176952\\
220	0.0093250878262989\\
221	0.00932456045544534\\
222	0.00932402364320896\\
223	0.0093234772205988\\
224	0.00932292101557764\\
225	0.00932235485300581\\
226	0.00932177855458387\\
227	0.0093211919387942\\
228	0.00932059482084144\\
229	0.00931998701259174\\
230	0.00931936832251089\\
231	0.00931873855560115\\
232	0.00931809751333696\\
233	0.00931744499359923\\
234	0.00931678079060848\\
235	0.00931610469485662\\
236	0.00931541649303729\\
237	0.00931471596797501\\
238	0.00931400289855272\\
239	0.00931327705963808\\
240	0.00931253822200812\\
241	0.00931178615227254\\
242	0.00931102061279544\\
243	0.00931024136161539\\
244	0.00930944815236406\\
245	0.00930864073418312\\
246	0.00930781885163949\\
247	0.0093069822446389\\
248	0.0093061306483377\\
249	0.0093052637930529\\
250	0.00930438140417041\\
251	0.0093034832020513\\
252	0.00930256890193639\\
253	0.00930163821384867\\
254	0.00930069084249383\\
255	0.0092997264871588\\
256	0.00929874484160814\\
257	0.00929774559397829\\
258	0.00929672842666973\\
259	0.00929569301623684\\
260	0.00929463903327553\\
261	0.00929356614230848\\
262	0.00929247400166804\\
263	0.00929136226337673\\
264	0.00929023057302517\\
265	0.00928907856964746\\
266	0.00928790588559415\\
267	0.00928671214640243\\
268	0.00928549697066392\\
269	0.00928425996989008\\
270	0.00928300074837557\\
271	0.00928171890306\\
272	0.00928041402338833\\
273	0.00927908569116927\\
274	0.00927773348042881\\
275	0.00927635695725176\\
276	0.0092749556795893\\
277	0.00927352919708048\\
278	0.00927207705092298\\
279	0.00927059877371125\\
280	0.00926909388926583\\
281	0.00926756191245897\\
282	0.00926600234903667\\
283	0.00926441469543684\\
284	0.00926279843860374\\
285	0.00926115305579856\\
286	0.00925947801440592\\
287	0.00925777277173657\\
288	0.00925603677482595\\
289	0.00925426946022866\\
290	0.0092524702538089\\
291	0.00925063857052676\\
292	0.00924877381422034\\
293	0.00924687537738379\\
294	0.00924494264094119\\
295	0.00924297497401639\\
296	0.00924097173369885\\
297	0.00923893226480545\\
298	0.00923685589963861\\
299	0.00923474195774062\\
300	0.00923258974564453\\
301	0.00923039855662175\\
302	0.00922816767042654\\
303	0.00922589635303781\\
304	0.0092235838563983\\
305	0.00922122941815141\\
306	0.00921883226137546\\
307	0.00921639159431492\\
308	0.00921390661010708\\
309	0.00921137648650256\\
310	0.0092088003855786\\
311	0.00920617745344888\\
312	0.00920350681998642\\
313	0.00920078759859547\\
314	0.00919801888610708\\
315	0.00919519976283647\\
316	0.00919232929209615\\
317	0.00918940651995461\\
318	0.00918643047506029\\
319	0.00918340016848936\\
320	0.00918031459362137\\
321	0.00917717272604791\\
322	0.00917397352351965\\
323	0.00917071592593803\\
324	0.00916739885539842\\
325	0.00916402121629215\\
326	0.00916058189547558\\
327	0.00915707976251482\\
328	0.00915351367001522\\
329	0.00914988245404487\\
330	0.00914618493466163\\
331	0.00914241991655318\\
332	0.00913858618980006\\
333	0.00913468253077488\\
334	0.00913070770319948\\
335	0.0091266604594075\\
336	0.00912253954191899\\
337	0.00911834368555332\\
338	0.00911407162045904\\
339	0.00910972207626186\\
340	0.00910529378589186\\
341	0.00910078549893887\\
342	0.00909619595261919\\
343	0.00909152377963481\\
344	0.00908676758765\\
345	0.00908192597161524\\
346	0.00907699751383327\\
347	0.00907198078462663\\
348	0.00906687434498613\\
349	0.00906167675267459\\
350	0.00905638655439896\\
351	0.00905100241924003\\
352	0.00904552286378997\\
353	0.00903994608758106\\
354	0.00903427025068874\\
355	0.00902849347266619\\
356	0.0090226138315067\\
357	0.00901662936258298\\
358	0.00901053805740051\\
359	0.00900433786148541\\
360	0.00899802666873709\\
361	0.00899160232238428\\
362	0.00898506261350247\\
363	0.0089784052788706\\
364	0.00897162799864498\\
365	0.008964728393827\\
366	0.00895770402349778\\
367	0.00895055238178909\\
368	0.00894327089455504\\
369	0.00893585691570293\\
370	0.00892830772313262\\
371	0.00892062051422059\\
372	0.00891279240076288\\
373	0.00890482040325693\\
374	0.00889670144435709\\
375	0.00888843234130537\\
376	0.00888000979720011\\
377	0.00887143039126845\\
378	0.00886269056886668\\
379	0.00885378663281679\\
380	0.00884471474983117\\
381	0.0088354709255763\\
382	0.00882605098805673\\
383	0.00881645057115677\\
384	0.00880666509483924\\
385	0.00879668973995517\\
386	0.00878651941543478\\
387	0.00877614872036761\\
388	0.00876557191870218\\
389	0.00875478279931898\\
390	0.00874377548020158\\
391	0.00873254406419085\\
392	0.00872108240461992\\
393	0.00870938406743709\\
394	0.00869744231546698\\
395	0.00868525009351515\\
396	0.00867280001506848\\
397	0.00866008435157889\\
398	0.0086470950255808\\
399	0.00863382360912698\\
400	0.00862026132908542\\
401	0.00860639908035345\\
402	0.00859222744626044\\
403	0.00857773672126252\\
404	0.00856291692610347\\
405	0.00854775783126497\\
406	0.00853224903854124\\
407	0.00851637874887452\\
408	0.00850014017392674\\
409	0.00848352729068936\\
410	0.00846653532930649\\
411	0.00844916134249677\\
412	0.00843140494628865\\
413	0.0084132692340222\\
414	0.00839476293802955\\
415	0.00837589427803379\\
416	0.00835665512812732\\
417	0.00833703712372604\\
418	0.00831703164985451\\
419	0.00829662982675444\\
420	0.00827582249080892\\
421	0.00825460016794972\\
422	0.0082329530375656\\
423	0.00821087088992473\\
424	0.00818834308085885\\
425	0.0081653586062401\\
426	0.00814190635765207\\
427	0.00811797489406353\\
428	0.00809355238571992\\
429	0.00806862659615527\\
430	0.00804318486362214\\
431	0.00801721408211817\\
432	0.00799070068231922\\
433	0.00796363061299112\\
434	0.0079359893240296\\
435	0.00790776175364678\\
436	0.00787893232556888\\
437	0.00784948497019549\\
438	0.00781940320103334\\
439	0.00778867029193811\\
440	0.00775726930641245\\
441	0.00772518524823265\\
442	0.0076924019746264\\
443	0.00765889704039598\\
444	0.00762464640987071\\
445	0.00758962423797925\\
446	0.0075538026267833\\
447	0.00751715137017973\\
448	0.00747963769494051\\
449	0.00744122593783304\\
450	0.0074018779348395\\
451	0.00736154953242777\\
452	0.00732018995426554\\
453	0.00727774139820778\\
454	0.00723413764358201\\
455	0.0071893023992448\\
456	0.00714314881497628\\
457	0.00709611890625548\\
458	0.00707814715998119\\
459	0.00705950064542732\\
460	0.007040111390837\\
461	0.00701990586931755\\
462	0.006998799285829\\
463	0.00697669296911973\\
464	0.0069534716361377\\
465	0.00692900000368419\\
466	0.00690311877158677\\
467	0.00687563960169342\\
468	0.00684633893625431\\
469	0.00681495038705406\\
470	0.00678115540694164\\
471	0.00674457210883074\\
472	0.00670474309112749\\
473	0.00666369476105502\\
474	0.00662197457892596\\
475	0.00657957827687717\\
476	0.00653650234801006\\
477	0.00649274409124263\\
478	0.00644830169130089\\
479	0.00640317456720802\\
480	0.00635736400836668\\
481	0.0063108771065416\\
482	0.00626375221462998\\
483	0.00621601161985001\\
484	0.00616766531824011\\
485	0.00611872205674312\\
486	0.00606917904683886\\
487	0.00601901680727206\\
488	0.0059682650676501\\
489	0.00591694912971216\\
490	0.00586505640756299\\
491	0.00581251825469725\\
492	0.00575937889418669\\
493	0.00570569420173042\\
494	0.00565153429849766\\
495	0.00559698631085357\\
496	0.00554215845861785\\
497	0.00548718488524454\\
498	0.00543223164892994\\
499	0.00537750445415099\\
500	0.0053232586911779\\
501	0.00526980852182763\\
502	0.00521754039160833\\
503	0.00516692999283385\\
504	0.00511856338139452\\
505	0.00507316311871932\\
506	0.0050316199649441\\
507	0.0049905546194071\\
508	0.00494928986390594\\
509	0.00490747668529065\\
510	0.00486469649171915\\
511	0.00482092246210445\\
512	0.00477612639420774\\
513	0.00473027858039279\\
514	0.00468334734606081\\
515	0.00463529756612277\\
516	0.00458609874967004\\
517	0.00453572889551623\\
518	0.0044841677197608\\
519	0.00443139563389439\\
520	0.00437739300423209\\
521	0.00432213901410709\\
522	0.00426560983196854\\
523	0.00420777542104398\\
524	0.00414859175016062\\
525	0.00408799773907985\\
526	0.00402593207117221\\
527	0.00396233451736068\\
528	0.00389714705872905\\
529	0.00383031591407272\\
530	0.00376179331226578\\
531	0.00369153993728676\\
532	0.00361959527846768\\
533	0.0035462583284732\\
534	0.00347208389533472\\
535	0.00339908980553376\\
536	0.00332748166785388\\
537	0.00325747574193593\\
538	0.00318929462662516\\
539	0.00312316045408918\\
540	0.00305928458020588\\
541	0.00299785326245797\\
542	0.00293903789899305\\
543	0.00288294885860575\\
544	0.00282960002502755\\
545	0.00277886000119549\\
546	0.00272909952961711\\
547	0.00267981308881545\\
548	0.0026309828134207\\
549	0.00258257253401098\\
550	0.00253452574506337\\
551	0.00248676442562614\\
552	0.00243918939882931\\
553	0.00239168320911\\
554	0.00234410597758729\\
555	0.00229629841183345\\
556	0.0022480992879957\\
557	0.0021993849561593\\
558	0.00215013390444389\\
559	0.0021003217604376\\
560	0.0020499215771393\\
561	0.00199890396187714\\
562	0.00194723786732274\\
563	0.00189489179927033\\
564	0.00184183539852713\\
565	0.00178804137997534\\
566	0.00173348768596366\\
567	0.00167815996825155\\
568	0.00162205285376998\\
569	0.00156516806544259\\
570	0.00150750949614769\\
571	0.00144908358804972\\
572	0.00138989961244892\\
573	0.00132996979794916\\
574	0.00126930973751084\\
575	0.00120793876053359\\
576	0.00114588022454223\\
577	0.00108316193301972\\
578	0.00101981685596092\\
579	0.000955885432697557\\
580	0.000891416745494425\\
581	0.000826449428908736\\
582	0.00076103325331982\\
583	0.000695249245061119\\
584	0.000629112420938282\\
585	0.000564944790052522\\
586	0.000503134290179678\\
587	0.000444772794001453\\
588	0.000390227629686837\\
589	0.000340264159084332\\
590	0.000293674396315564\\
591	0.00024937252729716\\
592	0.000206953440128856\\
593	0.000166169107244796\\
594	0.000127115113967414\\
595	8.96111722547531e-05\\
596	5.42660945238511e-05\\
597	2.28062284332059e-05\\
598	2.9204464504877e-07\\
599	0\\
600	0\\
};
\addplot [color=red!25!mycolor17,solid,forget plot]
  table[row sep=crcr]{%
1	0.00752975719577589\\
2	0.00752975205343657\\
3	0.00752974681777696\\
4	0.00752974148710355\\
5	0.00752973605969214\\
6	0.00752973053378722\\
7	0.00752972490760141\\
8	0.00752971917931491\\
9	0.00752971334707491\\
10	0.00752970740899495\\
11	0.00752970136315436\\
12	0.00752969520759761\\
13	0.00752968894033371\\
14	0.00752968255933553\\
15	0.00752967606253913\\
16	0.00752966944784316\\
17	0.00752966271310816\\
18	0.00752965585615583\\
19	0.00752964887476838\\
20	0.00752964176668774\\
21	0.00752963452961493\\
22	0.00752962716120926\\
23	0.00752961965908754\\
24	0.00752961202082343\\
25	0.00752960424394652\\
26	0.00752959632594162\\
27	0.00752958826424795\\
28	0.00752958005625826\\
29	0.00752957169931803\\
30	0.00752956319072461\\
31	0.00752955452772636\\
32	0.00752954570752172\\
33	0.00752953672725836\\
34	0.00752952758403223\\
35	0.00752951827488665\\
36	0.00752950879681133\\
37	0.00752949914674141\\
38	0.0075294893215565\\
39	0.00752947931807964\\
40	0.00752946913307629\\
41	0.00752945876325331\\
42	0.00752944820525786\\
43	0.00752943745567639\\
44	0.00752942651103344\\
45	0.00752941536779064\\
46	0.00752940402234548\\
47	0.0075293924710302\\
48	0.00752938071011062\\
49	0.0075293687357849\\
50	0.00752935654418234\\
51	0.00752934413136215\\
52	0.00752933149331217\\
53	0.0075293186259476\\
54	0.00752930552510964\\
55	0.00752929218656425\\
56	0.00752927860600069\\
57	0.00752926477903021\\
58	0.00752925070118459\\
59	0.00752923636791476\\
60	0.0075292217745893\\
61	0.007529206916493\\
62	0.0075291917888253\\
63	0.00752917638669876\\
64	0.00752916070513756\\
65	0.00752914473907581\\
66	0.00752912848335602\\
67	0.00752911193272739\\
68	0.00752909508184412\\
69	0.00752907792526378\\
70	0.00752906045744548\\
71	0.00752904267274817\\
72	0.00752902456542876\\
73	0.00752900612964038\\
74	0.00752898735943041\\
75	0.00752896824873868\\
76	0.00752894879139544\\
77	0.00752892898111949\\
78	0.00752890881151607\\
79	0.00752888827607488\\
80	0.00752886736816801\\
81	0.00752884608104782\\
82	0.00752882440784475\\
83	0.00752880234156517\\
84	0.00752877987508915\\
85	0.00752875700116815\\
86	0.00752873371242279\\
87	0.0075287100013404\\
88	0.00752868586027273\\
89	0.00752866128143343\\
90	0.0075286362568956\\
91	0.0075286107785893\\
92	0.00752858483829895\\
93	0.00752855842766072\\
94	0.00752853153815989\\
95	0.00752850416112809\\
96	0.00752847628774064\\
97	0.00752844790901365\\
98	0.00752841901580124\\
99	0.00752838959879258\\
100	0.00752835964850895\\
101	0.00752832915530077\\
102	0.00752829810934449\\
103	0.00752826650063949\\
104	0.00752823431900494\\
105	0.00752820155407651\\
106	0.00752816819530313\\
107	0.00752813423194364\\
108	0.00752809965306338\\
109	0.00752806444753073\\
110	0.00752802860401361\\
111	0.00752799211097584\\
112	0.00752795495667353\\
113	0.00752791712915137\\
114	0.00752787861623882\\
115	0.00752783940554629\\
116	0.00752779948446119\\
117	0.00752775884014398\\
118	0.00752771745952413\\
119	0.00752767532929593\\
120	0.00752763243591434\\
121	0.0075275887655907\\
122	0.00752754430428839\\
123	0.00752749903771837\\
124	0.00752745295133474\\
125	0.00752740603033007\\
126	0.00752735825963083\\
127	0.00752730962389257\\
128	0.00752726010749509\\
129	0.00752720969453758\\
130	0.00752715836883358\\
131	0.00752710611390588\\
132	0.00752705291298136\\
133	0.0075269987489857\\
134	0.00752694360453804\\
135	0.00752688746194548\\
136	0.00752683030319757\\
137	0.00752677210996064\\
138	0.00752671286357203\\
139	0.00752665254503426\\
140	0.00752659113500905\\
141	0.0075265286138113\\
142	0.00752646496140288\\
143	0.00752640015738638\\
144	0.0075263341809987\\
145	0.0075262670111046\\
146	0.00752619862619003\\
147	0.00752612900435545\\
148	0.00752605812330895\\
149	0.00752598596035932\\
150	0.00752591249240895\\
151	0.0075258376959466\\
152	0.00752576154704007\\
153	0.00752568402132878\\
154	0.00752560509401615\\
155	0.00752552473986181\\
156	0.00752544293317384\\
157	0.00752535964780071\\
158	0.00752527485712319\\
159	0.00752518853404599\\
160	0.00752510065098941\\
161	0.00752501117988075\\
162	0.00752492009214558\\
163	0.00752482735869887\\
164	0.00752473294993598\\
165	0.00752463683572344\\
166	0.00752453898538968\\
167	0.00752443936771548\\
168	0.0075243379509243\\
169	0.00752423470267244\\
170	0.00752412959003905\\
171	0.00752402257951598\\
172	0.00752391363699737\\
173	0.00752380272776916\\
174	0.00752368981649834\\
175	0.0075235748672221\\
176	0.00752345784333668\\
177	0.00752333870758614\\
178	0.00752321742205086\\
179	0.00752309394813589\\
180	0.00752296824655904\\
181	0.00752284027733886\\
182	0.0075227099997823\\
183	0.00752257737247225\\
184	0.00752244235325482\\
185	0.0075223048992264\\
186	0.00752216496672052\\
187	0.0075220225112945\\
188	0.00752187748771579\\
189	0.00752172984994817\\
190	0.00752157955113768\\
191	0.00752142654359825\\
192	0.00752127077879721\\
193	0.00752111220734044\\
194	0.0075209507789573\\
195	0.00752078644248533\\
196	0.00752061914585465\\
197	0.00752044883607212\\
198	0.00752027545920519\\
199	0.00752009896036555\\
200	0.00751991928369243\\
201	0.00751973637233562\\
202	0.00751955016843826\\
203	0.00751936061311927\\
204	0.00751916764645548\\
205	0.00751897120746353\\
206	0.00751877123408139\\
207	0.00751856766314955\\
208	0.007518360430392\\
209	0.00751814947039674\\
210	0.00751793471659604\\
211	0.0075177161012464\\
212	0.00751749355540804\\
213	0.00751726700892417\\
214	0.00751703639039983\\
215	0.00751680162718042\\
216	0.00751656264532977\\
217	0.00751631936960797\\
218	0.00751607172344869\\
219	0.00751581962893617\\
220	0.00751556300678189\\
221	0.00751530177630066\\
222	0.00751503585538647\\
223	0.00751476516048784\\
224	0.00751448960658274\\
225	0.00751420910715318\\
226	0.00751392357415917\\
227	0.00751363291801246\\
228	0.00751333704754961\\
229	0.00751303587000479\\
230	0.007512729290982\\
231	0.00751241721442677\\
232	0.00751209954259747\\
233	0.00751177617603611\\
234	0.0075114470135386\\
235	0.00751111195212445\\
236	0.00751077088700609\\
237	0.00751042371155747\\
238	0.00751007031728229\\
239	0.00750971059378157\\
240	0.00750934442872073\\
241	0.00750897170779599\\
242	0.0075085923147003\\
243	0.00750820613108868\\
244	0.00750781303654287\\
245	0.00750741290853539\\
246	0.0075070056223931\\
247	0.00750659105125999\\
248	0.00750616906605933\\
249	0.0075057395354553\\
250	0.00750530232581381\\
251	0.0075048573011628\\
252	0.00750440432315164\\
253	0.00750394325101007\\
254	0.00750347394150628\\
255	0.00750299624890437\\
256	0.00750251002492099\\
257	0.0075020151186813\\
258	0.00750151137667419\\
259	0.00750099864270668\\
260	0.00750047675785759\\
261	0.00749994556043041\\
262	0.00749940488590532\\
263	0.00749885456689047\\
264	0.00749829443307233\\
265	0.00749772431116531\\
266	0.00749714402486035\\
267	0.00749655339477289\\
268	0.00749595223838985\\
269	0.00749534037001628\\
270	0.00749471760072198\\
271	0.00749408373829001\\
272	0.00749343858717124\\
273	0.0074927819484534\\
274	0.00749211361985863\\
275	0.00749143339577178\\
276	0.00749074106721274\\
277	0.00749003642144281\\
278	0.00748931924166022\\
279	0.00748858930723845\\
280	0.00748784639371893\\
281	0.00748709027274426\\
282	0.00748632071199015\\
283	0.00748553747509582\\
284	0.00748474032159292\\
285	0.00748392900683273\\
286	0.00748310328191177\\
287	0.0074822628935955\\
288	0.00748140758424024\\
289	0.00748053709171292\\
290	0.00747965114930884\\
291	0.007478749485667\\
292	0.00747783182468312\\
293	0.00747689788541996\\
294	0.00747594738201499\\
295	0.0074749800235849\\
296	0.00747399551412713\\
297	0.0074729935524178\\
298	0.00747197383190605\\
299	0.00747093604060444\\
300	0.00746987986097498\\
301	0.00746880496981065\\
302	0.00746771103811184\\
303	0.00746659773095754\\
304	0.0074654647073704\\
305	0.00746431162017531\\
306	0.00746313811585005\\
307	0.00746194383436596\\
308	0.00746072840901369\\
309	0.0074594914662036\\
310	0.00745823262521787\\
311	0.00745695149787069\\
312	0.00745564768800999\\
313	0.00745432079083502\\
314	0.00745297039232933\\
315	0.0074515960700461\\
316	0.00745019739601793\\
317	0.00744877393411322\\
318	0.00744732523925988\\
319	0.00744585085715755\\
320	0.00744435032397979\\
321	0.00744282316606823\\
322	0.00744126889962242\\
323	0.00743968703038992\\
324	0.00743807705336373\\
325	0.00743643845249687\\
326	0.00743477070044707\\
327	0.00743307325836974\\
328	0.00743134557578306\\
329	0.007429587090537\\
330	0.00742779722892859\\
331	0.00742597540601873\\
332	0.00742412102622348\\
333	0.00742223348427516\\
334	0.00742031216667975\\
335	0.00741835645384029\\
336	0.00741636572308849\\
337	0.00741433935301452\\
338	0.00741227672988319\\
339	0.00741017725826789\\
340	0.00740804038339811\\
341	0.00740586565793154\\
342	0.007403653038201\\
343	0.00740140271340758\\
344	0.00739911405810567\\
345	0.00739678643373589\\
346	0.00739441918769577\\
347	0.00739201165231525\\
348	0.00738956314374368\\
349	0.00738707296073435\\
350	0.00738454038327247\\
351	0.00738196466808564\\
352	0.00737934504954275\\
353	0.00737668074415684\\
354	0.0073739709498462\\
355	0.00737121484528177\\
356	0.00736841158958132\\
357	0.00736556032300077\\
358	0.00736266016990002\\
359	0.00735971024460689\\
360	0.00735670964922029\\
361	0.00735365742967529\\
362	0.00735055259845563\\
363	0.00734739413805547\\
364	0.00734418099895487\\
365	0.00734091209734935\\
366	0.0073375863125907\\
367	0.00733420248428568\\
368	0.00733075940898739\\
369	0.00732725583639804\\
370	0.00732369046498075\\
371	0.00732006193684797\\
372	0.00731636883174691\\
373	0.00731260965987817\\
374	0.00730878285311568\\
375	0.00730488675383421\\
376	0.00730091959977377\\
377	0.00729687950186456\\
378	0.00729276441000695\\
379	0.00728857206466934\\
380	0.00728429996771987\\
381	0.00727994554468052\\
382	0.00727550599739177\\
383	0.00727097823098026\\
384	0.00726635879220191\\
385	0.00726164378880268\\
386	0.00725682877613685\\
387	0.00725190857554885\\
388	0.00724687691834787\\
389	0.00724172555728899\\
390	0.00723644137563214\\
391	0.00723101554785491\\
392	0.00722543988330375\\
393	0.00721970529499155\\
394	0.00721380166796882\\
395	0.00720771770449372\\
396	0.00720144074128882\\
397	0.00719495653310726\\
398	0.00718824899550521\\
399	0.00718129989806737\\
400	0.00717408849727336\\
401	0.00716659109563358\\
402	0.00715878051056588\\
403	0.0071506254326964\\
404	0.00714208964907079\\
405	0.00713313110317246\\
406	0.00712370076604672\\
407	0.00711374132651605\\
408	0.00710318559509768\\
409	0.0070919547364818\\
410	0.00707995392375266\\
411	0.00706706986772846\\
412	0.00705316692827304\\
413	0.00703808231943389\\
414	0.00702167943378768\\
415	0.00700498846860179\\
416	0.0069880052658891\\
417	0.00697072568106828\\
418	0.00695314558986649\\
419	0.00693526089418565\\
420	0.00691706752383685\\
421	0.00689856142557651\\
422	0.00687973851619553\\
423	0.0068605945394869\\
424	0.00684112469188092\\
425	0.00682132287849355\\
426	0.00680118231290836\\
427	0.00678069868600467\\
428	0.00675986817327228\\
429	0.0067386870870183\\
430	0.00671715189842607\\
431	0.00669525926230584\\
432	0.00667300604491383\\
433	0.00665038935527493\\
434	0.00662740658050536\\
435	0.00660405542568729\\
436	0.00658033395887662\\
437	0.00655624066174944\\
438	0.0065317744859354\\
439	0.00650693491354474\\
440	0.00648172201726202\\
441	0.00645613645862427\\
442	0.00643017960258967\\
443	0.00640385381117351\\
444	0.00637716262060013\\
445	0.00635011095562376\\
446	0.00632270539431041\\
447	0.00629495451382842\\
448	0.00626686941319138\\
449	0.00623846479105154\\
450	0.00620976236471191\\
451	0.00618080754270315\\
452	0.0061516398098084\\
453	0.00612229921115789\\
454	0.00609283475388691\\
455	0.00606330621852963\\
456	0.00603378593943971\\
457	0.00600435309360236\\
458	0.00597472622139851\\
459	0.005944936642734\\
460	0.00591497869761773\\
461	0.00588488049490957\\
462	0.00585470720823154\\
463	0.00582453831237067\\
464	0.00579447066612896\\
465	0.00576462228241092\\
466	0.0057351371544715\\
467	0.00570619111318079\\
468	0.0056779990950215\\
469	0.00565082417820276\\
470	0.00562498881388385\\
471	0.00560088871720802\\
472	0.00557900960011686\\
473	0.00555729142567498\\
474	0.00553516701095325\\
475	0.00551262943556047\\
476	0.00548967173836112\\
477	0.00546628689209586\\
478	0.00544246779344238\\
479	0.00541820726369776\\
480	0.00539349805074807\\
481	0.00536833279070289\\
482	0.00534270362687262\\
483	0.00531660233968035\\
484	0.00529002037339372\\
485	0.00526294907004596\\
486	0.00523538001615325\\
487	0.00520730590458239\\
488	0.00517871950737512\\
489	0.00514961353753942\\
490	0.00511998114587108\\
491	0.00508981678447065\\
492	0.00505911483619851\\
493	0.00502786944273571\\
494	0.00499607435605367\\
495	0.00496372307028727\\
496	0.00493080396722983\\
497	0.00489730203712122\\
498	0.00486319740704639\\
499	0.00482846311294119\\
500	0.00479306325290543\\
501	0.00475696059332692\\
502	0.00472010551273883\\
503	0.00468243136231268\\
504	0.0046438481817106\\
505	0.00460423429291239\\
506	0.0045634250990333\\
507	0.00452134962894445\\
508	0.00447795628043695\\
509	0.00443319610689204\\
510	0.00438702368694223\\
511	0.00433939085082558\\
512	0.00429024426436863\\
513	0.00423951819484293\\
514	0.00418714193967163\\
515	0.00413305427633992\\
516	0.00407720885976529\\
517	0.00401958143675479\\
518	0.00396017969591317\\
519	0.0038990565281705\\
520	0.00383632709739516\\
521	0.00377219079310824\\
522	0.00370696148339691\\
523	0.00364150103709476\\
524	0.00357730663765718\\
525	0.003514572456412\\
526	0.003453502088954\\
527	0.00339430491184796\\
528	0.00333719045874468\\
529	0.0032823599463531\\
530	0.00322999434560389\\
531	0.00318023771765325\\
532	0.00313317193087718\\
533	0.00308877696155659\\
534	0.00304677637041605\\
535	0.00300533620988421\\
536	0.00296445707726527\\
537	0.00292412483708432\\
538	0.00288430801836305\\
539	0.00284495531333978\\
540	0.002805993459618\\
541	0.00276732591444782\\
542	0.00272883235327698\\
543	0.00269037068913318\\
544	0.00265178253040087\\
545	0.00261290383274353\\
546	0.00257366270674616\\
547	0.00253401370536589\\
548	0.00249390732631952\\
549	0.00245329074706882\\
550	0.00241210889436304\\
551	0.00237030589512094\\
552	0.00232782673648705\\
553	0.00228461998623048\\
554	0.00224065046301698\\
555	0.0021958884073572\\
556	0.00215031064061035\\
557	0.00210389946353566\\
558	0.00205663759187896\\
559	0.00200850842755855\\
560	0.00195949635800379\\
561	0.00190958708395927\\
562	0.00185876795497196\\
563	0.00180702829309604\\
564	0.00175435968293312\\
565	0.00170075620342841\\
566	0.00164621457913793\\
567	0.0015907342281127\\
568	0.00153431721326273\\
569	0.00147696824692171\\
570	0.00141869494997049\\
571	0.00135950808434287\\
572	0.00129942189673999\\
573	0.00123845564645323\\
574	0.00117663625677103\\
575	0.00111400030455689\\
576	0.00105059644939085\\
577	0.000986530157595525\\
578	0.000921762352719769\\
579	0.000856219106191558\\
580	0.000791664481391484\\
581	0.000729004586915961\\
582	0.000668600899830575\\
583	0.000611598515788096\\
584	0.000558751325735728\\
585	0.000508153571256534\\
586	0.000460174865235934\\
587	0.000414142193949086\\
588	0.000370019893892401\\
589	0.000327223396935425\\
590	0.000285420585329276\\
591	0.000244524545387522\\
592	0.000204480456411576\\
593	0.000165271055315995\\
594	0.000126870159937007\\
595	8.96111722547517e-05\\
596	5.42660945238508e-05\\
597	2.28062284332056e-05\\
598	2.9204464504877e-07\\
599	0\\
600	0\\
};
\addplot [color=mycolor19,solid,forget plot]
  table[row sep=crcr]{%
1	0.00695390139793097\\
2	0.00695389015754209\\
3	0.00695387871314316\\
4	0.00695386706103132\\
5	0.00695385519743642\\
6	0.00695384311851986\\
7	0.00695383082037335\\
8	0.00695381829901761\\
9	0.00695380555040107\\
10	0.00695379257039861\\
11	0.0069537793548102\\
12	0.00695376589935953\\
13	0.00695375219969261\\
14	0.00695373825137641\\
15	0.00695372404989741\\
16	0.0069537095906601\\
17	0.00695369486898552\\
18	0.00695367988010977\\
19	0.00695366461918243\\
20	0.00695364908126501\\
21	0.00695363326132935\\
22	0.00695361715425597\\
23	0.00695360075483249\\
24	0.00695358405775181\\
25	0.00695356705761055\\
26	0.0069535497489072\\
27	0.0069535321260403\\
28	0.00695351418330684\\
29	0.00695349591490013\\
30	0.00695347731490815\\
31	0.00695345837731154\\
32	0.00695343909598167\\
33	0.00695341946467864\\
34	0.0069533994770493\\
35	0.00695337912662519\\
36	0.00695335840682044\\
37	0.00695333731092963\\
38	0.00695331583212566\\
39	0.0069532939634575\\
40	0.00695327169784803\\
41	0.00695324902809163\\
42	0.00695322594685197\\
43	0.00695320244665956\\
44	0.00695317851990943\\
45	0.00695315415885856\\
46	0.0069531293556235\\
47	0.00695310410217771\\
48	0.0069530783903491\\
49	0.00695305221181729\\
50	0.006953025558111\\
51	0.00695299842060523\\
52	0.0069529707905186\\
53	0.00695294265891042\\
54	0.00695291401667788\\
55	0.00695288485455306\\
56	0.00695285516309997\\
57	0.00695282493271153\\
58	0.00695279415360646\\
59	0.00695276281582613\\
60	0.00695273090923136\\
61	0.00695269842349913\\
62	0.00695266534811933\\
63	0.00695263167239128\\
64	0.0069525973854204\\
65	0.00695256247611462\\
66	0.00695252693318083\\
67	0.00695249074512126\\
68	0.00695245390022984\\
69	0.00695241638658834\\
70	0.0069523781920626\\
71	0.00695233930429862\\
72	0.00695229971071861\\
73	0.00695225939851692\\
74	0.00695221835465596\\
75	0.00695217656586201\\
76	0.00695213401862091\\
77	0.00695209069917379\\
78	0.00695204659351266\\
79	0.00695200168737586\\
80	0.00695195596624356\\
81	0.00695190941533302\\
82	0.00695186201959392\\
83	0.00695181376370349\\
84	0.0069517646320617\\
85	0.00695171460878611\\
86	0.0069516636777069\\
87	0.00695161182236169\\
88	0.00695155902599015\\
89	0.00695150527152884\\
90	0.00695145054160554\\
91	0.00695139481853385\\
92	0.00695133808430745\\
93	0.00695128032059439\\
94	0.00695122150873117\\
95	0.00695116162971682\\
96	0.00695110066420685\\
97	0.00695103859250703\\
98	0.00695097539456708\\
99	0.00695091104997433\\
100	0.0069508455379472\\
101	0.00695077883732847\\
102	0.00695071092657865\\
103	0.00695064178376904\\
104	0.00695057138657477\\
105	0.00695049971226762\\
106	0.00695042673770889\\
107	0.00695035243934188\\
108	0.00695027679318451\\
109	0.0069501997748216\\
110	0.00695012135939712\\
111	0.0069500415216063\\
112	0.00694996023568754\\
113	0.00694987747541421\\
114	0.00694979321408632\\
115	0.00694970742452203\\
116	0.006949620079049\\
117	0.00694953114949558\\
118	0.00694944060718188\\
119	0.00694934842291061\\
120	0.00694925456695786\\
121	0.00694915900906359\\
122	0.00694906171842212\\
123	0.00694896266367224\\
124	0.00694886181288735\\
125	0.00694875913356523\\
126	0.00694865459261785\\
127	0.00694854815636073\\
128	0.00694843979050242\\
129	0.00694832946013351\\
130	0.00694821712971561\\
131	0.00694810276307007\\
132	0.00694798632336657\\
133	0.00694786777311141\\
134	0.00694774707413566\\
135	0.00694762418758307\\
136	0.00694749907389779\\
137	0.00694737169281185\\
138	0.00694724200333245\\
139	0.00694710996372896\\
140	0.00694697553151976\\
141	0.00694683866345879\\
142	0.00694669931552195\\
143	0.00694655744289313\\
144	0.00694641299995012\\
145	0.00694626594025021\\
146	0.00694611621651549\\
147	0.00694596378061796\\
148	0.00694580858356443\\
149	0.00694565057548093\\
150	0.00694548970559717\\
151	0.00694532592223043\\
152	0.00694515917276932\\
153	0.00694498940365725\\
154	0.00694481656037552\\
155	0.00694464058742624\\
156	0.00694446142831484\\
157	0.00694427902553231\\
158	0.00694409332053714\\
159	0.00694390425373695\\
160	0.00694371176446977\\
161	0.006943515790985\\
162	0.006943316270424\\
163	0.00694311313880048\\
164	0.00694290633098026\\
165	0.00694269578066109\\
166	0.00694248142035172\\
167	0.00694226318135076\\
168	0.00694204099372528\\
169	0.00694181478628887\\
170	0.00694158448657936\\
171	0.00694135002083621\\
172	0.00694111131397747\\
173	0.00694086828957625\\
174	0.00694062086983692\\
175	0.00694036897557088\\
176	0.00694011252617177\\
177	0.00693985143959037\\
178	0.00693958563230906\\
179	0.00693931501931578\\
180	0.00693903951407757\\
181	0.00693875902851364\\
182	0.006938473472968\\
183	0.00693818275618153\\
184	0.0069378867852637\\
185	0.00693758546566369\\
186	0.00693727870114104\\
187	0.00693696639373578\\
188	0.00693664844373811\\
189	0.00693632474965749\\
190	0.00693599520819113\\
191	0.00693565971419217\\
192	0.00693531816063699\\
193	0.00693497043859225\\
194	0.00693461643718118\\
195	0.00693425604354933\\
196	0.00693388914282978\\
197	0.0069335156181077\\
198	0.00693313535038432\\
199	0.00693274821854031\\
200	0.00693235409929843\\
201	0.00693195286718566\\
202	0.00693154439449466\\
203	0.00693112855124446\\
204	0.00693070520514063\\
205	0.00693027422153472\\
206	0.00692983546338289\\
207	0.00692938879120407\\
208	0.00692893406303712\\
209	0.00692847113439755\\
210	0.00692799985823328\\
211	0.00692752008487976\\
212	0.00692703166201432\\
213	0.00692653443460973\\
214	0.00692602824488693\\
215	0.00692551293226704\\
216	0.00692498833332255\\
217	0.00692445428172762\\
218	0.00692391060820754\\
219	0.00692335714048752\\
220	0.0069227937032403\\
221	0.00692222011803323\\
222	0.00692163620327414\\
223	0.00692104177415652\\
224	0.00692043664260375\\
225	0.00691982061721227\\
226	0.00691919350319398\\
227	0.00691855510231751\\
228	0.00691790521284864\\
229	0.00691724362948962\\
230	0.00691657014331765\\
231	0.00691588454172209\\
232	0.00691518660834084\\
233	0.00691447612299557\\
234	0.00691375286162589\\
235	0.00691301659622249\\
236	0.00691226709475902\\
237	0.00691150412112304\\
238	0.00691072743504577\\
239	0.00690993679203053\\
240	0.0069091319432803\\
241	0.00690831263562384\\
242	0.00690747861144072\\
243	0.00690662960858509\\
244	0.00690576536030822\\
245	0.0069048855951798\\
246	0.00690399003700792\\
247	0.00690307840475774\\
248	0.00690215041246895\\
249	0.00690120576917171\\
250	0.00690024417880139\\
251	0.00689926534011188\\
252	0.00689826894658748\\
253	0.00689725468635337\\
254	0.00689622224208467\\
255	0.00689517129091393\\
256	0.00689410150433732\\
257	0.00689301254811908\\
258	0.00689190408219455\\
259	0.00689077576057163\\
260	0.00688962723123057\\
261	0.00688845813602224\\
262	0.00688726811056464\\
263	0.00688605678413775\\
264	0.00688482377957676\\
265	0.00688356871316349\\
266	0.00688229119451611\\
267	0.00688099082647716\\
268	0.00687966720499996\\
269	0.00687831991903377\\
270	0.00687694855040877\\
271	0.00687555267372398\\
272	0.00687413185624691\\
273	0.00687268565784921\\
274	0.00687121363104095\\
275	0.00686971532124631\\
276	0.00686819026753304\\
277	0.00686663800347691\\
278	0.00686505805409404\\
279	0.00686344993264916\\
280	0.0068618131435009\\
281	0.00686014718256823\\
282	0.00685845153720273\\
283	0.00685672568606032\\
284	0.00685496909897197\\
285	0.00685318123681425\\
286	0.00685136155137918\\
287	0.00684950948524381\\
288	0.00684762447163972\\
289	0.00684570593432244\\
290	0.00684375328744113\\
291	0.0068417659354086\\
292	0.00683974327277196\\
293	0.00683768468408408\\
294	0.00683558954377621\\
295	0.00683345721603199\\
296	0.00683128705466313\\
297	0.00682907840298717\\
298	0.00682683059370771\\
299	0.00682454294879728\\
300	0.00682221477938361\\
301	0.00681984538563942\\
302	0.00681743405667633\\
303	0.00681498007044334\\
304	0.00681248269363003\\
305	0.00680994118157487\\
306	0.00680735477817774\\
307	0.00680472271581441\\
308	0.00680204421524527\\
309	0.00679931848549755\\
310	0.00679654472366677\\
311	0.00679372211450116\\
312	0.00679084982945135\\
313	0.00678792702453238\\
314	0.00678495283606589\\
315	0.00678192637496652\\
316	0.00677884673122257\\
317	0.00677571301116319\\
318	0.00677252431404008\\
319	0.00676927972679045\\
320	0.00676597832402508\\
321	0.00676261916802415\\
322	0.00675920130874011\\
323	0.0067557237838069\\
324	0.00675218561855415\\
325	0.00674858582602453\\
326	0.00674492340699199\\
327	0.00674119734997746\\
328	0.00673740663125787\\
329	0.00673355021486284\\
330	0.00672962705255205\\
331	0.00672563608376393\\
332	0.00672157623552386\\
333	0.0067174464222968\\
334	0.00671324554576512\\
335	0.0067089724945067\\
336	0.00670462614354091\\
337	0.00670020535369889\\
338	0.00669570897075152\\
339	0.00669113582416517\\
340	0.006686484725114\\
341	0.00668175446227064\\
342	0.00667694378810846\\
343	0.0066720514101265\\
344	0.00666707601442083\\
345	0.0066620162650849\\
346	0.0066568708036065\\
347	0.00665163824824929\\
348	0.00664631719336097\\
349	0.00664090620841075\\
350	0.0066354038361421\\
351	0.00662980858811405\\
352	0.00662411894676957\\
353	0.00661833337075079\\
354	0.00661245029465419\\
355	0.00660646812903499\\
356	0.00660038526113502\\
357	0.00659420005779262\\
358	0.00658791087504414\\
359	0.0065815160877256\\
360	0.00657501416938438\\
361	0.00656840376917955\\
362	0.00656168322022079\\
363	0.00655485078322227\\
364	0.00654790469801547\\
365	0.00654084318433572\\
366	0.00653366444282823\\
367	0.00652636665631342\\
368	0.00651894799135598\\
369	0.00651140660018733\\
370	0.00650374062303331\\
371	0.00649594819089508\\
372	0.00648802742880782\\
373	0.00647997645952718\\
374	0.00647179340737348\\
375	0.00646347640134645\\
376	0.00645502357491394\\
377	0.00644643305523614\\
378	0.00643770292244578\\
379	0.0064288310908781\\
380	0.00641981501860316\\
381	0.00641065128683896\\
382	0.0064013378902591\\
383	0.00639187317032329\\
384	0.00638225559557552\\
385	0.0063724837922119\\
386	0.00636255658396266\\
387	0.0063524730422203\\
388	0.00634223255301183\\
389	0.00633183492038799\\
390	0.00632128057070477\\
391	0.00631057048172071\\
392	0.00629970623276913\\
393	0.00628869011538802\\
394	0.00627752526468924\\
395	0.00626621581572812\\
396	0.00625476709009516\\
397	0.00624318581914798\\
398	0.0062314804117944\\
399	0.00621966127660698\\
400	0.00620774121040289\\
401	0.00619573586840969\\
402	0.00618366433499069\\
403	0.00617154981904448\\
404	0.00615942050553408\\
405	0.00614731060645685\\
406	0.00613526167730345\\
407	0.00612332430667249\\
408	0.00611156033221292\\
409	0.0061000470048815\\
410	0.00608888767423589\\
411	0.00607819528708678\\
412	0.0060681075480346\\
413	0.00605879216119026\\
414	0.00605039235996826\\
415	0.00604185743285749\\
416	0.00603318689318356\\
417	0.00602438049187169\\
418	0.00601543826557051\\
419	0.00600636059380947\\
420	0.00599714826356753\\
421	0.00598780253140995\\
422	0.00597832514686596\\
423	0.00596871821590453\\
424	0.00595898350381705\\
425	0.0059491197902136\\
426	0.00593911287090259\\
427	0.00592893537021737\\
428	0.00591858525772538\\
429	0.00590806053171042\\
430	0.00589735922382137\\
431	0.00588647940413584\\
432	0.0058754191866799\\
433	0.00586417673545369\\
434	0.00585275027102884\\
435	0.00584113807777788\\
436	0.00582933851177035\\
437	0.00581735000938853\\
438	0.00580517109684997\\
439	0.00579280040080217\\
440	0.00578023666036211\\
441	0.00576747874230785\\
442	0.00575452565753874\\
443	0.0057413765754582\\
444	0.00572803083883235\\
445	0.00571448797890645\\
446	0.00570074773024775\\
447	0.00568681004396134\\
448	0.00567267509619579\\
449	0.00565834328446983\\
450	0.00564381517755388\\
451	0.00562909122899558\\
452	0.0056141718403278\\
453	0.00559905734025508\\
454	0.00558374783688781\\
455	0.00556824300784175\\
456	0.0055525418165271\\
457	0.00553664223866819\\
458	0.0055205460104428\\
459	0.0055042548798025\\
460	0.00548777105199359\\
461	0.00547109670729628\\
462	0.00545423336834274\\
463	0.00543718144625986\\
464	0.00541993961752241\\
465	0.00540250397815816\\
466	0.00538486693980148\\
467	0.00536701575891807\\
468	0.00534893050941499\\
469	0.00533058140256037\\
470	0.00531192525664044\\
471	0.00529290082854101\\
472	0.00527342265064941\\
473	0.00525345928920426\\
474	0.00523299400938363\\
475	0.00521200905849416\\
476	0.00519048576086233\\
477	0.00516840498273882\\
478	0.00514574679734546\\
479	0.00512249027630091\\
480	0.00509861340375058\\
481	0.00507409300201966\\
482	0.00504890468123225\\
483	0.00502302294423405\\
484	0.00499642067975817\\
485	0.00496906674974231\\
486	0.0049409272542987\\
487	0.00491196446639418\\
488	0.00488214330691491\\
489	0.00485142886448882\\
490	0.0048197850881542\\
491	0.00478717504516566\\
492	0.00475356100831737\\
493	0.00471890451952105\\
494	0.0046831664209815\\
495	0.00464630683692947\\
496	0.00460828516353634\\
497	0.0045690598648893\\
498	0.00452858806221636\\
499	0.00448682481535209\\
500	0.00444372186236416\\
501	0.00439922444155628\\
502	0.00435326183881953\\
503	0.0043057663241142\\
504	0.00425668069219755\\
505	0.00420596463645533\\
506	0.00415360351552566\\
507	0.00409961616185581\\
508	0.00404406752357827\\
509	0.00398708599350592\\
510	0.00392888654377625\\
511	0.00386979945639612\\
512	0.00381098264892001\\
513	0.00375343889528498\\
514	0.00369735913363088\\
515	0.00364294323764303\\
516	0.00359039584672001\\
517	0.00353991996381999\\
518	0.00349170750589745\\
519	0.00344592571380514\\
520	0.00340269812280261\\
521	0.00336207834154239\\
522	0.00332401368087333\\
523	0.00328787877285572\\
524	0.00325235286091564\\
525	0.00321743390702476\\
526	0.00318310583236878\\
527	0.00314933622795304\\
528	0.00311607420835653\\
529	0.00308324867862273\\
530	0.00305076739670388\\
531	0.00301851739566748\\
532	0.00298636765065977\\
533	0.0029541753336562\\
534	0.00292180350962232\\
535	0.00288922044366352\\
536	0.00285639051087333\\
537	0.00282327447655823\\
538	0.00278982996972125\\
539	0.00275601217596438\\
540	0.00272177476417683\\
541	0.002687071039504\\
542	0.00265185528873788\\
543	0.00261608420867369\\
544	0.00257971821736867\\
545	0.0025427223019672\\
546	0.00250506177064863\\
547	0.00246670110567965\\
548	0.00242760442670507\\
549	0.0023877360820317\\
550	0.00234706140132897\\
551	0.00230554737593107\\
552	0.00226317048307561\\
553	0.00221991333580777\\
554	0.00217575964277463\\
555	0.00213069413955516\\
556	0.00208470244627019\\
557	0.0020377709651064\\
558	0.00198988703819046\\
559	0.00194103911958854\\
560	0.00189121696355546\\
561	0.00184041183165241\\
562	0.00178861672312973\\
563	0.00173582662502646\\
564	0.00168203875846467\\
565	0.0016272528277323\\
566	0.00157147127416107\\
567	0.00151469961622109\\
568	0.00145694766974197\\
569	0.00139823255111959\\
570	0.00133858101940291\\
571	0.00127804978197904\\
572	0.00121673136249583\\
573	0.00115457311231883\\
574	0.00109149386651943\\
575	0.00102737860674039\\
576	0.000964660699241159\\
577	0.000903620067808924\\
578	0.000844693973416802\\
579	0.00078896063058065\\
580	0.000735683997274819\\
581	0.000684393506326076\\
582	0.000635262737748697\\
583	0.000587719065319528\\
584	0.000541596458167286\\
585	0.000496834403922398\\
586	0.00045297776209589\\
587	0.0004098793501683\\
588	0.000367452930850881\\
589	0.000325673544872563\\
590	0.00028455397376655\\
591	0.000244105317769071\\
592	0.000204333810489543\\
593	0.000165233217676712\\
594	0.000126870159937007\\
595	8.96111722547522e-05\\
596	5.4266094523851e-05\\
597	2.28062284332059e-05\\
598	2.9204464504877e-07\\
599	0\\
600	0\\
};
\addplot [color=red!50!mycolor17,solid,forget plot]
  table[row sep=crcr]{%
1	0.00628094651839509\\
2	0.00628094102369023\\
3	0.00628093542892778\\
4	0.00628092973228313\\
5	0.00628092393189816\\
6	0.00628091802588075\\
7	0.00628091201230395\\
8	0.00628090588920548\\
9	0.00628089965458708\\
10	0.0062808933064137\\
11	0.00628088684261299\\
12	0.00628088026107447\\
13	0.00628087355964898\\
14	0.00628086673614781\\
15	0.00628085978834202\\
16	0.00628085271396175\\
17	0.00628084551069534\\
18	0.00628083817618871\\
19	0.00628083070804443\\
20	0.00628082310382097\\
21	0.00628081536103192\\
22	0.00628080747714516\\
23	0.00628079944958188\\
24	0.00628079127571589\\
25	0.00628078295287261\\
26	0.00628077447832827\\
27	0.00628076584930893\\
28	0.00628075706298955\\
29	0.00628074811649311\\
30	0.00628073900688952\\
31	0.00628072973119478\\
32	0.00628072028636988\\
33	0.00628071066931981\\
34	0.00628070087689251\\
35	0.00628069090587783\\
36	0.00628068075300646\\
37	0.00628067041494876\\
38	0.00628065988831376\\
39	0.00628064916964789\\
40	0.00628063825543392\\
41	0.00628062714208971\\
42	0.00628061582596708\\
43	0.0062806043033505\\
44	0.00628059257045588\\
45	0.00628058062342928\\
46	0.00628056845834568\\
47	0.00628055607120759\\
48	0.00628054345794367\\
49	0.00628053061440746\\
50	0.00628051753637593\\
51	0.00628050421954806\\
52	0.00628049065954342\\
53	0.00628047685190062\\
54	0.00628046279207592\\
55	0.00628044847544159\\
56	0.00628043389728444\\
57	0.00628041905280421\\
58	0.00628040393711189\\
59	0.00628038854522819\\
60	0.00628037287208172\\
61	0.00628035691250748\\
62	0.00628034066124487\\
63	0.00628032411293615\\
64	0.00628030726212452\\
65	0.00628029010325225\\
66	0.00628027263065894\\
67	0.0062802548385795\\
68	0.00628023672114227\\
69	0.00628021827236698\\
70	0.0062801994861628\\
71	0.00628018035632629\\
72	0.00628016087653928\\
73	0.00628014104036676\\
74	0.00628012084125468\\
75	0.00628010027252782\\
76	0.0062800793273875\\
77	0.00628005799890928\\
78	0.00628003628004066\\
79	0.0062800141635987\\
80	0.0062799916422676\\
81	0.00627996870859623\\
82	0.00627994535499566\\
83	0.00627992157373659\\
84	0.0062798973569467\\
85	0.00627987269660813\\
86	0.00627984758455471\\
87	0.00627982201246914\\
88	0.00627979597188038\\
89	0.00627976945416064\\
90	0.00627974245052258\\
91	0.00627971495201631\\
92	0.0062796869495264\\
93	0.00627965843376886\\
94	0.00627962939528797\\
95	0.00627959982445314\\
96	0.00627956971145567\\
97	0.00627953904630546\\
98	0.0062795078188277\\
99	0.00627947601865933\\
100	0.00627944363524576\\
101	0.00627941065783719\\
102	0.00627937707548512\\
103	0.00627934287703853\\
104	0.00627930805114029\\
105	0.00627927258622331\\
106	0.00627923647050663\\
107	0.00627919969199157\\
108	0.00627916223845761\\
109	0.00627912409745839\\
110	0.00627908525631748\\
111	0.00627904570212421\\
112	0.00627900542172933\\
113	0.00627896440174055\\
114	0.00627892262851818\\
115	0.00627888008817048\\
116	0.00627883676654913\\
117	0.00627879264924439\\
118	0.00627874772158035\\
119	0.00627870196861005\\
120	0.00627865537511043\\
121	0.00627860792557735\\
122	0.00627855960422035\\
123	0.0062785103949574\\
124	0.00627846028140953\\
125	0.00627840924689548\\
126	0.00627835727442593\\
127	0.00627830434669812\\
128	0.00627825044608983\\
129	0.0062781955546537\\
130	0.00627813965411117\\
131	0.00627808272584647\\
132	0.00627802475090034\\
133	0.00627796570996379\\
134	0.00627790558337171\\
135	0.00627784435109629\\
136	0.00627778199274037\\
137	0.0062777184875307\\
138	0.00627765381431106\\
139	0.00627758795153517\\
140	0.00627752087725963\\
141	0.00627745256913659\\
142	0.00627738300440642\\
143	0.00627731215989005\\
144	0.00627724001198138\\
145	0.00627716653663946\\
146	0.00627709170938051\\
147	0.00627701550526985\\
148	0.00627693789891362\\
149	0.00627685886445043\\
150	0.00627677837554275\\
151	0.00627669640536831\\
152	0.0062766129266111\\
153	0.00627652791145255\\
154	0.00627644133156215\\
155	0.00627635315808825\\
156	0.00627626336164851\\
157	0.00627617191232016\\
158	0.00627607877963023\\
159	0.00627598393254549\\
160	0.00627588733946222\\
161	0.00627578896819584\\
162	0.00627568878597031\\
163	0.00627558675940736\\
164	0.0062754828545156\\
165	0.00627537703667921\\
166	0.00627526927064674\\
167	0.00627515952051947\\
168	0.00627504774973969\\
169	0.00627493392107862\\
170	0.00627481799662442\\
171	0.00627469993776958\\
172	0.00627457970519841\\
173	0.00627445725887422\\
174	0.00627433255802615\\
175	0.00627420556113595\\
176	0.00627407622592435\\
177	0.0062739445093374\\
178	0.00627381036753238\\
179	0.0062736737558635\\
180	0.00627353462886746\\
181	0.00627339294024866\\
182	0.00627324864286411\\
183	0.0062731016887082\\
184	0.00627295202889711\\
185	0.00627279961365297\\
186	0.00627264439228783\\
187	0.00627248631318715\\
188	0.00627232532379321\\
189	0.00627216137058811\\
190	0.00627199439907659\\
191	0.00627182435376837\\
192	0.00627165117816039\\
193	0.00627147481471861\\
194	0.0062712952048595\\
195	0.00627111228893133\\
196	0.00627092600619503\\
197	0.00627073629480474\\
198	0.00627054309178807\\
199	0.00627034633302594\\
200	0.00627014595323224\\
201	0.00626994188593301\\
202	0.00626973406344522\\
203	0.00626952241685541\\
204	0.00626930687599775\\
205	0.00626908736943188\\
206	0.00626886382442027\\
207	0.00626863616690529\\
208	0.00626840432148594\\
209	0.00626816821139397\\
210	0.00626792775846993\\
211	0.00626768288313855\\
212	0.00626743350438385\\
213	0.00626717953972388\\
214	0.00626692090518494\\
215	0.00626665751527545\\
216	0.00626638928295935\\
217	0.00626611611962919\\
218	0.00626583793507867\\
219	0.00626555463747476\\
220	0.00626526613332943\\
221	0.00626497232747089\\
222	0.00626467312301442\\
223	0.00626436842133268\\
224	0.00626405812202566\\
225	0.00626374212288999\\
226	0.00626342031988799\\
227	0.00626309260711611\\
228	0.00626275887677293\\
229	0.00626241901912664\\
230	0.00626207292248208\\
231	0.00626172047314722\\
232	0.00626136155539926\\
233	0.00626099605145005\\
234	0.00626062384141118\\
235	0.0062602448032584\\
236	0.00625985881279563\\
237	0.00625946574361841\\
238	0.00625906546707674\\
239	0.00625865785223759\\
240	0.00625824276584659\\
241	0.00625782007228937\\
242	0.00625738963355231\\
243	0.00625695130918266\\
244	0.00625650495624811\\
245	0.00625605042929581\\
246	0.00625558758031067\\
247	0.00625511625867318\\
248	0.0062546363111165\\
249	0.00625414758168291\\
250	0.00625364991167956\\
251	0.00625314313963339\\
252	0.00625262710124551\\
253	0.00625210162934443\\
254	0.0062515665538387\\
255	0.0062510217016683\\
256	0.00625046689675522\\
257	0.00624990195995269\\
258	0.00624932670899319\\
259	0.00624874095843501\\
260	0.0062481445196072\\
261	0.00624753720055269\\
262	0.00624691880596951\\
263	0.00624628913714975\\
264	0.006245647991916\\
265	0.00624499516455529\\
266	0.00624433044575023\\
267	0.00624365362250734\\
268	0.00624296447808276\\
269	0.00624226279190632\\
270	0.00624154833950568\\
271	0.00624082089243628\\
272	0.00624008021823186\\
273	0.00623932608041917\\
274	0.00623855823872793\\
275	0.0062377764498981\\
276	0.00623698047030328\\
277	0.0062361700637432\\
278	0.0062353450193773\\
279	0.00623450512588287\\
280	0.00623365013075074\\
281	0.00623277977116969\\
282	0.00623189378012479\\
283	0.00623099188634747\\
284	0.00623007381426606\\
285	0.00622913928395706\\
286	0.00622818801109727\\
287	0.00622721970691689\\
288	0.00622623407815404\\
289	0.00622523082701093\\
290	0.00622420965111174\\
291	0.00622317024346303\\
292	0.00622211229241692\\
293	0.00622103548163765\\
294	0.0062199394900721\\
295	0.00621882399192518\\
296	0.00621768865664089\\
297	0.00621653314889005\\
298	0.00621535712856596\\
299	0.0062141602507891\\
300	0.00621294216592259\\
301	0.00621170251959943\\
302	0.00621044095276355\\
303	0.00620915710172548\\
304	0.006207850598234\\
305	0.00620652106956347\\
306	0.00620516813861468\\
307	0.00620379142402274\\
308	0.00620239054025357\\
309	0.0062009650976434\\
310	0.00619951470226192\\
311	0.00619803895528635\\
312	0.00619653745105236\\
313	0.00619500977157134\\
314	0.00619345547183289\\
315	0.00619187404272906\\
316	0.00619026483260075\\
317	0.00618862701493381\\
318	0.00618696005667385\\
319	0.0061852634724058\\
320	0.00618353677000882\\
321	0.00618177945060518\\
322	0.00617999100850778\\
323	0.00617817093116535\\
324	0.00617631869910501\\
325	0.00617443378587163\\
326	0.00617251565796289\\
327	0.00617056377475961\\
328	0.00616857758845009\\
329	0.0061665565439479\\
330	0.00616450007880163\\
331	0.00616240762309565\\
332	0.00616027859934056\\
333	0.00615811242235246\\
334	0.0061559084991195\\
335	0.00615366622865523\\
336	0.00615138500183909\\
337	0.00614906420124558\\
338	0.00614670320096608\\
339	0.0061443013664249\\
340	0.00614185805416666\\
341	0.00613937261154164\\
342	0.00613684437664284\\
343	0.00613427267874974\\
344	0.00613165683802625\\
345	0.00612899616513374\\
346	0.00612628996073316\\
347	0.00612353751484298\\
348	0.00612073810601039\\
349	0.00611789100024825\\
350	0.0061149954497102\\
351	0.00611205069110537\\
352	0.00610905594355766\\
353	0.00610601040567376\\
354	0.0061029132518177\\
355	0.00609976362747239\\
356	0.00609656064378319\\
357	0.00609330337230913\\
358	0.00608999084475669\\
359	0.00608662207816113\\
360	0.00608319621868035\\
361	0.00607971329950062\\
362	0.00607617618023874\\
363	0.00607258442884062\\
364	0.00606893764994095\\
365	0.0060652354907542\\
366	0.00606147764795087\\
367	0.00605766387572508\\
368	0.00605379399530891\\
369	0.00604986790625016\\
370	0.00604588559984586\\
371	0.00604184717521095\\
372	0.00603775285855215\\
373	0.00603360302627036\\
374	0.00602939823240385\\
375	0.0060251392402592\\
376	0.0060208270556653\\
377	0.00601646295147004\\
378	0.00601204844818144\\
379	0.00600758513639789\\
380	0.00600307396102361\\
381	0.00599851261440092\\
382	0.00599388544244718\\
383	0.00598919227205787\\
384	0.00598443314402723\\
385	0.0059796081968888\\
386	0.00597471767623049\\
387	0.00596976194436436\\
388	0.00596474149005861\\
389	0.00595965693759411\\
390	0.0059545090529708\\
391	0.00594929875289887\\
392	0.00594402711429341\\
393	0.00593869538304248\\
394	0.00593330498149172\\
395	0.00592785751390258\\
396	0.00592235476888149\\
397	0.00591679871743948\\
398	0.00591119150489946\\
399	0.00590553543428755\\
400	0.0058998329380851\\
401	0.00589408653422213\\
402	0.00588829876088577\\
403	0.00588247208299798\\
404	0.00587660876093922\\
405	0.00587071066903032\\
406	0.00586477904703945\\
407	0.00585881416217398\\
408	0.00585281485363119\\
409	0.00584677790125402\\
410	0.0058406970847465\\
411	0.00583456223492706\\
412	0.0058283578137602\\
413	0.00582206104411124\\
414	0.00581564132950739\\
415	0.00580909611989684\\
416	0.00580242278769309\\
417	0.00579561862140608\\
418	0.00578868081793819\\
419	0.00578160647315369\\
420	0.00577439257025361\\
421	0.00576703596549714\\
422	0.00575953337116851\\
423	0.00575188133728517\\
424	0.00574407623924873\\
425	0.00573611429910769\\
426	0.00572799174826807\\
427	0.00571970501971221\\
428	0.0057112504374799\\
429	0.00570262421128045\\
430	0.00569382243073522\\
431	0.00568484105921746\\
432	0.005675675927249\\
433	0.00566632272540006\\
434	0.00565677699666341\\
435	0.00564703412839972\\
436	0.00563708934390052\\
437	0.00562693769319163\\
438	0.0056165740425122\\
439	0.00560599306326688\\
440	0.00559518922013013\\
441	0.00558415675807559\\
442	0.00557288968817381\\
443	0.00556138177202994\\
444	0.00554962650456028\\
445	0.00553761709472205\\
446	0.00552534644538259\\
447	0.00551280713336373\\
448	0.00549999138768484\\
449	0.0054868910588211\\
450	0.00547349758831802\\
451	0.00545980198065547\\
452	0.0054457947712163\\
453	0.00543146599114788\\
454	0.00541680513034979\\
455	0.00540180109703208\\
456	0.00538644216815536\\
457	0.00537071598176243\\
458	0.00535460946385435\\
459	0.00533810875409222\\
460	0.00532119911108202\\
461	0.00530386483378619\\
462	0.00528608918842895\\
463	0.00526785433637298\\
464	0.00524914126685343\\
465	0.00522992969894428\\
466	0.00521019785529346\\
467	0.00518992263385337\\
468	0.0051690801278424\\
469	0.00514764554730165\\
470	0.00512559328073365\\
471	0.00510289727727141\\
472	0.00507953175229053\\
473	0.005055469974674\\
474	0.00503068410002096\\
475	0.00500514388755872\\
476	0.00497881545852904\\
477	0.00495166159305947\\
478	0.00492364845257451\\
479	0.00489474256434685\\
480	0.00486491000694632\\
481	0.00483411653850486\\
482	0.00480232771466033\\
483	0.004769508966389\\
484	0.00473562561016074\\
485	0.00470064279906671\\
486	0.00466452525591064\\
487	0.00462723672678512\\
488	0.00458873886863503\\
489	0.00454898918719202\\
490	0.00450793510723609\\
491	0.00446551549812167\\
492	0.00442167582343966\\
493	0.0043763724822605\\
494	0.00432957844475902\\
495	0.00428129072677211\\
496	0.00423154041810466\\
497	0.0041804059763298\\
498	0.00412803078363405\\
499	0.0040746463120854\\
500	0.00402060285304561\\
501	0.00396743933357704\\
502	0.00391556399806644\\
503	0.00386515616431572\\
504	0.00381640428098737\\
505	0.00376949956777484\\
506	0.00372462926229314\\
507	0.00368196700638546\\
508	0.0036416591034706\\
509	0.00360380526237516\\
510	0.00356843183757421\\
511	0.00353545553400809\\
512	0.00350392253949669\\
513	0.00347300447686475\\
514	0.00344269685930694\\
515	0.00341298179059693\\
516	0.00338382592998904\\
517	0.00335517866997427\\
518	0.00332697080439069\\
519	0.00329911409782002\\
520	0.00327150234383665\\
521	0.0032440147440854\\
522	0.00321652281051031\\
523	0.00318892662393476\\
524	0.00316120027083114\\
525	0.00313331491948637\\
526	0.0031052391674743\\
527	0.00307693956129006\\
528	0.00304838130493532\\
529	0.00301952916079463\\
530	0.00299034852332022\\
531	0.00296080660805138\\
532	0.00293087363594622\\
533	0.00290052379454109\\
534	0.00286973531170303\\
535	0.00283848542659518\\
536	0.00280675046465217\\
537	0.00277450591734095\\
538	0.00274172652008376\\
539	0.00270838631990362\\
540	0.00267445872297293\\
541	0.00263991651194983\\
542	0.00260473182461684\\
543	0.00256887609111131\\
544	0.00253231993928337\\
545	0.00249503310058525\\
546	0.00245698458609395\\
547	0.00241814303104803\\
548	0.00237847717142211\\
549	0.002337956497622\\
550	0.00229655183905641\\
551	0.00225424597099964\\
552	0.00221102370153545\\
553	0.00216687040382646\\
554	0.00212177211836048\\
555	0.00207571568159297\\
556	0.002028688891195\\
557	0.00198068071385247\\
558	0.00193168153175534\\
559	0.00188168342228085\\
560	0.00183068044001795\\
561	0.00177866892288609\\
562	0.00172564781376172\\
563	0.00167161948272084\\
564	0.00161659226410366\\
565	0.00156058298271428\\
566	0.00150362990328009\\
567	0.00144583137164707\\
568	0.00138714982852563\\
569	0.00132751977693301\\
570	0.00126683277350473\\
571	0.00120492049445878\\
572	0.00114382947878174\\
573	0.00108426163518732\\
574	0.00102666342567682\\
575	0.000971846090115743\\
576	0.000918265947001419\\
577	0.000866422173777079\\
578	0.000816462067759592\\
579	0.000767915527104241\\
580	0.000720194215446356\\
581	0.000673645505827697\\
582	0.000628150962911244\\
583	0.000583436223402909\\
584	0.000539226991113701\\
585	0.000495475690639145\\
586	0.000452196148255144\\
587	0.000409418505768188\\
588	0.000367181587116058\\
589	0.000325525791425008\\
590	0.000284484611477045\\
591	0.000244081548911079\\
592	0.000204328087251261\\
593	0.000165233217676711\\
594	0.000126870159937007\\
595	8.96111722547522e-05\\
596	5.42660945238508e-05\\
597	2.28062284332056e-05\\
598	2.9204464504877e-07\\
599	0\\
600	0\\
};
\addplot [color=red!40!mycolor19,solid,forget plot]
  table[row sep=crcr]{%
1	0.00609949007723919\\
2	0.00609948663958327\\
3	0.00609948313907455\\
4	0.00609947957456534\\
5	0.00609947594488679\\
6	0.00609947224884837\\
7	0.00609946848523761\\
8	0.00609946465281958\\
9	0.00609946075033646\\
10	0.00609945677650724\\
11	0.00609945273002708\\
12	0.00609944860956702\\
13	0.00609944441377342\\
14	0.00609944014126752\\
15	0.00609943579064496\\
16	0.00609943136047536\\
17	0.00609942684930172\\
18	0.00609942225563992\\
19	0.00609941757797826\\
20	0.00609941281477694\\
21	0.00609940796446747\\
22	0.00609940302545211\\
23	0.00609939799610336\\
24	0.00609939287476341\\
25	0.00609938765974349\\
26	0.00609938234932327\\
27	0.00609937694175034\\
28	0.00609937143523951\\
29	0.00609936582797223\\
30	0.00609936011809597\\
31	0.00609935430372347\\
32	0.00609934838293216\\
33	0.00609934235376347\\
34	0.0060993362142221\\
35	0.00609932996227535\\
36	0.00609932359585234\\
37	0.00609931711284339\\
38	0.00609931051109918\\
39	0.00609930378842994\\
40	0.00609929694260481\\
41	0.00609928997135099\\
42	0.00609928287235284\\
43	0.00609927564325116\\
44	0.00609926828164235\\
45	0.00609926078507749\\
46	0.00609925315106148\\
47	0.00609924537705219\\
48	0.0060992374604595\\
49	0.00609922939864443\\
50	0.0060992211889181\\
51	0.0060992128285409\\
52	0.00609920431472136\\
53	0.00609919564461529\\
54	0.00609918681532465\\
55	0.00609917782389656\\
56	0.00609916866732222\\
57	0.00609915934253584\\
58	0.00609914984641353\\
59	0.00609914017577214\\
60	0.00609913032736824\\
61	0.00609912029789677\\
62	0.00609911008399001\\
63	0.00609909968221624\\
64	0.00609908908907858\\
65	0.0060990783010137\\
66	0.00609906731439054\\
67	0.00609905612550898\\
68	0.00609904473059852\\
69	0.00609903312581693\\
70	0.00609902130724886\\
71	0.00609900927090432\\
72	0.00609899701271745\\
73	0.00609898452854482\\
74	0.00609897181416409\\
75	0.00609895886527241\\
76	0.00609894567748484\\
77	0.00609893224633285\\
78	0.00609891856726264\\
79	0.00609890463563345\\
80	0.00609889044671596\\
81	0.00609887599569055\\
82	0.00609886127764551\\
83	0.00609884628757535\\
84	0.00609883102037887\\
85	0.00609881547085742\\
86	0.00609879963371292\\
87	0.00609878350354604\\
88	0.00609876707485416\\
89	0.00609875034202935\\
90	0.00609873329935646\\
91	0.00609871594101096\\
92	0.00609869826105684\\
93	0.00609868025344447\\
94	0.00609866191200839\\
95	0.00609864323046511\\
96	0.00609862420241078\\
97	0.00609860482131893\\
98	0.00609858508053802\\
99	0.00609856497328922\\
100	0.00609854449266368\\
101	0.00609852363162027\\
102	0.00609850238298286\\
103	0.00609848073943784\\
104	0.00609845869353143\\
105	0.00609843623766696\\
106	0.00609841336410217\\
107	0.00609839006494641\\
108	0.0060983663321577\\
109	0.00609834215753996\\
110	0.00609831753273993\\
111	0.00609829244924423\\
112	0.00609826689837624\\
113	0.00609824087129303\\
114	0.00609821435898207\\
115	0.0060981873522581\\
116	0.00609815984175969\\
117	0.00609813181794595\\
118	0.00609810327109312\\
119	0.00609807419129092\\
120	0.00609804456843922\\
121	0.00609801439224414\\
122	0.00609798365221457\\
123	0.00609795233765821\\
124	0.00609792043767789\\
125	0.00609788794116754\\
126	0.00609785483680825\\
127	0.00609782111306418\\
128	0.00609778675817847\\
129	0.00609775176016894\\
130	0.00609771610682384\\
131	0.00609767978569751\\
132	0.00609764278410584\\
133	0.00609760508912176\\
134	0.00609756668757063\\
135	0.00609752756602546\\
136	0.00609748771080222\\
137	0.00609744710795483\\
138	0.00609740574327021\\
139	0.00609736360226324\\
140	0.00609732067017151\\
141	0.00609727693195015\\
142	0.00609723237226632\\
143	0.00609718697549387\\
144	0.00609714072570772\\
145	0.00609709360667812\\
146	0.00609704560186496\\
147	0.00609699669441184\\
148	0.00609694686714002\\
149	0.00609689610254236\\
150	0.0060968443827771\\
151	0.00609679168966141\\
152	0.00609673800466514\\
153	0.00609668330890394\\
154	0.00609662758313276\\
155	0.00609657080773896\\
156	0.00609651296273535\\
157	0.00609645402775305\\
158	0.00609639398203434\\
159	0.00609633280442522\\
160	0.00609627047336795\\
161	0.00609620696689341\\
162	0.0060961422626133\\
163	0.00609607633771221\\
164	0.00609600916893961\\
165	0.0060959407326015\\
166	0.00609587100455211\\
167	0.00609579996018535\\
168	0.00609572757442607\\
169	0.0060956538217213\\
170	0.00609557867603111\\
171	0.00609550211081947\\
172	0.00609542409904487\\
173	0.00609534461315076\\
174	0.00609526362505591\\
175	0.00609518110614439\\
176	0.00609509702725557\\
177	0.00609501135867384\\
178	0.00609492407011807\\
179	0.00609483513073107\\
180	0.00609474450906861\\
181	0.00609465217308845\\
182	0.00609455809013899\\
183	0.00609446222694786\\
184	0.00609436454961022\\
185	0.00609426502357682\\
186	0.00609416361364184\\
187	0.00609406028393064\\
188	0.00609395499788708\\
189	0.0060938477182607\\
190	0.00609373840709372\\
191	0.00609362702570769\\
192	0.00609351353468993\\
193	0.00609339789387978\\
194	0.00609328006235448\\
195	0.00609315999841487\\
196	0.00609303765957081\\
197	0.00609291300252628\\
198	0.0060927859831643\\
199	0.00609265655653147\\
200	0.00609252467682226\\
201	0.00609239029736297\\
202	0.00609225337059552\\
203	0.00609211384806071\\
204	0.00609197168038146\\
205	0.00609182681724539\\
206	0.00609167920738737\\
207	0.00609152879857161\\
208	0.00609137553757333\\
209	0.00609121937016029\\
210	0.00609106024107379\\
211	0.00609089809400936\\
212	0.00609073287159712\\
213	0.0060905645153817\\
214	0.00609039296580175\\
215	0.00609021816216924\\
216	0.00609004004264809\\
217	0.00608985854423248\\
218	0.00608967360272485\\
219	0.00608948515271332\\
220	0.0060892931275486\\
221	0.00608909745932065\\
222	0.0060888980788347\\
223	0.00608869491558677\\
224	0.00608848789773869\\
225	0.00608827695209266\\
226	0.0060880620040652\\
227	0.00608784297766047\\
228	0.00608761979544311\\
229	0.00608739237851043\\
230	0.00608716064646387\\
231	0.0060869245173799\\
232	0.00608668390778015\\
233	0.00608643873260086\\
234	0.00608618890516139\\
235	0.00608593433713219\\
236	0.00608567493850159\\
237	0.00608541061754193\\
238	0.00608514128077454\\
239	0.00608486683293366\\
240	0.00608458717692941\\
241	0.00608430221380937\\
242	0.00608401184271884\\
243	0.0060837159608598\\
244	0.00608341446344822\\
245	0.00608310724366962\\
246	0.00608279419263279\\
247	0.00608247519932137\\
248	0.00608215015054305\\
249	0.00608181893087611\\
250	0.00608148142261296\\
251	0.00608113750570028\\
252	0.0060807870576751\\
253	0.00608042995359647\\
254	0.00608006606597174\\
255	0.00607969526467671\\
256	0.00607931741686841\\
257	0.00607893238688942\\
258	0.00607854003616171\\
259	0.00607814022306827\\
260	0.00607773280281994\\
261	0.00607731762730415\\
262	0.00607689454491185\\
263	0.00607646340033768\\
264	0.00607602403434751\\
265	0.00607557628350515\\
266	0.00607511997984932\\
267	0.00607465495050848\\
268	0.0060741810172383\\
269	0.00607369799586306\\
270	0.00607320569559741\\
271	0.00607270391822066\\
272	0.00607219245707472\\
273	0.00607167109586992\\
274	0.00607113960735399\\
275	0.00607059775219016\\
276	0.00607004527956652\\
277	0.00606948193612833\\
278	0.0060689075145937\\
279	0.00606832212225923\\
280	0.00606772596503167\\
281	0.00606711885765475\\
282	0.00606650061256708\\
283	0.00606587103995243\\
284	0.00606522994780081\\
285	0.00606457714198132\\
286	0.00606391242632902\\
287	0.00606323560274758\\
288	0.00606254647133033\\
289	0.00606184483050241\\
290	0.00606113047718779\\
291	0.00606040320700509\\
292	0.00605966281449736\\
293	0.00605890909340178\\
294	0.00605814183696667\\
295	0.0060573608383249\\
296	0.00605656589093458\\
297	0.0060557567891005\\
298	0.00605493332859318\\
299	0.00605409530738581\\
300	0.00605324252653497\\
301	0.0060523747912365\\
302	0.00605149191209669\\
303	0.0060505937066678\\
304	0.00604968000131025\\
305	0.00604875063345861\\
306	0.006047805454388\\
307	0.00604684433259849\\
308	0.00604586715795675\\
309	0.00604487384674168\\
310	0.00604386434769657\\
311	0.0060428386489718\\
312	0.00604179678508451\\
313	0.00604073884058972\\
314	0.00603966493946368\\
315	0.00603857518426247\\
316	0.00603746942339006\\
317	0.00603634639281748\\
318	0.00603520227378819\\
319	0.00603403668044389\\
320	0.00603284922021106\\
321	0.00603163949369277\\
322	0.00603040709455885\\
323	0.00602915160943468\\
324	0.00602787261778861\\
325	0.00602656969181786\\
326	0.00602524239633337\\
327	0.00602389028864327\\
328	0.00602251291843567\\
329	0.00602110982766015\\
330	0.00601968055040881\\
331	0.00601822461279687\\
332	0.00601674153284308\\
333	0.00601523082035019\\
334	0.00601369197678595\\
335	0.00601212449516497\\
336	0.006010527859932\\
337	0.00600890154684694\\
338	0.00600724502287205\\
339	0.00600555774606172\\
340	0.00600383916545539\\
341	0.00600208872097596\\
342	0.00600030584332892\\
343	0.0059984899538931\\
344	0.00599664046461557\\
345	0.00599475677791284\\
346	0.00599283828658081\\
347	0.00599088437371715\\
348	0.00598889441266128\\
349	0.00598686776695855\\
350	0.00598480379035758\\
351	0.00598270182685029\\
352	0.00598056121077298\\
353	0.00597838126699589\\
354	0.00597616131123422\\
355	0.00597390065052333\\
356	0.00597159858391212\\
357	0.00596925440342943\\
358	0.00596686739533017\\
359	0.00596443684136008\\
360	0.00596196201846727\\
361	0.00595944218832653\\
362	0.00595687655876742\\
363	0.00595426432285699\\
364	0.00595160465835615\\
365	0.00594889672710658\\
366	0.00594613967433101\\
367	0.00594333262782399\\
368	0.00594047469700265\\
369	0.00593756497177677\\
370	0.00593460252118352\\
371	0.00593158639171427\\
372	0.00592851560523585\\
373	0.00592538915637774\\
374	0.00592220600921807\\
375	0.00591896509306117\\
376	0.00591566529708073\\
377	0.00591230546367939\\
378	0.00590888438083942\\
379	0.00590540077532086\\
380	0.00590185331409043\\
381	0.0058982406413478\\
382	0.0058945615607665\\
383	0.00589081484917345\\
384	0.00588699925315538\\
385	0.00588311348689868\\
386	0.00587915622972221\\
387	0.00587512612325945\\
388	0.00587102176824432\\
389	0.00586684172085428\\
390	0.00586258448862514\\
391	0.00585824852581624\\
392	0.00585383222813431\\
393	0.0058493339267348\\
394	0.00584475188141529\\
395	0.00584008427291369\\
396	0.00583532919422693\\
397	0.00583048464087574\\
398	0.00582554850005957\\
399	0.00582051853867892\\
400	0.00581539239025816\\
401	0.00581016754088863\\
402	0.00580484131444731\\
403	0.00579941085753478\\
404	0.00579387312485534\\
405	0.0057882248661641\\
406	0.00578246261647293\\
407	0.00577658269202535\\
408	0.00577058119559946\\
409	0.0057644540365423\\
410	0.00575819697446965\\
411	0.0057518056932042\\
412	0.00574527592104245\\
413	0.00573860362103671\\
414	0.00573178525222442\\
415	0.00572481715987743\\
416	0.00571769557054584\\
417	0.00571041658686079\\
418	0.00570297618209032\\
419	0.00569537019445206\\
420	0.00568759432119414\\
421	0.00567964411248218\\
422	0.00567151496514972\\
423	0.0056632021163124\\
424	0.00565470063701057\\
425	0.00564600542546769\\
426	0.0056371111976312\\
427	0.00562801247359913\\
428	0.00561870356763885\\
429	0.00560917857772503\\
430	0.00559943137445791\\
431	0.00558945558930404\\
432	0.0055792446020997\\
433	0.00556879152770817\\
434	0.00555808920160435\\
435	0.00554713016404182\\
436	0.00553590664433304\\
437	0.00552441054698277\\
438	0.00551263343706816\\
439	0.00550056651711529\\
440	0.00548820060611104\\
441	0.00547552611821937\\
442	0.0054625330403664\\
443	0.00544921090865481\\
444	0.00543554878338468\\
445	0.00542153522134404\\
446	0.00540715823967644\\
447	0.00539240528487763\\
448	0.00537726322626625\\
449	0.00536171836902219\\
450	0.00534575638316665\\
451	0.00532936225790829\\
452	0.00531252026494597\\
453	0.00529521391674586\\
454	0.00527742591865204\\
455	0.00525913811865879\\
456	0.00524033143738105\\
457	0.00522098557186356\\
458	0.00520107913309478\\
459	0.00518059002488525\\
460	0.00515949530239541\\
461	0.00513777091349513\\
462	0.00511539163234859\\
463	0.00509233104181511\\
464	0.00506856155377677\\
465	0.00504405459075588\\
466	0.00501878070847056\\
467	0.00499270703806631\\
468	0.00496579830873763\\
469	0.00493802272386192\\
470	0.00490934976632691\\
471	0.00487974891411962\\
472	0.00484918962606995\\
473	0.0048176412409891\\
474	0.00478507269140209\\
475	0.00475145198857918\\
476	0.00471674534696897\\
477	0.00468091556270191\\
478	0.00464391755227902\\
479	0.00460569392592616\\
480	0.00456619201277347\\
481	0.00452536715465093\\
482	0.00448318728872223\\
483	0.00443963912314147\\
484	0.00439473634050376\\
485	0.00434853028216572\\
486	0.00430112419214191\\
487	0.00425269202561763\\
488	0.00420350323911374\\
489	0.00415437121824339\\
490	0.00410637070479002\\
491	0.00405966503355721\\
492	0.00401442487833185\\
493	0.00397082558574463\\
494	0.00392904397100763\\
495	0.00388925106606612\\
496	0.00385160133021408\\
497	0.00381621843275766\\
498	0.00378317542947194\\
499	0.00375246742237028\\
500	0.00372397408550459\\
501	0.00369632046858735\\
502	0.00366926832203856\\
503	0.00364280960574355\\
504	0.00361692350771421\\
505	0.00359157468192518\\
506	0.00356671175220235\\
507	0.00354226635502932\\
508	0.00351815313048391\\
509	0.00349427124709755\\
510	0.0034705082972028\\
511	0.00344674770856446\\
512	0.00342292178581892\\
513	0.00339900788620017\\
514	0.00337498108408554\\
515	0.00335081455714488\\
516	0.00332648013014735\\
517	0.00330194898684415\\
518	0.00327719254604736\\
519	0.00325218347362656\\
520	0.00322689676323357\\
521	0.00320131075848223\\
522	0.00317540789701119\\
523	0.00314917367059463\\
524	0.00312259320271672\\
525	0.00309565133254662\\
526	0.00306833269407408\\
527	0.00304062178171857\\
528	0.00301250299167785\\
529	0.0029839606266123\\
530	0.00295497885052935\\
531	0.00292554158190216\\
532	0.00289563231772567\\
533	0.00286523389183085\\
534	0.00283432820493941\\
535	0.00280289616912907\\
536	0.00277091764986146\\
537	0.00273837140739797\\
538	0.00270523504060233\\
539	0.00267148493771352\\
540	0.00263709624074912\\
541	0.00260204283283811\\
542	0.00256629736102423\\
543	0.0025298313108894\\
544	0.00249261515356304\\
545	0.00245461858992128\\
546	0.00241581091084629\\
547	0.00237616150442382\\
548	0.00233564055112879\\
549	0.00229421974472738\\
550	0.00225188413966668\\
551	0.00220861946225582\\
552	0.00216441225788334\\
553	0.00211925011293704\\
554	0.00207312192622846\\
555	0.00202601822576494\\
556	0.00197793150135645\\
557	0.00192885657953965\\
558	0.00187879103415134\\
559	0.00182773633444033\\
560	0.00177570121612501\\
561	0.00172270419382382\\
562	0.00166883078065186\\
563	0.00161408765059616\\
564	0.00155842221278116\\
565	0.00150176416055747\\
566	0.00144398692427377\\
567	0.00138484423759802\\
568	0.00132527813048257\\
569	0.00126704480548041\\
570	0.00121055312007744\\
571	0.00115653787994659\\
572	0.00110330586081578\\
573	0.00105101228541044\\
574	0.00100035963156781\\
575	0.000951144518913378\\
576	0.000902948852799658\\
577	0.000855365674177795\\
578	0.000808355188236387\\
579	0.000762216035417117\\
580	0.00071690457967\\
581	0.00067198153227095\\
582	0.000627351837523727\\
583	0.000583007319138113\\
584	0.000538986303744318\\
585	0.000495338379819224\\
586	0.000452116120303638\\
587	0.000409372268862694\\
588	0.000367156964357134\\
589	0.000325514548405105\\
590	0.000284480845955471\\
591	0.000244080684870838\\
592	0.000204328087251261\\
593	0.000165233217676712\\
594	0.000126870159937008\\
595	8.96111722547531e-05\\
596	5.42660945238511e-05\\
597	2.28062284332057e-05\\
598	2.9204464504877e-07\\
599	0\\
600	0\\
};
\addplot [color=red!75!mycolor17,solid,forget plot]
  table[row sep=crcr]{%
1	0.0060542548985145\\
2	0.00605425199370972\\
3	0.00605424903553986\\
4	0.00605424602303839\\
5	0.00605424295522111\\
6	0.006054239831086\\
7	0.00605423664961272\\
8	0.00605423340976236\\
9	0.00605423011047707\\
10	0.00605422675067964\\
11	0.00605422332927326\\
12	0.00605421984514103\\
13	0.00605421629714559\\
14	0.00605421268412883\\
15	0.00605420900491139\\
16	0.00605420525829226\\
17	0.00605420144304841\\
18	0.00605419755793437\\
19	0.00605419360168178\\
20	0.00605418957299889\\
21	0.00605418547057024\\
22	0.00605418129305609\\
23	0.00605417703909203\\
24	0.00605417270728841\\
25	0.00605416829622995\\
26	0.00605416380447523\\
27	0.00605415923055607\\
28	0.00605415457297718\\
29	0.0060541498302155\\
30	0.00605414500071974\\
31	0.00605414008290975\\
32	0.00605413507517608\\
33	0.00605412997587927\\
34	0.00605412478334934\\
35	0.00605411949588519\\
36	0.00605411411175401\\
37	0.00605410862919055\\
38	0.0060541030463966\\
39	0.00605409736154033\\
40	0.00605409157275556\\
41	0.00605408567814111\\
42	0.00605407967576014\\
43	0.00605407356363946\\
44	0.00605406733976871\\
45	0.00605406100209975\\
46	0.00605405454854584\\
47	0.00605404797698082\\
48	0.0060540412852385\\
49	0.00605403447111171\\
50	0.00605402753235154\\
51	0.0060540204666665\\
52	0.00605401327172166\\
53	0.00605400594513783\\
54	0.00605399848449061\\
55	0.00605399088730956\\
56	0.00605398315107722\\
57	0.0060539752732282\\
58	0.00605396725114822\\
59	0.00605395908217315\\
60	0.00605395076358791\\
61	0.00605394229262556\\
62	0.00605393366646624\\
63	0.00605392488223606\\
64	0.00605391593700604\\
65	0.00605390682779104\\
66	0.00605389755154856\\
67	0.00605388810517763\\
68	0.00605387848551762\\
69	0.00605386868934704\\
70	0.00605385871338232\\
71	0.00605384855427655\\
72	0.00605383820861819\\
73	0.00605382767292976\\
74	0.00605381694366659\\
75	0.00605380601721534\\
76	0.00605379488989271\\
77	0.00605378355794394\\
78	0.00605377201754145\\
79	0.00605376026478337\\
80	0.00605374829569191\\
81	0.00605373610621198\\
82	0.00605372369220956\\
83	0.00605371104947008\\
84	0.0060536981736968\\
85	0.00605368506050917\\
86	0.00605367170544116\\
87	0.00605365810393945\\
88	0.00605364425136173\\
89	0.00605363014297491\\
90	0.00605361577395319\\
91	0.00605360113937629\\
92	0.00605358623422746\\
93	0.00605357105339158\\
94	0.00605355559165315\\
95	0.00605353984369429\\
96	0.00605352380409261\\
97	0.00605350746731915\\
98	0.00605349082773621\\
99	0.00605347387959511\\
100	0.00605345661703403\\
101	0.00605343903407568\\
102	0.00605342112462494\\
103	0.0060534028824666\\
104	0.00605338430126275\\
105	0.00605336537455043\\
106	0.00605334609573917\\
107	0.00605332645810821\\
108	0.00605330645480415\\
109	0.00605328607883798\\
110	0.00605326532308261\\
111	0.00605324418026994\\
112	0.00605322264298808\\
113	0.00605320070367846\\
114	0.00605317835463289\\
115	0.00605315558799049\\
116	0.00605313239573474\\
117	0.00605310876969023\\
118	0.00605308470151957\\
119	0.00605306018272017\\
120	0.00605303520462078\\
121	0.00605300975837828\\
122	0.00605298383497411\\
123	0.00605295742521092\\
124	0.00605293051970883\\
125	0.00605290310890187\\
126	0.00605287518303431\\
127	0.00605284673215675\\
128	0.00605281774612234\\
129	0.00605278821458291\\
130	0.00605275812698484\\
131	0.00605272747256499\\
132	0.00605269624034661\\
133	0.00605266441913503\\
134	0.00605263199751339\\
135	0.00605259896383815\\
136	0.00605256530623462\\
137	0.00605253101259243\\
138	0.00605249607056078\\
139	0.00605246046754374\\
140	0.00605242419069537\\
141	0.00605238722691476\\
142	0.006052349562841\\
143	0.00605231118484809\\
144	0.00605227207903958\\
145	0.00605223223124333\\
146	0.00605219162700609\\
147	0.00605215025158783\\
148	0.00605210808995623\\
149	0.00605206512678079\\
150	0.00605202134642706\\
151	0.00605197673295059\\
152	0.00605193127009082\\
153	0.0060518849412649\\
154	0.00605183772956136\\
155	0.00605178961773358\\
156	0.00605174058819324\\
157	0.00605169062300365\\
158	0.00605163970387276\\
159	0.00605158781214639\\
160	0.00605153492880093\\
161	0.00605148103443623\\
162	0.00605142610926808\\
163	0.00605137013312077\\
164	0.00605131308541933\\
165	0.00605125494518182\\
166	0.00605119569101119\\
167	0.00605113530108726\\
168	0.00605107375315838\\
169	0.00605101102453295\\
170	0.00605094709207084\\
171	0.00605088193217454\\
172	0.00605081552078026\\
173	0.00605074783334872\\
174	0.00605067884485589\\
175	0.00605060852978345\\
176	0.00605053686210912\\
177	0.00605046381529679\\
178	0.00605038936228641\\
179	0.00605031347548376\\
180	0.00605023612674995\\
181	0.00605015728739076\\
182	0.00605007692814574\\
183	0.00604999501917712\\
184	0.0060499115300585\\
185	0.00604982642976324\\
186	0.00604973968665278\\
187	0.00604965126846458\\
188	0.00604956114229992\\
189	0.00604946927461135\\
190	0.00604937563119001\\
191	0.00604928017715268\\
192	0.00604918287692849\\
193	0.00604908369424541\\
194	0.0060489825921166\\
195	0.00604887953282627\\
196	0.00604877447791537\\
197	0.00604866738816707\\
198	0.0060485582235918\\
199	0.00604844694341212\\
200	0.00604833350604725\\
201	0.00604821786909728\\
202	0.00604809998932707\\
203	0.00604797982264992\\
204	0.00604785732411068\\
205	0.00604773244786887\\
206	0.0060476051471812\\
207	0.00604747537438382\\
208	0.00604734308087434\\
209	0.00604720821709327\\
210	0.00604707073250529\\
211	0.00604693057558015\\
212	0.006046787693773\\
213	0.00604664203350458\\
214	0.00604649354014091\\
215	0.0060463421579725\\
216	0.00604618783019338\\
217	0.00604603049887952\\
218	0.00604587010496691\\
219	0.00604570658822923\\
220	0.00604553988725512\\
221	0.00604536993942488\\
222	0.00604519668088687\\
223	0.0060450200465334\\
224	0.00604483996997613\\
225	0.00604465638352118\\
226	0.00604446921814338\\
227	0.00604427840346052\\
228	0.00604408386770679\\
229	0.00604388553770576\\
230	0.00604368333884301\\
231	0.00604347719503816\\
232	0.00604326702871631\\
233	0.00604305276077907\\
234	0.00604283431057514\\
235	0.00604261159587007\\
236	0.0060423845328158\\
237	0.00604215303591947\\
238	0.00604191701801181\\
239	0.00604167639021485\\
240	0.00604143106190916\\
241	0.00604118094070057\\
242	0.00604092593238629\\
243	0.00604066594092043\\
244	0.00604040086837903\\
245	0.00604013061492462\\
246	0.00603985507877001\\
247	0.00603957415614173\\
248	0.00603928774124293\\
249	0.00603899572621561\\
250	0.00603869800110246\\
251	0.00603839445380817\\
252	0.00603808497006022\\
253	0.00603776943336931\\
254	0.00603744772498924\\
255	0.00603711972387649\\
256	0.00603678530664946\\
257	0.00603644434754738\\
258	0.00603609671838921\\
259	0.00603574228853227\\
260	0.00603538092483121\\
261	0.00603501249159726\\
262	0.00603463685055806\\
263	0.00603425386081876\\
264	0.00603386337882462\\
265	0.00603346525832618\\
266	0.00603305935034801\\
267	0.00603264550316234\\
268	0.00603222356226987\\
269	0.00603179337039027\\
270	0.00603135476746624\\
271	0.0060309075906859\\
272	0.00603045167453047\\
273	0.00602998685085563\\
274	0.00602951294901667\\
275	0.00602902979604564\\
276	0.0060285372168711\\
277	0.00602803503448028\\
278	0.00602752306947831\\
279	0.00602700113593192\\
280	0.00602646903952961\\
281	0.0060259265821394\\
282	0.00602537356172687\\
283	0.00602480977227086\\
284	0.00602423500367617\\
285	0.00602364904168356\\
286	0.00602305166777651\\
287	0.00602244265908472\\
288	0.00602182178828396\\
289	0.00602118882349233\\
290	0.00602054352816206\\
291	0.00601988566096721\\
292	0.00601921497568633\\
293	0.00601853122107978\\
294	0.00601783414076147\\
295	0.00601712347306377\\
296	0.00601639895089535\\
297	0.00601566030159045\\
298	0.0060149072467486\\
299	0.00601413950206287\\
300	0.00601335677713433\\
301	0.00601255877527002\\
302	0.00601174519325995\\
303	0.00601091572112823\\
304	0.00601007004185066\\
305	0.00600920783102938\\
306	0.00600832875651105\\
307	0.00600743247793097\\
308	0.00600651864615927\\
309	0.00600558690261738\\
310	0.00600463687842416\\
311	0.0060036681933215\\
312	0.00600268045432796\\
313	0.00600167325409898\\
314	0.00600064616910949\\
315	0.00599959875827926\\
316	0.00599853056443055\\
317	0.00599744112762684\\
318	0.00599633002043226\\
319	0.00599519680657051\\
320	0.0059940410407279\\
321	0.005992862268351\\
322	0.00599166002543922\\
323	0.00599043383833153\\
324	0.00598918322348752\\
325	0.00598790768726258\\
326	0.00598660672567669\\
327	0.005985279824177\\
328	0.00598392645739348\\
329	0.00598254608888788\\
330	0.00598113817089524\\
331	0.0059797021440581\\
332	0.00597823743715262\\
333	0.00597674346680675\\
334	0.00597521963720975\\
335	0.00597366533981282\\
336	0.0059720799530204\\
337	0.00597046284187164\\
338	0.00596881335771161\\
339	0.00596713083785189\\
340	0.00596541460521975\\
341	0.00596366396799539\\
342	0.00596187821923701\\
343	0.00596005663649259\\
344	0.00595819848139854\\
345	0.00595630299926424\\
346	0.00595436941864167\\
347	0.00595239695087973\\
348	0.00595038478966199\\
349	0.00594833211052729\\
350	0.00594623807037185\\
351	0.00594410180693178\\
352	0.0059419224382442\\
353	0.00593969906208525\\
354	0.00593743075538152\\
355	0.00593511657359166\\
356	0.00593275555005242\\
357	0.00593034669528196\\
358	0.0059278889962312\\
359	0.0059253814154743\\
360	0.00592282289034598\\
361	0.00592021233215843\\
362	0.00591754862594789\\
363	0.00591483062968437\\
364	0.00591205717345138\\
365	0.00590922705859407\\
366	0.0059063390568347\\
367	0.00590339190935374\\
368	0.00590038432583566\\
369	0.00589731498347839\\
370	0.00589418252596617\\
371	0.0058909855624063\\
372	0.0058877226662318\\
373	0.0058843923740743\\
374	0.00588099318461349\\
375	0.0058775235574158\\
376	0.00587398191177944\\
377	0.00587036662561085\\
378	0.00586667603436179\\
379	0.00586290843004274\\
380	0.00585906206024255\\
381	0.00585513512668948\\
382	0.00585112578103346\\
383	0.00584703212292182\\
384	0.0058428521980019\\
385	0.00583858399582633\\
386	0.00583422544765775\\
387	0.00582977442416987\\
388	0.00582522873304119\\
389	0.00582058611644157\\
390	0.00581584424840936\\
391	0.00581100073211716\\
392	0.0058060530970288\\
393	0.00580099879595051\\
394	0.00579583520198189\\
395	0.00579055960537552\\
396	0.00578516921031676\\
397	0.00577966113164156\\
398	0.00577403239152266\\
399	0.0057682799161579\\
400	0.00576240053247438\\
401	0.00575639096487995\\
402	0.00575024783207291\\
403	0.00574396764398991\\
404	0.00573754679892206\\
405	0.00573098158081809\\
406	0.00572426815680069\\
407	0.00571740257477954\\
408	0.0057103807610439\\
409	0.00570319851758423\\
410	0.00569585151864958\\
411	0.00568833530593727\\
412	0.00568064528179326\\
413	0.00567277669913962\\
414	0.00566472464014436\\
415	0.00565648400799689\\
416	0.00564804951792851\\
417	0.00563941568793487\\
418	0.00563057682899589\\
419	0.0056215270346997\\
420	0.00561226017019854\\
421	0.00560276986037648\\
422	0.00559304947721987\\
423	0.00558309212656227\\
424	0.00557289063323799\\
425	0.00556243752687089\\
426	0.00555172502862437\\
427	0.00554074503770892\\
428	0.00552948911145671\\
429	0.00551794844589325\\
430	0.00550611385673614\\
431	0.00549397575973935\\
432	0.00548152415026784\\
433	0.00546874858210162\\
434	0.00545563814522781\\
435	0.00544218144097815\\
436	0.00542836654737445\\
437	0.00541418098913318\\
438	0.00539961173926262\\
439	0.00538464523869321\\
440	0.0053692673161546\\
441	0.00535346313975109\\
442	0.0053372171746004\\
443	0.00532051313740289\\
444	0.00530333394848065\\
445	0.00528566168259585\\
446	0.0052674775156372\\
447	0.00524876156771987\\
448	0.00522949276320377\\
449	0.00520964896449091\\
450	0.00518920757373021\\
451	0.00516814511954083\\
452	0.00514643720112049\\
453	0.00512405864723791\\
454	0.00510098378576996\\
455	0.00507718651561285\\
456	0.00505264043758917\\
457	0.00502731926123773\\
458	0.00500119338249417\\
459	0.00497423113914844\\
460	0.00494640367555496\\
461	0.00491768390259255\\
462	0.00488804416224272\\
463	0.00485745557444009\\
464	0.00482588695608326\\
465	0.00479330279722531\\
466	0.0047596576280196\\
467	0.00472489779996188\\
468	0.00468897354242522\\
469	0.0046518417927839\\
470	0.00461347006042861\\
471	0.00457384167076456\\
472	0.00453296284521914\\
473	0.0044908721285255\\
474	0.0044476528958311\\
475	0.00440344963230884\\
476	0.00435848950460793\\
477	0.00431330090536759\\
478	0.00426911901941018\\
479	0.00422609131441622\\
480	0.0041843734761192\\
481	0.00414412701115376\\
482	0.00410551496252874\\
483	0.00406869694891489\\
484	0.00403382195664205\\
485	0.00400101794806777\\
486	0.0039703749638837\\
487	0.00394192340717972\\
488	0.00391560420247279\\
489	0.00389078961723187\\
490	0.00386655392115217\\
491	0.00384289551839856\\
492	0.00381980222562067\\
493	0.00379724964136457\\
494	0.00377519960045295\\
495	0.00375359895964029\\
496	0.00373237900858465\\
497	0.0037114559002514\\
498	0.00369073269417876\\
499	0.00367010384999713\\
500	0.00364946334530463\\
501	0.00362877905153483\\
502	0.00360803028005762\\
503	0.00358719451586058\\
504	0.00356624782116794\\
505	0.00354516538132623\\
506	0.00352392219824332\\
507	0.00350249392209815\\
508	0.00348085778748331\\
509	0.00345899358143589\\
510	0.00343688451126113\\
511	0.00341451775218712\\
512	0.0033918824440211\\
513	0.0033689676683681\\
514	0.00334576254116211\\
515	0.00332225629935928\\
516	0.00329843837371602\\
517	0.00327429843792487\\
518	0.0032498264231834\\
519	0.00322501248710322\\
520	0.00319984692763712\\
521	0.00317432003781731\\
522	0.00314842190777655\\
523	0.00312214225160906\\
524	0.00309547037273244\\
525	0.00306839512133135\\
526	0.00304090484325028\\
527	0.00301298732007266\\
528	0.00298462970066008\\
529	0.00295581842514291\\
530	0.00292653914324031\\
531	0.00289677662978668\\
532	0.00286651470130723\\
533	0.00283573613813255\\
534	0.00280442261571909\\
535	0.00277255463848415\\
536	0.00274011147920116\\
537	0.00270707112800494\\
538	0.0026734102563127\\
539	0.00263910420253288\\
540	0.00260412698838064\\
541	0.00256845137703628\\
542	0.00253204898738828\\
543	0.00249489048234822\\
544	0.00245694585391358\\
545	0.00241818483357086\\
546	0.00237857746439831\\
547	0.00233809488657614\\
548	0.00229671045645981\\
549	0.00225441111083414\\
550	0.00221118503939151\\
551	0.00216702202657525\\
552	0.00212191382248228\\
553	0.0020758545648019\\
554	0.00202884126536382\\
555	0.00198087510715297\\
556	0.00193196503358525\\
557	0.00188213043144942\\
558	0.00183148136505412\\
559	0.00177997739204913\\
560	0.00172755807796478\\
561	0.00167414569304286\\
562	0.00161956559511978\\
563	0.00156366756377305\\
564	0.00150631068618296\\
565	0.00144923707526241\\
566	0.00139360112641226\\
567	0.00133997155536227\\
568	0.00128796297954385\\
569	0.00123625110336431\\
570	0.0011849967424863\\
571	0.00113500063362126\\
572	0.00108630229033245\\
573	0.00103861044929871\\
574	0.000991302384123051\\
575	0.000944334643980418\\
576	0.00089771161392153\\
577	0.000851824028850601\\
578	0.000806587975280099\\
579	0.000761530778505787\\
580	0.000716607563069663\\
581	0.000671842104011154\\
582	0.000627277405667915\\
583	0.000582965786062491\\
584	0.000538962689738965\\
585	0.000495324737689842\\
586	0.000452108368453091\\
587	0.000409368226051223\\
588	0.000367155166688129\\
589	0.000325513960353391\\
590	0.000284480716949123\\
591	0.000244080684870838\\
592	0.000204328087251262\\
593	0.000165233217676712\\
594	0.000126870159937009\\
595	8.96111722547531e-05\\
596	5.42660945238512e-05\\
597	2.28062284332062e-05\\
598	2.9204464504877e-07\\
599	0\\
600	0\\
};
\addplot [color=red!80!mycolor19,solid,forget plot]
  table[row sep=crcr]{%
1	0.00604379698173766\\
2	0.00604379373848888\\
3	0.0060437904349912\\
4	0.00604378707016116\\
5	0.00604378364289623\\
6	0.00604378015207447\\
7	0.0060437765965541\\
8	0.00604377297517326\\
9	0.0060437692867496\\
10	0.00604376553007986\\
11	0.00604376170393957\\
12	0.00604375780708261\\
13	0.0060437538382409\\
14	0.00604374979612386\\
15	0.00604374567941812\\
16	0.00604374148678706\\
17	0.00604373721687039\\
18	0.00604373286828375\\
19	0.00604372843961818\\
20	0.00604372392943977\\
21	0.00604371933628912\\
22	0.00604371465868094\\
23	0.00604370989510351\\
24	0.0060437050440182\\
25	0.00604370010385904\\
26	0.0060436950730321\\
27	0.00604368994991509\\
28	0.00604368473285671\\
29	0.0060436794201762\\
30	0.00604367401016273\\
31	0.00604366850107489\\
32	0.00604366289114001\\
33	0.00604365717855372\\
34	0.00604365136147922\\
35	0.00604364543804676\\
36	0.00604363940635292\\
37	0.00604363326446007\\
38	0.00604362701039563\\
39	0.00604362064215146\\
40	0.00604361415768315\\
41	0.00604360755490934\\
42	0.00604360083171102\\
43	0.00604359398593074\\
44	0.00604358701537194\\
45	0.00604357991779815\\
46	0.00604357269093224\\
47	0.00604356533245563\\
48	0.00604355784000749\\
49	0.00604355021118384\\
50	0.00604354244353681\\
51	0.00604353453457374\\
52	0.00604352648175627\\
53	0.00604351828249949\\
54	0.00604350993417095\\
55	0.00604350143408981\\
56	0.00604349277952583\\
57	0.00604348396769837\\
58	0.00604347499577549\\
59	0.00604346586087274\\
60	0.00604345656005232\\
61	0.0060434470903219\\
62	0.00604343744863352\\
63	0.00604342763188252\\
64	0.00604341763690638\\
65	0.00604340746048358\\
66	0.00604339709933237\\
67	0.00604338655010954\\
68	0.00604337580940924\\
69	0.00604336487376163\\
70	0.0060433537396317\\
71	0.00604334240341784\\
72	0.00604333086145045\\
73	0.00604331910999072\\
74	0.00604330714522906\\
75	0.00604329496328367\\
76	0.00604328256019913\\
77	0.00604326993194488\\
78	0.00604325707441361\\
79	0.00604324398341973\\
80	0.00604323065469773\\
81	0.00604321708390063\\
82	0.00604320326659809\\
83	0.00604318919827488\\
84	0.00604317487432911\\
85	0.0060431602900703\\
86	0.00604314544071769\\
87	0.00604313032139822\\
88	0.00604311492714479\\
89	0.00604309925289413\\
90	0.00604308329348493\\
91	0.00604306704365572\\
92	0.00604305049804286\\
93	0.00604303365117828\\
94	0.00604301649748742\\
95	0.00604299903128696\\
96	0.00604298124678256\\
97	0.0060429631380665\\
98	0.00604294469911541\\
99	0.00604292592378771\\
100	0.00604290680582129\\
101	0.00604288733883078\\
102	0.00604286751630521\\
103	0.00604284733160519\\
104	0.00604282677796039\\
105	0.00604280584846656\\
106	0.00604278453608295\\
107	0.00604276283362935\\
108	0.00604274073378318\\
109	0.0060427182290765\\
110	0.00604269531189297\\
111	0.00604267197446476\\
112	0.00604264820886932\\
113	0.00604262400702621\\
114	0.0060425993606937\\
115	0.00604257426146556\\
116	0.00604254870076742\\
117	0.00604252266985342\\
118	0.00604249615980252\\
119	0.00604246916151483\\
120	0.00604244166570795\\
121	0.00604241366291311\\
122	0.00604238514347133\\
123	0.00604235609752938\\
124	0.00604232651503577\\
125	0.00604229638573664\\
126	0.00604226569917148\\
127	0.00604223444466896\\
128	0.00604220261134241\\
129	0.0060421701880854\\
130	0.00604213716356718\\
131	0.00604210352622806\\
132	0.00604206926427466\\
133	0.00604203436567501\\
134	0.00604199881815365\\
135	0.0060419626091866\\
136	0.00604192572599635\\
137	0.00604188815554639\\
138	0.00604184988453604\\
139	0.00604181089939501\\
140	0.00604177118627776\\
141	0.00604173073105795\\
142	0.0060416895193226\\
143	0.00604164753636629\\
144	0.00604160476718508\\
145	0.0060415611964705\\
146	0.0060415168086032\\
147	0.00604147158764674\\
148	0.00604142551734103\\
149	0.00604137858109577\\
150	0.00604133076198368\\
151	0.00604128204273377\\
152	0.0060412324057242\\
153	0.00604118183297524\\
154	0.00604113030614201\\
155	0.00604107780650709\\
156	0.00604102431497294\\
157	0.00604096981205427\\
158	0.00604091427787014\\
159	0.00604085769213601\\
160	0.00604080003415563\\
161	0.00604074128281264\\
162	0.00604068141656227\\
163	0.00604062041342257\\
164	0.00604055825096566\\
165	0.00604049490630886\\
166	0.0060404303561054\\
167	0.00604036457653526\\
168	0.00604029754329556\\
169	0.00604022923159099\\
170	0.00604015961612381\\
171	0.00604008867108396\\
172	0.0060400163701386\\
173	0.00603994268642189\\
174	0.00603986759252411\\
175	0.00603979106048093\\
176	0.00603971306176223\\
177	0.00603963356726089\\
178	0.00603955254728118\\
179	0.00603946997152703\\
180	0.00603938580909002\\
181	0.00603930002843719\\
182	0.00603921259739859\\
183	0.00603912348315443\\
184	0.00603903265222221\\
185	0.00603894007044345\\
186	0.0060388457029702\\
187	0.0060387495142512\\
188	0.00603865146801789\\
189	0.00603855152727004\\
190	0.00603844965426113\\
191	0.00603834581048335\\
192	0.00603823995665241\\
193	0.00603813205269206\\
194	0.00603802205771807\\
195	0.00603790993002215\\
196	0.00603779562705542\\
197	0.00603767910541146\\
198	0.00603756032080915\\
199	0.00603743922807512\\
200	0.00603731578112576\\
201	0.00603718993294888\\
202	0.00603706163558518\\
203	0.00603693084010897\\
204	0.00603679749660884\\
205	0.00603666155416769\\
206	0.00603652296084248\\
207	0.00603638166364351\\
208	0.00603623760851322\\
209	0.00603609074030466\\
210	0.00603594100275937\\
211	0.00603578833848494\\
212	0.00603563268893194\\
213	0.00603547399437057\\
214	0.00603531219386657\\
215	0.00603514722525689\\
216	0.00603497902512458\\
217	0.00603480752877349\\
218	0.006034632670202\\
219	0.00603445438207663\\
220	0.00603427259570483\\
221	0.00603408724100724\\
222	0.00603389824648951\\
223	0.00603370553921332\\
224	0.00603350904476698\\
225	0.0060333086872352\\
226	0.00603310438916856\\
227	0.00603289607155193\\
228	0.00603268365377257\\
229	0.00603246705358736\\
230	0.00603224618708944\\
231	0.00603202096867407\\
232	0.00603179131100386\\
233	0.00603155712497328\\
234	0.00603131831967223\\
235	0.00603107480234912\\
236	0.00603082647837303\\
237	0.00603057325119501\\
238	0.00603031502230869\\
239	0.00603005169120995\\
240	0.00602978315535595\\
241	0.00602950931012283\\
242	0.00602923004876309\\
243	0.00602894526236156\\
244	0.00602865483979086\\
245	0.00602835866766542\\
246	0.00602805663029497\\
247	0.00602774860963676\\
248	0.00602743448524672\\
249	0.00602711413422966\\
250	0.00602678743118839\\
251	0.00602645424817156\\
252	0.00602611445462051\\
253	0.00602576791731476\\
254	0.00602541450031635\\
255	0.00602505406491296\\
256	0.00602468646955961\\
257	0.00602431156981905\\
258	0.00602392921830078\\
259	0.0060235392645986\\
260	0.00602314155522674\\
261	0.00602273593355424\\
262	0.0060223222397381\\
263	0.00602190031065444\\
264	0.00602146997982814\\
265	0.00602103107736065\\
266	0.00602058342985596\\
267	0.00602012686034462\\
268	0.0060196611882056\\
269	0.00601918622908599\\
270	0.00601870179481818\\
271	0.00601820769333418\\
272	0.0060177037285767\\
273	0.00601718970040628\\
274	0.00601666540450337\\
275	0.00601613063226448\\
276	0.00601558517069031\\
277	0.00601502880226538\\
278	0.00601446130483493\\
279	0.00601388245153339\\
280	0.00601329201074085\\
281	0.00601268974597786\\
282	0.00601207541579783\\
283	0.00601144877367671\\
284	0.00601080956790019\\
285	0.0060101575414481\\
286	0.00600949243187622\\
287	0.00600881397119506\\
288	0.006008121885746\\
289	0.00600741589607442\\
290	0.00600669571679988\\
291	0.00600596105648334\\
292	0.00600521161749125\\
293	0.00600444709585661\\
294	0.00600366718113684\\
295	0.00600287155626848\\
296	0.00600205989741867\\
297	0.00600123187383337\\
298	0.00600038714768226\\
299	0.00599952537390029\\
300	0.00599864620002601\\
301	0.00599774926603647\\
302	0.0059968342041787\\
303	0.00599590063879805\\
304	0.00599494818616345\\
305	0.00599397645428955\\
306	0.00599298504275645\\
307	0.00599197354252766\\
308	0.00599094153576698\\
309	0.00598988859565647\\
310	0.0059888142862175\\
311	0.00598771816213875\\
312	0.00598659976861655\\
313	0.005985458641214\\
314	0.00598429430574684\\
315	0.00598310627819779\\
316	0.00598189406463208\\
317	0.00598065716095028\\
318	0.00597939505211107\\
319	0.00597810721187351\\
320	0.00597679310253201\\
321	0.0059754521746443\\
322	0.00597408386675148\\
323	0.00597268760509062\\
324	0.0059712628032988\\
325	0.00596980886210919\\
326	0.00596832516903809\\
327	0.00596681109806301\\
328	0.00596526600929152\\
329	0.00596368924862003\\
330	0.00596208014738282\\
331	0.00596043802199029\\
332	0.00595876217355639\\
333	0.00595705188751468\\
334	0.0059553064332226\\
335	0.00595352506355342\\
336	0.00595170701447526\\
337	0.00594985150461702\\
338	0.00594795773482025\\
339	0.00594602488767673\\
340	0.00594405212705106\\
341	0.00594203859758758\\
342	0.00593998342420066\\
343	0.00593788571154831\\
344	0.00593574454348784\\
345	0.00593355898251308\\
346	0.00593132806917237\\
347	0.00592905082146658\\
348	0.00592672623422611\\
349	0.00592435327846604\\
350	0.0059219309007185\\
351	0.005919458022341\\
352	0.00591693353879988\\
353	0.00591435631892751\\
354	0.00591172520415271\\
355	0.00590903900770255\\
356	0.00590629651377466\\
357	0.00590349647667853\\
358	0.00590063761994472\\
359	0.00589771863540054\\
360	0.00589473818221175\\
361	0.00589169488589031\\
362	0.00588858733725277\\
363	0.00588541409133603\\
364	0.00588217366626758\\
365	0.00587886454208901\\
366	0.00587548515953178\\
367	0.00587203391874112\\
368	0.00586850917794364\\
369	0.00586490925205651\\
370	0.00586123241123398\\
371	0.00585747687934691\\
372	0.00585364083239288\\
373	0.00584972239683445\\
374	0.00584571964786044\\
375	0.00584163060756524\\
376	0.00583745324304039\\
377	0.00583318546437371\\
378	0.00582882512254872\\
379	0.00582437000723769\\
380	0.00581981784447777\\
381	0.00581516629421113\\
382	0.00581041294774655\\
383	0.0058055553250961\\
384	0.00580059087218987\\
385	0.0057955169579579\\
386	0.00579033087126014\\
387	0.00578502981765638\\
388	0.00577961091601375\\
389	0.00577407119492913\\
390	0.00576840758897508\\
391	0.00576261693475519\\
392	0.00575669596675555\\
393	0.00575064131297993\\
394	0.00574444949035536\\
395	0.00573811689989189\\
396	0.00573163982157519\\
397	0.0057250144089629\\
398	0.00571823668345897\\
399	0.00571130252832844\\
400	0.00570420768255797\\
401	0.00569694773434462\\
402	0.00568951811419927\\
403	0.00568191408721808\\
404	0.00567413074496403\\
405	0.00566616299687093\\
406	0.00565800556104346\\
407	0.00564965295490358\\
408	0.00564109948494235\\
409	0.00563233923572894\\
410	0.00562336605821633\\
411	0.00561417355720179\\
412	0.00560475507766134\\
413	0.00559510369100685\\
414	0.005585212183069\\
415	0.0055750730417266\\
416	0.00556467844256432\\
417	0.00555402023031086\\
418	0.00554308990211027\\
419	0.00553187859084382\\
420	0.00552037704796187\\
421	0.00550857562560085\\
422	0.00549646425690355\\
423	0.00548403243420698\\
424	0.00547126918874049\\
425	0.00545816305475796\\
426	0.00544470204193996\\
427	0.00543087362543778\\
428	0.00541666475211721\\
429	0.00540206177521736\\
430	0.00538705039846046\\
431	0.00537161563139494\\
432	0.00535574174438478\\
433	0.00533941222309604\\
434	0.00532260972527357\\
435	0.00530531604468355\\
436	0.0052875120838388\\
437	0.00526917772433\\
438	0.00525029172219893\\
439	0.00523083201498075\\
440	0.00521077665230321\\
441	0.00519010343163706\\
442	0.00516878988349136\\
443	0.00514681334450342\\
444	0.00512415101997684\\
445	0.00510078002476416\\
446	0.00507667740522985\\
447	0.00505182028960638\\
448	0.00502618457628976\\
449	0.00499974289798137\\
450	0.00497246427710687\\
451	0.00494432117897287\\
452	0.00491528257570121\\
453	0.00488530972768461\\
454	0.00485435170143604\\
455	0.00482235931248683\\
456	0.00478928719534402\\
457	0.00475509663674163\\
458	0.00471975949187953\\
459	0.00468326338633841\\
460	0.00464561844076083\\
461	0.00460686637231825\\
462	0.00456709256807714\\
463	0.00452644208904394\\
464	0.00448514072074695\\
465	0.00444389923794743\\
466	0.00440357687039676\\
467	0.00436430832544291\\
468	0.00432623602995577\\
469	0.00428950815751473\\
470	0.00425427567283625\\
471	0.00422068738185323\\
472	0.00418888285783103\\
473	0.00415898311876477\\
474	0.00413107750704409\\
475	0.00410520609121975\\
476	0.00408133284316304\\
477	0.00405911043870122\\
478	0.00403743665823675\\
479	0.0040163131829262\\
480	0.00399573266567172\\
481	0.00397567727603209\\
482	0.00395611732407331\\
483	0.00393701007397081\\
484	0.00391829896134864\\
485	0.00389991353540765\\
486	0.00388177063464587\\
487	0.00386377740380242\\
488	0.00384583705763168\\
489	0.00382788266290682\\
490	0.00380989663293803\\
491	0.00379185963450906\\
492	0.00377375088435137\\
493	0.00375554856466038\\
494	0.00373723036853259\\
495	0.00371877417539081\\
496	0.00370015884067846\\
497	0.00368136505947589\\
498	0.00366237622430955\\
499	0.00364317913778042\\
500	0.00362376435192982\\
501	0.0036041230010163\\
502	0.00358424630052364\\
503	0.00356412563595359\\
504	0.00354375264480726\\
505	0.00352311928443311\\
506	0.00350221787714692\\
507	0.00348104112326624\\
508	0.00345958207301436\\
509	0.00343783405051128\\
510	0.0034157905286329\\
511	0.00339344496435606\\
512	0.0033707907054899\\
513	0.0033478209780482\\
514	0.00332452886784109\\
515	0.00330090729577894\\
516	0.00327694898667039\\
517	0.00325264643170242\\
518	0.00322799184531334\\
519	0.00320297711777891\\
520	0.00317759376545378\\
521	0.00315183288108833\\
522	0.00312568508669098\\
523	0.00309914048826814\\
524	0.00307218862662699\\
525	0.00304481842431517\\
526	0.00301701812888583\\
527	0.00298877525281087\\
528	0.00296007651051954\\
529	0.00293090775321448\\
530	0.00290125390231508\\
531	0.00287109888260652\\
532	0.00284042555644818\\
533	0.00280921566075377\\
534	0.00277744974899561\\
535	0.002745107141692\\
536	0.00271216588991355\\
537	0.00267860275770603\\
538	0.00264439323105173\\
539	0.00260951156316653\\
540	0.00257393086867269\\
541	0.00253762328263359\\
542	0.00250056020476996\\
543	0.00246271265464424\\
544	0.00242405177078421\\
545	0.00238454949588862\\
546	0.00234417950976365\\
547	0.00230291826941214\\
548	0.00226075729913756\\
549	0.00221769056880834\\
550	0.00217371502577192\\
551	0.00212883183532615\\
552	0.00208304983956979\\
553	0.00203638997391316\\
554	0.00198896231752856\\
555	0.00194072070159704\\
556	0.00189160323636402\\
557	0.00184153124543727\\
558	0.0017902977245096\\
559	0.00173782298923212\\
560	0.00168399686198448\\
561	0.0016286614295943\\
562	0.00157372532456192\\
563	0.00152029620686089\\
564	0.00146903631266427\\
565	0.0014186652881098\\
566	0.00136848386464934\\
567	0.00131849638653166\\
568	0.00126901022566881\\
569	0.00122069256463649\\
570	0.0011736084075615\\
571	0.00112676679780215\\
572	0.00108013609015035\\
573	0.00103366236371688\\
574	0.000987377889550623\\
575	0.000941691633156011\\
576	0.000896566979252233\\
577	0.000851509644641342\\
578	0.000806470720006763\\
579	0.0007614808141167\\
580	0.000716583918777049\\
581	0.000671829456087438\\
582	0.000627270354366442\\
583	0.000582961784628753\\
584	0.000538960396115172\\
585	0.000495323454366403\\
586	0.000452107712405483\\
587	0.000409367941602314\\
588	0.000367155075774456\\
589	0.0003255139412327\\
590	0.000284480716949123\\
591	0.000244080684870837\\
592	0.00020432808725126\\
593	0.00016523321767671\\
594	0.000126870159937007\\
595	8.96111722547515e-05\\
596	5.42660945238509e-05\\
597	2.28062284332055e-05\\
598	2.9204464504877e-07\\
599	0\\
600	0\\
};
\addplot [color=red,solid,forget plot]
  table[row sep=crcr]{%
1	0.00604132329944266\\
2	0.00604131925264746\\
3	0.00604131512902119\\
4	0.00604131092717107\\
5	0.00604130664568032\\
6	0.00604130228310772\\
7	0.00604129783798725\\
8	0.00604129330882765\\
9	0.00604128869411197\\
10	0.00604128399229723\\
11	0.00604127920181386\\
12	0.00604127432106537\\
13	0.0060412693484278\\
14	0.00604126428224932\\
15	0.00604125912084976\\
16	0.00604125386252011\\
17	0.00604124850552209\\
18	0.0060412430480875\\
19	0.00604123748841791\\
20	0.00604123182468409\\
21	0.00604122605502535\\
22	0.0060412201775492\\
23	0.0060412141903307\\
24	0.00604120809141197\\
25	0.00604120187880152\\
26	0.00604119555047378\\
27	0.00604118910436851\\
28	0.00604118253839014\\
29	0.0060411758504072\\
30	0.00604116903825171\\
31	0.00604116209971853\\
32	0.00604115503256479\\
33	0.00604114783450912\\
34	0.00604114050323102\\
35	0.00604113303637021\\
36	0.00604112543152595\\
37	0.00604111768625627\\
38	0.0060411097980773\\
39	0.00604110176446248\\
40	0.00604109358284187\\
41	0.00604108525060133\\
42	0.00604107676508178\\
43	0.00604106812357833\\
44	0.00604105932333964\\
45	0.00604105036156682\\
46	0.00604104123541281\\
47	0.00604103194198142\\
48	0.00604102247832642\\
49	0.00604101284145067\\
50	0.00604100302830524\\
51	0.0060409930357884\\
52	0.00604098286074466\\
53	0.00604097249996382\\
54	0.00604096195017998\\
55	0.0060409512080705\\
56	0.00604094027025489\\
57	0.00604092913329385\\
58	0.00604091779368803\\
59	0.00604090624787712\\
60	0.00604089449223851\\
61	0.00604088252308624\\
62	0.00604087033666972\\
63	0.00604085792917265\\
64	0.00604084529671161\\
65	0.00604083243533483\\
66	0.00604081934102101\\
67	0.00604080600967785\\
68	0.00604079243714074\\
69	0.00604077861917138\\
70	0.00604076455145631\\
71	0.00604075022960549\\
72	0.00604073564915083\\
73	0.00604072080554458\\
74	0.00604070569415793\\
75	0.00604069031027922\\
76	0.00604067464911248\\
77	0.00604065870577569\\
78	0.00604064247529903\\
79	0.0060406259526232\\
80	0.0060406091325977\\
81	0.00604059200997878\\
82	0.00604057457942783\\
83	0.00604055683550932\\
84	0.00604053877268885\\
85	0.00604052038533122\\
86	0.00604050166769827\\
87	0.00604048261394694\\
88	0.00604046321812696\\
89	0.00604044347417879\\
90	0.0060404233759313\\
91	0.00604040291709949\\
92	0.00604038209128214\\
93	0.00604036089195942\\
94	0.00604033931249039\\
95	0.00604031734611052\\
96	0.00604029498592913\\
97	0.00604027222492669\\
98	0.00604024905595212\\
99	0.00604022547172015\\
100	0.0060402014648083\\
101	0.00604017702765421\\
102	0.0060401521525525\\
103	0.00604012683165182\\
104	0.00604010105695174\\
105	0.00604007482029962\\
106	0.00604004811338731\\
107	0.0060400209277479\\
108	0.00603999325475223\\
109	0.00603996508560554\\
110	0.00603993641134386\\
111	0.0060399072228304\\
112	0.00603987751075174\\
113	0.00603984726561425\\
114	0.00603981647773998\\
115	0.00603978513726282\\
116	0.00603975323412439\\
117	0.00603972075806994\\
118	0.00603968769864406\\
119	0.00603965404518634\\
120	0.00603961978682697\\
121	0.00603958491248219\\
122	0.00603954941084964\\
123	0.00603951327040361\\
124	0.00603947647939026\\
125	0.00603943902582262\\
126	0.00603940089747556\\
127	0.0060393620818806\\
128	0.00603932256632063\\
129	0.00603928233782457\\
130	0.00603924138316183\\
131	0.00603919968883666\\
132	0.00603915724108239\\
133	0.00603911402585568\\
134	0.00603907002883038\\
135	0.00603902523539151\\
136	0.00603897963062892\\
137	0.00603893319933103\\
138	0.00603888592597824\\
139	0.00603883779473625\\
140	0.0060387887894494\\
141	0.00603873889363357\\
142	0.00603868809046935\\
143	0.00603863636279456\\
144	0.0060385836930971\\
145	0.00603853006350736\\
146	0.00603847545579058\\
147	0.00603841985133902\\
148	0.00603836323116402\\
149	0.00603830557588792\\
150	0.00603824686573561\\
151	0.00603818708052626\\
152	0.00603812619966465\\
153	0.0060380642021324\\
154	0.00603800106647893\\
155	0.00603793677081245\\
156	0.00603787129279056\\
157	0.00603780460961084\\
158	0.00603773669800116\\
159	0.00603766753420982\\
160	0.00603759709399551\\
161	0.00603752535261709\\
162	0.00603745228482317\\
163	0.00603737786484147\\
164	0.00603730206636806\\
165	0.00603722486255624\\
166	0.00603714622600539\\
167	0.00603706612874949\\
168	0.00603698454224547\\
169	0.0060369014373613\\
170	0.00603681678436404\\
171	0.00603673055290728\\
172	0.0060366427120188\\
173	0.00603655323008767\\
174	0.00603646207485128\\
175	0.00603636921338198\\
176	0.00603627461207375\\
177	0.00603617823662822\\
178	0.00603608005204077\\
179	0.00603598002258623\\
180	0.00603587811180431\\
181	0.00603577428248475\\
182	0.00603566849665226\\
183	0.00603556071555106\\
184	0.00603545089962927\\
185	0.00603533900852289\\
186	0.00603522500103951\\
187	0.00603510883514169\\
188	0.00603499046793007\\
189	0.00603486985562615\\
190	0.00603474695355463\\
191	0.00603462171612562\\
192	0.00603449409681624\\
193	0.00603436404815203\\
194	0.00603423152168799\\
195	0.00603409646798919\\
196	0.00603395883661098\\
197	0.00603381857607889\\
198	0.00603367563386799\\
199	0.00603352995638198\\
200	0.00603338148893178\\
201	0.00603323017571368\\
202	0.006033075959787\\
203	0.00603291878305142\\
204	0.00603275858622366\\
205	0.00603259530881392\\
206	0.00603242888910147\\
207	0.00603225926411005\\
208	0.00603208636958272\\
209	0.00603191013995585\\
210	0.006031730508333\\
211	0.00603154740645792\\
212	0.00603136076468708\\
213	0.00603117051196166\\
214	0.00603097657577886\\
215	0.00603077888216255\\
216	0.0060305773556334\\
217	0.00603037191917825\\
218	0.00603016249421888\\
219	0.00602994900058007\\
220	0.0060297313564569\\
221	0.0060295094783815\\
222	0.00602928328118882\\
223	0.00602905267798188\\
224	0.00602881758009614\\
225	0.00602857789706305\\
226	0.00602833353657291\\
227	0.00602808440443683\\
228	0.00602783040454781\\
229	0.00602757143884117\\
230	0.0060273074072538\\
231	0.00602703820768281\\
232	0.00602676373594302\\
233	0.00602648388572377\\
234	0.00602619854854449\\
235	0.00602590761370964\\
236	0.00602561096826228\\
237	0.00602530849693701\\
238	0.00602500008211152\\
239	0.00602468560375743\\
240	0.00602436493938971\\
241	0.00602403796401531\\
242	0.0060237045500804\\
243	0.0060233645674167\\
244	0.0060230178831864\\
245	0.00602266436182607\\
246	0.00602230386498922\\
247	0.00602193625148769\\
248	0.00602156137723161\\
249	0.00602117909516823\\
250	0.00602078925521923\\
251	0.00602039170421677\\
252	0.00601998628583804\\
253	0.00601957284053832\\
254	0.00601915120548276\\
255	0.00601872121447631\\
256	0.00601828269789236\\
257	0.00601783548259966\\
258	0.00601737939188752\\
259	0.0060169142453895\\
260	0.00601643985900513\\
261	0.00601595604482015\\
262	0.00601546261102462\\
263	0.00601495936182939\\
264	0.00601444609738052\\
265	0.00601392261367178\\
266	0.00601338870245519\\
267	0.00601284415114934\\
268	0.00601228874274578\\
269	0.00601172225571316\\
270	0.0060111444638991\\
271	0.00601055513642994\\
272	0.00600995403760808\\
273	0.00600934092680692\\
274	0.00600871555836368\\
275	0.00600807768146947\\
276	0.00600742704005722\\
277	0.00600676337268714\\
278	0.0060060864124301\\
279	0.00600539588674823\\
280	0.00600469151737163\\
281	0.00600397302017226\\
282	0.00600324010503467\\
283	0.00600249247572393\\
284	0.00600172982975035\\
285	0.00600095185823096\\
286	0.00600015824574771\\
287	0.00599934867020256\\
288	0.00599852280266881\\
289	0.00599768030723909\\
290	0.00599682084086989\\
291	0.00599594405322201\\
292	0.00599504958649758\\
293	0.00599413707527296\\
294	0.00599320614632769\\
295	0.00599225641846931\\
296	0.0059912875023539\\
297	0.00599029900030216\\
298	0.00598929050611112\\
299	0.0059882616048609\\
300	0.0059872118727169\\
301	0.00598614087672656\\
302	0.00598504817461133\\
303	0.00598393331455281\\
304	0.00598279583497354\\
305	0.00598163526431188\\
306	0.00598045112079092\\
307	0.00597924291218078\\
308	0.00597801013555466\\
309	0.00597675227703763\\
310	0.00597546881154838\\
311	0.00597415920253324\\
312	0.00597282290169219\\
313	0.00597145934869616\\
314	0.00597006797089525\\
315	0.00596864818301673\\
316	0.00596719938685158\\
317	0.00596572097092812\\
318	0.00596421231018518\\
319	0.00596267276563555\\
320	0.00596110168401903\\
321	0.00595949839744503\\
322	0.00595786222302412\\
323	0.00595619246248798\\
324	0.00595448840179789\\
325	0.00595274931074066\\
326	0.00595097444251194\\
327	0.00594916303328601\\
328	0.00594731430177211\\
329	0.00594542744875613\\
330	0.00594350165662732\\
331	0.00594153608888943\\
332	0.00593952988965533\\
333	0.00593748218312489\\
334	0.00593539207304499\\
335	0.00593325864215104\\
336	0.00593108095158899\\
337	0.00592885804031686\\
338	0.0059265889244846\\
339	0.00592427259679165\\
340	0.00592190802582186\\
341	0.00591949415535425\\
342	0.00591702990364704\\
343	0.00591451416269395\\
344	0.00591194579745183\\
345	0.00590932364503768\\
346	0.00590664651389399\\
347	0.00590391318292038\\
348	0.00590112240057143\\
349	0.00589827288391688\\
350	0.005895363317663\\
351	0.00589239235313299\\
352	0.00588935860720373\\
353	0.00588626066119526\\
354	0.00588309705971205\\
355	0.00587986630943471\\
356	0.00587656687785905\\
357	0.00587319719197867\\
358	0.00586975563690877\\
359	0.0058662405544477\\
360	0.00586265024157304\\
361	0.00585898294886799\\
362	0.00585523687887468\\
363	0.00585141018436922\\
364	0.00584750096655261\\
365	0.00584350727315025\\
366	0.00583942709641713\\
367	0.00583525837107253\\
368	0.0058309989721399\\
369	0.0058266467126732\\
370	0.00582219934137166\\
371	0.00581765454007395\\
372	0.00581300992110441\\
373	0.00580826302448079\\
374	0.00580341131499454\\
375	0.00579845217914324\\
376	0.00579338292190432\\
377	0.00578820076334026\\
378	0.00578290283502511\\
379	0.00577748617628099\\
380	0.0057719477302107\\
381	0.0057662843395138\\
382	0.00576049274206254\\
383	0.00575456956620739\\
384	0.00574851132579433\\
385	0.00574231441499663\\
386	0.00573597510292741\\
387	0.00572948952787593\\
388	0.00572285369113555\\
389	0.00571606345047797\\
390	0.00570911451306406\\
391	0.00570200242799517\\
392	0.00569472257847336\\
393	0.0056872701735383\\
394	0.0056796402393677\\
395	0.00567182761012981\\
396	0.00566382691837387\\
397	0.00565563258491992\\
398	0.00564723880811962\\
399	0.00563863955222837\\
400	0.00562982853558676\\
401	0.00562079922056375\\
402	0.00561154480310268\\
403	0.0056020582030006\\
404	0.00559233204912305\\
405	0.00558235866513765\\
406	0.00557213005533737\\
407	0.00556163788835732\\
408	0.00555087348540174\\
409	0.00553982780368971\\
410	0.00552849141492641\\
411	0.00551685448156283\\
412	0.00550490672915224\\
413	0.00549263740829848\\
414	0.00548003525961385\\
415	0.00546708849749552\\
416	0.00545378479780484\\
417	0.0054401112754309\\
418	0.00542605441187016\\
419	0.0054116000085708\\
420	0.00539673314876577\\
421	0.00538143816631\\
422	0.00536569862809995\\
423	0.00534949732924722\\
424	0.00533281630784088\\
425	0.00531563700179651\\
426	0.00529794031994624\\
427	0.00527970666589112\\
428	0.00526091616858311\\
429	0.00524154930988612\\
430	0.00522158670897148\\
431	0.00520100901224537\\
432	0.00517979689563761\\
433	0.00515793102197964\\
434	0.00513539190146522\\
435	0.00511215960693481\\
436	0.00508821329996635\\
437	0.0050635306819359\\
438	0.00503808544981885\\
439	0.00501184283113749\\
440	0.00498475355696809\\
441	0.00495677028297857\\
442	0.00492784673408646\\
443	0.004897939422489\\
444	0.00486701000135425\\
445	0.00483502847171133\\
446	0.00480197750681774\\
447	0.00476785824140474\\
448	0.00473269801185335\\
449	0.00469656058548712\\
450	0.0046595596934835\\
451	0.00462187671951774\\
452	0.00458393393414347\\
453	0.00454674949141144\\
454	0.00451044051279913\\
455	0.00447513188695105\\
456	0.00444095508964119\\
457	0.00440804619863455\\
458	0.00437654283428951\\
459	0.00434657956098983\\
460	0.00431828136589076\\
461	0.00429175330593373\\
462	0.00426706771200028\\
463	0.00424424587992064\\
464	0.00422323385217898\\
465	0.0042034722512688\\
466	0.00418422708684037\\
467	0.00416550069813697\\
468	0.00414728742576752\\
469	0.00412957230571683\\
470	0.00411232980882289\\
471	0.00409552275599316\\
472	0.00407910158538559\\
473	0.00406300421735797\\
474	0.00404715689755966\\
475	0.00403147654769942\\
476	0.00401587544089208\\
477	0.00400027999628674\\
478	0.0039846749808583\\
479	0.00396904354869625\\
480	0.00395336747204813\\
481	0.0039376274716649\\
482	0.00392180365609093\\
483	0.00390587607449125\\
484	0.00388982537664086\\
485	0.00387363355521733\\
486	0.00385728471522756\\
487	0.00384076577288425\\
488	0.00382406692090728\\
489	0.00380718053226114\\
490	0.00379009904886561\\
491	0.0037728150640437\\
492	0.00375532140232142\\
493	0.00373761119101988\\
494	0.00371967791683657\\
495	0.00370151545967474\\
496	0.00368311809563851\\
497	0.00366448046189818\\
498	0.00364559747892885\\
499	0.0036264642317458\\
500	0.00360707582316718\\
501	0.00358742735026712\\
502	0.00356751390046808\\
503	0.00354733054335028\\
504	0.00352687231790604\\
505	0.00350613421524147\\
506	0.00348511115709701\\
507	0.00346379797101353\\
508	0.00344218936347826\\
509	0.00342027989286092\\
510	0.00339806394424488\\
511	0.00337553570809208\\
512	0.00335268916009062\\
513	0.00332951803879312\\
514	0.00330601582100793\\
515	0.00328217569492879\\
516	0.00325799053100626\\
517	0.00323345285057013\\
518	0.00320855479219949\\
519	0.00318328807580211\\
520	0.00315764396430225\\
521	0.00313161322275183\\
522	0.00310518607459123\\
523	0.00307835215483463\\
524	0.00305110046025353\\
525	0.00302341929672944\\
526	0.00299529622407598\\
527	0.00296671799879961\\
528	0.00293767051549207\\
529	0.00290813874783711\\
530	0.0028781066905927\\
531	0.0028475573044006\\
532	0.00281647246590783\\
533	0.00278483292649969\\
534	0.00275261828398187\\
535	0.0027198069728484\\
536	0.0026863762804127\\
537	0.00265230239814621\\
538	0.00261756052016224\\
539	0.00258212500404021\\
540	0.00254596961341247\\
541	0.00250906786742201\\
542	0.00247139352909162\\
543	0.00243292127202608\\
544	0.00239362755840972\\
545	0.00235349175704467\\
546	0.00231249727013801\\
547	0.00227064453612754\\
548	0.00222794134987418\\
549	0.00218440577001577\\
550	0.00214013908144317\\
551	0.00209511622696201\\
552	0.00204927801102705\\
553	0.00200254729293833\\
554	0.00195472088222525\\
555	0.00190573084441342\\
556	0.00185549613625779\\
557	0.00180390471392892\\
558	0.00175081620496203\\
559	0.00169770725359811\\
560	0.00164609814802067\\
561	0.0015966641520743\\
562	0.00154794430499767\\
563	0.00149930786152006\\
564	0.00145063724791586\\
565	0.00140192736334399\\
566	0.0013538859561472\\
567	0.00130694443089763\\
568	0.00126075754464001\\
569	0.00121470715974243\\
570	0.00116868405784354\\
571	0.00112272218384932\\
572	0.00107685136522041\\
573	0.00103136023930306\\
574	0.000986412336410125\\
575	0.000941498720255945\\
576	0.000896516120383694\\
577	0.000851490573360361\\
578	0.000806462517147885\\
579	0.000761476872772062\\
580	0.000716581800501493\\
581	0.00067182827412899\\
582	0.00062726968466177\\
583	0.000582961403638049\\
584	0.000538960186060454\\
585	0.000495323349035172\\
586	0.000452107667820891\\
587	0.000409367927672478\\
588	0.000367155072958957\\
589	0.000325513941232698\\
590	0.000284480716949121\\
591	0.000244080684870835\\
592	0.00020432808725126\\
593	0.000165233217676711\\
594	0.000126870159937007\\
595	8.9611172254752e-05\\
596	5.42660945238507e-05\\
597	2.28062284332057e-05\\
598	2.9204464504877e-07\\
599	0\\
600	0\\
};
\addplot [color=mycolor20,solid,forget plot]
  table[row sep=crcr]{%
1	0.00604070286126672\\
2	0.00604069768145596\\
3	0.00604069240011982\\
4	0.00604068701536704\\
5	0.00604068152527344\\
6	0.00604067592788133\\
7	0.00604067022119914\\
8	0.00604066440320073\\
9	0.00604065847182493\\
10	0.006040652424975\\
11	0.00604064626051793\\
12	0.00604063997628406\\
13	0.00604063357006636\\
14	0.00604062703961985\\
15	0.00604062038266108\\
16	0.00604061359686745\\
17	0.00604060667987656\\
18	0.00604059962928574\\
19	0.00604059244265129\\
20	0.00604058511748774\\
21	0.00604057765126749\\
22	0.00604057004141987\\
23	0.00604056228533052\\
24	0.0060405543803408\\
25	0.00604054632374706\\
26	0.00604053811279983\\
27	0.00604052974470326\\
28	0.00604052121661428\\
29	0.00604051252564191\\
30	0.0060405036688465\\
31	0.0060404946432389\\
32	0.00604048544577976\\
33	0.00604047607337873\\
34	0.00604046652289367\\
35	0.00604045679112977\\
36	0.00604044687483879\\
37	0.00604043677071818\\
38	0.00604042647541023\\
39	0.00604041598550124\\
40	0.00604040529752056\\
41	0.00604039440793977\\
42	0.0060403833131717\\
43	0.00604037200956956\\
44	0.00604036049342594\\
45	0.00604034876097188\\
46	0.0060403368083759\\
47	0.00604032463174296\\
48	0.00604031222711347\\
49	0.00604029959046228\\
50	0.00604028671769763\\
51	0.00604027360465999\\
52	0.00604026024712113\\
53	0.00604024664078291\\
54	0.00604023278127614\\
55	0.00604021866415943\\
56	0.00604020428491812\\
57	0.00604018963896296\\
58	0.00604017472162895\\
59	0.00604015952817414\\
60	0.00604014405377826\\
61	0.00604012829354153\\
62	0.00604011224248334\\
63	0.00604009589554082\\
64	0.00604007924756761\\
65	0.00604006229333234\\
66	0.00604004502751723\\
67	0.00604002744471677\\
68	0.00604000953943605\\
69	0.00603999130608937\\
70	0.00603997273899869\\
71	0.00603995383239199\\
72	0.00603993458040168\\
73	0.00603991497706302\\
74	0.00603989501631235\\
75	0.00603987469198551\\
76	0.00603985399781589\\
77	0.00603983292743276\\
78	0.00603981147435948\\
79	0.00603978963201152\\
80	0.00603976739369456\\
81	0.00603974475260268\\
82	0.00603972170181615\\
83	0.00603969823429948\\
84	0.00603967434289937\\
85	0.00603965002034245\\
86	0.00603962525923319\\
87	0.00603960005205163\\
88	0.00603957439115098\\
89	0.00603954826875544\\
90	0.00603952167695767\\
91	0.00603949460771633\\
92	0.00603946705285358\\
93	0.00603943900405253\\
94	0.0060394104528545\\
95	0.00603938139065637\\
96	0.00603935180870775\\
97	0.00603932169810817\\
98	0.00603929104980418\\
99	0.00603925985458628\\
100	0.00603922810308589\\
101	0.0060391957857722\\
102	0.00603916289294902\\
103	0.0060391294147513\\
104	0.00603909534114195\\
105	0.00603906066190825\\
106	0.00603902536665826\\
107	0.00603898944481726\\
108	0.00603895288562394\\
109	0.00603891567812664\\
110	0.00603887781117927\\
111	0.00603883927343742\\
112	0.00603880005335422\\
113	0.00603876013917593\\
114	0.00603871951893777\\
115	0.00603867818045937\\
116	0.00603863611134024\\
117	0.00603859329895501\\
118	0.00603854973044864\\
119	0.00603850539273156\\
120	0.00603846027247447\\
121	0.0060384143561033\\
122	0.00603836762979378\\
123	0.00603832007946602\\
124	0.00603827169077896\\
125	0.00603822244912454\\
126	0.00603817233962188\\
127	0.00603812134711125\\
128	0.00603806945614792\\
129	0.0060380166509958\\
130	0.00603796291562085\\
131	0.00603790823368464\\
132	0.00603785258853736\\
133	0.00603779596321088\\
134	0.0060377383404116\\
135	0.00603767970251324\\
136	0.00603762003154909\\
137	0.00603755930920449\\
138	0.0060374975168089\\
139	0.0060374346353279\\
140	0.0060373706453548\\
141	0.00603730552710236\\
142	0.00603723926039393\\
143	0.00603717182465477\\
144	0.00603710319890287\\
145	0.00603703336173978\\
146	0.00603696229134103\\
147	0.00603688996544644\\
148	0.00603681636135023\\
149	0.00603674145589082\\
150	0.00603666522544051\\
151	0.00603658764589482\\
152	0.0060365086926617\\
153	0.00603642834065032\\
154	0.00603634656425989\\
155	0.00603626333736804\\
156	0.00603617863331896\\
157	0.00603609242491136\\
158	0.00603600468438619\\
159	0.00603591538341399\\
160	0.0060358244930821\\
161	0.00603573198388151\\
162	0.00603563782569354\\
163	0.00603554198777613\\
164	0.00603544443874995\\
165	0.00603534514658427\\
166	0.00603524407858244\\
167	0.00603514120136712\\
168	0.00603503648086523\\
169	0.0060349298822928\\
170	0.00603482137013902\\
171	0.0060347109081507\\
172	0.00603459845931581\\
173	0.00603448398584715\\
174	0.00603436744916537\\
175	0.00603424880988208\\
176	0.00603412802778226\\
177	0.00603400506180656\\
178	0.00603387987003342\\
179	0.00603375240966048\\
180	0.00603362263698609\\
181	0.00603349050739019\\
182	0.00603335597531511\\
183	0.00603321899424582\\
184	0.00603307951668995\\
185	0.00603293749415748\\
186	0.00603279287714004\\
187	0.00603264561508996\\
188	0.00603249565639885\\
189	0.00603234294837584\\
190	0.00603218743722551\\
191	0.00603202906802543\\
192	0.00603186778470337\\
193	0.00603170353001399\\
194	0.00603153624551522\\
195	0.00603136587154434\\
196	0.0060311923471935\\
197	0.0060310156102849\\
198	0.0060308355973455\\
199	0.00603065224358139\\
200	0.00603046548285162\\
201	0.00603027524764155\\
202	0.00603008146903589\\
203	0.00602988407669103\\
204	0.00602968299880715\\
205	0.00602947816209949\\
206	0.00602926949176939\\
207	0.00602905691147463\\
208	0.00602884034329925\\
209	0.00602861970772277\\
210	0.00602839492358883\\
211	0.0060281659080733\\
212	0.00602793257665158\\
213	0.0060276948430654\\
214	0.00602745261928877\\
215	0.00602720581549342\\
216	0.00602695434001352\\
217	0.00602669809930934\\
218	0.00602643699793054\\
219	0.00602617093847832\\
220	0.00602589982156705\\
221	0.00602562354578477\\
222	0.00602534200765297\\
223	0.00602505510158548\\
224	0.00602476271984619\\
225	0.00602446475250616\\
226	0.00602416108739942\\
227	0.00602385161007798\\
228	0.00602353620376556\\
229	0.00602321474931045\\
230	0.00602288712513714\\
231	0.00602255320719681\\
232	0.00602221286891671\\
233	0.00602186598114831\\
234	0.00602151241211429\\
235	0.00602115202735406\\
236	0.00602078468966836\\
237	0.0060204102590622\\
238	0.00602002859268669\\
239	0.00601963954477947\\
240	0.00601924296660363\\
241	0.0060188387063855\\
242	0.00601842660925068\\
243	0.00601800651715886\\
244	0.00601757826883696\\
245	0.00601714169971099\\
246	0.0060166966418361\\
247	0.00601624292382536\\
248	0.0060157803707767\\
249	0.00601530880419837\\
250	0.00601482804193277\\
251	0.00601433789807843\\
252	0.00601383818291052\\
253	0.00601332870279949\\
254	0.00601280926012784\\
255	0.00601227965320519\\
256	0.00601173967618163\\
257	0.00601118911895872\\
258	0.00601062776709899\\
259	0.00601005540173316\\
260	0.00600947179946547\\
261	0.00600887673227664\\
262	0.00600826996742509\\
263	0.00600765126734565\\
264	0.0060070203895459\\
265	0.00600637708650021\\
266	0.00600572110554137\\
267	0.0060050521887495\\
268	0.00600437007283856\\
269	0.00600367448903991\\
270	0.00600296516298329\\
271	0.00600224181457464\\
272	0.00600150415787112\\
273	0.00600075190095296\\
274	0.00599998474579196\\
275	0.0059992023881168\\
276	0.0059984045172748\\
277	0.00599759081609012\\
278	0.00599676096071829\\
279	0.00599591462049686\\
280	0.005995051457792\\
281	0.00599417112784122\\
282	0.00599327327859168\\
283	0.0059923575505342\\
284	0.0059914235765328\\
285	0.00599047098164961\\
286	0.00598949938296499\\
287	0.00598850838939272\\
288	0.00598749760149035\\
289	0.00598646661126413\\
290	0.0059854150019687\\
291	0.00598434234790135\\
292	0.0059832482141907\\
293	0.00598213215657961\\
294	0.00598099372120207\\
295	0.00597983244435416\\
296	0.00597864785225872\\
297	0.00597743946082362\\
298	0.0059762067753936\\
299	0.00597494929049533\\
300	0.00597366648957539\\
301	0.00597235784473117\\
302	0.00597102281643418\\
303	0.00596966085324572\\
304	0.00596827139152416\\
305	0.00596685385512431\\
306	0.00596540765508857\\
307	0.00596393218932975\\
308	0.00596242684230389\\
309	0.00596089098467301\\
310	0.005959323972958\\
311	0.00595772514918041\\
312	0.00595609384049316\\
313	0.00595442935879954\\
314	0.00595273100035977\\
315	0.00595099804538495\\
316	0.00594922975761662\\
317	0.00594742538389277\\
318	0.00594558415369862\\
319	0.00594370527870151\\
320	0.00594178795226922\\
321	0.00593983134897069\\
322	0.00593783462405842\\
323	0.00593579691293132\\
324	0.0059337173305774\\
325	0.00593159497099554\\
326	0.00592942890659603\\
327	0.0059272181875775\\
328	0.0059249618412794\\
329	0.00592265887150968\\
330	0.00592030825784598\\
331	0.00591790895490918\\
332	0.00591545989160667\\
333	0.00591295997034463\\
334	0.00591040806620996\\
335	0.00590780302611918\\
336	0.00590514366793275\\
337	0.00590242877953272\\
338	0.00589965711786155\\
339	0.00589682740791893\\
340	0.00589393834171451\\
341	0.00589098857718418\\
342	0.00588797673707183\\
343	0.00588490140775219\\
344	0.0058817611380014\\
345	0.00587855443771407\\
346	0.00587527977656284\\
347	0.00587193558259683\\
348	0.00586852024077776\\
349	0.00586503209146127\\
350	0.00586146942880643\\
351	0.00585783049911073\\
352	0.00585411349906987\\
353	0.00585031657395378\\
354	0.00584643781567608\\
355	0.00584247526076415\\
356	0.00583842688824244\\
357	0.0058342906174133\\
358	0.00583006430552663\\
359	0.00582574574533156\\
360	0.00582133266250288\\
361	0.00581682271293385\\
362	0.0058122134798853\\
363	0.00580750247098027\\
364	0.00580268711502714\\
365	0.00579776475864244\\
366	0.00579273266261862\\
367	0.00578758799798392\\
368	0.00578232784208436\\
369	0.00577694917450169\\
370	0.00577144887262362\\
371	0.0057658237069727\\
372	0.00576007033631017\\
373	0.00575418530222471\\
374	0.00574816502335601\\
375	0.00574200578947321\\
376	0.00573570375528036\\
377	0.00572925493391668\\
378	0.00572265519015235\\
379	0.00571590023328521\\
380	0.00570898560975258\\
381	0.00570190669547667\\
382	0.00569465868796211\\
383	0.00568723659813321\\
384	0.00567963524178665\\
385	0.00567184923042796\\
386	0.00566387296297438\\
387	0.00565570061850434\\
388	0.00564732614852558\\
389	0.00563874326842201\\
390	0.00562994544885467\\
391	0.00562092590397597\\
392	0.00561167757684297\\
393	0.0056021931235798\\
394	0.00559246489594869\\
395	0.00558248492210143\\
396	0.00557224488527129\\
397	0.00556173610014804\\
398	0.00555094948656369\\
399	0.00553987553928657\\
400	0.00552850428898409\\
401	0.00551682525811876\\
402	0.00550482743946284\\
403	0.00549249927104814\\
404	0.00547982863506249\\
405	0.00546680280346584\\
406	0.00545340840135631\\
407	0.00543963139660914\\
408	0.00542545709129351\\
409	0.00541087024193215\\
410	0.00539585526736261\\
411	0.00538039629792995\\
412	0.00536447723384199\\
413	0.00534808180737729\\
414	0.00533119353783731\\
415	0.0053137956986983\\
416	0.00529587151388881\\
417	0.00527740443553554\\
418	0.00525837840941861\\
419	0.00523877743823224\\
420	0.00521858532440277\\
421	0.00519778529766863\\
422	0.00517635939137337\\
423	0.00515428747691237\\
424	0.00513154562729427\\
425	0.00510810224031989\\
426	0.00508391495677255\\
427	0.00505893916759164\\
428	0.00503312793904506\\
429	0.00500643350322129\\
430	0.00497881596042503\\
431	0.00495024230059195\\
432	0.00492068858544584\\
433	0.00489014442261511\\
434	0.00485861893677142\\
435	0.0048261487135882\\
436	0.00479280832293565\\
437	0.00475872421739817\\
438	0.00472409309455758\\
439	0.0046897229291746\\
440	0.00465601899576579\\
441	0.00462308586759372\\
442	0.00459103525379076\\
443	0.00455998509454473\\
444	0.00453005799718654\\
445	0.00450137864226732\\
446	0.00447406996077481\\
447	0.00444824758907592\\
448	0.00442401202173317\\
449	0.00440143778640491\\
450	0.00438055781551818\\
451	0.00436134261793029\\
452	0.00434351468438177\\
453	0.00432616324410694\\
454	0.00430929422423594\\
455	0.00429290781882491\\
456	0.00427699674321518\\
457	0.00426154508181158\\
458	0.00424652720145676\\
459	0.00423190685177035\\
460	0.00421763662973471\\
461	0.004203658089645\\
462	0.0041899028287776\\
463	0.00417629507945139\\
464	0.0041627565142112\\
465	0.004149235382823\\
466	0.00413571810586105\\
467	0.00412218965680862\\
468	0.00410863376027093\\
469	0.00409503317653824\\
470	0.00408137008153313\\
471	0.0040676265459852\\
472	0.00405378510905811\\
473	0.00403982942760372\\
474	0.00402574495899102\\
475	0.00401151960005543\\
476	0.00399714414979814\\
477	0.0039826119151884\\
478	0.00396791625508381\\
479	0.00395305065346735\\
480	0.00393800879180704\\
481	0.00392278461619124\\
482	0.00390737239387512\\
483	0.00389176675294497\\
484	0.00387596269825045\\
485	0.00385995559697531\\
486	0.00384374112887292\\
487	0.00382731520011974\\
488	0.00381067382738024\\
489	0.00379381305678002\\
490	0.00377672896545118\\
491	0.00375941765990687\\
492	0.00374187527095354\\
493	0.0037240979450396\\
494	0.00370608183220922\\
495	0.00368782307117653\\
496	0.00366931777244603\\
497	0.00365056200083808\\
498	0.00363155175914802\\
499	0.0036122829748172\\
500	0.00359275149114712\\
501	0.00357295305835337\\
502	0.00355288332358055\\
503	0.00353253781991185\\
504	0.00351191195442204\\
505	0.00349100099532988\\
506	0.00346980005829799\\
507	0.00344830409190142\\
508	0.00342650786223379\\
509	0.00340440593654245\\
510	0.00338199266568192\\
511	0.00335926216506952\\
512	0.00333620829390472\\
513	0.00331282463253544\\
514	0.00328910445785155\\
515	0.00326504071658356\\
516	0.00324062599638607\\
517	0.00321585249459072\\
518	0.00319071198452735\\
519	0.00316519577933604\\
520	0.00313929469323235\\
521	0.00311299900024939\\
522	0.00308629839056859\\
523	0.0030591819246678\\
524	0.00303163798565948\\
525	0.00300365423037989\\
526	0.00297521754003508\\
527	0.00294631397152773\\
528	0.00291692871099567\\
529	0.00288704603161653\\
530	0.00285664925839858\\
531	0.00282572074352561\\
532	0.00279424185689702\\
533	0.00276219299786336\\
534	0.00272955363586645\\
535	0.00269630238983699\\
536	0.00266241715889706\\
537	0.00262787532050961\\
538	0.00259265401715977\\
539	0.00255673055832753\\
540	0.00252008296455489\\
541	0.00248269066657626\\
542	0.00244453538964067\\
543	0.00240560232750984\\
544	0.00236588246753321\\
545	0.00232537716519942\\
546	0.00228415675127574\\
547	0.00224224875684687\\
548	0.00219959959602875\\
549	0.00215614081365568\\
550	0.0021116893143768\\
551	0.00206616683318269\\
552	0.00201950322840626\\
553	0.00197160885480873\\
554	0.00192240234639964\\
555	0.00187175949147671\\
556	0.00182016778867611\\
557	0.0017700079133193\\
558	0.00172189222910075\\
559	0.00167483274808183\\
560	0.00162776118290781\\
561	0.00158051982821157\\
562	0.0015330935121052\\
563	0.00148554422170036\\
564	0.00143866178250926\\
565	0.00139285578388573\\
566	0.0013474996131113\\
567	0.00130214936842766\\
568	0.00125674675529522\\
569	0.00121131505068596\\
570	0.00116589341138373\\
571	0.00112058409493486\\
572	0.00107579239048085\\
573	0.00103115420667899\\
574	0.000986382803600775\\
575	0.00094149076835946\\
576	0.000896513086086129\\
577	0.000851489245902153\\
578	0.000806461868510235\\
579	0.000761476522194667\\
580	0.000716581604601778\\
581	0.000671828163310488\\
582	0.00062726962209076\\
583	0.000582961369628161\\
584	0.000538960169322208\\
585	0.000495323342111254\\
586	0.000452107665705174\\
587	0.000409367927260633\\
588	0.000367155072958956\\
589	0.000325513941232701\\
590	0.000284480716949122\\
591	0.000244080684870839\\
592	0.000204328087251261\\
593	0.000165233217676712\\
594	0.000126870159937008\\
595	8.9611172254752e-05\\
596	5.42660945238509e-05\\
597	2.28062284332057e-05\\
598	2.9204464504877e-07\\
599	0\\
600	0\\
};
\addplot [color=mycolor21,solid,forget plot]
  table[row sep=crcr]{%
1	0.0060405115386396\\
2	0.00604050497580479\\
3	0.00604049827929662\\
4	0.0060404914465168\\
5	0.00604048447481996\\
6	0.0060404773615129\\
7	0.00604047010385364\\
8	0.00604046269905084\\
9	0.00604045514426286\\
10	0.00604044743659698\\
11	0.0060404395731087\\
12	0.00604043155080067\\
13	0.00604042336662207\\
14	0.00604041501746772\\
15	0.00604040650017708\\
16	0.00604039781153354\\
17	0.00604038894826346\\
18	0.00604037990703524\\
19	0.00604037068445846\\
20	0.00604036127708301\\
21	0.00604035168139799\\
22	0.00604034189383093\\
23	0.00604033191074684\\
24	0.00604032172844708\\
25	0.00604031134316853\\
26	0.00604030075108255\\
27	0.00604028994829395\\
28	0.00604027893083998\\
29	0.00604026769468939\\
30	0.00604025623574118\\
31	0.0060402445498238\\
32	0.0060402326326939\\
33	0.00604022048003529\\
34	0.00604020808745783\\
35	0.00604019545049642\\
36	0.00604018256460975\\
37	0.0060401694251792\\
38	0.00604015602750773\\
39	0.00604014236681863\\
40	0.0060401284382545\\
41	0.00604011423687585\\
42	0.00604009975766\\
43	0.00604008499549986\\
44	0.00604006994520267\\
45	0.00604005460148872\\
46	0.00604003895899013\\
47	0.00604002301224949\\
48	0.00604000675571866\\
49	0.00603999018375738\\
50	0.00603997329063187\\
51	0.00603995607051361\\
52	0.00603993851747789\\
53	0.00603992062550243\\
54	0.00603990238846597\\
55	0.00603988380014686\\
56	0.00603986485422158\\
57	0.00603984554426332\\
58	0.00603982586374054\\
59	0.00603980580601526\\
60	0.00603978536434184\\
61	0.00603976453186516\\
62	0.00603974330161922\\
63	0.00603972166652551\\
64	0.00603969961939137\\
65	0.00603967715290844\\
66	0.00603965425965089\\
67	0.00603963093207381\\
68	0.00603960716251152\\
69	0.00603958294317581\\
70	0.00603955826615418\\
71	0.00603953312340808\\
72	0.0060395075067711\\
73	0.00603948140794716\\
74	0.00603945481850856\\
75	0.00603942772989412\\
76	0.00603940013340739\\
77	0.00603937202021446\\
78	0.00603934338134211\\
79	0.00603931420767584\\
80	0.00603928448995767\\
81	0.00603925421878412\\
82	0.0060392233846041\\
83	0.00603919197771672\\
84	0.00603915998826905\\
85	0.00603912740625393\\
86	0.00603909422150771\\
87	0.00603906042370773\\
88	0.00603902600237024\\
89	0.00603899094684771\\
90	0.00603895524632647\\
91	0.00603891888982421\\
92	0.00603888186618738\\
93	0.00603884416408854\\
94	0.00603880577202371\\
95	0.00603876667830962\\
96	0.00603872687108093\\
97	0.00603868633828738\\
98	0.00603864506769076\\
99	0.00603860304686205\\
100	0.00603856026317835\\
101	0.0060385167038197\\
102	0.00603847235576585\\
103	0.00603842720579312\\
104	0.00603838124047081\\
105	0.00603833444615807\\
106	0.00603828680900013\\
107	0.00603823831492472\\
108	0.00603818894963851\\
109	0.00603813869862311\\
110	0.0060380875471314\\
111	0.00603803548018332\\
112	0.00603798248256189\\
113	0.00603792853880897\\
114	0.00603787363322096\\
115	0.0060378177498443\\
116	0.00603776087247091\\
117	0.00603770298463356\\
118	0.00603764406960102\\
119	0.00603758411037308\\
120	0.0060375230896755\\
121	0.00603746098995471\\
122	0.00603739779337252\\
123	0.00603733348180048\\
124	0.00603726803681418\\
125	0.00603720143968747\\
126	0.00603713367138634\\
127	0.00603706471256278\\
128	0.00603699454354828\\
129	0.00603692314434729\\
130	0.00603685049463051\\
131	0.0060367765737278\\
132	0.00603670136062105\\
133	0.00603662483393677\\
134	0.0060365469719385\\
135	0.00603646775251884\\
136	0.00603638715319152\\
137	0.006036305151083\\
138	0.00603622172292386\\
139	0.00603613684504009\\
140	0.00603605049334386\\
141	0.00603596264332425\\
142	0.00603587327003759\\
143	0.00603578234809756\\
144	0.00603568985166492\\
145	0.00603559575443699\\
146	0.00603550002963689\\
147	0.00603540265000233\\
148	0.00603530358777422\\
149	0.00603520281468479\\
150	0.00603510030194546\\
151	0.00603499602023439\\
152	0.00603488993968359\\
153	0.00603478202986581\\
154	0.00603467225978081\\
155	0.00603456059784149\\
156	0.00603444701185953\\
157	0.00603433146903064\\
158	0.0060342139359194\\
159	0.00603409437844369\\
160	0.00603397276185872\\
161	0.00603384905074067\\
162	0.00603372320896976\\
163	0.0060335951997131\\
164	0.00603346498540687\\
165	0.00603333252773816\\
166	0.00603319778762635\\
167	0.00603306072520401\\
168	0.00603292129979739\\
169	0.00603277946990622\\
170	0.00603263519318343\\
171	0.00603248842641388\\
172	0.00603233912549306\\
173	0.00603218724540499\\
174	0.0060320327401998\\
175	0.00603187556297068\\
176	0.00603171566583043\\
177	0.00603155299988751\\
178	0.0060313875152214\\
179	0.00603121916085778\\
180	0.00603104788474286\\
181	0.00603087363371748\\
182	0.00603069635349043\\
183	0.00603051598861155\\
184	0.00603033248244412\\
185	0.00603014577713679\\
186	0.00602995581359512\\
187	0.0060297625314524\\
188	0.00602956586904013\\
189	0.00602936576335804\\
190	0.00602916215004345\\
191	0.00602895496334029\\
192	0.00602874413606756\\
193	0.00602852959958735\\
194	0.00602831128377235\\
195	0.00602808911697287\\
196	0.00602786302598349\\
197	0.00602763293600906\\
198	0.0060273987706305\\
199	0.00602716045176988\\
200	0.00602691789965512\\
201	0.00602667103278441\\
202	0.0060264197678899\\
203	0.00602616401990116\\
204	0.00602590370190799\\
205	0.006025638725123\\
206	0.00602536899884355\\
207	0.00602509443041335\\
208	0.00602481492518334\\
209	0.00602453038647252\\
210	0.00602424071552795\\
211	0.00602394581148433\\
212	0.00602364557132305\\
213	0.00602333988983085\\
214	0.00602302865955784\\
215	0.00602271177077484\\
216	0.00602238911143023\\
217	0.00602206056710623\\
218	0.00602172602097441\\
219	0.00602138535375063\\
220	0.00602103844364905\\
221	0.00602068516633558\\
222	0.00602032539488046\\
223	0.00601995899970986\\
224	0.00601958584855677\\
225	0.00601920580641077\\
226	0.00601881873546681\\
227	0.00601842449507293\\
228	0.00601802294167679\\
229	0.00601761392877109\\
230	0.00601719730683757\\
231	0.00601677292328973\\
232	0.00601634062241428\\
233	0.0060159002453106\\
234	0.00601545162982939\\
235	0.00601499461050892\\
236	0.00601452901851008\\
237	0.00601405468154942\\
238	0.00601357142383042\\
239	0.00601307906597272\\
240	0.00601257742493947\\
241	0.00601206631396247\\
242	0.00601154554246535\\
243	0.00601101491598445\\
244	0.00601047423608748\\
245	0.00600992330028988\\
246	0.00600936190196889\\
247	0.00600878983027507\\
248	0.00600820687004161\\
249	0.00600761280169106\\
250	0.00600700740113942\\
251	0.00600639043969812\\
252	0.00600576168397293\\
253	0.00600512089576077\\
254	0.00600446783194364\\
255	0.00600380224438007\\
256	0.00600312387979381\\
257	0.00600243247966014\\
258	0.00600172778008923\\
259	0.00600100951170685\\
260	0.00600027739953232\\
261	0.00599953116285375\\
262	0.00599877051510059\\
263	0.00599799516371429\\
264	0.00599720481001554\\
265	0.00599639914906835\\
266	0.0059955778695411\\
267	0.0059947406535649\\
268	0.00599388717658837\\
269	0.00599301710722961\\
270	0.00599213010712454\\
271	0.00599122583077206\\
272	0.00599030392537543\\
273	0.00598936403067997\\
274	0.00598840577880696\\
275	0.00598742879408339\\
276	0.00598643269286741\\
277	0.00598541708336929\\
278	0.00598438156546767\\
279	0.00598332573052086\\
280	0.0059822491611728\\
281	0.00598115143115341\\
282	0.00598003210507315\\
283	0.00597889073821138\\
284	0.00597772687629799\\
285	0.00597654005528827\\
286	0.00597532980113045\\
287	0.00597409562952539\\
288	0.00597283704567818\\
289	0.00597155354404147\\
290	0.00597024460804948\\
291	0.00596890970984297\\
292	0.00596754830998413\\
293	0.00596615985716149\\
294	0.00596474378788404\\
295	0.00596329952616421\\
296	0.0059618264831893\\
297	0.00596032405698086\\
298	0.00595879163204187\\
299	0.00595722857899214\\
300	0.00595563425419103\\
301	0.00595400799934622\\
302	0.00595234914110841\\
303	0.00595065699065172\\
304	0.00594893084323893\\
305	0.00594716997776979\\
306	0.00594537365631309\\
307	0.0059435411236291\\
308	0.00594167160668233\\
309	0.00593976431413166\\
310	0.00593781843580287\\
311	0.00593583314214417\\
312	0.0059338075836631\\
313	0.00593174089034451\\
314	0.00592963217105018\\
315	0.00592748051289843\\
316	0.00592528498062269\\
317	0.00592304461590789\\
318	0.00592075843670291\\
319	0.00591842543650829\\
320	0.00591604458363759\\
321	0.0059136148204504\\
322	0.00591113506255556\\
323	0.00590860419798202\\
324	0.00590602108631484\\
325	0.0059033845577936\\
326	0.00590069341237401\\
327	0.0058979464187612\\
328	0.00589514231339425\\
329	0.0058922797993854\\
330	0.00588935754541443\\
331	0.0058863741845763\\
332	0.00588332831317656\\
333	0.00588021848945845\\
334	0.00587704323226054\\
335	0.0058738010196226\\
336	0.00587049028732595\\
337	0.00586710942736092\\
338	0.00586365678631363\\
339	0.00586013066366021\\
340	0.00585652930994459\\
341	0.0058528509248092\\
342	0.00584909365496575\\
343	0.0058452555921781\\
344	0.00584133477099838\\
345	0.00583732916634428\\
346	0.00583323669094712\\
347	0.00582905519265595\\
348	0.00582478245157597\\
349	0.00582041617703999\\
350	0.00581595400455628\\
351	0.00581139349257955\\
352	0.00580673211910165\\
353	0.00580196727811239\\
354	0.00579709627591544\\
355	0.00579211632706821\\
356	0.00578702455004549\\
357	0.00578181796285351\\
358	0.00577649347849695\\
359	0.00577104790028927\\
360	0.00576547791702843\\
361	0.00575978009807038\\
362	0.00575395088834157\\
363	0.00574798660334017\\
364	0.00574188342418147\\
365	0.00573563739272935\\
366	0.00572924440677059\\
367	0.00572270021484352\\
368	0.00571600040952584\\
369	0.005709140423154\\
370	0.0057021155240592\\
371	0.00569492080962159\\
372	0.0056875511982807\\
373	0.0056800014216376\\
374	0.00567226601295422\\
375	0.00566433929319607\\
376	0.00565621535729714\\
377	0.00564788805911819\\
378	0.00563935099452979\\
379	0.0056305974823533\\
380	0.00562162054289488\\
381	0.00561241287384588\\
382	0.00560296682340784\\
383	0.00559327436064272\\
384	0.00558332704312636\\
385	0.00557311598119243\\
386	0.00556263179377011\\
387	0.0055518645746237\\
388	0.0055408038787054\\
389	0.00552943871860183\\
390	0.00551775757784875\\
391	0.00550574847396325\\
392	0.00549339911978506\\
393	0.00548069695745734\\
394	0.00546762919783081\\
395	0.00545418286695475\\
396	0.00544034485903358\\
397	0.0054261019943971\\
398	0.00541144107993989\\
399	0.00539634896805736\\
400	0.00538081260476591\\
401	0.00536481898707886\\
402	0.00534835499273021\\
403	0.00533140743911178\\
404	0.00531396296337945\\
405	0.00529600816592463\\
406	0.00527752883261459\\
407	0.00525850891416183\\
408	0.00523892916154191\\
409	0.00521876400267892\\
410	0.00519797662168044\\
411	0.00517652869294469\\
412	0.00515438080539511\\
413	0.00513149313071235\\
414	0.0051078266056657\\
415	0.00508334309874557\\
416	0.00505800565036955\\
417	0.00503178177338743\\
418	0.00500464863693634\\
419	0.00497660150946273\\
420	0.00494765572762768\\
421	0.00491785471270213\\
422	0.00488728058215687\\
423	0.00485606816725119\\
424	0.00482455303040914\\
425	0.0047935031221998\\
426	0.00476300169002057\\
427	0.00473313885047253\\
428	0.00470401144068946\\
429	0.00467572247426099\\
430	0.00464837987732808\\
431	0.00462209462361826\\
432	0.00459697793237357\\
433	0.00457313715956891\\
434	0.00455066990763092\\
435	0.00452965565608938\\
436	0.00451014421183016\\
437	0.00449213987423766\\
438	0.00447557991696671\\
439	0.00445976160220178\\
440	0.00444438329017133\\
441	0.00442945124921817\\
442	0.00441496622615378\\
443	0.00440092238274243\\
444	0.00438730644245399\\
445	0.00437409738269767\\
446	0.00436126490607438\\
447	0.00434876889534301\\
448	0.00433655936263782\\
449	0.0043245771894617\\
450	0.00431275609427381\\
451	0.00430102640784474\\
452	0.00428932972198311\\
453	0.0042776548476847\\
454	0.00426598925603829\\
455	0.00425431918821024\\
456	0.00424262985238187\\
457	0.0042309056985038\\
458	0.00421913077775146\\
459	0.0042072891883445\\
460	0.00419536560038398\\
461	0.00418334583722357\\
462	0.00417121746949017\\
463	0.00415897034334175\\
464	0.0041465969142752\\
465	0.00413409127308022\\
466	0.00412144755919448\\
467	0.00410866002540348\\
468	0.00409572310204968\\
469	0.00408263145707985\\
470	0.0040693800473816\\
471	0.00405596415608319\\
472	0.00404237940998745\\
473	0.00402862177137887\\
474	0.00401468749958905\\
475	0.00400057308067138\\
476	0.00398627512949324\\
477	0.00397179029839382\\
478	0.00395711528042726\\
479	0.00394224681010434\\
480	0.00392718166136174\\
481	0.00391191664262769\\
482	0.00389644858905306\\
483	0.00388077435224436\\
484	0.00386489078816142\\
485	0.00384879474420444\\
486	0.00383248304685515\\
487	0.00381595249145306\\
488	0.0037991998355973\\
489	0.00378222179513981\\
490	0.00376501503962558\\
491	0.00374757618723257\\
492	0.00372990179928235\\
493	0.00371198837440514\\
494	0.00369383234244635\\
495	0.00367543005819135\\
496	0.00365677779495725\\
497	0.00363787173804987\\
498	0.00361870797801236\\
499	0.00359928250350329\\
500	0.00357959119355817\\
501	0.00355962980915254\\
502	0.00353939398401156\\
503	0.00351887921460329\\
504	0.00349808084924341\\
505	0.00347699407622836\\
506	0.00345561391090399\\
507	0.00343393518156551\\
508	0.00341195251407696\\
509	0.00338966031509233\\
510	0.00336705275376157\\
511	0.00334412374181056\\
512	0.00332086691189175\\
513	0.00329727559410764\\
514	0.00327334279062074\\
515	0.00324906114828117\\
516	0.00322442292923013\\
517	0.00319941997947536\\
518	0.00317404369548875\\
519	0.00314828498894659\\
520	0.00312213424983111\\
521	0.00309558130823694\\
522	0.00306861539539113\\
523	0.00304122510460897\\
524	0.00301339835318134\\
525	0.00298512234654203\\
526	0.00295638354651234\\
527	0.00292716764599154\\
528	0.00289745955318733\\
529	0.0028672433893974\\
530	0.0028365025055124\\
531	0.00280521952386659\\
532	0.00277337641388418\\
533	0.00274095461225293\\
534	0.00270793520146081\\
535	0.00267429916488165\\
536	0.0026400277413916\\
537	0.00260510289781278\\
538	0.0025695079186138\\
539	0.00253322814582862\\
540	0.00249625223342463\\
541	0.00245857508153023\\
542	0.00242020649288695\\
543	0.00238123570251701\\
544	0.00234160205815798\\
545	0.00230124504563887\\
546	0.00226002773265821\\
547	0.00221783170899319\\
548	0.00217459767527134\\
549	0.0021302495305581\\
550	0.00208471834938659\\
551	0.00203792665729891\\
552	0.00198977932691693\\
553	0.00194012959396351\\
554	0.00189105467690076\\
555	0.0018437709329238\\
556	0.00179836906095719\\
557	0.00175287820462187\\
558	0.00170717331469996\\
559	0.00166116212177324\\
560	0.00161488078334582\\
561	0.00156841503513133\\
562	0.0015224984313825\\
563	0.00147763480850627\\
564	0.00143310019390703\\
565	0.00138847730609411\\
566	0.00134373361196189\\
567	0.00129889564446826\\
568	0.00125400069337501\\
569	0.00120908421228601\\
570	0.00116444936949479\\
571	0.0011201870485504\\
572	0.00107576112986418\\
573	0.00103114982609361\\
574	0.000986381581866229\\
575	0.000941490291903521\\
576	0.000896512873534784\\
577	0.000851489140303056\\
578	0.000806461811104773\\
579	0.000761476490067704\\
580	0.000716581586462604\\
581	0.000671828153146611\\
582	0.000627269616642477\\
583	0.000582961366995044\\
584	0.000538960168256757\\
585	0.00049532334179279\\
586	0.00045210766564535\\
587	0.000409367927260643\\
588	0.000367155072958964\\
589	0.000325513941232703\\
590	0.000284480716949125\\
591	0.000244080684870838\\
592	0.000204328087251261\\
593	0.000165233217676711\\
594	0.000126870159937007\\
595	8.96111722547522e-05\\
596	5.42660945238511e-05\\
597	2.2806228433206e-05\\
598	2.9204464504877e-07\\
599	0\\
600	0\\
};
\addplot [color=black!20!mycolor21,solid,forget plot]
  table[row sep=crcr]{%
1	0.00604041973083569\\
2	0.00604041163451223\\
3	0.00604040336646474\\
4	0.00604039492318405\\
5	0.0060403863010932\\
6	0.00604037749654635\\
7	0.00604036850582758\\
8	0.00604035932514956\\
9	0.00604034995065246\\
10	0.00604034037840258\\
11	0.00604033060439107\\
12	0.00604032062453276\\
13	0.00604031043466472\\
14	0.00604030003054496\\
15	0.00604028940785117\\
16	0.0060402785621792\\
17	0.00604026748904179\\
18	0.00604025618386714\\
19	0.00604024464199752\\
20	0.0060402328586877\\
21	0.00604022082910366\\
22	0.00604020854832101\\
23	0.0060401960113235\\
24	0.00604018321300155\\
25	0.00604017014815062\\
26	0.00604015681146975\\
27	0.00604014319755995\\
28	0.00604012930092258\\
29	0.00604011511595774\\
30	0.0060401006369627\\
31	0.0060400858581301\\
32	0.00604007077354647\\
33	0.00604005537719037\\
34	0.00604003966293073\\
35	0.00604002362452516\\
36	0.00604000725561808\\
37	0.0060399905497391\\
38	0.00603997350030111\\
39	0.00603995610059853\\
40	0.00603993834380532\\
41	0.00603992022297334\\
42	0.00603990173103038\\
43	0.00603988286077825\\
44	0.00603986360489084\\
45	0.00603984395591222\\
46	0.0060398239062547\\
47	0.00603980344819681\\
48	0.00603978257388132\\
49	0.0060397612753132\\
50	0.00603973954435765\\
51	0.00603971737273794\\
52	0.00603969475203342\\
53	0.00603967167367738\\
54	0.00603964812895493\\
55	0.00603962410900086\\
56	0.00603959960479752\\
57	0.00603957460717259\\
58	0.00603954910679688\\
59	0.00603952309418217\\
60	0.00603949655967891\\
61	0.00603946949347404\\
62	0.0060394418855886\\
63	0.00603941372587553\\
64	0.00603938500401733\\
65	0.00603935570952367\\
66	0.00603932583172915\\
67	0.00603929535979078\\
68	0.00603926428268575\\
69	0.00603923258920885\\
70	0.00603920026797016\\
71	0.00603916730739258\\
72	0.00603913369570929\\
73	0.00603909942096132\\
74	0.00603906447099502\\
75	0.0060390288334596\\
76	0.00603899249580445\\
77	0.00603895544527667\\
78	0.00603891766891847\\
79	0.00603887915356454\\
80	0.00603883988583942\\
81	0.0060387998521549\\
82	0.0060387590387073\\
83	0.0060387174314748\\
84	0.00603867501621471\\
85	0.00603863177846084\\
86	0.00603858770352055\\
87	0.00603854277647222\\
88	0.00603849698216225\\
89	0.00603845030520236\\
90	0.00603840272996667\\
91	0.00603835424058895\\
92	0.00603830482095957\\
93	0.00603825445472271\\
94	0.00603820312527338\\
95	0.00603815081575444\\
96	0.00603809750905357\\
97	0.00603804318780031\\
98	0.00603798783436296\\
99	0.0060379314308455\\
100	0.00603787395908446\\
101	0.00603781540064582\\
102	0.00603775573682165\\
103	0.00603769494862711\\
104	0.00603763301679705\\
105	0.00603756992178262\\
106	0.00603750564374808\\
107	0.00603744016256737\\
108	0.0060373734578205\\
109	0.00603730550879026\\
110	0.00603723629445852\\
111	0.00603716579350268\\
112	0.00603709398429197\\
113	0.00603702084488379\\
114	0.00603694635301978\\
115	0.00603687048612213\\
116	0.00603679322128945\\
117	0.00603671453529285\\
118	0.00603663440457184\\
119	0.00603655280523012\\
120	0.00603646971303127\\
121	0.00603638510339442\\
122	0.00603629895138968\\
123	0.00603621123173367\\
124	0.00603612191878477\\
125	0.00603603098653822\\
126	0.00603593840862124\\
127	0.0060358441582879\\
128	0.0060357482084139\\
129	0.00603565053149122\\
130	0.00603555109962257\\
131	0.00603544988451557\\
132	0.00603534685747704\\
133	0.00603524198940678\\
134	0.00603513525079143\\
135	0.00603502661169791\\
136	0.00603491604176677\\
137	0.00603480351020528\\
138	0.00603468898578028\\
139	0.00603457243681071\\
140	0.00603445383116012\\
141	0.00603433313622849\\
142	0.0060342103189442\\
143	0.00603408534575543\\
144	0.00603395818262127\\
145	0.00603382879500259\\
146	0.00603369714785249\\
147	0.00603356320560641\\
148	0.0060334269321718\\
149	0.0060332882909176\\
150	0.00603314724466295\\
151	0.00603300375566584\\
152	0.00603285778561102\\
153	0.0060327092955976\\
154	0.00603255824612614\\
155	0.00603240459708521\\
156	0.00603224830773736\\
157	0.00603208933670466\\
158	0.00603192764195358\\
159	0.00603176318077928\\
160	0.00603159590978937\\
161	0.0060314257848869\\
162	0.00603125276125283\\
163	0.00603107679332769\\
164	0.00603089783479258\\
165	0.00603071583854953\\
166	0.00603053075670093\\
167	0.00603034254052844\\
168	0.00603015114047072\\
169	0.0060299565061008\\
170	0.00602975858610219\\
171	0.00602955732824444\\
172	0.00602935267935756\\
173	0.00602914458530571\\
174	0.00602893299095985\\
175	0.0060287178401695\\
176	0.00602849907573345\\
177	0.00602827663936958\\
178	0.00602805047168352\\
179	0.00602782051213645\\
180	0.00602758669901166\\
181	0.00602734896938018\\
182	0.00602710725906535\\
183	0.00602686150260615\\
184	0.00602661163321953\\
185	0.00602635758276164\\
186	0.00602609928168786\\
187	0.00602583665901176\\
188	0.00602556964226287\\
189	0.0060252981574434\\
190	0.00602502212898375\\
191	0.00602474147969693\\
192	0.00602445613073175\\
193	0.00602416600152512\\
194	0.00602387100975294\\
195	0.00602357107128012\\
196	0.00602326610010947\\
197	0.00602295600832944\\
198	0.00602264070606097\\
199	0.0060223201014031\\
200	0.00602199410037796\\
201	0.00602166260687438\\
202	0.00602132552259091\\
203	0.00602098274697774\\
204	0.00602063417717792\\
205	0.00602027970796776\\
206	0.00601991923169633\\
207	0.00601955263822441\\
208	0.00601917981486271\\
209	0.00601880064630949\\
210	0.00601841501458756\\
211	0.00601802279898083\\
212	0.00601762387597035\\
213	0.00601721811916998\\
214	0.00601680539926162\\
215	0.00601638558393026\\
216	0.0060159585377985\\
217	0.00601552412236119\\
218	0.00601508219591953\\
219	0.00601463261351521\\
220	0.00601417522686442\\
221	0.00601370988429164\\
222	0.00601323643066339\\
223	0.0060127547073219\\
224	0.00601226455201865\\
225	0.00601176579884772\\
226	0.00601125827817919\\
227	0.00601074181659219\\
228	0.00601021623680778\\
229	0.00600968135762151\\
230	0.00600913699383562\\
231	0.00600858295619075\\
232	0.00600801905129705\\
233	0.00600744508156469\\
234	0.0060068608451333\\
235	0.0060062661358006\\
236	0.00600566074294979\\
237	0.00600504445147547\\
238	0.00600441704170816\\
239	0.00600377828933684\\
240	0.00600312796532969\\
241	0.00600246583585235\\
242	0.00600179166218383\\
243	0.00600110520062961\\
244	0.00600040620243174\\
245	0.00599969441367572\\
246	0.00599896957519369\\
247	0.00599823142246401\\
248	0.00599747968550672\\
249	0.00599671408877461\\
250	0.00599593435104\\
251	0.00599514018527647\\
252	0.00599433129853582\\
253	0.00599350739181975\\
254	0.0059926681599462\\
255	0.00599181329141022\\
256	0.00599094246823893\\
257	0.00599005536584096\\
258	0.00598915165284961\\
259	0.00598823099095999\\
260	0.00598729303475947\\
261	0.00598633743155098\\
262	0.0059853638211689\\
263	0.0059843718357903\\
264	0.0059833610997498\\
265	0.00598233122934154\\
266	0.00598128183261407\\
267	0.0059802125091595\\
268	0.00597912284989707\\
269	0.00597801243685141\\
270	0.00597688084292488\\
271	0.00597572763166461\\
272	0.00597455235702391\\
273	0.0059733545631182\\
274	0.00597213378397527\\
275	0.00597088954328013\\
276	0.00596962135411411\\
277	0.00596832871868851\\
278	0.00596701112807231\\
279	0.00596566806191407\\
280	0.0059642989881579\\
281	0.00596290336275316\\
282	0.00596148062935757\\
283	0.00596003021903368\\
284	0.0059585515499382\\
285	0.00595704402700358\\
286	0.00595550704161166\\
287	0.00595393997125856\\
288	0.00595234217921039\\
289	0.00595071301414884\\
290	0.00594905180980607\\
291	0.00594735788458802\\
292	0.00594563054118526\\
293	0.00594386906617028\\
294	0.00594207272958046\\
295	0.00594024078448527\\
296	0.00593837246653645\\
297	0.00593646699349897\\
298	0.00593452356476108\\
299	0.00593254136082331\\
300	0.00593051954277351\\
301	0.00592845725174483\\
302	0.00592635360834158\\
303	0.00592420771203587\\
304	0.00592201864053301\\
305	0.0059197854490976\\
306	0.00591750716982406\\
307	0.00591518281083395\\
308	0.00591281135547901\\
309	0.00591039176156725\\
310	0.00590792296046258\\
311	0.00590540385611892\\
312	0.00590283332406944\\
313	0.00590021021036055\\
314	0.00589753333042427\\
315	0.00589480146791016\\
316	0.00589201337346598\\
317	0.0058891677634614\\
318	0.00588626331865267\\
319	0.00588329868278457\\
320	0.00588027246112666\\
321	0.0058771832189402\\
322	0.00587402947987144\\
323	0.0058708097242668\\
324	0.00586752238740279\\
325	0.00586416585761891\\
326	0.0058607384743371\\
327	0.00585723852597493\\
328	0.00585366424790664\\
329	0.00585001382025991\\
330	0.00584628536560832\\
331	0.00584247694659282\\
332	0.00583858656350006\\
333	0.00583461215180657\\
334	0.00583055157955604\\
335	0.00582640264457449\\
336	0.00582216307181674\\
337	0.0058178305107879\\
338	0.00581340253306314\\
339	0.0058088766299555\\
340	0.00580425021035672\\
341	0.00579952059865069\\
342	0.00579468503224592\\
343	0.00578974065955761\\
344	0.00578468453962854\\
345	0.00577951363976895\\
346	0.00577422483167486\\
347	0.00576881488723711\\
348	0.00576328047393433\\
349	0.00575761814956387\\
350	0.00575182435598938\\
351	0.00574589541383171\\
352	0.00573982751635916\\
353	0.00573361672235197\\
354	0.00572725894847346\\
355	0.00572074996134833\\
356	0.00571408536631671\\
357	0.00570726059326061\\
358	0.00570027088200122\\
359	0.00569311126585978\\
360	0.00568577655283809\\
361	0.00567826130420807\\
362	0.00567055981040699\\
363	0.00566266606433347\\
364	0.00565457373248421\\
365	0.00564627612493743\\
366	0.00563776616607646\\
367	0.00562903636909487\\
368	0.00562007881702173\\
369	0.00561088513657466\\
370	0.00560144652750161\\
371	0.00559175389375945\\
372	0.00558179789037141\\
373	0.00557156895084197\\
374	0.00556105733721312\\
375	0.00555025316917609\\
376	0.00553914644108116\\
377	0.00552772706280232\\
378	0.00551598490973305\\
379	0.00550390987707951\\
380	0.00549149193592538\\
381	0.00547872118685052\\
382	0.00546558790448464\\
383	0.0054520825630505\\
384	0.00543819582848082\\
385	0.00542391849660757\\
386	0.00540924134281841\\
387	0.00539415472574238\\
388	0.00537864806235072\\
389	0.00536270919272062\\
390	0.00534632333267171\\
391	0.00532947120568322\\
392	0.00531212333194729\\
393	0.00529424825184008\\
394	0.00527581284121551\\
395	0.00525678251559689\\
396	0.00523712156378698\\
397	0.00521679365642495\\
398	0.00519576258982919\\
399	0.00517399334701273\\
400	0.00515145359893942\\
401	0.00512811592022666\\
402	0.00510396008989104\\
403	0.00507897457342055\\
404	0.00505316281341298\\
405	0.00502654890918629\\
406	0.00499919242962357\\
407	0.00497119736069615\\
408	0.00494277929217888\\
409	0.0049146626275627\\
410	0.00488691175444974\\
411	0.00485959695242314\\
412	0.00483279448306534\\
413	0.00480658649718681\\
414	0.00478106067117138\\
415	0.00475630951499183\\
416	0.00473242920785781\\
417	0.0047095176753174\\
418	0.00468767160117613\\
419	0.00466698195385279\\
420	0.00464752785040454\\
421	0.00462936801802073\\
422	0.00461252901153922\\
423	0.00459698912504476\\
424	0.00458252040870373\\
425	0.00456843631081279\\
426	0.00455474838043038\\
427	0.00454146452366907\\
428	0.00452858822431064\\
429	0.00451611773544137\\
430	0.00450404487672025\\
431	0.00449235441662874\\
432	0.00448102327740238\\
433	0.00447001987603998\\
434	0.00445930405550695\\
435	0.0044488278667083\\
436	0.00443853618507812\\
437	0.0044283692003319\\
438	0.00441826682854435\\
439	0.00440820415116507\\
440	0.00439817136788632\\
441	0.00438815749405422\\
442	0.0043781504687486\\
443	0.00436813731957653\\
444	0.00435810438547989\\
445	0.00434803759671732\\
446	0.00433792285124789\\
447	0.00432774646157792\\
448	0.00431749565613897\\
449	0.0043071591014906\\
450	0.00429672738417001\\
451	0.00428619335106061\\
452	0.00427555179762931\\
453	0.00426479752868549\\
454	0.00425392541315694\\
455	0.00424293044073412\\
456	0.00423180777714498\\
457	0.00422055281468181\\
458	0.00420916121386276\\
459	0.00419762893149913\\
460	0.00418595223013638\\
461	0.00417412766416368\\
462	0.00416215203923854\\
463	0.0041500223447995\\
464	0.00413773566535698\\
465	0.00412528912332364\\
466	0.00411267988230371\\
467	0.00409990514827473\\
468	0.00408696216842916\\
469	0.00407384822755778\\
470	0.00406056064202319\\
471	0.00404709675159173\\
472	0.00403345390966303\\
473	0.00401962947274033\\
474	0.00400562079028181\\
475	0.00399142519628373\\
476	0.00397704000392844\\
477	0.00396246250337754\\
478	0.0039476899592021\\
479	0.00393271960750239\\
480	0.00391754865278553\\
481	0.00390217426468383\\
482	0.00388659357460285\\
483	0.00387080367238742\\
484	0.00385480160307617\\
485	0.00383858436378452\\
486	0.00382214890070395\\
487	0.00380549210614036\\
488	0.00378861081544281\\
489	0.00377150180369735\\
490	0.00375416178217399\\
491	0.00373658739451004\\
492	0.00371877521260873\\
493	0.00370072173222481\\
494	0.00368242336820189\\
495	0.00366387644931887\\
496	0.00364507721269443\\
497	0.0036260217976926\\
498	0.00360670623926831\\
499	0.0035871264606905\\
500	0.00356727826558349\\
501	0.00354715732922336\\
502	0.00352675918902258\\
503	0.00350607923413049\\
504	0.00348511269407485\\
505	0.00346385462636618\\
506	0.00344229990298504\\
507	0.00342044319567329\\
508	0.00339827895995153\\
509	0.00337580141779204\\
510	0.00335300453888381\\
511	0.00332988202044153\\
512	0.00330642726552966\\
513	0.00328263335990133\\
514	0.00325849304739117\\
515	0.00323399870395426\\
516	0.0032091423105132\\
517	0.00318391542486947\\
518	0.00315830915305462\\
519	0.00313231412065579\\
520	0.00310592044485016\\
521	0.00307911770814284\\
522	0.00305189493513307\\
523	0.00302424057405301\\
524	0.00299614248535714\\
525	0.002967587940314\\
526	0.0029385636334032\\
527	0.00290905571339415\\
528	0.00287904983933047\\
529	0.00284853126932706\\
530	0.0028174849921919\\
531	0.00278589591474815\\
532	0.00275374912178569\\
533	0.00272103023008121\\
534	0.00268772585133614\\
535	0.00265382415755588\\
536	0.00261931558188282\\
537	0.00258419417888923\\
538	0.00254846100558979\\
539	0.00251216052959282\\
540	0.00247532628282182\\
541	0.00243788997122817\\
542	0.00239976415995061\\
543	0.00236075191454028\\
544	0.00232080359268986\\
545	0.00227985780608986\\
546	0.00223785281570349\\
547	0.00219472782639527\\
548	0.00215041177863719\\
549	0.00210482397663955\\
550	0.00205785625152206\\
551	0.00200963534708887\\
552	0.00196294539021133\\
553	0.00191846763346604\\
554	0.00187458162861202\\
555	0.00183052002201858\\
556	0.00178605773398743\\
557	0.00174121417877548\\
558	0.00169604534204883\\
559	0.00165063743477819\\
560	0.00160552143434389\\
561	0.00156143381812764\\
562	0.00151771985037248\\
563	0.00147385304555012\\
564	0.00142980369459937\\
565	0.00138559882693199\\
566	0.00134127297237882\\
567	0.00129686098742586\\
568	0.00125239367639895\\
569	0.00120835592750733\\
570	0.00116438756191665\\
571	0.00112018262134879\\
572	0.00107576048997661\\
573	0.0010311496404989\\
574	0.00098638150776886\\
575	0.000941490258173559\\
576	0.000896512856511153\\
577	0.000851489130996976\\
578	0.000806461805889532\\
579	0.000761476487129886\\
580	0.000716581584829257\\
581	0.000671828152282903\\
582	0.00062726961623216\\
583	0.000582961366832659\\
584	0.000538960168209238\\
585	0.000495323341784078\\
586	0.000452107665645342\\
587	0.000409367927260636\\
588	0.000367155072958956\\
589	0.0003255139412327\\
590	0.000284480716949123\\
591	0.000244080684870839\\
592	0.000204328087251262\\
593	0.000165233217676712\\
594	0.000126870159937008\\
595	8.96111722547525e-05\\
596	5.4266094523851e-05\\
597	2.28062284332056e-05\\
598	2.9204464504877e-07\\
599	0\\
600	0\\
};
\addplot [color=black!50!mycolor20,solid,forget plot]
  table[row sep=crcr]{%
1	0.00604035455130753\\
2	0.00604034491250777\\
3	0.00604033506129769\\
4	0.00604032499311465\\
5	0.00604031470330184\\
6	0.00604030418710641\\
7	0.00604029343967774\\
8	0.00604028245606537\\
9	0.00604027123121725\\
10	0.00604025975997777\\
11	0.00604024803708565\\
12	0.00604023605717213\\
13	0.00604022381475875\\
14	0.00604021130425541\\
15	0.00604019851995815\\
16	0.00604018545604706\\
17	0.00604017210658414\\
18	0.006040158465511\\
19	0.00604014452664662\\
20	0.00604013028368517\\
21	0.00604011573019356\\
22	0.00604010085960915\\
23	0.00604008566523733\\
24	0.00604007014024908\\
25	0.00604005427767858\\
26	0.00604003807042058\\
27	0.00604002151122794\\
28	0.00604000459270902\\
29	0.00603998730732511\\
30	0.00603996964738761\\
31	0.00603995160505559\\
32	0.00603993317233276\\
33	0.00603991434106494\\
34	0.00603989510293712\\
35	0.00603987544947057\\
36	0.00603985537202005\\
37	0.0060398348617708\\
38	0.00603981390973551\\
39	0.00603979250675143\\
40	0.00603977064347726\\
41	0.00603974831039\\
42	0.00603972549778182\\
43	0.0060397021957569\\
44	0.0060396783942282\\
45	0.00603965408291424\\
46	0.00603962925133564\\
47	0.00603960388881196\\
48	0.00603957798445819\\
49	0.00603955152718133\\
50	0.00603952450567691\\
51	0.00603949690842561\\
52	0.00603946872368945\\
53	0.00603943993950841\\
54	0.0060394105436967\\
55	0.00603938052383908\\
56	0.00603934986728715\\
57	0.00603931856115561\\
58	0.00603928659231845\\
59	0.00603925394740509\\
60	0.00603922061279656\\
61	0.00603918657462144\\
62	0.00603915181875213\\
63	0.00603911633080062\\
64	0.00603908009611462\\
65	0.00603904309977344\\
66	0.00603900532658388\\
67	0.00603896676107607\\
68	0.00603892738749933\\
69	0.00603888718981799\\
70	0.00603884615170701\\
71	0.00603880425654783\\
72	0.00603876148742409\\
73	0.00603871782711716\\
74	0.00603867325810185\\
75	0.00603862776254197\\
76	0.00603858132228596\\
77	0.00603853391886239\\
78	0.00603848553347546\\
79	0.00603843614700055\\
80	0.00603838573997964\\
81	0.00603833429261675\\
82	0.00603828178477338\\
83	0.00603822819596396\\
84	0.00603817350535117\\
85	0.00603811769174128\\
86	0.00603806073357962\\
87	0.00603800260894578\\
88	0.00603794329554906\\
89	0.00603788277072365\\
90	0.00603782101142409\\
91	0.00603775799422036\\
92	0.00603769369529337\\
93	0.00603762809043015\\
94	0.0060375611550191\\
95	0.00603749286404532\\
96	0.00603742319208591\\
97	0.00603735211330515\\
98	0.0060372796014499\\
99	0.00603720562984483\\
100	0.00603713017138775\\
101	0.00603705319854488\\
102	0.00603697468334623\\
103	0.00603689459738088\\
104	0.00603681291179239\\
105	0.00603672959727415\\
106	0.00603664462406469\\
107	0.0060365579619432\\
108	0.00603646958022498\\
109	0.00603637944775674\\
110	0.00603628753291225\\
111	0.0060361938035878\\
112	0.0060360982271977\\
113	0.0060360007706699\\
114	0.00603590140044156\\
115	0.00603580008245474\\
116	0.00603569678215202\\
117	0.00603559146447224\\
118	0.00603548409384623\\
119	0.00603537463419263\\
120	0.00603526304891366\\
121	0.00603514930089098\\
122	0.00603503335248161\\
123	0.0060349151655138\\
124	0.00603479470128311\\
125	0.00603467192054829\\
126	0.00603454678352742\\
127	0.00603441924989387\\
128	0.00603428927877249\\
129	0.00603415682873569\\
130	0.00603402185779955\\
131	0.0060338843234201\\
132	0.00603374418248944\\
133	0.00603360139133191\\
134	0.00603345590570039\\
135	0.00603330768077241\\
136	0.00603315667114645\\
137	0.00603300283083806\\
138	0.00603284611327618\\
139	0.00603268647129906\\
140	0.00603252385715058\\
141	0.00603235822247626\\
142	0.00603218951831917\\
143	0.00603201769511605\\
144	0.00603184270269301\\
145	0.0060316644902614\\
146	0.00603148300641334\\
147	0.00603129819911738\\
148	0.00603111001571373\\
149	0.00603091840290947\\
150	0.00603072330677369\\
151	0.00603052467273213\\
152	0.00603032244556186\\
153	0.00603011656938548\\
154	0.00602990698766532\\
155	0.00602969364319684\\
156	0.00602947647810232\\
157	0.0060292554338236\\
158	0.00602903045111478\\
159	0.00602880147003427\\
160	0.0060285684299365\\
161	0.00602833126946292\\
162	0.00602808992653266\\
163	0.00602784433833239\\
164	0.00602759444130564\\
165	0.00602734017114131\\
166	0.00602708146276157\\
167	0.00602681825030869\\
168	0.00602655046713143\\
169	0.00602627804576997\\
170	0.00602600091794035\\
171	0.00602571901451754\\
172	0.00602543226551752\\
173	0.00602514060007818\\
174	0.006024843946439\\
175	0.00602454223191927\\
176	0.00602423538289514\\
177	0.00602392332477499\\
178	0.00602360598197339\\
179	0.00602328327788333\\
180	0.00602295513484693\\
181	0.00602262147412411\\
182	0.00602228221585955\\
183	0.00602193727904763\\
184	0.00602158658149546\\
185	0.00602123003978353\\
186	0.00602086756922437\\
187	0.00602049908381879\\
188	0.00602012449620984\\
189	0.006019743717634\\
190	0.00601935665787035\\
191	0.00601896322518649\\
192	0.00601856332628224\\
193	0.00601815686623016\\
194	0.00601774374841347\\
195	0.00601732387446088\\
196	0.00601689714417841\\
197	0.00601646345547823\\
198	0.00601602270430429\\
199	0.00601557478455476\\
200	0.00601511958800134\\
201	0.00601465700420523\\
202	0.00601418692042981\\
203	0.00601370922155019\\
204	0.00601322378995926\\
205	0.00601273050547047\\
206	0.0060122292452175\\
207	0.00601171988355051\\
208	0.00601120229192933\\
209	0.0060106763388133\\
210	0.00601014188954833\\
211	0.00600959880625064\\
212	0.00600904694768804\\
213	0.00600848616915798\\
214	0.00600791632236334\\
215	0.00600733725528554\\
216	0.00600674881205547\\
217	0.00600615083282223\\
218	0.00600554315362001\\
219	0.00600492560623318\\
220	0.00600429801805994\\
221	0.00600366021197484\\
222	0.00600301200619012\\
223	0.00600235321411653\\
224	0.00600168364422354\\
225	0.0060010030998996\\
226	0.00600031137931222\\
227	0.0059996082752687\\
228	0.0059988935750774\\
229	0.00599816706040992\\
230	0.0059974285071646\\
231	0.0059966776853314\\
232	0.00599591435885837\\
233	0.00599513828552009\\
234	0.00599434921678806\\
235	0.00599354689770343\\
236	0.00599273106675183\\
237	0.00599190145574067\\
238	0.00599105778967874\\
239	0.00599019978665826\\
240	0.00598932715773907\\
241	0.00598843960683508\\
242	0.00598753683060262\\
243	0.00598661851833054\\
244	0.00598568435183174\\
245	0.00598473400533584\\
246	0.00598376714538252\\
247	0.00598278343071531\\
248	0.00598178251217499\\
249	0.00598076403259272\\
250	0.00597972762668163\\
251	0.00597867292092701\\
252	0.00597759953347416\\
253	0.00597650707401343\\
254	0.00597539514366176\\
255	0.00597426333484019\\
256	0.00597311123114652\\
257	0.005971938407222\\
258	0.00597074442861153\\
259	0.00596952885161539\\
260	0.00596829122313056\\
261	0.00596703108047722\\
262	0.00596574795120211\\
263	0.00596444135284604\\
264	0.00596311079268821\\
265	0.00596175576758368\\
266	0.00596037576370285\\
267	0.00595897025622848\\
268	0.00595753870903681\\
269	0.00595608057436188\\
270	0.00595459529244277\\
271	0.00595308229115259\\
272	0.00595154098560904\\
273	0.00594997077776536\\
274	0.00594837105598171\\
275	0.00594674119457572\\
276	0.00594508055335192\\
277	0.00594338847710932\\
278	0.00594166429512658\\
279	0.00593990732062399\\
280	0.00593811685020162\\
281	0.00593629216325293\\
282	0.00593443252135306\\
283	0.00593253716762066\\
284	0.00593060532605261\\
285	0.00592863620083044\\
286	0.00592662897559697\\
287	0.00592458281270248\\
288	0.00592249685241824\\
289	0.00592037021211663\\
290	0.00591820198541585\\
291	0.00591599124128818\\
292	0.00591373702313046\\
293	0.00591143834779619\\
294	0.00590909420458879\\
295	0.00590670355421709\\
296	0.00590426532771432\\
297	0.00590177842532272\\
298	0.00589924171534324\\
299	0.00589665403294819\\
300	0.00589401417896816\\
301	0.00589132091877082\\
302	0.00588857298127079\\
303	0.00588576905794149\\
304	0.00588290780192623\\
305	0.0058799878273027\\
306	0.00587700770851719\\
307	0.00587396597990175\\
308	0.00587086113495447\\
309	0.00586769162621109\\
310	0.00586445586609552\\
311	0.00586115222654217\\
312	0.00585777903781928\\
313	0.0058543345873161\\
314	0.00585081711820609\\
315	0.00584722482787601\\
316	0.00584355586639754\\
317	0.00583980833493617\\
318	0.00583598028405877\\
319	0.00583206971192252\\
320	0.00582807456232503\\
321	0.0058239927225891\\
322	0.00581982202124891\\
323	0.00581556022549524\\
324	0.00581120503832473\\
325	0.0058067540953129\\
326	0.00580220496087143\\
327	0.00579755512370845\\
328	0.00579280199116947\\
329	0.00578794288452105\\
330	0.00578297503252372\\
331	0.00577789556368191\\
332	0.00577270149724791\\
333	0.00576738973298516\\
334	0.00576195703974413\\
335	0.00575640004094692\\
336	0.00575071519598767\\
337	0.00574489878061114\\
338	0.0057389468653358\\
339	0.00573285529206363\\
340	0.0057266196498375\\
341	0.00572023525148013\\
342	0.00571369711339678\\
343	0.00570699993620161\\
344	0.00570013810097834\\
345	0.00569310572673465\\
346	0.00568589673648015\\
347	0.0056785048609832\\
348	0.00567092364497663\\
349	0.00566314645599178\\
350	0.00565516649457982\\
351	0.00564697679917844\\
352	0.0056385702728284\\
353	0.00562993971370758\\
354	0.00562107784745955\\
355	0.00561197736864539\\
356	0.00560263100200876\\
357	0.00559303154919701\\
358	0.00558317192088877\\
359	0.00557304518645487\\
360	0.00556264462551128\\
361	0.00555196377299146\\
362	0.00554099644978443\\
363	0.00552973676717171\\
364	0.0055181790880087\\
365	0.00550631792034881\\
366	0.00549414770945814\\
367	0.00548166248157145\\
368	0.00546885527818469\\
369	0.00545571730669379\\
370	0.00544223620615092\\
371	0.00542839179521135\\
372	0.00541416081700759\\
373	0.00539951832034958\\
374	0.00538443766169766\\
375	0.00536889081424874\\
376	0.00535284846039011\\
377	0.00533627990800812\\
378	0.00531915330324797\\
379	0.00530143601499595\\
380	0.00528309520580291\\
381	0.005264098651944\\
382	0.00524441589473051\\
383	0.00522401983064083\\
384	0.00520288888166286\\
385	0.00518100993417347\\
386	0.00515838231517751\\
387	0.00513502332663846\\
388	0.0051109739445012\\
389	0.00508630711750994\\
390	0.00506114078397471\\
391	0.00503610410007305\\
392	0.00501125655682597\\
393	0.00498664792446473\\
394	0.00496233283192958\\
395	0.00493837095736671\\
396	0.00491482711891741\\
397	0.00489177121564937\\
398	0.00486927795068279\\
399	0.00484742624496875\\
400	0.00482629821873276\\
401	0.0048059775741019\\
402	0.00478654718902244\\
403	0.00476808569265558\\
404	0.00475066251264518\\
405	0.00473433095851641\\
406	0.00471911843610295\\
407	0.00470501322286141\\
408	0.00469189080838035\\
409	0.00467909168690972\\
410	0.00466662958433403\\
411	0.0046545159743818\\
412	0.00464275949432454\\
413	0.00463136529122577\\
414	0.00462033432580973\\
415	0.00460966263109742\\
416	0.00459934058267878\\
417	0.00458935216047305\\
418	0.00457967434280781\\
419	0.00457027675662039\\
420	0.00456112164386289\\
421	0.00455216435069586\\
422	0.00454335500797968\\
423	0.00453464140084274\\
424	0.00452598108331289\\
425	0.00451736707410352\\
426	0.00450879128376258\\
427	0.0045002445499172\\
428	0.00449171671181201\\
429	0.00448319673197135\\
430	0.00447467287969468\\
431	0.0044661329667574\\
432	0.00445756464614725\\
433	0.00444895577249972\\
434	0.00444029480843122\\
435	0.004431571252739\\
436	0.00442277607768038\\
437	0.00441390208889142\\
438	0.00440494409640422\\
439	0.00439589753527377\\
440	0.00438675784669873\\
441	0.00437752052518183\\
442	0.0043681811668333\\
443	0.00435873551675346\\
444	0.00434917951305033\\
445	0.0043395093245137\\
446	0.00432972137728854\\
447	0.00431981236625983\\
448	0.00430977924699373\\
449	0.00429961920502578\\
450	0.00428932960160009\\
451	0.00427890789942998\\
452	0.00426835159385037\\
453	0.00425765821680932\\
454	0.00424682533924926\\
455	0.00423585057161851\\
456	0.00422473156234556\\
457	0.00421346599421916\\
458	0.00420205157877209\\
459	0.00419048604896569\\
460	0.00417876715071148\\
461	0.00416689263402309\\
462	0.00415486024482622\\
463	0.00414266771859025\\
464	0.00413031277684553\\
465	0.00411779312564403\\
466	0.00410510645374804\\
467	0.0040922504305949\\
468	0.00407922270410051\\
469	0.00406602089837641\\
470	0.00405264261144239\\
471	0.00403908541301646\\
472	0.00402534684245244\\
473	0.00401142440687092\\
474	0.00399731557948974\\
475	0.00398301779810677\\
476	0.0039685284636279\\
477	0.00395384493851744\\
478	0.0039389645451738\\
479	0.00392388456423045\\
480	0.00390860223278089\\
481	0.00389311474252117\\
482	0.00387741923780042\\
483	0.00386151281356359\\
484	0.00384539251316722\\
485	0.00382905532604283\\
486	0.00381249818518186\\
487	0.00379571796441353\\
488	0.00377871147545074\\
489	0.00376147546468013\\
490	0.00374400660966948\\
491	0.00372630151536363\\
492	0.00370835670993593\\
493	0.0036901686402604\\
494	0.00367173366696607\\
495	0.00365304805903214\\
496	0.00363410798788008\\
497	0.00361490952091579\\
498	0.00359544861447277\\
499	0.00357572110610464\\
500	0.00355572270617314\\
501	0.0035354489886765\\
502	0.00351489538126131\\
503	0.00349405715436267\\
504	0.00347292940941807\\
505	0.00345150706610567\\
506	0.00342978484856397\\
507	0.00340775727056105\\
508	0.00338541861959726\\
509	0.00336276293994635\\
510	0.00333978401467161\\
511	0.00331647534669245\\
512	0.00329283013903121\\
513	0.00326884127444044\\
514	0.0032445012947022\\
515	0.00321980238000906\\
516	0.00319473632898957\\
517	0.00316929454013484\\
518	0.00314346799563162\\
519	0.00311724724892223\\
520	0.00309062241771182\\
521	0.00306358318464741\\
522	0.00303611880853064\\
523	0.00300821814972765\\
524	0.0029798697144472\\
525	0.00295106172382384\\
526	0.00292178221532365\\
527	0.00289201918595378\\
528	0.00286176078938083\\
529	0.00283099560279367\\
530	0.00279971298377082\\
531	0.00276790353115984\\
532	0.00273555964225991\\
533	0.00270267619243534\\
534	0.00266925192065413\\
535	0.00263529296920442\\
536	0.00260086606279785\\
537	0.00256597321413507\\
538	0.00253054433721722\\
539	0.00249444974095169\\
540	0.00245754414628867\\
541	0.00241976540072678\\
542	0.00238103965629058\\
543	0.0023413125141704\\
544	0.00230052829688574\\
545	0.00225862371766199\\
546	0.00221552624303288\\
547	0.00217115084894059\\
548	0.00212537826171806\\
549	0.00207910769893968\\
550	0.00203457920162987\\
551	0.00199229323501405\\
552	0.00194984963616151\\
553	0.00190703680818334\\
554	0.00186376888205866\\
555	0.00182006478878576\\
556	0.00177598546288734\\
557	0.00173161014088666\\
558	0.00168713765704981\\
559	0.00164367024345205\\
560	0.00160078372768793\\
561	0.00155771264955085\\
562	0.00151441038560274\\
563	0.00147090213189777\\
564	0.00142722036978999\\
565	0.00138339816260417\\
566	0.00133946689808092\\
567	0.00129553652878565\\
568	0.00125204062884596\\
569	0.00120834736434655\\
570	0.00116438695026368\\
571	0.00112018252877128\\
572	0.00107576046202357\\
573	0.00103114962906728\\
574	0.000986381502459647\\
575	0.000941490255454449\\
576	0.000896512855016981\\
577	0.000851489130158736\\
578	0.00080646180541855\\
579	0.000761476486870332\\
580	0.000716581584693804\\
581	0.000671828152219676\\
582	0.000627269616207832\\
583	0.000582961366825552\\
584	0.000538960168207982\\
585	0.000495323341784074\\
586	0.000452107665645344\\
587	0.000409367927260635\\
588	0.000367155072958959\\
589	0.0003255139412327\\
590	0.00028448071694912\\
591	0.000244080684870837\\
592	0.000204328087251259\\
593	0.00016523321767671\\
594	0.000126870159937007\\
595	8.96111722547517e-05\\
596	5.42660945238506e-05\\
597	2.28062284332055e-05\\
598	2.9204464504877e-07\\
599	0\\
600	0\\
};
\addplot [color=black!60!mycolor21,solid,forget plot]
  table[row sep=crcr]{%
1	0.00604030315615854\\
2	0.00604029213280765\\
3	0.00604028085850546\\
4	0.00604026932762131\\
5	0.00604025753440102\\
6	0.00604024547296424\\
7	0.00604023313730183\\
8	0.0060402205212733\\
9	0.00604020761860375\\
10	0.00604019442288126\\
11	0.00604018092755395\\
12	0.00604016712592705\\
13	0.00604015301115993\\
14	0.00604013857626299\\
15	0.00604012381409466\\
16	0.00604010871735813\\
17	0.0060400932785982\\
18	0.00604007749019795\\
19	0.00604006134437541\\
20	0.00604004483318011\\
21	0.00604002794848966\\
22	0.00604001068200617\\
23	0.00603999302525264\\
24	0.00603997496956928\\
25	0.00603995650610979\\
26	0.00603993762583747\\
27	0.00603991831952148\\
28	0.00603989857773269\\
29	0.00603987839083977\\
30	0.00603985774900511\\
31	0.00603983664218049\\
32	0.00603981506010304\\
33	0.00603979299229069\\
34	0.00603977042803793\\
35	0.00603974735641119\\
36	0.00603972376624438\\
37	0.00603969964613416\\
38	0.00603967498443524\\
39	0.0060396497692556\\
40	0.00603962398845151\\
41	0.00603959762962259\\
42	0.00603957068010675\\
43	0.00603954312697506\\
44	0.00603951495702644\\
45	0.00603948615678234\\
46	0.0060394567124814\\
47	0.0060394266100738\\
48	0.00603939583521576\\
49	0.00603936437326383\\
50	0.0060393322092691\\
51	0.00603929932797119\\
52	0.0060392657137925\\
53	0.00603923135083194\\
54	0.0060391962228589\\
55	0.00603916031330686\\
56	0.00603912360526713\\
57	0.0060390860814823\\
58	0.00603904772433975\\
59	0.00603900851586495\\
60	0.00603896843771466\\
61	0.00603892747117013\\
62	0.00603888559713016\\
63	0.00603884279610392\\
64	0.00603879904820394\\
65	0.00603875433313866\\
66	0.00603870863020521\\
67	0.00603866191828188\\
68	0.00603861417582054\\
69	0.00603856538083893\\
70	0.00603851551091298\\
71	0.00603846454316885\\
72	0.00603841245427489\\
73	0.0060383592204337\\
74	0.00603830481737387\\
75	0.00603824922034165\\
76	0.00603819240409257\\
77	0.00603813434288305\\
78	0.00603807501046174\\
79	0.00603801438006077\\
80	0.00603795242438711\\
81	0.00603788911561362\\
82	0.00603782442537004\\
83	0.00603775832473398\\
84	0.00603769078422181\\
85	0.0060376217737793\\
86	0.00603755126277238\\
87	0.00603747921997769\\
88	0.00603740561357308\\
89	0.00603733041112803\\
90	0.00603725357959399\\
91	0.00603717508529463\\
92	0.00603709489391608\\
93	0.00603701297049695\\
94	0.00603692927941852\\
95	0.00603684378439462\\
96	0.00603675644846161\\
97	0.00603666723396823\\
98	0.00603657610256547\\
99	0.00603648301519629\\
100	0.00603638793208539\\
101	0.00603629081272888\\
102	0.00603619161588401\\
103	0.00603609029955871\\
104	0.00603598682100124\\
105	0.00603588113668981\\
106	0.00603577320232221\\
107	0.00603566297280529\\
108	0.0060355504022446\\
109	0.00603543544393402\\
110	0.00603531805034534\\
111	0.00603519817311788\\
112	0.00603507576304822\\
113	0.00603495077007992\\
114	0.00603482314329318\\
115	0.00603469283089476\\
116	0.00603455978020791\\
117	0.00603442393766219\\
118	0.00603428524878369\\
119	0.00603414365818501\\
120	0.00603399910955568\\
121	0.0060338515456524\\
122	0.00603370090828966\\
123	0.00603354713833032\\
124	0.00603339017567638\\
125	0.00603322995925995\\
126	0.00603306642703442\\
127	0.00603289951596574\\
128	0.00603272916202395\\
129	0.00603255530017487\\
130	0.0060323778643721\\
131	0.00603219678754922\\
132	0.00603201200161214\\
133	0.00603182343743191\\
134	0.0060316310248376\\
135	0.0060314346926097\\
136	0.00603123436847357\\
137	0.0060310299790934\\
138	0.00603082145006639\\
139	0.0060306087059174\\
140	0.00603039167009382\\
141	0.00603017026496083\\
142	0.00602994441179719\\
143	0.00602971403079119\\
144	0.00602947904103725\\
145	0.00602923936053267\\
146	0.00602899490617513\\
147	0.00602874559376031\\
148	0.00602849133798023\\
149	0.0060282320524218\\
150	0.00602796764956598\\
151	0.00602769804078748\\
152	0.00602742313635467\\
153	0.00602714284543029\\
154	0.00602685707607232\\
155	0.00602656573523567\\
156	0.00602626872877395\\
157	0.00602596596144213\\
158	0.00602565733689928\\
159	0.00602534275771209\\
160	0.00602502212535866\\
161	0.00602469534023269\\
162	0.00602436230164816\\
163	0.00602402290784449\\
164	0.00602367705599188\\
165	0.00602332464219712\\
166	0.00602296556150962\\
167	0.00602259970792783\\
168	0.00602222697440578\\
169	0.00602184725285984\\
170	0.00602146043417563\\
171	0.00602106640821494\\
172	0.00602066506382275\\
173	0.00602025628883411\\
174	0.00601983997008088\\
175	0.00601941599339837\\
176	0.00601898424363137\\
177	0.00601854460464016\\
178	0.0060180969593057\\
179	0.00601764118953426\\
180	0.00601717717626128\\
181	0.00601670479945442\\
182	0.00601622393811539\\
183	0.00601573447028078\\
184	0.00601523627302126\\
185	0.0060147292224395\\
186	0.00601421319366627\\
187	0.00601368806085453\\
188	0.00601315369717146\\
189	0.00601260997478819\\
190	0.00601205676486678\\
191	0.00601149393754458\\
192	0.00601092136191525\\
193	0.00601033890600672\\
194	0.00600974643675521\\
195	0.00600914381997563\\
196	0.00600853092032748\\
197	0.00600790760127641\\
198	0.00600727372505088\\
199	0.00600662915259358\\
200	0.00600597374350736\\
201	0.00600530735599531\\
202	0.00600462984679469\\
203	0.00600394107110405\\
204	0.00600324088250367\\
205	0.00600252913286865\\
206	0.00600180567227416\\
207	0.00600107034889287\\
208	0.00600032300888391\\
209	0.00599956349627308\\
210	0.00599879165282408\\
211	0.00599800731790024\\
212	0.00599721032831651\\
213	0.00599640051818154\\
214	0.00599557771872936\\
215	0.00599474175814054\\
216	0.00599389246135252\\
217	0.00599302964985902\\
218	0.00599215314149819\\
219	0.00599126275022954\\
220	0.00599035828589953\\
221	0.00598943955399564\\
222	0.00598850635538901\\
223	0.00598755848606573\\
224	0.00598659573684676\\
225	0.00598561789309678\\
226	0.00598462473442191\\
227	0.00598361603435678\\
228	0.0059825915600411\\
229	0.00598155107188605\\
230	0.00598049432323093\\
231	0.00597942105999027\\
232	0.00597833102029219\\
233	0.00597722393410799\\
234	0.00597609952287374\\
235	0.00597495749910423\\
236	0.00597379756599957\\
237	0.005972619417045\\
238	0.00597142273560416\\
239	0.00597020719450606\\
240	0.00596897245562604\\
241	0.00596771816946078\\
242	0.00596644397469735\\
243	0.00596514949777626\\
244	0.00596383435244846\\
245	0.00596249813932576\\
246	0.00596114044542479\\
247	0.00595976084370408\\
248	0.00595835889259425\\
249	0.00595693413552139\\
250	0.00595548610042442\\
251	0.00595401429926767\\
252	0.00595251822755056\\
253	0.00595099736381837\\
254	0.00594945116917886\\
255	0.00594787908683215\\
256	0.00594628054162402\\
257	0.00594465493963533\\
258	0.00594300166782471\\
259	0.00594132009374463\\
260	0.00593960956535445\\
261	0.00593786941095441\\
262	0.005936098939254\\
263	0.00593429743954531\\
264	0.00593246418184334\\
265	0.00593059841689974\\
266	0.00592869937769454\\
267	0.00592676628006091\\
268	0.00592479832244917\\
269	0.00592279468568871\\
270	0.00592075453274704\\
271	0.00591867700848528\\
272	0.00591656123940989\\
273	0.00591440633341983\\
274	0.00591221137954805\\
275	0.00590997544769704\\
276	0.00590769758836703\\
277	0.00590537683237556\\
278	0.00590301219056673\\
279	0.00590060265350832\\
280	0.00589814719117372\\
281	0.00589564475260517\\
282	0.00589309426555345\\
283	0.00589049463608816\\
284	0.00588784474817003\\
285	0.00588514346317461\\
286	0.00588238961935384\\
287	0.00587958203121711\\
288	0.00587671948880924\\
289	0.00587380075685598\\
290	0.00587082457374034\\
291	0.00586778965026332\\
292	0.0058646946681321\\
293	0.00586153827810502\\
294	0.00585831909770823\\
295	0.00585503570842117\\
296	0.0058516866522094\\
297	0.00584827042726107\\
298	0.005844785482755\\
299	0.00584123021243587\\
300	0.00583760294666414\\
301	0.00583390194251364\\
302	0.00583012537284562\\
303	0.00582627131476884\\
304	0.00582233773560924\\
305	0.00581832247727876\\
306	0.00581422323984336\\
307	0.00581003756541961\\
308	0.00580576282366509\\
309	0.00580139619656889\\
310	0.00579693467624333\\
311	0.00579237510283586\\
312	0.00578771420092257\\
313	0.00578294857752554\\
314	0.00577807472093154\\
315	0.00577308899970213\\
316	0.00576798766021564\\
317	0.00576276682629832\\
318	0.00575742250012165\\
319	0.00575195056427633\\
320	0.0057463467852586\\
321	0.00574060681863975\\
322	0.00573472621622314\\
323	0.00572870043553005\\
324	0.00572252485198907\\
325	0.00571619477423348\\
326	0.0057097054628966\\
327	0.00570305215303784\\
328	0.00569623007880404\\
329	0.00568923449356538\\
330	0.00568206071621986\\
331	0.00567470417398735\\
332	0.00566716044728615\\
333	0.0056594253182353\\
334	0.00565149482309718\\
335	0.00564336531262988\\
336	0.00563503349789744\\
337	0.00562649645778915\\
338	0.00561775163546331\\
339	0.00560879679679947\\
340	0.00559962992549786\\
341	0.00559024902400292\\
342	0.00558065177867284\\
343	0.00557083503636195\\
344	0.00556079387954658\\
345	0.00555051953043704\\
346	0.00553999868927008\\
347	0.00552921699659402\\
348	0.00551815895815368\\
349	0.00550680787097888\\
350	0.00549514575425154\\
351	0.00548315328290356\\
352	0.00547080958467503\\
353	0.00545809225929875\\
354	0.00544497741749707\\
355	0.00543143971394483\\
356	0.00541745244928231\\
357	0.00540298796483707\\
358	0.00538801795719036\\
359	0.00537251368367673\\
360	0.00535644652525741\\
361	0.00533978882818451\\
362	0.00532251507603197\\
363	0.00530460349848151\\
364	0.00528603825579737\\
365	0.00526681238042775\\
366	0.00524693171268722\\
367	0.00522642014082995\\
368	0.00520532655642379\\
369	0.00518373409893584\\
370	0.00516206586647816\\
371	0.00514044460870371\\
372	0.00511889942685035\\
373	0.005097461684701\\
374	0.00507616541703685\\
375	0.0050550469394263\\
376	0.00503414834652415\\
377	0.00501351824451031\\
378	0.00499320950024386\\
379	0.0049732792138121\\
380	0.00495378852032667\\
381	0.00493480214886549\\
382	0.00491638764213609\\
383	0.00489861409876216\\
384	0.0048815502818452\\
385	0.00486526185955928\\
386	0.00484980747629532\\
387	0.0048352332527258\\
388	0.00482156524752625\\
389	0.0048087991710341\\
390	0.00479688634022394\\
391	0.00478524062734394\\
392	0.004773860368139\\
393	0.00476275899406963\\
394	0.00475194851101444\\
395	0.00474143907290504\\
396	0.00473123849955345\\
397	0.00472135174083783\\
398	0.00471178029418218\\
399	0.0047025215893457\\
400	0.0046935683648113\\
401	0.00468490807480182\\
402	0.0046765223856452\\
403	0.00466838684684614\\
404	0.00466047086612644\\
405	0.00465273816968375\\
406	0.00464514801062233\\
407	0.00463765747607518\\
408	0.00463022821741895\\
409	0.00462285557451509\\
410	0.00461553393591348\\
411	0.00460825674039417\\
412	0.00460101649287061\\
413	0.00459380480350022\\
414	0.00458661236496382\\
415	0.00457942916295125\\
416	0.00457224459312569\\
417	0.00456504808471705\\
418	0.00455782910114952\\
419	0.00455057744375803\\
420	0.00454328362249896\\
421	0.00453593924048389\\
422	0.00452853733886829\\
423	0.00452107263056984\\
424	0.00451354120769885\\
425	0.00450593911397278\\
426	0.00449826238226361\\
427	0.00449050707496815\\
428	0.0044826693259891\\
429	0.00447474538262651\\
430	0.00446673164499422\\
431	0.00445862470053646\\
432	0.00445042135051302\\
433	0.00444211862500887\\
434	0.00443371378317185\\
435	0.00442520429581949\\
436	0.00441658780808193\\
437	0.00440786208314939\\
438	0.00439902493336187\\
439	0.00439007420042411\\
440	0.00438100775895551\\
441	0.00437182351869122\\
442	0.00436251942513567\\
443	0.00435309345852253\\
444	0.00434354363100991\\
445	0.00433386798214939\\
446	0.00432406457285262\\
447	0.00431413147827598\\
448	0.00430406678026078\\
449	0.00429386856017406\\
450	0.00428353489313293\\
451	0.00427306384456706\\
452	0.0042624534691652\\
453	0.00425170180958279\\
454	0.00424080689494397\\
455	0.00422976673918413\\
456	0.00421857933929183\\
457	0.00420724267351705\\
458	0.00419575469961823\\
459	0.00418411335321687\\
460	0.00417231654631692\\
461	0.00416036216602138\\
462	0.00414824807344308\\
463	0.00413597210275987\\
464	0.00412353206031631\\
465	0.0041109257237\\
466	0.0040981508407988\\
467	0.00408520512884452\\
468	0.00407208627344699\\
469	0.00405879192761907\\
470	0.00404531971079127\\
471	0.00403166720781047\\
472	0.00401783196791326\\
473	0.00400381150366288\\
474	0.00398960328983551\\
475	0.0039752047622424\\
476	0.0039606133164768\\
477	0.00394582630657787\\
478	0.00393084104360233\\
479	0.00391565479409372\\
480	0.00390026477843745\\
481	0.00388466816908875\\
482	0.00386886208865883\\
483	0.00385284360784337\\
484	0.00383660974317631\\
485	0.00382015745459029\\
486	0.00380348364276359\\
487	0.00378658514623244\\
488	0.00376945873824518\\
489	0.0037521011233333\\
490	0.00373450893357197\\
491	0.00371667872450067\\
492	0.00369860697067291\\
493	0.0036802900608012\\
494	0.00366172429246222\\
495	0.00364290586632503\\
496	0.00362383087986406\\
497	0.00360449532051765\\
498	0.003584895058252\\
499	0.00356502583749236\\
500	0.003544883268384\\
501	0.0035244628173497\\
502	0.00350375979691629\\
503	0.00348276935479163\\
504	0.00346148646218569\\
505	0.0034399059013872\\
506	0.00341802225263116\\
507	0.00339582988032334\\
508	0.00337332291873124\\
509	0.0033504952573046\\
510	0.00332734052585997\\
511	0.00330385207995603\\
512	0.00328002298690368\\
513	0.00325584601300512\\
514	0.00323131361280798\\
515	0.00320641792140324\\
516	0.0031811507511027\\
517	0.00315550359421973\\
518	0.00312946763416657\\
519	0.00310303376769541\\
520	0.00307619264188358\\
521	0.00304893471043217\\
522	0.00302125031506175\\
523	0.00299312979930897\\
524	0.00296456366392906\\
525	0.00293554277548647\\
526	0.00290605864302008\\
527	0.00287610378238502\\
528	0.00284567218440825\\
529	0.00281475988321566\\
530	0.00278336563667194\\
531	0.00275149226689524\\
532	0.00271915008403821\\
533	0.00268641628363034\\
534	0.00265328435369643\\
535	0.00261968582869031\\
536	0.0025854688063535\\
537	0.00255052199012215\\
538	0.00251478365610303\\
539	0.00247819002717404\\
540	0.00244068417819549\\
541	0.00240220348343551\\
542	0.00236268161584073\\
543	0.00232205422193516\\
544	0.00228024789926472\\
545	0.00223717673401289\\
546	0.00219271866604341\\
547	0.00214821380497191\\
548	0.00210568170443776\\
549	0.00206477854581583\\
550	0.00202367834717305\\
551	0.00198207658025245\\
552	0.00193997595629076\\
553	0.00189740567257551\\
554	0.00185441906889823\\
555	0.00181108860982139\\
556	0.00176750323728597\\
557	0.00172451269574327\\
558	0.00168246236750703\\
559	0.00164022170205425\\
560	0.00159770829159438\\
561	0.00155494262970847\\
562	0.00151195463762704\\
563	0.00146877499999463\\
564	0.00142543401352537\\
565	0.00138195982240711\\
566	0.00133854778582265\\
567	0.00129544180006035\\
568	0.00125203948509083\\
569	0.00120834728092063\\
570	0.00116438693696481\\
571	0.0011201825245913\\
572	0.0010757604602725\\
573	0.00103114962823804\\
574	0.00098638150202912\\
575	0.000941490255217017\\
576	0.000896512854883693\\
577	0.000851489130084093\\
578	0.000806461805377778\\
579	0.00076147648684911\\
580	0.000716581584684075\\
581	0.000671828152215967\\
582	0.000627269616206792\\
583	0.000582961366825369\\
584	0.00053896016820798\\
585	0.00049532334178407\\
586	0.000452107665645336\\
587	0.000409367927260632\\
588	0.000367155072958954\\
589	0.000325513941232697\\
590	0.000284480716949123\\
591	0.000244080684870839\\
592	0.000204328087251263\\
593	0.000165233217676714\\
594	0.000126870159937009\\
595	8.96111722547525e-05\\
596	5.42660945238512e-05\\
597	2.28062284332059e-05\\
598	2.9204464504877e-07\\
599	0\\
600	0\\
};
\addplot [color=black!80!mycolor21,solid,forget plot]
  table[row sep=crcr]{%
1	0.00604026546700612\\
2	0.0060402533658538\\
3	0.00604024098238466\\
4	0.00604022831004924\\
5	0.00604021534214751\\
6	0.00604020207182548\\
7	0.00604018849207171\\
8	0.00604017459571358\\
9	0.00604016037541399\\
10	0.00604014582366729\\
11	0.0060401309327957\\
12	0.00604011569494533\\
13	0.00604010010208226\\
14	0.00604008414598843\\
15	0.00604006781825751\\
16	0.00604005111029071\\
17	0.00604003401329235\\
18	0.00604001651826554\\
19	0.0060399986160076\\
20	0.00603998029710551\\
21	0.00603996155193108\\
22	0.00603994237063621\\
23	0.00603992274314792\\
24	0.00603990265916339\\
25	0.00603988210814473\\
26	0.00603986107931386\\
27	0.00603983956164693\\
28	0.00603981754386914\\
29	0.0060397950144488\\
30	0.00603977196159192\\
31	0.00603974837323622\\
32	0.00603972423704519\\
33	0.006039699540402\\
34	0.0060396742704033\\
35	0.00603964841385282\\
36	0.00603962195725503\\
37	0.0060395948868083\\
38	0.00603956718839841\\
39	0.00603953884759133\\
40	0.00603950984962644\\
41	0.00603948017940923\\
42	0.00603944982150397\\
43	0.00603941876012617\\
44	0.00603938697913498\\
45	0.0060393544620254\\
46	0.00603932119192022\\
47	0.00603928715156196\\
48	0.00603925232330455\\
49	0.00603921668910482\\
50	0.00603918023051389\\
51	0.00603914292866831\\
52	0.00603910476428106\\
53	0.0060390657176325\\
54	0.0060390257685607\\
55	0.00603898489645214\\
56	0.00603894308023186\\
57	0.00603890029835354\\
58	0.00603885652878928\\
59	0.00603881174901931\\
60	0.00603876593602144\\
61	0.00603871906626033\\
62	0.00603867111567635\\
63	0.00603862205967457\\
64	0.00603857187311328\\
65	0.00603852053029232\\
66	0.00603846800494131\\
67	0.00603841427020753\\
68	0.00603835929864351\\
69	0.00603830306219463\\
70	0.00603824553218625\\
71	0.00603818667931065\\
72	0.00603812647361384\\
73	0.00603806488448191\\
74	0.00603800188062737\\
75	0.00603793743007505\\
76	0.00603787150014784\\
77	0.0060378040574521\\
78	0.00603773506786285\\
79	0.00603766449650872\\
80	0.00603759230775658\\
81	0.00603751846519583\\
82	0.00603744293162264\\
83	0.00603736566902364\\
84	0.00603728663855948\\
85	0.00603720580054809\\
86	0.00603712311444767\\
87	0.00603703853883924\\
88	0.00603695203140918\\
89	0.00603686354893113\\
90	0.00603677304724792\\
91	0.00603668048125302\\
92	0.00603658580487165\\
93	0.00603648897104176\\
94	0.00603638993169448\\
95	0.00603628863773451\\
96	0.0060361850390201\\
97	0.00603607908434256\\
98	0.0060359707214058\\
99	0.00603585989680519\\
100	0.00603574655600643\\
101	0.00603563064332395\\
102	0.00603551210189894\\
103	0.00603539087367727\\
104	0.00603526689938696\\
105	0.0060351401185154\\
106	0.00603501046928619\\
107	0.00603487788863583\\
108	0.00603474231218996\\
109	0.00603460367423942\\
110	0.00603446190771583\\
111	0.00603431694416723\\
112	0.0060341687137331\\
113	0.00603401714511918\\
114	0.00603386216557229\\
115	0.00603370370085445\\
116	0.00603354167521696\\
117	0.00603337601137441\\
118	0.00603320663047808\\
119	0.00603303345208935\\
120	0.0060328563941528\\
121	0.00603267537296921\\
122	0.00603249030316817\\
123	0.00603230109768058\\
124	0.00603210766771103\\
125	0.00603190992271004\\
126	0.00603170777034592\\
127	0.00603150111647674\\
128	0.00603128986512214\\
129	0.00603107391843494\\
130	0.00603085317667271\\
131	0.00603062753816934\\
132	0.00603039689930652\\
133	0.00603016115448515\\
134	0.00602992019609692\\
135	0.00602967391449568\\
136	0.00602942219796904\\
137	0.00602916493271007\\
138	0.006028902002789\\
139	0.00602863329012514\\
140	0.00602835867445888\\
141	0.00602807803332407\\
142	0.00602779124202049\\
143	0.0060274981735866\\
144	0.00602719869877273\\
145	0.00602689268601452\\
146	0.00602658000140678\\
147	0.00602626050867775\\
148	0.00602593406916396\\
149	0.00602560054178549\\
150	0.00602525978302194\\
151	0.00602491164688882\\
152	0.00602455598491491\\
153	0.00602419264611996\\
154	0.00602382147699359\\
155	0.00602344232147472\\
156	0.00602305502093198\\
157	0.00602265941414526\\
158	0.00602225533728789\\
159	0.00602184262391019\\
160	0.00602142110492413\\
161	0.00602099060858907\\
162	0.0060205509604988\\
163	0.00602010198356985\\
164	0.00601964349803142\\
165	0.00601917532141646\\
166	0.00601869726855445\\
167	0.00601820915156574\\
168	0.00601771077985748\\
169	0.00601720196012123\\
170	0.00601668249633258\\
171	0.00601615218975228\\
172	0.00601561083892954\\
173	0.00601505823970703\\
174	0.00601449418522819\\
175	0.00601391846594625\\
176	0.00601333086963577\\
177	0.00601273118140585\\
178	0.00601211918371597\\
179	0.00601149465639383\\
180	0.00601085737665546\\
181	0.00601020711912762\\
182	0.00600954365587254\\
183	0.00600886675641485\\
184	0.00600817618777094\\
185	0.00600747171448041\\
186	0.00600675309863981\\
187	0.00600602009993884\\
188	0.00600527247569823\\
189	0.0060045099809102\\
190	0.00600373236828059\\
191	0.00600293938827294\\
192	0.00600213078915456\\
193	0.00600130631704403\\
194	0.00600046571596032\\
195	0.00599960872787332\\
196	0.00599873509275525\\
197	0.00599784454863329\\
198	0.00599693683164267\\
199	0.00599601167608023\\
200	0.00599506881445815\\
201	0.00599410797755745\\
202	0.0059931288944809\\
203	0.00599213129270514\\
204	0.00599111489813133\\
205	0.00599007943513419\\
206	0.00598902462660888\\
207	0.00598795019401522\\
208	0.00598685585741859\\
209	0.0059857413355276\\
210	0.00598460634572723\\
211	0.00598345060410724\\
212	0.00598227382548555\\
213	0.00598107572342554\\
214	0.00597985601024703\\
215	0.00597861439703013\\
216	0.00597735059361167\\
217	0.00597606430857348\\
218	0.00597475524922178\\
219	0.00597342312155775\\
220	0.00597206763023817\\
221	0.00597068847852598\\
222	0.00596928536823059\\
223	0.00596785799963718\\
224	0.00596640607142497\\
225	0.00596492928057418\\
226	0.00596342732226176\\
227	0.00596189988974522\\
228	0.00596034667423549\\
229	0.00595876736475797\\
230	0.00595716164800227\\
231	0.00595552920816083\\
232	0.00595386972675607\\
233	0.00595218288245648\\
234	0.00595046835088112\\
235	0.00594872580439209\\
236	0.00594695491187441\\
237	0.00594515533850142\\
238	0.00594332674548392\\
239	0.00594146878979969\\
240	0.00593958112389889\\
241	0.00593766339537894\\
242	0.00593571524662091\\
243	0.00593373631437569\\
244	0.00593172622928573\\
245	0.00592968461532358\\
246	0.00592761108912331\\
247	0.00592550525917501\\
248	0.00592336672484548\\
249	0.0059211950751797\\
250	0.00591898988742878\\
251	0.0059167507252386\\
252	0.0059144771364235\\
253	0.00591216865023706\\
254	0.00590982477404137\\
255	0.00590744498926841\\
256	0.00590502874656275\\
257	0.00590257545999976\\
258	0.00590008450029143\\
259	0.00589755518693252\\
260	0.00589498677931358\\
261	0.00589237846695294\\
262	0.0058897293591971\\
263	0.00588703847501438\\
264	0.00588430473372601\\
265	0.00588152694691003\\
266	0.00587870381006663\\
267	0.00587583392198569\\
268	0.00587291582604199\\
269	0.00586994800842341\\
270	0.00586692889637673\\
271	0.00586385685648655\\
272	0.00586073019300609\\
273	0.0058575471462592\\
274	0.00585430589114777\\
275	0.00585100453578624\\
276	0.00584764112030902\\
277	0.00584421361589246\\
278	0.00584071992404575\\
279	0.005837157876233\\
280	0.00583352523390158\\
281	0.005829819689003\\
282	0.00582603886510844\\
283	0.0058221803192368\\
284	0.00581824154453196\\
285	0.00581421997394661\\
286	0.00581011298511246\\
287	0.00580591790660013\\
288	0.00580163202579745\\
289	0.00579725259865931\\
290	0.00579277686160511\\
291	0.00578820204585885\\
292	0.00578352539453689\\
293	0.00577874418278647\\
294	0.00577385574125485\\
295	0.00576885748311349\\
296	0.00576374693476257\\
297	0.00575852177017284\\
298	0.00575317984855776\\
299	0.00574771925464148\\
300	0.00574213834001177\\
301	0.00573643576217835\\
302	0.00573061051281166\\
303	0.00572466194125788\\
304	0.00571858977587763\\
305	0.00571239410950164\\
306	0.00570607534071958\\
307	0.00569963405213641\\
308	0.0056930707978446\\
309	0.00568638576265987\\
310	0.00567957813374577\\
311	0.00567264463828431\\
312	0.0056655794880873\\
313	0.00565837646431389\\
314	0.00565102888368784\\
315	0.00564352956717776\\
316	0.00563587080655857\\
317	0.00562804430077823\\
318	0.00562004110318492\\
319	0.005611851568863\\
320	0.00560346529909061\\
321	0.00559487108331325\\
322	0.00558605683930868\\
323	0.00557700955256866\\
324	0.00556771521640067\\
325	0.00555815877491492\\
326	0.00554832407201765\\
327	0.00553819381099456\\
328	0.00552774953117986\\
329	0.0055169716024414\\
330	0.00550583909829185\\
331	0.00549432998080299\\
332	0.00548242125738638\\
333	0.00547008920475404\\
334	0.0054573097112811\\
335	0.00544405879011807\\
336	0.00543031344118811\\
337	0.0054160527628837\\
338	0.00540125912742442\\
339	0.00538592000330503\\
340	0.00537003044851019\\
341	0.00535359644526481\\
342	0.00533663934030161\\
343	0.00531920174680257\\
344	0.00530144252339929\\
345	0.00528361270366819\\
346	0.00526572555361539\\
347	0.00524779599459207\\
348	0.00522984077687911\\
349	0.00521187866831396\\
350	0.00519393066326458\\
351	0.00517602024434208\\
352	0.00515817393827313\\
353	0.00514041964728142\\
354	0.00512278743406033\\
355	0.00510530997362699\\
356	0.00508802247112745\\
357	0.00507096222996936\\
358	0.00505417115077324\\
359	0.00503769643319219\\
360	0.00502158814721375\\
361	0.00500589866402464\\
362	0.00499068173694374\\
363	0.00497599110596318\\
364	0.0049618784526552\\
365	0.0049483904821479\\
366	0.00493556483546533\\
367	0.00492342444071178\\
368	0.00491196978455394\\
369	0.00490116839863309\\
370	0.0048906324354223\\
371	0.00488027870923257\\
372	0.00487012001402301\\
373	0.00486016882262852\\
374	0.00485043709822925\\
375	0.00484093607625654\\
376	0.00483167592810388\\
377	0.0048226653696756\\
378	0.00481391127770395\\
379	0.00480541826302987\\
380	0.0047971882060768\\
381	0.00478921976522046\\
382	0.00478150787669776\\
383	0.0047740432764957\\
384	0.00476681208989165\\
385	0.00475979555689918\\
386	0.00475296999263243\\
387	0.00474630712384782\\
388	0.00473977499796648\\
389	0.00473333973708328\\
390	0.00472696851258153\\
391	0.00472065829735722\\
392	0.00471440615456371\\
393	0.0047082083800895\\
394	0.00470206048044941\\
395	0.00469595716595798\\
396	0.00468989236351199\\
397	0.00468385925379096\\
398	0.00467785033801642\\
399	0.00467185753945659\\
400	0.00466587234443052\\
401	0.00465988598636044\\
402	0.00465388967405033\\
403	0.00464787486131385\\
404	0.00464183354823129\\
405	0.00463575859381497\\
406	0.00462964400391982\\
407	0.00462348513651544\\
408	0.00461727863610413\\
409	0.00461102107911178\\
410	0.00460470887362543\\
411	0.00459833843936909\\
412	0.00459190635956627\\
413	0.00458540926575913\\
414	0.00457884387749456\\
415	0.00457220703776355\\
416	0.0045654957449966\\
417	0.00455870716969015\\
418	0.00455183867148872\\
419	0.00454488780577619\\
420	0.00453785231597264\\
421	0.00453073010934476\\
422	0.00452351921611247\\
423	0.00451621773483804\\
424	0.0045088237842544\\
425	0.00450133550763517\\
426	0.00449375107627498\\
427	0.00448606869187626\\
428	0.00447828658765964\\
429	0.00447040302805675\\
430	0.00446241630691595\\
431	0.0044543247442354\\
432	0.00444612668155861\\
433	0.00443782047631773\\
434	0.00442940449557253\\
435	0.00442087710976413\\
436	0.0044122366872682\\
437	0.00440348159058331\\
438	0.00439461017483153\\
439	0.00438562078690467\\
440	0.00437651176441105\\
441	0.00436728143444992\\
442	0.0043579281122518\\
443	0.00434845009973193\\
444	0.00433884568401337\\
445	0.00432911313598069\\
446	0.00431925070892445\\
447	0.00430925663732789\\
448	0.00429912913582875\\
449	0.00428886639836054\\
450	0.00427846659744047\\
451	0.00426792788352919\\
452	0.00425724838437782\\
453	0.00424642620436951\\
454	0.00423545942386197\\
455	0.00422434609853724\\
456	0.00421308425876252\\
457	0.00420167190896466\\
458	0.00419010702701767\\
459	0.00417838756364069\\
460	0.00416651144180048\\
461	0.0041544765561112\\
462	0.00414228077222317\\
463	0.00412992192619294\\
464	0.00411739782383058\\
465	0.00410470624002143\\
466	0.00409184491801969\\
467	0.00407881156871013\\
468	0.00406560386983372\\
469	0.00405221946517284\\
470	0.00403865596368979\\
471	0.00402491093861322\\
472	0.00401098192646624\\
473	0.00399686642602873\\
474	0.00398256189722682\\
475	0.00396806575994199\\
476	0.00395337539273091\\
477	0.00393848813144712\\
478	0.00392340126775386\\
479	0.0039081120475174\\
480	0.00389261766906804\\
481	0.00387691528131579\\
482	0.00386100198170571\\
483	0.00384487481399766\\
484	0.00382853076585254\\
485	0.00381196676620695\\
486	0.00379517968241609\\
487	0.00377816631714323\\
488	0.00376092340497299\\
489	0.0037434476087237\\
490	0.00372573551543333\\
491	0.00370778363199189\\
492	0.00368958838039254\\
493	0.00367114609257297\\
494	0.00365245300481891\\
495	0.00363350525170209\\
496	0.00361429885952708\\
497	0.00359482973926421\\
498	0.00357509367895112\\
499	0.00355508633555229\\
500	0.00353480322627609\\
501	0.00351423971936296\\
502	0.00349339102437709\\
503	0.00347225218205858\\
504	0.00345081805382653\\
505	0.00342908331106559\\
506	0.00340704242438401\\
507	0.00338468965310275\\
508	0.00336201903532531\\
509	0.00333902437905489\\
510	0.00331569925497228\\
511	0.00329203699167503\\
512	0.00326803067441413\\
513	0.00324367314866261\\
514	0.00321895703022545\\
515	0.00319387472407181\\
516	0.00316841845466331\\
517	0.0031425803112958\\
518	0.00311635231290443\\
519	0.00308972649795068\\
520	0.00306269504647383\\
521	0.003035250443222\\
522	0.00300738569307181\\
523	0.00297909460281731\\
524	0.00295037214700139\\
525	0.00292121493997158\\
526	0.00289162182508419\\
527	0.00286159456307669\\
528	0.00283113903050781\\
529	0.00280026820797992\\
530	0.00276905767127757\\
531	0.00273751343323209\\
532	0.00270557045782752\\
533	0.00267307211406851\\
534	0.00263992242027967\\
535	0.00260606339991704\\
536	0.00257144120591396\\
537	0.00253600462960201\\
538	0.00249969764946177\\
539	0.0024624594910934\\
540	0.0024242224843366\\
541	0.00238491137467294\\
542	0.00234444454378673\\
543	0.00230273639459776\\
544	0.00225966796547375\\
545	0.00221668865662068\\
546	0.00217582521296366\\
547	0.0021362607008898\\
548	0.00209639682394114\\
549	0.00205600872796146\\
550	0.00201508651193405\\
551	0.00197366306424266\\
552	0.00193178370444651\\
553	0.00188950895448212\\
554	0.00184691508122861\\
555	0.0018042815234374\\
556	0.00176267919822897\\
557	0.00172129441691307\\
558	0.00167960212397915\\
559	0.0016376158968852\\
560	0.00159536277689962\\
561	0.00155287132132885\\
562	0.00151016995834333\\
563	0.0014672865789184\\
564	0.00142424694170975\\
565	0.00138130050605624\\
566	0.00133853402642386\\
567	0.00129544164993273\\
568	0.00125203947380946\\
569	0.00120834727902135\\
570	0.00116438693634367\\
571	0.00112018252432505\\
572	0.00107576046014404\\
573	0.00103114962817055\\
574	0.000986381501991801\\
575	0.000941490255195813\\
576	0.000896512854871852\\
577	0.000851489130077657\\
578	0.000806461805374505\\
579	0.000761476486847638\\
580	0.000716581584683521\\
581	0.000671828152215824\\
582	0.000627269616206769\\
583	0.00058296136682538\\
584	0.000538960168207986\\
585	0.000495323341784082\\
586	0.00045210766564535\\
587	0.000409367927260641\\
588	0.000367155072958961\\
589	0.000325513941232703\\
590	0.000284480716949125\\
591	0.000244080684870838\\
592	0.00020432808725126\\
593	0.000165233217676711\\
594	0.000126870159937008\\
595	8.9611172254752e-05\\
596	5.42660945238509e-05\\
597	2.28062284332057e-05\\
598	2.9204464504877e-07\\
599	0\\
600	0\\
};
\addplot [color=black,solid,forget plot]
  table[row sep=crcr]{%
1	0.00604024220115463\\
2	0.00604022941156559\\
3	0.00604021631894476\\
4	0.00604020291611037\\
5	0.00604018919571038\\
6	0.00604017515021865\\
7	0.0060401607719305\\
8	0.00604014605295886\\
9	0.00604013098522959\\
10	0.00604011556047734\\
11	0.00604009977024089\\
12	0.00604008360585844\\
13	0.00604006705846304\\
14	0.00604005011897763\\
15	0.00604003277811003\\
16	0.00604001502634798\\
17	0.0060399968539538\\
18	0.00603997825095909\\
19	0.00603995920715936\\
20	0.00603993971210836\\
21	0.00603991975511225\\
22	0.00603989932522407\\
23	0.00603987841123736\\
24	0.00603985700168037\\
25	0.00603983508480957\\
26	0.00603981264860322\\
27	0.00603978968075492\\
28	0.00603976616866678\\
29	0.0060397420994427\\
30	0.00603971745988097\\
31	0.00603969223646742\\
32	0.00603966641536785\\
33	0.00603963998242046\\
34	0.00603961292312817\\
35	0.00603958522265073\\
36	0.00603955686579649\\
37	0.00603952783701423\\
38	0.00603949812038457\\
39	0.00603946769961138\\
40	0.00603943655801282\\
41	0.00603940467851219\\
42	0.00603937204362877\\
43	0.00603933863546809\\
44	0.00603930443571237\\
45	0.00603926942561029\\
46	0.00603923358596707\\
47	0.00603919689713374\\
48	0.00603915933899659\\
49	0.00603912089096622\\
50	0.00603908153196619\\
51	0.00603904124042176\\
52	0.00603899999424794\\
53	0.00603895777083757\\
54	0.00603891454704906\\
55	0.00603887029919381\\
56	0.00603882500302325\\
57	0.00603877863371572\\
58	0.00603873116586302\\
59	0.00603868257345652\\
60	0.00603863282987321\\
61	0.00603858190786107\\
62	0.00603852977952445\\
63	0.00603847641630884\\
64	0.00603842178898546\\
65	0.00603836586763539\\
66	0.0060383086216335\\
67	0.00603825001963159\\
68	0.0060381900295418\\
69	0.00603812861851904\\
70	0.00603806575294322\\
71	0.00603800139840123\\
72	0.00603793551966834\\
73	0.00603786808068912\\
74	0.00603779904455802\\
75	0.00603772837349957\\
76	0.00603765602884802\\
77	0.00603758197102642\\
78	0.00603750615952539\\
79	0.00603742855288145\\
80	0.00603734910865462\\
81	0.00603726778340568\\
82	0.00603718453267291\\
83	0.00603709931094821\\
84	0.00603701207165275\\
85	0.00603692276711206\\
86	0.00603683134853047\\
87	0.00603673776596517\\
88	0.00603664196829936\\
89	0.0060365439032152\\
90	0.00603644351716582\\
91	0.0060363407553467\\
92	0.00603623556166671\\
93	0.0060361278787182\\
94	0.00603601764774663\\
95	0.00603590480861926\\
96	0.00603578929979333\\
97	0.00603567105828357\\
98	0.00603555001962885\\
99	0.00603542611785818\\
100	0.00603529928545591\\
101	0.00603516945332611\\
102	0.00603503655075626\\
103	0.00603490050538015\\
104	0.00603476124313985\\
105	0.00603461868824691\\
106	0.00603447276314273\\
107	0.00603432338845804\\
108	0.00603417048297137\\
109	0.00603401396356695\\
110	0.00603385374519139\\
111	0.00603368974080943\\
112	0.00603352186135891\\
113	0.0060333500157048\\
114	0.00603317411059178\\
115	0.00603299405059647\\
116	0.00603280973807822\\
117	0.00603262107312888\\
118	0.00603242795352169\\
119	0.00603223027465891\\
120	0.00603202792951842\\
121	0.00603182080859922\\
122	0.00603160879986567\\
123	0.00603139178869086\\
124	0.00603116965779838\\
125	0.00603094228720326\\
126	0.00603070955415151\\
127	0.00603047133305845\\
128	0.0060302274954458\\
129	0.00602997790987749\\
130	0.00602972244189418\\
131	0.00602946095394636\\
132	0.00602919330532638\\
133	0.00602891935209879\\
134	0.00602863894702952\\
135	0.00602835193951357\\
136	0.00602805817550143\\
137	0.00602775749742385\\
138	0.00602744974411521\\
139	0.00602713475073561\\
140	0.00602681234869119\\
141	0.00602648236555314\\
142	0.0060261446249751\\
143	0.00602579894660905\\
144	0.00602544514601946\\
145	0.00602508303459615\\
146	0.00602471241946532\\
147	0.00602433310339914\\
148	0.00602394488472337\\
149	0.00602354755722375\\
150	0.00602314091005041\\
151	0.00602272472762075\\
152	0.00602229878952048\\
153	0.00602186287040315\\
154	0.00602141673988773\\
155	0.00602096016245453\\
156	0.00602049289733946\\
157	0.00602001469842612\\
158	0.00601952531413678\\
159	0.0060190244873208\\
160	0.00601851195514175\\
161	0.00601798744896242\\
162	0.00601745069422821\\
163	0.00601690141034843\\
164	0.00601633931057578\\
165	0.00601576410188413\\
166	0.00601517548484403\\
167	0.00601457315349658\\
168	0.00601395679522537\\
169	0.00601332609062629\\
170	0.00601268071337537\\
171	0.00601202033009492\\
172	0.00601134460021717\\
173	0.0060106531758464\\
174	0.00600994570161868\\
175	0.00600922181455949\\
176	0.00600848114393944\\
177	0.00600772331112782\\
178	0.00600694792944373\\
179	0.00600615460400528\\
180	0.00600534293157628\\
181	0.00600451250041084\\
182	0.00600366289009542\\
183	0.00600279367138868\\
184	0.00600190440605857\\
185	0.00600099464671707\\
186	0.00600006393665222\\
187	0.00599911180965739\\
188	0.0059981377898579\\
189	0.00599714139153454\\
190	0.00599612211894405\\
191	0.00599507946613675\\
192	0.0059940129167705\\
193	0.00599292194392151\\
194	0.00599180600989156\\
195	0.00599066456601142\\
196	0.00598949705244036\\
197	0.00598830289796187\\
198	0.00598708151977458\\
199	0.00598583232327963\\
200	0.00598455470186254\\
201	0.00598324803667099\\
202	0.00598191169638715\\
203	0.00598054503699518\\
204	0.00597914740154348\\
205	0.00597771811990127\\
206	0.00597625650851\\
207	0.00597476187012911\\
208	0.0059732334935763\\
209	0.00597167065346244\\
210	0.00597007260992117\\
211	0.00596843860833389\\
212	0.00596676787905015\\
213	0.00596505963710449\\
214	0.00596331308193016\\
215	0.00596152739707178\\
216	0.00595970174989727\\
217	0.00595783529131195\\
218	0.00595592715547681\\
219	0.0059539764595338\\
220	0.00595198230334206\\
221	0.00594994376922937\\
222	0.0059478599217644\\
223	0.00594572980755597\\
224	0.0059435524550875\\
225	0.00594132687459575\\
226	0.00593905205800509\\
227	0.00593672697893076\\
228	0.00593435059276621\\
229	0.00593192183687351\\
230	0.00592943963089844\\
231	0.00592690287723542\\
232	0.00592431046167212\\
233	0.00592166125424854\\
234	0.00591895411037039\\
235	0.00591618787222353\\
236	0.00591336137054312\\
237	0.00591047342679857\\
238	0.00590752285586539\\
239	0.00590450846926347\\
240	0.00590142907905286\\
241	0.0058982835024888\\
242	0.00589507056754937\\
243	0.00589178911946169\\
244	0.00588843802836297\\
245	0.00588501619824405\\
246	0.00588152257732885\\
247	0.00587795617004913\\
248	0.00587431605076752\\
249	0.00587060137939107\\
250	0.00586681141898741\\
251	0.00586294555546768\\
252	0.00585900331931978\\
253	0.00585498440925583\\
254	0.00585088871745866\\
255	0.00584671635585565\\
256	0.00584246768248407\\
257	0.00583814332650342\\
258	0.00583374420970669\\
259	0.00582927156141854\\
260	0.0058247269223566\\
261	0.00582011213126507\\
262	0.00581542928577073\\
263	0.0058106806658248\\
264	0.00580586860408352\\
265	0.00580099528198818\\
266	0.00579606241162627\\
267	0.00579107060308566\\
268	0.00578601800834475\\
269	0.00578090264365087\\
270	0.00577572237782153\\
271	0.00577047491949016\\
272	0.00576515780320051\\
273	0.00575976837424582\\
274	0.00575430377214284\\
275	0.00574876091262352\\
276	0.00574313646802086\\
277	0.00573742684592163\\
278	0.00573162816595557\\
279	0.00572573623459187\\
280	0.00571974651781746\\
281	0.00571365411158153\\
282	0.00570745370991107\\
283	0.00570113957063665\\
284	0.00569470547871148\\
285	0.00568814470717387\\
286	0.00568144997589694\\
287	0.00567461340840066\\
288	0.00566762648717802\\
289	0.0056604800082285\\
290	0.00565316403581248\\
291	0.00564566785886966\\
292	0.00563797995111085\\
293	0.00563008793754007\\
294	0.005621978571147\\
295	0.00561363772479723\\
296	0.00560505040503344\\
297	0.00559620079670585\\
298	0.0055870723502374\\
299	0.00557764792713239\\
300	0.00556791002437553\\
301	0.00555784110485093\\
302	0.00554742406491736\\
303	0.00553664279888841\\
304	0.00552548302651021\\
305	0.00551393362608928\\
306	0.00550198829571939\\
307	0.00548964775149012\\
308	0.00547692265350177\\
309	0.00546383750077499\\
310	0.00545050411544554\\
311	0.00543706551203978\\
312	0.00542352426312607\\
313	0.00540988338268015\\
314	0.00539614638453567\\
315	0.00538231736054493\\
316	0.00536840113025545\\
317	0.00535440347191429\\
318	0.00534033101754344\\
319	0.00532619131263288\\
320	0.00531199291781422\\
321	0.00529774552004493\\
322	0.00528346005357307\\
323	0.00526914883070616\\
324	0.00525482568205746\\
325	0.0052405061054705\\
326	0.00522620742230711\\
327	0.00521194893974284\\
328	0.00519775212120448\\
329	0.00518364079415995\\
330	0.00516964162363341\\
331	0.00515578212083989\\
332	0.00514209199396442\\
333	0.00512860302430427\\
334	0.00511534880080009\\
335	0.00510236423489391\\
336	0.00508968465564633\\
337	0.00507734632258802\\
338	0.00506538684370882\\
339	0.00505384123288094\\
340	0.00504273935097975\\
341	0.00503210235176745\\
342	0.0050219378023539\\
343	0.00501223303726919\\
344	0.00500285468968795\\
345	0.00499356955559161\\
346	0.00498438739563515\\
347	0.00497531818451975\\
348	0.00496637204863509\\
349	0.00495755918753014\\
350	0.00494888977653276\\
351	0.00494037384742799\\
352	0.00493202114184904\\
353	0.00492384098152797\\
354	0.00491584209842258\\
355	0.00490803242980753\\
356	0.00490041888681241\\
357	0.00489300709639044\\
358	0.0048858010512431\\
359	0.00487880271345722\\
360	0.00487201164452336\\
361	0.00486542462797597\\
362	0.00485903530661304\\
363	0.00485283386790705\\
364	0.00484680682774371\\
365	0.00484093698516495\\
366	0.00483520365176605\\
367	0.00482958330142587\\
368	0.00482405084203799\\
369	0.00481858178443982\\
370	0.00481317112843914\\
371	0.00480781825833003\\
372	0.00480252205246886\\
373	0.00479728084542065\\
374	0.0047920923938142\\
375	0.00478695384774833\\
376	0.00478186173200704\\
377	0.00477681194038512\\
378	0.00477179974536357\\
379	0.00476681982690542\\
380	0.00476186632440995\\
381	0.00475693291591714\\
382	0.00475201292834093\\
383	0.00474709948159236\\
384	0.00474218566766209\\
385	0.00473726476257569\\
386	0.00473233046398241\\
387	0.00472737713907369\\
388	0.00472240005545754\\
389	0.00471739555086495\\
390	0.00471236108295356\\
391	0.00470729403270115\\
392	0.00470219168751057\\
393	0.00469705126007499\\
394	0.00469186991009814\\
395	0.0046866447686414\\
396	0.00468137296467694\\
397	0.00467605165319088\\
398	0.0046706780439002\\
399	0.00466524942932727\\
400	0.00465976321063176\\
401	0.00465421691925515\\
402	0.00464860823213952\\
403	0.00464293497810728\\
404	0.00463719513305938\\
405	0.00463138680214551\\
406	0.00462550818845853\\
407	0.00461955755337387\\
408	0.00461353313115817\\
409	0.00460743324585354\\
410	0.00460125630724931\\
411	0.00459500075370312\\
412	0.00458866504993294\\
413	0.00458224768832692\\
414	0.00457574718895802\\
415	0.00456916209832784\\
416	0.00456249098687029\\
417	0.00455573244556927\\
418	0.00454888508158583\\
419	0.00454194751318431\\
420	0.00453491836442718\\
421	0.00452779626024947\\
422	0.00452057982260584\\
423	0.00451326766834301\\
424	0.00450585840873635\\
425	0.00449835064883985\\
426	0.0044907429866612\\
427	0.00448303401218258\\
428	0.00447522230625659\\
429	0.00446730643941554\\
430	0.00445928497063901\\
431	0.00445115644612937\\
432	0.00444291939814595\\
433	0.00443457234394397\\
434	0.00442611378485224\\
435	0.00441754220550516\\
436	0.0044088560732158\\
437	0.00440005383744301\\
438	0.00439113392927468\\
439	0.00438209476090384\\
440	0.00437293472510363\\
441	0.00436365219470771\\
442	0.00435424552210145\\
443	0.00434471303872821\\
444	0.00433505305461357\\
445	0.00432526385790813\\
446	0.00431534371444779\\
447	0.00430529086732777\\
448	0.00429510353648548\\
449	0.00428477991828609\\
450	0.00427431818510563\\
451	0.00426371648490848\\
452	0.00425297294081893\\
453	0.00424208565068619\\
454	0.00423105268664214\\
455	0.00421987209465014\\
456	0.00420854189404392\\
457	0.00419706007705395\\
458	0.0041854246083197\\
459	0.00417363342438504\\
460	0.00416168443317458\\
461	0.00414957551344839\\
462	0.00413730451423243\\
463	0.00412486925422227\\
464	0.00411226752115713\\
465	0.00409949707116105\\
466	0.00408655562804788\\
467	0.00407344088258621\\
468	0.00406015049171967\\
469	0.00404668207773832\\
470	0.00403303322739574\\
471	0.00401920149096636\\
472	0.00400518438123652\\
473	0.00399097937242283\\
474	0.00397658389901009\\
475	0.00396199535450056\\
476	0.00394721109006594\\
477	0.00393222841309141\\
478	0.00391704458560201\\
479	0.00390165682255867\\
480	0.00388606229001174\\
481	0.00387025810309765\\
482	0.00385424132386436\\
483	0.00383800895890885\\
484	0.00382155795681001\\
485	0.00380488520533814\\
486	0.00378798752842205\\
487	0.00377086168285347\\
488	0.00375350435470764\\
489	0.00373591215545927\\
490	0.00371808161777201\\
491	0.00370000919094156\\
492	0.0036816912359725\\
493	0.00366312402027269\\
494	0.00364430371195221\\
495	0.00362522637371953\\
496	0.00360588795637567\\
497	0.00358628429191823\\
498	0.00356641108628137\\
499	0.0035462639117581\\
500	0.00352583819917636\\
501	0.00350512922993411\\
502	0.00348413212804223\\
503	0.00346284185237889\\
504	0.00344125318943081\\
505	0.00341936074688669\\
506	0.00339715894856318\\
507	0.00337464203128774\\
508	0.0033518040445461\\
509	0.00332863885393061\\
510	0.0033051401497136\\
511	0.00328130146222999\\
512	0.00325711618620332\\
513	0.00323257761671049\\
514	0.00320767900018014\\
515	0.00318241360468962\\
516	0.00315677481490683\\
517	0.00313075625836541\\
518	0.00310435197142525\\
519	0.00307755661533165\\
520	0.00305036575533494\\
521	0.00302277621898349\\
522	0.00299478655359284\\
523	0.00296639760769014\\
524	0.00293761326714388\\
525	0.00290844138313012\\
526	0.00287889567964912\\
527	0.00284903708034976\\
528	0.00281890494915272\\
529	0.00278843855151698\\
530	0.00275749196392408\\
531	0.00272596507989511\\
532	0.00269380524096619\\
533	0.00266096644932591\\
534	0.00262740435415402\\
535	0.00259307084256343\\
536	0.00255791304693705\\
537	0.00252187174740753\\
538	0.00248488045298894\\
539	0.00244686410993284\\
540	0.0024077377673102\\
541	0.00236740477819526\\
542	0.00232575612961217\\
543	0.00228403044480058\\
544	0.00224447891824068\\
545	0.00220617374851022\\
546	0.00216751041189044\\
547	0.00212829967778029\\
548	0.00208853590537212\\
549	0.00204824471904266\\
550	0.00200746452793853\\
551	0.00196624548713097\\
552	0.0019246531104606\\
553	0.00188276877233496\\
554	0.00184135597079297\\
555	0.00180079542484969\\
556	0.00175995385435818\\
557	0.00171878217399611\\
558	0.00167730425837276\\
559	0.0016355463416697\\
560	0.00159353515574625\\
561	0.00155129727986493\\
562	0.00150885875639163\\
563	0.0014662435534718\\
564	0.00142371991366068\\
565	0.00138129869913588\\
566	0.00133853400693808\\
567	0.00129544164841737\\
568	0.00125203947353977\\
569	0.0012083472789297\\
570	0.00116438693630348\\
571	0.0011201825243051\\
572	0.00107576046013345\\
573	0.00103114962816466\\
574	0.000986381501988496\\
575	0.000941490255193976\\
576	0.000896512854870878\\
577	0.000851489130077175\\
578	0.000806461805374287\\
579	0.000761476486847549\\
580	0.000716581584683501\\
581	0.000671828152215815\\
582	0.000627269616206765\\
583	0.00058296136682537\\
584	0.000538960168207977\\
585	0.00049532334178407\\
586	0.000452107665645345\\
587	0.000409367927260634\\
588	0.000367155072958957\\
589	0.0003255139412327\\
590	0.000284480716949121\\
591	0.000244080684870836\\
592	0.00020432808725126\\
593	0.00016523321767671\\
594	0.000126870159937007\\
595	8.96111722547526e-05\\
596	5.4266094523851e-05\\
597	2.2806228433206e-05\\
598	2.9204464504877e-07\\
599	0\\
600	0\\
};
\end{axis}
\end{tikzpicture}% 
  \caption{Discrete Time w/ nFPC}
\end{subfigure}\\

\leavevmode\smash{\makebox[0pt]{\hspace{-7em}% HORIZONTAL POSITION           
  \rotatebox[origin=l]{90}{\hspace{20em}% VERTICAL POSITION
    Depth $\delta^-$}%
}}\hspace{0pt plus 1filll}\null

Time (s)

\vspace{1cm}
\begin{subfigure}{\linewidth}
  \centering
  \tikzsetnextfilename{altdeltalegend}
  \definecolor{mycolor1}{rgb}{0.00000,1.00000,0.14286}%
\definecolor{mycolor2}{rgb}{0.00000,1.00000,0.28571}%
\definecolor{mycolor3}{rgb}{0.00000,1.00000,0.42857}%
\definecolor{mycolor4}{rgb}{0.00000,1.00000,0.57143}%
\definecolor{mycolor5}{rgb}{0.00000,1.00000,0.71429}%
\definecolor{mycolor6}{rgb}{0.00000,1.00000,0.85714}%
\definecolor{mycolor7}{rgb}{0.00000,1.00000,1.00000}%
\definecolor{mycolor8}{rgb}{0.00000,0.87500,1.00000}%
\definecolor{mycolor9}{rgb}{0.00000,0.62500,1.00000}%
\definecolor{mycolor10}{rgb}{0.12500,0.00000,1.00000}%
\definecolor{mycolor11}{rgb}{0.25000,0.00000,1.00000}%
\definecolor{mycolor12}{rgb}{0.37500,0.00000,1.00000}%
\definecolor{mycolor13}{rgb}{0.50000,0.00000,1.00000}%
\definecolor{mycolor14}{rgb}{0.62500,0.00000,1.00000}%
\definecolor{mycolor15}{rgb}{0.75000,0.00000,1.00000}%
\definecolor{mycolor16}{rgb}{0.87500,0.00000,1.00000}%
\definecolor{mycolor17}{rgb}{1.00000,0.00000,1.00000}%
\definecolor{mycolor18}{rgb}{1.00000,0.00000,0.87500}%
\definecolor{mycolor19}{rgb}{1.00000,0.00000,0.62500}%
\definecolor{mycolor20}{rgb}{0.85714,0.00000,0.00000}%
\definecolor{mycolor21}{rgb}{0.71429,0.00000,0.00000}%
%[trim axis left, trim axis right]
\begin{tikzpicture}
\begin{axis}[%
    hide axis,
    scale only axis,
    height=0pt,
    width=0pt,
    point meta min=-19,
    point meta max=19,
    colormap={mymap}{[1pt] rgb(0pt)=(0,1,0); rgb(7pt)=(0,1,1); rgb(15pt)=(0,0,1); rgb(23pt)=(1,0,1); rgb(31pt)=(1,0,0); rgb(38pt)=(0,0,0)},
    colorbar horizontal,
    colorbar style={width=15cm,xtick={{-15},{-10},{-5},{0},{5},{10},{15}}}
    %colorbar style={separate axis lines,every outer x axis line/.append style={black},every x tick label/.append style={font=\color{black}},every outer y axis line/.append style={black},every y tick label/.append style={font=\color{black}},yticklabels={{-19},{-17},{-15},{-13},{-11},{-9},{-7},{-5},{-3},{-1},{1},{3},{5},{7},{9},{11},{13},{15},{17},{19}}}
]%
    \addplot [draw=none] coordinates {(0,0)};
\end{axis}
\end{tikzpicture}
 
\end{subfigure}%
  \caption{Optimal sell depths $\delta^-$ for Markov state $Z=(\rho = 0, \Delta S = 0)$, implying neutral imbalance and no previous price change. We expect no change in midprice.}
  \label{fig:comp_dm_z8}
\end{figure}

\begin{figure}
\centering
\begin{subfigure}{.45\linewidth}
  \centering
  \setlength\figureheight{\linewidth} 
  \setlength\figurewidth{\linewidth}
  \tikzsetnextfilename{dp_colorbar/dm_cts_z15}
  % This file was created by matlab2tikz.
%
%The latest updates can be retrieved from
%  http://www.mathworks.com/matlabcentral/fileexchange/22022-matlab2tikz-matlab2tikz
%where you can also make suggestions and rate matlab2tikz.
%
\definecolor{mycolor1}{rgb}{1.00000,0.00000,1.00000}%
%
\begin{tikzpicture}[trim axis left, trim axis right]

\begin{axis}[%
width=\figurewidth,
height=\figureheight,
at={(0\figurewidth,0\figureheight)},
scale only axis,
every outer x axis line/.append style={black},
every x tick label/.append style={font=\color{black}},
xmin=0,
xmax=100,
xlabel={Time},
every outer y axis line/.append style={black},
every y tick label/.append style={font=\color{black}},
ymin=0,
ymax=0.015,
ylabel={Depth $\delta^-$},
axis background/.style={fill=white},
axis x line*=bottom,
axis y line*=left,
yticklabel style={
        /pgf/number format/fixed,
        /pgf/number format/precision=3
},
scaled y ticks=false,
legend style={legend cell align=left,align=left,draw=black,font=\footnotesize, at={(0.98,0.02)},anchor=south east},
every axis legend/.code={\renewcommand\addlegendentry[2][]{}}  %ignore legend locally
]
\addplot [color=green,dashed]
  table[row sep=crcr]{%
0.01	0\\
1.01	0\\
2.01	0\\
3.01	0\\
4.01	0\\
5.01	0\\
6.01	0\\
7.01	0\\
8.01	0\\
9.01	0\\
10.01	0\\
11.01	0\\
12.01	0\\
13.01	0\\
14.01	0\\
15.01	0\\
16.01	0\\
17.01	0\\
18.01	0\\
19.01	0\\
20.01	0\\
21.01	0\\
22.01	0\\
23.01	0\\
24.01	0\\
25.01	0\\
26.01	0\\
27.01	0\\
28.01	0\\
29.01	0\\
30.01	0\\
31.01	0\\
32.01	0\\
33.01	0\\
34.01	0\\
35.01	0\\
36.01	0\\
37.01	0\\
38.01	0\\
39.01	0\\
40.01	0\\
41.01	0\\
42.01	0\\
43.01	0\\
44.01	0\\
45.01	0\\
46.01	0\\
47.01	0\\
48.01	0\\
49.01	0\\
50.01	0\\
51.01	0\\
52.01	0\\
53.01	0\\
54.01	0\\
55.01	0\\
56.01	0\\
57.01	0\\
58.01	0\\
59.01	0\\
60.01	0\\
61.01	0\\
62.01	0\\
63.01	0\\
64.01	0\\
65.01	0\\
66.01	0\\
67.01	0\\
68.01	0\\
69.01	0\\
70.01	0\\
71.01	0\\
72.01	0\\
73.01	0\\
74.01	0\\
75.01	0\\
76.01	0\\
77.01	0\\
78.01	0\\
79.01	0\\
80.01	0\\
81.01	0\\
82.01	0\\
83.01	0\\
84.01	0\\
85.01	0\\
86.01	0\\
87.01	0\\
88.01	0\\
89.01	0\\
90.01	0\\
91.01	0\\
92.01	0\\
93.01	0\\
94.01	0\\
95.01	0\\
96.01	0\\
97.01	0\\
98.01	0\\
99.01	0.00331175016347651\\
99.02	0.00335383459458577\\
99.03	0.00339629425946782\\
99.04	0.00343913299375206\\
99.05	0.00348235448270438\\
99.06	0.00352596247377274\\
99.07	0.00356996078207136\\
99.08	0.0036143532932759\\
99.09	0.003659143966672\\
99.1	0.00370433683836476\\
99.11	0.00374993602465781\\
99.12	0.00379594572561081\\
99.13	0.00384237022878453\\
99.14	0.00388921386374265\\
99.15	0.00393648045349505\\
99.16	0.00398417385589169\\
99.17	0.00403229796388142\\
99.18	0.00408085670576604\\
99.19	0.00412985404544918\\
99.2	0.00417929398267937\\
99.21	0.00422918055328676\\
99.22	0.00427951782957623\\
99.23	0.00433030992114294\\
99.24	0.00438156097518348\\
99.25	0.00443327517680532\\
99.26	0.00448545674933452\\
99.27	0.00453810995462083\\
99.28	0.00459123909334014\\
99.29	0.00464484850529341\\
99.3	0.00469894256970183\\
99.31	0.00475352570549745\\
99.32	0.00480860237160862\\
99.33	0.00486417706723978\\
99.34	0.00492025433214455\\
99.35	0.00497683874689156\\
99.36	0.00503393493312218\\
99.37	0.00509154755379893\\
99.38	0.00514968131344361\\
99.39	0.0052083409583641\\
99.4	0.00526753127686883\\
99.41	0.00532725709946772\\
99.42	0.00538752330062317\\
99.43	0.00544833479915421\\
99.44	0.0055096965586415\\
99.45	0.00557161358783625\\
99.46	0.00563409094107324\\
99.47	0.00569713371868781\\
99.48	0.0057607470674359\\
99.49	0.00582493618091646\\
99.5	0.0058897062999978\\
99.51	0.00595506271324757\\
99.52	0.00602101075736671\\
99.53	0.00608755581762727\\
99.54	0.00615470332831403\\
99.55	0.00622245877317017\\
99.56	0.00629082768584681\\
99.57	0.00635981565035658\\
99.58	0.00642942830153121\\
99.59	0.00649967132548313\\
99.6	0.00657055046007118\\
99.61	0.00664207149537041\\
99.62	0.00671424027414604\\
99.63	0.00678706269233151\\
99.64	0.00686054469951087\\
99.65	0.0069346922994052\\
99.66	0.00700951155036349\\
99.67	0.00708500856585769\\
99.68	0.00716118951498209\\
99.69	0.00723806060688493\\
99.7	0.00731562810494498\\
99.71	0.00739389832935691\\
99.72	0.007472877657644\\
99.73	0.00755257252517513\\
99.74	0.0076329894256864\\
99.75	0.00771413491180722\\
99.76	0.00779601559559102\\
99.77	0.00787863814905056\\
99.78	0.00796200930469796\\
99.79	0.00804613585608947\\
99.8	0.00813102465837497\\
99.81	0.00821668262885248\\
99.82	0.00830311674752747\\
99.83	0.00839033405767729\\
99.84	0.00847834166642063\\
99.85	0.00856714674529227\\
99.86	0.00865675653082306\\
99.87	0.00874717832512533\\
99.88	0.00883841949648396\\
99.89	0.00893048747995297\\
99.9	0.00902338977795818\\
99.91	0.00911713396090572\\
99.92	0.0092117276677969\\
99.93	0.00930717860684955\\
99.94	0.009403494556126\\
99.95	0.00950068336416817\\
99.96	0.00959875295063989\\
99.97	0.009697711306977\\
99.98	0.00979756649704546\\
99.99	0.00989832665780802\\
100	0.01\\
};
\addlegendentry{$q=-4$};

\addplot [color=mycolor1,dashed]
  table[row sep=crcr]{%
0.01	0\\
1.01	0\\
2.01	0\\
3.01	0\\
4.01	0\\
5.01	0\\
6.01	0\\
7.01	0\\
8.01	0\\
9.01	0\\
10.01	0\\
11.01	0\\
12.01	0\\
13.01	0\\
14.01	0\\
15.01	0\\
16.01	0\\
17.01	0\\
18.01	0\\
19.01	0\\
20.01	0\\
21.01	0\\
22.01	0\\
23.01	0\\
24.01	0\\
25.01	0\\
26.01	0\\
27.01	0\\
28.01	0\\
29.01	0\\
30.01	0\\
31.01	0\\
32.01	0\\
33.01	0\\
34.01	0\\
35.01	0\\
36.01	0\\
37.01	0\\
38.01	0\\
39.01	0\\
40.01	0\\
41.01	0\\
42.01	0\\
43.01	0\\
44.01	0\\
45.01	0\\
46.01	0\\
47.01	0\\
48.01	0\\
49.01	0\\
50.01	0\\
51.01	0\\
52.01	0\\
53.01	0\\
54.01	0\\
55.01	0\\
56.01	0\\
57.01	0\\
58.01	0\\
59.01	0\\
60.01	0\\
61.01	0\\
62.01	0\\
63.01	0\\
64.01	0\\
65.01	0\\
66.01	0\\
67.01	0\\
68.01	0\\
69.01	0\\
70.01	0\\
71.01	0\\
72.01	0\\
73.01	0\\
74.01	0\\
75.01	0\\
76.01	0\\
77.01	0\\
78.01	0\\
79.01	0\\
80.01	0\\
81.01	0\\
82.01	0\\
83.01	0\\
84.01	0\\
85.01	0\\
86.01	0\\
87.01	0\\
88.01	0\\
89.01	0\\
90.01	0\\
91.01	0\\
92.01	0\\
93.01	0\\
94.01	0\\
95.01	0\\
96.01	0\\
97.01	0\\
98.01	0\\
99.01	0.00328845890704418\\
99.02	0.00333351020433462\\
99.03	0.00337872989622107\\
99.04	0.00342410781940597\\
99.05	0.00346969067490262\\
99.06	0.00351546976428445\\
99.07	0.00356143509901703\\
99.08	0.00360757607931048\\
99.09	0.00365388146525549\\
99.1	0.00370033934650487\\
99.11	0.00374693711037871\\
99.12	0.00379366140830112\\
99.13	0.00384049812047099\\
99.14	0.0038874580034141\\
99.15	0.00393483565423353\\
99.16	0.00398263494425564\\
99.17	0.00403085980130233\\
99.18	0.0040795142118173\\
99.19	0.00412860222311627\\
99.2	0.00417812794569628\\
99.21	0.00422809555566403\\
99.22	0.00427850924718199\\
99.23	0.00432937310681449\\
99.24	0.0043806912758079\\
99.25	0.00443246795197619\\
99.26	0.00448470739168515\\
99.27	0.00453741391194087\\
99.28	0.00459059189258777\\
99.29	0.0046442457786222\\
99.3	0.00469838008262783\\
99.31	0.00475299938733924\\
99.32	0.00480810834834083\\
99.33	0.00486371169690803\\
99.34	0.00491981424299882\\
99.35	0.0049764208784027\\
99.36	0.00503353658005642\\
99.37	0.00509116641355064\\
99.38	0.00514931553692431\\
99.39	0.00520798920455344\\
99.4	0.00526719277124689\\
99.41	0.00532693169656051\\
99.42	0.00538721084840616\\
99.43	0.00544803513735539\\
99.44	0.00550940951836646\\
99.45	0.00557133899114194\\
99.46	0.0056338286004846\\
99.47	0.00569688343665094\\
99.48	0.0057605086360643\\
99.49	0.00582470938215867\\
99.5	0.00588949090583306\\
99.51	0.00595485848591075\\
99.52	0.00602081744960348\\
99.53	0.00608737317298039\\
99.54	0.00615453108144196\\
99.55	0.00622229665019873\\
99.56	0.00629067540475496\\
99.57	0.00635967292139696\\
99.58	0.00642929482768636\\
99.59	0.0064995468029579\\
99.6	0.00657043457882205\\
99.61	0.00664196393967204\\
99.62	0.0067141407231954\\
99.63	0.00678697082088982\\
99.64	0.00686046017858321\\
99.65	0.00693461479695774\\
99.66	0.0070094407320778\\
99.67	0.00708494409592144\\
99.68	0.00716113105691529\\
99.69	0.00723800782441484\\
99.7	0.00731558066288144\\
99.71	0.0073938558945165\\
99.72	0.0074728398998023\\
99.73	0.0075525391180442\\
99.74	0.00763296004791393\\
99.75	0.0077141092479932\\
99.76	0.00779599333731718\\
99.77	0.00787861899591701\\
99.78	0.00796199296536071\\
99.79	0.00804612204929144\\
99.8	0.00813101311396239\\
99.81	0.00821667308876694\\
99.82	0.00830310896676325\\
99.83	0.00839032780519167\\
99.84	0.00847833672598379\\
99.85	0.00856714291626138\\
99.86	0.00865675362882351\\
99.87	0.00874717618262004\\
99.88	0.00883841796320915\\
99.89	0.00893048642319673\\
99.9	0.00902338908265498\\
99.91	0.00911713352951744\\
99.92	0.00921172741994723\\
99.93	0.00930717847867511\\
99.94	0.00940349449930366\\
99.95	0.00950068334457319\\
99.96	0.00959875294658504\\
99.97	0.009697711306977\\
99.98	0.00979756649704546\\
99.99	0.00989832665780802\\
100	0.01\\
};
\addlegendentry{$q=-3$};

\addplot [color=red,dashed]
  table[row sep=crcr]{%
0.01	0\\
1.01	0\\
2.01	0\\
3.01	0\\
4.01	0\\
5.01	0\\
6.01	0\\
7.01	0\\
8.01	0\\
9.01	0\\
10.01	0\\
11.01	0\\
12.01	0\\
13.01	0\\
14.01	0\\
15.01	0\\
16.01	0\\
17.01	0\\
18.01	0\\
19.01	0\\
20.01	0\\
21.01	0\\
22.01	0\\
23.01	0\\
24.01	0\\
25.01	0\\
26.01	0\\
27.01	0\\
28.01	0\\
29.01	0\\
30.01	0\\
31.01	0\\
32.01	0\\
33.01	0\\
34.01	0\\
35.01	0\\
36.01	0\\
37.01	0\\
38.01	0\\
39.01	0\\
40.01	0\\
41.01	0\\
42.01	0\\
43.01	0\\
44.01	0\\
45.01	0\\
46.01	0\\
47.01	0\\
48.01	0\\
49.01	0\\
50.01	0\\
51.01	0\\
52.01	0\\
53.01	0\\
54.01	0\\
55.01	0\\
56.01	0\\
57.01	0\\
58.01	0\\
59.01	0\\
60.01	0\\
61.01	0\\
62.01	0\\
63.01	0\\
64.01	0\\
65.01	0\\
66.01	0\\
67.01	0\\
68.01	0\\
69.01	0\\
70.01	0\\
71.01	0\\
72.01	0\\
73.01	0\\
74.01	0\\
75.01	0\\
76.01	0\\
77.01	0\\
78.01	0\\
79.01	0\\
80.01	0\\
81.01	0\\
82.01	0\\
83.01	0\\
84.01	0\\
85.01	0\\
86.01	0\\
87.01	0\\
88.01	0\\
89.01	0\\
90.01	0\\
91.01	0\\
92.01	0\\
93.01	0\\
94.01	0\\
95.01	0\\
96.01	0\\
97.01	0\\
98.01	0\\
99.01	0.00104946028175264\\
99.02	0.00125306395411695\\
99.03	0.00145818452931908\\
99.04	0.0016648453630583\\
99.05	0.00187301328746817\\
99.06	0.00208271075060248\\
99.07	0.00229396160842808\\
99.08	0.00250679044560442\\
99.09	0.00272122260386963\\
99.1	0.002937284219982\\
99.11	0.0031550022516486\\
99.12	0.00337440450742363\\
99.13	0.00359551968105016\\
99.14	0.00380493250998101\\
99.15	0.00385656934463024\\
99.16	0.00390855237475965\\
99.17	0.00396087846322035\\
99.18	0.00401354417880869\\
99.19	0.00406654578056501\\
99.2	0.00411987921666678\\
99.21	0.00417354011061869\\
99.22	0.00422753536402045\\
99.23	0.00428190027022628\\
99.24	0.00433663114616696\\
99.25	0.0043917239783746\\
99.26	0.00444717440792617\\
99.27	0.00450297771469116\\
99.28	0.0045591288008455\\
99.29	0.00461562217361109\\
99.3	0.00467245192717746\\
99.31	0.00472961172375959\\
99.32	0.0047870947737426\\
99.33	0.0048448938148608\\
99.34	0.00490300109035484\\
99.35	0.00496140832634903\\
99.36	0.00502010670781841\\
99.37	0.00507908684954088\\
99.38	0.0051383387442909\\
99.39	0.00519785176102771\\
99.4	0.00525761461537342\\
99.41	0.00531761533853545\\
99.42	0.00537815543682032\\
99.43	0.00543924041152297\\
99.44	0.00550087518071841\\
99.45	0.00556306471542116\\
99.46	0.0056258140408721\\
99.47	0.00568912823785012\\
99.48	0.00575301234814035\\
99.49	0.00581747134619418\\
99.5	0.00588251024853272\\
99.51	0.00594813411434026\\
99.52	0.00601434804608568\\
99.53	0.00608115719017392\\
99.54	0.00614856673762911\\
99.55	0.00621658192481169\\
99.56	0.00628520803417149\\
99.57	0.00635445039503925\\
99.58	0.00642431438445891\\
99.59	0.00649480542806346\\
99.6	0.00656592900099709\\
99.61	0.00663769062888668\\
99.62	0.00671009588886575\\
99.63	0.00678315041065447\\
99.64	0.00685685987769925\\
99.65	0.00693123002837576\\
99.66	0.00700626665725987\\
99.67	0.00708197561647047\\
99.68	0.00715836281708942\\
99.69	0.00723543421531813\\
99.7	0.00731319582612591\\
99.71	0.00739165372783235\\
99.72	0.00747081406389988\\
99.73	0.00755068304485562\\
99.74	0.00763126695028889\\
99.75	0.00771257213096943\\
99.76	0.00779460501109486\\
99.77	0.00787737209067585\\
99.78	0.00796087994806859\\
99.79	0.0080451352426644\\
99.8	0.00813014471774746\\
99.81	0.00821591520353212\\
99.82	0.00830245362039213\\
99.83	0.00838976698229511\\
99.84	0.00847786240045663\\
99.85	0.008566747087229\\
99.86	0.00865642836024134\\
99.87	0.00874691364680851\\
99.88	0.0088382104886279\\
99.89	0.00893032654678435\\
99.9	0.0090232696070851\\
99.91	0.00911704758574675\\
99.92	0.00921166853545841\\
99.93	0.0093071406518502\\
99.94	0.009403472280396\\
99.95	0.00950067192378203\\
99.96	0.00959874824977478\\
99.97	0.00969771009962486\\
99.98	0.00979756649704546\\
99.99	0.00989832665780802\\
100	0.01\\
};
\addlegendentry{$q=-2$};

\addplot [color=blue,dashed]
  table[row sep=crcr]{%
0.01	0\\
1.01	0\\
2.01	0\\
3.01	0\\
4.01	0\\
5.01	0\\
6.01	0\\
7.01	0\\
8.01	0\\
9.01	0\\
10.01	0\\
11.01	0\\
12.01	0\\
13.01	0\\
14.01	0\\
15.01	0\\
16.01	0\\
17.01	0\\
18.01	0\\
19.01	0\\
20.01	0\\
21.01	0\\
22.01	0\\
23.01	0\\
24.01	0\\
25.01	0\\
26.01	0\\
27.01	0\\
28.01	0\\
29.01	0\\
30.01	0\\
31.01	0\\
32.01	0\\
33.01	0\\
34.01	0\\
35.01	0\\
36.01	0\\
37.01	0\\
38.01	0\\
39.01	0\\
40.01	0\\
41.01	0\\
42.01	0\\
43.01	0\\
44.01	0\\
45.01	0\\
46.01	0\\
47.01	0\\
48.01	0\\
49.01	0\\
50.01	0\\
51.01	0\\
52.01	0\\
53.01	0\\
54.01	0\\
55.01	0\\
56.01	0\\
57.01	0\\
58.01	0\\
59.01	0\\
60.01	0\\
61.01	0\\
62.01	0\\
63.01	0\\
64.01	0\\
65.01	0\\
66.01	0\\
67.01	0\\
68.01	0\\
69.01	0\\
70.01	0\\
71.01	0\\
72.01	0\\
73.01	0\\
74.01	0\\
75.01	0\\
76.01	0\\
77.01	0\\
78.01	0\\
79.01	0\\
80.01	0\\
81.01	0\\
82.01	0\\
83.01	0\\
84.01	0\\
85.01	0\\
86.01	0\\
87.01	0\\
88.01	0\\
89.01	0\\
90.01	0\\
91.01	0\\
92.01	0\\
93.01	0\\
94.01	0\\
95.01	0\\
96.01	0\\
97.01	0\\
98.01	0\\
99.01	0\\
99.02	0\\
99.03	0\\
99.04	0\\
99.05	0\\
99.06	0\\
99.07	0\\
99.08	0\\
99.09	0\\
99.1	0\\
99.11	0\\
99.12	0\\
99.13	0\\
99.14	1.34192545933262e-05\\
99.15	0.000186052377303365\\
99.16	0.000359788996779725\\
99.17	0.000534644183888877\\
99.18	0.000710633127091898\\
99.19	0.000887771451375849\\
99.2	0.00106607520955938\\
99.21	0.00124556089630498\\
99.22	0.00142623390412353\\
99.23	0.00160807155888041\\
99.24	0.00179109029889443\\
99.25	0.00197530702975846\\
99.26	0.00216073914018688\\
99.27	0.00234740451849972\\
99.28	0.00253532156977716\\
99.29	0.00272450923372016\\
99.3	0.00291498700325548\\
99.31	0.00310677494392576\\
99.32	0.00329989371410826\\
99.33	0.00349436458610888\\
99.34	0.00369020946818107\\
99.35	0.00388745077047987\\
99.36	0.00408611158292441\\
99.37	0.00428621572535727\\
99.38	0.00448778793606267\\
99.39	0.00469085371780874\\
99.4	0.00489543938389655\\
99.41	0.00510157207241651\\
99.42	0.00516645903340951\\
99.43	0.00523160273235865\\
99.44	0.00529730533150437\\
99.45	0.0053635695660718\\
99.46	0.00543039808954691\\
99.47	0.00549779347615066\\
99.48	0.00556577769541155\\
99.49	0.0056343774394848\\
99.5	0.0057035963880993\\
99.51	0.00577343816246014\\
99.52	0.00584390632119317\\
99.53	0.00591500435610443\\
99.54	0.00598673568774476\\
99.55	0.00605910366076959\\
99.56	0.0061321115390831\\
99.57	0.00620576250075546\\
99.58	0.00628005963270098\\
99.59	0.00635500592510444\\
99.6	0.00643060426558196\\
99.61	0.00650685743306189\\
99.62	0.00658376809137045\\
99.63	0.00666133878250572\\
99.64	0.00673957191958258\\
99.65	0.00681846977943027\\
99.66	0.00689803449482267\\
99.67	0.00697826804632059\\
99.68	0.00705917225370364\\
99.69	0.00714074876711371\\
99.7	0.00722299905762269\\
99.71	0.00730592440728626\\
99.72	0.00738952589167773\\
99.73	0.00747380436075851\\
99.74	0.00755876044180094\\
99.75	0.00764439452717997\\
99.76	0.00773070676151409\\
99.77	0.00781769702811535\\
99.78	0.007905364934706\\
99.79	0.00799370979835622\\
99.8	0.00808273062959414\\
99.81	0.008172426115636\\
99.82	0.0082627946026807\\
99.83	0.00835383407720883\\
99.84	0.00844554214622215\\
99.85	0.00853791601635481\\
99.86	0.00863095247178254\\
99.87	0.00872464785085084\\
99.88	0.00881899802133705\\
99.89	0.00891399835425527\\
99.9	0.0090096436961058\\
99.91	0.00910592833996096\\
99.92	0.00920284599502165\\
99.93	0.0093003897536808\\
99.94	0.00939855205657116\\
99.95	0.00949732465545663\\
99.96	0.00959669857381528\\
99.97	0.00969666406494989\\
99.98	0.00979721056744917\\
99.99	0.00989832665780802\\
100	0.01\\
};
\addlegendentry{$q=-1$};

\addplot [color=black,solid]
  table[row sep=crcr]{%
0.01	0\\
1.01	0\\
2.01	0\\
3.01	0\\
4.01	0\\
5.01	0\\
6.01	0\\
7.01	0\\
8.01	0\\
9.01	0\\
10.01	0\\
11.01	0\\
12.01	0\\
13.01	0\\
14.01	0\\
15.01	0\\
16.01	0\\
17.01	0\\
18.01	0\\
19.01	0\\
20.01	0\\
21.01	0\\
22.01	0\\
23.01	0\\
24.01	0\\
25.01	0\\
26.01	0\\
27.01	0\\
28.01	0\\
29.01	0\\
30.01	0\\
31.01	0\\
32.01	0\\
33.01	0\\
34.01	0\\
35.01	0\\
36.01	0\\
37.01	0\\
38.01	0\\
39.01	0\\
40.01	0\\
41.01	0\\
42.01	0\\
43.01	0\\
44.01	0\\
45.01	0\\
46.01	0\\
47.01	0\\
48.01	0\\
49.01	0\\
50.01	0\\
51.01	0\\
52.01	0\\
53.01	0\\
54.01	0\\
55.01	0\\
56.01	0\\
57.01	0\\
58.01	0\\
59.01	0\\
60.01	0\\
61.01	0\\
62.01	0\\
63.01	0\\
64.01	0\\
65.01	0\\
66.01	0\\
67.01	0\\
68.01	0\\
69.01	0\\
70.01	0\\
71.01	0\\
72.01	0\\
73.01	0\\
74.01	0\\
75.01	0\\
76.01	0\\
77.01	0\\
78.01	0\\
79.01	0\\
80.01	0\\
81.01	0\\
82.01	0\\
83.01	0\\
84.01	0\\
85.01	0\\
86.01	0\\
87.01	0\\
88.01	0\\
89.01	0\\
90.01	0\\
91.01	0\\
92.01	0\\
93.01	0\\
94.01	0\\
95.01	0\\
96.01	0\\
97.01	0\\
98.01	0\\
99.01	0\\
99.02	0\\
99.03	0\\
99.04	0\\
99.05	0\\
99.06	0\\
99.07	0\\
99.08	0\\
99.09	0\\
99.1	0\\
99.11	0\\
99.12	0\\
99.13	0\\
99.14	0\\
99.15	0\\
99.16	0\\
99.17	0\\
99.18	0\\
99.19	0\\
99.2	0\\
99.21	0\\
99.22	0\\
99.23	0\\
99.24	0\\
99.25	0\\
99.26	0\\
99.27	0\\
99.28	0\\
99.29	0\\
99.3	0\\
99.31	0\\
99.32	0\\
99.33	0\\
99.34	0\\
99.35	0\\
99.36	0\\
99.37	0\\
99.38	0\\
99.39	0\\
99.4	0\\
99.41	0\\
99.42	0.000142507484711799\\
99.43	0.00028603094158923\\
99.44	0.000430279959184088\\
99.45	0.000575263732640401\\
99.46	0.000720991715020906\\
99.47	0.000867473608317394\\
99.48	0.00101470001362642\\
99.49	0.00116265714149537\\
99.5	0.00131135440874501\\
99.51	0.00146080148760354\\
99.52	0.00161100831282696\\
99.53	0.00176198508903214\\
99.54	0.00191374229825145\\
99.55	0.00206629070771789\\
99.56	0.00221964137789047\\
99.57	0.00237380567073002\\
99.58	0.00252879525823614\\
99.59	0.00268462213125676\\
99.6	0.00284129860858223\\
99.61	0.00299883734633685\\
99.62	0.00315725134768115\\
99.63	0.00331655397283938\\
99.64	0.00347675894946723\\
99.65	0.00363788038337595\\
99.66	0.00379993276962971\\
99.67	0.00396293100403452\\
99.68	0.0041268903950377\\
99.69	0.00429182667605687\\
99.7	0.00445775601826144\\
99.71	0.00462469504382956\\
99.72	0.00479266083977007\\
99.73	0.00496167097234289\\
99.74	0.00513174350189687\\
99.75	0.00530289699829066\\
99.76	0.00547515055692814\\
99.77	0.00564852381544173\\
99.78	0.00582303697105938\\
99.79	0.00599871079869321\\
99.8	0.00617556666979046\\
99.81	0.00635362657199012\\
99.82	0.00653291312963158\\
99.83	0.00671344962516493\\
99.84	0.00689526002151581\\
99.85	0.00707836898546157\\
99.86	0.00726280191207965\\
99.87	0.00744858495033297\\
99.88	0.00763574502986251\\
99.89	0.0078243098890616\\
99.9	0.00801430810451265\\
99.91	0.0082057689535157\\
99.92	0.00839872238447791\\
99.93	0.00859319927334343\\
99.94	0.00878923145914393\\
99.95	0.00898685178139854\\
99.96	0.0091860941194868\\
99.97	0.0093869934341281\\
99.98	0.00958958581111154\\
99.99	0.0097939085074319\\
100	0.01\\
};
\addlegendentry{$q=0$};

\addplot [color=blue,solid]
  table[row sep=crcr]{%
0.01	0.00922742732202166\\
1.01	0.0092274130399497\\
2.01	0.00922739792741873\\
3.01	0.00922738222131146\\
4.01	0.00922736589596059\\
5.01	0.00922734892435787\\
6.01	0.00922733127806274\\
7.01	0.00922731292710504\\
8.01	0.00922729383988779\\
9.01	0.00922727398311702\\
10.01	0.00922725332186595\\
11.01	0.00922723182015138\\
12.01	0.0092272094430556\\
13.01	0.00922718616149337\\
14.01	0.00922716195212826\\
15.01	0.00922713676520024\\
16.01	0.00922711052046818\\
17.01	0.0092270831603759\\
18.01	0.00922705462948393\\
19.01	0.00922702486907789\\
20.01	0.00922699381699539\\
21.01	0.00922696140742542\\
22.01	0.0092269275705294\\
23.01	0.00922689223118527\\
24.01	0.00922685530378333\\
25.01	0.00922681666996403\\
26.01	0.00922677608529397\\
27.01	0.00922673280502235\\
28.01	0.00922668419077714\\
29.01	0.00922662106269991\\
30.01	0.00922651492566772\\
31.01	0.00922633446516904\\
32.01	0.00922613612752173\\
33.01	0.00922592990882459\\
34.01	0.00922571540224445\\
35.01	0.0092254921680819\\
36.01	0.00922525972966405\\
37.01	0.00922501756860596\\
38.01	0.00922476511943845\\
39.01	0.00922450176378566\\
40.01	0.00922422682443228\\
41.01	0.00922393955799911\\
42.01	0.00922363913562648\\
43.01	0.00922332459722462\\
44.01	0.00922299485833425\\
45.01	0.00922264875532557\\
46.01	0.00922228502688403\\
47.01	0.00922190241754074\\
48.01	0.00922150004276925\\
49.01	0.00922107782147045\\
50.01	0.00922063429811563\\
51.01	0.009220163730359\\
52.01	0.00921966145111034\\
53.01	0.00921912045848829\\
54.01	0.00921852278350417\\
55.01	0.00921781276658482\\
56.01	0.00921683754706864\\
57.01	0.00921537198395175\\
58.01	0.00921366886788504\\
59.01	0.00921188506743619\\
60.01	0.00921001457423623\\
61.01	0.00920805067758686\\
62.01	0.00920598582982146\\
63.01	0.00920381142141993\\
64.01	0.00920151730993209\\
65.01	0.00919909069106331\\
66.01	0.00919651353506634\\
67.01	0.00919376018707084\\
68.01	0.0091908105038698\\
69.01	0.00918760132951691\\
70.01	0.00918388721130242\\
71.01	0.00917883389870093\\
72.01	0.00917123417438804\\
73.01	0.00916016398907375\\
74.01	0.00914478415822474\\
75.01	0.0091278394184736\\
76.01	0.00910898099118535\\
77.01	0.0090863842332633\\
78.01	0.00905202319190744\\
79.01	0.00901183623729243\\
80.01	0.00896862825047214\\
81.01	0.00891994859362619\\
82.01	0.00886281994865658\\
83.01	0.00879447242289795\\
84.01	0.00869084382124653\\
85.01	0.00857162949842379\\
86.01	0.00844580465448999\\
87.01	0.00831061982868391\\
88.01	0.00815379040255077\\
89.01	0.0079242568210914\\
90.01	0.00766057592530038\\
91.01	0.00738581406637025\\
92.01	0.00709892259758851\\
93.01	0.00679870494687675\\
94.01	0.00648388363909919\\
95.01	0.00615365911855247\\
96.01	0.00580990015028039\\
97.01	0.00544683142545227\\
98.01	0.00501480028686344\\
99.01	0.00417782896024826\\
99.02	0.00416297391082926\\
99.03	0.00414787847786715\\
99.04	0.00413253714368365\\
99.05	0.00411694425386603\\
99.06	0.00410109401350807\\
99.07	0.00408498048333106\\
99.08	0.00406859757568049\\
99.09	0.00405193905039347\\
99.1	0.0040349985105342\\
99.11	0.00401776939798825\\
99.12	0.00400024498891175\\
99.13	0.00398241838903006\\
99.14	0.0039642825287796\\
99.15	0.00394583015828669\\
99.16	0.00392705384217749\\
99.17	0.00390794595421406\\
99.18	0.00388849867174659\\
99.19	0.00386870396997163\\
99.2	0.00384855361599089\\
99.21	0.00382803916266256\\
99.22	0.003807151942236\\
99.23	0.00378588305976082\\
99.24	0.0037642233862604\\
99.25	0.0037421635516597\\
99.26	0.00371969393745665\\
99.27	0.0036968046691257\\
99.28	0.00367348560824153\\
99.29	0.0036497263443105\\
99.3	0.00362551618629647\\
99.31	0.00360084413655559\\
99.32	0.00357569888582327\\
99.33	0.00355006881594236\\
99.34	0.00352394198998781\\
99.35	0.00349730614200647\\
99.36	0.00347014866635337\\
99.37	0.00344245660660522\\
99.38	0.00341425888549942\\
99.39	0.0033855659980539\\
99.4	0.00335636581065441\\
99.41	0.00332664584999986\\
99.42	0.00329639329011699\\
99.43	0.00326559492261193\\
99.44	0.00323425675308237\\
99.45	0.00320238362600423\\
99.46	0.0031699622610237\\
99.47	0.00313697899702963\\
99.48	0.00310341977913769\\
99.49	0.00306927014512682\\
99.5	0.00303451521130009\\
99.51	0.00299913965773969\\
99.52	0.00296312771292437\\
99.53	0.00292646313767548\\
99.54	0.00288912920839561\\
99.55	0.00285110869956154\\
99.56	0.00281238386543081\\
99.57	0.00277293642091844\\
99.58	0.00273274752159758\\
99.59	0.00269179774277483\\
99.6	0.00265006705758736\\
99.61	0.00260753481406595\\
99.62	0.00256417971110353\\
99.63	0.0025199797732653\\
99.64	0.00247491232437151\\
99.65	0.0024289539597794\\
99.66	0.00238208051728548\\
99.67	0.00233426704656365\\
99.68	0.00228548777704836\\
99.69	0.00223571608416403\\
99.7	0.00218492445379856\\
99.71	0.00213308444490778\\
99.72	0.00208016665012936\\
99.73	0.00202614065427561\\
99.74	0.00197097499056455\\
99.75	0.00191463709443744\\
99.76	0.00185709325479893\\
99.77	0.00179830856250254\\
99.78	0.00173824685588996\\
99.79	0.00167687066317651\\
99.8	0.00161414114145789\\
99.81	0.00155001801209408\\
99.82	0.00148445949220548\\
99.83	0.00141742222199301\\
99.84	0.00134886118756874\\
99.85	0.00127872963895532\\
99.86	0.00120697900288187\\
99.87	0.00113355878996945\\
99.88	0.00105841649586151\\
99.89	0.000981497495812871\\
99.9	0.000902744932204142\\
99.91	0.000822099594396612\\
99.92	0.000739499790285361\\
99.93	0.000654881208843966\\
99.94	0.000568176772882604\\
99.95	0.000479316481161323\\
99.96	0.000388227238910565\\
99.97	0.000294832675710451\\
99.98	0.000199052949567459\\
99.99	0.000100804535899853\\
100	0\\
};
\addlegendentry{$q=1$};

\addplot [color=red,solid]
  table[row sep=crcr]{%
0.01	0.00402362825950001\\
1.01	0.00402351587962618\\
2.01	0.00402339029379895\\
3.01	0.00402325978601299\\
4.01	0.00402312414476495\\
5.01	0.00402298314765832\\
6.01	0.00402283656067248\\
7.01	0.00402268413737226\\
8.01	0.00402252561806117\\
9.01	0.00402236072891558\\
10.01	0.00402218918130391\\
11.01	0.00402201067232423\\
12.01	0.00402182489147851\\
13.01	0.00402163155271235\\
14.01	0.004021430481385\\
15.01	0.00402122150854012\\
16.01	0.00402100390779924\\
17.01	0.00402077711062517\\
18.01	0.00402054066519825\\
19.01	0.00402029409346165\\
20.01	0.00402003688971778\\
21.01	0.00401976851939623\\
22.01	0.004019488417831\\
23.01	0.00401919598837888\\
24.01	0.00401889059640013\\
25.01	0.00401857154232075\\
26.01	0.0040182379328601\\
27.01	0.00401788805657958\\
28.01	0.00401751641407862\\
29.01	0.00401710090792436\\
30.01	0.00401654043053287\\
31.01	0.00401524077524837\\
32.01	0.00401356328581714\\
33.01	0.00401181919838981\\
34.01	0.00401000510163408\\
35.01	0.00400811731348965\\
36.01	0.00400615184786791\\
37.01	0.00400410437593584\\
38.01	0.00400197018113882\\
39.01	0.00399974410753812\\
40.01	0.0039974205029865\\
41.01	0.00399499316308984\\
42.01	0.00399245526468203\\
43.01	0.00398979908676834\\
44.01	0.00398701559597362\\
45.01	0.00398409504346138\\
46.01	0.0039810266773342\\
47.01	0.00397779871373349\\
48.01	0.00397439986095259\\
49.01	0.0039708263726504\\
50.01	0.003967087031578\\
51.01	0.00396313903742749\\
52.01	0.00395893722306839\\
53.01	0.00395444771955571\\
54.01	0.00394961335081805\\
55.01	0.00394427121862288\\
56.01	0.0039377329305626\\
57.01	0.00392734508317033\\
58.01	0.00391272863316338\\
59.01	0.00389740460488283\\
60.01	0.00388131958531887\\
61.01	0.00386441396860172\\
62.01	0.00384662097911993\\
63.01	0.00382786546168491\\
64.01	0.00380806233451378\\
65.01	0.00378711476944264\\
66.01	0.00376491376922966\\
67.01	0.0037413440497164\\
68.01	0.00371626118832227\\
69.01	0.00368942935742867\\
70.01	0.00366029305354543\\
71.01	0.00362606065237299\\
72.01	0.00357017100527519\\
73.01	0.0035039506031637\\
74.01	0.00343495441141227\\
75.01	0.00336092299662772\\
76.01	0.00328298424162559\\
77.01	0.00320051274712849\\
78.01	0.00311921445800905\\
79.01	0.00304018641402235\\
80.01	0.00295782854570667\\
81.01	0.00286161099820042\\
82.01	0.00270468057325484\\
83.01	0.0025219402204106\\
84.01	0.00234611554577139\\
85.01	0.00217116590119847\\
86.01	0.00198846164238502\\
87.01	0.00179611854344243\\
88.01	0.00159085725273375\\
89.01	0.00140325190254498\\
90.01	0.00122620645439562\\
91.01	0.00104124992940481\\
92.01	0.00084792810317127\\
93.01	0.000645779928157975\\
94.01	0.000434474210155834\\
95.01	0.000215534575809176\\
96.01	1.96404939382718e-05\\
97.01	0\\
98.01	1.73472347597681e-18\\
99.01	0\\
99.02	0\\
99.03	0\\
99.04	0\\
99.05	1.73472347597681e-18\\
99.06	0\\
99.07	0\\
99.08	1.73472347597681e-18\\
99.09	0\\
99.1	0\\
99.11	1.73472347597681e-18\\
99.12	0\\
99.13	0\\
99.14	0\\
99.15	0\\
99.16	0\\
99.17	1.73472347597681e-18\\
99.18	0\\
99.19	0\\
99.2	0\\
99.21	0\\
99.22	0\\
99.23	0\\
99.24	1.73472347597681e-18\\
99.25	0\\
99.26	0\\
99.27	1.73472347597681e-18\\
99.28	0\\
99.29	0\\
99.3	0\\
99.31	0\\
99.32	0\\
99.33	0\\
99.34	0\\
99.35	0\\
99.36	0\\
99.37	0\\
99.38	0\\
99.39	0\\
99.4	0\\
99.41	0\\
99.42	0\\
99.43	0\\
99.44	0\\
99.45	0\\
99.46	0\\
99.47	0\\
99.48	0\\
99.49	0\\
99.5	0\\
99.51	1.73472347597681e-18\\
99.52	0\\
99.53	0\\
99.54	0\\
99.55	0\\
99.56	0\\
99.57	0\\
99.58	0\\
99.59	0\\
99.6	0\\
99.61	1.73472347597681e-18\\
99.62	0\\
99.63	0\\
99.64	0\\
99.65	1.73472347597681e-18\\
99.66	0\\
99.67	1.73472347597681e-18\\
99.68	0\\
99.69	0\\
99.7	0\\
99.71	0\\
99.72	0\\
99.73	0\\
99.74	0\\
99.75	0\\
99.76	0\\
99.77	1.73472347597681e-18\\
99.78	0\\
99.79	0\\
99.8	0\\
99.81	0\\
99.82	0\\
99.83	0\\
99.84	0\\
99.85	0\\
99.86	0\\
99.87	0\\
99.88	0\\
99.89	0\\
99.9	0\\
99.91	0\\
99.92	0\\
99.93	0\\
99.94	0\\
99.95	0\\
99.96	0\\
99.97	0\\
99.98	0\\
99.99	0\\
100	0\\
};
\addlegendentry{$q=2$};

\addplot [color=mycolor1,solid]
  table[row sep=crcr]{%
0.01	0.00180108844777369\\
1.01	0.00179984246580634\\
2.01	0.00179823173510353\\
3.01	0.00179655773666776\\
4.01	0.00179481775455851\\
5.01	0.00179300893492088\\
6.01	0.00179112827705123\\
7.01	0.00178917262374545\\
8.01	0.00178713865088356\\
9.01	0.00178502285626283\\
10.01	0.00178282154804389\\
11.01	0.00178053083528039\\
12.01	0.00177814663585204\\
13.01	0.00177566479313577\\
14.01	0.00177308174165706\\
15.01	0.00177039578766906\\
16.01	0.00176760016414396\\
17.01	0.00176468662391234\\
18.01	0.00176164945348996\\
19.01	0.0017584826310553\\
20.01	0.00175517980316267\\
21.01	0.00175173427327766\\
22.01	0.00174813899362994\\
23.01	0.00174438656115718\\
24.01	0.00174046921478758\\
25.01	0.00173637880981325\\
26.01	0.00173210661519279\\
27.01	0.00172764193619225\\
28.01	0.00172296286916978\\
29.01	0.00171798170291983\\
30.01	0.00171252229161471\\
31.01	0.00170711068742848\\
32.01	0.00170173956209494\\
33.01	0.00169615557190633\\
34.01	0.00169034803230301\\
35.01	0.001684305455391\\
36.01	0.00167801546038683\\
37.01	0.00167146467134716\\
38.01	0.00166463860039271\\
39.01	0.00165752151542814\\
40.01	0.00165009629585938\\
41.01	0.00164234429673888\\
42.01	0.00163424525838736\\
43.01	0.00162577666146905\\
44.01	0.00161690984190359\\
45.01	0.00160761456231607\\
46.01	0.00159785989871978\\
47.01	0.00158761256245439\\
48.01	0.00157684419635927\\
49.01	0.00156559129391649\\
50.01	0.00155426769843272\\
51.01	0.00154290317002742\\
52.01	0.00153084890945793\\
53.01	0.00151802570807007\\
54.01	0.0015043147273238\\
55.01	0.00148932950688072\\
56.01	0.00147010534762902\\
57.01	0.00143763784887765\\
58.01	0.00140509227545263\\
59.01	0.00137107252375266\\
60.01	0.00133548917503814\\
61.01	0.00129824502343478\\
62.01	0.00125923407963422\\
63.01	0.00121834037200655\\
64.01	0.00117543653588249\\
65.01	0.00113038264286854\\
66.01	0.00108302726747121\\
67.01	0.00103320416477684\\
68.01	0.000980687385888791\\
69.01	0.000925145846401404\\
70.01	0.000865895158193228\\
71.01	0.000801309291955115\\
72.01	0.000740779584327448\\
73.01	0.00068271907955952\\
74.01	0.000606375862709216\\
75.01	0.000482764482176146\\
76.01	0.000354261142763266\\
77.01	0.000221793341475427\\
78.01	9.0130111263095e-05\\
79.01	0\\
80.01	0\\
81.01	0\\
82.01	1.73472347597681e-18\\
83.01	0\\
84.01	0\\
85.01	0\\
86.01	1.73472347597681e-18\\
87.01	0\\
88.01	1.73472347597681e-18\\
89.01	1.73472347597681e-18\\
90.01	0\\
91.01	1.73472347597681e-18\\
92.01	0\\
93.01	0\\
94.01	1.73472347597681e-18\\
95.01	1.73472347597681e-18\\
96.01	1.73472347597681e-18\\
97.01	0\\
98.01	1.73472347597681e-18\\
99.01	0\\
99.02	0\\
99.03	0\\
99.04	0\\
99.05	1.73472347597681e-18\\
99.06	0\\
99.07	0\\
99.08	1.73472347597681e-18\\
99.09	0\\
99.1	0\\
99.11	1.73472347597681e-18\\
99.12	0\\
99.13	0\\
99.14	0\\
99.15	0\\
99.16	0\\
99.17	1.73472347597681e-18\\
99.18	0\\
99.19	0\\
99.2	0\\
99.21	0\\
99.22	0\\
99.23	0\\
99.24	1.73472347597681e-18\\
99.25	0\\
99.26	0\\
99.27	1.73472347597681e-18\\
99.28	0\\
99.29	0\\
99.3	0\\
99.31	0\\
99.32	0\\
99.33	0\\
99.34	0\\
99.35	0\\
99.36	0\\
99.37	0\\
99.38	0\\
99.39	0\\
99.4	0\\
99.41	0\\
99.42	0\\
99.43	0\\
99.44	0\\
99.45	0\\
99.46	0\\
99.47	0\\
99.48	0\\
99.49	0\\
99.5	0\\
99.51	1.73472347597681e-18\\
99.52	0\\
99.53	0\\
99.54	0\\
99.55	0\\
99.56	0\\
99.57	0\\
99.58	0\\
99.59	0\\
99.6	0\\
99.61	1.73472347597681e-18\\
99.62	0\\
99.63	0\\
99.64	0\\
99.65	1.73472347597681e-18\\
99.66	0\\
99.67	1.73472347597681e-18\\
99.68	0\\
99.69	0\\
99.7	0\\
99.71	0\\
99.72	0\\
99.73	0\\
99.74	0\\
99.75	0\\
99.76	0\\
99.77	1.73472347597681e-18\\
99.78	0\\
99.79	0\\
99.8	0\\
99.81	0\\
99.82	0\\
99.83	0\\
99.84	0\\
99.85	0\\
99.86	0\\
99.87	0\\
99.88	0\\
99.89	0\\
99.9	0\\
99.91	0\\
99.92	0\\
99.93	0\\
99.94	0\\
99.95	0\\
99.96	0\\
99.97	0\\
99.98	0\\
99.99	0\\
100	0\\
};
\addlegendentry{$q=3$};

\addplot [color=green,solid]
  table[row sep=crcr]{%
0.01	0.000436939440003583\\
1.01	0.000435249038411013\\
2.01	0.000433715076373102\\
3.01	0.000432121203667831\\
4.01	0.000430464765526696\\
5.01	0.000428742950372802\\
6.01	0.000426952775594999\\
7.01	0.000425091071466512\\
8.01	0.000423154462925453\\
9.01	0.000421139348972532\\
10.01	0.000419041879963573\\
11.01	0.000416857936936011\\
12.01	0.000414583145707113\\
13.01	0.000412213189092001\\
14.01	0.00040974666013685\\
15.01	0.000407204605007089\\
16.01	0.000404596272407359\\
17.01	0.000401868534377913\\
18.01	0.000399012668723412\\
19.01	0.000396019856755268\\
20.01	0.000392880264043667\\
21.01	0.000389582869429453\\
22.01	0.000386115257936448\\
23.01	0.000382463367585647\\
24.01	0.000378611172048396\\
25.01	0.000374540243915152\\
26.01	0.000370228904079811\\
27.01	0.000365648712250342\\
28.01	0.000360735918349649\\
29.01	0.000355043552395995\\
30.01	0.00034501410892479\\
31.01	0.000333219282760679\\
32.01	0.000321032933456837\\
33.01	0.000308373354441102\\
34.01	0.00029521944760502\\
35.01	0.000281548946667593\\
36.01	0.000267338333795478\\
37.01	0.000252562748476483\\
38.01	0.000237195887947015\\
39.01	0.000221209899548568\\
40.01	0.000204575270032792\\
41.01	0.000187260725853388\\
42.01	0.000169233069767932\\
43.01	0.000150455923702175\\
44.01	0.000130887770278712\\
45.01	0.000110486949943736\\
46.01	8.92105669107138e-05\\
47.01	6.70151802680437e-05\\
48.01	4.38790378111831e-05\\
49.01	2.00812055629711e-05\\
50.01	6.16418725146728e-07\\
51.01	0\\
52.01	0\\
53.01	0\\
54.01	1.73472347597681e-18\\
55.01	1.73472347597681e-18\\
56.01	0\\
57.01	0\\
58.01	0\\
59.01	0\\
60.01	0\\
61.01	0\\
62.01	0\\
63.01	0\\
64.01	0\\
65.01	0\\
66.01	0\\
67.01	0\\
68.01	0\\
69.01	0\\
70.01	1.73472347597681e-18\\
71.01	0\\
72.01	0\\
73.01	1.73472347597681e-18\\
74.01	0\\
75.01	0\\
76.01	0\\
77.01	0\\
78.01	0\\
79.01	0\\
80.01	0\\
81.01	0\\
82.01	1.73472347597681e-18\\
83.01	0\\
84.01	0\\
85.01	0\\
86.01	1.73472347597681e-18\\
87.01	0\\
88.01	1.73472347597681e-18\\
89.01	1.73472347597681e-18\\
90.01	0\\
91.01	1.73472347597681e-18\\
92.01	0\\
93.01	0\\
94.01	1.73472347597681e-18\\
95.01	1.73472347597681e-18\\
96.01	1.73472347597681e-18\\
97.01	0\\
98.01	1.73472347597681e-18\\
99.01	0\\
99.02	0\\
99.03	0\\
99.04	0\\
99.05	1.73472347597681e-18\\
99.06	0\\
99.07	0\\
99.08	1.73472347597681e-18\\
99.09	0\\
99.1	0\\
99.11	1.73472347597681e-18\\
99.12	0\\
99.13	0\\
99.14	0\\
99.15	0\\
99.16	0\\
99.17	1.73472347597681e-18\\
99.18	0\\
99.19	0\\
99.2	0\\
99.21	0\\
99.22	0\\
99.23	0\\
99.24	1.73472347597681e-18\\
99.25	0\\
99.26	0\\
99.27	1.73472347597681e-18\\
99.28	0\\
99.29	0\\
99.3	0\\
99.31	0\\
99.32	0\\
99.33	0\\
99.34	0\\
99.35	0\\
99.36	0\\
99.37	0\\
99.38	0\\
99.39	0\\
99.4	0\\
99.41	0\\
99.42	0\\
99.43	0\\
99.44	0\\
99.45	0\\
99.46	0\\
99.47	0\\
99.48	0\\
99.49	0\\
99.5	0\\
99.51	1.73472347597681e-18\\
99.52	0\\
99.53	0\\
99.54	0\\
99.55	0\\
99.56	0\\
99.57	0\\
99.58	0\\
99.59	0\\
99.6	0\\
99.61	1.73472347597681e-18\\
99.62	0\\
99.63	0\\
99.64	0\\
99.65	1.73472347597681e-18\\
99.66	0\\
99.67	1.73472347597681e-18\\
99.68	0\\
99.69	0\\
99.7	0\\
99.71	0\\
99.72	0\\
99.73	0\\
99.74	0\\
99.75	0\\
99.76	0\\
99.77	1.73472347597681e-18\\
99.78	0\\
99.79	0\\
99.8	0\\
99.81	0\\
99.82	0\\
99.83	0\\
99.84	0\\
99.85	0\\
99.86	0\\
99.87	0\\
99.88	0\\
99.89	0\\
99.9	0\\
99.91	0\\
99.92	0\\
99.93	0\\
99.94	0\\
99.95	0\\
99.96	0\\
99.97	0\\
99.98	0\\
99.99	0\\
100	0\\
};
\addlegendentry{$q=4$};

\end{axis}
\end{tikzpicture}%

  \caption{Continuous Time}
\end{subfigure}%
\hfill%
\begin{subfigure}{.45\linewidth}
  \centering
  \setlength\figureheight{\linewidth} 
  \setlength\figurewidth{\linewidth}
  \tikzsetnextfilename{dp_colorbar/dm_dscr_z15}
  % This file was created by matlab2tikz.
%
%The latest updates can be retrieved from
%  http://www.mathworks.com/matlabcentral/fileexchange/22022-matlab2tikz-matlab2tikz
%where you can also make suggestions and rate matlab2tikz.
%
\definecolor{mycolor1}{rgb}{1.00000,0.00000,1.00000}%
%
\begin{tikzpicture}

\begin{axis}[%
width=4.564in,
height=3.803in,
at={(1.067in,0.513in)},
scale only axis,
every outer x axis line/.append style={black},
every x tick label/.append style={font=\color{black}},
xmin=0,
xmax=100,
xlabel={Time},
every outer y axis line/.append style={black},
every y tick label/.append style={font=\color{black}},
ymin=0,
ymax=0.015,
ylabel={Depth $\delta$},
axis background/.style={fill=white},
title={Z=15},
axis x line*=bottom,
axis y line*=left,
legend style={legend cell align=left,align=left,draw=black}
]
\addplot [color=green,dashed]
  table[row sep=crcr]{%
1	0.0122327606316414\\
2	0.012241522057822\\
3	0.0122505439155707\\
4	0.0122598297553432\\
5	0.0122693826350237\\
6	0.012279205027388\\
7	0.012289298715094\\
8	0.0122996646760778\\
9	0.0123103031544958\\
10	0.0123212138640972\\
11	0.0123323970708161\\
12	0.0123438601925528\\
13	0.0123555950386767\\
14	0.0123675832077272\\
15	0.0123798181383589\\
16	0.0123922919220944\\
17	0.0124049951859417\\
18	0.0124179168490719\\
19	0.0124310356266748\\
20	0.0124443197451163\\
21	0.0124577176870763\\
22	0.0124711589225573\\
23	0.0124846853127427\\
24	0.0124982474848527\\
25	0.012511787186506\\
26	0.0125252316719989\\
27	0.0125388468111412\\
28	0.0125527990068079\\
29	0.012567079642699\\
30	0.0125816785416316\\
31	0.0125965842095548\\
32	0.0126117831474384\\
33	0.0126272595804471\\
34	0.0126429948698342\\
35	0.0126589650740243\\
36	0.0126751355269413\\
37	0.0126905800261\\
38	0.0127064191489913\\
39	0.0127229337549721\\
40	0.0127401769159195\\
41	0.0127581650269329\\
42	0.0127769001634385\\
43	0.0127963650035619\\
44	0.0128165844624822\\
45	0.0128375798199826\\
46	0.0128593680795727\\
47	0.0128819614778757\\
48	0.0129053704793083\\
49	0.0129295498474715\\
50	0.0129539447314585\\
51	0.0129780726543019\\
52	0.0130018492520287\\
53	0.0130276145365274\\
54	0.0130531189295267\\
55	0.013077990109756\\
56	0.0131022486945868\\
57	0.0131260879584517\\
58	0.0131497615244743\\
59	0.0131732433489588\\
60	0.0131965475314924\\
61	0.0132196466585857\\
62	0.0132425450126211\\
63	0.0132652629027781\\
64	0.013287851666599\\
65	0.0133106623280383\\
66	0.0133337769836722\\
67	0.0133591616733824\\
68	0.0133850191726385\\
69	0.0134111173069747\\
70	0.0134369346682186\\
71	0.0134612352303047\\
72	0.0134856037193887\\
73	0.0135100647937358\\
74	0.013534594106904\\
75	0.013559157506947\\
76	0.0135835908959974\\
77	0.0136078314992819\\
78	0.0136300809313359\\
79	0.0136515665992234\\
80	0.0136722646348219\\
81	0.0136920171315763\\
82	0.0137106095932766\\
83	0.0137286654461089\\
84	0.0137462465757114\\
85	0.0137623201421256\\
86	0.0137775050281561\\
87	0.01379271403945\\
88	0.0138076796863335\\
89	0.013821319253111\\
90	0.0138344704942092\\
91	0.0138472674773515\\
92	0.0138598474407153\\
93	0.0138728061381877\\
94	0.013887029133783\\
95	0.0139037923926731\\
96	0.0139270290006153\\
97	0.0139715613125096\\
98	0.0140755501155958\\
99	0\\
100	0\\
};
\addlegendentry{$q=-4$};

\addplot [color=mycolor1,dashed]
  table[row sep=crcr]{%
1	0.0117188170935237\\
2	0.0117306652698127\\
3	0.0117429096743193\\
4	0.0117555606001355\\
5	0.0117686282489777\\
6	0.0117821226810193\\
7	0.0117960537675407\\
8	0.0118104311719426\\
9	0.0118252644577415\\
10	0.0118405637383399\\
11	0.0118563426599663\\
12	0.0118726324617226\\
13	0.0118894277179298\\
14	0.0119067012395361\\
15	0.0119244562546676\\
16	0.0119426944788827\\
17	0.0119614156482702\\
18	0.0119806165037037\\
19	0.0120002883132638\\
20	0.0120204092071249\\
21	0.0120409200608136\\
22	0.0120617367423536\\
23	0.012083053809787\\
24	0.012104861188102\\
25	0.0121271461351857\\
26	0.0121498932691813\\
27	0.0121730761831054\\
28	0.0121966589106519\\
29	0.0122206010777664\\
30	0.0122448284114289\\
31	0.0122692717523863\\
32	0.0122938567059575\\
33	0.0123184969082716\\
34	0.0123430916315677\\
35	0.0123675163962651\\
36	0.0123903374999865\\
37	0.0124115583881214\\
38	0.0124327012648518\\
39	0.0124543856686423\\
40	0.012476729429032\\
41	0.0124998238060895\\
42	0.0125237004378295\\
43	0.0125483175301616\\
44	0.0125736466452735\\
45	0.0125996705302306\\
46	0.0126263637123744\\
47	0.0126536900652924\\
48	0.0126815968999723\\
49	0.0127099775677733\\
50	0.0127386531047608\\
51	0.012767632638374\\
52	0.012796976678315\\
53	0.0128266090984458\\
54	0.0128564260304907\\
55	0.0128868309920126\\
56	0.0129179453392102\\
57	0.0129492283379129\\
58	0.0129800666081532\\
59	0.0130101120923736\\
60	0.0130377302212764\\
61	0.0130648578004952\\
62	0.0130914199231992\\
63	0.0131173464107898\\
64	0.0131425288367032\\
65	0.0131673222267289\\
66	0.0131918299726905\\
67	0.013216034277627\\
68	0.0132398830908501\\
69	0.0132633332051133\\
70	0.0132874667748315\\
71	0.0133138365869203\\
72	0.0133398407301026\\
73	0.0133656919237774\\
74	0.0133916256423548\\
75	0.0134176120152253\\
76	0.0134435975183044\\
77	0.0134696206927688\\
78	0.0134979872064833\\
79	0.013526295047492\\
80	0.0135542916460223\\
81	0.0135819116043021\\
82	0.0136090604613035\\
83	0.0136353875792836\\
84	0.0136608899324634\\
85	0.0136839201253191\\
86	0.0137052795969719\\
87	0.0137263229713489\\
88	0.0137471985247367\\
89	0.0137659838894507\\
90	0.0137843083005153\\
91	0.0138025843373601\\
92	0.0138209614507844\\
93	0.0138398695231636\\
94	0.0138613215606166\\
95	0.0138894431312715\\
96	0.013925543792143\\
97	0.0139715613125096\\
98	0.0140755501155958\\
99	0\\
100	0\\
};
\addlegendentry{$q=-3$};

\addplot [color=red,dashed]
  table[row sep=crcr]{%
1	0.0102709580499725\\
2	0.010281563369072\\
3	0.010292582102453\\
4	0.0103040316655793\\
5	0.0103159303038681\\
6	0.0103282971421568\\
7	0.0103411522457359\\
8	0.010354516708772\\
9	0.0103684128154397\\
10	0.0103828643933117\\
11	0.0103978973564456\\
12	0.0104135351939038\\
13	0.0104298014667672\\
14	0.0104467241755461\\
15	0.0104643326364947\\
16	0.0104826574700699\\
17	0.0105017305045356\\
18	0.0105215844908043\\
19	0.0105422524187287\\
20	0.0105637664610809\\
21	0.0105861622341779\\
22	0.0106094979982446\\
23	0.0106338139771202\\
24	0.0106591515116842\\
25	0.0106855528951657\\
26	0.0107130611511757\\
27	0.010741720000705\\
28	0.0107715742260739\\
29	0.010802673584565\\
30	0.0108350071096665\\
31	0.0108685994341274\\
32	0.0109034901001623\\
33	0.0109397278197828\\
34	0.0109773917256394\\
35	0.0110166459370303\\
36	0.0110552203897181\\
37	0.0110977486314485\\
38	0.0111416762584311\\
39	0.0111874662894916\\
40	0.0112367524444499\\
41	0.0112944140689771\\
42	0.0113542590386678\\
43	0.0114158262939671\\
44	0.0114790656123249\\
45	0.0115439585004534\\
46	0.0116104705923482\\
47	0.0116785437745188\\
48	0.0117480816096814\\
49	0.0118189186003087\\
50	0.0118906944978455\\
51	0.0119633576791547\\
52	0.0120346561095056\\
53	0.0121037888850088\\
54	0.0121698739737614\\
55	0.0122279467797018\\
56	0.0122814698727606\\
57	0.012336506268143\\
58	0.0123929905653289\\
59	0.0124505089611907\\
60	0.01250652929889\\
61	0.0125638994053487\\
62	0.0126225039614988\\
63	0.0126820544130006\\
64	0.0127422062361778\\
65	0.0128018722105154\\
66	0.0128555686963515\\
67	0.0129057704861722\\
68	0.0129545669867357\\
69	0.0130016072701518\\
70	0.0130449403196094\\
71	0.013084902895006\\
72	0.0131233286706024\\
73	0.0131600244581203\\
74	0.0131952549803496\\
75	0.0132296069523315\\
76	0.01326340025322\\
77	0.0132940287863121\\
78	0.0133240202672836\\
79	0.0133536487911245\\
80	0.0133828154380139\\
81	0.0134114322445559\\
82	0.0134394244153484\\
83	0.0134669745121238\\
84	0.0134940944783442\\
85	0.0135229775828559\\
86	0.013553300417067\\
87	0.0135826847719989\\
88	0.0136109729044239\\
89	0.0136400769814614\\
90	0.0136685968057691\\
91	0.013696341564945\\
92	0.0137232969189836\\
93	0.0137498611905214\\
94	0.0137779259554712\\
95	0.0138158749324622\\
96	0.0138757626800631\\
97	0.0139621673973833\\
98	0.0140755501155958\\
99	0\\
100	0\\
};
\addlegendentry{$q=-2$};

\addplot [color=blue,dashed]
  table[row sep=crcr]{%
1	0.00696266678300967\\
2	0.00697407602410268\\
3	0.0069859610560876\\
4	0.00699834474908146\\
5	0.00701125131491679\\
6	0.00702470641419099\\
7	0.00703873728159877\\
8	0.00705337286457887\\
9	0.00706864392778429\\
10	0.00708458295699298\\
11	0.00710122362787537\\
12	0.00711860182755134\\
13	0.00713675645756719\\
14	0.00715572885890603\\
15	0.00717556294846286\\
16	0.00719630533805207\\
17	0.0072180054519068\\
18	0.00724071576011226\\
19	0.00726449251394272\\
20	0.00728939763495815\\
21	0.00731549996724737\\
22	0.00734286844257857\\
23	0.00737157676614258\\
24	0.00740170378354605\\
25	0.00743333384065768\\
26	0.00746655703611329\\
27	0.00750146899616408\\
28	0.00753816929183361\\
29	0.00757675676447259\\
30	0.00761735020637924\\
31	0.00766008499747461\\
32	0.00770512001860023\\
33	0.00775266670123059\\
34	0.00780307646824082\\
35	0.00785710738072752\\
36	0.00791754763436833\\
37	0.0079802615944167\\
38	0.0080458099583471\\
39	0.00811586911773047\\
40	0.00819559079318031\\
41	0.00827853940493495\\
42	0.0083644074040341\\
43	0.00845330614277967\\
44	0.00854536255092997\\
45	0.00864070759235894\\
46	0.00873947339316896\\
47	0.008841788480229\\
48	0.0089477718710809\\
49	0.00905752947720104\\
50	0.00917117414942731\\
51	0.00928858526987464\\
52	0.00940427094424446\\
53	0.00951649651222525\\
54	0.00962298834456926\\
55	0.00970805341192905\\
56	0.00977984868148762\\
57	0.009854736189011\\
58	0.00993279098107374\\
59	0.0100146390509547\\
60	0.0101040286200362\\
61	0.0101953663496222\\
62	0.0102884759572755\\
63	0.0103827928711533\\
64	0.0104778404210339\\
65	0.0105732559399415\\
66	0.0106583464882014\\
67	0.0107408684014955\\
68	0.010822614385659\\
69	0.0109033451817943\\
70	0.0109788420309948\\
71	0.0110499045347061\\
72	0.0111207746787612\\
73	0.0111903355374338\\
74	0.0112582487723513\\
75	0.011326219736867\\
76	0.0113955543662148\\
77	0.0114604731860944\\
78	0.0115265590503869\\
79	0.0115946330502547\\
80	0.011664374843197\\
81	0.0117357782886367\\
82	0.0118088834229557\\
83	0.0118845317440119\\
84	0.0119639441718954\\
85	0.012050478640946\\
86	0.0121645462105752\\
87	0.012289336771521\\
88	0.0124146773866143\\
89	0.0125402773446041\\
90	0.0126655152963389\\
91	0.0127899080838297\\
92	0.0129132089901202\\
93	0.0130354663147911\\
94	0.0131576833482414\\
95	0.0132849489166765\\
96	0.0134453474499231\\
97	0.0137304419781218\\
98	0.0140755501155958\\
99	0\\
100	0\\
};
\addlegendentry{$q=-1$};

\addplot [color=black,solid]
  table[row sep=crcr]{%
1	0.0069535399396323\\
2	0.0069535399396323\\
3	0.0069535399396323\\
4	0.0069535399396323\\
5	0.0069535399396323\\
6	0.0069535399396323\\
7	0.0069535399396323\\
8	0.0069535399396323\\
9	0.0069535399396323\\
10	0.0069535399396323\\
11	0.0069535399396323\\
12	0.0069535399396323\\
13	0.0069535399396323\\
14	0.0069535399396323\\
15	0.0069535399396323\\
16	0.0069535399396323\\
17	0.0069535399396323\\
18	0.0069535399396323\\
19	0.0069535399396323\\
20	0.0069535399396323\\
21	0.0069535399396323\\
22	0.0069535399396323\\
23	0.0069535399396323\\
24	0.0069535399396323\\
25	0.0069535399396323\\
26	0.0069535399396323\\
27	0.0069535399396323\\
28	0.0069535399396323\\
29	0.0069535399396323\\
30	0.0069535399396323\\
31	0.0069535399396323\\
32	0.0069535399396323\\
33	0.0069535399396323\\
34	0.0069535399396323\\
35	0.0069535399396323\\
36	0.0069535399396323\\
37	0.0069535399396323\\
38	0.0069535399396323\\
39	0.0069535399396323\\
40	0.0069535399396323\\
41	0.0069535399396323\\
42	0.0069535399396323\\
43	0.0069535399396323\\
44	0.0069535399396323\\
45	0.0069535399396323\\
46	0.0069535399396323\\
47	0.0069535399396323\\
48	0.0069535399396323\\
49	0.0069535399396323\\
50	0.0069535399396323\\
51	0.0069535399396323\\
52	0.00696027783267133\\
53	0.00697299671308261\\
54	0.00698657675953214\\
55	0.0070011446971794\\
56	0.00701684118719758\\
57	0.00703382593193524\\
58	0.00705228738988878\\
59	0.00707244823104831\\
60	0.00709457244918762\\
61	0.00711897426419684\\
62	0.00714589746726174\\
63	0.00716869992745805\\
64	0.00719429039367552\\
65	0.00722315004847161\\
66	0.00728020871235676\\
67	0.00737330386823627\\
68	0.00747216693218319\\
69	0.00757828644269027\\
70	0.00769633490019724\\
71	0.0078219484968913\\
72	0.00795285098323012\\
73	0.00808949725914809\\
74	0.00823162513640673\\
75	0.00837397560283416\\
76	0.0085228596890992\\
77	0.00868283792378282\\
78	0.00884766761610357\\
79	0.00901754273325763\\
80	0.00918788597327563\\
81	0.00934803350457614\\
82	0.00951064346236894\\
83	0.00967590203938414\\
84	0.00984670278679874\\
85	0.0100302965634104\\
86	0.0101930953735268\\
87	0.0103551969237519\\
88	0.0105160183316585\\
89	0.0106749180557875\\
90	0.0108304902743305\\
91	0.0109806504953299\\
92	0.011123550218616\\
93	0.0112610305818502\\
94	0.011396274611093\\
95	0.011536591915825\\
96	0.0116985505525057\\
97	0.0119313279515455\\
98	0.0119846348539947\\
99	0\\
100	0\\
};
\addlegendentry{$q=0$};

\addplot [color=blue,solid]
  table[row sep=crcr]{%
1	0.0140747890152839\\
2	0.0140747853067208\\
3	0.0140747814898151\\
4	0.0140747775603974\\
5	0.0140747735155351\\
6	0.0140747693515032\\
7	0.0140747650607592\\
8	0.0140747606318368\\
9	0.0140747560483417\\
10	0.0140747513062544\\
11	0.0140747464221367\\
12	0.0140747413874906\\
13	0.014074736188711\\
14	0.0140747307957396\\
15	0.0140747251195047\\
16	0.0140747188363629\\
17	0.0140746892835858\\
18	0.0140746199071754\\
19	0.0140745481717254\\
20	0.0140744739493555\\
21	0.0140743970942218\\
22	0.0140743174259298\\
23	0.0140742346795541\\
24	0.0140741483594454\\
25	0.0140740574021\\
26	0.014073960557788\\
27	0.0140738599566251\\
28	0.0140737553611988\\
29	0.0140736465126963\\
30	0.014073533128553\\
31	0.0140734148996141\\
32	0.014073291486317\\
33	0.0140731625124396\\
34	0.0140730275517528\\
35	0.0140728860909299\\
36	0.0140727374003998\\
37	0.0140725799860309\\
38	0.0140720464440718\\
39	0.0140705294753545\\
40	0.0140689606425357\\
41	0.0140673364972055\\
42	0.014065653183951\\
43	0.0140639066111707\\
44	0.0140620920611754\\
45	0.014060204184613\\
46	0.0140582363747563\\
47	0.0140561809923816\\
48	0.014054029155822\\
49	0.0140517704777872\\
50	0.0140493927258234\\
51	0.0140468813584505\\
52	0.014044218817105\\
53	0.0140413832111486\\
54	0.0140383468965883\\
55	0.0140318474827155\\
56	0.0140242787915779\\
57	0.0140164057083802\\
58	0.0140082085829942\\
59	0.0139996600516814\\
60	0.0139907282707493\\
61	0.0139813769434149\\
62	0.01397156464403\\
63	0.0139612439060719\\
64	0.0139503604044524\\
65	0.0139388516775686\\
66	0.0139264519209255\\
67	0.01391300485925\\
68	0.0138983067396105\\
69	0.0138717190905066\\
70	0.0138402116152021\\
71	0.0138032183439128\\
72	0.0137642523093567\\
73	0.0137224605122615\\
74	0.0136767886098927\\
75	0.0136209545126161\\
76	0.0135620737592675\\
77	0.0134996559309287\\
78	0.013432882253133\\
79	0.0133479929115408\\
80	0.0132318936792259\\
81	0.013091439493059\\
82	0.0129442082898861\\
83	0.012789546522986\\
84	0.012626515584212\\
85	0.0124493165633761\\
86	0.0122243530586431\\
87	0.0119881118982065\\
88	0.011739777735739\\
89	0.0114778950676377\\
90	0.0112018141990376\\
91	0.010909072124832\\
92	0.0106022834277032\\
93	0.0102795724167244\\
94	0.00993137033858306\\
95	0.00955101288282799\\
96	0.00912465292637579\\
97	0.00824476209521964\\
98	0.00407555011559578\\
99	0\\
100	0\\
};
\addlegendentry{$q=1$};

\addplot [color=red,solid]
  table[row sep=crcr]{%
1	0.0140131419108131\\
2	0.0140124158162123\\
3	0.0140116665278782\\
4	0.014010892898994\\
5	0.0140100937203853\\
6	0.0140092680910095\\
7	0.014008415232017\\
8	0.0140075339143292\\
9	0.0140066228448268\\
10	0.0140056807209616\\
11	0.0140047062337551\\
12	0.0140036978124129\\
13	0.0140026535428182\\
14	0.0140015712954017\\
15	0.0140004485977575\\
16	0.0139992821208324\\
17	0.0139980372303798\\
18	0.0139966885177159\\
19	0.0139952851922683\\
20	0.013993822329178\\
21	0.0139922941220683\\
22	0.0139906936029604\\
23	0.013989012123546\\
24	0.0139872382053057\\
25	0.0139853546189215\\
26	0.0139830892787469\\
27	0.0139789092124666\\
28	0.0139745902387157\\
29	0.0139701257499622\\
30	0.0139655086758974\\
31	0.0139607314474885\\
32	0.0139557859601876\\
33	0.0139506635360103\\
34	0.0139453548806164\\
35	0.0139398500152443\\
36	0.0139341380835649\\
37	0.0139282064841823\\
38	0.0139215291127289\\
39	0.0139132494872816\\
40	0.0139046762880946\\
41	0.0138957887819354\\
42	0.0138865556893053\\
43	0.0138769251814139\\
44	0.0138669105576204\\
45	0.0138565183828406\\
46	0.013845722919349\\
47	0.0138344910794825\\
48	0.0138227850556276\\
49	0.0138105613879769\\
50	0.0137977697513191\\
51	0.0137843512349195\\
52	0.013770235441885\\
53	0.0137553349298749\\
54	0.0137395342039856\\
55	0.0137181695943548\\
56	0.0136945031875181\\
57	0.0136633017396435\\
58	0.0136254936003157\\
59	0.0135861628752301\\
60	0.0135451901743813\\
61	0.0135024438323053\\
62	0.0134577763201429\\
63	0.0134110078656356\\
64	0.0133619239504832\\
65	0.0133102894262138\\
66	0.0132562187548939\\
67	0.0131989041215644\\
68	0.0131375648918451\\
69	0.0130570273506728\\
70	0.0129672719789755\\
71	0.0128799991873934\\
72	0.0127883625333282\\
73	0.0126904654314209\\
74	0.0125734096485984\\
75	0.0124400919478998\\
76	0.012300672920036\\
77	0.0121542941251746\\
78	0.0119995487933104\\
79	0.0118172874393079\\
80	0.0116026918830551\\
81	0.0114120955547575\\
82	0.0112141523588551\\
83	0.0110083908932307\\
84	0.0107938588018927\\
85	0.010577379001518\\
86	0.0104177862729992\\
87	0.0102514642963198\\
88	0.0100797791511674\\
89	0.00990011712857725\\
90	0.00971281494860763\\
91	0.0095213768432415\\
92	0.00933091037119082\\
93	0.00913507579011389\\
94	0.0089299129425915\\
95	0.0087147000474069\\
96	0.00827718649022398\\
97	0.00632133460307733\\
98	0.00407555011559578\\
99	0\\
100	0\\
};
\addlegendentry{$q=2$};

\addplot [color=mycolor1,solid]
  table[row sep=crcr]{%
1	0.0137701680573566\\
2	0.0137665932044743\\
3	0.0137629117712479\\
4	0.0137591193277895\\
5	0.0137552107147427\\
6	0.0137511805544953\\
7	0.0137470259076089\\
8	0.0137427449137119\\
9	0.0137383327324756\\
10	0.013733784335973\\
11	0.0137290944426782\\
12	0.0137242571616614\\
13	0.0137192657136528\\
14	0.0137141128126003\\
15	0.0137087906469272\\
16	0.0137032909542647\\
17	0.0136976597292383\\
18	0.0136919287327977\\
19	0.0136859965191235\\
20	0.0136798515107889\\
21	0.0136734790267163\\
22	0.013666862227869\\
23	0.0136599813561021\\
24	0.0136528121777675\\
25	0.0136453219859478\\
26	0.0136371160966439\\
27	0.0136260117248677\\
28	0.0136145015052309\\
29	0.0136026114874057\\
30	0.0135903225938376\\
31	0.0135776141542072\\
32	0.0135644637256027\\
33	0.013550846896024\\
34	0.0135367370818913\\
35	0.0135221053614088\\
36	0.0135069205081391\\
37	0.0134911499430103\\
38	0.013475650444195\\
39	0.0134618684065498\\
40	0.0134475005382067\\
41	0.0134324903049592\\
42	0.0134167511270578\\
43	0.0134001188163241\\
44	0.0133771851941023\\
45	0.0133501134992671\\
46	0.0133221402773208\\
47	0.0132932135562792\\
48	0.0132632763447413\\
49	0.0132322658634036\\
50	0.0132001125181892\\
51	0.0131667384085978\\
52	0.0131320546488386\\
53	0.0130959548139344\\
54	0.0130583035157773\\
55	0.013026827467008\\
56	0.0129954251156187\\
57	0.0129533411439786\\
58	0.0129020323456521\\
59	0.0128487426063729\\
60	0.012793334759399\\
61	0.0127356578542819\\
62	0.0126755685252279\\
63	0.0126129190001478\\
64	0.0125474360908037\\
65	0.012478723329547\\
66	0.012406385462252\\
67	0.0123310224627413\\
68	0.0122524759856178\\
69	0.0121955659437535\\
70	0.0121223723366172\\
71	0.0120268963747487\\
72	0.0119268429872047\\
73	0.0118227406772296\\
74	0.0116945929420149\\
75	0.0115277077437531\\
76	0.0113552798374595\\
77	0.0111772918531823\\
78	0.0109944964154016\\
79	0.0108397659615434\\
80	0.0107346360348897\\
81	0.0106255650825685\\
82	0.0105131150129121\\
83	0.0103982252022044\\
84	0.0102806923492592\\
85	0.0101609525846018\\
86	0.0100348669550475\\
87	0.00990542750144545\\
88	0.00977254152256229\\
89	0.00963271907182141\\
90	0.00948542100562233\\
91	0.00933003221642036\\
92	0.00916544862748719\\
93	0.00899179295123485\\
94	0.00880899567312919\\
95	0.00847358171863326\\
96	0.00734600028981367\\
97	0.00586739649120498\\
98	0.00407555011559578\\
99	0\\
100	0\\
};
\addlegendentry{$q=3$};

\addplot [color=green,solid]
  table[row sep=crcr]{%
1	0.0134720254946476\\
2	0.0134667840002946\\
3	0.0134613917977598\\
4	0.0134558438388858\\
5	0.0134501331454569\\
6	0.0134442477718128\\
7	0.013438170999875\\
8	0.0134319055216595\\
9	0.013425467866156\\
10	0.0134188519181047\\
11	0.0134120510835525\\
12	0.0134050582473476\\
13	0.0133978658185527\\
14	0.0133904657010935\\
15	0.0133828493470008\\
16	0.0133750083507678\\
17	0.0133669322243053\\
18	0.0133586079072552\\
19	0.0133500267485475\\
20	0.0133411816188853\\
21	0.0133320616986511\\
22	0.0133226559561399\\
23	0.0133129534056457\\
24	0.0133029438011369\\
25	0.0132926201431537\\
26	0.0132825806884392\\
27	0.0132764831833719\\
28	0.0132689373947882\\
29	0.0132557368854051\\
30	0.0132421565350406\\
31	0.0132281782625461\\
32	0.0132137820880791\\
33	0.0131989457842485\\
34	0.0131836444452043\\
35	0.0131678499618763\\
36	0.013151530426545\\
37	0.0131346497480047\\
38	0.0131171522919514\\
39	0.01309894889047\\
40	0.0130799762917019\\
41	0.0130601621672877\\
42	0.0130394240624307\\
43	0.0130176430428208\\
44	0.0129871267568911\\
45	0.0129509067503821\\
46	0.0129135846315742\\
47	0.0128751159829357\\
48	0.0128354546454393\\
49	0.0127945527982676\\
50	0.0127523609580503\\
51	0.0127088279306244\\
52	0.0126639016558636\\
53	0.0126175324887667\\
54	0.0125696831148186\\
55	0.0125202074910236\\
56	0.0124691564891979\\
57	0.0124317672558142\\
58	0.0124056690887418\\
59	0.0123782087026684\\
60	0.0123492922121896\\
61	0.012318684009363\\
62	0.0122860583715728\\
63	0.012251256164234\\
64	0.0122143817935398\\
65	0.0121747691199453\\
66	0.0121262291201672\\
67	0.0120388755397677\\
68	0.0119483220600089\\
69	0.0118531751669201\\
70	0.0117235133944667\\
71	0.0115630221741897\\
72	0.0113978677821986\\
73	0.0112286425469846\\
74	0.0110897806425187\\
75	0.0110056500983793\\
76	0.0109197013311663\\
77	0.0108324349380534\\
78	0.0107446101229008\\
79	0.0106562374605092\\
80	0.0105640666156659\\
81	0.0104679168593401\\
82	0.010367590898843\\
83	0.0102629765679872\\
84	0.0101544321151487\\
85	0.0100434433798738\\
86	0.0099270810727797\\
87	0.00980475556964079\\
88	0.00967581506317478\\
89	0.00953962853680043\\
90	0.00939493691026074\\
91	0.00924339423504485\\
92	0.00908201703891517\\
93	0.00891275907712221\\
94	0.00861876478833807\\
95	0.0078199693350052\\
96	0.00674579869692106\\
97	0.00586739649120498\\
98	0.00407555011559578\\
99	0\\
100	0\\
};
\addlegendentry{$q=4$};

\end{axis}
\end{tikzpicture}% 
  \caption{Discrete Time}
\end{subfigure}\\
\vspace{1cm}
\begin{subfigure}{.45\linewidth}
  \centering
  \setlength\figureheight{\linewidth} 
  \setlength\figurewidth{\linewidth}
  \tikzsetnextfilename{dp_colorbar/dm_cts_nFPC_z15}
  % This file was created by matlab2tikz.
%
%The latest updates can be retrieved from
%  http://www.mathworks.com/matlabcentral/fileexchange/22022-matlab2tikz-matlab2tikz
%where you can also make suggestions and rate matlab2tikz.
%
\definecolor{mycolor1}{rgb}{1.00000,0.00000,1.00000}%
%
\begin{tikzpicture}[trim axis left, trim axis right]

\begin{axis}[%
width=\figurewidth,
height=\figureheight,
at={(0\figurewidth,0\figureheight)},
scale only axis,
every outer x axis line/.append style={black},
every x tick label/.append style={font=\color{black}},
xmin=0,
xmax=100,
%xlabel={Time},
every outer y axis line/.append style={black},
every y tick label/.append style={font=\color{black}},
ymin=0,
ymax=0.015,
%ylabel={Depth $\delta^-$},
axis background/.style={fill=white},
axis x line*=bottom,
axis y line*=left,
yticklabel style={
        /pgf/number format/fixed,
        /pgf/number format/precision=3
},
scaled y ticks=false,
legend style={legend cell align=left,align=left,draw=black,font=\footnotesize, at={(0.98,0.02)},anchor=south east},
every axis legend/.code={\renewcommand\addlegendentry[2][]{}}  %ignore legend locally
]
\addplot [color=green,dashed]
  table[row sep=crcr]{%
0.01	0\\
1.01	0\\
2.01	0\\
3.01	0\\
4.01	0\\
5.01	0\\
6.01	0\\
7.01	0\\
8.01	0\\
9.01	0\\
10.01	0\\
11.01	0\\
12.01	0\\
13.01	0\\
14.01	0\\
15.01	0\\
16.01	0\\
17.01	0\\
18.01	0\\
19.01	0\\
20.01	0\\
21.01	0\\
22.01	0\\
23.01	0\\
24.01	0\\
25.01	0\\
26.01	0\\
27.01	0\\
28.01	0\\
29.01	0\\
30.01	0\\
31.01	0\\
32.01	0\\
33.01	0\\
34.01	0\\
35.01	0\\
36.01	0\\
37.01	0\\
38.01	0\\
39.01	0\\
40.01	0\\
41.01	0\\
42.01	0\\
43.01	0\\
44.01	0\\
45.01	0\\
46.01	0\\
47.01	0\\
48.01	0\\
49.01	0\\
50.01	0\\
51.01	0\\
52.01	0\\
53.01	0\\
54.01	0\\
55.01	0\\
56.01	0\\
57.01	0\\
58.01	0\\
59.01	0\\
60.01	0\\
61.01	0\\
62.01	0\\
63.01	0\\
64.01	0\\
65.01	0\\
66.01	0\\
67.01	0\\
68.01	0\\
69.01	0\\
70.01	0\\
71.01	0\\
72.01	0\\
73.01	0\\
74.01	0\\
75.01	0\\
76.01	0\\
77.01	0\\
78.01	0\\
79.01	0\\
80.01	0\\
81.01	0\\
82.01	0\\
83.01	0\\
84.01	0\\
85.01	0\\
86.01	0\\
87.01	0\\
88.01	0\\
89.01	0\\
90.01	0\\
91.01	0\\
92.01	0\\
93.01	0\\
94.01	0\\
95.01	0\\
96.01	0\\
97.01	0\\
98.01	0.000386588687883091\\
99.01	0.00337828656858661\\
99.02	0.00341954072443339\\
99.03	0.00346117385023638\\
99.04	0.00350318948458161\\
99.05	0.00354559119907409\\
99.06	0.00358838259864556\\
99.07	0.00363156732186504\\
99.08	0.00367514904125229\\
99.09	0.00371913146359422\\
99.1	0.0037635183302642\\
99.11	0.00380831341754438\\
99.12	0.00385352053695097\\
99.13	0.00389914353556257\\
99.14	0.00394518629635159\\
99.15	0.00399165273851862\\
99.16	0.00403854681783014\\
99.17	0.00408587252695911\\
99.18	0.00413363389582897\\
99.19	0.00418183495431657\\
99.2	0.00423047976694747\\
99.21	0.00427957243576882\\
99.22	0.00432911710069517\\
99.23	0.00437911793985739\\
99.24	0.00442957916995488\\
99.25	0.00448050504661093\\
99.26	0.00453189986473142\\
99.27	0.0045837679588668\\
99.28	0.00463611370357735\\
99.29	0.00468894151380188\\
99.3	0.00474225584522977\\
99.31	0.00479606119467643\\
99.32	0.00485036210046225\\
99.33	0.00490516314279497\\
99.34	0.00496046894415566\\
99.35	0.00501628416968814\\
99.36	0.00507261352759206\\
99.37	0.0051294617695196\\
99.38	0.00518683369097572\\
99.39	0.00524473413172222\\
99.4	0.00530316797618543\\
99.41	0.00536214015386766\\
99.42	0.00542165563976247\\
99.43	0.00548171945477372\\
99.44	0.00554233666613847\\
99.45	0.00560351238785381\\
99.46	0.00566525178110757\\
99.47	0.00572756005471303\\
99.48	0.00579044246554756\\
99.49	0.00585390431899538\\
99.5	0.00591795096939432\\
99.51	0.00598258782048674\\
99.52	0.00604782032587452\\
99.53	0.00611365398947834\\
99.54	0.00618009436600108\\
99.55	0.00624714706139553\\
99.56	0.00631481773333638\\
99.57	0.00638311209169663\\
99.58	0.00645203589902817\\
99.59	0.00652159497104708\\
99.6	0.00659179517712308\\
99.61	0.00666264244077372\\
99.62	0.00673414273681778\\
99.63	0.00680630208973471\\
99.64	0.00687912657950685\\
99.65	0.00695262234212994\\
99.66	0.0070267955701283\\
99.67	0.00710165251307478\\
99.68	0.00717719947811546\\
99.69	0.00725344283049917\\
99.7	0.00733038899411186\\
99.71	0.0074080444520159\\
99.72	0.00748641574699432\\
99.73	0.00756550948210005\\
99.74	0.00764533232121015\\
99.75	0.00772589098958526\\
99.76	0.00780719227443403\\
99.77	0.00788924302548281\\
99.78	0.00797205015555061\\
99.79	0.00805562064112923\\
99.8	0.00813996152296881\\
99.81	0.00822507990666871\\
99.82	0.00831098296327383\\
99.83	0.00839767792987635\\
99.84	0.00848517211022314\\
99.85	0.00857347287532854\\
99.86	0.00866258766409292\\
99.87	0.00875252398392687\\
99.88	0.00884328941138109\\
99.89	0.00893489159278211\\
99.9	0.00902733824487381\\
99.91	0.00912063715546484\\
99.92	0.00921479618408196\\
99.93	0.00930982326262945\\
99.94	0.00940572639605445\\
99.95	0.00950251366301854\\
99.96	0.00960019321657535\\
99.97	0.00969877328485451\\
99.98	0.00979826217175179\\
99.99	0.00989866825762563\\
100	0.01\\
};
\addlegendentry{$q=-4$};

\addplot [color=mycolor1,dashed]
  table[row sep=crcr]{%
0.01	0\\
1.01	0\\
2.01	0\\
3.01	0\\
4.01	0\\
5.01	0\\
6.01	0\\
7.01	0\\
8.01	0\\
9.01	0\\
10.01	0\\
11.01	0\\
12.01	0\\
13.01	0\\
14.01	0\\
15.01	0\\
16.01	0\\
17.01	0\\
18.01	0\\
19.01	0\\
20.01	0\\
21.01	0\\
22.01	0\\
23.01	0\\
24.01	0\\
25.01	0\\
26.01	0\\
27.01	0\\
28.01	0\\
29.01	0\\
30.01	0\\
31.01	0\\
32.01	0\\
33.01	0\\
34.01	0\\
35.01	0\\
36.01	0\\
37.01	0\\
38.01	0\\
39.01	0\\
40.01	0\\
41.01	0\\
42.01	0\\
43.01	0\\
44.01	0\\
45.01	0\\
46.01	0\\
47.01	0\\
48.01	0\\
49.01	0\\
50.01	0\\
51.01	0\\
52.01	0\\
53.01	0\\
54.01	0\\
55.01	0\\
56.01	0\\
57.01	0\\
58.01	0\\
59.01	0\\
60.01	0\\
61.01	0\\
62.01	0\\
63.01	0\\
64.01	0\\
65.01	0\\
66.01	0\\
67.01	0\\
68.01	0\\
69.01	0\\
70.01	0\\
71.01	0\\
72.01	0\\
73.01	0\\
74.01	0\\
75.01	0\\
76.01	0\\
77.01	0\\
78.01	0\\
79.01	0\\
80.01	0\\
81.01	0\\
82.01	0\\
83.01	0\\
84.01	0\\
85.01	0\\
86.01	0\\
87.01	0\\
88.01	0\\
89.01	0\\
90.01	0\\
91.01	0\\
92.01	0\\
93.01	0\\
94.01	0\\
95.01	0\\
96.01	0\\
97.01	0\\
98.01	0\\
99.01	0.00337828226131128\\
99.02	0.00341953654113725\\
99.03	0.00346116978864239\\
99.04	0.00350318554243863\\
99.05	0.00354558737415644\\
99.06	0.00358837888875248\\
99.07	0.00363156372482022\\
99.08	0.00367514555490337\\
99.09	0.00371912808581235\\
99.1	0.00376351505894353\\
99.11	0.00380831025060157\\
99.12	0.00385351747232472\\
99.13	0.00389914057121316\\
99.14	0.00394518343026037\\
99.15	0.00399164996868758\\
99.16	0.00403854414228137\\
99.17	0.00408586994373441\\
99.18	0.0041336314029893\\
99.19	0.00418183254997663\\
99.2	0.0042304774492322\\
99.21	0.00427957020281343\\
99.22	0.00432911495064515\\
99.23	0.00437911587086857\\
99.24	0.00442957718019346\\
99.25	0.0044805031342536\\
99.26	0.00453189802796542\\
99.27	0.00458376619589004\\
99.28	0.00463611201259857\\
99.29	0.00468893989304077\\
99.3	0.00474225429291715\\
99.31	0.00479605970905447\\
99.32	0.00485036067978466\\
99.33	0.00490516178532726\\
99.34	0.00496046764817539\\
99.35	0.00501628293348524\\
99.36	0.00507261234946913\\
99.37	0.00512946064779225\\
99.38	0.00518683262397297\\
99.39	0.0052447331177869\\
99.4	0.00530316701367457\\
99.41	0.005362139241153\\
99.42	0.00542165477523084\\
99.43	0.00548171863682759\\
99.44	0.00554233589319639\\
99.45	0.00560351165835098\\
99.46	0.00566525109349632\\
99.47	0.00572755940746341\\
99.48	0.00579044185714788\\
99.49	0.00585390374795281\\
99.5	0.0059179504342355\\
99.51	0.00598258731975837\\
99.52	0.00604781985814402\\
99.53	0.00611365355333448\\
99.54	0.00618009396005464\\
99.55	0.00624714668428\\
99.56	0.00631481738370866\\
99.57	0.00638311176823765\\
99.58	0.00645203560044371\\
99.59	0.0065215946960684\\
99.6	0.00659179492450768\\
99.61	0.00666264220930608\\
99.62	0.00673414252531063\\
99.63	0.00680630189702865\\
99.64	0.00687912640447119\\
99.65	0.00695262218366352\\
99.66	0.00702679542716034\\
99.67	0.00710165238456571\\
99.68	0.00717719936305772\\
99.69	0.00725344272791803\\
99.7	0.00733038890306619\\
99.71	0.00740804437159896\\
99.72	0.00748641567633449\\
99.73	0.00756550942036152\\
99.74	0.00764533226759364\\
99.75	0.00772589094332859\\
99.76	0.0078071922348127\\
99.77	0.00788924299181055\\
99.78	0.00797205012717976\\
99.79	0.00805562061745114\\
99.8	0.0081399615034141\\
99.81	0.00822507989070742\\
99.82	0.00831098295041551\\
99.83	0.00839767791967001\\
99.84	0.00848517210225699\\
99.85	0.0085734728692297\\
99.86	0.00866258765952688\\
99.87	0.00875252398059677\\
99.88	0.00884328940902682\\
99.89	0.00893489159117915\\
99.9	0.00902733824383187\\
99.91	0.00912063715482618\\
99.92	0.00921479618371945\\
99.93	0.00930982326244423\\
99.94	0.00940572639597333\\
99.95	0.0095025136629909\\
99.96	0.0096001932165697\\
99.97	0.00969877328485451\\
99.98	0.0097982621717518\\
99.99	0.00989866825762563\\
100	0.01\\
};
\addlegendentry{$q=-3$};

\addplot [color=red,dashed]
  table[row sep=crcr]{%
0.01	0\\
1.01	0\\
2.01	0\\
3.01	0\\
4.01	0\\
5.01	0\\
6.01	0\\
7.01	0\\
8.01	0\\
9.01	0\\
10.01	0\\
11.01	0\\
12.01	0\\
13.01	0\\
14.01	0\\
15.01	0\\
16.01	0\\
17.01	0\\
18.01	0\\
19.01	0\\
20.01	0\\
21.01	0\\
22.01	0\\
23.01	0\\
24.01	0\\
25.01	0\\
26.01	0\\
27.01	0\\
28.01	0\\
29.01	0\\
30.01	0\\
31.01	0\\
32.01	0\\
33.01	0\\
34.01	0\\
35.01	0\\
36.01	0\\
37.01	0\\
38.01	0\\
39.01	0\\
40.01	0\\
41.01	0\\
42.01	0\\
43.01	0\\
44.01	0\\
45.01	0\\
46.01	0\\
47.01	0\\
48.01	0\\
49.01	0\\
50.01	0\\
51.01	0\\
52.01	0\\
53.01	0\\
54.01	0\\
55.01	0\\
56.01	0\\
57.01	0\\
58.01	0\\
59.01	0\\
60.01	0\\
61.01	0\\
62.01	0\\
63.01	0\\
64.01	0\\
65.01	0\\
66.01	0\\
67.01	0\\
68.01	0\\
69.01	0\\
70.01	0\\
71.01	0\\
72.01	0\\
73.01	0\\
74.01	0\\
75.01	0\\
76.01	0\\
77.01	0\\
78.01	0\\
79.01	0\\
80.01	0\\
81.01	0\\
82.01	0\\
83.01	0\\
84.01	0\\
85.01	0\\
86.01	0\\
87.01	0\\
88.01	0\\
89.01	0\\
90.01	0\\
91.01	0\\
92.01	0\\
93.01	0\\
94.01	0\\
95.01	0\\
96.01	0\\
97.01	0\\
98.01	0\\
99.01	0.00337794512537842\\
99.02	0.00341920636951997\\
99.03	0.00346084650937926\\
99.04	0.00350286908385008\\
99.05	0.00354527766483987\\
99.06	0.00358807585757756\\
99.07	0.00363126730092422\\
99.08	0.00367485566768671\\
99.09	0.00371884466493414\\
99.1	0.00376323803431739\\
99.11	0.00380803955239153\\
99.12	0.00385325303094139\\
99.13	0.00389888231730995\\
99.14	0.00394493129473008\\
99.15	0.00399140388265912\\
99.16	0.00403830403711683\\
99.17	0.00408563575102626\\
99.18	0.00413340305455804\\
99.19	0.00418160998596475\\
99.2	0.00423026060940814\\
99.21	0.00427935902655147\\
99.22	0.00432890937690489\\
99.23	0.00437891583817394\\
99.24	0.00442938262661121\\
99.25	0.00448031399737134\\
99.26	0.00453171424486916\\
99.27	0.00458358770314124\\
99.28	0.00463593874621069\\
99.29	0.00468877178845529\\
99.3	0.00474209128497913\\
99.31	0.00479590173198754\\
99.32	0.0048502076671655\\
99.33	0.0049050136700596\\
99.34	0.00496032436246343\\
99.35	0.00501614440880655\\
99.36	0.00507247851654706\\
99.37	0.00512933143656773\\
99.38	0.00518670796357587\\
99.39	0.00524461293650739\\
99.4	0.00530305123896248\\
99.41	0.00536202779961808\\
99.42	0.00542154759264429\\
99.43	0.00548161563812457\\
99.44	0.00554223700248002\\
99.45	0.00560341679889745\\
99.46	0.00566516018776163\\
99.47	0.00572747237709148\\
99.48	0.0057903586229804\\
99.49	0.00585382423004071\\
99.5	0.00591787455185232\\
99.51	0.00598251499141552\\
99.52	0.00604775100160814\\
99.53	0.00611358808564699\\
99.54	0.00618003179755359\\
99.55	0.00624708774262441\\
99.56	0.00631476157790545\\
99.57	0.00638305901267143\\
99.58	0.00645198580890939\\
99.59	0.00652154778180702\\
99.6	0.00659175080024555\\
99.61	0.00666260078729736\\
99.62	0.00673410371788326\\
99.63	0.00680626561604403\\
99.64	0.006879092561375\\
99.65	0.00695259068954072\\
99.66	0.00702676619279462\\
99.67	0.00710162532050356\\
99.68	0.00717717437967758\\
99.69	0.00725341973550465\\
99.7	0.00733036781189061\\
99.71	0.00740802509200435\\
99.72	0.00748639811882837\\
99.73	0.00756549349571853\\
99.74	0.0076453178869692\\
99.75	0.00772587801838046\\
99.76	0.00780718067783092\\
99.77	0.00788923271585624\\
99.78	0.00797204104623336\\
99.79	0.00805561264657055\\
99.8	0.00813995455890341\\
99.81	0.00822507389029681\\
99.82	0.00831097781345295\\
99.83	0.00839767356732553\\
99.84	0.0084851684577402\\
99.85	0.00857346985802142\\
99.86	0.00866258520962567\\
99.87	0.00875252202278125\\
99.88	0.00884328787713478\\
99.89	0.00893489042240442\\
99.9	0.00902733737903997\\
99.91	0.00912063653888998\\
99.92	0.009214795765876\\
99.93	0.00930982299667403\\
99.94	0.00940572624140342\\
99.95	0.00950251358432325\\
99.96	0.00960019318453634\\
99.97	0.00969877327670119\\
99.98	0.00979826217175179\\
99.99	0.00989866825762563\\
100	0.01\\
};
\addlegendentry{$q=-2$};

\addplot [color=blue,dashed]
  table[row sep=crcr]{%
0.01	0\\
1.01	0\\
2.01	0\\
3.01	0\\
4.01	0\\
5.01	0\\
6.01	0\\
7.01	0\\
8.01	0\\
9.01	0\\
10.01	0\\
11.01	0\\
12.01	0\\
13.01	0\\
14.01	0\\
15.01	0\\
16.01	0\\
17.01	0\\
18.01	0\\
19.01	0\\
20.01	0\\
21.01	0\\
22.01	0\\
23.01	0\\
24.01	0\\
25.01	0\\
26.01	0\\
27.01	0\\
28.01	0\\
29.01	0\\
30.01	0\\
31.01	0\\
32.01	0\\
33.01	0\\
34.01	0\\
35.01	0\\
36.01	0\\
37.01	0\\
38.01	0\\
39.01	0\\
40.01	0\\
41.01	0\\
42.01	0\\
43.01	0\\
44.01	0\\
45.01	0\\
46.01	0\\
47.01	0\\
48.01	0\\
49.01	0\\
50.01	0\\
51.01	0\\
52.01	0\\
53.01	0\\
54.01	0\\
55.01	0\\
56.01	0\\
57.01	0\\
58.01	0\\
59.01	0\\
60.01	0\\
61.01	0\\
62.01	0\\
63.01	0\\
64.01	0\\
65.01	0\\
66.01	0\\
67.01	0\\
68.01	0\\
69.01	0\\
70.01	0\\
71.01	0\\
72.01	0\\
73.01	0\\
74.01	0\\
75.01	0\\
76.01	0\\
77.01	0\\
78.01	0\\
79.01	0\\
80.01	0\\
81.01	0\\
82.01	0\\
83.01	0\\
84.01	0\\
85.01	0\\
86.01	0\\
87.01	0\\
88.01	0\\
89.01	0\\
90.01	0\\
91.01	0\\
92.01	0\\
93.01	0\\
94.01	0\\
95.01	0\\
96.01	0\\
97.01	0\\
98.01	0\\
99.01	0.00335442711367033\\
99.02	0.00339602125744558\\
99.03	0.00343799324950865\\
99.04	0.0034803466119229\\
99.05	0.00352308489923335\\
99.06	0.0035662116987579\\
99.07	0.00360973063088092\\
99.08	0.00365364534934916\\
99.09	0.00369795954156998\\
99.1	0.00374267692891206\\
99.11	0.00378780126700835\\
99.12	0.00383333634606152\\
99.13	0.00387928599115167\\
99.14	0.00392565406254658\\
99.15	0.0039724444560142\\
99.16	0.00401966110313765\\
99.17	0.0040673079716326\\
99.18	0.00411538906566701\\
99.19	0.00416390842624637\\
99.2	0.0042128701315636\\
99.21	0.00426227829732716\\
99.22	0.00431213707709156\\
99.23	0.00436245066259037\\
99.24	0.00441322328407146\\
99.25	0.00446445921063472\\
99.26	0.00451616275057213\\
99.27	0.00456833825171019\\
99.28	0.00462099010175469\\
99.29	0.00467412272863787\\
99.3	0.00472774060086789\\
99.31	0.00478184822788059\\
99.32	0.00483645016039364\\
99.33	0.00489155099076289\\
99.34	0.00494715535334105\\
99.35	0.00500326792483865\\
99.36	0.00505989342468713\\
99.37	0.00511703661540424\\
99.38	0.00517470230296157\\
99.39	0.00523289533568309\\
99.4	0.00529162054313569\\
99.41	0.00535088279818377\\
99.42	0.00541068701734764\\
99.43	0.0054710381611638\\
99.44	0.00553194123454689\\
99.45	0.00559340128715346\\
99.46	0.00565542341374731\\
99.47	0.00571801275456649\\
99.48	0.00578117449569182\\
99.49	0.00584491386941689\\
99.5	0.00590923615461952\\
99.51	0.00597414667713447\\
99.52	0.00603965081012756\\
99.53	0.00610575397447076\\
99.54	0.00617246163911862\\
99.55	0.00623977932148552\\
99.56	0.00630771258782391\\
99.57	0.00637626705360334\\
99.58	0.00644544838389017\\
99.59	0.00651526229372785\\
99.6	0.00658571454851766\\
99.61	0.00665681096439979\\
99.62	0.00672855740863903\\
99.63	0.00680095980001549\\
99.64	0.00687402410920575\\
99.65	0.00694775635916377\\
99.66	0.00702216262550119\\
99.67	0.00709724903686717\\
99.68	0.00717302177532716\\
99.69	0.0072494870767408\\
99.7	0.00732665123113838\\
99.71	0.00740452058309589\\
99.72	0.00748310153210824\\
99.73	0.00756240052588584\\
99.74	0.00764242406140549\\
99.75	0.00772317869143584\\
99.76	0.00780467102489812\\
99.77	0.00788690772722355\\
99.78	0.00796989552070725\\
99.79	0.00805364118485809\\
99.8	0.00813815155674424\\
99.81	0.00822343353133399\\
99.82	0.00830949406183134\\
99.83	0.00839634016000594\\
99.84	0.00848397889651701\\
99.85	0.00857241740123047\\
99.86	0.00866166286352905\\
99.87	0.00875172253261461\\
99.88	0.00884260371780208\\
99.89	0.00893431378880451\\
99.9	0.0090268601760084\\
99.91	0.00912025037073873\\
99.92	0.00921449192551288\\
99.93	0.00930959245428256\\
99.94	0.00940555963266315\\
99.95	0.00950240119814921\\
99.96	0.00960012495031559\\
99.97	0.00969873875100284\\
99.98	0.00979825052448599\\
99.99	0.00989866825762563\\
100	0.01\\
};
\addlegendentry{$q=-1$};

\addplot [color=black,solid]
  table[row sep=crcr]{%
0.01	0\\
1.01	0\\
2.01	0\\
3.01	0\\
4.01	0\\
5.01	0\\
6.01	0\\
7.01	0\\
8.01	0\\
9.01	0\\
10.01	0\\
11.01	0\\
12.01	0\\
13.01	0\\
14.01	0\\
15.01	0\\
16.01	0\\
17.01	0\\
18.01	0\\
19.01	0\\
20.01	0\\
21.01	0\\
22.01	0\\
23.01	0\\
24.01	0\\
25.01	0\\
26.01	0\\
27.01	0\\
28.01	0\\
29.01	0\\
30.01	0\\
31.01	0\\
32.01	0\\
33.01	0\\
34.01	0\\
35.01	0\\
36.01	0\\
37.01	0\\
38.01	0\\
39.01	0\\
40.01	0\\
41.01	0\\
42.01	0\\
43.01	0\\
44.01	0\\
45.01	0\\
46.01	0\\
47.01	0\\
48.01	0\\
49.01	0\\
50.01	0\\
51.01	0\\
52.01	0\\
53.01	0\\
54.01	0\\
55.01	0\\
56.01	0\\
57.01	0\\
58.01	0\\
59.01	0\\
60.01	0\\
61.01	0\\
62.01	0\\
63.01	0\\
64.01	0\\
65.01	0\\
66.01	0\\
67.01	0\\
68.01	0\\
69.01	0\\
70.01	0\\
71.01	0\\
72.01	0\\
73.01	0\\
74.01	0\\
75.01	0\\
76.01	0\\
77.01	0\\
78.01	0\\
79.01	0\\
80.01	0\\
81.01	0\\
82.01	0\\
83.01	0\\
84.01	0\\
85.01	0\\
86.01	0\\
87.01	0\\
88.01	0\\
89.01	0\\
90.01	0\\
91.01	0\\
92.01	0\\
93.01	0\\
94.01	0\\
95.01	0\\
96.01	0\\
97.01	0\\
98.01	0\\
99.01	0.00188347420949276\\
99.02	0.00193952357968626\\
99.03	0.00199595186539415\\
99.04	0.00205276251364543\\
99.05	0.00210995900922383\\
99.06	0.0021675448751667\\
99.07	0.00222552367327132\\
99.08	0.00228389900460891\\
99.09	0.00234267451004639\\
99.1	0.002401853870776\\
99.11	0.00246144080885295\\
99.12	0.00252143908774134\\
99.13	0.00258185251286822\\
99.14	0.00264268493218622\\
99.15	0.00270394023674476\\
99.16	0.00276562236126998\\
99.17	0.00282773528475357\\
99.18	0.0028902830310507\\
99.19	0.00295326966948699\\
99.2	0.00301669931547491\\
99.21	0.00308057613113965\\
99.22	0.00314490432595475\\
99.23	0.00320968815738764\\
99.24	0.00327493193155525\\
99.25	0.0033406400038899\\
99.26	0.00340681677981562\\
99.27	0.00347346671543517\\
99.28	0.00354059431822787\\
99.29	0.00360820414775853\\
99.3	0.00367630081639761\\
99.31	0.00374488899005289\\
99.32	0.00381397338891286\\
99.33	0.003883558788202\\
99.34	0.00395365001894827\\
99.35	0.00402425196876301\\
99.36	0.00409536958263347\\
99.37	0.00416700786372826\\
99.38	0.00423917187421598\\
99.39	0.00431186673610018\\
99.4	0.00438509763218818\\
99.41	0.00445886980695187\\
99.42	0.00453318856740286\\
99.43	0.00460805928398217\\
99.44	0.00468348739146472\\
99.45	0.00475947838987908\\
99.46	0.00483603784544274\\
99.47	0.00491317139151319\\
99.48	0.00499088472955523\\
99.49	0.00506918363012486\\
99.5	0.00514807393387008\\
99.51	0.00522756155254905\\
99.52	0.00530765247006595\\
99.53	0.00538835274352498\\
99.54	0.00546966850430286\\
99.55	0.00555160595914044\\
99.56	0.00563417139125364\\
99.57	0.00571737116146438\\
99.58	0.00580121170935183\\
99.59	0.00588569955442464\\
99.6	0.00597084129731456\\
99.61	0.00605664362099196\\
99.62	0.00614311329200395\\
99.63	0.00623025716173545\\
99.64	0.00631808216769411\\
99.65	0.00640659533481932\\
99.66	0.00649580377681639\\
99.67	0.00658571469751614\\
99.68	0.00667633539226089\\
99.69	0.00676767324931748\\
99.7	0.00685973575131798\\
99.71	0.00695253047672899\\
99.72	0.0070460651013502\\
99.73	0.0071403473998536\\
99.74	0.00723538524735233\\
99.75	0.00733118662099236\\
99.76	0.00742775960157701\\
99.77	0.0075251123752253\\
99.78	0.00762325323506514\\
99.79	0.00772219058296238\\
99.8	0.00782193293128693\\
99.81	0.00792248890471694\\
99.82	0.00802386724208233\\
99.83	0.0081260767982489\\
99.84	0.00822912654604423\\
99.85	0.00833302557822688\\
99.86	0.00843778310950011\\
99.87	0.00854340847857167\\
99.88	0.00864991115026125\\
99.89	0.00875730071765702\\
99.9	0.00886558690432315\\
99.91	0.00897477956655976\\
99.92	0.00908488869571742\\
99.93	0.00919592442056783\\
99.94	0.00930789700973289\\
99.95	0.009420816874174\\
99.96	0.00953469456974397\\
99.97	0.00964954079980366\\
99.98	0.00976536641790577\\
99.99	0.00988218243054826\\
100	0.01\\
};
\addlegendentry{$q=0$};

\addplot [color=blue,solid]
  table[row sep=crcr]{%
0.01	0.00162143504819226\\
1.01	0.00162143504819226\\
2.01	0.00162143504819226\\
3.01	0.00162143504819226\\
4.01	0.00162143504819226\\
5.01	0.00162143504819226\\
6.01	0.00162143504819226\\
7.01	0.00162143504819226\\
8.01	0.00162143504819226\\
9.01	0.00162143504819226\\
10.01	0.00162143504819226\\
11.01	0.00162143504819226\\
12.01	0.00162143504819226\\
13.01	0.00162143504819226\\
14.01	0.00162143504819226\\
15.01	0.00162143504819226\\
16.01	0.00162143504819226\\
17.01	0.00162143504819226\\
18.01	0.00162143504819226\\
19.01	0.00162143504819226\\
20.01	0.00162143504819226\\
21.01	0.00162143504819226\\
22.01	0.00162143504819226\\
23.01	0.00162143504819226\\
24.01	0.00162143504819226\\
25.01	0.00162143504819226\\
26.01	0.00162143504819226\\
27.01	0.00162143504819226\\
28.01	0.00162143504819226\\
29.01	0.00162143504819226\\
30.01	0.00162143504819226\\
31.01	0.00162143504819226\\
32.01	0.00162143504819226\\
33.01	0.00162143504819226\\
34.01	0.00162143504819226\\
35.01	0.00162143504819226\\
36.01	0.00162143504819226\\
37.01	0.00162143504819226\\
38.01	0.00162143504819226\\
39.01	0.00162143504819225\\
40.01	0.00162143504819221\\
41.01	0.0016214350481921\\
42.01	0.00162143504819177\\
43.01	0.00162143504819085\\
44.01	0.00162143504818824\\
45.01	0.00162143504818086\\
46.01	0.00162143504816007\\
47.01	0.00162143504810191\\
48.01	0.00162143504794009\\
49.01	0.00162143504749311\\
50.01	0.00162143504626871\\
51.01	0.00162143504294763\\
52.01	0.00162143503404522\\
53.01	0.00162143501052086\\
54.01	0.0016214349494443\\
55.01	0.00162143479433139\\
56.01	0.00162143441135684\\
57.01	0.00162143350012686\\
58.01	0.00162143143786484\\
59.01	0.00162142708863956\\
60.01	0.00162141882862004\\
61.01	0.00162140552473795\\
62.01	0.00162138889599654\\
63.01	0.00162137148007616\\
64.01	0.00162135232211013\\
65.01	0.00162132985636778\\
66.01	0.00162130145470477\\
67.01	0.0016212628304976\\
68.01	0.0016212097158356\\
69.01	0.00162114423258658\\
70.01	0.00162107328571607\\
71.01	0.00162099704961776\\
72.01	0.00162091422619813\\
73.01	0.00162082301169986\\
74.01	0.0016207217752044\\
75.01	0.00162061329719376\\
76.01	0.00162051680403822\\
77.01	0.00162045008134704\\
78.01	0.00162038202707512\\
79.01	0.00162019636512219\\
80.01	0.00161963055478129\\
81.01	0.00161790949924456\\
82.01	0.00161268396694046\\
83.01	0.0016002582561518\\
84.01	0.00158386659166719\\
85.01	0.00155951436451762\\
86.01	0.00152327426365535\\
87.01	0.00146417119193138\\
88.01	0.0013733235365292\\
89.01	0.00121366773820042\\
90.01	0.000933806065604585\\
91.01	0.000545681375545037\\
92.01	0.000157353162382811\\
93.01	0\\
94.01	0\\
95.01	0\\
96.01	0\\
97.01	0\\
98.01	0\\
99.01	0\\
99.02	0\\
99.03	0\\
99.04	0\\
99.05	0\\
99.06	0\\
99.07	0\\
99.08	0\\
99.09	0\\
99.1	0\\
99.11	0\\
99.12	0\\
99.13	0\\
99.14	0\\
99.15	0\\
99.16	0\\
99.17	0\\
99.18	0\\
99.19	0\\
99.2	0\\
99.21	0\\
99.22	0\\
99.23	0\\
99.24	0\\
99.25	0\\
99.26	0\\
99.27	0\\
99.28	0\\
99.29	0\\
99.3	0\\
99.31	0\\
99.32	0\\
99.33	0\\
99.34	0\\
99.35	0\\
99.36	0\\
99.37	0\\
99.38	0\\
99.39	0\\
99.4	0\\
99.41	0\\
99.42	0\\
99.43	0\\
99.44	0\\
99.45	0\\
99.46	0\\
99.47	0\\
99.48	0\\
99.49	0\\
99.5	0\\
99.51	0\\
99.52	0\\
99.53	0\\
99.54	0\\
99.55	0\\
99.56	0\\
99.57	0\\
99.58	0\\
99.59	0\\
99.6	0\\
99.61	0\\
99.62	0\\
99.63	0\\
99.64	0\\
99.65	0\\
99.66	0\\
99.67	0\\
99.68	0\\
99.69	0\\
99.7	0\\
99.71	0\\
99.72	0\\
99.73	0\\
99.74	0\\
99.75	0\\
99.76	0\\
99.77	0\\
99.78	0\\
99.79	0\\
99.8	0\\
99.81	0\\
99.82	0\\
99.83	0\\
99.84	0\\
99.85	0\\
99.86	0\\
99.87	0\\
99.88	0\\
99.89	0\\
99.9	0\\
99.91	0\\
99.92	0\\
99.93	0\\
99.94	0\\
99.95	0\\
99.96	0\\
99.97	0\\
99.98	0\\
99.99	0\\
100	0\\
};
\addlegendentry{$q=1$};

\addplot [color=red,solid]
  table[row sep=crcr]{%
0.01	0\\
1.01	0\\
2.01	0\\
3.01	0\\
4.01	0\\
5.01	0\\
6.01	0\\
7.01	0\\
8.01	0\\
9.01	0\\
10.01	0\\
11.01	0\\
12.01	0\\
13.01	0\\
14.01	0\\
15.01	0\\
16.01	0\\
17.01	0\\
18.01	0\\
19.01	0\\
20.01	0\\
21.01	0\\
22.01	0\\
23.01	0\\
24.01	0\\
25.01	0\\
26.01	0\\
27.01	0\\
28.01	0\\
29.01	0\\
30.01	0\\
31.01	0\\
32.01	0\\
33.01	0\\
34.01	0\\
35.01	0\\
36.01	0\\
37.01	0\\
38.01	0\\
39.01	0\\
40.01	0\\
41.01	0\\
42.01	0\\
43.01	0\\
44.01	0\\
45.01	0\\
46.01	0\\
47.01	0\\
48.01	1.73472347597681e-18\\
49.01	0\\
50.01	0\\
51.01	0\\
52.01	0\\
53.01	0\\
54.01	0\\
55.01	0\\
56.01	0\\
57.01	0\\
58.01	0\\
59.01	0\\
60.01	0\\
61.01	0\\
62.01	0\\
63.01	0\\
64.01	0\\
65.01	0\\
66.01	0\\
67.01	0\\
68.01	0\\
69.01	0\\
70.01	0\\
71.01	0\\
72.01	0\\
73.01	0\\
74.01	0\\
75.01	0\\
76.01	0\\
77.01	1.73472347597681e-18\\
78.01	0\\
79.01	0\\
80.01	0\\
81.01	0\\
82.01	0\\
83.01	0\\
84.01	0\\
85.01	0\\
86.01	0\\
87.01	0\\
88.01	0\\
89.01	0\\
90.01	0\\
91.01	0\\
92.01	0\\
93.01	0\\
94.01	0\\
95.01	0\\
96.01	0\\
97.01	0\\
98.01	0\\
99.01	0\\
99.02	0\\
99.03	0\\
99.04	0\\
99.05	0\\
99.06	0\\
99.07	0\\
99.08	0\\
99.09	0\\
99.1	0\\
99.11	0\\
99.12	0\\
99.13	0\\
99.14	0\\
99.15	0\\
99.16	0\\
99.17	0\\
99.18	0\\
99.19	0\\
99.2	0\\
99.21	0\\
99.22	0\\
99.23	0\\
99.24	0\\
99.25	0\\
99.26	0\\
99.27	0\\
99.28	0\\
99.29	0\\
99.3	0\\
99.31	0\\
99.32	0\\
99.33	0\\
99.34	0\\
99.35	0\\
99.36	0\\
99.37	0\\
99.38	0\\
99.39	0\\
99.4	0\\
99.41	0\\
99.42	0\\
99.43	0\\
99.44	0\\
99.45	0\\
99.46	0\\
99.47	0\\
99.48	0\\
99.49	0\\
99.5	0\\
99.51	0\\
99.52	0\\
99.53	0\\
99.54	0\\
99.55	0\\
99.56	0\\
99.57	0\\
99.58	0\\
99.59	0\\
99.6	0\\
99.61	0\\
99.62	0\\
99.63	0\\
99.64	0\\
99.65	0\\
99.66	0\\
99.67	0\\
99.68	0\\
99.69	0\\
99.7	0\\
99.71	0\\
99.72	0\\
99.73	0\\
99.74	0\\
99.75	0\\
99.76	0\\
99.77	0\\
99.78	0\\
99.79	0\\
99.8	0\\
99.81	0\\
99.82	0\\
99.83	0\\
99.84	0\\
99.85	0\\
99.86	0\\
99.87	0\\
99.88	0\\
99.89	0\\
99.9	0\\
99.91	0\\
99.92	0\\
99.93	0\\
99.94	0\\
99.95	0\\
99.96	0\\
99.97	0\\
99.98	0\\
99.99	0\\
100	0\\
};
\addlegendentry{$q=2$};

\addplot [color=mycolor1,solid]
  table[row sep=crcr]{%
0.01	0\\
1.01	0\\
2.01	0\\
3.01	0\\
4.01	0\\
5.01	0\\
6.01	0\\
7.01	0\\
8.01	0\\
9.01	0\\
10.01	0\\
11.01	0\\
12.01	0\\
13.01	0\\
14.01	0\\
15.01	0\\
16.01	0\\
17.01	0\\
18.01	0\\
19.01	0\\
20.01	0\\
21.01	0\\
22.01	0\\
23.01	0\\
24.01	0\\
25.01	0\\
26.01	0\\
27.01	0\\
28.01	0\\
29.01	0\\
30.01	0\\
31.01	0\\
32.01	0\\
33.01	0\\
34.01	0\\
35.01	0\\
36.01	0\\
37.01	0\\
38.01	0\\
39.01	0\\
40.01	0\\
41.01	0\\
42.01	0\\
43.01	0\\
44.01	0\\
45.01	0\\
46.01	0\\
47.01	0\\
48.01	1.73472347597681e-18\\
49.01	0\\
50.01	0\\
51.01	0\\
52.01	0\\
53.01	0\\
54.01	0\\
55.01	0\\
56.01	0\\
57.01	0\\
58.01	0\\
59.01	0\\
60.01	0\\
61.01	0\\
62.01	0\\
63.01	0\\
64.01	0\\
65.01	0\\
66.01	0\\
67.01	0\\
68.01	0\\
69.01	0\\
70.01	0\\
71.01	0\\
72.01	0\\
73.01	0\\
74.01	0\\
75.01	0\\
76.01	0\\
77.01	1.73472347597681e-18\\
78.01	0\\
79.01	0\\
80.01	0\\
81.01	0\\
82.01	0\\
83.01	0\\
84.01	0\\
85.01	0\\
86.01	0\\
87.01	0\\
88.01	0\\
89.01	0\\
90.01	0\\
91.01	0\\
92.01	0\\
93.01	0\\
94.01	0\\
95.01	0\\
96.01	0\\
97.01	0\\
98.01	0\\
99.01	0\\
99.02	0\\
99.03	0\\
99.04	0\\
99.05	0\\
99.06	0\\
99.07	0\\
99.08	0\\
99.09	0\\
99.1	0\\
99.11	0\\
99.12	0\\
99.13	0\\
99.14	0\\
99.15	0\\
99.16	0\\
99.17	0\\
99.18	0\\
99.19	0\\
99.2	0\\
99.21	0\\
99.22	0\\
99.23	0\\
99.24	0\\
99.25	0\\
99.26	0\\
99.27	0\\
99.28	0\\
99.29	0\\
99.3	0\\
99.31	0\\
99.32	0\\
99.33	0\\
99.34	0\\
99.35	0\\
99.36	0\\
99.37	0\\
99.38	0\\
99.39	0\\
99.4	0\\
99.41	0\\
99.42	0\\
99.43	0\\
99.44	0\\
99.45	0\\
99.46	0\\
99.47	0\\
99.48	0\\
99.49	0\\
99.5	0\\
99.51	0\\
99.52	0\\
99.53	0\\
99.54	0\\
99.55	0\\
99.56	0\\
99.57	0\\
99.58	0\\
99.59	0\\
99.6	0\\
99.61	0\\
99.62	0\\
99.63	0\\
99.64	0\\
99.65	0\\
99.66	0\\
99.67	0\\
99.68	0\\
99.69	0\\
99.7	0\\
99.71	0\\
99.72	0\\
99.73	0\\
99.74	0\\
99.75	0\\
99.76	0\\
99.77	0\\
99.78	0\\
99.79	0\\
99.8	0\\
99.81	0\\
99.82	0\\
99.83	0\\
99.84	0\\
99.85	0\\
99.86	0\\
99.87	0\\
99.88	0\\
99.89	0\\
99.9	0\\
99.91	0\\
99.92	0\\
99.93	0\\
99.94	0\\
99.95	0\\
99.96	0\\
99.97	0\\
99.98	0\\
99.99	0\\
100	0\\
};
\addlegendentry{$q=3$};

\addplot [color=green,solid]
  table[row sep=crcr]{%
0.01	0\\
1.01	0\\
2.01	0\\
3.01	0\\
4.01	0\\
5.01	0\\
6.01	0\\
7.01	0\\
8.01	0\\
9.01	0\\
10.01	0\\
11.01	0\\
12.01	0\\
13.01	0\\
14.01	0\\
15.01	0\\
16.01	0\\
17.01	0\\
18.01	0\\
19.01	0\\
20.01	0\\
21.01	0\\
22.01	0\\
23.01	0\\
24.01	0\\
25.01	0\\
26.01	0\\
27.01	0\\
28.01	0\\
29.01	0\\
30.01	0\\
31.01	0\\
32.01	0\\
33.01	0\\
34.01	0\\
35.01	0\\
36.01	0\\
37.01	0\\
38.01	0\\
39.01	0\\
40.01	0\\
41.01	0\\
42.01	0\\
43.01	0\\
44.01	0\\
45.01	0\\
46.01	0\\
47.01	0\\
48.01	1.73472347597681e-18\\
49.01	0\\
50.01	0\\
51.01	0\\
52.01	0\\
53.01	0\\
54.01	0\\
55.01	0\\
56.01	0\\
57.01	0\\
58.01	0\\
59.01	0\\
60.01	0\\
61.01	0\\
62.01	0\\
63.01	0\\
64.01	0\\
65.01	0\\
66.01	0\\
67.01	0\\
68.01	0\\
69.01	0\\
70.01	0\\
71.01	0\\
72.01	0\\
73.01	0\\
74.01	0\\
75.01	0\\
76.01	0\\
77.01	1.73472347597681e-18\\
78.01	0\\
79.01	0\\
80.01	0\\
81.01	0\\
82.01	0\\
83.01	0\\
84.01	0\\
85.01	0\\
86.01	0\\
87.01	0\\
88.01	0\\
89.01	0\\
90.01	0\\
91.01	0\\
92.01	0\\
93.01	0\\
94.01	0\\
95.01	0\\
96.01	0\\
97.01	0\\
98.01	0\\
99.01	0\\
99.02	0\\
99.03	0\\
99.04	0\\
99.05	0\\
99.06	0\\
99.07	0\\
99.08	0\\
99.09	0\\
99.1	0\\
99.11	0\\
99.12	0\\
99.13	0\\
99.14	0\\
99.15	0\\
99.16	0\\
99.17	0\\
99.18	0\\
99.19	0\\
99.2	0\\
99.21	0\\
99.22	0\\
99.23	0\\
99.24	0\\
99.25	0\\
99.26	0\\
99.27	0\\
99.28	0\\
99.29	0\\
99.3	0\\
99.31	0\\
99.32	0\\
99.33	0\\
99.34	0\\
99.35	0\\
99.36	0\\
99.37	0\\
99.38	0\\
99.39	0\\
99.4	0\\
99.41	0\\
99.42	0\\
99.43	0\\
99.44	0\\
99.45	0\\
99.46	0\\
99.47	0\\
99.48	0\\
99.49	0\\
99.5	0\\
99.51	0\\
99.52	0\\
99.53	0\\
99.54	0\\
99.55	0\\
99.56	0\\
99.57	0\\
99.58	0\\
99.59	0\\
99.6	0\\
99.61	0\\
99.62	0\\
99.63	0\\
99.64	0\\
99.65	0\\
99.66	0\\
99.67	0\\
99.68	0\\
99.69	0\\
99.7	0\\
99.71	0\\
99.72	0\\
99.73	0\\
99.74	0\\
99.75	0\\
99.76	0\\
99.77	0\\
99.78	0\\
99.79	0\\
99.8	0\\
99.81	0\\
99.82	0\\
99.83	0\\
99.84	0\\
99.85	0\\
99.86	0\\
99.87	0\\
99.88	0\\
99.89	0\\
99.9	0\\
99.91	0\\
99.92	0\\
99.93	0\\
99.94	0\\
99.95	0\\
99.96	0\\
99.97	0\\
99.98	0\\
99.99	0\\
100	0\\
};
\addlegendentry{$q=4$};

\end{axis}
\end{tikzpicture}%
 
  \caption{Continuous Time w/ nFPC}
\end{subfigure}%
\hfill%
\begin{subfigure}{.45\linewidth}
  \centering
  \setlength\figureheight{\linewidth} 
  \setlength\figurewidth{\linewidth}
  \tikzsetnextfilename{dp_colorbar/dm_dscr_nFPC_z15}
  % This file was created by matlab2tikz.
%
%The latest updates can be retrieved from
%  http://www.mathworks.com/matlabcentral/fileexchange/22022-matlab2tikz-matlab2tikz
%where you can also make suggestions and rate matlab2tikz.
%
\definecolor{mycolor1}{rgb}{1.00000,0.00000,1.00000}%
%
\begin{tikzpicture}[trim axis left, trim axis right]

\begin{axis}[%
width=\figurewidth,
height=\figureheight,
at={(0\figurewidth,0\figureheight)},
scale only axis,
every outer x axis line/.append style={black},
every x tick label/.append style={font=\color{black}},
xmin=0,
xmax=100,
%xlabel={Time},
every outer y axis line/.append style={black},
every y tick label/.append style={font=\color{black}},
ymin=0,
ymax=0.015,
%ylabel={Depth $\delta^+$},
axis background/.style={fill=white},
axis x line*=bottom,
axis y line*=left,
yticklabel style={
        /pgf/number format/fixed,
        /pgf/number format/precision=3
},
scaled y ticks=false,
legend style={legend cell align=left,align=left,draw=black,font=\footnotesize, at={(0.98,0.02)},anchor=south east},
every axis legend/.code={\renewcommand\addlegendentry[2][]{}}  %ignore legend locally
]
\addplot [color=green,dashed]
  table[row sep=crcr]{%
1	0.0110455900738313\\
2	0.0110580658419961\\
3	0.0110711002023015\\
4	0.0110847186127849\\
5	0.011098947011392\\
6	0.0111138115686288\\
7	0.0111293383533924\\
8	0.0111455528900724\\
9	0.0111624795806907\\
10	0.0111801409614611\\
11	0.0111985567600587\\
12	0.0112177427225301\\
13	0.011237709201904\\
14	0.0112584595946247\\
15	0.0112799890528045\\
16	0.0113022851250408\\
17	0.0113253364165301\\
18	0.0113490753902364\\
19	0.011373362331623\\
20	0.011398075956214\\
21	0.0114230352769138\\
22	0.0114479928477834\\
23	0.011471374969726\\
24	0.0114947398506091\\
25	0.0115189546331043\\
26	0.0115440435517189\\
27	0.0115700309248424\\
28	0.0115969413676922\\
29	0.0116248002340417\\
30	0.0116536398840924\\
31	0.0116834886735304\\
32	0.0117143689176055\\
33	0.0117463019183949\\
34	0.0117793081126445\\
35	0.0118134075452996\\
36	0.0118486212132247\\
37	0.0118849754575801\\
38	0.0119221351203645\\
39	0.0119603630217099\\
40	0.012000447587354\\
41	0.0120427263453424\\
42	0.0120891770130887\\
43	0.0121365777521125\\
44	0.0121848349365603\\
45	0.0122338121987253\\
46	0.0122833556034042\\
47	0.0123332665891129\\
48	0.0123833847464205\\
49	0.0124335044426007\\
50	0.0124833659630434\\
51	0.0125327020752989\\
52	0.0125812214612951\\
53	0.0126285171151238\\
54	0.0126739498597923\\
55	0.0127169436838526\\
56	0.0127550898468969\\
57	0.0127919097843038\\
58	0.0128267594141972\\
59	0.0128594959001294\\
60	0.0128926643343223\\
61	0.0129261447385129\\
62	0.0129755266230054\\
63	0.0130286155819347\\
64	0.01308106914823\\
65	0.0131327173915753\\
66	0.0131834776323277\\
67	0.013233398777312\\
68	0.0132714278428884\\
69	0.0133090211115002\\
70	0.0133464232949045\\
71	0.0133874495048413\\
72	0.0134268569837115\\
73	0.0134612592362577\\
74	0.0134943750174216\\
75	0.0135268935899649\\
76	0.0135590004382286\\
77	0.0135874994915551\\
78	0.0136129285920795\\
79	0.0136378832610413\\
80	0.0136617356552583\\
81	0.013683673960379\\
82	0.0137032409774594\\
83	0.0137208351231345\\
84	0.0137367768512176\\
85	0.0137511467620862\\
86	0.0137635406378348\\
87	0.0137746336380689\\
88	0.0137848781595571\\
89	0.0137944730921645\\
90	0.0138038563586677\\
91	0.0138132229079681\\
92	0.0138227275629359\\
93	0.0138326426430243\\
94	0.0138436332014765\\
95	0.0138574154315624\\
96	0.0138783790039005\\
97	0.0139179224337987\\
98	0.0140058180264177\\
99	0\\
100	0\\
};
\addlegendentry{$q=-4$};

\addplot [color=mycolor1,dashed]
  table[row sep=crcr]{%
1	0.00993725649741715\\
2	0.00994743908148045\\
3	0.00995810956404846\\
4	0.00996929519187683\\
5	0.00998102523150166\\
6	0.00999333122873422\\
7	0.0100062473246203\\
8	0.0100198106422231\\
9	0.010034061761987\\
10	0.0100490453075863\\
11	0.0100648106694385\\
12	0.0100814129003547\\
13	0.0100989138287825\\
14	0.0101173834499466\\
15	0.0101369016517615\\
16	0.0101575601408662\\
17	0.0101794630802216\\
18	0.0102027310663123\\
19	0.0102275077167328\\
20	0.010253960817155\\
21	0.0102822909185747\\
22	0.0103127551382908\\
23	0.0103469405010079\\
24	0.0103834219573124\\
25	0.0104214986876766\\
26	0.0104612225610747\\
27	0.0105026329818961\\
28	0.0105457418207953\\
29	0.0105904934466075\\
30	0.0106367925732236\\
31	0.010684886305683\\
32	0.0107349767371514\\
33	0.0107870782513646\\
34	0.0108411770875308\\
35	0.0108972207748731\\
36	0.0109551016116553\\
37	0.0110146171596997\\
38	0.0110679656416736\\
39	0.0111136633318416\\
40	0.0111611611350395\\
41	0.0112104653192837\\
42	0.0112615510488951\\
43	0.0113144945252057\\
44	0.0113693662707487\\
45	0.0114259491659851\\
46	0.0114840983088909\\
47	0.0115433802594881\\
48	0.0116047575583339\\
49	0.0116665806572984\\
50	0.0117303611817939\\
51	0.0117959660250449\\
52	0.011863128475386\\
53	0.011931355243533\\
54	0.0119942627636737\\
55	0.012060544277407\\
56	0.0121325121090892\\
57	0.0122056816426013\\
58	0.0122798155140526\\
59	0.0123546103874823\\
60	0.0124295605965819\\
61	0.0125040499180606\\
62	0.0125773193351345\\
63	0.0126484616186782\\
64	0.0127163888925465\\
65	0.0127800816693448\\
66	0.0128372911983206\\
67	0.0128819721583471\\
68	0.0129377934149165\\
69	0.0129931791207402\\
70	0.0130475756700926\\
71	0.0131005037865686\\
72	0.0131570590102849\\
73	0.0132174657469469\\
74	0.0132670203012515\\
75	0.0133135493019146\\
76	0.0133590018515661\\
77	0.0134065253902449\\
78	0.0134559425450279\\
79	0.0135042967524793\\
80	0.0135514186663528\\
81	0.0135969729039468\\
82	0.0136322586220696\\
83	0.0136635805896388\\
84	0.0136896446077703\\
85	0.0137137713083481\\
86	0.0137356710415909\\
87	0.0137550051541237\\
88	0.0137715958172534\\
89	0.0137865131110447\\
90	0.0137990331112737\\
91	0.0138101317004599\\
92	0.0138208052690158\\
93	0.0138315709036979\\
94	0.0138431850534444\\
95	0.0138573254549391\\
96	0.0138783790039005\\
97	0.0139179224337987\\
98	0.0140058180264177\\
99	0\\
100	0\\
};
\addlegendentry{$q=-3$};

\addplot [color=red,dashed]
  table[row sep=crcr]{%
1	0.00845092557785682\\
2	0.00845656823941849\\
3	0.00846249107576248\\
4	0.00846871059035578\\
5	0.00847524450565671\\
6	0.00848211186795421\\
7	0.00848933315827441\\
8	0.00849693040842963\\
9	0.00850492732109275\\
10	0.00851334939277234\\
11	0.00852222403875616\\
12	0.00853158071912726\\
13	0.00854145106361232\\
14	0.00855186898865415\\
15	0.00856287079869339\\
16	0.00857449531764242\\
17	0.00858678440978791\\
18	0.00859978386050781\\
19	0.00861354459996147\\
20	0.00862812540429216\\
21	0.00864359717802262\\
22	0.00866004806934649\\
23	0.00867748307320201\\
24	0.0086959640757178\\
25	0.00871558977544893\\
26	0.00873644734090162\\
27	0.0087585514122345\\
28	0.00878164749295886\\
29	0.00880459120053278\\
30	0.00882338650485376\\
31	0.0088404307082489\\
32	0.00885883486335252\\
33	0.00887888961031317\\
34	0.00890101799341021\\
35	0.00892590423691198\\
36	0.00895483132840351\\
37	0.0089906163796001\\
38	0.0090408605051884\\
39	0.00910378174186449\\
40	0.00916942222967661\\
41	0.00923795615880716\\
42	0.00930957852060985\\
43	0.00938449145442318\\
44	0.00946288840424228\\
45	0.00954494274252755\\
46	0.00963069465067402\\
47	0.00971976082099793\\
48	0.00980496696703615\\
49	0.00986937744349283\\
50	0.00993814592443868\\
51	0.0100119720976435\\
52	0.010091464182447\\
53	0.0101774151149637\\
54	0.0102763767387847\\
55	0.0103797217785801\\
56	0.0104873311297228\\
57	0.0105994090813687\\
58	0.0107161515061263\\
59	0.0108377369718138\\
60	0.010964323063165\\
61	0.0110960446215847\\
62	0.0112330516965564\\
63	0.0113745177345102\\
64	0.0115190868409274\\
65	0.0116689361195459\\
66	0.0118068149881645\\
67	0.011921020819691\\
68	0.0120361397725554\\
69	0.0121513860103602\\
70	0.0122653080138773\\
71	0.0123653408301589\\
72	0.0124650462912041\\
73	0.0125640957472384\\
74	0.0126715293894912\\
75	0.0127771856867971\\
76	0.0128766812873347\\
77	0.0129617370189626\\
78	0.0130429843392688\\
79	0.0131194818731913\\
80	0.0131910198144678\\
81	0.0132530081282395\\
82	0.0133214089066423\\
83	0.0133861696491701\\
84	0.0134477731984394\\
85	0.0135067464386294\\
86	0.0135631957202715\\
87	0.013616831142668\\
88	0.0136673924845396\\
89	0.0137135717499587\\
90	0.0137455637238906\\
91	0.0137691815452557\\
92	0.0137904700127987\\
93	0.0138105125141738\\
94	0.0138302301789764\\
95	0.0138515483058692\\
96	0.0138771841329783\\
97	0.0139179224337987\\
98	0.0140058180264177\\
99	0\\
100	0\\
};
\addlegendentry{$q=-2$};

\addplot [color=blue,dashed]
  table[row sep=crcr]{%
1	0.00681554850718535\\
2	0.00681588973092251\\
3	0.00681624790604493\\
4	0.00681662403229282\\
5	0.00681701918263336\\
6	0.00681743450916522\\
7	0.00681787124928804\\
8	0.00681833073211409\\
9	0.0068188143851154\\
10	0.00681932374100495\\
11	0.00681986044480288\\
12	0.00682042626094946\\
13	0.0068210230805929\\
14	0.00682165293148809\\
15	0.00682231800084379\\
16	0.00682302069294455\\
17	0.00682376371377654\\
18	0.00682455017083602\\
19	0.00682538367532876\\
20	0.00682626823522245\\
21	0.00682720746955256\\
22	0.0068282024376938\\
23	0.00682925825974799\\
24	0.0068303846220623\\
25	0.00683159975836174\\
26	0.00683294794645898\\
27	0.00683455738287878\\
28	0.00683683256702084\\
29	0.00684109416742677\\
30	0.00685170549342327\\
31	0.00686637214688786\\
32	0.006882117146803\\
33	0.00689901996077394\\
34	0.00691716712181518\\
35	0.00693666354110884\\
36	0.00695764583303007\\
37	0.00698024282228485\\
38	0.00700415259648881\\
39	0.00702896958052679\\
40	0.00705474132081956\\
41	0.00708151927823191\\
42	0.00710935815038134\\
43	0.0071383147682871\\
44	0.00716844468766515\\
45	0.00719978697145568\\
46	0.00723232259049564\\
47	0.00726586038409361\\
48	0.00730028131436559\\
49	0.00733651719634623\\
50	0.00737469959653813\\
51	0.00741498112945589\\
52	0.0074575578790814\\
53	0.00750268940781908\\
54	0.00755025072979743\\
55	0.00760052206817181\\
56	0.00765385351790057\\
57	0.00771066220684896\\
58	0.00777144807873026\\
59	0.00783681469444389\\
60	0.00790749797473147\\
61	0.00798441341844466\\
62	0.00806875988614439\\
63	0.00816244126907436\\
64	0.00826904811391383\\
65	0.00839598143236846\\
66	0.00856300219363242\\
67	0.00877525537066877\\
68	0.00899933696993162\\
69	0.00923504581303584\\
70	0.00947869461855486\\
71	0.00971000767935811\\
72	0.00990465437474219\\
73	0.0101095951680048\\
74	0.0103258548766747\\
75	0.0105546924270711\\
76	0.0107977370384361\\
77	0.0110620631704474\\
78	0.0113293632347916\\
79	0.0115421906689915\\
80	0.011763453473767\\
81	0.0119973082073932\\
82	0.0122334666990685\\
83	0.0124319372635047\\
84	0.0125612878259806\\
85	0.0126817069884855\\
86	0.0127970668994875\\
87	0.0129045895389597\\
88	0.0130013449527564\\
89	0.0131014145585904\\
90	0.0132126248377904\\
91	0.0133188642268457\\
92	0.0134158310019865\\
93	0.013511266137075\\
94	0.0136047815916573\\
95	0.0136968616847195\\
96	0.013792309010026\\
97	0.0138803621983305\\
98	0.0140058180264177\\
99	0\\
100	0\\
};
\addlegendentry{$q=-1$};

\addplot [color=black,solid]
  table[row sep=crcr]{%
1	0.00621792783154248\\
2	0.00621795363151064\\
3	0.006217980731081\\
4	0.00621800920860938\\
5	0.00621803914829472\\
6	0.00621807064061459\\
7	0.00621810378283594\\
8	0.0062181386795362\\
9	0.00621817544314337\\
10	0.00621821419451566\\
11	0.00621825506366069\\
12	0.0062182981909585\\
13	0.00621834372977078\\
14	0.00621839185148292\\
15	0.00621844275128388\\
16	0.00621849665212127\\
17	0.00621855381158176\\
18	0.0062186145199329\\
19	0.00621867908373063\\
20	0.00621874781355397\\
21	0.0062188211347773\\
22	0.00621890017299207\\
23	0.00621898736594281\\
24	0.00621908806271129\\
25	0.0062192143396067\\
26	0.00621939307734149\\
27	0.00621968087587299\\
28	0.00622018200515726\\
29	0.00622102489640633\\
30	0.006222065928412\\
31	0.00622318211632672\\
32	0.00622437914036729\\
33	0.00622566332427819\\
34	0.00622704153512148\\
35	0.00622851935407994\\
36	0.00623009471218396\\
37	0.00623174686285373\\
38	0.00623345815815957\\
39	0.00623523145880468\\
40	0.0062370697610722\\
41	0.00623897606923901\\
42	0.0062409531792925\\
43	0.00624300321551038\\
44	0.0062451266250654\\
45	0.00624732106274971\\
46	0.00624958270784438\\
47	0.00625192203351148\\
48	0.00625437951091605\\
49	0.0062569642653318\\
50	0.00625968667240331\\
51	0.00626255775449155\\
52	0.00626558564303955\\
53	0.00626876738880855\\
54	0.00627212092530033\\
55	0.00627566847937362\\
56	0.00627943664884499\\
57	0.00628345771370841\\
58	0.00628777189884787\\
59	0.0062924320428559\\
60	0.00629751206215357\\
61	0.00630312702680858\\
62	0.00630947542351325\\
63	0.00631689409173776\\
64	0.006325890501855\\
65	0.00633704058716358\\
66	0.00634993194303342\\
67	0.00636319983109054\\
68	0.00637669793633343\\
69	0.00639006327006573\\
70	0.00640259934075841\\
71	0.00641406232098269\\
72	0.00642600457558528\\
73	0.00643861422240916\\
74	0.00645234168823419\\
75	0.00646831096392316\\
76	0.00648942213072244\\
77	0.00652294593690508\\
78	0.0065899276595373\\
79	0.00672850623736217\\
80	0.00688408450938001\\
81	0.00706463096777866\\
82	0.00728639351234022\\
83	0.00758917216479368\\
84	0.00797753818783012\\
85	0.00838912380471636\\
86	0.00881866262566954\\
87	0.00926903541158117\\
88	0.00974152623661196\\
89	0.0102143184308963\\
90	0.0106757434510928\\
91	0.0110774318305254\\
92	0.0113693047466867\\
93	0.0116714488375805\\
94	0.0119840727133986\\
95	0.0123074839379101\\
96	0.0126303225374715\\
97	0.0130412667576705\\
98	0.0136048426333631\\
99	0\\
100	0\\
};
\addlegendentry{$q=0$};

\addplot [color=blue,solid]
  table[row sep=crcr]{%
1	0.0139983083805517\\
2	0.0139983082720273\\
3	0.0139983081574379\\
4	0.0139983080363954\\
5	0.0139983079090009\\
6	0.0139983077753098\\
7	0.0139983076341143\\
8	0.0139983074828703\\
9	0.0139983073165798\\
10	0.0139983071255936\\
11	0.0139983068924995\\
12	0.0139983065898676\\
13	0.0139983061849815\\
14	0.0139983056621921\\
15	0.013998305057157\\
16	0.0139983044295312\\
17	0.0139983037771064\\
18	0.0139983030960604\\
19	0.0139983023790402\\
20	0.0139983016112709\\
21	0.0139983007634654\\
22	0.0139982997814332\\
23	0.0139982985788664\\
24	0.013998297060026\\
25	0.0139982952283117\\
26	0.0139982933076538\\
27	0.0139982913154207\\
28	0.0139982892426073\\
29	0.0139982870772956\\
30	0.0139982848063752\\
31	0.0139982824215368\\
32	0.0139982799282731\\
33	0.0139982773437443\\
34	0.0139982746681239\\
35	0.0139982718935258\\
36	0.013998269010109\\
37	0.0139982660076548\\
38	0.0139982628758252\\
39	0.0139982596013694\\
40	0.0139982561655939\\
41	0.0139982525386186\\
42	0.0139982486663215\\
43	0.0139982444414905\\
44	0.0139982396436937\\
45	0.0139982338273148\\
46	0.0139982261626455\\
47	0.0139982153866421\\
48	0.0139982004346597\\
49	0.013998182158501\\
50	0.0139981625593857\\
51	0.0139981420252606\\
52	0.0139981206092228\\
53	0.013998098155511\\
54	0.0139980744000645\\
55	0.0139980488305364\\
56	0.0139980203327613\\
57	0.0139979865168737\\
58	0.0139979432389766\\
59	0.0139976318273502\\
60	0.0139972043922104\\
61	0.0139967529347835\\
62	0.0139962718635278\\
63	0.01399575394748\\
64	0.013995194357183\\
65	0.0139945963173215\\
66	0.0139939544103745\\
67	0.0139932487227211\\
68	0.0139924440877367\\
69	0.0139911937998497\\
70	0.0139898463883936\\
71	0.0139884163899688\\
72	0.0139868807363929\\
73	0.0139852082736702\\
74	0.013983356982005\\
75	0.0139812313196168\\
76	0.0139777517095134\\
77	0.013973616531471\\
78	0.0139691494821928\\
79	0.013964214158185\\
80	0.0139514536016275\\
81	0.0139344793329246\\
82	0.0139107111233564\\
83	0.0138838507138313\\
84	0.0138477349623043\\
85	0.0138081269900492\\
86	0.013736240631162\\
87	0.0136473980836543\\
88	0.0135470430361267\\
89	0.0133243650786069\\
90	0.0130820992896335\\
91	0.0126355529612964\\
92	0.0121478036924208\\
93	0.0116210555269654\\
94	0.0110409883808773\\
95	0.0103634447591698\\
96	0.00918448118066964\\
97	0.00722167411404017\\
98	0.00400581802641767\\
99	0\\
100	0\\
};
\addlegendentry{$q=1$};

\addplot [color=red,solid]
  table[row sep=crcr]{%
1	0.0139245392166005\\
2	0.0139245378203568\\
3	0.0139245363514915\\
4	0.0139245347992432\\
5	0.0139245331576809\\
6	0.0139245314317816\\
7	0.0139245296261191\\
8	0.0139245277306554\\
9	0.0139245257231864\\
10	0.0139245235575464\\
11	0.0139245211335468\\
12	0.0139245182386413\\
13	0.0139245144653401\\
14	0.0139245092097827\\
15	0.0139245021355708\\
16	0.0139244939094515\\
17	0.0139244853833913\\
18	0.0139244765389551\\
19	0.0139244673498546\\
20	0.0139244577696135\\
21	0.0139244476988419\\
22	0.013924436901818\\
23	0.0139244248103099\\
24	0.0139244101461545\\
25	0.0139243906702063\\
26	0.013924237896993\\
27	0.0139240662485454\\
28	0.0139238888741228\\
29	0.0139237054071376\\
30	0.0139235153981125\\
31	0.0139233183180689\\
32	0.0139231136539381\\
33	0.0139229011275037\\
34	0.0139226805548886\\
35	0.013922451429106\\
36	0.0139222130570585\\
37	0.0139219646638605\\
38	0.0139217054655661\\
39	0.0139214346687592\\
40	0.0139211513825356\\
41	0.0139208545924508\\
42	0.0139205431123277\\
43	0.0139202154806119\\
44	0.0139198697130677\\
45	0.0139195026790389\\
46	0.0139191085170946\\
47	0.0139186749789327\\
48	0.0139181780829358\\
49	0.0139175226027776\\
50	0.0139166661680981\\
51	0.0139157672933066\\
52	0.0139148310113012\\
53	0.0139138566709414\\
54	0.0139128405886136\\
55	0.0139117783846599\\
56	0.0139106646228975\\
57	0.0139094885576432\\
58	0.0139082219134894\\
59	0.0139035937526516\\
60	0.0138974969498809\\
61	0.0138910967847774\\
62	0.0138843455455878\\
63	0.0138771613390056\\
64	0.0138694092466516\\
65	0.0138603275814683\\
66	0.0138503406141517\\
67	0.0138397361302318\\
68	0.0138282880259625\\
69	0.0138112612958869\\
70	0.0137930111444582\\
71	0.0137737719994472\\
72	0.0137533280914355\\
73	0.0137313371446556\\
74	0.0137031626087889\\
75	0.0136697451461063\\
76	0.0136193502677612\\
77	0.0135604069971424\\
78	0.0134973267826613\\
79	0.0134292481808521\\
80	0.0133626213272515\\
81	0.0132526603301399\\
82	0.0130496754137029\\
83	0.0128320506032241\\
84	0.012603996387161\\
85	0.0123530187471663\\
86	0.0119599087030613\\
87	0.0114832829858986\\
88	0.010975074275643\\
89	0.0105572257988587\\
90	0.0101048122462807\\
91	0.00982588095647541\\
92	0.00954189884861001\\
93	0.00922397477561944\\
94	0.00880247522304508\\
95	0.00769651298244612\\
96	0.00673001832143179\\
97	0.00574637998611275\\
98	0.00400581802641767\\
99	0\\
100	0\\
};
\addlegendentry{$q=2$};

\addplot [color=mycolor1,solid]
  table[row sep=crcr]{%
1	0.0138659281825004\\
2	0.0138658303664556\\
3	0.0138657284708293\\
4	0.0138656221337992\\
5	0.0138655109202661\\
6	0.0138653944050993\\
7	0.0138652723292665\\
8	0.0138651444583232\\
9	0.0138650103605435\\
10	0.0138648694949254\\
11	0.0138647211728643\\
12	0.0138645643799941\\
13	0.0138643972884491\\
14	0.0138642160275985\\
15	0.0138640125094838\\
16	0.0138637564037386\\
17	0.0138634166197179\\
18	0.0138630640967847\\
19	0.0138626981450955\\
20	0.0138623180178758\\
21	0.0138619228880535\\
22	0.0138615117666275\\
23	0.0138610831948392\\
24	0.0138606340410762\\
25	0.0138601546144108\\
26	0.0138580041589181\\
27	0.0138556098224583\\
28	0.0138531361555401\\
29	0.0138505790042018\\
30	0.0138479335717341\\
31	0.0138451940841394\\
32	0.0138423533955504\\
33	0.013839403382497\\
34	0.0138363387669431\\
35	0.0138331568375606\\
36	0.0138298501579202\\
37	0.0138264085172796\\
38	0.0138228207826792\\
39	0.0138190758423261\\
40	0.0138151622390994\\
41	0.0138110672022985\\
42	0.0138067764237643\\
43	0.0138022737649553\\
44	0.0137975408243028\\
45	0.0137925561702599\\
46	0.0137872935733119\\
47	0.0137817157837313\\
48	0.0137757454786795\\
49	0.0137682035150454\\
50	0.0137582480484675\\
51	0.0137477773588493\\
52	0.0137362444823498\\
53	0.013724017890704\\
54	0.0137112819435204\\
55	0.013697989944474\\
56	0.0136840876981937\\
57	0.0136695166819895\\
58	0.0136542206517211\\
59	0.0136417400219856\\
60	0.0136299858319327\\
61	0.0136175957092921\\
62	0.0136044746383344\\
63	0.0135904563512442\\
64	0.0135751176689904\\
65	0.0135482951130647\\
66	0.0135139517849026\\
67	0.0134729206133208\\
68	0.0134299869659956\\
69	0.0133901687966344\\
70	0.0133483630254395\\
71	0.013303721695035\\
72	0.0132556063839647\\
73	0.0132029130739205\\
74	0.0130893343139838\\
75	0.0129335155393375\\
76	0.0127833506564155\\
77	0.0126277866300801\\
78	0.0124475946068981\\
79	0.0122503212283224\\
80	0.0120352698675127\\
81	0.0118382330606605\\
82	0.0115127760722277\\
83	0.0111610068162654\\
84	0.0107917217872007\\
85	0.0104026883932588\\
86	0.0101489457463948\\
87	0.00995666357602472\\
88	0.00976697868687867\\
89	0.00957643399279868\\
90	0.00939558684152857\\
91	0.00920822989858796\\
92	0.00897855980677334\\
93	0.00842245474939861\\
94	0.00748614997919531\\
95	0.00709933536237606\\
96	0.00659458310959697\\
97	0.00574637998611275\\
98	0.00400581802641767\\
99	0\\
100	0\\
};
\addlegendentry{$q=3$};

\addplot [color=green,solid]
  table[row sep=crcr]{%
1	0.0137934540964822\\
2	0.0137920822172213\\
3	0.0137906552941932\\
4	0.0137891692232524\\
5	0.0137876190349365\\
6	0.0137859986619449\\
7	0.0137843018531052\\
8	0.0137825244217185\\
9	0.0137806633430117\\
10	0.0137787127499051\\
11	0.0137766655642113\\
12	0.0137745137387683\\
13	0.0137722477079747\\
14	0.013769854326488\\
15	0.0137673076174875\\
16	0.0137642538154483\\
17	0.0137601599183347\\
18	0.0137559135221988\\
19	0.0137515064838808\\
20	0.0137469300826655\\
21	0.0137421750415046\\
22	0.0137372316698655\\
23	0.0137320903609994\\
24	0.0137267431809246\\
25	0.0137211891919787\\
26	0.0137172292745451\\
27	0.0137133172931002\\
28	0.0137092661503339\\
29	0.0137050702508529\\
30	0.0137007235066934\\
31	0.0136962183852499\\
32	0.0136915431315267\\
33	0.013686674884002\\
34	0.013681571007666\\
35	0.0136758920841648\\
36	0.0136699121098025\\
37	0.0136637467300657\\
38	0.0136573747939205\\
39	0.013650775046622\\
40	0.0136439353666763\\
41	0.0136368423034981\\
42	0.0136294807701638\\
43	0.0136218336291618\\
44	0.0136138811660326\\
45	0.0136056004152668\\
46	0.0135969649725736\\
47	0.0135879508647332\\
48	0.013578571937541\\
49	0.0135700573734081\\
50	0.0135632395167464\\
51	0.0135557162512302\\
52	0.0135400632882504\\
53	0.01352087067752\\
54	0.0135008948792161\\
55	0.0134800769465166\\
56	0.0134583318837203\\
57	0.0134355222927865\\
58	0.0134101765001236\\
59	0.0133800158311734\\
60	0.0133483586758046\\
61	0.0133150073984438\\
62	0.0132797599746472\\
63	0.01324241044225\\
64	0.0132028895590607\\
65	0.0131717196046901\\
66	0.013128624100755\\
67	0.0130202144267264\\
68	0.0129061707761707\\
69	0.0127855041042009\\
70	0.0126573033069973\\
71	0.0125204864924476\\
72	0.0123736180735568\\
73	0.0122151095257336\\
74	0.0121008584942676\\
75	0.0120087275463001\\
76	0.0119099762978608\\
77	0.0118014285062564\\
78	0.0115223760489935\\
79	0.0111965821677563\\
80	0.0108583570943391\\
81	0.0105063547064552\\
82	0.0103502268726332\\
83	0.0102060253919138\\
84	0.0100646509320887\\
85	0.00993094654733278\\
86	0.00979740506831434\\
87	0.00966105444861244\\
88	0.00952161592045265\\
89	0.00937462806720677\\
90	0.00921206089831327\\
91	0.00900704371056902\\
92	0.00838247713451574\\
93	0.00771699577961124\\
94	0.00743248223339164\\
95	0.00708638392997788\\
96	0.00659458310959697\\
97	0.00574637998611275\\
98	0.00400581802641767\\
99	0\\
100	0\\
};
\addlegendentry{$q=4$};

\end{axis}
\end{tikzpicture}%
 
  \caption{Discrete Time w/ nFPC}
\end{subfigure}\\

\leavevmode\smash{\makebox[0pt]{\hspace{-7em}% HORIZONTAL POSITION           
  \rotatebox[origin=l]{90}{\hspace{20em}% VERTICAL POSITION
    Depth $\delta^-$}%
}}\hspace{0pt plus 1filll}\null

Time (s)

\vspace{1cm}
\begin{subfigure}{\linewidth}
  \centering
  \tikzsetnextfilename{altdeltalegend}
  \definecolor{mycolor1}{rgb}{0.00000,1.00000,0.14286}%
\definecolor{mycolor2}{rgb}{0.00000,1.00000,0.28571}%
\definecolor{mycolor3}{rgb}{0.00000,1.00000,0.42857}%
\definecolor{mycolor4}{rgb}{0.00000,1.00000,0.57143}%
\definecolor{mycolor5}{rgb}{0.00000,1.00000,0.71429}%
\definecolor{mycolor6}{rgb}{0.00000,1.00000,0.85714}%
\definecolor{mycolor7}{rgb}{0.00000,1.00000,1.00000}%
\definecolor{mycolor8}{rgb}{0.00000,0.87500,1.00000}%
\definecolor{mycolor9}{rgb}{0.00000,0.62500,1.00000}%
\definecolor{mycolor10}{rgb}{0.12500,0.00000,1.00000}%
\definecolor{mycolor11}{rgb}{0.25000,0.00000,1.00000}%
\definecolor{mycolor12}{rgb}{0.37500,0.00000,1.00000}%
\definecolor{mycolor13}{rgb}{0.50000,0.00000,1.00000}%
\definecolor{mycolor14}{rgb}{0.62500,0.00000,1.00000}%
\definecolor{mycolor15}{rgb}{0.75000,0.00000,1.00000}%
\definecolor{mycolor16}{rgb}{0.87500,0.00000,1.00000}%
\definecolor{mycolor17}{rgb}{1.00000,0.00000,1.00000}%
\definecolor{mycolor18}{rgb}{1.00000,0.00000,0.87500}%
\definecolor{mycolor19}{rgb}{1.00000,0.00000,0.62500}%
\definecolor{mycolor20}{rgb}{0.85714,0.00000,0.00000}%
\definecolor{mycolor21}{rgb}{0.71429,0.00000,0.00000}%
%[trim axis left, trim axis right]
\begin{tikzpicture}
\begin{axis}[%
    hide axis,
    scale only axis,
    height=0pt,
    width=0pt,
    point meta min=-19,
    point meta max=19,
    colormap={mymap}{[1pt] rgb(0pt)=(0,1,0); rgb(7pt)=(0,1,1); rgb(15pt)=(0,0,1); rgb(23pt)=(1,0,1); rgb(31pt)=(1,0,0); rgb(38pt)=(0,0,0)},
    colorbar horizontal,
    colorbar style={width=15cm,xtick={{-15},{-10},{-5},{0},{5},{10},{15}}}
    %colorbar style={separate axis lines,every outer x axis line/.append style={black},every x tick label/.append style={font=\color{black}},every outer y axis line/.append style={black},every y tick label/.append style={font=\color{black}},yticklabels={{-19},{-17},{-15},{-13},{-11},{-9},{-7},{-5},{-3},{-1},{1},{3},{5},{7},{9},{11},{13},{15},{17},{19}}}
]%
    \addplot [draw=none] coordinates {(0,0)};
\end{axis}
\end{tikzpicture}
 
\end{subfigure}%
  \caption{Optimal sell depths $\delta^{-}$ for Markov state $Z=(\rho = +1, \Delta S = +1)$, implying heavy imbalance in favor of buy pressure, and having previously seen an upward price change. We expect the midprice to rise.}
  \label{fig:comp_dm_z15}
\end{figure}

\FloatBarrier
\subsection{Comparing Optimal Control Performance}

\begin{figure}
  \centering
\begin{subfigure}{\linewidth}
  \setlength\figureheight{0.5\linewidth} 
  \setlength\figurewidth{\linewidth}
  \tikzsetnextfilename{ORCL_comp4stoch}
  % This file was created by matlab2tikz.
%
%The latest updates can be retrieved from
%  http://www.mathworks.com/matlabcentral/fileexchange/22022-matlab2tikz-matlab2tikz
%where you can also make suggestions and rate matlab2tikz.
%
%
\begin{tikzpicture}[trim axis left, trim axis right]

\begin{axis}[%
width=\figurewidth,
height=\figureheight,
at={(0\figurewidth,0\figureheight)},
scale only axis,
every outer x axis line/.append style={black},
every x tick label/.append style={font=\color{black}},
xmin=9.5,
xmax=16,
xlabel={Time (h)},
every outer y axis line/.append style={black},
every y tick label/.append style={font=\color{black}},
ymin=-0.1,
ymax=0.25,
ylabel={Normalized PnL},
title={Strategy Performance using Optimal Stochastic Control},
axis background/.style={fill=white},
axis x line*=bottom,
axis y line*=left,
yticklabel style={
        /pgf/number format/fixed,
        /pgf/number format/precision=3
},
scaled y ticks=false,
legend style={legend cell align=left,align=left,draw=black,font=\small, legend pos=north west},
]
\addplot [color=cts_plot_color,solid,line width=1.5pt]
  table[row sep=crcr]{%
9.50027777777778	0.013029315960912\\
9.50583333333333	0.027864698814415\\
9.51138888888889	0.0658672429738636\\
9.51694444444444	0.00800661523744189\\
9.5225	-0.000899886766644353\\
9.52805555555556	0.00451407618006027\\
9.53361111111111	-0.00904705816935735\\
9.53916666666667	-0.000320570733841338\\
9.54472222222222	-0.01231214894334\\
9.55027777777778	-0.00528945411466411\\
9.55583333333333	-0.00739204427562812\\
9.56138888888889	0.000106265970203148\\
9.56694444444444	0.00217911169125714\\
9.5725	0.00646505653289955\\
9.57805555555555	0.00646505653289955\\
9.58361111111111	0.00054264018703161\\
9.58916666666667	0.0096815309218638\\
9.59472222222222	0.00296710083368474\\
9.60027777777778	0.00771941328838683\\
9.60583333333333	-0.00898343481041701\\
9.61138888888889	-0.00868731399312383\\
9.61694444444444	-0.00795347759948156\\
9.6225	-0.0061767526957216\\
9.62805555555556	-0.00344893584730434\\
9.63361111111111	-0.00653368454878604\\
9.63916666666667	-0.0060742050735985\\
9.64472222222222	-0.00504622344556753\\
9.65027777777778	-0.00466585690483796\\
9.65583333333333	-0.00332466050015554\\
9.66138888888889	-0.00560834577342344\\
9.66694444444444	-0.00442386250424964\\
9.6725	-0.00396878405855884\\
9.67805555555555	-0.0061847477406465\\
9.68361111111111	-0.00304082251739974\\
9.68916666666667	-0.00352504084216268\\
9.69472222222222	-0.00468274725120326\\
9.70027777777778	-0.00409050561661647\\
9.70583333333333	-0.00352182472404664\\
9.71138888888889	-0.00352182472404664\\
9.71694444444444	-0.00559467044510063\\
9.7225	-0.00301382881289665\\
9.72805555555555	-0.00172047049919201\\
9.73361111111111	-0.003793316220246\\
9.73916666666667	-0.003793316220246\\
9.74472222222222	-0.003793316220246\\
9.75027777777778	-0.00260883295107241\\
9.75583333333333	-0.00142434968189882\\
9.76138888888889	-0.0023969578572156\\
9.76694444444444	-0.0021797023062031\\
9.7725	-0.00496464710001442\\
9.77805555555556	-0.00489037631665236\\
9.78361111111111	-0.00480617153028672\\
9.78916666666667	-0.00282633654598255\\
9.79472222222222	-0.00340977223018559\\
9.80027777777778	-0.00266751092221102\\
9.80583333333333	-0.00266751092221102\\
9.81138888888889	-0.00416309586273107\\
9.81694444444444	-0.00430984138049258\\
9.8225	-0.00182137887900532\\
9.82805555555555	-0.00525113525629357\\
9.83361111111111	-0.00377053116982658\\
9.83916666666667	-0.00317828953523979\\
9.84472222222222	-0.00244557163978926\\
9.85027777777778	-0.00228992708335939\\
9.85583333333333	-0.00244557163978926\\
9.86138888888889	-0.00228885850428042\\
9.86694444444444	-0.00196799883741226\\
9.8725	-0.00115641567568231\\
9.87805555555556	-0.0009445405818255\\
9.88361111111111	-0.00050794350366816\\
9.88916666666667	-0.00050794350366816\\
9.89472222222222	-0.000172590448823596\\
9.90027777777778	-0.000172590448823596\\
9.90583333333333	-0.000468711266116782\\
9.91138888888889	-3.45701752148681e-05\\
9.91694444444444	-3.45701752148681e-05\\
9.9225	-3.45701752148681e-05\\
9.92805555555555	-3.45701752148681e-05\\
9.93361111111111	-3.45701752148681e-05\\
9.93916666666667	0.000122142960293971\\
9.94472222222222	0.000619183666677032\\
9.95027777777778	0.000159025846504827\\
9.95583333333333	0.0020641663274063\\
9.96138888888889	0.00628731612929569\\
9.96694444444444	-0.00347149427904405\\
9.9725	-0.00435985673092445\\
9.97805555555555	-0.00272275100331465\\
9.98361111111111	-0.00272275100331465\\
9.98916666666667	0.00131491712896905\\
9.99472222222222	0.00190715876355626\\
10.0002777777778	0.00190715876355626\\
10.0058333333333	0.00190715876355626\\
10.0113888888889	0.00190715876355626\\
10.0169444444444	-0.00203761278946184\\
10.0225	-0.00347059396614383\\
10.0280555555556	-0.00347059396614383\\
10.0336111111111	-0.00317447314885065\\
10.0391666666667	-0.00317447314885065\\
10.0447222222222	-0.00232971738015875\\
10.0502777777778	-0.00203359656286556\\
10.0558333333333	-0.0018142550357224\\
10.0613888888889	-0.0018142550357224\\
10.0669444444444	-0.00373857559092975\\
10.0725	-0.00373857559092975\\
10.0780555555556	-0.00856739625456624\\
10.0836111111111	-0.00856739625456624\\
10.0891666666667	-0.00856739625456624\\
10.0947222222222	-0.00856739625456624\\
10.1002777777778	-0.000448511857814258\\
10.1058333333333	0.00110647108986034\\
10.1113888888889	0.00110647108986034\\
10.1169444444444	-0.00218001594391252\\
10.1225	-0.00218001594391252\\
10.1280555555556	0.00285403795007586\\
10.1336111111111	0.00285403795007586\\
10.1391666666667	0.00285403795007586\\
10.1447222222222	-0.000536300867729521\\
10.1502777777778	-0.000536300867729521\\
10.1558333333333	-0.000536300867729521\\
10.1613888888889	-0.000536300867729521\\
10.1669444444444	-0.00254134460225484\\
10.1725	-0.00194910296766804\\
10.1780555555556	-0.00194910296766804\\
10.1836111111111	-0.00156367267896288\\
10.1891666666667	-0.00156367267896288\\
10.1947222222222	-0.00156367267896288\\
10.2002777777778	-0.00156367267896288\\
10.2058333333333	-0.00156367267896288\\
10.2113888888889	-0.00156367267896288\\
10.2169444444444	-0.00156367267896288\\
10.2225	-0.00137557517149398\\
10.2280555555556	-0.000882121572964113\\
10.2336111111111	-0.00160644702196507\\
10.2391666666667	-0.00138919147095215\\
10.2447222222222	-0.000712704112929402\\
10.2502777777778	-0.00109307065365897\\
10.2558333333333	-0.000500829019071751\\
10.2613888888889	0.00210909637822064\\
10.2669444444444	0.00131952196239759\\
10.2725	0.00434597910215365\\
10.2780555555556	0.00138477092921842\\
10.2836111111111	-0.000194621193305707\\
10.2891666666667	-0.00155474511477684\\
10.2947222222222	-0.00155474511477684\\
10.3002777777778	-0.000389400856758182\\
10.3058333333333	-0.000389400856758182\\
10.3113888888889	-0.00135662857491843\\
10.3169444444444	0.00160457959801596\\
10.3225	0.000716217146135561\\
10.3280555555556	0.000856693406998873\\
10.3336111111111	-3.16690448815278e-05\\
10.3391666666667	-3.16690448815278e-05\\
10.3447222222222	-0.000623910679468532\\
10.3502777777778	0.000264451772411659\\
10.3558333333333	0.00129329048515537\\
10.3613888888889	0.00129329048515537\\
10.3669444444444	0.00129329048515537\\
10.3725	0.00129329048515537\\
10.3780555555556	0.00129329048515537\\
10.3836111111111	0.00129329048515537\\
10.3891666666667	0.00129329048515537\\
10.3947222222222	0.00129329048515537\\
10.4002777777778	0.00145000362066463\\
10.4058333333333	0.00145000362066463\\
10.4113888888889	0.00204224525525143\\
10.4169444444444	0.00204224525525143\\
10.4225	0.00204224525525143\\
10.4280555555556	0.00282258439730755\\
10.4336111111111	0.0036121588131306\\
10.4391666666667	0.0036121588131306\\
10.4447222222222	0.00424128333392782\\
10.4502777777778	0.00424128333392782\\
10.4558333333333	0.00424128333392782\\
10.4613888888889	0.00424128333392782\\
10.4669444444444	0.00424128333392782\\
10.4725	0.00474927924507824\\
10.4780555555556	0.00563764169695864\\
10.4836111111111	0.00563764169695864\\
10.4891666666667	0.00496862077222182\\
10.4947222222222	0.00496862077222182\\
10.5002777777778	0.00496862077222182\\
10.5058333333333	0.00666109995254624\\
10.5113888888889	0.00622535995910163\\
10.5169444444444	0.00659911715125716\\
10.5225	0.00659911715125716\\
10.5280555555556	0.00711249351956411\\
10.5336111111111	0.00827783777758298\\
10.5391666666667	0.00857395859487638\\
10.5447222222222	0.00857395859487638\\
10.5502777777778	0.00768559614299597\\
10.5558333333333	0.00827783777758319\\
10.5613888888889	0.00827783777758319\\
10.5669444444444	0.0100354236701891\\
10.5725	0.00855481958372186\\
10.5780555555556	0.00885094040101505\\
10.5836111111111	0.00885094040101505\\
10.5891666666667	0.00914706121830824\\
10.5947222222222	0.00914706121830824\\
10.6002777777778	0.00752683795569004\\
10.6058333333333	0.00752683795569004\\
10.6113888888889	0.00618671013008574\\
10.6169444444444	0.00618671013008574\\
10.6225	0.00618671013008574\\
10.6280555555556	0.00811907959027641\\
10.6336111111111	0.00811907959027641\\
10.6391666666667	0.00682657301127855\\
10.6447222222222	0.00682657301127855\\
10.6502777777778	0.00851905219160297\\
10.6558333333333	0.00851905219160297\\
10.6613888888889	0.0104353962715148\\
10.6669444444444	0.0128043628098624\\
10.6725	0.00883431202005109\\
10.6780555555556	0.0091304328373447\\
10.6836111111111	0.0115209781086832\\
10.6891666666667	0.0115209781086832\\
10.6947222222222	0.0115209781086832\\
10.7002777777778	0.00960463402877141\\
10.7058333333333	0.0118170989259764\\
10.7113888888889	0.0144630472704625\\
10.7169444444444	0.0144630472704625\\
10.7225	0.0144630472704625\\
10.7280555555556	0.0124093405605632\\
10.7336111111111	0.0107891172979446\\
10.7391666666667	0.0107891172979446\\
10.7447222222222	0.00949661071894671\\
10.7502777777778	0.00821333941371577\\
10.7558333333333	0.00821333941371577\\
10.7613888888889	0.0130500605264679\\
10.7669444444444	0.0136423021610547\\
10.7725	0.015932403433122\\
10.7780555555556	0.015932403433122\\
10.7836111111111	0.015932403433122\\
10.7891666666667	0.015932403433122\\
10.7947222222222	0.015932403433122\\
10.8002777777778	0.0162093852392607\\
10.8058333333333	0.0165055060565539\\
10.8113888888889	0.0165055060565539\\
10.8169444444444	0.0145891619766421\\
10.8225	0.0145891619766421\\
10.8280555555556	0.0151814036112289\\
10.8336111111111	0.0151814036112289\\
10.8391666666667	0.0119181831642087\\
10.8447222222222	0.0171710136531893\\
10.8502777777778	0.0171710136531893\\
10.8558333333333	0.0157704837643672\\
10.8613888888889	0.0179231813686995\\
10.8669444444444	0.0205094034580605\\
10.8725	0.0211016450926473\\
10.8780555555556	0.0211016450926473\\
10.8836111111111	0.0211016450926473\\
10.8891666666667	0.0211016450926473\\
10.8947222222222	0.0211016450926473\\
10.9002777777778	0.0211016450926473\\
10.9058333333333	0.0211016450926473\\
10.9113888888889	0.019762635765236\\
10.9169444444444	0.019762635765236\\
10.9225	0.0178302663050457\\
10.9280555555556	0.0178302663050457\\
10.9336111111111	0.0143813430606276\\
10.9391666666667	0.0200869720081789\\
10.9447222222222	0.0144303454571024\\
10.9502777777778	0.0165874197808524\\
10.9558333333333	0.0175004115538275\\
10.9613888888889	0.0186564548556034\\
10.9669444444444	0.016963488986564\\
10.9725	0.0159672802489774\\
10.9780555555556	0.0198204348899416\\
10.9836111111111	0.0220200102323855\\
10.9891666666667	0.0223454333877305\\
10.9947222222222	0.0234407163756993\\
11.0002777777778	0.0240712167345856\\
11.0058333333333	0.0240712167345856\\
11.0113888888889	0.024663458369172\\
11.0169444444444	0.0261440624556392\\
11.0225	0.0261440624556388\\
11.0280555555556	0.0261440624556388\\
11.0336111111111	0.0261440624556388\\
11.0391666666667	0.0261440624556388\\
11.0447222222222	0.0250716170597066\\
11.0502777777778	0.0234829896634147\\
11.0558333333333	0.0258519562017623\\
11.0613888888889	0.0258519562017623\\
11.0669444444444	0.0258519562017623\\
11.0725	0.0240752312980019\\
11.0780555555556	0.0240752312980019\\
11.0836111111111	0.0240752312980019\\
11.0891666666667	0.0240752312980019\\
11.0947222222222	0.0240752312980019\\
11.1002777777778	0.0240752312980019\\
11.1058333333333	0.0293211602858479\\
11.1113888888889	0.0299134019204343\\
11.1169444444444	0.0302095227377275\\
11.1225	0.0302095227377275\\
11.1280555555556	0.0302095227377275\\
11.1336111111111	0.0302095227377275\\
11.1391666666667	0.0302095227377275\\
11.1447222222222	0.0302095227377275\\
11.1502777777778	0.0305056435550207\\
11.1558333333333	0.0287470300076698\\
11.1613888888889	0.0287470300076698\\
11.1669444444444	0.0287470300076698\\
11.1725	0.0287470300076698\\
11.1780555555556	0.0287470300076698\\
11.1836111111111	0.031551736539462\\
11.1891666666667	0.029339271642257\\
11.1947222222222	0.029339271642257\\
11.2002777777778	0.0322216145489116\\
11.2058333333333	0.0290431508249638\\
11.2113888888889	0.0290431508249638\\
11.2169444444444	0.0321439781740496\\
11.2225	0.0321439781740496\\
11.2280555555556	0.0321439781740496\\
11.2336111111111	0.0321439781740496\\
11.2391666666667	0.0321439781740496\\
11.2447222222222	0.0324400989913428\\
11.2502777777778	0.0263058075516164\\
11.2558333333333	0.0266019283689096\\
11.2613888888889	0.0267900258763785\\
11.2669444444444	0.0271899984777042\\
11.2725	0.0271899984777042\\
11.2780555555556	0.0311952136589489\\
11.2836111111111	0.0311952136589489\\
11.2891666666667	0.0311952136589489\\
11.2947222222222	0.0311952136589489\\
11.3002777777778	0.0346697982001902\\
11.3058333333333	0.0346697982001902\\
11.3113888888889	0.0346697982001902\\
11.3169444444444	0.0349659190174838\\
11.3225	0.0349659190174838\\
11.3280555555556	0.0349659190174838\\
11.3336111111111	0.0382041089965569\\
11.3391666666667	0.0382041089965569\\
11.3447222222222	0.0349467800063289\\
11.3502777777778	0.0318459526572435\\
11.3558333333333	0.029072841991778\\
11.3613888888889	0.029072841991778\\
11.3669444444444	0.0326262917992988\\
11.3725	0.0361797416068204\\
11.3780555555556	0.0361797416068204\\
11.3836111111111	0.0361797416068204\\
11.3891666666667	0.0364758624241136\\
11.3947222222222	0.0333750350750278\\
11.4002777777778	0.0333750350750278\\
11.4058333333333	0.0241893982821388\\
11.4113888888889	0.0241893982821388\\
11.4169444444444	0.0241893982821388\\
11.4225	0.0351517599787894\\
11.4280555555556	0.0351517599787894\\
11.4336111111111	0.0421606553163972\\
11.4391666666667	0.0421606553163972\\
11.4447222222222	0.0392185861546175\\
11.4502777777778	0.0392185861546175\\
11.4558333333333	0.0347776309799283\\
11.4613888888889	0.0333549519929121\\
11.4669444444444	0.0333549519929121\\
11.4725	0.0333549519929121\\
11.4780555555556	0.0333549519929121\\
11.4836111111111	0.0375994314710775\\
11.4891666666667	0.0383259118246596\\
11.4947222222222	0.0319601573414044\\
11.5002777777778	0.0319601573414044\\
11.5058333333333	0.0319601573414044\\
11.5113888888889	0.0408136966265315\\
11.5169444444444	0.0374795883461534\\
11.5225	0.0374795883461534\\
11.5280555555556	0.0374795883461534\\
11.5336111111111	0.0374795883461534\\
11.5391666666667	0.0374795883461534\\
11.5447222222222	0.039468188343771\\
11.5502777777778	0.0377757091634466\\
11.5558333333333	0.0377757091634466\\
11.5613888888889	0.04205790615083\\
11.5669444444444	0.0426310087742618\\
11.5725	0.0426310087742618\\
11.5780555555556	0.0426310087742618\\
11.5836111111111	0.0426310087742618\\
11.5891666666667	0.04071466469435\\
11.5947222222222	0.04071466469435\\
11.6002777777778	0.0387822952341598\\
11.6058333333333	0.0387822952341598\\
11.6113888888889	0.0387822952341598\\
11.6169444444444	0.0387822952341598\\
11.6225	0.036357360005159\\
11.6280555555556	0.039490311285369\\
11.6336111111111	0.0418592778237166\\
11.6391666666667	0.0418592778237166\\
11.6447222222222	0.0418592778237166\\
11.6502777777778	0.039678408792838\\
11.6558333333333	0.039678408792838\\
11.6613888888889	0.039678408792838\\
11.6669444444444	0.039678408792838\\
11.6725	0.0348079648464185\\
11.6780555555556	0.0326055883306573\\
11.6836111111111	0.0326055883306573\\
11.6891666666667	0.0389930667344428\\
11.6947222222222	0.0389930667344428\\
11.7002777777778	0.0389930667344428\\
11.7058333333333	0.0389930667344428\\
11.7113888888889	0.0389930667344428\\
11.7169444444444	0.0389930667344428\\
11.7225	0.0389930667344428\\
11.7280555555556	0.0389930667344428\\
11.7336111111111	0.037029801319599\\
11.7391666666667	0.03533042063814\\
11.7447222222222	0.0337202857569661\\
11.7502777777778	0.0306584352347767\\
11.7558333333333	0.0306584352347767\\
11.7613888888889	0.0293827693031818\\
11.7669444444444	0.0283707499006521\\
11.7725	0.0263467865507101\\
11.7780555555556	0.0253673238678274\\
11.7836111111111	0.0243878618744367\\
11.7891666666667	0.0224289394797373\\
11.7947222222222	0.0384350531775655\\
11.8002777777778	0.0389192715023276\\
11.8058333333333	0.0389192715023276\\
11.8113888888889	0.0389192715023276\\
11.8169444444444	0.0389192715023276\\
11.8225	0.0389192715023276\\
11.8280555555556	0.0389192715023276\\
11.8336111111111	0.0389192715023276\\
11.8391666666667	0.0382282418316839\\
11.8447222222222	0.0368548530176112\\
11.8502777777778	0.0407981282591218\\
11.8558333333333	0.0407981282591218\\
11.8613888888889	0.0427921087138959\\
11.8669444444444	0.0410153838101351\\
11.8725	0.0395347797236675\\
11.8780555555556	0.0377972870574581\\
11.8836111111111	0.0363940271932922\\
11.8891666666667	0.0337683502835682\\
11.8947222222222	0.0396907666294387\\
11.9002777777778	0.0396907666294387\\
11.9058333333333	0.0401842202279685\\
11.9113888888889	0.038813196447712\\
11.9169444444444	0.038813196447712\\
11.9225	0.038813196447712\\
11.9280555555556	0.0375335122503342\\
11.9336111111111	0.0375335122503342\\
11.9391666666667	0.0349783862150943\\
11.9447222222222	0.0349783862150943\\
11.9502777777778	0.0349783862150943\\
11.9558333333333	0.0349783862150943\\
11.9613888888889	0.0411077230970541\\
11.9669444444444	0.0414038439143473\\
11.9725	0.0414038439143473\\
11.9780555555556	0.0397838321460953\\
11.9836111111111	0.0420280500826657\\
11.9891666666667	0.0420280500826657\\
11.9947222222222	0.0420280500826657\\
12.0002777777778	0.0420280500826657\\
12.0058333333333	0.0376134201640654\\
12.0113888888889	0.036336482014694\\
12.0169444444444	0.0350605158646766\\
12.0225	0.0350605158646766\\
12.0280555555556	0.0350605158646766\\
12.0336111111111	0.0350605158646766\\
12.0391666666667	0.0440687959151212\\
12.0447222222222	0.0462972861926051\\
12.0502777777778	0.0462972861926051\\
12.0558333333333	0.0443649167324144\\
12.0613888888889	0.0410454249411222\\
12.0669444444444	0.0362519696805269\\
12.0725	0.0423310991619054\\
12.0780555555556	0.0390527043858619\\
12.0836111111111	0.0449751207317323\\
12.0891666666667	0.04696372072935\\
12.0947222222222	0.0454831166428823\\
12.1002777777778	0.0454831166428823\\
12.1058333333333	0.0479686596410827\\
12.1113888888889	0.0482647804583759\\
12.1169444444444	0.0466060283072736\\
12.1225	0.0454607772756505\\
12.1280555555556	0.0454607772756505\\
12.1336111111111	0.0454607772756505\\
12.1391666666667	0.0466452605448241\\
12.1447222222222	0.0484219854485849\\
12.1502777777778	0.0484219854485849\\
12.1558333333333	0.0486338605424417\\
12.1613888888889	0.0486338605424417\\
12.1669444444444	0.0486338605424417\\
12.1725	0.0486338605424417\\
12.1780555555556	0.0486338605424417\\
12.1836111111111	0.0486338605424417\\
12.1891666666667	0.0506278409972159\\
12.1947222222222	0.0495222229943221\\
12.2002777777778	0.0495222229943221\\
12.2058333333333	0.0495222229943221\\
12.2113888888889	0.0495222229943221\\
12.2169444444444	0.0495222229943221\\
12.2225	0.0495222229943221\\
12.2280555555556	0.0484219854485849\\
12.2336111111111	0.0484219854485849\\
12.2391666666667	0.0484219854485849\\
12.2447222222222	0.0484219854485849\\
12.2502777777778	0.0504948311696389\\
12.2558333333333	0.0471823402210625\\
12.2613888888889	0.0471823402210625\\
12.2669444444444	0.0446812898683003\\
12.2725	0.0446812898683003\\
12.2780555555556	0.0446812898683003\\
12.2836111111111	0.0436657478397503\\
12.2891666666667	0.0436657478397503\\
12.2947222222222	0.0436657478397503\\
12.3002777777778	0.0436657478397503\\
12.3058333333333	0.0490290952018303\\
12.3113888888889	0.0490290952018303\\
12.3169444444444	0.0490290952018303\\
12.3225	0.049395564087784\\
12.3280555555556	0.049395564087784\\
12.3336111111111	0.0511722889915448\\
12.3391666666667	0.0498321611659413\\
12.3447222222222	0.0498321611659413\\
12.3502777777778	0.0488043910322766\\
12.3558333333333	0.0488043910322766\\
12.3613888888889	0.0488043910322766\\
12.3669444444444	0.0504244028005285\\
12.3725	0.0504244028005285\\
12.3780555555556	0.0504244028005285\\
12.3836111111111	0.0504244028005285\\
12.3891666666667	0.049733373129884\\
12.3947222222222	0.049733373129884\\
12.4002777777778	0.049733373129884\\
12.4058333333333	0.049733373129884\\
12.4113888888889	0.049733373129884\\
12.4169444444444	0.049733373129884\\
12.4225	0.0486721479519002\\
12.4280555555556	0.0486721479519002\\
12.4336111111111	0.0488128745119974\\
12.4391666666667	0.0520949423518013\\
12.4447222222222	0.0541193097415373\\
12.4502777777778	0.0541193097415373\\
12.4558333333333	0.0541193097415373\\
12.4613888888889	0.0541193097415373\\
12.4669444444444	0.0536835697480927\\
12.4725	0.0555999138280037\\
12.4780555555556	0.0555999138280037\\
12.4836111111111	0.0555999138280037\\
12.4891666666667	0.0547115513761233\\
12.4947222222222	0.0547115513761233\\
12.5002777777778	0.0533151930130921\\
12.5058333333333	0.0544996762822656\\
12.5113888888889	0.0544996762822656\\
12.5169444444444	0.0544996762822656\\
12.5225	0.0562764011860264\\
12.5280555555556	0.0580531260897873\\
12.5336111111111	0.0587230040992364\\
12.5391666666667	0.0590382639276849\\
12.5447222222222	0.0590382639276849\\
12.5502777777778	0.0593343847449781\\
12.5558333333333	0.0581499014758045\\
12.5613888888889	0.0581499014758045\\
12.5669444444444	0.0581499014758045\\
12.5725	0.0581499014758045\\
12.5780555555556	0.0581499014758045\\
12.5836111111111	0.0596305055622717\\
12.5891666666667	0.0596305055622717\\
12.5947222222222	0.0584460222930977\\
12.6002777777778	0.0584460222930977\\
12.6058333333333	0.0584460222930977\\
12.6113888888889	0.0567535431127733\\
12.6169444444444	0.0554296521618153\\
12.6225	0.0544460887817308\\
12.6280555555556	0.0577034177719592\\
12.6336111111111	0.0577034177719592\\
12.6391666666667	0.0577034177719592\\
12.6447222222222	0.0577034177719592\\
12.6502777777778	0.056262045923043\\
12.6558333333333	0.056262045923043\\
12.6613888888889	0.056262045923043\\
12.6669444444444	0.0585606643927296\\
12.6725	0.0609296309310772\\
12.6780555555556	0.0609296309310772\\
12.6836111111111	0.063594718286718\\
12.6891666666667	0.063594718286718\\
12.6947222222222	0.0616783742068062\\
12.7002777777778	0.0600897468105147\\
12.7058333333333	0.0600897468105147\\
12.7113888888889	0.0600897468105147\\
12.7169444444444	0.0600897468105147\\
12.7225	0.0600897468105147\\
12.7280555555556	0.0600897468105147\\
12.7336111111111	0.0600897468105147\\
12.7391666666667	0.0600897468105147\\
12.7447222222222	0.0618664717142755\\
12.7502777777778	0.0618664717142755\\
12.7558333333333	0.0618664717142755\\
12.7613888888889	0.0618664717142755\\
12.7669444444444	0.0618664717142755\\
12.7725	0.0618664717142755\\
12.7780555555556	0.0618664717142755\\
12.7836111111111	0.0618664717142755\\
12.7891666666667	0.0618664717142755\\
12.7947222222222	0.0635643313517565\\
12.8002777777778	0.0635643313517565\\
12.8058333333333	0.0635643313517565\\
12.8113888888889	0.0635643313517565\\
12.8169444444444	0.0635643313517565\\
12.8225	0.0630509549834491\\
12.8280555555556	0.0630509549834491\\
12.8336111111111	0.0630509549834491\\
12.8391666666667	0.0630509549834491\\
12.8447222222222	0.0643750574287741\\
12.8502777777778	0.0646711782460673\\
12.8558333333333	0.0646711782460673\\
12.8613888888889	0.0649672990633605\\
12.8669444444444	0.0649672990633605\\
12.8725	0.0649672990633605\\
12.8780555555556	0.0655595406979469\\
12.8836111111111	0.0636747924843618\\
12.8891666666667	0.0636747924843618\\
12.8947222222222	0.0629837628137172\\
12.9002777777778	0.0629837628137172\\
12.9058333333333	0.0629837628137172\\
12.9113888888889	0.0640586657766798\\
12.9169444444444	0.0659918923215828\\
12.9225	0.0659918923215828\\
12.9280555555556	0.0659918923215828\\
12.9336111111111	0.0651035298697019\\
12.9391666666667	0.0643718805533304\\
12.9447222222222	0.0643718805533304\\
12.9502777777778	0.0643718805533304\\
12.9558333333333	0.0644492247756313\\
12.9613888888889	0.0660619126765131\\
12.9669444444444	0.0660619126765131\\
12.9725	0.0654696710419259\\
12.9780555555556	0.0648774294073387\\
12.9836111111111	0.0654696710419259\\
12.9891666666667	0.0654696710419259\\
12.9947222222222	0.0654696710419259\\
13.0002777777778	0.0648774294073387\\
13.0058333333333	0.0648774294073387\\
13.0113888888889	0.0648774294073387\\
13.0169444444444	0.0648774294073387\\
13.0225	0.0648774294073387\\
13.0280555555556	0.0645813085900455\\
13.0336111111111	0.0645813085900455\\
13.0391666666667	0.0645813085900455\\
13.0447222222222	0.0645813085900455\\
13.0502777777778	0.0656021819116502\\
13.0558333333333	0.0656021819116502\\
13.0613888888889	0.0654156414005681\\
13.0669444444444	0.0654156414005681\\
13.0725	0.0654156414005681\\
13.0780555555556	0.0664905443635306\\
13.0836111111111	0.0664905443635306\\
13.0891666666667	0.0664905443635306\\
13.0947222222222	0.0664905443635306\\
13.1002777777778	0.0664905443635306\\
13.1058333333333	0.0657902794191191\\
13.1113888888889	0.0657902794191191\\
13.1169444444444	0.0671508021659154\\
13.1225	0.0641396627973914\\
13.1280555555556	0.0641396627973914\\
13.1336111111111	0.0641396627973914\\
13.1391666666667	0.0645762598755488\\
13.1447222222222	0.0645762598755488\\
13.1502777777778	0.0660712433400224\\
13.1558333333333	0.0660712433400224\\
13.1613888888889	0.0669596057919028\\
13.1669444444444	0.0673007229743424\\
13.1725	0.0673007229743424\\
13.1780555555556	0.0673007229743424\\
13.1836111111111	0.0673007229743424\\
13.1891666666667	0.0673007229743424\\
13.1947222222222	0.0673007229743424\\
13.2002777777778	0.0673007229743424\\
13.2058333333333	0.0673007229743424\\
13.2113888888889	0.0673007229743424\\
13.2169444444444	0.0689837432614082\\
13.2225	0.0689837432614082\\
13.2280555555556	0.0689837432614082\\
13.2336111111111	0.0695759848959954\\
13.2391666666667	0.0718524965541448\\
13.2447222222222	0.079540803969919\\
13.2502777777778	0.0846887735270579\\
13.2558333333333	0.090273612634557\\
13.2613888888889	0.0936716849762216\\
13.2669444444444	0.0936716849762216\\
13.2725	0.0936716849762216\\
13.2780555555556	0.0971340362702832\\
13.2836111111111	0.107908310533937\\
13.2891666666667	0.108591649987875\\
13.2947222222222	0.108591649987875\\
13.3002777777778	0.108591649987875\\
13.3058333333333	0.104833177182411\\
13.3113888888889	0.109480012439754\\
13.3169444444444	0.0290587622885941\\
13.3225	0.0247565692053435\\
13.3280555555556	0.0345770345073454\\
13.3336111111111	0.030222974722969\\
13.3391666666667	0.0349609077996642\\
13.3447222222222	0.0316885488376693\\
13.3502777777778	0.0360223377872467\\
13.3558333333333	0.0360223377872467\\
13.3613888888889	0.0360223377872467\\
13.3669444444444	0.0360223377872467\\
13.3725	0.0360223377872467\\
13.3780555555556	0.0448071742699933\\
13.3836111111111	0.054580397241147\\
13.3891666666667	0.054580397241147\\
13.3947222222222	0.054580397241147\\
13.4002777777778	0.0757281453894228\\
13.4058333333333	0.0930292561501905\\
13.4113888888889	0.111124962944339\\
13.4169444444444	0.0325434600868136\\
13.4225	0.0325434600868136\\
13.4280555555556	0.0328788131416573\\
13.4336111111111	0.0283527551588189\\
13.4391666666667	0.0283527551588189\\
13.4447222222222	0.0283527551588189\\
13.4502777777778	0.0283527551588189\\
13.4558333333333	0.0380695801711545\\
13.4613888888889	0.0380695801711545\\
13.4669444444444	0.0341807773087884\\
13.4725	0.034081989272731\\
13.4780555555556	0.034081989272731\\
13.4836111111111	0.034081989272731\\
13.4891666666667	0.0389295026556373\\
13.4947222222222	0.0349081889235476\\
13.5002777777778	0.0399729578574672\\
13.5058333333333	0.0445391478268876\\
13.5113888888889	0.0305129279255054\\
13.5169444444444	0.0343624985503202\\
13.5225	0.03094866791895\\
13.5280555555556	0.03094866791895\\
13.5336111111111	0.0315600485646918\\
13.5391666666667	0.0315600485646918\\
13.5447222222222	0.0315600485646918\\
13.5502777777778	0.0318370303708304\\
13.5558333333333	0.0297641846497762\\
13.5613888888889	0.0297641846497762\\
13.5669444444444	0.0321331511881236\\
13.5725	0.037034195254535\\
13.5780555555556	0.0351732246273379\\
13.5836111111111	0.0351732246273379\\
13.5891666666667	0.0351732246273379\\
13.5947222222222	0.0351732246273379\\
13.6002777777778	0.0335361188997281\\
13.6058333333333	0.0335361188997281\\
13.6113888888889	0.0338322397170217\\
13.6169444444444	0.0338322397170217\\
13.6225	0.0338322397170217\\
13.6280555555556	0.0341283605343149\\
13.6336111111111	0.0341283605343149\\
13.6391666666667	0.0386490380599967\\
13.6447222222222	0.0386490380599967\\
13.6502777777778	0.0386490380599967\\
13.6558333333333	0.0386490380599967\\
13.6613888888889	0.0386490380599967\\
13.6669444444444	0.0386490380599967\\
13.6725	0.0386490380599967\\
13.6780555555556	0.0386490380599967\\
13.6836111111111	0.0386490380599967\\
13.6891666666667	0.0368723131562359\\
13.6947222222222	0.0350955882524751\\
13.7002777777778	0.0358392836721684\\
13.7058333333333	0.0358392836721684\\
13.7113888888889	0.0358392836721684\\
13.7169444444444	0.0379909946595026\\
13.7225	0.042433875497984\\
13.7280555555556	0.044670758221917\\
13.7336111111111	0.040713164287278\\
13.7391666666667	0.040713164287278\\
13.7447222222222	0.0432794636068618\\
13.7502777777778	0.0487168535905383\\
13.7558333333333	0.0515856068832749\\
13.7613888888889	0.0515856068832749\\
13.7669444444444	0.0515856068832749\\
13.7725	0.0515856068832749\\
13.7780555555556	0.0493803375250796\\
13.7836111111111	0.0493803375250796\\
13.7891666666667	0.0426587590085327\\
13.7947222222222	0.0394261688678796\\
13.8002777777778	0.0412817590379203\\
13.8058333333333	0.04324658144915\\
13.8113888888889	0.04324658144915\\
13.8169444444444	0.04324658144915\\
13.8225	0.04324658144915\\
13.8280555555556	0.04324658144915\\
13.8336111111111	0.0453586594077554\\
13.8391666666667	0.0453586594077554\\
13.8447222222222	0.0453586594077554\\
13.8502777777778	0.0453586594077554\\
13.8558333333333	0.0435819345039946\\
13.8613888888889	0.0435819345039946\\
13.8669444444444	0.0418052096002338\\
13.8725	0.040028484696473\\
13.8780555555556	0.040028484696473\\
13.8836111111111	0.040028484696473\\
13.8891666666667	0.0421013304175269\\
13.8947222222222	0.0421013304175269\\
13.9002777777778	0.0421013304175269\\
13.9058333333333	0.0421013304175269\\
13.9113888888889	0.0421013304175269\\
13.9169444444444	0.0406207263310593\\
13.9225	0.0423974512348201\\
13.9280555555556	0.0442584218620173\\
13.9336111111111	0.0442584218620173\\
13.9391666666667	0.0463704998206227\\
13.9447222222222	0.0463704998206227\\
13.9502777777778	0.044437061781353\\
13.9558333333333	0.044437061781353\\
13.9613888888889	0.044437061781353\\
13.9669444444444	0.0491524913498443\\
13.9725	0.0421776755492156\\
13.9780555555556	0.0421776755492156\\
13.9836111111111	0.0446308878109996\\
13.9891666666667	0.0446308878109996\\
13.9947222222222	0.0471094346555579\\
14.0002777777778	0.0555646056957093\\
14.0058333333333	0.0555646056957093\\
14.0113888888889	0.0428314105520905\\
14.0169444444444	0.0453408533513019\\
14.0225	0.0530612510118603\\
14.0280555555556	0.0558155386487054\\
14.0336111111111	0.0642300916533607\\
14.0391666666667	0.0642300916533607\\
14.0447222222222	0.0670359055690983\\
14.0502777777778	0.0727110911869758\\
14.0558333333333	0.07832275203899\\
14.0613888888889	0.0811285831985364\\
14.0669444444444	0.0867402464971505\\
14.0725	0.0895460784096122\\
14.0780555555556	0.0895460784096122\\
14.0836111111111	0.0895460784096122\\
14.0891666666667	0.0895460784096122\\
14.0947222222222	0.0895460784096122\\
14.1002777777778	0.0895460784096122\\
14.1058333333333	0.0895460784096122\\
14.1113888888889	0.0895460784096122\\
14.1169444444444	0.0895460784096122\\
14.1225	0.0895460784096122\\
14.1280555555556	0.0895460784096122\\
14.1336111111111	0.0870676252176332\\
14.1391666666667	0.0380154213309564\\
14.1447222222222	0.0474409777796086\\
14.1502777777778	0.0474409777796086\\
14.1558333333333	0.05988560330801\\
14.1613888888889	0.066088911575594\\
14.1669444444444	0.0691908463086248\\
14.1725	0.0513923367152974\\
14.1780555555556	0.054181801780957\\
14.1836111111111	0.0570134791591842\\
14.1891666666667	0.0570134791591842\\
14.1947222222222	0.0598109901519667\\
14.2002777777778	0.0598109901519667\\
14.2058333333333	0.0598109901519667\\
14.2113888888889	0.0598109901519667\\
14.2169444444444	0.0598109901519667\\
14.2225	0.048366395092591\\
14.2280555555556	0.0508758378918019\\
14.2336111111111	0.0508758378918019\\
14.2391666666667	0.0508758378918019\\
14.2447222222222	0.05660426527341\\
14.2502777777778	0.0594717688296495\\
14.2558333333333	0.0594717688296495\\
14.2613888888889	0.0594717688296495\\
14.2669444444444	0.0652102474144792\\
14.2725	0.07095036464084\\
14.2780555555556	0.0766905758325669\\
14.2836111111111	0.079560683774499\\
14.2891666666667	0.0858931432359678\\
14.2947222222222	0.0893554945300294\\
14.3002777777778	0.096280198064753\\
14.3058333333333	0.0997425500529238\\
14.3113888888889	0.0997425500529238\\
14.3169444444444	0.0997425500529238\\
14.3225	0.0997425500529238\\
14.3280555555556	0.0997425500529238\\
14.3336111111111	0.0997425500529238\\
14.3391666666667	0.0997425500529238\\
14.3447222222222	0.0997425500529238\\
14.3502777777778	0.0997425500529238\\
14.3558333333333	0.0997425500529238\\
14.3613888888889	0.0997425500529238\\
14.3669444444444	0.0997425500529238\\
14.3725	0.0997425500529238\\
14.3780555555556	0.0343909472504333\\
14.3836111111111	0.0343909472504333\\
14.3891666666667	0.0311336182602053\\
14.3947222222222	0.0311336182602053\\
14.4002777777778	0.0347659333340068\\
14.4058333333333	0.0386155039588211\\
14.4113888888889	0.0386155039588211\\
14.4169444444444	0.0423256669018516\\
14.4225	0.0423256669018516\\
14.4280555555556	0.0389485076623252\\
14.4336111111111	0.0352545815939402\\
14.4391666666667	0.0315446301452765\\
14.4447222222222	0.0353942007700912\\
14.4502777777778	0.0353942007700912\\
14.4558333333333	0.0330879637245189\\
14.4613888888889	0.0332326307567051\\
14.4669444444444	0.0332326307567051\\
14.4725	0.0332326307567051\\
14.4780555555556	0.0335287515739983\\
14.4836111111111	0.0335287515739983\\
14.4891666666667	0.0335287515739983\\
14.4947222222222	0.0335287515739983\\
14.5002777777778	0.0335287515739983\\
14.5058333333333	0.0345567332020293\\
14.5113888888889	0.0351489748366157\\
14.5169444444444	0.0351489748366157\\
14.5225	0.0351489748366157\\
14.5280555555556	0.0351489748366157\\
14.5336111111111	0.0351489748366157\\
14.5391666666667	0.0354450956539089\\
14.5447222222222	0.0360373372884952\\
14.5502777777778	0.0371026725487995\\
14.5558333333333	0.0371026725487995\\
14.5613888888889	0.0371026725487995\\
14.5669444444444	0.0376949141833859\\
14.5725	0.037991035000679\\
14.5780555555556	0.0442266661231993\\
14.5836111111111	0.0421538204021453\\
14.5891666666667	0.0421538204021453\\
14.5947222222222	0.0445620191780438\\
14.6002777777778	0.0445620191780438\\
14.6058333333333	0.0500529573236207\\
14.6113888888889	0.0500529573236207\\
14.6169444444444	0.0500529573236207\\
14.6225	0.0422254004968504\\
14.6280555555556	0.0422254004968504\\
14.6336111111111	0.0427188540953803\\
14.6391666666667	0.0427188540953803\\
14.6447222222222	0.0427188540953803\\
14.6502777777778	0.0427188540953803\\
14.6558333333333	0.0406460083743263\\
14.6613888888889	0.0406460083743263\\
14.6669444444444	0.0406460083743263\\
14.6725	0.0406460083743263\\
14.6780555555556	0.0430149749126735\\
14.6836111111111	0.0396623156702373\\
14.6891666666667	0.0396623156702373\\
14.6947222222222	0.0435118862950521\\
14.7002777777778	0.0435118862950521\\
14.7058333333333	0.0435118862950521\\
14.7113888888889	0.0435118862950521\\
14.7169444444444	0.0435118862950521\\
14.7225	0.043910138965377\\
14.7280555555556	0.046772559102254\\
14.7336111111111	0.046772559102254\\
14.7391666666667	0.0568185443962179\\
14.7447222222222	0.0409551998390425\\
14.7502777777778	0.0444006263367394\\
14.7558333333333	0.0481107892797698\\
14.7613888888889	0.041102677700784\\
14.7669444444444	0.0381414695278496\\
14.7725	0.0381414695278496\\
14.7780555555556	0.0381414695278496\\
14.7836111111111	0.0388113475372987\\
14.7891666666667	0.0388113475372987\\
14.7947222222222	0.0388113475372987\\
14.8002777777778	0.0388113475372987\\
14.8058333333333	0.0391074683545919\\
14.8113888888889	0.0394812255467474\\
14.8169444444444	0.0405536709426792\\
14.8225	0.0408497917599724\\
14.8280555555556	0.058972682805587\\
14.8336111111111	0.0621061704106449\\
14.8391666666667	0.0621061704106449\\
14.8447222222222	0.044562156401867\\
14.8502777777778	0.0475625968123528\\
14.8558333333333	0.0475625968123528\\
14.8613888888889	0.0475625968123528\\
14.8669444444444	0.0475625968123528\\
14.8725	0.0475625968123528\\
14.8780555555556	0.0475625968123528\\
14.8836111111111	0.0475625968123528\\
14.8891666666667	0.0475625968123528\\
14.8947222222222	0.0475625968123528\\
14.9002777777778	0.0426273309544213\\
14.9058333333333	0.0426273309544213\\
14.9113888888889	0.0429234517717149\\
14.9169444444444	0.0412863460441051\\
14.9225	0.0437341778487326\\
14.9280555555556	0.0437341778487326\\
14.9336111111111	0.0437341778487326\\
14.9391666666667	0.0437341778487326\\
14.9447222222222	0.0437341778487326\\
14.9502777777778	0.0437341778487326\\
14.9558333333333	0.0512300948603345\\
14.9613888888889	0.0453061017738314\\
14.9669444444444	0.0433412793626017\\
14.9725	0.0411895683752674\\
14.9780555555556	0.0411895683752674\\
14.9836111111111	0.041485689192561\\
14.9891666666667	0.0417818100098542\\
14.9947222222222	0.042374051644441\\
15.0002777777778	0.0426510334505793\\
15.0058333333333	0.0426510334505793\\
15.0113888888889	0.0408551695356635\\
15.0169444444444	0.0408551695356635\\
15.0225	0.0429280152567179\\
15.0280555555556	0.0429280152567179\\
15.0336111111111	0.0429280152567179\\
15.0391666666667	0.0454534834362077\\
15.0447222222222	0.0432241360740111\\
15.0502777777778	0.0432241360740111\\
15.0558333333333	0.0432241360740111\\
15.0613888888889	0.0458892234296523\\
15.0669444444444	0.0520525986344663\\
15.0725	0.0520525986344663\\
15.0780555555556	0.0520525986344663\\
15.0836111111111	0.0523487194517603\\
15.0891666666667	0.0458340614713044\\
15.0947222222222	0.0458340614713044\\
15.1002777777778	0.0505719945479987\\
15.1058333333333	0.0505719945479987\\
15.1113888888889	0.0505719945479987\\
15.1169444444444	0.0505719945479987\\
15.1225	0.0460206019823861\\
15.1280555555556	0.0460206019823861\\
15.1336111111111	0.0445399978959189\\
15.1391666666667	0.0448361187132121\\
15.1447222222222	0.0463955880659595\\
15.1502777777778	0.0483604104771892\\
15.1558333333333	0.0470354509471519\\
15.1613888888889	0.0476276925817387\\
15.1669444444444	0.0492963941756752\\
15.1725	0.0494937269569119\\
15.1780555555556	0.0521980465501036\\
15.1836111111111	0.0521980465501036\\
15.1891666666667	0.0507947866859378\\
15.1947222222222	0.0530649651882285\\
15.2002777777778	0.0556348516156655\\
15.2058333333333	0.0556348516156655\\
15.2113888888889	0.0513151482716764\\
15.2169444444444	0.0528349845956945\\
15.2225	0.0548484555665566\\
15.2280555555556	0.0548484555665566\\
15.2336111111111	0.0593296782718758\\
15.2391666666667	0.0593296782718758\\
15.2447222222222	0.0476894835165847\\
15.2502777777778	0.0502465475624012\\
15.2558333333333	0.0464737562277364\\
15.2613888888889	0.0471436342371855\\
15.2669444444444	0.0475692802738489\\
15.2725	0.0518554479768202\\
15.2780555555556	0.0679901975492113\\
15.2836111111111	0.0541251154568561\\
15.2891666666667	0.0541251154568561\\
15.2947222222222	0.0541251154568561\\
15.3002777777778	0.0578794851534667\\
15.3058333333333	0.0473143366591061\\
15.3113888888889	0.0473143366591061\\
15.3169444444444	0.0510558839740966\\
15.3225	0.043864738635617\\
15.3280555555556	0.0474181884431386\\
15.3336111111111	0.0509716382506602\\
15.3391666666667	0.0513083787136804\\
15.3447222222222	0.0452454861069474\\
15.3502777777778	0.0452454861069474\\
15.3558333333333	0.0572478890189066\\
15.3613888888889	0.0572478890189066\\
15.3669444444444	0.0572478890189066\\
15.3725	0.0504171459168814\\
15.3780555555556	0.0468636961093602\\
15.3836111111111	0.0471598169266534\\
15.3891666666667	0.0544071928025599\\
15.3947222222222	0.070039607002322\\
15.4002777777778	0.0602676200316367\\
15.4058333333333	0.0602676200316367\\
15.4113888888889	0.0602676200316367\\
15.4169444444444	0.0638984140614593\\
15.4225	0.0675352830809744\\
15.4280555555556	0.06457407490804\\
15.4336111111111	0.0583555377448772\\
15.4391666666667	0.055690450389236\\
15.4447222222222	0.055690450389236\\
15.4502777777778	0.0533214838508884\\
15.4558333333333	0.0543494654789198\\
15.4613888888889	0.054645586296213\\
15.4669444444444	0.054645586296213\\
15.4725	0.0567972972835472\\
15.4780555555556	0.0567972972835472\\
15.4836111111111	0.0567972972835472\\
15.4891666666667	0.0567972972835472\\
15.4947222222222	0.0550205723797864\\
15.5002777777778	0.0550205723797864\\
15.5058333333333	0.0550205723797864\\
15.5113888888889	0.0550205723797864\\
15.5169444444444	0.0570934181008408\\
15.5225	0.0593543613293646\\
15.5280555555556	0.0593543613293646\\
15.5336111111111	0.0577895115194606\\
15.5391666666667	0.0565261629840067\\
15.5447222222222	0.0580856323367537\\
15.5502777777778	0.0562300421667131\\
15.5558333333333	0.0583817531540474\\
15.5613888888889	0.0614604422249396\\
15.5669444444444	0.0614604422249396\\
15.5725	0.0691680107109068\\
15.5780555555556	0.0691680107109068\\
15.5836111111111	0.0673078665951516\\
15.5891666666667	0.0631621751530436\\
15.5947222222222	0.0631621751530436\\
15.6002777777778	0.0631621751530436\\
15.6058333333333	0.0654323536553343\\
15.6113888888889	0.0699813811871304\\
15.6169444444444	0.0624117707322071\\
15.6225	0.062110269457758\\
15.6280555555556	0.0573723363810631\\
15.6336111111111	0.0594451821021171\\
15.6391666666667	0.0582104294947824\\
15.6447222222222	0.05508795392773\\
15.6502777777778	0.0568646788314904\\
15.6558333333333	0.0578991424272762\\
15.6613888888889	0.0568851075168446\\
15.6669444444444	0.0569572842966019\\
15.6725	0.0595381668658775\\
15.6780555555556	0.0613060858192541\\
15.6836111111111	0.0613060858192541\\
15.6891666666667	0.0613060858192541\\
15.6947222222222	0.0613060858192541\\
15.7002777777778	0.0597880932496532\\
15.7058333333333	0.0591958516150668\\
15.7113888888889	0.0591958516150668\\
15.7169444444444	0.0591958516150668\\
15.7225	0.0591958516150668\\
15.7280555555556	0.0591958516150668\\
15.7336111111111	0.0609725765188264\\
15.7391666666667	0.0609725765188264\\
15.7447222222222	0.062157059788\\
15.7502777777778	0.0627493014225863\\
15.7558333333333	0.0625274513886531\\
15.7613888888889	0.0654886595615871\\
15.7669444444444	0.0640080554751199\\
15.7725	0.0614172389028407\\
15.7780555555556	0.058995872387587\\
15.7836111111111	0.0612105987917702\\
15.7891666666667	0.0606810707615872\\
15.7947222222222	0.0608622141756818\\
15.8002777777778	0.0613214525496589\\
15.8058333333333	0.0598961804597198\\
15.8113888888889	0.0603198695992478\\
15.8169444444444	0.0607466891321549\\
15.8225	0.0607466891321549\\
15.8280555555556	0.0607466891321558\\
15.8336111111111	0.0613389307667422\\
15.8391666666667	0.061289466188782\\
15.8447222222222	0.061289466188782\\
15.8502777777778	0.061289466188782\\
15.8558333333333	0.061289466188782\\
15.8613888888889	0.0617901812936087\\
15.8669444444444	0.0635041925929652\\
15.8725	0.0635041925929652\\
15.8780555555556	0.0646886758621388\\
15.8836111111111	0.0664654007658992\\
15.8891666666667	0.0664654007658992\\
15.8947222222222	0.0664654007658992\\
15.9002777777778	0.0655770383140188\\
15.9058333333333	0.0647365076368971\\
15.9113888888889	0.0647365076368971\\
15.9169444444444	0.0659009198411478\\
15.9225	0.0656047990238538\\
15.9280555555556	0.0658287514580001\\
15.9336111111111	0.0623112622310733\\
15.9391666666667	0.0631996246829537\\
15.9447222222222	0.0640879871348337\\
15.9502777777778	0.0651949775644215\\
15.9558333333333	0.0657872191990078\\
15.9613888888889	0.0657872191990078\\
15.9669444444444	0.0657872191990078\\
15.9725	0.0664421744379984\\
15.9780555555556	0.0679277885589267\\
15.9836111111111	0.0695164836690324\\
15.9891666666667	0.0695164836690324\\
15.9947222222222	0.0658513744232272\\
};
\addlegendentry{Cts Stoch Ctrl};

\addplot [color=dscr_plot_color,solid,line width=1.5pt]
  table[row sep=crcr]{%
9.50027777777778	0.013029315960912\\
9.50583333333333	0.0197210736578596\\
9.51138888888889	0.0276189596341472\\
9.51694444444444	0.00598654198937404\\
9.5225	-0.00551415581790399\\
9.52805555555556	0.0104108086787287\\
9.53361111111111	-0.0122263963391346\\
9.53916666666667	-0.0127025541271545\\
9.54472222222222	-0.0140623883911364\\
9.55027777777778	-0.0137662675738428\\
9.55583333333333	-0.0137662675738424\\
9.56138888888889	-0.0139967359346436\\
9.56694444444444	-0.0139620600248174\\
9.5725	-0.0115930934864701\\
9.57805555555555	-0.0115930934864701\\
9.58361111111111	-0.014552879132049\\
9.58916666666667	-0.00377947490332764\\
9.59472222222222	-0.00526007898979525\\
9.60027777777778	0.00171714451652137\\
9.60583333333333	-0.0112031427459798\\
9.61138888888889	-0.0114992635632729\\
9.61694444444444	-0.0202815616953708\\
9.6225	-0.0211480920173618\\
9.62805555555556	-0.0154140461277317\\
9.63361111111111	-0.0156985143372483\\
9.63916666666667	-0.0176072621445683\\
9.64472222222222	-0.0175476612771446\\
9.65027777777778	-0.0163090837386354\\
9.65583333333333	-0.0150080467053805\\
9.66138888888889	-0.0204391938990349\\
9.66694444444444	-0.0201430730817413\\
9.6725	-0.0231468726930269\\
9.67805555555555	-0.0229140906479468\\
9.68361111111111	-0.021137365744186\\
9.68916666666667	-0.0214334865614801\\
9.69472222222222	-0.0225082289158951\\
9.70027777777778	-0.0222121080986019\\
9.70583333333333	-0.0233087151505697\\
9.71138888888889	-0.0218281110641025\\
9.71694444444444	-0.0230125943332765\\
9.7225	-0.0224203526986901\\
9.72805555555555	-0.02373030875174\\
9.73361111111111	-0.0240264295690332\\
9.73916666666667	-0.0240264295690332\\
9.74472222222222	-0.0240264295690332\\
9.75027777777778	-0.0228419462998596\\
9.75583333333333	-0.021657463030686\\
9.76138888888889	-0.0233875375858746\\
9.76694444444444	-0.022499175133995\\
9.7725	-0.0248187050778202\\
9.77805555555556	-0.0248187050778206\\
9.78361111111111	-0.0248187050778206\\
9.78916666666667	-0.0236342218086474\\
9.79472222222222	-0.0242264634432342\\
9.80027777777778	-0.0214623811281357\\
9.80583333333333	-0.0226468643973093\\
9.81138888888889	-0.023962821067828\\
9.81694444444444	-0.023962821067828\\
9.8225	-0.0229389766452898\\
9.82805555555555	-0.0246783966800908\\
9.83361111111111	-0.0252706383146778\\
9.83916666666667	-0.0249745174973846\\
9.84472222222222	-0.0243822758627983\\
9.85027777777778	-0.0250261875790675\\
9.85583333333333	-0.0245370433360909\\
9.86138888888889	-0.0245356208087348\\
9.86694444444444	-0.0257201040779088\\
9.8725	-0.0243863633719369\\
9.87805555555556	-0.0246469213825024\\
9.88361111111111	-0.0234624381133284\\
9.88916666666667	-0.0234624381133284\\
9.89472222222222	-0.0233634431538714\\
9.90027777777778	-0.0233634431538714\\
9.90583333333333	-0.0239556847884582\\
9.91138888888889	-0.0233634431538714\\
9.91694444444444	-0.0227712015192842\\
9.9225	-0.023363443153871\\
9.92805555555555	-0.0234949165552158\\
9.93361111111111	-0.0233918191636048\\
9.93916666666667	-0.0221261985144385\\
9.94472222222222	-0.0224223193317321\\
9.95027777777778	-0.0233106817836125\\
9.95583333333333	-0.0213539199091254\\
9.96138888888889	-0.0210577990918318\\
9.96694444444444	-0.0267747151004731\\
9.9725	-0.0293971701524183\\
9.97805555555555	-0.0220423954738222\\
9.98361111111111	-0.0220423954738222\\
9.98916666666667	-0.0220423954738222\\
9.99472222222222	-0.0241152411948766\\
10.0002777777778	-0.0241152411948766\\
10.0058333333333	-0.0241152411948766\\
10.0113888888889	-0.0241152411948766\\
10.0169444444444	-0.0241152411948766\\
10.0225	-0.0293642785261659\\
10.0280555555556	-0.0293642785261659\\
10.0336111111111	-0.0293642785261659\\
10.0391666666667	-0.0300778928289549\\
10.0447222222222	-0.0274786974006018\\
10.0502777777778	-0.0284004760490981\\
10.0558333333333	-0.0276773539541034\\
10.0613888888889	-0.0276773539541034\\
10.0669444444444	-0.0282695955886897\\
10.0725	-0.0282695955886897\\
10.0780555555556	-0.0282695955886902\\
10.0836111111111	-0.0279734747713965\\
10.0891666666667	-0.0279734747713965\\
10.0947222222222	-0.0279734747713965\\
10.1002777777778	-0.0273812331368098\\
10.1058333333333	-0.0252728246090272\\
10.1113888888889	-0.023628270016356\\
10.1169444444444	-0.0248938906655218\\
10.1225	-0.0248938906655218\\
10.1280555555556	-0.0210017422027043\\
10.1336111111111	-0.0210017422027043\\
10.1391666666667	-0.0210017422027043\\
10.1447222222222	-0.0222929703204454\\
10.1502777777778	-0.0222929703204454\\
10.1558333333333	-0.0222929703204454\\
10.1613888888889	-0.0222929703204454\\
10.1669444444444	-0.0269739022418062\\
10.1725	-0.0268749072823484\\
10.1780555555556	-0.0268749072823484\\
10.1836111111111	-0.026874907282348\\
10.1891666666667	-0.0265787864650548\\
10.1947222222222	-0.0265787864650548\\
10.2002777777778	-0.0262826656477616\\
10.2058333333333	-0.0262826656477616\\
10.2113888888889	-0.0262826656477616\\
10.2169444444444	-0.0262826656477616\\
10.2225	-0.0262826656477616\\
10.2280555555556	-0.0259865448304684\\
10.2336111111111	-0.0268749072823488\\
10.2391666666667	-0.0268749072823493\\
10.2447222222222	-0.0256904240131752\\
10.2502777777778	-0.0265787864650556\\
10.2558333333333	-0.0253943031958821\\
10.2613888888889	-0.0248020615612953\\
10.2669444444444	-0.0259865448304693\\
10.2725	-0.0245059407440017\\
10.2780555555556	-0.024183654435323\\
10.2836111111111	-0.0262565001563774\\
10.2891666666667	-0.0298099499638977\\
10.2947222222222	-0.0298099499638977\\
10.3002777777778	-0.0252018854078178\\
10.3058333333333	-0.0252018854078178\\
10.3113888888889	-0.0240676496929706\\
10.3169444444444	-0.0255482537794378\\
10.3225	-0.0234754080583834\\
10.3280555555556	-0.0231778647137341\\
10.3336111111111	-0.0231778647137341\\
10.3391666666667	-0.0245423849536344\\
10.3447222222222	-0.0245423849536344\\
10.3502777777778	-0.0257552442325418\\
10.3558333333333	-0.0237854959030992\\
10.3613888888889	-0.0237854959030992\\
10.3669444444444	-0.0237854959030992\\
10.3725	-0.0237854959030992\\
10.3780555555556	-0.0237854959030992\\
10.3836111111111	-0.0237854959030992\\
10.3891666666667	-0.024377737537686\\
10.3947222222222	-0.024377737537686\\
10.4002777777778	-0.0234893750858052\\
10.4058333333333	-0.0240816167203924\\
10.4113888888889	-0.023193254268512\\
10.4169444444444	-0.023193254268512\\
10.4225	-0.023193254268512\\
10.4280555555556	-0.023193254268512\\
10.4336111111111	-0.023193254268512\\
10.4391666666667	-0.023193254268512\\
10.4447222222222	-0.0237854959030988\\
10.4502777777778	-0.0237854959030988\\
10.4558333333333	-0.0237854959030988\\
10.4613888888889	-0.0237854959030988\\
10.4669444444444	-0.0237499330963711\\
10.4725	-0.0234538122790771\\
10.4780555555556	-0.0230545940701729\\
10.4836111111111	-0.0230545940701729\\
10.4891666666667	-0.0236876886260399\\
10.4947222222222	-0.0233915678087463\\
10.5002777777778	-0.0233915678087463\\
10.5058333333333	-0.0230598841847246\\
10.5113888888889	-0.0235490284277017\\
10.5169444444444	-0.0235490284277012\\
10.5225	-0.0238451492449944\\
10.5280555555556	-0.0238451492449948\\
10.5336111111111	-0.0229085410393746\\
10.5391666666667	-0.0222281846268966\\
10.5447222222222	-0.0215347676938532\\
10.5502777777778	-0.0201374261702265\\
10.5558333333333	-0.0268930226296646\\
10.5613888888889	-0.0268930226296646\\
10.5669444444444	-0.0248201769086098\\
10.5725	-0.0248201769086098\\
10.5780555555556	-0.023554556259444\\
10.5836111111111	-0.0195153593612217\\
10.5891666666667	-0.0162860073409905\\
10.5947222222222	-0.0162860073409905\\
10.6002777777778	-0.0294225177913962\\
10.6058333333333	-0.0294225177913962\\
10.6113888888889	-0.0294225177913962\\
10.6169444444444	-0.0294225177913962\\
10.6225	-0.0294225177913962\\
10.6280555555556	-0.027941913704929\\
10.6336111111111	-0.027941913704929\\
10.6391666666667	-0.0291263969741034\\
10.6447222222222	-0.0291263969741034\\
10.6502777777778	-0.0279419137049298\\
10.6558333333333	-0.0279419137049298\\
10.6613888888889	-0.0264613096184622\\
10.6669444444444	-0.024643731793422\\
10.6725	-0.0261243358798896\\
10.6780555555556	-0.0279010607836504\\
10.6836111111111	-0.024015927352108\\
10.6891666666667	-0.021943081631054\\
10.6947222222222	-0.021943081631054\\
10.7002777777778	-0.0290499812460959\\
10.7058333333333	-0.0272732563423355\\
10.7113888888889	-0.0252004106212815\\
10.7169444444444	-0.0252004106212815\\
10.7225	-0.0252004106212815\\
10.7280555555556	-0.0287538604288027\\
10.7336111111111	-0.0287538604288027\\
10.7391666666667	-0.0272732563423355\\
10.7447222222222	-0.0272732563423355\\
10.7502777777778	-0.0272732563423355\\
10.7558333333333	-0.0272732563423355\\
10.7613888888889	-0.0275693771596287\\
10.7669444444444	-0.0308267061498567\\
10.7725	-0.0275693771596287\\
10.7780555555556	-0.0275693771596287\\
10.7836111111111	-0.0275693771596287\\
10.7891666666667	-0.0275693771596287\\
10.7947222222222	-0.0254965314385747\\
10.8002777777778	-0.0213508399964659\\
10.8058333333333	-0.0169746975984287\\
10.8113888888889	-0.0169746975984287\\
10.8169444444444	-0.027723161798883\\
10.8225	-0.027723161798883\\
10.8280555555556	-0.0274270409815898\\
10.8336111111111	-0.0274270409815898\\
10.8391666666667	-0.0271309201642967\\
10.8447222222222	-0.0243668378491973\\
10.8502777777778	-0.0205804221294506\\
10.8558333333333	-0.027216452778108\\
10.8613888888889	-0.0257358486916408\\
10.8669444444444	-0.0216807777295982\\
10.8725	-0.0194780634879293\\
10.8780555555556	-0.0194780634879293\\
10.8836111111111	-0.0194780634879293\\
10.8891666666667	-0.0194780634879293\\
10.8947222222222	-0.0194780634879293\\
10.9002777777778	-0.0194780634879293\\
10.9058333333333	-0.0194780634879293\\
10.9113888888889	-0.027327126873126\\
10.9169444444444	-0.027327126873126\\
10.9225	-0.027327126873126\\
10.9280555555556	-0.027327126873126\\
10.9336111111111	-0.0270310060558328\\
10.9391666666667	-0.0211285576273699\\
10.9447222222222	-0.0265398697186449\\
10.9502777777778	-0.025847210597243\\
10.9558333333333	-0.0241037390759532\\
10.9613888888889	-0.0234125024819095\\
10.9669444444444	-0.0222280192127367\\
10.9725	-0.0213396567608571\\
10.9780555555556	-0.0209274520969969\\
10.9836111111111	-0.0206313312797039\\
10.9891666666667	-0.0197443913551796\\
10.9947222222222	-0.019924428325906\\
11.0002777777778	-0.0189420229290674\\
11.0058333333333	-0.0189420229290674\\
11.0113888888889	-0.018349781294481\\
11.0169444444444	-0.0162002212841406\\
11.0225	-0.0150992799148871\\
11.0280555555556	-0.0150992799148871\\
11.0336111111111	-0.0193686774643559\\
11.0391666666667	-0.0180851992357246\\
11.0447222222222	-0.0185883421952905\\
11.0502777777778	-0.0221417920028121\\
11.0558333333333	-0.0174038589261165\\
11.0613888888889	-0.0174038589261165\\
11.0669444444444	-0.0174038589261165\\
11.0725	-0.0215086974469455\\
11.0780555555556	-0.0215086974469455\\
11.0836111111111	-0.0215086974469455\\
11.0891666666667	-0.0215086974469455\\
11.0947222222222	-0.0215086974469455\\
11.1002777777778	-0.0215086974469455\\
11.1058333333333	-0.0173274431981094\\
11.1113888888889	-0.00844431680256637\\
11.1169444444444	-0.00386950983447912\\
11.1225	-0.00386950983447912\\
11.1280555555556	-0.00386950983447912\\
11.1336111111111	-0.00386950983447912\\
11.1391666666667	-0.00386950983447912\\
11.1447222222222	-0.00166211159001829\\
11.1502777777778	-0.0198853034953411\\
11.1558333333333	-0.0185515623649962\\
11.1613888888889	-0.0185515623649962\\
11.1669444444444	-0.0164787166439422\\
11.1725	-0.0164787166439422\\
11.1780555555556	-0.0164787166439422\\
11.1836111111111	-0.0141097501055937\\
11.1891666666667	-0.0141097501055937\\
11.1947222222222	-0.0141097501055937\\
11.2002777777778	-0.0116999306459672\\
11.2058333333333	-0.0202874343474776\\
11.2113888888889	-0.0199913135301844\\
11.2169444444444	-0.0173262261745436\\
11.2225	-0.0173262261745436\\
11.2280555555556	-0.0173262261745436\\
11.2336111111111	-0.0173262261745436\\
11.2391666666667	-0.0109678411306445\\
11.2447222222222	-0.00754656163421243\\
11.2502777777778	-0.0215941085676191\\
11.2558333333333	-0.018927598684622\\
11.2613888888889	-0.0194842775124819\\
11.2669444444444	-0.0194842775124819\\
11.2725	-0.0194842775124819\\
11.2780555555556	-0.0191867341678326\\
11.2836111111111	-0.0191867341678326\\
11.2891666666667	-0.0191867341678326\\
11.2947222222222	-0.0165216468121914\\
11.3002777777778	-0.010453774690168\\
11.3058333333333	-0.010453774690168\\
11.3113888888889	-0.0196335200262648\\
11.3169444444444	-0.0172645534879172\\
11.3225	-0.0172645534879172\\
11.3280555555556	-0.0172645534879172\\
11.3336111111111	-0.014599466132276\\
11.3391666666667	-0.014599466132276\\
11.3447222222222	-0.014599466132276\\
11.3502777777778	-0.0204187850865342\\
11.3558333333333	-0.020122664269241\\
11.3613888888889	-0.0198265434519478\\
11.3669444444444	-0.0168653352790138\\
11.3725	-0.0138685642993518\\
11.3780555555556	-0.0138685642993518\\
11.3836111111111	-0.0138685642993518\\
11.3891666666667	-0.0105703823878439\\
11.3947222222222	-0.0132354697434851\\
11.4002777777778	-0.0132354697434851\\
11.4058333333333	-0.0159005570991263\\
11.4113888888889	-0.0159005570991263\\
11.4169444444444	-0.0159005570991263\\
11.4225	-0.00938589911866951\\
11.4280555555556	-0.0182695236374731\\
11.4336111111111	-0.0123471072916039\\
11.4391666666667	-0.0123471072916039\\
11.4447222222222	-0.0176772820028859\\
11.4502777777778	-0.0176772820028859\\
11.4558333333333	-0.0176772820028859\\
11.4613888888889	-0.017265077339025\\
11.4669444444444	-0.017265077339025\\
11.4725	-0.017265077339025\\
11.4780555555556	-0.017265077339025\\
11.4836111111111	-0.0154883524352646\\
11.4891666666667	-0.0157757178509607\\
11.4947222222222	-0.0169602011201347\\
11.5002777777778	-0.0169602011201347\\
11.5058333333333	-0.0169602011201347\\
11.5113888888889	-0.0123537414447838\\
11.5169444444444	-0.0165402858081716\\
11.5225	-0.0165402858081716\\
11.5280555555556	-0.0165402858081716\\
11.5336111111111	-0.0165402858081716\\
11.5391666666667	-0.0165402858081716\\
11.5447222222222	-0.0150596817217044\\
11.5502777777778	-0.0162441649908784\\
11.5558333333333	-0.0162441649908784\\
11.5613888888889	-0.014895034305756\\
11.5669444444444	-0.00757253362034365\\
11.5725	-0.00561495224876022\\
11.5780555555556	-0.00561495224876022\\
11.5836111111111	-0.00561495224876022\\
11.5891666666667	-0.0162753016713238\\
11.5947222222222	-0.0144985767675634\\
11.6002777777778	-0.0159791808540306\\
11.6058333333333	-0.0159791808540306\\
11.6113888888889	-0.0159791808540306\\
11.6169444444444	-0.0159791808540306\\
11.6225	-0.015979180854031\\
11.6280555555556	-0.0147946975848574\\
11.6336111111111	-0.0112272654025276\\
11.6391666666667	-0.0089904691752691\\
11.6447222222222	-0.0107671940790291\\
11.6502777777778	-0.0125439189827899\\
11.6558333333333	-0.0125439189827899\\
11.6613888888889	-0.00998484208636497\\
11.6669444444444	-0.00998484208636497\\
11.6725	-0.0166449558429752\\
11.6780555555556	-0.0166449558429752\\
11.6836111111111	-0.0166449558429752\\
11.6891666666667	-0.0163474124983259\\
11.6947222222222	-0.0163474124983259\\
11.7002777777778	-0.0163474124983259\\
11.7058333333333	-0.0163474124983259\\
11.7113888888889	-0.0163474124983259\\
11.7169444444444	-0.0163474124983259\\
11.7225	-0.0163474124983259\\
11.7280555555556	-0.0163474124983259\\
11.7336111111111	-0.0163474124983259\\
11.7391666666667	-0.0163474124983259\\
11.7447222222222	-0.0160512916810327\\
11.7502777777778	-0.0160512916810327\\
11.7558333333333	-0.0160512916810327\\
11.7613888888889	-0.0160512916810327\\
11.7669444444444	-0.0160512916810327\\
11.7725	-0.0154590500464464\\
11.7780555555556	-0.0154590500464464\\
11.7836111111111	-0.0151629292291532\\
11.7891666666667	-0.0154590500464472\\
11.7947222222222	-0.0157551708637396\\
11.8002777777778	-0.0142745667772728\\
11.8058333333333	-0.0142745667772728\\
11.8113888888889	-0.0142745667772728\\
11.8169444444444	-0.0142745667772728\\
11.8225	-0.0142745667772728\\
11.8280555555556	-0.0142745667772728\\
11.8336111111111	-0.0139784459599796\\
11.8391666666667	-0.0139784459599796\\
11.8447222222222	-0.0136823251426864\\
11.8502777777778	-0.0127939626908064\\
11.8558333333333	-0.0130900835081\\
11.8613888888889	-0.0106180195781423\\
11.8669444444444	-0.0118380656540444\\
11.8725	-0.0113619078660249\\
11.8780555555556	-0.0113619078660249\\
11.8836111111111	-0.0122502703179053\\
11.8891666666667	-0.0119541495006117\\
11.8947222222222	-0.0107696662314377\\
11.9002777777778	-0.0116580286833181\\
11.9058333333333	-0.0116580286833181\\
11.9113888888889	-0.0113619078660244\\
11.9169444444444	-0.0113619078660244\\
11.9225	-0.0110657870487313\\
11.9280555555556	-0.0110657870487313\\
11.9336111111111	-0.0110657870487313\\
11.9391666666667	-0.0110657870487313\\
11.9447222222222	-0.0110657870487313\\
11.9502777777778	-0.0110657870487313\\
11.9558333333333	-0.0110657870487313\\
11.9613888888889	-0.00988130377955725\\
11.9669444444444	-0.00958518296226322\\
11.9725	-0.00958518296226322\\
11.9780555555556	-0.0107696662314372\\
11.9836111111111	-0.0107696662314372\\
11.9891666666667	-0.0107696662314372\\
11.9947222222222	-0.0107696662314372\\
12.0002777777778	-0.0107696662314372\\
12.0058333333333	-0.0107609108298401\\
12.0113888888889	-0.0101686691952537\\
12.0169444444444	-0.0101686691952537\\
12.0225	-0.0101686691952537\\
12.0280555555556	-0.0101686691952537\\
12.0336111111111	-0.00987254837796054\\
12.0391666666667	-0.00839194429149334\\
12.0447222222222	-0.00661521938773254\\
12.0502777777778	-0.00661521938773254\\
12.0558333333333	-0.0101686691952537\\
12.0613888888889	-0.00987254837796054\\
12.0669444444444	-0.00987254837796096\\
12.0725	-0.00868806510878737\\
12.0780555555556	-0.00928030674337417\\
12.0836111111111	-0.00809582347420016\\
12.0891666666667	-0.00657965658100486\\
12.0947222222222	-0.00598741494641765\\
12.1002777777778	-0.00598741494641765\\
12.1058333333333	-0.00539517331183043\\
12.1113888888889	-0.00865250230205759\\
12.1169444444444	-0.00776413985017761\\
12.1225	-0.0083563814847644\\
12.1280555555556	-0.0083563814847644\\
12.1336111111111	-0.0083563814847644\\
12.1391666666667	-0.00618454080425307\\
12.1447222222222	-0.00480994717700177\\
12.1502777777778	-0.00319947456992013\\
12.1558333333333	-0.00750911651823342\\
12.1613888888889	-0.00750911651823342\\
12.1669444444444	-0.00750911651823342\\
12.1725	-0.00750911651823342\\
12.1780555555556	-0.00750911651823342\\
12.1836111111111	-0.00750911651823342\\
12.1891666666667	-0.00839747897011425\\
12.1947222222222	-0.00839747897011425\\
12.2002777777778	-0.00839747897011425\\
12.2058333333333	-0.00839747897011425\\
12.2113888888889	-0.00839747897011425\\
12.2169444444444	-0.00839747897011425\\
12.2225	-0.00839747897011425\\
12.2280555555556	-0.00839747897011425\\
12.2336111111111	-0.00839747897011425\\
12.2391666666667	-0.00839747897011425\\
12.2447222222222	-0.00569682880774494\\
12.2502777777778	-0.00407076894512434\\
12.2558333333333	-0.00825731330851213\\
12.2613888888889	-0.00825731330851213\\
12.2669444444444	-0.00824855790691499\\
12.2725	-0.00824855790691499\\
12.2780555555556	-0.00824855790691499\\
12.2836111111111	-0.00824855790691499\\
12.2891666666667	-0.00795243708962181\\
12.2947222222222	-0.00795243708962181\\
12.3002777777778	-0.00795243708962181\\
12.3058333333333	-0.00676795382044864\\
12.3113888888889	-0.00676795382044864\\
12.3169444444444	-0.00676795382044864\\
12.3225	-0.00647183300315545\\
12.3280555555556	-0.00647183300315545\\
12.3336111111111	-0.00518835477452326\\
12.3391666666667	-0.00356229491190265\\
12.3447222222222	-0.00686047682341089\\
12.3502777777778	-0.00686047682341089\\
12.3558333333333	-0.00686047682341089\\
12.3613888888889	-0.00686047682341089\\
12.3669444444444	-0.00656435600611686\\
12.3725	-0.00567599355423688\\
12.3780555555556	-0.00567599355423688\\
12.3836111111111	-0.00567599355423688\\
12.3891666666667	-0.00478763110235648\\
12.3947222222222	-0.00478763110235648\\
12.4002777777778	-0.00478763110235648\\
12.4058333333333	-0.00478763110235648\\
12.4113888888889	-0.00478763110235648\\
12.4169444444444	-0.00478763110235648\\
12.4225	-0.00385841572919879\\
12.4280555555556	-0.00385841572919879\\
12.4336111111111	-0.0035622949119056\\
12.4391666666667	-0.0027799429192411\\
12.4447222222222	-0.00154721389632853\\
12.4502777777778	-0.00154721389632853\\
12.4558333333333	-0.00154721389632853\\
12.4613888888889	-0.00154721389632853\\
12.4669444444444	-0.000954972261741312\\
12.4725	0.00169504359075149\\
12.4780555555556	0.00276806655639917\\
12.4836111111111	0.00276806655639917\\
12.4891666666667	0.00381910548056626\\
12.4947222222222	0.00381910548056626\\
12.5002777777778	-0.000785383251124116\\
12.5058333333333	0.000267626616705574\\
12.5113888888889	0.000267626616705574\\
12.5169444444444	0.000267626616705574\\
12.5225	0.000267626616705574\\
12.5280555555556	0.00164222024395687\\
12.5336111111111	-0.000134504659804773\\
12.5391666666667	-0.00161510874627197\\
12.5447222222222	-0.00102286711168475\\
12.5502777777778	0.00104997860936966\\
12.5558333333333	0.000161616157489255\\
12.5613888888889	0.000161616157489255\\
12.5669444444444	0.000161616157489255\\
12.5725	-0.000726746294390725\\
12.5780555555556	-0.000726746294390725\\
12.5836111111111	0.00121462602531894\\
12.5891666666667	0.00121462602531894\\
12.5947222222222	0.000586821584004049\\
12.6002777777778	0.000586821584004049\\
12.6058333333333	0.000586821584004049\\
12.6113888888889	0.000389695726167787\\
12.6169444444444	-0.00168314999488662\\
12.6225	-0.000498666725713034\\
12.6280555555556	0.00080190039002781\\
12.6336111111111	0.00080190039002781\\
12.6391666666667	0.00080190039002781\\
12.6447222222222	0.00080190039002781\\
12.6502777777778	-8.64620618525908e-05\\
12.6558333333333	-8.64620618525908e-05\\
12.6613888888889	-8.64620618525908e-05\\
12.6669444444444	0.0013941420246146\\
12.6725	0.00326986188783359\\
12.6780555555556	0.00326986188783359\\
12.6836111111111	-9.61719386025918e-05\\
12.6891666666667	-9.61719386025918e-05\\
12.6947222222222	-9.61719386025918e-05\\
12.7002777777778	-9.61719386025918e-05\\
12.7058333333333	-9.61719386025918e-05\\
12.7113888888889	-9.61719386025918e-05\\
12.7169444444444	-0.000984534390483203\\
12.7225	-9.61719386025918e-05\\
12.7280555555556	-9.61719386025918e-05\\
12.7336111111111	-9.61719386025918e-05\\
12.7391666666667	-9.61719386025918e-05\\
12.7447222222222	0.00108831133057142\\
12.7502777777778	0.00108831133057142\\
12.7558333333333	0.00108831133057142\\
12.7613888888889	0.00108831133057142\\
12.7669444444444	0.00108831133057142\\
12.7725	0.00108831133057142\\
12.7780555555556	0.00230835740647353\\
12.7836111111111	0.00230835740647353\\
12.7891666666667	0.00230835740647353\\
12.7947222222222	0.000868606241285726\\
12.8002777777778	0.000868606241285726\\
12.8058333333333	0.000868606241285726\\
12.8113888888889	0.000868606241285726\\
12.8169444444444	0.000868606241285726\\
12.8225	0.00175696869316571\\
12.8280555555556	0.00175696869316571\\
12.8336111111111	0.00175696869316571\\
12.8391666666667	0.00175696869316571\\
12.8447222222222	0.00264533114504611\\
12.8502777777778	0.0036791493730795\\
12.8558333333333	0.0036791493730795\\
12.8613888888889	0.00502758314845756\\
12.8669444444444	0.00642095101800634\\
12.8725	0.00642095101800634\\
12.8780555555556	0.00215999522642612\\
12.8836111111111	0.00156775359183933\\
12.8891666666667	0.00156775359183933\\
12.8947222222222	0.00156775359183933\\
12.9002777777778	0.00127163277454614\\
12.9058333333333	0.00186387440913293\\
12.9113888888889	0.00249167885044783\\
12.9169444444444	0.00110714217710144\\
12.9225	0.00110714217710144\\
12.9280555555556	0.00110714217710144\\
12.9336111111111	0.00110714217710144\\
12.9391666666667	0.00110714217710144\\
12.9447222222222	0.00110714217710144\\
12.9502777777778	0.00110714217710144\\
12.9558333333333	0.00110714217710102\\
12.9613888888889	0.00154871877054804\\
12.9669444444444	0.00154871877054804\\
12.9725	0.000767789305238789\\
12.9780555555556	0.000883873151805625\\
12.9836111111111	0.00117999396909923\\
12.9891666666667	0.00117999396909923\\
12.9947222222222	0.00117999396909923\\
13.0002777777778	0.000767789305238368\\
13.0058333333333	0.000767789305238368\\
13.0113888888889	0.000767789305238368\\
13.0169444444444	0.000767789305238368\\
13.0225	0.000767789305238368\\
13.0280555555556	0.00106391012253197\\
13.0336111111111	0.00106391012253197\\
13.0391666666667	0.00106391012253197\\
13.0447222222222	0.00106391012253197\\
13.0502777777778	0.00116290508198932\\
13.0558333333333	0.00116290508198932\\
13.0613888888889	0.00142346309255483\\
13.0669444444444	0.00142346309255483\\
13.0725	0.00142346309255483\\
13.0780555555556	0.00175514671657654\\
13.0836111111111	0.00175514671657654\\
13.0891666666667	0.00175514671657654\\
13.0947222222222	0.00215436492548072\\
13.1002777777778	0.00215436492548072\\
13.1058333333333	0.000733361706436427\\
13.1113888888889	0.000733361706436427\\
13.1169444444444	0.0015980413331863\\
13.1225	-0.000222911703794373\\
13.1280555555556	0.000423424200127612\\
13.1336111111111	0.000423424200127612\\
13.1391666666667	0.000836098781502938\\
13.1447222222222	0.000836098781502938\\
13.1502777777778	0.00230762431726446\\
13.1558333333333	0.00230762431726446\\
13.1613888888889	0.00414866144600727\\
13.1669444444444	0.00446873486370231\\
13.1725	0.00446873486370231\\
13.1780555555556	0.00446873486370231\\
13.1836111111111	0.00446873486370231\\
13.1891666666667	0.00446873486370231\\
13.1947222222222	0.00446873486370231\\
13.2002777777778	0.00446873486370231\\
13.2058333333333	0.00446873486370231\\
13.2113888888889	0.006145650010137\\
13.2169444444444	0.00506097649828868\\
13.2225	0.00506097649828868\\
13.2280555555556	0.00703401246201656\\
13.2336111111111	0.0132503498680701\\
13.2391666666667	0.0180856610091299\\
13.2447222222222	0.0257551296658649\\
13.2502777777778	0.0257551296658649\\
13.2558333333333	0.0308865680693745\\
13.2613888888889	0.0308865680693745\\
13.2669444444444	0.0308865680693745\\
13.2725	0.0308865680693745\\
13.2780555555556	0.0308865680693745\\
13.2836111111111	0.0308865680693745\\
13.2891666666667	0.0315781778454274\\
13.2947222222222	0.0315781778454274\\
13.3002777777778	0.0315781778454274\\
13.3058333333333	0.0278279753620791\\
13.3113888888889	0.0324665402973103\\
13.3169444444444	-0.052689199289742\\
13.3225	-0.0523930784724488\\
13.3280555555556	-0.0378837488844701\\
13.3336111111111	-0.0513756303712435\\
13.3391666666667	-0.0386424352276247\\
13.3447222222222	-0.0456803249996331\\
13.3502777777778	-0.041405547908199\\
13.3558333333333	-0.041405547908199\\
13.3613888888889	-0.041405547908199\\
13.3669444444444	-0.041405547908199\\
13.3725	-0.041405547908199\\
13.3780555555556	-0.0326515994981216\\
13.3836111111111	-0.0179892470697555\\
13.3891666666667	-0.0179892470697555\\
13.3947222222222	-0.0131119384033289\\
13.4002777777778	0.0135092079775066\\
13.4058333333333	0.0308105405350936\\
13.4113888888889	0.0428908857111296\\
13.4169444444444	-0.0583824338032411\\
13.4225	-0.0583824338032411\\
13.4280555555556	-0.0580863129859479\\
13.4336111111111	-0.0580863129859479\\
13.4391666666667	-0.0580863129859479\\
13.4447222222222	-0.0580863129859479\\
13.4502777777778	-0.0580863129859479\\
13.4558333333333	-0.0485563525632219\\
13.4613888888889	-0.0485563525632219\\
13.4669444444444	-0.0490195085484834\\
13.4725	-0.0592172924451868\\
13.4780555555556	-0.0592172924451868\\
13.4836111111111	-0.0592172924451868\\
13.4891666666667	-0.0544793593684916\\
13.4947222222222	-0.0583289299933064\\
13.5002777777778	-0.0452529636838222\\
13.5058333333333	-0.0407137753337863\\
13.5113888888889	-0.0554643644324611\\
13.5169444444444	-0.0477652231828319\\
13.5225	-0.0547050876299068\\
13.5280555555556	-0.0547050876299068\\
13.5336111111111	-0.0514477586396792\\
13.5391666666667	-0.0514477586396792\\
13.5447222222222	-0.0514477586396792\\
13.5502777777778	-0.0535206043607332\\
13.5558333333333	-0.0535206043607332\\
13.5613888888889	-0.05322448354344\\
13.5669444444444	-0.0529283627261468\\
13.5725	-0.0457207184931966\\
13.5780555555556	-0.0519392556563586\\
13.5836111111111	-0.0519392556563586\\
13.5891666666667	-0.0519392556563586\\
13.5947222222222	-0.0519392556563586\\
13.6002777777778	-0.0519392556563586\\
13.6058333333333	-0.0519392556563586\\
13.6113888888889	-0.0516431348390654\\
13.6169444444444	-0.0516431348390654\\
13.6225	-0.0516431348390654\\
13.6280555555556	-0.0516431348390654\\
13.6336111111111	-0.0516431348390654\\
13.6391666666667	-0.0472013225796634\\
13.6447222222222	-0.0472013225796634\\
13.6502777777778	-0.0472013225796634\\
13.6558333333333	-0.0472013225796634\\
13.6613888888889	-0.0472013225796634\\
13.6669444444444	-0.0472013225796634\\
13.6725	-0.0472013225796634\\
13.6780555555556	-0.0472013225796634\\
13.6836111111111	-0.0472013225796634\\
13.6891666666667	-0.0472013225796634\\
13.6947222222222	-0.0513470140217714\\
13.7002777777778	-0.0504430702881821\\
13.7058333333333	-0.0504430702881821\\
13.7113888888889	-0.0504430702881821\\
13.7169444444444	-0.0483702245671281\\
13.7225	-0.0462432845767385\\
13.7280555555556	-0.04553448917807\\
13.7336111111111	-0.047607334899124\\
13.7391666666667	-0.0451842740914412\\
13.7447222222222	-0.0451842740914412\\
13.7502777777778	-0.0426488628445266\\
13.7558333333333	-0.0398016436737716\\
13.7613888888889	-0.0398016436737716\\
13.7669444444444	-0.0398016436737716\\
13.7725	-0.0398016436737716\\
13.7780555555556	-0.0419718654854604\\
13.7836111111111	-0.0419718654854604\\
13.7891666666667	-0.0487421368702952\\
13.7947222222222	-0.0486479772372455\\
13.8002777777778	-0.0468712523334847\\
13.8058333333333	-0.0449654417803982\\
13.8113888888889	-0.0449654417803982\\
13.8169444444444	-0.0449654417803982\\
13.8225	-0.0449654417803982\\
13.8280555555556	-0.0449654417803982\\
13.8336111111111	-0.0428925960593442\\
13.8391666666667	-0.0428925960593442\\
13.8447222222222	-0.0428925960593442\\
13.8502777777778	-0.0428925960593442\\
13.8558333333333	-0.0464460458668658\\
13.8613888888889	-0.0464460458668658\\
13.8669444444444	-0.0464460458668658\\
13.8725	-0.0482227707706266\\
13.8780555555556	-0.0482227707706266\\
13.8836111111111	-0.0479266499533334\\
13.8891666666667	-0.0455576834149863\\
13.8947222222222	-0.0455576834149863\\
13.9002777777778	-0.0455576834149863\\
13.9058333333333	-0.0455576834149863\\
13.9113888888889	-0.0455576834149863\\
13.9169444444444	-0.0470382875014539\\
13.9225	-0.0452615625976931\\
13.9280555555556	-0.0434848376939323\\
13.9336111111111	-0.0434848376939323\\
13.9391666666667	-0.039548109307631\\
13.9447222222222	-0.039548109307631\\
13.9502777777778	-0.0476865035639269\\
13.9558333333333	-0.0476865035639269\\
13.9613888888889	-0.0476865035639269\\
13.9669444444444	-0.0431905970351901\\
13.9725	-0.0476865035639273\\
13.9780555555556	-0.0476865035639273\\
13.9836111111111	-0.0456136578428729\\
13.9891666666667	-0.0456136578428729\\
13.9947222222222	-0.0432446913045249\\
14.0002777777778	-0.0348699805237027\\
14.0058333333333	-0.0348699805237027\\
14.0113888888889	-0.0481954173019079\\
14.0169444444444	-0.0434033899558771\\
14.0225	-0.035777664557351\\
14.0280555555556	-0.0302089899344119\\
14.0336111111111	-0.0217933982890594\\
14.0391666666667	-0.0217933982890594\\
14.0447222222222	-0.0217933982890594\\
14.0502777777778	-0.0133192578947086\\
14.0558333333333	-0.00490154985592733\\
14.0613888888889	-0.00209564643610781\\
14.0669444444444	0.000710257032371143\\
14.0725	0.000710257032371143\\
14.0780555555556	0.000710257032371143\\
14.0836111111111	0.000710257032371143\\
14.0891666666667	0.000710257032371143\\
14.0947222222222	0.000710257032371143\\
14.1002777777778	0.000710257032371143\\
14.1058333333333	0.000710257032371143\\
14.1113888888889	0.000710257032371143\\
14.1169444444444	0.000710257032371143\\
14.1225	0.000710257032371143\\
14.1280555555556	0.000710257032371143\\
14.1336111111111	-0.00177646578228183\\
14.1391666666667	-0.0475461068134383\\
14.1447222222222	-0.0413281601096759\\
14.1502777777778	-0.0413281601096759\\
14.1558333333333	-0.0382238197473684\\
14.1613888888889	-0.0382238197473684\\
14.1669444444444	-0.0320517373481859\\
14.1725	-0.0474755135251013\\
14.1780555555556	-0.0448104261694601\\
14.1836111111111	-0.0420912445444833\\
14.1891666666667	-0.0393096034247608\\
14.1947222222222	-0.0337261683947623\\
14.2002777777778	-0.0308978054638665\\
14.2058333333333	-0.0308978054638665\\
14.2113888888889	-0.0308978054638665\\
14.2169444444444	-0.0308978054638665\\
14.2225	-0.044704019501381\\
14.2280555555556	-0.0423350529630334\\
14.2336111111111	-0.0395408799580669\\
14.2391666666667	-0.0367093478938582\\
14.2447222222222	-0.028148600105989\\
14.2502777777778	-0.0224265705275226\\
14.2558333333333	-0.019564881819064\\
14.2613888888889	-0.0167030910786026\\
14.2669444444444	-0.00811759299017918\\
14.2725	-0.00525575353118444\\
14.2780555555556	-0.00239391372569902\\
14.2836111111111	-0.00239391372569902\\
14.2891666666667	0.00392200785926239\\
14.2947222222222	0.0073760895253175\\
14.3002777777778	0.0073760895253175\\
14.3058333333333	0.0073760895253175\\
14.3113888888889	0.0073760895253175\\
14.3169444444444	0.0073760895253175\\
14.3225	0.0073760895253175\\
14.3280555555556	0.0073760895253175\\
14.3336111111111	0.0073760895253175\\
14.3391666666667	0.0073760895253175\\
14.3447222222222	0.0073760895253175\\
14.3502777777778	0.0073760895253175\\
14.3558333333333	0.0073760895253175\\
14.3613888888889	0.0073760895253175\\
14.3669444444444	0.0073760895253175\\
14.3725	0.0073760895253175\\
14.3780555555556	-0.0618168135798721\\
14.3836111111111	-0.0618168135798721\\
14.3891666666667	-0.0618168135798721\\
14.3947222222222	-0.0618168135798721\\
14.4002777777778	-0.0582633637723505\\
14.4058333333333	-0.0615206927625785\\
14.4113888888889	-0.0615206927625785\\
14.4169444444444	-0.0579672429550573\\
14.4225	-0.0579672429550573\\
14.4280555555556	-0.057596130757774\\
14.4336111111111	-0.0617418221998824\\
14.4391666666667	-0.0614457013825888\\
14.4447222222222	-0.0611495805652956\\
14.4502777777778	-0.0611495805652956\\
14.4558333333333	-0.0596533784964781\\
14.4613888888889	-0.0596377972147692\\
14.4669444444444	-0.059341676397476\\
14.4725	-0.059341676397476\\
14.4780555555556	-0.059341676397476\\
14.4836111111111	-0.059341676397476\\
14.4891666666667	-0.059341676397476\\
14.4947222222222	-0.059341676397476\\
14.5002777777778	-0.059341676397476\\
14.5058333333333	-0.059341676397476\\
14.5113888888889	-0.059341676397476\\
14.5169444444444	-0.0592403500107703\\
14.5225	-0.0592403500107703\\
14.5280555555556	-0.0592403500107703\\
14.5336111111111	-0.0592403500107703\\
14.5391666666667	-0.0589442291934771\\
14.5447222222222	-0.0589442291934771\\
14.5502777777778	-0.0592403500107703\\
14.5558333333333	-0.0592403500107703\\
14.5613888888889	-0.0592403500107703\\
14.5669444444444	-0.0592403500107703\\
14.5725	-0.0586481083761839\\
14.5780555555556	-0.052609138266233\\
14.5836111111111	-0.058827675429395\\
14.5891666666667	-0.0563296232417221\\
14.5947222222222	-0.053960656703374\\
14.6002777777778	-0.053960656703374\\
14.6058333333333	-0.0485899745791773\\
14.6113888888889	-0.0485899745791773\\
14.6169444444444	-0.0485899745791773\\
14.6225	-0.0590515792750829\\
14.6280555555556	-0.0590515792750829\\
14.6336111111111	-0.0560903711021485\\
14.6391666666667	-0.0560903711021485\\
14.6447222222222	-0.0560903711021485\\
14.6502777777778	-0.0560903711021485\\
14.6558333333333	-0.0581632168232025\\
14.6613888888889	-0.0578670960059093\\
14.6669444444444	-0.0578670960059093\\
14.6725	-0.0602360625442569\\
14.6780555555556	-0.0560903711021485\\
14.6836111111111	-0.0693322310064605\\
14.6891666666667	-0.0693322310064605\\
14.6947222222222	-0.0651959368796947\\
14.7002777777778	-0.0651959368796947\\
14.7058333333333	-0.0651959368796947\\
14.7113888888889	-0.0651959368796947\\
14.7169444444444	-0.0651959368796947\\
14.7225	-0.0681571450526295\\
14.7280555555556	-0.0648998160624015\\
14.7336111111111	-0.0648998160624015\\
14.7391666666667	-0.0646036952451083\\
14.7447222222222	-0.0684532658699235\\
14.7502777777778	-0.0643075744278147\\
14.7558333333333	-0.0643075744278147\\
14.7613888888889	-0.0681571450526295\\
14.7669444444444	-0.0716115999006933\\
14.7725	-0.0716115999006933\\
14.7780555555556	-0.0716115999006933\\
14.7836111111111	-0.0677620292758785\\
14.7891666666667	-0.0677620292758785\\
14.7947222222222	-0.0710193582661061\\
14.8002777777778	-0.0710193582661061\\
14.8058333333333	-0.0763703294979138\\
14.8113888888889	-0.0707440339693374\\
14.8169444444444	-0.0721837851345256\\
14.8225	-0.0679082251718028\\
14.8280555555556	-0.0419796878884824\\
14.8336111111111	-0.0388552041402926\\
14.8391666666667	-0.0388552041402926\\
14.8447222222222	-0.0563861417755345\\
14.8502777777778	-0.0534249336026001\\
14.8558333333333	-0.0534249336026001\\
14.8613888888889	-0.0534249336026001\\
14.8669444444444	-0.0534249336026001\\
14.8725	-0.0534249336026001\\
14.8780555555556	-0.0534249336026001\\
14.8836111111111	-0.0534249336026001\\
14.8891666666667	-0.0534249336026001\\
14.8947222222222	-0.0534249336026001\\
14.9002777777778	-0.0557939001409477\\
14.9058333333333	-0.0557939001409477\\
14.9113888888889	-0.0534249336026001\\
14.9169444444444	-0.0531288127853069\\
14.9225	-0.0545553226024389\\
14.9280555555556	-0.0545553226024389\\
14.9336111111111	-0.0527785976986781\\
14.9391666666667	-0.0527785976986781\\
14.9447222222222	-0.0527785976986781\\
14.9502777777778	-0.0527785976986781\\
14.9558333333333	-0.0475780990961223\\
14.9613888888889	-0.054388877893871\\
14.9669444444444	-0.0564617236149254\\
14.9725	-0.0561656027976323\\
14.9780555555556	-0.0561656027976323\\
14.9836111111111	-0.0561656027976323\\
14.9891666666667	-0.0555733611630459\\
14.9947222222222	-0.0535005154419919\\
15.0002777777778	-0.0535005154419919\\
15.0058333333333	-0.0535005154419919\\
15.0113888888889	-0.0552772403457527\\
15.0169444444444	-0.0552772403457527\\
15.0225	-0.0552772403457527\\
15.0280555555556	-0.0552772403457527\\
15.0336111111111	-0.0552772403457527\\
15.0391666666667	-0.0529082738074051\\
15.0447222222222	-0.0557403963310142\\
15.0502777777778	-0.0557403963310142\\
15.0558333333333	-0.0557403963310142\\
15.0613888888889	-0.053075308975373\\
15.0669444444444	-0.0476285805000093\\
15.0725	-0.0448661170537333\\
15.0780555555556	-0.0448661170537333\\
15.0836111111111	-0.0422010296980921\\
15.0891666666667	-0.0554784524091326\\
15.0947222222222	-0.0554784524091326\\
15.1002777777778	-0.0504443985151443\\
15.1058333333333	-0.0504443985151443\\
15.1113888888889	-0.0458061946147466\\
15.1169444444444	-0.0458061946147466\\
15.1225	-0.0574734387998229\\
15.1280555555556	-0.0548239327258906\\
15.1336111111111	-0.0566006576296514\\
15.1391666666667	-0.0566006576296514\\
15.1447222222222	-0.0532142429900973\\
15.1502777777778	-0.051271073377769\\
15.1558333333333	-0.0557128856371702\\
15.1613888888889	-0.054232281550703\\
15.1669444444444	-0.0526975831949002\\
15.1725	-0.0587496756494941\\
15.1780555555556	-0.0560845882938535\\
15.1836111111111	-0.0560845882938535\\
15.1891666666667	-0.0554923466592671\\
15.1947222222222	-0.0534195009382131\\
15.2002777777778	-0.0480899166863308\\
15.2058333333333	-0.0480899166863308\\
15.2113888888889	-0.0550016512658104\\
15.2169444444444	-0.0535210471793433\\
15.2225	-0.051565345681771\\
15.2280555555556	-0.051565345681771\\
15.2336111111111	-0.0493759461966357\\
15.2391666666667	-0.0493759461966357\\
15.2447222222222	-0.0564828458116789\\
15.2502777777778	-0.0541138792733313\\
15.2558333333333	-0.0552983625425053\\
15.2613888888889	-0.0547061209079185\\
15.2669444444444	-0.054129477255683\\
15.2725	-0.0554394333087332\\
15.2780555555556	-0.0396258440522822\\
15.2836111111111	-0.0534419806063529\\
15.2891666666667	-0.0534419806063529\\
15.2947222222222	-0.0534419806063529\\
15.3002777777778	-0.0497594451495056\\
15.3058333333333	-0.0524245325051476\\
15.3113888888889	-0.0487419970483003\\
15.3169444444444	-0.0409904000922827\\
15.3225	-0.0572770450434239\\
15.3280555555556	-0.053131353601316\\
15.3336111111111	-0.053131353601316\\
15.3391666666667	-0.0560175703942602\\
15.3447222222222	-0.0560175703942602\\
15.3502777777778	-0.0560175703942602\\
15.3558333333333	-0.0449445463903218\\
15.3613888888889	-0.0449445463903218\\
15.3669444444444	-0.0406167231330992\\
15.3725	-0.0560283495538985\\
15.3780555555556	-0.0560283495538985\\
15.3836111111111	-0.0560283495538985\\
15.3891666666667	-0.0488673556695197\\
15.3947222222222	-0.0373790249083894\\
15.4002777777778	-0.0546327108383945\\
15.4058333333333	-0.0546327108383945\\
15.4113888888889	-0.0546327108383945\\
15.4169444444444	-0.0475258112233513\\
15.4225	-0.0403215355176742\\
15.4280555555556	-0.0432827436906086\\
15.4336111111111	-0.0530547306612922\\
15.4391666666667	-0.0530547306612922\\
15.4447222222222	-0.0521663682094126\\
15.4502777777778	-0.0521663682094126\\
15.4558333333333	-0.0515741265748263\\
15.4613888888889	-0.0512780057575331\\
15.4669444444444	-0.0509818849402399\\
15.4725	-0.0486129184018927\\
15.4780555555556	-0.0486129184018927\\
15.4836111111111	-0.0486129184018927\\
15.4891666666667	-0.0486129184018927\\
15.4947222222222	-0.0486129184018927\\
15.5002777777778	-0.0461898575942099\\
15.5058333333333	-0.0461898575942099\\
15.5113888888889	-0.0500394282190247\\
15.5169444444444	-0.0500394282190247\\
15.5225	-0.0466530135794706\\
15.5280555555556	-0.0466530135794706\\
15.5336111111111	-0.049910342569699\\
15.5391666666667	-0.0496142217524058\\
15.5447222222222	-0.0481336176659386\\
15.5502777777778	-0.0504484899349506\\
15.5558333333333	-0.046248704223507\\
15.5613888888889	-0.0413676389574386\\
15.5669444444444	-0.050843505110829\\
15.5725	-0.0459764863848081\\
15.5780555555556	-0.0459764863848081\\
15.5836111111111	-0.0477532112885689\\
15.5891666666667	-0.0495299361923297\\
15.5947222222222	-0.0495299361923297\\
15.6002777777778	-0.0495299361923297\\
15.6058333333333	-0.0474570904712757\\
15.6113888888889	-0.0430158686712746\\
15.6169444444444	-0.048049922565263\\
15.6225	-0.0471615601133817\\
15.6280555555556	-0.0474420829483241\\
15.6336111111111	-0.0459614788618569\\
15.6391666666667	-0.0427546312755151\\
15.6447222222222	-0.0468613230918825\\
15.6502777777778	-0.0456768398227089\\
15.6558333333333	-0.0436045845610558\\
15.6613888888889	-0.0450851886475234\\
15.6669444444444	-0.044196826195643\\
15.6725	-0.0426621278398398\\
15.6780555555556	-0.0426621278398398\\
15.6836111111111	-0.0426621278398398\\
15.6891666666667	-0.0410150790448047\\
15.6947222222222	-0.0410150790448047\\
15.7002777777778	-0.042822873015428\\
15.7058333333333	-0.0422306313808408\\
15.7113888888889	-0.0422306313808408\\
15.7169444444444	-0.0422306313808408\\
15.7225	-0.0422306313808408\\
15.7280555555556	-0.0422306313808408\\
15.7336111111111	-0.0415093040969287\\
15.7391666666667	-0.0415093040969287\\
15.7447222222222	-0.0407506177537749\\
15.7502777777778	-0.0328163091697003\\
15.7558333333333	-0.0445361293790918\\
15.7613888888889	-0.0347914887814482\\
15.7669444444444	-0.0325230992531773\\
15.7725	-0.0406851066324949\\
15.7780555555556	-0.0403889858152013\\
15.7836111111111	-0.0383167305535477\\
15.7891666666667	-0.0390754168967016\\
15.7947222222222	-0.0390754168967016\\
15.8002777777778	-0.0381870544448212\\
15.8058333333333	-0.037890933627528\\
15.8113888888889	-0.0375948128102348\\
15.8169444444444	-0.0375948128102348\\
15.8225	-0.0375948128102348\\
15.8280555555556	-0.037890933627528\\
15.8336111111111	-0.0372986919929414\\
15.8391666666667	-0.0372986919929416\\
15.8447222222222	-0.0372986919929416\\
15.8502777777778	-0.0372986919929416\\
15.8558333333333	-0.0372986919929416\\
15.8613888888889	-0.0371696063436159\\
15.8669444444444	-0.0355493134622207\\
15.8725	-0.0347615280204158\\
15.8780555555556	-0.0347615280204158\\
15.8836111111111	-0.0347615280204158\\
15.8891666666667	-0.0339729753258623\\
15.8947222222222	-0.0339729753258623\\
15.9002777777778	-0.0386110986452331\\
15.9058333333333	-0.0383149778279399\\
15.9113888888889	-0.0383149778279399\\
15.9169444444444	-0.0377227361933535\\
15.9225	-0.0377227361933535\\
15.9280555555556	-0.0374266153760604\\
15.9336111111111	-0.0363773256700509\\
15.9391666666667	-0.0360812048527577\\
15.9447222222222	-0.0369701577640381\\
15.9502777777778	-0.0350138658070646\\
15.9558333333333	-0.0350138658070646\\
15.9613888888889	-0.0350138658070646\\
15.9669444444444	-0.0356061074416518\\
15.9725	-0.0344216241724782\\
15.9780555555556	-0.0344216241724782\\
15.9836111111111	-0.0338293825378919\\
15.9891666666667	-0.0346089828130371\\
15.9947222222222	-0.0357006922207798\\
};
\addlegendentry{Cts Stoch Ctrl w nFPC};

\addplot [color=cts_nFPC_plot_color,solid,line width=1.5pt]
  table[row sep=crcr]{%
9.50027777777778	0.00651465798045599\\
9.50583333333333	0.0200445396026153\\
9.51138888888889	0.0270451721856112\\
9.51694444444444	0.0174153800991449\\
9.5225	0.0265153206998117\\
9.52805555555556	0.0447450874263107\\
9.53361111111111	0.046603099408904\\
9.53916666666667	0.0261707630156523\\
9.54472222222222	0.0258949357735249\\
9.55027777777778	0.0286412274727425\\
9.55583333333333	0.0242240013259122\\
9.56138888888889	0.0260250749775928\\
9.56694444444444	0.0291449923131466\\
9.5725	0.0290997660447632\\
9.57805555555555	0.0290997660447632\\
9.58361111111111	0.03058037013123\\
9.58916666666667	0.0335040185963986\\
9.59472222222222	0.0320670569117455\\
9.60027777777778	0.024283724091004\\
9.60583333333333	0.0237432062674186\\
9.61138888888889	0.0201389363121283\\
9.61694444444444	0.0211689122859709\\
9.6225	0.0268021191607941\\
9.62805555555556	0.0277221793725048\\
9.63361111111111	0.026529314779487\\
9.63916666666667	0.0261809076460262\\
9.64472222222222	0.0267852031711157\\
9.65027777777778	0.0261576487717202\\
9.65583333333333	0.0261576487717202\\
9.66138888888889	0.0288234232098251\\
9.66694444444444	0.0300070423924282\\
9.6725	0.0304860973839293\\
9.67805555555555	0.0293291369944308\\
9.68361111111111	0.0304149581095141\\
9.68916666666667	0.0294693825263812\\
9.69472222222222	0.030860199134917\\
9.70027777777778	0.0311650879796876\\
9.70583333333333	0.0303755713050265\\
9.71138888888889	0.0318637647476124\\
9.71694444444444	0.0309207306601339\\
9.7225	0.0321102889706195\\
9.72805555555555	0.0324632312580554\\
9.73361111111111	0.0324752968537227\\
9.73916666666667	0.0324752968537227\\
9.74472222222222	0.0324752968537227\\
9.75027777777778	0.0336597801228963\\
9.75583333333333	0.0348442633920698\\
9.76138888888889	0.0339087734052872\\
9.76694444444444	0.0338508535023585\\
9.7725	0.0354745370385622\\
9.77805555555556	0.0354745370385622\\
9.78361111111111	0.0360121070375505\\
9.78916666666667	0.0361645595245702\\
9.79472222222222	0.0360214802968167\\
9.80027777777778	0.0370815358515861\\
9.80583333333333	0.0370815358515861\\
9.81138888888889	0.0370876140518991\\
9.81694444444444	0.0373904407777211\\
9.8225	0.0384191435861967\\
9.82805555555555	0.0375030998715784\\
9.83361111111111	0.0378052448394507\\
9.83916666666667	0.0384988563190337\\
9.84472222222222	0.0384988563190337\\
9.85027777777778	0.0391991515161995\\
9.85583333333333	0.0394142663553523\\
9.86138888888889	0.037823334563378\\
9.86694444444444	0.0384154208101019\\
9.8725	0.0392088538254531\\
9.87805555555556	0.0394578471078444\\
9.88361111111111	0.0392538946577874\\
9.88916666666667	0.0392538946577874\\
9.89472222222222	0.0395516041216764\\
9.90027777777778	0.0395516041216764\\
9.90583333333333	0.0398580816130423\\
9.91138888888889	0.0401542024303355\\
9.91694444444444	0.0401542024303355\\
9.9225	0.0400995307947374\\
9.92805555555555	0.0401082988222144\\
9.93361111111111	0.0401082988222144\\
9.93916666666667	0.0401602363204419\\
9.94472222222222	0.04030243120466\\
9.95027777777778	0.04030243120466\\
9.95583333333333	0.0421249112549149\\
9.96138888888889	0.0443178215135495\\
9.96694444444444	0.0412711668161853\\
9.9725	0.042751770902652\\
9.97805555555555	0.0411530269572411\\
9.98361111111111	0.0411530269572411\\
9.98916666666667	0.0434652026822263\\
9.99472222222222	0.0440570122735278\\
10.0002777777778	0.0440570122735278\\
10.0058333333333	0.0440570122735278\\
10.0113888888889	0.0440570122735278\\
10.0169444444444	0.0400042205605599\\
10.0225	0.0403797392554138\\
10.0280555555556	0.0403797392554138\\
10.0336111111111	0.0406720578857378\\
10.0391666666667	0.0406753554539282\\
10.0447222222222	0.0422062813331845\\
10.0502777777778	0.0424477305148796\\
10.0558333333333	0.0433295566424198\\
10.0613888888889	0.0433295566424198\\
10.0669444444444	0.0407969029261894\\
10.0725	0.0407969029261894\\
10.0780555555556	0.0411478084689101\\
10.0836111111111	0.0411478084689101\\
10.0891666666667	0.0411478084689101\\
10.0947222222222	0.0411478084689101\\
10.1002777777778	0.0421892125103307\\
10.1058333333333	0.0440564516499785\\
10.1113888888889	0.0443591676036528\\
10.1169444444444	0.0410849314077858\\
10.1225	0.0410849314077858\\
10.1280555555556	0.0410849314077858\\
10.1336111111111	0.0410849314077858\\
10.1391666666667	0.0410849314077858\\
10.1447222222222	0.0378022111379179\\
10.1502777777778	0.0378022111379179\\
10.1558333333333	0.0378022111379179\\
10.1613888888889	0.0378022111379179\\
10.1669444444444	0.0413660252897712\\
10.1725	0.0413703170777\\
10.1780555555556	0.0413703170777\\
10.1836111111111	0.041301472373834\\
10.1891666666667	0.0417596925000573\\
10.1947222222222	0.0417596925000573\\
10.2002777777778	0.0417596925000573\\
10.2058333333333	0.0417596925000573\\
10.2113888888889	0.0417596925000573\\
10.2169444444444	0.0417596925000573\\
10.2225	0.0421980082015685\\
10.2280555555556	0.0434313957837689\\
10.2336111111111	0.0444612217579636\\
10.2391666666667	0.0446600184939979\\
10.2447222222222	0.0444553460365318\\
10.2502777777778	0.0454351568482393\\
10.2558333333333	0.0442490578736269\\
10.2613888888889	0.0468331392511278\\
10.2669444444444	0.0474249488424294\\
10.2725	0.0458175836313272\\
10.2780555555556	0.0478427490229996\\
10.2836111111111	0.0436970575808916\\
10.2891666666667	0.0423586483786421\\
10.2947222222222	0.0423586483786421\\
10.3002777777778	0.0435474234357445\\
10.3058333333333	0.0435474234357445\\
10.3113888888889	0.0434960493683359\\
10.3169444444444	0.0465090368383444\\
10.3225	0.0479965522710601\\
10.3280555555556	0.046500968033926\\
10.3336111111111	0.0456186837823595\\
10.3391666666667	0.0447370272390079\\
10.3447222222222	0.0441525759646425\\
10.3502777777778	0.0441525759646425\\
10.3558333333333	0.0451866937267664\\
10.3613888888889	0.0451866937267664\\
10.3669444444444	0.0451866937267664\\
10.3725	0.0451866937267664\\
10.3780555555556	0.0451866937267664\\
10.3836111111111	0.0451866937267664\\
10.3891666666667	0.045043614499013\\
10.3947222222222	0.045043614499013\\
10.4002777777778	0.0446040429477961\\
10.4058333333333	0.0446040429477961\\
10.4113888888889	0.0453477613362132\\
10.4169444444444	0.0453477613362132\\
10.4225	0.0453477613362132\\
10.4280555555556	0.046090895718674\\
10.4336111111111	0.046090895718674\\
10.4391666666667	0.046090895718674\\
10.4447222222222	0.0460362240830759\\
10.4502777777778	0.0460362240830759\\
10.4558333333333	0.0460362240830759\\
10.4613888888889	0.0460362240830759\\
10.4669444444444	0.0463328988200085\\
10.4725	0.0460868249371383\\
10.4780555555556	0.0469280598541165\\
10.4836111111111	0.0469280598541165\\
10.4891666666667	0.0463505206172089\\
10.4947222222222	0.0470689085949639\\
10.5002777777778	0.0470689085949639\\
10.5058333333333	0.047910143511942\\
10.5113888888889	0.0482796145095872\\
10.5169444444444	0.0482758123226176\\
10.5225	0.0482758123226176\\
10.5280555555556	0.0482758123226176\\
10.5336111111111	0.0488468088470052\\
10.5391666666667	0.0488468088470052\\
10.5447222222222	0.049132307109199\\
10.5502777777778	0.049132307109199\\
10.5558333333333	0.0494295357657707\\
10.5613888888889	0.0494295357657707\\
10.5669444444444	0.0494295357657707\\
10.5725	0.0494295357657707\\
10.5780555555556	0.0500141730264187\\
10.5836111111111	0.0506108200684743\\
10.5891666666667	0.0511954573291224\\
10.5947222222222	0.0511987548973127\\
10.6002777777778	0.0495729227414263\\
10.6058333333333	0.0495729227414263\\
10.6113888888889	0.0482337821772737\\
10.6169444444444	0.0482337821772737\\
10.6225	0.0482337821772737\\
10.6280555555556	0.0501635486705747\\
10.6336111111111	0.0501635486705747\\
10.6391666666667	0.0500947039667087\\
10.6447222222222	0.0489559758440293\\
10.6502777777778	0.0506855732128878\\
10.6558333333333	0.0506855732128878\\
10.6613888888889	0.0506855732128878\\
10.6669444444444	0.0529998681156369\\
10.6725	0.04923234553483\\
10.6780555555556	0.0531748158887857\\
10.6836111111111	0.0531748158887857\\
10.6891666666667	0.0534603141509794\\
10.6947222222222	0.0534603141509794\\
10.7002777777778	0.0515383611777994\\
10.7058333333333	0.0537534017830715\\
10.7113888888889	0.0540389000452653\\
10.7169444444444	0.0540389000452653\\
10.7225	0.0540389000452653\\
10.7280555555556	0.0540389000452653\\
10.7336111111111	0.0524130678893784\\
10.7391666666667	0.0543319876773573\\
10.7447222222222	0.051748948365724\\
10.7502777777778	0.0505151699414724\\
10.7558333333333	0.0505151699414724\\
10.7613888888889	0.0526415262550694\\
10.7669444444444	0.0550781112234604\\
10.7725	0.0573924061262095\\
10.7780555555556	0.0573924061262095\\
10.7836111111111	0.0573924061262095\\
10.7891666666667	0.0573924061262095\\
10.7947222222222	0.0573924061262095\\
10.8002777777778	0.057963402650597\\
10.8058333333333	0.058548039911245\\
10.8113888888889	0.058548039911245\\
10.8169444444444	0.056626086938065\\
10.8225	0.056626086938065\\
10.8280555555556	0.0572171022988287\\
10.8336111111111	0.0572171022988287\\
10.8391666666667	0.0561910504392158\\
10.8447222222222	0.0582614073621766\\
10.8502777777778	0.060625002317626\\
10.8558333333333	0.0609263462299384\\
10.8613888888889	0.0616764518219443\\
10.8669444444444	0.0616852198494213\\
10.8725	0.0622785693232866\\
10.8780555555556	0.0622785693232866\\
10.8836111111111	0.0622785693232866\\
10.8891666666667	0.0622785693232866\\
10.8947222222222	0.0622785693232866\\
10.9002777777778	0.0622785693232866\\
10.9058333333333	0.0622785693232866\\
10.9113888888889	0.0588654256723275\\
10.9169444444444	0.0588654256723275\\
10.9225	0.0587386645476921\\
10.9280555555556	0.0571312993365891\\
10.9336111111111	0.0546887532470368\\
10.9391666666667	0.0621743089913325\\
10.9447222222222	0.056544204226185\\
10.9502777777778	0.060353733632553\\
10.9558333333333	0.0602097263558239\\
10.9613888888889	0.0614181207671955\\
10.9669444444444	0.0603797453467011\\
10.9725	0.0602041054157616\\
10.9780555555556	0.0596045568575515\\
10.9836111111111	0.0609183134064565\\
10.9891666666667	0.0608240125912719\\
10.9947222222222	0.0612817887979883\\
11.0002777777778	0.06124148982751\\
11.0058333333333	0.06124148982751\\
11.0113888888889	0.0606492481929219\\
11.0169444444444	0.0620604485484612\\
11.0225	0.0620604485484612\\
11.0280555555556	0.0620604485484612\\
11.0336111111111	0.0610793554727681\\
11.0391666666667	0.0627866766456006\\
11.0447222222222	0.0828581917789997\\
11.0502777777778	0.0827893470751341\\
11.0558333333333	0.0848733993572442\\
11.0613888888889	0.0848733993572442\\
11.0669444444444	0.0848733993572442\\
11.0725	0.0828005536361903\\
11.0780555555556	0.0828005536361903\\
11.0836111111111	0.0828005536361903\\
11.0891666666667	0.0828005536361903\\
11.0947222222222	0.0828005536361903\\
11.1002777777778	0.0828005536361903\\
11.1058333333333	0.0828005536361903\\
11.1113888888889	0.0860536188827986\\
11.1169444444444	0.0860536188827986\\
11.1225	0.0860536188827986\\
11.1280555555556	0.0860536188827986\\
11.1336111111111	0.0860536188827986\\
11.1391666666667	0.0860536188827986\\
11.1447222222222	0.0863459375131226\\
11.1502777777778	0.0866498486906376\\
11.1558333333333	0.0869245011440224\\
11.1613888888889	0.0869245011440224\\
11.1669444444444	0.0869277987122123\\
11.1725	0.0869277987122123\\
11.1780555555556	0.0869277987122123\\
11.1836111111111	0.0869277987122123\\
11.1891666666667	0.0847097249217391\\
11.1947222222222	0.0847097249217391\\
11.2002777777778	0.0876162614590754\\
11.2058333333333	0.0844070677801053\\
11.2113888888889	0.0815873231294851\\
11.2169444444444	0.0846161497324629\\
11.2225	0.0846161497324629\\
11.2280555555556	0.0846161497324629\\
11.2336111111111	0.0846161497324629\\
11.2391666666667	0.0881711881865799\\
11.2447222222222	0.0881711881865799\\
11.2502777777778	0.0849181509842803\\
11.2558333333333	0.0881832703347297\\
11.2613888888889	0.0881838242543687\\
11.2669444444444	0.0847818211143601\\
11.2725	0.0847818211143601\\
11.2780555555556	0.0885854881310534\\
11.2836111111111	0.0885854881310534\\
11.2891666666667	0.0885854881310534\\
11.2947222222222	0.0885854881310534\\
11.3002777777778	0.0888778067613774\\
11.3058333333333	0.0888778067613774\\
11.3113888888889	0.0888778067613774\\
11.3169444444444	0.0891633050235707\\
11.3225	0.0891633050235707\\
11.3280555555556	0.0891633050235707\\
11.3336111111111	0.0891633050235707\\
11.3391666666667	0.0891633050235707\\
11.3447222222222	0.0891633050235707\\
11.3502777777778	0.0861338060664093\\
11.3558333333333	0.0861338060664093\\
11.3613888888889	0.0833998740069021\\
11.3669444444444	0.0899274787818029\\
11.3725	0.086970562396797\\
11.3780555555556	0.086970562396797\\
11.3836111111111	0.086970562396797\\
11.3891666666667	0.086970562396797\\
11.3947222222222	0.086970562396797\\
11.4002777777778	0.086970562396797\\
11.4058333333333	0.077927369786894\\
11.4113888888889	0.077927369786894\\
11.4169444444444	0.0719375137628995\\
11.4225	0.0913490578727498\\
11.4280555555556	0.0913490578727498\\
11.4336111111111	0.0916345561349435\\
11.4391666666667	0.0916345561349435\\
11.4447222222222	0.0916345561349435\\
11.4502777777778	0.0916345561349435\\
11.4558333333333	0.0891203615271763\\
11.4613888888889	0.0873893917699097\\
11.4669444444444	0.0873893917699097\\
11.4725	0.0858594925283649\\
11.4780555555556	0.0858594925283649\\
11.4836111111111	0.0879812013612104\\
11.4891666666667	0.0881902833135684\\
11.4947222222222	0.0882000749151273\\
11.5002777777778	0.0882000749151273\\
11.5058333333333	0.0882000749151273\\
11.5113888888889	0.0920549471020703\\
11.5169444444444	0.090330159696537\\
11.5225	0.090330159696537\\
11.5280555555556	0.090330159696537\\
11.5336111111111	0.090330159696537\\
11.5391666666667	0.090330159696537\\
11.5447222222222	0.0923976338346928\\
11.5502777777778	0.0906252007188604\\
11.5558333333333	0.0906252007188604\\
11.5613888888889	0.0910800046442494\\
11.5669444444444	0.0932950452495215\\
11.5725	0.0932950452495215\\
11.5780555555556	0.0932950452495215\\
11.5836111111111	0.0932950452495215\\
11.5891666666667	0.0913730922763415\\
11.5947222222222	0.0913730922763415\\
11.6002777777778	0.0894417100776017\\
11.6058333333333	0.0879118108360569\\
11.6113888888889	0.0879118108360569\\
11.6169444444444	0.0879118108360569\\
11.6225	0.0904827487117684\\
11.6280555555556	0.092254102032631\\
11.6336111111111	0.092262870060108\\
11.6391666666667	0.092262870060108\\
11.6447222222222	0.0879031058447143\\
11.6502777777778	0.0857809649685831\\
11.6558333333333	0.0857809649685831\\
11.6613888888889	0.0857809649685831\\
11.6669444444444	0.0857809649685831\\
11.6725	0.0808743033071966\\
11.6780555555556	0.0786993175886138\\
11.6836111111111	0.0786993175886138\\
11.6891666666667	0.0880819342541435\\
11.6947222222222	0.0880819342541435\\
11.7002777777778	0.0880819342541435\\
11.7058333333333	0.0880819342541435\\
11.7113888888889	0.0880819342541435\\
11.7169444444444	0.0880819342541435\\
11.7225	0.0880819342541435\\
11.7280555555556	0.0880819342541435\\
11.7336111111111	0.0860548436795841\\
11.7391666666667	0.0842841429264024\\
11.7447222222222	0.0826865052258992\\
11.7502777777778	0.078265218456461\\
11.7558333333333	0.0743570991498332\\
11.7613888888889	0.0743570991498332\\
11.7669444444444	0.0743570991498332\\
11.7725	0.0733746736962139\\
11.7780555555556	0.0724643667074089\\
11.7836111111111	0.071554706917513\\
11.7891666666667	0.0822150563400787\\
11.7947222222222	0.0829761002655253\\
11.8002777777778	0.0830693548705894\\
11.8058333333333	0.0830693548705894\\
11.8113888888889	0.0830693548705894\\
11.8169444444444	0.0830693548705894\\
11.8225	0.0830693548705894\\
11.8280555555556	0.0830693548705894\\
11.8336111111111	0.0830693548705894\\
11.8391666666667	0.0824082685321361\\
11.8447222222222	0.0836603702313531\\
11.8502777777778	0.0836603702313531\\
11.8558333333333	0.0829992838929003\\
11.8613888888889	0.0843451530314749\\
11.8669444444444	0.0826141832742091\\
11.8725	0.0832003148962095\\
11.8780555555556	0.0814296141430278\\
11.8836111111111	0.0799559214028091\\
11.8891666666667	0.077378386356437\\
11.8947222222222	0.0790272885738052\\
11.9002777777778	0.077674367359841\\
11.9058333333333	0.0752754827046198\\
11.9113888888889	0.073928260930088\\
11.9169444444444	0.073928260930088\\
11.9225	0.0726490493888428\\
11.9280555555556	0.0726490493888428\\
11.9336111111111	0.0726490493888428\\
11.9391666666667	0.0832648651355022\\
11.9447222222222	0.0832648651355022\\
11.9502777777778	0.0832648651355022\\
11.9558333333333	0.0832648651355022\\
11.9613888888889	0.0850431786858583\\
11.9669444444444	0.0832697513502879\\
11.9725	0.0832697513502879\\
11.9780555555556	0.0816344899688417\\
11.9836111111111	0.0839535948349169\\
11.9891666666667	0.0839535948349169\\
11.9947222222222	0.0839535948349169\\
12.0002777777778	0.0839535948349169\\
12.0058333333333	0.0810070890596618\\
12.0113888888889	0.0785614707333441\\
12.0169444444444	0.0773447713418571\\
12.0225	0.0773447713418571\\
12.0280555555556	0.0760603326762928\\
12.0336111111111	0.0760603326762928\\
12.0391666666667	0.0779369259019658\\
12.0447222222222	0.0801117940097214\\
12.0502777777778	0.0801117940097214\\
12.0558333333333	0.0854419687210038\\
12.0613888888889	0.0837566921770701\\
12.0669444444444	0.0774709621092213\\
12.0725	0.0778980901228201\\
12.0780555555556	0.0766866621708119\\
12.0836111111111	0.0782647734004501\\
12.0891666666667	0.0802076807682366\\
12.0947222222222	0.078792160560467\\
12.1002777777778	0.078792160560467\\
12.1058333333333	0.0793843542313104\\
12.1113888888889	0.0883963837595779\\
12.1169444444444	0.0877787508453512\\
12.1225	0.0866002917267567\\
12.1280555555556	0.0866002917267567\\
12.1336111111111	0.0866002917267567\\
12.1391666666667	0.0879887433802656\\
12.1447222222222	0.0897721741925552\\
12.1502777777778	0.0897721741925552\\
12.1558333333333	0.0886672315933114\\
12.1613888888889	0.0886672315933114\\
12.1669444444444	0.0886672315933114\\
12.1725	0.0886672315933114\\
12.1780555555556	0.0886672315933114\\
12.1836111111111	0.0886672315933114\\
12.1891666666667	0.0905969980866124\\
12.1947222222222	0.0905969980866124\\
12.2002777777778	0.0905969980866124\\
12.2058333333333	0.0905969980866124\\
12.2113888888889	0.0905969980866124\\
12.2169444444444	0.0905969980866124\\
12.2225	0.0905969980866124\\
12.2280555555556	0.089458269963933\\
12.2336111111111	0.089458269963933\\
12.2391666666667	0.089458269963933\\
12.2447222222222	0.0932553414709474\\
12.2502777777778	0.0932586390391377\\
12.2558333333333	0.08985094454186\\
12.2613888888889	0.08985094454186\\
12.2669444444444	0.0888316432319347\\
12.2725	0.0888316432319347\\
12.2780555555556	0.0888316432319347\\
12.2836111111111	0.0896029720697787\\
12.2891666666667	0.0896029720697787\\
12.2947222222222	0.0896029720697787\\
12.3002777777778	0.0896029720697787\\
12.3058333333333	0.0909404969284925\\
12.3113888888889	0.0909404969284925\\
12.3169444444444	0.0909404969284925\\
12.3225	0.0912798840941867\\
12.3280555555556	0.0912798840941867\\
12.3336111111111	0.0930581976445428\\
12.3391666666667	0.0916295310563031\\
12.3447222222222	0.0916295310563031\\
12.3502777777778	0.0906723239005567\\
12.3558333333333	0.0906723239005567\\
12.3613888888889	0.0906723239005567\\
12.3669444444444	0.0906723239005567\\
12.3725	0.0906723239005567\\
12.3780555555556	0.0906723239005567\\
12.3836111111111	0.0906723239005567\\
12.3891666666667	0.0906723239005567\\
12.3947222222222	0.0906723239005567\\
12.4002777777778	0.0906723239005567\\
12.4058333333333	0.0906723239005567\\
12.4113888888889	0.0906723239005567\\
12.4169444444444	0.0906723239005567\\
12.4225	0.0900958005856957\\
12.4280555555556	0.0900958005856957\\
12.4336111111111	0.0902256721890746\\
12.4391666666667	0.0905217450426232\\
12.4447222222222	0.092355122118565\\
12.4502777777778	0.092355122118565\\
12.4558333333333	0.092355122118565\\
12.4613888888889	0.092355122118565\\
12.4669444444444	0.0929472583803277\\
12.4725	0.0969965449617303\\
12.4780555555556	0.0969965449617303\\
12.4836111111111	0.0969965449617303\\
12.4891666666667	0.0962095523548458\\
12.4947222222222	0.0962095523548458\\
12.5002777777778	0.0956169215461064\\
12.5058333333333	0.0956169215461064\\
12.5113888888889	0.0956169215461064\\
12.5169444444444	0.0956169215461064\\
12.5225	0.0973963746824071\\
12.5280555555556	0.0991789728690801\\
12.5336111111111	0.100025907088804\\
12.5391666666667	0.101294587243652\\
12.5447222222222	0.101294587243652\\
12.5502777777778	0.101294587243652\\
12.5558333333333	0.100112437384709\\
12.5613888888889	0.100112437384709\\
12.5669444444444	0.100112437384709\\
12.5725	0.100112437384709\\
12.5780555555556	0.100112437384709\\
12.5836111111111	0.100112437384709\\
12.5891666666667	0.100112437384709\\
12.5947222222222	0.100112437384709\\
12.6002777777778	0.100112437384709\\
12.6058333333333	0.100112437384709\\
12.6113888888889	0.101885593695406\\
12.6169444444444	0.101730936400427\\
12.6225	0.10077372924468\\
12.6280555555556	0.10255716005697\\
12.6336111111111	0.10255716005697\\
12.6391666666667	0.10255716005697\\
12.6447222222222	0.10255716005697\\
12.6502777777778	0.101078889380733\\
12.6558333333333	0.101078889380733\\
12.6613888888889	0.101078889380733\\
12.6669444444444	0.103409767702938\\
12.6725	0.105780322887882\\
12.6780555555556	0.105780322887882\\
12.6836111111111	0.103466591415952\\
12.6891666666667	0.103466591415952\\
12.6947222222222	0.103466591415952\\
12.7002777777778	0.101917142625618\\
12.7058333333333	0.101917142625618\\
12.7113888888889	0.101917142625618\\
12.7169444444444	0.101917142625618\\
12.7225	0.101917142625618\\
12.7280555555556	0.101917142625618\\
12.7336111111111	0.101917142625618\\
12.7391666666667	0.101917142625618\\
12.7447222222222	0.101917142625618\\
12.7502777777778	0.101917142625618\\
12.7558333333333	0.101917142625618\\
12.7613888888889	0.101917142625618\\
12.7669444444444	0.101917142625618\\
12.7725	0.101917142625618\\
12.7780555555556	0.100824107716271\\
12.7836111111111	0.100824107716271\\
12.7891666666667	0.100824107716271\\
12.7947222222222	0.101595436554115\\
12.8002777777778	0.101595436554115\\
12.8058333333333	0.101595436554115\\
12.8113888888889	0.101595436554115\\
12.8169444444444	0.101595436554115\\
12.8225	0.102186062483263\\
12.8280555555556	0.102186062483263\\
12.8336111111111	0.102186062483263\\
12.8391666666667	0.102186062483263\\
12.8447222222222	0.10343816418248\\
12.8502777777778	0.104920356915543\\
12.8558333333333	0.104920356915543\\
12.8613888888889	0.105219775301026\\
12.8669444444444	0.105219775301026\\
12.8725	0.105219775301026\\
12.8780555555556	0.105631333375949\\
12.8836111111111	0.103790055556251\\
12.8891666666667	0.103790055556251\\
12.8947222222222	0.103148518766586\\
12.9002777777778	0.104333002035761\\
12.9058333333333	0.10551907395153\\
12.9113888888889	0.105519627871169\\
12.9169444444444	0.105528395898646\\
12.9225	0.105528395898646\\
12.9280555555556	0.105528395898646\\
12.9336111111111	0.104640033446766\\
12.9391666666667	0.103893134517199\\
12.9447222222222	0.103893134517199\\
12.9502777777778	0.103766373392564\\
12.9558333333333	0.103475895485072\\
12.9613888888889	0.103898162645533\\
12.9669444444444	0.103898162645533\\
12.9725	0.104200361663139\\
12.9780555555556	0.104503188388961\\
12.9836111111111	0.105389062042748\\
12.9891666666667	0.105389062042748\\
12.9947222222222	0.105389062042748\\
13.0002777777778	0.104796820408161\\
13.0058333333333	0.104796820408161\\
13.0113888888889	0.104796820408161\\
13.0169444444444	0.104796820408161\\
13.0225	0.104796820408161\\
13.0280555555556	0.104305948618571\\
13.0336111111111	0.104199520108075\\
13.0391666666667	0.104199520108075\\
13.0447222222222	0.104199520108075\\
13.0502777777778	0.105255479859885\\
13.0558333333333	0.105255479859885\\
13.0613888888889	0.105060728887588\\
13.0669444444444	0.105060728887588\\
13.0725	0.105060728887588\\
13.0780555555556	0.106307540412698\\
13.0836111111111	0.106307540412698\\
13.0891666666667	0.106307540412698\\
13.0947222222222	0.10710375043162\\
13.1002777777778	0.10710375043162\\
13.1058333333333	0.107350576093835\\
13.1113888888889	0.107350576093835\\
13.1169444444444	0.107152536501657\\
13.1225	0.108042799420442\\
13.1280555555556	0.108983487950944\\
13.1336111111111	0.108983487950944\\
13.1391666666667	0.107692576171274\\
13.1447222222222	0.107692576171274\\
13.1502777777778	0.108129008460186\\
13.1558333333333	0.108890609787431\\
13.1613888888889	0.107958084966024\\
13.1669444444444	0.107659890134022\\
13.1725	0.107659890134022\\
13.1780555555556	0.107659890134022\\
13.1836111111111	0.107659890134022\\
13.1891666666667	0.107659890134022\\
13.1947222222222	0.107659890134022\\
13.2002777777778	0.107659890134022\\
13.2058333333333	0.107659890134022\\
13.2113888888889	0.109291386968364\\
13.2169444444444	0.108249098583408\\
13.2225	0.108249098583408\\
13.2280555555556	0.110176716235043\\
13.2336111111111	0.112358698210581\\
13.2391666666667	0.112309403055504\\
13.2447222222222	0.114925195256068\\
13.2502777777778	0.114925195256068\\
13.2558333333333	0.112525332076245\\
13.2613888888889	0.115874109426301\\
13.2669444444444	0.115874109426301\\
13.2725	0.115874109426301\\
13.2780555555556	0.119386131001666\\
13.2836111111111	0.112644350905843\\
13.2891666666667	0.113612385678269\\
13.2947222222222	0.113612385678269\\
13.3002777777778	0.113612385678269\\
13.3058333333333	0.114203401039033\\
13.3113888888889	0.123809059240604\\
13.3169444444444	0.109049631454977\\
13.3225	0.109049631454977\\
13.3280555555556	0.123992225045843\\
13.3336111111111	0.109367737785436\\
13.3391666666667	0.109658487019832\\
13.3447222222222	0.11925333301821\\
13.3502777777778	0.11925333301821\\
13.3558333333333	0.11925333301821\\
13.3613888888889	0.11925333301821\\
13.3669444444444	0.11925333301821\\
13.3725	0.11925333301821\\
13.3780555555556	0.105597376018202\\
13.3836111111111	0.120126290251189\\
13.3891666666667	0.120126290251189\\
13.3947222222222	0.12496559317288\\
13.4002777777778	0.151797006450967\\
13.4058333333333	0.174842595252089\\
13.4113888888889	0.192821634812083\\
13.4169444444444	0.199611372125185\\
13.4225	0.199611372125185\\
13.4280555555556	0.19450517893564\\
13.4336111111111	0.189924666162603\\
13.4391666666667	0.189924666162603\\
13.4447222222222	0.189924666162603\\
13.4502777777778	0.189924666162603\\
13.4558333333333	0.199746540794264\\
13.4613888888889	0.199746540794264\\
13.4669444444444	0.199951616652405\\
13.4725	0.133899626399702\\
13.4780555555556	0.133899626399702\\
13.4836111111111	0.133899626399702\\
13.4891666666667	0.138738929321393\\
13.4947222222222	0.139199240679409\\
13.5002777777778	0.135554903792815\\
13.5058333333333	0.140098085897213\\
13.5113888888889	0.134042210392129\\
13.5169444444444	0.134042210392129\\
13.5225	0.130556379014651\\
13.5280555555556	0.130556379014651\\
13.5336111111111	0.13095201851241\\
13.5391666666667	0.13095201851241\\
13.5447222222222	0.13095201851241\\
13.5502777777778	0.128429126040987\\
13.5558333333333	0.128425323854018\\
13.5613888888889	0.128425323854018\\
13.5669444444444	0.128717642484342\\
13.5725	0.131141844089107\\
13.5780555555556	0.131141844089107\\
13.5836111111111	0.131141844089107\\
13.5891666666667	0.131141844089107\\
13.5947222222222	0.131141844089107\\
13.6002777777778	0.129507314069564\\
13.6058333333333	0.129507314069564\\
13.6113888888889	0.129799632699888\\
13.6169444444444	0.129799632699888\\
13.6225	0.129799632699888\\
13.6280555555556	0.130096307436821\\
13.6336111111111	0.130096307436821\\
13.6391666666667	0.134592390604681\\
13.6447222222222	0.134592390604681\\
13.6502777777778	0.134592390604681\\
13.6558333333333	0.134592390604681\\
13.6613888888889	0.134592390604681\\
13.6669444444444	0.134592390604681\\
13.6725	0.134592390604681\\
13.6780555555556	0.134592390604681\\
13.6836111111111	0.134592390604681\\
13.6891666666667	0.132822371609449\\
13.6947222222222	0.13105343706591\\
13.7002777777778	0.133143818841918\\
13.7058333333333	0.133143818841918\\
13.7113888888889	0.133143818841918\\
13.7169444444444	0.135358128085287\\
13.7225	0.139852266423311\\
13.7280555555556	0.138192193892809\\
13.7336111111111	0.133454260816114\\
13.7391666666667	0.135964690876776\\
13.7447222222222	0.13858048307734\\
13.7502777777778	0.133208880133901\\
13.7558333333333	0.133208880133901\\
13.7613888888889	0.133208880133901\\
13.7669444444444	0.133208880133901\\
13.7725	0.133208880133901\\
13.7780555555556	0.131130662829948\\
13.7836111111111	0.131130662829948\\
13.7891666666667	0.131724493111131\\
13.7947222222222	0.131736558706798\\
13.8002777777778	0.133565221108786\\
13.8058333333333	0.135569222125974\\
13.8113888888889	0.135569222125974\\
13.8169444444444	0.135569222125974\\
13.8225	0.135569222125974\\
13.8280555555556	0.135569222125974\\
13.8336111111111	0.137648091997607\\
13.8391666666667	0.137648091997607\\
13.8447222222222	0.137648091997607\\
13.8502777777778	0.137648091997607\\
13.8558333333333	0.13587744529416\\
13.8613888888889	0.13587744529416\\
13.8669444444444	0.134107426298928\\
13.8725	0.136476392837275\\
13.8780555555556	0.136476392837275\\
13.8836111111111	0.136476392837275\\
13.8891666666667	0.135602526654144\\
13.8947222222222	0.135602526654144\\
13.9002777777778	0.135602526654144\\
13.9058333333333	0.135602526654144\\
13.9113888888889	0.135602526654144\\
13.9169444444444	0.137675372375198\\
13.9225	0.139543545638786\\
13.9280555555556	0.141421640387543\\
13.9336111111111	0.141421640387543\\
13.9391666666667	0.143537940171428\\
13.9447222222222	0.143537940171428\\
13.9502777777778	0.143537940171428\\
13.9558333333333	0.143537940171428\\
13.9613888888889	0.143537940171428\\
13.9669444444444	0.135440538106976\\
13.9725	0.133224613158168\\
13.9780555555556	0.133224613158168\\
13.9836111111111	0.137803628923891\\
13.9891666666667	0.137803628923891\\
13.9947222222222	0.140273965307235\\
14.0002777777778	0.145918601724636\\
14.0058333333333	0.145918601724636\\
14.0113888888889	0.14355080107316\\
14.0169444444444	0.148678880390524\\
14.0225	0.151179661619718\\
14.0280555555556	0.156711439355605\\
14.0336111111111	0.159511306378516\\
14.0391666666667	0.159511306378516\\
14.0447222222222	0.162311790751565\\
14.0502777777778	0.17085224519403\\
14.0558333333333	0.176460863612373\\
14.0613888888889	0.17926613767482\\
14.0669444444444	0.17926613767482\\
14.0725	0.17926613767482\\
14.0780555555556	0.17926613767482\\
14.0836111111111	0.17926613767482\\
14.0891666666667	0.17926613767482\\
14.0947222222222	0.17926613767482\\
14.1002777777778	0.17926613767482\\
14.1058333333333	0.17926613767482\\
14.1113888888889	0.17926613767482\\
14.1169444444444	0.17926613767482\\
14.1225	0.13188680690787\\
14.1280555555556	0.13188680690787\\
14.1336111111111	0.134854039231384\\
14.1391666666667	0.129181497807394\\
14.1447222222222	0.132369982093756\\
14.1502777777778	0.132369982093756\\
14.1558333333333	0.141644817036567\\
14.1613888888889	0.150917810722099\\
14.1669444444444	0.160203762716599\\
14.1725	0.162868850072239\\
14.1780555555556	0.16567000193353\\
14.1836111111111	0.16567000193353\\
14.1891666666667	0.16567000193353\\
14.1947222222222	0.168471899825501\\
14.2002777777778	0.171341976228772\\
14.2058333333333	0.171341976228772\\
14.2113888888889	0.171341976228772\\
14.2169444444444	0.171341976228772\\
14.2225	0.174078355599488\\
14.2280555555556	0.17665466522698\\
14.2336111111111	0.179597129854155\\
14.2391666666667	0.179597129854155\\
14.2447222222222	0.179597129854155\\
14.2502777777778	0.179597129854155\\
14.2558333333333	0.179597129854155\\
14.2613888888889	0.179597129854155\\
14.2669444444444	0.179597129854155\\
14.2725	0.179597129854155\\
14.2780555555556	0.179597129854155\\
14.2836111111111	0.12925659091427\\
14.2891666666667	0.129138451055326\\
14.2947222222222	0.136127210811426\\
14.3002777777778	0.1431574586385\\
14.3058333333333	0.146678692323709\\
14.3113888888889	0.150202224014358\\
14.3169444444444	0.16078801383563\\
14.3225	0.16078801383563\\
14.3280555555556	0.171382532244277\\
14.3336111111111	0.185517411857769\\
14.3391666666667	0.185517411857769\\
14.3447222222222	0.18905211811953\\
14.3502777777778	0.18905211811953\\
14.3558333333333	0.18905211811953\\
14.3613888888889	0.128643471391673\\
14.3669444444444	0.128643471391673\\
14.3725	0.132910352978768\\
14.3780555555556	0.129128400783996\\
14.3836111111111	0.125872660440364\\
14.3891666666667	0.125872660440364\\
14.3947222222222	0.125872660440364\\
14.4002777777778	0.129478047746113\\
14.4058333333333	0.126225604970671\\
14.4113888888889	0.126225604970671\\
14.4169444444444	0.12983099227642\\
14.4225	0.12983099227642\\
14.4280555555556	0.12983099227642\\
14.4336111111111	0.12983099227642\\
14.4391666666667	0.1261323845633\\
14.4447222222222	0.126437273408071\\
14.4502777777778	0.126437273408071\\
14.4558333333333	0.127621784721553\\
14.4613888888889	0.127921203107037\\
14.4669444444444	0.131477950482748\\
14.4725	0.131477950482748\\
14.4780555555556	0.131477950482748\\
14.4836111111111	0.131477950482748\\
14.4891666666667	0.131477950482748\\
14.4947222222222	0.131477950482748\\
14.5002777777778	0.131477950482748\\
14.5058333333333	0.129548888292508\\
14.5113888888889	0.129851604246182\\
14.5169444444444	0.130263354246191\\
14.5225	0.130263354246191\\
14.5280555555556	0.130263354246191\\
14.5336111111111	0.130263354246191\\
14.5391666666667	0.130263354246191\\
14.5447222222222	0.130862191017159\\
14.5502777777778	0.13324869361046\\
14.5558333333333	0.13324869361046\\
14.5613888888889	0.13324869361046\\
14.5669444444444	0.132012836273878\\
14.5725	0.132617143504132\\
14.5780555555556	0.137109013612899\\
14.5836111111111	0.135329799911045\\
14.5891666666667	0.135329799911045\\
14.5947222222222	0.137701099859623\\
14.6002777777778	0.137701099859623\\
14.6058333333333	0.143044210067733\\
14.6113888888889	0.143044210067733\\
14.6169444444444	0.143044210067733\\
14.6225	0.140573381244599\\
14.6280555555556	0.140573381244599\\
14.6336111111111	0.138056463319801\\
14.6391666666667	0.138056463319801\\
14.6447222222222	0.138056463319801\\
14.6502777777778	0.135936490063845\\
14.6558333333333	0.133865232989387\\
14.6613888888889	0.13416465137487\\
14.6669444444444	0.13416465137487\\
14.6725	0.134173419402347\\
14.6780555555556	0.136542385940695\\
14.6836111111111	0.135188963620343\\
14.6891666666667	0.135188963620343\\
14.6947222222222	0.139238381140955\\
14.7002777777778	0.139238381140955\\
14.7058333333333	0.139238381140955\\
14.7113888888889	0.139238381140955\\
14.7169444444444	0.139238381140955\\
14.7225	0.136731563680472\\
14.7280555555556	0.134504792026342\\
14.7336111111111	0.134504792026342\\
14.7391666666667	0.14100254280116\\
14.7447222222222	0.137449092993638\\
14.7502777777778	0.134630079704922\\
14.7558333333333	0.138166622306805\\
14.7613888888889	0.134755367383501\\
14.7669444444444	0.134758664951692\\
14.7725	0.134758664951692\\
14.7780555555556	0.134758664951692\\
14.7836111111111	0.138166886689794\\
14.7891666666667	0.138166886689794\\
14.7947222222222	0.135351170969267\\
14.8002777777778	0.135351170969267\\
14.8058333333333	0.135351170969267\\
14.8113888888889	0.135743512898836\\
14.8169444444444	0.139239881848569\\
14.8225	0.141358482716117\\
14.8280555555556	0.145193263191686\\
14.8336111111111	0.142644828209305\\
14.8391666666667	0.145785123585497\\
14.8447222222222	0.140750832961835\\
14.8502777777778	0.143718065285348\\
14.8558333333333	0.143718065285348\\
14.8613888888889	0.143718065285348\\
14.8669444444444	0.143718065285348\\
14.8725	0.143718065285348\\
14.8780555555556	0.143718065285348\\
14.8836111111111	0.143718065285348\\
14.8891666666667	0.143718065285348\\
14.8947222222222	0.143718065285348\\
14.9002777777778	0.136659128630599\\
14.9058333333333	0.136659128630599\\
14.9113888888889	0.14110756608816\\
14.9169444444444	0.14342723747143\\
14.9225	0.14342723747143\\
14.9280555555556	0.14342723747143\\
14.9336111111111	0.14131296351822\\
14.9391666666667	0.14131296351822\\
14.9447222222222	0.14131296351822\\
14.9502777777778	0.14131296351822\\
14.9558333333333	0.138825857120929\\
14.9613888888889	0.138830362597414\\
14.9669444444444	0.136768723437416\\
14.9725	0.137161065366985\\
14.9780555555556	0.137161065366985\\
14.9836111111111	0.137460483752468\\
14.9891666666667	0.137757158489401\\
14.9947222222222	0.139715161919702\\
15.0002777777778	0.139715161919702\\
15.0058333333333	0.139715161919702\\
15.0113888888889	0.139715161919702\\
15.0169444444444	0.139715161919702\\
15.0225	0.139715161919702\\
15.0280555555556	0.139715161919702\\
15.0336111111111	0.139715161919702\\
15.0391666666667	0.139646317215836\\
15.0447222222222	0.137428313589183\\
15.0502777777778	0.137428313589183\\
15.0558333333333	0.137428313589183\\
15.0613888888889	0.140093400944824\\
15.0669444444444	0.143473692341246\\
15.0725	0.146230228056714\\
15.0780555555556	0.146230228056714\\
15.0836111111111	0.148895315412356\\
15.0891666666667	0.140128044926088\\
15.0947222222222	0.14042471966302\\
15.1002777777778	0.14714431203075\\
15.1058333333333	0.14714431203075\\
15.1113888888889	0.141265349747711\\
15.1169444444444	0.141265349747711\\
15.1225	0.139145376491755\\
15.1280555555556	0.141218222212809\\
15.1336111111111	0.138500635739368\\
15.1391666666667	0.138500635739368\\
15.1447222222222	0.142037178341251\\
15.1502777777778	0.142037178341251\\
15.1558333333333	0.140709615844324\\
15.1613888888889	0.142235975077285\\
15.1669444444444	0.141119110238155\\
15.1725	0.147730055108595\\
15.1780555555556	0.1522164873177\\
15.1836111111111	0.1522164873177\\
15.1891666666667	0.14866693162959\\
15.1947222222222	0.141560032014546\\
15.2002777777778	0.144452395483614\\
15.2058333333333	0.144452395483614\\
15.2113888888889	0.142349153072781\\
15.2169444444444	0.143832090569479\\
15.2225	0.142386388213608\\
15.2280555555556	0.142386388213608\\
15.2336111111111	0.149094835795001\\
15.2391666666667	0.149094835795001\\
15.2447222222222	0.141032134049951\\
15.2502777777778	0.140963289346085\\
15.2558333333333	0.139617275769896\\
15.2613888888889	0.142429453686442\\
15.2669444444444	0.14257164857066\\
15.2725	0.147603858141472\\
15.2780555555556	0.163619653673509\\
15.2836111111111	0.1640478319742\\
15.2891666666667	0.1640478319742\\
15.2947222222222	0.1640478319742\\
15.3002777777778	0.137076507545129\\
15.3058333333333	0.14124281220484\\
15.3113888888889	0.145043087674578\\
15.3169444444444	0.137275101721082\\
15.3225	0.133776965034005\\
15.3280555555556	0.137644071713774\\
15.3336111111111	0.137934820948169\\
15.3391666666667	0.134921326459067\\
15.3447222222222	0.135234983331314\\
15.3502777777778	0.135234983331314\\
15.3558333333333	0.142945858382823\\
15.3613888888889	0.142945858382823\\
15.3669444444444	0.142945858382823\\
15.3725	0.143629653603179\\
15.3780555555556	0.140379030521479\\
15.3836111111111	0.137363150712947\\
15.3891666666667	0.140920892308397\\
15.3947222222222	0.152327882384689\\
15.4002777777778	0.156472537777146\\
15.4058333333333	0.156472537777146\\
15.4113888888889	0.156472537777146\\
15.4169444444444	0.160088477660287\\
15.4225	0.163707011346506\\
15.4280555555556	0.160746764388019\\
15.4336111111111	0.157786722101957\\
15.4391666666667	0.152653534155721\\
15.4447222222222	0.147928385187868\\
15.4502777777778	0.147928385187868\\
15.4558333333333	0.148590578662729\\
15.4613888888889	0.147108454481193\\
15.4669444444444	0.147413343325963\\
15.4725	0.149538126545246\\
15.4780555555556	0.151838248379727\\
15.4836111111111	0.151838248379727\\
15.4891666666667	0.150059034677872\\
15.4947222222222	0.150059034677872\\
15.5002777777778	0.150745612299871\\
15.5058333333333	0.150745612299871\\
15.5113888888889	0.148970476042706\\
15.5169444444444	0.151342740149244\\
15.5225	0.152819804227128\\
15.5280555555556	0.152819804227128\\
15.5336111111111	0.151292072605759\\
15.5391666666667	0.151292072605759\\
15.5447222222222	0.15291414021454\\
15.5502777777778	0.152787379089905\\
15.5558333333333	0.150291651426922\\
15.5613888888889	0.15075697654901\\
15.5669444444444	0.15075697654901\\
15.5725	0.158592670524834\\
15.5780555555556	0.158592670524834\\
15.5836111111111	0.156808292164216\\
15.5891666666667	0.154939414827497\\
15.5947222222222	0.154939414827497\\
15.6002777777778	0.154939414827497\\
15.6058333333333	0.157259086210768\\
15.6113888888889	0.161920367499494\\
15.6169444444444	0.162578767267453\\
15.6225	0.163468433378136\\
15.6280555555556	0.162313954699102\\
15.6336111111111	0.15876050489158\\
15.6391666666667	0.158819564338011\\
15.6447222222222	0.155754389374401\\
15.6502777777778	0.157089765391449\\
15.6558333333333	0.15790679434137\\
15.6613888888889	0.159093611020775\\
15.6669444444444	0.158404635492715\\
15.6725	0.160177595849371\\
15.6780555555556	0.158570230638269\\
15.6836111111111	0.158570230638269\\
15.6891666666667	0.160278110838163\\
15.6947222222222	0.160278110838163\\
15.7002777777778	0.160340705405093\\
15.7058333333333	0.160340705405093\\
15.7113888888889	0.160340705405093\\
15.7169444444444	0.160340705405093\\
15.7225	0.160335333822195\\
15.7280555555556	0.160335333822195\\
15.7336111111111	0.160782347387362\\
15.7391666666667	0.160782347387362\\
15.7447222222222	0.161829141248804\\
15.7502777777778	0.160535830023612\\
15.7558333333333	0.161933049257854\\
15.7613888888889	0.16713727702609\\
15.7669444444444	0.167729421195022\\
15.7725	0.165803908199743\\
15.7780555555556	0.166318251821486\\
15.7836111111111	0.16716518604121\\
15.7891666666667	0.166910235721576\\
15.7947222222222	0.167935520527753\\
15.8002777777778	0.167935520527753\\
15.8058333333333	0.169194461816557\\
15.8113888888889	0.169486780446881\\
15.8169444444444	0.169779099077205\\
15.8225	0.169779099077205\\
15.8280555555556	0.16956502630392\\
15.8336111111111	0.169638376484271\\
15.8391666666667	0.170658865184909\\
15.8447222222222	0.170658865184909\\
15.8502777777778	0.170658865184909\\
15.8558333333333	0.170658865184909\\
15.8613888888889	0.170809757932783\\
15.8669444444444	0.170672068525051\\
15.8725	0.171491586273065\\
15.8780555555556	0.172330653569869\\
15.8836111111111	0.172330653569869\\
15.8891666666667	0.170512500433951\\
15.8947222222222	0.170512500433951\\
15.9002777777778	0.170062453683582\\
15.9058333333333	0.170367896447992\\
15.9113888888889	0.170367896447992\\
15.9169444444444	0.170982284769016\\
15.9225	0.170697370512778\\
15.9280555555556	0.171413580102545\\
15.9336111111111	0.171367117972545\\
15.9391666666667	0.173039618351133\\
15.9447222222222	0.172529424760554\\
15.9502777777778	0.174108398800963\\
15.9558333333333	0.174108398800963\\
15.9613888888889	0.174108398800963\\
15.9669444444444	0.17366704991425\\
15.9725	0.17366704991425\\
15.9780555555556	0.17366704991425\\
15.9836111111111	0.174829018509867\\
15.9891666666667	0.174829018509867\\
15.9947222222222	0.176509773505735\\
};
\addlegendentry{Dscr Stoch Ctrl};

\addplot [color=dscr_nFPC_plot_color,solid,line width=1.5pt]
  table[row sep=crcr]{%
9.50027777777778	-0.0142137992300856\\
9.50583333333333	-0.0153761969429108\\
9.51138888888889	-0.0290905628591349\\
9.51694444444444	-0.0260964080325325\\
9.5225	-0.0203035170035144\\
9.52805555555556	-0.0192954959882675\\
9.53361111111111	-0.0241575453023504\\
9.53916666666667	-0.0245737665998322\\
9.54472222222222	-0.0222485590679709\\
9.55027777777778	-0.0212359473770928\\
9.55583333333333	-0.0286389678094288\\
9.56138888888889	-0.0234575260954245\\
9.56694444444444	-0.0234575260954245\\
9.5725	-0.0224527305087399\\
9.57805555555555	-0.0224527305087399\\
9.58361111111111	-0.0236257867281079\\
9.58916666666667	-0.0230411985847551\\
9.59472222222222	-0.0271209470504013\\
9.60027777777778	-0.0218699550765727\\
9.60583333333333	-0.0206798480245503\\
9.61138888888889	-0.0295523366250158\\
9.61694444444444	-0.0255144063555439\\
9.6225	-0.0249600334174756\\
9.62805555555556	-0.0245293236452927\\
9.63361111111111	-0.0242808074467895\\
9.63916666666667	-0.0232286034798215\\
9.64472222222222	-0.0229302006781965\\
9.65027777777778	-0.0220469444639478\\
9.65583333333333	-0.0223434126105447\\
9.66138888888889	-0.0251242760451779\\
9.66694444444444	-0.024234162221201\\
9.6725	-0.0229058703526148\\
9.67805555555555	-0.0263684823328728\\
9.68361111111111	-0.0276964965553402\\
9.68916666666667	-0.0300587109285411\\
9.69472222222222	-0.0192853612367702\\
9.70027777777778	-0.020227224776873\\
9.70583333333333	-0.0222117302422664\\
9.71138888888889	-0.0207311261557996\\
9.71694444444444	-0.0195379945731772\\
9.7225	-0.0236843604249808\\
9.72805555555555	-0.0145048166760908\\
9.73361111111111	-0.0116719318488218\\
9.73916666666667	-0.0116719318488218\\
9.74472222222222	-0.0116719318488218\\
9.75027777777778	-0.0100193231824488\\
9.75583333333333	-0.0110855097481426\\
9.76138888888889	-0.0136778813075275\\
9.76694444444444	-0.0171109216687139\\
9.7725	-0.0124571935505693\\
9.77805555555556	-0.0124512970811373\\
9.78361111111111	-0.0124426487676894\\
9.78916666666667	-0.010380127122673\\
9.79472222222222	-0.0106747089328214\\
9.80027777777778	-0.00854176698333869\\
9.80583333333333	-0.00854176698333869\\
9.81138888888889	-0.00953880286560122\\
9.81694444444444	-0.00973412987661639\\
9.8225	-0.0133392789183861\\
9.82805555555555	-0.0108431786624353\\
9.83361111111111	-0.0100976600475517\\
9.83916666666667	-0.00891500364414681\\
9.84472222222222	-0.00773132702906194\\
9.85027777777778	-0.00841695609905393\\
9.85583333333333	-0.00841695609905393\\
9.86138888888889	-0.00820224754450059\\
9.86694444444444	-0.00790644227818264\\
9.8725	-0.00675070067257796\\
9.87805555555556	-0.00703850966293226\\
9.88361111111111	-0.00526114061009051\\
9.88916666666667	-0.00496239371001459\\
9.89472222222222	-0.0052642907567088\\
9.90027777777778	-0.0052642907567088\\
9.90583333333333	-0.00579380264300988\\
9.91138888888889	-0.00527081256904745\\
9.91694444444444	-0.00438427698293577\\
9.9225	-0.00497497961037735\\
9.92805555555555	-0.00505020614120394\\
9.93361111111111	-0.00504189431426504\\
9.93916666666667	-0.0050359978448331\\
9.94472222222222	-0.00414588402085703\\
9.95027777777778	-0.00502835000330548\\
9.95583333333333	-0.00230219371136086\\
9.96138888888889	6.59661728984359e-05\\
9.96694444444444	-0.00567006335417183\\
9.9725	-0.00572356444239335\\
9.97805555555555	-0.00186996046115068\\
9.98361111111111	-0.00186996046115068\\
9.98916666666667	0.000501477092046578\\
9.99472222222222	-0.00157136862900783\\
10.0002777777778	-0.00157136862900783\\
10.0058333333333	-0.00157136862900783\\
10.0113888888889	-0.00157136862900783\\
10.0169444444444	-0.00304969073114281\\
10.0225	-0.00658859575578534\\
10.0280555555556	-0.00573959536319075\\
10.0336111111111	-0.00747135521543832\\
10.0391666666667	-0.00638336257927563\\
10.0447222222222	-0.00814142759445333\\
10.0502777777778	-0.00907726486614728\\
10.0558333333333	-0.00879138193439901\\
10.0613888888889	-0.00879138193439901\\
10.0669444444444	-0.0147894124456029\\
10.0725	-0.0147894124456029\\
10.0780555555556	-0.0153698975433476\\
10.0836111111111	-0.0153698975433476\\
10.0891666666667	-0.0153698975433476\\
10.0947222222222	-0.0153698975433476\\
10.1002777777778	-0.0147769103258223\\
10.1058333333333	-0.012422629508851\\
10.1113888888889	-0.0107742279507548\\
10.1169444444444	-0.0119689598022938\\
10.1225	-0.0119689598022938\\
10.1280555555556	-0.0100244374269044\\
10.1336111111111	-0.0100244374269044\\
10.1391666666667	-0.0100244374269044\\
10.1447222222222	-0.012471211881614\\
10.1502777777778	-0.012471211881614\\
10.1558333333333	-0.012471211881614\\
10.1613888888889	-0.012471211881614\\
10.1669444444444	-0.0132612623772216\\
10.1725	-0.0131220043265578\\
10.1780555555556	-0.0131220043265578\\
10.1836111111111	-0.0125280113198749\\
10.1891666666667	-0.011934018313192\\
10.1947222222222	-0.011934018313192\\
10.2002777777778	-0.011934018313192\\
10.2058333333333	-0.011934018313192\\
10.2113888888889	-0.011934018313192\\
10.2169444444444	-0.011934018313192\\
10.2225	-0.011339494694273\\
10.2280555555556	-0.00985676692225945\\
10.2336111111111	-0.0107443837912014\\
10.2391666666667	-0.0104583473702193\\
10.2447222222222	-0.0104583473702193\\
10.2502777777778	-0.0113383979951608\\
10.2558333333333	-0.0107464456230549\\
10.2613888888889	-0.0083076072338097\\
10.2669444444444	-0.00949209050298329\\
10.2725	-0.00742107164769802\\
10.2780555555556	-0.009493917368752\\
10.2836111111111	-0.00918720274018771\\
10.2891666666667	-0.0127406525477085\\
10.2947222222222	-0.0127406525477085\\
10.3002777777778	-0.012571412340311\\
10.3058333333333	-0.012571412340311\\
10.3113888888889	-0.0138872642763697\\
10.3169444444444	-0.0120020063443886\\
10.3225	-0.0121119061585628\\
10.3280555555556	-0.00732249731389231\\
10.3336111111111	-0.00732249731389231\\
10.3391666666667	-0.00751782432490833\\
10.3447222222222	-0.00763792480509733\\
10.3502777777778	-0.00851207896060732\\
10.3558333333333	-0.00769127584272394\\
10.3613888888889	-0.00769127584272394\\
10.3669444444444	-0.00769127584272394\\
10.3725	-0.00769127584272394\\
10.3780555555556	-0.00769127584272394\\
10.3836111111111	-0.00769127584272394\\
10.3891666666667	-0.00827762100787921\\
10.3947222222222	-0.00827762100787921\\
10.4002777777778	-0.00769156510520526\\
10.4058333333333	-0.00650533046393516\\
10.4113888888889	-0.00887429700228234\\
10.4169444444444	-0.00887429700228234\\
10.4225	-0.00887429700228234\\
10.4280555555556	-0.00769169645435704\\
10.4336111111111	-0.00650725077191536\\
10.4391666666667	-0.00650725077191536\\
10.4447222222222	-0.00706317566843647\\
10.4502777777778	-0.00706317566843647\\
10.4558333333333	-0.00706317566843647\\
10.4613888888889	-0.00706317566843647\\
10.4669444444444	-0.00693792470080506\\
10.4725	-0.00696327616863116\\
10.4780555555556	-0.00770612023867588\\
10.4836111111111	-0.00770612023867588\\
10.4891666666667	-0.00770612023867588\\
10.4947222222222	-0.0073736826833166\\
10.5002777777778	-0.0073736826833166\\
10.5058333333333	-0.0064829222630978\\
10.5113888888889	-0.00648051943183353\\
10.5169444444444	-0.00639137850201736\\
10.5225	-0.00639137850201736\\
10.5280555555556	-0.00620493787899683\\
10.5336111111111	-0.00463013119393866\\
10.5391666666667	-0.00396384052721108\\
10.5447222222222	-0.00262989824137346\\
10.5502777777778	-0.00262989824137346\\
10.5558333333333	-0.00629230571553153\\
10.5613888888889	-0.00629230571553153\\
10.5669444444444	-0.0183021106015747\\
10.5725	-0.0183021106015747\\
10.5780555555556	-0.0170421829375775\\
10.5836111111111	-0.0204227902002728\\
10.5891666666667	-0.0173498208944504\\
10.5947222222222	-0.0156633244637713\\
10.6002777777778	-0.0168500734368413\\
10.6058333333333	-0.0168500734368413\\
10.6113888888889	-0.017815540029884\\
10.6169444444444	-0.017815540029884\\
10.6225	-0.017815540029884\\
10.6280555555556	-0.0162594915485944\\
10.6336111111111	-0.0162594915485944\\
10.6391666666667	-0.0171497367217231\\
10.6447222222222	-0.0171497367217231\\
10.6502777777778	-0.0158197350707122\\
10.6558333333333	-0.0158197350707122\\
10.6613888888889	-0.0143368845416395\\
10.6669444444444	-0.01247316679789\\
10.6725	-0.0139561831491617\\
10.6780555555556	-0.0121770554141367\\
10.6836111111111	-0.00798113852523238\\
10.6891666666667	-0.00590547541762244\\
10.6947222222222	-0.00590547541762244\\
10.7002777777778	-0.00738891183211118\\
10.7058333333333	-0.00560936954179443\\
10.7113888888889	-0.00353346918192913\\
10.7169444444444	-0.00353346918192913\\
10.7225	-0.00353346918192913\\
10.7280555555556	-0.00501715002521968\\
10.7336111111111	-0.00620446562241524\\
10.7391666666667	-0.00472104414939167\\
10.7447222222222	-0.00650326018385218\\
10.7502777777778	-0.00739414924930908\\
10.7558333333333	-0.00739414924930908\\
10.7613888888889	-0.00698618374464226\\
10.7669444444444	-0.00491082028045927\\
10.7725	-0.00304249664737047\\
10.7780555555556	-0.00304249664737047\\
10.7836111111111	-0.00304249664737047\\
10.7891666666667	-0.00304249664737047\\
10.7947222222222	-0.000966833539760536\\
10.8002777777778	0.00741875194304429\\
10.8058333333333	0.0118270387829356\\
10.8113888888889	0.0118270387829356\\
10.8169444444444	0.0102717494822276\\
10.8225	0.0102717494822276\\
10.8280555555556	0.0108637860156155\\
10.8336111111111	0.0108637860156155\\
10.8391666666667	0.0101972902476894\\
10.8447222222222	-0.00598241489643954\\
10.8502777777778	-0.00215819573945\\
10.8558333333333	-0.00807097737893331\\
10.8613888888889	-0.00770140727281371\\
10.8669444444444	-0.00561991323831175\\
10.8725	-0.00120345244673584\\
10.8780555555556	-0.00120345244673584\\
10.8836111111111	-0.00120345244673584\\
10.8891666666667	-0.00120345244673584\\
10.8947222222222	-0.00120345244673584\\
10.9002777777778	-0.00120345244673584\\
10.9058333333333	-0.00120345244673584\\
10.9113888888889	-0.0112394715577492\\
10.9169444444444	-0.0112394715577492\\
10.9225	-0.0127208822983051\\
10.9280555555556	-0.012721229627608\\
10.9336111111111	-0.0118484652285839\\
10.9391666666667	-0.00874630391574263\\
10.9447222222222	-0.0112860813303385\\
10.9502777777778	-0.00846878672713571\\
10.9558333333333	-0.00719040427172507\\
10.9613888888889	-0.00665478726613479\\
10.9669444444444	-0.00672484548084828\\
10.9725	-0.00613648643619224\\
10.9780555555556	-0.0054345895963848\\
10.9836111111111	-0.0041244344522553\\
10.9891666666667	-0.0041244344522553\\
10.9947222222222	-0.00431188068528873\\
11.0002777777778	-0.00286207053681678\\
11.0058333333333	-0.00286207053681678\\
11.0113888888889	-0.0030421785689739\\
11.0169444444444	-0.00209108636880737\\
11.0225	-0.000981829630460366\\
11.0280555555556	-0.000981829630460366\\
11.0336111111111	-0.00158025699696\\
11.0391666666667	-0.000256515677122606\\
11.0447222222222	0.00173497613730264\\
11.0502777777778	0.00173497613730264\\
11.0558333333333	0.00529022779879853\\
11.0613888888889	0.00529022779879853\\
11.0669444444444	0.00529022779879853\\
11.0725	0.00785879096353189\\
11.0780555555556	0.00785879096353189\\
11.0836111111111	0.00637592117316864\\
11.0891666666667	0.00637592117316864\\
11.0947222222222	0.00637592117316864\\
11.1002777777778	0.00637592117316864\\
11.1058333333333	0.00860176729801769\\
11.1113888888889	0.0154191172545937\\
11.1169444444444	0.0201394370824234\\
11.1225	0.0201394370824234\\
11.1280555555556	0.0201394370824234\\
11.1336111111111	0.022440529452734\\
11.1391666666667	0.022440529452734\\
11.1447222222222	0.0246326666936496\\
11.1502777777778	-0.00448849670039961\\
11.1558333333333	-0.00406072245569584\\
11.1613888888889	-0.00406072245569584\\
11.1669444444444	-0.00182007926301323\\
11.1725	0.000674138242966199\\
11.1780555555556	0.000674138242966199\\
11.1836111111111	0.0030449066352885\\
11.1891666666667	0.00126629901027941\\
11.1947222222222	0.00126629901027941\\
11.2002777777778	-0.00221170640241942\\
11.2058333333333	-0.00488704234042602\\
11.2113888888889	-0.00191718585404366\\
11.2169444444444	0.000747901501597543\\
11.2225	0.000747901501597543\\
11.2280555555556	0.000747901501597543\\
11.2336111111111	0.000747901501597543\\
11.2391666666667	0.000784218239662801\\
11.2447222222222	0.00424117741417435\\
11.2502777777778	0.00172569710973976\\
11.2558333333333	-0.00948855138789692\\
11.2613888888889	-0.00918183675933178\\
11.2669444444444	-0.00591619594216405\\
11.2725	-0.00591619594216405\\
11.2780555555556	0.00268465811175524\\
11.2836111111111	0.00268465811175524\\
11.2891666666667	0.00268465811175524\\
11.2947222222222	-0.000427152496636654\\
11.3002777777778	0.00272027667941675\\
11.3058333333333	0.00272027667941675\\
11.3113888888889	0.00272027667941675\\
11.3169444444444	0.00538716588903224\\
11.3225	0.00538716588903224\\
11.3280555555556	0.00538716588903224\\
11.3336111111111	0.0080544996872776\\
11.3391666666667	0.0080544996872776\\
11.3447222222222	0.0080544996872776\\
11.3502777777778	-0.00746968215416511\\
11.3558333333333	-0.00746968215416511\\
11.3613888888889	-0.00983636670818091\\
11.3669444444444	-0.0095344032103177\\
11.3725	-0.00656488321044442\\
11.3780555555556	-0.00922768858175341\\
11.3836111111111	-0.00922768858175341\\
11.3891666666667	-0.0059688205843802\\
11.3947222222222	-0.0059688205843802\\
11.4002777777778	-0.0059688205843802\\
11.4058333333333	-0.0112942422964807\\
11.4113888888889	-0.0112942422964807\\
11.4169444444444	-0.0106967484963289\\
11.4225	-0.000922135442862176\\
11.4280555555556	-0.000922135442862176\\
11.4336111111111	0.00500892921645458\\
11.4391666666667	0.00500892921645458\\
11.4447222222222	0.00500892921645458\\
11.4502777777778	0.00500892921645458\\
11.4558333333333	0.00293604590866783\\
11.4613888888889	0.00115932100490787\\
11.4669444444444	0.00115932100490787\\
11.4725	-2.34108921700438e-05\\
11.4780555555556	-2.34108921700438e-05\\
11.4836111111111	0.00175331401159076\\
11.4891666666667	0.00241217346093718\\
11.4947222222222	0.0033105838145921\\
11.5002777777778	0.0033105838145921\\
11.5058333333333	0.0033105838145921\\
11.5113888888889	-0.00042469042681475\\
11.5169444444444	-0.00161032767937991\\
11.5225	-0.00161032767937991\\
11.5280555555556	-0.00161032767937991\\
11.5336111111111	-0.00161032767937991\\
11.5391666666667	-0.00161032767937991\\
11.5447222222222	-0.000125253580650324\\
11.5502777777778	-0.00131449575815271\\
11.5558333333333	-0.00131449575815271\\
11.5613888888889	0.00538557645579517\\
11.5669444444444	0.0108297587425714\\
11.5725	0.00245176079748638\\
11.5780555555556	0.00245176079748638\\
11.5836111111111	0.00245176079748638\\
11.5891666666667	0.000972908083115279\\
11.5947222222222	0.00275552945630803\\
11.6002777777778	0.00126873963792799\\
11.6058333333333	8.65383530861918e-05\\
11.6113888888889	8.65383530861918e-05\\
11.6169444444444	8.65383530861918e-05\\
11.6225	-0.000812915569023966\\
11.6280555555556	0.000376037712412435\\
11.6336111111111	0.000376037712412435\\
11.6391666666667	0.00251161318175255\\
11.6447222222222	0.000737170262323962\\
11.6502777777778	-0.00103708362658717\\
11.6558333333333	-0.00103708362658717\\
11.6613888888889	0.00145017961524061\\
11.6669444444444	0.00145017961524061\\
11.6725	-0.000346424184942465\\
11.6780555555556	-0.000346424184942465\\
11.6836111111111	-0.000346424184942465\\
11.6891666666667	0.00172716711905\\
11.6947222222222	0.00172716711905\\
11.7002777777778	0.00172716711905\\
11.7058333333333	0.00172716711905\\
11.7113888888889	0.00172716711905\\
11.7169444444444	0.00172716711905\\
11.7225	0.00172716711905\\
11.7280555555556	0.00172716711905\\
11.7336111111111	0.000241804124254008\\
11.7391666666667	0.000241804124254008\\
11.7447222222222	0.000833351100234667\\
11.7502777777778	0.000829815839607196\\
11.7558333333333	0.00112558932759752\\
11.7613888888889	0.00112558932759752\\
11.7669444444444	0.00112558932759752\\
11.7725	0.00112558932759752\\
11.7780555555556	0.00142433622767302\\
11.7836111111111	0.00142433622767302\\
11.7891666666667	0.00172308312774894\\
11.7947222222222	0.00350437460592522\\
11.8002777777778	0.00380277740755061\\
11.8058333333333	0.00380277740755061\\
11.8113888888889	0.00380277740755061\\
11.8169444444444	0.00380277740755061\\
11.8225	0.00380277740755061\\
11.8280555555556	0.00380277740755061\\
11.8336111111111	0.00380277740755061\\
11.8391666666667	0.00350912760510667\\
11.8447222222222	0.00351175368788898\\
11.8502777777778	0.00440011613976896\\
11.8558333333333	0.00410646633732502\\
11.8613888888889	0.00492079128199163\\
11.8669444444444	0.00373154910448925\\
11.8725	0.00522358024076249\\
11.8780555555556	0.00402275333579411\\
11.8836111111111	0.0043122574153923\\
11.8891666666667	0.00431191008608902\\
11.8947222222222	0.00550086336752542\\
11.9002777777778	0.00550086336752542\\
11.9058333333333	0.00550086336752542\\
11.9113888888889	0.00550086336752542\\
11.9169444444444	0.00550086336752542\\
11.9225	0.0046116942615563\\
11.9280555555556	0.0046116942615563\\
11.9336111111111	0.0046116942615563\\
11.9391666666667	0.00491044116163222\\
11.9447222222222	0.00491044116163222\\
11.9502777777778	0.00491044116163222\\
11.9558333333333	0.00491044116163222\\
11.9613888888889	0.00610635148061227\\
11.9669444444444	0.00764968531536602\\
11.9725	0.00764968531536602\\
11.9780555555556	0.00646337518042371\\
11.9836111111111	0.00973491246702297\\
11.9891666666667	0.00973491246702297\\
11.9947222222222	0.00973491246702297\\
12.0002777777778	0.00973491246702297\\
12.0058333333333	0.00765405414828699\\
12.0113888888889	0.00676816271125626\\
12.0169444444444	0.00706690961133217\\
12.0225	0.00706690961133217\\
12.0280555555556	0.00706690961133217\\
12.0336111111111	0.00706690961133217\\
12.0391666666667	0.00854825928073784\\
12.0447222222222	0.0103265231916439\\
12.0502777777778	0.0103265231916439\\
12.0558333333333	0.00884409223940794\\
12.0613888888889	0.00765485006190514\\
12.0669444444444	0.0085419560066371\\
12.0725	0.00884869721964803\\
12.0780555555556	0.00884869721964803\\
12.0836111111111	0.0100339260717601\\
12.0891666666667	0.0115228419851666\\
12.0947222222222	0.0121114978334201\\
12.1002777777778	0.0121114978334201\\
12.1058333333333	0.0137313599705506\\
12.1113888888889	0.0138197333209858\\
12.1169444444444	0.0129901968698105\\
12.1225	0.0123287036745985\\
12.1280555555556	0.0123287036745985\\
12.1336111111111	0.0123287036745985\\
12.1391666666667	0.0145491192732552\\
12.1447222222222	0.0144956181850337\\
12.1502777777778	0.0143005837773695\\
12.1558333333333	0.0145182647269834\\
12.1613888888889	0.0145182647269834\\
12.1669444444444	0.0145182647269834\\
12.1725	0.0145182647269834\\
12.1780555555556	0.0145182647269834\\
12.1836111111111	0.0145182647269834\\
12.1891666666667	0.0160075171268981\\
12.1947222222222	0.0154050269099463\\
12.2002777777778	0.0154050269099463\\
12.2058333333333	0.0154050269099463\\
12.2113888888889	0.0154050269099463\\
12.2169444444444	0.0154050269099463\\
12.2225	0.0154050269099463\\
12.2280555555556	0.0154050269099463\\
12.2336111111111	0.0154050269099463\\
12.2391666666667	0.0154050269099463\\
12.2447222222222	0.0166905965887481\\
12.2502777777778	0.0149138716849881\\
12.2558333333333	0.0137232026839013\\
12.2613888888889	0.0137232026839013\\
12.2669444444444	0.0137232026839013\\
12.2725	0.0137232026839013\\
12.2780555555556	0.0137232026839013\\
12.2836111111111	0.0131291341835458\\
12.2891666666667	0.0128354843811018\\
12.2947222222222	0.0128354843811018\\
12.3002777777778	0.0128354843811018\\
12.3058333333333	0.0137245924159207\\
12.3113888888889	0.0137245924159207\\
12.3169444444444	0.0137245924159207\\
12.3225	0.0140273813746916\\
12.3280555555556	0.0140273813746916\\
12.3336111111111	0.0137939903056837\\
12.3391666666667	0.0129048211997145\\
12.3447222222222	0.0129044738704113\\
12.3502777777778	0.0134989942584771\\
12.3558333333333	0.0134989942584771\\
12.3613888888889	0.0134989942584771\\
12.3669444444444	0.0146842231105892\\
12.3725	0.0155808973894085\\
12.3780555555556	0.0155808973894085\\
12.3836111111111	0.0155808973894085\\
12.3891666666667	0.0164692598412889\\
12.3947222222222	0.0164692598412889\\
12.4002777777778	0.0164692598412889\\
12.4058333333333	0.0164692598412889\\
12.4113888888889	0.0164692598412889\\
12.4169444444444	0.0164692598412889\\
12.4225	0.0170461633553557\\
12.4280555555556	0.0170461633553557\\
12.4336111111111	0.0170461633553557\\
12.4391666666667	0.0176259082074016\\
12.4447222222222	0.0188103914765752\\
12.4502777777778	0.0188103914765752\\
12.4558333333333	0.0188103914765752\\
12.4613888888889	0.0188103914765752\\
12.4669444444444	0.0194025955244297\\
12.4725	0.0213341226859607\\
12.4780555555556	0.0213341226859607\\
12.4836111111111	0.0213341226859607\\
12.4891666666667	0.0223790413812779\\
12.4947222222222	0.0223790413812779\\
12.5002777777778	0.0175181607672678\\
12.5058333333333	0.0186274175056148\\
12.5113888888889	0.0186274175056148\\
12.5169444444444	0.0186274175056148\\
12.5225	0.0186274175056148\\
12.5280555555556	0.02005452050386\\
12.5336111111111	0.0206568100402217\\
12.5391666666667	0.0209892100088483\\
12.5447222222222	0.0209892100088483\\
12.5502777777778	0.0208183401591857\\
12.5558333333333	0.0197684855683028\\
12.5613888888889	0.0197684855683028\\
12.5669444444444	0.0197684855683028\\
12.5725	0.0191700582018032\\
12.5780555555556	0.0191700582018032\\
12.5836111111111	0.0201976787043474\\
12.5891666666667	0.0201976787043474\\
12.5947222222222	0.019537877735764\\
12.6002777777778	0.019537877735764\\
12.6058333333333	0.019537877735764\\
12.6113888888889	0.0208594916082065\\
12.6169444444444	0.0199649434244128\\
12.6225	0.0199649434244128\\
12.6280555555556	0.0184843393379452\\
12.6336111111111	0.0184843393379452\\
12.6391666666667	0.0184843393379452\\
12.6447222222222	0.0184843393379452\\
12.6502777777778	0.020027673172699\\
12.6558333333333	0.020027673172699\\
12.6613888888889	0.020027673172699\\
12.6669444444444	0.0219468524952462\\
12.6725	0.0219468524952462\\
12.6780555555556	0.0219468524952462\\
12.6836111111111	0.0217086293354075\\
12.6891666666667	0.0217086293354075\\
12.6947222222222	0.0217086293354075\\
12.7002777777778	0.0205258974383296\\
12.7058333333333	0.0205258974383296\\
12.7113888888889	0.0205258974383296\\
12.7169444444444	0.0205258974383296\\
12.7225	0.0205258974383296\\
12.7280555555556	0.0205258974383296\\
12.7336111111111	0.0205258974383296\\
12.7391666666667	0.0205258974383296\\
12.7447222222222	0.021849638758167\\
12.7502777777778	0.021849638758167\\
12.7558333333333	0.021849638758167\\
12.7613888888889	0.021849638758167\\
12.7669444444444	0.021849638758167\\
12.7725	0.021849638758167\\
12.7780555555556	0.0231352084369687\\
12.7836111111111	0.0231352084369687\\
12.7891666666667	0.0231352084369687\\
12.7947222222222	0.0243560084442084\\
12.8002777777778	0.0243560084442084\\
12.8058333333333	0.0243560084442084\\
12.8113888888889	0.0243560084442084\\
12.8169444444444	0.0243560084442084\\
12.8225	0.0249465903324562\\
12.8280555555556	0.0249465903324562\\
12.8336111111111	0.0249465903324562\\
12.8391666666667	0.0249465903324562\\
12.8447222222222	0.02583675463831\\
12.8502777777778	0.0269310094344027\\
12.8558333333333	0.0198052516929772\\
12.8613888888889	0.0210524647104377\\
12.8669444444444	0.0210524647104377\\
12.8725	0.0210524647104377\\
12.8780555555556	0.0226860934126623\\
12.8836111111111	0.0220262924440789\\
12.8891666666667	0.0220262924440789\\
12.8947222222222	0.0220262924440789\\
12.9002777777778	0.0220262924440789\\
12.9058333333333	0.0220262924440789\\
12.9113888888889	0.0226829380479827\\
12.9169444444444	0.0225900749004729\\
12.9225	0.0225900749004729\\
12.9280555555556	0.0225900749004729\\
12.9336111111111	0.0225900749004729\\
12.9391666666667	0.0222877683512664\\
12.9447222222222	0.0222877683512664\\
12.9502777777778	0.0222877683512664\\
12.9558333333333	0.0231742428662279\\
12.9613888888889	0.0231757818733735\\
12.9669444444444	0.0231757818733735\\
12.9725	0.023467814995615\\
12.9780555555556	0.0228755733610278\\
12.9836111111111	0.0232344239266079\\
12.9891666666667	0.0232344239266079\\
12.9947222222222	0.0232344239266079\\
13.0002777777778	0.0230543158944508\\
13.0058333333333	0.0230543158944508\\
13.0113888888889	0.0230543158944508\\
13.0169444444444	0.0230543158944508\\
13.0225	0.0230543158944508\\
13.0280555555556	0.0233456778034156\\
13.0336111111111	0.0233456778034156\\
13.0391666666667	0.0233456778034156\\
13.0447222222222	0.0233456778034156\\
13.0502777777778	0.0234128775639642\\
13.0558333333333	0.0234128775639642\\
13.0613888888889	0.0234128775639642\\
13.0669444444444	0.0234128775639642\\
13.0725	0.0234128775639642\\
13.0780555555556	0.0247512278086278\\
13.0836111111111	0.0247512278086278\\
13.0891666666667	0.0247512278086278\\
13.0947222222222	0.0251725995935533\\
13.1002777777778	0.0251725995935533\\
13.1058333333333	0.0251725620068205\\
13.1113888888889	0.0251725620068205\\
13.1169444444444	0.0259574213391957\\
13.1225	0.0268954785825211\\
13.1280555555556	0.0268954785825211\\
13.1336111111111	0.0268954785825211\\
13.1391666666667	0.0277776553024882\\
13.1447222222222	0.0277776553024882\\
13.1502777777778	0.028964420555994\\
13.1558333333333	0.028964420555994\\
13.1613888888889	0.0298524951492513\\
13.1669444444444	0.0273349776712128\\
13.1725	0.0273349776712128\\
13.1780555555556	0.0273349776712128\\
13.1836111111111	0.0273349776712128\\
13.1891666666667	0.0273349776712128\\
13.1947222222222	0.0273349776712128\\
13.2002777777778	0.0273349776712128\\
13.2058333333333	0.0273349776712128\\
13.2113888888889	0.0273349776712128\\
13.2169444444444	0.0258543735847452\\
13.2225	0.0258543735847452\\
13.2280555555556	0.0258543735847452\\
13.2336111111111	0.0282247397114198\\
13.2391666666667	0.0308915784391575\\
13.2447222222222	0.0338550685964241\\
13.2502777777778	0.0338550685964241\\
13.2558333333333	0.0335666709595448\\
13.2613888888889	0.0335666709595448\\
13.2669444444444	0.0335666709595448\\
13.2725	0.0374187125992092\\
13.2780555555556	0.0380162063993611\\
13.2836111111111	0.0387209849685169\\
13.2891666666667	0.0348714143437017\\
13.2947222222222	0.0399105653353224\\
13.3002777777778	0.0399105653353224\\
13.3058333333333	0.035764873893214\\
13.3113888888889	0.0313230616338129\\
13.3169444444444	0.0262949042092556\\
13.3225	0.0215569711325596\\
13.3280555555556	0.0268885454321685\\
13.3336111111111	0.0172418094291155\\
13.3391666666667	0.025805104255399\\
13.3447222222222	0.0360393929170313\\
13.3502777777778	0.0407797970085761\\
13.3558333333333	0.0407797970085761\\
13.3613888888889	0.0407797970085761\\
13.3669444444444	0.0407797970085761\\
13.3725	0.0407797970085761\\
13.3780555555556	0.0360444900146633\\
13.3836111111111	0.0416656793056072\\
13.3891666666667	0.0416656793056072\\
13.3947222222222	0.0416656793056072\\
13.4002777777778	0.0419476590803905\\
13.4058333333333	0.0422367093764287\\
13.4113888888889	0.0425262134560269\\
13.4169444444444	0.0366152241599638\\
13.4225	0.0366152241599638\\
13.4280555555556	0.0310516583796747\\
13.4336111111111	0.025665707311392\\
13.4391666666667	0.025665707311392\\
13.4447222222222	0.025665707311392\\
13.4502777777778	0.025665707311392\\
13.4558333333333	0.036256670534191\\
13.4613888888889	0.036256670534191\\
13.4669444444444	0.0275325100603679\\
13.4725	0.0373021607080963\\
13.4780555555556	0.0373021607080963\\
13.4836111111111	0.0373021607080963\\
13.4891666666667	0.0373021607080963\\
13.4947222222222	0.0378896434343543\\
13.5002777777778	0.0430502945968702\\
13.5058333333333	0.0430502945968702\\
13.5113888888889	0.0392993473123029\\
13.5169444444444	0.0437403529176162\\
13.5225	0.039890782292801\\
13.5280555555556	0.039890782292801\\
13.5336111111111	0.0296673507453388\\
13.5391666666667	0.0296673507453388\\
13.5447222222222	0.0296673507453388\\
13.5502777777778	0.0208695876002725\\
13.5558333333333	0.0159250442894011\\
13.5613888888889	0.0135587255856004\\
13.5669444444444	0.00913892487369177\\
13.5725	0.0345093832751608\\
13.5780555555556	0.0321487285637521\\
13.5836111111111	0.0321487285637521\\
13.5891666666667	0.0321487285637521\\
13.5947222222222	0.0321487285637521\\
13.6002777777778	0.0300758828426981\\
13.6058333333333	0.0300758828426981\\
13.6113888888889	0.0280116854350912\\
13.6169444444444	0.0280116854350912\\
13.6225	0.0280116854350912\\
13.6280555555556	0.0260640906816686\\
13.6336111111111	0.0260640906816686\\
13.6391666666667	0.028690150658833\\
13.6447222222222	0.028690150658833\\
13.6502777777778	0.028690150658833\\
13.6558333333333	0.028690150658833\\
13.6613888888889	0.028690150658833\\
13.6669444444444	0.028690150658833\\
13.6725	0.028690150658833\\
13.6780555555556	0.028690150658833\\
13.6836111111111	0.028690150658833\\
13.6891666666667	0.0265779428784923\\
13.6947222222222	0.0354999099568104\\
13.7002777777778	0.0338377454244208\\
13.7058333333333	0.0338377454244208\\
13.7113888888889	0.0338377454244208\\
13.7169444444444	0.0365010059142933\\
13.7225	0.0365010059142933\\
13.7280555555556	0.0323568534793309\\
13.7336111111111	0.0323568534793309\\
13.7391666666667	0.0353118759203529\\
13.7447222222222	0.0382753660776195\\
13.7502777777778	0.0415377921654795\\
13.7558333333333	0.0418365390655554\\
13.7613888888889	0.0418365390655554\\
13.7669444444444	0.0418365390655554\\
13.7725	0.0418365390655554\\
13.7780555555556	0.0391759217221766\\
13.7836111111111	0.0391759217221766\\
13.7891666666667	0.0365735641148219\\
13.7947222222222	0.0343471258624954\\
13.8002777777778	0.0343471258624954\\
13.8058333333333	0.0367178437729391\\
13.8113888888889	0.0367178437729391\\
13.8169444444444	0.0367178437729391\\
13.8225	0.0367178437729391\\
13.8280555555556	0.0367178437729391\\
13.8336111111111	0.0393136795679555\\
13.8391666666667	0.0393136795679555\\
13.8447222222222	0.0393136795679555\\
13.8502777777778	0.0393136795679555\\
13.8558333333333	0.0349726611148324\\
13.8613888888889	0.0349726611148324\\
13.8669444444444	0.032846314305556\\
13.8725	0.0348797979673242\\
13.8780555555556	0.0348797979673242\\
13.8836111111111	0.0348797979673242\\
13.8891666666667	0.038379746686623\\
13.8947222222222	0.038379746686623\\
13.9002777777778	0.038379746686623\\
13.9058333333333	0.038379746686623\\
13.9113888888889	0.038379746686623\\
13.9169444444444	0.0409207178048765\\
13.9225	0.0412133033615426\\
13.9280555555556	0.0412133033615426\\
13.9336111111111	0.0412133033615426\\
13.9391666666667	0.0412097681009151\\
13.9447222222222	0.0412097681009151\\
13.9502777777778	0.0388415471455056\\
13.9558333333333	0.0388415471455056\\
13.9613888888889	0.0388415471455056\\
13.9669444444444	0.0415015282635151\\
13.9725	0.0388371864908124\\
13.9780555555556	0.0388371864908124\\
13.9836111111111	0.0364765317794041\\
13.9891666666667	0.0364765317794041\\
13.9947222222222	0.0364765317794041\\
14.0002777777778	0.0429813857850969\\
14.0058333333333	0.0429813857850969\\
14.0113888888889	0.0376494072971066\\
14.0169444444444	0.0376494072971066\\
14.0225	0.0435554413311902\\
14.0280555555556	0.0438480268878559\\
14.0336111111111	0.0441406124445216\\
14.0391666666667	0.0441406124445216\\
14.0447222222222	0.0444331980011873\\
14.0502777777778	0.044725783557853\\
14.0558333333333	0.0450183691145188\\
14.0613888888889	0.0456035402278502\\
14.0669444444444	0.0461887113411816\\
14.0725	0.0441234100800025\\
14.0780555555556	0.0473643954344357\\
14.0836111111111	0.0476569809911014\\
14.0891666666667	0.0476569809911014\\
14.0947222222222	0.0476569809911014\\
14.1002777777778	0.0476569809911014\\
14.1058333333333	0.0476569809911014\\
14.1113888888889	0.0476569809911014\\
14.1169444444444	0.0479527544790917\\
14.1225	0.0479527544790917\\
14.1280555555556	0.0479527544790917\\
14.1336111111111	0.0479461377413963\\
14.1391666666667	0.045428789347164\\
14.1447222222222	0.0492834570696108\\
14.1502777777778	0.0492834570696108\\
14.1558333333333	0.0492834570696108\\
14.1613888888889	0.0492834570696108\\
14.1669444444444	0.0492834570696108\\
14.1725	0.0493786586329425\\
14.1780555555556	0.0493786586329425\\
14.1836111111111	0.0493786586329425\\
14.1891666666667	0.0496744321209328\\
14.1947222222222	0.0496744321209328\\
14.2002777777778	0.0499702056089232\\
14.2058333333333	0.0499702056089232\\
14.2113888888889	0.0499702056089232\\
14.2169444444444	0.0499702056089232\\
14.2225	0.0532209178614558\\
14.2280555555556	0.0505554831765117\\
14.2336111111111	0.0505554831765117\\
14.2391666666667	0.0505554831765117\\
14.2447222222222	0.0505554831765117\\
14.2502777777778	0.0505554831765117\\
14.2558333333333	0.0508542300765876\\
14.2613888888889	0.0508542300765876\\
14.2669444444444	0.0511529769766635\\
14.2725	0.0511529769766635\\
14.2780555555556	0.0514517238767394\\
14.2836111111111	0.0514543499595213\\
14.2891666666667	0.0514543499595213\\
14.2947222222222	0.0520518437596732\\
14.3002777777778	0.0520518437596732\\
14.3058333333333	0.052649337559825\\
14.3113888888889	0.0529480844599009\\
14.3169444444444	0.0535455782600528\\
14.3225	0.0541430720602046\\
14.3280555555556	0.0550393127604323\\
14.3336111111111	0.0550393127604323\\
14.3391666666667	0.0550393127604323\\
14.3447222222222	0.0553380596605083\\
14.3502777777778	0.0559355534606601\\
14.3558333333333	0.0559355534606601\\
14.3613888888889	0.056234300360736\\
14.3669444444444	0.0565330472608119\\
14.3725	0.0569364528471099\\
14.3780555555556	0.0490039205284729\\
14.3836111111111	0.0452936079543219\\
14.3891666666667	0.0416866570585779\\
14.3947222222222	0.0416866570585779\\
14.4002777777778	0.0458220999183205\\
14.4058333333333	0.0420304269509602\\
14.4113888888889	0.0420304269509602\\
14.4169444444444	0.0461658698107027\\
14.4225	0.0461658698107027\\
14.4280555555556	0.0420288266820419\\
14.4336111111111	0.0420288266820419\\
14.4391666666667	0.0455527224506431\\
14.4447222222222	0.0382781374306487\\
14.4502777777778	0.0382781374306487\\
14.4558333333333	0.032357304235359\\
14.4613888888889	0.0472085534961583\\
14.4669444444444	0.0530774687538056\\
14.4725	0.0530774687538056\\
14.4780555555556	0.0499961601006826\\
14.4836111111111	0.0499961601006826\\
14.4891666666667	0.0499961601006826\\
14.4947222222222	0.0499961601006826\\
14.5002777777778	0.0499961601006826\\
14.5058333333333	0.0437885576425182\\
14.5113888888889	0.0433420344571993\\
14.5169444444444	0.0410026430779785\\
14.5225	0.0435171279981622\\
14.5280555555556	0.0435171279981622\\
14.5336111111111	0.0435171279981622\\
14.5391666666667	0.0435171279981622\\
14.5447222222222	0.0435171279981622\\
14.5502777777778	0.0520753952266145\\
14.5558333333333	0.0520753952266145\\
14.5613888888889	0.0520753952266145\\
14.5669444444444	0.0461412276811429\\
14.5725	0.0442361794317176\\
14.5780555555556	0.0538030723544636\\
14.5836111111111	0.051445532865922\\
14.5891666666667	0.051752274078933\\
14.5947222222222	0.051752274078933\\
14.6002777777778	0.051752274078933\\
14.6058333333333	0.0549866426956712\\
14.6113888888889	0.0549866426956712\\
14.6169444444444	0.0549866426956712\\
14.6225	0.0520299045349988\\
14.6280555555556	0.0520299045349988\\
14.6336111111111	0.0524467930706218\\
14.6391666666667	0.0524467930706218\\
14.6447222222222	0.0524467930706218\\
14.6502777777778	0.0471825613358012\\
14.6558333333333	0.0471825613358012\\
14.6613888888889	0.0446852714517886\\
14.6669444444444	0.0422769428541548\\
14.6725	0.0511070662847364\\
14.6780555555556	0.051801260987009\\
14.6836111111111	0.0481902567028894\\
14.6891666666667	0.0481902567028894\\
14.6947222222222	0.0532214311787972\\
14.7002777777778	0.0532214311787972\\
14.7058333333333	0.0532214311787972\\
14.7113888888889	0.0532214311787972\\
14.7169444444444	0.0532214311787972\\
14.7225	0.053524493023202\\
14.7280555555556	0.0508594056675608\\
14.7336111111111	0.0508594056675608\\
14.7391666666667	0.0582630702489782\\
14.7447222222222	0.0550064868416882\\
14.7502777777778	0.0517491578514607\\
14.7558333333333	0.0600411848847584\\
14.7613888888889	0.0561916142599432\\
14.7669444444444	0.0527008942007077\\
14.7725	0.0527008942007077\\
14.7780555555556	0.0527008942007077\\
14.7836111111111	0.0532969568267719\\
14.7891666666667	0.0532969568267719\\
14.7947222222222	0.0500396278365443\\
14.8002777777778	0.0500396278365443\\
14.8058333333333	0.0474756268905313\\
14.8113888888889	0.0448468562729554\\
14.8169444444444	0.0409198621345701\\
14.8225	0.0390035726569538\\
14.8280555555556	0.0562460189260852\\
14.8336111111111	0.0597976418678381\\
14.8391666666667	0.0597976418678381\\
14.8447222222222	0.0538943830924445\\
14.8502777777778	0.0573049890629694\\
14.8558333333333	0.0573049890629694\\
14.8613888888889	0.0573049890629694\\
14.8669444444444	0.0573049890629694\\
14.8725	0.0573049890629694\\
14.8780555555556	0.0573049890629694\\
14.8836111111111	0.0573049890629694\\
14.8891666666667	0.0573049890629694\\
14.8947222222222	0.0573049890629694\\
14.9002777777778	0.0517181676603578\\
14.9058333333333	0.0517181676603578\\
14.9113888888889	0.0561423870503954\\
14.9169444444444	0.0517064712604245\\
14.9225	0.0517064712604245\\
14.9280555555556	0.0517064712604245\\
14.9336111111111	0.0498198465424895\\
14.9391666666667	0.0498198465424895\\
14.9447222222222	0.0498198465424895\\
14.9502777777778	0.0498198465424895\\
14.9558333333333	0.0524848963113988\\
14.9613888888889	0.0626146685033625\\
14.9669444444444	0.0578767354266677\\
14.9725	0.0552175445404585\\
14.9780555555556	0.0552175445404585\\
14.9836111111111	0.0552175445404585\\
14.9891666666667	0.0526535435944454\\
14.9947222222222	0.050566308186419\\
15.0002777777778	0.0481991435020461\\
15.0058333333333	0.0481991435020461\\
15.0113888888889	0.0458324234063039\\
15.0169444444444	0.0458324234063039\\
15.0225	0.0458324234063039\\
15.0280555555556	0.0458324234063039\\
15.0336111111111	0.0458324234063039\\
15.0391666666667	0.043465859699221\\
15.0447222222222	0.0602222335399576\\
15.0502777777778	0.0602222335399576\\
15.0558333333333	0.057778040470783\\
15.0613888888889	0.057778040470783\\
15.0669444444444	0.064815264989994\\
15.0725	0.064815264989994\\
15.0780555555556	0.064815264989994\\
15.0836111111111	0.0651055662733801\\
15.0891666666667	0.0598572788808594\\
15.0947222222222	0.0637685738560158\\
15.1002777777778	0.0646812686524612\\
15.1058333333333	0.0646812686524612\\
15.1113888888889	0.0646812686524612\\
15.1169444444444	0.0646812686524612\\
15.1225	0.0620206513090824\\
15.1280555555556	0.0597144145190205\\
15.1336111111111	0.0560303342731504\\
15.1391666666667	0.0560303342731504\\
15.1447222222222	0.058400700399825\\
15.1502777777778	0.0607714183102687\\
15.1558333333333	0.0590005898759398\\
15.1613888888889	0.0590005898759398\\
15.1669444444444	0.0575199857894722\\
15.1725	0.0644048565153094\\
15.1780555555556	0.0672115765637862\\
15.1836111111111	0.0672115765637862\\
15.1891666666667	0.0647655797179032\\
15.1947222222222	0.0647655797179032\\
15.2002777777778	0.0680280058057632\\
15.2058333333333	0.0680280058057632\\
15.2113888888889	0.0675763159949857\\
15.2169444444444	0.0678658200745839\\
15.2225	0.0643139092742071\\
15.2280555555556	0.0643139092742071\\
15.2336111111111	0.0696414504656321\\
15.2391666666667	0.0696414504656321\\
15.2447222222222	0.0696440765484144\\
15.2502777777778	0.0696467026311967\\
15.2558333333333	0.0601869202678306\\
15.2613888888889	0.0555607483240243\\
15.2669444444444	0.0617934457893742\\
15.2725	0.0672742639610957\\
15.2780555555556	0.0708097213848981\\
15.2836111111111	0.0669715778098889\\
15.2891666666667	0.0669715778098889\\
15.2947222222222	0.0669715778098889\\
15.3002777777778	0.0711197402668465\\
15.3058333333333	0.0712589642470859\\
15.3113888888889	0.0715577111471618\\
15.3169444444444	0.0715629633127264\\
15.3225	0.0588957111455703\\
15.3280555555556	0.0628056956031716\\
15.3336111111111	0.0630233765527847\\
15.3391666666667	0.0667086784907404\\
15.3447222222222	0.0664140966805925\\
15.3502777777778	0.0664140966805925\\
15.3558333333333	0.070857573300755\\
15.3613888888889	0.070857573300755\\
15.3669444444444	0.070857573300755\\
15.3725	0.0673127473245192\\
15.3780555555556	0.0635190965566204\\
15.3836111111111	0.0599715432185308\\
15.3891666666667	0.0640496753266426\\
15.3947222222222	0.0729085126061877\\
15.4002777777778	0.0729596192811435\\
15.4058333333333	0.0729596192811435\\
15.4113888888889	0.0729596192811435\\
15.4169444444444	0.0732491233607417\\
15.4225	0.0738312129970056\\
15.4280555555556	0.0738312129970056\\
15.4336111111111	0.0703960598929646\\
15.4391666666667	0.0644179470362238\\
15.4447222222222	0.0644179470362238\\
15.4502777777778	0.0617256227445463\\
15.4558333333333	0.0624442579393636\\
15.4613888888889	0.0621837556897328\\
15.4669444444444	0.0603676687266853\\
15.4725	0.0629937287038496\\
15.4780555555556	0.0629937287038496\\
15.4836111111111	0.0629937287038496\\
15.4891666666667	0.0608109831686215\\
15.4947222222222	0.0608109831686215\\
15.5002777777778	0.0585875350121108\\
15.5058333333333	0.0585875350121108\\
15.5113888888889	0.0626429979811388\\
15.5169444444444	0.0650105977335329\\
15.5225	0.0673278145557551\\
15.5280555555556	0.0673278145557551\\
15.5336111111111	0.0653193728040177\\
15.5391666666667	0.0618108880480093\\
15.5447222222222	0.0587831369593792\\
15.5502777777778	0.0591477881209108\\
15.5558333333333	0.0620243246470798\\
15.5613888888889	0.0677072793115827\\
15.5669444444444	0.0677072793115827\\
15.5725	0.0736328466178754\\
15.5780555555556	0.0736328466178754\\
15.5836111111111	0.0714031381301912\\
15.5891666666667	0.0712081037225261\\
15.5947222222222	0.0712081037225261\\
15.6002777777778	0.0712081037225261\\
15.6058333333333	0.0712081037225261\\
15.6113888888889	0.0712081037225261\\
15.6169444444444	0.0740054900903276\\
15.6225	0.0692637165500594\\
15.6280555555556	0.0707451354610328\\
15.6336111111111	0.0728197325541833\\
15.6391666666667	0.0765846740394875\\
15.6447222222222	0.0733390285419192\\
15.6502777777778	0.0751180354300122\\
15.6558333333333	0.0757241751292547\\
15.6613888888889	0.0757241751292547\\
15.6669444444444	0.0761418175350287\\
15.6725	0.0776205947557276\\
15.6780555555556	0.0793965130053996\\
15.6836111111111	0.0793965130053996\\
15.6891666666667	0.0793965130053996\\
15.6947222222222	0.0793965130053996\\
15.7002777777778	0.078335664328614\\
15.7058333333333	0.07893037697805\\
15.7113888888889	0.07893037697805\\
15.7169444444444	0.07893037697805\\
15.7225	0.0783464471704026\\
15.7280555555556	0.0783464471704026\\
15.7336111111111	0.0801279250733452\\
15.7391666666667	0.0801279250733452\\
15.7447222222222	0.0801279250733452\\
15.7502777777778	0.0801331772389099\\
15.7558333333333	0.0807445470359121\\
15.7613888888889	0.0834173575719676\\
15.7669444444444	0.0822355003855764\\
15.7725	0.0795719520370808\\
15.7780555555556	0.0770186258143284\\
15.7836111111111	0.0782012263622538\\
15.7891666666667	0.0776107865816408\\
15.7947222222222	0.0782094073162901\\
15.8002777777778	0.0786510465154206\\
15.8058333333333	0.079848884747259\\
15.8113888888889	0.0802341464943683\\
15.8169444444444	0.0810128287519752\\
15.8225	0.0810128287519752\\
15.8280555555556	0.0813061172412473\\
15.8336111111111	0.0818983479077033\\
15.8391666666667	0.0823060383069964\\
15.8447222222222	0.0823060383069964\\
15.8502777777778	0.0823060383069964\\
15.8558333333333	0.0823060383069964\\
15.8613888888889	0.0828982423548509\\
15.8669444444444	0.0855687626552473\\
15.8725	0.0855687626552473\\
15.8780555555556	0.0855687626552473\\
15.8836111111111	0.0870544638393465\\
15.8891666666667	0.0870570899221284\\
15.8947222222222	0.0870570899221284\\
15.9002777777778	0.086168727470248\\
15.9058333333333	0.0852134503144795\\
15.9113888888889	0.0852134503144795\\
15.9169444444444	0.0852651246532888\\
15.9225	0.0843785640553835\\
15.9280555555556	0.0840860983223698\\
15.9336111111111	0.0798425776674964\\
15.9391666666667	0.0810633776747361\\
15.9447222222222	0.0813594609052973\\
15.9502777777778	0.0820634636727734\\
15.9558333333333	0.0826579517499648\\
15.9613888888889	0.0826579517499648\\
15.9669444444444	0.0826579517499648\\
15.9725	0.0832525962158154\\
15.9780555555556	0.0839467365128878\\
15.9836111111111	0.079421853148477\\
15.9891666666667	0.0801111199462783\\
15.9947222222222	0.0804716536219057\\
};
\addlegendentry{Dscr Stoch Ctrl w nFPC};

\addplot [color=black,dashed,line width=1.5pt]
  table[row sep=crcr]{%
9.50000138888889	0\\
9.50111194444444	0.00118448326917386\\
9.50449111111111	0.000444181225939921\\
9.50566027777778	-0.000444181225940254\\
9.50606805555556	-0.000444181225940254\\
9.50631	-0.000444181225940254\\
9.50650638888889	-0.000444181225940254\\
9.51661222222222	0.000444181225939921\\
9.52385472222222	0.00192478531240736\\
9.54045444444444	0.00148060408646722\\
9.54540055555556	0.00325732899022801\\
9.55986555555556	0.00488599348534202\\
9.57838833333333	0.00458987266804844\\
9.58957805555556	0.00414569144210808\\
9.60126472222222	0.00222090612970072\\
9.61190555555556	0.00340538939887458\\
9.619735	0.00503405389398859\\
9.62554638888889	0.00488599348534202\\
9.64076333333333	0.00621853716316245\\
9.65760166666667	0.00488599348534202\\
9.66701305555556	0.00621853716316245\\
9.67307194444444	0.00518211430263538\\
9.67669166666667	0.00458987266804844\\
9.6815325	0.00518211430263538\\
9.68680777777778	0.00444181225940188\\
9.69267194444445	0.00488599348534202\\
9.70423222222222	0.00399763103346151\\
9.71677611111111	0.00458987266804844\\
9.72511361111111	0.00458987266804844\\
9.72996777777778	0.00429375185075509\\
9.75026555555556	0.00488599348534202\\
9.76833111111111	0.00533017471128217\\
9.78471444444444	0.00607047675451589\\
9.79037527777778	0.00547823511992873\\
9.81136361111111	0.00562629552857552\\
9.82086055555556	0.0066627183891026\\
9.82731972222222	0.00651465798045603\\
9.84485055555556	0.00577435593722231\\
9.85223111111111	0.00577435593722231\\
9.86587111111111	0.00533017471128217\\
9.88255583333333	0.00577435593722231\\
9.89415277777778	0.00547823511992873\\
9.91960944444445	0.00577435593722231\\
9.94306083333333	0.00547823511992873\\
9.95475	0.00444181225940188\\
9.96290444444444	0.00414569144210808\\
9.98785638888889	0.00414569144210808\\
10.0118361111111	0.00384957062481472\\
10.0204633333333	0.00473793307669523\\
10.0404466666667	0.00547823511992873\\
10.0483366666667	0.00607047675451589\\
10.0616297222222	0.0066627183891026\\
10.0668508333333	0.00695883920639617\\
10.0777827777778	0.0066627183891026\\
10.1034741666667	0.00740302043233632\\
10.1108475	0.00784720165827646\\
10.1550372222222	0.00784720165827646\\
10.16618	0.0066627183891026\\
10.1707808333333	0.0059224163458691\\
10.2189616666667	0.0059224163458691\\
10.2282147222222	0.00518211430263538\\
10.240315	0.00547823511992873\\
10.2543530555556	0.00533017471128217\\
10.2621152777778	0.00488599348534202\\
10.2734305555556	0.00429375185075509\\
10.2854647222222	0.00458987266804844\\
10.3064977777778	0.00488599348534202\\
10.3092838888889	0.00458987266804844\\
10.3233986111111	0.00458987266804844\\
10.3397141666667	0.00488599348534202\\
10.3514469444444	0.00577435593722231\\
10.4036355555556	0.00547823511992873\\
10.4414047222222	0.00533017471128217\\
10.4481336111111	0.00607047675451589\\
10.4776477777778	0.00636659757180924\\
10.5069688888889	0.00651465798045603\\
10.5134291666667	0.00636659757180924\\
10.5302111111111	0.00651465798045603\\
10.5530877777778	0.00710689961504274\\
10.5634966666667	0.00725496002368953\\
10.5904716666667	0.00784720165827646\\
10.63591	0.00725496002368953\\
10.6638466666667	0.00799526206692325\\
10.6850736111111	0.00814332247557004\\
10.7270925	0.00814332247557004\\
10.7423130555556	0.00755108084098288\\
10.7578022222222	0.00784720165827646\\
10.7705291666667	0.00799526206692325\\
10.8023508333333	0.0082913828842166\\
10.8253030555556	0.00769914124962989\\
10.8364372222222	0.00695883920639617\\
10.8500147222222	0.00784720165827646\\
10.8563391666667	0.00784720165827646\\
10.8578588888889	0.00784720165827646\\
10.8637102777778	0.00784720165827646\\
10.9066405555556	0.00843944329286339\\
10.9104805555556	0.00814332247557004\\
10.9281575	0.00725496002368953\\
10.93547	0.00725496002368953\\
10.9427863888889	0.00725496002368953\\
10.9527866666667	0.00681077879774938\\
10.9581725	0.0066627183891026\\
10.9717638888889	0.00636659757180924\\
10.9817988888889	0.00636659757180924\\
10.9891697222222	0.0059224163458691\\
10.9943269444444	0.0059224163458691\\
10.9978458333333	0.00651465798045603\\
11.0110175	0.00695883920639617\\
11.0310494444444	0.00710689961504274\\
11.0401572222222	0.00784720165827646\\
11.0407494444444	0.00814332247557004\\
11.0514430555556	0.00784720165827646\\
11.0756444444444	0.00814332247557004\\
11.1090888888889	0.00843944329286339\\
11.1482363888889	0.0082913828842166\\
11.1502294444444	0.00858750370151018\\
11.1523738888889	0.0082913828842166\\
11.1590180555556	0.00814332247557004\\
11.19621	0.00858750370151018\\
11.2022375	0.00917974533609689\\
11.2217275	0.00903168492745032\\
11.2496625	0.00873556411015675\\
11.2578869444444	0.00932780574474368\\
11.2664038888889	0.00903168492745032\\
11.2764038888889	0.00903168492745032\\
11.2937172222222	0.00903168492745032\\
11.3126269444444	0.00873556411015675\\
11.3320022222222	0.00903168492745032\\
11.3483225	0.00917974533609689\\
11.3697255555556	0.00903168492745032\\
11.4184836111111	0.00917974533609689\\
11.4332438888889	0.00932780574474368\\
11.4529455555556	0.00843944329286339\\
11.4572761111111	0.00755108084098288\\
11.4834966666667	0.00814332247557004\\
11.4896191666667	0.00784720165827646\\
11.507555	0.00755108084098288\\
11.5398627777778	0.00755108084098288\\
11.5581236111111	0.00769914124962989\\
11.5624258333333	0.00769914124962989\\
11.5875011111111	0.00784720165827646\\
11.6045205555556	0.00755108084098288\\
11.6202886111111	0.00725496002368953\\
11.6376652777778	0.00814332247557004\\
11.6450027777778	0.0082913828842166\\
11.6699430555556	0.00843944329286339\\
11.6889369444444	0.00799526206692325\\
11.7334233333333	0.00784720165827646\\
11.7474783333333	0.00755108084098288\\
11.7712441666667	0.00725496002368953\\
11.7946125	0.00710689961504274\\
11.8165530555556	0.00695883920639617\\
11.8568216666667	0.00710689961504274\\
11.8634066666667	0.00784720165827646\\
11.8742052777778	0.00755108084098288\\
11.9017461111111	0.00755108084098288\\
11.9382172222222	0.00725496002368953\\
11.9668552777778	0.00784720165827646\\
11.9831680555556	0.00814332247557004\\
12.0152302777778	0.00755108084098288\\
12.0410402777778	0.00784720165827646\\
12.0659138888889	0.00740302043233632\\
12.0802863888889	0.00755108084098288\\
12.1096377777778	0.00784720165827646\\
12.1141016666667	0.00725496002368953\\
12.1285986111111	0.00695883920639617\\
12.1501658333333	0.00784720165827646\\
12.1707508333333	0.00755108084098288\\
12.2163308333333	0.00725496002368953\\
12.2483086111111	0.00784720165827646\\
12.2638022222222	0.00725496002368953\\
12.3048261111111	0.00695883920639617\\
12.3189922222222	0.00725496002368953\\
12.3421622222222	0.00725496002368953\\
12.4179988888889	0.00695883920639617\\
12.4183472222222	0.00636659757180924\\
12.4358361111111	0.00636659757180924\\
12.4645227777778	0.00695883920639617\\
12.4886102777778	0.00695883920639617\\
12.5043041666667	0.00725496002368953\\
12.5381361111111	0.00725496002368953\\
12.5463455555556	0.00725496002368953\\
12.5686047222222	0.00695883920639617\\
12.6062816666667	0.00710689961504274\\
12.6082305555556	0.00784720165827646\\
12.6273311111111	0.00755108084098288\\
12.6641488888889	0.00784720165827646\\
12.6708919444444	0.00814332247557004\\
12.6928519444444	0.00799526206692325\\
12.7142405555556	0.00725496002368953\\
12.7739005555556	0.00740302043233632\\
12.8030041666667	0.00695883920639617\\
12.8443927777778	0.00725496002368953\\
12.8551388888889	0.00725496002368953\\
12.8969294444444	0.00681077879774938\\
12.9146408333333	0.00725496002368953\\
12.9491175	0.0066627183891026\\
12.9549816666667	0.0059224163458691\\
12.9668175	0.0066627183891026\\
12.9781347222222	0.00636659757180924\\
13.0114336111111	0.00636659757180924\\
13.0486619444444	0.00607047675451589\\
13.0732966666667	0.00621853716316245\\
13.1032852777778	0.00636659757180924\\
13.116925	0.00562629552857552\\
13.1348447222222	0.00562629552857552\\
13.1584116666667	0.00533017471128217\\
13.1660561111111	0.00458987266804844\\
13.2154888888889	0.00444181225940188\\
13.2304272222222	0.00370151021616816\\
13.2503633333333	0.00370151021616816\\
13.2549833333333	0.00281314776428787\\
13.2839944444444	0.00222090612970072\\
13.2933194444444	0.00192478531240736\\
13.3080980555556	0.00148060408646722\\
13.3115980555556	0.00103642286052708\\
13.3331252777778	0.00118448326917386\\
13.3376194444444	0.00162866449511401\\
13.3424838888889	0.00192478531240736\\
13.3746313888889	0.00133254367782043\\
13.398155	0.000740302043233498\\
13.4147963888889	0.000444181225939921\\
13.4325194444444	0.00103642286052708\\
13.462665	0.00148060408646722\\
13.4690611111111	0.00177672490376057\\
13.4701041666667	0.00118448326917386\\
13.4955783333333	0.00148060408646722\\
13.4979116666667	0.00162866449511401\\
13.5115202777778	0.00222090612970072\\
13.5307294444444	0.00281314776428787\\
13.5484705555556	0.00340538939887458\\
13.5777555555556	0.00384957062481472\\
13.6270827777778	0.00399763103346151\\
13.6723563888889	0.00370151021616816\\
13.6992080555556	0.00399763103346151\\
13.7310766666667	0.00384957062481472\\
13.7689858333333	0.00340538939887458\\
13.7937802777778	0.00429375185075509\\
13.8282872222222	0.00370151021616816\\
13.8549211111111	0.00384957062481472\\
13.8717405555556	0.00384957062481472\\
13.9166705555556	0.00399763103346151\\
13.9217688888889	0.00429375185075509\\
13.9476483333333	0.00384957062481472\\
13.9660916666667	0.00340538939887458\\
13.9812091666667	0.00399763103346151\\
13.9997763888889	0.00325732899022801\\
14.0106	0.00370151021616816\\
14.0627572222222	0.00340538939887458\\
14.1176233333333	0.00310926858158123\\
14.1365577777778	0.00296120817293444\\
14.17165	0.00340538939887458\\
14.2231411111111	0.00370151021616816\\
14.2760219444444	0.00340538939887458\\
14.2885908333333	0.00281314776428787\\
14.3411905555556	0.00281314776428787\\
14.368325	0.00192478531240736\\
14.3840313888889	0.00236896653834751\\
14.4053222222222	0.00236896653834751\\
14.4225655555556	0.00192478531240736\\
14.4367030555556	0.00222090612970072\\
14.4520211111111	0.00236896653834751\\
14.4533102777778	0.00236896653834751\\
14.4578727777778	0.00236896653834751\\
14.4599613888889	0.00281314776428787\\
14.5024269444444	0.00281314776428787\\
14.507665	0.00325732899022801\\
14.5165019444444	0.00355344980752137\\
14.5168766666667	0.00355344980752137\\
14.5488716666667	0.00384957062481472\\
14.5726927777778	0.00414569144210808\\
14.5799011111111	0.00399763103346151\\
14.6048725	0.00340538939887458\\
14.6189983333333	0.00340538939887458\\
14.66641	0.00340538939887458\\
14.6784186111111	0.00399763103346151\\
14.6920227777778	0.00384957062481472\\
14.7191602777778	0.00310926858158123\\
14.7349805555556	0.00281314776428787\\
14.7480891666667	0.00266508735564108\\
14.7585238888889	0.00236896653834751\\
14.7803658333333	0.00266508735564108\\
14.8047269444444	0.00370151021616816\\
14.8113877777778	0.00340538939887458\\
14.8155066666667	0.00384957062481472\\
14.8235269444444	0.00384957062481472\\
14.8320288888889	0.00296120817293444\\
14.8568383333333	0.00310926858158123\\
14.8972325	0.00355344980752137\\
14.9078869444444	0.00370151021616816\\
14.9220372222222	0.00384957062481472\\
14.9338605555556	0.00399763103346151\\
14.9539852777778	0.00310926858158123\\
14.9585955555556	0.00370151021616816\\
14.9717413888889	0.00340538939887458\\
14.9892997222222	0.00340538939887458\\
14.9914952777778	0.00384957062481472\\
15.0314038888889	0.00370151021616816\\
15.0412358333333	0.00340538939887458\\
15.0798611111111	0.00340538939887458\\
15.0870244444444	0.00370151021616816\\
15.0985155555556	0.00414569144210808\\
15.1000338888889	0.00399763103346151\\
15.1191172222222	0.00370151021616816\\
15.1300683333333	0.00429375185075509\\
15.1524230555556	0.00429375185075509\\
15.1694680555556	0.00399763103346151\\
15.1745933333333	0.00399763103346151\\
15.1849508333333	0.00384957062481472\\
15.1979472222222	0.00340538939887458\\
15.2080444444444	0.00429375185075509\\
15.2189502777778	0.00429375185075509\\
15.2220713888889	0.00399763103346151\\
15.2415544444444	0.00355344980752137\\
15.2552663888889	0.00355344980752137\\
15.2646333333333	0.00414569144210808\\
15.2699113888889	0.00340538939887458\\
15.2731491666667	0.00281314776428787\\
15.2786141666667	0.00236896653834751\\
15.3051338888889	0.00236896653834751\\
15.316735	0.00192478531240736\\
15.3242872222222	0.00251702694699429\\
15.3341147222222	0.00222090612970072\\
15.3505097222222	0.00266508735564108\\
15.3694041666667	0.00192478531240736\\
15.3756863888889	0.00251702694699429\\
15.3963508333333	0.00222090612970072\\
15.4294527777778	0.00266508735564108\\
15.4454355555556	0.00340538939887458\\
15.4588930555556	0.00399763103346151\\
15.4733547222222	0.00370151021616816\\
15.5135641666667	0.00429375185075509\\
15.5217897222222	0.00429375185075509\\
15.5437788888889	0.00429375185075509\\
15.5500508333333	0.00355344980752137\\
15.5597272222222	0.00355344980752137\\
15.5888180555556	0.00399763103346151\\
15.6159077777778	0.00458987266804844\\
15.6229730555556	0.00458987266804844\\
15.6363211111111	0.00473793307669523\\
15.6427986111111	0.00488599348534202\\
15.6605063888889	0.00518211430263538\\
15.6657394444444	0.00518211430263538\\
15.6722908333333	0.00488599348534202\\
15.6980930555556	0.00503405389398859\\
15.7221111111111	0.00577435593722231\\
15.7326002777778	0.00562629552857552\\
15.7433066666667	0.00562629552857552\\
15.7480433333333	0.00488599348534202\\
15.7532744444444	0.00458987266804844\\
15.7581419444444	0.00458987266804844\\
15.7654947222222	0.00473793307669523\\
15.7747011111111	0.00518211430263538\\
15.7921936111111	0.00577435593722231\\
15.7990469444444	0.00681077879774938\\
15.8197083333333	0.00636659757180924\\
15.8295508333333	0.00577435593722231\\
15.8390375	0.00607047675451589\\
15.86611	0.00547823511992873\\
15.8901183333333	0.00518211430263538\\
15.9027716666667	0.00547823511992873\\
15.9148186111111	0.00533017471128217\\
15.9162177777778	0.00562629552857552\\
15.9256441666667	0.00607047675451589\\
15.9288980555556	0.00695883920639617\\
15.9360080555556	0.00636659757180924\\
15.9495427777778	0.00636659757180924\\
15.9634108333333	0.0066627183891026\\
15.9772219444444	0.0066627183891026\\
15.9899863888889	0.00681077879774938\\
15.9945927777778	0.00636659757180924\\
15.9986611111111	0.00607047675451589\\
};
\addlegendentry{Midprice};

\end{axis}
\end{tikzpicture}%

  \caption{Performance comparison of the four stochastic control methods.}
  \label{fig:ORCL_comp4stoch}
\end{subfigure}\\
\vspace{1cm}
\begin{subfigure}{\linewidth}
  \centering
  \setlength\figureheight{0.5\linewidth} 
  \setlength\figurewidth{\linewidth}
  \tikzsetnextfilename{ORCL_comp4stoch_inv}
  % This file was created by matlab2tikz.
%
%The latest updates can be retrieved from
%  http://www.mathworks.com/matlabcentral/fileexchange/22022-matlab2tikz-matlab2tikz
%where you can also make suggestions and rate matlab2tikz.
%
\definecolor{mycolor1}{rgb}{0.25098,0.00000,0.38824}%
\definecolor{mycolor2}{rgb}{0.00000,0.11765,0.38824}%
\definecolor{mycolor3}{rgb}{0.58039,0.26275,0.00000}%
\definecolor{mycolor4}{rgb}{0.58039,0.38824,0.00000}%
%
\begin{tikzpicture}[trim axis left, trim axis right]

\begin{axis}[%
width=\figurewidth,
height=\figureheight,
at={(0\figurewidth,0\figureheight)},
scale only axis,
every outer x axis line/.append style={black},
every x tick label/.append style={font=\color{black}},
xmin=9.5,
xmax=16,
xlabel={Time (h)},
every outer y axis line/.append style={black},
every y tick label/.append style={font=\color{black}},
ymin=-20,
ymax=20,
ylabel={Inventory},
title={Strategy Inventory using Optimal Stochastic Control},
axis background/.style={fill=white},
axis x line*=bottom,
axis y line*=left,
yticklabel style={
        /pgf/number format/fixed,
        /pgf/number format/precision=3
},
scaled y ticks=false,
legend style={legend cell align=left,align=left,draw=black,font=\small, legend pos=north west},
]
\addplot [color=mycolor1,solid,line width=1.5pt]
  table[row sep=crcr]{%
9.50027777777778	2\\
9.50583333333333	4\\
9.51138888888889	9\\
9.51694444444444	1\\
9.5225	-2\\
9.52805555555556	-1\\
9.53361111111111	-4\\
9.53916666666667	-2\\
9.54472222222222	-5\\
9.55027777777778	-3\\
9.55583333333333	-4\\
9.56138888888889	2\\
9.56694444444444	3\\
9.5725	7\\
9.57805555555555	7\\
9.58361111111111	2\\
9.58916666666667	8\\
9.59472222222222	4\\
9.60027777777778	6\\
9.60583333333333	2\\
9.61138888888889	2\\
9.61694444444444	2\\
9.6225	3\\
9.62805555555556	5\\
9.63361111111111	2\\
9.63916666666667	4\\
9.64472222222222	3\\
9.65027777777778	4\\
9.65583333333333	5\\
9.66138888888889	1\\
9.66694444444444	1\\
9.6725	4\\
9.67805555555555	2\\
9.68361111111111	4\\
9.68916666666667	3\\
9.69472222222222	2\\
9.70027777777778	2\\
9.70583333333333	3\\
9.71138888888889	3\\
9.71694444444444	1\\
9.7225	3\\
9.72805555555555	4\\
9.73361111111111	2\\
9.73916666666667	2\\
9.74472222222222	2\\
9.75027777777778	3\\
9.75583333333333	4\\
9.76138888888889	3\\
9.76694444444444	2\\
9.7725	-3\\
9.77805555555556	-4\\
9.78361111111111	-5\\
9.78916666666667	-2\\
9.79472222222222	-3\\
9.80027777777778	-2\\
9.80583333333333	-2\\
9.81138888888889	-5\\
9.81694444444444	-7\\
9.8225	-10\\
9.82805555555555	-1\\
9.83361111111111	2\\
9.83916666666667	3\\
9.84472222222222	4\\
9.85027777777778	2\\
9.85583333333333	4\\
9.86138888888889	4\\
9.86694444444444	3\\
9.8725	4\\
9.87805555555556	3\\
9.88361111111111	4\\
9.88916666666667	4\\
9.89472222222222	5\\
9.90027777777778	5\\
9.90583333333333	4\\
9.91138888888889	4\\
9.91694444444444	4\\
9.9225	4\\
9.92805555555555	4\\
9.93361111111111	4\\
9.93916666666667	5\\
9.94472222222222	6\\
9.95027777777778	5\\
9.95583333333333	7\\
9.96138888888889	9\\
9.96694444444444	3\\
9.9725	2\\
9.97805555555555	3\\
9.98361111111111	3\\
9.98916666666667	5\\
9.99472222222222	5\\
10.0002777777778	5\\
10.0058333333333	5\\
10.0113888888889	5\\
10.0169444444444	3\\
10.0225	2\\
10.0280555555556	2\\
10.0336111111111	2\\
10.0391666666667	2\\
10.0447222222222	2\\
10.0502777777778	2\\
10.0558333333333	3\\
10.0613888888889	3\\
10.0669444444444	5\\
10.0725	5\\
10.0780555555556	12\\
10.0836111111111	12\\
10.0891666666667	12\\
10.0947222222222	12\\
10.1002777777778	3\\
10.1058333333333	2\\
10.1113888888889	2\\
10.1169444444444	4\\
10.1225	4\\
10.1280555555556	2\\
10.1336111111111	2\\
10.1391666666667	2\\
10.1447222222222	4\\
10.1502777777778	4\\
10.1558333333333	4\\
10.1613888888889	4\\
10.1669444444444	6\\
10.1725	3\\
10.1780555555556	3\\
10.1836111111111	5\\
10.1891666666667	3\\
10.1947222222222	3\\
10.2002777777778	3\\
10.2058333333333	3\\
10.2113888888889	3\\
10.2169444444444	3\\
10.2225	4\\
10.2280555555556	5\\
10.2336111111111	3\\
10.2391666666667	3\\
10.2447222222222	4\\
10.2502777777778	3\\
10.2558333333333	3\\
10.2613888888889	5\\
10.2669444444444	4\\
10.2725	6\\
10.2780555555556	4\\
10.2836111111111	3\\
10.2891666666667	2\\
10.2947222222222	2\\
10.3002777777778	3\\
10.3058333333333	3\\
10.3113888888889	2\\
10.3169444444444	4\\
10.3225	3\\
10.3280555555556	3\\
10.3336111111111	2\\
10.3391666666667	2\\
10.3447222222222	1\\
10.3502777777778	2\\
10.3558333333333	4\\
10.3613888888889	4\\
10.3669444444444	4\\
10.3725	4\\
10.3780555555556	4\\
10.3836111111111	4\\
10.3891666666667	4\\
10.3947222222222	4\\
10.4002777777778	3\\
10.4058333333333	3\\
10.4113888888889	3\\
10.4169444444444	3\\
10.4225	3\\
10.4280555555556	4\\
10.4336111111111	5\\
10.4391666666667	5\\
10.4447222222222	4\\
10.4502777777778	4\\
10.4558333333333	4\\
10.4613888888889	4\\
10.4669444444444	4\\
10.4725	3\\
10.4780555555556	2\\
10.4836111111111	2\\
10.4891666666667	4\\
10.4947222222222	4\\
10.5002777777778	4\\
10.5058333333333	2\\
10.5113888888889	3\\
10.5169444444444	3\\
10.5225	3\\
10.5280555555556	2\\
10.5336111111111	1\\
10.5391666666667	1\\
10.5447222222222	1\\
10.5502777777778	2\\
10.5558333333333	2\\
10.5613888888889	2\\
10.5669444444444	1\\
10.5725	2\\
10.5780555555556	2\\
10.5836111111111	2\\
10.5891666666667	2\\
10.5947222222222	2\\
10.6002777777778	3\\
10.6058333333333	3\\
10.6113888888889	4\\
10.6169444444444	4\\
10.6225	4\\
10.6280555555556	3\\
10.6336111111111	3\\
10.6391666666667	4\\
10.6447222222222	4\\
10.6502777777778	3\\
10.6558333333333	3\\
10.6613888888889	2\\
10.6669444444444	1\\
10.6725	3\\
10.6780555555556	3\\
10.6836111111111	2\\
10.6891666666667	2\\
10.6947222222222	2\\
10.7002777777778	3\\
10.7058333333333	2\\
10.7113888888889	1\\
10.7169444444444	1\\
10.7225	1\\
10.7280555555556	2\\
10.7336111111111	3\\
10.7391666666667	3\\
10.7447222222222	4\\
10.7502777777778	5\\
10.7558333333333	5\\
10.7613888888889	3\\
10.7669444444444	3\\
10.7725	2\\
10.7780555555556	2\\
10.7836111111111	2\\
10.7891666666667	2\\
10.7947222222222	2\\
10.8002777777778	2\\
10.8058333333333	2\\
10.8113888888889	2\\
10.8169444444444	3\\
10.8225	3\\
10.8280555555556	3\\
10.8336111111111	3\\
10.8391666666667	6\\
10.8447222222222	3\\
10.8502777777778	3\\
10.8558333333333	4\\
10.8613888888889	3\\
10.8669444444444	2\\
10.8725	2\\
10.8780555555556	2\\
10.8836111111111	2\\
10.8891666666667	2\\
10.8947222222222	2\\
10.9002777777778	2\\
10.9058333333333	2\\
10.9113888888889	3\\
10.9169444444444	3\\
10.9225	4\\
10.9280555555556	4\\
10.9336111111111	7\\
10.9391666666667	2\\
10.9447222222222	6\\
10.9502777777778	5\\
10.9558333333333	5\\
10.9613888888889	5\\
10.9669444444444	8\\
10.9725	11\\
10.9780555555556	6\\
10.9836111111111	5\\
10.9891666666667	8\\
10.9947222222222	4\\
11.0002777777778	3\\
11.0058333333333	3\\
11.0113888888889	3\\
11.0169444444444	2\\
11.0225	2\\
11.0280555555556	2\\
11.0336111111111	2\\
11.0391666666667	2\\
11.0447222222222	3\\
11.0502777777778	4\\
11.0558333333333	3\\
11.0613888888889	3\\
11.0669444444444	3\\
11.0725	4\\
11.0780555555556	4\\
11.0836111111111	4\\
11.0891666666667	4\\
11.0947222222222	4\\
11.1002777777778	4\\
11.1058333333333	2\\
11.1113888888889	2\\
11.1169444444444	2\\
11.1225	2\\
11.1280555555556	2\\
11.1336111111111	2\\
11.1391666666667	2\\
11.1447222222222	2\\
11.1502777777778	2\\
11.1558333333333	3\\
11.1613888888889	3\\
11.1669444444444	3\\
11.1725	3\\
11.1780555555556	3\\
11.1836111111111	2\\
11.1891666666667	3\\
11.1947222222222	3\\
11.2002777777778	2\\
11.2058333333333	3\\
11.2113888888889	3\\
11.2169444444444	2\\
11.2225	2\\
11.2280555555556	2\\
11.2336111111111	2\\
11.2391666666667	2\\
11.2447222222222	2\\
11.2502777777778	4\\
11.2558333333333	4\\
11.2613888888889	4\\
11.2669444444444	4\\
11.2725	4\\
11.2780555555556	3\\
11.2836111111111	3\\
11.2891666666667	3\\
11.2947222222222	3\\
11.3002777777778	2\\
11.3058333333333	2\\
11.3113888888889	2\\
11.3169444444444	2\\
11.3225	2\\
11.3280555555556	2\\
11.3336111111111	1\\
11.3391666666667	1\\
11.3447222222222	2\\
11.3502777777778	3\\
11.3558333333333	4\\
11.3613888888889	4\\
11.3669444444444	3\\
11.3725	2\\
11.3780555555556	2\\
11.3836111111111	2\\
11.3891666666667	2\\
11.3947222222222	3\\
11.4002777777778	3\\
11.4058333333333	6\\
11.4113888888889	6\\
11.4169444444444	6\\
11.4225	3\\
11.4280555555556	3\\
11.4336111111111	1\\
11.4391666666667	1\\
11.4447222222222	2\\
11.4502777777778	2\\
11.4558333333333	4\\
11.4613888888889	5\\
11.4669444444444	5\\
11.4725	5\\
11.4780555555556	5\\
11.4836111111111	3\\
11.4891666666667	3\\
11.4947222222222	7\\
11.5002777777778	7\\
11.5058333333333	7\\
11.5113888888889	2\\
11.5169444444444	4\\
11.5225	4\\
11.5280555555556	4\\
11.5336111111111	4\\
11.5391666666667	4\\
11.5447222222222	3\\
11.5502777777778	4\\
11.5558333333333	4\\
11.5613888888889	2\\
11.5669444444444	2\\
11.5725	2\\
11.5780555555556	2\\
11.5836111111111	2\\
11.5891666666667	3\\
11.5947222222222	3\\
11.6002777777778	4\\
11.6058333333333	4\\
11.6113888888889	4\\
11.6169444444444	4\\
11.6225	6\\
11.6280555555556	4\\
11.6336111111111	3\\
11.6391666666667	3\\
11.6447222222222	3\\
11.6502777777778	4\\
11.6558333333333	4\\
11.6613888888889	4\\
11.6669444444444	4\\
11.6725	6\\
11.6780555555556	7\\
11.6836111111111	7\\
11.6891666666667	4\\
11.6947222222222	4\\
11.7002777777778	4\\
11.7058333333333	4\\
11.7113888888889	4\\
11.7169444444444	4\\
11.7225	4\\
11.7280555555556	4\\
11.7336111111111	5\\
11.7391666666667	6\\
11.7447222222222	7\\
11.7502777777778	9\\
11.7558333333333	9\\
11.7613888888889	10\\
11.7669444444444	11\\
11.7725	13\\
11.7780555555556	14\\
11.7836111111111	15\\
11.7891666666667	17\\
11.7947222222222	4\\
11.8002777777778	4\\
11.8058333333333	4\\
11.8113888888889	4\\
11.8169444444444	4\\
11.8225	4\\
11.8280555555556	4\\
11.8336111111111	4\\
11.8391666666667	5\\
11.8447222222222	7\\
11.8502777777778	3\\
11.8558333333333	3\\
11.8613888888889	2\\
11.8669444444444	3\\
11.8725	4\\
11.8780555555556	5\\
11.8836111111111	6\\
11.8891666666667	8\\
11.8947222222222	4\\
11.9002777777778	4\\
11.9058333333333	4\\
11.9113888888889	5\\
11.9169444444444	5\\
11.9225	5\\
11.9280555555556	6\\
11.9336111111111	6\\
11.9391666666667	8\\
11.9447222222222	8\\
11.9502777777778	8\\
11.9558333333333	8\\
11.9613888888889	4\\
11.9669444444444	4\\
11.9725	4\\
11.9780555555556	5\\
11.9836111111111	4\\
11.9891666666667	4\\
11.9947222222222	4\\
12.0002777777778	4\\
12.0058333333333	7\\
12.0113888888889	8\\
12.0169444444444	9\\
12.0225	9\\
12.0280555555556	9\\
12.0336111111111	9\\
12.0391666666667	4\\
12.0447222222222	3\\
12.0502777777778	3\\
12.0558333333333	4\\
12.0613888888889	6\\
12.0669444444444	9\\
12.0725	5\\
12.0780555555556	8\\
12.0836111111111	4\\
12.0891666666667	3\\
12.0947222222222	4\\
12.1002777777778	4\\
12.1058333333333	3\\
12.1113888888889	3\\
12.1169444444444	4\\
12.1225	5\\
12.1280555555556	5\\
12.1336111111111	5\\
12.1391666666667	4\\
12.1447222222222	3\\
12.1502777777778	3\\
12.1558333333333	3\\
12.1613888888889	3\\
12.1669444444444	3\\
12.1725	3\\
12.1780555555556	3\\
12.1836111111111	3\\
12.1891666666667	2\\
12.1947222222222	3\\
12.2002777777778	3\\
12.2058333333333	3\\
12.2113888888889	3\\
12.2169444444444	3\\
12.2225	3\\
12.2280555555556	4\\
12.2336111111111	4\\
12.2391666666667	4\\
12.2447222222222	4\\
12.2502777777778	3\\
12.2558333333333	5\\
12.2613888888889	5\\
12.2669444444444	7\\
12.2725	7\\
12.2780555555556	7\\
12.2836111111111	8\\
12.2891666666667	8\\
12.2947222222222	8\\
12.3002777777778	8\\
12.3058333333333	4\\
12.3113888888889	4\\
12.3169444444444	4\\
12.3225	4\\
12.3280555555556	4\\
12.3336111111111	3\\
12.3391666666667	4\\
12.3447222222222	4\\
12.3502777777778	5\\
12.3558333333333	5\\
12.3613888888889	5\\
12.3669444444444	4\\
12.3725	4\\
12.3780555555556	4\\
12.3836111111111	4\\
12.3891666666667	5\\
12.3947222222222	5\\
12.4002777777778	5\\
12.4058333333333	5\\
12.4113888888889	5\\
12.4169444444444	5\\
12.4225	11\\
12.4280555555556	11\\
12.4336111111111	12\\
12.4391666666667	4\\
12.4447222222222	2\\
12.4502777777778	2\\
12.4558333333333	2\\
12.4613888888889	2\\
12.4669444444444	3\\
12.4725	2\\
12.4780555555556	2\\
12.4836111111111	2\\
12.4891666666667	3\\
12.4947222222222	3\\
12.5002777777778	5\\
12.5058333333333	4\\
12.5113888888889	4\\
12.5169444444444	4\\
12.5225	3\\
12.5280555555556	2\\
12.5336111111111	2\\
12.5391666666667	2\\
12.5447222222222	2\\
12.5502777777778	2\\
12.5558333333333	3\\
12.5613888888889	3\\
12.5669444444444	3\\
12.5725	3\\
12.5780555555556	3\\
12.5836111111111	2\\
12.5891666666667	2\\
12.5947222222222	3\\
12.6002777777778	3\\
12.6058333333333	3\\
12.6113888888889	4\\
12.6169444444444	5\\
12.6225	6\\
12.6280555555556	4\\
12.6336111111111	4\\
12.6391666666667	4\\
12.6447222222222	4\\
12.6502777777778	5\\
12.6558333333333	5\\
12.6613888888889	5\\
12.6669444444444	4\\
12.6725	3\\
12.6780555555556	3\\
12.6836111111111	2\\
12.6891666666667	2\\
12.6947222222222	3\\
12.7002777777778	4\\
12.7058333333333	4\\
12.7113888888889	4\\
12.7169444444444	4\\
12.7225	4\\
12.7280555555556	4\\
12.7336111111111	4\\
12.7391666666667	4\\
12.7447222222222	3\\
12.7502777777778	3\\
12.7558333333333	3\\
12.7613888888889	3\\
12.7669444444444	3\\
12.7725	3\\
12.7780555555556	3\\
12.7836111111111	3\\
12.7891666666667	3\\
12.7947222222222	2\\
12.8002777777778	2\\
12.8058333333333	2\\
12.8113888888889	2\\
12.8169444444444	2\\
12.8225	3\\
12.8280555555556	3\\
12.8336111111111	3\\
12.8391666666667	3\\
12.8447222222222	2\\
12.8502777777778	2\\
12.8558333333333	2\\
12.8613888888889	2\\
12.8669444444444	2\\
12.8725	2\\
12.8780555555556	2\\
12.8836111111111	4\\
12.8891666666667	4\\
12.8947222222222	5\\
12.9002777777778	5\\
12.9058333333333	5\\
12.9113888888889	4\\
12.9169444444444	3\\
12.9225	3\\
12.9280555555556	3\\
12.9336111111111	4\\
12.9391666666667	5\\
12.9447222222222	5\\
12.9502777777778	5\\
12.9558333333333	6\\
12.9613888888889	3\\
12.9669444444444	3\\
12.9725	4\\
12.9780555555556	5\\
12.9836111111111	4\\
12.9891666666667	4\\
12.9947222222222	4\\
13.0002777777778	5\\
13.0058333333333	5\\
13.0113888888889	5\\
13.0169444444444	5\\
13.0225	5\\
13.0280555555556	6\\
13.0336111111111	6\\
13.0391666666667	6\\
13.0447222222222	6\\
13.0502777777778	4\\
13.0558333333333	4\\
13.0613888888889	5\\
13.0669444444444	5\\
13.0725	5\\
13.0780555555556	3\\
13.0836111111111	3\\
13.0891666666667	3\\
13.0947222222222	3\\
13.1002777777778	3\\
13.1058333333333	6\\
13.1113888888889	6\\
13.1169444444444	10\\
13.1225	3\\
13.1280555555556	3\\
13.1336111111111	3\\
13.1391666666667	4\\
13.1447222222222	4\\
13.1502777777778	6\\
13.1558333333333	6\\
13.1613888888889	6\\
13.1669444444444	6\\
13.1725	6\\
13.1780555555556	6\\
13.1836111111111	6\\
13.1891666666667	6\\
13.1947222222222	6\\
13.2002777777778	6\\
13.2058333333333	6\\
13.2113888888889	6\\
13.2169444444444	7\\
13.2225	7\\
13.2280555555556	7\\
13.2336111111111	7\\
13.2391666666667	8\\
13.2447222222222	11\\
13.2502777777778	13\\
13.2558333333333	15\\
13.2613888888889	16\\
13.2669444444444	16\\
13.2725	16\\
13.2780555555556	17\\
13.2836111111111	20\\
13.2891666666667	20\\
13.2947222222222	20\\
13.3002777777778	20\\
13.3058333333333	19\\
13.3113888888889	20\\
13.3169444444444	3\\
13.3225	2\\
13.3280555555556	4\\
13.3336111111111	3\\
13.3391666666667	4\\
13.3447222222222	3\\
13.3502777777778	4\\
13.3558333333333	4\\
13.3613888888889	4\\
13.3669444444444	4\\
13.3725	4\\
13.3780555555556	6\\
13.3836111111111	8\\
13.3891666666667	8\\
13.3947222222222	8\\
13.4002777777778	12\\
13.4058333333333	15\\
13.4113888888889	18\\
13.4169444444444	4\\
13.4225	4\\
13.4280555555556	4\\
13.4336111111111	3\\
13.4391666666667	3\\
13.4447222222222	3\\
13.4502777777778	3\\
13.4558333333333	5\\
13.4613888888889	5\\
13.4669444444444	4\\
13.4725	4\\
13.4780555555556	4\\
13.4836111111111	4\\
13.4891666666667	5\\
13.4947222222222	4\\
13.5002777777778	5\\
13.5058333333333	6\\
13.5113888888889	2\\
13.5169444444444	3\\
13.5225	2\\
13.5280555555556	2\\
13.5336111111111	2\\
13.5391666666667	2\\
13.5447222222222	2\\
13.5502777777778	2\\
13.5558333333333	1\\
13.5613888888889	1\\
13.5669444444444	2\\
13.5725	4\\
13.5780555555556	3\\
13.5836111111111	3\\
13.5891666666667	3\\
13.5947222222222	3\\
13.6002777777778	2\\
13.6058333333333	2\\
13.6113888888889	2\\
13.6169444444444	2\\
13.6225	2\\
13.6280555555556	2\\
13.6336111111111	2\\
13.6391666666667	4\\
13.6447222222222	4\\
13.6502777777778	4\\
13.6558333333333	4\\
13.6613888888889	4\\
13.6669444444444	4\\
13.6725	4\\
13.6780555555556	4\\
13.6836111111111	4\\
13.6891666666667	3\\
13.6947222222222	2\\
13.7002777777778	2\\
13.7058333333333	2\\
13.7113888888889	2\\
13.7169444444444	3\\
13.7225	5\\
13.7280555555556	6\\
13.7336111111111	4\\
13.7391666666667	4\\
13.7447222222222	5\\
13.7502777777778	7\\
13.7558333333333	8\\
13.7613888888889	8\\
13.7669444444444	8\\
13.7725	8\\
13.7780555555556	7\\
13.7836111111111	7\\
13.7891666666667	4\\
13.7947222222222	2\\
13.8002777777778	3\\
13.8058333333333	4\\
13.8113888888889	4\\
13.8169444444444	4\\
13.8225	4\\
13.8280555555556	4\\
13.8336111111111	5\\
13.8391666666667	5\\
13.8447222222222	5\\
13.8502777777778	5\\
13.8558333333333	4\\
13.8613888888889	4\\
13.8669444444444	3\\
13.8725	2\\
13.8780555555556	2\\
13.8836111111111	2\\
13.8891666666667	3\\
13.8947222222222	3\\
13.9002777777778	3\\
13.9058333333333	3\\
13.9113888888889	3\\
13.9169444444444	2\\
13.9225	3\\
13.9280555555556	4\\
13.9336111111111	4\\
13.9391666666667	5\\
13.9447222222222	5\\
13.9502777777778	4\\
13.9558333333333	4\\
13.9613888888889	4\\
13.9669444444444	6\\
13.9725	3\\
13.9780555555556	3\\
13.9836111111111	4\\
13.9891666666667	4\\
13.9947222222222	5\\
14.0002777777778	8\\
14.0058333333333	8\\
14.0113888888889	3\\
14.0169444444444	4\\
14.0225	7\\
14.0280555555556	8\\
14.0336111111111	11\\
14.0391666666667	11\\
14.0447222222222	12\\
14.0502777777778	14\\
14.0558333333333	16\\
14.0613888888889	17\\
14.0669444444444	19\\
14.0725	20\\
14.0780555555556	20\\
14.0836111111111	20\\
14.0891666666667	20\\
14.0947222222222	20\\
14.1002777777778	20\\
14.1058333333333	20\\
14.1113888888889	20\\
14.1169444444444	20\\
14.1225	20\\
14.1280555555556	20\\
14.1336111111111	19\\
14.1391666666667	2\\
14.1447222222222	5\\
14.1502777777778	5\\
14.1558333333333	9\\
14.1613888888889	11\\
14.1669444444444	12\\
14.1725	5\\
14.1780555555556	6\\
14.1836111111111	7\\
14.1891666666667	7\\
14.1947222222222	8\\
14.2002777777778	8\\
14.2058333333333	8\\
14.2113888888889	8\\
14.2169444444444	8\\
14.2225	3\\
14.2280555555556	4\\
14.2336111111111	4\\
14.2391666666667	4\\
14.2447222222222	6\\
14.2502777777778	7\\
14.2558333333333	7\\
14.2613888888889	7\\
14.2669444444444	9\\
14.2725	11\\
14.2780555555556	13\\
14.2836111111111	14\\
14.2891666666667	16\\
14.2947222222222	17\\
14.3002777777778	19\\
14.3058333333333	20\\
14.3113888888889	20\\
14.3169444444444	20\\
14.3225	20\\
14.3280555555556	20\\
14.3336111111111	20\\
14.3391666666667	20\\
14.3447222222222	20\\
14.3502777777778	20\\
14.3558333333333	20\\
14.3613888888889	20\\
14.3669444444444	20\\
14.3725	20\\
14.3780555555556	3\\
14.3836111111111	3\\
14.3891666666667	2\\
14.3947222222222	2\\
14.4002777777778	3\\
14.4058333333333	4\\
14.4113888888889	4\\
14.4169444444444	5\\
14.4225	5\\
14.4280555555556	4\\
14.4336111111111	3\\
14.4391666666667	2\\
14.4447222222222	3\\
14.4502777777778	3\\
14.4558333333333	2\\
14.4613888888889	2\\
14.4669444444444	2\\
14.4725	2\\
14.4780555555556	2\\
14.4836111111111	2\\
14.4891666666667	2\\
14.4947222222222	2\\
14.5002777777778	2\\
14.5058333333333	2\\
14.5113888888889	2\\
14.5169444444444	2\\
14.5225	2\\
14.5280555555556	2\\
14.5336111111111	2\\
14.5391666666667	2\\
14.5447222222222	2\\
14.5502777777778	2\\
14.5558333333333	2\\
14.5613888888889	2\\
14.5669444444444	2\\
14.5725	2\\
14.5780555555556	5\\
14.5836111111111	4\\
14.5891666666667	4\\
14.5947222222222	5\\
14.6002777777778	5\\
14.6058333333333	7\\
14.6113888888889	7\\
14.6169444444444	7\\
14.6225	4\\
14.6280555555556	4\\
14.6336111111111	4\\
14.6391666666667	4\\
14.6447222222222	4\\
14.6502777777778	4\\
14.6558333333333	3\\
14.6613888888889	3\\
14.6669444444444	3\\
14.6725	3\\
14.6780555555556	4\\
14.6836111111111	2\\
14.6891666666667	2\\
14.6947222222222	4\\
14.7002777777778	4\\
14.7058333333333	4\\
14.7113888888889	4\\
14.7169444444444	4\\
14.7225	4\\
14.7280555555556	5\\
14.7336111111111	5\\
14.7391666666667	8\\
14.7447222222222	3\\
14.7502777777778	4\\
14.7558333333333	5\\
14.7613888888889	3\\
14.7669444444444	2\\
14.7725	2\\
14.7780555555556	2\\
14.7836111111111	2\\
14.7891666666667	2\\
14.7947222222222	2\\
14.8002777777778	2\\
14.8058333333333	2\\
14.8113888888889	2\\
14.8169444444444	2\\
14.8225	2\\
14.8280555555556	9\\
14.8336111111111	10\\
14.8391666666667	10\\
14.8447222222222	4\\
14.8502777777778	5\\
14.8558333333333	5\\
14.8613888888889	5\\
14.8669444444444	5\\
14.8725	5\\
14.8780555555556	5\\
14.8836111111111	5\\
14.8891666666667	5\\
14.8947222222222	5\\
14.9002777777778	3\\
14.9058333333333	3\\
14.9113888888889	3\\
14.9169444444444	2\\
14.9225	3\\
14.9280555555556	3\\
14.9336111111111	3\\
14.9391666666667	3\\
14.9447222222222	3\\
14.9502777777778	3\\
14.9558333333333	6\\
14.9613888888889	4\\
14.9669444444444	3\\
14.9725	2\\
14.9780555555556	2\\
14.9836111111111	2\\
14.9891666666667	2\\
14.9947222222222	2\\
15.0002777777778	2\\
15.0058333333333	2\\
15.0113888888889	1\\
15.0169444444444	1\\
15.0225	2\\
15.0280555555556	2\\
15.0336111111111	2\\
15.0391666666667	3\\
15.0447222222222	2\\
15.0502777777778	2\\
15.0558333333333	2\\
15.0613888888889	3\\
15.0669444444444	5\\
15.0725	5\\
15.0780555555556	5\\
15.0836111111111	5\\
15.0891666666667	2\\
15.0947222222222	2\\
15.1002777777778	5\\
15.1058333333333	5\\
15.1113888888889	5\\
15.1169444444444	5\\
15.1225	3\\
15.1280555555556	3\\
15.1336111111111	2\\
15.1391666666667	2\\
15.1447222222222	3\\
15.1502777777778	4\\
15.1558333333333	3\\
15.1613888888889	3\\
15.1669444444444	4\\
15.1725	4\\
15.1780555555556	5\\
15.1836111111111	5\\
15.1891666666667	4\\
15.1947222222222	5\\
15.2002777777778	6\\
15.2058333333333	6\\
15.2113888888889	4\\
15.2169444444444	5\\
15.2225	6\\
15.2280555555556	6\\
15.2336111111111	8\\
15.2391666666667	8\\
15.2447222222222	3\\
15.2502777777778	4\\
15.2558333333333	2\\
15.2613888888889	2\\
15.2669444444444	2\\
15.2725	4\\
15.2780555555556	9\\
15.2836111111111	5\\
15.2891666666667	5\\
15.2947222222222	5\\
15.3002777777778	6\\
15.3058333333333	3\\
15.3113888888889	3\\
15.3169444444444	4\\
15.3225	2\\
15.3280555555556	3\\
15.3336111111111	4\\
15.3391666666667	4\\
15.3447222222222	2\\
15.3502777777778	2\\
15.3558333333333	5\\
15.3613888888889	5\\
15.3669444444444	5\\
15.3725	3\\
15.3780555555556	2\\
15.3836111111111	2\\
15.3891666666667	4\\
15.3947222222222	8\\
15.4002777777778	5\\
15.4058333333333	5\\
15.4113888888889	5\\
15.4169444444444	6\\
15.4225	7\\
15.4280555555556	6\\
15.4336111111111	4\\
15.4391666666667	3\\
15.4447222222222	3\\
15.4502777777778	2\\
15.4558333333333	2\\
15.4613888888889	2\\
15.4669444444444	2\\
15.4725	3\\
15.4780555555556	3\\
15.4836111111111	3\\
15.4891666666667	3\\
15.4947222222222	2\\
15.5002777777778	2\\
15.5058333333333	2\\
15.5113888888889	2\\
15.5169444444444	3\\
15.5225	4\\
15.5280555555556	4\\
15.5336111111111	3\\
15.5391666666667	2\\
15.5447222222222	3\\
15.5502777777778	2\\
15.5558333333333	3\\
15.5613888888889	4\\
15.5669444444444	4\\
15.5725	7\\
15.5780555555556	7\\
15.5836111111111	6\\
15.5891666666667	4\\
15.5947222222222	4\\
15.6002777777778	4\\
15.6058333333333	5\\
15.6113888888889	7\\
15.6169444444444	2\\
15.6225	1\\
15.6280555555556	-2\\
15.6336111111111	-1\\
15.6391666666667	-2\\
15.6447222222222	-4\\
15.6502777777778	-3\\
15.6558333333333	-3\\
15.6613888888889	-4\\
15.6669444444444	-5\\
15.6725	-3\\
15.6780555555556	-2\\
15.6836111111111	-2\\
15.6891666666667	-2\\
15.6947222222222	-2\\
15.7002777777778	-4\\
15.7058333333333	-5\\
15.7113888888889	-5\\
15.7169444444444	-5\\
15.7225	-5\\
15.7280555555556	-5\\
15.7336111111111	-3\\
15.7391666666667	-3\\
15.7447222222222	-2\\
15.7502777777778	-2\\
15.7558333333333	-3\\
15.7613888888889	-2\\
15.7669444444444	-3\\
15.7725	-5\\
15.7780555555556	-8\\
15.7836111111111	-6\\
15.7891666666667	-7\\
15.7947222222222	-12\\
15.8002777777778	-13\\
15.8058333333333	-7\\
15.8113888888889	-8\\
15.8169444444444	-9\\
15.8225	-9\\
15.8280555555556	-5\\
15.8336111111111	-5\\
15.8391666666667	-8\\
15.8447222222222	-8\\
15.8502777777778	-8\\
15.8558333333333	-8\\
15.8613888888889	-7\\
15.8669444444444	-4\\
15.8725	-4\\
15.8780555555556	-3\\
15.8836111111111	-2\\
15.8891666666667	-2\\
15.8947222222222	-2\\
15.9002777777778	-3\\
15.9058333333333	-5\\
15.9113888888889	-5\\
15.9169444444444	-5\\
15.9225	-6\\
15.9280555555556	-11\\
15.9336111111111	-4\\
15.9391666666667	-6\\
15.9447222222222	-2\\
15.9502777777778	-5\\
15.9558333333333	-6\\
15.9613888888889	-6\\
15.9669444444444	-6\\
15.9725	-7\\
15.9780555555556	-9\\
15.9836111111111	-11\\
15.9891666666667	-11\\
15.9947222222222	0\\
};
\addlegendentry{Cts Stoch Ctrl};

\addplot [color=mycolor2,solid,line width=1.5pt]
  table[row sep=crcr]{%
9.50027777777778	2\\
9.50583333333333	3\\
9.51138888888889	4\\
9.51694444444444	1\\
9.5225	-1\\
9.52805555555556	2\\
9.53361111111111	-3\\
9.53916666666667	-3\\
9.54472222222222	-2\\
9.55027777777778	-2\\
9.55583333333333	-2\\
9.56138888888889	-2\\
9.56694444444444	-2\\
9.5725	-1\\
9.57805555555555	-1\\
9.58361111111111	-3\\
9.58916666666667	3\\
9.59472222222222	2\\
9.60027777777778	5\\
9.60583333333333	2\\
9.61138888888889	2\\
9.61694444444444	-2\\
9.6225	-2\\
9.62805555555556	2\\
9.63361111111111	2\\
9.63916666666667	-2\\
9.64472222222222	2\\
9.65027777777778	4\\
9.65583333333333	5\\
9.66138888888889	2\\
9.66694444444444	3\\
9.6725	2\\
9.67805555555555	2\\
9.68361111111111	3\\
9.68916666666667	2\\
9.69472222222222	2\\
9.70027777777778	2\\
9.70583333333333	2\\
9.71138888888889	3\\
9.71694444444444	2\\
9.7225	2\\
9.72805555555555	2\\
9.73361111111111	2\\
9.73916666666667	2\\
9.74472222222222	2\\
9.75027777777778	3\\
9.75583333333333	4\\
9.76138888888889	2\\
9.76694444444444	2\\
9.7725	-2\\
9.77805555555556	-3\\
9.78361111111111	-4\\
9.78916666666667	-2\\
9.79472222222222	-3\\
9.80027777777778	-1\\
9.80583333333333	-2\\
9.81138888888889	-4\\
9.81694444444444	-4\\
9.8225	-6\\
9.82805555555555	-1\\
9.83361111111111	-2\\
9.83916666666667	-2\\
9.84472222222222	-2\\
9.85027777777778	-5\\
9.85583333333333	-2\\
9.86138888888889	-2\\
9.86694444444444	-4\\
9.8725	-3\\
9.87805555555556	-4\\
9.88361111111111	-2\\
9.88916666666667	-2\\
9.89472222222222	-2\\
9.90027777777778	-2\\
9.90583333333333	-3\\
9.91138888888889	-3\\
9.91694444444444	-2\\
9.9225	-3\\
9.92805555555555	-4\\
9.93361111111111	-5\\
9.93916666666667	-2\\
9.94472222222222	-2\\
9.95027777777778	-3\\
9.95583333333333	-2\\
9.96138888888889	-2\\
9.96694444444444	-5\\
9.9725	-7\\
9.97805555555555	-2\\
9.98361111111111	-2\\
9.98916666666667	-2\\
9.99472222222222	-3\\
10.0002777777778	-3\\
10.0058333333333	-3\\
10.0113888888889	-3\\
10.0169444444444	-3\\
10.0225	-6\\
10.0280555555556	-6\\
10.0336111111111	-6\\
10.0391666666667	-7\\
10.0447222222222	-5\\
10.0502777777778	-8\\
10.0558333333333	-2\\
10.0613888888889	-2\\
10.0669444444444	-2\\
10.0725	-2\\
10.0780555555556	-1\\
10.0836111111111	-2\\
10.0891666666667	-2\\
10.0947222222222	-2\\
10.1002777777778	-3\\
10.1058333333333	-5\\
10.1113888888889	-6\\
10.1169444444444	-5\\
10.1225	-5\\
10.1280555555556	-7\\
10.1336111111111	-7\\
10.1391666666667	-7\\
10.1447222222222	-6\\
10.1502777777778	-6\\
10.1558333333333	-6\\
10.1613888888889	-6\\
10.1669444444444	-2\\
10.1725	-4\\
10.1780555555556	-4\\
10.1836111111111	-1\\
10.1891666666667	-2\\
10.1947222222222	-2\\
10.2002777777778	-2\\
10.2058333333333	-2\\
10.2113888888889	-2\\
10.2169444444444	-2\\
10.2225	-2\\
10.2280555555556	-2\\
10.2336111111111	-3\\
10.2391666666667	-3\\
10.2447222222222	-2\\
10.2502777777778	-3\\
10.2558333333333	-2\\
10.2613888888889	-2\\
10.2669444444444	-3\\
10.2725	-2\\
10.2780555555556	-2\\
10.2836111111111	-3\\
10.2891666666667	-5\\
10.2947222222222	-5\\
10.3002777777778	-2\\
10.3058333333333	-2\\
10.3113888888889	-2\\
10.3169444444444	-3\\
10.3225	-2\\
10.3280555555556	-2\\
10.3336111111111	-2\\
10.3391666666667	-3\\
10.3447222222222	-3\\
10.3502777777778	-5\\
10.3558333333333	-2\\
10.3613888888889	-2\\
10.3669444444444	-2\\
10.3725	-2\\
10.3780555555556	-2\\
10.3836111111111	-2\\
10.3891666666667	-3\\
10.3947222222222	-3\\
10.4002777777778	-2\\
10.4058333333333	-2\\
10.4113888888889	-2\\
10.4169444444444	-2\\
10.4225	-2\\
10.4280555555556	-2\\
10.4336111111111	-2\\
10.4391666666667	-2\\
10.4447222222222	-3\\
10.4502777777778	-3\\
10.4558333333333	-3\\
10.4613888888889	-3\\
10.4669444444444	-4\\
10.4725	-4\\
10.4780555555556	-5\\
10.4836111111111	-5\\
10.4891666666667	-2\\
10.4947222222222	-3\\
10.5002777777778	-3\\
10.5058333333333	-4\\
10.5113888888889	-3\\
10.5169444444444	-1\\
10.5225	-2\\
10.5280555555556	-4\\
10.5336111111111	-7\\
10.5391666666667	-8\\
10.5447222222222	-9\\
10.5502777777778	-11\\
10.5558333333333	-3\\
10.5613888888889	-3\\
10.5669444444444	-5\\
10.5725	-5\\
10.5780555555556	-6\\
10.5836111111111	-9\\
10.5891666666667	-11\\
10.5947222222222	-11\\
10.6002777777778	-2\\
10.6058333333333	-2\\
10.6113888888889	-2\\
10.6169444444444	-2\\
10.6225	-2\\
10.6280555555556	-3\\
10.6336111111111	-3\\
10.6391666666667	-2\\
10.6447222222222	-2\\
10.6502777777778	-3\\
10.6558333333333	-3\\
10.6613888888889	-4\\
10.6669444444444	-5\\
10.6725	-4\\
10.6780555555556	-3\\
10.6836111111111	-5\\
10.6891666666667	-6\\
10.6947222222222	-6\\
10.7002777777778	-2\\
10.7058333333333	-3\\
10.7113888888889	-4\\
10.7169444444444	-4\\
10.7225	-4\\
10.7280555555556	-2\\
10.7336111111111	-2\\
10.7391666666667	-3\\
10.7447222222222	-3\\
10.7502777777778	-3\\
10.7558333333333	-3\\
10.7613888888889	-3\\
10.7669444444444	-1\\
10.7725	-3\\
10.7780555555556	-3\\
10.7836111111111	-3\\
10.7891666666667	-3\\
10.7947222222222	-4\\
10.8002777777778	-6\\
10.8058333333333	-8\\
10.8113888888889	-8\\
10.8169444444444	-2\\
10.8225	-2\\
10.8280555555556	-2\\
10.8336111111111	-2\\
10.8391666666667	-2\\
10.8447222222222	-4\\
10.8502777777778	-6\\
10.8558333333333	-2\\
10.8613888888889	-3\\
10.8669444444444	-5\\
10.8725	-6\\
10.8780555555556	-6\\
10.8836111111111	-6\\
10.8891666666667	-6\\
10.8947222222222	-6\\
10.9002777777778	-6\\
10.9058333333333	-6\\
10.9113888888889	-2\\
10.9169444444444	-2\\
10.9225	-2\\
10.9280555555556	-2\\
10.9336111111111	-2\\
10.9391666666667	-6\\
10.9447222222222	-2\\
10.9502777777778	-2\\
10.9558333333333	-3\\
10.9613888888889	-2\\
10.9669444444444	-2\\
10.9725	-2\\
10.9780555555556	-1\\
10.9836111111111	-2\\
10.9891666666667	-2\\
10.9947222222222	-3\\
11.0002777777778	-5\\
11.0058333333333	-5\\
11.0113888888889	-5\\
11.0169444444444	-7\\
11.0225	-8\\
11.0280555555556	-8\\
11.0336111111111	-3\\
11.0391666666667	-4\\
11.0447222222222	-3\\
11.0502777777778	-1\\
11.0558333333333	-4\\
11.0613888888889	-4\\
11.0669444444444	-4\\
11.0725	-2\\
11.0780555555556	-2\\
11.0836111111111	-2\\
11.0891666666667	-2\\
11.0947222222222	-2\\
11.1002777777778	-2\\
11.1058333333333	-4\\
11.1113888888889	-8\\
11.1169444444444	-10\\
11.1225	-10\\
11.1280555555556	-10\\
11.1336111111111	-10\\
11.1391666666667	-10\\
11.1447222222222	-11\\
11.1502777777778	-2\\
11.1558333333333	-2\\
11.1613888888889	-2\\
11.1669444444444	-3\\
11.1725	-3\\
11.1780555555556	-3\\
11.1836111111111	-4\\
11.1891666666667	-4\\
11.1947222222222	-4\\
11.2002777777778	-5\\
11.2058333333333	-2\\
11.2113888888889	-2\\
11.2169444444444	-3\\
11.2225	-3\\
11.2280555555556	-3\\
11.2336111111111	-3\\
11.2391666666667	-5\\
11.2447222222222	-6\\
11.2502777777778	-1\\
11.2558333333333	-2\\
11.2613888888889	-2\\
11.2669444444444	-2\\
11.2725	-2\\
11.2780555555556	-2\\
11.2836111111111	-2\\
11.2891666666667	-2\\
11.2947222222222	-3\\
11.3002777777778	-5\\
11.3058333333333	-5\\
11.3113888888889	-2\\
11.3169444444444	-3\\
11.3225	-3\\
11.3280555555556	-3\\
11.3336111111111	-4\\
11.3391666666667	-4\\
11.3447222222222	-4\\
11.3502777777778	-2\\
11.3558333333333	-2\\
11.3613888888889	-2\\
11.3669444444444	-3\\
11.3725	-4\\
11.3780555555556	-4\\
11.3836111111111	-4\\
11.3891666666667	-5\\
11.3947222222222	-4\\
11.4002777777778	-4\\
11.4058333333333	-3\\
11.4113888888889	-3\\
11.4169444444444	-3\\
11.4225	-5\\
11.4280555555556	-2\\
11.4336111111111	-4\\
11.4391666666667	-4\\
11.4447222222222	-2\\
11.4502777777778	-2\\
11.4558333333333	-2\\
11.4613888888889	-2\\
11.4669444444444	-2\\
11.4725	-2\\
11.4780555555556	-2\\
11.4836111111111	-3\\
11.4891666666667	-3\\
11.4947222222222	-2\\
11.5002777777778	-2\\
11.5058333333333	-2\\
11.5113888888889	-5\\
11.5169444444444	-2\\
11.5225	-2\\
11.5280555555556	-2\\
11.5336111111111	-2\\
11.5391666666667	-2\\
11.5447222222222	-3\\
11.5502777777778	-2\\
11.5558333333333	-2\\
11.5613888888889	-3\\
11.5669444444444	-7\\
11.5725	-8\\
11.5780555555556	-8\\
11.5836111111111	-8\\
11.5891666666667	-2\\
11.5947222222222	-3\\
11.6002777777778	-2\\
11.6058333333333	-2\\
11.6113888888889	-2\\
11.6169444444444	-2\\
11.6225	-2\\
11.6280555555556	-3\\
11.6336111111111	-5\\
11.6391666666667	-6\\
11.6447222222222	-5\\
11.6502777777778	-4\\
11.6558333333333	-4\\
11.6613888888889	-5\\
11.6669444444444	-5\\
11.6725	-2\\
11.6780555555556	-2\\
11.6836111111111	-2\\
11.6891666666667	-2\\
11.6947222222222	-2\\
11.7002777777778	-2\\
11.7058333333333	-2\\
11.7113888888889	-2\\
11.7169444444444	-2\\
11.7225	-2\\
11.7280555555556	-2\\
11.7336111111111	-2\\
11.7391666666667	-2\\
11.7447222222222	-2\\
11.7502777777778	-2\\
11.7558333333333	-2\\
11.7613888888889	-2\\
11.7669444444444	-2\\
11.7725	-2\\
11.7780555555556	-2\\
11.7836111111111	-2\\
11.7891666666667	-2\\
11.7947222222222	-1\\
11.8002777777778	-2\\
11.8058333333333	-2\\
11.8113888888889	-2\\
11.8169444444444	-2\\
11.8225	-2\\
11.8280555555556	-2\\
11.8336111111111	-2\\
11.8391666666667	-2\\
11.8447222222222	-2\\
11.8502777777778	-3\\
11.8558333333333	-2\\
11.8613888888889	-4\\
11.8669444444444	-3\\
11.8725	-3\\
11.8780555555556	-3\\
11.8836111111111	-2\\
11.8891666666667	-2\\
11.8947222222222	-3\\
11.9002777777778	-2\\
11.9058333333333	-2\\
11.9113888888889	-2\\
11.9169444444444	-2\\
11.9225	-2\\
11.9280555555556	-2\\
11.9336111111111	-2\\
11.9391666666667	-2\\
11.9447222222222	-2\\
11.9502777777778	-2\\
11.9558333333333	-2\\
11.9613888888889	-3\\
11.9669444444444	-3\\
11.9725	-3\\
11.9780555555556	-2\\
11.9836111111111	-2\\
11.9891666666667	-2\\
11.9947222222222	-2\\
12.0002777777778	-2\\
12.0058333333333	-2\\
12.0113888888889	-2\\
12.0169444444444	-2\\
12.0225	-2\\
12.0280555555556	-2\\
12.0336111111111	-2\\
12.0391666666667	-3\\
12.0447222222222	-4\\
12.0502777777778	-4\\
12.0558333333333	-2\\
12.0613888888889	-2\\
12.0669444444444	-2\\
12.0725	-3\\
12.0780555555556	-2\\
12.0836111111111	-3\\
12.0891666666667	-4\\
12.0947222222222	-4\\
12.1002777777778	-4\\
12.1058333333333	-4\\
12.1113888888889	-2\\
12.1169444444444	-3\\
12.1225	-2\\
12.1280555555556	-2\\
12.1336111111111	-2\\
12.1391666666667	-4\\
12.1447222222222	-5\\
12.1502777777778	-6\\
12.1558333333333	-3\\
12.1613888888889	-3\\
12.1669444444444	-3\\
12.1725	-3\\
12.1780555555556	-3\\
12.1836111111111	-3\\
12.1891666666667	-2\\
12.1947222222222	-2\\
12.2002777777778	-2\\
12.2058333333333	-2\\
12.2113888888889	-2\\
12.2169444444444	-2\\
12.2225	-2\\
12.2280555555556	-2\\
12.2336111111111	-2\\
12.2391666666667	-2\\
12.2447222222222	-4\\
12.2502777777778	-5\\
12.2558333333333	-2\\
12.2613888888889	-2\\
12.2669444444444	-2\\
12.2725	-2\\
12.2780555555556	-2\\
12.2836111111111	-2\\
12.2891666666667	-2\\
12.2947222222222	-2\\
12.3002777777778	-2\\
12.3058333333333	-3\\
12.3113888888889	-3\\
12.3169444444444	-3\\
12.3225	-3\\
12.3280555555556	-3\\
12.3336111111111	-4\\
12.3391666666667	-5\\
12.3447222222222	-2\\
12.3502777777778	-2\\
12.3558333333333	-2\\
12.3613888888889	-2\\
12.3669444444444	-2\\
12.3725	-3\\
12.3780555555556	-3\\
12.3836111111111	-3\\
12.3891666666667	-4\\
12.3947222222222	-4\\
12.4002777777778	-4\\
12.4058333333333	-4\\
12.4113888888889	-4\\
12.4169444444444	-4\\
12.4225	-2\\
12.4280555555556	-2\\
12.4336111111111	-2\\
12.4391666666667	-5\\
12.4447222222222	-7\\
12.4502777777778	-7\\
12.4558333333333	-7\\
12.4613888888889	-7\\
12.4669444444444	-7\\
12.4725	-9\\
12.4780555555556	-10\\
12.4836111111111	-10\\
12.4891666666667	-11\\
12.4947222222222	-11\\
12.5002777777778	-3\\
12.5058333333333	-4\\
12.5113888888889	-4\\
12.5169444444444	-4\\
12.5225	-4\\
12.5280555555556	-5\\
12.5336111111111	-3\\
12.5391666666667	-1\\
12.5447222222222	-2\\
12.5502777777778	-4\\
12.5558333333333	-3\\
12.5613888888889	-3\\
12.5669444444444	-3\\
12.5725	-2\\
12.5780555555556	-2\\
12.5836111111111	-4\\
12.5891666666667	-4\\
12.5947222222222	-3\\
12.6002777777778	-3\\
12.6058333333333	-3\\
12.6113888888889	-3\\
12.6169444444444	-1\\
12.6225	-2\\
12.6280555555556	-3\\
12.6336111111111	-3\\
12.6391666666667	-3\\
12.6447222222222	-3\\
12.6502777777778	-2\\
12.6558333333333	-2\\
12.6613888888889	-2\\
12.6669444444444	-3\\
12.6725	-4\\
12.6780555555556	-4\\
12.6836111111111	-2\\
12.6891666666667	-2\\
12.6947222222222	-2\\
12.7002777777778	-2\\
12.7058333333333	-2\\
12.7113888888889	-2\\
12.7169444444444	-1\\
12.7225	-2\\
12.7280555555556	-2\\
12.7336111111111	-2\\
12.7391666666667	-2\\
12.7447222222222	-3\\
12.7502777777778	-3\\
12.7558333333333	-3\\
12.7613888888889	-3\\
12.7669444444444	-3\\
12.7725	-3\\
12.7780555555556	-4\\
12.7836111111111	-4\\
12.7891666666667	-4\\
12.7947222222222	-2\\
12.8002777777778	-2\\
12.8058333333333	-2\\
12.8113888888889	-2\\
12.8169444444444	-2\\
12.8225	-3\\
12.8280555555556	-3\\
12.8336111111111	-3\\
12.8391666666667	-3\\
12.8447222222222	-4\\
12.8502777777778	-5\\
12.8558333333333	-5\\
12.8613888888889	-6\\
12.8669444444444	-7\\
12.8725	-7\\
12.8780555555556	-3\\
12.8836111111111	-2\\
12.8891666666667	-2\\
12.8947222222222	-2\\
12.9002777777778	-2\\
12.9058333333333	-3\\
12.9113888888889	-4\\
12.9169444444444	-2\\
12.9225	-2\\
12.9280555555556	-2\\
12.9336111111111	-2\\
12.9391666666667	-2\\
12.9447222222222	-2\\
12.9502777777778	-2\\
12.9558333333333	-3\\
12.9613888888889	-5\\
12.9669444444444	-5\\
12.9725	-2\\
12.9780555555556	-2\\
12.9836111111111	-3\\
12.9891666666667	-3\\
12.9947222222222	-3\\
13.0002777777778	-2\\
13.0058333333333	-2\\
13.0113888888889	-2\\
13.0169444444444	-2\\
13.0225	-2\\
13.0280555555556	-2\\
13.0336111111111	-2\\
13.0391666666667	-2\\
13.0447222222222	-2\\
13.0502777777778	-4\\
13.0558333333333	-4\\
13.0613888888889	-3\\
13.0669444444444	-3\\
13.0725	-3\\
13.0780555555556	-4\\
13.0836111111111	-4\\
13.0891666666667	-4\\
13.0947222222222	-5\\
13.1002777777778	-5\\
13.1058333333333	3\\
13.1113888888889	3\\
13.1169444444444	7\\
13.1225	2\\
13.1280555555556	4\\
13.1336111111111	4\\
13.1391666666667	5\\
13.1447222222222	5\\
13.1502777777778	7\\
13.1558333333333	7\\
13.1613888888889	9\\
13.1669444444444	9\\
13.1725	9\\
13.1780555555556	9\\
13.1836111111111	9\\
13.1891666666667	9\\
13.1947222222222	9\\
13.2002777777778	9\\
13.2058333333333	9\\
13.2113888888889	10\\
13.2169444444444	9\\
13.2225	9\\
13.2280555555556	10\\
13.2336111111111	13\\
13.2391666666667	15\\
13.2447222222222	18\\
13.2502777777778	18\\
13.2558333333333	20\\
13.2613888888889	20\\
13.2669444444444	20\\
13.2725	20\\
13.2780555555556	20\\
13.2836111111111	20\\
13.2891666666667	20\\
13.2947222222222	20\\
13.3002777777778	20\\
13.3058333333333	19\\
13.3113888888889	20\\
13.3169444444444	2\\
13.3225	2\\
13.3280555555556	5\\
13.3336111111111	2\\
13.3391666666667	5\\
13.3447222222222	3\\
13.3502777777778	4\\
13.3558333333333	4\\
13.3613888888889	4\\
13.3669444444444	4\\
13.3725	4\\
13.3780555555556	6\\
13.3836111111111	9\\
13.3891666666667	9\\
13.3947222222222	10\\
13.4002777777778	15\\
13.4058333333333	18\\
13.4113888888889	20\\
13.4169444444444	2\\
13.4225	2\\
13.4280555555556	2\\
13.4336111111111	2\\
13.4391666666667	2\\
13.4447222222222	2\\
13.4502777777778	2\\
13.4558333333333	4\\
13.4613888888889	4\\
13.4669444444444	4\\
13.4725	2\\
13.4780555555556	2\\
13.4836111111111	2\\
13.4891666666667	3\\
13.4947222222222	2\\
13.5002777777778	5\\
13.5058333333333	6\\
13.5113888888889	2\\
13.5169444444444	4\\
13.5225	2\\
13.5280555555556	2\\
13.5336111111111	3\\
13.5391666666667	3\\
13.5447222222222	3\\
13.5502777777778	2\\
13.5558333333333	2\\
13.5613888888889	2\\
13.5669444444444	2\\
13.5725	5\\
13.5780555555556	2\\
13.5836111111111	2\\
13.5891666666667	2\\
13.5947222222222	2\\
13.6002777777778	2\\
13.6058333333333	2\\
13.6113888888889	2\\
13.6169444444444	2\\
13.6225	2\\
13.6280555555556	2\\
13.6336111111111	2\\
13.6391666666667	4\\
13.6447222222222	4\\
13.6502777777778	4\\
13.6558333333333	4\\
13.6613888888889	4\\
13.6669444444444	4\\
13.6725	4\\
13.6780555555556	4\\
13.6836111111111	4\\
13.6891666666667	4\\
13.6947222222222	2\\
13.7002777777778	2\\
13.7058333333333	2\\
13.7113888888889	2\\
13.7169444444444	3\\
13.7225	4\\
13.7280555555556	4\\
13.7336111111111	3\\
13.7391666666667	4\\
13.7447222222222	4\\
13.7502777777778	5\\
13.7558333333333	6\\
13.7613888888889	6\\
13.7669444444444	6\\
13.7725	6\\
13.7780555555556	5\\
13.7836111111111	5\\
13.7891666666667	2\\
13.7947222222222	2\\
13.8002777777778	3\\
13.8058333333333	4\\
13.8113888888889	4\\
13.8169444444444	4\\
13.8225	4\\
13.8280555555556	4\\
13.8336111111111	5\\
13.8391666666667	5\\
13.8447222222222	5\\
13.8502777777778	5\\
13.8558333333333	3\\
13.8613888888889	3\\
13.8669444444444	3\\
13.8725	2\\
13.8780555555556	2\\
13.8836111111111	2\\
13.8891666666667	3\\
13.8947222222222	3\\
13.9002777777778	3\\
13.9058333333333	3\\
13.9113888888889	3\\
13.9169444444444	2\\
13.9225	3\\
13.9280555555556	4\\
13.9336111111111	4\\
13.9391666666667	6\\
13.9447222222222	6\\
13.9502777777778	2\\
13.9558333333333	2\\
13.9613888888889	2\\
13.9669444444444	4\\
13.9725	2\\
13.9780555555556	2\\
13.9836111111111	3\\
13.9891666666667	3\\
13.9947222222222	4\\
14.0002777777778	7\\
14.0058333333333	7\\
14.0113888888889	2\\
14.0169444444444	4\\
14.0225	7\\
14.0280555555556	9\\
14.0336111111111	12\\
14.0391666666667	12\\
14.0447222222222	12\\
14.0502777777778	15\\
14.0558333333333	18\\
14.0613888888889	19\\
14.0669444444444	20\\
14.0725	20\\
14.0780555555556	20\\
14.0836111111111	20\\
14.0891666666667	20\\
14.0947222222222	20\\
14.1002777777778	20\\
14.1058333333333	20\\
14.1113888888889	20\\
14.1169444444444	20\\
14.1225	20\\
14.1280555555556	20\\
14.1336111111111	19\\
14.1391666666667	3\\
14.1447222222222	5\\
14.1502777777778	5\\
14.1558333333333	6\\
14.1613888888889	6\\
14.1669444444444	8\\
14.1725	2\\
14.1780555555556	3\\
14.1836111111111	4\\
14.1891666666667	5\\
14.1947222222222	7\\
14.2002777777778	8\\
14.2058333333333	8\\
14.2113888888889	8\\
14.2169444444444	8\\
14.2225	2\\
14.2280555555556	3\\
14.2336111111111	4\\
14.2391666666667	5\\
14.2447222222222	8\\
14.2502777777778	10\\
14.2558333333333	11\\
14.2613888888889	12\\
14.2669444444444	15\\
14.2725	16\\
14.2780555555556	17\\
14.2836111111111	17\\
14.2891666666667	19\\
14.2947222222222	20\\
14.3002777777778	20\\
14.3058333333333	20\\
14.3113888888889	20\\
14.3169444444444	20\\
14.3225	20\\
14.3280555555556	20\\
14.3336111111111	20\\
14.3391666666667	20\\
14.3447222222222	20\\
14.3502777777778	20\\
14.3558333333333	20\\
14.3613888888889	20\\
14.3669444444444	20\\
14.3725	20\\
14.3780555555556	2\\
14.3836111111111	2\\
14.3891666666667	2\\
14.3947222222222	2\\
14.4002777777778	3\\
14.4058333333333	2\\
14.4113888888889	2\\
14.4169444444444	3\\
14.4225	3\\
14.4280555555556	3\\
14.4336111111111	2\\
14.4391666666667	2\\
14.4447222222222	2\\
14.4502777777778	2\\
14.4558333333333	2\\
14.4613888888889	2\\
14.4669444444444	2\\
14.4725	2\\
14.4780555555556	2\\
14.4836111111111	2\\
14.4891666666667	2\\
14.4947222222222	2\\
14.5002777777778	2\\
14.5058333333333	2\\
14.5113888888889	2\\
14.5169444444444	2\\
14.5225	2\\
14.5280555555556	2\\
14.5336111111111	2\\
14.5391666666667	2\\
14.5447222222222	2\\
14.5502777777778	2\\
14.5558333333333	2\\
14.5613888888889	2\\
14.5669444444444	2\\
14.5725	2\\
14.5780555555556	5\\
14.5836111111111	2\\
14.5891666666667	3\\
14.5947222222222	4\\
14.6002777777778	4\\
14.6058333333333	6\\
14.6113888888889	6\\
14.6169444444444	6\\
14.6225	2\\
14.6280555555556	2\\
14.6336111111111	3\\
14.6391666666667	3\\
14.6447222222222	3\\
14.6502777777778	3\\
14.6558333333333	2\\
14.6613888888889	2\\
14.6669444444444	2\\
14.6725	1\\
14.6780555555556	3\\
14.6836111111111	-4\\
14.6891666666667	-4\\
14.6947222222222	-2\\
14.7002777777778	-2\\
14.7058333333333	-2\\
14.7113888888889	-2\\
14.7169444444444	-2\\
14.7225	-3\\
14.7280555555556	-2\\
14.7336111111111	-2\\
14.7391666666667	-2\\
14.7447222222222	-3\\
14.7502777777778	-2\\
14.7558333333333	-2\\
14.7613888888889	-3\\
14.7669444444444	-4\\
14.7725	-4\\
14.7780555555556	-4\\
14.7836111111111	-3\\
14.7891666666667	-3\\
14.7947222222222	-4\\
14.8002777777778	-4\\
14.8058333333333	-6\\
14.8113888888889	-4\\
14.8169444444444	-5\\
14.8225	-3\\
14.8280555555556	7\\
14.8336111111111	8\\
14.8391666666667	8\\
14.8447222222222	2\\
14.8502777777778	3\\
14.8558333333333	3\\
14.8613888888889	3\\
14.8669444444444	3\\
14.8725	3\\
14.8780555555556	3\\
14.8836111111111	3\\
14.8891666666667	3\\
14.8947222222222	3\\
14.9002777777778	2\\
14.9058333333333	2\\
14.9113888888889	3\\
14.9169444444444	3\\
14.9225	2\\
14.9280555555556	2\\
14.9336111111111	3\\
14.9391666666667	3\\
14.9447222222222	3\\
14.9502777777778	3\\
14.9558333333333	5\\
14.9613888888889	3\\
14.9669444444444	2\\
14.9725	2\\
14.9780555555556	2\\
14.9836111111111	2\\
14.9891666666667	2\\
14.9947222222222	3\\
15.0002777777778	3\\
15.0058333333333	3\\
15.0113888888889	2\\
15.0169444444444	2\\
15.0225	2\\
15.0280555555556	2\\
15.0336111111111	2\\
15.0391666666667	3\\
15.0447222222222	2\\
15.0502777777778	2\\
15.0558333333333	2\\
15.0613888888889	3\\
15.0669444444444	5\\
15.0725	6\\
15.0780555555556	6\\
15.0836111111111	7\\
15.0891666666667	2\\
15.0947222222222	2\\
15.1002777777778	5\\
15.1058333333333	5\\
15.1113888888889	7\\
15.1169444444444	7\\
15.1225	2\\
15.1280555555556	3\\
15.1336111111111	2\\
15.1391666666667	2\\
15.1447222222222	4\\
15.1502777777778	5\\
15.1558333333333	2\\
15.1613888888889	3\\
15.1669444444444	4\\
15.1725	1\\
15.1780555555556	2\\
15.1836111111111	2\\
15.1891666666667	2\\
15.1947222222222	3\\
15.2002777777778	5\\
15.2058333333333	5\\
15.2113888888889	2\\
15.2169444444444	3\\
15.2225	4\\
15.2280555555556	4\\
15.2336111111111	5\\
15.2391666666667	5\\
15.2447222222222	2\\
15.2502777777778	3\\
15.2558333333333	2\\
15.2613888888889	2\\
15.2669444444444	2\\
15.2725	2\\
15.2780555555556	7\\
15.2836111111111	3\\
15.2891666666667	3\\
15.2947222222222	3\\
15.3002777777778	4\\
15.3058333333333	3\\
15.3113888888889	4\\
15.3169444444444	6\\
15.3225	2\\
15.3280555555556	3\\
15.3336111111111	3\\
15.3391666666667	2\\
15.3447222222222	2\\
15.3502777777778	2\\
15.3558333333333	5\\
15.3613888888889	5\\
15.3669444444444	6\\
15.3725	2\\
15.3780555555556	2\\
15.3836111111111	2\\
15.3891666666667	4\\
15.3947222222222	7\\
15.4002777777778	2\\
15.4058333333333	2\\
15.4113888888889	2\\
15.4169444444444	4\\
15.4225	6\\
15.4280555555556	5\\
15.4336111111111	2\\
15.4391666666667	2\\
15.4447222222222	2\\
15.4502777777778	2\\
15.4558333333333	2\\
15.4613888888889	2\\
15.4669444444444	2\\
15.4725	3\\
15.4780555555556	3\\
15.4836111111111	3\\
15.4891666666667	3\\
15.4947222222222	3\\
15.5002777777778	4\\
15.5058333333333	4\\
15.5113888888889	2\\
15.5169444444444	2\\
15.5225	4\\
15.5280555555556	4\\
15.5336111111111	2\\
15.5391666666667	2\\
15.5447222222222	3\\
15.5502777777778	2\\
15.5558333333333	4\\
15.5613888888889	6\\
15.5669444444444	2\\
15.5725	4\\
15.5780555555556	4\\
15.5836111111111	3\\
15.5891666666667	2\\
15.5947222222222	2\\
15.6002777777778	2\\
15.6058333333333	3\\
15.6113888888889	5\\
15.6169444444444	2\\
15.6225	2\\
15.6280555555556	2\\
15.6336111111111	3\\
15.6391666666667	5\\
15.6447222222222	2\\
15.6502777777778	3\\
15.6558333333333	4\\
15.6613888888889	2\\
15.6669444444444	2\\
15.6725	4\\
15.6780555555556	4\\
15.6836111111111	4\\
15.6891666666667	5\\
15.6947222222222	5\\
15.7002777777778	3\\
15.7058333333333	3\\
15.7113888888889	3\\
15.7169444444444	3\\
15.7225	2\\
15.7280555555556	2\\
15.7336111111111	4\\
15.7391666666667	4\\
15.7447222222222	5\\
15.7502777777778	11\\
15.7558333333333	2\\
15.7613888888889	8\\
15.7669444444444	9\\
15.7725	2\\
15.7780555555556	2\\
15.7836111111111	5\\
15.7891666666667	2\\
15.7947222222222	3\\
15.8002777777778	2\\
15.8058333333333	2\\
15.8113888888889	2\\
15.8169444444444	2\\
15.8225	2\\
15.8280555555556	3\\
15.8336111111111	1\\
15.8391666666667	2\\
15.8447222222222	2\\
15.8502777777778	3\\
15.8558333333333	3\\
15.8613888888889	4\\
15.8669444444444	8\\
15.8725	9\\
15.8780555555556	9\\
15.8836111111111	9\\
15.8891666666667	10\\
15.8947222222222	10\\
15.9002777777778	2\\
15.9058333333333	2\\
15.9113888888889	2\\
15.9169444444444	3\\
15.9225	2\\
15.9280555555556	2\\
15.9336111111111	2\\
15.9391666666667	2\\
15.9447222222222	5\\
15.9502777777778	2\\
15.9558333333333	2\\
15.9613888888889	2\\
15.9669444444444	3\\
15.9725	2\\
15.9780555555556	2\\
15.9836111111111	2\\
15.9891666666667	3\\
15.9947222222222	7\\
};
\addlegendentry{Cts Stoch Ctrl w nFPC};

\addplot [color=mycolor3,solid,line width=1.5pt]
  table[row sep=crcr]{%
9.50027777777778	1\\
9.50583333333333	3\\
9.51138888888889	4\\
9.51694444444444	3\\
9.5225	5\\
9.52805555555556	9\\
9.53361111111111	9\\
9.53916666666667	4\\
9.54472222222222	3\\
9.55027777777778	4\\
9.55583333333333	2\\
9.56138888888889	3\\
9.56694444444444	5\\
9.5725	5\\
9.57805555555555	5\\
9.58361111111111	6\\
9.58916666666667	7\\
9.59472222222222	6\\
9.60027777777778	3\\
9.60583333333333	3\\
9.61138888888889	2\\
9.61694444444444	2\\
9.6225	6\\
9.62805555555556	5\\
9.63361111111111	3\\
9.63916666666667	3\\
9.64472222222222	4\\
9.65027777777778	4\\
9.65583333333333	4\\
9.66138888888889	4\\
9.66694444444444	4\\
9.6725	3\\
9.67805555555555	2\\
9.68361111111111	3\\
9.68916666666667	2\\
9.69472222222222	2\\
9.70027777777778	2\\
9.70583333333333	2\\
9.71138888888889	3\\
9.71694444444444	2\\
9.7225	2\\
9.72805555555555	2\\
9.73361111111111	2\\
9.73916666666667	2\\
9.74472222222222	2\\
9.75027777777778	3\\
9.75583333333333	4\\
9.76138888888889	3\\
9.76694444444444	2\\
9.7725	3\\
9.77805555555556	3\\
9.78361111111111	2\\
9.78916666666667	4\\
9.79472222222222	3\\
9.80027777777778	5\\
9.80583333333333	5\\
9.81138888888889	4\\
9.81694444444444	3\\
9.8225	2\\
9.82805555555555	4\\
9.83361111111111	5\\
9.83916666666667	6\\
9.84472222222222	6\\
9.85027777777778	3\\
9.85583333333333	5\\
9.86138888888889	3\\
9.86694444444444	3\\
9.8725	4\\
9.87805555555556	3\\
9.88361111111111	3\\
9.88916666666667	3\\
9.89472222222222	3\\
9.90027777777778	3\\
9.90583333333333	2\\
9.91138888888889	3\\
9.91694444444444	3\\
9.9225	2\\
9.92805555555555	2\\
9.93361111111111	2\\
9.93916666666667	3\\
9.94472222222222	2\\
9.95027777777778	2\\
9.95583333333333	4\\
9.96138888888889	5\\
9.96694444444444	3\\
9.9725	4\\
9.97805555555555	3\\
9.98361111111111	3\\
9.98916666666667	4\\
9.99472222222222	4\\
10.0002777777778	4\\
10.0058333333333	4\\
10.0113888888889	4\\
10.0169444444444	2\\
10.0225	2\\
10.0280555555556	2\\
10.0336111111111	2\\
10.0391666666667	2\\
10.0447222222222	3\\
10.0502777777778	2\\
10.0558333333333	2\\
10.0613888888889	2\\
10.0669444444444	5\\
10.0725	5\\
10.0780555555556	4\\
10.0836111111111	4\\
10.0891666666667	4\\
10.0947222222222	4\\
10.1002777777778	3\\
10.1058333333333	2\\
10.1113888888889	2\\
10.1169444444444	4\\
10.1225	4\\
10.1280555555556	4\\
10.1336111111111	4\\
10.1391666666667	4\\
10.1447222222222	6\\
10.1502777777778	6\\
10.1558333333333	6\\
10.1613888888889	6\\
10.1669444444444	2\\
10.1725	3\\
10.1780555555556	3\\
10.1836111111111	3\\
10.1891666666667	3\\
10.1947222222222	3\\
10.2002777777778	3\\
10.2058333333333	3\\
10.2113888888889	3\\
10.2169444444444	3\\
10.2225	2\\
10.2280555555556	3\\
10.2336111111111	4\\
10.2391666666667	4\\
10.2447222222222	4\\
10.2502777777778	5\\
10.2558333333333	3\\
10.2613888888889	5\\
10.2669444444444	5\\
10.2725	4\\
10.2780555555556	5\\
10.2836111111111	3\\
10.2891666666667	2\\
10.2947222222222	2\\
10.3002777777778	3\\
10.3058333333333	3\\
10.3113888888889	3\\
10.3169444444444	5\\
10.3225	6\\
10.3280555555556	5\\
10.3336111111111	4\\
10.3391666666667	3\\
10.3447222222222	2\\
10.3502777777778	2\\
10.3558333333333	4\\
10.3613888888889	4\\
10.3669444444444	4\\
10.3725	4\\
10.3780555555556	4\\
10.3836111111111	4\\
10.3891666666667	3\\
10.3947222222222	3\\
10.4002777777778	2\\
10.4058333333333	2\\
10.4113888888889	2\\
10.4169444444444	2\\
10.4225	2\\
10.4280555555556	3\\
10.4336111111111	3\\
10.4391666666667	3\\
10.4447222222222	2\\
10.4502777777778	2\\
10.4558333333333	2\\
10.4613888888889	2\\
10.4669444444444	2\\
10.4725	4\\
10.4780555555556	3\\
10.4836111111111	3\\
10.4891666666667	5\\
10.4947222222222	4\\
10.5002777777778	4\\
10.5058333333333	3\\
10.5113888888889	2\\
10.5169444444444	2\\
10.5225	2\\
10.5280555555556	2\\
10.5336111111111	2\\
10.5391666666667	2\\
10.5447222222222	2\\
10.5502777777778	2\\
10.5558333333333	2\\
10.5613888888889	2\\
10.5669444444444	2\\
10.5725	2\\
10.5780555555556	2\\
10.5836111111111	2\\
10.5891666666667	2\\
10.5947222222222	2\\
10.6002777777778	3\\
10.6058333333333	3\\
10.6113888888889	4\\
10.6169444444444	4\\
10.6225	4\\
10.6280555555556	3\\
10.6336111111111	3\\
10.6391666666667	3\\
10.6447222222222	4\\
10.6502777777778	3\\
10.6558333333333	3\\
10.6613888888889	3\\
10.6669444444444	2\\
10.6725	4\\
10.6780555555556	2\\
10.6836111111111	2\\
10.6891666666667	2\\
10.6947222222222	2\\
10.7002777777778	3\\
10.7058333333333	2\\
10.7113888888889	2\\
10.7169444444444	2\\
10.7225	2\\
10.7280555555556	2\\
10.7336111111111	3\\
10.7391666666667	2\\
10.7447222222222	4\\
10.7502777777778	5\\
10.7558333333333	5\\
10.7613888888889	4\\
10.7669444444444	3\\
10.7725	2\\
10.7780555555556	2\\
10.7836111111111	2\\
10.7891666666667	2\\
10.7947222222222	2\\
10.8002777777778	2\\
10.8058333333333	2\\
10.8113888888889	2\\
10.8169444444444	3\\
10.8225	3\\
10.8280555555556	3\\
10.8336111111111	3\\
10.8391666666667	4\\
10.8447222222222	3\\
10.8502777777778	2\\
10.8558333333333	2\\
10.8613888888889	2\\
10.8669444444444	2\\
10.8725	2\\
10.8780555555556	2\\
10.8836111111111	2\\
10.8891666666667	2\\
10.8947222222222	2\\
10.9002777777778	2\\
10.9058333333333	2\\
10.9113888888889	4\\
10.9169444444444	4\\
10.9225	4\\
10.9280555555556	5\\
10.9336111111111	7\\
10.9391666666667	3\\
10.9447222222222	7\\
10.9502777777778	5\\
10.9558333333333	6\\
10.9613888888889	4\\
10.9669444444444	6\\
10.9725	7\\
10.9780555555556	10\\
10.9836111111111	10\\
10.9891666666667	13\\
10.9947222222222	13\\
11.0002777777778	14\\
11.0058333333333	14\\
11.0113888888889	15\\
11.0169444444444	14\\
11.0225	14\\
11.0280555555556	14\\
11.0336111111111	15\\
11.0391666666667	14\\
11.0447222222222	3\\
11.0502777777778	3\\
11.0558333333333	2\\
11.0613888888889	2\\
11.0669444444444	2\\
11.0725	3\\
11.0780555555556	3\\
11.0836111111111	3\\
11.0891666666667	3\\
11.0947222222222	3\\
11.1002777777778	3\\
11.1058333333333	3\\
11.1113888888889	2\\
11.1169444444444	2\\
11.1225	2\\
11.1280555555556	2\\
11.1336111111111	2\\
11.1391666666667	2\\
11.1447222222222	2\\
11.1502777777778	2\\
11.1558333333333	2\\
11.1613888888889	2\\
11.1669444444444	2\\
11.1725	2\\
11.1780555555556	2\\
11.1836111111111	2\\
11.1891666666667	3\\
11.1947222222222	3\\
11.2002777777778	2\\
11.2058333333333	3\\
11.2113888888889	4\\
11.2169444444444	3\\
11.2225	3\\
11.2280555555556	3\\
11.2336111111111	3\\
11.2391666666667	2\\
11.2447222222222	2\\
11.2502777777778	3\\
11.2558333333333	2\\
11.2613888888889	2\\
11.2669444444444	3\\
11.2725	3\\
11.2780555555556	2\\
11.2836111111111	2\\
11.2891666666667	2\\
11.2947222222222	2\\
11.3002777777778	2\\
11.3058333333333	2\\
11.3113888888889	2\\
11.3169444444444	2\\
11.3225	2\\
11.3280555555556	2\\
11.3336111111111	2\\
11.3391666666667	2\\
11.3447222222222	2\\
11.3502777777778	3\\
11.3558333333333	3\\
11.3613888888889	4\\
11.3669444444444	2\\
11.3725	3\\
11.3780555555556	3\\
11.3836111111111	3\\
11.3891666666667	3\\
11.3947222222222	3\\
11.4002777777778	3\\
11.4058333333333	6\\
11.4113888888889	6\\
11.4169444444444	8\\
11.4225	2\\
11.4280555555556	2\\
11.4336111111111	2\\
11.4391666666667	2\\
11.4447222222222	2\\
11.4502777777778	2\\
11.4558333333333	3\\
11.4613888888889	4\\
11.4669444444444	4\\
11.4725	5\\
11.4780555555556	5\\
11.4836111111111	4\\
11.4891666666667	4\\
11.4947222222222	4\\
11.5002777777778	4\\
11.5058333333333	4\\
11.5113888888889	2\\
11.5169444444444	3\\
11.5225	3\\
11.5280555555556	3\\
11.5336111111111	3\\
11.5391666666667	3\\
11.5447222222222	2\\
11.5502777777778	3\\
11.5558333333333	3\\
11.5613888888889	3\\
11.5669444444444	2\\
11.5725	2\\
11.5780555555556	2\\
11.5836111111111	2\\
11.5891666666667	3\\
11.5947222222222	3\\
11.6002777777778	4\\
11.6058333333333	5\\
11.6113888888889	5\\
11.6169444444444	5\\
11.6225	3\\
11.6280555555556	2\\
11.6336111111111	2\\
11.6391666666667	2\\
11.6447222222222	4\\
11.6502777777778	5\\
11.6558333333333	5\\
11.6613888888889	5\\
11.6669444444444	5\\
11.6725	7\\
11.6780555555556	8\\
11.6836111111111	8\\
11.6891666666667	3\\
11.6947222222222	3\\
11.7002777777778	3\\
11.7058333333333	3\\
11.7113888888889	3\\
11.7169444444444	3\\
11.7225	3\\
11.7280555555556	3\\
11.7336111111111	4\\
11.7391666666667	5\\
11.7447222222222	6\\
11.7502777777778	9\\
11.7558333333333	12\\
11.7613888888889	12\\
11.7669444444444	12\\
11.7725	13\\
11.7780555555556	14\\
11.7836111111111	15\\
11.7891666666667	3\\
11.7947222222222	3\\
11.8002777777778	3\\
11.8058333333333	3\\
11.8113888888889	3\\
11.8169444444444	3\\
11.8225	3\\
11.8280555555556	3\\
11.8336111111111	3\\
11.8391666666667	4\\
11.8447222222222	3\\
11.8502777777778	3\\
11.8558333333333	4\\
11.8613888888889	3\\
11.8669444444444	4\\
11.8725	4\\
11.8780555555556	5\\
11.8836111111111	6\\
11.8891666666667	8\\
11.8947222222222	7\\
11.9002777777778	8\\
11.9058333333333	10\\
11.9113888888889	11\\
11.9169444444444	11\\
11.9225	12\\
11.9280555555556	12\\
11.9336111111111	12\\
11.9391666666667	3\\
11.9447222222222	3\\
11.9502777777778	3\\
11.9558333333333	3\\
11.9613888888889	2\\
11.9669444444444	3\\
11.9725	3\\
11.9780555555556	4\\
11.9836111111111	3\\
11.9891666666667	3\\
11.9947222222222	3\\
12.0002777777778	3\\
12.0058333333333	5\\
12.0113888888889	7\\
12.0169444444444	8\\
12.0225	8\\
12.0280555555556	9\\
12.0336111111111	9\\
12.0391666666667	8\\
12.0447222222222	7\\
12.0502777777778	7\\
12.0558333333333	4\\
12.0613888888889	5\\
12.0669444444444	9\\
12.0725	9\\
12.0780555555556	10\\
12.0836111111111	9\\
12.0891666666667	8\\
12.0947222222222	9\\
12.1002777777778	9\\
12.1058333333333	9\\
12.1113888888889	4\\
12.1169444444444	4\\
12.1225	5\\
12.1280555555556	5\\
12.1336111111111	5\\
12.1391666666667	4\\
12.1447222222222	3\\
12.1502777777778	3\\
12.1558333333333	4\\
12.1613888888889	4\\
12.1669444444444	4\\
12.1725	4\\
12.1780555555556	4\\
12.1836111111111	4\\
12.1891666666667	3\\
12.1947222222222	3\\
12.2002777777778	3\\
12.2058333333333	3\\
12.2113888888889	3\\
12.2169444444444	3\\
12.2225	3\\
12.2280555555556	4\\
12.2336111111111	4\\
12.2391666666667	4\\
12.2447222222222	2\\
12.2502777777778	2\\
12.2558333333333	4\\
12.2613888888889	4\\
12.2669444444444	5\\
12.2725	5\\
12.2780555555556	5\\
12.2836111111111	4\\
12.2891666666667	4\\
12.2947222222222	4\\
12.3002777777778	4\\
12.3058333333333	3\\
12.3113888888889	3\\
12.3169444444444	3\\
12.3225	3\\
12.3280555555556	3\\
12.3336111111111	2\\
12.3391666666667	3\\
12.3447222222222	3\\
12.3502777777778	4\\
12.3558333333333	4\\
12.3613888888889	4\\
12.3669444444444	4\\
12.3725	4\\
12.3780555555556	4\\
12.3836111111111	4\\
12.3891666666667	4\\
12.3947222222222	4\\
12.4002777777778	4\\
12.4058333333333	4\\
12.4113888888889	4\\
12.4169444444444	4\\
12.4225	8\\
12.4280555555556	8\\
12.4336111111111	9\\
12.4391666666667	10\\
12.4447222222222	8\\
12.4502777777778	8\\
12.4558333333333	8\\
12.4613888888889	8\\
12.4669444444444	8\\
12.4725	5\\
12.4780555555556	5\\
12.4836111111111	5\\
12.4891666666667	6\\
12.4947222222222	6\\
12.5002777777778	7\\
12.5058333333333	7\\
12.5113888888889	7\\
12.5169444444444	7\\
12.5225	6\\
12.5280555555556	5\\
12.5336111111111	4\\
12.5391666666667	3\\
12.5447222222222	3\\
12.5502777777778	3\\
12.5558333333333	4\\
12.5613888888889	4\\
12.5669444444444	4\\
12.5725	4\\
12.5780555555556	4\\
12.5836111111111	4\\
12.5891666666667	4\\
12.5947222222222	4\\
12.6002777777778	4\\
12.6058333333333	4\\
12.6113888888889	3\\
12.6169444444444	3\\
12.6225	4\\
12.6280555555556	3\\
12.6336111111111	3\\
12.6391666666667	3\\
12.6447222222222	3\\
12.6502777777778	4\\
12.6558333333333	4\\
12.6613888888889	4\\
12.6669444444444	3\\
12.6725	2\\
12.6780555555556	2\\
12.6836111111111	3\\
12.6891666666667	3\\
12.6947222222222	3\\
12.7002777777778	4\\
12.7058333333333	4\\
12.7113888888889	4\\
12.7169444444444	4\\
12.7225	4\\
12.7280555555556	4\\
12.7336111111111	4\\
12.7391666666667	4\\
12.7447222222222	4\\
12.7502777777778	4\\
12.7558333333333	4\\
12.7613888888889	4\\
12.7669444444444	4\\
12.7725	4\\
12.7780555555556	5\\
12.7836111111111	5\\
12.7891666666667	5\\
12.7947222222222	4\\
12.8002777777778	4\\
12.8058333333333	4\\
12.8113888888889	4\\
12.8169444444444	4\\
12.8225	4\\
12.8280555555556	4\\
12.8336111111111	4\\
12.8391666666667	4\\
12.8447222222222	3\\
12.8502777777778	2\\
12.8558333333333	2\\
12.8613888888889	2\\
12.8669444444444	2\\
12.8725	2\\
12.8780555555556	2\\
12.8836111111111	4\\
12.8891666666667	4\\
12.8947222222222	5\\
12.9002777777778	3\\
12.9058333333333	2\\
12.9113888888889	2\\
12.9169444444444	2\\
12.9225	2\\
12.9280555555556	2\\
12.9336111111111	3\\
12.9391666666667	4\\
12.9447222222222	4\\
12.9502777777778	4\\
12.9558333333333	5\\
12.9613888888889	4\\
12.9669444444444	4\\
12.9725	4\\
12.9780555555556	4\\
12.9836111111111	3\\
12.9891666666667	3\\
12.9947222222222	3\\
13.0002777777778	4\\
13.0058333333333	4\\
13.0113888888889	4\\
13.0169444444444	4\\
13.0225	4\\
13.0280555555556	6\\
13.0336111111111	7\\
13.0391666666667	7\\
13.0447222222222	7\\
13.0502777777778	5\\
13.0558333333333	5\\
13.0613888888889	6\\
13.0669444444444	6\\
13.0725	6\\
13.0780555555556	5\\
13.0836111111111	5\\
13.0891666666667	5\\
13.0947222222222	4\\
13.1002777777778	4\\
13.1058333333333	3\\
13.1113888888889	3\\
13.1169444444444	4\\
13.1225	4\\
13.1280555555556	6\\
13.1336111111111	6\\
13.1391666666667	4\\
13.1447222222222	4\\
13.1502777777778	4\\
13.1558333333333	5\\
13.1613888888889	3\\
13.1669444444444	2\\
13.1725	2\\
13.1780555555556	2\\
13.1836111111111	2\\
13.1891666666667	2\\
13.1947222222222	2\\
13.2002777777778	2\\
13.2058333333333	2\\
13.2113888888889	3\\
13.2169444444444	2\\
13.2225	2\\
13.2280555555556	3\\
13.2336111111111	4\\
13.2391666666667	4\\
13.2447222222222	5\\
13.2502777777778	5\\
13.2558333333333	4\\
13.2613888888889	5\\
13.2669444444444	5\\
13.2725	5\\
13.2780555555556	6\\
13.2836111111111	4\\
13.2891666666667	4\\
13.2947222222222	4\\
13.3002777777778	4\\
13.3058333333333	4\\
13.3113888888889	6\\
13.3169444444444	3\\
13.3225	3\\
13.3280555555556	6\\
13.3336111111111	3\\
13.3391666666667	3\\
13.3447222222222	5\\
13.3502777777778	5\\
13.3558333333333	5\\
13.3613888888889	5\\
13.3669444444444	5\\
13.3725	5\\
13.3780555555556	2\\
13.3836111111111	5\\
13.3891666666667	5\\
13.3947222222222	6\\
13.4002777777778	11\\
13.4058333333333	15\\
13.4113888888889	18\\
13.4169444444444	19\\
13.4225	19\\
13.4280555555556	18\\
13.4336111111111	17\\
13.4391666666667	17\\
13.4447222222222	17\\
13.4502777777778	17\\
13.4558333333333	19\\
13.4613888888889	19\\
13.4669444444444	19\\
13.4725	5\\
13.4780555555556	5\\
13.4836111111111	5\\
13.4891666666667	6\\
13.4947222222222	6\\
13.5002777777778	5\\
13.5058333333333	6\\
13.5113888888889	4\\
13.5169444444444	4\\
13.5225	3\\
13.5280555555556	3\\
13.5336111111111	3\\
13.5391666666667	3\\
13.5447222222222	3\\
13.5502777777778	2\\
13.5558333333333	2\\
13.5613888888889	2\\
13.5669444444444	2\\
13.5725	3\\
13.5780555555556	3\\
13.5836111111111	3\\
13.5891666666667	3\\
13.5947222222222	3\\
13.6002777777778	2\\
13.6058333333333	2\\
13.6113888888889	2\\
13.6169444444444	2\\
13.6225	2\\
13.6280555555556	2\\
13.6336111111111	2\\
13.6391666666667	4\\
13.6447222222222	4\\
13.6502777777778	4\\
13.6558333333333	4\\
13.6613888888889	4\\
13.6669444444444	4\\
13.6725	4\\
13.6780555555556	4\\
13.6836111111111	4\\
13.6891666666667	3\\
13.6947222222222	2\\
13.7002777777778	3\\
13.7058333333333	3\\
13.7113888888889	3\\
13.7169444444444	4\\
13.7225	6\\
13.7280555555556	5\\
13.7336111111111	3\\
13.7391666666667	4\\
13.7447222222222	5\\
13.7502777777778	3\\
13.7558333333333	3\\
13.7613888888889	3\\
13.7669444444444	3\\
13.7725	3\\
13.7780555555556	2\\
13.7836111111111	2\\
13.7891666666667	2\\
13.7947222222222	2\\
13.8002777777778	3\\
13.8058333333333	4\\
13.8113888888889	4\\
13.8169444444444	4\\
13.8225	4\\
13.8280555555556	4\\
13.8336111111111	5\\
13.8391666666667	5\\
13.8447222222222	5\\
13.8502777777778	5\\
13.8558333333333	4\\
13.8613888888889	4\\
13.8669444444444	3\\
13.8725	4\\
13.8780555555556	4\\
13.8836111111111	4\\
13.8891666666667	3\\
13.8947222222222	3\\
13.9002777777778	3\\
13.9058333333333	3\\
13.9113888888889	3\\
13.9169444444444	4\\
13.9225	5\\
13.9280555555556	6\\
13.9336111111111	6\\
13.9391666666667	7\\
13.9447222222222	7\\
13.9502777777778	7\\
13.9558333333333	7\\
13.9613888888889	7\\
13.9669444444444	4\\
13.9725	3\\
13.9780555555556	3\\
13.9836111111111	5\\
13.9891666666667	5\\
13.9947222222222	6\\
14.0002777777778	8\\
14.0058333333333	8\\
14.0113888888889	7\\
14.0169444444444	9\\
14.0225	10\\
14.0280555555556	12\\
14.0336111111111	13\\
14.0391666666667	13\\
14.0447222222222	14\\
14.0502777777778	17\\
14.0558333333333	19\\
14.0613888888889	20\\
14.0669444444444	20\\
14.0725	20\\
14.0780555555556	20\\
14.0836111111111	20\\
14.0891666666667	20\\
14.0947222222222	20\\
14.1002777777778	20\\
14.1058333333333	20\\
14.1113888888889	20\\
14.1169444444444	20\\
14.1225	4\\
14.1280555555556	4\\
14.1336111111111	5\\
14.1391666666667	3\\
14.1447222222222	4\\
14.1502777777778	4\\
14.1558333333333	7\\
14.1613888888889	10\\
14.1669444444444	13\\
14.1725	14\\
14.1780555555556	15\\
14.1836111111111	15\\
14.1891666666667	15\\
14.1947222222222	16\\
14.2002777777778	17\\
14.2058333333333	17\\
14.2113888888889	17\\
14.2169444444444	17\\
14.2225	18\\
14.2280555555556	19\\
14.2336111111111	20\\
14.2391666666667	20\\
14.2447222222222	20\\
14.2502777777778	20\\
14.2558333333333	20\\
14.2613888888889	20\\
14.2669444444444	20\\
14.2725	20\\
14.2780555555556	20\\
14.2836111111111	3\\
14.2891666666667	3\\
14.2947222222222	5\\
14.3002777777778	7\\
14.3058333333333	8\\
14.3113888888889	9\\
14.3169444444444	12\\
14.3225	12\\
14.3280555555556	15\\
14.3336111111111	19\\
14.3391666666667	19\\
14.3447222222222	20\\
14.3502777777778	20\\
14.3558333333333	20\\
14.3613888888889	3\\
14.3669444444444	3\\
14.3725	4\\
14.3780555555556	3\\
14.3836111111111	2\\
14.3891666666667	2\\
14.3947222222222	2\\
14.4002777777778	3\\
14.4058333333333	2\\
14.4113888888889	2\\
14.4169444444444	3\\
14.4225	3\\
14.4280555555556	3\\
14.4336111111111	3\\
14.4391666666667	2\\
14.4447222222222	2\\
14.4502777777778	2\\
14.4558333333333	2\\
14.4613888888889	2\\
14.4669444444444	3\\
14.4725	3\\
14.4780555555556	3\\
14.4836111111111	3\\
14.4891666666667	3\\
14.4947222222222	3\\
14.5002777777778	3\\
14.5058333333333	2\\
14.5113888888889	2\\
14.5169444444444	2\\
14.5225	2\\
14.5280555555556	2\\
14.5336111111111	2\\
14.5391666666667	2\\
14.5447222222222	2\\
14.5502777777778	3\\
14.5558333333333	3\\
14.5613888888889	3\\
14.5669444444444	2\\
14.5725	2\\
14.5780555555556	4\\
14.5836111111111	3\\
14.5891666666667	3\\
14.5947222222222	4\\
14.6002777777778	4\\
14.6058333333333	6\\
14.6113888888889	6\\
14.6169444444444	6\\
14.6225	5\\
14.6280555555556	5\\
14.6336111111111	4\\
14.6391666666667	4\\
14.6447222222222	4\\
14.6502777777778	3\\
14.6558333333333	2\\
14.6613888888889	2\\
14.6669444444444	2\\
14.6725	2\\
14.6780555555556	3\\
14.6836111111111	2\\
14.6891666666667	2\\
14.6947222222222	4\\
14.7002777777778	4\\
14.7058333333333	4\\
14.7113888888889	4\\
14.7169444444444	4\\
14.7225	3\\
14.7280555555556	2\\
14.7336111111111	2\\
14.7391666666667	4\\
14.7447222222222	3\\
14.7502777777778	2\\
14.7558333333333	3\\
14.7613888888889	2\\
14.7669444444444	2\\
14.7725	2\\
14.7780555555556	2\\
14.7836111111111	3\\
14.7891666666667	3\\
14.7947222222222	2\\
14.8002777777778	2\\
14.8058333333333	2\\
14.8113888888889	2\\
14.8169444444444	3\\
14.8225	4\\
14.8280555555556	6\\
14.8336111111111	5\\
14.8391666666667	6\\
14.8447222222222	4\\
14.8502777777778	5\\
14.8558333333333	5\\
14.8613888888889	5\\
14.8669444444444	5\\
14.8725	5\\
14.8780555555556	5\\
14.8836111111111	5\\
14.8891666666667	5\\
14.8947222222222	5\\
14.9002777777778	2\\
14.9058333333333	2\\
14.9113888888889	4\\
14.9169444444444	5\\
14.9225	5\\
14.9280555555556	5\\
14.9336111111111	4\\
14.9391666666667	4\\
14.9447222222222	4\\
14.9502777777778	4\\
14.9558333333333	3\\
14.9613888888889	3\\
14.9669444444444	2\\
14.9725	2\\
14.9780555555556	2\\
14.9836111111111	2\\
14.9891666666667	2\\
14.9947222222222	3\\
15.0002777777778	3\\
15.0058333333333	3\\
15.0113888888889	3\\
15.0169444444444	3\\
15.0225	3\\
15.0280555555556	3\\
15.0336111111111	3\\
15.0391666666667	3\\
15.0447222222222	2\\
15.0502777777778	2\\
15.0558333333333	2\\
15.0613888888889	3\\
15.0669444444444	4\\
15.0725	5\\
15.0780555555556	5\\
15.0836111111111	6\\
15.0891666666667	2\\
15.0947222222222	2\\
15.1002777777778	6\\
15.1058333333333	6\\
15.1113888888889	4\\
15.1169444444444	4\\
15.1225	3\\
15.1280555555556	4\\
15.1336111111111	2\\
15.1391666666667	2\\
15.1447222222222	4\\
15.1502777777778	4\\
15.1558333333333	3\\
15.1613888888889	4\\
15.1669444444444	3\\
15.1725	6\\
15.1780555555556	8\\
15.1836111111111	8\\
15.1891666666667	6\\
15.1947222222222	3\\
15.2002777777778	4\\
15.2058333333333	4\\
15.2113888888889	3\\
15.2169444444444	4\\
15.2225	3\\
15.2280555555556	3\\
15.2336111111111	6\\
15.2391666666667	6\\
15.2447222222222	3\\
15.2502777777778	3\\
15.2558333333333	2\\
15.2613888888889	3\\
15.2669444444444	3\\
15.2725	5\\
15.2780555555556	10\\
15.2836111111111	10\\
15.2891666666667	10\\
15.2947222222222	10\\
15.3002777777778	3\\
15.3058333333333	4\\
15.3113888888889	5\\
15.3169444444444	3\\
15.3225	2\\
15.3280555555556	3\\
15.3336111111111	3\\
15.3391666666667	2\\
15.3447222222222	2\\
15.3502777777778	2\\
15.3558333333333	4\\
15.3613888888889	4\\
15.3669444444444	4\\
15.3725	4\\
15.3780555555556	3\\
15.3836111111111	2\\
15.3891666666667	3\\
15.3947222222222	6\\
15.4002777777778	7\\
15.4058333333333	7\\
15.4113888888889	7\\
15.4169444444444	8\\
15.4225	9\\
15.4280555555556	8\\
15.4336111111111	7\\
15.4391666666667	5\\
15.4447222222222	3\\
15.4502777777778	3\\
15.4558333333333	3\\
15.4613888888889	2\\
15.4669444444444	2\\
15.4725	3\\
15.4780555555556	4\\
15.4836111111111	4\\
15.4891666666667	3\\
15.4947222222222	3\\
15.5002777777778	3\\
15.5058333333333	3\\
15.5113888888889	2\\
15.5169444444444	3\\
15.5225	4\\
15.5280555555556	4\\
15.5336111111111	3\\
15.5391666666667	3\\
15.5447222222222	4\\
15.5502777777778	4\\
15.5558333333333	3\\
15.5613888888889	3\\
15.5669444444444	3\\
15.5725	6\\
15.5780555555556	6\\
15.5836111111111	5\\
15.5891666666667	4\\
15.5947222222222	4\\
15.6002777777778	4\\
15.6058333333333	5\\
15.6113888888889	7\\
15.6169444444444	7\\
15.6225	7\\
15.6280555555556	6\\
15.6336111111111	4\\
15.6391666666667	4\\
15.6447222222222	2\\
15.6502777777778	3\\
15.6558333333333	3\\
15.6613888888889	4\\
15.6669444444444	2\\
15.6725	4\\
15.6780555555556	3\\
15.6836111111111	3\\
15.6891666666667	4\\
15.6947222222222	4\\
15.7002777777778	3\\
15.7058333333333	3\\
15.7113888888889	3\\
15.7169444444444	3\\
15.7225	2\\
15.7280555555556	2\\
15.7336111111111	3\\
15.7391666666667	3\\
15.7447222222222	4\\
15.7502777777778	3\\
15.7558333333333	4\\
15.7613888888889	7\\
15.7669444444444	7\\
15.7725	5\\
15.7780555555556	5\\
15.7836111111111	6\\
15.7891666666667	5\\
15.7947222222222	3\\
15.8002777777778	3\\
15.8058333333333	2\\
15.8113888888889	2\\
15.8169444444444	2\\
15.8225	2\\
15.8280555555556	3\\
15.8336111111111	2\\
15.8391666666667	2\\
15.8447222222222	2\\
15.8502777777778	2\\
15.8558333333333	2\\
15.8613888888889	3\\
15.8669444444444	3\\
15.8725	4\\
15.8780555555556	5\\
15.8836111111111	5\\
15.8891666666667	3\\
15.8947222222222	3\\
15.9002777777778	2\\
15.9058333333333	2\\
15.9113888888889	2\\
15.9169444444444	3\\
15.9225	2\\
15.9280555555556	2\\
15.9336111111111	4\\
15.9391666666667	2\\
15.9447222222222	4\\
15.9502777777778	2\\
15.9558333333333	2\\
15.9613888888889	2\\
15.9669444444444	3\\
15.9725	3\\
15.9780555555556	3\\
15.9836111111111	2\\
15.9891666666667	2\\
15.9947222222222	2\\
};
\addlegendentry{Dscr Stoch Ctrl};

\addplot [color=mycolor4,solid,line width=1.5pt]
  table[row sep=crcr]{%
9.50027777777778	-2\\
9.50583333333333	-2\\
9.51138888888889	-4\\
9.51694444444444	-3\\
9.5225	-2\\
9.52805555555556	-2\\
9.53361111111111	-3\\
9.53916666666667	-3\\
9.54472222222222	-2\\
9.55027777777778	-2\\
9.55583333333333	-5\\
9.56138888888889	-3\\
9.56694444444444	-3\\
9.5725	-2\\
9.57805555555555	-2\\
9.58361111111111	-3\\
9.58916666666667	-3\\
9.59472222222222	-5\\
9.60027777777778	-3\\
9.60583333333333	-3\\
9.61138888888889	-5\\
9.61694444444444	-3\\
9.6225	-3\\
9.62805555555556	-4\\
9.63361111111111	-3\\
9.63916666666667	-2\\
9.64472222222222	-3\\
9.65027777777778	-2\\
9.65583333333333	-2\\
9.66138888888889	-5\\
9.66694444444444	-6\\
9.6725	-5\\
9.67805555555555	-8\\
9.68361111111111	-9\\
9.68916666666667	-11\\
9.69472222222222	-5\\
9.70027777777778	-6\\
9.70583333333333	-7\\
9.71138888888889	-6\\
9.71694444444444	-5\\
9.7225	-9\\
9.72805555555555	-5\\
9.73361111111111	-3\\
9.73916666666667	-3\\
9.74472222222222	-3\\
9.75027777777778	-2\\
9.75583333333333	-3\\
9.76138888888889	-5\\
9.76694444444444	-8\\
9.7725	-4\\
9.77805555555556	-5\\
9.78361111111111	-6\\
9.78916666666667	-3\\
9.79472222222222	-3\\
9.80027777777778	-2\\
9.80583333333333	-2\\
9.81138888888889	-3\\
9.81694444444444	-5\\
9.8225	-3\\
9.82805555555555	-5\\
9.83361111111111	-4\\
9.83916666666667	-3\\
9.84472222222222	-2\\
9.85027777777778	-5\\
9.85583333333333	-5\\
9.86138888888889	-5\\
9.86694444444444	-6\\
9.8725	-3\\
9.87805555555556	-4\\
9.88361111111111	-2\\
9.88916666666667	-2\\
9.89472222222222	-3\\
9.90027777777778	-3\\
9.90583333333333	-4\\
9.91138888888889	-4\\
9.91694444444444	-3\\
9.9225	-4\\
9.92805555555555	-3\\
9.93361111111111	-4\\
9.93916666666667	-5\\
9.94472222222222	-4\\
9.95027777777778	-5\\
9.95583333333333	-3\\
9.96138888888889	-2\\
9.96694444444444	-5\\
9.9725	-5\\
9.97805555555555	-3\\
9.98361111111111	-3\\
9.98916666666667	-2\\
9.99472222222222	-3\\
10.0002777777778	-3\\
10.0058333333333	-3\\
10.0113888888889	-3\\
10.0169444444444	-4\\
10.0225	-6\\
10.0280555555556	-5\\
10.0336111111111	-7\\
10.0391666666667	-5\\
10.0447222222222	-8\\
10.0502777777778	-11\\
10.0558333333333	-14\\
10.0613888888889	-14\\
10.0669444444444	-5\\
10.0725	-5\\
10.0780555555556	-2\\
10.0836111111111	-2\\
10.0891666666667	-2\\
10.0947222222222	-2\\
10.1002777777778	-3\\
10.1058333333333	-5\\
10.1113888888889	-6\\
10.1169444444444	-5\\
10.1225	-5\\
10.1280555555556	-6\\
10.1336111111111	-6\\
10.1391666666667	-6\\
10.1447222222222	-4\\
10.1502777777778	-4\\
10.1558333333333	-4\\
10.1613888888889	-4\\
10.1669444444444	-4\\
10.1725	-5\\
10.1780555555556	-5\\
10.1836111111111	-4\\
10.1891666666667	-4\\
10.1947222222222	-4\\
10.2002777777778	-4\\
10.2058333333333	-4\\
10.2113888888889	-4\\
10.2169444444444	-4\\
10.2225	-3\\
10.2280555555556	-2\\
10.2336111111111	-3\\
10.2391666666667	-3\\
10.2447222222222	-3\\
10.2502777777778	-4\\
10.2558333333333	-4\\
10.2613888888889	-3\\
10.2669444444444	-4\\
10.2725	-3\\
10.2780555555556	-4\\
10.2836111111111	-4\\
10.2891666666667	-6\\
10.2947222222222	-6\\
10.3002777777778	-6\\
10.3058333333333	-6\\
10.3113888888889	-7\\
10.3169444444444	-6\\
10.3225	-6\\
10.3280555555556	-3\\
10.3336111111111	-3\\
10.3391666666667	-3\\
10.3447222222222	-3\\
10.3502777777778	-5\\
10.3558333333333	-4\\
10.3613888888889	-4\\
10.3669444444444	-4\\
10.3725	-4\\
10.3780555555556	-4\\
10.3836111111111	-4\\
10.3891666666667	-5\\
10.3947222222222	-5\\
10.4002777777778	-5\\
10.4058333333333	-4\\
10.4113888888889	-8\\
10.4169444444444	-8\\
10.4225	-8\\
10.4280555555556	-7\\
10.4336111111111	-6\\
10.4391666666667	-6\\
10.4447222222222	-6\\
10.4502777777778	-6\\
10.4558333333333	-6\\
10.4613888888889	-6\\
10.4669444444444	-7\\
10.4725	-7\\
10.4780555555556	-6\\
10.4836111111111	-6\\
10.4891666666667	-6\\
10.4947222222222	-7\\
10.5002777777778	-7\\
10.5058333333333	-9\\
10.5113888888889	-10\\
10.5169444444444	-11\\
10.5225	-11\\
10.5280555555556	-13\\
10.5336111111111	-17\\
10.5391666666667	-18\\
10.5447222222222	-20\\
10.5502777777778	-20\\
10.5558333333333	-17\\
10.5613888888889	-17\\
10.5669444444444	-7\\
10.5725	-7\\
10.5780555555556	-8\\
10.5836111111111	-6\\
10.5891666666667	-8\\
10.5947222222222	-9\\
10.6002777777778	-8\\
10.6058333333333	-8\\
10.6113888888889	-7\\
10.6169444444444	-7\\
10.6225	-7\\
10.6280555555556	-8\\
10.6336111111111	-8\\
10.6391666666667	-7\\
10.6447222222222	-7\\
10.6502777777778	-8\\
10.6558333333333	-8\\
10.6613888888889	-9\\
10.6669444444444	-10\\
10.6725	-9\\
10.6780555555556	-10\\
10.6836111111111	-12\\
10.6891666666667	-13\\
10.6947222222222	-13\\
10.7002777777778	-12\\
10.7058333333333	-13\\
10.7113888888889	-14\\
10.7169444444444	-14\\
10.7225	-14\\
10.7280555555556	-13\\
10.7336111111111	-12\\
10.7391666666667	-13\\
10.7447222222222	-11\\
10.7502777777778	-10\\
10.7558333333333	-10\\
10.7613888888889	-10\\
10.7669444444444	-11\\
10.7725	-12\\
10.7780555555556	-12\\
10.7836111111111	-12\\
10.7891666666667	-12\\
10.7947222222222	-13\\
10.8002777777778	-17\\
10.8058333333333	-19\\
10.8113888888889	-19\\
10.8169444444444	-18\\
10.8225	-18\\
10.8280555555556	-18\\
10.8336111111111	-18\\
10.8391666666667	-17\\
10.8447222222222	-6\\
10.8502777777778	-8\\
10.8558333333333	-5\\
10.8613888888889	-5\\
10.8669444444444	-6\\
10.8725	-8\\
10.8780555555556	-8\\
10.8836111111111	-8\\
10.8891666666667	-8\\
10.8947222222222	-8\\
10.9002777777778	-8\\
10.9058333333333	-8\\
10.9113888888889	-3\\
10.9169444444444	-3\\
10.9225	-2\\
10.9280555555556	-2\\
10.9336111111111	-2\\
10.9391666666667	-5\\
10.9447222222222	-2\\
10.9502777777778	-4\\
10.9558333333333	-4\\
10.9613888888889	-4\\
10.9669444444444	-2\\
10.9725	-2\\
10.9780555555556	-2\\
10.9836111111111	-2\\
10.9891666666667	-2\\
10.9947222222222	-3\\
11.0002777777778	-4\\
11.0058333333333	-4\\
11.0113888888889	-3\\
11.0169444444444	-4\\
11.0225	-5\\
11.0280555555556	-5\\
11.0336111111111	-4\\
11.0391666666667	-5\\
11.0447222222222	-6\\
11.0502777777778	-6\\
11.0558333333333	-8\\
11.0613888888889	-8\\
11.0669444444444	-8\\
11.0725	-9\\
11.0780555555556	-9\\
11.0836111111111	-8\\
11.0891666666667	-8\\
11.0947222222222	-8\\
11.1002777777778	-8\\
11.1058333333333	-9\\
11.1113888888889	-12\\
11.1169444444444	-14\\
11.1225	-14\\
11.1280555555556	-14\\
11.1336111111111	-15\\
11.1391666666667	-15\\
11.1447222222222	-16\\
11.1502777777778	-5\\
11.1558333333333	-5\\
11.1613888888889	-5\\
11.1669444444444	-6\\
11.1725	-7\\
11.1780555555556	-7\\
11.1836111111111	-8\\
11.1891666666667	-7\\
11.1947222222222	-7\\
11.2002777777778	-6\\
11.2058333333333	-5\\
11.2113888888889	-6\\
11.2169444444444	-7\\
11.2225	-7\\
11.2280555555556	-7\\
11.2336111111111	-7\\
11.2391666666667	-7\\
11.2447222222222	-8\\
11.2502777777778	-7\\
11.2558333333333	-3\\
11.2613888888889	-3\\
11.2669444444444	-4\\
11.2725	-4\\
11.2780555555556	-7\\
11.2836111111111	-7\\
11.2891666666667	-7\\
11.2947222222222	-6\\
11.3002777777778	-7\\
11.3058333333333	-7\\
11.3113888888889	-7\\
11.3169444444444	-8\\
11.3225	-8\\
11.3280555555556	-8\\
11.3336111111111	-9\\
11.3391666666667	-9\\
11.3447222222222	-9\\
11.3502777777778	-4\\
11.3558333333333	-4\\
11.3613888888889	-3\\
11.3669444444444	-3\\
11.3725	-4\\
11.3780555555556	-3\\
11.3836111111111	-3\\
11.3891666666667	-4\\
11.3947222222222	-4\\
11.4002777777778	-4\\
11.4058333333333	-2\\
11.4113888888889	-2\\
11.4169444444444	-2\\
11.4225	-5\\
11.4280555555556	-5\\
11.4336111111111	-7\\
11.4391666666667	-7\\
11.4447222222222	-7\\
11.4502777777778	-7\\
11.4558333333333	-6\\
11.4613888888889	-5\\
11.4669444444444	-5\\
11.4725	-4\\
11.4780555555556	-4\\
11.4836111111111	-5\\
11.4891666666667	-5\\
11.4947222222222	-5\\
11.5002777777778	-5\\
11.5058333333333	-5\\
11.5113888888889	-3\\
11.5169444444444	-2\\
11.5225	-2\\
11.5280555555556	-2\\
11.5336111111111	-2\\
11.5391666666667	-2\\
11.5447222222222	-3\\
11.5502777777778	-2\\
11.5558333333333	-2\\
11.5613888888889	-6\\
11.5669444444444	-9\\
11.5725	-5\\
11.5780555555556	-5\\
11.5836111111111	-5\\
11.5891666666667	-4\\
11.5947222222222	-5\\
11.6002777777778	-4\\
11.6058333333333	-3\\
11.6113888888889	-3\\
11.6169444444444	-3\\
11.6225	-2\\
11.6280555555556	-3\\
11.6336111111111	-3\\
11.6391666666667	-4\\
11.6447222222222	-3\\
11.6502777777778	-2\\
11.6558333333333	-2\\
11.6613888888889	-3\\
11.6669444444444	-3\\
11.6725	-2\\
11.6780555555556	-2\\
11.6836111111111	-2\\
11.6891666666667	-3\\
11.6947222222222	-3\\
11.7002777777778	-3\\
11.7058333333333	-3\\
11.7113888888889	-3\\
11.7169444444444	-3\\
11.7225	-3\\
11.7280555555556	-3\\
11.7336111111111	-2\\
11.7391666666667	-2\\
11.7447222222222	-2\\
11.7502777777778	-2\\
11.7558333333333	-2\\
11.7613888888889	-2\\
11.7669444444444	-2\\
11.7725	-2\\
11.7780555555556	-2\\
11.7836111111111	-2\\
11.7891666666667	-2\\
11.7947222222222	-3\\
11.8002777777778	-3\\
11.8058333333333	-3\\
11.8113888888889	-3\\
11.8169444444444	-3\\
11.8225	-3\\
11.8280555555556	-3\\
11.8336111111111	-3\\
11.8391666666667	-2\\
11.8447222222222	-2\\
11.8502777777778	-3\\
11.8558333333333	-2\\
11.8613888888889	-3\\
11.8669444444444	-2\\
11.8725	-3\\
11.8780555555556	-2\\
11.8836111111111	-2\\
11.8891666666667	-2\\
11.8947222222222	-3\\
11.9002777777778	-3\\
11.9058333333333	-3\\
11.9113888888889	-3\\
11.9169444444444	-3\\
11.9225	-2\\
11.9280555555556	-2\\
11.9336111111111	-2\\
11.9391666666667	-2\\
11.9447222222222	-2\\
11.9502777777778	-2\\
11.9558333333333	-2\\
11.9613888888889	-3\\
11.9669444444444	-4\\
11.9725	-4\\
11.9780555555556	-3\\
11.9836111111111	-5\\
11.9891666666667	-5\\
11.9947222222222	-5\\
12.0002777777778	-5\\
12.0058333333333	-3\\
12.0113888888889	-2\\
12.0169444444444	-2\\
12.0225	-2\\
12.0280555555556	-2\\
12.0336111111111	-2\\
12.0391666666667	-3\\
12.0447222222222	-4\\
12.0502777777778	-4\\
12.0558333333333	-3\\
12.0613888888889	-2\\
12.0669444444444	-2\\
12.0725	-2\\
12.0780555555556	-2\\
12.0836111111111	-3\\
12.0891666666667	-4\\
12.0947222222222	-4\\
12.1002777777778	-4\\
12.1058333333333	-5\\
12.1113888888889	-5\\
12.1169444444444	-4\\
12.1225	-3\\
12.1280555555556	-3\\
12.1336111111111	-3\\
12.1391666666667	-5\\
12.1447222222222	-5\\
12.1502777777778	-5\\
12.1558333333333	-5\\
12.1613888888889	-5\\
12.1669444444444	-5\\
12.1725	-5\\
12.1780555555556	-5\\
12.1836111111111	-5\\
12.1891666666667	-6\\
12.1947222222222	-5\\
12.2002777777778	-5\\
12.2058333333333	-5\\
12.2113888888889	-5\\
12.2169444444444	-5\\
12.2225	-5\\
12.2280555555556	-5\\
12.2336111111111	-5\\
12.2391666666667	-5\\
12.2447222222222	-6\\
12.2502777777778	-5\\
12.2558333333333	-4\\
12.2613888888889	-4\\
12.2669444444444	-4\\
12.2725	-4\\
12.2780555555556	-4\\
12.2836111111111	-3\\
12.2891666666667	-2\\
12.2947222222222	-2\\
12.3002777777778	-2\\
12.3058333333333	-3\\
12.3113888888889	-3\\
12.3169444444444	-3\\
12.3225	-3\\
12.3280555555556	-3\\
12.3336111111111	-3\\
12.3391666666667	-2\\
12.3447222222222	-2\\
12.3502777777778	-2\\
12.3558333333333	-2\\
12.3613888888889	-2\\
12.3669444444444	-3\\
12.3725	-4\\
12.3780555555556	-4\\
12.3836111111111	-4\\
12.3891666666667	-5\\
12.3947222222222	-5\\
12.4002777777778	-5\\
12.4058333333333	-5\\
12.4113888888889	-5\\
12.4169444444444	-5\\
12.4225	-3\\
12.4280555555556	-3\\
12.4336111111111	-3\\
12.4391666666667	-5\\
12.4447222222222	-7\\
12.4502777777778	-7\\
12.4558333333333	-7\\
12.4613888888889	-7\\
12.4669444444444	-7\\
12.4725	-9\\
12.4780555555556	-9\\
12.4836111111111	-9\\
12.4891666666667	-10\\
12.4947222222222	-10\\
12.5002777777778	-4\\
12.5058333333333	-5\\
12.5113888888889	-5\\
12.5169444444444	-5\\
12.5225	-5\\
12.5280555555556	-6\\
12.5336111111111	-6\\
12.5391666666667	-6\\
12.5447222222222	-6\\
12.5502777777778	-6\\
12.5558333333333	-5\\
12.5613888888889	-5\\
12.5669444444444	-5\\
12.5725	-4\\
12.5780555555556	-4\\
12.5836111111111	-5\\
12.5891666666667	-5\\
12.5947222222222	-4\\
12.6002777777778	-4\\
12.6058333333333	-4\\
12.6113888888889	-5\\
12.6169444444444	-4\\
12.6225	-4\\
12.6280555555556	-3\\
12.6336111111111	-3\\
12.6391666666667	-3\\
12.6447222222222	-3\\
12.6502777777778	-4\\
12.6558333333333	-4\\
12.6613888888889	-4\\
12.6669444444444	-5\\
12.6725	-5\\
12.6780555555556	-5\\
12.6836111111111	-5\\
12.6891666666667	-5\\
12.6947222222222	-5\\
12.7002777777778	-4\\
12.7058333333333	-4\\
12.7113888888889	-4\\
12.7169444444444	-4\\
12.7225	-4\\
12.7280555555556	-4\\
12.7336111111111	-4\\
12.7391666666667	-4\\
12.7447222222222	-5\\
12.7502777777778	-5\\
12.7558333333333	-5\\
12.7613888888889	-5\\
12.7669444444444	-5\\
12.7725	-5\\
12.7780555555556	-6\\
12.7836111111111	-6\\
12.7891666666667	-6\\
12.7947222222222	-7\\
12.8002777777778	-7\\
12.8058333333333	-7\\
12.8113888888889	-7\\
12.8169444444444	-7\\
12.8225	-7\\
12.8280555555556	-7\\
12.8336111111111	-7\\
12.8391666666667	-7\\
12.8447222222222	-8\\
12.8502777777778	-9\\
12.8558333333333	-3\\
12.8613888888889	-4\\
12.8669444444444	-4\\
12.8725	-4\\
12.8780555555556	-5\\
12.8836111111111	-4\\
12.8891666666667	-4\\
12.8947222222222	-4\\
12.9002777777778	-4\\
12.9058333333333	-4\\
12.9113888888889	-5\\
12.9169444444444	-5\\
12.9225	-5\\
12.9280555555556	-5\\
12.9336111111111	-5\\
12.9391666666667	-4\\
12.9447222222222	-4\\
12.9502777777778	-4\\
12.9558333333333	-3\\
12.9613888888889	-4\\
12.9669444444444	-4\\
12.9725	-4\\
12.9780555555556	-3\\
12.9836111111111	-4\\
12.9891666666667	-4\\
12.9947222222222	-4\\
13.0002777777778	-3\\
13.0058333333333	-3\\
13.0113888888889	-3\\
13.0169444444444	-3\\
13.0225	-3\\
13.0280555555556	-2\\
13.0336111111111	-2\\
13.0391666666667	-2\\
13.0447222222222	-2\\
13.0502777777778	-4\\
13.0558333333333	-4\\
13.0613888888889	-4\\
13.0669444444444	-4\\
13.0725	-4\\
13.0780555555556	-6\\
13.0836111111111	-6\\
13.0891666666667	-6\\
13.0947222222222	-7\\
13.1002777777778	-7\\
13.1058333333333	-6\\
13.1113888888889	-6\\
13.1169444444444	-5\\
13.1225	-5\\
13.1280555555556	-5\\
13.1336111111111	-5\\
13.1391666666667	-4\\
13.1447222222222	-4\\
13.1502777777778	-3\\
13.1558333333333	-3\\
13.1613888888889	-3\\
13.1669444444444	-5\\
13.1725	-5\\
13.1780555555556	-5\\
13.1836111111111	-5\\
13.1891666666667	-5\\
13.1947222222222	-5\\
13.2002777777778	-5\\
13.2058333333333	-5\\
13.2113888888889	-5\\
13.2169444444444	-6\\
13.2225	-6\\
13.2280555555556	-6\\
13.2336111111111	-5\\
13.2391666666667	-4\\
13.2447222222222	-3\\
13.2502777777778	-3\\
13.2558333333333	-3\\
13.2613888888889	-3\\
13.2669444444444	-3\\
13.2725	-2\\
13.2780555555556	-2\\
13.2836111111111	-2\\
13.2891666666667	-3\\
13.2947222222222	-2\\
13.3002777777778	-2\\
13.3058333333333	-3\\
13.3113888888889	-4\\
13.3169444444444	-5\\
13.3225	-6\\
13.3280555555556	-5\\
13.3336111111111	-7\\
13.3391666666667	-5\\
13.3447222222222	-3\\
13.3502777777778	-2\\
13.3558333333333	-2\\
13.3613888888889	-2\\
13.3669444444444	-2\\
13.3725	-2\\
13.3780555555556	-3\\
13.3836111111111	-2\\
13.3891666666667	-2\\
13.3947222222222	-2\\
13.4002777777778	-2\\
13.4058333333333	-2\\
13.4113888888889	-2\\
13.4169444444444	-3\\
13.4225	-3\\
13.4280555555556	-4\\
13.4336111111111	-5\\
13.4391666666667	-5\\
13.4447222222222	-5\\
13.4502777777778	-5\\
13.4558333333333	-3\\
13.4613888888889	-3\\
13.4669444444444	-5\\
13.4725	-3\\
13.4780555555556	-3\\
13.4836111111111	-3\\
13.4891666666667	-3\\
13.4947222222222	-3\\
13.5002777777778	-2\\
13.5058333333333	-2\\
13.5113888888889	-3\\
13.5169444444444	-2\\
13.5225	-3\\
13.5280555555556	-3\\
13.5336111111111	-6\\
13.5391666666667	-6\\
13.5447222222222	-6\\
13.5502777777778	-9\\
13.5558333333333	-11\\
13.5613888888889	-12\\
13.5669444444444	-14\\
13.5725	-3\\
13.5780555555556	-4\\
13.5836111111111	-4\\
13.5891666666667	-4\\
13.5947222222222	-4\\
13.6002777777778	-5\\
13.6058333333333	-5\\
13.6113888888889	-6\\
13.6169444444444	-6\\
13.6225	-6\\
13.6280555555556	-7\\
13.6336111111111	-7\\
13.6391666666667	-6\\
13.6447222222222	-6\\
13.6502777777778	-6\\
13.6558333333333	-6\\
13.6613888888889	-6\\
13.6669444444444	-6\\
13.6725	-6\\
13.6780555555556	-6\\
13.6836111111111	-6\\
13.6891666666667	-7\\
13.6947222222222	-3\\
13.7002777777778	-4\\
13.7058333333333	-4\\
13.7113888888889	-4\\
13.7169444444444	-3\\
13.7225	-3\\
13.7280555555556	-5\\
13.7336111111111	-5\\
13.7391666666667	-4\\
13.7447222222222	-3\\
13.7502777777778	-2\\
13.7558333333333	-2\\
13.7613888888889	-2\\
13.7669444444444	-2\\
13.7725	-2\\
13.7780555555556	-3\\
13.7836111111111	-3\\
13.7891666666667	-4\\
13.7947222222222	-5\\
13.8002777777778	-5\\
13.8058333333333	-4\\
13.8113888888889	-4\\
13.8169444444444	-4\\
13.8225	-4\\
13.8280555555556	-4\\
13.8336111111111	-3\\
13.8391666666667	-3\\
13.8447222222222	-3\\
13.8502777777778	-3\\
13.8558333333333	-5\\
13.8613888888889	-5\\
13.8669444444444	-6\\
13.8725	-5\\
13.8780555555556	-5\\
13.8836111111111	-5\\
13.8891666666667	-3\\
13.8947222222222	-3\\
13.9002777777778	-3\\
13.9058333333333	-3\\
13.9113888888889	-3\\
13.9169444444444	-2\\
13.9225	-2\\
13.9280555555556	-2\\
13.9336111111111	-2\\
13.9391666666667	-2\\
13.9447222222222	-2\\
13.9502777777778	-3\\
13.9558333333333	-3\\
13.9613888888889	-3\\
13.9669444444444	-2\\
13.9725	-3\\
13.9780555555556	-3\\
13.9836111111111	-4\\
13.9891666666667	-4\\
13.9947222222222	-4\\
14.0002777777778	-2\\
14.0058333333333	-2\\
14.0113888888889	-4\\
14.0169444444444	-4\\
14.0225	-2\\
14.0280555555556	-2\\
14.0336111111111	-2\\
14.0391666666667	-2\\
14.0447222222222	-2\\
14.0502777777778	-2\\
14.0558333333333	-2\\
14.0613888888889	-2\\
14.0669444444444	-2\\
14.0725	-3\\
14.0780555555556	-2\\
14.0836111111111	-2\\
14.0891666666667	-2\\
14.0947222222222	-2\\
14.1002777777778	-2\\
14.1058333333333	-2\\
14.1113888888889	-2\\
14.1169444444444	-2\\
14.1225	-2\\
14.1280555555556	-2\\
14.1336111111111	-2\\
14.1391666666667	-3\\
14.1447222222222	-2\\
14.1502777777778	-2\\
14.1558333333333	-2\\
14.1613888888889	-2\\
14.1669444444444	-2\\
14.1725	-2\\
14.1780555555556	-2\\
14.1836111111111	-2\\
14.1891666666667	-2\\
14.1947222222222	-2\\
14.2002777777778	-2\\
14.2058333333333	-2\\
14.2113888888889	-2\\
14.2169444444444	-2\\
14.2225	-1\\
14.2280555555556	-2\\
14.2336111111111	-2\\
14.2391666666667	-2\\
14.2447222222222	-2\\
14.2502777777778	-2\\
14.2558333333333	-2\\
14.2613888888889	-2\\
14.2669444444444	-2\\
14.2725	-2\\
14.2780555555556	-2\\
14.2836111111111	-2\\
14.2891666666667	-2\\
14.2947222222222	-2\\
14.3002777777778	-2\\
14.3058333333333	-2\\
14.3113888888889	-2\\
14.3169444444444	-2\\
14.3225	-2\\
14.3280555555556	-2\\
14.3336111111111	-2\\
14.3391666666667	-2\\
14.3447222222222	-2\\
14.3502777777778	-2\\
14.3558333333333	-2\\
14.3613888888889	-2\\
14.3669444444444	-2\\
14.3725	-2\\
14.3780555555556	-4\\
14.3836111111111	-5\\
14.3891666666667	-6\\
14.3947222222222	-6\\
14.4002777777778	-5\\
14.4058333333333	-6\\
14.4113888888889	-6\\
14.4169444444444	-5\\
14.4225	-5\\
14.4280555555556	-6\\
14.4336111111111	-6\\
14.4391666666667	-5\\
14.4447222222222	-7\\
14.4502777777778	-7\\
14.4558333333333	-9\\
14.4613888888889	-5\\
14.4669444444444	-3\\
14.4725	-3\\
14.4780555555556	-4\\
14.4836111111111	-4\\
14.4891666666667	-4\\
14.4947222222222	-4\\
14.5002777777778	-4\\
14.5058333333333	-6\\
14.5113888888889	-6\\
14.5169444444444	-7\\
14.5225	-6\\
14.5280555555556	-6\\
14.5336111111111	-6\\
14.5391666666667	-6\\
14.5447222222222	-6\\
14.5502777777778	-2\\
14.5558333333333	-2\\
14.5613888888889	-2\\
14.5669444444444	-5\\
14.5725	-6\\
14.5780555555556	-2\\
14.5836111111111	-3\\
14.5891666666667	-3\\
14.5947222222222	-3\\
14.6002777777778	-3\\
14.6058333333333	-2\\
14.6113888888889	-2\\
14.6169444444444	-2\\
14.6225	-3\\
14.6280555555556	-3\\
14.6336111111111	-3\\
14.6391666666667	-3\\
14.6447222222222	-3\\
14.6502777777778	-5\\
14.6558333333333	-5\\
14.6613888888889	-6\\
14.6669444444444	-7\\
14.6725	-3\\
14.6780555555556	-3\\
14.6836111111111	-5\\
14.6891666666667	-5\\
14.6947222222222	-3\\
14.7002777777778	-3\\
14.7058333333333	-3\\
14.7113888888889	-3\\
14.7169444444444	-3\\
14.7225	-3\\
14.7280555555556	-4\\
14.7336111111111	-4\\
14.7391666666667	-2\\
14.7447222222222	-3\\
14.7502777777778	-4\\
14.7558333333333	-2\\
14.7613888888889	-3\\
14.7669444444444	-4\\
14.7725	-4\\
14.7780555555556	-4\\
14.7836111111111	-4\\
14.7891666666667	-4\\
14.7947222222222	-5\\
14.8002777777778	-5\\
14.8058333333333	-6\\
14.8113888888889	-7\\
14.8169444444444	-9\\
14.8225	-10\\
14.8280555555556	-4\\
14.8336111111111	-3\\
14.8391666666667	-3\\
14.8447222222222	-5\\
14.8502777777778	-4\\
14.8558333333333	-4\\
14.8613888888889	-4\\
14.8669444444444	-4\\
14.8725	-4\\
14.8780555555556	-4\\
14.8836111111111	-4\\
14.8891666666667	-4\\
14.8947222222222	-4\\
14.9002777777778	-6\\
14.9058333333333	-6\\
14.9113888888889	-4\\
14.9169444444444	-6\\
14.9225	-6\\
14.9280555555556	-6\\
14.9336111111111	-7\\
14.9391666666667	-7\\
14.9447222222222	-7\\
14.9502777777778	-7\\
14.9558333333333	-6\\
14.9613888888889	-2\\
14.9669444444444	-4\\
14.9725	-5\\
14.9780555555556	-5\\
14.9836111111111	-5\\
14.9891666666667	-6\\
14.9947222222222	-7\\
15.0002777777778	-8\\
15.0058333333333	-8\\
15.0113888888889	-9\\
15.0169444444444	-9\\
15.0225	-9\\
15.0280555555556	-9\\
15.0336111111111	-9\\
15.0391666666667	-10\\
15.0447222222222	-4\\
15.0502777777778	-4\\
15.0558333333333	-5\\
15.0613888888889	-5\\
15.0669444444444	-3\\
15.0725	-3\\
15.0780555555556	-3\\
15.0836111111111	-3\\
15.0891666666667	-5\\
15.0947222222222	-3\\
15.1002777777778	-2\\
15.1058333333333	-2\\
15.1113888888889	-2\\
15.1169444444444	-2\\
15.1225	-3\\
15.1280555555556	-4\\
15.1336111111111	-6\\
15.1391666666667	-6\\
15.1447222222222	-5\\
15.1502777777778	-4\\
15.1558333333333	-5\\
15.1613888888889	-5\\
15.1669444444444	-6\\
15.1725	-3\\
15.1780555555556	-2\\
15.1836111111111	-2\\
15.1891666666667	-3\\
15.1947222222222	-3\\
15.2002777777778	-2\\
15.2058333333333	-2\\
15.2113888888889	-2\\
15.2169444444444	-2\\
15.2225	-4\\
15.2280555555556	-4\\
15.2336111111111	-2\\
15.2391666666667	-2\\
15.2447222222222	-2\\
15.2502777777778	-2\\
15.2558333333333	-6\\
15.2613888888889	-8\\
15.2669444444444	-5\\
15.2725	-3\\
15.2780555555556	-2\\
15.2836111111111	-3\\
15.2891666666667	-3\\
15.2947222222222	-3\\
15.3002777777778	-2\\
15.3058333333333	-2\\
15.3113888888889	-2\\
15.3169444444444	-2\\
15.3225	-5\\
15.3280555555556	-4\\
15.3336111111111	-4\\
15.3391666666667	-3\\
15.3447222222222	-3\\
15.3502777777778	-3\\
15.3558333333333	-2\\
15.3613888888889	-2\\
15.3669444444444	-2\\
15.3725	-3\\
15.3780555555556	-4\\
15.3836111111111	-5\\
15.3891666666667	-4\\
15.3947222222222	-2\\
15.4002777777778	-2\\
15.4058333333333	-2\\
15.4113888888889	-2\\
15.4169444444444	-2\\
15.4225	-2\\
15.4280555555556	-2\\
15.4336111111111	-3\\
15.4391666666667	-5\\
15.4447222222222	-5\\
15.4502777777778	-6\\
15.4558333333333	-6\\
15.4613888888889	-6\\
15.4669444444444	-7\\
15.4725	-6\\
15.4780555555556	-6\\
15.4836111111111	-6\\
15.4891666666667	-7\\
15.4947222222222	-7\\
15.5002777777778	-8\\
15.5058333333333	-8\\
15.5113888888889	-6\\
15.5169444444444	-5\\
15.5225	-4\\
15.5280555555556	-4\\
15.5336111111111	-5\\
15.5391666666667	-7\\
15.5447222222222	-9\\
15.5502777777778	-9\\
15.5558333333333	-8\\
15.5613888888889	-6\\
15.5669444444444	-6\\
15.5725	-4\\
15.5780555555556	-4\\
15.5836111111111	-5\\
15.5891666666667	-5\\
15.5947222222222	-5\\
15.6002777777778	-5\\
15.6058333333333	-5\\
15.6113888888889	-5\\
15.6169444444444	-3\\
15.6225	-6\\
15.6280555555556	-5\\
15.6336111111111	-4\\
15.6391666666667	-2\\
15.6447222222222	-4\\
15.6502777777778	-3\\
15.6558333333333	-3\\
15.6613888888889	-3\\
15.6669444444444	-4\\
15.6725	-3\\
15.6780555555556	-2\\
15.6836111111111	-2\\
15.6891666666667	-2\\
15.6947222222222	-2\\
15.7002777777778	-3\\
15.7058333333333	-3\\
15.7113888888889	-3\\
15.7169444444444	-3\\
15.7225	-4\\
15.7280555555556	-4\\
15.7336111111111	-2\\
15.7391666666667	-2\\
15.7447222222222	-2\\
15.7502777777778	-2\\
15.7558333333333	-3\\
15.7613888888889	-2\\
15.7669444444444	-3\\
15.7725	-5\\
15.7780555555556	-8\\
15.7836111111111	-7\\
15.7891666666667	-8\\
15.7947222222222	-7\\
15.8002777777778	-8\\
15.8058333333333	-10\\
15.8113888888889	-11\\
15.8169444444444	-13\\
15.8225	-13\\
15.8280555555556	-12\\
15.8336111111111	-12\\
15.8391666666667	-7\\
15.8447222222222	-7\\
15.8502777777778	-7\\
15.8558333333333	-7\\
15.8613888888889	-6\\
15.8669444444444	-3\\
15.8725	-3\\
15.8780555555556	-3\\
15.8836111111111	-2\\
15.8891666666667	-2\\
15.8947222222222	-2\\
15.9002777777778	-3\\
15.9058333333333	-5\\
15.9113888888889	-5\\
15.9169444444444	-6\\
15.9225	-8\\
15.9280555555556	-13\\
15.9336111111111	-4\\
15.9391666666667	-7\\
15.9447222222222	-6\\
15.9502777777778	-8\\
15.9558333333333	-9\\
15.9613888888889	-9\\
15.9669444444444	-9\\
15.9725	-10\\
15.9780555555556	-11\\
15.9836111111111	-5\\
15.9891666666667	-5\\
15.9947222222222	-2\\
};
\addlegendentry{Dscr Stoch Ctrl w nFPC};

\end{axis}
\end{tikzpicture}%

  \caption{Inventory comparison of the four stochastic control methods.}
  \label{fig:ORCL_comp4stoch_inv}
\end{subfigure}%
\caption{Comparison of the four stochastic control methods.}
\end{figure}
In \autoref{fig:ORCL_comp4stoch} we plot the normalized PnL for the four strategies, calibrated and backtested using data for \texttt{ORCL} from 2013-05-15. At a glance the four plots show obvious similarities in trajectory, as well as in the distinct spikes between 13h and 14.5h. Nevertheless the correlation of arithmetic returns, \autoref{tbl:ORCL_comp4stoch_corr}, shows that the strategies' returns were uncorrelated. Indeed, while the overall paths are similar, on close inspection the individual returns do show markedly different behavior.
\begin{table}[H]
\centering
\ra{1.2}
\begin{tabular}{@{} r *{4}{c} @{}}
\toprule
& Cts & \cellbreak{t}{c}{Cts \\ w nFPC} & Dscr & \cellbreak{t}{c}{Dscr \\ w nFPC} \\
\cmidrule{2-5}
Cts          &  1.0000  & & & \\
Cts w nFPC   & -0.0109  &  1.0000 &  & \\
Dscr         & -0.0120  &  0.0122 &   1.0000 &  \\
Dscr w nFPC  & -0.0015  &  0.0034 &  -0.0165 &   1.0000 \\
\bottomrule
\end{tabular}
\caption{Correlation of returns}
\label{tbl:ORCL_comp4stoch_corr}
\end{table}
However, we find instead that the returns are co-integrated. On running the Engle-Granger cointegration test with statistics computed using an augmented Dickey-Fuller test of residuals, the $\tau$-test and $z$-test both returned $p$-values of 0.001, thus rejecting the null hypothesis of no co-integration. Indeed, the co-integration relation plotted in \autoref{fig:cointeg_relation} displays stationarity, thus confirming the existence of a co-integration relation.
\begin{figure}[H]
  \centering
  \setlength\figureheight{0.2\linewidth} 
  \setlength\figurewidth{0.35\linewidth}
  \tikzsetnextfilename{cointeg_relation}
  % This file was created by matlab2tikz.
%
%The latest updates can be retrieved from
%  http://www.mathworks.com/matlabcentral/fileexchange/22022-matlab2tikz-matlab2tikz
%where you can also make suggestions and rate matlab2tikz.
%
\begin{tikzpicture}[trim axis left, trim axis right]

\begin{axis}[%
width=\figurewidth,
height=\figureheight,
at={(0\figurewidth,0\figureheight)},
scale only axis,
every outer x axis line/.append style={black},
every x tick label/.append style={font=\color{black}},
xmin=9.5,
xmax=16,
xlabel={Time [h]},
every outer y axis line/.append style={black},
every y tick label/.append style={font=\color{black}},
ymin=-0.03,
ymax=0.04,
ylabel={},
title={Cointegrating Relation},
axis background/.style={fill=white},
axis x line*=bottom,
axis y line*=left,
yticklabel style={
        /pgf/number format/fixed,
        /pgf/number format/precision=3
},
scaled y ticks=false,
legend style={legend cell align=left,align=left,draw=black,font=\small, legend pos=north west},
]
\addplot [color=blue,solid,line width=1.5pt,forget plot]
  table[row sep=crcr]{%
9.50027777777778	-0.00649349697383166\\
9.50305555555556	-0.00538515058711511\\
9.50583333333333	-0.000975442708040742\\
9.50861111111111	-0.00112799142180334\\
9.51138888888889	0.033729101479292\\
9.51416666666667	-0.00110771530981036\\
9.51694444444444	-0.00584896295802729\\
9.51972222222222	-0.00588648571818438\\
9.5225	-0.0116231692842139\\
9.52527777777778	-0.0038711819602893\\
9.52805555555556	-0.02480029916492\\
9.53083333333333	-0.0277847545105595\\
9.53361111111111	-0.0205461321681931\\
9.53638888888889	-0.00988560832347582\\
9.53916666666667	-0.00405778892009148\\
9.54194444444444	-0.00970791148360503\\
9.54472222222222	-0.0157985369079592\\
9.5475	-0.00658850531093523\\
9.55027777777778	-0.0103401195680479\\
9.55305555555555	-0.00816667039863216\\
9.55583333333333	-0.00816667039863216\\
9.55861111111111	-0.0101722716437027\\
9.56138888888889	-0.00303273107411231\\
9.56416666666667	-0.00215686989180905\\
9.56694444444444	-0.00209379193385206\\
9.56972222222222	-0.0011596755536112\\
9.5725	9.3435083667005e-05\\
9.57527777777778	9.3435083667005e-05\\
9.57805555555555	9.3435083667005e-05\\
9.58083333333333	-0.00374285038001263\\
9.58361111111111	-0.00374285038001263\\
9.58638888888889	-0.00139099666894348\\
9.58916666666667	-0.00380295827271597\\
9.59194444444444	-0.00742307463604285\\
9.59472222222222	-0.00742307463604285\\
9.5975	-0.00890684895506191\\
9.60027777777778	-0.00697228352898695\\
9.60305555555555	-0.00568865009693123\\
9.60583333333333	-0.0143880487683933\\
9.60861111111111	-0.0128752517477776\\
9.61138888888889	-0.00934896370826435\\
9.61416666666667	-0.00397948339594045\\
9.61694444444444	-0.00397948339594045\\
9.61972222222222	0.00101295422161863\\
9.6225	-0.00376745066858546\\
9.62527777777778	-0.00276375326382203\\
9.62805555555556	-0.00575349229346011\\
9.63083333333333	-0.0080918727077899\\
9.63361111111111	-0.00829552338694106\\
9.63638888888889	-0.00695175341036102\\
9.63916666666667	-0.00668907964820036\\
9.64194444444444	-0.00647065598072599\\
9.64472222222222	-0.00602884082137385\\
9.6475	-0.00602884082137385\\
9.65027777777778	-0.0066620904138623\\
9.65305555555556	-0.0061721596748715\\
9.65583333333333	-0.0061721596748715\\
9.65861111111111	-0.00504968986139192\\
9.66138888888889	-0.00437993924044874\\
9.66416666666667	-0.00452990000234495\\
9.66694444444444	-0.00415981457511554\\
9.66972222222222	-0.00428455285920534\\
9.6725	-0.00214494268584359\\
9.67527777777778	-0.00178879935937043\\
9.67805555555555	-0.00285553404383201\\
9.68083333333333	-0.00238670202173334\\
9.68361111111111	-0.000922304366886534\\
9.68638888888889	-0.000930588407873039\\
9.68916666666667	1.15333855854865e-05\\
9.69194444444444	-0.00339960761853483\\
9.69472222222222	-0.00478668506822644\\
9.6975	-0.00417679250844341\\
9.70027777777778	-0.00417679250844341\\
9.70305555555555	-0.00417679250844341\\
9.70583333333333	-0.00179314834201113\\
9.70861111111111	-0.00179314834201113\\
9.71138888888889	-0.00395530716453507\\
9.71416666666667	-0.00395530716453507\\
9.71694444444444	-0.00525571010900612\\
9.71972222222222	-0.00294732894062307\\
9.7225	-0.00201821541193409\\
9.72527777777778	-0.000695618940231374\\
9.72805555555555	-0.00324041257404949\\
9.73083333333333	-0.00324041257404949\\
9.73361111111111	-0.0061349406472448\\
9.73638888888889	-0.0061349406472448\\
9.73916666666667	-0.0061349406472448\\
9.74194444444444	-0.0061349406472448\\
9.74472222222222	-0.0061349406472448\\
9.7475	-0.0061349406472448\\
9.75027777777778	-0.00684919357992936\\
9.75305555555556	-0.00656934258469412\\
9.75583333333333	-0.00656934258469412\\
9.75861111111111	-0.00498561427943606\\
9.76138888888889	-0.00498561427943606\\
9.76416666666667	-0.00460330572331415\\
9.76694444444444	-0.00414779261303325\\
9.76972222222222	-0.00769271381746917\\
9.7725	-0.00750021104793655\\
9.77527777777778	-0.00742809625755005\\
9.77805555555556	-0.00742809625755005\\
9.78083333333333	-0.00742809625755005\\
9.78361111111111	-0.00753802167530077\\
9.78638888888889	-0.00724018222732601\\
9.78916666666667	-0.00724018222732601\\
9.79194444444444	-0.00722823138337043\\
9.79472222222222	-0.00722823138337043\\
9.7975	-0.00679810045739406\\
9.80027777777778	-0.0096812725722636\\
9.80305555555555	-0.00880757790236553\\
9.80583333333333	-0.00880757790236553\\
9.80861111111111	-0.00857392437367246\\
9.81138888888889	-0.0089700928972851\\
9.81416666666667	-0.0089700928972851\\
9.81694444444444	-0.00915299588466386\\
9.81972222222222	-0.009479693564149\\
9.8225	-0.00646698618787172\\
9.82527777777778	-0.00665422869736743\\
9.82805555555555	-0.00920097650195285\\
9.83083333333333	-0.00876412916700366\\
9.83361111111111	-0.00766345241438942\\
9.83638888888889	-0.00796846319736431\\
9.83916666666667	-0.00796846319736431\\
9.84194444444444	-0.00810539373882271\\
9.84472222222222	-0.00810539373882271\\
9.8475	-0.0071051550956986\\
9.85027777777778	-0.0074728698242274\\
9.85305555555555	-0.00806573341368087\\
9.85583333333333	-0.00806573341368087\\
9.85861111111111	-0.00806573341368087\\
9.86138888888889	-0.00742340836845611\\
9.86416666666667	-0.00751979133793359\\
9.86694444444444	-0.00654734720829235\\
9.86972222222222	-0.00742400199558782\\
9.8725	-0.00742400199558782\\
9.87527777777778	-0.00700315323183133\\
9.87805555555556	-0.00700315323183133\\
9.88083333333333	-0.00700315323183133\\
9.88361111111111	-0.00801767772623417\\
9.88638888888889	-0.00812691194266817\\
9.88916666666667	-0.00812691194266817\\
9.89194444444444	-0.00747378919964214\\
9.89472222222222	-0.00785995241283687\\
9.8975	-0.00785995241283687\\
9.90027777777778	-0.00785995241283687\\
9.90305555555556	-0.00752617881600386\\
9.90583333333333	-0.00763448846887603\\
9.90861111111111	-0.00763448846887603\\
9.91138888888889	-0.00793361637538804\\
9.91416666666667	-0.00793361637538804\\
9.91694444444444	-0.00869461776872314\\
9.91972222222222	-0.00802236341406048\\
9.9225	-0.00802236341406048\\
9.92527777777778	-0.00802236341406048\\
9.92805555555555	-0.00790099528600495\\
9.93083333333333	-0.00798008079389526\\
9.93361111111111	-0.00798008079389526\\
9.93638888888889	-0.00877751704671036\\
9.93916666666667	-0.00877751704671036\\
9.94194444444444	-0.00877751704671036\\
9.94472222222222	-0.00843802883182967\\
9.9475	-0.00843802883182967\\
9.95027777777778	-0.00792024960652324\\
9.95305555555555	-0.00881364009303782\\
9.95583333333333	-0.00910266827678017\\
9.95861111111111	-0.00674285549111066\\
9.96138888888889	-0.00674285549111066\\
9.96416666666667	-0.00970807727294066\\
9.96694444444444	-0.00910514051704815\\
9.96972222222222	-0.00856554841122204\\
9.9725	-0.00856554841122204\\
9.97527777777778	-0.00856554841122204\\
9.97805555555555	-0.0131945416871447\\
9.98083333333333	-0.0131945416871447\\
9.98361111111111	-0.0131945416871447\\
9.98638888888889	-0.0131945416871447\\
9.98916666666667	-0.0108453535867021\\
9.99194444444444	-0.0108453535867021\\
9.99472222222222	-0.00817646445886957\\
9.9975	-0.00817646445886957\\
10.0002777777778	-0.00817646445886957\\
10.0030555555556	-0.00817646445886957\\
10.0058333333333	-0.00817646445886957\\
10.0086111111111	-0.00817646445886957\\
10.0113888888889	-0.00817646445886957\\
10.0141666666667	-0.00678997619591088\\
10.0169444444444	-0.0101409741209769\\
10.0197222222222	-0.00948767442491721\\
10.0225	-0.00654160801987926\\
10.0252777777778	-0.00685203765986437\\
10.0280555555556	-0.00685203765986437\\
10.0308333333333	-0.00653037780473897\\
10.0336111111111	-0.0060265580232845\\
10.0363888888889	-0.0060265580232845\\
10.0391666666667	-0.00589917070041958\\
10.0419444444444	-0.00687265453356445\\
10.0447222222222	-0.00687265453356445\\
10.0475	-0.00579358018561683\\
10.0502777777778	-0.00564020672708305\\
10.0530555555556	-0.00564020672708305\\
10.0558333333333	-0.00637204540060725\\
10.0586111111111	-0.00637204540060725\\
10.0613888888889	-0.00637204540060725\\
10.0641666666667	-0.00474127014894181\\
10.0669444444444	-0.00476668300433708\\
10.0697222222222	-0.00476668300433708\\
10.0725	-0.00476668300433708\\
10.0752777777778	-0.00661303649942249\\
10.0780555555556	-0.00950791108629428\\
10.0808333333333	-0.00972633475376895\\
10.0836111111111	-0.00972633475376895\\
10.0863888888889	-0.00972633475376895\\
10.0891666666667	-0.00972633475376895\\
10.0919444444444	-0.00972633475376895\\
10.0947222222222	-0.00972633475376895\\
10.0975	-0.00972633475376895\\
10.1002777777778	-0.00263106991317219\\
10.1030555555556	-0.00263106991317219\\
10.1058333333333	-0.00415543053484159\\
10.1086111111111	-0.00415543053484159\\
10.1113888888889	-0.0060787430081996\\
10.1141666666667	-0.00683169397285731\\
10.1169444444444	-0.00683169397285731\\
10.1197222222222	-0.00683169397285731\\
10.1225	-0.00683169397285731\\
10.1252777777778	-0.0066051998390105\\
10.1280555555556	-0.00537955164566052\\
10.1308333333333	-0.00537955164566052\\
10.1336111111111	-0.00537955164566052\\
10.1363888888889	-0.00537955164566052\\
10.1391666666667	-0.00537955164566052\\
10.1419444444444	-0.00511713110277952\\
10.1447222222222	-0.00575665345529167\\
10.1475	-0.00575665345529167\\
10.1502777777778	-0.00575665345529167\\
10.1530555555556	-0.00575665345529167\\
10.1558333333333	-0.00575665345529167\\
10.1586111111111	-0.00575665345529167\\
10.1613888888889	-0.00575665345529167\\
10.1641666666667	-0.00575665345529167\\
10.1669444444444	-0.00528610871784833\\
10.1697222222222	-0.00528610871784833\\
10.1725	-0.00481933054907513\\
10.1752777777778	-0.00481933054907513\\
10.1780555555556	-0.00481933054907513\\
10.1808333333333	-0.00484842143966381\\
10.1836111111111	-0.00462663205365013\\
10.1863888888889	-0.00522502364903915\\
10.1891666666667	-0.00522502364903915\\
10.1919444444444	-0.00522502364903915\\
10.1947222222222	-0.00522502364903915\\
10.1975	-0.00522502364903915\\
10.2002777777778	-0.00544344731651352\\
10.2030555555556	-0.00544344731651352\\
10.2058333333333	-0.00544344731651352\\
10.2086111111111	-0.00544344731651352\\
10.2113888888889	-0.00544344731651352\\
10.2141666666667	-0.00544344731651352\\
10.2169444444444	-0.00544344731651352\\
10.2197222222222	-0.00562844084154815\\
10.2225	-0.00562844084154815\\
10.2252777777778	-0.00633370988372447\\
10.2280555555556	-0.00633370988372447\\
10.2308333333333	-0.00644405346317211\\
10.2336111111111	-0.00644405346317211\\
10.2363888888889	-0.00644405346317211\\
10.2391666666667	-0.00640200581719614\\
10.2419444444444	-0.00652650465448875\\
10.2447222222222	-0.00652650465448875\\
10.2475	-0.00652650465448875\\
10.2502777777778	-0.00627788831678242\\
10.2530555555556	-0.00608356642112058\\
10.2558333333333	-0.00635443015453961\\
10.2586111111111	-0.00509855793664975\\
10.2613888888889	-0.00599106925795246\\
10.2641666666667	-0.00517562662295784\\
10.2669444444444	-0.0056840892189849\\
10.2697222222222	-0.0056840892189849\\
10.2725	-0.00393599515159796\\
10.2752777777778	-0.00709643500722194\\
10.2780555555556	-0.00709643500722194\\
10.2808333333333	-0.00451667560135383\\
10.2836111111111	-0.00578628080000925\\
10.2863888888889	-0.00275057195773\\
10.2891666666667	-0.00275057195773\\
10.2919444444444	-0.00275057195773\\
10.2947222222222	-0.00275057195773\\
10.2975	-0.00113622257146076\\
10.3002777777778	-0.00546839851745495\\
10.3030555555556	-0.00546839851745495\\
10.3058333333333	-0.00546839851745495\\
10.3086111111111	-0.00443882559353184\\
10.3113888888889	-0.00677287732376407\\
10.3141666666667	-0.00677287732376407\\
10.3169444444444	-0.00447922200199391\\
10.3197222222222	-0.00738479577781859\\
10.3225	-0.00738479577781859\\
10.3252777777778	-0.00719899404313226\\
10.3280555555556	-0.0086837023715358\\
10.3308333333333	-0.00925863944672306\\
10.3336111111111	-0.00925863944672306\\
10.3363888888889	-0.00925863944672306\\
10.3391666666667	-0.00786752447863063\\
10.3419444444444	-0.00786752447863063\\
10.3447222222222	-0.00820823014091753\\
10.3475	-0.00708216419164919\\
10.3502777777778	-0.00610561550424188\\
10.3530555555556	-0.007669253049363\\
10.3558333333333	-0.0071971788351018\\
10.3586111111111	-0.0071971788351018\\
10.3613888888889	-0.0071971788351018\\
10.3641666666667	-0.0071971788351018\\
10.3669444444444	-0.0071971788351018\\
10.3697222222222	-0.0071971788351018\\
10.3725	-0.0071971788351018\\
10.3752777777778	-0.0071971788351018\\
10.3780555555556	-0.0071971788351018\\
10.3808333333333	-0.0071971788351018\\
10.3836111111111	-0.0071971788351018\\
10.3863888888889	-0.0071971788351018\\
10.3891666666667	-0.00649511156327611\\
10.3919444444444	-0.00649511156327611\\
10.3947222222222	-0.00649511156327611\\
10.3975	-0.00705180094231762\\
10.4002777777778	-0.00705180094231762\\
10.4030555555556	-0.00705180094231762\\
10.4058333333333	-0.00704869002944691\\
10.4086111111111	-0.00599982223520678\\
10.4113888888889	-0.00650972811361511\\
10.4141666666667	-0.00650972811361511\\
10.4169444444444	-0.00650972811361511\\
10.4197222222222	-0.00650972811361511\\
10.4225	-0.00650972811361511\\
10.4252777777778	-0.00642578996721999\\
10.4280555555556	-0.00642578996721999\\
10.4308333333333	-0.00606929785617492\\
10.4336111111111	-0.00606929785617492\\
10.4363888888889	-0.00606929785617492\\
10.4391666666667	-0.00606929785617492\\
10.4419444444444	-0.00527613633249867\\
10.4447222222222	-0.00478063515680082\\
10.4475	-0.00478063515680082\\
10.4502777777778	-0.00478063515680082\\
10.4530555555556	-0.00478063515680082\\
10.4558333333333	-0.00478063515680082\\
10.4586111111111	-0.00478063515680082\\
10.4613888888889	-0.00478063515680082\\
10.4641666666667	-0.00478063515680082\\
10.4669444444444	-0.00495805545718207\\
10.4697222222222	-0.00495805545718207\\
10.4725	-0.00457179762433508\\
10.4752777777778	-0.00457179762433508\\
10.4780555555556	-0.00400513358474151\\
10.4808333333333	-0.00400513358474151\\
10.4836111111111	-0.00400513358474151\\
10.4863888888889	-0.00389139301421087\\
10.4891666666667	-0.00400200655805228\\
10.4919444444444	-0.0045971854975179\\
10.4947222222222	-0.0045971854975179\\
10.4975	-0.0045971854975179\\
10.5002777777778	-0.0045971854975179\\
10.5030555555556	-0.00379306876907388\\
10.5058333333333	-0.00377390342264628\\
10.5086111111111	-0.00406837345048314\\
10.5113888888889	-0.00398097303184232\\
10.5141666666667	-0.00349883957181686\\
10.5169444444444	-0.00363845874796816\\
10.5197222222222	-0.0034200350804938\\
10.5225	-0.0034200350804938\\
10.5252777777778	-0.00297482911114948\\
10.5280555555556	-0.00297482911114948\\
10.5308333333333	-0.00323227862924958\\
10.5336111111111	-0.0032789997198349\\
10.5363888888889	-0.0032789997198349\\
10.5391666666667	-0.00372834458263956\\
10.5419444444444	-0.00397219211245276\\
10.5447222222222	-0.00482898623540353\\
10.5475	-0.00482898623540353\\
10.5502777777778	-0.00674805116573405\\
10.5530555555556	-0.00647106935959538\\
10.5558333333333	6.07705522919846e-05\\
10.5586111111111	6.07705522919846e-05\\
10.5613888888889	6.07705522919846e-05\\
10.5641666666667	0.00422987673227056\\
10.5669444444444	0.00468067190693647\\
10.5697222222222	0.00320006782046928\\
10.5725	0.00320006782046928\\
10.5752777777778	0.00216268069091\\
10.5780555555556	0.00189427569950093\\
10.5808333333333	-0.00175753460082482\\
10.5836111111111	-6.09684578351498e-05\\
10.5863888888889	-0.00347816465440904\\
10.5891666666667	-0.00347816465440904\\
10.5919444444444	-0.00409598890508385\\
10.5947222222222	-0.00409598890508385\\
10.5975	0.00520341142579466\\
10.6002777777778	0.0049849877583203\\
10.6030555555556	0.0049849877583203\\
10.6058333333333	0.0049849877583203\\
10.6086111111111	0.00447359485131983\\
10.6113888888889	0.00447359485131983\\
10.6141666666667	0.00447359485131983\\
10.6169444444444	0.00447359485131983\\
10.6197222222222	0.00447359485131983\\
10.6225	0.00447359485131983\\
10.6252777777778	0.0040593541280294\\
10.6280555555556	0.0040593541280294\\
10.6308333333333	0.0040593541280294\\
10.6336111111111	0.0040593541280294\\
10.6363888888889	0.00399050926045652\\
10.6391666666667	0.00399050926045652\\
10.6419444444444	0.00439503454827412\\
10.6447222222222	0.00439503454827412\\
10.6475	0.00411308797487269\\
10.6502777777778	0.00411308797487269\\
10.6530555555556	0.00411308797487269\\
10.6558333333333	0.00411308797487269\\
10.6586111111111	0.00439512226507893\\
10.6613888888889	0.00439512226507893\\
10.6641666666667	0.00391982349623386\\
10.6669444444444	0.00391982349623386\\
10.6697222222222	0.0036409956212862\\
10.6725	0.00292252946054679\\
10.6752777777778	0.00292252946054679\\
10.6780555555556	0.00247813431710287\\
10.6808333333333	0.000433430834555593\\
10.6836111111111	0.000468739522499902\\
10.6863888888889	0.000468739522499902\\
10.6891666666667	-0.00192059566944173\\
10.6919444444444	-0.00192059566944173\\
10.6947222222222	-0.00192059566944173\\
10.6975	0.00263039444387428\\
10.7002777777778	0.00263039444387428\\
10.7030555555556	0.00209476525989089\\
10.7058333333333	0.00209476525989089\\
10.7086111111111	0.00235129166320309\\
10.7113888888889	0.00235129166320309\\
10.7141666666667	0.00235129166320309\\
10.7169444444444	0.00235129166320309\\
10.7197222222222	0.00235129166320309\\
10.7225	0.00235129166320309\\
10.7252777777778	0.00235129166320309\\
10.7280555555556	0.00346116401253293\\
10.7308333333333	0.00285263803997485\\
10.7336111111111	0.00285263803997485\\
10.7363888888889	0.000536436510039368\\
10.7391666666667	0.000536436510039368\\
10.7419444444444	0.000536436510039368\\
10.7447222222222	0.000813188633376294\\
10.7475	0.000813188633376294\\
10.7502777777778	0.000293954448435817\\
10.7530555555556	0.000293954448435817\\
10.7558333333333	0.000293954448435817\\
10.7586111111111	0.00374202473751246\\
10.7613888888889	0.00444455680525832\\
10.7641666666667	0.00444455680525832\\
10.7669444444444	0.00581504015339188\\
10.7697222222222	0.00472292181601881\\
10.7725	0.00419720747535385\\
10.7752777777778	0.00419720747535385\\
10.7780555555556	0.00419720747535385\\
10.7808333333333	0.00419720747535385\\
10.7836111111111	0.00419720747535385\\
10.7863888888889	0.00419720747535385\\
10.7891666666667	0.00419720747535385\\
10.7919444444444	0.00419720747535385\\
10.7947222222222	0.00190929356596877\\
10.7975	-0.000480128375205714\\
10.8002777777778	-0.00414061523818057\\
10.8030555555556	-0.00889195048812161\\
10.8058333333333	-0.00889195048812161\\
10.8086111111111	-0.00889195048812161\\
10.8113888888889	-0.00889195048812161\\
10.8141666666667	-0.00162860904571695\\
10.8169444444444	-0.00162860904571695\\
10.8197222222222	-0.00162860904571695\\
10.8225	-0.00162860904571695\\
10.8252777777778	-0.00242498271711365\\
10.8280555555556	-0.00168121822739273\\
10.8308333333333	-0.00168121822739273\\
10.8336111111111	-0.00168121822739273\\
10.8363888888889	-0.00359637746946731\\
10.8391666666667	-0.00455466611229721\\
10.8419444444444	0.00427031678429906\\
10.8447222222222	0.00383982010392844\\
10.8475	-0.00119104682698011\\
10.8502777777778	-0.00119104682698011\\
10.8530555555556	-0.000356731737055055\\
10.8558333333333	0.00435817753179403\\
10.8586111111111	0.00691492134572371\\
10.8613888888889	0.00501715686815961\\
10.8641666666667	0.0035820533944668\\
10.8669444444444	0.00384809574852551\\
10.8697222222222	0.00323510990239859\\
10.8725	0.000989954586625857\\
10.8752777777778	0.000989954586625857\\
10.8780555555556	0.000989954586625857\\
10.8808333333333	0.000989954586625857\\
10.8836111111111	0.000989954586625857\\
10.8863888888889	0.000989954586625857\\
10.8891666666667	0.000989954586625857\\
10.8919444444444	0.000989954586625857\\
10.8947222222222	0.000989954586625857\\
10.8975	0.000989954586625857\\
10.9002777777778	0.000989954586625857\\
10.9030555555556	0.000989954586625857\\
10.9058333333333	0.000989954586625857\\
10.9086111111111	0.0110884570718523\\
10.9113888888889	0.0103226248195016\\
10.9141666666667	0.0103226248195016\\
10.9169444444444	0.0103226248195016\\
10.9197222222222	0.0089769513780318\\
10.9225	0.0089769513780318\\
10.9252777777778	0.0089769513780318\\
10.9280555555556	0.00954808374741396\\
10.9308333333333	0.00744720893969062\\
10.9336111111111	0.00642931569310305\\
10.9363888888889	0.00771298550401771\\
10.9391666666667	0.00398772965335148\\
10.9419444444444	0.00650096516170836\\
10.9447222222222	0.00525128078738206\\
10.9475	0.00226887974968632\\
10.9502777777778	0.00451400982144688\\
10.9530555555556	0.00437764566756866\\
10.9558333333333	0.00372471595727394\\
10.9586111111111	0.00387382567732883\\
10.9613888888889	0.00374577411250833\\
10.9641666666667	0.00214530222403609\\
10.9669444444444	0.00157360543758654\\
10.9697222222222	0.000187872537181378\\
10.9725	-0.0002306078206817\\
10.9752777777778	0.00375435109349142\\
10.9780555555556	0.00327484074363805\\
10.9808333333333	0.004334169080169\\
10.9836111111111	0.0043102426126077\\
10.9863888888889	0.00425921097954145\\
10.9891666666667	0.00401494375771642\\
10.9919444444444	0.00464543395012847\\
10.9947222222222	0.00514894128745372\\
10.9975	0.00484491960100878\\
11.0002777777778	0.00453900840197054\\
11.0030555555556	0.00453900840197054\\
11.0058333333333	0.00453900840197054\\
11.0086111111111	0.00436324003260934\\
11.0113888888889	0.00497064738710863\\
11.0141666666667	0.00497064738710863\\
11.0169444444444	0.00401662226969571\\
11.0197222222222	0.00361103215108132\\
11.0225	0.00279895938380733\\
11.0252777777778	0.00279895938380733\\
11.0280555555556	0.00279895938380733\\
11.0308333333333	0.00651547473345048\\
11.0336111111111	0.00651547473345048\\
11.0363888888889	0.00447823114233263\\
11.0391666666667	0.00447823114233263\\
11.0419444444444	-0.00187081310701281\\
11.0447222222222	-0.0040815262199202\\
11.0475	-0.0040815262199202\\
11.0502777777778	-0.00302461300173925\\
11.0530555555556	-0.00488017527022671\\
11.0558333333333	-0.00619071727507413\\
11.0586111111111	-0.00619071727507413\\
11.0613888888889	-0.00619071727507413\\
11.0641666666667	-0.00619071727507413\\
11.0669444444444	-0.00619071727507413\\
11.0697222222222	-0.00932819672732551\\
11.0725	-0.00514245340354024\\
11.0752777777778	-0.00514245340354024\\
11.0780555555556	-0.00514245340354024\\
11.0808333333333	-0.00460025490848364\\
11.0836111111111	-0.00460025490848364\\
11.0863888888889	-0.00460025490848364\\
11.0891666666667	-0.00460025490848364\\
11.0919444444444	-0.00460025490848364\\
11.0947222222222	-0.00460025490848364\\
11.0975	-0.00460025490848364\\
11.1002777777778	-0.00460025490848364\\
11.1030555555556	-0.0035483789486006\\
11.1058333333333	-0.00325235034282935\\
11.1086111111111	-0.00580220800376616\\
11.1113888888889	-0.0128607851960059\\
11.1141666666667	-0.0176650625725748\\
11.1169444444444	-0.0176650625725748\\
11.1197222222222	-0.0176650625725748\\
11.1225	-0.0176650625725748\\
11.1252777777778	-0.0176650625725748\\
11.1280555555556	-0.0176650625725748\\
11.1308333333333	-0.0185064370670022\\
11.1336111111111	-0.0185064370670022\\
11.1363888888889	-0.0185064370670022\\
11.1391666666667	-0.0185064370670022\\
11.1419444444444	-0.0185064370670022\\
11.1447222222222	-0.021040031097212\\
11.1475	-0.021040031097212\\
11.1502777777778	0.00323775967389646\\
11.1530555555556	-0.000397622364041266\\
11.1558333333333	0.000241376030457353\\
11.1586111111111	0.000241376030457353\\
11.1613888888889	0.000241376030457353\\
11.1641666666667	0.000241376030457353\\
11.1669444444444	-0.002108032856356\\
11.1697222222222	-0.002108032856356\\
11.1725	-0.00302002188389873\\
11.1752777777778	-0.00302002188389873\\
11.1780555555556	-0.00302002188389873\\
11.1808333333333	-0.00302002188389873\\
11.1836111111111	-0.00282955562028487\\
11.1863888888889	-0.00282955562028487\\
11.1891666666667	-0.00360373267536378\\
11.1919444444444	-0.00360373267536378\\
11.1947222222222	-0.00360373267536378\\
11.1975	-0.00225973734925416\\
11.2002777777778	-0.00225973734925416\\
11.2030555555556	0.00301434255015807\\
11.2058333333333	0.00301434255015807\\
11.2086111111111	0.00301434255015807\\
11.2113888888889	0.00271171136170833\\
11.2141666666667	0.00271171136170833\\
11.2169444444444	0.00179628989826021\\
11.2197222222222	0.00179628989826021\\
11.2225	0.00179628989826021\\
11.2252777777778	0.00179628989826021\\
11.2280555555556	0.00179628989826021\\
11.2308333333333	0.00179628989826021\\
11.2336111111111	0.00179628989826021\\
11.2363888888889	-0.000532855799099535\\
11.2391666666667	-0.00416994297215397\\
11.2419444444444	-0.00416994297215397\\
11.2447222222222	-0.00766142231245038\\
11.2475	-0.00766142231245038\\
11.2502777777778	-0.0013586256408946\\
11.2530555555556	-0.0013586256408946\\
11.2558333333333	-8.88853402647664e-05\\
11.2586111111111	0.000397483457657517\\
11.2613888888889	0.000397483457657517\\
11.2641666666667	0.000811940859643779\\
11.2669444444444	0.000811940859643779\\
11.2697222222222	0.000811940859643779\\
11.2725	0.000811940859643779\\
11.2752777777778	0.00591023593046435\\
11.2780555555556	0.000101628910653906\\
11.2808333333333	0.000101628910653906\\
11.2836111111111	0.000101628910653906\\
11.2863888888889	0.000101628910653906\\
11.2891666666667	0.000101628910653906\\
11.2919444444444	0.000101628910653906\\
11.2947222222222	-0.000726377500423809\\
11.2975	-0.00298223131002392\\
11.3002777777778	-0.00298223131002392\\
11.3030555555556	-0.00298223131002392\\
11.3058333333333	-0.00298223131002392\\
11.3086111111111	0.00575471538895851\\
11.3113888888889	0.00378890238168722\\
11.3141666666667	0.00120261951736938\\
11.3169444444444	0.00126108763623655\\
11.3197222222222	0.00126108763623655\\
11.3225	0.00126108763623655\\
11.3252777777778	0.00126108763623655\\
11.3280555555556	0.00126108763623655\\
11.3308333333333	0.00155817710766533\\
11.3336111111111	0.00155817710766533\\
11.3363888888889	0.00155817710766533\\
11.3391666666667	0.00155817710766533\\
11.3419444444444	-0.00169915188256267\\
11.3447222222222	-0.00169915188256267\\
11.3475	-0.00044075359161878\\
11.3502777777778	0.00624493893235682\\
11.3530555555556	0.00624493893235682\\
11.3558333333333	0.00325340459941697\\
11.3586111111111	0.00487154909091347\\
11.3613888888889	0.00487154909091347\\
11.3641666666667	0.00275453862264555\\
11.3669444444444	0.00381146544806796\\
11.3697222222222	0.00381146544806796\\
11.3725	0.00511909168189512\\
11.3752777777778	0.00511909168189512\\
11.3780555555556	0.0060927234021817\\
11.3808333333333	0.0060927234021817\\
11.3836111111111	0.0060927234021817\\
11.3863888888889	0.00276447322250563\\
11.3891666666667	0.00276447322250563\\
11.3919444444444	0.00162945888069094\\
11.3947222222222	0.00162945888069094\\
11.3975	0.00162945888069094\\
11.4002777777778	0.00162945888069094\\
11.4030555555556	-0.000430638968206\\
11.4058333333333	-0.000430638968206\\
11.4086111111111	-0.000430638968206\\
11.4113888888889	-0.000430638968206\\
11.4141666666667	0.000524588457569366\\
11.4169444444444	0.00147874751339923\\
11.4197222222222	-0.00154611934995153\\
11.4225	-0.00283403083038067\\
11.4252777777778	0.00371867919385611\\
11.4280555555556	0.00371867919385611\\
11.4308333333333	0.00392312475934403\\
11.4336111111111	0.0040890374783056\\
11.4363888888889	0.0040890374783056\\
11.4391666666667	0.0040890374783056\\
11.4419444444444	0.00507859433106788\\
11.4447222222222	0.00507859433106788\\
11.4475	0.00507859433106788\\
11.4502777777778	0.00507859433106788\\
11.4530555555556	0.00422109068502012\\
11.4558333333333	0.00228872122482944\\
11.4586111111111	0.00132424413365912\\
11.4613888888889	0.00182655228701707\\
11.4641666666667	0.00182655228701707\\
11.4669444444444	0.00182655228701707\\
11.4697222222222	0.00182655228701707\\
11.4725	0.00280249408267856\\
11.4752777777778	0.00280249408267856\\
11.4780555555556	0.00280249408267856\\
11.4808333333333	0.00280249408267856\\
11.4836111111111	0.00433306509912886\\
11.4863888888889	0.00433306509912886\\
11.4891666666667	0.00495632982141329\\
11.4919444444444	-0.000867704354943887\\
11.4947222222222	-0.000867704354943887\\
11.4975	-0.000867704354943887\\
11.5002777777778	-0.000867704354943887\\
11.5030555555556	-0.000867704354943887\\
11.5058333333333	-0.000867704354943887\\
11.5086111111111	0.00232975005743049\\
11.5113888888889	0.00458438722757441\\
11.5141666666667	0.00422374705211696\\
11.5169444444444	0.00538458087558762\\
11.5197222222222	0.00538458087558762\\
11.5225	0.00538458087558762\\
11.5252777777778	0.00538458087558762\\
11.5280555555556	0.00538458087558762\\
11.5308333333333	0.00538458087558762\\
11.5336111111111	0.00538458087558762\\
11.5363888888889	0.00538458087558762\\
11.5391666666667	0.00538458087558762\\
11.5419444444444	0.00538458087558762\\
11.5447222222222	0.00500360216979872\\
11.5475	0.0052492983636651\\
11.5502777777778	0.0052492983636651\\
11.5530555555556	0.0052492983636651\\
11.5558333333333	0.0052492983636651\\
11.5586111111111	0.00506232556074922\\
11.5613888888889	0.00592496467251714\\
11.5641666666667	0.00193359423046751\\
11.5669444444444	-0.00168062762886841\\
11.5697222222222	-0.00168062762886841\\
11.5725	-6.1230006134845e-05\\
11.5752777777778	-6.1230006134845e-05\\
11.5780555555556	-6.1230006134845e-05\\
11.5808333333333	-6.1230006134845e-05\\
11.5836111111111	-6.1230006134845e-05\\
11.5863888888889	-6.1230006134845e-05\\
11.5891666666667	0.00710916812662943\\
11.5919444444444	0.00710916812662943\\
11.5947222222222	0.00514682605674487\\
11.5975	0.00514682605674487\\
11.6002777777778	0.00553631691800234\\
11.6030555555556	0.00553631691800234\\
11.6058333333333	0.00651206469989556\\
11.6086111111111	0.00651206469989556\\
11.6113888888889	0.00651206469989556\\
11.6141666666667	0.00651206469989556\\
11.6169444444444	0.00651206469989556\\
11.6197222222222	0.00454046324199463\\
11.6225	0.00350269911244632\\
11.6252777777778	0.00350269911244632\\
11.6280555555556	0.00469796424050311\\
11.6308333333333	0.00348128443428465\\
11.6336111111111	0.00443241835453757\\
11.6363888888889	0.00443241835453757\\
11.6391666666667	0.00200166537355452\\
11.6419444444444	0.00550979313916179\\
11.6447222222222	0.00550979313916179\\
11.6475	0.00604208254520955\\
11.6502777777778	0.00604208254520955\\
11.6530555555556	0.00604208254520955\\
11.6558333333333	0.00604208254520955\\
11.6586111111111	0.00604208254520955\\
11.6613888888889	0.0032450183645982\\
11.6641666666667	0.0032450183645982\\
11.6669444444444	0.0032450183645982\\
11.6697222222222	0.0032450183645982\\
11.6725	0.00568715594496214\\
11.6752777777778	0.00425742805909396\\
11.6780555555556	0.00425742805909396\\
11.6808333333333	0.00425742805909396\\
11.6836111111111	0.00425742805909396\\
11.6863888888889	0.000198282756649834\\
11.6891666666667	0.00633413315428792\\
11.6919444444444	0.00633413315428792\\
11.6947222222222	0.00633413315428792\\
11.6975	0.00633413315428792\\
11.7002777777778	0.00633413315428792\\
11.7030555555556	0.00633413315428792\\
11.7058333333333	0.00633413315428792\\
11.7086111111111	0.00633413315428792\\
11.7113888888889	0.00633413315428792\\
11.7141666666667	0.00633413315428792\\
11.7169444444444	0.00633413315428792\\
11.7197222222222	0.00633413315428792\\
11.7225	0.00633413315428792\\
11.7252777777778	0.00633413315428792\\
11.7280555555556	0.00633413315428792\\
11.7308333333333	0.00633413315428792\\
11.7336111111111	0.00563408775622971\\
11.7363888888889	0.00456373626756365\\
11.7391666666667	0.00456373626756365\\
11.7419444444444	0.00456373626756365\\
11.7447222222222	0.00308643343138309\\
11.7475	0.00114021913468897\\
11.7502777777778	0.00159650708483538\\
11.7530555555556	0.00207565518665361\\
11.7558333333333	0.0028766924179386\\
11.7586111111111	0.00160102648634367\\
11.7613888888889	0.00160102648634367\\
11.7641666666667	0.00160102648634367\\
11.7669444444444	0.00058900708381399\\
11.7697222222222	0.00058900708381399\\
11.7725	-0.00152280375513498\\
11.7752777777778	-0.00228812039421675\\
11.7780555555556	-0.00228812039421675\\
11.7808333333333	-0.00228812039421675\\
11.7836111111111	-0.0031628557077822\\
11.7863888888889	-0.00446997502104709\\
11.7891666666667	-0.00879960403510613\\
11.7919444444444	0.00529366645360901\\
11.7947222222222	0.00650326394016813\\
11.7975	0.0057531274744317\\
11.8002777777778	0.0057531274744317\\
11.8030555555556	0.0057531274744317\\
11.8058333333333	0.0057531274744317\\
11.8086111111111	0.0057531274744317\\
11.8113888888889	0.0057531274744317\\
11.8141666666667	0.0057531274744317\\
11.8169444444444	0.0057531274744317\\
11.8197222222222	0.0057531274744317\\
11.8225	0.0057531274744317\\
11.8252777777778	0.0057531274744317\\
11.8280555555556	0.0057531274744317\\
11.8308333333333	0.00553470380695733\\
11.8336111111111	0.00553470380695733\\
11.8363888888889	0.00553470380695733\\
11.8391666666667	0.0051858909914619\\
11.8419444444444	0.0051858909914619\\
11.8447222222222	0.00314831783893128\\
11.8475	0.00611150004198784\\
11.8502777777778	0.00611150004198784\\
11.8530555555556	0.00611150004198784\\
11.8558333333333	0.00667214056461065\\
11.8586111111111	0.00531398496034208\\
11.8613888888889	0.00606682375573188\\
11.8641666666667	0.00606682375573188\\
11.8669444444444	0.00623977637590897\\
11.8697222222222	0.00602604303744504\\
11.8725	0.00365418287555679\\
11.8752777777778	0.0029847913590502\\
11.8780555555556	0.0029847913590502\\
11.8808333333333	0.00265446698902612\\
11.8836111111111	0.00265446698902612\\
11.8863888888889	0.00265446698902612\\
11.8891666666667	0.000726144903517432\\
11.8919444444444	0.00475437500462933\\
11.8947222222222	0.00475437500462933\\
11.8975	0.00475437500462933\\
11.9002777777778	0.00589026190875366\\
11.9030555555556	0.00589026190875366\\
11.9058333333333	0.00723590268547597\\
11.9086111111111	0.00723590268547597\\
11.9113888888889	0.00612504645301591\\
11.9141666666667	0.00612504645301591\\
11.9169444444444	0.00612504645301591\\
11.9197222222222	0.00668617081804655\\
11.9225	0.00668617081804655\\
11.9252777777778	0.00668617081804655\\
11.9280555555556	0.00540648662066881\\
11.9308333333333	0.00540648662066881\\
11.9336111111111	0.00540648662066881\\
11.9363888888889	0.00401906451836541\\
11.9391666666667	-0.00102906873439055\\
11.9419444444444	-0.00102906873439055\\
11.9447222222222	-0.00102906873439055\\
11.9475	-0.00102906873439055\\
11.9502777777778	-0.00102906873439055\\
11.9530555555556	-0.00102906873439055\\
11.9558333333333	-0.00102906873439055\\
11.9586111111111	0.00315756564151932\\
11.9613888888889	0.00315756564151932\\
11.9641666666667	0.00349979459993827\\
11.9669444444444	0.00330095393447926\\
11.9697222222222	0.00330095393447926\\
11.9725	0.00330095393447926\\
11.9752777777778	0.00330095393447926\\
11.9780555555556	0.00356931615477027\\
11.9808333333333	0.00356931615477027\\
11.9836111111111	0.00379347885870606\\
11.9863888888889	0.00379347885870606\\
11.9891666666667	0.00379347885870606\\
11.9919444444444	0.00379347885870606\\
11.9947222222222	0.00379347885870606\\
11.9975	0.00379347885870606\\
12.0002777777778	0.00379347885870606\\
12.0030555555556	0.00379347885870606\\
12.0058333333333	0.00117996442308744\\
12.0086111111111	0.00139703733555873\\
12.0113888888889	0.000658886461165064\\
12.0141666666667	-0.000294089863633025\\
12.0169444444444	-0.000294089863633025\\
12.0197222222222	-0.000294089863633025\\
12.0225	-0.000294089863633025\\
12.0252777777778	0.000162198086513377\\
12.0280555555556	0.000162198086513377\\
12.0308333333333	0.000162198086513377\\
12.0336111111111	-5.62255809609859e-05\\
12.0363888888889	-5.62255809609859e-05\\
12.0391666666667	0.00665164669503861\\
12.0419444444444	0.00614678132152057\\
12.0447222222222	0.00614678132152057\\
12.0475	0.00614678132152057\\
12.0502777777778	0.00614678132152057\\
12.0530555555556	0.00614678132152057\\
12.0558333333333	0.00548402621688812\\
12.0586111111111	0.00548402621688812\\
12.0613888888889	0.00297962969319959\\
12.0641666666667	0.000911834686163361\\
12.0669444444444	9.47739492141875e-05\\
12.0697222222222	0.00503631722103495\\
12.0725	0.00503631722103495\\
12.0752777777778	0.00382436217873982\\
12.0780555555556	0.00262512117918696\\
12.0808333333333	0.00667986109427873\\
12.0836111111111	0.00667986109427873\\
12.0863888888889	0.00631549742461216\\
12.0891666666667	0.00631549742461216\\
12.0919444444444	0.00452379991389094\\
12.0947222222222	0.0046856628520772\\
12.0975	0.0046856628520772\\
12.1002777777778	0.0046856628520772\\
12.1030555555556	0.0059316972504784\\
12.1058333333333	0.0059316972504784\\
12.1086111111111	0.0059316972504784\\
12.1113888888889	0.00539670431548958\\
12.1141666666667	0.00202375325334897\\
12.1169444444444	0.003605403800326\\
12.1197222222222	0.003605403800326\\
12.1225	0.00355750881025263\\
12.1252777777778	0.00355750881025263\\
12.1280555555556	0.00355750881025263\\
12.1308333333333	0.00355750881025263\\
12.1336111111111	0.00355750881025263\\
12.1363888888889	0.00326562204889629\\
12.1391666666667	0.00183489249311463\\
12.1419444444444	0.00198370489853946\\
12.1447222222222	0.00198370489853946\\
12.1475	0.00198370489853946\\
12.1502777777778	0.000867106010185817\\
12.1530555555556	0.00447702938647501\\
12.1558333333333	0.00457077514469662\\
12.1586111111111	0.00457077514469662\\
12.1613888888889	0.00457077514469662\\
12.1641666666667	0.00457077514469662\\
12.1669444444444	0.00457077514469662\\
12.1697222222222	0.00457077514469662\\
12.1725	0.00457077514469662\\
12.1752777777778	0.00457077514469662\\
12.1780555555556	0.00457077514469662\\
12.1808333333333	0.00457077514469662\\
12.1836111111111	0.00457077514469662\\
12.1863888888889	0.00598995816439077\\
12.1891666666667	0.00598995816439077\\
12.1919444444444	0.00598995816439077\\
12.1947222222222	0.00510463549089176\\
12.1975	0.00510463549089176\\
12.2002777777778	0.00510463549089176\\
12.2030555555556	0.00510463549089176\\
12.2058333333333	0.00510463549089176\\
12.2086111111111	0.00510463549089176\\
12.2113888888889	0.00510463549089176\\
12.2141666666667	0.00510463549089176\\
12.2169444444444	0.00510463549089176\\
12.2197222222222	0.00510463549089176\\
12.2225	0.00510463549089176\\
12.2252777777778	0.00440892323297215\\
12.2280555555556	0.00440892323297215\\
12.2308333333333	0.00440892323297215\\
12.2336111111111	0.00440892323297215\\
12.2363888888889	0.00440892323297215\\
12.2391666666667	0.00440892323297215\\
12.2419444444444	0.00059793770191835\\
12.2447222222222	0.00059793770191835\\
12.2475	0.00059793770191835\\
12.2502777777778	0.00211984710612638\\
12.2530555555556	0.00105701252844185\\
12.2558333333333	0.0035413390031332\\
12.2586111111111	0.0035413390031332\\
12.2613888888889	0.0035413390031332\\
12.2641666666667	0.00139593025850426\\
12.2669444444444	0.00139593025850426\\
12.2697222222222	0.00139593025850426\\
12.2725	0.00139593025850426\\
12.2752777777778	0.00139593025850426\\
12.2780555555556	0.00139593025850426\\
12.2808333333333	0.00139593025850426\\
12.2836111111111	0.000323594998057723\\
12.2863888888889	0.000212541837416449\\
12.2891666666667	0.000212541837416449\\
12.2919444444444	0.000212541837416449\\
12.2947222222222	0.000212541837416449\\
12.2975	0.000212541837416449\\
12.3002777777778	0.000212541837416449\\
12.3030555555556	-0.000689606345487235\\
12.3058333333333	0.00390195342461383\\
12.3086111111111	0.00390195342461383\\
12.3113888888889	0.00390195342461383\\
12.3141666666667	0.00390195342461383\\
12.3169444444444	0.00390195342461383\\
12.3197222222222	0.00381872154018782\\
12.3225	0.00381872154018782\\
12.3252777777778	0.00381872154018782\\
12.3280555555556	0.00381872154018782\\
12.3308333333333	0.00381872154018782\\
12.3336111111111	0.00410233527281322\\
12.3363888888889	0.0029029263221929\\
12.3391666666667	0.00239543940060359\\
12.3419444444444	0.00239543940060359\\
12.3447222222222	0.00482836053877905\\
12.3475	0.00392325039470508\\
12.3502777777778	0.00392325039470508\\
12.3530555555556	0.00392325039470508\\
12.3558333333333	0.00392325039470508\\
12.3586111111111	0.00392325039470508\\
12.3613888888889	0.00392325039470508\\
12.3641666666667	0.00392325039470508\\
12.3669444444444	0.00489146983149748\\
12.3697222222222	0.00390833764551548\\
12.3725	0.00390833764551548\\
12.3752777777778	0.00390833764551548\\
12.3780555555556	0.00390833764551548\\
12.3808333333333	0.00390833764551548\\
12.3836111111111	0.00390833764551548\\
12.3863888888889	0.00390833764551548\\
12.3891666666667	0.00223721493641641\\
12.3919444444444	0.00223721493641641\\
12.3947222222222	0.00223721493641641\\
12.3975	0.00223721493641641\\
12.4002777777778	0.00223721493641641\\
12.4030555555556	0.00223721493641641\\
12.4058333333333	0.00223721493641641\\
12.4086111111111	0.00223721493641641\\
12.4113888888889	0.00223721493641641\\
12.4141666666667	0.00223721493641641\\
12.4169444444444	0.00223721493641641\\
12.4197222222222	0.000576102680043013\\
12.4225	0.000484451104098818\\
12.4252777777778	0.000484451104098818\\
12.4280555555556	0.000484451104098818\\
12.4308333333333	0.000360618008078005\\
12.4336111111111	0.000360618008078005\\
12.4363888888889	0.00348707370382108\\
12.4391666666667	0.00274845341650275\\
12.4419444444444	0.00276632468276482\\
12.4447222222222	0.00277914869057244\\
12.4475	0.00277914869057244\\
12.4502777777778	0.00277914869057244\\
12.4530555555556	0.00277914869057244\\
12.4558333333333	0.00277914869057244\\
12.4586111111111	0.00277914869057244\\
12.4613888888889	0.00277914869057244\\
12.4641666666667	0.00249483239107479\\
12.4669444444444	0.00147967477085102\\
12.4697222222222	0.00165240950598957\\
12.4725	-0.000703404385445056\\
12.4752777777778	-0.00149488407128198\\
12.4780555555556	-0.00149488407128198\\
12.4808333333333	-0.00149488407128198\\
12.4836111111111	-0.00149488407128198\\
12.4863888888889	-0.00149488407128198\\
12.4891666666667	-0.00326100220360845\\
12.4919444444444	-0.00326100220360845\\
12.4947222222222	-0.00326100220360845\\
12.4975	-0.00326100220360845\\
12.5002777777778	0.000726854221455554\\
12.5030555555556	0.000726854221455554\\
12.5058333333333	0.000729029681848667\\
12.5086111111111	0.000729029681848667\\
12.5113888888889	0.000729029681848667\\
12.5141666666667	0.000729029681848667\\
12.5169444444444	0.000729029681848667\\
12.5197222222222	0.0018736161692628\\
12.5225	0.0018736161692628\\
12.5252777777778	0.0018736161692628\\
12.5280555555556	0.00148135429912668\\
12.5308333333333	0.00294068483871479\\
12.5336111111111	0.00294068483871479\\
12.5363888888889	0.00294068483871479\\
12.5391666666667	0.00377583398683833\\
12.5419444444444	0.00333898665188898\\
12.5447222222222	0.00333898665188898\\
12.5475	0.00216861887741003\\
12.5502777777778	0.00216861887741003\\
12.5530555555556	0.00244322740779447\\
12.5558333333333	0.00244322740779447\\
12.5586111111111	0.00244322740779447\\
12.5613888888889	0.00244322740779447\\
12.5641666666667	0.00244322740779447\\
12.5669444444444	0.00244322740779447\\
12.5697222222222	0.00331730819353493\\
12.5725	0.00331730819353493\\
12.5752777777778	0.00331730819353493\\
12.5780555555556	0.00331730819353493\\
12.5808333333333	0.00331730819353493\\
12.5836111111111	0.00299018305133206\\
12.5863888888889	0.00299018305133206\\
12.5891666666667	0.00299018305133206\\
12.5919444444444	0.00299018305133206\\
12.5947222222222	0.00251002934747501\\
12.5975	0.00251002934747501\\
12.6002777777778	0.00251002934747501\\
12.6030555555556	0.00251002934747501\\
12.6058333333333	0.00251002934747501\\
12.6086111111111	-0.000150184681157477\\
12.6113888888889	-0.000150184681157477\\
12.6141666666667	-0.000150184681157477\\
12.6169444444444	0.000436914769650463\\
12.6197222222222	-0.00108030206024858\\
12.6225	-0.00108030206024858\\
12.6252777777778	-0.00108030206024858\\
12.6280555555556	0.0011255254751717\\
12.6308333333333	0.0011255254751717\\
12.6336111111111	0.0011255254751717\\
12.6363888888889	0.0011255254751717\\
12.6391666666667	0.0011255254751717\\
12.6419444444444	0.0011255254751717\\
12.6447222222222	0.0011255254751717\\
12.6475	0.0011255254751717\\
12.6502777777778	0.000300263410497622\\
12.6530555555556	0.000300263410497622\\
12.6558333333333	0.000300263410497622\\
12.6586111111111	0.000300263410497622\\
12.6613888888889	0.000300263410497622\\
12.6641666666667	-0.00011313069726847\\
12.6669444444444	-2.29961524776685e-05\\
12.6697222222222	0.000120284713368756\\
12.6725	0.000120284713368756\\
12.6752777777778	0.000120284713368756\\
12.6780555555556	0.000120284713368756\\
12.6808333333333	0.000329844007235704\\
12.6836111111111	0.0061772562504723\\
12.6863888888889	0.0061772562504723\\
12.6891666666667	0.0061772562504723\\
12.6919444444444	0.0061772562504723\\
12.6947222222222	0.0042609121705605\\
12.6975	0.0042609121705605\\
12.7002777777778	0.00365517141189325\\
12.7030555555556	0.00365517141189325\\
12.7058333333333	0.00365517141189325\\
12.7086111111111	0.00365517141189325\\
12.7113888888889	0.00365517141189325\\
12.7141666666667	0.00365517141189325\\
12.7169444444444	0.00431044241431711\\
12.7197222222222	0.00365517141189325\\
12.7225	0.00365517141189325\\
12.7252777777778	0.00365517141189325\\
12.7280555555556	0.00365517141189325\\
12.7308333333333	0.00365517141189325\\
12.7336111111111	0.00365517141189325\\
12.7363888888889	0.00365517141189325\\
12.7391666666667	0.00365517141189325\\
12.7419444444444	0.00407418709766493\\
12.7447222222222	0.00407418709766493\\
12.7475	0.00407418709766493\\
12.7502777777778	0.00407418709766493\\
12.7530555555556	0.00407418709766493\\
12.7558333333333	0.00407418709766493\\
12.7586111111111	0.00407418709766493\\
12.7613888888889	0.00407418709766493\\
12.7641666666667	0.00407418709766493\\
12.7669444444444	0.00407418709766493\\
12.7697222222222	0.00407418709766493\\
12.7725	0.00407418709766493\\
12.7752777777778	0.00309249637845472\\
12.7780555555556	0.00309249637845472\\
12.7808333333333	0.00309249637845472\\
12.7836111111111	0.00309249637845472\\
12.7863888888889	0.00309249637845472\\
12.7891666666667	0.00309249637845472\\
12.7919444444444	0.0051319563729836\\
12.7947222222222	0.0051319563729836\\
12.7975	0.0051319563729836\\
12.8002777777778	0.0051319563729836\\
12.8030555555556	0.0051319563729836\\
12.8058333333333	0.0051319563729836\\
12.8086111111111	0.0051319563729836\\
12.8113888888889	0.0051319563729836\\
12.8141666666667	0.0051319563729836\\
12.8169444444444	0.0051319563729836\\
12.8197222222222	0.0035375520747483\\
12.8225	0.0035375520747483\\
12.8252777777778	0.0035375520747483\\
12.8280555555556	0.0035375520747483\\
12.8308333333333	0.0035375520747483\\
12.8336111111111	0.0035375520747483\\
12.8363888888889	0.0035375520747483\\
12.8391666666667	0.0035375520747483\\
12.8419444444444	0.0035375520747483\\
12.8447222222222	0.0034361021827602\\
12.8475	0.00204301788145984\\
12.8502777777778	0.00204301788145984\\
12.8530555555556	0.00204301788145984\\
12.8558333333333	0.00464848948027933\\
12.8586111111111	0.00464848948027933\\
12.8613888888889	0.0033875840449882\\
12.8641666666667	0.00235981260024558\\
12.8669444444444	0.00235981260024558\\
12.8697222222222	0.00235981260024558\\
12.8725	0.00235981260024558\\
12.8752777777778	0.00535148114711861\\
12.8780555555556	0.00535148114711861\\
12.8808333333333	0.00535148114711861\\
12.8836111111111	0.00479893198523574\\
12.8863888888889	0.00479893198523574\\
12.8891666666667	0.00479893198523574\\
12.8919444444444	0.00433580382094365\\
12.8947222222222	0.00433580382094365\\
12.8975	0.0041334480013439\\
12.9002777777778	0.0041334480013439\\
12.9030555555556	0.00327525682360822\\
12.9058333333333	0.00327525682360822\\
12.9086111111111	0.00327525682360822\\
12.9113888888889	0.00364678717689901\\
12.9141666666667	0.00391835784268795\\
12.9169444444444	0.00663211095297823\\
12.9197222222222	0.00663211095297823\\
12.9225	0.00663211095297823\\
12.9252777777778	0.00663211095297823\\
12.9280555555556	0.00663211095297823\\
12.9308333333333	0.00663211095297823\\
12.9336111111111	0.00605933311640283\\
12.9363888888889	0.00605933311640283\\
12.9391666666667	0.00570355024899148\\
12.9419444444444	0.00570355024899148\\
12.9447222222222	0.00570355024899148\\
12.9475	0.00574858126157994\\
12.9502777777778	0.00574858126157994\\
12.9530555555556	0.00544193355501364\\
12.9558333333333	0.0056049840227319\\
12.9586111111111	0.00617873925960697\\
12.9613888888889	0.00674138743076513\\
12.9641666666667	0.00674138743076513\\
12.9669444444444	0.00674138743076513\\
12.9697222222222	0.00651103893878321\\
12.9725	0.00651103893878321\\
12.9752777777778	0.00610557106567342\\
12.9780555555556	0.00594214284209794\\
12.9808333333333	0.00594214284209794\\
12.9836111111111	0.00587004972106894\\
12.9863888888889	0.00587004972106894\\
12.9891666666667	0.00587004972106894\\
12.9919444444444	0.00587004972106894\\
12.9947222222222	0.00587004972106894\\
12.9975	0.00587004972106894\\
13.0002777777778	0.00585810182619388\\
13.0030555555556	0.00585810182619388\\
13.0058333333333	0.00585810182619388\\
13.0086111111111	0.00585810182619388\\
13.0113888888889	0.00585810182619388\\
13.0141666666667	0.00585810182619388\\
13.0169444444444	0.00585810182619388\\
13.0197222222222	0.00585810182619388\\
13.0225	0.00585810182619388\\
13.0252777777778	0.00566717588066908\\
13.0280555555556	0.00541140219040953\\
13.0308333333333	0.00541140219040953\\
13.0336111111111	0.0054492101824868\\
13.0363888888889	0.0054492101824868\\
13.0391666666667	0.0054492101824868\\
13.0419444444444	0.0054492101824868\\
13.0447222222222	0.0054492101824868\\
13.0475	0.0054492101824868\\
13.0502777777778	0.00599736975924\\
13.0530555555556	0.00599736975924\\
13.0558333333333	0.00599736975924\\
13.0586111111111	0.00568782123733002\\
13.0613888888889	0.00568782123733002\\
13.0641666666667	0.00568782123733002\\
13.0669444444444	0.00568782123733002\\
13.0697222222222	0.00568782123733002\\
13.0725	0.00568782123733002\\
13.0752777777778	0.00558579146806008\\
13.0780555555556	0.00558579146806008\\
13.0808333333333	0.00558579146806008\\
13.0836111111111	0.00558579146806008\\
13.0863888888889	0.00558579146806008\\
13.0891666666667	0.00558579146806008\\
13.0919444444444	0.00558579146806008\\
13.0947222222222	0.00485440239574106\\
13.0975	0.00485440239574106\\
13.1002777777778	0.00485440239574106\\
13.1030555555556	0.00485440239574106\\
13.1058333333333	0.00511462382637362\\
13.1086111111111	0.00511462382637362\\
13.1113888888889	0.00511462382637362\\
13.1141666666667	0.00511462382637362\\
13.1169444444444	0.00562071959686531\\
13.1197222222222	0.00550612981685622\\
13.1225	0.00329349339485012\\
13.1252777777778	0.00329349339485012\\
13.1280555555556	0.00248257212936756\\
13.1308333333333	0.00248257212936756\\
13.1336111111111	0.00248257212936756\\
13.1363888888889	0.00275080073671169\\
13.1391666666667	0.00275080073671169\\
13.1419444444444	0.00275080073671169\\
13.1447222222222	0.00275080073671169\\
13.1475	0.00275080073671169\\
13.1502777777778	0.00257139236742486\\
13.1530555555556	0.00257139236742486\\
13.1558333333333	0.00230083876470788\\
13.1586111111111	0.0024383804610772\\
13.1613888888889	0.00183777764751254\\
13.1641666666667	0.00183777764751254\\
13.1669444444444	0.00296924343547271\\
13.1697222222222	0.00296924343547271\\
13.1725	0.00296924343547271\\
13.1752777777778	0.00296924343547271\\
13.1780555555556	0.00296924343547271\\
13.1808333333333	0.00296924343547271\\
13.1836111111111	0.00296924343547271\\
13.1863888888889	0.00296924343547271\\
13.1891666666667	0.00296924343547271\\
13.1919444444444	0.00296924343547271\\
13.1947222222222	0.00296924343547271\\
13.1975	0.00296924343547271\\
13.2002777777778	0.00296924343547271\\
13.2030555555556	0.00296924343547271\\
13.2058333333333	0.00296924343547271\\
13.2086111111111	0.00296924343547271\\
13.2113888888889	0.00115274481331039\\
13.2141666666667	0.00115274481331039\\
13.2169444444444	0.00454747422216335\\
13.2197222222222	0.00454747422216335\\
13.2225	0.00454747422216335\\
13.2252777777778	0.00454747422216335\\
13.2280555555556	0.00240735706075813\\
13.2308333333333	-0.000736328942714751\\
13.2336111111111	-0.00322751358919298\\
13.2363888888889	-0.00358268679098103\\
13.2391666666667	-0.00547520301888064\\
13.2419444444444	-0.00505471959882195\\
13.2447222222222	-0.00545684054765904\\
13.2475	-0.00288285710188681\\
13.2502777777778	-0.000308870990520129\\
13.2530555555556	-0.00164308163007005\\
13.2558333333333	0.00244891821720027\\
13.2586111111111	0.00465736057222611\\
13.2613888888889	0.00465736057222611\\
13.2641666666667	0.00465736057222611\\
13.2669444444444	0.00465736057222611\\
13.2697222222222	0.00465736057222611\\
13.2725	0.00324889491091555\\
13.2752777777778	0.00535439062019299\\
13.2780555555556	0.00524515640375899\\
13.2808333333333	0.0196837267142269\\
13.2836111111111	0.0181567052271508\\
13.2863888888889	0.0193935762431854\\
13.2891666666667	0.0193935762431854\\
13.2919444444444	0.0175510543294101\\
13.2947222222222	0.0175510543294101\\
13.2975	0.0175510543294101\\
13.3002777777778	0.0175510543294101\\
13.3030555555556	0.0194181421196792\\
13.3058333333333	0.0178646756479301\\
13.3086111111111	0.0190485318234364\\
13.3113888888889	0.0173017943851157\\
13.3141666666667	0.00677452881411743\\
13.3169444444444	0.00677452881411743\\
13.3197222222222	0.00677452881411743\\
13.3225	0.00398629625555693\\
13.3252777777778	0.00398629625555693\\
13.3280555555556	-0.0041532595792259\\
13.3308333333333	-0.0041532595792259\\
13.3336111111111	0.0101670114649716\\
13.3363888888889	0.0101670114649716\\
13.3391666666667	0.00227834560340112\\
13.3419444444444	-0.00257032629787179\\
13.3447222222222	-0.0029533298224713\\
13.3475	-0.0035059756487873\\
13.3502777777778	-0.0035059756487873\\
13.3530555555556	-0.0035059756487873\\
13.3558333333333	-0.0035059756487873\\
13.3586111111111	-0.0035059756487873\\
13.3613888888889	-0.0035059756487873\\
13.3641666666667	-0.0035059756487873\\
13.3669444444444	-0.0035059756487873\\
13.3697222222222	-0.0035059756487873\\
13.3725	-0.0035059756487873\\
13.3752777777778	0.00540441031592567\\
13.3780555555556	0.00540441031592567\\
13.3808333333333	0.003056714136547\\
13.3836111111111	-0.00285419853915299\\
13.3863888888889	-0.00285419853915299\\
13.3891666666667	-0.00285419853915299\\
13.3919444444444	-0.00817091186924699\\
13.3947222222222	-0.00817091186924699\\
13.3975	-0.0086174774140342\\
13.4002777777778	-0.0162941439403942\\
13.4030555555556	-0.0195146347168337\\
13.4058333333333	-0.0200472601684522\\
13.4086111111111	-0.0173550021038588\\
13.4113888888889	-0.0173550021038588\\
13.4141666666667	-0.0216862545303849\\
13.4169444444444	-0.0214863158554788\\
13.4197222222222	-0.0214863158554788\\
13.4225	-0.0214863158554788\\
13.4252777777778	-0.0175211770232975\\
13.4280555555556	-0.0175211770232975\\
13.4308333333333	-0.0156231633071625\\
13.4336111111111	-0.0184507131736571\\
13.4363888888889	-0.0184507131736571\\
13.4391666666667	-0.0184507131736571\\
13.4419444444444	-0.0184507131736571\\
13.4447222222222	-0.0184507131736571\\
13.4475	-0.0184507131736571\\
13.4502777777778	-0.0184507131736571\\
13.4530555555556	-0.0231249933425059\\
13.4558333333333	-0.0231249933425059\\
13.4586111111111	-0.0231249933425059\\
13.4613888888889	-0.0231249933425059\\
13.4641666666667	-0.0183303480929262\\
13.4669444444444	-0.0235551026702636\\
13.4697222222222	-0.00163389928491944\\
13.4725	0.00376048970146869\\
13.4752777777778	0.00376048970146869\\
13.4780555555556	0.00376048970146869\\
13.4808333333333	0.00376048970146869\\
13.4836111111111	0.00376048970146869\\
13.4863888888889	0.00339409552033214\\
13.4891666666667	0.00339409552033214\\
13.4919444444444	0.00183395906951881\\
13.4947222222222	0.00183395906951881\\
13.4975	-3.68228112807446e-05\\
13.5002777777778	-0.0033386453743738\\
13.5030555555556	-0.0033386453743738\\
13.5058333333333	-0.00373457066282618\\
13.5086111111111	-0.00425583370567932\\
13.5113888888889	-0.00335770007611736\\
13.5141666666667	-0.00681096003971388\\
13.5169444444444	-0.00681096003971388\\
13.5197222222222	-0.00245995094739211\\
13.5225	-0.00245995094739211\\
13.5252777777778	-0.00245995094739211\\
13.5280555555556	-0.00245995094739211\\
13.5308333333333	-0.00172616790722044\\
13.5336111111111	-0.00065366964272346\\
13.5363888888889	-0.00065366964272346\\
13.5391666666667	-0.00065366964272346\\
13.5419444444444	-0.00065366964272346\\
13.5447222222222	-0.00065366964272346\\
13.5475	0.00421333934065045\\
13.5502777777778	0.00526534382542839\\
13.5530555555556	0.00526534382542839\\
13.5558333333333	0.00500177823973965\\
13.5586111111111	0.00500177823973965\\
13.5613888888889	0.00564857851057934\\
13.5641666666667	0.00564857851057934\\
13.5669444444444	0.00931133905876583\\
13.5697222222222	-0.000414751016054643\\
13.5725	-0.0012417650284747\\
13.5752777777778	-0.0012417650284747\\
13.5780555555556	0.00234731430984772\\
13.5808333333333	0.00234731430984772\\
13.5836111111111	0.00234731430984772\\
13.5863888888889	0.00234731430984772\\
13.5891666666667	0.00234731430984772\\
13.5919444444444	0.00234731430984772\\
13.5947222222222	0.00234731430984772\\
13.5975	0.00234731430984772\\
13.6002777777778	0.00204878214801173\\
13.6030555555556	0.00204878214801173\\
13.6058333333333	0.00204878214801173\\
13.6086111111111	0.00204878214801173\\
13.6113888888889	0.00277739102984638\\
13.6141666666667	0.00277739102984638\\
13.6169444444444	0.00277739102984638\\
13.6197222222222	0.00277739102984638\\
13.6225	0.00277739102984638\\
13.6252777777778	0.00368024135186254\\
13.6280555555556	0.00368024135186254\\
13.6308333333333	0.00368024135186254\\
13.6336111111111	0.00368024135186254\\
13.6363888888889	0.00236716533893134\\
13.6391666666667	0.00236716533893134\\
13.6419444444444	0.00236716533893134\\
13.6447222222222	0.00236716533893134\\
13.6475	0.00236716533893134\\
13.6502777777778	0.00236716533893134\\
13.6530555555556	0.00236716533893134\\
13.6558333333333	0.00236716533893134\\
13.6586111111111	0.00236716533893134\\
13.6613888888889	0.00236716533893134\\
13.6641666666667	0.00236716533893134\\
13.6669444444444	0.00236716533893134\\
13.6697222222222	0.00236716533893134\\
13.6725	0.00236716533893134\\
13.6752777777778	0.00236716533893134\\
13.6780555555556	0.00236716533893134\\
13.6808333333333	0.00236716533893134\\
13.6836111111111	0.00236716533893134\\
13.6863888888889	0.00236716533893134\\
13.6891666666667	0.00199153791828603\\
13.6919444444444	0.00199153791828603\\
13.6947222222222	0.000638906133112509\\
13.6975	0.00141016105533062\\
13.7002777777778	0.000580999936336766\\
13.7030555555556	0.000580999936336766\\
13.7058333333333	0.000580999936336766\\
13.7086111111111	0.000580999936336766\\
13.7113888888889	0.000580999936336766\\
13.7141666666667	0.000580999936336766\\
13.7169444444444	-0.000556670919129762\\
13.7197222222222	0.000720831855274292\\
13.7225	0.000720831855274292\\
13.7252777777778	0.000720831855274292\\
13.7280555555556	0.00453989796402942\\
13.7308333333333	0.00137929214551102\\
13.7336111111111	0.00379438765000782\\
13.7363888888889	0.00379438765000782\\
13.7391666666667	3.48061998137281e-05\\
13.7419444444444	0.000588288928388364\\
13.7447222222222	0.000588288928388364\\
13.7475	0.0048708641144217\\
13.7502777777778	0.0048708641144217\\
13.7530555555556	0.00553022671045944\\
13.7558333333333	0.00553022671045944\\
13.7586111111111	0.00553022671045944\\
13.7613888888889	0.00553022671045944\\
13.7641666666667	0.00553022671045944\\
13.7669444444444	0.00553022671045944\\
13.7697222222222	0.00553022671045944\\
13.7725	0.00553022671045944\\
13.7752777777778	0.00553022671045944\\
13.7780555555556	0.00663685326452774\\
13.7808333333333	0.00663685326452774\\
13.7836111111111	0.00663685326452774\\
13.7863888888889	0.00663685326452774\\
13.7891666666667	0.00564971557435265\\
13.7919444444444	0.00777435973694557\\
13.7947222222222	0.00315746335694485\\
13.7975	0.00315746335694485\\
13.8002777777778	0.0030528918545874\\
13.8030555555556	0.0030528918545874\\
13.8058333333333	0.00203321655934091\\
13.8086111111111	0.00203321655934091\\
13.8113888888889	0.00203321655934091\\
13.8141666666667	0.00203321655934091\\
13.8169444444444	0.00203321655934091\\
13.8197222222222	0.00203321655934091\\
13.8225	0.00203321655934091\\
13.8252777777778	0.00203321655934091\\
13.8280555555556	0.00203321655934091\\
13.8308333333333	0.000928679830659083\\
13.8336111111111	0.000928679830659083\\
13.8363888888889	0.000928679830659083\\
13.8391666666667	0.000928679830659083\\
13.8419444444444	0.000928679830659083\\
13.8447222222222	0.000928679830659083\\
13.8475	0.000928679830659083\\
13.8502777777778	0.000928679830659083\\
13.8530555555556	0.000928679830659083\\
13.8558333333333	0.00398930472649974\\
13.8586111111111	0.00398930472649974\\
13.8613888888889	0.00398930472649974\\
13.8641666666667	0.0036188471193204\\
13.8669444444444	0.0036188471193204\\
13.8697222222222	0.0036188471193204\\
13.8725	0.00156757955823333\\
13.8752777777778	0.00156757955823333\\
13.8780555555556	0.00156757955823333\\
13.8808333333333	0.00134915589075897\\
13.8836111111111	0.00134915589075897\\
13.8863888888889	0.000897805307170612\\
13.8891666666667	0.000705321253523567\\
13.8919444444444	0.000705321253523567\\
13.8947222222222	0.000705321253523567\\
13.8975	0.000705321253523567\\
13.9002777777778	0.000705321253523567\\
13.9030555555556	0.000705321253523567\\
13.9058333333333	0.000705321253523567\\
13.9086111111111	0.000705321253523567\\
13.9113888888889	0.000705321253523567\\
13.9141666666667	0.000705321253523567\\
13.9169444444444	-0.00134861267886728\\
13.9197222222222	-0.00134861267886728\\
13.9225	-0.00165306678233246\\
13.9252777777778	-0.00176981832674794\\
13.9280555555556	-0.00176981832674794\\
13.9308333333333	-0.00176981832674794\\
13.9336111111111	-0.00176981832674794\\
13.9363888888889	-0.00176981832674794\\
13.9391666666667	-0.00331204528805838\\
13.9419444444444	-0.00331204528805838\\
13.9447222222222	-0.00331204528805838\\
13.9475	-0.00331204528805838\\
13.9502777777778	0.00162345187139966\\
13.9530555555556	0.00162345187139966\\
13.9558333333333	0.00162345187139966\\
13.9586111111111	0.00162345187139966\\
13.9613888888889	0.00162345187139966\\
13.9641666666667	0.00162345187139966\\
13.9669444444444	0.00492657259871394\\
13.9697222222222	0.00492657259871394\\
13.9725	0.00302939814835552\\
13.9752777777778	0.00302939814835552\\
13.9780555555556	0.00302939814835552\\
13.9808333333333	0.00302939814835552\\
13.9836111111111	0.00319013401340277\\
13.9863888888889	0.00319013401340277\\
13.9891666666667	0.00319013401340277\\
13.9919444444444	0.00304372160786194\\
13.9947222222222	0.00304372160786194\\
13.9975	0.00165664560629711\\
14.0002777777778	0.000937904345281872\\
14.0030555555556	0.000937904345281872\\
14.0058333333333	0.000937904345281872\\
14.0086111111111	0.0019872507150694\\
14.0113888888889	0.00082451079043515\\
14.0141666666667	-0.00202244062898492\\
14.0169444444444	-0.00202244062898492\\
14.0197222222222	-0.00240418332486366\\
14.0225	-0.0029747809840734\\
14.0252777777778	-0.0029747809840734\\
14.0280555555556	-0.00640014803174247\\
14.0308333333333	-0.00872922047663286\\
14.0336111111111	-0.00529469094087328\\
14.0363888888889	-0.00529469094087328\\
14.0391666666667	-0.00529469094087328\\
14.0419444444444	-0.00529469094087328\\
14.0447222222222	-0.00359071109938649\\
14.0475	-0.00692763340692137\\
14.0502777777778	-0.00730711189898279\\
14.0530555555556	-0.0107400470629649\\
14.0558333333333	-0.0100038975501232\\
14.0586111111111	-0.0101108789247053\\
14.0613888888889	-0.0104782646347996\\
14.0641666666667	-0.00984909581597903\\
14.0669444444444	-0.00715024541260823\\
14.0697222222222	-0.00455954188878772\\
14.0725	-0.0035892539824822\\
14.0752777777778	-0.0035892539824822\\
14.0780555555556	-0.00477429220572614\\
14.0808333333333	-0.00488127358030832\\
14.0836111111111	-0.00488127358030832\\
14.0863888888889	-0.00488127358030832\\
14.0891666666667	-0.00488127358030832\\
14.0919444444444	-0.00488127358030832\\
14.0947222222222	-0.00488127358030832\\
14.0975	-0.00488127358030832\\
14.1002777777778	-0.00488127358030832\\
14.1030555555556	-0.00488127358030832\\
14.1058333333333	-0.00488127358030832\\
14.1086111111111	-0.00488127358030832\\
14.1113888888889	-0.00488127358030832\\
14.1141666666667	-0.00488127358030832\\
14.1169444444444	-0.00498942059436726\\
14.1197222222222	0.0118417588885876\\
14.1225	0.0118417588885876\\
14.1252777777778	0.0118417588885876\\
14.1280555555556	0.0118417588885876\\
14.1308333333333	0.0118417588885876\\
14.1336111111111	0.0101458846326138\\
14.1363888888889	-0.000278780154585982\\
14.1391666666667	-0.00221028879491506\\
14.1419444444444	-0.00611191909904994\\
14.1447222222222	8.66933190912592e-05\\
14.1475	8.66933190912592e-05\\
14.1502777777778	8.66933190912592e-05\\
14.1530555555556	0.00494182605631271\\
14.1558333333333	0.00694668452619643\\
14.1586111111111	0.00584937735137019\\
14.1613888888889	0.0098558259379169\\
14.1641666666667	0.00648482598713342\\
14.1669444444444	0.00510635925937376\\
14.1697222222222	-0.00229687853521979\\
14.1725	-0.00229687853521979\\
14.1752777777778	-0.00229687853521979\\
14.1780555555556	-0.00246831629708046\\
14.1808333333333	-0.00246831629708046\\
14.1836111111111	-0.0016423527639163\\
14.1863888888889	-0.00380228479100163\\
14.1891666666667	-0.00380228479100163\\
14.1919444444444	-0.00380228479100163\\
14.1947222222222	-0.00611856370212784\\
14.1975	-0.00611856370212784\\
14.2002777777778	-0.0093325333055245\\
14.2030555555556	-0.0093325333055245\\
14.2058333333333	-0.0093325333055245\\
14.2086111111111	-0.0093325333055245\\
14.2113888888889	-0.0093325333055245\\
14.2141666666667	-0.0093325333055245\\
14.2169444444444	-0.0093325333055245\\
14.2197222222222	-0.0093325333055245\\
14.2225	-0.012754108601405\\
14.2252777777778	-0.0117796424933126\\
14.2280555555556	-0.0119326782081059\\
14.2308333333333	-0.0119326782081059\\
14.2336111111111	-0.0150389971520069\\
14.2363888888889	-0.0150389971520069\\
14.2391666666667	-0.0171275825601154\\
14.2419444444444	-0.0192277390403802\\
14.2447222222222	-0.0177137059258736\\
14.2475	-0.0190668670880082\\
14.2502777777778	-0.0190668670880082\\
14.2530555555556	-0.0212869307577212\\
14.2558333333333	-0.0212869307577212\\
14.2586111111111	-0.0212869307577212\\
14.2613888888889	-0.0233978354715119\\
14.2641666666667	-0.0233978354715119\\
14.2669444444444	-0.0241013980861217\\
14.2697222222222	-0.0204722215091566\\
14.2725	-0.0204722215091566\\
14.2752777777778	-0.0225831624141297\\
14.2780555555556	-0.0169521854388367\\
14.2808333333333	-0.0169521854388367\\
14.2836111111111	0.00380009049931134\\
14.2863888888889	0.0035042682889217\\
14.2891666666667	0.00551578893312872\\
14.2919444444444	0.00508323574770207\\
14.2947222222222	0.00372917483286177\\
14.2975	0.00594390511518362\\
14.3002777777778	0.00815643146253784\\
14.3030555555556	0.0101494211111984\\
14.3058333333333	0.0101494211111984\\
14.3086111111111	0.00889771085402088\\
14.3113888888889	0.00878847663758687\\
14.3141666666667	0.00606354627830205\\
14.3169444444444	0.00480947960360692\\
14.3197222222222	0.00459101117073891\\
14.3225	0.00459101117073891\\
14.3252777777778	0.0031182289119984\\
14.3280555555556	0.000499679150044442\\
14.3308333333333	-0.00201056311224114\\
14.3336111111111	-0.00452163904701642\\
14.3363888888889	-0.00452163904701642\\
14.3391666666667	-0.00452163904701642\\
14.3419444444444	-0.00463087326345042\\
14.3447222222222	-0.00588655320970928\\
14.3475	-0.00588655320970928\\
14.3502777777778	-0.00610502164257729\\
14.3530555555556	-0.00610502164257729\\
14.3558333333333	-0.00610502164257729\\
14.3586111111111	-0.00610502164257729\\
14.3613888888889	0.0152454979817546\\
14.3641666666667	0.0152454979817546\\
14.3669444444444	0.0151362637653205\\
14.3697222222222	0.0150264483131463\\
14.3725	0.0134729818413974\\
14.3752777777778	0.00340319571351147\\
14.3780555555556	0.00340319571351147\\
14.3808333333333	0.00591641860333547\\
14.3836111111111	0.00591641860333547\\
14.3863888888889	0.00397793996674262\\
14.3891666666667	0.00397793996674262\\
14.3919444444444	0.00397793996674262\\
14.3947222222222	0.00397793996674262\\
14.3975	0.00219629326991875\\
14.4002777777778	0.00219629326991875\\
14.4030555555556	0.00219629326991875\\
14.4058333333333	0.0109903244115691\\
14.4086111111111	0.0109903244115691\\
14.4113888888889	0.0109903244115691\\
14.4141666666667	0.0109903244115691\\
14.4169444444444	0.00928652558397435\\
14.4197222222222	0.00928652558397435\\
14.4225	0.00928652558397435\\
14.4252777777778	0.0104022926868017\\
14.4280555555556	0.00714830176809584\\
14.4308333333333	0.0065123070443548\\
14.4336111111111	0.0065123070443548\\
14.4363888888889	0.00541678768044115\\
14.4391666666667	0.00260935476542149\\
14.4419444444444	0.00395556322942178\\
14.4447222222222	0.00879208107405929\\
14.4475	0.00879208107405929\\
14.4502777777778	0.00879208107405929\\
14.4530555555556	0.00520684028476593\\
14.4558333333333	0.00712633228374302\\
14.4586111111111	0.00976038720160498\\
14.4613888888889	0.00172290933290073\\
14.4641666666667	0.00172290933290073\\
14.4669444444444	-0.00190494226477562\\
14.4697222222222	-0.00190494226477562\\
14.4725	-0.00190494226477562\\
14.4752777777778	-0.00190494226477562\\
14.4780555555556	-0.000482167627042202\\
14.4808333333333	-0.000482167627042202\\
14.4836111111111	-0.000482167627042202\\
14.4863888888889	-0.000482167627042202\\
14.4891666666667	-0.000482167627042202\\
14.4919444444444	-0.000482167627042202\\
14.4947222222222	-0.000482167627042202\\
14.4975	-0.000482167627042202\\
14.5002777777778	-0.000482167627042202\\
14.5030555555556	0.00190023831251776\\
14.5058333333333	0.00350085607078979\\
14.5086111111111	0.00414882729221133\\
14.5113888888889	0.00414882729221133\\
14.5141666666667	0.00507749646884319\\
14.5169444444444	0.00478319411099675\\
14.5197222222222	0.00386379447899525\\
14.5225	0.00386379447899525\\
14.5252777777778	0.00386379447899525\\
14.5280555555556	0.00386379447899525\\
14.5308333333333	0.00386379447899525\\
14.5336111111111	0.00386379447899525\\
14.5363888888889	0.00386379447899525\\
14.5391666666667	0.00394149162881408\\
14.5419444444444	0.00413124613605189\\
14.5447222222222	0.0043210006432897\\
14.5475	0.00491324227787607\\
14.5502777777778	0.00162771474955936\\
14.5530555555556	0.00162771474955936\\
14.5558333333333	0.00162771474955936\\
14.5586111111111	0.00162771474955936\\
14.5613888888889	0.00162771474955936\\
14.5641666666667	0.00415161311632802\\
14.5669444444444	0.00482876311781616\\
14.5697222222222	0.00482876311781616\\
14.5725	0.00516992502917185\\
14.5752777777778	0.00452028095711026\\
14.5780555555556	0.00185735348111299\\
14.5808333333333	0.00586547203023558\\
14.5836111111111	0.00586547203023558\\
14.5863888888889	0.00336233519590217\\
14.5891666666667	0.00391070968881493\\
14.5919444444444	0.00391070968881493\\
14.5947222222222	0.00372913122292709\\
14.5975	0.00372913122292709\\
14.6002777777778	0.00372913122292709\\
14.6030555555556	0.00372913122292709\\
14.6058333333333	0.00217784260069869\\
14.6086111111111	0.00217784260069869\\
14.6113888888889	0.00217784260069869\\
14.6141666666667	0.00217784260069869\\
14.6169444444444	0.00217784260069869\\
14.6197222222222	0.00402579091357796\\
14.6225	0.00402579091357796\\
14.6252777777778	0.00402579091357796\\
14.6280555555556	0.00402579091357796\\
14.6308333333333	0.00603951462665754\\
14.6336111111111	0.00307669383581814\\
14.6363888888889	0.00307669383581814\\
14.6391666666667	0.00307669383581814\\
14.6419444444444	0.00307669383581814\\
14.6447222222222	0.00307669383581814\\
14.6475	0.00307669383581814\\
14.6502777777778	0.00575462040694407\\
14.6530555555556	0.00594654010487842\\
14.6558333333333	0.00594654010487842\\
14.6586111111111	0.00653486254330803\\
14.6613888888889	0.00653486254330803\\
14.6641666666667	0.00653486254330803\\
14.6669444444444	0.00741544703543594\\
14.6697222222222	0.00568306284327181\\
14.6725	0.00593106344333279\\
14.6752777777778	0.00332377545165446\\
14.6780555555556	0.00414671338272896\\
14.6808333333333	0.0105575838784621\\
14.6836111111111	0.0123625978329006\\
14.6863888888889	0.0123625978329006\\
14.6891666666667	0.0123625978329006\\
14.6919444444444	0.00941501597622642\\
14.6947222222222	0.0098830357999721\\
14.6975	0.0098830357999721\\
14.7002777777778	0.0098830357999721\\
14.7030555555556	0.0098830357999721\\
14.7058333333333	0.0098830357999721\\
14.7086111111111	0.0098830357999721\\
14.7113888888889	0.0098830357999721\\
14.7141666666667	0.0098830357999721\\
14.7169444444444	0.0098830357999721\\
14.7197222222222	0.0144517743999764\\
14.7225	0.0132452427671671\\
14.7252777777778	0.0132452427671671\\
14.7280555555556	0.0154705138951553\\
14.7308333333333	0.0154705138951553\\
14.7336111111111	0.0154705138951553\\
14.7363888888889	0.0170401374229377\\
14.7391666666667	0.020282708682788\\
14.7419444444444	0.00971195178017322\\
14.7447222222222	0.00971195178017322\\
14.7475	0.00687244410300372\\
14.7502777777778	0.0122918960341146\\
14.7530555555556	0.0117138188199422\\
14.7558333333333	0.0117138188199422\\
14.7586111111111	0.0101646017865719\\
14.7613888888889	0.0101646017865719\\
14.7641666666667	0.011026637402245\\
14.7669444444444	0.011026637402245\\
14.7697222222222	0.011026637402245\\
14.7725	0.011026637402245\\
14.7752777777778	0.011026637402245\\
14.7780555555556	0.011026637402245\\
14.7808333333333	0.00657915627603297\\
14.7836111111111	0.0074283154038123\\
14.7863888888889	0.0074283154038123\\
14.7891666666667	0.0074283154038123\\
14.7919444444444	0.0120234248469208\\
14.7947222222222	0.012022253408634\\
14.7975	0.012022253408634\\
14.8002777777778	0.012022253408634\\
14.8030555555556	0.0143953202874254\\
14.8058333333333	0.0172028448397978\\
14.8086111111111	0.0172028448397978\\
14.8113888888889	0.0142483628010671\\
14.8141666666667	0.00955363027006105\\
14.8169444444444	0.0165766032482601\\
14.8197222222222	0.0165766032482601\\
14.8225	0.0136670558760711\\
14.8252777777778	0.00269611028606476\\
14.8280555555556	0.00499779379646524\\
14.8308333333333	0.00499779379646524\\
14.8336111111111	0.00543330366251847\\
14.8363888888889	0.00431773545480857\\
14.8391666666667	0.00431773545480857\\
14.8419444444444	0.00177615547503041\\
14.8447222222222	0.00365170660893266\\
14.8475	0.00365170660893266\\
14.8502777777778	0.00216676305681151\\
14.8530555555556	0.00216676305681151\\
14.8558333333333	0.00216676305681151\\
14.8586111111111	0.00216676305681151\\
14.8613888888889	0.00216676305681151\\
14.8641666666667	0.00216676305681151\\
14.8669444444444	0.00216676305681151\\
14.8697222222222	0.00216676305681151\\
14.8725	0.00216676305681151\\
14.8752777777778	0.00216676305681151\\
14.8780555555556	0.00216676305681151\\
14.8808333333333	0.00216676305681151\\
14.8836111111111	0.00216676305681151\\
14.8863888888889	0.00216676305681151\\
14.8891666666667	0.00216676305681151\\
14.8919444444444	0.00216676305681151\\
14.8947222222222	0.00216676305681151\\
14.8975	0.00505826347954613\\
14.9002777777778	0.0035292978072242\\
14.9030555555556	0.0035292978072242\\
14.9058333333333	0.0035292978072242\\
14.9086111111111	-0.00111992484746892\\
14.9113888888889	-0.00111992484746892\\
14.9141666666667	0.00105465326048484\\
14.9169444444444	-0.00217754726670719\\
14.9197222222222	-0.00151700563966783\\
14.9225	0.00132250203750137\\
14.9252777777778	0.00132250203750137\\
14.9280555555556	0.00132250203750137\\
14.9308333333333	0.00201233002472495\\
14.9336111111111	0.00145286921512833\\
14.9363888888889	0.00145286921512833\\
14.9391666666667	0.00145286921512833\\
14.9419444444444	0.00145286921512833\\
14.9447222222222	0.00145286921512833\\
14.9475	0.00145286921512833\\
14.9502777777778	0.00145286921512833\\
14.9530555555556	0.00502188645573776\\
14.9558333333333	0.00502188645573776\\
14.9586111111111	0.00219465170503617\\
14.9613888888889	0.000416173718232909\\
14.9641666666667	0.00244508422058087\\
14.9669444444444	0.00244508422058087\\
14.9697222222222	-0.000197467698163839\\
14.9725	0.000907882924645682\\
14.9752777777778	0.000907882924645682\\
14.9780555555556	0.000907882924645682\\
14.9808333333333	0.000907882924645682\\
14.9836111111111	0.00109763743188393\\
14.9863888888889	0.00200744767178034\\
14.9891666666667	0.00178902400430596\\
14.9919444444444	0.00246899492087275\\
14.9947222222222	0.000919912341976109\\
14.9975	0.000919912341976109\\
15.0002777777778	0.00206242741189722\\
15.0030555555556	0.00206242741189722\\
15.0058333333333	0.00206242741189722\\
15.0086111111111	0.00206242741189722\\
15.0113888888889	0.00244247620562974\\
15.0141666666667	0.00244247620562974\\
15.0169444444444	0.00244247620562974\\
15.0197222222222	0.00451532192668414\\
15.0225	0.00451532192668414\\
15.0252777777778	0.00451532192668414\\
15.0280555555556	0.00451532192668414\\
15.0308333333333	0.00451532192668414\\
15.0336111111111	0.00451532192668414\\
15.0363888888889	0.00618317089279789\\
15.0391666666667	0.00618317089279789\\
15.0419444444444	-0.000206153841361481\\
15.0447222222222	0.000703951955058277\\
15.0475	0.000703951955058277\\
15.0502777777778	0.000703951955058277\\
15.0530555555556	0.00159764998056705\\
15.0558333333333	0.00159764998056705\\
15.0586111111111	0.00135017048302091\\
15.0613888888889	0.00135017048302091\\
15.0641666666667	-0.000277977658999763\\
15.0669444444444	-0.000277977658999763\\
15.0697222222222	-0.000277977658999763\\
15.0725	-0.00329485714244587\\
15.0752777777778	-0.00329485714244587\\
15.0780555555556	-0.00329485714244587\\
15.0808333333333	-0.00468796874492963\\
15.0836111111111	-0.00601744932818183\\
15.0863888888889	0.00436789048507152\\
15.0891666666667	0.00229504476401754\\
15.0919444444444	0.00280046017461059\\
15.0947222222222	0.000759521985915404\\
15.0975	0.00220973127135182\\
15.1002777777778	-0.000936554933766744\\
15.1030555555556	-0.000936554933766744\\
15.1058333333333	-0.000936554933766744\\
15.1086111111111	-0.00324720040892632\\
15.1113888888889	-0.00226931102478031\\
15.1141666666667	-0.00226931102478031\\
15.1169444444444	-0.00226931102478031\\
15.1197222222222	0.00351118837584558\\
15.1225	0.00351118837584558\\
15.1252777777778	0.00351118837584558\\
15.1280555555556	0.00166375978713299\\
15.1308333333333	0.00277078312562187\\
15.1336111111111	0.00380615353229455\\
15.1363888888889	0.00410227434958774\\
15.1391666666667	0.00410227434958774\\
15.1419444444444	0.0040252005691803\\
15.1447222222222	0.00104083148939188\\
15.1475	0.000705507024817222\\
15.1502777777778	0.000705507024817222\\
15.1530555555556	0.000705507024817222\\
15.1558333333333	0.00377599797888938\\
15.1586111111111	0.00288763552700898\\
15.1613888888889	0.00273389271800472\\
15.1641666666667	0.00420870370441589\\
15.1669444444444	0.00420870370441589\\
15.1697222222222	0.00330660737665563\\
15.1725	0.00400427532332464\\
15.1752777777778	0.00155655454795401\\
15.1780555555556	0.00212275502878484\\
15.1808333333333	0.00212275502878484\\
15.1836111111111	0.00212275502878484\\
15.1863888888889	0.00243796049274954\\
15.1891666666667	0.00243796049274954\\
15.1919444444444	0.00570385024516203\\
15.1947222222222	0.00570385024516203\\
15.1975	0.00570385024516203\\
15.2002777777778	0.0021221762361656\\
15.2030555555556	0.0021221762361656\\
15.2058333333333	0.0021221762361656\\
15.2086111111111	0.00168514394769716\\
15.2113888888889	0.00381300270035104\\
15.2141666666667	0.00381300270035104\\
15.2169444444444	0.00360806274112978\\
15.2197222222222	0.00529230369888332\\
15.2225	0.00599127675814465\\
15.2252777777778	0.00599127675814465\\
15.2280555555556	0.00599127675814465\\
15.2308333333333	0.00405883069167032\\
15.2336111111111	0.00452646289893257\\
15.2363888888889	0.00452646289893257\\
15.2391666666667	0.00452646289893257\\
15.2419444444444	0.00996332618615935\\
15.2447222222222	0.000991694975460498\\
15.2475	0.000991694975460498\\
15.2502777777778	0.00182486608183502\\
15.2530555555556	0.00182486608183502\\
15.2558333333333	0.00286281891405134\\
15.2586111111111	0.00361188793826374\\
15.2613888888889	0.00378836254038758\\
15.2641666666667	0.00378836254038758\\
15.2669444444444	0.00145922098179085\\
15.2697222222222	1.43288453295434e-05\\
15.2725	0.00291996284886856\\
15.2752777777778	0.00135408451351521\\
15.2780555555556	0.000408134552714368\\
15.2808333333333	-0.00177087110526126\\
15.2836111111111	-0.00201465756644368\\
15.2863888888889	-0.00201465756644368\\
15.2891666666667	-0.00201465756644368\\
15.2919444444444	-0.00201465756644368\\
15.2947222222222	-0.00201465756644368\\
15.2975	-0.00201465756644368\\
15.3002777777778	0.00708804912802718\\
15.3030555555556	0.00267749745004028\\
15.3058333333333	-0.00304224330926485\\
15.3086111111111	-0.00304224330926485\\
15.3113888888889	-0.00721779883947648\\
15.3141666666667	-0.00721779883947648\\
15.3169444444444	-0.00643635622326994\\
15.3197222222222	0.010065581055514\\
15.3225	0.00426016004545327\\
15.3252777777778	0.00374584288208707\\
15.3280555555556	0.00195226366231295\\
15.3308333333333	0.00711641288170866\\
15.3336111111111	0.00532283366193468\\
15.3363888888889	0.00530062104169345\\
15.3391666666667	0.00751152023978181\\
15.3419444444444	0.00549013755799956\\
15.3447222222222	0.00144491448645067\\
15.3475	0.00144491448645067\\
15.3502777777778	0.00144491448645067\\
15.3530555555556	-0.00111353251943922\\
15.3558333333333	0.000915715938923031\\
15.3586111111111	0.000915715938923031\\
15.3613888888889	0.000915715938923031\\
15.3641666666667	0.000915715938923031\\
15.3669444444444	-0.00227655887871488\\
15.3697222222222	-0.00621669245237842\\
15.3725	0.0033137928210408\\
15.3752777777778	-0.000195743286142688\\
15.3780555555556	0.00230221992015826\\
15.3808333333333	0.00496684332660936\\
15.3836111111111	0.00496684332660936\\
15.3863888888889	0.00496684332660936\\
15.3891666666667	0.00417715357331273\\
15.3919444444444	0.00640299042542527\\
15.3947222222222	0.0040441714290457\\
15.3975	0.00550774490096449\\
15.4002777777778	0.00550774490096449\\
15.4030555555556	0.00550774490096449\\
15.4058333333333	0.00550774490096449\\
15.4086111111111	0.00550774490096449\\
15.4113888888889	0.00550774490096449\\
15.4141666666667	0.00550774490096449\\
15.4169444444444	0.00250597862393827\\
15.4197222222222	0.00250597862393827\\
15.4225	-0.000669441701650061\\
15.4252777777778	-0.000669441701650061\\
15.4280555555556	-0.000394805946936954\\
15.4308333333333	0.00290220644011499\\
15.4336111111111	0.00290220644011499\\
15.4363888888889	0.00230955756509451\\
15.4391666666667	0.00424649377777867\\
15.4419444444444	0.00548822292490022\\
15.4447222222222	0.00526979925742586\\
15.4475	0.00294775303895538\\
15.4502777777778	0.00388525777692123\\
15.4530555555556	0.00455150467797309\\
15.4558333333333	0.00397838968457087\\
15.4586111111111	0.00538089997184267\\
15.4613888888889	0.00467785158774169\\
15.4641666666667	0.00467785158774169\\
15.4669444444444	0.00501515473550281\\
15.4697222222222	0.00479673106802845\\
15.4725	0.00370446575853605\\
15.4752777777778	0.00288736338916658\\
15.4780555555556	0.00288736338916658\\
15.4808333333333	0.00288736338916658\\
15.4836111111111	0.00288736338916658\\
15.4863888888889	0.00288736338916658\\
15.4891666666667	0.00431751874731633\\
15.4919444444444	0.00254079384355552\\
15.4947222222222	0.00254079384355552\\
15.4975	-0.000138309245832448\\
15.5002777777778	0.00132258627568469\\
15.5030555555556	0.00132258627568469\\
15.5058333333333	0.00132258627568469\\
15.5086111111111	0.00132258627568469\\
15.5113888888889	0.00330985391928158\\
15.5141666666667	0.00367427688526486\\
15.5169444444444	0.00367427688526486\\
15.5197222222222	0.00250935582560294\\
15.5225	0.00206535698176881\\
15.5252777777778	0.00206535698176881\\
15.5280555555556	0.00206535698176881\\
15.5308333333333	0.00418025293419394\\
15.5336111111111	0.00418025293419394\\
15.5363888888889	0.00418025293419394\\
15.5391666666667	0.00398132779376525\\
15.5419444444444	0.00398132779376525\\
15.5447222222222	0.00497952147351419\\
15.5475	0.00497952147351419\\
15.5502777777778	0.00474311927925714\\
15.5530555555556	0.00557750319688053\\
15.5558333333333	0.00363180740505087\\
15.5586111111111	0.00205081820098794\\
15.5613888888889	0.00086691441859179\\
15.5641666666667	0.0078564717777776\\
15.5669444444444	0.0078564717777776\\
15.5697222222222	0.00739711400270751\\
15.5725	0.0070238374113679\\
15.5752777777778	0.0070238374113679\\
15.5780555555556	0.0070238374113679\\
15.5808333333333	0.00792339696470501\\
15.5836111111111	0.00792339696470501\\
15.5863888888889	0.00582346591423322\\
15.5891666666667	0.00582346591423322\\
15.5919444444444	0.00582346591423322\\
15.5947222222222	0.00582346591423322\\
15.5975	0.00582346591423322\\
15.6002777777778	0.00582346591423322\\
15.6030555555556	0.00582346591423322\\
15.6058333333333	0.00574063153286961\\
15.6086111111111	0.00535785165437738\\
15.6113888888889	0.00535785165437738\\
15.6141666666667	0.000244711487799528\\
15.6169444444444	0.000244711487799528\\
15.6197222222222	0.000700939201943532\\
15.6225	0.000705679904255598\\
15.6252777777778	-0.00395688226719898\\
15.6280555555556	-0.00395688226719898\\
15.6308333333333	-0.00395688226719898\\
15.6336111111111	-0.00247237488040544\\
15.6363888888889	-0.00388633317958285\\
15.6391666666667	-0.00747015060619995\\
15.6419444444444	-0.00804889162756441\\
15.6447222222222	-0.00528783723978241\\
15.6475	-0.00528783723978241\\
15.6502777777778	-0.00550966855766194\\
15.6530555555556	-0.00655426564245269\\
15.6558333333333	-0.00651560871467522\\
15.6586111111111	-0.00685913370264967\\
15.6613888888889	-0.00685913370264967\\
15.6641666666667	-0.00853949344714965\\
15.6669444444444	-0.00735018146361757\\
15.6697222222222	-0.00735018146361757\\
15.6725	-0.00707185205320858\\
15.6752777777778	-0.00707185205320858\\
15.6780555555556	-0.00538227685438197\\
15.6808333333333	-0.00538227685438197\\
15.6836111111111	-0.00538227685438197\\
15.6863888888889	-0.00720388021842494\\
15.6891666666667	-0.00720388021842494\\
15.6919444444444	-0.00720388021842494\\
15.6947222222222	-0.00720388021842494\\
15.6975	-0.00528706792508394\\
15.7002777777778	-0.00702275987651073\\
15.7030555555556	-0.00798882249858634\\
15.7058333333333	-0.00826930037524399\\
15.7086111111111	-0.00826930037524399\\
15.7113888888889	-0.00826930037524399\\
15.7141666666667	-0.00826930037524399\\
15.7169444444444	-0.00826930037524399\\
15.7197222222222	-0.00826930037524399\\
15.7225	-0.00805388328108772\\
15.7252777777778	-0.00805388328108772\\
15.7280555555556	-0.00805388328108772\\
15.7308333333333	-0.00805388328108772\\
15.7336111111111	-0.00761940188779862\\
15.7363888888889	-0.00761940188779862\\
15.7391666666667	-0.00761940188779862\\
15.7419444444444	-0.00761940188779862\\
15.7447222222222	-0.00736640463192861\\
15.7475	-0.00688085342804291\\
15.7502777777778	-0.0121691222540295\\
15.7530555555556	-0.0186138218895313\\
15.7558333333333	-0.00446613142467966\\
15.7586111111111	-0.0067771004893813\\
15.7613888888889	-0.0115187890700201\\
15.7641666666667	-0.0147210233410633\\
15.7669444444444	-0.0144508144874143\\
15.7697222222222	-0.00891284121207034\\
15.7725	-0.00936326935237203\\
15.7752777777778	-0.0112521749306121\\
15.7780555555556	-0.0112521749306121\\
15.7808333333333	-0.011989889787995\\
15.7836111111111	-0.0112992538357789\\
15.7863888888889	-0.0109627036011603\\
15.7891666666667	-0.0109627036011603\\
15.7919444444444	-0.0110666631295696\\
15.7947222222222	-0.011364666008207\\
15.7975	-0.011364666008207\\
15.8002777777778	-0.0117221801845837\\
15.8030555555556	-0.0135065467977001\\
15.8058333333333	-0.0142510852999072\\
15.8086111111111	-0.0142510852999072\\
15.8113888888889	-0.0142905316199134\\
15.8141666666667	-0.0144322979139253\\
15.8169444444444	-0.0142522746811408\\
15.8197222222222	-0.0142522746811408\\
15.8225	-0.0142522746811408\\
15.8252777777778	-0.0140894981424989\\
15.8280555555556	-0.0140650415377209\\
15.8308333333333	-0.0140650415377209\\
15.8336111111111	-0.0141522483958324\\
15.8363888888889	-0.0147034082354335\\
15.8391666666667	-0.0147133029882295\\
15.8419444444444	-0.0147133029882295\\
15.8447222222222	-0.0147133029882295\\
15.8475	-0.0147133029882295\\
15.8502777777778	-0.0147133029882295\\
15.8530555555556	-0.0147133029882295\\
15.8558333333333	-0.0147133029882295\\
15.8586111111111	-0.0145779415028689\\
15.8613888888889	-0.0145779415028689\\
15.8641666666667	-0.0145779415028689\\
15.8669444444444	-0.0149866247400564\\
15.8697222222222	-0.0149866247400564\\
15.8725	-0.0158588364682189\\
15.8752777777778	-0.0158588364682189\\
15.8780555555556	-0.0149724260515353\\
15.8808333333333	-0.0149724260515353\\
15.8836111111111	-0.0137389349174272\\
15.8863888888889	-0.0136756584544633\\
15.8891666666667	-0.0136756584544633\\
15.8919444444444	-0.0136756584544633\\
15.8947222222222	-0.0136756584544633\\
15.8975	-0.0106581655424187\\
15.9002777777778	-0.0106581655424187\\
15.9030555555556	-0.0112346661430836\\
15.9058333333333	-0.011476337498077\\
15.9086111111111	-0.011476337498077\\
15.9113888888889	-0.011476337498077\\
15.9141666666667	-0.011476337498077\\
15.9169444444444	-0.0109859241038684\\
15.9197222222222	-0.0108566678989503\\
15.9225	-0.0108566678989503\\
15.9252777777778	-0.0111950720045715\\
15.9280555555556	-0.0109986300721761\\
15.9308333333333	-0.0108728855093979\\
15.9336111111111	-0.0137219810129858\\
15.9363888888889	-0.0140925613664116\\
15.9391666666667	-0.0140925613664116\\
15.9419444444444	-0.0133110673441797\\
15.9447222222222	-0.0124755099046115\\
15.9475	-0.0124755099046115\\
15.9502777777778	-0.0136298452602929\\
15.9530555555556	-0.013254973042008\\
15.9558333333333	-0.013254973042008\\
15.9586111111111	-0.013254973042008\\
15.9613888888889	-0.013254973042008\\
15.9641666666667	-0.013254973042008\\
15.9669444444444	-0.0126613395683223\\
15.9697222222222	-0.0130975055976899\\
15.9725	-0.0130975055976899\\
15.9752777777778	-0.0130975055976899\\
15.9780555555556	-0.0118656978641306\\
15.9808333333333	-0.0118656978641306\\
15.9836111111111	-0.00947214698383887\\
15.9863888888889	-0.00983479984970423\\
15.9891666666667	-0.00914912521660121\\
15.9919444444444	-0.0132319087945714\\
15.9947222222222	-0.0127378739448555\\
15.9975	-0.0142912960173226\\
};
\end{axis}
\end{tikzpicture}%

  \caption{Co-integration relation of the four stochastic control methods.}
  \label{fig:cointeg_relation}
\end{figure}
From the inventory plot in \autoref{fig:ORCL_comp4stoch_inv} we see that the strategies always avoid maintaining zero inventory. This is not surprising: in both the discrete and continuous cases we found that the value function ansatz $h(t,z,q)$ was non-negative, and was equal to zero at zero inventory. The interpretation is that there is no added value to having zero inventory, whereas non-zero inventory can at worst have zero value. Thus it is always profitable, from a value-function standpoint, to have non-zero inventory. Further, we see that the strategies rarely cross the zero-inventory barrier. This is likely attributed to the backtesting algorithm itself, which gives priority to executing buy market orders above sell market orders - we suspect that once the strategy gets into either positive or negative inventory territory, the ansatz function $h$ rarely produces the circumstances to cross the inventory sign barrier by virtue of the non-linear mark-to-market behavior on either side of zero inventory.

Concerning trade execution, the number of executed market orders and filled limit orders generated by each strategy are presented in \autoref{tbl:ORCL_comp4stoch_numt}. The surge in market orders seen for the Cts w nFPC strategy can be explained by the difference in the $\delta^\pm$ plots. As mentioned already, from the stochastic analysis chapter, we know that if $q < 0 $ and $\delta^+ =0$, or if $q > 0$ and $\delta^+ = 1/\kappa$, then we execute a buy MO. Likewise, if $q < 0 $ and $\delta^- =1/\kappa$, or if $q > 0$ and $\delta^- = 0$, then we execute a sell MO. In \autoref{fig:comp_dp_z1} we see that we have $\delta^+ = 0$ for almost all inventory values, and in \autoref{fig:comp_dp_z15} we have  $\delta^+ = 1/\kappa$ for almost all inventory values. This tells us that when the Markov chain state is in one of the non-neutral states, the Cts w nFPC strategy will execute market orders when possible, as it expects prices to move in the corresponding direction. Regarding overall number of trades, it should be noted that we did not include the cost of market order execution in the stochastic control problem. Thus, actual performance would have been negatively affected in proportion to the number of market orders listed.
\begin{table}
\centering
\ra{1.2}
\begin{tabular}{@{} r *{2}{c} @{}}
\toprule
& Market Orders & Limit Orders \\
\midrule
Cts          &  536 & 1280 \\
Cts w nFPC   & 1010 & 1306 \\
Dscr         &  559 & 1285 \\
Dscr w nFPC  &  523 & 1287 \\
\bottomrule
\end{tabular}
\caption{Number of trades comparison of the four stochastic control methods.}
\label{tbl:ORCL_comp4stoch_numt}
\end{table}

To better see how the strategies differ in behaviour, in the figures that follow we show a short sample path on a fine timescale spanning about 2 minutes. In \autoref{fig:samplepath_paths} we plot the midprice path (black line), the optimal posting depths on either side of the real bid/ask prices (gray lines), our execution of market orders (dark blue and dark green), and track incoming external market orders (light blue and light green) that either fill our limit orders (solid lines) or do not (dashed lines). \autoref{fig:samplepath_depths} plots just the optimal depths as they react to the changing Markov state, allowing a better comparison of the behaviours, as well as highlighting the almost-symmetric behaviour between $\delta^+$ and $\delta^-$. \autoref{fig:samplepath_pnl} and \autoref{fig:samplepath_inv} show the effect on PnL and inventory, respectively. 

\begin{figure}
\centering
\begin{subfigure}{.45\linewidth}
  \centering
  \setlength\figureheight{\linewidth} 
  \setlength\figurewidth{\linewidth}
  \tikzsetnextfilename{samplepath_cts_paths}
  % This file was created by matlab2tikz.
%
%The latest updates can be retrieved from
%  http://www.mathworks.com/matlabcentral/fileexchange/22022-matlab2tikz-matlab2tikz
%where you can also make suggestions and rate matlab2tikz.
%
\definecolor{mycolor1}{rgb}{0.65098,0.80784,0.89020}%
\definecolor{mycolor2}{rgb}{0.69804,0.87451,0.54118}%
\definecolor{mycolor3}{rgb}{0.20000,0.62745,0.17255}%
\definecolor{mycolor4}{rgb}{0.12157,0.47059,0.70588}%
%
\begin{tikzpicture}[trim axis left, trim axis right]

\begin{axis}[%
width=\figurewidth,
height=\figureheight,
at={(0\figurewidth,0\figureheight)},
scale only axis,
every outer x axis line/.append style={black},
every x tick label/.append style={font=\color{black}},
xmin=10.975,
xmax=11,
every outer y axis line/.append style={black},
every y tick label/.append style={font=\color{black}},
ymin=33.93,
ymax=34.03,
axis background/.style={fill=white},
axis x line*=bottom,
axis y line*=left,
legend style={legend cell align=left,align=left,draw=black,font=\footnotesize,legend pos=south west},
every axis legend/.code={\renewcommand\addlegendentry[2][]{}}  %ignore legend locally
]
\addplot [color=black,solid,line width=3.0pt]
  table[row sep=crcr]{%
10.975	33.995\\
10.9752777777778	33.995\\
10.9755555555556	33.995\\
10.9758333333333	33.995\\
10.9761111111111	33.995\\
10.9763888888889	33.995\\
10.9766666666667	33.995\\
10.9769444444444	33.995\\
10.9772222222222	33.995\\
10.9775	33.995\\
10.9777777777778	33.995\\
10.9780555555556	33.99\\
10.9783333333333	33.99\\
10.9786111111111	33.99\\
10.9788888888889	33.99\\
10.9791666666667	33.99\\
10.9794444444444	33.995\\
10.9797222222222	33.995\\
10.98	33.995\\
10.9802777777778	33.99\\
10.9805555555556	33.99\\
10.9808333333333	33.99\\
10.9811111111111	33.995\\
10.9813888888889	33.985\\
10.9816666666667	33.985\\
10.9819444444444	33.985\\
10.9822222222222	33.985\\
10.9825	33.985\\
10.9827777777778	33.985\\
10.9830555555556	33.985\\
10.9833333333333	33.985\\
10.9836111111111	33.985\\
10.9838888888889	33.985\\
10.9841666666667	33.985\\
10.9844444444444	33.98\\
10.9847222222222	33.98\\
10.985	33.985\\
10.9852777777778	33.985\\
10.9855555555556	33.98\\
10.9858333333333	33.985\\
10.9861111111111	33.985\\
10.9863888888889	33.975\\
10.9866666666667	33.975\\
10.9869444444444	33.975\\
10.9872222222222	33.975\\
10.9875	33.975\\
10.9877777777778	33.975\\
10.9880555555556	33.975\\
10.9883333333333	33.975\\
10.9886111111111	33.975\\
10.9888888888889	33.975\\
10.9891666666667	33.97\\
10.9894444444444	33.97\\
10.9897222222222	33.97\\
10.99	33.97\\
10.9902777777778	33.97\\
10.9905555555556	33.97\\
10.9908333333333	33.97\\
10.9911111111111	33.975\\
10.9913888888889	33.97\\
10.9916666666667	33.97\\
10.9919444444444	33.97\\
10.9922222222222	33.965\\
10.9925	33.965\\
10.9927777777778	33.965\\
10.9930555555556	33.965\\
10.9933333333333	33.955\\
10.9936111111111	33.965\\
10.9938888888889	33.965\\
10.9941666666667	33.965\\
10.9944444444444	33.975\\
10.9947222222222	33.975\\
10.995	33.965\\
10.9952777777778	33.975\\
10.9955555555556	33.975\\
10.9958333333333	33.975\\
10.9961111111111	33.975\\
10.9963888888889	33.975\\
10.9966666666667	33.975\\
10.9969444444444	33.97\\
10.9972222222222	33.97\\
10.9975	33.975\\
10.9977777777778	33.985\\
10.9980555555556	33.995\\
10.9983333333333	33.995\\
10.9986111111111	33.995\\
10.9988888888889	33.995\\
10.9991666666667	33.995\\
10.9994444444444	33.995\\
10.9997222222222	33.995\\
11	33.995\\
};
\addlegendentry{$S$};

\addplot [color=gray,solid,line width=2.0pt,forget plot]
  table[row sep=crcr]{%
10.975	34.01\\
10.9752777777778	34.01\\
10.9755555555556	34.01\\
10.9758333333333	34.01\\
10.9761111111111	34.01\\
10.9763888888889	34.01\\
10.9766666666667	34.01\\
10.9769444444444	34.01\\
10.9772222222222	34.01\\
10.9775	34.0052561166706\\
10.9777777777778	34.0052561166706\\
10.9780555555556	34.01\\
10.9783333333333	34.0062994730593\\
10.9786111111111	34.0062994730593\\
10.9788888888889	34.0062994730593\\
10.9791666666667	34.0071550219196\\
10.9794444444444	34.0001000871007\\
10.9797222222222	34.0071550219196\\
10.98	34.0036479471728\\
10.9802777777778	34.01\\
10.9805555555556	34.0071550219196\\
10.9808333333333	34.0071550219196\\
10.9811111111111	34.0011202389371\\
10.9813888888889	34\\
10.9816666666667	34\\
10.9819444444444	34\\
10.9822222222222	34\\
10.9825	34\\
10.9827777777778	34\\
10.9830555555556	34\\
10.9833333333333	33.9971550219196\\
10.9836111111111	33.9971550219196\\
10.9838888888889	33.9971550219196\\
10.9841666666667	33.9971550219196\\
10.9844444444444	34\\
10.9847222222222	34\\
10.985	33.9908354309502\\
10.9852777777778	33.9971550219196\\
10.9855555555556	34\\
10.9858333333333	33.9908354309502\\
10.9861111111111	33.9971550219196\\
10.9863888888889	33.99\\
10.9866666666667	33.99\\
10.9869444444444	33.99\\
10.9872222222222	33.99\\
10.9875	33.99\\
10.9877777777778	33.99\\
10.9880555555556	33.99\\
10.9883333333333	33.9886751273379\\
10.9886111111111	33.9873880856129\\
10.9888888888889	33.9857997647327\\
10.9891666666667	33.99\\
10.9894444444444	33.9857997647327\\
10.9897222222222	33.9857997647327\\
10.99	33.9873880856129\\
10.9902777777778	33.9873880856129\\
10.9905555555556	33.9873880856129\\
10.9908333333333	33.9873880856129\\
10.9911111111111	33.9811202389371\\
10.9913888888889	33.99\\
10.9916666666667	33.99\\
10.9919444444444	33.99\\
10.9922222222222	33.98\\
10.9925	33.98\\
10.9927777777778	33.98\\
10.9930555555556	33.98\\
10.9933333333333	33.97\\
10.9936111111111	33.9740236282595\\
10.9938888888889	33.98\\
10.9941666666667	33.98\\
10.9944444444444	33.989227427322\\
10.9947222222222	33.99\\
10.995	33.98\\
10.9952777777778	33.9818010884478\\
10.9955555555556	33.99\\
10.9958333333333	33.99\\
10.9961111111111	33.9873367199577\\
10.9963888888889	33.9852561166706\\
10.9966666666667	33.9852561166706\\
10.9969444444444	33.99\\
10.9972222222222	33.99\\
10.9975	33.9801000871007\\
10.9977777777778	33.992511774143\\
10.9980555555556	34.0049796089352\\
10.9983333333333	34.01\\
10.9986111111111	34.01\\
10.9988888888889	34.01\\
10.9991666666667	34.01\\
10.9994444444444	34.01\\
10.9997222222222	34.01\\
11	34.01\\
};
\addplot [color=gray,solid,line width=2.0pt]
  table[row sep=crcr]{%
10.975	33.99\\
10.9752777777778	33.99\\
10.9755555555556	33.99\\
10.9758333333333	33.99\\
10.9761111111111	33.99\\
10.9763888888889	33.99\\
10.9766666666667	33.99\\
10.9769444444444	33.9871550219196\\
10.9772222222222	33.9871550219196\\
10.9775	33.9847077974139\\
10.9777777777778	33.9847077974139\\
10.9780555555556	33.98\\
10.9783333333333	33.9757997647327\\
10.9786111111111	33.9757997647327\\
10.9788888888889	33.9757997647327\\
10.9791666666667	33.9762994730593\\
10.9794444444444	33.98\\
10.9797222222222	33.9862994730593\\
10.98	33.9833360719777\\
10.9802777777778	33.98\\
10.9805555555556	33.9762994730593\\
10.9808333333333	33.9762994730593\\
10.9811111111111	33.98\\
10.9813888888889	33.98\\
10.9816666666667	33.98\\
10.9819444444444	33.98\\
10.9822222222222	33.98\\
10.9825	33.98\\
10.9827777777778	33.98\\
10.9830555555556	33.98\\
10.9833333333333	33.9762994730593\\
10.9836111111111	33.9762994730593\\
10.9838888888889	33.9762994730593\\
10.9841666666667	33.9762994730593\\
10.9844444444444	33.97\\
10.9847222222222	33.97\\
10.985	33.97\\
10.9852777777778	33.9762994730593\\
10.9855555555556	33.97\\
10.9858333333333	33.97\\
10.9861111111111	33.9762994730593\\
10.9863888888889	33.97\\
10.9866666666667	33.97\\
10.9869444444444	33.97\\
10.9872222222222	33.97\\
10.9875	33.97\\
10.9877777777778	33.97\\
10.9880555555556	33.97\\
10.9883333333333	33.9673880856129\\
10.9886111111111	33.967182933211\\
10.9888888888889	33.96566340596\\
10.9891666666667	33.96\\
10.9894444444444	33.95566340596\\
10.9897222222222	33.95566340596\\
10.99	33.957182933211\\
10.9902777777778	33.957182933211\\
10.9905555555556	33.957182933211\\
10.9908333333333	33.957182933211\\
10.9911111111111	33.96\\
10.9913888888889	33.96\\
10.9916666666667	33.96\\
10.9919444444444	33.96\\
10.9922222222222	33.96\\
10.9925	33.96\\
10.9927777777778	33.96\\
10.9930555555556	33.96\\
10.9933333333333	33.95\\
10.9936111111111	33.9518010884478\\
10.9938888888889	33.96\\
10.9941666666667	33.96\\
10.9944444444444	33.9640236282595\\
10.9947222222222	33.97\\
10.995	33.96\\
10.9952777777778	33.96043693944\\
10.9955555555556	33.9671550219196\\
10.9958333333333	33.9671550219196\\
10.9961111111111	33.9652561166706\\
10.9963888888889	33.9647077974139\\
10.9966666666667	33.9647077974139\\
10.9969444444444	33.96\\
10.9972222222222	33.9586751273379\\
10.9975	33.96\\
10.9977777777778	33.9708354309501\\
10.9980555555556	33.982118374971\\
10.9983333333333	33.99\\
10.9986111111111	33.99\\
10.9988888888889	33.99\\
10.9991666666667	33.9871550219196\\
10.9994444444444	33.9871550219196\\
10.9997222222222	33.9871550219196\\
11	33.9871550219196\\
};
\addlegendentry{$S \pm \delta^\pm$};

\addplot [color=mycolor1,solid,line width=2.0pt,forget plot]
  table[row sep=crcr]{%
10.9791666666667	33.99\\
10.9791666666667	34.0071550219196\\
};
\addplot [color=mycolor1,solid,line width=2.0pt,mark=*,mark options={solid,fill=mycolor1},forget plot]
  table[row sep=crcr]{%
10.9791666666667	34.0071550219196\\
};
\addplot [color=mycolor1,solid,line width=2.0pt,forget plot]
  table[row sep=crcr]{%
10.9811111111111	33.995\\
10.9811111111111	34.0011202389371\\
};
\addplot [color=mycolor1,solid,line width=2.0pt,mark=*,mark options={solid,fill=mycolor1},forget plot]
  table[row sep=crcr]{%
10.9811111111111	34.0011202389371\\
};
\addplot [color=mycolor1,solid,line width=2.0pt,forget plot]
  table[row sep=crcr]{%
10.9913888888889	33.97\\
10.9913888888889	33.99\\
};
\addplot [color=mycolor1,solid,line width=2.0pt,mark=*,mark options={solid,fill=mycolor1},forget plot]
  table[row sep=crcr]{%
10.9913888888889	33.99\\
};
\addplot [color=mycolor1,solid,line width=2.0pt,forget plot]
  table[row sep=crcr]{%
10.9936111111111	33.965\\
10.9936111111111	33.9740236282595\\
};
\addplot [color=mycolor1,solid,line width=2.0pt,mark=*,mark options={solid,fill=mycolor1},forget plot]
  table[row sep=crcr]{%
10.9936111111111	33.9740236282595\\
};
\addplot [color=mycolor1,solid,line width=2.0pt,forget plot]
  table[row sep=crcr]{%
10.9944444444444	33.975\\
10.9944444444444	33.989227427322\\
};
\addplot [color=mycolor1,solid,line width=2.0pt,mark=*,mark options={solid,fill=mycolor1},forget plot]
  table[row sep=crcr]{%
10.9944444444444	33.989227427322\\
};
\addplot [color=mycolor1,solid,line width=2.0pt,forget plot]
  table[row sep=crcr]{%
10.9977777777778	33.985\\
10.9977777777778	33.992511774143\\
};
\addplot [color=mycolor1,solid,line width=2.0pt,mark=*,mark options={solid,fill=mycolor1},forget plot]
  table[row sep=crcr]{%
10.9977777777778	33.992511774143\\
};
\addplot [color=mycolor1,solid,line width=2.0pt,forget plot]
  table[row sep=crcr]{%
10.9980555555556	33.995\\
10.9980555555556	34.0049796089352\\
};
\addplot [color=mycolor1,solid,line width=2.0pt,mark=*,mark options={solid,fill=mycolor1},forget plot]
  table[row sep=crcr]{%
10.9980555555556	34.0049796089352\\
};
\addplot [color=mycolor1,solid,line width=2.0pt,forget plot]
  table[row sep=crcr]{%
10.9983333333333	33.995\\
10.9983333333333	34.01\\
};
\addplot [color=mycolor1,solid,line width=2.0pt,mark=*,mark options={solid,fill=mycolor1}]
  table[row sep=crcr]{%
10.9983333333333	34.01\\
};
\addlegendentry{Ext Buy MO lifts our sell LO};

\addplot [color=mycolor1,dashed,line width=2.0pt,forget plot]
  table[row sep=crcr]{%
10.9911111111111	33.975\\
10.9911111111111	33.98\\
};
\addplot [color=mycolor1,dashed,line width=2.0pt,mark=o,mark options={solid},forget plot]
  table[row sep=crcr]{%
10.9911111111111	33.98\\
};
\addplot [color=mycolor1,dashed,line width=2.0pt,forget plot]
  table[row sep=crcr]{%
10.9952777777778	33.975\\
10.9952777777778	33.97\\
};
\addplot [color=mycolor1,dashed,line width=2.0pt,mark=o,mark options={solid},forget plot]
  table[row sep=crcr]{%
10.9952777777778	33.97\\
};
\addplot [color=mycolor1,dashed,line width=2.0pt,forget plot]
  table[row sep=crcr]{%
10.9975	33.975\\
10.9975	33.98\\
};
\addplot [color=mycolor1,dashed,line width=2.0pt,mark=o,mark options={solid}]
  table[row sep=crcr]{%
10.9975	33.98\\
};
\addlegendentry{Ext Buy MO arrives};

\addplot [color=mycolor2,solid,line width=2.0pt,forget plot]
  table[row sep=crcr]{%
10.9769444444444	33.995\\
10.9769444444444	33.9871550219196\\
};
\addplot [color=mycolor2,solid,line width=2.0pt,mark=*,mark options={solid,fill=mycolor2},forget plot]
  table[row sep=crcr]{%
10.9769444444444	33.9871550219196\\
};
\addplot [color=mycolor2,solid,line width=2.0pt,forget plot]
  table[row sep=crcr]{%
10.9775	33.995\\
10.9775	33.9847077974139\\
};
\addplot [color=mycolor2,solid,line width=2.0pt,mark=*,mark options={solid,fill=mycolor2},forget plot]
  table[row sep=crcr]{%
10.9775	33.9847077974139\\
};
\addplot [color=mycolor2,solid,line width=2.0pt,forget plot]
  table[row sep=crcr]{%
10.9780555555556	33.99\\
10.9780555555556	33.98\\
};
\addplot [color=mycolor2,solid,line width=2.0pt,mark=*,mark options={solid,fill=mycolor2},forget plot]
  table[row sep=crcr]{%
10.9780555555556	33.98\\
};
\addplot [color=mycolor2,solid,line width=2.0pt,forget plot]
  table[row sep=crcr]{%
10.9833333333333	33.985\\
10.9833333333333	33.9762994730593\\
};
\addplot [color=mycolor2,solid,line width=2.0pt,mark=*,mark options={solid,fill=mycolor2},forget plot]
  table[row sep=crcr]{%
10.9833333333333	33.9762994730593\\
};
\addplot [color=mycolor2,solid,line width=2.0pt,forget plot]
  table[row sep=crcr]{%
10.9855555555556	33.98\\
10.9855555555556	33.97\\
};
\addplot [color=mycolor2,solid,line width=2.0pt,mark=*,mark options={solid,fill=mycolor2},forget plot]
  table[row sep=crcr]{%
10.9855555555556	33.97\\
};
\addplot [color=mycolor2,solid,line width=2.0pt,forget plot]
  table[row sep=crcr]{%
10.9880555555556	33.975\\
10.9880555555556	33.97\\
};
\addplot [color=mycolor2,solid,line width=2.0pt,mark=*,mark options={solid,fill=mycolor2},forget plot]
  table[row sep=crcr]{%
10.9880555555556	33.97\\
};
\addplot [color=mycolor2,solid,line width=2.0pt,forget plot]
  table[row sep=crcr]{%
10.9886111111111	33.975\\
10.9886111111111	33.967182933211\\
};
\addplot [color=mycolor2,solid,line width=2.0pt,mark=*,mark options={solid,fill=mycolor2},forget plot]
  table[row sep=crcr]{%
10.9886111111111	33.967182933211\\
};
\addplot [color=mycolor2,solid,line width=2.0pt,forget plot]
  table[row sep=crcr]{%
10.9933333333333	33.955\\
10.9933333333333	33.95\\
};
\addplot [color=mycolor2,solid,line width=2.0pt,mark=*,mark options={solid,fill=mycolor2},forget plot]
  table[row sep=crcr]{%
10.9933333333333	33.95\\
};
\addplot [color=mycolor2,solid,line width=2.0pt,forget plot]
  table[row sep=crcr]{%
10.995	33.965\\
10.995	33.96\\
};
\addplot [color=mycolor2,solid,line width=2.0pt,mark=*,mark options={solid,fill=mycolor2},forget plot]
  table[row sep=crcr]{%
10.995	33.96\\
};
\addplot [color=mycolor2,solid,line width=2.0pt,forget plot]
  table[row sep=crcr]{%
10.9963888888889	33.975\\
10.9963888888889	33.9647077974139\\
};
\addplot [color=mycolor2,solid,line width=2.0pt,mark=*,mark options={solid,fill=mycolor2},forget plot]
  table[row sep=crcr]{%
10.9963888888889	33.9647077974139\\
};
\addplot [color=mycolor2,solid,line width=2.0pt,forget plot]
  table[row sep=crcr]{%
10.9988888888889	33.995\\
10.9988888888889	33.99\\
};
\addplot [color=mycolor2,solid,line width=2.0pt,mark=*,mark options={solid,fill=mycolor2}]
  table[row sep=crcr]{%
10.9988888888889	33.99\\
};
\addlegendentry{Ext Sell MO lifts our buy LO};

\addplot [color=mycolor2,dashed,line width=2.0pt,forget plot]
  table[row sep=crcr]{%
10.9802777777778	33.99\\
10.9802777777778	33.99\\
};
\addplot [color=mycolor2,dashed,line width=2.0pt,mark=o,mark options={solid},forget plot]
  table[row sep=crcr]{%
10.9802777777778	33.99\\
};
\addplot [color=mycolor2,dashed,line width=2.0pt,forget plot]
  table[row sep=crcr]{%
10.9813888888889	33.985\\
10.9813888888889	33.99\\
};
\addplot [color=mycolor2,dashed,line width=2.0pt,mark=o,mark options={solid},forget plot]
  table[row sep=crcr]{%
10.9813888888889	33.99\\
};
\addplot [color=mycolor2,dashed,line width=2.0pt,forget plot]
  table[row sep=crcr]{%
10.9841666666667	33.985\\
10.9841666666667	33.98\\
};
\addplot [color=mycolor2,dashed,line width=2.0pt,mark=o,mark options={solid},forget plot]
  table[row sep=crcr]{%
10.9841666666667	33.98\\
};
\addplot [color=mycolor2,dashed,line width=2.0pt,forget plot]
  table[row sep=crcr]{%
10.9863888888889	33.975\\
10.9863888888889	33.98\\
};
\addplot [color=mycolor2,dashed,line width=2.0pt,mark=o,mark options={solid},forget plot]
  table[row sep=crcr]{%
10.9863888888889	33.98\\
};
\addplot [color=mycolor2,dashed,line width=2.0pt,forget plot]
  table[row sep=crcr]{%
10.9891666666667	33.97\\
10.9891666666667	33.97\\
};
\addplot [color=mycolor2,dashed,line width=2.0pt,mark=o,mark options={solid},forget plot]
  table[row sep=crcr]{%
10.9891666666667	33.97\\
};
\addplot [color=mycolor2,dashed,line width=2.0pt,forget plot]
  table[row sep=crcr]{%
10.9913888888889	33.97\\
10.9913888888889	33.97\\
};
\addplot [color=mycolor2,dashed,line width=2.0pt,mark=o,mark options={solid},forget plot]
  table[row sep=crcr]{%
10.9913888888889	33.97\\
};
\addplot [color=mycolor2,dashed,line width=2.0pt,forget plot]
  table[row sep=crcr]{%
10.9969444444444	33.97\\
10.9969444444444	33.97\\
};
\addplot [color=mycolor2,dashed,line width=2.0pt,mark=o,mark options={solid},forget plot]
  table[row sep=crcr]{%
10.9969444444444	33.97\\
};
\addplot [color=mycolor2,dashed,line width=2.0pt,forget plot]
  table[row sep=crcr]{%
10.9980555555556	33.995\\
10.9980555555556	34\\
};
\addplot [color=mycolor2,dashed,line width=2.0pt,mark=o,mark options={solid}]
  table[row sep=crcr]{%
10.9980555555556	34\\
};
\addlegendentry{Ext Sell MO arrives};

\addplot [color=mycolor3,solid,line width=2.0pt,forget plot]
  table[row sep=crcr]{%
10.975	33.995\\
10.975	34\\
};
\addplot [color=mycolor3,solid,line width=2.0pt,mark=*,mark options={solid,fill=mycolor3},forget plot]
  table[row sep=crcr]{%
10.975	34\\
};
\addplot [color=mycolor3,solid,line width=2.0pt,forget plot]
  table[row sep=crcr]{%
10.9947222222222	33.975\\
10.9947222222222	33.98\\
};
\addplot [color=mycolor3,solid,line width=2.0pt,mark=*,mark options={solid,fill=mycolor3},forget plot]
  table[row sep=crcr]{%
10.9947222222222	33.98\\
};
\addplot [color=mycolor3,solid,line width=2.0pt,forget plot]
  table[row sep=crcr]{%
10.9983333333333	33.995\\
10.9983333333333	34\\
};
\addplot [color=mycolor3,solid,line width=2.0pt,mark=*,mark options={solid,fill=mycolor3}]
  table[row sep=crcr]{%
10.9983333333333	34\\
};
\addlegendentry{Our Buy MO};

\addplot [color=mycolor4,solid,line width=2.0pt,forget plot]
  table[row sep=crcr]{%
10.9858333333333	33.985\\
10.9858333333333	33.98\\
};
\addplot [color=mycolor4,solid,line width=2.0pt,mark=*,mark options={solid,fill=mycolor4},forget plot]
  table[row sep=crcr]{%
10.9858333333333	33.98\\
};
\addplot [color=mycolor4,solid,line width=2.0pt,forget plot]
  table[row sep=crcr]{%
10.9911111111111	33.975\\
10.9911111111111	33.97\\
};
\addplot [color=mycolor4,solid,line width=2.0pt,mark=*,mark options={solid,fill=mycolor4}]
  table[row sep=crcr]{%
10.9911111111111	33.97\\
};
\addlegendentry{Our Sell MO};

\end{axis}
\end{tikzpicture}%

\end{subfigure}%
\hfill%
\begin{subfigure}{.45\linewidth}
  \centering
  \setlength\figureheight{\linewidth} 
  \setlength\figurewidth{\linewidth}
  \tikzsetnextfilename{samplepath_dscr_paths}
  % This file was created by matlab2tikz.
%
%The latest updates can be retrieved from
%  http://www.mathworks.com/matlabcentral/fileexchange/22022-matlab2tikz-matlab2tikz
%where you can also make suggestions and rate matlab2tikz.
%
\definecolor{mycolor1}{rgb}{0.65098,0.80784,0.89020}%
\definecolor{mycolor2}{rgb}{0.69804,0.87451,0.54118}%
\definecolor{mycolor3}{rgb}{0.20000,0.62745,0.17255}%
\definecolor{mycolor4}{rgb}{0.12157,0.47059,0.70588}%
%
\begin{tikzpicture}[trim axis left, trim axis right]

\begin{axis}[%
width=\figurewidth,
height=\figureheight,
at={(0\figurewidth,0\figureheight)},
scale only axis,
every outer x axis line/.append style={black},
every x tick label/.append style={font=\color{black}},
xmin=10.975,
xmax=11,
every outer y axis line/.append style={black},
every y tick label/.append style={font=\color{black}},
xlabel={Time [h]},
ylabel={Price [\$]},
ymin=33.93,
ymax=34.03,
axis background/.style={fill=white},
axis x line*=bottom,
axis y line*=left,
legend style={legend cell align=left,align=left,draw=black,legend pos = south west},
every axis legend/.code={\renewcommand\addlegendentry[2][]{}}  %ignore legend locally
]
\addplot [color=black,solid,line width=3.0pt]
  table[row sep=crcr]{%
10.975	33.995\\
10.9752777777778	33.995\\
10.9755555555556	33.995\\
10.9758333333333	33.995\\
10.9761111111111	33.995\\
10.9763888888889	33.995\\
10.9766666666667	33.995\\
10.9769444444444	33.995\\
10.9772222222222	33.995\\
10.9775	33.995\\
10.9777777777778	33.995\\
10.9780555555556	33.99\\
10.9783333333333	33.99\\
10.9786111111111	33.99\\
10.9788888888889	33.99\\
10.9791666666667	33.99\\
10.9794444444444	33.995\\
10.9797222222222	33.995\\
10.98	33.995\\
10.9802777777778	33.99\\
10.9805555555556	33.99\\
10.9808333333333	33.99\\
10.9811111111111	33.995\\
10.9813888888889	33.985\\
10.9816666666667	33.985\\
10.9819444444444	33.985\\
10.9822222222222	33.985\\
10.9825	33.985\\
10.9827777777778	33.985\\
10.9830555555556	33.985\\
10.9833333333333	33.985\\
10.9836111111111	33.985\\
10.9838888888889	33.985\\
10.9841666666667	33.985\\
10.9844444444444	33.98\\
10.9847222222222	33.98\\
10.985	33.985\\
10.9852777777778	33.985\\
10.9855555555556	33.98\\
10.9858333333333	33.985\\
10.9861111111111	33.985\\
10.9863888888889	33.975\\
10.9866666666667	33.975\\
10.9869444444444	33.975\\
10.9872222222222	33.975\\
10.9875	33.975\\
10.9877777777778	33.975\\
10.9880555555556	33.975\\
10.9883333333333	33.975\\
10.9886111111111	33.975\\
10.9888888888889	33.975\\
10.9891666666667	33.97\\
10.9894444444444	33.97\\
10.9897222222222	33.97\\
10.99	33.97\\
10.9902777777778	33.97\\
10.9905555555556	33.97\\
10.9908333333333	33.97\\
10.9911111111111	33.975\\
10.9913888888889	33.97\\
10.9916666666667	33.97\\
10.9919444444444	33.97\\
10.9922222222222	33.965\\
10.9925	33.965\\
10.9927777777778	33.965\\
10.9930555555556	33.965\\
10.9933333333333	33.955\\
10.9936111111111	33.965\\
10.9938888888889	33.965\\
10.9941666666667	33.965\\
10.9944444444444	33.975\\
10.9947222222222	33.975\\
10.995	33.965\\
10.9952777777778	33.975\\
10.9955555555556	33.975\\
10.9958333333333	33.975\\
10.9961111111111	33.975\\
10.9963888888889	33.975\\
10.9966666666667	33.975\\
10.9969444444444	33.97\\
10.9972222222222	33.97\\
10.9975	33.975\\
10.9977777777778	33.985\\
10.9980555555556	33.995\\
10.9983333333333	33.995\\
10.9986111111111	33.995\\
10.9988888888889	33.995\\
10.9991666666667	33.995\\
10.9994444444444	33.995\\
10.9997222222222	33.995\\
11	33.995\\
};
\addlegendentry{$S$};

\addplot [color=gray,solid,line width=2.0pt,forget plot]
  table[row sep=crcr]{%
10.975	34.0059182180567\\
10.9752777777778	34.0059182180567\\
10.9755555555556	34.0059182180567\\
10.9758333333333	34.0059182180567\\
10.9761111111111	34.0059182180567\\
10.9763888888889	34.0059182180567\\
10.9766666666667	34.0059182180567\\
10.9769444444444	34.0056834278805\\
10.9772222222222	34.0056834278805\\
10.9775	34.003372553231\\
10.9777777777778	34.003372553231\\
10.9780555555556	34.0011849949641\\
10.9783333333333	34.0056119818102\\
10.9786111111111	34.0056119818102\\
10.9788888888889	34.0056119818102\\
10.9791666666667	34.0056119818102\\
10.9794444444444	34.0108655723145\\
10.9797222222222	34.0056119818102\\
10.98	34.0010067763665\\
10.9802777777778	34.0011380381246\\
10.9805555555556	34.005546685923\\
10.9808333333333	34.005546685923\\
10.9811111111111	34.0090040506672\\
10.9813888888889	33.9911380381246\\
10.9816666666667	34.0000275388031\\
10.9819444444444	34.0000275388031\\
10.9822222222222	33.9977551732797\\
10.9825	33.9977551732797\\
10.9827777777778	33.9977551732797\\
10.9830555555556	33.9977551732797\\
10.9833333333333	33.9954944216281\\
10.9836111111111	33.9954944216281\\
10.9838888888889	33.9954944216281\\
10.9841666666667	33.9954684661003\\
10.9844444444444	33.993101665781\\
10.9847222222222	34.0000186652223\\
10.985	34.0011119245755\\
10.9852777777778	33.9954684661003\\
10.9855555555556	33.9910686277899\\
10.9858333333333	34.0011006501926\\
10.9861111111111	33.9954478023027\\
10.9863888888889	33.9810514999642\\
10.9866666666667	33.9900156366446\\
10.9869444444444	33.9900156366446\\
10.9872222222222	33.9900156366446\\
10.9875	33.9900156366446\\
10.9877777777778	33.9900156366446\\
10.9880555555556	33.9900140341327\\
10.9883333333333	33.9876562360065\\
10.9886111111111	33.9876375213784\\
10.9888888888889	33.9853790049406\\
10.9891666666667	33.9810116677959\\
10.9894444444444	33.9853790049406\\
10.9897222222222	33.9853790049406\\
10.99	33.9876375213784\\
10.9902777777778	33.9876375213784\\
10.9905555555556	33.9876375213784\\
10.9908333333333	33.9876375213784\\
10.9911111111111	33.9888198884016\\
10.9913888888889	33.9809874902617\\
10.9916666666667	33.9876161066837\\
10.9919444444444	33.9876161066837\\
10.9922222222222	33.9730379405805\\
10.9925	33.9800102121667\\
10.9927777777778	33.9800102121667\\
10.9930555555556	33.9800102121667\\
10.9933333333333	33.9642875309228\\
10.9936111111111	33.9826749211683\\
10.9938888888889	33.9800102121667\\
10.9941666666667	33.9800102121667\\
10.9944444444444	33.9926749211683\\
10.9947222222222	33.9853499342462\\
10.995	33.9709596128336\\
10.9952777777778	33.9926749211683\\
10.9955555555556	33.9853499342462\\
10.9958333333333	33.9853499342462\\
10.9961111111111	33.9830762963274\\
10.9963888888889	33.9830762963274\\
10.9966666666667	33.9830762963274\\
10.9969444444444	33.9809874902617\\
10.9972222222222	33.9876161066837\\
10.9975	33.9907069403664\\
10.9977777777778	34.0010643450663\\
10.9980555555556	34.0106903541472\\
10.9983333333333	34.0076375213784\\
10.9986111111111	34.0100122475993\\
10.9988888888889	34.0076161066837\\
10.9991666666667	34.0053499342462\\
10.9994444444444	34.0053499342462\\
10.9997222222222	34.0053499342462\\
11	34.0053499342462\\
};
\addplot [color=gray,solid,line width=2.0pt]
  table[row sep=crcr]{%
10.975	33.9856881493956\\
10.9752777777778	33.9856881493956\\
10.9755555555556	33.9856881493956\\
10.9758333333333	33.9856881493956\\
10.9761111111111	33.9856881493956\\
10.9763888888889	33.9856881493956\\
10.9766666666667	33.9856881493956\\
10.9769444444444	33.9856142359539\\
10.9772222222222	33.9856142359539\\
10.9775	33.9832955920654\\
10.9777777777778	33.9832955920654\\
10.9780555555556	33.9711624114765\\
10.9783333333333	33.9755486178901\\
10.9786111111111	33.9755486178901\\
10.9788888888889	33.9755486178901\\
10.9791666666667	33.9755486178901\\
10.9794444444444	33.99\\
10.9797222222222	33.9855486178901\\
10.98	33.9809099219393\\
10.9802777777778	33.9711208098915\\
10.9805555555556	33.9754956793259\\
10.9808333333333	33.9754956793259\\
10.9811111111111	33.9889849747738\\
10.9813888888889	33.9711208098915\\
10.9816666666667	33.98\\
10.9819444444444	33.98\\
10.9822222222222	33.9777226346755\\
10.9825	33.9777226346755\\
10.9827777777778	33.9777226346755\\
10.9830555555556	33.9777226346755\\
10.9833333333333	33.9754692311835\\
10.9836111111111	33.9754692311835\\
10.9838888888889	33.9754692311835\\
10.9841666666667	33.9754484906363\\
10.9844444444444	33.9630985298242\\
10.9847222222222	33.97\\
10.985	33.98\\
10.9852777777778	33.9754484906363\\
10.9855555555556	33.9610612302116\\
10.9858333333333	33.98\\
10.9861111111111	33.9754276427221\\
10.9863888888889	33.9610434964238\\
10.9866666666667	33.97\\
10.9869444444444	33.97\\
10.9872222222222	33.97\\
10.9875	33.97\\
10.9877777777778	33.97\\
10.9880555555556	33.97\\
10.9883333333333	33.9676383300123\\
10.9886111111111	33.9676170353958\\
10.9888888888889	33.965350979661\\
10.9891666666667	33.9510014369277\\
10.9894444444444	33.955350979661\\
10.9897222222222	33.955350979661\\
10.99	33.9576170353958\\
10.9902777777778	33.9576170353958\\
10.9905555555556	33.9576170353958\\
10.9908333333333	33.9576170353958\\
10.9911111111111	33.968814477513\\
10.9913888888889	33.9509756900264\\
10.9916666666667	33.9575924686509\\
10.9919444444444	33.9575924686509\\
10.9922222222222	33.9530335411951\\
10.9925	33.96\\
10.9927777777778	33.96\\
10.9930555555556	33.96\\
10.9933333333333	33.9442861734524\\
10.9936111111111	33.96\\
10.9938888888889	33.96\\
10.9941666666667	33.96\\
10.9944444444444	33.97\\
10.9947222222222	33.9653172698093\\
10.995	33.9509459789529\\
10.9952777777778	33.97\\
10.9955555555556	33.9653172698093\\
10.9958333333333	33.9653172698093\\
10.9961111111111	33.9630430896799\\
10.9963888888889	33.9630430896799\\
10.9966666666667	33.9630430896799\\
10.9969444444444	33.9509756900264\\
10.9972222222222	33.9575924686509\\
10.9975	33.97\\
10.9977777777778	33.98\\
10.9980555555556	33.99\\
10.9983333333333	33.9876170353958\\
10.9986111111111	33.99\\
10.9988888888889	33.9875924686509\\
10.9991666666667	33.9853172698093\\
10.9994444444444	33.9853172698093\\
10.9997222222222	33.9853172698093\\
11	33.9853172698093\\
};
\addlegendentry{$S \pm \delta^\pm$};

\addplot [color=mycolor1,solid,line width=2.0pt,forget plot]
  table[row sep=crcr]{%
10.9811111111111	33.995\\
10.9811111111111	34.0090040506672\\
};
\addplot [color=mycolor1,solid,line width=2.0pt,mark=*,mark options={solid,fill=mycolor1},forget plot]
  table[row sep=crcr]{%
10.9811111111111	34.0090040506672\\
};
\addplot [color=mycolor1,solid,line width=2.0pt,forget plot]
  table[row sep=crcr]{%
10.9936111111111	33.965\\
10.9936111111111	33.9826749211683\\
};
\addplot [color=mycolor1,solid,line width=2.0pt,mark=*,mark options={solid,fill=mycolor1},forget plot]
  table[row sep=crcr]{%
10.9936111111111	33.9826749211683\\
};
\addplot [color=mycolor1,solid,line width=2.0pt,forget plot]
  table[row sep=crcr]{%
10.9952777777778	33.975\\
10.9952777777778	33.9926749211683\\
};
\addplot [color=mycolor1,solid,line width=2.0pt,mark=*,mark options={solid,fill=mycolor1},forget plot]
  table[row sep=crcr]{%
10.9952777777778	33.9926749211683\\
};
\addplot [color=mycolor1,solid,line width=2.0pt,forget plot]
  table[row sep=crcr]{%
10.9975	33.975\\
10.9975	33.9907069403664\\
};
\addplot [color=mycolor1,solid,line width=2.0pt,mark=*,mark options={solid,fill=mycolor1},forget plot]
  table[row sep=crcr]{%
10.9975	33.9907069403664\\
};
\addplot [color=mycolor1,solid,line width=2.0pt,forget plot]
  table[row sep=crcr]{%
10.9983333333333	33.995\\
10.9983333333333	34.0076375213784\\
};
\addplot [color=mycolor1,solid,line width=2.0pt,mark=*,mark options={solid,fill=mycolor1}]
  table[row sep=crcr]{%
10.9983333333333	34.0076375213784\\
};
\addlegendentry{Ext Buy MO lifts our sell LO};

\addplot [color=mycolor1,dashed,line width=2.0pt,forget plot]
  table[row sep=crcr]{%
10.9791666666667	33.99\\
10.9791666666667	34\\
};
\addplot [color=mycolor1,dashed,line width=2.0pt,mark=o,mark options={solid},forget plot]
  table[row sep=crcr]{%
10.9791666666667	34\\
};
\addplot [color=mycolor1,dashed,line width=2.0pt,forget plot]
  table[row sep=crcr]{%
10.9911111111111	33.975\\
10.9911111111111	33.98\\
};
\addplot [color=mycolor1,dashed,line width=2.0pt,mark=o,mark options={solid},forget plot]
  table[row sep=crcr]{%
10.9911111111111	33.98\\
};
\addplot [color=mycolor1,dashed,line width=2.0pt,forget plot]
  table[row sep=crcr]{%
10.9913888888889	33.97\\
10.9913888888889	33.98\\
};
\addplot [color=mycolor1,dashed,line width=2.0pt,mark=o,mark options={solid},forget plot]
  table[row sep=crcr]{%
10.9913888888889	33.98\\
};
\addplot [color=mycolor1,dashed,line width=2.0pt,forget plot]
  table[row sep=crcr]{%
10.9944444444444	33.975\\
10.9944444444444	33.97\\
};
\addplot [color=mycolor1,dashed,line width=2.0pt,mark=o,mark options={solid},forget plot]
  table[row sep=crcr]{%
10.9944444444444	33.97\\
};
\addplot [color=mycolor1,dashed,line width=2.0pt,forget plot]
  table[row sep=crcr]{%
10.9977777777778	33.985\\
10.9977777777778	33.98\\
};
\addplot [color=mycolor1,dashed,line width=2.0pt,mark=o,mark options={solid},forget plot]
  table[row sep=crcr]{%
10.9977777777778	33.98\\
};
\addplot [color=mycolor1,dashed,line width=2.0pt,forget plot]
  table[row sep=crcr]{%
10.9980555555556	33.995\\
10.9980555555556	34\\
};
\addplot [color=mycolor1,dashed,line width=2.0pt,mark=o,mark options={solid}]
  table[row sep=crcr]{%
10.9980555555556	34\\
};
\addlegendentry{Ext Buy MO arrives};

\addplot [color=mycolor2,solid,line width=2.0pt,forget plot]
  table[row sep=crcr]{%
10.9769444444444	33.995\\
10.9769444444444	33.9856142359539\\
};
\addplot [color=mycolor2,solid,line width=2.0pt,mark=*,mark options={solid,fill=mycolor2},forget plot]
  table[row sep=crcr]{%
10.9769444444444	33.9856142359539\\
};
\addplot [color=mycolor2,solid,line width=2.0pt,forget plot]
  table[row sep=crcr]{%
10.9775	33.995\\
10.9775	33.9832955920654\\
};
\addplot [color=mycolor2,solid,line width=2.0pt,mark=*,mark options={solid,fill=mycolor2},forget plot]
  table[row sep=crcr]{%
10.9775	33.9832955920654\\
};
\addplot [color=mycolor2,solid,line width=2.0pt,forget plot]
  table[row sep=crcr]{%
10.9802777777778	33.99\\
10.9802777777778	33.9711208098915\\
};
\addplot [color=mycolor2,solid,line width=2.0pt,mark=*,mark options={solid,fill=mycolor2},forget plot]
  table[row sep=crcr]{%
10.9802777777778	33.9711208098915\\
};
\addplot [color=mycolor2,solid,line width=2.0pt,forget plot]
  table[row sep=crcr]{%
10.9813888888889	33.985\\
10.9813888888889	33.9711208098915\\
};
\addplot [color=mycolor2,solid,line width=2.0pt,mark=*,mark options={solid,fill=mycolor2},forget plot]
  table[row sep=crcr]{%
10.9813888888889	33.9711208098915\\
};
\addplot [color=mycolor2,solid,line width=2.0pt,forget plot]
  table[row sep=crcr]{%
10.9833333333333	33.985\\
10.9833333333333	33.9754692311835\\
};
\addplot [color=mycolor2,solid,line width=2.0pt,mark=*,mark options={solid,fill=mycolor2},forget plot]
  table[row sep=crcr]{%
10.9833333333333	33.9754692311835\\
};
\addplot [color=mycolor2,solid,line width=2.0pt,forget plot]
  table[row sep=crcr]{%
10.9841666666667	33.985\\
10.9841666666667	33.9754484906363\\
};
\addplot [color=mycolor2,solid,line width=2.0pt,mark=*,mark options={solid,fill=mycolor2},forget plot]
  table[row sep=crcr]{%
10.9841666666667	33.9754484906363\\
};
\addplot [color=mycolor2,solid,line width=2.0pt,forget plot]
  table[row sep=crcr]{%
10.9855555555556	33.98\\
10.9855555555556	33.9610612302116\\
};
\addplot [color=mycolor2,solid,line width=2.0pt,mark=*,mark options={solid,fill=mycolor2},forget plot]
  table[row sep=crcr]{%
10.9855555555556	33.9610612302116\\
};
\addplot [color=mycolor2,solid,line width=2.0pt,forget plot]
  table[row sep=crcr]{%
10.9863888888889	33.975\\
10.9863888888889	33.9610434964238\\
};
\addplot [color=mycolor2,solid,line width=2.0pt,mark=*,mark options={solid,fill=mycolor2},forget plot]
  table[row sep=crcr]{%
10.9863888888889	33.9610434964238\\
};
\addplot [color=mycolor2,solid,line width=2.0pt,forget plot]
  table[row sep=crcr]{%
10.9880555555556	33.975\\
10.9880555555556	33.97\\
};
\addplot [color=mycolor2,solid,line width=2.0pt,mark=*,mark options={solid,fill=mycolor2},forget plot]
  table[row sep=crcr]{%
10.9880555555556	33.97\\
};
\addplot [color=mycolor2,solid,line width=2.0pt,forget plot]
  table[row sep=crcr]{%
10.9886111111111	33.975\\
10.9886111111111	33.9676170353958\\
};
\addplot [color=mycolor2,solid,line width=2.0pt,mark=*,mark options={solid,fill=mycolor2},forget plot]
  table[row sep=crcr]{%
10.9886111111111	33.9676170353958\\
};
\addplot [color=mycolor2,solid,line width=2.0pt,forget plot]
  table[row sep=crcr]{%
10.9913888888889	33.97\\
10.9913888888889	33.9509756900264\\
};
\addplot [color=mycolor2,solid,line width=2.0pt,mark=*,mark options={solid,fill=mycolor2},forget plot]
  table[row sep=crcr]{%
10.9913888888889	33.9509756900264\\
};
\addplot [color=mycolor2,solid,line width=2.0pt,forget plot]
  table[row sep=crcr]{%
10.9933333333333	33.955\\
10.9933333333333	33.9442861734524\\
};
\addplot [color=mycolor2,solid,line width=2.0pt,mark=*,mark options={solid,fill=mycolor2},forget plot]
  table[row sep=crcr]{%
10.9933333333333	33.9442861734524\\
};
\addplot [color=mycolor2,solid,line width=2.0pt,forget plot]
  table[row sep=crcr]{%
10.995	33.965\\
10.995	33.9509459789529\\
};
\addplot [color=mycolor2,solid,line width=2.0pt,mark=*,mark options={solid,fill=mycolor2},forget plot]
  table[row sep=crcr]{%
10.995	33.9509459789529\\
};
\addplot [color=mycolor2,solid,line width=2.0pt,forget plot]
  table[row sep=crcr]{%
10.9980555555556	33.995\\
10.9980555555556	33.99\\
};
\addplot [color=mycolor2,solid,line width=2.0pt,mark=*,mark options={solid,fill=mycolor2},forget plot]
  table[row sep=crcr]{%
10.9980555555556	33.99\\
};
\addplot [color=mycolor2,solid,line width=2.0pt,forget plot]
  table[row sep=crcr]{%
10.9988888888889	33.995\\
10.9988888888889	33.9875924686509\\
};
\addplot [color=mycolor2,solid,line width=2.0pt,mark=*,mark options={solid,fill=mycolor2}]
  table[row sep=crcr]{%
10.9988888888889	33.9875924686509\\
};
\addlegendentry{Ext Sell MO lifts our buy LO};

\addplot [color=mycolor2,dashed,line width=2.0pt,forget plot]
  table[row sep=crcr]{%
10.9780555555556	33.99\\
10.9780555555556	33.99\\
};
\addplot [color=mycolor2,dashed,line width=2.0pt,mark=o,mark options={solid},forget plot]
  table[row sep=crcr]{%
10.9780555555556	33.99\\
};
\addplot [color=mycolor2,dashed,line width=2.0pt,forget plot]
  table[row sep=crcr]{%
10.9891666666667	33.97\\
10.9891666666667	33.97\\
};
\addplot [color=mycolor2,dashed,line width=2.0pt,mark=o,mark options={solid},forget plot]
  table[row sep=crcr]{%
10.9891666666667	33.97\\
};
\addplot [color=mycolor2,dashed,line width=2.0pt,forget plot]
  table[row sep=crcr]{%
10.9963888888889	33.975\\
10.9963888888889	33.97\\
};
\addplot [color=mycolor2,dashed,line width=2.0pt,mark=o,mark options={solid},forget plot]
  table[row sep=crcr]{%
10.9963888888889	33.97\\
};
\addplot [color=mycolor2,dashed,line width=2.0pt,forget plot]
  table[row sep=crcr]{%
10.9969444444444	33.97\\
10.9969444444444	33.97\\
};
\addplot [color=mycolor2,dashed,line width=2.0pt,mark=o,mark options={solid}]
  table[row sep=crcr]{%
10.9969444444444	33.97\\
};
\addlegendentry{Ext Sell MO arrives};

\end{axis}
\end{tikzpicture}%
 
\end{subfigure}\\
\vspace{1cm}
\begin{subfigure}{.45\linewidth}
  \centering
  \setlength\figureheight{\linewidth} 
  \setlength\figurewidth{\linewidth}
  \tikzsetnextfilename{samplepath_cts_nFPC_paths}
  % This file was created by matlab2tikz.
%
%The latest updates can be retrieved from
%  http://www.mathworks.com/matlabcentral/fileexchange/22022-matlab2tikz-matlab2tikz
%where you can also make suggestions and rate matlab2tikz.
%
\definecolor{mycolor1}{rgb}{0.65098,0.80784,0.89020}%
\definecolor{mycolor2}{rgb}{0.69804,0.87451,0.54118}%
\definecolor{mycolor3}{rgb}{0.20000,0.62745,0.17255}%
\definecolor{mycolor4}{rgb}{0.12157,0.47059,0.70588}%
%
\begin{tikzpicture}[trim axis left, trim axis right]

\begin{axis}[%
width=\figurewidth,
height=\figureheight,
at={(0\figurewidth,0\figureheight)},
scale only axis,
every outer x axis line/.append style={black},
every x tick label/.append style={font=\color{black}},
xmin=10.975,
xmax=11,
xlabel={Time (h)},
every outer y axis line/.append style={black},
every y tick label/.append style={font=\color{black}},
ymin=33.93,
ymax=34.03,
ylabel={Price},
axis background/.style={fill=white},
axis x line*=bottom,
axis y line*=left,
legend style={legend cell align=left,align=left,draw=black,legend pos = south west},
every axis legend/.code={\renewcommand\addlegendentry[2][]{}}  %ignore legend locally
]
\addplot [color=black,solid,line width=3.0pt]
  table[row sep=crcr]{%
10.975	33.995\\
10.9752777777778	33.995\\
10.9755555555556	33.995\\
10.9758333333333	33.995\\
10.9761111111111	33.995\\
10.9763888888889	33.995\\
10.9766666666667	33.995\\
10.9769444444444	33.995\\
10.9772222222222	33.995\\
10.9775	33.995\\
10.9777777777778	33.995\\
10.9780555555556	33.99\\
10.9783333333333	33.99\\
10.9786111111111	33.99\\
10.9788888888889	33.99\\
10.9791666666667	33.99\\
10.9794444444444	33.995\\
10.9797222222222	33.995\\
10.98	33.995\\
10.9802777777778	33.99\\
10.9805555555556	33.99\\
10.9808333333333	33.99\\
10.9811111111111	33.995\\
10.9813888888889	33.985\\
10.9816666666667	33.985\\
10.9819444444444	33.985\\
10.9822222222222	33.985\\
10.9825	33.985\\
10.9827777777778	33.985\\
10.9830555555556	33.985\\
10.9833333333333	33.985\\
10.9836111111111	33.985\\
10.9838888888889	33.985\\
10.9841666666667	33.985\\
10.9844444444444	33.98\\
10.9847222222222	33.98\\
10.985	33.985\\
10.9852777777778	33.985\\
10.9855555555556	33.98\\
10.9858333333333	33.985\\
10.9861111111111	33.985\\
10.9863888888889	33.975\\
10.9866666666667	33.975\\
10.9869444444444	33.975\\
10.9872222222222	33.975\\
10.9875	33.975\\
10.9877777777778	33.975\\
10.9880555555556	33.975\\
10.9883333333333	33.975\\
10.9886111111111	33.975\\
10.9888888888889	33.975\\
10.9891666666667	33.97\\
10.9894444444444	33.97\\
10.9897222222222	33.97\\
10.99	33.97\\
10.9902777777778	33.97\\
10.9905555555556	33.97\\
10.9908333333333	33.97\\
10.9911111111111	33.975\\
10.9913888888889	33.97\\
10.9916666666667	33.97\\
10.9919444444444	33.97\\
10.9922222222222	33.965\\
10.9925	33.965\\
10.9927777777778	33.965\\
10.9930555555556	33.965\\
10.9933333333333	33.955\\
10.9936111111111	33.965\\
10.9938888888889	33.965\\
10.9941666666667	33.965\\
10.9944444444444	33.975\\
10.9947222222222	33.975\\
10.995	33.965\\
10.9952777777778	33.975\\
10.9955555555556	33.975\\
10.9958333333333	33.975\\
10.9961111111111	33.975\\
10.9963888888889	33.975\\
10.9966666666667	33.975\\
10.9969444444444	33.97\\
10.9972222222222	33.97\\
10.9975	33.975\\
10.9977777777778	33.985\\
10.9980555555556	33.995\\
10.9983333333333	33.995\\
10.9986111111111	33.995\\
10.9988888888889	33.995\\
10.9991666666667	33.995\\
10.9994444444444	33.995\\
10.9997222222222	33.995\\
11	33.995\\
};
\addlegendentry{$S$};

\addplot [color=gray,solid,line width=2.0pt,forget plot]
  table[row sep=crcr]{%
10.975	34.01\\
10.9752777777778	34.01\\
10.9755555555556	34.01\\
10.9758333333333	34.01\\
10.9761111111111	34.01\\
10.9763888888889	34.01\\
10.9766666666667	34.01\\
10.9769444444444	34.01\\
10.9772222222222	34.01\\
10.9775	34.0081732365245\\
10.9777777777778	34.01\\
10.9780555555556	34.0084246177287\\
10.9783333333333	34.0084246177287\\
10.9786111111111	34.0084246177287\\
10.9788888888889	34.0084246177287\\
10.9791666666667	34.000526743864\\
10.9794444444444	34.01\\
10.9797222222222	34.01\\
10.98	34.01\\
10.9802777777778	34.01\\
10.9805555555556	34.01\\
10.9808333333333	34.000526743864\\
10.9811111111111	34.01\\
10.9813888888889	34\\
10.9816666666667	34\\
10.9819444444444	34\\
10.9822222222222	34\\
10.9825	34\\
10.9827777777778	34\\
10.9830555555556	34\\
10.9833333333333	34\\
10.9836111111111	34\\
10.9838888888889	34\\
10.9841666666667	34\\
10.9844444444444	34\\
10.9847222222222	33.9916214350482\\
10.985	34\\
10.9852777777778	34\\
10.9855555555556	33.990526743864\\
10.9858333333333	34\\
10.9861111111111	34\\
10.9863888888889	33.99\\
10.9866666666667	33.99\\
10.9869444444444	33.99\\
10.9872222222222	33.99\\
10.9875	33.99\\
10.9877777777778	33.99\\
10.9880555555556	33.99\\
10.9883333333333	33.99\\
10.9886111111111	33.99\\
10.9888888888889	33.99\\
10.9891666666667	33.9867116094193\\
10.9894444444444	33.9867116094193\\
10.9897222222222	33.9867116094193\\
10.99	33.9886320646659\\
10.9902777777778	33.9886320646659\\
10.9905555555556	33.9886320646659\\
10.9908333333333	33.9805261798833\\
10.9911111111111	33.99\\
10.9913888888889	33.99\\
10.9916666666667	33.99\\
10.9919444444444	33.99\\
10.9922222222222	33.98\\
10.9925	33.98\\
10.9927777777778	33.98\\
10.9930555555556	33.98\\
10.9933333333333	33.9616214350482\\
10.9936111111111	33.98\\
10.9938888888889	33.98\\
10.9941666666667	33.9716214350482\\
10.9944444444444	33.99\\
10.9947222222222	33.99\\
10.995	33.970526743864\\
10.9952777777778	33.99\\
10.9955555555556	33.99\\
10.9958333333333	33.99\\
10.9961111111111	33.99\\
10.9963888888889	33.99\\
10.9966666666667	33.99\\
10.9969444444444	33.99\\
10.9972222222222	33.9805261798833\\
10.9975	33.983421792079\\
10.9977777777778	33.99\\
10.9980555555556	34.0012009559832\\
10.9983333333333	34.0049120415607\\
10.9986111111111	34.0064200267923\\
10.9988888888889	34.0033430597809\\
10.9991666666667	34.0012009559832\\
10.9994444444444	34.0012009559832\\
10.9997222222222	34.0012009559832\\
11	34.0012009559832\\
};
\addplot [color=gray,solid,line width=2.0pt]
  table[row sep=crcr]{%
10.975	33.99\\
10.9752777777778	33.99\\
10.9755555555556	33.99\\
10.9758333333333	33.99\\
10.9761111111111	33.99\\
10.9763888888889	33.99\\
10.9766666666667	33.99\\
10.9769444444444	33.99\\
10.9772222222222	33.99\\
10.9775	33.986063979387\\
10.9777777777778	33.99\\
10.9780555555556	33.9767116094193\\
10.9783333333333	33.9767116094193\\
10.9786111111111	33.9767116094193\\
10.9788888888889	33.9767116094193\\
10.9791666666667	33.97\\
10.9794444444444	33.99\\
10.9797222222222	33.99\\
10.98	33.99\\
10.9802777777778	33.98\\
10.9805555555556	33.98\\
10.9808333333333	33.97\\
10.9811111111111	33.99\\
10.9813888888889	33.98\\
10.9816666666667	33.98\\
10.9819444444444	33.98\\
10.9822222222222	33.98\\
10.9825	33.98\\
10.9827777777778	33.98\\
10.9830555555556	33.98\\
10.9833333333333	33.9784246177287\\
10.9836111111111	33.9784246177287\\
10.9838888888889	33.9784246177287\\
10.9841666666667	33.98\\
10.9844444444444	33.9686320646659\\
10.9847222222222	33.96\\
10.985	33.98\\
10.9852777777778	33.98\\
10.9855555555556	33.96\\
10.9858333333333	33.98\\
10.9861111111111	33.98\\
10.9863888888889	33.97\\
10.9866666666667	33.97\\
10.9869444444444	33.97\\
10.9872222222222	33.97\\
10.9875	33.97\\
10.9877777777778	33.97\\
10.9880555555556	33.97\\
10.9883333333333	33.97\\
10.9886111111111	33.9686320646659\\
10.9888888888889	33.97\\
10.9891666666667	33.95599716387\\
10.9894444444444	33.95599716387\\
10.9897222222222	33.95599716387\\
10.99	33.9573430261438\\
10.9902777777778	33.9573430261438\\
10.9905555555556	33.9573430261438\\
10.9908333333333	33.95\\
10.9911111111111	33.97\\
10.9913888888889	33.96\\
10.9916666666667	33.96\\
10.9919444444444	33.96\\
10.9922222222222	33.96\\
10.9925	33.96\\
10.9927777777778	33.96\\
10.9930555555556	33.96\\
10.9933333333333	33.94\\
10.9936111111111	33.96\\
10.9938888888889	33.96\\
10.9941666666667	33.95\\
10.9944444444444	33.97\\
10.9947222222222	33.97\\
10.995	33.95\\
10.9952777777778	33.97\\
10.9955555555556	33.97\\
10.9958333333333	33.97\\
10.9961111111111	33.97\\
10.9963888888889	33.9681732365245\\
10.9966666666667	33.97\\
10.9969444444444	33.9584246177287\\
10.9972222222222	33.95\\
10.9975	33.960526743864\\
10.9977777777778	33.97\\
10.9980555555556	33.98\\
10.9983333333333	33.9833430597809\\
10.9986111111111	33.9855601432366\\
10.9988888888889	33.98\\
10.9991666666667	33.98\\
10.9994444444444	33.98\\
10.9997222222222	33.98\\
11	33.98\\
};
\addlegendentry{$S \pm \delta^\pm$};

\addplot [color=mycolor1,solid,line width=2.0pt,forget plot]
  table[row sep=crcr]{%
10.9811111111111	33.995\\
10.9811111111111	34.01\\
};
\addplot [color=mycolor1,solid,line width=2.0pt,mark=*,mark options={solid,fill=mycolor1},forget plot]
  table[row sep=crcr]{%
10.9811111111111	34.01\\
};
\addplot [color=mycolor1,solid,line width=2.0pt,forget plot]
  table[row sep=crcr]{%
10.9911111111111	33.975\\
10.9911111111111	33.99\\
};
\addplot [color=mycolor1,solid,line width=2.0pt,mark=*,mark options={solid,fill=mycolor1},forget plot]
  table[row sep=crcr]{%
10.9911111111111	33.99\\
};
\addplot [color=mycolor1,solid,line width=2.0pt,forget plot]
  table[row sep=crcr]{%
10.9936111111111	33.965\\
10.9936111111111	33.98\\
};
\addplot [color=mycolor1,solid,line width=2.0pt,mark=*,mark options={solid,fill=mycolor1},forget plot]
  table[row sep=crcr]{%
10.9936111111111	33.98\\
};
\addplot [color=mycolor1,solid,line width=2.0pt,forget plot]
  table[row sep=crcr]{%
10.9944444444444	33.975\\
10.9944444444444	33.99\\
};
\addplot [color=mycolor1,solid,line width=2.0pt,mark=*,mark options={solid,fill=mycolor1},forget plot]
  table[row sep=crcr]{%
10.9944444444444	33.99\\
};
\addplot [color=mycolor1,solid,line width=2.0pt,forget plot]
  table[row sep=crcr]{%
10.9952777777778	33.975\\
10.9952777777778	33.99\\
};
\addplot [color=mycolor1,solid,line width=2.0pt,mark=*,mark options={solid,fill=mycolor1},forget plot]
  table[row sep=crcr]{%
10.9952777777778	33.99\\
};
\addplot [color=mycolor1,solid,line width=2.0pt,forget plot]
  table[row sep=crcr]{%
10.9975	33.975\\
10.9975	33.983421792079\\
};
\addplot [color=mycolor1,solid,line width=2.0pt,mark=*,mark options={solid,fill=mycolor1},forget plot]
  table[row sep=crcr]{%
10.9975	33.983421792079\\
};
\addplot [color=mycolor1,solid,line width=2.0pt,forget plot]
  table[row sep=crcr]{%
10.9977777777778	33.985\\
10.9977777777778	33.99\\
};
\addplot [color=mycolor1,solid,line width=2.0pt,mark=*,mark options={solid,fill=mycolor1},forget plot]
  table[row sep=crcr]{%
10.9977777777778	33.99\\
};
\addplot [color=mycolor1,solid,line width=2.0pt,forget plot]
  table[row sep=crcr]{%
10.9980555555556	33.995\\
10.9980555555556	34.0012009559832\\
};
\addplot [color=mycolor1,solid,line width=2.0pt,mark=*,mark options={solid,fill=mycolor1},forget plot]
  table[row sep=crcr]{%
10.9980555555556	34.0012009559832\\
};
\addplot [color=mycolor1,solid,line width=2.0pt,forget plot]
  table[row sep=crcr]{%
10.9983333333333	33.995\\
10.9983333333333	34.0049120415607\\
};
\addplot [color=mycolor1,solid,line width=2.0pt,mark=*,mark options={solid,fill=mycolor1}]
  table[row sep=crcr]{%
10.9983333333333	34.0049120415607\\
};
\addlegendentry{Ext Buy MO lifts our sell LO};

\addplot [color=mycolor1,dashed,line width=2.0pt,forget plot]
  table[row sep=crcr]{%
10.9791666666667	33.99\\
10.9791666666667	34\\
};
\addplot [color=mycolor1,dashed,line width=2.0pt,mark=o,mark options={solid},forget plot]
  table[row sep=crcr]{%
10.9791666666667	34\\
};
\addplot [color=mycolor1,dashed,line width=2.0pt,forget plot]
  table[row sep=crcr]{%
10.9913888888889	33.97\\
10.9913888888889	33.98\\
};
\addplot [color=mycolor1,dashed,line width=2.0pt,mark=o,mark options={solid}]
  table[row sep=crcr]{%
10.9913888888889	33.98\\
};
\addlegendentry{Ext Buy MO arrives};

\addplot [color=mycolor2,solid,line width=2.0pt,forget plot]
  table[row sep=crcr]{%
10.9769444444444	33.995\\
10.9769444444444	33.99\\
};
\addplot [color=mycolor2,solid,line width=2.0pt,mark=*,mark options={solid,fill=mycolor2},forget plot]
  table[row sep=crcr]{%
10.9769444444444	33.99\\
};
\addplot [color=mycolor2,solid,line width=2.0pt,forget plot]
  table[row sep=crcr]{%
10.9775	33.995\\
10.9775	33.986063979387\\
};
\addplot [color=mycolor2,solid,line width=2.0pt,mark=*,mark options={solid,fill=mycolor2},forget plot]
  table[row sep=crcr]{%
10.9775	33.986063979387\\
};
\addplot [color=mycolor2,solid,line width=2.0pt,forget plot]
  table[row sep=crcr]{%
10.9780555555556	33.99\\
10.9780555555556	33.9767116094193\\
};
\addplot [color=mycolor2,solid,line width=2.0pt,mark=*,mark options={solid,fill=mycolor2},forget plot]
  table[row sep=crcr]{%
10.9780555555556	33.9767116094193\\
};
\addplot [color=mycolor2,solid,line width=2.0pt,forget plot]
  table[row sep=crcr]{%
10.9802777777778	33.99\\
10.9802777777778	33.98\\
};
\addplot [color=mycolor2,solid,line width=2.0pt,mark=*,mark options={solid,fill=mycolor2},forget plot]
  table[row sep=crcr]{%
10.9802777777778	33.98\\
};
\addplot [color=mycolor2,solid,line width=2.0pt,forget plot]
  table[row sep=crcr]{%
10.9813888888889	33.985\\
10.9813888888889	33.98\\
};
\addplot [color=mycolor2,solid,line width=2.0pt,mark=*,mark options={solid,fill=mycolor2},forget plot]
  table[row sep=crcr]{%
10.9813888888889	33.98\\
};
\addplot [color=mycolor2,solid,line width=2.0pt,forget plot]
  table[row sep=crcr]{%
10.9833333333333	33.985\\
10.9833333333333	33.9784246177287\\
};
\addplot [color=mycolor2,solid,line width=2.0pt,mark=*,mark options={solid,fill=mycolor2},forget plot]
  table[row sep=crcr]{%
10.9833333333333	33.9784246177287\\
};
\addplot [color=mycolor2,solid,line width=2.0pt,forget plot]
  table[row sep=crcr]{%
10.9841666666667	33.985\\
10.9841666666667	33.98\\
};
\addplot [color=mycolor2,solid,line width=2.0pt,mark=*,mark options={solid,fill=mycolor2},forget plot]
  table[row sep=crcr]{%
10.9841666666667	33.98\\
};
\addplot [color=mycolor2,solid,line width=2.0pt,forget plot]
  table[row sep=crcr]{%
10.9855555555556	33.98\\
10.9855555555556	33.96\\
};
\addplot [color=mycolor2,solid,line width=2.0pt,mark=*,mark options={solid,fill=mycolor2},forget plot]
  table[row sep=crcr]{%
10.9855555555556	33.96\\
};
\addplot [color=mycolor2,solid,line width=2.0pt,forget plot]
  table[row sep=crcr]{%
10.9863888888889	33.975\\
10.9863888888889	33.97\\
};
\addplot [color=mycolor2,solid,line width=2.0pt,mark=*,mark options={solid,fill=mycolor2},forget plot]
  table[row sep=crcr]{%
10.9863888888889	33.97\\
};
\addplot [color=mycolor2,solid,line width=2.0pt,forget plot]
  table[row sep=crcr]{%
10.9880555555556	33.975\\
10.9880555555556	33.97\\
};
\addplot [color=mycolor2,solid,line width=2.0pt,mark=*,mark options={solid,fill=mycolor2},forget plot]
  table[row sep=crcr]{%
10.9880555555556	33.97\\
};
\addplot [color=mycolor2,solid,line width=2.0pt,forget plot]
  table[row sep=crcr]{%
10.9886111111111	33.975\\
10.9886111111111	33.9686320646659\\
};
\addplot [color=mycolor2,solid,line width=2.0pt,mark=*,mark options={solid,fill=mycolor2},forget plot]
  table[row sep=crcr]{%
10.9886111111111	33.9686320646659\\
};
\addplot [color=mycolor2,solid,line width=2.0pt,forget plot]
  table[row sep=crcr]{%
10.9891666666667	33.97\\
10.9891666666667	33.95599716387\\
};
\addplot [color=mycolor2,solid,line width=2.0pt,mark=*,mark options={solid,fill=mycolor2},forget plot]
  table[row sep=crcr]{%
10.9891666666667	33.95599716387\\
};
\addplot [color=mycolor2,solid,line width=2.0pt,forget plot]
  table[row sep=crcr]{%
10.9913888888889	33.97\\
10.9913888888889	33.96\\
};
\addplot [color=mycolor2,solid,line width=2.0pt,mark=*,mark options={solid,fill=mycolor2},forget plot]
  table[row sep=crcr]{%
10.9913888888889	33.96\\
};
\addplot [color=mycolor2,solid,line width=2.0pt,forget plot]
  table[row sep=crcr]{%
10.9933333333333	33.955\\
10.9933333333333	33.94\\
};
\addplot [color=mycolor2,solid,line width=2.0pt,mark=*,mark options={solid,fill=mycolor2},forget plot]
  table[row sep=crcr]{%
10.9933333333333	33.94\\
};
\addplot [color=mycolor2,solid,line width=2.0pt,forget plot]
  table[row sep=crcr]{%
10.995	33.965\\
10.995	33.95\\
};
\addplot [color=mycolor2,solid,line width=2.0pt,mark=*,mark options={solid,fill=mycolor2},forget plot]
  table[row sep=crcr]{%
10.995	33.95\\
};
\addplot [color=mycolor2,solid,line width=2.0pt,forget plot]
  table[row sep=crcr]{%
10.9963888888889	33.975\\
10.9963888888889	33.9681732365245\\
};
\addplot [color=mycolor2,solid,line width=2.0pt,mark=*,mark options={solid,fill=mycolor2},forget plot]
  table[row sep=crcr]{%
10.9963888888889	33.9681732365245\\
};
\addplot [color=mycolor2,solid,line width=2.0pt,forget plot]
  table[row sep=crcr]{%
10.9969444444444	33.97\\
10.9969444444444	33.9584246177287\\
};
\addplot [color=mycolor2,solid,line width=2.0pt,mark=*,mark options={solid,fill=mycolor2},forget plot]
  table[row sep=crcr]{%
10.9969444444444	33.9584246177287\\
};
\addplot [color=mycolor2,solid,line width=2.0pt,forget plot]
  table[row sep=crcr]{%
10.9988888888889	33.995\\
10.9988888888889	33.98\\
};
\addplot [color=mycolor2,solid,line width=2.0pt,mark=*,mark options={solid,fill=mycolor2}]
  table[row sep=crcr]{%
10.9988888888889	33.98\\
};
\addlegendentry{Ext Sell MO lifts our buy LO};

\addplot [color=mycolor2,dashed,line width=2.0pt,forget plot]
  table[row sep=crcr]{%
10.9980555555556	33.995\\
10.9980555555556	34\\
};
\addplot [color=mycolor2,dashed,line width=2.0pt,mark=o,mark options={solid}]
  table[row sep=crcr]{%
10.9980555555556	34\\
};
\addlegendentry{Ext Sell MO arrives};

\addplot [color=mycolor3,solid,line width=2.0pt,forget plot]
  table[row sep=crcr]{%
10.9811111111111	33.995\\
10.9811111111111	34\\
};
\addplot [color=mycolor3,solid,line width=2.0pt,mark=*,mark options={solid,fill=mycolor3},forget plot]
  table[row sep=crcr]{%
10.9811111111111	34\\
};
\addplot [color=mycolor3,solid,line width=2.0pt,forget plot]
  table[row sep=crcr]{%
10.985	33.985\\
10.985	33.99\\
};
\addplot [color=mycolor3,solid,line width=2.0pt,mark=*,mark options={solid,fill=mycolor3},forget plot]
  table[row sep=crcr]{%
10.985	33.99\\
};
\addplot [color=mycolor3,solid,line width=2.0pt,forget plot]
  table[row sep=crcr]{%
10.9911111111111	33.975\\
10.9911111111111	33.98\\
};
\addplot [color=mycolor3,solid,line width=2.0pt,mark=*,mark options={solid,fill=mycolor3},forget plot]
  table[row sep=crcr]{%
10.9911111111111	33.98\\
};
\addplot [color=mycolor3,solid,line width=2.0pt,forget plot]
  table[row sep=crcr]{%
10.9936111111111	33.965\\
10.9936111111111	33.97\\
};
\addplot [color=mycolor3,solid,line width=2.0pt,mark=*,mark options={solid,fill=mycolor3},forget plot]
  table[row sep=crcr]{%
10.9936111111111	33.97\\
};
\addplot [color=mycolor3,solid,line width=2.0pt,forget plot]
  table[row sep=crcr]{%
10.9944444444444	33.975\\
10.9944444444444	33.98\\
};
\addplot [color=mycolor3,solid,line width=2.0pt,mark=*,mark options={solid,fill=mycolor3},forget plot]
  table[row sep=crcr]{%
10.9944444444444	33.98\\
};
\addplot [color=mycolor3,solid,line width=2.0pt,forget plot]
  table[row sep=crcr]{%
10.9952777777778	33.975\\
10.9952777777778	33.98\\
};
\addplot [color=mycolor3,solid,line width=2.0pt,mark=*,mark options={solid,fill=mycolor3}]
  table[row sep=crcr]{%
10.9952777777778	33.98\\
};
\addlegendentry{Our Buy MO};

\addplot [color=mycolor4,solid,line width=2.0pt,forget plot]
  table[row sep=crcr]{%
10.9791666666667	33.99\\
10.9791666666667	33.98\\
};
\addplot [color=mycolor4,solid,line width=2.0pt,mark=*,mark options={solid,fill=mycolor4},forget plot]
  table[row sep=crcr]{%
10.9791666666667	33.98\\
};
\addplot [color=mycolor4,solid,line width=2.0pt,forget plot]
  table[row sep=crcr]{%
10.9808333333333	33.99\\
10.9808333333333	33.98\\
};
\addplot [color=mycolor4,solid,line width=2.0pt,mark=*,mark options={solid,fill=mycolor4},forget plot]
  table[row sep=crcr]{%
10.9808333333333	33.98\\
};
\addplot [color=mycolor4,solid,line width=2.0pt,forget plot]
  table[row sep=crcr]{%
10.9847222222222	33.98\\
10.9847222222222	33.97\\
};
\addplot [color=mycolor4,solid,line width=2.0pt,mark=*,mark options={solid,fill=mycolor4},forget plot]
  table[row sep=crcr]{%
10.9847222222222	33.97\\
};
\addplot [color=mycolor4,solid,line width=2.0pt,forget plot]
  table[row sep=crcr]{%
10.9855555555556	33.98\\
10.9855555555556	33.97\\
};
\addplot [color=mycolor4,solid,line width=2.0pt,mark=*,mark options={solid,fill=mycolor4},forget plot]
  table[row sep=crcr]{%
10.9855555555556	33.97\\
};
\addplot [color=mycolor4,solid,line width=2.0pt,forget plot]
  table[row sep=crcr]{%
10.9908333333333	33.97\\
10.9908333333333	33.96\\
};
\addplot [color=mycolor4,solid,line width=2.0pt,mark=*,mark options={solid,fill=mycolor4},forget plot]
  table[row sep=crcr]{%
10.9908333333333	33.96\\
};
\addplot [color=mycolor4,solid,line width=2.0pt,forget plot]
  table[row sep=crcr]{%
10.9933333333333	33.955\\
10.9933333333333	33.95\\
};
\addplot [color=mycolor4,solid,line width=2.0pt,mark=*,mark options={solid,fill=mycolor4},forget plot]
  table[row sep=crcr]{%
10.9933333333333	33.95\\
};
\addplot [color=mycolor4,solid,line width=2.0pt,forget plot]
  table[row sep=crcr]{%
10.9941666666667	33.965\\
10.9941666666667	33.96\\
};
\addplot [color=mycolor4,solid,line width=2.0pt,mark=*,mark options={solid,fill=mycolor4},forget plot]
  table[row sep=crcr]{%
10.9941666666667	33.96\\
};
\addplot [color=mycolor4,solid,line width=2.0pt,forget plot]
  table[row sep=crcr]{%
10.995	33.965\\
10.995	33.96\\
};
\addplot [color=mycolor4,solid,line width=2.0pt,mark=*,mark options={solid,fill=mycolor4},forget plot]
  table[row sep=crcr]{%
10.995	33.96\\
};
\addplot [color=mycolor4,solid,line width=2.0pt,forget plot]
  table[row sep=crcr]{%
10.9972222222222	33.97\\
10.9972222222222	33.96\\
};
\addplot [color=mycolor4,solid,line width=2.0pt,mark=*,mark options={solid,fill=mycolor4},forget plot]
  table[row sep=crcr]{%
10.9972222222222	33.96\\
};
\addplot [color=mycolor4,solid,line width=2.0pt,forget plot]
  table[row sep=crcr]{%
10.9977777777778	33.985\\
10.9977777777778	33.98\\
};
\addplot [color=mycolor4,solid,line width=2.0pt,mark=*,mark options={solid,fill=mycolor4}]
  table[row sep=crcr]{%
10.9977777777778	33.98\\
};
\addlegendentry{Our Sell MO};

\end{axis}
\end{tikzpicture}%

\end{subfigure}%
\hfill%
\begin{subfigure}{.45\linewidth}
  \centering
  \setlength\figureheight{\linewidth} 
  \setlength\figurewidth{\linewidth}
  \tikzsetnextfilename{samplepath_dscr_nFPC_paths}
  % This file was created by matlab2tikz.
%
%The latest updates can be retrieved from
%  http://www.mathworks.com/matlabcentral/fileexchange/22022-matlab2tikz-matlab2tikz
%where you can also make suggestions and rate matlab2tikz.
%
\definecolor{mycolor1}{rgb}{0.65098,0.80784,0.89020}%
\definecolor{mycolor2}{rgb}{0.69804,0.87451,0.54118}%
\definecolor{mycolor3}{rgb}{0.20000,0.62745,0.17255}%
\definecolor{mycolor4}{rgb}{0.12157,0.47059,0.70588}%
%
\begin{tikzpicture}[trim axis left, trim axis right]

\begin{axis}[%
width=\figurewidth,
height=\figureheight,
at={(0\figurewidth,0\figureheight)},
scale only axis,
every outer x axis line/.append style={black},
every x tick label/.append style={font=\color{black}},
xmin=10.975,
xmax=11,
every outer y axis line/.append style={black},
every y tick label/.append style={font=\color{black}},
ymin=33.93,
ymax=34.03,
axis background/.style={fill=white},
axis x line*=bottom,
axis y line*=left,
legend style={legend cell align=left,align=left,draw=black,legend pos = south west},
every axis legend/.code={\renewcommand\addlegendentry[2][]{}}  %ignore legend locally
]
\addplot [color=black,solid,line width=3.0pt]
  table[row sep=crcr]{%
10.975	33.995\\
10.9752777777778	33.995\\
10.9755555555556	33.995\\
10.9758333333333	33.995\\
10.9761111111111	33.995\\
10.9763888888889	33.995\\
10.9766666666667	33.995\\
10.9769444444444	33.995\\
10.9772222222222	33.995\\
10.9775	33.995\\
10.9777777777778	33.995\\
10.9780555555556	33.99\\
10.9783333333333	33.99\\
10.9786111111111	33.99\\
10.9788888888889	33.99\\
10.9791666666667	33.99\\
10.9794444444444	33.995\\
10.9797222222222	33.995\\
10.98	33.995\\
10.9802777777778	33.99\\
10.9805555555556	33.99\\
10.9808333333333	33.99\\
10.9811111111111	33.995\\
10.9813888888889	33.985\\
10.9816666666667	33.985\\
10.9819444444444	33.985\\
10.9822222222222	33.985\\
10.9825	33.985\\
10.9827777777778	33.985\\
10.9830555555556	33.985\\
10.9833333333333	33.985\\
10.9836111111111	33.985\\
10.9838888888889	33.985\\
10.9841666666667	33.985\\
10.9844444444444	33.98\\
10.9847222222222	33.98\\
10.985	33.985\\
10.9852777777778	33.985\\
10.9855555555556	33.98\\
10.9858333333333	33.985\\
10.9861111111111	33.985\\
10.9863888888889	33.975\\
10.9866666666667	33.975\\
10.9869444444444	33.975\\
10.9872222222222	33.975\\
10.9875	33.975\\
10.9877777777778	33.975\\
10.9880555555556	33.975\\
10.9883333333333	33.975\\
10.9886111111111	33.975\\
10.9888888888889	33.975\\
10.9891666666667	33.97\\
10.9894444444444	33.97\\
10.9897222222222	33.97\\
10.99	33.97\\
10.9902777777778	33.97\\
10.9905555555556	33.97\\
10.9908333333333	33.97\\
10.9911111111111	33.975\\
10.9913888888889	33.97\\
10.9916666666667	33.97\\
10.9919444444444	33.97\\
10.9922222222222	33.965\\
10.9925	33.965\\
10.9927777777778	33.965\\
10.9930555555556	33.965\\
10.9933333333333	33.955\\
10.9936111111111	33.965\\
10.9938888888889	33.965\\
10.9941666666667	33.965\\
10.9944444444444	33.975\\
10.9947222222222	33.975\\
10.995	33.965\\
10.9952777777778	33.975\\
10.9955555555556	33.975\\
10.9958333333333	33.975\\
10.9961111111111	33.975\\
10.9963888888889	33.975\\
10.9966666666667	33.975\\
10.9969444444444	33.97\\
10.9972222222222	33.97\\
10.9975	33.975\\
10.9977777777778	33.985\\
10.9980555555556	33.995\\
10.9983333333333	33.995\\
10.9986111111111	33.995\\
10.9988888888889	33.995\\
10.9991666666667	33.995\\
10.9994444444444	33.995\\
10.9997222222222	33.995\\
11	33.995\\
};
\addlegendentry{$S$};

\addplot [color=gray,solid,line width=2.0pt,forget plot]
  table[row sep=crcr]{%
10.975	34.0095591449678\\
10.9752777777778	34.0095591449678\\
10.9755555555556	34.0095591449678\\
10.9758333333333	34.0095591449678\\
10.9761111111111	34.0095591449678\\
10.9763888888889	34.0095591449678\\
10.9766666666667	34.0095591449678\\
10.9769444444444	34.0077907684168\\
10.9772222222222	34.0077907684168\\
10.9775	34.0044883451971\\
10.9777777777778	34\\
10.9780555555556	34.0070144313823\\
10.9783333333333	34.0070144313823\\
10.9786111111111	34.0070144313823\\
10.9788888888889	34.0070144313823\\
10.9791666666667	34.0118829629196\\
10.9794444444444	34.0070144313823\\
10.9797222222222	34.0070144313823\\
10.98	34\\
10.9802777777778	34.0095591449678\\
10.9805555555556	34.0095591449678\\
10.9808333333333	34.012650854911\\
10.9811111111111	34.0021523461235\\
10.9813888888889	33.9997593349026\\
10.9816666666667	34.000221604306\\
10.9819444444444	34.000221604306\\
10.9822222222222	33.9999930529003\\
10.9825	33.9999930529003\\
10.9827777777778	33.9999930529003\\
10.9830555555556	33.9999930529003\\
10.9833333333333	33.9995591449678\\
10.9836111111111	33.9995591449678\\
10.9838888888889	33.9995591449678\\
10.9841666666667	33.9924887877916\\
10.9844444444444	33.9997969182872\\
10.9847222222222	34.0034558419258\\
10.985	33.9997969182872\\
10.9852777777778	33.9924887877916\\
10.9855555555556	34.0018829629196\\
10.9858333333333	33.9997258918278\\
10.9861111111111	33.9920935862006\\
10.9863888888889	33.9870144313823\\
10.9866666666667	33.9901671828515\\
10.9869444444444	33.9901671828515\\
10.9872222222222	33.9901671828515\\
10.9875	33.9901671828515\\
10.9877777777778	33.9901671828515\\
10.9880555555556	33.9901581955578\\
10.9883333333333	33.9896650872252\\
10.9886111111111	33.9888653999139\\
10.9888888888889	33.9814256768415\\
10.9891666666667	33.9858026788123\\
10.9894444444444	33.9858026788123\\
10.9897222222222	33.9858026788123\\
10.99	33.987668687197\\
10.9902777777778	33.987668687197\\
10.9905555555556	33.987668687197\\
10.9908333333333	33.9915804064569\\
10.9911111111111	33.98\\
10.9913888888889	33.9877907684168\\
10.9916666666667	33.9897969182872\\
10.9919444444444	33.9841301013182\\
10.9922222222222	33.9797969182872\\
10.9925	33.98018095728\\
10.9927777777778	33.98018095728\\
10.9930555555556	33.9750986475475\\
10.9933333333333	33.9733875791588\\
10.9936111111111	33.98018095728\\
10.9938888888889	33.98018095728\\
10.9941666666667	33.9834558419258\\
10.9944444444444	33.9902110954469\\
10.9947222222222	33.9833086626523\\
10.995	33.9821207619446\\
10.9952777777778	33.99018095728\\
10.9955555555556	33.9877907684168\\
10.9958333333333	33.9877907684168\\
10.9961111111111	33.9849396889425\\
10.9963888888889	33.9844883451971\\
10.9966666666667	33.98\\
10.9969444444444	33.9867367765128\\
10.9972222222222	33.9923883907621\\
10.9975	33.9917432433565\\
10.9977777777778	34.0025065370642\\
10.9980555555556	34.0067367765128\\
10.9983333333333	34.0096650872252\\
10.9986111111111	34.0101581955578\\
10.9988888888889	34.0088653999139\\
10.9991666666667	34.0062335570531\\
10.9994444444444	34.0062335570531\\
10.9997222222222	34.0062335570531\\
11	34.0062335570531\\
};
\addplot [color=gray,solid,line width=2.0pt]
  table[row sep=crcr]{%
10.975	33.9878764041058\\
10.9752777777778	33.9878764041058\\
10.9755555555556	33.9878764041058\\
10.9758333333333	33.9878764041058\\
10.9761111111111	33.9878764041058\\
10.9763888888889	33.9878764041058\\
10.9766666666667	33.9878764041058\\
10.9769444444444	33.9870520397391\\
10.9772222222222	33.9870520397391\\
10.9775	33.9842616086803\\
10.9777777777778	33.9798455400393\\
10.9780555555556	33.9767675356892\\
10.9783333333333	33.9767675356892\\
10.9786111111111	33.9767675356892\\
10.9788888888889	33.9767675356892\\
10.9791666666667	33.98\\
10.9794444444444	33.9867675356893\\
10.9797222222222	33.9867675356893\\
10.98	33.9795147998989\\
10.9802777777778	33.9778764041057\\
10.9805555555556	33.9778764041057\\
10.9808333333333	33.98\\
10.9811111111111	33.9815306001365\\
10.9813888888889	33.9796529109682\\
10.9816666666667	33.98\\
10.9819444444444	33.98\\
10.9822222222222	33.9799665589495\\
10.9825	33.9799665589495\\
10.9827777777778	33.9799665589495\\
10.9830555555556	33.9799665589495\\
10.9833333333333	33.9778764041057\\
10.9836111111111	33.9778764041057\\
10.9838888888889	33.9778764041057\\
10.9841666666667	33.9721754260084\\
10.9844444444444	33.9697303226573\\
10.9847222222222	33.97\\
10.985	33.9797303226573\\
10.9852777777778	33.9721754260084\\
10.9855555555556	33.97\\
10.9858333333333	33.9797017484381\\
10.9861111111111	33.9719521233684\\
10.9863888888889	33.9667675356893\\
10.9866666666667	33.97\\
10.9869444444444	33.97\\
10.9872222222222	33.97\\
10.9875	33.97\\
10.9877777777778	33.97\\
10.9880555555556	33.97\\
10.9883333333333	33.9689490076283\\
10.9886111111111	33.9677351358409\\
10.9888888888889	33.9610819402726\\
10.9891666666667	33.9556856460776\\
10.9894444444444	33.9556856460776\\
10.9897222222222	33.9556856460776\\
10.99	33.9574744593794\\
10.9902777777778	33.9574744593794\\
10.9905555555556	33.9574744593794\\
10.9908333333333	33.96\\
10.9911111111111	33.9598455400393\\
10.9913888888889	33.9570520397391\\
10.9916666666667	33.9597303226574\\
10.9919444444444	33.9539290070915\\
10.9922222222222	33.9597303226574\\
10.9925	33.96\\
10.9927777777778	33.96\\
10.9930555555556	33.9550224734259\\
10.9933333333333	33.95\\
10.9936111111111	33.96\\
10.9938888888889	33.96\\
10.9941666666667	33.96\\
10.9944444444444	33.97\\
10.9947222222222	33.9626552154609\\
10.995	33.96\\
10.9952777777778	33.97\\
10.9955555555556	33.9670520397391\\
10.9958333333333	33.9670520397391\\
10.9961111111111	33.9645163776871\\
10.9963888888889	33.9642616086803\\
10.9966666666667	33.9598455400393\\
10.9969444444444	33.9562695844837\\
10.9972222222222	33.96\\
10.9975	33.97\\
10.9977777777778	33.98\\
10.9980555555556	33.9862695844837\\
10.9983333333333	33.9889490076283\\
10.9986111111111	33.99\\
10.9988888888889	33.9877351358409\\
10.9991666666667	33.9858243720109\\
10.9994444444444	33.9858243720109\\
10.9997222222222	33.9858243720109\\
11	33.9858243720109\\
};
\addlegendentry{$S \pm \delta^\pm$};

\addplot [color=mycolor1,solid,line width=2.0pt,forget plot]
  table[row sep=crcr]{%
10.9811111111111	33.995\\
10.9811111111111	34.0021523461235\\
};
\addplot [color=mycolor1,solid,line width=2.0pt,mark=*,mark options={solid,fill=mycolor1},forget plot]
  table[row sep=crcr]{%
10.9811111111111	34.0021523461235\\
};
\addplot [color=mycolor1,solid,line width=2.0pt,forget plot]
  table[row sep=crcr]{%
10.9911111111111	33.975\\
10.9911111111111	33.98\\
};
\addplot [color=mycolor1,solid,line width=2.0pt,mark=*,mark options={solid,fill=mycolor1},forget plot]
  table[row sep=crcr]{%
10.9911111111111	33.98\\
};
\addplot [color=mycolor1,solid,line width=2.0pt,forget plot]
  table[row sep=crcr]{%
10.9913888888889	33.97\\
10.9913888888889	33.9877907684168\\
};
\addplot [color=mycolor1,solid,line width=2.0pt,mark=*,mark options={solid,fill=mycolor1},forget plot]
  table[row sep=crcr]{%
10.9913888888889	33.9877907684168\\
};
\addplot [color=mycolor1,solid,line width=2.0pt,forget plot]
  table[row sep=crcr]{%
10.9936111111111	33.965\\
10.9936111111111	33.98018095728\\
};
\addplot [color=mycolor1,solid,line width=2.0pt,mark=*,mark options={solid,fill=mycolor1},forget plot]
  table[row sep=crcr]{%
10.9936111111111	33.98018095728\\
};
\addplot [color=mycolor1,solid,line width=2.0pt,forget plot]
  table[row sep=crcr]{%
10.9944444444444	33.975\\
10.9944444444444	33.9902110954469\\
};
\addplot [color=mycolor1,solid,line width=2.0pt,mark=*,mark options={solid,fill=mycolor1},forget plot]
  table[row sep=crcr]{%
10.9944444444444	33.9902110954469\\
};
\addplot [color=mycolor1,solid,line width=2.0pt,forget plot]
  table[row sep=crcr]{%
10.9977777777778	33.985\\
10.9977777777778	34.0025065370642\\
};
\addplot [color=mycolor1,solid,line width=2.0pt,mark=*,mark options={solid,fill=mycolor1}]
  table[row sep=crcr]{%
10.9977777777778	34.0025065370642\\
};
\addlegendentry{Ext Buy MO lifts our sell LO};

\addplot [color=mycolor1,dashed,line width=2.0pt,forget plot]
  table[row sep=crcr]{%
10.9791666666667	33.99\\
10.9791666666667	34\\
};
\addplot [color=mycolor1,dashed,line width=2.0pt,mark=o,mark options={solid},forget plot]
  table[row sep=crcr]{%
10.9791666666667	34\\
};
\addplot [color=mycolor1,dashed,line width=2.0pt,forget plot]
  table[row sep=crcr]{%
10.9952777777778	33.975\\
10.9952777777778	33.97\\
};
\addplot [color=mycolor1,dashed,line width=2.0pt,mark=o,mark options={solid},forget plot]
  table[row sep=crcr]{%
10.9952777777778	33.97\\
};
\addplot [color=mycolor1,dashed,line width=2.0pt,forget plot]
  table[row sep=crcr]{%
10.9975	33.975\\
10.9975	33.98\\
};
\addplot [color=mycolor1,dashed,line width=2.0pt,mark=o,mark options={solid},forget plot]
  table[row sep=crcr]{%
10.9975	33.98\\
};
\addplot [color=mycolor1,dashed,line width=2.0pt,forget plot]
  table[row sep=crcr]{%
10.9980555555556	33.995\\
10.9980555555556	34\\
};
\addplot [color=mycolor1,dashed,line width=2.0pt,mark=o,mark options={solid},forget plot]
  table[row sep=crcr]{%
10.9980555555556	34\\
};
\addplot [color=mycolor1,dashed,line width=2.0pt,forget plot]
  table[row sep=crcr]{%
10.9983333333333	33.995\\
10.9983333333333	34\\
};
\addplot [color=mycolor1,dashed,line width=2.0pt,mark=o,mark options={solid}]
  table[row sep=crcr]{%
10.9983333333333	34\\
};
\addlegendentry{Ext Buy MO arrives};

\addplot [color=mycolor2,solid,line width=2.0pt,forget plot]
  table[row sep=crcr]{%
10.9769444444444	33.995\\
10.9769444444444	33.9870520397391\\
};
\addplot [color=mycolor2,solid,line width=2.0pt,mark=*,mark options={solid,fill=mycolor2},forget plot]
  table[row sep=crcr]{%
10.9769444444444	33.9870520397391\\
};
\addplot [color=mycolor2,solid,line width=2.0pt,forget plot]
  table[row sep=crcr]{%
10.9775	33.995\\
10.9775	33.9842616086803\\
};
\addplot [color=mycolor2,solid,line width=2.0pt,mark=*,mark options={solid,fill=mycolor2},forget plot]
  table[row sep=crcr]{%
10.9775	33.9842616086803\\
};
\addplot [color=mycolor2,solid,line width=2.0pt,forget plot]
  table[row sep=crcr]{%
10.9802777777778	33.99\\
10.9802777777778	33.9778764041057\\
};
\addplot [color=mycolor2,solid,line width=2.0pt,mark=*,mark options={solid,fill=mycolor2},forget plot]
  table[row sep=crcr]{%
10.9802777777778	33.9778764041057\\
};
\addplot [color=mycolor2,solid,line width=2.0pt,forget plot]
  table[row sep=crcr]{%
10.9833333333333	33.985\\
10.9833333333333	33.9778764041057\\
};
\addplot [color=mycolor2,solid,line width=2.0pt,mark=*,mark options={solid,fill=mycolor2},forget plot]
  table[row sep=crcr]{%
10.9833333333333	33.9778764041057\\
};
\addplot [color=mycolor2,solid,line width=2.0pt,forget plot]
  table[row sep=crcr]{%
10.9841666666667	33.985\\
10.9841666666667	33.9721754260084\\
};
\addplot [color=mycolor2,solid,line width=2.0pt,mark=*,mark options={solid,fill=mycolor2},forget plot]
  table[row sep=crcr]{%
10.9841666666667	33.9721754260084\\
};
\addplot [color=mycolor2,solid,line width=2.0pt,forget plot]
  table[row sep=crcr]{%
10.9855555555556	33.98\\
10.9855555555556	33.97\\
};
\addplot [color=mycolor2,solid,line width=2.0pt,mark=*,mark options={solid,fill=mycolor2},forget plot]
  table[row sep=crcr]{%
10.9855555555556	33.97\\
};
\addplot [color=mycolor2,solid,line width=2.0pt,forget plot]
  table[row sep=crcr]{%
10.9880555555556	33.975\\
10.9880555555556	33.97\\
};
\addplot [color=mycolor2,solid,line width=2.0pt,mark=*,mark options={solid,fill=mycolor2},forget plot]
  table[row sep=crcr]{%
10.9880555555556	33.97\\
};
\addplot [color=mycolor2,solid,line width=2.0pt,forget plot]
  table[row sep=crcr]{%
10.9886111111111	33.975\\
10.9886111111111	33.9677351358409\\
};
\addplot [color=mycolor2,solid,line width=2.0pt,mark=*,mark options={solid,fill=mycolor2},forget plot]
  table[row sep=crcr]{%
10.9886111111111	33.9677351358409\\
};
\addplot [color=mycolor2,solid,line width=2.0pt,forget plot]
  table[row sep=crcr]{%
10.9891666666667	33.97\\
10.9891666666667	33.9556856460776\\
};
\addplot [color=mycolor2,solid,line width=2.0pt,mark=*,mark options={solid,fill=mycolor2},forget plot]
  table[row sep=crcr]{%
10.9891666666667	33.9556856460776\\
};
\addplot [color=mycolor2,solid,line width=2.0pt,forget plot]
  table[row sep=crcr]{%
10.9933333333333	33.955\\
10.9933333333333	33.95\\
};
\addplot [color=mycolor2,solid,line width=2.0pt,mark=*,mark options={solid,fill=mycolor2},forget plot]
  table[row sep=crcr]{%
10.9933333333333	33.95\\
};
\addplot [color=mycolor2,solid,line width=2.0pt,forget plot]
  table[row sep=crcr]{%
10.995	33.965\\
10.995	33.96\\
};
\addplot [color=mycolor2,solid,line width=2.0pt,mark=*,mark options={solid,fill=mycolor2},forget plot]
  table[row sep=crcr]{%
10.995	33.96\\
};
\addplot [color=mycolor2,solid,line width=2.0pt,forget plot]
  table[row sep=crcr]{%
10.9963888888889	33.975\\
10.9963888888889	33.9642616086803\\
};
\addplot [color=mycolor2,solid,line width=2.0pt,mark=*,mark options={solid,fill=mycolor2},forget plot]
  table[row sep=crcr]{%
10.9963888888889	33.9642616086803\\
};
\addplot [color=mycolor2,solid,line width=2.0pt,forget plot]
  table[row sep=crcr]{%
10.9969444444444	33.97\\
10.9969444444444	33.9562695844837\\
};
\addplot [color=mycolor2,solid,line width=2.0pt,mark=*,mark options={solid,fill=mycolor2},forget plot]
  table[row sep=crcr]{%
10.9969444444444	33.9562695844837\\
};
\addplot [color=mycolor2,solid,line width=2.0pt,forget plot]
  table[row sep=crcr]{%
10.9980555555556	33.995\\
10.9980555555556	33.9862695844837\\
};
\addplot [color=mycolor2,solid,line width=2.0pt,mark=*,mark options={solid,fill=mycolor2},forget plot]
  table[row sep=crcr]{%
10.9980555555556	33.9862695844837\\
};
\addplot [color=mycolor2,solid,line width=2.0pt,forget plot]
  table[row sep=crcr]{%
10.9988888888889	33.995\\
10.9988888888889	33.9877351358409\\
};
\addplot [color=mycolor2,solid,line width=2.0pt,mark=*,mark options={solid,fill=mycolor2}]
  table[row sep=crcr]{%
10.9988888888889	33.9877351358409\\
};
\addlegendentry{Ext Sell MO lifts our buy LO};

\addplot [color=mycolor2,dashed,line width=2.0pt,forget plot]
  table[row sep=crcr]{%
10.9780555555556	33.99\\
10.9780555555556	33.99\\
};
\addplot [color=mycolor2,dashed,line width=2.0pt,mark=o,mark options={solid},forget plot]
  table[row sep=crcr]{%
10.9780555555556	33.99\\
};
\addplot [color=mycolor2,dashed,line width=2.0pt,forget plot]
  table[row sep=crcr]{%
10.9813888888889	33.985\\
10.9813888888889	33.99\\
};
\addplot [color=mycolor2,dashed,line width=2.0pt,mark=o,mark options={solid},forget plot]
  table[row sep=crcr]{%
10.9813888888889	33.99\\
};
\addplot [color=mycolor2,dashed,line width=2.0pt,forget plot]
  table[row sep=crcr]{%
10.9863888888889	33.975\\
10.9863888888889	33.98\\
};
\addplot [color=mycolor2,dashed,line width=2.0pt,mark=o,mark options={solid},forget plot]
  table[row sep=crcr]{%
10.9863888888889	33.98\\
};
\addplot [color=mycolor2,dashed,line width=2.0pt,forget plot]
  table[row sep=crcr]{%
10.9913888888889	33.97\\
10.9913888888889	33.97\\
};
\addplot [color=mycolor2,dashed,line width=2.0pt,mark=o,mark options={solid}]
  table[row sep=crcr]{%
10.9913888888889	33.97\\
};
\addlegendentry{Ext Sell MO arrives};

\addplot [color=mycolor4,solid,line width=2.0pt,forget plot]
  table[row sep=crcr]{%
10.98	33.995\\
10.98	33.99\\
};
\addplot [color=mycolor4,solid,line width=2.0pt,mark=*,mark options={solid,fill=mycolor4},forget plot]
  table[row sep=crcr]{%
10.98	33.99\\
};
\addplot [color=mycolor4,solid,line width=2.0pt,forget plot]
  table[row sep=crcr]{%
10.9911111111111	33.975\\
10.9911111111111	33.97\\
};
\addplot [color=mycolor4,solid,line width=2.0pt,mark=*,mark options={solid,fill=mycolor4},forget plot]
  table[row sep=crcr]{%
10.9911111111111	33.97\\
};

\end{axis}
\end{tikzpicture}%
 
\end{subfigure}\\
\leavevmode\smash{\makebox[0pt]{\hspace{-7em}% HORIZONTAL POSITION           
  \rotatebox[origin=l]{90}{\hspace{20em}% VERTICAL POSITION
    Price}%
}}\hspace{0pt plus 1filll}\null

Time (h)

\vspace{1cm}
\begin{subfigure}{\linewidth}
  \centering
  \setlength\figureheight{\linewidth} 
  \setlength\figurewidth{\linewidth}
  \tikzsetnextfilename{samplepathslegend}
  \definecolor{mycolor1}{rgb}{0.65098,0.80784,0.89020}%
\definecolor{mycolor2}{rgb}{0.69804,0.87451,0.54118}%
\definecolor{mycolor3}{rgb}{0.20000,0.62745,0.17255}%
\definecolor{mycolor4}{rgb}{0.12157,0.47059,0.70588}%
\begin{tikzpicture}
    \begingroup
    % inits/clears the lists (which might be populated from previous
    % axes):
    \csname pgfplots@init@cleared@structures\endcsname
    \pgfplotsset{legend cell align=left,legend columns = 2,legend style={at={(0,1)},anchor=north west},legend style={draw=black,column sep=1ex},
    legend entries={Midprice,
    				Midprice $\pm \delta^\pm$,
    				Our Sell MO,
    				Our Buy MO,
    				Ext Buy MO lifts our sell LO,
    				Ext Sell MO lifts our buy LO,
    				Ext Buy MO arrives,
    				Ext Sell MO arrives}}%
    \csname pgfplots@addlegendimage\endcsname{line width=2pt,black,solid,sharp plot}
    \csname pgfplots@addlegendimage\endcsname{line width=2pt,gray,solid,sharp plot}
    \csname pgfplots@addlegendimage\endcsname{line width=1.5pt,mycolor4,solid,mark=*,mark options={solid,fill=mycolor4},sharp plot}%sell
    \csname pgfplots@addlegendimage\endcsname{line width=1.5pt,mycolor3,solid,mark=*,mark options={solid,fill=mycolor3},sharp plot}%buy
    \csname pgfplots@addlegendimage\endcsname{line width=1.5pt,mycolor1,solid,mark=*,mark options={solid,fill=mycolor1},sharp plot}% ext buy lifts
    \csname pgfplots@addlegendimage\endcsname{line width=1.5pt,mycolor2,solid,mark=*,mark options={solid,fill=mycolor2},sharp plot}%ext sell lifts    
    \csname pgfplots@addlegendimage\endcsname{line width=1pt,mycolor1,dashed,mark=o,mark options={solid},sharp plot}%ext buy
    \csname pgfplots@addlegendimage\endcsname{line width=1pt,mycolor2,dashed,mark=o,mark options={solid},sharp plot}%ext sell


    % draws the legend:
    \csname pgfplots@createlegend\endcsname
    \endgroup
\end{tikzpicture}
\end{subfigure}%
  \caption{Sample paths of the optimal trading strategies, showing price, limit order posting depths, executed market orders, and filled limit orders.}
  \label{fig:samplepath_paths}
\end{figure}

\begin{figure}
\centering
\begin{subfigure}{.45\linewidth}
  \centering
  \setlength\figureheight{\linewidth} 
  \setlength\figurewidth{\linewidth}
  \tikzsetnextfilename{samplepath_cts_depths}
  % This file was created by matlab2tikz.
%
%The latest updates can be retrieved from
%  http://www.mathworks.com/matlabcentral/fileexchange/22022-matlab2tikz-matlab2tikz
%where you can also make suggestions and rate matlab2tikz.
%
\begin{tikzpicture}[trim axis left, trim axis right]

\begin{axis}[%
width=\figurewidth,
height=\figureheight,
at={(0\figurewidth,0\figureheight)},
scale only axis,
every outer x axis line/.append style={black},
every x tick label/.append style={font=\color{black}},
xmin=10.975,
xmax=11,
every outer y axis line/.append style={black},
every y tick label/.append style={font=\color{black}},
ymin=-0.01,
ymax=0.015,
ytick={-0.01,-0.005,0,0.005,0.01,0.015},
yticklabels={{ 0.01},{0.005},{    0},{0.005},{ 0.01},{0.015}},
axis background/.style={fill=white},
axis x line*=bottom,
axis y line*=left,
yticklabel style={
        /pgf/number format/fixed,
        /pgf/number format/precision=3
},
scaled y ticks=false,
legend style={legend cell align=left,align=left,draw=black}
]
\addplot [color=white!60!black,solid,line width=2.0pt]
  table[row sep=crcr]{%
10.975	0.01\\
10.9752777777778	0.01\\
10.9755555555556	0.01\\
10.9758333333333	0.01\\
10.9761111111111	0.01\\
10.9763888888889	0.01\\
10.9766666666667	0.01\\
10.9769444444444	0.01\\
10.9772222222222	0.01\\
10.9775	0.00525611667063562\\
10.9777777777778	0.00525611667063562\\
10.9780555555556	0.01\\
10.9783333333333	0.00629947305926094\\
10.9786111111111	0.00629947305926094\\
10.9788888888889	0.00629947305926094\\
10.9791666666667	0.00715502191955105\\
10.9794444444444	0.000100087100726632\\
10.9797222222222	0.00715502191955105\\
10.98	0.0036479471727682\\
10.9802777777778	0.01\\
10.9805555555556	0.00715502191955105\\
10.9808333333333	0.00715502191955105\\
10.9811111111111	0.00112023893709131\\
10.9813888888889	0.01\\
10.9816666666667	0.01\\
10.9819444444444	0.01\\
10.9822222222222	0.01\\
10.9825	0.01\\
10.9827777777778	0.01\\
10.9830555555556	0.01\\
10.9833333333333	0.00715502191955105\\
10.9836111111111	0.00715502191955105\\
10.9838888888889	0.00715502191955105\\
10.9841666666667	0.00715502191955105\\
10.9844444444444	0.01\\
10.9847222222222	0.01\\
10.985	0.000835430950151666\\
10.9852777777778	0.00715502191955105\\
10.9855555555556	0.01\\
10.9858333333333	0.000835430950151666\\
10.9861111111111	0.00715502191955105\\
10.9863888888889	0.01\\
10.9866666666667	0.01\\
10.9869444444444	0.01\\
10.9872222222222	0.01\\
10.9875	0.01\\
10.9877777777778	0.01\\
10.9880555555556	0.01\\
10.9883333333333	0.00867512733790384\\
10.9886111111111	0.00738808561287795\\
10.9888888888889	0.00579976473268557\\
10.9891666666667	0.01\\
10.9894444444444	0.00579976473268557\\
10.9897222222222	0.00579976473268557\\
10.99	0.00738808561287795\\
10.9902777777778	0.00738808561287795\\
10.9905555555556	0.00738808561287795\\
10.9908333333333	0.00738808561287795\\
10.9911111111111	0.00112023893709131\\
10.9913888888889	0.01\\
10.9916666666667	0.01\\
10.9919444444444	0.01\\
10.9922222222222	0.01\\
10.9925	0.01\\
10.9927777777778	0.01\\
10.9930555555556	0.01\\
10.9933333333333	0.01\\
10.9936111111111	0.00402362825950001\\
10.9938888888889	0.01\\
10.9941666666667	0.01\\
10.9944444444444	0.00922742732202166\\
10.9947222222222	0.01\\
10.995	0.01\\
10.9952777777778	0.00180108844777369\\
10.9955555555556	0.01\\
10.9958333333333	0.01\\
10.9961111111111	0.00733671995772021\\
10.9963888888889	0.00525611667063562\\
10.9966666666667	0.00525611667063562\\
10.9969444444444	0.01\\
10.9972222222222	0.01\\
10.9975	0.000100087100726632\\
10.9977777777778	0.0025117741429678\\
10.9980555555556	0.00497960893516648\\
10.9983333333333	0.01\\
10.9986111111111	0.01\\
10.9988888888889	0.01\\
10.9991666666667	0.01\\
10.9994444444444	0.01\\
10.9997222222222	0.01\\
11	0.01\\
};
\addlegendentry{$\delta^-$ (sell depth)};

\addplot [color=white!40!black,solid,line width=2.0pt]
  table[row sep=crcr]{%
10.975	-0\\
10.9752777777778	-0\\
10.9755555555556	-0\\
10.9758333333333	-0\\
10.9761111111111	-0\\
10.9763888888889	-0\\
10.9766666666667	-0\\
10.9769444444444	-0.00284497808044895\\
10.9772222222222	-0.00284497808044895\\
10.9775	-0.00529220258614084\\
10.9777777777778	-0.00529220258614084\\
10.9780555555556	-0\\
10.9783333333333	-0.00420023526731443\\
10.9786111111111	-0.00420023526731443\\
10.9788888888889	-0.00420023526731443\\
10.9791666666667	-0.00370052694073906\\
10.9794444444444	-0.01\\
10.9797222222222	-0.00370052694073906\\
10.98	-0.00666392802235069\\
10.9802777777778	-0\\
10.9805555555556	-0.00370052694073906\\
10.9808333333333	-0.00370052694073906\\
10.9811111111111	-0.01\\
10.9813888888889	-0\\
10.9816666666667	-0\\
10.9819444444444	-0\\
10.9822222222222	-0\\
10.9825	-0\\
10.9827777777778	-0\\
10.9830555555556	-0\\
10.9833333333333	-0.00370052694073906\\
10.9836111111111	-0.00370052694073906\\
10.9838888888889	-0.00370052694073906\\
10.9841666666667	-0.00370052694073906\\
10.9844444444444	-0\\
10.9847222222222	-0\\
10.985	-0.01\\
10.9852777777778	-0.00370052694073906\\
10.9855555555556	-0\\
10.9858333333333	-0.01\\
10.9861111111111	-0.00370052694073906\\
10.9863888888889	-0\\
10.9866666666667	-0\\
10.9869444444444	-0\\
10.9872222222222	-0\\
10.9875	-0\\
10.9877777777778	-0\\
10.9880555555556	-0\\
10.9883333333333	-0.00261191438712204\\
10.9886111111111	-0.0028170667890354\\
10.9888888888889	-0.00433659404002523\\
10.9891666666667	-0\\
10.9894444444444	-0.00433659404002523\\
10.9897222222222	-0.00433659404002523\\
10.99	-0.0028170667890354\\
10.9902777777778	-0.0028170667890354\\
10.9905555555556	-0.0028170667890354\\
10.9908333333333	-0.0028170667890354\\
10.9911111111111	-0.01\\
10.9913888888889	-0\\
10.9916666666667	-0\\
10.9919444444444	-0\\
10.9922222222222	-0\\
10.9925	-0\\
10.9927777777778	-0\\
10.9930555555556	-0\\
10.9933333333333	-0\\
10.9936111111111	-0.00819891155222631\\
10.9938888888889	-0\\
10.9941666666667	-0\\
10.9944444444444	-0.00597637174049999\\
10.9947222222222	-0\\
10.995	-0\\
10.9952777777778	-0.00956306055999642\\
10.9955555555556	-0.00284497808044895\\
10.9958333333333	-0.00284497808044895\\
10.9961111111111	-0.00474388332936438\\
10.9963888888889	-0.00529220258614084\\
10.9966666666667	-0.00529220258614084\\
10.9969444444444	-0\\
10.9972222222222	-0.00132487266209616\\
10.9975	-0.01\\
10.9977777777778	-0.00916456904984833\\
10.9980555555556	-0.00788162502896627\\
10.9983333333333	-0\\
10.9986111111111	-0\\
10.9988888888889	-0\\
10.9991666666667	-0.00284497808044895\\
10.9994444444444	-0.00284497808044895\\
10.9997222222222	-0.00284497808044895\\
11	-0.00284497808044895\\
};
\addlegendentry{$\delta^+$ (buy depth)};

\end{axis}
\end{tikzpicture}%

\end{subfigure}%
\hfill%
\begin{subfigure}{.45\linewidth}
  \centering
  \setlength\figureheight{\linewidth} 
  \setlength\figurewidth{\linewidth}
  \tikzsetnextfilename{samplepath_dscr_depths}
  % This file was created by matlab2tikz.
%
%The latest updates can be retrieved from
%  http://www.mathworks.com/matlabcentral/fileexchange/22022-matlab2tikz-matlab2tikz
%where you can also make suggestions and rate matlab2tikz.
%
\begin{tikzpicture}

\begin{axis}[%
width=3.742in,
height=3.694in,
at={(1.889in,0.622in)},
scale only axis,
every outer x axis line/.append style={black},
every x tick label/.append style={font=\color{black}},
xmin=10.975,
xmax=11,
xlabel={Time (h)},
every outer y axis line/.append style={black},
every y tick label/.append style={font=\color{black}},
ymin=-0.01,
ymax=0.015,
ytick={-0.01,-0.005,0,0.005,0.01,0.015},
yticklabels={{ 0.01},{0.005},{    0},{0.005},{ 0.01},{0.015}},
ylabel={LO Posting Depths},
axis background/.style={fill=white},
axis x line*=bottom,
axis y line*=left,
legend style={legend cell align=left,align=left,draw=black}
]
\addplot [color=white!60!black,solid,line width=2.0pt]
  table[row sep=crcr]{%
10.975	0.00591821805671686\\
10.9752777777778	0.00591821805671686\\
10.9755555555556	0.00591821805671686\\
10.9758333333333	0.00591821805671686\\
10.9761111111111	0.00591821805671686\\
10.9763888888889	0.00591821805671686\\
10.9766666666667	0.00591821805671686\\
10.9769444444444	0.00568342788052975\\
10.9772222222222	0.00568342788052975\\
10.9775	0.00337255323101976\\
10.9777777777778	0.00337255323101976\\
10.9780555555556	0.00118499496413139\\
10.9783333333333	0.00561198181015415\\
10.9786111111111	0.00561198181015415\\
10.9788888888889	0.00561198181015415\\
10.9791666666667	0.00561198181015415\\
10.9794444444444	0.0108655723144592\\
10.9797222222222	0.00561198181015415\\
10.98	0.00100677636654667\\
10.9802777777778	0.00113803812455127\\
10.9805555555556	0.00554668592302211\\
10.9808333333333	0.00554668592302211\\
10.9811111111111	0.00900405066718764\\
10.9813888888889	0.00113803812455127\\
10.9816666666667	0.0100275388030968\\
10.9819444444444	0.0100275388030968\\
10.9822222222222	0.00775517327970128\\
10.9825	0.00775517327970128\\
10.9827777777778	0.00775517327970128\\
10.9830555555556	0.00775517327970128\\
10.9833333333333	0.00549442162806638\\
10.9836111111111	0.00549442162806638\\
10.9838888888889	0.00549442162806638\\
10.9841666666667	0.00546846610034147\\
10.9844444444444	0.0031016657810417\\
10.9847222222222	0.0100186652222556\\
10.985	0.0111119245754859\\
10.9852777777778	0.00546846610034147\\
10.9855555555556	0.00106862778989292\\
10.9858333333333	0.0111006501925519\\
10.9861111111111	0.00544780230268703\\
10.9863888888889	0.00105149996421414\\
10.9866666666667	0.010015636644651\\
10.9869444444444	0.010015636644651\\
10.9872222222222	0.010015636644651\\
10.9875	0.010015636644651\\
10.9877777777778	0.010015636644651\\
10.9880555555556	0.0100140341327421\\
10.9883333333333	0.00765623600652009\\
10.9886111111111	0.00763752137837441\\
10.9888888888889	0.00537900494058132\\
10.9891666666667	0.00101166779590929\\
10.9894444444444	0.00537900494058132\\
10.9897222222222	0.00537900494058132\\
10.99	0.00763752137837441\\
10.9902777777778	0.00763752137837441\\
10.9905555555556	0.00763752137837441\\
10.9908333333333	0.00763752137837441\\
10.9911111111111	0.00881988840160116\\
10.9913888888889	0.000987490261742568\\
10.9916666666667	0.0076161066836826\\
10.9919444444444	0.0076161066836826\\
10.9922222222222	0.00303794058045924\\
10.9925	0.0100102121666927\\
10.9927777777778	0.0100102121666927\\
10.9930555555556	0.0100102121666927\\
10.9933333333333	0.00428753092284626\\
10.9936111111111	0.0126749211683205\\
10.9938888888889	0.0100102121666927\\
10.9941666666667	0.0100102121666927\\
10.9944444444444	0.0126749211683205\\
10.9947222222222	0.00534993424615531\\
10.995	0.000959612833554843\\
10.9952777777778	0.0126749211683205\\
10.9955555555556	0.00534993424615531\\
10.9958333333333	0.00534993424615531\\
10.9961111111111	0.00307629632740271\\
10.9963888888889	0.00307629632740271\\
10.9966666666667	0.00307629632740271\\
10.9969444444444	0.000987490261742568\\
10.9972222222222	0.0076161066836826\\
10.9975	0.0107069403663601\\
10.9977777777778	0.0110643450662585\\
10.9980555555556	0.0106903541471763\\
10.9983333333333	0.00763752137837441\\
10.9986111111111	0.0100122475992815\\
10.9988888888889	0.0076161066836826\\
10.9991666666667	0.00534993424615531\\
10.9994444444444	0.00534993424615531\\
10.9997222222222	0.00534993424615531\\
11	0.00534993424615531\\
};
\addlegendentry{$\delta^-$ (sell depth)};

\addplot [color=white!40!black,solid,line width=2.0pt]
  table[row sep=crcr]{%
10.975	-0.0043118506044314\\
10.9752777777778	-0.0043118506044314\\
10.9755555555556	-0.0043118506044314\\
10.9758333333333	-0.0043118506044314\\
10.9761111111111	-0.0043118506044314\\
10.9763888888889	-0.0043118506044314\\
10.9766666666667	-0.0043118506044314\\
10.9769444444444	-0.00438576404607996\\
10.9772222222222	-0.00438576404607996\\
10.9775	-0.00670440793461694\\
10.9777777777778	-0.00670440793461694\\
10.9780555555556	-0.00883758852350922\\
10.9783333333333	-0.00445138210990432\\
10.9786111111111	-0.00445138210990432\\
10.9788888888889	-0.00445138210990432\\
10.9791666666667	-0.00445138210990432\\
10.9794444444444	-0\\
10.9797222222222	-0.00445138210990432\\
10.98	-0.00909007806073707\\
10.9802777777778	-0.00887919010845352\\
10.9805555555556	-0.00450432067409909\\
10.9808333333333	-0.00450432067409909\\
10.9811111111111	-0.00101502522624212\\
10.9813888888889	-0.00887919010845352\\
10.9816666666667	-0\\
10.9819444444444	-0\\
10.9822222222222	-0.00227736532449263\\
10.9825	-0.00227736532449263\\
10.9827777777778	-0.00227736532449263\\
10.9830555555556	-0.00227736532449263\\
10.9833333333333	-0.00453076881645172\\
10.9836111111111	-0.00453076881645172\\
10.9838888888889	-0.00453076881645172\\
10.9841666666667	-0.00455150936370967\\
10.9844444444444	-0.00690147017578766\\
10.9847222222222	-0\\
10.985	-0\\
10.9852777777778	-0.00455150936370967\\
10.9855555555556	-0.0089387697884026\\
10.9858333333333	-0\\
10.9861111111111	-0.00457235727789788\\
10.9863888888889	-0.0089565035762392\\
10.9866666666667	-0\\
10.9869444444444	-0\\
10.9872222222222	-0\\
10.9875	-0\\
10.9877777777778	-0\\
10.9880555555556	-0\\
10.9883333333333	-0.00236166998765341\\
10.9886111111111	-0.00238296460419384\\
10.9888888888889	-0.00464902033895635\\
10.9891666666667	-0.00899856307228236\\
10.9894444444444	-0.00464902033895635\\
10.9897222222222	-0.00464902033895635\\
10.99	-0.00238296460419384\\
10.9902777777778	-0.00238296460419384\\
10.9905555555556	-0.00238296460419384\\
10.9908333333333	-0.00238296460419384\\
10.9911111111111	-0.00118552248698985\\
10.9913888888889	-0.00902430997363722\\
10.9916666666667	-0.00240753134907238\\
10.9919444444444	-0.00240753134907238\\
10.9922222222222	-0.00696645880494782\\
10.9925	-0\\
10.9927777777778	-0\\
10.9930555555556	-0\\
10.9933333333333	-0.00571382654755853\\
10.9936111111111	-0\\
10.9938888888889	-0\\
10.9941666666667	-0\\
10.9944444444444	-0\\
10.9947222222222	-0.00468273019067273\\
10.995	-0.00905402104706181\\
10.9952777777778	-0\\
10.9955555555556	-0.00468273019067273\\
10.9958333333333	-0.00468273019067273\\
10.9961111111111	-0.0069569103200494\\
10.9963888888889	-0.0069569103200494\\
10.9966666666667	-0.0069569103200494\\
10.9969444444444	-0.00902430997363722\\
10.9972222222222	-0.00240753134907238\\
10.9975	-0\\
10.9977777777778	-0\\
10.9980555555556	-0\\
10.9983333333333	-0.00238296460419384\\
10.9986111111111	-0\\
10.9988888888889	-0.00240753134907238\\
10.9991666666667	-0.00468273019067273\\
10.9994444444444	-0.00468273019067273\\
10.9997222222222	-0.00468273019067273\\
11	-0.00468273019067273\\
};
\addlegendentry{$\delta^+$ (buy depth)};

\end{axis}
\end{tikzpicture}% 
\end{subfigure}\\
\vspace{1cm}
\begin{subfigure}{.45\linewidth}
  \centering
  \setlength\figureheight{\linewidth} 
  \setlength\figurewidth{\linewidth}
  \tikzsetnextfilename{samplepath_cts_nFPC_depths}
  % This file was created by matlab2tikz.
%
%The latest updates can be retrieved from
%  http://www.mathworks.com/matlabcentral/fileexchange/22022-matlab2tikz-matlab2tikz
%where you can also make suggestions and rate matlab2tikz.
%
\begin{tikzpicture}

\begin{axis}[%
width=3.742in,
height=3.694in,
at={(1.889in,0.622in)},
scale only axis,
every outer x axis line/.append style={black},
every x tick label/.append style={font=\color{black}},
xmin=10.975,
xmax=11,
xlabel={Time (h)},
every outer y axis line/.append style={black},
every y tick label/.append style={font=\color{black}},
ymin=-0.01,
ymax=0.015,
ytick={-0.01,-0.005,0,0.005,0.01,0.015},
yticklabels={{ 0.01},{0.005},{    0},{0.005},{ 0.01},{0.015}},
ylabel={LO Posting Depths},
axis background/.style={fill=white},
axis x line*=bottom,
axis y line*=left,
legend style={legend cell align=left,align=left,draw=black}
]
\addplot [color=white!60!black,solid,line width=2.0pt]
  table[row sep=crcr]{%
10.975	0.01\\
10.9752777777778	0.01\\
10.9755555555556	0.01\\
10.9758333333333	0.01\\
10.9761111111111	0.01\\
10.9763888888889	0.01\\
10.9766666666667	0.01\\
10.9769444444444	0.01\\
10.9772222222222	0.01\\
10.9775	0.00817323652453748\\
10.9777777777778	0.01\\
10.9780555555556	0.00842461772871728\\
10.9783333333333	0.00842461772871728\\
10.9786111111111	0.00842461772871728\\
10.9788888888889	0.00842461772871728\\
10.9791666666667	0.000526743864001341\\
10.9794444444444	0.01\\
10.9797222222222	0.01\\
10.98	0.01\\
10.9802777777778	0.01\\
10.9805555555556	0.01\\
10.9808333333333	0.000526743864001341\\
10.9811111111111	0.01\\
10.9813888888889	0.01\\
10.9816666666667	0.01\\
10.9819444444444	0.01\\
10.9822222222222	0.01\\
10.9825	0.01\\
10.9827777777778	0.01\\
10.9830555555556	0.01\\
10.9833333333333	0.01\\
10.9836111111111	0.01\\
10.9838888888889	0.01\\
10.9841666666667	0.01\\
10.9844444444444	0.01\\
10.9847222222222	0.00162143504819226\\
10.985	0.01\\
10.9852777777778	0.01\\
10.9855555555556	0.000526743864001341\\
10.9858333333333	0.01\\
10.9861111111111	0.01\\
10.9863888888889	0.01\\
10.9866666666667	0.01\\
10.9869444444444	0.01\\
10.9872222222222	0.01\\
10.9875	0.01\\
10.9877777777778	0.01\\
10.9880555555556	0.01\\
10.9883333333333	0.01\\
10.9886111111111	0.01\\
10.9888888888889	0.01\\
10.9891666666667	0.00671160941927011\\
10.9894444444444	0.00671160941927011\\
10.9897222222222	0.00671160941927011\\
10.99	0.00863206466588298\\
10.9902777777778	0.00863206466588298\\
10.9905555555556	0.00863206466588298\\
10.9908333333333	0.000526179883314139\\
10.9911111111111	0.01\\
10.9913888888889	0.01\\
10.9916666666667	0.01\\
10.9919444444444	0.01\\
10.9922222222222	0.01\\
10.9925	0.01\\
10.9927777777778	0.01\\
10.9930555555556	0.01\\
10.9933333333333	0.00162143504819226\\
10.9936111111111	0.01\\
10.9938888888889	0.01\\
10.9941666666667	0.00162143504819226\\
10.9944444444444	0.01\\
10.9947222222222	0.01\\
10.995	0.000526743864001341\\
10.9952777777778	0.01\\
10.9955555555556	0.01\\
10.9958333333333	0.01\\
10.9961111111111	0.01\\
10.9963888888889	0.01\\
10.9966666666667	0.01\\
10.9969444444444	0.01\\
10.9972222222222	0.000526179883314139\\
10.9975	0.00342179207905144\\
10.9977777777778	0\\
10.9980555555556	0.00120095598319226\\
10.9983333333333	0.00491204156068829\\
10.9986111111111	0.00642002679226435\\
10.9988888888889	0.00334305978087883\\
10.9991666666667	0.00120095598319226\\
10.9994444444444	0.00120095598319226\\
10.9997222222222	0.00120095598319226\\
11	0.00120095598319226\\
};
\addlegendentry{$\delta^-$ (sell depth)};

\addplot [color=white!40!black,solid,line width=2.0pt]
  table[row sep=crcr]{%
10.975	-0\\
10.9752777777778	-0\\
10.9755555555556	-0\\
10.9758333333333	-0\\
10.9761111111111	-0\\
10.9763888888889	-0\\
10.9766666666667	-0\\
10.9769444444444	-0\\
10.9772222222222	-0\\
10.9775	-0.00393602061303001\\
10.9777777777778	-0\\
10.9780555555556	-0.0032883905807299\\
10.9783333333333	-0.0032883905807299\\
10.9786111111111	-0.0032883905807299\\
10.9788888888889	-0.0032883905807299\\
10.9791666666667	-0.01\\
10.9794444444444	-0\\
10.9797222222222	-0\\
10.98	-0\\
10.9802777777778	-0\\
10.9805555555556	-0\\
10.9808333333333	-0.01\\
10.9811111111111	-0\\
10.9813888888889	-0\\
10.9816666666667	-0\\
10.9819444444444	-0\\
10.9822222222222	-0\\
10.9825	-0\\
10.9827777777778	-0\\
10.9830555555556	-0\\
10.9833333333333	-0.00157538227128272\\
10.9836111111111	-0.00157538227128272\\
10.9838888888889	-0.00157538227128272\\
10.9841666666667	-0\\
10.9844444444444	-0.00136793533411702\\
10.9847222222222	-0.01\\
10.985	-0\\
10.9852777777778	-0\\
10.9855555555556	-0.01\\
10.9858333333333	-0\\
10.9861111111111	-0\\
10.9863888888889	-0\\
10.9866666666667	-0\\
10.9869444444444	-0\\
10.9872222222222	-0\\
10.9875	-0\\
10.9877777777778	-0\\
10.9880555555556	-0\\
10.9883333333333	-0\\
10.9886111111111	-0.00136793533411702\\
10.9888888888889	-0\\
10.9891666666667	-0.00400283612999854\\
10.9894444444444	-0.00400283612999854\\
10.9897222222222	-0.00400283612999854\\
10.99	-0.00265697385624915\\
10.9902777777778	-0.00265697385624915\\
10.9905555555556	-0.00265697385624915\\
10.9908333333333	-0.01\\
10.9911111111111	-0\\
10.9913888888889	-0\\
10.9916666666667	-0\\
10.9919444444444	-0\\
10.9922222222222	-0\\
10.9925	-0\\
10.9927777777778	-0\\
10.9930555555556	-0\\
10.9933333333333	-0.01\\
10.9936111111111	-0\\
10.9938888888889	-0\\
10.9941666666667	-0.01\\
10.9944444444444	-0\\
10.9947222222222	-0\\
10.995	-0.01\\
10.9952777777778	-0\\
10.9955555555556	-0\\
10.9958333333333	-0\\
10.9961111111111	-0\\
10.9963888888889	-0.00182676347546253\\
10.9966666666667	-0\\
10.9969444444444	-0.00157538227128272\\
10.9972222222222	-0.01\\
10.9975	-0.00947325613599866\\
10.9977777777778	-0.01\\
10.9980555555556	-0.01\\
10.9983333333333	-0.00665694021912117\\
10.9986111111111	-0.00443985676341491\\
10.9988888888889	-0.01\\
10.9991666666667	-0.01\\
10.9994444444444	-0.01\\
10.9997222222222	-0.01\\
11	-0.01\\
};
\addlegendentry{$\delta^+$ (buy depth)};

\end{axis}
\end{tikzpicture}%
\end{subfigure}%
\hfill%
\begin{subfigure}{.45\linewidth}
  \centering
  \setlength\figureheight{\linewidth} 
  \setlength\figurewidth{\linewidth}
  \tikzsetnextfilename{samplepath_dscr_nFPC_depths}
  % This file was created by matlab2tikz.
%
%The latest updates can be retrieved from
%  http://www.mathworks.com/matlabcentral/fileexchange/22022-matlab2tikz-matlab2tikz
%where you can also make suggestions and rate matlab2tikz.
%
\begin{tikzpicture}

\begin{axis}[%
width=3.742in,
height=3.694in,
at={(1.889in,0.622in)},
scale only axis,
every outer x axis line/.append style={black},
every x tick label/.append style={font=\color{black}},
xmin=10.975,
xmax=11,
xlabel={Time (h)},
every outer y axis line/.append style={black},
every y tick label/.append style={font=\color{black}},
ymin=-0.015,
ymax=0.015,
ytick={-0.015,-0.01,-0.005,0,0.005,0.01,0.015},
yticklabels={{0.015},{ 0.01},{0.005},{    0},{0.005},{ 0.01},{0.015}},
ylabel={LO Posting Depths},
axis background/.style={fill=white},
axis x line*=bottom,
axis y line*=left,
legend style={legend cell align=left,align=left,draw=black}
]
\addplot [color=white!60!black,solid,line width=2.0pt]
  table[row sep=crcr]{%
10.975	0.00955914496776727\\
10.9752777777778	0.00955914496776727\\
10.9755555555556	0.00955914496776727\\
10.9758333333333	0.00955914496776727\\
10.9761111111111	0.00955914496776727\\
10.9763888888889	0.00955914496776727\\
10.9766666666667	0.00955914496776727\\
10.9769444444444	0.00779076841678636\\
10.9772222222222	0.00779076841678636\\
10.9775	0.0044883451970511\\
10.9777777777778	0\\
10.9780555555556	0.00701443138225722\\
10.9783333333333	0.00701443138225722\\
10.9786111111111	0.00701443138225722\\
10.9788888888889	0.00701443138225722\\
10.9791666666667	0.0118829629195527\\
10.9794444444444	0.00701443138225722\\
10.9797222222222	0.00701443138225722\\
10.98	0\\
10.9802777777778	0.00955914496776727\\
10.9805555555556	0.00955914496776727\\
10.9808333333333	0.01265085491098\\
10.9811111111111	0.00215234612348661\\
10.9813888888889	0.00975933490261401\\
10.9816666666667	0.0102216043060101\\
10.9819444444444	0.0102216043060101\\
10.9822222222222	0.0099930529003391\\
10.9825	0.0099930529003391\\
10.9827777777778	0.0099930529003391\\
10.9830555555556	0.0099930529003391\\
10.9833333333333	0.00955914496776727\\
10.9836111111111	0.00955914496776727\\
10.9838888888889	0.00955914496776727\\
10.9841666666667	0.00248878779155287\\
10.9844444444444	0.00979691828718088\\
10.9847222222222	0.0134558419257991\\
10.985	0.00979691828718088\\
10.9852777777778	0.00248878779155287\\
10.9855555555556	0.0118829629195527\\
10.9858333333333	0.00972589182777017\\
10.9861111111111	0.00209358620056745\\
10.9863888888889	0.00701443138225722\\
10.9866666666667	0.0101671828515352\\
10.9869444444444	0.0101671828515352\\
10.9872222222222	0.0101671828515352\\
10.9875	0.0101671828515352\\
10.9877777777778	0.0101671828515352\\
10.9880555555556	0.0101581955578191\\
10.9883333333333	0.00966508722521721\\
10.9886111111111	0.00886539991386342\\
10.9888888888889	0.00142567684150565\\
10.9891666666667	0.00580267881232323\\
10.9894444444444	0.00580267881232323\\
10.9897222222222	0.00580267881232323\\
10.99	0.00766868719702822\\
10.9902777777778	0.00766868719702822\\
10.9905555555556	0.00766868719702822\\
10.9908333333333	0.0115804064569313\\
10.9911111111111	0\\
10.9913888888889	0.00779076841678636\\
10.9916666666667	0.00979691828718088\\
10.9919444444444	0.00413010131820344\\
10.9922222222222	0.00979691828718088\\
10.9925	0.0101809572800362\\
10.9927777777778	0.0101809572800362\\
10.9930555555556	0.00509864754750646\\
10.9933333333333	0.0133875791588378\\
10.9936111111111	0.0101809572800362\\
10.9938888888889	0.0101809572800362\\
10.9941666666667	0.0134558419257991\\
10.9944444444444	0.0102110954468975\\
10.9947222222222	0.00330866265227938\\
10.995	0.0121207619446429\\
10.9952777777778	0.0101809572800362\\
10.9955555555556	0.00779076841678636\\
10.9958333333333	0.00779076841678636\\
10.9961111111111	0.00493968894246695\\
10.9963888888889	0.0044883451970511\\
10.9966666666667	0\\
10.9969444444444	0.00673677651284548\\
10.9972222222222	0.0123883907620873\\
10.9975	0.0117432433564966\\
10.9977777777778	0.0125065370642039\\
10.9980555555556	0.00673677651284548\\
10.9983333333333	0.00966508722521721\\
10.9986111111111	0.0101581955578191\\
10.9988888888889	0.00886539991386342\\
10.9991666666667	0.00623355705305608\\
10.9994444444444	0.00623355705305608\\
10.9997222222222	0.00623355705305608\\
11	0.00623355705305608\\
};
\addlegendentry{$\delta^-$ (sell depth)};

\addplot [color=white!40!black,solid,line width=2.0pt]
  table[row sep=crcr]{%
10.975	-0.00212359589424925\\
10.9752777777778	-0.00212359589424925\\
10.9755555555556	-0.00212359589424925\\
10.9758333333333	-0.00212359589424925\\
10.9761111111111	-0.00212359589424925\\
10.9763888888889	-0.00212359589424925\\
10.9766666666667	-0.00212359589424925\\
10.9769444444444	-0.00294796026086679\\
10.9772222222222	-0.00294796026086679\\
10.9775	-0.00573839131972788\\
10.9777777777778	-0.0101544599606959\\
10.9780555555556	-0.003232464310747\\
10.9783333333333	-0.003232464310747\\
10.9786111111111	-0.003232464310747\\
10.9788888888889	-0.003232464310747\\
10.9791666666667	-0\\
10.9794444444444	-0.003232464310747\\
10.9797222222222	-0.003232464310747\\
10.98	-0.0104852001010911\\
10.9802777777778	-0.00212359589424925\\
10.9805555555556	-0.00212359589424925\\
10.9808333333333	-0\\
10.9811111111111	-0.00846939986349565\\
10.9813888888889	-0.000347089031771669\\
10.9816666666667	-0\\
10.9819444444444	-0\\
10.9822222222222	-3.34410504625313e-05\\
10.9825	-3.34410504625313e-05\\
10.9827777777778	-3.34410504625313e-05\\
10.9830555555556	-3.34410504625313e-05\\
10.9833333333333	-0.00212359589424925\\
10.9836111111111	-0.00212359589424925\\
10.9838888888889	-0.00212359589424925\\
10.9841666666667	-0.00782457399156521\\
10.9844444444444	-0.000269677342648533\\
10.9847222222222	-0\\
10.985	-0.000269677342648533\\
10.9852777777778	-0.00782457399156521\\
10.9855555555556	-0\\
10.9858333333333	-0.000298251561890983\\
10.9861111111111	-0.00804787663159079\\
10.9863888888889	-0.003232464310747\\
10.9866666666667	-0\\
10.9869444444444	-0\\
10.9872222222222	-0\\
10.9875	-0\\
10.9877777777778	-0\\
10.9880555555556	-0\\
10.9883333333333	-0.00105099237168575\\
10.9886111111111	-0.00226486415908192\\
10.9888888888889	-0.00891805972735326\\
10.9891666666667	-0.00431435392241137\\
10.9894444444444	-0.00431435392241137\\
10.9897222222222	-0.00431435392241137\\
10.99	-0.0025255406205822\\
10.9902777777778	-0.0025255406205822\\
10.9905555555556	-0.0025255406205822\\
10.9908333333333	-0\\
10.9911111111111	-0.0101544599606959\\
10.9913888888889	-0.00294796026086679\\
10.9916666666667	-0.000269677342648533\\
10.9919444444444	-0.00607099290853057\\
10.9922222222222	-0.000269677342648533\\
10.9925	-0\\
10.9927777777778	-0\\
10.9930555555556	-0.00497752657405822\\
10.9933333333333	-0\\
10.9936111111111	-0\\
10.9938888888889	-0\\
10.9941666666667	-0\\
10.9944444444444	-0\\
10.9947222222222	-0.00734478453907611\\
10.995	-0\\
10.9952777777778	-0\\
10.9955555555556	-0.00294796026086679\\
10.9958333333333	-0.00294796026086679\\
10.9961111111111	-0.00548362231286444\\
10.9963888888889	-0.00573839131972788\\
10.9966666666667	-0.0101544599606959\\
10.9969444444444	-0.00373041551627003\\
10.9972222222222	-0\\
10.9975	-0\\
10.9977777777778	-0\\
10.9980555555556	-0.00373041551627003\\
10.9983333333333	-0.00105099237168575\\
10.9986111111111	-0\\
10.9988888888889	-0.00226486415908192\\
10.9991666666667	-0.00417562798910466\\
10.9994444444444	-0.00417562798910466\\
10.9997222222222	-0.00417562798910466\\
11	-0.00417562798910466\\
};
\addlegendentry{$\delta^+$ (buy depth)};

\end{axis}
\end{tikzpicture}% 
\end{subfigure}\\
\leavevmode\smash{\makebox[0pt]{\hspace{-7em}% HORIZONTAL POSITION           
  \rotatebox[origin=l]{90}{\hspace{20em}% VERTICAL POSITION
    PnL}%
}}\hspace{0pt plus 1filll}\null

Time (h)

\vspace{1cm}
\begin{subfigure}{\linewidth}
  \centering
  \setlength\figureheight{\linewidth} 
  \setlength\figurewidth{\linewidth}
  \tikzsetnextfilename{samplepathslegend}
  \definecolor{mycolor1}{rgb}{0.65098,0.80784,0.89020}%
\definecolor{mycolor2}{rgb}{0.69804,0.87451,0.54118}%
\definecolor{mycolor3}{rgb}{0.20000,0.62745,0.17255}%
\definecolor{mycolor4}{rgb}{0.12157,0.47059,0.70588}%
\begin{tikzpicture}
    \begingroup
    % inits/clears the lists (which might be populated from previous
    % axes):
    \csname pgfplots@init@cleared@structures\endcsname
    \pgfplotsset{legend cell align=left,legend columns = 2,legend style={at={(0,1)},anchor=north west},legend style={draw=black,column sep=1ex},
    legend entries={Midprice,
    				Midprice $\pm \delta^\pm$,
    				Our Sell MO,
    				Our Buy MO,
    				Ext Buy MO lifts our sell LO,
    				Ext Sell MO lifts our buy LO,
    				Ext Buy MO arrives,
    				Ext Sell MO arrives}}%
    \csname pgfplots@addlegendimage\endcsname{line width=2pt,black,solid,sharp plot}
    \csname pgfplots@addlegendimage\endcsname{line width=2pt,gray,solid,sharp plot}
    \csname pgfplots@addlegendimage\endcsname{line width=1.5pt,mycolor4,solid,mark=*,mark options={solid,fill=mycolor4},sharp plot}%sell
    \csname pgfplots@addlegendimage\endcsname{line width=1.5pt,mycolor3,solid,mark=*,mark options={solid,fill=mycolor3},sharp plot}%buy
    \csname pgfplots@addlegendimage\endcsname{line width=1.5pt,mycolor1,solid,mark=*,mark options={solid,fill=mycolor1},sharp plot}% ext buy lifts
    \csname pgfplots@addlegendimage\endcsname{line width=1.5pt,mycolor2,solid,mark=*,mark options={solid,fill=mycolor2},sharp plot}%ext sell lifts    
    \csname pgfplots@addlegendimage\endcsname{line width=1pt,mycolor1,dashed,mark=o,mark options={solid},sharp plot}%ext buy
    \csname pgfplots@addlegendimage\endcsname{line width=1pt,mycolor2,dashed,mark=o,mark options={solid},sharp plot}%ext sell


    % draws the legend:
    \csname pgfplots@createlegend\endcsname
    \endgroup
\end{tikzpicture}
\end{subfigure}%
  \caption{Sample paths of the optimal trading strategies, showing price, limit order posting depths, executed market orders, and filled limit orders.}
  \label{fig:samplepath_depths}
\end{figure}

\begin{figure}
\centering
\begin{subfigure}{.45\linewidth}
  \centering
  \setlength\figureheight{\linewidth} 
  \setlength\figurewidth{\linewidth}
  \tikzsetnextfilename{samplepath_cts_bookvalue}
  % This file was created by matlab2tikz.
%
%The latest updates can be retrieved from
%  http://www.mathworks.com/matlabcentral/fileexchange/22022-matlab2tikz-matlab2tikz
%where you can also make suggestions and rate matlab2tikz.
%
\begin{tikzpicture}[trim axis left, trim axis right]

\begin{axis}[%
width=\figurewidth,
height=\figureheight,
at={(0\figurewidth,0\figureheight)},
scale only axis,
separate axis lines,
every outer x axis line/.append style={black},
every x tick label/.append style={font=\color{black}},
xtick = {10.97,10.98,10.99,11},
xmin=10.975,
xmax=11,
every outer y axis line/.append style={black},
every y tick label/.append style={font=\color{black}},
ymin=1.56,
ymax=1.72,
axis background/.style={fill=white}
]
\addplot [color=black,solid,line width=2.0pt,forget plot]
  table[row sep=crcr]{%
10.975	1.62135739067691\\
10.9752777777778	1.62135739067691\\
10.9755555555556	1.62135739067691\\
10.9758333333333	1.62135739067691\\
10.9761111111111	1.62135739067691\\
10.9763888888889	1.62135739067691\\
10.9766666666667	1.62135739067691\\
10.9769444444444	1.6013573906769\\
10.9772222222222	1.6013573906769\\
10.9775	1.58420236875733\\
10.9777777777778	1.58420236875733\\
10.9780555555556	1.56949457134348\\
10.9783333333333	1.56949457134348\\
10.9786111111111	1.56949457134348\\
10.9788888888889	1.56949457134348\\
10.9791666666667	1.60579404440273\\
10.9794444444444	1.60579404440273\\
10.9797222222222	1.60579404440273\\
10.98	1.60579404440273\\
10.9802777777778	1.60579404440273\\
10.9805555555556	1.60579404440273\\
10.9808333333333	1.60579404440273\\
10.9811111111111	1.64294906632227\\
10.9813888888889	1.64294906632227\\
10.9816666666667	1.64294906632227\\
10.9819444444444	1.64294906632227\\
10.9822222222222	1.64294906632227\\
10.9825	1.64294906632227\\
10.9827777777778	1.64294906632227\\
10.9830555555556	1.64294906632227\\
10.9833333333333	1.63294906632228\\
10.9836111111111	1.63294906632228\\
10.9838888888889	1.63294906632228\\
10.9841666666667	1.63294906632228\\
10.9844444444444	1.63294906632228\\
10.9847222222222	1.63294906632228\\
10.985	1.63294906632228\\
10.9852777777778	1.63294906632228\\
10.9855555555556	1.62664959326301\\
10.9858333333333	1.636649593263\\
10.9861111111111	1.636649593263\\
10.9863888888889	1.636649593263\\
10.9866666666667	1.636649593263\\
10.9869444444444	1.636649593263\\
10.9872222222222	1.636649593263\\
10.9875	1.636649593263\\
10.9877777777778	1.636649593263\\
10.9880555555556	1.636649593263\\
10.9883333333333	1.636649593263\\
10.9886111111111	1.63926150765013\\
10.9888888888889	1.63926150765013\\
10.9891666666667	1.63926150765013\\
10.9894444444444	1.63926150765013\\
10.9897222222222	1.63926150765013\\
10.99	1.63926150765013\\
10.9902777777778	1.63926150765013\\
10.9905555555556	1.63926150765013\\
10.9908333333333	1.63926150765013\\
10.9911111111111	1.63926150765013\\
10.9913888888889	1.65038174658721\\
10.9916666666667	1.65038174658721\\
10.9919444444444	1.65038174658721\\
10.9922222222222	1.65038174658721\\
10.9925	1.65038174658721\\
10.9927777777778	1.65038174658721\\
10.9930555555556	1.65038174658721\\
10.9933333333333	1.66038174658721\\
10.9936111111111	1.66038174658721\\
10.9938888888889	1.66038174658721\\
10.9941666666667	1.66038174658721\\
10.9944444444444	1.6703817465872\\
10.9947222222222	1.66038174658721\\
10.995	1.66038174658721\\
10.9952777777778	1.66038174658721\\
10.9955555555556	1.66038174658721\\
10.9958333333333	1.66038174658721\\
10.9961111111111	1.66038174658721\\
10.9963888888889	1.66512562991656\\
10.9966666666667	1.66512562991656\\
10.9969444444444	1.66512562991656\\
10.9972222222222	1.66512562991656\\
10.9975	1.66512562991656\\
10.9977777777778	1.67522571701728\\
10.9980555555556	1.69773749116025\\
10.9983333333333	1.70271710009541\\
10.9986111111111	1.70271710009541\\
10.9988888888889	1.6827171000954\\
10.9991666666667	1.6827171000954\\
10.9994444444444	1.6827171000954\\
10.9997222222222	1.6827171000954\\
11	1.6827171000954\\
};
\end{axis}
\end{tikzpicture}%

\end{subfigure}%
\hfill%
\begin{subfigure}{.45\linewidth}
  \centering
  \setlength\figureheight{\linewidth} 
  \setlength\figurewidth{\linewidth}
  \tikzsetnextfilename{samplepath_dscr_bookvalue}
  % This file was created by matlab2tikz.
%
%The latest updates can be retrieved from
%  http://www.mathworks.com/matlabcentral/fileexchange/22022-matlab2tikz-matlab2tikz
%where you can also make suggestions and rate matlab2tikz.
%
\begin{tikzpicture}[trim axis left, trim axis right]

\begin{axis}[%
width=\figurewidth,
height=\figureheight,
at={(0\figurewidth,0\figureheight)},
scale only axis,
separate axis lines,
every outer x axis line/.append style={black},
every x tick label/.append style={font=\color{black}},
xtick = {10.97,10.98,10.99,11},
xmin=10.975,
xmax=11,
every outer y axis line/.append style={black},
every y tick label/.append style={font=\color{black}},
ymin=2.22,
ymax=2.29,
axis background/.style={fill=white}
]
\addplot [color=black,solid,line width=2.0pt,forget plot]
  table[row sep=crcr]{%
10.975	2.27140001455481\\
10.9752777777778	2.27140001455481\\
10.9755555555556	2.27140001455481\\
10.9758333333333	2.27140001455481\\
10.9761111111111	2.27140001455481\\
10.9763888888889	2.27140001455481\\
10.9766666666667	2.27140001455481\\
10.9769444444444	2.25571186515924\\
10.9772222222222	2.25571186515924\\
10.9775	2.24009762920537\\
10.9777777777778	2.24009762920537\\
10.9780555555556	2.24009762920537\\
10.9783333333333	2.24009762920537\\
10.9786111111111	2.24009762920537\\
10.9788888888889	2.24009762920537\\
10.9791666666667	2.24009762920537\\
10.9794444444444	2.24009762920537\\
10.9797222222222	2.24009762920537\\
10.98	2.24009762920537\\
10.9802777777778	2.22918770726608\\
10.9805555555556	2.22918770726608\\
10.9808333333333	2.22918770726608\\
10.9811111111111	2.26473439318914\\
10.9813888888889	2.24574941841536\\
10.9816666666667	2.24574941841536\\
10.9819444444444	2.24574941841536\\
10.9822222222222	2.24574941841536\\
10.9825	2.24574941841536\\
10.9827777777778	2.24574941841536\\
10.9830555555556	2.24574941841536\\
10.9833333333333	2.23802678373983\\
10.9836111111111	2.23802678373983\\
10.9838888888889	2.23802678373983\\
10.9841666666667	2.23255755255633\\
10.9844444444444	2.23255755255633\\
10.9847222222222	2.23255755255633\\
10.985	2.23255755255633\\
10.9852777777778	2.23255755255633\\
10.9855555555556	2.22710906192009\\
10.9858333333333	2.22710906192009\\
10.9861111111111	2.22710906192009\\
10.9863888888889	2.22168141919798\\
10.9866666666667	2.22168141919798\\
10.9869444444444	2.22168141919798\\
10.9872222222222	2.22168141919798\\
10.9875	2.22168141919798\\
10.9877777777778	2.22168141919798\\
10.9880555555556	2.22168141919792\\
10.9883333333333	2.22168141919792\\
10.9886111111111	2.22404308918556\\
10.9888888888889	2.22404308918556\\
10.9891666666667	2.22404308918556\\
10.9894444444444	2.22404308918556\\
10.9897222222222	2.22404308918556\\
10.99	2.22404308918556\\
10.9902777777778	2.22404308918556\\
10.9905555555556	2.22404308918556\\
10.9908333333333	2.22404308918556\\
10.9911111111111	2.22404308918556\\
10.9913888888889	2.22522861167261\\
10.9916666666667	2.22522861167261\\
10.9919444444444	2.22522861167261\\
10.9922222222222	2.22522861167261\\
10.9925	2.22522861167261\\
10.9927777777778	2.22522861167261\\
10.9930555555556	2.22522861167261\\
10.9933333333333	2.2352286116726\\
10.9936111111111	2.22951614259546\\
10.9938888888889	2.22951614259546\\
10.9941666666667	2.22951614259546\\
10.9944444444444	2.22951614259546\\
10.9947222222222	2.22951614259546\\
10.995	2.23419887278612\\
10.9952777777778	2.23515848561965\\
10.9955555555556	2.23515848561965\\
10.9958333333333	2.23515848561965\\
10.9961111111111	2.23515848561965\\
10.9963888888889	2.23515848561965\\
10.9966666666667	2.23515848561965\\
10.9969444444444	2.23515848561965\\
10.9972222222222	2.23515848561965\\
10.9975	2.25277459230335\\
10.9977777777778	2.25277459230335\\
10.9980555555556	2.24277459230336\\
10.9983333333333	2.28346494645052\\
10.9986111111111	2.28346494645052\\
10.9988888888889	2.26346494645054\\
10.9991666666667	2.26346494645054\\
10.9994444444444	2.26346494645054\\
10.9997222222222	2.26346494645054\\
11	2.26346494645054\\
};
\end{axis}
\end{tikzpicture}%
 
\end{subfigure}\\
\vspace{1cm}
\begin{subfigure}{.45\linewidth}
  \centering
  \setlength\figureheight{\linewidth} 
  \setlength\figurewidth{\linewidth}
  \tikzsetnextfilename{samplepath_cts_nFPC_bookvalue}
  % This file was created by matlab2tikz.
%
%The latest updates can be retrieved from
%  http://www.mathworks.com/matlabcentral/fileexchange/22022-matlab2tikz-matlab2tikz
%where you can also make suggestions and rate matlab2tikz.
%
\begin{tikzpicture}[trim axis left, trim axis right]

\begin{axis}[%
width=\figurewidth,
height=\figureheight,
at={(0\figurewidth,0\figureheight)},
scale only axis,
separate axis lines,
every outer x axis line/.append style={black},
every x tick label/.append style={font=\color{black}},
xmin=10.975,
xmax=11,
every outer y axis line/.append style={black},
every y tick label/.append style={font=\color{black}},
ymin=-7.4,
ymax=-7.05,
axis background/.style={fill=white}
]
\addplot [color=black,solid,line width=2.0pt,forget plot]
  table[row sep=crcr]{%
10.975	-7.07965620945679\\
10.9752777777778	-7.07965620945679\\
10.9755555555556	-7.07965620945679\\
10.9758333333333	-7.07965620945679\\
10.9761111111111	-7.07965620945679\\
10.9763888888889	-7.07965620945679\\
10.9766666666667	-7.07965620945679\\
10.9769444444444	-7.09965620945678\\
10.9772222222222	-7.09965620945678\\
10.9775	-7.11965620945679\\
10.9777777777778	-7.11965620945679\\
10.9780555555556	-7.1396562094568\\
10.9783333333333	-7.1396562094568\\
10.9786111111111	-7.1396562094568\\
10.9788888888889	-7.1396562094568\\
10.9791666666667	-7.10965620945683\\
10.9794444444444	-7.10965620945683\\
10.9797222222222	-7.10965620945683\\
10.98	-7.10965620945683\\
10.9802777777778	-7.12965620945684\\
10.9805555555556	-7.12965620945684\\
10.9808333333333	-7.11965620945685\\
10.9811111111111	-7.11912946559285\\
10.9813888888889	-7.13912946559284\\
10.9816666666667	-7.13912946559284\\
10.9819444444444	-7.13912946559284\\
10.9822222222222	-7.13912946559284\\
10.9825	-7.13912946559284\\
10.9827777777778	-7.13912946559284\\
10.9830555555556	-7.13912946559284\\
10.9833333333333	-7.14912946559284\\
10.9836111111111	-7.14912946559284\\
10.9838888888889	-7.14912946559284\\
10.9841666666667	-7.15755408332154\\
10.9844444444444	-7.15755408332154\\
10.9847222222222	-7.15755408332154\\
10.985	-7.17755408332155\\
10.9852777777778	-7.17755408332155\\
10.9855555555556	-7.18755408332154\\
10.9858333333333	-7.18755408332154\\
10.9861111111111	-7.18755408332154\\
10.9863888888889	-7.19755408332153\\
10.9866666666667	-7.19755408332153\\
10.9869444444444	-7.19755408332153\\
10.9872222222222	-7.19755408332153\\
10.9875	-7.19755408332153\\
10.9877777777778	-7.19755408332153\\
10.9880555555556	-7.19755408332153\\
10.9883333333333	-7.19755408332153\\
10.9886111111111	-7.19755408332153\\
10.9888888888889	-7.19755408332153\\
10.9891666666667	-7.19755408332153\\
10.9894444444444	-7.19755408332153\\
10.9897222222222	-7.19755408332153\\
10.99	-7.19755408332153\\
10.9902777777778	-7.19755408332153\\
10.9905555555556	-7.19755408332153\\
10.9908333333333	-7.2375540833215\\
10.9911111111111	-7.23702790343819\\
10.9913888888889	-7.23702790343819\\
10.9916666666667	-7.23702790343819\\
10.9919444444444	-7.23702790343819\\
10.9922222222222	-7.23702790343819\\
10.9925	-7.23702790343819\\
10.9927777777778	-7.23702790343819\\
10.9930555555556	-7.23702790343819\\
10.9933333333333	-7.28702790343819\\
10.9936111111111	-7.29540646838998\\
10.9938888888889	-7.29540646838998\\
10.9941666666667	-7.30540646838998\\
10.9944444444444	-7.32378503334178\\
10.9947222222222	-7.32378503334178\\
10.995	-7.33378503334177\\
10.9952777777778	-7.34325828947777\\
10.9955555555556	-7.34325828947777\\
10.9958333333333	-7.34325828947777\\
10.9961111111111	-7.34325828947777\\
10.9963888888889	-7.34325828947777\\
10.9966666666667	-7.34325828947777\\
10.9969444444444	-7.34325828947777\\
10.9972222222222	-7.36325828947776\\
10.9975	-7.35273210959444\\
10.9977777777778	-7.33931031751539\\
10.9980555555556	-7.32931031751539\\
10.9983333333333	-7.30810936153219\\
10.9986111111111	-7.30810936153219\\
10.9988888888889	-7.3136695047688\\
10.9991666666667	-7.3136695047688\\
10.9994444444444	-7.3136695047688\\
10.9997222222222	-7.3136695047688\\
11	-7.3136695047688\\
};
\end{axis}
\end{tikzpicture}%

\end{subfigure}%
\hfill%
\begin{subfigure}{.45\linewidth}
  \centering
  \setlength\figureheight{\linewidth} 
  \setlength\figurewidth{\linewidth}
  \tikzsetnextfilename{samplepath_dscr_nFPC_bookvalue}
  % This file was created by matlab2tikz.
%
%The latest updates can be retrieved from
%  http://www.mathworks.com/matlabcentral/fileexchange/22022-matlab2tikz-matlab2tikz
%where you can also make suggestions and rate matlab2tikz.
%
\begin{tikzpicture}

\begin{axis}[%
width=4.376in,
height=3.694in,
at={(1.256in,0.622in)},
scale only axis,
separate axis lines,
every outer x axis line/.append style={black},
every x tick label/.append style={font=\color{black}},
xmin=10.975,
xmax=11,
xlabel={Time (h)},
every outer y axis line/.append style={black},
every y tick label/.append style={font=\color{black}},
ymin=4.25,
ymax=4.5,
ylabel={Book Value},
axis background/.style={fill=white}
]
\addplot [color=black,solid,line width=2.0pt,forget plot]
  table[row sep=crcr]{%
10.975	4.31785157278387\\
10.9752777777778	4.31785157278387\\
10.9755555555556	4.31785157278387\\
10.9758333333333	4.31785157278387\\
10.9761111111111	4.31785157278387\\
10.9763888888889	4.31785157278387\\
10.9766666666667	4.31785157278387\\
10.9769444444444	4.29997516867812\\
10.9772222222222	4.29997516867812\\
10.9775	4.28292312893899\\
10.9777777777778	4.28292312893899\\
10.9780555555556	4.28292312893899\\
10.9783333333333	4.28292312893899\\
10.9786111111111	4.28292312893899\\
10.9788888888889	4.28292312893899\\
10.9791666666667	4.28292312893899\\
10.9794444444444	4.28292312893899\\
10.9797222222222	4.28292312893899\\
10.98	4.34292312893902\\
10.9802777777778	4.33340832904011\\
10.9805555555556	4.33340832904011\\
10.9808333333333	4.33340832904011\\
10.9811111111111	4.3760591839511\\
10.9813888888889	4.3760591839511\\
10.9816666666667	4.3760591839511\\
10.9819444444444	4.3760591839511\\
10.9822222222222	4.3760591839511\\
10.9825	4.3760591839511\\
10.9827777777778	4.3760591839511\\
10.9830555555556	4.3760591839511\\
10.9833333333333	4.36609262500156\\
10.9836111111111	4.36609262500156\\
10.9838888888889	4.36609262500156\\
10.9841666666667	4.3582162208958\\
10.9844444444444	4.3582162208958\\
10.9847222222222	4.3582162208958\\
10.985	4.3582162208958\\
10.9852777777778	4.3582162208958\\
10.9855555555556	4.35604079488735\\
10.9858333333333	4.35604079488735\\
10.9861111111111	4.35604079488735\\
10.9863888888889	4.35604079488735\\
10.9866666666667	4.35604079488735\\
10.9869444444444	4.35604079488735\\
10.9872222222222	4.35604079488735\\
10.9875	4.35604079488735\\
10.9877777777778	4.35604079488735\\
10.9880555555556	4.35604079488735\\
10.9883333333333	4.35604079488735\\
10.9886111111111	4.35709178725898\\
10.9888888888889	4.35709178725898\\
10.9891666666667	4.36600984698629\\
10.9894444444444	4.36600984698629\\
10.9897222222222	4.36600984698629\\
10.99	4.36600984698629\\
10.9902777777778	4.36600984698629\\
10.9905555555556	4.36600984698629\\
10.9908333333333	4.36600984698629\\
10.9911111111111	4.38759025344325\\
10.9913888888889	4.39759025344324\\
10.9916666666667	4.39759025344324\\
10.9919444444444	4.39759025344324\\
10.9922222222222	4.39759025344324\\
10.9925	4.39759025344324\\
10.9927777777778	4.39759025344324\\
10.9930555555556	4.39759025344324\\
10.9933333333333	4.4125677800173\\
10.9936111111111	4.41595535917614\\
10.9938888888889	4.41595535917614\\
10.9941666666667	4.41595535917614\\
10.9944444444444	4.42941120110194\\
10.9947222222222	4.42941120110194\\
10.995	4.43675598564101\\
10.9952777777778	4.43675598564101\\
10.9955555555556	4.43675598564101\\
10.9958333333333	4.43675598564101\\
10.9961111111111	4.43675598564101\\
10.9963888888889	4.44223960795387\\
10.9966666666667	4.44223960795387\\
10.9969444444444	4.45239406791461\\
10.9972222222222	4.45239406791461\\
10.9975	4.45239406791461\\
10.9977777777778	4.47413731127108\\
10.9980555555556	4.46413731127109\\
10.9983333333333	4.46413731127109\\
10.9986111111111	4.46413731127109\\
10.9988888888889	4.44413731127105\\
10.9991666666667	4.44413731127105\\
10.9994444444444	4.44413731127105\\
10.9997222222222	4.44413731127105\\
11	4.44413731127105\\
};
\end{axis}
\end{tikzpicture}% 
\end{subfigure}\\
\leavevmode\smash{\makebox[0pt]{\hspace{-7em}% HORIZONTAL POSITION           
  \rotatebox[origin=l]{90}{\hspace{20em}% VERTICAL POSITION
    PnL}%
}}\hspace{0pt plus 1filll}\null

Time (h)

\vspace{1cm}%
  \caption{Sample paths of the optimal trading strategies, showing price, limit order posting depths, executed market orders, and filled limit orders.}
  \label{fig:samplepath_pnl}
\end{figure}

\begin{figure}
\centering
\begin{subfigure}{.45\linewidth}
  \centering
  \setlength\figureheight{\linewidth} 
  \setlength\figurewidth{\linewidth}
  \tikzsetnextfilename{samplepath_cts_inventory}
  % This file was created by matlab2tikz.
%
%The latest updates can be retrieved from
%  http://www.mathworks.com/matlabcentral/fileexchange/22022-matlab2tikz-matlab2tikz
%where you can also make suggestions and rate matlab2tikz.
%
\begin{tikzpicture}[trim axis left, trim axis right]

\begin{axis}[%
width=\figurewidth,
height=\figureheight,
at={(0\figurewidth,0\figureheight)},
scale only axis,
separate axis lines,
every outer x axis line/.append style={black},
every x tick label/.append style={font=\color{black}},
xmin=10.975,
xmax=11,
every outer y axis line/.append style={black},
every y tick label/.append style={font=\color{black}},
ymin=1,
ymax=7,
axis background/.style={fill=white}
]
\addplot [color=black,solid,line width=2.0pt,forget plot]
  table[row sep=crcr]{%
10.975	2\\
10.9752777777778	2\\
10.9755555555556	2\\
10.9758333333333	2\\
10.9761111111111	2\\
10.9763888888889	2\\
10.9766666666667	2\\
10.9769444444444	3\\
10.9772222222222	3\\
10.9775	4\\
10.9777777777778	4\\
10.9780555555556	5\\
10.9783333333333	5\\
10.9786111111111	5\\
10.9788888888889	5\\
10.9791666666667	4\\
10.9794444444444	4\\
10.9797222222222	4\\
10.98	4\\
10.9802777777778	4\\
10.9805555555556	4\\
10.9808333333333	4\\
10.9811111111111	3\\
10.9813888888889	3\\
10.9816666666667	3\\
10.9819444444444	3\\
10.9822222222222	3\\
10.9825	3\\
10.9827777777778	3\\
10.9830555555556	3\\
10.9833333333333	4\\
10.9836111111111	4\\
10.9838888888889	4\\
10.9841666666667	4\\
10.9844444444444	4\\
10.9847222222222	4\\
10.985	4\\
10.9852777777778	4\\
10.9855555555556	5\\
10.9858333333333	4\\
10.9861111111111	4\\
10.9863888888889	4\\
10.9866666666667	4\\
10.9869444444444	4\\
10.9872222222222	4\\
10.9875	4\\
10.9877777777778	4\\
10.9880555555556	5\\
10.9883333333333	5\\
10.9886111111111	6\\
10.9888888888889	6\\
10.9891666666667	6\\
10.9894444444444	6\\
10.9897222222222	6\\
10.99	6\\
10.9902777777778	6\\
10.9905555555556	6\\
10.9908333333333	6\\
10.9911111111111	3\\
10.9913888888889	2\\
10.9916666666667	2\\
10.9919444444444	2\\
10.9922222222222	2\\
10.9925	2\\
10.9927777777778	2\\
10.9930555555556	2\\
10.9933333333333	3\\
10.9936111111111	2\\
10.9938888888889	2\\
10.9941666666667	2\\
10.9944444444444	1\\
10.9947222222222	2\\
10.995	3\\
10.9952777777778	3\\
10.9955555555556	3\\
10.9958333333333	3\\
10.9961111111111	3\\
10.9963888888889	4\\
10.9966666666667	4\\
10.9969444444444	4\\
10.9972222222222	4\\
10.9975	4\\
10.9977777777778	3\\
10.9980555555556	2\\
10.9983333333333	2\\
10.9986111111111	2\\
10.9988888888889	3\\
10.9991666666667	3\\
10.9994444444444	3\\
10.9997222222222	3\\
11	3\\
};
\end{axis}
\end{tikzpicture}%

\end{subfigure}%
\hfill%
\begin{subfigure}{.45\linewidth}
  \centering
  \setlength\figureheight{\linewidth} 
  \setlength\figurewidth{\linewidth}
  \tikzsetnextfilename{samplepath_dscr_inventory}
  % This file was created by matlab2tikz.
%
%The latest updates can be retrieved from
%  http://www.mathworks.com/matlabcentral/fileexchange/22022-matlab2tikz-matlab2tikz
%where you can also make suggestions and rate matlab2tikz.
%
\begin{tikzpicture}

\begin{axis}[%
width=4.653in,
height=3.694in,
at={(0.978in,0.622in)},
scale only axis,
separate axis lines,
every outer x axis line/.append style={black},
every x tick label/.append style={font=\color{black}},
xmin=10.975,
xmax=11,
xlabel={Time (h)},
every outer y axis line/.append style={black},
every y tick label/.append style={font=\color{black}},
ymin=6,
ymax=18,
ylabel={Inventory},
axis background/.style={fill=white}
]
\addplot [color=black,solid,line width=2.0pt,forget plot]
  table[row sep=crcr]{%
10.975	7\\
10.9752777777778	7\\
10.9755555555556	7\\
10.9758333333333	7\\
10.9761111111111	7\\
10.9763888888889	7\\
10.9766666666667	7\\
10.9769444444444	8\\
10.9772222222222	8\\
10.9775	9\\
10.9777777777778	9\\
10.9780555555556	9\\
10.9783333333333	9\\
10.9786111111111	9\\
10.9788888888889	9\\
10.9791666666667	9\\
10.9794444444444	9\\
10.9797222222222	9\\
10.98	9\\
10.9802777777778	10\\
10.9805555555556	10\\
10.9808333333333	10\\
10.9811111111111	9\\
10.9813888888889	10\\
10.9816666666667	10\\
10.9819444444444	10\\
10.9822222222222	10\\
10.9825	10\\
10.9827777777778	10\\
10.9830555555556	10\\
10.9833333333333	11\\
10.9836111111111	11\\
10.9838888888889	11\\
10.9841666666667	12\\
10.9844444444444	12\\
10.9847222222222	12\\
10.985	12\\
10.9852777777778	12\\
10.9855555555556	13\\
10.9858333333333	13\\
10.9861111111111	13\\
10.9863888888889	14\\
10.9866666666667	14\\
10.9869444444444	14\\
10.9872222222222	14\\
10.9875	14\\
10.9877777777778	14\\
10.9880555555556	15\\
10.9883333333333	15\\
10.9886111111111	16\\
10.9888888888889	16\\
10.9891666666667	16\\
10.9894444444444	16\\
10.9897222222222	16\\
10.99	16\\
10.9902777777778	16\\
10.9905555555556	16\\
10.9908333333333	16\\
10.9911111111111	16\\
10.9913888888889	17\\
10.9916666666667	17\\
10.9919444444444	17\\
10.9922222222222	17\\
10.9925	17\\
10.9927777777778	17\\
10.9930555555556	17\\
10.9933333333333	18\\
10.9936111111111	17\\
10.9938888888889	17\\
10.9941666666667	17\\
10.9944444444444	17\\
10.9947222222222	17\\
10.995	18\\
10.9952777777778	17\\
10.9955555555556	17\\
10.9958333333333	17\\
10.9961111111111	17\\
10.9963888888889	17\\
10.9966666666667	17\\
10.9969444444444	17\\
10.9972222222222	17\\
10.9975	16\\
10.9977777777778	16\\
10.9980555555556	17\\
10.9983333333333	16\\
10.9986111111111	16\\
10.9988888888889	17\\
10.9991666666667	17\\
10.9994444444444	17\\
10.9997222222222	17\\
11	17\\
};
\end{axis}
\end{tikzpicture}% 
\end{subfigure}\\
\vspace{1cm}
\begin{subfigure}{.45\linewidth}
  \centering
  \setlength\figureheight{\linewidth} 
  \setlength\figurewidth{\linewidth}
  \tikzsetnextfilename{samplepath_cts_nFPC_inventory}
  % This file was created by matlab2tikz.
%
%The latest updates can be retrieved from
%  http://www.mathworks.com/matlabcentral/fileexchange/22022-matlab2tikz-matlab2tikz
%where you can also make suggestions and rate matlab2tikz.
%
\begin{tikzpicture}[trim axis left, trim axis right]

\begin{axis}[%
width=\figurewidth,
height=\figureheight,
at={(0\figurewidth,0\figureheight)},
scale only axis,
separate axis lines,
every outer x axis line/.append style={black},
every x tick label/.append style={font=\color{black}},
xtick = {10.97,10.98,10.99,11},
xmin=10.975,
xmax=11,
ylabel near ticks,
yticklabel pos=right,
every outer y axis line/.append style={black},
every y tick label/.append style={font=\color{black}},
ymin=-4,
ymax=6,
axis background/.style={fill=white}
]
\addplot [color=black,solid,line width=2.0pt,forget plot]
  table[row sep=crcr]{%
10.975	2\\
10.9752777777778	2\\
10.9755555555556	2\\
10.9758333333333	2\\
10.9761111111111	2\\
10.9763888888889	2\\
10.9766666666667	2\\
10.9769444444444	3\\
10.9772222222222	3\\
10.9775	4\\
10.9777777777778	4\\
10.9780555555556	5\\
10.9783333333333	5\\
10.9786111111111	5\\
10.9788888888889	5\\
10.9791666666667	2\\
10.9794444444444	2\\
10.9797222222222	2\\
10.98	2\\
10.9802777777778	3\\
10.9805555555556	3\\
10.9808333333333	2\\
10.9811111111111	2\\
10.9813888888889	3\\
10.9816666666667	3\\
10.9819444444444	3\\
10.9822222222222	3\\
10.9825	3\\
10.9827777777778	3\\
10.9830555555556	3\\
10.9833333333333	4\\
10.9836111111111	4\\
10.9838888888889	4\\
10.9841666666667	5\\
10.9844444444444	5\\
10.9847222222222	1\\
10.985	2\\
10.9852777777778	2\\
10.9855555555556	2\\
10.9858333333333	2\\
10.9861111111111	2\\
10.9863888888889	3\\
10.9866666666667	3\\
10.9869444444444	3\\
10.9872222222222	3\\
10.9875	3\\
10.9877777777778	3\\
10.9880555555556	4\\
10.9883333333333	4\\
10.9886111111111	5\\
10.9888888888889	5\\
10.9891666666667	6\\
10.9894444444444	6\\
10.9897222222222	6\\
10.99	6\\
10.9902777777778	6\\
10.9905555555556	6\\
10.9908333333333	2\\
10.9911111111111	2\\
10.9913888888889	3\\
10.9916666666667	3\\
10.9919444444444	3\\
10.9922222222222	3\\
10.9925	3\\
10.9927777777778	3\\
10.9930555555556	3\\
10.9933333333333	1\\
10.9936111111111	2\\
10.9938888888889	2\\
10.9941666666667	1\\
10.9944444444444	2\\
10.9947222222222	2\\
10.995	2\\
10.9952777777778	2\\
10.9955555555556	2\\
10.9958333333333	2\\
10.9961111111111	2\\
10.9963888888889	3\\
10.9966666666667	3\\
10.9969444444444	4\\
10.9972222222222	2\\
10.9975	1\\
10.9977777777778	-2\\
10.9980555555556	-3\\
10.9983333333333	-4\\
10.9986111111111	-4\\
10.9988888888889	-3\\
10.9991666666667	-3\\
10.9994444444444	-3\\
10.9997222222222	-3\\
11	-3\\
};
\end{axis}
\end{tikzpicture}%

\end{subfigure}%
\hfill%
\begin{subfigure}{.45\linewidth}
  \centering
  \setlength\figureheight{\linewidth} 
  \setlength\figurewidth{\linewidth}
  \tikzsetnextfilename{samplepath_dscr_nFPC_inventory}
  % This file was created by matlab2tikz.
%
%The latest updates can be retrieved from
%  http://www.mathworks.com/matlabcentral/fileexchange/22022-matlab2tikz-matlab2tikz
%where you can also make suggestions and rate matlab2tikz.
%
\begin{tikzpicture}[trim axis left, trim axis right]

\begin{axis}[%
width=\figurewidth,
height=\figureheight,
at={(0\figurewidth,0\figureheight)},
scale only axis,
separate axis lines,
every outer x axis line/.append style={black},
every x tick label/.append style={font=\color{black}},
xtick = {10.97,10.98,10.99,11},
xmin=10.975,
xmax=11,
ylabel near ticks,
yticklabel pos=right,
every outer y axis line/.append style={black},
every y tick label/.append style={font=\color{black}},
ymin=5,
ymax=11,
axis background/.style={fill=white}
]
\addplot [color=black,solid,line width=2.0pt,forget plot]
  table[row sep=crcr]{%
10.975	6\\
10.9752777777778	6\\
10.9755555555556	6\\
10.9758333333333	6\\
10.9761111111111	6\\
10.9763888888889	6\\
10.9766666666667	6\\
10.9769444444444	7\\
10.9772222222222	7\\
10.9775	8\\
10.9777777777778	8\\
10.9780555555556	8\\
10.9783333333333	8\\
10.9786111111111	8\\
10.9788888888889	8\\
10.9791666666667	8\\
10.9794444444444	8\\
10.9797222222222	8\\
10.98	5\\
10.9802777777778	6\\
10.9805555555556	6\\
10.9808333333333	6\\
10.9811111111111	5\\
10.9813888888889	5\\
10.9816666666667	5\\
10.9819444444444	5\\
10.9822222222222	5\\
10.9825	5\\
10.9827777777778	5\\
10.9830555555556	5\\
10.9833333333333	6\\
10.9836111111111	6\\
10.9838888888889	6\\
10.9841666666667	7\\
10.9844444444444	7\\
10.9847222222222	7\\
10.985	7\\
10.9852777777778	7\\
10.9855555555556	8\\
10.9858333333333	8\\
10.9861111111111	8\\
10.9863888888889	8\\
10.9866666666667	8\\
10.9869444444444	8\\
10.9872222222222	8\\
10.9875	8\\
10.9877777777778	8\\
10.9880555555556	9\\
10.9883333333333	9\\
10.9886111111111	10\\
10.9888888888889	10\\
10.9891666666667	11\\
10.9894444444444	11\\
10.9897222222222	11\\
10.99	11\\
10.9902777777778	11\\
10.9905555555556	11\\
10.9908333333333	11\\
10.9911111111111	8\\
10.9913888888889	7\\
10.9916666666667	7\\
10.9919444444444	7\\
10.9922222222222	7\\
10.9925	7\\
10.9927777777778	7\\
10.9930555555556	7\\
10.9933333333333	8\\
10.9936111111111	7\\
10.9938888888889	7\\
10.9941666666667	7\\
10.9944444444444	6\\
10.9947222222222	6\\
10.995	7\\
10.9952777777778	7\\
10.9955555555556	7\\
10.9958333333333	7\\
10.9961111111111	7\\
10.9963888888889	8\\
10.9966666666667	8\\
10.9969444444444	9\\
10.9972222222222	9\\
10.9975	9\\
10.9977777777778	8\\
10.9980555555556	9\\
10.9983333333333	9\\
10.9986111111111	9\\
10.9988888888889	10\\
10.9991666666667	10\\
10.9994444444444	10\\
10.9997222222222	10\\
11	10\\
};
\end{axis}
\end{tikzpicture}%
 
\end{subfigure}\\
\leavevmode\smash{\makebox[0pt]{\hspace{-7em}% HORIZONTAL POSITION           
  \rotatebox[origin=l]{90}{\hspace{19em}% VERTICAL POSITION
    Inventory}%
}}\hspace{0pt plus 1filll}\null

Time (h)

\vspace{1cm}%
  \caption{Sample paths of the optimal trading strategies, showing price, limit order posting depths, executed market orders, and filled limit orders.}
  \label{fig:samplepath_inv}
\end{figure}

\FloatBarrier
\section{In-Sample Backtesting}

\subsection{Same-Day Calibration}
We begin our in-sample backtesting same-day calibration: calibration was run for each ticker and each trading day of 2013, and backtesting was then done for each strategy using the same day's calibration. Each backtest would yield the end of day PnL, average inventory held during the day, and the number of executed market orders and filled limit orders. In \autoref{tbl:IS_sameday} we show performance values for several metrics of interest, while \autoref{fig:IS_sameday_comp} compares the day-over-day performance of the various strategies. 

Since we are calibrating and backtesting using the same underlying data, the calibration should be best attuned to the price dynamics for that praticular day, and hence we expect the performance using same-day calibration to exceed that of the weekly offset calibration and the annual calibration (detailed in the sections that follow). Looking at the \% Win column in \autoref{tbl:IS_sameday} we see that trading on \texttt{FARO} very rarely produces positive PnL. This is not surprising and was mentioned at the conclusion of the exploratory data analysis chapter: \texttt{FARO} is highly illiquid, with daily volume hovering around 200k, and its bid-ask spread averages approximately 20 cents. This makes it never profitable to execute market orders (due to crossing the spread), and because our optimal strategies still force us to post limit orders at depths between 0 and $1/\kappa = 0.01 = 1\text{cent}$, the most probable occurrence is that our limit orders are lifted adversely. \texttt{NTAP} seems to be a borderline case for liquidity, with average volumes around 4m, and here the strategies exhibit weak regularity of profits. The most liquid stocks, \texttt{ORCL} and \texttt{NTAP}, with average volumes around 15m and 30m respectively, post extremely promising results: the stochastic control strategies produce positive EOD PnL more than 90\% of the time. The discrete time controller outperforms its continuous time counterpart, and in particular we highlight that in the case of \texttt{INTC}, we attain a very good Sharpe ratio of 2.5.

\begin{figure}
\centering
\begin{subfigure}{.45\linewidth}
  \centering
  \setlength\figureheight{\linewidth} 
  \setlength\figurewidth{\linewidth}
  \tikzsetnextfilename{IS_sameday_FARO}
  % This file was created by matlab2tikz.
%
%The latest updates can be retrieved from
%  http://www.mathworks.com/matlabcentral/fileexchange/22022-matlab2tikz-matlab2tikz
%where you can also make suggestions and rate matlab2tikz.
%
\definecolor{mycolor1}{rgb}{0.25098,0.00000,0.38824}%
\definecolor{mycolor2}{rgb}{0.00000,0.46275,0.00000}%
\definecolor{mycolor3}{rgb}{0.00000,0.34902,0.34902}%
\definecolor{mycolor4}{rgb}{0.58039,0.26275,0.00000}%
%
\begin{tikzpicture}[trim axis left, trim axis right]

\begin{axis}[%
width=\figurewidth,
height=\figureheight,
at={(0\figurewidth,0\figureheight)},
scale only axis,
every outer x axis line/.append style={black},
every x tick label/.append style={font=\color{black}},
xmin=1,
xmax=252,
%xlabel={Time (h)},
every outer y axis line/.append style={black},
every y tick label/.append style={font=\color{black}},
ymin=-1.1,
ymax=1.1,
%ylabel={Normalized PnL},
title={FARO},
axis background/.style={fill=white},
axis x line*=bottom,
axis y line*=left,
yticklabel style={
        /pgf/number format/fixed,
        /pgf/number format/precision=3
},
scaled y ticks=false,
]
\addplot [color=mycolor1,solid,line width=1.5pt,forget plot]
  table[row sep=crcr]{%
1	-0.0171582644099111\\
2	0.00198553845329328\\
3	-0.0918381341353484\\
4	-0.0986434275244136\\
5	-0.10053545379216\\
6	-0.0765475019464891\\
7	-0.0619222370274102\\
8	-0.708629331666439\\
9	-0.0932142611000022\\
10	-0.0941648419858469\\
11	-0.0324252434646838\\
12	-0.0464054629407883\\
13	-0.0459735160777416\\
14	-0.0679795826088153\\
15	-0.0396135878814431\\
16	-0.0274015640561488\\
17	-0.110112508170455\\
18	-0.0207975175600572\\
19	0.00664822964171117\\
20	0.0305409285433767\\
21	0.0124858716902851\\
22	-0.0469516904294046\\
23	-0.035362251491487\\
24	0.00640515053588226\\
25	-0.0564561570035896\\
26	-0.0640620435107844\\
27	-0.0466163810411621\\
28	-0.0636460269021971\\
29	-0.0590713368383734\\
30	-0.010462513018786\\
31	-0.0251229925079348\\
32	-0.0557200425544463\\
33	-0.103341621432614\\
34	-0.0773956403892493\\
35	-0.0365092147868603\\
36	0.0100489072284386\\
37	-0.0890099138389983\\
38	0.0336076766630253\\
39	-0.103949730541853\\
40	-0.9116126651753\\
41	-0.532670609912645\\
42	-0.351849497654309\\
43	0.00645839220614087\\
44	-0.257313121546475\\
45	-0.0635082838868881\\
46	-0.0531319967624851\\
47	0.013153055583196\\
48	-0.111001208924059\\
49	-0.0605525101196774\\
50	-0.016332751886972\\
51	-0.0491571843311811\\
52	0.0223958997868287\\
53	-0.0772677491286859\\
54	-0.00639802756883161\\
55	-0.0646782974058186\\
56	-0.0527780538430975\\
57	-0.0250415322516236\\
58	0.0134292871432345\\
59	-0.015126888078404\\
60	-0.0835960651984423\\
61	-0.086372312057223\\
62	0.00471824400921696\\
63	-0.0839530150416437\\
64	-0.0504905221235172\\
65	-0.00514006616616673\\
66	-0.108323298106884\\
67	-0.0155128701854274\\
68	0.0124478974112718\\
69	0.0141068263254035\\
70	0.00375355550701115\\
71	-0.205160939015571\\
72	-0.0100435249402997\\
73	-0.0200745061305122\\
74	-0.115347831271105\\
75	-0.0237109391031726\\
76	-0.135479522427706\\
77	0.00136345324069474\\
78	-0.0252246426626556\\
79	0.0562445274565356\\
80	-0.060622245960738\\
81	0.0714540967726391\\
82	-0.0357462251970239\\
83	-0.14184563478958\\
84	-0.0288407456551952\\
85	-0.0161809618453014\\
86	-0.00270174938537434\\
87	-0.16710314270281\\
88	-0.344546544008865\\
89	-0.0822310439453183\\
90	0.00184115954419039\\
91	-0.0901620515392069\\
92	-0.02814712928248\\
93	-0.0642351253433988\\
94	-0.0543397269401892\\
95	-0.127588246935\\
96	-0.12962178832341\\
97	-0.0934037887833329\\
98	-0.247396463887623\\
99	-0.0676838074453423\\
100	-0.0339345272701973\\
101	-0.0256718812675493\\
102	0.0142415822097371\\
103	-0.0596969405389897\\
104	-0.127715473920977\\
105	-0.085726574917314\\
106	-0.0483658503841326\\
107	-0.00795737318767066\\
108	-0.193668543730335\\
109	-0.229030841457497\\
110	-0.0687365352216405\\
111	0.0304523542243592\\
112	-0.00877099369150027\\
113	-0.0773206687783234\\
114	-0.0253415848181428\\
115	0.0581194867286839\\
116	-0.0142253085159781\\
117	-0.189137185984256\\
118	-0.0917052152473808\\
119	-0.0779583437011491\\
120	-0.0995032279734559\\
121	0.0263299830333238\\
122	-0.066827365652015\\
123	0.0135215893583277\\
124	0.0432558243268529\\
125	0.05736020086017\\
126	-0.0391372604231726\\
127	nan\\
128	-0.0434157192895887\\
129	-0.0102160832983049\\
130	-0.0592146150877287\\
131	-0.00566252535470318\\
132	-0.0507383647620117\\
133	-0.0287858549492007\\
134	-0.047826363634428\\
135	-0.0805724317409333\\
136	-0.0210173086284954\\
137	-0.0319044331325967\\
138	-0.0304715635181761\\
139	0.0042972322618874\\
140	-0.0101214222501411\\
141	-0.0146749057198633\\
142	-0.0504943830128512\\
143	-0.0645585909497196\\
144	-0.0776329979733335\\
145	-0.0971405013331328\\
146	-0.217125860947905\\
147	-0.0506980931886626\\
148	0.000643074417361048\\
149	-0.0266954743946673\\
150	-0.00488551393582025\\
151	-0.020341720430395\\
152	-0.0600976650527583\\
153	-0.0162281149997243\\
154	-0.00616110101063212\\
155	-0.0339752895706021\\
156	-0.00388022564675624\\
157	-0.0243355058151993\\
158	-0.00227270934430147\\
159	-0.0125325964452041\\
160	-0.0339467567985541\\
161	-0.0430175913719791\\
162	-0.0504876244347099\\
163	-0.0229053235371861\\
164	-0.0178023478383072\\
165	-0.0282115837509881\\
166	-0.000331926056490367\\
167	-0.0021869080185209\\
168	0.0189294409664905\\
169	-0.0808254202616795\\
170	-0.0651728287440223\\
171	-0.0116019260016085\\
172	-0.0333900082557554\\
173	-0.0203953257171169\\
174	-0.0131235963338227\\
175	-0.00557681283044115\\
176	-0.0137813869077116\\
177	-0.0163558795336526\\
178	-0.0233167203181002\\
179	-0.00672705190154837\\
180	-0.22506014020993\\
181	-0.0422140102548055\\
182	-0.111830932994618\\
183	-0.0770119129734678\\
184	-0.033865470656763\\
185	-0.0309516758892761\\
186	0.0017030471483138\\
187	0.00122428991185019\\
188	-0.129030064251804\\
189	-0.146425169717127\\
190	-0.165472509132876\\
191	0.029299834942611\\
192	-0.00806656440187077\\
193	0.00629841165134728\\
194	-0.0305743127447412\\
195	-0.0257328488148134\\
196	0.00520985929153253\\
197	0.0151726383005547\\
198	-0.0117093589421445\\
199	-0.00184518493792325\\
200	-0.0199496013094712\\
201	0.0150463734629994\\
202	-0.0817453192951282\\
203	-0.0455764355931173\\
204	0.0366252996905885\\
205	0.0243659707317058\\
206	0.0404684951656568\\
207	-0.0183992890629788\\
208	-0.0162286922500947\\
209	0.0316530234813035\\
210	-0.0915995685229851\\
211	-0.571450089369681\\
212	-0.522800509713424\\
213	-0.110285731204592\\
214	-0.0636330295621417\\
215	-0.113831964775362\\
216	-0.196734629325433\\
217	-0.0992367600758085\\
218	0.0409386988386661\\
219	-0.160882625082066\\
220	-0.0142251178772514\\
221	0.0141150152601961\\
222	0.0119435929361839\\
223	-0.0521911400526267\\
224	-0.0986690004686093\\
225	-0.00349592015040395\\
226	0.0159080741143208\\
227	0.0323391649223707\\
228	-0.0102717989347697\\
229	-0.0602873068144885\\
230	-0.0431071489260421\\
231	nan\\
232	-0.0530180516963446\\
233	-0.0207328360728421\\
234	-0.0565243675288807\\
235	-0.0192230995284595\\
236	-0.0166270880962157\\
237	-0.0979943053035664\\
238	-0.0169414260554841\\
239	-0.0217924199887439\\
240	-0.000568558703685773\\
241	-0.019621456236181\\
242	-0.129287753892855\\
243	-0.0723576800320373\\
244	-0.00299039452103355\\
245	0.0216702226468757\\
246	-0.0274519273566285\\
247	-0.0862383725984549\\
248	nan\\
249	-0.114382117250106\\
250	-0.189145284673407\\
251	-0.0214157588924525\\
252	-0.00544348336715652\\
};
\addplot [color=mycolor2,solid,line width=1.5pt,forget plot]
  table[row sep=crcr]{%
1	-0.0856203607850693\\
2	-0.0707747680009662\\
3	-0.104141767525771\\
4	-0.199497981634198\\
5	-0.202681998710269\\
6	-0.124272945470926\\
7	-0.0993665627651743\\
8	-0.183401978355264\\
9	-0.0100345814990514\\
10	-0.0351430873842089\\
11	-0.034657838205582\\
12	-0.0239896214341985\\
13	-0.0215512017589147\\
14	-0.0367101149924947\\
15	-0.0224441555157761\\
16	-0.0675490454834694\\
17	-0.108555182425727\\
18	-0.0722263810848897\\
19	-0.00771450276650055\\
20	-0.016270732313766\\
21	-0.0465005274262002\\
22	-0.0650023862255114\\
23	-0.0239450712140652\\
24	-0.0682625897914213\\
25	-0.0289944212941058\\
26	-0.127588160982961\\
27	-0.024696177113977\\
28	-0.0590517227982054\\
29	-0.0550493487370463\\
30	-0.0773557542425015\\
31	-0.0574895482736575\\
32	-0.0376033956692175\\
33	-0.236411619367621\\
34	-0.0415619297297856\\
35	-0.0774282743612988\\
36	-0.00417096847419357\\
37	-0.0584864611637903\\
38	-0.059937989907866\\
39	-0.258529585311911\\
40	-0.913876532859668\\
41	-0.241128583708994\\
42	-0.308320541558404\\
43	0.0221913295446278\\
44	-0.166382789956429\\
45	-0.0746443572646033\\
46	-0.083504386724256\\
47	-0.0280890090267684\\
48	-0.0812589613421778\\
49	-0.116398106578273\\
50	-0.0481030665944524\\
51	-0.0391416075350832\\
52	-0.0545576809466638\\
53	-0.156039215761556\\
54	-0.0678036386983275\\
55	-0.118155941332162\\
56	-0.0735552449067791\\
57	-0.0371177887306459\\
58	-0.0176973207914295\\
59	-0.0263150638058797\\
60	-0.0714150906462965\\
61	-0.0516524626348779\\
62	-0.104627480490001\\
63	0.046317926814774\\
64	-0.0625564482050183\\
65	-0.0418152620529059\\
66	-0.0382930905093672\\
67	-0.0550099913859691\\
68	-0.0618335286862652\\
69	-0.0752666300786891\\
70	-0.0281333125739904\\
71	0.00807330571377673\\
72	-0.0229210637247318\\
73	-0.0669069953283607\\
74	-0.0737350731420096\\
75	-0.0777053928440509\\
76	-0.136486911745524\\
77	-0.0366913682070886\\
78	-0.0698376112609589\\
79	-0.00337707611656408\\
80	-0.114332186553646\\
81	-0.0115305221444353\\
82	-0.115209215970164\\
83	0.439650394556965\\
84	0.00612820399059535\\
85	-0.0615294028012233\\
86	-0.0338466468133894\\
87	-0.136516720514761\\
88	-0.145849664598147\\
89	-0.0566279996114591\\
90	-0.011033454052542\\
91	-0.0779248342818201\\
92	-0.0605750749077891\\
93	-0.00706762663330628\\
94	-0.0475068278957096\\
95	-0.042571755182104\\
96	-0.0620456353807177\\
97	-0.168971234019066\\
98	-0.134466411627154\\
99	-0.107370959617784\\
100	-0.0355136658602324\\
101	-0.0826875458012084\\
102	-0.012497329406777\\
103	-0.0885485171161874\\
104	-0.0784231848579368\\
105	-0.148875945742619\\
106	-0.0540032742362043\\
107	-0.0192702772363494\\
108	-0.100003725668135\\
109	-0.0880473248130928\\
110	-0.0629689384357705\\
111	0.0271013275987113\\
112	0.0267954796082197\\
113	-0.00706077752500883\\
114	0.0128907769675273\\
115	0.079774740717241\\
116	0.00709653171822195\\
117	-0.0348048497337723\\
118	0.0137361248936539\\
119	-0.11069698020023\\
120	-0.0397754309647627\\
121	-0.0439333532540539\\
122	-0.0433208106039542\\
123	0.0108010198370059\\
124	0.0412562754054816\\
125	0.0495524503980965\\
126	-0.0230646480102991\\
127	nan\\
128	-0.0825196433645509\\
129	-0.0426437891756898\\
130	-0.0789593568157475\\
131	0.00171891840661759\\
132	-0.0439384492123908\\
133	-8.64259799509436e-05\\
134	-0.0226745101498006\\
135	-0.0893874893231492\\
136	-0.0649509389243762\\
137	-0.0610984608998195\\
138	0.0215272301610553\\
139	-0.0692698722440269\\
140	-0.0195680422436206\\
141	-0.110076383365166\\
142	-0.0456384545240186\\
143	-0.0149081746520216\\
144	-0.0960757775549577\\
145	-0.00629160189492709\\
146	-0.374685438067206\\
147	-0.0683852020732083\\
148	-0.016805086047098\\
149	-0.0293643955733932\\
150	-0.0109455526971296\\
151	-0.0332981944801438\\
152	-0.0379064416954155\\
153	-0.0279380860146927\\
154	-0.0187273673479832\\
155	-0.0162547540706838\\
156	-0.0878741281253934\\
157	-0.0439976559598474\\
158	-0.0429758734059628\\
159	-0.0221036345308733\\
160	-0.0548862232653352\\
161	-0.0484284016576985\\
162	-0.0522623362336953\\
163	-0.0490208043745991\\
164	-0.0123963327824419\\
165	-0.076196737440846\\
166	-0.0415989240108919\\
167	-0.0225447362634346\\
168	-0.0509916412046076\\
169	-0.117639659966885\\
170	-0.0363932875760041\\
171	-0.026816628718428\\
172	-0.0532919333856116\\
173	-0.0385311885296545\\
174	-0.0239080218807847\\
175	-0.0318015339407009\\
176	-0.0574657445212645\\
177	-0.00979371527250017\\
178	-0.0295716879791661\\
179	-0.0306115437632438\\
180	-0.0965882847989252\\
181	-0.0457428528654177\\
182	-0.00841085885731082\\
183	-0.0585946596159777\\
184	-0.00873880794793503\\
185	0.0272599193223205\\
186	-0.0432600087181072\\
187	-0.0264348076179931\\
188	-0.0652251652492641\\
189	-0.113779963847615\\
190	-0.156487555067705\\
191	-0.0516262787892315\\
192	0.00669748999338609\\
193	-0.0506072739666934\\
194	-0.0952110369833532\\
195	-0.0401777848420842\\
196	-0.0807671958922854\\
197	-0.0381874581366996\\
198	-0.0506746193639166\\
199	-0.0418030336911066\\
200	-0.0798617004081724\\
201	-0.0475251100419537\\
202	-0.0717503092598897\\
203	-0.0878079773248392\\
204	-0.0996163870508437\\
205	-0.0777651969959645\\
206	-0.0190252615980488\\
207	-0.0447288543526097\\
208	-0.0705843737607429\\
209	-0.0248370801208491\\
210	-0.127457303804752\\
211	-0.604418315878085\\
212	-0.291205074120635\\
213	0.0584940114227124\\
214	-0.0681439495242273\\
215	-0.159294342792978\\
216	-0.189538872966937\\
217	-0.0913258339765485\\
218	-0.0990296458364101\\
219	-0.116301064003621\\
220	-0.0966356484637969\\
221	-0.0351315812712979\\
222	-0.0261504556467591\\
223	-0.062852785198048\\
224	-0.170329954761776\\
225	-0.0582206542418083\\
226	0.0350050108017376\\
227	-0.0179826980976183\\
228	-0.0921585953559416\\
229	-0.0573584738895998\\
230	-0.0826404156044176\\
231	nan\\
232	-0.0205025824630308\\
233	-0.010472547699454\\
234	-0.101950489112216\\
235	-0.0766498745566366\\
236	-0.0593906968680247\\
237	-0.0793761003808785\\
238	-0.0560834056017775\\
239	-0.0654466634150648\\
240	-0.0687733104517934\\
241	-0.0449674093725098\\
242	-0.136999348902649\\
243	-0.0990008446675353\\
244	-0.0473733622107678\\
245	-0.0276760569690386\\
246	0.09745752787986\\
247	-0.147801343392415\\
248	nan\\
249	-0.0772661059886298\\
250	-0.0338821430775791\\
251	-0.0930978658052126\\
252	-0.028861778877178\\
};
\addplot [color=mycolor3,solid,line width=1.5pt,forget plot]
  table[row sep=crcr]{%
1	-0.0325100186027445\\
2	0.00499071688286546\\
3	-0.0852566445186826\\
4	-0.128535812985994\\
5	-0.126430414437586\\
6	-0.0756068325374878\\
7	-0.0538770823587646\\
8	-0.531721353957638\\
9	-0.0708225479627843\\
10	-0.119629808342178\\
11	-0.0561264147152898\\
12	-0.0533832179700141\\
13	-0.0498798625465059\\
14	-0.0719390766891468\\
15	-0.0429492169957707\\
16	-0.0382271953511934\\
17	-0.124846453570998\\
18	-0.0487833452673584\\
19	-0.0126515153877701\\
20	-0.00596509339016379\\
21	0.00925214405352224\\
22	-0.0543759280174415\\
23	-0.0246462118601773\\
24	-0.00474002932294398\\
25	-0.0441995006162151\\
26	-0.0547268005867932\\
27	-0.0420869121125324\\
28	-0.0595788926705136\\
29	-0.0632721249170167\\
30	-0.0269424411484572\\
31	-0.0352636697056688\\
32	-0.0591175241909321\\
33	-0.0774958445608846\\
34	-0.0795424262984346\\
35	-0.0480154443853273\\
36	0.00729087441898886\\
37	-0.105513532189232\\
38	0.0104821499505265\\
39	-0.117417090163649\\
40	-1.16087436964866\\
41	-0.450240024031655\\
42	-0.362511252113198\\
43	0.00253674922364051\\
44	-0.232219955047903\\
45	-0.0618511195038171\\
46	-0.0657308906530964\\
47	-0.00154341978978814\\
48	-0.0914535442590734\\
49	-0.0708698213294756\\
50	-0.019896182315092\\
51	-0.0492654348784088\\
52	0.00689967392370656\\
53	-0.0881481112551058\\
54	-0.032251385455512\\
55	-0.0645320177175673\\
56	-0.0556703886598468\\
57	-0.0270177890059823\\
58	0.00681735123120174\\
59	-0.00711459545958231\\
60	-0.0673024944336585\\
61	-0.0983070088896275\\
62	0.00824195941551491\\
63	-0.0820803854091777\\
64	-0.0605120601427456\\
65	-0.0208801293947695\\
66	-0.100844400164873\\
67	-0.0199379004327238\\
68	-0.0197793621417604\\
69	0.0174249481803315\\
70	0.00511681446935318\\
71	-0.207955344653329\\
72	-0.0214743855952195\\
73	-0.0449543766786345\\
74	-0.11663155533071\\
75	-0.0452597783201444\\
76	-0.127120023372661\\
77	0.0134075363327031\\
78	-0.0373553717434991\\
79	0.0322795907227259\\
80	-0.0659634130057415\\
81	0.063873091925299\\
82	-0.0553045738236964\\
83	-0.445143988821365\\
84	-0.0351244628148961\\
85	-0.0337146072651949\\
86	-0.0168501539127626\\
87	-0.168770320946046\\
88	-0.326141157109719\\
89	-0.0711067284362476\\
90	0.025136257067558\\
91	-0.0839147215733757\\
92	-0.0304584294714067\\
93	-0.065847661597966\\
94	-0.0553607015174642\\
95	-0.0964286893116665\\
96	-0.148385653282368\\
97	-0.11250924989955\\
98	-0.243201151236821\\
99	-0.0765028605896402\\
100	-0.0472184640709103\\
101	-0.0285262451235031\\
102	0.0151693078845256\\
103	-0.0600461727505705\\
104	-0.143224829065629\\
105	-0.10240229704266\\
106	-0.0548469475561593\\
107	-0.0173263081257618\\
108	-0.194189114583288\\
109	-0.146668309612629\\
110	-0.0783409137968528\\
111	0.0103246460058126\\
112	-0.0162356167413733\\
113	-0.0742648241775433\\
114	-0.0228181338709292\\
115	0.0260919976537978\\
116	-0.00824281311377484\\
117	-0.161457767517951\\
118	-0.0607936211132628\\
119	-0.0607122690762398\\
120	-0.0809099450505293\\
121	0.0151417891219907\\
122	-0.0582003299236888\\
123	0.00838915937655658\\
124	0.0328128854927279\\
125	0.0774547674302478\\
126	-0.0385892495767562\\
127	nan\\
128	-0.0504536046952919\\
129	-0.0183504982275549\\
130	-0.0545839997487897\\
131	-0.0033753513913728\\
132	-0.0338200504699539\\
133	-0.0304633685792612\\
134	-0.0332670993045126\\
135	-0.0534531211665265\\
136	-0.0444908058306384\\
137	-0.0324152882979398\\
138	-0.0367574902755926\\
139	0.0100006452739671\\
140	-0.0209455992891978\\
141	-0.0449060916024155\\
142	-0.0464680620917089\\
143	-0.0647606324716471\\
144	-0.0698059030936675\\
145	-0.0898400797659559\\
146	-0.193773187382333\\
147	-0.0186641424485681\\
148	0.00149580181566159\\
149	-0.0209297538160674\\
150	-0.00788979612878293\\
151	-0.0235582163011212\\
152	-0.0631215744085287\\
153	-0.0323508828645202\\
154	-0.0091303912188726\\
155	-0.033484960670348\\
156	-0.030396352128917\\
157	-0.0233276584988491\\
158	-0.00118789986850866\\
159	0.00345145667368415\\
160	-0.0345997498649027\\
161	-0.050138520735859\\
162	-0.0305551660095176\\
163	-0.0377359594650799\\
164	-0.0216983067651871\\
165	-0.0147785040235029\\
166	-0.0100912132086347\\
167	-0.000701390630210771\\
168	0.0098851367776351\\
169	-0.0924709707647595\\
170	-0.0563595480316546\\
171	-0.0124263278692122\\
172	-0.0221704130169706\\
173	-0.0221389455308899\\
174	-0.0171248785605318\\
175	-0.0107670654365861\\
176	-0.0125576959180354\\
177	-0.0134986533562417\\
178	-0.0167150236524654\\
179	-0.019728820316897\\
180	-0.153845242829659\\
181	-0.0396045744308492\\
182	-0.121507464052203\\
183	-0.107953671166394\\
184	-0.0237759297794944\\
185	-0.0326840755200716\\
186	-0.024306742888668\\
187	0.00292971611045153\\
188	-0.119092188456988\\
189	-0.15063649222065\\
190	-0.161002339383097\\
191	0.0338426364268917\\
192	-0.0054904172481059\\
193	0.0028448384348453\\
194	-0.0327301587023332\\
195	-0.0301497301114387\\
196	-0.0314348449158078\\
197	-0.00406469927148007\\
198	-0.0218253714970635\\
199	-0.0126471305489269\\
200	-0.0291676838754158\\
201	-0.00197921820880713\\
202	-0.0943460647947407\\
203	-0.0516507457586046\\
204	-0.00527532551066649\\
205	0.0154303085609422\\
206	0.0312347081868182\\
207	-0.0223158646075264\\
208	-0.0257662721930381\\
209	0.0252244562619509\\
210	-0.0743939428939186\\
211	-0.517811023063384\\
212	-0.480860080640541\\
213	-0.14066362713826\\
214	-0.0513391312381968\\
215	-0.110695575262891\\
216	-0.254960425034872\\
217	-0.100422858811898\\
218	0.0465248030180825\\
219	-0.168680130529138\\
220	-0.0225095117080068\\
221	-0.00106343297125548\\
222	0.0050177638227106\\
223	-0.0493416422209153\\
224	-0.0946918654901592\\
225	-0.0125458380134338\\
226	0.0469300862314703\\
227	0.0285000683526303\\
228	-0.00433467588116343\\
229	-0.0477687697144216\\
230	-0.0486051770547526\\
231	nan\\
232	-0.0611896769542466\\
233	-0.0117401651206573\\
234	-0.0466409178607115\\
235	-0.0340311045304937\\
236	-0.014574508177385\\
237	-0.081373367229482\\
238	-0.0179718313667354\\
239	-0.0159782856994813\\
240	-0.00317863495006516\\
241	-0.0356328237653631\\
242	-0.147350844523739\\
243	-0.0648150138084743\\
244	-0.0270338813110732\\
245	0.0129733570280646\\
246	-0.0323933530834971\\
247	-0.0894666557187541\\
248	nan\\
249	-0.111514129059471\\
250	-0.180058701160543\\
251	-0.0566108077699478\\
252	-0.0119279514298396\\
};
\addplot [color=mycolor4,solid,line width=1.5pt,forget plot]
  table[row sep=crcr]{%
1	-0.0772538560126161\\
2	-0.0487663806382359\\
3	-0.0995878386535013\\
4	-0.157039252752378\\
5	-0.155954965253489\\
6	-0.1017185443399\\
7	-0.125639905725187\\
8	-0.181654451310473\\
9	-0.0440025253326762\\
10	-0.00994264585330307\\
11	-0.0123795390126177\\
12	0.0788118452540155\\
13	-0.0435783745467516\\
14	-0.0642452120456839\\
15	-0.0219952026611009\\
16	-0.0474258407491828\\
17	-0.0905389213253349\\
18	-0.0533781055306928\\
19	0.000359855304758824\\
20	-0.0344969867867687\\
21	-0.0401698534279172\\
22	-0.0487567635568958\\
23	-0.0238892468923298\\
24	-0.0666232030207952\\
25	-0.0407890322268669\\
26	-0.104970479938002\\
27	-0.0236147782336466\\
28	-0.0607980424170458\\
29	-0.0555068034283317\\
30	-0.0774287796883188\\
31	-0.049138143932718\\
32	-0.0522260033186819\\
33	-0.189282852132894\\
34	-0.0328192420575641\\
35	-0.103372512784294\\
36	0.00498475053095889\\
37	0.329562491250644\\
38	-0.0323123540531002\\
39	-0.201009233049357\\
40	-0.716131976300465\\
41	-0.278139840653107\\
42	-0.320253212724183\\
43	0.0203601980201811\\
44	-0.172658880891437\\
45	-0.0447067801485141\\
46	-0.0593586765558188\\
47	-0.0149041438446444\\
48	-0.0731028566197628\\
49	-0.101943854933178\\
50	-0.0456367498315522\\
51	-0.0488950430817485\\
52	-0.0550089096320184\\
53	-0.0981461396597604\\
54	-0.0648742991789584\\
55	-0.117944324145923\\
56	-0.150944158551876\\
57	-0.034077891769953\\
58	-0.0550494949217167\\
59	-0.0322821339298258\\
60	0.00401846431606009\\
61	-0.0481181439006338\\
62	0.0133009045122569\\
63	0.0350441784052174\\
64	-0.0593319229365297\\
65	-0.0529940020969334\\
66	-0.0495275013937246\\
67	-0.0423220305753306\\
68	-0.0370759107265927\\
69	-0.0505337942644965\\
70	-0.0314051029149161\\
71	-0.00163753441109454\\
72	-0.0309281751018605\\
73	-0.0474279186149383\\
74	-0.0695188763270507\\
75	-0.0696380205340643\\
76	-0.130975676538113\\
77	-0.0785090684947946\\
78	-0.0804868878131417\\
79	0.00707339171628024\\
80	-0.0960045026717784\\
81	-0.00540077502783183\\
82	-0.0510521702582195\\
83	0.538837573666183\\
84	0.014358075808554\\
85	-0.0525077527322873\\
86	-0.0422992308608391\\
87	-0.12844297456689\\
88	-0.152795929405942\\
89	-0.0661209323204958\\
90	-0.00636353180036207\\
91	-0.0784302682475388\\
92	-0.0724413586266111\\
93	0.00679372805307658\\
94	-0.0359795590314484\\
95	-0.0289246614781889\\
96	-0.0462546345950233\\
97	-0.104294408805889\\
98	-0.195552680959809\\
99	-0.128297371866996\\
100	-0.0586787119142817\\
101	-0.0431536576590735\\
102	-0.00348452136081325\\
103	-0.0760628260968697\\
104	-0.0687245700787149\\
105	-0.131630097177762\\
106	-0.066856432252672\\
107	-0.0764328034015351\\
108	-0.0987506455691217\\
109	-0.0819510835955211\\
110	-0.0639284689433902\\
111	0.0345001595390402\\
112	0.0263698310461612\\
113	0.0243226854861509\\
114	0.00564551992382162\\
115	0.0485078109559253\\
116	-0.000119918449584913\\
117	-0.00890716060067398\\
118	-0.0171328596110289\\
119	-0.137819303108701\\
120	-0.0535083583698354\\
121	-0.0158429075775827\\
122	-0.0543100878572725\\
123	0.0240170738000622\\
124	0.00245797975176231\\
125	0.0424685835361417\\
126	-0.0270946795989594\\
127	nan\\
128	-0.0884490246860671\\
129	-0.01979352205602\\
130	-0.0602209425028455\\
131	-0.047390182858586\\
132	-0.00763040618558268\\
133	-0.0183942120197465\\
134	-0.018524605807832\\
135	-0.101818549201356\\
136	-0.0441827685465075\\
137	-0.0603019168075178\\
138	0.0106773139402215\\
139	-0.060134771308082\\
140	-0.0180420801104241\\
141	-0.0344071960710348\\
142	-0.0364838995842449\\
143	-0.0248453006625892\\
144	-0.0469324765491856\\
145	-0.0126376129666439\\
146	-0.325455703332194\\
147	-0.0974231984998782\\
148	-0.0209135001537699\\
149	-0.0228853857280722\\
150	-0.0111692977900301\\
151	-0.0272122272873205\\
152	-0.0355248017379502\\
153	-0.00719711559512561\\
154	-0.0147237214070352\\
155	-0.0194375470338417\\
156	-0.0846836984499125\\
157	-0.0547371498014214\\
158	-0.0451454071552599\\
159	-0.0321358202410631\\
160	-0.035721366820431\\
161	-0.0366210025217901\\
162	-0.043417480992881\\
163	-0.0715093319742274\\
164	-0.0105724490039661\\
165	-0.0640378778906103\\
166	-0.0070971258966624\\
167	-0.0102372292760136\\
168	-0.0668849024409976\\
169	-0.112004510426872\\
170	-0.0550437927387341\\
171	-0.0113708820647503\\
172	-0.044096804374606\\
173	-0.0414595416991836\\
174	0.00436535979800067\\
175	-0.0259001975317621\\
176	-0.038506748418225\\
177	-0.0197660370139826\\
178	-0.0409584252365702\\
179	-0.022123631327268\\
180	-0.0590341745343367\\
181	-0.0407592269820797\\
182	-0.0178894107077691\\
183	-0.0570585816082112\\
184	-0.0157515388879908\\
185	-0.0507842872328195\\
186	-0.0508017650246672\\
187	-0.0274600610542058\\
188	-0.11448340349484\\
189	-0.126498340267413\\
190	-0.166891557063536\\
191	-0.0624470421368618\\
192	-0.00312345797821631\\
193	-0.0365341283977062\\
194	-0.0864226769558292\\
195	-0.033297829277073\\
196	-0.0931575860883022\\
197	-0.0180807039183585\\
198	-0.0384201418606434\\
199	-0.0503471590960953\\
200	-0.0688841809215643\\
201	-0.0421318421198584\\
202	-0.0581141246155688\\
203	-0.0938432452427879\\
204	-0.0628536425622035\\
205	-0.0634822056026895\\
206	-0.0185695161655531\\
207	-0.0437535387441696\\
208	-0.0529967823525217\\
209	-0.0332262021893026\\
210	-0.0843441610302172\\
211	-0.448453757861155\\
212	-0.480672263477085\\
213	-0.10154146855894\\
214	-0.0901049318841872\\
215	-0.161506065537761\\
216	-0.196046565322263\\
217	-0.12929137635563\\
218	-0.0935565777276325\\
219	-0.123092636211245\\
220	-0.109638109821853\\
221	-0.0383269909900374\\
222	-0.00518380287887244\\
223	-0.0670863506228746\\
224	-0.150065305203484\\
225	-0.0687504059741254\\
226	-0.0359091376169781\\
227	-0.0396195571702265\\
228	-0.0835084658962133\\
229	-0.0609935393359201\\
230	-0.104173308186193\\
231	nan\\
232	-0.0285122764177975\\
233	-0.041851454588042\\
234	-0.113780873715656\\
235	-0.0997152866933102\\
236	-0.0772546548995375\\
237	-0.101285386367683\\
238	-0.0520203611978263\\
239	-0.0719311313284583\\
240	-0.0628166253878295\\
241	-0.0371527205573437\\
242	-0.0419428482393087\\
243	-0.0846577230990247\\
244	-0.0301119457497231\\
245	-0.0417833931591006\\
246	0.103713190789445\\
247	-0.166255659948543\\
248	nan\\
249	-0.0710819768332999\\
250	-0.0267233799613964\\
251	-0.0886930899149551\\
252	-0.055034461224752\\
};
\end{axis}
\end{tikzpicture}%

\end{subfigure}%
\hfill%
\begin{subfigure}{.45\linewidth}
  \centering
  \setlength\figureheight{\linewidth} 
  \setlength\figurewidth{\linewidth}
  \tikzsetnextfilename{IS_sameday_NTAP}
  % This file was created by matlab2tikz.
%
%The latest updates can be retrieved from
%  http://www.mathworks.com/matlabcentral/fileexchange/22022-matlab2tikz-matlab2tikz
%where you can also make suggestions and rate matlab2tikz.
%
%
\begin{tikzpicture}[trim axis left, trim axis right]

\begin{axis}[%
width=\figurewidth,
height=\figureheight,
at={(0\figurewidth,0\figureheight)},
scale only axis,
every outer x axis line/.append style={black},
every x tick label/.append style={font=\color{black}},
xmin=1,
xmax=252,
%xlabel={Time (h)},
every outer y axis line/.append style={black},
every y tick label/.append style={font=\color{black}},
ymin=-1.1,
ymax=1.1,
%ylabel={Normalized PnL},
title={NTAP},
axis background/.style={fill=white},
axis x line*=bottom,
axis y line*=left,
yticklabel style={
        /pgf/number format/fixed,
        /pgf/number format/precision=3
},
scaled y ticks=false,
]
\addplot [color=cts_plot_color,solid,line width=1.5pt,forget plot]
  table[row sep=crcr]{%
1	-0.146334170010575\\
2	-0.144423708590969\\
3	-0.113628960428223\\
4	-0.111609761510288\\
5	-0.110024540277312\\
6	0.140994376775195\\
7	-0.0332346473748883\\
8	0.0296803479021415\\
9	-0.118898954471748\\
10	-0.190455353230526\\
11	0.0454943820496442\\
12	0.103455191170325\\
13	-0.0311975823263387\\
14	-0.0822526439771115\\
15	-0.205542229063115\\
16	0.0420613010293632\\
17	0.0581810057255279\\
18	0.0934843564082873\\
19	-0.10407702781347\\
20	0.0495365578162272\\
21	-0.0651968426695856\\
22	-0.0362594470922687\\
23	0.0295027433549918\\
24	0.0402863872513736\\
25	-0.0304396733253954\\
26	0.0417105612437457\\
27	-0.0757320937702446\\
28	-0.133328387690426\\
29	-0.0105532138988593\\
30	0.00734746971436523\\
31	-0.271014263488796\\
32	-0.0225360665226154\\
33	0.0270826314618481\\
34	-0.11036823607362\\
35	-0.0669363295449461\\
36	-0.0684361096624793\\
37	-0.0726888900256057\\
38	-0.152911284350321\\
39	0.0115639181478459\\
40	0.0391571897768147\\
41	-0.0276917904604929\\
42	-0.0683789280157995\\
43	-0.00530031523669075\\
44	0.0219871415084311\\
45	-0.0129985277813901\\
46	-0.0401921637228632\\
47	0.0643914125935195\\
48	0.043394485507103\\
49	-0.0711739964749996\\
50	0.110457599167972\\
51	0.0100403529626161\\
52	-0.000832245014777877\\
53	0.0424172810391523\\
54	0.0434608810096843\\
55	-0.0443678189886408\\
56	0.111106607577829\\
57	0.119456010136082\\
58	-0.0320339589783253\\
59	0.0918476493070952\\
60	0.0686293798300882\\
61	0.0759622015320213\\
62	-0.0542835635266948\\
63	0.0686233034746546\\
64	-0.0118677391997202\\
65	-0.0113275428555239\\
66	0.0444465804486523\\
67	0.14811082690168\\
68	-0.0192722667572909\\
69	0.124365938609198\\
70	-0.441102148058369\\
71	-0.117697471659442\\
72	-0.0294122088199094\\
73	-0.233151140569622\\
74	-0.0593052258912813\\
75	0.020792681529729\\
76	0.0711701663125001\\
77	0.0283796709671788\\
78	0.160795073761954\\
79	-0.124474251962115\\
80	-0.17755262720032\\
81	0.0611409103664031\\
82	0.0809662461599879\\
83	-0.21441705287873\\
84	-0.00730563552259109\\
85	0.0570135157456549\\
86	0.0723259059553574\\
87	0.0258750433987791\\
88	-0.0925193045532482\\
89	0.015324672857035\\
90	0.0799573384106038\\
91	-0.231020564515516\\
92	0.0470950719066619\\
93	0.0388064634304863\\
94	0.128481093339731\\
95	0.183622826299882\\
96	0.0255416421079278\\
97	0.0711187932026343\\
98	0.208882847356938\\
99	0.0696713353354581\\
100	0.0718035162559606\\
101	0.0102230241949068\\
102	0.0365797660163733\\
103	-0.103865110790456\\
104	0.030650033039141\\
105	0.0410558602789736\\
106	-0.309681077456261\\
107	-0.0119963771443947\\
108	0.0738055865058521\\
109	-0.0638892439528518\\
110	0.0355662508557962\\
111	0.00149083921598245\\
112	0.034974244210311\\
113	-0.0145454016341785\\
114	-0.0337232241322727\\
115	0.00814165644218772\\
116	0.0973219229186651\\
117	-0.104761297194103\\
118	-0.419237465072863\\
119	-0.0523189998296781\\
120	-0.0768411737612388\\
121	-0.139236977502743\\
122	0.0276395965518297\\
123	-0.0140285426197914\\
124	-0.0806214971063888\\
125	0.0263867511605079\\
126	0.0460580304566267\\
127	nan\\
128	-0.173811622638303\\
129	-0.10974086476214\\
130	-0.0507990945628054\\
131	-0.0728025878038892\\
132	0.0222118388324187\\
133	0.0363185993848166\\
134	0.0734376842153292\\
135	0.0336667922929843\\
136	-0.00991920319171624\\
137	0.00937863012335889\\
138	-0.00761756646769313\\
139	0.0430429318019514\\
140	-0.0306077288933996\\
141	-0.419941923197977\\
142	-0.057246909240319\\
143	-0.0384410902577501\\
144	0.0883684527555557\\
145	0.0111990845774435\\
146	0.0108542440234786\\
147	-0.0738308480389174\\
148	0.0295952773286623\\
149	0.00673140392301664\\
150	-0.0167476548085527\\
151	-0.00477607857846891\\
152	0.123967067201216\\
153	0.051698844506967\\
154	0.0381213374465553\\
155	0.020716777358153\\
156	0.0235163562779327\\
157	0.215470792206479\\
158	0.0559155521377211\\
159	0.036605144719932\\
160	-0.039776044782485\\
161	-0.0665321773332405\\
162	-0.0700413707222049\\
163	0.00978652665770475\\
164	0.0436755401320482\\
165	0.0228987415853389\\
166	-0.0154884600655549\\
167	0.0184169731807707\\
168	-0.0401208134678286\\
169	0.0411759538028637\\
170	-0.0237934135621796\\
171	-0.0123871426362652\\
172	0.0739620296686734\\
173	0.0195615906981918\\
174	0.0651260876811813\\
175	-0.0303899452706762\\
176	0.0436287452408109\\
177	0.0191688803649383\\
178	0.00840355103418225\\
179	0.0958795337938052\\
180	0.0140446658526974\\
181	0.0676158648691198\\
182	0.00640791975164628\\
183	0.0303009828790512\\
184	-0.0994653235826126\\
185	0.0712246515022809\\
186	-0.0808762441478835\\
187	0.0155932579789001\\
188	-0.00931717780727907\\
189	0.0150731451006184\\
190	0.0188591527404255\\
191	0.00968067939096548\\
192	0.100237421294454\\
193	0.0525604985243461\\
194	0.000638530076900374\\
195	0.0986389076436885\\
196	0.0914807345213507\\
197	0.125061157001554\\
198	-0.016015464386295\\
199	0.0510021451385751\\
200	-0.0258011756544677\\
201	0.0459394562766339\\
202	-0.00941871633395888\\
203	-7.55520003988876e-05\\
204	-0.0901083796304973\\
205	-0.0935858887981964\\
206	-0.0177398883251711\\
207	0.00706535085873735\\
208	0.0371564549558536\\
209	0.00215051615313052\\
210	0.0318752027796627\\
211	0.105000940332606\\
212	0.0197618518867625\\
213	0.0936925777053488\\
214	0.0776326682904282\\
215	0.129981106334259\\
216	0.104501994143678\\
217	0.125550090491769\\
218	0.0355674462123794\\
219	0.0533954104517503\\
220	0.0139076815454674\\
221	-0.0810052530656197\\
222	0.0287692464938397\\
223	0.0261627527076821\\
224	-0.0655168760210977\\
225	-0.0508405112929092\\
226	-0.0675880657603052\\
227	-0.00671118307397814\\
228	0.0434520149266367\\
229	0.0612387726738441\\
230	0.00569900039774141\\
231	nan\\
232	0.0655467788499101\\
233	-0.0196756770261993\\
234	-0.0418391737939853\\
235	0.0736224743590198\\
236	0.0216778620711751\\
237	-0.00190011317706467\\
238	0.0537005389197896\\
239	0.0242413869476011\\
240	-0.113996554880806\\
241	-0.169723127166857\\
242	0.0409965470764547\\
243	-0.0171390619206672\\
244	-0.295443260283733\\
245	-0.0941032767048099\\
246	-0.0270864439245175\\
247	0.00205563752126393\\
248	nan\\
249	0.0385430468655989\\
250	0.0458987125601525\\
251	0.0291309838401741\\
252	-0.0863508709436403\\
};
\addplot [color=dscr_plot_color,solid,line width=1.5pt,forget plot]
  table[row sep=crcr]{%
1	0.0609275984551955\\
2	0.0950875034786651\\
3	-0.0171970693855623\\
4	0.0287749011815951\\
5	-0.00283688707041341\\
6	0.0555567487150617\\
7	0.0392869024075223\\
8	0.139976500341321\\
9	0.0906270023399637\\
10	0.0788664216530345\\
11	0.186313902380164\\
12	0.118506059519284\\
13	0.0301503457485284\\
14	0.13810005370944\\
15	0.0149860502373101\\
16	0.242167005827668\\
17	0.270649058113191\\
18	0.00152634842132761\\
19	-0.0424654509653028\\
20	0.0878302300898151\\
21	0.102956420640893\\
22	0.0170364926120804\\
23	-0.0534108804068856\\
24	0.114603615277228\\
25	0.117790824239874\\
26	0.072921939554139\\
27	0.0822530911206107\\
28	-0.0804882784749721\\
29	0.00107511764534981\\
30	0.239778491201471\\
31	0.224803502576105\\
32	0.113306746069711\\
33	0.0774034777100572\\
34	0.168385496479386\\
35	-0.033864300218401\\
36	-0.0411067939793546\\
37	-0.115648386636437\\
38	-0.0329113001965352\\
39	0.28292061161448\\
40	-0.0595391674234815\\
41	0.12648111600364\\
42	-0.00514071295008419\\
43	0.00396858621873165\\
44	0.111405936137243\\
45	-0.0209933476646561\\
46	0.00168009156635855\\
47	0.147158729873515\\
48	0.129572716471885\\
49	0.0548857562411508\\
50	0.165286372348446\\
51	0.0460633080491916\\
52	0.00742233808868739\\
53	0.0833581977307227\\
54	0.055425800453132\\
55	0.0390369778020091\\
56	0.0833721716900087\\
57	-0.0147107927370948\\
58	-0.0769346690440123\\
59	0.128980552397636\\
60	0.124655826947206\\
61	0.1306429932187\\
62	0.282038950133202\\
63	0.0417678517697073\\
64	0.00858961249779559\\
65	0.083873558429835\\
66	0.0802231534094313\\
67	0.140531047212986\\
68	0.150242012120184\\
69	0.241449521372468\\
70	0.270482989442111\\
71	0.0285657836968452\\
72	0.481369378366072\\
73	-0.104948428312733\\
74	-0.0541622011243189\\
75	0.22844345369596\\
76	0.399036238060357\\
77	0.292317909222674\\
78	0.247558173713441\\
79	-0.0306423072126751\\
80	0.298193027534852\\
81	0.0224741761338023\\
82	0.229317204467132\\
83	-0.107515207874432\\
84	0.0684554058759941\\
85	0.375037941168099\\
86	0.322164310817521\\
87	0.0939390734250137\\
88	0.237971591805726\\
89	0.0878030899202089\\
90	0.209256294831725\\
91	0.0750197434729755\\
92	0.0799208917754899\\
93	0.42989919425326\\
94	1.04223704402971\\
95	0.0134742073558479\\
96	0.0496027838228986\\
97	0.138381946924433\\
98	0.0790603777575076\\
99	0.0687445519884812\\
100	0.130697967786828\\
101	0.497451906971097\\
102	0.182296379355635\\
103	0.156338045993842\\
104	0.0965748238084229\\
105	0.0332511495905237\\
106	0.235527779582819\\
107	0.0940473071544629\\
108	0.217892921736545\\
109	0.0778135784284177\\
110	0.0928866581190771\\
111	0.116468856123083\\
112	0.0289182169645737\\
113	0.121005161796437\\
114	0.0156199141356453\\
115	0.100355176029493\\
116	0.19877707470414\\
117	0.00186972317330306\\
118	-0.0197838231672022\\
119	-0.104695465828348\\
120	0.0609763712451391\\
121	-0.0154013977547844\\
122	0.0830393037212593\\
123	0.044295299264677\\
124	0.0512542626286886\\
125	0.0633694787218436\\
126	0.327088551393902\\
127	nan\\
128	0.108762703693394\\
129	-0.0331187979584362\\
130	0.0161605783302232\\
131	0.0436341106679181\\
132	0.104394809531516\\
133	0.0786056495668824\\
134	0.101685868955278\\
135	0.0583805526960136\\
136	0.11990653429152\\
137	0.0611890641710924\\
138	0.128994225423853\\
139	0.163377760093438\\
140	0.30035296646803\\
141	0.0101577972613104\\
142	0.158542544930491\\
143	0.0585802179849182\\
144	0.0359168149655003\\
145	0.0561347604222076\\
146	0.0238099775565552\\
147	0.0590474125612155\\
148	0.042548754582079\\
149	0.157120351355349\\
150	0.0423260278722799\\
151	0.156414997167463\\
152	0.301620368641878\\
153	0.108746644996181\\
154	0.0329985337140441\\
155	0.138400611124417\\
156	0.0820424090692136\\
157	0.888417419446388\\
158	0.153773940083529\\
159	0.0943990042580011\\
160	0.0385683948458663\\
161	-0.0279894572048993\\
162	0.0760980130710236\\
163	-0.00957367808115706\\
164	0.0120820551230113\\
165	0.116586717969787\\
166	-0.12553876341692\\
167	0.0602045171665562\\
168	0.0182677827927335\\
169	0.142795875255276\\
170	0.0985693987018073\\
171	0.048321731189936\\
172	0.193646467807197\\
173	0.0808444492424373\\
174	0.0746833072657553\\
175	0.115658246323195\\
176	0.119148065788819\\
177	0.0863380504620934\\
178	0.00211501188468592\\
179	0.0735366365524754\\
180	0.0294108039856393\\
181	0.0690322315072054\\
182	0.00580010499653321\\
183	0.0940971452637962\\
184	0.0449619567982585\\
185	0.072686997432567\\
186	0.0683895235658366\\
187	0.00181437511091313\\
188	0.0659647441695265\\
189	0.0824313402119695\\
190	0.0926905965924258\\
191	0.0472561385423525\\
192	0.116962909059665\\
193	0.190410798314783\\
194	0.024985022164049\\
195	0.0628646265311809\\
196	0.0744729573793952\\
197	0.1005908398202\\
198	0.028693186001653\\
199	0.0656473870289947\\
200	0.027186821758811\\
201	0.272177789269849\\
202	0.106525902245672\\
203	0.0214425359878095\\
204	-0.0463844875978321\\
205	-0.0536553946432129\\
206	0.065712399236653\\
207	0.0351628287713756\\
208	0.144250905882125\\
209	0.0386007555359326\\
210	-0.0438214774575132\\
211	0.20440423781531\\
212	0.120078231799797\\
213	0.136132529038807\\
214	0.116124524754091\\
215	0.211140347605339\\
216	0.31695666981137\\
217	0.188475890604656\\
218	0.128809959505246\\
219	0.261604502121207\\
220	0.272955701250515\\
221	-0.0599373656583528\\
222	0.159493039250562\\
223	0.00512007204121944\\
224	0.0838935562615367\\
225	0.095609998226938\\
226	0.0746528434361931\\
227	0.103900449944385\\
228	0.0697632489517808\\
229	0.0763906594030649\\
230	0.222812154117533\\
231	nan\\
232	0.107424496536352\\
233	0.0322100952873769\\
234	0.214996356656436\\
235	0.304441427968825\\
236	0.0669956127574658\\
237	0.0349308505738761\\
238	0.120862133678185\\
239	-0.0440005338920748\\
240	0.00892630345809987\\
241	-0.00535660601520037\\
242	0.0545656999881955\\
243	0.0702386612435167\\
244	0.0112258666516098\\
245	0.00524987508684319\\
246	-0.00861602522280953\\
247	0.0395295871695247\\
248	nan\\
249	0.0258054019817371\\
250	0.0867563047498356\\
251	0.105287084366041\\
252	-0.00922145539284379\\
};
\addplot [color=cts_nFPC_plot_color,solid,line width=1.5pt,forget plot]
  table[row sep=crcr]{%
1	-0.221897345898991\\
2	-0.160668840235624\\
3	-0.279885385958042\\
4	-0.458021461871051\\
5	-0.427844593226991\\
6	0.0140876063628836\\
7	-0.206510527558516\\
8	-0.206428300917142\\
9	-0.223462578915298\\
10	-0.052753174298243\\
11	-0.141998968194952\\
12	0.0172065072920903\\
13	-0.122288994152593\\
14	0.0675137017402154\\
15	-0.105228543201608\\
16	-0.0925332175132246\\
17	0.00302070982548691\\
18	-0.174930786261241\\
19	0.106073819881101\\
20	-0.251512509707802\\
21	-0.114172592314085\\
22	-0.114501590715144\\
23	-0.183735921700641\\
24	-0.0809647374496597\\
25	-0.109233311446707\\
26	-0.172859308385964\\
27	-0.273959351585184\\
28	-0.240582979518584\\
29	-0.0236744820109217\\
30	-0.023961526511399\\
31	-0.371439868264196\\
32	-0.316771831149095\\
33	-0.0975116151337195\\
34	-0.288280975634245\\
35	-0.486296309861344\\
36	-0.380735790209438\\
37	-0.359507129081605\\
38	-0.313340173346978\\
39	-0.366952929246599\\
40	-0.22360468596119\\
41	-0.347311121844036\\
42	-0.269759877867614\\
43	-0.312871137904917\\
44	-0.193453502683584\\
45	-0.273410230700954\\
46	-0.12153311108687\\
47	-0.0972222878148076\\
48	-0.26496222824649\\
49	-0.393187912229119\\
50	-0.062924003325565\\
51	-0.229454618256752\\
52	-0.104693817552017\\
53	-0.262146006499089\\
54	-0.25145298189975\\
55	-0.24991900942359\\
56	-0.0746912866368648\\
57	-0.192475187142345\\
58	-0.204338770921482\\
59	-0.11177724124841\\
60	-0.153459353138629\\
61	-0.112876080573347\\
62	-0.324310870083684\\
63	-0.260881071860624\\
64	-0.242199610896453\\
65	-0.469899873913509\\
66	-0.14149280795039\\
67	-0.0995159235285039\\
68	-0.496832201991141\\
69	-0.20459102303056\\
70	-0.934599634568009\\
71	-0.497859637771278\\
72	-0.13596962129122\\
73	-0.559296072963275\\
74	-0.446300263861607\\
75	-0.308132504226949\\
76	-0.444610482037457\\
77	-0.544940010793985\\
78	-0.210195372656718\\
79	-0.286059975944909\\
80	-0.240496423803031\\
81	-0.0105768395022788\\
82	0.243881297539726\\
83	-0.146046194126093\\
84	-0.146196977012992\\
85	0.049753090356078\\
86	-0.286069734494076\\
87	-0.209887385794233\\
88	-0.532147433899234\\
89	-0.206428929429583\\
90	-0.14899877699009\\
91	-0.383165747682687\\
92	-0.257973289124885\\
93	0.0138185561852694\\
94	0.124092915540727\\
95	-0.261134851097768\\
96	-0.126134310301268\\
97	-0.185584770803984\\
98	-0.118268909954323\\
99	-0.154238864464098\\
100	0.029679930759376\\
101	-0.0046397654097486\\
102	-0.0913352042124955\\
103	-0.176450320984024\\
104	-0.0896893281668327\\
105	-0.1452356479199\\
106	-0.337174360678847\\
107	-0.0917525093738904\\
108	0.0253060404451461\\
109	-0.0495270858659681\\
110	-0.0284976617586047\\
111	-0.0888942319517275\\
112	-0.0360799734455534\\
113	-0.210576002029873\\
114	-0.140481119340807\\
115	-0.169065717275663\\
116	-0.0152268517606645\\
117	-0.332567104532107\\
118	-0.451082739930509\\
119	-0.397449048302414\\
120	-0.158616514672985\\
121	-0.342908024729248\\
122	-0.0434280932827393\\
123	-0.13446557111096\\
124	-0.312286159419527\\
125	-0.141320499143886\\
126	-0.0960068374850642\\
127	nan\\
128	-0.198018470670532\\
129	-0.207128686273796\\
130	-0.217747794475989\\
131	-0.245912579874692\\
132	0.0118144657748489\\
133	-0.0223462292829573\\
134	-0.112271478697404\\
135	-0.140456787743348\\
136	-0.0755598349197332\\
137	-0.246919679826006\\
138	-0.120572130405933\\
139	-0.0942241508565454\\
140	-0.0798148456920532\\
141	-0.227994612582025\\
142	-0.0745363501705154\\
143	-0.142023602627177\\
144	-0.009435055148019\\
145	-0.0707421247137581\\
146	-0.0850130486290516\\
147	-0.157320870299976\\
148	-0.142833692541075\\
149	-0.0141103706127006\\
150	-0.0788826816709233\\
151	-0.050786100929248\\
152	0.0809644518103945\\
153	-0.033335789331376\\
154	-0.0417381893153845\\
155	-0.0435194348387565\\
156	-0.182188324290648\\
157	0.147776634812953\\
158	-0.0362198988986458\\
159	0.0159161737390447\\
160	-0.0575344958920988\\
161	-0.230790098411685\\
162	-0.310720241373073\\
163	-0.130926207886537\\
164	0.0192152543262268\\
165	-0.0718739031353337\\
166	-0.16648021256982\\
167	-0.0778439010531418\\
168	-0.120067847069719\\
169	0.0138877607599651\\
170	-0.00310585985525876\\
171	-0.0372699641512008\\
172	-0.0440758804487057\\
173	-0.103906360103764\\
174	-0.0201963104618272\\
175	-0.0476365110943458\\
176	-0.0326325062188797\\
177	-0.0362745844317767\\
178	-0.00948586745426994\\
179	-0.0173009186567767\\
180	-0.122218548957123\\
181	0.00952649933091816\\
182	-0.189100955681924\\
183	-0.0488301695343668\\
184	-0.27331411928271\\
185	0.014679390592631\\
186	-0.144924022471503\\
187	-0.0760423226990721\\
188	-0.0859954592593946\\
189	-0.100959083711751\\
190	0.0145750868249861\\
191	-0.231798650348902\\
192	0.0639288739717163\\
193	-0.0901664453825316\\
194	-0.083539464937944\\
195	0.0470877271830689\\
196	-0.0507480920331644\\
197	0.00155213034442048\\
198	-0.189549396383355\\
199	-0.286625395390046\\
200	-0.128150457053247\\
201	-0.0412592839902325\\
202	-0.162549974568938\\
203	-0.031943905765726\\
204	-0.29773625833293\\
205	-0.325346780007507\\
206	-0.168221412130828\\
207	-0.0633227723266867\\
208	-0.117001835589108\\
209	-0.134381106726437\\
210	-0.0433483297145184\\
211	-0.0003632309699483\\
212	-0.0178307888589857\\
213	-0.114834280804606\\
214	0.0170399694552865\\
215	0.0270737923937779\\
216	-0.0959699037638495\\
217	-0.0694238716907414\\
218	-0.0161833311756387\\
219	-0.00507590309371556\\
220	-0.321434704125238\\
221	-0.265021543448214\\
222	-0.13978849243942\\
223	-0.0884519572481613\\
224	0.00173112514018595\\
225	-0.0909880378550325\\
226	-0.120763393679595\\
227	-0.0553755553259604\\
228	0.0353365240750126\\
229	0.0442624421193745\\
230	-0.0453609808381064\\
231	nan\\
232	-0.0439458818892116\\
233	-0.098423329001228\\
234	-0.162686460592699\\
235	0.0381143710994553\\
236	-0.0107080146672956\\
237	-0.0075731188465051\\
238	-0.0689378157327155\\
239	-0.0963057206277922\\
240	-0.283297652264101\\
241	-0.0672891405026618\\
242	0.000480376018287777\\
243	0.0619812667574372\\
244	0.198345472450602\\
245	0.0369578240377158\\
246	-0.111960725821409\\
247	0.0308970637992113\\
248	nan\\
249	0.00332902375952278\\
250	-0.0324401927198416\\
251	-0.00282069915454051\\
252	-0.10009411415034\\
};
\addplot [color=dscr_nFPC_plot_color,solid,line width=1.5pt,forget plot]
  table[row sep=crcr]{%
1	0.0776663767348847\\
2	0.0907141560058189\\
3	0.0454369998667038\\
4	0.184901366138002\\
5	0.10688206566116\\
6	0.31861578912269\\
7	0.0729730451922033\\
8	0.148629079977738\\
9	0.0807282085447651\\
10	0.112636765259078\\
11	-0.0729304265383075\\
12	0.143314519850656\\
13	0.0275546492612374\\
14	-0.036888453046952\\
15	0.0986327089522674\\
16	0.320057627928247\\
17	0.0311923977844148\\
18	0.198892346480778\\
19	-0.0125696648755947\\
20	0.105798659545416\\
21	0.0926620507902995\\
22	0.0549343229868392\\
23	0.168669085912424\\
24	0.0513615268880323\\
25	0.0567100336967936\\
26	0.0564280439319978\\
27	0.0821073974134711\\
28	0.0263977618174404\\
29	0.295885724292233\\
30	0.222209014156316\\
31	0.301671642633316\\
32	0.140113138300991\\
33	0.0666840027423222\\
34	0.269240433904748\\
35	0.157232494578222\\
36	0.00842305132795561\\
37	-0.13888135572351\\
38	0.0137876048954372\\
39	0.213049439032533\\
40	0.238143798649001\\
41	0.198060687778523\\
42	0.125590684203624\\
43	0.0424317611256796\\
44	0.108244275294955\\
45	0.142257591144978\\
46	0.0141186558128219\\
47	0.180176323935343\\
48	0.143820449404543\\
49	0.0349572248022134\\
50	0.217018598220828\\
51	0.0520519094901039\\
52	0.0342343908327333\\
53	0.218563665085742\\
54	0.152368891881181\\
55	0.0905876912997299\\
56	0.202451042500571\\
57	0.00752546296332642\\
58	0.0106105911060552\\
59	0.0103785619107652\\
60	0.157718186052121\\
61	0.171582539710896\\
62	0.370976143391494\\
63	0.175642975919649\\
64	0.0832387705566563\\
65	0.164668264224106\\
66	0.13707108823537\\
67	0.171821406704652\\
68	0.147935631735202\\
69	0.248181447570254\\
70	0.317021450067863\\
71	0.0848957242011071\\
72	0.540308030531828\\
73	-0.103522737041812\\
74	-0.0576962771897293\\
75	0.165189562522402\\
76	0.484005198570207\\
77	0.322791469160326\\
78	0.206461972664454\\
79	-0.0456141109591401\\
80	0.262991652579518\\
81	0.210455226196022\\
82	0.250106894140199\\
83	-0.13530868818557\\
84	0.0444441174315221\\
85	0.394214892109759\\
86	0.281215859942133\\
87	0.109748227189895\\
88	0.26364162623539\\
89	0.0587500460069281\\
90	0.209028617764171\\
91	0.0462243886058061\\
92	0.170370792820705\\
93	0.00916657634758967\\
94	0.962469782420417\\
95	0.708356047246591\\
96	0.171301391695062\\
97	0.0933137526503772\\
98	0.0159928971600543\\
99	0.441869217809132\\
100	0.0969920439870136\\
101	0.510370000209713\\
102	0.183711212036224\\
103	0.127959093365516\\
104	0.0482598455376544\\
105	0.101972072136078\\
106	0.126779293705925\\
107	0.0578144747682026\\
108	0.196216567530989\\
109	0.0348716120292757\\
110	0.0810212361258562\\
111	0.0227812853216696\\
112	-0.00790898768559478\\
113	0.0579328454514521\\
114	-0.0231021320139937\\
115	-0.0262234009538987\\
116	0.194646699024379\\
117	0.148702372396008\\
118	-0.0532389917708331\\
119	0.132590700824369\\
120	0.063995444079536\\
121	-0.0177346696411575\\
122	0.0952258138681367\\
123	0.0818859198686564\\
124	0.0515988720263589\\
125	0.0886858968332003\\
126	0.277491549324952\\
127	nan\\
128	0.0818727093008569\\
129	0.00344747290233926\\
130	0.0680292829861298\\
131	0.0560777409972058\\
132	0.0303767642564203\\
133	0.0722841731122094\\
134	0.163692085465261\\
135	0.0789787312394804\\
136	0.132075240415809\\
137	-0.0588214340247697\\
138	0.155823820197212\\
139	-0.0496488223140521\\
140	0.244844290147389\\
141	-0.0635252575949201\\
142	0.0771504696893602\\
143	0.0858840466685321\\
144	0.0581597249188646\\
145	0.0840240650747702\\
146	0.0562755164613257\\
147	0.0770581420101036\\
148	0.0620026264832715\\
149	0.193718811605707\\
150	0.0551868249721559\\
151	0.175545791949479\\
152	0.265226662285279\\
153	0.0910554779124941\\
154	0.0294235588541712\\
155	0.138893131048589\\
156	0.115875323159545\\
157	0.935671911456588\\
158	0.170399949043299\\
159	0.154007446706733\\
160	0.0459429833404734\\
161	-0.0271183257486216\\
162	0.0995257935699255\\
163	0.0439444915789618\\
164	0.0592577771226312\\
165	0.0906661460713747\\
166	-0.00934421917577832\\
167	0.0691920003913826\\
168	0.0404467400894151\\
169	0.116357079062839\\
170	0.0873829868840405\\
171	0.0689207344763225\\
172	0.115797796108756\\
173	0.102397043450317\\
174	0.102132131921362\\
175	0.0988197625440041\\
176	0.137725673675573\\
177	0.0417680894954063\\
178	0.126107383932523\\
179	0.0688883613033566\\
180	0.0301486515163635\\
181	0.112680280368023\\
182	0.0228471102546033\\
183	0.107690034234174\\
184	0.0306535256219218\\
185	0.0585133081083639\\
186	0.0991100593273178\\
187	0.00968107514038874\\
188	0.135917373184386\\
189	0.0993256221605311\\
190	0.0905046784876292\\
191	0.0566056281555663\\
192	0.144957112923868\\
193	0.191822031872813\\
194	0.058318735155575\\
195	0.0899274475943735\\
196	0.0975745966477257\\
197	0.108267845644366\\
198	0.0281725014584351\\
199	0.256188622568536\\
200	0.109476000772486\\
201	0.292313285262679\\
202	0.0647289062570355\\
203	0.0251197348948053\\
204	0.00895702473531189\\
205	0.340164948773289\\
206	0.0859461024205246\\
207	0.0634386156723113\\
208	0.170725975670945\\
209	0.0612547323867039\\
210	-0.0248490764958978\\
211	0.207818489945175\\
212	0.0964755625880126\\
213	0.154068018704167\\
214	0.123824293700171\\
215	0.210578372713573\\
216	-0.0237832666641564\\
217	0.19835660928597\\
218	0.113216170851969\\
219	0.194208040118491\\
220	0.397498485209749\\
221	-0.0876346285626362\\
222	0.186936948816548\\
223	-0.0238837197366306\\
224	0.149373899472736\\
225	0.190120750748549\\
226	0.0805772119481869\\
227	0.0273366850750235\\
228	0.0537738531530057\\
229	0.0975099375201429\\
230	0.101562787844492\\
231	nan\\
232	0.0783348603028827\\
233	0.0619807971405299\\
234	0.232791527437023\\
235	0.408532536242019\\
236	0.0664400521112427\\
237	0.1030434208326\\
238	0.125640660788505\\
239	0.541335529362101\\
240	-0.00255034936805533\\
241	-0.0038956891717937\\
242	0.0894642139762351\\
243	0.0648820468057703\\
244	-0.0134466406131337\\
245	0.151959596143075\\
246	0.0116627653897352\\
247	0.0431726523143674\\
248	nan\\
249	0.0288866414171665\\
250	0.0882512063336328\\
251	0.0821384908360475\\
252	0.0203376197223765\\
};
\end{axis}
\end{tikzpicture}%
 
\end{subfigure}\\
\vspace{1cm}
\begin{subfigure}{.45\linewidth}
  \centering
  \setlength\figureheight{\linewidth} 
  \setlength\figurewidth{\linewidth}
  \tikzsetnextfilename{IS_sameday_ORCL}
  % This file was created by matlab2tikz.
%
%The latest updates can be retrieved from
%  http://www.mathworks.com/matlabcentral/fileexchange/22022-matlab2tikz-matlab2tikz
%where you can also make suggestions and rate matlab2tikz.
%
%
\begin{tikzpicture}[trim axis left, trim axis right]

\begin{axis}[%
width=\figurewidth,
height=\figureheight,
at={(0\figurewidth,0\figureheight)},
scale only axis,
every outer x axis line/.append style={black},
every x tick label/.append style={font=\color{black}},
xmin=1,
xmax=252,
%xlabel={Time (h)},
every outer y axis line/.append style={black},
every y tick label/.append style={font=\color{black}},
ymin=-1.1,
ymax=1.1,
%ylabel={Normalized PnL},
title={ORCL},
axis background/.style={fill=white},
axis x line*=bottom,
axis y line*=left,
yticklabel style={
        /pgf/number format/fixed,
        /pgf/number format/precision=3
},
scaled y ticks=false,
]
\addplot [color=cts_plot_color,solid,line width=1.5pt,forget plot]
  table[row sep=crcr]{%
1	0.241223274287608\\
2	0.24745065209925\\
3	0.127746130518664\\
4	0.0203272197232222\\
5	0.060032618753344\\
6	0.137470270516756\\
7	0.223445930484044\\
8	0.164140903838792\\
9	0.154450640745265\\
10	0.053217762948819\\
11	0.23458028517867\\
12	0.14572885290245\\
13	0.521284360126471\\
14	0.177521212622025\\
15	0.125315186751598\\
16	0.209854204316187\\
17	0.210490303692473\\
18	0.159760094908392\\
19	0.197355786902324\\
20	0.13823422781648\\
21	0.128173567773223\\
22	0.110295740028735\\
23	-0.0459245507968338\\
24	0.100599672742255\\
25	0.110979164201082\\
26	0.1345583952384\\
27	0.142962944472724\\
28	0.162019958135607\\
29	0.167794129093981\\
30	0.0450090971134973\\
31	0.0602639492750183\\
32	0.139149109878398\\
33	0.248690911586315\\
34	-0.0655120377294428\\
35	0.115125661875154\\
36	0.0954162073168408\\
37	0.0669382802350141\\
38	0.168464080407859\\
39	0.0927190483618675\\
40	0.248860269652851\\
41	0.0973576369216187\\
42	0.169025168510289\\
43	0.150855917903857\\
44	0.23514066295871\\
45	0.108436124677382\\
46	0.101299615551686\\
47	0.125910215136719\\
48	0.0379509192574323\\
49	0.0558724696828833\\
50	0.156030085921235\\
51	0.139719738402309\\
52	0.136003184936705\\
53	0.191848662819897\\
54	0.206937220712009\\
55	0.290068723473809\\
56	0.18894115508058\\
57	-0.0523846010311696\\
58	0.107298264062165\\
59	0.31809637815883\\
60	0.226176092613965\\
61	0.191197845873848\\
62	0.180443154787067\\
63	0.127039685403067\\
64	0.219638567605891\\
65	0.202412509975261\\
66	0.1569711490769\\
67	0.28055513945065\\
68	0.245362139750547\\
69	0.185467343604661\\
70	0.060709656064548\\
71	-0.0285056144430903\\
72	0.14458350097571\\
73	0.207388957403149\\
74	0.230732506321745\\
75	0.257405714580662\\
76	0.228033641021165\\
77	0.147286436881983\\
78	0.240994694556747\\
79	0.145005857526444\\
80	0.221598854926346\\
81	0.157875393697151\\
82	0.126219773292884\\
83	0.178324864158897\\
84	0.195839825256848\\
85	0.0928970420706774\\
86	0.155724657821501\\
87	-0.000465347954131393\\
88	0.127258315497155\\
89	0.199381256230535\\
90	0.107231200134548\\
91	0.0301832428779352\\
92	-0.0985974903465689\\
93	0.0808793552600453\\
94	-0.00860132105104882\\
95	0.0729748833160431\\
96	0.078747488141741\\
97	0.139186196515515\\
98	-0.114958700000476\\
99	0.184070586197995\\
100	0.0369592578900176\\
101	0.0546503495155683\\
102	0.153141414163885\\
103	0.12884774990341\\
104	-0.00867147887933388\\
105	0.29183544379335\\
106	0.193418498046049\\
107	0.0920431132441799\\
108	0.163648754280062\\
109	0.133098513253821\\
110	0.237121761593614\\
111	0.0299341390334858\\
112	0.188724675552604\\
113	0.172244592484314\\
114	0.13218191213765\\
115	0.17903778535716\\
116	0.0739145224338239\\
117	0.16561557862633\\
118	0.0855004972167472\\
119	-0.216596498114913\\
120	0.271285706464782\\
121	0.19548447765328\\
122	0.142875503733172\\
123	0.0290956358200346\\
124	0.216256923540652\\
125	-0.128048224230666\\
126	0.217388801809032\\
127	nan\\
128	0.167492394919075\\
129	0.189328149615795\\
130	0.0959364006512009\\
131	0.108981973590273\\
132	0.0897098506433756\\
133	-0.14266997682801\\
134	0.147425894288815\\
135	0.0672618200057133\\
136	0.113532146342714\\
137	0.0447675775724201\\
138	-0.0210005222149592\\
139	0.0633252183434963\\
140	0.0220889265289387\\
141	0.0827438285466515\\
142	0.180875311005951\\
143	0.132219491113588\\
144	0.160563645809524\\
145	0.0890586783857221\\
146	0.110891060530117\\
147	0.0812720090836312\\
148	0.0860984505059703\\
149	0.100754722510865\\
150	0.0685619150860805\\
151	0.0487189280678173\\
152	0.0813039281390116\\
153	0.0166560683293825\\
154	0.0550293357495144\\
155	-0.000154935184961307\\
156	0.107560800217674\\
157	0.0316463098747111\\
158	0.12151499323743\\
159	0.0434969721000555\\
160	0.102875303407633\\
161	0.0988586081988907\\
162	0.0268600688505\\
163	0.156856156054873\\
164	0.078127863406937\\
165	0.0402222962530346\\
166	0.0495058588916819\\
167	0.034743352642104\\
168	0.0335260093509291\\
169	0.046336613013224\\
170	0.0209464277052586\\
171	0.0652117287117677\\
172	0.0324816015661456\\
173	0.0482661868731509\\
174	0.0497191253010622\\
175	-0.016562153628797\\
176	0.0863502930792854\\
177	0.070404280217022\\
178	0.00637605090709595\\
179	0.0393588987029311\\
180	0.142256298886192\\
181	0.16421400363218\\
182	0.120326306884496\\
183	0.0344992241089748\\
184	0.0473475310219116\\
185	0.0636802096642029\\
186	0.0631257278991188\\
187	0.12088653154436\\
188	0.124133771256082\\
189	0.0584762516333993\\
190	0.105455357015284\\
191	0.0682468045322736\\
192	0.140588009293907\\
193	0.164773444874498\\
194	0.231907967089212\\
195	0.205664160133516\\
196	0.248029570591285\\
197	0.149379353264988\\
198	0.10466505765627\\
199	0.137433651124083\\
200	0.133102222995278\\
201	0.069180818207139\\
202	0.143775577759242\\
203	0.0739502508429764\\
204	0.173104009949441\\
205	0.191529757841133\\
206	0.0716301103331933\\
207	0.163363505350456\\
208	0.0587338333643523\\
209	0.116289849557864\\
210	0.146317972320714\\
211	0.126036872851262\\
212	0.154504471955781\\
213	0.053379498807833\\
214	0.0772241375076522\\
215	0.143404427705278\\
216	0.068022502603057\\
217	0.00854996506683371\\
218	0.0341009738373644\\
219	0.0944268509716125\\
220	0.122148291575481\\
221	0.130590493220607\\
222	0.00489377884325976\\
223	0.0881278447922347\\
224	0.0231001602860589\\
225	0.0640433149208885\\
226	0.0506999679533119\\
227	0.11438020428506\\
228	0.0442678639845931\\
229	0.148451266074303\\
230	0.100813727774735\\
231	nan\\
232	-0.0234799535535712\\
233	0.141941938597241\\
234	0.141906435665674\\
235	0.101901315463783\\
236	0.176801485681818\\
237	0.0863980044614175\\
238	0.0254058724886769\\
239	0.098460953890485\\
240	0.058179083754149\\
241	0.0115479930953512\\
242	0.143984141545859\\
243	0.101118132867356\\
244	0.294196493306117\\
245	0.282158026086236\\
246	0.137386117329218\\
247	0.151718434304225\\
248	nan\\
249	0.11292797935102\\
250	0.0398592077364431\\
251	0.042703010834764\\
252	0.0483242416796185\\
};
\addplot [color=dscr_plot_color,solid,line width=1.5pt,forget plot]
  table[row sep=crcr]{%
1	0.273346665066574\\
2	0.189212312626568\\
3	0.211205152629585\\
4	0.179013768165097\\
5	0.169064698420172\\
6	0.0964873429571397\\
7	0.17558688117321\\
8	0.122105728437492\\
9	0.164119278284694\\
10	0.175848435914647\\
11	0.188204213997302\\
12	0.121890287422419\\
13	0.515050783158148\\
14	0.188028003721372\\
15	0.124428940917841\\
16	0.2220490561916\\
17	0.221516059081375\\
18	0.228332917589034\\
19	0.209887733030315\\
20	0.0875383764195\\
21	0.220762736769412\\
22	0.127634483875145\\
23	0.0257226860606545\\
24	0.117477215430799\\
25	0.075484976001504\\
26	0.108252039055186\\
27	0.160307065992963\\
28	0.0980375962929234\\
29	0.176103032120917\\
30	0.0663116570919216\\
31	0.100264220076255\\
32	0.130377427069623\\
33	0.251174961680548\\
34	0.0718727701614224\\
35	0.0819917590646432\\
36	0.0916017618022998\\
37	0.0302330760789778\\
38	0.102931926247013\\
39	0.126051405992587\\
40	0.112587374030975\\
41	0.164423714118142\\
42	0.259079847017375\\
43	0.097595037307296\\
44	0.2418257337092\\
45	0.147104344096026\\
46	0.083169093589618\\
47	0.123792769995301\\
48	0.0633159197964904\\
49	0.0641380249115127\\
50	0.198799785600185\\
51	0.166613207334672\\
52	0.171839442460413\\
53	0.0361148384642435\\
54	0.232225143110759\\
55	0.485188252045305\\
56	0.18868442443865\\
57	0.101076171797811\\
58	0.150680913939971\\
59	0.336958755375116\\
60	0.24505226999223\\
61	0.19123162197311\\
62	0.191380349176894\\
63	0.156593554694079\\
64	0.20705005338918\\
65	0.297069140122517\\
66	0.129093895116347\\
67	0.286397382291399\\
68	0.329564105009742\\
69	0.192508713170727\\
70	0.114080020307172\\
71	0.0032055876252193\\
72	0.183312677255579\\
73	-0.0566901234503444\\
74	0.193117321765846\\
75	0.320902819989265\\
76	0.218212823777351\\
77	0.131936428895831\\
78	0.165131312391019\\
79	0.171946799426226\\
80	0.231091102315572\\
81	0.196226962892521\\
82	0.416299269499185\\
83	0.12724580623387\\
84	0.24659822718522\\
85	0.118540866460038\\
86	0.143455446634539\\
87	0.118801469887235\\
88	0.192628530541603\\
89	0.280904679770813\\
90	0.126576332970002\\
91	0.039765584493218\\
92	0.108558612643594\\
93	0.186690131816102\\
94	0.120046391751876\\
95	0.129465249564172\\
96	0.111564228059796\\
97	0.189333991136525\\
98	0.0288264833224655\\
99	0.174054847095928\\
100	0.105890654914792\\
101	0.0894025798863696\\
102	0.0771669747740801\\
103	0.0839455837165869\\
104	0.0653931761400871\\
105	0.335273437562528\\
106	0.201584907139371\\
107	0.146825567491122\\
108	0.128345921588691\\
109	0.143479640545705\\
110	0.216413939197816\\
111	0.127768664139924\\
112	0.111208610819479\\
113	0.24728612832673\\
114	0.108781785973734\\
115	0.16325065491916\\
116	0.137459933321632\\
117	0.0673502806794637\\
118	-0.00791208882767072\\
119	0.156423076414051\\
120	0.29375609055909\\
121	0.118846864947344\\
122	0.156778739682479\\
123	0.313427626122744\\
124	0.297919902909278\\
125	0.0873580997993704\\
126	0.211048138151547\\
127	nan\\
128	0.212140002709667\\
129	0.214268833264449\\
130	0.244162177162469\\
131	0.157511081632883\\
132	0.194036613983136\\
133	0.139842108130623\\
134	0.168520280553994\\
135	0.106642265173287\\
136	0.145830393290165\\
137	0.0925566426252537\\
138	0.0627152187216657\\
139	0.0651092114789767\\
140	0.0976686090058129\\
141	0.075170588027189\\
142	0.0988593821602895\\
143	0.118902858429481\\
144	0.151727088886939\\
145	0.0436928544311631\\
146	0.113561753192908\\
147	0.0832077495537323\\
148	0.0539508409658656\\
149	0.0701439256636209\\
150	0.19270004503352\\
151	0.073373174575009\\
152	0.0557939043785814\\
153	0.0651762035367486\\
154	0.0606709459560761\\
155	0.027790196372764\\
156	0.115831565008953\\
157	0.0415731376565472\\
158	0.0851811833747788\\
159	0.0709144384693512\\
160	0.0658345812929325\\
161	0.0694815446780439\\
162	0.0747291028069253\\
163	0.0324007969749028\\
164	0.172988146407423\\
165	0.015647815386357\\
166	0.0262265276423608\\
167	0.0765869873848283\\
168	0.0991924831409798\\
169	0.103949430167673\\
170	0.101799169207094\\
171	0.0509383632951249\\
172	0.0860225548288399\\
173	0.21896668727248\\
174	0.097533578774753\\
175	0.0972294515542211\\
176	0.0977980045664691\\
177	0.0440044809180337\\
178	0.0711218674448398\\
179	0.135724878737634\\
180	0.153414262473705\\
181	0.260616041982053\\
182	0.111490494211181\\
183	0.107276117911108\\
184	0.0299050229668953\\
185	0.0769459936223605\\
186	0.0877406995209833\\
187	0.0595068027496756\\
188	0.0746817717294346\\
189	0.093149089038501\\
190	0.0604099500234195\\
191	-0.0634235302427407\\
192	0.121088758042349\\
193	0.10494311384696\\
194	0.122904842841552\\
195	0.121487137284002\\
196	0.323962352476322\\
197	0.0891732130448263\\
198	0.125268227571178\\
199	0.0629518651664227\\
200	0.138620723659144\\
201	0.0858094335144247\\
202	0.061835341186842\\
203	0.0403852289768351\\
204	0.0580164473040898\\
205	0.0946614662032086\\
206	0.180788875327448\\
207	0.102963476251914\\
208	0.0982523031321061\\
209	0.135988269401859\\
210	0.0471417200606991\\
211	0.142930524157119\\
212	0.103158214732359\\
213	0.0809070237730298\\
214	0.058940334861863\\
215	0.205968113911324\\
216	0.0323278939199908\\
217	0.0601643207931574\\
218	0.0674616700757591\\
219	0.147705767227897\\
220	0.113156192179139\\
221	0.0773647658653425\\
222	0.160827683540172\\
223	0.0739187978328479\\
224	0.0784007796469039\\
225	0.0472635467184074\\
226	0.0716836030065077\\
227	0.0793792271825583\\
228	0.0969261723876889\\
229	0.158513618723575\\
230	0.103394013234269\\
231	nan\\
232	0.0287296975388596\\
233	0.179287496936059\\
234	0.139689384171674\\
235	0.12502080733179\\
236	0.171006881758098\\
237	0.0964309718920965\\
238	0.0717890901144666\\
239	0.0315646560471625\\
240	0.10627969293463\\
241	0.0980803063454243\\
242	0.0841267982951749\\
243	0.0770993047727351\\
244	0.317135474553656\\
245	0.429330303626638\\
246	0.199891083029741\\
247	0.167428558387384\\
248	nan\\
249	0.128500149202004\\
250	0.136260048789033\\
251	0.0342568208026125\\
252	0.0668284671772081\\
};
\addplot [color=cts_nFPC_plot_color,solid,line width=1.5pt,forget plot]
  table[row sep=crcr]{%
1	-0.0846576225918833\\
2	0.155281601903694\\
3	0.0221751536407668\\
4	0.0337038384603742\\
5	0.0116085386732742\\
6	-0.0175877020692845\\
7	0.147556791679367\\
8	0.121744214406951\\
9	0.0752120937739852\\
10	0.028158031446299\\
11	0.11086890180201\\
12	0.065635175982436\\
13	0.503393570902585\\
14	-0.0381665301710048\\
15	0.0937484557313351\\
16	0.0741180141310412\\
17	0.149422872459086\\
18	0.0412212942352366\\
19	0.0395035093552537\\
20	-0.126479857853542\\
21	0.0275979621563786\\
22	-0.0477692996328487\\
23	-0.213250996409327\\
24	0.0700504263461083\\
25	-0.0622542471476072\\
26	-0.122209029044425\\
27	0.0481814026832434\\
28	0.0904226975773476\\
29	0.182607883130763\\
30	0.0611108060630259\\
31	0.00490750872288962\\
32	0.0429367998809651\\
33	0.0752582035032304\\
34	-0.190949119205558\\
35	-0.0863546057391088\\
36	-0.104020680247121\\
37	-0.138237662523695\\
38	-0.0511122550532538\\
39	-0.166307438326315\\
40	0.0836486133601431\\
41	0.176675160682967\\
42	0.0432541131287971\\
43	-0.0249965210769209\\
44	-0.0112820327496435\\
45	0.0235090691746489\\
46	-0.0283187733753653\\
47	0.0528482238468779\\
48	-0.018264867559768\\
49	-0.0289363217573902\\
50	0.0534948813879319\\
51	0.0954755842380508\\
52	0.0224398222063561\\
53	-0.00368548833273068\\
54	0.0120852123650379\\
55	0.0910526192212227\\
56	0.0799127668471845\\
57	-0.0693162163445281\\
58	0.0781683160041529\\
59	0.16649960400856\\
60	0.0778574941823046\\
61	0.0705434190131545\\
62	0.0221829696954044\\
63	-0.0379533784247016\\
64	0.0848255226525035\\
65	0.145029474590804\\
66	0.0732602800011299\\
67	-0.0524642139851114\\
68	-0.0531025130736515\\
69	0.0234836439557796\\
70	-0.0682540646419776\\
71	-0.132420767879116\\
72	-0.0428638835429499\\
73	-0.261856361126151\\
74	-0.126752828756445\\
75	0.0556289157318817\\
76	0.0434120064241584\\
77	-0.118498751536332\\
78	0.0348167372240673\\
79	-0.00329157252309795\\
80	-0.0468210966101787\\
81	0.0876246348069102\\
82	0.191871458697738\\
83	0.0373557634466459\\
84	-0.0121371214121342\\
85	-0.008343365232015\\
86	-0.0331095093905462\\
87	-0.0322958227548654\\
88	0.097098128135251\\
89	0.146093976955255\\
90	-0.0138021304815943\\
91	-0.121009711924756\\
92	-0.168487320043594\\
93	0.00986484180124042\\
94	-0.0457537434853789\\
95	0.00519213619101177\\
96	-0.0333620089540813\\
97	0.0901430146916826\\
98	-0.281331014338772\\
99	-0.157999852400096\\
100	-0.224051074956349\\
101	-0.091912764282101\\
102	-0.101582871739512\\
103	-0.0692772857726541\\
104	-0.181591250066822\\
105	0.00914738862360414\\
106	0.0710280024488038\\
107	-0.12063125893102\\
108	-0.0717977254365932\\
109	0.0293394093789232\\
110	-0.00226067818741903\\
111	-0.215804458070252\\
112	-0.0375676753045819\\
113	0.0140450866257482\\
114	-0.0175386469973769\\
115	0.0837311057976844\\
116	0.13808153812408\\
117	-0.0808118752226555\\
118	-0.206266489480464\\
119	-0.478207937042886\\
120	0.121972852644426\\
121	0.0477286903377687\\
122	0.173686494231355\\
123	-0.247103007819378\\
124	-0.0644129362762274\\
125	-0.156364581206834\\
126	0.174764318052435\\
127	nan\\
128	-0.0445714235812645\\
129	0.0861094272178511\\
130	0.0893130000150174\\
131	0.0814090499854394\\
132	0.132763938887605\\
133	-0.14753273975168\\
134	0.0254980440968309\\
135	-0.0634148125882246\\
136	0.0530169273621083\\
137	-0.0334542731792991\\
138	-0.110635913862955\\
139	0.055031731345818\\
140	-0.0213467072359347\\
141	-0.0101921442935641\\
142	-0.0104854653610358\\
143	-0.049982870049771\\
144	0.0162122832847615\\
145	0.000323636202974383\\
146	-0.113183707091048\\
147	0.0072187533983602\\
148	0.0369688155256217\\
149	0.036977467411456\\
150	0.0730615040923931\\
151	-0.0667703261663191\\
152	0.0251262446013779\\
153	0.00216973636669341\\
154	-0.0482272472139541\\
155	-0.0868730582214768\\
156	0.0653355503406741\\
157	-0.0383234539760709\\
158	-0.00108468463711581\\
159	-0.0541794247057601\\
160	0.0314131652349796\\
161	0.0504261119636444\\
162	0.00436060118335924\\
163	0.0270640076812079\\
164	-0.0210071368614834\\
165	-0.118898264773489\\
166	-0.0263563541011511\\
167	0.00696478598077608\\
168	-0.017189057992433\\
169	-0.0334184527152792\\
170	0.0339692809025743\\
171	-0.0465058854921497\\
172	-0.0402170466475977\\
173	0.0585487766266136\\
174	-0.0313781133397876\\
175	-0.121257760005896\\
176	-0.0339090820768931\\
177	0.00313004440230345\\
178	-0.066609273727825\\
179	-0.0468718897081484\\
180	-0.135170096099848\\
181	-0.255767981268249\\
182	-0.121621298945692\\
183	-0.173984423017701\\
184	-0.0605610390915512\\
185	-0.133077218584128\\
186	-0.0950356244983138\\
187	-0.0995823391090869\\
188	-0.0565236226010403\\
189	-0.0978886858219799\\
190	-0.156501853025451\\
191	-0.168322739663517\\
192	-0.0379973254617043\\
193	-0.0696391560601512\\
194	0.00455896402856274\\
195	0.0315317519874283\\
196	0.0772161557106037\\
197	-0.178778529657615\\
198	-0.0226173591654247\\
199	-0.103955509839379\\
200	-0.0382022970515123\\
201	-0.0531888661737467\\
202	-0.0067638461231929\\
203	-0.0388717218347404\\
204	-0.014229578474209\\
205	-0.00318179693005503\\
206	0.0594990759366699\\
207	0.0379238343155247\\
208	-0.0260218004275441\\
209	0.0175691806121162\\
210	-0.144790476407852\\
211	-0.00355621418883807\\
212	0.0583664576522994\\
213	0.0293825015424531\\
214	0.0464850777956799\\
215	0.0357380968930529\\
216	-0.0873788472858223\\
217	-0.0655897458475655\\
218	0.0247845237125166\\
219	-0.0142226174354622\\
220	0.0911153403224589\\
221	-0.111084180450963\\
222	0.123595853265444\\
223	0.0493089697404575\\
224	-0.0659406720165051\\
225	-0.01844582654628\\
226	0.0283824539060457\\
227	0.0425695384813175\\
228	0.0722915230285756\\
229	0.0504789694652664\\
230	0.0148724624634424\\
231	nan\\
232	-0.0608119860700085\\
233	0.0908345234576338\\
234	0.104946511449879\\
235	-0.0191603642490593\\
236	0.0830904007390058\\
237	-0.0133266661397632\\
238	0.0267752435700931\\
239	0.00878429212249776\\
240	-0.113693058333204\\
241	0.0464327740851524\\
242	-0.0545602570180026\\
243	0.0534725065018719\\
244	0.096906976517188\\
245	-0.154088322082056\\
246	0.0296050598383107\\
247	0.0750937502002928\\
248	nan\\
249	-0.00413260491543337\\
250	0.00108549800118402\\
251	0.0315328096198634\\
252	-0.0105493936908908\\
};
\addplot [color=dscr_nFPC_plot_color,solid,line width=1.5pt,forget plot]
  table[row sep=crcr]{%
1	0.307285424571827\\
2	0.196668817360928\\
3	0.235311054721116\\
4	0.184743829233353\\
5	0.209543650440518\\
6	0.128661231937815\\
7	0.211165829849492\\
8	0.126929036072532\\
9	0.176560187720251\\
10	0.196132126133114\\
11	0.21274191293403\\
12	0.174456184898567\\
13	0.572703712979235\\
14	0.322096582835897\\
15	0.184456548359416\\
16	0.171077512791466\\
17	0.123016581173625\\
18	0.240855607656172\\
19	0.223617221978723\\
20	0.092311525105119\\
21	0.151543712443178\\
22	0.0375241122849847\\
23	-0.0249626725291844\\
24	0.128432500374722\\
25	0.162439704142357\\
26	0.273526246000629\\
27	0.115437721821743\\
28	0.10513599732059\\
29	0.116937515160046\\
30	0.0535358982026782\\
31	0.0995758865662426\\
32	0.158641622242691\\
33	0.2456121908829\\
34	0.010453101695058\\
35	0.0866437001083825\\
36	0.0149861290420329\\
37	0.00740214696205961\\
38	0.128350439966792\\
39	0.185681358259925\\
40	0.287963410717822\\
41	0.179869089783602\\
42	0.293066608658142\\
43	0.128591434798244\\
44	0.2418772223055\\
45	0.142367074798148\\
46	0.165228010952215\\
47	0.157391745062771\\
48	0.14894638088329\\
49	0.0693477211394827\\
50	0.223055388648566\\
51	0.151292371649033\\
52	0.17714454462719\\
53	0.253825101458325\\
54	0.204814805173776\\
55	0.458250959253053\\
56	0.117121262800926\\
57	0.0751484741075734\\
58	0.169449344897272\\
59	0.359516561671493\\
60	0.248653564181917\\
61	0.193505802777672\\
62	0.198862693718709\\
63	0.148237051263375\\
64	0.220668635859849\\
65	0.334512495534136\\
66	0.157366619147144\\
67	0.106587072926027\\
68	0.383270727564274\\
69	0.184577521804748\\
70	0.0947805709273705\\
71	0.185044179569012\\
72	0.229267863603023\\
73	-0.0225791628539177\\
74	0.418248195298535\\
75	0.354985403548023\\
76	0.168185968911444\\
77	0.122753302189851\\
78	0.272541556765899\\
79	0.14686626464717\\
80	0.133695611235848\\
81	0.164791423363682\\
82	0.455604427542813\\
83	0.14690699834231\\
84	0.263048173024092\\
85	0.089409489762996\\
86	0.138639283744544\\
87	0.0988079958750348\\
88	0.164863828094923\\
89	0.292372170096296\\
90	0.142040274486306\\
91	0.0235983185342222\\
92	0.0566750373676545\\
93	0.0905864320161909\\
94	0.0607362801469435\\
95	0.131528118257272\\
96	0.102549382523037\\
97	0.200739353469501\\
98	0.235484344276673\\
99	0.252411325176238\\
100	0.11016282248878\\
101	0.095908983062567\\
102	0.123338851585889\\
103	0.0591781074202609\\
104	0.0388251493755226\\
105	0.464479776088965\\
106	0.153899403747447\\
107	0.162247133896345\\
108	0.313494898733717\\
109	0.183344363464974\\
110	0.215328929100844\\
111	0.0909984350513846\\
112	0.159267122409394\\
113	-0.0316582002231229\\
114	0.23093136426872\\
115	0.185409868378807\\
116	0.128344857874948\\
117	0.117662961563677\\
118	-0.0124790467832456\\
119	0.116807511327024\\
120	0.349616930350628\\
121	0.0984807943232725\\
122	0.155844982833761\\
123	0.280223838905043\\
124	0.300210581070049\\
125	0.0146208958647383\\
126	0.20932916520657\\
127	nan\\
128	0.199103491677951\\
129	0.120638911289335\\
130	0.258851138344847\\
131	0.140852647475561\\
132	0.190832634455336\\
133	0.0630070295587054\\
134	0.154781032796507\\
135	0.119844192414237\\
136	0.153248149228531\\
137	0.11548395888323\\
138	0.0536905330868437\\
139	0.0636464934007989\\
140	0.105827621322495\\
141	0.0799214990589647\\
142	0.103888019718052\\
143	0.0512625513101907\\
144	0.154113577374722\\
145	0.092318523498236\\
146	0.121029426515441\\
147	0.0901377859971579\\
148	0.0699504874455004\\
149	0.0348783431840218\\
150	0.166420248066132\\
151	0.0746463166148519\\
152	0.0912031998905761\\
153	0.0320691628387041\\
154	0.0228924044609168\\
155	0.0725241439741116\\
156	0.0796851476670739\\
157	0.158247387447405\\
158	0.0562948211631204\\
159	0.134534455960333\\
160	0.0842106066081413\\
161	0.0967467938158669\\
162	0.0660858574302106\\
163	0.0245804268173264\\
164	0.168232955722048\\
165	0.0850135652308796\\
166	0.0611008179592187\\
167	0.091185195989006\\
168	0.121902595724408\\
169	0.103652592552605\\
170	0.0298343021929557\\
171	0.0758839803205474\\
172	0.0967485937809668\\
173	0.0438856643312103\\
174	0.0969598726272626\\
175	0.0399342095009177\\
176	0.204225560407387\\
177	0.0493607021819654\\
178	0.0377742266476312\\
179	0.14099209272245\\
180	0.231308889608903\\
181	0.270558867329969\\
182	0.119964151532389\\
183	0.0937556819824004\\
184	0.081688611726089\\
185	0.063556976289782\\
186	0.0696681812845726\\
187	0.0261314132138828\\
188	0.195335402696224\\
189	0.104742809892857\\
190	0.0616857058362728\\
191	-0.0237057342954233\\
192	0.166182104706984\\
193	0.15416037199674\\
194	0.140790567550672\\
195	0.146180716896118\\
196	0.358567793747239\\
197	0.159105047433635\\
198	0.128428148945464\\
199	0.0483323438423279\\
200	0.185871189266475\\
201	0.121738494402955\\
202	0.101683570926425\\
203	0.16041813005798\\
204	0.134062442660448\\
205	0.0812758252239515\\
206	0.203226409153398\\
207	0.124056123826672\\
208	0.0972582039864547\\
209	0.186347102272649\\
210	0.0516254980297903\\
211	0.139205506868576\\
212	0.0892179945971371\\
213	0.0837651650602831\\
214	0.0481244793387642\\
215	0.282217433996174\\
216	0.028732311418992\\
217	0.0665234401348808\\
218	0.0685391920627127\\
219	0.0212470760301315\\
220	0.0552896147186778\\
221	0.131991653633165\\
222	0.0276060105381962\\
223	0.0673788238363792\\
224	0.112900847015324\\
225	0.05687194828695\\
226	0.0304754564341683\\
227	0.074048278512078\\
228	0.101169379633474\\
229	0.187547298111378\\
230	0.114790708437446\\
231	nan\\
232	0.0230624436458383\\
233	0.13865002676311\\
234	0.0887185122655012\\
235	0.0880199781819814\\
236	0.244309000173449\\
237	0.113573922025022\\
238	0.0230279663344548\\
239	0.137324963705612\\
240	0.0731413280680853\\
241	0.0885642099411079\\
242	0.108985810793237\\
243	0.0988798551165541\\
244	0.382457372710262\\
245	0.465094175049723\\
246	0.197487144557938\\
247	0.206654202096348\\
248	nan\\
249	0.151514847965492\\
250	0.127387638224365\\
251	0.0347612411880016\\
252	0.0943046211513907\\
};
\end{axis}
\end{tikzpicture}%

\end{subfigure}%
\hfill%
\begin{subfigure}{.45\linewidth}
  \centering
  \setlength\figureheight{\linewidth} 
  \setlength\figurewidth{\linewidth}
  \tikzsetnextfilename{IS_sameday_INTC}
  % This file was created by matlab2tikz.
%
%The latest updates can be retrieved from
%  http://www.mathworks.com/matlabcentral/fileexchange/22022-matlab2tikz-matlab2tikz
%where you can also make suggestions and rate matlab2tikz.
%
\definecolor{mycolor1}{rgb}{0.25098,0.00000,0.38824}%
\definecolor{mycolor2}{rgb}{0.00000,0.46275,0.00000}%
\definecolor{mycolor3}{rgb}{0.00000,0.34902,0.34902}%
\definecolor{mycolor4}{rgb}{0.58039,0.26275,0.00000}%
%
\begin{tikzpicture}[trim axis left, trim axis right]

\begin{axis}[%
width=\figurewidth,
height=\figureheight,
at={(0\figurewidth,0\figureheight)},
scale only axis,
every outer x axis line/.append style={black},
every x tick label/.append style={font=\color{black}},
xmin=1,
xmax=252,
%xlabel={Time (h)},
every outer y axis line/.append style={black},
every y tick label/.append style={font=\color{black}},
ymin=-1.1,
ymax=1.1,
%ylabel={Normalized PnL},
title={INTC},
axis background/.style={fill=white},
axis x line*=bottom,
axis y line*=left,
yticklabel style={
        /pgf/number format/fixed,
        /pgf/number format/precision=3
},
scaled y ticks=false,
]
\addplot [color=mycolor1,solid,line width=1.5pt,forget plot]
  table[row sep=crcr]{%
1	0.391589743909999\\
2	0.293083529682506\\
3	0.188933630622459\\
4	0.137562023861166\\
5	0.103150713225003\\
6	0.307578941910393\\
7	0.338294500475941\\
8	0.239104447513227\\
9	0.211827283956308\\
10	0.214517866655773\\
11	0.336718265752563\\
12	0.207215673870654\\
13	0.31850310303286\\
14	0.192046597654574\\
15	0.179448508958753\\
16	0.229834462662867\\
17	0.206629655940924\\
18	0.242404046396384\\
19	0.260557719208387\\
20	0.179553123507906\\
21	0.158273794341599\\
22	0.308850958221178\\
23	0.187928773474472\\
24	0.210443692503262\\
25	0.209761085362709\\
26	0.291570630073024\\
27	0.222689951228062\\
28	0.143289627521191\\
29	0.226480981385223\\
30	0.125171371227529\\
31	0.166586106296488\\
32	0.159410680897874\\
33	0.155251802870983\\
34	0.168237639261154\\
35	-0.0686538044199376\\
36	0.15770140032313\\
37	0.211210567515153\\
38	0.423660127674359\\
39	0.359910454100227\\
40	0.137236165283162\\
41	0.294336986097433\\
42	0.227326885180794\\
43	0.240341027609736\\
44	0.245485822371267\\
45	0.207293677312832\\
46	0.100200577045647\\
47	0.279989737706634\\
48	0.0928568278703446\\
49	0.173340720221563\\
50	0.182501881776688\\
51	0.132773709084778\\
52	0.216255834381277\\
53	0.249170078390131\\
54	0.178407444213397\\
55	0.260856351173351\\
56	0.251273752896795\\
57	0.260528389026425\\
58	0.448570971825313\\
59	0.34780586911256\\
60	0.121897622188909\\
61	0.157625766963898\\
62	0.173903993046632\\
63	0.118192786942004\\
64	0.282875234760071\\
65	0.271846713507021\\
66	0.245037388946468\\
67	0.296549738076209\\
68	0.408068413063436\\
69	0.363541161875419\\
70	0.173309066028106\\
71	0.226814147686051\\
72	0.472609891207332\\
73	0.342410509752256\\
74	0.456413095290415\\
75	0.444951932109454\\
76	0.385459576099748\\
77	0.453027499315521\\
78	0.452254308164672\\
79	0.0790376599986121\\
80	0.280051727140669\\
81	0.216799036380052\\
82	0.343739805470527\\
83	0.165918096320032\\
84	0.261102719099238\\
85	0.209991451932288\\
86	0.117624205199578\\
87	0.143599060028991\\
88	0.24884118900429\\
89	0.265238721313174\\
90	0.223775295905549\\
91	0.0499184605029373\\
92	0.0780738339099536\\
93	0.188837262009207\\
94	0.0927700468375558\\
95	0.214890774430686\\
96	0.219035329192683\\
97	0.27081400407282\\
98	0.210179507916276\\
99	0.20516652153545\\
100	0.276124930281051\\
101	0.219383027057866\\
102	0.357047990005264\\
103	0.180632457335695\\
104	0.21972784265787\\
105	0.302179316213804\\
106	0.241441667805397\\
107	0.279404438964498\\
108	0.233438674884992\\
109	0.357901302114471\\
110	0.361517583530466\\
111	0.155806280615884\\
112	0.219597870107508\\
113	0.379977264387118\\
114	0.248319341714596\\
115	0.297896826251599\\
116	0.220767379910672\\
117	0.265708522202168\\
118	0.229259386833002\\
119	0.28771867169118\\
120	0.133564535005963\\
121	0.286206389465077\\
122	0.218340325664506\\
123	0.301161029290859\\
124	0.486990711699025\\
125	0.142324905310972\\
126	0.215826086374088\\
127	nan\\
128	0.353981154708094\\
129	-0.0876606655766194\\
130	0.0897218957308472\\
131	0.126823732267754\\
132	0.445427602160508\\
133	0.194566159875542\\
134	0.256991519393359\\
135	0.242616077792208\\
136	0.111505500768326\\
137	0.366342534221969\\
138	0.224050507906716\\
139	0.0985754872832232\\
140	0.127024023511359\\
141	0.188686797285021\\
142	0.26359114586076\\
143	0.285506229463265\\
144	0.203257044360471\\
145	0.179764512445963\\
146	0.228174544655024\\
147	0.159415813173573\\
148	0.302615196957305\\
149	0.261725871424861\\
150	0.218591227381236\\
151	0.18113793898798\\
152	0.127977569845308\\
153	0.179001249139115\\
154	0.239555167804789\\
155	0.194784539785957\\
156	0.198471636396744\\
157	0.124605854266245\\
158	0.0589386829418371\\
159	0.230550759393515\\
160	0.236902700794932\\
161	-0.113058079575813\\
162	-0.212985845877257\\
163	0.208586353127076\\
164	0.128044538071642\\
165	0.325275299611676\\
166	0.215486122422006\\
167	0.142800169454835\\
168	0.222559479106779\\
169	0.297829803928806\\
170	0.18924898234552\\
171	0.193770509019651\\
172	0.236328431833781\\
173	0.288120799220058\\
174	0.126672227132185\\
175	0.136845165867239\\
176	0.113668634996973\\
177	0.462061822423768\\
178	0.0502408372055615\\
179	0.196139557246953\\
180	0.277954127277983\\
181	0.184504125597655\\
182	0.102103725066033\\
183	0.177909702961505\\
184	0.124227438696616\\
185	0.118452085691134\\
186	0.0806303006329809\\
187	0.124861207164269\\
188	0.132552169186704\\
189	0.137411816495894\\
190	0.153995132510856\\
191	0.227685285676196\\
192	0.241506758769529\\
193	0.199515612126498\\
194	0.188686139412486\\
195	0.228379602058814\\
196	0.237324333998516\\
197	0.300319125077605\\
198	0.281072139830337\\
199	0.276135399852095\\
200	0.253488793280004\\
201	0.285504271180965\\
202	0.171387822471255\\
203	0.27154615662202\\
204	0.215994096185234\\
205	0.118481807337922\\
206	0.168023083443109\\
207	0.248443651567233\\
208	0.326728341757265\\
209	0.221154986645938\\
210	0.133839353551871\\
211	0.136878419892747\\
212	0.100470278727643\\
213	0.265640288597569\\
214	0.131917664913351\\
215	0.18014086988717\\
216	0.177975179035934\\
217	0.205897906850193\\
218	0.0827645501757977\\
219	0.144541424398236\\
220	0.299431168272583\\
221	0.214330692316746\\
222	0.246778315161637\\
223	0.22449342903404\\
224	0.211913452007816\\
225	0.173767486513819\\
226	0.386812150512532\\
227	0.136299598668391\\
228	0.276795310724196\\
229	0.093957495086345\\
230	0.190155990631332\\
231	nan\\
232	0.13456243263902\\
233	0.266971665229883\\
234	0.262750616268233\\
235	0.347257025196628\\
236	0.242411287810825\\
237	0.157847882121487\\
238	0.161660624395257\\
239	0.00240282950344152\\
240	0.0977244784159328\\
241	0.08793921931565\\
242	0.0825696869284111\\
243	0.188293785863016\\
244	0.236325653540559\\
245	0.172480752113236\\
246	0.0157650550512185\\
247	0.19220692833729\\
248	nan\\
249	0.247991358901486\\
250	0.0546080130429217\\
251	0.228637802663637\\
252	0.136383000925083\\
};
\addplot [color=mycolor2,solid,line width=1.5pt,forget plot]
  table[row sep=crcr]{%
1	0.333768911677547\\
2	0.232727472512196\\
3	0.116003418925775\\
4	0.215428056281661\\
5	0.221651367048505\\
6	0.295577594094905\\
7	0.302622948454115\\
8	0.212614229441805\\
9	0.203897782621441\\
10	0.216710737439324\\
11	0.286118838314943\\
12	0.254478154886667\\
13	0.352074738164661\\
14	0.227737436907713\\
15	0.158455395450261\\
16	0.189556668605896\\
17	0.176027631670616\\
18	0.221673890089364\\
19	0.281797309611809\\
20	0.180142385389432\\
21	0.258122804274677\\
22	0.349488141899585\\
23	0.213851550819874\\
24	0.218444959937139\\
25	0.180183290415965\\
26	0.305180751825854\\
27	0.201939433280902\\
28	0.0920272419603295\\
29	0.179214109327316\\
30	0.112978938362586\\
31	0.155220863928893\\
32	0.152879316487404\\
33	0.124965794773272\\
34	0.178794410634309\\
35	0.226640116103109\\
36	0.160800171150487\\
37	0.11772465252331\\
38	0.383917476269602\\
39	0.295235919097192\\
40	0.173348804409072\\
41	0.33930547581791\\
42	0.282282878028601\\
43	0.305449256930086\\
44	0.227467800670768\\
45	0.256480418740308\\
46	0.169325811244331\\
47	0.308568373450275\\
48	0.168721701539817\\
49	0.150541367347649\\
50	0.209728663894474\\
51	0.149670481929516\\
52	0.20380574767565\\
53	0.248590605253363\\
54	0.191283231318807\\
55	0.262394776850264\\
56	0.18835938160362\\
57	0.144135022393152\\
58	0.458719122491559\\
59	0.328322654740615\\
60	0.129241793435649\\
61	0.148231240723555\\
62	0.153984723547677\\
63	0.176412025987317\\
64	0.256585142373645\\
65	0.296707290082554\\
66	0.216615736124191\\
67	0.391861267441821\\
68	0.539551082151974\\
69	0.381932499516413\\
70	0.19374053891556\\
71	0.189265385758965\\
72	0.510769133691676\\
73	0.4860766722571\\
74	0.434901837872652\\
75	0.364289203278551\\
76	0.382263698412221\\
77	0.421916862366552\\
78	0.470976858927345\\
79	0.232856649519314\\
80	0.310555098675914\\
81	0.223615540346806\\
82	0.395051098455704\\
83	0.325245896851892\\
84	0.273031040682037\\
85	0.272365011891521\\
86	0.164684847009387\\
87	0.156780790194564\\
88	0.282912733803722\\
89	0.166633515545989\\
90	0.173040849392663\\
91	0.12208622281088\\
92	0.0947661999218445\\
93	0.261955498874767\\
94	0.125055804156315\\
95	0.196422788020447\\
96	0.26160152179177\\
97	0.22893798390484\\
98	0.238335614964464\\
99	0.228498471249047\\
100	0.238334044277545\\
101	0.201928054118012\\
102	0.246041488795959\\
103	0.214430219738578\\
104	0.299664116995078\\
105	0.366473733680578\\
106	0.302133372992863\\
107	0.232177600536365\\
108	0.248595209240429\\
109	0.406193218599909\\
110	0.396003382499736\\
111	0.211418354595269\\
112	0.168549932837974\\
113	0.338452156382061\\
114	0.194493499240721\\
115	0.292511233319038\\
116	0.268471478622731\\
117	0.119209274157966\\
118	0.268405050549451\\
119	0.293597698373012\\
120	0.229391436912554\\
121	0.281351765324802\\
122	0.190421984864482\\
123	0.320019122184802\\
124	0.500882825729437\\
125	0.258168127608347\\
126	0.288642837129454\\
127	nan\\
128	0.296250080015567\\
129	0.247537697847365\\
130	0.100860578798212\\
131	0.204441407394484\\
132	0.611465214342782\\
133	0.359205774804975\\
134	0.24601898713452\\
135	0.241720110684394\\
136	0.224423730982619\\
137	0.305988730853051\\
138	0.243633638552224\\
139	0.11921399394659\\
140	0.156562062381879\\
141	0.134113042488412\\
142	0.266611715038007\\
143	0.253770583377862\\
144	0.193610712663937\\
145	0.111908954802266\\
146	0.12152313207422\\
147	0.182478773111015\\
148	0.275369655138293\\
149	0.246346687184395\\
150	0.229053550725161\\
151	0.182776466745665\\
152	0.160007290646792\\
153	0.20970990272689\\
154	0.246817649678624\\
155	0.195739910307888\\
156	0.229146372891214\\
157	0.180610373817526\\
158	0.116506756712066\\
159	0.271253408242694\\
160	0.232723622158607\\
161	0.0851512134271443\\
162	-0.21665706489368\\
163	0.206420233748082\\
164	0.178122696492527\\
165	0.345469368560939\\
166	0.205030777797029\\
167	0.144950239179236\\
168	0.320989347666679\\
169	0.231515837871226\\
170	0.23738881401316\\
171	0.199188252526374\\
172	0.256113712468798\\
173	0.239927922835006\\
174	0.155967877072355\\
175	0.144436324329898\\
176	0.0763658414985831\\
177	0.491003226431149\\
178	0.145318751398294\\
179	0.229928911782692\\
180	0.255313601647687\\
181	0.22302287025072\\
182	0.150244371017714\\
183	0.159384706676769\\
184	0.16842033604045\\
185	0.0693389467958679\\
186	0.216596099758733\\
187	0.17573546272982\\
188	0.253109541582453\\
189	0.154064031660957\\
190	0.213617156589859\\
191	0.32524689086249\\
192	0.18647972059609\\
193	0.17315070553096\\
194	0.208826256729927\\
195	0.275883432527285\\
196	0.236016421805217\\
197	0.289122125310656\\
198	0.236519644598414\\
199	0.23995119711274\\
200	0.316622693874422\\
201	0.334667899134426\\
202	0.190002958040152\\
203	0.259840042714512\\
204	0.203776192197029\\
205	0.166724697803691\\
206	0.15486690616195\\
207	0.216464628494421\\
208	0.34002218675414\\
209	0.269044858828246\\
210	0.160120053352544\\
211	0.16711692513657\\
212	0.0949436526703116\\
213	0.166791266125258\\
214	0.18123567302268\\
215	0.214317316756454\\
216	0.173972235563062\\
217	0.202964348332083\\
218	0.138156259099779\\
219	0.218869822815176\\
220	0.303252967796926\\
221	0.20734236891722\\
222	0.225295458148032\\
223	0.171392933134449\\
224	0.211171306349371\\
225	0.185110316986111\\
226	0.361321650227918\\
227	0.234325015260041\\
228	0.238319038825123\\
229	0.145888606827339\\
230	0.189702908840084\\
231	nan\\
232	0.141105032321184\\
233	0.235711900821984\\
234	0.265969117719761\\
235	0.351458488973552\\
236	0.276610963533824\\
237	0.163527875239468\\
238	0.199369637176612\\
239	0.136369455475376\\
240	0.190326157777735\\
241	0.157375171133857\\
242	0.118804195552336\\
243	0.205899398729245\\
244	0.290032432677362\\
245	0.235785658215783\\
246	0.170257244631804\\
247	0.223844571703125\\
248	nan\\
249	0.242059363952209\\
250	0.122399936418885\\
251	0.244697683538711\\
252	0.171282281300026\\
};
\addplot [color=mycolor3,solid,line width=1.5pt,forget plot]
  table[row sep=crcr]{%
1	0.208572316405276\\
2	0.14833928522974\\
3	0.112452439639978\\
4	0.266337300392211\\
5	0.0120323403425431\\
6	0.263507675247356\\
7	0.159542111451932\\
8	0.152597362358569\\
9	0.116236238787322\\
10	0.111906663812486\\
11	0.1311569919059\\
12	-0.0343539076140332\\
13	0.356571793766579\\
14	0.0627084160617568\\
15	0.212827954547176\\
16	0.222806229356876\\
17	0.0968225395783486\\
18	0.15451505562834\\
19	0.0793682455558322\\
20	0.0397822573314476\\
21	0.110762201260052\\
22	0.170844495596471\\
23	0.149005569580097\\
24	0.145570801763429\\
25	0.190640439888381\\
26	0.253043448393966\\
27	0.154751160429114\\
28	0.133345399236251\\
29	0.133746447931057\\
30	0.118200213093476\\
31	0.144103563707981\\
32	0.116215484747289\\
33	0.131497807212047\\
34	0.0996512135734099\\
35	-0.0549622985166351\\
36	0.0721231779860896\\
37	0.102196150637255\\
38	0.00287862113606428\\
39	0.0208494400265195\\
40	0.000749602455799581\\
41	0.18867244657714\\
42	0.184339612217394\\
43	0.207618866290416\\
44	0.212603832376325\\
45	0.113499174942538\\
46	0.0481654063884207\\
47	0.186856852491959\\
48	-0.0161360324000987\\
49	0.0892194608781958\\
50	0.186334719461304\\
51	0.0847005969954608\\
52	0.0702741138471873\\
53	0.138739645340319\\
54	0.138794243722763\\
55	0.211046661359078\\
56	0.163104894946067\\
57	0.0856356077735918\\
58	0.408394207216064\\
59	0.0914488368687861\\
60	0.0380296947566958\\
61	0.0294228998748581\\
62	0.166476024030666\\
63	0.257163473238519\\
64	0.135411156767448\\
65	0.259342077165369\\
66	0.210238592799974\\
67	0.0569762377215822\\
68	0.276963533820283\\
69	0.181804041040413\\
70	0.136633927347281\\
71	-0.0402782396901753\\
72	0.309016162919673\\
73	0.0302506439555826\\
74	0.138678739895202\\
75	0.307497593288483\\
76	0.157561932092823\\
77	0.18113060327347\\
78	0.314389029527146\\
79	0.00319218274424889\\
80	0.0646736506132777\\
81	0.0341700496948436\\
82	0.219765253663634\\
83	0.0168812213505443\\
84	0.00312959509183796\\
85	0.0881096242257643\\
86	0.0889515348774606\\
87	0.0578848377611993\\
88	0.151225591361362\\
89	0.0770278067728074\\
90	0.0873721787275192\\
91	0.0192649108371895\\
92	-0.0199207831327442\\
93	0.0660045344247142\\
94	-0.0900714839478612\\
95	0.0957358110783347\\
96	0.211435980662948\\
97	0.170553392991275\\
98	-0.147577545552589\\
99	0.0435593690473405\\
100	0.0499510936997321\\
101	0.0732619841892531\\
102	0.103389362818078\\
103	0.0805971442513251\\
104	0.0851609928902873\\
105	0.0427475961705217\\
106	-0.0695817687491329\\
107	0.100067474922637\\
108	-0.0234100108960775\\
109	0.115083647101147\\
110	0.0366651071495101\\
111	-0.00937704416345261\\
112	0.0673434751655782\\
113	0.137811286778276\\
114	0.172219534500304\\
115	0.0460793659112324\\
116	0.0744653430963214\\
117	-0.00771272225708336\\
118	0.103347606461473\\
119	0.0889482130516056\\
120	0.0330905249723488\\
121	0.279867457454173\\
122	0.118022330372072\\
123	0.144893113599767\\
124	0.271632379521963\\
125	0.123755428697926\\
126	0.245750514955849\\
127	nan\\
128	0.266692876001423\\
129	-0.244235681819987\\
130	0.0584428802819006\\
131	-0.0649785285550654\\
132	0.32357095219598\\
133	0.0950832914557975\\
134	0.191990251504753\\
135	0.0592509676938719\\
136	-0.12083280985335\\
137	0.331592897306785\\
138	0.195492325654146\\
139	0.0853091511933448\\
140	0.0372583421801789\\
141	0.0057495130501054\\
142	0.222978662986317\\
143	0.218238920865679\\
144	0.139552311059381\\
145	0.117306901470082\\
146	0.0720915058273515\\
147	0.158372843596323\\
148	0.174946820643204\\
149	0.232737890283557\\
150	0.204408361011538\\
151	0.14351041590253\\
152	0.100026915910576\\
153	0.107359358407358\\
154	0.220238984703566\\
155	0.11698559777673\\
156	0.141341483813451\\
157	0.00104025284951189\\
158	0.102813320260989\\
159	0.0242294312795385\\
160	0.128381336205801\\
161	0.0475288005886188\\
162	-0.218377892608336\\
163	0.17480745917009\\
164	0.168280444898345\\
165	0.161280497137455\\
166	0.0916941129866418\\
167	0.114502127716107\\
168	0.159315346414182\\
169	0.15133785175757\\
170	0.0694699544451333\\
171	0.141162050949331\\
172	0.0577416837540809\\
173	0.168479267232314\\
174	0.0929424426533464\\
175	0.0852990274274164\\
176	0.0776316049787917\\
177	0.415652016493405\\
178	0.031513855835861\\
179	0.105654741837782\\
180	0.135461668120968\\
181	0.144714562517296\\
182	0.0576025843059079\\
183	0.173600558636567\\
184	-0.00523260697827171\\
185	0.087154855783982\\
186	0.0662483112219375\\
187	-0.0071577759611377\\
188	0.0422000258068876\\
189	0.0382348798195439\\
190	0.151101356189055\\
191	0.18325449536047\\
192	0.125986542549519\\
193	0.0956149280239598\\
194	0.156143797706234\\
195	0.155043443224061\\
196	0.142346556131763\\
197	0.202670282531865\\
198	0.112836216107313\\
199	0.154964682787758\\
200	0.215752745741765\\
201	0.140914921856394\\
202	0.125518735750116\\
203	0.211421195194753\\
204	0.158945781151614\\
205	0.0603606236469995\\
206	0.0771360863332413\\
207	0.149773839856891\\
208	0.234330247526507\\
209	0.114414228458558\\
210	0.0291387543337991\\
211	0.235770065325781\\
212	0.0560025416422469\\
213	0.152738882932125\\
214	0.0460055754477534\\
215	0.135205809403947\\
216	0.109687425959324\\
217	0.0632596699055367\\
218	0.0150814666044855\\
219	-0.0174249104440113\\
220	0.108274618131137\\
221	0.163423865818873\\
222	0.21067852307861\\
223	0.186867085363917\\
224	0.155710650387992\\
225	0.0639343568153517\\
226	0.095177681638619\\
227	0.0239651780991623\\
228	0.264918260755787\\
229	0.0787859516178995\\
230	0.14036520374532\\
231	nan\\
232	0.127015693063667\\
233	0.180376110288811\\
234	0.194639865496199\\
235	0.0853837699158635\\
236	0.162784009425773\\
237	0.0952189101879141\\
238	0.241323865255618\\
239	0.0535040728906561\\
240	-0.0158618596943482\\
241	-0.00559769091562715\\
242	-0.0769981792985474\\
243	0.0614324396865506\\
244	0.0249844295403188\\
245	0.0681306089187595\\
246	0.0189679015240573\\
247	0.109719257209715\\
248	nan\\
249	0.216251968542567\\
250	0.00883984459256003\\
251	0.14948722704969\\
252	0.0751563761642884\\
};
\addplot [color=mycolor4,solid,line width=1.5pt,forget plot]
  table[row sep=crcr]{%
1	0.392998953875574\\
2	0.285368606498711\\
3	0.133775733676425\\
4	0.21359681628222\\
5	0.201399845217861\\
6	0.339070939052882\\
7	0.315383735662621\\
8	0.239336691481629\\
9	0.219384772430487\\
10	0.207565504343158\\
11	0.290470768193617\\
12	0.290423438654441\\
13	0.328122256717569\\
14	0.24587888372209\\
15	0.155476747066635\\
16	0.355657104331459\\
17	0.160744281254348\\
18	0.203565570229359\\
19	0.261012375652109\\
20	0.130010788902493\\
21	0.216354552062287\\
22	0.316665103893108\\
23	0.290370518183633\\
24	0.193269063478001\\
25	0.276208868963551\\
26	0.277597836948956\\
27	0.196126136400166\\
28	0.11349098926935\\
29	0.212652184598576\\
30	0.117545974831672\\
31	0.139699896592418\\
32	0.159847478042831\\
33	0.133223239455452\\
34	0.152081738806074\\
35	0.108932186949634\\
36	0.158272442957345\\
37	0.135098806472973\\
38	0.407475422927014\\
39	0.340596364007856\\
40	0.146528395458194\\
41	0.1234492870125\\
42	0.25654496666507\\
43	0.289233904646697\\
44	0.247735002262201\\
45	0.240336619229197\\
46	0.143454251629071\\
47	0.310608795783249\\
48	0.16908728329742\\
49	0.157904424351091\\
50	0.211968229501408\\
51	0.122311820944972\\
52	0.218908704779559\\
53	0.215069964389261\\
54	0.21363027007528\\
55	0.271944245736794\\
56	0.172953546312389\\
57	0.268530698047401\\
58	0.465879678575854\\
59	0.319761446367702\\
60	0.130171721579893\\
61	0.281651528310274\\
62	0.1514660856373\\
63	0.160234782667765\\
64	0.26024197970397\\
65	0.267193189518822\\
66	0.104453406427488\\
67	0.00925209259343471\\
68	0.602378205214577\\
69	0.348672128673846\\
70	0.167933010987475\\
71	0.202962381262112\\
72	0.564575867837199\\
73	0.526953874374029\\
74	0.483076804863291\\
75	0.40592573422906\\
76	0.450704573752827\\
77	0.478595425326174\\
78	0.524678804006999\\
79	0.160192946084838\\
80	0.330887311658275\\
81	0.224868763723493\\
82	0.402417161254912\\
83	0.262856350143555\\
84	0.298335414267989\\
85	0.186030426971146\\
86	0.218686413659908\\
87	0.161679122883166\\
88	0.298712411628303\\
89	0.187624791269267\\
90	0.210848849546735\\
91	0.111775903175298\\
92	0.0733470046105282\\
93	0.251997400631176\\
94	0.104140075851183\\
95	0.196334991050354\\
96	0.18226364843081\\
97	0.187900142499613\\
98	0.18585761354001\\
99	0.258802893195389\\
100	0.221021678152658\\
101	0.187889420231888\\
102	0.244225404564783\\
103	0.237233354539624\\
104	0.170818410167884\\
105	0.350748335622377\\
106	0.22076585341228\\
107	0.184624110460172\\
108	0.147493383880028\\
109	0.397213154052382\\
110	0.38050942841072\\
111	0.287599691147976\\
112	0.292371199387385\\
113	0.382625586115085\\
114	0.218038262062834\\
115	0.300346564068007\\
116	0.26624167279707\\
117	0.0951150743617846\\
118	0.236128349606966\\
119	0.280489454815599\\
120	0.127231324622025\\
121	0.280465625058357\\
122	0.173384669960189\\
123	0.29406996940685\\
124	0.135824863648377\\
125	0.15889807939581\\
126	0.219640188807317\\
127	nan\\
128	0.317359403967283\\
129	0.0867327775729016\\
130	0.081373840927681\\
131	0.21801929216247\\
132	0.650223143734301\\
133	0.300245621994336\\
134	0.128432197823438\\
135	0.257587675598922\\
136	0.186075432242773\\
137	0.371034112756539\\
138	0.186287192453459\\
139	0.113168531279885\\
140	0.121205323439328\\
141	0.115248498166329\\
142	0.287469105525996\\
143	0.260426548245732\\
144	0.198388642410851\\
145	0.143764244194818\\
146	0.19610083870649\\
147	0.133399046766265\\
148	0.281447435665594\\
149	0.226387144400128\\
150	0.251970447565208\\
151	0.17177546502284\\
152	0.157911648767173\\
153	0.216096103343097\\
154	0.257813156331301\\
155	0.271761369679223\\
156	0.217751631781759\\
157	0.105244542283106\\
158	0.102721124453486\\
159	0.291332892545082\\
160	0.259070393594845\\
161	0.0533171627521759\\
162	-0.220117010168094\\
163	0.189805774799252\\
164	0.166892333854392\\
165	0.319451689476314\\
166	0.203190135261707\\
167	0.133604856769013\\
168	0.376788700006355\\
169	0.311636027059345\\
170	0.247160964093518\\
171	0.173704092384282\\
172	0.31883034771726\\
173	0.248310717016367\\
174	0.179210605427053\\
175	0.124172132089872\\
176	0.0937191585444825\\
177	0.520066513706459\\
178	0.105735051500734\\
179	0.209705603744315\\
180	0.248620509281717\\
181	0.184004216722346\\
182	0.121080689274207\\
183	0.266215142127721\\
184	0.131353850513838\\
185	0.13004822913655\\
186	0.163066711670286\\
187	0.249120897411268\\
188	0.19510464035822\\
189	0.137426930058441\\
190	0.191562299748405\\
191	0.313463706996225\\
192	0.213103100447217\\
193	0.197744270395187\\
194	0.336783272859535\\
195	0.240875614514517\\
196	0.267654938880436\\
197	0.331018939960578\\
198	0.256115820408796\\
199	0.227763698143317\\
200	0.27059480946856\\
201	0.313451252411457\\
202	0.150141782483742\\
203	0.159315860196636\\
204	0.173230968063198\\
205	0.135937989592514\\
206	0.15667652721982\\
207	0.21466150600947\\
208	0.330313347043268\\
209	0.285985317619785\\
210	0.162361064419463\\
211	0.14191921846279\\
212	0.0919663256607726\\
213	0.16012823404491\\
214	0.191388696629016\\
215	0.216777385045652\\
216	0.252034413475323\\
217	0.186593841718372\\
218	0.130670628415025\\
219	0.208120387485163\\
220	0.308974178041681\\
221	0.19548874231866\\
222	0.151322423834585\\
223	0.168600042310865\\
224	0.221830038238226\\
225	0.212631475241632\\
226	0.423970353904841\\
227	0.14525549721858\\
228	0.28781921905292\\
229	0.144905861318691\\
230	0.217448528161204\\
231	nan\\
232	0.30634138144378\\
233	0.206408544757044\\
234	0.271443889056105\\
235	0.374295618746448\\
236	0.20200713095419\\
237	0.136261154465944\\
238	0.200976750226435\\
239	0.0632290851564593\\
240	0.214059118605928\\
241	0.14703172020519\\
242	0.115243106781836\\
243	0.170407332585463\\
244	0.293577089435767\\
245	0.200069357471978\\
246	0.112351472199517\\
247	0.224409256551706\\
248	nan\\
249	0.238828167014954\\
250	0.12115279753739\\
251	0.253506331650686\\
252	0.172527085171514\\
};
\end{axis}
\end{tikzpicture}%
 
\end{subfigure}\\

\leavevmode\smash{\makebox[0pt]{\hspace{-7em}% HORIZONTAL POSITION           
  \rotatebox[origin=l]{90}{\hspace{20em}% VERTICAL POSITION
    Normalized PnL}%
}}\hspace{0pt plus 1filll}\null

Trading Day Number of 2013

\vspace{1cm}
\begin{subfigure}{\linewidth}
  %\centering
  \setlength\figureheight{\linewidth} 
  \setlength\figurewidth{\linewidth}
  \tikzsetnextfilename{strategylegend}
  \resizebox{\linewidth}{!}{\definecolor{mycolor1}{rgb}{0.25098,0.00000,0.38824}%
\definecolor{mycolor2}{rgb}{0.00000,0.46275,0.00000}%
\definecolor{mycolor3}{rgb}{0.00000,0.34902,0.34902}%
\definecolor{mycolor4}{rgb}{0.58039,0.26275,0.00000}%
\begin{tikzpicture}
    \begingroup
    % inits/clears the lists (which might be populated from previous
    % axes):
    \csname pgfplots@init@cleared@structures\endcsname
    \pgfplotsset{legend style={at={(0,1)},anchor=north west},legend columns=-1,legend style={draw=black,column sep=1ex},
            legend entries={Cts Stoch Ctrl,Dscr Stoch Ctrl,Cts Stoch Ctrl w nFPC,Dscr Stoch Ctrl w nFPC}}%
    
    \csname pgfplots@addlegendimage\endcsname{line width=2pt,mycolor1,sharp plot}
    \csname pgfplots@addlegendimage\endcsname{line width=2pt,mycolor2,sharp plot}
    \csname pgfplots@addlegendimage\endcsname{line width=2pt,mycolor3,sharp plot}
    \csname pgfplots@addlegendimage\endcsname{line width=2pt,mycolor4,sharp plot}

    % draws the legend:
    \csname pgfplots@createlegend\endcsname
    \endgroup
\end{tikzpicture}
}
\end{subfigure}%
  \caption{End of day strategy performances: in-sample backtesting using same-day calibration.}
  \label{fig:IS_sameday_comp}
\end{figure}

\begin{table}
\centering
\ra{1.2}
\begin{tabular}{@{} *{8}{r} @{}}
\toprule
Strategy & Return & Sharpe & Trades & Inv & \% Win & Max Loss & Max Win \\
\midrule
\multicolumn{8}{l}{\texttt{FARO}} \\
Naive & -0.879 & -0.808 & 413 & 0.47 & 0.07 & -7.109 & 5.715 \\ 
Naive+ & 0.101 & 0.107 & 213 & 2.45 & 0.74 & -8.797 & 5.336 \\ 
Naive++ & 0.002 & 0.021 & 7 & 0.17 & 0.50 & -0.842 & 0.320 \\ 
Cont & -0.059 & -0.551 & 201 & 0.09 & 0.18 & -0.912 & 0.071 \\ 
Dscr & -0.064 & -0.695 & 210 & -0.02 & 0.08 & -0.914 & 0.440 \\ 
Cont w nFPC & -0.063 & -0.571 & 204 & 0.08 & 0.14 & -1.161 & 0.077 \\ 
Dscr w nFPC & -0.060 & -0.662 & 209 & -0.03 & 0.09 & -0.716 & 0.539 \\[2ex]
\multicolumn{8}{l}{\texttt{NTAP}} \\
Naive & -0.188 & -0.316 & 842 & -9.81 & 0.23 & -3.238 & 3.524 \\ 
Naive+ & 0.388 & 0.169 & 3562 & -9.73 & 0.74 & -19.367 & 10.201 \\ 
Naive++ & -0.005 & -0.012 & 157 & -0.90 & 0.54 & -2.888 & 2.558 \\ 
Cont & -0.006 & -0.062 & 2265 & 0.40 & 0.56 & -0.441 & 0.215 \\ 
Dscr & 0.099 & 0.767 & 1872 & 4.74 & 0.86 & -0.126 & 1.042 \\ 
Cont w nFPC & -0.141 & -0.951 & 2897 & 0.65 & 0.14 & -0.935 & 0.244 \\ 
Dscr w nFPC & 0.121 & 0.881 & 1738 & 2.82 & 0.89 & -0.139 & 0.962 \\[2ex]
\multicolumn{8}{l}{\texttt{ORCL}} \\
Naive & -0.105 & -0.253 & 484 & 1.40 & 0.28 & -1.837 & 2.180 \\ 
Naive+ & -0.034 & -0.011 & 4086 & -55.18 & 0.61 & -17.501 & 18.400 \\ 
Naive++ & 0.002 & 0.006 & 132 & 0.61 & 0.52 & -0.798 & 2.636 \\ 
Cont & 0.115 & 1.348 & 1874 & 1.94 & 0.92 & -0.217 & 0.521 \\ 
Dscr & 0.135 & 1.620 & 1898 & 3.93 & 0.98 & -0.063 & 0.515 \\ 
Cont w nFPC & -0.010 & -0.100 & 2455 & 1.32 & 0.48 & -0.478 & 0.503 \\ 
Dscr w nFPC & 0.144 & 1.501 & 1759 & 2.85 & 0.97 & -0.032 & 0.573 \\[2ex]
\multicolumn{8}{l}{\texttt{INTC}} \\
Naive & -0.082 & -0.228 & 258 & -5.21 & 0.33 & -1.465 & 1.425 \\ 
Naive+ & 0.365 & 0.134 & 3962 & -32.50 & 0.63 & -11.202 & 11.669 \\ 
Naive++ & -0.001 & -0.003 & 74 & -0.84 & 0.48 & -1.314 & 1.264 \\ 
Cont & 0.214 & 2.159 & 1577 & 5.17 & 0.97 & -0.213 & 0.487 \\ 
Dscr & 0.232 & 2.528 & 1642 & 4.48 & 0.98 & -0.217 & 0.611 \\ 
Cont w nFPC & 0.114 & 1.218 & 1894 & 2.01 & 0.90 & -0.244 & 0.416 \\ 
Dscr w nFPC &  0.226 & 2.202 & 1569 & 4.28 & 0.98 & -0.220 & 0.650 \\ 
\bottomrule
\end{tabular}
\caption{Averaged strategy performance results: in-sample backtesting using same-day calibration.}
\label{tbl:IS_sameday}
\end{table}

\FloatBarrier
\subsection{Week Offset Calibration}
The next type of in-sample backtesting done was to calibrate for each ticker and each trading day of 2013, and to use the results to backtest on the date given by the calibration date shifted forward 7 days. Thus, the calibration obtained on Monday, January 2, 2013 would be used to backtest on Monday, January 9, 2013. Performance values are given in \autoref{tbl:IS_week}, and \autoref{fig:IS_week_comp} compares the day-over-day performance of the various strategies. 

Most of the observations from the previous section apply here. Chiefly, the illiquid stock \texttt{FARO} produces negative PnL and the low-liquidity stock \texttt{NTAP} approximately breaks even. As expected, the week offset calibration underperforms same-day calibration, but remarkably the difference is very small: in the case of \texttt{INTC}, the discrete time controller still generates a Sharpe ratio of approximately 2.5, and in this case only returned negative PnL once during the trading year. 

The similarity of the results can be interpreted in several ways. First, it is possible that trading behaviour is stable across days of the week, such that substituting one Monday for another yields a similar calibration. This is readily testable by calibrating on a given day and backtesting on the subsequent trading day, instead of a one-week offset. On the other hand, even with dissimilar data, it's possible that the calculation of $\delta^\pm$ is stable with respect to day-over-day fluctuations of data.

\begin{figure}
\centering
\begin{subfigure}{.45\linewidth}
  \centering
  \setlength\figureheight{\linewidth} 
  \setlength\figurewidth{\linewidth}
  \tikzsetnextfilename{IS_week_FARO}
  % This file was created by matlab2tikz.
%
%The latest updates can be retrieved from
%  http://www.mathworks.com/matlabcentral/fileexchange/22022-matlab2tikz-matlab2tikz
%where you can also make suggestions and rate matlab2tikz.
%
%
\begin{tikzpicture}[trim axis left, trim axis right]

\begin{axis}[%
width=\figurewidth,
height=\figureheight,
at={(0\figurewidth,0\figureheight)},
scale only axis,
every outer x axis line/.append style={black},
every x tick label/.append style={font=\color{black}},
xmin=1,
xmax=240,
%xlabel={Time (h)},
every outer y axis line/.append style={black},
every y tick label/.append style={font=\color{black}},
ymin=-1.1,
ymax=1.1,
%ylabel={Normalized PnL},
title={FARO},
axis background/.style={fill=white},
axis x line*=bottom,
axis y line*=left,
yticklabel style={
        /pgf/number format/fixed,
        /pgf/number format/precision=3
},
scaled y ticks=false,
legend style={legend cell align=left,align=left,draw=black,font=\small, legend pos=north west},
every axis legend/.code={\renewcommand\addlegendentry[2][]{}}  %ignore legend locally
]
\addplot [color=cts_plot_color,solid,line width=1.5pt]
  table[row sep=crcr]{%
1	-0.0989928992435463\\
2	-0.0688180757696504\\
3	-0.667895329957383\\
4	-0.0630984607302803\\
5	-0.0804420804978697\\
6	-0.0238517422661603\\
7	-0.0638804461775067\\
8	-0.0912660237796769\\
9	-0.0490205942512218\\
10	-0.0252396718205314\\
11	-0.0360152481078483\\
12	-0.117630685374556\\
13	-0.00905645976076215\\
14	0.0305930218094014\\
15	0.00454532675490999\\
16	-0.0474566329727204\\
17	-0.0277740698587635\\
18	0.0104523122719218\\
19	-0.0471103202144804\\
20	-0.0466657064669461\\
21	-0.0379092006431071\\
22	-0.0962090318821978\\
23	-0.0667128402210335\\
24	0.00458328159485368\\
25	-0.0340792833717887\\
26	-0.0524606181198134\\
27	-0.101686333308474\\
28	-0.0930262650903495\\
29	-0.038440417353103\\
30	0.00646484296662299\\
31	0.0236287464914939\\
32	-0.106015086663405\\
33	-0.954925663307111\\
34	-0.393668338263199\\
35	-0.352469294096596\\
36	0.0309633242908584\\
37	-0.132497830787876\\
38	-0.0872340928201544\\
39	-0.0283111451658453\\
40	0.0118032806624083\\
41	-0.116258049308536\\
42	-0.097156710139102\\
43	-0.0129087881285822\\
44	-0.0396811580752417\\
45	0.0384162219195533\\
46	-0.0932070169147991\\
47	-0.0301884358829766\\
48	-0.061222235792448\\
49	-0.0703213827773295\\
50	-0.0246499622256796\\
51	0.0137850210254537\\
52	-0.00719359648918848\\
53	-0.0999153985217064\\
54	-0.0693644053212767\\
55	-0.0233426349342\\
56	-0.116730392046449\\
57	-0.0482703093536527\\
58	-0.0741330791335301\\
59	-0.0218726798057195\\
60	-0.00162410122724674\\
61	0.0331777422327288\\
62	0.00363315681677933\\
63	-0.155468008244286\\
64	-0.000435368998228069\\
65	-0.0261484374669145\\
66	-0.102217855734003\\
67	-0.0227363366769885\\
68	-0.133455366021963\\
69	-0.0269378322323864\\
70	-0.0195586074005721\\
71	0.0761346958181762\\
72	-0.0590198958472396\\
73	0.0767273777479881\\
74	-0.0432146734715059\\
75	-0.271416961782643\\
76	-0.0377477696742255\\
77	-0.0415751562377729\\
78	-0.0114566831881186\\
79	-0.172023901074782\\
80	-0.335974961081289\\
81	-0.0892871783058809\\
82	0.0538948598614626\\
83	-0.0365041983473961\\
84	-0.032151216144696\\
85	-0.118950280616404\\
86	-0.027891574310685\\
87	-0.130260925782951\\
88	-0.137685456840875\\
89	-0.0940534331095446\\
90	-0.211819331162252\\
91	-0.0483104244035162\\
92	-0.0168887195686571\\
93	-0.0190066197018946\\
94	-0.00888131484237648\\
95	-0.0640431796960462\\
96	-0.124070638642356\\
97	-0.0253571972739885\\
98	-0.0154399222921771\\
99	-0.189194549464084\\
100	-0.140447138688782\\
101	-0.0982887168231367\\
102	-0.00717265561770279\\
103	-0.0361224416123228\\
104	-0.168915984467289\\
105	-0.0495988604924424\\
106	0.0546528494487437\\
107	-0.012350967857789\\
108	-0.168601836127504\\
109	-0.0897402428653327\\
110	-0.0672065239533574\\
111	-0.0663273135132303\\
112	0.0247811747852366\\
113	-0.0828194756305478\\
114	0.0245182018522123\\
115	0.0397933168347158\\
116	0.062974447202189\\
117	-0.0271910791886851\\
118	-0.127458547111586\\
119	-0.05678167212853\\
120	0.0131570838509638\\
121	-0.0549489467788073\\
122	0.0129310344827583\\
123	-0.0323243914044978\\
124	-0.0543232834024399\\
125	-0.070463814773802\\
126	-0.0191957261505864\\
127	-0.00260670643491442\\
128	-0.0320702661926855\\
129	0.0063671403302416\\
130	-0.0113211497883461\\
131	-0.0367713400427507\\
132	-0.0594025300019686\\
133	-0.0642285458218825\\
134	-0.087358436414643\\
135	-0.0647255398601378\\
136	-0.445433279612283\\
137	-0.0385032218541256\\
138	0.00260985539699599\\
139	-0.0373202341780704\\
140	-0.00421998109767481\\
141	-0.0621682396887828\\
142	-0.0633737754817796\\
143	-0.0265361314481719\\
144	-0.0119949039905129\\
145	-0.0247057280739619\\
146	0.00727289519933497\\
147	-0.0177507320566373\\
148	-0.00847097312293046\\
149	0.00596429862640786\\
150	-0.027882114658807\\
151	-0.0494771736983482\\
152	-0.0298202347050229\\
153	-0.0190028326438429\\
154	-0.0331645592293618\\
155	-0.0370638724578523\\
156	-0.00901903851536128\\
157	-0.00814933180142084\\
158	0.019944331897309\\
159	-0.0646841749373628\\
160	-0.0594321816054181\\
161	-0.00903343215215287\\
162	-0.0396062748802371\\
163	-0.0081865280186242\\
164	-0.016514668612421\\
165	-0.0174400832815655\\
166	-0.0118513266090668\\
167	-0.0144099633674281\\
168	0.00432651465202523\\
169	-0.151139081185435\\
170	-0.0302992601490747\\
171	-0.0979672622465193\\
172	-0.0942754486622191\\
173	-0.0182954943922059\\
174	-0.0199049988690337\\
175	-0.0227851169555195\\
176	0.00710088148873564\\
177	-0.107164761151541\\
178	-0.155143080609608\\
179	-0.21007149595768\\
180	0.0205934345070606\\
181	-0.0216214277949158\\
182	0.0337096204960389\\
183	-0.028289187238836\\
184	-0.0262910610830008\\
185	0.0004377233990683\\
186	-0.00109071305727747\\
187	-0.03023765742818\\
188	-0.0206927050990663\\
189	-0.0137475930099623\\
190	0.02665776778776\\
191	-0.0818645181863823\\
192	-0.0381153134578919\\
193	0.0287286464700551\\
194	0.0230054256027323\\
195	0.0194148115766118\\
196	-0.0407011609226884\\
197	0.00289020608165716\\
198	0.0375356378763503\\
199	-0.0738193540642478\\
200	-0.37155921794098\\
201	-0.442696758209487\\
202	-0.156069615014703\\
203	-0.0457175362190334\\
204	-0.129216531434065\\
205	-0.198209611029361\\
206	-0.0983047051566051\\
207	0.0675492324282333\\
208	-0.213569427969743\\
209	0.0187020759304281\\
210	0.0529215326435023\\
211	0.0145119890494427\\
212	-0.0621550140246297\\
213	-0.104098893949249\\
214	-0.00834901235061623\\
215	0.0376713349480656\\
216	0.0258067182602828\\
217	-0.012936941253008\\
218	-0.0670234402278327\\
219	-0.0870676977658279\\
220	-0.200004204721159\\
221	-0.0422636938307848\\
222	-0.0274776526235301\\
223	-0.0340701501302789\\
224	-0.0289366581669569\\
225	-0.0653807276375892\\
226	0.022921097403135\\
227	-0.023858132599337\\
228	0.0219377068609881\\
229	-0.0243773524026548\\
230	-0.179711176723372\\
231	-0.0682080981424351\\
232	-0.0299513217494404\\
233	0.0369695198259879\\
234	-0.0261613586243667\\
235	-0.0783588174899534\\
236	-0.125299794306574\\
237	-0.189118515394342\\
238	-0.245097122884676\\
239	-0.0588247368973214\\
240	0.00712136618536608\\
};
\addlegendentry{Cts Stoch Ctrl};

\addplot [color=dscr_plot_color,solid,line width=1.5pt]
  table[row sep=crcr]{%
1	-0.0995966207847469\\
2	-0.0955400522120556\\
3	-0.512177498035373\\
4	-0.0967261190740271\\
5	-0.211912683247003\\
6	-0.0338607079611349\\
7	-0.116836593809303\\
8	-0.0409623574512794\\
9	-0.0425852649052626\\
10	-0.00129863427129358\\
11	-0.0660318197644637\\
12	0.0178080601603774\\
13	0.00342127380272925\\
14	-0.0111056135298548\\
15	-0.00435331133027751\\
16	-0.0237677600366383\\
17	-0.0510455454596241\\
18	-0.0286322627025294\\
19	-0.061962302384862\\
20	-0.121955195592133\\
21	-0.0529510596123478\\
22	-0.0200861021004944\\
23	-0.0361574715984256\\
24	-0.0431417467617645\\
25	-0.0648805322537758\\
26	-0.0167552813269773\\
27	-0.17986021889352\\
28	-0.0789448859615183\\
29	-0.0748290530065737\\
30	-0.0077930343590699\\
31	-0.149178049209888\\
32	-0.198054736337988\\
33	-1.0038236725938\\
34	-0.262097390911982\\
35	-0.239828027358685\\
36	-0.0124696859560573\\
37	-0.226144936698071\\
38	-0.0735066525503359\\
39	-0.0700205658496726\\
40	-0.0222913865291784\\
41	-0.0589095689411235\\
42	-0.0716890864347401\\
43	-0.0270988990549223\\
44	-0.0649422858573774\\
45	-0.0200871083350892\\
46	-0.132927008950368\\
47	-0.0922801810570446\\
48	-0.140860871266921\\
49	-0.144633073706183\\
50	-0.0340617688762246\\
51	-0.0681894912495955\\
52	-0.0176941658313543\\
53	-0.0975432358447838\\
54	-0.0643685155820067\\
55	-0.0505340692711008\\
56	-0.0805893343847505\\
57	-0.0572397681277407\\
58	-0.0663337478075647\\
59	-0.0412542044959246\\
60	0.0586294576198538\\
61	-0.0407102358099261\\
62	-0.041223148810457\\
63	-0.110900958494676\\
64	-0.0218544140904249\\
65	-0.042424893896799\\
66	-0.0953951082001896\\
67	-0.0346788590652774\\
68	-0.0658652561021561\\
69	-0.0880310618608504\\
70	-0.0727069410057454\\
71	0.0120642370796955\\
72	-0.109167350631398\\
73	-0.0201956017728097\\
74	-0.103969236682258\\
75	-0.359600927095318\\
76	-0.0300705348679874\\
77	-0.097839934921383\\
78	-0.0302775675279979\\
79	-0.141431348090031\\
80	-0.247613824379903\\
81	-0.0326466026311607\\
82	-0.0125117058597183\\
83	-0.07965222658458\\
84	-0.0570558236770073\\
85	-0.0493518547689478\\
86	-0.0416719787334401\\
87	-0.0579109075868733\\
88	-0.0639626690665008\\
89	-0.113038765677644\\
90	-0.178784398205381\\
91	-0.108905080503451\\
92	-0.0328384692683799\\
93	-0.104763872903824\\
94	0.00191596037330389\\
95	-0.0888854622250821\\
96	-0.0683551412843746\\
97	-0.099180498794051\\
98	-0.0509256190405727\\
99	-0.15026328364107\\
100	-0.0742637153916431\\
101	-0.0516295715497802\\
102	0.0242770366344454\\
103	0.0839232096717357\\
104	-0.179769057501029\\
105	-0.0258275395712541\\
106	0.0424755855636558\\
107	0.00248663600060657\\
108	-0.0289275930632835\\
109	-0.00369395621058976\\
110	-0.106396466252579\\
111	-0.0982966626263425\\
112	-0.0219554961838324\\
113	-0.0717743345283962\\
114	0.00943020367554314\\
115	0.00722407034168151\\
116	0.0483846617685872\\
117	-0.0884769242981952\\
118	-0.184789645876706\\
119	-0.0647511812312117\\
120	0.0070545849437567\\
121	-0.0767806901249797\\
122	-0.0168354487167097\\
123	-0.0040624654056175\\
124	-0.018566212068315\\
125	-0.126989593539228\\
126	-0.0424639278394763\\
127	-0.0556742845038517\\
128	-0.0907343854584827\\
129	-0.0645519926408665\\
130	-0.0194634235792512\\
131	-0.0628063420991571\\
132	-0.0697900275459092\\
133	-0.119152212217465\\
134	-0.0658455382524425\\
135	-0.0691443833399853\\
136	-0.312455111232868\\
137	-0.0771505696756136\\
138	-0.0123734576878019\\
139	-0.0378580518425196\\
140	-0.022769687717831\\
141	-0.0376493770356853\\
142	-0.0481374827047047\\
143	-0.0531258653283602\\
144	-0.0133596073462276\\
145	-0.0191608129577023\\
146	-0.0877179476605288\\
147	-0.0204313519799799\\
148	-0.0434196171494139\\
149	-0.0408269411776303\\
150	-0.0249320375719999\\
151	-0.0387166586153994\\
152	-0.0377281723270208\\
153	-0.0418160700872973\\
154	-0.00210806397559851\\
155	-0.0833320090062677\\
156	-0.030795451904264\\
157	-0.0352537916958394\\
158	-0.0667973037356563\\
159	-0.105641535465567\\
160	-0.0616612955004078\\
161	-0.029680134509237\\
162	-0.0586846507048888\\
163	-0.0205017613844083\\
164	-0.0298111638411698\\
165	-0.0304415903616888\\
166	-0.0236533784645829\\
167	-0.0301779675613731\\
168	-0.0182722644251138\\
169	-0.104992252705746\\
170	-0.0498070818522068\\
171	-0.051990351056985\\
172	-0.0469572620315705\\
173	-0.0366764619292028\\
174	-0.0424399731627968\\
175	-0.049488136588733\\
176	-0.0170293407156964\\
177	-0.0742677827912091\\
178	-0.103310744403417\\
179	0.0623521135985585\\
180	-0.0637176018713269\\
181	0.00181226441765912\\
182	-0.0151405642783778\\
183	-0.112582882290347\\
184	-0.0710612531919435\\
185	-0.0826123217636468\\
186	-0.0341104834331987\\
187	-0.0624070115857106\\
188	-0.0445500207790483\\
189	-0.0286283388943652\\
190	-0.0198824777312811\\
191	-0.0520016571908172\\
192	-0.0776483417156276\\
193	-0.0231520235268523\\
194	-0.0981868548338814\\
195	-0.0908561904747453\\
196	-0.0439273688123807\\
197	-0.0614124806165277\\
198	-0.0287415819349967\\
199	-0.107678632634983\\
200	-0.388457313733346\\
201	-0.365425918679538\\
202	-0.247662127724195\\
203	-0.0886696315432309\\
204	-0.214749939117925\\
205	-0.289661531025451\\
206	-0.12464695015031\\
207	0.0710077160828214\\
208	-0.124546382275244\\
209	-0.088345351154684\\
210	-0.041134188195587\\
211	-0.0420258755838985\\
212	-0.051193064389229\\
213	-0.0945689302627227\\
214	-0.103157097432378\\
215	-0.0215759017161257\\
216	-0.0515873728442342\\
217	-0.0897554795563876\\
218	-0.0690889866451098\\
219	-0.188361031378949\\
220	-0.123656524933847\\
221	-0.0680015348507405\\
222	-0.0111379335705933\\
223	-0.0971871659578753\\
224	-0.0542251743875227\\
225	-0.0750576630075023\\
226	-0.030428502194894\\
227	-0.0943968501723815\\
228	-0.0368365474947934\\
229	-0.0329932161979989\\
230	-0.175537100210999\\
231	-0.089484283073257\\
232	-0.0608288964275909\\
233	-0.0094390721646589\\
234	-0.0552722142990639\\
235	-0.0993661819923027\\
236	-0.0519124321229057\\
237	-0.102529599538556\\
238	-0.541925451157354\\
239	-0.109779154154819\\
240	-0.00811388632014382\\
};
\addlegendentry{Dscr Stoch Ctrl};

\addplot [color=cts_nFPC_plot_color,solid,line width=1.5pt]
  table[row sep=crcr]{%
1	-0.0871127387607132\\
2	-0.0671621857356237\\
3	-0.632697555545204\\
4	-0.0694814356123973\\
5	-0.0691230412178578\\
6	-0.0422207460143579\\
7	-0.0648862465784592\\
8	-0.10752398450512\\
9	-0.0755606342417569\\
10	-0.0417598794169953\\
11	-0.0154112824444694\\
12	-0.117615331477479\\
13	-0.0125899945470681\\
14	0.0115857485707156\\
15	0.00454236419622189\\
16	-0.0534978348494885\\
17	-0.0243232067686003\\
18	0.00278687631675928\\
19	-0.0452104403771012\\
20	-0.0646148612474375\\
21	-0.046210052065509\\
22	-0.0706441967021038\\
23	-0.0526246175332645\\
24	-0.000945564443374598\\
25	-0.0228867026732935\\
26	-0.0663359752444718\\
27	-0.087656575963981\\
28	-0.127380328273489\\
29	-0.0452214790818072\\
30	0.00405322635364281\\
31	0.00052020722235808\\
32	-0.108281849441816\\
33	-1.02176778576647\\
34	-0.376114842190908\\
35	-0.349716633911133\\
36	0.0175171279812335\\
37	-0.158787510706624\\
38	-0.0712138361658776\\
39	-0.078729309131304\\
40	0.00826410467480504\\
41	-0.11846550144202\\
42	-0.074061894643235\\
43	4.91571086225418e-05\\
44	-0.0580695561279427\\
45	0.0427260606103892\\
46	-0.0944141348480918\\
47	-0.0344478691372178\\
48	-0.0648990002944799\\
49	-0.0710173787321302\\
50	-0.0401795786327721\\
51	0.0133345198721642\\
52	-0.0123745008368418\\
53	-0.0851652664658814\\
54	-0.0657593318290426\\
55	-0.0295875583738101\\
56	-0.103082843692256\\
57	-0.0574530279289594\\
58	-0.0924995990785913\\
59	-0.0279504966776894\\
60	0.0139020240926115\\
61	0.0139137337164068\\
62	-0.00880540136011325\\
63	-0.165126019098238\\
64	0.00198936235308465\\
65	-0.0331332867964299\\
66	-0.10490095539328\\
67	-0.0475032673729588\\
68	-0.122445239523586\\
69	-0.0580330203451716\\
70	-0.0258812339251961\\
71	0.0532461764032767\\
72	-0.0569133427249097\\
73	0.0715442764578818\\
74	-0.0462194251440889\\
75	-0.257988561755775\\
76	-0.0369674602733516\\
77	-0.0667766588369836\\
78	-0.0171477686121267\\
79	-0.179315557830554\\
80	-0.387902144236105\\
81	-0.161451745808743\\
82	0.0436958210296877\\
83	-0.0677623063157492\\
84	-0.0268104110002942\\
85	-0.105237717065019\\
86	-0.0575493104914281\\
87	-0.0921772900792722\\
88	-0.160883761681185\\
89	-0.102512785292145\\
90	-0.223971803290721\\
91	-0.0615615429632503\\
92	-0.0471462843051883\\
93	-0.0286134509961212\\
94	0.00288034434676267\\
95	-0.0596597759119077\\
96	-0.110177106957168\\
97	-0.0219724807973389\\
98	-0.00979946423816803\\
99	-0.21114771897182\\
100	-0.186166769397773\\
101	-0.0766812080816469\\
102	0.0365495334683822\\
103	-0.035989298634138\\
104	-0.143145743292324\\
105	-0.0395262721749439\\
106	0.024762949640771\\
107	-0.00759370235250323\\
108	-0.132806480492229\\
109	-0.0978651926587319\\
110	-0.0672445933706186\\
111	-0.0672526152871426\\
112	0.0187921810901593\\
113	-0.0526130948611329\\
114	0.0400986413741723\\
115	0.0325336500416856\\
116	0.0522228442249863\\
117	-0.0194571105392318\\
118	-0.125818245366238\\
119	-0.055880407828404\\
120	0.010470065009321\\
121	-0.0556567401080457\\
122	-0.00590207960683699\\
123	-0.0260406329063047\\
124	-0.0485734163122704\\
125	-0.0827993430414907\\
126	-0.0085292832861981\\
127	-0.0165455819391737\\
128	-0.0546306843589225\\
129	-0.0252488143973061\\
130	-0.0219398049907086\\
131	-0.0330672621956983\\
132	-0.05641001499249\\
133	-0.0832481636049818\\
134	-0.0900604090212291\\
135	-0.0751512839576212\\
136	-0.489802983209772\\
137	-0.0588776241875546\\
138	-0.00117824887415718\\
139	-0.0370864406634303\\
140	-0.0114669060927844\\
141	-0.040358701582808\\
142	-0.0378270859155574\\
143	-0.0317965137995542\\
144	-0.00678407873008303\\
145	-0.0356809437699124\\
146	0.00572698822291418\\
147	-0.0134644417909712\\
148	0.0298364096577485\\
149	0.000767762267181203\\
150	-0.0327513423351131\\
151	-0.0552578806504685\\
152	-0.0521586795031856\\
153	-0.0367150968072826\\
154	-0.0367272079292918\\
155	-0.0341657499663833\\
156	-0.00844300339264999\\
157	-0.0162302252213209\\
158	-0.00493963017913874\\
159	-0.0787881448713488\\
160	-0.0533532645279524\\
161	-0.00671417779905864\\
162	-0.0278451638859784\\
163	-0.00709899415224736\\
164	-0.0156208065681947\\
165	-0.0171413951156905\\
166	-0.0104742401795647\\
167	-0.0222141693343199\\
168	-0.00287483775355411\\
169	-0.146653230399974\\
170	-0.0374486723297366\\
171	-0.0878182495128086\\
172	-0.111458298114415\\
173	-0.0281096007281745\\
174	-0.0196866449606567\\
175	-0.0259362501514982\\
176	0.00563173359451441\\
177	-0.120868971920394\\
178	-0.154689618340419\\
179	-0.138947084341368\\
180	0.015821020206797\\
181	-0.00981265201609758\\
182	0.0190127351499457\\
183	-0.0227041517136506\\
184	-0.0116083020631266\\
185	-0.00751710074770835\\
186	-0.00203245901230065\\
187	-0.028802678906054\\
188	-0.0233346722978996\\
189	-0.0161760358572218\\
190	0.0213079510418383\\
191	-0.0794955196422548\\
192	-0.0509948621203456\\
193	0.00416952791979274\\
194	0.00729646389411873\\
195	0.0143697065852701\\
196	-0.0392807858441568\\
197	-0.0189923853623441\\
198	0.0341561861612771\\
199	-0.063761165469273\\
200	-0.50156655845565\\
201	-0.436797526921779\\
202	-0.175165257234287\\
203	-0.0648459591692155\\
204	-0.159333403348048\\
205	-0.252520595313619\\
206	-0.10174281652234\\
207	0.0705711178181963\\
208	-0.193326966545528\\
209	-0.00357441891311646\\
210	0.0164799555596676\\
211	-0.0151477664151083\\
212	-0.0556676484912089\\
213	-0.110004209554791\\
214	-0.0137605681154881\\
215	0.0359769954649446\\
216	0.0218087154165005\\
217	-0.0156747562798948\\
218	-0.0660532630932026\\
219	-0.0958202863965884\\
220	-0.191385226247613\\
221	-0.0509812988828071\\
222	-0.0171487457888097\\
223	-0.0571172512965657\\
224	-0.0209639093423886\\
225	-0.0757461600179559\\
226	-0.00235452846095781\\
227	-0.0181633609544181\\
228	0.0213935081117822\\
229	-0.0216707847904582\\
230	-0.169209394530033\\
231	-0.0624835127866373\\
232	-0.0333515408441886\\
233	0.0280423280423287\\
234	-0.00563498784643804\\
235	-0.102178169542021\\
236	-0.127159758277303\\
237	-0.116743989877684\\
238	-0.227751413776023\\
239	-0.0761921366925282\\
240	0.00073549018986007\\
};
\addlegendentry{Cts Stoch Ctrl w NMC};

\addplot [color=dscr_nFPC_plot_color,solid,line width=1.5pt]
  table[row sep=crcr]{%
1	-0.0622236403141551\\
2	-0.0902214029339275\\
3	-0.542290931891746\\
4	-0.0717159931334488\\
5	-0.201243961290449\\
6	-0.0205159854710596\\
7	-0.108018101271503\\
8	-0.0345685658457754\\
9	-0.0395623404473825\\
10	-0.00710962724638317\\
11	-0.0546198265711766\\
12	-0.105298470261885\\
13	-0.0132681269301445\\
14	-0.00820262041878009\\
15	-0.0176897572022576\\
16	-0.00347117319601139\\
17	-0.0349042621221976\\
18	-0.0391791119222073\\
19	-0.0540654180710634\\
20	-0.141267072952724\\
21	-0.0265016512326562\\
22	-0.0236207598606104\\
23	-0.0898093876323304\\
24	-0.0421938775337798\\
25	-0.044784567276272\\
26	-0.0400194988061513\\
27	-0.166538317442239\\
28	-0.0864737932784008\\
29	-0.0945690265312077\\
30	-0.00689195049311092\\
31	-0.0944950695228009\\
32	-0.186096758507663\\
33	-0.903607340688565\\
34	-0.227193131115657\\
35	-0.279884735271594\\
36	0.00971134247837628\\
37	-0.210000909954266\\
38	-0.0716067594794867\\
39	-0.0741181373152859\\
40	-0.0292473061097523\\
41	-0.097086674407452\\
42	-0.0630845875417165\\
43	-0.0171196920681851\\
44	-0.0517142166576022\\
45	-0.00206728966938795\\
46	-0.104572269152522\\
47	-0.108397275023398\\
48	-0.142576199918195\\
49	-0.152529964917434\\
50	-0.0391588230642041\\
51	-0.0252705628230447\\
52	-0.0241722807265594\\
53	-0.0905879241140987\\
54	-0.0544375219064896\\
55	-0.0814547668301678\\
56	-0.0942634872744548\\
57	0.0279637082316996\\
58	-0.0599447231913957\\
59	-0.0157381403300757\\
60	0.0417783102585255\\
61	-0.061661302172879\\
62	-0.0457520433081582\\
63	-0.0818618772971108\\
64	-0.0349052106245809\\
65	-0.0424434788780984\\
66	-0.0919713935537352\\
67	-0.0467653613516443\\
68	-0.0670277227015725\\
69	-0.126994512926019\\
70	-0.0531452931165491\\
71	0.0378043105477498\\
72	-0.109593020500796\\
73	-0.0254771827917257\\
74	-0.147394867733764\\
75	-0.364728498154195\\
76	-0.040240959519834\\
77	-0.0877259709108337\\
78	-0.0293458855617777\\
79	-0.192838749231946\\
80	-0.283023964761692\\
81	-0.0336014430708085\\
82	0.00348932110596437\\
83	-0.094830800584839\\
84	-0.0429276027984096\\
85	-0.0340383290336537\\
86	-0.0285004894319218\\
87	-0.0394006803604519\\
88	-0.107173626622257\\
89	-0.124573224481829\\
90	-0.171644995829899\\
91	-0.0685585924587923\\
92	-0.0353170007152629\\
93	-0.0391334431322789\\
94	0.00680186908474811\\
95	-0.10301852240814\\
96	-0.0789571128223681\\
97	-0.0922587816312591\\
98	-0.0488741282267409\\
99	-0.163186487179174\\
100	-0.0761392650795775\\
101	-0.0651113068472757\\
102	0.0281762900870729\\
103	-0.00807480901727355\\
104	-0.183732382719522\\
105	-0.0170101720071374\\
106	0.00525375093447478\\
107	0.00858546593338885\\
108	-0.0143810136961229\\
109	-0.0269078616776616\\
110	-0.0795656252962227\\
111	-0.0486901493672376\\
112	-0.0200663686632242\\
113	-0.0155263591138719\\
114	0.00809581309591992\\
115	-0.0302542674343119\\
116	0.0533265966555904\\
117	-0.0390713988537171\\
118	-0.19324540637989\\
119	-0.105977750696868\\
120	0.00462628013723824\\
121	-0.0732629729062215\\
122	-0.0268577487282578\\
123	0.00364658019612162\\
124	-0.0109155092309639\\
125	-0.0853037956782845\\
126	-0.0683959688525811\\
127	-0.032261124762357\\
128	-0.0740880725455547\\
129	-0.0632909429370922\\
130	-0.0135285669830839\\
131	-0.0208576001258154\\
132	-0.0777235195157482\\
133	-0.081760721788352\\
134	-0.0495625237324592\\
135	-0.0184125134061292\\
136	-0.377538664557245\\
137	-0.0521078627366559\\
138	-0.00883882224766369\\
139	-0.033764027209326\\
140	-0.0173829910371007\\
141	-0.0306354702621573\\
142	-0.0571649221326191\\
143	-0.046745546798273\\
144	-0.0041950788230853\\
145	-0.0134369556759433\\
146	-0.069554468112651\\
147	-0.00406869956106517\\
148	-0.0316823769879335\\
149	-0.0502657568826392\\
150	-0.0361906807789412\\
151	-0.0338544277081945\\
152	-0.0530489218274422\\
153	-0.0553829325534318\\
154	-0.0254645099463233\\
155	-0.0471205961204492\\
156	-0.0549166270956556\\
157	-0.0420849087961308\\
158	-0.0722496556889746\\
159	-0.0765796518791437\\
160	-0.0466696959772314\\
161	-0.0132397259525\\
162	-0.0417859034419682\\
163	-0.0286568319333279\\
164	-0.0306270322886295\\
165	-0.0392944029199278\\
166	-0.0186315364531892\\
167	-0.0424968649572698\\
168	0.000576257089235774\\
169	-0.0971774988680029\\
170	-0.0403157068411932\\
171	-0.0589031861797398\\
172	-0.0504786300445456\\
173	-0.031341609912908\\
174	-0.0451807162447258\\
175	-0.0637446493978277\\
176	-0.00384236881034699\\
177	-0.100988253518279\\
178	-0.0894755998710421\\
179	-0.22857805558768\\
180	-0.0726527386712699\\
181	0.00551852627939679\\
182	-0.0233628481454359\\
183	-0.0911163022512063\\
184	-0.063396265323932\\
185	-0.0980072653995251\\
186	-0.0254073911692917\\
187	-0.0386653237938813\\
188	-0.047339457224938\\
189	-0.0325608424450099\\
190	-0.0347837238329513\\
191	-0.0678912332272933\\
192	-0.0877696378294019\\
193	-0.033562859116245\\
194	-0.077530346724354\\
195	-0.0823316433175613\\
196	-0.0349509240569361\\
197	-0.0592287524696285\\
198	0.00601338794253177\\
199	-0.0927109166083285\\
200	-0.405672275514796\\
201	-0.351153290804939\\
202	-0.176700005644929\\
203	-0.0901474091204997\\
204	-0.190453778383595\\
205	-0.220083408628351\\
206	-0.0902196008698294\\
207	-0.0426876463195926\\
208	-0.125707831921622\\
209	-0.0820683052049124\\
210	-0.0340953391818671\\
211	-0.0312351278705099\\
212	-0.044968800985478\\
213	-0.102542419634159\\
214	-0.0634441831626672\\
215	-0.00395905323249378\\
216	-0.0433834610118744\\
217	-0.0913445654734438\\
218	-0.0687076681557595\\
219	-0.164161494195099\\
220	-0.0540132435500447\\
221	-0.0553912723358738\\
222	-0.0432155009974497\\
223	-0.0993336302941086\\
224	-0.0592135988537866\\
225	-0.0723902567796435\\
226	-0.0425758656386836\\
227	-0.093635446694427\\
228	-0.0632205143073491\\
229	-0.0496741146261264\\
230	-0.265678949422655\\
231	-0.0974478471882132\\
232	-0.077840837789503\\
233	-0.010825795914125\\
234	-0.0463316960790487\\
235	-0.103536985524458\\
236	-0.0921391128858256\\
237	-0.121690837742565\\
238	-0.578050734139789\\
239	-0.159598840346705\\
240	-0.018074250222312\\
};
\addlegendentry{Dscr Stoch Ctrl w NMC};

\end{axis}
\end{tikzpicture}%

\end{subfigure}%
\hfill%
\begin{subfigure}{.45\linewidth}
  \centering
  \setlength\figureheight{\linewidth} 
  \setlength\figurewidth{\linewidth}
  \tikzsetnextfilename{IS_week_NTAP}
  % This file was created by matlab2tikz.
%
%The latest updates can be retrieved from
%  http://www.mathworks.com/matlabcentral/fileexchange/22022-matlab2tikz-matlab2tikz
%where you can also make suggestions and rate matlab2tikz.
%
%
\begin{tikzpicture}[trim axis left, trim axis right]

\begin{axis}[%
width=\figurewidth,
height=\figureheight,
at={(0\figurewidth,0\figureheight)},
scale only axis,
every outer x axis line/.append style={black},
every x tick label/.append style={font=\color{black}},
xmin=1,
xmax=240,
%xlabel={Time (h)},
every outer y axis line/.append style={black},
every y tick label/.append style={font=\color{black}},
ymin=-1.1,
ymax=1.1,
%ylabel={Normalized PnL},
title={NTAP},
axis background/.style={fill=white},
axis x line*=bottom,
axis y line*=left,
yticklabel style={
        /pgf/number format/fixed,
        /pgf/number format/precision=3
},
scaled y ticks=false,
legend style={legend cell align=left,align=left,draw=black,font=\small, legend pos=north west},
every axis legend/.code={\renewcommand\addlegendentry[2][]{}}  %ignore legend locally
]
\addplot [color=cts_plot_color,solid,line width=1.5pt]
  table[row sep=crcr]{%
1	0.0973842315996598\\
2	-0.0708650954069955\\
3	-0.0598632120162681\\
4	0.0341523605968625\\
5	-0.00652695944036494\\
6	0.0195894381677894\\
7	0.0408599009402559\\
8	-0.00716058821124536\\
9	-0.107392035132287\\
10	0.00298464446297911\\
11	0.0471406931020282\\
12	0.108773794298955\\
13	-0.452966515536883\\
14	-0.0388028055304427\\
15	-0.0840846957730187\\
16	-0.0693172209752143\\
17	0.0493268753503404\\
18	-0.0575102653515546\\
19	0.0879208592925131\\
20	0.0291898443665743\\
21	-0.060069693165109\\
22	-0.0924867213712007\\
23	0.0102158499293169\\
24	-0.259938753763427\\
25	-0.47725819357207\\
26	-0.0828920841600036\\
27	0.0277038108307295\\
28	-0.1424138666713\\
29	-0.387827848470006\\
30	-0.110188843538406\\
31	-0.00596173213012182\\
32	0.0890686220975857\\
33	0.0832315186734875\\
34	-0.0559277181235802\\
35	-0.0924300197756936\\
36	-0.136539617751062\\
37	-0.0737087964191357\\
38	-0.0730543724262139\\
39	-0.0493033828088148\\
40	0.0187322988045535\\
41	0.0564279433419993\\
42	0.00595373998704507\\
43	0.133030267515719\\
44	-0.0127000043369285\\
45	0.0609548342088399\\
46	0.0663499375021835\\
47	-0.0285298535590834\\
48	-0.0301305337910599\\
49	0.0887161151340703\\
50	0.0473640158063604\\
51	-0.0362600550821372\\
52	0.0691194282565057\\
53	0.01141673468283\\
54	0.0174280224562928\\
55	0.0258091918218855\\
56	-0.00493745151199758\\
57	0.0473061584795541\\
58	0.0608516700823546\\
59	0.0804747254843715\\
60	0.0457808092371504\\
61	0.0286536617594221\\
62	-0.612189903637762\\
63	-0.167087974967655\\
64	-0.00969747218529174\\
65	-0.180728932238483\\
66	0.0438671552521918\\
67	-0.0500913508345933\\
68	-0.0780695756780642\\
69	0.0173881705260624\\
70	0.157911004189968\\
71	-0.0874048978518429\\
72	0.066151280261778\\
73	-0.0122414001608518\\
74	0.118879503024411\\
75	0.00107303527881092\\
76	-0.172098736336661\\
77	-0.0932679357515626\\
78	-0.169801828013107\\
79	-0.0119978407380682\\
80	-0.382475612272398\\
81	-0.00786648138056621\\
82	0.0499885009466521\\
83	-0.113410775061386\\
84	-0.00702652821288167\\
85	-0.0754342779012364\\
86	-0.304693540635337\\
87	0.0868670029577257\\
88	0.10458939703196\\
89	0.136700888298652\\
90	-0.0686251927558937\\
91	-0.0468358464385195\\
92	0.119048894819612\\
93	0.0211078453801693\\
94	0.0867679032302116\\
95	-0.0144372302542293\\
96	0.0922244543083735\\
97	0.0379171669411518\\
98	-0.0269605319573218\\
99	-0.0797527451819617\\
100	-0.116439854831078\\
101	-0.0440602322030294\\
102	-0.0876475806521321\\
103	0.00906483133371041\\
104	-0.0858587851749569\\
105	-0.0136003442744996\\
106	-0.0641346111214305\\
107	-0.000110011637300687\\
108	0.0372919471763565\\
109	-0.196224750592391\\
110	-0.139840626420907\\
111	-0.0479629633617055\\
112	-0.125685940553263\\
113	-0.0537672428487077\\
114	-0.0899824359670461\\
115	-0.088545697187311\\
116	-0.00536329371751885\\
117	0.00903009781103259\\
118	-0.177805770611549\\
119	-0.00573457681851643\\
120	-0.0323039792870895\\
121	-0.0503208559062953\\
122	-0.100681625113905\\
123	-0.0518738459796998\\
124	0.0173260218852166\\
125	0.0162083975438982\\
126	-0.0342858021622865\\
127	-0.164456995559568\\
128	-0.02769136384579\\
129	0.110179548534562\\
130	0.00554586380144403\\
131	-0.127235181686079\\
132	-0.014302810202828\\
133	-0.0150363808756104\\
134	0.0371083786406148\\
135	-0.0112709696918206\\
136	0.011418024923865\\
137	-0.0448750784287619\\
138	0.016659210915472\\
139	0.0280238927523558\\
140	0.0517541544693311\\
141	0.0472509431435074\\
142	0.0725860815201903\\
143	0.0208445393261559\\
144	0.012004775627597\\
145	-0.0440407923369719\\
146	0.0131233609097689\\
147	-0.0142313260269168\\
148	0.00151835358913453\\
149	0.0363672232634753\\
150	-0.0122267145492334\\
151	-0.0420302152254878\\
152	0.024357017844225\\
153	0.0333996547270617\\
154	0.0564529242258631\\
155	0.0979691531352013\\
156	-0.051266094379007\\
157	-0.0113941383939634\\
158	-0.0475199148250848\\
159	0.0196101199306324\\
160	0.0356842358480718\\
161	-0.0409987474021925\\
162	-0.0276228242107958\\
163	0.0696573675515952\\
164	-0.0842302317220567\\
165	0.0430481430293894\\
166	0.00207627757627633\\
167	0.0348728848560369\\
168	0.0484201417835678\\
169	0.0356939905297771\\
170	0.0606889683700428\\
171	0.102790080003422\\
172	0.0270719625968143\\
173	-0.0891425154163665\\
174	0.00665998755658641\\
175	-0.0798350273128964\\
176	0.0430239494082436\\
177	0.0404849596397555\\
178	0.00773424163131437\\
179	0.0203945601793577\\
180	0.00208754748670289\\
181	0.10936434516018\\
182	0.0163821812803336\\
183	-0.00603797725033009\\
184	0.0498808149625003\\
185	0.0746372585555441\\
186	0.0745077027351688\\
187	-0.0392155062254381\\
188	0.000721633881269832\\
189	-0.0314516428962623\\
190	0.0318774689583385\\
191	0.0395510001309161\\
192	-0.0046065009632562\\
193	0.0123540943081898\\
194	-0.137494414718074\\
195	0.0162279491866922\\
196	0.0224618978454733\\
197	0.0394943147636381\\
198	0.000730236493346396\\
199	0.0373462725164568\\
200	0.0238811394288191\\
201	0.00402512529932606\\
202	0.0564962753322664\\
203	0.11231731008455\\
204	0.143079017599241\\
205	0.0425705047122373\\
206	0.062346163114291\\
207	0.0618846006026825\\
208	0.0928790567125626\\
209	0.0354268926922381\\
210	-0.0878627553214327\\
211	0.0396297912552752\\
212	-0.0455798560871387\\
213	-0.0790917174718304\\
214	0.0401246697427942\\
215	-0.0768316976721247\\
216	0.0200646394235883\\
217	0.0334186708568887\\
218	0.0499161413844089\\
219	-0.00873390501630864\\
220	-0.0994913239564507\\
221	-0.0218355431308969\\
222	0.0119232044776062\\
223	-0.122587064806905\\
224	0.0584418982669117\\
225	-0.00392136452251621\\
226	0.00936478759172981\\
227	-0.0342997397713096\\
228	-0.0713175880221822\\
229	-0.0116268589818055\\
230	0.0264458483768565\\
231	0.0609417742133519\\
232	0.030173869175963\\
233	-0.0688490661433105\\
234	-0.0325883067354387\\
235	0.0683853891061657\\
236	-0.0240907528205583\\
237	0.0326078720422878\\
238	0.00338068277755105\\
239	-0.0585670470297425\\
240	-0.0380622041802715\\
};
\addlegendentry{Cts Stoch Ctrl};

\addplot [color=dscr_plot_color,solid,line width=1.5pt]
  table[row sep=crcr]{%
1	0.0703720049266865\\
2	0.0131570971235728\\
3	0.138204783252761\\
4	0.054913776919686\\
5	0.0886245118733281\\
6	0.168782981974309\\
7	0.146748613683705\\
8	0.0305576335210845\\
9	0.127846792970914\\
10	-0.0331845819674926\\
11	0.266919256727259\\
12	0.140609833422398\\
13	0.0197032835328913\\
14	0.0270110335132634\\
15	-0.151363094159371\\
16	-0.0309665053476047\\
17	-0.0438607592736874\\
18	0.081021793472687\\
19	0.130374023075972\\
20	0.168100107226583\\
21	0.0554268607588556\\
22	-0.120777675628993\\
23	0.0435005833594051\\
24	0.254067640596002\\
25	0.223582152624015\\
26	0.122779978744148\\
27	0.0860531211174495\\
28	-0.354740091939946\\
29	-0.355126714292391\\
30	-0.0631804308522634\\
31	-0.0327048336344858\\
32	0.118512166123868\\
33	-0.0167039732512806\\
34	0.087181492555383\\
35	-0.0149865421487683\\
36	-0.0232170297438526\\
37	0.0161697736347325\\
38	-0.0164689248728257\\
39	0.0196389470066701\\
40	-0.00693261054426208\\
41	0.0969448129590094\\
42	0.0500657884045183\\
43	0.11975200340865\\
44	0.0129281507354426\\
45	0.0500729137213891\\
46	0.0494494047779015\\
47	0.0677095968173248\\
48	0.0242826694960415\\
49	0.090531120878199\\
50	-0.0648261507830045\\
51	-0.0275720051660725\\
52	0.186643372329312\\
53	0.0778485764868438\\
54	0.118872938736678\\
55	0.237407526901494\\
56	-0.0408340326712999\\
57	0.017711190838325\\
58	0.0781740766019589\\
59	0.12444187077108\\
60	0.0851912661283341\\
61	0.139778033270261\\
62	0.341818771808098\\
63	0.0650509044970772\\
64	0.310371121284102\\
65	-0.110531828864107\\
66	-0.0860367523964\\
67	0.002420316510972\\
68	0.211405940233653\\
69	0.379847057267015\\
70	0.114214490743908\\
71	-0.0441445609205973\\
72	0.148642814541579\\
73	-0.0681327531166301\\
74	0.16052083484256\\
75	-0.125258506529542\\
76	0.031466200062119\\
77	0.380966745135204\\
78	0.0261574240493854\\
79	0.0610611710914722\\
80	0.040053103160992\\
81	0.0572662523947591\\
82	0.165802990829075\\
83	0.11443974016285\\
84	0.0801875405391441\\
85	0.154084235128611\\
86	0.503845335895026\\
87	-0.108121660741786\\
88	0.0456741717295523\\
89	0.173027832842805\\
90	-0.00895336607024053\\
91	0.017883607886326\\
92	0.0891335436901179\\
93	0.217466107071817\\
94	0.148030251192651\\
95	0.0252119114804575\\
96	0.0825536653247613\\
97	-0.00327135297004967\\
98	0.0929930497265642\\
99	0.224847413314423\\
100	0.106663123770563\\
101	0.0641076046925142\\
102	0.12651608672365\\
103	-0.0173231033191183\\
104	0.174985279681904\\
105	-0.0736593350195474\\
106	0.106752694994312\\
107	0.122440134711149\\
108	0.0810235360476938\\
109	-0.0525456071667759\\
110	-0.0398475622394987\\
111	0.0555125133539333\\
112	-0.0492517300046208\\
113	0.0251673310518787\\
114	-0.00128025087771159\\
115	0.0520316601428443\\
116	0.0946532931646031\\
117	0.0547263047033345\\
118	0.0682001471849204\\
119	0.0725394662934728\\
120	-0.0438700704818943\\
121	0.0349106889074497\\
122	0.0388626562316958\\
123	0.128572894410691\\
124	0.110933513095272\\
125	0.0746946495013565\\
126	0.147597425692192\\
127	-0.160713631569919\\
128	0.140870131762671\\
129	0.0961105931250165\\
130	0.00479671563141531\\
131	0.11012099730051\\
132	0.0673907358049508\\
133	0.033143917369162\\
134	0.0388080060804038\\
135	0.039005304931326\\
136	-0.0032686173221429\\
137	0.0884789177185559\\
138	0.0423259515873512\\
139	0.134714071372903\\
140	0.0140861738915347\\
141	0.130199764376351\\
142	0.148298521857623\\
143	0.10963107031178\\
144	-0.00100301481994817\\
145	0.0222669739083351\\
146	0.0616899541537857\\
147	0.858103801715829\\
148	0.180998242470763\\
149	0.134713181055879\\
150	0.00536507856131954\\
151	-0.0139582803641756\\
152	0.126306284956143\\
153	0.0417697811435116\\
154	0.00700168211320959\\
155	0.0661718377738596\\
156	-0.0712701599376247\\
157	0.05861661684765\\
158	-0.00432438458340827\\
159	0.0919093275241118\\
160	-0.0243343033170994\\
161	0.0304260385092162\\
162	0.112061563671839\\
163	0.0936579920591389\\
164	0.166368897263938\\
165	0.10339254630715\\
166	0.134336774363349\\
167	0.0134596614583168\\
168	0.0791195462892243\\
169	-0.00833855563794509\\
170	0.111074495240072\\
171	0.0357613720189755\\
172	0.0693815857291547\\
173	0.0548262292885271\\
174	0.0701809990456316\\
175	0.0885283655052241\\
176	-0.00132236582863772\\
177	0.107546138557534\\
178	0.0751004693834156\\
179	0.0485959942035392\\
180	-0.0151512017364843\\
181	0.0936888582035544\\
182	0.176044433462656\\
183	0.0749982538589873\\
184	0.119839649693843\\
185	0.136264843230219\\
186	0.115244970463338\\
187	0.145324854694208\\
188	0.067693120205217\\
189	0.0323687640217347\\
190	0.226325967443027\\
191	0.124518444187915\\
192	0.0413785683165608\\
193	-0.0673277822677443\\
194	-0.0127395406619523\\
195	0.102882504336133\\
196	0.0622036747140454\\
197	-0.0163633714192136\\
198	0.0183265646790919\\
199	-0.0121286035448168\\
200	0.171304349298364\\
201	0.0756299722242753\\
202	0.14268579165767\\
203	0.123876883982222\\
204	0.106299717769913\\
205	0.357810796775369\\
206	0.206137903508627\\
207	0.0854257580335576\\
208	0.1501699794034\\
209	0.457609782231183\\
210	0.0899058480398283\\
211	0.237119578694487\\
212	0.0187829546946557\\
213	0.0914304801072561\\
214	0.0580026276419761\\
215	0.026639367472526\\
216	0.0777822471417843\\
217	0.0591151086329065\\
218	0.0850248715728726\\
219	0.105704023427833\\
220	-0.179455832253954\\
221	0.132402567886143\\
222	0.050460845647744\\
223	0.24869901501101\\
224	0.0648627812921855\\
225	0.0330572876890373\\
226	0.0987520949934108\\
227	-0.32733212757619\\
228	-0.157421786003843\\
229	0.0165227569390805\\
230	0.060339698416858\\
231	0.0986272514484145\\
232	0.0465909091451621\\
233	0.0859018863053148\\
234	-0.0411104150350423\\
235	0.0699766222008862\\
236	0.0182982984295398\\
237	0.00167502174536144\\
238	0.0619841721316854\\
239	0.060638188739647\\
240	-0.0046001609284566\\
};
\addlegendentry{Dscr Stoch Ctrl};

\addplot [color=cts_nFPC_plot_color,solid,line width=1.5pt]
  table[row sep=crcr]{%
1	-0.139805058667736\\
2	-0.0451404477631651\\
3	-0.158370684869451\\
4	-0.227529350882089\\
5	-0.245508724865619\\
6	-0.0709054916273988\\
7	-0.159481710576313\\
8	-0.267538397034946\\
9	0.0521049156195567\\
10	-0.178238146718729\\
11	-0.00522146613769647\\
12	-0.0808943546013343\\
13	0.0827574746868611\\
14	-0.0960108910954503\\
15	-0.0806871987593052\\
16	-0.0528525149356751\\
17	-0.410678949658928\\
18	-0.061181405137241\\
19	0.105880802053019\\
20	-0.220005930521351\\
21	-0.185003101970468\\
22	-0.21342065955324\\
23	-0.196639402325448\\
24	-0.284449823239727\\
25	-0.62615723249344\\
26	-0.235633565476951\\
27	-0.0351387672420759\\
28	-0.345606759420644\\
29	-0.444868848128163\\
30	-0.357640752526337\\
31	-0.566076818411576\\
32	-0.389955250992929\\
33	-0.331432449312558\\
34	-0.349072016290433\\
35	-0.26000583584631\\
36	-0.222837887603184\\
37	-0.162393288520881\\
38	-0.196353950358269\\
39	-0.200925898744061\\
40	-0.124280316409587\\
41	-0.23321618584913\\
42	-0.153131406215968\\
43	-0.0646832022754805\\
44	-0.118513096528325\\
45	-0.190410595635772\\
46	-0.396068228256934\\
47	-0.381999122273798\\
48	-0.258910314452487\\
49	-0.0281677600694823\\
50	-0.168146056425567\\
51	-0.221389263464318\\
52	-0.166807851495887\\
53	-0.0906337302012409\\
54	-0.157018191564938\\
55	-0.11634201589609\\
56	-0.289591401779817\\
57	-0.304850412481734\\
58	-0.194425723176599\\
59	-0.226253713194153\\
60	-0.362417784551983\\
61	-0.175631622079742\\
62	-1.08339291676473\\
63	-0.598541588805985\\
64	-0.11364061662608\\
65	-0.806238309103878\\
66	-0.559425384372708\\
67	-0.344960670065169\\
68	-0.393272737098969\\
69	-0.299285912334562\\
70	-0.1406947830123\\
71	-0.316468720178912\\
72	-0.481412017893969\\
73	-0.325153031044975\\
74	-0.260770225929821\\
75	-0.330789703944535\\
76	-0.201032549253977\\
77	-0.139015504498741\\
78	-0.20404752942388\\
79	0.0491519807258837\\
80	-0.380707648021428\\
81	-0.195191053418403\\
82	0.0309756482068927\\
83	-0.382271259936085\\
84	-0.253206985150912\\
85	-0.328310168564215\\
86	-0.8531098372815\\
87	-0.129444819039134\\
88	-0.109971187942221\\
89	-0.216353398648089\\
90	-0.0871309653400267\\
91	0.0148561179244617\\
92	-0.251188959527083\\
93	-0.575531995266241\\
94	-0.0853015766673952\\
95	-0.139304179117453\\
96	-0.00941462402616778\\
97	-0.122796320813025\\
98	-0.237271780360567\\
99	-0.158600706324494\\
100	-0.182355265848589\\
101	-0.1639725030951\\
102	-0.0895770323041142\\
103	-0.0609674295685618\\
104	-0.136358852213361\\
105	-0.167515313486299\\
106	-0.0458931996934408\\
107	-0.202565300929382\\
108	-0.163258275148794\\
109	-0.464295777616052\\
110	-0.253947745821125\\
111	-0.210567161259449\\
112	-0.164988155548863\\
113	-0.133259844991595\\
114	-0.132500221879267\\
115	-0.42913451718449\\
116	-0.06461795653533\\
117	-0.156329294214558\\
118	-0.0776825508361103\\
119	-0.252925793551627\\
120	-0.226815343485615\\
121	-0.198884384654073\\
122	-0.159777766329815\\
123	-0.0371167163108244\\
124	-0.0594494272206225\\
125	-0.0771971952089563\\
126	-0.0951316343720054\\
127	-0.0858963469206517\\
128	-0.152775050735581\\
129	-0.148170290773805\\
130	-0.17047129066974\\
131	-0.183520076724633\\
132	-0.229329484887191\\
133	-0.136802763409496\\
134	-0.0929533091569738\\
135	-0.121382548908818\\
136	-0.0329245581069627\\
137	0.0142572918880028\\
138	-0.0988132400802492\\
139	-0.10364364052205\\
140	-0.0336228633268921\\
141	-0.0655148977077766\\
142	0.0462071874361303\\
143	-0.0584706337534911\\
144	-0.0432530997445138\\
145	-0.0636850169984925\\
146	-0.056203599847353\\
147	-0.0640876360369058\\
148	-0.04328865383036\\
149	-0.0711619757335845\\
150	-0.0541234861951094\\
151	-0.276314170760101\\
152	0.0161816682374379\\
153	-0.0489838344333206\\
154	-0.0756218734484506\\
155	-0.0249367905650375\\
156	-0.111887997467713\\
157	-0.152478222658382\\
158	-0.239539869184267\\
159	-0.0406230069141187\\
160	-0.117504138062886\\
161	-0.114719327256121\\
162	-0.0593547328900494\\
163	0.029283525163585\\
164	-0.0491937260716997\\
165	0.0263109265209291\\
166	-0.106789555619347\\
167	-0.100624037266344\\
168	-0.0171202410538076\\
169	0.0360161424788116\\
170	-0.0153844101057019\\
171	-0.114226973436493\\
172	-0.0301528770846605\\
173	-0.17748137478675\\
174	-0.0808235320593774\\
175	-0.0568600144908419\\
176	-0.0893365363075065\\
177	-0.10107433274118\\
178	-0.12073908377315\\
179	-0.0385581121709723\\
180	-0.012864186318239\\
181	-0.0165695356914907\\
182	-0.0524252979060749\\
183	-0.129241184142247\\
184	-0.0747396618411098\\
185	-0.0336547427265626\\
186	0.0624152543784383\\
187	-0.133518269646842\\
188	-0.124926116654389\\
189	-0.0242659847043819\\
190	-0.123716209586504\\
191	-0.0638767211387936\\
192	-0.113361908956059\\
193	-0.290206004994798\\
194	-0.352184540335726\\
195	-0.12780729077896\\
196	-0.0618855806306413\\
197	-0.0585759520522991\\
198	-0.147400507388389\\
199	-0.153150853445248\\
200	-0.104298701807242\\
201	-0.102345525978899\\
202	-0.105568230893709\\
203	-0.00929950971616619\\
204	0.0316890251894685\\
205	0.0660363904060208\\
206	0.0509260506138231\\
207	-0.0167802905836541\\
208	-0.032518064905848\\
209	-0.157417949112782\\
210	-0.522363238474154\\
211	-0.0864178118798642\\
212	-0.157979290439029\\
213	-0.0836034338984622\\
214	-0.302538517422969\\
215	0.0403595552492635\\
216	-0.195632526009467\\
217	0.00429088957262867\\
218	0.011948960731785\\
219	-0.0616913728362621\\
220	-0.275275702185792\\
221	-0.0181740622095131\\
222	-0.028219994469852\\
223	-0.127663475429474\\
224	-0.0198263095684604\\
225	-0.0556835640337739\\
226	-0.0574177187395385\\
227	-0.114168063780801\\
228	-0.120117543015972\\
229	-0.214764456837728\\
230	0.0171972288329242\\
231	0.0305378553274413\\
232	-0.130151195857846\\
233	-0.106397177516726\\
234	-0.174768976494117\\
235	-0.00990093967864353\\
236	-0.211169153497108\\
237	-0.0039709544119715\\
238	-0.0615590173599486\\
239	-0.0682867244033581\\
240	-0.0266869421089509\\
};
\addlegendentry{Cts Stoch Ctrl w NMC};

\addplot [color=dscr_nFPC_plot_color,solid,line width=1.5pt]
  table[row sep=crcr]{%
1	0.212369623403168\\
2	-0.00118159023845058\\
3	0.157753925724068\\
4	0.111647395220853\\
5	0.16586166405704\\
6	-0.116395038591044\\
7	0.138953124208097\\
8	0.0321024325280366\\
9	-0.052403116069919\\
10	0.114618052283324\\
11	0.297238025276683\\
12	0.0671491776788578\\
13	-0.212781307507827\\
14	0.081646922126694\\
15	-0.226056725272632\\
16	0.0404602496968807\\
17	0.000680666591977871\\
18	0.0806172507259021\\
19	0.0159552937303739\\
20	-0.0336127503876021\\
21	0.0777662642801583\\
22	-0.076734598592604\\
23	0.11696151367665\\
24	0.365734670358325\\
25	0.392356680279431\\
26	0.114699594316019\\
27	0.140170961860565\\
28	0.24524768169929\\
29	-0.416168162249947\\
30	-0.0298495309624558\\
31	-0.0106368949264504\\
32	0.215252947741464\\
33	0.136959761243266\\
34	0.00311741319488611\\
35	0.120837635351075\\
36	0.000258481185795926\\
37	0.106711786231056\\
38	0.240149044185374\\
39	-0.00408549563017066\\
40	0.11430487848279\\
41	0.156602630599312\\
42	0.273672566733029\\
43	0.179691498906968\\
44	0.0504732603950427\\
45	0.0648037114127482\\
46	0.225451636934519\\
47	0.079321164598447\\
48	0.0926756523569687\\
49	0.190666439621348\\
50	0.0143780590219546\\
51	0.0104614697628026\\
52	-0.0311075043205512\\
53	0.133149597907239\\
54	0.191040626683126\\
55	0.283460747294866\\
56	0.0421407728199236\\
57	0.074077508772687\\
58	0.0949115047389401\\
59	0.160141378763195\\
60	0.0843346907672647\\
61	0.239714689728876\\
62	0.509461052518395\\
63	0.0756898001707959\\
64	0.301257622077714\\
65	-0.172227201653601\\
66	-0.16047462863718\\
67	0.0772336895619947\\
68	0.323360869671054\\
69	0.433692375758365\\
70	0.216192659093922\\
71	-0.0177929553874671\\
72	0.211459039269082\\
73	-0.0599224602969366\\
74	0.000613001967151697\\
75	0.210190508613215\\
76	0.0134040114181168\\
77	0.361261379018139\\
78	0.0917167832687542\\
79	0.0150254480996171\\
80	0.0723362153813598\\
81	0.0867688754032171\\
82	0.188182843146294\\
83	0.0227795228827394\\
84	0.197784317898375\\
85	0.0125877564587754\\
86	-0.31662809742699\\
87	0.48401003236161\\
88	0.0921178109073896\\
89	0.181043534287218\\
90	0.681490424825109\\
91	0.00591502711937267\\
92	0.07433148514321\\
93	0.287550970575212\\
94	0.185598759934311\\
95	0.113216234262766\\
96	0.0719882940982891\\
97	0.0482479111855568\\
98	0.285480911762139\\
99	0.183848037883026\\
100	-0.159110325158905\\
101	0.0398684872933536\\
102	-0.0792744700754039\\
103	-0.0751682628023988\\
104	0.149736153157471\\
105	-0.0740033491404619\\
106	-0.0382474644734932\\
107	0.151759076662519\\
108	0.201142861857653\\
109	-0.107029488133103\\
110	0.114500842902242\\
111	0.0386783462056631\\
112	-0.0484538651733601\\
113	0.0414201638588112\\
114	0.0800991816675072\\
115	-0.0516171378841521\\
116	0.0918823367908057\\
117	0.141347491709536\\
118	0.0970385551296132\\
119	0.101346471290196\\
120	-0.0340404665642237\\
121	0.0536278217177678\\
122	0.0165288375453295\\
123	0.102572874590666\\
124	0.076131463426304\\
125	0.0939550012837272\\
126	0.176470258246533\\
127	-0.0347861082713793\\
128	0.134383264300894\\
129	0.102422867910867\\
130	0.08361697508324\\
131	0.0241053627229993\\
132	0.105942882328945\\
133	0.0549778446143145\\
134	0.0896274931422403\\
135	0.0974137627825942\\
136	8.45005460011903e-05\\
137	0.0631686222401509\\
138	0.0462973009387984\\
139	0.153288958175477\\
140	0.053128591050959\\
141	0.173374349769575\\
142	0.16805372769577\\
143	0.121204835615406\\
144	0.00688519707297414\\
145	0.116676465566799\\
146	0.103081561392267\\
147	1.04816710190263\\
148	0.0703659678364617\\
149	0.157543928575852\\
150	0.0370815541820001\\
151	0.0341633036370514\\
152	0.0938146589510169\\
153	0.0713526119601484\\
154	0.0257126438468068\\
155	0.0921744525361482\\
156	-0.0133419076438335\\
157	0.0679668488457903\\
158	0.0387547479927556\\
159	0.0454334860942742\\
160	0.0672853526299744\\
161	0.0338795240070051\\
162	0.122006783264531\\
163	0.0890457436857605\\
164	0.132770386071602\\
165	0.117362027281264\\
166	0.0305973624725745\\
167	0.122373259649513\\
168	0.0685252275744998\\
169	0.000746608023432382\\
170	0.0919017084930504\\
171	0.0547934788780824\\
172	0.0911854255101162\\
173	0.0153503553359939\\
174	0.0551879275522795\\
175	0.121754012417702\\
176	0.0287637355629404\\
177	0.160092412283716\\
178	0.0733548694147488\\
179	0.0615450437058364\\
180	0.0111449395886488\\
181	0.10070701874267\\
182	0.166610940092815\\
183	0.035861073512564\\
184	0.0905977208344743\\
185	0.0881121216659243\\
186	0.12854481584928\\
187	0.0129837999862982\\
188	0.122492096573611\\
189	0.0399920041654045\\
190	0.232237891167446\\
191	0.0762168853372158\\
192	0.0200281033117737\\
193	-0.0487205089614086\\
194	0.00311190983816331\\
195	0.103245091180965\\
196	0.109929210220211\\
197	0.148389922574749\\
198	0.00834126057065443\\
199	-0.0142926457061776\\
200	0.141882144618257\\
201	0.0871911752251949\\
202	0.13713817597731\\
203	0.117296782813323\\
204	0.174313487446561\\
205	0.36528597127677\\
206	0.232943162009742\\
207	0.13457553763455\\
208	0.170120923700974\\
209	0.438604887097346\\
210	0.143951307792897\\
211	0.0238981361581802\\
212	-0.0679737356839339\\
213	0.125707003215686\\
214	0.0768447897813109\\
215	0.0396662792630399\\
216	0.138629982343903\\
217	0.0356179356058343\\
218	0.0908604404199727\\
219	0.104401043197881\\
220	-0.115806795130364\\
221	0.0588956673446209\\
222	0.0532713405008364\\
223	0.208714599789417\\
224	0.0429985474335868\\
225	0.119813575912389\\
226	0.120468810280838\\
227	0.228373720150135\\
228	-0.145456806935693\\
229	0.0349564804506123\\
230	0.0691140386050729\\
231	0.102494497826076\\
232	0.0740934067708173\\
233	-0.123339326075203\\
234	-0.014782199130979\\
235	0.0981118486193388\\
236	0.00447697501266885\\
237	-0.0606588291369753\\
238	0.0902325955025149\\
239	0.059260665473269\\
240	0.00297640142370169\\
};
\addlegendentry{Dscr Stoch Ctrl w NMC};

\end{axis}
\end{tikzpicture}%
 
\end{subfigure}\\
\vspace{1cm}
\begin{subfigure}{.45\linewidth}
  \centering
  \setlength\figureheight{\linewidth} 
  \setlength\figurewidth{\linewidth}
  \tikzsetnextfilename{IS_week_ORCL}
  % This file was created by matlab2tikz.
%
%The latest updates can be retrieved from
%  http://www.mathworks.com/matlabcentral/fileexchange/22022-matlab2tikz-matlab2tikz
%where you can also make suggestions and rate matlab2tikz.
%
\definecolor{mycolor1}{rgb}{0.25098,0.00000,0.38824}%
\definecolor{mycolor2}{rgb}{0.00000,0.46275,0.00000}%
\definecolor{mycolor3}{rgb}{0.00000,0.34902,0.34902}%
\definecolor{mycolor4}{rgb}{0.58039,0.26275,0.00000}%
%
\begin{tikzpicture}[trim axis left, trim axis right]

\begin{axis}[%
width=\figurewidth,
height=\figureheight,
at={(0\figurewidth,0\figureheight)},
scale only axis,
every outer x axis line/.append style={black},
every x tick label/.append style={font=\color{black}},
xmin=1,
xmax=240,
%xlabel={Time (h)},
every outer y axis line/.append style={black},
every y tick label/.append style={font=\color{black}},
ymin=-1.1,
ymax=1.1,
%ylabel={Normalized PnL},
title={ORCL},
axis background/.style={fill=white},
axis x line*=bottom,
axis y line*=left,
yticklabel style={
        /pgf/number format/fixed,
        /pgf/number format/precision=3
},
scaled y ticks=false,
legend style={legend cell align=left,align=left,draw=black,font=\small, legend pos=north west},
every axis legend/.code={\renewcommand\addlegendentry[2][]{}}  %ignore legend locally
]
\addplot [color=mycolor1,solid,line width=1.5pt]
  table[row sep=crcr]{%
1	0.106540754541501\\
2	0.22519372901346\\
3	0.137704826456894\\
4	0.168538911227994\\
5	0.0300561834533785\\
6	0.13903655298721\\
7	0.126874101627404\\
8	0.237833677362031\\
9	0.130642833885921\\
10	0.101937030161955\\
11	0.156042487038076\\
12	0.241810122294663\\
13	0.161519843532851\\
14	0.0904173528628341\\
15	0.0383617227233518\\
16	0.108386493967483\\
17	-0.032647944717102\\
18	0.11826176428271\\
19	0.0865219332948834\\
20	0.148362063470914\\
21	0.175519765254118\\
22	0.108810693701721\\
23	0.193732306006846\\
24	0.0994486582297581\\
25	0.0752718234446712\\
26	0.131260170381711\\
27	0.049909224561027\\
28	-0.0366442919029737\\
29	0.0980880419790949\\
30	0.0754522444194005\\
31	0.168759918810082\\
32	0.150401317875709\\
33	0.207770742854203\\
34	0.106264370122263\\
35	-0.0108015303115805\\
36	0.148488992902124\\
37	0.169522807546045\\
38	0.00780302608514987\\
39	-0.114989717867627\\
40	0.0967554207335205\\
41	0.0208527812867942\\
42	0.0620210884541054\\
43	0.172283427442278\\
44	0.130877263274654\\
45	0.143522461323539\\
46	0.0733889442296028\\
47	0.201432577003317\\
48	0.309802582383725\\
49	0.121277788123214\\
50	-0.132300214479061\\
51	0.138173316342143\\
52	0.17565710017519\\
53	0.200976957200634\\
54	0.184688371728441\\
55	0.18607799584135\\
56	0.0550629689159508\\
57	0.191717667592766\\
58	0.192061274875198\\
59	0.258033393314315\\
60	0.265595365748545\\
61	0.078564545224155\\
62	0.139117902133239\\
63	0.0355346657015414\\
64	0.210116966047322\\
65	0.0367465719449428\\
66	0.144908813949386\\
67	0.300505847984412\\
68	0.213455634362758\\
69	0.136804229125549\\
70	0.244318534955863\\
71	0.180124989382077\\
72	0.179980606683071\\
73	0.182654914049672\\
74	0.15752146785494\\
75	0.0840143360724211\\
76	0.104615061355028\\
77	0.0484170713308239\\
78	0.167493834086675\\
79	-0.0344109502234701\\
80	0.146015856657944\\
81	0.206825097223236\\
82	0.0300674464814801\\
83	0.0608210664331526\\
84	-0.107472698835373\\
85	0.125432346243525\\
86	-0.0142956830596146\\
87	0.0407122910096369\\
88	0.112004966557289\\
89	0.10239431667985\\
90	-0.180466543256322\\
91	0.100504235088528\\
92	0.0440663122405075\\
93	0.0576425718320533\\
94	0.11093020719874\\
95	0.0840501164846748\\
96	-0.0013289362316018\\
97	0.164458704586842\\
98	0.140011164944835\\
99	0.0969858976414885\\
100	0.00836844899112041\\
101	0.237147295459091\\
102	0.0597236105534419\\
103	0.126470853540951\\
104	0.0146315140526567\\
105	0.133716930340285\\
106	0.101109854033775\\
107	0.0997633675058042\\
108	0.111676793506579\\
109	0.00355380804801269\\
110	-0.157212346042898\\
111	0.273065472967105\\
112	0.205900026021536\\
113	0.0997771262975968\\
114	-0.140400800628824\\
115	0.209978541658699\\
116	-0.203085906962451\\
117	0.206988909267803\\
118	0.147020419666974\\
119	0.161570903225032\\
120	0.195837674365191\\
121	0.108088242674058\\
122	0.0719793282939159\\
123	-0.0791268545270439\\
124	0.169441288708443\\
125	0.115612722044645\\
126	0.109637229538418\\
127	0.0640339334916849\\
128	0.0451473112000344\\
129	0.0364309227551844\\
130	0.0490332483132823\\
131	0.0437042104070921\\
132	0.157818382592803\\
133	0.0794614974287867\\
134	0.163798639187689\\
135	0.117969394594341\\
136	0.22576095036217\\
137	0.126883311502032\\
138	0.0781489835973646\\
139	0.0162281486427537\\
140	0.130748727268543\\
141	0.0511427420779248\\
142	0.077178168633474\\
143	0.054232453028918\\
144	0.151581378585396\\
145	-0.00281670124734755\\
146	0.128790874779945\\
147	-0.0269832512909599\\
148	0.0770599271074546\\
149	0.00466176316042218\\
150	0.0796886207761633\\
151	0.109268808323537\\
152	0.08773189845648\\
153	0.0778706326410299\\
154	-0.0571944272605597\\
155	-0.0123174595114863\\
156	-0.000303449326891686\\
157	0.0133312865871319\\
158	-0.0255703307017186\\
159	0.0782193822713527\\
160	-0.010557305457592\\
161	0.0645361926212708\\
162	0.118548149997695\\
163	0.00483175004533483\\
164	0.0232509846706501\\
165	0.127367418449002\\
166	0.0801217980547043\\
167	0.0221770221356782\\
168	0.0153861567969524\\
169	0.200554715140204\\
170	0.172803787809462\\
171	0.251390032188578\\
172	0.0631820688585877\\
173	0.0406025199863487\\
174	0.0751247999867299\\
175	0.0265710690877486\\
176	0.0978538979144822\\
177	0.154206005405942\\
178	0.160271917054417\\
179	0.112793232603879\\
180	0.131014144354607\\
181	0.182082003168149\\
182	0.15734682731923\\
183	0.177836859277068\\
184	0.216075653365485\\
185	0.106006125985158\\
186	0.0302139317292166\\
187	0.122603808735947\\
188	0.175198306217881\\
189	0.12038397078862\\
190	0.10797620742374\\
191	0.156454209563141\\
192	0.0857024349421306\\
193	0.110144412756735\\
194	0.0937727814791615\\
195	0.0907290933407674\\
196	0.0489790112177488\\
197	0.137430549025354\\
198	0.0367857861188821\\
199	0.117020251879886\\
200	0.124634438767712\\
201	0.105245387705113\\
202	0.0858524000999755\\
203	0.0631476246688538\\
204	0.0977850741680623\\
205	-0.013206364918819\\
206	-0.00211264214248321\\
207	0.0573987144102122\\
208	0.00484426790904826\\
209	0.115707009147385\\
210	0.154632156567978\\
211	0.0321881421261843\\
212	0.0871493925104386\\
213	0.029303984948224\\
214	0.1179825367604\\
215	-0.0145495339508366\\
216	0.0940658411903503\\
217	0.0734611839077353\\
218	0.153341423861865\\
219	0.0713073538391985\\
220	0.0437609588143221\\
221	-0.00231820108185265\\
222	0.162239565127336\\
223	0.155770854003913\\
224	0.102784958684766\\
225	0.0966195948139982\\
226	0.035157020915888\\
227	0.0958560777098545\\
228	0.0480149986944441\\
229	0.0306402568862938\\
230	0.0948192037587309\\
231	0.0986593751249876\\
232	0.285580744072392\\
233	0.132028209341729\\
234	0.0710948097076419\\
235	0.187737164656444\\
236	0.0160842048662836\\
237	0.156303332833646\\
238	0.0900505896142218\\
239	0.0566559933095659\\
240	0.0274556501064303\\
};
\addlegendentry{Cts Stoch Ctrl};

\addplot [color=mycolor2,solid,line width=1.5pt]
  table[row sep=crcr]{%
1	0.112923418391793\\
2	0.192712292915206\\
3	0.134469330581035\\
4	0.170487041797623\\
5	0.17817365170928\\
6	0.186990828628621\\
7	0.151059478170748\\
8	0.505273049610193\\
9	0.178868120574461\\
10	0.11755788465129\\
11	0.204926251447202\\
12	0.296505559149932\\
13	0.246645023489756\\
14	0.0876852876535064\\
15	0.220064166280156\\
16	0.128552760614477\\
17	-0.0950319854256295\\
18	0.10683909209394\\
19	0.0645682801076503\\
20	0.109773325585976\\
21	0.173838831675686\\
22	0.0886950556373176\\
23	0.184222369341749\\
24	0.0602171869621153\\
25	0.0554222878699282\\
26	0.144894913923229\\
27	0.259138582690354\\
28	0.0463473827036197\\
29	0.136326408813607\\
30	0.0930191959540357\\
31	0.12371609733179\\
32	0.156809166070012\\
33	0.111153983064208\\
34	0.220766724816837\\
35	0.230022579735752\\
36	0.142761604348176\\
37	0.150913532749018\\
38	0.154405151621384\\
39	-0.0294960200195824\\
40	0.157307126324037\\
41	0.0475095656302157\\
42	0.0386791181444514\\
43	0.161476224077651\\
44	0.191136250237844\\
45	0.176776143966389\\
46	0.0939732556459778\\
47	0.204200170887122\\
48	0.424550880623221\\
49	0.144020314543328\\
50	0.0698174359059105\\
51	0.127687679101662\\
52	0.316509619384463\\
53	0.248117737623612\\
54	0.124371510496666\\
55	0.208886491583142\\
56	0.214228667412326\\
57	0.188064306088318\\
58	0.155777078338759\\
59	0.322047725350016\\
60	0.280624094583949\\
61	0.19757320108001\\
62	0.0874350738906626\\
63	-0.0151023490518649\\
64	0.235424496993248\\
65	0.0326258966291267\\
66	0.165608359343947\\
67	0.286711863570512\\
68	0.139080405453994\\
69	0.0931955401458295\\
70	0.161503683616151\\
71	0.165770013565127\\
72	0.223966493837649\\
73	0.19073929887774\\
74	0.36574095675131\\
75	0.146535593130328\\
76	0.230563260896064\\
77	0.0673456566490985\\
78	0.136627217487627\\
79	0.0851349757280626\\
80	0.151514254112261\\
81	0.278531581958965\\
82	0.140633225838834\\
83	0.0447319412170896\\
84	0.104874066966397\\
85	0.187521647075365\\
86	0.138686464214135\\
87	0.117956645425514\\
88	0.104604782506671\\
89	0.163154360770709\\
90	0.0729033011745306\\
91	0.264990941368631\\
92	0.162329823321742\\
93	0.0999594668515173\\
94	0.0434662501743662\\
95	0.0532909411535066\\
96	0.0262081530643639\\
97	0.179240682874133\\
98	0.106951584691941\\
99	0.0989505790322949\\
100	0.167068836860499\\
101	0.246601702597136\\
102	0.117742479350114\\
103	0.149214656694155\\
104	0.201893183385329\\
105	0.0721716455637645\\
106	0.179009240686497\\
107	0.130747435232052\\
108	0.0968144382473444\\
109	0.00950761569782395\\
110	0.193849187502614\\
111	0.316556741988046\\
112	0.0972773580949286\\
113	0.160733278146426\\
114	0.252398944836731\\
115	0.295491438238362\\
116	-0.0968912449955398\\
117	0.211274248342138\\
118	0.204990587616484\\
119	0.210164523551879\\
120	0.210867544909957\\
121	0.213400808040063\\
122	0.159464133005557\\
123	0.198893199629936\\
124	0.171433456405851\\
125	0.140188288963804\\
126	0.135215464009214\\
127	0.152881341011124\\
128	0.0761457670972769\\
129	0.0668657437985615\\
130	0.0773760206579396\\
131	0.0610678193577542\\
132	0.0631810145409384\\
133	0.0776166488189116\\
134	0.132965822295822\\
135	0.0742394003104476\\
136	0.0937795903791547\\
137	0.0944396117072238\\
138	0.0799011385574969\\
139	0.0907755703832218\\
140	0.144447946727802\\
141	0.0811941294787575\\
142	0.0728851627935245\\
143	0.0579803719635675\\
144	0.0828800556940454\\
145	0.0470926882002771\\
146	0.112712754596622\\
147	0.050692884878265\\
148	0.0641142427090044\\
149	0.018586049573798\\
150	0.081208720921814\\
151	0.0902307697592179\\
152	0.0217764091055521\\
153	0.059315368820206\\
154	0.0303699889860479\\
155	-0.00567818102576588\\
156	-0.013855552036576\\
157	0.0667253741364553\\
158	0.0640701642930541\\
159	0.0764566858800782\\
160	0.115443883202301\\
161	0.0365955959895032\\
162	0.0966460499652526\\
163	0.0914523473754671\\
164	0.100724657014396\\
165	0.0964732018164757\\
166	0.0698258889662013\\
167	0.102635320341467\\
168	0.134113711784774\\
169	0.125577126518279\\
170	0.222178988345205\\
171	0.0866631001821743\\
172	0.0659650639069685\\
173	0.0244563480601626\\
174	0.0607816957133521\\
175	0.0608616850147566\\
176	0.077458620507703\\
177	0.0791640584904546\\
178	0.091749172578733\\
179	0.0900993690880803\\
180	0.0121419852774606\\
181	0.104941054281327\\
182	0.106226881002613\\
183	0.0765421272085126\\
184	0.123213705596731\\
185	0.123533287985628\\
186	0.055787356062088\\
187	0.115019847172415\\
188	0.0771813689532249\\
189	0.129146925356191\\
190	0.165716900397474\\
191	0.126098416984155\\
192	0.0240025889892594\\
193	0.0992562391747015\\
194	0.0886494568074968\\
195	0.143915646264521\\
196	0.0852920656956927\\
197	0.076812083818993\\
198	0.132979492358577\\
199	0.0719460591369591\\
200	0.149270069290304\\
201	0.0645629352984282\\
202	0.0635900219145438\\
203	0.0426620654723346\\
204	0.142297946982187\\
205	0.0518713609233419\\
206	0.0841803891249813\\
207	0.0637596282429846\\
208	0.139401890792643\\
209	0.133365557218228\\
210	0.103145486656201\\
211	0.116369202817268\\
212	0.0629943612364632\\
213	0.0977578389807265\\
214	0.0193980264568238\\
215	0.0723188321357767\\
216	0.0709584359826685\\
217	0.130629598835936\\
218	0.12892357535481\\
219	0.0932272323586785\\
220	0.0607947283334159\\
221	0.0369955029295741\\
222	0.140757546047835\\
223	0.130888010867174\\
224	0.127851681521066\\
225	0.105680733008434\\
226	0.0496777262960564\\
227	0.052252925431684\\
228	0.131790473369414\\
229	0.0154541041723531\\
230	0.101212899601714\\
231	0.0868216474614345\\
232	0.269855169694504\\
233	0.467827369177334\\
234	0.148344388952312\\
235	0.157478288640751\\
236	0.0797374430686309\\
237	0.127831031494324\\
238	0.135819959681923\\
239	0.028376835862097\\
240	0.0927673752145075\\
};
\addlegendentry{Dscr Stoch Ctrl};

\addplot [color=mycolor3,solid,line width=1.5pt]
  table[row sep=crcr]{%
1	0.0109418068400895\\
2	0.145878834632259\\
3	0.00840977134466853\\
4	0.0439828285924121\\
5	0.0462227863971218\\
6	0.123276975480219\\
7	0.106304551970127\\
8	0.500466130890592\\
9	0.118089643974495\\
10	0.024027008310466\\
11	0.0947657189261147\\
12	0.149428645578148\\
13	-0.0404436870807658\\
14	0.112563451462946\\
15	0.0606882933443012\\
16	-0.00661088245083104\\
17	-0.138940691109414\\
18	0.00161622733695933\\
19	-0.105407089360194\\
20	-0.060433107959002\\
21	0.0975090561589264\\
22	-0.0137701058352276\\
23	0.0821009274644704\\
24	0.00234473768295178\\
25	-0.0716433364625198\\
26	0.0306850177363901\\
27	0.179407046638373\\
28	-0.20825710135667\\
29	-0.0250649262809243\\
30	-0.046874646514069\\
31	-0.0125944901064334\\
32	-0.0743018134745574\\
33	0.0671966480045781\\
34	-0.138327646701389\\
35	0.082476923599092\\
36	-0.0449672744580462\\
37	-0.0131805321034779\\
38	0.0146329418799298\\
39	-0.121466376763797\\
40	0.0174924340375877\\
41	-0.0613204026675408\\
42	-0.0135581421155556\\
43	0.0144414173730219\\
44	0.0905268074692689\\
45	0.0519784142346992\\
46	-0.088459998838066\\
47	0.046796969253838\\
48	0.156203116512954\\
49	0.15473471433827\\
50	-0.0730330715159118\\
51	0.000944632235859662\\
52	0.0609975035451418\\
53	0.129115249593117\\
54	0.0816886026587709\\
55	0.103735744507912\\
56	-0.0987276735776261\\
57	0.180672731599778\\
58	0.0829950654879805\\
59	0.114407230948071\\
60	-0.0966680083755922\\
61	-0.0250957638902034\\
62	0.0314253272579883\\
63	-0.0840202772862022\\
64	0.0447922468224132\\
65	-0.0669029355811453\\
66	-0.0120292107397511\\
67	0.0474986216119333\\
68	0.0797352358583845\\
69	-0.158987124630962\\
70	0.0460094411566786\\
71	0.0337594795197594\\
72	0.0215496387555915\\
73	0.0988449935513265\\
74	0.132520249095029\\
75	-0.104789353351203\\
76	-0.0789238497263615\\
77	0.0195957170096018\\
78	0.0391288842926782\\
79	-0.0216270465561866\\
80	0.0523778697792018\\
81	0.0539092457677933\\
82	-0.00761281630562344\\
83	-0.074930186611399\\
84	-0.148654531689077\\
85	-0.0149304191812535\\
86	-0.0456195357660466\\
87	-0.0336696107135641\\
88	-0.0198176716035008\\
89	-0.0237195155828747\\
90	-0.335598921465332\\
91	-0.134649814258177\\
92	-0.0984485515198874\\
93	-0.0952073468963271\\
94	-0.00662518541499159\\
95	-0.0163127052484753\\
96	-0.179603525680058\\
97	0.0296975572832644\\
98	-0.032622822317478\\
99	0.0776828619047956\\
100	-0.0955040148838274\\
101	0.0440574533905225\\
102	-0.0838289339201449\\
103	-0.0862236636975325\\
104	-0.00817334818549639\\
105	0.0811713934920186\\
106	-0.019590235609851\\
107	0.0294520330682153\\
108	-0.0286529274687698\\
109	-0.328817454572584\\
110	-0.527812092324924\\
111	0.230907037070111\\
112	0.0409853877969768\\
113	0.00869906959706923\\
114	-0.214616950890463\\
115	-0.0878386738937054\\
116	-0.226810944060554\\
117	0.080558191630008\\
118	0.0787376586396832\\
119	-0.0301520248005351\\
120	0.118010003653557\\
121	-0.0171988802068232\\
122	0.0202083413283912\\
123	-0.177809684428788\\
124	-0.0608281578444691\\
125	0.0516497066578072\\
126	0.0206662602518183\\
127	0.00867666965859449\\
128	-0.145586911058566\\
129	0.0184556556871515\\
130	-0.0268410415795488\\
131	0.00908303519926278\\
132	0.121409745837948\\
133	0.0359239516831623\\
134	0.0678644035434147\\
135	0.0903842582521414\\
136	-0.0585251997721324\\
137	-0.0447806576784392\\
138	-0.0209519860753186\\
139	0.00721439443024314\\
140	-0.0447676368453893\\
141	-0.0660017964094734\\
142	-0.0275399308853287\\
143	-0.0387822206962715\\
144	-0.0458608761039165\\
145	-0.105390591321028\\
146	-0.0342478236669038\\
147	-0.0168491734330536\\
148	0.022098230198952\\
149	-0.131764616515259\\
150	-0.0700485201844154\\
151	-0.0325076628581467\\
152	-0.0179182653495157\\
153	0.0590234864036128\\
154	-0.0901377462546244\\
155	-0.0527608535680072\\
156	0.0207403445438522\\
157	0.03053987361856\\
158	-0.239092544684608\\
159	-0.0708837727828399\\
160	-0.0213406128421758\\
161	0.0215174122532895\\
162	0.00935327477704696\\
163	-0.0237277352749333\\
164	-0.102155600817695\\
165	0.00952620227369642\\
166	-0.0569473268258413\\
167	-0.13853814511086\\
168	-0.0452574574847635\\
169	-0.0894066831986427\\
170	-0.296973259070747\\
171	0.071847862859746\\
172	-0.115167160220225\\
173	-0.011577149730968\\
174	-0.148201450245767\\
175	-0.103206325428874\\
176	-0.151904818384523\\
177	-0.0505014241429798\\
178	-0.19258058808214\\
179	-0.122016976310281\\
180	-0.104626715711469\\
181	-0.0530244665156968\\
182	-0.0539844189779024\\
183	-0.111756018244114\\
184	-0.0244708380017296\\
185	0.0210803067977749\\
186	-0.062673942733755\\
187	0.017654761451592\\
188	-0.063271199506945\\
189	-0.060097204798916\\
190	-0.111202435190112\\
191	-0.0542101791684325\\
192	0.0174020116922894\\
193	-0.00138711837754103\\
194	0.0369310438569653\\
195	0.0546130069888784\\
196	-0.0332784543482989\\
197	-0.0383237014017327\\
198	-0.0221978573299285\\
199	-0.138590196388663\\
200	-0.0907861470349187\\
201	0.120435729778013\\
202	0.004132231550445\\
203	0.0296689846708275\\
204	-0.00424960309322448\\
205	-0.05266630301145\\
206	0.0427530077108709\\
207	0.0164248431113527\\
208	0.0971378954763436\\
209	-0.0189748475353819\\
210	0.0302643281886083\\
211	0.0696524218591655\\
212	0.0318520491321496\\
213	-0.0721762693711667\\
214	0.0425615578128655\\
215	-0.069847806051081\\
216	0.0375617487782085\\
217	0.043568261554469\\
218	0.0856928174385157\\
219	0.00125275551296378\\
220	-0.0427781503450971\\
221	0.000336150994249034\\
222	0.0521792135351439\\
223	0.0360084786341543\\
224	0.0498991208105158\\
225	-0.00175725938858798\\
226	0.029125300301572\\
227	-0.0196330065279931\\
228	-0.117394149463417\\
229	0.0477810863996858\\
230	0.0505065195961207\\
231	0.01334884575943\\
232	0.144975856733256\\
233	-0.268790341265156\\
234	0.07914064675577\\
235	0.0255335704288297\\
236	-0.0397899714597549\\
237	0.0554887644016089\\
238	-0.0621743058775188\\
239	0.0311680076379203\\
240	0.0558388221293866\\
};
\addlegendentry{Cts Stoch Ctrl w NMC};

\addplot [color=mycolor4,solid,line width=1.5pt]
  table[row sep=crcr]{%
1	0.117522439302889\\
2	0.193714637909439\\
3	0.123929666788155\\
4	0.197049044642514\\
5	0.196741908614173\\
6	0.203276744323998\\
7	0.151442786258534\\
8	0.515631578925498\\
9	0.174761457375557\\
10	0.113943418018702\\
11	0.17732934325146\\
12	0.357976994655604\\
13	0.260072384965373\\
14	0.0779663072622854\\
15	0.15331852717122\\
16	0.0141890393194416\\
17	-0.0985477725873703\\
18	0.119917594298331\\
19	0.184108978333709\\
20	0.231928204355273\\
21	0.160007605473972\\
22	0.11651153809355\\
23	0.0714925474706558\\
24	0.0411507665190115\\
25	0.0608148687413444\\
26	0.142368659730332\\
27	0.259574559352778\\
28	-0.000225617507666213\\
29	0.0817481942431757\\
30	0.119646443068115\\
31	0.141367471535399\\
32	-0.00271194585200402\\
33	0.264725023949842\\
34	0.235708479038162\\
35	-0.0367915336278034\\
36	0.126199239945149\\
37	0.17232339664511\\
38	0.144672170659601\\
39	-0.0509230029957569\\
40	0.153658198983958\\
41	0.0531559174436159\\
42	0.173256452518617\\
43	0.183248008774112\\
44	0.154017730793622\\
45	0.162582285547935\\
46	0.213947291144197\\
47	0.192688391917162\\
48	0.456095667365722\\
49	0.123815233021952\\
50	0.0283121951575793\\
51	0.138765021716711\\
52	0.359246118361702\\
53	0.240163035817193\\
54	0.165847841289187\\
55	0.196903878209283\\
56	0.120099563595465\\
57	0.234132857822863\\
58	0.17343793387391\\
59	0.34006916179467\\
60	0.29408677567746\\
61	0.184165853680406\\
62	0.0855733369314742\\
63	0.195588250275169\\
64	0.28447593148572\\
65	0.0102000830937004\\
66	0.101335375622015\\
67	0.316386493706967\\
68	0.0943858279056424\\
69	0.188981078462164\\
70	0.209841998632369\\
71	0.138478152135852\\
72	0.222550470727845\\
73	0.18489860809711\\
74	0.363786262449125\\
75	0.179650842731286\\
76	0.289483241634642\\
77	0.0295019247261798\\
78	0.150599675938237\\
79	0.105417826738771\\
80	0.135691261963458\\
81	0.295644285678406\\
82	0.150874944464978\\
83	0.0545669374461137\\
84	0.0873131379979612\\
85	0.19550109907606\\
86	0.119086071425362\\
87	0.106233947871963\\
88	0.0561278617824999\\
89	0.173697362259545\\
90	0.274176988551816\\
91	0.305926651371315\\
92	0.155653635886955\\
93	0.0573664571729875\\
94	0.102821300614347\\
95	0.0664114582579276\\
96	0.0282669561140941\\
97	0.186042783853509\\
98	0.146111602289072\\
99	0.322645366393515\\
100	0.172001204607572\\
101	0.178900952253933\\
102	0.0817572551815943\\
103	0.188758635781794\\
104	-0.117634568405045\\
105	-0.00187137047687233\\
106	0.216451560884194\\
107	0.0974223269735625\\
108	0.131927563954563\\
109	-0.0217154021619471\\
110	0.165111976471387\\
111	0.331079821873786\\
112	0.0657827271909845\\
113	0.136327083558669\\
114	0.251579013349062\\
115	0.291330408771043\\
116	-0.192229030409068\\
117	0.213322437441027\\
118	0.200027651337103\\
119	0.223880872923399\\
120	0.2352026106347\\
121	0.22842888734839\\
122	0.14409797354605\\
123	0.0709912760727584\\
124	0.185754939760091\\
125	0.140282997990943\\
126	0.153200693508223\\
127	0.151588996855974\\
128	0.0633639445554077\\
129	0.0546685960226589\\
130	0.0983179524532892\\
131	0.0606520765779729\\
132	0.0971383284516079\\
133	0.049631003466941\\
134	0.0951527364190741\\
135	0.0887399282346796\\
136	0.112079108429553\\
137	0.0530819187221686\\
138	0.065453954340685\\
139	0.0848063167436592\\
140	0.170373909268454\\
141	0.0965789706188669\\
142	0.0698544663643723\\
143	0.0263559498605503\\
144	0.0109615419713951\\
145	0.0476218969985001\\
146	0.0573330062587148\\
147	0.138634665850133\\
148	0.0463345241101315\\
149	0.0911891381636153\\
150	0.0532012411551162\\
151	0.149601621123438\\
152	0.109804357655505\\
153	0.062630262902223\\
154	0.138119538371798\\
155	0.0355255564446561\\
156	0.0586076575792843\\
157	0.0713423187238186\\
158	0.101763964638775\\
159	0.170611506126705\\
160	0.135508862104957\\
161	0.108134521168962\\
162	0.0939754587806221\\
163	0.12775658034352\\
164	0.130360203153521\\
165	0.19723414203133\\
166	0.204249119465143\\
167	0.0688947815331206\\
168	0.115628748792134\\
169	0.183185443841869\\
170	0.241802923080661\\
171	0.128708735109009\\
172	0.0623649121456092\\
173	0.0957188186721063\\
174	0.0788766427076832\\
175	0.0477373231714646\\
176	0.118147561798162\\
177	0.141350252646352\\
178	0.100574952960529\\
179	0.0866405299435105\\
180	0.172138509760234\\
181	0.124250831156419\\
182	0.171243673466863\\
183	0.117832392224321\\
184	0.130135808112528\\
185	0.240905574362623\\
186	0.14967302678588\\
187	0.198516613063936\\
188	0.0822103927105676\\
189	0.173489204880383\\
190	0.179844945892797\\
191	0.204214436623029\\
192	0.161743898901856\\
193	0.113713550348592\\
194	0.0767547382265253\\
195	0.178596296045314\\
196	0.107042005726179\\
197	0.0970248591337719\\
198	0.186066909464693\\
199	0.0181066151886708\\
200	0.14210058370006\\
201	0.0667126876463618\\
202	0.0636870802454166\\
203	0.0381247566042119\\
204	0.200432813975418\\
205	0.0500092295437106\\
206	-0.0380267829305514\\
207	0.0495462285338741\\
208	0.0221751660978788\\
209	0.0339316530600687\\
210	0.118316377604758\\
211	0.0452085236356668\\
212	0.0889751281956243\\
213	0.0912477817340114\\
214	0.0302893356311725\\
215	0.0803666279453793\\
216	0.0763116015300408\\
217	0.0888185395726834\\
218	0.157442893440473\\
219	0.108562899370013\\
220	0.0207624886165965\\
221	0.0804043858439082\\
222	0.12329431091289\\
223	0.0937410056368895\\
224	0.151374053547168\\
225	0.107806900632565\\
226	0.242945555862427\\
227	0.161135749395884\\
228	0.143708820165988\\
229	0.0241578259737574\\
230	0.0860202367078583\\
231	0.0952288237094876\\
232	0.284729820159629\\
233	0.480241305839283\\
234	0.119121831351708\\
235	0.203526683032668\\
236	0.0775028183851073\\
237	0.156236473007947\\
238	0.13776956097407\\
239	0.0353417569283897\\
240	0.125371603723996\\
};
\addlegendentry{Dscr Stoch Ctrl w NMC};

\end{axis}
\end{tikzpicture}%

\end{subfigure}%
\hfill%
\begin{subfigure}{.45\linewidth}
  \centering
  \setlength\figureheight{\linewidth} 
  \setlength\figurewidth{\linewidth}
  \tikzsetnextfilename{IS_week_INTC}
  % This file was created by matlab2tikz.
%
%The latest updates can be retrieved from
%  http://www.mathworks.com/matlabcentral/fileexchange/22022-matlab2tikz-matlab2tikz
%where you can also make suggestions and rate matlab2tikz.
%
\definecolor{mycolor1}{rgb}{0.25098,0.00000,0.38824}%
\definecolor{mycolor2}{rgb}{0.00000,0.46275,0.00000}%
\definecolor{mycolor3}{rgb}{0.00000,0.34902,0.34902}%
\definecolor{mycolor4}{rgb}{0.58039,0.26275,0.00000}%
%
\begin{tikzpicture}[trim axis left, trim axis right]

\begin{axis}[%
width=\figurewidth,
height=\figureheight,
at={(0\figurewidth,0\figureheight)},
scale only axis,
every outer x axis line/.append style={black},
every x tick label/.append style={font=\color{black}},
xmin=1,
xmax=240,
%xlabel={Time (h)},
every outer y axis line/.append style={black},
every y tick label/.append style={font=\color{black}},
ymin=-1.1,
ymax=1.1,
%ylabel={Normalized PnL},
title={INTC},
axis background/.style={fill=white},
axis x line*=bottom,
axis y line*=left,
yticklabel style={
        /pgf/number format/fixed,
        /pgf/number format/precision=3
},
scaled y ticks=false,
legend style={legend cell align=left,align=left,draw=black,font=\small, legend pos=north west},
every axis legend/.code={\renewcommand\addlegendentry[2][]{}}  %ignore legend locally
]
\addplot [color=mycolor1,solid,line width=1.5pt]
  table[row sep=crcr]{%
1	0.331396165929232\\
2	0.339499560527892\\
3	0.232328043849591\\
4	0.197393490825541\\
5	0.187155719449737\\
6	0.200682473766741\\
7	0.412030727012545\\
8	0.35601657438774\\
9	0.163684184463028\\
10	0.182946063259209\\
11	0.172032262672843\\
12	0.174864171813871\\
13	0.261566506485612\\
14	0.151522528504462\\
15	0.142727887295511\\
16	0.172723516061412\\
17	0.184583636968767\\
18	0.228810977335256\\
19	0.213349599109731\\
20	0.281919660012566\\
21	0.215171620276083\\
22	0.139159231125741\\
23	0.246242067987401\\
24	0.130420505342434\\
25	0.170657309778014\\
26	0.147901755594369\\
27	0.125948448738494\\
28	0.186841504058438\\
29	0.00559335767240588\\
30	0.160846370738877\\
31	0.462772567964411\\
32	0.347895542091314\\
33	0.142161610438638\\
34	0.291376159385902\\
35	0.234377829130777\\
36	0.254543649532816\\
37	0.26089771889818\\
38	0.20834383439368\\
39	0.112842876861812\\
40	0.282708033243972\\
41	0.132373693541792\\
42	0.204999204462943\\
43	0.191642904976306\\
44	0.0791157700167747\\
45	0.23253508553312\\
46	0.258140655442665\\
47	0.185172214396056\\
48	0.269169380874173\\
49	0.201375091344223\\
50	0.203393862379776\\
51	0.00366377257170145\\
52	0.35212774409079\\
53	0.147912902166732\\
54	0.1099710341438\\
55	0.17431052186199\\
56	0.11078847173706\\
57	0.27719300493022\\
58	0.235813610013044\\
59	0.297249380549499\\
60	0.381681392360562\\
61	0.37292235246887\\
62	0.310305236672697\\
63	0.192715494550387\\
64	0.513016301733233\\
65	0.39527714349713\\
66	0.473736729339163\\
67	0.414661375366238\\
68	0.303235645894483\\
69	0.454398148670253\\
70	0.450480002206538\\
71	0.0284376859931493\\
72	0.267376965048921\\
73	0.206263380914674\\
74	0.281272246948003\\
75	0.120186737962823\\
76	0.257181072375543\\
77	0.199314056667815\\
78	0.102551826069036\\
79	0.131731636196664\\
80	0.229895694997589\\
81	0.186041980900323\\
82	0.151277440577492\\
83	0.12629960725065\\
84	0.0765255571591537\\
85	0.209743306206626\\
86	0.0675038498785687\\
87	0.202461621892708\\
88	0.227436013723439\\
89	0.252333893016006\\
90	0.206217049333736\\
91	0.282054259749008\\
92	0.257341327129602\\
93	0.217129960209356\\
94	0.332322480792552\\
95	0.247922733285289\\
96	0.174838578493935\\
97	0.236392619359868\\
98	0.261828324054605\\
99	0.158428430462519\\
100	0.342854593637918\\
101	0.341170741327908\\
102	0.23907081865857\\
103	0.169479645423366\\
104	0.375611643634223\\
105	0.240994824258382\\
106	0.257928213532449\\
107	0.0546482155050077\\
108	0.184666550029001\\
109	0.205457942223012\\
110	0.29714493301848\\
111	0.0917917241698569\\
112	0.275552181780701\\
113	0.204890118860845\\
114	0.269997714887668\\
115	0.476751883158293\\
116	0.144670618224455\\
117	0.173428940452699\\
118	0.156269682517367\\
119	0.364237850831936\\
120	-0.115324549293291\\
121	0.0853628463398832\\
122	0.141094818844601\\
123	0.169126983818796\\
124	0.234607316033912\\
125	0.239551026639516\\
126	0.087215096382643\\
127	0.304838454261983\\
128	0.195457975725964\\
129	0.100972716893283\\
130	0.0863760760545377\\
131	0.149572380384725\\
132	0.276045381896443\\
133	0.280013038603723\\
134	0.186258060209268\\
135	0.200429073352436\\
136	0.195578757832808\\
137	0.148190948773964\\
138	0.28583211952428\\
139	0.251036536316024\\
140	0.230860747909843\\
141	0.158839227327434\\
142	0.0916500073104917\\
143	0.156287547561436\\
144	0.233617238564273\\
145	0.188228953514499\\
146	0.171158554219468\\
147	0.154081577052322\\
148	0.0622471289795272\\
149	0.259402956603548\\
150	0.271502326866788\\
151	-0.10060751455371\\
152	-0.139074807257041\\
153	0.213535759940408\\
154	0.0924034655681529\\
155	0.348920501470298\\
156	0.195452437404994\\
157	0.109835696601448\\
158	0.231827550536351\\
159	0.234788159073377\\
160	0.220621074150899\\
161	0.176291802370658\\
162	0.235761737355467\\
163	0.138998875283066\\
164	0.140666989609675\\
165	0.108207981570183\\
166	0.483907363104017\\
167	0.0349358815471575\\
168	0.187229953398813\\
169	0.31027402624019\\
170	0.199015927057617\\
171	0.0352144648524511\\
172	0.209246576078776\\
173	0.104267401983979\\
174	0.108414092446744\\
175	0.0859657807463441\\
176	0.0962301606047842\\
177	0.162055011565105\\
178	0.170216251394938\\
179	0.179749324478075\\
180	0.258227752002927\\
181	0.241996935939608\\
182	0.191658750598612\\
183	0.18754644678136\\
184	0.256807138275276\\
185	0.252857408032043\\
186	0.282199883870709\\
187	0.180111190075556\\
188	0.291396360105658\\
189	0.233430677193213\\
190	0.300815151407373\\
191	0.172997606444136\\
192	0.28653597005103\\
193	0.193495394545922\\
194	0.107924701857685\\
195	0.154758975897866\\
196	0.237898723751482\\
197	0.298265044594303\\
198	0.189354294478767\\
199	0.140597279265902\\
200	0.134596492053173\\
201	0.0861547112516496\\
202	0.20521091733216\\
203	0.173426599329909\\
204	0.168969798598249\\
205	0.152523540885011\\
206	0.206229119521072\\
207	0.0976579696137138\\
208	-0.00052714493799049\\
209	0.273780106347652\\
210	0.225342111888984\\
211	0.242345275420585\\
212	0.221112279550754\\
213	0.210286467652625\\
214	0.150701491463962\\
215	0.353641582992934\\
216	0.190098992939186\\
217	0.247625593672221\\
218	0.0971579894278747\\
219	0.19531994041142\\
220	0.0577919934340389\\
221	0.14909675795604\\
222	0.272251760887914\\
223	0.298766493925412\\
224	0.334485682858427\\
225	0.140286032574761\\
226	0.183874510484128\\
227	-0.0110254331231015\\
228	0.0520541985910703\\
229	0.0491171985192197\\
230	0.059079545184912\\
231	0.188934566487484\\
232	0.301985520520548\\
233	0.166160641556414\\
234	0.0240694899771623\\
235	0.155298633663994\\
236	0.0757047514639142\\
237	0.218416052396393\\
238	0.0437393522047888\\
239	0.218708176172619\\
240	0.130366145827001\\
};
\addlegendentry{Cts Stoch Ctrl};

\addplot [color=mycolor2,solid,line width=1.5pt]
  table[row sep=crcr]{%
1	0.258333332954965\\
2	0.298395915529189\\
3	0.204460619410076\\
4	0.264537072893063\\
5	0.194028835985588\\
6	0.290043130874557\\
7	0.348624890845606\\
8	0.35376605128522\\
9	0.267731020440209\\
10	0.141381343936286\\
11	0.107878453349953\\
12	0.177655570640551\\
13	0.272815031263431\\
14	0.199514888824382\\
15	0.267999262376953\\
16	0.354507053216405\\
17	0.234410193587408\\
18	0.230534176127727\\
19	0.199747528094638\\
20	0.350938655199766\\
21	0.216702474382561\\
22	0.114961760483312\\
23	0.194664585765652\\
24	0.127795872755482\\
25	0.157200473697505\\
26	0.167493121103629\\
27	0.0721925917056041\\
28	0.171111866732889\\
29	0.175354876754656\\
30	0.207560652220519\\
31	0.40887494508836\\
32	0.309137534629374\\
33	0.219826378984648\\
34	0.308612699856069\\
35	0.275133411727778\\
36	0.313865407449887\\
37	0.239210558033938\\
38	0.235689301957969\\
39	0.158216734915965\\
40	0.286676782585371\\
41	0.211368003680231\\
42	0.177475200397231\\
43	0.189314930582791\\
44	0.108170016063631\\
45	0.188442385264394\\
46	0.15584155352183\\
47	0.169407296464634\\
48	0.261543002758309\\
49	0.175792619966598\\
50	0.182953346845535\\
51	0.368667868260818\\
52	0.277245186634247\\
53	0.144979832605964\\
54	0.0574344310237376\\
55	0.131839709986538\\
56	0.152033815912936\\
57	0.193663694314966\\
58	0.212576282637062\\
59	0.363073434156807\\
60	0.521575479588552\\
61	0.410403309034656\\
62	0.232375464914921\\
63	0.19794554132765\\
64	0.548604400931705\\
65	0.559621634035289\\
66	0.415365188282705\\
67	0.298417289044937\\
68	0.337567002987758\\
69	0.468130975862967\\
70	0.414045770081976\\
71	0.163477531651022\\
72	0.368690000407711\\
73	0.243850048745596\\
74	0.420018345680921\\
75	0.3222049914706\\
76	0.323600313651368\\
77	0.143580283982501\\
78	0.123218850479158\\
79	0.134753391844057\\
80	0.279129752713564\\
81	0.161562231784907\\
82	0.206015154260378\\
83	0.123146862553987\\
84	0.059725079087019\\
85	0.255374408832185\\
86	0.115968479223038\\
87	0.200505032799341\\
88	0.275391614774295\\
89	0.221220475366963\\
90	0.217397467591595\\
91	0.228009217698381\\
92	0.194216045485652\\
93	0.211797652459087\\
94	0.258855238671246\\
95	0.197983753993186\\
96	0.281002729397494\\
97	0.335487447234812\\
98	0.225876259127117\\
99	0.244016400175303\\
100	0.392929861837949\\
101	0.392713878074444\\
102	0.206742453910765\\
103	0.143318971072785\\
104	0.38639609995595\\
105	0.222693636582825\\
106	0.287398216102626\\
107	0.259245640933127\\
108	0.164391671686726\\
109	0.215785589565327\\
110	0.296681475183561\\
111	0.150769021537656\\
112	0.345731647608159\\
113	0.153837507573348\\
114	0.329308186018068\\
115	0.517449725627535\\
116	0.252682418242897\\
117	0.196645416834867\\
118	0.199004907643989\\
119	0.264827645469955\\
120	0.254807080459137\\
121	0.170303052930315\\
122	0.237393767625636\\
123	0.325365464025871\\
124	0.239881071150188\\
125	0.262581422452126\\
126	0.200218309008649\\
127	0.292323794910911\\
128	0.170192237965674\\
129	0.111783371502943\\
130	0.124641064092572\\
131	0.124932328295501\\
132	0.303711270155845\\
133	0.263464289045227\\
134	0.165222789319563\\
135	0.158291183586955\\
136	0.144916816511215\\
137	0.143712800781857\\
138	0.264111554009897\\
139	0.213144415567275\\
140	0.235821628012592\\
141	0.189318631641625\\
142	0.157050194742795\\
143	0.229485762168262\\
144	0.252461030128294\\
145	0.213214105989781\\
146	0.216827782904002\\
147	0.225418533377847\\
148	0.142295964844881\\
149	0.293273847340325\\
150	0.230694473401839\\
151	-0.0290205589527297\\
152	0.434314809037508\\
153	0.21459631022965\\
154	0.106030697649942\\
155	0.281527526696193\\
156	0.206550133011625\\
157	0.10552874054675\\
158	0.312181138697747\\
159	0.218242480844117\\
160	0.255238413553204\\
161	0.163903650890966\\
162	0.234358318517292\\
163	0.126576310836501\\
164	0.10933802770406\\
165	0.115327257630403\\
166	0.491016467956258\\
167	0.093757686457163\\
168	0.192926585170794\\
169	0.178573401887598\\
170	0.202110611943852\\
171	0.12642510417572\\
172	0.189080339845646\\
173	0.16983946905113\\
174	0.115307657091625\\
175	0.207716283446279\\
176	0.163308247534264\\
177	0.25780752848014\\
178	0.156230348988371\\
179	0.164409323338058\\
180	0.348233303518176\\
181	0.169747414507807\\
182	0.201973744532911\\
183	0.237984198370188\\
184	0.315310723622873\\
185	0.228023350302429\\
186	0.305665418061777\\
187	0.221124458755576\\
188	0.264628158716188\\
189	0.299405388580942\\
190	0.307069658467044\\
191	0.171963812712652\\
192	0.300471630654666\\
193	0.19181371750441\\
194	0.120457413360201\\
195	0.163388899984384\\
196	0.19291988058835\\
197	0.309929306966694\\
198	0.253187510897678\\
199	0.170589840671008\\
200	0.150686101339681\\
201	0.0795666788394481\\
202	0.191510242058821\\
203	0.223195481602274\\
204	0.184635372461343\\
205	0.190701865687302\\
206	0.192178851757808\\
207	0.111384067996687\\
208	0.209319393784229\\
209	0.283444064222149\\
210	0.176259046557202\\
211	0.195980508062143\\
212	0.157087947486856\\
213	0.215830191036432\\
214	0.175110057113848\\
215	0.34003388168324\\
216	0.246261042321727\\
217	0.231807908796562\\
218	0.152812342322448\\
219	0.20432434531518\\
220	0.116735867179073\\
221	0.164762705388815\\
222	0.219568779865992\\
223	0.262637931904502\\
224	0.299628542292272\\
225	0.190477112331715\\
226	0.230601207988424\\
227	0.0475988718710717\\
228	0.123431475627929\\
229	0.15349633391494\\
230	0.137010963157796\\
231	0.232134636089899\\
232	0.324024915436665\\
233	0.222921603271774\\
234	0.268069673236259\\
235	0.228965930981914\\
236	0.138180232581363\\
237	0.228457999486799\\
238	0.100837622403079\\
239	0.23211054311038\\
240	0.161369278141647\\
};
\addlegendentry{Dscr Stoch Ctrl};

\addplot [color=mycolor3,solid,line width=1.5pt]
  table[row sep=crcr]{%
1	0.222263544979983\\
2	0.198970727955836\\
3	0.169310262332087\\
4	0.0912973472287008\\
5	0.173651820119614\\
6	0.239583164062507\\
7	-0.245476694886574\\
8	0.174925499866295\\
9	0.0680706430093557\\
10	0.157516659413538\\
11	0.0963856778844056\\
12	0.0892999584982044\\
13	-0.00176492844837237\\
14	0.115167579841054\\
15	0.0364517122605558\\
16	0.26829907687146\\
17	0.214228000551541\\
18	0.193554284591098\\
19	0.154239469944987\\
20	0.293929077121696\\
21	0.18854769543431\\
22	0.130327349308429\\
23	0.141708361014342\\
24	0.0832319728063623\\
25	0.15614472821394\\
26	0.0967042968054644\\
27	0.127208849008037\\
28	0.142079724750189\\
29	-0.0479619881322133\\
30	0.0892294670628884\\
31	0.456779897933269\\
32	0.0567962565264658\\
33	0.0583162148837402\\
34	0.189257565320723\\
35	0.17685619700167\\
36	0.261157895701715\\
37	0.0683143416369446\\
38	0.124588910574566\\
39	-0.0185815047606836\\
40	0.167890119206015\\
41	0.064563428605367\\
42	0.145843118442487\\
43	0.174248817777444\\
44	0.0378259429956586\\
45	0.21059007305075\\
46	0.0977694810039008\\
47	0.146351159366483\\
48	0.115176947943702\\
49	0.0614340856714658\\
50	0.118935770061037\\
51	0.321129814564622\\
52	0.231855707053304\\
53	0.0346677316525492\\
54	0.0723543314181321\\
55	0.098054996954259\\
56	0.0125058064020707\\
57	0.127311308232199\\
58	0.11313964609865\\
59	0.230723532227329\\
60	0.0530659720434264\\
61	0.222639583307394\\
62	0.151346094798832\\
63	0.0538002873713941\\
64	0.256445775476438\\
65	0.086102613386259\\
66	0.153692825836058\\
67	0.216198560426405\\
68	0.118637674059301\\
69	0.164226326535527\\
70	0.153463201655866\\
71	0.14544786723175\\
72	-0.0374005200029325\\
73	0.0396359688827504\\
74	0.153506821585801\\
75	-0.0285609755531888\\
76	0.190453598770038\\
77	0.201402188396788\\
78	0.00786736934922523\\
79	0.122893831598667\\
80	0.0311697740142221\\
81	0.0601400445151178\\
82	0.0859423969598438\\
83	-0.0224194326125029\\
84	-0.0600735038533097\\
85	0.0675887153718381\\
86	-0.0611846187475456\\
87	0.141916685254974\\
88	0.204401464139166\\
89	0.112019687506728\\
90	0.119988482990356\\
91	0.131665774338503\\
92	0.122983586224832\\
93	0.0865212242274805\\
94	0.110947215307438\\
95	-0.000743889725715881\\
96	-0.0671739297741392\\
97	0.0622732145249389\\
98	0.0390279660461109\\
99	-0.00481379357654755\\
100	0.118590306347889\\
101	0.128082594350603\\
102	0.0296610151694145\\
103	0.0639681453031256\\
104	0.157516505382841\\
105	0.208561931646692\\
106	0.213239520374583\\
107	0.102162200860606\\
108	-0.0527594320045033\\
109	-0.051380484793228\\
110	0.0868102464672762\\
111	-0.0691089289026354\\
112	0.271470296006428\\
113	0.217230860351698\\
114	0.108924567404139\\
115	0.215944703670225\\
116	0.0623487461021289\\
117	0.120114863901387\\
118	0.118440628177787\\
119	0.1648179163908\\
120	-0.204111674658723\\
121	0.164222719830404\\
122	-0.0465832206163647\\
123	0.37914444890461\\
124	0.222533055307076\\
125	0.107047112388434\\
126	0.0155077164104214\\
127	0.238052378971289\\
128	0.15505731929797\\
129	0.078774639017682\\
130	0.032030375820157\\
131	0.0213811545568484\\
132	0.125799470415069\\
133	0.24500326793308\\
134	0.200201285031303\\
135	0.103378102876811\\
136	0.0263338064988217\\
137	0.133048567597795\\
138	0.277908890894589\\
139	0.210647826861229\\
140	0.201419127130224\\
141	0.125624400125455\\
142	0.0710552965948081\\
143	0.0699236324174838\\
144	0.221989622609864\\
145	0.112480349000936\\
146	0.0736233962979803\\
147	0.0554132695274645\\
148	0.166062207505647\\
149	0.108744265969822\\
150	0.141065973325588\\
151	0.042384884228189\\
152	-0.187711816121461\\
153	0.164985041377612\\
154	0.0471598023015995\\
155	0.15841873469733\\
156	0.178194085904888\\
157	0.103533648435791\\
158	0.244930318815808\\
159	0.141433280301853\\
160	0.0100685663924666\\
161	0.223703250082627\\
162	0.226642172970313\\
163	0.0344924108116539\\
164	0.118934798955324\\
165	0.144287147385437\\
166	0.119690123089361\\
167	-0.054667435632899\\
168	0.131515424593767\\
169	0.0511534862318332\\
170	0.15829546431751\\
171	0.0816477239829299\\
172	0.198986572935073\\
173	0.058119449290335\\
174	0.0154527156878105\\
175	0.00432947808770741\\
176	0.0507748476349964\\
177	0.0655263846953395\\
178	-0.0243542346727458\\
179	0.0900148775303378\\
180	0.327811157135771\\
181	0.0853700157049899\\
182	0.0280877814698996\\
183	0.0443266562505952\\
184	-0.00478779993089406\\
185	0.185080069839642\\
186	0.141368505121706\\
187	0.0385564626016761\\
188	0.0107984086131524\\
189	0.309174999086628\\
190	0.142794079666083\\
191	0.0866662105881236\\
192	0.0747867724204601\\
193	0.122154897912091\\
194	-0.033878869324388\\
195	0.127752103886536\\
196	0.117556375180358\\
197	0.223381078835774\\
198	0.0636512001013263\\
199	0.0411018843320269\\
200	0.0665209616784442\\
201	0.0034822650256495\\
202	0.180577771885791\\
203	0.098043797933666\\
204	0.0982138996781343\\
205	0.143793466483495\\
206	0.206028069048729\\
207	0.0581586532885447\\
208	-0.000660918346393593\\
209	0.145873773214626\\
210	0.155977134829783\\
211	0.149805124882246\\
212	0.161150072278889\\
213	0.0632454266821189\\
214	0.124546356489698\\
215	-0.0259147557937289\\
216	0.17240477549684\\
217	0.254343278339333\\
218	0.0265230558401982\\
219	0.128974739237414\\
220	-0.0160821089086349\\
221	0.11242760220697\\
222	0.210078385990293\\
223	0.147920332022308\\
224	0.156700244992677\\
225	0.140204815123486\\
226	0.115769063010685\\
227	-0.0902461393897774\\
228	-0.0577940379838994\\
229	0.0136258129000245\\
230	-0.0038275794047052\\
231	0.0594622397943263\\
232	-0.0185177454823751\\
233	-0.021099347311368\\
234	0.00366900162136685\\
235	0.148855295962291\\
236	-0.0429862627762104\\
237	0.200662917717979\\
238	0.0145005061079729\\
239	0.213501695674985\\
240	0.0145070136869115\\
};
\addlegendentry{Cts Stoch Ctrl w NMC};

\addplot [color=mycolor4,solid,line width=1.5pt]
  table[row sep=crcr]{%
1	0.336471609378567\\
2	0.321901294212283\\
3	0.235637541051299\\
4	0.209567440381408\\
5	0.199943897443242\\
6	0.325737707160169\\
7	0.366342338959215\\
8	0.358690803445911\\
9	0.284912139158231\\
10	0.120970256205192\\
11	0.0718793308997572\\
12	0.196037788187212\\
13	0.286110669682671\\
14	0.150864562185575\\
15	0.223967996687471\\
16	0.297118293543899\\
17	0.206895412737444\\
18	0.202657897266101\\
19	0.24845499023069\\
20	0.330014448597852\\
21	0.140739592158399\\
22	0.133881057088247\\
23	0.198959963207385\\
24	0.127717038457593\\
25	0.16665259999308\\
26	0.147283365473374\\
27	0.104950386922479\\
28	0.330240919119253\\
29	0.115057256357791\\
30	0.190303890197856\\
31	0.443960766772709\\
32	0.35098610972577\\
33	0.186367214921145\\
34	0.32853383736874\\
35	0.233873946740099\\
36	0.339381876811973\\
37	0.259994707452077\\
38	0.128036514193369\\
39	0.115246714317089\\
40	0.283972916843903\\
41	0.213083135863214\\
42	0.114903358806584\\
43	0.193165214478117\\
44	0.0556806464852635\\
45	0.181773927933091\\
46	0.17394854504711\\
47	0.18309774144252\\
48	0.225072513157133\\
49	0.188569552091848\\
50	0.240912591060741\\
51	-0.117791703153207\\
52	0.204186368310798\\
53	0.123050499607986\\
54	0.0754905978466561\\
55	0.147646993405581\\
56	0.430238704229078\\
57	0.223424225258602\\
58	0.0903696073776\\
59	0.364015430420251\\
60	0.486715309260562\\
61	0.391788180484186\\
62	0.215197045617866\\
63	0.16897447583751\\
64	0.526687819575454\\
65	0.647345693571796\\
66	0.450892354260948\\
67	0.341601298346067\\
68	0.402101220385606\\
69	0.535899666427412\\
70	0.500292351444691\\
71	0.11130991247058\\
72	0.33797126780722\\
73	0.266042027761\\
74	0.453772473312595\\
75	0.315265519497982\\
76	0.0664533765119052\\
77	0.439780505268266\\
78	0.0912445196438349\\
79	0.138047523013254\\
80	0.291415697930149\\
81	0.152100645682413\\
82	0.203256320908266\\
83	0.223401154038658\\
84	0.0592868177624871\\
85	0.284406218384738\\
86	0.108826076280922\\
87	0.203368054884026\\
88	0.162811533331163\\
89	0.205339842738819\\
90	0.217146679095861\\
91	0.127711117158194\\
92	0.188616880823886\\
93	0.194290973988217\\
94	0.280702334793909\\
95	0.209514532675422\\
96	0.193473916785821\\
97	0.291554419936054\\
98	0.171912827529474\\
99	0.180459779138088\\
100	0.318739493675968\\
101	0.398703139959641\\
102	0.323188290723377\\
103	0.325320456440727\\
104	-0.00091818887207803\\
105	0.233315207242249\\
106	0.284865140598491\\
107	0.27440529907213\\
108	0.0991397943366665\\
109	0.15602470541477\\
110	0.253334717248088\\
111	0.0852614878821786\\
112	0.301336583033504\\
113	0.154614011473175\\
114	0.311064998254319\\
115	0.112089018661595\\
116	0.139321915280773\\
117	0.179933889357738\\
118	0.262349426034681\\
119	0.293060625319897\\
120	0.116884334158766\\
121	0.116181136466744\\
122	0.167734551947296\\
123	0.308227920264372\\
124	0.237265210757983\\
125	0.256604132389172\\
126	0.180037877296538\\
127	0.232202413379556\\
128	0.183201589253406\\
129	0.106547648899202\\
130	0.105046483429241\\
131	0.120939257901037\\
132	0.322220498678864\\
133	0.265203885854655\\
134	0.164748550729207\\
135	0.129343575805158\\
136	0.167127662526121\\
137	0.113708057432757\\
138	0.278475600551741\\
139	0.222309964064261\\
140	0.260274102865244\\
141	0.156198848615877\\
142	0.10816112086139\\
143	0.217704312843464\\
144	0.254454599293711\\
145	0.193981820748988\\
146	0.113682305327934\\
147	0.162646706255191\\
148	0.112161902892448\\
149	0.255496648459768\\
150	0.256116799899766\\
151	-0.0460722496585199\\
152	0.459679736144816\\
153	0.207389011750971\\
154	0.0590343649122772\\
155	0.271662389479131\\
156	0.189129631292154\\
157	0.110758706881691\\
158	0.392456629933447\\
159	0.303547945896831\\
160	0.24009414475569\\
161	0.251549095054938\\
162	0.197068162855751\\
163	0.144759067160211\\
164	0.122587145000078\\
165	0.105557013597745\\
166	0.500044066275686\\
167	0.0801768278836259\\
168	0.212238014976256\\
169	0.231437824023911\\
170	0.186113018718118\\
171	0.066590852023465\\
172	0.264249249586093\\
173	0.128818168869885\\
174	0.136928633257533\\
175	0.138323253603058\\
176	0.243815395181275\\
177	0.12571589852038\\
178	0.142408549287779\\
179	0.175069093258628\\
180	0.304949745976365\\
181	0.211449709725533\\
182	0.123026312578697\\
183	0.358392738783192\\
184	0.264683479484792\\
185	0.254724651954903\\
186	0.302951492642233\\
187	0.245464395768922\\
188	0.2229113329183\\
189	0.296411850700688\\
190	0.349401735868073\\
191	0.164826243797203\\
192	0.135606559707778\\
193	0.174499364352956\\
194	0.103898300875638\\
195	0.153645718696207\\
196	0.184919779975383\\
197	0.312815815002384\\
198	0.257011361557203\\
199	0.16599290711904\\
200	0.162844421211338\\
201	0.198670798121436\\
202	0.164096823162197\\
203	0.213928780875623\\
204	0.220455941675477\\
205	0.258098591121641\\
206	0.185514906774204\\
207	0.11007872898255\\
208	0.220027752152494\\
209	0.309911013321452\\
210	0.186712279668909\\
211	0.209909330706734\\
212	0.140409002633512\\
213	0.21218285773143\\
214	0.163341836080128\\
215	0.365074278535047\\
216	0.179371538662177\\
217	0.30634839897908\\
218	0.135259764457064\\
219	0.0801512735900101\\
220	0.0782481925200402\\
221	0.133845774965093\\
222	0.237810382581997\\
223	0.2844476036017\\
224	0.265764801532066\\
225	0.162367295591022\\
226	0.184405066416753\\
227	0.00870622503042605\\
228	0.236135869541002\\
229	0.174734406783864\\
230	0.119449335951609\\
231	0.0709973235572263\\
232	0.338610186189586\\
233	0.150710070162335\\
234	0.127081749206416\\
235	0.226708013939924\\
236	0.0730620679571756\\
237	0.22773623918712\\
238	0.0767991856584905\\
239	0.243246619476617\\
240	0.169123449736759\\
};
\addlegendentry{Dscr Stoch Ctrl w NMC};

\end{axis}
\end{tikzpicture}%
 
\end{subfigure}\\

\leavevmode\smash{\makebox[0pt]{\hspace{-7em}% HORIZONTAL POSITION           
  \rotatebox[origin=l]{90}{\hspace{20em}% VERTICAL POSITION
    Normalized PnL}%
}}\hspace{0pt plus 1filll}\null

Trading Day Number of 2013

\vspace{1cm}
\begin{subfigure}{\linewidth}
  %\centering
  \setlength\figureheight{\linewidth} 
  \setlength\figurewidth{\linewidth}
  \tikzsetnextfilename{strategylegend}
  \resizebox{\linewidth}{!}{\definecolor{mycolor1}{rgb}{0.25098,0.00000,0.38824}%
\definecolor{mycolor2}{rgb}{0.00000,0.46275,0.00000}%
\definecolor{mycolor3}{rgb}{0.00000,0.34902,0.34902}%
\definecolor{mycolor4}{rgb}{0.58039,0.26275,0.00000}%
\begin{tikzpicture}
    \begingroup
    % inits/clears the lists (which might be populated from previous
    % axes):
    \csname pgfplots@init@cleared@structures\endcsname
    \pgfplotsset{legend style={at={(0,1)},anchor=north west},legend columns=-1,legend style={draw=black,column sep=1ex},
            legend entries={Cts Stoch Ctrl,Dscr Stoch Ctrl,Cts Stoch Ctrl w nFPC,Dscr Stoch Ctrl w nFPC}}%
    
    \csname pgfplots@addlegendimage\endcsname{line width=2pt,mycolor1,sharp plot}
    \csname pgfplots@addlegendimage\endcsname{line width=2pt,mycolor2,sharp plot}
    \csname pgfplots@addlegendimage\endcsname{line width=2pt,mycolor3,sharp plot}
    \csname pgfplots@addlegendimage\endcsname{line width=2pt,mycolor4,sharp plot}

    % draws the legend:
    \csname pgfplots@createlegend\endcsname
    \endgroup
\end{tikzpicture}
}
\end{subfigure}%
  \caption{End of day strategy performances: in-sample backtesting using a one-week offset for calibration.}
  \label{fig:IS_week_comp}
\end{figure}

\begin{table}
\centering
\ra{1.2}
\begin{tabular}{@{} *{9}{r} @{}}
\toprule
Strategy & Return & Sharpe & \# MO & \# LO & Inv & \% Win & Max Loss & Max Win \\
\midrule
\multicolumn{9}{l}{\texttt{FARO}} \\ 
Naive & -1.072 & -0.444 & 435 & 0 & 1.74 & 0.08 & -34.276 & 2.984 \\ 
Naive+ & 0.045 & 0.046 & 0 & 213 & 2.26 & 0.74 & -8.764 & 5.363 \\ 
Naive++ & -0.003 & -0.027 & 0 & 6 & 0.25 & 0.50 & -1.138 & 0.444 \\ 
Cts & -0.060 & -0.565 & 53 & 149 & 0.08 & 0.20 & -0.955 & 0.077 \\ 
Dscr & -0.076 & -0.764 & 80 & 133 & -0.07 & 0.07 & -1.004 & 0.084 \\ 
Cts w nFPC & -0.065 & -0.590 & 56 & 149 & 0.09 & 0.17 & -1.022 & 0.072 \\ 
Dscr w nFPC & -0.076 & -0.778 & 78 & 133 & 0.07 & 0.07 & -0.904 & 0.053 \\[2ex]
\multicolumn{9}{l}{\texttt{NTAP}} \\ 
Naive & -0.303 & -0.316 & 854 & 0 & -10.96 & 0.20 & -9.463 & 4.349 \\ 
Naive+ & 0.290 & 0.122 & 0 & 3537 & -13.83 & 0.73 & -19.806 & 10.266 \\ 
Naive++ & -0.048 & -0.084 & 0 & 156 & -1.79 & 0.52 & -6.310 & 2.670 \\ 
Cts & -0.016 & -0.165 & 830 & 1405 & 0.37 & 0.52 & -0.612 & 0.158 \\ 
Dscr & 0.070 & 0.593 & 460 & 1388 & 5.06 & 0.79 & -0.355 & 0.858 \\ 
Cts w nFPC & -0.156 & -0.987 & 1506 & 1425 & 0.70 & 0.09 & -1.083 & 0.106 \\ 
Dscr w nFPC & 0.091 & 0.656 & 332 & 1401 & 3.16 & 0.85 & -0.416 & 1.048 \\[2ex]
\multicolumn{9}{l}{\texttt{ORCL}} \\ 
Naive & -0.112 & -0.248 & 492 & 0 & 3.66 & 0.28 & -3.197 & 2.452 \\ 
Naive+ & 0.066 & 0.022 & 0 & 4049 & -50.06 & 0.64 & -17.396 & 18.873 \\ 
Naive++ & 0.002 & 0.005 & 0 & 134 & 0.64 & 0.49 & -1.691 & 2.537 \\ 
Cts & 0.098 & 1.181 & 545 & 1318 & 1.86 & 0.90 & -0.203 & 0.310 \\ 
Dscr & 0.126 & 1.547 & 578 & 1310 & 4.01 & 0.97 & -0.097 & 0.505 \\ 
Cts w nFPC & -0.013 & -0.130 & 1069 & 1365 & 1.39 & 0.47 & -0.528 & 0.500 \\ 
Dscr w nFPC & 0.135 & 1.459 & 416 & 1338 & 3.16 & 0.96 & -0.192 & 0.516 \\[2ex]
\multicolumn{9}{l}{\texttt{INTC}} \\ 
Naive & -0.057 & -0.179 & 274 & 0 & -3.63 & 0.31 & -0.954 & 1.766 \\ 
Naive+ & 0.375 & 0.138 & 0 & 3925 & -25.43 & 0.65 & -11.060 & 11.465 \\ 
Naive++ & 0.013 & 0.055 & 0 & 77 & -0.47 & 0.53 & -1.815 & 1.126 \\ 
Cts & 0.202 & 1.995 & 423 & 1139 & 4.76 & 0.98 & -0.139 & 0.513 \\ 
Dscr & 0.226 & 2.494 & 501 & 1136 & 4.62 & 0.99 & -0.029 & 0.560 \\ 
Cts w nFPC & 0.107 & 1.111 & 681 & 1187 & 1.64 & 0.87 & -0.245 & 0.457 \\ 
Dscr w nFPC & 0.215 & 2.027 & 401 & 1156 & 3.78 & 0.99 & -0.118 & 0.647 \\ 
\bottomrule
\end{tabular}
\caption{Averaged strategy performance results: in-sample backtesting using a one-week offset for calibration.}
\label{tbl:IS_week}
\end{table}

\FloatBarrier
\subsection{Annual Calibration}
The second type of out-of-sample backtesting done was to calibrate using data amalgamated from the entire 2013 trading year. This was a very rich calibration source, as it effectively ensured that every possible state of the Markov chain would have had sufficient observations. Further, this caused us to fix the imbalance bins $\rho$ for the entire year, rather than having bins (and hence what it means to be `heavy buy imbalance' and `neutral imbalance') vary each day. Performance values are given in \autoref{tbl:IS_annual}, and \autoref{fig:IS_annual_comp} compares the day-over-day performance of the various strategies. 

Here we backtest only the more liquid of the stocks, \texttt{ORCL} and \texttt{INTC}. In comparing \autoref{tbl:IS_annual} with \autoref{tbl:IS_week}, we note some interesting observations. Again we see the most liquid stock, \texttt{INTC}, posting on average the better results using the strategies, suggesting that using a liquid stock is key.  (\texttt{INTC} started the year at \$21.38 and gained 21.42\% over the year, while \texttt{ORCL} started at \$34.69 and climbed 10.29\%. However, \texttt{NTAP} started at \$34.30 and gained 19.94\%, similar in performance to \texttt{INTC}, and yet performed substantially worse.) Whereas we have seen thus far that the nFPC strategies underperform the regular calibration, here the roles were reversed in terms of performance, number of market orders used, and average inventory held. Across the strategies we see stability in the number of limit orders used, which suggests that this isn't so much strategy dependent as it is externally dependent on outside agents submitting their market orders. In the case of \texttt{ORCL} we see that the Cts Stoch Ctrl strategy was particularly susceptible to the sharp downward spikes on days 55, 100, and 119, which corresponded to large sell-offs in the market. 


\begin{figure}
\centering
\begin{subfigure}{.45\linewidth}
  \centering
  \setlength\figureheight{\linewidth} 
  \setlength\figurewidth{\linewidth}
  \tikzsetnextfilename{IS_annual_ORCL}
  % This file was created by matlab2tikz.
%
%The latest updates can be retrieved from
%  http://www.mathworks.com/matlabcentral/fileexchange/22022-matlab2tikz-matlab2tikz
%where you can also make suggestions and rate matlab2tikz.
%
\definecolor{mycolor1}{rgb}{   0.20000  , 0.62745 ,  0.17255}% Dark Green
\definecolor{mycolor2}{rgb}{   0.12157  , 0.47059 ,  0.70588}% Dark Blue
\definecolor{mycolor3}{rgb}{   0.69804  , 0.87451 ,  0.54118}% Light Green
\definecolor{mycolor4}{rgb}{   0.65098  , 0.80784 ,  0.89020}% Light Blue
%
\begin{tikzpicture}[trim axis left, trim axis right]

\begin{axis}[%
width=\figurewidth,
height=\figureheight,
at={(0\figurewidth,0\figureheight)},
scale only axis,
every outer x axis line/.append style={black},
every x tick label/.append style={font=\color{black}},
xmin=1,
xmax=252,
%xlabel={Time (h)},
every outer y axis line/.append style={black},
every y tick label/.append style={font=\color{black}},
ymin=-0.8,
ymax=0.8,
%ylabel={Normalized PnL},
title={ORCL},
axis background/.style={fill=white},
axis x line*=bottom,
axis y line*=left,
yticklabel style={
        /pgf/number format/fixed,
        /pgf/number format/precision=3
},
scaled y ticks=false,
]
\addplot [color=mycolor1,solid,line width=1pt,forget plot]
  table[row sep=crcr]{%
1	-0.143152744598688\\
2	0.0409566773302763\\
3	0.0225720068215832\\
4	-0.0650511708916325\\
5	0.00606454290473\\
6	-0.0254824530777331\\
7	0.0389772133852268\\
8	-0.0114111094916207\\
9	-0.0270662195722461\\
10	-0.0855628988030846\\
11	0.0649951064606233\\
12	0.00133072102315445\\
13	0.393077107457149\\
14	-0.00370215310303141\\
15	0.0122839443285884\\
16	-0.0310875346980275\\
17	0.043483201502087\\
18	0.0424801826151211\\
19	-0.147306409434344\\
20	-0.10992458583361\\
21	-0.0962793977367626\\
22	-0.0576680286191835\\
23	-0.219546062859651\\
24	0.00362378008515\\
25	-0.108928643579458\\
26	-0.131042895444256\\
27	0.0114815785053352\\
28	-0.0822703709955286\\
29	0.0201343521373436\\
30	0.0382040436018392\\
31	-0.0500538335436623\\
32	-0.0176756122811914\\
33	0.0775656899759344\\
34	-0.2764653864822\\
35	-0.161615860562715\\
36	-0.0956478820635329\\
37	-0.270159027824385\\
38	-0.183830015523017\\
39	-0.179310083118617\\
40	-0.00940138580350346\\
41	-0.180985242258304\\
42	-0.19719625310921\\
43	-0.0824227714707286\\
44	-0.0531953383210056\\
45	-0.0925788166066226\\
46	-0.0588850425026199\\
47	-0.0347490286673981\\
48	-0.100533762597442\\
49	0.00205268421010893\\
50	-0.0486530554235266\\
51	0.0597717282980598\\
52	-0.0121448160451146\\
53	-0.181976377014897\\
54	-0.0399372084786851\\
55	-0.0877379612378367\\
56	-0.0701567205115872\\
57	-0.349815535018211\\
58	-0.0480335028794422\\
59	0.0769337253679048\\
60	0.00278944024552383\\
61	-0.0473646691752706\\
62	-0.0155302822075744\\
63	-0.127111759802825\\
64	0.00921385738268186\\
65	0.00240678774855579\\
66	-0.0311700232224119\\
67	-0.0925817813805951\\
68	-0.16980536098477\\
69	-0.05954691398974\\
70	-0.135488068826615\\
71	-0.278645562400599\\
72	-0.0781971703793798\\
73	-0.270850179290615\\
74	-0.15449502674324\\
75	-0.0645404258648556\\
76	-0.0734500683439551\\
77	-0.0970099619552773\\
78	0.0343089263116042\\
79	-0.0220254386537158\\
80	-0.0442356829584443\\
81	0.0968352774717633\\
82	0.122400452743693\\
83	-0.121869683409915\\
84	-0.121181915917986\\
85	-0.0884818093938658\\
86	-0.060038117851821\\
87	-0.1569169145631\\
88	0.0138989723853876\\
89	0.0269543638594793\\
90	-0.0320964376181165\\
91	-0.155649278913946\\
92	-0.205566908993275\\
93	-0.0497061943267014\\
94	-0.0737985376265631\\
95	-0.0617678464555871\\
96	-0.0760827785560391\\
97	-0.0427962379198825\\
98	-0.48245336165393\\
99	-0.270097408331084\\
100	-0.200745587840738\\
101	-0.240067478279848\\
102	-0.180853369016357\\
103	-0.0731967785924826\\
104	-0.233633926996703\\
105	-0.0756747359577371\\
106	-0.0659827454379421\\
107	-0.174511843997212\\
108	-0.197334044158415\\
109	-0.125061437084109\\
110	0.00163813062197134\\
111	-0.232989620678163\\
112	-0.138649697715001\\
113	-0.0701052742784381\\
114	-0.0852794488629931\\
115	-0.112522439392022\\
116	-0.0317477071254898\\
117	-0.154550200882653\\
118	-0.378287738056346\\
119	-0.704683591375321\\
120	-0.023404944316434\\
121	-0.0446406264371165\\
122	-0.0058047879683146\\
123	-0.154248404465989\\
124	-0.0506591020030976\\
125	-0.169844449650414\\
126	0.0673153639089035\\
127	-0.107346566699149\\
128	-0.0802971367218684\\
129	-0.0299497705794451\\
130	-0.0595761638187059\\
131	-0.0530883940998834\\
132	-0.162185071780507\\
133	-0.115011748174835\\
134	-0.0889645884489991\\
135	-0.00188943158720356\\
136	-0.0272731277886099\\
137	-0.107747834854862\\
138	-0.227810407875211\\
139	-0.0273238988812028\\
140	-0.123337225647539\\
141	-0.0732425625456187\\
142	-0.0242438481459873\\
143	-0.0709720888898088\\
144	0.00405181146134272\\
145	-0.0556437877204816\\
146	-0.215098235804878\\
147	-0.0596423649910278\\
148	-0.0198103625232419\\
149	-0.0317729478597815\\
150	-0.0721309593712943\\
151	-0.0991151788172514\\
152	-0.0884387712389177\\
153	-0.0968651256745322\\
154	-0.107491391129115\\
155	-0.186157852539664\\
156	-0.0845809615461386\\
157	-0.233604749845871\\
158	-0.158251744371543\\
159	-0.209757519603007\\
160	-0.119598643023831\\
161	-0.2716873816551\\
162	-0.088638283444982\\
163	-0.057641899827942\\
164	-0.119637287885132\\
165	-0.270060307437976\\
166	-0.1316211013809\\
167	-0.0882053915257603\\
168	-0.212273720091381\\
169	-0.163287930562959\\
170	-0.0814100188432132\\
171	-0.103277467734153\\
172	-0.141165514663665\\
173	-0.0701834757049924\\
174	-0.0885747285781201\\
175	-0.172232724418938\\
176	-0.059724561539514\\
177	-0.125900417482458\\
178	-0.138295864979637\\
179	-0.118017132530749\\
180	-0.0817441610397176\\
181	-0.366232220109311\\
182	-0.0772019082926824\\
183	-0.191672660254052\\
184	-0.0861969923477891\\
185	-0.123284607585518\\
186	-0.137455256807513\\
187	-0.158738857772212\\
188	-0.116281588057889\\
189	-0.126799652024237\\
190	-0.178604090811128\\
191	-0.264911600566118\\
192	-0.146225201168938\\
193	-0.123713463299111\\
194	-0.164904878102206\\
195	-0.108532603092511\\
196	-0.0636764057699469\\
197	-0.0536678189814755\\
198	-0.183158397264326\\
199	-0.123283516416816\\
200	-0.139879283060345\\
201	-0.10990845652968\\
202	-0.046241660977155\\
203	-0.0706757150549093\\
204	-0.00112441208515265\\
205	-0.0241871460250395\\
206	0.00485155862698148\\
207	-0.0204329767171499\\
208	-0.0151218651688404\\
209	-0.00723833840901775\\
210	-0.139161210405707\\
211	-0.161273996041664\\
212	0.00132204213310409\\
213	-0.00745434975615357\\
214	-0.0287389723815516\\
215	-0.0526478859955804\\
216	-0.154981025371928\\
217	-0.0716134550215987\\
218	-0.0917724279642875\\
219	-0.0665064951265261\\
220	-0.0178675594637534\\
221	-0.16746116139023\\
222	0.033571211903165\\
223	-0.0575307379321925\\
224	-0.100541134147284\\
225	-0.131513682341395\\
226	-0.0459363995993681\\
227	-0.000283787111564833\\
228	-0.00027757569332935\\
229	0.0362952069415284\\
230	-0.0177461784420121\\
231	-0.083375347496053\\
232	-0.0810836709781799\\
233	-0.0488485458900137\\
234	-0.0742196521054875\\
235	-0.0448109158400766\\
236	-0.0393384448078818\\
237	-0.018157458888292\\
238	-0.0517682293819131\\
239	-0.108671614294697\\
240	-0.188868189752239\\
241	-0.0266460894078831\\
242	-0.0729181094505936\\
243	0.00641299882468128\\
244	-0.0181581957487398\\
245	-0.426311894516413\\
246	0.014583934132769\\
247	-0.0224994188023452\\
248	-0.0464222913087153\\
249	-0.0694982949806928\\
250	-0.106972192995484\\
251	-0.053550611555725\\
252	-0.034315649862337\\
};
\addplot [color=mycolor2,solid,line width=1pt,forget plot]
  table[row sep=crcr]{%
1	0.30391980123581\\
2	0.279989607342037\\
3	0.104493659719645\\
4	0.0711411679674797\\
5	0.0974758416324635\\
6	0.135020030966423\\
7	0.21937719933309\\
8	0.134574186167451\\
9	0.186300080256134\\
10	0.165612719953199\\
11	0.13467833008811\\
12	0.167423759863081\\
13	0.0273710924965377\\
14	0.161202230918598\\
15	0.202008352694582\\
16	0.224383770282168\\
17	0.0778875140564405\\
18	0.231587921719369\\
19	-0.0217627123551034\\
20	0.095600070443442\\
21	0.166199510372964\\
22	-0.0202893831428722\\
23	0.221450352248142\\
24	0.126287642618672\\
25	0.190188837247395\\
26	0.291873965036333\\
27	0.124267978835074\\
28	0.100806563660975\\
29	0.0847839138166134\\
30	0.0184649111040544\\
31	0.100482289360945\\
32	0.16322775334867\\
33	0.251461720958779\\
34	-0.0327838632566798\\
35	0.167909224439716\\
36	0.0272785283611255\\
37	0.000414903580518349\\
38	0.164491640329409\\
39	0.0458338234426272\\
40	0.284592653235583\\
41	-0.00469052712511058\\
42	-0.0530235264860993\\
43	0.132213458598435\\
44	0.204612238819433\\
45	0.144748710447075\\
46	0.179901756658058\\
47	0.127701704248544\\
48	0.172020381164583\\
49	0.192321130551971\\
50	0.188266078681895\\
51	0.140330754542413\\
52	0.18785893331547\\
53	0.223954011310401\\
54	0.239709368828676\\
55	0.486706883215756\\
56	0.1082978357118\\
57	0.0998090973931565\\
58	0.172609838678833\\
59	0.298166624198851\\
60	0.239835531571686\\
61	0.189575299654216\\
62	0.197292977773833\\
63	0.202448284904525\\
64	0.254431687248642\\
65	0.316089304850331\\
66	0.0540213799023794\\
67	0.11644299390196\\
68	0.34064603090327\\
69	0.182537492085717\\
70	0.124607912690121\\
71	0.189841042030954\\
72	0.258984142277515\\
73	0.0487250763856916\\
74	0.189573268955994\\
75	0.356903731255983\\
76	0.161101512843304\\
77	0.173778379565565\\
78	0.208139671810986\\
79	0.308844330850879\\
80	0.141538332254659\\
81	0.200884667976744\\
82	0.430060445304225\\
83	0.174214190395833\\
84	0.212347003142854\\
85	0.0867898893046956\\
86	0.162241064088688\\
87	0.0693328406846551\\
88	0.183453531641591\\
89	0.195073659996617\\
90	0.123014106886756\\
91	0.0380106331322261\\
92	0.0656316116706567\\
93	0.0860833603847741\\
94	0.129749892656321\\
95	0.14792478529521\\
96	0.210088176815601\\
97	0.156258497308003\\
98	0.304256198738895\\
99	0.252164816960141\\
100	0.0470737533076383\\
101	0.130023133307429\\
102	0.129347946294781\\
103	0.0565983208316409\\
104	0.0306932829263233\\
105	0.117249818901254\\
106	0.196575424815801\\
107	0.176874117858134\\
108	0.365597873585027\\
109	0.0325231578156801\\
110	0.182150007632377\\
111	0.133743185556336\\
112	0.203613666874533\\
113	-0.0375880811795595\\
114	0.22260110900325\\
115	0.193040861680235\\
116	0.116315638975275\\
117	0.156173099150138\\
118	0.0525915009837263\\
119	0.164384087683327\\
120	0.30997565831735\\
121	0.122539354378317\\
122	0.147175061761096\\
123	0.282246833667623\\
124	0.298841277387712\\
125	-0.0884394250286415\\
126	0.213293322054991\\
127	0.141715715659852\\
128	0.243149666667379\\
129	0.123140309991338\\
130	0.262474005936294\\
131	0.174208796072005\\
132	0.188610750698997\\
133	0.0738162531852026\\
134	0.190907578740153\\
135	0.128948008096204\\
136	0.142425552244105\\
137	0.158521782566722\\
138	0.122888370221202\\
139	0.0795776118784478\\
140	0.131723683502111\\
141	0.047383308862499\\
142	0.170470022526545\\
143	0.0375614020833121\\
144	0.0617909046525308\\
145	0.107317561146912\\
146	0.158026016810783\\
147	0.0222572004844339\\
148	0.0666107895413924\\
149	0.0769429369736428\\
150	0.177912628218722\\
151	0.098541499388799\\
152	0.0974846296761735\\
153	0.00673842209381607\\
154	-0.0129916213644158\\
155	0.0326269794361613\\
156	0.0768455126175127\\
157	0.175945939976982\\
158	0.085317143846952\\
159	0.169323414048746\\
160	0.048264874903101\\
161	0.16615597552273\\
162	0.104752538805686\\
163	0.0232288833476614\\
164	0.162816446149189\\
165	0.0803752218020178\\
166	0.0888071110841557\\
167	0.111380501676856\\
168	0.116422322098607\\
169	0.0990401583308419\\
170	-0.010285782641404\\
171	0.11072438210708\\
172	0.103206811668615\\
173	-0.0192713433878968\\
174	0.127101802996817\\
175	0.0363792966517226\\
176	0.209016980770479\\
177	0.187960713636485\\
178	0.0151744102249499\\
179	0.167414848029729\\
180	0.231697582998873\\
181	0.14935345067665\\
182	0.114407949157651\\
183	0.115477621828263\\
184	0.116492715019623\\
185	0.0656513081444051\\
186	0.0697816799157596\\
187	0.110921044416324\\
188	0.177450322752631\\
189	0.159118350529589\\
190	0.0639034839441202\\
191	0.199424647280944\\
192	0.153925170278027\\
193	0.146756515987654\\
194	0.123793168501931\\
195	0.160023645195434\\
196	0.274514140833035\\
197	0.167228216438165\\
198	0.174231906711933\\
199	0.0851806523486165\\
200	0.219216833428879\\
201	0.148161637411238\\
202	0.129580344988186\\
203	0.185103320574473\\
204	0.165675996125955\\
205	0.0936761677088966\\
206	0.183987332029324\\
207	0.142261652992772\\
208	0.0877171160285701\\
209	0.180112708799717\\
210	0.0609501957840159\\
211	0.148070777786537\\
212	0.055851394988721\\
213	0.0709838521940563\\
214	0.041187241862375\\
215	0.246162596118826\\
216	0.0424474302863783\\
217	-0.0170189896255588\\
218	0.0329206564737356\\
219	0.00463526319300386\\
220	0.0538715832589857\\
221	0.178639197376422\\
222	0.043028709851543\\
223	0.0843213186930804\\
224	0.117533888698174\\
225	0.0511575883166564\\
226	0.0239408676355892\\
227	0.0873734921264623\\
228	0.114638751291501\\
229	0.202883531990841\\
230	0.112748172133995\\
231	0.0194997633866025\\
232	0.113754809343357\\
233	0.124258726004152\\
234	0.101914799678255\\
235	0.104712832530443\\
236	0.188882488383911\\
237	0.098442717105064\\
238	0.254361223677934\\
239	0.168421614203036\\
240	0.187890168720843\\
241	0.0302354516987999\\
242	0.131689780483186\\
243	0.0998327464269098\\
244	0.269633425535159\\
245	0.438931459095862\\
246	0.275483120449328\\
247	0.207058701736482\\
248	0.0874100256086577\\
249	0.133859657932863\\
250	0.0816213207678088\\
251	0.0416854316167692\\
252	0.107435330244426\\
};
\addplot [color=mycolor3,solid,line width=1pt,forget plot]
  table[row sep=crcr]{%
1	0.232985180645052\\
2	0.278661369202366\\
3	0.177784445622526\\
4	0.0974783024399574\\
5	0.0327501169678138\\
6	0.115835166200431\\
7	0.228182960974266\\
8	0.133677083274036\\
9	0.117138736406273\\
10	0.0407993933299398\\
11	0.224901868201894\\
12	0.127772225688058\\
13	0.484476142631701\\
14	0.158329884387747\\
15	0.222959432027228\\
16	0.168627969078757\\
17	0.223764755537495\\
18	0.191147521777371\\
19	0.188269006298159\\
20	0.0633774412278127\\
21	0.100655573303472\\
22	0.0552923224973704\\
23	0.00296293629926227\\
24	0.146565657641003\\
25	0.0539323835085516\\
26	0.088952312466428\\
27	0.103644176185121\\
28	0.0459792583841881\\
29	0.167203997305459\\
30	0.0275495644912131\\
31	0.0385576079915185\\
32	0.140369493831539\\
33	0.0908384340048176\\
34	-0.0948591341170648\\
35	0.0864392597988126\\
36	-0.0354566217863406\\
37	-0.010317418128354\\
38	0.100727021651652\\
39	0.0185516200215174\\
40	0.0999079395420016\\
41	0.127182620672739\\
42	0.198457523865318\\
43	0.161341126958339\\
44	0.162018082003606\\
45	0.0896753081223347\\
46	0.0706439778174085\\
47	0.0862713664879718\\
48	0.0150859046368243\\
49	0.0657548690991912\\
50	0.120804460730971\\
51	0.116836047940295\\
52	0.173762024633149\\
53	0.0408085503252444\\
54	0.196231599131226\\
55	0.303153416169213\\
56	0.213132313445921\\
57	0.00328884448705292\\
58	0.0816711766253685\\
59	0.326281946274648\\
60	0.100711841711582\\
61	0.117978675757647\\
62	0.165274080760713\\
63	0.0512366903200091\\
64	0.183001481869082\\
65	0.22740384885254\\
66	0.18120229904944\\
67	0.160902639904801\\
68	0.244873158782679\\
69	0.198132558155759\\
70	0.111828439479246\\
71	-0.0292877791293691\\
72	0.175789026392167\\
73	0.0777978826200172\\
74	0.318573788431889\\
75	0.225325607071812\\
76	0.176900250172823\\
77	0.0943795291426934\\
78	0.267498491033146\\
79	0.162023593040026\\
80	0.21221640032508\\
81	0.192799867501877\\
82	0.332764168099998\\
83	0.192159036173535\\
84	0.0485391971680672\\
85	0.147201294910995\\
86	0.134205673668886\\
87	0.0413551405682738\\
88	0.163274976214799\\
89	0.181687987700759\\
90	0.0922938076229452\\
91	0.0115536515993344\\
92	-0.140280495851299\\
93	0.119840477478217\\
94	-0.00590111106983454\\
95	0.0956018488099142\\
96	0.0819497429315121\\
97	0.123005840782121\\
98	-0.0913045949948849\\
99	0.124815016842876\\
100	0.0548683207746685\\
101	0.0947713389582852\\
102	0.120853429885179\\
103	0.0977217262898269\\
104	-0.0362412098370985\\
105	0.326377020260979\\
106	0.191980074306645\\
107	0.130274208205258\\
108	0.107792579916992\\
109	-0.0195827679726336\\
110	0.239851413357581\\
111	0.0911675169453773\\
112	0.127287854404605\\
113	0.176936165888826\\
114	0.146426576735801\\
115	0.17153665839019\\
116	0.104597987250093\\
117	0.159312657578851\\
118	0.0485939492601045\\
119	-0.119306879716023\\
120	0.32113588646771\\
121	0.164477978725228\\
122	0.111832535583704\\
123	-0.0289101023348679\\
124	0.342743014178556\\
125	-0.137819552344678\\
126	0.179713536266101\\
127	0.102257290230638\\
128	0.0794973462296643\\
129	0.169663688572771\\
130	0.118037309848645\\
131	0.172209657125192\\
132	0.112773605073544\\
133	0.113802253707354\\
134	0.161743871723443\\
135	0.027181216214204\\
136	0.0992267616906097\\
137	0.0451809622425888\\
138	0.0307367073514741\\
139	0.00834368907549202\\
140	0.0438643539213626\\
141	0.0622992966938269\\
142	0.142075461545313\\
143	0.124319704085495\\
144	0.155064036520721\\
145	0.112355960956459\\
146	0.129968237608048\\
147	0.124265765213282\\
148	0.089205432552081\\
149	0.110339676984122\\
150	0.0865363232719383\\
151	0.056711945542368\\
152	0.0604724903236844\\
153	0.0132143447916864\\
154	0.146084319432537\\
155	0.0143784265959894\\
156	0.128361495663678\\
157	0.0269224735038067\\
158	0.124822343461558\\
159	0.101653009775558\\
160	0.0513006823572526\\
161	0.0968342157936758\\
162	0.0790767557675106\\
163	0.193633892620425\\
164	0.0870793980974061\\
165	0.0521911180987885\\
166	0.0428379242719558\\
167	0.10077308471489\\
168	0.038952356186051\\
169	0.113450984150026\\
170	0.0845369523509786\\
171	0.0936409666937151\\
172	0.109651245944451\\
173	0.179757522467018\\
174	0.0354537216892541\\
175	0.0151809728304655\\
176	0.0945005562075589\\
177	0.0149260439944415\\
178	0.00281069854134182\\
179	0.0335558826282202\\
180	0.155767841494501\\
181	0.0543554310967031\\
182	0.138057298858646\\
183	-0.0211471483376159\\
184	0.0267091244602764\\
185	0.0651297930594392\\
186	0.000786096804094837\\
187	0.10292943681807\\
188	0.172169738924283\\
189	0.0647197961089958\\
190	0.184515558782605\\
191	0.0511900478827466\\
192	0.14964999585032\\
193	0.127198865510193\\
194	0.260012355734284\\
195	0.227666905713059\\
196	0.248227046444054\\
197	0.12498511758634\\
198	0.159341835522063\\
199	0.148673188238109\\
200	0.167152469016004\\
201	0.14479397098422\\
202	0.174397908105001\\
203	0.0673452735247626\\
204	0.131226284747402\\
205	0.108445392792424\\
206	0.165797086956511\\
207	0.169913012420754\\
208	0.0636082840378056\\
209	0.0620139396004335\\
210	0.0562393195838818\\
211	0.0688769521494154\\
212	0.117228698734344\\
213	0.0880519489204243\\
214	0.0931417105478116\\
215	0.191311934292677\\
216	0.133069703542529\\
217	0.0847130762616071\\
218	0.0803582190909989\\
219	0.138757357023911\\
220	0.123093160196518\\
221	0.0833453108336262\\
222	0.189295334350849\\
223	0.079163850824988\\
224	0.0203166165120443\\
225	0.0989683606295565\\
226	0.00651819309494288\\
227	0.0717527520079922\\
228	0.0825436823681192\\
229	0.0870800327581871\\
230	0.125969715648678\\
231	0.0520229824858752\\
232	-0.0228189242434396\\
233	0.161682078513971\\
234	0.128832122256469\\
235	0.152389538249389\\
236	0.188106371248526\\
237	0.0909801730509801\\
238	0.00679257366476614\\
239	0.0135800578969971\\
240	0.157213526162243\\
241	0.0288904217185857\\
242	0.104194652248243\\
243	0.104010133085349\\
244	0.255648086856961\\
245	0.364429955687002\\
246	0.0673541865664417\\
247	0.176558312207671\\
248	0.0671039842472772\\
249	0.0445631345055255\\
250	0.108440838773946\\
251	0.0534500570486243\\
252	0.0917498830878987\\
};
\addplot [color=mycolor4,solid,line width=1pt,forget plot]
  table[row sep=crcr]{%
1	0.27639918926549\\
2	0.21155945953622\\
3	0.207492605393876\\
4	0.160139160324937\\
5	0.165479784839595\\
6	0.125073629404836\\
7	0.197829837790079\\
8	0.118702865642066\\
9	0.165089329716041\\
10	0.161538787011453\\
11	0.19580344812474\\
12	0.127316910070779\\
13	0.474496491286064\\
14	0.180050973300325\\
15	0.0993212969582757\\
16	0.195156699322351\\
17	0.198532647257166\\
18	0.188298656265802\\
19	0.211657684387069\\
20	0.0626875689116539\\
21	0.163931551455658\\
22	0.131771254868048\\
23	-0.00315271753168663\\
24	0.0787402723718865\\
25	0.0453231588317941\\
26	0.0894584600383547\\
27	0.166341554645226\\
28	0.0844377317306865\\
29	0.159084791145838\\
30	0.0526663675139228\\
31	0.0623373498696312\\
32	0.11945868160562\\
33	0.212381459920464\\
34	0.0209163567438843\\
35	0.0149111992343755\\
36	0.0877389429888953\\
37	0.0299015923204245\\
38	0.0676225102386918\\
39	0.119742769412006\\
40	0.0816240977702925\\
41	0.194699654303858\\
42	0.21275686985415\\
43	0.0915241793980571\\
44	0.164354192431146\\
45	0.128275075633929\\
46	0.0566021414925927\\
47	0.106947366994773\\
48	0.00885450759402141\\
49	0.0622596231879802\\
50	0.167817160381626\\
51	0.116056004094141\\
52	0.156960052608011\\
53	0.0487993586743903\\
54	0.166073155840938\\
55	0.40888991168613\\
56	0.157105693678344\\
57	0.0740134910320949\\
58	0.118001485633177\\
59	0.236909011594077\\
60	0.201353462292212\\
61	0.13557854172803\\
62	0.157627403304817\\
63	0.105009925749533\\
64	0.144682515232769\\
65	0.195943856230458\\
66	0.101357041301963\\
67	0.227810409516371\\
68	0.227654535421307\\
69	0.142485829739997\\
70	0.0844971435701694\\
71	0.0406893614329708\\
72	0.132250488974365\\
73	-0.028164811726563\\
74	0.0702665979208937\\
75	0.226417512723961\\
76	0.151366309590113\\
77	0.15186766327251\\
78	0.160882965886393\\
79	0.119989986242234\\
80	0.186278816347836\\
81	0.167459124509021\\
82	0.298649172065754\\
83	0.169794370348843\\
84	0.172102501150202\\
85	0.10402313182636\\
86	0.128497938498235\\
87	0.123765813880071\\
88	0.145672467618361\\
89	0.197300881455733\\
90	0.11954366976266\\
91	0.0378913409256639\\
92	0.0912287534477121\\
93	0.139569403217225\\
94	0.144401548157279\\
95	0.0969862124168518\\
96	0.0963345993519173\\
97	0.119569214466929\\
98	0.0120137533667836\\
99	0.162922884216401\\
100	0.0967948968883721\\
101	0.0519572488941979\\
102	0.0708806015535982\\
103	0.0569403659887177\\
104	0.0371553896048364\\
105	0.257619063542184\\
106	0.174366979201043\\
107	0.0998755760222771\\
108	0.0769646924915516\\
109	0.163258335810131\\
110	0.189208772934403\\
111	0.0870240481764104\\
112	0.0961675141419996\\
113	0.159549733883943\\
114	0.107770071084745\\
115	0.163726119614005\\
116	0.0813024799791784\\
117	0.103036055979575\\
118	0.0051992963078465\\
119	0.155133859446972\\
120	0.26573111681967\\
121	0.160423562044355\\
122	0.143375609446446\\
123	0.268451511416498\\
124	0.293980385583363\\
125	0.0408807978982486\\
126	0.204596687406779\\
127	0.187397723747391\\
128	0.161320128691823\\
129	0.132550163832802\\
130	0.222926595135384\\
131	0.108433128023631\\
132	0.197314897482464\\
133	0.142841163453813\\
134	0.155059050338522\\
135	0.120882177243395\\
136	0.118120095764545\\
137	0.103900837186136\\
138	0.0585840380984916\\
139	0.0673575673034174\\
140	0.0806967742107236\\
141	0.0761542780972822\\
142	0.111974229824667\\
143	0.109195985086925\\
144	0.131682835045461\\
145	0.0960223016899116\\
146	0.123142595616906\\
147	0.0916802495254794\\
148	0.0493302201987418\\
149	0.0809375684407428\\
150	0.148103387204611\\
151	0.0761853140612121\\
152	0.0702567139928381\\
153	0.053148628011654\\
154	0.101598604579821\\
155	0.0457749226417626\\
156	0.101303352539949\\
157	0.0355607019801478\\
158	0.0485429137702689\\
159	0.0118533548508159\\
160	0.0641164528980895\\
161	0.0752134823370411\\
162	0.0685664083529915\\
163	0.0436990463152773\\
164	0.117130029217376\\
165	0.0698969146054663\\
166	0.0315670037669296\\
167	0.0640988053172732\\
168	0.0746331568685694\\
169	0.0828505034364849\\
170	0.113497221110076\\
171	0.0644964108987353\\
172	0.0849738885158042\\
173	0.181369947577198\\
174	0.0840224549847493\\
175	0.0971815598235255\\
176	0.0966495775071051\\
177	0.0543883939075832\\
178	0.0773925270985083\\
179	0.129462014301531\\
180	0.16224732391565\\
181	0.215811512520427\\
182	0.120947380754986\\
183	0.0773099357478404\\
184	0.0438866453904282\\
185	0.073895078461553\\
186	0.0572866402734621\\
187	0.0676904147510946\\
188	0.0711673362608009\\
189	0.112812436941538\\
190	0.107596090182604\\
191	0.0425067462018149\\
192	0.118865299757302\\
193	0.100907240011908\\
194	0.10016762524532\\
195	0.137821720080934\\
196	0.218515169444105\\
197	0.129052416600099\\
198	0.135845709334879\\
199	0.066528557717827\\
200	0.121554717409691\\
201	0.0961636298085788\\
202	0.117253122533433\\
203	0.0497705231919543\\
204	0.0951939424467456\\
205	0.0879054494519945\\
206	0.154745078511307\\
207	0.110314152312057\\
208	0.0836100748894903\\
209	0.146884965031663\\
210	0.0565817460467028\\
211	0.102581251934584\\
212	0.0713936205600112\\
213	0.0728126816662681\\
214	0.0411561861587021\\
215	0.189970984626892\\
216	0.0489161925568649\\
217	0.0714902653976091\\
218	0.0686970681125924\\
219	0.127734251014221\\
220	0.0899410427676556\\
221	0.0902039767100507\\
222	0.168210094571749\\
223	0.0609878756769018\\
224	0.0699860116246424\\
225	0.0266330888162964\\
226	0.0695080629070155\\
227	0.0597590110332024\\
228	0.0980603191800073\\
229	0.129449383905655\\
230	0.0870174619857003\\
231	0.0566516008915038\\
232	0.0312845703438675\\
233	0.11888581710355\\
234	0.11658074205095\\
235	0.0653655101536513\\
236	0.140861632630684\\
237	0.0903054193630153\\
238	0.0523090028468483\\
239	-0.00650629789209997\\
240	0.108269648266976\\
241	0.0567961650517652\\
242	0.0966328430306656\\
243	0.068512438300021\\
244	0.236861122884258\\
245	0.374265369788632\\
246	0.11198907773798\\
247	0.148390098963562\\
248	0.0446932786848131\\
249	0.0960325619798381\\
250	0.127640008925875\\
251	0.0290784157024081\\
252	0.0804453804514989\\
};
\end{axis}
\end{tikzpicture}%

\end{subfigure}%
\hfill%
\begin{subfigure}{.45\linewidth}
  \centering
  \setlength\figureheight{\linewidth} 
  \setlength\figurewidth{\linewidth}
  \tikzsetnextfilename{IS_annual_INTC}
  % This file was created by matlab2tikz.
%
%The latest updates can be retrieved from
%  http://www.mathworks.com/matlabcentral/fileexchange/22022-matlab2tikz-matlab2tikz
%where you can also make suggestions and rate matlab2tikz.
%
%
\begin{tikzpicture}[trim axis left, trim axis right]

\begin{axis}[%
width=\figurewidth,
height=\figureheight,
at={(0\figurewidth,0\figureheight)},
scale only axis,
every outer x axis line/.append style={black},
every x tick label/.append style={font=\color{black}},
xmin=1,
xmax=252,
%xlabel={Time (h)},
every outer y axis line/.append style={black},
every y tick label/.append style={font=\color{black}},
ymin=-0.8,
ymax=0.8,
%ylabel={Normalized PnL},
title={INTC},
axis background/.style={fill=white},
axis x line*=bottom,
axis y line*=left,
yticklabel style={
        /pgf/number format/fixed,
        /pgf/number format/precision=3
},
scaled y ticks=false,
]
\addplot [color=cts_plot_color,solid,line width=1pt,forget plot]
  table[row sep=crcr]{%
1	0.108261969453122\\
2	0.128092766008791\\
3	0.138042602261726\\
4	0.0689827355724957\\
5	-0.0508908268687132\\
6	0.11538950403179\\
7	0.14626078138833\\
8	0.131540265756651\\
9	0.0320565799796215\\
10	0.150855952311756\\
11	0.138601678341993\\
12	-0.198193726608908\\
13	0.176304349601666\\
14	0.106431495798279\\
15	0.124571842575772\\
16	0.169655559701445\\
17	0.0755018234871014\\
18	0.129893264309841\\
19	0.116225134188291\\
20	0.0360262567184804\\
21	0.0247422126180168\\
22	0.211414718911429\\
23	0.138335248304581\\
24	0.116255681041543\\
25	0.120163088231376\\
26	0.174552526651406\\
27	0.113898655980305\\
28	0.0625369538875912\\
29	0.0828337756485767\\
30	0.078848504424959\\
31	0.0778406626193225\\
32	0.0613091077904826\\
33	0.0981156080686699\\
34	0.0437305896462215\\
35	-0.107978727945947\\
36	0.0422570380736357\\
37	0.0518807637461655\\
38	0.138515799274456\\
39	-0.0082199773927469\\
40	0.103146673228178\\
41	0.124075226268223\\
42	0.120275057214139\\
43	0.175268195805374\\
44	0.115844802852367\\
45	0.0387918535917847\\
46	0.0474052750595506\\
47	0.0829883775743784\\
48	0.00618608785344248\\
49	0.0913658967434522\\
50	0.134210002920354\\
51	0.0505766095096737\\
52	0.00543701888886731\\
53	0.0807066908067748\\
54	0.156978531580506\\
55	0.0979806277522267\\
56	0.0474474090495496\\
57	0.0374303132845947\\
58	0.0554990886649717\\
59	0.171774404775077\\
60	0.0512813079475249\\
61	0.0503689941991482\\
62	0.119893637511322\\
63	0.00707292088765692\\
64	0.115872241811019\\
65	0.17328633932529\\
66	0.135219371640266\\
67	0.141174535281971\\
68	0.2010688432868\\
69	0.227756892412636\\
70	0.0561618573866091\\
71	-0.0394852814489606\\
72	0.211159472217932\\
73	0.00122970503154915\\
74	0.149456372593392\\
75	0.208616975679732\\
76	0.0960263888191837\\
77	0.0131105074853763\\
78	0.17404792333262\\
79	-0.158162085792078\\
80	0.0374647796877087\\
81	-0.00403344021345173\\
82	0.148567930878915\\
83	-0.0514163669677407\\
84	0.102201960858149\\
85	0.10829618996525\\
86	0.00591430843927303\\
87	-0.0150519084078956\\
88	0.143416662067332\\
89	0.0230792443964536\\
90	0.0596959518151509\\
91	0.0190487985205076\\
92	-0.0514062438074665\\
93	0.0455170523808247\\
94	0.0754234224032671\\
95	0.119293461388266\\
96	0.145551946718334\\
97	0.168239562035543\\
98	-0.0200800110482956\\
99	0.0647420743323313\\
100	0.0265974945583752\\
101	0.0503832233438308\\
102	0.109468013496597\\
103	0.0365425315970992\\
104	0.0382582389951579\\
105	0.172335727977483\\
106	-0.0965722398996357\\
107	0.0268956385755057\\
108	0.0597835470846484\\
109	0.0613502089919859\\
110	0.0272245350806087\\
111	0.0134598791399228\\
112	0.0220557718365536\\
113	0.144781742916602\\
114	0.164265216477742\\
115	0.0929810910255108\\
116	-0.0171582414088864\\
117	0.00184779714519647\\
118	-0.0861258770145403\\
119	0.0992043294393057\\
120	-0.0568703746190836\\
121	0.162277518790955\\
122	0.067038771555409\\
123	-0.00698610227058504\\
124	0.22732290240923\\
125	0.0798645984348988\\
126	0.100308116841178\\
127	0.0869853473120074\\
128	0.153042019602568\\
129	-0.282473459336083\\
130	0.00167938058314608\\
131	-0.0668692982940619\\
132	0.340573246594684\\
133	0.101383537283402\\
134	0.153128721439757\\
135	-0.0225515513667713\\
136	0.0850346045170208\\
137	0.257659489851652\\
138	0.134822504798924\\
139	0.0161452839639641\\
140	-0.15772702724084\\
141	0.00416626455367591\\
142	0.156677954482728\\
143	0.195150362533003\\
144	0.00823489172142039\\
145	0.0665959869364099\\
146	0.0500482970538794\\
147	0.0392330862105301\\
148	0.216426980671116\\
149	0.160708215279891\\
150	0.175253288271324\\
151	0.171441017487954\\
152	0.0149874172867495\\
153	0.0625221919521225\\
154	0.0825572865332587\\
155	0.0547155864908492\\
156	0.058490385203267\\
157	-0.0693070489521058\\
158	0.063095140026463\\
159	-0.191704169063219\\
160	0.108825520224136\\
161	-0.156982715492995\\
162	-0.28203914522936\\
163	0.0743978044883306\\
164	-0.121542879949486\\
165	0.109117879553276\\
166	0.00988417645492044\\
167	-0.0186351783008256\\
168	0.103413433901317\\
169	0.0844681789726084\\
170	0.0241571341510587\\
171	0.0964149750578668\\
172	0.0697235101198282\\
173	0.151217538217463\\
174	0.0152478812507011\\
175	0.0469930948941769\\
176	0.0973377614788725\\
177	0.321871466660433\\
178	-0.0621356190449453\\
179	0.0547220381542117\\
180	0.0458075373278752\\
181	0.115797502019694\\
182	0.0458726525250665\\
183	0.0630251303152078\\
184	0.0289874319378473\\
185	0.0566523090266537\\
186	0.0464686055954117\\
187	0.0169041473736411\\
188	-0.0508784022615498\\
189	0.0101772592828562\\
190	0.0366323527817701\\
191	0.112460797610547\\
192	0.0335893508647951\\
193	0.0736036364243488\\
194	0.0178994386564783\\
195	-0.0307879324856051\\
196	0.140180273122916\\
197	0.0897795818278904\\
198	0.0254436555103005\\
199	0.132509305805736\\
200	0.20357032745515\\
201	0.0845321588225085\\
202	0.0958021331742496\\
203	0.00833383662002339\\
204	0.129727041431246\\
205	0.0267717333567808\\
206	0.143753897530091\\
207	0.0540151526406507\\
208	0.122186064640078\\
209	0.0761791761842384\\
210	0.0121409206149818\\
211	0.0826400930819239\\
212	0.00312684435651294\\
213	0.1407315318068\\
214	0.0472542886965665\\
215	0.0860742202844107\\
216	0.0223579836328475\\
217	0.154575549988418\\
218	-0.00386652868985407\\
219	-0.0101279877341988\\
220	0.00698156876633294\\
221	0.13803333410012\\
222	0.0362115875140923\\
223	0.112508556997716\\
224	0.00492072431518957\\
225	0.0141581289172317\\
226	-0.0197276703929077\\
227	0.0834549195547045\\
228	0.23227820315365\\
229	0.0732053400641598\\
230	0.0144336996587433\\
231	-0.0272787330211376\\
232	0.0238914469341749\\
233	0.162125941547513\\
234	0.0877290034606728\\
235	0.0417879921061364\\
236	0.155411021937718\\
237	-0.00941043830190998\\
238	0.138194685593565\\
239	0.027954220969785\\
240	0.00136754272925214\\
241	-0.0252780656416822\\
242	-0.0919386524572901\\
243	0.0665961182252593\\
244	-0.105726711005937\\
245	-0.0474528158135998\\
246	0.116695784046653\\
247	0.102962311067144\\
248	-0.0823809768660973\\
249	0.0429709001119464\\
250	0.0399148361903201\\
251	0.114042220458798\\
252	-0.026146132380294\\
};
\addplot [color=dscr_plot_color,solid,line width=1pt,forget plot]
  table[row sep=crcr]{%
1	0.400025196631865\\
2	0.289485874992995\\
3	0.143207326104979\\
4	0.243534612788485\\
5	0.23273237047548\\
6	0.321566798063289\\
7	0.353725948682474\\
8	0.240002928416267\\
9	0.25770585528097\\
10	0.219076703553454\\
11	0.298949557450466\\
12	0.39296100834028\\
13	0.45820569588233\\
14	0.292822969364162\\
15	0.164316992992746\\
16	0.34163596830614\\
17	0.172440767072072\\
18	0.20295896933454\\
19	0.0994961896218226\\
20	0.137024469542746\\
21	0.158075644492136\\
22	0.355102889629264\\
23	0.211411050710463\\
24	0.154976180694573\\
25	0.300136897018099\\
26	0.399675729145874\\
27	0.151307908407647\\
28	0.13135293513083\\
29	0.242607741538753\\
30	0.110744857252593\\
31	0.137701019293833\\
32	0.197685834163507\\
33	0.119475404105709\\
34	0.430259917507794\\
35	0.569928951324634\\
36	0.219365380469569\\
37	0.168043520159332\\
38	0.382184956949107\\
39	0.132895937597891\\
40	0.228181848396486\\
41	0.210621201201309\\
42	0.220479849762868\\
43	0.314370929058208\\
44	0.263582718270233\\
45	0.183039829840106\\
46	0.151693839090666\\
47	0.293368955598131\\
48	0.17021097670925\\
49	0.159629438891879\\
50	0.208881013474258\\
51	0.111307683897903\\
52	0.184636566297049\\
53	0.21355416240471\\
54	0.205207948532088\\
55	0.261933874424987\\
56	0.219852062839298\\
57	0.249392049128776\\
58	-0.0428184873773782\\
59	0.22287204255617\\
60	0.1320697395179\\
61	0.268097115925098\\
62	0.136297325280829\\
63	0.424003789442383\\
64	0.292833012482542\\
65	0.261727501422772\\
66	0.0886749255115792\\
67	-0.0610766578155111\\
68	0.505145055965026\\
69	0.350789225158246\\
70	0.237405158695785\\
71	0.222315992049437\\
72	0.438474496255641\\
73	0.414560734833353\\
74	0.483882398665746\\
75	0.368617517441074\\
76	0.416301586563577\\
77	0.197430177545447\\
78	0.431532323362498\\
79	0.392375703280779\\
80	0.329906495294956\\
81	0.0981611612556981\\
82	0.399812549200479\\
83	0.28243717140467\\
84	0.07633978815363\\
85	0.418005946039606\\
86	0.236327526412087\\
87	0.162709764878771\\
88	0.270992847249807\\
89	0.204143419121148\\
90	0.223875198219955\\
91	0.254775191205843\\
92	0.194024329600252\\
93	0.276556895402207\\
94	0.114518886463483\\
95	0.214271077989601\\
96	0.210659286640705\\
97	0.232079647257014\\
98	0.233110999959891\\
99	0.184291008316362\\
100	0.244636985945496\\
101	0.214831811122427\\
102	0.341040387263145\\
103	0.27435146237683\\
104	0.254566345764663\\
105	0.175157981890586\\
106	0.439605533212178\\
107	0.438615042639915\\
108	0.356455549434069\\
109	0.345698184639719\\
110	0.375471176434594\\
111	0.313113526357139\\
112	0.36757207715139\\
113	0.0709416221678153\\
114	0.268290358867691\\
115	0.276679141303827\\
116	0.275069204433356\\
117	0.093835704590033\\
118	0.428782612063659\\
119	0.293660498715605\\
120	0.503533149629406\\
121	0.308300109163703\\
122	0.179479697708016\\
123	0.338610481255123\\
124	0.203473906669954\\
125	0.190967714336239\\
126	0.230348134959505\\
127	0.0207556124967796\\
128	0.322062702413738\\
129	0.172771853905164\\
130	0.105973225770651\\
131	0.209468175513551\\
132	0.594849762799291\\
133	0.351307634057394\\
134	0.119695303316431\\
135	0.251372084375492\\
136	0.369175993778912\\
137	0.40972633504259\\
138	0.184532873559399\\
139	0.115845608055244\\
140	0.102774469028738\\
141	0.148209803836426\\
142	0.308731434451249\\
143	0.0389804948730261\\
144	0.178460586093457\\
145	0.145564931429177\\
146	0.228581298697616\\
147	0.172958927676682\\
148	0.290221498465205\\
149	0.226202489771086\\
150	0.224054262503606\\
151	0.164182116430292\\
152	0.118562806984135\\
153	0.19814779100876\\
154	0.272938211089922\\
155	0.297317450149003\\
156	0.151590062341798\\
157	0.0782426753595337\\
158	0.108338340676328\\
159	0.116780629145222\\
160	0.266310664860373\\
161	0.0375722879569656\\
162	0.334313994093827\\
163	0.193067591204357\\
164	0.0945683640197671\\
165	0.295460463182257\\
166	0.141025956010348\\
167	0.170059100252603\\
168	0.409777608871741\\
169	0.342821356730035\\
170	0.234790248838244\\
171	0.158906521124168\\
172	0.201765036954251\\
173	0.25845115186197\\
174	0.159288398134003\\
175	0.136266441463357\\
176	0.0939117395263683\\
177	0.478013239843091\\
178	0.111135213296954\\
179	0.204906635518907\\
180	0.211068461257253\\
181	0.212856584986603\\
182	0.1302657773789\\
183	0.264169912690202\\
184	0.130887314402289\\
185	0.148028749158154\\
186	0.136431043810465\\
187	0.262025212646355\\
188	0.1364443298524\\
189	0.241477403681927\\
190	0.207918187591207\\
191	0.334477331877957\\
192	0.25252338219494\\
193	0.142888169325405\\
194	0.385118980173844\\
195	0.259601950501781\\
196	0.242717423219597\\
197	0.334942807702739\\
198	0.277083386116088\\
199	0.263849276931672\\
200	0.164608737552717\\
201	0.311227764988255\\
202	0.161831156549722\\
203	0.124662159070175\\
204	0.190175905915047\\
205	0.154885138646797\\
206	0.152752678054191\\
207	0.221228556609678\\
208	0.306518282995364\\
209	0.283061067964265\\
210	0.170942426750386\\
211	0.159236096137171\\
212	0.226451057169982\\
213	0.172323019436497\\
214	0.209952635183128\\
215	0.249718523091676\\
216	0.285596128671762\\
217	0.194485128359416\\
218	0.117046625199036\\
219	0.215614699658539\\
220	0.310325721564494\\
221	0.198987490177524\\
222	0.159648581371898\\
223	0.153184993040198\\
224	0.225814155323589\\
225	0.236255134343643\\
226	0.38244016652343\\
227	0.535003461699192\\
228	0.367347870882156\\
229	0.12883689423119\\
230	0.118319169438353\\
231	0.0940884259534607\\
232	0.271043466169644\\
233	0.236413726507604\\
234	0.289174964409929\\
235	0.353370719316426\\
236	0.296373550871967\\
237	0.120196128157652\\
238	0.25854734740425\\
239	0.242399693511239\\
240	0.230987706946474\\
241	0.163328693908056\\
242	0.100431646128926\\
243	0.0969579056847306\\
244	0.337428050798324\\
245	0.16791435040434\\
246	0.134798980496439\\
247	0.230652077519846\\
248	0.0463452603567209\\
249	0.251665763229872\\
250	0.096994551905227\\
251	0.245968407349172\\
252	0.171368477784535\\
};
\addplot [color=cts_nFPC_plot_color,solid,line width=1pt,forget plot]
  table[row sep=crcr]{%
1	0.409380580106947\\
2	0.27193430943486\\
3	0.178633165957182\\
4	0.148859863916421\\
5	0.149620386058862\\
6	0.34361760692377\\
7	0.36614825032781\\
8	0.23673791757725\\
9	0.206894746722539\\
10	0.211356413735345\\
11	0.295780453536132\\
12	0.384865099253492\\
13	0.35781698684539\\
14	0.17438113070194\\
15	0.170063852865386\\
16	0.232028978916103\\
17	0.181874464902405\\
18	0.223114389515592\\
19	0.230471626661809\\
20	0.141764963810315\\
21	0.109620945485864\\
22	0.278292721953501\\
23	0.173892989797102\\
24	0.213824308912708\\
25	0.200100541272195\\
26	0.281239543824297\\
27	0.198786873667823\\
28	0.139241686630429\\
29	0.228269441644398\\
30	0.107632184443418\\
31	0.169870855548386\\
32	0.143766718059292\\
33	0.128061188216047\\
34	0.131630966596605\\
35	-0.0954846673123338\\
36	0.147181374377658\\
37	0.210790733801998\\
38	0.375070826982312\\
39	0.343891284089822\\
40	0.119193926766073\\
41	0.319590625082945\\
42	0.225559793021216\\
43	0.229237427561732\\
44	0.241921523780251\\
45	0.192909010350702\\
46	0.108443354208804\\
47	0.302865258354305\\
48	0.142924600521329\\
49	0.174855726139533\\
50	0.182751857268009\\
51	0.0918169160759578\\
52	0.24180243653747\\
53	0.228939581764732\\
54	0.185538531146522\\
55	0.272640832981793\\
56	0.223928193814315\\
57	0.18178910613155\\
58	0.414482576829753\\
59	0.354672731465507\\
60	0.111470427760016\\
61	0.133252489477692\\
62	0.181725478681812\\
63	0.0885336621433856\\
64	0.279642889218802\\
65	0.270429410455935\\
66	0.253138638387396\\
67	0.273434374873234\\
68	0.392720125825363\\
69	0.375823631244993\\
70	0.210807158008849\\
71	0.179171019038376\\
72	0.492557778980544\\
73	0.423404373501245\\
74	0.492272856813045\\
75	0.439761483108558\\
76	0.424347593855491\\
77	0.487760676427891\\
78	0.46097684216883\\
79	0.0590160939723585\\
80	0.288649108290585\\
81	0.235549722358523\\
82	0.317057606915869\\
83	0.145178019536038\\
84	0.286178880151742\\
85	0.20119686854961\\
86	0.114231095510779\\
87	0.153127713450194\\
88	0.250894245375757\\
89	0.173864482674641\\
90	0.223289013824408\\
91	0.0577492439533075\\
92	0.085000101744777\\
93	0.205718958380988\\
94	0.107677638286136\\
95	0.199032046652932\\
96	0.232388102792328\\
97	0.288248213805986\\
98	0.210344287020462\\
99	0.267256848592102\\
100	0.248611174180272\\
101	0.193721719649013\\
102	0.342035188023794\\
103	0.206343995638513\\
104	0.228219352708996\\
105	0.356067573191545\\
106	0.258668397240784\\
107	0.230974298471121\\
108	0.21193763186034\\
109	0.333248838462054\\
110	0.353555588826717\\
111	0.270876264851258\\
112	0.130311540427804\\
113	0.346860300506474\\
114	0.220663944374247\\
115	0.287830332001458\\
116	0.247232970046368\\
117	0.00411269993083423\\
118	0.201302330785844\\
119	0.244306484449321\\
120	0.110321484448345\\
121	0.286837380682217\\
122	0.192233138788586\\
123	0.286896762673856\\
124	0.467512441305329\\
125	0.136551270367038\\
126	0.200089086389807\\
127	0.139754707102902\\
128	0.376603083845216\\
129	-0.112362630923675\\
130	0.0963751543522127\\
131	0.160238131673804\\
132	0.494160971320074\\
133	0.192763598562732\\
134	0.258852819620232\\
135	0.198793739942111\\
136	0.0772412568234956\\
137	0.361338737252661\\
138	0.216348137619985\\
139	0.0799999247508229\\
140	0.0887463865524794\\
141	0.158350641839326\\
142	0.281346916587545\\
143	0.288344441948139\\
144	0.178514861911213\\
145	0.187919095374932\\
146	0.208220503424345\\
147	0.186405806747519\\
148	0.296180267554823\\
149	0.252002000595093\\
150	0.220689717195736\\
151	0.1475334615605\\
152	0.0781904373778997\\
153	0.166892732909485\\
154	0.250505436609991\\
155	0.12836360005422\\
156	0.151438178202751\\
157	0.0980722749930247\\
158	0.0428009275899794\\
159	0.203565147048779\\
160	0.23460762973018\\
161	-0.0928936078142307\\
162	0.362410744668385\\
163	0.221298677773816\\
164	0.10870159556042\\
165	0.32369033678377\\
166	0.179326842861759\\
167	0.11460198778673\\
168	0.224127949158287\\
169	0.236202271186356\\
170	0.214297383420241\\
171	0.16279919907474\\
172	0.225286230661745\\
173	0.280749481582824\\
174	0.118911143173455\\
175	0.126703391579931\\
176	0.0826144520914961\\
177	0.491622823111333\\
178	0.0493655248960054\\
179	0.168911722625979\\
180	0.30318804875908\\
181	0.201703543347544\\
182	0.0706610087682732\\
183	0.17794680869249\\
184	0.110157856966721\\
185	0.110230615842454\\
186	0.0914484597764864\\
187	0.0802714998621025\\
188	0.168402777423651\\
189	0.139017606521816\\
190	0.171176339974814\\
191	0.223574319050662\\
192	0.249414827143633\\
193	0.213222108211419\\
194	0.15198747869594\\
195	0.215945917084024\\
196	0.242437963951711\\
197	0.29537203569712\\
198	0.241708620475021\\
199	0.256598645296036\\
200	0.28042826758908\\
201	0.299629141091671\\
202	0.166764359209815\\
203	0.274088336778844\\
204	0.184108704454264\\
205	0.129093431066118\\
206	0.168039784552491\\
207	0.251668373495046\\
208	0.31561851388323\\
209	0.184706359043372\\
210	0.142017640380836\\
211	0.117052993005956\\
212	0.0869291264767735\\
213	0.1909669015156\\
214	0.189691005175916\\
215	0.183460272819179\\
216	0.173166750528578\\
217	0.202027719958232\\
218	0.0930957313862146\\
219	0.138249271373509\\
220	0.305953074424776\\
221	0.210265138386714\\
222	0.244593446158739\\
223	0.221718798607988\\
224	0.210299006473937\\
225	0.157382541847867\\
226	0.335655189075567\\
227	0.155814562687372\\
228	0.235590066510059\\
229	0.0848620520949471\\
230	0.194292567063813\\
231	0.0376709917765818\\
232	0.145787988029175\\
233	0.266685767386097\\
234	0.283173014459159\\
235	0.346831904405895\\
236	0.283766440153289\\
237	0.171911302548391\\
238	0.190836017686627\\
239	-0.018794808486488\\
240	0.0811257407635893\\
241	0.053872168241526\\
242	0.0261874708864481\\
243	0.177658788689333\\
244	0.295416479804735\\
245	0.172544246960114\\
246	0.0304703150618709\\
247	0.172742471818508\\
248	0.0832171581474543\\
249	0.233200000360545\\
250	0.0297006908133427\\
251	0.218301474930152\\
252	0.122818939325019\\
};
\addplot [color=dscr_nFPC_plot_color,solid,line width=1pt,forget plot]
  table[row sep=crcr]{%
1	0.283273087692138\\
2	0.240316778927671\\
3	0.137653392760366\\
4	0.201236934782089\\
5	0.221140275876572\\
6	0.251296212310154\\
7	0.270557769072591\\
8	0.217977026580689\\
9	0.201250083044183\\
10	0.185208306121869\\
11	0.238339465663979\\
12	0.261169771904389\\
13	0.338185918164731\\
14	0.228966183863086\\
15	0.126543002623788\\
16	0.151940509535204\\
17	0.129892623893044\\
18	0.168731034247013\\
19	0.251184515727683\\
20	0.1583458534619\\
21	0.164593960788328\\
22	0.279105868178504\\
23	0.17376133822048\\
24	0.171316029127863\\
25	0.172821809253734\\
26	0.252598091995637\\
27	0.168954357199089\\
28	0.100642958012295\\
29	0.188485610610372\\
30	0.0989764054929328\\
31	0.132799948994482\\
32	0.136071627038489\\
33	0.127254994636462\\
34	0.08087665797576\\
35	0.111354823448533\\
36	0.142328870532445\\
37	0.130205052838473\\
38	0.348205073989094\\
39	0.275314450989157\\
40	0.136299055886808\\
41	0.320090601721998\\
42	0.277624593230822\\
43	0.257289582595687\\
44	0.224395373978443\\
45	0.213228047228006\\
46	0.140799475714779\\
47	0.29921413510376\\
48	0.137393880035162\\
49	0.147384908525771\\
50	0.197323559869352\\
51	0.116622001217631\\
52	0.174138123754495\\
53	0.139537260621815\\
54	0.180161290517066\\
55	0.211560413598572\\
56	0.15125387860992\\
57	0.125787149595859\\
58	0.377173961655109\\
59	0.26554493146359\\
60	0.116986877582731\\
61	0.056453939702643\\
62	0.154792580815771\\
63	0.12999231655439\\
64	0.204947200277858\\
65	0.209665157436112\\
66	0.194888746683896\\
67	0.294524304577694\\
68	0.419785927631256\\
69	0.336487305463687\\
70	0.184217482744818\\
71	0.138335161388858\\
72	0.345454118439998\\
73	0.363480053973329\\
74	0.359592653684877\\
75	0.24855273768956\\
76	0.33459961092932\\
77	0.324792148310527\\
78	0.363512295235183\\
79	0.178902745393211\\
80	0.26282220608998\\
81	0.16998484223114\\
82	0.263019176501658\\
83	0.183664923988541\\
84	0.238934331833857\\
85	0.148155833967507\\
86	0.131405047048942\\
87	0.114923145850592\\
88	0.227926738503707\\
89	0.157054960525412\\
90	0.172325139680059\\
91	0.114043160615563\\
92	0.0675482699161709\\
93	0.259976044856538\\
94	0.12063786444581\\
95	0.178178491144231\\
96	0.250204231541912\\
97	0.213122960428201\\
98	0.205246879298343\\
99	0.19422336790515\\
100	0.214020931823415\\
101	0.18836614260077\\
102	0.268840973091279\\
103	0.176192979489467\\
104	0.220030265393452\\
105	0.385065069201689\\
106	0.297235770043008\\
107	0.190663506137877\\
108	0.199613710179333\\
109	0.345840245466668\\
110	0.301675180345023\\
111	0.210672103905532\\
112	0.110655011953572\\
113	0.347288591725608\\
114	0.191450442542822\\
115	0.245353941828268\\
116	0.228123935413177\\
117	0.126394819916494\\
118	0.256853779744032\\
119	0.293139118367735\\
120	0.214653964445232\\
121	0.294154935349773\\
122	0.124279370181123\\
123	0.276192811541769\\
124	0.45951329793021\\
125	0.175305012966312\\
126	0.186092583923094\\
127	0.165932777248396\\
128	0.278157298857628\\
129	0.112953435447527\\
130	0.150969216962796\\
131	0.218189567929002\\
132	0.49036693108004\\
133	0.28789781485678\\
134	0.22838365469535\\
135	0.205794561316854\\
136	0.164159648177851\\
137	0.367575111792948\\
138	0.210280503476175\\
139	0.0895855818902679\\
140	0.10673885401738\\
141	0.113912371979412\\
142	0.22395712697017\\
143	0.241405766611521\\
144	0.12774057546991\\
145	0.133102895071512\\
146	0.132456514168308\\
147	0.178479549178216\\
148	0.235783750522147\\
149	0.194451896430442\\
150	0.232691407344574\\
151	0.133024873179756\\
152	0.110139243817735\\
153	0.159785045004232\\
154	0.193643744646674\\
155	0.191344866554914\\
156	0.175868871562936\\
157	0.107915475541562\\
158	0.0846068408555347\\
159	0.268030891426995\\
160	0.203658402509391\\
161	0.0508564099185756\\
162	0.176993373097338\\
163	0.196957232190911\\
164	0.141260081948049\\
165	0.232706056231707\\
166	0.157594387012881\\
167	0.147582672748645\\
168	0.20574509674136\\
169	0.162553219305875\\
170	0.225110208133347\\
171	0.162450741126946\\
172	0.206768424752949\\
173	0.219456510400262\\
174	0.163042393274452\\
175	0.130001100939952\\
176	0.0761447236803113\\
177	0.459698210875096\\
178	0.128157552239729\\
179	0.186392285368906\\
180	0.212710476948183\\
181	0.178402319391698\\
182	0.146430611396294\\
183	0.148017251278876\\
184	0.0983002951631574\\
185	0.0919187014010666\\
186	0.184831650690781\\
187	0.111439416106827\\
188	0.170579944139033\\
189	0.115653536587723\\
190	0.15838335891106\\
191	0.257248991952529\\
192	0.170759971397798\\
193	0.165232398129494\\
194	0.11122537234962\\
195	0.203180169512002\\
196	0.189460428867282\\
197	0.272761983575787\\
198	0.210208070092263\\
199	0.223625344823139\\
200	0.276918743366371\\
201	0.262106504556999\\
202	0.142909868159749\\
203	0.254316477075702\\
204	0.197011312048393\\
205	0.131448404084982\\
206	0.1245363354743\\
207	0.171582931655926\\
208	0.22235204774326\\
209	0.252182392440717\\
210	0.134548510607566\\
211	0.123938888858439\\
212	0.0886640687956931\\
213	0.136045242535645\\
214	0.162052673779445\\
215	0.153550337855269\\
216	0.143325695584478\\
217	0.176495716993906\\
218	0.0793907205854168\\
219	0.193708915239853\\
220	0.256557800957156\\
221	0.196553332515899\\
222	0.192383535404911\\
223	0.153964367190734\\
224	0.206541446830496\\
225	0.145044240403608\\
226	0.325670435538243\\
227	0.207984538466702\\
228	0.23075179976574\\
229	0.12456462402548\\
230	0.173456457851575\\
231	0.0896133221556759\\
232	0.146806969809239\\
233	0.195924987979096\\
234	0.25524905466495\\
235	0.292511388073308\\
236	0.285128107836885\\
237	0.155291041618852\\
238	0.20710478613284\\
239	0.0579010948829961\\
240	0.124683825178111\\
241	0.124424328729406\\
242	0.0995039212783834\\
243	0.187818890061266\\
244	0.255363058788249\\
245	0.210114773516928\\
246	0.117299555109926\\
247	0.213112131471332\\
248	0.132084222046144\\
249	0.2059814896587\\
250	0.0935808953524389\\
251	0.235375107982328\\
252	0.144734221927228\\
};
\end{axis}
\end{tikzpicture}%
 
\end{subfigure}\\

\leavevmode\smash{\makebox[0pt]{\hspace{-7em}% HORIZONTAL POSITION           
  \rotatebox[origin=l]{90}{\hspace{7em}% VERTICAL POSITION
    Normalized PnL}%
}}\hspace{0pt plus 1filll}\null

Trading Day Number of 2013

\vspace{1cm}
\begin{subfigure}{\linewidth}
  %\centering
  \setlength\figureheight{\linewidth} 
  \setlength\figurewidth{\linewidth}
  \tikzsetnextfilename{strategylegend}
  \resizebox{\linewidth}{!}{\definecolor{mycolor1}{rgb}{0.25098,0.00000,0.38824}%
\definecolor{mycolor2}{rgb}{0.00000,0.46275,0.00000}%
\definecolor{mycolor3}{rgb}{0.00000,0.34902,0.34902}%
\definecolor{mycolor4}{rgb}{0.58039,0.26275,0.00000}%
\begin{tikzpicture}
    \begingroup
    % inits/clears the lists (which might be populated from previous
    % axes):
    \csname pgfplots@init@cleared@structures\endcsname
    \pgfplotsset{legend style={at={(0,1)},anchor=north west},legend columns=-1,legend style={draw=black,column sep=1ex},
            legend entries={Cts Stoch Ctrl,Dscr Stoch Ctrl,Cts Stoch Ctrl w nFPC,Dscr Stoch Ctrl w nFPC}}%
    
    \csname pgfplots@addlegendimage\endcsname{line width=2pt,mycolor1,sharp plot}
    \csname pgfplots@addlegendimage\endcsname{line width=2pt,mycolor2,sharp plot}
    \csname pgfplots@addlegendimage\endcsname{line width=2pt,mycolor3,sharp plot}
    \csname pgfplots@addlegendimage\endcsname{line width=2pt,mycolor4,sharp plot}

    % draws the legend:
    \csname pgfplots@createlegend\endcsname
    \endgroup
\end{tikzpicture}
}
\end{subfigure}%
  \caption{End of day strategy performances: in-sample backtesting on 2013 data, using amalgamated annual 2013 data for calibration.}
  \label{fig:IS_annual_comp}
\end{figure}

\begin{table}
\centering
\ra{1.2}
\begin{tabular}{@{} *{9}{r} @{}}
\toprule
Strategy & Return & Sharpe & \# MO & \# LO & Inv & \% Win & Max Loss & Max Win \\
\midrule
\multicolumn{9}{l}{\texttt{ORCL}} \\ 
Cts & -0.089 & -0.875 & 1540 & 1383 & 1.19 & 0.14 & -0.705 & 0.393 \\ 
Dscr & 0.140 & 1.596 & 368 & 1344 & 0.46 & 0.96 & -0.088 & 0.487 \\ 
Cts w nFPC & 0.113 & 1.327 & 476 & 1338 & 2.67 & 0.94 & -0.140 & 0.484 \\ 
Dscr w nFPC & 0.118 & 1.735 & 590 & 1337 & 3.43 & 0.99 & -0.028 & 0.474 \\[2ex]
\multicolumn{9}{l}{\texttt{INTC}} \\ 
Cts & 0.065 & 0.743 & 888 & 1207 & 1.22 & 0.84 & -0.282 & 0.341 \\ 
Dscr & 0.235 & 2.189 & 380 & 1170 & 1.19 & 0.99 & -0.061 & 0.595 \\ 
Cts w nFPC & 0.209 & 2.030 & 396 & 1160 & 5.58 & 0.98 & -0.112 & 0.494 \\ 
Dscr w nFPC & 0.197 & 2.588 & 576 & 1164 & 3.78 & 1.00 &  & 0.490 \\ 
\bottomrule
\end{tabular}
\caption{Averaged strategy performance results: in-sample backtesting on 2013 data, using amalgamated annual 2013 data for calibration.}
\label{tbl:IS_annual}
\end{table}

\FloatBarrier
\section{Out-of-Sample Backtesting}
From the in-sample backtesting we draw several conclusions:
\begin{itemize}
\item underlying stocks need to be highly liquid;
\item underlying stocks need to have the smallest possible bid-ask spread;
\item calibrating over a larger period of time produces comparable returns and slightly improved Sharpe ratios, and is therefore preferred;
\item there is no clear victor between regular calibration and the nFPC method.
\end{itemize}
For out-of-sample backtesting we move to using 2014 data that has hitherto remained untouched. We elect to test all four strategies on two stocks, \texttt{INTC} and \texttt{AAPL}, that have average daily trading volumes of 30m and 45m respectively, and each have a typical bid-ask spread of the minimum 1 cent. \texttt{AAPL} underwent a 7-for-1 stock split on 2014-06-09, and prices are adjusted correspondingly in the underlying data; although prior to the split the bid-ask spread was an order of magnitude greater than 1 cent, we retain the 1 cent spread assumption on the adjusted pre-split prices to stay consistent with what was observed after the split. Additionally, we use a sliding calibration window of 1 month (21 trading days) and thus begin trading on the 22nd trading day of the year. 

\begin{figure}
\centering
\begin{subfigure}{.45\linewidth}
  \centering
  \setlength\figureheight{\linewidth} 
  \setlength\figurewidth{\linewidth}
  \tikzsetnextfilename{OOS_annual_INTC}
  % This file was created by matlab2tikz.
%
%The latest updates can be retrieved from
%  http://www.mathworks.com/matlabcentral/fileexchange/22022-matlab2tikz-matlab2tikz
%where you can also make suggestions and rate matlab2tikz.
%
%
\begin{tikzpicture}[trim axis left, trim axis right]

\begin{axis}[%
width=\figurewidth,
height=\figureheight,
at={(0\figurewidth,0\figureheight)},
scale only axis,
every outer x axis line/.append style={black},
every x tick label/.append style={font=\color{black}},
xmin=1,
xmax=252,
%xlabel={Time (h)},
every outer y axis line/.append style={black},
every y tick label/.append style={font=\color{black}},
ymin=-0.5,
ymax=2.5,
%ylabel={Normalized PnL},
title={INTC},
axis background/.style={fill=white},
axis x line*=bottom,
axis y line*=left,
yticklabel style={
        /pgf/number format/fixed,
        /pgf/number format/precision=3
},
scaled y ticks=false,
]
\addplot [color=cts_plot_color,solid,line width=1pt,forget plot]
  table[row sep=crcr]{%
1	0.26963564527033\\
2	0.0812791859976301\\
3	0.258353823089972\\
4	0.15375066826374\\
5	0.228086326367345\\
6	0.00266454176664818\\
7	0.150739182461208\\
8	-0.0644458160201035\\
9	0.250525165811975\\
10	-0.0285372062002428\\
11	0.203747624534175\\
12	0.22683416062517\\
13	0.204539234207424\\
14	0.13565308087828\\
15	0.0317687692307175\\
16	0.150204777884517\\
17	0.13271364583099\\
18	0.134897480299307\\
19	0.131319491506605\\
20	0.146890840586434\\
21	0.216529851766005\\
22	0.158858310518688\\
23	0.238241452954011\\
24	0.270521049539988\\
25	0.143719556639425\\
26	0.189414497397985\\
27	0.149020596653023\\
28	0.207340316730591\\
29	0.23792092507481\\
30	0.137422051983286\\
31	0.192376382841841\\
32	0.169928619540353\\
33	0.264152170973586\\
34	0.153683635295779\\
35	0.142520021356101\\
36	0.180871890047619\\
37	0.152846540684806\\
38	0.180991992124792\\
39	0.214234918752381\\
40	0.218064673442335\\
41	0.178464442986413\\
42	0.192860129299274\\
43	0.0842375583401652\\
44	0.384279415327714\\
45	0.173144733589689\\
46	0.228212061854239\\
47	0.151652151261517\\
48	0.204532786024088\\
49	0.163170341401239\\
50	0.211462893201068\\
51	0.185078717063482\\
52	0.1987472756297\\
53	0.194047354948597\\
54	0.174244446842754\\
55	0.189119742796312\\
56	0.199030406767797\\
57	0.193003098013475\\
58	0.217258056419755\\
59	0.173391125020354\\
60	0.143682148706881\\
61	0.148468507898605\\
62	0.198986770769308\\
63	0.185457370152702\\
64	0.150938247954111\\
65	0.125606029397477\\
66	0.168335192432876\\
67	0.138808441298083\\
68	0.137400435760667\\
69	0.15187022714063\\
70	0.140644766268518\\
71	0.0929942134693372\\
72	0.215480323637217\\
73	0.262338504529912\\
74	0.145758450019345\\
75	0.151275450604712\\
76	0.0534438078011343\\
77	0.0956144844396152\\
78	0.107329466952255\\
79	0.045108910668381\\
80	0.149376952011719\\
81	0.175905502929114\\
82	0.211292600547571\\
83	0.22565364241326\\
84	0.260802075855425\\
85	0.156382146075261\\
86	0.258945342929534\\
87	0.0850168600720129\\
88	0.265570399568249\\
89	0.0854455046515042\\
90	0.215153502438407\\
91	0.202230887113168\\
92	0.488961611221297\\
93	0.210220284619878\\
94	0.216692334027058\\
95	0.154499998447018\\
96	0.141034120002053\\
97	0.196537666281983\\
98	0.0716830272821748\\
99	0.175614630399463\\
100	0.392026200357144\\
101	0.315821525763047\\
102	0.141645708196059\\
103	0.034904645034563\\
104	0.223932864294619\\
105	-0.00301139156718667\\
106	0.179475025467556\\
107	0.288106415124689\\
108	0.270322782227211\\
109	0.231763037444338\\
110	0.235044492783833\\
111	0.242151485785364\\
112	0.283142805497629\\
113	0.410239161511713\\
114	0.457329738987022\\
115	0.163295557384188\\
116	0.0576422344031585\\
117	0.357266903302322\\
118	0.248757746039589\\
119	0.242130630474179\\
120	0.157057700486511\\
121	0.309180048060807\\
122	0.283555469628606\\
123	0.251315341750423\\
124	0.356696698483964\\
125	0.458118033458421\\
126	0.352989048833637\\
127	0.435612823980314\\
128	0.323941215981985\\
129	0.359916633729929\\
130	0.472228428689651\\
131	0.245733320362786\\
132	0.290200915209536\\
133	0.0715036032832169\\
134	0.281097669088852\\
135	0.265406976307438\\
136	0.170036310559606\\
137	0.17719066740288\\
138	0.280646708738482\\
139	-0.168861468056154\\
140	0.216660049689778\\
141	0.207726978849197\\
142	0.135314904905273\\
143	0.11650868619581\\
144	0.151975250219257\\
145	0.217026410045547\\
146	0.15524410913185\\
147	0.206243551871065\\
148	0.0699133688479681\\
149	0.141622012723338\\
150	0.241747506628732\\
151	0.224252632294149\\
152	0.161701823075847\\
153	0.10739841021324\\
154	0.292867162922703\\
155	0.236723852248063\\
156	0.301778375246838\\
157	0.258859523234779\\
158	0.222663439838195\\
159	0.245020590082906\\
160	0.224907286079295\\
161	0.252239218797636\\
162	0.186127122384629\\
163	0.170460229110415\\
164	0.207290277373665\\
165	0.364478133554319\\
166	0.223792380821682\\
167	0.342369486794757\\
168	0.229921450699335\\
169	0.273706321413039\\
170	0.174287954053077\\
171	0.190691419275898\\
172	0.0941464507426687\\
173	0.240257226040656\\
174	0.294616548373612\\
175	0.32186066540851\\
176	0.485112630185237\\
177	0.190239194471975\\
178	0.279471961772849\\
179	0.271365925377265\\
180	0.195290886392595\\
181	0.231170779113027\\
182	0.184397587115079\\
183	0.134930576506312\\
184	0.224089720228972\\
185	0.278560185795872\\
186	0.121069174307033\\
187	0.133158089650564\\
188	0.50178704744723\\
189	0.362567554371031\\
190	0.240806600831905\\
191	0.020784597456922\\
192	0.384424121703955\\
193	0.208887052544037\\
194	0.165595879016578\\
195	0.311471846871098\\
196	0.180777500213411\\
197	0.229751859247128\\
198	0.059276885845721\\
199	0.102799770615309\\
200	0.174098004828106\\
201	0.254053910153812\\
202	0.128744654348108\\
203	0.490190973864465\\
204	0.321416793127848\\
205	0.268611762334539\\
206	0.292449905369116\\
207	0.378407822058957\\
208	0.346020049635607\\
209	0.233442674888068\\
210	0.250753983279198\\
211	0.174683799564852\\
212	0.164391811815535\\
213	0.190571676502179\\
214	0.135897002147125\\
215	0.313678451824508\\
216	0.336960341145604\\
217	0.213052382876897\\
218	0.300299501516183\\
219	0.215036330072671\\
220	0.219294973212254\\
221	0.382081482096947\\
222	0.381633897082985\\
223	0.24677589631451\\
224	0.21403387585827\\
225	0.0686160494410331\\
226	0.198850541000567\\
227	0.233261180209299\\
228	0.0645541918868825\\
};
\addplot [color=dscr_plot_color,solid,line width=1pt,forget plot]
  table[row sep=crcr]{%
1	0.441997056889738\\
2	0.458233304284025\\
3	0.418471389199401\\
4	0.419234295384876\\
5	0.127552429123174\\
6	0.0796121196023828\\
7	0.202368102971599\\
8	0.271997015033251\\
9	0.0857094245573307\\
10	0.263960810630823\\
11	0.0554872054849208\\
12	0.236343824454965\\
13	0.0209659235606106\\
14	0.257873000756714\\
15	0.0608117960876181\\
16	0.174978399376741\\
17	0.32004932114849\\
18	0.24266544543884\\
19	0.240914574064539\\
20	0.323668351713643\\
21	0.310469306838916\\
22	0.225709470087306\\
23	0.107706802380786\\
24	0.330137661943864\\
25	-0.0101153514236141\\
26	0.131490150492565\\
27	0.315546644353201\\
28	0.37517513504854\\
29	0.130409224088904\\
30	0.101828419323733\\
31	0.0998818754669931\\
32	0.508742884606918\\
33	0.462665254414907\\
34	0.432326193122878\\
35	0.467169477869723\\
36	0.437176474362773\\
37	0.157355344246135\\
38	0.407835337669032\\
39	0.338243433626207\\
40	0.124076619144464\\
41	-0.0405716182279245\\
42	0.232958706445838\\
43	0.167458586462093\\
44	0.504429284894597\\
45	0.349568451452751\\
46	0.615332434672446\\
47	0.447192573173332\\
48	0.451207318256972\\
49	0.585387680158007\\
50	0.0896605918252703\\
51	0.721380422989914\\
52	0.533777667296209\\
53	0.140101469021222\\
54	0.180596690307499\\
55	0.236319225699871\\
56	0.397199609860212\\
57	0.194928572246658\\
58	0.351168328545488\\
59	0.252754659627726\\
60	0.123120692509247\\
61	0.123347838899411\\
62	0.253683698135599\\
63	0.423955495912861\\
64	0.0983369543420489\\
65	0.0130255437892288\\
66	0.08318138711644\\
67	0.225741836150945\\
68	0.379246813378094\\
69	0.0895323112943477\\
70	0.267174781542726\\
71	0.245486387600415\\
72	0.301704848320736\\
73	0.222532495158896\\
74	0.289105044726425\\
75	0.1421450258803\\
76	0.0565186572378707\\
77	-0.0090057084367307\\
78	0.231017407906282\\
79	0.31270517007323\\
80	0.265077921282586\\
81	0.157779619894752\\
82	0.173797116426202\\
83	0.326594662928318\\
84	0.0175443499031492\\
85	0.236100307304464\\
86	0.242425591463638\\
87	0.199519975733049\\
88	0.513603538272743\\
89	0.209908850004939\\
90	0.444400332593423\\
91	0.210945642470453\\
92	0.652785629998448\\
93	0.297957427163893\\
94	0.290466476971488\\
95	0.152633902308132\\
96	0.162930520827283\\
97	0.148302635533775\\
98	0.108377719925966\\
99	0.369756303400631\\
100	0.197213773837795\\
101	0.363949556126501\\
102	0.309348394742462\\
103	0.292370757128491\\
104	0.417333972480264\\
105	0.291049431909279\\
106	0.271067874727222\\
107	0.364220408611214\\
108	0.389582320392795\\
109	-0.0216957169073585\\
110	0.394158027682089\\
111	0.345934500395396\\
112	0.434094443967498\\
113	0.446988486558916\\
114	0.965020714428128\\
115	0.262264857615027\\
116	0.491818002746111\\
117	0.563558199130077\\
118	0.195332712364632\\
119	0.47197638072801\\
120	0.306874655592442\\
121	0.316062120176029\\
122	0.134614102600616\\
123	0.189209550147757\\
124	0.610803997841028\\
125	0.773172715096116\\
126	0.131160295087331\\
127	0.903739306448733\\
128	0.393172845883635\\
129	0.524763642248934\\
130	0.728699231507389\\
131	0.103518770517311\\
132	0.472974223004033\\
133	-0.0133687739298978\\
134	0.214201321169624\\
135	0.18482438021321\\
136	0.385421759030822\\
137	0.298936137228685\\
138	0.467121218345748\\
139	0.312275694654746\\
140	0.231948058844622\\
141	0.345688708520898\\
142	0.30596408697264\\
143	0.229122166643363\\
144	0.290419900275788\\
145	0.211774128098993\\
146	0.380140896150045\\
147	0.396805253353331\\
148	0.114987179307665\\
149	0.443290266916919\\
150	0.615359237397012\\
151	0.322190260820541\\
152	0.448351699883985\\
153	0.291794908096244\\
154	0.478597493771408\\
155	0.531014666351418\\
156	0.231478249055541\\
157	0.457496655346694\\
158	0.42143405905386\\
159	0.55153244337475\\
160	0.386384231224012\\
161	0.422942166322309\\
162	0.252613395439516\\
163	0.318789151753325\\
164	0.394194261985352\\
165	0.651806146605628\\
166	0.495521213851066\\
167	0.680707809063964\\
168	0.69822257954559\\
169	0.539427359618056\\
170	0.62496861118153\\
171	0.573095622141557\\
172	0.429544268763001\\
173	0.950166893392434\\
174	1.41792238398612\\
175	0.814800920828198\\
176	0.968861581868494\\
177	1.12895162725433\\
178	1.26668843735552\\
179	0.884959708604826\\
180	0.801532307436512\\
181	0.367328912433657\\
182	0.679598936067132\\
183	0.648422353695511\\
184	0.194714623632761\\
185	0.745110531290363\\
186	0.556700477237455\\
187	0.391545597904276\\
188	0.955124080948623\\
189	0.421607199484748\\
190	0.42106135006137\\
191	0.677710526542774\\
192	0.322448526774788\\
193	0.482191507386851\\
194	0.50367914683191\\
195	0.535553328270742\\
196	0.391783169714067\\
197	0.464787066713261\\
198	0.253043310973074\\
199	0.357658202308609\\
200	0.560817580638952\\
201	0.634138613836285\\
202	0.270348573752544\\
203	0.347741867118189\\
204	0.776140050772969\\
205	0.3901587047626\\
206	0.325488227131837\\
207	0.0712602916953453\\
208	0.361444723650599\\
209	0.318604996768405\\
210	0.448470618353861\\
211	0.264678811250693\\
212	0.197332535165579\\
213	0.720233609765952\\
214	0.501477842815664\\
215	0.511393651859714\\
216	0.29403130395263\\
217	0.892152892925317\\
218	0.710760536886796\\
219	0.623916832187122\\
220	0.255456816336281\\
221	0.442414153165319\\
222	0.806856491594673\\
223	0.256680637304243\\
224	0.534325025619459\\
225	0.0833247533131612\\
226	0.396014089362232\\
227	0.334916075613173\\
228	0.563247735737518\\
};
\addplot [color=cts_nFPC_plot_color,solid,line width=1pt,forget plot]
  table[row sep=crcr]{%
1	0.6143130460286\\
2	0.479930045169186\\
3	0.395687042528969\\
4	0.437680042861911\\
5	0.339412958095725\\
6	0.268583810098898\\
7	0.214741031069543\\
8	0.259989868511312\\
9	0.365682632156957\\
10	0.261371395812812\\
11	0.325121448560579\\
12	0.358345956387868\\
13	0.336931285447313\\
14	0.195134682784106\\
15	0.306473805844229\\
16	0.283472686814215\\
17	0.339020300929932\\
18	0.357114288622482\\
19	0.279691256840746\\
20	0.385718894427536\\
21	0.314944909234571\\
22	0.229341799515829\\
23	0.320358906572409\\
24	0.421980101396289\\
25	0.288599031723427\\
26	0.28223425709022\\
27	0.302793797708563\\
28	0.3851793902035\\
29	0.38323615022619\\
30	0.249959668586474\\
31	0.288848474529826\\
32	0.494468165623729\\
33	0.474606410430397\\
34	0.338058625019283\\
35	0.470671354705772\\
36	0.467075685549844\\
37	0.391668357741845\\
38	0.481964455628875\\
39	0.350921896784284\\
40	0.397572497620522\\
41	0.340835024510644\\
42	0.307555999868785\\
43	0.513283632669537\\
44	0.708241722672039\\
45	0.715256802026324\\
46	0.57651944450147\\
47	0.41455306651409\\
48	0.530996918914574\\
49	0.540449524887044\\
50	0.566215379881527\\
51	0.72668052256212\\
52	0.672422060412128\\
53	0.421132753233327\\
54	0.294486786651721\\
55	0.375290488637248\\
56	0.397771752437363\\
57	0.379520390483487\\
58	0.385436348368195\\
59	0.523440446631788\\
60	0.321760128221773\\
61	0.356777677997279\\
62	0.381024224555597\\
63	0.412970324091504\\
64	0.29411464088543\\
65	0.341986464474481\\
66	0.458760927536746\\
67	0.352059362915712\\
68	0.369615974203577\\
69	0.312037162787901\\
70	0.26682291862681\\
71	0.25775898336278\\
72	0.463459629157065\\
73	0.440889359704075\\
74	0.308115343416008\\
75	0.313947832265368\\
76	0.288464284088744\\
77	0.20112925319551\\
78	0.244941812053245\\
79	0.321729251208538\\
80	0.266891123062674\\
81	0.241290245588802\\
82	0.371884435521172\\
83	0.330283611456308\\
84	0.419558393629129\\
85	0.249081168070741\\
86	0.397607253089301\\
87	0.308391157851021\\
88	0.38482423389251\\
89	0.350035763773043\\
90	0.327141943033486\\
91	0.349920348898649\\
92	0.860732190446409\\
93	0.490696014004677\\
94	0.332274720034596\\
95	0.331837771997554\\
96	0.347288929152006\\
97	0.28183379533041\\
98	0.371908043123139\\
99	0.534427972798377\\
100	0.600488041210758\\
101	0.464816979699518\\
102	0.328770456986064\\
103	0.319545216410469\\
104	0.434670651070785\\
105	0.317602655621071\\
106	0.300558370027516\\
107	0.492895826334102\\
108	0.436140561080273\\
109	0.496162960244502\\
110	0.397342679498699\\
111	0.34350110621099\\
112	0.605828989718724\\
113	1.12160114247488\\
114	0.877520174669197\\
115	0.601985857132689\\
116	0.505117046294082\\
117	0.580147029450329\\
118	0.364701368002105\\
119	0.461390103271463\\
120	0.475961791452906\\
121	0.636312675435545\\
122	0.461184839970346\\
123	0.524407487976284\\
124	0.627356385918564\\
125	0.779287921647462\\
126	0.528132719464313\\
127	0.806808394032914\\
128	0.713179706889436\\
129	0.644634376292113\\
130	0.750479293565096\\
131	0.466589214432448\\
132	0.471596066345577\\
133	0.591344207392925\\
134	0.475290201438928\\
135	0.579483599004693\\
136	0.0920632659524396\\
137	0.305517603023685\\
138	0.448485306231749\\
139	-0.156220844596798\\
140	0.377601692703897\\
141	0.321355729328189\\
142	0.269449593531328\\
143	0.230196222342016\\
144	0.297067668534244\\
145	0.400282916784218\\
146	0.29773848365603\\
147	0.405130075847202\\
148	0.212734903030616\\
149	0.435126486220765\\
150	0.602114149302921\\
151	0.503808779652743\\
152	0.42710373673571\\
153	0.507270310228639\\
154	0.688778305910511\\
155	0.56841094598978\\
156	0.624685768111744\\
157	0.483911950107121\\
158	0.43197846366263\\
159	0.514815411763773\\
160	0.594440882113467\\
161	0.446133995355554\\
162	0.470665743973826\\
163	0.614662874395886\\
164	0.578876589338569\\
165	0.79398237024307\\
166	0.616583372615095\\
167	0.753937576151404\\
168	0.513372693258185\\
169	0.570371098114885\\
170	0.628141878020989\\
171	0.541215334926928\\
172	0.819242822106041\\
173	0.974200058605348\\
174	1.1428078838766\\
175	1.19442572473451\\
176	0.99447293097858\\
177	0.934276617957967\\
178	1.15350480992241\\
179	0.81820187395044\\
180	0.540691902475096\\
181	0.267658458706957\\
182	0.704204288496644\\
183	0.676524025615314\\
184	0.473728962275772\\
185	0.506625115894345\\
186	0.340421751589954\\
187	0.343948646288413\\
188	1.01198712198112\\
189	0.797312783345937\\
190	0.650620432524172\\
191	0.181751201276961\\
192	0.644353240978866\\
193	0.48010862920146\\
194	0.547020736786603\\
195	0.429320415872187\\
196	0.415714348731314\\
197	0.486656626384004\\
198	0.576145975331464\\
199	0.23402762688151\\
200	0.574212014030322\\
201	0.629827558237091\\
202	0.547943448198315\\
203	0.758425130020845\\
204	0.613676310362735\\
205	0.706066874063763\\
206	0.546946790915805\\
207	0.644994233831756\\
208	0.858924282139581\\
209	0.382715157200909\\
210	0.706751384614788\\
211	0.60271264361827\\
212	0.452936961606829\\
213	0.641518292045949\\
214	0.941727566678151\\
215	0.834407948052902\\
216	0.864800493013229\\
217	0.710179449388589\\
218	0.993267266063656\\
219	0.836689268716272\\
220	0.647034644669472\\
221	0.552319077243002\\
222	0.680901338974365\\
223	0.346174738416085\\
224	0.41058763900373\\
225	0.288588065988688\\
226	0.436397261638319\\
227	0.405515094425726\\
228	0.450649936257121\\
};
\addplot [color=dscr_nFPC_plot_color,solid,line width=1pt,forget plot]
  table[row sep=crcr]{%
1	0.61183707372905\\
2	0.466620129440472\\
3	0.394683124208813\\
4	0.437022836894612\\
5	0.344573007145789\\
6	0.282851712672088\\
7	0.197207407774155\\
8	0.267218441654412\\
9	0.352372969236646\\
10	0.261484994028186\\
11	0.301613950547206\\
12	0.324321085967892\\
13	0.319203853570605\\
14	0.193670456555661\\
15	0.301887798736334\\
16	0.274618007001055\\
17	0.333518577183689\\
18	0.366021308168541\\
19	0.248674800961759\\
20	0.398834743605126\\
21	0.276907388594029\\
22	0.233783670958193\\
23	0.332944360032859\\
24	0.418581845809891\\
25	0.296090048655316\\
26	0.273159968348499\\
27	0.304526968525399\\
28	0.370341944333678\\
29	0.34449717965225\\
30	0.253937264852245\\
31	0.296780191115426\\
32	0.486003506011523\\
33	0.458172884369276\\
34	0.333708821644002\\
35	0.462627588633211\\
36	0.431960709590673\\
37	0.373107971250565\\
38	0.448924510940559\\
39	0.33334121923638\\
40	0.388203124130375\\
41	0.340447211926684\\
42	0.32742636934098\\
43	0.490659028835647\\
44	0.718549820702009\\
45	0.670509686346988\\
46	0.598172555645278\\
47	0.436633672612086\\
48	0.509143832143232\\
49	0.551869076374524\\
50	0.53214049759813\\
51	0.719395596354385\\
52	0.686159032286584\\
53	0.424892664560722\\
54	0.319932380990227\\
55	0.354325652330032\\
56	0.373804329198854\\
57	0.381014367797213\\
58	0.398937162641295\\
59	0.53954522058515\\
60	0.347548468617298\\
61	0.359514983246846\\
62	0.380252442821248\\
63	0.403043454867974\\
64	0.300420036702677\\
65	0.335704252356505\\
66	0.433742198071703\\
67	0.343009562421407\\
68	0.355035669913453\\
69	0.299669809463617\\
70	0.255010718434588\\
71	0.237623542440356\\
72	0.441655633110088\\
73	0.417229909972626\\
74	0.277405652787983\\
75	0.312250795184475\\
76	0.285206997869146\\
77	0.192416921825856\\
78	0.244699634034665\\
79	0.319392525372523\\
80	0.273615756350554\\
81	0.234098171453855\\
82	0.345795162830772\\
83	0.308326468758271\\
84	0.409363212153008\\
85	0.24978418167321\\
86	0.369596719215507\\
87	0.310088791961515\\
88	0.368596578058314\\
89	0.306042448086602\\
90	0.321482156500573\\
91	0.355035190359323\\
92	0.876045539876714\\
93	0.472307336675702\\
94	0.316690064588974\\
95	0.318955325077107\\
96	0.355925626333873\\
97	0.290895711414747\\
98	0.372477427031126\\
99	0.531165242461047\\
100	0.582814797433697\\
101	0.474795200066837\\
102	0.349562171000063\\
103	0.305562670786552\\
104	0.416969105461709\\
105	0.303036569756862\\
106	0.288727095077769\\
107	0.480284693747882\\
108	0.386268607123281\\
109	0.484777221701001\\
110	0.39838347963515\\
111	0.332223180031924\\
112	0.610375759876839\\
113	1.22934602650305\\
114	0.923197050723522\\
115	0.574684147374411\\
116	0.478881922705708\\
117	0.558833030217281\\
118	0.388078279830345\\
119	0.46155486135149\\
120	0.457026590166114\\
121	0.634457999430658\\
122	0.441220416437272\\
123	0.530989908958742\\
124	0.643813195859474\\
125	0.75780192173065\\
126	0.529499419613188\\
127	0.811385260812246\\
128	0.711894681611713\\
129	0.630476049675469\\
130	0.715368563828518\\
131	0.456320742191189\\
132	0.478137451812145\\
133	0.590157955531636\\
134	0.425839528538212\\
135	0.589922320978104\\
136	0.36260767905207\\
137	0.289002892665452\\
138	0.460498205436308\\
139	0.320992096460161\\
140	0.360295826079102\\
141	0.338387682884993\\
142	0.275147006254992\\
143	0.224866615019689\\
144	0.287742833153562\\
145	0.388742470207407\\
146	0.386866286895132\\
147	0.426206333167008\\
148	0.412320434805305\\
149	0.429643201720446\\
150	0.614149912256539\\
151	0.480322179227098\\
152	0.447881796988888\\
153	0.488366938099405\\
154	0.66034536132001\\
155	0.532795255170833\\
156	0.644817367688909\\
157	0.517119759577625\\
158	0.427461619817519\\
159	0.490598804493259\\
160	0.636849920990711\\
161	0.461819470358364\\
162	0.464586044099153\\
163	0.645982020528883\\
164	0.547049407391825\\
165	0.778966710029067\\
166	0.595285276132172\\
167	0.738129379116837\\
168	0.646473260784414\\
169	0.555043508355322\\
170	0.603213532473551\\
171	0.509503545879259\\
172	0.844802719081944\\
173	0.92802465980639\\
174	1.55398272353678\\
175	1.1342870808597\\
176	1.33754657326245\\
177	2.07162688110104\\
178	1.48405035960051\\
179	1.02974185784727\\
180	0.772742110300907\\
181	0.78889126881632\\
182	0.652203178397897\\
183	0.618920410516353\\
184	0.687614006461787\\
185	0.747948863885012\\
186	0.559829799337336\\
187	0.596254225566214\\
188	1.02855992251955\\
189	0.803903610425394\\
190	0.634162482408358\\
191	0.649654588813321\\
192	0.617738661328253\\
193	0.590543927761638\\
194	0.531900513415453\\
195	0.512827672932275\\
196	0.435104748886764\\
197	0.486466963956121\\
198	0.527544433379397\\
199	0.528124417781775\\
200	0.541514120653209\\
201	0.587564613594762\\
202	0.488679523640645\\
203	1.33838736344592\\
204	0.708431844757965\\
205	0.66967559080376\\
206	0.524206902858684\\
207	0.609217289937578\\
208	0.852130301356892\\
209	0.739238028597512\\
210	0.677504640390143\\
211	0.55882001008687\\
212	0.443030014053196\\
213	0.878715210639802\\
214	0.956193332481327\\
215	0.778955134733541\\
216	0.887600466724404\\
217	0.901012765523267\\
218	0.955994055705178\\
219	1.01593272600577\\
220	1.00390361795808\\
221	0.816658835565289\\
222	0.68737504970322\\
223	0.705917331394664\\
224	0.528333013403345\\
225	0.277274120974258\\
226	0.401202478704402\\
227	0.383694447047517\\
228	0.479012741293375\\
};
\end{axis}
\end{tikzpicture}%

\end{subfigure}%
\hfill%
\begin{subfigure}{.45\linewidth}
  \centering
  \setlength\figureheight{\linewidth} 
  \setlength\figurewidth{\linewidth}
  \tikzsetnextfilename{OOS_annual_AAPL}
  % This file was created by matlab2tikz.
%
%The latest updates can be retrieved from
%  http://www.mathworks.com/matlabcentral/fileexchange/22022-matlab2tikz-matlab2tikz
%where you can also make suggestions and rate matlab2tikz.
%
%
\begin{tikzpicture}[trim axis left, trim axis right]

\begin{axis}[%
width=\figurewidth,
height=\figureheight,
at={(0\figurewidth,0\figureheight)},
scale only axis,
every outer x axis line/.append style={black},
every x tick label/.append style={font=\color{black}},
xmin=1,
xmax=252,
%xlabel={Time (h)},
every outer y axis line/.append style={black},
every y tick label/.append style={font=\color{black}},
ymin=-0.5,
ymax=2.5,
%ylabel={Normalized PnL},
title={AAPL},
axis background/.style={fill=white},
axis x line*=bottom,
axis y line*=left,
yticklabel style={
        /pgf/number format/fixed,
        /pgf/number format/precision=3
},
scaled y ticks=false,
]
\addplot [color=cts_plot_color,solid,line width=1pt,forget plot]
  table[row sep=crcr]{%
1	0.510987910390083\\
2	0.50493899806683\\
3	0.54333235039816\\
4	0.497479959177795\\
5	0.498866792659851\\
6	0.583676323725802\\
7	0.523166608602107\\
8	0.435874820908009\\
9	0.567830213116335\\
10	0.518426123215793\\
11	0.435644759110937\\
12	0.436771436444335\\
13	0.468266810055287\\
14	0.580023023511957\\
15	0.46352318185389\\
16	0.437173201793071\\
17	0.377340237152658\\
18	0.644177546256084\\
19	0.488407638526946\\
20	0.473846235744826\\
21	0.328366867926963\\
22	0.411824769917912\\
23	0.257628422878803\\
24	0.419336926576297\\
25	0.284582981199671\\
26	0.551447511265182\\
27	0.343543349787213\\
28	0.470603633864958\\
29	0.395418892323493\\
30	0.435937969426288\\
31	0.366070562879117\\
32	0.393248546480879\\
33	0.322909586908209\\
34	0.433433353306567\\
35	0.459980964463616\\
36	0.522403580488045\\
37	0.440135131708724\\
38	0.407740874646837\\
39	0.359496854881543\\
40	0.212926998888887\\
41	0.0708760847158397\\
42	0.120855795950528\\
43	0.227491801523226\\
44	0.457576096279289\\
45	0.409559707438116\\
46	0.176433420422788\\
47	0.117898075662125\\
48	0.474786049178819\\
49	0.289103379809835\\
50	0.123284435041617\\
51	0.355734383354284\\
52	0.241006285248412\\
53	0.135990744828084\\
54	0.348533898052694\\
55	0.0374594163111564\\
56	0.416587117781412\\
57	0.507876772983342\\
58	0.748064449494588\\
59	1.02485187605485\\
60	0.573285316623427\\
61	0.567844134447056\\
62	0.473715049866141\\
63	0.419608150960151\\
64	0.53416621312512\\
65	0.404616468188424\\
66	0.454675816563084\\
67	0.399771432829398\\
68	0.486904313452766\\
69	0.434953267733016\\
70	0.370062267833206\\
71	0.33021234317118\\
72	0.376413371174028\\
73	0.480376772247229\\
74	0.595014729615327\\
75	0.359143532690909\\
76	0.381733434019907\\
77	0.331805625875993\\
78	0.489379006333125\\
79	0.607107849238533\\
80	0.525190544353949\\
81	0.60136129887222\\
82	0.736187211753486\\
83	0.590861209859749\\
84	0.54081257054675\\
85	0.549193877837785\\
86	0.513938123262339\\
87	0.563538927572976\\
88	1.39174913662952\\
89	1.06123649601527\\
90	0.909921213617733\\
91	0.870278732836959\\
92	0.869686433140742\\
93	0.670608290242773\\
94	0.599688346873008\\
95	0.591595198860895\\
96	0.62536824396781\\
97	0.599380940066415\\
98	0.608238454014307\\
99	0.572288105692664\\
100	0.684094620826021\\
101	0.264464953575519\\
102	0.242272183874215\\
103	0.286686858049356\\
104	0.440685455801696\\
105	0.433852067585813\\
106	0.254831895547666\\
107	0.864063620586768\\
108	0.227658085017016\\
109	0.00807582560044849\\
110	0.206283731387004\\
111	0.120221406652525\\
112	0.317286039208535\\
113	0.389013192673175\\
114	0.441713587462249\\
115	0.0583722396375666\\
116	0.344621790605737\\
117	0.168916486344438\\
118	-0.0265651591675015\\
119	0.154059026035574\\
120	0.0938252319507085\\
121	-0.016160260163571\\
122	0.237119467306883\\
123	0.177862639984112\\
124	0.521481897491611\\
125	0.0900766131325231\\
126	0.388269595772568\\
127	0.225595336495003\\
128	0.320536904903836\\
129	0.333620645793912\\
130	0.240874820690198\\
131	0.0503960929392621\\
132	0.239320678484777\\
133	-0.000976002136468723\\
134	0.12464431947248\\
135	0.241596975901332\\
136	0.133327137827351\\
137	0.121822078766212\\
138	0.122419301940224\\
139	0.167293373148065\\
140	0.0981874192905491\\
141	0.146526835025966\\
142	0.255848991808334\\
143	-0.0325730096281847\\
144	0.157770982340905\\
145	0.183713842621101\\
146	0.190241392691877\\
147	0.984068324753296\\
148	0.304072732178635\\
149	0.0930642553527545\\
150	0.340735276245742\\
151	0.556380221872139\\
152	-0.0170946287958654\\
153	0.0288735844085977\\
154	0.433914554861058\\
155	0.277351848471341\\
156	0.291285349057398\\
157	0.854272414320157\\
158	0.484865397465263\\
159	0.664879775022002\\
160	0.249591700864865\\
161	0.249940209786659\\
162	0.368704231617722\\
163	0.682903821955991\\
164	0.0659224181477588\\
165	0.847220557837157\\
166	0.337313521156171\\
167	0.434664951730829\\
168	0.320559975020865\\
169	0.193371944435979\\
170	0.31022860042775\\
171	0.333773847928745\\
172	0.19618191008808\\
173	0.591095055851438\\
174	0.579012455065681\\
175	0.695330152417599\\
176	0.643074899823606\\
177	1.004276615396\\
178	0.751402356130219\\
179	0.438676136204901\\
180	0.149257931471269\\
181	0.263571433492917\\
182	0.258680367087105\\
183	0.278454690789311\\
184	0.292016068969924\\
185	0.238189085057527\\
186	0.102324625661413\\
187	0.162482953670979\\
188	0.225289623547628\\
189	0.202583113576629\\
190	0.198730035172748\\
191	0.172514968095086\\
192	0.208502729628415\\
193	0.130171157342652\\
194	0.210914337667021\\
195	0.183132856647801\\
196	0.0948476083282829\\
197	-0.0151176467513982\\
198	0.25021846742453\\
199	0.0824319451460317\\
200	0.232247987500126\\
201	-0.094486593950673\\
202	0.378670181803119\\
203	0.0591114198662748\\
204	0.436557111418969\\
205	-0.0825038280628991\\
206	0.612610999445504\\
207	0.0145678775478772\\
208	0.852020514419013\\
209	0.0220201228365819\\
210	0.249433094466648\\
211	0.211775295229903\\
212	0.215543504570929\\
213	0.510812445064065\\
214	0.0424863680110536\\
215	0.640344373483863\\
216	0.466728203057596\\
217	0.710346831073175\\
218	1.0832979135758\\
219	0.706973517141022\\
220	0.618711183426892\\
221	0.405786491068403\\
222	0.416112907257908\\
223	0.31620314832479\\
224	0.258904007018663\\
225	-0.0286929094134621\\
226	0.154216793162184\\
227	0.180243562066718\\
228	0.380460190665007\\
};
\addplot [color=dscr_plot_color,solid,line width=1pt,forget plot]
  table[row sep=crcr]{%
1	0.572635042165019\\
2	0.525371269501357\\
3	0.544273463843038\\
4	0.547893744695127\\
5	0.573476863400697\\
6	0.643483010002963\\
7	0.531551497826567\\
8	0.407054615353053\\
9	0.624037773736192\\
10	0.53512297544551\\
11	0.45468548974163\\
12	0.510906798837668\\
13	0.502453952383931\\
14	0.566745870574096\\
15	0.547159396725753\\
16	0.471147187418905\\
17	0.396233396559053\\
18	0.701668524082678\\
19	0.56005406423644\\
20	0.495003931317454\\
21	0.352615470588501\\
22	0.457295451043031\\
23	0.258543741797873\\
24	0.436719387826526\\
25	0.318297728040905\\
26	0.577368424738107\\
27	0.367655938271027\\
28	0.487547671081847\\
29	0.415472216315897\\
30	0.473525460460465\\
31	0.420294586361968\\
32	0.410321559384572\\
33	0.342244805015436\\
34	0.469239534803419\\
35	0.515853781871504\\
36	0.582156120922253\\
37	0.457307041263558\\
38	0.394788288771593\\
39	0.385113330389144\\
40	0.235948719101105\\
41	0.400942916606883\\
42	0.304571079423648\\
43	0.305014238699459\\
44	0.489321671155054\\
45	0.530632960497969\\
46	0.470639655579753\\
47	0.531030631836087\\
48	0.484307348082753\\
49	0.571534545518245\\
50	0.430643074971713\\
51	0.521244600778244\\
52	0.506112204817425\\
53	0.59799964193009\\
54	0.460493800920078\\
55	0.39972939372257\\
56	0.528341612534132\\
57	0.93251378411044\\
58	0.788039219503593\\
59	1.24239217102466\\
60	0.620827361725037\\
61	0.662399567897206\\
62	0.507839562313441\\
63	0.422417112958209\\
64	0.559172422393611\\
65	0.534423352350451\\
66	0.528338248908195\\
67	0.404106186040187\\
68	0.544200762071391\\
69	0.465003498754473\\
70	0.387289421212477\\
71	0.341424620413702\\
72	0.37659582977998\\
73	0.524530446070264\\
74	0.685860252446102\\
75	0.359902859577453\\
76	0.385004767468789\\
77	0.346797286265547\\
78	0.500149506443763\\
79	0.693192511741897\\
80	0.554846888224278\\
81	0.639765997564432\\
82	0.773161132834359\\
83	0.698713619656617\\
84	0.688899915413481\\
85	0.645967822859769\\
86	0.590945687821289\\
87	0.630998860581595\\
88	1.54002203675934\\
89	1.10219062938273\\
90	0.941410750300312\\
91	0.876347096642116\\
92	0.927675955264196\\
93	0.736457138499131\\
94	0.597346193606566\\
95	0.646684734793865\\
96	0.634874205297542\\
97	0.62835258551894\\
98	0.675595337205851\\
99	0.653796405358777\\
100	0.737388547687769\\
101	0.62777820059426\\
102	0.79542268298354\\
103	0.760843019957251\\
104	0.756773110365475\\
105	0.556725751476698\\
106	1.09330571320605\\
107	0.983215959480354\\
108	0.714636106396028\\
109	0.809368527850809\\
110	0.687341057001014\\
111	0.734101462980017\\
112	0.617561335357786\\
113	0.574611519573093\\
114	0.965952128940091\\
115	0.886939048224717\\
116	0.594250570789462\\
117	0.665176312112676\\
118	1.19935417898177\\
119	0.662515194071923\\
120	0.798063171721589\\
121	0.794905268038086\\
122	0.665431213690731\\
123	0.663331997098926\\
124	0.91094476579677\\
125	1.02506151872686\\
126	0.776122190650635\\
127	0.999822137144747\\
128	0.813001698992531\\
129	0.806469121232209\\
130	0.787598789362484\\
131	0.61572210610844\\
132	0.64772597630447\\
133	0.570987441888115\\
134	0.564444970523129\\
135	0.813461409503097\\
136	0.735398332580561\\
137	0.880953871432894\\
138	0.69222242115663\\
139	0.49523489812302\\
140	0.825142218352522\\
141	0.56425658731651\\
142	0.646557779219151\\
143	0.748906682165934\\
144	0.781736851212993\\
145	0.531121478290193\\
146	0.828659661972717\\
147	1.2677332950082\\
148	1.2009992569898\\
149	0.983329293004528\\
150	0.754969416845888\\
151	2.03911088916297\\
152	1.68534178230356\\
153	0.934638706885989\\
154	1.04101196580143\\
155	0.95167029581856\\
156	1.10052258309991\\
157	0.985975851774984\\
158	0.64656458366843\\
159	0.735104591313398\\
160	0.905149290819487\\
161	0.99097536566092\\
162	0.988659418150452\\
163	1.493241778725\\
164	1.07319944541052\\
165	0.90944017259873\\
166	0.888126281901074\\
167	0.910214867476074\\
168	1.01239819194476\\
169	0.783286350063079\\
170	0.786179638031346\\
171	0.849510337039679\\
172	1.2498639358168\\
173	1.38835374953801\\
174	1.2943758283755\\
175	1.16579547386066\\
176	1.40344779060007\\
177	1.92716133241095\\
178	1.55072475231388\\
179	1.17975451895814\\
180	1.06893169096079\\
181	1.14143952076062\\
182	1.05464850543528\\
183	0.936937774868218\\
184	0.771663667539874\\
185	0.716077728209522\\
186	0.832706228134726\\
187	0.919150639201475\\
188	0.857306274313929\\
189	0.725683012190071\\
190	1.00790703624562\\
191	0.755606867478911\\
192	0.695699263453031\\
193	0.634550146903165\\
194	0.67165677282236\\
195	0.522734593201175\\
196	0.627253510728576\\
197	0.767766578774296\\
198	1.09444026153036\\
199	0.802462147153355\\
200	0.89928481419554\\
201	0.746587125979502\\
202	0.889974266938327\\
203	0.867220307656495\\
204	0.911802165797712\\
205	0.88268123479031\\
206	1.08727121017108\\
207	0.812756995270985\\
208	1.43015262232194\\
209	1.24128104528242\\
210	0.87627933153593\\
211	0.806449328934028\\
212	0.753983080570651\\
213	1.21543876792055\\
214	1.53399873691462\\
215	1.05810667044878\\
216	1.10045843413441\\
217	1.18213350261358\\
218	1.50811364266658\\
219	1.47155875025947\\
220	1.46118264030364\\
221	1.25802298488489\\
222	1.16027912581942\\
223	0.921543427357968\\
224	0.648692744037811\\
225	0.778791275789721\\
226	0.727660391711022\\
227	0.661829859084177\\
228	0.876544836797172\\
};
\addplot [color=cts_nFPC_plot_color,solid,line width=1pt,forget plot]
  table[row sep=crcr]{%
1	0.527013146032987\\
2	0.48628432283232\\
3	0.522973185561087\\
4	0.508828838082412\\
5	0.522409077325843\\
6	0.577311592731682\\
7	0.509378530488385\\
8	0.39896233910821\\
9	0.542667264872306\\
10	0.515872005250187\\
11	0.416546408079227\\
12	0.469539047479547\\
13	0.471262235064219\\
14	0.592222545441584\\
15	0.451203534280804\\
16	0.449823397531805\\
17	0.371180866256153\\
18	0.607399875882882\\
19	0.511206413126081\\
20	0.456925041132563\\
21	0.332134492494144\\
22	0.405654196348516\\
23	0.247929965935124\\
24	0.410925015429706\\
25	0.278497770577268\\
26	0.515211180533612\\
27	0.328237461580854\\
28	0.468354144416127\\
29	0.393698412954098\\
30	0.420735534620526\\
31	0.369983001347717\\
32	0.393406182075785\\
33	0.318810755781903\\
34	0.424333956775293\\
35	0.45197342804232\\
36	0.515579865987258\\
37	0.416400685259106\\
38	0.407839645652108\\
39	0.346114512317509\\
40	0.234109943204588\\
41	0.360818600160251\\
42	0.287161727598284\\
43	0.276328209136919\\
44	0.460010376495236\\
45	0.437908891396074\\
46	0.434171673700962\\
47	0.419241931432476\\
48	0.442441792061359\\
49	0.485111411499514\\
50	0.403264115155883\\
51	0.485031817001208\\
52	0.441481319597241\\
53	0.509139789136969\\
54	0.407626020094027\\
55	0.36275346833271\\
56	0.512151192209563\\
57	0.84456574003006\\
58	0.724620926245467\\
59	0.950683653412964\\
60	0.57845036623355\\
61	0.567699785885092\\
62	0.469751720401259\\
63	0.406687353013248\\
64	0.521256375904405\\
65	0.438986942588531\\
66	0.449350064327186\\
67	0.398197454670796\\
68	0.499971651817582\\
69	0.423758670303705\\
70	0.332857371786908\\
71	0.311940257747638\\
72	0.382665585076899\\
73	0.474071930629586\\
74	0.594359976019543\\
75	0.359279669243655\\
76	0.370916159704761\\
77	0.329997932417529\\
78	0.474826548616345\\
79	0.583579407461127\\
80	0.510933416273131\\
81	0.5897407828434\\
82	0.713589519710599\\
83	0.61465941502711\\
84	0.527333391725127\\
85	0.548473513556272\\
86	0.500555334611658\\
87	0.527047696728748\\
88	1.40479776223565\\
89	1.03198653817977\\
90	0.91771637170633\\
91	0.86744104187522\\
92	0.903990957156392\\
93	0.685572197176585\\
94	0.596676712370254\\
95	0.598635108513479\\
96	0.622123437724841\\
97	0.596881680282268\\
98	0.608757345223565\\
99	0.60256051917148\\
100	0.721969155737155\\
101	0.630528333130804\\
102	0.767320331159116\\
103	0.789767368971804\\
104	0.739379499647864\\
105	0.518871476200234\\
106	0.981071546442119\\
107	0.906496063059201\\
108	0.70360087340482\\
109	0.755562361395981\\
110	0.640021218342395\\
111	0.685232289578357\\
112	0.624939466069253\\
113	0.622322644059531\\
114	0.878197905195552\\
115	0.845055476433579\\
116	0.643890227317995\\
117	0.696685451274484\\
118	1.11639017747854\\
119	0.640210545515495\\
120	0.795721218136573\\
121	0.806656860518244\\
122	0.654060807058385\\
123	0.607271001406742\\
124	0.918422786390599\\
125	0.90609645766652\\
126	0.771032068074388\\
127	0.974015774815021\\
128	0.783591541472671\\
129	0.744971522913362\\
130	0.778007324245103\\
131	0.594474194205859\\
132	0.624254835512701\\
133	0.556729800051238\\
134	0.551638641700323\\
135	0.808062769143168\\
136	0.731988122822473\\
137	0.85103919266452\\
138	0.653369201744911\\
139	0.491849406840323\\
140	0.780605107054779\\
141	0.517237476445531\\
142	0.623673354856675\\
143	0.682511820177687\\
144	0.739958425964614\\
145	0.511453648023659\\
146	0.776831176285804\\
147	1.28130605114685\\
148	1.07548053047316\\
149	0.82310424365751\\
150	0.732531511718311\\
151	1.60500830445564\\
152	1.4261072291789\\
153	0.850466314618884\\
154	0.99534655944459\\
155	0.85045224747868\\
156	1.05313316772091\\
157	0.902488572334056\\
158	0.672516603521923\\
159	0.713571192892587\\
160	0.907579955149111\\
161	0.945679549020306\\
162	0.938556955022972\\
163	1.38130553598916\\
164	1.03287190510014\\
165	0.93015744338379\\
166	0.867973475972517\\
167	0.895542691050974\\
168	1.00415129028628\\
169	0.725210307977858\\
170	0.79879133918383\\
171	0.791557908092451\\
172	1.1561272497787\\
173	1.32944033993067\\
174	1.20339516964102\\
175	1.13725439202096\\
176	1.29222567636411\\
177	1.80303487409323\\
178	1.58299084686873\\
179	1.13856940151076\\
180	0.979020720871724\\
181	1.02680890272232\\
182	1.00943414558605\\
183	0.903116062906233\\
184	0.752471853706713\\
185	0.685979158780566\\
186	0.766120184834204\\
187	0.887134543376903\\
188	0.87404978625041\\
189	0.667394784375535\\
190	0.933887631078078\\
191	0.668007163996999\\
192	0.64975615016881\\
193	0.584845856297677\\
194	0.635870218019499\\
195	0.490961925753884\\
196	0.59669659252607\\
197	0.721791020441764\\
198	0.995935668952698\\
199	0.682791103903349\\
200	0.836700426015663\\
201	0.686017377066084\\
202	0.848252328844145\\
203	0.790383604599665\\
204	0.841922082035889\\
205	0.76831350475571\\
206	1.01145468348288\\
207	0.7224784784437\\
208	0.977310129834231\\
209	0.98428511542579\\
210	0.838339405391479\\
211	0.78819792946779\\
212	0.741045341939021\\
213	1.1482434427019\\
214	1.35642301201117\\
215	1.04440629731667\\
216	1.0495885963816\\
217	1.19694920180998\\
218	1.42580567828732\\
219	1.44711298169312\\
220	1.45726146428035\\
221	1.23322571472921\\
222	1.09817753707606\\
223	0.908983180048035\\
224	0.616097616728871\\
225	0.720872500917441\\
226	0.66214419416451\\
227	0.603137699242246\\
228	0.820104113439825\\
};
\addplot [color=dscr_nFPC_plot_color,solid,line width=1pt,forget plot]
  table[row sep=crcr]{%
1	0.549968256347378\\
2	0.529162917392576\\
3	0.576395197253715\\
4	0.526678335871561\\
5	0.539975706503024\\
6	0.632483232115493\\
7	0.539582045476306\\
8	0.439506059045925\\
9	0.613158090984689\\
10	0.555394311798505\\
11	0.465541103710198\\
12	0.493356656994018\\
13	0.51042874695835\\
14	0.59510944024271\\
15	0.532978506719299\\
16	0.465081493699681\\
17	0.415816050672101\\
18	0.69052704081578\\
19	0.583859503130896\\
20	0.491959776551032\\
21	0.37324536770292\\
22	0.440973661208236\\
23	0.270097545899761\\
24	0.461925295370307\\
25	0.316710046783775\\
26	0.579432884829812\\
27	0.358465875835747\\
28	0.514080265460167\\
29	0.416399192755409\\
30	0.448240410496426\\
31	0.394581799440262\\
32	0.421524185321602\\
33	0.32458897153917\\
34	0.455944832426325\\
35	0.520795109637355\\
36	0.566303946438182\\
37	0.477427066975175\\
38	0.414388774236902\\
39	0.354649151325897\\
40	0.243666051947293\\
41	0.422988960388855\\
42	0.300397187009708\\
43	0.312854643837493\\
44	0.472316110722165\\
45	0.505605616035618\\
46	0.46484248756415\\
47	0.465652120302308\\
48	0.483660377592654\\
49	0.545204440840352\\
50	0.436947018410702\\
51	0.51133002568318\\
52	0.475607811667886\\
53	0.581177363519852\\
54	0.453402850408356\\
55	0.39486082169253\\
56	0.553770968950097\\
57	0.886678039186445\\
58	0.832978135318167\\
59	1.19104418823622\\
60	0.648237349747213\\
61	0.61996907429614\\
62	0.519346529758409\\
63	0.419539773553721\\
64	0.563319786866697\\
65	0.494177348271629\\
66	0.514299860319154\\
67	0.424315287090703\\
68	0.526553092390261\\
69	0.457650465566024\\
70	0.360895162151618\\
71	0.342834617318656\\
72	0.399550602495141\\
73	0.518248197954059\\
74	0.654648032301695\\
75	0.370563382661612\\
76	0.406004172878663\\
77	0.350617779368454\\
78	0.513581330983037\\
79	0.643475399588613\\
80	0.544146811535351\\
81	0.672169760977239\\
82	0.79586575866552\\
83	0.697151958472333\\
84	0.64104783073341\\
85	0.605282791916975\\
86	0.582840652378036\\
87	0.632677011108779\\
88	1.49728435274927\\
89	1.16059734637857\\
90	0.959549778538781\\
91	0.901931511505811\\
92	0.971675566942361\\
93	0.746076643465853\\
94	0.604197514019656\\
95	0.65738008198325\\
96	0.655912739646712\\
97	0.624580041873934\\
98	0.669973175721948\\
99	0.664721383162933\\
100	0.798097053860049\\
101	0.672829511574294\\
102	0.81368515527951\\
103	0.816648115158968\\
104	0.693179790783279\\
105	0.551079044919066\\
106	1.08170781066368\\
107	0.924839040091002\\
108	0.74882756401343\\
109	0.786218879413868\\
110	0.65756151699529\\
111	0.74992348861896\\
112	0.655669026627503\\
113	0.625202631630227\\
114	0.990353636079273\\
115	0.896857526768284\\
116	0.644724670988897\\
117	0.743869904974882\\
118	1.18177348376587\\
119	0.658468910744254\\
120	0.824455372349693\\
121	0.81400181033904\\
122	0.703975438104185\\
123	0.630186426223042\\
124	0.965166479114125\\
125	0.976269236393907\\
126	0.745466808667985\\
127	1.04168193482869\\
128	0.804381241851947\\
129	0.800901865473201\\
130	0.790463516442074\\
131	0.627026346124255\\
132	0.643704767599913\\
133	0.561307399573567\\
134	0.579348336001171\\
135	0.807270925967026\\
136	0.740698784136677\\
137	0.901711607909822\\
138	0.686179287851513\\
139	0.508073153953537\\
140	0.823909870381861\\
141	0.554137141109483\\
142	0.654060847657772\\
143	0.721756343353926\\
144	0.779529226047671\\
145	0.531017504345761\\
146	0.8006212187954\\
147	1.33266777842362\\
148	1.24985569092528\\
149	0.935960180904252\\
150	0.771005895799184\\
151	2.18018428102887\\
152	1.69161368211221\\
153	1.00038171093251\\
154	1.04280617061884\\
155	0.948522403921837\\
156	1.14850712404679\\
157	0.939472361399624\\
158	0.694276811330867\\
159	0.740906915630114\\
160	0.939777070117689\\
161	1.01869003984382\\
162	0.976205995572552\\
163	1.43561690584349\\
164	1.06248323603473\\
165	0.938467914833425\\
166	0.927424439354192\\
167	0.922253188003247\\
168	1.00044375170679\\
169	0.765288308279841\\
170	0.774352022498486\\
171	0.818970218398529\\
172	1.22295451410339\\
173	1.41834361348849\\
174	1.31186379353161\\
175	1.20018216310318\\
176	1.36145158461151\\
177	1.88402908727715\\
178	1.57384216265156\\
179	1.19704471287921\\
180	1.07338403341461\\
181	1.11218570608091\\
182	1.06947436116827\\
183	0.965656166006739\\
184	0.776058806057357\\
185	0.7070261302204\\
186	0.821812090308181\\
187	0.943818787711232\\
188	0.871184159906395\\
189	0.715180653441377\\
190	1.00064359360506\\
191	0.76497218294952\\
192	0.722006515249328\\
193	0.635286545640924\\
194	0.648597611874939\\
195	0.511488062072491\\
196	0.617196157989858\\
197	0.80319428326138\\
198	1.10259133580778\\
199	0.796242832121232\\
200	0.970899529573888\\
201	0.736560379665237\\
202	0.88403753862138\\
203	0.896699858279534\\
204	0.901912016009564\\
205	0.843569634284451\\
206	1.13739703593133\\
207	0.837456812215811\\
208	1.18495691303143\\
209	1.19158121374004\\
210	0.894671909199632\\
211	0.831011460238132\\
212	0.811970015943177\\
213	1.23048187142181\\
214	1.57146604683578\\
215	1.11925007577664\\
216	1.09848564717802\\
217	1.20834252232944\\
218	1.63264444958277\\
219	1.48362999633601\\
220	1.38547278532348\\
221	1.27869662370463\\
222	1.14868649081685\\
223	0.920040030239484\\
224	0.659517992192129\\
225	0.760530726445564\\
226	0.70403284527378\\
227	0.66205098498412\\
228	0.875006123471908\\
};
\end{axis}
\end{tikzpicture}%
 
\end{subfigure}\\

\leavevmode\smash{\makebox[0pt]{\hspace{-7em}% HORIZONTAL POSITION           
  \rotatebox[origin=l]{90}{\hspace{7em}% VERTICAL POSITION
    Normalized PnL}%
}}\hspace{0pt plus 1filll}\null

Trading Day Number of 2014

\vspace{1cm}
\begin{subfigure}{\linewidth}
  %\centering
  \setlength\figureheight{\linewidth} 
  \setlength\figurewidth{\linewidth}
  \tikzsetnextfilename{strategylegend}
  \resizebox{\linewidth}{!}{\definecolor{mycolor1}{rgb}{0.25098,0.00000,0.38824}%
\definecolor{mycolor2}{rgb}{0.00000,0.46275,0.00000}%
\definecolor{mycolor3}{rgb}{0.00000,0.34902,0.34902}%
\definecolor{mycolor4}{rgb}{0.58039,0.26275,0.00000}%
\begin{tikzpicture}
    \begingroup
    % inits/clears the lists (which might be populated from previous
    % axes):
    \csname pgfplots@init@cleared@structures\endcsname
    \pgfplotsset{legend style={at={(0,1)},anchor=north west},legend columns=-1,legend style={draw=black,column sep=1ex},
            legend entries={Cts Stoch Ctrl,Dscr Stoch Ctrl,Cts Stoch Ctrl w nFPC,Dscr Stoch Ctrl w nFPC}}%
    
    \csname pgfplots@addlegendimage\endcsname{line width=2pt,mycolor1,sharp plot}
    \csname pgfplots@addlegendimage\endcsname{line width=2pt,mycolor2,sharp plot}
    \csname pgfplots@addlegendimage\endcsname{line width=2pt,mycolor3,sharp plot}
    \csname pgfplots@addlegendimage\endcsname{line width=2pt,mycolor4,sharp plot}

    % draws the legend:
    \csname pgfplots@createlegend\endcsname
    \endgroup
\end{tikzpicture}
}
\end{subfigure}%
  \caption{End of day strategy performances: out-of-sample backtesting on 2014 data, using amalgamated annual 2013 data for calibration.}
  \label{fig:OOS_annual_comp}
\end{figure}

\begin{table}
\centering
\ra{1.2}
\begin{tabular}{@{} *{9}{r} @{}}
\toprule
Strategy & Return & Sharpe & \# MO & \# LO & Inv & \% Win & Max Loss & Max Win \\
\midrule
\multicolumn{9}{l}{\texttt{INTC}} \\ 
Cts & 0.209 & 2.112 & 2118 & 1758 & 0.44 & 0.98 & -0.169 & 0.502 \\ 
Dscr & 0.372 & 1.591 & 949 & 1770 & -5.89 & 0.98 & -0.041 & 1.418 \\ 
Cts w nFPC & 0.483 & 2.364 & 704 & 1693 & 1.46 & 1.00 & -0.156 & 1.194 \\ 
Dscr w nFPC & 0.515 & 2.033 & 490 & 1629 & 2.81 & 1.00 &  & 2.072 \\  [2ex]
\multicolumn{9}{l}{\texttt{AAPL}} \\ 
Cts & 0.378 & 1.571 & 3853 & 6297 & -5.80 & 0.96 & -0.094 & 1.392 \\ 
Dscr & 0.761 & 2.457 & 830 & 5566 & 4.05 & 1.00 &  & 2.039 \\ 
Cts w nFPC & 0.710 & 2.479 & 1276 & 5689 & 2.93 & 1.00 &  & 1.803 \\ 
Dscr w nFPC & 0.764 & 2.442 & 796 & 5559 & 3.85 & 1.00 &  & 2.180 \\
\bottomrule
\end{tabular}
\caption{Averaged strategy performance results: out-of-sample backtesting on 2014 data, using amalgamated annual 2013 data for calibration.}
\label{tbl:OOS_annual}
\end{table}

The results in \autoref{tbl:OOS_annual} show the incredible out-of-sample results. The strategies post very solid returns and Sharpe ratios, and provide positive returns on almost every day that was tested. The strategies were run assuming our trade volume was 1 stock for every market order or limit order executed. Thus, to quantify these results in dollar terms, we can do a back of the envelope calculation by assuming for each of the 249 trading days of 2014, we trade 100 shares at a time, and multiply the average return (0.4 and 0.7 for \texttt{INTC} and \texttt{AAPL}, respectively) by the average share price during the year (\$30 and \$95, respectively). Thus, we conclude:

\begin{center}
{\bf Trading \texttt{INTC} would have generated revenue of \$298,800.} \par
{\bf Trading \texttt{AAPL} would have generated revenue of \$1,655,850.}
\end{center}

The capital requirements would have been the full price of the shares, \$30 and \$100 each, multiplied by the maximum long/short inventory of $20 \times 100$ shares - thus \$260,000. This represents both a compound annual growth rate (CAGR) and return on investment (ROI) of 652\%.